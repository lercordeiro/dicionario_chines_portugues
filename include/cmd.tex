%
% Dictionary functions
%

% Até agora o melhor estilo para capítulos
\chapterstyle{verville}

% Ajuste das margens                                                                                                     
\setlrmarginsandblock{3cm}{2cm}{*}
\setulmarginsandblock{2cm}{*}{1}
\checkandfixthelayout

% Linguagem principal
\setmainlanguage[variant=brazilian]{portuguese}
\setotherlanguages{chinese,english}
\hyphenation{
post-gresql
or-a-cle
mi-cro-soft
}


% Ajuste das fontes... No Tofu do Google
\setmainfont[
    Ligatures=TeX,
    BoldFont={NotoSerifCJKsc-SemiBold},
    BoldSlantedFont={NotoSerifCJKsc-SemiBold},
    AutoFakeSlant=0.25,
    SlantedFeatures={FakeSlant=0.25},
    BoldSlantedFeatures={FakeSlant=0.25}]
    {NotoSerifCJKsc-ExtraLight}
\setsansfont[Ligatures=TeX]{NotoSansCJKsc-DemiLight}
\setmonofont[Ligatures=TeX]{NotoSansMonoCJKsc-Regular}

% Sumário
\makeatletter
\renewcommand{\@pnumwidth}{2em} 
\renewcommand{\@tocrmarg}{4em}
\makeatother
\renewcommand\cftbeforechapterskip{5pt plus 1pt}

% Ajustes de espaçamento
\setlength{\parindent}{0cm}
\setlength{\parskip}{-1.5mm}                                                                                             
\setlength{\columnsep}{.8em}
\setlength{\columnseprule}{0.1mm}

% Headers & Footers
\setheadfoot{14pt}{28pt}
\makeevenfoot{plain}{\thepage}{汉葡词典}{}
\makeoddfoot{plain}{}{汉葡词典}{\thepage}
\makefootrule{plain}{\textwidth}{\normalrulethickness}{4pt}
\makepagestyle{dicionario}
\makeevenhead{dicionario}{\rightmark}{}{\leftmark}
\makeoddhead{dicionario}{\rightmark}{}{\leftmark}
\makeevenfoot{dicionario}{\thepage}{汉葡词典}{}
\makeoddfoot{dicionario}{}{汉葡词典}{\thepage}
\makeheadrule{dicionario}{\textwidth}{\normalrulethickness}

% Seções têm uma caixa arredondada em volta do nome, para o A-Z dos pinyin.
\newcommand{\boxedsec}[1]{%
    \begin{tcolorbox}[%
        nobeforeafter,
        colframe=black,%
        colback=black!5!white,%
        boxrule=2pt,%
        leftrule=3mm,%
        left=0mm,%
        right=0mm,%
        top=0mm,%
        bottom=0mm]
    \Large\bfseries#1
    \end{tcolorbox}
}
\setsecheadstyle{\boxedsec}
\setbeforesecskip{2sp}
\setaftersecskip{1sp}
\newcommand{\sectionbreak}{\phantomsection}%this is global effect

% Comandos
\newbool{f_veja}
\newbool{f_exemplo}
\renewcommand\stacktype{S}
\renewcommand\stackalignment{l}
\setstackgap{S}{2.5pt}

% Programa para converter pinyins numéricos para pinyins com acentos
\directlua{dofile "include/tex-sx-pinyin-tonemarks.lua"}

%% wrap converter in a TeX macro
\protected\def\pinyin#1{%
    %% switch to appropriate hyphenation pattern goes here
    \directlua{packagedata.pinyintones.convert ([==[#1]==])}%
}

% comando \&
\DeclareRobustCommand{\&}%
{
    \ifdim\fontdimen1\font>0pt
        \textsl{\symbol{`\&}}%
    \else
        \symbol{`\&}%
    \fi
}

% comando \dul{texto} --- underline
\NewDocumentCommand{\dul}{m}{\underline{#1}}

% comandos \dictpinyin{pin1yin1} e \dpy{pin1yin1}
\NewDocumentCommand{\dictpinyin}{m}{\guillemotleft\pinyin{#1}\guillemotright} 
\NewDocumentCommand{\dpy}{m}%
{%
    \StrSubstitute{#1}{r5}{r}[\rA]%
    \StrSubstitute{\rA}{v}{ü}[\rB]%
    \StrSubstitute{\rB}{V}{Ü}[\rC]%
    \edef\py{\dictpinyin{\rC}}%
    \mbox{}\py
}

% enumerate para as definições do dicionário
\NewDocumentCommand{\dictenumerate}{>{\SplitList{|}}m}
{%
    \begin{enumerate*}[left=0pt,mode=unboxed,font=\bfseries]
        \ProcessList{#1}{\insertitem}
    \end{enumerate*}
}
\NewDocumentCommand{\insertitem}{m}{\item #1}

% Lista Veja
\newenvironment{listaveja}%
{\list{}% empty label
    {
        \setlength{\topsep}{0ex}
        \setlength{\itemsep}{0ex}
        \setlength{\leftmargin}{0ex}
        \setlength{\labelsep}{0ex}
        \setlength{\parsep}{0pt}
        \setlength{\partopsep}{0pt}
        \setlength{\rightmargin}{0ex}
        \setlength{\listparindent}{0em}
        \setlength{\itemindent}{0ex}
        \setlength{\labelwidth}{0ex}
    }%
}%
{\endlist}
\newcommand{\vejaitem}[2]{\item[\addstackgap{\stackunder{#1}{\tiny\dpy{#2}}}]}

\newcommand\vejalst{}
\newcommand\exemplolst{}

\listadd{\vejalst}{}% Initialize list
\listadd{\exemplolst}{}% Initialize list

%%% EOF
