%%%%%%%%%%%%%%%%%%%%%%%%%%%%%%%%%%%%%%%%%%%%%%%%%%%%%%%%%%%%%%%%%%%%%%%%%%%%%%%
%%%%%%%%%%%%%%%%%%%%%%%%%%%%%%%%%%%%%%%%%%%%%%%%%%%%%%%%%%%%%%%%%%%%%%%%%%%%%%%
%%%%%                                                                     %%%%%
%%%%% Funções e Ajustes dos Documentos do Dicionário                      %%%%%
%%%%%                                                                     %%%%%
%%%%%%%%%%%%%%%%%%%%%%%%%%%%%%%%%%%%%%%%%%%%%%%%%%%%%%%%%%%%%%%%%%%%%%%%%%%%%%%
%%%%%%%%%%%%%%%%%%%%%%%%%%%%%%%%%%%%%%%%%%%%%%%%%%%%%%%%%%%%%%%%%%%%%%%%%%%%%%%

%%% Espaçamento das linhas normal
\SingleSpacing

%%% Hyperref em modo 'draft' não gera os hiperlinks
\hypersetup{final}

%%% Ajustes da separação das colunas quando em modo texto de 2 colunas
\setlength{\columnsep}{1.2em}
\setlength{\columnseprule}{0.1mm}

%%% Estilo do capítulo, o melhor que encontrei
\chapterstyle{dash}

%%% Sem identação
\setlength{\parindent}{0cm}
\setlength{\parskip}{0.15\baselineskip}

%%% Ajuste das margens do documento
\setlrmarginsandblock{3cm}{2cm}{*}
\setulmarginsandblock{2cm}{*}{1}
\checkandfixthelayout

%%% Pra evitar viúvas e órfãs
\clubpenalty=10000
\widowpenalty=10000
\raggedbottom

%%% Usando a fonte NoTofu do Google.
\babelfont{rm}[
 Renderer=Harfbuzz,
 Ligatures=TeX,
 BoldFont={NotoSerifCJKsc-Bold},
 BoldSlantedFont={NotoSansCJKsc-Regular},
 AutoFakeSlant=0.25,
 SlantedFeatures={FakeSlant=0.25},
 BoldSlantedFeatures={FakeSlant=0.25}]{Noto Serif CJK SC}
\babelfont{sf}[Renderer=Harfbuzz,Ligatures=TeX]{Noto Sans CJK SC}
\babelfont{tt}[Renderer=Harfbuzz,Ligatures=TeX]{Noto Sans Mono CJK SC}

%%% Ajustes do MultiCol: parar com a indentação do primeiro parágrafo
%\AddToHook{env/multicols/begin}{\AddToHookNext{para/begin}{\OmitIndent}}

%%% Ajustes do Sumário
\setcounter{secnumdepth}{2}
\makeatletter
\renewcommand{\@pnumwidth}{2em} 
\renewcommand{\@tocrmarg}{4em}
\makeatother
\renewcommand\cftbeforechapterskip{5pt plus 1pt}

%%% Cria 'Lista de Hanzis'
\newcommand{\listlohname}{Primeiros Hanzis}
\newlistof{listoffirsthanzis}{loh}{\listlohname}

%%% Ajustes de Cabeçalhos e Rodapés
\setheadfoot{14pt}{28pt}

% Estilo "plain"
\makefootrule{plain}{\textwidth}{\normalrulethickness}{2pt}
\ifdraftdoc
 \makeevenfoot{plain}{\thepage}{汉葡词典}{Draft}
 \makeoddfoot{plain}{Draft}{汉葡词典}{\thepage}
\else
 \makeevenfoot{plain}{\thepage}{汉葡词典}{}
 \makeoddfoot{plain}{}{汉葡词典}{\thepage}
\fi

% Estilo "dictionary"
\makepagestyle{dictionary}
\makeheadrule{dictionary}{\textwidth}{\normalrulethickness}
\makefootrule{dictionary}{\textwidth}{\normalrulethickness}{2pt}
\ifdraftdoc
 \makeevenhead{dictionary}{\rightmark}{Draft}{\leftmark}
 \makeoddhead{dictionary}{\rightmark}{Draft}{\leftmark}
 \makeevenfoot{dictionary}{\thepage}{汉葡词典}{Draft}
 \makeoddfoot{dictionary}{Draft}{汉葡词典}{\thepage}
\else
 \makeevenhead{dictionary}{\rightmark}{}{\leftmark}
 \makeoddhead{dictionary}{\rightmark}{}{\leftmark}
 \makeevenfoot{dictionary}{\thepage}{汉葡词典}{}
 \makeoddfoot{dictionary}{}{汉葡词典}{\thepage}
\fi

\newcommand{\boxedsec}[1]
 {%
  \begin{tcolorbox}%
   [%
    enhanced,%
    nobeforeafter,%
    before={\noindent},%
    colframe=black,%
    colback=black!20!white,%
    boxrule=2pt,%
    leftrule=4mm,%
    left=0mm,%
    right=0mm,%
    top=0mm,%
    bottom=0mm%
   ]
   \hfill\LARGE\bfseries#1
  \end{tcolorbox}
 }
\setsecheadstyle{\boxedsec}
\newcommand{\sectionbreak}{\phantomsection}

\newcommand{\boxedsubsec}[1]
 {%
  \begin{tcolorbox}%
   [%
    enhanced,%
    nobeforeafter,%
    before={\noindent},%
    colframe=black,%
    colback=black!10!white,%
    boxrule=2pt,%
    leftrule=2mm,%
    left=0mm,%
    right=0mm,%
    top=0mm,%
    bottom=0mm%
   ]
   \hfill\Large#1
  \end{tcolorbox}
 }
\setsubsecheadstyle{\boxedsubsec}

%%% Estilo das caixas dos verbetes
\newtcolorbox{lightbox}%
 {%
  enhanced,%
  size=fbox,%
  colframe=black,%
  colback=white,%
  boxrule=1pt,%
  toprule=3pt,%
  left=0mm,%
  right=0mm,%
  top=0mm,%
  bottom=0mm,%
  middle=0mm,%
  nobeforeafter,%
  segmentation empty,%
  before={\noindent}%
 }
\newtcolorbox{darkbox}%
 {%
  enhanced,%
  size=fbox,%
  colframe=black,%
  colback=black!5!white,%
  boxrule=1pt,%
  toprule=3pt,%
  left=0mm,%
  right=0mm,%
  top=0mm,%
  bottom=0mm,%
  middle=0mm,%
  nobeforeafter,%
  segmentation empty,%
  before={\noindent}%
 }


%%% Variáveis tipo "bool" para dizer se tem ou não os campos
%%% "Veja", "Veja também", "Synonym"" e "Antonym"
%%% nas definições dos verbetes
\newbool{hassee}
\newbool{hasseealso}
\newbool{hassynonym}
\newbool{hasantonym}

%%% Converte os pinyins numéricos em pinyins com marcação de tom
\directlua{dofile "include/tex-sx-pinyin-tonemarks.lua"}

%%% Comandos genéricos usados no Dicionário

% Função "\pinyin" faz a conversão
\protected\def\pinyin#1{\directlua{packagedata.pinyintones.convert ([==[#1]==])}}

\ExplSyntaxOn

% Comando "\dictpinyin", coloca o pinyin entre «»
\NewDocumentCommand{\dictpinyin}{m}{\guillemotleft\pinyin{#1}\guillemotright} 

% Comando "\dpy", gera a string do pinyin utilizada no Dicionário
% Este comando realiza uma série de substituições antes
\NewDocumentCommand{\dpy}{m}%
 {%
  \StrSubstitute{#1}{5}{}[\result]%
  \StrSubstitute{\result}{v}{ü}[\result]%
  \StrSubstitute{\result}{V}{Ü}[\result]%
  \edef\py{\dictpinyin{\result}}%
  \mbox{}\py
 }

% Yin, Yang e Os Oito Trigramas
\newfontfamily\dejavusans{DejaVu Sans}
\DeclareRobustCommand{\Yin}{{\dejavusans\symbol{"268A}}}
\DeclareRobustCommand{\Yang}{{\dejavusans\symbol{"268B}}}
\DeclareRobustCommand{\TrigramHeaven}{{\dejavusans\symbol{"2630}}}
\DeclareRobustCommand{\TrigramLake}{{\dejavusans\symbol{"2631}}}
\DeclareRobustCommand{\TrigramFire}{{\dejavusans\symbol{"2632}}}
\DeclareRobustCommand{\TrigramThunder}{{\dejavusans\symbol{"2633}}}
\DeclareRobustCommand{\TrigramWind}{{\dejavusans\symbol{"2634}}}
\DeclareRobustCommand{\TrigramWater}{{\dejavusans\symbol{"2635}}}
\DeclareRobustCommand{\TrigramMountain}{{\dejavusans\symbol{"2636}}}
\DeclareRobustCommand{\TrigramEarth}{{\dejavusans\symbol{"2637}}}

% Comando "\&", insere o caracgter "&"
%\DeclareRobustCommand{\&}%
% {%
%  \ifdim\fontdimen1\font>0pt%
%   \textsl{\symbol{`\&}}%
%  \else%
%   \symbol{`\&}%
%  \fi%
% }

\NewDocumentCommand{\setvar}{mm}
 {
  % clear an existing variable or allocate a new one
  \tl_clear_new:c { g__youthdoo_var_#1_tl }
  % set to the stated value
  \tl_gset:cn { g__youthdoo_var_#1_tl } { #2 }
 }

\NewExpandableDocumentCommand{\usevar}{m}
 {
  % deliver the contents
  \tl_use:c { g__youthdoo_var_#1_tl }
 }

% Ambiente "enumerate" especial utilizado no dicionário, coloca as definições 
% do verbete em uma lista numerada em linha
\NewDocumentCommand{\dictenumerate}{>{\SplitList{|}}m}
 {%
  \begin{enumerate*}[nosep,label={\arabic*},left=0pt,mode=boxed,font=\bfseries]
   \ProcessList{#1}{\insertitem}
  \end{enumerate*}
 }
\NewDocumentCommand{\insertitem}{>{\TrimSpaces}m}{\item #1}

% Ambiente "enumerate" especial utilizado no dicionário, coloca os exemplos
% das definições do verbete em uma lista numerada em linha, utilizando
% algarismos romanos
\makeatletter
\NewDocumentCommand{\dictexamples}{m>{\SplitList{|}}m}
 {%
  \def\@theword{#1}% 
  \begin{sloppypar}
   \begin{enumerate}[nosep,label=\alph*),left=0pt,mode=boxed,font=\bfseries]
    \ProcessList{#2}{\insertexample}
   \end{enumerate}
  \end{sloppypar}
 }
\NewDocumentCommand{\insertexample}{>{\TrimSpaces}m}
 {%
   \IfSubStr{#1}{===}
   {% Com traducao
     \StrCut{#1}{===}\csH\csP%
     \StrSubstitute{\csH}{\@theword}{\underline{\@theword}}[\csHUL]%
     \item\foreignlanguage{chinese-hans}{\csHUL}\\*\textit{\footnotesize``\csP''}
   }
   {% Sem traducao
     \StrSubstitute{#1}{\@theword}{\underline{\@theword}}[\csHUL]%
     \item\foreignlanguage{chinese-hans}{\csHUL}
   }
 }  
\makeatother

%%% Cria listas especializadas (seelist, seealsolist, synonymlist e antonymlist)
\newlist{seelist}{enumerate}{1}
\newlist{seealsolist}{enumerate}{1}
\newlist{synonymlist}{enumerate}{1}
\newlist{antonymlist}{enumerate}{1}

\setlist[seelist]{label={\roman*)},topsep=0pt,nosep,noitemsep,font=\bfseries}
\setlist[seealsolist]{label={\roman*)},topsep=0pt,nosep,noitemsep,font=\bfseries}
\setlist[synonymlist]{label={\roman*)},topsep=0pt,nosep,noitemsep,font=\bfseries}
\setlist[antonymlist]{label={\roman*)},topsep=0pt,nosep,noitemsep,font=\bfseries}

%%% Cria e inicializa a lista "Veja", "Veja também", "Sinônimo(s)" e "Antônimo(s)"
\newcommand\seerefl{}
\newcommand\seealsorefl{}
\newcommand\synonymrefl{}
\newcommand\antonymrefl{}

%%% Comando "\areference", adiciona um item "Veja", "Veja também", "Antônimo(s)" ou "Sinônimo(s)" na lista,
%%% com os pinyins abaixo dos caracteres
\newcommand{\areference}[2]
 {
  \StrLen{#1}[\hlen]%
  \StrLen{#2}[\plen]%
  \ifnumcomp{\hlen+\plen}{>}{24}
   {%
    \foreignlanguage{chinese-hans}{#1}\ (pág.~\pageref{\l_label_tl : #1 : #2})\\*
    \dpy{#2}
   }
   {%
    \foreignlanguage{chinese-hans}{#1}\ \dpy{#2}\ (pág.~\pageref{\l_label_tl : #1 : #2})
   }
 }

%%% Comando "\definition", gera o texto da definição
\NewDocumentCommand{\definition}{sommo}
 {%
  \begin{midsloppypar}
   \IfBooleanTF{#1}%
    {% Substantivo Próprio
     {\small\ding{108}}\ (\textit{S.P.})\IfValueT{#2}{~[clas.:~#2]}{\ \dictenumerate{#4}}\par
    }%
    {%
     {\small\ding{108}}\ (\textit{#3})\IfValueT{#2}{~[clas.:~#2]}{\ \dictenumerate{#4}}\par
    }%
  \end{midsloppypar}
  \IfValueT{#5}%
   {%
    \IfSubStr{#5}{|}{\textbf{Exemplos:}}{\textbf{Exemplo:}}\dictexamples{\l_hanzi_tl}{#5}\par
   }%
 }

%%% Comando "Variante de"
\NewDocumentCommand{\variantof}{m}
 {
  {\small\ding{108}}\ Variante\ de\ \foreignlanguage{chinese-hans}{#1}\ (p.~\pageref{\l_label_tl : #1 : \l_pinyin_tl})\par
 }

%%% Comando "Veja"
\NewDocumentCommand{\seeref}{m}{\booltrue{hassee}\listgadd{\seerefl}{\l_hanzi_tl : #1}}

%%% Comando "Veja também"
\NewDocumentCommand{\seealsoref}{mm}{\booltrue{hasseealso}\listgadd{\seealsorefl}{#1 : #2}}

%%% Comando "Sinônimo(s)"
\NewDocumentCommand{\synonymref}{mm}{\booltrue{hassynonym}\listgadd{\synonymrefl}{#1 : #2}}

%%% Comando "Antônimo(s)"
\NewDocumentCommand{\antonymref}{mm}{\booltrue{hasantonym}\listgadd{\antonymrefl}{#1 : #2}}

%%% Ambiente "DictionaryEntries" para definir o início das entradas dos verbetes
\NewDocumentEnvironment{DictionaryEntries}{m}%
 {%
  \tl_set:Nn \l_label_tl {#1}
  \pagestyle{dictionary}
  \twocolumn
 }%
 {%
  \onecolumn
 }%

%%% Ambiente "EntryWithPhonetic", para os verbetes
\NewDocumentEnvironment{EntryWithPhonetic}{mO{}mO{}mmooo}%
 {%
  \leavevmode
  \markboth{#1{\tiny\dpy{#3}}}{#1{\tiny\dpy{#3}}}
  \tl_set:Nn \l_hanzi_tl {#1}
  \tl_set:Nn \l_pinyin_tl {#3}
  \tl_set:Nn \l_strokes_tl {#5}
  \boolfalse{hassee}\renewcommand\seerefl{}
  \boolfalse{hasseealso}\renewcommand\seealsorefl{}
  \boolfalse{hassynonym}\renewcommand\synonymrefl{}
  \boolfalse{hasantonym}\renewcommand\antonymrefl{}
  \label{\l_label_tl : #1 : #3}
  \StrLen{#1}[\hlen]%
  \StrLen{#3}[\plen]%
  \begin{lightbox}
   \ifnumcomp{\hlen}{>}{10}
    {%
     \mbox{}\hfill\textsuperscript{\tiny(#5画)}\\
     {\Large\foreignlanguage{chinese-hans}{#1}}
    }
    {%
     {\Large\foreignlanguage{chinese-hans}{#1}}\hfill\textsuperscript{\tiny(#5画)}
    }
   \tcblower
   \ifnumcomp{\plen}{>}{25}
    {%
     {\footnotesize#2\ \dpy{#3}\ #4}\\
    }
    {%
      {\footnotesize#2\ \dpy{#3}\ #4}
    }
   \IfValueT{#7}{\mbox{}\hfill{\tiny#7}}{}%
   \IfValueT{#8}{\mbox{}\hfill{\tiny#8}}{}%
   \IfValueT{#9}{\mbox{}\hfill{\tiny#9}}{}%
   \mbox{}\hfill\IfSubStr{#6}{、}{\tiny Radicais\ #6}{\tiny Radical\ #6}
  \end{lightbox}
 }%
 {%
  \ifbool{hassee}%
   {% Processa as referências "Veja"
    \RenewDocumentCommand\do{>{\SplitArgument{1}{:}}m}{\item \areference ##1}
    \textbf{Veja:}%
    \begin{seelist}
     \dolistloop{\seerefl}
    \end{seelist}
   }{}%
  \ifbool{hasseealso}%
   {% Processa as referências "Veja também"
    \RenewDocumentCommand\do{>{\SplitArgument{1}{:}}m}{\item \areference ##1}
    \textbf{Veja\ também:}%
    \begin{seealsolist}
     \dolistloop{\seealsorefl}
    \end{seealsolist}
   }{}%
  \ifbool{hassynonym}%
   {% Processa as referências "Sinônimo"
    \RenewDocumentCommand\do{>{\SplitArgument{1}{:}}m}{\item \areference ##1}
    \textbf{Sinônimo(s):}%
    \begin{synonymlist}
     \dolistloop{\synonymrefl}
    \end{synonymlist}
   }{}%
  \ifbool{hasantonym}%
   {% Processa as referências "Antônimo"
    \RenewDocumentCommand\do{>{\SplitArgument{1}{:}}m}{\item \areference ##1}
    \textbf{Antônimo(s):}%
    \begin{antonymlist}
     \dolistloop{\antonymrefl}
    \end{antonymlist}
   }{}%
 }

%%% Ambiente "Entry", para os verbetes
\NewDocumentEnvironment{Entry}{mmmooo}%
 {%
  \leavevmode
  \markboth{#1{\tiny(#2画)}}{#1{\tiny(#2画)}}
  \tl_set:Nn \l_hanzi_tl {#1}
  \tl_set:Nn \l_strokes_tl {#2}
  \StrLen{#1}[\hlen]%
  \begin{lightbox}
   \ifnumcomp{\hlen}{>}{10}
    {%
     \mbox{}\hfill\textsuperscript{\tiny(#2画)}\\
     {\Large\foreignlanguage{chinese-hans}{#1}}
    }
    {%
     {\LARGE#1}\hfill\textsuperscript{\tiny(#2画)}
    }
   \tcblower
   \IfValueT{#4}{\mbox{}\hfill{\tiny#4}}{}%
   \IfValueT{#5}{\mbox{}\hfill{\tiny#5}}{}%
   \IfValueT{#6}{\mbox{}\hfill{\tiny#6}}{}%
   \mbox{}\hfill\IfSubStr{#3}{,}{\tiny Radicais\ #3}{\tiny Radical\ #3}
  \end{lightbox}\par
 }{}

%%% Ambiente "Phonetics", para as diversas entradas de fonética da palavra
\NewDocumentEnvironment{Phonetics}{mO{}mO{}O{}}%
 {%
  \tl_set:Nn \l_pinyin_tl {#3}
  \boolfalse{hassee}\renewcommand\seerefl{}
  \boolfalse{hasseealso}\renewcommand\seealsorefl{}
  \boolfalse{hassynonym}\renewcommand\synonymrefl{}
  \boolfalse{hasantonym}\renewcommand\antonymrefl{}
  \label{\l_label_tl : #1 : #3}
   \ding{93}\ #2\ \dpy{#3}\ #4\ \ding{93}\hfill \textbf{#5}
 }%
 {%  
  \ifbool{hassee}%
   {% Processa as referências "Veja"
    \RenewDocumentCommand\do{>{\SplitArgument{1}{:}}m}{\item \areference ##1}
    \textbf{Veja:}%
    \begin{seelist}
     \dolistloop{\seerefl}
    \end{seelist}
   }{}%
  \ifbool{hasseealso}%
   {% Processa as referências "Veja também"
    \RenewDocumentCommand\do{>{\SplitArgument{1}{:}}m}{\item \areference ##1}
    \textbf{Veja\ também:}%
    \begin{seealsolist}
     \dolistloop{\seealsorefl}
    \end{seealsolist}
   }{}%
  \ifbool{hassynonym}%
   {% Processa as referências "Sinônimo"
    \RenewDocumentCommand\do{>{\SplitArgument{1}{:}}m}{\item \areference ##1}
    \textbf{Sinônimo(s):}%
    \begin{synonymlist}
     \dolistloop{\synonymrefl}
    \end{synonymlist}
  }{}%
  \ifbool{hasantonym}%
   {% Processa as referências "Antônimo"
    \RenewDocumentCommand\do{>{\SplitArgument{1}{:}}m}{\item \areference ##1}
    \textbf{Antônimo(s):}%
    \begin{antonymlist}
     \dolistloop{\antonymrefl}
    \end{antonymlist}
   }{}%
 }

\ExplSyntaxOff

%%%%% EOF %%%%%
