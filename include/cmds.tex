%%%%%%%%%%%%%%%%%%%%%%%%%%%%%%%%%%%%%%%%%%%%%%%%%%%%%%%%%%%%%%%%%%%%%%%%%%%%%%%
%%%%%%%%%%%%%%%%%%%%%%%%%%%%%%%%%%%%%%%%%%%%%%%%%%%%%%%%%%%%%%%%%%%%%%%%%%%%%%%
%%%%%                                                                     %%%%%
%%%%% Funções e Ajustes dos Documentos do Dicionário                      %%%%%
%%%%%                                                                     %%%%%
%%%%%%%%%%%%%%%%%%%%%%%%%%%%%%%%%%%%%%%%%%%%%%%%%%%%%%%%%%%%%%%%%%%%%%%%%%%%%%%
%%%%%%%%%%%%%%%%%%%%%%%%%%%%%%%%%%%%%%%%%%%%%%%%%%%%%%%%%%%%%%%%%%%%%%%%%%%%%%%

%%% Espaçamento das linhas normal
\SingleSpacing

%%% Hyperref em modo 'draft' não gera os hiperlinks
\hypersetup{final}

%%% Largura da entrada do verbete
\def\entrywidth{.485\textwidth}

%%% Estilo do capítulo, o melhor que encontrei
\chapterstyle{verville}

%%% Sem identação
\setlength{\parindent}{0cm}
\setlength{\parskip}{0.6\baselineskip}

%%% Ajuste das margens do documento
\setlrmarginsandblock{3cm}{2cm}{*}
\setulmarginsandblock{2cm}{*}{1}
\checkandfixthelayout

%%% Pra evitar viúvas e órfãs
\clubpenalty=10000
\widowpenalty=10000
\raggedbottom

%%% Usando a fonte NoTofu do Google.
\babelfont{rm}[
 Renderer=Node,
 Ligatures=TeX,
 BoldFont={NotoSerifCJKsc-SemiBold},
 BoldSlantedFont={NotoSerifCJKsc-SemiBold},
 AutoFakeSlant=0.25,
 SlantedFeatures={FakeSlant=0.25},
 BoldSlantedFeatures={FakeSlant=0.25}]
 {Noto Serif CJK SC Light}
\babelfont{sf}[Renderer=Harfbuzz,Ligatures=TeX]{Noto Sans CJK SC Light}
\babelfont{tt}[Renderer=Harfbuzz,Ligatures=TeX]{Noto Sans Mono CJK SC}

%%% Ajustes do MultiCol: parar com a indentação do primeiro parágrafo
\AddToHook{env/multicols/begin}{\AddToHookNext{para/begin}{\OmitIndent}}

%%% Ajustes do Sumário
\makeatletter
\renewcommand{\@pnumwidth}{2em} 
\renewcommand{\@tocrmarg}{4em}
\makeatother
\renewcommand\cftbeforechapterskip{5pt plus 1pt}

%%% Ajustes da separação das colunas quando em modo texto de 2 colunas
\setlength{\columnsep}{0.8em}
\setlength{\columnseprule}{0.1mm}

%%% Ajustes para o "stackengine"
\renewcommand\stacktype{S}
\renewcommand\stackalignment{c}

%%% Ajustes de Cabeçalhos e Rodapés
\setheadfoot{14pt}{28pt}

% Estilo "plain"
\makefootrule{plain}{\textwidth}{\normalrulethickness}{2pt}
\ifdraftdoc
 \makeevenfoot{plain}{\thepage}{汉葡词典}{Draft}
 \makeoddfoot{plain}{Draft}{汉葡词典}{\thepage}
\else
 \makeevenfoot{plain}{\thepage}{汉葡词典}{}
 \makeoddfoot{plain}{}{汉葡词典}{\thepage}
\fi

% Estilo "dictionary"
\makepagestyle{dictionary}
\makeheadrule{dictionary}{\textwidth}{\normalrulethickness}
\makefootrule{dictionary}{\textwidth}{\normalrulethickness}{2pt}
\ifdraftdoc
 \makeevenhead{dictionary}{\rightmark}{Draft}{\leftmark}
 \makeoddhead{dictionary}{\rightmark}{Draft}{\leftmark}
 \makeevenfoot{dictionary}{\thepage}{汉葡词典}{Draft}
 \makeoddfoot{dictionary}{Draft}{汉葡词典}{\thepage}
\else
 \makeevenhead{dictionary}{\rightmark}{}{\leftmark}
 \makeoddhead{dictionary}{\rightmark}{}{\leftmark}
 \makeevenfoot{dictionary}{\thepage}{汉葡词典}{}
 \makeoddfoot{dictionary}{}{汉葡词典}{\thepage}
\fi

%%% Estilo das Seções
\newcommand{\boxedsec}[1]
 {%
  \begin{tcolorbox}%
   [%
    enhanced,%
    nobeforeafter,%
    before={\noindent},%
    colframe=black,%
    colback=black!15!white,%
    boxrule=2pt,%
    leftrule=2mm,%
    left=0mm,%
    right=0mm,%
    top=0mm,%
    bottom=0mm%
   ]
   \hfill\LARGE\bfseries#1
  \end{tcolorbox}
 }
\setsecheadstyle{\boxedsec}
\setbeforesecskip{1ex plus .25ex minus .25ex}
\setaftersecskip{.25ex plus .25ex minus .25ex}
\newcommand{\sectionbreak}{\phantomsection}

%%% Estilo das caixas dos verbetes
\newtcolorbox{lightbox}%
 {%
  enhanced,%
  size=fbox,%
  colframe=black,%
  colback=white,%
  boxrule=1pt,%
  toprule=3pt,%
  left=0mm,%
  right=0mm,%
  top=0mm,%
  bottom=0mm,%
  middle=0mm,%
  nobeforeafter,%
  segmentation empty,%
  before={\noindent}%
 }
\newtcolorbox{darkbox}%
 {%
  enhanced,%
  size=fbox,%
  colframe=black,%
  colback=black!5!white,%
  boxrule=1pt,%
  toprule=3pt,%
  left=0mm,%
  right=0mm,%
  top=0mm,%
  bottom=0mm,%
  middle=0mm,%
  nobeforeafter,%
  segmentation empty,%
  before={\noindent}%
 }


%%% Variáveis tipo "bool" para dizer se tem ou não os campos
%%% "Veja" e "Veja também" nas definições dos verbetes
\newbool{f_see}
\newbool{f_seealso}

%%% Converte os pinyins numéricos em pinyins com marcação de tom
\directlua{dofile "include/tex-sx-pinyin-tonemarks.lua"}

%%% Comandos genéricos usados no Dicionário

% Função "\pinyin" faz a conversão
\protected\def\pinyin#1{%
 \directlua{packagedata.pinyintones.convert ([==[#1]==])}%
}

\ExplSyntaxOn

% Comando "\dictpinyin", coloca o pinyin entre «»
\NewDocumentCommand{\dictpinyin}{m}{\guillemotleft\pinyin{#1}\guillemotright} 

% Comando "\dpy", gera a string do pinyin utilizada no Dicionário
% Este comando realiza uma série de substituições antes
\NewDocumentCommand{\dpy}{m}%
 {%
  \StrSubstitute{#1}{5}{}[\result]%
  \StrSubstitute{\result}{v}{ü}[\result]%
  \StrSubstitute{\result}{V}{Ü}[\result]%
  \edef\py{\dictpinyin{\result}}%
  \mbox{}\py
 }

% Comando "\&", insere o caracgter "&"
\DeclareRobustCommand{\&}%
 {%
  \ifdim\fontdimen1\font>0pt%
   \textsl{\symbol{`\&}}%
  \else%
   \symbol{`\&}%
  \fi%
 }

\NewDocumentCommand{\setvar}{mm}
 {
  % clear an existing variable or allocate a new one
  \tl_clear_new:c { g__youthdoo_var_#1_tl }
  % set to the stated value
  \tl_gset:cn { g__youthdoo_var_#1_tl } { #2 }
 }

\NewExpandableDocumentCommand{\usevar}{m}
 {
  % deliver the contents
  \tl_use:c { g__youthdoo_var_#1_tl }
 }

% Ambiente "enumerate" especial utilizado no dicionário, coloca as definições 
% do verbete em uma lista numerada em linha
\NewDocumentCommand{\dictenumerate}{>{\SplitList{|}}m}
 {%
  \begin{enumerate*}[nosep,label=(\arabic*),left=0pt,mode=unboxed,font=\bfseries]
   \ProcessList{#1}{\insertitem}
  \end{enumerate*}
 }
\NewDocumentCommand{\insertitem}{>{\TrimSpaces}m}{\item #1}

% Ambiente "enumerate" especial utilizado no dicionário, coloca os exemplos
% das definições do verbete em uma lista numerada em linha, utilizando
% algarismos romanos
\makeatletter
\NewDocumentCommand{\dictexamples}{m>{\SplitList{|}}m}
 {%
  \def\@theword{#1}%
  \begin{enumerate}[nosep,label=(\roman*),left=0pt,mode=unboxed,font=\bfseries]
   \ProcessList{#2}{\insertexample}
  \end{enumerate}
 }
\NewDocumentCommand{\insertexample}{>{\TrimSpaces}m}
 {
   \IfSubStr{#1}{===}
   {% Com traducao
     \StrCut{#1}{===}\csA\csB%
     \item\StrSubstitute{\csA}{\@theword}{\underline{\@theword}}\\{\small``\csB''}
   }
   {% Sem traducao
     \item\StrSubstitute{#1}{\@theword}{\underline{\@theword}}
   }
 }  
\makeatother

%%% Cria listas especializadas (seelist e seealsolist)
\newlist{seelist}{enumerate}{1}
\newlist{seealsolist}{enumerate}{1}

\setlist[seelist]{label={(\alph*)},topsep=0pt,nosep,noitemsep}%,leftmargin=\parindent}
\setlist[seealsolist]{label={(\alph*)},topsep=0pt,nosep,noitemsep}%,leftmargin=\parindent}

%%% Cria e inicializa a lista "\seerefl", "Veja"
\newcommand\seerefl{}
\listadd{\seerefl}{}% Inicializa a lista

%%% Cria e inicializa a lista "\seealsorefl", "Veja também"
\newcommand\seealsorefl{}
\listadd{\seealsorefl}{}% Inicializa a lista

%%% Comando "\seeitem", adiciona um item "Veja" ou "Veja também" na lista,
%%% com os pinyins abaixo dos caracteres
\newcommand{\seeitem}[2]{#1~\dpy{#2}\ (p.~\pageref{\l_label_tl : #1 : #2})}

%%% Comando "\definition", gera o texto da definição
\NewDocumentCommand{\definition}{sommo}
 {%
  \IfBooleanTF{#1}%
   {% Substantivo Próprio
    {\small\ding{108}}\ (\textit{S.P.})\IfValueT{#2}{~[clas.:~#2]}{\ \dictenumerate{#4}}\par
   }%
   {%
    {\small\ding{108}}\ (\textit{#3})\IfValueT{#2}{~[clas.: #2]}{\ \dictenumerate{#4}}\par
   }%
  \IfValueT{#5}%
   {%
    \IfSubStr{#5}{|}{\textbf{Exemplos:}}{\textbf{Exemplo:}}\dictexamples{\l_hanzi_tl}{#5}
   }%
 }

%%% Comando "Variante de"
\NewDocumentCommand{\variantof}{m}
 {
  {\small\ding{108}}\ Variante\ de\ #1\ (p.~\pageref{\l_label_tl : #1 : \l_pinyin_tl})\par
 }

%%% Comando "Veja"
\NewDocumentCommand{\seeref}{m}
 {%
  \booltrue{f_see}
  \listgadd{\seerefl}{\l_hanzi_tl : #1}
 }

%%% Comando "Veja também"
\NewDocumentCommand{\seealsoref}{mm}
 {%
  \booltrue{f_seealso}
  \listgadd{\seealsorefl}{#1 : #2}
 }

%%% Ambiente "DictionaryEntries" para definir o início das entradas dos verbetes
\NewDocumentEnvironment{DictionaryEntries}{m}%
 {%
  \tl_set:Nn \l_label_tl {#1}
  \pagestyle{dictionary}
  \twocolumn
 }%
 {%
  \onecolumn
  \pagestyle{plain}
 }%

%%% Ambiente "EntryWithPhonetic", para os verbetes
\NewDocumentEnvironment{EntryWithPhonetic}{mO{}mO{}mmooo}%
 {%
  \leavevmode
  \markboth{#1{\tiny\dpy{#3}}}{#1{\tiny\dpy{#3}}}
  \tl_set:Nn \l_hanzi_tl {#1}
  \tl_set:Nn \l_pinyin_tl {#3}
  \tl_set:Nn \l_strokes_tl {#5}
  \boolfalse{f_see}\renewcommand\seerefl{}\listadd{\seerefl}{}% Inicializa a lista
  \boolfalse{f_seealso}\renewcommand\seealsorefl{}\listadd{\seealsorefl}{}% Inicializa a lista
  \begin{minipage}[t][][t]{\entrywidth}
   \label{\l_label_tl : #1 : #3}
   \StrLen{#1}[\hanzilen]
   \ifthenelse{\hanzilen > 1}
    {%
     \begin{lightbox}
      {\Large#1}\hfill\textsuperscript{\tiny(#5画)}
      \tcblower
      {\footnotesize#2\ \dpy{#3}\ #4}%
      \IfValueT{#7}{\mbox{}\hfill{\tiny#7}}{}%
      \IfValueT{#8}{\mbox{}\hfill{\tiny#8}}{}%
      \IfValueT{#9}{\mbox{}\hfill{\tiny#9}}{}%
      \mbox{}\hfill\IfSubStr{#6}{、}{\tiny Radicais\ #6}{\tiny Radical\ #6}
     \end{lightbox}
    }%
    {%
     \begin{darkbox}
      {\Large#1}\hfill\textsuperscript{\tiny(#5画)}
      \tcblower
      {\footnotesize#2\ \dpy{#3}\ #4}%
      \IfValueT{#7}{\mbox{}\hfill{\tiny#7}}{}%
      \IfValueT{#8}{\mbox{}\hfill{\tiny#8}}{}%
      \IfValueT{#9}{\mbox{}\hfill{\tiny#9}}{}%
      \mbox{}\hfill\IfSubStr{#6}{、}{\tiny Radicais\ #6}{\tiny Radical\ #6}
     \end{darkbox}
    }%
 }%
 {%
  \ifbool{f_see}%
   {% Processa as referências "Veja"
    \RenewDocumentCommand\do{>{\SplitArgument{1}{:}}m}{\item \seeitem ##1}
    \textbf{Veja:}%
    \begin{seelist}
     \dolistloop{\seerefl}
    \end{seelist}
   }{}%
  \ifbool{f_seealso}%
   {% Processa as referências "Veja também"
    \RenewDocumentCommand\do{>{\SplitArgument{1}{:}}m}{\item \seeitem ##1}
    \textbf{Veja\ também:}%
    \begin{seealsolist}
     \dolistloop{\seealsorefl}
    \end{seealsolist}
   }{}%
  \end{minipage}
 }

%%% Ambiente "EntryWithPhonetic*", para os verbetes muito compridos
\NewDocumentEnvironment{EntryWithPhonetic*}{mO{}mO{}mmooo}%
 {%
  \leavevmode
  \markboth{#1{\tiny\dpy{#3}}}{#1{\tiny\dpy{#3}}}
  \tl_set:Nn \l_hanzi_tl {#1}
  \tl_set:Nn \l_pinyin_tl {#3}
  \tl_set:Nn \l_strokes_tl {#5}
  \boolfalse{f_see} \renewcommand\seerefl{} \listadd{\seerefl}{}% Inicializa a lista
  \boolfalse{f_seealso} \renewcommand\seealsorefl{} \listadd{\seealsorefl}{}% Inicializa a lista
  \begin{minipage}[t][][t]{\entrywidth}
   \label{\l_label_tl : #1 : #3}
   \begin{lightbox}
    \mbox{}\hfill\textsuperscript{\tiny(#5画)}\\
    {\LARGE#1}
    \tcblower
    {\footnotesize#2\ \dpy{#3}\ #4}\\
    \IfValueT{#7}{\mbox{}\hfill{\tiny#7}}{}%
    \IfValueT{#8}{\mbox{}\hfill{\tiny#8}}{}%
    \IfValueT{#9}{\mbox{}\hfill{\tiny#9}}{}%
    \mbox{}\hfill\IfSubStr{#6}{、}{\tiny Radicais\ #6}{\tiny Radical\ #6}                                      
   \end{lightbox}
 }%
 {%
  \ifbool{f_see}%
   {% Processa as referências "Veja"
    \RenewDocumentCommand\do{>{\SplitArgument{1}{:}}m}{\item \seeitem ##1}
    \textbf{Veja:}%
    \begin{seelist}
     \dolistloop{\seerefl}
    \end{seelist}
   }{}%
  \ifbool{f_seealso}%
   {% Processa as referências "Veja também"
    \RenewDocumentCommand\do{>{\SplitArgument{1}{:}}m}{\item \seeitem ##1}
    \textbf{Veja\ também:}%
    \begin{seealsolist}
     \dolistloop{\seealsorefl}
    \end{seealsolist}
   }{}%
  \end{minipage}
 }

%%% Ambiente "Entry", para os verbetes
\NewDocumentEnvironment{Entry}{mmmooo}%
 {%
  \leavevmode
  \markboth{#1{\tiny(#2画)}}{#1{\tiny(#2画)}}
  \tl_set:Nn \l_hanzi_tl {#1}
  \tl_set:Nn \l_strokes_tl {#2}
  \begin{minipage}[t][][t]{\entrywidth}
   \StrLen{#1}[\hanzilen]
   \ifthenelse{\hanzilen > 1}
    {%
     \begin{lightbox}
      {\LARGE#1}\hfill\textsuperscript{\tiny(#2画)}
      \tcblower
      \IfValueT{#4}{\mbox{}\hfill{\tiny#4}}{}%
      \IfValueT{#5}{\mbox{}\hfill{\tiny#5}}{}%
      \IfValueT{#5}{\mbox{}\hfill{\tiny#6}}{}%
      \mbox{}\hfill\IfSubStr{#3}{、}{\tiny Radicais\ #3}{\tiny Radical\ #3}
     \end{lightbox}
    }%
    {%
     \begin{darkbox}
      {\LARGE#1}\hfill\textsuperscript{\tiny(#2画)}
      \tcblower
      \IfValueT{#4}{\mbox{}\hfill{\tiny#4}}{}%
      \IfValueT{#5}{\mbox{}\hfill{\tiny#5}}{}%
      \IfValueT{#5}{\mbox{}\hfill{\tiny#6}}{}%
      \mbox{}\hfill\IfSubStr{#3}{、}{\tiny Radicais\ #3}{\tiny Radical\ #3}
     \end{darkbox}
    }
 }%
 {%
  \end{minipage}
 }

 %%% Ambiente "Entry*", para os verbetes muito compridos
\NewDocumentEnvironment{Entry*}{mmmooo}%
 {%
  \leavevmode
  \markboth{#1{\tiny(#2画)}}{#1{\tiny(#2画)}}
  \tl_set:Nn \l_hanzi_tl {#1}
  \tl_set:Nn \l_strokes_tl {#2}
  \begin{minipage}[t][][t]{\entrywidth}
   \begin{lightbox}
    \mbox{}\hfill\textsuperscript{\tiny(#2画)}\\
    {\LARGE#1}
    \tcblower
    \IfValueT{#4}{\mbox{}\hfill{\tiny#4}}{}%
    \IfValueT{#5}{\mbox{}\hfill{\tiny#5}}{}%
    \IfValueT{#6}{\mbox{}\hfill{\tiny#6}}{}%
    \mbox{}\hfill\IfSubStr{#3}{、}{\tiny Radicais\ #3}{\tiny Radical\ #3}
   \end{lightbox}
 }%
 {%
  \end{minipage}
 }


%%% Ambiente "Phonetics", para as diversas entradas de fonética da palavra
\NewDocumentEnvironment{Phonetics}{mO{}mO{}O{}}%
 {%
  \tl_set:Nn \l_pinyin_tl {#3}
  \boolfalse{f_see} \renewcommand\seerefl{} \listadd{\seerefl}{}% Inicializa a lista
  \boolfalse{f_seealso} \renewcommand\seealsorefl{} \listadd{\seealsorefl}{}% Inicializa a lista
  \label{\l_label_tl : #1 : #3}
  \ding{93}\ #2\ \dpy{#3}\ #4\ \ding{93}\hfill \textbf{#5}\\
 }%
 {%  
  \ifbool{f_see}%
   {% Processa as referências "Veja"
    \RenewDocumentCommand\do{>{\SplitArgument{1}{:}}m}{\item \seeitem ##1}
    \textbf{Veja:}%
    \begin{seelist}
     \dolistloop{\seerefl}
    \end{seelist}
   }{}%
  \ifbool{f_seealso}%
   {% Processa as referências "Veja também"
    \RenewDocumentCommand\do{>{\SplitArgument{1}{:}}m}{\item \seeitem ##1}
    \textbf{Veja\ também:}%
    \begin{seealsolist}
     \dolistloop{\seealsorefl}
    \end{seealsolist}
   }{}%
 }

\ExplSyntaxOff

%%%%% EOF %%%%%
