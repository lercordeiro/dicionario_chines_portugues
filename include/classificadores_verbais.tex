\DefTblrTemplate{caption}{default}{}
\DefTblrTemplate{capcont}{default}{ \UseTblrTemplate{conthead-text}{default} }
\DefTblrTemplate{contfoot-text}{default}{Continua na próxima página.}
\DefTblrTemplate{conthead-text}{default}{(Continuação)}
\DefTblrTemplate{firsthead}{default}{ \UseTblrTemplate{caption}{default} }
\DefTblrTemplate{middlehead,lasthead}{default}{ \UseTblrTemplate{conthead}{default} }
\DefTblrTemplate{firstfoot,middlefoot}{default}{ \UseTblrTemplate{contfoot}{default} }
\DefTblrTemplate{lastfoot}{default}{ \UseTblrTemplate{note}{default} \UseTblrTemplate{remark}{default} }

\begin{longtblr}
{
  colspec = {|c|c|X|X|}, hlines,
  width = 1\linewidth,
  rowhead = 1, rowfoot = 0,
  row{1} = {font=\bfseries, fg=white, bg=black},
}
\textbf{Hanzi} & \textbf{Pinyin} & \textbf{Descrição}\\
    遍 & \dpy{bian4}  & o número de vezes que uma ação foi concluída \\
    场 & \dpy{chang3} & a duração de um evento ocorrendo dentro de outro evento\\
    次 & \dpy{ci4}    & vezes (ao contrário de 遍, 次 refere-se ao número de vezes, independente de ter sido concluído ou não)\\
    顿 & \dpy{dun4}   & ações sem repetição\\
    回 & \dpy{hui2}   & ocorrências (usado coloquialmente)\\
    声 & \dpy{sheng1} & gritos, expressões\\
    趟 & \dpy{tang4}  & viagens, visitas\\
    下 & \dpy{xia4}   & ações breves e frequentemente repentinas, muito mais comum em cantonês do que em dialetos do norte\\
\end{longtblr}
