
%%%%%%%%%%%%%%%%%%%%%%%%%%%%%%%%%%%%%%%%%
% LuaLaTex
%
% Dicionário Chinês -> Português
% Autor: Luiz Eduardo Roncato Cordeiro
%
% Licença:
% CC BY-NC-SA 3.0 (http://creativecommons.org/licenses/by-nc-sa/3.0/)
%%%%%%%%%%%%%%%%%%%%%%%%%%%%%%%%%%%%%%%%%

\documentclass[a4paper,9pt,twoside,openany]{memoir}

\usepackage[brazilian]{babel}
\usepackage{fontspec}
\usepackage[dvipsnames]{xcolor}
\usepackage{multicol}
\usepackage{imakeidx}
\usepackage[inline]{enumitem}
\usepackage{zhnumber}
\usepackage{tikz}
\usepackage[hyperindex,hidelinks]{hyperref}
\usepackage{pifont}
\usepackage{xstring}
\usepackage{tabularray}
\usepackage[most]{tcolorbox}
\usepackage{luacode}
\usepackage{parskip}
\usepackage{stackengine}

% Meus Comandos
%%%%%%%%%%%%%%%%%%%%%%%%%%%%%%%%%%%%%%%%%%%%%%%%%%%%%%%%%%%%%%%%%%%%%%%%%%%%%%%
%%%%%%%%%%%%%%%%%%%%%%%%%%%%%%%%%%%%%%%%%%%%%%%%%%%%%%%%%%%%%%%%%%%%%%%%%%%%%%%
%%%%%                                                                     %%%%%
%%%%% Funções e Ajustes dos Documentos do Dicionário                      %%%%%
%%%%%                                                                     %%%%%
%%%%%%%%%%%%%%%%%%%%%%%%%%%%%%%%%%%%%%%%%%%%%%%%%%%%%%%%%%%%%%%%%%%%%%%%%%%%%%%
%%%%%%%%%%%%%%%%%%%%%%%%%%%%%%%%%%%%%%%%%%%%%%%%%%%%%%%%%%%%%%%%%%%%%%%%%%%%%%%

%%% Hyperref em modo 'draft' não gera os hiperlinks
\hypersetup{final}

%%% Largura da entrada do verbete
\def\entrywidth{.49\textwidth}

%%% Estilo do capítulo, o melhor que encontrei
\chapterstyle{verville}

%%% Sem identação
\setlength\parindent{0pt}

%%% Ajuste das margens do documento
\setlrmarginsandblock{3cm}{2cm}{*}
\setulmarginsandblock{2cm}{*}{1}
\checkandfixthelayout

%%% Pra evitar viúvas e órfãs
\clubpenalty=10000
\widowpenalty=10000
\raggedbottom

%%% Espaçamento das linhas, 1 e meio
\OnehalfSpacing

%%% Usando a fonte NoTofu do Google.
\babelfont{rm}[
 Renderer=Node,
 Ligatures=TeX,
 BoldFont={NotoSerifCJKsc-SemiBold},
 BoldSlantedFont={NotoSerifCJKsc-SemiBold},
 AutoFakeSlant=0.25,
 SlantedFeatures={FakeSlant=0.25},
 BoldSlantedFeatures={FakeSlant=0.25}]
 {Noto Serif CJK SC Light}
\babelfont{sf}[Renderer=Harfbuzz,Ligatures=TeX]{Noto Sans CJK SC Light}
\babelfont{tt}[Renderer=Harfbuzz,Ligatures=TeX]{Noto Sans Mono CJK SC}

%%% Ajustes do Sumário
\makeatletter
\renewcommand{\@pnumwidth}{2em} 
\renewcommand{\@tocrmarg}{4em}
\makeatother
\renewcommand\cftbeforechapterskip{5pt plus 1pt}

%%% Ajustes da separação das colunas quando em modo texto de 2 colunas
\setlength{\columnsep}{.8em}
\setlength{\columnseprule}{0.1mm}

%%% Ajustes para o "stackengine"
\renewcommand\stacktype{S}
\renewcommand\stackalignment{c}

%%% Ajustes de Cabeçalhos e Rodapés
\setheadfoot{14pt}{28pt}

% Estilo "plain"
\makefootrule{plain}{\textwidth}{\normalrulethickness}{2pt}
\ifdraftdoc
 \makeevenfoot{plain}{\thepage}{汉葡词典}{Draft}
 \makeoddfoot{plain}{Draft}{汉葡词典}{\thepage}
\else
 \makeevenfoot{plain}{\thepage}{汉葡词典}{}
 \makeoddfoot{plain}{}{汉葡词典}{\thepage}
\fi

% Estilo "dictionary"
\makepagestyle{dictionary}
\makeheadrule{dictionary}{\textwidth}{\normalrulethickness}
\makefootrule{dictionary}{\textwidth}{\normalrulethickness}{2pt}
\ifdraftdoc
 \makeevenhead{dictionary}{\rightmark}{Draft}{\leftmark}
 \makeoddhead{dictionary}{\rightmark}{Draft}{\leftmark}
 \makeevenfoot{dictionary}{\thepage}{汉葡词典}{Draft}
 \makeoddfoot{dictionary}{Draft}{汉葡词典}{\thepage}
\else
 \makeevenhead{dictionary}{\rightmark}{}{\leftmark}
 \makeoddhead{dictionary}{\rightmark}{}{\leftmark}
 \makeevenfoot{dictionary}{\thepage}{汉葡词典}{}
 \makeoddfoot{dictionary}{}{汉葡词典}{\thepage}
\fi

%%% Estilo das Seções
\newcommand{\boxedsec}[1]
 {%
  \begin{tcolorbox}%
   [%
    %hbox,
    %before skip=2sp plus 1sp minus 1sp,
    %after skip=2sp plus 1sp minus 1sp,
    colframe=black,%
    colback=black!15!white,%
    boxrule=2pt,%
    leftrule=2mm,%
    left=0mm,%
    right=0mm,%
    top=0mm,%
    bottom=0mm%
   ]
   \hfill\LARGE\bfseries#1
  \end{tcolorbox}
 }
\setsecheadstyle{\boxedsec}
\setbeforesecskip{-2ex plus -.5ex minus -.25ex}
\setaftersecskip{.5ex plus .25ex}
\newcommand{\sectionbreak}{\phantomsection}

%%% Variáveis tipo "bool" para dizer se tem ou não os campos
%%% "Veja" e "Veja também" nas definições dos verbetes
\newbool{f_see}
\newbool{f_seealso}

%%% Converte os pinyins numéricos em pinyins com marcação de tom
\directlua{dofile "include/tex-sx-pinyin-tonemarks.lua"}

% Função "\pinyin" faz a conversão
\protected\def\pinyin#1{%
 \directlua{packagedata.pinyintones.convert ([==[#1]==])}%
}

% Comando "\dictpinyin", coloca o pinyin entre «»
\NewDocumentCommand{\dictpinyin}{m}{\guillemotleft\pinyin{#1}\guillemotright} 

% Comando "\dpy", gera a string do pinyin utilizada no Dicionário
% Este comando realiza uma série de substituições antes
\NewDocumentCommand{\dpy}{m}%
 {%
  \StrSubstitute{#1}{r5}{r}[\result]%
  \StrSubstitute{\result}{v}{ü}[\result]%
  \StrSubstitute{\result}{V}{Ü}[\result]%
  \edef\py{\dictpinyin{\result}}%
  \mbox{}\py
 }

%%% Comandos genéricos usados no Dicionário

% Comando "\&", insere o caracgter "&"
\DeclareRobustCommand{\&}%
 {%
  \ifdim\fontdimen1\font>0pt%
   \textsl{\symbol{`\&}}%
  \else%
   \symbol{`\&}%
  \fi%
 }

% Comando "\dul{text}", sublinha o texto dado
\NewDocumentCommand{\dul}{m}{\underline{#1}}

% Ambiente "enumerate" especial utilizado no dicionário, coloca as definições 
% do verbete em uma lista numerada em linha
\NewDocumentCommand{\dictenumerate}{>{\SplitList{|}}m}
 {%
  \begin{enumerate*}[nosep,label=(\arabic*),left=0pt,mode=unboxed,font=\bfseries]
   \ProcessList{#1}{\insertitem}
  \end{enumerate*}
 }
\NewDocumentCommand{\insertitem}{>{\TrimSpaces}m}{\item #1}

% Ambiente "enumerate" especial utilizado no dicionário, coloca os exemplos
% das definições do verbete em uma lista numerada em linha, utilizando
% algarismos romanos
\makeatletter
\NewDocumentCommand{\dictexamples}{m>{\SplitList{|}}m}
 {%
  \def\@theword{#1}%
  \begin{enumerate}[nosep,label=(\roman*),left=0pt,mode=unboxed,font=\bfseries]
   \ProcessList{#2}{\insertexample}
  \end{enumerate}
 }
\NewDocumentCommand{\insertexample}{>{\TrimSpaces}m}
 {
  \IfSubStr{#1}{mud::::}
   {% Sublinhado Manual
    \StrBehind{#1}{mud::::}[output]%
    \IfSubStr{\output}{___}
    {% Com traducao
     \StrCut{\output}{___}\csA\csB%
     \item\csA\\{\small``\csB''}
    }
    {% Sem traducao
     \item\output
    }
   }
   {% Sublinhado Automático
    \IfSubStr{#1}{___}
    {% Com traducao
      \StrCut{#1}{___}\csA\csB%
      \item\StrSubstitute{\csA}{\@theword}{\underline{\@theword}}\\{\small``\csB''}
    }
    {% Sem traducao
      \item\StrSubstitute{#1}{\@theword}{\underline{\@theword}}
    }
   }
 }  
\makeatother

%%%%% EOF %%%%%

%%%%%%%%%%%%%%%%%%%%%%%%%%%%%%%%%%%%%%%%%%%%%%%%%%%%%%%%%%%%%%%%%%%%%%%%%%%%%%%
%%%%%%%%%%%%%%%%%%%%%%%%%%%%%%%%%%%%%%%%%%%%%%%%%%%%%%%%%%%%%%%%%%%%%%%%%%%%%%%
%%%%%                                                                     %%%%%
%%%%% Funções e Ajustes dos Documentos do Dicionário                      %%%%%
%%%%%                                                                     %%%%%
%%%%%%%%%%%%%%%%%%%%%%%%%%%%%%%%%%%%%%%%%%%%%%%%%%%%%%%%%%%%%%%%%%%%%%%%%%%%%%%
%%%%%%%%%%%%%%%%%%%%%%%%%%%%%%%%%%%%%%%%%%%%%%%%%%%%%%%%%%%%%%%%%%%%%%%%%%%%%%%

%%% Espaçamento das linhas normal
\SingleSpacing

%%% Hyperref em modo 'draft' não gera os hiperlinks
\hypersetup{final}

%%% Largura da entrada do verbete
\def\entrywidth{.49\textwidth}

%%% Estilo do capítulo, o melhor que encontrei
\chapterstyle{verville}

%%% Sem identação
\setlength{\parindent}{0cm}
\setlength{\parskip}{0.6\baselineskip}

%%% Ajuste das margens do documento
\setlrmarginsandblock{3cm}{2cm}{*}
\setulmarginsandblock{2cm}{*}{1}
\checkandfixthelayout

%%% Pra evitar viúvas e órfãs
\clubpenalty=10000
\widowpenalty=10000
\raggedbottom

%%% Usando a fonte NoTofu do Google.
\babelfont{rm}[
 Renderer=Node,
 Ligatures=TeX,
 BoldFont={NotoSerifCJKsc-SemiBold},
 BoldSlantedFont={NotoSerifCJKsc-SemiBold},
 AutoFakeSlant=0.25,
 SlantedFeatures={FakeSlant=0.25},
 BoldSlantedFeatures={FakeSlant=0.25}]
 {Noto Serif CJK SC Light}
\babelfont{sf}[Renderer=Harfbuzz,Ligatures=TeX]{Noto Sans CJK SC Light}
\babelfont{tt}[Renderer=Harfbuzz,Ligatures=TeX]{Noto Sans Mono CJK SC}

%%% Ajustes do MultiCol: parar com a indentação do primeiro parágrafo
\AddToHook{env/multicols/begin}{\AddToHookNext{para/begin}{\OmitIndent}}

%%% Ajustes do Sumário
\makeatletter
\renewcommand{\@pnumwidth}{2em} 
\renewcommand{\@tocrmarg}{4em}
\makeatother
\renewcommand\cftbeforechapterskip{5pt plus 1pt}

%%% Ajustes da separação das colunas quando em modo texto de 2 colunas
\setlength{\columnsep}{0.8em}
\setlength{\columnseprule}{0.1mm}

%%% Ajustes para o "stackengine"
\renewcommand\stacktype{S}
\renewcommand\stackalignment{c}

%%% Ajustes de Cabeçalhos e Rodapés
\setheadfoot{14pt}{28pt}

% Estilo "plain"
\makefootrule{plain}{\textwidth}{\normalrulethickness}{2pt}
\ifdraftdoc
 \makeevenfoot{plain}{\thepage}{汉葡词典}{Draft}
 \makeoddfoot{plain}{Draft}{汉葡词典}{\thepage}
\else
 \makeevenfoot{plain}{\thepage}{汉葡词典}{}
 \makeoddfoot{plain}{}{汉葡词典}{\thepage}
\fi

% Estilo "dictionary"
\makepagestyle{dictionary}
\makeheadrule{dictionary}{\textwidth}{\normalrulethickness}
\makefootrule{dictionary}{\textwidth}{\normalrulethickness}{2pt}
\ifdraftdoc
 \makeevenhead{dictionary}{\rightmark}{Draft}{\leftmark}
 \makeoddhead{dictionary}{\rightmark}{Draft}{\leftmark}
 \makeevenfoot{dictionary}{\thepage}{汉葡词典}{Draft}
 \makeoddfoot{dictionary}{Draft}{汉葡词典}{\thepage}
\else
 \makeevenhead{dictionary}{\rightmark}{}{\leftmark}
 \makeoddhead{dictionary}{\rightmark}{}{\leftmark}
 \makeevenfoot{dictionary}{\thepage}{汉葡词典}{}
 \makeoddfoot{dictionary}{}{汉葡词典}{\thepage}
\fi

%%% Estilo das Seções
\newcommand{\boxedsec}[1]
 {%
  \begin{tcolorbox}%
   [%
    enhanced,%
    nobeforeafter,%
    before={\noindent},%
    colframe=black,%
    colback=black!15!white,%
    boxrule=2pt,%
    leftrule=2mm,%
    left=0mm,%
    right=0mm,%
    top=0mm,%
    bottom=0mm%
   ]
   \hfill\LARGE\bfseries#1
  \end{tcolorbox}
 }
\setsecheadstyle{\boxedsec}
\setbeforesecskip{1ex plus .25ex minus .25ex}
\setaftersecskip{.25ex plus .25ex minus .25ex}
\newcommand{\sectionbreak}{\phantomsection}

%%% Estilo das caixas dos verbetes
\newtcolorbox{lightbox}%
 {%
  enhanced,%
  size=fbox,%
  colframe=black,%
  colback=white,%
  boxrule=1pt,%
  toprule=3pt,%
  left=0mm,%
  right=0mm,%
  top=0mm,%
  bottom=0mm,%
  middle=0mm,%
  nobeforeafter,%
  segmentation empty,%
  before={\noindent}%
 }
\newtcolorbox{darkbox}%
 {%
  enhanced,%
  size=fbox,%
  colframe=black,%
  colback=black!5!white,%
  boxrule=1pt,%
  toprule=3pt,%
  left=0mm,%
  right=0mm,%
  top=0mm,%
  bottom=0mm,%
  middle=0mm,%
  nobeforeafter,%
  segmentation empty,%
  before={\noindent}%
 }


%%% Variáveis tipo "bool" para dizer se tem ou não os campos
%%% "Veja" e "Veja também" nas definições dos verbetes
\newbool{f_see}
\newbool{f_seealso}

%%% Converte os pinyins numéricos em pinyins com marcação de tom
\directlua{dofile "include/tex-sx-pinyin-tonemarks.lua"}

%%% Comandos genéricos usados no Dicionário

% Função "\pinyin" faz a conversão
\protected\def\pinyin#1{%
 \directlua{packagedata.pinyintones.convert ([==[#1]==])}%
}

% Comando "\dictpinyin", coloca o pinyin entre «»
\NewDocumentCommand{\dictpinyin}{m}{\guillemotleft\pinyin{#1}\guillemotright} 

% Comando "\dpy", gera a string do pinyin utilizada no Dicionário
% Este comando realiza uma série de substituições antes
\NewDocumentCommand{\dpy}{m}%
 {%
  \StrSubstitute{#1}{5}{}[\result]%
  \StrSubstitute{\result}{v}{ü}[\result]%
  \StrSubstitute{\result}{V}{Ü}[\result]%
  \edef\py{\dictpinyin{\result}}%
  \mbox{}\py
 }

% Comando "\&", insere o caracgter "&"
\DeclareRobustCommand{\&}%
 {%
  \ifdim\fontdimen1\font>0pt%
   \textsl{\symbol{`\&}}%
  \else%
   \symbol{`\&}%
  \fi%
 }

% Comando "\dul{text}", sublinha o texto dado
\NewDocumentCommand{\dul}{m}{\underline{#1}}

% Ambiente "enumerate" especial utilizado no dicionário, coloca as definições 
% do verbete em uma lista numerada em linha
\NewDocumentCommand{\dictenumerate}{>{\SplitList{|}}m}
 {%
  \begin{enumerate*}[nosep,label=(\arabic*),left=0pt,mode=unboxed,font=\bfseries]
   \ProcessList{#1}{\insertitem}
  \end{enumerate*}
 }
\NewDocumentCommand{\insertitem}{>{\TrimSpaces}m}{\item #1}

% Ambiente "enumerate" especial utilizado no dicionário, coloca os exemplos
% das definições do verbete em uma lista numerada em linha, utilizando
% algarismos romanos
\makeatletter
\NewDocumentCommand{\dictexamples}{m>{\SplitList{|}}m}
 {%
  \def\@theword{#1}%
  \begin{enumerate}[nosep,label=(\roman*),left=0pt,mode=unboxed,font=\bfseries]
   \ProcessList{#2}{\insertexample}
  \end{enumerate}
 }
\NewDocumentCommand{\insertexample}{>{\TrimSpaces}m}
 {
  \IfSubStr{#1}{mud::::}
   {% Sublinhado Manual
    \StrBehind{#1}{mud::::}[output]%
    \IfSubStr{\output}{___}
    {% Com traducao
     \StrCut{\output}{___}\csA\csB%
     \item\csA\\{\small``\csB''}
    }
    {% Sem traducao
     \item\output
    }
   }
   {% Sublinhado Automático
    \IfSubStr{#1}{___}
    {% Com traducao
      \StrCut{#1}{___}\csA\csB%
      \item\StrSubstitute{\csA}{\@theword}{\underline{\@theword}}\\{\small``\csB''}
    }
    {% Sem traducao
      \item\StrSubstitute{#1}{\@theword}{\underline{\@theword}}
    }
   }
 }  
\makeatother

\ExplSyntaxOn

%%% Cria listas especializadas (seelist e seealsolist)
\newlist{seelist}{enumerate}{1}
\newlist{seealsolist}{enumerate}{1}

\setlist[seelist]{label={(\alph*)},topsep=0pt,nosep,noitemsep}%,leftmargin=\parindent}
\setlist[seealsolist]{label={(\alph*)},topsep=0pt,nosep,noitemsep}%,leftmargin=\parindent}

%%% Cria e inicializa a lista "\seerefl", "Veja"
\newcommand\seerefl{}
\listadd{\seerefl}{}% Inicializa a lista

%%% Cria e inicializa a lista "\seealsorefl", "Veja também"
\newcommand\seealsorefl{}
\listadd{\seealsorefl}{}% Inicializa a lista

%%% Comando "\seeitem", adiciona um item "Veja" ou "Veja também" na lista,
%%% com os pinyins abaixo dos caracteres
\newcommand{\seeitem}[2]{#1~\dpy{#2}\ (p.~\pageref{#1:#2})}

%%% Comando "\definition", gera o texto da definição
\NewDocumentCommand{\definition}{sommo}
 {%
  \IfBooleanTF{#1}%
   {% Substantivo Próprio
    {\small\ding{108}}\ (\textit{S.P.})\IfValueT{#2}{~[clas.:~#2]}{\ \dictenumerate{#4}}\par
   }%
   {%
    {\small\ding{108}}\ (\textit{#3})\IfValueT{#2}{~[clas.: #2]}{\ \dictenumerate{#4}}\par
   }%
  \IfValueT{#5}%
   {%
    \IfSubStr{#5}{|}{\textbf{Exemplos:}}{\textbf{Exemplo:}}\dictexamples{\l_hanzi_tl}{#5}
   }%
 }

%%% Comando "Variante de"
\NewDocumentCommand{\variantof}{m}
 {
  {\small\ding{108}}\ Variante\ de\ #1\ (p.~\pageref{#1:\l_pinyin_tl})\par
 }

%%% Comando "Veja"
\NewDocumentCommand{\seeref}{mm}
 {%
  \booltrue{f_see}
  \listgadd{\seerefl}{#1:#2}
 }

%%% Comando "Veja também"
\NewDocumentCommand{\seealsoref}{mm}
 {%
  \booltrue{f_seealso}
  \listgadd{\seealsorefl}{#1:#2}
 }

\ExplSyntaxOff

%%%%% EOF %%%%%

%%%%%                         %%%%%
%%%%% Dictionary environments %%%%%
%%%%%                         %%%%%

\ExplSyntaxOn

%%%%% "entry" environment
\NewDocumentEnvironment{entry}{mmooo}%
 {%
  \leavevmode%
  \markboth{#1{\tiny(#2画)}}{#1{\tiny(#2画)}}
  \tl_set:Nn \l_hanzi_tl {#1}
  \tl_set:Nn \l_strokes_tl {#2}
  \begin{minipage}[t][][t]{.49\textwidth}
   \vspace{2sp}
   \begin{tcolorbox}[size=title,colframe=black,colback=white,boxrule=1pt,toprule=2pt,left=0mm,right=0mm,top=0mm,bottom=0mm]
    {\LARGE#1}\hfill\textsuperscript{\tiny(#2画)}\\
    \IfValueT{#3}{\mbox{}\hfill{\tiny#3}}{}%
    \IfValueT{#4}{\mbox{}\hfill{\tiny#4}}{}%
    \IfValueT{#5}{\mbox{}\hfill{\tiny#5}}{}
   \end{tcolorbox}
 }%
 {%
  \end{minipage}
 }

%%%%% "entry*" environment
\NewDocumentEnvironment{entry*}{mmooo}%
 {%
  \leavevmode%
  \markboth{#1{\tiny(#2画)}}{#1{\tiny(#2画)}}
  \tl_set:Nn \l_hanzi_tl {#1}
  \tl_set:Nn \l_strokes_tl {#2}
  \begin{minipage}[t][][t]{.49\textwidth}
   \vspace{2sp}
   \begin{tcolorbox}[size=title,colframe=black,colback=white,boxrule=1pt,toprule=2pt,left=0mm,right=0mm,top=0mm,bottom=0mm]
    \mbox{}\hfill\textsuperscript{\tiny(#2画)}\\
    {\LARGE#1}\\
    \IfValueT{#3}{\mbox{}\hfill{\tiny#3}}{}%
    \IfValueT{#4}{\mbox{}\hfill{\tiny#4}}{}%
    \IfValueT{#5}{\mbox{}\hfill{\tiny#5}}{}
   \end{tcolorbox}
 }%
 {%
  \end{minipage}
 }

%%%%% "phonetics" environment
\NewDocumentEnvironment{phonetics}{mO{}mO{}O{}}%
 {%
  \tl_set:Nn \l_pinyin_tl {#3}
  \boolfalse{f_example} \renewcommand\examplel{} \listadd{\examplel}{}% Initialize list
  \boolfalse{f_see} \renewcommand\seerefl{} \listadd{\seerefl}{}% Initialize list
  \boolfalse{f_seealso} \renewcommand\seealsorefl{} \listadd{\seealsorefl}{}% Initialize list
  \label{#1:#3}
  \index[sradical]{\l_hanzi_tl@\l_hanzi_tl \dpy{#3}}
  \ding{93}\ #2\ \dpy{#3}\ #4\ \ding{93}\hfill #5\\
 }
 {%  
  \ifbool{f_example}%
   {% Process "examples"
    \RenewDocumentCommand\do{m}%                                                                                                
     {%
      \IfSubStr{##1}{manual::::}%
       {% Manual underline
        \StrBehind{##1}{manual::::}[\sM]%
        \IfSubStr{\sM}{::::}%
         {% With translation
          \StrCut{\sM}{::::}{\sE}{\sT}%
          \mbox{}\enskip\sE\\
          \mbox{}\enskip$\hookrightarrow$\ \sT\\
         }%
         {% Without translation
          \mbox{}\enskip\sM\\
         }%
       }%
       {%
        \IfSubStr{##1}{::::}%
         {% With translation
          \StrCut{##1}{::::}{\sE}{\sT}%
          \mbox{}\enskip\StrSubstitute{\sE}{\l_hanzi_tl}{\underline{\l_hanzi_tl}}\\
          \mbox{}\enskip$\hookrightarrow$\ \sT\\
         }
         {%
          \IfSubStr{##1}{ERRO}{##1}%
           {% There aren't entry words in text
            \mbox{}\enskip\StrSubstitute{##1}{\l_hanzi_tl}{\underline{\l_hanzi_tl}}\\
           }%
         }%
       }%
     }%
    \textbf{Exemplos:}\\
    \dolistloop{\examplel}
   }{}%
  \ifbool{f_see}%
   {% Process "see" references
    \RenewDocumentCommand\do{>{\SplitArgument{1}{:}}m}{\item \seeitem ##1}
    \textbf{Veja:\ }%
    \begin{itemize*}[label={}, itemjoin={{,\ }}, itemjoin*={{\ e~}}]
     \dolistloop{\seerefl}
    \end{itemize*}\par
   }{}%
  \ifbool{f_seealso}%
   {% Process "seealso" references
    \RenewDocumentCommand\do{>{\SplitArgument{1}{:}}m}{\item \seeitem ##1}
    \textbf{Veja\ também:\ }%
    \begin{itemize*}[label={}, itemjoin={{,\ }}, itemjoin*={{\ e~}}]
     \dolistloop{\seealsorefl}
    \end{itemize*}\par
   }{}%
 }

\ExplSyntaxOff

%%%%% EOF %%%%%


% Ajustes do PDF
\hypersetup{
  linktoc=page,
  colorlinks=true,
  urlcolor=blue,
  linkcolor=blue,
  citecolor=blue,
  pdftitle={汉葡词典 - Dicionário Chinês-Português},
  pdfsubject={Dicionário Chinês-Português -- Ordenado por Número de Traços},
  pdfauthor={Luiz Eduardo Roncato Cordeiro, AKA 罗学凯},
  pdfkeywords={dicionário, chinês, português, instituto confúcio}
}

% Índices Remissivos
\makeindex[title=Índice Remissivo por Radical,intoc=true,columns=3,columnsep=15pt,columnseprule=true,noautomatic=true,name=sradical]
\indexsetup{level=\chapter*,toclevel=chapter,headers={\indexname}{\indexname}}

%%%
%%% Documento começa aqui!
%%%

\begin{document}
\addfontfeatures{CharacterWidth=Proportional}

\begin{titlingpage}
  \raggedleft
  \rule{1pt}{\textheight}
  \hspace{0.1\textwidth}
  \parbox[b]{0.75\textwidth}{
    \vspace{0.05\textheight}
    {\HUGE\bfseries 汉葡词典}\\[2\baselineskip] % Title
    {\Large\textsc{Dicionário Chinês-Português}\\%
     \large\textsc{\zhtoday}}\\% Date
    [4\baselineskip]
    {\Large\textsc{罗学凯}\\%
     \small Luiz Eduardo Roncato Cordeiro\\% Author
            Aluno do Instituto Confúcio da UNESP}\\%
    \vspace{0.5\textheight}\\%
    {Instituto Confúcio, Curso de Chinês}\\[\baselineskip] % Publisher?
  }
  \newpage
  \raggedright
  \setlength{\parindent}{0pt}
  \setlength{\parskip}{\baselineskip}
  \mbox{}
  \vfill
  \footnotesize
  \textcopyright{} 2024 por Luiz Eduardo Roncato Cordeiro, está licenciado sob CC BY-NC-SA 4.0\\
  \begin{itemize}
    \item Para visualizar uma cópia desta licença, visite:\\ \url{http://creativecommons.org/licenses/by-nc-sa/4.0/}
    \item Este trabalho ainda está em andamento e o ``código fonte'' está localizado em:\\ \url{https://github.com/lercordeiro/dicionario_chines_portugues}
    \item A última versão compilada também pode ser encontrada em:\\ \url{https://ler.cordeiro.nom.br/}
  \end{itemize}
%   \begin{tabular}{ll}
%   First edition: & T.B.D. \\
%   \end{tabular}
\end{titlingpage}


\clearpage
\pagestyle{empty}
\tableofcontents

\clearpage
\pagestyle{empty}
\chapter{汉葡词典}

%%%%%%%%%%%%%%%%%%%%%%%%
%
% https://en.wikipedia.org/wiki/Chinese_character_orders
%
%%%%%%%%%%%%%%%%%%%%%%%%

Dicionário Chinês-Português ordenado primeiro pelo número de traços
de cada caracter, depois pela ordem do caracter na tabela UTF-8.
As definilções são ordenadas e agrupadas pelo pinyin em cada verbete.

\clearpage
\pagestyle{dicionario}
\begin{multicols}{2}
%%%
%%% 0画
%%%

\section*{0画}\addcontentsline{toc}{section}{0画}

\begin{entry}{T-恤}{0,9}
  \begin{phonetics}{T-恤}{[t]-xu4}
    \definition{s.}{camiseta | pulôver | suéter}
  \end{phonetics}
\end{entry}

%%%%% EOF %%%%%


%%%
%%% 1画
%%%

\section*{1画}\addcontentsline{toc}{section}{1画}

\begin{entry}{一}{1}{⼀}[Kangxi 1]
  \begin{phonetics}{一}{yi1}[(quando usado sozinho)][HSK 1]
    \definition{adv.}{uma vez; assim que; indica que duas ações ocorreram em um intervalo de tempo muito curto, uma terminando e a outra começando imediatamente em seguida | indica que primeiro se realiza uma ação e, em seguida, o resultado dessa ação  | indica uma ação única, indicando que a ação é muito curta ou apenas uma tentativa}
    \definition{num.}{um; 1 | pronunciado como \dpy{yao1} quando dito número a número | igual; refere-se ao mesmo ou igual | inteiro; todo; por toda parte | exclusivo ou único | refere-se a algo específico | também; caso contrário; referindo-se a outro ou mais um}
    \definition{part.}{antes de certas palavras para dar ênfase}
    \definition{prep.}{cada; por; toda vez}
    \definition{s.}{uma nota da escala em Gongchepu (工尺谱), correspondente ao 17 na notação musical numerada}
  \seealsoref{工尺谱}{gong1 che3 pu3}
  \end{phonetics}
  \begin{phonetics}{一}{yi2}[(antes de quarto tom)][HSK 1]
    \definition{num.}{um; 1 | um (artigo)}
  \end{phonetics}
  \begin{phonetics}{一}{yi4}[][HSK 1]
    \definition{adv.}{uma vez | assim que | ao}
    \definition{num.}{um; 1 | um (artigo)}
  \end{phonetics}
\end{entry}

\begin{entry}{一下}{1,3}{⼀、⼀}
  \begin{phonetics}{一下}{yi2xia4}
    \definition{adv.}{em um curto tempo | rapidamente}
  \end{phonetics}
\end{entry}

\begin{entry}{一下儿}{1,3,2}{⼀、⼀、⼉}
  \begin{phonetics}{一下儿}{yi2 xia4r5}[][HSK 1,5]
    \definition{s.}{um tempo; um momento}
  \end{phonetics}
\end{entry}

\begin{entry}{一下子}{1,3,3}{⼀、⼀、⼦}
  \begin{phonetics}{一下子}{yi2 xia4 zi5}[][HSK 5]
    \definition{adv.}{tudo de uma vez; de repente; em pouco tempo; em um curto espaço de tempo}
  \end{phonetics}
\end{entry}

\begin{entry}{一个样}{1,3,10}{⼀、⼈、⽊}
  \begin{phonetics}{一个样}{yi2ge5yang4}
    \definition{adj.}{igual | mesmo}
  \seealsoref{一样}{yi2yang4}
  \end{phonetics}
\end{entry}

\begin{entry}{一口气}{1,3,4}{⼀、⼝、⽓}
  \begin{phonetics}{一口气}{yi4 kou3 qi4}[][HSK 5]
    \definition{adv.}{em um só fôlego; sem pausa}
  \end{phonetics}
\end{entry}

\begin{entry}{一切}{1,4}{⼀、⼑}
  \begin{phonetics}{一切}{yi2qie4}[][HSK 3]
    \definition{pron.}{tudo; todo; todas as coisas}
  \end{phonetics}
\end{entry}

\begin{entry}{一方面}{1,4,9}{⼀、⽅、⾯}
  \begin{phonetics}{一方面}{yi4 fang1 mian4}[][HSK 3]
    \definition{s.}{um lado; um dos dois aspectos opostos ou um lado de algo que está relacionado a outro}
  \end{phonetics}
\end{entry}

\begin{entry}{一方面……,一方面……}{1,4,9,1,4,9}{⼀、⽅、⾯、⼀、⽅、⾯}
  \begin{phonetics}{一方面……,一方面……}{yi4 fang1 mian4 yi4 fang1 mian4}[][HSK 3]
    \definition{conj.}{por um lado\dots, por outro lado\dots; conecta duas orações paralelas (devem ser usadas juntas)}
  \end{phonetics}
\end{entry}

\begin{entry}{一半}{1,5}{⼀、⼗}
  \begin{phonetics}{一半}{yi2ban4}[][HSK 1]
    \definition{num.}{metade; em parte; uma metade}
  \end{phonetics}
\end{entry}

\begin{entry}{一句话}{1,5,8}{⼀、⼝、⾔}
  \begin{phonetics}{一句话}{yi2 ju4 hua4}[][HSK 5]
    \definition{s.}{em resumo; em uma palavra; expressar um conteúdo complexo de forma sucinta | trabalho fácil; fácil de fazer; descrever uma tarefa ou trabalho como muito simples e fácil de realizar}
  \end{phonetics}
\end{entry}

\begin{entry}{一旦}{1,5}{⼀、⽇}
  \begin{phonetics}{一旦}{yi2dan4}[][HSK 5]
    \definition{adv.}{uma vez; no caso; agora que | de repente; uma vez}
    \definition{s.}{em um único dia; em um tempo muito curto;}
  \end{phonetics}
\end{entry}

\begin{entry}{一生}{1,5}{⼀、⽣}
  \begin{phonetics}{一生}{yi4 sheng1}[][HSK 2]
    \definition{s.}{vida inteira; toda a vida; ao longo da vida; todo o tempo desde o nascimento até a morte; às vezes exagerado para indicar um longo período de tempo no curso da vida}
  \end{phonetics}
\end{entry}

\begin{entry}{一边}{1,5}{⼀、⾡}
  \begin{phonetics}{一边}{yi4bian1}[][HSK 1]
    \definition{adj.}{igual; idêntico; da mesma forma}
    \definition{adv.}{enquanto; ao mesmo tempo; simultaneamente; indica que uma ação ocorre simultaneamente a outra ação}
    \definition{s.}{lado; um lado; um aspecto | ambos os lados; ao lado de}
  \end{phonetics}
\end{entry}

\begin{entry}{一会儿}{1,6,2}{⼀、⼈、⼉}
  \begin{phonetics}{一会儿}{yi2 hui4r5}[][HSK 1,2]
    \definition{adv.}{agora\dots agora\dots; um momento\dots o próximo\dots; usado antes de dois antônimos, indica a alternância de situações}
    \definition{s.}{um pouquinho de tempo; muito pouco tempo}
  \end{phonetics}
\end{entry}

\begin{entry}{一共}{1,6}{⼀、⼋}
  \begin{phonetics}{一共}{yi2gong4}[][HSK 2]
    \definition{adv.}{completamente; em tudo; no todo}
  \end{phonetics}
\end{entry}

\begin{entry}{一再}{1,6}{⼀、⼌}
  \begin{phonetics}{一再}{yi2zai4}[][HSK 4]
    \definition{adv.}{repetidamente; de novo e de novo; repetidas vezes}
  \end{phonetics}
\end{entry}

\begin{entry}{一同}{1,6}{⼀、⼝}
  \begin{phonetics}{一同}{yi4tong2}
    \definition{adv.}{juntos, ao mesmo tempo}
  \end{phonetics}
\end{entry}

\begin{entry}{一向}{1,6}{⼀、⼝}
  \begin{phonetics}{一向}{yi2xiang4}[][HSK 5]
    \definition{adv.}{desde o início; indica do passado até o presente}
  \end{phonetics}
\end{entry}

\begin{entry}{一行}{1,6}{⼀、⾏}
  \begin{phonetics}{一行}{yi1xing2}
    \definition{s.}{festa | delegação}
  \end{phonetics}
\end{entry}

\begin{entry}{一齐}{1,6}{⼀、⿑}
  \begin{phonetics}{一齐}{yi4qi2}
    \definition{adv.}{tudo ao mesmo tempo | em uníssono | junto}
  \end{phonetics}
\end{entry}

\begin{entry}{一块}{1,7}{⼀、⼟}
  \begin{phonetics}{一块}{yi2kuai4}
    \definition{adv.}{(principalmente mandarim) juntos}
  \end{phonetics}
\end{entry}

\begin{entry}{一块儿}{1,7,2}{⼀、⼟、⼉}
  \begin{phonetics}{一块儿}{yi2 kuai4r5}[][HSK 1]
    \definition{adv.}{juntos; em conjunto}
    \definition{s.}{no mesmo lugar; no mesmo local}
  \end{phonetics}
\end{entry}

\begin{entry}{一时}{1,7}{⼀、⽇}
  \begin{phonetics}{一时}{yi4shi2}
    \definition{adv.}{por pouco tempo | por um tempo | temporariamente | momentaneamente | uma vez | de tempos em tempos | ocasionalmente}
  \end{phonetics}
\end{entry}

\begin{entry}{一身}{1,7}{⼀、⾝}
  \begin{phonetics}{一身}{yi4 shen1}[][HSK 5]
    \definition{s.}{o corpo inteiro; em todo o corpo | um terno; (um conjunto completo de) roupas | sozinho; uma única pessoa; relativo a uma única pessoa}
  \end{phonetics}
\end{entry}

\begin{entry}{一些}{1,8}{⼀、⼆}
  \begin{phonetics}{一些}{yi4 xie1}[][HSK 1]
    \definition{clas.}{alguns; um número de; quantidade indeterminada | um pouco; uma pequena quantidade | mais de um; mais de uma vez; indica mais de um ou mais de uma vez, etc. | uma ligeira mudança no grau, intensidade; usado após certos verbos, adjetivos, etc., para indicar uma quantidade muito pequena}
    \definition{pron.}{uns; alguns}
  \end{phonetics}
\end{entry}

\begin{entry}{一定}{1,8}{⼀、⼧}
  \begin{phonetics}{一定}{yi2ding4}[][HSK 2]
    \definition{adj.}{certo; particular; tendo um certo nível de especificidade; (objeto, situação) determinado em um ou mais | devido; certo; sempre foi assim, não vai mudar | fixo; especificado; há requisitos claros quanto à maneira, método, quantidade, etc.}
    \definition{adv.}{certamente; necessariamente; expressando determinação ou certeza | certamente; indica especulação ou avaliação de que um evento ou situação definitivamente acontecerá ou realmente existirá}
  \end{phonetics}
\end{entry}

\begin{entry}{一直}{1,8}{⼀、⽬}
  \begin{phonetics}{一直}{yi4zhi2}[][HSK 2]
    \definition{adv.}{direto; indica que permanece inalterado em uma direção | sempre; continuamente; o tempo todo; o tempo todo; indica que a ação é sempre ininterrupta ou o estado é sempre inalterado | de um ponto a outro sem enfatizar nenhuma exceção}
  \end{phonetics}
\end{entry}

\begin{entry}{一带}{1,9}{⼀、⼱}
  \begin{phonetics}{一带}{yi2 dai4}[][HSK 5]
    \definition{s.}{a área em torno de um determinado local; refere-se a um determinado local e suas proximidades.}
  \end{phonetics}
\end{entry}

\begin{entry}{一律}{1,9}{⼀、⼻}
  \begin{phonetics}{一律}{yi2lv4}[][HSK 4]
    \definition{adj.}{igual; semelhante; uniforme; parecido; idêntico}
    \definition{adv.}{todos; tudo; sem exceção; enfatiza que todos devem ser assim, sem exceção, e é usado principalmente em regulamentos ou requisitos}
  \end{phonetics}
\end{entry}

\begin{entry}{一战}{1,9}{⼀、⼽}
  \begin{phonetics}{一战}{yi2zhan4}
    \definition*{s.}{Primeira Guerra Mundial}
  \end{phonetics}
\end{entry}

\begin{entry}{一点儿}{1,9,2}{⼀、⽕、⼉}
  \begin{phonetics}{一点儿}{yi4dian3r5}[][HSK 1]
    \definition{adv.}{um pouco; uma pitada; uma gota; uma amostra; uma pequena quantidade; ({adj.} + (一)点儿, 一点儿 + {s.} ou 有 + (一)点儿 + {s.})}
  \end{phonetics}
\end{entry}

\begin{entry}{一点点}{1,9,9}{⼀、⽕、⽕}
  \begin{phonetics}{一点点}{yi4 dian3 dian3}[][HSK 2]
    \definition{adj.}{um pouco; muito pouco ou um pouquinho}
  \end{phonetics}
\end{entry}

\begin{entry}{一样}{1,10}{⼀、⽊}
  \begin{phonetics}{一样}{yi2yang4}[][HSK 1]
    \definition{adj.}{o mesmo; igualmente; semelhante; tão\dots quanto\dots}
    \definition{part.}{na mesma medida; anexado a verbos ou palavras nominais, indica uma comparação ou semelhança, equivalente a 似的}
  \seealsoref{似的}{shi4de5}
  \end{phonetics}
\end{entry}

\begin{entry}{一流}{1,10}{⼀、⽔}
  \begin{phonetics}{一流}{yi4liu2}[][HSK 5]
    \definition{adj.}{clássico; de primeira linha; de primeira classe; o melhor}
    \definition[些]{s.}{tipo; mesmo tipo; da mesma classe; da mesma categoria; uma categoria}
  \end{phonetics}
\end{entry}

\begin{entry}{一致}{1,10}{⼀、⾄}
  \begin{phonetics}{一致}{yi2zhi4}[][HSK 4]
    \definition{adj.}{equado; idêntico; uniforme; unânime; nenhuma diferença (de opinião ou ação)}
    \definition{adv.}{juntos; em conjunto}
  \end{phonetics}
\end{entry}

\begin{entry}{一般}{1,10}{⼀、⾈}
  \begin{phonetics}{一般}{yi4ban1}[][HSK 2]
    \definition{adj.}{o mesmo que; exatamente como | geral; ordinário; comum | médio; medíocre; o grau ou nível não é muito alto}
    \definition{adv.}{frequentemente; geralmente}
  \end{phonetics}
\end{entry}

\begin{entry}{一般来说}{1,10,7,9}{⼀、⾈、⽊、⾔}
  \begin{phonetics}{一般来说}{yi4 ban1 lai2 shuo1}[][HSK 4]
    \definition{expr.}{de modo geral; na média; no caso usual; a declaração usual}
  \end{phonetics}
\end{entry}

\begin{entry}{一起}{1,10}{⼀、⾛}
  \begin{phonetics}{一起}{yi4qi3}[][HSK 1]
    \definition{adv.}{juntos; em companhia; indica o mesmo local, ao mesmo tempo que se faz algo | no total; em todos; no conjunto}
    \definition{s.}{no mesmo lugar}
  \end{phonetics}
\end{entry}

\begin{entry}{一部分}{1,10,4}{⼀、⾢、⼑}
  \begin{phonetics}{一部分}{yi2 bu4 fen4}[][HSK 2]
    \definition{adj.}{parcial}
    \definition{adv.}{parcialmente}
    \definition{num.}{parte; porção; seção; fração}
  \end{phonetics}
\end{entry}

\begin{entry}{一……就……}{1,12}{⼀、⼪}
  \begin{phonetics}{一……就……}{yi1 jiu4}
    \definition{expr.}{logo que |  uma vez que}
  \end{phonetics}
\end{entry}

\begin{entry}{一辈子}{1,12,3}{⼀、⾞、⼦}
  \begin{phonetics}{一辈子}{yi2bei4zi5}[][HSK 5]
    \definition{s.}{uma vida inteira; vida inteira; toda a vida; durante toda a vida; enquanto se vive; todo o tempo entre o nascimento e a morte}
  \end{phonetics}
\end{entry}

\begin{entry}{一道}{1,12}{⼀、⾡}
  \begin{phonetics}{一道}{yi2dao4}
    \definition{adv.}{juntos | ao lado}
  \end{phonetics}
\end{entry}

\begin{entry}{一路}{1,13}{⼀、⾜}
  \begin{phonetics}{一路}{yi2 lu4}[][HSK 5]
    \definition{adv.}{o tempo todo; persistentemente; continuamente | juntos; sem parar; continuamente}
    \definition{s.}{o mesmo caminho; a mesma rota; ao longo de toda a viagem, ao longo do caminho | do mesmo tipo; da mesma categoria}
  \end{phonetics}
\end{entry}

\begin{entry}{一路平安}{1,13,5,6}{⼀、⾜、⼲、⼧}
  \begin{phonetics}{一路平安}{yi2 lu4 ping2 an1}[][HSK 2]
    \definition{expr.}{Boa viagem!; Tenha uma boa viagem!}
    \definition{v.}{ter uma viagem agradável}
  \end{phonetics}
\end{entry}

\begin{entry}{一路顺风}{1,13,9,4}{⼀、⾜、⾴、⾵}
  \begin{phonetics}{一路顺风}{yi2 lu4 shun4 feng1}[][HSK 2]
    \definition{expr.}{ter uma viagem agradável; toda a viagem foi segura e tranquila; é uma metáfora para cada etapa do processo de lidar com algo que ocorre sem problemas | Tenha uma boa viagem!; Boa viagem!}
  \end{phonetics}
\end{entry}

\begin{entry}{乙}{1}{⼄}
  \begin{phonetics}{乙}{yi3}[][HSK 5]
    \definition*{s.}{sobrenome Yi}
    \definition*{s.}{o segundo lugar do Tian Gan}
    \definition{num.}{segundo}
    \definition{s.}{uma nota da escala em Gongchepu (工尺谱); nível superior na música tradicional chinesa}
  \seealsoref{工尺谱}{gong1 che3 pu3}
  \end{phonetics}
\end{entry}

%%%%% EOF %%%%%


%%%
%%% 2画
%%%

\section*{2画}\addcontentsline{toc}{section}{2画}

\begin{entry}{七}{2}{⼀}
  \begin{phonetics}{七}{qi1}[][HSK 1]
    \definition{num.}{sete; 7}
  \end{phonetics}
\end{entry}

\begin{entry}{七夕}{2,3}{⼀、⼣}
  \begin{phonetics}{七夕}{qi1xi1}
    \definition*{s.}{Dia dos Namorados Chinês, quando o vaqueiro e a tecelã (牛郎织女) têm permissão para se reunirem anualmente | Festival das Meninas | Festival Duplo Sete, noite do sétimo mês lunar}
    \seeref{牛郎织女}{niu2lang2zhi1nv3}
  \end{phonetics}
\end{entry}

\begin{entry}{九}{2}{⼄}
  \begin{phonetics}{九}{jiu3}[][HSK 1]
    \definition{num.}{nove; 9}
  \end{phonetics}
\end{entry}

\begin{entry}{了}{2}{⼅}
  \begin{phonetics}{了}{le5}[][HSK 1]
    \definition{part.}{usada depois de verbos ou adjetivos para indicar que uma ação ou mudança foi concluída | usada no final de uma frase ou em uma pausa na frase, indica uma mudança, significa o surgimento de uma nova situação e expressa uma insistência ou um conselho contra algo}
  \end{phonetics}
  \begin{phonetics}{了}{liao3}
    \definition{v.}{terminar | alcançar | entender claramente}
  \end{phonetics}
  \begin{phonetics}{了}{liao4}
    \definition{adj.}{brilhantes (olhos)}
    \definition{v.}{observar | olhar para fora | olhar para baixo de um lugar mais alto | compreender claramente}
  \end{phonetics}
\end{entry}

\begin{entry}{了不起}{2,4,10}{⼅、⼀、⾛}
  \begin{phonetics}{了不起}{liao3bu5qi3}[][HSK 4]
    \definition{adj.}{incrível; fantástico; extraordinário | sério; grave}
  \end{phonetics}
\end{entry}

\begin{entry}{了解}{2,13}{⼅、⾓}
  \begin{phonetics}{了解}{liao3jie3}[][HSK 4]
    \definition{v.}{entender; compreender | investigar; indagar sobre}
  \end{phonetics}
\end{entry}

\begin{entry}{二}{2}{⼆}[Kangxi 7]
  \begin{phonetics}{二}{er4}[][HSK 1]
    \definition{num.}{dois; 2 | (dialeto de Pequim) estúpido}
  \end{phonetics}
\end{entry}

\begin{entry}{二手}{2,4}{⼆、⼿}
  \begin{phonetics}{二手}{er4 shou3}[][HSK 4]
    \definition{adj.}{usado; de segunda mão; refere-se especificamente a usados e revendidos}
  \end{phonetics}
\end{entry}

\begin{entry}{二战}{2,9}{⼆、⼽}
  \begin{phonetics}{二战}{er4zhan4}
    \definition*{s.}{Segunda Guerra Mundial}
  \end{phonetics}
\end{entry}

\begin{entry}{人}{2}{⼈}[Kangxi 9]
  \begin{phonetics}{人}{ren2}[][HSK 1]
    \definition[个,位]{s.}{pessoa | gente}
  \end{phonetics}
\end{entry}

\begin{entry}{人口}{2,3}{⼈、⼝}
  \begin{phonetics}{人口}{ren2kou3}[][HSK 2]
    \definition{s.}{pessoas | população}
  \end{phonetics}
\end{entry}

\begin{entry}{人工}{2,3}{⼈、⼯}
  \begin{phonetics}{人工}{ren2gong1}[][HSK 3]
    \definition{adj.}{feito pelo homem; artificial}
    \definition[个]{s.}{trabalho manual; trabalho feito à mão | mão de obra; homem-dia; uma unidade de cálculo da quantidade de trabalho realizado}
  \end{phonetics}
\end{entry}

\begin{entry}{人才}{2,3}{⼈、⼿}
  \begin{phonetics}{人才}{ren2cai2}[][HSK 3]
    \definition{adj.}{aparência bonita, elegante}
    \definition[个]{s.}{talento; pessoal qualificado; pessoa com capacidade}
  \end{phonetics}
\end{entry}

\begin{entry}{人们}{2,5}{⼈、⼈}
  \begin{phonetics}{人们}{ren2 men5}[][HSK 2]
    \definition{s.}{homens |  pessoas | o público}
  \end{phonetics}
\end{entry}

\begin{entry}{人民}{2,5}{⼈、⽒}
  \begin{phonetics}{人民}{ren2 min2}[][HSK 3]
    \definition[群,批,个]{s.}{o povo}
  \end{phonetics}
\end{entry}

\begin{entry}{人民币}{2,5,4}{⼈、⽒、⼱}
  \begin{phonetics}{人民币}{ren2min2bi4}[][HSK 3]
    \definition*[块,张,元]{s.}{Renminbi (RMB); Yuan Chinês (CYN); nome da moeda chinesa}
  \end{phonetics}
\end{entry}

\begin{entry}{人生}{2,5}{⼈、⽣}
  \begin{phonetics}{人生}{ren2sheng1}[][HSK 3]
    \definition{s.}{vida (tempo de alguém na Terra)}
  \end{phonetics}
\end{entry}

\begin{entry}{人权}{2,6}{⼈、⽊}
  \begin{phonetics}{人权}{ren2quan2}
    \definition*{s.}{Direitos Humanos}
  \seealsoref{人权法}{ren2quan2fa3}
  \end{phonetics}
\end{entry}

\begin{entry}{人权法}{2,6,8}{⼈、⽊、⽔}
  \begin{phonetics}{人权法}{ren2quan2fa3}
    \definition*{s.}{Direitos Humanos}
  \seealsoref{人权}{ren2quan2}
  \end{phonetics}
\end{entry}

\begin{entry}{人行道}{2,6,12}{⼈、⾏、⾡}
  \begin{phonetics}{人行道}{ren2xing2dao4}
    \definition{s.}{calçada}
  \end{phonetics}
\end{entry}

\begin{entry}{人员}{2,7}{⼈、⼝}
  \begin{phonetics}{人员}{ren2yuan2}[][HSK 3]
    \definition[个,位,名]{s.}{funcionários | pessoal}
  \end{phonetics}
\end{entry}

\begin{entry}{人材}{2,7}{⼈、⽊}
  \begin{phonetics}{人材}{ren2cai2}
    \variantof{人才}
  \end{phonetics}
\end{entry}

\begin{entry}{人间}{2,7}{⼈、⾨}
  \begin{phonetics}{人间}{ren2jian1}
    \definition{s.}{o mundo humano | a Terra}
  \end{phonetics}
\end{entry}

\begin{entry}{人鱼}{2,8}{⼈、⿂}
  \begin{phonetics}{人鱼}{ren2yu2}
    \definition{s.}{sereia | peixe-boi | salamandra gigante}
  \end{phonetics}
\end{entry}

\begin{entry}{人类}{2,9}{⼈、⽶}
  \begin{phonetics}{人类}{ren2lei4}[][HSK 3]
    \definition[种]{s.}{humano; humanidade; raça humana}
  \end{phonetics}
\end{entry}

\begin{entry}{人家}{2,10}{⼈、⼧}
  \begin{phonetics}{人家}{ren2jia1}[][HSK 4]
    \definition[对]{s.}{lar; família; família do noivo; casa do futuro marido}
  \end{phonetics}
  \begin{phonetics}{人家}{ren2jia5}
    \definition{pron.}{outros; uma pessoa ou pessoas diferentes do falante ou ouvinte; refere-se a alguém diferente de si mesmo ou de outra pessoa | certa pessoa ou pessoas (a pessoa ou pessoas mencionadas em um contexto próximo, aproximadamente equivalente ao pronome de terceira pessoa);  refere-se a uma pessoa ou algumas pessoas, com significado semelhante a ``他'' | eu; mim (usado retoricamente no lugar do primeiro pronome pessoal, muitas vezes expressando descontentamento de forma jocosa; geralmente usado quando se fala com pessoas próximas, para significar ``自己'', usado principamente por meninas)}
  \seealsoref{他}{ta1}
  \seealsoref{自己}{zi4ji3}
  \end{phonetics}
\end{entry}

\begin{entry}{人海}{2,10}{⼈、⽔}
  \begin{phonetics}{人海}{ren2hai3}
    \definition{s.}{uma multidão | um mar de pessoas}
  \end{phonetics}
\end{entry}

\begin{entry}{人道}{2,12}{⼈、⾡}
  \begin{phonetics}{人道}{ren2dao4}
    \definition{s.}{solidariedade humana | humanitarismo | humano | a ``maneira humana'', um dos estágios do ciclo de reencarnação (budismo) | relação sexual}
  \end{phonetics}
\end{entry}

\begin{entry}{人像}{2,13}{⼈、⼈}
  \begin{phonetics}{人像}{ren2xiang4}
    \definition{s.}{``retrato'' de uma pessoa (esboço, foto, escultura, etc.)}
  \end{phonetics}
\end{entry}

\begin{entry}{人数}{2,13}{⼈、⽁}
  \begin{phonetics}{人数}{ren2 shu4}[][HSK 2]
    \definition{s.}{número de pessoas}
  \end{phonetics}
\end{entry}

\begin{entry}{人群}{2,13}{⼈、⽺}
  \begin{phonetics}{人群}{ren2 qun2}[][HSK 3]
    \definition{s.}{multidão; ajuntamento; torpel; aglomeração; um grupo de pessoas}
  \end{phonetics}
\end{entry}

\begin{entry}{儿}{2}{⼉}
  \begin{phonetics}{儿}{er2}
    \definition{s.}{criança | filho}
  \end{phonetics}
  \begin{phonetics}{儿}{r5}
    \definition{suf.}{sufixo diminutivo não silábico | final retroflexo}
  \end{phonetics}
  \begin{phonetics}{儿}{ren2}
    \definition{s.}{pessoa, radical em caracteres chineses}
    \variantof{人}
  \end{phonetics}
\end{entry}

\begin{entry}{儿子}{2,3}{⼉、⼦}
  \begin{phonetics}{儿子}{er2zi5}
    \definition{s.}{filho}
  \seealsoref{女儿}{nv3'er2}
  \end{phonetics}
\end{entry}

\begin{entry}{儿童}{2,12}{⼉、⽴}
  \begin{phonetics}{儿童}{er2tong2}[][HSK 4]
    \definition[个,群]{s.}{criança; menor de idade (mais jovem do que ``少年'')}
  \seealsoref{少年}{shao4 nian2}
  \end{phonetics}
\end{entry}

\begin{entry}{儿媳}{2,13}{⼉、⼥}
  \begin{phonetics}{儿媳}{er2xi2}
    \definition{s.}{esposa do filho}
  \end{phonetics}
\end{entry}

\begin{entry}{入乡随俗}{2,3,11,9}{⼊、⼄、⾩、⼈}
  \begin{phonetics}{入乡随俗}{ru4xiang1-sui2su2}
    \definition{expr.}{Em roma, faça como os romanos!}
  \end{phonetics}
\end{entry}

\begin{entry}{入口}{2,3}{⼊、⼝}
  \begin{phonetics}{入口}{ru4kou3}[][HSK 2]
    \definition[个]{s.}{entrada | enseada}
    \definition{v.}{entra na boca | importar}
  \end{phonetics}
\end{entry}

\begin{entry}{入门}{2,3}{⼊、⾨}
  \begin{phonetics}{入门}{ru4men2}
    \definition{s.}{curso elementar | ABC | guia}
    \definition{v.+compl.}{atravessar o limiar | aprender o ABC de | introduzir um assunto | aprender os rudimentos de um assunto}
  \end{phonetics}
\end{entry}

\begin{entry}{入党}{2,10}{⼊、⼉}
  \begin{phonetics}{入党}{ru4dang3}
    \definition{v.}{ingressar em um partido político (especialmente o partido comunista)}
  \end{phonetics}
\end{entry}

\begin{entry}{入境}{2,14}{⼊、⼟}
  \begin{phonetics}{入境}{ru4jing4}
    \definition{s.}{imigração}
    \definition{v.+compl.}{entrar em um país | imigrar}
  \end{phonetics}
\end{entry}

\begin{entry}{八}{2}{⼋}[Kangxi 12]
  \begin{phonetics}{八}{ba1}[][HSK 1]
    \definition{num.}{oito; 8}
  \end{phonetics}
\end{entry}

\begin{entry}{八八六}{2,2,4}{⼋、⼋、⼋}
  \begin{phonetics}{八八六}{ba1 ba1 liu4}
    \definition{expr.}{\emph{Bye bye!} (em salas de bate-papo e mensagens de texto)}
  \end{phonetics}
\end{entry}

\begin{entry}{几}{2}{⼏}
  \begin{phonetics}{几}{ji1}
    \definition{adv.}{quase}
    \definition{s.}{mesa pequena}
  \end{phonetics}
  \begin{phonetics}{几}{ji3}[][HSK 1]
    \definition{adv.}{quantos?, (até 10 itens) | vários | alguns}
  \end{phonetics}
\end{entry}

\begin{entry}{几乎}{2,5}{⼏、⼃}
  \begin{phonetics}{几乎}{ji1hu1}[][HSK 4]
    \definition{adv.}{quase; praticamente; próximo a | perto de; quase; à beira de}
  \end{phonetics}
\end{entry}

\begin{entry}{几何}{2,7}{⼏、⼈}
  \begin{phonetics}{几何}{ji3he2}
    \definition{s.}{geometria}
  \end{phonetics}
\end{entry}

\begin{entry}{刀}{2}{⼑}[Kangxi 18]
  \begin{phonetics}{刀}{dao1}[][HSK 3]
    \definition*{s.}{sobrenome Dao}
    \definition{clas.}{para cortes de faca ou facadas | para cem folhas (de papel)}
    \definition[把]{s.}{faca; espada | algo em forma de faca | moeda antiga em forma de faca}
  \end{phonetics}
\end{entry}

\begin{entry}{力}{2}{⼒}[Kangxi 19]
  \begin{phonetics}{力}{li4}[][HSK 3]
    \definition*{s.}{sobrenome Li}
    \definition{adv.}{energicamente; arduamente; vigorosamente}
    \definition{s.}{poder; força; habilidade; capacidade | força; energia; poder | força física}
    \definition{v.}{fazer tudo o que puder; fazer todo o esforço}
  \end{phonetics}
\end{entry}

\begin{entry}{力气}{2,4}{⼒、⽓}
  \begin{phonetics}{力气}{li4qi5}[][HSK 4]
    \definition[点,把]{s.}{força física | esforço}
  \end{phonetics}
\end{entry}

\begin{entry}{力量}{2,12}{⼒、⾥}
  \begin{phonetics}{力量}{li4liang5}[][HSK 3]
    \definition[出]{s.}{força física; força espiritual | habilidade; capacidade | eficácia; efeito | força (pessoa ou grupo que tem muito poder ou influência)}
  \end{phonetics}
\end{entry}

\begin{entry}{十}{2}{⼗}[Kangxi 24]
  \begin{phonetics}{十}{shi2}[][HSK 1]
    \definition{num.}{dez; 10 | dezena}
  \end{phonetics}
\end{entry}

\begin{entry}{十分}{2,4}{⼗、⼑}
  \begin{phonetics}{十分}{shi2fen1}[][HSK 2]
    \definition{adv.}{muito | extremamente | totalmente | absolutamente}
  \end{phonetics}
\end{entry}

\begin{entry}{十足}{2,7}{⼗、⾜}
  \begin{phonetics}{十足}{shi2zu2}
    \definition{adj.}{amplo | completo | cento por cento | tom puro (de alguma cor)}
  \end{phonetics}
\end{entry}

\begin{entry}{厂}{2}{⼚}[Kangxi 27]
  \begin{phonetics}{厂}{chang3}[][HSK 3]
    \definition[家]{s.}{fábrica; moinho; planta; obra | pátio; depósito}
  \end{phonetics}
  \begin{phonetics}{厂}{han3}
    \definition{s.}{radical ``penhasco'' em caracteres chineses (radical Kangxi 27)}
  \end{phonetics}
\end{entry}

\begin{entry}{又}{2}{⼜}[Kangxi 29]
  \begin{phonetics}{又}{you4}[][HSK 2]
    \definition{adv.}{mais uma vez | (usado para dar ênfase) de qualquer maneira | e ainda | e também}
  \end{phonetics}
\end{entry}

\begin{entry}{又一次}{2,1,6}{⼜、⼀、⽋}
  \begin{phonetics}{又一次}{you4yi2ci4}
    \definition{adv.}{outra vez | mais uma vez | de novo}
  \end{phonetics}
\end{entry}

\begin{entry}{又及}{2,3}{⼜、⼃}
  \begin{phonetics}{又及}{you4ji2}
    \definition{s.}{P.S., \emph{postscript}}
  \end{phonetics}
\end{entry}

\begin{entry}{又名}{2,6}{⼜、⼝}
  \begin{phonetics}{又名}{you4ming2}
    \definition{s.}{também conhecido como | nome alternativo}
  \end{phonetics}
\end{entry}

\begin{entry}{又称}{2,10}{⼜、⽲}
  \begin{phonetics}{又称}{you4cheng1}
    \definition{s.}{também conhecido como}
  \end{phonetics}
\end{entry}

%%%%% EOF %%%%%


%%%
%%% 3画
%%%

\section*{3画}\addcontentsline{toc}{section}{3画}

\begin{entry}{万}{3}{⼀}
  \begin{phonetics}{万}{wan4}[][HSK 2]
    \definition*{s.}{sobrenome Wan}
    \definition{adj.}{um grande número}
    \definition{num.}{dez mil; 10.000; 1.0000}
  \end{phonetics}
\end{entry}

\begin{entry}{万一}{3,1}{⼀、⼀}
  \begin{phonetics}{万一}{wan4yi1}[][HSK 4]
    \definition{conj.}{por via das dúvidas; se por acaso; só por precaução; expressa uma suposição muito improvável (usado para coisas desagradáveis)}
    \definition{num.}{um décimo milionésimo; uma porcentagem muito pequena}
    \definition{s.}{contingência; eventualidade; contingências muito improváveis}
  \end{phonetics}
\end{entry}

\begin{entry}{万万}{3,3}{⼀、⼀}
  \begin{phonetics}{万万}{wan4wan4}
    \definition{adv.}{absolutamente | totalmente}
  \end{phonetics}
\end{entry}

\begin{entry}{万圣节}{3,5,5}{⼀、⼟、⾋}
  \begin{phonetics}{万圣节}{wan4sheng4jie2}
    \definition*{s.}{Dia de Todos os Santos}
  \seealsoref{万圣节前夕}{wan4sheng4jie2qian2xi1}
  \end{phonetics}
\end{entry}

\begin{entry}{万圣节前夕}{3,5,5,9,3}{⼀、⼟、⾋、⼑、⼣}
  \begin{phonetics}{万圣节前夕}{wan4sheng4jie2qian2xi1}
    \definition*{s.}{Véspera do Dia de Todos os Santos | \emph{Halloween}}
  \seealsoref{万圣节}{wan4sheng4jie2}
  \end{phonetics}
\end{entry}

\begin{entry}{丈夫}{3,4}{⼀、⼤}
  \begin{phonetics}{丈夫}{zhang4fu5}[][HSK 4]
    \definition[个,位,名]{s.}{marido; esposo}
  \end{phonetics}
\end{entry}

\begin{entry}{三}{3}{⼀}
  \begin{phonetics}{三}{san1}[][HSK 1]
    \definition*{s.}{sobrenome San}
    \definition{num.}{três; 3}
  \end{phonetics}
\end{entry}

\begin{entry}{三角}{3,7}{⼀、⾓}
  \begin{phonetics}{三角}{san1jiao3}
    \definition{s.}{triângulo}
  \end{phonetics}
\end{entry}

\begin{entry}{三角恋爱}{3,7,10,10}{⼀、⾓、⼼、⽖}
  \begin{phonetics}{三角恋爱}{san1jiao3lian4'ai4}
    \definition{s.}{triângulo amoroso}
  \end{phonetics}
\end{entry}

\begin{entry}{三明治}{3,8,8}{⼀、⽇、⽔}
  \begin{phonetics}{三明治}{san1ming2zhi4}
    \definition{s.}{(empréstimo linguístico) sanduíche}
  \end{phonetics}
\end{entry}

\begin{entry}{三轮车}{3,8,4}{⼀、⾞、⾞}
  \begin{phonetics}{三轮车}{san1lun2che1}
    \definition{s.}{triciclo}
  \end{phonetics}
\end{entry}

\begin{entry}{上}{3}{⼀}
  \begin{phonetics}{上}{shang4}[][HSK 1]
    \definition{adv.}{acima | em cima | sobre}
    \definition{v.}{subir | entrar em | frequentar (aula ou universidade)}
  \end{phonetics}
\end{entry}

\begin{entry}{上下}{3,3}{⼀、⼀}
  \begin{phonetics}{上下}{shang4 xia4}[][HSK 5]
    \definition{adv.}{para cima e para baixo}
    \definition[顶]{s.}{alto e baixo | de cima para baixo; para cima e para baixo | superioridade ou inferioridade relativa | (após números redondos) aproximadamente; mais ou menos; por aí | hierarquia em termos de cargo e posição social}
    \definition{v.}{subir ou descer | subir e descer}
  \end{phonetics}
\end{entry}

\begin{entry}{上个月}{3,3,4}{⼀、⼈、⽉}
  \begin{phonetics}{上个月}{shang4 ge4 yue4}[][HSK 4]
    \definition{s.}{mês passado; refere-se à hora de um mês atrás, ou seja, o mês passado mais próximo da hora atual}
  \end{phonetics}
\end{entry}

\begin{entry}{上门}{3,3}{⼀、⾨}
  \begin{phonetics}{上门}{shang4 men2}[][HSK 4]
    \definition{v.}{chamar; visitar; aparecer; ir ou vir para ver alguém; ir até a porta; ir até a casa de alguém | trancar a porta; fechar a porta durante a noite | casar-se e morar com a família da noiva}
  \end{phonetics}
\end{entry}

\begin{entry}{上升}{3,4}{⼀、⼗}
  \begin{phonetics}{上升}{shang4 sheng1}[][HSK 3]
    \definition{v.}{elevar; subir; mover-se para cima}
  \end{phonetics}
\end{entry}

\begin{entry}{上午}{3,4}{⼀、⼗}
  \begin{phonetics}{上午}{shang4wu3}[][HSK 1]
    \definition{adv.}{manhã | de manhã | período antes do meio-dia}
  \end{phonetics}
\end{entry}

\begin{entry}{上车}{3,4}{⼀、⾞}
  \begin{phonetics}{上车}{shang4 che1}[][HSK 1]
    \definition{v.}{entrar (em ônibus, trem, carro, etc.)}
  \end{phonetics}
\end{entry}

\begin{entry}{上去}{3,5}{⼀、⼛}
  \begin{phonetics}{上去}{shang4 qu4}[][HSK 3]
    \definition{v.}{subir (a partir da minha localização) | ascender a um lugar (ou estado) considerado mais elevado (ou acima)}
  \end{phonetics}
\end{entry}

\begin{entry}{上古}{3,5}{⼀、⼝}
  \begin{phonetics}{上古}{shang4gu3}
    \definition{s.}{o passado distante | tempos antigos | antiguidade}
  \end{phonetics}
\end{entry}

\begin{entry}{上边}{3,5}{⼀、⾡}
  \begin{phonetics}{上边}{shang4bian5}[][HSK 1]
    \definition{adv.}{acima de | parte de cima | por cima}
  \end{phonetics}
\end{entry}

\begin{entry}{上当}{3,6}{⼀、⼹}
  \begin{phonetics}{上当}{shang4dang4}
    \definition{v.+compl.}{ser enganado | morder uma isca | ser manipulado | ser joguete nas mãos de alguém}
  \end{phonetics}
\end{entry}

\begin{entry}{上次}{3,6}{⼀、⽋}
  \begin{phonetics}{上次}{shang4 ci4}[][HSK 1]
    \definition{adv.}{última vez}
  \end{phonetics}
\end{entry}

\begin{entry}{上级}{3,6}{⼀、⽷}
  \begin{phonetics}{上级}{shang4ji2}[][HSK 5]
    \definition[个]{s.}{nível superior; organização ou pessoa em nível superior; organizações ou pessoas de nível superior dentro do mesmo sistema organizacional}
  \end{phonetics}
\end{entry}

\begin{entry}{上网}{3,6}{⼀、⽹}
  \begin{phonetics}{上网}{shang4 wang3}[][HSK 1]
    \definition{v.}{conectar à \emph{Internet} | fazer \emph{upload} | ficar \emph{online}}
  \end{phonetics}
\end{entry}

\begin{entry}{上衣}{3,6}{⼀、⾐}
  \begin{phonetics}{上衣}{shang4 yi1}[][HSK 3]
    \definition{s.}{jaqueta; vestimenta externa superior}
  \end{phonetics}
\end{entry}

\begin{entry}{上访}{3,6}{⼀、⾔}
  \begin{phonetics}{上访}{shang4fang3}
    \definition{v.}{buscar uma audiência com superiores (especialmente funcionários do governo) para fazer uma petição por algo}
  \end{phonetics}
\end{entry}

\begin{entry}{上声}{3,7}{⼀、⼠}
  \begin{phonetics}{上声}{shang3sheng1}
    \definition{s.}{tom descendente e ascendente | terceiro tom no mandarim moderno}
  \end{phonetics}
\end{entry}

\begin{entry}{上来}{3,7}{⼀、⽊}
  \begin{phonetics}{上来}{shang4 lai2}[][HSK 3]
    \definition{v.}{subir (para a minha localização) | estar no começo | vir à tona | usado depois de um verbo para indicar sucesso em fazer algo}
  \end{phonetics}
\end{entry}

\begin{entry}{上周}{3,8}{⼀、⼝}
  \begin{phonetics}{上周}{shang4 zhou1}[][HSK 2]
    \definition{s.}{semana passada}
  \end{phonetics}
\end{entry}

\begin{entry}{上坡路}{3,8,13}{⼀、⼟、⾜}
  \begin{phonetics}{上坡路}{shang4po1lu4}
    \definition{s.}{aclive | progresso | (fig.) tendência ascendente}
  \end{phonetics}
\end{entry}

\begin{entry}{上学}{3,8}{⼀、⼦}
  \begin{phonetics}{上学}{shang4 xue2}[][HSK 1]
    \definition{v.}{ir à escola | frequentar a escola | estar na escola | iniciar as aulas}
  \end{phonetics}
\end{entry}

\begin{entry}{上询}{3,8}{⼀、⾔}
  \begin{phonetics}{上询}{shang4 xun2}
    \definition{adv.}{primeira dezena do mês}
  \end{phonetics}
\end{entry}

\begin{entry}{上面}{3,9}{⼀、⾯}
  \begin{phonetics}{上面}{shang4 mian4}[][HSK 3]
    \definition{s.}{uma posição mais alta que algo; uma posição acima/acima de algo | superfície do objeto | aspecto | a parte acima mencionada | autoridades superiores | os mais velhos; a geração mais velha da família}
  \end{phonetics}
\end{entry}

\begin{entry}{上海}{3,10}{⼀、⽔}
  \begin{phonetics}{上海}{shang4hai3}
    \definition*{s.}{Shangai (Xangai)}
  \end{phonetics}
\end{entry}

\begin{entry}{上涨}{3,10}{⼀、⽔}
  \begin{phonetics}{上涨}{shang4 zhang3}[][HSK 5]
    \definition{v.}{subir; ir para cima; ascender}
  \end{phonetics}
\end{entry}

\begin{entry}{上班}{3,10}{⼀、⽟}
  \begin{phonetics}{上班}{shang4 ban1}[][HSK 1]
    \definition{v.+compl.}{ir para o trabalho | ir para o emprego | estar de plantão}
  \end{phonetics}
\end{entry}

\begin{entry}{上课}{3,10}{⼀、⾔}
  \begin{phonetics}{上课}{shang4 ke4}[][HSK 1]
    \definition{v.}{assistir à aula | ir para a aula | ir dar uma aula}
  \end{phonetics}
\end{entry}

\begin{entry}{上楼}{3,13}{⼀、⽊}
  \begin{phonetics}{上楼}{shang4 lou2}[][HSK 4]
    \definition{v.}{subir as escadas; ir para o andar de cima}
  \end{phonetics}
\end{entry}

\begin{entry}{上演}{3,14}{⼀、⽔}
  \begin{phonetics}{上演}{shang4yan3}
    \definition{s.}{exibição | encenação}
    \definition{v.}{exibir (um filme) | encenar (uma peça)}
  \end{phonetics}
\end{entry}

\begin{entry}{下}{3}{⼀}
  \begin{phonetics}{下}{xia4}[][HSK 1,2]
    \definition{adv.}{abaixo | em baixo de}
    \definition{clas.}{para número de vezes para ações}
    \definition{v.}{descer | chegar a (uma decisão, conclusão, etc.) | recusar}
  \end{phonetics}
\end{entry}

\begin{entry}{下个月}{3,3,4}{⼀、⼈、⽉}
  \begin{phonetics}{下个月}{xia4 ge4 yue4}[][HSK 4]
    \definition{s.}{próximo mês; mês que vem; refere-se ao próximo mês do mês atual}
  \end{phonetics}
\end{entry}

\begin{entry}{下午}{3,4}{⼀、⼗}
  \begin{phonetics}{下午}{xia4wu3}[][HSK 1]
    \definition{adv.}{tarde | à tarde | período logo após o meio-dia}
  \end{phonetics}
\end{entry}

\begin{entry}{下午茶}{3,4,9}{⼀、⼗、⾋}
  \begin{phonetics}{下午茶}{xia4wu3cha2}
    \definition{s.}{chá da tarde (normalmente chás com doces)}
  \end{phonetics}
\end{entry}

\begin{entry}{下巴}{3,4}{⼀、⼰}
  \begin{phonetics}{下巴}{xia4ba5}
    \definition[个]{s.}{queixo}
  \end{phonetics}
\end{entry}

\begin{entry}{下水道}{3,4,12}{⼀、⽔、⾡}
  \begin{phonetics}{下水道}{xia4shui3dao4}
    \definition{s.}{esgoto}
  \end{phonetics}
\end{entry}

\begin{entry}{下车}{3,4}{⼀、⾞}
  \begin{phonetics}{下车}{xia4 che1}[][HSK 1]
    \definition{v.}{descer ou sair (de ônibus, carro, etc.)}
  \end{phonetics}
\end{entry}

\begin{entry}{下去}{3,5}{⼀、⼛}
  \begin{phonetics}{下去}{xia4 qu4}[][HSK 3]
    \definition{part.}{usado depois de verbos para indicar de alto a baixo | usado depois de um verbo para indicar continuação}
    \definition{v.}{descer (a partir da minha localização)
continuar
obter; crescer; tornar-se}
  \end{phonetics}
\end{entry}

\begin{entry}{下边}{3,5}{⼀、⾡}
  \begin{phonetics}{下边}{xia4bian5}[][HSK 1]
    \definition{adv.}{em baixo | abaixo | parte de baixo}
  \end{phonetics}
\end{entry}

\begin{entry}{下旬}{3,6}{⼀、⽇}
  \begin{phonetics}{下旬}{xia4xun2}
    \definition{adv.}{última dezena do mês}
  \end{phonetics}
\end{entry}

\begin{entry}{下次}{3,6}{⼀、⽋}
  \begin{phonetics}{下次}{xia4 ci4}[][HSK 1]
    \definition{s.}{próxima vez}
  \end{phonetics}
\end{entry}

\begin{entry}{下来}{3,7}{⼀、⽊}
  \begin{phonetics}{下来}{xia4 lai5}[][HSK 3]
    \definition{part.}{usado depois de um verbo para indicar que uma ação ou comportamento está se movendo em direção ao falante ou que a ação está continuando ou sendo concluída | usado depois de um adjetivo para indicar que um certo estado começou a aparecer e continuará a se desenvolver.}
    \definition{v.}{descer (para a minha localização) | (colheitas/frutas/vegetais, etc.) ser colhido; estar maduro o suficiente para ser colhido | (período de tempo) acabar; passar; chegar ao fim}
  \end{phonetics}
\end{entry}

\begin{entry}{下周}{3,8}{⼀、⼝}
  \begin{phonetics}{下周}{xia4 zhou1}[][HSK 2]
    \definition{s.}{próxima semana}
  \end{phonetics}
\end{entry}

\begin{entry}{下线}{3,8}{⼀、⽷}
  \begin{phonetics}{下线}{xia4xian4}
    \definition{v.}{ficar \emph{offline} | (um produto) sair da linha de montagem | pessoa abaixo de si em um esquema de pirâmide}
  \end{phonetics}
\end{entry}

\begin{entry}{下降}{3,8}{⼀、⾩}
  \begin{phonetics}{下降}{xia4 jiang4}[][HSK 4]
    \definition{v.}{cair; despencar; declinar; descer; diminuir; ir para baixo}
  \end{phonetics}
\end{entry}

\begin{entry}{下雨}{3,8}{⼀、⾬}
  \begin{phonetics}{下雨}{xia4 yu3}[][HSK 1]
    \definition{v.+compl.}{chover}
  \end{phonetics}
\end{entry}

\begin{entry}{下面}{3,9}{⼀、⾯}
  \begin{phonetics}{下面}{xia4 mian4}[][HSK 3]
    \definition{s.}{em baixo; abaixo; parte de baixo | próximo; seguinte | subordinado; o nível inferior; homens nos níveis inferiores}
    \definition{v.}{cozinhar macarrão}
  \end{phonetics}
\end{entry}

\begin{entry}{下海}{3,10}{⼀、⽔}
  \begin{phonetics}{下海}{xia4hai3}
    \definition{v.+compl.}{ir para o mar; (barco) deixar o porto e iniciar uma jornada | ir pescar no mar | tornar-se ator profissional}
  \end{phonetics}
\end{entry}

\begin{entry}{下班}{3,10}{⼀、⽟}
  \begin{phonetics}{下班}{xia4 ban1}[][HSK 1]
    \definition{v.+compl.}{sair do trabalho}
  \end{phonetics}
\end{entry}

\begin{entry}{下课}{3,10}{⼀、⾔}
  \begin{phonetics}{下课}{xia4 ke4}[][HSK 1]
    \definition{v.+compl.}{acabar a aula | terminar a aula}
  \end{phonetics}
\end{entry}

\begin{entry}{下载}{3,10}{⼀、⾞}
  \begin{phonetics}{下载}{xia4zai3}[][HSK 4]
    \definition{v.}{\emph{download}; baixar; salvar informações da \emph{Web} em um dispositivo, como um computador}
  \end{phonetics}
\end{entry}

\begin{entry}{下蛋}{3,11}{⼀、⾍}
  \begin{phonetics}{下蛋}{xia4dan4}
    \definition{v.}{botar ovos}
  \end{phonetics}
\end{entry}

\begin{entry}{下雪}{3,11}{⼀、⾬}
  \begin{phonetics}{下雪}{xia4 xue3}[][HSK 2]
    \definition[场,次]{s.}{neve}
    \definition{v.+compl.}{nevar}
  \end{phonetics}
\end{entry}

\begin{entry}{下崽}{3,12}{⼀、⼭}
  \begin{phonetics}{下崽}{xia4zai3}
    \definition{v.}{dar à luz (animais) | parir}
  \end{phonetics}
\end{entry}

\begin{entry}{下楼}{3,13}{⼀、⽊}
  \begin{phonetics}{下楼}{xia4 lou2}[][HSK 4]
    \definition{v.}{descer as escadas}
  \end{phonetics}
\end{entry}

\begin{entry}{与}{3}{⼀}
  \begin{phonetics}{与}{yu3}
    \definition{conj.}{e, com}
  \end{phonetics}
  \begin{phonetics}{与}{yu4}
    \definition{v.}{fazer parte de}
  \end{phonetics}
\end{entry}

\begin{entry}{与其}{3,8}{⼀、⼋}
  \begin{phonetics}{与其}{yu3qi2}
    \definition{conj.}{mais do que}
  \end{phonetics}
\end{entry}

\begin{entry}{与其……不如……}{3,8,4,6}{⼀、⼋、⼀、⼥}
  \begin{phonetics}{与其……不如……}{yu3qi2 bu4ru2}
    \definition{conj.}{ao invés de\dots melhor que\dots}
  \end{phonetics}
\end{entry}

\begin{entry}{与其……宁可……}{3,8,5,5}{⼀、⼋、⼧、⼝}
  \begin{phonetics}{与其……宁可……}{yu3qi2 ning4ke3}
    \definition{conj.}{ao invés de\dots melhor que\dots}
  \end{phonetics}
\end{entry}

\begin{entry}{个}{3}{⼈}
  \begin{phonetics}{个}{ge3}
    \definition{pron.}{usado em 自个儿}
    \seeref{自个儿}{zi4ge3r5}
  \end{phonetics}
  \begin{phonetics}{个}{ge4}[][HSK 1]
    \definition{clas.}{para objetos e pessoas em geral}
    \definition{pron.}{isto | aquilo}
    \definition{s.}{indivíduo | tamanho}
  \end{phonetics}
\end{entry}

\begin{entry}{个人}{3,2}{⼈、⼈}
  \begin{phonetics}{个人}{ge4ren2}[][HSK 3]
    \definition{pron.}{pessoal; si mesmo}
    \definition[个]{s.}{indivíduo}
  \end{phonetics}
\end{entry}

\begin{entry}{个儿}{3,2}{⼈、⼉}
  \begin{phonetics}{个儿}{ge4r5}[][HSK 5]
    \definition{s.}{tamanho; altura; estatura; tamanho do corpo ou do objeto |
pessoas ou coisas consideradas isoladamente; referir-se a uma pessoa ou coisa individualmente}
  \end{phonetics}
\end{entry}

\begin{entry}{个子}{3,3}{⼈、⼦}
  \begin{phonetics}{个子}{ge4zi5}[][HSK 2]
    \definition{s.}{altura | estatura}
  \end{phonetics}
\end{entry}

\begin{entry}{个体}{3,7}{⼈、⼈}
  \begin{phonetics}{个体}{ge4ti3}[][HSK 4]
    \definition{s.}{pessoa ou organismo individual}
  \end{phonetics}
\end{entry}

\begin{entry}{个别}{3,7}{⼈、⼑}
  \begin{phonetics}{个别}{ge4bie2}[][HSK 4]
    \definition{adj.}{muito poucos; excepcionais}
    \definition{adv.}{separadamente; individualmente; isoladamente}
  \end{phonetics}
\end{entry}

\begin{entry}{个性}{3,8}{⼈、⼼}
  \begin{phonetics}{个性}{ge4xing4}[][HSK 3]
    \definition{s.}{caráter individual; individualidade; personalidade}
  \end{phonetics}
\end{entry}

\begin{entry}{久}{3}{⼃}
  \begin{phonetics}{久}{jiu3}[][HSK 3]
    \definition{adj.}{por muito tempo | duração de tempo especificada}
  \end{phonetics}
\end{entry}

\begin{entry}{义务}{3,5}{⼂、⼒}
  \begin{phonetics}{义务}{yi4wu4}[][HSK 4]
    \definition{s.}{dever; obrigação; responsabilidades perante a lei, em oposição a ``权利''.}
  \seealsoref{权利}{quan2li4}
  \end{phonetics}
\end{entry}

\begin{entry}{之一}{3,1}{⼂、⼀}
  \begin{phonetics}{之一}{zhi1 yi1}[][HSK 4]
    \definition{s.}{um de (algo); pertence a um ou a todo um grupo de coisas com as mesmas características}
  \end{phonetics}
\end{entry}

\begin{entry}{之下}{3,3}{⼂、⼀}
  \begin{phonetics}{之下}{zhi1 xia4}[][HSK 5]
    \definition{s.}{usado para indicar algo abaixo de um determinado intervalo, posição, grau, etc.; indica um aspecto inferior em termos de alcance, posição, status, nível, Chengdu, etc. | usado para indicar as condições sob as quais algo acontece | usado para indicar o humor, estado em que alguém faz algo; expressa um determinado comportamento em um determinado estado de espírito ou situação}
  \end{phonetics}
\end{entry}

\begin{entry}{之中}{3,4}{⼂、⼁}
  \begin{phonetics}{之中}{zhi1 zhong1}[][HSK 5]
    \definition{prep.}{em; no meio de; entre}
  \end{phonetics}
\end{entry}

\begin{entry}{之内}{3,4}{⼂、⼌}
  \begin{phonetics}{之内}{zhi1 nei4}[][HSK 5]
    \definition{adv.}{em; dentro de; indica dentro de um determinado intervalo, limite ou período de tempo, etc.}
  \end{phonetics}
\end{entry}

\begin{entry}{之外}{3,5}{⼂、⼣}
  \begin{phonetics}{之外}{zhi1 wai4}[][HSK 5]
    \definition{adv.}{lado de fora; exceto; além de; além disso; refere-se a algo que excede um determinado limite}
  \end{phonetics}
\end{entry}

\begin{entry}{之后}{3,6}{⼂、⼝}
  \begin{phonetics}{之后}{zhi1 hou4}[][HSK 4]
    \definition{adv.}{mais tarde; posteriormente; depois; desde então; para indicar que é depois de um determinado tempo ou de uma determinada coisa, ``以后'' é usado com frequência na linguagem falada; às vezes, também pode indicar que é depois de um determinado lugar ou local,  ``后面'' é usado com frequência na linguagem falada}
  \seealsoref{后面}{hou4mian4}
  \seealsoref{以后}{yi3 hou4}
  \end{phonetics}
\end{entry}

\begin{entry}{之间}{3,7}{⼂、⾨}
  \begin{phonetics}{之间}{zhi1 jian1}[][HSK 4]
    \definition{prep.}{(depois de um substantivo) entre; dentro de duas delimitações de tempo, local ou quantitativas | colocado após certos verbos ou advérbios de duas sílabas para indicar um curto período de tempo}
  \end{phonetics}
\end{entry}

\begin{entry}{之前}{3,9}{⼂、⼑}
  \begin{phonetics}{之前}{zhi1 qian2}[][HSK 4]
    \definition{adv.}{(referindo-se ao tempo) antes, antes de, atrás | (referindo-se ao local físico) na frente de | (usado independentemente) no passado, antes disso}
  \end{phonetics}
\end{entry}

\begin{entry}{也}{3}{⼄}
  \begin{phonetics}{也}{ye3}[][HSK 1]
    \definition*{s.}{sobrenome Ye}
    \definition{adv.}{também | (em frases negativas) nem, tampouco}
  \end{phonetics}
\end{entry}

\begin{entry}{也好}{3,6}{⼄、⼥}
  \begin{phonetics}{也好}{ye3 hao3}[][HSK 5]
    \definition{part.}{pode não ser uma má ideia; também pode ser | [reduplicado] se\dots ou\dots; não importa se | pode não ser uma má ideia | se\dots ou\dots; usado em conjunto, significa que não está condicionado a uma determinada situação}
  \end{phonetics}
\end{entry}

\begin{entry}{也有今天}{3,6,4,4}{⼄、⽉、⼈、⼤}
  \begin{phonetics}{也有今天}{ye3you3jin1tian1}
    \definition{expr.}{obter apenas o que merece | todo cachorro tem seu dia | obter a sua parte (coisas boas ou ruins) | servir alguém bem}
  \end{phonetics}
\end{entry}

\begin{entry}{也许}{3,6}{⼄、⾔}
  \begin{phonetics}{也许}{ye3xu3}[][HSK 2]
    \definition{adv.}{possivelmente | talvez}
  \end{phonetics}
\end{entry}

\begin{entry}{也就是}{3,12,9}{⼄、⼪、⽇}
  \begin{phonetics}{也就是}{ye3jiu4shi4}
    \definition{adv.}{i.e., isso é | ou seja}
  \end{phonetics}
\end{entry}

\begin{entry}{也就是说}{3,12,9,9}{⼄、⼪、⽇、⾔}
  \begin{phonetics}{也就是说}{ye3jiu4shi4shuo1}
    \definition{adv.}{em outras palavras | então | isto é | por isso}
  \end{phonetics}
\end{entry}

\begin{entry}{习惯}{3,11}{⼄、⼼}
  \begin{phonetics}{习惯}{xi2guan4}[][HSK 2]
    \definition[个]{s.}{hábito | costume | prática usual}
    \definition{v.}{ser acostumado a | ter o hábito de}
  \end{phonetics}
\end{entry}

\begin{entry}{乡}{3}{⼄}
  \begin{phonetics}{乡}{xiang1}[][HSK 5]
    \definition[个]{s.}{país; campo; vilarejo; área rural | local de origem; vila ou cidade natal | município (uma unidade administrativa rural subordinada ao condado) | vila natal; cidade natal | terra ou local famoso por produzir algo}
  \end{phonetics}
\end{entry}

\begin{entry}{乡巴佬}{3,4,8}{⼄、⼰、⼈}
  \begin{phonetics}{乡巴佬}{xiang1ba1lao3}
    \definition{s.}{aldeão | caipira}
  \end{phonetics}
\end{entry}

\begin{entry}{乡村}{3,7}{⼄、⽊}
  \begin{phonetics}{乡村}{xiang1 cun1}[][HSK 5]
    \definition{adj.}{rural | rústico}
    \definition{s.}{vila; campo; área rural; principalmente envolvido na agricultura; áreas com distribuição populacional mais dispersa em relação às cidades}
  \end{phonetics}
\end{entry}

\begin{entry}{于}{3}{⼆}
  \begin{phonetics}{于}{yu2}
    \definition*{s.}{sobrenome Yu}
    \definition{prep.}{indica tempo, local, extensão, etc. | indica a direção da ação | usada após um verbo para indicar doação, entrega, etc. | relacionamento do objeto ou da entidade introduzida | indica o ponto inicial ou o ponto de partida | indica comparação}
  \end{phonetics}
\end{entry}

\begin{entry}{于是}{3,9}{⼆、⽇}
  \begin{phonetics}{于是}{yu2shi4}[][HSK 4]
    \definition{conj.}{então; portanto; consequentemente; como resultado; indica que o último segue o primeiro e que o último é frequentemente causado pelo primeiro}
  \end{phonetics}
\end{entry}

\begin{entry}{亏}{3}{⼆}
  \begin{phonetics}{亏}{kui1}[][HSK 5]
    \definition{adv.}{felizmente; por sorte; graças a | contrariamente, expressando sarcasmo}
    \definition{s.}{prejuízo}
    \definition{v.}{perder dinheiro, etc.; ter um déficit; ter prejuízo | ter falta de; ser deficiente; carecer de | tratar injustamente; causar prejuízo; trair a confiança}
  \end{phonetics}
\end{entry}

\begin{entry}{亿}{3}{⼈}
  \begin{phonetics}{亿}{yi4}[][HSK 2]
    \definition{num.}{cem milhões; 100.000.000; 1.0000.0000}
  \end{phonetics}
\end{entry}

\begin{entry}{千}{3}{⼗}
  \begin{phonetics}{千}{qian1}[][HSK 2]
    \definition{num.}{mil; 1.000; 1000}
  \end{phonetics}
\end{entry}

\begin{entry}{千万}{3,3}{⼗、⼀}
  \begin{phonetics}{千万}{qian1wan4}[][HSK 3]
    \definition{adv.}{(usado para indicar desejos fortes) por todos os meios; sob quaisquer circunstâncias}
    \definition{num.}{dez milhões; milhões e milhões}
  \end{phonetics}
\end{entry}

\begin{entry}{千千万万}{3,3,3,3}{⼗、⼗、⼀、⼀}
  \begin{phonetics}{千千万万}{qian1qian1wan4wan4}
    \definition{num.}{inumerável | números incontáveis | milhares e milhares}
  \end{phonetics}
\end{entry}

\begin{entry}{千古}{3,5}{⼗、⼝}
  \begin{phonetics}{千古}{qian1gu3}
    \definition{adv.}{por toda a eternidade | em todas as idades}
    \definition{s.}{eternidade (usada em um dístico elegíaco, coroa de flores, etc., dedicada aos mortos)}
  \end{phonetics}
\end{entry}

\begin{entry}{千年}{3,6}{⼗、⼲}
  \begin{phonetics}{千年}{qian1nian2}
    \definition{s.}{milênio}
  \end{phonetics}
\end{entry}

\begin{entry}{千克}{3,7}{⼗、⼗}
  \begin{phonetics}{千克}{qian1 ke4}[][HSK 2]
    \definition{clas.}{kg | quilo | quilograma}
  \end{phonetics}
\end{entry}

\begin{entry}{卫生}{3,5}{⼙、⽣}
  \begin{phonetics}{卫生}{wei4 sheng1}[][HSK 3]
    \definition{adj.}{bom para a saúde; higiênico}
    \definition{s.}{higiene; saneamento}
  \end{phonetics}
\end{entry}

\begin{entry}{卫生巾}{3,5,3}{⼙、⽣、⼱}
  \begin{phonetics}{卫生巾}{wei4sheng1jin1}
    \definition{s.}{absorvente higiênico}
  \end{phonetics}
\end{entry}

\begin{entry}{卫生厅}{3,5,4}{⼙、⽣、⼚}
  \begin{phonetics}{卫生厅}{wei4sheng1ting1}
    \definition*{s.}{Departamento de Saúde (da província)}
  \end{phonetics}
\end{entry}

\begin{entry}{卫生防疫}{3,5,6,9}{⼙、⽣、⾩、⽧}
  \begin{phonetics}{卫生防疫}{wei4sheng1 fang2yi4}
    \definition{s.}{prevenção contra a epidemia}
  \end{phonetics}
\end{entry}

\begin{entry}{卫生局}{3,5,7}{⼙、⽣、⼫}
  \begin{phonetics}{卫生局}{wei4sheng1ju2}
    \definition*{s.}{Departamento de Saúde | Escritório de Saúde}
  \end{phonetics}
\end{entry}

\begin{entry}{卫生纸}{3,5,7}{⼙、⽣、⽷}
  \begin{phonetics}{卫生纸}{wei4sheng1zhi3}
    \definition{s.}{papel higiênico}
  \end{phonetics}
\end{entry}

\begin{entry}{卫生间}{3,5,7}{⼙、⽣、⾨}
  \begin{phonetics}{卫生间}{wei4sheng1jian1}[][HSK 3]
    \definition[间,个]{s.}{banheiro; sanitário; \emph{toilette}}
  \end{phonetics}
\end{entry}

\begin{entry}{卫生套}{3,5,10}{⼙、⽣、⼤}
  \begin{phonetics}{卫生套}{wei4sheng1tao4}
    \definition[只]{s.}{preservativo | camisinha}
  \end{phonetics}
\end{entry}

\begin{entry}{卫生部}{3,5,10}{⼙、⽣、⾢}
  \begin{phonetics}{卫生部}{wei4sheng1bu4}
    \definition*{s.}{Ministério da Saúde}
  \end{phonetics}
\end{entry}

\begin{entry}{卫生球}{3,5,11}{⼙、⽣、⽟}
  \begin{phonetics}{卫生球}{wei4sheng1qiu2}
    \definition{s.}{naftalina}
  \end{phonetics}
\end{entry}

\begin{entry}{卫生棉}{3,5,12}{⼙、⽣、⽊}
  \begin{phonetics}{卫生棉}{wei4sheng1mian2}
    \definition{s.}{absorvente | algodão absorvente esterilizado (usado para curativos ou limpeza de feridas) | absorvente tampão}
  \end{phonetics}
\end{entry}

\begin{entry}{卫生署}{3,5,13}{⼙、⽣、⽹}
  \begin{phonetics}{卫生署}{wei4sheng1shu3}
    \definition*{s.}{Agência de Saúde (ou Escritório, ou Departamento)}
  \end{phonetics}
\end{entry}

\begin{entry}{卫星}{3,9}{⼙、⽇}
  \begin{phonetics}{卫星}{wei4xing1}[][HSK 5]
    \definition[个,颗]{s.}{satélite; lua; corpos celestes orbitando planetas | satélite artificial | algo que gira em torno de um centro}
  \end{phonetics}
\end{entry}

\begin{entry}{叉}{3}{⼜}
  \begin{phonetics}{叉}{cha1}[][HSK 5]
    \definition{s.}{garfo; forquilha | símbolo de cruz, ``×''}
    \definition{v.}{trabalhar com um garfo; garfar; pegar coisas com um garfo}
  \end{phonetics}
  \begin{phonetics}{叉}{cha2}
    \definition{v.}{bloquear; emperrar; congestionar}
  \end{phonetics}
  \begin{phonetics}{叉}{cha3}
    \definition{v.}{separar de modo a formar uma bifurcação; bifurcar}
  \end{phonetics}
\end{entry}

\begin{entry}{叉子}{3,3}{⼜、⼦}
  \begin{phonetics}{叉子}{cha1zi5}[][HSK 5]
    \definition[把]{s.}{garfo; ferramenta com mais de duas pontas em uma extremidade | tridente; forquilha; ferramentas de agricultura antigas}
  \end{phonetics}
\end{entry}

\begin{entry}{及}{3}{⼃}
  \begin{phonetics}{及}{ji2}
    \definition{conj.}{e | bem como}
  \end{phonetics}
\end{entry}

\begin{entry}{及时}{3,7}{⼃、⽇}
  \begin{phonetics}{及时}{ji2shi2}[][HSK 3]
    \definition{adj.}{oportuno; a tempo; sazonal}
    \definition{adv.}{prontamente; sem demora}
  \end{phonetics}
\end{entry}

\begin{entry}{及格}{3,10}{⼃、⽊}
  \begin{phonetics}{及格}{ji2ge2}[][HSK 4]
    \definition{v.+compl.}{passar; passar em um teste, exame, etc.}
  \end{phonetics}
\end{entry}

\begin{entry}{口}{3}{⼝}[Kangxi 30]
  \begin{phonetics}{口}{kou3}[][HSK 1]
    \definition{clas.}{para coisas com bocas (pessoas, animais domésticos, canhões, etc.) | para mordidas ou bocados}
    \definition{s.}{boca}
  \end{phonetics}
\end{entry}

\begin{entry}{口号}{3,5}{⼝、⼝}
  \begin{phonetics}{口号}{kou3 hao4}[][HSK 5]
    \definition[个]{s.}{\emph{slogan}; palavra de ordem; lema}
  \end{phonetics}
\end{entry}

\begin{entry}{口语}{3,9}{⼝、⾔}
  \begin{phonetics}{口语}{kou3 yu3}[][HSK 4]
    \definition[门]{s.}{linguagem oral; linguagem falada; linguagem coloquial; linguagem usada em conversas}
  \end{phonetics}
\end{entry}

\begin{entry}{口音}{3,9}{⼝、⾳}
  \begin{phonetics}{口音}{kou3yin1}
    \definition{s.}{sons da fala oral (linguística)}
  \end{phonetics}
  \begin{phonetics}{口音}{kou3yin5}
    \definition{s.}{sotaque | voz}
  \end{phonetics}
\end{entry}

\begin{entry}{口香糖}{3,9,16}{⼝、⾹、⽶}
  \begin{phonetics}{口香糖}{kou3xiang1tang2}
    \definition{s.}{goma de mascar | chiclete}
  \end{phonetics}
\end{entry}

\begin{entry}{口袋}{3,11}{⼝、⾐}
  \begin{phonetics}{口袋}{kou3dai4}[][HSK 4]
    \definition[个]{s.}{bolso | saco; sacola; artigos de tecido ou couro}
  \end{phonetics}
\end{entry}

\begin{entry}{口袋妖怪}{3,11,7,8}{⼝、⾐、⼥、⼼}
  \begin{phonetics}{口袋妖怪}{kou3dai4 yao1guai4}
    \definition*{s.}{\emph{Pokémon}}
  \end{phonetics}
\end{entry}

\begin{entry}{土}{3}{⼟}[Kangxi 32]
  \begin{phonetics}{土}{tu3}[][HSK 3]
    \definition*{s.}{sobrenome Tu}
    \definition{adj.}{local; nativo | folclórico; popular; indígena | fora de moda; antiquado; inculto; rústico}
    \definition{s.}{solo; terra | terreno; chão}
  \end{phonetics}
\end{entry}

\begin{entry}{土地}{3,6}{⼟、⼟}
  \begin{phonetics}{土地}{tu3di4}[][HSK 4]
    \definition[片,块]{s.}{terra; solo; chão; superfície terrestre da Terra usada para cultivar, construir edifícios e viver | território}
  \end{phonetics}
  \begin{phonetics}{土地}{tu3di5}
    \definition{s.}{deus da audeia; deus local; \emph{genius loci} deidade protetora de um local; (superstição) refere-se ao deus da terra que governa uma pequena área}
  \end{phonetics}
\end{entry}

\begin{entry}{土豆}{3,7}{⼟、⾖}
  \begin{phonetics}{土豆}{tu3dou4}[][HSK 5]
    \definition[个,片,块,斤]{s.}{batata; denominação comum da batata}
  \end{phonetics}
\end{entry}

\begin{entry}{土豆泥}{3,7,8}{⼟、⾖、⽔}
  \begin{phonetics}{土豆泥}{tu3dou4ni2}
    \definition{s.}{purê de batatas}
  \end{phonetics}
\end{entry}

\begin{entry}{土鸡}{3,7}{⼟、⿃}
  \begin{phonetics}{土鸡}{tu3ji1}
    \definition{s.}{galinha caipira}
  \end{phonetics}
\end{entry}

\begin{entry}{士兵}{3,7}{⼠、⼋}
  \begin{phonetics}{士兵}{shi4bing1}[][HSK 4]
    \definition[名,个]{s.}{soldado; militar; termo coletivo para oficiais não comissionados e soldados; os membros mais jovens do exército}
  \end{phonetics}
\end{entry}

\begin{entry}{夕阳}{3,6}{⼣、⾩}
  \begin{phonetics}{夕阳}{xi1yang2}
    \definition{s.}{pôr do sol}
  \seealsoref{日出}{ri4chu1}
  \end{phonetics}
\end{entry}

\begin{entry}{大}{3}{⼤}[Kangxi 37]
  \begin{phonetics}{大}{da4}[][HSK 1]
    \definition{adj.}{grande | enorme | maior | largo | profundo | mais velho (que) | mais antigo | mais velho | muito}
    \definition{s.}{(dialeto) pai | irmão mais velho ou mais novo do pai}
  \end{phonetics}
  \begin{phonetics}{大}{dai4}
    \definition{s.}{usado em 大夫 \dpy{dai4fu5}: médico, doutor}
    \seeref{大夫}{dai4fu5}
  \end{phonetics}
\end{entry}

\begin{entry}{大人}{3,2}{⼤、⼈}
  \begin{phonetics}{大人}{da4 ren2}[][HSK 2]
    \definition{s.}{adulto}
  \end{phonetics}
\end{entry}

\begin{entry}{大于}{3,3}{⼤、⼆}
  \begin{phonetics}{大于}{da4 yu2}[][HSK 5]
    \definition{v.}{ser maior, mais numeroso, mais importante, etc. do que}
  \end{phonetics}
\end{entry}

\begin{entry}{大口}{3,3}{⼤、⼝}
  \begin{phonetics}{大口}{da4kou3}
    \definition{s.}{grande bocado (de comida, bebida, fumo, etc.)}
  \end{phonetics}
\end{entry}

\begin{entry}{大大}{3,3}{⼤、⼤}
  \begin{phonetics}{大大}{da4 da4}[][HSK 2]
    \definition{adv.}{muito; enormemente}
  \end{phonetics}
\end{entry}

\begin{entry}{大小}{3,3}{⼤、⼩}
  \begin{phonetics}{大小}{da4 xiao3}[][HSK 2]
    \definition{adv.}{no mínimo}
    \definition[家]{s.}{tamanho | grau de antiguidade | adultos e crianças | grande ou pequeno}
  \end{phonetics}
\end{entry}

\begin{entry}{大门}{3,3}{⼤、⾨}
  \begin{phonetics}{大门}{da4 men2}[][HSK 2]
    \definition{s.}{portão | entrada}
  \end{phonetics}
\end{entry}

\begin{entry}{大马}{3,3}{⼤、⾺}
  \begin{phonetics}{大马}{da4ma3}
    \definition*{s.}{Malásia}
  \end{phonetics}
\end{entry}

\begin{entry}{大厅}{3,4}{⼤、⼚}
  \begin{phonetics}{大厅}{da4 ting1}[][HSK 5]
    \definition{s.}{\emph{hall}; saguão, uma sala grande para reuniões ou atividades em um edifício de grande porte}
  \end{phonetics}
\end{entry}

\begin{entry}{大夫}{3,4}{⼤、⼤}
  \begin{phonetics}{大夫}{da4fu1}
    \definition{s.}{oficial sênior (na China Imperial)}
  \end{phonetics}
  \begin{phonetics}{大夫}{dai4fu5}[][HSK 3]
    \definition{s.}{médico, doutor}
  \end{phonetics}
\end{entry}

\begin{entry}{大巴}{3,4}{⼤、⼰}
  \begin{phonetics}{大巴}{da4 ba1}[][HSK 4]
    \definition{s.}{ônibus}
  \end{phonetics}
\end{entry}

\begin{entry}{大方}{3,4}{⼤、⽅}
  \begin{phonetics}{大方}{da4fang5}[][HSK 4]
    \definition{adj.}{generoso | não afetado; natural e equilibrado |  de bom gosto}
  \end{phonetics}
\end{entry}

\begin{entry}{大众}{3,6}{⼤、⼈}
  \begin{phonetics}{大众}{da4 zhong4}[][HSK 4]
    \definition{s.}{massas; população; pessoas comuns; público em geral}
  \end{phonetics}
\end{entry}

\begin{entry}{大伙儿}{3,6,2}{⼤、⼈、⼉}
  \begin{phonetics}{大伙儿}{da4huo3r5}[][HSK 5]
    \definition{pron.}{todos nós; todos vocês; todo mundo; todos | equivalente a ``大家''}
  \seealsoref{大家}{da4jia1}
  \end{phonetics}
\end{entry}

\begin{entry}{大会}{3,6}{⼤、⼈}
  \begin{phonetics}{大会}{da4 hui4}[][HSK 4]
    \definition{s.}{sessão plenária; reunião geral de membros; reuniões convocadas por partidos políticos socialistas | reunião de massa; comício de massa}
  \end{phonetics}
\end{entry}

\begin{entry}{大全}{3,6}{⼤、⼊}
  \begin{phonetics}{大全}{da4quan2}
    \definition{s.}{coleção abrangente}
  \end{phonetics}
\end{entry}

\begin{entry}{大后天}{3,6,4}{⼤、⼝、⼤}
  \begin{phonetics}{大后天}{da4 hou4 tian1}
    \definition{adv.}{daqui a três dias}
  \end{phonetics}
\end{entry}

\begin{entry}{大多}{3,6}{⼤、⼣}
  \begin{phonetics}{大多}{da4 duo1}[][HSK 4]
    \definition{adv.}{majoritariamente; em sua maior parte; em sua maioria; em grande parte}
  \end{phonetics}
\end{entry}

\begin{entry}{大多数}{3,6,13}{⼤、⼣、⽁}
  \begin{phonetics}{大多数}{da4 duo1 shu4}[][HSK 2]
    \definition{s.}{a grande maioria | a vasta maioria | a maior parte}
  \end{phonetics}
\end{entry}

\begin{entry}{大妈}{3,6}{⼤、⼥}
  \begin{phonetics}{大妈}{da4 ma1}[][HSK 4]
    \definition{s.}{tia; esposa do irmão mais velho do pai | tia (homenagem às mulheres idosas)}
  \end{phonetics}
\end{entry}

\begin{entry}{大戏}{3,6}{⼤、⼽}
  \begin{phonetics}{大戏}{da4xi4}
    \definition*{s.}{Drama, Ópera Chinesa}
  \end{phonetics}
\end{entry}

\begin{entry}{大爷}{3,6}{⼤、⽗}
  \begin{phonetics}{大爷}{da4 ye5}[][HSK 4]
    \definition{s.}{irmão mais velho do pai; tio | tio (homenagem aos homens mais velhos)}
  \end{phonetics}
\end{entry}

\begin{entry}{大约}{3,6}{⼤、⽷}
  \begin{phonetics}{大约}{da4yue1}[][HSK 3]
    \definition{adv.}{aproximadamente; sobre | provavelmente}
  \end{phonetics}
\end{entry}

\begin{entry}{大自然}{3,6,12}{⼤、⾃、⽕}
  \begin{phonetics}{大自然}{da4 zi4 ran2}[][HSK 2]
    \definition{s.}{natureza}
  \end{phonetics}
\end{entry}

\begin{entry}{大衣}{3,6}{⼤、⾐}
  \begin{phonetics}{大衣}{da4 yi1}[][HSK 2]
    \definition{s.}{sobretudo}
  \end{phonetics}
\end{entry}

\begin{entry}{大声}{3,7}{⼤、⼠}
  \begin{phonetics}{大声}{da4 sheng1}[][HSK 2]
    \definition{adj.}{alto volume | em voz alta}
  \end{phonetics}
\end{entry}

\begin{entry}{大纲}{3,7}{⼤、⽷}
  \begin{phonetics}{大纲}{da4 gang1}[][HSK 5]
    \definition{s.}{esboço; compêndio; programa de estudos; resumo; fundamentos da organização sistemática de conteúdos (livros, discursos, programas, etc.)}
  \end{phonetics}
\end{entry}

\begin{entry}{大豆}{3,7}{⼤、⾖}
  \begin{phonetics}{大豆}{da4dou4}
    \definition{s.}{soja}
  \end{phonetics}
\end{entry}

\begin{entry}{大陆}{3,7}{⼤、⾩}
  \begin{phonetics}{大陆}{da4 lu4}[][HSK 4]
    \definition*{s.}{China continental; refere-se especificamente à vasta porção terrestre do território da China}
    \definition[个,块]{s.}{terra firme; continente; vasta extensão de terra}
  \end{phonetics}
\end{entry}

\begin{entry}{大事}{3,8}{⼤、⼅}
  \begin{phonetics}{大事}{da4 shi4}[][HSK 5]
    \definition[件,桩]{s.}{grande evento; grande acontecimento; assunto importante; grande questão; algo importante |
situação geral | em grande escala; em grande estilo; em grande parte}
  \end{phonetics}
\end{entry}

\begin{entry}{大使馆}{3,8,11}{⼤、⼈、⾷}
  \begin{phonetics}{大使馆}{da4shi3guan3}[][HSK 3]
    \definition[座,个]{s.}{embaixada}
  \end{phonetics}
\end{entry}

\begin{entry}{大姐}{3,8}{⼤、⼥}
  \begin{phonetics}{大姐}{da4 jie3}[][HSK 4]
    \definition[个]{s.}{irmã mais velha (também um termo educado para se dirigir a uma garota ou mulher um pouco mais velha do que a pessoa que fala)}
  \end{phonetics}
\end{entry}

\begin{entry}{大学}{3,8}{⼤、⼦}
  \begin{phonetics}{大学}{da4 xue2}[][HSK 1]
    \definition[所]{s.}{faculdade | universidade}
  \end{phonetics}
\end{entry}

\begin{entry}{大学生}{3,8,5}{⼤、⼦、⽣}
  \begin{phonetics}{大学生}{da4 xue2 sheng1}[][HSK 1]
    \definition{s.}{estudante universitário}
  \end{phonetics}
\end{entry}

\begin{entry}{大规模}{3,8,14}{⼤、⾒、⽊}
  \begin{phonetics}{大规模}{da4 gui1 mo2}[][HSK 4]
    \definition{adj.}{em larga escala; extensivo; maciço; massa}
    \definition{adj.}{em larga escala; extensivo; maciço; massivo}
  \end{phonetics}
\end{entry}

\begin{entry}{大雨}{3,8}{⼤、⾬}
  \begin{phonetics}{大雨}{da4yu3}
    \definition[场]{s.}{chuva pesada, forte}
  \end{phonetics}
\end{entry}

\begin{entry}{大前天}{3,9,4}{⼤、⼑、⼤}
  \begin{phonetics}{大前天}{da4qian2tian1}
    \definition{adv.}{três dias atrás}
  \end{phonetics}
\end{entry}

\begin{entry}{大型}{3,9}{⼤、⼟}
  \begin{phonetics}{大型}{da4xing2}[][HSK 4]
    \definition{adj.}{grande; em larga escala; tamanho e volume grandes | larga escala (importante e influente)}
  \end{phonetics}
\end{entry}

\begin{entry}{大奖赛}{3,9,14}{⼤、⼤、⾙}
  \begin{phonetics}{大奖赛}{da4 jiang3 sai4}[][HSK 5]
    \definition{s.}{grande competição; grande prêmio; \emph{grand prix}}
  \end{phonetics}
\end{entry}

\begin{entry}{大战}{3,9}{⼤、⼽}
  \begin{phonetics}{大战}{da4zhan4}
    \definition{s.}{guerra}
    \definition{v.}{guerrear | lutar em uma guerra}
  \end{phonetics}
\end{entry}

\begin{entry}{大洋洲}{3,9,9}{⼤、⽔、⽔}
  \begin{phonetics}{大洋洲}{da4yang2zhou1}
    \definition*{s.}{Oceania}
  \end{phonetics}
\end{entry}

\begin{entry}{大神}{3,9}{⼤、⽰}
  \begin{phonetics}{大神}{da4shen2}
    \definition{s.}{deidade | (gíria da Internet) guru | \emph{expert} | gênio}
  \end{phonetics}
\end{entry}

\begin{entry}{大胆}{3,9}{⼤、⾁}
  \begin{phonetics}{大胆}{da4 dan3}[][HSK 5]
    \definition{adj.}{ousado; atrevido; audacioso; corajoso; destemido}
  \end{phonetics}
\end{entry}

\begin{entry}{大哥}{3,10}{⼤、⼝}
  \begin{phonetics}{大哥}{da4 ge1}[][HSK 4]
    \definition{s.}{irmão mais velho | \emph{big brother} (endereço educado para um homem da mesma idade que você) | líder de gangue; pessoa mais poderosa em uma organização que realiza atividades ilegais na sociedade}
  \end{phonetics}
\end{entry}

\begin{entry}{大家}{3,10}{⼤、⼧}
  \begin{phonetics}{大家}{da4jia1}[][HSK 2]
    \definition{pron.}{todos}
  \end{phonetics}
\end{entry}

\begin{entry}{大海}{3,10}{⼤、⽔}
  \begin{phonetics}{大海}{da4 hai3}[][HSK 2]
    \definition{s.}{mar | oceano}
  \end{phonetics}
\end{entry}

\begin{entry}{大脑}{3,10}{⼤、⾁}
  \begin{phonetics}{大脑}{da4 nao3}[][HSK 5]
    \definition{s.}{cérebro; encéfalo}
  \end{phonetics}
\end{entry}

\begin{entry}{大致}{3,10}{⼤、⾄}
  \begin{phonetics}{大致}{da4zhi4}[][HSK 5]
    \definition{adj.}{geral; no todo}
    \definition{adv.}{grosso modo; aproximadamente; mais ou menos; indica uma estimativa aproximada da situação}
  \end{phonetics}
\end{entry}

\begin{entry}{大部分}{3,10,4}{⼤、⾢、⼑}
  \begin{phonetics}{大部分}{da4 bu4 fen4}[][HSK 2]
    \definition{s.}{a maioria | a maior parte}
  \end{phonetics}
\end{entry}

\begin{entry}{大都}{3,10}{⼤、⾢}
  \begin{phonetics}{大都}{da4 dou1}[][HSK 5]
  \end{phonetics}
  \begin{phonetics}{大都}{da4 du1}
    \definition{adv.}{em sua maior parte; na maior parte; indica que a maioria das pessoas ou coisas em um determinado intervalo tem a mesma natureza e características | também pronunciado como \dpy{da4dou1} na língua falada}
  \end{phonetics}
\end{entry}

\begin{entry}{大象}{3,11}{⼤、⾗}
  \begin{phonetics}{大象}{da4xiang4}[][HSK 5]
    \definition[只,头,群,个]{s.}{elefante}
  \end{phonetics}
\end{entry}

\begin{entry}{大猩猩}{3,12,12}{⼤、⽝、⽝}
  \begin{phonetics}{大猩猩}{da4xing1xing5}
    \definition{s.}{gorila}
  \end{phonetics}
\end{entry}

\begin{entry}{大量}{3,12}{⼤、⾥}
  \begin{phonetics}{大量}{da4 liang4}[][HSK 2]
    \definition{adj.}{numeroso | em massa | grande em número ou quantidade | generoso | magnânimo}
  \end{phonetics}
\end{entry}

\begin{entry}{大楼}{3,13}{⼤、⽊}
  \begin{phonetics}{大楼}{da4 lou2}[][HSK 4]
    \definition[座,幢]{s.}{edifício; mansão; edifício de vários andares disponível para uso residencial e comercial}
  \end{phonetics}
\end{entry}

\begin{entry}{大概}{3,13}{⼤、⽊}
  \begin{phonetics}{大概}{da4gai4}[][HSK 3]
    \definition{adj.}{geral; grosseiro; aproximado}
    \definition{adv.}{sobre; provavelmente
geralmente; brevemente}
    \definition{s.}{ideia geral; esboço geral}
  \end{phonetics}
\end{entry}

\begin{entry}{大腿}{3,13}{⼤、⾁}
  \begin{phonetics}{大腿}{da4tui3}
    \definition{s.}{coxa}
  \end{phonetics}
\end{entry}

\begin{entry}{大蒜}{3,13}{⼤、⾋}
  \begin{phonetics}{大蒜}{da4suan4}
    \definition[瓣,头]{s.}{alho}
  \end{phonetics}
\end{entry}

\begin{entry}{大熊猫}{3,14,11}{⼤、⽕、⽝}
  \begin{phonetics}{大熊猫}{da4 xiong2 mao1}[][HSK 5]
    \definition{s.}{panda gigante}
  \end{phonetics}
\end{entry}

\begin{entry}{大赛}{3,14}{⼤、⾙}
  \begin{phonetics}{大赛}{da4sai4}
    \definition{s.}{grande concurso, competição}
  \end{phonetics}
\end{entry}

\begin{entry}{女}{3}{⼥}[Kangxi 38]
  \begin{phonetics}{女}{nv3}[][HSK 1]
    \definition{adj.}{feminino}
  \end{phonetics}
\end{entry}

\begin{entry}{女人}{3,2}{⼥、⼈}
  \begin{phonetics}{女人}{nv3ren2}[][HSK 1]
    \definition[个,位]{s.}{mulher}
  \end{phonetics}
\end{entry}

\begin{entry}{女儿}{3,2}{⼥、⼉}
  \begin{phonetics}{女儿}{nv3'er2}[][HSK 1]
    \definition{s.}{filha}
  \seealsoref{儿子}{er2zi5}
  \end{phonetics}
\end{entry}

\begin{entry}{女士}{3,3}{⼥、⼠}
  \begin{phonetics}{女士}{nv3shi4}[][HSK 4]
    \definition{pron.}{Sra.; Senhorita; Senhora; título honorífico para mulheres (agora usado em contextos diplomáticos)}
    \definition[位,个]{s.}{senhora; madame}
  \end{phonetics}
\end{entry}

\begin{entry}{女子}{3,3}{⼥、⼦}
  \begin{phonetics}{女子}{nv3 zi3}[][HSK 3]
    \definition[位]{s.}{mulher; feminino}
  \end{phonetics}
\end{entry}

\begin{entry}{女王}{3,4}{⼥、⽟}
  \begin{phonetics}{女王}{nv3wang2}
    \definition{s.}{rainha}
  \end{phonetics}
\end{entry}

\begin{entry}{女生}{3,5}{⼥、⽣}
  \begin{phonetics}{女生}{nv3sheng1}[][HSK 1]
    \definition[个]{s.}{aluna | estudante so sexo feminino}
  \end{phonetics}
\end{entry}

\begin{entry}{女性}{3,8}{⼥、⼼}
  \begin{phonetics}{女性}{nv3 xing4}[][HSK 5]
    \definition[位,名]{s.}{mulher; feminino; feminilidade}
  \end{phonetics}
\end{entry}

\begin{entry}{女朋友}{3,8,4}{⼥、⽉、⼜}
  \begin{phonetics}{女朋友}{nv3peng2you5}[][HSK 1]
    \definition{s.}{namorada}
  \end{phonetics}
\end{entry}

\begin{entry}{女孩}{3,9}{⼥、⼦}
  \begin{phonetics}{女孩}{nv3hai2}
    \definition{s.}{menina | garota}
  \end{phonetics}
\end{entry}

\begin{entry}{女孩儿}{3,9,2}{⼥、⼦、⼉}
  \begin{phonetics}{女孩儿}{nv3hai2r5}[][HSK 1]
  \end{phonetics}
\end{entry}

\begin{entry}{女婿}{3,12}{⼥、⼥}
  \begin{phonetics}{女婿}{nv3xu5}
    \definition{s.}{marido da filha}
  \end{phonetics}
\end{entry}

\begin{entry}{子}{3}{⼦}
  \begin{phonetics}{子}{zi3}
    \definition{adj.}{jovem | pequeno | tenro}
    \definition{clas.}{para objetos finos que podem ser pinçados com os dedos}
    \definition{pron.}{você}
    \definition{s.}{filho | pessoa | antigo título de respeito para um homem culto ou virtuoso | semente | ovo; ova | coisas pequenas e duras | moeda de cobre; cobre | o quarto título da classificação dos cinco títulos feudais de nobreza; visconde}
  \end{phonetics}
  \begin{phonetics}{子}{zi5}[][HSK 1]
    \definition{clas.}{sufixos de palavras de medida individuais}
    \definition{suf.}{sufixo para substantivos}
  \end{phonetics}
\end{entry}

\begin{entry}{子女}{3,3}{⼦、⼥}
  \begin{phonetics}{子女}{zi3 nv3}[][HSK 3]
    \definition{s.}{crianças; descendência; filhos e filhas}
  \end{phonetics}
\end{entry}

\begin{entry}{子弹}{3,11}{⼦、⼸}
  \begin{phonetics}{子弹}{zi3dan4}[][HSK 5]
    \definition[粒,颗,发]{s.}{bala; cartucho; munição;}
  \end{phonetics}
\end{entry}

\begin{entry}{寸}{3}{⼨}
  \begin{phonetics}{寸}{cun4}[][HSK 5]
    \definition*{s.}{sobrenome Cun}
    \definition{adj.}{muito pouco; muito curto; pequeno}
    \definition{clas.}{cun, uma unidade de comprimento (=13 decímetros)}
  \end{phonetics}
\end{entry}

\begin{entry}{小}{3}{⼩}[Kangxi 42]
  \begin{phonetics}{小}{xiao3}[][HSK 1,2]
    \definition{adj.}{pequeno | jovem}
  \end{phonetics}
\end{entry}

\begin{entry}{小小}{3,3}{⼩、⼩}
  \begin{phonetics}{小小}{xiao3xiao3}
    \definition{adj.}{muito pequeno}
  \end{phonetics}
\end{entry}

\begin{entry}{小区}{3,4}{⼩、⼖}
  \begin{phonetics}{小区}{xiao3qu1}
    \definition{s.}{conjunto habitacional, comunidade, bairro | célula (telecomunicações)}
  \end{phonetics}
\end{entry}

\begin{entry}{小心}{3,4}{⼩、⼼}
  \begin{phonetics}{小心}{xiao3xin1}[][HSK 2]
    \definition{adj.}{cuidado}
  \end{phonetics}
\end{entry}

\begin{entry}{小气鬼}{3,4,9}{⼩、⽓、⿁}
  \begin{phonetics}{小气鬼}{xiao3qi4gui3}
    \definition{adj.}{avarento | mão-de-vaca | miserável | pão-duro}
  \end{phonetics}
\end{entry}

\begin{entry}{小白菜}{3,5,11}{⼩、⽩、⾋}
  \begin{phonetics}{小白菜}{xiao3bai2cai4}
    \definition[棵]{s.}{\emph{bok choy} | couve chinesa}
  \end{phonetics}
\end{entry}

\begin{entry}{小众}{3,6}{⼩、⼈}
  \begin{phonetics}{小众}{xiao3zhong4}
    \definition{s.}{minoria da população | nicho (mercado, etc.)}
  \end{phonetics}
\end{entry}

\begin{entry}{小伙子}{3,6,3}{⼩、⼈、⼦}
  \begin{phonetics}{小伙子}{xiao3huo3zi5}[][HSK 4]
    \definition[个]{s.}{rapaz jovem; jovem colega}
  \end{phonetics}
\end{entry}

\begin{entry}{小吃}{3,6}{⼩、⼝}
  \begin{phonetics}{小吃}{xiao3chi1}[][HSK 4]
    \definition{s.}{lanche; petiscos; comida com especialidades locais, não muito para uma porção | prato frio; prato feito; cortes de frios na culinária ocidental | pratos pequenos e baratos; pratos simples em restaurantes com porções pequenas e preços baixos}
  \end{phonetics}
\end{entry}

\begin{entry}{小声}{3,7}{⼩、⼠}
  \begin{phonetics}{小声}{xiao3 sheng1}[][HSK 2]
    \definition{v.}{falar em voz baixa | sussurar}
  \end{phonetics}
\end{entry}

\begin{entry}{小时}{3,7}{⼩、⽇}
  \begin{phonetics}{小时}{xiao3shi2}[][HSK 1]
    \definition{adv.}{hora | para horas}
    \definition[个]{s.}{hora}
  \end{phonetics}
\end{entry}

\begin{entry}{小时候}{3,7,10}{⼩、⽇、⼈}
  \begin{phonetics}{小时候}{xiao3 shi2 hou5}[][HSK 2]
    \definition{s.}{na infância | quando alguém era jovem}
  \end{phonetics}
\end{entry}

\begin{entry}{小姐}{3,8}{⼩、⼥}
  \begin{phonetics}{小姐}{xiao3jie5}[][HSK 1]
    \definition[个,位]{s.}{senhorita | jovem senhora | (gíria) prostituta}
  \end{phonetics}
\end{entry}

\begin{entry}{小学}{3,8}{⼩、⼦}
  \begin{phonetics}{小学}{xiao3xue2}[][HSK 1]
    \definition{s.}{escola ensino fundamental}
  \end{phonetics}
\end{entry}

\begin{entry}{小学生}{3,8,5}{⼩、⼦、⽣}
  \begin{phonetics}{小学生}{xiao3xue2sheng1}[][HSK 1]
    \definition{s.}{aluno, estudante de escola primária}
  \end{phonetics}
\end{entry}

\begin{entry}{小朋友}{3,8,4}{⼩、⽉、⼜}
  \begin{phonetics}{小朋友}{xiao3peng2you3}[][HSK 1]
    \definition{s.}{criança | [termo de tratamento usado por um adulto para uma criança] amiguinho}
  \end{phonetics}
\end{entry}

\begin{entry}{小狗}{3,8}{⼩、⽝}
  \begin{phonetics}{小狗}{xiao3 gou3}
    \definition{s.}{filhote de cachorro}
  \end{phonetics}
\end{entry}

\begin{entry}{小组}{3,8}{⼩、⽷}
  \begin{phonetics}{小组}{xiao3 zu3}[][HSK 2]
    \definition[个]{s.}{grupo}
  \end{phonetics}
\end{entry}

\begin{entry}{小型}{3,9}{⼩、⼟}
  \begin{phonetics}{小型}{xiao3 xing2}[][HSK 4]
    \definition{adj.}{de tamanho pequeno; em pequena escala; miniatura; tipo pequeno; tamanho de bolso; tipo compacto}
    \definition{s.}{(Mediterrâneo) escunas, pequenos veleiros de pesca ou turismo | pequeno \emph{rover} lunar (duas pessoas)}
  \end{phonetics}
\end{entry}

\begin{entry}{小孩儿}{3,9,2}{⼩、⼦、⼉}
  \begin{phonetics}{小孩儿}{xiao3hai2r5}[][HSK 1]
    \definition[个]{s.}{criança | bebê}
  \end{phonetics}
\end{entry}

\begin{entry}{小屋}{3,9}{⼩、⼫}
  \begin{phonetics}{小屋}{xiao3wu1}
    \definition{s.}{cabana | chalé | cabine}
  \end{phonetics}
\end{entry}

\begin{entry}{小树}{3,9}{⼩、⽊}
  \begin{phonetics}{小树}{xiao3shu4}
    \definition[棵]{s.}{muda | arbusto | árvore pequena}
  \end{phonetics}
\end{entry}

\begin{entry}{小洋白菜}{3,9,5,11}{⼩、⽔、⽩、⾋}
  \begin{phonetics}{小洋白菜}{xiao3 yang2bai2cai4}
    \definition{s.}{couve de bruxelas}
  \end{phonetics}
\end{entry}

\begin{entry}{小说}{3,9}{⼩、⾔}
  \begin{phonetics}{小说}{xiao3shuo1}[][HSK 2]
    \definition[本,部]{s.}{romance | ficção}
  \end{phonetics}
\end{entry}

\begin{entry}{小偷儿}{3,11,2}{⼩、⼈、⼉}
  \begin{phonetics}{小偷儿}{xiao3 tou1er5}[][HSK 5]
    \definition{s.}{ladrão insignificante (ou furtivo); ladrãozinho | ladrão}
  \end{phonetics}
\end{entry}

\begin{entry}{小腿}{3,13}{⼩、⾁}
  \begin{phonetics}{小腿}{xiao3tui3}
    \definition{s.}{perna (do joelho ao calcanhar) | haste}
  \end{phonetics}
\end{entry}

\begin{entry}{山}{3}{⼭}[Kangxi 46]
  \begin{phonetics}{山}{shan1}[][HSK 1]
    \definition*{s.}{sobrenome Shan}
    \definition[座]{s.}{montanha | monte | qualquer coisa que se assemelhe a uma montanha}
  \end{phonetics}
\end{entry}

\begin{entry}{山区}{3,4}{⼭、⼖}
  \begin{phonetics}{山区}{shan1 qu1}[][HSK 5]
    \definition[个]{s.}{área montanhosa; região montanhosa | colina; serra; montanha | distrito montanhoso}
  \end{phonetics}
\end{entry}

\begin{entry}{山东}{3,5}{⼭、⼀}
  \begin{phonetics}{山东}{shan1dong1}
    \definition*{s.}{Shandong}
  \end{phonetics}
\end{entry}

\begin{entry}{山羊}{3,6}{⼭、⽺}
  \begin{phonetics}{山羊}{shan1yang2}
    \definition{s.}{cabra | (ginástica) cavalo de salto de pequeno porte}
  \end{phonetics}
\end{entry}

\begin{entry}{山体}{3,7}{⼭、⼈}
  \begin{phonetics}{山体}{shan1ti3}
    \definition{s.}{forma de uma montanha}
  \end{phonetics}
\end{entry}

\begin{entry}{山谷}{3,7}{⼭、⾕}
  \begin{phonetics}{山谷}{shan1gu3}
    \definition{s.}{vale | ravina}
  \end{phonetics}
\end{entry}

\begin{entry}{山顶}{3,8}{⼭、⾴}
  \begin{phonetics}{山顶}{shan1ding3}
    \definition{s.}{cume da montanha}
  \end{phonetics}
\end{entry}

\begin{entry}{山寨}{3,14}{⼭、⼧}
  \begin{phonetics}{山寨}{shan1zhai4}
    \definition{s.}{fortaleza fortificada da vila | fortaleza da montanha (especialmente de bandidos) | falsificação | imitação | (fig.) pechincha}
  \end{phonetics}
\end{entry}

\begin{entry}{工}{3}{⼯}[Kangxi 48]
  \begin{phonetics}{工}{gong1}
    \definition{s.}{trabalho | trabalhador | habilidade | profissão | comércio | ofício}
  \end{phonetics}
\end{entry}

\begin{entry}{工人}{3,2}{⼯、⼈}
  \begin{phonetics}{工人}{gong1ren2}[][HSK 1]
    \definition{s.}{trabalhador | operário | mão de obra de fábrica}
  \end{phonetics}
\end{entry}

\begin{entry}{工厂}{3,2}{⼯、⼚}
  \begin{phonetics}{工厂}{gong1chang3}[][HSK 3]
    \definition[家,座,个]{s.}{fábrica; moinho; planta; obras}
  \end{phonetics}
\end{entry}

\begin{entry}{工夫}{3,4}{⼯、⼤}
  \begin{phonetics}{工夫}{gong1 fu1}
    \definition[个]{s.}{tempo | tempo livre; lazer}
  \end{phonetics}
  \begin{phonetics}{工夫}{gong1 fu5}[][HSK 3]
    \definition{s.}{(um período de) tempo | tempo livre}
  \end{phonetics}
\end{entry}

\begin{entry}{工尺谱}{3,4,14}{⼯、⼫、⾔}
  \begin{phonetics}{工尺谱}{gong1 che3 pu3}
    \definition{s.}{notação musical tradicional chinesa que usa caracteres chineses para representar notas musicais}
  \end{phonetics}
\end{entry}

\begin{entry}{工艺}{3,4}{⼯、⾋}
  \begin{phonetics}{工艺}{gong1 yi4}[][HSK 5]
    \definition{s.}{técnica; tecnologia; arte industrial; técnicas ou métodos de fabricação e processamento de produtos | artesanato; arte artesanal}
  \end{phonetics}
\end{entry}

\begin{entry}{工艺品}{3,4,9}{⼯、⾋、⼝}
  \begin{phonetics}{工艺品}{gong1 yi4 pin3}[][HSK 5]
    \definition[个,件]{s.}{trabalho manual; artesanato; habilidade manual; artigo artesanal; itens delicados produzidos com técnicas artesanais. Por exemplo, esculturas em jade, esmaltes Jingtailan, bordados, etc.}
  \end{phonetics}
\end{entry}

\begin{entry}{工业}{3,5}{⼯、⼀}
  \begin{phonetics}{工业}{gong1ye4}[][HSK 3]
    \definition{s.}{indústria}
  \end{phonetics}
\end{entry}

\begin{entry}{工作}{3,7}{⼯、⼈}
  \begin{phonetics}{工作}{gong1zuo4}[][HSK 1]
    \definition[个,份,项]{s.}{trabalho | tarefa}
    \definition{v.}{trabalhar | operar (uma máquina)}
  \end{phonetics}
\end{entry}

\begin{entry}{工作日}{3,7,4}{⼯、⼈、⽇}
  \begin{phonetics}{工作日}{gong1 zuo4 ri4}[][HSK 5]
    \definition{s.}{dia de trabalho; dia útil; dias em que você deveria estar trabalhando de acordo com as regras | horas de trabalho por dia; horas do dia para fazer o trabalho necessário}
  \end{phonetics}
\end{entry}

\begin{entry}{工具}{3,8}{⼯、⼋}
  \begin{phonetics}{工具}{gong1ju4}[][HSK 3]
    \definition[个]{s.}{ferramenta; implemento | ferramenta; meio; instrumento}
  \end{phonetics}
\end{entry}

\begin{entry}{工资}{3,10}{⼯、⾙}
  \begin{phonetics}{工资}{gong1zi1}[][HSK 3]
    \definition[份,个,年,月,天]{s.}{pagamento; salário}
  \end{phonetics}
\end{entry}

\begin{entry}{工程}{3,12}{⼯、⽲}
  \begin{phonetics}{工程}{gong1 cheng2}[][HSK 4]
    \definition[个,项]{s.}{projeto; programa; trabalhos que utilizam equipamentos grandes e complexos, como projetos de reconstrução urbana e projetos de cestas de alimentos, etc. | engenharia; departamentos de produção e manufatura usam equipamentos grandes e complexos para realizar seu trabalho}
  \end{phonetics}
\end{entry}

\begin{entry}{工程师}{3,12,6}{⼯、⽲、⼱}
  \begin{phonetics}{工程师}{gong1cheng2shi1}[][HSK 3]
    \definition[个,名]{s.}{engenheiro}
  \end{phonetics}
\end{entry}

\begin{entry}{工龄}{3,13}{⼯、⿒}
  \begin{phonetics}{工龄}{gong1ling2}
    \definition{s.}{tempo de serviço | senioridade}
  \end{phonetics}
\end{entry}

\begin{entry}{已}{3}{⼰}
  \begin{phonetics}{已}{yi3}[][HSK 3]
    \definition{adv.}{já | depois; mais tarde; depois de um tempo}
  \end{phonetics}
\end{entry}

\begin{entry}{已久}{3,3}{⼰、⼃}
  \begin{phonetics}{已久}{yi3jiu3}
    \definition{adv.}{já faz muito tempo}
  \end{phonetics}
\end{entry}

\begin{entry}{已灭}{3,5}{⼰、⽕}
  \begin{phonetics}{已灭}{yi3mie4}
    \definition{adj.}{extinto}
  \end{phonetics}
\end{entry}

\begin{entry}{已知}{3,8}{⼰、⽮}
  \begin{phonetics}{已知}{yi3zhi1}
    \definition{v.}{conhecido (ter ciência)}
  \end{phonetics}
\end{entry}

\begin{entry}{已经}{3,8}{⼰、⽷}
  \begin{phonetics}{已经}{yi3jing1}[][HSK 2]
    \definition{adv.}{já}
  \end{phonetics}
\end{entry}

\begin{entry}{已故}{3,9}{⼰、⽁}
  \begin{phonetics}{已故}{yi3gu4}
    \definition{adj.}{morto | atrasado}
  \end{phonetics}
\end{entry}

\begin{entry}{已婚}{3,11}{⼰、⼥}
  \begin{phonetics}{已婚}{yi3hun1}
    \definition{adj.}{casado}
  \end{phonetics}
\end{entry}

\begin{entry}{已然}{3,12}{⼰、⽕}
  \begin{phonetics}{已然}{yi3ran2}
    \definition{adv.}{já | já ser assim}
  \end{phonetics}
\end{entry}

\begin{entry}{干}{3}{⼲}
  \begin{phonetics}{干}{gan1}[][HSK 1]
    \definition*{s.}{sobrenome Gan}
    \definition{v.}{preocupar | ignorar | interferir}
  \end{phonetics}
  \begin{phonetics}{干}{gan4}[][HSK 1]
    \definition{v.}{fazer | gerenciar | trabalhar | (gíria) matar | (vulgar) foder}
  \end{phonetics}
\end{entry}

\begin{entry}{干与}{3,3}{⼲、⼀}
  \begin{phonetics}{干与}{gan1yu4}
    \variantof{干预}
  \end{phonetics}
\end{entry}

\begin{entry}{干什么}{3,4,3}{⼲、⼈、⼃}
  \begin{phonetics}{干什么}{gan4 shen2 me5}[][HSK 1]
    \definition{v.}{o que fazer? | o que está fazendo?}
  \end{phonetics}
\end{entry}

\begin{entry}{干吗}{3,6}{⼲、⼝}
  \begin{phonetics}{干吗}{gan4 ma2}[][HSK 3]
    \definition{pron.}{por que?}
    \definition{v.}{o que fazer?}
  \end{phonetics}
\end{entry}

\begin{entry}{干你屁事}{3,7,7,8}{⼲、⼈、⼫、⼅}
  \begin{phonetics}{干你屁事}{gan1 ni3 pi4shi4}
    \definition{interj.}{Foda-se!}
  \end{phonetics}
\end{entry}

\begin{entry}{干扰}{3,7}{⼲、⼿}
  \begin{phonetics}{干扰}{gan1rao3}[][HSK 5]
    \definition{v.}{perturbar; incomodar | interferir; interromper o funcionamento adequado de equipamentos eletrônicos com sinais eletrônicos dispersos}
  \end{phonetics}
\end{entry}

\begin{entry}{干净}{3,8}{⼲、⼎}
  \begin{phonetics}{干净}{gan1jing4}[][HSK 1]
    \definition{adj.}{limpo | arrumado}
  \end{phonetics}
\end{entry}

\begin{entry}{干杯}{3,8}{⼲、⽊}
  \begin{phonetics}{干杯}{gan1bei1}[][HSK 2]
    \definition{interj.}{Saúde!}
    \definition{v.+compl.}{fazer um brinde | brindar até a última gota}
  \end{phonetics}
\end{entry}

\begin{entry}{干活}{3,9}{⼲、⽔}
  \begin{phonetics}{干活}{gan4huo2}
    \definition{v.+compl.}{trabalhar | trabalhar em um emprego}
  \end{phonetics}
\end{entry}

\begin{entry}{干活儿}{3,9,2}{⼲、⽔、⼉}
  \begin{phonetics}{干活儿}{gan4huo2r5}[][HSK 2]
    \definition{v.}{trabalhar em um emprego}
  \end{phonetics}
\end{entry}

\begin{entry}{干脆}{3,10}{⼲、⾁}
  \begin{phonetics}{干脆}{gan1cui4}[][HSK 5]
    \definition{adj.}{claro; direto; (falando, fazendo coisas) sem hesitação; atitude clara}
    \definition{adv.}{justamente; diretamente; sem maiores considerações}
  \end{phonetics}
\end{entry}

\begin{entry}{干预}{3,10}{⼲、⾴}
  \begin{phonetics}{干预}{gan1yu4}[][HSK 5]
    \definition{s.}{intromissão; intervenção}
    \definition{v.}{intrometer-se; intervir; interpor-se;}
  \end{phonetics}
\end{entry}

\begin{entry}{广}{3}{⼴}
  \begin{phonetics}{广}{an1}
    \definition{s.}{mais comum em nomes de pessoas; o mesmo que ``庵''}[广安是我的朋友。(An'an é meu amigo.)]
  \seealsoref{庵}{an1}
  \end{phonetics}
  \begin{phonetics}{广}{guang3}[][HSK 5]
    \definition{adj.}{largo; vasto; amplo; extenso; (oposto a ``狭'')}
  \seealsoref{狭}{xia2}
  \end{phonetics}
  \begin{phonetics}{广}{yan3}
    \definition[家]{s.}{casa ou edifício construído contra ou ao longo da encosta de uma montanha ou penhasco}
  \end{phonetics}
\end{entry}

\begin{entry}{广大}{3,3}{⼴、⼤}
  \begin{phonetics}{广大}{guang3da4}[][HSK 3]
    \definition{adj.}{muito difundido | (uma área ou espaço) vasto; extenso; em grande escala | numeroso}
  \end{phonetics}
\end{entry}

\begin{entry}{广东}{3,5}{⼴、⼀}
  \begin{phonetics}{广东}{guang3dong1}
    \definition*{s.}{Guangdong}
  \end{phonetics}
\end{entry}

\begin{entry}{广场}{3,6}{⼴、⼟}
  \begin{phonetics}{广场}{guang3chang3}[][HSK 2]
    \definition{s.}{praça | praça pública | esplanada}
  \end{phonetics}
\end{entry}

\begin{entry}{广场舞}{3,6,14}{⼴、⼟、⾇}
  \begin{phonetics}{广场舞}{guang3chang3wu3}
    \definition{s.}{quadrilha, uma rotina de exercícios tocada com música em quadrados públicos, parques e praças, popular especialmente entre mulheres de meia-idade e aposentados na China}
  \end{phonetics}
\end{entry}

\begin{entry}{广告}{3,7}{⼴、⼝}
  \begin{phonetics}{广告}{guang3gao4}[][HSK 2]
    \definition[项]{s.}{publicidade | anúncio publicitário}
  \end{phonetics}
\end{entry}

\begin{entry}{广泛}{3,7}{⼴、⽔}
  \begin{phonetics}{广泛}{guang3fan4}[][HSK 5]
    \definition{adj.}{amplo; extenso; de grande alcance; disseminado; escopo e cobertura amplos}
  \end{phonetics}
\end{entry}

\begin{entry}{广播}{3,15}{⼴、⼿}
  \begin{phonetics}{广播}{guang3bo1}[][HSK 3]
    \definition[个]{s.}{programa de rádio; transmissão (de rádio)}
    \definition{v.}{transmitir; estar no ar | espalhar-se amplamente; ser conhecido em toda parte}
  \end{phonetics}
\end{entry}

\begin{entry}{才}{3}{⼿}
  \begin{phonetics}{才}{cai2}[][HSK 2,4]
    \definition*{s.}{sobrenome Cai}
    \definition{adv.}{indica que algo aconteceu há pouco tempo, agora mesmo | indica que algo acontece ou termina tarde | indica que algo só acontece sob certas condições, ou por um motivo ou propósito específico, seguido do que acontece depois, geralmente é precedida por palavras como “somente”, “deve”, “porque” ou “devido a” | em comparação, indica uma pequena quantidade, poucas ocorrências, pouca habilidade, etc.; meramente | indica ênfase no que está sendo dito, e o caractere “呢” é frequentemente usado no final da frase}
    \definition{conj.}{apenas quando}
    \definition{s.}{capacidade; talento; dom | pessoa capacitada}
  \seealsoref{呢}{ne5}
  \end{phonetics}
\end{entry}

\begin{entry}{才华}{3,6}{⼿、⼗}
  \begin{phonetics}{才华}{cai2hua2}
    \definition[份]{s.}{talento}
  \end{phonetics}
\end{entry}

\begin{entry}{才能}{3,10}{⼿、⾁}
  \begin{phonetics}{才能}{cai2 neng2}[][HSK 3]
    \definition[间]{s.}{talento | habilidade | dom | capacidade}
  \end{phonetics}
\end{entry}

\begin{entry}{才略}{3,11}{⼿、⽥}
  \begin{phonetics}{才略}{cai2lve4}
    \definition{s.}{habilidade e sagacidade}
  \end{phonetics}
\end{entry}

\begin{entry}{门}{3}{⾨}[Kangxi 169]
  \begin{phonetics}{门}{men2}[][HSK 1]
    \definition*{s.}{sobrenome Men}
    \definition{clas.}{para canhão | para lição de casa, tecnologia, etc.}
    \definition{s.}{porta | portão | entrada; saída | interruptor | válvula |maneira | método | acesso | família | casa | escola (de pensamento) | seita (religiosa) | ramo de estudo | categoria; classe | filo}
  \end{phonetics}
\end{entry}

\begin{entry}{门口}{3,3}{⾨、⼝}
  \begin{phonetics}{门口}{men2kou3}[][HSK 1]
    \definition[个]{s.}{porta | portão}
  \end{phonetics}
\end{entry}

\begin{entry}{门诊}{3,7}{⾨、⾔}
  \begin{phonetics}{门诊}{men2 zhen3}[][HSK 5]
    \definition{s.}{(no hospital) clínica ambulatorial; seção para pacientes ambulatoriais; local onde os médicos atendem pacientes que não estão internados no hospital}
  \end{phonetics}
\end{entry}

\begin{entry}{门票}{3,11}{⾨、⽰}
  \begin{phonetics}{门票}{men2piao4}[][HSK 1]
    \definition{s.}{bilhete de entrada | bilhete de admissão}
  \end{phonetics}
\end{entry}

\begin{entry}{飞}{3}{⾶}[Kangxi 183]
  \begin{phonetics}{飞}{fei1}[][HSK 1]
    \definition*{s.}{sobrenome Fei}
    \definition{adj.}{inesperado | acidental | infundado | sem fundamento}
    \definition{adv.}{rapidamente}
    \definition{s.}{roda livre de uma bicicleta}
    \definition{v.}{voar | esvoaçar | flutuar no ar | volatilizar}
  \end{phonetics}
\end{entry}

\begin{entry}{飞机}{3,6}{⾶、⽊}
  \begin{phonetics}{飞机}{fei1ji1}[][HSK 1]
    \definition[架]{s.}{avião}
  \end{phonetics}
\end{entry}

\begin{entry}{飞机票}{3,6,11}{⾶、⽊、⽰}
  \begin{phonetics}{飞机票}{fei1ji1 piao4}
    \definition[张]{s.}{bilhete de avião}
  \seealsoref{机票}{ji1 piao4}
  \end{phonetics}
\end{entry}

\begin{entry}{飞行}{3,6}{⾶、⾏}
  \begin{phonetics}{飞行}{fei1 xing2}[][HSK 3]
    \definition{s.}{voo | aviação}
    \definition{v.}{voar; fazer um voo | (aviões, foguetes, etc.) voar no ar}
  \end{phonetics}
\end{entry}

\begin{entry}{飞船}{3,11}{⾶、⾈}
  \begin{phonetics}{飞船}{fei1chuan2}
    \definition{s.}{espaçonave | dirigível | aeronave}
  \end{phonetics}
\end{entry}

\begin{entry}{飞碟}{3,14}{⾶、⽯}
  \begin{phonetics}{飞碟}{fei1die2}
    \definition{s.}{disco-voador, OVNI, \emph{UFO} | \emph{frisbee}}
  \end{phonetics}
\end{entry}

\begin{entry}{马}{3}{⾺}[Kangxi 187]
  \begin{phonetics}{马}{ma3}[][HSK 3]
    \definition*{s.}{sobrenome Ma}
    \definition{adj.}{grande}
    \definition[匹]{s.}{cavalo | a peça do cavalo no xadrez chinês}
  \end{phonetics}
\end{entry}

\begin{entry}{马上}{3,3}{⾺、⼀}
  \begin{phonetics}{马上}{ma3shang4}[][HSK 1]
    \definition{adv.}{já | imediatamente | de imediato | sem demora}
  \end{phonetics}
\end{entry}

\begin{entry}{马马虎虎}{3,3,8,8}{⾺、⾺、⾌、⾌}
  \begin{phonetics}{马马虎虎}{ma3ma3hu3hu3}
    \definition{adj.}{descuidado | casual | tolerável | vago | mais ou menos}
  \end{phonetics}
\end{entry}

\begin{entry}{马耳他}{3,6,5}{⾺、⽿、⼈}
  \begin{phonetics}{马耳他}{ma3'er3ta1}
    \definition*{s.}{Malta}
  \end{phonetics}
\end{entry}

\begin{entry}{马克思列宁主义}{3,7,9,6,5,5,3}{⾺、⼗、⼼、⼑、⼧、⼂、⼂}
  \begin{phonetics}{马克思列宁主义}{ma3ke4si1 lie4ning2zhu3yi4}
    \definition*{s.}{Marxismo-Leninismo}
  \end{phonetics}
\end{entry}

\begin{entry}{马尾}{3,7}{⾺、⼫}
  \begin{phonetics}{马尾}{ma3wei3}
    \definition{s.}{(penteado) rabo de cavalo | cauda de cavalo}
  \end{phonetics}
\end{entry}

\begin{entry}{马路}{3,13}{⾺、⾜}
  \begin{phonetics}{马路}{ma3lu4}[][HSK 1]
    \definition[条]{s.}{rua | estrada}
  \end{phonetics}
\end{entry}

%%%%% EOF %%%%%


%%%
%%% 4画
%%%

\section*{4画}\addcontentsline{toc}{section}{4画}

\begin{entry}{不}{4}[Radical 一]
  \begin{phonetics}{不}{bu2}[(antes de quarto tom)][1]
    \definition{adv.}{não}
    \definition{pref.}{prefixo negativo}
  \end{phonetics}
  \begin{phonetics}{不}{bu4}[][HSK 1]
    \definition{adv.}{não}
    \definition{pref.}{prefixo negativo}
  \end{phonetics}
  \begin{phonetics}{不}{bu5}[][HSK 1]
    \definition{adv.}{não (em expressões ``v.+不+v.'')}
  \end{phonetics}
\end{entry}

\begin{entry}{不一会儿}{4,1,6,2}
  \begin{phonetics}{不一会儿}{bu4 yi2 hui4r5}[][HSK 2]
    \definition{expr.}{em um momento | em pouco tempo |em breve}
  \end{phonetics}
\end{entry}

\begin{entry}{不一定}{4,1,8}
  \begin{phonetics}{不一定}{bu4 yi2 ding4}[][HSK 2]
    \definition{adv.}{talvez | incerto | não tenho certeza | não necessariamente}
  \end{phonetics}
\end{entry}

\begin{entry}{不久}{4,3}
  \begin{phonetics}{不久}{bu4 jiu3}[][HSK 2]
    \definition{adj.}{em breve | futuro próximo | logo depois | não muito depois | não muito tempo (antes ou depois de algo)}
  \end{phonetics}
\end{entry}

\begin{entry}{不大}{4,3}
  \begin{phonetics}{不大}{bu2 da4}[][HSK 1]
    \definition{adv.}{não muito | não frequentemente | raramente |dificilmente | escassamente}
  \end{phonetics}
\end{entry}

\begin{entry}{不大离}{4,3,10}
  \begin{phonetics}{不大离}{bu2da4li2}
    \definition{adj.}{bem perto | quase certo | nada mal}
  \end{phonetics}
\end{entry}

\begin{entry}{不公}{4,4}
  \begin{phonetics}{不公}{bu4gong1}
    \definition{adj.}{injusto}
  \end{phonetics}
\end{entry}

\begin{entry}{不太}{4,4}
  \begin{phonetics}{不太}{bu2 tai4}[][HSK 2]
    \definition{adv.}{não bastante | não muito}
  \end{phonetics}
\end{entry}

\begin{entry}{不少}{4,4}
  \begin{phonetics}{不少}{bu4 shao3}[][HSK 2]
    \definition{adj.}{muitos | bastante | não poucos}
  \end{phonetics}
\end{entry}

\begin{entry}{不日}{4,4}
  \begin{phonetics}{不日}{bu2ri4}
    \definition{adv.}{em alguns dias}
  \end{phonetics}
\end{entry}

\begin{entry}{不止}{4,4}
  \begin{phonetics}{不止}{bu4zhi3}
    \definition{adv.}{incessantemente | sem fim | mais que | não limitado a}
  \end{phonetics}
\end{entry}

\begin{entry}{不可避免}{4,5,16,7}
  \begin{phonetics}{不可避免}{bu4ke3bi4mian3}
    \definition{adj./adv.}{inevitável}
  \end{phonetics}
\end{entry}

\begin{entry}{不对}{4,5}
  \begin{phonetics}{不对}{bu2dui4}[][HSK 1]
    \definition{adj.}{incorreto | errado | anormal | estranho | estar em desacordo com | ser difícil de conviver}
  \end{phonetics}
\end{entry}

\begin{entry}{不用}{4,5}
  \begin{phonetics}{不用}{bu2yong4}[][HSK 1]
    \definition{v.}{não precisar}
    \seeref{甭}{beng2}
  \end{phonetics}
\end{entry}

\begin{entry}{不同}{4,6}
  \begin{phonetics}{不同}{bu4 tong2}
    \definition{adj.}{diferente | distinto}
  \end{phonetics}
\end{entry}

\begin{entry}{不好意思}{4,6,13,9}
  \begin{phonetics}{不好意思}{bu4 hao3 yi4 si5}[][HSK 2]
    \definition{adj.}{pedir desculpas (por incomodar alguém) | sentir-se envergonhado | achar isso embaraçoso}
  \end{phonetics}
\end{entry}

\begin{entry}{不如}{4,6}
  \begin{phonetics}{不如}{bu4ru2}[][HSK 2]
    \definition{conj.}{em vez de | melhor que | seria melhor}
    \definition{v.}{ser inferior a | não ser igual a | não ser tão bom quanto | não poder fazer melhor que}
  \end{phonetics}
\end{entry}

\begin{entry}{不成话}{4,6,8}
  \begin{phonetics}{不成话}{bu4cheng2hua4}
    \definition{expr.}{sem razão | demasiado irracionável}
    \seeref{不是话}{bu2shi4hua4}
    \seeref{不像话}{bu2xiang4hua4}
  \end{phonetics}
\end{entry}

\begin{entry}{不行}{4,6}
  \begin{phonetics}{不行}{bu4 xing2}[][HSK 2]
    \definition{adj.}{não funciona | não é bom}
    \definition{adv.}{profundamente | terrivelmente | extremamente}
    \definition{v.}{não fazer | não ser permitido | estar fora de questão | estar à beira da morte}
  \end{phonetics}
\end{entry}

\begin{entry}{不论……也……}{4,6,3}
  \begin{phonetics}{不论……也……}{bu2lun4 ye3}
    \definition{conj.}{não apenas\dots, (o que, quem, como, etc.), \dots}
  \end{phonetics}
\end{entry}

\begin{entry}{不论……都……}{4,6,10}
  \begin{phonetics}{不论……都……}{bu2lun4 dou1}
    \definition{conj.}{não apenas\dots, (o que, quem, como, etc.), \dots}
  \end{phonetics}
\end{entry}

\begin{entry}{不过}{4,6}
  \begin{phonetics}{不过}{bu2guo4}[][HSK 2]
    \definition{conj.}{mas | contudo | no entanto}
  \end{phonetics}
\end{entry}

\begin{entry}{不但}{4,7}
  \begin{phonetics}{不但}{bu2 dan4}[][HSK 2]
    \definition{conj.}{não somente}
  \end{phonetics}
\end{entry}

\begin{entry}{不但……而且……}{4,7,6,5}
  \begin{phonetics}{不但……而且……}{bu2 dan4 er2qie3}[][HSK 2]
    \definition{conj.}{não só\dots mas também\dots}
  \end{phonetics}
\end{entry}

\begin{entry}{不到}{4,8}
  \begin{phonetics}{不到}{bu2dao4}
    \definition{adj.}{insuficiente}
    \definition{adv.}{menos que}
    \definition{v.}{não chegar}
  \end{phonetics}
\end{entry}

\begin{entry}{不注意}{4,8,13}
  \begin{phonetics}{不注意}{bu2zhu4yi4}
    \definition{adj.}{impensado | distraído}
    \definition{s.}{descuido | distração}
  \end{phonetics}
\end{entry}

\begin{entry}{不客气}{4,9,4}
  \begin{phonetics}{不客气}{bu2 ke4qi5}[][HSK 1]
    \definition{adj.}{indelicado | rude | brusco}
    \definition{expr.}{de nada | não há de que | não mencione isso}
  \end{phonetics}
\end{entry}

\begin{entry}{不是话}{4,9,8}
  \begin{phonetics}{不是话}{bu2shi4hua4}
    \definition{expr.}{sem razão | demasiado irracionável}
    \seeref{不像话}{bu2xiang4hua4}
    \seeref{不成话}{bu4cheng2hua4}
  \end{phonetics}
\end{entry}

\begin{entry}{不要}{4,9}
  \begin{phonetics}{不要}{bu2 yao4}[][HSK 2]
    \definition{adv.}{nada de (pedir a alguém para não fazer) | não}
  \end{phonetics}
\end{entry}

\begin{entry}{不够}{4,11}
  \begin{phonetics}{不够}{bu2 gou4}[][HSK 2]
    \definition{adv.}{insuficiente}
    \definition{v.}{não ser suficiente}
  \end{phonetics}
\end{entry}

\begin{entry}{不断}{4,11}
  \begin{phonetics}{不断}{bu2duan4}
    \definition{adv.}{continuamente | sem fim}
  \end{phonetics}
\end{entry}

\begin{entry}{不像话}{4,13,8}
  \begin{phonetics}{不像话}{bu2xiang4hua4}
    \definition{expr.}{sem razão | demasiado irracionável}
    \seeref{不是话}{bu2shi4hua4}
    \seeref{不成话}{bu4cheng2hua4}
  \end{phonetics}
\end{entry}

\begin{entry}{不满}{4,13}
  \begin{phonetics}{不满}{bu4 man3}[][HSK 2]
    \definition{adj.}{ressentido | insatisfeito | descontente}
    \definition{v.}{estar descontente com |ser menor que}
  \end{phonetics}
\end{entry}

\begin{entry}{不错}{4,13}
  \begin{phonetics}{不错}{bu2 cuo4}[][HSK 2]
    \definition{adj.}{correto | não (é) mau | bastante bom | certo}
  \end{phonetics}
\end{entry}

\begin{entry}{不管……也……}{4,14,3}
  \begin{phonetics}{不管……也……}{bu4guan3 ye3}
    \definition{conj.}{não apenas\dots, (o que, quem, como, etc.), \dots}
  \end{phonetics}
\end{entry}

\begin{entry}{不管……都……}{4,14,10}
  \begin{phonetics}{不管……都……}{bu4guan3 dou1}
    \definition{conj.}{não apenas\dots, (o que, quem, como, etc.), \dots}
  \end{phonetics}
\end{entry}

\begin{entry}{专业}{4,5}
  \begin{phonetics}{专业}{zhuan1ye4}
    \definition[门,个]{s.}{área de atuação | especialidade}
  \end{phonetics}
\end{entry}

\begin{entry}{专业人士}{4,5,2,3}
  \begin{phonetics}{专业人士}{zhuan1ye4ren2shi4}
    \definition{s.}{profissional}
  \end{phonetics}
\end{entry}

\begin{entry}{专业人才}{4,5,2,3}
  \begin{phonetics}{专业人才}{zhuan1ye4ren2cai2}
    \definition{s.}{especialista (em uma área)}
  \end{phonetics}
\end{entry}

\begin{entry}{专业化}{4,5,4}
  \begin{phonetics}{专业化}{zhuan1ye4hua4}
    \definition{s.}{especialização}
  \end{phonetics}
\end{entry}

\begin{entry}{专业户}{4,5,4}
  \begin{phonetics}{专业户}{zhuan1ye4hu4}
    \definition{s.}{indústria caseira | empresa familiar produzindo um produto especial}
  \end{phonetics}
\end{entry}

\begin{entry}{专业性}{4,5,8}
  \begin{phonetics}{专业性}{zhuan1ye4xing4}
    \definition{s.}{profissionalismo | expertise}
  \end{phonetics}
\end{entry}

\begin{entry}{专业教育}{4,5,11,8}
  \begin{phonetics}{专业教育}{zhuan1ye4jiao4yu4}
    \definition{s.}{educação especializada | escola técnica}
  \end{phonetics}
\end{entry}

\begin{entry}{中}{4}[Radical 丨]
  \begin{phonetics}{中}{zhong1}
    \definition*{s.}{China}
    \definition*{s.}{sobrenome Zhong}
    \definition{s.}{centro | meio | médio | intermediário | média | meio caminho entre dois extremos | intermediador}
  \seealsoref{中国}{zhong1guo2}
  \end{phonetics}
  \begin{phonetics}{中}{zhong4}
    \definition{v.}{acertar | encaixar exatamente |ser atingido por | cair em | ser afetado por | sofrer | sustentar}
  \end{phonetics}
\end{entry}

\begin{entry}{中小学}{4,3,8}
  \begin{phonetics}{中小学}{zhong1 xiao3 xue2}[][HSK 2]
    \definition{s.}{escolas primárias e secundárias}
  \end{phonetics}
\end{entry}

\begin{entry}{中午}{4,4}
  \begin{phonetics}{中午}{zhong1wu3}[][HSK 1]
    \definition[个]{s.}{meio-dia}
  \end{phonetics}
\end{entry}

\begin{entry}{中心}{4,4}
  \begin{phonetics}{中心}{zhong1xin1}[][HSK 2]
    \definition[个]{s.}{núcleo | coração | meio | centro |chave}
  \end{phonetics}
\end{entry}

\begin{entry}{中文}{4,4}
  \begin{phonetics}{中文}{zhong1wen2}[][HSK 1]
    \definition{s.}{chinês, língua chinesa}
  \end{phonetics}
\end{entry}

\begin{entry}{中东}{4,5}
  \begin{phonetics}{中东}{zhong1dong1}
    \definition*{s.}{Oriente Médio}
  \end{phonetics}
\end{entry}

\begin{entry}{中央情报局}{4,5,11,7,7}
  \begin{phonetics}{中央情报局}{zhong1yang1 qing2bao4ju2}
    \definition*{s.}{Agência Central de Inteligência dos EUA, CIA}
  \end{phonetics}
\end{entry}

\begin{entry}{中年}{4,6}
  \begin{phonetics}{中年}{zhong1 nian2}[][HSK 2]
    \definition{s.}{meia-idade}
  \end{phonetics}
\end{entry}

\begin{entry}{中级}{4,6}
  \begin{phonetics}{中级}{zhong1 ji2}[][HSK 2]
    \definition{adj.}{nível médio | nível intermediário}
  \end{phonetics}
\end{entry}

\begin{entry}{中医}{4,7}
  \begin{phonetics}{中医}{zhong1 yi1}[][HSK 2]
    \definition{s.}{ciência médica tradicional chinesa | médico de medicina tradicional chinesa | praticante de medicina chinesa}
  \end{phonetics}
\end{entry}

\begin{entry}{中间}{4,7}
  \begin{phonetics}{中间}{zhong1jian1}[][HSK 1]
    \definition{adv.}{central | centro | no meio}
  \end{phonetics}
\end{entry}

\begin{entry}{中国}{4,8}
  \begin{phonetics}{中国}{zhong1guo2}[][HSK 1]
    \definition*{s.}{China}
  \end{phonetics}
\end{entry}

\begin{entry}{中国人}{4,8,2}
  \begin{phonetics}{中国人}{zhong1guo2ren2}
    \definition{s.}{chinês | pessoa ou povo da China}
  \end{phonetics}
\end{entry}

\begin{entry}{中国城}{4,8,9}
  \begin{phonetics}{中国城}{zhong1guo2cheng2}
    \definition*{s.}{Bairro Chinês, \emph{Chinatown}}
    \seeref{唐人街}{tang2ren2 jie1}
  \end{phonetics}
\end{entry}

\begin{entry}{中国科学院}{4,8,9,8,9}
  \begin{phonetics}{中国科学院}{zhong1guo2 ke1xue2yuan4}
    \definition*{s.}{Academia Chinesa de Ciências}
  \end{phonetics}
\end{entry}

\begin{entry}{中国通}{4,8,10}
  \begin{phonetics}{中国通}{zhong1guo2tong1}
    \definition*{s.}{Conhecedor da China, especialista em tudo sobre a China}
  \end{phonetics}
\end{entry}

\begin{entry}{中学}{4,8}
  \begin{phonetics}{中学}{zhong1xue2}[][HSK 1]
    \definition[个]{s.}{escola ensino médio}
  \end{phonetics}
\end{entry}

\begin{entry}{中学生}{4,8,5}
  \begin{phonetics}{中学生}{zhong1xue2sheng1}[][HSK 1]
    \definition{s.}{aluno, estudante de escola ensino médio}
  \end{phonetics}
\end{entry}

\begin{entry}{中性}{4,8}
  \begin{phonetics}{中性}{zhong1xing4}
    \definition{adj.}{neutro}
  \end{phonetics}
\end{entry}

\begin{entry}{中询}{4,8}
  \begin{phonetics}{中询}{zhong1 xun2}
    \definition{adv.}{segunda dezena do mês | meio do mês | em meados do mês}
  \end{phonetics}
\end{entry}

\begin{entry}{中秋节}{4,9,5}
  \begin{phonetics}{中秋节}{zhong1qiu1jie2}
    \definition*{s.}{Festival do Meio-Outono | Festival do Bolo Lunar (15º dia do oitavo mês lunar)}
  \end{phonetics}
\end{entry}

\begin{entry}{中药}{4,9}
  \begin{phonetics}{中药}{zhong1yao4}
    \definition[服,种]{s.}{medicina tradicional chinesa}
  \end{phonetics}
\end{entry}

\begin{entry}{中情局}{4,11,7}
  \begin{phonetics}{中情局}{zhong1qing2ju2}
    \definition*{s.}{Agência Central de Inteligência dos EUA, CIA (abreviação de 中央情报局)}
    \seeref{中央情报局}{zhong1yang1 qing2bao4ju2}
  \end{phonetics}
\end{entry}

\begin{entry}{中意}{4,13}
  \begin{phonetics}{中意}{zhong4yi4}
    \definition{s.}{ser do seu agrado | começar a gostar muito de algo ou de alguém}
  \end{phonetics}
\end{entry}

\begin{entry}{中餐}{4,16}
  \begin{phonetics}{中餐}{zhong1 can1}[][HSK 2]
    \definition[分,顿]{s.}{comida chinesa | almoço}
  \end{phonetics}
\end{entry}

\begin{entry}{丰收}{4,6}
  \begin{phonetics}{丰收}{feng1shou1}
    \definition{s.}{colheita abundante}
  \end{phonetics}
\end{entry}

\begin{entry}{为}{4}[Radical 丶]
  \begin{phonetics}{为}{wei2}[][HSK 0]
    \definition{prep.}{como (na capacidade de) | por (na voz passiva)}
    \definition{v.}{tomar algo como | agir como | servir como | comportar-se como | tornar-se}
  \end{phonetics}
  \begin{phonetics}{为}{wei4}[][HSK 2]
    \definition{prep.}{para | porque}
  \end{phonetics}
\end{entry}

\begin{entry}{为什么}{4,4,3}
  \begin{phonetics}{为什么}{wei4shen2me5}[][HSK 2]
    \definition{adv.}{por que?}
  \end{phonetics}
\end{entry}

\begin{entry}{乌克兰}{4,7,5}
  \begin{phonetics}{乌克兰}{wu1ke4lan2}
    \definition*{s.}{Ucrânia}
  \end{phonetics}
\end{entry}

\begin{entry}{乌龟}{4,7}
  \begin{phonetics}{乌龟}{wu1gui1}
    \definition{s.}{tartaruga}
  \end{phonetics}
\end{entry}

\begin{entry}{书}{4}[Radical 乙]
  \begin{phonetics}{书}{shu1}[][HSK 1]
    \definition[本,册,部]{s.}{livro | carta | documento}
  \end{phonetics}
\end{entry}

\begin{entry}{书包}{4,5}
  \begin{phonetics}{书包}{shu1bao1}[][HSK 1]
    \definition[个,款]{s.}{mochila escolar}
  \end{phonetics}
\end{entry}

\begin{entry}{书记}{4,5}
  \begin{phonetics}{书记}{shu1ji5}
    \definition{s.}{secretário (chefe de um ramo de um partido socialista ou comunista) | atendente | balconista | escriturário}
  \end{phonetics}
\end{entry}

\begin{entry}{书店}{4,8}
  \begin{phonetics}{书店}{shu1dian4}[][HSK 1]
    \definition[家]{s.}{livraria}
  \end{phonetics}
\end{entry}

\begin{entry}{云}{4}[Radical 二]
  \begin{phonetics}{云}{yun2}[][HSK 2]
    \definition*{s.}{sobrenome Yun}
    \definition[朵]{s.}{nuvem}
  \end{phonetics}
\end{entry}

\begin{entry}{云云}{4,4}
  \begin{phonetics}{云云}{yun2yun2}
    \definition{adv.}{e assim por diante | assim e assim}
  \end{phonetics}
\end{entry}

\begin{entry}{云南}{4,9}
  \begin{phonetics}{云南}{yun2nan2}
    \definition*{s.}{Yunnan}
  \end{phonetics}
\end{entry}

\begin{entry}{云端}{4,14}
  \begin{phonetics}{云端}{yun2duan1}
    \definition{s.}{alto nas nuvens | (computação) a nuvem}
  \end{phonetics}
\end{entry}

\begin{entry}{互}{4}[Radical ⼆]
  \begin{phonetics}{互}{hu4}
    \definition{adj.}{mútuo | recíproco}
  \end{phonetics}
\end{entry}

\begin{entry}{互动}{4,6}
  \begin{phonetics}{互动}{hu4dong4}
    \definition{s.}{interativo}
    \definition{v.}{interagir}
  \end{phonetics}
\end{entry}

\begin{entry}{互利}{4,7}
  \begin{phonetics}{互利}{hu4li4}
    \definition{s.}{benefício mútuo}
  \end{phonetics}
\end{entry}

\begin{entry}{互相}{4,9}
  \begin{phonetics}{互相}{hu4xiang1}
    \definition{adv.}{mutuamente | um ao outro}
  \end{phonetics}
\end{entry}

\begin{entry}{五}{4}[Radical 二]
  \begin{phonetics}{五}{wu3}[][HSK 1]
    \definition{num.}{cinco; 5}
  \end{phonetics}
\end{entry}

\begin{entry}{五五}{4,4}
  \begin{phonetics}{五五}{wu3wu3}
    \definition{num.}{50-50}
    \definition{s.}{igual (partilha, parceria, etc.)}
  \end{phonetics}
\end{entry}

\begin{entry}{五体投地}{4,7,7,6}
  \begin{phonetics}{五体投地}{wu3ti3tou2di4}
    \definition{expr.}{prostrar-se em admiração | adular alguém}
  \end{phonetics}
\end{entry}

\begin{entry}{井}{4}[Radical 二][Kangxi 7]
  \begin{phonetics}{井}{jing3}
    \definition{adj.}{puro | ordenado}
    \definition[口]{s.}{poço}
  \end{phonetics}
\end{entry}

\begin{entry}{什么}{4,3}
  \begin{phonetics}{什么}{shen2me5}[][HSK 1]
    \definition{pron.}{que? | o que?}
    \definition{pron.}{algo | qualquer coisa}
  \end{phonetics}
\end{entry}

\begin{entry}{什么时候}{4,3,7,10}
  \begin{phonetics}{什么时候}{shen2me5shi2hou5}
    \definition{adv.}{quando? | a que horas?}
  \end{phonetics}
\end{entry}

\begin{entry}{什么样}{4,3,10}
  \begin{phonetics}{什么样}{shen2 me5 yang4}[][HSK 2]
    \definition{pron.}{que tipo? | o quê? | que tipo?}
  \end{phonetics}
\end{entry}

\begin{entry}{仅}{4}[Radical 人]
  \begin{phonetics}{仅}{jin3}
    \definition{adv.}{apenas | meramente}
  \end{phonetics}
\end{entry}

\begin{entry}{仅仅}{4,4}
  \begin{phonetics}{仅仅}{jin3jin3}
    \definition{adv.}{meramente | somente | apenas}
  \end{phonetics}
\end{entry}

\begin{entry}{今天}{4,4}
  \begin{phonetics}{今天}{jin1tian1}[][HSK 1]
    \definition{adv.}{hoje | no presente | agora}
  \end{phonetics}
\end{entry}

\begin{entry}{今后}{4,6}
  \begin{phonetics}{今后}{jin1 hou4}[][HSK 2]
    \definition{s.}{de agora em diante | daqui em diante | no futuro}
  \end{phonetics}
\end{entry}

\begin{entry}{今年}{4,6}
  \begin{phonetics}{今年}{jin1nian2}[][HSK 1]
    \definition{adv.}{este ano}
  \end{phonetics}
\end{entry}

\begin{entry}{介绍}{4,8}
  \begin{phonetics}{介绍}{jie4shao4}[][HSK 1]
    \definition{s.}{introdução | apresentação}
    \definition{v.}{fazer uma apresentação | apresentar (alguém para alguém) | apresentar (alguém para um emprego, etc.)}
  \end{phonetics}
\end{entry}

\begin{entry}{仍然}{4,12}
  \begin{phonetics}{仍然}{reng2ran2}
    \definition{adv.}{ainda}
  \end{phonetics}
\end{entry}

\begin{entry}{从}{4}[Radical ⼈]
  \begin{phonetics}{从}{cong2}[][HSK 1]
    \definition*{s.}{sobrenome Cong}
    \definition{prep.}{de | desde | a partir de}
  \end{phonetics}
\end{entry}

\begin{entry}{从小}{4,3}
  \begin{phonetics}{从小}{cong2 xiao3}[][HSK 2]
    \definition{adv.}{desde a infância | desde muito jovem | quando criança}
  \end{phonetics}
\end{entry}

\begin{entry}{从不}{4,4}
  \begin{phonetics}{从不}{cong2bu4}
    \definition{adv.}{nunca}
  \end{phonetics}
\end{entry}

\begin{entry}{从未}{4,5}
  \begin{phonetics}{从未}{cong2wei4}
    \definition{adv.}{nunca}
  \end{phonetics}
\end{entry}

\begin{entry}{从而}{4,6}
  \begin{phonetics}{从而}{cong2'er2}
    \definition{conj.}{assim | desse modo}
  \end{phonetics}
\end{entry}

\begin{entry}{从来}{4,7}
  \begin{phonetics}{从来}{cong2lai2}
    \definition{adv.}{do passado até o presente | o tempo todo | sempre | nunca (se usado em uma sentença negativa)}
  \end{phonetics}
\end{entry}

\begin{entry}{以上}{4,3}
  \begin{phonetics}{以上}{yi3 shang4}[][HSK 2]
    \definition{s.}{mais que | sobre | acima | o acima | o precedente | o acima mencionado}
  \end{phonetics}
\end{entry}

\begin{entry}{以下}{4,3}
  \begin{phonetics}{以下}{yi3 xia4}[][HSK 2]
    \definition[所]{s.}{abaixo | sob | seguinte}
  \end{phonetics}
\end{entry}

\begin{entry}{以及}{4,3}
  \begin{phonetics}{以及}{yi3ji2}
    \definition{conj.}{assim como | juntamente como}
  \end{phonetics}
\end{entry}

\begin{entry}{以为}{4,4}
  \begin{phonetics}{以为}{yi3wei2}[][HSK 2]
    \definition{v.}{pensar, ou seja, considerar que\dots (geralmente há uma implicação de que a noção está errada --- exceto ao expressar a própria opnião atual)}
  \end{phonetics}
\end{entry}

\begin{entry}{以外}{4,5}
  \begin{phonetics}{以外}{yi3 wai4}[][HSK 2]
    \definition{s.}{além | exceto | fora | diferente de}
  \end{phonetics}
\end{entry}

\begin{entry}{以后}{4,6}
  \begin{phonetics}{以后}{yi3 hou4}[][HSK 2]
    \definition{adv.}{depois de | depois | após}
  \end{phonetics}
\end{entry}

\begin{entry}{以此}{4,6}
  \begin{phonetics}{以此}{yi3ci3}
    \definition{adv.}{devido a esta | deste modo | por isso | com isso}
  \end{phonetics}
\end{entry}

\begin{entry}{以至}{4,6}
  \begin{phonetics}{以至}{yi3zhi4}
    \definition{adv.}{até}
    \definition{conj.}{a tal ponto que\dots}
  \seealsoref{以至于}{yi3zhi4yu2}
  \end{phonetics}
\end{entry}

\begin{entry}{以至于}{4,6,3}
  \begin{phonetics}{以至于}{yi3zhi4yu2}
    \definition{adv.}{até}
    \definition{conj.}{na medida em que\dots}
  \seealsoref{以至}{yi3zhi4}
  \end{phonetics}
\end{entry}

\begin{entry}{以色列}{4,6,6}
  \begin{phonetics}{以色列}{yi3se4lie4}
    \definition*{s.}{Israel}
  \end{phonetics}
\end{entry}

\begin{entry}{以免}{4,7}
  \begin{phonetics}{以免}{yi3mian3}
    \definition{conj.}{para evitar isso}
  \end{phonetics}
\end{entry}

\begin{entry}{以来}{4,7}
  \begin{phonetics}{以来}{yi3lai2}
    \definition{prep.}{desde (um evento anterior)}
  \end{phonetics}
\end{entry}

\begin{entry}{以求}{4,7}
  \begin{phonetics}{以求}{yi3qiu2}
    \definition{conj.}{a fim de}
  \end{phonetics}
\end{entry}

\begin{entry}{以便}{4,9}
  \begin{phonetics}{以便}{yi3bian4}
    \definition{conj.}{a fim de | para que | assim como}
  \end{phonetics}
\end{entry}

\begin{entry}{以前}{4,9}
  \begin{phonetics}{以前}{yi3qian2}[][HSK 2]
    \definition{adv.}{antes de | antes}
  \end{phonetics}
\end{entry}

\begin{entry}{以期}{4,12}
  \begin{phonetics}{以期}{yi3qi1}
    \definition{v.}{tentando | esperando | esperando por}
  \end{phonetics}
\end{entry}

\begin{entry}{元}{4}[Radical 儿]
  \begin{phonetics}{元}{yuan2}[][HSK 1]
    \definition*{s.}{sobrenome Yuan | Dinastia Yuan (1279-1368)}
    \definition{clas.}{unidade monetária da China}
  \end{phonetics}
\end{entry}

\begin{entry}{元气}{4,4}
  \begin{phonetics}{元气}{yuan2qi4}
    \definition{s.}{força | vigor | vitalidade | energial vital}
  \end{phonetics}
\end{entry}

\begin{entry}{元旦}{4,5}
  \begin{phonetics}{元旦}{yuan2dan4}
    \definition*{s.}{Dia de Ano Novo (1 de janeiro)}
  \end{phonetics}
\end{entry}

\begin{entry}{元来}{4,7}
  \begin{phonetics}{元来}{yuan2lai2}
    \variantof{原来}
  \end{phonetics}
\end{entry}

\begin{entry}{元夜}{4,8}
  \begin{phonetics}{元夜}{yuan2ye4}
    \definition*{s.}{Festival das Lanternas}
  \seealsoref{元宵}{yuan2xiao1}
  \seealsoref{元宵节}{yuan2xiao1jie2}
  \end{phonetics}
\end{entry}

\begin{entry}{元宵}{4,10}
  \begin{phonetics}{元宵}{yuan2xiao1}
    \definition*{s.}{Festival das Lanternas}
  \seealsoref{元宵节}{yuan2xiao1jie2}
  \seealsoref{元夜}{yuan2ye4}
  \end{phonetics}
\end{entry}

\begin{entry}{元宵节}{4,10,5}
  \begin{phonetics}{元宵节}{yuan2xiao1jie2}
    \definition*{s.}{Festival das Lanternas (15º~dia do primeiro mês lunar)}
  \seealsoref{元宵}{yuan2xiao1}
  \seealsoref{元夜}{yuan2ye4}
  \end{phonetics}
\end{entry}

\begin{entry}{公元}{4,4}
  \begin{phonetics}{公元}{gong1yuan2}
    \definition{s.}{D.C. (Depois de~Cristo)}
  \seealsoref{前}{qian2}
    \example{公元293年}[293 d.C.]
  \end{phonetics}
\end{entry}

\begin{entry}{公开}{4,4}
  \begin{phonetics}{公开}{gong1kai1}
    \definition{s.}{aberto | público}
    \definition{v.}{tornar público | liberar}
  \end{phonetics}
\end{entry}

\begin{entry}{公斤}{4,4}
  \begin{phonetics}{公斤}{gong1jin1}[][HSK 2]
    \definition{clas.}{quilograma (kg)}
  \end{phonetics}
\end{entry}

\begin{entry}{公车}{4,4}
  \begin{phonetics}{公车}{gong1che1}
    \definition{s.}{abreviação de~公共汽车, ônibus}
    \seeref{公共汽车}{gong1gong4qi4che1}
  \end{phonetics}
\end{entry}

\begin{entry}{公司}{4,5}
  \begin{phonetics}{公司}{gong1si1}[][HSK 2]
    \definition[家]{s.}{empresa | companhia | corporação | firma}
  \end{phonetics}
\end{entry}

\begin{entry}{公司治理}{4,5,8,11}
  \begin{phonetics}{公司治理}{gong1si1zhi4li3}
    \definition{s.}{governança corporativa}
  \end{phonetics}
\end{entry}

\begin{entry}{公平}{4,5}
  \begin{phonetics}{公平}{gong1ping2}[][HSK 2]
    \definition{adj.}{justo | imparcial | equitativo}
  \end{phonetics}
\end{entry}

\begin{entry}{公用电话}{4,5,5,8}
  \begin{phonetics}{公用电话}{gong1yong4dian4hua4}
    \definition[部]{s.}{telefone público}
  \end{phonetics}
\end{entry}

\begin{entry}{公交车}{4,6,4}
  \begin{phonetics}{公交车}{gong1 jiao1 che1}[][HSK 2]
    \definition[辆]{s.}{ônibus urbano | veículo de transporte público}
  \end{phonetics}
\end{entry}

\begin{entry}{公共汽车}{4,6,7,4}
  \begin{phonetics}{公共汽车}{gong1gong4qi4che1}[][HSK 2]
    \definition[辆,班]{s.}{ônibus}
    \seeref{公车}{gong1che1}
  \end{phonetics}
\end{entry}

\begin{entry}{公克}{4,7}
  \begin{phonetics}{公克}{gong1ke4}
    \definition{s.}{grama (medida de peso)}
  \end{phonetics}
\end{entry}

\begin{entry}{公园}{4,7}
  \begin{phonetics}{公园}{gong1yuan2}[][HSK 2]
    \definition[座]{s.}{parque (para recreação pública)}
  \end{phonetics}
\end{entry}

\begin{entry}{公里}{4,7}
  \begin{phonetics}{公里}{gong1li3}[][HSK 2]
    \definition{s.}{quilômetro}
  \end{phonetics}
\end{entry}

\begin{entry}{公寓}{4,12}
  \begin{phonetics}{公寓}{gong1yu4}
    \definition[套]{s.}{prédio de apartamentos | pensão}
  \end{phonetics}
\end{entry}

\begin{entry}{公路}{4,13}
  \begin{phonetics}{公路}{gong1 lu4}[][HSK 2]
    \definition[条]{s.}{rodovia | via de trânsito | estrada | auto-estrada}
  \end{phonetics}
\end{entry}

\begin{entry}{六}{4}[Radical 八]
  \begin{phonetics}{六}{liu4}[][HSK 1]
    \definition{num.}{seis; 6}
  \end{phonetics}
\end{entry}

\begin{entry}{内存}{4,6}
  \begin{phonetics}{内存}{nei4cun2}
    \definition{s.}{armazenamento interno | memória do computador | RAM (\emph{random access memory})}
  \seealsoref{随机存取存储器}{sui2ji1cun2qu3cun2chu3qi4}
  \seealsoref{随机存取记忆体}{sui2ji1cun2qu3ji4yi4ti3}
  \end{phonetics}
\end{entry}

\begin{entry}{内省}{4,9}
  \begin{phonetics}{内省}{nei4xing3}
    \definition{s.}{introspecção}
    \definition{v.}{refletir sobre si mesmo}
  \end{phonetics}
\end{entry}

\begin{entry}{内燃机}{4,16,6}
  \begin{phonetics}{内燃机}{nei4ran2ji1}
    \definition{s.}{motor de combustão interna}
  \end{phonetics}
\end{entry}

\begin{entry}{凤凰}{4,11}
  \begin{phonetics}{凤凰}{feng4huang2}
    \definition{s.}{fênix}
  \end{phonetics}
\end{entry}

\begin{entry}{分}{4}[Radical 刀]
  \begin{phonetics}{分}{fen1}[][HSK 1]
    \definition{s.}{parte ou subdivisão | fração | um décimo (de certas unidades) | unidade de comprimento equivalente a 0,33cm | minuto (unidade de tempo) | minuto (unidade de medida angular) | um ponto (em esportes e jogos) | 0,01 yuan (unidade de dinheiro)}
    \definition{v.}{dividir | separar | distribuir | atribuir | distinguir (bom e mau)}
  \end{phonetics}
  \begin{phonetics}{分}{fen4}[][HSK 2]
    \definition{s.}{parte | ingrediente | componente}
  \end{phonetics}
\end{entry}

\begin{entry}{分子}{4,3}
  \begin{phonetics}{分子}{fen1zi3}
    \definition{s.}{molécula | (matemática) numerador de uma fração}
  \end{phonetics}
  \begin{phonetics}{分子}{fen4zi3}
    \definition{s.}{membros de uma classe ou grupo | elementos políticos (como intelectuais ou extremistas)}
  \end{phonetics}
\end{entry}

\begin{entry}{分公司}{4,4,5}
  \begin{phonetics}{分公司}{fen1gong1si1}
    \definition{s.}{sucursal | filial de companhia}
  \end{phonetics}
\end{entry}

\begin{entry}{分开}{4,4}
  \begin{phonetics}{分开}{fen1 kai1}[][HSK 2]
    \definition{v.+compl.}{separar | dividir | desacoplar | desempacotar | quebrar | desmembrar | romper | desfazer | desvincular | distribuir | separar de (em) | dividir ... de ... | separar de}
  \end{phonetics}
\end{entry}

\begin{entry}{分手}{4,4}
  \begin{phonetics}{分手}{fen1shou3}
    \definition{v.+compl.}{separar | separar-se do companheiro | dizer adeus}
  \end{phonetics}
\end{entry}

\begin{entry}{分钟}{4,9}
  \begin{phonetics}{分钟}{fen1zhong1}[][HSK 2]
    \definition{s.}{minuto (usado em intervalos de tempo)}
  \end{phonetics}
\end{entry}

\begin{entry}{分量}{4,12}
  \begin{phonetics}{分量}{fen1liang4}
    \definition{s.}{componente vetorial}
  \end{phonetics}
  \begin{phonetics}{分量}{fen4liang4}
    \definition{s.}{tamanho da porção (comida)}
  \end{phonetics}
  \begin{phonetics}{分量}{fen4liang5}
    \definition{s.}{quantidade | peso | medida}
  \end{phonetics}
\end{entry}

\begin{entry}{分数}{4,13}
  \begin{phonetics}{分数}{fen1 shu4}[][HSK 2]
    \definition{s.}{fração | número fracionário | marca | nota | ponto}
  \end{phonetics}
\end{entry}

\begin{entry}{切割}{4,12}
  \begin{phonetics}{切割}{qie1ge1}
    \definition{v.}{cortar}
  \end{phonetics}
\end{entry}

\begin{entry}{办}{4}[Radical 力]
  \begin{phonetics}{办}{ban4}[][HSK 2]
    \definition{v.}{lidar com | lidar | gerenciar | configurar}
  \end{phonetics}
\end{entry}

\begin{entry}{办公}{4,4}
  \begin{phonetics}{办公}{ban4gong1}
    \definition{v.+compl.}{lidar com negócios oficiais | trabalhar (especialmente em um escritório)}
  \end{phonetics}
\end{entry}

\begin{entry}{办公室}{4,4,9}
  \begin{phonetics}{办公室}{ban4gong1shi4}[][HSK 2]
    \definition[间]{s.}{gabinete | escritório}
  \end{phonetics}
\end{entry}

\begin{entry}{办法}{4,8}
  \begin{phonetics}{办法}{ban4fa3}[][HSK 2]
    \definition[条,个]{s.}{meio (de se fazer alguma coisa) | método | medida}
  \end{phonetics}
\end{entry}

\begin{entry}{勾}{4}[Radical ⼓]
  \begin{phonetics}{勾}{gou1}
    \definition*{s.}{sobrenome Gou}
    \definition{v.}{atrair | excitar | marcar | atacar | delinear | conspirar}
    \variantof{钩}
  \end{phonetics}
  \begin{phonetics}{勾}{gou4}
    \definition{s.}{usado em 勾当}
    \seeref{勾当}{gou4dang4}
  \end{phonetics}
\end{entry}

\begin{entry}{勾当}{4,6}
  \begin{phonetics}{勾当}{gou4dang4}
    \definition{s.}{negócio obscuro}
  \end{phonetics}
\end{entry}

\begin{entry}{化}{4}[Radical 匕]
  \begin{phonetics}{化}{hua1}
    \variantof{花}
  \end{phonetics}
\end{entry}

\begin{entry}{化学}{4,8}
  \begin{phonetics}{化学}{hua4xue2}
    \definition{s.}{química (disciplina)}
  \end{phonetics}
\end{entry}

\begin{entry}{区}{4}[Radical 匸]
  \begin{phonetics}{区}{ou1}
    \definition*{s.}{sobrenome Ou}
  \end{phonetics}
  \begin{phonetics}{区}{qu1}
    \definition[个]{s.}{área | região | distrito}
  \end{phonetics}
\end{entry}

\begin{entry}{区域}{4,11}
  \begin{phonetics}{区域}{qu1yu4}
    \definition{s.}{área | região | distrito}
  \end{phonetics}
\end{entry}

\begin{entry}{升起}{4,10}
  \begin{phonetics}{升起}{sheng1qi3}
    \definition{v.}{levantar | içar | subir}
  \end{phonetics}
\end{entry}

\begin{entry}{午}{4}[Radical 十]
  \begin{phonetics}{午}{wu3}
    \definition{s.}{período entre 11h00 e 13h00, meio-dia}
  \end{phonetics}
\end{entry}

\begin{entry}{午休}{4,6}
  \begin{phonetics}{午休}{wu3xiu1}
    \definition{s.}{pausa para almoço | cochilo na hora do almoço | intervalo do meio-dia}
  \end{phonetics}
\end{entry}

\begin{entry}{午后}{4,6}
  \begin{phonetics}{午后}{wu3hou4}
    \definition{s.}{tarde | período da tarde}
  \end{phonetics}
\end{entry}

\begin{entry}{午饭}{4,7}
  \begin{phonetics}{午饭}{wu3fan4}[][HSK 1]
    \definition[份,顿,次,餐]{s.}{almoço}
  \seealsoref{午餐}{wu3can1}
  \end{phonetics}
\end{entry}

\begin{entry}{午夜}{4,8}
  \begin{phonetics}{午夜}{wu3ye4}
    \definition{s.}{meia-noite}
  \end{phonetics}
\end{entry}

\begin{entry}{午前}{4,9}
  \begin{phonetics}{午前}{wu3qian2}
    \definition{s.}{\emph{A.M.} | manhã | período da manhã}
  \end{phonetics}
\end{entry}

\begin{entry}{午宴}{4,10}
  \begin{phonetics}{午宴}{wu3yan4}
    \definition{s.}{banquete de almoço}
  \end{phonetics}
\end{entry}

\begin{entry}{午睡}{4,13}
  \begin{phonetics}{午睡}{wu3 shui4}[][HSK 2]
    \definition{s.}{siesta}
    \definition{v.}{tirar uma soneca}
  \end{phonetics}
\end{entry}

\begin{entry}{午餐}{4,16}
  \begin{phonetics}{午餐}{wu3 can1}[][HSK 2]
    \definition[份,顿,次]{s.}{almoço}
  \seealsoref{午饭}{wu3fan4}
  \end{phonetics}
\end{entry}

\begin{entry}{历史}{4,5}
  \begin{phonetics}{历史}{li4shi3}
    \definition[门,段]{s.}{história}
  \end{phonetics}
\end{entry}

\begin{entry}{友好}{4,6}
  \begin{phonetics}{友好}{you3hao3}[][HSK 2]
    \definition{adj.}{amigável}
    \definition{s.}{amigo próximo, íntimo}
  \end{phonetics}
\end{entry}

\begin{entry}{双}{4}[Radical 又]
  \begin{phonetics}{双}{shuang1}
    \definition*{s.}{sobrenome Shuang}
    \definition{s.}{dobro | par | dupla | ambos | número par}
  \end{phonetics}
\end{entry}

\begin{entry}{双方同意}{4,4,6,13}
  \begin{phonetics}{双方同意}{shuang1fang1tong2yi4}
    \definition{s.}{acordo bilateral}
  \end{phonetics}
\end{entry}

\begin{entry}{双打}{4,5}
  \begin{phonetics}{双打}{shuang1da3}
    \definition[场]{s.}{duplas (em esportes)}
  \end{phonetics}
\end{entry}

\begin{entry}{双层床}{4,7,7}
  \begin{phonetics}{双层床}{shuang1ceng2chuang2}
    \definition{s.}{beliche}
  \end{phonetics}
\end{entry}

\begin{entry}{反对}{4,5}
  \begin{phonetics}{反对}{fan3dui4}
    \definition{v.}{contrariar | opor-se | lutar contra}
  \end{phonetics}
\end{entry}

\begin{entry}{反对派}{4,5,9}
  \begin{phonetics}{反对派}{fan3dui4pai4}
    \definition{s.}{facção de oposição}
  \end{phonetics}
\end{entry}

\begin{entry}{反对党}{4,5,10}
  \begin{phonetics}{反对党}{fan3dui4dang3}
    \definition{s.}{partido de oposição}
  \end{phonetics}
\end{entry}

\begin{entry}{反对票}{4,5,11}
  \begin{phonetics}{反对票}{fan3dui4piao4}
    \definition{s.}{voto dissidente}
  \end{phonetics}
\end{entry}

\begin{entry}{反正}{4,5}
  \begin{phonetics}{反正}{fan3zheng4}
    \definition{adv.}{de qualquer maneira | em qualquer caso | aconteça o que acontecer}
  \end{phonetics}
\end{entry}

\begin{entry}{反应}{4,7}
  \begin{phonetics}{反应}{fan3ying4}
    \definition[个]{s.}{reação | resposta | reação química}
    \definition{v.}{reagir | responder}
  \end{phonetics}
\end{entry}

\begin{entry}{反复}{4,9}
  \begin{phonetics}{反复}{fan3fu4}
    \definition{adv.}{de novo e de novo | repetidamente}
  \end{phonetics}
\end{entry}

\begin{entry}{反省}{4,9}
  \begin{phonetics}{反省}{fan3xing3}
    \definition{v.}{examinar a consciência | questionar-se | refletir sobre si mesmo | sondar a alma}
  \end{phonetics}
\end{entry}

\begin{entry}{天}{4}[Radical 大]
  \begin{phonetics}{天}{tian1}[][HSK 1]
    \definition{s.}{dia | céu | paraíso}
  \end{phonetics}
\end{entry}

\begin{entry}{天上}{4,3}
  \begin{phonetics}{天上}{tian1 shang4}[][HSK 2]
    \definition{s.}{o céu | paraíso}
  \end{phonetics}
\end{entry}

\begin{entry}{天下}{4,3}
  \begin{phonetics}{天下}{tian1xia4}
    \definition{s.}{terra sob o céu | o mundo todo | toda a China | reino}
  \end{phonetics}
\end{entry}

\begin{entry}{天才}{4,3}
  \begin{phonetics}{天才}{tian1cai2}
    \definition{adj.}{talentoso | superdotado | genial}
    \definition{s.}{talento | dom | gênio}
  \end{phonetics}
\end{entry}

\begin{entry}{天公}{4,4}
  \begin{phonetics}{天公}{tian1gong1}
    \definition{s.}{céu, paraíso | senhor do céu}
  \end{phonetics}
\end{entry}

\begin{entry}{天天}{4,4}
  \begin{phonetics}{天天}{tian1tian1}
    \definition{adv.}{todo dia}
  \end{phonetics}
\end{entry}

\begin{entry}{天气}{4,4}
  \begin{phonetics}{天气}{tian1qi4}[][HSK 1]
    \definition{s.}{clima, tempo}
  \end{phonetics}
\end{entry}

\begin{entry}{天花板}{4,7,8}
  \begin{phonetics}{天花板}{tian1hua1ban3}
    \definition{s.}{teto}
  \end{phonetics}
\end{entry}

\begin{entry}{天使}{4,8}
  \begin{phonetics}{天使}{tian1shi3}
    \definition{s.}{anjo}
  \end{phonetics}
\end{entry}

\begin{entry}{天择}{4,8}
  \begin{phonetics}{天择}{tian1ze2}
    \definition{s.}{seleção natural}
  \end{phonetics}
\end{entry}

\begin{entry}{天柱}{4,9}
  \begin{phonetics}{天柱}{tian1zhu4}
    \definition{s.}{pilar celestial, que sustenta o céu}
  \end{phonetics}
\end{entry}

\begin{entry}{天堂}{4,11}
  \begin{phonetics}{天堂}{tian1tang2}
    \definition{s.}{paraíso, céu}
  \end{phonetics}
\end{entry}

\begin{entry}{天然}{4,12}
  \begin{phonetics}{天然}{tian1ran2}
    \definition{adj.}{natural}
  \end{phonetics}
\end{entry}

\begin{entry}{天鹅}{4,12}
  \begin{phonetics}{天鹅}{tian1'e2}
    \definition{s.}{cisne}
  \end{phonetics}
\end{entry}

\begin{entry}{太}{4}[Radical 大]
  \begin{phonetics}{太}{tai4}[][HSK 1]
    \definition{adv.}{excessivamente | demais | muito}
  \end{phonetics}
\end{entry}

\begin{entry}{太太}{4,4}
  \begin{phonetics}{太太}{tai4tai5}[][HSK 2]
    \definition[个,位]{s.}{esposa | madame| mulher casada}
  \end{phonetics}
\end{entry}

\begin{entry}{太平洋}{4,5,9}
  \begin{phonetics}{太平洋}{tai4ping2 yang2}
    \definition*{s.}{Oceano Pacífico}
  \end{phonetics}
\end{entry}

\begin{entry}{太阳}{4,6}
  \begin{phonetics}{太阳}{tai4yang5}[][HSK 2]
    \definition[个]{s.}{sol | abreviação de 太阳穴}
    \seeref{太阳穴}{tai4yang2xue2}
  \end{phonetics}
\end{entry}

\begin{entry}{太阳日}{4,6,4}
  \begin{phonetics}{太阳日}{tai4yang2ri4}
    \definition{s.}{dia solar}
  \end{phonetics}
\end{entry}

\begin{entry}{太阳风}{4,6,4}
  \begin{phonetics}{太阳风}{tai4yang2feng1}
    \definition{s.}{vento solar}
  \end{phonetics}
\end{entry}

\begin{entry}{太阳穴}{4,6,5}
  \begin{phonetics}{太阳穴}{tai4yang2xue2}
    \definition{s.}{têmpora (nas laterais da cabeça humana)}
  \end{phonetics}
\end{entry}

\begin{entry}{太阳灯}{4,6,6}
  \begin{phonetics}{太阳灯}{tai4yang2deng1}
    \definition{s.}{lâmpada solar (com células fotovoltaicas)}
  \end{phonetics}
\end{entry}

\begin{entry}{太阳雨}{4,6,8}
  \begin{phonetics}{太阳雨}{tai4yang2yu3}
    \definition{s.}{banho de sol}
  \end{phonetics}
\end{entry}

\begin{entry}{太阳窗}{4,6,12}
  \begin{phonetics}{太阳窗}{tai4yang2chuang1}
    \definition{s.}{teto solar (de veículos)}
  \end{phonetics}
\end{entry}

\begin{entry}{太阳镜}{4,6,16}
  \begin{phonetics}{太阳镜}{tai4yang2jing4}
    \definition{s.}{óculos de sol}
  \end{phonetics}
\end{entry}

\begin{entry}{太阳翼}{4,6,17}
  \begin{phonetics}{太阳翼}{tai4yang2yi4}
    \definition{s.}{painel solar}
  \end{phonetics}
\end{entry}

\begin{entry}{太极拳}{4,7,10}
  \begin{phonetics}{太极拳}{tai4ji2quan2}
    \definition*{s.}{Tai Chi Chuan, Taiji, T'aichi ou T'aichichuan; forma tradicional de exercício físico ou relaxamento}
  \end{phonetics}
\end{entry}

\begin{entry}{太空}{4,8}
  \begin{phonetics}{太空}{tai4kong1}
    \definition{s.}{espaço sideral | espaço exterior}
  \end{phonetics}
\end{entry}

\begin{entry}{夫妻}{4,8}
  \begin{phonetics}{夫妻}{fu1qi1}
    \definition{s.}{casal | marido e eposa}
  \end{phonetics}
\end{entry}

\begin{entry}{孔}{4}[Radical 子]
  \begin{phonetics}{孔}{kong3}
    \definition*{s.}{sobrenome Kong}
    \definition{clas.}{para habitações em cavernas}
    \definition[个]{s.}{buraco}
  \end{phonetics}
\end{entry}

\begin{entry}{孔子}{4,3}
  \begin{phonetics}{孔子}{kong3zi3}
    \definition*{s.}{Confúcio (551-479 aC), pensador e filósofo social chinês}
  \seealsoref{孔夫子}{kong3fu1zi3}
  \end{phonetics}
\end{entry}

\begin{entry}{孔子学院}{4,3,8,9}
  \begin{phonetics}{孔子学院}{kong3zi3 xue2yuan4}
    \definition*{s.}{Instituto Confúcio, organização estabelecida internacionalmente pela República Popular da China, que promove a língua e a cultura chinesas}
  \end{phonetics}
\end{entry}

\begin{entry}{孔夫子}{4,4,3}
  \begin{phonetics}{孔夫子}{kong3fu1zi3}
    \definition*{s.}{Confúcio (551-479 aC), pensador e filósofo social chinês}
  \seealsoref{孔子}{kong3zi3}
  \end{phonetics}
\end{entry}

\begin{entry}{孔雀}{4,11}
  \begin{phonetics}{孔雀}{kong3que4}
    \definition{s.}{pavão}
  \end{phonetics}
\end{entry}

\begin{entry}{少}{4}[Radical 小]
  \begin{phonetics}{少}{shao3}
    \definition{adj.}{pouco, poucos}
    \definition{v.}{sentir falta | faltar | parar (de fazer algo)}
  \end{phonetics}
  \begin{phonetics}{少}{shao4}
    \definition{s.}{jovem}
  \end{phonetics}
\end{entry}

\begin{entry}{少年}{4,6}
  \begin{phonetics}{少年}{shao4 nian2}[][HSK 2]
    \definition[个]{s.}{adolescente | juventude precoce | menor | juventude | adolescente}
  \end{phonetics}
\end{entry}

\begin{entry}{少数}{4,13}
  \begin{phonetics}{少数}{shao3 shu4}[][HSK 2]
    \definition{s.}{pequeno número | poucos | minoria}
  \end{phonetics}
\end{entry}

\begin{entry}{尤其}{4,8}
  \begin{phonetics}{尤其}{you2qi2}
    \definition{adv.}{especialmente | particularmente}
  \end{phonetics}
\end{entry}

\begin{entry}{巴西}{4,6}
  \begin{phonetics}{巴西}{ba1xi1}
    \definition*{s.}{Brasil}
  \end{phonetics}
\end{entry}

\begin{entry}{巴西人}{4,6,2}
  \begin{phonetics}{巴西人}{ba1xi1ren2}
    \definition[个,位]{s.}{brasileiro | pessoa ou povo do Brasil}
    \example{他是巴西人。}[Ele é brasileiro.]
  \end{phonetics}
\end{entry}

\begin{entry}{巴西战舞}{4,6,9,14}
  \begin{phonetics}{巴西战舞}{ba1xi1zhan4wu3}
    \definition{s.}{capoeira}
  \end{phonetics}
\end{entry}

\begin{entry}{巴勒斯坦}{4,11,12,8}
  \begin{phonetics}{巴勒斯坦}{ba1le4si1tan3}
    \definition*{s.}{Palestina}
  \end{phonetics}
\end{entry}

\begin{entry}{幻觉}{4,9}
  \begin{phonetics}{幻觉}{huan4jue2}
    \definition{s.}{ilusão | alucinação}
  \end{phonetics}
\end{entry}

\begin{entry}{开}{4}[Radical 廾]
  \begin{phonetics}{开}{kai1}[][HSK 1]
    \definition{clas.}{quilate (ouro)}
    \definition{v.}{abrir | ligar | dirigir | iniciar (alguma coisa) | começar | ferver | escrever  (uma receita, cheque, fatura, etc.) | operar (um veículo) | abreviação de Kelvin 开尔文}
    \seeref{开尔文}{kai1'er3wen2}
  \end{phonetics}
\end{entry}

\begin{entry}{开口}{4,3}
  \begin{phonetics}{开口}{kai1kou3}
    \definition{v.}{abrir a boca de alguém | começar a falar}
  \end{phonetics}
\end{entry}

\begin{entry}{开心}{4,4}
  \begin{phonetics}{开心}{kai1xin1}[][HSK 2]
    \definition{v.}{sentir-se feliz | regozijar-se | divertir-se | tirar sarro de alguém}
  \end{phonetics}
\end{entry}

\begin{entry}{开车}{4,4}
  \begin{phonetics}{开车}{kai1 che1}[][HSK 1]
    \definition{v.+compl.}{conduzir | dirigir}
  \end{phonetics}
\end{entry}

\begin{entry}{开发区}{4,5,4}
  \begin{phonetics}{开发区}{kai1fa1qu1}
    \definition{s.}{zona de desenvolvimento}
  \end{phonetics}
\end{entry}

\begin{entry}{开头}{4,5}
  \begin{phonetics}{开头}{kai1tou2}
    \definition{s.}{início | começo}
    \definition{v.+compl.}{iniciar | começar | fazer um começo}
  \end{phonetics}
\end{entry}

\begin{entry}{开尔文}{4,5,4}
  \begin{phonetics}{开尔文}{kai1'er3wen2}
    \definition{s.}{Kelvin, temperatura absoluta | K, escala de temperatura}
  \end{phonetics}
\end{entry}

\begin{entry}{开会}{4,6}
  \begin{phonetics}{开会}{kai1 hui4}[][HSK 1]
    \definition{v.+compl.}{realizar uma reunião | ter uma reunião | participar de uma reunião (conferência)}
  \end{phonetics}
\end{entry}

\begin{entry}{开机}{4,6}
  \begin{phonetics}{开机}{kai1 ji1}[][HSK 2]
    \definition{v.}{começar a filmar um filme ou programa de TV | iniciar uma máquina}
  \end{phonetics}
\end{entry}

\begin{entry}{开启}{4,7}
  \begin{phonetics}{开启}{kai1qi3}
    \definition{v.}{abrir | iniciar | (computação) ativar}
  \end{phonetics}
\end{entry}

\begin{entry}{开花}{4,7}
  \begin{phonetics}{开花}{kai1hua1}
    \definition{v.}{florescer | (fig.) explodir, abrir-se | (fig.) explodir de alegria | (fig.) começar a existir de repente em todos os lugares}
  \end{phonetics}
\end{entry}

\begin{entry}{开夜车}{4,8,4}
  \begin{phonetics}{开夜车}{kai1ye4che1}
    \definition{expr.}{trabalho noturno | (literalmente) ``conduzir carro à noite''}
  \end{phonetics}
\end{entry}

\begin{entry}{开始}{4,8}
  \begin{phonetics}{开始}{kai1shi3}
    \definition{adv.}{inicial}
    \definition[个]{s.}{começo | início}
    \definition{v.}{começar | iniciar}
  \end{phonetics}
\end{entry}

\begin{entry}{开学}{4,8}
  \begin{phonetics}{开学}{kai1 xue2}[][HSK 2]
    \definition{v.}{iniciar as aulas | iniciar o semestre | começar as aulas}
  \end{phonetics}
\end{entry}

\begin{entry}{开玩笑}{4,8,10}
  \begin{phonetics}{开玩笑}{kai1 wan2xiao4}[][HSK 1]
    \definition{v.}{contar uma piada | brincar | fazer piada de | pregar uma peça | provocar}
  \end{phonetics}
\end{entry}

\begin{entry}{开锁}{4,12}
  \begin{phonetics}{开锁}{kai1suo3}
    \definition{v.}{desbloquear | destravar}
  \end{phonetics}
\end{entry}

\begin{entry}{引擎}{4,16}
  \begin{phonetics}{引擎}{yin3qing2}
    \definition[台]{s.}{motor | (empréstimo linguístico) \emph{engine}}
  \end{phonetics}
\end{entry}

\begin{entry}{心中}{4,4}
  \begin{phonetics}{心中}{xin1zhong1}[][HSK 2]
    \definition{adv.}{nos pensamentos | no coração}
    \definition{s.}{ponto central}
  \end{phonetics}
\end{entry}

\begin{entry}{心机}{4,6}
  \begin{phonetics}{心机}{xin1ji1}
    \definition{s.}{pensamento | esquema}
  \end{phonetics}
\end{entry}

\begin{entry}{心声}{4,7}
  \begin{phonetics}{心声}{xin1sheng1}
    \definition{s.}{desejo sincero | voz interior | aspiração}
  \end{phonetics}
\end{entry}

\begin{entry}{心里}{4,7}
  \begin{phonetics}{心里}{xin1 li3}[][HSK 2]
    \definition[把]{s.}{no coração | no coração de alguém | na mente}
  \end{phonetics}
\end{entry}

\begin{entry}{心疼}{4,10}
  \begin{phonetics}{心疼}{xin1teng2}
    \definition{adj.}{angustiado}
    \definition{v.}{sentir pena de alguém | arrepender-se | ressentir-se | ficar angustiado}
  \end{phonetics}
\end{entry}

\begin{entry}{心情}{4,11}
  \begin{phonetics}{心情}{xin1qing2}[][HSK 2]
    \definition{s.}{humor | sentimento | estado de espírito}
  \end{phonetics}
\end{entry}

\begin{entry}{手}{4}[Radical 手][Kangxi 64]
  \begin{phonetics}{手}{shou3}[][HSK 1]
    \definition{adj.}{conveniente}
    \definition{clas.}{de habilidade}
    \definition[双,只]{s.}{mão | pessoa envolvida em certos tipos de trabalho | pessoa qualificada para certos tipos de trabalho}
    \definition{v.}{segurar (formal)}
  \end{phonetics}
\end{entry}

\begin{entry}{手工}{4,3}
  \begin{phonetics}{手工}{shou3gong1}
    \definition{s.}{trabalho manual | artesanato}
  \end{phonetics}
\end{entry}

\begin{entry}{手工艺人}{4,3,4,2}
  \begin{phonetics}{手工艺人}{shou3gong1 yi4ren2}
    \definition{s.}{artesão}
  \end{phonetics}
\end{entry}

\begin{entry}{手边}{4,5}
  \begin{phonetics}{手边}{shou3bian1}
    \definition{adv.}{à mão | na mão}
  \end{phonetics}
\end{entry}

\begin{entry}{手机}{4,6}
  \begin{phonetics}{手机}{shou3ji1}[][HSK 1]
    \definition[部,支]{s.}{telefone celular ou móvel}
  \end{phonetics}
\end{entry}

\begin{entry}{手刹}{4,8}
  \begin{phonetics}{手刹}{shou3sha1}
    \definition{s.}{freio de mão}
  \end{phonetics}
\end{entry}

\begin{entry}{手表}{4,8}
  \begin{phonetics}{手表}{shou3biao3}[][HSK 2]
    \definition[块,只,个]{s.}{relógio de pulso}
  \end{phonetics}
\end{entry}

\begin{entry}{手指}{4,9}
  \begin{phonetics}{手指}{shou3zhi3}
    \definition[个,只]{s.}{dedo}
  \end{phonetics}
\end{entry}

\begin{entry}{手臂}{4,17}
  \begin{phonetics}{手臂}{shou3bi4}
    \definition{s.}{braço}
  \end{phonetics}
\end{entry}

\begin{entry}{支}{4}[Radical 支][Kangxi 65]
  \begin{phonetics}{支}{zhi1}
    \definition*{s.}{sobrenome Zhi}
    \definition{clas.}{para varetas como canetas e armas | para divisões do exército e para canções ou composições}
    \definition{v.}{sacar dinheiro | erguer | criar | suportar | sustentar}
  \end{phonetics}
\end{entry}

\begin{entry}{支支吾吾}{4,4,7,7}
  \begin{phonetics}{支支吾吾}{zhi1zhi1wu2wu2}
    \definition{v.}{falhar | murmurar | paralisar | gaguejar}
  \end{phonetics}
\end{entry}

\begin{entry}{支应}{4,7}
  \begin{phonetics}{支应}{zhi1ying4}
    \definition{v.}{lidar com | fornecer}
  \end{phonetics}
\end{entry}

\begin{entry}{支承}{4,8}
  \begin{phonetics}{支承}{zhi1cheng2}
    \definition{v.}{suportar o peso de (um edifício) | suportar}
  \end{phonetics}
\end{entry}

\begin{entry}{支持}{4,9}
  \begin{phonetics}{支持}{zhi1chi2}
    \definition[个]{s.}{apoio | suporte}
    \definition{v.}{apoiar | ser a favor de | suportar}
  \end{phonetics}
\end{entry}

\begin{entry}{支根}{4,10}
  \begin{phonetics}{支根}{zhi1gen1}
    \definition{s.}{raiz ramificada | raízes de apoio | radícula}
  \end{phonetics}
\end{entry}

\begin{entry}{支票}{4,11}
  \begin{phonetics}{支票}{zhi1piao4}
    \definition[本]{s.}{cheque (banco)}
  \end{phonetics}
\end{entry}

\begin{entry}{文化}{4,4}
  \begin{phonetics}{文化}{wen2hua4}
    \definition[个,种]{s.}{cultura | civilização}
  \end{phonetics}
\end{entry}

\begin{entry}{文化水平}{4,4,4,5}
  \begin{phonetics}{文化水平}{wen2hua4 shui3ping2}
    \definition{s.}{nível educacional}
  \end{phonetics}
\end{entry}

\begin{entry}{文化史}{4,4,5}
  \begin{phonetics}{文化史}{wen2hua4shi3}
    \definition*{s.}{História Cultural}
  \end{phonetics}
\end{entry}

\begin{entry}{文化层}{4,4,7}
  \begin{phonetics}{文化层}{wen2hua4ceng2}
    \definition{s.}{nível de cultura (em sítio arqueológico)}
  \end{phonetics}
\end{entry}

\begin{entry}{文化宫}{4,4,9}
  \begin{phonetics}{文化宫}{wen2hua4gong1}
    \definition{s.}{palácio cultural}
  \end{phonetics}
\end{entry}

\begin{entry}{文化热}{4,4,10}
  \begin{phonetics}{文化热}{wen2hua4re4}
    \definition{s.}{mania cultural | febre cultural}
  \end{phonetics}
\end{entry}

\begin{entry}{文化圈}{4,4,11}
  \begin{phonetics}{文化圈}{wen2hua4quan1}
    \definition{s.}{esfera de influência cultural}
  \end{phonetics}
\end{entry}

\begin{entry}{文化障碍}{4,4,13,13}
  \begin{phonetics}{文化障碍}{wen2hua4zhang4'ai4}
    \definition{s.}{barreira cultural}
  \end{phonetics}
\end{entry}

\begin{entry}{文学系}{4,8,7}
  \begin{phonetics}{文学系}{wen2xue2 xi4}
    \definition*{s.}{Faculdade de Letras}
  \end{phonetics}
\end{entry}

\begin{entry}{文明}{4,8}
  \begin{phonetics}{文明}{wen2ming2}
    \definition{adj.}{civilizado}
    \definition[个]{s.}{civilização | cultura}
  \end{phonetics}
\end{entry}

\begin{entry}{斤}{4}[Radical 斤]
  \begin{phonetics}{斤}{jin1}[][HSK 2]
    \definition{clas.}{peso igual a 500 g}
  \end{phonetics}
\end{entry}

\begin{entry}{方向}{4,6}
  \begin{phonetics}{方向}{fang1xiang4}[][HSK 2]
    \definition[个]{s.}{direção | orientação | alvo | meta | objetivo}
  \end{phonetics}
\end{entry}

\begin{entry}{方言}{4,7}
  \begin{phonetics}{方言}{fang1yan2}
    \definition*{s.}{o primeiro dicionário de dialeto chinês, editado por Yang Xiong 扬雄 no século I, contendo mais de 9.000 caracteres}
    \definition{s.}{dialeto}
  \seealsoref{扬雄}{yang2xiong2}
  \end{phonetics}
\end{entry}

\begin{entry}{方法}{4,8}
  \begin{phonetics}{方法}{fang1fa3}[][HSK 2]
    \definition[个]{s.}{método | meio}
  \end{phonetics}
\end{entry}

\begin{entry}{方便}{4,9}
  \begin{phonetics}{方便}{fang1bian4}[][HSK 2]
    \definition{adj.}{conveniente | adequado}
    \definition{v.}{facilitar, facilitar as coisas | ter dinheiro de sobra | (eufemismo) aliviar-se}
  \end{phonetics}
\end{entry}

\begin{entry}{方便面}{4,9,9}
  \begin{phonetics}{方便面}{fang1 bian4 mian4}[][HSK 2]
    \definition{s.}{macarrão instantâneo}
  \end{phonetics}
\end{entry}

\begin{entry}{方面}{4,9}
  \begin{phonetics}{方面}{fang1mian4}[][HSK 2]
    \definition[个]{s.}{lado | campo | aspecto}
  \end{phonetics}
\end{entry}

\begin{entry}{方案}{4,10}
  \begin{phonetics}{方案}{fang1'an4}
    \definition[个,套]{s.}{plano | programa (para uma ação, etc.) | proposta | proposta de projeto de lei}
  \end{phonetics}
\end{entry}

\begin{entry}{无}{4}[Radical 无][Kangxi 71]
  \begin{phonetics}{无}{wu2}
    \definition{adv.}{não ter algo | não há\dots}
  \end{phonetics}
\end{entry}

\begin{entry}{无人}{4,2}
  \begin{phonetics}{无人}{wu2ren2}
    \definition{adj.}{não tripulado | desabitado}
  \end{phonetics}
\end{entry}

\begin{entry}{无人机}{4,2,6}
  \begin{phonetics}{无人机}{wu2ren2ji1}
    \definition{s.}{\emph{drone} | veículo aéreo não tripulado}
  \end{phonetics}
\end{entry}

\begin{entry}{无论……也……}{4,6,3}
  \begin{phonetics}{无论……也……}{wu2lun4 ye3}
    \definition{conj.}{não apenas\dots, (o que, quem, como, etc.), \dots}
  \end{phonetics}
\end{entry}

\begin{entry}{无视}{4,8}
  \begin{phonetics}{无视}{wu2shi4}
    \definition{v.}{ignorar | desconsiderar}
  \end{phonetics}
\end{entry}

\begin{entry}{无故}{4,9}
  \begin{phonetics}{无故}{wu2gu4}
    \definition{adv.}{sem causa ou razão | sem motivo}
  \end{phonetics}
\end{entry}

\begin{entry}{无骨}{4,9}
  \begin{phonetics}{无骨}{wu2 gu3}
    \definition{adj.}{desossado}
  \end{phonetics}
\end{entry}

\begin{entry}{无敌}{4,10}
  \begin{phonetics}{无敌}{wu2di2}
    \definition{adj.}{invencível | inigualável}
  \end{phonetics}
\end{entry}

\begin{entry}{无氧}{4,10}
  \begin{phonetics}{无氧}{wu2yang3}
    \definition{adj.}{anaeróbico}
  \end{phonetics}
\end{entry}

\begin{entry}{日}{4}[Radical 日][Kangxi 72]
  \begin{phonetics}{日}{ri4}[][HSK 1]
    \definition*{s.}{Japão, abreviação de~日本}
    \definition{clas.}{dia (mais usado em escrita) | data, dia do mês}
    \seeref{日本}{ri4ben3}
  \end{phonetics}
\end{entry}

\begin{entry}{日子}{4,3}
  \begin{phonetics}{日子}{ri4zi5}[][HSK 2]
    \definition{s.}{dia | uma data (calendário) | dias de vida de alguém}
  \end{phonetics}
\end{entry}

\begin{entry}{日出}{4,5}
  \begin{phonetics}{日出}{ri4chu1}
    \definition{s.}{nascer do sol}
  \seealsoref{夕阳}{xi1yang2}
  \end{phonetics}
\end{entry}

\begin{entry}{日本}{4,5}
  \begin{phonetics}{日本}{ri4ben3}
    \definition*{s.}{Japão}
  \end{phonetics}
\end{entry}

\begin{entry}{日本人}{4,5,2}
  \begin{phonetics}{日本人}{ri4ben3ren2}
    \definition{s.}{japonês | pessoa ou povo do Japão}
  \end{phonetics}
\end{entry}

\begin{entry}{日光灯}{4,6,6}
  \begin{phonetics}{日光灯}{ri4guang1deng1}
    \definition{s.}{lâmpada fluorescente}
  \end{phonetics}
\end{entry}

\begin{entry}{日报}{4,7}
  \begin{phonetics}{日报}{ri4 bao4}[][HSK 2]
    \definition[张]{s.}{diário | jornal diários}
  \end{phonetics}
\end{entry}

\begin{entry}{日常}{4,11}
  \begin{phonetics}{日常}{ri4chang2}
    \definition{adv.}{diariamente | dia-a-dia | todo dia}
  \end{phonetics}
\end{entry}

\begin{entry}{日期}{4,12}
  \begin{phonetics}{日期}{ri4qi1}[][HSK 1]
    \definition{s.}{data}
  \end{phonetics}
\end{entry}

\begin{entry}{月}{4}[Radical 月][Kangxi 74]
  \begin{phonetics}{月}{yue4}[][HSK 1]
    \definition[个,轮]{s.}{mês}
  \end{phonetics}
\end{entry}

\begin{entry}{月月}{4,4}
  \begin{phonetics}{月月}{yue4yue4}
    \definition{adv.}{todo mês}
  \end{phonetics}
\end{entry}

\begin{entry}{月份}{4,6}
  \begin{phonetics}{月份}{yue4 fen4}[][HSK 2]
    \definition{s.}{mês}
  \end{phonetics}
\end{entry}

\begin{entry}{月径}{4,8}
  \begin{phonetics}{月径}{yue4jing4}
    \definition{s.}{diâmetro da lua | diâmetro da órbita da lua | caminho iluminado pela lua}
  \end{phonetics}
\end{entry}

\begin{entry}{月亮}{4,9}
  \begin{phonetics}{月亮}{yue4liang5}[][HSK 2]
    \definition{s.}{lua}
  \end{phonetics}
\end{entry}

\begin{entry}{月相}{4,9}
  \begin{phonetics}{月相}{yue4xiang4}
    \definition{s.}{fases da lua, a saber: lua nova 朔, lua crescente 上弦, lua cheia 望 e lua minguante 下弦}
  \end{phonetics}
\end{entry}

\begin{entry}{月饼}{4,9}
  \begin{phonetics}{月饼}{yue4bing3}
    \definition[张]{s.}{bolo da lua}
  \end{phonetics}
\end{entry}

\begin{entry}{月球}{4,11}
  \begin{phonetics}{月球}{yue4qiu2}
    \definition{s.}{a lua}
  \end{phonetics}
\end{entry}

\begin{entry}{月壤}{4,20}
  \begin{phonetics}{月壤}{yue4rang3}
    \definition{s.}{solo lunar}
  \end{phonetics}
\end{entry}

\begin{entry}{木头}{4,5}
  \begin{phonetics}{木头}{mu4tou5}
    \definition{adj.}{estúpido | cabeça-dura}
    \definition[块,根]{s.}{tronco (de madeira)}
  \end{phonetics}
\end{entry}

\begin{entry}{木偶}{4,11}
  \begin{phonetics}{木偶}{mu4'ou3}
    \definition{s.}{fantoche, marionete}
  \end{phonetics}
\end{entry}

\begin{entry}{歹徒}{4,10}
  \begin{phonetics}{歹徒}{dai3tu2}
    \definition{s.}{malfeitor | gangster | bandido}
  \end{phonetics}
\end{entry}

\begin{entry}{比}{4}[Radical 匕][Kangxi 81]
  \begin{phonetics}{比}{bi3}[][HSK 1]
    \definition*{s.}{Bélgica, abreviação de 比利时}
    \definition{part.}{partícula usada para comparação (superioridade)}
    \definition{prep.}{que | do que | (seguido por um substantivo e adjetivo) mais \{adj.\} do que \{s.\}}
    \definition{s.}{razão (taxa)}
    \definition{v.}{comparar | contrastar | gesticular (com as mãos)}
    \seeref{比利时}{bi3li4shi2}
  \end{phonetics}
\end{entry}

\begin{entry}{比亚迪}{4,6,8}
  \begin{phonetics}{比亚迪}{bi3ya4di2}
    \definition*{s.}{Montadora BYD}
  \end{phonetics}
\end{entry}

\begin{entry}{比如}{4,6}
  \begin{phonetics}{比如}{bi3ru2}[][HSK 2]
    \definition{conj.}{por exemplo | como}
  \end{phonetics}
\end{entry}

\begin{entry}{比如说}{4,6,9}
  \begin{phonetics}{比如说}{bi3 ru2 shuo1}[][HSK 2]
    \definition{adv.}{por exemplo}
  \end{phonetics}
\end{entry}

\begin{entry}{比利时}{4,7,7}
  \begin{phonetics}{比利时}{bi3li4shi2}
    \definition*{s.}{Bélgica}
  \end{phonetics}
\end{entry}

\begin{entry}{比拼}{4,9}
  \begin{phonetics}{比拼}{bi3pin1}
    \definition{s.}{concurso}
    \definition{v.}{competir ferozmente}
  \end{phonetics}
\end{entry}

\begin{entry}{比较}{4,10}
  \begin{phonetics}{比较}{bi3jiao4}
    \definition{adv.}{comparativamente | relativamente}
    \definition{s.}{comparação}
    \definition{v.}{comparar}
  \end{phonetics}
\end{entry}

\begin{entry}{比萨饼}{4,11,9}
  \begin{phonetics}{比萨饼}{bi3sa4bing3}
    \definition[张]{s.}{pizza}
  \end{phonetics}
\end{entry}

\begin{entry}{比赛}{4,14}
  \begin{phonetics}{比赛}{bi3sai4}
    \definition[场,次]{s.}{competição | concurso}
    \definition{v.}{competir}
  \end{phonetics}
\end{entry}

\begin{entry}{毛}{4}[Radical 毛][Kangxi 82]
  \begin{phonetics}{毛}{mao2}[][HSK 1]
    \definition*{s.}{sobrenome Mao}
    \definition{clas.}{1 mao corresponde a 10 centavos}
  \end{phonetics}
\end{entry}

\begin{entry}{气}{4}
  \begin{phonetics}{气}{qi4}[][HSK 2]
    \definition[口]{s.}{gás | ar | respiração | clima | cheiro | odor | espírito | moral | ares | maneira | estilo | insulto | intimidação | energia vital | energia da vida}
    \definition{v.}{ficar bravo | ficar enfurecido | irritar | enfurecer}
  \end{phonetics}
\end{entry}

\begin{entry}{气质}{4,8}
  \begin{phonetics}{气质}{qi4zhi4}
    \definition{s.}{traços de personalidade, temperamento, disposição | aura, ar, sentimento, \emph{vibe} | refinamento, sofisticação, classe}
  \end{phonetics}
\end{entry}

\begin{entry}{气球}{4,11}
  \begin{phonetics}{气球}{qi4qiu2}
    \definition{s.}{balão}
  \end{phonetics}
\end{entry}

\begin{entry}{气温}{4,12}
  \begin{phonetics}{气温}{qi4 wen1}[][HSK 2]
    \definition[个]{s.}{temperatura do ar}
  \end{phonetics}
\end{entry}

\begin{entry}{水}{4}[Radical 水][Kangxi 85]
  \begin{phonetics}{水}{shui3}[][HSK 1]
    \definition*{s.}{sobrenome Shui}
    \definition{clas.}{para número de lavagens}
    \definition{s.}{água | líquido | encargos ou receitas adicionais}
  \end{phonetics}
\end{entry}

\begin{entry}{水平}{4,5}
  \begin{phonetics}{水平}{shui3ping2}[][HSK 2]
    \definition{s.}{nível (de realização, etc.) | padrão | nível horizontal}
  \end{phonetics}
\end{entry}

\begin{entry}{水平以下}{4,5,4,3}
  \begin{phonetics}{水平以下}{shui3ping2 yi3xia4}
    \definition{s.}{sub-nível}
  \end{phonetics}
\end{entry}

\begin{entry}{水平尺}{4,5,4}
  \begin{phonetics}{水平尺}{shui3ping2chi3}
    \definition{s.}{nível espiritual}
  \end{phonetics}
\end{entry}

\begin{entry}{水平仪}{4,5,5}
  \begin{phonetics}{水平仪}{shui3ping2yi2}
    \definition{s.}{nível (dispositivo para determinar horizontal) | nível espiritual | nível de topógrafo}
  \end{phonetics}
\end{entry}

\begin{entry}{水平视差}{4,5,8,9}
  \begin{phonetics}{水平视差}{shui3ping2 shi4cha1}
    \definition{s.}{paralaxe horizontal}
  \end{phonetics}
\end{entry}

\begin{entry}{水平度}{4,5,9}
  \begin{phonetics}{水平度}{shui3ping2 du4}
    \definition{s.}{nivelamento}
  \end{phonetics}
\end{entry}

\begin{entry}{水平轴}{4,5,9}
  \begin{phonetics}{水平轴}{shui3ping2zhou2}
    \definition{s.}{eixo horizontal}
  \end{phonetics}
\end{entry}

\begin{entry}{水平面}{4,5,9}
  \begin{phonetics}{水平面}{shui3ping2mian4}
    \definition{s.}{plano horizontal | nível-da-água | superfície horizontal}
  \end{phonetics}
\end{entry}

\begin{entry}{水边}{4,5}
  \begin{phonetics}{水边}{shui3bian1}
    \definition{s.}{beira d'água | beira-mar | costa (de mar, lago ou rio)}
  \end{phonetics}
\end{entry}

\begin{entry}{水污染}{4,6,9}
  \begin{phonetics}{水污染}{shui3wu1ran3}
    \definition{s.}{poluição da água}
  \end{phonetics}
\end{entry}

\begin{entry}{水灵}{4,7}
  \begin{phonetics}{水灵}{shui3ling2}
    \definition{adj.}{cheio de vida (sobre uma pessoa, etc.) | úmido e brilhante (sobre os olhos) | fresco (sobre frutas, etc.) | brilhante | aparência saudável}
  \end{phonetics}
\end{entry}

\begin{entry}{水果}{4,8}
  \begin{phonetics}{水果}{shui3guo3}[][HSK 1]
    \definition[个]{s.}{fruta}
  \end{phonetics}
\end{entry}

\begin{entry}{水波}{4,8}
  \begin{phonetics}{水波}{shui3bo1}
    \definition{s.}{ondulação (na água) | onda}
  \end{phonetics}
\end{entry}

\begin{entry}{水饺}{4,9}
  \begin{phonetics}{水饺}{shui3jiao3}
    \definition{s.}{\emph{dumplings} | pastéis chineses cozidos}
  \end{phonetics}
\end{entry}

\begin{entry}{水瓶}{4,10}
  \begin{phonetics}{水瓶}{shui3 ping2}
    \definition{s.}{garrada de água}
  \end{phonetics}
\end{entry}

\begin{entry}{水培}{4,11}
  \begin{phonetics}{水培}{shui3pei2}
    \definition{v.}{cultivar plantas hidroponicamente}
  \end{phonetics}
\end{entry}

\begin{entry}{水豚}{4,11}
  \begin{phonetics}{水豚}{shui3tun2}
    \definition{s.}{capivara}
  \end{phonetics}
\end{entry}

\begin{entry}{水路}{4,13}
  \begin{phonetics}{水路}{shui3lu4}
    \definition{s.}{hidrovia}
  \end{phonetics}
\end{entry}

\begin{entry}{水槽}{4,15}
  \begin{phonetics}{水槽}{shui3cao2}
    \definition{s.}{pia (de cozinha)}
  \end{phonetics}
\end{entry}

\begin{entry}{火}{4}[Radical 火][Kangxi 86]
  \begin{phonetics}{火}{huo3}
    \definition*{s.}{sobrenome Huo}
    \definition{adj.}{urgente | ardente ou flamejante | quente (popular)}
    \definition{clas.}{para unidades militares (antigo)}
    \definition{s.}{fogo | munição | calor interno (medicina chinesa)}
  \end{phonetics}
\end{entry}

\begin{entry}{火车}{4,4}
  \begin{phonetics}{火车}{huo3che1}[][HSK 1]
    \definition[列,节,班,趟]{s.}{trem | comboio}
  \end{phonetics}
\end{entry}

\begin{entry}{火车司机}{4,4,5,6}
  \begin{phonetics}{火车司机}{huo3che1 si1ji1}
    \definition{s.}{maquinista de trem}
  \end{phonetics}
\end{entry}

\begin{entry}{火柴}{4,10}
  \begin{phonetics}{火柴}{huo3chai2}
    \definition[根,盒]{s.}{fósforo (palito de fósforo)}
  \end{phonetics}
\end{entry}

\begin{entry}{火海}{4,10}
  \begin{phonetics}{火海}{huo3hai3}
    \definition{s.}{um mar de chamas}
  \end{phonetics}
\end{entry}

\begin{entry}{父母亲}{4,5,9}
  \begin{phonetics}{父母亲}{fu4mu3qin1}
    \definition{s.}{pais}
  \end{phonetics}
\end{entry}

\begin{entry}{父亲}{4,9}
  \begin{phonetics}{父亲}{fu4qin1}
    \definition[个]{s.}{pai}
  \end{phonetics}
\end{entry}

\begin{entry}{片}{4}[Radical 片][Kangxi 91]
  \begin{phonetics}{片}{pian4}[][HSK 2]
    \definition{adj.}{parcial | incompleto | que só tem um lado}
    \definition{clas.}{para CDs, filmes, DVDs, etc. | para fatias, comprimidos, extensão de terra, área de água | usado com numeral~一:~para  cenário, cena, sentimento, atmosfera, som etc.}
    \definition{s.}{uma fatia | floco | filme | pedaço fino}
    \definition{v.}{fatiar | esculpir fino}
  \end{phonetics}
\end{entry}

\begin{entry}{牙}{4}[Radical 牙][Kangxi 92]
  \begin{phonetics}{牙}{ya2}
    \definition[颗]{s.}{dente | marfim}
  \end{phonetics}
\end{entry}

\begin{entry}{牙行}{4,6}
  \begin{phonetics}{牙行}{ya2hang2}
    \definition{s.}{corretor | \emph{broker}}
  \end{phonetics}
\end{entry}

\begin{entry}{牙医}{4,7}
  \begin{phonetics}{牙医}{ya2yi1}
    \definition{s.}{dentista}
  \end{phonetics}
\end{entry}

\begin{entry}{牙刷}{4,8}
  \begin{phonetics}{牙刷}{ya2shua1}
    \definition[把]{s.}{escova de dentes}
  \end{phonetics}
\end{entry}

\begin{entry}{牙线}{4,8}
  \begin{phonetics}{牙线}{ya2xian4}
    \definition[条]{s.}{fio dental}
  \end{phonetics}
\end{entry}

\begin{entry}{牙齿}{4,8}
  \begin{phonetics}{牙齿}{ya2chi3}
    \definition{adv.}{dental}
    \definition[颗]{s.}{dente}
  \end{phonetics}
\end{entry}

\begin{entry}{牙膏}{4,14}
  \begin{phonetics}{牙膏}{ya2gao1}
    \definition[管]{s.}{pasta de dente}
  \end{phonetics}
\end{entry}

\begin{entry}{牛}{4}[Radical 牛][Kangxi 93]
  \begin{phonetics}{牛}{niu2}
    \definition*{s.}{sobrenome Niu}
    \definition[条,头]{s.}{boi | touro | vaca | (gíria) incrível}
  \end{phonetics}
\end{entry}

\begin{entry}{牛人}{4,2}
  \begin{phonetics}{牛人}{niu2ren2}
    \definition{s.}{(coloquial) o cara | verdadeiro especialista | \emph{badass}}
  \end{phonetics}
\end{entry}

\begin{entry}{牛仔裤}{4,5,12}
  \begin{phonetics}{牛仔裤}{niu2zai3ku4}
    \definition[条]{s.}{calça de ganga, jeans}
  \end{phonetics}
\end{entry}

\begin{entry}{牛奶}{4,5}
  \begin{phonetics}{牛奶}{niu2nai3}[][HSK 1]
    \definition[瓶,杯]{s.}{leite de vaca}
  \end{phonetics}
\end{entry}

\begin{entry}{牛肉}{4,6}
  \begin{phonetics}{牛肉}{niu2rou4}
    \definition{s.}{carne de vaca | bife}
  \end{phonetics}
\end{entry}

\begin{entry}{牛郎织女}{4,8,8,3}
  \begin{phonetics}{牛郎织女}{niu2lang2zhi1nv3}
    \definition*{s.}{Vaqueiro e Tecelã (personagens de contos folclóricos) | amantes separados | Altair e Vega (estrelas)}
  \end{phonetics}
\end{entry}

\begin{entry}{牛顿}{4,10}
  \begin{phonetics}{牛顿}{niu2dun4}
    \definition*{s.}{Newton (nome) | newton (N, unidade de força do SI)}
  \end{phonetics}
\end{entry}

\begin{entry}{犬}{4}[Radical 犬][Kangxi 94]
  \begin{phonetics}{犬}{quan3}
    \definition{s.}{cachorro}
  \end{phonetics}
\end{entry}

\begin{entry}{王}{4}[Radical 玉]
  \begin{phonetics}{王}{wang2}
    \definition*{s.}{sobrenome Wang}
    \definition{adj.}{grande | ótimo}
    \definition{s.}{rei ou monarca | melhor ou mais forte do seu tipo}
  \end{phonetics}
  \begin{phonetics}{王}{wang4}
    \definition{v.}{(literário) (um monarca) reinar (um reino)}
  \end{phonetics}
\end{entry}

\begin{entry}{王五}{4,4}
  \begin{phonetics}{王五}{wang2wu3}
    \definition{s.}{Wang Wu | Zé Ninguém | nome para uma pessoa não especificada, 3 de 3}
  \seealsoref{李四}{li3si4}
  \seealsoref{张三}{zhang1san1}
  \end{phonetics}
\end{entry}

\begin{entry}{王朝}{4,12}
  \begin{phonetics}{王朝}{wang2chao2}
    \definition{s.}{dinastia}
  \end{phonetics}
\end{entry}

\begin{entry}{瓦}{4}[Radical 瓦][Kangxi 98]
  \begin{phonetics}{瓦}{wa3}
    \definition{s.}{telha | abreviação de 瓦特}
    \seeref{瓦特}{wa3te4}
  \end{phonetics}
\end{entry}

\begin{entry}{瓦努阿图}{4,7,7,8}
  \begin{phonetics}{瓦努阿图}{wa3nu3'a1tu2}
    \definition*{s.}{Vanuatu, país do sudoeste do Oceano Pacífico}
  \end{phonetics}
\end{entry}

\begin{entry}{瓦特}{4,10}
  \begin{phonetics}{瓦特}{wa3te4}
    \definition{s.}{(empréstimo linguístico) watt | medida de potência}
  \end{phonetics}
\end{entry}

\begin{entry}{艺人}{4,2}
  \begin{phonetics}{艺人}{yi4ren2}
    \definition{s.}{artista | ator}
  \end{phonetics}
\end{entry}

\begin{entry}{见}{4}[Radical 見]
  \begin{phonetics}{见}{jian4}[][HSK 1]
    \definition{s.}{opinião, visão}
    \definition{v.}{ver | entrevistar | encontrar alguém | parecer (ser alguma coisa)}
  \end{phonetics}
  \begin{phonetics}{见}{xian4}[][HSK 0]
    \definition{v.}{aparecer | também escrito como 现}
    \seeref{现}{xian4}
  \end{phonetics}
\end{entry}

\begin{entry}{见过}{4,6}
  \begin{phonetics}{见过}{jian4 guo4}[][HSK 2]
    \definition{s.}{visto (ver)}
  \end{phonetics}
\end{entry}

\begin{entry}{见到}{4,8}
  \begin{phonetics}{见到}{jian4 dao4}[][HSK 2]
    \definition{v.}{ver | esbarrar em | encontrar-se com}
  \end{phonetics}
\end{entry}

\begin{entry}{见面}{4,9}
  \begin{phonetics}{见面}{jian4 mian4}[][HSK 1]
    \definition{v.+compl.}{encontrar-se com alguém | ver alguém face-a-face}
  \end{phonetics}
\end{entry}

\begin{entry}{计划}{4,6}
  \begin{phonetics}{计划}{ji4hua4}[][HSK 2]
    \definition[个,项]{s.}{plano | projeto | programa}
    \definition{v.}{planejar | mapear}
  \end{phonetics}
\end{entry}

\begin{entry}{计算机}{4,14,6}
  \begin{phonetics}{计算机}{ji4 suan4 ji1}[][HSK 2]
    \definition[部,台]{s.}{computador | calculadora}
  \end{phonetics}
\end{entry}

\begin{entry}{认为}{4,4}
  \begin{phonetics}{认为}{ren4wei2}[][HSK 2]
    \definition{v.}{pensar | considerar | segurar | julgar}
  \end{phonetics}
\end{entry}

\begin{entry}{认识}{4,7}
  \begin{phonetics}{认识}{ren4shi5}[][HSK 1]
    \definition{s.}{conhecimento | saber | entendimento}
    \definition{v.}{estar familiarizado com | conhecer alguém | saber | reconhecer}
  \end{phonetics}
\end{entry}

\begin{entry}{认真}{4,10}
  \begin{phonetics}{认真}{ren4zhen1}[][HSK 1]
    \definition{adj.}{sério | consciencioso}
    \definition{adv.}{seriamente}
    \definition{v.}{levar a sério}
  \end{phonetics}
\end{entry}

\begin{entry}{车}{4}[Radical 車][Kangxi 159]
  \begin{phonetics}{车}{che1}[][HSK 1]
    \definition*{s.}{sobrenome Che}
    \definition[辆]{s.}{carro | veículo | viatura}
  \end{phonetics}
  \begin{phonetics}{车}{ju1}[][HSK 0]
    \definition{s.}{(arcaico) carruagem de guerra | torre (no xadrez)}
  \end{phonetics}
\end{entry}

\begin{entry}{车上}{4,3}
  \begin{phonetics}{车上}{che1 shang5}[][HSK 1]
    \definition{adv.}{no carro | dentro do veículo}
  \end{phonetics}
\end{entry}

\begin{entry}{车子}{4,3}
  \begin{phonetics}{车子}{che1zi5}
    \definition{s.}{qualquer veículo (carro, bicicleta, caminhão, etc)}
  \end{phonetics}
\end{entry}

\begin{entry}{车水马龙}{4,4,3,5}
  \begin{phonetics}{车水马龙}{che1shui3-ma3long2}
    \definition{expr.}{tráfego engarrafado | engarrafamento | (literalmente) ``fluxo interminável de cavalos e carruagens''}
  \end{phonetics}
\end{entry}

\begin{entry}{车主}{4,5}
  \begin{phonetics}{车主}{che1zhu3}
    \definition{s.}{proprietário do carro}
  \end{phonetics}
\end{entry}

\begin{entry}{车次}{4,6}
  \begin{phonetics}{车次}{che1ci4}
    \definition{s.}{número do trem}
  \end{phonetics}
\end{entry}

\begin{entry}{车库}{4,7}
  \begin{phonetics}{车库}{che1ku4}
    \definition{s.}{garagem}
  \end{phonetics}
\end{entry}

\begin{entry}{车站}{4,10}
  \begin{phonetics}{车站}{che1zhan4}[][HSK 1]
    \definition[处,个]{s.}{estação | ponto de ônibus}
  \end{phonetics}
\end{entry}

\begin{entry}{车票}{4,11}
  \begin{phonetics}{车票}{che1piao4}[][HSK 1]
    \definition{s.}{bilhete (de ônibus, trem, metrô)}
  \end{phonetics}
\end{entry}

\begin{entry}{车辆}{4,11}
  \begin{phonetics}{车辆}{che1 liang4}[][HSK 2]
    \definition{s.}{veículo | carro}
  \end{phonetics}
\end{entry}

\begin{entry}{车牌}{4,12}
  \begin{phonetics}{车牌}{che1pai2}
    \definition{s.}{matrícula | placa de carro}
  \end{phonetics}
\end{entry}

\begin{entry}{长}{4}[Radical 長]
  \begin{phonetics}{长}{chang2}[][HSK 2]
    \definition{adj.}{comprido | longo}
  \end{phonetics}
  \begin{phonetics}{长}{zhang3}[][HSK 2]
    \definition{s.}{chefe | ancião}
    \definition{v.}{crescer | desenvolver | aumentar | melhorar}
  \end{phonetics}
\end{entry}

\begin{entry}{长大}{4,3}
  \begin{phonetics}{长大}{zhang3 da4}[][HSK 2]
    \definition{v.}{crescer | ser criado}
  \end{phonetics}
\end{entry}

\begin{entry}{长城}{4,9}
  \begin{phonetics}{长城}{chang2cheng2}
    \definition*{s.}{Grande Muralha}
  \end{phonetics}
\end{entry}

\begin{entry}{长颈鹿}{4,11,11}
  \begin{phonetics}{长颈鹿}{chang2jing3lu4}
    \definition[只]{s.}{girafa}
  \end{phonetics}
\end{entry}

\begin{entry}{队}{4}[Radical 阜]
  \begin{phonetics}{队}{dui4}[][HSK 2]
    \definition[个]{s.}{esquadrão | equipe | grupo}
  \end{phonetics}
\end{entry}

\begin{entry}{队友}{4,4}
  \begin{phonetics}{队友}{dui4you3}
    \definition{s.}{companheiro de equipe}
  \end{phonetics}
\end{entry}

\begin{entry}{队长}{4,4}
  \begin{phonetics}{队长}{dui4 zhang3}[][HSK 2]
    \definition{s.}{capitão (de equipe) | líder da equipe}
  \end{phonetics}
\end{entry}

\begin{entry}{风}{4}[Radical 風][Kangxi 182]
  \begin{phonetics}{风}{feng1}[][HSK 1]
    \definition[阵,丝]{s.}{vento}
  \end{phonetics}
\end{entry}

\begin{entry}{风扇}{4,10}
  \begin{phonetics}{风扇}{feng1shan4}
    \definition{s.}{ventilador elétrico}
  \end{phonetics}
\end{entry}

\begin{entry}{风景}{4,12}
  \begin{phonetics}{风景}{feng1jing3}
    \definition{s.}{cenário | paisagem}
  \end{phonetics}
\end{entry}

\begin{entry}{风筝}{4,12}
  \begin{phonetics}{风筝}{feng1zheng5}
    \definition{s.}{pipa | papagaio | pandorga}
  \end{phonetics}
\end{entry}

%%%%% EOF %%%%%


%%%
%%% 5画
%%%

\section*{5画}\addcontentsline{toc}{section}{5画}

\begin{entry}{㐌}{5}[Radical 乙]
  \begin{phonetics}{㐌}{ta1}
    \variantof{它}
  \end{phonetics}
\end{entry}

\begin{entry}{世代}{5,5}
  \begin{phonetics}{世代}{shi4dai4}
    \definition{adv.}{por muitas gerações, eras}
    \definition{s.}{geração | era}
  \end{phonetics}
\end{entry}

\begin{entry}{世界}{5,9}
  \begin{phonetics}{世界}{shi4jie4}
    \definition[个]{s.}{mundo}
  \end{phonetics}
\end{entry}

\begin{entry}{世界杯}{5,9,8}
  \begin{phonetics}{世界杯}{shi4jie4bei1}
    \definition*{s.}{Copa do Mundo}
  \end{phonetics}
\end{entry}

\begin{entry}{世锦赛}{5,13,14}
  \begin{phonetics}{世锦赛}{shi4jin3sai4}
    \definition*{s.}{Campeonato Mundial}
  \end{phonetics}
\end{entry}

\begin{entry}{丘陵}{5,10}
  \begin{phonetics}{丘陵}{qiu1ling2}
    \definition{s.}{colinas}
  \end{phonetics}
\end{entry}

\begin{entry}{东}{5}[Radical ⼀]
  \begin{phonetics}{东}{dong1}
    \definition*{s.}{sobrenome Dong}
    \definition{s.}{leste}
  \end{phonetics}
\end{entry}

\begin{entry}{东方}{5,4}
  \begin{phonetics}{东方}{dong1fang1}
    \definition*{s.}{sobrenome Dongfang}
    \definition{s.}{leste | oriente}
  \end{phonetics}
\end{entry}

\begin{entry}{东方学院}{5,4,8,9}
  \begin{phonetics}{东方学院}{dong1fang1 xue2yuan4}
    \definition*{s.}{Instituto Oriental}
  \end{phonetics}
\end{entry}

\begin{entry}{东北}{5,5}
  \begin{phonetics}{东北}{dong1bei3}
    \definition*{s.}{Nordeste da China | Manchúria}
    \definition{s.}{nordeste}
  \end{phonetics}
\end{entry}

\begin{entry}{东半球}{5,5,11}
  \begin{phonetics}{东半球}{dong1ban4qiu2}
    \definition*{s.}{Hemisfério Oriental}
  \end{phonetics}
\end{entry}

\begin{entry}{东边}{5,5}
  \begin{phonetics}{东边}{dong1bian5}
    \definition{s.}{este | leste | lado leste | oriente}
  \end{phonetics}
\end{entry}

\begin{entry}{东西}{5,6}
  \begin{phonetics}{东西}{dong1xi1}
    \definition{s.}{leste e oeste}
  \end{phonetics}
  \begin{phonetics}{东西}{dong1xi5}
    \definition[个,件]{s.}{coisa | material | pessoa}
  \end{phonetics}
\end{entry}

\begin{entry}{东面}{5,9}
  \begin{phonetics}{东面}{dong1mian4}
    \definition{s.}{lado leste (de algo)}
  \end{phonetics}
\end{entry}

\begin{entry}{东部}{5,10}
  \begin{phonetics}{东部}{dong1bu4}
    \definition{s.}{leste | oriente}
  \end{phonetics}
\end{entry}

\begin{entry}{丝}{5}[Radical 一]
  \begin{phonetics}{丝}{si1}
    \definition{adj.}{filiforme | delgado como um fio | que se assemelha a um fio}
    \definition{clas.}{um traço (de fumaça, etc.) | um pouquinho, etc.}
    \definition{s.}{seda | (cozinha) pedaços ou tiras de julienne, tiras cortadas finas}
  \end{phonetics}
\end{entry}

\begin{entry}{主义}{5,3}
  \begin{phonetics}{主义}{zhu3yi4}
    \definition{s.}{ideologia}
    \definition{suf.}{"ismo"}
  \end{phonetics}
\end{entry}

\begin{entry}{主席}{5,10}
  \begin{phonetics}{主席}{zhu3xi2}
    \definition*[个,位]{s.}{Presidente (da China) | Primeiro-Ministro}
  \end{phonetics}
\end{entry}

\begin{entry}{主席台}{5,10,5}
  \begin{phonetics}{主席台}{zhu3xi2tai2}
    \definition[个]{s.}{plataforma | tribuna}
  \end{phonetics}
\end{entry}

\begin{entry}{主席团}{5,10,6}
  \begin{phonetics}{主席团}{zhu3xi2tuan2}
    \definition{s.}{presídio}
  \end{phonetics}
\end{entry}

\begin{entry}{乐观}{5,6}
  \begin{phonetics}{乐观}{le4guan1}
    \definition{adj.}{otimista | esperançoso}
  \end{phonetics}
\end{entry}

\begin{entry}{乐园}{5,7}
  \begin{phonetics}{乐园}{le4yuan2}
    \definition{s.}{paraíso}
  \end{phonetics}
\end{entry}

\begin{entry}{乐高}{5,10}
  \begin{phonetics}{乐高}{le4gao1}
    \definition*{s.}{Lego (brinquedo)}
  \end{phonetics}
\end{entry}

\begin{entry}{他}{5}[Radical 人]
  \begin{phonetics}{他}{ta1}
    \definition{pron.}{ele | se, o, lhe | si, consigo, ele}
    \seeref{怹}{tan1}
  \end{phonetics}
\end{entry}

\begin{entry}{他们}{5,5}
  \begin{phonetics}{他们}{ta1men5}
    \definition{pron.}{eles | se, os, lhes | si, consigo, eles}
  \end{phonetics}
\end{entry}

\begin{entry}{他们的}{5,5,8}
  \begin{phonetics}{他们的}{ta1men5 de5}
    \definition{pron.}{deles}
  \end{phonetics}
\end{entry}

\begin{entry}{他妈的}{5,6,8}
  \begin{phonetics}{他妈的}{ta1ma1de5}
    \definition{interj.}{Dane-se! | Foda-se!}
  \end{phonetics}
\end{entry}

\begin{entry}{他的}{5,8}
  \begin{phonetics}{他的}{ta1 de5}
    \definition{pron.}{dele}
  \end{phonetics}
\end{entry}

\begin{entry}{付}{5}[Radical 人]
  \begin{phonetics}{付}{fu4}
    \definition*{s.}{sobrenome Fu}
    \definition{clas.}{para pares ou conjuntos de coisas}
    \definition{v.}{pagar}
  \end{phonetics}
\end{entry}

\begin{entry}{付款}{5,12}
  \begin{phonetics}{付款}{fu4kuan3}
    \definition{s.}{pagamento}
    \definition{v.+compl.}{pagar uma quantia em dinheiro}
  \end{phonetics}
\end{entry}

\begin{entry}{仙}{5}[Radical 人]
  \begin{phonetics}{仙}{xian1}
    \definition{s.}{imortal}
  \end{phonetics}
\end{entry}

\begin{entry}{代价}{5,6}
  \begin{phonetics}{代价}{dai4jia4}
    \definition{s.}{preço | custo}
  \end{phonetics}
\end{entry}

\begin{entry}{代言}{5,7}
  \begin{phonetics}{代言}{dai4yan2}
    \definition{v.}{ser um porta-voz | ser um embaixador (para uma marca) | endossar}
  \end{phonetics}
\end{entry}

\begin{entry}{代表团}{5,8,6}
  \begin{phonetics}{代表团}{dai4biao3tuan2}
    \definition[个]{s.}{delegação}
  \end{phonetics}
\end{entry}

\begin{entry}{代称}{5,10}
  \begin{phonetics}{代称}{dai4cheng1}
    \definition{s.}{nome alternativo | antonomásia}
    \definition{v.}{referir-se a algo ou alguém por outro nome}
  \end{phonetics}
\end{entry}

\begin{entry}{令人}{5,2}
  \begin{phonetics}{令人}{ling4ren2}
    \definition{v.}{causar alguém (a fazer alguma coisa) | fazer alguém ficar zangado, encantado, etc.}
  \end{phonetics}
\end{entry}

\begin{entry}{仪式}{5,6}
  \begin{phonetics}{仪式}{yi2shi4}
    \definition{s.}{cerimônia}
  \end{phonetics}
\end{entry}

\begin{entry}{们}{5}[Radical 人]
  \begin{phonetics}{们}{men5}
    \definition{part.}{sufixo para plural de pronomes e substantivos referentes a indivíduos}
  \end{phonetics}
\end{entry}

\begin{entry}{兄弟}{5,7}
  \begin{phonetics}{兄弟}{xiong1di4}
    \definition{adj.}{fraternal | \emph{brotherly}}
    \definition{pron.}{eu, me (termo de uso humilde por homens em discurso público)}
    \definition[个]{s.}{irmãos | irmão mais novo | \emph{brothers}}
  \end{phonetics}
\end{entry}

\begin{entry}{兰花}{5,7}
  \begin{phonetics}{兰花}{lan2hua1}
    \definition{s.}{orquídea}
  \end{phonetics}
\end{entry}

\begin{entry}{写}{5}[Radical 冖]
  \begin{phonetics}{写}{xie3}
    \definition{v.}{escrever}
  \end{phonetics}
\end{entry}

\begin{entry}{写字}{5,6}
  \begin{phonetics}{写字}{xie3zi4}
    \definition{v.}{escrever (à mão) | praticar caligrafia}
  \end{phonetics}
\end{entry}

\begin{entry}{写字匠}{5,6,6}
  \begin{phonetics}{写字匠}{xie3zi4 jiang4}
    \definition{s.}{calígrafo}
  \end{phonetics}
\end{entry}

\begin{entry}{写作}{5,7}
  \begin{phonetics}{写作}{xie3zuo4}
    \definition{s.}{escrita | redação | composição}
    \definition{v.}{escrever}
  \end{phonetics}
\end{entry}

\begin{entry}{写真}{5,10}
  \begin{phonetics}{写真}{xie3zhen1}
    \definition{s.}{retrato}
    \definition{v.}{descrever algo com precisão}
  \end{phonetics}
\end{entry}

\begin{entry}{写意}{5,13}
  \begin{phonetics}{写意}{xie3yi4}
    \definition{s.}{estilo de pintura chinesa à mão livre, caracterizado por traços ousados em vez de detalhes precisos}
    \definition{v.}{sugerir (em vez de descrever em detalhes)}
  \end{phonetics}
  \begin{phonetics}{写意}{xie4yi4}
    \definition{adj.}{confortável | agradável | relaxado}
  \end{phonetics}
\end{entry}

\begin{entry}{写照}{5,13}
  \begin{phonetics}{写照}{xie3zhao4}
    \definition{s.}{retrato}
  \end{phonetics}
\end{entry}

\begin{entry}{冬天}{5,4}
  \begin{phonetics}{冬天}{dong1tian1}
    \definition{s.}{inverno}
  \end{phonetics}
\end{entry}

\begin{entry}{冬瓜}{5,5}
  \begin{phonetics}{冬瓜}{dong1gua1}
    \definition{s.}{melão de inverno}
  \end{phonetics}
\end{entry}

\begin{entry}{出}{5}[Radical ⼐]
  \begin{phonetics}{出}{chu1}
    \definition{clas.}{para dramas, peças, óperas, etc.}
    \definition{v.}{sair | ir para fora | vir para fora}
  \end{phonetics}
\end{entry}

\begin{entry}{出口}{5,3}
  \begin{phonetics}{出口}{chu1kou3}
    \definition[个]{s.}{exportação}
    \definition{v.+compl.}{exportar}
  \end{phonetics}
\end{entry}

\begin{entry}{出击}{5,5}
  \begin{phonetics}{出击}{chu1ji1}
    \definition{v.}{atacar}
  \end{phonetics}
\end{entry}

\begin{entry}{出去}{5,5}
  \begin{phonetics}{出去}{chu1qu4}
    \definition{v.}{sair | ir para fora (a partir da minha localização)}
  \end{phonetics}
\end{entry}

\begin{entry}{出发}{5,5}
  \begin{phonetics}{出发}{chu1fa1}
    \definition{v.}{partir | começar (uma jornada)}
  \end{phonetics}
\end{entry}

\begin{entry}{出汗}{5,6}
  \begin{phonetics}{出汗}{chu1han4}
    \definition{v.}{transpirar | suar}
  \end{phonetics}
\end{entry}

\begin{entry}{出行}{5,6}
  \begin{phonetics}{出行}{chu1xing2}
    \definition{v.}{sair para algum lugar (viagem relativamente curta) | partir em uma viagem (viagem mais longa)}
  \end{phonetics}
\end{entry}

\begin{entry}{出来}{5,7}
  \begin{phonetics}{出来}{chu1lai2}
    \definition{v.}{sair | vir para fora (para a minha localização)}
  \end{phonetics}
\end{entry}

\begin{entry}{出版}{5,8}
  \begin{phonetics}{出版}{chu1ban3}
    \definition{v.}{publicar | editar}
  \end{phonetics}
\end{entry}

\begin{entry}{出版社}{5,8,7}
  \begin{phonetics}{出版社}{chu1ban3she4}
    \definition{s.}{editora}
  \end{phonetics}
\end{entry}

\begin{entry}{出差}{5,9}
  \begin{phonetics}{出差}{chu1chai1}
    \definition{v.+compl.}{fazer uma viagem oficial ou de negócios}
  \end{phonetics}
\end{entry}

\begin{entry}{出租}{5,10}
  \begin{phonetics}{出租}{chu1zu1}
    \definition{v.}{alugar | arrendar}
  \end{phonetics}
\end{entry}

\begin{entry}{出租车}{5,10,4}
  \begin{phonetics}{出租车}{chu1zu1che1}
    \definition{s.}{táxi}
  \seealsoref{出租汽车}{chu1zu1qi4che1}
  \end{phonetics}
\end{entry}

\begin{entry}{出租司机}{5,10,5,6}
  \begin{phonetics}{出租司机}{chu1zu1si1ji1}
    \definition{s.}{motorista de táxi}
  \end{phonetics}
\end{entry}

\begin{entry}{出租汽车}{5,10,7,4}
  \begin{phonetics}{出租汽车}{chu1zu1qi4che1}
    \definition[辆]{s.}{táxi}
  \seealsoref{出租车}{chu1zu1che1}
  \end{phonetics}
\end{entry}

\begin{entry}{出站}{5,10}
  \begin{phonetics}{出站}{chu1 zhan4}
    \definition{s.}{saída da estação}
  \end{phonetics}
\end{entry}

\begin{entry}{功夫}{5,4}
  \begin{phonetics}{功夫}{gong1fu5}
    \definition*{s.}{Gongfu (Kung Fu), arte marcial}
    \definition{s.}{esforço | habilidade}
  \end{phonetics}
\end{entry}

\begin{entry}{功臣}{5,6}
  \begin{phonetics}{功臣}{gong1chen2}
    \definition{s.}{oficial meritório | pessoa que presta serviço excepcional, herói | (fig.) alguém que desempenha um papel vital}
  \end{phonetics}
\end{entry}

\begin{entry}{加}{5}[Radical 力]
  \begin{phonetics}{加}{jia1}
    \definition*{s.}{Canadá, abreviação de~加拿大 | sobrenome Jia}
    \seeref{加拿大}{jia1na2da4}
  \end{phonetics}
\end{entry}

\begin{entry}{加入}{5,2}
  \begin{phonetics}{加入}{jia1ru4}
    \definition{v.}{tornar-se um membro | juntar-se | participar de | adicionar em}
  \end{phonetics}
\end{entry}

\begin{entry}{加工}{5,3}
  \begin{phonetics}{加工}{jia1gong1}
    \definition{s.}{processo | trabalho (de uma máquina)}
    \definition{v.}{processar}
  \end{phonetics}
\end{entry}

\begin{entry}{加油}{5,8}
  \begin{phonetics}{加油}{jia1you2}
    \definition{v.+compl.}{lubrificar | encher o tanque de combustível | fazer um esforço maior | fazer um esforço extra}
  \end{phonetics}
\end{entry}

\begin{entry}{加拿大}{5,10,3}
  \begin{phonetics}{加拿大}{jia1na2da4}
    \definition{s.}{Canadá}
  \end{phonetics}
\end{entry}

\begin{entry}{加拿大人}{5,10,3,2}
  \begin{phonetics}{加拿大人}{jia1na2da4ren2}
    \definition{s.}{canadense | pessoa ou povo do Canadá}
  \end{phonetics}
\end{entry}

\begin{entry}{加速}{5,10}
  \begin{phonetics}{加速}{jia1su4}
    \definition{v.}{acelerar | agilizar}
  \end{phonetics}
\end{entry}

\begin{entry}{加速度}{5,10,9}
  \begin{phonetics}{加速度}{jia1su4du4}
    \definition{s.}{aceleração}
  \end{phonetics}
\end{entry}

\begin{entry}{务实}{5,8}
  \begin{phonetics}{务实}{wu4shi2}
    \definition{adj.}{pragmático}
    \definition{v.}{lidar com assuntos concretos}
  \end{phonetics}
\end{entry}

\begin{entry}{包}{5}[Radical 勹]
  \begin{phonetics}{包}{bao1}
    \definition*{s.}{sobrenome Bao}
    \definition{clas.}{pacotes, sacos, sacolas, embrulhos}
    \definition[个,只]{s.}{bolsa | pacote | recipiente | embrulho}
    \definition{v.}{contratar | cobrir | segurar ou abraçar | incluir | assumir o comando | embrulhar}
  \end{phonetics}
\end{entry}

\begin{entry}{包子}{5,3}
  \begin{phonetics}{包子}{bao1zi5}
    \definition[个]{s.}{pão recheado cozido no vapor}
  \end{phonetics}
\end{entry}

\begin{entry}{包干}{5,3}
  \begin{phonetics}{包干}{bao1gan1}
    \definition{s.}{tarefa alocada}
    \definition{v.}{ter a responsabilidade total sobre um trabalho}
  \end{phonetics}
\end{entry}

\begin{entry}{包办}{5,4}
  \begin{phonetics}{包办}{bao1ban4}
    \definition{v.}{comandar todo o show | comprometer-se a fazer tudo sozinho}
  \end{phonetics}
\end{entry}

\begin{entry}{包括}{5,9}
  \begin{phonetics}{包括}{bao1kuo4}
    \definition{v.}{compreender | consistir em | incluir | incorporar | envolver}
  \end{phonetics}
\end{entry}

\begin{entry}{包容}{5,10}
  \begin{phonetics}{包容}{bao1rong2}
    \definition{adj.}{inclusivo}
    \definition{v.}{perdoar | mostrar tolerância | conter | segurar}
  \end{phonetics}
\end{entry}

\begin{entry}{包租}{5,10}
  \begin{phonetics}{包租}{bao1zu1}
    \definition{s.}{aluguel fixo para terras agrícolas}
    \definition{v.}{fretar | alugar | alugar um terreno ou uma casa para subarrendar}
  \end{phonetics}
\end{entry}

\begin{entry}{匆匆}{5,5}
  \begin{phonetics}{匆匆}{cong1cong1}
    \definition{adv.}{apressadamente}
  \end{phonetics}
\end{entry}

\begin{entry}{北}{5}[Radical 匕]
  \begin{phonetics}{北}{bei3}
    \definition{s.}{norte}
    \definition{v.}{(clássico) ser derrotado}
  \end{phonetics}
\end{entry}

\begin{entry}{北大西洋公约组织}{5,3,6,9,4,6,8,8}
  \begin{phonetics}{北大西洋公约组织}{bei3 da4xi1 yang2 gong1 yue1 zu3zhi1}
    \definition*{s.}{Organização do Tratado do Atlântico Norte, OTAN}
  \end{phonetics}
\end{entry}

\begin{entry}{北方}{5,4}
  \begin{phonetics}{北方}{bei3fang1}
    \definition{s.}{norte | a parte norte de um país}
  \end{phonetics}
\end{entry}

\begin{entry}{北边}{5,5}
  \begin{phonetics}{北边}{bei3bian1}
    \definition{adv.}{lado norte | ao norte de}
  \end{phonetics}
\end{entry}

\begin{entry}{北约}{5,6}
  \begin{phonetics}{北约}{bei3yue1}
    \definition*{s.}{OTAN (Organização do Tratado do Atlântico Norte), abreviação de 北大西洋公约组织}
    \seeref{北大西洋公约组织}{bei3 da4xi1 yang2 gong1 yue1 zu3zhi1}
  \end{phonetics}
\end{entry}

\begin{entry}{北极}{5,7}
  \begin{phonetics}{北极}{bei3ji2}
    \definition*{s.}{Ártico | Pólo Norte}
    \definition{s.}{pólo norte magnético}
  \end{phonetics}
\end{entry}

\begin{entry}{北京}{5,8}
  \begin{phonetics}{北京}{bei3jing1}
    \definition*{s.}{Beijing (Pequim), Capital da República Popular da China | Beijing (Pequim), governo da RPC}
  \end{phonetics}
\end{entry}

\begin{entry}{北面}{5,9}
  \begin{phonetics}{北面}{bei3mian4}
    \definition{s.}{lado norte}
  \end{phonetics}
\end{entry}

\begin{entry}{半}{5}[Radical 十]
  \begin{phonetics}{半}{ban4}
    \definition{adj.}{incompleto}
    \definition{num.}{(depois de um número) ``e meio''}
    \definition{pref.}{semi}
    \definition{s.}{metade}
  \end{phonetics}
\end{entry}

\begin{entry}{半音}{5,9}
  \begin{phonetics}{半音}{ban4yin1}
    \definition{s.}{semitom}
  \end{phonetics}
\end{entry}

\begin{entry}{半球}{5,11}
  \begin{phonetics}{半球}{ban4qiu2}
    \definition{s.}{hemisfério}
  \end{phonetics}
\end{entry}

\begin{entry}{卡片}{5,4}
  \begin{phonetics}{卡片}{ka3pian4}
    \definition{s.}{cartão}
  \end{phonetics}
\end{entry}

\begin{entry}{卡片游戏}{5,4,12,6}
  \begin{phonetics}{卡片游戏}{ka3pian4 you2xi4}
    \definition{s.}{carta de baralho}
  \end{phonetics}
\end{entry}

\begin{entry}{卡车司机}{5,4,5,6}
  \begin{phonetics}{卡车司机}{ka3che1 si1ji1}
    \definition{s.}{motorista de caminhão}
  \end{phonetics}
\end{entry}

\begin{entry}{卡通}{5,10}
  \begin{phonetics}{卡通}{ka3tong1}
    \definition{s.}{(empréstimo linguístico) \emph{cartoon}}
  \end{phonetics}
\end{entry}

\begin{entry}{卢旺达}{5,8,6}
  \begin{phonetics}{卢旺达}{lu2wang4da2}
    \definition*{s.}{Ruanda}
  \end{phonetics}
\end{entry}

\begin{entry}{厉害}{5,10}
  \begin{phonetics}{厉害}{li4hai5}
    \definition{adj.}{severo | rigoroso | exigente | radical | violento | feroz}
  \end{phonetics}
\end{entry}

\begin{entry}{厺}{5}
  \begin{phonetics}{厺}{qu4}
    \variantof{去}
  \end{phonetics}
\end{entry}

\begin{entry}{去}{5}[Radical 厶]
  \begin{phonetics}{去}{qu4}
    \definition{v.}{ir | (eufenismo) morrer}
  \end{phonetics}
\end{entry}

\begin{entry}{去年}{5,6}
  \begin{phonetics}{去年}{qu4nian2}
    \definition{s.}{ano passado}
  \end{phonetics}
\end{entry}

\begin{entry}{去死}{5,6}
  \begin{phonetics}{去死}{qu4si3}
    \definition{interj.}{Caia morto! | Vá para o Inferno!}
  \end{phonetics}
\end{entry}

\begin{entry}{发}{5}[Radical ⼜]
  \begin{phonetics}{发}{fa1}
    \definition{clas.}{para tiros (rodadas)}
    \definition{v.}{enviar | mandar}
  \end{phonetics}
  \begin{phonetics}{发}{fa4}
    \definition{s.}{cabelo}
  \end{phonetics}
\end{entry}

\begin{entry}{发生}{5,5}
  \begin{phonetics}{发生}{fa1sheng1}
    \definition{v.}{acontecer | ocorrer}
  \end{phonetics}
\end{entry}

\begin{entry}{发动机}{5,6,6}
  \begin{phonetics}{发动机}{fa1dong4ji1}
    \definition[台]{s.}{motor}
  \end{phonetics}
\end{entry}

\begin{entry}{发抖}{5,7}
  \begin{phonetics}{发抖}{fa1dou3}
    \definition{v.}{tremer | sacudir | estremecer}
  \end{phonetics}
\end{entry}

\begin{entry}{发财}{5,7}
  \begin{phonetics}{发财}{fa1cai2}
    \definition{v.+compl.}{ficar rico | fazer fortuna}
  \end{phonetics}
\end{entry}

\begin{entry}{发明者}{5,8,8}
  \begin{phonetics}{发明者}{fa1ming2zhe3}
    \definition{s.}{inventor}
  \end{phonetics}
\end{entry}

\begin{entry}{发现}{5,8}
  \begin{phonetics}{发现}{fa1xian4}
    \definition{s.}{descoberta}
    \definition{v.}{perceber, tornar-se ciente de | descobrir, encontrar, detectar}
  \end{phonetics}
\end{entry}

\begin{entry}{发现者}{5,8,8}
  \begin{phonetics}{发现者}{fa1xian4 zhe3}
    \definition{s.}{descobridor}
  \end{phonetics}
\end{entry}

\begin{entry}{发表}{5,8}
  \begin{phonetics}{发表}{fa1biao3}
    \definition{v.}{emitir | publicar}
  \end{phonetics}
\end{entry}

\begin{entry}{发型}{5,9}
  \begin{phonetics}{发型}{fa4xing2}
    \definition{s.}{penteado}
  \end{phonetics}
\end{entry}

\begin{entry}{发音}{5,9}
  \begin{phonetics}{发音}{fa1yin1}
    \definition{s.}{pronúncia}
    \definition{v.}{pronunciar}
  \end{phonetics}
\end{entry}

\begin{entry}{发展}{5,10}
  \begin{phonetics}{发展}{fa1zhan3}
    \definition{s.}{desenvolvimento}
    \definition{v.}{desenvolver}
  \end{phonetics}
\end{entry}

\begin{entry}{发烧}{5,10}
  \begin{phonetics}{发烧}{fa1shao1}
    \definition{v.}{ter febre}
  \end{phonetics}
\end{entry}

\begin{entry}{发票}{5,11}
  \begin{phonetics}{发票}{fa1piao4}
    \definition{s.}{fatura | recibo | conta}
  \end{phonetics}
\end{entry}

\begin{entry}{发愁}{5,13}
  \begin{phonetics}{发愁}{fa1chou2}
    \definition{v.+compl.}{preocupar-se | ficar ansioso | ficar triste}
  \end{phonetics}
\end{entry}

\begin{entry}{发簪}{5,18}
  \begin{phonetics}{发簪}{fa4zan1}
    \definition{s.}{grampo de cabelo}
  \end{phonetics}
\end{entry}

\begin{entry}{古}{5}[Radical ⼝]
  \begin{phonetics}{古}{gu3}
    \definition*{s.}{sobrenome Gu}
    \definition{adj.}{anciente | antigo | velho}
    \definition{pref.}{``paleo''}
  \end{phonetics}
\end{entry}

\begin{entry}{古人}{5,2}
  \begin{phonetics}{古人}{gu3ren2}
    \definition{s.}{pessoas dos tempos antigos | os antigos | espécies humanas extintas, como \emph{Homo erectus} ou \emph{Homo neanderthalensis} | (literário) pessoa falecida}
  \end{phonetics}
\end{entry}

\begin{entry}{古老}{5,6}
  \begin{phonetics}{古老}{gu3lao3}
    \definition{adj.}{ancestral | antigo | velho}
  \end{phonetics}
\end{entry}

\begin{entry}{古城}{5,9}
  \begin{phonetics}{古城}{gu3cheng2}
    \definition{s.}{cidade antiga}
  \end{phonetics}
\end{entry}

\begin{entry}{古铜色}{5,11,6}
  \begin{phonetics}{古铜色}{gu3tong2se4}
    \definition{s.}{bronze (cor)}
  \end{phonetics}
\end{entry}

\begin{entry}{句}{5}[Radical 口]
  \begin{phonetics}{句}{gou4}
    \variantof{勾}
  \end{phonetics}
  \begin{phonetics}{句}{ju4}
    \definition{clas.}{para orações, frases ou linhas de versos}
    \definition{s.}{sentença | cláusula | frase}
  \end{phonetics}
\end{entry}

\begin{entry}{句子}{5,3}
  \begin{phonetics}{句子}{ju4zi5}
    \definition[个]{s.}{sentença | frase | oração}
  \end{phonetics}
\end{entry}

\begin{entry}{另外}{5,5}
  \begin{phonetics}{另外}{ling4wai4}
    \definition{adv./pron.}{além disso}
  \end{phonetics}
\end{entry}

\begin{entry}{只}{5}[Radical 口]
  \begin{phonetics}{只}{zhi1}
    \definition{clas.}{para pássaros, gatos, cãezinhos, etc.}
  \end{phonetics}
  \begin{phonetics}{只}{zhi3}
    \definition{adv.}{apenas | só}
  \end{phonetics}
\end{entry}

\begin{entry}{只好}{5,6}
  \begin{phonetics}{只好}{zhi3hao3}
    \definition{adv.}{ser forçado a | ter que | sem nenhuma opção melhor | não ter outro remédio senão}
  \end{phonetics}
\end{entry}

\begin{entry}{只有……才……}{5,6,3}
  \begin{phonetics}{只有……才……}{zhi3you3 cai2}
    \definition{conj.}{só se\dots então\dots}
  \end{phonetics}
\end{entry}

\begin{entry}{只身}{5,7}
  \begin{phonetics}{只身}{zhi1shen1}
    \definition{adv.}{sozinho | por si só}
  \end{phonetics}
\end{entry}

\begin{entry}{只怕}{5,8}
  \begin{phonetics}{只怕}{zhi3pa4}
    \definition{adv.}{receio que\dots | talvez | muito provavelmente}
  \end{phonetics}
\end{entry}

\begin{entry}{只要}{5,9}
  \begin{phonetics}{只要}{zhi3yao4}
    \definition{conj.}{se apenas | contanto que}
  \end{phonetics}
\end{entry}

\begin{entry}{只要……就……}{5,9,12}
  \begin{phonetics}{只要……就……}{zhi3yao4 jiu4}
    \definition{conj.}{contanto que/desde que/se somente\dots, então\dots}
  \end{phonetics}
\end{entry}

\begin{entry}{只消}{5,10}
  \begin{phonetics}{只消}{zhi3xiao1}
    \definition{conj.}{desde que}
  \end{phonetics}
\end{entry}

\begin{entry}{只读}{5,10}
  \begin{phonetics}{只读}{zhi3du2}
    \definition{s.}{somente leitura (computação) | \emph{read-only}}
  \end{phonetics}
\end{entry}

\begin{entry}{只顾}{5,10}
  \begin{phonetics}{只顾}{zhi3gu4}
    \definition{adv.}{exclusivamente preocupado (com uma coisa)}
    \definition{v.}{cuidar de apenas um aspecto}
  \end{phonetics}
\end{entry}

\begin{entry}{只得}{5,11}
  \begin{phonetics}{只得}{zhi3de5}
    \definition{v.}{ser obrigado a | não ter outra alternativa senão}
  \end{phonetics}
\end{entry}

\begin{entry}{叫}{5}[Radical 口]
  \begin{phonetics}{叫}{jiao4}
    \definition{v.}{ser chamado | chamar-se | pedir | por (indica agente na voz passiva) | chamar | gritar | ordenar}
  \end{phonetics}
\end{entry}

\begin{entry}{叮嘱}{5,15}
  \begin{phonetics}{叮嘱}{ding1zhu3}
    \definition{v.}{exortar | avisar | insistir de novo e de novo}
  \end{phonetics}
\end{entry}

\begin{entry}{可}{5}[Radical 口]
  \begin{phonetics}{可}{ke3}
    \definition{adv.}{muito | realmente}
  \end{phonetics}
\end{entry}

\begin{entry}{可口可乐}{5,3,5,5}
  \begin{phonetics}{可口可乐}{ke3kou3ke3le4}
    \definition*{s.}{(empréstimo linguístico) Coca-Cola}
  \end{phonetics}
\end{entry}

\begin{entry}{可以}{5,4}
  \begin{phonetics}{可以}{ke3yi3}
    \definition{v.}{ser capaz de | poder}
  \end{phonetics}
\end{entry}

\begin{entry}{可卡因}{5,5,6}
  \begin{phonetics}{可卡因}{ke3ka3yin1}
    \definition{s.}{(empréstimo linguístico) cocaína}
  \end{phonetics}
\end{entry}

\begin{entry}{可怕}{5,8}
  \begin{phonetics}{可怕}{ke3pa4}
    \definition{adj.}{horrível | terrível | formidável | assustador | hediondo}
    \definition{adv.}{terrivelmente}
  \end{phonetics}
\end{entry}

\begin{entry}{可是}{5,9}
  \begin{phonetics}{可是}{ke3shi4}
    \definition{adv.}{(usado para dar ênfase) de fato}
    \definition{conj.}{porém | contudo | mas}
  \end{phonetics}
\end{entry}

\begin{entry}{可爱}{5,10}
  \begin{phonetics}{可爱}{ke3'ai4}
    \definition{adj.}{adorável | querido | fofo}
  \end{phonetics}
\end{entry}

\begin{entry}{可能}{5,10}
  \begin{phonetics}{可能}{ke3neng2}
    \definition{adj.}{possível | provável}
    \definition{adv.}{possivelmente | provavelmente}
    \definition[个]{s.}{possibilidade | probabilidade}
  \end{phonetics}
\end{entry}

\begin{entry}{可惜}{5,11}
  \begin{phonetics}{可惜}{ke3xi1}
    \definition{adj.}{é uma pena | que pena}
    \definition{adv.}{infelizmente | que pena | é uma pena}
  \end{phonetics}
\end{entry}

\begin{entry}{可编程}{5,12,12}
  \begin{phonetics}{可编程}{ke3bian1cheng2}
    \definition{adj.}{programável}
  \end{phonetics}
\end{entry}

\begin{entry*}{可擦写可编程只读存储器}{5,17,5,5,12,12,5,10,6,12,16}
  \begin{phonetics}{可擦写可编程只读存储器}{ke3ca1xie3ke3bian1cheng2zhi1du2cun2chu3qi4}
    \definition{s.}{EPROM (\emph{erasable programmable read-only memory})}
  \end{phonetics}
\end{entry*}

\begin{entry}{台}{5}[Radical 口]
  \begin{phonetics}{台}{tai2}
    \definition*{s.}{sobrenome Tai}
    \definition{clas.}{para aparelhos e máquinas}
    \definition{s.}{estação de transmissão | contador | \emph{help desk} | suporte técnico | plataforma | terraço | tufão}
  \end{phonetics}
\end{entry}

\begin{entry}{台下}{5,3}
  \begin{phonetics}{台下}{tai2xia4}
    \definition{s.}{platéia | fora do palco}
  \end{phonetics}
\end{entry}

\begin{entry}{台风}{5,4}
  \begin{phonetics}{台风}{tai2feng1}
    \definition{s.}{tufão}
  \end{phonetics}
\end{entry}

\begin{entry}{右}{5}[Radical 口]
  \begin{phonetics}{右}{you4}
    \definition{s.}{(política) a Direita}
    \definition{s.}{direita}
  \end{phonetics}
\end{entry}

\begin{entry}{右手}{5,4}
  \begin{phonetics}{右手}{you4shou3}
    \definition{s.}{mão direita | lado direito}
  \end{phonetics}
\end{entry}

\begin{entry}{右边}{5,5}
  \begin{phonetics}{右边}{you4bian5}
    \definition{adv.}{à direita | ao lado direito}
  \end{phonetics}
\end{entry}

\begin{entry}{右侧}{5,8}
  \begin{phonetics}{右侧}{you4ce4}
    \definition{s.}{lateral direita | lado direito}
  \end{phonetics}
\end{entry}

\begin{entry}{右转}{5,8}
  \begin{phonetics}{右转}{you4zhuan3}
    \definition{v.}{virar à direita}
  \end{phonetics}
\end{entry}

\begin{entry}{右面}{5,9}
  \begin{phonetics}{右面}{you4mian4}
    \definition{s.}{lado direito}
  \end{phonetics}
\end{entry}

\begin{entry}{右倾}{5,10}
  \begin{phonetics}{右倾}{you4qing1}
    \definition{adj.}{conservador | reacionário}
  \end{phonetics}
\end{entry}

\begin{entry}{右袒}{5,10}
  \begin{phonetics}{右袒}{you4tan3}
    \definition{v.}{ser tendencioso | ser parcial | favorecer um lado | tomar partido}
  \end{phonetics}
\end{entry}

\begin{entry}{号}{5}[Radical 口]
  \begin{phonetics}{号}{hao2}
    \definition[个]{s.}{rugido | choro}
  \end{phonetics}
  \begin{phonetics}{号}{hao4}
    \definition{clas.}{para indicar o número de pessoas}
    \definition{num.}{dia do mês | usado para indicar o número de pessoas}
    \definition[个]{s.}{número ordinal | dia de um mês | marca | sinal | estabelecimento comercial | tamanho | buzina (instrumento de sopro) | toque de corneta | nome suposto}
    \definition{suf.}{sufixo de navio}
    \definition{v.}{tomar um pulso}
  \end{phonetics}
\end{entry}

\begin{entry}{号角}{5,7}
  \begin{phonetics}{号角}{hao4jiao3}
    \definition{s.}{corneta | trombeta}
  \end{phonetics}
\end{entry}

\begin{entry}{号码}{5,8}
  \begin{phonetics}{号码}{hao4ma3}
    \definition[堆,个]{s.}{número}
  \end{phonetics}
\end{entry}

\begin{entry}{司机}{5,6}
  \begin{phonetics}{司机}{si1ji1}
    \definition{s.}{condutor | motorista | chofer}
  \end{phonetics}
\end{entry}

\begin{entry}{囘}{5}
  \begin{phonetics}{囘}{hui2}
    \variantof{回}
  \end{phonetics}
\end{entry}

\begin{entry}{四}{5}[Radical 囗]
  \begin{phonetics}{四}{si4}
    \definition{num.}{quatro; 4}
  \end{phonetics}
\end{entry}

\begin{entry}{四川}{5,3}
  \begin{phonetics}{四川}{si4chuan1}
    \definition*{s.}{Sichuan}
  \end{phonetics}
\end{entry}

\begin{entry}{四季分明}{5,8,4,8}
  \begin{phonetics}{四季分明}{si4ji4-fen1ming2}
    \definition{expr.}{as quatro estações são muito distintas}
  \end{phonetics}
\end{entry}

\begin{entry}{四季如春}{5,8,6,9}
  \begin{phonetics}{四季如春}{si4ji4-ru2chun1}
    \definition{expr.}{é primavera todo o ano | clima favorável durante todo o ano | quatro estações como a primavera}
  \end{phonetics}
\end{entry}

\begin{entry}{圣地}{5,6}
  \begin{phonetics}{圣地}{sheng4di4}
    \definition{s.}{terra santa (de uma religião) | lugar sagrado | santuário | cidade santa (como Jerusalém, Meca, etc.) | centro de interesse histórico}
  \end{phonetics}
\end{entry}

\begin{entry}{圣诞节}{5,8,5}
  \begin{phonetics}{圣诞节}{sheng4dan4jie2}
    \definition*{s.}{Natal}
  \end{phonetics}
\end{entry}

\begin{entry}{处}{5}[Radical ⼡]
  \begin{phonetics}{处}{chu3}
    \definition{v.}{residir | viver | habitar | estar dentro | estar situado em | ficar | se dar bem com | estar em uma posição de | lidar com | disciplinar | punir}
  \end{phonetics}
  \begin{phonetics}{处}{chu4}
    \definition{clas.}{para locais ou itens de danos: lugar, local}
    \definition{s.}{local | localização | lugar | ponto | escritório | departamento}
  \end{phonetics}
\end{entry}

\begin{entry}{处处}{5,5}
  \begin{phonetics}{处处}{chu4chu4}
    \definition{adv.}{em todos os lugares | em todos os aspectos}
  \end{phonetics}
\end{entry}

\begin{entry}{处罚}{5,9}
  \begin{phonetics}{处罚}{chu3fa2}
    \definition{v.}{penalizar | punir}
  \end{phonetics}
\end{entry}

\begin{entry}{外}{5}[Radical 夕]
  \begin{phonetics}{外}{wai4}
    \definition{s.}{fora | por fora | exterior | estrangeiro}
  \end{phonetics}
\end{entry}

\begin{entry}{外公}{5,4}
  \begin{phonetics}{外公}{wai4gong1}
    \definition{s.}{avô materno}
  \end{phonetics}
\end{entry}

\begin{entry}{外水}{5,4}
  \begin{phonetics}{外水}{wai4shui3}
    \definition{s.}{renda extra}
  \end{phonetics}
\end{entry}

\begin{entry}{外号}{5,5}
  \begin{phonetics}{外号}{wai4hao4}
    \definition{s.}{apelido}
  \end{phonetics}
\end{entry}

\begin{entry}{外边}{5,5}
  \begin{phonetics}{外边}{wai4bian5}
    \definition{adv.}{fora do país | superfície externa | fora | lugar diferente de sua casa}
  \end{phonetics}
\end{entry}

\begin{entry}{外交}{5,6}
  \begin{phonetics}{外交}{wai4jiao1}
    \definition{adj.}{diplomático}
    \definition[个]{s.}{diplomacia | relações exteriores}
  \end{phonetics}
\end{entry}

\begin{entry}{外协}{5,6}
  \begin{phonetics}{外协}{wai4xie2}
    \definition{s.}{terceirização | pessoas que julgam os outros pela aparência}
  \seealsoref{外貌协会}{wai4mao4xie2hui4}
  \end{phonetics}
\end{entry}

\begin{entry}{外孙}{5,6}
  \begin{phonetics}{外孙}{wai4sun1}
    \definition{s.}{filho da filha}
  \end{phonetics}
\end{entry}

\begin{entry}{外孙女}{5,6,3}
  \begin{phonetics}{外孙女}{wai4sun1nv3}
    \definition{s.}{filha da filha}
  \end{phonetics}
\end{entry}

\begin{entry}{外衣}{5,6}
  \begin{phonetics}{外衣}{wai4yi1}
    \definition{s.}{aparência | roupa de cima}
  \end{phonetics}
\end{entry}

\begin{entry}{外围}{5,7}
  \begin{phonetics}{外围}{wai4wei2}
    \definition{adv.}{arredores}
  \end{phonetics}
\end{entry}

\begin{entry}{外事}{5,8}
  \begin{phonetics}{外事}{wai4shi4}
    \definition{s.}{assuntos ou relações exteriores}
  \end{phonetics}
\end{entry}

\begin{entry}{外国}{5,8}
  \begin{phonetics}{外国}{wai4guo2}
    \definition[个]{s.}{país estrangeiro}
  \end{phonetics}
\end{entry}

\begin{entry}{外国人}{5,8,2}
  \begin{phonetics}{外国人}{wai4guo2ren2}
    \definition{s.}{estrangeiro | pessoa de fora do país}
  \end{phonetics}
\end{entry}

\begin{entry}{外语}{5,9}
  \begin{phonetics}{外语}{wai4yu3}
    \definition[门]{s.}{língua estrangeira}
  \end{phonetics}
\end{entry}

\begin{entry}{外贸}{5,9}
  \begin{phonetics}{外贸}{wai4mao4}
    \definition{s.}{comércio exterior}
  \end{phonetics}
\end{entry}

\begin{entry}{外面}{5,9}
  \begin{phonetics}{外面}{wai4mian4}
    \definition{adv.}{fora | por fora | exterior | superfície}
  \end{phonetics}
\end{entry}

\begin{entry}{外海}{5,10}
  \begin{phonetics}{外海}{wai4hai3}
    \definition{s.}{mar aberto}
  \end{phonetics}
\end{entry}

\begin{entry}{外积}{5,10}
  \begin{phonetics}{外积}{wai4ji1}
    \definition{s.}{produto exterior | (matemática) o produto vetorial de dois vetores}
  \end{phonetics}
\end{entry}

\begin{entry}{外婆}{5,11}
  \begin{phonetics}{外婆}{wai4po2}
    \definition{s.}{avó materna}
  \end{phonetics}
\end{entry}

\begin{entry}{外插}{5,12}
  \begin{phonetics}{外插}{wai4cha1}
    \definition{v.}{extrapolar | (computação) conectar (um dispositivo periférico, etc.)}
  \end{phonetics}
\end{entry}

\begin{entry}{外貌协会}{5,14,6,6}
  \begin{phonetics}{外貌协会}{wai4mao4xie2hui4}
    \definition{s.}{``o clube da boa aparência'': pessoas que dão grande importância à aparência de uma pessoa}
  \seealsoref{外协}{wai4xie2}
  \end{phonetics}
\end{entry}

\begin{entry}{失去}{5,5}
  \begin{phonetics}{失去}{shi1qu4}
    \definition{v.}{perder}
  \end{phonetics}
\end{entry}

\begin{entry}{失眠}{5,10}
  \begin{phonetics}{失眠}{shi1mian2}
    \definition{s.}{insônia}
    \definition{v.}{ter insônia}
  \end{phonetics}
\end{entry}

\begin{entry}{失望}{5,11}
  \begin{phonetics}{失望}{shi1wang4}
    \definition{adj.}{desapontado}
    \definition{v.}{perder a esperança | desesperar}
  \end{phonetics}
\end{entry}

\begin{entry}{失落}{5,12}
  \begin{phonetics}{失落}{shi1luo4}
    \definition{s.}{frustração | decepção | perda}
    \definition{v.}{perder (algo) | cair (algo) | sentir uma sensação de perda}
  \end{phonetics}
\end{entry}

\begin{entry}{失意}{5,13}
  \begin{phonetics}{失意}{shi1yi4}
    \definition{adj.}{desapontado | frustrado}
  \end{phonetics}
\end{entry}

\begin{entry}{头}{5}[Radical 大]
  \begin{phonetics}{头}{tou2}
    \definition{clas.}{para suínos ou gado}
    \definition[个]{s.}{cabeça}
  \end{phonetics}
  \begin{phonetics}{头}{tou5}
    \definition{suf.}{sufixo para nomes}
  \end{phonetics}
\end{entry}

\begin{entry}{头发}{5,5}
  \begin{phonetics}{头发}{tou2fa5}
    \definition{s.}{cabelo}
  \end{phonetics}
\end{entry}

\begin{entry}{头号}{5,5}
  \begin{phonetics}{头号}{tou2hao4}
    \definition{adj.}{primeira classe | número um | \emph{top rank}}
  \end{phonetics}
\end{entry}

\begin{entry}{头头}{5,5}
  \begin{phonetics}{头头}{tou2tou2}
    \definition{s.}{chefe | o cabeça}
  \end{phonetics}
\end{entry}

\begin{entry}{头脑风暴}{5,10,4,15}
  \begin{phonetics}{头脑风暴}{tou2nao3feng1bao4}
    \definition{s.}{\emph{brainstorm}}
  \end{phonetics}
\end{entry}

\begin{entry}{头像}{5,13}
  \begin{phonetics}{头像}{tou2xiang4}
    \definition{s.}{retrato | busto | avatar | imagem de perfil (computação)}
  \end{phonetics}
\end{entry}

\begin{entry}{奶奶}{5,5}
  \begin{phonetics}{奶奶}{nai3nai5}
    \definition[位]{s.}{avó (paterna) | (respeitoso) dona da casa}
  \end{phonetics}
\end{entry}

\begin{entry}{宁}{5}[Radical 宀]
  \begin{phonetics}{宁}{ning2}
    \definition*{s.}{sobrenome Ning}
    \definition{adj.}{calmo, pacífico, sereno | saudável}
  \end{phonetics}
  \begin{phonetics}{宁}{ning4}
    \definition{conj.}{mais\dots do que\dots, melhor\dots do que\dots}
  \end{phonetics}
\end{entry}

\begin{entry}{宁可}{5,5}
  \begin{phonetics}{宁可}{ning4ke3}
    \definition{conj.}{mais\dots do que\dots | melhor\dots do que\dots}
  \end{phonetics}
\end{entry}

\begin{entry}{宁可……也不……}{5,5,3,4}
  \begin{phonetics}{宁可……也不……}{ning4ke3 ye3bu4}
    \definition{conj.}{em vez de\dots}
  \end{phonetics}
\end{entry}

\begin{entry}{宁可……也要……}{5,5,3,9}
  \begin{phonetics}{宁可……也要……}{ning4ke3 ye3yao4}
    \definition{conj.}{mesmo que tenhamos que\dots nós iremos\dots}
  \end{phonetics}
\end{entry}

\begin{entry}{宁肯}{5,8}
  \begin{phonetics}{宁肯}{ning4ken3}
    \definition{conj.}{mais\dots do que\dots, melhor\dots do que\dots}
  \end{phonetics}
\end{entry}

\begin{entry}{宁愿}{5,14}
  \begin{phonetics}{宁愿}{ning4yuan4}
    \definition{conj.}{mais\dots do que\dots, melhor\dots do que\dots}
  \end{phonetics}
\end{entry}

\begin{entry}{它}{5}[Radical 宀]
  \begin{phonetics}{它}{ta1}
    \definition{pron.}{ele (para objetos inanimados) | se, o, lhe | si, consigo, eles}
  \end{phonetics}
\end{entry}

\begin{entry}{它们}{5,5}
  \begin{phonetics}{它们}{ta1men5}
    \definition{pron.}{eles (para objetos inanimados) | se, os, lhes | si, consigo, eles}
  \end{phonetics}
\end{entry}

\begin{entry}{对}{5}[Radical ⼨]
  \begin{phonetics}{对}{dui4}
    \definition{adj.}{correto | sim}
    \definition{clas.}{para casais}
    \definition{prep.}{com | para | para com}
  \end{phonetics}
\end{entry}

\begin{entry}{对不起}{5,4,10}
  \begin{phonetics}{对不起}{dui4bu5qi3}
    \definition{interj.}{Desculpe! | Desculpe-me! | Perdoe-me! | Desculpe? (por favor, repita)}
    \definition{v.}{desculpar | pedir desculpas | perdoar}
  \end{phonetics}
\end{entry}

\begin{entry}{对手}{5,4}
  \begin{phonetics}{对手}{dui4shou3}
    \definition{s.}{oponente | rival | concorrente | adversário}
  \end{phonetics}
\end{entry}

\begin{entry}{对……有兴趣}{5,6,6,15}
  \begin{phonetics}{对……有兴趣}{dui4 you3xing4qu4}
    \definition{expr.}{estar interessado em\dots | ter interesse em\dots | interessar-se por\dots}
    \seeref{对……感兴趣}{dui4 gan3xing4qu4}
  \end{phonetics}
\end{entry}

\begin{entry}{对话}{5,8}
  \begin{phonetics}{对话}{dui4hua4}
    \definition[个]{s.}{diálogo | conversa}
    \definition{v.}{dialogar | conversar}
  \end{phonetics}
\end{entry}

\begin{entry}{对……说}{5,9}
  \begin{phonetics}{对……说}{dui4 shuo5}
    \definition{v.}{dizer a alguém}
  \end{phonetics}
\end{entry}

\begin{entry}{对面}{5,9}
  \begin{phonetics}{对面}{dui4mian4}
    \definition{s.}{lado oposto}
  \end{phonetics}
\end{entry}

\begin{entry}{对得起}{5,11,10}
  \begin{phonetics}{对得起}{dui4de5qi3}
    \definition{v.}{não decepcionar alguém | tratar alguém de maneira justa | ser digno de}
  \end{phonetics}
\end{entry}

\begin{entry}{对……感兴趣}{5,13,6,15}
  \begin{phonetics}{对……感兴趣}{dui4 gan3xing4qu4}
    \definition{expr.}{estar interessado em\dots | ter interesse em\dots | interessar-se por\dots}
    \seeref{对……有兴趣}{dui4 you3xing4qu4}
  \end{phonetics}
\end{entry}

\begin{entry}{对……熟悉}{5,15,11}
  \begin{phonetics}{对……熟悉}{dui4 shu2xi1}
    \definition{expr.}{estar familiarizado com\dots}
  \end{phonetics}
\end{entry}

\begin{entry}{左}{5}[Radical 工]
  \begin{phonetics}{左}{zuo3}
    \definition*{s.}{sobrenome Zuo}
    \definition{p.l.}{esquerda}
  \end{phonetics}
\end{entry}

\begin{entry}{左右}{5,5}
  \begin{phonetics}{左右}{zuo3you4}
    \definition{adv.}{cerca de | aproximadamente}
  \end{phonetics}
\end{entry}

\begin{entry}{左边}{5,5}
  \begin{phonetics}{左边}{zuo3bian5}
    \definition{s.}{esquerda | lado esquerdo}
  \end{phonetics}
\end{entry}

\begin{entry}{左派}{5,9}
  \begin{phonetics}{左派}{zuo3pai4}
    \definition{s.}{(política) esquerda | esquerdista}
  \end{phonetics}
\end{entry}

\begin{entry}{左面}{5,9}
  \begin{phonetics}{左面}{zuo3mian4}
    \definition{s.}{esquerda | lado esquerdo}
  \end{phonetics}
\end{entry}

\begin{entry}{左倾}{5,10}
  \begin{phonetics}{左倾}{zuo3qing1}
    \definition{s.}{esquerdista | progressivo}
  \end{phonetics}
\end{entry}

\begin{entry}{左袒}{5,10}
  \begin{phonetics}{左袒}{zuo3tan3}
    \definition{v.}{ser tendencioso | ser parcial para | favorecer um lado | tomar partido com}
  \end{phonetics}
\end{entry}

\begin{entry}{左舷}{5,11}
  \begin{phonetics}{左舷}{zuo3xian2}
    \definition{s.}{porto (lado de um navio)}
  \end{phonetics}
\end{entry}

\begin{entry}{左翼}{5,17}
  \begin{phonetics}{左翼}{zuo3yi4}
    \definition{s.}{esquerda (política)}
  \end{phonetics}
\end{entry}

\begin{entry}{巧合}{5,6}
  \begin{phonetics}{巧合}{qiao3he2}
    \definition{s.}{coincidência}
    \definition{v.}{coincidir}
  \end{phonetics}
\end{entry}

\begin{entry}{巧克力}{5,7,2}
  \begin{phonetics}{巧克力}{qiao3ke4li4}
    \definition[块]{s.}{(empréstimo linguístico) chocolate}
  \end{phonetics}
\end{entry}

\begin{entry}{市中心}{5,4,4}
  \begin{phonetics}{市中心}{shi4zhong1xin1}
    \definition{s.}{centro da cidade}
  \end{phonetics}
\end{entry}

\begin{entry}{市区}{5,4}
  \begin{phonetics}{市区}{shi4qu1}
    \definition{s.}{centro da cidade | distrito urbano}
  \end{phonetics}
\end{entry}

\begin{entry}{市场}{5,6}
  \begin{phonetics}{市场}{shi4chang3}
    \definition{s.}{mercado (também no abstrato)}
  \end{phonetics}
\end{entry}

\begin{entry}{布}{5}[Radical 巾]
  \begin{phonetics}{布}{bu4}
    \definition{s.}{pano | tecido}
    \definition{v.}{declarar | anunciar | espalhar | fazer conhecer}
  \end{phonetics}
\end{entry}

\begin{entry}{布谷鸟}{5,7,5}
  \begin{phonetics}{布谷鸟}{bu4gu3niao3}
    \definition{s.}{cuco (pássaro)}
  \seealsoref{杜鹃}{du4juan1}
  \seealsoref{杜鹃鸟}{du4juan1niao3}
  \seealsoref{杜宇}{du4yu3}
  \end{phonetics}
\end{entry}

\begin{entry}{布署}{5,13}
  \begin{phonetics}{布署}{bu4shu3}
    \variantof{部署}
  \end{phonetics}
\end{entry}

\begin{entry}{帅}{5}[Radical 巾]
  \begin{phonetics}{帅}{shuai4}
    \definition*{s.}{sobrenome Shuai}
    \definition{adj.}{elegante | agradável à vista | gracioso | inteligente}
    \definition{interj.}{Legal!}
    \definition{s.}{comandante em chefe}
  \end{phonetics}
\end{entry}

\begin{entry}{平}{5}[Radical 干]
  \begin{phonetics}{平}{ping2}
    \definition*{s.}{sobrenome Ping}
    \definition{adj.}{calmo | pacífico}
    \definition{s.}{plano | nível}
    \definition{v.}{fazer a mesma pontuação | marcar uma pontuação}
  \end{phonetics}
\end{entry}

\begin{entry}{平台}{5,5}
  \begin{phonetics}{平台}{ping2tai2}
    \definition{s.}{plataforma | terraço | edifício de telhado plano}
  \end{phonetics}
\end{entry}

\begin{entry}{平地}{5,6}
  \begin{phonetics}{平地}{ping2di4}
    \definition{v.}{nivelar a terra | aplanar}
  \end{phonetics}
\end{entry}

\begin{entry}{平时}{5,7}
  \begin{phonetics}{平时}{ping2shi2}
    \definition{adv.}{normalmente | em tempos normais | em tempos de paz}
  \end{phonetics}
\end{entry}

\begin{entry}{幼儿园}{5,2,7}
  \begin{phonetics}{幼儿园}{you4'er2yuan2}
    \definition{s.}{jardim de infância | berçário}
  \end{phonetics}
\end{entry}

\begin{entry}{归}{5}[Radical ⼹]
  \begin{phonetics}{归}{gui1}
    \definition*{s.}{sobrenome Gui}
    \definition{s.}{divisão no ábaco com divisor de um dígito}
    \definition{v.}{retornar | voltar a | retribuir a | (uma responsabilidade) a ser resolvido por | pertencer | reunir-se | (usado entre dois verbos idênticos) apesar}
  \end{phonetics}
\end{entry}

\begin{entry}{必定}{5,8}
  \begin{phonetics}{必定}{bi4ding4}
    \definition{adv.}{sem falta | certamente | com certeza | definitivamente | inevitavelmente | com determinação}
    \definition{v.}{estar vinculado a | ter certeza de}
  \end{phonetics}
\end{entry}

\begin{entry}{必然}{5,12}
  \begin{phonetics}{必然}{bi4ran2}
    \definition{adv.}{sem falta | certamente | definitivamente | inevitavelmente}
  \end{phonetics}
\end{entry}

\begin{entry}{扑克}{5,7}
  \begin{phonetics}{扑克}{pu1ke4}
    \definition{s.}{(empréstimo linguístico) (jogo) \emph{poker}  | baralho}
  \end{phonetics}
\end{entry}

\begin{entry}{扒犁}{5,11}
  \begin{phonetics}{扒犁}{pa2li2}
    \definition{s.}{trenó}
    \seeref{爬犁}{pa2li2}
  \end{phonetics}
\end{entry}

\begin{entry}{打}{5}[Radical 手]
  \begin{phonetics}{打}{da2}
    \definition{s.}{(empréstimo linguístico) dúzia}
  \end{phonetics}
  \begin{phonetics}{打}{da3}
    \definition{adv.}{desde}
    \definition{v.}{jogar (um jogo) | bater | atacar | acertar | quebrar | digitar | misturar | construir | lutar | pegar | fazer | amarrar | atirar | calcular}
  \end{phonetics}
\end{entry}

\begin{entry}{打工}{5,3}
  \begin{phonetics}{打工}{da3gong1}
    \definition{v.}{(para alunos) ter um emprego fora do horário de aula ou durante as férias | trabalhar em um emprego temporá rio ou casual}
  \end{phonetics}
\end{entry}

\begin{entry}{打工人}{5,3,2}
  \begin{phonetics}{打工人}{da3gong1ren2}
    \definition{s.}{trabalhador}
  \end{phonetics}
\end{entry}

\begin{entry}{打电话}{5,5,8}
  \begin{phonetics}{打电话}{da3dian4hua4}
    \definition{v.}{telefonar | fazer uma chamada telefônica | dar um telefonema}
  \seealsoref{给……打电话}{gei3 da3dian4hua4}
  \end{phonetics}
\end{entry}

\begin{entry}{打压}{5,6}
  \begin{phonetics}{打压}{da3ya1}
    \definition{v.}{reprimir | derrotar}
  \end{phonetics}
\end{entry}

\begin{entry}{打屁股}{5,7,8}
  \begin{phonetics}{打屁股}{da3pi4gu5}
    \definition{v.}{dar um tapa no bumbum de alguém}
  \end{phonetics}
\end{entry}

\begin{entry}{打扮}{5,7}
  \begin{phonetics}{打扮}{da3ban5}
    \definition{v.}{arranjar-se | enfeitar-se}
  \end{phonetics}
\end{entry}

\begin{entry}{打扰}{5,7}
  \begin{phonetics}{打扰}{da3rao3}
    \definition{v.}{perturbar | incomodar}
  \end{phonetics}
\end{entry}

\begin{entry}{打针}{5,7}
  \begin{phonetics}{打针}{da3zhen1}
    \definition{v.+compl.}{dar injeção | levar injeção}
  \end{phonetics}
\end{entry}

\begin{entry}{打的}{5,8}
  \begin{phonetics}{打的}{da3di1}
    \definition{v.+compl.}{(coloquial) pegar um táxi | ir de táxi}
  \end{phonetics}
\end{entry}

\begin{entry}{打架}{5,9}
  \begin{phonetics}{打架}{da3jia4}
    \definition{v.+compl.}{lutar | brigar | participar de lutas, brigas}
  \end{phonetics}
\end{entry}

\begin{entry}{打结}{5,9}
  \begin{phonetics}{打结}{da3jie2}
    \definition{v.}{dar um nó | amarrar}
  \end{phonetics}
\end{entry}

\begin{entry}{打骂}{5,9}
  \begin{phonetics}{打骂}{da3ma4}
    \definition{v.}{bater e repreender}
  \end{phonetics}
\end{entry}

\begin{entry}{打猎}{5,11}
  \begin{phonetics}{打猎}{da3lie4}
    \definition{v.}{ir caçar}
  \end{phonetics}
\end{entry}

\begin{entry}{打球}{5,11}
  \begin{phonetics}{打球}{da3qiu2}
    \definition{v.}{jogar bola (com as mãos) | jogar (basquetebol, handbol, etc.)}
  \end{phonetics}
\end{entry}

\begin{entry}{打搅}{5,12}
  \begin{phonetics}{打搅}{da3jiao3}
    \definition{v.}{perturbar | incomodar}
  \end{phonetics}
\end{entry}

\begin{entry}{打算}{5,14}
  \begin{phonetics}{打算}{da3suan4}
    \definition[个]{s.}{plano | intenção}
    \definition{v.}{pensar | planejar | pretender}
  \end{phonetics}
\end{entry}

\begin{entry}{打瞌睡}{5,15,13}
  \begin{phonetics}{打瞌睡}{da3ke1shui4}
    \definition{v.}{cochilar}
  \end{phonetics}
\end{entry}

\begin{entry}{打磨}{5,16}
  \begin{phonetics}{打磨}{da3mo2}
    \definition{v.}{polir | fazer brilhar}
  \end{phonetics}
\end{entry}

\begin{entry}{扔}{5}[Radical 手]
  \begin{phonetics}{扔}{reng1}
    \definition{v.}{lançar | atirar}
  \end{phonetics}
\end{entry}

\begin{entry}{扔下}{5,3}
  \begin{phonetics}{扔下}{reng1xia4}
    \definition{v.}{lançar (uma bomba) | derrubar}
  \end{phonetics}
\end{entry}

\begin{entry}{扔弃}{5,7}
  \begin{phonetics}{扔弃}{reng1qi4}
    \definition{v.}{abandonar | descartar | jogar fora}
  \end{phonetics}
\end{entry}

\begin{entry}{扔掉}{5,11}
  \begin{phonetics}{扔掉}{reng1diao4}
    \definition{v.}{jogar fora}
  \end{phonetics}
\end{entry}

\begin{entry}{斥骂}{5,9}
  \begin{phonetics}{斥骂}{chi4ma4}
    \definition{v.}{repreender}
  \end{phonetics}
\end{entry}

\begin{entry}{旧}{5}[Radical 日]
  \begin{phonetics}{旧}{jiu4}
    \definition{adj.}{velho | antigo | desgastado (com a idade)}
  \end{phonetics}
\end{entry}

\begin{entry}{未}{5}[Radical 木]
  \begin{phonetics}{未}{wei4}
    \definition{adv.}{não ter | ainda não}
  \end{phonetics}
\end{entry}

\begin{entry}{未必}{5,5}
  \begin{phonetics}{未必}{wei4bi4}
    \definition{adv.}{não pode | não necessariamente}
  \end{phonetics}
\end{entry}

\begin{entry}{本}{5}[Radical 木]
  \begin{phonetics}{本}{ben3}
    \definition{adj.}{o atual | original | inerente}
    \definition{adv.}{originalmente}
    \definition{clas.}{para livros, dicionários, periódicos, arquivos, etc.}
    \definition{s.}{raiz | caule | origem | fonte}
  \end{phonetics}
\end{entry}

\begin{entry}{本子}{5,3}
  \begin{phonetics}{本子}{ben3zi5}
    \definition[本]{s.}{caderno}
  \end{phonetics}
\end{entry}

\begin{entry}{本来}{5,7}
  \begin{phonetics}{本来}{ben3lai2}
    \definition{adv.}{originalmente | apropriadamente | legalmente}
  \end{phonetics}
\end{entry}

\begin{entry}{正}{5}[Radical 止]
  \begin{phonetics}{正}{zheng1}
    \definition{s.}{primeiro mês do ano lunar}
  \end{phonetics}
  \begin{phonetics}{正}{zheng4}
    \definition{adj.}{reto | vertical | adequado | principal | (matemática) positivo}
    \definition{adv.}{agora mesmo | no processo de}
    \definition{v.}{corrigir | retificar}
  \end{phonetics}
\end{entry}

\begin{entry}{正正}{5,5}
  \begin{phonetics}{正正}{zheng4zheng4}
    \definition{adv.}{na hora certa | ordenadamente}
  \end{phonetics}
\end{entry}

\begin{entry}{正在}{5,6}
  \begin{phonetics}{正在}{zheng4zai4}
    \definition{adv.}{no processo de | atualmente | em andamento}
    \definition{v.}{estar a~+~v.inf. | estar~+~v.ger.}
  \end{phonetics}
\end{entry}

\begin{entry}{正宗}{5,8}
  \begin{phonetics}{正宗}{zheng4zong1}
    \definition{adj.}{autêntico | genuíno | \emph{old school} | (fig.) tradicional}
  \end{phonetics}
\end{entry}

\begin{entry}{正常}{5,11}
  \begin{phonetics}{正常}{zheng4chang2}
    \definition{adj.}{regular | normal | ordinário}
  \end{phonetics}
\end{entry}

\begin{entry}{母亲}{5,9}
  \begin{phonetics}{母亲}{mu3qin1}
    \definition[个]{s.}{mãe}
  \end{phonetics}
\end{entry}

\begin{entry}{母语}{5,9}
  \begin{phonetics}{母语}{mu3yu3}
    \definition{s.}{língua materna | língua nativa}
  \end{phonetics}
\end{entry}

\begin{entry}{民主}{5,5}
  \begin{phonetics}{民主}{min2zhu3}
    \definition{adj.}{democrático}
    \definition{s.}{democracia}
  \end{phonetics}
\end{entry}

\begin{entry}{民众}{5,6}
  \begin{phonetics}{民众}{min2zhong4}
    \definition{s.}{a população | as massas | as pessoas comuns}
  \end{phonetics}
\end{entry}

\begin{entry}{永不}{5,4}
  \begin{phonetics}{永不}{yong3bu4}
    \definition{adv.}{nunca}
  \end{phonetics}
\end{entry}

\begin{entry}{永远}{5,7}
  \begin{phonetics}{永远}{yong3yuan3}
    \definition{adv.}{para sempre, sempre | permanentemente}
  \end{phonetics}
\end{entry}

\begin{entry}{汉}{5}[Radical 水]
  \begin{phonetics}{汉}{han4}
    \definition{s.}{grupo étnico Han | chinês (língua) | dinastia Han (206 a.C.-220d.C.) | homem}
  \end{phonetics}
\end{entry}

\begin{entry}{汉字}{5,6}
  \begin{phonetics}{汉字}{han4zi4}
    \definition[个]{s.}{caracter chinês}
  \end{phonetics}
\end{entry}

\begin{entry}{汉服}{5,8}
  \begin{phonetics}{汉服}{han4fu2}
    \definition{s.}{vestido chinês tradicional Han}
  \end{phonetics}
\end{entry}

\begin{entry}{汉语}{5,9}
  \begin{phonetics}{汉语}{han4yu3}
    \definition[门]{s.}{língua chinesa, mandarim}
  \end{phonetics}
\end{entry}

\begin{entry}{汉堡王}{5,12,4}
  \begin{phonetics}{汉堡王}{han4bao3wang2}
    \definition*{s.}{Burguer King (restaurante \emph{fast-food})}
  \end{phonetics}
\end{entry}

\begin{entry}{汉堡包}{5,12,5}
  \begin{phonetics}{汉堡包}{han4bao3bao1}
    \definition[个]{s.}{hambúrguer}
  \end{phonetics}
\end{entry}

\begin{entry}{汉葡词典}{5,12,7,8}
  \begin{phonetics}{汉葡词典}{han4-pu2 ci2dian3}
    \definition[部,本]{s.}{dicionário chinês-português}
  \seealsoref{葡汉词典}{pu2-han4 ci2dian3}
  \end{phonetics}
\end{entry}

\begin{entry}{灭火}{5,4}
  \begin{phonetics}{灭火}{mie4huo3}
    \definition{s.}{combate a incêndios}
    \definition{v.}{extinguir um incêndio}
  \end{phonetics}
\end{entry}

\begin{entry}{犯法}{5,8}
  \begin{phonetics}{犯法}{fan4fa3}
    \definition{v.}{violar (quebrar) a lei}
  \end{phonetics}
\end{entry}

\begin{entry}{犯罪}{5,13}
  \begin{phonetics}{犯罪}{fan4zui4}
    \definition{v.+compl.}{cometer  um crime (uma ofensa)}
  \end{phonetics}
\end{entry}

\begin{entry}{玄学}{5,8}
  \begin{phonetics}{玄学}{xuan2xue2}
    \definition{s.}{Escola Philosófica Wei e Jin amalgamando os ideais daoísta e confucionistas | tradução da metafísica (形而上学)}
    \seeref{形而上学}{xing2'er2shang4xue2}
  \end{phonetics}
\end{entry}

\begin{entry}{玉}{5}[Radical 玉][Kangxi 96]
  \begin{phonetics}{玉}{yu4}
    \definition[块]{s.}{jade}
  \end{phonetics}
\end{entry}

\begin{entry}{玉米}{5,6}
  \begin{phonetics}{玉米}{yu4mi3}
    \definition[粒]{s.}{milho}
  \end{phonetics}
\end{entry}

\begin{entry}{玉米片}{5,6,4}
  \begin{phonetics}{玉米片}{yu4mi3pian4}
    \definition{s.}{flocos de milho | chips de tortilha}
  \end{phonetics}
\end{entry}

\begin{entry}{玉米花}{5,6,7}
  \begin{phonetics}{玉米花}{yu4mi3hua1}
    \definition{s.}{pipoca}
  \end{phonetics}
\end{entry}

\begin{entry}{玉米面}{5,6,9}
  \begin{phonetics}{玉米面}{yu4mi3mian4}
    \definition{s.}{fubá | farinha de milho}
  \end{phonetics}
\end{entry}

\begin{entry}{玉米饼}{5,6,9}
  \begin{phonetics}{玉米饼}{yu4mi3bing3}
    \definition{s.}{tortilha mexicana | bolo de milho}
  \end{phonetics}
\end{entry}

\begin{entry}{玉米笋}{5,6,10}
  \begin{phonetics}{玉米笋}{yu4mi3sun3}
    \definition{s.}{broto de milho}
  \end{phonetics}
\end{entry}

\begin{entry}{玉米粉}{5,6,10}
  \begin{phonetics}{玉米粉}{yu4mi3fen3}
    \definition{s.}{amido de milho | farinha de milho}
  \end{phonetics}
\end{entry}

\begin{entry}{玉米糁}{5,6,14}
  \begin{phonetics}{玉米糁}{yu4mi3san3}
    \definition{s.}{grãos de milho}
  \end{phonetics}
\end{entry}

\begin{entry}{玉米糕}{5,6,16}
  \begin{phonetics}{玉米糕}{yu4mi3gao1}
    \definition{s.}{bolo de milho | polenta}
  \end{phonetics}
\end{entry}

\begin{entry}{甘心}{5,4}
  \begin{phonetics}{甘心}{gan1xin1}
    \definition{v.}{estar disposto a | resignar-se a}
  \end{phonetics}
\end{entry}

\begin{entry}{甘薯}{5,16}
  \begin{phonetics}{甘薯}{gan1shu3}
    \definition{s.}{batata doce}
  \end{phonetics}
\end{entry}

\begin{entry}{生}{5}[Radical 生][Kangxi 100]
  \begin{phonetics}{生}{sheng1}
    \definition{adj.}{vida | estudante | cru | não cozido}
    \definition{v.}{nascer | dar a luz | crescer}
  \end{phonetics}
\end{entry}

\begin{entry}{生日}{5,4}
  \begin{phonetics}{生日}{sheng1ri4}
    \definition[个]{s.}{aniversário}
  \end{phonetics}
\end{entry}

\begin{entry}{生气}{5,4}
  \begin{phonetics}{生气}{sheng1qi4}
    \definition{s.}{vitalidade | vigor}
    \definition{v.+compl.}{irritar-se | zangar-se | ofender-se | ficar com raiva}
  \end{phonetics}
\end{entry}

\begin{entry}{生长}{5,4}
  \begin{phonetics}{生长}{sheng1zhang3}
    \definition{v.}{crescer | amadurecer | ser criado}
  \end{phonetics}
\end{entry}

\begin{entry}{生态}{5,8}
  \begin{phonetics}{生态}{sheng1tai4}
    \definition{adj.}{ecológico}
    \definition{s.}{ecologia}
  \end{phonetics}
\end{entry}

\begin{entry}{生物}{5,8}
  \begin{phonetics}{生物}{sheng1wu4}
    \definition{adj.}{biológico}
    \definition{s.}{biologia (disciplina) | organismo | ser vivo}
  \end{phonetics}
\end{entry}

\begin{entry}{生的}{5,8}
  \begin{phonetics}{生的}{sheng1de5}
    \definition{conj.}{para evitar isso | para que\dots não\dots}
  \end{phonetics}
\end{entry}

\begin{entry}{生鱼片}{5,8,4}
  \begin{phonetics}{生鱼片}{sheng1yu2pian4}
    \definition{s.}{fatias de peixe cru, \emph{sashimi}}
  \end{phonetics}
\end{entry}

\begin{entry}{生活}{5,9}
  \begin{phonetics}{生活}{sheng1huo2}
    \definition[道]{s.}{vida | atividade | meios de subsistência}
    \definition{v.}{viver}
  \end{phonetics}
\end{entry}

\begin{entry}{生活垃圾}{5,9,8,6}
  \begin{phonetics}{生活垃圾}{sheng1huo2la1ji1}
    \definition{s.}{lixo doméstico}
  \end{phonetics}
\end{entry}

\begin{entry}{生活型}{5,9,9}
  \begin{phonetics}{生活型}{sheng1huo2 xing2}
    \definition{s.}{forma de vida}
  \end{phonetics}
\end{entry}

\begin{entry}{生理}{5,11}
  \begin{phonetics}{生理}{sheng1li3}
    \definition{adj.}{fisiológico}
    \definition{s.}{fisiologia}
  \end{phonetics}
\end{entry}

\begin{entry}{生菜}{5,11}
  \begin{phonetics}{生菜}{sheng1cai4}
    \definition{s.}{alface}
  \end{phonetics}
\end{entry}

\begin{entry}{生意}{5,13}
  \begin{phonetics}{生意}{sheng1yi4}
    \definition{s.}{força vital | vitalidade}
  \end{phonetics}
  \begin{phonetics}{生意}{sheng1yi5}
    \definition{s.}{negócio}
  \end{phonetics}
\end{entry}

\begin{entry}{用}{5}[Radical 用][Kangxi 101]
  \begin{phonetics}{用}{yong4}
    \definition{v.}{usar}
  \end{phonetics}
\end{entry}

\begin{entry}{用心}{5,4}
  \begin{phonetics}{用心}{yong4xin1}
    \definition{s.}{motivo | intenção}
    \definition{v.+compl.}{ser diligente ou atencioso}
  \end{phonetics}
\end{entry}

\begin{entry}{用处}{5,5}
  \begin{phonetics}{用处}{yong4chu5}
    \definition[个]{s.}{usabilidade | utilidade}
  \end{phonetics}
\end{entry}

\begin{entry}{用料}{5,10}
  \begin{phonetics}{用料}{yong4liao4}
    \definition{s.}{ingredientes | materiais}
  \end{phonetics}
\end{entry}

\begin{entry}{田}{5}[Radical 田][Kangxi 102]
  \begin{phonetics}{田}{tian2}
    \definition*{s.}{sobrenome Tian}
    \definition[片]{s.}{fazenda | campo}
  \end{phonetics}
\end{entry}

\begin{entry}{田园}{5,7}
  \begin{phonetics}{田园}{tian2yuan2}
    \definition{adj.}{bucólico}
    \definition{s.}{campo | interior | rural}
  \end{phonetics}
\end{entry}

\begin{entry}{甲骨文}{5,9,4}
  \begin{phonetics}{甲骨文}{jia3gu3wen2}
    \definition{s.}{escrituras de oráculos | inscrições em ossos de oráculos (forma original de escritura chinesa)}
  \end{phonetics}
\end{entry}

\begin{entry}{电子}{5,3}
  \begin{phonetics}{电子}{dian4zi3}
    \definition{s.}{eletrônico | elétron}
  \end{phonetics}
\end{entry}

\begin{entry}{电子名片}{5,3,6,4}
  \begin{phonetics}{电子名片}{dian4zi3 ming2pian4}
    \definition{s.}{cartão de visita eletrônico}
  \end{phonetics}
\end{entry}

\begin{entry}{电子邮件}{5,3,7,6}
  \begin{phonetics}{电子邮件}{dian4zi3you2jian4}
    \definition[封,份]{s.}{correio eletrônico, \emph{e-mail}}
  \seealsoref{电邮}{dian4you2}
  \end{phonetics}
\end{entry}

\begin{entry}{电车司机}{5,4,5,6}
  \begin{phonetics}{电车司机}{dian4che1 si1ji1}
    \definition{s.}{motorista de bonde}
  \end{phonetics}
\end{entry}

\begin{entry}{电冰箱}{5,6,15}
  \begin{phonetics}{电冰箱}{dian4bing1xiang1}
    \definition[台]{s.}{frigorífico | refrigerador}
  \end{phonetics}
\end{entry}

\begin{entry}{电动}{5,6}
  \begin{phonetics}{电动}{dian4dong4}
    \definition{adj.}{movido a eletricidade | elétrico}
  \end{phonetics}
\end{entry}

\begin{entry}{电动车}{5,6,4}
  \begin{phonetics}{电动车}{dian4dong4che1}
    \definition{s.}{veículo elétrico (\emph{scooter}, bicicleta, carro, etc.)}
  \end{phonetics}
\end{entry}

\begin{entry}{电灯泡}{5,6,8}
  \begin{phonetics}{电灯泡}{dian4deng1pao4}
    \definition{s.}{lâmpada elétrica | (gíria) terceiro convidado indesejado}
  \end{phonetics}
\end{entry}

\begin{entry}{电邮}{5,7}
  \begin{phonetics}{电邮}{dian4you2}
    \definition{s.}{correio eletrônico, \emph{e-mail} | abreviação de~电子邮件}
  \seealsoref{电子邮件}{dian4zi3you2jian4}
  \end{phonetics}
\end{entry}

\begin{entry}{电视}{5,8}
  \begin{phonetics}{电视}{dian4shi4}
    \definition[台,个]{s.}{televisão | TV | televisor}
  \end{phonetics}
\end{entry}

\begin{entry}{电视机}{5,8,6}
  \begin{phonetics}{电视机}{dian4shi4ji1}
    \definition[台]{s.}{aparelho de televisão | televisor}
  \end{phonetics}
\end{entry}

\begin{entry}{电话}{5,8}
  \begin{phonetics}{电话}{dian4hua4}
    \definition[部]{s.}{telefone}
    \definition[通]{s.}{chamada telefônica}
  \end{phonetics}
\end{entry}

\begin{entry}{电脑}{5,10}
  \begin{phonetics}{电脑}{dian4nao3}
    \definition[台]{s.}{computador}
  \end{phonetics}
\end{entry}

\begin{entry}{电脑语言}{5,10,9,7}
  \begin{phonetics}{电脑语言}{dian4nao3yu3yan2}
    \definition{s.}{linguagem de programação | linguagem de computador}
  \end{phonetics}
\end{entry}

\begin{entry}{电梯}{5,11}
  \begin{phonetics}{电梯}{dian4ti1}
    \definition[台,部]{s.}{elevador | ascensor}
  \end{phonetics}
\end{entry}

\begin{entry}{电梯司机}{5,11,5,6}
  \begin{phonetics}{电梯司机}{dian4ti1 si1ji1}
    \definition{s.}{ascensorista}
  \end{phonetics}
\end{entry}

\begin{entry}{电影}{5,15}
  \begin{phonetics}{电影}{dian4ying3}
    \definition[部,片,幕,场]{s.}{filme}
  \end{phonetics}
\end{entry}

\begin{entry}{电影艺术}{5,15,4,5}
  \begin{phonetics}{电影艺术}{dian4ying3 yi4shu4}
    \definition{s.}{arte cinematográfica}
  \end{phonetics}
\end{entry}

\begin{entry}{电影术}{5,15,5}
  \begin{phonetics}{电影术}{dian4ying3 shu4}
    \definition{s.}{cinematografia}
  \end{phonetics}
\end{entry}

\begin{entry}{电影节}{5,15,5}
  \begin{phonetics}{电影节}{dian4ying3jie2}
    \definition{s.}{festival de cinema}
  \end{phonetics}
\end{entry}

\begin{entry}{电影奖}{5,15,9}
  \begin{phonetics}{电影奖}{dian4ying3jiang3}
    \definition{s.}{premiações de cinema}
  \end{phonetics}
\end{entry}

\begin{entry}{电影界}{5,15,9}
  \begin{phonetics}{电影界}{dian4ying3jie4}
    \definition{s.}{indústria cinematográfica}
  \end{phonetics}
\end{entry}

\begin{entry}{电影院}{5,15,9}
  \begin{phonetics}{电影院}{dian4ying3yuan4}
    \definition[次,家,座]{s.}{sala de cinema}
  \end{phonetics}
\end{entry}

\begin{entry}{电影音乐}{5,15,9,5}
  \begin{phonetics}{电影音乐}{dian4ying3 yin1yue4}
    \definition{s.}{música cinematográfica}
  \end{phonetics}
\end{entry}

\begin{entry}{电影票}{5,15,11}
  \begin{phonetics}{电影票}{dian4ying3piao4}
    \definition{s.}{ingresso de filme}
  \end{phonetics}
\end{entry}

\begin{entry}{电器}{5,16}
  \begin{phonetics}{电器}{dian4qi4}
    \definition{s.}{aparelho elétrico}
  \end{phonetics}
\end{entry}

\begin{entry}{白}{5}[Radical 白][Kangxi 106]
  \begin{phonetics}{白}{bai2}
    \definition*{s.}{sobrenome Bai}
    \definition{adj.}{branco | claro | puro | límpido | simples | em branco | grátis}
    \definition{adv.}{em vão | sem propósito | por nada}
    \definition{s.}{parte falada na ópera | diálogo | dialeto}
  \end{phonetics}
\end{entry}

\begin{entry}{白天}{5,4}
  \begin{phonetics}{白天}{bai2tian1}
    \definition{adv.}{dia | de dia}
    \definition[个]{s.}{dia}
  \end{phonetics}
\end{entry}

\begin{entry}{白色}{5,6}
  \begin{phonetics}{白色}{bai2se4}
    \definition{s.}{cor branca}
  \end{phonetics}
\end{entry}

\begin{entry}{白苋}{5,7}
  \begin{phonetics}{白苋}{bai2xian4}
    \definition{s.}{amaranto branco | brotos e folhas tenras de espinafre chinês usados como alimento}
  \end{phonetics}
\end{entry}

\begin{entry}{白拣}{5,8}
  \begin{phonetics}{白拣}{bai2jian3}
    \definition{s.}{uma escolha barata}
    \definition{v.}{escolher algo que não custa nada}
  \end{phonetics}
\end{entry}

\begin{entry}{白菜}{5,11}
  \begin{phonetics}{白菜}{bai2cai4}
    \definition[棵,个]{s.}{acelga | repolho chinês}
  \end{phonetics}
\end{entry}

\begin{entry}{白萝卜}{5,11,2}
  \begin{phonetics}{白萝卜}{bai2luo2bo5}
    \definition{s.}{rabanete branco}
  \end{phonetics}
\end{entry}

\begin{entry}{白蛋白}{5,11,5}
  \begin{phonetics}{白蛋白}{bai2dan4bai2}
    \definition{s.}{albumina}
  \end{phonetics}
\end{entry}

\begin{entry}{白鹄}{5,12}
  \begin{phonetics}{白鹄}{bai2hu2}
    \definition{s.}{cisne branco}
  \end{phonetics}
\end{entry}

\begin{entry}{白痴}{5,13}
  \begin{phonetics}{白痴}{bai2chi1}
    \definition{adj./s.}{estúpido | imbecil}
  \end{phonetics}
\end{entry}

\begin{entry}{皮}{5}[Radical 皮][Kangxi 107]
  \begin{phonetics}{皮}{pi2}
    \definition*{s.}{sobrenome Pi}
    \definition{adj.}{safado}
    \definition{pref.}{``pico'' (um trilhonésimo)}
    \definition[张]{s.}{couro | pele | pelagem}
  \end{phonetics}
\end{entry}

\begin{entry}{皮下}{5,3}
  \begin{phonetics}{皮下}{pi2xia4}
    \definition{adj.}{(injeção) subcutâneo | sob a pele}
  \end{phonetics}
\end{entry}

\begin{entry}{皮卡}{5,5}
  \begin{phonetics}{皮卡}{pi2ka3}
    \definition{s.}{(empréstimo linguístico) \emph{pick-up} | caminhonete}
  \end{phonetics}
\end{entry}

\begin{entry}{皮卡丘}{5,5,5}
  \begin{phonetics}{皮卡丘}{pi2ka3qiu1}
    \definition*{s.}{\emph{Pikachu} (Pokémon, 口袋妖怪)}
  \seealsoref{口袋妖怪}{kou3dai4 yao1guai4}
  \end{phonetics}
\end{entry}

\begin{entry}{皮肤}{5,8}
  \begin{phonetics}{皮肤}{pi2fu1}
    \definition[层,块]{s.}{pele}
  \end{phonetics}
\end{entry}

\begin{entry}{礼节}{5,5}
  \begin{phonetics}{礼节}{li3jie2}
    \definition{s.}{protocolo | cerimônia | etiqueta}
  \end{phonetics}
\end{entry}

\begin{entry}{礼让}{5,5}
  \begin{phonetics}{礼让}{li3rang4}
    \definition{s.}{cortesia}
    \definition{v.}{mostrar consideração por (outros) | ceder a (outro veículo, etc.)}
  \end{phonetics}
\end{entry}

\begin{entry}{礼物}{5,8}
  \begin{phonetics}{礼物}{li3wu4}
    \definition[件,个,份]{s.}{prenda | lembrança | presente}
  \end{phonetics}
\end{entry}

\begin{entry}{立刻}{5,8}
  \begin{phonetics}{立刻}{li4ke4}
    \definition{adv.}{imediatamente}
  \end{phonetics}
\end{entry}

\begin{entry}{立法}{5,8}
  \begin{phonetics}{立法}{li4fa3}
    \definition{s.}{legislação}
    \definition{v.}{promulgar leis | legislar}
  \end{phonetics}
\end{entry}

\begin{entry}{纠葛}{5,12}
  \begin{phonetics}{纠葛}{jiu1ge2}
    \definition{s.}{emaranhado | disputa}
  \end{phonetics}
\end{entry}

\begin{entry}{节日}{5,4}
  \begin{phonetics}{节日}{jie2ri4}
    \definition[个]{s.}{festival | feriado}
  \end{phonetics}
\end{entry}

\begin{entry}{节奏}{5,9}
  \begin{phonetics}{节奏}{jie2zou4}
    \definition{s.}{ritmo | cadência | batida}
  \end{phonetics}
\end{entry}

\begin{entry}{讨生活}{5,5,9}
  \begin{phonetics}{讨生活}{tao3sheng1huo2}
    \definition{v.}{ganhar a vida}
  \end{phonetics}
\end{entry}

\begin{entry}{让}{5}[Radical 言]
  \begin{phonetics}{让}{rang4}
    \definition{v.}{deixar alguém fazer alguma coisa |fazer alguém (sentir-se triste, etc.) | permitir | conceder}
  \end{phonetics}
\end{entry}

\begin{entry}{让步}{5,7}
  \begin{phonetics}{让步}{rang4bu4}
    \definition{v.+compl.}{fazer uma concessão | entregar | desistir | comprometer}
  \end{phonetics}
\end{entry}

\begin{entry}{记住}{5,7}
  \begin{phonetics}{记住}{ji4-zhu4}
    \definition{v.}{decorar | memorizar | ter em mente}
  \end{phonetics}
\end{entry}

\begin{entry}{记性}{5,8}
  \begin{phonetics}{记性}{ji4xing5}
    \definition{s.}{memória (habilidade em reter informações)}
  \end{phonetics}
\end{entry}

\begin{entry}{记得}{5,11}
  \begin{phonetics}{记得}{ji4de5}
    \definition{v.}{lembrar | lembrar-se}
  \end{phonetics}
\end{entry}

\begin{entry}{边}{5}[Radical 辵]
  \begin{phonetics}{边}{bian1}
    \definition{adv.}{simultaneamente}
    \definition[个]{s.}{fronteira | limite | borda | margem | lado}
  \end{phonetics}
  \begin{phonetics}{边}{bian5}
    \definition{suf.}{sufixo de uma palavra de localidade}
  \end{phonetics}
\end{entry}

\begin{entry}{边关}{5,6}
  \begin{phonetics}{边关}{bian1guan1}
    \definition{s.}{posto de fronteira | posição defensiva estratégica na fronteira}
  \end{phonetics}
\end{entry}

\begin{entry}{边防}{5,6}
  \begin{phonetics}{边防}{bian1fang2}
    \definition{s.}{defesa da fronteira}
  \end{phonetics}
\end{entry}

\begin{entry}{闪存盘}{5,6,11}
  \begin{phonetics}{闪存盘}{shan3cun2pan2}
    \definition{s.}{unidade de memória \emph{USB} | cartão de memória}
  \seealsoref{优盘}{you1pan2}
  \end{phonetics}
\end{entry}

\begin{entry}{鸟儿}{5,2}
  \begin{phonetics}{鸟儿}{niao3r5}
    \definition[只]{s.}{pássaro | ave}
  \end{phonetics}
\end{entry}

\begin{entry}{龙}{5}[Radical 龍]
  \begin{phonetics}{龙}{long2}
    \definition*{s.}{sobrenome Long}
    \definition{adj.}{imperial}
    \definition[条]{s.}{dragão chinês | (fig.) imperador | dragão | (forma ligada) dinossauro}
  \end{phonetics}
\end{entry}

\begin{entry}{龙山}{5,3}
  \begin{phonetics}{龙山}{long2shan1}
    \definition*{s.}{Longshan}
  \end{phonetics}
\end{entry}

\begin{entry}{龙虾}{5,9}
  \begin{phonetics}{龙虾}{long2xia1}
    \definition{s.}{lagosta}
  \end{phonetics}
\end{entry}

%%%%% EOF %%%%%


%%%
%%% 6画
%%%

\section*{6画}\addcontentsline{toc}{section}{6画}

\begin{entry}{丢}{6}{⼛}
  \begin{phonetics}{丢}{diu1}[][HSK 5]
    \definition{v.}{perder; extraviar; estar ausente | lançar; atirar | colocar (deixar) de lado | deixar (para trás)}
  \end{phonetics}
\end{entry}

\begin{entry}{丢下}{6,3}{⼛、⼀}
  \begin{phonetics}{丢下}{diu1xia4}
    \definition{v.}{abandonar}
  \end{phonetics}
\end{entry}

\begin{entry}{丢开}{6,4}{⼛、⼶}
  \begin{phonetics}{丢开}{diu1kai1}
    \definition{v.}{jogar fora ou deixar de lado | esquecer por um tempo}
  \end{phonetics}
\end{entry}

\begin{entry}{丢失}{6,5}{⼛、⼤}
  \begin{phonetics}{丢失}{diu1shi1}
    \definition{v.}{perder}
  \end{phonetics}
\end{entry}

\begin{entry}{丢弃}{6,7}{⼛、⼶}
  \begin{phonetics}{丢弃}{diu1qi4}
    \definition{v.}{jogar fora | descartar}
  \end{phonetics}
\end{entry}

\begin{entry}{丢官}{6,8}{⼛、⼧}
  \begin{phonetics}{丢官}{diu1guan1}
    \definition{v.}{perder um cargo oficial}
  \end{phonetics}
\end{entry}

\begin{entry}{丢掉}{6,11}{⼛、⼿}
  \begin{phonetics}{丢掉}{diu1diao4}
    \definition{v.}{jogar fora | descartar |perder}
  \end{phonetics}
\end{entry}

\begin{entry}{丢脸}{6,11}{⼛、⾁}
  \begin{phonetics}{丢脸}{diu1lian3}
    \definition{adj.}{vergonhoso}
  \end{phonetics}
\end{entry}

\begin{entry}{乒}{6}{⼃}
  \begin{phonetics}{乒}{ping1}
    \definition{interj.}{(onomatopéia) estalo; estouro; estrondo | (onomatopéia)  ``ping''}
    \definition{s.}{(abreviação) tênis de mesa; pingue-pongue | (abreviação) bola de tênis de mesa; bola de pingue-pongue}
  \end{phonetics}
\end{entry}

\begin{entry}{乒乓球}{6,6,11}{⼃、⼃、⽟}
  \begin{phonetics}{乒乓球}{ping1pang1qiu2}
    \definition[个]{s.}{tênis de mesa |ping-pong}
  \end{phonetics}
\end{entry}

\begin{entry}{乓}{6}{⼃}
  \begin{phonetics}{乓}{pang1}
    \definition{interj.}{(onomatopéia) barulho repentino feito por tiros, uma porta batendo, coisas quebrando, etc.; estrondo; estouro; batida; colisão}
  \end{phonetics}
\end{entry}

\begin{entry}{买}{6}{⼄}
  \begin{phonetics}{买}{mai3}[][HSK 1]
    \definition*{s.}{sobrenome Mai}
    \definition{v.}{comprar; adquirir | comprar; subornar; usar dinheiro ou outros meios para angariar apoio| pedir; obter; trocar dinheiro por coisas}
  \end{phonetics}
\end{entry}

\begin{entry}{买东西}{6,5,6}{⼄、⼀、⾑}
  \begin{phonetics}{买东西}{mai3dong1xi5}
    \definition{v.}{fazer compras}
  \end{phonetics}
\end{entry}

\begin{entry}{买卖}{6,8}{⼄、⼗}
  \begin{phonetics}{买卖}{mai3 mai4}[][HSK 5]
    \definition[种,笔]{s.}{negócio; compra e venda; transação | (privado) loja; armazém;}
  \end{phonetics}
\end{entry}

\begin{entry}{争}{6}{⼑}
  \begin{phonetics}{争}{zheng1}[][HSK 3]
    \definition*{s.}{sobrenome Zheng}
    \definition{adv.}{como; por que}
    \definition{v.}{competir; disputar; lutar; esforçar-se para obter ou alcançar | discutir; argumentar; contestar; debater | faltar; estar em falta}
  \end{phonetics}
\end{entry}

\begin{entry}{争风吃醋}{6,4,6,15}{⼑、⾵、⼝、⾣}
  \begin{phonetics}{争风吃醋}{zheng1feng1chi1cu4}
    \definition{v.}{rivalizar com alguém pelo carinho de um homem ou mulher |estar com ciúmes de um rival em um caso de amor}
  \end{phonetics}
\end{entry}

\begin{entry}{争议}{6,5}{⼑、⾔}
  \begin{phonetics}{争议}{zheng1yi4}[][HSK 5]
    \definition{s.}{disputa; controvérsia; situações e questões em que há divergências de opinião}
    \definition{v.}{debater; discutir}
  \end{phonetics}
\end{entry}

\begin{entry}{争先}{6,6}{⼑、⼉}
  \begin{phonetics}{争先}{zheng1xian1}
    \definition{v.}{competir para ser o primeiro |contestar o primeiro lugar}
  \end{phonetics}
\end{entry}

\begin{entry}{争论}{6,6}{⼑、⾔}
  \begin{phonetics}{争论}{zheng1lun4}[][HSK 4]
    \definition{s.}{debate; discussão; argumentação; disputa}
    \definition{v.}{discutir; disputar; debater; argumentar; contestar}
  \end{phonetics}
\end{entry}

\begin{entry}{争取}{6,8}{⼑、⼜}
  \begin{phonetics}{争取}{zheng1qu3}[][HSK 3]
    \definition{v.}{lutar por; conquistar; vencer; se esforçar para conseguir}
  \end{phonetics}
\end{entry}

\begin{entry}{争霸}{6,21}{⼑、⾬}
  \begin{phonetics}{争霸}{zheng1ba4}
    \definition{s.}{hegemonia | uma luta de poder}
    \definition{v.}{disputar a hegemonia}
  \end{phonetics}
\end{entry}

\begin{entry}{亚军}{6,6}{⼆、⼍}
  \begin{phonetics}{亚军}{ya4jun1}[][HSK 5]
    \definition{s.}{segundo lugar; vice-campeão; medalhista de prata}
  \end{phonetics}
\end{entry}

\begin{entry}{亚运会}{6,7,6}{⼆、⾡、⼈}
  \begin{phonetics}{亚运会}{ya4 yun4 hui4}[][HSK 4]
    \definition*{s.}{Jogos Asiáticos}
  \end{phonetics}
\end{entry}

\begin{entry}{亚细亚洲}{6,8,6,9}{⼆、⽷、⼆、⽔}
  \begin{phonetics}{亚细亚洲}{ya4xi4ya4zhou1}
    \definition*{s.}{Ásia}
  \end{phonetics}
\end{entry}

\begin{entry}{亚洲}{6,9}{⼆、⽔}
  \begin{phonetics}{亚洲}{ya4zhou1}
    \definition*{s.}{Ásia, abreviação de~亚细亚洲}
  \seealsoref{亚细亚洲}{ya4xi4ya4zhou1}
  \end{phonetics}
\end{entry}

\begin{entry}{亚洲人}{6,9,2}{⼆、⽔、⼈}
  \begin{phonetics}{亚洲人}{ya4zhou1ren2}
    \definition{s.}{asiático | pessoa ou povo da Ásia}
  \end{phonetics}
\end{entry}

\begin{entry}{交}{6}{⼇}
  \begin{phonetics}{交}{jiao1}[][HSK 2]
    \definition*{s.}{sobrenome Jiao}
    \definition{adv.}{mutuamente; recíprocamente; um ao outro | juntos; simultaneamente}
    \definition{s.}{amigo; conhecido; amizade; relacionamento | transação comercial; negócio; barganha | queda}
    \definition{v.}{entregar | (de lugares ou períodos de tempo) cruzar; encontrar; unir | chegar (a uma determinada hora ou estação); estabelecer-se; vir | cruzar; intersectar | associar-se a | ter relações sexuais | acasalar; reproduzir-se | transferir as coisas para as partes interessadas | unir (lugares ou períodos de tempo)}
  \end{phonetics}
\end{entry}

\begin{entry}{交叉}{6,3}{⼇、⼜}
  \begin{phonetics}{交叉}{jiao1cha1}
    \definition{v.}{cruzar | sobrepor}
  \end{phonetics}
\end{entry}

\begin{entry}{交叉口}{6,3,3}{⼇、⼜、⼝}
  \begin{phonetics}{交叉口}{jiao1cha1kou3}
    \definition{s.}{intersecção (rodovia)}
  \end{phonetics}
\end{entry}

\begin{entry}{交叉点}{6,3,9}{⼇、⼜、⽕}
  \begin{phonetics}{交叉点}{jiao1cha1dian3}
    \definition{s.}{encruzilhada | cruzamento | junção}
  \end{phonetics}
\end{entry}

\begin{entry}{交代}{6,5}{⼇、⼈}
  \begin{phonetics}{交代}{jiao1dai4}[][HSK 5]
    \definition{v.}{contar; entregar | ordenar; insistir; contar aos outros sobre suas intenções, instruções | contar; admitir}
  \end{phonetics}
\end{entry}

\begin{entry}{交运}{6,7}{⼇、⾡}
  \begin{phonetics}{交运}{jiao1yun4}
    \definition{v.}{despachar (bagagem em um aeroporto, etc.) | entregar para transporte}
  \end{phonetics}
\end{entry}

\begin{entry}{交际}{6,7}{⼇、⾩}
  \begin{phonetics}{交际}{jiao1ji4}[][HSK 4]
    \definition{s.}{contato; comunicação; relações sociais; contato interpessoal, socialização}
  \end{phonetics}
\end{entry}

\begin{entry}{交往}{6,8}{⼇、⼻}
  \begin{phonetics}{交往}{jiao1wang3}[][HSK 3]
    \definition{v.}{estar em contato com; associar-se a; interagir}
  \end{phonetics}
\end{entry}

\begin{entry}{交易}{6,8}{⼇、⽇}
  \begin{phonetics}{交易}{jiao1yi4}[][HSK 3]
    \definition[笔,桩,个,场]{s.}{negócio; comércio; transação comercial; transação; atividades de compra e venda de mercadorias}
    \definition{v.}{negociar; comprar e vender mercadorias}
  \end{phonetics}
\end{entry}

\begin{entry}{交朋友}{6,8,4}{⼇、⽉、⼜}
  \begin{phonetics}{交朋友}{jiao1 peng2 you3}[][HSK 2]
    \definition{v.}{fazer amizade com alguém; fazer amigos}
  \end{phonetics}
\end{entry}

\begin{entry}{交杯酒}{6,8,10}{⼇、⽊、⾣}
  \begin{phonetics}{交杯酒}{jiao1bei1jiu3}
    \definition{s.}{copo de vinho nupcial}
  \end{phonetics}
\end{entry}

\begin{entry}{交响}{6,9}{⼇、⼝}
  \begin{phonetics}{交响}{jiao1xiang3}
    \definition{s.}{sinfonia}
  \end{phonetics}
\end{entry}

\begin{entry}{交界}{6,9}{⼇、⽥}
  \begin{phonetics}{交界}{jiao1jie4}
    \definition{s.}{fronteira comum | limite comum | interface}
  \end{phonetics}
\end{entry}

\begin{entry}{交给}{6,9}{⼇、⽷}
  \begin{phonetics}{交给}{jiao1 gei3}[][HSK 2]
    \definition{v.}{entregar para | dar para}
  \end{phonetics}
\end{entry}

\begin{entry}{交费}{6,9}{⼇、⾙}
  \begin{phonetics}{交费}{jiao1 fei4}[][HSK 3]
    \definition{v.}{pagar taxas ou impostos; pagar uma taxa ou imposto}
  \end{phonetics}
\end{entry}

\begin{entry}{交换}{6,10}{⼇、⼿}
  \begin{phonetics}{交换}{jiao1huan4}[][HSK 4]
    \definition{v.}{trocar; permutar; comutar; intercambiar}
  \end{phonetics}
\end{entry}

\begin{entry}{交流}{6,10}{⼇、⽔}
  \begin{phonetics}{交流}{jiao1liu2}[][HSK 3]
    \definition{v.}{trocar; interagir; comunicar-se; compartilhar o que cada um tem com o outro}
  \end{phonetics}
\end{entry}

\begin{entry}{交班}{6,10}{⼇、⽟}
  \begin{phonetics}{交班}{jiao1ban1}
    \definition{v.}{passar para o próximo turno de trabalho}
  \end{phonetics}
\end{entry}

\begin{entry}{交通}{6,10}{⼇、⾡}
  \begin{phonetics}{交通}{jiao1tong1}[][HSK 2]
    \definition{s.}{tráfego | ligação; conexão | transporte; termo genérico para todos os tipos de transporte, como ferroviário e rodoviário}
    \definition{v.}{conspirar; fazer amizades; conchavar | estar conectado; estar ligado; estar vinculado | associar-se a; conspirar com}
  \end{phonetics}
\end{entry}

\begin{entry}{交通警察}{6,10,19,14}{⼇、⾡、⾔、⼧}
  \begin{phonetics}{交通警察}{jiao1tong1jing3cha2}
    \definition{s.}{policial de trânsito}
  \seealsoref{交警}{jiao1 jing3}
  \end{phonetics}
\end{entry}

\begin{entry}{交叠}{6,13}{⼇、⼜}
  \begin{phonetics}{交叠}{jiao1die2}
    \definition{s.}{sobreposição}
  \end{phonetics}
\end{entry}

\begin{entry}{交媾}{6,13}{⼇、⼥}
  \begin{phonetics}{交媾}{jiao1gou4}
    \definition{v.}{copular | ter relações sexuais}
  \end{phonetics}
\end{entry}

\begin{entry}{交警}{6,19}{⼇、⾔}
  \begin{phonetics}{交警}{jiao1 jing3}[][HSK 3]
    \definition{s.}{policial de trânsito, abreviação de 交通警察}
  \seealsoref{交通警察}{jiao1tong1jing3cha2}
  \end{phonetics}
\end{entry}

\begin{entry}{亦}{6}{⼇}
  \begin{phonetics}{亦}{yi4}
    \definition{adv.}{também | igualmente | apenas | embora | já}
  \end{phonetics}
\end{entry}

\begin{entry}{产}{6}{⼇}
  \begin{phonetics}{产}{chan3}
    \definition*{s.}{sobrenome Chan}
    \definition{s.}{produto | propriedade; espólio | (abreviação) indústria}
    \definition{v.}{dar à luz; ser entregue a | produzir; render | separar um ser humano ou animal de sua mãe}
  \end{phonetics}
\end{entry}

\begin{entry}{产业}{6,5}{⼇、⼀}
  \begin{phonetics}{产业}{chan3ye4}[][HSK 5]
    \definition{s.}{patrimônio; propriedade; bens pessoais, como terrenos, casas, fábricas, etc. | indústria; refere-se especificamente à produção industrial moderna | setor; indústria; indústrias e setores da economia nacional}
  \end{phonetics}
\end{entry}

\begin{entry}{产生}{6,5}{⼇、⽣}
  \begin{phonetics}{产生}{chan3sheng1}[][HSK 3]
    \definition{v.}{produzir; evoluir; emergir; provocar; vir a ser; dar origem a; criar coisas novas e novos fenômenos a partir do que já existe}
  \end{phonetics}
\end{entry}

\begin{entry}{产后}{6,6}{⼇、⼝}
  \begin{phonetics}{产后}{chan3hou4}
    \definition{s.}{pós-parto}
  \end{phonetics}
\end{entry}

\begin{entry}{产品}{6,9}{⼇、⼝}
  \begin{phonetics}{产品}{chan3pin3}[][HSK 4]
    \definition[个,件,种,批,项,类]{s.}{produto; item produzido}
  \end{phonetics}
\end{entry}

\begin{entry}{件}{6}{⼈}
  \begin{phonetics}{件}{jian4}[][HSK 2]
    \definition*{s.}{sobrenome Jian}
    \definition{clas.}{item; peça; artigo; usado para coisas individuais}
    \definition{s.}{refere-se a coisas que podem ser contadas uma a uma | papel; carta; documento; correspondência}
  \end{phonetics}
\end{entry}

\begin{entry}{价}{6}{⼈}
  \begin{phonetics}{价}{jia4}[][HSK 5]
    \definition{s.}{preço | valor; (figurativo) valores (éticos, culturais etc.) | (química) valência}
  \end{phonetics}
\end{entry}

\begin{entry}{价值}{6,10}{⼈、⼈}
  \begin{phonetics}{价值}{jia4zhi2}[][HSK 3]
    \definition{s.}{valor; o trabalho social necessário condensado nos produtos | valor; importância; efeitos positivos}
  \end{phonetics}
\end{entry}

\begin{entry}{价格}{6,10}{⼈、⽊}
  \begin{phonetics}{价格}{jia4ge2}[][HSK 3]
    \definition[个,种]{s.}{preço; tarifa; o valor monetário da mercadoria}
  \end{phonetics}
\end{entry}

\begin{entry}{价钱}{6,10}{⼈、⾦}
  \begin{phonetics}{价钱}{jia4 qian2}[][HSK 3]
    \definition[个,种,笔]{s.}{preço}
  \end{phonetics}
\end{entry}

\begin{entry}{任}{6}{⼈}
  \begin{phonetics}{任}{ren4}[][HSK 3]
    \definition{clas.}{usado para o número de mandatos cumpridos em um cargo oficial}
    \definition{conj.}{não importa (como, o que, etc.); orações de conexão, ou usadas antes de pronomes interrogativos, para expressar incondicionalidade, equivalente a 不管 ou 无论}
    \definition{s.}{escritório; posto oficial; cargo | dever; fardo; responsabilidade}
    \definition{v.}{nomear; designar alguém para um cargo | assumir um emprego; assumir um posto; assumir uma posição | deixar; permitir; dar rédea solta a | suportar; empreender | ceder; permitir sem restrições; deixar (alguém) fazer o que quiser}
  \seealsoref{不管}{bu4guan3}
  \seealsoref{无论}{wu2lun4}
  \end{phonetics}
\end{entry}

\begin{entry}{任务}{6,5}{⼈、⼒}
  \begin{phonetics}{任务}{ren4wu5}[][HSK 3]
    \definition[项,个,种,些]{s.}{tarefa; dever; missão; designação; trabalho designado; responsabilidades designadas}
  \end{phonetics}
\end{entry}

\begin{entry}{任何}{6,7}{⼈、⼈}
  \begin{phonetics}{任何}{ren4he2}[][HSK 3]
    \definition{pron.}{qualquer; qualquer que seja; o que for; não importa o que}
  \end{phonetics}
\end{entry}

\begin{entry}{份}{6}{⼈}
  \begin{phonetics}{份}{fen4}
    \definition{clas.}{usado para emparelhar itens em grupos | usado para jornais, documentos, etc. | usado para partes de um todo | usado para aparência, estado, etc.}
    \definition{s.}{porção; parte | a unidade de divisão; usado após 省, 县, 年, 月,  indica a unidade de divisão | grau; extensão de algo}
  \seealsoref{年}{nian2}
  \seealsoref{省}{sheng3}
  \seealsoref{县}{xian4}
  \seealsoref{月}{yue4}
  \end{phonetics}
\end{entry}

\begin{entry}{企}{6}{⼈}
  \begin{phonetics}{企}{qi3}
    \definition{v.}{ficar na ponta dos pés | esperar ansiosamente por algo; ansiar por | planejar um projeto}
  \end{phonetics}
\end{entry}

\begin{entry}{企业}{6,5}{⼈、⼀}
  \begin{phonetics}{企业}{qi3ye4}[][HSK 4]
    \definition[家,个]{s.}{empresa; estabelecimento; empreendimento; negócio; setores envolvidos em atividades econômicas como produção, transporte, comércio, etc., como fábricas, minas, ferrovias, empresas comerciais, etc.}
  \end{phonetics}
\end{entry}

\begin{entry}{伊马姆}{6,3,8}{⼈、⾺、⼥}
  \begin{phonetics}{伊马姆}{yi1ma3mu3}
    \definition*{s.}{Islã}
  \seealsoref{伊玛目}{yi1ma3mu4}
  \seealsoref{伊曼}{yi1man4}
  \seealsoref{伊斯兰}{yi1si1lan2}
  \end{phonetics}
\end{entry}

\begin{entry}{伊玛目}{6,7,5}{⼈、⽟、⽬}
  \begin{phonetics}{伊玛目}{yi1ma3mu4}
    \definition*{s.}{Islã}
  \seealsoref{伊马姆}{yi1ma3mu3}
  \seealsoref{伊曼}{yi1man4}
  \seealsoref{伊斯兰}{yi1si1lan2}
  \end{phonetics}
\end{entry}

\begin{entry}{伊朗}{6,10}{⼈、⽉}
  \begin{phonetics}{伊朗}{yi1lang3}
    \definition*{s.}{Irã}
  \end{phonetics}
\end{entry}

\begin{entry}{伊曼}{6,11}{⼈、⽈}
  \begin{phonetics}{伊曼}{yi1man4}
    \definition*{s.}{Islã}
  \seealsoref{伊马姆}{yi1ma3mu3}
  \seealsoref{伊玛目}{yi1ma3mu4}
  \seealsoref{伊斯兰}{yi1si1lan2}
  \end{phonetics}
\end{entry}

\begin{entry}{伊斯兰}{6,12,5}{⼈、⽄、⼋}
  \begin{phonetics}{伊斯兰}{yi1si1lan2}
    \definition*{s.}{Islã}
  \seealsoref{伊马姆}{yi1ma3mu3}
  \seealsoref{伊玛目}{yi1ma3mu4}
  \seealsoref{伊曼}{yi1man4}
  \end{phonetics}
\end{entry}

\begin{entry}{休兵}{6,7}{⼈、⼋}
  \begin{phonetics}{休兵}{xiu1bing1}
    \definition{s.}{armistício}
    \definition{v.}{cessar fogo}
  \end{phonetics}
\end{entry}

\begin{entry}{休闲}{6,7}{⼈、⾨}
  \begin{phonetics}{休闲}{xiu1xian2}[][HSK 5]
    \definition{s.}{ócio; lazer; tempo livre}
    \definition{v.}{desfrutar do lazer; sair de férias; aproveitar o tempo livre; parar de trabalhar ou estudar, estar em um estado de lazer e descontração | ficar ocioso}
  \end{phonetics}
\end{entry}

\begin{entry}{休息}{6,10}{⼈、⼼}
  \begin{phonetics}{休息}{xiu1xi5}[][HSK 1]
    \definition{s.}{descanço}
    \definition{v.}{descansar; descansar um pouco; fazer uma pausa; interromper o trabalho, os estudos ou as atividades para recuperar as energias | dormir}
  \end{phonetics}
\end{entry}

\begin{entry}{休息室}{6,10,9}{⼈、⼼、⼧}
  \begin{phonetics}{休息室}{xiu1xi1shi4}
    \definition{s.}{saguão | salão}
  \end{phonetics}
\end{entry}

\begin{entry}{休假}{6,11}{⼈、⼈}
  \begin{phonetics}{休假}{xiu1 jia4}[][HSK 2]
    \definition{v.+compl.}{ter um feriado; tirar férias; sair de férias}
  \end{phonetics}
\end{entry}

\begin{entry}{休憩}{6,16}{⼈、⼼}
  \begin{phonetics}{休憩}{xiu1qi4}
    \definition{v.}{relaxar | descansar | dar um tempo}
  \end{phonetics}
\end{entry}

\begin{entry}{休整}{6,16}{⼈、⽁}
  \begin{phonetics}{休整}{xiu1zheng3}
    \definition{v.}{(militar) descansar e reorganizar}
  \end{phonetics}
\end{entry}

\begin{entry}{众}{6}{⼈}
  \begin{phonetics}{众}{zhong4}
    \definition*{s.}{Câmara dos Deputados, abreviação de 众议院}
    \definition{adj.}{numeroso}
    \definition{adv.}{muitos}
    \definition{s.}{multidão}
  \seealsoref{众议院}{zhong4yi4yuan4}
  \end{phonetics}
\end{entry}

\begin{entry}{众议院}{6,5,9}{⼈、⾔、⾩}
  \begin{phonetics}{众议院}{zhong4yi4yuan4}
    \definition*{s.}{Casa baixa da Assembléia Bicameral | Câmara dos Deputados}
  \end{phonetics}
\end{entry}

\begin{entry}{众多}{6,6}{⼈、⼣}
  \begin{phonetics}{众多}{zhong4 duo1}[][HSK 5]
    \definition{adj.}{muitos; numerosos; multitudinários}
  \end{phonetics}
\end{entry}

\begin{entry}{优}{6}{⼈}
  \begin{phonetics}{优}{you1}
    \definition{adj.}{excelente | superior}
  \end{phonetics}
\end{entry}

\begin{entry}{优于}{6,3}{⼈、⼆}
  \begin{phonetics}{优于}{you1yu2}
    \definition{v.}{superar}
  \end{phonetics}
\end{entry}

\begin{entry}{优先}{6,6}{⼈、⼉}
  \begin{phonetics}{优先}{you1xian1}[][HSK 5]
    \definition{adj.}{anterior; sênior; subjacente}
    \definition{v.}{ter prioridade; ter precedência; colocar-se à frente de outras pessoas ou assuntos}
  \end{phonetics}
\end{entry}

\begin{entry}{优伶}{6,7}{⼈、⼈}
  \begin{phonetics}{优伶}{you1ling2}
    \definition{s.}{ator}
  \end{phonetics}
\end{entry}

\begin{entry}{优秀}{6,7}{⼈、⽲}
  \begin{phonetics}{优秀}{you1xiu4}[][HSK 4]
    \definition{adj.}{esplêndido; excelente; extraordinário; excepcional; notável; descreve moral, qualidades, realizações, aprendizado, etc. muito bons.}
  \end{phonetics}
\end{entry}

\begin{entry}{优良}{6,7}{⼈、⾉}
  \begin{phonetics}{优良}{you1 liang2}[][HSK 4]
    \definition{adj.}{ótimo; bom; excelente; (variedade, qualidade, desempenho, estilo, etc.) muito bom}
  \end{phonetics}
\end{entry}

\begin{entry}{优势}{6,8}{⼈、⼒}
  \begin{phonetics}{优势}{you1shi4}[][HSK 3]
    \definition[种,个]{s.}{vantagem; superioridade; preponderância; posição dominante; uma situação favorável que permite superar o adversário}
  \end{phonetics}
\end{entry}

\begin{entry}{优质}{6,8}{⼈、⾙}
  \begin{phonetics}{优质}{you1zhi4}
    \definition{adj.}{excelente qualidade}
  \end{phonetics}
\end{entry}

\begin{entry}{优厚}{6,9}{⼈、⼚}
  \begin{phonetics}{优厚}{you1hou4}
    \definition{adj.}{generoso}
  \end{phonetics}
\end{entry}

\begin{entry}{优点}{6,9}{⼈、⽕}
  \begin{phonetics}{优点}{you1dian3}[][HSK 3]
    \definition[个,项,种,些]{s.}{mérito; virtude; ponto forte; vantagem (em oposição a 缺点)}
  \seealsoref{缺点}{que1dian3}
  \end{phonetics}
\end{entry}

\begin{entry}{优美}{6,9}{⼈、⽺}
  \begin{phonetics}{优美}{you1mei3}[][HSK 4]
    \definition{adj.}{fino; elegante; gracioso; bonito}
  \end{phonetics}
\end{entry}

\begin{entry}{优选}{6,9}{⼈、⾡}
  \begin{phonetics}{优选}{you1xuan3}
    \definition{v.}{otimizar}
  \end{phonetics}
\end{entry}

\begin{entry}{优格}{6,10}{⼈、⽊}
  \begin{phonetics}{优格}{you1ge2}
    \definition{s.}{iogurte}
  \end{phonetics}
\end{entry}

\begin{entry}{优盘}{6,11}{⼈、⽫}
  \begin{phonetics}{优盘}{you1pan2}
    \definition{s.}{unidade de memória USB}
  \seealsoref{闪存盘}{shan3cun2pan2}
  \end{phonetics}
\end{entry}

\begin{entry}{优惠}{6,12}{⼈、⼼}
  \begin{phonetics}{优惠}{you1hui4}[][HSK 5]
    \definition{adj.}{especial; pechincha; reduzido; com desconto | favorável; preferencial; melhores condições ou tratamento do que o normal, permitindo que as pessoas obtenham mais benefícios}
  \end{phonetics}
\end{entry}

\begin{entry}{优等}{6,12}{⼈、⽵}
  \begin{phonetics}{优等}{you1deng3}
    \definition{adj.}{excelente | de primeira linha | alta classe | da mais alta ordem, superior}
  \end{phonetics}
\end{entry}

\begin{entry}{优裕}{6,12}{⼈、⾐}
  \begin{phonetics}{优裕}{you1yu4}
    \definition{adj.}{abundante | bastante}
    \definition{s.}{abundância}
  \end{phonetics}
\end{entry}

\begin{entry}{伙}{6}{⼈}
  \begin{phonetics}{伙}{huo3}[][HSK 4]
    \definition{clas.}{grupo; multidão; banda}
    \definition{s.}{iguaria; alimentação; refeições | parceiro; companheiro | coletivo de colegas}
    \definition{v.}{combinar; unir}
  \end{phonetics}
\end{entry}

\begin{entry}{伙伴}{6,7}{⼈、⼈}
  \begin{phonetics}{伙伴}{huo3ban4}[][HSK 4]
    \definition[个,位,群]{s.}{parceiro; companheiro; antigo sistema militar de dez pessoas para uma fogueira, o chefe da fogueira, uma pessoa encarregada de cozinhar, com a fogueira é chamado de parceiro da fogueira, agora se refere à participação comum em uma determinada organização ou engajada em certas atividades}
  \end{phonetics}
\end{entry}

\begin{entry}{会}{6}{⼈}
  \begin{phonetics}{会}{hui4}[][HSK 1,2]
    \definition{adv.}{um momento}
    \definition{clas.}{momento; um curto período de tempo}
    \definition{s.}{reunião; festa; conferência; reunião com um objetivo específico | reunião; reunião no trabalho | feira do templo; festival religioso | associação; sociedade; sindicato; certas organizações públicas | oportunidade; ocasião; momento oportuno | cidade principal; capital; cidade central}
    \definition{suf.}{união; grupo; associação}
    \definition{v.}{ser provável que; ter certeza de; indica que é possível realizar (é possível responder à pergunta separadamente) |  poder; ser capaz de; significa saber como fazer ou ter a capacidade de fazer (geralmente se refere a coisas que precisam ser aprendidas) | saber; compreender; entender | encontrar; ver | reunir-se; reunir; agregar; juntar | destacar-se em; ser bom em; ser hábil em; indica proficiência | pagar (ou custear) contas}
  \end{phonetics}
  \begin{phonetics}{会}{kuai4}
    \definition[个,场,次]{s.}{contabilidade}
    \definition{v.}{computar; calcular; equilibrar uma conta}
  \end{phonetics}
\end{entry}

\begin{entry}{会计}{6,4}{⼈、⾔}
  \begin{phonetics}{会计}{kuai4ji4}[][HSK 4]
    \definition[个,位,名]{s.}{contabilidade | contador; contabilista; guarda-livros; pessoal que trabalha como contador}
  \end{phonetics}
\end{entry}

\begin{entry}{会议}{6,5}{⼈、⾔}
  \begin{phonetics}{会议}{hui4yi4}[][HSK 3]
    \definition[次,届,个,场]{s.}{reunião; conferência; reunião organizada pela organização relevante para ouvir opiniões, discutir questões e distribuir tarefas | conselho; congresso; um órgão ou organização permanente que discute e trata frequentemente assuntos importantes}
  \end{phonetics}
\end{entry}

\begin{entry}{会员}{6,7}{⼈、⼝}
  \begin{phonetics}{会员}{hui4 yuan2}[][HSK 3]
    \definition[位,名,个,些]{s.}{membro; associado; membros de certos grupos ou organizações}
  \end{phonetics}
\end{entry}

\begin{entry}{会首}{6,9}{⼈、⾸}
  \begin{phonetics}{会首}{hui4shou3}
    \definition{s.}{chefe de uma sociedade | patrocinador de uma organização}
  \end{phonetics}
\end{entry}

\begin{entry}{会谈}{6,10}{⼈、⾔}
  \begin{phonetics}{会谈}{hui4 tan2}[][HSK 5]
    \definition{v.}{manter conversações; comumente usado em assuntos internacionais ou atividades diplomáticas}
  \end{phonetics}
\end{entry}

\begin{entry}{伞}{6}{⼈}
  \begin{phonetics}{伞}{san3}[][HSK 4]
    \definition*{s.}{sobrenome San}
    \definition[把]{s.}{guarda-chuva; proteção contra chuva ou sol | algo que tem o formato de um guarda-chuva}
  \end{phonetics}
\end{entry}

\begin{entry}{伟}{6}{⼈}
  \begin{phonetics}{伟}{wei3}
    \definition{adj.}{grande | ótimo}
  \end{phonetics}
\end{entry}

\begin{entry}{伟大}{6,3}{⼈、⼤}
  \begin{phonetics}{伟大}{wei3da4}[][HSK 3]
    \definition{adj.}{ótimo; excelente; extrovertido; descreve uma pessoa com moral e qualidades excelentes, habilidades e realizações excepcionais, que inspira grande respeito | ótimo; magnífico; descreve algo de grande importância, com impacto significativo, acima do normal, algo notável}
  \end{phonetics}
\end{entry}

\begin{entry}{传}{6}{⼈}
  \begin{phonetics}{传}{chuan2}[][HSK 3]
    \definition{v.}{passar; passar adiante | passar adiante; legar; passar de\dots para\dots; passar da geração anterior para a seguinte | transmitir (conhecimento, habilidade, etc.); comunicar; ensinar | espalhar; propagar | transmitir; conduzir; transferir | transmitir; expressar | convocar; dar ordem para chamar alguém | infectar; ser contagioso | enviar documentos por e-mail ou fax}
  \end{phonetics}
  \begin{phonetics}{传}{zhuan4}
    \definition{s.}{comentários sobre clássicos; obras que explicam as escrituras| biografia | romances sobre eventos históricos; obras que narram histórias históricas}
  \end{phonetics}
\end{entry}

\begin{entry}{传达}{6,6}{⼈、⾡}
  \begin{phonetics}{传达}{chuan2da2}[][HSK 5]
    \definition{s.}{recepção e registro de chamadas em um estabelecimento público | zelador; recepcionista}
    \definition{v.}{passar adiante (informações, etc.); transmitir; retransmitir; comunicar}
  \end{phonetics}
\end{entry}

\begin{entry}{传来}{6,7}{⼈、⽊}
  \begin{phonetics}{传来}{chuan2 lai2}[][HSK 3]
    \definition{v.}{(um som) passar; transmitir de algum lugar para o local onde o locutor se encontra | (notícias) chegar; transmitir mensagens ou informações}
  \end{phonetics}
\end{entry}

\begin{entry}{传承}{6,8}{⼈、⼿}
  \begin{phonetics}{传承}{chuan2cheng2}
    \definition{s.}{herança | tradição continuada}
    \definition{v.}{transmitir (para as gerações futuras) | passar adiante (desde os tempos antigos)}
  \end{phonetics}
\end{entry}

\begin{entry}{传给}{6,9}{⼈、⽷}
  \begin{phonetics}{传给}{chuan2gei3}
    \definition{v.}{passar para | transferir para | entregar a}
  \end{phonetics}
\end{entry}

\begin{entry}{传统}{6,9}{⼈、⽷}
  \begin{phonetics}{传统}{chuan2tong3}[][HSK 4]
    \definition{adj.}{tradicional; histórico; transmitido de geração em geração | antiquado, conservador e fora de sintonia com os tempos}
    \definition[个]{s.}{tradição; costume; fatores sociais, como costumes, moral, ideias, estilos, artes, instituições etc., que são transmitidos de uma geração para outra e que são característicos da sociedade}
  \end{phonetics}
\end{entry}

\begin{entry}{传说}{6,9}{⼈、⾔}
  \begin{phonetics}{传说}{chuan2shuo1}[][HSK 3]
    \definition[个,种,段]{s.}{lenda; conto popular; folclore; coisas lendárias; especificamente, lendas populares}
    \definition{v.}{dizer que; ser dito; passar de boca em boca; transmitir oralmente, segundo a tradição}
  \end{phonetics}
\end{entry}

\begin{entry}{传真}{6,10}{⼈、⼗}
  \begin{phonetics}{传真}{chuan2zhen1}[][HSK 5]
    \definition[台,部,份]{s.}{\emph{FAX}, facsímile; texto, diagramas, fotografias, etc., transmitidos por aparelho de fax}
    \definition{v.}{enviar um fax}
  \end{phonetics}
\end{entry}

\begin{entry}{传递}{6,10}{⼈、⾡}
  \begin{phonetics}{传递}{chuan2 di4}[][HSK 5]
    \definition{v.}{transmitir; entregar; transferir; passar adiante}
  \end{phonetics}
\end{entry}

\begin{entry}{传播}{6,15}{⼈、⼿}
  \begin{phonetics}{传播}{chuan2bo1}[][HSK 3]
    \definition{v.}{espalhar; difundir; propagar; disseminar}
  \end{phonetics}
\end{entry}

\begin{entry}{伤}{6}{⼈}
  \begin{phonetics}{伤}{shang1}[][HSK 3]
    \definition*{s.}{sobrenome Shang}
    \definition[处]{s.}{ferida; ferimento}
    \definition{v.}{ferir; machucar | ter os sentimentos feridos | estar angustiado | enjoar de algo; desenvolver aversão a algo | ser prejudicial a; entravar}
  \end{phonetics}
\end{entry}

\begin{entry}{伤心}{6,4}{⼈、⼼}
  \begin{phonetics}{伤心}{shang1xin1}[][HSK 3]
    \definition{v.+compl.}{estar triste; lamentar; estar com o coração partido; sentir-se triste por causa de infortúnio ou decepção}
  \end{phonetics}
\end{entry}

\begin{entry}{伤害}{6,10}{⼈、⼧}
  \begin{phonetics}{伤害}{shang1hai4}[][HSK 4]
    \definition{v.}{ferir; prejudicar; machucar; magoar; causar danos físicos ou mentais}
  \end{phonetics}
\end{entry}

\begin{entry}{伦}{6}{⼈}
  \begin{phonetics}{伦}{lun2}
    \definition*{s.}{sobrenome Lun}
    \definition{s.}{relações humanas (especialmente como concebidas pela ética feudal) | lógica; ordem | par; correspondência; (mesma) classe | ética; relações humanas | sequência lógica; ordem | o mesmo tipo; semelhante; igual}
  \end{phonetics}
\end{entry}

\begin{entry}{伦敦}{6,12}{⼈、⽁}
  \begin{phonetics}{伦敦}{lun2dun1}
    \definition*{s.}{Londres}
  \end{phonetics}
\end{entry}

\begin{entry}{伪}{6}{⼈}
  \begin{phonetics}{伪}{wei3}
    \definition{adj.}{falso; falsificado | fantoche; colaboracionista; ilegal | forjado; falso}
    \definition{pref.}{pseudo-; quasi-; quase-}
  \end{phonetics}
\end{entry}

\begin{entry}{似}{6}{⼈}
  \begin{phonetics}{似}{shi4}
    \definition{v.}{ver; parecer}
  \end{phonetics}
  \begin{phonetics}{似}{si4}
    \definition*{s.}{sobrenome Si}
    \definition{adv.}{parece; como se}
    \definition{v.}{ser semelhante; parecer-se com | parecer; aparecer | exceder}
  \end{phonetics}
\end{entry}

\begin{entry}{似乎}{6,5}{⼈、⼃}
  \begin{phonetics}{似乎}{si4hu1}[][HSK 4]
    \definition{adv.}{como se; aparentemente; se parece como}
  \end{phonetics}
\end{entry}

\begin{entry}{似的}{6,8}{⼈、⽩}
  \begin{phonetics}{似的}{shi4de5}[][HSK 4]
    \definition{part.}{como; como\dots como; como se (embora); usada após uma palavra ou frase para indicar uma semelhança com algo ou uma situação | usada para indicar alto grau}
  \end{phonetics}
\end{entry}

\begin{entry}{似曾相识}{6,12,9,7}{⼈、⽈、⽬、⾔}
  \begin{phonetics}{似曾相识}{si4ceng2xiang1shi2}
    \definition{s.}{\emph{déjà vu} (a experiência de ver exatamente a mesma situação pela segunda vez) | situação aparentemente familiar}
  \end{phonetics}
\end{entry}

\begin{entry}{充}{6}{⼉}
  \begin{phonetics}{充}{chong1}
    \definition{adj.}{suficiente; completo; amplo; cheio}
    \definition{s.}{sobrenome Chong}
    \definition{v.}{encher; carregar; atulhar | servir como; agir como | fingir ser; posar como; passar algo como}
  \end{phonetics}
\end{entry}

\begin{entry}{充分}{6,4}{⼉、⼑}
  \begin{phonetics}{充分}{chong1fen4}[][HSK 4]
    \definition{adj.}{cheio; amplo; abundante; suficiente; adequado}
    \definition{adv.}{totalmente; até o fim}
  \end{phonetics}
\end{entry}

\begin{entry}{充电}{6,5}{⼉、⽥}
  \begin{phonetics}{充电}{chong1 dian4}[][HSK 4]
    \definition{v.}{carregar (uma bateria); conectar uma fonte de alimentação CC aos terminais da bateria para recarregar a bateria | relaxar; passar o tempo livre; ``recarregar as baterias''; estudar para adquirir mais conhecimento; reabastecer (ou ampliar) o conhecimento; metaforicamente falando, para reabastecer a força física e a energia por meio do descanso e da recreação; também metaforicamente falando, para reabastecer novos conhecimentos e desenvolver novas habilidades por meio do reaprendizado}
  \end{phonetics}
\end{entry}

\begin{entry}{充电器}{6,5,16}{⼉、⽥、⼝}
  \begin{phonetics}{充电器}{chong1dian4qi4}[][HSK 4]
    \definition{s.}{carregador de bateria; dispositivo para alimentar uma bateria com energia, forçando uma corrente através dela}
  \end{phonetics}
\end{entry}

\begin{entry}{充足}{6,7}{⼉、⾜}
  \begin{phonetics}{充足}{chong1zu2}[][HSK 5]
    \definition{adj.}{bastante; adequado; suficiente}
  \end{phonetics}
\end{entry}

\begin{entry}{充满}{6,13}{⼉、⽔}
  \begin{phonetics}{充满}{chong1man3}[][HSK 3]
    \definition{v.}{preencher; encher; cobrir completamente| estar cheio de; estar repleto de; estar transbordando de; estar impregnado de}
  \end{phonetics}
\end{entry}

\begin{entry}{兆}{6}{⼉}
  \begin{phonetics}{兆}{zhao4}
    \definition{num.}{trilhão}
  \end{phonetics}
\end{entry}

\begin{entry}{先}{6}{⼉}
  \begin{phonetics}{先}{xian1}[][HSK 1]
    \definition*{s.}{sobrenome Xian}
    \definition{adv.}{primeiro; antes; mais cedo; com antecedência | no momento; por enquanto; em um curto espaço de tempo; temporariamente}
    \definition{s.}{início; começo; em ordem cronológica ou de precedência | ancestral; geração mais velha; antepassado | tardio; falecido; morto (honrar os mortos)}
  \end{phonetics}
\end{entry}

\begin{entry}{先不先}{6,4,6}{⼉、⼀、⼉}
  \begin{phonetics}{先不先}{xian1bu4xian1}
    \definition{adv.}{(dialeto) antes de tudo | em primeiro lugar}
  \end{phonetics}
\end{entry}

\begin{entry}{先天}{6,4}{⼉、⼤}
  \begin{phonetics}{先天}{xian1tian1}
    \definition{adj.}{congênito | inato | natural}
    \definition{s.}{período embrionário}
  \end{phonetics}
\end{entry}

\begin{entry}{先生}{6,5}{⼉、⽣}
  \begin{phonetics}{先生}{xian1sheng5}[][HSK 1]
    \definition[个,位]{s.}{professor; títulos honoríficos para professores, médicos, etc. | marido; antigamente, referia-se ao marido de outra pessoa ou ao próprio marido (ambos com pronomes pessoais como determinantes) | médico; títulos usados para se referir aos médicos no passado | refere-se a pessoas cuja profissão envolve contar histórias, adivinhação, etc.; antigamente, era chamado de contador | senhor; \emph{sir}; título dado aos intelectuais}
  \end{phonetics}
\end{entry}

\begin{entry}{先后}{6,6}{⼉、⼝}
  \begin{phonetics}{先后}{xian1 hou4}[][HSK 5]
    \definition{adv.}{sucessivamente; um após o outro}
    \definition{s.}{prioridade; ordem; cedo ou tarde; primeiro e último}
  \end{phonetics}
\end{entry}

\begin{entry}{先有}{6,6}{⼉、⽉}
  \begin{phonetics}{先有}{xian1you3}
    \definition{adj.}{preexistente | anterior}
  \end{phonetics}
\end{entry}

\begin{entry}{先进}{6,7}{⼉、⾡}
  \begin{phonetics}{先进}{xian1jin4}[][HSK 3]
    \definition{adj.}{avançado; progressos rápidos e nível elevado, podendo servir de exemplo a seguir}
    \definition{s.}{indivíduos ou grupos avançados}
  \end{phonetics}
\end{entry}

\begin{entry}{先到先得}{6,8,6,11}{⼉、⼑、⼉、⼻}
  \begin{phonetics}{先到先得}{xian1dao4xian1de2}
    \definition{expr.}{primeiro a chegar | primeiro a ser servido}
  \end{phonetics}
\end{entry}

\begin{entry}{先前}{6,9}{⼉、⼑}
  \begin{phonetics}{先前}{xian1qian2}[][HSK 5]
    \definition[出]{s.}{antes; anteriormente; geralmente se refere ao passado ou a um certo tempo anterior}
  \end{phonetics}
\end{entry}

\begin{entry}{先烈}{6,10}{⼉、⽕}
  \begin{phonetics}{先烈}{xian1lie4}
    \definition{s.}{mártir}
  \end{phonetics}
\end{entry}

\begin{entry}{先验}{6,10}{⼉、⾺}
  \begin{phonetics}{先验}{xian1yan4}
    \definition{adj.}{(filosofia) a priori}
  \end{phonetics}
\end{entry}

\begin{entry}{先期}{6,12}{⼉、⽉}
  \begin{phonetics}{先期}{xian1qi1}
    \definition{adv.}{antecipadamente}
    \definition{s.}{prematuro | \emph{front-end}}
  \end{phonetics}
\end{entry}

\begin{entry}{光}{6}{⼉}
  \begin{phonetics}{光}{guang1}[][HSK 3]
    \definition*{s.}{sobrenome Guang}
    \definition{adj.}{suave; liso; brilhante | esgotado; sem nada sobrando | brilhante}
    \definition{adv.}{somente; sozinho; meramente}
    \definition{s.}{luz; raio | cenário; paisagem | honra; glória; brilho | claridade | favor; graça | momento | corpo celeste; referindo-se especificamente a corpos celestes, como o sol, a lua e as estrelas}
    \definition{v.}{glorificar; recuperar; reconquistar | estar nu; expor}
  \end{phonetics}
\end{entry}

\begin{entry}{光污染}{6,6,9}{⼉、⽔、⽊}
  \begin{phonetics}{光污染}{guang1 wu1ran3}
    \definition{s.}{poluição luminosa}
  \end{phonetics}
\end{entry}

\begin{entry}{光明}{6,8}{⼉、⽇}
  \begin{phonetics}{光明}{guang1ming2}[][HSK 3]
    \definition{adj.}{brilhante; luminoso | sincero; ingênuo; metáfora da justiça e da esperança | justo; honesto; franco}
    \definition{s.}{luz}
  \end{phonetics}
\end{entry}

\begin{entry}{光线}{6,8}{⼉、⽷}
  \begin{phonetics}{光线}{guang1 xian4}[][HSK 5]
    \definition[条,道]{s.}{luz; feixe luminoso; raio de luz}
  \end{phonetics}
\end{entry}

\begin{entry}{光临}{6,9}{⼉、⼁}
  \begin{phonetics}{光临}{guang1lin2}[][HSK 4]
    \definition{v.}{honrar com sua presença, uma palavra de honra, usada para dizer que um convidado chegou}
  \end{phonetics}
\end{entry}

\begin{entry}{光荣}{6,9}{⼉、⾋}
  \begin{phonetics}{光荣}{guang1rong2}[][HSK 5]
    \definition{adj.}{honroso; honrado; glorioso; por fazer algo que é benéfico para o país ou para a coletividade e que é considerado por todos como digno de respeito ou elogio}
    \definition{s.}{honra; glória; crédito; sentimento de honra decorrente do fato de ser respeitado ou elogiado por fazer algo importante ou grandioso}
  \end{phonetics}
\end{entry}

\begin{entry}{光盘}{6,11}{⼉、⽫}
  \begin{phonetics}{光盘}{guang1pan2}[][HSK 4]
    \definition[片,张]{s.}{CD; disco compacto; um disco circular feito de plástico rígido composto que usa um laser para registrar e ler informações}
  \end{phonetics}
\end{entry}

\begin{entry}{光槃}{6,14}{⼉、⽊}
  \begin{phonetics}{光槃}{guang1pan2}
    \variantof{光盘}
  \end{phonetics}
\end{entry}

\begin{entry}{全}{6}{⼊}
  \begin{phonetics}{全}{quan2}[][HSK 2]
    \definition*{adj.}{completo; total; inteiro}
    \definition*{s.}{sobrenome Quan}
    \definition{adv.}{inteiramente; totalmente; completamente; significa 100\%; equivalente a 完全 ou 全然}
    \definition{v.}{manter intacto; tornar perfeito ou completo; completar}
  \seealsoref{全然}{quan2ran2}
  \seealsoref{完全}{wan2quan2}
  \end{phonetics}
\end{entry}

\begin{entry}{全世界}{6,5,9}{⼊、⼀、⽥}
  \begin{phonetics}{全世界}{quan2 shi4 jie4}[][HSK 5]
    \definition[种]{s.}{mundo inteiro; mundo todo | em todo o mundo}
  \end{phonetics}
\end{entry}

\begin{entry}{全场}{6,6}{⼊、⼟}
  \begin{phonetics}{全场}{quan2 chang3}[][HSK 3]
    \definition{s.}{toda a audiência; todos os presentes; todo o público}
  \end{phonetics}
\end{entry}

\begin{entry}{全年}{6,6}{⼊、⼲}
  \begin{phonetics}{全年}{quan2 nian2}[][HSK 2]
    \definition{s.}{ano inteiro | anual; todo ano}
  \end{phonetics}
\end{entry}

\begin{entry}{全体}{6,7}{⼊、⼈}
  \begin{phonetics}{全体}{quan2 ti3}[][HSK 2]
    \definition{s.}{(frequentemente referido a pessoas) todos; número total; todos | por todo o corpo | todos; inteiro; a soma de todas as partes; a soma de todos os indivíduos (geralmente se refere a pessoas)}
  \end{phonetics}
\end{entry}

\begin{entry}{全身}{6,7}{⼊、⾝}
  \begin{phonetics}{全身}{quan2 shen1}[][HSK 2]
    \definition{s.}{corpo inteiro; por todo o corpo; todo o corpo}
  \end{phonetics}
\end{entry}

\begin{entry}{全国}{6,8}{⼊、⼞}
  \begin{phonetics}{全国}{quan2 guo2}[][HSK 2]
    \definition{s.}{toda a nação (ou país); em todo o país; em todo o território nacional | toda a nação; todo o país}
  \end{phonetics}
\end{entry}

\begin{entry}{全面}{6,9}{⼊、⾯}
  \begin{phonetics}{全面}{quan2mian4}[][HSK 3]
    \definition{adj.}{geral; completo; abrangente; onipotente}
    \definition{s.}{todos os aspectos; cada aspecto}
  \seealsoref{片面}{pian4mian4}
  \end{phonetics}
\end{entry}

\begin{entry}{全家}{6,10}{⼊、⼧}
  \begin{phonetics}{全家}{quan2 jia1}[][HSK 2]
    \definition{s.}{toda a família; a família inteira}
  \end{phonetics}
\end{entry}

\begin{entry}{全部}{6,10}{⼊、⾢}
  \begin{phonetics}{全部}{quan2bu4}[][HSK 2]
    \definition{adv.}{tudo; total; inteiro; completo; aplica-se a todos, sem exceção}
    \definition{s.}{totalidade; total; integridade; a soma de todas as partes; o todo}
  \end{phonetics}
\end{entry}

\begin{entry}{全都}{6,10}{⼊、⾢}
  \begin{phonetics}{全都}{quan2 dou1}[][HSK 5]
    \definition{adv.}{tudo; todos; sem exceção}
  \end{phonetics}
\end{entry}

\begin{entry}{全都不}{6,10,4}{⼊、⾢、⼀}
  \begin{phonetics}{全都不}{quan2dou1 bu4}
    \definition{adj.}{nada; nenhum; nenhum deles; nada disso}
  \end{phonetics}
\end{entry}

\begin{entry}{全球}{6,11}{⼊、⽟}
  \begin{phonetics}{全球}{quan2 qiu2}[][HSK 3]
    \definition[门]{s.}{o mundo inteiro; a Terra inteira}
  \end{phonetics}
\end{entry}

\begin{entry}{全职}{6,11}{⼊、⽿}
  \begin{phonetics}{全职}{quan2zhi2}
    \definition{s.}{período integral | tempo inteiro | (trabalho) \emph{full-time}}
  \end{phonetics}
\end{entry}

\begin{entry}{全然}{6,12}{⼊、⽕}
  \begin{phonetics}{全然}{quan2ran2}
    \definition{adv.}{completamente; inteiramente}
  \end{phonetics}
\end{entry}

\begin{entry}{共}{6}{⼋}
  \begin{phonetics}{共}{gong4}[][HSK 4]
    \definition*{s.}{sobrenome Gong}
    \definition*{s.}{Abreviação de Partido Comunista, 共产党}
    \definition{adj.}{conjunto; mútuo; geral; comum; o mesmo para todos}
    \definition{adv.}{juntos; juntamente; conjuntamente | em sua totalidade; em todos}
    \definition{v.}{compartilhar com; empreender ou realizar em conjunto}
  \seealsoref{共产党}{gong4chan3dang3}
  \end{phonetics}
\end{entry}

\begin{entry}{共计}{6,4}{⼋、⾔}
  \begin{phonetics}{共计}{gong4ji4}[][HSK 5]
    \definition{s.}{total; total geral; agregado; montante}
    \definition{v.}{contar até; somar até; totalizar}
  \end{phonetics}
\end{entry}

\begin{entry}{共产}{6,6}{⼋、⼇}
  \begin{phonetics}{共产}{gong4chan3}
    \definition{adj.}{comunista}
    \definition{s.}{comunismo}
  \end{phonetics}
\end{entry}

\begin{entry}{共产党}{6,6,10}{⼋、⼇、⼉}
  \begin{phonetics}{共产党}{gong4chan3dang3}
    \definition*{s.}{Partido Comunista}
  \end{phonetics}
\end{entry}

\begin{entry}{共同}{6,6}{⼋、⼝}
  \begin{phonetics}{共同}{gong4tong2}[][HSK 3]
    \definition{adj.}{comum; compartilhado; colaborativo; todos têm}
    \definition{adv.}{juntos; conjuntamente; todos juntos (fazemos)}
  \end{phonetics}
\end{entry}

\begin{entry}{共同体}{6,6,7}{⼋、⼝、⼈}
  \begin{phonetics}{共同体}{gong4tong2ti3}
    \definition{s.}{comunidade}
  \end{phonetics}
\end{entry}

\begin{entry}{共有}{6,6}{⼋、⽉}
  \begin{phonetics}{共有}{gong4 you3}[][HSK 3]
    \definition{v.}{compartilhar; possuir (por todos); possuir ou desfrutar em conjunto}
  \end{phonetics}
\end{entry}

\begin{entry}{共享}{6,8}{⼋、⼇}
  \begin{phonetics}{共享}{gong4 xiang3}[][HSK 5]
    \definition{v.}{compartilhar; desfrutar juntos; aproveitar as coisas boas juntos}
  \end{phonetics}
\end{entry}

\begin{entry}{兲}{6}{⼋}
  \begin{phonetics}{兲}{tian1}
    \variantof{天}
  \end{phonetics}
\end{entry}

\begin{entry}{关}{6}{⼋}
  \begin{phonetics}{关}{guan1}[][HSK 1,4]
    \definition*{s.}{sobrenome Guan}
    \definition{s.}{passagem; ponto de controle | alfândega; escritórios de cobrança de impostos para exportação e importação de mercadorias | ponto de inflexão ou barreira; ponto de virada ou dificuldade | momento crítico; mecanismo}
    \definition{v.}{fechar; encerrar; amarrar algo | fechar; trancar | encerrar; sair do mercado; falir | conceder ou sacar o pagamento de um salário | desligar | envolver; preocupar-se; conectar-se}
  \end{phonetics}
\end{entry}

\begin{entry}{关上}{6,3}{⼋、⼀}
  \begin{phonetics}{关上}{guan1 shang4}[][HSK 1]
    \definition{v.}{fechar (uma porta); fechar um objeto | desligar (luz, equipamento elétrico etc.); parar ou encerrar (uma atividade, situação, etc.)}
  \end{phonetics}
\end{entry}

\begin{entry}{关于}{6,3}{⼋、⼆}
  \begin{phonetics}{关于}{guan1yu2}[][HSK 4]
    \definition{prep.}{sobre; relativo a; pertencente a; uma questão de; com relação a}
  \end{phonetics}
\end{entry}

\begin{entry}{关心}{6,4}{⼋、⼼}
  \begin{phonetics}{关心}{guan1xin1}[][HSK 2]
    \definition{v.}{cuidar; preocupar-se com; manifestar interesse por; demonstrar solicitude por; (colocar uma pessoa ou coisa) sempre no coração; valorizar e cuidar}
  \end{phonetics}
\end{entry}

\begin{entry}{关机}{6,6}{⼋、⽊}
  \begin{phonetics}{关机}{guan1 ji1}[][HSK 2]
    \definition{v.}{encerrar; terminar; refere-se especificamente à conclusão das filmagens de um filme ou série de TV | desligar; desligar a fonte de alimentação; parar o funcionamento da máquina}
  \end{phonetics}
\end{entry}

\begin{entry}{关闭}{6,6}{⼋、⾨}
  \begin{phonetics}{关闭}{guan1bi4}[][HSK 4]
    \definition{v.}{fechar | (empresa) falir}
  \end{phonetics}
\end{entry}

\begin{entry}{关怀}{6,7}{⼋、⼼}
  \begin{phonetics}{关怀}{guan1huai2}[][HSK 5]
    \definition{v.}{mostrar cuidado amoroso por; mostrar solicitude por; cuidar, amar, apoiar ou ajudar os fracos ou grupos em dificuldade | geralmente usado para superiores para subordinados, anciãos para juniores ou organizações para indivíduos}
  \end{phonetics}
\end{entry}

\begin{entry}{关系}{6,7}{⼋、⽷}
  \begin{phonetics}{关系}{guan1xi5}[][HSK 3]
    \definition[个,种]{s.}{relações; conexões; relacionamento; a interligação entre pessoas ou coisas | consequência; impacto; significado a influência ou importância de algo; algo digno de nota (geralmente usado com 没有, 有). | causa; razão (geralmente usado com 由于 ou 因为); refere-se genericamente a causas, condições, etc. | credenciais que mostram filiação a uma organização; documento que comprova a existência de algum tipo de relação organizacional}
    \definition{v.}{preocupar; afetar; ter influência sobre; ter a ver com}
  \seealsoref{没有}{mei2 you3}
  \seealsoref{因为}{yin1wei4}
  \seealsoref{由于}{you2yu2}
  \seealsoref{有}{you3}
  \end{phonetics}
\end{entry}

\begin{entry}{关注}{6,8}{⼋、⽔}
  \begin{phonetics}{关注}{guan1 zhu4}[][HSK 3]
    \definition{v.}{prestar atenção em; seguir algo de perto; seguir (nas redes sociais)}
  \end{phonetics}
\end{entry}

\begin{entry}{关键}{6,13}{⼋、⾦}
  \begin{phonetics}{关键}{guan1jian4}[][HSK 5]
    \definition{adj.}{crucial; decisivo; importante; que pode determinar o curso e o resultado dos eventos}
    \definition[个]{s.}{chave; ponto crucial; aspectos ou condições mais importantes que determinam o desenvolvimento e o resultado de algo}
  \end{phonetics}
\end{entry}

\begin{entry}{兴}{6}{⼋}
  \begin{phonetics}{兴}{xing1}
    \definition*{s.}{sobrenome Xing}
    \definition{adv.}{talvez (dialeto)}
    \definition{v.}{subir | florescer | tornar-se popular | começar | encorajar | levantar-se | (frequentemente usado em negativas) permitir (dialeto)}
  \end{phonetics}
  \begin{phonetics}{兴}{xing4}
    \definition{s.}{sentimento ou desejo de fazer algo | interesse em algo | excitação}
  \end{phonetics}
\end{entry}

\begin{entry}{兴奋}{6,8}{⼋、⼤}
  \begin{phonetics}{兴奋}{xing1fen4}[][HSK 4]
    \definition{adj.}{animado; excitante; empolgante;}
    \definition{s.}{excitação; empolgação}
    \definition{v.}{excitar; intoxicar}
  \end{phonetics}
\end{entry}

\begin{entry}{兴趣}{6,15}{⼋、⾛}
  \begin{phonetics}{兴趣}{xing4 qu4}[][HSK 4]
    \definition[个]{s.}{interesse (desejo de conhecer sobre alguma coisa ou coisa no qual está interessado) | \emph{hobby}}
  \end{phonetics}
\end{entry}

\begin{entry}{再}{6}{⼌}
  \begin{phonetics}{再}{zai4}[][HSK 1]
    \definition{adv.}{mais uma vez; além disso; ainda mais; indica a repetição ou continuação de uma mesma ação ou comportamento; refere-se principalmente a ações ou comportamentos não realizados ou contínuos | usado antes do adjetivo, indica intensificação, equivalente a 更 ou 更加 | (para uma ação adiada, precedida por uma expressão de tempo ou condição) então; somente então; depois de algo; indica que a ação ocorrerá após a conclusão de outra ação | além disso; indica um complemento, equivalente a 另外 ou 又 | próxima vez; indica que a ação ocorrerá após um determinado período de tempo | novamente; de novo}
  \seealsoref{更}{geng4}
  \seealsoref{更加}{geng4 jia1}
  \seealsoref{另外}{ling4wai4}
  \seealsoref{又}{you4}
  \end{phonetics}
\end{entry}

\begin{entry}{再三}{6,3}{⼌、⼀}
  \begin{phonetics}{再三}{zai4san1}[][HSK 4]
    \definition{adv.}{repetidamente; repetidas vezes; de novo e de novo}
  \end{phonetics}
\end{entry}

\begin{entry}{再也}{6,3}{⼌、⼄}
  \begin{phonetics}{再也}{zai4 ye3}[][HSK 5]
    \definition{adv.}{não mais; nunca mais; uma determinada situação ou ação nunca mais ocorrerá}
  \end{phonetics}
\end{entry}

\begin{entry}{再不}{6,4}{⼌、⼀}
  \begin{phonetics}{再不}{zai4bu4}
    \definition{adv.}{nunca mais}
  \end{phonetics}
\end{entry}

\begin{entry}{再见}{6,4}{⼌、⾒}
  \begin{phonetics}{再见}{zai4jian4}[][HSK 1]
    \definition{v.}{adeus; tchau; até logo; até mais; até mais tarde}
  \end{phonetics}
\end{entry}

\begin{entry}{再发}{6,5}{⼌、⼜}
  \begin{phonetics}{再发}{zai4fa1}
    \definition{v.}{reenviar}
  \end{phonetics}
\end{entry}

\begin{entry}{再生}{6,5}{⼌、⽣}
  \begin{phonetics}{再生}{zai4sheng1}
    \definition{s.}{reciclagem | regeneração}
    \definition{v.}{reciclar | renascer | regenerar}
  \end{phonetics}
\end{entry}

\begin{entry}{再次}{6,6}{⼌、⽋}
  \begin{phonetics}{再次}{zai4 ci4}[][HSK 5]
    \definition{adv.}{mais uma vez; uma segunda vez; outra vez}
  \end{phonetics}
\end{entry}

\begin{entry}{再审}{6,8}{⼌、⼧}
  \begin{phonetics}{再审}{zai4shen3}
    \definition{s.}{novo julgamento | revisão}
    \definition{v.}{ouvir um caso novamente}
  \end{phonetics}
\end{entry}

\begin{entry}{再者}{6,8}{⼌、⽼}
  \begin{phonetics}{再者}{zai4zhe3}
    \definition{conj.}{além do mais | além disso}
  \end{phonetics}
\end{entry}

\begin{entry}{再育}{6,8}{⼌、⾁}
  \begin{phonetics}{再育}{zai4yu4}
    \definition{v.}{aumentar | multiplicar | proliferar}
  \end{phonetics}
\end{entry}

\begin{entry}{再临}{6,9}{⼌、⼁}
  \begin{phonetics}{再临}{zai4lin2}
    \definition{v.}{vir de novo}
  \end{phonetics}
\end{entry}

\begin{entry}{再度}{6,9}{⼌、⼴}
  \begin{phonetics}{再度}{zai4du4}
    \definition{adv.}{outra vez | mais uma vez}
  \end{phonetics}
\end{entry}

\begin{entry}{再说}{6,9}{⼌、⾔}
  \begin{phonetics}{再说}{zai4shuo1}
    \definition{conj.}{além do mais | além disso | o que mais}
    \definition{v.}{adiar uma discussão para mais tarde | dizer novamente}
  \end{phonetics}
\end{entry}

\begin{entry}{再读}{6,10}{⼌、⾔}
  \begin{phonetics}{再读}{zai4du2}
    \definition{v.}{ler novamente | rever (uma lição, etc.)}
  \end{phonetics}
\end{entry}

\begin{entry}{军}{6}{⼍}
  \begin{phonetics}{军}{jun1}
    \definition*{s.}{sobrenome Jun}
    \definition{s.}{forças armadas; exército; tropas | exército; contingente; muitas pessoas participando de uma atividade | exército; unidades militares}
  \end{phonetics}
\end{entry}

\begin{entry}{军人}{6,2}{⼍、⼈}
  \begin{phonetics}{军人}{jun1 ren2}[][HSK 5]
    \definition{s.}{soldado; militar; pessoal militar; pessoas com status militar; pessoas servindo nas forças armadas}
  \end{phonetics}
\end{entry}

\begin{entry}{军装}{6,12}{⼍、⾐}
  \begin{phonetics}{军装}{jun1zhuang1}
    \definition{s.}{uniforme militar}
  \end{phonetics}
\end{entry}

\begin{entry}{农}{6}{⼍}
  \begin{phonetics}{农}{nong2}
    \definition*{s.}{sobrenome Nong}
    \definition{s.}{agricultura; criação de animais | camponês; fazendeiro}
  \end{phonetics}
\end{entry}

\begin{entry}{农业}{6,5}{⼍、⼀}
  \begin{phonetics}{农业}{nong2ye4}[][HSK 3]
    \definition{s.}{agricultura}
  \end{phonetics}
\end{entry}

\begin{entry}{农民}{6,5}{⼍、⽒}
  \begin{phonetics}{农民}{nong2min2}[][HSK 3]
    \definition[个,位,名,些]{s.}{fazendeiro; camponês; campesinato; trabalhadores que participam da produção agrícola há muito tempo}
  \end{phonetics}
\end{entry}

\begin{entry}{农产品}{6,6,9}{⼍、⼇、⼝}
  \begin{phonetics}{农产品}{nong2 chan3 pin3}[][HSK 5]
    \definition{s.}{produtos agrícolas}
  \end{phonetics}
\end{entry}

\begin{entry}{农村}{6,7}{⼍、⽊}
  \begin{phonetics}{农村}{nong2cun1}[][HSK 3]
    \definition{s.}{aldeia; campo; área rural; locais onde vivem os trabalhadores principalmente dedicados à produção agrícola}
  \end{phonetics}
\end{entry}

\begin{entry}{冰}{6}{⼎}
  \begin{phonetics}{冰}{bing1}[][HSK 4]
    \definition[块,层,些]{s.}{gelo; água em estado sólido |  algo parecido com gelo | (gíria) metanfetamina}
    \definition{v.}{colocar gelo; colocar gelo ao redor; colocar no gelo; resfriar objetos com gelo ou água fria | sentir frio}
  \end{phonetics}
\end{entry}

\begin{entry}{冰天雪地}{6,4,11,6}{⼎、⼤、⾬、⼟}
  \begin{phonetics}{冰天雪地}{bing1tian1-xue3di4}
    \definition{expr.}{um mundo de gelo e neve}
  \end{phonetics}
\end{entry}

\begin{entry}{冰球}{6,11}{⼎、⽟}
  \begin{phonetics}{冰球}{bing1qiu2}
    \definition{s.}{hóquei no gelo}
  \end{phonetics}
\end{entry}

\begin{entry}{冰雪}{6,11}{⼎、⾬}
  \begin{phonetics}{冰雪}{bing1 xue3}[][HSK 4]
    \definition{adj.}{puro como gelo e neve; descreve uma pessoa como pura}
    \definition{s.}{gelo e neve}
  \end{phonetics}
\end{entry}

\begin{entry}{冰棍}{6,12}{⼎、⽊}
  \begin{phonetics}{冰棍}{bing1gun4}
    \definition[根]{s.}{picolé}
  \end{phonetics}
\end{entry}

\begin{entry}{冰箱}{6,15}{⼎、⾋}
  \begin{phonetics}{冰箱}{bing1xiang1}[][HSK 4]
    \definition[台,个]{s.}{geladeira; freezer; refrigerador; aparelhos para congelar alimentos ou medicamentos com gelo para mantê-los frios}
  \end{phonetics}
\end{entry}

\begin{entry}{冰激凌}{6,16,10}{⼎、⽔、⼎}
  \begin{phonetics}{冰激凌}{bing1ji1ling2}
    \definition{s.}{sorvete}
  \end{phonetics}
\end{entry}

\begin{entry}{冰糕}{6,16}{⼎、⽶}
  \begin{phonetics}{冰糕}{bing1gao1}
    \definition{s.}{sorvete | picolé}
  \end{phonetics}
\end{entry}

\begin{entry}{冲}{6}{⼎}
  \begin{phonetics}{冲}{chong1}[][HSK 4]
    \definition{s.}{via pública; local importante; via de passagem; via local importante | um trecho de planície em uma área montanhosa | (astronomia) oposição; os planetas externos orbitam até ficarem alinhados com a Terra e o Sol, e a Terra está no meio}
    \definition{v.}{atacar; apressar; correr; passar rapidamente; passar por um obstáculo | colidir; chocar; bater | despejar água fervente sobre | enxaguar; dar descarga; lavar | revelar (filme) | neutralizar a má sorte}
  \end{phonetics}
  \begin{phonetics}{冲}{chong4}
    \definition{adj.}{poderoso; com vigor; com muita força; vigoroso | forte; odor forte e pungente (olfato)}
    \definition{prep.}{de frente; em direção a | na força de; com base em; em virtude de}
    \definition{v.}{estampar (máquina de estamparia)}
  \end{phonetics}
\end{entry}

\begin{entry}{冲动}{6,6}{⼎、⼒}
  \begin{phonetics}{冲动}{chong1dong4}[][HSK 5]
    \definition{adj.}{impulsivo; impetuoso}
    \definition{s.}{impulso; impetuosidade; impulso de movimento; fenômeno psicológico no qual as emoções são particularmente fortes e o controle racional é fraco}
    \definition{v.}{ficar animado; ser impetuoso; agir por impulso}
  \end{phonetics}
\end{entry}

\begin{entry}{冲突}{6,9}{⼎、⽳}
  \begin{phonetics}{冲突}{chong1tu1}[][HSK 5]
    \definition{v.}{chocar-se; entrar em conflito; conflitar | contradizer; duas coisas opostas que interferem uma na outra}
  \end{phonetics}
\end{entry}

\begin{entry}{冲浪}{6,10}{⼎、⽔}
  \begin{phonetics}{冲浪}{chong1lang4}
    \definition{s.}{surfe}
    \definition{v.}{surfar}
  \end{phonetics}
\end{entry}

\begin{entry}{冲锋}{6,12}{⼎、⾦}
  \begin{phonetics}{冲锋}{chong1feng1}
    \definition{v.}{cobrar | tomar de assalto}
  \end{phonetics}
\end{entry}

\begin{entry}{决}{6}{⼎}
  \begin{phonetics}{决}{jue2}
    \definition{v.}{decidir; determinar | executar uma pessoa | (de um dique, etc.) romper; desabar}
  \end{phonetics}
\end{entry}

\begin{entry}{决不}{6,4}{⼎、⼀}
  \begin{phonetics}{决不}{jue2 bu4}[][HSK 5]
    \definition{adv.}{definitivamente não; certamente não; sob nenhuma circunstância; de forma alguma}
  \end{phonetics}
\end{entry}

\begin{entry}{决心}{6,4}{⼎、⼼}
  \begin{phonetics}{决心}{jue2xin1}[][HSK 3]
    \definition{s.}{resolução; determinação; determinação inabalável}
    \definition{v.}{secidir-se; decidir fazer algo e não vacilar nem mudar de ideia}
  \end{phonetics}
\end{entry}

\begin{entry}{决定}{6,8}{⼎、⼧}
  \begin{phonetics}{决定}{jue2ding4}[][HSK 3]
    \definition{adj.}{decisivo; as leis objetivas levam as coisas a se desenvolverem e mudarem em determinada direção}
    \definition[项,个]{s.}{decisão; resolução; assuntos decididos}
    \definition{v.}{decidir; determinar; algo se torna a base ou o pré-requisito para outra coisa; desempenha um papel dominante | decidir; resolver; tomar uma decisão; propor uma forma de agir}
  \end{phonetics}
\end{entry}

\begin{entry}{决赛}{6,14}{⼎、⾙}
  \begin{phonetics}{决赛}{jue2sai4}[][HSK 3]
    \definition[场]{s.}{finais (de uma competição); em competições esportivas, a última partida ou rodada disputada para determinar a classificação}
  \end{phonetics}
\end{entry}

\begin{entry}{划}{6}{⼑}
  \begin{phonetics}{划}{hua2}[][HSK 4]
    \definition{adj.}{rentável; vale (o esforço); compensa (fazer alguma coisa)}
    \definition{v.}{remar | ser vantajoso para alguém; ser uma pechincha | arranhar; cortar a superfície de; cortar em outra coisa com um objeto pontiagudo | arranhar; golpear;  esfregar uma coisa ou varrer sobre outra}
  \end{phonetics}
  \begin{phonetics}{划}{hua4}[][HSK 4]
    \definition*{s.}{sobrenome Hua}
    \definition{s.}{traço de um caracter chinês}
    \definition{v.}{delimitar; diferenciar; delinear | transferir; ceder | planejar; programar | desenhar; marcar; delinear; fazer linhas ou escrever como marcadores com uma caneta ou objeto semelhante a uma caneta}
  \end{phonetics}
\end{entry}

\begin{entry}{划分}{6,4}{⼑、⼑}
  \begin{phonetics}{划分}{hua4fen1}[][HSK 5]
    \definition{v.}{dividir; particionar; reparticionar | diferenciar; encontrar aspectos diferentes}
  \end{phonetics}
\end{entry}

\begin{entry}{划船}{6,11}{⼑、⾈}
  \begin{phonetics}{划船}{hua2 chuan2}[][HSK 3]
    \definition[次,回]{s.}{remo (ato de remar); passeios de barco; a atividade ou esporte de “remar um barco com remos”}
    \definition{v.}{remar um barco; a ação ou comportamento de mover um barco na água usando remos}
  \end{phonetics}
\end{entry}

\begin{entry}{划艇}{6,12}{⼑、⾈}
  \begin{phonetics}{划艇}{hua2ting3}
    \definition{s.}{barco a remo}
  \end{phonetics}
\end{entry}

\begin{entry}{列}{6}{⼑}
  \begin{phonetics}{列}{lie4}[][HSK 4]
    \definition{v.}{organizar; formar uma linha; alinhar | listar; inserir em uma lista}
  \end{phonetics}
\end{entry}

\begin{entry}{列入}{6,2}{⼑、⼊}
  \begin{phonetics}{列入}{lie4 ru4}[][HSK 4]
    \definition{v.}{incluir em uma lista}
  \end{phonetics}
\end{entry}

\begin{entry}{列为}{6,4}{⼑、⼂}
  \begin{phonetics}{列为}{lie4 wei2}[][HSK 4]
    \definition{v.}{ser classificado como; ser listado como}
  \end{phonetics}
\end{entry}

\begin{entry}{列车}{6,4}{⼑、⾞}
  \begin{phonetics}{列车}{lie4che1}[][HSK 4]
    \definition{s.}{trem; trem em uma composição contínua, puxado por uma locomotiva e equipado com uma tripulação e marcações prescritas; geralmente um trem de passageiros}
  \end{phonetics}
\end{entry}

\begin{entry}{刘}{6}{⼑}
  \begin{phonetics}{刘}{liu2}
    \definition*{s.}{sobrenome Liu}
    \definition{s.}{(clássico) um tipo de machado de batalha}
    \definition{v.}{matar}
  \end{phonetics}
\end{entry}

\begin{entry}{刚}{6}{⼑}
  \begin{phonetics}{刚}{gang1}[][HSK 2]
    \definition*{s.}{sobrenome Gang}
    \definition{adj.}{duro; firme; rígido; forte; (personalidade, atitude) forte; (vontade) determinada}
    \definition{adv.}{apenas; exatamente; justamente | apenas; apenas por pouco; significa atingir um certo nível com dificuldade | apenas; há pouco tempo; indica que a ação ou situação ocorreu há pouco tempo | assim que; somente neste momento; aconteceu que; use a palavra 就 para indicar que duas coisas estão intimamente relacionadas}
  \seealsoref{就}{jiu4}
  \end{phonetics}
\end{entry}

\begin{entry}{刚才}{6,3}{⼑、⼿}
  \begin{phonetics}{刚才}{gang1cai2}[][HSK 2]
    \definition{s.}{agora mesmo; há pouco; há pouco tempo; referindo-se ao período recente que acabou de passar}
  \end{phonetics}
\end{entry}

\begin{entry}{刚刚}{6,6}{⼑、⼑}
  \begin{phonetics}{刚刚}{gang1 gang5}[][HSK 2]
    \definition{adv.}{apenas; somente; exatamente; refere-se a algo que é adequado em termos de grau, quantidade, tempo, etc., nem mais nem menos, nem cedo nem tarde, atingindo um estado satisfatório ou que atende exatamente às necessidades | agora mesmo; há pouco; há um momento atrás; referindo-se a um período de tempo muito curto no passado}
  \end{phonetics}
\end{entry}

\begin{entry}{创}{6}{⼑}
  \begin{phonetics}{创}{chuang1}
    \definition{s.}{ferimento; trauma}
  \end{phonetics}
  \begin{phonetics}{创}{chuang4}
    \definition{v.}{começar (fazer algo); alcançar (algo pela primeira vez); estabelecer; fazer pela primeira vez | estabelecer; fundar; criar; perceber algo novo, como um começo | ferir; machucar}
  \end{phonetics}
\end{entry}

\begin{entry}{创业}{6,5}{⼑、⼀}
  \begin{phonetics}{创业}{chuang4ye4}[][HSK 3]
    \definition{s.}{empreendedorismo}
    \definition{v.}{começar um empreendimento; iniciar/fundar um negócio, uma empresa;}
  \end{phonetics}
\end{entry}

\begin{entry}{创立}{6,5}{⼑、⽴}
  \begin{phonetics}{创立}{chuang4li4}[][HSK 5]
    \definition{v.}{fundar; originar; estabelecer}
  \end{phonetics}
\end{entry}

\begin{entry}{创作}{6,7}{⼑、⼈}
  \begin{phonetics}{创作}{chuang4zuo4}[][HSK 3]
    \definition[个]{s.}{criação; trabalho criativo; obras literárias e artísticas}
    \definition{v.}{escrever; criar; produzir; compor; criar obras artísticas}
  \end{phonetics}
\end{entry}

\begin{entry}{创造}{6,10}{⼑、⾡}
  \begin{phonetics}{创造}{chuang4zao4}[][HSK 3]
    \definition{s.}{criação; inovação; primeiro a concluir ou a alcançar resultados}
    \definition{v.}{criar; produzir; trazer à tona; fazer ou estabelecer pela primeira vez; referir-se de maneira geral a fazer ou estabelecer}
  \end{phonetics}
\end{entry}

\begin{entry}{创意}{6,13}{⼑、⼼}
  \begin{phonetics}{创意}{chuang4yi4}
    \definition{adj.}{criativo}
    \definition{s.}{criatividade}
  \end{phonetics}
\end{entry}

\begin{entry}{创新}{6,13}{⼑、⽄}
  \begin{phonetics}{创新}{chuang4xin1}[][HSK 3]
    \definition[个,种,次]{s.}{inovação; algo novo ou diferente, uma ideia}
    \definition{v.}{trazer novas ideias; inovar; abrir novos caminhos; criar ou fazer algo novo, diferente do que era antes}
  \end{phonetics}
\end{entry}

\begin{entry}{动}{6}{⼒}
  \begin{phonetics}{动}{dong4}[][HSK 1]
    \definition{adj.}{não estacionário; móvel; variável; mutável}
    \definition{adv.}{facilmente; frequentemente}
    \definition{s.}{ação; movimento}
    \definition{v.}{mover; mexer; (pessoas ou coisas) mudar a posição ou o estado original | agir; começar a agir; entrar em ação | alterar; mudar; alterar a posição ou o estado original | usar; utilizar; tornar ativo | despertar; tocar (o coração de alguém); provocar mudanças emocionais, reações | [geralmente na forma negativa] comer ou beber | emocionar; deixar emocionado}
  \end{phonetics}
\end{entry}

\begin{entry}{动人}{6,2}{⼒、⼈}
  \begin{phonetics}{动人}{dong4 ren2}[][HSK 3]
    \definition{adj.}{comovente; emocionante; tocante}
  \end{phonetics}
\end{entry}

\begin{entry}{动力}{6,2}{⼒、⼒}
  \begin{phonetics}{动力}{dong4li4}
    \definition[种,个]{s.}{poder; a força que faz com que as máquinas funcionem, por exemplo, energia elétrica, eólica, hidráulica, etc. | ímpeto; força motriz (ou propulsora); refere-se, de maneira geral, à força que impulsiona o desenvolvimento das coisas}
  \end{phonetics}
\end{entry}

\begin{entry}{动手}{6,4}{⼒、⼿}
  \begin{phonetics}{动手}{dong4shou3}[][HSK 5]
    \definition{v.+compl.}{iniciar o trabalho; começar a trabalhar | tocar; manusear; manipular | bater; levantar a mão (para bater); espancar}
  \end{phonetics}
\end{entry}

\begin{entry}{动机}{6,6}{⼒、⽊}
  \begin{phonetics}{动机}{dong4ji1}[][HSK 5]
    \definition[部]{s.}{motivo; razão; intenção; ideias que motivam as pessoas a se envolverem em determinados comportamentos}
  \end{phonetics}
\end{entry}

\begin{entry}{动作}{6,7}{⼒、⼈}
  \begin{phonetics}{动作}{dong4zuo4}[][HSK 1]
    \definition[个]{s.}{movimento; ação; atividade de todo o corpo ou parte do corpo}
    \definition{v.}{agir; começar a se mover; entrar em ação}
  \end{phonetics}
\end{entry}

\begin{entry}{动员}{6,7}{⼒、⼝}
  \begin{phonetics}{动员}{dong4yuan2}[][HSK 5]
    \definition{v.}{despertar; mobilizar; iniciar (para fazer algo ou participar de uma atividade) | mobilizar toda a nação; transferir dos setores militar, político e econômico para uma situação de guerra}
  \end{phonetics}
\end{entry}

\begin{entry}{动身}{6,7}{⼒、⾝}
  \begin{phonetics}{动身}{dong4shen1}
    \definition{v.+compl.}{fazer uma jornada | começar uma jornada | partir | partir em uma jornada | sair (para um lugar distante)}
  \end{phonetics}
\end{entry}

\begin{entry}{动态}{6,8}{⼒、⼼}
  \begin{phonetics}{动态}{dong4tai4}[][HSK 5]
    \definition{s.}{tendências; desenvolvimentos; tendência geral dos assuntos; causa provável de ação; curso dos acontecimentos | expressão; comportamento ativo | estado dinâmico; condição dinâmica; de ou em relação a um estado de movimento}
  \end{phonetics}
\end{entry}

\begin{entry}{动物}{6,8}{⼒、⽜}
  \begin{phonetics}{动物}{dong4wu4}[][HSK 2]
    \definition[个,只,群,种]{s.}{animal; uma grande classe de seres vivos, que se alimentam principalmente de matéria orgânica, possuem sistema nervoso, são sensíveis e capazes de se mover; refere-se a todos os tipos de coisas concretas ou abstratas}
  \end{phonetics}
\end{entry}

\begin{entry}{动物园}{6,8,7}{⼒、⽜、⼞}
  \begin{phonetics}{动物园}{dong4 wu4 yuan2}[][HSK 2]
    \definition[个,座,家]{s.}{jardim zoológico; zoo; parque que cria muitos tipos de animais (especialmente animais com valor científico ou raros na região) para exibição ao público}
  \end{phonetics}
\end{entry}

\begin{entry}{动画片}{6,8,4}{⼒、⽥、⽚}
  \begin{phonetics}{动画片}{dong4hua4pian4}[][HSK 4]
    \definition[部]{s.}{desenho animado; animações; filme de animação}
  \end{phonetics}
\end{entry}

\begin{entry}{动感}{6,13}{⼒、⼼}
  \begin{phonetics}{动感}{dong4gan3}
    \definition{adj.}{dinâmica | vívida}
    \definition{adv.}{dinamicamente}
    \definition{s.}{senso de movimento (geralmente em uma obra de arte estática)}
  \end{phonetics}
\end{entry}

\begin{entry}{动摇}{6,13}{⼒、⼿}
  \begin{phonetics}{动摇}{dong4 yao2}[][HSK 4]
    \definition{adj.}{instável}
    \definition{v.}{ondular; pairar; agitar; balançar; sacudir | hesitar; vacilar; esmorecer; abalar}
  \end{phonetics}
\end{entry}

\begin{entry}{动漫}{6,14}{⼒、⽔}
  \begin{phonetics}{动漫}{dong4man4}
    \definition{s.}{desenhos animados | quadrinhos | anime | mangá}
  \end{phonetics}
\end{entry}

\begin{entry}{匈奴}{6,5}{⼓、⼥}
  \begin{phonetics}{匈奴}{xiong1nu2}
    \definition*{s.}{Xiongnu, um povo da estepe oriental que criou um império que floresceu na época das dinastias Qin e Han}
  \end{phonetics}
\end{entry}

\begin{entry}{匠}{6}{⼕}
  \begin{phonetics}{匠}{jiang4}
    \definition{s.}{artesão}
  \end{phonetics}
\end{entry}

\begin{entry}{华}{6}{⼗}
  \begin{phonetics}{华}{hua2}
    \definition*{s.}{China; refere-se à China (anteriormente conhecida como Huaxia, 华夏, mais tarde chamada de Zhonghua, 中华, ou simplesmente Hua, 华)}
    \definition{adj.}{esplêndido; magnífico | próspero; florescente | chamativo; extravagante; vaidoso | grisalho}
    \definition{s.}{corona; um halo colorido ao redor do sol ou da lua causado pela difração da luz através das nuvens | creme; melhor parte; a melhor parte das coisas | chinês; refere-se à nacionalidade Han (língua e escrita) | vezes; anos; refere-se a (bons) momentos | elixir; essência líquida; substâncias formadas pela sedimentação de minerais na água de nascente | Seu, palavra honorífica, usada para se referir a coisas relacionadas à outra pessoa}
  \seealsoref{华夏}{hua2xia4}
  \seealsoref{中华}{zhong1hua2}
  \end{phonetics}
  \begin{phonetics}{华}{hua4}
    \definition*{s.}{sobrenome Hua}
    \definition*{s.}{Huashan Mountain (na província de Shaanxi)}
  \end{phonetics}
\end{entry}

\begin{entry}{华人}{6,2}{⼗、⼈}
  \begin{phonetics}{华人}{hua2 ren2}[][HSK 3]
    \definition[名,位,个]{s.}{Chinês; chinês étnico | chineses no exterior; refere-se a cidadãos estrangeiros de ascendência chinesa que obtiveram a nacionalidade do país em que residem}
  \end{phonetics}
\end{entry}

\begin{entry}{华氏}{6,4}{⼗、⽒}
  \begin{phonetics}{华氏}{hua2shi4}
    \definition{s.}{graus Fahrenheit (°F)}
  \end{phonetics}
\end{entry}

\begin{entry}{华语}{6,9}{⼗、⾔}
  \begin{phonetics}{华语}{hua2 yu3}[][HSK 5]
    \definition*{s.}{Chinês (idioma)}
  \end{phonetics}
\end{entry}

\begin{entry}{华夏}{6,10}{⼗、⼢}
  \begin{phonetics}{华夏}{hua2xia4}
    \definition*{s.}{Huaxia, nome antigo da China | Catai}
  \end{phonetics}
\end{entry}

\begin{entry}{华盛顿}{6,11,10}{⼗、⽫、⾴}
  \begin{phonetics}{华盛顿}{hua2sheng4dun4}
    \definition*{s.}{Washington}
  \end{phonetics}
\end{entry}

\begin{entry}{华裔}{6,13}{⼗、⾐}
  \begin{phonetics}{华裔}{hua2yi4}
    \definition{s.}{descendente de chinês}
  \end{phonetics}
\end{entry}

\begin{entry}{协议}{6,5}{⼗、⾔}
  \begin{phonetics}{协议}{xie2yi4}[][HSK 5]
    \definition[项]{s.}{acordo; tratado; decisão conjunta alcançada através de negociação e consulta}
    \definition{v.}{concordar em}
  \end{phonetics}
\end{entry}

\begin{entry}{协议书}{6,5,4}{⼗、⾔、⼄}
  \begin{phonetics}{协议书}{xie2 yi4 shu1}[][HSK 5]
    \definition{s.}{contrato | protocolo}
  \end{phonetics}
\end{entry}

\begin{entry}{危}{6}{⼙}
  \begin{phonetics}{危}{wei1}
    \definition*{s.}{sobrenome Wei}
    \definition*{s.}{Wei, a décima segunda das vinte e oito constelações em que a esfera celeste foi dividida, consistindo de três estrelas em forma de triângulo obtuso, uma em Aquário e duas em Pégaso | Wei, uma das mansões lunares}
    \definition{adj.}{arriscado; inseguro; perigoso (oposto a 安) | estar gravemente doente; estar morrendo | alto; íngreme}
    \definition{s.}{perigo | cumeeira (de um telhado)}
    \definition{v.}{pôr em perigo; colocar em perigo; comprometer}
  \seealsoref{安}{an1}
  \end{phonetics}
\end{entry}

\begin{entry}{危急}{6,9}{⼙、⼼}
  \begin{phonetics}{危急}{wei1ji2}
    \definition{adj.}{crítico | desesperadora (situação)}
  \end{phonetics}
\end{entry}

\begin{entry}{危险}{6,9}{⼙、⾩}
  \begin{phonetics}{危险}{wei1xian3}[][HSK 3]
    \definition{adj.}{arriscado; perigoso}
  \end{phonetics}
\end{entry}

\begin{entry}{危害}{6,10}{⼙、⼧}
  \begin{phonetics}{危害}{wei1hai4}[][HSK 3]
    \definition[个,种]{s.}{prejuízo; perigo; dano}
    \definition{v.}{destruir; prejudicar; pôr em perigo; pôr em risco}
  \end{phonetics}
\end{entry}

\begin{entry}{危难}{6,10}{⼙、⾫}
  \begin{phonetics}{危难}{wei1nan4}
    \definition{s.}{calamidade}
  \end{phonetics}
\end{entry}

\begin{entry}{压}{6}{⼚}
  \begin{phonetics}{压}{ya1}[][HSK 3]
    \definition{v.}{pressionar; empurrar para baixo; segurar; pesar | acalmar emoções agitadas ou situações ruins; tranquilizar | intimidar; reprimir; exercer pressão sobre; usar poder, posição ou padrões morais para coagir ou restringir as pessoas, impedindo-as de se expressar, decidir ou se desenvolver livremente | aproximar-se; estar chegando perto | arquivar; deixar de lado | pressionar; metáfora para uma grande carga emocional e psicológica | superar; ultrapassar; voz, capacidade e presença mais fortes do que os outros | apostar em um determinado resultado ao jogar | pressionar; força na superfície de contato do objeto}
  \end{phonetics}
  \begin{phonetics}{压}{ya4}
    \definition{adv.}{fundamentalmente; nunca (usado principalmente em frases negativas)}
  \seealsoref{压根儿}{ya4 gen1r5}
  \end{phonetics}
\end{entry}

\begin{entry}{压力}{6,2}{⼚、⼒}
  \begin{phonetics}{压力}{ya1li4}[][HSK 3]
    \definition[份,个]{s.}{pressão; força atuando perpendicularmente à superfície de um objeto | pressão; força esmagadora; metáfora para a força que coage e intimida as pessoas (principalmente nos aspectos espirituais e psicológicos) | tensão; fardo; os encargos econômicos, psicológicos e espirituais impostos pelo mundo exterior}
  \end{phonetics}
\end{entry}

\begin{entry}{压岁钱}{6,6,10}{⼚、⼭、⾦}
  \begin{phonetics}{压岁钱}{ya1sui4qian2}
    \definition{s.}{dinheiro da sorte | dinheiro dado às crianças como presente no Ano Novo Chinês}
  \end{phonetics}
\end{entry}

\begin{entry}{压根儿}{6,10,2}{⼚、⽊、⼉}
  \begin{phonetics}{压根儿}{ya4 gen1r5}
    \definition{adv.}{fundamentalmente; nunca (usado principalmente em frases negativas)}
  \end{phonetics}
\end{entry}

\begin{entry}{压碎}{6,13}{⼚、⽯}
  \begin{phonetics}{压碎}{ya1sui4}
    \definition{v.}{esmagar em pedaços}
  \end{phonetics}
\end{entry}

\begin{entry}{压韵}{6,13}{⼚、⾳}
  \begin{phonetics}{压韵}{ya1yun4}
    \variantof{押韵}
  \end{phonetics}
\end{entry}

\begin{entry}{吃}{6}{⼝}
  \begin{phonetics}{吃}{chi1}[][HSK 1]
    \definition{s.}{alimentos; necessidades básicas}
    \definition{v.}{comer; pegar; fazer; colocar alimentos na boca, mastigar e engolir (incluindo sugar e beber) | viver; depender de algo para viver | aniquilar; eliminar (usado principalmente em jogos de guerra e jogos de tabuleiro) | esgotar; exaurir; ser um fardo; ser um esforço | absorver | sofrer; incorrer | entender; compreender | entrar um objeto em outro | expressar aceitação psicológica | fazer suas refeições; comer}
  \end{phonetics}
\end{entry}

\begin{entry}{吃力}{6,2}{⼝、⼒}
  \begin{phonetics}{吃力}{chi1li4}[][HSK 5]
    \definition{adj.}{suado; extenuante; trabalhoso; laborioso | cansado; fatigado}
  \end{phonetics}
\end{entry}

\begin{entry}{吃饭}{6,7}{⼝、⾷}
  \begin{phonetics}{吃饭}{chi1 fan4}[][HSK 1]
    \definition{v.+compl.}{comer; ter (comer) uma refeição | manter-se vivo;  ganhar a vida; refere-se à vida ou à sobrevivência em geral}
  \end{phonetics}
\end{entry}

\begin{entry}{吃屎}{6,9}{⼝、⼫}
  \begin{phonetics}{吃屎}{chi1 shi3}
    \definition{expr.}{Coma merda!}
  \end{phonetics}
\end{entry}

\begin{entry}{吃惊}{6,11}{⼝、⼼}
  \begin{phonetics}{吃惊}{chi1jing1}[][HSK 4]
    \definition{v.+compl.}{ficar assustado; ficar chocado; ficar espantado; pegar de surpresa; ficar assustado inesperadamente}
  \end{phonetics}
\end{entry}

\begin{entry}{各}{6}{⼝}
  \begin{phonetics}{各}{ge4}[][HSK 3]
    \definition{adv.}{de várias maneiras; de diversas formas; respectivamente; indica que algo é feito separadamente ou que possui uma determinada característica separadamente}
    \definition{pron.}{todo; todos; cada; refere-se a todos os indivíduos dentro de um determinado intervalo, equivalente a 每个}
  \seealsoref{每个}{mei3ge4}
  \end{phonetics}
\end{entry}

\begin{entry}{各个}{6,3}{⼝、⼈}
  \begin{phonetics}{各个}{ge4 ge4}[][HSK 4]
    \definition{adv./pron.}{cada | um a um; um após o outro}
  \end{phonetics}
\end{entry}

\begin{entry}{各地}{6,6}{⼝、⼟}
  \begin{phonetics}{各地}{ge4 di4}[][HSK 3]
    \definition{s.}{em todos os lugares; em vários locais}
  \end{phonetics}
\end{entry}

\begin{entry}{各自}{6,6}{⼝、⾃}
  \begin{phonetics}{各自}{ge4zi4}[][HSK 3]
    \definition{pron.}{por si mesmo; por conta própria; cada um por si | cada um; indica cada uma das partes envolvidas}
  \end{phonetics}
\end{entry}

\begin{entry}{各位}{6,7}{⼝、⼈}
  \begin{phonetics}{各位}{ge4 wei4}[][HSK 3]
    \definition{pron.}{todos; toda a gente; todo mundo | cada um}
  \end{phonetics}
\end{entry}

\begin{entry}{各种}{6,9}{⼝、⽲}
  \begin{phonetics}{各种}{ge4 zhong3}[][HSK 3]
    \definition{adv.}{todos os tipos; vários tipos}
  \end{phonetics}
\end{entry}

\begin{entry}{合}{6}{⼝}
  \begin{phonetics}{合}{he2}[][HSK 3]
    \definition{adj.}{todo; completo; inteiro}
    \definition{clas.}{usado para rodadas | 100ml | medida para grãos secos igual a um décimo de 升, ou um centésimo de 斗}
    \definition{s.}{casamento; união matrimonial | (astronomia) conjunção | nota da escala em Gongchepu (工尺谱), correspondente ao 5 na notação musical numerada}
    \definition{v.}{fechar | juntar; combinar (oposto de 分) | adequar-se; concordar; conformar-se a | ser igual a; somar | ser adequado}
  \seealsoref{斗}{dou4}
  \seealsoref{分}{fen1}
  \seealsoref{工尺谱}{gong1 che3 pu3}
  \seealsoref{升}{sheng1}
  \end{phonetics}
\end{entry}

\begin{entry}{合同}{6,6}{⼝、⼝}
  \begin{phonetics}{合同}{he2tong5}[][HSK 4]
    \definition[个,份]{s.}{contrato; acordo; uma disposição para observância mútua por duas ou mais partes na condução de um assunto com o objetivo de determinar seus respectivos direitos e obrigações.}
  \end{phonetics}
\end{entry}

\begin{entry}{合并}{6,6}{⼝、⼲}
  \begin{phonetics}{合并}{he2bing4}[][HSK 5]
    \definition{v.}{fundir; amalgamar; combinar várias coisas em uma coisa só | (doença) ser complicada por outra doença; uma doença levar a outra, ataques simultâneos (de várias doenças)}
  \end{phonetics}
\end{entry}

\begin{entry}{合成}{6,6}{⼝、⼽}
  \begin{phonetics}{合成}{he2cheng2}[][HSK 5]
    \definition{s.}{compor; integrar; combinar; misturar | sintetizar, reação química para transformar uma substância com uma composição simples em uma substância com uma composição complexa}
  \end{phonetics}
\end{entry}

\begin{entry}{合作}{6,7}{⼝、⼈}
  \begin{phonetics}{合作}{he2zuo4}[][HSK 3]
    \definition{v.}{cooperar; colaborar; trabalhar em conjunto; trabalhar em conjunto para realizar algo ou concluir uma tarefa}
  \end{phonetics}
\end{entry}

\begin{entry}{合法}{6,8}{⼝、⽔}
  \begin{phonetics}{合法}{he2fa3}[][HSK 3]
    \definition{adj.}{legal; legítimo; lícito;  justo; válido; em conformidade com as disposições legais}
  \end{phonetics}
\end{entry}

\begin{entry}{合宪性}{6,9,8}{⼝、⼧、⼼}
  \begin{phonetics}{合宪性}{he2xian4xing4}
    \definition{s.}{constitucionalismo}
  \end{phonetics}
\end{entry}

\begin{entry}{合适}{6,9}{⼝、⾡}
  \begin{phonetics}{合适}{he2shi4}[][HSK 2]
    \definition{adj.}{correto; adequado; apropriado; conveniente; em conformidade com a realidade ou com os requisitos objetivos}
  \end{phonetics}
\end{entry}

\begin{entry}{合格}{6,10}{⼝、⽊}
  \begin{phonetics}{合格}{he2ge2}[][HSK 3]
    \definition{adj.}{qualificado; dentro dos padrões; em conformidade com os requisitos ou normas}
  \end{phonetics}
\end{entry}

\begin{entry}{合资}{6,10}{⼝、⾙}
  \begin{phonetics}{合资}{he2zi1}
    \definition{s.}{\emph{joint-venture} com capitais mistos}
  \end{phonetics}
\end{entry}

\begin{entry}{合理}{6,11}{⼝、⽟}
  \begin{phonetics}{合理}{he2li3}[][HSK 3]
    \definition{adj.}{racional; razoável; equitativo; razoável ou lógico}
  \end{phonetics}
\end{entry}

\begin{entry}{吉}{6}{⼝}
  \begin{phonetics}{吉}{ji2}
    \definition*{s.}{sobrenome Ji}
    \definition*{s.}{Província de Jilin, 吉林}
    \definition{adj.}{sortudo; propício; auspicioso (oposto de 凶)}
  \seealsoref{吉林}{ji2lin2}
  \seealsoref{凶}{xiong1}
  \end{phonetics}
\end{entry}

\begin{entry}{吉他}{6,5}{⼝、⼈}
  \begin{phonetics}{吉他}{ji2ta1}
    \definition[把]{s.}{(empréstimo linguístico) guitarra}
  \end{phonetics}
\end{entry}

\begin{entry}{吉林}{6,8}{⼝、⽊}
  \begin{phonetics}{吉林}{ji2lin2}
    \definition*{s.}{Província de Jilin}
  \end{phonetics}
\end{entry}

\begin{entry}{同}{6}{⼝}
  \begin{phonetics}{同}{tong2}
    \definition{adj.}{junto}
    \definition{adv.}{junto com}
  \end{phonetics}
\end{entry}

\begin{entry}{同伙}{6,6}{⼝、⼈}
  \begin{phonetics}{同伙}{tong2huo3}
    \definition[个]{s.}{cúmplice | colega}
  \end{phonetics}
\end{entry}

\begin{entry}{同时}{6,7}{⼝、⽇}
  \begin{phonetics}{同时}{tong2shi2}[][HSK 2]
    \definition{conj.}{além disso; além do mais; ainda mais; indica uma relação de equivalência, geralmente com um significado mais profundo}
    \definition{s.}{enquanto isso; ao mesmo tempo}
  \end{phonetics}
\end{entry}

\begin{entry}{同事}{6,8}{⼝、⼅}
  \begin{phonetics}{同事}{tong2shi4}[][HSK 2]
    \definition[个,位,名]{s.}{companheiro; colega; colega de trabalho; pessoas que trabalham juntas}
    \definition{v.}{trabalhar no mesmo lugar; trabalhar juntos; trabalhar na mesma unidade}
  \end{phonetics}
\end{entry}

\begin{entry}{同学}{6,8}{⼝、⼦}
  \begin{phonetics}{同学}{tong2xue2}[][HSK 1]
    \definition[位,个,些]{s.}{colega de escola; colega de turma; colega de estudos; pessoas que estudam na mesma escola}
  \end{phonetics}
\end{entry}

\begin{entry}{同性恋}{6,8,10}{⼝、⼼、⼼}
  \begin{phonetics}{同性恋}{tong2xing4lian4}
    \definition{s.}{homossexualidade | pessoa gay | amor gay}
  \end{phonetics}
\end{entry}

\begin{entry}{同屋}{6,9}{⼝、⼫}
  \begin{phonetics}{同屋}{tong2wu1}
    \definition[个]{s.}{companheiro de quarto | colega de quarto}
  \end{phonetics}
\end{entry}

\begin{entry}{同砚}{6,9}{⼝、⽯}
  \begin{phonetics}{同砚}{tong2yan4}
    \definition[位,个]{s.}{colega de classe | colega estudante}
  \end{phonetics}
\end{entry}

\begin{entry}{同样}{6,10}{⼝、⽊}
  \begin{phonetics}{同样}{tong2 yang4}[][HSK 2]
    \definition{adj.}{igual; semelhante; similar; idêntico; sem diferença}
  \end{phonetics}
\end{entry}

\begin{entry}{同流合污}{6,10,6,6}{⼝、⽔、⼝、⽔}
  \begin{phonetics}{同流合污}{tong2liu2he2wu1}
    \definition{expr.}{chafurdar na lama com alguém | seguir o mau exemplo dos outros}
  \end{phonetics}
\end{entry}

\begin{entry}{同情}{6,11}{⼝、⼼}
  \begin{phonetics}{同情}{tong2qing2}[][HSK 4]
    \definition{s.}{simpatia}
    \definition{v.}{simpatizar com; solidarizar-se; compadecer-se; ter empatia emocional pelo que os outros estão passando}
  \end{phonetics}
\end{entry}

\begin{entry}{同意}{6,13}{⼝、⼼}
  \begin{phonetics}{同意}{tong2yi4}[][HSK 3]
    \definition{v.}{concordar; consentir; aprovar; concordar com; dizer sim}
  \end{phonetics}
\end{entry}

\begin{entry}{名}{6}{⼝}
  \begin{phonetics}{名}{ming2}[][HSK 2]
    \definition*{s.}{sobrenome Ming}
    \definition*{v.}{nome próprio (é) | expressar; descrever | possuir; tomar; ter}
    \definition{adj.}{notável; famoso; conhecido; renomado}
    \definition{clas.}{usado para pessoas | usado para classificação por ordem}
    \definition{s.}{nome; denominação | desculpa; pretexto | fama; reputação}
  \end{phonetics}
\end{entry}

\begin{entry}{名人}{6,2}{⼝、⼈}
  \begin{phonetics}{名人}{ming2 ren2}[][HSK 4]
    \definition{s.}{celebridade; pessoa famosa}
  \end{phonetics}
\end{entry}

\begin{entry}{名片}{6,4}{⼝、⽚}
  \begin{phonetics}{名片}{ming2pian4}[][HSK 4]
    \definition[张,盒,叠]{s.}{cartão de visita; um pedaço de papel retangular com o nome, o cargo, o endereço etc. impressos}
  \end{phonetics}
\end{entry}

\begin{entry}{名字}{6,6}{⼝、⼦}
  \begin{phonetics}{名字}{ming2zi5}[][HSK 1]
    \definition[个]{s.}{nome; nome próprio | nome (de uma coisa)}
  \end{phonetics}
\end{entry}

\begin{entry}{名单}{6,8}{⼝、⼗}
  \begin{phonetics}{名单}{ming2 dan1}[][HSK 2]
    \definition[个,份]{s.}{lista com nomes de pessoas ou nomes de organizações}
  \end{phonetics}
\end{entry}

\begin{entry}{名称}{6,10}{⼝、⽲}
  \begin{phonetics}{名称}{ming2 cheng1}[][HSK 2]
    \definition[个,种]{s.}{nomes, apelidos e formas de se referir a pessoas ou coisas}
  \end{phonetics}
\end{entry}

\begin{entry}{名牌儿}{6,12,2}{⼝、⽚、⼉}
  \begin{phonetics}{名牌儿}{ming2 pai2r5}[][HSK 4]
    \definition*{s.}{Marca famosa}
  \end{phonetics}
\end{entry}

\begin{entry}{后}{6}{⼝}
  \begin{phonetics}{后}{hou4}[][HSK 1]
    \definition*{s.}{sobrenome Hou}
    \definition{s.}{atrás; traseiro; a direção oposta àquela para a qual a pessoa está voltada; a direção oposta àquela para a qual a parte de trás de uma casa está voltada (o oposto de 前)  | depois; mais tarde no tempo; futuro (em oposição a 先 ou 前) | último | posteridade; descendência | rainha; imperatriz | governante; soberano; monarca antigo}
  \seealsoref{前}{qian2}
  \seealsoref{先}{xian1}
  \end{phonetics}
\end{entry}

\begin{entry}{后天}{6,4}{⼝、⼤}
  \begin{phonetics}{后天}{hou4 tian1}[][HSK 1]
    \definition{s.}{depois de amanhã; período em que uma pessoa ou animal vive e cresce sozinho após deixar o útero materno (em oposição a 先天)}
  \seealsoref{先天}{xian1tian1}
  \end{phonetics}
\end{entry}

\begin{entry}{后头}{6,5}{⼝、⼤}
  \begin{phonetics}{后头}{hou4 tou5}[][HSK 4]
    \definition{adv.}{posteriormente | atrás | mais tarde}
    \definition{s.}{a parte de trás | a parte traseira}
  \end{phonetics}
\end{entry}

\begin{entry}{后边}{6,5}{⼝、⾡}
  \begin{phonetics}{后边}{hou4 bian5}[][HSK 1]
    \definition{adv.}{costas; traseira; atrás}
  \end{phonetics}
\end{entry}

\begin{entry}{后年}{6,6}{⼝、⼲}
  \begin{phonetics}{后年}{hou4nian2}[][HSK 3]
    \definition{s.}{daqui a dois anos; no ano seguinte ao próximo ano}
  \end{phonetics}
\end{entry}

\begin{entry}{后来}{6,7}{⼝、⽊}
  \begin{phonetics}{后来}{hou4lai2}[][HSK 2]
    \definition{adv.}{mais tarde; depois; refere-se a um período posterior a um determinado momento no passado}
  \end{phonetics}
\end{entry}

\begin{entry}{后果}{6,8}{⼝、⽊}
  \begin{phonetics}{后果}{hou4guo3}[][HSK 3]
    \definition{s.}{consequência; resultado (geralmente negativo)}
  \end{phonetics}
\end{entry}

\begin{entry}{后面}{6,9}{⼝、⾯}
  \begin{phonetics}{后面}{hou4mian4}
    \definition{adv.}{parte de trás; retaguarda; atrás; a parte posterior do espaço ou localização | mais tarde; depois; no futuro; a parte posterior de um artigo ou discurso em relação ao que está sendo narrado no momento}
  \end{phonetics}
\end{entry}

\begin{entry}{后悔}{6,10}{⼝、⼼}
  \begin{phonetics}{后悔}{hou4hui3}[][HSK 5]
    \definition{v.}{lamentar; ter remorso; arrepender-se}
  \end{phonetics}
\end{entry}

\begin{entry}{吐}{6}{⼝}
  \begin{phonetics}{吐}{tu3}[][HSK 5]
    \definition{v.}{cuspir; sair pela boca | surgir ou aparecer pela boca ou por uma fenda | dizer; contar; falar abertamente}
  \end{phonetics}
  \begin{phonetics}{吐}{tu4}[][HSK 5]
    \definition{v.}{vomitar; sair pela boca | vomitar; expelir; metáfora para ser forçado a devolver bens usurpados}
  \end{phonetics}
\end{entry}

\begin{entry}{向}{6}{⼝}
  \begin{phonetics}{向}{xiang4}[][HSK 2]
    \definition*{s.}{sobrenome Xiang}
    \definition{adv.}{sempre; o tempo todo}
    \definition{prep.}{em direção a; para}
    \definition{s.}{direção | a janela voltada para o norte}
    \definition{v.}{encarar; virar-se para | estar do lado de; ser parcial com; tomar o partido de alguém}
  \end{phonetics}
\end{entry}

\begin{entry}{向上}{6,3}{⼝、⼀}
  \begin{phonetics}{向上}{xiang4 shang4}[][HSK 5]
    \definition{v.}{mover-se; subir; ir para um lugar mais alto; ir para um lugar mais alto em relação a um determinado ponto; ir para um desenvolvimento mais alto que o atual | avançar; continuar se aperfeiçoar; subir na vida; desenvolver-se em direção ao progresso}
  \end{phonetics}
\end{entry}

\begin{entry}{向导}{6,6}{⼝、⼨}
  \begin{phonetics}{向导}{xiang4dao3}[][HSK 5]
    \definition{s.}{guia}
    \definition{v.}{agir como um guia; mostrar a alguém o caminho; levar alguém a algum lugar}
  \end{phonetics}
\end{entry}

\begin{entry}{向汪}{6,7}{⼝、⽔}
  \begin{phonetics}{向汪}{xiang4wang1}
    \definition{v.}{esperar que}
  \end{phonetics}
\end{entry}

\begin{entry}{向往}{6,8}{⼝、⼻}
  \begin{phonetics}{向往}{xiang4wang3}
    \definition{v.}{ansiar por | esperar ansiosamente por}
  \end{phonetics}
\end{entry}

\begin{entry}{向前}{6,9}{⼝、⼑}
  \begin{phonetics}{向前}{xiang4 qian2}[][HSK 5]
    \definition{adv.}{para frente; adiante;}
    \definition{v.}{avançar; ir em direção à frente; mover-se para frente; avançar um pouco mais}
  \end{phonetics}
\end{entry}

\begin{entry}{吓}{6}{⼝}
  \begin{phonetics}{吓}{xia4}[][HSK 5]
    \definition{interj.}{interjeição que demonstra espanto; Interjeição que expressa insatisfação}
    \definition{v.}{ameaçar; intimidar; usar ameaças ou meios coercitivos para intimidar ou assustar}
  \end{phonetics}
\end{entry}

\begin{entry}{吓人}{6,2}{⼝、⼈}
  \begin{phonetics}{吓人}{xia4ren2}
    \definition{adj.}{apavorante | assustador}
    \definition{v.+compl.}{assustar-se | tomar um susto}
  \end{phonetics}
\end{entry}

\begin{entry}{吗}{6}{⼝}
  \begin{phonetics}{吗}{ma2}
    \definition{adv.}{(coloquial) que?}
  \end{phonetics}
  \begin{phonetics}{吗}{ma3}
    \definition{s.}{usada em 吗啡, morfina}
  \seealsoref{吗啡}{ma3fei1}
  \end{phonetics}
  \begin{phonetics}{吗}{ma5}[][HSK 1]
    \definition{part.}{usado no final de uma pergunta | como uma pausa em uma frase antes de introduzir o ponto principal | usado no final de uma pergunta retórica}
  \end{phonetics}
\end{entry}

\begin{entry}{吗啡}{6,11}{⼝、⼝}
  \begin{phonetics}{吗啡}{ma3fei1}
    \definition{s.}{morfina (empréstimo linguístico)}
  \end{phonetics}
\end{entry}

\begin{entry}{吸}{6}{⼝}
  \begin{phonetics}{吸}{xi1}[][HSK 4]
    \definition{v.}{inalar; inspirar; aspirar; itroduzir líquidos, gases, etc. no corpo | absorver; sugar | atrair; atrair para si mesmo; atrair (interesse, investimento etc.)}
  \end{phonetics}
\end{entry}

\begin{entry}{吸引}{6,4}{⼝、⼸}
  \begin{phonetics}{吸引}{xi1yin3}[][HSK 4]
    \definition{v.}{atrair; apelar para; chamar a atenção de outros objetos, forças ou pessoas para si mesmo}
  \end{phonetics}
\end{entry}

\begin{entry}{吸收}{6,6}{⼝、⽁}
  \begin{phonetics}{吸收}{xi1shou1}[][HSK 4]
    \definition{v.}{imbuir; absorver; assimilar; sugar;  chupar; (animais, plantas, etc.) extrair material de fora dos tecidos para o interior dos tecidos | absorver; chupar;  sugar alguma substância de fora para dentro | recrutar; alistar; inscrever-se; matricular-se; admitir; (organizações ou coletivos) aceitar novos membros | absorver; aproveitar e usar a experiência, o conhecimento, o dinheiro e outras coisas valiosas de outras pessoas | absorver; diminuir, atenuar ou eliminar determinados efeitos ou fenômenos}
  \end{phonetics}
\end{entry}

\begin{entry}{吸烟}{6,10}{⼝、⽕}
  \begin{phonetics}{吸烟}{xi1yan1}[][HSK 4]
    \definition{v.+compl.}{fumar}
  \end{phonetics}
\end{entry}

\begin{entry}{吸铁石}{6,10,5}{⼝、⾦、⽯}
  \begin{phonetics}{吸铁石}{xi1tie3shi2}
    \definition{s.}{imã | magneto}
  \seealsoref{磁铁}{ci2tie3}
  \end{phonetics}
\end{entry}

\begin{entry}{吸管}{6,14}{⼝、⽵}
  \begin{phonetics}{吸管}{xi1 guan3}[][HSK 4]
    \definition[根,个]{s.}{tubo de sucção; sugador; canudo (para beber); refere-se ao tubo fino usado para sugar bebidas | conta-gotas; pipeta; cateter para bombeamento de líquidos usando pressão de ar}
  \end{phonetics}
\end{entry}

\begin{entry}{回}{6}{⼞}
  \begin{phonetics}{回}{hui2}[][HSK 1,2]
    \definition*{s.}{grupo étnico Hui (mulçumanos chineses)}
    \definition{clas.}{usado para coisas, ações, número de vezes |  um trecho de um conto; um capítulo de um romance em capítulos | seção ou capítulo (de um livro clássico)}
    \definition{s.}{sobrenome Hui}
    \definition{v.}{circular; enrolar | retornar; voltar; voltar ao lugar de origem | dar meia-volta | responder; contestar | relatar; reportar; responder}
  \end{phonetics}
\end{entry}

\begin{entry}{回忆}{6,4}{⼞、⼼}
  \begin{phonetics}{回忆}{hui2yi4}[][HSK 5]
    \definition[个,段]{s.}{memória; lembrança de eventos ou experiências passadas}
    \definition{v.}{lembrar; recordar}
  \end{phonetics}
\end{entry}

\begin{entry}{回去}{6,5}{⼞、⼛}
  \begin{phonetics}{回去}{hui2 qu4}[][HSK 1]
    \definition{v.}{retornar; voltar; estar de volta ; (a partir da minha localização)}
  \end{phonetics}
\end{entry}

\begin{entry}{回头}{6,5}{⼞、⼤}
  \begin{phonetics}{回头}{hui2 tou2}[][HSK 5]
    \definition{adv.}{mais tarde; depois de um tempo}
    \definition{conj.}{ou então; usado no início da segunda metade de uma frase para indicar o que acontecerá se você não fizer o que fez na primeira metade da frase}
    \definition{v.}{dar a meia-volta; virar a cabeça; virar a cabeça para trás | retornar; voltar | arrepender-se; corrigir seu caminho; reconhecer e corrigir erros}
  \end{phonetics}
\end{entry}

\begin{entry}{回收}{6,6}{⼞、⽁}
  \begin{phonetics}{回收}{hui2shou1}[][HSK 5]
    \definition{v.}{reciclar; reciclar itens (geralmente resíduos ou produtos antigos) para reutilização | recuperar; recolher; recuperar o que foi emitido ou demitido}
  \end{phonetics}
\end{entry}

\begin{entry}{回报}{6,7}{⼞、⼿}
  \begin{phonetics}{回报}{hui2bao4}[][HSK 5]
    \definition{s.}{recompensa; pagamento; benefícios recebidos como resultado de assistência, esforço ou afeto | retornos; benefícios recebidos por meio de investimentos}
    \definition{v.}{pagar de volta; beneficiar pessoas ou organizações que os ajudaram ou cuidaram deles de alguma forma}
  \end{phonetics}
\end{entry}

\begin{entry}{回来}{6,7}{⼞、⽊}
  \begin{phonetics}{回来}{hui2 lai5}[][HSK 1]
    \definition{v.}{voltar; regressar (para a minha localização) | retornar; usado após um verbo, significa ``vir ao lugar original''}
  \end{phonetics}
\end{entry}

\begin{entry}{回到}{6,8}{⼞、⼑}
  \begin{phonetics}{回到}{hui2 dao4}[][HSK 1]
    \definition{v.}{retornar para; voltar e chegar (ao lugar onde estava originalmente); (após uma mudança nas circunstâncias) retornar ao estado original}
  \end{phonetics}
\end{entry}

\begin{entry}{回国}{6,8}{⼞、⼞}
  \begin{phonetics}{回国}{hui2 guo2}[][HSK 2]
    \definition{v.}{regressar ao seu país (terra natal); referindo-se a voltar do exterior}
  \end{phonetics}
\end{entry}

\begin{entry}{回信}{6,9}{⼞、⼈}
  \begin{phonetics}{回信}{hui2 xin4}[][HSK 5]
    \definition[封]{s.}{uma carta em resposta; uma mensagem verbal em resposta}
    \definition{v.+compl.}{escrever em resposta; escrever de volta; responder uma carta; responder verbalmente uma mensagem}
  \end{phonetics}
\end{entry}

\begin{entry}{回复}{6,9}{⼞、⼢}
  \begin{phonetics}{回复}{hui2 fu4}[][HSK 4]
    \definition{v.}{responder (a uma carta) | retornar ao estado normal; restaurar algo ao seu estado original}
  \end{phonetics}
\end{entry}

\begin{entry}{回家}{6,10}{⼞、⼧}
  \begin{phonetics}{回家}{hui2 jia1}[][HSK 1]
    \definition{v.}{ir (voltar) para casa; estar em casa; voltar para casa}
  \end{phonetics}
\end{entry}

\begin{entry}{回顾}{6,10}{⼞、⾴}
  \begin{phonetics}{回顾}{hui2gu4}[][HSK 5]
    \definition{v.}{olhar para trás | revisar; fazer uma retrospectiva; olhar para trás, pensar no passado}
  \end{phonetics}
\end{entry}

\begin{entry}{回旋}{6,11}{⼞、⽅}
  \begin{phonetics}{回旋}{hui2xuan2}
    \definition{v.}{circular | rodar | dar a volta}
  \end{phonetics}
\end{entry}

\begin{entry}{回答}{6,12}{⼞、⽵}
  \begin{phonetics}{回答}{hui2da2}[][HSK 1]
    \definition[个]{s.}{resposta}
    \definition{v.}{responder; explicar a questão; expressar opinião sobre a solicitação}
  \end{phonetics}
\end{entry}

\begin{entry}{回避}{6,16}{⼞、⾌}
  \begin{phonetics}{回避}{hui2bi4}
    \definition{v.}{fugir (de um problema); em direito, refere-se especificamente à não participação nos procedimentos de um caso de um oficial de justiça, etc., que tenha interesse no caso ou nas partes do caso | esquivar-se; evadir-se; evitar (encontrar alguém)}
  \end{phonetics}
\end{entry}

\begin{entry}{因}{6}{⼞}
  \begin{phonetics}{因}{yin1}
    \definition*{s.}{sobrenome Yin}
    \definition{conj.}{porque; orações de conexão, indicando relações de causa e efeito}
    \definition{prep.}{com base em; à luz de; de acordo com; a introdução da ação comportamental equivale a 按照 ou 根据}
    \definition{s.}{causa; motivo; condições em que algo ocorre ou causa um determinado resultado (em oposição a 果)}
    \definition{v.}{seguir; continuar; fazer como sempre fez | estar em conformidade com; estar de acordo com; depender; contar com}
  \seealsoref{按照}{an4zhao4}
  \seealsoref{根据}{gen1ju4}
  \seealsoref{果}{guo3}
  \end{phonetics}
\end{entry}

\begin{entry}{因为}{6,4}{⼞、⼂}
  \begin{phonetics}{因为}{yin1wei4}[][HSK 2]
    \definition{conj.}{porque; indica o motivo e a frase seguinte indica o resultado}
    \definition{prep.}{por causa de; por conta de; indica razão ou justificativa}
  \end{phonetics}
\end{entry}

\begin{entry}{因为……所以……}{6,4,8,4}{⼞、⼂、⼾、⼈}
  \begin{phonetics}{因为……所以……}{yin1wei4 suo3yi3}[][HSK 2]
    \definition{conj.}{porque\dots portanto\dots}
  \end{phonetics}
\end{entry}

\begin{entry}{因此}{6,6}{⼞、⽌}
  \begin{phonetics}{因此}{yin1ci3}[][HSK 3]
    \definition{conj.}{assim; portanto; consequentemente}
  \end{phonetics}
\end{entry}

\begin{entry}{因此就}{6,6,12}{⼞、⽌、⼪}
  \begin{phonetics}{因此就}{yin1ci3 jiu4}
    \definition{conj.}{portanto}
  \end{phonetics}
\end{entry}

\begin{entry}{因而}{6,6}{⼞、⽽}
  \begin{phonetics}{因而}{yin1'er2}[][HSK 5]
    \definition{conj.}{; como resultado; com o resultado que; conecta frases, indicando relação de causa e efeito}
  \end{phonetics}
\end{entry}

\begin{entry}{团}{6}{⼞}
  \begin{phonetics}{团}{tuan2}[][HSK 3]
    \definition*{s.}{Liga da Juventude Comunista da China; Liga}
    \definition{adj.}{redondo; circular}
    \definition{clas.}{usado para algo em forma de bola}
    \definition[个]{s.}{bolinho de massa; comida em forma de bola feita de arroz ou farinha | algo em forma de bola | grupo; corpo; sociedade; organização; um grupo envolvido em um determinado trabalho ou atividade | regimento; unidade organizacional militar, geralmente abaixo do nível de divisão e acima do nível de batalhão}
    \definition{v.}{enrolar algo para formar uma bola; rolar | reunir; unir; conglomerar}
  \end{phonetics}
\end{entry}

\begin{entry}{团长}{6,4}{⼞、⾧}
  \begin{phonetics}{团长}{tuan2 zhang3}[][HSK 5]
    \definition{s.}{comandante do regimento | chefe (ou presidente) de uma delegação, trupe, etc. | líder de uma delegação}
  \end{phonetics}
\end{entry}

\begin{entry}{团队}{6,4}{⼞、⾩}
  \begin{phonetics}{团队}{tuan2dui4}
    \definition{s.}{equipe}
  \end{phonetics}
\end{entry}

\begin{entry}{团体}{6,7}{⼞、⼈}
  \begin{phonetics}{团体}{tuan2ti3}[][HSK 3]
    \definition[种,个]{s.}{equipe; grupo; organização; um grupo de pessoas com objetivos e interesses comuns}
  \end{phonetics}
\end{entry}

\begin{entry}{团结}{6,9}{⼞、⽷}
  \begin{phonetics}{团结}{tuan2jie2}[][HSK 3]
    \definition{adj.}{unido; amigável; harmonioso; relação harmoniosa e coexistência harmoniosa}
    \definition{v.}{unir; reunir}
  \end{phonetics}
\end{entry}

\begin{entry}{在}{6}{⼟}
  \begin{phonetics}{在}{zai4}[][HSK 1]
    \definition{adv.}{em processo de; em curso de}
    \definition{prep.}{em; no (um lugar ou momento); indica tempo, local, âmbito, etc.}
    \definition{v.}{existir; estar vivo | estar em; estar no; estar em (um lugar); indica a localização de pessoas ou coisas | permanecer; ficar | depender de; residir em; repousar com | ingressar ou pertencer a uma organização; ser membro de uma organização}
  \end{phonetics}
\end{entry}

\begin{entry}{在下}{6,3}{⼟、⼀}
  \begin{phonetics}{在下}{zai4xia4}
    \definition{pron.}{eu mesmo (humildemente)}
  \end{phonetics}
\end{entry}

\begin{entry}{在于}{6,3}{⼟、⼆}
  \begin{phonetics}{在于}{zai4yu2}[][HSK 4]
    \definition{v.}{ser responsável por; caber a;  ser da competência de;  apontar a essência das coisas, ou do que elas se tratam | depender de; ser determinado por;  ser devido a (um determinado atributo)/(de um assunto a ser determinado)}
  \end{phonetics}
\end{entry}

\begin{entry}{在内}{6,4}{⼟、⼌}
  \begin{phonetics}{在内}{zai4 nei4}[][HSK 5]
    \definition{adj.}{incluido}
    \definition{adv.}{dentro; internamente; entre eles}
    \definition{v.}{ser incluído}
  \end{phonetics}
\end{entry}

\begin{entry}{在乎}{6,5}{⼟、⼃}
  \begin{phonetics}{在乎}{zai4hu5}[][HSK 4]
    \definition{v.}{preocupar-se; preocupar-se com; levar a sério | ser responsável por; caber ao; ser da competência de}
  \end{phonetics}
\end{entry}

\begin{entry}{在地}{6,6}{⼟、⼟}
  \begin{phonetics}{在地}{zai4di4}
    \definition{s.}{local}
  \end{phonetics}
\end{entry}

\begin{entry}{在场}{6,6}{⼟、⼟}
  \begin{phonetics}{在场}{zai4 chang3}[][HSK 5]
    \definition{v.}{estar presente; estar no local; estar em cena; estar presente onde as coisas estão acontecendo}
  \end{phonetics}
\end{entry}

\begin{entry}{在此}{6,6}{⼟、⽌}
  \begin{phonetics}{在此}{zai4ci3}
    \definition{adv.}{aqui}
  \end{phonetics}
\end{entry}

\begin{entry}{在行}{6,6}{⼟、⾏}
  \begin{phonetics}{在行}{zai4hang2}
    \definition{v.}{ser adepto de algo | ser um especialista em um comércio ou profissão}
  \end{phonetics}
\end{entry}

\begin{entry}{在线}{6,8}{⼟、⽷}
  \begin{phonetics}{在线}{zai4xian4}
    \definition{s.}{\emph{online}}
  \end{phonetics}
\end{entry}

\begin{entry}{在家}{6,10}{⼟、⼧}
  \begin{phonetics}{在家}{zai4 jia1}[][HSK 1]
    \definition{v.}{estar em; estar em casa; estar no local de trabalho ou alojamento; sem sair de casa | continuar sendo um leigo; permanecer leigo; para monges, freiras, taoístas e outros que 出家, as pessoas comuns são consideradas leigas}
  \seealsoref{出家}{chu1 jia1}
  \end{phonetics}
\end{entry}

\begin{entry}{在教}{6,11}{⼟、⽁}
  \begin{phonetics}{在教}{zai4jiao4}
    \definition{v.}{ser um crente (em uma religião)}
  \end{phonetics}
\end{entry}

\begin{entry}{在意}{6,13}{⼟、⼼}
  \begin{phonetics}{在意}{zai4yi4}
    \definition{v.+compl.}{preocupar-se | importar-se | levar a sério}
  \end{phonetics}
\end{entry}

\begin{entry}{地}{6}{⼟}
  \begin{phonetics}{地}{de5}[][HSK 1]
    \definition{part.}{(estrutural) utilizada antes de um verbo ou adjetivo, ligando-o ao adjunto adverbial modificador precedente}
  \end{phonetics}
  \begin{phonetics}{地}{di4}[][HSK 1]
    \definition*{s.}{sobrenome Di}
    \definition[块,片]{s.}{a Terra | terra; solo | campos | chão; piso | posição; situação | contexto; base | distância percorrida (medida em 里 ou paradas 站) | indicando estado de espírito | território | lugar; local | parte do espaço | distância}
  \end{phonetics}
\end{entry}

\begin{entry}{地上}{6,3}{⼟、⼀}
  \begin{phonetics}{地上}{di4 shang5}[][HSK 1]
    \definition{adv.}{no chão; no solo; em terra}
  \end{phonetics}
\end{entry}

\begin{entry}{地下}{6,3}{⼟、⼀}
  \begin{phonetics}{地下}{di4 xia4}[][HSK 4]
    \definition{s.}{subterrâneo | secreta (atividade) | recursos ocultos}
  \end{phonetics}
\end{entry}

\begin{entry}{地下室}{6,3,9}{⼟、⼀、⼧}
  \begin{phonetics}{地下室}{di4xia4shi4}
    \definition{s.}{subterrâneo | porão}
  \end{phonetics}
\end{entry}

\begin{entry}{地区}{6,4}{⼟、⼖}
  \begin{phonetics}{地区}{di4qu1}[][HSK 3]
    \definition[个,片]{s.}{área; distrito; região; um lugar maior | prefeitura; unidade administrativa | latitudes; localidade; lado | em determinadas circunstâncias, algumas regiões administrativas locais da China, como Hong Kong e Macau, participam individualmente em algumas atividades internacionais}
    \definition{suf.}{como sufixo do nome da cidade, significa prefeitura ou condado}
  \end{phonetics}
\end{entry}

\begin{entry}{地方}{6,4}{⼟、⽅}
  \begin{phonetics}{地方}{di4fang1}
    \definition[个]{s.}{distrito; localidade;  em oposição a 中央, o número total de unidades administrativas em todos os níveis abaixo do centro | governo local e população; refere-se a outros setores que não o militar}
  \seealsoref{中央}{zhong1yang1}
  \end{phonetics}
  \begin{phonetics}{地方}{di4fang5}[][HSK 1,4]
    \definition[个,处,块]{s.}{lugar; cômodo; área; refere-se a um espaço específico | parte}
  \end{phonetics}
\end{entry}

\begin{entry}{地位}{6,7}{⼟、⼈}
  \begin{phonetics}{地位}{di4wei4}[][HSK 4]
    \definition{s.}{lugar; status; posição; posição da pessoa ou do grupo nas relações sociais | lugar; posição (ocupada por uma pessoa ou coisa); espaço ocupado por uma pessoa ou coisa}
  \end{phonetics}
\end{entry}

\begin{entry}{地址}{6,7}{⼟、⼟}
  \begin{phonetics}{地址}{di4zhi3}[][HSK 4]
    \definition[个]{s.}{endereço; local de residência ou correspondência}
  \end{phonetics}
\end{entry}

\begin{entry}{地形}{6,7}{⼟、⼺}
  \begin{phonetics}{地形}{di4 xing2}[][HSK 5]
    \definition{s.}{topografia; forma do terreno; relevo; disposição do terreno; característica do relevo; característica da superfície; terreno}
  \end{phonetics}
\end{entry}

\begin{entry}{地图}{6,8}{⼟、⼞}
  \begin{phonetics}{地图}{di4tu2}[][HSK 1]
    \definition[张,本]{s.}{mapa; mapa que mostra a distribuição de coisas e fenômenos na superfície da Terra, com símbolos e textos, e às vezes também com cores}
  \end{phonetics}
\end{entry}

\begin{entry}{地带}{6,9}{⼟、⼱}
  \begin{phonetics}{地带}{di4 dai4}[][HSK 5]
    \definition[个]{s.}{distrito; região; zona; área de uma determinada natureza ou extensão}
  \end{phonetics}
\end{entry}

\begin{entry}{地点}{6,9}{⼟、⽕}
  \begin{phonetics}{地点}{di4dian3}[][HSK 1]
    \definition[个]{s.}{lugar; local; região; localização}
  \end{phonetics}
\end{entry}

\begin{entry}{地狱}{6,9}{⼟、⽝}
  \begin{phonetics}{地狱}{di4yu4}
    \definition*{s.}{\emph{Naraka} (Budismo)}
    \definition{adj.}{infernal}
    \definition{s.}{inferno | submundo}
  \end{phonetics}
\end{entry}

\begin{entry}{地砖}{6,9}{⼟、⽯}
  \begin{phonetics}{地砖}{di4zhuan1}
    \definition{s.}{ladrilho de piso}
  \end{phonetics}
\end{entry}

\begin{entry}{地面}{6,9}{⼟、⾯}
  \begin{phonetics}{地面}{di4 mian4}[][HSK 4]
    \definition{s.}{a superfície da Terra | térreo; piso; camada de material colocada no chão dentro e ao redor dos edifícios | localidade; chão | região; território; principalmente áreas administrativas}
  \end{phonetics}
\end{entry}

\begin{entry}{地核}{6,10}{⼟、⽊}
  \begin{phonetics}{地核}{di4he2}
    \definition{s.}{(geologia) núcleo da Terra}
  \end{phonetics}
\end{entry}

\begin{entry}{地铁}{6,10}{⼟、⾦}
  \begin{phonetics}{地铁}{di4tie3}[][HSK 2]
    \definition[条,班,列,趟]{s.}{metrô; trem subterrâneo; também se refere ao vagão do metrô}
  \end{phonetics}
\end{entry}

\begin{entry}{地铁站}{6,10,10}{⼟、⾦、⽴}
  \begin{phonetics}{地铁站}{di4 tie3 zhan4}[][HSK 2]
    \definition[个,座]{s.}{estação de metrô}
  \end{phonetics}
\end{entry}

\begin{entry}{地球}{6,11}{⼟、⽟}
  \begin{phonetics}{地球}{di4qiu2}[][HSK 2]
    \definition[个]{s.}{o planeta Terra}
  \end{phonetics}
\end{entry}

\begin{entry}{地理}{6,11}{⼟、⽟}
  \begin{phonetics}{地理}{di4li3}
    \definition{s.}{geografia}
  \end{phonetics}
\end{entry}

\begin{entry}{地震}{6,15}{⼟、⾬}
  \begin{phonetics}{地震}{di4zhen4}[][HSK 5]
    \definition[场,次,级]{s.}{sismo; terremoto; tremor de terra; vibrações na crosta terrestre}
    \definition{v.}{sacudir com vibrações sísmicas}
  \end{phonetics}
\end{entry}

\begin{entry}{场}{6}{⼟}
  \begin{phonetics}{场}{chang2}
    \definition{clas.}{usado para descrever o desenrolar dos acontecimentos}
    \definition{s.}{eira; espaço aberto e plano; um terreno plano, geralmente usado para secar grãos e moer cereais | mercado; feira rural}
  \end{phonetics}
  \begin{phonetics}{场}{chang3}[][HSK 2]
    \definition*{s.}{sobrenome Chang}
    \definition{clas.}{usado para atividades culturais, recreativas e esportivas | usado para pequenos trechos de uma peça}
    \definition{s.}{um local amplo utilizado para um fim específico | palco; campo | cena | (física) campo (por exemplo: campo manético) | (para atividades recreativas, esportivas ou outras) | um lugar onde as pessoas se reúnem | fazenda; quinta | abertura; encerramento; refere-se ao processo completo de uma apresentação ou competição | local; ponto; o local onde ocorreu o incidente}
  \end{phonetics}
\end{entry}

\begin{entry}{场合}{6,6}{⼟、⼝}
  \begin{phonetics}{场合}{chang3he2}[][HSK 3]
    \definition[个,些,种,类]{s.}{ocasião; situação; um certo tempo, lugar ou situação}
  \end{phonetics}
\end{entry}

\begin{entry}{场所}{6,8}{⼟、⼾}
  \begin{phonetics}{场所}{chang3suo3}[][HSK 3]
    \definition{s.}{lugar; sítio; arena; local da atividade}
  \end{phonetics}
\end{entry}

\begin{entry}{场面}{6,9}{⼟、⾯}
  \begin{phonetics}{场面}{chang3mian4}[][HSK 5]
    \definition[个,种,番]{s.}{espetáculo; cena (em teatro, ficção, etc.); uma cena em uma produção teatral, cinematográfica ou televisiva que consiste em um cenário, música e personagens | cena; ocasião; literatura narrativa que consiste em situações da vida em que os personagens se relacionam entre si em determinadas ocasiões | orquestra ou instrumentos de acompanhamento (em ópera); refere-se às pessoas e aos instrumentos musicais que acompanham a apresentação de uma ópera, divididos em dois tipos: música de sopro e cordas é uma cena cultural, e gongos e tambores são uma cena marcial | situação; referência geral a uma situação em um determinado contexto | frente; fachada; aparência; espetáculo superficial}
  \end{phonetics}
\end{entry}

\begin{entry}{场景}{6,12}{⼟、⽇}
  \begin{phonetics}{场景}{chang3jing3}
    \definition{s.}{cena | cenário | situação | contexto}
  \end{phonetics}
\end{entry}

\begin{entry}{多}{6}{⼣}
  \begin{phonetics}{多}{duo1}[][HSK 1,2]
    \definition*{s.}{sobrenome Duo}
    \definition{adj.}{grande quantidade (oposto de 少, 寡) | excessivo; desnecessário | excessivo; em demasia; indica um grande grau de diferença | mais do que o número correto ou necessário; em excesso}
    \definition{adv.}{acima de um valor especificado; e mais | em que medida; usado em frases interrogativas para indagar sobre grau ou quantidade, equivalente a 多么 | uma extensão não especificada; usado em frases exclamativas para expressar um alto grau, equivalente a 多么 | quase; significa que a maior parte do intervalo é assim | mais;  sobre; ímpar; usado depois de um quantificador para indicar uma fração}
    \definition{num.}{(após um número) ímpar}
    \definition{pref.}{multi- | poli-}
    \definition{v.}{ter (uma quantidade específica) a mais ou a mais (oposto a 少) | ter algo em abundância  | (em perguntas) até que ponto | (em exclamações) até que ponto | ter mais}
  \seealsoref{多么}{duo1me5}
  \seealsoref{寡}{gua3}
  \seealsoref{少}{shao3}
  \end{phonetics}
\end{entry}

\begin{entry}{多久}{6,3}{⼣、⼃}
  \begin{phonetics}{多久}{duo1 jiu3}[][HSK 2]
    \definition{pron.}{quanto tempo?; quanto tempo; perguntar quanto tempo leva}
  \end{phonetics}
\end{entry}

\begin{entry}{多么}{6,3}{⼣、⼃}
  \begin{phonetics}{多么}{duo1me5}[][HSK 2]
    \definition{adv.}{(em exclamações) como; o quê; em que medida; usado em frases exclamativas, indica um grau muito alto | em grau indeterminado; usado em frases declarativas, indica um grau mais profundo | como (usado em uma frase interrogativa para perguntar sobre grau ou número)}
  \end{phonetics}
\end{entry}

\begin{entry}{多大}{6,3}{⼣、⼤}
  \begin{phonetics}{多大}{duo1da4}
    \definition{adj.}{quantos anos? | que idade? | quão grande?}
  \end{phonetics}
\end{entry}

\begin{entry}{多云}{6,4}{⼣、⼆}
  \begin{phonetics}{多云}{duo1 yun2}[][HSK 2]
    \definition{adj.}{céu nublado; em meteorologia, refere-se a condições atmosféricas em que a cobertura de nuvens médias e baixas ocupa entre 40\% e 70\% da área do céu, ou a cobertura de nuvens altas ocupa entre 60\% e 100\% da área do céu}
  \end{phonetics}
\end{entry}

\begin{entry}{多少}{6,4}{⼣、⼩}
  \begin{phonetics}{多少}{duo1shao3}
    \definition{adv.}{um pouco; mais ou menos; até certo ponto}
    \definition{s.}{número; quantidade; volume}
  \end{phonetics}
  \begin{phonetics}{多少}{duo1shao5}[][HSK 1]
    \definition{adv.}{quantos?; quanto?; usado em perguntas para perguntar sobre quantidade | expressar uma quantidade ou número não especificado; quantidade indefinida}
  \end{phonetics}
\end{entry}

\begin{entry}{多年}{6,6}{⼣、⼲}
  \begin{phonetics}{多年}{duo1 nian2}[][HSK 4]
    \definition{adv.}{por muitos anos; durante muitos anos}
  \end{phonetics}
\end{entry}

\begin{entry}{多次}{6,6}{⼣、⽋}
  \begin{phonetics}{多次}{duo1 ci4}[][HSK 4]
    \definition{adv.}{muitas vezes; de vez em quando; repetidamente; em muitas ocasiões}
  \end{phonetics}
\end{entry}

\begin{entry}{多咱}{6,9}{⼣、⼝}
  \begin{phonetics}{多咱}{duo1 zan5}
    \definition{adv.}{que horas?; quando?}
  \end{phonetics}
\end{entry}

\begin{entry}{多种}{6,9}{⼣、⽲}
  \begin{phonetics}{多种}{duo1 zhong3}[][HSK 4]
    \definition{adj.}{diverso; vários tipos de; múltiplo; diversificado}
  \end{phonetics}
\end{entry}

\begin{entry}{多重}{6,9}{⼣、⾥}
  \begin{phonetics}{多重}{duo1chong2}
    \definition{pref.}{multi (facetado, cultural, étnico, etc.)}
  \end{phonetics}
\end{entry}

\begin{entry}{多样}{6,10}{⼣、⽊}
  \begin{phonetics}{多样}{duo1 yang4}[][HSK 4]
    \definition{adj.}{diversos; variados; diversificado}
    \definition{s.}{diversidade}
  \end{phonetics}
\end{entry}

\begin{entry}{多数}{6,13}{⼣、⽁}
  \begin{phonetics}{多数}{duo1 shu4}[][HSK 2]
    \definition{adj.}{maioria; a maioria; plural}
    \definition{pref.}{pluri-}
  \end{phonetics}
\end{entry}

\begin{entry}{夹}{6}{⼤}
  \begin{phonetics}{夹}{ga1}
    \definition{s.}{axila; sovaco; atualmente, costuma-se escrever 胳肢窝}
  \seealsoref{胳肢窝}{ga1 zhi1 wo1}
  \end{phonetics}
  \begin{phonetics}{夹}{jia1}[][HSK 5]
    \definition{s.}{clipe, grampo, pasta, etc.}
    \definition{v.}{colocar no meio; pressionar de ambos os lados; aplicar força ou ação ao mesmo objeto de ambos os lados ao mesmo tempo | misturar; mesclar; intercalar}
  \end{phonetics}
  \begin{phonetics}{夹}{jia2}
    \definition{adj.}{forrado; com camada dupla; duas camadas (roupas, colchas, etc.) | pinçado; voz deliberadamente engraçada}
  \end{phonetics}
\end{entry}

\begin{entry}{夹肢窝}{6,8,12}{⼤、⾁、⽳}
  \begin{phonetics}{夹肢窝}{jia1 zhi1 wo1}
    \definition{s.}{axila; sovaco; também escrito como 胳肢窝}
  \seealsoref{胳肢窝}{ga1 zhi1 wo1}
  \end{phonetics}
\end{entry}

\begin{entry}{夺}{6}{⼤}
  \begin{phonetics}{夺}{duo2}
    \definition{v.}{tomar à força; apreender; arrancar; roubar | forçar a passagem; empurrar para abrir | lutar por; competir por; esforçar-se por; obter primeiro | privar; perder | perder; tirar | decidir; tomar uma decisão | omitir (palavra em um texto)}
  \end{phonetics}
\end{entry}

\begin{entry}{夺冠}{6,9}{⼤、⼍}
  \begin{phonetics}{夺冠}{duo2guan4}
    \definition{v.}{apoderar-se da coroa | (fig.) ganhar um campeonato | ganhar a medalha de ouro}
  \end{phonetics}
\end{entry}

\begin{entry}{奸}{6}{⼥}
  \begin{phonetics}{奸}{jian1}
    \definition{adj.}{perverso; maligno; traiçoeiro; malicioso}
    \definition{s.}{traidor; espião | pessoa perversa; pessoa traiçoeira | relações sexuais ilícitas; comportamento sexual impróprio}
    \definition{v.}{ter relações sexuais ilícitas}
  \end{phonetics}
\end{entry}

\begin{entry}{奸夫}{6,4}{⼥、⼤}
  \begin{phonetics}{奸夫}{jian1fu1}
    \definition{s.}{homem adúltero}
  \end{phonetics}
\end{entry}

\begin{entry}{她}{6}{⼥}
  \begin{phonetics}{她}{ta1}[][HSK 1]
    \definition{pron.}{ela | ela; referir-se a coisas que se ama ou aprecia, como a pátria, a bandeira nacional, etc.}
  \end{phonetics}
\end{entry}

\begin{entry}{她们}{6,5}{⼥、⼈}
  \begin{phonetics}{她们}{ta1men5}[][HSK 1]
    \definition{pron.}{elas; referindo-se a várias mulheres: em textos escritos, use 她们 quando todas as pessoas forem mulheres e 他们 quando houver homens e mulheres}
  \seealsoref{他们}{ta1men5}
  \end{phonetics}
\end{entry}

\begin{entry}{她们的}{6,5,8}{⼥、⼈、⽩}
  \begin{phonetics}{她们的}{ta1men5 de5}
    \definition{pron.}{delas}
  \end{phonetics}
\end{entry}

\begin{entry}{她的}{6,8}{⼥、⽩}
  \begin{phonetics}{她的}{ta1 de5}
    \definition{pron.}{dela}
  \end{phonetics}
\end{entry}

\begin{entry}{好}{6}{⼥}
  \begin{phonetics}{好}{hao3}[][HSK 1,2,4]
    \definition{adj.}{bom; ótimo; agradável; vantajoso; satisfatório | amigável; gentil; amistoso; amável | saudável; bem | pronto; concluído; usado após um verbo para indicar conclusão ou perfeição | fácil (de fazer); conveniente; responsável (por)}
    \definition{adv.}{muito; bastante; tão; usado na frente de uma palavra de quantidade ou uma palavra de tempo para indicar muito ou por muito tempo | em que medida; como; usado antes de adjetivos e verbos para indicar profundidade e com exclamação}
    \definition{interj.}{O.K.; tudo bem; aprovação, acordo ou encerramento | (no início de uma frase ou oração) expressa concordância (ou desaprovação, surpresa, etc.)}
    \definition{prep.}{de modo a; para que}
    \definition{s.}{referindo-se a palavras de elogio ou aplauso | saudações; cumprimentos}
    \definition{suf.}{sufixo que indica conclusão ou prontidão | depois de um pronome significa ``olá''}
    \definition{v.}{deve; precisa; tem que; deveria | apaixonar-se}
  \end{phonetics}
  \begin{phonetics}{好}{hao4}
    \definition*{s.}{sobrenome Hao}
    \definition{adv.}{algo que acontece com frequência, que é fácil de acontecer}
    \definition{v.}{gostar; amar; ter afeição por}
  \end{phonetics}
\end{entry}

\begin{entry}{好人}{6,2}{⼥、⼈}
  \begin{phonetics}{好人}{hao3 ren2}[][HSK 2]
    \definition[个,位,名]{s.}{pessoa boa (ou excelente) (oposto de 坏人) | pessoa saudável | pessoa gentil que tenta se dar bem com todos (muitas vezes em detrimento dos princípios)}
  \seealsoref{坏人}{huai4 ren2}
  \end{phonetics}
\end{entry}

\begin{entry}{好久}{6,3}{⼥、⼃}
  \begin{phonetics}{好久}{hao3jiu3}[][HSK 2]
    \definition{adv.}{por muito tempo | por eras (no passado)}
  \end{phonetics}
\end{entry}

\begin{entry}{好友}{6,4}{⼥、⼜}
  \begin{phonetics}{好友}{hao3you3}[][HSK 4]
    \definition[位,个]{s.}{bom amigo; amigo próximo}
  \end{phonetics}
\end{entry}

\begin{entry}{好心}{6,4}{⼥、⼼}
  \begin{phonetics}{好心}{hao3xin1}
    \definition{s.}{bondade | boas intenções}
  \end{phonetics}
\end{entry}

\begin{entry}{好处}{6,5}{⼥、⼡}
  \begin{phonetics}{好处}{hao3chu4}[][HSK 2]
    \definition[个]{s.}{bom; benefício; vantagem; fatores favoráveis a pessoas ou coisas | ganho; lucro; algo que não se deveria receber, dado por outra pessoa ou obtido através de uma oportunidade; geralmente tem conotação pejorativa}
  \end{phonetics}
\end{entry}

\begin{entry}{好汉}{6,5}{⼥、⽔}
  \begin{phonetics}{好汉}{hao3han4}
    \definition[条]{s.}{herói | pessoa forte e corajosa}
  \end{phonetics}
\end{entry}

\begin{entry}{好生}{6,5}{⼥、⽣}
  \begin{phonetics}{好生}{hao3sheng1}
    \definition{adv.}{bastante; extremamente | cuidadosamente; apropriadamente}
  \end{phonetics}
\end{entry}

\begin{entry}{好用}{6,5}{⼥、⽤}
  \begin{phonetics}{好用}{hao3yong4}
    \definition{adj.}{fácil de usar | adequado ao uso}
  \end{phonetics}
\end{entry}

\begin{entry}{好吃}{6,6}{⼥、⼝}
  \begin{phonetics}{好吃}{hao3chi1}[][HSK 1]
    \definition{adj.}{bom; saboroso; delicioso; descreve o sabor agradável de algo, que as pessoas gostam de comer}
  \end{phonetics}
  \begin{phonetics}{好吃}{hao4chi1}
    \definition{v.}{ser guloso; gostar de comer boa comida}
  \end{phonetics}
\end{entry}

\begin{entry}{好多}{6,6}{⼥、⼣}
  \begin{phonetics}{好多}{hao3 duo1}[][HSK 2]
    \definition{adj.}{muitos; uma boa quantidade; uma grande quantidade; uma quantidade enorme}
    \definition{pron.}{quantos?; quanto?; frequentemente usado para perguntar sobre quantidade}
  \end{phonetics}
\end{entry}

\begin{entry}{好好}{6,6}{⼥、⼥}
  \begin{phonetics}{好好}{hao3 hao3}[][HSK 3]
    \definition{adj.}{realmente bom/bem; em perfeitas condições; quando tudo está bem}
    \definition{adv.}{diretamente; seriamente; cuidadosamente; com todo o empenho; ao máximo}
  \end{phonetics}
\end{entry}

\begin{entry}{好听}{6,7}{⼥、⼝}
  \begin{phonetics}{好听}{hao3 ting1}[][HSK 1]
    \definition{adj.}{agradável de ouvir (de som ou voz) | bom; palatável; satisfatório (de palavras)  | decente; honrado (de ações, etc.); descreve uma coisa que parece prestigiosa | interessante; descreve palavras, histórias e outras coisas interessantes}
  \end{phonetics}
\end{entry}

\begin{entry}{好运}{6,7}{⼥、⾡}
  \begin{phonetics}{好运}{hao3 yun4}[][HSK 5]
    \definition{s.}{boa sorte, fortuna ou oportunidade}
  \end{phonetics}
\end{entry}

\begin{entry}{好事}{6,8}{⼥、⼅}
  \begin{phonetics}{好事}{hao3 shi4}[][HSK 2]
    \definition[个,件]{s.}{boa ação; gentileza | (antigo) obra de caridade | acontecimento feliz; evento festivo}
  \end{phonetics}
  \begin{phonetics}{好事}{hao4 shi4}
    \definition[个,件]{s.}{intrometido; gostar de se meter na vida dos outros}
  \end{phonetics}
\end{entry}

\begin{entry}{好奇}{6,8}{⼥、⼤}
  \begin{phonetics}{好奇}{hao4qi2}[][HSK 3]
    \definition{adj.}{curioso; curiosidade e interesse por coisas não conhecidas}
    \definition{s.}{curiosidade}
    \definition{v.}{ser ou estar curioso}
  \end{phonetics}
\end{entry}

\begin{entry}{好学}{6,8}{⼥、⼦}
  \begin{phonetics}{好学}{hao3xue2}
    \definition{adj.}{fácil de aprender}
  \end{phonetics}
  \begin{phonetics}{好学}{hao4xue2}
    \definition{s.}{estudioso | erudito}
  \end{phonetics}
\end{entry}

\begin{entry}{好玩儿}{6,8,2}{⼥、⽟、⼉}
  \begin{phonetics}{好玩儿}{hao3 wan2r5}[][HSK 1]
    \definition{adj.}{divertido; interessante; capaz de despertar interesse}
  \end{phonetics}
\end{entry}

\begin{entry}{好看}{6,9}{⼥、⽬}
  \begin{phonetics}{好看}{hao3 kan4}[][HSK 1]
    \definition{adj.}{de boa aparência; agradável; bonito | interessante; descreve o enredo ou conteúdo de filmes, romances, performances, etc., como sendo cativante, agradável ou apreciável}
  \end{phonetics}
\end{entry}

\begin{entry}{好象}{6,11}{⼥、⾗}
  \begin{phonetics}{好象}{hao3xiang4}
    \variantof{好像}
  \end{phonetics}
\end{entry}

\begin{entry}{好像}{6,13}{⼥、⼈}
  \begin{phonetics}{好像}{hao3xiang4}[][HSK 2]
    \definition{adv.}{como se; um pouco parecido; como se fosse}
    \definition{v.}{parecer; ser como; parecer-se com}
  \end{phonetics}
\end{entry}

\begin{entry}{如}{6}{⼥}
  \begin{phonetics}{如}{ru2}
    \definition{conj.}{por exemplo}
  \end{phonetics}
\end{entry}

\begin{entry}{如下}{6,3}{⼥、⼀}
  \begin{phonetics}{如下}{ru2 xia4}[][HSK 5]
    \definition{adv.}{como descrito ou listado abaixo; conforme segue; conforme abaixo}
  \end{phonetics}
\end{entry}

\begin{entry}{如今}{6,4}{⼥、⼈}
  \begin{phonetics}{如今}{ru2jin1}[][HSK 4]
    \definition{s.}{agora; hoje em dia; atualmente; no presente}
  \end{phonetics}
\end{entry}

\begin{entry}{如同}{6,6}{⼥、⼝}
  \begin{phonetics}{如同}{ru2 tong2}[][HSK 5]
    \definition{v.}{parecer que. usado principalmente em metáforas}
  \end{phonetics}
\end{entry}

\begin{entry}{如此}{6,6}{⼥、⽌}
  \begin{phonetics}{如此}{ru2 ci3}[][HSK 5]
    \definition{adv.}{assim; tal; dessa forma; dessa maneira; refere-se a uma situação mencionada anteriormente, equivalente a 这样}
  \seealsoref{这样}{zhe4 yang4}
  \end{phonetics}
\end{entry}

\begin{entry}{如何}{6,7}{⼥、⼈}
  \begin{phonetics}{如何}{ru2he2}[][HSK 3]
    \definition{pron.}{como?; o que?; usado para perguntar como resolver um problema | como?; o que?; usado para perguntar sobre a situação ou obter a opinião de outras pessoas}
  \end{phonetics}
\end{entry}

\begin{entry}{如果}{6,8}{⼥、⽊}
  \begin{phonetics}{如果}{ru2guo3}[][HSK 2]
    \definition{conj.}{se; no caso de; na eventualidade de; supondo que; para expressar suposições, pode-se usar 要是 na linguagem falada.}
  \seealsoref{要是}{yao4shi5}
  \end{phonetics}
\end{entry}

\begin{entry}{如画}{6,8}{⼥、⽥}
  \begin{phonetics}{如画}{ru2hua4}
    \definition{adj.}{pitoresco}
  \end{phonetics}
\end{entry}

\begin{entry}{妆}{6}{⼥}
  \begin{phonetics}{妆}{zhuang1}
    \definition{s.}{maquiagem | adorno | enxoval | maquiagem e figurino de palco}
    \definition{v.}{maquiar-se | enfeitar-se}
  \end{phonetics}
\end{entry}

\begin{entry}{妆扮}{6,7}{⼥、⼿}
  \begin{phonetics}{妆扮}{zhuang1ban4}
    \variantof{装扮}
  \end{phonetics}
\end{entry}

\begin{entry}{妈}{6}{⼥}
  \begin{phonetics}{妈}{ma1}[][HSK 1]
    \definition[个,位]{s.}{mãe; mamãe | uma forma de tratamento para uma mulher casada uma geração mais velha | (antigo) uma forma de tratamento para uma empregada doméstica de meia-idade ou velha}
  \seealsoref{妈妈}{ma1 ma5}
  \end{phonetics}
\end{entry}

\begin{entry}{妈妈}{6,6}{⼥、⼥}
  \begin{phonetics}{妈妈}{ma1 ma5}[][HSK 1]
    \definition[个,位]{s.}{mamãe; mãe | uma forma de chamar uma mulher de meia-idade; títulos de respeito para mulheres mais velhas}
  \end{phonetics}
\end{entry}

\begin{entry}{字}{6}{⼦}
  \begin{phonetics}{字}{zi4}[][HSK 1]
    \definition[个]{s.}{palavra; caractere; texto | pronúncia (de uma palavra ou caractere); som do caractere | tipo de impressão; estilo de caligrafia; forma de um caractere escrito ou impresso; refere-se às diferentes formas dos caracteres chineses; também se refere às diferentes escolas de caligrafia | escritas; obras de caligrafia | recibo; compromisso por escrito; documento | nome de estilo masculino adotado aos vinte anos de idade | sobrenome | um número indicado num contador elétrico, contador de água, etc.; registrar dos números dos medidores de consumo de água e eletricidade}
    \definition{v.}{ficar noiva (nos tempos antigos)}
  \end{phonetics}
\end{entry}

\begin{entry}{字母}{6,5}{⼦、⽏}
  \begin{phonetics}{字母}{zi4mu3}[][HSK 4]
    \definition[个]{s.}{letra; letras de um alfabeto | caractere que representa uma consoante inicial (em fonologia)}
  \end{phonetics}
\end{entry}

\begin{entry}{字字珠玉}{6,6,10,5}{⼦、⼦、⽟、⽟}
  \begin{phonetics}{字字珠玉}{zi4zi4zhu1yu4}
    \definition{expr.}{cada palavra é uma jóia}
    \definition{s.}{escrita magnífica}
  \end{phonetics}
\end{entry}

\begin{entry}{字典}{6,8}{⼦、⼋}
  \begin{phonetics}{字典}{zi4 dian3}[][HSK 2]
    \definition[本,册,部]{s.}{dicionário de caracteres chineses (contendo verbetes de caracteres únicos, em contraste com 词典 que contém verbetes para palavras com um ou mais caracteres)}
  \seealsoref{词典}{ci2dian3}
  \end{phonetics}
\end{entry}

\begin{entry}{字眼}{6,11}{⼦、⽬}
  \begin{phonetics}{字眼}{zi4yan3}
    \definition[个]{s.}{palavras | redação}
  \end{phonetics}
\end{entry}

\begin{entry}{字脚}{6,11}{⼦、⾁}
  \begin{phonetics}{字脚}{zi4jiao3}
    \definition[典]{s.}{gancho no final da pincelada | serifa}
  \end{phonetics}
\end{entry}

\begin{entry}{存}{6}{⼦}
  \begin{phonetics}{存}{cun2}[][HSK 3]
    \definition{v.}{existir; viver; sobreviver | armazenar; manter | acumular; coletar | depositar | sair com; verificar | reservar; reter | permanecer em equilíbrio; estar em estoque | estimar; abrigar}
  \end{phonetics}
\end{entry}

\begin{entry}{存在}{6,6}{⼦、⼟}
  \begin{phonetics}{存在}{cun2zai4}[][HSK 3]
    \definition{s.}{existência; ser; ente; o mundo objetivo, que não depende da consciência humana para mudar, ou seja, a matéria}
    \definition{v.}{existir; ser; as coisas ocupam continuamente o tempo e o espaço; na verdade, ainda não desapareceram}
  \end{phonetics}
\end{entry}

\begin{entry}{存款}{6,12}{⼦、⽋}
  \begin{phonetics}{存款}{cun2 kuan3}[][HSK 5]
    \definition[笔]{s.}{depósito; poupança bancária}
    \definition{v.}{depositar dinheiro; colocar dinheiro no banco}
  \end{phonetics}
\end{entry}

\begin{entry}{孙}{6}{⼦}
  \begin{phonetics}{孙}{sun1}
    \definition*{s.}{sobrenome Sun}
    \definition{s.}{neto; neta | gerações abaixo da do neto | parentes pertencentes à geração do neto | segundo crescimento das plantas}
  \end{phonetics}
\end{entry}

\begin{entry}{孙女}{6,3}{⼦、⼥}
  \begin{phonetics}{孙女}{sun1nv3}[][HSK 4]
    \definition{s.}{filha do filho; neta}
  \end{phonetics}
\end{entry}

\begin{entry}{孙子}{6,3}{⼦、⼦}
  \begin{phonetics}{孙子}{sun1zi3}
    \definition*{s.}{Sun Tzu, também conhecido por Sun Wu (孙武), general, estrategista e filósofo autor do ``Arte da Guerra'' (孙子兵法)}
  \seealsoref{孙武}{sun1wu3}
  \seealsoref{孙子兵法}{sun1zi3 bing1fa3}
  \end{phonetics}
  \begin{phonetics}{孙子}{sun1zi5}[][HSK 4]
    \definition{s.}{filho do filho; neto}
  \end{phonetics}
\end{entry}

\begin{entry}{孙子兵法}{6,3,7,8}{⼦、⼦、⼋、⽔}
  \begin{phonetics}{孙子兵法}{sun1zi3 bing1fa3}
    \definition*{s.}{``Arte da Guerra'', escrito por Sun Tzu (孫子)}
  \seealsoref{孙武}{sun1wu3}
  \seealsoref{孙子}{sun1zi3}
  \end{phonetics}
\end{entry}

\begin{entry}{孙武}{6,8}{⼦、⽌}
  \begin{phonetics}{孙武}{sun1wu3}
    \definition*{s.}{Sun Wu, também conhecido por Sun Tzu (孙子), general, estrategista e filósofo autor do ``Arte da Guerra'' (孙子兵法)}
  \seealsoref{孙子}{sun1zi3}
  \seealsoref{孙子兵法}{sun1zi3 bing1fa3}
  \end{phonetics}
\end{entry}

\begin{entry}{宇宙}{6,8}{⼧、⼧}
  \begin{phonetics}{宇宙}{yu3zhou4}
    \definition{s.}{universo | cosmos}
  \end{phonetics}
\end{entry}

\begin{entry}{宇航员}{6,10,7}{⼧、⾈、⼝}
  \begin{phonetics}{宇航员}{yu3hang2yuan2}
    \definition{s.}{astronauta}
  \end{phonetics}
\end{entry}

\begin{entry}{守}{6}{⼧}
  \begin{phonetics}{守}{shou3}[][HSK 4]
    \definition*{s.}{sobrenome Shou}
    \definition{adv.}{próximo; perto de; perto de algum lugar em posição, perto de algum lugar}
    \definition{v.}{guardar; defender; estar presente para cuidar; não ir embora | manter vigilância; defender do ataque do oponente em uma luta ou confronto | observar; cumprir; respeitar; fazer as coisas como elas devem ser feitas | manter, observar a integridade; honrar a palavra de alguém; manter a palavra de alguém}
  \end{phonetics}
\end{entry}

\begin{entry}{守门员}{6,3,7}{⼧、⾨、⼝}
  \begin{phonetics}{守门员}{shou3men2yuan2}
    \definition{s.}{goleiro}
  \end{phonetics}
\end{entry}

\begin{entry}{安}{6}{⼧}
  \begin{phonetics}{安}{an1}[][HSK 4]
    \definition*{s.}{sobrenome An}
    \definition{adj.}{pacífico; quieto; tranquilo; calmo | seguro; protegido (oposto a 危) | com boa saúde | em paz; bem}
    \definition{adv.}{pacificamente; silenciosamente | com segurança; em segurança | (em perguntas retóricas) como?}
    \definition{pron.}{usado como pronome interrogativo, como em 哪里,怎么; 谁,何,如何}
    \definition{s.}{segurança; proteção; paz | ampère; (eletricidade) abreviação de ampère, 安培}
    \definition{v.}{tranquilizar (a mente de alguém); acalmar | contentar-se; ficar satisfeito | colocar em uma posição adequada; encontrar um lugar para | instalar; consertar; encaixar; configurar | trazer (uma acusação contra alguém); dar (a alguém um apelido); reivindicar (crédito por algo) | abrigar (uma intenção) | acalmar; estabilizar | sentir-se satisfeito e à vontade}
  \seealsoref{安培}{an1pei2}
  \seealsoref{何}{he2}
  \seealsoref{哪里}{na3 li3}
  \seealsoref{如何}{ru2he2}
  \seealsoref{谁}{shei2}
  \seealsoref{危}{wei1}
  \seealsoref{怎么}{zen3me5}
  \end{phonetics}
\end{entry}

\begin{entry}{安全}{6,6}{⼧、⼊}
  \begin{phonetics}{安全}{an1quan2}[][HSK 2]
    \definition{adj.}{seguro; protegido; sem perigo; sem ameaças; sem acidentes}
    \definition{s.}{segurança; proteção; refere-se a um estado ou conceito, geralmente indicando ausência de ameaças ou perigo}
  \end{phonetics}
\end{entry}

\begin{entry}{安神}{6,9}{⼧、⽰}
  \begin{phonetics}{安神}{an1shen2}
    \definition{v.+compl.}{acalmar os nervos | aliviar a inquietação pela tranquilização da mente e do corpo}
  \end{phonetics}
\end{entry}

\begin{entry}{安家}{6,10}{⼧、⼧}
  \begin{phonetics}{安家}{an1jia1}
    \definition{v.+compl.}{montar uma casa | estabelecer-se}
  \end{phonetics}
\end{entry}

\begin{entry}{安培}{6,11}{⼧、⼟}
  \begin{phonetics}{安培}{an1pei2}
    \definition{clas.}{A; (empréstimo linguístico) ampere; (física) unidade de corrente elétrica}
  \end{phonetics}
\end{entry}

\begin{entry}{安排}{6,11}{⼧、⼿}
  \begin{phonetics}{安排}{an1pai2}[][HSK 3]
    \definition{s.}{plano; programação; organização; tabela do plano de atividades ou horários}
    \definition{v.}{organizar (assuntos) de acordo com a sequência ou regras; tratar as coisas de acordo com uma determinada ordem ou regras | atribuir tarefas a alguém; colocar as pessoas nos cargos de trabalho determinados, conforme planejado}
  \end{phonetics}
\end{entry}

\begin{entry}{安装}{6,12}{⼧、⾐}
  \begin{phonetics}{安装}{an1zhuang1}[][HSK 3]
    \definition{v.}{instalar; consertar; configurar; fixar máquinas ou equipamentos (geralmente conjuntos) em um determinado local, de acordo com métodos e especificações específicos}
  \end{phonetics}
\end{entry}

\begin{entry}{安置}{6,13}{⼧、⽹}
  \begin{phonetics}{安置}{an1zhi4}[][HSK 4]
    \definition{v.}{providenciar; encontrar um lugar para; ajudar a estabelecer-se; colocar pessoas ou coisas em uma determinada posição ou organizá-las adequadamente}
  \end{phonetics}
\end{entry}

\begin{entry}{安静}{6,14}{⼧、⾭}
  \begin{phonetics}{安静}{an1jing4}[][HSK 2]
    \definition{adj.}{silencioso; tranquilo; sem som; sem barulho e sem algazarra}
  \end{phonetics}
\end{entry}

\begin{entry}{安慰}{6,15}{⼧、⼼}
  \begin{phonetics}{安慰}{an1wei4}[][HSK 5]
    \definition{adj.}{confortar; tranquilizar; consolar; apaziguar;}
    \definition[个]{s.}{conforto; consolo; comportamento que alivia a dor de alguém e o acalma com palavras ou gestos}
    \definition{v.}{confortar; consolar; acalmar e confortar; deixar a mente tranquila}
  \end{phonetics}
\end{entry}

\begin{entry}{寺}{6}{⼨}
  \begin{phonetics}{寺}{si4}
    \definition{s.}{Templo Budista | Mesquita}
  \end{phonetics}
\end{entry}

\begin{entry}{寺庙}{6,8}{⼨、⼴}
  \begin{phonetics}{寺庙}{si4miao4}
    \definition{s.}{templo | mosteiro | santuário}
  \end{phonetics}
\end{entry}

\begin{entry}{寻找}{6,7}{⼨、⼿}
  \begin{phonetics}{寻找}{xun2zhao3}[][HSK 4]
    \definition{v.}{buscar; procurar; pesquisar; encontrar, que pode ser usado tanto para coisas concretas quanto para coisas abstratas}
  \end{phonetics}
\end{entry}

\begin{entry}{寻求}{6,7}{⼨、⽔}
  \begin{phonetics}{寻求}{xun2 qiu2}[][HSK 5]
    \definition{v.}{procurar; perseguir; explorar; ir em busca de}
  \end{phonetics}
\end{entry}

\begin{entry}{导}{6}{⼨}
  \begin{phonetics}{导}{dao3}
    \definition[个,位,名,些]{s.}{guia turístico | diretor}
    \definition{v.}{liderar; guiar | conduzir; transmitir | ensinar; instruir; dar orientação a}
  \end{phonetics}
\end{entry}

\begin{entry}{导致}{6,10}{⼨、⾄}
  \begin{phonetics}{导致}{dao3zhi4}[][HSK 4]
    \definition{v.}{causar; levar a; dar origem a (um resultado ruim)}
  \end{phonetics}
\end{entry}

\begin{entry}{导弹}{6,11}{⼨、⼸}
  \begin{phonetics}{导弹}{dao3dan4}
    \definition[枚]{s.}{míssil (guiado)}
  \end{phonetics}
\end{entry}

\begin{entry}{导游}{6,12}{⼨、⽔}
  \begin{phonetics}{导游}{dao3you2}[][HSK 4]
    \definition[个,位,名]{s.}{guia turístico; pessoas que trabalham como guias turísticos}
    \definition{v.}{guiar; conduzir um passeio turístico}
  \end{phonetics}
\end{entry}

\begin{entry}{导演}{6,14}{⼨、⽔}
  \begin{phonetics}{导演}{dao3yan3}[][HSK 3]
    \definition[位,名,个]{s.}{diretor; pessoa que exerce a função de diretor}
    \definition{v.}{dirigir (um filme, peça, etc.); ensaio de peças teatrais ou filmagem de filmes e séries de TV; organização e orientação do trabalho de produção}
  \end{phonetics}
\end{entry}

\begin{entry}{尧}{6}{⼪}
  \begin{phonetics}{尧}{yao2}
    \definition*{s.}{Yao, um monarca lendário da China antiga}
    \definition*{s.}{sobrenome Yao}
  \end{phonetics}
\end{entry}

\begin{entry}{尽}{6}{⼫}
  \begin{phonetics}{尽}{jin3}
    \definition{adv.}{na maior extensão possível | na extremidade mais distante de | usado antes de palavras que indicam direção, o mesmo que 最 | de agora em diante}
    \definition{prep.}{dentro dos limites de}
    \definition{v.}{dar prioridade a | deixar que certas pessoas ou coisas tenham precedência}
  \seealsoref{最}{zui4}
  \end{phonetics}
  \begin{phonetics}{尽}{jin4}
    \definition*{s.}{sobrenome Jin}
    \definition{adj.}{exausto; acabado | ao máximo; ao limite | tudo; exaustivo}
    \definition{v.}{esgotar | tentar o seu melhor; fazer o melhor uso possível | morrer; falecer | terminar | chegar ao fim ao máximo; alcançar extremos}
  \end{phonetics}
\end{entry}

\begin{entry}{尽力}{6,2}{⼫、⼒}
  \begin{phonetics}{尽力}{jin4li4}[][HSK 4]
    \definition{v.+compl.}{esforçar-se ao máximo; esforçar-se ao máximo; usar toda a sua força; fazer algo com seu melhor esforço}
  \end{phonetics}
\end{entry}

\begin{entry}{尽可能}{6,5,10}{⼫、⼝、⾁}
  \begin{phonetics}{尽可能}{jin3 ke3 neng2}[][HSK 5]
    \definition{adv.}{na medida do possível; com o melhor de sua capacidade; tentar fazer algo, atingir um determinado nível ou extensão}
  \end{phonetics}
\end{entry}

\begin{entry}{尽快}{6,7}{⼫、⼼}
  \begin{phonetics}{尽快}{jin3kuai4}[][HSK 4]
    \definition{adv.}{com toda a velocidade; o mais rápido possível; o mais breve possível}
  \end{phonetics}
\end{entry}

\begin{entry}{尽量}{6,12}{⼫、⾥}
  \begin{phonetics}{尽量}{jin3liang4}[][HSK 3]
    \definition{adv.}{tanto quanto possível; da melhor maneira possível}
  \end{phonetics}
\end{entry}

\begin{entry}{尽管}{6,14}{⼫、⽵}
  \begin{phonetics}{尽管}{jin3guan3}[][HSK 5]
    \definition{adv.}{justo; livremente; faça o que quiser, não se preocupe, não há restrições de movimento ou comportamento}
    \definition{conj.}{no entanto; embora; apesar de ; normalmente usado no início de uma frase anterior para introduzir um fato, seguido de 但是, etc. para introduzir um resultado que o fato não deveria ter; às vezes, também pode ser usado no início de uma frase posterior.}
  \seealsoref{但是}{dan4 shi4}
  \end{phonetics}
\end{entry}

\begin{entry}{岁}{6}{⼭}
  \begin{phonetics}{岁}{sui4}[][HSK 1]
    \definition{clas.}{usado para anos (de idade)}
    \definition{s.}{ano (literário) | colheita do ano (literário) | idade | tempo (literário) | ano (de idade) | ano (para as colheitas)}
  \end{phonetics}
\end{entry}

\begin{entry}{岁月}{6,4}{⼭、⽉}
  \begin{phonetics}{岁月}{sui4yue4}[][HSK 5]
    \definition{s.}{anos; ano e mês; refere-se a tempo em geral}
  \end{phonetics}
\end{entry}

\begin{entry}{岂}{6}{⼭}
  \begin{phonetics}{岂}{qi3}
    \definition*{s.}{sobrenome Qi}
    \definition{adv.}{expressa uma pergunta retórica, equivalente a 哪里, 怎么 e 难道}
  \seealsoref{哪里}{na3 li3}
  \seealsoref{难道}{nan2dao4}
  \seealsoref{怎么}{zen3me5}
  \end{phonetics}
\end{entry}

\begin{entry}{岂有此理}{6,6,6,11}{⼭、⽉、⽌、⽟}
  \begin{phonetics}{岂有此理}{qi3you3ci3li3}
    \definition{interj.}{Que exorbitante! | Absurdo! | Como isso pode ser assim? | Ridículo!}
  \end{phonetics}
\end{entry}

\begin{entry}{巡逻}{6,11}{⾡、⾡}
  \begin{phonetics}{巡逻}{xun2luo2}
    \definition{s.}{patrulha}
    \definition{v.}{patrulhar (polícia, exército ou marinha)}
  \end{phonetics}
\end{entry}

\begin{entry}{师}{6}{⼱}
  \begin{phonetics}{师}{shi1}
    \definition*{s.}{sobrenome Shi}
    \definition{s.}{professor | mestre | especialista | modelo | divisão do exército}
    \definition{v.}{despachar tropas}
  \end{phonetics}
\end{entry}

\begin{entry}{师傅}{6,12}{⼱、⼈}
  \begin{phonetics}{师傅}{shi1fu5}[][HSK 5]
    \definition[个,位,名]{s.}{mestre; um trabalhador qualificado; título honorífico para pessoas habilidosas | mestre; professor (em certos ofícios); pessoas que ensinam técnicas em áreas como engenharia, comércio e teatro}
  \end{phonetics}
\end{entry}

\begin{entry}{年}{6}{⼲}
  \begin{phonetics}{年}{nian2}[][HSK 1]
    \definition*{s.}{sobrenome Nian}
    \definition{clas.}{ano; usado para calcular o número de anos}
    \definition{s.}{ano | idade | um período (época) da história | colheita anual | Ano Novo | artigos para o dia de Ano Novo | um período da vida de uma pessoa; fases da vida humana divididas por idade}
  \end{phonetics}
\end{entry}

\begin{entry}{年代}{6,5}{⼲、⼈}
  \begin{phonetics}{年代}{nian2dai4}[][HSK 3]
    \definition[个]{s.}{idade; anos; tempo; um período de tempo com características distintas na história | uma década de um século; período de dez anos}
  \end{phonetics}
\end{entry}

\begin{entry}{年级}{6,6}{⼲、⽷}
  \begin{phonetics}{年级}{nian2ji2}[][HSK 2]
    \definition[个]{s.}{série; ano; níveis divididos de acordo com o tempo de estudo dos alunos na escola}
  \end{phonetics}
\end{entry}

\begin{entry}{年纪}{6,6}{⼲、⽷}
  \begin{phonetics}{年纪}{nian2ji4}[][HSK 3]
    \definition[把,个]{s.}{idade (de uma pessoa)}
  \end{phonetics}
\end{entry}

\begin{entry}{年初}{6,7}{⼲、⾐}
  \begin{phonetics}{年初}{nian2 chu1}[][HSK 3]
    \definition{s.}{o começo do ano; os primeiros dias do ano}
  \end{phonetics}
\end{entry}

\begin{entry}{年底}{6,8}{⼲、⼴}
  \begin{phonetics}{年底}{nian2 di3}[][HSK 3]
    \definition[个]{s.}{fim de ano; o fim do ano; geralmente os últimos dias de dezembro ou o fim do ano}
  \end{phonetics}
\end{entry}

\begin{entry}{年货}{6,8}{⼲、⾙}
  \begin{phonetics}{年货}{nian2huo4}
    \definition{s.}{mercadorias vendidas no Ano Novo Chinês}
  \end{phonetics}
\end{entry}

\begin{entry}{年前}{6,9}{⼲、⼑}
  \begin{phonetics}{年前}{nian2 qian2}[][HSK 5]
    \definition{s.}{antes do final do ano; antes do ano novo}
  \end{phonetics}
\end{entry}

\begin{entry}{年度}{6,9}{⼲、⼴}
  \begin{phonetics}{年度}{nian2du4}[][HSK 5]
    \definition{s.}{ano; de acordo com a natureza e as necessidades de um negócio, há um prazo de doze meses com data de início e término definidas}
  \end{phonetics}
\end{entry}

\begin{entry}{年轻}{6,9}{⼲、⾞}
  \begin{phonetics}{年轻}{nian2qing1}[][HSK 2]
    \definition{adj.}{jovem; não muito velho (geralmente se refere a pessoas entre 10 e 20 anos)}
  \end{phonetics}
\end{entry}

\begin{entry}{年龄}{6,13}{⼲、⿒}
  \begin{phonetics}{年龄}{nian2ling2}[][HSK 5]
    \definition[个]{s.}{idade; animais, plantas e outros seres vivos vivem e crescem no mundo durante um determinado número de anos}
  \end{phonetics}
\end{entry}

\begin{entry}{并}{6}{⼲}
  \begin{phonetics}{并}{bing4}[][HSK 3,4]
    \definition{adv.}{lado a lado; igualmente; simultaneamente | (usado para reforçar uma negação) na verdade; definitivamente | mesmo assim | (usado para reforçar uma negação) na verdade; de forma alguma | todos; indica o conjunto completo, equivalente a 全部}
    \definition{conj.}{e; além disso}
    \definition{v.}{combinar; fundir; incorporar | ficar (ou colocar) lado a lado | estar paralelo a | anexar; juntar}
  \seealsoref{全部}{quan2bu4}
  \end{phonetics}
\end{entry}

\begin{entry}{并且}{6,5}{⼲、⼀}
  \begin{phonetics}{并且}{bing4qie3}[][HSK 3]
    \definition{conj.}{e; bem como; usado entre verbos, adjetivos ou frases paralelas para indicar que várias ações são realizadas ao mesmo tempo ou que propriedades existem ao mesmo tempo | além disso; além do mais; ademais; usado na segunda metade de uma frase complexa para expressar um significado adicional}
  \end{phonetics}
\end{entry}

\begin{entry}{并排}{6,11}{⼲、⼿}
  \begin{phonetics}{并排}{bing4pai2}
    \definition{adv.}{lado a lado}
  \end{phonetics}
\end{entry}

\begin{entry}{庆}{6}{⼴}
  \begin{phonetics}{庆}{qing4}
    \definition*{s.}{sobrenome Qing}
    \definition{s.}{celebração | ocasião para celebração; um aniversário que vale a pena comemorar}
    \definition{v.}{celebrar; felicitar; comemorar}
  \end{phonetics}
\end{entry}

\begin{entry}{庆祝}{6,9}{⼴、⽰}
  \begin{phonetics}{庆祝}{qing4zhu4}[][HSK 3]
    \definition{v.}{celebrar; comemorar; festejar; realizar atividades para comemorar ou celebrar festivais comuns e eventos felizes}
  \end{phonetics}
\end{entry}

\begin{entry}{延长}{6,4}{⼵、⾧}
  \begin{phonetics}{延长}{yan2chang2}[][HSK 4]
    \definition{v.}{estender; prolongar; alongar; aumentar o tempo, a distância ou a duração de algo específico}
  \end{phonetics}
\end{entry}

\begin{entry}{延伸}{6,7}{⼵、⼈}
  \begin{phonetics}{延伸}{yan2shen1}[][HSK 5]
    \definition{v.}{estender; esticar; alongar; estender-se}
  \end{phonetics}
\end{entry}

\begin{entry}{延续}{6,11}{⼵、⽷}
  \begin{phonetics}{延续}{yan2xu4}[][HSK 4]
    \definition{v.}{durar; continuar; prosseguir; continuar como antes; prolongar}
  \end{phonetics}
\end{entry}

\begin{entry}{延期}{6,12}{⼵、⽉}
  \begin{phonetics}{延期}{yan2qi1}[][HSK 4]
    \definition{v.+compl.}{atrasar; adiar; postergar}
  \end{phonetics}
\end{entry}

\begin{entry}{异常}{6,11}{⼶、⼱}
  \begin{phonetics}{异常}{yi4chang2}
    \definition{adj.}{extraordinário | anormal}
    \definition{adv.}{extraordinariamente | excepcionalmente}
    \definition{s.}{anormalidade}
  \end{phonetics}
\end{entry}

\begin{entry}{式}{6}{⼷}
  \begin{phonetics}{式}{shi4}[][HSK 5]
    \definition*{s.}{sobrenome Shi}
    \definition{s.}{tipo; estilo | forma; padrão | ritual; cerimônia | fórmula; conjunto de símbolos que expressam uma lei natural na ciência natural | humor; modo; categoria gramatical que expressa a atitude subjetiva do falante em relação ao que está sendo dito, como narrativa, imperativa e condicional}
  \end{phonetics}
\end{entry}

\begin{entry}{当}{6}{⼹}
  \begin{phonetics}{当}{dang1}[][HSK 2]
    \definition*{s.}{sobrenome Dang}
    \definition{adj.}{igual; adequado; compatível}
    \definition{interj.}{(onomatopéia) barulho metálico, som de um gongo ou sino}
    \definition{prep.}{na presença de alguém; na cara de alguém | exatamente em (um momento ou lugar); em algum momento, em algum lugar | na frente de alguém}
    \definition{s.}{topo; cume |uma lacuna no espaço ou no tempo; refere-se a um espaço ou intervalo de tempo}
    \definition{v.}{dever; ter que; dever ser | trabalhar como; servir como; ser; assumir; desempenhar a função de | suportar; aceitar; merecer | dirigir; gerenciar; estar no comando; ser responsável por;  presidir | conter; bloquear; segurar; reter; resistir}
  \end{phonetics}
  \begin{phonetics}{当}{dang4}
    \definition{adj.}{adequado; correto; apropriado | igual; o mesmo}
    \definition{pron.}{naquele mesmo (dia, etc.); refere-se ao momento em que algo aconteceu}
    \definition{s.}{algo penhorado; penhor; garantia; objetos físicos penhorados em casas de penhores}
    \definition{v.}{corresponder; ser igual a; combinar | tratar como; considerar como; tomar como | pensar que; achar que | penhorar; empréstimo com garantia real em uma loja de penhores}
  \end{phonetics}
\end{entry}

\begin{entry}{当中}{6,4}{⼹、⼁}
  \begin{phonetics}{当中}{dang1 zhong1}[][HSK 3]
    \definition{prep.}{no meio; no centro | entre; dentro}
  \end{phonetics}
\end{entry}

\begin{entry}{当代}{6,5}{⼹、⼈}
  \begin{phonetics}{当代}{dang1dai4}[][HSK 5]
    \definition{s.}{a era atual; a era contemporânea}
  \end{phonetics}
\end{entry}

\begin{entry}{当地}{6,6}{⼹、⼟}
  \begin{phonetics}{当地}{dang1di4}
    \definition{s.}{local; o lugar onde as pessoas e as coisas estão ou onde as coisas acontecem}
  \end{phonetics}
\end{entry}

\begin{entry}{当场}{6,6}{⼹、⼟}
  \begin{phonetics}{当场}{dang1chang3}[][HSK 5]
    \definition{adv.}{na hora; de imediato; na mesma hora}
  \end{phonetics}
\end{entry}

\begin{entry}{当年}{6,6}{⼹、⼲}
  \begin{phonetics}{当年}{dang1 nian2}[][HSK 5]
    \definition{adv.}{durante esse período; durante esse tempo | naquela época; naqueles dias | naqueles anos | nessa ocasião}
  \end{phonetics}
  \begin{phonetics}{当年}{dang4 nian2}
    \definition{s.}{no mesmo ano; naquele mesmo ano}
  \end{phonetics}
\end{entry}

\begin{entry}{当初}{6,7}{⼹、⾐}
  \begin{phonetics}{当初}{dang1chu1}[][HSK 3]
    \definition{s.}{no começo; originalmente; no início; em primeiro lugar; refere-se a algo que aconteceu no passado, seja em geral ou especificamente}
  \end{phonetics}
\end{entry}

\begin{entry}{当时}{6,7}{⼹、⽇}
  \begin{phonetics}{当时}{dang1shi2}[][HSK 2]
    \definition{s.}{naquela época; aquela ocasião; aquela vez; refere-se a algo que aconteceu no passado}
    \definition{v.}{ser o momento adequado; acontecer no momento certo}
  \end{phonetics}
  \begin{phonetics}{当时}{dang4shi2}
    \definition{adv.}{(depois de fazer algo ou algo acontecer) imediatamente; de imediato; agora mesmo}
  \end{phonetics}
\end{entry}

\begin{entry}{当前}{6,9}{⼹、⼑}
  \begin{phonetics}{当前}{dang1qian2}[][HSK 5]
    \definition{s.}{presente; atual}
    \definition{v.}{estar diante de alguém; estar frente a frente com alguém; na frente de, geralmente refere-se a uma situação perigosa}
  \end{phonetics}
\end{entry}

\begin{entry}{当选}{6,9}{⼹、⾡}
  \begin{phonetics}{当选}{dang1xuan3}[][HSK 5]
    \definition{v.}{ser eleito}
  \end{phonetics}
\end{entry}

\begin{entry}{当然}{6,12}{⼹、⽕}
  \begin{phonetics}{当然}{dang1ran2}[][HSK 3]
    \definition{adj.}{natural; verdadeiro; espontâneo}
    \definition{adv.}{sem dúvida; certamente; claro}
  \end{phonetics}
\end{entry}

\begin{entry}{忙}{6}{⼼}
  \begin{phonetics}{忙}{mang2}[][HSK 1]
    \definition*{s.}{sobrenome Mang}
    \definition{adj.}{ocupado; movimentado; totalmente ocupado; muitas coisas para fazer, sem tempo livre (oposto de 闲) | imperativo; ansioso; urgente}
    \definition{v.}{apressar-se; agitar-se; fazer algo com urgência e constantemente | trabalhar; fazer}
  \seealsoref{闲}{xian2}
  \end{phonetics}
\end{entry}

\begin{entry}{戏}{6}{⼽}
  \begin{phonetics}{戏}{xi4}[][HSK 5]
    \definition*{s.}{sobrenome Xi}
    \definition[场,部,出,台]{s.}{drama; peça; espetáculo; \emph{show}}
    \definition{v.}{brincar; praticar esportes; jogar | zombar; brincar; provocar}
  \end{phonetics}
\end{entry}

\begin{entry}{戏弄}{6,7}{⼽、⼶}
  \begin{phonetics}{戏弄}{xi4nong4}
    \definition{v.}{zombar de | pregar peças | provocar}
  \end{phonetics}
\end{entry}

\begin{entry}{戏法}{6,8}{⼽、⽔}
  \begin{phonetics}{戏法}{xi4fa3}
    \definition{s.}{truque de mágica | prestidigitação}
  \end{phonetics}
\end{entry}

\begin{entry}{戏耍}{6,9}{⼽、⽽}
  \begin{phonetics}{戏耍}{xi4shua3}
    \definition{v.}{divertir-me | brincar com | provocar}
  \end{phonetics}
\end{entry}

\begin{entry}{戏院}{6,9}{⼽、⾩}
  \begin{phonetics}{戏院}{xi4yuan4}
    \definition{s.}{teatro}
  \end{phonetics}
\end{entry}

\begin{entry}{戏剧}{6,10}{⼽、⼑}
  \begin{phonetics}{戏剧}{xi4ju4}[][HSK 5]
    \definition{s.}{drama; peça; teatro | roteiro; peça; cenário}
  \end{phonetics}
\end{entry}

\begin{entry}{戏剧化地}{6,10,4,6}{⼽、⼑、⼔、⼟}
  \begin{phonetics}{戏剧化地}{xi4ju4hua4di4}
    \definition{adv.}{dramaticamente | teatralmente}
  \end{phonetics}
\end{entry}

\begin{entry}{戏剧性}{6,10,8}{⼽、⼑、⼼}
  \begin{phonetics}{戏剧性}{xi4ju4xing4}
    \definition{adj.}{dramático}
  \end{phonetics}
\end{entry}

\begin{entry}{戏剧家}{6,10,10}{⼽、⼑、⼧}
  \begin{phonetics}{戏剧家}{xi4ju4jia1}
    \definition{s.}{dramaturgo}
  \end{phonetics}
\end{entry}

\begin{entry}{戏剧效果}{6,10,10,8}{⼽、⼑、⽁、⽊}
  \begin{phonetics}{戏剧效果}{xi4ju4xiao4guo3}
    \definition{s.}{efeito dramático}
  \end{phonetics}
\end{entry}

\begin{entry}{戏剧般}{6,10,10}{⼽、⼑、⾈}
  \begin{phonetics}{戏剧般}{xi4ju4ban1}
    \definition{adj.}{melodramático}
  \end{phonetics}
\end{entry}

\begin{entry}{戏剧编剧}{6,10,12,10}{⼽、⼑、⽷、⼑}
  \begin{phonetics}{戏剧编剧}{xi4ju4bian1ju4}
    \definition{s.}{dramaturgo}
  \end{phonetics}
\end{entry}

\begin{entry}{戏剧演出}{6,10,14,5}{⼽、⼑、⽔、⼐}
  \begin{phonetics}{戏剧演出}{xi4ju4yan3chu1}
    \definition{s.}{performance dramática}
  \end{phonetics}
\end{entry}

\begin{entry}{戏谑}{6,11}{⼽、⾔}
  \begin{phonetics}{戏谑}{xi4xue4}
    \definition{v.}{brincar | fazer piadas | ridicularizar}
  \end{phonetics}
\end{entry}

\begin{entry}{成}{6}{⼽}
  \begin{phonetics}{成}{cheng2}[][HSK 2]
    \definition*{s.}{sobrenome Cheng}
    \definition{adj.}{capaz; competente | totalmente crescido; totalmente desenvolvido; maduro | estabelecido; Já definido; pronto para uso | em números ou quantidades consideráveis; inteiro; suficiente: enfatiza a quantidade ou a duração}
    \definition{clas.}{um décimo}
    \definition{interj.}{O.K.; tudo bem}
    \definition{s.}{resultado; conquista}
    \definition{v.}{ter sucesso; conseguir; ser bem-sucedido | tornar-se; transformar-se | ajudar a completar; realizar}
  \end{phonetics}
\end{entry}

\begin{entry}{成人}{6,2}{⼽、⼈}
  \begin{phonetics}{成人}{cheng2ren2}[][HSK 4]
    \definition[个]{s.}{adulto; crescido; pessoa adulta}
    \definition{v.}{crescer; tornar-se adulto}
  \end{phonetics}
\end{entry}

\begin{entry}{成为}{6,4}{⼽、⼂}
  \begin{phonetics}{成为}{cheng2wei2}[][HSK 2]
    \definition{v.}{tornar-se; transformar-se; revelar-se; passar de uma situação, identidade ou estado para outro}
  \end{phonetics}
\end{entry}

\begin{entry}{成长}{6,4}{⼽、⾧}
  \begin{phonetics}{成长}{cheng2zhang3}[][HSK 3]
    \definition{v.}{crescer; amadurecer; tornar-se adulto; o desenvolvimento de seres humanos, animais ou plantas desde a infância até a maturidade}
  \end{phonetics}
\end{entry}

\begin{entry}{成功}{6,5}{⼽、⼒}
  \begin{phonetics}{成功}{cheng2gong1}[][HSK 3]
    \definition{adj.}{bem-sucedido; frutífero}
    \definition[个,次]{s.}{sucesso}
    \definition{v.}{ter sucesso; obter os resultados esperados}
  \end{phonetics}
\end{entry}

\begin{entry}{成本}{6,5}{⼽、⽊}
  \begin{phonetics}{成本}{cheng2ben3}[][HSK 5]
    \definition{s.}{custo principal; custo; custo capitalizado; custo final; primeiro custo; custo próprio; custo de produção de um produto}
  \end{phonetics}
\end{entry}

\begin{entry}{成立}{6,5}{⼽、⽴}
  \begin{phonetics}{成立}{cheng2li4}[][HSK 3]
    \definition{v.}{fundar; estabelecer; criar; (organizações, instituições, etc.) começar a existir e a funcionar | ser válido; ser sustentável; fazer sentido; (teorias, pontos de vista, razões, etc.) fundamentados e válidos}
  \end{phonetics}
\end{entry}

\begin{entry}{成交}{6,6}{⼽、⼇}
  \begin{phonetics}{成交}{cheng2jiao1}[][HSK 5]
    \definition{v.+compl.}{fechar um acordo; fazer uma barganha; concluir uma transação}
  \end{phonetics}
\end{entry}

\begin{entry}{成吉思汗}{6,6,9,6}{⼽、⼝、⼼、⽔}
  \begin{phonetics}{成吉思汗}{cheng2ji2si1han2}
    \definition*{s.}{Genghis Khan (1162-1227), fundador e governante do Império Mongol}
  \end{phonetics}
\end{entry}

\begin{entry}{成色}{6,6}{⼽、⾊}
  \begin{phonetics}{成色}{cheng2se4}
    \definition{v.}{sair-se bem | ser bem sucedido}
  \end{phonetics}
\end{entry}

\begin{entry}{成员}{6,7}{⼽、⼝}
  \begin{phonetics}{成员}{cheng2yuan2}[][HSK 3]
    \definition[个,些,名,位]{s.}{membro; membros de um grupo ou família}
  \end{phonetics}
\end{entry}

\begin{entry}{成批}{6,7}{⼽、⼿}
  \begin{phonetics}{成批}{cheng2pi1}
    \definition{s.}{em lotes | a granel}
  \end{phonetics}
\end{entry}

\begin{entry}{成果}{6,8}{⼽、⽊}
  \begin{phonetics}{成果}{cheng2guo3}[][HSK 3]
    \definition[个]{s.}{realização; resultado; conquista; recompensas no trabalho ou na carreira}
  \end{phonetics}
\end{entry}

\begin{entry}{成活}{6,9}{⼽、⽔}
  \begin{phonetics}{成活}{cheng2huo2}
    \definition{v.}{sobreviver}
  \end{phonetics}
\end{entry}

\begin{entry}{成语}{6,9}{⼽、⾔}
  \begin{phonetics}{成语}{cheng2yu3}[][HSK 5]
    \definition{s.}{expressão idiomática; frase de conjunto (frases de quatro caracteres em chinês, geralmente com alusões literárias)}
  \end{phonetics}
\end{entry}

\begin{entry}{成家}{6,10}{⼽、⼧}
  \begin{phonetics}{成家}{cheng2jia1}
    \definition{v.}{tornar-se um especialista reconhecido | estabelecer-se e casar-se (de um homem)}
  \end{phonetics}
\end{entry}

\begin{entry}{成效}{6,10}{⼽、⽁}
  \begin{phonetics}{成效}{cheng2xiao4}[][HSK 5]
    \definition{s.}{efeito; resultado}
  \end{phonetics}
\end{entry}

\begin{entry}{成都}{6,10}{⼽、⾢}
  \begin{phonetics}{成都}{cheng2du1}
    \definition*{s.}{Chengdu}
  \end{phonetics}
\end{entry}

\begin{entry}{成婚}{6,11}{⼽、⼥}
  \begin{phonetics}{成婚}{cheng2hun1}
    \definition{v.}{casar-se}
  \end{phonetics}
\end{entry}

\begin{entry}{成绩}{6,11}{⼽、⽷}
  \begin{phonetics}{成绩}{cheng2ji4}[][HSK 2]
    \definition[项,个]{s.}{realização; sucesso; resultado (de trabalho ou estudo); refere-se à pontuação obtida em exames e competições; classificação, também se refere aos resultados alcançados no trabalho}
  \end{phonetics}
\end{entry}

\begin{entry}{成就}{6,12}{⼽、⼪}
  \begin{phonetics}{成就}{cheng2jiu4}[][HSK 3]
    \definition[个,项]{s.}{realização; conquista; sucesso; realizações profissionais}
    \definition{v.}{realizar; alcançar; completar; concluir (carreira)}
  \end{phonetics}
\end{entry}

\begin{entry}{成熟}{6,15}{⼽、⽕}
  \begin{phonetics}{成熟}{cheng2shu2}[][HSK 3]
    \definition{adj.}{maduro; amadurecido; totalmente desenvolvido; descreve que as oportunidades, condições, etc. estão perfeitas e que não haverá nenhum problema}
    \definition{v.}{amadurecer; atingir a maturidade; estar totalmente desenvolvido; frutas e outros frutos totalmente maduros, referindo-se ao desenvolvimento completo de organismos vivos}
  \end{phonetics}
\end{entry}

\begin{entry}{成器}{6,16}{⼽、⼝}
  \begin{phonetics}{成器}{cheng2qi4}
    \definition{v.}{tornar-se uma pessoa digna de respeito | fazer algo de si mesmo}
  \end{phonetics}
\end{entry}

\begin{entry}{扛}{6}{⼿}
  \begin{phonetics}{扛}{gang1}
    \definition{v.}{levantar com as duas mãos | carregar alguma coisa juntos (duas ou mais pessoas)}
  \end{phonetics}
  \begin{phonetics}{扛}{kang2}
    \definition{v.}{carregar objetos nos ombros |  suportar; aguentar | lidar; assumir}
  \end{phonetics}
\end{entry}

\begin{entry}{执行}{6,6}{⼿、⾏}
  \begin{phonetics}{执行}{zhi2xing2}[][HSK 5]
    \definition{v.}{executar; implementar; realizar}
  \end{phonetics}
\end{entry}

\begin{entry}{执着}{6,11}{⼿、⽬}
  \begin{phonetics}{执着}{zhi2zhuo2}
    \definition{s.}{(budismo) apego}
    \definition{v.}{estar fortemente apegado a | ser dedicado | apegar-se a}
  \end{phonetics}
\end{entry}

\begin{entry}{扩}{6}{⼿}
  \begin{phonetics}{扩}{kuo4}
    \definition{v.}{expandir; ampliar; estender; alargar}
  \end{phonetics}
\end{entry}

\begin{entry}{扩大}{6,3}{⼿、⼤}
  \begin{phonetics}{扩大}{kuo4da4}[][HSK 4]
    \definition{v.}{ampliar; expandir; estender; alargar}
  \end{phonetics}
\end{entry}

\begin{entry}{扩展}{6,10}{⼿、⼫}
  \begin{phonetics}{扩展}{kuo4 zhan3}[][HSK 4]
    \definition{v.}{esticar; expandir; estender; espalhar}
  \end{phonetics}
\end{entry}

\begin{entry}{扫}{6}{⼿}
  \begin{phonetics}{扫}{sao3}[][HSK 4]
    \definition{v.}{varrer; limpar | passar rapidamente ao longo ou sobre; varrer | juntar tudo}
  \end{phonetics}
  \begin{phonetics}{扫}{sao4}
  \seealsoref{扫帚}{sao4zhou5}
  \end{phonetics}
\end{entry}

\begin{entry}{扫兴}{6,6}{⼿、⼋}
  \begin{phonetics}{扫兴}{sao3xing4}
    \definition{v.+compl.}{sentir-se decepcionado | entristecer alguém}
  \end{phonetics}
\end{entry}

\begin{entry}{扫帚}{6,8}{⼿、⼱}
  \begin{phonetics}{扫帚}{sao4zhou5}
    \definition[把]{s.}{vassoura; ferramenta de varredura feita de varas de bambu, etc., maior que uma vassora}
  \end{phonetics}
\end{entry}

\begin{entry}{扬雄}{6,12}{⼿、⾫}
  \begin{phonetics}{扬雄}{yang2xiong2}
    \definition*{s.}{Yang Xiong (53 AC-18 DC), estudioso, poeta e lexicógrafo, autor do primeiro dicionário de dialeto chinês 方言}
  \seealsoref{方言}{fang1yan2}
  \end{phonetics}
\end{entry}

\begin{entry}{收}{6}{⽁}
  \begin{phonetics}{收}{shou1}[][HSK 2]
    \definition{expr.}{aos cuidados de (usado na linha de endereço após o nome)}
    \definition{v.}{recolocar; juntar; reunir e juntar coisas espalhadas ou dispersas | recolher; cobrar | ganhar; obter (benefícios econômicos) | colher; recolher; colher ou cortar frutas, legumes, cereais maduros, etc. | aceitar; receber; acolher | controlar; restringir; restringir, controlar os sentimentos ou ações, para voltar ao estado normal | finalizar; parar; concluir; encerrar | prender; deter; colocar sob custódia}
  \end{phonetics}
\end{entry}

\begin{entry}{收入}{6,2}{⽁、⼊}
  \begin{phonetics}{收入}{shou1ru4}[][HSK 2]
    \definition[笔,个]{s.}{renda; salário; dinheiro recebido}
    \definition{v.}{receber dinheiro | coletar; receber}
  \end{phonetics}
\end{entry}

\begin{entry}{收买}{6,6}{⽁、⼄}
  \begin{phonetics}{收买}{shou1mai3}
    \definition{v.}{subornar | comprar}
  \end{phonetics}
\end{entry}

\begin{entry}{收回}{6,6}{⽁、⼞}
  \begin{phonetics}{收回}{shou1 hui2}[][HSK 4]
    \definition{v.}{retomar; recuperar; relembrar; recordar; receber de volta o que foi enviado ou emprestado, ou o dinheiro que foi emprestado ou usado | sacar; retirar; recolher; rescindir; cancelar (uma opinião, ordem, etc.)}
  \end{phonetics}
\end{entry}

\begin{entry}{收听}{6,7}{⽁、⼝}
  \begin{phonetics}{收听}{shou1 ting1}[][HSK 3]
    \definition{v.}{ouvir (rádio)}
  \end{phonetics}
\end{entry}

\begin{entry}{收到}{6,8}{⽁、⼑}
  \begin{phonetics}{收到}{shou1 dao4}[][HSK 2]
    \definition{v.}{conseguir; obter; receber; alcançar}
  \end{phonetics}
\end{entry}

\begin{entry}{收购}{6,8}{⽁、⾙}
  \begin{phonetics}{收购}{shou1 gou4}[][HSK 5]
    \definition{v.}{comprar; adquirir; comprar muito em vários lugares | adquirir uma empresa; obter o controle efetivo de uma empresa por meio de dinheiro, transações de ações, etc.}
  \end{phonetics}
\end{entry}

\begin{entry}{收拾}{6,9}{⽁、⼿}
  \begin{phonetics}{收拾}{shou1shi5}[][HSK 5]
    \definition{v.}{arrumar; empacotar; limpar; organizar, policiar, restaurar a normalidade em situações adversas | consertar; reparar; restaurar algo que está danificado ao seu estado ou função original |  punir; punir alguém, geralmente com medidas mais severas | matar}
  \end{phonetics}
\end{entry}

\begin{entry}{收看}{6,9}{⽁、⽬}
  \begin{phonetics}{收看}{shou1 kan4}[][HSK 3]
    \definition{v.}{assistir (a um programa de TV)}
  \end{phonetics}
\end{entry}

\begin{entry}{收费}{6,9}{⽁、⾙}
  \begin{phonetics}{收费}{shou1 fei4}[][HSK 3]
    \definition{v.}{cobrar; cobrar taxas}
  \end{phonetics}
\end{entry}

\begin{entry}{收音机}{6,9,6}{⽁、⾳、⽊}
  \begin{phonetics}{收音机}{shou1yin1ji1}[][HSK 3]
    \definition[部,台]{s.}{rádio; sem fio; um termo geral para receptores de rádio}
  \end{phonetics}
\end{entry}

\begin{entry}{收益}{6,10}{⽁、⽫}
  \begin{phonetics}{收益}{shou1yi4}[][HSK 4]
    \definition{s.}{lucro; renda; benefício; ganhos; vantagens ou benefícios obtidos}
  \end{phonetics}
\end{entry}

\begin{entry}{收获}{6,10}{⽁、⾋}
  \begin{phonetics}{收获}{shou1huo4}[][HSK 4]
    \definition[次,番,份]{s.}{resultados; ganhos; metaforicamente falando, conhecimento, experiência, etc. obtidos em estudo ou trabalho; os resultados obtidos por meio de trabalho árduo | colheita; colheita de safras}
    \definition{v.}{colher; juntar as colheitas;}
  \end{phonetics}
\end{entry}

\begin{entry}{收据}{6,11}{⽁、⼿}
  \begin{phonetics}{收据}{shou1ju4}
    \definition[张]{s.}{recibo | \emph{voucher}}
  \end{phonetics}
\end{entry}

\begin{entry}{收敛}{6,11}{⽁、⽁}
  \begin{phonetics}{收敛}{shou1lian3}
    \definition{v.}{diminuir | desaparecer | fazer desaparecer | exercer restrição | conter (alegria, arrogância, etc.) | constringir | (matemática) convergir}
  \end{phonetics}
\end{entry}

\begin{entry}{收集}{6,12}{⽁、⾫}
  \begin{phonetics}{收集}{shou1 ji2}[][HSK 5]
    \definition{v.}{coletar; reunir; recolher}
  \end{phonetics}
\end{entry}

\begin{entry}{早}{6}{⽇}
  \begin{phonetics}{早}{zao3}[][HSK 1]
    \definition{adj.}{precoce; antes do previsto ou planejado; antes do tempo; antes de um determinado momento |}
    \definition{adv.}{há muito tempo; desde cedo; por muito tempo; há muito tempo atrás}
    \definition{interj.}{bom dia; saudações, usadas para cumprimentar uns aos outros ao se encontrarem pela manhã}
    \definition[个]{s.}{manhã}
  \end{phonetics}
\end{entry}

\begin{entry}{早上}{6,3}{⽇、⼀}
  \begin{phonetics}{早上}{zao3shang5}[][HSK 1]
    \definition[个]{s.}{de manhã cedo; madrugada; o período antes e depois do nascer do sol; geralmente, desde o amanhecer até às 8h ou 9h da manhã; às vezes também se refere ao período entre o amanhecer e o meio-dia}
  \end{phonetics}
\end{entry}

\begin{entry}{早亡}{6,3}{⽇、⼇}
  \begin{phonetics}{早亡}{zao3wang2}
    \definition[个]{s.}{morte prematura}
    \definition{v.}{morrer prematuramente}
  \end{phonetics}
\end{entry}

\begin{entry}{早已}{6,3}{⽇、⼰}
  \begin{phonetics}{早已}{zao3 yi3}[][HSK 3]
    \definition{adv.}{há muito tempo; por muito tempo | (dialeto) no passado}
  \end{phonetics}
\end{entry}

\begin{entry}{早车}{6,4}{⽇、⾞}
  \begin{phonetics}{早车}{zao3che1}
    \definition{s.}{trem matutino | ônibus matutino}
  \end{phonetics}
\end{entry}

\begin{entry}{早安}{6,6}{⽇、⼧}
  \begin{phonetics}{早安}{zao3'an1}
    \definition{interj.}{Bom dia!}
  \end{phonetics}
\end{entry}

\begin{entry}{早早儿}{6,6,2}{⽇、⽇、⼉}
  \begin{phonetics}{早早儿}{zao3zao3r5}
    \definition{adv.}{o mais cedo possível | o mais breve possível}
  \end{phonetics}
\end{entry}

\begin{entry}{早饭}{6,7}{⽇、⾷}
  \begin{phonetics}{早饭}{zao3 fan4}[][HSK 1]
    \definition[份,顿]{s.}{o café da manhã}
  \end{phonetics}
\end{entry}

\begin{entry}{早知}{6,8}{⽇、⽮}
  \begin{phonetics}{早知}{zao3zhi1}
    \definition{v.}{prever | se alguém soubesse antes, \dots}
  \end{phonetics}
\end{entry}

\begin{entry}{早前}{6,9}{⽇、⼑}
  \begin{phonetics}{早前}{zao3qian2}
    \definition{adv.}{previamente}
  \end{phonetics}
\end{entry}

\begin{entry}{早晚}{6,11}{⽇、⽇}
  \begin{phonetics}{早晚}{zao3 wan3}
    \definition{adv./s.}{manhã e noite | mais cedo ou mais tarde; cedo ou tarde | algum tempo no futuro; algum dia; em algum momento no futuro}
  \end{phonetics}
\end{entry}

\begin{entry}{早晨}{6,11}{⽇、⽇}
  \begin{phonetics}{早晨}{zao3 chen2}[][HSK 2]
    \definition[个,段,番]{s.}{manhã cedo; manhãzinha; o período do amanhecer às oito ou nove horas; às vezes, o período da meia-noite ao meio-dia}
  \end{phonetics}
\end{entry}

\begin{entry}{早就}{6,12}{⽇、⼪}
  \begin{phonetics}{早就}{zao3 jiu4}[][HSK 2]
    \definition{adv.}{já; há muito tempo; há muito tempo atrás}
  \end{phonetics}
\end{entry}

\begin{entry}{早期}{6,12}{⽇、⽉}
  \begin{phonetics}{早期}{zao3 qi1}[][HSK 5]
    \definition{s.}{prófase; estágio inicial; fase inicial; a fase inicial de uma determinada época, processo ou vida de uma pessoa}
  \end{phonetics}
\end{entry}

\begin{entry}{早餐}{6,16}{⽇、⾷}
  \begin{phonetics}{早餐}{zao3 can1}[][HSK 2]
    \definition[份,桌,顿]{s.}{café da manhã; desejum}
  \end{phonetics}
\end{entry}

\begin{entry}{曲}{6}{⽈}
  \begin{phonetics}{曲}{qu1}
    \definition*{s.}{sobrenome Qu}
    \definition{adj.}{dobrado; curvado (oposto a 直) | errado; injustificável | torto}
    \definition{v.}{dobrar | torcer}
  \seealsoref{直}{zhi2}
  \end{phonetics}
\end{entry}

\begin{entry}{曲棍球}{6,12,11}{⽈、⽊、⽟}
  \begin{phonetics}{曲棍球}{qu1gun4qiu2}
    \definition{s.}{hóquei em campo}
  \end{phonetics}
\end{entry}

\begin{entry}{有}{6}{⽉}
  \begin{phonetics}{有}{you3}[][HSK 1]
    \definition*{s.}{sobrenome You}
    \definition{pref.}{usado antes do nome de certas dinastias ou etnias}
    \definition{v.}{ter; possuir; indica posse ou propriedade | existe; há; indica que certas coisas existem em certos lugares | fazer uma estimativa ou uma comparação; expressar estimativa ou comparação | indicar ação; indica que algo aconteceu ou ocorreu | usado antes de substantivos abstratos, indica quantidade ou grandeza | em termos gerais, semelhante a 某; refere-se de maneira geral a algo semelhante | usado antes de pessoa, hora e lugar, indica a existência parcial | usado antes de certos verbos para formar uma expressão idiomática, indicando cortesia, polidez}
  \seealsoref{某}{mou3}
  \end{phonetics}
\end{entry}

\begin{entry}{有(一)些}{6,1,8}{⽉、⼀、⼆}
  \begin{phonetics}{有(一)些}{you3 (yi4) xie1}[][HSK 1]
    \definition{adv.}{em vez disso; em vez de; de certa forma}
    \definition{pron.}{de certa forma}
  \seealsoref{有些}{you3 xie1}
  \end{phonetics}
\end{entry}

\begin{entry}{有(一)点儿}{6,1,9,2}{⽉、⼀、⽕、⼉}
  \begin{phonetics}{有(一)点儿}{you3 yi4 dian3r5}[][HSK 2]
    \definition{adv.}{um pouco (有点儿 + {s.} ou {v. mental})}
  \seealsoref{有点儿}{you3 dian3r5}
  \end{phonetics}
\end{entry}

\begin{entry}{有人}{6,2}{⽉、⼈}
  \begin{phonetics}{有人}{you3 ren2}[][HSK 2]
    \definition{adj.}{ocupado (como no banheiro)}
    \definition{pron.}{qualquer um; alguém}
    \definition[所]{s.}{pessoas}
    \definition{v.}{ter alguém ali}
  \end{phonetics}
\end{entry}

\begin{entry}{有力}{6,2}{⽉、⼒}
  \begin{phonetics}{有力}{you3 li4}[][HSK 5]
    \definition{adj.}{forte; vigoroso; poderoso; energético}
  \end{phonetics}
\end{entry}

\begin{entry}{有用}{6,5}{⽉、⽤}
  \begin{phonetics}{有用}{you3yong4}[][HSK 1]
    \definition{adj.}{útil; prático; funcional}
  \end{phonetics}
\end{entry}

\begin{entry}{有名}{6,6}{⽉、⼝}
  \begin{phonetics}{有名}{you3ming2}[][HSK 1]
    \definition{adj.}{conhecido; famoso; célebre; nome conhecido por todos}
  \end{phonetics}
\end{entry}

\begin{entry}{有名无实}{6,6,4,8}{⽉、⼝、⽆、⼧}
  \begin{phonetics}{有名无实}{you3ming2wu2shi2}
    \definition{v.}{(literal) tem um nome, mas não tem realidade | existe apenas no nome}
  \end{phonetics}
\end{entry}

\begin{entry}{有利}{6,7}{⽉、⼑}
  \begin{phonetics}{有利}{you3li4}[][HSK 3]
    \definition{adj.}{benéfico; favorável; vantajoso}
  \end{phonetics}
\end{entry}

\begin{entry}{有利于}{6,7,3}{⽉、⼑、⼆}
  \begin{phonetics}{有利于}{you3 li4 yu2}[][HSK 5]
    \definition{prep.}{em favor de; fazer para; em benefício de; aproveitar}
  \end{phonetics}
\end{entry}

\begin{entry}{有劲儿}{6,7,2}{⽉、⼒、⼉}
  \begin{phonetics}{有劲儿}{you3 jin4er5}[][HSK 4]
    \definition{adj.}{forte; enérgico; energético}
  \end{phonetics}
\end{entry}

\begin{entry}{有劳}{6,7}{⽉、⼒}
  \begin{phonetics}{有劳}{you3lao2}
    \definition{v.}{posso incomodá-lo; desculpe incomodá-lo | (educado) obrigado pelo seu trabalho (usado ao pedir um favor ou após ter recebido um)}
  \end{phonetics}
\end{entry}

\begin{entry}{有时}{6,7}{⽉、⽇}
  \begin{phonetics}{有时}{you3 shi2}[][HSK 1]
    \definition{expr.}{às vezes; ocasionalmente; de vez em quando}
  \seealsoref{有的时候}{you3 de5 shi2 hou4}
  \seealsoref{有时候}{you3 shi2 hou5}
  \end{phonetics}
\end{entry}

\begin{entry}{有时……有时……}{6,7,6,7}{⽉、⽇、⽉、⽇}
  \begin{phonetics}{有时……有时……}{you3shi2 you3shi2}
    \definition{adv.}{às vezes\dots às vezes\dots}
  \end{phonetics}
\end{entry}

\begin{entry}{有时候}{6,7,10}{⽉、⽇、⼈}
  \begin{phonetics}{有时候}{you3 shi2 hou5}[][HSK 1]
    \definition{adv.}{às vezes; indica um momento incerto, mas não frequente}
  \seealsoref{有的时候}{you3 de5 shi2 hou4}
  \seealsoref{有时}{you3 shi2}
  \end{phonetics}
\end{entry}

\begin{entry}{有些}{6,8}{⽉、⼆}
  \begin{phonetics}{有些}{you3 xie1}[][HSK 1]
    \definition{adv.}{um pouco; bastante; ligeiramente}
    \definition{pron.}{uma parte; alguns}
    \definition{v.}{usado para indicar que há alguns, mas não muitos;}
  \seealsoref{有(一)些}{you3 (yi4) xie1}
  \end{phonetics}
\end{entry}

\begin{entry}{有的}{6,8}{⽉、⽩}
  \begin{phonetics}{有的}{you3 de5}[][HSK 1]
    \definition{pron.}{algum, alguns}
  \end{phonetics}
\end{entry}

\begin{entry}{有的时候}{6,8,7,10}{⽉、⽩、⽇、⼈}
  \begin{phonetics}{有的时候}{you3 de5 shi2 hou4}
    \definition{adv.}{às vezes; ocasionalmente}
  \seealsoref{有时}{you3 shi2}
  \seealsoref{有时候}{you3 shi2 hou5}
  \end{phonetics}
\end{entry}

\begin{entry}{有的是}{6,8,9}{⽉、⽩、⽇}
  \begin{phonetics}{有的是}{you3 de5 shi4}[][HSK 3]
    \definition{expr.}{ter em abundância; não faltar; enfatizar que há muitos}
  \end{phonetics}
\end{entry}

\begin{entry}{有空儿}{6,8,2}{⽉、⽳、⼉}
  \begin{phonetics}{有空儿}{you3 kong4r5}[][HSK 2]
    \definition{v.}{estar livre; ter tempo livre}
  \end{phonetics}
\end{entry}

\begin{entry}{有限}{6,8}{⽉、⾩}
  \begin{phonetics}{有限}{you3 xian4}[][HSK 4]
    \definition{adj.}{finito; limitado; restrito | indica baixo grau; indica pouco número; número baixo; nível baixo}
  \end{phonetics}
\end{entry}

\begin{entry}{有限公司}{6,8,4,5}{⽉、⾩、⼋、⼝}
  \begin{phonetics}{有限公司}{you3xian4gong1si1}
    \definition{s.}{companhia limitada | corporação}
  \end{phonetics}
\end{entry}

\begin{entry}{有毒}{6,9}{⽉、⽏}
  \begin{phonetics}{有毒}{you3 du2}[][HSK 5]
    \definition{adj.}{venenoso; tóxico; nocivo; geralmente é usada para descrever as propriedades nocivas à saúde de produtos químicos, plantas ou animais.}
  \end{phonetics}
\end{entry}

\begin{entry}{有点儿}{6,9,2}{⽉、⽕、⼉}
  \begin{phonetics}{有点儿}{you3 dian3r5}
    \definition{adv.}{um pouco; indica um grau inferior, equivalente a 稍微 (usado principalmente para coisas que são insatisfatórias)}
    \definition{v.}{há um pouco; tem (ou ser de) algum; existem alguns}
  \seealsoref{稍微}{shao1wei1}
  \seealsoref{有(一)点儿}{you3 yi4 dian3r5}
  \end{phonetics}
\end{entry}

\begin{entry}{有害}{6,10}{⽉、⼧}
  \begin{phonetics}{有害}{you3 hai4}[][HSK 5]
    \definition{adj.}{prejudicial; nocivo; danoso; que pode causar danos ou prejuízos a algo}
  \end{phonetics}
\end{entry}

\begin{entry}{有效}{6,10}{⽉、⽁}
  \begin{phonetics}{有效}{you3 xiao4}[][HSK 3]
    \definition{adj.}{válido; eficiente; eficaz; capaz de alcançar os objetivos esperados}
  \end{phonetics}
\end{entry}

\begin{entry}{有着}{6,11}{⽉、⽬}
  \begin{phonetics}{有着}{you3 zhe5}[][HSK 5]
    \definition{v.}{ter; possuir; haver; existir}
  \end{phonetics}
\end{entry}

\begin{entry}{有道理}{6,12,11}{⽉、⾡、⽟}
  \begin{phonetics}{有道理}{you3dao4li5}
    \definition{adj.}{razoável}
    \definition{v.}{fazer sentido}
  \end{phonetics}
\end{entry}

\begin{entry}{有意思}{6,13,9}{⽉、⼼、⼼}
  \begin{phonetics}{有意思}{you3 yi4 si5}[][HSK 2]
    \definition{adj.}{significativo; significativo e intrigante | interessante; agradável}
    \definition{v.}{ter interesse por; ser atraído sexualmente}
  \end{phonetics}
\end{entry}

\begin{entry}{有趣}{6,15}{⽉、⾛}
  \begin{phonetics}{有趣}{you3qu4}[][HSK 4]
    \definition{adj.}{interessante; fascinante; divertido; excitante; estimulante}
  \end{phonetics}
\end{entry}

\begin{entry}{朵}{6}{⽊}
  \begin{phonetics}{朵}{duo3}[][HSK 5]
    \definition*{s.}{sobrenome Duo}
    \definition{clas.}{usado para flores, nuvens ou coisas que se assemelham a flores e nuvens}
  \end{phonetics}
\end{entry}

\begin{entry}{机}{6}{⽊}
  \begin{phonetics}{机}{ji1}
    \definition*{s.}{sobrenome Ji}
    \definition{adj.}{flexível; perspicaz; destreza; agilidade}
    \definition[台]{s.}{máquina; motor | avião; aeronave; aeroplano; refere-se especificamente a aeronaves | ponto crucial; os fatores-chave para a ocorrência e mudança das coisas | chance; ocasião; oportunidade; um momento crítico ou oportuno para o desenvolvimento e mudança das coisas | organismo; funções vitais dos organismos | besta; mecanismo de disparo de flechas de madeira em uma besta antiga | assuntos importantes; assuntos extremamente importantes e confidenciais | ideia; intenção}
  \end{phonetics}
\end{entry}

\begin{entry}{机甲}{6,5}{⽊、⽥}
  \begin{phonetics}{机甲}{ji1jia3}
    \definition{s.}{\emph{mecha} (robôs operados pelo homem em mangá japonês)}
  \end{phonetics}
\end{entry}

\begin{entry}{机会}{6,6}{⽊、⼈}
  \begin{phonetics}{机会}{ji1hui4}[][HSK 2]
    \definition[个,次,种,些]{s.}{chance; oportunidade; momento favorável raro}
  \end{phonetics}
\end{entry}

\begin{entry}{机场}{6,6}{⽊、⼟}
  \begin{phonetics}{机场}{ji1chang3}[][HSK 1]
    \definition[个,家,处,座]{s.}{aeródromo; campo de aviação; aeroporto; campo de voo}
  \end{phonetics}
\end{entry}

\begin{entry}{机制}{6,8}{⽊、⼑}
  \begin{phonetics}{机制}{ji1 zhi4}[][HSK 5]
    \definition{s.}{mecanismo; processado por máquina; feito por máquina}
  \end{phonetics}
\end{entry}

\begin{entry}{机构}{6,8}{⽊、⽊}
  \begin{phonetics}{机构}{ji1gou4}[][HSK 4]
    \definition[所]{s.}{órgão; organização; instituição; instalações; aparelhamento; configuração | mecanismo; funcionamento interno de uma máquina ou unidade | estrutura interna de uma organização}
  \end{phonetics}
\end{entry}

\begin{entry}{机械}{6,11}{⽊、⽊}
  \begin{phonetics}{机械}{ji1xie4}
    \definition{s.}{máquina | maquinaria | mecânica}
  \end{phonetics}
\end{entry}

\begin{entry}{机票}{6,11}{⽊、⽰}
  \begin{phonetics}{机票}{ji1 piao4}[][HSK 1]
    \definition[张]{s.}{passagem aérea; passagem de avião}
  \seealsoref{飞机票}{fei1ji1 piao4}
  \end{phonetics}
\end{entry}

\begin{entry}{机遇}{6,12}{⽊、⾡}
  \begin{phonetics}{机遇}{ji1yu4}[][HSK 4]
    \definition[个]{s.}{chance; oportunidade; circunstâncias favoráveis}
  \end{phonetics}
\end{entry}

\begin{entry}{机器}{6,16}{⽊、⼝}
  \begin{phonetics}{机器}{ji1qi4}[][HSK 3]
    \definition[台,部,个]{s.}{máquina; maquinário; motor; dispositivos e máquinas que são montados a partir de peças, podem funcionar, transformar energia ou produzir trabalho útil podem ser usados como ferramentas de produção, reduzindo a intensidade do trabalho humano e aumentando a produtividade | aparato; sistema político e econômico}
  \end{phonetics}
\end{entry}

\begin{entry}{机器人}{6,16,2}{⽊、⼝、⼈}
  \begin{phonetics}{机器人}{ji1 qi4 ren2}[][HSK 5]
    \definition[个]{s.}{androide; golem | pessoa mecânica | robô}
  \end{phonetics}
\end{entry}

\begin{entry}{杀}{6}{⽊}
  \begin{phonetics}{杀}{sha1}[][HSK 5]
    \definition{adv.}{em extremo; excessivamente; usado após um verbo, indica grau intenso}
    \definition{v.}{matar; abater; esquartejar | lutar; entrar em batalha | enfraquecer; reduzir; diminuir | decolar; neutralizar}
  \end{phonetics}
\end{entry}

\begin{entry}{杀气}{6,4}{⽊、⽓}
  \begin{phonetics}{杀气}{sha1qi4}
    \definition{s.}{espírito assassino | aura de morte}
    \definition{v.}{desabafar a raiva de alguém}
  \end{phonetics}
\end{entry}

\begin{entry}{杀毒}{6,9}{⽊、⽏}
  \begin{phonetics}{杀毒}{sha1 du2}[][HSK 5]
    \definition{s.}{(computação) antivírus}
    \definition{v.}{esterilizar; desinfetar | (computação) eliminar um vírus}
  \end{phonetics}
\end{entry}

\begin{entry}{杂志}{6,7}{⽊、⼼}
  \begin{phonetics}{杂志}{za2zhi4}[][HSK 3]
    \definition[本,期,份,种]{s.}{jornal; revista; publicação}
  \end{phonetics}
\end{entry}

\begin{entry}{杂志社}{6,7,7}{⽊、⼼、⽰}
  \begin{phonetics}{杂志社}{za2zhi4she4}
    \definition{s.}{editora de revista}
  \end{phonetics}
\end{entry}

\begin{entry}{杂技}{6,7}{⽊、⼿}
  \begin{phonetics}{杂技}{za2ji4}
    \definition[场]{s.}{acrobacia}
  \end{phonetics}
\end{entry}

\begin{entry}{权}{6}{⽊}
  \begin{phonetics}{权}{quan2}
    \definition*{s.}{sobrenome Quan}
    \definition{adv.}{provisoriamente; por enquanto}
    \definition{s.}{(literário) contrapeso; peso deslizante de uma balança romana | poder; autoridade | direito | posição vantajosa | conveniência}
    \definition{v.}{pesar; medir o peso}
  \end{phonetics}
\end{entry}

\begin{entry}{权利}{6,7}{⽊、⼑}
  \begin{phonetics}{权利}{quan2li4}[][HSK 4]
    \definition[项,种,个,条,份]{s.}{direito; interesse; os poderes e benefícios (em oposição a 义务) exercidos por um cidadão ou pessoa jurídica de acordo com a lei}
  \seealsoref{义务}{yi4wu4}
  \end{phonetics}
\end{entry}

\begin{entry}{次}{6}{⽋}
  \begin{phonetics}{次}{ci4}[][HSK 1,4]
    \definition*{s.}{sobrenome Ci}
    \definition{adj.}{de segunda categoria; de qualidade inferior}
    \definition{clas.}{usado para coisas ou ações que podem ser repetidas}
    \definition{num.}{segundo; próximo}
    \definition{pref.}{(química) hipo-, radical ácido ou composto contendo dois átomos de oxigênio a menos}
    \definition{s.}{ordem; sequência; classificação | local de parada em uma viagem; escala}
  \end{phonetics}
\end{entry}

\begin{entry}{欢}{6}{⽋}
  \begin{phonetics}{欢}{huan1}
    \definition*{s.}{Huan}
    \definition{adj.}{alegre; feliz; jubilante | vigoroso; energético; em pleno andamento; com grande impulso}
    \definition{s.}{amante; querida; um apelido usado por mulheres nos tempos antigos para se referir aos seus amantes; agora, geralmente se refere a alguém de quem você gosta ou com quem tem um relacionamento romântico}
  \end{phonetics}
\end{entry}

\begin{entry}{欢乐}{6,5}{⽋、⼃}
  \begin{phonetics}{欢乐}{huan1le4}[][HSK 3]
    \definition{adj.}{feliz; alegre; felicidade (geralmente coletiva)}
  \end{phonetics}
\end{entry}

\begin{entry}{欢快}{6,7}{⽋、⼼}
  \begin{phonetics}{欢快}{huan1kuai4}
    \definition{adj.}{feliz e sem ansiedade | vívido}
  \end{phonetics}
\end{entry}

\begin{entry}{欢迎}{6,7}{⽋、⾡}
  \begin{phonetics}{欢迎}{huan1ying2}[][HSK 2]
    \definition{adj.}{bem-vindo}
    \definition{v.}{dar as boas-vindas; cumprimentar; receber com alegria | dar as boas-vindas; receber favoravelmente (bem)}
  \end{phonetics}
\end{entry}

\begin{entry}{此}{6}{⽌}
  \begin{phonetics}{此}{ci3}[][HSK 4]
    \definition*{s.}{sobrenome Ci}
    \definition{pron.}{esse; essa; isso; este; esta; isto; indica ou se refere a uma pessoa ou coisa que está mais próxima, equivalente a 这 ou 这个 (em oposição a 彼) | aqui e agora; refere-se a um tempo ou lugar recente, equivalente a 这会儿 ou 这里}
  \seealsoref{彼}{bi3}
  \seealsoref{这}{zhe4}
  \seealsoref{这会儿}{zhe4 hui4r5}
  \seealsoref{这里}{zhe4 li3}
  \seealsoref{这个}{zhe4ge5}
  \end{phonetics}
\end{entry}

\begin{entry}{此外}{6,5}{⽌、⼣}
  \begin{phonetics}{此外}{ci3wai4}[][HSK 4]
    \definition{conj.}{além disso; em adição; além das coisas ou situações mencionadas acima}
  \end{phonetics}
\end{entry}

\begin{entry}{此后}{6,6}{⽌、⼝}
  \begin{phonetics}{此后}{ci3 hou4}[][HSK 5]
    \definition{s.}{daqui em diante; doravante; depois disso; após isso; de agora em diante}
  \end{phonetics}
\end{entry}

\begin{entry}{此时}{6,7}{⽌、⽇}
  \begin{phonetics}{此时}{ci3 shi2}[][HSK 5]
    \definition{s.}{agora; no presente; agora mesmo; neste momento; por enquanto}
  \end{phonetics}
\end{entry}

\begin{entry}{此刻}{6,8}{⽌、⼑}
  \begin{phonetics}{此刻}{ci3 ke4}[][HSK 5]
    \definition{s.}{agora; no momento; exatamente agora; neste momento}
  \end{phonetics}
\end{entry}

\begin{entry}{死}{6}{⽍}
  \begin{phonetics}{死}{si3}[][HSK 3]
    \definition{adj.}{até a morte | implacável; mortal | fixo; rígido; inflexível | intransitável; fechado | (expressando raiva, reclamação, etc., às vezes jocosamente) maldito}
    \definition{adv.}{(frequentemente no negativo) teimosamente; inflexivelmente}
    \definition{v.}{morrer; estar morto (oposto a 生 e 活)}
  \seealsoref{活}{huo2}
  \seealsoref{生}{sheng1}
  \end{phonetics}
\end{entry}

\begin{entry}{死亡}{6,3}{⽍、⼇}
  \begin{phonetics}{死亡}{si3wang2}
    \definition{s.}{morte}
    \definition{v.}{morrer}
  \end{phonetics}
\end{entry}

\begin{entry}{毕}{6}{⽐}
  \begin{phonetics}{毕}{bi4}
    \definition*{s.}{sobrenome Bi}
    \definition*{s.}{Bi, uma das mansões lunares; a décima nona das vinte e oito constelações ( Vinte e Oito Constelações ) em que a esfera celeste foi dividida, consistindo de oito estrelas, seis em Híades e duas em Touro}
    \definition{adv.}{tudo; completamente; totalmente}
    \definition{v.}{terminar; realizar; concluir  | completar; terminar}
  \end{phonetics}
\end{entry}

\begin{entry}{毕业}{6,5}{⽐、⼀}
  \begin{phonetics}{毕业}{bi4ye4}[][HSK 4]
    \definition{s.}{formatura}
    \definition{v.+compl.}{formar-se}
  \end{phonetics}
\end{entry}

\begin{entry}{毕业生}{6,5,5}{⽐、⼀、⽣}
  \begin{phonetics}{毕业生}{bi4 ye4 sheng1}[][HSK 4]
    \definition[个]{s.}{diplomado; graduado; bacharel; pessoa que recebeu um diploma, grau ou certificado}
  \end{phonetics}
\end{entry}

\begin{entry}{毕竟}{6,11}{⽐、⾳}
  \begin{phonetics}{毕竟}{bi4jing4}[][HSK 5]
    \definition{adv.}{afinal de contas; quando tudo estiver dito e feito; em última análise; indica um resultado que não pode ser alterado, enfatizando que se trata de uma causa ou fato que precisa ser enfocado para referência | significa 到底, 究竟, 终究, indicando a conclusão final alcançada}
  \seealsoref{到底}{dao4di3}
  \seealsoref{究竟}{jiu1jing4}
  \seealsoref{终究}{zhong1jiu1}
  \end{phonetics}
\end{entry}

\begin{entry}{汗}{6}{⽔}
  \begin{phonetics}{汗}{han2}
    \definition*{s.}{abreviação de Khan}[他是成吉思汗。___Ele é Genghis Khan.]
  \end{phonetics}
  \begin{phonetics}{汗}{han4}[][HSK 5]
    \definition{s.}{suor; transpiração; perspiração}
  \end{phonetics}
\end{entry}

\begin{entry}{汗水}{6,4}{⽔、⽔}
  \begin{phonetics}{汗水}{han4shui3}
    \definition*{s.}{Rio Han (Hanshui)}
    \definition{s.}{suor | transpiração}
  \end{phonetics}
\end{entry}

\begin{entry}{汗液}{6,11}{⽔、⽔}
  \begin{phonetics}{汗液}{han4ye4}
    \definition{s.}{suor}
  \end{phonetics}
\end{entry}

\begin{entry}{汗腺}{6,13}{⽔、⾁}
  \begin{phonetics}{汗腺}{han4xian4}
    \definition{s.}{glândula sudorípara}
  \end{phonetics}
\end{entry}

\begin{entry}{江}{6}{⽔}
  \begin{phonetics}{江}{jiang1}[][HSK 4]
    \definition*{s.}{sobrenome Jiang}
    \definition*{s.}{Rio Changjiang}
    \definition[条,道]{s.}{rio grande}
  \end{phonetics}
\end{entry}

\begin{entry}{江水}{6,4}{⽔、⽔}
  \begin{phonetics}{江水}{jiang1shui3}
    \definition{s.}{água do rio}
  \end{phonetics}
\end{entry}

\begin{entry}{江西}{6,6}{⽔、⾑}
  \begin{phonetics}{江西}{jiang1xi1}
    \definition*{s.}{Jiangxi}
  \end{phonetics}
\end{entry}

\begin{entry}{江苏}{6,7}{⽔、⾋}
  \begin{phonetics}{江苏}{jiang1su1}
    \definition*{s.}{Província de Jiangsu}
  \end{phonetics}
\end{entry}

\begin{entry}{江南水乡}{6,9,4,3}{⽔、⼗、⽔、⼄}
  \begin{phonetics}{江南水乡}{jiang1nan2shui3xiang1}
    \definition*{s.}{Vila Aquática de Jiangnan | Cidades Aquáticas}
  \end{phonetics}
\end{entry}

\begin{entry}{池}{6}{⽔}
  \begin{phonetics}{池}{chi2}
    \definition*{s.}{sobrenome Chi}
    \definition[个,片]{s.}{piscina; lagoa | qualquer espaço fechado com laterais elevadas | baias (em um teatro); a parte frontal do salão principal do teatro | fosso}
  \end{phonetics}
\end{entry}

\begin{entry}{池子}{6,3}{⽔、⼦}
  \begin{phonetics}{池子}{chi2 zi5}[][HSK 5]
    \definition{s.}{lago; lagoa; viveiro | piscina; piscina do balneário | (antigo) arquibancada (primeiras fileiras em um teatro) | pista de dança de um salão de baile}
  \end{phonetics}
\end{entry}

\begin{entry}{污}{6}{⽔}
  \begin{phonetics}{污}{wu1}
    \definition{adj.}{sujo; imundo; imundo | corrupto}
    \definition{s.}{sujeira; imundície | esgoto; água suja; coisas sujas}
    \definition{v.}{contaminar; sujar | manchar}
  \end{phonetics}
\end{entry}

\begin{entry}{污水}{6,4}{⽔、⽔}
  \begin{phonetics}{污水}{wu1shui3}[][HSK 5]
    \definition{s.}{água suja (ou poluída, residual); esgoto; lodo | efluente; drenagem; água suja; água poluída; água residual}
  \end{phonetics}
\end{entry}

\begin{entry}{污染}{6,9}{⽔、⽊}
  \begin{phonetics}{污染}{wu1ran3}[][HSK 5]
    \definition{s.}{poluição}
    \definition{v.}{poluir; contaminar com substâncias nocivas e prejudiciais; refere-se especificamente à destruição do ambiente natural causada por substâncias nocivas, tais como gases, líquidos e resíduos emitidos por indústrias, minas, veículos, etc. | contaminar; metáfora de que pensamentos prejudiciais causam efeitos negativos nas pessoas}
  \end{phonetics}
\end{entry}

\begin{entry}{污染区}{6,9,4}{⽔、⽊、⼖}
  \begin{phonetics}{污染区}{wu1ran3qu1}
    \definition{s.}{área contaminada}
  \end{phonetics}
\end{entry}

\begin{entry}{污染物}{6,9,8}{⽔、⽊、⽜}
  \begin{phonetics}{污染物}{wu1ran3wu4}
    \definition{s.}{poluente}
  \seealsoref{污染物质}{wu1ran3 wu4zhi4}
  \end{phonetics}
\end{entry}

\begin{entry}{污染物质}{6,9,8,8}{⽔、⽊、⽜、⾙}
  \begin{phonetics}{污染物质}{wu1ran3 wu4zhi4}
    \definition{s.}{poluente}
  \seealsoref{污染物}{wu1ran3wu4}
  \end{phonetics}
\end{entry}

\begin{entry}{汤}{6}{⽔}
  \begin{phonetics}{汤}{shang1}
    \definition{s.}{correnteza forte}
  \end{phonetics}
  \begin{phonetics}{汤}{tang1}[][HSK 3]
    \definition*{s.}{sobrenome Tang}
    \definition[勺,碗,杯,锅]{s.}{água quente; água fervente | fontes termais | água utilizada para ferver algo| sopa; caldo | uma preparação líquida de ervas medicinais; decocção}
  \end{phonetics}
\end{entry}

\begin{entry}{灯}{6}{⽕}
  \begin{phonetics}{灯}{deng1}[][HSK 2]
    \definition*{s.}{sobrenome Deng}
    \definition[盏,个]{s.}{lâmpada; luz; lanterna; dispositivo luminoso, usado principalmente para iluminação | queimador; um aparelho que brilha e aquece como uma lâmpada e pode ser usado para aquecer | tubo; válvula; o nome popular dado aos tubos eletrônicos com formato semelhante a lâmpadas encontrados em aparelhos antigos, como rádios}
  \end{phonetics}
\end{entry}

\begin{entry}{灯丝}{6,5}{⽕、⼀}
  \begin{phonetics}{灯丝}{deng1si1}
    \definition{s.}{filamento (de uma lâmpada)}
  \end{phonetics}
\end{entry}

\begin{entry}{灯号}{6,5}{⽕、⼝}
  \begin{phonetics}{灯号}{deng1hao4}
    \definition{s.}{sinal luminoso | luz indicadora}
  \end{phonetics}
\end{entry}

\begin{entry}{灯光}{6,6}{⽕、⼉}
  \begin{phonetics}{灯光}{deng1 guang1}[][HSK 4]
    \definition{s.}{iluminação; luminosidade da lâmpada | luminação (palco); equipamento de iluminação para palco ou estúdio}
  \end{phonetics}
\end{entry}

\begin{entry}{灯泡}{6,8}{⽕、⽔}
  \begin{phonetics}{灯泡}{deng1pao4}
    \definition[个]{s.}{lâmpada | (gíria) terceiro indesejado estragando encontro de casal}
  \seealsoref{电灯泡}{dian4deng1pao4}
  \end{phonetics}
\end{entry}

\begin{entry}{灯标}{6,9}{⽕、⽊}
  \begin{phonetics}{灯标}{deng1biao1}
    \definition{s.}{farol | luz de farol}
  \end{phonetics}
\end{entry}

\begin{entry}{灰}{6}{⽕}
  \begin{phonetics}{灰}{hui1}
    \definition{adj.}{cinza (cor) | desanimado; desencorajado; deprimido}
    \definition[把,堆]{s.}{cinzas; pó que sobra após a queima de um objeto | pó; poeira; substância em pó | cal; argamassa (de cal)}
  \end{phonetics}
\end{entry}

\begin{entry}{灰色}{6,6}{⽕、⾊}
  \begin{phonetics}{灰色}{hui1 se4}[][HSK 5]
    \definition{adj.}{obscuro; ambíguo | sombrio; pessimista}
    \definition[个]{s.}{cor cinza; acinzentado}
  \end{phonetics}
\end{entry}

\begin{entry}{爷爷}{6,6}{⽗、⽗}
  \begin{phonetics}{爷爷}{ye2ye5}[][HSK 1]
    \definition[个,位]{s.}{avô (paterno)}
  \end{phonetics}
\end{entry}

\begin{entry}{百}{6}{⽩}
  \begin{phonetics}{百}{bai3}[][HSK 1]
    \definition{adj.}{todos; todos os tipos de; multifacetados; numerosos}
    \definition{adv.}{muito; sempre}
    \definition{num.}{cem; 100}
  \end{phonetics}
\end{entry}

\begin{entry}{百分}{6,4}{⽩、⼑}
  \begin{phonetics}{百分}{bai3fen1}
    \definition{num.}{por cento}
    \definition{s.}{porcentagem}
  \end{phonetics}
\end{entry}

\begin{entry}{百货}{6,8}{⽩、⾙}
  \begin{phonetics}{百货}{bai3 huo4}[][HSK 4]
    \definition{s.}{mercadorias em geral; loja de departamentos; um termo geral para bens que incluem principalmente roupas, utensílios e necessidades diárias gerais}
  \end{phonetics}
\end{entry}

\begin{entry}{百般}{6,10}{⽩、⾈}
  \begin{phonetics}{百般}{bai3ban1}
    \definition{adv.}{de todas as maneiras possíveis | por todos os meios}
  \end{phonetics}
\end{entry}

\begin{entry}{竹子}{6,3}{⽵、⼦}
  \begin{phonetics}{竹子}{zhu2zi5}[][HSK 5]
    \definition[块,株]{s.}{bambu; nome genérico para os tipos de bambu}
  \end{phonetics}
\end{entry}

\begin{entry}{竹马}{6,3}{⽵、⾺}
  \begin{phonetics}{竹马}{zhu2ma3}
    \definition{s.}{cavalo de bambu | vara de bambu usada como cavalo de brinquedo}
  \end{phonetics}
\end{entry}

\begin{entry}{竹排}{6,11}{⽵、⼿}
  \begin{phonetics}{竹排}{zhu2pai2}
    \definition{s.}{jangada de bambu}
  \end{phonetics}
\end{entry}

\begin{entry}{竹编}{6,12}{⽵、⽷}
  \begin{phonetics}{竹编}{zhu2bian1}
    \definition{s.}{vime | tecelagem de bambu}
  \end{phonetics}
\end{entry}

\begin{entry}{米}{6}{⽶}[Kangxi 119]
  \begin{phonetics}{米}{mi3}[][HSK 2,3]
    \definition*{s.}{sobrenome Mi}
    \definition{clas.}{metro (m); unidade principal de comprimento do sistema métrico}
    \definition[粒,斤]{s.}{arroz | sementes descascadas; refere-se a sementes comestíveis descascadas ou sem casca | qualquer coisa que se assemelhe a um grão de arroz}
  \end{phonetics}
\end{entry}

\begin{entry}{米饭}{6,7}{⽶、⾷}
  \begin{phonetics}{米饭}{mi3fan4}[][HSK 1]
    \definition{s.}{arroz (cozido)}
  \end{phonetics}
\end{entry}

\begin{entry}{红}{6}{⽷}
  \begin{phonetics}{红}{hong2}[][HSK 2]
    \definition*{s.}{sobrenome Hong}
    \definition{adj.}{vermelho | popular; bem-sucedido; símbolo de sucesso ou valorização | vermelho; revolucionário; símbolo da revolução | festivo; símbolo de alegria}
    \definition{s.}{tecido vermelho, bandeirinhas, etc. usados em ocasiões festivas | bônus; dividendo}
  \end{phonetics}
\end{entry}

\begin{entry}{红包}{6,5}{⽷、⼓}
  \begin{phonetics}{红包}{hong2 bao1}[][HSK 4]
    \definition[个]{s.}{saco de papel vermelho ou envelope contendo dinheiro como presente, gorjeta ou bônus | suborno; propina}
  \end{phonetics}
\end{entry}

\begin{entry}{红色}{6,6}{⽷、⾊}
  \begin{phonetics}{红色}{hong2 se4}[][HSK 2]
    \definition{adj.}{vermelho; revolucionário; símbolo da revolução ou da consciência política elevada}
    \definition{s.}{cor vermelha}
  \end{phonetics}
\end{entry}

\begin{entry}{红宝石}{6,8,5}{⽷、⼧、⽯}
  \begin{phonetics}{红宝石}{hong2bao3shi2}
    \definition{s.}{rubi}
  \end{phonetics}
\end{entry}

\begin{entry}{红线}{6,8}{⽷、⽷}
  \begin{phonetics}{红线}{hong2xian4}
    \definition{s.}{linha vermelha}
  \end{phonetics}
\end{entry}

\begin{entry}{红茶}{6,9}{⽷、⾋}
  \begin{phonetics}{红茶}{hong2 cha2}[][HSK 3]
    \definition[杯,壶,斤,种]{s.}{chá preto; chá acabado produzido através de fermentação completa}
  \end{phonetics}
\end{entry}

\begin{entry}{红烧}{6,10}{⽷、⽕}
  \begin{phonetics}{红烧}{hong2shao1}
    \definition{s.}{guisado em molho de soja (prato)}
  \end{phonetics}
\end{entry}

\begin{entry}{红酒}{6,10}{⽷、⾣}
  \begin{phonetics}{红酒}{hong2 jiu3}[][HSK 3]
    \definition[瓶,杯,壶,斤,箱]{s.}{vinho tinto}
  \end{phonetics}
\end{entry}

\begin{entry}{红绿灯}{6,11,6}{⽷、⽷、⽕}
  \begin{phonetics}{红绿灯}{hong2lv4deng1}
    \definition[个]{s.}{semáforo | sinal de trânsito}
  \end{phonetics}
\end{entry}

\begin{entry}{红薯}{6,16}{⽷、⾋}
  \begin{phonetics}{红薯}{hong2shu3}
    \definition{s.}{batata doce}
  \end{phonetics}
\end{entry}

\begin{entry}{约}{6}{⽷}
  \begin{phonetics}{约}{yao1}
    \definition{adj.}{econômico; frugal | simples; breve | indistinto}
    \definition{adv.}{cerca de; ao redor; aproximadamente}
    \definition{s.}{pacto; acordo; nomeação; coisa prometida}
    \definition{v.}{marcar uma consulta; organizar | perguntar ou convidar com antecedência | restringir; conter | reduzir (fração aproximada)}
  \end{phonetics}
  \begin{phonetics}{约}{yue1}[][HSK 3]
    \definition*{s.}{sobrenome Yue}
    \definition{adj.}{econômico; frugal | simples; breve; resumido | indistinto; confuso}
    \definition{adv.}{cerca de; ao redor; aproximadamente}
    \definition{s.}{pacto; acordo; nomeação; o que foi combinado}
    \definition{v.}{combinar; propor ou discutir antecipadamente (o que deve ser respeitado por todos) | convidar com antecedência | restringir; conter | reduzir (fração aproximada)}
  \end{phonetics}
\end{entry}

\begin{entry}{约会}{6,6}{⽷、⼈}
  \begin{phonetics}{约会}{yue1hui4}[][HSK 4]
    \definition[个,次]{s.}{data; compromisso; engajamento; reunião pré-agendada}
    \definition{v.}{marcar uma reunião; marcar um encontro;}
  \end{phonetics}
\end{entry}

\begin{entry}{约束}{6,7}{⽷、⽊}
  \begin{phonetics}{约束}{yue1shu4}[][HSK 5]
    \definition{adj.}{amarrado}
    \definition{s.}{restrição; constrangimento; engajamento}
    \definition{v.}{amarrar; prender; reprimir; restringir; manter dentro de si}
  \end{phonetics}
\end{entry}

\begin{entry}{级}{6}{⽷}
  \begin{phonetics}{级}{ji2}[][HSK 2]
    \definition{clas.}{usado para degraus, escadas, pisos de torres, etc.}
    \definition[个,种]{s.}{nível; classificação; grau; classe | série; turma; qualquer uma das divisões anuais de um curso escolar | degrau}
  \end{phonetics}
\end{entry}

\begin{entry}{纪}{6}{⽷}
  \begin{phonetics}{纪}{ji3}
    \definition*{s.}{Ji}
    \definition{s.}{disciplina | um período de doze anos (na China antiga); um período de anos | (geologia) subdivisão de uma era geológica; período}
    \definition{v.}{colocar por escrito; registrar; mesmo significado de 记, usado principalmente em 记录, 纪年, 纪元, 纪传, etc. | classificar (fios de seda)}
  \seealsoref{记}{ji4}
  \seealsoref{纪传}{ji4 zhuan4}
  \seealsoref{记录}{ji4lu4}
  \seealsoref{纪年}{ji4nian2}
  \seealsoref{纪元}{ji4yuan2}
  \end{phonetics}
  \begin{phonetics}{纪}{ji4}
    \definition*{s.}{sobrenome Ji}
    \definition{s.}{disciplina | idade; época | (geologia) período | um período de doze anos (na China antiga); um período de anos | (geologia) subdivisão de uma era geológica}
    \definition{v.}{colocar por escrito; registrar | registrar, mesmo significado de 记, usado principalmente em 记录, 纪年, 纪元, 纪传, etc. | classificar (fios de seda)}
  \seealsoref{记}{ji4}
  \seealsoref{纪传}{ji4 zhuan4}
  \seealsoref{记录}{ji4lu4}
  \seealsoref{纪年}{ji4nian2}
  \seealsoref{纪元}{ji4yuan2}
  \end{phonetics}
\end{entry}

\begin{entry}{纪元}{6,4}{⽷、⼉}
  \begin{phonetics}{纪元}{ji4yuan2}
    \definition{s.}{o início de uma era (por exemplo, o reinado de um imperador) | época; era}
  \end{phonetics}
\end{entry}

\begin{entry}{纪传}{6,6}{⽷、⼈}
  \begin{phonetics}{纪传}{ji4 zhuan4}
    \definition{s.}{crônica; biografia}
  \end{phonetics}
\end{entry}

\begin{entry}{纪传体}{6,6,7}{⽷、⼈、⼈}
  \begin{phonetics}{纪传体}{ji4 zhuan4 ti3}
    \definition{s.}{história apresentada em uma série de biografias | gênero histórico baseado em biografia}
  \end{phonetics}
\end{entry}

\begin{entry}{纪年}{6,6}{⽷、⼲}
  \begin{phonetics}{纪年}{ji4nian2}
    \definition{s.}{cronologia; uma maneira de numerar os anos | registro cronológico de eventos; anais; um dos gêneros de livros históricos é organizar fatos históricos em ordem cronológica}
  \end{phonetics}
\end{entry}

\begin{entry}{纪录}{6,8}{⽷、⼹}
  \begin{phonetics}{纪录}{ji4lu4}[][HSK 3]
    \definition[项,个]{s.}{recorde (esportes); o número mais alto ou mais baixo registrado em um determinado período de tempo}
  \end{phonetics}
\end{entry}

\begin{entry}{纪念}{6,8}{⽷、⼼}
  \begin{phonetics}{纪念}{ji4nian4}[][HSK 3]
    \definition[个,次]{s.}{lembrança; recordação; usado para representar uma lembrança (objeto)}
    \definition{v.}{comemorar; expressar saudade por pessoas ou coisas através de objetos ou ações}
  \end{phonetics}
\end{entry}

\begin{entry}{纪律}{6,9}{⽷、⼻}
  \begin{phonetics}{纪律}{ji4lv4}[][HSK 4]
    \definition{s.}{disciplina; código de conduta que cada membro da vida coletiva deve observar}
  \end{phonetics}
\end{entry}

\begin{entry}{网}{6}{⽹}[Kangxi 122]
  \begin{phonetics}{网}{wang3}[][HSK 2]
    \definition[张]{s.}{rede; um dispositivo feito de corda ou barbante para capturar peixes e pássaros | algo que parece uma rede | rede; uma rede de organizações; um sistema}
    \definition{v.}{pegar com uma rede | cobrir como com uma rede}
  \end{phonetics}
\end{entry}

\begin{entry}{网上}{6,3}{⽹、⼀}
  \begin{phonetics}{网上}{wang3 shang4}[][HSK 1]
    \definition{s.}{\emph{online}; refere-se a acessar a Internet através de um computador ou celular para pesquisar e consultar informações na rede}
  \end{phonetics}
\end{entry}

\begin{entry}{网上银行}{6,3,11,6}{⽹、⼀、⾦、⾏}
  \begin{phonetics}{网上银行}{wang3shang4yin2hang2}
    \definition[个]{s.}{banco \emph{online} | acesso a operações bancárias via \emph{Internet}}
  \seealsoref{网银}{wang3yin2}
  \end{phonetics}
\end{entry}

\begin{entry}{网友}{6,4}{⽹、⼜}
  \begin{phonetics}{网友}{wang3 you3}[][HSK 1]
    \definition{s.}{internauta; usuário da \emph{Internet}; amigos que se conhecem pela Internet; também usado como forma de tratamento entre internautas}
  \end{phonetics}
\end{entry}

\begin{entry}{网址}{6,7}{⽹、⼟}
  \begin{phonetics}{网址}{wang3 zhi3}[][HSK 4]
    \definition{s.}{\emph{website}; endereço da \emph{web}; endereço de um \emph{site} na \emph{Internet}, que os usuários podem acessar, consultar e obter recursos de informações nesse \emph{site} clicando nele}
  \end{phonetics}
\end{entry}

\begin{entry}{网际网络}{6,7,6,9}{⽹、⾩、⽹、⽷}
  \begin{phonetics}{网际网络}{wang3ji4wang3luo4}
    \definition*{s.}{\emph{Internet}}
  \seealsoref{互联网}{hu4lian2wang3}
  \seealsoref{网际网路}{wang3ji4wang3lu4}
  \seealsoref{网路}{wang3lu4}
  \end{phonetics}
\end{entry}

\begin{entry}{网际网路}{6,7,6,13}{⽹、⾩、⽹、⾜}
  \begin{phonetics}{网际网路}{wang3ji4wang3lu4}
    \definition*{s.}{\emph{Internet}}
  \seealsoref{互联网}{hu4lian2wang3}
  \seealsoref{网际网络}{wang3ji4wang3luo4}
  \seealsoref{网路}{wang3lu4}
  \end{phonetics}
\end{entry}

\begin{entry}{网络}{6,9}{⽹、⽷}
  \begin{phonetics}{网络}{wang3luo4}[][HSK 4]
    \definition{s.}{rede; um sistema que consiste em ramificações interconectadas; em um sistema elétrico, um circuito ou parte de um circuito que consiste em vários elementos que permitem a transmissão de sinais elétricos de acordo com determinados requisitos | rede; rede de computadores}
  \end{phonetics}
\end{entry}

\begin{entry}{网站}{6,10}{⽹、⽴}
  \begin{phonetics}{网站}{wang3zhan4}[][HSK 2]
    \definition[个,家]{s.}{\emph{web}; \emph{website}; um site virtual na Internet para uma organização ou indivíduo, geralmente consistindo em uma página inicial e muitas páginas da web}
  \end{phonetics}
\end{entry}

\begin{entry}{网罟}{6,10}{⽹、⽹}
  \begin{phonetics}{网罟}{wang3gu3}
    \definition{s.}{(fig.) a rede da justiça | rede usada para capturar peixes (ou outros animais, como pássaros)}
  \end{phonetics}
\end{entry}

\begin{entry}{网球}{6,11}{⽹、⽟}
  \begin{phonetics}{网球}{wang3qiu2}[][HSK 2]
    \definition[个,颗,些]{s.}{tênis (esporte) | bola de tênis}
  \end{phonetics}
\end{entry}

\begin{entry}{网银}{6,11}{⽹、⾦}
  \begin{phonetics}{网银}{wang3yin2}
    \definition{s.}{banco \emph{online} | acesso a operações bancárias via \emph{Internet}}
  \seealsoref{网上银行}{wang3shang4yin2hang2}
  \end{phonetics}
\end{entry}

\begin{entry}{网路}{6,13}{⽹、⾜}
  \begin{phonetics}{网路}{wang3lu4}
    \definition{s.}{\emph{Internet}}
  \seealsoref{互联网}{hu4lian2wang3}
  \seealsoref{网际网路}{wang3ji4wang3lu4}
  \seealsoref{网际网络}{wang3ji4wang3luo4}
  \end{phonetics}
\end{entry}

\begin{entry}{羊}{6}{⽺}[Kangxi 123]
  \begin{phonetics}{羊}{yang2}[][HSK 3]
    \definition*{s.}{sobrenome Yang}
    \definition[只,头,群]{s.}{carneiro; ovelha; bode; cabra; antílope}
  \end{phonetics}
\end{entry}

\begin{entry}{羽毛}{6,4}{⽻、⽑}
  \begin{phonetics}{羽毛}{yu3mao2}
    \definition{s.}{pena | plumagem | pluma}
  \end{phonetics}
\end{entry}

\begin{entry}{羽毛笔}{6,4,10}{⽻、⽑、⽵}
  \begin{phonetics}{羽毛笔}{yu3mao2bi3}
    \definition{s.}{caneta de pena}
  \end{phonetics}
\end{entry}

\begin{entry}{羽毛球}{6,4,11}{⽻、⽑、⽟}
  \begin{phonetics}{羽毛球}{yu3mao2qiu2}[][HSK 5]
    \definition[只]{s.}{\emph{badminton}; esporte com bola, as regras e equipamentos são bastante semelhantes ao tênis | peteca}
  \end{phonetics}
\end{entry}

\begin{entry}{羽林}{6,8}{⽻、⽊}
  \begin{phonetics}{羽林}{yu3lin2}
    \definition{s.}{escolta armada}
  \end{phonetics}
\end{entry}

\begin{entry}{羽冠}{6,9}{⽻、⼍}
  \begin{phonetics}{羽冠}{yu3guan1}
    \definition{s.}{crista emplumada (de pássaro)}
  \end{phonetics}
\end{entry}

\begin{entry}{羽绒服}{6,9,8}{⽻、⽷、⽉}
  \begin{phonetics}{羽绒服}{yu3rong2fu2}[][HSK 5]
    \definition[件]{s.}{jaqueta de plumas; peça de vestuário com enchimento de plumas; casaco cujo interior é preenchido com penas de pato ou ganso}
  \end{phonetics}
\end{entry}

\begin{entry}{羽流}{6,10}{⽻、⽔}
  \begin{phonetics}{羽流}{yu3liu2}
    \definition{s.}{pluma}
  \end{phonetics}
\end{entry}

\begin{entry}{老}{6}{⽼}[Kangxi 125]
  \begin{phonetics}{老}{lao3}[][HSK 1,2]
    \definition*{s.}{sobrenome Lao}
    \definition{adj.}{velho; envelhecido; idade avançada | antigo; de longa data; existe há muito tempo | antigo; desatualizado; obsoleto; ultrapassado  | antigo; tradicional; original | coberto de vegetação; os vegetais cresceram além do período ideal para serem consumidos | resistente; endurecido; alimentos muito cozidos | escuro; profundo; (sobre cores) | último nascido; o mais novo | veterano; experiente; sofisticado}
    \definition{adv.}{longo; por muito tempo | sempre (fazendo algo) | muito}
    \definition{pref.}{usado para designar pessoas, ordem de classificação, certos nomes de animais e plantas}
    \definition{s.}{idosos; pessoas mais velhas | ancião; sênior; um título respeitoso para pessoas mais velhas}
    \definition{v.}{envelhecer | morrer; referindo-se à morte de um idoso}
  \end{phonetics}
\end{entry}

\begin{entry}{老人}{6,2}{⽼、⼈}
  \begin{phonetics}{老人}{lao3 ren2}[][HSK 1]
    \definition[位]{s.}{homem ou mulher de idade avançada; o idoso; o velho}
  \end{phonetics}
\end{entry}

\begin{entry}{老人家}{6,2,10}{⽼、⼈、⼧}
  \begin{phonetics}{老人家}{lao3 ren2 jia1}
    \definition{s.}{senhor ancião | madame | senhora | termo educado para mulher ou homem velho}
  \end{phonetics}
\end{entry}

\begin{entry}{老公}{6,4}{⽼、⼋}
  \begin{phonetics}{老公}{lao3 gong1}[][HSK 4]
    \definition[个]{s.}{marido; esposo}
  \end{phonetics}
\end{entry}

\begin{entry}{老太太}{6,4,4}{⽼、⼤、⼤}
  \begin{phonetics}{老太太}{lao3 tai4 tai5}[][HSK 3]
    \definition[位,名,个]{s.}{velha senhora; (em tratamento direto)Venerável Senhora; uma maneira respeitosa de chamar uma senhora idosa; título honorífico para mulheres idosas | (forma de tratamento) sua velha mãe; minha velha mãe, avó ou sogra; referindo-se à própria mãe, à mãe do outro ou à mãe de outra pessoa, à sogra ou à sogra política}
  \end{phonetics}
\end{entry}

\begin{entry}{老头儿}{6,5,2}{⽼、⼤、⼉}
  \begin{phonetics}{老头儿}{lao3 tou2r5}[][HSK 3]
    \definition{s.}{(coloquial) (com um tom de intimidade) velho; velho amigo}
  \seealsoref{老头子}{lao3 tou2zi5}
  \end{phonetics}
\end{entry}

\begin{entry}{老头子}{6,5,3}{⽼、⼤、⼦}
  \begin{phonetics}{老头子}{lao3 tou2zi5}
    \definition{s.}{velho antiquado (ou velho rabugento) | (referindo-se ao marido idoso) meu velho | chefe de uma sociedade secreta | (coloquial) velho; velho rabugento}
  \seealsoref{老头儿}{lao3 tou2r5}
  \end{phonetics}
\end{entry}

\begin{entry}{老师}{6,6}{⽼、⼱}
  \begin{phonetics}{老师}{lao3shi1}[][HSK 1]
    \definition[个,位]{s.}{professor; título honorífico para professores; refere-se, de maneira geral, a pessoas que transmitem cultura e tecnologia ou que são dignas de admiração em termos de ideias, moralidade e conhecimentos profissionais}
  \end{phonetics}
\end{entry}

\begin{entry}{老年}{6,6}{⽼、⼲}
  \begin{phonetics}{老年}{lao3 nian2}[][HSK 2]
    \definition[个]{s.}{idoso; velhice; idade acima de 60 ou 70 anos}
  \end{phonetics}
\end{entry}

\begin{entry}{老百姓}{6,6,8}{⽼、⽩、⼥}
  \begin{phonetics}{老百姓}{lao3bai3xing4}[][HSK 3]
    \definition[些]{s.}{povo; civis; pessoas comuns; residentes (em contraste com militares e funcionários públicos)}
  \end{phonetics}
\end{entry}

\begin{entry}{老兵}{6,7}{⽼、⼋}
  \begin{phonetics}{老兵}{lao3bing1}
    \definition{s.}{velho soldado | veterano de guerra | veterano (alguém que tem muita experiência em algum domínio)}
  \end{phonetics}
\end{entry}

\begin{entry}{老实}{6,8}{⽼、⼧}
  \begin{phonetics}{老实}{lao3shi5}[][HSK 4]
    \definition{adj.}{franco; sincero; honesto | bom; bem-comportado | ingênuo; simplório; meio bobo; facilmente enganado; eufemismo para pouco inteligente}
  \end{phonetics}
\end{entry}

\begin{entry}{老朋友}{6,8,4}{⽼、⽉、⼜}
  \begin{phonetics}{老朋友}{lao3 peng2 you3}[][HSK 2]
    \definition[个,位,名]{s.}{velho amigo; refere-se a amigos que conhecemos há muito tempo e com quem temos uma relação íntima}
  \end{phonetics}
\end{entry}

\begin{entry}{老板}{6,8}{⽼、⽊}
  \begin{phonetics}{老板}{lao3ban3}[][HSK 3]
    \definition[个,位]{s.}{chefe; dono; líder; refere-se ao gerente de uma empresa comercial ou industrial | antigo título honorífico dado a atores famosos de ópera ou atores que também eram diretores de companhias de ópera}
  \end{phonetics}
\end{entry}

\begin{entry}{老虎}{6,8}{⽼、⾌}
  \begin{phonetics}{老虎}{lao3hu3}
    \definition[只]{s.}{tigre}
  \seealsoref{虎}{hu3}
  \end{phonetics}
\end{entry}

\begin{entry}{老是}{6,9}{⽼、⽇}
  \begin{phonetics}{老是}{lao3 shi4}[][HSK 2]
    \definition{adv.}{sempre; indica que a ação continua ou que o estado permanece inalterado, equivalente a 一直}
  \seealsoref{一直}{yi4zhi2}
  \end{phonetics}
\end{entry}

\begin{entry}{老家}{6,10}{⽼、⼧}
  \begin{phonetics}{老家}{lao3 jia1}[][HSK 4]
    \definition{s.}{cidade natal; local de origem | lugar nativo; refere-se às gerações anteriores da família ou ao local onde a pessoa nasceu ou viveu}
  \end{phonetics}
\end{entry}

\begin{entry}{老婆}{6,11}{⽼、⼥}
  \begin{phonetics}{老婆}{lao3po2}[][HSK 4]
    \definition[个]{s.}{esposa}
  \end{phonetics}
\end{entry}

\begin{entry}{考}{6}{⽼}
  \begin{phonetics}{考}{kao3}[][HSK 1]
    \definition*{s.}{sobrenome Kao}
    \definition{adj.}{antigo; velho; com idade avançada}
    \definition{s.}{o pai falecido de alguém}
    \definition{v.}{examinar; dar (fazer) um exame, teste ou questionário | verificar; inspecionar | estudar; verificar; investigar | perguntar; testar; fazer perguntas para que o outro responda, a fim de testar suas habilidades em determinada área}
  \end{phonetics}
\end{entry}

\begin{entry}{考生}{6,5}{⽼、⽣}
  \begin{phonetics}{考生}{kao3 sheng1}[][HSK 2]
    \definition{s.}{candidato a exame; alunos inscritos para o exame de admissão}
  \end{phonetics}
\end{entry}

\begin{entry}{考试}{6,8}{⽼、⾔}
  \begin{phonetics}{考试}{kao3shi4}[][HSK 1]
    \definition[次]{s.}{teste; exame; prova; atividades realizadas para verificar conhecimentos ou habilidades}
    \definition{v.+compl.}{testar; avaliar; avaliar conhecimentos e habilidades por meio de perguntas escritas ou orais.}
  \end{phonetics}
\end{entry}

\begin{entry}{考核}{6,10}{⽼、⽊}
  \begin{phonetics}{考核}{kao3he2}[][HSK 5]
    \definition{v.}{examinar; checar; avaliar; avaliar (a proficiência de alguém)}
  \end{phonetics}
\end{entry}

\begin{entry}{考虑}{6,10}{⽼、⾌}
  \begin{phonetics}{考虑}{kao3lv4}[][HSK 4]
    \definition{v.}{considerar; refletir sobre; levar em conta}
  \end{phonetics}
\end{entry}

\begin{entry}{考验}{6,10}{⽼、⾺}
  \begin{phonetics}{考验}{kao3yan4}[][HSK 3]
    \definition[场,个,种]{s.}{teste; julgamento; atividade realizada para verificar se as habilidades, ideias, moral e qualidades de uma pessoa atendem aos requisitos}
    \definition{v.}{testar; testar as capacidades, ideias, moral e qualidades de uma pessoa através de situações, ações ou ambientes difíceis, para verificar se elas atendem aos requisitos}
  \end{phonetics}
\end{entry}

\begin{entry}{考察}{6,14}{⽼、⼧}
  \begin{phonetics}{考察}{kao3cha2}[][HSK 4]
    \definition{v.}{inspecionar; investigar; observar e estudar}
  \end{phonetics}
\end{entry}

\begin{entry}{而}{6}{⽽}[Kangxi 126]
  \begin{phonetics}{而}{er2}[][HSK 4]
    \definition{conj.}{e (coordenação) | e ainda (restrição) | conexão de componentes com continuidade semântica | conecxão de componentes afirmativos e negativos que se complementam | conexão de componentes com significados opostos para indicar um contraste |  conexão de componentes de causa e efeito no raciocínio | significa “chegar” ou “alcançar” | conexão de componentes que indicam tempo ou modo ao verbo | inserido entre o sujeito e o predicado, significa 如果}
  \seealsoref{如果}{ru2guo3}
  \end{phonetics}
\end{entry}

\begin{entry}{而且}{6,5}{⽽、⼀}
  \begin{phonetics}{而且}{er2 qie3}[][HSK 2]
    \definition{conj.}{e também; indica igualdade | e isso; não só\dots mas (também); indica um passo adiante}
  \end{phonetics}
\end{entry}

\begin{entry}{而况}{6,7}{⽽、⼎}
  \begin{phonetics}{而况}{er2kuang4}
    \definition{conj.}{além disso | além do mais}
  \end{phonetics}
\end{entry}

\begin{entry}{而是}{6,9}{⽽、⽇}
  \begin{phonetics}{而是}{er2 shi4}[][HSK 4]
    \definition{conj.}{mas; em vez disso; geralmente usada em conjunto com 不是 para formar o correlativo 不是……而是, indicando uma relação paralela}
  \seealsoref{不是……而是}{bu4shi4 er2 shi4}
  \end{phonetics}
\end{entry}

\begin{entry}{耳}{6}{⽿}
  \begin{phonetics}{耳}{er3}
    \definition*{s.}{sobrenome Er}
    \definition{part.}{(clássico) somente; apenas}
    \definition{s.}{orelha | coisa parecida com uma orelha | em ambos os lados; lado | orelha de um utensílio}
  \end{phonetics}
\end{entry}

\begin{entry}{耳朵}{6,6}{⽿、⽊}
  \begin{phonetics}{耳朵}{er3duo5}[][HSK 5]
    \definition[双,只,个,对]{s.}{orelha; ouvido; órgão da audição e do equilíbrio}
  \end{phonetics}
\end{entry}

\begin{entry}{耳机}{6,6}{⽿、⽊}
  \begin{phonetics}{耳机}{er3 ji1}[][HSK 4]
    \definition[副,个,对]{s.}{fone de ouvido; receptor (de telefone); dispositivos que permitem que uma pessoa ouça sons sozinha, como ouvir música, histórias, chamadas telefônicas etc., usados na cabeça ou inseridos nos ouvidos}
  \end{phonetics}
\end{entry}

\begin{entry}{肉}{6}{⾁}[Kangxi 130]
  \begin{phonetics}{肉}{rou4}[][HSK 1]
    \definition{adj.}{não crocante; mole | lento (em movimento); preguiçoso | carnal; erótico}
    \definition[块]{s.}{carne (especialmente carne de porco) | carne | polpa (da fruta)}
  \end{phonetics}
\end{entry}

\begin{entry}{肉桂}{6,10}{⾁、⽊}
  \begin{phonetics}{肉桂}{rou4gui4}
    \definition{s.}{canela}
  \seealsoref{官桂}{guan1gui4}
  \end{phonetics}
\end{entry}

\begin{entry}{肌}{6}{⾁}
  \begin{phonetics}{肌}{ji1}
    \definition[块,片]{s.}{músculo; carne | pele;}
  \end{phonetics}
\end{entry}

\begin{entry}{肌肉}{6,6}{⾁、⾁}
  \begin{phonetics}{肌肉}{ji1rou4}[][HSK 5]
    \definition[身,块]{s.}{músculo; um dos tecidos básicos dos músculos humanos e de alguns animais, composto principalmente de células musculares fibrosas, pode se contrair, é o movimento do corpo e o corpo de digestão, respiração, circulação, excreção e outros processos fisiológicos da fonte de energia; pode ser dividido em três tipos: músculo liso, músculo esquelético e músculo cardíaco}
  \end{phonetics}
\end{entry}

\begin{entry}{自}{6}{⾃}[Kangxi 132]
  \begin{phonetics}{自}{zi4}[][HSK 4]
    \definition*{s.}{sobrenome Zi}
    \definition{adv.}{certamente; com certeza; é claro; naturalmente}
    \definition{prep.}{de; desde; a partir de; apresenta o ponto de partida, a fonte ou o horário de início do comportamento da ação, equivalente a 从 e 由}
    \definition{pron.}{si mesmo; próprio | próprio; indica que a ação é iniciada por e direcionada a si mesmo | por si mesmo; indica que a ação é autoiniciada e não é causada por uma força externa}
    \definition{v.}{iniciar}
  \seealsoref{从}{cong2}
  \seealsoref{由}{you2}
  \end{phonetics}
\end{entry}

\begin{entry}{自个儿}{6,3,2}{⾃、⼈、⼉}
  \begin{phonetics}{自个儿}{zi4ge3r5}
    \definition{pron.}{(dialeto) a si mesmo, por si mesmo}
  \end{phonetics}
\end{entry}

\begin{entry}{自己}{6,3}{⾃、⼰}
  \begin{phonetics}{自己}{zi4ji3}[][HSK 2]
    \definition{pron.}{a si próprio; a si mesmo; refere-se ao substantivo ou pronome precedente (enfatiza principalmente que não é devido a forças externas)}
  \end{phonetics}
\end{entry}

\begin{entry}{自己动手}{6,3,6,4}{⾃、⼰、⼒、⼿}
  \begin{phonetics}{自己动手}{zi4ji3dong4shou3}
    \definition{v.}{fazer (algo) sozinho | ajudar-se a}
  \end{phonetics}
\end{entry}

\begin{entry}{自从}{6,4}{⾃、⼈}
  \begin{phonetics}{自从}{zi4cong2}[][HSK 3]
    \definition{prep.}{de; desde; a partir de; referir-se a um momento ou evento específico no passado}
  \end{phonetics}
\end{entry}

\begin{entry}{自主}{6,5}{⾃、⼂}
  \begin{phonetics}{自主}{zi4zhu3}[][HSK 3]
    \definition{v.}{agir por conta própria; decidir por si mesmo; manter a iniciativa em suas próprias mãos; tomar suas próprias decisões}
  \end{phonetics}
\end{entry}

\begin{entry}{自由}{6,5}{⾃、⽥}
  \begin{phonetics}{自由}{zi4you2}[][HSK 2]
    \definition{adj.}{livre; irrestrito}
    \definition[个]{s.}{liberdade; o direito de agir de acordo com a própria vontade dentro do âmbito da lei | liberdade; filosoficamente, liberdade é definida como o processo de as pessoas reconhecerem as leis que governam o desenvolvimento das coisas e aplicá-las conscientemente na prática}
  \end{phonetics}
\end{entry}

\begin{entry}{自由泳}{6,5,8}{⾃、⽥、⽔}
  \begin{phonetics}{自由泳}{zi4you2yong3}
    \definition{s.}{natação de estilo livre}
  \end{phonetics}
\end{entry}

\begin{entry}{自动}{6,6}{⾃、⼒}
  \begin{phonetics}{自动}{zi4dong4}[][HSK 3]
    \definition{adj.}{automático; auto-atuante; uso de dispositivos mecânicos, elétricos, etc, para funcionar automaticamente, sem necessidade de controle humano}
    \definition{adv.}{voluntariamente; por vontade própria; por iniciativa própria | automaticamente; espontaneamente; refere-se a movimentos, mudanças, etc., que não são causados pela ação humana, mas sim pelo próprio objeto}
  \end{phonetics}
\end{entry}

\begin{entry}{自动化}{6,6,4}{⾃、⼒、⼔}
  \begin{phonetics}{自动化}{zi4dong4hua4}
    \definition{s.}{automação}
  \end{phonetics}
\end{entry}

\begin{entry}{自杀}{6,6}{⾃、⽊}
  \begin{phonetics}{自杀}{zi4 sha1}[][HSK 5]
    \definition{s.}{suicídio; auto-assassinato; auto-sacrifício}
    \definition{v.}{cometer suicídio; tentar suicídio; matar-se}
  \end{phonetics}
\end{entry}

\begin{entry}{自行车}{6,6,4}{⾃、⾏、⾞}
  \begin{phonetics}{自行车}{zi4xing2che1}[][HSK 2]
    \definition[辆]{s.}{bicicleta; um veículo de duas rodas que é impulsionado para a frente com os pedais}
  \end{phonetics}
\end{entry}

\begin{entry}{自行车架}{6,6,4,9}{⾃、⾏、⾞、⽊}
  \begin{phonetics}{自行车架}{zi4xing2che1jia4}
    \definition{s.}{suporte para bicicleta | bicicletário}
  \end{phonetics}
\end{entry}

\begin{entry}{自行车馆}{6,6,4,11}{⾃、⾏、⾞、⾷}
  \begin{phonetics}{自行车馆}{zi4xing2che1guan3}
    \definition{s.}{estádio de ciclismo | velódromo}
  \end{phonetics}
\end{entry}

\begin{entry}{自行车赛}{6,6,4,14}{⾃、⾏、⾞、⾙}
  \begin{phonetics}{自行车赛}{zi4xing2che1sai4}
    \definition{s.}{corrida de bicicleta}
  \end{phonetics}
\end{entry}

\begin{entry}{自我}{6,7}{⾃、⼽}
  \begin{phonetics}{自我}{zi4wo3}
    \definition{pref.}{auto}
    \definition{pron.}{a si mesmo | eu próprio | (psicologia) ego}
  \end{phonetics}
\end{entry}

\begin{entry}{自我介绍}{6,7,4,8}{⾃、⼽、⼈、⽷}
  \begin{phonetics}{自我介绍}{zi4wo3jie4shao4}
    \definition{s.}{defesa pessoal | auto-defesa}
  \end{phonetics}
\end{entry}

\begin{entry}{自我安慰}{6,7,6,15}{⾃、⼽、⼧、⼼}
  \begin{phonetics}{自我安慰}{zi4wo3'an1wei4}
    \definition{v.}{confortar-se | consolar-se | tranquilizar-se}
  \end{phonetics}
\end{entry}

\begin{entry}{自我防卫}{6,7,6,3}{⾃、⼽、⾩、⼙}
  \begin{phonetics}{自我防卫}{zi4wo3fang2wei4}
    \definition{s.}{defesa pessoal | auto-defesa}
  \end{phonetics}
\end{entry}

\begin{entry}{自我吹嘘}{6,7,7,14}{⾃、⼽、⼝、⼝}
  \begin{phonetics}{自我吹嘘}{zi4wo3chui1xu1}
    \definition{expr.}{tocar a própria buzina}
  \end{phonetics}
\end{entry}

\begin{entry}{自我批评}{6,7,7,7}{⾃、⼽、⼿、⾔}
  \begin{phonetics}{自我批评}{zi4wo3pi1ping2}
    \definition{s.}{autocrítica}
  \end{phonetics}
\end{entry}

\begin{entry}{自我实现}{6,7,8,8}{⾃、⼽、⼧、⾒}
  \begin{phonetics}{自我实现}{zi4wo3shi2xian4}
    \definition{s.}{(psicologia) auto-atualização, auto-realização}
  \end{phonetics}
\end{entry}

\begin{entry}{自我的人}{6,7,8,2}{⾃、⼽、⽩、⼈}
  \begin{phonetics}{自我的人}{zi4wo3de5ren2}
    \definition{s.}{(minha, sua) própria pessoa | (afirmar) a própria personalidade}
  \end{phonetics}
\end{entry}

\begin{entry}{自我保存}{6,7,9,6}{⾃、⼽、⼈、⼦}
  \begin{phonetics}{自我保存}{zi4wo3 bao3cun2}
    \definition{v.}{autopreservação}
  \end{phonetics}
\end{entry}

\begin{entry}{自我陶醉}{6,7,10,15}{⾃、⼽、⾩、⾣}
  \begin{phonetics}{自我陶醉}{zi4wo3tao2zui4}
    \definition{s.}{narcisista | auto-imbuído | satisfeito consigo mesmo}
  \end{phonetics}
\end{entry}

\begin{entry}{自我催眠}{6,7,13,10}{⾃、⼽、⼈、⽬}
  \begin{phonetics}{自我催眠}{zi4wo3cui1mian2}
    \definition{v.}{consolar-me | tranquilizar-me}
  \end{phonetics}
\end{entry}

\begin{entry}{自我意识}{6,7,13,7}{⾃、⼽、⼼、⾔}
  \begin{phonetics}{自我意识}{zi4wo3yi4shi2}
    \definition{s.}{autoapresentação}
    \definition{v.}{apresentar-se}
  \end{phonetics}
\end{entry}

\begin{entry}{自我解嘲}{6,7,13,15}{⾃、⼽、⾓、⼝}
  \begin{phonetics}{自我解嘲}{zi4wo3jie3chao2}
    \definition{s.}{referir-se às próprias fraquezas ou falhas com humor autodepreciativo}
  \end{phonetics}
\end{entry}

\begin{entry}{自来水}{6,7,4}{⾃、⽊、⽔}
  \begin{phonetics}{自来水}{zi4lai2shui3}
    \definition{s.}{água corrente | água da torneira}
  \end{phonetics}
\end{entry}

\begin{entry}{自身}{6,7}{⾃、⾝}
  \begin{phonetics}{自身}{zi4 shen1}[][HSK 3]
    \definition{pron.}{eu mesmo (enfatizando que não é outra pessoa ou outra coisa)}
  \end{phonetics}
\end{entry}

\begin{entry}{自责}{6,8}{⾃、⾙}
  \begin{phonetics}{自责}{zi4ze2}
    \definition{v.}{culpar-se}
  \end{phonetics}
\end{entry}

\begin{entry}{自信}{6,9}{⾃、⼈}
  \begin{phonetics}{自信}{zi4xin4}[][HSK 4]
    \definition{adj.}{confiante; descreve a crença em suas próprias habilidades, decisões, etc., tendo confiança em si mesmo}
    \definition[份,种]{s.}{autoconfiança; confiança em si mesmo}
    \definition{v.}{acreditar em si mesmo;}
  \end{phonetics}
\end{entry}

\begin{entry}{自觉}{6,9}{⾃、⾒}
  \begin{phonetics}{自觉}{zi4jue2}[][HSK 3]
    \definition{adj.}{autoconsciente; de ​​livre e espontânea vontade; controlar o próprio comportamento e agir por iniciativa própria}
    \definition{v.}{estar ciente de}
  \end{phonetics}
\end{entry}

\begin{entry}{自救}{6,11}{⾃、⽁}
  \begin{phonetics}{自救}{zi4jiu4}
    \definition{v.}{sair a si mesmo de problemas}
  \end{phonetics}
\end{entry}

\begin{entry}{自然}{6,12}{⾃、⽕}
  \begin{phonetics}{自然}{zi4ran2}[][HSK 3]
    \definition{adj.}{natural; no curso normal dos eventos; formado ou desenvolvido sem intervenção humana; algo que se desenvolve livremente}
    \definition{adv.}{naturalmente; definitivamente; certamente, isso significa que, de acordo com a lógica, deve ser assim}
    \definition{conj.}{usado para ligar duas frases, com a segunda introduzindo informações adicionais ou adversativas; indica explicação complementar ou uma mudança de significado}
    \definition{s.}{natureza; mundo natural; tudo o que não foi criado pelo ser humano}
  \end{phonetics}
\end{entry}

\begin{entry}{自愿}{6,14}{⾃、⽕}
  \begin{phonetics}{自愿}{zi4yuan4}[][HSK 5]
    \definition{adv.}{voluntariamente; por iniciativa própria; por vontade própria}
    \definition{s.}{voluntário}
  \end{phonetics}
\end{entry}

\begin{entry}{自豪}{6,14}{⾃、⾗}
  \begin{phonetics}{自豪}{zi4hao2}[][HSK 5]
    \definition{adj.}{orgulhar-se de; ter orgulho de; sentir-se honrado por possuir qualidades excelentes ou ter alcançado grandes conquistas, seja por si mesmo ou por um grupo ou indivíduo relacionado a si}
  \end{phonetics}
\end{entry}

\begin{entry}{自燃}{6,16}{⾃、⽕}
  \begin{phonetics}{自燃}{zi4ran2}
    \definition{s.}{combustão espontânea}
  \end{phonetics}
\end{entry}

\begin{entry}{至}{6}{⾄}
  \begin{phonetics}{至}{zhi4}[][HSK 5]
    \definition{adv.}{a maior parte; extremamente; indica o grau mais alto, equivalente a 极 ou 最}
    \definition{prep.}{para; até; chegar a um determinado ponto}
    \definition{s.}{extremo, máximo}
    \definition{v.}{chegar; alcançar}
  \seealsoref{极}{ji2}
  \seealsoref{最}{zui4}
  \end{phonetics}
\end{entry}

\begin{entry}{至于}{6,3}{⾄、⼆}
  \begin{phonetics}{至于}{zhi4yu2}
    \definition{conj.}{para | quanto a | a respeiro de}
  \end{phonetics}
\end{entry}

\begin{entry}{至今}{6,4}{⾄、⼈}
  \begin{phonetics}{至今}{zhi4jin1}[][HSK 3]
    \definition{adv.}{até agora; até o momento; até hoje}
  \end{phonetics}
\end{entry}

\begin{entry}{至少}{6,4}{⾄、⼩}
  \begin{phonetics}{至少}{zhi4shao3}[][HSK 3]
    \definition{adv.}{pelo menos; indica o limite mínimo}
  \end{phonetics}
\end{entry}

\begin{entry}{舌}{6}{⾆}
  \begin{phonetics}{舌}{she2}
    \definition*{s.}{sobrenome She}
    \definition[片,条]{s.}{língua (de um ser humano ou animal); glossa | algo em forma de língua | língua de sino; badalo}
  \end{phonetics}
\end{entry}

\begin{entry}{舌头}{6,5}{⾆、⼤}
  \begin{phonetics}{舌头}{she2tou5}
    \definition[个]{s.}{língua | soldado inimigo capturado com o propósito de extrair informações}
  \end{phonetics}
\end{entry}

\begin{entry}{色}{6}{⾊}[Kangxi 139]
  \begin{phonetics}{色}{se4}[][HSK 4]
    \definition[种]{s.}{cor | aparência; semblante; expressão | tipo; gênero; descrição | cena; cenário;  paisagem | qualidade (de metais preciosos, mercadorias, etc.) | aparência feminina; beleza feminina}
  \end{phonetics}
  \begin{phonetics}{色}{shai3}
    \definition[4]{s.}{cor}
  \end{phonetics}
\end{entry}

\begin{entry}{色狼}{6,10}{⾊、⽝}
  \begin{phonetics}{色狼}{se4lang2}
    \definition*{s.}{Sátiro}
    \definition{adj.}{depravado | tarado}
  \end{phonetics}
\end{entry}

\begin{entry}{色彩}{6,11}{⾊、⼺}
  \begin{phonetics}{色彩}{se4cai3}[][HSK 4]
    \definition[种,丝]{s.}{cor; matiz; tonalidade | cor; sabor; característica; metáfora para um determinado estado de espírito ou tendência de pensamento}
  \end{phonetics}
\end{entry}

\begin{entry}{芋头}{6,5}{⾋、⼤}
  \begin{phonetics}{芋头}{yu4tou5}
    \definition{s.}{taro, similar ao inhame e batata doce}
  \end{phonetics}
\end{entry}

\begin{entry}{芋头色}{6,5,6}{⾋、⼤、⾊}
  \begin{phonetics}{芋头色}{yu4tou5se4}
    \definition{s.}{cor lilás}
  \end{phonetics}
\end{entry}

\begin{entry}{芝麻}{6,11}{⾋、⿇}
  \begin{phonetics}{芝麻}{zhi1ma5}
    \definition{s.}{semente de gergelim}
  \end{phonetics}
\end{entry}

\begin{entry}{虫}{6}{⾍}[Kangxi 142]
  \begin{phonetics}{虫}{chong2}
    \definition[只,条]{s.}{inseto; verme | (pejorativo) pessoas que se comportam de forma desprezível | fã; viciado | forma inferior de vida animal, incluindo insetos, larvas de insetos, vermes e criaturas semelhantes | pessoa com uma característica indesejável específica}
  \end{phonetics}
\end{entry}

\begin{entry}{虫子}{6,3}{⾍、⼦}
  \begin{phonetics}{虫子}{chong2 zi5}[][HSK 4]
    \definition[条,只,种]{s.}{percevejo; besouro; inseto; verme; criaturas semelhantes a insetos}
  \end{phonetics}
\end{entry}

\begin{entry}{血}{6}{⾎}[Kangxi 143]
  \begin{phonetics}{血}{xie3}
  \end{phonetics}
  \begin{phonetics}{血}{xue4}[][HSK 3]
    \definition[滴,袋,口,毫升]{s.}{sangue | parente consanguíneo; com laços de parentesco | pessoa ativa e animada; metáfora para uma personalidade ou espírito forte e sincero | medicina tradicional chinesa refere-se à menstruação}
  \end{phonetics}
\end{entry}

\begin{entry}{血汗}{6,6}{⾎、⽔}
  \begin{phonetics}{血汗}{xue4han4}
    \definition{s.}{(fig.) suor e labuta, trabalho duro}
  \end{phonetics}
\end{entry}

\begin{entry}{行}{6}{⾏}
  \begin{phonetics}{行}{hang2}[][HSK 3]
    \definition{adj.}{temporário; improvisado | capaz; competente}
    \definition{adv.}{logo; em breve}
    \definition{clas.}{linha; fileira; coisas usadas para formar filas, linhas}
    \definition{s.}{comportamento; conduta | linha; fileira | empresa comercial; certas instituições comerciais | comércio; profissão; ramo de atividade | especialista; conhecedor; refere-se ao conhecimento e experiência em um determinado setor}
    \definition{v.}{ir; caminhar; viajar | estar atualizado; circular | fazer; executar; realizar | (antes de um verbo dissílabo, indicando a realização de alguma ação) | ficar bem; vai dar certo | (remédio) fazer efeito | classificar (entre irmãos e irmãs por ordem de idade)}
  \end{phonetics}
  \begin{phonetics}{行}{heng2}
    \definition{s.}{usado em 道行}
  \seealsoref{道行}{dao4 heng2}
  \end{phonetics}
  \begin{phonetics}{行}{xing2}[][HSK 1]
    \definition*{s.}{sobrenome Xing}
    \definition{adj.}{de viajar; relacionado a viagens | temporário; improvisado; provisório | capaz; competente}
    \definition{adv.}{em breve}
    \definition{s.}{comportamento; conduta | caligrafia cursiva (na caligrafia chinesa); escrita cursiva}
    \definition{v.}{ir | fazer uma viagem | estar em voga; prevalecer; circular | fazer; executar; realizar; envolver-se em | estar tudo bem; O.K. | indica a realização de uma determinada atividade (usado principalmente antes de verbos dissilábicos) | (em medicina) fazer efeito}
  \end{phonetics}
\end{entry}

\begin{entry}{行人}{6,2}{⾏、⼈}
  \begin{phonetics}{行人}{xing2ren2}[][HSK 2]
    \definition[个]{s.}{pedestre; transeunte; viajante à pé; pessoas caminhando na estrada}
  \end{phonetics}
\end{entry}

\begin{entry}{行为}{6,4}{⾏、⼂}
  \begin{phonetics}{行为}{xing2wei2}[][HSK 2]
    \definition[个,种,类]{s.}{ação; comportamento; conduta; atividades que são controladas por pensamentos e manifestadas externamente}
  \end{phonetics}
\end{entry}

\begin{entry}{行凶}{6,4}{⾏、⼐}
  \begin{phonetics}{行凶}{xing2xiong1}
    \definition{v.+compl.}{cometer agressão física ou assassinato | fazer algo violento}
  \end{phonetics}
\end{entry}

\begin{entry}{行业}{6,5}{⾏、⼀}
  \begin{phonetics}{行业}{hang2ye4}[][HSK 4]
    \definition[种,个]{s.}{comércio; indústria; setor; profissão; categorias em negócios e indústria referem-se a ocupações em geral}
  \end{phonetics}
\end{entry}

\begin{entry}{行礼}{6,5}{⾏、⽰}
  \begin{phonetics}{行礼}{xing2li3}
    \definition{v.}{saudar | fazer saudação}
  \end{phonetics}
\end{entry}

\begin{entry}{行动}{6,6}{⾏、⼒}
  \begin{phonetics}{行动}{xing2dong4}[][HSK 2]
    \definition[次,场,项]{s.}{ação; operação; comportamento;}
    \definition{v.}{circular; mover-se; andar | agir; tomar medidas; atividades para atingir um determinado propósito}
  \end{phonetics}
\end{entry}

\begin{entry}{行李}{6,7}{⾏、⽊}
  \begin{phonetics}{行李}{xing2li5}[][HSK 3]
    \definition[点,个]{s.}{bagagem, malas, cestas de vime, etc. que você leva quando sai de casa}
  \end{phonetics}
\end{entry}

\begin{entry}{行进}{6,7}{⾏、⾡}
  \begin{phonetics}{行进}{xing2jin4}
    \definition{s.}{avançar | movimentar-se para frente}
  \end{phonetics}
\end{entry}

\begin{entry}{行驶}{6,8}{⾏、⾺}
  \begin{phonetics}{行驶}{xing2 shi3}[][HSK 5]
    \definition{v.}{ir; navegar; viajar (utilizando um veículo, navio, etc.);}
  \end{phonetics}
\end{entry}

\begin{entry}{行星}{6,9}{⾏、⽇}
  \begin{phonetics}{行星}{xing2xing1}
    \definition[颗]{s.}{planeta}
  \seealsoref{惑星}{huo4xing1}
  \end{phonetics}
\end{entry}

\begin{entry}{衣}{6}{⾐}[Kangxi 145]
  \begin{phonetics}{衣}{yi1}
    \definition[件]{s.}{roupa}
  \end{phonetics}
  \begin{phonetics}{衣}{yi4}
    \definition{v.}{vestir | vestir-se}
  \end{phonetics}
\end{entry}

\begin{entry}{衣甲}{6,5}{⾐、⽥}
  \begin{phonetics}{衣甲}{yi1jia3}
    \definition{s.}{armadura}
  \end{phonetics}
\end{entry}

\begin{entry}{衣服}{6,8}{⾐、⽉}
  \begin{phonetics}{衣服}{yi1fu5}[][HSK 1]
    \definition[套,件]{s.}{roupas; vestuário; algo que se veste para cobrir o corpo e se proteger do frio}
  \end{phonetics}
\end{entry}

\begin{entry}{衣柜}{6,8}{⾐、⽊}
  \begin{phonetics}{衣柜}{yi1gui4}
    \definition[个]{s.}{armário | guarda-roupa}
  \end{phonetics}
\end{entry}

\begin{entry}{衣架}{6,9}{⾐、⽊}
  \begin{phonetics}{衣架}{yi1 jia4}[][HSK 3]
    \definition[个,副,组]{s.}{cabideiro; móvel para pendurar roupas | estatura; figura; refere-se ao tipo físico de uma pessoa; estrutura corporal}
  \end{phonetics}
\end{entry}

\begin{entry}{西}{6}{⾑}
  \begin{phonetics}{西}{xi1}[][HSK 1]
    \definition*{s.}{sobrenome Xi}
    \definition*{s.}{abreviatura de Espanha | Paraíso Ocidental}
    \definition{s.}{oeste; uma das quatro direções básicas, o lado onde o sol se põe (oposto ao 东) | ocidental; refere-se ao Ocidente (principalmente aos países europeus e americanos) | aqui e ali; em contraposição a 东, significa 到处 ou 零散, 没有次序}
  \seealsoref{到处}{dao4chu4}
  \seealsoref{东}{dong1}
  \seealsoref{零散}{ling2san3}
  \seealsoref{没有次序}{mei2you3 ci4xu4}
  \end{phonetics}
\end{entry}

\begin{entry}{西文}{6,4}{⾑、⽂}
  \begin{phonetics}{西文}{xi1wen2}
    \definition{s.}{espanhol | língua espanhola}
  \seealsoref{西班牙文}{xi1ban1ya2wen2}
  \end{phonetics}
\end{entry}

\begin{entry}{西方}{6,4}{⾑、⽅}
  \begin{phonetics}{西方}{xi1 fang1}[][HSK 2]
    \definition{s.}{oeste | o Ocidente; o Oeste; países europeus e americanos | Paraíso Ocidental, termo budista}
  \end{phonetics}
\end{entry}

\begin{entry}{西兰花}{6,5,7}{⾑、⼋、⾋}
  \begin{phonetics}{西兰花}{xi1lan2hua1}
    \definition{s.}{brócolis}
  \end{phonetics}
\end{entry}

\begin{entry}{西北}{6,5}{⾑、⼔}
  \begin{phonetics}{西北}{xi1 bei3}[][HSK 2]
    \definition{s.}{noroeste | noroeste da China; o Noroeste}
  \end{phonetics}
\end{entry}

\begin{entry}{西半球}{6,5,11}{⾑、⼗、⽟}
  \begin{phonetics}{西半球}{xi1ban4qiu2}
    \definition{s.}{hemisfério oeste}
  \end{phonetics}
\end{entry}

\begin{entry}{西瓜}{6,5}{⾑、⽠}
  \begin{phonetics}{西瓜}{xi1gua1}[][HSK 4]
    \definition[个,颗,粒]{s.}{melancia; fruto que é uma baga de formato grande, globular ou oval, com muita polpa aguada e doce}
  \end{phonetics}
\end{entry}

\begin{entry}{西边}{6,5}{⾑、⾡}
  \begin{phonetics}{西边}{xi1bian1}[][HSK 1]
    \definition{s.}{lado oeste; (oeste) Uma das quatro direções principais; uma das direções cardeais, oposta ao 东方}
  \seealsoref{东方}{dong1 fang1}
  \end{phonetics}
\end{entry}

\begin{entry}{西安}{6,6}{⾑、⼧}
  \begin{phonetics}{西安}{xi1'an1}
    \definition*{s.}{Xi'an}
  \end{phonetics}
\end{entry}

\begin{entry}{西红柿}{6,6,9}{⾑、⽷、⽊}
  \begin{phonetics}{西红柿}{xi1hong2shi4}[][HSK 5]
    \definition[种,只]{s.}{tomate;}
  \end{phonetics}
\end{entry}

\begin{entry}{西西}{6,6}{⾑、⾑}
  \begin{phonetics}{西西}{xi1xi1}
    \definition{num.}{centímetro cúbico}
  \end{phonetics}
\end{entry}

\begin{entry}{西医}{6,7}{⾑、⼖}
  \begin{phonetics}{西医}{xi1 yi1}[][HSK 2]
    \definition[名,位]{s.}{medicina ocidental; medicina introduzida na China a partir da Europa e da América | um médico treinado em medicina ocidental}
  \end{phonetics}
\end{entry}

\begin{entry}{西南}{6,9}{⾑、⼗}
  \begin{phonetics}{西南}{xi1 nan2}[][HSK 2]
    \definition{s.}{sudoeste | o Sudoeste; Sudoeste da China}
  \end{phonetics}
\end{entry}

\begin{entry}{西药}{6,9}{⾑、⾋}
  \begin{phonetics}{西药}{xi1 yao4}
    \definition{s.}{medicina ocidental; refere-se aos medicamentos usados ​​na medicina ocidental, geralmente feitos por métodos sintéticos ou extraídos de produtos naturais, como comprimidos anti-inflamatórios, aspirina, tintura de iodo, penicilina, etc.}
  \end{phonetics}
\end{entry}

\begin{entry}{西语}{6,9}{⾑、⾔}
  \begin{phonetics}{西语}{xi1yu3}
    \definition{s.}{espanhol | língua espanhola}
  \seealsoref{西班牙语}{xi1ban1ya2yu3}
  \end{phonetics}
\end{entry}

\begin{entry}{西面}{6,9}{⾑、⾯}
  \begin{phonetics}{西面}{xi1mian4}
    \definition{s.}{oeste | lado oeste}
  \end{phonetics}
\end{entry}

\begin{entry}{西班牙文}{6,10,4,4}{⾑、⽟、⽛、⽂}
  \begin{phonetics}{西班牙文}{xi1ban1ya2wen2}
    \definition{s.}{espanhol, língua espanhola}
  \seealsoref{西文}{xi1wen2}
  \end{phonetics}
\end{entry}

\begin{entry}{西班牙语}{6,10,4,9}{⾑、⽟、⽛、⾔}
  \begin{phonetics}{西班牙语}{xi1ban1ya2yu3}
    \definition{s.}{espanhol | língua espanhola}
  \seealsoref{西语}{xi1yu3}
  \end{phonetics}
\end{entry}

\begin{entry}{西部}{6,10}{⾑、⾢}
  \begin{phonetics}{西部}{xi1 bu4}[][HSK 3]
    \definition{s.}{(EUA) filme de faroeste; filme de \emph{cowboys}; um faroeste | filme da região ocidental (China) | parte ocidental; região oeste da China}
  \end{phonetics}
\end{entry}

\begin{entry}{西装}{6,12}{⾑、⾐}
  \begin{phonetics}{西装}{xi1 zhuang1}[][HSK 5]
    \definition[件,套,个]{s.}{terno; roupas de estilo ocidental; roupas ocidentais, divididas em masculinas e femininas}
  \end{phonetics}
\end{entry}

\begin{entry}{西蓝花}{6,13,7}{⾑、⾋、⾋}
  \begin{phonetics}{西蓝花}{xi1lan2hua1}
    \variantof{西兰花}
  \end{phonetics}
\end{entry}

\begin{entry}{西餐}{6,16}{⾑、⾷}
  \begin{phonetics}{西餐}{xi1 can1}[][HSK 2]
    \definition[份,顿,桌]{s.}{comida ocidental; comida de estilo ocidental, comida com garfo e faca (diferente da 中餐)}
  \seealsoref{中餐}{zhong1 can1}
  \end{phonetics}
\end{entry}

\begin{entry}{观}{6}{⾒}
  \begin{phonetics}{观}{guan1}
    \definition*{s.}{Templo taoísta; `Koon'}
    \definition{s.}{visão; vista | perspectiva; visão; conceito | aparência; perspectiva | alcance de visão | noção; ideia; conhecimento ou visão das coisas | ponto de vista; postura; uma visão de uma coisa}
    \definition{v.}{olhar para; assistir; observar | contemplar}
  \end{phonetics}
  \begin{phonetics}{观}{guan4}
    \definition*{s.}{sobrenome Guan}
    \definition{s.}{mosteiro taoísta | torre de vigia do portão do palácio | plataforma}
  \end{phonetics}
\end{entry}

\begin{entry}{观众}{6,6}{⾒、⼈}
  \begin{phonetics}{观众}{guan1zhong4}[][HSK 3]
    \definition[位,名,批,个]{s.}{espectador; público; audiência; pessoas que assistem a espetáculos ou competições}
  \end{phonetics}
\end{entry}

\begin{entry}{观念}{6,8}{⾒、⼼}
  \begin{phonetics}{观念}{guan1nian4}[][HSK 3]
    \definition[种,个]{s.}{ideia; conceito; consciência ideológica}
  \end{phonetics}
\end{entry}

\begin{entry}{观点}{6,9}{⾒、⽕}
  \begin{phonetics}{观点}{guan1dian3}[][HSK 2]
    \definition[个,种]{s.}{ponto de vista; perspectiva; a visão ou atitude que se tem sobre algo a partir de uma determinada posição ou perspectiva | ponto de vista; perspectiva; a posição ou perspectiva adotada ao analisar uma questão}
  \end{phonetics}
\end{entry}

\begin{entry}{观看}{6,9}{⾒、⽬}
  \begin{phonetics}{观看}{guan1 kan4}[][HSK 3]
    \definition{v.}{assistir; ver propositadamente; observar}
  \end{phonetics}
\end{entry}

\begin{entry}{观察}{6,14}{⾒、⼧}
  \begin{phonetics}{观察}{guan1cha2}[][HSK 3]
    \definition{v.}{assistir; pesquisar; observar; examinar cuidadosamente coisas ou fenômenos}
  \end{phonetics}
\end{entry}

\begin{entry}{讲}{6}{⾔}
  \begin{phonetics}{讲}{jiang3}[][HSK 2]
    \definition[种]{s.}{palestra; discurso}
    \definition{v.}{contar; falar | explicar; transmitir oralmente; esclarecer | negociar; barganhar | ser exigente com; valorizar; dar importância}
  \end{phonetics}
\end{entry}

\begin{entry}{讲究}{6,7}{⾔、⽳}
  \begin{phonetics}{讲究}{jiang3jiu5}[][HSK 4]
    \definition{adj.}{requintado; elegante; de bom gosto; exigente com a vida e com outros aspectos, buscando alto nível, qualidade e detalhes}
    \definition{s.}{estudo cuidadoso; algo que merece atenção; elementos e aspectos que merecem atenção especial}
    \definition{v.}{dar ênfase a; ser específico sobre; prestar atenção a}
  \end{phonetics}
\end{entry}

\begin{entry}{讲话}{6,8}{⾔、⾔}
  \begin{phonetics}{讲话}{jiang3 hua4}[][HSK 2]
    \definition[个]{s.}{discurso; palestra | guia; introdução}
    \definition{v.}{falar; conversar; dirigir-se a alguém | criticar}
  \end{phonetics}
\end{entry}

\begin{entry}{讲述}{6,8}{⾔、⾡}
  \begin{phonetics}{讲述}{jiang3shu4}
    \definition{v.}{falar sobre | narrar | descrever}
  \end{phonetics}
\end{entry}

\begin{entry}{讲座}{6,10}{⾔、⼴}
  \begin{phonetics}{讲座}{jiang3zuo4}[][HSK 4]
    \definition[个]{s.}{palestra; um curso de palestras; a forma de instrução usada para ensinar um determinado assunto ou tópico, geralmente por meio de palestras ao vivo, seriados de rádio ou televisão ou seriados de jornal.}
  \end{phonetics}
\end{entry}

\begin{entry}{许}{6}{⾔}
  \begin{phonetics}{许}{xu3}
    \definition*{s.}{sobrenome Xu}
    \definition{adv.}{um pouco | talvez}
    \definition{v.}{permitir | prometer | elogiar}
  \end{phonetics}
\end{entry}

\begin{entry}{许可}{6,5}{⾔、⼝}
  \begin{phonetics}{许可}{xu3ke3}[][HSK 5]
    \definition{v.}{permitir; autorizar}
  \end{phonetics}
\end{entry}

\begin{entry}{许多}{6,6}{⾔、⼣}
  \begin{phonetics}{许多}{xu3duo1}[][HSK 2]
    \definition{num.}{muitos; muito; numerosos; uma grande quantidade de}
  \end{phonetics}
\end{entry}

\begin{entry}{论}{6}{⾔}
  \begin{phonetics}{论}{lun2}
    \definition*{s.}{Os Analectos de Confúcio, registro dos ditos e feitos de Confúcio e seus discípulos}
  \end{phonetics}
  \begin{phonetics}{论}{lun4}
    \definition*{s.}{sobrenome Lun}
    \definition{prep.}{por (uma certa unidade de medida) | de acordo com (um certo sistema ou princípio)}
    \definition{s.}{visão; opinião; declaração | (frequentemente em títulos) dissertação; ensaio; tratado | teoria; doutrina | ideia; palavras ou artigos que analisam e explicam coisas}
    \definition{v.}{discutir; falar sobre; discursar sobre; comentar | mencionar; considerar; falar de | decidir sobre; determinar | decidir sobre a natureza da culpa; punir | argumentar; analisar e explicar coisas | considerar; ponderar; medir; avaliar}
  \end{phonetics}
\end{entry}

\begin{entry}{论文}{6,4}{⾔、⽂}
  \begin{phonetics}{论文}{lun4wen2}[][HSK 4]
    \definition[篇]{s.}{tese; redação; artigo; artigo que discute ou examina uma questão}
  \end{phonetics}
\end{entry}

\begin{entry}{设}{6}{⾔}
  \begin{phonetics}{设}{she4}
    \definition*{s.}{sobrenome She}
    \definition{conj.}{se; no caso | (matemática) dado; suponha; se}
    \definition{v.}{configurar; estabelecer; encontrar; colocar em prática}
  \end{phonetics}
\end{entry}

\begin{entry}{设计}{6,4}{⾔、⾔}
  \begin{phonetics}{设计}{she4ji4}[][HSK 3]
    \definition[份]{s.}{plano; esquema; refere-se a um plano de design ou a um projeto para um plano, etc.}
    \definition{v.}{planejar; projetar; formular métodos, desenhos, etc. com antecedência, de acordo com determinados requisitos de finalidade, antes de iniciar oficialmente um trabalho | arquitetar; idear; tramar; fazer um plano}
  \end{phonetics}
\end{entry}

\begin{entry}{设立}{6,5}{⾔、⽴}
  \begin{phonetics}{设立}{she4li4}[][HSK 3]
    \definition{v.}{fundar; estabelecer; começar}
  \end{phonetics}
\end{entry}

\begin{entry}{设备}{6,8}{⾔、⼡}
  \begin{phonetics}{设备}{she4bei4}[][HSK 3]
    \definition[台,套]{s.}{instalação; equipamento; montagem; um conjunto de edifícios ou equipamentos necessários para executar uma determinada tarefa ou suprir uma determinada necessidade}
  \end{phonetics}
\end{entry}

\begin{entry}{设施}{6,9}{⾔、⽅}
  \begin{phonetics}{设施}{she4shi1}[][HSK 4]
    \definition{s.}{facilidade; instalação; instituições, sistemas, organizações, edifícios, etc., estabelecidos para realizar um trabalho ou atender a uma necessidade}
  \end{phonetics}
\end{entry}

\begin{entry}{设想}{6,13}{⾔、⼼}
  \begin{phonetics}{设想}{she4xiang3}[][HSK 5]
    \definition[个,种]{s.}{plano provisório (ou ideia); (item, tipo) refere-se a algo hipotético ou imaginário}
    \definition{v.}{imaginar; prever; conceber; supor | ter consideração por}
  \end{phonetics}
\end{entry}

\begin{entry}{设置}{6,13}{⾔、⽹}
  \begin{phonetics}{设置}{she4zhi4}[][HSK 4]
    \definition{v.}{estabelecer; colocar em prática; estabelecer ou criar instituições, empregos, profissões ou códigos, etc. | encaixar; ajustar; instalar; configurar; colocar}
  \end{phonetics}
\end{entry}

\begin{entry}{访}{6}{⾔}
  \begin{phonetics}{访}{fang3}
    \definition{v.}{visitar; fazer uma visita; ligar para | procurar por meio de investigação ou busca; tentar obter; obter uma entrevista | entrevistar | investigar; procurar por meio de investigação (pesquisar)}
  \end{phonetics}
\end{entry}

\begin{entry}{访问}{6,6}{⾔、⾨}
  \begin{phonetics}{访问}{fang3wen4}[][HSK 3]
    \definition{v.}{visitar; ligar; entrevistar; visitar e conversar com um objetivo específico | visitar um \emph{site}}
  \end{phonetics}
\end{entry}

\begin{entry}{负}{6}{⾙}
  \begin{phonetics}{负}{fu4}
    \definition{adj.}{negativo; menor que zero | negativo; referindo-se ao que recebe elétrons (oposto a 正)}
    \definition{v.}{carregar; transportar nas costas ou nos ombros | suportar; assumir; encarar | confiar em; contar com; depender | sofrer | aproveitar; desfrutar | ter dívidas | trair; violar | perder; ser derrotado}
  \seealsoref{正}{zheng4}
  \end{phonetics}
\end{entry}

\begin{entry}{负担}{6,8}{⾙、⼿}
  \begin{phonetics}{负担}{fu4dan1}[][HSK 4]
    \definition{s.}{carga; fardo; frete; ônus; pressão ou responsabilidade, despesas, etc.}
    \definition{v.}{carregar; carregar (um fardo); assumir (responsabilidade, trabalho, despesas, etc.)}
  \end{phonetics}
\end{entry}

\begin{entry}{负责}{6,8}{⾙、⾙}
  \begin{phonetics}{负责}{fu4ze2}[][HSK 3]
    \definition{adj.}{consciencioso; ser sério e responsável}
    \definition{v.}{ser responsável por; estar encarregado de; assumir responsabilidades}
  \end{phonetics}
\end{entry}

\begin{entry}{负责人}{6,8,2}{⾙、⾙、⼈}
  \begin{phonetics}{负责人}{fu4 ze2 ren2}[][HSK 5]
    \definition{s.}{pessoa responsável; pessoa encarregada; pessoas com responsabilidades de liderança}
  \end{phonetics}
\end{entry}

\begin{entry}{赱}{6}{⼟}
  \begin{phonetics}{赱}{zou3}
    \variantof{走}
  \end{phonetics}
\end{entry}

\begin{entry}{达}{6}{⾡}
  \begin{phonetics}{达}{da2}
    \definition*{s.}{sobrenome Da}
    \definition{adj.}{eminente; distinto; refere-se a um funcionário distinto; \emph{status} elevado | otimista; de mente aberta}
    \definition{v.}{prolongar | alcançar; atingir; equivaler a | entender completamente; compreender (assuntos) | expressar; comunicar}
  \end{phonetics}
\end{entry}

\begin{entry}{达成}{6,6}{⾡、⼽}
  \begin{phonetics}{达成}{da2cheng2}[][HSK 5]
    \definition{v.}{concluir; chegar (a um acordo); conseguir; obter (principalmente como resultado de uma negociação)}
  \end{phonetics}
\end{entry}

\begin{entry}{达到}{6,8}{⾡、⼑}
  \begin{phonetics}{达到}{da2dao4}[][HSK 3]
    \definition{v.}{alcançar; atender o padrão; atingir (refere-se principalmente a coisas abstratas ou graus); chegar a um determinado ponto ou grau}
  \end{phonetics}
\end{entry}

\begin{entry}{迅速}{6,10}{⾡、⾡}
  \begin{phonetics}{迅速}{xun4su4}[][HSK 4]
    \definition{adv.}{rapidamente; velozmente; prontamente}
  \end{phonetics}
\end{entry}

\begin{entry}{过}{6}{⾡}
  \begin{phonetics}{过}{guo1}
    \definition*{s.}{sobrenome Guo}
  \end{phonetics}
  \begin{phonetics}{过}{guo4}[][HSK 1,2]
    \definition{adv.}{excessivamente; em excesso}
    \definition{clas.}{tempo; número de vezes usado para a ação}
    \definition{s.}{falha; erro; demérito; equívoco; negligência; (oposto a 功)}
    \definition{v.}{cruzar; passar; mudar-se de um lugar para outro; passar por | exceder; ir além; ultrapassar; usado após um adjetivo, significa ``mais do que'' | gastar (tempo); passar (tempo); exceder (um determinado limite ou limite) | celebrar; comemorar | mudar; transferir; transferir de um lado para o outro | passar por um processo; passar por; submeter a (algum tipo de tratamento) | visitar; fazer uma visita | falecer; morrer | infectar; ser contagioso; espalhar | exceder; ir além; usado após o verbo com o sufixo 得, significa ``superar'' ou ``passar'' | viver | revisar; examinar; usar os olhos para ver ou a mente para lembrar}
  \seealsoref{得}{de5}
  \seealsoref{功}{gong1}
  \end{phonetics}
  \begin{phonetics}{过}{guo5}
    \definition{part.}{usado depois de um verbo para indicar conclusão | usado depois de um verbo para indicar que uma ação ou mudança ocorreu | usado depois de um adjetivo para indicar que algo já teve uma certa qualidade ou estado e para compará-lo com o presente}
  \end{phonetics}
\end{entry}

\begin{entry}{过于}{6,3}{⾡、⼆}
  \begin{phonetics}{过于}{guo4yu2}[][HSK 5]
    \definition{adv.}{demais; indevidamente; excessivamente; advérbios de grau ou quantidade excessiva}
  \end{phonetics}
\end{entry}

\begin{entry}{过不惯}{6,4,11}{⾡、⼀、⼼}
  \begin{phonetics}{过不惯}{guo4 bu5 guan4}
    \definition{v.}{não se acostumar | não se habituar}
  \seealsoref{过惯}{guo4guan4}
  \end{phonetics}
\end{entry}

\begin{entry}{过分}{6,4}{⾡、⼑}
  \begin{phonetics}{过分}{guo4fen4}[][HSK 4]
    \definition{adj.}{excessivo; muito longe; demais; falar ou agir além dos limites ou graus adequados}
    \definition{adv.}{excessivamente; indevidamente; muito mesmo}
  \end{phonetics}
\end{entry}

\begin{entry}{过去}{6,5}{⾡、⼛}
  \begin{phonetics}{过去}{guo4 qu4}[][HSK 2,3]
    \definition{adv.}{(no) passado}
    \definition{s.}{o passado; refere-se a um período anterior; também se refere a coisas anteriores}
    \definition{v.}{atravessar; passar; sair do local onde o interlocutor se encontra e deslocar-se para outro local | acabar; passar; ficar para trás; indica que já passou por uma determinada fase | passar; indica que um determinado período ou situação já não existe mais | falecer | ir lá | passar por}
  \end{phonetics}
\end{entry}

\begin{entry}{过节}{6,5}{⾡、⾋}
  \begin{phonetics}{过节}{guo4jie2}
    \definition{v.+compl.}{celebrar festividades | comemorar um festival}
  \end{phonetics}
\end{entry}

\begin{entry}{过关}{6,6}{⾡、⼋}
  \begin{phonetics}{过关}{guo4guan1}
    \definition{v.+compl.}{passar uma barreira | passar por uma provação | passar em um teste | atingir um padrão | passar pela alfândega}
  \end{phonetics}
\end{entry}

\begin{entry}{过年}{6,6}{⾡、⼲}
  \begin{phonetics}{过年}{guo4 nian2}[][HSK 2]
    \definition{v.+compl.}{comemorar o Ano Novo; comemorar o Festival da Primavera; passar o Ano Novo; passar o Festival da Primavera; realizar atividades comemorativas durante o Ano Novo ou o Festival da Primavera}
  \end{phonetics}
\end{entry}

\begin{entry}{过来}{6,7}{⾡、⽊}
  \begin{phonetics}{过来}{guo4 lai2}[][HSK 2]
    \definition{v.}{vir até aqui | ser capaz de cuidar de | lidar com | administrar}
  \end{phonetics}
\end{entry}

\begin{entry}{过度}{6,9}{⾡、⼴}
  \begin{phonetics}{过度}{guo4du4}[][HSK 5]
    \definition{adj.}{excessivo; acima do limite; além do limite; além do que é apropriado}
  \end{phonetics}
\end{entry}

\begin{entry}{过惯}{6,11}{⾡、⼼}
  \begin{phonetics}{过惯}{guo4guan4}
    \definition{v.}{estar acostumado (a um certo estilo de vida, etc.)}
  \seealsoref{过不惯}{guo4 bu5 guan4}
  \end{phonetics}
\end{entry}

\begin{entry}{过敏}{6,11}{⾡、⽁}
  \begin{phonetics}{过敏}{guo4min3}[][HSK 5]
    \definition{adj.}{sensível; excessivamente sensível; resposta acima do normal; ceticismo excessivo}
    \definition{v.}{ser alérgico a}
  \end{phonetics}
\end{entry}

\begin{entry}{过期}{6,12}{⾡、⽉}
  \begin{phonetics}{过期}{guo4qi1}
    \definition{v.+compl.}{exceder a data | passar a data | expirar (passar a data de expiração)}
  \end{phonetics}
\end{entry}

\begin{entry}{过程}{6,12}{⾡、⽲}
  \begin{phonetics}{过程}{guo4cheng2}[][HSK 3]
    \definition[个,段]{s.}{curso dos eventos; processo; o processo pelo qual as coisas acontecem ou se desenvolvem.}
  \end{phonetics}
\end{entry}

\begin{entry}{过瘾}{6,16}{⾡、⽧}
  \begin{phonetics}{过瘾}{guo4yin3}
    \definition{adj.}{gratificante | imensamente agradável | satisfatório}
    \definition{v.+compl.}{satisfazer um desejo | se divertir com algo}
  \end{phonetics}
\end{entry}

\begin{entry}{那}{6}{⾢}
  \begin{phonetics}{那}{na1}
    \definition*{s.}{sobrenome Na}
  \end{phonetics}
  \begin{phonetics}{那}{na3}
    \definition{adv.}{expressa negação em perguntas retóricas}
    \definition{pron.}{qual? | qualquer que seja; qualquer que; para expressar incerteza em uma declaração | variante de 哪}
  \seealsoref{哪}{na3}
  \end{phonetics}
  \begin{phonetics}{那}{na4}[][HSK 1,2]
    \definition{conj.}{então; nessa situação; nesse caso; o mesmo que 那么}
    \definition{pron.}{aquele; aquilo; indica pessoas ou coisas distantes | aquele; aquilo; expressa muitas coisas, sem se referir especificamente a uma pessoa ou coisa, e é frequentemente usado em conjunto com 这}
  \seealsoref{那么}{na4 me5}
  \seealsoref{这}{zhe4}
  \end{phonetics}
  \begin{phonetics}{那}{ne4}
    \definition{conj.}{então; nesse caso; o mesmo que 那么}
    \definition{pron.}{aquele; aquilo; pronúncia coloquial de 那 (\dpy{na4})}
  \seealsoref{那么}{na4 me5}
  \end{phonetics}
  \begin{phonetics}{那}{nei4}
    \definition{conj.}{então; o mesmo que 那么}
    \definition{pron.}{aquele; aquilo; A pronúncia coloquial de 那 (\dpy{na4})}
  \seealsoref{那么}{na4 me5}
  \end{phonetics}
  \begin{phonetics}{那}{nuo2}
    \definition*{s.}{sobrenome Nuo}
  \end{phonetics}
\end{entry}

\begin{entry}{那儿}{6,2}{⾢、⼉}
  \begin{phonetics}{那儿}{na4r5}[][HSK 1]
    \definition{pron.}{lá; ali; naquele lugar | então; naquela época (usado após 打, 从 e 由)}
  \seealsoref{从}{cong2}
  \seealsoref{打}{da3}
  \seealsoref{由}{you2}
  \end{phonetics}
\end{entry}

\begin{entry}{那个}{6,3}{⾢、⼈}
  \begin{phonetics}{那个}{na4ge5}
    \definition{pron.}{aquele | usado antes de verbos e adjetivos para indicar exagero | para substituir o discurso direto inconveniente}
  \end{phonetics}
\end{entry}

\begin{entry}{那么}{6,3}{⾢、⼃}
  \begin{phonetics}{那么}{na4 me5}[][HSK 2]
    \definition{conj.}{então; nesse caso; afirmar o resultado esperado ou fazer um julgamento}
    \definition{pron.}{assim; dessa maneira; indica a natureza, o estado, a forma, o grau, etc. | assim; sobre; colocado antes do numeral, indica uma estimativa}
  \end{phonetics}
\end{entry}

\begin{entry}{那边}{6,5}{⾢、⾡}
  \begin{phonetics}{那边}{na4 bian5}[][HSK 1]
    \definition{pron.}{ali; acolá; aquele lado}
  \end{phonetics}
\end{entry}

\begin{entry}{那会儿}{6,6,2}{⾢、⼈、⼉}
  \begin{phonetics}{那会儿}{na4 hui4r5}[][HSK 2]
    \definition{pron.}{então; naquela época; refere-se ao passado ou ao futuro}
  \end{phonetics}
\end{entry}

\begin{entry}{那时}{6,7}{⾢、⽇}
  \begin{phonetics}{那时}{na4 shi2}[][HSK 2]
    \definition{pron.}{então; naquela época; naqueles dias; geralmente se refere a um período de tempo distante do presente}
  \seealsoref{那时候}{na4 shi2 hou5}
  \end{phonetics}
\end{entry}

\begin{entry}{那时候}{6,7,10}{⾢、⽇、⼈}
  \begin{phonetics}{那时候}{na4 shi2 hou5}[][HSK 2]
    \definition{adv.}{naquela hora; em algum momento no passado}
  \seealsoref{那时}{na4 shi2}
  \end{phonetics}
\end{entry}

\begin{entry}{那里}{6,7}{⾢、⾥}
  \begin{phonetics}{那里}{na4 li3}[][HSK 1]
    \definition{pron./s.}{lá; ali; aquele lugar; indica um lugar distante}
  \end{phonetics}
\end{entry}

\begin{entry}{那些}{6,8}{⾢、⼆}
  \begin{phonetics}{那些}{na4 xie1}[][HSK 1]
    \definition{pron.}{aqueles; indica duas ou mais pessoas ou coisas}
  \end{phonetics}
\end{entry}

\begin{entry}{那咱}{6,9}{⾢、⼝}
  \begin{phonetics}{那咱}{na4 zan5}
    \definition{s.}{(informal) naquela época; então | (antigo) naquela época}
  \end{phonetics}
\end{entry}

\begin{entry}{那样}{6,10}{⾢、⽊}
  \begin{phonetics}{那样}{na4 yang4}[][HSK 2]
    \definition{pron.}{assim; tal; desse tipo; desse gênero; dessa natureza; desse tipo; indica a natureza, o estado, a maneira, o grau ou refere-se a uma ação ou situação específica}
  \end{phonetics}
\end{entry}

\begin{entry}{那麽}{6,14}{⾢、⿇}
  \begin{phonetics}{那麽}{na4 me5}
    \variantof{那么}
  \end{phonetics}
\end{entry}

\begin{entry}{闭}{6}{⾨}
  \begin{phonetics}{闭}{bi4}
    \definition*{s.}{sobrenome Bi}
    \definition{v.}{fechar; encerrar | bloquear; obstruir; parar}
  \end{phonetics}
\end{entry}

\begin{entry}{闭幕}{6,13}{⾨、⼱}
  \begin{phonetics}{闭幕}{bi4 mu4}[][HSK 5]
    \definition{v.+compl.}{fechar; concluir; (conferência, exposição, etc.) terminar | cair a cortina; abaixar a cortina; terminar a apresentação e a cortina se fechar em frente ao palco}
  \end{phonetics}
\end{entry}

\begin{entry}{闭幕式}{6,13,6}{⾨、⼱、⼷}
  \begin{phonetics}{闭幕式}{bi4 mu4 shi4}[][HSK 5]
    \definition{s.}{cerimônia de encerramento; cerimônia formal realizada no final de uma conferência ou exposição}
  \end{phonetics}
\end{entry}

\begin{entry}{闭嘴}{6,16}{⾨、⼝}
  \begin{phonetics}{闭嘴}{bi4zui3}
    \definition{expr.}{Cale-se!}
  \end{phonetics}
\end{entry}

\begin{entry}{问}{6}{⾨}
  \begin{phonetics}{问}{wen4}[][HSK 1]
    \definition*{s.}{sobrenome Wen}
    \definition{prep.}{de; introduzir o objeto da ação, equivalente a 向 e 跟}
    \definition{v.}{perguntar; indagar; fazer com que as pessoas respondam ou esclareçam coisas que não sabem ou não têm certeza | perguntar (ou indagar) sobre | examinar; interrogar | intervir; responsabilizar; investigar | cuidar; preocupar-se; gerenciar; interferir}
  \seealsoref{跟}{gen1}
  \seealsoref{向}{xiang4}
  \end{phonetics}
\end{entry}

\begin{entry}{问市}{6,5}{⾨、⼱}
  \begin{phonetics}{问市}{wen4shi4}
    \definition{v.}{chegar ao mercado | bater o mercado | atingir o mercado}
  \end{phonetics}
\end{entry}

\begin{entry}{问安}{6,6}{⾨、⼧}
  \begin{phonetics}{问安}{wen4'an1}
    \definition{s.}{saudações}
    \definition{v.}{dar cumprimentos a | prestar homenagem}
  \end{phonetics}
\end{entry}

\begin{entry}{问卷}{6,8}{⾨、⼙}
  \begin{phonetics}{问卷}{wen4juan4}
    \definition[份]{s.}{questionário}
  \end{phonetics}
\end{entry}

\begin{entry}{问候}{6,10}{⾨、⼈}
  \begin{phonetics}{问候}{wen4hou4}[][HSK 4]
    \definition{s.}{homenagem | saudação}
    \definition{v.}{prestar homenagem; enviar uma saudação;  dar os respeitos (cumprimentos) a alguém | (fig.) (coloquial) fazer referência ofensiva a (alguém querido pela pessoa com quem se está falando)}
  \end{phonetics}
\end{entry}

\begin{entry}{问鼎}{6,12}{⾨、⿍}
  \begin{phonetics}{问鼎}{wen4ding3}
    \definition{v.}{visar (o primeiro lugar, etc.) | aspirar ao trono}
  \end{phonetics}
\end{entry}

\begin{entry}{问路}{6,13}{⾨、⾜}
  \begin{phonetics}{问路}{wen4 lu4}[][HSK 2]
    \definition{v.}{perguntar o caminho; pedir direções}
  \end{phonetics}
\end{entry}

\begin{entry}{问题}{6,15}{⾨、⾴}
  \begin{phonetics}{问题}{wen4ti2}[][HSK 2]
    \definition{adj.}{desqualificado; indesejável; anormal, não atende aos requisitos}
    \definition[个,种,类,串]{s.}{pergunta; problema; perguntas a serem respondidas | problema; questão; contradições que precisam ser estudadas e resolvidas | problema; acidente; incidente | chave; ponto crucial; pontos importantes}
  \end{phonetics}
\end{entry}

\begin{entry}{闯}{6}{⾨}
  \begin{phonetics}{闯}{chuang3}[][HSK 5]
    \definition*{s.}{sobrenome Chuang}
    \definition{v.}{apressar-se; correr | moderar a si mesmo (lutando contra dificuldades e perigos); aventurar-se no mundo | incorrer; causar (um desastre, etc.)}
  \end{phonetics}
\end{entry}

\begin{entry}{防}{6}{⾩}
  \begin{phonetics}{防}{fang2}[][HSK 3]
    \definition*{s.}{sobrenome Fang}
    \definition{s.}{defesa | dique; aterro | barragem; represa; estrutura para conter a água}
    \definition{v.}{proteger contra; prevenir contra; tomar precauções contra | defender-se contra}
  \end{phonetics}
\end{entry}

\begin{entry}{防止}{6,4}{⾩、⽌}
  \begin{phonetics}{防止}{fang2zhi3}[][HSK 3]
    \definition{v.}{evitar; prevenir; prevenir; proteger contra; preparar-se com antecedência para evitar que coisas ruins aconteçam}
  \end{phonetics}
\end{entry}

\begin{entry}{防护}{6,7}{⾩、⼿}
  \begin{phonetics}{防护}{fang2hu4}
    \definition{v.}{defender | proteger}
  \end{phonetics}
\end{entry}

\begin{entry}{防治}{6,8}{⾩、⽔}
  \begin{phonetics}{防治}{fang2zhi4}[][HSK 5]
    \definition{s.}{tratamento preventivo; prevenção e cura; profilaxia e tratamento}
  \end{phonetics}
\end{entry}

\begin{entry}{防晒}{6,10}{⾩、⽇}
  \begin{phonetics}{防晒}{fang2shai4}
    \definition{s.}{protetor solar}
  \end{phonetics}
\end{entry}

\begin{entry}{阳}{6}{⾩}
  \begin{phonetics}{阳}{yang2}
    \definition*{s.}{Yang (o princípio positivo de Yin e Yang)}
    \definition{s.}{positivo (eletricidade) | sol}
  \seealsoref{阴}{yin1}
  \seealsoref{阴阳}{yin1yang2}
  \end{phonetics}
\end{entry}

\begin{entry}{阳台}{6,5}{⾩、⼝}
  \begin{phonetics}{阳台}{yang2tai2}[][HSK 4]
    \definition{s.}{varanda; terraço; sacada; pequeno terraço do edifício com grades para se refrescar, tomar sol ou olhar o horizonte}
  \end{phonetics}
\end{entry}

\begin{entry}{阳光}{6,6}{⾩、⼉}
  \begin{phonetics}{阳光}{yang2guang1}[][HSK 3]
    \definition{adj.}{alegre; otimista; personalidade positiva e alegre; cheio de vitalidade juvenil | aberto; transparente; público; conduzido sob supervisão pública}
    \definition[缕,束,道]{s.}{luz do sol; raio de sol}
  \end{phonetics}
\end{entry}

\begin{entry}{阴}{6}{⾩}
  \begin{phonetics}{阴}{yin1}[][HSK 2]
    \definition*{s.}{sobrenome Yin}
    \definition*{s.}{Yin (o princípio negativo de Yin e Yang)}
    \definition{adj.}{nublado; opaco; sombrio | escondido; secreto; não exposto | sinistro | do mundo inferior; dos fantasmas | (física) negativo; cátodo | nublado; mais de 80\% do céu estão cobertos por nuvens | em talhe-doce; rebaixado | (matéria) carregada negativamente}
    \definition[片]{s.}{a Lua; refere-se a Taiyin | sombra; lugar sombrio | partes íntimas (especialmente da mulher) | ao norte de uma colina ou ao sul de um rio | verso | entalhe}
  \seealsoref{阳}{yang2}
  \seealsoref{阴阳}{yin1yang2}
  \end{phonetics}
\end{entry}

\begin{entry}{阴天}{6,4}{⾩、⼤}
  \begin{phonetics}{阴天}{yin1 tian1}[][HSK 2]
    \definition[个]{s.}{nublado; céu nublado; dia nublado; uma condição climática em que 80\% do céu está coberto por nuvens e apenas um pouco de sol pode ser visto}
  \end{phonetics}
\end{entry}

\begin{entry}{阴阳}{6,6}{⾩、⾩}
  \begin{phonetics}{阴阳}{yin1yang2}
    \definition*{s.}{Yin e Yang}
  \seealsoref{阳}{yang2}
  \seealsoref{阴}{yin1}
  \end{phonetics}
\end{entry}

\begin{entry}{阵}{6}{⾩}
  \begin{phonetics}{阵}{zhen4}[][HSK 4]
    \definition{clas.}{passagens que expressam a passagem de eventos ou ações}
    \definition{s.}{matriz de batalha (formação); termo tático antigo para as fileiras ou formações de uma equipe de combate | \emph{front}; frente de batalha; posição | um período de tempo}
  \end{phonetics}
\end{entry}

\begin{entry}{阵地}{6,6}{⾩、⼟}
  \begin{phonetics}{阵地}{zhen4di4}
    \definition{s.}{posição (militar) | frente de batalha | \emph{front}}
  \end{phonetics}
\end{entry}

\begin{entry}{阶}{6}{⾩}
  \begin{phonetics}{阶}{jie1}
    \definition{s.}{degrau; escada; escadaria | classificação | escala | ordem | estágio}
  \end{phonetics}
\end{entry}

\begin{entry}{阶段}{6,9}{⾩、⽎}
  \begin{phonetics}{阶段}{jie1duan4}[][HSK 4]
    \definition{s.}{estágio; fase; período; bancada; gradação}
  \end{phonetics}
\end{entry}

\begin{entry}{页}{6}{⾴}[Kangxi 181]
  \begin{phonetics}{页}{ye4}[][HSK 1]
    \definition{clas.}{página; folha de papel; lâmina; antigamente, referia-se a uma folha de um livro encadernado; atualmente, refere-se a uma das faces de um livro impresso em ambos os lados}
    \definition{s.}{página; folha de papel; folhas soltas de um livro}
  \end{phonetics}
\end{entry}

\begin{entry}{齐}{6}{⿑}[Kangxi 210]
  \begin{phonetics}{齐}{qi2}[][HSK 3]
    \definition*{s.}{sobrenome Qi}
    \definition*{s.}{Qi, um estado da Dinastia Zhou | Dinastia Qi do Sul (479-502), uma das Dinastias do Sul | Dinastia Qi do Norte (550-577), uma das Dinastias do Norte}
    \definition{adj.}{arrumado; uniforme; regular; comprimento, tamanho, etc. são praticamente iguais; uniformes | semelhante; similar; da mesma forma; de acordo| tudo pronto; todos presentes; completo; perfeito}
    \definition{adv.}{juntos; simultaneamente; ao mesmo tempo}
    \definition{v.}{estar no mesmo nível que; alcançar o mesmo nível | estar nivelado em um ponto ou ao longo de uma linha; tornar consistente; harmonizar}
  \end{phonetics}
\end{entry}

\begin{entry}{齐全}{6,6}{⿑、⼊}
  \begin{phonetics}{齐全}{qi2quan2}[][HSK 5]
    \definition{adj.}{completo; tudo pronto}
  \end{phonetics}
\end{entry}

%%%%% EOF %%%%%


%%%
%%% 7画
%%%

\section*{7画}\addcontentsline{toc}{section}{7画}

\begin{entry}{两}{7}{⼀}
  \begin{phonetics}{两}{liang3}[][HSK 1,2]
    \definition*{s.}{sobrenome Liang}
    \definition{clas.}{liang, uma unidade de peso (=50 gramas)}
    \definition{num.}{dois (sempre usado antes de classificadores) | poucos; alguns; indica um número indeterminado}
    \definition{s.}{ambos (lados); qualquer (lado)}
  \end{phonetics}
\end{entry}

\begin{entry}{两边}{7,5}{⼀、⾡}
  \begin{phonetics}{两边}{liang3 bian1}[][HSK 4]
    \definition{s.}{ambos os lados; ambas as direções; ambos os lugares | ambas as partes; ambos os lados}
  \end{phonetics}
\end{entry}

\begin{entry}{两岸}{7,8}{⼀、⼭}
  \begin{phonetics}{两岸}{liang3 an4}[][HSK 5]
    \definition{s.}{ambos os lados; ambas as margens; ambas as costas; entre os dois lados do estreito; bilateral}
  \end{phonetics}
\end{entry}

\begin{entry}{两码事}{7,8,8}{⼀、⽯、⼅}
  \begin{phonetics}{两码事}{liang3ma3shi4}
    \definition{expr.}{duas coisas completamente diferentes}
  \end{phonetics}
\end{entry}

\begin{entry}{严}{7}{⼀}
  \begin{phonetics}{严}{yan2}[][HSK 4]
    \definition*{s.}{sobrenome Yan}
    \definition{adj.}{rígido; rigoroso; estrito; severo}
    \definition{s.}{pai; refere-se ao pai}
  \end{phonetics}
\end{entry}

\begin{entry}{严厉}{7,5}{⼀、⼚}
  \begin{phonetics}{严厉}{yan2li4}[][HSK 5]
    \definition{adj.}{severo; rigoroso}
  \end{phonetics}
\end{entry}

\begin{entry}{严肃}{7,8}{⼀、⾀}
  \begin{phonetics}{严肃}{yan2su4}[][HSK 5]
    \definition{adj.}{sério; solene; sincero; (expressão, atmosfera, etc.) faz as pessoas se sentirem admiradas e desconfortáveis | sóbrio; grave; sério; sincero}
    \definition{v.}{aplicar rigorosamente; fazer algo sério}
  \end{phonetics}
\end{entry}

\begin{entry}{严重}{7,9}{⼀、⾥}
  \begin{phonetics}{严重}{yan2zhong4}[][HSK 4]
    \definition{adj.}{sério; grave; crítico; severo}
  \end{phonetics}
\end{entry}

\begin{entry}{严重打伤}{7,9,5,6}{⼀、⾥、⼿、⼈}
  \begin{phonetics}{严重打伤}{yan2zhong4 da3 shang1}
    \definition{s.}{gravemente ferido}
  \end{phonetics}
\end{entry}

\begin{entry}{严重伤害}{7,9,6,10}{⼀、⾥、⼈、⼧}
  \begin{phonetics}{严重伤害}{yan2zhong4 shang1hai4}
    \definition{s.}{ferimento grave}
  \end{phonetics}
\end{entry}

\begin{entry}{严重关切}{7,9,6,4}{⼀、⾥、⼋、⼑}
  \begin{phonetics}{严重关切}{yan2zhong4guan1qie4}
    \definition{s.}{preocupação séria}
  \end{phonetics}
\end{entry}

\begin{entry}{严重危害}{7,9,6,10}{⼀、⾥、⼙、⼧}
  \begin{phonetics}{严重危害}{yan2zhong4wei1hai4}
    \definition{s.}{danos graves}
  \end{phonetics}
\end{entry}

\begin{entry}{严重后果}{7,9,6,8}{⼀、⾥、⼝、⽊}
  \begin{phonetics}{严重后果}{yan2zhong4hou4guo3}
    \definition{s.}{consequências sérias | repercursões graves}
  \end{phonetics}
\end{entry}

\begin{entry}{严重地}{7,9,6}{⼀、⾥、⼟}
  \begin{phonetics}{严重地}{yan2zhong4 di4}
    \definition{adv.}{seriamente | gravemente}
  \end{phonetics}
\end{entry}

\begin{entry}{严重问题}{7,9,6,15}{⼀、⾥、⾨、⾴}
  \begin{phonetics}{严重问题}{yan2zhong4wen4ti2}
    \definition{s.}{problema sério}
  \end{phonetics}
\end{entry}

\begin{entry}{严重性}{7,9,8}{⼀、⾥、⼼}
  \begin{phonetics}{严重性}{yan2zhong4xing4}
    \definition{s.}{seriedade | gravidade}
  \end{phonetics}
\end{entry}

\begin{entry}{严重破坏}{7,9,10,7}{⼀、⾥、⽯、⼟}
  \begin{phonetics}{严重破坏}{yan2zhong4 po4huai4}
    \definition{s.}{destruição grave}
  \end{phonetics}
\end{entry}

\begin{entry}{严格}{7,10}{⼀、⽊}
  \begin{phonetics}{严格}{yan2ge2}[][HSK 4]
    \definition{adj.}{rígido; estrito; rigoroso; muito consciente e meticuloso na implementação de sistemas e no domínio de padrões}
    \definition{v.}{tornar (sistemas, provisões, etc.) rigorosos;}
  \end{phonetics}
\end{entry}

\begin{entry}{乱}{7}{⼄}
  \begin{phonetics}{乱}{luan4}[][HSK 3]
    \definition{adj.}{bagunçado; confuso; desordenado | turbulento; perturbado (estado de espírito) | arbitrário; aleatório}
    \definition{adv.}{em confusão ou desordem; em um estado de espírito confuso}
    \definition{s.}{caos; tumulto; agitação; turbilhão | comportamento sexual promíscuo; promiscuidade}
    \definition{v.}{confundir; embaralhar; misturar}
  \end{phonetics}
\end{entry}

\begin{entry}{亩}{7}{⼇}
  \begin{phonetics}{亩}{mu3}
    \definition{clas.}{usado para campos | unidade de área igual a um décimo quinto de um hectare}
  \end{phonetics}
\end{entry}

\begin{entry}{估计}{7,4}{⼈、⾔}
  \begin{phonetics}{估计}{gu1ji4}[][HSK 5]
    \definition{v.}{fazer contas; estimar; calcular; julgar a natureza, quantidade, mudança, etc. de uma coisa em uma determinada situação | parecer; parecer como se; aparentar; fazer inferências aproximadas sobre a natureza, a quantidade e a mudança das coisas com base em determinadas circunstâncias}
  \end{phonetics}
\end{entry}

\begin{entry}{伲}{7}{⼈}
  \begin{phonetics}{伲}{ni4}
    \definition{pron.}{(dialeto) eu | meu | nosso | nós}
  \seealsoref{你}{ni3}
  \end{phonetics}
\end{entry}

\begin{entry}{伴侣}{7,8}{⼈、⼈}
  \begin{phonetics}{伴侣}{ban4lv3}
    \definition{s.}{companheiro | parceiro}
  \end{phonetics}
\end{entry}

\begin{entry}{伸}{7}{⼈}
  \begin{phonetics}{伸}{shen1}[][HSK 5]
    \definition{v.}{alongar; esticar; estender}
  \end{phonetics}
\end{entry}

\begin{entry}{但}{7}{⼈}
  \begin{phonetics}{但}{dan4}[][HSK 2]
    \definition*{s.}{sobrenome Dan}
    \definition{adv.}{apenas; meramente; indica uma restrição ao âmbito da ação, equivalente a 只 ou 仅}
    \definition{conj.}{mas; ainda assim; mesmo assim; no entanto; contudo; usado na última oração, conecta duas orações, expressando uma relação de transição, equivalente a 可是 ou 不过}
  \seealsoref{不过}{bu2guo4}
  \seealsoref{仅}{jin3}
  \seealsoref{可是}{ke3shi4}
  \seealsoref{只}{zhi3}
  \end{phonetics}
\end{entry}

\begin{entry}{但是}{7,9}{⼈、⽇}
  \begin{phonetics}{但是}{dan4 shi4}[][HSK 2]
    \definition{conj.}{mas; contudo; no entanto; mesmo assim; usado na segunda parte da frase para indicar uma mudança, geralmente acompanhada de expressões como 虽然 ou 尽管}
  \seealsoref{尽管}{jin3guan3}
  \seealsoref{虽然}{sui1 ran2}
  \end{phonetics}
\end{entry}

\begin{entry}{位}{7}{⼈}
  \begin{phonetics}{位}{wei4}[][HSK 2]
    \definition*{s.}{sobrenome Wei}
    \definition{clas.}{usado para pessoas (com cortesia, respeito) | usado para bits binários}[十六位 (16 bits)]
    \definition{s.}{lugar; localização; o lugar onde ou onde alguém está localizado | posto; \emph{status}; posição; a posição de uma pessoa em uma determinada área da vida social | trono; refere-se especificamente ao status do imperador | lugar; dígito; a posição de cada dígito em um número}
  \end{phonetics}
\end{entry}

\begin{entry}{位于}{7,3}{⼈、⼆}
  \begin{phonetics}{位于}{wei4yu2}[][HSK 4]
    \definition{v.}{estar localizado; estar situado}
  \end{phonetics}
\end{entry}

\begin{entry}{位子}{7,3}{⼈、⼦}
  \begin{phonetics}{位子}{wei4zi5}
    \definition{s.}{lugar | assento}
  \end{phonetics}
\end{entry}

\begin{entry}{位居}{7,8}{⼈、⼫}
  \begin{phonetics}{位居}{wei4ju1}
    \definition{v.}{estar localizado em}
  \end{phonetics}
\end{entry}

\begin{entry}{位置}{7,13}{⼈、⽹}
  \begin{phonetics}{位置}{wei4zhi4}[][HSK 4]
    \definition[通,个]{s.}{assento; lugar; localização | lugar; posição; \emph{status} | posição (por exemplo: cargo no escritório)}
  \end{phonetics}
\end{entry}

\begin{entry}{低}{7}{⼈}
  \begin{phonetics}{低}{di1}[][HSK 2]
    \definition*{s.}{sobrenome Di}
    \definition{adj.}{baixo; distância pequena de baixo para cima; próximo ao solo | abaixo da média; abaixo do padrão geral | inferior (em grau); de nível inferior}
    \definition{v.}{deixar cair; pendurar; abaixar (a cabeça)}
  \end{phonetics}
\end{entry}

\begin{entry}{低于}{7,3}{⼈、⼆}
  \begin{phonetics}{低于}{di1 yu2}[][HSK 5]
    \definition{v.}{ser inferior a; algo ou fenômeno é, de alguma forma, inferior ou pior do que outra coisa}
  \end{phonetics}
\end{entry}

\begin{entry}{低潮}{7,15}{⼈、⽔}
  \begin{phonetics}{低潮}{di1chao2}
    \definition{s.}{maré baixa/vazante; o nível mais baixo da maré durante um ciclo de maré (distinto da 高潮) | vazante baixa; o ponto mais baixo; uma metáfora para o baixo estágio de desenvolvimento das coisas}
  \seealsoref{高潮}{gao1chao2}
  \end{phonetics}
\end{entry}

\begin{entry}{住}{7}{⼈}
  \begin{phonetics}{住}{zhu4}[][HSK 1]
    \definition{adv.}{firmemente; indica estabilidade ou firmeza}
    \definition{v.}{viver; residir; morar; ficar | parar; cessar | (após um verbo) com firmeza; até parar | hospedar; acomodar | parar; interromper | ser competente; ser qualificado; estar à altura; usado com 得 ou 不, indica que a força é suficiente (ou insuficiente)}
  \seealsoref{不}{bu4}
  \seealsoref{得}{de5}
  \end{phonetics}
\end{entry}

\begin{entry}{住处}{7,5}{⼈、⼡}
  \begin{phonetics}{住处}{zhu4chu4}
    \definition{s.}{morada | habitação | residência}
  \end{phonetics}
\end{entry}

\begin{entry}{住宅}{7,6}{⼈、⼧}
  \begin{phonetics}{住宅}{zhu4zhai2}
    \definition{s.}{residência}
  \end{phonetics}
\end{entry}

\begin{entry}{住房}{7,8}{⼈、⼾}
  \begin{phonetics}{住房}{zhu4fang2}[][HSK 2]
    \definition{s.}{habitação}
  \end{phonetics}
\end{entry}

\begin{entry}{住所}{7,8}{⼈、⼾}
  \begin{phonetics}{住所}{zhu4suo3}
    \definition[处]{s.}{morada | habitação | residência}
  \end{phonetics}
\end{entry}

\begin{entry}{住院}{7,9}{⼈、⾩}
  \begin{phonetics}{住院}{zhu4 yuan4}[][HSK 2]
    \definition{v.}{estar hospitalizado | estar no hospital}
  \end{phonetics}
\end{entry}

\begin{entry}{住嘴}{7,16}{⼈、⼝}
  \begin{phonetics}{住嘴}{zhu4zui3}
    \definition{interj.}{Cale-se!}
    \definition{v.}{calar | calar-se}
  \end{phonetics}
\end{entry}

\begin{entry}{体力}{7,2}{⼈、⼒}
  \begin{phonetics}{体力}{ti3 li4}[][HSK 5]
    \definition{s.}{força física; vigor físico (ou corporal); a força do corpo humano para sustentar suas próprias atividades}
  \end{phonetics}
\end{entry}

\begin{entry}{体内}{7,4}{⼈、⼌}
  \begin{phonetics}{体内}{ti3nei4}
    \definition{adj.}{dentro do corpo | \emph{in vivo} (versus \emph{in vitro} | interno a}
  \end{phonetics}
\end{entry}

\begin{entry}{体会}{7,6}{⼈、⼈}
  \begin{phonetics}{体会}{ti3hui4}[][HSK 3]
    \definition{s.}{conhecimento; compreensão; experiência pessoal}
    \definition{v.}{perceber; saber (ou aprender) com a experiência}
  \end{phonetics}
\end{entry}

\begin{entry}{体现}{7,8}{⼈、⾒}
  \begin{phonetics}{体现}{ti3xian4}[][HSK 3]
    \definition{v.}{refletir; incorporar; encarnar}
  \end{phonetics}
\end{entry}

\begin{entry}{体育}{7,8}{⼈、⾁}
  \begin{phonetics}{体育}{ti3yu4}[][HSK 2]
    \definition{s.}{cultura física; treinamento físico; educação cuja principal tarefa é desenvolver a capacidade física e fortalecer a constituição física, alcançada através da participação em várias atividades esportivas | esportes; atividades esportivas; refere-se a esportes}
  \end{phonetics}
\end{entry}

\begin{entry}{体育场}{7,8,6}{⼈、⾁、⼟}
  \begin{phonetics}{体育场}{ti3 yu4 chang3}[][HSK 2]
    \definition[个,座]{s.}{estádio; campo esportivo; espaço ao ar livre para a prática de exercícios físicos ou competições esportivas}
  \end{phonetics}
\end{entry}

\begin{entry}{体育馆}{7,8,11}{⼈、⾁、⾷}
  \begin{phonetics}{体育馆}{ti3 yu4 guan3}[][HSK 2]
    \definition[个]{s.}{ginásio | estádio}
  \end{phonetics}
\end{entry}

\begin{entry}{体重}{7,9}{⼈、⾥}
  \begin{phonetics}{体重}{ti3 zhong4}[][HSK 4]
    \definition{s.}{peso corporal}
  \end{phonetics}
\end{entry}

\begin{entry}{体积}{7,10}{⼈、⽲}
  \begin{phonetics}{体积}{ti3ji1}[][HSK 5]
    \definition[个]{s.}{volume; quantidade; o tamanho do espaço ocupado pelo objeto}
  \end{phonetics}
\end{entry}

\begin{entry}{体验}{7,10}{⼈、⾺}
  \begin{phonetics}{体验}{ti3yan4}[][HSK 3]
    \definition[种]{s.}{experiência}
    \definition{v.}{aprender através da prática; aprender através da experiência pessoal}
  \end{phonetics}
\end{entry}

\begin{entry}{体检}{7,11}{⼈、⽊}
  \begin{phonetics}{体检}{ti3 jian3}[][HSK 4]
    \definition{s.}{exame clínico}
    \definition{v.}{fazer um exame médico}
  \end{phonetics}
\end{entry}

\begin{entry}{体操}{7,16}{⼈、⼿}
  \begin{phonetics}{体操}{ti3 cao1}[][HSK 4]
    \definition{s.}{ginástica; esportes, exercícios ou performances de vários movimentos, sem armas ou com o auxílio de determinados equipamentos}
  \end{phonetics}
\end{entry}

\begin{entry}{何}{7}{⼈}
  \begin{phonetics}{何}{he2}
    \definition*{s.}{sobrenome He}
    \definition{adv.}{expressa exclamação, equivalente a 多么}
    \definition{pron.}{O que?; Onde?; Por que? | expressa uma pergunta retórica, equivalente a 岂, 怎}
  \seealsoref{多么}{duo1me5}
  \seealsoref{岂}{qi3}
  \seealsoref{怎}{zen3}
  \end{phonetics}
\end{entry}

\begin{entry}{何不}{7,4}{⼈、⼀}
  \begin{phonetics}{何不}{he2bu4}
    \definition{adv.}{por que não?; use o tom interrogativo para expressar "deveria" ou "pode"}
  \end{phonetics}
\end{entry}

\begin{entry}{何况}{7,7}{⼈、⼎}
  \begin{phonetics}{何况}{he2kuang4}
    \definition{conj.}{além disso | muito menos}
  \end{phonetics}
\end{entry}

\begin{entry}{佛}{7}{⼈}
  \begin{phonetics}{佛}{fo2}
    \definition*{s.}{Buda, abreviação de 佛陀 | Budismo}
  \seealsoref{佛陀}{fo2tuo2}
  \end{phonetics}
  \begin{phonetics}{佛}{fu2}
    \definition{adv.}{aparentemente}
    \definition{s.}{ornamento da cabeça (feminino)}
  \end{phonetics}
\end{entry}

\begin{entry}{佛陀}{7,7}{⼈、⾩}
  \begin{phonetics}{佛陀}{fo2tuo2}
    \definition{s.}{Buda (uma pessoa que atingiu a Budeidade, ou especificamente Siddhartha Gautama)}
  \end{phonetics}
\end{entry}

\begin{entry}{作}{7}{⼈}
  \begin{phonetics}{作}{zuo1}
    \definition{adj.}{(gíria) incômodo}
    \definition{s.}{trabalhador | oficina | (pessoa) de alta manutenção}
  \end{phonetics}
  \begin{phonetics}{作}{zuo4}
    \definition{s.}{escritos ou obras}
    \definition{v.}{fazer | crescer | escrever ou compor | fingir | considerar como | sentir}
  \end{phonetics}
\end{entry}

\begin{entry}{作为}{7,4}{⼈、⼂}
  \begin{phonetics}{作为}{zuo4wei2}[][HSK 4]
    \definition{prep.}{como; na capacidade de; no caráter de; no papel de; em termos de uma certa identidade de uma pessoa ou de uma certa natureza de uma coisa}
    \definition{s.}{ato; ação; conduta; feito; comportamento | conquista; realização; especificamente, uma boa ação}
    \definition{v.}{considerar como; tomar por; olhar como; tratar como | realizar; fazer conquistas; deixar uma marca}
  \end{phonetics}
\end{entry}

\begin{entry}{作文}{7,4}{⼈、⽂}
  \begin{phonetics}{作文}{zuo4wen2}[][HSK 2]
    \definition[篇]{s.}{ensaio |  composição | redação}
    \definition{v.+compl.}{(de alunos) para escrever uma redação}
  \end{phonetics}
\end{entry}

\begin{entry}{作业}{7,5}{⼈、⼀}
  \begin{phonetics}{作业}{zuo4ye4}[][HSK 2]
    \definition[份,个]{s.}{tarefa escolar | trabalho | tarefa | operação}
  \end{phonetics}
\end{entry}

\begin{entry}{作出}{7,5}{⼈、⼐}
  \begin{phonetics}{作出}{zuo4 chu1}[][HSK 4]
    \definition{v.}{mostrar; tomar (decisões, conclusões, etc. por meio de consideração ou discussão); formar (uma conclusão, decisão, etc.) por meio de consideração ou discussão}
  \end{phonetics}
\end{entry}

\begin{entry}{作用}{7,5}{⼈、⽤}
  \begin{phonetics}{作用}{zuo4yong4}[][HSK 2]
    \definition{s.}{efeito | ação | função}
    \definition{v.}{afetar | agir em}
  \end{phonetics}
\end{entry}

\begin{entry}{作者}{7,8}{⼈、⽼}
  \begin{phonetics}{作者}{zuo4zhe3}[][HSK 3]
    \definition[位,名,个]{s.}{autor; escritor; uma pessoa que escreve artigos ou cria obras de arte}
  \end{phonetics}
\end{entry}

\begin{entry}{作品}{7,9}{⼈、⼝}
  \begin{phonetics}{作品}{zuo4pin3}[][HSK 3]
    \definition[个,部,篇,幅]{s.}{obra de arte; obras concluídas de literatura e arte}
  \end{phonetics}
\end{entry}

\begin{entry}{作家}{7,10}{⼈、⼧}
  \begin{phonetics}{作家}{zuo4jia1}[][HSK 2]
    \definition[位,个]{s.}{autor | escritor}
  \end{phonetics}
\end{entry}

\begin{entry}{你}{7}{⼈}
  \begin{phonetics}{你}{ni3}[][HSK 1]
    \definition{pron.}{você (segunda pessoa do singular); refere-se à pessoa com quem se está conversando | (referindo-se a qualquer pessoa) você; um; qualquer um | com 我 ou 你 em estruturas paralelas para indicar várias ou muitas pessoas se comportando da mesma maneira}
  \seealsoref{您}{nin2}
  \seealsoref{我}{wo3}
  \end{phonetics}
\end{entry}

\begin{entry}{你们}{7,5}{⼈、⼈}
  \begin{phonetics}{你们}{ni3men5}[][HSK 1]
    \definition{pron.}{você (segunda pessoa do plural); refere-se a mais de uma pessoa ou a várias pessoas, incluindo a outra parte}
  \end{phonetics}
\end{entry}

\begin{entry}{你们的}{7,5,8}{⼈、⼈、⽩}
  \begin{phonetics}{你们的}{ni3men5 de5}
    \definition{pron.}{vossos}
  \end{phonetics}
\end{entry}

\begin{entry}{你好}{7,6}{⼈、⼥}
  \begin{phonetics}{你好}{ni3hao3}
    \definition{interj.}{Olá! | Oi!}
  \end{phonetics}
\end{entry}

\begin{entry}{你的}{7,8}{⼈、⽩}
  \begin{phonetics}{你的}{ni3 de5}
    \definition{pron.}{seu}
  \end{phonetics}
\end{entry}

\begin{entry}{克}{7}{⼗}
  \begin{phonetics}{克}{ke4}[][HSK 2]
    \definition*{s.}{sobrenome Ke}
    \definition{clas.}{grama (g) | unidade tibetana de volume ou medida seca (com capacidade para cerca de 25 jin,  斤, de cevada) | unidade tibetana de área de terra equivalente a cerca de 1 mu, 亩}
    \definition{v.}{poder; ser capaz de | tolerar; conter; restringir; suprimir| subjugar; capturar; conquistar (uma cidade, etc.) | digerir (alimentos) | reduzir; diminuir | definir um limite de tempo}
  \seealsoref{斤}{jin1}
  \seealsoref{亩}{mu3}
  \end{phonetics}
\end{entry}

\begin{entry}{克服}{7,8}{⼗、⽉}
  \begin{phonetics}{克服}{ke4fu2}[][HSK 3]
    \definition{v.}{sobrepujar; superar; conquistar | suportar (dificuldades, inconveniências, etc.)}
  \end{phonetics}
\end{entry}

\begin{entry}{免费}{7,9}{⼉、⾙}
  \begin{phonetics}{免费}{mian3fei4}[][HSK 4]
    \definition{s.}{gratuito; sem custo}
    \definition{v.+compl.}{isentar de taxas; tonar grátis}
  \end{phonetics}
\end{entry}

\begin{entry}{免得}{7,11}{⼉、⼻}
  \begin{phonetics}{免得}{mian3de5}
    \definition{conj.}{de modo a não | para evitar | para que não}
  \end{phonetics}
\end{entry}

\begin{entry}{免税}{7,12}{⼉、⽲}
  \begin{phonetics}{免税}{mian3shui4}
    \definition{adj.}{isento de impostos (tributação)}
    \definition{s.}{livre de impostos | isenção de impostos}
    \definition{v.+compl.}{isentar impostos}
  \end{phonetics}
\end{entry}

\begin{entry}{兵}{7}{⼋}
  \begin{phonetics}{兵}{bing1}[][HSK 4]
    \definition[名]{s.}{armas; armamentos | soldado; pessoal militar | exército; tropas | soldado raso; membro mais jovem do exército | assuntos militares (estratégia) | peão, uma das peças do xadrez chinês}
  \end{phonetics}
\end{entry}

\begin{entry}{兵器}{7,16}{⼋、⼝}
  \begin{phonetics}{兵器}{bing1qi4}
    \definition{s.}{armas | armamento}
  \end{phonetics}
\end{entry}

\begin{entry}{况且}{7,5}{⼎、⼀}
  \begin{phonetics}{况且}{kuang4qie3}
    \definition{conj.}{além disso | além do mais}
  \end{phonetics}
\end{entry}

\begin{entry}{冷}{7}{⼎}
  \begin{phonetics}{冷}{leng3}[][HSK 1]
    \definition*{s.}{sobrenome Leng}
    \definition{adj.}{frio; baixa temperatura; sensação de frio | gelado; frio por natureza; sem entusiasmo; sem gentileza | desolado; pouco frequentado; quieto; sem agitação | negligenciado; indesejável; ignorado por todos | raro; estranho; incomum | feito em segredo; filmado de forma escondida; lançado secretamente}
    \definition{v.}{esfriar; resfriar | esfriar; congelar; tornar-se indiferente, apático | ignorar}
  \end{phonetics}
\end{entry}

\begin{entry}{冷门}{7,3}{⼎、⾨}
  \begin{phonetics}{冷门}{leng3men2}
    \definition{s.}{uma profissão, ofício ou ramo de aprendizagem que recebe pouca atenção | um vencedor inesperado; azarão}
  \end{phonetics}
\end{entry}

\begin{entry}{冷静}{7,14}{⼎、⾭}
  \begin{phonetics}{冷静}{leng3jing4}[][HSK 4]
    \definition{adj.}{calmo; descreve uma pessoa que consegue ficar atenta em uma situação importante ou de emergência e não toma decisões aleatórias por causa de seus sentimentos no momento | (lugar) tranquilo; quieto; deserto}
  \end{phonetics}
\end{entry}

\begin{entry}{冻}{7}{⼎}
  \begin{phonetics}{冻}{dong4}[][HSK 5]
    \definition*{s.}{sobrenome Dong}
    \definition{s.}{geleia; gelatina;}
    \definition{v.}{congelar; ser congelado | ficar com frio ou sentir frio}
  \end{phonetics}
\end{entry}

\begin{entry}{初}{7}{⾐}
  \begin{phonetics}{初}{chu1}[][HSK 3]
    \definition*{s.}{sobrenome Chu}
    \definition{adj.}{primeiro (em ordem) | elementar; rudimentar | original}
    \definition{adv.}{pela primeira vez}
    \definition{pref.}{anexado aos numerais de um a dez para indicar ordem (primeiro ao décimo)}
    \definition{s.}{no início de; na primeira parte de | o estágio júnior (pleno; sênior)}
  \end{phonetics}
\end{entry}

\begin{entry}{初中}{7,4}{⾐、⼁}
  \begin{phonetics}{初中}{chu1 zhong1}[][HSK 3]
    \definition[所,个]{s.}{ensino médio; ensino fundamental}
  \end{phonetics}
\end{entry}

\begin{entry}{初心}{7,4}{⾐、⼼}
  \begin{phonetics}{初心}{chu1xin1}
    \definition{s.}{intenção original de alguém, aspiração, etc. | (budismo) ``mente do iniciante'' (ter a mente aberta quando estudando um assunto como um iniciante no assunto teria)}
  \end{phonetics}
\end{entry}

\begin{entry}{初级}{7,6}{⾐、⽷}
  \begin{phonetics}{初级}{chu1ji2}[][HSK 3]
    \definition{adj.}{elementar; primário; júnior; inicial}
  \end{phonetics}
\end{entry}

\begin{entry}{初步}{7,7}{⾐、⽌}
  \begin{phonetics}{初步}{chu1bu4}[][HSK 3]
    \definition{adj.}{inicial; preliminar}
  \end{phonetics}
\end{entry}

\begin{entry}{初期}{7,12}{⾐、⽉}
  \begin{phonetics}{初期}{chu1 qi1}[][HSK 5]
    \definition{s.}{primórdio; estágio inicial; primeiros dias; estágio preliminar; período inicial}
  \end{phonetics}
\end{entry}

\begin{entry}{判断}{7,11}{⼑、⽄}
  \begin{phonetics}{判断}{pan4duan4}[][HSK 3]
    \definition[个]{s.}{julgamento}
    \definition{v.}{julgar; decidir}
  \end{phonetics}
\end{entry}

\begin{entry}{利}{7}{⼑}
  \begin{phonetics}{利}{li4}
    \definition*{s.}{sobrenome Li}
    \definition{adj.}{afiado; cortante | favorável; conveniente; sem dificuldades; sem ou com poucas dificuldades}
    \definition{s.}{benefício; vantagem | lucro; ganhos; juros}
    \definition{v.}{beneficiar; tornar vantajoso}
  \end{phonetics}
\end{entry}

\begin{entry}{利用}{7,5}{⼑、⽤}
  \begin{phonetics}{利用}{li4yong4}[][HSK 3]
    \definition{v.}{usar; utilizar; fazer uso de | explorar; tirar vantagem de}
  \end{phonetics}
\end{entry}

\begin{entry}{利息}{7,10}{⼑、⼼}
  \begin{phonetics}{利息}{li4xi1}[][HSK 4]
    \definition{s.}{acréscimo; juros; dinheiro recebido além do valor principal como resultado de depósitos ou empréstimos (diferenciado de 本金)}
  \seealsoref{本金}{ben3 jin1}
  \end{phonetics}
\end{entry}

\begin{entry}{利润}{7,10}{⼑、⽔}
  \begin{phonetics}{利润}{li4run4}[][HSK 5]
    \definition[笔]{s.}{lucro; o dinheiro ganho com atividades comerciais e industriais}
  \end{phonetics}
\end{entry}

\begin{entry}{利益}{7,10}{⼑、⽫}
  \begin{phonetics}{利益}{li4yi4}[][HSK 4]
    \definition[个,种]{s.}{ganho; lucro; juros; benefício}
  \end{phonetics}
\end{entry}

\begin{entry}{别}{7}{⼑}
  \begin{phonetics}{别}{bie2}[][HSK 1,4]
    \definition*{s.}{sobrenome Bie}
    \definition{adv.}{não; nada de (pedir a alguém para não fazer); é melhor não | talvez, usado em conjunto com a palavra 是 para indicar especulação.}
    \definition{pron.}{outro; algum outro}
    \definition{s.}{distinção; diferença | classificação}
    \definition{v.}{deixar; partir; separar | diferenciar; distinguir; encontrar aspectos diferentes | fixar objetos com pinos | girar; virar | aderir; colar; preder}
  \seealsoref{是}{shi4}
  \end{phonetics}
  \begin{phonetics}{别}{bie4}
    \definition{v.}{fazer com que alguém mude seus hábitos, opiniões, etc.}
  \end{phonetics}
\end{entry}

\begin{entry}{别人}{7,2}{⼑、⼈}
  \begin{phonetics}{别人}{bie2 ren2}[][HSK 1]
    \definition{pron.}{outros; outras pessoas}
    \definition{s.}{outros; pessoas; outras pessoas; refere-se a alguém diferente de si mesmo}
  \end{phonetics}
\end{entry}

\begin{entry}{别的}{7,8}{⼑、⽩}
  \begin{phonetics}{别的}{bie2 de5}[][HSK 1]
    \definition{pron.}{outro; o resto}
  \end{phonetics}
\end{entry}

\begin{entry}{别说}{7,9}{⼑、⾔}
  \begin{phonetics}{别说}{bie2shuo1}
    \definition{v.}{não falar de | não mencionar}
  \end{phonetics}
\end{entry}

\begin{entry}{助手}{7,4}{⼒、⼿}
  \begin{phonetics}{助手}{zhu4shou3}[][HSK 5]
    \definition[个]{s.}{ajudante; auxiliar; assistente; alguém que ajuda os outros com seu trabalho}
  \end{phonetics}
\end{entry}

\begin{entry}{助兴}{7,6}{⼒、⼋}
  \begin{phonetics}{助兴}{zhu4xing4}
    \definition{v.+compl.}{animar as coisas | juntar-se à diversão}
  \end{phonetics}
\end{entry}

\begin{entry}{助理}{7,11}{⼒、⽟}
  \begin{phonetics}{助理}{zhu4li3}[][HSK 5]
    \definition[个,名,位]{s.}{deputado; assistente; auxiliar do diretor responsável (geralmente usado em cargos) | ajudante; assistente; pessoa que auxilia o responsável a fazer as coisas}
  \end{phonetics}
\end{entry}

\begin{entry}{努力}{7,2}{⼒、⼒}
  \begin{phonetics}{努力}{nu3li4}[][HSK 2]
    \definition{adj.}{extenuante; árduo | diligente; trabalhador; quem faz as coisas com o máximo de capacidade ou esforço possível}
    \definition{s.}{esforço; tentativa; fazer o melhor possível}
    \definition{v.}{fazer grandes esforços; esforçar-se; empenhar-se | esforçar-se; usar toda a força possível}
  \end{phonetics}
\end{entry}

\begin{entry}{劳工同事}{7,3,6,8}{⼒、⼯、⼝、⼅}
  \begin{phonetics}{劳工同事}{lao2gong1 tong2shi4}
    \definition{s.}{colaborador | colega de trabalho}
  \end{phonetics}
\end{entry}

\begin{entry}{劳动}{7,6}{⼒、⼒}
  \begin{phonetics}{劳动}{lao2dong4}[][HSK 5]
    \definition[次]{s.}{trabalho; mão de obra; atividades intelectuais ou físicas que podem criar valor | trabalho físico; trabalho manual; referindo-se especificamente ao trabalho físico}
    \definition{v.}{realizar trabalho físico}
  \end{phonetics}
\end{entry}

\begin{entry}{医}{7}{⼖}
  \begin{phonetics}{医}{yi1}
    \definition{s.}{médico | medicina}
    \definition{v.}{curar | tratar}
  \end{phonetics}
\end{entry}

\begin{entry}{医生}{7,5}{⼖、⽣}
  \begin{phonetics}{医生}{yi1sheng1}[][HSK 1]
    \definition[位,个,名]{s.}{médico; clínico; pessoa que possui conhecimentos médicos e cuja profissão é tratar doenças}
  \end{phonetics}
\end{entry}

\begin{entry}{医疗}{7,7}{⼖、⽧}
  \begin{phonetics}{医疗}{yi1 liao2}[][HSK 4]
    \definition{s.}{tratamento médico; tratamento de doenças}
  \end{phonetics}
\end{entry}

\begin{entry}{医学}{7,8}{⼖、⼦}
  \begin{phonetics}{医学}{yi1 xue2}[][HSK 4]
    \definition{s.}{medicina; iatrologia; ciência médica; ciência da prevenção e do tratamento de doenças e da proteção e promoção da saúde humana}
  \end{phonetics}
\end{entry}

\begin{entry}{医院}{7,9}{⼖、⾩}
  \begin{phonetics}{医院}{yi1yuan4}[][HSK 1]
    \definition[家,所,个]{s.}{hospital; instituições que tratam e cuidam de pacientes, e também realizam exames de saúde, prevenção de doenças, etc.}
  \end{phonetics}
\end{entry}

\begin{entry}{即}{7}{⼙}
  \begin{phonetics}{即}{ji2}
    \definition{conj.}{e | até | mesmo se/embora}
  \end{phonetics}
\end{entry}

\begin{entry}{即使}{7,8}{⼙、⼈}
  \begin{phonetics}{即使}{ji2shi3}[][HSK 5]
    \definition{conj.}{mesmo; mesmo que; mesmo se; apesar de; expressando uma concessão hipotética}
  \end{phonetics}
\end{entry}

\begin{entry}{即或}{7,8}{⼙、⼽}
  \begin{phonetics}{即或}{ji2huo4}
    \definition{conj.}{mesmo se/embora}
  \end{phonetics}
\end{entry}

\begin{entry}{即若}{7,8}{⼙、⾋}
  \begin{phonetics}{即若}{ji2ruo4}
    \definition{conj.}{mesmo se/embora}
  \end{phonetics}
\end{entry}

\begin{entry}{即便}{7,9}{⼙、⼈}
  \begin{phonetics}{即便}{ji2bian4}
    \definition{conj.}{mesmo se/embora}
  \end{phonetics}
\end{entry}

\begin{entry}{即将}{7,9}{⼙、⼨}
  \begin{phonetics}{即将}{ji2jiang1}[][HSK 4]
    \definition{adv.}{em breve; estar prestes a; estar a ponto de}
  \end{phonetics}
\end{entry}

\begin{entry}{即是}{7,9}{⼙、⽇}
  \begin{phonetics}{即是}{ji2shi4}
    \definition{conj.}{aquilo é}
  \end{phonetics}
\end{entry}

\begin{entry}{却}{7}{⼙}
  \begin{phonetics}{却}{que4}[][HSK 4]
    \definition{adv.}{mas; contudo; no entanto; enquanto; indica um ponto de virada}
    \definition{v.}{recuar; retroceder | afastar; repelir; desencorajar | declinar; recusar; rejeitar}
    \definition{v.aux.}{usado depois de certos verbos para indicar a conclusão de uma ação}
  \end{phonetics}
\end{entry}

\begin{entry}{却是}{7,9}{⼙、⽇}
  \begin{phonetics}{却是}{que4shi4}
    \definition{conj.}{no entanto | realmente | o fato é\dots | mas isso é\dots}
  \end{phonetics}
\end{entry}

\begin{entry}{县}{7}{⼛}
  \begin{phonetics}{县}{xian4}[][HSK 4]
    \definition[个]{s.}{condado; unidade de divisão administrativa}
  \end{phonetics}
\end{entry}

\begin{entry}{君主立宪制}{7,5,5,9,8}{⼝、⼂、⽴、⼧、⼑}
  \begin{phonetics}{君主立宪制}{jun1zhu3li4xian4zhi4}
    \definition{s.}{monarquia constitucional}
  \end{phonetics}
\end{entry}

\begin{entry}{吟诗}{7,8}{⼝、⾔}
  \begin{phonetics}{吟诗}{yin2shi1}
    \definition{v.}{recitar poesia}
  \end{phonetics}
\end{entry}

\begin{entry}{否认}{7,4}{⼝、⾔}
  \begin{phonetics}{否认}{fou3ren4}[][HSK 3]
    \definition{v.}{negar; repudiar}
  \end{phonetics}
\end{entry}

\begin{entry}{否则}{7,6}{⼝、⼑}
  \begin{phonetics}{否则}{fou3ze2}[][HSK 4]
    \definition{conj.}{senão; se não; ou então; se não for isso}
  \end{phonetics}
\end{entry}

\begin{entry}{否定}{7,8}{⼝、⼧}
  \begin{phonetics}{否定}{fou3ding4}[][HSK 3]
    \definition{adj.}{negativo}
    \definition{s.}{negativo (resposta); negação}
    \definition{v.}{rejeitar; negar}
  \end{phonetics}
\end{entry}

\begin{entry}{吧}{7}{⼝}
  \begin{phonetics}{吧}{ba1}
    \definition{s.}{som de estalo, som crepitante}
    \definition{v.}{puxar o cachimbo; fumar | abreviação de ``bar''}
  \end{phonetics}
  \begin{phonetics}{吧}{ba5}[][HSK 1]
    \definition{part.}{indica discussão, sugestão, solicitação ou comando no final de uma frase | indica concordância ou aprovação no final de uma frase | indica uma pergunta ou especulação no final de uma frase | indica incerteza no final de uma frase | em uma frase, indica uma pausa, carrega um tom hipotético, frequentemente apresenta um contraste e implica um dilema}
  \end{phonetics}
\end{entry}

\begin{entry}{吨}{7}{⼝}
  \begin{phonetics}{吨}{dun1}[][HSK 5]
    \definition{clas.}{tonelada}
  \end{phonetics}
\end{entry}

\begin{entry}{含}{7}{⼝}
  \begin{phonetics}{含}{han2}[][HSK 4]
    \definition{v.}{manter na boca (sem engolir ou cuspir) | conter; incluir | cuidar; acalentar; abrigar}
  \end{phonetics}
\end{entry}

\begin{entry}{含义}{7,3}{⼝、⼂}
  \begin{phonetics}{含义}{han2yi4}[][HSK 4]
    \definition[个,种,层]{s.}{sentido; mensagem; significado; implicação}
  \end{phonetics}
\end{entry}

\begin{entry}{含有}{7,6}{⼝、⽉}
  \begin{phonetics}{含有}{han2 you3}[][HSK 4]
    \definition{v.}{conter; ter; incluir}
  \end{phonetics}
\end{entry}

\begin{entry}{含金量}{7,8,12}{⼝、⾦、⾥}
  \begin{phonetics}{含金量}{han2jin1liang4}
    \definition{adj.}{conteúdo de ouro | (fig.) valioso}
  \end{phonetics}
\end{entry}

\begin{entry}{含量}{7,12}{⼝、⾥}
  \begin{phonetics}{含量}{han2 liang4}[][HSK 4]
    \definition{s.}{conteúdo; a quantidade de um componente contido em uma substância}
  \end{phonetics}
\end{entry}

\begin{entry}{听}{7}{⼝}
  \begin{phonetics}{听}{ting1}[][HSK 1]
    \definition{clas.}{latas; usado para bebidas e alimentos para levar consigo}
    \definition{s.}{lata; embalagem metálica; recipiente cilíndrico utilizado para armazenar bebidas, alimentos, etc.}
    \definition{v.}{ouvir; escutar | obedecer; dar ouvidos; aceitar | supervisionar; administrar; tratar (assuntos políticos); julgar (casos) | permitir; deixar ser; deixar fazer}
  \end{phonetics}
  \begin{phonetics}{听}{yin3}
    \definition[个]{s.}{lata; embalagem metálica}
  \end{phonetics}
\end{entry}

\begin{entry}{听力}{7,2}{⼝、⼒}
  \begin{phonetics}{听力}{ting1li4}[][HSK 3]
    \definition{s.}{audição; capacidade auditiva | compreensão auditiva (na aprendizagem de línguas)}
  \end{phonetics}
\end{entry}

\begin{entry}{听力理解}{7,2,11,13}{⼝、⼒、⽟、⾓}
  \begin{phonetics}{听力理解}{ting1li4li3jie3}
    \definition{s.}{compreensão auditiva}
  \end{phonetics}
\end{entry}

\begin{entry}{听小骨}{7,3,9}{⼝、⼩、⾻}
  \begin{phonetics}{听小骨}{ting1xiao3gu3}
    \definition{s.}{ossículos (do ouvido médio)}
  \seealsoref{听骨}{ting1gu3}
  \end{phonetics}
\end{entry}

\begin{entry}{听见}{7,4}{⼝、⾒}
  \begin{phonetics}{听见}{ting1 jian4}[][HSK 1]
    \definition{v.}{ouvir; conseguir ouvir}
  \end{phonetics}
\end{entry}

\begin{entry}{听写}{7,5}{⼝、⼍}
  \begin{phonetics}{听写}{ting1 xie3}[][HSK 1]
    \definition{s.}{ditado}
    \definition{v.}{ouvir e escrever}
  \end{phonetics}
\end{entry}

\begin{entry}{听众}{7,6}{⼝、⼈}
  \begin{phonetics}{听众}{ting1 zhong4}[][HSK 3]
    \definition{s.}{audiência; ouvintes}
  \end{phonetics}
\end{entry}

\begin{entry}{听会}{7,6}{⼝、⼈}
  \begin{phonetics}{听会}{ting1hui4}
    \definition{v.}{participar de uma reunião (e ouvir o que é discutido)}
  \end{phonetics}
\end{entry}

\begin{entry}{听戏}{7,6}{⼝、⼽}
  \begin{phonetics}{听戏}{ting1xi4}
    \definition{v.}{assistir a uma ópera | ver uma ópera}
  \end{phonetics}
\end{entry}

\begin{entry}{听讲}{7,6}{⼝、⾔}
  \begin{phonetics}{听讲}{ting1 jiang3}[][HSK 2]
    \definition{v.+compl.}{assistir a uma palestra; ouvir palestras ou discursos}
  \end{phonetics}
\end{entry}

\begin{entry}{听来}{7,7}{⼝、⽊}
  \begin{phonetics}{听来}{ting1lai2}
    \definition{v.}{ouvir de algum lugar | soar (antigo, estrangeiro, excitante, certo, etc.) | soar como se (ou seja, dar uma impressão ao ouvinte)}
  \end{phonetics}
\end{entry}

\begin{entry}{听凭}{7,8}{⼝、⼏}
  \begin{phonetics}{听凭}{ting1ping2}
    \definition{v.}{permitir (alguém a fazer o que desejar)}
  \end{phonetics}
\end{entry}

\begin{entry}{听到}{7,8}{⼝、⼑}
  \begin{phonetics}{听到}{ting1 dao4}[][HSK 1]
    \definition{v.}{ouvir, escutar; ouvir atentamente, escutar atentamente}
  \end{phonetics}
\end{entry}

\begin{entry}{听命}{7,8}{⼝、⼝}
  \begin{phonetics}{听命}{ting1ming4}
    \definition{v.}{obedecer ordens | receber ordens}
  \end{phonetics}
\end{entry}

\begin{entry}{听说}{7,9}{⼝、⾔}
  \begin{phonetics}{听说}{ting1 shuo1}[][HSK 2]
    \definition{v.}{ser informado; ouvir falar de; ouvir dizer | ouvir e falar}
  \end{phonetics}
\end{entry}

\begin{entry}{听骨}{7,9}{⼝、⾻}
  \begin{phonetics}{听骨}{ting1gu3}
    \definition{s.}{ossículos (do ouvido médio)}
  \seealsoref{听小骨}{ting1xiao3gu3}
  \end{phonetics}
\end{entry}

\begin{entry}{听断}{7,11}{⼝、⽄}
  \begin{phonetics}{听断}{ting1duan4}
    \definition{v.}{ouvir e decidir | julgar (ou seja, ouvir e julgar em um tribunal)}
  \end{phonetics}
\end{entry}

\begin{entry}{听随}{7,11}{⼝、⾩}
  \begin{phonetics}{听随}{ting1sui2}
    \definition{v.}{permitir | obedecer}
  \end{phonetics}
\end{entry}

\begin{entry}{启发}{7,5}{⼝、⼜}
  \begin{phonetics}{启发}{qi3fa1}[][HSK 5]
    \definition{s.}{iluminação; esclarecimento; fenômenos e princípios que levam as pessoas a refletir e a abrir suas mentes}
    \definition{v.}{despertar; inspirar; esclarecer; orientar, fazer com que compreendam}
  \end{phonetics}
\end{entry}

\begin{entry}{启动}{7,6}{⼝、⼒}
  \begin{phonetics}{启动}{qi3 dong4}[][HSK 5]
    \definition{v.}{ligar (uma máquina); acionar; ligar máquinas, equipamentos elétricos, etc., para começar a trabalhar | entrar em vigor; começar a vigorar e a ser implementados planos, projetos, documentos jurídicos, etc.}
  \end{phonetics}
\end{entry}

\begin{entry}{启事}{7,8}{⼝、⼅}
  \begin{phonetics}{启事}{qi3shi4}[][HSK 5]
    \definition{s.}{aviso; anúncio; texto publicado em jornais ou afixado em paredes com o objetivo de divulgar publicamente algo}
  \end{phonetics}
\end{entry}

\begin{entry}{吵}{7}{⼝}
  \begin{phonetics}{吵}{chao3}[][HSK 3]
    \definition{adj.}{barulhento; ruidoso}
    \definition{v.}{perturbar fazendo barulho; fazer barulho | discutir; brigar; disputar}
  \end{phonetics}
\end{entry}

\begin{entry}{吵架}{7,9}{⼝、⽊}
  \begin{phonetics}{吵架}{chao3jia4}[][HSK 3]
    \definition{v.+compl.}{brigar; discutir; ter uma briga}
  \end{phonetics}
\end{entry}

\begin{entry}{吹}{7}{⼝}
  \begin{phonetics}{吹}{chui1}[][HSK 2]
    \definition{v.}{soprar; baforar | tocar (instrumentos de sopro) | (do vento) soprar | gabar-se; vangloriar-se | elogiar; louvar aos céus; adular | (relacionamento) romper; separar-se; (assunto) fracassar}
  \end{phonetics}
\end{entry}

\begin{entry}{吹牛}{7,4}{⼝、⽜}
  \begin{phonetics}{吹牛}{chui1niu2}
    \definition{v.+compl.}{ogulhar-se | gabar-se | destacar-se}
  \end{phonetics}
\end{entry}

\begin{entry}{吾}{7}{⼝}
  \begin{phonetics}{吾}{wu2}
    \definition*{s.}{sobrenome Wu}
    \definition{pron.}{eu | (antigo) meu}
  \end{phonetics}
\end{entry}

\begin{entry}{呀}{7}{⼝}
  \begin{phonetics}{呀}{ya5}[][HSK 4]
    \definition{part.}{usado no lugar de 啊 quando a palavra anterior termina com o som a, e, i, o ou ü}
  \seealsoref{啊}{a5}
  \end{phonetics}
\end{entry}

\begin{entry}{呆}{7}{⼝}
  \begin{phonetics}{呆}{dai1}[][HSK 5]
    \definition*{s.}{sobrenome Dai}
    \definition{adj.}{maçante; de raciocínio lento | em branco; de madeira; rígido; inflexível}
    \definition{v.}{ficar; permanecer;}
  \end{phonetics}
\end{entry}

\begin{entry}{告别}{7,7}{⼝、⼑}
  \begin{phonetics}{告别}{gao4bie2}[][HSK 3]
    \definition{v.+compl.}{dizer adeus a | deixar; partir de | prestar as últimas homenagens ao falecido}
  \end{phonetics}
\end{entry}

\begin{entry}{告诉}{7,7}{⼝、⾔}
  \begin{phonetics}{告诉}{gao4su4}
    \definition{v.}{dizer; informar (dar a conhecer); dizer aos outros, para que todos saibam}
  \end{phonetics}
  \begin{phonetics}{告诉}{gao4su5}[][HSK 1]
    \definition{v.}{dizer; informar (dar a conhecer)}
  \end{phonetics}
\end{entry}

\begin{entry}{告急}{7,9}{⼝、⼼}
  \begin{phonetics}{告急}{gao4ji2}
    \definition{v.}{estar em estado de emergência | relatar uma emergência | solicitar assistência de emergência}
  \end{phonetics}
\end{entry}

\begin{entry}{员}{7}{⼝}
  \begin{phonetics}{员}{yuan2}[][HSK 3]
    \definition{clas.}{para comandantes militares}
    \definition{s.}{uma pessoa envolvida em algum campo de atividade; refere-se a pessoas que trabalham ou estudam | membro; refere-se aos membros de um grupo ou organização}
  \end{phonetics}
\end{entry}

\begin{entry}{员工}{7,3}{⼝、⼯}
  \begin{phonetics}{员工}{yuan2gong1}[][HSK 3]
    \definition[位,名,个]{s.}{funcionário; atendente; balconista; empregado; trabalhador; pessoal}
  \end{phonetics}
\end{entry}

\begin{entry}{园林}{7,8}{⼞、⽊}
  \begin{phonetics}{园林}{yuan2lin2}[][HSK 5]
    \definition{s.}{parque; jardim; área paisagística com plantas e árvores para as pessoas apreciarem e descansarem.}
  \end{phonetics}
\end{entry}

\begin{entry}{囯}{7}{⼞}
  \begin{phonetics}{囯}{guo2}
    \variantof{国}
  \end{phonetics}
\end{entry}

\begin{entry}{困}{7}{⼞}
  \begin{phonetics}{困}{kun4}[][HSK 3]
    \definition{adj.}{cansado | sonolento}
    \definition{v.}{estar encalhado; estar em grande pressão | cercar; prender; sitiar; cercar; rodear}
  \end{phonetics}
\end{entry}

\begin{entry}{困扰}{7,7}{⼞、⼿}
  \begin{phonetics}{困扰}{kun4 rao3}[][HSK 5]
    \definition{v.}{perturbar; deixar perplexo; perseguir}
  \end{phonetics}
\end{entry}

\begin{entry}{困难}{7,10}{⼞、⾫}
  \begin{phonetics}{困难}{kun4nan5}[][HSK 3]
    \definition{adj.}{dificuldades financeiras; circunstâncias difíceis | complicado; nodoso; difícil; duro;}
    \definition[种]{s.}{dificuldade; situação difícil}
  \end{phonetics}
\end{entry}

\begin{entry}{围}{7}{⼞}
  \begin{phonetics}{围}{wei2}[][HSK 3]
    \definition*{s.}{sobrenome Wei}
    \definition{clas.}{o comprimento dos dois polegares e indicadores ou o comprimento de ambos os braços quando unidos}
    \definition{s.}{em volta de tudo; ao redor}
    \definition{v.}{cercar; rodear; circundar; encurralar | enrolar; envolver}
  \end{phonetics}
\end{entry}

\begin{entry}{围巾}{7,3}{⼞、⼱}
  \begin{phonetics}{围巾}{wei2jin1}[][HSK 4]
    \definition[条]{s.}{lenço; cachecol; echarpe; gravata; tiras longas de malha ou tecido usadas ao redor do pescoço para aquecimento, proteção do colarinho ou decoração}
  \end{phonetics}
\end{entry}

\begin{entry}{围绕}{7,9}{⼞、⽷}
  \begin{phonetics}{围绕}{wei2rao4}[][HSK 5]
    \definition{v.}{girar; circundar; dar voltas; girar em torno de algo; cercar | concentrar-se em; centrar-se em; centrar-se em uma questão ou evento (para realizar atividades)}
  \end{phonetics}
\end{entry}

\begin{entry}{坏}{7}{⼟}
  \begin{phonetics}{坏}{huai4}[][HSK 1]
    \definition{adj.}{ruim; prejudicial; insatisfatório; péssimo | mal; extremamente; indica um grau profundo, geralmente usado após verbos ou adjetivos que expressam estado psicológico | podre; estragado; impróprio; prejudicial ao uso}
    \definition[种]{s.}{ideia maligna; truque sujo; péssima ideia}
    \definition{v.}{estragar; destruir; corromper}
  \end{phonetics}
\end{entry}

\begin{entry}{坏人}{7,2}{⼟、⼈}
  \begin{phonetics}{坏人}{huai4 ren2}[][HSK 2]
    \definition[个,种]{s.}{malfeitor; canalha; pessoa má; pessoa de má qualidade; pessoa que faz coisas ruins}
  \end{phonetics}
\end{entry}

\begin{entry}{坏处}{7,5}{⼟、⼡}
  \begin{phonetics}{坏处}{huai4 chu4}[][HSK 2]
    \definition[个]{s.}{dano; prejuízo; desvantagem; fatores prejudiciais a pessoas ou coisas}
  \end{phonetics}
\end{entry}

\begin{entry}{坏蛋}{7,11}{⼟、⾍}
  \begin{phonetics}{坏蛋}{huai4dan4}
    \definition{s.}{bastardo | canalha | pessoa má}
  \end{phonetics}
\end{entry}

\begin{entry}{坐}{7}{⼟}
  \begin{phonetics}{坐}{zuo4}[][HSK 1]
    \definition*{s.}{sobrenome Zuo}
    \definition{adv.}{sem motivo algum; sem causa ou razão; sem motivo aparente}
    \definition{prep.}{porque; pelo fato de que; pela razão de que; pelo motivo de que}
    \definition{s.}{assento; lugar; posição}
    \definition{v.}{sentar; sentar-se; ocupar um lugar; colocar os glúteos sobre um objeto para apoiar o peso corporal | pegar; viajar de; pegar carona | ter as costas voltadas para | colocar (uma panela, chaleira, etc.) no fogo | recuo; coice (de rifles, armas, etc.)  | produzir frutos; formar sementes | ser punido; ser acusado de crime | contrair (ou ter) uma doença; sofrer de uma doença | (um edifício) afundar; ceder}
  \end{phonetics}
\end{entry}

\begin{entry}{坐下}{7,3}{⼟、⼀}
  \begin{phonetics}{坐下}{zuo4 xia5}[][HSK 1]
    \definition{v.}{sentar-se; tomar um assento; passar da posição em pé para a posição sentada}
  \end{phonetics}
\end{entry}

\begin{entry}{坐车}{7,4}{⼟、⾞}
  \begin{phonetics}{坐车}{zuo4che1}
    \definition{v.}{andar de carro, ônibus, trem, etc.}
  \end{phonetics}
\end{entry}

\begin{entry}{坐好}{7,6}{⼟、⼥}
  \begin{phonetics}{坐好}{zuo4hao3}
    \definition{v.}{sentar-se corretamente | sentar direito}
  \end{phonetics}
\end{entry}

\begin{entry}{坐享}{7,8}{⼟、⼇}
  \begin{phonetics}{坐享}{zuo4xiang3}
    \definition{v.}{curtir algo sem levantar um dedo}
  \end{phonetics}
\end{entry}

\begin{entry}{坐垫}{7,9}{⼟、⼟}
  \begin{phonetics}{坐垫}{zuo4dian4}
    \definition[块]{s.}{assento (motocicleta) | almofada}
  \end{phonetics}
\end{entry}

\begin{entry}{坐标}{7,9}{⼟、⽊}
  \begin{phonetics}{坐标}{zuo4biao1}
    \definition{s.}{coordenada (geometria)}
  \end{phonetics}
\end{entry}

\begin{entry}{坑}{7}{⼟}
  \begin{phonetics}{坑}{keng1}
    \definition{s.}{poço | depressão | túnel | buraco no chão}
    \definition{v.}{enganar | trapacear}
  \end{phonetics}
\end{entry}

\begin{entry}{坑人}{7,2}{⼟、⼈}
  \begin{phonetics}{坑人}{keng1ren2}
    \definition{v.+compl.}{trapacear alguém}
  \end{phonetics}
\end{entry}

\begin{entry}{块}{7}{⼟}
  \begin{phonetics}{块}{kuai4}[][HSK 1]
    \definition{clas.}{usado para coisas em pedaços | usado para coisas em pedaços ou em algumas formas de folhas | usado para moedas de prata ou notas de papel equivalentes a 圆}
    \definition{s.}{pedaço; pedaço (de terra); peça; algo que forma um pedaço ou massa}
  \seealsoref{圆}{yuan2}
  \end{phonetics}
\end{entry}

\begin{entry}{坚决}{7,6}{⼟、⼎}
  \begin{phonetics}{坚决}{jian1jue2}[][HSK 3]
    \definition{adj.}{firme; resoluto}
  \end{phonetics}
\end{entry}

\begin{entry}{坚守}{7,6}{⼟、⼧}
  \begin{phonetics}{坚守}{jian1shou3}
    \definition{v.}{agarrar-se}
  \end{phonetics}
\end{entry}

\begin{entry}{坚固}{7,8}{⼟、⼞}
  \begin{phonetics}{坚固}{jian1gu4}[][HSK 4]
    \definition{adj.}{firme; sólido; robusto; forte; durável; firmemente unidos e inquebráveis}
  \end{phonetics}
\end{entry}

\begin{entry}{坚定}{7,8}{⼟、⼧}
  \begin{phonetics}{坚定}{jian1ding4}[][HSK 5]
    \definition{adj.}{firme; inabalável; inamovível; (posição, opinião, vontade, etc.) firme e estável, inabalável}
    \definition{v.}{fortalecer}
  \end{phonetics}
\end{entry}

\begin{entry}{坚持}{7,9}{⼟、⼿}
  \begin{phonetics}{坚持}{jian1chi2}[][HSK 3]
    \definition{v.}{persistir e; perseverar em; sustentar; insistir em; manter-se fiel a; aderir a}
  \end{phonetics}
\end{entry}

\begin{entry}{坚强}{7,12}{⼟、⼸}
  \begin{phonetics}{坚强}{jian1qiang2}[][HSK 3]
    \definition{adj.}{forte; firme; convicto}
    \definition{v.}{fortalecer; tornar forte}
  \end{phonetics}
\end{entry}

\begin{entry}{坠}{7}{⼟}
  \begin{phonetics}{坠}{zhui4}
    \definition{v.}{cair | pesar | fazer vergar com o peso}
  \end{phonetics}
\end{entry}

\begin{entry}{坠落}{7,12}{⼟、⾋}
  \begin{phonetics}{坠落}{zhui4luo4}
    \definition{v.}{cair}
  \end{phonetics}
\end{entry}

\begin{entry}{声}{7}{⼠}
  \begin{phonetics}{声}{sheng1}[][HSK 5]
    \definition{clas.}{indica o número de vezes que um som é emitido}
    \definition{s.}{som; voz | reputação | consoante inicial (de uma sílaba chinesa) | tom; tom de voz | informação; notícia}
    \definition{v.}{declarar; anunciar; emitir um som}
  \end{phonetics}
\end{entry}

\begin{entry}{声明}{7,8}{⼠、⽇}
  \begin{phonetics}{声明}{sheng1ming2}[][HSK 3]
    \definition[项,份]{s.}{declaração}
    \definition{v.}{declarar}
    \definition{v.}{declarar; anunciar}
  \end{phonetics}
\end{entry}

\begin{entry}{声音}{7,9}{⼠、⾳}
  \begin{phonetics}{声音}{sheng1yin1}[][HSK 2]
    \definition[个,种]{s.}{som; voz; a percepção auditiva das ondas sonoras}
  \end{phonetics}
\end{entry}

\begin{entry}{壳}{7}{⼠}
  \begin{phonetics}{壳}{ke2}
    \definition{s.}{casca (de ovo, noz, caranguejo, etc.) | caixa | invólucro | alojamento (de uma máquina ou dispositivo)}
  \end{phonetics}
\end{entry}

\begin{entry}{妖}{7}{⼥}
  \begin{phonetics}{妖}{yao1}
    \definition{adj.}{enfeitiçante | encantador}
    \definition{s.}{\emph{goblin} | bruxa | diabo | monstro | fantasma | demônio}
  \end{phonetics}
\end{entry}

\begin{entry}{妙招}{7,8}{⼥、⼿}
  \begin{phonetics}{妙招}{miao4zhao1}
    \definition{adj.}{escorregadio}
    \definition{s.}{movimento inteligente | maneira inteligente de fazer algo}
  \end{phonetics}
\end{entry}

\begin{entry}{宋}{7}{⼧}
  \begin{phonetics}{宋}{song4}
    \definition*{s.}{sobrenome Song}
    \definition{s.}{Dinastia Song (960-1279) | Song das dinastias do sul (420-479)}
  \end{phonetics}
\end{entry}

\begin{entry}{完}{7}{⼧}
  \begin{phonetics}{完}{wan2}[][HSK 2]
    \definition*{s.}{sobrenome Wan}
    \definition{adj.}{inteiro; intacto; completo}
    \definition{v.}{acabar; terminar; completar | pagar | estar terminado; estar pronto para | esgotar; ser usado}
  \end{phonetics}
\end{entry}

\begin{entry}{完了}{7,2}{⼧、⼅}
  \begin{phonetics}{完了}{wan2 le5}[][HSK 5]
    \definition{v.}{acabar; terminar; concluir; chegar ao fim}
  \end{phonetics}
\end{entry}

\begin{entry}{完人}{7,2}{⼧、⼈}
  \begin{phonetics}{完人}{wan2ren2}
    \definition{s.}{pessoa perfeita}
  \end{phonetics}
\end{entry}

\begin{entry}{完全}{7,6}{⼧、⼊}
  \begin{phonetics}{完全}{wan2quan2}[][HSK 2]
    \definition{adj.}{inteiro; completo; não falta nada, está tudo completo}
    \definition{adv.}{completamente; representa tudo}
  \end{phonetics}
\end{entry}

\begin{entry}{完成}{7,6}{⼧、⼽}
  \begin{phonetics}{完成}{wan2cheng2}[][HSK 2]
    \definition{v.}{realizar; completar; terminar; cumprir; levar ao sucesso}
  \end{phonetics}
\end{entry}

\begin{entry}{完毕}{7,6}{⼧、⽐}
  \begin{phonetics}{完毕}{wan2bi4}
    \definition{v.}{completar | terminar | acabar}
  \end{phonetics}
\end{entry}

\begin{entry}{完完全全}{7,7,6,6}{⼧、⼧、⼊、⼊}
  \begin{phonetics}{完完全全}{wan2wan2quan2quan2}
    \definition{adv.}{completamente}
  \end{phonetics}
\end{entry}

\begin{entry}{完备}{7,8}{⼧、⼡}
  \begin{phonetics}{完备}{wan2bei4}
    \definition{adj.}{completo | impecável | perfeito}
    \definition{v.}{não deixar nada a desejar}
  \end{phonetics}
\end{entry}

\begin{entry}{完美}{7,9}{⼧、⽺}
  \begin{phonetics}{完美}{wan2mei3}[][HSK 3]
    \definition{adj.}{perfeito; impecável; consumado}
    \definition{adv.}{perfeitamente}
    \definition{s.}{perfeição}
  \end{phonetics}
\end{entry}

\begin{entry}{完善}{7,12}{⼧、⼝}
  \begin{phonetics}{完善}{wan2shan4}[][HSK 3]
    \definition{adj.}{perfeito; consumado}
    \definition{v.}{refinar; melhorar; tornar perfeito}
  \end{phonetics}
\end{entry}

\begin{entry}{完税}{7,12}{⼧、⽲}
  \begin{phonetics}{完税}{wan2shui4}
    \definition{v.}{pagar imposto}
  \end{phonetics}
\end{entry}

\begin{entry}{完满}{7,13}{⼧、⽔}
  \begin{phonetics}{完满}{wan2man3}
    \definition{adj.}{satisfatório | bem-sucedido}
  \end{phonetics}
\end{entry}

\begin{entry}{完整}{7,16}{⼧、⽁}
  \begin{phonetics}{完整}{wan2zheng3}[][HSK 3]
    \definition{adj.}{intacto; inteiro; completo; integrado}
  \end{phonetics}
\end{entry}

\begin{entry}{寿司}{7,5}{⼨、⼝}
  \begin{phonetics}{寿司}{shou4 si1}[][HSK 5]
    \definition[份]{s.}{\emph{sushi}; iguaria tradicional japonesa}
  \end{phonetics}
\end{entry}

\begin{entry}{尾巴}{7,4}{⼫、⼰}
  \begin{phonetics}{尾巴}{wei3ba5}[][HSK 4]
    \definition{s.}{cauda; projeções na extremidade do corpo de certos animais | parte semelhante a uma cauda; refere-se, em geral, ao final de algo | apêndice; anexo; adepto servil; pessoa que segue ou concorda com outra pessoa | (figura de linguagem) alguém que faz sombra a outro | fim; remanescente; parte restante (ou inacabada)}
  \end{phonetics}
\end{entry}

\begin{entry}{尿}{7}{⼫}
  \begin{phonetics}{尿}{niao4}
    \definition[泡]{s.}{urina}
    \definition{v.}{urinar}
  \end{phonetics}
  \begin{phonetics}{尿}{sui1}
    \definition{s.}{(coloquial) urina}
  \end{phonetics}
\end{entry}

\begin{entry}{局}{7}{⼫}
  \begin{phonetics}{局}{ju2}[][HSK 4]
    \definition{s.}{tabuleiro de xadrez | jogo; turno; \emph{set} | situação; estado das coisas | tolerância; grandeza ou pequenez da mente; grau de tolerância de uma pessoa em relação às outras | reunião de pessoas em festas | ardil; artidício; estratagema; armadilha | parte; porção; parcela | nome de determinadas lojas}
  \end{phonetics}
\end{entry}

\begin{entry}{局长}{7,4}{⼫、⾧}
  \begin{phonetics}{局长}{ju2 zhang3}[][HSK 5]
    \definition[位,个]{s.}{comissário; diretor; principais chefes de gabinete do governo}
  \end{phonetics}
\end{entry}

\begin{entry}{局面}{7,9}{⼫、⾯}
  \begin{phonetics}{局面}{ju2mian4}[][HSK 5]
    \definition[种]{s.}{aspecto; fase; situação; o estado das coisas em um período de tempo, em sua maior parte abstraído | escopo; escala}
  \end{phonetics}
\end{entry}

\begin{entry}{屁股}{7,8}{⼫、⾁}
  \begin{phonetics}{屁股}{pi4gu5}
    \definition{s.}{nádega | quadris}
  \end{phonetics}
\end{entry}

\begin{entry}{屁话}{7,8}{⼫、⾔}
  \begin{phonetics}{屁话}{pi4hua4}
    \definition{s.}{absurdo | tolice | besteira}
  \end{phonetics}
\end{entry}

\begin{entry}{层}{7}{⼫}
  \begin{phonetics}{层}{ceng2}[][HSK 2]
    \definition{clas.}{usado para coisas que se sobrepõem e se acumulam, como andares, camadas e estratos | usado para coisas que podem ser divididas em itens e etapas | usado para coisas que podem ser removidas ou apagadas da superfície de um objeto}
    \definition{s.}{camada; nível; coisas que se sobrepõem | nível; classificação; camada}
    \definition{v.}{sobrepor; empilhar camada sobre camada}
  \end{phonetics}
\end{entry}

\begin{entry}{层次}{7,6}{⼫、⽋}
  \begin{phonetics}{层次}{ceng2ci4}[][HSK 5]
    \definition{s.}{disposição ordenada do conteúdo (de um discurso ou texto) | nível ou estrutura administrativa; distinções entre a mesma coisa devido a diferenças de tamanho, altura, etc. | nível; níveis de afiliação}
  \end{phonetics}
\end{entry}

\begin{entry}{层层}{7,7}{⼫、⼫}
  \begin{phonetics}{层层}{ceng2ceng2}
    \definition{s.}{camada sobre camada}
  \end{phonetics}
\end{entry}

\begin{entry}{希望}{7,11}{⼱、⽉}
  \begin{phonetics}{希望}{xi1wang4}[][HSK 3]
    \definition[个]{s.}{esperança; desejo; expectativa | aquilo em que a esperança é depositada}
    \definition{v.}{ter esperança; desejar; esperar}
  \end{phonetics}
\end{entry}

\begin{entry}{床}{7}{⼴}
  \begin{phonetics}{床}{chuang2}[][HSK 1]
    \definition{clas.}{para colchas, roupas de cama, etc.}
    \definition[张]{s.}{cama; sofá; móveis para dormir | algo com o formato de uma cama}
  \end{phonetics}
\end{entry}

\begin{entry}{库}{7}{⼴}
  \begin{phonetics}{库}{ku4}[][HSK 5]
    \definition{s.}{depósito; tesouraria; armazém; almoxarifado; edifícios e equipamentos para armazenamento de mercadorias | banco de dados}
  \end{phonetics}
\end{entry}

\begin{entry}{应}{7}{⼴}
  \begin{phonetics}{应}{ying1}[][HSK 4,5]
    \definition{v.}{ecoar; responder; responder a; responder às chamadas, saudações, perguntas, etc. de outras pessoas | conceder; cumprir | adequar; adaptar; responder a | lidar com; enfrentar; abordar | tornar-se realidade; ser cumprido}
  \end{phonetics}
\end{entry}

\begin{entry}{应对}{7,5}{⼴、⼨}
  \begin{phonetics}{应对}{ying4dui4}
    \definition{v.}{responder | manusear | lidar}
  \end{phonetics}
\end{entry}

\begin{entry}{应用}{7,5}{⼴、⽤}
  \begin{phonetics}{应用}{ying4yong4}[][HSK 3]
    \definition{adj.}{aplicado (na vida ou na produção); usado diretamente na vida ou na produção}
    \definition{s.}{aplicativo}
    \definition{v.}{usar; aplicar}
  \end{phonetics}
\end{entry}

\begin{entry}{应用程序}{7,5,12,7}{⼴、⽤、⽲、⼴}
  \begin{phonetics}{应用程序}{ying4yong4cheng2xu4}
    \definition{s.}{aplicativo | programa de computador}
  \end{phonetics}
\end{entry}

\begin{entry}{应用程序接口}{7,5,12,7,11,3}{⼴、⽤、⽲、⼴、⼿、⼝}
  \begin{phonetics}{应用程序接口}{ying4yong4cheng2xu4jie1kou3}
    \definition{s.}{API (\emph{application programming interface})}
  \seealsoref{应用程序编程接口}{ying4yong4cheng2xu4bian1cheng2jie1kou3}
  \end{phonetics}
\end{entry}

\begin{entry*}{应用程序编程接口}{7,5,12,7,12,12,11,3}{⼴、⽤、⽲、⼴、⽷、⽲、⼿、⼝}
  \begin{phonetics}{应用程序编程接口}{ying4yong4cheng2xu4bian1cheng2jie1kou3}
    \definition{s.}{API (\emph{application programming interface})}
  \seealsoref{应用程序接口}{ying4yong4cheng2xu4jie1kou3}
  \end{phonetics}
\end{entry*}

\begin{entry}{应当}{7,6}{⼴、⼹}
  \begin{phonetics}{应当}{ying1 dang1}[][HSK 3]
    \definition{v.}{dever}
  \end{phonetics}
\end{entry}

\begin{entry}{应该}{7,8}{⼴、⾔}
  \begin{phonetics}{应该}{ying1gai1}[][HSK 2]
    \definition{v.}{dever | ter de}
  \end{phonetics}
\end{entry}

\begin{entry}{弄}{7}{⼶}
  \begin{phonetics}{弄}{long4}
    \definition{s.}{rua estreita; beco; viela; travessa}
  \end{phonetics}
  \begin{phonetics}{弄}{nong4}[][HSK 2]
    \definition{v.}{fazer, realizar; tratar; organizar | obter; buscar; tentar conseguir; encontrar uma maneira de conseguir | brincar com; enganar | pregar uma peça; brincar; manipular | mexer com; perturbar}
  \end{phonetics}
\end{entry}

\begin{entry}{弟}{7}{⼸}
  \begin{phonetics}{弟}{di4}[][HSK 1]
    \definition*{s.}{sobrenome Di}
    \definition[个]{s.}{irmão mais novo | (entre amigos homens) eu | geralmente se refere a colegas do sexo masculino mais jovens na família ou entre parentes | forma humilde que os amigos usam para se referir uns aos outros, usada principalmente em correspondência}
  \end{phonetics}
\end{entry}

\begin{entry}{弟弟}{7,7}{⼸、⼸}
  \begin{phonetics}{弟弟}{di4 di5}[][HSK 1]
    \definition[个,位]{s.}{irmão mais novo | primo}
  \end{phonetics}
\end{entry}

\begin{entry}{弟妹}{7,8}{⼸、⼥}
  \begin{phonetics}{弟妹}{di4mei4}
    \definition{s.}{esposa do irmão mais novo}
  \end{phonetics}
\end{entry}

\begin{entry}{张}{7}{⼸}
  \begin{phonetics}{张}{zhang1}[][HSK 3]
    \definition*{s.}{sobrenome Zhang}
    \definition*{s.}{Zhang, uma das mansões lunares}
    \definition{adj.}{nervoso; tenso}
    \definition{clas.}{para papel, couro, etc. | para camas, mesas, etc. | para a boca e o rosto | para arcos}
    \definition{s.}{folha de papel}
    \definition{v.}{consertar (uma corda de arco); encordoar (um instrumento musical ou um arco) | abrir; espalhar; esticar | expor; exibir |expandir; estender | ampliar; exagerar | olhar | dar rédea solta a; satisfazer | iniciar um negócio; abrir uma loja | colocar em bom uso; dar liberdade para | pegar com uma rede; montar armadilhas para capturar pássaros e animais}
  \end{phonetics}
\end{entry}

\begin{entry}{张三}{7,3}{⼸、⼀}
  \begin{phonetics}{张三}{zhang1san1}
    \definition*{s.}{Zhang San | Zé Ninguém | nome para uma pessoa não especificada, 1 de 3}
  \seealsoref{李四}{li3si4}
  \seealsoref{王五}{wang2wu3}
  \end{phonetics}
\end{entry}

\begin{entry}{张狂}{7,7}{⼸、⽝}
  \begin{phonetics}{张狂}{zhang1kuang2}
    \definition{adj.}{impetuoso | frenético | insolente}
  \end{phonetics}
\end{entry}

\begin{entry}{形式}{7,6}{⼺、⼷}
  \begin{phonetics}{形式}{xing2shi4}[][HSK 3]
    \definition[种,个]{s.}{forma; formato; modalidade | aparência, estrutura ou estado de algo}
  \end{phonetics}
\end{entry}

\begin{entry}{形成}{7,6}{⼺、⼽}
  \begin{phonetics}{形成}{xing2cheng2}[][HSK 3]
    \definition{v.}{moldar; formar; tomar forma | tornar-se algo ou algo através do desenvolvimento e da mudança}
  \end{phonetics}
\end{entry}

\begin{entry}{形而上学}{7,6,3,8}{⼺、⽽、⼀、⼦}
  \begin{phonetics}{形而上学}{xing2'er2shang4xue2}
    \definition{s.}{metafísica}
  \end{phonetics}
\end{entry}

\begin{entry}{形状}{7,7}{⼺、⽝}
  \begin{phonetics}{形状}{xing2zhuang4}[][HSK 3]
    \definition[个]{s.}{forma; aparência | a aparência de um objeto ou figura formada pela combinação de superfícies ou linhas externas}
  \end{phonetics}
\end{entry}

\begin{entry}{形势}{7,8}{⼺、⼒}
  \begin{phonetics}{形势}{xing2shi4}[][HSK 4]
    \definition[个]{s.}{terreno; características topográficas; situação geográfica, principalmente de uma perspectiva militar | situação; circunstâncias; a situação geral, a tendência de como as coisas estão se desenvolvendo e mudando | geralmente não é usado em situações pessoais}
  \end{phonetics}
\end{entry}

\begin{entry}{形态}{7,8}{⼺、⼼}
  \begin{phonetics}{形态}{xing2tai4}[][HSK 5]
    \definition{s.}{forma; forma como as coisas se apresentam | forma; padrão; postura | morfologia; forma; (gramática) refere-se às formas internas de mudança das palavras, incluindo a formação de palavras e as mudanças morfológicas}
  \end{phonetics}
\end{entry}

\begin{entry}{形容}{7,10}{⼺、⼧}
  \begin{phonetics}{形容}{xing2rong2}[][HSK 4]
    \definition{s.}{aparência; semblante}
    \definition{v.}{descrever}
  \end{phonetics}
\end{entry}

\begin{entry}{形象}{7,11}{⼺、⾗}
  \begin{phonetics}{形象}{xing2xiang4}[][HSK 3]
    \definition{adj.}{vívido}
    \definition[个]{s.}{imagem; forma; figura | uma forma ou gesto específico que pode despertar os pensamentos ou emoções das pessoas | imagem literária; imagem artística | pessoas ou coisas com características diferentes criadas na literatura, no cinema e em outras artes}
  \end{phonetics}
\end{entry}

\begin{entry}{彻底}{7,8}{⼻、⼴}
  \begin{phonetics}{彻底}{che4di3}[][HSK 4]
    \definition{adj.}{minucioso; completo; exaustivo; profundo e completo; nada é deixado de fora}
  \end{phonetics}
\end{entry}

\begin{entry}{忍}{7}{⼼}
  \begin{phonetics}{忍}{ren3}[][HSK 5]
    \definition{v.}{suportar; aguentar; tolerar; aturar | ter coragem para; ser insensível o suficiente para; ser capaz de endurecer o coração e fazer coisas que não se devem fazer por uma questão de razão}
  \end{phonetics}
\end{entry}

\begin{entry}{忍不住}{7,4,7}{⼼、⼀、⼈}
  \begin{phonetics}{忍不住}{ren3bu5zhu4}[][HSK 5]
    \definition{v.}{incapaz de suportar; não conseguir evitar fazer algo; não conseguir se controlar}
  \end{phonetics}
\end{entry}

\begin{entry}{忍受}{7,8}{⼼、⼜}
  \begin{phonetics}{忍受}{ren3shou4}[][HSK 5]
    \definition{v.}{suportar; sofrer; aguentar; tolerar; suportar com dificuldade o sofrimento, as dificuldades e as adversidades da vida}
  \end{phonetics}
\end{entry}

\begin{entry}{忍耐}{7,9}{⼼、⽽}
  \begin{phonetics}{忍耐}{ren3nai4}
    \definition{s.}{paciência | resistência}
    \definition{v.}{suportar | resistir | exercer paciência}
  \end{phonetics}
\end{entry}

\begin{entry}{志愿}{7,14}{⼼、⽕}
  \begin{phonetics}{志愿}{zhi4 yuan4}[][HSK 3]
    \definition{s.}{desejo; ideal; aspiração; meta que se espera alcançar}
    \definition{v.}{ser voluntário; tomar a iniciativa e esteja disposto a fazer um trabalho que não gere renda ou que tenha renda muito baixa, mas que possa ajudar outras pessoas}
  \end{phonetics}
\end{entry}

\begin{entry}{志愿者}{7,14,8}{⼼、⽕、⽼}
  \begin{phonetics}{志愿者}{zhi4yuan4zhe3}[][HSK 3]
    \definition{s.}{voluntário; pessoa que se voluntaria para servir em atividades de assistência social, eventos de grande porte, conferências, etc.}
  \end{phonetics}
\end{entry}

\begin{entry}{忘}{7}{⼼}
  \begin{phonetics}{忘}{wang4}[][HSK 1]
    \definition{v.}{esquecer | ignorar; negligenciar}
  \end{phonetics}
\end{entry}

\begin{entry}{忘本}{7,5}{⼼、⽊}
  \begin{phonetics}{忘本}{wang4ben3}
    \definition{v.}{esquecer as próprias raízes}
  \end{phonetics}
\end{entry}

\begin{entry}{忘记}{7,5}{⼼、⾔}
  \begin{phonetics}{忘记}{wang4ji4}[][HSK 1]
    \definition{v.}{esquecer | ignorar; negligenciar | sair da memória de alguém; não ser lembrado | descartar da mente; ignorar}
  \end{phonetics}
\end{entry}

\begin{entry}{忘却}{7,7}{⼼、⼙}
  \begin{phonetics}{忘却}{wang4que4}
    \definition{v.}{esquecer}
  \end{phonetics}
\end{entry}

\begin{entry}{忘怀}{7,7}{⼼、⼼}
  \begin{phonetics}{忘怀}{wang4huai2}
    \definition{v.}{esquecer}
  \end{phonetics}
\end{entry}

\begin{entry}{忘恩}{7,10}{⼼、⼼}
  \begin{phonetics}{忘恩}{wang4'en1}
    \definition{v.}{ser ingrato}
  \end{phonetics}
\end{entry}

\begin{entry}{忘掉}{7,11}{⼼、⼿}
  \begin{phonetics}{忘掉}{wang4diao4}
    \definition{v.}{esquecer}
  \end{phonetics}
\end{entry}

\begin{entry}{忘餐}{7,16}{⼼、⾷}
  \begin{phonetics}{忘餐}{wang4can1}
    \definition{v.}{esquecer as refeições}
  \end{phonetics}
\end{entry}

\begin{entry}{忧郁}{7,8}{⼼、⾢}
  \begin{phonetics}{忧郁}{you1yu4}
    \definition{adj.}{deprimido | melancólico | desanimado}
    \definition{s.}{depressão | melancolia}
  \end{phonetics}
\end{entry}

\begin{entry}{快}{7}{⼼}
  \begin{phonetics}{快}{kuai4}[][HSK 1]
    \definition*{s.}{sobrenome Kuai}
    \definition{adj.}{rápido; veloz (oposto a 慢) | apressado | perspicaz; ágil; inteligente; de ​​mente rápida | (de uma faca, espada, etc.) afiado (oposto a 钝) | direto; franco; sem rodeios | satisfeito; feliz; gratificado | rápido; veloz; alta velocidade; tempo de execução curto | satisfeito; feliz; contente | engenhoso; ágil | afiado; facas, tesouras, machados e outros objetos afiados | sincero}
    \definition{adv.}{em breve; antes de muito tempo; estar prestes a | rapidamente}
    \definition{s.}{policial; polícia | (antigo) oficial encarregado de efetuar prisões}
  \seealsoref{钝}{dun4}
  \seealsoref{慢}{man4}
  \end{phonetics}
\end{entry}

\begin{entry}{快乐}{7,5}{⼼、⼃}
  \begin{phonetics}{快乐}{kuai4le4}[][HSK 2]
    \definition{adj.}{feliz; alegre; animado; prazeiroso}
    \definition{s.}{felicidade | alegria}
  \end{phonetics}
\end{entry}

\begin{entry}{快活}{7,9}{⼼、⽔}
  \begin{phonetics}{快活}{kuai4huo5}[][HSK 5]
    \definition{adj.}{feliz; alegre; contente; animado}
  \end{phonetics}
\end{entry}

\begin{entry}{快点儿}{7,9,2}{⼼、⽕、⼉}
  \begin{phonetics}{快点儿}{kuai4 dian3r5}[][HSK 2]
    \definition{v.}{apressar-se}
  \end{phonetics}
\end{entry}

\begin{entry}{快要}{7,9}{⼼、⾑}
  \begin{phonetics}{快要}{kuai4 yao4}[][HSK 2]
    \definition{adv.}{estar prestes a; estar indo para; estar à beira de; em breve; em pouco tempo; indica que a situação está prestes a ocorrer}
  \end{phonetics}
\end{entry}

\begin{entry}{快递}{7,10}{⼼、⾡}
  \begin{phonetics}{快递}{kuai4 di4}[][HSK 4]
    \definition[个]{s.}{correio rápido; entrega expressa; entrega rápida}
    \definition{v.}{entregar (serviço de entrega rápida por transportadoras especializadas)}
  \end{phonetics}
\end{entry}

\begin{entry}{快速}{7,10}{⼼、⾡}
  \begin{phonetics}{快速}{kuai4 su4}[][HSK 3]
    \definition{adj.}{rápido; veloz; de alta velocidade}
  \end{phonetics}
\end{entry}

\begin{entry}{快餐}{7,16}{⼼、⾷}
  \begin{phonetics}{快餐}{kuai4 can1}[][HSK 2]
    \definition[份,顿]{s.}{pedido (comida) rápido; \emph{fast food}; refere-se a refeições simples preparadas com antecedência e que podem ser servidas rapidamente}
  \end{phonetics}
\end{entry}

\begin{entry}{怀旧}{7,5}{⼼、⽇}
  \begin{phonetics}{怀旧}{huai2jiu4}
    \definition{s.}{nostalgia}
    \definition{v.}{sentir-se nostálgico}
  \end{phonetics}
\end{entry}

\begin{entry}{怀念}{7,8}{⼼、⼼}
  \begin{phonetics}{怀念}{huai2nian4}[][HSK 4]
    \definition{v.}{pensar em; valorizar a memória de}
  \end{phonetics}
\end{entry}

\begin{entry}{怀疑}{7,14}{⼼、⽦}
  \begin{phonetics}{怀疑}{huai2yi2}[][HSK 4]
    \definition{v.}{duvidar; suspeitar | supor}
  \end{phonetics}
\end{entry}

\begin{entry}{我}{7}{⼽}
  \begin{phonetics}{我}{wo3}[][HSK 1]
    \definition{pron.}{eu; mim | um; qualquer um; usado para contrastar 他 e 我; refere-se a muitas pessoas em geral}
  \seealsoref{他}{ta1}
  \end{phonetics}
\end{entry}

\begin{entry}{我们}{7,5}{⼽、⼈}
  \begin{phonetics}{我们}{wo3men5}[][HSK 1]
    \definition{pron.}{nós; nos}
  \end{phonetics}
\end{entry}

\begin{entry}{我们的}{7,5,8}{⼽、⼈、⽩}
  \begin{phonetics}{我们的}{wo3men5 de5}
    \definition{pron.}{nosso, nossos}
  \end{phonetics}
\end{entry}

\begin{entry}{我去}{7,5}{⼽、⼛}
  \begin{phonetics}{我去}{wo3qu4}
    \definition{interj.}{(gíria) O que\dots!! | Oh meu Deus! | Isso é insano!}
  \end{phonetics}
\end{entry}

\begin{entry}{我的}{7,8}{⼽、⽩}
  \begin{phonetics}{我的}{wo3 de5}
    \definition{pron.}{meu, meus}
  \end{phonetics}
\end{entry}

\begin{entry}{戒}{7}{⼽}
  \begin{phonetics}{戒}{jie4}[][HSK 5]
    \definition[个]{s.}{advertência; exortação | disciplina monástica budista; preceitos budistas | anel (dedo)}
    \definition{v.}{proteger-se contra; estar preparado; estar atento | advertir; exortar; admoestar | abandonar; parar; desistir; desistir (de um hábito ruim)}
  \end{phonetics}
\end{entry}

\begin{entry}{扮演}{7,14}{⼿、⽔}
  \begin{phonetics}{扮演}{ban4yan3}[][HSK 5]
    \definition{v.}{desempenhar o papel de; ter um papel (em uma peça, etc.); atuar}
  \end{phonetics}
\end{entry}

\begin{entry}{扶}{7}{⼿}
  \begin{phonetics}{扶}{fu2}[][HSK 5]
    \definition*{s.}{sobrenome Fu}
    \definition{v.}{segurar; apoiar com a mão; segurar algo com o apoio das mãos para que ninguém, objeto ou pessoa caia | dar apoio a; ajudar uma pessoa deitada ou caída a se levantar com as mãos; endireitar um objeto caído com as mãos | ajudar; tirar de baixo}
  \end{phonetics}
\end{entry}

\begin{entry}{扶梯}{7,11}{⼿、⽊}
  \begin{phonetics}{扶梯}{fu2ti1}
    \definition{s.}{escada rolante}
  \end{phonetics}
\end{entry}

\begin{entry}{批}{7}{⼿}
  \begin{phonetics}{批}{pi1}[][HSK 4]
    \definition{adj.}{(compra ou venda) atacado; a granel; em grandes quantidades}
    \definition{clas.}{para mercadorias a granel, grande número de pessoas}
    \definition{s.}{fibras de algodão, linho, etc., prontas para serem estiradas e torcidas | anotação; comentário}
    \definition{v.}{escrever comentários ou críticas sobre documentos subordinados, textos de outras pessoas, tarefas etc. | refutar; criticar | dar um tapa}
  \end{phonetics}
\end{entry}

\begin{entry}{批评}{7,7}{⼿、⾔}
  \begin{phonetics}{批评}{pi1ping2}[][HSK 3]
    \definition{s.}{crítica}
    \definition{v.}{criticar; comentar sobre}
  \end{phonetics}
\end{entry}

\begin{entry}{批准}{7,10}{⼿、⼎}
  \begin{phonetics}{批准}{pi1zhun3}[][HSK 3]
    \definition{v.}{aprovar}
  \end{phonetics}
\end{entry}

\begin{entry}{找}{7}{⼿}
  \begin{phonetics}{找}{zhao3}[][HSK 1]
    \definition{v.}{procurar; tentar encontrar; buscar | querer ver; visitar; abordar; solicitar | dar troco | descobrir; esforçar-se para ver ou obter a pessoa ou coisa desejada | examinar; investigar; completar as partes que faltam | causar intencionalmente (um resultado indesejável, negativo)}
  \end{phonetics}
\end{entry}

\begin{entry}{找见}{7,4}{⼿、⾒}
  \begin{phonetics}{找见}{zhao3jian4}
    \definition{v.}{encontrar (algo que está procurando)}
  \end{phonetics}
\end{entry}

\begin{entry}{找出}{7,5}{⼿、⼐}
  \begin{phonetics}{找出}{zhao3 chu1}[][HSK 2]
    \definition{v.}{encontrar | procurar}
  \end{phonetics}
\end{entry}

\begin{entry}{找回}{7,6}{⼿、⼞}
  \begin{phonetics}{找回}{zhao3hui2}
    \definition{v.}{recuperar algo}
  \end{phonetics}
\end{entry}

\begin{entry}{找寻}{7,6}{⼿、⼨}
  \begin{phonetics}{找寻}{zhao3xun2}
    \definition{v.}{encontrar falhas | procurar | buscar}
  \end{phonetics}
\end{entry}

\begin{entry}{找事}{7,8}{⼿、⼅}
  \begin{phonetics}{找事}{zhao3shi4}
    \definition{v.}{procurar emprego | começar uma briga}
  \end{phonetics}
\end{entry}

\begin{entry}{找到}{7,8}{⼿、⼑}
  \begin{phonetics}{找到}{zhao3 dao4}[][HSK 1]
    \definition{v.}{encontrar; procurar; achar; encontar através de pesquisa, exploração, etc.;  ver ou encontrar coisas ou padrões que os antepassados não viram}
  \end{phonetics}
\end{entry}

\begin{entry}{找钱}{7,10}{⼿、⾦}
  \begin{phonetics}{找钱}{zhao3qian2}
    \definition{v.}{dar troco}
  \end{phonetics}
\end{entry}

\begin{entry}{找着}{7,11}{⼿、⽬}
  \begin{phonetics}{找着}{zhao3zhao2}
    \definition{v.}{encontrar}
  \end{phonetics}
\end{entry}

\begin{entry}{找遍}{7,12}{⼿、⾡}
  \begin{phonetics}{找遍}{zhao3bian4}
    \definition{v.}{pentear | pesquisar em todos os lugares}
  \end{phonetics}
\end{entry}

\begin{entry}{找零}{7,13}{⼿、⾬}
  \begin{phonetics}{找零}{zhao3ling2}
    \definition{v.}{trocar dinheiro | dar troco}
  \end{phonetics}
\end{entry}

\begin{entry}{找辙}{7,16}{⼿、⾞}
  \begin{phonetics}{找辙}{zhao3zhe2}
    \definition{v.}{procurar um pretexto}
  \end{phonetics}
\end{entry}

\begin{entry}{技巧}{7,5}{⼿、⼯}
  \begin{phonetics}{技巧}{ji4qiao3}[][HSK 4]
    \definition{s.}{habilidade; técnica; habilidades engenhosas expressas em artes, artesanato, esportes, etc.}
  \end{phonetics}
\end{entry}

\begin{entry}{技术}{7,5}{⼿、⽊}
  \begin{phonetics}{技术}{ji4shu4}[][HSK 3]
    \definition[种,门,项]{s.}{habilidade; técnica; tecnologia}
  \end{phonetics}
\end{entry}

\begin{entry}{技俩}{7,9}{⼿、⼈}
  \begin{phonetics}{技俩}{ji4liang3}
    \definition{s.}{truque | estratagema | ardil | esquema | estratégia | tática}
  \end{phonetics}
\end{entry}

\begin{entry}{技能}{7,10}{⼿、⾁}
  \begin{phonetics}{技能}{ji4 neng2}[][HSK 5]
    \definition[种,项]{s.}{habilidade técnica; domínio de uma habilidade ou técnica; capacidade de adquirir e aplicar conhecimento}
  \end{phonetics}
\end{entry}

\begin{entry}{抄}{7}{⼿}
  \begin{phonetics}{抄}{chao1}[][HSK 4]
    \definition*{s.}{sobrenome Chao}
    \definition{v.}{copiar; transcrever | plagiar | revistar e confiscar; fazer uma batida | pegar um atalho | dobrar (os braços) | agarrar; pegar}
  \end{phonetics}
\end{entry}

\begin{entry}{抄写}{7,5}{⼿、⼍}
  \begin{phonetics}{抄写}{chao1 xie3}[][HSK 4]
    \definition{v.}{copiar; transcrever}
  \end{phonetics}
\end{entry}

\begin{entry}{把}{7}{⼿}
  \begin{phonetics}{把}{ba3}[][HSK 3]
    \definition{clas.}{para objetos com alça | para objetos pequenos:~punhado}
    \definition{part.}{partícula tornando o substantivo seguinte um objeto direto}
    \definition{v.}{conter | alcançar | segurar}
  \end{phonetics}
  \begin{phonetics}{把}{ba4}
    \definition{v.}{lidar}
  \end{phonetics}
\end{entry}

\begin{entry}{把风}{7,4}{⼿、⾵}
  \begin{phonetics}{把风}{ba3feng1}
    \definition{v.}{estar atento | vigiar (durante uma atividade clandestina)}
  \end{phonetics}
\end{entry}

\begin{entry}{把关}{7,6}{⼿、⼋}
  \begin{phonetics}{把关}{ba3guan1}
    \definition{v.}{verificar estritamente | examinar cuidadosamente para ver se algo é feito de acordo com um padrão fixo | fazer a verificação final | guardar uma passagem, fronteira}
  \end{phonetics}
\end{entry}

\begin{entry}{把守}{7,6}{⼿、⼧}
  \begin{phonetics}{把守}{ba3shou3}
    \definition{v.}{vigiar | guardar}
  \end{phonetics}
\end{entry}

\begin{entry}{把式}{7,6}{⼿、⼷}
  \begin{phonetics}{把式}{ba3shi4}
    \definition{s.}{pessoa qualificada em um comércio}
  \end{phonetics}
\end{entry}

\begin{entry}{把戏}{7,6}{⼿、⼽}
  \begin{phonetics}{把戏}{ba3xi4}
    \definition{s.}{acrobacia | malabarismo | truque barato}
  \end{phonetics}
\end{entry}

\begin{entry}{把玩}{7,8}{⼿、⽟}
  \begin{phonetics}{把玩}{ba3wan2}
    \definition{v.}{brincar com | mexer com}
  \end{phonetics}
\end{entry}

\begin{entry}{把持}{7,9}{⼿、⼿}
  \begin{phonetics}{把持}{ba3chi2}
    \definition{v.}{controlar | dominar | monopolizar}
  \end{phonetics}
\end{entry}

\begin{entry}{把柄}{7,9}{⼿、⽊}
  \begin{phonetics}{把柄}{ba3bing3}
    \definition{s.}{(figurativo) informações que podem ser usadas contra alguém}
  \end{phonetics}
\end{entry}

\begin{entry}{把脉}{7,9}{⼿、⾁}
  \begin{phonetics}{把脉}{ba3mai4}
    \definition{v.}{sentir ou tomar o pulso de alguém}
  \end{phonetics}
\end{entry}

\begin{entry}{把握}{7,12}{⼿、⼿}
  \begin{phonetics}{把握}{ba3wo4}[][HSK 3]
    \definition{s.}{seguro | garantia | certeza}
    \definition{v.}{agarrar | segurar | aproveitar}
  \end{phonetics}
\end{entry}

\begin{entry}{把稳}{7,14}{⼿、⽲}
  \begin{phonetics}{把稳}{ba3wen3}
    \definition{adj.}{confiável}
  \end{phonetics}
\end{entry}

\begin{entry}{抓}{7}{⼿}
  \begin{phonetics}{抓}{zhua1}[][HSK 3]
    \definition{v.}{agarrar | arranhar | capturar | compreender; conhecer a chave ou a chave das coisas ou problemas | focar em algo; fortalecer o poder de fazer (algo) ou administrar (algum aspecto) | atrair a atenção de alguém}
  \end{phonetics}
\end{entry}

\begin{entry}{抓住}{7,7}{⼿、⼈}
  \begin{phonetics}{抓住}{zhua1 zhu4}[][HSK 3]
    \definition{v.}{apanhar; prender; capturar (uma pessoa ou animal) e ter sucesso | segurar; agarrar; segurar algo e deixá-lo imóvel}
  \end{phonetics}
\end{entry}

\begin{entry}{抓紧}{7,10}{⼿、⽷}
  \begin{phonetics}{抓紧}{zhua1jin3}[][HSK 4]
    \definition{v.}{agarrar com firmeza; segurar firme e não soltar | prestar muita atenção a}
  \end{phonetics}
\end{entry}

\begin{entry}{投}{7}{⼿}
  \begin{phonetics}{投}{tou2}[][HSK 4]
    \definition*{s.}{sobrenome Tou}
    \definition{pron.}{para; indica tempo, equivalente a 到, 临 | para; em direção a; indica orientação, direção, equivalente a 朝 ou 向}
    \definition{s.}{um jogo durante uma festa em que o vencedor era decidido pelo número de flechas lançadas em um pote distante | jogo de dados}
    \definition{v.}{lançar; arremessar; atirar | deixar cair; colocar em; lançar | mergulhar em; lançar-se em; pular dentro | lançar; projetar; sombrear | entregar; postar; enviar | ir até; ir para; buscar; juntar-se | sentir-se atraído por; adaptar-se a; concordar com; atender a}
  \seealsoref{朝}{chao2}
  \seealsoref{到}{dao4}
  \seealsoref{临}{lin2}
  \seealsoref{向}{xiang4}
  \end{phonetics}
\end{entry}

\begin{entry}{投入}{7,2}{⼿、⼊}
  \begin{phonetics}{投入}{tou2ru4}[][HSK 4]
    \definition{adj.}{sisudo; dedicado; devotado; absorto}
    \definition{s.}{investimento; insumo; refere-se à aplicação de recursos}
    \definition{v.}{lançar em; colocar em; jogar em; por em | entrar em uma situação; participar de | aplicar; investir; colocar fundos em}
  \end{phonetics}
\end{entry}

\begin{entry}{投诉}{7,7}{⼿、⾔}
  \begin{phonetics}{投诉}{tou2su4}[][HSK 4]
    \definition{v.}{reclamar; queixar-se; reclamar às autoridades ou pessoas envolvidas}
  \end{phonetics}
\end{entry}

\begin{entry}{投资}{7,10}{⼿、⾙}
  \begin{phonetics}{投资}{tou2zi1}[][HSK 4]
    \definition[次]{s.}{investimento}
    \definition{v.}{investir; aplicar dinheiro; investir dinheiro em negócios}
  \end{phonetics}
\end{entry}

\begin{entry}{投资人}{7,10,2}{⼿、⾙、⼈}
  \begin{phonetics}{投资人}{tou2zi1ren2}
    \definition{s.}{investidor}
  \seealsoref{投资家}{tou2zi1jia1}
  \seealsoref{投资者}{tou2zi1zhe3}
  \end{phonetics}
\end{entry}

\begin{entry}{投资风险}{7,10,4,9}{⼿、⾙、⾵、⾩}
  \begin{phonetics}{投资风险}{tou2zi1feng1xian3}
    \definition{s.}{risco de investimento}
  \end{phonetics}
\end{entry}

\begin{entry}{投资回报率}{7,10,6,7,11}{⼿、⾙、⼞、⼿、⽞}
  \begin{phonetics}{投资回报率}{tou2zi1hui2bao4lv4}
    \definition{s.}{retorno sobre o investimento (ROI)}
  \end{phonetics}
\end{entry}

\begin{entry}{投资者}{7,10,8}{⼿、⾙、⽼}
  \begin{phonetics}{投资者}{tou2zi1zhe3}
    \definition{s.}{investidor}
  \seealsoref{投资家}{tou2zi1jia1}
  \seealsoref{投资人}{tou2zi1ren2}
  \end{phonetics}
\end{entry}

\begin{entry}{投资家}{7,10,10}{⼿、⾙、⼧}
  \begin{phonetics}{投资家}{tou2zi1jia1}
    \definition{s.}{investidor}
  \seealsoref{投资人}{tou2zi1ren2}
  \seealsoref{投资者}{tou2zi1zhe3}
  \end{phonetics}
\end{entry}

\begin{entry}{投递}{7,10}{⼿、⾡}
  \begin{phonetics}{投递}{tou2di4}
    \definition{v.}{despachar | enviar}
  \end{phonetics}
\end{entry}

\begin{entry}{投票}{7,11}{⼿、⽰}
  \begin{phonetics}{投票}{tou2piao4}
    \definition{v.+compl.}{votar | depositar um voto}
  \end{phonetics}
\end{entry}

\begin{entry}{折}{7}{⼿}
  \begin{phonetics}{折}{she2}
    \definition{v.}{estalar; quebrar | perder dinheiro em um negócio}
  \end{phonetics}
  \begin{phonetics}{折}{zhe1}
    \definition{v.}{rolar; virar | despejar algo de um recipiente em outro; ficar despejando algo entre dois recipientes}
  \end{phonetics}
  \begin{phonetics}{折}{zhe2}[][HSK 4]
    \definition*{s.}{sobrenome Zhe}
    \definition{clas.}{uma passagem em um roteiro de ópera miscelânea de Yuan, aproximadamente equivalente a uma cena ou ato em uma ópera moderna}
    \definition[张,个,些]{s.}{fratura; quebra | abatimento; desconto | traços dos caracteres chineses que têm o formato de "𠃍" e "乚", etc. | pasta; livreto; \emph{folder}}
    \definition{v.}{estalar; quebrar; fazer quebrar | perder; sofrer a perda de | voltar para trás; mudar de direção; retornar |ser convencido; estar cheio de admiração | equivaler a; converter em | dobrar}
  \end{phonetics}
\end{entry}

\begin{entry}{折转}{7,8}{⼿、⾞}
  \begin{phonetics}{折转}{zhe2zhuan3}
    \definition{s.}{reflexo (ângulo)}
    \definition{v.}{voltar atrás}
  \end{phonetics}
\end{entry}

\begin{entry}{抢}{7}{⼿}
  \begin{phonetics}{抢}{qiang1}
    \definition{prep.}{contra; direção relativa inversa}
    \definition{v.}{bater; tocar}
  \end{phonetics}
  \begin{phonetics}{抢}{qiang3}[][HSK 5]
    \definition{v.}{roubar; saquear | agarrar; apanhar; arrebatar | disputar; lutar por; ser o primeiro; competir para ser o primeiro | correr; apressar-se; fazer uma incursão | raspar; arranhar; raspar ou esfregar uma camada da superfície de um objeto}
  \end{phonetics}
\end{entry}

\begin{entry}{抢掠}{7,11}{⼿、⼿}
  \begin{phonetics}{抢掠}{qiang3lve4}
    \definition{s.}{saque | pilhagem}
    \definition{v.}{saquear | pilhar}
  \end{phonetics}
\end{entry}

\begin{entry}{抢救}{7,11}{⼿、⽁}
  \begin{phonetics}{抢救}{qiang3jiu4}[][HSK 5]
    \definition{v.}{salvar; resgatar; prestar de socorro ou assistência rápidos em situações de emergência | salvar; tomar medidas rápidas para evitar ou minimizar perdas iminentes.}
  \end{phonetics}
\end{entry}

\begin{entry}{护士}{7,3}{⼿、⼠}
  \begin{phonetics}{护士}{hu4shi5}[][HSK 4]
    \definition[名,位]{s.}{enfermeiro; pessoas especializadas em enfermagem em hospitais ou instituições epidemiológicas}
  \end{phonetics}
\end{entry}

\begin{entry}{护照}{7,13}{⼿、⽕}
  \begin{phonetics}{护照}{hu4zhao4}[][HSK 2]
    \definition[本,个]{s.}{passaporte; documento emitido pela autoridade competente do país para comprovar a nacionalidade e a identidade dos cidadãos que viajam para o exterior}
  \end{phonetics}
\end{entry}

\begin{entry}{报}{7}{⼿}
  \begin{phonetics}{报}{bao4}[][HSK 3]
    \definition[份,张]{s.}{jornal | recompensa | relatório | vingança}
    \definition{v.}{anunciar | informar}
  \end{phonetics}
\end{entry}

\begin{entry}{报名}{7,6}{⼿、⼝}
  \begin{phonetics}{报名}{bao4ming2}[][HSK 2]
    \definition{v.+compl.}{inscrever-se; alistar-se; registrar seu nome; cadastrar-se; matricular-se; informar seu nome à pessoa responsável, órgão, grupo etc., indicando que você deseja participar de alguma atividade ou organização}
  \end{phonetics}
\end{entry}

\begin{entry}{报告}{7,7}{⼿、⼝}
  \begin{phonetics}{报告}{bao4gao4}[][HSK 3]
    \definition[份,篇,分,个,通]{s.}{relatório | discurso | palestra | aconselhamento}
    \definition{v.}{relatar | dar a conhecer | informar}
  \end{phonetics}
\end{entry}

\begin{entry}{报纸}{7,7}{⼿、⽷}
  \begin{phonetics}{报纸}{bao4zhi3}[][HSK 2]
    \definition[分,期,张]{s.}{jornal; publicações periódicas cujo conteúdo principal é notícias, geralmente referem-se a jornais diários | papel jornal; um tipo de papel usado para imprimir jornais ou publicações em geral}
  \end{phonetics}
\end{entry}

\begin{entry}{报到}{7,8}{⼿、⼑}
  \begin{phonetics}{报到}{bao4dao4}[][HSK 3]
    \definition{v.+compl.}{apresentar-se para o serviço | fazer check-in | registrar-se | assinar}
  \end{phonetics}
\end{entry}

\begin{entry}{报答}{7,12}{⼿、⽵}
  \begin{phonetics}{报答}{bao4da2}[][HSK 5]
    \definition{v.}{reembolsar; devolver; retribuir; pagar de volta; mostrar seu apreço de forma tangível}
  \end{phonetics}
\end{entry}

\begin{entry}{报道}{7,12}{⼿、⾡}
  \begin{phonetics}{报道}{bao4dao4}[][HSK 3]
    \definition[个,篇,分]{s.}{história | reportagem}
    \definition{v.}{cobrir | relatar (notícias)}
  \end{phonetics}
\end{entry}

\begin{entry}{报酬}{7,13}{⼿、⾣}
  \begin{phonetics}{报酬}{bao4chou5}
    \definition{s.}{recompensa | remuneração}
  \end{phonetics}
\end{entry}

\begin{entry}{报警}{7,19}{⼿、⾔}
  \begin{phonetics}{报警}{bao4jing3}[][HSK 5]
    \definition{v.}{relatar (um incidente) à polícia; relatar uma situação crítica ou sinalizar uma emergência às autoridades competentes}
  \end{phonetics}
\end{entry}

\begin{entry}{拒绝}{7,9}{⼿、⽷}
  \begin{phonetics}{拒绝}{ju4jue2}[][HSK 5]
    \definition{v.}{recusar; rejeitar; declinar; não aceitar (pedidos, sugestões ou presentes)}
  \end{phonetics}
\end{entry}

\begin{entry}{改}{7}{⽁}
  \begin{phonetics}{改}{gai3}[][HSK 2]
    \definition*{s.}{sobrenome Gai}
    \definition{v.}{mudar; converter; transformar; alterar; substituir | alterar; revisar; aperfeiçoar; modificar | corrigir; retificar; remediar; consertar}
  \end{phonetics}
\end{entry}

\begin{entry}{改正}{7,5}{⽁、⽌}
  \begin{phonetics}{改正}{gai3 zheng4}[][HSK 4]
    \definition{v.}{corrigir; emendar; mudar o errado para o correto}
  \end{phonetics}
\end{entry}

\begin{entry}{改良}{7,7}{⽁、⾉}
  \begin{phonetics}{改良}{gai3liang2}
    \definition{v.}{melhorar (algo) | reformar (um sistema)}
  \end{phonetics}
\end{entry}

\begin{entry}{改进}{7,7}{⽁、⾡}
  \begin{phonetics}{改进}{gai3jin4}[][HSK 3]
    \definition[个]{s.}{melhoria}
    \definition{v.}{aprimorar; aperfeiçoar; melhorar; tornar melhor
modificar}
  \end{phonetics}
\end{entry}

\begin{entry}{改变}{7,8}{⽁、⼜}
  \begin{phonetics}{改变}{gai3bian4}[][HSK 2]
    \definition{v.}{mudar; alterar; transformar; converter; moldar; modificar | causar mudanças; alterar}
  \end{phonetics}
\end{entry}

\begin{entry}{改革}{7,9}{⽁、⾰}
  \begin{phonetics}{改革}{gai3ge2}[][HSK 5]
    \definition[项,次,种]{s.}{reforma; reformação; iniciativas para aprimorar a inovação}
    \definition{v.}{reformar; transformar as antigas partes irracionais das coisas em novas que possam ser adaptadas à situação objetiva}
  \end{phonetics}
\end{entry}

\begin{entry}{改造}{7,10}{⽁、⾡}
  \begin{phonetics}{改造}{gai3 zao4}[][HSK 3]
    \definition{v.}{transformar; renovar | remodelar}
  \end{phonetics}
\end{entry}

\begin{entry}{改善}{7,12}{⽁、⼝}
  \begin{phonetics}{改善}{gai3shan4}[][HSK 4]
    \definition{v.}{melhorar; amenizar; mudar a situação original para torná-la melhor}
  \end{phonetics}
\end{entry}

\begin{entry}{改善关系}{7,12,6,7}{⽁、⼝、⼋、⽷}
  \begin{phonetics}{改善关系}{gai3shan4guan1xi5}
    \definition{v.}{melhorar a relação}
  \end{phonetics}
\end{entry}

\begin{entry}{改善通讯}{7,12,10,5}{⽁、⼝、⾡、⾔}
  \begin{phonetics}{改善通讯}{gai3shan4tong1xun4}
    \definition{v.}{melhorar a comunicação}
  \end{phonetics}
\end{entry}

\begin{entry}{时}{7}{⽇}
  \begin{phonetics}{时}{shi2}[][HSK 3]
    \definition*{s.}{sobrenome Shi}
    \definition{adj.}{atual; presente | a tempo; feito a tempo}
    \definition{adv.}{de vez em quando; ocasionalmente; de ​​tempos em tempos | às vezes\dots às vezes\dots}
    \definition{clas.}{hora; horas}
    \definition{s.}{dias; tempos; longo período de tempo | tempo; tempo fixo | hora; hora do dia | temporada | chance; oportunidade | atualidade; presente | tempo verbal}
  \end{phonetics}
\end{entry}

\begin{entry}{时代}{7,5}{⽇、⼈}
  \begin{phonetics}{时代}{shi2dai4}[][HSK 3]
    \definition[个]{s.}{idade; era; tempos; época | um período na vida de alguém}
  \end{phonetics}
\end{entry}

\begin{entry}{时光}{7,6}{⽇、⼉}
  \begin{phonetics}{时光}{shi2guang1}[][HSK 5]
    \definition[台]{s.}{tempo; passagem do tempo | dias; horas; anos; épocas; períodos}
  \end{phonetics}
\end{entry}

\begin{entry}{时机}{7,6}{⽇、⽊}
  \begin{phonetics}{时机}{shi2ji1}[][HSK 5]
    \definition{s.}{oportunidade; momento oportuno}
  \end{phonetics}
\end{entry}

\begin{entry}{时时}{7,7}{⽇、⽇}
  \begin{phonetics}{时时}{shi2shi2}
    \definition{adv.}{muitas vezes | constantemente}
  \end{phonetics}
\end{entry}

\begin{entry}{时间}{7,7}{⽇、⾨}
  \begin{phonetics}{时间}{shi2jian1}[][HSK 1]
    \definition[段]{s.}{tempo; refere-se à forma de existência do movimento da matéria, um sistema contínuo composto pelo passado, presente e futuro | tempo; período (duração); um período de tempo com início e fim | tempo (um ponto); em algum momento do tempo}
  \end{phonetics}
\end{entry}

\begin{entry}{时事}{7,8}{⽇、⼅}
  \begin{phonetics}{时事}{shi2shi4}[][HSK 5]
    \definition{s.}{acontecimentos atuais; assuntos atuais; eventos atuais | tendências atuais | como as coisas estão indo | a situação atual}
  \end{phonetics}
\end{entry}

\begin{entry}{时刻}{7,8}{⽇、⼑}
  \begin{phonetics}{时刻}{shi2ke4}[][HSK 3]
    \definition{adv.}{constantemente; sempre}
    \definition[个,段]{s.}{tempo; hora; momento; conjuntura}
  \end{phonetics}
\end{entry}

\begin{entry}{时差}{7,9}{⽇、⼯}
  \begin{phonetics}{时差}{shi2cha1}
    \definition{s.}{diferença de tempo | \emph{jet lag}}
  \end{phonetics}
\end{entry}

\begin{entry}{时候}{7,10}{⽇、⼈}
  \begin{phonetics}{时候}{shi2hou5}[][HSK 1]
    \definition[个]{s.}{(um ponto no) tempo; momento; um determinado momento no tempo | (a duração do) tempo; um período de tempo com início e fim}
  \end{phonetics}
\end{entry}

\begin{entry}{时常}{7,11}{⽇、⼱}
  \begin{phonetics}{时常}{shi2chang2}[][HSK 5]
    \definition{adv.}{frequentemente; com frequência}
  \end{phonetics}
\end{entry}

\begin{entry}{旷野}{7,11}{⽇、⾥}
  \begin{phonetics}{旷野}{kuang4ye3}
    \definition{s.}{região selvagem}
  \end{phonetics}
\end{entry}

\begin{entry}{更}{7}{⽈}
  \begin{phonetics}{更}{geng1}
    \definition*{s.}{sobrenome Geng}
    \definition{clas.}{um dos cinco períodos de duas horas em que a noite era anteriormente dividida; vigília; antigamente, a noite era dividida em cinco turnos, cada um com aproximadamente duas horas de duração}
    \definition{v.}{alterar; substituir | experimentar}
  \end{phonetics}
  \begin{phonetics}{更}{geng4}[][HSK 2]
    \definition{adv.}{mais; ainda mais | além disso; além do mais; ainda mais}
  \end{phonetics}
\end{entry}

\begin{entry}{更加}{7,5}{⽈、⼒}
  \begin{phonetics}{更加}{geng4 jia1}[][HSK 3]
    \definition{adv.}{mais; ainda mais; em maior grau}
  \end{phonetics}
\end{entry}

\begin{entry}{更换}{7,10}{⽈、⼿}
  \begin{phonetics}{更换}{geng1 huan4}[][HSK 5]
    \definition{v.}{alterar; mudar; substituir; comutar}
  \end{phonetics}
\end{entry}

\begin{entry}{更新}{7,13}{⽈、⽄}
  \begin{phonetics}{更新}{geng1xin1}[][HSK 5]
    \definition{v.}{renovar; atualizar; substituir; remover o antigo e substituir pelo novo}
  \end{phonetics}
\end{entry}

\begin{entry}{李}{7}{⽊}
  \begin{phonetics}{李}{li3}
    \definition*{s.}{sobrenome Li}
    \definition{s.}{ameixa}
  \end{phonetics}
\end{entry}

\begin{entry}{李子}{7,3}{⽊、⼦}
  \begin{phonetics}{李子}{li3zi5}
    \definition[个]{s.}{ameixa}
  \end{phonetics}
\end{entry}

\begin{entry}{李四}{7,5}{⽊、⼞}
  \begin{phonetics}{李四}{li3si4}
    \definition*{s.}{Li Si | Zé Ninguém | nome para uma pessoa não especificada, 2 de 3}
  \seealsoref{王五}{wang2wu3}
  \seealsoref{张三}{zhang1san1}
  \end{phonetics}
\end{entry}

\begin{entry}{材料}{7,10}{⽊、⽃}
  \begin{phonetics}{材料}{cai2liao4}[][HSK 4]
    \definition[份,个,种]{s.}{material; algo para fazer um produto acabado | material (figura de linguagem) | dados; material para estudo, pesquisa, etc.; conteúdo de uma obra}
  \end{phonetics}
\end{entry}

\begin{entry}{村}{7}{⽊}
  \begin{phonetics}{村}{cun1}[][HSK 3]
    \definition{adj.}{rústico; grosseiro}
    \definition{s.}{aldeia; vila}
  \end{phonetics}
\end{entry}

\begin{entry}{杜宇}{7,6}{⽊、⼧}
  \begin{phonetics}{杜宇}{du4yu3}
    \definition{s.}{cuco (pássaro)}
  \seealsoref{布谷鸟}{bu4gu3niao3}
  \seealsoref{杜鹃}{du4juan1}
  \seealsoref{杜鹃鸟}{du4juan1niao3}
  \end{phonetics}
\end{entry}

\begin{entry}{杜鹃}{7,12}{⽊、⿃}
  \begin{phonetics}{杜鹃}{du4juan1}
    \definition{s.}{cuco (pássaro)}
  \seealsoref{布谷鸟}{bu4gu3niao3}
  \seealsoref{杜鹃鸟}{du4juan1niao3}
  \seealsoref{杜宇}{du4yu3}
  \end{phonetics}
\end{entry}

\begin{entry}{杜鹃鸟}{7,12,5}{⽊、⿃、⿃}
  \begin{phonetics}{杜鹃鸟}{du4juan1niao3}
    \definition{s.}{cuco (pássaro)}
  \seealsoref{布谷鸟}{bu4gu3niao3}
  \seealsoref{杜鹃}{du4juan1}
  \seealsoref{杜宇}{du4yu3}
  \end{phonetics}
\end{entry}

\begin{entry}{束}{7}{⽊}
  \begin{phonetics}{束}{shu4}[][HSK 3]
    \definition*{s.}{sobrenome Shu}
    \definition{clas.}{para cachos, molhos, feixes, feixes de luz, etc.}
    \definition{s.}{monte; pacote; maço; feixe; cacho}
    \definition{v.}{atar; amarrar; vincular | controlar; restringir}
  \end{phonetics}
\end{entry}

\begin{entry}{束腰}{7,13}{⽊、⾁}
  \begin{phonetics}{束腰}{shu4yao1}
    \definition{s.}{cinto | cinta | cinturão}
  \end{phonetics}
\end{entry}

\begin{entry}{杠}{7}{⽊}
  \begin{phonetics}{杠}{gang1}
    \definition{s.}{mastro de bandeira | poste | passarela}
  \end{phonetics}
  \begin{phonetics}{杠}{gang4}
    \definition{s.}{vara grossa | barra | linha grossa | padrão, critério | hífen, traço}
    \definition{v.}{marcar com uma linha grossa | afiar (faca, navalha, etc.)}
  \end{phonetics}
\end{entry}

\begin{entry}{条}{7}{⽊}
  \begin{phonetics}{条}{tiao2}[][HSK 2]
    \definition*{s.}{sobrenome Tiao}
    \definition{clas.}{usado para objetos longos e finos; usado para sintetizar certas coisas longas e retangulares em quantidades fixas | usado para itemização | aplicado ao corpo humano}
    \definition{s.}{galho; galhos finos e longos | tira; faixa | item; artigo | ordem; método | nota; anotação em papel}
  \end{phonetics}
\end{entry}

\begin{entry}{条目}{7,5}{⽊、⽬}
  \begin{phonetics}{条目}{tiao2mu4}
    \definition{s.}{cláusulas e subcláusulas (em documento formal) | verbete (em um dicionário, enciclopédia, etc.)}
  \end{phonetics}
\end{entry}

\begin{entry}{条件}{7,6}{⽊、⼈}
  \begin{phonetics}{条件}{tiao2jian4}[][HSK 2]
    \definition[个,项,些]{s.}{condição; termo; fator; fatores que restringem a ocorrência, existência ou desenvolvimento das coisas | requisito; pré-requisito; qualificação; requisitos ou padrões estabelecidos para determinadas coisas | situação; estado; condição}
  \end{phonetics}
\end{entry}

\begin{entry}{条例}{7,8}{⽊、⼈}
  \begin{phonetics}{条例}{tiao2li4}
    \definition{s.}{código de conduta | ordenanças | regulamentos | regras | estatutos}
  \end{phonetics}
\end{entry}

\begin{entry}{条贯}{7,8}{⽊、⾙}
  \begin{phonetics}{条贯}{tiao2guan4}
    \definition{s.}{ordem | procedimentos | sequência | sistema}
  \end{phonetics}
\end{entry}

\begin{entry}{条幅}{7,12}{⽊、⼱}
  \begin{phonetics}{条幅}{tiao2fu2}
    \definition{s.}{faixa | banner | pergaminho de parede (para pintura ou caligrafia)}
  \end{phonetics}
\end{entry}

\begin{entry}{来}{7}{⽊}
  \begin{phonetics}{来}{lai2}[][HSK 1]
    \definition*{s.}{sobrenome Lai}
    \definition{part.}{usado após uma palavra numérica ou de quantidade; indica uma quantidade aproximada | usado depois de numerais como 一, 二, 三; para listar razões ou fatos, etc.}
    \definition{s.}{usado após uma expressão de tempo para indicar uma duração que vai do passado ao presente}
    \definition{v.}{vir; chegar; de outro lugar para o lugar onde o interlocutor se encontra | aparecer; acontecer; vir; (problemas, coisas, etc.) ocorrerem; surgirem | substitui um verbo com significado específico, indicando a realização de uma ação específica | estar indo para; usado antes de outro verbo, indica que algo será feito | vir para fazer algo; usado após outro verbo, indica que se vai fazer algo | usado para indicar um propósito; expressar o objetivo, fazer algo usando o método, a atitude ou a direção anteriores | usado com 得 ou 不 para indicar possibilidade, capacidade ou hábito}
  \seealsoref{不}{bu4}
  \seealsoref{得}{de5}
  \end{phonetics}
\end{entry}

\begin{entry}{来不及}{7,4,3}{⽊、⼀、⼃}
  \begin{phonetics}{来不及}{lai2bu5ji2}[][HSK 4]
    \definition{v.}{ser tarde demais; não ter tempo; não ter tempo suficiente (para fazer algo); não ser possível participar ou se atualizar devido a restrições de tempo}
  \end{phonetics}
\end{entry}

\begin{entry}{来自}{7,6}{⽊、⾃}
  \begin{phonetics}{来自}{lai2zi4}[][HSK 2]
    \definition{v.}{vir de (um local) | \emph{From:} (cabeçalho de \emph{e -mail})}
  \end{phonetics}
\end{entry}

\begin{entry}{来到}{7,8}{⽊、⼑}
  \begin{phonetics}{来到}{lai2 dao4}[][HSK 1]
    \definition{v.}{chegar; vir}
  \end{phonetics}
\end{entry}

\begin{entry}{来信}{7,9}{⽊、⼈}
  \begin{phonetics}{来信}{lai2 xin4}[][HSK 5]
    \definition{s.}{sua carta; carta recebida; carta ao interlocutor}
    \definition{v.}{enviar uma carta para aqui; enviar uma carta para o remetente}
  \end{phonetics}
\end{entry}

\begin{entry}{来得及}{7,11,3}{⽊、⼻、⼃}
  \begin{phonetics}{来得及}{lai2de5ji2}[][HSK 4]
    \definition{v.}{ainda ter tempo; ser capaz de fazê-lo; ser capaz de fazer algo a tempo; ainda ter tempo de chegar lá ou de se atualizar}
  \end{phonetics}
\end{entry}

\begin{entry}{来源}{7,13}{⽊、⽔}
  \begin{phonetics}{来源}{lai2yuan2}[][HSK 4]
    \definition{s.}{origem; causa; fonte; tabula rasa (ou seja, o lugar de onde as coisas vêm)}
    \definition{v.}{originar-se; surgir; ter origem; (algo) originar (seguido de 于)}
  \seealsoref{于}{yu2}
  \end{phonetics}
\end{entry}

\begin{entry}{极}{7}{⽊}
  \begin{phonetics}{极}{ji2}[][HSK 4]
    \definition*{s.}{sobrenome Ji}
    \definition{adj.}{máximo; extremo; final; supremo}
    \definition{adv.}{extremamente; excessivamente}
    \definition{s.}{o ponto máximo, mais alto; extremo; ápice; ponto culminante |
pólo; as extremidades norte e sul da Terra; as extremidades de um ímã; a extremidade de uma fonte de alimentação ou de um aparelho elétrico onde a corrente entra ou sai do aparelho}
    \definition{v.}{chegar ao fim de; levar a extremos}
  \end{phonetics}
\end{entry}

\begin{entry}{……极了}{7,2}{⽊、⼅}
  \begin{phonetics}{……极了}{ji2le5}[][HSK 3]
    \definition{expr.}{extremamente}
  \end{phonetics}
\end{entry}

\begin{entry}{极其}{7,8}{⽊、⼋}
  \begin{phonetics}{极其}{ji2qi2}[][HSK 4]
    \definition{adv.}{mais; extremamente; excessivamente}
  \end{phonetics}
\end{entry}

\begin{entry}{步}{7}{⽌}
  \begin{phonetics}{步}{bu4}[][HSK 3]
    \definition*{s.}{sobrenome Bu}
    \definition{clas.}{uma unidade antiga para medida de comprimento, equivalente a cinco chi}
    \definition{s.}{ritmo | passo | estágio | passo | condição | situação | estado}
    \definition{v.}{ir a pé | andar | pisar | contar passos}
  \end{phonetics}
\end{entry}

\begin{entry}{步行}{7,6}{⽌、⾏}
  \begin{phonetics}{步行}{bu4 xing2}[][HSK 4]
    \definition{v.}{caminhar; ir a pé; andar a pé (diferente de andar de carro, a cavalo, etc.)}
  \end{phonetics}
\end{entry}

\begin{entry}{每}{7}{⽏}
  \begin{phonetics}{每}{mei3}[][HSK 3]
    \definition*{s.}{sobrenome Mei}
    \definition{adv.}{frequentemente; todo}
    \definition{pron.}{cada; cada um; cada qual;  todo}
  \end{phonetics}
\end{entry}

\begin{entry}{每个人}{7,3,2}{⽏、⼈、⼈}
  \begin{phonetics}{每个人}{mei3ge5ren2}
    \definition{pron.}{todo mundo | todos}
  \end{phonetics}
\end{entry}

\begin{entry}{每天}{7,4}{⽏、⼤}
  \begin{phonetics}{每天}{mei3tian1}
    \definition{adv.}{todo dia | cada dia}
  \end{phonetics}
\end{entry}

\begin{entry}{每次}{7,6}{⽏、⽋}
  \begin{phonetics}{每次}{mei3ci4}
    \definition{adv.}{toda vez | cada vez}
  \end{phonetics}
\end{entry}

\begin{entry}{求}{7}{⽔}
  \begin{phonetics}{求}{qiu2}[][HSK 2]
    \definition*{s.}{sobrenome Qiu}
    \definition{v.}{implorar; solicitar; suplicar; rogar | lutar por; buscar; investigar | tentar; procurar; tentar obter | demandar}
  \end{phonetics}
\end{entry}

\begin{entry}{汹涌}{7,10}{⽔、⽔}
  \begin{phonetics}{汹涌}{xiong1yong3}
    \definition{adj.}{turbulento}
    \definition{v.}{aumentar ou emergir violentamente (oceano, rio, lago, etc.)}
  \end{phonetics}
\end{entry}

\begin{entry}{汽水}{7,4}{⽔、⽔}
  \begin{phonetics}{汽水}{qi4 shui3}[][HSK 4]
    \definition[罐,瓶]{s.}{refrigerante; refrigerante gaseificado; bebida refrescante, feita com a pressão de dióxido de carbono para dissolver na água e adicionar açúcar, suco de frutas, especiarias etc.}
  \end{phonetics}
\end{entry}

\begin{entry}{汽车}{7,4}{⽔、⾞}
  \begin{phonetics}{汽车}{qi4 che1}[][HSK 1]
    \definition[辆,种,款]{s.}{automóvel; carro; veículo motorizado; veículo movido a motor de combustão interna, que circula principalmente em rodovias ou ruas, geralmente com quatro ou mais pneus de borracha, usado para transportar pessoas ou mercadorias}
  \end{phonetics}
\end{entry}

\begin{entry}{汽油}{7,8}{⽔、⽔}
  \begin{phonetics}{汽油}{qi4you2}[][HSK 4]
    \definition{s.}{gasolina; mistura líquida de hidrocarbonetos com volatilidade e combustibilidade, que é usada como combustível a partir do fracionamento ou craqueamento do petróleo}
  \end{phonetics}
\end{entry}

\begin{entry}{沉}{7}{⽔}
  \begin{phonetics}{沉}{chen2}[][HSK 4]
    \definition{adj.}{profundo | pesado | pesado (sentir-se pesado)}
    \definition{v.}{afundar; submergir; imergir | manter baixo; abaixar | descansar; parar}
  \end{phonetics}
\end{entry}

\begin{entry}{沉重}{7,9}{⽔、⾥}
  \begin{phonetics}{沉重}{chen2zhong4}[][HSK 4]
    \definition{adj.}{(pressão, fardo, etc.) muito pesado; profundo | sério; pesado; humor pouco animador; fardo pesado de pensamentos}
  \end{phonetics}
\end{entry}

\begin{entry}{沉默}{7,16}{⽔、⿊}
  \begin{phonetics}{沉默}{chen2mo4}[][HSK 4]
    \definition{adj.}{silencioso; reticente; taciturno; não comunicativo}
    \definition{v.}{silenciar; não falar por causa de alguma coisa}
  \end{phonetics}
\end{entry}

\begin{entry}{沙}{7}{⽔}
  \begin{phonetics}{沙}{sha1}
    \definition*{s.}{sobrenome Sha}
    \definition[粒]{s.}{areia | cascalho | grânulo | pó}
  \end{phonetics}
\end{entry}

\begin{entry}{沙子}{7,3}{⽔、⼦}
  \begin{phonetics}{沙子}{sha1 zi5}[][HSK 3]
    \definition[粒,把]{s.}{areia; grão | \emph{pellets}; grãos pequenos}
  \end{phonetics}
\end{entry}

\begin{entry}{沙发}{7,5}{⽔、⼜}
  \begin{phonetics}{沙发}{sha1fa1}[][HSK 3]
    \definition[套,组,个,张]{s.}{sofá; divã}
  \end{phonetics}
\end{entry}

\begin{entry}{沙鱼}{7,8}{⽔、⿂}
  \begin{phonetics}{沙鱼}{sha1yu2}
    \variantof{鲨鱼}
  \end{phonetics}
\end{entry}

\begin{entry}{沙特}{7,10}{⽔、⽜}
  \begin{phonetics}{沙特}{sha1te4}
    \definition*{s.}{Saudita | abreviação de 沙特阿拉伯}
  \seealsoref{沙特阿拉伯}{sha1te4 a1la1bo2}
  \end{phonetics}
\end{entry}

\begin{entry}{沙特阿拉伯}{7,10,7,8,7}{⽔、⽜、⾩、⼿、⼈}
  \begin{phonetics}{沙特阿拉伯}{sha1te4 a1la1bo2}
    \definition*{s.}{Arábia Saudita}
  \end{phonetics}
\end{entry}

\begin{entry}{沙漠}{7,13}{⽔、⽔}
  \begin{phonetics}{沙漠}{sha1mo4}[][HSK 5]
    \definition[个]{s.}{deserto; superfície totalmente coberta por areia, sem água corrente, clima seco e vegetação escassa}
  \end{phonetics}
\end{entry}

\begin{entry}{沟}{7}{⽔}
  \begin{phonetics}{沟}{gou1}[][HSK 5]
    \definition[条]{s.}{canal; vala; sarjeta; trincheira; cursos d'água ou fortificações escavados | ranhura; sulco raso; uma depressão que se assemelha a uma vala | ravina; barranco; cursos d'água}
  \end{phonetics}
\end{entry}

\begin{entry}{沟通}{7,10}{⽔、⾡}
  \begin{phonetics}{沟通}{gou1tong1}[][HSK 5]
    \definition{v.}{comunicar; comunicar-se para entender as ideias, opiniões, etc. | conectar; ligar; estabelecer um paralelo entre os dois}
  \end{phonetics}
\end{entry}

\begin{entry}{没}{7}{⽔}
  \begin{phonetics}{没}{mei2}[][HSK 1]
    \definition{adv.}{não; nunca; negar que uma ação ou situação tenha ocorrido, com o significado de 不曾}
    \definition{pref.}{não (prefixo negativo para verbos, traduzido para outras línguas com verbos no pretérito)}
    \definition{v.}{não possuir; não ter | não existe; não há | ninguém; usado antes de 谁, 什么, 哪个, significa 全都不 | não ser tão bom quanto; ser inferior a; não chega a; não é tão bom quanto | menor que; insuficiente}
  \seealsoref{不曾}{bu4 ceng2}
  \seealsoref{哪个}{na3ge5}
  \seealsoref{全都不}{quan2dou1 bu4}
  \seealsoref{谁}{shei2}
  \seealsoref{什么}{shen2me5}
  \end{phonetics}
  \begin{phonetics}{没}{mo4}
    \definition{adj.}{último; final}
    \definition{v.}{afundar na água; submergir | transbordar; subir além; exceder ou ultrapassar | esconder-se; desaparecer; sumir; ocultar-se | confiscar; expropriar | morrer}
    \variantof{没}
  \end{phonetics}
\end{entry}

\begin{entry}{没了}{7,2}{⽔、⼅}
  \begin{phonetics}{没了}{mei2le5}
    \definition{v.}{estar morto | deixar de existir}
  \end{phonetics}
\end{entry}

\begin{entry}{没什么}{7,4,3}{⽔、⼈、⼃}
  \begin{phonetics}{没什么}{mei2 shen2 me5}[][HSK 1]
    \definition{expr.}{não é nada; está tudo bem; não importa}
  \end{phonetics}
\end{entry}

\begin{entry}{没用}{7,5}{⽔、⽤}
  \begin{phonetics}{没用}{mei2 yong4}[][HSK 3]
    \definition{adj.}{inútil; imprestável; sem valor; sem préstimo; vão; que não serve para nada}
  \end{phonetics}
\end{entry}

\begin{entry}{没关系}{7,6,7}{⽔、⼋、⽷}
  \begin{phonetics}{没关系}{mei2guan1xi5}[][HSK 1]
    \definition{v.}{está tudo bem; não é nada; não importa; não se preocupe}
  \seealsoref{没有关系}{mei2you3guan1xi5}
  \end{phonetics}
\end{entry}

\begin{entry}{没有}{7,6}{⽔、⽉}
  \begin{phonetics}{没有}{mei2 you3}[][HSK 1]
    \definition{adv.}{ainda não; (usado com o pretérito) não; ação ou estado negativo ocorreu}
    \definition{v.}{não há; não tem; não existe}
  \end{phonetics}
\end{entry}

\begin{entry}{没有关系}{7,6,6,7}{⽔、⽉、⼋、⽷}
  \begin{phonetics}{没有关系}{mei2you3guan1xi5}
    \definition{v.}{não ter problema | não ter importância | não fazer mal}
  \seealsoref{没关系}{mei2guan1xi5}
  \end{phonetics}
\end{entry}

\begin{entry}{没有次序}{7,6,6,7}{⽔、⽉、⽋、⼴}
  \begin{phonetics}{没有次序}{mei2you3 ci4xu4}
    \definition{adj.}{sem ordem; nenhuma ordem}
  \end{phonetics}
\end{entry}

\begin{entry}{没有意思}{7,6,13,9}{⽔、⽉、⼼、⼼}
  \begin{phonetics}{没有意思}{mei2you3yi4si5}
    \definition{adj.}{tedioso | chato | sem interesse}
  \end{phonetics}
\end{entry}

\begin{entry}{没事儿}{7,8,2}{⽔、⼅、⼉}
  \begin{phonetics}{没事儿}{mei2 shi4r5}[][HSK 1]
    \definition{expr.}{fora de perigo; nada sério | não importa; não é nada; está tudo bem; não importa | está tudo bem; sem problemas; não se preocupe com isso; não é grande coisa; não há nada errado}
    \definition{v.}{não ter nada para fazer; ser livre; estar perdido | estar desempregado; estar sem trabalho | não ter responsabilidade}
  \end{phonetics}
\end{entry}

\begin{entry}{没法儿}{7,8,2}{⽔、⽔、⼉}
  \begin{phonetics}{没法儿}{mei2 fa3r5}[][HSK 4]
    \definition{adv.}{não pode; sem chance}
  \end{phonetics}
\end{entry}

\begin{entry}{没想到}{7,13,8}{⽔、⼼、⼑}
  \begin{phonetics}{没想到}{mei2 xiang3 dao4}[][HSK 4]
    \definition{expr.}{não esperava; inesperado}
  \end{phonetics}
\end{entry}

\begin{entry}{没错}{7,13}{⽔、⾦}
  \begin{phonetics}{没错}{mei2 cuo4}[][HSK 4]
    \definition{adv.}{está certo; é isso mesmo; não há como errar}
  \end{phonetics}
\end{entry}

\begin{entry}{灵感}{7,13}{⽕、⼼}
  \begin{phonetics}{灵感}{ling2gan3}
    \definition{s.}{inspiração | explosão de criatividade em empreendimento científico ou artístico}
  \end{phonetics}
\end{entry}

\begin{entry}{灵魂}{7,13}{⽕、⿁}
  \begin{phonetics}{灵魂}{ling2hun2}
    \definition{s.}{alma | espírito}
  \end{phonetics}
\end{entry}

\begin{entry}{灶台}{7,5}{⽕、⼝}
  \begin{phonetics}{灶台}{zao4tai2}
    \definition{s.}{fogão}
  \end{phonetics}
\end{entry}

\begin{entry}{灾}{7}{⽕}
  \begin{phonetics}{灾}{zai1}[][HSK 5]
    \definition[个,场]{s.}{calamidade; desastre | infortúnio pessoal; adversidade | azar}
  \end{phonetics}
\end{entry}

\begin{entry}{灾区}{7,4}{⽕、⼖}
  \begin{phonetics}{灾区}{zai1 qu1}[][HSK 5]
    \definition{s.}{área de desastre; área afetada por catástrofes}
  \end{phonetics}
\end{entry}

\begin{entry}{灾害}{7,10}{⽕、⼧}
  \begin{phonetics}{灾害}{zai1hai4}[][HSK 5]
    \definition[个]{s.}{desastre; calamidade; danos causados pela seca, inundações, pragas, granizo, guerras, etc.}
  \end{phonetics}
\end{entry}

\begin{entry}{灾难}{7,10}{⽕、⾫}
  \begin{phonetics}{灾难}{zai1nan4}[][HSK 5]
    \definition[场,次]{s.}{desastre; sofrimento; calamidade; catástrofe; danos e sofrimentos causados por desastres naturais ou guerras}
  \end{phonetics}
\end{entry}

\begin{entry}{状况}{7,7}{⽝、⼎}
  \begin{phonetics}{状况}{zhuang4kuang4}[][HSK 3]
    \definition[个,种]{s.}{estado; \emph{status}; condição; estado de coisas}
  \end{phonetics}
\end{entry}

\begin{entry}{状态}{7,8}{⽝、⼼}
  \begin{phonetics}{状态}{zhuang4tai4}[][HSK 3]
    \definition[种,个]{s.}{\emph{status}; estado; condição; estado de coisas; a forma em que uma pessoa ou coisa aparece}
  \end{phonetics}
\end{entry}

\begin{entry}{犹豫}{7,15}{⽝、⾗}
  \begin{phonetics}{犹豫}{you2yu4}[][HSK 5]
    \definition{adj.}{hesitante; indeciso, incapaz de decidir ou agir}
    \definition{v.}{hesitar; ser indeciso}
  \end{phonetics}
\end{entry}

\begin{entry}{狂}{7}{⽝}
  \begin{phonetics}{狂}{kuang2}[][HSK 5]
    \definition*{s.}{sobrenome Kuang}
    \definition{adj.}{louco; maluco | violento; selvagem | selvagem; delirante; furioso; desenfreado; desinibido; sem restrições | arrogante; autoritário}
  \end{phonetics}
\end{entry}

\begin{entry}{狂欢节}{7,6,5}{⽝、⽋、⾋}
  \begin{phonetics}{狂欢节}{kuang2huan1jie2}
    \definition*{s.}{Carnaval}
  \end{phonetics}
\end{entry}

\begin{entry}{男}{7}{⽥}
  \begin{phonetics}{男}{nan2}[][HSK 1]
    \definition{adj.}{homem; macho; masculino (em oposição a 女)}
    \definition[个,位]{s.}{filho; menino | homem | barão (o mais baixo de cinco ordens de nobreza)}
  \seealsoref{女}{nv3}
  \end{phonetics}
\end{entry}

\begin{entry}{男人}{7,2}{⽥、⼈}
  \begin{phonetics}{男人}{nan2 ren2}[][HSK 1]
    \definition[个]{s.}{homem adulto; macho; cavalheiro | marido}
  \end{phonetics}
\end{entry}

\begin{entry}{男士}{7,3}{⽥、⼠}
  \begin{phonetics}{男士}{nan2 shi4}[][HSK 4]
    \definition{s.}{cavalheiro; \emph{gentleman}}
  \end{phonetics}
\end{entry}

\begin{entry}{男女}{7,3}{⽥、⼥}
  \begin{phonetics}{男女}{nan2 nv3}[][HSK 4]
    \definition{s.}{homens e mulheres; masculino e feminino}
  \end{phonetics}
\end{entry}

\begin{entry}{男子}{7,3}{⽥、⼦}
  \begin{phonetics}{男子}{nan2zi3}[][HSK 3]
    \definition[名]{s.}{homem; macho}
  \end{phonetics}
\end{entry}

\begin{entry}{男生}{7,5}{⽥、⽣}
  \begin{phonetics}{男生}{nan2 sheng1}[][HSK 1]
    \definition[个]{s.}{menino; estudante; estudante do sexo masculino; aluno do sexo masculino}
  \end{phonetics}
\end{entry}

\begin{entry}{男性}{7,8}{⽥、⼼}
  \begin{phonetics}{男性}{nan2 xing4}[][HSK 5]
    \definition{s.}{masculino; homem; masculinidade}
  \end{phonetics}
\end{entry}

\begin{entry}{男朋友}{7,8,4}{⽥、⽉、⼜}
  \begin{phonetics}{男朋友}{nan2 peng2 you5}[][HSK 1]
    \definition{s.}{namorado}
  \end{phonetics}
\end{entry}

\begin{entry}{男孩儿}{7,9,2}{⽥、⼦、⼉}
  \begin{phonetics}{男孩儿}{nan2hai2r5}[][HSK 1]
    \definition{s.}{menino; rapaz}
  \end{phonetics}
\end{entry}

\begin{entry}{疗养}{7,9}{⽧、⼋}
  \begin{phonetics}{疗养}{liao2 yang3}[][HSK 4]
    \definition{v.}{recuperar; convalescer; tratar pessoas com doenças crônicas ou debilitantes em instituições médicas especializadas com foco na recuperação}
  \end{phonetics}
\end{entry}

\begin{entry}{社}{7}{⽰}
  \begin{phonetics}{社}{she4}[][HSK 5]
    \definition[个]{s.}{agência; sociedade; órgão organizado; organização; comunidade | comuna popular | o deus da terra, sacrifícios a ele ou altares para tais sacrifícios; na antiguidade, o deus da terra, o local onde ele era venerado, o dia da veneração e o ritual eram chamados de 社 | agência de notícias |  imprensa}
  \end{phonetics}
\end{entry}

\begin{entry}{社区}{7,4}{⽰、⼖}
  \begin{phonetics}{社区}{she4qu1}[][HSK 5]
    \definition{s.}{bairro; comunidade residencial; bairros da cidade, divididos de acordo com a localização geográfica | distrito; comunidade (para pessoas da mesma classe social, etc.) ; lugar onde pessoas com características comuns, como classe social, vivem juntas}
  \end{phonetics}
\end{entry}

\begin{entry}{社会}{7,6}{⽰、⼈}
  \begin{phonetics}{社会}{she4hui4}[][HSK 3]
    \definition[个,种]{s.}{sociedade | comunidade}
  \end{phonetics}
\end{entry}

\begin{entry}{私人}{7,2}{⽲、⼈}
  \begin{phonetics}{私人}{si1ren2}[][HSK 5]
    \definition{adj.}{privado; pertencente a um indivíduo ou exercido a título individual; não público | interpessoal}
    \definition[个]{s.}{algo privado; pessoas que se aproximam de você por motivos pessoais ou interesses próprios}
  \end{phonetics}
\end{entry}

\begin{entry}{私人诊所}{7,2,7,8}{⽲、⼈、⾔、⼾}
  \begin{phonetics}{私人诊所}{si1ren2 zhen3suo3}
    \definition[些]{s.}{clínica privada}
  \end{phonetics}
\end{entry}

\begin{entry}{私人信件}{7,2,9,6}{⽲、⼈、⼈、⼈}
  \begin{phonetics}{私人信件}{si1ren2 xin4jian4}
    \definition{s.}{carta pessoal}
  \end{phonetics}
\end{entry}

\begin{entry}{私人钥匙}{7,2,9,11}{⽲、⼈、⾦、⼔}
  \begin{phonetics}{私人钥匙}{si1ren2yao4shi5}
    \definition{s.}{(criptografia) chave privada}
  \end{phonetics}
\end{entry}

\begin{entry}{私生活}{7,5,9}{⽲、⽣、⽔}
  \begin{phonetics}{私生活}{si1sheng1huo2}
    \definition{s.}{vida privada}
  \end{phonetics}
\end{entry}

\begin{entry}{私自}{7,6}{⽲、⾃}
  \begin{phonetics}{私自}{si1zi4}
    \definition{adj.}{privado | pessoal}
    \definition{adv.}{secretamente | sem aprovação explícita}
  \end{phonetics}
\end{entry}

\begin{entry}{究竟}{7,11}{⽳、⾳}
  \begin{phonetics}{究竟}{jiu1jing4}[][HSK 4]
    \definition{adv.}{de fato; exatamente; usado em frases interrogativas para buscar | afinal de contas, no final; ênfase em fatos ou motivos}
    \definition{s.}{resultado; desfecho; a causa, o efeito ou a história completa do que aconteceu}
  \end{phonetics}
\end{entry}

\begin{entry}{穷}{7}{⽳}
  \begin{phonetics}{穷}{qiong2}[][HSK 4]
    \definition{adj.}{remoto; isolado; de difícil acesso | pobre; atingido pela pobreza | situação difícil, sem saída}
    \definition{adv.}{completamente | extremamente}
    \definition{v.}{exaurir; esgotar; consmir | ir até o fim; perseguir completamente perseguido; sondar profundamente | gastar}
  \end{phonetics}
\end{entry}

\begin{entry}{穷人}{7,2}{⽳、⼈}
  \begin{phonetics}{穷人}{qiong2 ren2}[][HSK 4]
    \definition{s.}{os pobres; pessoas pobres}
  \end{phonetics}
\end{entry}

\begin{entry}{系}{7}{⽷}
  \begin{phonetics}{系}{ji4}
    \definition{v.}{amarrar; prender; abotoar}
  \end{phonetics}
  \begin{phonetics}{系}{xi4}[][HSK 3,4]
    \definition*{s.}{sobrenome Xi}
    \definition{s.}{faculdade (da universidade) | departamento}
    \definition{v.}{sistema; série | departamento; faculdade}
    \definition{v.}{relacionar-se com; suportar; depender de | sentir-se ansioso; estar preocupado | amarrar; prender | ser}
  \end{phonetics}
\end{entry}

\begin{entry}{系囚}{7,5}{⽷、⼞}
  \begin{phonetics}{系囚}{xi4qiu2}
    \definition{s.}{prisioneiro}
  \end{phonetics}
\end{entry}

\begin{entry}{系列}{7,6}{⽷、⼑}
  \begin{phonetics}{系列}{xi4lie4}[][HSK 4]
    \definition{s.}{série; conjunto; conjunto de coisas relacionadas (matemática)}
  \end{phonetics}
\end{entry}

\begin{entry}{系统}{7,9}{⽷、⽷}
  \begin{phonetics}{系统}{xi4tong3}[][HSK 4]
    \definition{adj.}{sistemático; organizado}
    \definition[个]{s.}{sistema; relação de tipos semelhantes (ou seja, grupo de coisas semelhantes)}
  \end{phonetics}
\end{entry}

\begin{entry}{纯}{7}{⽷}
  \begin{phonetics}{纯}{chun2}[][HSK 4]
    \definition{adj.}{puro; não misturado; livre de impurezas | simples; puro e simples | habilidoso; proficiente; bem versado}
  \end{phonetics}
\end{entry}

\begin{entry}{纯净水}{7,8,4}{⽷、⼎、⽔}
  \begin{phonetics}{纯净水}{chun2 jing4 shui3}[][HSK 4]
    \definition{s.}{água purificada}
  \end{phonetics}
\end{entry}

\begin{entry}{纯真}{7,10}{⽷、⼗}
  \begin{phonetics}{纯真}{chun2zhen1}
    \definition{adj.}{inocente e não afetado | puro e não adulterado}
    \definition{s.}{inocência}
  \end{phonetics}
\end{entry}

\begin{entry}{纷纷}{7,7}{⽷、⽷}
  \begin{phonetics}{纷纷}{fen1fen1}[][HSK 4]
    \definition{adj.}{numeroso e confuso; muitos e desordenados}
    \definition{adv.}{um após o outro; em sucessão; em rápida sucessão}
  \end{phonetics}
\end{entry}

\begin{entry}{纸}{7}{⽷}
  \begin{phonetics}{纸}{zhi3}[][HSK 2]
    \definition{clas.}{para documentos, cartas, etc.}
    \definition[张,沓]{s.}{papel}
  \end{phonetics}
\end{entry}

\begin{entry}{纸巾}{7,3}{⽷、⼱}
  \begin{phonetics}{纸巾}{zhi3jin1}
    \definition[张,包]{s.}{lenço | guardanapo | papel toalha}
  \end{phonetics}
\end{entry}

\begin{entry}{纸币}{7,4}{⽷、⼱}
  \begin{phonetics}{纸币}{zhi3bi4}
    \definition[张]{s.}{nota (dinheiro) | cédula}
  \end{phonetics}
\end{entry}

\begin{entry}{纸尿裤}{7,7,12}{⽷、⼫、⾐}
  \begin{phonetics}{纸尿裤}{zhi3niao4ku4}
    \definition{s.}{fralda descartável}
  \end{phonetics}
\end{entry}

\begin{entry}{纸张}{7,7}{⽷、⼸}
  \begin{phonetics}{纸张}{zhi3zhang1}
    \definition{s.}{papel}
  \end{phonetics}
\end{entry}

\begin{entry}{纸烟}{7,10}{⽷、⽕}
  \begin{phonetics}{纸烟}{zhi3yan1}
    \definition{s.}{cigarro}
  \end{phonetics}
\end{entry}

\begin{entry}{纹路}{7,13}{⽷、⾜}
  \begin{phonetics}{纹路}{wen2lu4}
    \definition{s.}{padrão de linhas | rugas | veias | veias (em mármore ou impressão digital) | grãos (em madeira, etc.)}
  \end{phonetics}
\end{entry}

\begin{entry}{肚}{7}{⾁}
  \begin{phonetics}{肚}{du3}
    \definition{s.}{tripas | entranhas}
  \end{phonetics}
  \begin{phonetics}{肚}{du4}
    \definition{s.}{barriga}
  \end{phonetics}
\end{entry}

\begin{entry}{肚子}{7,3}{⾁、⼦}
  \begin{phonetics}{肚子}{du4zi5}[][HSK 4]
    \definition[个,只]{s.}{abdômen; barriguinha; ventre; barriga}
  \end{phonetics}
\end{entry}

\begin{entry}{肠}{7}{⾁}
  \begin{phonetics}{肠}{chang2}[][HSK 5]
    \definition{s.}{intestinos | salsicha; linguiça | coração; sentimentos; emoções}
  \end{phonetics}
\end{entry}

\begin{entry}{良心}{7,4}{⾉、⼼}
  \begin{phonetics}{良心}{liang2xin1}
    \definition{s.}{consciência}
  \end{phonetics}
\end{entry}

\begin{entry}{良田}{7,5}{⾉、⽥}
  \begin{phonetics}{良田}{liang2tian2}
    \definition{s.}{terra agrícola boa | terra fértil}
  \end{phonetics}
\end{entry}

\begin{entry}{良好}{7,6}{⾉、⼥}
  \begin{phonetics}{良好}{liang2hao3}[][HSK 4]
    \definition{adj.}{bom; ótimo; bem}
  \end{phonetics}
\end{entry}

\begin{entry}{芥}{7}{⾋}
  \begin{phonetics}{芥}{gai4}
    \definition{s.}{usado em 芥蓝 \dpy{gai4lan2}}
  \seealsoref{芥蓝}{gai4lan2}
  \end{phonetics}
  \begin{phonetics}{芥}{jie4}
    \definition{s.}{mostarda}
  \end{phonetics}
\end{entry}

\begin{entry}{芥兰}{7,5}{⾋、⼋}
  \begin{phonetics}{芥兰}{gai4lan2}
    \variantof{芥蓝}
  \end{phonetics}
  \begin{phonetics}{芥兰}{jie4lan2}
    \definition{s.}{couve}
  \end{phonetics}
\end{entry}

\begin{entry}{芥蓝}{7,13}{⾋、⾋}
  \begin{phonetics}{芥蓝}{gai4lan2}
    \definition{s.}{brócolis chinês | couve chinesa | mostarda}
  \seealsoref{格兰菜}{ge2lan2cai4}
  \end{phonetics}
\end{entry}

\begin{entry}{芦笋}{7,10}{⾋、⽵}
  \begin{phonetics}{芦笋}{lu2sun3}
    \definition{s.}{aspargos}
  \end{phonetics}
\end{entry}

\begin{entry}{芯片}{7,4}{⾋、⽚}
  \begin{phonetics}{芯片}{xin1pian4}
    \definition{s.}{chip de computador | microchip}
  \end{phonetics}
\end{entry}

\begin{entry}{花}{7}{⾋}
  \begin{phonetics}{花}{hua1}[][HSK 1,2,4]
    \definition*{s.}{sobrenome Hua}
    \definition{adj.}{multicolorido; colorido | embaçado; obscuro; deslumbrado e confuso | extravagante; florido; vistoso}
    \definition[朵,支,束,把,盆,簇]{s.}{flor; órgãos de reprodução sexual de plantas com sementes | flor; planta ornamental |  qualquer coisa que se assemelhe a uma flor | fogos de artifício | padrão; design; design decorativo | flor; metáfora para a essência de uma causa | prostituta; cortesã; referindo-se a prostitutas ou a assuntos relacionados a prostitutas | algodão | varíola | ferimento; ferida; lesões traumáticas sofridas em combate}
    \definition{v.}{gastar; despender; consumir}
  \end{phonetics}
\end{entry}

\begin{entry}{花儿}{7,2}{⾋、⼉}
  \begin{phonetics}{花儿}{hua1r5}
    \definition[朵,支,束,把,盆,簇]{s.}{flor}
  \end{phonetics}
\end{entry}

\begin{entry}{花生}{7,5}{⾋、⽣}
  \begin{phonetics}{花生}{hua1sheng1}
    \definition[粒]{s.}{amendoim}
  \end{phonetics}
\end{entry}

\begin{entry}{花园}{7,7}{⾋、⼞}
  \begin{phonetics}{花园}{hua1 yuan2}[][HSK 2]
    \definition[个,座]{s.}{jardim; um local onde se plantam flores e árvores para passear e descansar}
  \end{phonetics}
\end{entry}

\begin{entry}{花店}{7,8}{⾋、⼴}
  \begin{phonetics}{花店}{hua1dian4}
    \definition{s.}{floricultura}
  \end{phonetics}
\end{entry}

\begin{entry}{花茶}{7,9}{⾋、⾋}
  \begin{phonetics}{花茶}{hua1cha2}
    \definition[杯,壶]{s.}{chá perfumado}
  \end{phonetics}
\end{entry}

\begin{entry}{花样游泳}{7,10,12,8}{⾋、⽊、⽔、⽔}
  \begin{phonetics}{花样游泳}{hua1yang4you2yong3}
    \definition{s.}{nado sincronizado}
  \end{phonetics}
\end{entry}

\begin{entry}{花椰菜}{7,12,11}{⾋、⽊、⾋}
  \begin{phonetics}{花椰菜}{hua1ye1cai4}
    \definition{s.}{couve-flor}
  \end{phonetics}
\end{entry}

\begin{entry}{芹菜}{7,11}{⾋、⾋}
  \begin{phonetics}{芹菜}{qin2cai4}
    \definition{s.}{salsão}
  \end{phonetics}
\end{entry}

\begin{entry}{苏格兰}{7,10,5}{⾋、⽊、⼋}
  \begin{phonetics}{苏格兰}{su1ge2lan2}
    \definition*{s.}{Escócia}
  \end{phonetics}
\end{entry}

\begin{entry}{补}{7}{⾐}
  \begin{phonetics}{补}{bu3}[][HSK 3]
    \definition*{s.}{sobrenome Bu}
    \definition{s.}{benefício | ajuda | uso}
    \definition{v.}{consertar | remendar | preencher | adicionar suplemento | suprir | compensar |nutrir}
  \end{phonetics}
\end{entry}

\begin{entry}{补充}{7,6}{⾐、⼉}
  \begin{phonetics}{补充}{bu3chong1}[][HSK 3]
    \definition{adj.}{adicional | suplementar}
    \definition[个]{s.}{aditivo | suplemento}
    \definition{v.}{reabastecer | suplementar | complementar}
  \end{phonetics}
\end{entry}

\begin{entry}{补贴}{7,9}{⾐、⾙}
  \begin{phonetics}{补贴}{bu3tie1}[][HSK 5]
    \definition[笔,项,种,份]{s.}{subsídio; ajuda de custo; custos de indenização ou assistência concedida a empresas ou indivíduos pelo estado ou governo}
    \definition{v.}{subsidiar; compensar a falta de dinheiro ou coisas; refere-se principalmente à compensação financeira ou ajuda dada pelo estado ou governo a empresas ou indivíduos}
  \end{phonetics}
\end{entry}

\begin{entry}{补偿}{7,11}{⾐、⼈}
  \begin{phonetics}{补偿}{bu3chang2}[][HSK 5]
    \definition{v.}{compensar (perda, consumo); compensar (deficiências, diferenças)}
  \end{phonetics}
\end{entry}

\begin{entry}{角}{7}{⾓}
  \begin{phonetics}{角}{jiao3}[][HSK 2]
    \definition*{s.}{Jiao, uma das mansões lunares}
    \definition{clas.}{uma unidade monetária fracionária na China (=1/10 de um yuan ou 10 fen)}
    \definition[个,只,对]{s.}{chifre; o objeto duro que cresce na cabeça de bovinos, ovinos, veados, etc. | buzina; corneta; instrumentos musicais tocados no exército antigo | algo com a forma de um chifre | cabo; promontório; península | esquina; canto; a junção entre duas arestas de um objeto | ângulo}
  \end{phonetics}
  \begin{phonetics}{角}{jue2}
    \definition*{s.}{sobrenome Jue}
    \definition{s.}{papel (teatro)}
    \definition{v.}{competir}
  \end{phonetics}
\end{entry}

\begin{entry}{角色}{7,6}{⾓、⾊}
  \begin{phonetics}{角色}{jue2se4}[][HSK 4]
    \definition{s.}{papel; personagem em uma peça; personagem representado por um ator | papel; função; parte}
  \end{phonetics}
\end{entry}

\begin{entry}{角度}{7,9}{⾓、⼴}
  \begin{phonetics}{角度}{jiao3du4}[][HSK 2]
    \definition[个,种]{s.}{perspectiva; ponto de vista; o ponto de partida para ver as coisas | ângulo; o tamanho do ângulo; normalmente expresso em graus ou radianos}
  \end{phonetics}
\end{entry}

\begin{entry}{言论}{7,6}{⾔、⾔}
  \begin{phonetics}{言论}{yan2lun4}
    \definition{s.}{expressão de opinião |  visualizações | comentários | argumentos}
  \end{phonetics}
\end{entry}

\begin{entry}{言语}{7,9}{⾔、⾔}
  \begin{phonetics}{言语}{yan2 yu3}[][HSK 5]
    \definition{s.}{verbal; fala; linguagem falada; conversa; palavras}
  \end{phonetics}
\end{entry}

\begin{entry}{证}{7}{⾔}
  \begin{phonetics}{证}{zheng4}[][HSK 3]
    \definition{s.}{evidência; prova; testemunho; testemunha | certificado; cartão | doença; enfermidade}
    \definition{v.}{provar; verificar; demonstrar}
  \end{phonetics}
\end{entry}

\begin{entry}{证书}{7,4}{⾔、⼄}
  \begin{phonetics}{证书}{zheng4shu1}[][HSK 5]
    \definition[张,份,些]{s.}{certificado; documentos emitidos por instituições, grupos, etc., que comprovem experiência, nível, honras, poderes, etc.}
  \end{phonetics}
\end{entry}

\begin{entry}{证件}{7,6}{⾔、⼈}
  \begin{phonetics}{证件}{zheng4jian4}[][HSK 3]
    \definition[个,本,张]{s.}{documentos; credenciais; certificado}
  \end{phonetics}
\end{entry}

\begin{entry}{证实}{7,8}{⾔、⼧}
  \begin{phonetics}{证实}{zheng4shi2}[][HSK 5]
    \definition{v.}{verificar; afirmar; confirmar; corroborar; demonstrar; autenticar; provar que é verdadeiro}
  \end{phonetics}
\end{entry}

\begin{entry}{证明}{7,8}{⾔、⽇}
  \begin{phonetics}{证明}{zheng4ming2}[][HSK 3]
    \definition[个,份]{s.}{certificado; testemunho; identificação; certificado ou carta de certificação; documentos que comprovem identidade, experiência, etc., como carteira de estudante, carteira de trabalho, certificado de graduação, etc.}
    \definition{v.}{provar; testemunhar; sustentar; usar materiais confiáveis ​​para mostrar ou determinar a autenticidade de uma pessoa ou coisa}
  \end{phonetics}
\end{entry}

\begin{entry}{证据}{7,11}{⾔、⼿}
  \begin{phonetics}{证据}{zheng4ju4}[][HSK 3]
    \definition{s.}{prova; evidência; testemunho; fatos ou materiais relevantes que podem provar a autenticidade de algo}
  \end{phonetics}
\end{entry}

\begin{entry}{评价}{7,6}{⾔、⼈}
  \begin{phonetics}{评价}{ping2jia4}[][HSK 3]
    \definition[个,项,条,份]{s.}{avaliação; apreciação}
    \definition{v.}{estimar; avaliar}
  \end{phonetics}
\end{entry}

\begin{entry}{评论}{7,6}{⾔、⾔}
  \begin{phonetics}{评论}{ping2lun4}[][HSK 5]
    \definition[篇]{s.}{revisão; comentário; artigos ou comentários críticos}
    \definition{v.}{discutir; comentar sobre algo ou alguém}
  \end{phonetics}
\end{entry}

\begin{entry}{评估}{7,7}{⾔、⼈}
  \begin{phonetics}{评估}{ping2gu1}[][HSK 5]
    \definition{v.}{estimar; avaliar; apreciar; avaliar e estimar (coisas abstratas)}
  \end{phonetics}
\end{entry}

\begin{entry}{诅咒}{7,8}{⾔、⼝}
  \begin{phonetics}{诅咒}{zu3zhou4}
    \definition{v.}{amaldiçoar}
  \end{phonetics}
\end{entry}

\begin{entry}{诊断}{7,11}{⾔、⽄}
  \begin{phonetics}{诊断}{zhen3duan4}[][HSK 5]
    \definition{s.}{diagnóstico; diacrisis}
    \definition{v.}{diagnosticar; após examinar os sintomas do paciente, determinar a doença e seu desenvolvimento}
  \end{phonetics}
\end{entry}

\begin{entry}{词}{7}{⾔}
  \begin{phonetics}{词}{ci2}[][HSK 2]
    \definition[个,组,句,段,首]{s.}{palavra; termo; antigamente, referia-se a palavras vazias; atualmente, refere-se a palavras com forma fonética fixa e significado específico na língua; a menor unidade que pode ser usada de forma independente | discurso; declaração; linguagem; texto | ci (um tipo de poesia clássica chinesa, originária da dinastia Tang e plenamente desenvolvida na dinastia Song); gênero poético escrito de acordo com uma estrutura fixa, com versos de comprimentos variados | palavras; redação; refere-se genericamente ao teatro; a parte da letra cantada em harmonia com a melodia em canções e certas artes vocais}
  \end{phonetics}
\end{entry}

\begin{entry}{词汇}{7,5}{⾔、⽔}
  \begin{phonetics}{词汇}{ci2hui4}[][HSK 4]
    \definition[个,组,批,串,堆]{s.}{vocabulário; termo geral para palavras usadas em um idioma}
  \end{phonetics}
\end{entry}

\begin{entry}{词典}{7,8}{⾔、⼋}
  \begin{phonetics}{词典}{ci2dian3}[][HSK 2]
    \definition[本,部]{s.}{dicionário, livro de referência que reúne palavras e explicações para consulta}
  \seealsoref{字典}{zi4 dian3}
  \end{phonetics}
\end{entry}

\begin{entry}{词语}{7,9}{⾔、⾔}
  \begin{phonetics}{词语}{ci2yu3}[][HSK 2]
    \definition[个,租]{s.}{termo; palavra; expressão; conjunto de palavras e frases}
  \end{phonetics}
\end{entry}

\begin{entry}{谷}{7}{⾕}[Kangxi 150]
  \begin{phonetics}{谷}{gu3}
    \definition{adj.}{bom; gentil;}
    \definition{s.}{vale; ravina; desfiladeiro; garganta; faixa estreita de terra com uma saída no meio de duas colinas ou dois platôs | arroz não descascado | salário de funcionário (na época feudal) |calha; cocho; canal | fossa sob o cerebelo (anatomia); valécula | dificuldade; dilema}
    \definition{v.}{criar (filhos) | crescer}
  \end{phonetics}
\end{entry}

\begin{entry}{豆角}{7,7}{⾖、⾓}
  \begin{phonetics}{豆角}{dou4jiao3}
    \definition{s.}{feijão verde}
  \end{phonetics}
\end{entry}

\begin{entry}{豆制品}{7,8,9}{⾖、⼑、⼝}
  \begin{phonetics}{豆制品}{dou4 zhi4 pin3}[][HSK 5]
    \definition{s.}{produtos de soja}
  \end{phonetics}
\end{entry}

\begin{entry}{豆荚}{7,9}{⾖、⾋}
  \begin{phonetics}{豆荚}{dou4jia2}
    \definition{s.}{vagem (de legumes)}
  \end{phonetics}
\end{entry}

\begin{entry}{豆腐}{7,14}{⾖、⾁}
  \begin{phonetics}{豆腐}{dou4fu5}[][HSK 4]
    \definition[块,盒,斤,盘,锅]{s.}{\emph{tofu}}
  \end{phonetics}
\end{entry}

\begin{entry}{财产}{7,6}{⾙、⼇}
  \begin{phonetics}{财产}{cai2chan3}[][HSK 4]
    \definition{s.}{ativos; propriedade; pertences; refere-se à posse de riqueza material, como dinheiro, bens, casas, terras, etc.}
  \end{phonetics}
\end{entry}

\begin{entry}{财富}{7,12}{⾙、⼧}
  \begin{phonetics}{财富}{cai2fu4}[][HSK 4]
    \definition{s.}{riqueza; fortuna}
  \end{phonetics}
\end{entry}

\begin{entry}{赤}{7}{⾚}[Kangxi 155]
  \begin{phonetics}{赤}{chi4}
    \definition*{s.}{sobrenome Chi}
    \definition{adj.}{vermelho; de cor vermelha | leal; sincero; de coração único | nu; sem roupa}
  \end{phonetics}
\end{entry}

\begin{entry}{走}{7}{⾛}[Kangxi 156]
  \begin{phonetics}{走}{zou3}[][HSK 1]
    \definition{v.}{andar; caminhar | correr | mover; movimentar; deslocar | sair; partir; ir embora | visitar; fazer uma visita; (entre amigos e familiares) troca de visitas | passar por; atravessar; ultrapassar | vazar; revelar; divulgar | afastar-se do original; alterar ou perder a forma, o sabor, a cor, etc. originais}
  \end{phonetics}
\end{entry}

\begin{entry}{走开}{7,4}{⾛、⼶}
  \begin{phonetics}{走开}{zou3 kai1}[][HSK 2]
    \definition{v.}{ir embora | fugir | ir para outro lugar}
  \end{phonetics}
\end{entry}

\begin{entry}{走去}{7,5}{⾛、⼛}
  \begin{phonetics}{走去}{zou3qu4}
    \definition{v.}{caminhar até (para)}
  \end{phonetics}
\end{entry}

\begin{entry}{走过}{7,6}{⾛、⾡}
  \begin{phonetics}{走过}{zou3 guo4}[][HSK 2]
    \definition{v.}{passar}
  \end{phonetics}
\end{entry}

\begin{entry}{走秀}{7,7}{⾛、⽲}
  \begin{phonetics}{走秀}{zou3xiu4}
    \definition{s.}{desfile de moda}
    \definition{v.}{andar na passarela (em um desfile de moda)}
  \end{phonetics}
\end{entry}

\begin{entry}{走进}{7,7}{⾛、⾡}
  \begin{phonetics}{走进}{zou3 jin4}[][HSK 2]
    \definition{v.}{entrar}
  \end{phonetics}
\end{entry}

\begin{entry}{走势}{7,8}{⾛、⼒}
  \begin{phonetics}{走势}{zou3shi4}
    \definition{s.}{caminho | tendência}
  \end{phonetics}
\end{entry}

\begin{entry}{走卒}{7,8}{⾛、⼗}
  \begin{phonetics}{走卒}{zou3zu2}
    \definition{s.}{lacaio (masculino) | peão (isto é, soldado de infantaria) | servo}
  \end{phonetics}
\end{entry}

\begin{entry}{走鬼}{7,9}{⾛、⿁}
  \begin{phonetics}{走鬼}{zou3gui3}
    \definition{s.}{vendedor ambulante sem licença}
  \end{phonetics}
\end{entry}

\begin{entry}{走索}{7,10}{⾛、⽷}
  \begin{phonetics}{走索}{zou3suo3}
    \definition{v.}{andar na corda bamba}
  \seealsoref{走绳}{zou3sheng2}
  \end{phonetics}
\end{entry}

\begin{entry}{走绳}{7,11}{⾛、⽷}
  \begin{phonetics}{走绳}{zou3sheng2}
    \definition{v.}{andar na corda bamba}
  \seealsoref{走索}{zou3suo3}
  \end{phonetics}
\end{entry}

\begin{entry}{走路}{7,13}{⾛、⾜}
  \begin{phonetics}{走路}{zou3 lu4}[][HSK 1]
    \definition{v.}{caminhar; ir a pé; andar em pé sobre a terra | sair; ir embora; partir}
  \end{phonetics}
\end{entry}

\begin{entry}{足}{7}{⾜}[Kangxi 157]
  \begin{phonetics}{足}{ju4}
    \definition{adj.}{excessivo}
  \end{phonetics}
  \begin{phonetics}{足}{zu2}
    \definition{adj.}{amplo}
    \definition{s.}{pé}
    \definition{v.}{ser suficiente}
  \end{phonetics}
\end{entry}

\begin{entry}{足月}{7,4}{⾜、⽉}
  \begin{phonetics}{足月}{zu2yue4}
    \definition{s.}{gestação completa}
  \end{phonetics}
\end{entry}

\begin{entry}{足足}{7,7}{⾜、⾜}
  \begin{phonetics}{足足}{zu2zu2}
    \definition{adv.}{tanto quanto | extremamente | completamente | não menos que}
  \end{phonetics}
\end{entry}

\begin{entry}{足够}{7,11}{⾜、⼣}
  \begin{phonetics}{足够}{zu2 gou4}[][HSK 3]
    \definition{adj.}{bastante; amplo; suficiente; na medida em que deve ser ou pode atender às necessidades}
    \definition{v.}{satisfazer; ser suficiente; estar a contento}
  \end{phonetics}
\end{entry}

\begin{entry}{足球}{7,11}{⾜、⽟}
  \begin{phonetics}{足球}{zu2qiu2}[][HSK 3]
    \definition[个,只,颗,袋]{s.}{futebol | bola de futebol}
  \end{phonetics}
\end{entry}

\begin{entry}{足球队}{7,11,4}{⾜、⽟、⾩}
  \begin{phonetics}{足球队}{zu2qiu2dui4}
    \definition{s.}{time de futebol}
  \end{phonetics}
\end{entry}

\begin{entry}{足球协会}{7,11,6,6}{⾜、⽟、⼗、⼈}
  \begin{phonetics}{足球协会}{zu2qiu2xie2hui4}
    \definition*{s.}{Associação de Futebol}
  \end{phonetics}
\end{entry}

\begin{entry}{足球场}{7,11,6}{⾜、⽟、⼟}
  \begin{phonetics}{足球场}{zu2qiu2chang3}
    \definition{s.}{campo de futebol}
  \end{phonetics}
\end{entry}

\begin{entry}{足球迷}{7,11,9}{⾜、⽟、⾡}
  \begin{phonetics}{足球迷}{zu2qiu2mi2}
    \definition{s.}{fã de futebol}
  \end{phonetics}
\end{entry}

\begin{entry}{足球赛}{7,11,14}{⾜、⽟、⾙}
  \begin{phonetics}{足球赛}{zu2qiu2sai4}
    \definition{s.}{competição de futebol | partida de futebol}
  \end{phonetics}
\end{entry}

\begin{entry}{身上}{7,3}{⾝、⼀}
  \begin{phonetics}{身上}{shen1 shang5}[][HSK 1]
    \definition{s.}{no corpo de alguém | em um;  com um}
  \end{phonetics}
\end{entry}

\begin{entry}{身亡}{7,3}{⾝、⼇}
  \begin{phonetics}{身亡}{shen1wang2}
    \definition{v.}{morrer}
  \end{phonetics}
\end{entry}

\begin{entry}{身边}{7,5}{⾝、⾡}
  \begin{phonetics}{身边}{shen1 bian1}[][HSK 2]
    \definition{adv.}{ao redor; ao lado de alguém; perto do corpo | carregar consigo (transportar); à mão}
  \end{phonetics}
\end{entry}

\begin{entry}{身份}{7,6}{⾝、⼈}
  \begin{phonetics}{身份}{shen1fen4}[][HSK 4]
    \definition[种]{s.}{status; capacidade; identidade; refere-se à origem, ao status e às qualificações de uma pessoa | dignidade; posição honrada; referência especial ao status respeitável}
  \end{phonetics}
\end{entry}

\begin{entry}{身份证}{7,6,7}{⾝、⼈、⾔}
  \begin{phonetics}{身份证}{shen1 fen4 zheng4}[][HSK 3]
    \definition[张]{s.}{ID; bilhete de identidade; carteira de identidade}
  \end{phonetics}
\end{entry}

\begin{entry}{身体}{7,7}{⾝、⼈}
  \begin{phonetics}{身体}{shen1ti3}[][HSK 1]
    \definition[具,个]{s.}{corpo | saúde; saúde das pessoas}
  \end{phonetics}
\end{entry}

\begin{entry}{身体乳}{7,7,8}{⾝、⼈、⼄}
  \begin{phonetics}{身体乳}{shen1ti3 ru3}
    \definition{s.}{loção corporal}
  \end{phonetics}
\end{entry}

\begin{entry}{身体能力}{7,7,10,2}{⾝、⼈、⾁、⼒}
  \begin{phonetics}{身体能力}{shen1ti3 neng2li4}
    \definition{s.}{habilidade física}
  \end{phonetics}
\end{entry}

\begin{entry}{身材}{7,7}{⾝、⽊}
  \begin{phonetics}{身材}{shen1cai2}[][HSK 4]
    \definition[副,种,个,具]{s.}{figura; estatura; altura e peso corporal}
  \end{phonetics}
\end{entry}

\begin{entry}{身高}{7,10}{⾝、⾼}
  \begin{phonetics}{身高}{shen1 gao1}[][HSK 4]
    \definition[个,种,段]{s.}{estatura; altura (de uma pessoa);}
  \end{phonetics}
\end{entry}

\begin{entry}{辛苦}{7,8}{⾟、⾋}
  \begin{phonetics}{辛苦}{xin1ku3}[][HSK 5]
    \definition{adj.}{difícil; trabalhoso; árduo; descreve muito trabalho, alta intensidade e pouco descanso}
    \definition{s.}{dificuldades}
    \definition{v.}{trabalhar duro; passar por grandes dificuldades; passar por dificuldades}
  \end{phonetics}
\end{entry}

\begin{entry}{迎接}{7,11}{⾡、⼿}
  \begin{phonetics}{迎接}{ying2jie1}[][HSK 3]
    \definition{v.}{conhecer; cumprimentar; dar as boas-vindas}
  \end{phonetics}
\end{entry}

\begin{entry}{运}{7}{⾡}
  \begin{phonetics}{运}{yun4}[][HSK 5]
    \definition*{s.}{sobrenome Yun}
    \definition{s.}{sorte; destino; fortuna}
    \definition{v.}{mover; deslocar | transportar; levar | usar; empunhar; utilizar}
  \end{phonetics}
\end{entry}

\begin{entry}{运气}{7,4}{⾡、⽓}
  \begin{phonetics}{运气}{yun4qi5}[][HSK 4]
    \definition{adj.}{sortudo; afortunado}
    \definition{s.}{sorte; fortuna}
  \end{phonetics}
\end{entry}

\begin{entry}{运用}{7,5}{⾡、⽤}
  \begin{phonetics}{运用}{yun4yong4}[][HSK 4]
    \definition{v.}{usar; utilizar; manejar; aplicar; explorar as coisas de acordo com suas características}
  \end{phonetics}
\end{entry}

\begin{entry}{运动}{7,6}{⾡、⼒}
  \begin{phonetics}{运动}{yun4dong4}[][HSK 2]
    \definition[场]{s.}{esporte | desporto}
    \definition{v.}{exercitar | mover-se}
  \end{phonetics}
\end{entry}

\begin{entry}{运动会}{7,6,6}{⾡、⼒、⼈}
  \begin{phonetics}{运动会}{yun4 dong4 hui4}[][HSK 4]
    \definition[个]{s.}{jogos; encontro esportivo; dia de esportes; reunião atlética}
  \end{phonetics}
\end{entry}

\begin{entry}{运动场}{7,6,6}{⾡、⼒、⼟}
  \begin{phonetics}{运动场}{yun4dong4chang3}
    \definition{s.}{campo desportivo | campo de jogos}
  \end{phonetics}
\end{entry}

\begin{entry}{运动员}{7,6,7}{⾡、⼒、⼝}
  \begin{phonetics}{运动员}{yun4 dong4 yuan2}[][HSK 4]
    \definition[名,个]{s.}{jogador; atleta; esportista; pessoas que participam de competições esportivas}
  \end{phonetics}
\end{entry}

\begin{entry}{运动学}{7,6,8}{⾡、⼒、⼦}
  \begin{phonetics}{运动学}{yun4dong4xue2}
    \definition{s.}{cinemática}
  \end{phonetics}
\end{entry}

\begin{entry}{运动服}{7,6,8}{⾡、⼒、⽉}
  \begin{phonetics}{运动服}{yun4dong4fu2}
    \definition{s.}{roupa para prática de esporte}
  \end{phonetics}
\end{entry}

\begin{entry}{运动衫}{7,6,8}{⾡、⼒、⾐}
  \begin{phonetics}{运动衫}{yun4dong4shan1}
    \definition[件]{s.}{moletom | camisa esportiva}
  \end{phonetics}
\end{entry}

\begin{entry}{运动家}{7,6,10}{⾡、⼒、⼧}
  \begin{phonetics}{运动家}{yun4dong4jia1}
    \definition{s.}{ativista | atleta | esportista}
  \end{phonetics}
\end{entry}

\begin{entry}{运动病}{7,6,10}{⾡、⼒、⽧}
  \begin{phonetics}{运动病}{yun4dong4bing4}
    \definition{s.}{enjôo (movimento, carro, etc.)}
  \end{phonetics}
\end{entry}

\begin{entry}{运动鞋}{7,6,15}{⾡、⼒、⾰}
  \begin{phonetics}{运动鞋}{yun4dong4xie2}
    \definition{s.}{tênis | sapatos esportivos}
  \end{phonetics}
\end{entry}

\begin{entry}{运行}{7,6}{⾡、⾏}
  \begin{phonetics}{运行}{yun4xing2}[][HSK 5]
    \definition{v.}{(corpos celestes, etc.) mover-se ao longo do curso | (figurativo) funcionar, estar em operação | (serviço de trem, etc.) operar | (computador) executar um programa}
  \end{phonetics}
\end{entry}

\begin{entry}{运河}{7,8}{⾡、⽔}
  \begin{phonetics}{运河}{yun4he2}
    \definition{s.}{canal (em um rio)}
  \end{phonetics}
\end{entry}

\begin{entry}{运输}{7,13}{⾡、⾞}
  \begin{phonetics}{运输}{yun4shu1}[][HSK 3]
    \definition{v.}{enviar; transportar; usar um carro, navio, avião, etc. para transportar pessoas ou coisas de um lugar para outro}
  \end{phonetics}
\end{entry}

\begin{entry}{近}{7}{⾡}
  \begin{phonetics}{近}{jin4}[][HSK 2]
    \definition{adj.}{próximo; perto; distância espacial ou temporal curta (oposto de 远) | íntimo; intimamente relacionado; relação estreita | fácil de entender}
  \seealsoref{远}{yuan3}
  \end{phonetics}
\end{entry}

\begin{entry}{近代}{7,5}{⾡、⼈}
  \begin{phonetics}{近代}{jin4dai4}[][HSK 4]
    \definition{s.}{tempos modernos; era passada relativamente próxima à era moderna, geralmente referida na história chinesa como 1840 a 1919 | na história mundial, geralmente se refere à era capitalista}
  \end{phonetics}
\end{entry}

\begin{entry}{近来}{7,7}{⾡、⽊}
  \begin{phonetics}{近来}{jin4lai2}[][HSK 5]
    \definition{adv.}{ultimamente; recentemente; de ​​tarde; refere-se a um período de tempo entre o passado imediato e o presente}
  \end{phonetics}
\end{entry}

\begin{entry}{近期}{7,12}{⾡、⽉}
  \begin{phonetics}{近期}{jin4 qi1}[][HSK 3]
    \definition{adv.}{num futuro próximo; brevemente}
  \end{phonetics}
\end{entry}

\begin{entry}{返回}{7,6}{⾡、⼞}
  \begin{phonetics}{返回}{fan3 hui2}[][HSK 5]
    \definition{v.}{retornar; ir (voltar); reverter; recorrer; retroceder; voltar para (o lugar original)}
  \end{phonetics}
\end{entry}

\begin{entry}{还}{7}{⾡}
  \begin{phonetics}{还}{hai2}[][HSK 1]
    \definition{adv.}{ainda; indica que a ação ou estado permanece inalterado, equivalente a 仍然 | também; além disso; em adição; indica que há um aumento ou suplemento além do escopo já indicado | ainda mais; usado com 比 para indicar que as características e o grau das coisas comparadas aumentaram, o que é equivalente a 更加 razoavelmente; medianamente; usado antes de um adjetivo, indica que algo atinge apenas o nível mínimo exigido | mesmo; usado na primeira parte da frase como complemento, e na segunda parte como conclusão, equivalente a 尚且 | que expressa realização ou descoberta; expressa surpresa por algo que não se esperava, mas que acabou acontecendo | tão cedo quanto; por um curto período de tempo; indica que já era assim há muito tempo | para dar ênfase; para reforçar o tom}
  \seealsoref{比}{bi3}
  \seealsoref{更加}{geng4 jia1}
  \seealsoref{仍然}{reng2ran2}
  \seealsoref{尚且}{shang4qie3}
  \end{phonetics}
  \begin{phonetics}{还}{huan2}[][HSK 1]
    \definition*{s.}{sobrenome Huan}
    \definition{v.}{voltar; retornar; voltar ao lugar original ou restaurar o estado original | retribuir; devolver; reembolsar; devolver o dinheiro ou os bens emprestados ao seu proprietário | dar ou fazer algo em troca; retribuir as ações dos outros}
  \end{phonetics}
\end{entry}

\begin{entry}{还有}{7,6}{⾡、⽉}
  \begin{phonetics}{还有}{hai2 you3}[][HSK 1]
    \definition{adv.}{também; ainda; além disso; então novamente; enfatizar as partes complementares, excedentes ou não mencionadas além do que já é conhecido}
  \end{phonetics}
\end{entry}

\begin{entry}{还是}{7,9}{⾡、⽇}
  \begin{phonetics}{还是}{hai2shi5}[][HSK 1]
    \definition{adv.}{ainda; ainda assim; não é a continuação de um determinado estado, fenômeno ou ação; o resultado é o mesmo de antes, sem mudanças  |que expressa uma preferência por uma alternativa; expressa comparação ou escolha feita após consideração cuidadosa, frequentemente usado para fazer sugestões | que expressa realização ou descoberta; indica que o resultado final foi inesperado}
    \definition{conj.}{ou (somente para frases interrogativas); indica várias opções, geralmente usado em perguntas | tudo; se; não importa; independentemente de; significa que, independentemente das mudanças que ocorram, o resultado permanecerá o mesmo}
  \end{phonetics}
\end{entry}

\begin{entry}{这}{7}{⾡}
  \begin{phonetics}{这}{zhe4}[][HSK 1]
    \definition{pron.}{este, isto; substitui pessoas ou coisas que estão mais próximas | agora; em vez de 这时候, tem o efeito de reforçar a ênfase}
  \seealsoref{这时候}{zhe4 shi2 hou5}
  \end{phonetics}
  \begin{phonetics}{这}{zhei4}
    \definition{pron.}{(coloquial) este}
  \end{phonetics}
\end{entry}

\begin{entry}{这儿}{7,2}{⾡、⼉}
  \begin{phonetics}{这儿}{zhe4r5}[][HSK 1]
    \definition{pron.}{aqui | agora; neste momento (utilizado apenas após 打, 从, 由)}
  \seealsoref{从}{cong2}
  \seealsoref{打}{da3}
  \seealsoref{由}{you2}
  \end{phonetics}
\end{entry}

\begin{entry}{这个}{7,3}{⾡、⼈}
  \begin{phonetics}{这个}{zhe4ge5}
    \definition{pron.}{isto; este | isso; em vez das coisas mencionadas anteriormente | assim; tal; usado antes de verbos e adjetivos, indica um grau muito profundo, com um sentido exagerado | usado junto com 那个 para indicar pessoas ou objetos indefinidos}
  \seealsoref{那个}{na4ge5}
  \end{phonetics}
\end{entry}

\begin{entry}{这么}{7,3}{⾡、⼃}
  \begin{phonetics}{这么}{zhe4 me5}[][HSK 2]
    \definition{adv.}{como este | desta maneira}
  \end{phonetics}
\end{entry}

\begin{entry}{这边}{7,5}{⾡、⾡}
  \begin{phonetics}{这边}{zhe4 bian1}[][HSK 1]
    \definition{pron.}{aqui; deste lado; refere-se a um lugar próximo}
  \end{phonetics}
\end{entry}

\begin{entry}{这时}{7,7}{⾡、⽇}
  \begin{phonetics}{这时}{zhe4 shi2}[][HSK 2]
    \definition{adv.}{neste momento}
  \end{phonetics}
\end{entry}

\begin{entry}{这时候}{7,7,10}{⾡、⽇、⼈}
  \begin{phonetics}{这时候}{zhe4 shi2 hou5}[][HSK 2]
    \definition{adv.}{neste momento}
  \end{phonetics}
\end{entry}

\begin{entry}{这里}{7,7}{⾡、⾥}
  \begin{phonetics}{这里}{zhe4 li3}[][HSK 1]
    \definition{pron.}{aqui; pronomes demonstrativo, indicando locais próximos}
  \end{phonetics}
\end{entry}

\begin{entry}{这些}{7,8}{⾡、⼆}
  \begin{phonetics}{这些}{zhe4 xie1}[][HSK 1]
    \definition{pron.}{estes; pronome demonstrativo, que indicam duas ou mais pessoas ou coisas que estão próximas}
  \end{phonetics}
\end{entry}

\begin{entry}{这样}{7,10}{⾡、⽊}
  \begin{phonetics}{这样}{zhe4 yang4}[][HSK 2]
    \definition{adv.}{assim | dessa maneira | deste modo}
  \end{phonetics}
\end{entry}

\begin{entry}{这麽}{7,14}{⾡、⿇}
  \begin{phonetics}{这麽}{zhe4 me5}
    \variantof{这么}
  \end{phonetics}
\end{entry}

\begin{entry}{进}{7}{⾡}
  \begin{phonetics}{进}{jin4}[][HSK 1]
    \definition*{s.}{sobrenome Jin}
    \definition{clas.}{para seções em um edifício ou complexo residencial; qualquer uma das várias fileiras de casas em um complexo residencial de estilo antigo}
    \definition{s.}{(matemática) base de um sistema numérico}
    \definition{v.}{avançar; ir adiante; seguir em frente; (oposto a 退) | entrar; entrar em; entrar ou sair; (oposto a 出) | receber | comer; tomar; beber | submeter; apresentar | marcar um gol}
    \definition{v.aux.}{usado após um verbo, significa ``para dentro''}
  \seealsoref{出}{chu1}
  \seealsoref{退}{tui4}
  \end{phonetics}
\end{entry}

\begin{entry}{进一步}{7,1,7}{⾡、⼀、⽌}
  \begin{phonetics}{进一步}{jin4 yi2 bu4}[][HSK 3]
    \definition{adv.}{mais; dar um passo adiante; avançar um passo}
  \end{phonetics}
\end{entry}

\begin{entry}{进入}{7,2}{⾡、⼊}
  \begin{phonetics}{进入}{jin4 ru4}[][HSK 2]
    \definition{v.}{entrar; entrar em}
  \end{phonetics}
\end{entry}

\begin{entry}{进口}{7,3}{⾡、⼝}
  \begin{phonetics}{进口}{jin4kou3}[][HSK 4]
    \definition{adj.}{importado}
    \definition{s.}{importação; entrada de um edifício ou local, também chamada de 入口}
    \definition{v.+compl.}{importar; comprar ou transportar mercadorias de outro país ou região | entrar no porto; navegar em direção a um porto}
  \seealsoref{入口}{ru4kou3}
  \end{phonetics}
\end{entry}

\begin{entry}{进化}{7,4}{⾡、⼔}
  \begin{phonetics}{进化}{jin4hua4}[][HSK 5]
    \definition[个]{s.}{evolução; os organismos se desenvolvem e evoluem do simples para o complexo e de níveis baixos para altos}
    \definition{v.}{evoluir; um termo geral usado para descrever uma mudança gradual para melhor}
  \end{phonetics}
\end{entry}

\begin{entry}{进出口}{7,5,3}{⾡、⼐、⼝}
  \begin{phonetics}{进出口}{jin4chu1kou3}
    \definition{s.}{importação e exportação}
    \definition{v.}{importar e exportar}
  \end{phonetics}
\end{entry}

\begin{entry}{进去}{7,5}{⾡、⼛}
  \begin{phonetics}{进去}{jin4 qu4}[][HSK 1]
    \definition{v.}{entrar (a partir da minha localização)}
    \definition{v.aux.}{usado depois de um verbo, significa ``ir para dentro''; para um determinado intervalo ou período de tempo}
  \end{phonetics}
\end{entry}

\begin{entry}{进行}{7,6}{⾡、⾏}
  \begin{phonetics}{进行}{jin4xing2}[][HSK 2]
    \definition{v.}{continuar; estar em andamento; estar em progresso | fazer; conduzir; realizar; executar | marchar; avançar; prosseguir; estar em marcha}
  \end{phonetics}
\end{entry}

\begin{entry}{进行编程}{7,6,12,12}{⾡、⾏、⽷、⽲}
  \begin{phonetics}{进行编程}{jin4xing2bian1cheng2}
    \definition{s.}{programa de computador executável}
  \end{phonetics}
\end{entry}

\begin{entry}{进来}{7,7}{⾡、⽊}
  \begin{phonetics}{进来}{jin4 lai2}[][HSK 1]
    \definition{v.}{entrar (para a minha localização)}
  \end{phonetics}
\end{entry}

\begin{entry}{进步}{7,7}{⾡、⽌}
  \begin{phonetics}{进步}{jin4bu4}[][HSK 3]
    \definition{adj.}{progressivo}
    \definition[个]{s.}{avanço; progresso; melhora}
    \definition{v.}{avançar; progredir; melhorar}
  \end{phonetics}
\end{entry}

\begin{entry}{进展}{7,10}{⾡、⼫}
  \begin{phonetics}{进展}{jin4zhan3}[][HSK 3]
    \definition{v.}{fazer progresso; progredir}
  \end{phonetics}
\end{entry}

\begin{entry}{远}{7}{⾡}
  \begin{phonetics}{远}{yuan3}[][HSK 1]
    \definition*{s.}{sobrenome Yuan}
    \definition{adj.}{distante (no tempo ou no espaço); longe; remoto; Longa distância espacial ou temporal (em oposição a 近) | (relações de parentesco) distante | com grande diferença}
    \definition{v.}{manter-se afastado de; não se aproximar}
  \seealsoref{近}{jin4}
  \end{phonetics}
\end{entry}

\begin{entry}{远天}{7,4}{⾡、⼤}
  \begin{phonetics}{远天}{yuan3tian1}
    \definition{s.}{paraíso | o céu distante}
  \end{phonetics}
\end{entry}

\begin{entry}{远方}{7,4}{⾡、⽅}
  \begin{phonetics}{远方}{yuan3fang1}
    \definition{s.}{longe | um local distante}
  \end{phonetics}
\end{entry}

\begin{entry}{远处}{7,5}{⾡、⼡}
  \begin{phonetics}{远处}{yuan3 chu4}[][HSK 5]
    \definition{s.}{distância; lugar distante}
  \end{phonetics}
\end{entry}

\begin{entry}{远远}{7,7}{⾡、⾡}
  \begin{phonetics}{远远}{yuan3yuan3}
    \definition{adv.}{de longe}
  \end{phonetics}
\end{entry}

\begin{entry}{远征}{7,8}{⾡、⼻}
  \begin{phonetics}{远征}{yuan3zheng1}
    \definition{s.}{uma expedição militar | marcha para regiões remotas}
  \end{phonetics}
\end{entry}

\begin{entry}{违反}{7,4}{⾡、⼜}
  \begin{phonetics}{违反}{wei2fan3}[][HSK 5]
    \definition{v.}{violar; transgredir; contrariar; não estar em conformidade (com as regras, regulamentos, etc.)}
  \end{phonetics}
\end{entry}

\begin{entry}{违法}{7,8}{⾡、⽔}
  \begin{phonetics}{违法}{wei2 fa3}[][HSK 5]
    \definition{v.}{ser ilegal; infringir a lei; violar a lei ou os regulamentos}
  \end{phonetics}
\end{entry}

\begin{entry}{违规}{7,8}{⾡、⾒}
  \begin{phonetics}{违规}{wei2 gui1}[][HSK 5]
    \definition{v.}{violar (regras); infringir as regras e regulamentos}
  \end{phonetics}
\end{entry}

\begin{entry}{违宪}{7,9}{⾡、⼧}
  \begin{phonetics}{违宪}{wei2xian4}
    \definition{adj.}{inconstitucional}
  \end{phonetics}
\end{entry}

\begin{entry}{连}{7}{⾡}
  \begin{phonetics}{连}{lian2}[][HSK 3]
    \definition*{s.}{sobrenome Lian}
    \definition{adv.}{em sucessão; um após o outro; repetidamente | até}
    \definition{prep.}{incluindo}
    \definition{s.}{companhia | conjunção}
    \definition{v.}{ligar; juntar; conectar | envolver (em problemas); implicar | costurar; coser}
  \end{phonetics}
\end{entry}

\begin{entry}{连忙}{7,6}{⾡、⼼}
  \begin{phonetics}{连忙}{lian2mang2}[][HSK 3]
    \definition{adv.}{prontamente; imediatamente; apressadamente}
  \end{phonetics}
\end{entry}

\begin{entry}{连接}{7,11}{⾡、⼿}
  \begin{phonetics}{连接}{lian2 jie1}[][HSK 5]
    \definition[条]{s.}{conexão}
    \definition{v.}{ligar; unir; relacionar, conectar; anexar}
  \end{phonetics}
\end{entry}

\begin{entry}{连续}{7,11}{⾡、⽷}
  \begin{phonetics}{连续}{lian2xu4}[][HSK 3]
    \definition{adv.}{continuamente; sucessivamente; em uma fileira}
  \end{phonetics}
\end{entry}

\begin{entry}{连续剧}{7,11,10}{⾡、⽷、⼑}
  \begin{phonetics}{连续剧}{lian2 xu4 ju4}[][HSK 3]
    \definition{s.}{série; novela}
  \end{phonetics}
\end{entry}

\begin{entry}{连锁反应}{7,12,4,7}{⾡、⾦、⼜、⼴}
  \begin{phonetics}{连锁反应}{lian2suo3fan3ying4}
    \definition{s.}{reação em cadeia}
  \end{phonetics}
\end{entry}

\begin{entry}{迟}{7}{⾡}
  \begin{phonetics}{迟}{chi2}[][HSK 5]
    \definition*{s.}{sobrenome Chi}
    \definition{adj.}{lento; tardio; demorado | atrasado | lento; obtuso}
  \end{phonetics}
\end{entry}

\begin{entry}{迟到}{7,8}{⾡、⼑}
  \begin{phonetics}{迟到}{chi2dao4}[][HSK 4]
    \definition{v.}{chegar atrasado; atrasar-se}
  \end{phonetics}
\end{entry}

\begin{entry}{邮包}{7,5}{⾢、⼓}
  \begin{phonetics}{邮包}{you2bao1}
    \definition{s.}{encomenda postal}
  \end{phonetics}
\end{entry}

\begin{entry}{邮市}{7,5}{⾢、⼱}
  \begin{phonetics}{邮市}{you2shi4}
    \definition{s.}{mercado postal}
  \end{phonetics}
\end{entry}

\begin{entry}{邮电}{7,5}{⾢、⽥}
  \begin{phonetics}{邮电}{you2dian4}
    \definition*{s.}{Correios e Telecomunicações}
  \end{phonetics}
\end{entry}

\begin{entry}{邮件}{7,6}{⾢、⼈}
  \begin{phonetics}{邮件}{you2 jian4}[][HSK 3]
    \definition[封,个]{s.}{correspondência; correio; assunto postal; um termo geral para cartas, encomendas, etc. recebidas, transportadas e entregues pelos correios | \emph{e-mail}; mensagens enviadas e recebidas por meio eletrônico}
  \end{phonetics}
\end{entry}

\begin{entry}{邮局}{7,7}{⾢、⼫}
  \begin{phonetics}{邮局}{you2ju2}[][HSK 4]
    \definition[家]{s.}{correio; agência dos correios; organizações que lidam com serviços postais}
  \end{phonetics}
\end{entry}

\begin{entry}{邮费}{7,9}{⾢、⾙}
  \begin{phonetics}{邮费}{you2fei4}
    \definition{s.}{postagem}
    \definition{v.}{postar}
  \end{phonetics}
\end{entry}

\begin{entry}{邮迷}{7,9}{⾢、⾡}
  \begin{phonetics}{邮迷}{you2mi2}
    \definition{s.}{filatelista | colecionador de selos}
  \end{phonetics}
\end{entry}

\begin{entry}{邮资}{7,10}{⾢、⾙}
  \begin{phonetics}{邮资}{you2zi1}
    \definition{s.}{postagem}
  \end{phonetics}
\end{entry}

\begin{entry}{邮递}{7,10}{⾢、⾡}
  \begin{phonetics}{邮递}{you2di4}
    \definition{v.}{enviar por correio}
  \end{phonetics}
\end{entry}

\begin{entry}{邮票}{7,11}{⾢、⽰}
  \begin{phonetics}{邮票}{you2 piao4}[][HSK 3]
    \definition[枚,张,套,版]{s.}{selo; selo postal; um \emph{voucher} vendido pelos correios e afixado na correspondência para indicar que a postagem foi paga}
  \end{phonetics}
\end{entry}

\begin{entry}{邮箱}{7,15}{⾢、⾋}
  \begin{phonetics}{邮箱}{you2 xiang1}[][HSK 3]
    \definition{s.}{caixa de correio | \emph{mailbox}; refere-se ao endereço de \emph{e-mail}}
  \end{phonetics}
\end{entry}

\begin{entry}{邻居}{7,8}{⾢、⼫}
  \begin{phonetics}{邻居}{lin2ju1}[][HSK 5]
    \definition[个,位,家]{s.}{vizinho; pessoas ou famílias que moram muito perto}
  \end{phonetics}
\end{entry}

\begin{entry}{里}{7}{⾥}[Kangxi 166]
  \begin{phonetics}{里}{li3}[][HSK 1]
    \definition*{s.}{sobrenome Li}
    \definition{clas.}{li, uma unidade chinesa de comprimento (= 1/2 quilômetro)}
    \definition{s.}{forro; revestimento; interior; parte de trás do tecido | interno; dentro; no interior | vizinhança; vizinhos | cidade natal; local de origem}
  \end{phonetics}
\end{entry}

\begin{entry}{里头}{7,5}{⾥、⼤}
  \begin{phonetics}{里头}{li3 tou5}[][HSK 2]
    \definition{s.}{dentro}
  \end{phonetics}
\end{entry}

\begin{entry}{里边}{7,5}{⾥、⾡}
  \begin{phonetics}{里边}{li3 bian5}[][HSK 1]
    \definition{s.}{em; dentro; no interior}
  \end{phonetics}
\end{entry}

\begin{entry}{里面}{7,9}{⾥、⾯}
  \begin{phonetics}{里面}{li3 mian4}[][HSK 3]
    \definition{s.}{dentro; interior}
  \end{phonetics}
\end{entry}

\begin{entry}{里斯本}{7,12,5}{⾥、⽄、⽊}
  \begin{phonetics}{里斯本}{li3si1ben3}
    \definition*{s.}{Lisboa}
  \end{phonetics}
\end{entry}

\begin{entry}{里斯本大学}{7,12,5,3,8}{⾥、⽄、⽊、⼤、⼦}
  \begin{phonetics}{里斯本大学}{li3si1ben3 da4xue2}
    \definition*{s.}{Universidade de Lisboa}
  \end{phonetics}
\end{entry}

\begin{entry}{针}{7}{⾦}
  \begin{phonetics}{针}{zhen1}[][HSK 4]
    \definition*{s.}{sobrenome Zhen}
    \definition[根]{s.}{agulha; ferramentas para costura de roupas | objetos semelhantes a agulhas; algo longo e fino como uma agulha | injeção | ponto de costura | pontos de acupuntura na medicina chinesa}
  \end{phonetics}
\end{entry}

\begin{entry}{针对}{7,5}{⾦、⼨}
  \begin{phonetics}{针对}{zhen1dui4}[][HSK 4]
    \definition{prep.}{em conexão com; de acordo com; à luz de; introdução de objetos de comportamento com uma finalidade clara}
    \definition{v.}{contrariar; apontar para; ter como objetivo; ser direcionado contra; fazer algo especificamente sobre um problema ou uma pessoa}
  \end{phonetics}
\end{entry}

\begin{entry}{闲}{7}{⾨}
  \begin{phonetics}{闲}{xian2}[][HSK 5]
    \definition{adj.}{ocioso; não ocupado; desocupado; sem coisas para fazer; sem atividades; tempo livre | desocupado; (casa, objeto, etc.) não em uso; ocioso | não oficial; não sério; não relacionado ao negócio}
    \definition{s.}{lazer; tempo livre}
  \end{phonetics}
\end{entry}

\begin{entry}{间}{7}{⾨}
  \begin{phonetics}{间}{jian1}[][HSK 1]
    \definition{clas.}{a menor unidade de uma casa; a menor unidade habitacional; cômodo}
    \definition{s.}{espaço entre duas partes  | (em um) tempo ou espaço definido | sala; quarto | uma seção de uma sala ou o espaço lateral entre dois pares de pilares | com um tempo ou espaço definido}
  \end{phonetics}
  \begin{phonetics}{间}{jian4}
    \definition{s.}{espaço entre as duas partes; abertura; lacuna}
    \definition{v.}{separar | semear a discórdia | desbastar (mudas); podar; remover ou arrancar as mudas em excesso}
  \end{phonetics}
\end{entry}

\begin{entry}{间或}{7,8}{⾨、⼽}
  \begin{phonetics}{间或}{jian4huo4}
    \definition{adv.}{às vezes | ocasionalmente | de vez em quando}
  \end{phonetics}
\end{entry}

\begin{entry}{间接}{7,11}{⾨、⼿}
  \begin{phonetics}{间接}{jian4jie1}[][HSK 5]
    \definition{adj.}{indireto; de segunda mão; em oposição a 直接}
  \seealsoref{直接}{zhi2jie1}
  \end{phonetics}
\end{entry}

\begin{entry}{闷热}{7,10}{⾨、⽕}
  \begin{phonetics}{闷热}{men1re4}
    \definition{adj.}{abafado | quente e abafado | sufocantemente quente | quente e sensual}
  \end{phonetics}
\end{entry}

\begin{entry}{阻止}{7,4}{⾩、⽌}
  \begin{phonetics}{阻止}{zu3zhi3}[][HSK 4]
    \definition{v.}{parar; reter; conter; interromper; impedir o avanço; impedir o movimento; obstruir}
  \end{phonetics}
\end{entry}

\begin{entry}{阻击}{7,5}{⾩、⼐}
  \begin{phonetics}{阻击}{zu3ji1}
    \definition{v.}{verificar | parar}
  \end{phonetics}
\end{entry}

\begin{entry}{阻碍}{7,13}{⾩、⽯}
  \begin{phonetics}{阻碍}{zu3'ai4}[][HSK 5]
    \definition{s.}{obstáculo; impedimento; barreira}
    \definition{v.}{bloquear; impedir; obstruir; impedir o bom andamento ou desenvolvimento}
  \end{phonetics}
\end{entry}

\begin{entry}{阿}{7}{⾩}
  \begin{phonetics}{阿}{a1}
    \definition{pref.}{em dialetos do sul para formar termos carinhosos, antes de nomes de animais de estimação, sobrenomes monossilábicos ou números que denotam ordem de antiguidade em uma; anexado a 大, 二, 三,\dots\ para indicar classificação (e, às vezes, intimidade) | antes dos termos de parentesco; na frente de um sobrenome, de um nome próprio ou de um determinado título, com uma conotação de intimidade | em alguns contextos, pode soar infantil ou muito informal (por exemplo, chamar um colega de trabalho por ``阿 + Nome'' sem intimidade)}[阿妈 (mamãe) | 阿明 (forma carinhosa de chamar alguém chamado Ming)]
  \end{phonetics}
  \begin{phonetics}{阿}{e1}
    \definition*{s.}{sobrenome E}
    \definition*{s.}{Dong'e (um condado na província de Shandong)}
    \definition{s.}{grande monte (ou colina) | um lugar sinuoso (montanha, água, etc.)}
    \definition{v.}{bajular; satisfazer}
  \end{phonetics}
\end{entry}

\begin{entry}{阿姨}{7,9}{⾩、⼥}
  \begin{phonetics}{阿姨}{a1yi2}[][HSK 4]
    \definition[个,位]{s.}{tia; uma forma de tratamento para uma mulher da geração dos pais; dirigir-se a uma mulher que tem aproximadamente a mesma idade da sua mãe, geralmente não é parente | babá em uma família; professora em um jardim de infância | tia; irmã da mãe (mais comum no sul da China)}[阿姨,生日快乐!(Tia, feliz aniversário!) | 阿姨,这个苹果多少钱一斤?(Tia/Senhora, quanto custa o quilo dessas maçãs?) | 阿姨,我想喝水。(Tia/Babá, eu quero beber água.)]
  \end{phonetics}
\end{entry}

\begin{entry}{阿哥}{7,10}{⾩、⼝}
  \begin{phonetics}{阿哥}{a1ge1}
    \definition{s.}{irmão mais velho (afetivo)}[阿哥,帮我拿一下书包!(Irmão, ajude-me com minha mochila escolar!)]
  \end{phonetics}
\end{entry}

\begin{entry}{附件}{7,6}{⾩、⼈}
  \begin{phonetics}{附件}{fu4jian4}[][HSK 5]
    \definition*{s.}{\emph{Adnexa Uteri} ; refere-se à genitália interna feminina que não seja o útero, as trompas de falópio e os ovários}
    \definition{s.}{apêndice; documentos que acompanham o documento principal | acessório; anexo; peças ou sobressalentes que não sejam peças principais de máquinas e equipamentos | anexo; documentos ou itens relevantes emitidos com o documento}
  \end{phonetics}
\end{entry}

\begin{entry}{附近}{7,7}{⾩、⾡}
  \begin{phonetics}{附近}{fu4jin4}[][HSK 4]
    \definition{adj.}{perto; vizinho}
    \definition{s.}{vizinhança; bairro}
  \end{phonetics}
\end{entry}

\begin{entry}{陆地}{7,6}{⾩、⼟}
  \begin{phonetics}{陆地}{lu4di4}[][HSK 4]
    \definition[块,片]{s.}{terra; terra seca (em oposição ao mar); superfície da Terra, excluindo os oceanos (e, às vezes, rios e lagos)}
  \end{phonetics}
\end{entry}

\begin{entry}{陆续}{7,11}{⾩、⽷}
  \begin{phonetics}{陆续}{lu4xu4}[][HSK 4]
    \definition{adv.}{sucessivamente; um após o outro; intermitentemente}
  \end{phonetics}
\end{entry}

\begin{entry}{陆路}{7,13}{⾩、⾜}
  \begin{phonetics}{陆路}{lu4lu4}
    \definition{s.}{rota terrestre}
  \end{phonetics}
\end{entry}

\begin{entry}{饭}{7}{⾷}
  \begin{phonetics}{饭}{fan4}[][HSK 1]
    \definition{s.}{(empréstimo linguístico) fã, devoto}
    \definition[顿,碗]{s.}{cereais cozidos; grãos cozidos | refeição; alimentos consumidos diariamente em horários regulares | trabalho; meio de subsistência; meio de vida}
  \end{phonetics}
\end{entry}

\begin{entry}{饭店}{7,8}{⾷、⼴}
  \begin{phonetics}{饭店}{fan4dian4}[][HSK 1]
    \definition[家,个]{s.}{restaurante | hotel; hotel grande e bem equipado}
  \end{phonetics}
\end{entry}

\begin{entry}{饭馆}{7,11}{⾷、⾷}
  \begin{phonetics}{饭馆}{fan4 guan3}[][HSK 2]
    \definition[家,个]{s.}{restaurante; lanchonete}
  \end{phonetics}
\end{entry}

\begin{entry}{饮食}{7,9}{⾷、⾷}
  \begin{phonetics}{饮食}{yin3shi2}[][HSK 5]
    \definition{s.}{dieta; alimentos e bebidas}
    \definition{v.}{comer; beber}
  \end{phonetics}
\end{entry}

\begin{entry}{饮料}{7,10}{⾷、⽃}
  \begin{phonetics}{饮料}{yin3liao4}[][HSK 5]
    \definition[杯,瓶,种]{s.}{bebida; drinque; líquidos processados e fabricados para consumo, como vinho, chá, refrigerantes, suco de laranja, etc.}
  \end{phonetics}
\end{entry}

\begin{entry}{驱}{7}{⾺}
  \begin{phonetics}{驱}{qu1}
    \definition{v.}{expulsar | repelir}
  \end{phonetics}
\end{entry}

\begin{entry}{驴}{7}{⾺}
  \begin{phonetics}{驴}{lv2}
    \definition[头]{s.}{burro | asno | jumento | jegue}
  \end{phonetics}
\end{entry}

\begin{entry}{鸡}{7}{⿃}
  \begin{phonetics}{鸡}{ji1}[][HSK 2]
    \definition*{s.}{sobrenome Ji}
    \definition[只]{s.}{galo, galinha, frango | palavra ofensiva para uma mulher que ganha dinheiro fazendo sexo com um homem}
  \end{phonetics}
\end{entry}

\begin{entry}{鸡蛋}{7,11}{⿃、⾍}
  \begin{phonetics}{鸡蛋}{ji1dan4}[][HSK 1]
    \definition[个,枚,筐,箱,打]{s.}{ovo de galinha}
  \end{phonetics}
\end{entry}

\begin{entry}{麦当劳}{7,6,7}{⿆、⼹、⼒}
  \begin{phonetics}{麦当劳}{mai4dang1lao2}
    \definition*{s.}{McDonald's (empresa de \emph{fast-food})}
  \seealsoref{麦当劳叔叔}{mai4dang1lao2 shu1shu5}
  \end{phonetics}
\end{entry}

\begin{entry}{麦当劳叔叔}{7,6,7,8,8}{⿆、⼹、⼒、⼜、⼜}
  \begin{phonetics}{麦当劳叔叔}{mai4dang1lao2 shu1shu5}
    \definition*{s.}{Ronald McDonald}
  \seealsoref{麦当劳}{mai4dang1lao2}
  \end{phonetics}
\end{entry}

\begin{entry}{麦淇淋}{7,11,11}{⿆、⽔、⽔}
  \begin{phonetics}{麦淇淋}{mai4qi2lin2}
    \definition{s.}{(empréstimo linguístico) margarina}
  \end{phonetics}
\end{entry}

\begin{entry}{龟速}{7,10}{⿔、⾡}
  \begin{phonetics}{龟速}{gui1su4}
    \definition{adv.}{tão lento quanto uma tartaruga}
  \end{phonetics}
\end{entry}

%%%%% EOF %%%%%


%%%
%%% 8画
%%%

\section*{8画}\addcontentsline{toc}{section}{8画}

\begin{entry}{丧}{8}{⼗}
  \begin{phonetics}{丧}{sang1}
    \definition{adj.}{decepcionado; deprimido; desanimado}
    \definition{v.}{perder | desanimar; frustrar}
  \end{phonetics}
  \begin{phonetics}{丧}{sang4}
    \definition{adj.}{decepcionado | desanimado}
    \definition{v.}{estar enlutado (do cônjuge etc.) | morrer}
  \end{phonetics}
\end{entry}

\begin{entry}{丧钟}{8,9}{⼗、⾦}
  \begin{phonetics}{丧钟}{sang1zhong1}
    \definition{s.}{sentença de morte}
  \end{phonetics}
\end{entry}

\begin{entry}{乖}{8}{⼃}
  \begin{phonetics}{乖}{guai1}
    \definition{adj.}{(de uma criança) bem comportado; bom; obediente | inteligente; astuto; esperto | (de caráter, comportamento, etc.) estranho; anormal; irracional}
    \definition{v.}{perverter; ser contrário à razão; ir contra | (de caráter, comportamento, etc.) ser anormal; ser estranho}
  \end{phonetics}
\end{entry}

\begin{entry}{乖乖}{8,8}{⼃、⼃}
  \begin{phonetics}{乖乖}{guai1guai1}
    \definition{adj.}{bem-comportado (criança) | obediente}
  \end{phonetics}
  \begin{phonetics}{乖乖}{guai1guai5}
    \definition{expr.}{Graças a Deus! | Oh meu Deus!}
  \end{phonetics}
\end{entry}

\begin{entry}{乳}{8}{⼄}
  \begin{phonetics}{乳}{ru3}
    \definition{adj.}{recém-nascido (animal); lactente}
    \definition{s.}{mama; peito | leite (em geral) | qualquer líquido semelhante ao leite}
    \definition{v.}{dar à luz}
  \end{phonetics}
\end{entry}

\begin{entry}{乳房}{8,8}{⼄、⼾}
  \begin{phonetics}{乳房}{ru3fang2}
    \definition{s.}{seio | mama | úbere}
  \end{phonetics}
\end{entry}

\begin{entry}{事}{8}{⼅}
  \begin{phonetics}{事}{shi4}[][HSK 1]
    \definition[件,桩,回]{s.}{assunto; questão; coisa; negócio | problema; acidente | emprego; trabalho | responsabilidade; envolvimento | caso, coisa; o que aconteceu}
    \definition{v.}{servir; atender | estar envolvido em; dedicar-se a}
  \end{phonetics}
\end{entry}

\begin{entry}{事儿}{8,2}{⼅、⼉}
  \begin{phonetics}{事儿}{shi4r5}
    \definition[件,桩]{s.}{o emprego | negócio | afazeres | assunto que precisa ser resolvido | matéria}
  \end{phonetics}
\end{entry}

\begin{entry}{事业}{8,5}{⼅、⼀}
  \begin{phonetics}{事业}{shi4ye4}[][HSK 3]
    \definition[个]{s.}{causa; carreira; empreendimento; atividades regulares realizadas por pessoas com um determinado objetivo, escala e sistema que têm impacto no desenvolvimento social | instituição; instalações; unidade de trabalho apoiada financeiramente pelo governo; refere-se especificamente a empresas que não têm rendimentos de produção, são financiadas pelo Estado e não realizam contabilidade económica}
  \end{phonetics}
\end{entry}

\begin{entry}{事件}{8,6}{⼅、⼈}
  \begin{phonetics}{事件}{shi4jian4}[][HSK 3]
    \definition[个,件,次]{s.}{evento; incidente; grandes eventos na história ou na sociedade}
  \end{phonetics}
\end{entry}

\begin{entry}{事先}{8,6}{⼅、⼉}
  \begin{phonetics}{事先}{shi4xian1}[][HSK 4]
    \definition{adv.}{antes; de antemão; com antecedência; antecipadamente}
  \end{phonetics}
\end{entry}

\begin{entry}{事实}{8,8}{⼅、⼧}
  \begin{phonetics}{事实}{shi4shi2}[][HSK 3]
    \definition[个,件]{s.}{mito; lenda; uma narrativa sobre alguém ou algo que foi transmitida oralmente}
    \definition{v.}{dizer; contar; ser dito; contar a história}
  \end{phonetics}
\end{entry}

\begin{entry}{事实上}{8,8,3}{⼅、⼧、⼀}
  \begin{phonetics}{事实上}{shi4 shi2 shang4}[][HSK 3]
    \definition{adv.}{realmente; de fato; na realidade; na verdade; de fato}
  \end{phonetics}
\end{entry}

\begin{entry}{事物}{8,8}{⼅、⽜}
  \begin{phonetics}{事物}{shi4wu4}[][HSK 4]
    \definition{s.}{coisa; objeto; todos os objetos e fenômenos que existem objetivamente}
  \end{phonetics}
\end{entry}

\begin{entry}{事故}{8,9}{⼅、⽁}
  \begin{phonetics}{事故}{shi4gu4}[][HSK 3]
    \definition[起,桩,次,场]{s.}{acidente; perdas ou desastres repentinos, muitas vezes relacionados ao transporte, produção, trabalho e segurança pessoal}
  \end{phonetics}
\end{entry}

\begin{entry}{事情}{8,11}{⼅、⼼}
  \begin{phonetics}{事情}{shi4qing5}[][HSK 2]
    \definition[件,个,些,种]{s.}{assunto; questão; coisa; negócio | erro; acidente; infortúnio | (coloquial) emprego; trabalho}
  \end{phonetics}
\end{entry}

\begin{entry}{些}{8}{⼆}
  \begin{phonetics}{些}{xie1}[][HSK 4]
    \definition{adv.}{um pouco; um pouco mais; usado após um adjetivo ou parte de um verbo para indicar uma pequena quantidade, equivalente a 一点儿}
    \definition{clas.}{alguns; um pouco; denota uma quantidade indefinida}
  \seealsoref{一点儿}{yi4dian3r5}
  \end{phonetics}
\end{entry}

\begin{entry}{些许}{8,6}{⼆、⾔}
  \begin{phonetics}{些许}{xie1xu3}
    \definition{num.}{um pouco}
  \end{phonetics}
\end{entry}

\begin{entry}{享受}{8,8}{⼇、⼜}
  \begin{phonetics}{享受}{xiang3shou4}[][HSK 5]
    \definition[种]{s.}{prazer}
    \definition{v.}{aproveitar; desfrutar}
  \end{phonetics}
\end{entry}

\begin{entry}{京}{8}{⼇}
  \begin{phonetics}{京}{jing1}
    \definition*{s.}{sobrenome Jing}
    \definition*{s.}{Pequim (Beijing), abreviação de 北京}
    \definition{num.}{dez milhões (um numeral antigo); 10.000.000; 1000.0000}
    \definition{s.}{capital de um país}
  \seealsoref{北京}{bei3 jing1}
  \end{phonetics}
\end{entry}

\begin{entry}{京二胡}{8,2,9}{⼇、⼆、⾁}
  \begin{phonetics}{京二胡}{jing1'er4hu2}
    \definition{s.}{um tipo de violino chinês semelhante ao 二胡 de duas cordas, usado principalmente para acompanhamento do canto da ópera de Pequim | também chamado de 京胡 | jing'erhu, um violino de duas cordas, intermediário em tamanho e tom entre o 京胡 e o 二胡, usado para acompanhar a ópera chinesa}
  \seealsoref{二胡}{er4hu2}
  \seealsoref{京胡}{jing1hu2}
  \end{phonetics}
\end{entry}

\begin{entry}{京胡}{8,9}{⼇、⾁}
  \begin{phonetics}{京胡}{jing1hu2}
    \definition{s.}{jinghu, um instrumento de arco de duas cordas com registro agudo; violino da ópera de Pequim | também chamado de 京二胡 | jinghu, um 二胡 (violino de duas cordas) menor e mais agudo, usado para acompanhar a ópera chinesa}
  \seealsoref{二胡}{er4hu2}
  \seealsoref{胡琴}{hu2qin2}
  \seealsoref{京二胡}{jing1'er4hu2}
  \end{phonetics}
\end{entry}

\begin{entry}{京剧}{8,10}{⼇、⼑}
  \begin{phonetics}{京剧}{jing1ju4}[][HSK 3]
    \definition*[场,段]{s.}{Ópera de Pequim}
  \end{phonetics}
\end{entry}

\begin{entry}{佩}{8}{⼈}
  \begin{phonetics}{佩}{pei4}
    \definition{s.}{um ornamento usado como pingente amarrados em cintos nos tempos antigos}
    \definition{v.}{vestir (na cintura, etc.) | (arcaico) admirar | (arcaico) usar, especialmente uma pistola ou espada, na cintura}
  \end{phonetics}
\end{entry}

\begin{entry}{佩服}{8,8}{⼈、⽉}
  \begin{phonetics}{佩服}{pei4fu2}
    \definition{v.}{admirar}
  \end{phonetics}
\end{entry}

\begin{entry}{使}{8}{⼈}
  \begin{phonetics}{使}{shi3}[][HSK 3]
    \definition{conj.}{se; supondo; usado como a primeira cláusula de uma frase complexa; indica uma relação hipotética; equivalente a 假如}
    \definition{s.}{enviado; mensageiro; pessoas em uma missão}
    \definition{v.}{enviar; despachar; dizer a alguém para fazer algo | usar; empregar; aplicar | deixar; chamar; habilitar}
  \seealsoref{假如}{jia3ru2}
  \end{phonetics}
\end{entry}

\begin{entry}{使用}{8,5}{⼈、⽤}
  \begin{phonetics}{使用}{shi3yong4}[][HSK 2]
    \definition{v.}{usar; empregar; aplicar; fazer com que pessoas, equipamentos, fundos, etc. sirvam a um determinado propósito}
  \end{phonetics}
\end{entry}

\begin{entry}{使劲}{8,7}{⼈、⼒}
  \begin{phonetics}{使劲}{shi3 jin4}[][HSK 4]
    \definition{v.+compl.}{colocar energia; exercer toda a sua força | esforçar-se para ajudar; colocar energia para ajudar}
  \end{phonetics}
\end{entry}

\begin{entry}{使得}{8,11}{⼈、⼻}
  \begin{phonetics}{使得}{shi3 de5}[][HSK 5]
    \definition{v.}{ser utilizável; poder ser usado | ser viável; ser exequível; ser possível;  poder fazer | fazer; tornar; causar um determinado resultado (intenção, plano, coisa)}
  \end{phonetics}
\end{entry}

\begin{entry}{例}{8}{⼈}
  \begin{phonetics}{例}{li4}
    \definition{adj.}{regular; rotineiro}
    \definition{s.}{exemplo; instância | precedente | caso; instância | regras; estatutos; regulamentos}
    \definition{v.}{analogizar}
  \end{phonetics}
\end{entry}

\begin{entry}{例子}{8,3}{⼈、⼦}
  \begin{phonetics}{例子}{li4 zi5}[][HSK 2]
    \definition[个]{s.}{exemplo; algo usado para ajudar a explicar ou provar uma determinada situação ou afirmação}
  \end{phonetics}
\end{entry}

\begin{entry}{例外}{8,5}{⼈、⼣}
  \begin{phonetics}{例外}{li4wai4}[][HSK 5]
    \definition[个]{s.}{exceção; situações que não se enquadram nas regras gerais ou nas leis comuns}
    \definition{v.}{ser excepcional; ser uma exceção}
  \end{phonetics}
\end{entry}

\begin{entry}{例如}{8,6}{⼈、⼥}
  \begin{phonetics}{例如}{li4ru2}[][HSK 2]
    \definition{conj.}{por exemplo; tal como; como por exemplo; colocado antes do exemplo, indica que o exemplo vem a seguir}
  \end{phonetics}
\end{entry}

\begin{entry}{供}{8}{⼈}
  \begin{phonetics}{供}{gong1}
    \definition*{s.}{sobrenome Gong}
    \definition{v.}{fornecer; alimentar |  fornecer algo (para uso ou conveniência de); fornecer algumas condições de exploração à outra parte}
  \end{phonetics}
  \begin{phonetics}{供}{gong4}
    \definition{s.}{oferendas | confissão}
    \definition{v.}{depositar (oferendas) | confessar}
  \end{phonetics}
\end{entry}

\begin{entry}{供应}{8,7}{⼈、⼴}
  \begin{phonetics}{供应}{gong1 ying4}[][HSK 4]
    \definition{v.}{fornecer; prover de}
  \end{phonetics}
\end{entry}

\begin{entry}{依旧}{8,5}{⼈、⽇}
  \begin{phonetics}{依旧}{yi1jiu4}[][HSK 5]
    \definition{adv.}{ainda; como antes; como sempre}
  \end{phonetics}
\end{entry}

\begin{entry}{依法}{8,8}{⼈、⽔}
  \begin{phonetics}{依法}{yi1 fa3}[][HSK 5]
    \definition{adv.}{e acordo com regras (ou métodos) fixas | de acordo com a lei; por força da lei; em conformidade com as disposições legais}
  \end{phonetics}
\end{entry}

\begin{entry}{依偎}{8,11}{⼈、⼈}
  \begin{phonetics}{依偎}{yi1wei1}
    \definition{v.}{aninhar-se | aconchegar-se}
  \end{phonetics}
\end{entry}

\begin{entry}{依据}{8,11}{⼈、⼿}
  \begin{phonetics}{依据}{yi1ju4}[][HSK 5]
    \definition{prep.}{julgando por; de acordo com; à luz de; com base em; de acordo com; introduzir algo que possa servir como premissa ou base}
    \definition{s.}{base; evidência; fundamento; base para tomar uma decisão ou realizar uma ação}
    \definition{v.}{basear-se em; confiar em; depdender de; usar algo como premissa ou base}
  \end{phonetics}
\end{entry}

\begin{entry}{依然}{8,12}{⼈、⽕}
  \begin{phonetics}{依然}{yi1ran2}[][HSK 4]
    \definition{adv.}{ainda; como antes;}
    \definition{v.}{estar quieto; estar como antes; estar como o original, sem alterações}
  \end{phonetics}
\end{entry}

\begin{entry}{依照}{8,13}{⼈、⽕}
  \begin{phonetics}{依照}{yi1 zhao4}[][HSK 5]
    \definition{prep.}{de acordo com; à luz de; introduzir certos padrões para os eventos, o que equivale a 按照}
    \definition{v.}{seguir (com base em algo)}
  \seealsoref{按照}{an4zhao4}
  \end{phonetics}
\end{entry}

\begin{entry}{依靠}{8,15}{⼈、⾮}
  \begin{phonetics}{依靠}{yi1kao4}[][HSK 4]
    \definition{s.}{apoio; suporte; algo em que se apoiar; alguém ou algo em quem você pode confiar}
    \definition{v.}{depender de; confiar em (alguém ou alguma coisa para atingir um determinado objetivo)}
  \end{phonetics}
\end{entry}

\begin{entry}{兔}{8}{⼉}
  \begin{phonetics}{兔}{tu4}[][HSK 5]
    \definition[只]{s.}{lebre; coelho}
  \end{phonetics}
\end{entry}

\begin{entry}{兔子}{8,3}{⼉、⼦}
  \begin{phonetics}{兔子}{tu4zi5}
    \definition[只]{s.}{coelho | lebre}
  \end{phonetics}
\end{entry}

\begin{entry}{其}{8}{⼋}
  \begin{phonetics}{其}{qi2}[][HSK 5]
    \definition*{s.}{sobrenome Qi}
    \definition{adv.}{fazer uma suposição ou uma réplica | expressar comando, ordem}
    \definition{pron.}{dele (dela, deles, delas) | ele, ela, isso, eles; elas | isso; tal | isso (referindo-se a nenhuma pessoa ou coisa específica)}
    \definition{suf.}{sufixo de palavra, anexado ao advérbio}
  \end{phonetics}
\end{entry}

\begin{entry}{其中}{8,4}{⼋、⼁}
  \begin{phonetics}{其中}{qi2zhong1}[][HSK 2]
    \definition{pron.}{dentro; entre (os quais, eles, etc.); em (o qual, ele, etc.); nas pessoas ou coisas mencionadas anteriormente}
  \end{phonetics}
\end{entry}

\begin{entry}{其他}{8,5}{⼋、⼈}
  \begin{phonetics}{其他}{qi2ta1}[][HSK 2]
    \definition{pron.}{outra pessoa/outra coisa | outras coisas; outras pessoas; em substituição de outras pessoas ou coisas}
  \end{phonetics}
\end{entry}

\begin{entry}{其次}{8,6}{⼋、⽋}
  \begin{phonetics}{其次}{qi2ci4}[][HSK 3]
    \definition{adj.}{secundário}
    \definition{conj.}{próximo; então; em segundo lugar; mais tarde na ordem}
  \end{phonetics}
\end{entry}

\begin{entry}{其余}{8,7}{⼋、⼈}
  \begin{phonetics}{其余}{qi2yu2}[][HSK 4]
    \definition{pron.}{o restante; os outros}
  \end{phonetics}
\end{entry}

\begin{entry}{其实}{8,8}{⼋、⼧}
  \begin{phonetics}{其实}{qi2shi2}[][HSK 3]
    \definition{adv.}{na verdade; na realidade; a primeira parte é a situação aparente, e 其实 é usado para introduzir a situação real}
  \end{phonetics}
\end{entry}

\begin{entry}{具}{8}{⼋}
  \begin{phonetics}{具}{ju4}
    \definition*{s.}{sobrenome Ju}
    \definition{clas.}{(literário) usado para caixões, cadáveres e certos objetos}
    \definition{s.}{utensílio; ferramenta; implemento | capacidade; habilidade}
    \definition{v.}{possuir; ter | fornecer; prover | declarar; enumerar}
  \end{phonetics}
\end{entry}

\begin{entry}{具有}{8,6}{⼋、⽉}
  \begin{phonetics}{具有}{ju4 you3}[][HSK 3]
    \definition{v.}{ter; possuir; ser provido de}
  \end{phonetics}
\end{entry}

\begin{entry}{具体}{8,7}{⼋、⼈}
  \begin{phonetics}{具体}{ju4ti3}[][HSK 3]
    \definition{adj.}{específico; particular | concreto; específico; mais detalhado; muito detalhado; muito claro | concreto; real; não é abstrato, tem uma forma definida; pode ser visto ou sentido}
    \definition{v.}{incorporar; objetivar; combinar teorias, princípios, padrões, etc. com pessoas ou coisas específicas}
  \end{phonetics}
\end{entry}

\begin{entry}{具备}{8,8}{⼋、⼡}
  \begin{phonetics}{具备}{ju4bei4}[][HSK 4]
    \definition{v.}{ter; possuir; ser provido de}
  \end{phonetics}
\end{entry}

\begin{entry}{典}{8}{⼋}
  \begin{phonetics}{典}{dian3}
    \definition{s.}{lei; cânone; padrão; sistema; regulamentos | trabalho padrão de bolsa de estudos; livros que podem servir como padrões ou especificações | alusão; citação literária | cerimônia; uma grande cerimônia (nos tempos antigos, a etiqueta era um dos sistemas importantes do estado) | modelo; normas; regras}
    \definition{v.}{estar no comando de | hipotecar; usar imóveis ou casas como garantia ao pedir dinheiro emprestado}
  \end{phonetics}
\end{entry}

\begin{entry}{典礼}{8,5}{⼋、⽰}
  \begin{phonetics}{典礼}{dian3li3}[][HSK 5]
    \definition[个,次,场]{s.}{cerimônia; celebração; comemoração}
  \end{phonetics}
\end{entry}

\begin{entry}{典型}{8,9}{⼋、⼟}
  \begin{phonetics}{典型}{dian3xing2}[][HSK 4]
    \definition{adj.}{típico; representativo}
    \definition[个]{s.}{modelo; caso típico; indivíduo ou evento representativo | personagens típicos; personalidades modelo (em obras literárias); personagens na literatura e na arte que refletem a natureza de uma determinada sociedade e têm uma personalidade distinta}
  \end{phonetics}
\end{entry}

\begin{entry}{凭}{8}{⼏}
  \begin{phonetics}{凭}{ping2}[][HSK 5]
    \definition{prep.}{introduzir a ação ou o comportamento com base em algo; quando a frase nominal após 凭 é longa, pode-se adicionar 着 após 凭}
    \definition[张]{s.}{prova; evidência}
    \definition{v.}{apoiar-se; encostar-se | confiar em; depender de | basear-se em; tomar como base}
  \seealsoref{着}{zhe5}
  \end{phonetics}
\end{entry}

\begin{entry}{函}{8}{⼐}
  \begin{phonetics}{函}{han2}
    \definition*{s.}{sobrenome Han}
    \definition[封]{s.}{caixa; envelope; capa | carta}
  \end{phonetics}
\end{entry}

\begin{entry}{函数}{8,13}{⼐、⽁}
  \begin{phonetics}{函数}{han2shu4}
    \definition{s.}{função (matemática)}
  \end{phonetics}
\end{entry}

\begin{entry}{刮}{8}{⼑}
  \begin{phonetics}{刮}{gua1}
    \definition{v.}{barbear; raspar; depilar | untar com (pasta, etc.)  | extorquir; pilhar; adquirir avidamente (propriedade) por vários meios | (do vento) soprar}
  \end{phonetics}
\end{entry}

\begin{entry}{刮风}{8,4}{⼑、⾵}
  \begin{phonetics}{刮风}{gua1feng1}
    \definition{v.+compl.}{ventar | fazer vento}
  \end{phonetics}
\end{entry}

\begin{entry}{到}{8}{⼑}
  \begin{phonetics}{到}{dao4}[][HSK 1]
    \definition*{s.}{sobrenome Dao}
    \definition{adj.}{atencioso}
    \definition{prep.}{a; até; para; indica o tempo em que a ação ou comportamento foi alcançado}
    \definition{v.}{ir para; partir para | chegar; alcançar; chegar a | como complemento de um verbo para mostrar o resultado de uma ação}
  \end{phonetics}
\end{entry}

\begin{entry}{到处}{8,5}{⼑、⼡}
  \begin{phonetics}{到处}{dao4chu4}[][HSK 2]
    \definition{adv.}{em todos os lugares; em todos os locais; por toda parte}
  \end{phonetics}
\end{entry}

\begin{entry}{到达}{8,6}{⼑、⾡}
  \begin{phonetics}{到达}{dao4da2}[][HSK 3]
    \definition{v.}{chegar (a um determinado local, a uma determinada fase); alcançar}
  \end{phonetics}
\end{entry}

\begin{entry}{到来}{8,7}{⼑、⽊}
  \begin{phonetics}{到来}{dao4 lai2}[][HSK 5]
    \definition{v.}{chegar; chegar aqui de outro lugar}
  \end{phonetics}
\end{entry}

\begin{entry}{到底}{8,8}{⼑、⼴}
  \begin{phonetics}{到底}{dao4di3}[][HSK 3]
    \definition{adv.}{na terra (usado em frases interrogativas para expressar a determinação de alguém em encontrar uma resposta definitiva) | afinal | finalmente; por fim; no fim; indica uma situação que finalmente se concretizou após várias mudanças ou reviravoltas}
  \end{phonetics}
\end{entry}

\begin{entry}{制订}{8,4}{⼑、⾔}
  \begin{phonetics}{制订}{zhi4 ding4}[][HSK 4]
    \definition{v.}{esboçar; formular; elaborar; mapear}
  \end{phonetics}
\end{entry}

\begin{entry}{制成}{8,6}{⼑、⼽}
  \begin{phonetics}{制成}{zhi4 cheng2}[][HSK 5]
    \definition{v.}{fabricar; ser feito de; produzir}
  \end{phonetics}
\end{entry}

\begin{entry}{制约}{8,6}{⼑、⽷}
  \begin{phonetics}{制约}{zhi4yue1}[][HSK 5]
    \definition{v.}{limitar; verificar; restringir; a existência e a mudança de uma coisa determinam a existência e a mudança de outra coisa}
  \end{phonetics}
\end{entry}

\begin{entry}{制作}{8,7}{⼑、⼈}
  \begin{phonetics}{制作}{zhi4zuo4}[][HSK 3]
    \definition{v.}{fazer; produzir; itens feitos com matérias-primas, geralmente pequenos e feitos à mão | fazer; produzir; criar gráficos, anúncios, filmes, jogos, etc., utilizando texto, imagens, sons, imagens, etc.}
  \end{phonetics}
\end{entry}

\begin{entry}{制定}{8,8}{⼑、⼧}
  \begin{phonetics}{制定}{zhi4ding4}[][HSK 3]
    \definition{v.}{rascunhar; formular; elaborar; estabelecer (leis, regulamentos, planos, etc.)}
  \end{phonetics}
\end{entry}

\begin{entry}{制度}{8,9}{⼑、⼴}
  \begin{phonetics}{制度}{zhi4du4}[][HSK 3]
    \definition[项,条,套,种]{s.}{regulamentação; regulamento; procedimentos operacionais ou diretrizes de conduta que todos devem seguir | sistema; o sistema político, econômico e cultural formado sob determinadas condições históricas}
  \end{phonetics}
\end{entry}

\begin{entry}{制造}{8,10}{⼑、⾡}
  \begin{phonetics}{制造}{zhi4zao4}[][HSK 3]
    \definition{v.}{fazer; produzir; manufaturar; transformar matérias-primas em produtos acabados | criar; agitar; criar artificialmente uma situação ou atmosfera desfavorável}
  \end{phonetics}
\end{entry}

\begin{entry}{制裁}{8,12}{⼑、⾐}
  \begin{phonetics}{制裁}{zhi4cai2}
    \definition{s.}{punição | sanção (inclusive econômica)}
    \definition{v.}{punir}
  \end{phonetics}
\end{entry}

\begin{entry}{刷}{8}{⼑}
  \begin{phonetics}{刷}{shua1}[][HSK 4]
    \definition{s.}{escova; pincel | (onomatopéia) farfalhar; descreve o som de uma passagem rápida}
    \definition{v.}{escovar; esfregar; remover com uma escova | borrar; colar; aplicar com um pincel | eliminar; remover; limpar}
  \end{phonetics}
  \begin{phonetics}{刷}{shua4}
    \definition{v.}{selecionar}
  \end{phonetics}
\end{entry}

\begin{entry}{刷子}{8,3}{⼑、⼦}
  \begin{phonetics}{刷子}{shua1zi5}[][HSK 4]
    \definition[把]{s.}{escova; escovão; utensílio feito de lã, fio de plástico, fio de metal, etc., para remover sujeira ou aplicar óleo de unção, etc., geralmente longo ou oval, alguns com alças}
  \end{phonetics}
\end{entry}

\begin{entry}{刷牙}{8,4}{⼑、⽛}
  \begin{phonetics}{刷牙}{shua1ya2}[][HSK 4]
    \definition{s.}{escovar os dentes}
  \end{phonetics}
\end{entry}

\begin{entry}{刹}{8}{⼑}
  \begin{phonetics}{刹}{cha4}
    \definition*{s.}{abreviação de Kshatara, 刹多罗, sânscrito ``ksetra''}
    \definition{s.}{mosteiro, templo ou santuário budista}
  \seealsoref{刹多罗}{sha1duo1luo2}
  \end{phonetics}
  \begin{phonetics}{刹}{sha1}
    \definition{v.}{acionar o(s) freio(s); frear; brecar}
  \end{phonetics}
\end{entry}

\begin{entry}{刹多罗}{8,6,8}{⼑、⼣、⽹}
  \begin{phonetics}{刹多罗}{sha1duo1luo2}
    \definition*{s.}{Kshatara, sânscrito ``ksetra''}
  \end{phonetics}
\end{entry}

\begin{entry}{刺}{8}{⼑}
  \begin{phonetics}{刺}{ci1}
    \definition{s.}{(onomatopéia) som de rasgo, fricção, etc.}
  \end{phonetics}
  \begin{phonetics}{刺}{ci4}[][HSK 4]
    \definition*{s.}{sobrenome Ci}
    \definition{s.}{espinho; farpa; algo afiado como uma agulha | cartão de visita | saliências; projeções pequenas e pontiagudas na superfície de um objeto ou na pele de uma pessoa}
    \definition{v.}{esfaquear; perfurar | irritar; estimular | assassinar | espionar; detectar | criticar}
  \end{phonetics}
\end{entry}

\begin{entry}{刺猬}{8,12}{⼑、⽝}
  \begin{phonetics}{刺猬}{ci4wei5}
    \definition{s.}{porco-espinho | ouriço}
  \end{phonetics}
\end{entry}

\begin{entry}{刺激}{8,16}{⼑、⽔}
  \begin{phonetics}{刺激}{ci4ji1}[][HSK 4]
    \definition{adj.}{animado; entusiasmado; sensação de empolgação e nervosismo}
    \definition[个]{s.}{estímulo; estimulação; fortes efeitos físicos ou psicológicos}
    \definition{v.}{irritar; provocar; estimular | incentivar; estimular; incitar; (por algum meio) para mudar as coisas para melhor, para fazer coisas positivas}
  \end{phonetics}
\end{entry}

\begin{entry}{刻}{8}{⼑}
  \begin{phonetics}{刻}{ke4}[][HSK 2,5]
    \definition{adj.}{cruel; severo; rude; indelicado | no mais alto grau}
    \definition{clas.}{um quarto (de uma hora, 15min)}
    \definition[件]{s.}{quarto (de hora); momento}
    \definition{v.}{esculpir; inscrever; gravar; talhar com uma faca (padrões, texto, etc.) | definir um limite de tempo | imprimir (CD)}
  \end{phonetics}
\end{entry}

\begin{entry}{刻画}{8,8}{⼑、⽥}
  \begin{phonetics}{刻画}{ke4hua4}
    \definition{v.}{retratar | tirar um retrato}
  \end{phonetics}
\end{entry}

\begin{entry}{刻钟}{8,9}{⼑、⾦}
  \begin{phonetics}{刻钟}{ke4 zhong1}
    \definition{s.}{um quarto de hora}
  \end{phonetics}
\end{entry}

\begin{entry}{势}{8}{⼒}
  \begin{phonetics}{势}{shi4}
    \definition{s.}{poder; força; influência | momentum; tendência | aparência externa de um objeto natural; fenômenos ou situações naturais | situação; estado de coisas; circunstâncias | sinal; gesto | genitais masculinos}
  \end{phonetics}
\end{entry}

\begin{entry}{势力}{8,2}{⼒、⼒}
  \begin{phonetics}{势力}{shi4li4}[][HSK 5]
    \definition{s.}{força; poder; influência; forças políticas, econômicas, militares, etc.}
  \end{phonetics}
\end{entry}

\begin{entry}{单}{8}{⼗}
  \begin{phonetics}{单}{chan2}
    \definition{s.}{usado em 单于 \dpy{chan2yu2}}
  \seealsoref{单于}{chan2yu2}
  \end{phonetics}
  \begin{phonetics}{单}{dan1}[][HSK 4]
    \definition*{s.}{sobrenome Dan}
    \definition{adj.}{sozinho; único | ímpar; número ímpar (oposto a 双) | simples; poucos projetos e tipos; estrutura e ideias simples | fino; fraco; frágil}
    \definition{adv.}{isoladamente; sozinho; indica que uma ação ou coisa está dentro de um escopo limitado e não é combinada com outras; equivale a 只 ou 仅}
    \definition[个]{s.}{lençol; um único pedaço grande de pano usado para cobrir | conta; lista; pedaços de papel para anotações detalhadas (geralmente folhas soltas)}
  \seealsoref{仅}{jin3}
  \seealsoref{双}{shuang1}
  \seealsoref{只}{zhi3}
  \end{phonetics}
  \begin{phonetics}{单}{shan4}
    \definition*{s.}{sobrenome Shan}
    \definition{s.}{material de tecido de largura simples (dupla) | número singular (plural)}
  \end{phonetics}
\end{entry}

\begin{entry}{单一}{8,1}{⼗、⼀}
  \begin{phonetics}{单一}{dan1 yi1}[][HSK 5]
    \definition{adj.}{único; unitário; exclusivo}
  \end{phonetics}
\end{entry}

\begin{entry}{单于}{8,3}{⼗、⼆}
  \begin{phonetics}{单于}{chan2yu2}
    \definition{s.}{rei de Xiongnu (匈奴)}
  \seealsoref{匈奴}{xiong1nu2}
  \end{phonetics}
\end{entry}

\begin{entry}{单元}{8,4}{⼗、⼉}
  \begin{phonetics}{单元}{dan1yuan2}[][HSK 3]
    \definition[个,组,套]{s.}{unidade (de algo); um conjunto completo, com parágrafos e sistemas próprios, que forma uma unidade independente}
  \end{phonetics}
\end{entry}

\begin{entry}{单位}{8,7}{⼗、⼈}
  \begin{phonetics}{单位}{dan1wei4}[][HSK 2]
    \definition[个,家]{s.}{unidade (como padrão de medida) | unidade (como uma organização, departamento, divisão, seção, etc.) | unidade (grupo de pessoas como um todo) | unidade de trabalho (local de trabalho, especialmente na República Popular da China antes da reforma econômica)}
  \end{phonetics}
\end{entry}

\begin{entry}{单纯}{8,7}{⼗、⽷}
  \begin{phonetics}{单纯}{dan1chun2}[][HSK 4]
    \definition{adj.}{puro; simples; descomplicado}
    \definition{adv.}{sozinho; puramente; meramente}
  \end{phonetics}
\end{entry}

\begin{entry}{单质}{8,8}{⼗、⾙}
  \begin{phonetics}{单质}{dan1zhi4}
    \definition{s.}{substância simples (consistindo puramente de um elemento, como diamante, grafite, etc.)}
  \end{phonetics}
\end{entry}

\begin{entry}{单独}{8,9}{⼗、⽝}
  \begin{phonetics}{单独}{dan1du2}[][HSK 4]
    \definition{adv.}{solo; sozinho; por si mesmo; por conta própria}
  \end{phonetics}
\end{entry}

\begin{entry}{单调}{8,10}{⼗、⾔}
  \begin{phonetics}{单调}{dan1diao4}[][HSK 4]
    \definition{adj.}{maçante; monótono}
  \end{phonetics}
\end{entry}

\begin{entry}{单脚滑行车}{8,11,12,6,4}{⼗、⾁、⽔、⾏、⾞}
  \begin{phonetics}{单脚滑行车}{dan1jiao3hua2xing2che1}
    \definition{s.}{\emph{scooter}}
  \end{phonetics}
\end{entry}

\begin{entry}{卖}{8}{⼗}
  \begin{phonetics}{卖}{mai4}[][HSK 2]
    \definition*{s.}{sobrenome Mai}
    \definition{clas.}{um prato (nos tempos antigos); antigamente, os restaurantes chamavam cada prato vendido de 一卖 (uma porção)}
    \definition{v.}{vender (oposto de 买) | trair (o próprio país ou amigos); alcançar objetivos pessoais à custa dos interesses do país, da nação e dos outros | não poupar esforços; esforçar-se ao máximo; tentar fazer o máximo possível | mostrar-se intencionalmente; exibir-se | vender o próprio trabalho; trabalhar em troca de dinheiro}
  \seealsoref{买}{mai3}
  \end{phonetics}
\end{entry}

\begin{entry}{卧}{8}{⾂}
  \begin{phonetics}{卧}{wo4}
    \definition{v.}{agachar | deitar}
  \end{phonetics}
\end{entry}

\begin{entry}{卧车}{8,4}{⾂、⾞}
  \begin{phonetics}{卧车}{wo4che1}
    \definition{s.}{um carro-leito | vagão-leito}
  \end{phonetics}
\end{entry}

\begin{entry}{卧式}{8,6}{⾂、⼷}
  \begin{phonetics}{卧式}{wo4shi4}
    \definition{adj.}{horizontal}
  \end{phonetics}
\end{entry}

\begin{entry}{卧床}{8,7}{⾂、⼴}
  \begin{phonetics}{卧床}{wo4chuang2}
    \definition{adj.}{acamado}
    \definition{s.}{cama}
    \definition{v.}{deitar na cama}
  \end{phonetics}
\end{entry}

\begin{entry}{卧室}{8,9}{⾂、⼧}
  \begin{phonetics}{卧室}{wo4shi4}[][HSK 5]
    \definition[间,个]{s.}{quarto de dormir; quarto de uma casa usado para dormir}
  \end{phonetics}
\end{entry}

\begin{entry}{卧倒}{8,10}{⾂、⼈}
  \begin{phonetics}{卧倒}{wo4dao3}
    \definition{v.}{cair no chão | deitar-se}
  \end{phonetics}
\end{entry}

\begin{entry}{卧病}{8,10}{⾂、⽧}
  \begin{phonetics}{卧病}{wo4bing4}
    \definition{s.}{acamado | doente na cama}
  \end{phonetics}
\end{entry}

\begin{entry}{卧舱}{8,10}{⾂、⾈}
  \begin{phonetics}{卧舱}{wo4cang1}
    \definition{s.}{cabine de dormir em um barco ou trem}
  \end{phonetics}
\end{entry}

\begin{entry}{卧推}{8,11}{⾂、⼿}
  \begin{phonetics}{卧推}{wo4tui1}
    \definition{s.}{supino}
  \end{phonetics}
\end{entry}

\begin{entry}{卧榻}{8,14}{⾂、⽊}
  \begin{phonetics}{卧榻}{wo4ta4}
    \definition{s.}{um sofá | uma cama estreita}
  \end{phonetics}
\end{entry}

\begin{entry}{卷}{8}{⼙}
  \begin{phonetics}{卷}{juan3}[][HSK 4]
    \definition{clas.}{para pequenas coisas enroladas (maço de papel dinheiro, carretel de filme, etc.) | para rolos, carretéis, bobinas, etc.}
    \definition[张]{s.}{rolo; carretel; bobina}
    \definition{v.}{enrolar; dobrar algo em um cilindro ou semicírculo | varrer; carregar; levar junto | envolver-se; participar}
  \end{phonetics}
  \begin{phonetics}{卷}{juan4}[][HSK 4]
    \definition{clas.}{para capítulos, seções ou volumes; fascículos}
    \definition{s.}{livro; livros e pinturas que são enrolados para coleção; geralmente se refere a pinturas e caligrafia | papel de exame | arquivo; dossiê}
  \end{phonetics}
\end{entry}

\begin{entry}{厕}{8}{⼚}
  \begin{phonetics}{厕}{ce4}
    \definition[个,间]{s.}{latrina; fossa sanitária; (componente formador de palavras)}
  \seealsoref{茅厕}{mao2ce4}
  \end{phonetics}
  \begin{phonetics}{厕}{si5}
    \definition{s.}{componente formador de palavras | latrina; fossa sanitária}
  \seealsoref{茅厕}{mao2ce4}
  \end{phonetics}
\end{entry}

\begin{entry}{厕纸}{8,7}{⼚、⽷}
  \begin{phonetics}{厕纸}{ce4zhi3}
    \definition{s.}{papel higiênico}
  \end{phonetics}
\end{entry}

\begin{entry}{厕所}{8,8}{⼚、⼾}
  \begin{phonetics}{厕所}{ce4suo3}
    \definition[间,处]{s.}{lavatório | \emph{toilette}}
  \end{phonetics}
\end{entry}

\begin{entry}{参}{8}{⼛}
  \begin{phonetics}{参}{can1}
    \definition{v.}{juntar-se; entrar; tomar parte em; participar | referir; consultar; comparar com outros materiais | ligar para prestar homenagem a; fazer uma visita |  (significado antigo) acusar um funcionário perante o imperador; relatar ou expor ao imperador | explorar e compreender (verdade, significado, etc.)}
  \end{phonetics}
\end{entry}

\begin{entry}{参与}{8,3}{⼛、⼀}
  \begin{phonetics}{参与}{can1yu4}[][HSK 4]
    \definition{v.}{participar de; tomar parte em; ter uma mão em; envolver-se em; participar (no planejamento, discussão e condução dos assuntos)}
  \end{phonetics}
\end{entry}

\begin{entry}{参加}{8,5}{⼛、⼒}
  \begin{phonetics}{参加}{can1jia1}[][HSK 2]
    \definition{v.}{aderir (a organizações); participar; participar (de atividades); participar de alguma organização ou atividade | dar (conselho, sugestão, etc.)}
  \end{phonetics}
\end{entry}

\begin{entry}{参考}{8,6}{⼛、⽼}
  \begin{phonetics}{参考}{can1kao3}[][HSK 4]
    \definition{v.}{consultar; referir-se a; acessar informações relevantes para estudo ou pesquisa | consultar; referir-se a; lidar com coisas, observar, ler, aprender e usar materiais relevantes}
  \end{phonetics}
\end{entry}

\begin{entry}{参观}{8,6}{⼛、⾒}
  \begin{phonetics}{参观}{can1guan1}[][HSK 2]
    \definition{v.}{visitar; dar uma olhada; observação no local (resultados do trabalho, carreira, instalações, locais históricos e pontos turísticos, etc.)}
  \end{phonetics}
\end{entry}

\begin{entry}{叔}{8}{⼜}
  \begin{phonetics}{叔}{shu1}
    \definition*{s.}{sobrenome Shu}
    \definition{s.}{irmão mais novo do pai; tio (por parte de pai)| irmão mais novo do marido | terceiro entre irmãos | tio | uma forma de tratamento para um homem um pouco mais jovem que o pai; tio | terceiro tio (de quatro irmãos) | primo mais novo da mãe}
  \end{phonetics}
\end{entry}

\begin{entry}{叔叔}{8,8}{⼜、⼜}
  \begin{phonetics}{叔叔}{shu1shu5}
    \definition[个]{s.}{tio; irmão mais novo do pai | tio, dirigindo-se a um homem da mesma geração que o pai e mais jovem em idade}
  \end{phonetics}
\end{entry}

\begin{entry}{取}{8}{⼜}
  \begin{phonetics}{取}{qu3}[][HSK 2]
    \definition{v.}{pegar; obter; buscar; pegar de um lugar; pegar nas mãos | visar; procurar; obter; provocar | adotar; assumir; escolher; selecionar}
  \end{phonetics}
\end{entry}

\begin{entry}{取水}{8,4}{⼜、⽔}
  \begin{phonetics}{取水}{qu3shui3}
    \definition{v.}{obter água (de um poço, etc.)}
  \end{phonetics}
\end{entry}

\begin{entry}{取现}{8,8}{⼜、⾒}
  \begin{phonetics}{取现}{qu3xian4}
    \definition{v.}{sacar dinheiro}
  \end{phonetics}
\end{entry}

\begin{entry}{取胜}{8,9}{⼜、⾁}
  \begin{phonetics}{取胜}{qu3sheng4}
    \definition{v.}{prevalecer sobre os oponentes | marcar uma vitória}
  \end{phonetics}
\end{entry}

\begin{entry}{取悦}{8,10}{⼜、⼼}
  \begin{phonetics}{取悦}{qu3yue4}
    \definition{v.}{tentar agradar}
  \end{phonetics}
\end{entry}

\begin{entry}{取消}{8,10}{⼜、⽔}
  \begin{phonetics}{取消}{qu3xiao1}[][HSK 3]
    \definition{v.}{cancelar; suspender; anular; abolir; revogar; rescindir; tornar o sistema original, regulamentos, qualificações, direitos, etc. inválidos}
  \end{phonetics}
\end{entry}

\begin{entry}{取得}{8,11}{⼜、⼻}
  \begin{phonetics}{取得}{qu3 de2}[][HSK 2]
    \definition{v.}{ganhar; adquirir; obter; ser o primeiro a conseguir}
  \end{phonetics}
\end{entry}

\begin{entry}{受}{8}{⼜}
  \begin{phonetics}{受}{shou4}[][HSK 3]
    \definition{v.}{receber; aceitar | sofrer; ser submetido a | aguentar; suportar; tolerar | ser agradável}
  \end{phonetics}
\end{entry}

\begin{entry}{受不了}{8,4,2}{⼜、⼀、⼅}
  \begin{phonetics}{受不了}{shou4bu5liao3}[][HSK 4]
    \definition{adj.}{intolerável; insuportável}
    \definition{v.}{ser insuportável; não poder suportar algo; não suportar algo}
  \end{phonetics}
\end{entry}

\begin{entry}{受伤}{8,6}{⼜、⼈}
  \begin{phonetics}{受伤}{shou4shang1}[][HSK 3]
    \definition{v.}{ser ferido; sofrer uma lesão}
  \end{phonetics}
\end{entry}

\begin{entry}{受灾}{8,7}{⼜、⽕}
  \begin{phonetics}{受灾}{shou4 zai1}[][HSK 5]
    \definition{v.}{ser atingido por um desastre natural (ou calamidade) | ser atingido por uma adversidade natural}
  \end{phonetics}
\end{entry}

\begin{entry}{受到}{8,8}{⼜、⼑}
  \begin{phonetics}{受到}{shou4dao4}[][HSK 2]
    \definition{v.}{receber; receber itens, mensagens, instruções, etc. fornecidos por outras pessoas}
  \end{phonetics}
\end{entry}

\begin{entry}{受限}{8,8}{⼜、⾩}
  \begin{phonetics}{受限}{shou4xian4}
    \definition{v.}{ser limitado | ser restrito | ser constrangido}
  \end{phonetics}
\end{entry}

\begin{entry}{受得了}{8,11,2}{⼜、⼻、⼅}
  \begin{phonetics}{受得了}{shou4de5liao3}
    \definition{v.}{suportar | aguentar}
  \end{phonetics}
\end{entry}

\begin{entry}{变}{8}{⼜}
  \begin{phonetics}{变}{bian4}[][HSK 2]
    \definition{adj.}{alterado; mutável; que pode mudar; que está mudando ou já mudou}
    \definition{s.}{uma reviravolta inesperada nos acontecimentos; mudanças significativas repentinas}
    \definition{v.}{mudar; tornar-se diferente; fazer mudanças | tornar-se; transformar-se; natureza, estado ou situação diferentes dos originais | alterar; mudar; transformar}
  \end{phonetics}
\end{entry}

\begin{entry}{变为}{8,4}{⼜、⼂}
  \begin{phonetics}{变为}{bian4 wei2}[][HSK 3]
    \definition{v.}{transformar-se em; tornar-se | mudar para}
  \end{phonetics}
\end{entry}

\begin{entry}{变化}{8,4}{⼜、⼔}
  \begin{phonetics}{变化}{bian4hua4}[][HSK 3]
    \definition[个]{s.}{mudança; variação; a nova situação após uma mudança em pessoas ou coisas}
    \definition{v.}{mudar;  variar}
  \end{phonetics}
\end{entry}

\begin{entry}{变心}{8,4}{⼜、⼼}
  \begin{phonetics}{变心}{bian4xin1}
    \definition{v.+compl.}{deixar de ser fiel}
  \end{phonetics}
\end{entry}

\begin{entry}{变节}{8,5}{⼜、⾋}
  \begin{phonetics}{变节}{bian4jie2}
    \definition{s.}{traição | deserção | vira-casaca}
    \definition{v.}{mudar de lado politicamente}
  \end{phonetics}
\end{entry}

\begin{entry}{变动}{8,6}{⼜、⼒}
  \begin{phonetics}{变动}{bian4 dong4}[][HSK 5]
    \definition{v.}{mudar; alterar; oscilar; modificar; variar}
  \end{phonetics}
\end{entry}

\begin{entry}{变异}{8,6}{⼜、⼶}
  \begin{phonetics}{变异}{bian4yi4}
    \definition{s.}{variação | mutação}
  \end{phonetics}
\end{entry}

\begin{entry}{变成}{8,6}{⼜、⼽}
  \begin{phonetics}{变成}{bian4 cheng2}[][HSK 2]
    \definition{v.}{crescer; tornar-se; fazer; desenvolver-se; revelar-se; resultar; acontecer; passar a ser; passar para; acumular-se; converter-se; transformar-se; transformar-se em; mudar-se em; transformação da situação ou condição anterior para a situação ou condição atual}
  \end{phonetics}
\end{entry}

\begin{entry}{变迁}{8,6}{⼜、⾡}
  \begin{phonetics}{变迁}{bian4qian1}
    \definition{s.}{mudanças | vicissitudes}
  \end{phonetics}
\end{entry}

\begin{entry}{变更}{8,7}{⼜、⽈}
  \begin{phonetics}{变更}{bian4geng1}
    \definition{v.}{alterar | mudar | modificar}
  \end{phonetics}
\end{entry}

\begin{entry}{变性}{8,8}{⼜、⼼}
  \begin{phonetics}{变性}{bian4xing4}
    \definition{s.}{desnaturação | transexual}
    \definition{v.}{desnaturar | mudar de sexo}
  \end{phonetics}
\end{entry}

\begin{entry}{变装}{8,12}{⼜、⾐}
  \begin{phonetics}{变装}{bian4zhuang1}
    \definition{v.}{trocar de roupa | vestir-se | vestir uma fantasia | disfarçar-se ou fantasiar-se de personagem real ou ficcional, \emph{cosplay} | travestir-se}
  \end{phonetics}
\end{entry}

\begin{entry}{变数}{8,13}{⼜、⽁}
  \begin{phonetics}{变数}{bian4shu4}
    \definition{s.}{(matemática) variável}
  \end{phonetics}
\end{entry}

\begin{entry}{呢}{8}{⼝}
  \begin{phonetics}{呢}{ne5}[][HSK 1]
    \definition{part.}{usada no final de frases interrogativas (especificamente perguntas, perguntas de escolha e perguntas retóricas) para indicar um tom interrogativo | usada no final de uma frase declarativa, indica que uma ação ou situação está em andamento | usada em frases para indicar uma pausa (muitas vezes em pares) | usada no final de uma frase declarativa para confirmar um fato e convencer o interlocutor (com um tom de indicação e exagero)}
  \end{phonetics}
  \begin{phonetics}{呢}{ni2}
    \definition{s.}{(tecido feito de) lã; tecido de lã (para roupas pesadas); tecido de lã pesada; revestimento ou roupa de lã}
  \end{phonetics}
\end{entry}

\begin{entry}{周}{8}{⼝}
  \begin{phonetics}{周}{zhou1}[][HSK 2]
    \definition*{s.}{sobrenome Zhou}
    \definition*{s.}{Dinastia Zhou (1046-256 BC) | Dinastia Zhou do Norte (557-581), uma das Dinastias do Norte |  A Dinastia Zhou Posterior (951-960), uma das Cinco Dinastias}
    \definition{adj.}{universal; inteiro; por toda parte | atencioso; pensativo; completo; minucioso}
    \definition{adv.}{semanalmente}
    \definition{clas.}{usado para rodadas, voltas}
    \definition{s.}{periferia; arredores; círculo | semana | ciclo}
    \definition{v.}{fazer um circuito; mover-se em um curso circular | ajudar alguém}
  \end{phonetics}
\end{entry}

\begin{entry}{周末}{8,5}{⼝、⽊}
  \begin{phonetics}{周末}{zhou1mo4}[][HSK 2]
    \definition[个]{s.}{final-de-semana}
  \end{phonetics}
\end{entry}

\begin{entry}{周年}{8,6}{⼝、⼲}
  \begin{phonetics}{周年}{zhou1nian2}[][HSK 2]
    \definition{s.}{aniversário}
  \end{phonetics}
\end{entry}

\begin{entry}{周围}{8,7}{⼝、⼞}
  \begin{phonetics}{周围}{zhou1wei2}[][HSK 3]
    \definition{s.}{ao redor; redondeza; vizinhança; a parte ao redor do centro}
  \end{phonetics}
\end{entry}

\begin{entry}{周期}{8,12}{⼝、⽉}
  \begin{phonetics}{周期}{zhou1qi1}[][HSK 5]
    \definition{s.}{período; ciclo; no processo de mudança e movimento das coisas, certas características se repetem várias vezes, com um intervalo de tempo entre cada repetição | período; ciclo; refere-se a um processo em que certas características se repetem várias vezes, e o tempo decorrido entre duas ocorrências consecutivas | classificação dos elementos na tabela periódica}
  \end{phonetics}
\end{entry}

\begin{entry}{味}{8}{⼝}
  \begin{phonetics}{味}{wei4}
    \definition{clas.}{para medicamentos}
    \definition{s.}{cheiro | gosto}
  \end{phonetics}
\end{entry}

\begin{entry}{味儿}{8,2}{⼝、⼉}
  \begin{phonetics}{味儿}{wei4r5}[][HSK 4]
    \definition{s.}{gosto; sabor; propriedade de uma substância que dá à língua uma determinada sensação de sabor | cheiro; odor; propriedade de uma substância que dá ao nariz um determinado sentido de cheiro | interesse; significado; deleite}
  \end{phonetics}
\end{entry}

\begin{entry}{味道}{8,12}{⼝、⾡}
  \begin{phonetics}{味道}{wei4dao5}[][HSK 2]
    \definition[个,股,种]{s.}{gosto; sabor | sensação; gosto; experiência | interesse; deleite | cheiro; odor}
  \end{phonetics}
\end{entry}

\begin{entry}{呵}{8}{⼝}
  \begin{phonetics}{呵}{a1}
    \variantof{啊}
  \end{phonetics}
  \begin{phonetics}{呵}{he1}
    \definition{interj.}{Meu Deus!| Ah!; Oh!}
    \definition{v.}{expirar (com a boca aberta) | repreender}
  \end{phonetics}
\end{entry}

\begin{entry}{呶}{8}{⼝}
  \begin{phonetics}{呶}{nao2}
    \definition{interj.}{(onomatopéia) ruído alto e contínuo}
    \definition{v.}{(literário) gritar; clamar; falar ruidosamente}
  \seealsoref{努}{nu3}
  \end{phonetics}
\end{entry}

\begin{entry}{呼}{8}{⼝}
  \begin{phonetics}{呼}{hu1}
    \definition*{s.}{sobrenome Hu}
    \definition{interj.}{(onomatopéia) descreve o som do vento}
    \definition{v.}{expirar | gritar; clamar | chamar; ligar; ligar para alguém}
  \end{phonetics}
\end{entry}

\begin{entry}{呼吸}{8,6}{⼝、⼝}
  \begin{phonetics}{呼吸}{hu1xi1}[][HSK 4]
    \definition{s.}{um suspiro; metáfora para um período de tempo muito curto}
    \definition{v.}{respirar}
  \end{phonetics}
\end{entry}

\begin{entry}{呼啸}{8,11}{⼝、⼝}
  \begin{phonetics}{呼啸}{hu1xiao4}
    \definition{v.}{assobiar}
  \end{phonetics}
\end{entry}

\begin{entry}{命}{8}{⼝}
  \begin{phonetics}{命}{ming4}
    \definition[条]{s.}{vida | sorte; destino; fado | ordem; comando; instrução | atribuição de um nome, título etc.}
    \definition{v.}{ordenar; nomear | atribuir (um nome etc.)}
  \end{phonetics}
\end{entry}

\begin{entry}{命令}{8,5}{⼝、⼈}
  \begin{phonetics}{命令}{ming4ling4}[][HSK 5]
    \definition[道,个]{s.}{ordem; comando; instruções emitidas pelos superiores aos subordinados}
    \definition{v.}{ordenar; comandar}
  \end{phonetics}
\end{entry}

\begin{entry}{命运}{8,7}{⼝、⾡}
  \begin{phonetics}{命运}{ming4yun4}[][HSK 3]
    \definition[个]{s.}{tendência de desenvolvimento; tendência de futuro; metáfora para a direção e tendência do desenvolvimento e das mudanças | destino; sina; sorte; refere-se à vida e à morte, à riqueza e à pobreza e a todas as experiências da vida}
  \end{phonetics}
\end{entry}

\begin{entry}{和}{8}{⼝}
  \begin{phonetics}{和}{he2}[][HSK 1]
    \definition*{s.}{sobrenome He}
    \definition{adj.}{gentil; suave; amável | harmonioso; em boas condições}
    \definition{conj.}{e (somente para palavras); unidos com}
    \definition{prep.}{relacionado com | para; com; indica correlação; comparação, etc.}
    \definition{s.}{soma; soma total | japonês; refere-se ao Japão}
    \definition{v.}{disputar; reconciliar; acabar com a guerra ou a disputa | empatar; (próxima edição ou torneio) sem vencedor}
  \end{phonetics}
  \begin{phonetics}{和}{he4}
    \definition{v.}{compor um poema em resposta (ao poema de alguém) usando a mesma sequência de rimas | juntar-se à cantoria; cantar junto com outros em harmonia}
  \end{phonetics}
  \begin{phonetics}{和}{hu2}
    \definition{v.}{completar um conjunto de Mahjong, 麻将, ou cartas de baralho}
  \seealsoref{麻将}{ma2jiang4}
  \end{phonetics}
  \begin{phonetics}{和}{huo2}
    \definition{v.}{combinar uma substância em pó (farinha, gesso, etc.) com água; adicionar líquido ao pó e mexer ou amassar até ficar pegajoso ou espesso}
  \end{phonetics}
  \begin{phonetics}{和}{huo4}
    \definition{clas.}{usado para enxágues de roupas | usado para fervuras de ervas medicinais}
    \definition{v.}{misturar (ingredientes); misturar pós ou grãos; misturar com água para obter uma consistência mais líquida}
  \end{phonetics}
\end{entry}

\begin{entry}{和平}{8,5}{⼝、⼲}
  \begin{phonetics}{和平}{he2ping2}[][HSK 3]
    \definition{adj.}{pacífico; moderado; não violento | pacífico; tranquilo; sereno}
    \definition{s.}{paz ;ausência de guerra}
  \end{phonetics}
\end{entry}

\begin{entry}{和平共处}{8,5,6,5}{⼝、⼲、⼋、⼡}
  \begin{phonetics}{和平共处}{he2ping2gong4chu3}
    \definition{s.}{coexistência pacífica de nações, sociedades, etc.}
  \end{phonetics}
\end{entry}

\begin{entry}{和谐}{8,11}{⼝、⾔}
  \begin{phonetics}{和谐}{he2xie2}
    \definition{adj.}{harmonioso}
    \definition{s.}{harmonia}
    \definition{v.}{(eufemismo) censurar}
  \end{phonetics}
\end{entry}

\begin{entry}{咒骂}{8,9}{⼝、⾺}
  \begin{phonetics}{咒骂}{zhou4ma4}
    \definition{v.}{xingar | amaldiçoar | execrar}
  \end{phonetics}
\end{entry}

\begin{entry}{咖}{8}{⼝}
  \begin{phonetics}{咖}{ka1}
    \definition[杯]{s.}{classe | café | graduação}
  \end{phonetics}
\end{entry}

\begin{entry}{咖啡}{8,11}{⼝、⼝}
  \begin{phonetics}{咖啡}{ka1fei1}[][HSK 3]
    \definition[杯,瓶,罐,壶,包,袋,盒]{s.}{(empréstimo linguístico) café}
  \end{phonetics}
\end{entry}

\begin{entry}{咖啡色}{8,11,6}{⼝、⼝、⾊}
  \begin{phonetics}{咖啡色}{ka1fei1 se4}
    \definition{s.}{cor café}
  \end{phonetics}
\end{entry}

\begin{entry}{咖啡馆}{8,11,11}{⼝、⼝、⾷}
  \begin{phonetics}{咖啡馆}{ka1fei1guan3}
    \definition[家]{s.}{cafeteria}
  \end{phonetics}
\end{entry}

\begin{entry}{固}{8}{⼞}
  \begin{phonetics}{固}{gu4}
    \definition*{s.}{sobrenome Gu}
    \definition{adj.}{sólido; firme; forte | duro; sólido | mal informado; superficial; ignorante}
    \definition{adv.}{firmemente; resolutamente | originalmente; em primeiro lugar | certamente; reconhecidamente; seguramente}
    \definition{conj.}{usado da mesma forma que 固然}
    \definition{v.}{solidificar; consolidar; fortalecer | defender; proteger}
  \seealsoref{固然}{gu4ran2}
  \end{phonetics}
\end{entry}

\begin{entry}{固定}{8,8}{⼞、⼧}
  \begin{phonetics}{固定}{gu4ding4}[][HSK 4]
    \definition{adj.}{fixo; regular; inalterado ou imóvel}
    \definition{v.}{consertar; tornar fixo, não mover novamente; colocar as coisas em ordem, não mudá-las novamente}
  \end{phonetics}
\end{entry}

\begin{entry}{固然}{8,12}{⼞、⽕}
  \begin{phonetics}{固然}{gu4ran2}
    \definition{conj.}{usado para introduzir uma cláusula adversativa admitindo primeiro um certo fato | admitir um fato sem negar outro}
  \end{phonetics}
\end{entry}

\begin{entry}{国}{8}{⼞}
  \begin{phonetics}{国}{guo2}[][HSK 1]
    \definition*{s.}{sobrenome Guo}
    \definition{adj.}{nacional; do estado; representante do país | o melhor de um país}
    \definition[个]{s.}{estado; nação; país}
  \end{phonetics}
\end{entry}

\begin{entry}{国人}{8,2}{⼞、⼈}
  \begin{phonetics}{国人}{guo2ren2}
    \definition{s.}{compatriota}
  \end{phonetics}
\end{entry}

\begin{entry}{国内}{8,4}{⼞、⼌}
  \begin{phonetics}{国内}{guo2 nei4}[][HSK 3]
    \definition{s.}{interno (a um país); doméstico; lar; dentro de um determinado país}
  \end{phonetics}
\end{entry}

\begin{entry}{国外}{8,5}{⼞、⼣}
  \begin{phonetics}{国外}{guo2 wai4}[][HSK 1]
    \definition{adj.}{externo; no exterior; fora do país; outros lugares fora do país; geralmente chamados de exterior;  exterior não é o mesmo que estrangeiro}
  \end{phonetics}
\end{entry}

\begin{entry}{国民}{8,5}{⼞、⽒}
  \begin{phonetics}{国民}{guo2 min2}[][HSK 5]
    \definition[个]{s.}{membro de uma nação; povo de uma nação}
  \end{phonetics}
\end{entry}

\begin{entry}{国庆}{8,6}{⼞、⼴}
  \begin{phonetics}{国庆}{guo2 qing4}[][HSK 3]
    \definition*{s.}{Dia Nacional; o dia em que um país comemora sua independência ou fundação}
  \end{phonetics}
\end{entry}

\begin{entry}{国庆节}{8,6,5}{⼞、⼴、⾋}
  \begin{phonetics}{国庆节}{guo2qing4jie2}
    \definition*{s.}{Dia Nacional (1~de~outubro)}
  \end{phonetics}
\end{entry}

\begin{entry}{国际}{8,7}{⼞、⾩}
  \begin{phonetics}{国际}{guo2ji4}[][HSK 2]
    \definition{adj.}{internacional; entre países; entre nações}
    \definition{s.}{internacional; o mundo; entre nações; entre países de todo o mundo}
  \end{phonetics}
\end{entry}

\begin{entry}{国际儿童节}{8,7,2,12,5}{⼞、⾩、⼉、⽴、⾋}
  \begin{phonetics}{国际儿童节}{guo2ji4 er2tong2jie2}
    \definition*{s.}{Dia Internacional das Crianças (1~de~junho)}
  \end{phonetics}
\end{entry}

\begin{entry}{国际妇女节}{8,7,6,3,5}{⼞、⾩、⼥、⼥、⾋}
  \begin{phonetics}{国际妇女节}{guo2ji4 fu4nv3jie2}
    \definition*{s.}{Dia Internacional das Mulheres (8~de~março)}
  \end{phonetics}
\end{entry}

\begin{entry}{国际劳动节}{8,7,7,6,5}{⼞、⾩、⼒、⼒、⾋}
  \begin{phonetics}{国际劳动节}{guo2ji4 lao2dong4 jie2}
    \definition*{s.}{Dia Internacional dos Trabalhadores (1~de~maio)}
  \end{phonetics}
\end{entry}

\begin{entry}{国语}{8,9}{⼞、⾔}
  \begin{phonetics}{国语}{guo2yu3}
    \definition*{s.}{Língua Chinesa (Mandarim), enfatizando sua natureza nacional}
  \end{phonetics}
\end{entry}

\begin{entry}{国家}{8,10}{⼞、⼧}
  \begin{phonetics}{国家}{guo2jia1}[][HSK 1]
    \definition[个]{s.}{país; estado; nação; um lugar reconhecido internacionalmente e com soberania independente, incluindo as pessoas e as instituições administrativas desse lugar}
  \end{phonetics}
\end{entry}

\begin{entry}{国宾馆}{8,10,11}{⼞、⼧、⾷}
  \begin{phonetics}{国宾馆}{guo2bin1guan3}
    \definition{s.}{pousada estadual}
  \end{phonetics}
\end{entry}

\begin{entry}{国旗}{8,14}{⼞、⽅}
  \begin{phonetics}{国旗}{guo2qi2}
    \definition[面]{s.}{bandeira (de um país)}
  \end{phonetics}
\end{entry}

\begin{entry}{国歌}{8,14}{⼞、⽋}
  \begin{phonetics}{国歌}{guo2ge1}
    \definition{s.}{hino nacional}
  \end{phonetics}
\end{entry}

\begin{entry}{国籍}{8,20}{⼞、⽵}
  \begin{phonetics}{国籍}{guo2ji2}[][HSK 5]
    \definition{s.}{nacionalidade; cidadania; refere-se à identidade de um indivíduo como pertencente a um Estado | identidade nacional (de um avião, navio, etc.)}
  \end{phonetics}
\end{entry}

\begin{entry}{图}{8}{⼞}
  \begin{phonetics}{图}{tu2}[][HSK 3]
    \definition*{s.}{sobrenome Tu}
    \definition[张]{s.}{mapa; gráfico; imagem; desenho | plano; esquema; tentativa}
    \definition{v.}{procurar; perseguir; esperar obter| desenhar; retratar; pintar | imaginar; planejar; pensar; maquinar}
  \end{phonetics}
\end{entry}

\begin{entry}{图书馆}{8,4,11}{⼞、⼄、⾷}
  \begin{phonetics}{图书馆}{tu2shu1guan3}[][HSK 1]
    \definition[个,家]{s.}{biblioteca; instituição que coleta, organiza e armazena livros e materiais para leitura e consulta}
  \end{phonetics}
\end{entry}

\begin{entry}{图片}{8,4}{⼞、⽚}
  \begin{phonetics}{图片}{tu2 pian4}[][HSK 2]
    \definition[张,幅]{s.}{imagem; fotografia; um termo geral para imagens, fotografias, decalques, etc. usados para ilustrar algo}
  \end{phonetics}
\end{entry}

\begin{entry}{图画}{8,8}{⼞、⽥}
  \begin{phonetics}{图画}{tu2 hua4}[][HSK 3]
    \definition[幅,张,套]{s.}{desenho; imagem; pintura}
  \end{phonetics}
\end{entry}

\begin{entry}{图案}{8,10}{⼞、⽊}
  \begin{phonetics}{图案}{tu2'an4}[][HSK 4]
    \definition{s.}{padrão; desenho; padrões e gráficos usados para decoração de edifícios, tecidos, artes e artesanato, etc.}
  \end{phonetics}
\end{entry}

\begin{entry}{坦}{8}{⼟}
  \begin{phonetics}{坦}{tan3}
    \definition{adj.}{nivelado; suave; plano | calmo; composto | aberto; sincero; franco}
    \definition{s.}{sobrenome Tan}
  \end{phonetics}
\end{entry}

\begin{entry}{坦克}{8,7}{⼟、⼗}
  \begin{phonetics}{坦克}{tan3ke4}
    \definition{s.}{(empréstimo linguístico) tanque (veículo militar)}
  \end{phonetics}
\end{entry}

\begin{entry}{垃}{8}{⼟}
  \begin{phonetics}{垃}{la1}
    \definition[堆]{s.}{lixo}
  \end{phonetics}
\end{entry}

\begin{entry}{垃圾}{8,6}{⼟、⼟}
  \begin{phonetics}{垃圾}{la1 ji1}[][HSK 4]
    \definition{adj.}{lixo; inútil, ruim ou prejudicial}
    \definition[个]{s.}{entulho; lixo; refugo; rejeito; resíduo; coisa inútil que é jogada fora; metáfora para alguém ou algo que perdeu seu valor ou serve a um propósito ruim}
  \end{phonetics}
\end{entry}

\begin{entry}{垃圾工}{8,6,3}{⼟、⼟、⼯}
  \begin{phonetics}{垃圾工}{la1ji1gong1}
    \definition{s.}{lixeiro | gari}
  \end{phonetics}
\end{entry}

\begin{entry}{垃圾车}{8,6,4}{⼟、⼟、⾞}
  \begin{phonetics}{垃圾车}{la1ji1che1}
    \definition{s.}{caminhão de lixo}
  \end{phonetics}
\end{entry}

\begin{entry}{垃圾电邮}{8,6,5,7}{⼟、⼟、⽥、⾢}
  \begin{phonetics}{垃圾电邮}{la1ji1dian4you2}
    \definition{s.}{\emph{e-mail} de \emph{spam}}
  \end{phonetics}
\end{entry}

\begin{entry}{垃圾邮件}{8,6,7,6}{⼟、⼟、⾢、⼈}
  \begin{phonetics}{垃圾邮件}{la1ji1you2jian4}
    \definition{s.}{\emph{spam}, \emph{e-mail} não solicitado}
  \end{phonetics}
\end{entry}

\begin{entry}{垃圾食品}{8,6,9,9}{⼟、⼟、⾷、⼝}
  \begin{phonetics}{垃圾食品}{la1ji1shi2pin3}
    \definition{s.}{\emph{junk food}}
  \end{phonetics}
\end{entry}

\begin{entry}{垃圾堆}{8,6,11}{⼟、⼟、⼟}
  \begin{phonetics}{垃圾堆}{la1ji1dui1}
    \definition{s.}{depósito de lixo}
  \end{phonetics}
\end{entry}

\begin{entry}{垃圾筒}{8,6,12}{⼟、⼟、⽵}
  \begin{phonetics}{垃圾筒}{la1ji1tong3}
    \definition{s.}{cesto de lixo}
  \end{phonetics}
\end{entry}

\begin{entry}{垃圾箱}{8,6,15}{⼟、⼟、⾋}
  \begin{phonetics}{垃圾箱}{la1ji1xiang1}
    \definition{s.}{cesto de lixo}
  \end{phonetics}
\end{entry}

\begin{entry}{备}{8}{⼡}
  \begin{phonetics}{备}{bei4}
    \definition*{s.}{sobrenome Bei}
    \definition{adv.}{totalmente; de todas as maneiras possíveis | todos; tudo}
    \definition{s.}{equipamento}
    \definition{v.}{estar equipar com; ter; possuir | preparar; ficar pronto | providenciar (ou preparar) contra; tomar precauções contra}
  \end{phonetics}
\end{entry}

\begin{entry}{备份}{8,6}{⼡、⼈}
  \begin{phonetics}{备份}{bei4fen4}
    \definition{s.}{cópia de segurança | \emph{backup}}
  \end{phonetics}
\end{entry}

\begin{entry}{备胎}{8,9}{⼡、⾁}
  \begin{phonetics}{备胎}{bei4tai1}
    \definition{s.}{pneu sobressalente | (gíria) substituto}
  \end{phonetics}
\end{entry}

\begin{entry}{夜}{8}{⼣}
  \begin{phonetics}{夜}{ye4}[][HSK 2]
    \definition{s.}{noite; tarde; noturno; o período do anoitecer ao amanhecer (em oposição a 日 ou 昼); em meteorologia, refere-se especificamente ao período das 20h do dia atual às 8h do dia seguinte}
  \seealsoref{日}{ri4}
  \seealsoref{昼}{zhou4}
  \end{phonetics}
\end{entry}

\begin{entry}{夜生活}{8,5,9}{⼣、⽣、⽔}
  \begin{phonetics}{夜生活}{ye4sheng1huo2}
    \definition{s.}{vida noturna}
  \end{phonetics}
\end{entry}

\begin{entry}{夜鸟}{8,5}{⼣、⿃}
  \begin{phonetics}{夜鸟}{ye4niao3}
    \definition{s.}{ave noturna}
  \end{phonetics}
\end{entry}

\begin{entry}{夜场}{8,6}{⼣、⼟}
  \begin{phonetics}{夜场}{ye4chang3}
    \definition{s.}{show noturno (em um teatro, etc.) | local de entretenimento noturno (bar, boate, discoteca, etc.)}
  \end{phonetics}
\end{entry}

\begin{entry}{夜里}{8,7}{⼣、⾥}
  \begin{phonetics}{夜里}{ye4li5}[][HSK 2]
    \definition{s.}{noturno; à noite; o período do anoitecer ao amanhecer}
  \end{phonetics}
\end{entry}

\begin{entry}{夜间}{8,7}{⼣、⾨}
  \begin{phonetics}{夜间}{ye4 jian1}[][HSK 5]
    \definition{s.}{noite; à noite; noturno; durante a noite}
  \end{phonetics}
\end{entry}

\begin{entry}{夜夜}{8,8}{⼣、⼣}
  \begin{phonetics}{夜夜}{ye4ye4}
    \definition{adv.}{toda noite}
  \end{phonetics}
\end{entry}

\begin{entry}{夜店}{8,8}{⼣、⼴}
  \begin{phonetics}{夜店}{ye4dian4}
    \definition{s.}{boate | \emph{nightclub}}
  \end{phonetics}
\end{entry}

\begin{entry}{夜晚}{8,11}{⼣、⽇}
  \begin{phonetics}{夜晚}{ye4wan3}
    \definition[个]{s.}{noite}
  \end{phonetics}
\end{entry}

\begin{entry}{夜深人静}{8,11,2,14}{⼣、⽔、⼈、⾭}
  \begin{phonetics}{夜深人静}{ye4shen1ren2jing4}
    \definition{expr.}{``Na calada da noite.''}
  \end{phonetics}
\end{entry}

\begin{entry}{夜幕}{8,13}{⼣、⼱}
  \begin{phonetics}{夜幕}{ye4mu4}
    \definition{s.}{cortina da noite}
  \end{phonetics}
\end{entry}

\begin{entry}{奇}{8}{⼤}
  \begin{phonetics}{奇}{qi2}
    \definition{adj.}{ímpar (número); singular; solteiro; não em pares (ao contrário de 偶)}
    \definition{s.}{lotes ímpares; quantidade fracionária (acima daquela mencionada em um número redondo)}
  \seealsoref{偶}{ou3}
  \end{phonetics}
\end{entry}

\begin{entry}{奇怪}{8,8}{⼤、⼼}
  \begin{phonetics}{奇怪}{qi2guai4}[][HSK 3]
    \definition{adj.}{estranho; diferente do habitual; raramente visto, até um pouco irracional | estranho; esquisito; a descrição é diferente do imaginado e é difícil de entender}
    \definition{v.}{ficar perplexo; maravilhar-se; sentir-se surpreso; sentir-se estranho; sentir-se incompreensível}
  \end{phonetics}
\end{entry}

\begin{entry}{奇迹}{8,9}{⼤、⾡}
  \begin{phonetics}{奇迹}{qi2ji4}
    \definition{adj.}{milagroso}
    \definition{s.}{milagre}
  \end{phonetics}
\end{entry}

\begin{entry}{奋}{8}{⼤}
  \begin{phonetics}{奋}{fen4}
    \definition{adv.}{energicamente; com força e espírito}
    \definition{v.}{esforçar-se; agir vigorosamente; preparar-se | levantar | aplicar energia; resolver; animar-se | acenar; sacudir; levantar}
  \end{phonetics}
\end{entry}

\begin{entry}{奋斗}{8,4}{⼤、⽃}
  \begin{phonetics}{奋斗}{fen4dou4}[][HSK 4]
    \definition{v.}{lutar; esforçar-se; batalhar; trabalhar duro para atingir um determinado objetivo}
  \end{phonetics}
\end{entry}

\begin{entry}{奋战}{8,9}{⼤、⼽}
  \begin{phonetics}{奋战}{fen4zhan4}
    \definition{v.}{lutar bravamente | trabalhar duro}
  \end{phonetics}
\end{entry}

\begin{entry}{妹}{8}{⼥}
  \begin{phonetics}{妹}{mei4}[][HSK 1]
    \definition*{s.}{sobrenome Mei}
    \definition[个]{s.}{irmã mais nova | parente do sexo feminino da mesma geração | jovem garota; jovem mulher ou menina}
  \seealsoref{妹妹}{mei4 mei5}
  \end{phonetics}
\end{entry}

\begin{entry}{妹夫}{8,4}{⼥、⼤}
  \begin{phonetics}{妹夫}{mei4fu5}
    \definition{s.}{marido da irmã mais nova}
  \end{phonetics}
\end{entry}

\begin{entry}{妹妹}{8,8}{⼥、⼥}
  \begin{phonetics}{妹妹}{mei4 mei5}[][HSK 1]
    \definition[个]{s.}{irmã mais nova}
  \end{phonetics}
\end{entry}

\begin{entry}{妻}{8}{⼥}
  \begin{phonetics}{妻}{qi1}
    \definition{s.}{esposa}
  \end{phonetics}
  \begin{phonetics}{妻}{qi4}
    \definition{v.}{casar uma mulher com (alguém)}
  \end{phonetics}
\end{entry}

\begin{entry}{妻子}{8,3}{⼥、⼦}
  \begin{phonetics}{妻子}{qi1zi3}
    \definition{s.}{esposa e filhos; (chinês antigo) refere-se a esposas, filhos e filhas}
  \end{phonetics}
  \begin{phonetics}{妻子}{qi1zi5}[][HSK 4]
    \definition{s.}{esposa (não é usado como um termo carinhoso)}
  \end{phonetics}
\end{entry}

\begin{entry}{始}{8}{⼥}
  \begin{phonetics}{始}{shi3}
    \definition*{s.}{sobrenome Shi}
    \definition{adv.}{somente então; não\dots até}
    \definition{s.}{começo; início}
    \definition{v.}{começar; iniciar}
  \end{phonetics}
\end{entry}

\begin{entry}{始终}{8,8}{⼥、⽷}
  \begin{phonetics}{始终}{shi3zhong1}[][HSK 3]
    \definition{adv.}{sempre; o tempo todo; durante todo; do começo ao fim; indica continuidade do início ao fim}
    \definition{s.}{todo o processo do começo ao fim}
  \end{phonetics}
\end{entry}

\begin{entry}{姐}{8}{⼥}
  \begin{phonetics}{姐}{jie3}[][HSK 1]
    \definition[个,位]{s.}{irmã mais velha; irmã | termo genérico para mulheres jovens | mulheres da mesma geração que são mais velhas do que você (geralmente não inclui aquelas que podem ser chamadas de cunhadas) | um título respeitoso para mulheres jovens profissionais em determinados cargos}
  \seealsoref{姐姐}{jie3 jie5}
  \end{phonetics}
\end{entry}

\begin{entry}{姐夫}{8,4}{⼥、⼤}
  \begin{phonetics}{姐夫}{jie3fu5}
    \definition{s.}{marido da irmã mais velha}
  \end{phonetics}
\end{entry}

\begin{entry}{姐妹}{8,8}{⼥、⼥}
  \begin{phonetics}{姐妹}{jie3 mei4}[][HSK 4]
    \definition[个]{s.}{irmãs}
  \end{phonetics}
\end{entry}

\begin{entry}{姐姐}{8,8}{⼥、⼥}
  \begin{phonetics}{姐姐}{jie3 jie5}[][HSK 1]
    \definition[个]{s.}{irmã mais velha}
  \end{phonetics}
\end{entry}

\begin{entry}{姑}{8}{⼥}
  \begin{phonetics}{姑}{gu1}
    \definition{adv.}{provisoriamente; por enquanto}
    \definition[个,位,名,些]{s.}{irmã do pai; tia | irmã do marido; cunhada | mãe do marido; sogra | freira; mulher que exerce uma ocupação religiosa | a irmã do pai de alguém | mulheres jovens (no campo)}
  \end{phonetics}
\end{entry}

\begin{entry}{姑且}{8,5}{⼥、⼀}
  \begin{phonetics}{姑且}{gu1qie3}
    \definition{adv.}{provisoriamente | por enquanto}
  \end{phonetics}
\end{entry}

\begin{entry}{姑娘}{8,10}{⼥、⼥}
  \begin{phonetics}{姑娘}{gu1niang5}[][HSK 3]
    \definition[位,名,个,些]{s.}{menina; jovem senhora; mulher solteira | filha}
  \end{phonetics}
\end{entry}

\begin{entry}{姓}{8}{⼥}
  \begin{phonetics}{姓}{xing4}[][HSK 2]
    \definition[个]{s.}{sobrenome; nome de família; um caractere que representa um sistema familiar, os chineses colocam o sobrenome em primeiro lugar e o nome em segundo}
    \definition{v.}{ter como sobrenome; tratar um ou mais caracteres como sobrenome}
  \end{phonetics}
\end{entry}

\begin{entry}{姓氏}{8,4}{⼥、⽒}
  \begin{phonetics}{姓氏}{xing4shi4}
    \definition{s.}{sobrenome}
  \end{phonetics}
\end{entry}

\begin{entry}{姓名}{8,6}{⼥、⼝}
  \begin{phonetics}{姓名}{xing4ming2}[][HSK 2]
    \definition{s.}{nome; nome completo; sobrenome e nome próprio}
  \end{phonetics}
\end{entry}

\begin{entry}{委}{8}{⼥}
  \begin{phonetics}{委}{wei1}
    \definition{adj./adv.}{o mesmo que 逶 em 逶迤 sinuoso, curvo}
  \seealsoref{逶}{wei1}
  \seealsoref{逶迤}{wei1yi2}
  \end{phonetics}
  \begin{phonetics}{委}{wei3}
    \definition*{s.}{sobrenome Wei}
    \definition{adj.}{indireto; desviado | apático; abatido | sinuoso; tortuoso | desanimado; apático; sem inspiração}
    \definition{adv.}{realmente; certamente; na verdade}
    \definition{s.}{membro do comitê | comitê; comissão; conselho}
    \definition{v.}{confiar; nomear |  jogar fora; deixar de lado | culpar os outros | confiar | descartar; abandonar | mudar; empurrar | acumular}
  \end{phonetics}
\end{entry}

\begin{entry}{委内瑞拉}{8,4,13,8}{⼥、⼌、⽟、⼿}
  \begin{phonetics}{委内瑞拉}{wei3nei4rui4la1}
    \definition*{s.}{Venezuela}
  \end{phonetics}
\end{entry}

\begin{entry}{委托}{8,6}{⼥、⼿}
  \begin{phonetics}{委托}{wei3tuo1}[][HSK 5]
    \definition{v.}{confiar; confiar uma tarefa a outra pessoa ou instituição (para que seja realizada)}
  \end{phonetics}
\end{entry}

\begin{entry}{季}{8}{⼦}
  \begin{phonetics}{季}{ji4}[][HSK 4]
    \definition*{s.}{sobrenome Ji}
    \definition{s.}{estação; o ano é dividido em quatro estações, primavera, verão, outono e inverno, e uma estação dura três meses | temporada | o fim de uma era | o último mês de uma temporada | o quarto ou mais novo entre irmãos; último na ordem de precedência}
  \end{phonetics}
\end{entry}

\begin{entry}{季节}{8,5}{⼦、⾋}
  \begin{phonetics}{季节}{ji4jie2}[][HSK 4]
    \definition[个]{s.}{estação (clima); um período característico do ano}
  \end{phonetics}
\end{entry}

\begin{entry}{季度}{8,9}{⼦、⼴}
  \begin{phonetics}{季度}{ji4du4}[][HSK 4]
    \definition[个]{s.}{trimestre; período de tempo trimestral}
  \end{phonetics}
\end{entry}

\begin{entry}{孤}{8}{⼦}
  \begin{phonetics}{孤}{gu1}
    \definition*{s.}{sobrenome Gu}
    \definition{adj.}{sozinho; solitário; isolado}
    \definition{pron.}{eu; meu humilde eu (usado por príncipes feudais); título autoproclamado dos príncipes feudais}
    \definition[个,名,位]{s.}{órfão}
  \end{phonetics}
\end{entry}

\begin{entry}{孤独}{8,9}{⼦、⽝}
  \begin{phonetics}{孤独}{gu1du2}
    \definition{adj.}{solitário}
  \end{phonetics}
\end{entry}

\begin{entry}{学}{8}{⼦}
  \begin{phonetics}{学}{xue2}[][HSK 1]
    \definition[所]{s.}{aprendizagem; conhecimento; sabedoria; erudição | objeto de estudo; ramo do conhecimento | escola; faculdade | teoria; doutrina}
    \definition{v.}{estudar; aprender | imitar; copiar}
  \end{phonetics}
\end{entry}

\begin{entry}{学习}{8,3}{⼦、⼄}
  \begin{phonetics}{学习}{xue2xi2}[][HSK 1]
    \definition{s.}{estudo}
    \definition{v.}{estudar; aprender; adquirir conhecimentos ou habilidades através da leitura, da audição, da pesquisa e da prática}
  \end{phonetics}
\end{entry}

\begin{entry}{学分}{8,4}{⼦、⼑}
  \begin{phonetics}{学分}{xue2fen1}[][HSK 4]
    \definition{s.}{créditos de um curso; uma unidade de medida do peso e do tempo do curso no ensino superior; cada curso vale um crédito para uma aula por semana durante um semestre; alunos devem concluir o número necessário de créditos para se formar}
  \end{phonetics}
\end{entry}

\begin{entry}{学术}{8,5}{⼦、⽊}
  \begin{phonetics}{学术}{xue2shu4}[][HSK 4]
    \definition[个]{s.}{aprendizagem; aprendizado; ciências; aprendizado sistemático e especializado}
  \end{phonetics}
\end{entry}

\begin{entry}{学生}{8,5}{⼦、⽣}
  \begin{phonetics}{学生}{xue2sheng5}[][HSK 1]
    \definition{s.}{aluno; estudante; pupilo}
  \end{phonetics}
\end{entry}

\begin{entry}{学生证}{8,5,7}{⼦、⽣、⾔}
  \begin{phonetics}{学生证}{xue2sheng5zheng4}
    \definition{s.}{cartão de identidade de estudante}
  \end{phonetics}
\end{entry}

\begin{entry}{学会}{8,6}{⼦、⼈}
  \begin{phonetics}{学会}{xue2hui4}
    \definition{s.}{instituto | associação (acadêmica) | sociedade científica, douta ou erudita}
    \definition{v.}{aprender | dominar (um assunto)}
  \end{phonetics}
\end{entry}

\begin{entry}{学好}{8,6}{⼦、⼥}
  \begin{phonetics}{学好}{xue2hao3}
    \definition{v.}{seguir bons exemplos | aprender bem}
  \end{phonetics}
\end{entry}

\begin{entry}{学年}{8,6}{⼦、⼲}
  \begin{phonetics}{学年}{xue2 nian2}[][HSK 4]
    \definition{s.}{ano letivo; ano acadêmico}
  \end{phonetics}
\end{entry}

\begin{entry}{学问}{8,6}{⼦、⾨}
  \begin{phonetics}{学问}{xue2wen4}[][HSK 4]
    \definition[个]{s.}{aprendizado, conhecimento, erudição; a compreensão correta do mundo objetivo que alguém tem | conhecimento; aprendizado sistemático; conhecimento sistemático sobre algo ou uma ciência que pode ser aprendido em um livro ou em uma experiência prática}
  \end{phonetics}
\end{entry}

\begin{entry}{学位}{8,7}{⼦、⼈}
  \begin{phonetics}{学位}{xue2wei4}[][HSK 5]
    \definition{s.}{grau; grau acadêmico; título concedido com base no nível acadêmico profissional, como doutorado, mestrado, etc.}
  \end{phonetics}
\end{entry}

\begin{entry}{学时}{8,7}{⼦、⽇}
  \begin{phonetics}{学时}{xue2 shi2}[][HSK 4]
    \definition{s.}{hora-aula; hora de aula | período}
  \end{phonetics}
\end{entry}

\begin{entry}{学者}{8,8}{⼦、⽼}
  \begin{phonetics}{学者}{xue2 zhe3}[][HSK 5]
    \definition[位]{s.}{erudito; homem culto; pessoas que fazem pesquisas acadêmicas geralmente se referem àquelas que alcançaram certo sucesso acadêmico}
  \end{phonetics}
\end{entry}

\begin{entry}{学科}{8,9}{⼦、⽲}
  \begin{phonetics}{学科}{xue2 ke1}[][HSK 5]
    \definition{s.}{ramo do aprendizado; disciplina | disciplina escolar; curso de estudo | cursos teóricos oferecidos em treinamento militar ou físico (oposto a 术科)  | disciplina acadêmica | curso | assunto; tema}
  \seealsoref{术科}{shu4ke1}
  \end{phonetics}
\end{entry}

\begin{entry}{学费}{8,9}{⼦、⾙}
  \begin{phonetics}{学费}{xue2 fei4}[][HSK 3]
    \definition[笔]{s.}{mensalidade (taxa); prêmio; taxas que os alunos devem pagar para estudar na escola, conforme estabelecido pela escola | preço pelo que se aprendeu ao custo do próprio bolso; a metáfora do preço a pagar para obter uma determinada experiência | custo; preço; todas as despesas necessárias durante o período de estudos do aluno}
  \end{phonetics}
\end{entry}

\begin{entry}{学院}{8,9}{⼦、⾩}
  \begin{phonetics}{学院}{xue2yuan4}[][HSK 1]
    \definition[个,所]{s.}{academia; instituto; um tipo de instituição de ensino superior que se concentra em uma determinada área de especialização, como faculdades de engenharia, faculdades de música, faculdades de educação, etc.}
  \end{phonetics}
\end{entry}

\begin{entry}{学校}{8,10}{⼦、⽊}
  \begin{phonetics}{学校}{xue2xiao4}[][HSK 1]
    \definition[所,个]{s.}{escola; instituição de ensino}
  \end{phonetics}
\end{entry}

\begin{entry}{学期}{8,12}{⼦、⽉}
  \begin{phonetics}{学期}{xue2qi1}[][HSK 2]
    \definition[个,段]{s.}{semestre; período escolar; um ano acadêmico é dividido em dois semestres, um semestre do início do outono até as férias de inverno e um semestre do início da primavera até as férias de verão}
  \end{phonetics}
\end{entry}

\begin{entry}{官}{8}{⼧}
  \begin{phonetics}{官}{guan1}[][HSK 4]
    \definition*{s.}{sobrenome Guan}
    \definition{adj.}{propriedade do governo; pertencente ao governo ou ao público | público}
    \definition[个,位,名,些]{s.}{funcionário do governo; oficial; servidor público; titular de cargo; funcionário público nomeado acima de um determinado nível | órgão (parte do tecido do corpo)}
  \end{phonetics}
\end{entry}

\begin{entry}{官方}{8,4}{⼧、⽅}
  \begin{phonetics}{官方}{guan1fang1}[][HSK 4]
    \definition{s.}{autoridade; (do ou pelo) governo | oficial (de uma organização ou instituição)}
  \end{phonetics}
\end{entry}

\begin{entry}{官桂}{8,10}{⼧、⽊}
  \begin{phonetics}{官桂}{guan1gui4}
    \definition{s.}{canela}
  \seealsoref{肉桂}{rou4gui4}
  \end{phonetics}
\end{entry}

\begin{entry}{定}{8}{⼧}
  \begin{phonetics}{定}{ding4}[][HSK 4]
    \definition{adj.}{calmo; estável}
    \definition{adv.}{certamente; com certeza; definitivamente; espressa certeza ou necessidade}
    \definition{v.}{decidir; fixar; definir; determinar; ter certeza | acalmar; estabilizar; tornar estável | assinar (um jornal, etc.); reservar (assentos, ingressos, etc.); encomendar (mercadorias, etc.)}
  \end{phonetics}
\end{entry}

\begin{entry}{定期}{8,12}{⼧、⽉}
  \begin{phonetics}{定期}{ding4qi1}[][HSK 3]
    \definition{adj.}{regular; periódico; em intervalos regulares; com prazo determinado; por tempo limitado}
    \definition{v.}{fixar (definir) uma data; determinar a data; confirmar a data}
  \end{phonetics}
\end{entry}

\begin{entry}{宝}{8}{⼧}
  \begin{phonetics}{宝}{bao3}[][HSK 4]
    \definition*{s.}{sobrenome Bao}
    \definition{adj.}{antigo; precioso; estimado}
    \definition{pron.}{estimado; um termo educado usado para se referir à família, loja, etc. de alguém}
    \definition[个,件]{s.}{tesouro; objeto estimado; coisa preciosa | dinheiro; moeda; moeda antiga com furo quadrado no centro; moeda de prata}
  \end{phonetics}
\end{entry}

\begin{entry}{宝贝}{8,4}{⼧、⾙}
  \begin{phonetics}{宝贝}{bao3bei4}[][HSK 4]
    \definition{adj.}{excêntrico; estranho; imprestável; um termo depreciativo para uma pessoa incompetente ou ridícula}
    \definition[个,件]{s.}{tesouro; objeto estimado; coisa preciosa | querida; \emph{darling}; \emph{baby}; apelido para crianças}
  \end{phonetics}
\end{entry}

\begin{entry}{宝石}{8,5}{⼧、⽯}
  \begin{phonetics}{宝石}{bao3 shi2}[][HSK 4]
    \definition[颗,枚,块]{s.}{gema; jóia; pedra preciosa; mineral precioso que tem um brilho lindo e uma dureza de mais de sete graus, não é afetado pela atmosfera ou por produtos químicos e pode ser usado como decoração, suporte de instrumentos ou abrasivos}
  \end{phonetics}
\end{entry}

\begin{entry}{宝宝}{8,8}{⼧、⼧}
  \begin{phonetics}{宝宝}{bao3 bao5}[][HSK 4]
    \definition[个]{s.}{querida; \emph{darling}; \emph{baby}; apelido para crianças}
  \end{phonetics}
\end{entry}

\begin{entry}{宝贵}{8,9}{⼧、⾙}
  \begin{phonetics}{宝贵}{bao3gui4}[][HSK 4]
    \definition{adj.}{precioso; extremamente valioso, muito raro, pode ser usado para descrever coisas específicas, também pode ser usado para descrever coisas abstratas | valioso; como um tesouro}
  \end{phonetics}
\end{entry}

\begin{entry}{实}{8}{⼧}
  \begin{phonetics}{实}{shi2}
    \definition{adj.}{sólido; cheio por dentro; sem espaços vazios (oposto de 虚) | verdadeiro; real; atual; sincero | forte; eficaz; concreto; real}
    \definition{adv.}{verdadeiramente; realmente; de fato; originalmente}
    \definition{s.}{fato; realidade | semente; fruto}
    \definition{v.}{preencher}
  \seealsoref{虚}{xu1}
  \end{phonetics}
\end{entry}

\begin{entry}{实力}{8,2}{⼧、⼒}
  \begin{phonetics}{实力}{shi2li4}[][HSK 3]
    \definition{s.}{força real; geralmente se refere à força militar e econômica de um país, grupo ou indivíduo, e também se refere à capacidade de um indivíduo ou grupo em uma competição}
  \end{phonetics}
\end{entry}

\begin{entry}{实习}{8,3}{⼧、⼄}
  \begin{phonetics}{实习}{shi2xi2}[][HSK 2]
    \definition{s.}{estagiário; prática; estágio}
    \definition{v.}{aplicar e testar os conhecimentos teóricos aprendidos no trabalho prático, a fim de exercitar a capacidade profissional}
  \end{phonetics}
\end{entry}

\begin{entry}{实用}{8,5}{⼧、⽤}
  \begin{phonetics}{实用}{shi2yong4}[][HSK 4]
    \definition{adj.}{prático; pragmático; funcional; atende aos requisitos reais da aplicação}
    \definition{v.}{colocar em uso prático}
  \end{phonetics}
\end{entry}

\begin{entry}{实在}{8,6}{⼧、⼟}
  \begin{phonetics}{实在}{shi2zai4}[][HSK 2]
    \definition{adj.}{honesto; sincero | verdadeiro; honesto; realista; não é falso, não é enganador}
    \definition{adv.}{verdadeiramente; de fato; na verdade; usado para reforçar o tom afirmativo, enfatizando que a situação é realmente assim}
  \end{phonetics}
\end{entry}

\begin{entry}{实行}{8,6}{⼧、⾏}
  \begin{phonetics}{实行}{shi2xing2}[][HSK 3]
    \definition{v.}{praticar; implementar; executar; colocar em prática; realizar (programa, política, plano, etc.) por meio de ação}
  \end{phonetics}
\end{entry}

\begin{entry}{实际}{8,7}{⼧、⾩}
  \begin{phonetics}{实际}{shi2ji4}[][HSK 2]
    \definition{adj.}{real; efetivo; concreto; prático | factual; prático; realista; de acordo com os fatos}
    \definition{s.}{realidade; prática; coisas e situações que existem objetivamente}
  \end{phonetics}
\end{entry}

\begin{entry}{实际上}{8,7,3}{⼧、⾩、⼀}
  \begin{phonetics}{实际上}{shi2 ji4 shang4}[][HSK 3]
    \definition{adv.}{de fato; na verdade}
  \end{phonetics}
\end{entry}

\begin{entry}{实现}{8,8}{⼧、⾒}
  \begin{phonetics}{实现}{shi2xian4}[][HSK 2]
    \definition{v.}{alcançar; atingir; realizar; concretizar; tornar (ideais, planos, etc.) realidade}
  \end{phonetics}
\end{entry}

\begin{entry}{实施}{8,9}{⼧、⽅}
  \begin{phonetics}{实施}{shi2shi1}[][HSK 4]
    \definition{v.}{colocar em vigor; implementar (leis, políticas, etc.); executar; trazer (colocar) algo em vigor; fazer cumprir; colocar algo em (prática)}
  \end{phonetics}
\end{entry}

\begin{entry}{实验}{8,10}{⼧、⾺}
  \begin{phonetics}{实验}{shi2yan4}[][HSK 3]
    \definition[个,次]{s.}{teste; experimento; trabalho de laboratório}
    \definition{v.}{testar; experimentar; realizar uma operação ou se envolver em uma atividade para testar uma teoria ou hipótese científica}
  \end{phonetics}
\end{entry}

\begin{entry}{实验室}{8,10,9}{⼧、⾺、⼧}
  \begin{phonetics}{实验室}{shi2 yan4 shi4}[][HSK 3]
    \definition[个,间]{s.}{laboratório; salas especiais para experimentos científicos}
  \end{phonetics}
\end{entry}

\begin{entry}{实惠}{8,12}{⼧、⼼}
  \begin{phonetics}{实惠}{shi2hui4}[][HSK 5]
    \definition{adj.}{sólido; substancial; benefícios práticos}
    \definition{s.}{benefício material; benefícios tangíveis; benefícios reais}
  \end{phonetics}
\end{entry}

\begin{entry}{宠}{8}{⼧}
  \begin{phonetics}{宠}{chong3}
    \definition*{s.}{sobrenome Chong}
    \definition{v.}{mimar; estragar; conceder favor a | regalar; encontrar favor com alguém; estar nas boas graças de alguém}
  \end{phonetics}
\end{entry}

\begin{entry}{宠物}{8,8}{⼧、⽜}
  \begin{phonetics}{宠物}{chong3wu4}
    \definition{s.}{animal de estimação}
  \end{phonetics}
\end{entry}

\begin{entry}{尚}{8}{⼩}
  \begin{phonetics}{尚}{shang4}
    \definition*{s.}{sobrenome Shang}
    \definition{adv.}{ainda}
    \definition{s.}{costume predominante; refere-se à tendência predominante na sociedade; coisas que geralmente são admiradas pelas pessoas}
    \definition{v.}{valorizar; estimar; dar grande importância a}
  \end{phonetics}
\end{entry}

\begin{entry}{尚且}{8,5}{⼩、⼀}
  \begin{phonetics}{尚且}{shang4qie3}
    \definition{conj.}{até | ainda}
  \end{phonetics}
\end{entry}

\begin{entry}{尚且……何况……}{8,5,7,7}{⼩、⼀、⼈、⼎}
  \begin{phonetics}{尚且……何况……}{shang4qie3 he2kuang4}
    \definition{conj.}{ainda que\dots, \dots}
  \end{phonetics}
\end{entry}

\begin{entry}{居}{8}{⼫}
  \begin{phonetics}{居}{ju1}
    \definition*{s.}{sobrenome Ju}
    \definition{s.}{residência; casa | restaurante (em nomes de restaurantes)}
    \definition{v.}{residir; morar; viver | ocupar uma determinada posição; ocupar (um lugar); estar (em uma determinada posição) | reivindicar; afirmar | armazenar; guardar | ficar parado; estar parado}
  \end{phonetics}
\end{entry}

\begin{entry}{居民}{8,5}{⼫、⽒}
  \begin{phonetics}{居民}{ju1min2}[][HSK 4]
    \definition[个,户,位]{s.}{residente; habitante; pessoas que estão fixas em um único lugar}
  \end{phonetics}
\end{entry}

\begin{entry}{居住}{8,7}{⼫、⼈}
  \begin{phonetics}{居住}{ju1zhu4}[][HSK 4]
    \definition{v.}{viver; residir; morar; habitar}
  \end{phonetics}
\end{entry}

\begin{entry}{居然}{8,12}{⼫、⽕}
  \begin{phonetics}{居然}{ju1ran2}[][HSK 5]
    \definition{adv.}{inesperadamente; para surpresa de alguém; além da expectativa (expressão idiomática) |}
    \definition{v.}{ir tão longe a ponto de; ter a impudência de; ter o descaramento de;}
  \end{phonetics}
\end{entry}

\begin{entry}{屈}{8}{⼫}
  \begin{phonetics}{屈}{qu1}
    \definition*{s.}{sobrenome Qu}
    \definition[个]{s.}{injustiça; tratamento injusto | erro; queixa; injustiça}
    \definition{v.}{dobrar; curvar; encurvar | subjugar; submeter | tratar mal; tratar injustamente (ou deslealmente) | estar errado}
  \end{phonetics}
\end{entry}

\begin{entry}{屈原}{8,10}{⼫、⼚}
  \begin{phonetics}{屈原}{qu1yuan2}
    \definition*{s.}{Qu Yuan, poeta, é uma figura histórica famosa na cultura chinesa que viveu durante o Período dos Reinos Combatentes (340-278 a.C.).}
  \end{phonetics}
\end{entry}

\begin{entry}{届}{8}{⼫}
  \begin{phonetics}{届}{jie4}[][HSK 5]
    \definition{clas.}{sessões (de uma conferência); anos (de graduação); quantificador, ligeiramente equivalente a 次, usado para reuniões regulares ou turmas de formandos, etc.}
    \definition{v.}{vencer o prazo}
  \seealsoref{次}{ci4}
  \end{phonetics}
\end{entry}

\begin{entry}{岭}{8}{⼭}
  \begin{phonetics}{岭}{ling3}
    \definition{s.}{cordilheira}
  \end{phonetics}
\end{entry}

\begin{entry}{岸}{8}{⼭}
  \begin{phonetics}{岸}{an4}[][HSK 5]
    \definition{adj.}{arrogante; orgulhoso; grandioso (de maneira sombria ou condescendente)}
    \definition[条,道,段,面]{s.}{margem; costa; litoral; terreno à beira da água}
  \end{phonetics}
\end{entry}

\begin{entry}{岸上}{8,3}{⼭、⼀}
  \begin{phonetics}{岸上}{an4 shang4}[][HSK 5]
    \definition{s.}{em terra; costa; margem | na margem do rio; na beira do rio}
  \end{phonetics}
\end{entry}

\begin{entry}{帘}{8}{⼱}
  \begin{phonetics}{帘}{lian2}
    \definition{s.}{cortina | tela (pendurada) | bandeira usada como placa de loja}
  \end{phonetics}
\end{entry}

\begin{entry}{幷}{8}{⼲}
  \begin{phonetics}{幷}{bing4}
    \variantof{并}
  \end{phonetics}
\end{entry}

\begin{entry}{幸亏}{8,3}{⼲、⼆}
  \begin{phonetics}{幸亏}{xing4kui1}
    \definition{adv.}{felizmente}
  \end{phonetics}
\end{entry}

\begin{entry}{幸运}{8,7}{⼲、⾡}
  \begin{phonetics}{幸运}{xing4yun4}[][HSK 3]
    \definition{adj.}{sortudo; feliz; afortunado}
    \definition[个,点,丝]{s.}{boa sorte; boa fortuna}
  \end{phonetics}
\end{entry}

\begin{entry}{幸运儿}{8,7,2}{⼲、⾡、⼉}
  \begin{phonetics}{幸运儿}{xing4yun4'er2}
    \definition{s.}{pessoa de sorte}
  \end{phonetics}
\end{entry}

\begin{entry}{幸运抽奖}{8,7,8,9}{⼲、⾡、⼿、⼤}
  \begin{phonetics}{幸运抽奖}{xing4yun4chou1jiang3}
    \definition{s.}{loteria | sorteio}
  \end{phonetics}
\end{entry}

\begin{entry}{幸福}{8,13}{⼲、⽰}
  \begin{phonetics}{幸福}{xing4fu2}[][HSK 3]
    \definition{adj.}{feliz; a vida, a família e outras circunstâncias deixam as pessoas satisfeitas e felizes}
    \definition{s.}{felicidade; bem estar; sensação ou experiência satisfatória e feliz, etc.}
  \end{phonetics}
\end{entry}

\begin{entry}{底}{8}{⼴}
  \begin{phonetics}{底}{de5}
    \definition{part.}{usada após uma palavra ou frase que é usada como determinante para indicar subordinação à palavra central}
  \end{phonetics}
  \begin{phonetics}{底}{di3}[][HSK 4]
    \definition*{s.}{sobrenome Di}
    \definition{pron.}{o que? |  isto; isso; aqui | assim; tal}
    \definition{s.}{base; fundo; parte inferior de um objeto | detalhes; o cerne da questão; base, fonte ou contexto de uma coisa | rascunho; cópia mantida como registro; rascunho que pode ser usado como base | final de um ano ou mês | chão; fundo; fundação | a última parte de algo}
  \end{phonetics}
\end{entry}

\begin{entry}{底下}{8,3}{⼴、⼀}
  \begin{phonetics}{底下}{di3 xia4}[][HSK 3]
    \definition{adv.}{em baixo; abaixo; sob | próximo; mais tarde; depois; daqui para a frente}
  \end{phonetics}
\end{entry}

\begin{entry}{底气}{8,4}{⼴、⽓}
  \begin{phonetics}{底气}{di3qi4}
    \definition{s.}{capacidade pulmonar | ousadia | confiança | autoconfiança | vigor}
  \end{phonetics}
\end{entry}

\begin{entry}{店}{8}{⼴}
  \begin{phonetics}{店}{dian4}[][HSK 2]
    \definition[家,间,个]{s.}{loja; armazém; loja de venda de mercadorias | pousada; pequena pousada com instalações simples | usado para nomes de lugares}
  \end{phonetics}
\end{entry}

\begin{entry}{店主}{8,5}{⼴、⼂}
  \begin{phonetics}{店主}{dian4zhu3}
    \definition{s.}{lojista | dono de loja}
  \end{phonetics}
\end{entry}

\begin{entry}{店员}{8,7}{⼴、⼝}
  \begin{phonetics}{店员}{dian4yuan2}
    \definition{s.}{assistente de loja | balconista | vendedor}
  \end{phonetics}
\end{entry}

\begin{entry}{建}{8}{⼵}
  \begin{phonetics}{建}{jian4}[][HSK 3]
    \definition*{s.}{sobrenome Jian}
    \definition*{s.}{Província de Fujian | Rio Jian Jiang (na província de Fujian)}
    \definition{v.}{construir; construir; erigir | estabelecer; configurar; fundar | propor; defender; apresentar (suas próprias opiniões)}
  \end{phonetics}
\end{entry}

\begin{entry}{建立}{8,5}{⼵、⽴}
  \begin{phonetics}{建立}{jian4li4}[][HSK 3]
    \definition{v.}{estabelecer; construir; começar a construir | vir a ser; começar a surgir; começar a se formar}
  \end{phonetics}
\end{entry}

\begin{entry}{建立者}{8,5,8}{⼵、⽴、⽼}
  \begin{phonetics}{建立者}{jian4li4zhe3}
    \definition{s.}{fundador}
  \end{phonetics}
\end{entry}

\begin{entry}{建议}{8,5}{⼵、⾔}
  \begin{phonetics}{建议}{jian4yi4}[][HSK 3]
    \definition[个,点,条]{s.}{proposta; sugestão; recomendação; para que alguém ou alguma coisa evolua para melhor, para o coletivo; pontos de vista e opiniões apresentados pelos líderes, etc.}
    \definition{v.}{propor; sugerir; recomendar; em relação a determinada pessoa ou situação, apresentar seus pontos de vista e opiniões ao coletivo, aos líderes ou a indivíduos, para que as coisas evoluam para melhor}
  \end{phonetics}
\end{entry}

\begin{entry}{建成}{8,6}{⼵、⼽}
  \begin{phonetics}{建成}{jian4 cheng2}[][HSK 3]
    \definition{v.}{terminar a construção}
  \end{phonetics}
\end{entry}

\begin{entry}{建设}{8,6}{⼵、⾔}
  \begin{phonetics}{建设}{jian4she4}[][HSK 3]
    \definition{s.}{reconstrução; desenvolvimento; trabalhos relacionados com a construção}
    \definition{v.}{construir; edificar; (Estado ou coletividade) criar novos empreendimentos ou aumento de novas instalações}
  \end{phonetics}
\end{entry}

\begin{entry}{建设性}{8,6,8}{⼵、⾔、⼼}
  \begin{phonetics}{建设性}{jian4she4xing4}
    \definition{adj.}{construtivo}
    \definition{s.}{construtividade}
  \end{phonetics}
\end{entry}

\begin{entry}{建设者}{8,6,8}{⼵、⾔、⽼}
  \begin{phonetics}{建设者}{jian4she4zhe3}
    \definition{s.}{construtor}
  \end{phonetics}
\end{entry}

\begin{entry}{建造}{8,10}{⼵、⾡}
  \begin{phonetics}{建造}{jian4 zao4}[][HSK 5]
    \definition{adj.}{indireto; de segunda mão; ter um relacionamento por meio de um terceiro (em oposição a 直接)}
  \seealsoref{直接}{zhi2jie1}
  \end{phonetics}
\end{entry}

\begin{entry}{建筑}{8,12}{⼵、⽵}
  \begin{phonetics}{建筑}{jian4zhu4}[][HSK 5]
    \definition[座,幢,排]{s.}{construção; estrutura; edifício; prédio}
    \definition{v.}{construir; erguer; edificar; construir casas, estradas, pontes, etc.}
  \end{phonetics}
\end{entry}

\begin{entry}{廻}{8}{⼵}
  \begin{phonetics}{廻}{hui2}
    \variantof{回}
  \end{phonetics}
\end{entry}

\begin{entry}{录}{8}{⼹}
  \begin{phonetics}{录}{lu4}[][HSK 3]
    \definition{s.}{registro; cadastro; coleção; seleções}
    \definition{v.}{copiar; gravar; escrever; copiar; registrar | contratar; selecionar; empregar; adotar ou nomear | gravar em fita magnética}
  \end{phonetics}
\end{entry}

\begin{entry}{录取}{8,8}{⼹、⼜}
  \begin{phonetics}{录取}{lu4qu3}[][HSK 4]
    \definition{v.}{aceitar; admitir; recrutar; entrar; matricular (os aprovados no exame)}
  \end{phonetics}
\end{entry}

\begin{entry}{录音}{8,9}{⼹、⾳}
  \begin{phonetics}{录音}{lu4yin1}[][HSK 3]
    \definition[段,个]{s.}{gravação de som; som gravado com equipamento especializado}
    \definition{v.+compl.}{gravar; converter o som em sinal elétrico e, em seguida, gravá-lo por meios mecânicos, ópticos ou eletromagnéticos}
  \end{phonetics}
\end{entry}

\begin{entry}{录音机}{8,9,6}{⼹、⾳、⽊}
  \begin{phonetics}{录音机}{lu4yin1ji1}
    \definition[台]{s.}{gravador de áudio}
  \end{phonetics}
\end{entry}

\begin{entry}{录像机}{8,13,6}{⼹、⼈、⽊}
  \begin{phonetics}{录像机}{lu4xiang4ji1}
    \definition[台]{s.}{gravador de vídeo | VCR}
  \end{phonetics}
\end{entry}

\begin{entry}{录像带}{8,13,9}{⼹、⼈、⼱}
  \begin{phonetics}{录像带}{lu4xiang4dai4}
    \definition[盘]{s.}{video-cassete}
  \end{phonetics}
\end{entry}

\begin{entry}{彼}{8}{⼻}
  \begin{phonetics}{彼}{bi3}
    \definition{s.}{aquele; aquilo (oposto a 此) ; outro | a outra parte}
  \seealsoref{此}{ci3}
  \end{phonetics}
\end{entry}

\begin{entry}{彼此}{8,6}{⼻、⽌}
  \begin{phonetics}{彼此}{bi3ci3}[][HSK 5]
    \definition{pron.}{um ao outro; uns com os outros; este e aquele têm algum tipo de relacionamento; ambas as partes}
  \end{phonetics}
\end{entry}

\begin{entry}{往}{8}{⼻}
  \begin{phonetics}{往}{wang3}[][HSK 2]
    \definition{adj.}{passado; anterior}
    \definition{prep.}{para; em direção a; na direção de}
    \definition{v.}{ir}
  \end{phonetics}
\end{entry}

\begin{entry}{往日}{8,4}{⼻、⽇}
  \begin{phonetics}{往日}{wang3ri4}
    \definition{adv.}{dias passados}
    \definition{s.}{o passado}
  \end{phonetics}
\end{entry}

\begin{entry}{往生}{8,5}{⼻、⽣}
  \begin{phonetics}{往生}{wang3sheng1}
    \definition{v.}{renascer | morrer | (Budismo) viver no paraíso}
  \end{phonetics}
\end{entry}

\begin{entry}{往来}{8,7}{⼻、⽊}
  \begin{phonetics}{往来}{wang3lai2}
    \definition{s.}{contatos | negociações}
  \end{phonetics}
\end{entry}

\begin{entry}{往返}{8,7}{⼻、⾡}
  \begin{phonetics}{往返}{wang3fan3}
    \definition{s.}{ida e volta}
    \definition{v.}{ir e voltar | ir e vir}
  \end{phonetics}
\end{entry}

\begin{entry}{往事}{8,8}{⼻、⼅}
  \begin{phonetics}{往事}{wang3shi4}
    \definition{s.}{acontecimentos anteriores | eventos passados}
  \end{phonetics}
\end{entry}

\begin{entry}{往例}{8,8}{⼻、⼈}
  \begin{phonetics}{往例}{wang3li4}
    \definition{s.}{prática (habitual) do passado | precedente}
  \end{phonetics}
\end{entry}

\begin{entry}{往往}{8,8}{⼻、⼻}
  \begin{phonetics}{往往}{wang3wang3}[][HSK 3]
    \definition{adv.}{frequentemente; muitas vezes; mais frequentemente do que não; indica que uma situação existe ou ocorre com frequência}
  \end{phonetics}
\end{entry}

\begin{entry}{往昔}{8,8}{⼻、⽇}
  \begin{phonetics}{往昔}{wang3xi1}
    \definition{s.}{o passado}
  \end{phonetics}
\end{entry}

\begin{entry}{往复}{8,9}{⼻、⼢}
  \begin{phonetics}{往复}{wang3fu4}
    \definition{s.}{para trás e para frente (por exemplo, da ação do pistão ou da bomba)}
    \definition{v.}{ir e voltar | fazer uma viagem de volta}
  \end{phonetics}
\end{entry}

\begin{entry}{往迹}{8,9}{⼻、⾡}
  \begin{phonetics}{往迹}{wang3ji4}
    \definition{s.}{eventos passados}
  \end{phonetics}
\end{entry}

\begin{entry}{往程}{8,12}{⼻、⽲}
  \begin{phonetics}{往程}{wang3cheng2}
    \definition{s.}{saída (de uma viagem de ônibus ou trem, etc.)}
  \end{phonetics}
\end{entry}

\begin{entry}{征求}{8,7}{⼻、⽔}
  \begin{phonetics}{征求}{zheng1qiu2}[][HSK 4]
    \definition{v.}{procurar; buscar; solicitar; pedir abertamente opiniões, pontos de vista, etc.}
  \end{phonetics}
\end{entry}

\begin{entry}{征服}{8,8}{⼻、⽉}
  \begin{phonetics}{征服}{zheng1fu2}[][HSK 4]
    \definition{v.}{conquistar; cativar | subjugar; dominar}
  \end{phonetics}
\end{entry}

\begin{entry}{念}{8}{⼼}
  \begin{phonetics}{念}{nian4}[][HSK 3]
    \definition*{s.}{sobrenome Nian}
    \definition{num.}{vinte; 20; capitalização do número 廿}
    \definition{s.}{ideia; pensamento; pensamentos ou intenções internas}
    \definition{v.}{ler em voz alta | estudar; frequentar a escola | considerar; levar em conta | sentir falta; pensar em; pensar sobre; pensar frequentemente sobre}
  \seealsoref{廿}{nian4}
  \end{phonetics}
\end{entry}

\begin{entry}{忽}{8}{⼼}
  \begin{phonetics}{忽}{hu1}
    \definition*{s.}{sobrenome Hu}
    \definition{adv.}{agora\dots, agora\dots | de repente; subitamente}[天气忽冷忽热。___O clima está frio em um minuto e quente no outro.]
    \definition{v.}{negligenciar; ignorar; não prestar atenção; não levar a sério}
  \end{phonetics}
\end{entry}

\begin{entry}{忽视}{8,8}{⼼、⾒}
  \begin{phonetics}{忽视}{hu1shi4}[][HSK 4]
    \definition{v.}{ignorar; negligenciar; menosprezar; desprezar; dar de ombros}
  \end{phonetics}
\end{entry}

\begin{entry}{忽然}{8,12}{⼼、⽕}
  \begin{phonetics}{忽然}{hu1ran2}[][HSK 2]
    \definition{adv.}{repentinamente; de repente; sem aviso prévio; significa que algo aconteceu de forma rápida e inesperada}
  \end{phonetics}
\end{entry}

\begin{entry}{态}{8}{⼼}
  \begin{phonetics}{态}{tai4}
    \definition{s.}{forma; aparência; condição | (física) estado | (linguística) voz}[气态___estado gasoso | 被动态___voz passiva]
  \end{phonetics}
\end{entry}

\begin{entry}{态度}{8,9}{⼼、⼴}
  \begin{phonetics}{态度}{tai4du5}[][HSK 2]
    \definition[种,个]{s.}{maneira; comportamento; atitude; comportamento e expressão facial das pessoas | atitude; abordagem; opinião sobre o assunto e medidas tomadas}
  \end{phonetics}
\end{entry}

\begin{entry}{怕}{8}{⼼}
  \begin{phonetics}{怕}{pa4}[][HSK 2]
    \definition{adv.}{(expressando suposição, julgamento, estimativa, etc.) talvez; suponho; receio (que)}
    \definition{adv.}{por medo; talvez; suponho}
    \definition{v.}{temer; ter medo; recear; sentir medo, ficar nervoso | estar preocupado com; estar preocupado por (ou sobre); ter medo de que algo possa acontecer | ser afetado por; não conseguir suportar; não aguentar mais}
  \end{phonetics}
\end{entry}

\begin{entry}{性}{8}{⼼}
  \begin{phonetics}{性}{xing4}[][HSK 3]
    \definition[个]{s.}{natureza; caráter; personalidade | propriedade; qualidade; natureza e características das coisas | sexo; gênero | sexualidade; relacionado com a reprodução e a sexualidade | caráter; temperamento}
    \definition{suf.}{indica uma determinada propriedade ou característica de algo; segue um substantivo, verbo ou adjetivo, formando um substantivo abstrato ou um adjetivo que expressa uma propriedade}
  \end{phonetics}
\end{entry}

\begin{entry}{性生活}{8,5,9}{⼼、⽣、⽔}
  \begin{phonetics}{性生活}{xing4sheng1huo2}
    \definition{s.}{vida sexual}
  \end{phonetics}
\end{entry}

\begin{entry}{性别}{8,7}{⼼、⼑}
  \begin{phonetics}{性别}{xing4bie2}[][HSK 3]
    \definition[种]{s.}{sexo; gênero}
  \end{phonetics}
\end{entry}

\begin{entry}{性质}{8,8}{⼼、⾙}
  \begin{phonetics}{性质}{xing4zhi4}[][HSK 4]
    \definition[个,种,类]{s.}{natureza; qualidade; caráter; propriedade; propriedade fundamental que distingue uma coisa de outra}
  \end{phonetics}
\end{entry}

\begin{entry}{性侵}{8,9}{⼼、⼈}
  \begin{phonetics}{性侵}{xing4qin1}
    \definition{s.}{agressão sexual}
    \definition{v.}{agredir sexualmente}
  \end{phonetics}
\end{entry}

\begin{entry}{性格}{8,10}{⼼、⽊}
  \begin{phonetics}{性格}{xing4ge2}[][HSK 3]
    \definition[种,个]{s.}{caráter; temperamento; as características psicológicas manifestadas na atitude e no comportamento em relação às pessoas e às coisas}
  \end{phonetics}
\end{entry}

\begin{entry}{性能}{8,10}{⼼、⾁}
  \begin{phonetics}{性能}{xing4neng2}[][HSK 5]
    \definition{s.}{natureza; propriedade; desempenho; função (de uma máquina, etc.); grau de conformidade dos produtos mecânicos ou outros produtos industriais com os requisitos de projeto}
  \end{phonetics}
\end{entry}

\begin{entry}{怪}{8}{⼼}
  \begin{phonetics}{怪}{guai4}[][HSK 4,5]
    \definition*{s.}{sobrenome Guai}
    \definition{adj.}{estranho; esquisito; desconcertante | peculiar; excêntrico; pitoresco; monstruoso; anormal; incomum}
    \definition{adv.}{bastante; muito}
    \definition{s.}{monstro; demônio | diabo; ser maligno}
    \definition{v.}{culpar | achar algo estranho; maravilhar-se com; ficar surpreso | repreender; culpar; reclamar}
  \end{phonetics}
\end{entry}

\begin{entry}{怪兽}{8,11}{⼼、⼋}
  \begin{phonetics}{怪兽}{guai4shou4}
    \definition{s.}{animal raro | animal mítico | monstro}
  \end{phonetics}
\end{entry}

\begin{entry}{怪癖}{8,18}{⼼、⽧}
  \begin{phonetics}{怪癖}{guai4pi3}
    \definition{adj.}{peculiar}
    \definition{s.}{excentricidade | peculiaridade | hobby estranho}
  \end{phonetics}
\end{entry}

\begin{entry}{或}{8}{⼽}
  \begin{phonetics}{或}{huo4}[][HSK 2]
    \definition{adv.}{talvez; possivelmente; provavelmente | (geralmente na forma negativa) um pouco; ligeiramente}
    \definition{conj.}{ou (indicando escolha); ou\dots ou\dots}
    \definition{pron.}{alguém; algumas pessoas; refere-se a alguém ou algo, equivalente a 有人 ou 有的}
  \seealsoref{有的}{you3 de5}
  \seealsoref{有人}{you3 ren2}
  \end{phonetics}
\end{entry}

\begin{entry}{或许}{8,6}{⼽、⾔}
  \begin{phonetics}{或许}{huo4xu3}[][HSK 4]
    \definition{adv.}{talvez; possivelmente; receio; não tenho certeza}
  \end{phonetics}
\end{entry}

\begin{entry}{或者}{8,8}{⼽、⽼}
  \begin{phonetics}{或者}{huo4zhe3}[][HSK 2]
    \definition{adv.}{talvez; possivelmente}
    \definition{conj.}{ou (usado em expressões afirmativas); ou\dots ou\dots; usado em frases narrativas para indicar uma relação de escolha | ou (usado para indicar equação); indica relação de equivalência, indicando que os objetos anterior e posterior são iguais}
  \end{phonetics}
\end{entry}

\begin{entry}{或是}{8,9}{⼽、⽇}
  \begin{phonetics}{或是}{huo4 shi4}[][HSK 5]
    \definition{adv.}{um ou outro; o outro}
    \definition{conj.}{ou; às vezes, é apenas uma de duas coisas}
  \end{phonetics}
\end{entry}

\begin{entry}{房}{8}{⼾}
  \begin{phonetics}{房}{fang2}
    \definition*{s.}{sobrenome Fang}
    \definition*{s.}{Fang, a quarta das vinte e oito constelações nas quais a esfera celeste foi dividida, consistindo de quatro estrelas quase em linha reta em Escorpião}
    \definition[幢,个,间]{s.}{casa; edifício | sala; quarto; câmara | estrutura semelhante a uma casa | um ramo de uma família extensa | loja; estoque | local de trabalho do artesão; oficina; moinho}
  \end{phonetics}
\end{entry}

\begin{entry}{房子}{8,3}{⼾、⼦}
  \begin{phonetics}{房子}{fang2 zi5}[][HSK 1]
    \definition[栋,幢,座,套,间]{s.}{casa; edifício; prédio}
  \end{phonetics}
\end{entry}

\begin{entry}{房东}{8,5}{⼾、⼀}
  \begin{phonetics}{房东}{fang2dong1}[][HSK 3]
    \definition[个,位,名]{s.}{dono;  proprietário; senhorio; pessoas que alugam ou emprestam imóveis (para os 房客 )}
  \seealsoref{房客}{fang2ke4}
  \end{phonetics}
\end{entry}

\begin{entry}{房主}{8,5}{⼾、⼂}
  \begin{phonetics}{房主}{fang2zhu3}
    \definition{s.}{proprietário | dono de um imóvel}
  \end{phonetics}
\end{entry}

\begin{entry}{房间}{8,7}{⼾、⾨}
  \begin{phonetics}{房间}{fang2jian1}[][HSK 1]
    \definition[个,间,套]{s.}{sala; câmara; escritório; apartamento; divisões internas da casa}
  \end{phonetics}
\end{entry}

\begin{entry}{房客}{8,9}{⼾、⼧}
  \begin{phonetics}{房客}{fang2ke4}[][HSK 3]
    \definition{s.}{inquilino (de um quarto ou casa); hóspede (oposto a 房东) | inquilino; hóspede; pessoas que alugam ou emprestam imóveis para moradia (para o 房东)}
  \seealsoref{房东}{fang2dong1}
  \end{phonetics}
\end{entry}

\begin{entry}{房屋}{8,9}{⼾、⼫}
  \begin{phonetics}{房屋}{fang2 wu1}[][HSK 3]
    \definition[间,所,套]{s.}{casas; habitação; edifícios}
  \end{phonetics}
\end{entry}

\begin{entry}{房租}{8,10}{⼾、⽲}
  \begin{phonetics}{房租}{fang2 zu1}[][HSK 3]
    \definition[笔]{s.}{aluguel}
  \end{phonetics}
\end{entry}

\begin{entry}{所}{8}{⼾}
  \begin{phonetics}{所}{suo3}[][HSK 3]
    \definition*{s.}{sobrenome Suo}
    \definition{clas.}{usado para casas, etc.}
    \definition{part.}{usado com 为 ou 被 para indicar voz passiva | usado antes do verbo para formar um substantivo ou para qualificar um substantivo | usado antes do verbo na estrutura sujeito-predicado usada como complemento, indica que o termo central é o objeto}
    \definition{s.}{lugar | usado como nome de órgãos governamentais ou outros locais de trabalho}
  \seealsoref{被}{bei4}
  \seealsoref{为}{wei4}
  \end{phonetics}
\end{entry}

\begin{entry}{所以}{8,4}{⼾、⼈}
  \begin{phonetics}{所以}{suo3 yi3}[][HSK 2]
    \definition{conj.}{assim; portanto; como resultado; conecta frases, expressa resultados e costuma corresponder a expressões como 因为 e 由于}
    \definition[个]{s.}{motivo real; causa real; comportamento adequado}
  \seealsoref{因为}{yin1wei4}
  \seealsoref{由于}{you2yu2}
  \end{phonetics}
\end{entry}

\begin{entry}{所长}{8,4}{⼾、⾧}
  \begin{phonetics}{所长}{suo3 chang2}
    \definition{s.}{aquilo em que alguém é bom; o ponto forte de alguém; o forte de alguém}
  \end{phonetics}
  \begin{phonetics}{所长}{suo3 zhang3}[][HSK 3]
    \definition{s.}{chefe de um instituto, etc. | superintendente}
  \end{phonetics}
\end{entry}

\begin{entry}{所在}{8,6}{⼾、⼟}
  \begin{phonetics}{所在}{suo3 zai4}[][HSK 5]
    \definition[个]{s.}{lugar; local; localização | o lugar onde alguém ou algo está}
  \end{phonetics}
\end{entry}

\begin{entry}{所有}{8,6}{⼾、⽉}
  \begin{phonetics}{所有}{suo3you3}[][HSK 2]
    \definition{adj.}{todo | tudo}
    \definition{adj.}{tudo}
    \definition{s.}{bens; posses;}
    \definition{v.}{possuir; ter}
  \end{phonetics}
\end{entry}

\begin{entry}{承}{8}{⼿}
  \begin{phonetics}{承}{cheng2}
    \definition*{s.}{sobrenome Cheng}
    \definition{v.}{suportar; segurar; carregar; sustentar | empreender; contratar (para fazer um trabalho) | estar em dívida (com alguém por uma gentileza); receber um favor | continuar; prosseguir | receber de cima (instruções, mandato)}
  \end{phonetics}
\end{entry}

\begin{entry}{承办}{8,4}{⼿、⼒}
  \begin{phonetics}{承办}{cheng2ban4}[][HSK 5]
    \definition{v.}{empreender}
  \end{phonetics}
\end{entry}

\begin{entry}{承认}{8,4}{⼿、⾔}
  \begin{phonetics}{承认}{cheng2ren4}[][HSK 4]
    \definition{s.}{reconhecimento (diplomático, artístico, etc.)}
    \definition{v.}{admitir; reconhecer | dar reconhecimento diplomático; reconhecer}
  \end{phonetics}
\end{entry}

\begin{entry}{承受}{8,8}{⼿、⼜}
  \begin{phonetics}{承受}{cheng2shou4}[][HSK 4]
    \definition{v.}{suportar; resistir; realizar (tarefas, dificuldades, pressões, etc.); submeter-se a (testes, etc.) | herdar}
  \end{phonetics}
\end{entry}

\begin{entry}{承担}{8,8}{⼿、⼿}
  \begin{phonetics}{承担}{cheng2dan1}[][HSK 4]
    \definition{v.}{suportar; empreender; assumir; tomar conta de algo}
  \end{phonetics}
\end{entry}

\begin{entry}{披}{8}{⼿}
  \begin{phonetics}{披}{pi1}[][HSK 5]
    \definition{v.}{colocar sobre os ombros; enrolar em volta; cobrir ou colocar sobre os ombros | abrir; desenrolar; espalhar | abrir-se; rachar}
  \end{phonetics}
\end{entry}

\begin{entry}{抬}{8}{⼿}
  \begin{phonetics}{抬}{tai2}[][HSK 5]
    \definition{clas.}{para objetos que precisam ser carregados por pessoas quando transportados (por exemplo, móveis)}
    \definition{v.}{levantar; elevar; puxar para cima | (por duas ou mais pessoas) carregar; transportar; duas ou mais pessoas carregando algo com as mãos ou nos ombros | discutir, debater (geralmente sem sentido ou sem importância)}
  \end{phonetics}
\end{entry}

\begin{entry}{抬头}{8,5}{⼿、⼤}
  \begin{phonetics}{抬头}{tai2 tou2}[][HSK 5]
    \definition{s.}{(em recibos, contas, etc.) nome do comprador ou beneficiário, ou espaço para preencher esse nome | nome do comprador ou beneficiário; refere-se ao cabeçalho do documento ou da fatura}
    \definition{v.}{levantar a cabeça | ganhar terreno; olhar para cima; subir | começar uma nova linha, como sinal de respeito, ao mencionar o destinatário em cartas, correspondência oficial, etc.}
  \end{phonetics}
\end{entry}

\begin{entry}{抬杠}{8,7}{⼿、⽊}
  \begin{phonetics}{抬杠}{tai2gang4}
    \definition{v.+compl.}{discutir pelo prazer em discutir | discutir obstinadamente | brigar}
  \end{phonetics}
\end{entry}

\begin{entry}{抱}{8}{⼿}
  \begin{phonetics}{抱}{bao4}[][HSK 4]
    \definition*{s.}{sobrenome Bao}
    \definition{clas.}{braçada; medida dos dois braços}
    \definition{v.}{carregar no peito; segurar com ambos os braços; abraçar | ter o primeiro filho ou neto | adotar um bebê ou criança | ficar juntos, unidos | encaixar ou servir perfeitamente (roupas e sapatos do tamanho certo) | estimar; nutrir; abrigar; ter em mente | continuar; sobrecarregar com | chocar ovos}
  \end{phonetics}
\end{entry}

\begin{entry}{抱怨}{8,9}{⼿、⼼}
  \begin{phonetics}{抱怨}{bao4yuan4}[][HSK 5]
    \definition{v.}{reclamar ou expressar descontentamento ou insatisfação; falar com os outros sobre pessoas ou coisas com as quais você não está satisfeito}
  \end{phonetics}
\end{entry}

\begin{entry}{抵}{8}{⼿}
  \begin{phonetics}{抵}{di3}
    \definition{v.}{apoiar; sustentar | resistir; suportar | compensar; fazer o bem | hipotecar; dar como garantia; garantir | equilibrar; cancelar; compensar | ser igual a; corresponder | alcançar; chegar a | colidir; dar cabeçada (por animais com chifres)}
  \end{phonetics}
\end{entry}

\begin{entry}{抵抗}{8,7}{⼿、⼿}
  \begin{phonetics}{抵抗}{di3kang4}
    \definition{s.}{resistência}
    \definition{v.}{resistir}
  \end{phonetics}
\end{entry}

\begin{entry}{抹}{8}{⼿}
  \begin{phonetics}{抹}{ma1}
    \definition{v.}{esfregar; limpar | deslizar algo para fora; tirar}
  \end{phonetics}
  \begin{phonetics}{抹}{mo3}
    \definition{v.}{colocar; aplicar; untar; engessar | limpar | anular; apagar | (para nuvem, etc.) irradiar; raiar; riscar; traçar | riscar; cancelar; marcar; remover; excluir}
  \end{phonetics}
  \begin{phonetics}{抹}{mo4}
    \definition{v.}{rebocar; engessar; alisar a massa ou o gesso com uma espátula | virar; contornar; dar uma volta de perto}
  \end{phonetics}
\end{entry}

\begin{entry}{抹泪}{8,8}{⼿、⽔}
  \begin{phonetics}{抹泪}{mo3lei4}
    \definition{v.}{limpar as lágrimas | (figurativo) derramar lágrimas}
  \end{phonetics}
\end{entry}

\begin{entry}{押}{8}{⼿}
  \begin{phonetics}{押}{ya1}
    \definition{v.}{deter sob custódia | escoltar e proteger | hipotecar | penhorar}
  \end{phonetics}
\end{entry}

\begin{entry}{押后}{8,6}{⼿、⼝}
  \begin{phonetics}{押后}{ya1hou4}
    \definition{v.}{encerrar | adiar}
  \end{phonetics}
\end{entry}

\begin{entry}{押运}{8,7}{⼿、⾡}
  \begin{phonetics}{押运}{ya1yun4}
    \definition{v.}{escoltar sob guarda | escoltar (bens ou fundos)}
  \end{phonetics}
\end{entry}

\begin{entry}{押注}{8,8}{⼿、⽔}
  \begin{phonetics}{押注}{ya1zhu4}
    \definition{v.}{apostar}
  \end{phonetics}
\end{entry}

\begin{entry}{押金}{8,8}{⼿、⾦}
  \begin{phonetics}{押金}{ya1jin1}[][HSK 5]
    \definition[笔,份,些]{s.}{caução; sinal; depósito; dinheiro como garantia}
  \end{phonetics}
\end{entry}

\begin{entry}{押送}{8,9}{⼿、⾡}
  \begin{phonetics}{押送}{ya1song4}
    \definition{v.}{enviar sob escolta | transportar um detido}
  \end{phonetics}
\end{entry}

\begin{entry}{押租}{8,10}{⼿、⽲}
  \begin{phonetics}{押租}{ya1zu1}
    \definition{s.}{depósito de aluguel}
  \end{phonetics}
\end{entry}

\begin{entry}{押韵}{8,13}{⼿、⾳}
  \begin{phonetics}{押韵}{ya1yun4}
    \definition{v.}{rimar}
  \end{phonetics}
\end{entry}

\begin{entry}{抽}{8}{⼿}
  \begin{phonetics}{抽}{chou1}[][HSK 4]
    \definition{v.}{retirar; tirar (do meio); retirar, puxar ou arrancar algo que está preso ou emaranhado em outra coisa | tirar, retirar (uma parte de um todo) | (certas plantas) começar a crescer, produzir | bombear | encolher; contrair | chicotear; açoitar; surrar | dirigir; conduzir | encontrar tempo; libertar-se; sair de alguma coisa}
  \end{phonetics}
\end{entry}

\begin{entry}{抽奖}{8,9}{⼿、⼤}
  \begin{phonetics}{抽奖}{chou1 jiang3}[][HSK 4]
    \definition{s.}{loteria; sorteio de loteria}
  \end{phonetics}
\end{entry}

\begin{entry}{抽烟}{8,10}{⼿、⽕}
  \begin{phonetics}{抽烟}{chou1yan1}[][HSK 4]
    \definition{v.+compl.}{fumar (um cigarro ou um cachimbo)}
  \end{phonetics}
\end{entry}

\begin{entry}{担}{8}{⼿}
  \begin{phonetics}{担}{dan1}
    \definition{v.}{carregar em uma vara de ombro e baldes; carregar nos ombros | assumir; empreender; não ter medo de correr riscos}
  \end{phonetics}
  \begin{phonetics}{担}{dan4}
    \definition{clas.}{dan, uma unidade de peso (=50 quilogramas) ; 100 jin = 1 dan | usado em coisas usadas para transportar cargas}
    \definition{s.}{carga; fardo; cargas de mercadorias transportadas em uma vara de ombro por um mascate itinerante}
  \end{phonetics}
\end{entry}

\begin{entry}{担心}{8,4}{⼿、⼼}
  \begin{phonetics}{担心}{dan1xin1}[][HSK 4]
    \definition{v.}{preocupar-se; ficar ansioso; sentir-se desconfortável com algo}
  \end{phonetics}
\end{entry}

\begin{entry}{担任}{8,6}{⼿、⼈}
  \begin{phonetics}{担任}{dan1ren4}[][HSK 4]
    \definition{v.}{servir como; assumir o cargo de; ocupar o posto de; ocupar um determinado cargo ou emprego}
  \end{phonetics}
\end{entry}

\begin{entry}{担保}{8,9}{⼿、⼈}
  \begin{phonetics}{担保}{dan1bao3}[][HSK 4]
    \definition{v.}{garantir; atestar; expressar responsabilidade e garantir que não haverá problemas ou que eles serão resolvidos}
  \end{phonetics}
\end{entry}

\begin{entry}{拆}{8}{⼿}
  \begin{phonetics}{拆}{chai1}[][HSK 5]
    \definition{v.}{rasgar; desmontar; separar o que está unido | derrubar; desmantelar; demolir; refere-se especificamente à demolição de edifícios}
  \end{phonetics}
\end{entry}

\begin{entry}{拆除}{8,9}{⼿、⾩}
  \begin{phonetics}{拆除}{chai1 chu2}[][HSK 5]
    \definition{v.}{desmantelar; demolir; derrubar; remover (um edifício, etc.)}
  \end{phonetics}
\end{entry}

\begin{entry}{拉}{8}{⼿}
  \begin{phonetics}{拉}{la1}[][HSK 2]
    \definition{s.}{abreviação de América Latina, 拉丁美洲}
    \definition{v.}{puxar; arrastar; rebocar | transportar por veículo; rebocar | arrastar (ou puxar) para fora | mover (tropas para um lugar) | dar uma mãozinha; ajudar | arrastar para dentro; implicar; envolver | criar (criança) | atrair; conquistar; solicitar; angariar votos | bater-papo | organizar; preparar | ter dívidas; estar endividado | pressionar; recrutar à força | (no tênis, tênis de mesa, etc.) levantar (a bola) | tocar (certos instrumentos musicais); puxar uma parte do instrumento para que ele emita som | prolongar; espaçar | envolver-se em | (coloquial) esvaziar os intestinos | levantar, uma das técnicas do tênis de mesa | destruir; esmagar; quebrar}
  \seealsoref{拉丁美洲}{la1ding1 mei3zhou1}
  \end{phonetics}
  \begin{phonetics}{拉}{la4}
    \definition{s.}{usado em 拉拉蛄 \dpy{la4la4gu3}}
  \seealsoref{拉拉蛄}{la4la4gu3}
  \end{phonetics}
\end{entry}

\begin{entry}{拉丁美洲}{8,2,9,9}{⼿、⼀、⽺、⽔}
  \begin{phonetics}{拉丁美洲}{la1ding1 mei3zhou1}
    \definition*{s.}{América Latina; o nome coletivo dos países da América Central e do Sul é ``América Latina'', devido ao fato de a maioria de seus habitantes ser descendente de povos latinos e de a língua falada ser do grupo latino}
  \end{phonetics}
\end{entry}

\begin{entry}{拉开}{8,4}{⼿、⼶}
  \begin{phonetics}{拉开}{la1 kai1}[][HSK 4]
    \definition{v.}{puxar para abrir; recuar| ampliar; espaçar; distanciar; afastar; separar}
  \end{phonetics}
\end{entry}

\begin{entry}{拉拉队}{8,8,4}{⼿、⼿、⾩}
  \begin{phonetics}{拉拉队}{la1la1dui4}
    \definition{s.}{claque | torcida}
  \end{phonetics}
\end{entry}

\begin{entry}{拉拉蛄}{8,8,11}{⼿、⼿、⾍}
  \begin{phonetics}{拉拉蛄}{la4la4gu3}
    \variantof{蝲蝲蛄}
  \end{phonetics}
\end{entry}

\begin{entry}{拍}{8}{⼿}
  \begin{phonetics}{拍}{pai1}[][HSK 3]
    \definition[个,副,对]{s.}{bastão; raquete | batida; tempo; (música) uma unidade para medir a duração de uma nota musical}
    \definition{v.}{tirar (uma foto); usar uma câmera para capturar imagens de pessoas e objetos em filme | dar um tapinha; bater suavemente com as mãos ou ferramentas | bater asas | bater (ondas do mar) | enviar (um telegrama, etc.) | bajular}
  \end{phonetics}
\end{entry}

\begin{entry}{拍马}{8,3}{⼿、⾺}
  \begin{phonetics}{拍马}{pai1ma3}
    \definition{v.}{instigar um cavalo dando tapinhas em seu traseiro | lisonjear | bajular}
  \seealsoref{拍马屁}{pai1ma3pi4}
  \end{phonetics}
\end{entry}

\begin{entry}{拍马屁}{8,3,7}{⼿、⾺、⼫}
  \begin{phonetics}{拍马屁}{pai1ma3pi4}
    \definition{s.}{puxa-saco | bajulador}
    \definition{v.}{puxar o saco | bajular}
  \seealsoref{拍马}{pai1ma3}
  \end{phonetics}
\end{entry}

\begin{entry}{拍摄}{8,13}{⼿、⼿}
  \begin{phonetics}{拍摄}{pai1 she4}[][HSK 5]
    \definition{s.}{fotografar; tirar (uma foto); usar uma câmera fotográfica para capturar imagens de pessoas e objetos}
  \end{phonetics}
\end{entry}

\begin{entry}{拍照}{8,13}{⼿、⽕}
  \begin{phonetics}{拍照}{pai1 zhao4}[][HSK 4]
    \definition{v.+compl.}{fotografar; tirar uma foto}
  \end{phonetics}
\end{entry}

\begin{entry}{拐}{8}{⼿}
  \begin{phonetics}{拐}{guai3}
    \definition[支,根,副]{s.}{muleta; bengala; uma bengala com uma barra horizontal na parte superior, usada por pessoas com doenças ou deficiências nos membros inferiores para ajudá-las a caminhar |
sete; forma falada do numeral 七 | esquina; curva; canto}
    \definition{v.}{virar; girar; mudar de direção enquanto se move | enganar | mudar; transformar | mancar}
  \seealsoref{七}{qi1}
  \end{phonetics}
\end{entry}

\begin{entry}{拔}{8}{⼿}
  \begin{phonetics}{拔}{ba2}[][HSK 5]
    \definition{v.aux.}{puxar para cima; puxar para fora; arrastar para fora | extrair; sugar | escolher; selecionar | superar; destacar-se entre | apreender; capturar | esfriar na água; mergulhar algo em água fria para que esfrie}
  \end{phonetics}
\end{entry}

\begin{entry}{拔尖}{8,6}{⼿、⼩}
  \begin{phonetics}{拔尖}{ba2jian1}
    \definition{adj.}{topo de linha | fora do comum | o melhor}
    \definition{v.+compl.}{empurrar-se para a frente | sentir que é superior aos outros}
  \end{phonetics}
\end{entry}

\begin{entry}{拖}{8}{⼿}
  \begin{phonetics}{拖}{tuo1}
    \definition{v.}{puxar; arrastar; transportar; puxar um objeto para movê-lo contra o solo ou outra superfície | esfregar; limpar o chão com uma ferramenta especial para esfregar | atrasar; prolongar; procrastinar; arrastar; coisas que deveriam ser feitas nunca são iniciadas ou concluídas; uma certa nota é prolongada por um longo tempo | atrasar; conter; segurar; restringir}
  \end{phonetics}
\end{entry}

\begin{entry}{拖拉机}{8,8,6}{⼿、⼿、⽊}
  \begin{phonetics}{拖拉机}{tuo1la1ji1}
    \definition[台]{s.}{trator}
  \end{phonetics}
\end{entry}

\begin{entry}{拖鞋}{8,15}{⼿、⾰}
  \begin{phonetics}{拖鞋}{tuo1xie2}
    \definition[双,只]{s.}{chinelos | sandálias}
  \end{phonetics}
\end{entry}

\begin{entry}{招}{8}{⼿}
  \begin{phonetics}{招}{zhao1}
    \definition{adj.}{contagioso}
    \definition{s.}{um movimento (xadrez) | uma manobra | dispositivo | truque}
    \definition{v.}{recrutar | provocar | acenar | incorrer | infectar | confessar}
  \end{phonetics}
\end{entry}

\begin{entry}{招手}{8,4}{⼿、⼿}
  \begin{phonetics}{招手}{zhao1 shou3}[][HSK 5]
    \definition{v.+compl.}{acenar; chamar a atenção; levantar a mão e acenar com a palma, para indicar que a outra pessoa se aproxime ou para cumprimentá-la}
  \end{phonetics}
\end{entry}

\begin{entry}{招生}{8,5}{⼿、⽣}
  \begin{phonetics}{招生}{zhao1 sheng1}[][HSK 5]
    \definition{v.+compl.}{conseguir alunos; matricular novos alunos; recrutar novos alunos}
  \end{phonetics}
\end{entry}

\begin{entry}{招呼}{8,8}{⼿、⼝}
  \begin{phonetics}{招呼}{zhao1 hu5}[][HSK 4]
    \definition{v.}{chamar; chamar a atenção com palavras ou gestos | cumprimentar; saudar; cumprimentar ou despedir-se das pessoas com palavras ou gestos | pedir a alguém para fazer algo; fazer solicitações, pedir ajuda ou fazer coisas | receber e dar boas-vindas aos convidados}
  \end{phonetics}
\end{entry}

\begin{entry}{招数}{8,13}{⼿、⽁}
  \begin{phonetics}{招数}{zhao1shu4}
    \definition{s.}{estratégia | movimento (no xadrez, no palco, nas artes marciais) | esquema | truque}
  \end{phonetics}
\end{entry}

\begin{entry}{拥有}{8,6}{⼿、⽉}
  \begin{phonetics}{拥有}{yong1you3}[][HSK 5]
    \definition{v.}{possuir; deter; ter (grande quantidade de terras, população, bens, etc.)}
  \end{phonetics}
\end{entry}

\begin{entry}{拥抱}{8,8}{⼿、⼿}
  \begin{phonetics}{拥抱}{yong1bao4}[][HSK 5]
    \definition[个,次]{s.}{abraço;}
    \definition{v.}{abraçar; segurar em seus braços; abraçar para demonstrar afeto}
  \end{phonetics}
\end{entry}

\begin{entry}{拧}{8}{⼿}
  \begin{phonetics}{拧}{ning2}
    \definition{v.}{torcer | beliscar; torcer a pele com os dedos e virá-la com força}
  \end{phonetics}
  \begin{phonetics}{拧}{ning3}
    \definition{adj.}{errado; equivocado; de cabeça para baixo; oposto}
    \definition{v.}{torcer; parafusar | divergir; discordar; estar em desacordo}
  \end{phonetics}
  \begin{phonetics}{拧}{ning4}
    \definition{adj.}{teimoso}
  \end{phonetics}
\end{entry}

\begin{entry}{拧开}{8,4}{⼿、⼶}
  \begin{phonetics}{拧开}{ning3kai1}
    \definition{v.}{desaparafusar | desatarrachar | torcer (uma tampa) | abrir (uma torneira) | ligar (girando um botão) | girar (maçaneta da porta)}
  \end{phonetics}
\end{entry}

\begin{entry}{拨}{8}{⼿}
  \begin{phonetics}{拨}{bo1}
    \definition{clas.}{usado para agrupar pessoas; grupo; lote}
    \definition{v.}{mover (mexer) com a mão, o pé, o bastão, etc.; usar as mãos, os pés ou os bastões para mover objetos | atribuir; alocar; reservar | virar-se; inverter a marcha | dedilhar (uma corda de violão) com os dedos ou com um instrumento | chamar (alguém)}
  \end{phonetics}
\end{entry}

\begin{entry}{拨转}{8,8}{⼿、⾞}
  \begin{phonetics}{拨转}{bo1zhuan3}
    \definition{v.}{transferir (fundos, etc.) | virar | dar a volta}
  \end{phonetics}
\end{entry}

\begin{entry}{放}{8}{⽅}
  \begin{phonetics}{放}{fang4}[][HSK 1]
    \definition{v.}{deixar ir; libertar; soltar | ceder; deixar-se levar | levar para se alimentar; pastar | soltar; liberar (ou expelir) | exibir (um filme, etc.); reproduzir (um disco, etc.) | acender; inflamar | emprestar (dinheiro) com juros | tornar maior ou mais longo; soltar; abaixar | moderar (a atitude ou o comportamento de alguém) | (de flores) florescer; abrir | colocar; posicionar; deitar | fazer com que algo (ou alguém) caia no chão | deixar de lado; guardar (para uso futuro); conservar | (seguido por 着\dots 不\dots) permitir que algo permaneça (por fazer, por pegar, por usar, etc.) | adicionar; colocar | colocar em pastagem; soltar para caçar | deixar de lado; suspender; interromper | remover; aliviar; livrar-se; proteger; libertar | deixar-se levar; sem restrições; libertino | mandar embora; tirar o prisioneiro da prisão e deportá-lo para uma região remota | distribuir; emitir; lançar | atear fogo | expandir; ampliar; prolongar | reajustar-se até certo ponto; controlar suas ações, adotar uma determinada atitude, atingir um certo equilíbrio | derrubar}
  \end{phonetics}
\end{entry}

\begin{entry}{放下}{8,3}{⽅、⼀}
  \begin{phonetics}{放下}{fang4 xia4}[][HSK 2]
    \definition{v.}{deitar-se; colocar no chão| deixar ir; soltar; desistir; largar | colocar; acomodar; depositar}
  \end{phonetics}
\end{entry}

\begin{entry}{放大}{8,3}{⽅、⼤}
  \begin{phonetics}{放大}{fang4da4}[][HSK 5]
    \definition{v.}{amplificar; magnificar; aumentar; ampliar; aumentar o tamanho de imagens, textos, sons, etc.}
  \end{phonetics}
\end{entry}

\begin{entry}{放飞}{8,3}{⽅、⾶}
  \begin{phonetics}{放飞}{fang4fei1}
    \definition{s.}{deixar voar}
  \end{phonetics}
\end{entry}

\begin{entry}{放心}{8,4}{⽅、⼼}
  \begin{phonetics}{放心}{fang4xin1}[][HSK 2]
    \definition{adj.}{despreocupado}
    \definition{v.}{confiar; ter confiança em alguém; sentir-se aliviado; ficar tranquilo; ficar com a consciência tranquila}
  \end{phonetics}
\end{entry}

\begin{entry}{放出}{8,5}{⽅、⼐}
  \begin{phonetics}{放出}{fang4chu1}
    \definition{v.}{liberar | libertar}
  \end{phonetics}
\end{entry}

\begin{entry}{放电}{8,5}{⽅、⽥}
  \begin{phonetics}{放电}{fang4dian4}
    \definition{s.}{descarga elétrica}
  \end{phonetics}
\end{entry}

\begin{entry}{放任}{8,6}{⽅、⼈}
  \begin{phonetics}{放任}{fang4ren4}
    \definition{v.}{ignorar | saciar-se | deixar sozinho}
  \end{phonetics}
\end{entry}

\begin{entry}{放过}{8,6}{⽅、⾡}
  \begin{phonetics}{放过}{fang4guo4}
    \definition{v.}{deixar | deixar alguém escapar impune | passar despercebido}
  \end{phonetics}
\end{entry}

\begin{entry}{放弃}{8,7}{⽅、⼶}
  \begin{phonetics}{放弃}{fang4qi4}[][HSK 5]
    \definition{v.}{desistir, abandonar; descartar (direitos originais, reivindicações, opiniões, etc.)}
  \end{phonetics}
\end{entry}

\begin{entry}{放弃权利}{8,7,6,7}{⽅、⼶、⽊、⼑}
  \begin{phonetics}{放弃权利}{fang4qi4 quan2li4}
    \definition{s.}{renúncia}
  \end{phonetics}
\end{entry}

\begin{entry}{放弃者}{8,7,8}{⽅、⼶、⽼}
  \begin{phonetics}{放弃者}{fang4qi4zhe3}
    \definition{s.}{desistente}
  \end{phonetics}
\end{entry}

\begin{entry}{放走}{8,7}{⽅、⾛}
  \begin{phonetics}{放走}{fang4zou3}
    \definition{v.}{permitir (uma pessoa ou um animal) ir | liberar | libertar}
  \end{phonetics}
\end{entry}

\begin{entry}{放到}{8,8}{⽅、⼑}
  \begin{phonetics}{放到}{fang4 dao4}[][HSK 3]
    \definition{v.}{colocar em; meter}
  \end{phonetics}
\end{entry}

\begin{entry}{放学}{8,8}{⽅、⼦}
  \begin{phonetics}{放学}{fang4 xue2}[][HSK 1]
    \definition{v.+compl.}{encerrar; sair da escola; as aulas terminaram; a escola acabou (por hoje); voltar para casa depois de um dia ou meio dia de aula}
  \end{phonetics}
\end{entry}

\begin{entry}{放松}{8,8}{⽅、⽊}
  \begin{phonetics}{放松}{fang4song1}[][HSK 4]
    \definition{adj.}{relaxado; afrouxado; solto; desprendido}
    \definition{v.}{relaxar; afrouxar; soltar; desprender}
  \end{phonetics}
\end{entry}

\begin{entry}{放养}{8,9}{⽅、⼋}
  \begin{phonetics}{放养}{fang4yang3}
    \definition{v.}{criar (gado, peixes, culturas, etc.) | crescer | criar}
  \end{phonetics}
\end{entry}

\begin{entry}{放假}{8,11}{⽅、⼈}
  \begin{phonetics}{放假}{fang4 jia4}[][HSK 1]
    \definition{v.}{tirar férias (ou feriado); ter um dia de folga}
    \definition{v.+compl.}{tirar férias (ou feriado); começar as férias; ter um dia de folga; estar de férias (feriado)}
  \end{phonetics}
\end{entry}

\begin{entry}{放肆}{8,13}{⽅、⾀}
  \begin{phonetics}{放肆}{fang4si4}
    \definition{adj.}{atrevido | pesunçoso | devasso}
  \end{phonetics}
\end{entry}

\begin{entry}{放鞭炮}{8,18,9}{⽅、⾰、⽕}
  \begin{phonetics}{放鞭炮}{fang4bian1pao4}
    \definition{s.}{um conjunto de bombinhas ou traques}
  \end{phonetics}
\end{entry}

\begin{entry}{斩获}{8,10}{⽄、⾋}
  \begin{phonetics}{斩获}{zhan3huo4}
    \definition{v.}{matar ou capturar (em batalha) | (figurativo) (esportes) marcar (um gol), ganhar (uma medalha) | (figurativo) colher recompensas, obter ganhos}
  \end{phonetics}
\end{entry}

\begin{entry}{明}{8}{⽇}
  \begin{phonetics}{明}{ming2}
    \definition*{s.}{sobrenome Ming}
    \definition*{s.}{Dinastia Ming (1368-1644)}
    \definition{adj.}{claro; brilhante; brilhante | claro; distinto; de fácil entendimento | aberto; evidente; explícito; exposto | de ​​olhos aguçados; boa visão; visão nítida | honesto}
    \definition{adv.}{claramente; definitivamente; aparentemente; de fato}
    \definition{s.}{imediatamente a seguir no tempo; ao lado deste ano e hoje; visão}
    \definition{v.}{mostrar; revelar; tornar conhecido; deixar claro | entender; compreender}
  \end{phonetics}
\end{entry}

\begin{entry}{明天}{8,4}{⽇、⼤}
  \begin{phonetics}{明天}{ming2tian1}[][HSK 1]
    \definition{s.}{amanhã | futuro próximo}
  \end{phonetics}
\end{entry}

\begin{entry}{明白}{8,5}{⽇、⽩}
  \begin{phonetics}{明白}{ming2bai5}[][HSK 1]
    \definition{adj.}{claro; óbvio; evidente; inequívoco | sensato; razoável | aberto; franco; inequívoco; explícito}
    \definition{v.}{entender; compreender; saber}
  \end{phonetics}
\end{entry}

\begin{entry}{明年}{8,6}{⽇、⼲}
  \begin{phonetics}{明年}{ming2 nian2}[][HSK 1]
    \definition{s.}{próximo ano}
  \end{phonetics}
\end{entry}

\begin{entry}{明明}{8,8}{⽇、⽇}
  \begin{phonetics}{明明}{ming2ming2}[][HSK 5]
    \definition{adv.}{obviamente; claramente; sem dúvida; indica que o fenômeno ou princípio é evidente}
  \end{phonetics}
\end{entry}

\begin{entry}{明亮}{8,9}{⽇、⼇}
  \begin{phonetics}{明亮}{ming2 liang4}[][HSK 5]
    \definition{adj.}{claro; bem iluminado | brilhante; resplandecente | claro; simples; compreensível}
  \end{phonetics}
\end{entry}

\begin{entry}{明星}{8,9}{⽇、⽇}
  \begin{phonetics}{明星}{ming2xing1}[][HSK 2]
    \definition[个,位,颗,名]{s.}{estrela; ator, atleta, cantor famosos, etc. | talento de ponta; profissional de destaque; também é usado como metáfora para pessoas ou grupos que se destacam pelo seu bom desempenho ou excelência | estrela brilhante; estrela resplandecente; referindo-se a estrelas muito brilhantes}
  \end{phonetics}
\end{entry}

\begin{entry}{明显}{8,9}{⽇、⽇}
  \begin{phonetics}{明显}{ming2xian3}[][HSK 3]
    \definition{adj.}{claro; óbvio; distinto; claramente visível}
  \end{phonetics}
\end{entry}

\begin{entry}{明珠}{8,10}{⽇、⽟}
  \begin{phonetics}{明珠}{ming2zhu1}
    \definition{s.}{pérola | jóia (de grande valor)}
  \end{phonetics}
\end{entry}

\begin{entry}{明确}{8,12}{⽇、⽯}
  \begin{phonetics}{明确}{ming2que4}[][HSK 3]
    \definition{adj.}{claro; definido; específico}
    \definition{v.}{deixar claro; tornar definitivo; tornar um ponto de vista, uma tarefa, etc. claro, compreensível e definitivo}
  \end{phonetics}
\end{entry}

\begin{entry}{易}{8}{⽇}
  \begin{phonetics}{易}{yi4}
    \definition*{s.}{sobrenome Yi}
    \definition{adj.}{fácil | amigável; pacífico}
    \definition{v.}{modificar; transformar | trocar | subestimar; desprezar}
  \end{phonetics}
\end{entry}

\begin{entry}{昔}{8}{⽇}
  \begin{phonetics}{昔}{xi1}
    \definition{s.}{tempos antigos; o passado; era uma vez}
  \end{phonetics}
\end{entry}

\begin{entry}{昔日}{8,4}{⽇、⽇}
  \begin{phonetics}{昔日}{xi1ri4}
    \definition{adj.}{passado}
  \end{phonetics}
\end{entry}

\begin{entry}{朋}{8}{⽉}
  \begin{phonetics}{朋}{peng2}
    \definition*{s.}{sobrenome Peng}
    \definition{s.}{amigo}
    \definition{v.}{(literário) rivalizar; igualar; comparar | (literário) reunir-se em grupo; juntar-se em grupo}
  \end{phonetics}
\end{entry}

\begin{entry}{朋友}{8,4}{⽉、⼜}
  \begin{phonetics}{朋友}{peng2you5}[][HSK 1]
    \definition[个,位,帮,群]{s.}{amigo; pessoas que têm um bom relacionamento, uma boa relação, se entendem e se ajudam mutuamente | namorado; namorada}
  \end{phonetics}
\end{entry}

\begin{entry}{服}{8}{⽉}
  \begin{phonetics}{服}{fu2}
    \definition*{s.}{sobrenome Fu}
    \definition{s.}{roupas | vestuário de luto; refere-se a roupas de luto}
    \definition{v.}{vestir (roupas) | tomar (remédio) | envolver-se em; servir | obedecer; ser convencido | convencer; persuadir | adaptar-se; acostumar-se a}
  \end{phonetics}
  \begin{phonetics}{服}{fu4}
    \definition{clas.}{usado para remédio: dose; usado na medicina tradicional chinesa}
  \end{phonetics}
\end{entry}

\begin{entry}{服从}{8,4}{⽉、⼈}
  \begin{phonetics}{服从}{fu2cong2}[][HSK 5]
    \definition{v.}{obedecer; submeter-se a; estar subordinado a}
  \end{phonetics}
\end{entry}

\begin{entry}{服务}{8,5}{⽉、⼒}
  \begin{phonetics}{服务}{fu2 wu4}[][HSK 2]
    \definition{v.}{prestar serviço a; estar a serviço de; servir; trabalhar para o benefício coletivo (ou de outras pessoas) ou para uma causa específica | trabalhar; servir}
  \end{phonetics}
\end{entry}

\begin{entry}{服务员}{8,5,7}{⽉、⼒、⼝}
  \begin{phonetics}{服务员}{fu2wu4yuan2}
    \definition{s.}{atendente | garçom | garçonete | pessoal de atendimento ao cliente}
  \end{phonetics}
\end{entry}

\begin{entry}{服装}{8,12}{⽉、⾐}
  \begin{phonetics}{服装}{fu2zhuang1}[][HSK 3]
    \definition[套,件,身]{s.}{roupas; vestuário; trajes; termo genérico para roupas, sapatos e chapéus, geralmente referido especificamente a roupas}
  \end{phonetics}
\end{entry}

\begin{entry}{杯}{8}{⽊}
  \begin{phonetics}{杯}{bei1}[][HSK 1]
    \definition{clas.}{para certos recipientes de líquidos: copo, xícara, etc.}
    \definition[只,个]{s.}{copo; caneca; xícara | taça; troféu; prêmio em forma de taça}
  \end{phonetics}
\end{entry}

\begin{entry}{杯子}{8,3}{⽊、⼦}
  \begin{phonetics}{杯子}{bei1 zi5}[][HSK 1]
    \definition[个,只,种]{s.}{xícara; copo; recipiente para bebidas ou outros líquidos, geralmente cilíndrico ou com a parte inferior ligeiramente mais estreita, com capacidade geralmente pequena}
  \end{phonetics}
\end{entry}

\begin{entry}{杯具}{8,8}{⽊、⼋}
  \begin{phonetics}{杯具}{bei1ju4}
    \definition{s.}{parachoque | fiasco | (gíria) tragédia}
  \end{phonetics}
\end{entry}

\begin{entry}{松}{8}{⽊}
  \begin{phonetics}{松}{song1}[][HSK 4]
    \definition*{s.}{sobrenome Song}
    \definition{adj.}{solto; frouxo; folgado | abastado; rico; próspero | leve e crocante; macio}
    \definition[棵]{s.}{pinheiro | fio de carne seca; carne moída seca; alimentos macios ou quebradiços |}
    \definition{v.}{afrouxar; relaxar; soltar}
  \end{phonetics}
\end{entry}

\begin{entry}{松木}{8,4}{⽊、⽊}
  \begin{phonetics}{松木}{song1mu4}
    \definition{s.}{pinheiro}
  \end{phonetics}
\end{entry}

\begin{entry}{松树}{8,9}{⽊、⽊}
  \begin{phonetics}{松树}{song1 shu4}[][HSK 4]
    \definition[棵]{s.}{pinheiro; conífera comum, geralmente com folhas longas e pontiagudas e cones lenhosos}
  \end{phonetics}
\end{entry}

\begin{entry}{板}{8}{⽊}
  \begin{phonetics}{板}{ban3}[][HSK 3]
    \definition{adj.}{rígido; não natural; inflexível}
    \definition[块,个]{s.}{tábua; placa; prato; objeto rígido em forma de placa | veneziana; persiana; refere-se especificamente aos painéis de portas de lojas | badalos (instrumento musical que marca o ritmo) | uma batida acentuada (ritmo) na música e na ópera tradicional | chefe}
    \definition{v.}{parecer sério | corrigir maus hábitos ou defeitos | ser rígido como uma tábua}
  \end{phonetics}
\end{entry}

\begin{entry}{构}{8}{⽊}
  \begin{phonetics}{构}{gou4}
    \definition{s.}{composição literária}
    \definition{v.}{construir | formar | compor}
    \variantof{够}
  \end{phonetics}
\end{entry}

\begin{entry}{构成}{8,6}{⽊、⼽}
  \begin{phonetics}{构成}{gou4cheng2}[][HSK 4]
    \definition{s.}{parte; componente; composição; estrutura}
    \definition{v.}{formar; compor; constituir; compor; encaixar muitas partes para formar um todo | consistir; causar; formar (principalmente em termos jurídicos)}
  \end{phonetics}
\end{entry}

\begin{entry}{构造}{8,10}{⽊、⾡}
  \begin{phonetics}{构造}{gou4 zao4}[][HSK 4]
    \definition[种]{s.}{estrutura; construção; disposição, organização e inter-relação dos componentes}
    \definition{v.}{formar; construir}
  \end{phonetics}
\end{entry}

\begin{entry}{枕}{8}{⽊}
  \begin{phonetics}{枕}{zhen3}
    \definition{s.}{travesseiro | almofada}
  \end{phonetics}
\end{entry}

\begin{entry}{果}{8}{⽊}
  \begin{phonetics}{果}{guo3}
    \definition*{s.}{sobrenome Guo}
    \definition{adj.}{resoluto; determinado; sem exitação}
    \definition{adv.}{realmente; como esperado; com certeza; isso significa que as coisas são consistentes com as expectativas, equivalente a 果然}
    \definition{conj.}{se realmente; se de fato}
    \definition[个,些,种]{s.}{fruta; fruto da planta | resultado; consequência; o resultado final de um assunto (em oposição à 因)}
  \seealsoref{果然}{guo3ran2}
  \seealsoref{因}{yin1}
  \end{phonetics}
\end{entry}

\begin{entry}{果子}{8,3}{⽊、⼦}
  \begin{phonetics}{果子}{guo3zi5}
    \definition{s.}{fruta}
  \end{phonetics}
\end{entry}

\begin{entry}{果汁}{8,5}{⽊、⽔}
  \begin{phonetics}{果汁}{guo3zhi1}[][HSK 3]
    \definition[杯,瓶,种]{s.}{suco; suco de frutas frescas; também se refere a bebidas feitas com suco de frutas frescas}
  \end{phonetics}
\end{entry}

\begin{entry}{果实}{8,8}{⽊、⼧}
  \begin{phonetics}{果实}{guo3shi2}[][HSK 4]
    \definition[种]{s.}{fruta; o órgão que se desenvolve a partir do ovário ou com outras partes da flor após a fertilização da flor | ganhos; frutos;  uma metáfora para conquista ou recompensa por trabalho árduo}
  \end{phonetics}
\end{entry}

\begin{entry}{果然}{8,12}{⽊、⽕}
  \begin{phonetics}{果然}{guo3ran2}[][HSK 3]
    \definition{adv.}{realmente; como esperado; com certeza; indica que os fatos correspondem ao que foi dito ou esperado}
    \definition{conj.}{se realmente; se de fato; suponha que os fatos correspondam ao que foi dito ou esperado}
  \end{phonetics}
\end{entry}

\begin{entry}{果酱}{8,13}{⽊、⾣}
  \begin{phonetics}{果酱}{guo3jiang4}
    \definition{s.}{geléia | compota ou doce (de frutas)}
  \end{phonetics}
\end{entry}

\begin{entry}{枪}{8}{⽊}
  \begin{phonetics}{枪}{qiang1}[][HSK 5]
    \definition*{s.}{sobrenome Qiang}
    \definition{s.}{lança | arma; rifle; arma de fogo | uma coisa em forma de arma | enxada; ferramenta para cavar a terra}
    \definition{v.}{escrever artigos ou responder perguntas para outras pessoas}
  \end{phonetics}
\end{entry}

\begin{entry}{枫}{8}{⽊}
  \begin{phonetics}{枫}{feng1}
    \definition[棵]{s.}{goma doce chinesa | bordo; \emph{maple}}
  \end{phonetics}
\end{entry}

\begin{entry}{枫叶}{8,5}{⽊、⼝}
  \begin{phonetics}{枫叶}{feng1ye4}
    \definition{s.}{folha de bordo (maple, tipo de árvore)}
  \end{phonetics}
\end{entry}

\begin{entry}{柜}{8}{⽊}
  \begin{phonetics}{柜}{gui4}
    \definition{s.}{baú; armário; gabinete | loja; balcão}
  \end{phonetics}
  \begin{phonetics}{柜}{ju3}
    \definition{s.}{faia; salgueiro}
  \end{phonetics}
\end{entry}

\begin{entry}{柜子}{8,3}{⽊、⼦}
  \begin{phonetics}{柜子}{gui4 zi5}[][HSK 5]
    \definition[个]{s.}{gabinete; armário; dispositivo para guardar roupas, documentos, livros, etc.}
  \end{phonetics}
\end{entry}

\begin{entry}{欣赏}{8,12}{⽋、⾙}
  \begin{phonetics}{欣赏}{xin1shang3}[][HSK 5]
    \definition{v.}{apreciar; admirar; valorizar; apreciar as coisas boas e descubrir o prazer que elas proporcionam | apreciar; gostar; considerar bom}
  \end{phonetics}
\end{entry}

\begin{entry}{欧}{8}{⽋}
  \begin{phonetics}{欧}{ou1}
    \definition*{s.}{sobrenome Ou | Europa, abreviação de 欧洲}
  \seealsoref{欧洲}{ou1zhou1}
  \end{phonetics}
\end{entry}

\begin{entry}{欧洲}{8,9}{⽋、⽔}
  \begin{phonetics}{欧洲}{ou1zhou1}
    \definition*{s.}{Europa}
  \end{phonetics}
\end{entry}

\begin{entry}{欧洲人}{8,9,2}{⽋、⽔、⼈}
  \begin{phonetics}{欧洲人}{ou1zhou1ren2}
    \definition{s.}{europeu | pessoa ou povo da Europa}
  \end{phonetics}
\end{entry}

\begin{entry}{欧洲共同体}{8,9,6,6,7}{⽋、⽔、⼋、⼝、⼈}
  \begin{phonetics}{欧洲共同体}{ou1zhou1 gong4tong2ti3}
    \definition*{s.}{Comunidade Europeia}
  \end{phonetics}
\end{entry}

\begin{entry}{欧盟}{8,13}{⽋、⽫}
  \begin{phonetics}{欧盟}{ou1meng2}
    \definition*{s.}{Uniáo Europeia}
  \end{phonetics}
\end{entry}

\begin{entry}{武}{8}{⽌}
  \begin{phonetics}{武}{wu3}
    \definition*{s.}{sobrenome Wu}
    \definition{s.}{arte marcial}
  \end{phonetics}
\end{entry}

\begin{entry}{武力}{8,2}{⽌、⼒}
  \begin{phonetics}{武力}{wu3li4}
    \definition{s.}{forças armadas | militares}
  \end{phonetics}
\end{entry}

\begin{entry}{武士}{8,3}{⽌、⼠}
  \begin{phonetics}{武士}{wu3shi4}
    \definition{s.}{samurai | guerreiro}
  \end{phonetics}
\end{entry}

\begin{entry}{武大戏}{8,3,6}{⽌、⼤、⼽}
  \begin{phonetics}{武大戏}{wu3 da4xi4}
    \definition*{s.}{Drama de Luta Acrobática | Drama Wu}
  \end{phonetics}
\end{entry}

\begin{entry}{武艺}{8,4}{⽌、⾋}
  \begin{phonetics}{武艺}{wu3yi4}
    \definition{s.}{arte marcial | habilidade militar}
  \end{phonetics}
\end{entry}

\begin{entry}{武术}{8,5}{⽌、⽊}
  \begin{phonetics}{武术}{wu3shu4}[][HSK 3]
    \definition[种,套,门]{s.}{arte marcial; autodefesa; \emph{wushu}; um esporte tradicional chinês que utiliza técnicas com os punhos, pernas, pés ou armas como facas e espadas}
  \end{phonetics}
\end{entry}

\begin{entry}{武官}{8,8}{⽌、⼧}
  \begin{phonetics}{武官}{wu3guan1}
    \definition{s.}{oficial militar}
  \end{phonetics}
\end{entry}

\begin{entry}{武断}{8,11}{⽌、⽄}
  \begin{phonetics}{武断}{wu3duan4}
    \definition{adj.}{arbitrário | dogmático | subjetivo}
  \end{phonetics}
\end{entry}

\begin{entry}{武装}{8,12}{⽌、⾐}
  \begin{phonetics}{武装}{wu3zhuang1}
    \definition{s.}{forças armadas | militar | arma}
    \definition{v.}{armar}
  \end{phonetics}
\end{entry}

\begin{entry}{武器}{8,16}{⽌、⼝}
  \begin{phonetics}{武器}{wu3qi4}[][HSK 3]
    \definition[批,个,件,种]{s.}{arma; equipamentos e dispositivos utilizados diretamente para matar inimigos ou destruir suas instalações defensivas e ofensivas | armas; armamento; metáfora usada como ferramenta de luta}
  \end{phonetics}
\end{entry}

\begin{entry}{河}{8}{⽔}
  \begin{phonetics}{河}{he2}[][HSK 2]
    \definition*{s.}{o sistema da Via Láctea | o rio Amarelo; o rio Huanghe}
    \definition*{s.}{sobrenome He}
    \definition[条,道]{s.}{rio; refere-se a grandes cursos de água}
  \end{phonetics}
\end{entry}

\begin{entry}{河蚌}{8,10}{⽔、⾍}
  \begin{phonetics}{河蚌}{he2bang4}
    \definition{s.}{mexilhões | bivalves cultivados em rios e lagos}
  \end{phonetics}
\end{entry}

\begin{entry}{油}{8}{⽔}
  \begin{phonetics}{油}{you2}[][HSK 2]
    \definition*{s.}{sobrenome You}
    \definition{adj.}{oleoso; gorduroso}
    \definition[瓶,滴,层]{s.}{óleo; gordura; graxa; petróleo}
    \definition{v.}{aplicar óleo de tungue, verniz ou tinta | estar manchado ou sujo com óleo ou graxa | aplicar óleo de tungue ou tinta}
  \end{phonetics}
\end{entry}

\begin{entry}{治}{8}{⽔}
  \begin{phonetics}{治}{zhi4}[][HSK 4]
    \definition*{s.}{sobrenome Zhi}
    \definition{adj.}{calmo e pacífico}
    \definition{s.}{sede de um antigo governo local}
    \definition{v.}{reger; administrar; governar; gerenciar; gerir | tratar (uma doença); curar; sarar | eliminar; controlar pragas | controlar (um rio); restaurar um curso d'água por meio de dragagem | punir; castigar | estudar; pesquisar; explorar}
  \end{phonetics}
\end{entry}

\begin{entry}{治安}{8,6}{⽔、⼧}
  \begin{phonetics}{治安}{zhi4'an1}[][HSK 5]
    \definition{s.}{ordem pública; segurança pública; ordem social estável}
  \end{phonetics}
\end{entry}

\begin{entry}{治疗}{8,7}{⽔、⽧}
  \begin{phonetics}{治疗}{zhi4liao2}[][HSK 4]
    \definition{s.}{diagnóstico; tratamento}
    \definition{v.}{tratar; curar; remediar; eliminar doenças por meio de medicamentos, cirurgia, etc.}
  \end{phonetics}
\end{entry}

\begin{entry}{治理}{8,11}{⽔、⽟}
  \begin{phonetics}{治理}{zhi4li3}[][HSK 5]
    \definition{s.}{governo | governança}
    \definition{v.}{dirigir; gerenciar; governar; administrar | tratar; aproveitar; colocar sob controle; colocar em ordem}
  \end{phonetics}
\end{entry}

\begin{entry}{治愈}{8,13}{⽔、⼼}
  \begin{phonetics}{治愈}{zhi4yu4}
    \definition{v.}{curar | restaurar a saúde}
  \end{phonetics}
\end{entry}

\begin{entry}{泄气}{8,4}{⽔、⽓}
  \begin{phonetics}{泄气}{xie4qi4}
    \definition{adj.}{decepcionante | frustrante | patético}
    \definition{v.+compl.}{perder o coração | sentir-se desencorajado | ficar desanimado}
  \end{phonetics}
\end{entry}

\begin{entry}{法}{8}{⽔}
  \begin{phonetics}{法}{fa3}[][HSK 4]
    \definition*{s.}{França, abreviação de 法国}
    \definition*{s.}{sobrenome Fa}
    \definition*{s.}{Doutrina budista; o dharma}
    \definition{adj.}{(usado após advérbios negativos) legal; cumpridor da lei}
    \definition{clas.}{farad (F), medida de capacitância}
    \definition{s.}{lei; termo geral para regras de comportamento estabelecidas ou endossadas pelo Estado | maneira; método; modo; meios | padrão; modelo | artes mágicas; feitiço}
    \definition{v.}{seguir; imitar; aprender (os pontos fortes dos outros) |}
  \seealsoref{法国}{fa3guo2}
  \end{phonetics}
\end{entry}

\begin{entry}{法文}{8,4}{⽔、⽂}
  \begin{phonetics}{法文}{fa3wen2}
    \definition[份]{s.}{françês, língua francesa}
  \end{phonetics}
\end{entry}

\begin{entry}{法网}{8,6}{⽔、⽹}
  \begin{phonetics}{法网}{fa3wang3}
    \definition*{s.}{Torneio de Roland Garros (French Open), torneio de tênis}
  \end{phonetics}
\end{entry}

\begin{entry}{法制}{8,8}{⽔、⼑}
  \begin{phonetics}{法制}{fa3 zhi4}[][HSK 5]
    \definition{s.}{legalidade; instituições jurídicas; sistema jurídico}
  \end{phonetics}
\end{entry}

\begin{entry}{法国}{8,8}{⽔、⼞}
  \begin{phonetics}{法国}{fa3guo2}
    \definition*{s.}{França}
  \end{phonetics}
\end{entry}

\begin{entry}{法国人}{8,8,2}{⽔、⼞、⼈}
  \begin{phonetics}{法国人}{fa3guo2ren2}
    \definition{s.}{francês | pessoa ou povo da França}
  \end{phonetics}
\end{entry}

\begin{entry}{法官}{8,8}{⽔、⼧}
  \begin{phonetics}{法官}{fa3 guan1}[][HSK 4]
    \definition[位]{s.}{juiz; justiça; termo genérico para um membro do judiciário em um tribunal de justiça}
  \end{phonetics}
\end{entry}

\begin{entry}{法规}{8,8}{⽔、⾒}
  \begin{phonetics}{法规}{fa3 gui1}[][HSK 5]
    \definition{s.}{lei e regulamento; estatuto; termo geral para leis, decretos, regulamentos, regras, estatutos, etc.}
  \end{phonetics}
\end{entry}

\begin{entry}{法律}{8,9}{⽔、⼻}
  \begin{phonetics}{法律}{fa3lv4}[][HSK 4]
    \definition[项,条,套,个]{s.}{lei; estatuto; regras de conduta formuladas pelo legislativo e cuja aplicação é garantida pelo poder estatal}
  \end{phonetics}
\end{entry}

\begin{entry}{法语}{8,9}{⽔、⾔}
  \begin{phonetics}{法语}{fa3yu3}
    \definition{s.}{françês, língua francesa}
  \end{phonetics}
\end{entry}

\begin{entry}{法院}{8,9}{⽔、⾩}
  \begin{phonetics}{法院}{fa3yuan4}[][HSK 4]
    \definition[所,座]{s.}{tribunal; corte; órgãos estatais que exercem poder judicial independente}
  \end{phonetics}
\end{entry}

\begin{entry}{泡}{8}{⽔}
  \begin{phonetics}{泡}{pao1}
    \definition{adj.}{estufado | inchado | esponjoso}
    \definition{clas.}{para urina ou fezes}
    \definition{s.}{pequeno lago (especialmente em nomes de lugares)}
  \end{phonetics}
  \begin{phonetics}{泡}{pao4}
    \definition{clas.}{para ocorrências de uma ação | para número de infusões}
    \definition{s.}{bolha | espuma}
    \definition{v.}{encharcar | infundir | pegar (uma garota) | sair com (um parceiro sexual)}
  \end{phonetics}
\end{entry}

\begin{entry}{波}{8}{⽔}
  \begin{phonetics}{波}{bo1}
    \definition*{s.}{Polônia, abreviação de 波兰}
    \definition*{s.}{sobrenome Bo}
    \definition{s.}{ondas, a superfície irregular da água em rios, lagos e oceanos | onda, o processo de propagação da vibração | mudanças inesperadas; uma reviravolta inesperada nos acontecimentos; metáfora para mudanças inesperadas nas coisas | olhos; metáfora do olhar errante}
  \seealsoref{波兰}{bo1lan2}
  \end{phonetics}
\end{entry}

\begin{entry}{波兰}{8,5}{⽔、⼋}
  \begin{phonetics}{波兰}{bo1lan2}
    \definition*{s.}{Polônia}
  \end{phonetics}
\end{entry}

\begin{entry}{波音}{8,9}{⽔、⾳}
  \begin{phonetics}{波音}{bo1yin1}
    \definition*{s.}{Boeing (empresa aeroespacial)}
    \definition{s.}{mordente (música)}
  \end{phonetics}
\end{entry}

\begin{entry}{泥}{8}{⽔}
  \begin{phonetics}{泥}{ni2}
    \definition{s.}{lama | argila | pasta | polpa}
  \end{phonetics}
  \begin{phonetics}{泥}{ni4}
    \definition{adj.}{contido}
  \end{phonetics}
\end{entry}

\begin{entry}{泥潭}{8,15}{⽔、⽔}
  \begin{phonetics}{泥潭}{ni2tan2}
    \definition{s.}{atoleiro | lamaçal | charco | pântano}
  \end{phonetics}
\end{entry}

\begin{entry}{注}{8}{⽔}
  \begin{phonetics}{注}{zhu4}
    \definition{s.}{apostas (em jogos de azar) | notas (em um texto)}
    \definition{v.}{derramar; encher | concentrar-se em; fixar-se em; focar em  | anotar; explicar com notas | registrar; gravar | irrigar | dar exegese ou explicação}
  \end{phonetics}
\end{entry}

\begin{entry}{注册}{8,5}{⽔、⼌}
  \begin{phonetics}{注册}{zhu4ce4}[][HSK 5]
    \definition{v.}{inscrever-se; matricular-se; registrar-se; registrar-se junto à autoridade ou escola competente para obter status legal; refere-se especificamente ao usuário de uma determinada rede de computadores que insere o nome de usuário, senha, etc. na rede para obter permissão para usar a rede}
  \end{phonetics}
\end{entry}

\begin{entry}{注册人}{8,5,2}{⽔、⼌、⼈}
  \begin{phonetics}{注册人}{zhu4ce4ren2}
    \definition{s.}{registrante}
  \end{phonetics}
\end{entry}

\begin{entry}{注册表}{8,5,8}{⽔、⼌、⾐}
  \begin{phonetics}{注册表}{zhu4ce4biao3}
    \definition*{s.}{Registro do \emph{Windows}}
  \end{phonetics}
\end{entry}

\begin{entry}{注册商标}{8,5,11,9}{⽔、⼌、⼝、⽊}
  \begin{phonetics}{注册商标}{zhu4ce4shang1biao1}
    \definition{s.}{marca registrada}
  \end{phonetics}
\end{entry}

\begin{entry}{注视}{8,8}{⽔、⾒}
  \begin{phonetics}{注视}{zhu4shi4}[][HSK 5]
    \definition{v.}{olhar atentamente para; observar atentamente}
  \end{phonetics}
\end{entry}

\begin{entry}{注重}{8,9}{⽔、⾥}
  \begin{phonetics}{注重}{zhu4zhong4}[][HSK 5]
    \definition{v.}{enfatizar; dar ênfase a; dar ênfase a; prestar atenção a; dar importância a}
  \end{phonetics}
\end{entry}

\begin{entry}{注射}{8,10}{⽔、⼨}
  \begin{phonetics}{注射}{zhu4she4}[][HSK 5]
    \definition{v.}{injetar; usar uma seringa para administrar medicamento líquido em um organismo}
  \end{phonetics}
\end{entry}

\begin{entry}{注意}{8,13}{⽔、⼼}
  \begin{phonetics}{注意}{zhu4yi4}[][HSK 3]
    \definition{v.}{prestar atenção; notar; ficar de olho; concentrar os pensamentos em um aspecto específico}
  \end{phonetics}
\end{entry}

\begin{entry}{注意力}{8,13,2}{⽔、⼼、⼒}
  \begin{phonetics}{注意力}{zhu4yi4li4}
    \definition{s.}{atenção}
  \end{phonetics}
\end{entry}

\begin{entry}{注意力缺失症}{8,13,2,10,5,10}{⽔、⼼、⼒、⽸、⼤、⽧}
  \begin{phonetics}{注意力缺失症}{zhu4yi4li4que1shi1zheng4}
    \definition{s.}{transtorno de déficit de atenção}
  \end{phonetics}
\end{entry}

\begin{entry}{注意地}{8,13,6}{⽔、⼼、⼟}
  \begin{phonetics}{注意地}{zhu4yi4di4}
    \definition{s.}{área de cuidado, de observação}
  \end{phonetics}
\end{entry}

\begin{entry}{泪}{8}{⽔}
  \begin{phonetics}{泪}{lei4}[][HSK 4]
    \definition[滴]{s.}{lágrima}
  \end{phonetics}
\end{entry}

\begin{entry}{泪水}{8,4}{⽔、⽔}
  \begin{phonetics}{泪水}{lei4 shui3}[][HSK 4]
    \definition{s.}{lágrima}
  \end{phonetics}
\end{entry}

\begin{entry}{泳池}{8,6}{⽔、⽔}
  \begin{phonetics}{泳池}{yong3chi2}
    \definition{s.}{piscina}
  \seealsoref{游泳池}{you2 yong3 chi2}
  \seealsoref{游泳馆}{you2yong3guan3}
  \end{phonetics}
\end{entry}

\begin{entry}{泳衣}{8,6}{⽔、⾐}
  \begin{phonetics}{泳衣}{yong3yi1}
    \definition{s.}{roupa de banho | maiô}
  \seealsoref{游泳衣}{you2yong3yi1}
  \end{phonetics}
\end{entry}

\begin{entry}{泼}{8}{⽔}
  \begin{phonetics}{泼}{po1}[][HSK 5]
    \definition{adj.}{rude e irracional; mal-humorado}
    \definition{v.}{espalhar; salpicar; derramar; derramar ou espalhar o líquido com força para fora |}
  \end{phonetics}
\end{entry}

\begin{entry}{浅}{8}{⽔}
  \begin{phonetics}{浅}{jian1}
    \definition{adj.}{murmurando, fluindo suavemente, gorgolejando suavemente}
    \definition{s.}{(onomatopéia) som de água em movimento |}
  \end{phonetics}
  \begin{phonetics}{浅}{qian3}[][HSK 4]
    \definition{adj.}{raso; superficial;  (em oposição a 深) | fácil; simples; redação, conteúdo, etc. simples e fáceis de entender | superficial; não é profundo em aprendizado, percepção e sabedoria | não próximo; não íntimo; sentimentos não profundos | (cor) claro; pálido;  cor pouco intensa; leve |experiência breve; duração de tempo breve | baixo grau; peso leve; nível baixo}
  \seealsoref{深}{shen1}
  \end{phonetics}
\end{entry}

\begin{entry}{炎热}{8,10}{⽕、⽕}
  \begin{phonetics}{炎热}{yan2re4}
    \definition{adj.}{extremamente quente | escaldante (clima)}
  \end{phonetics}
\end{entry}

\begin{entry}{炒}{8}{⽕}
  \begin{phonetics}{炒}{chao3}
    \definition{v.}{saltear; refogar; aquecer os alimentos em uma panela e mexer repetidamente para cozinhá-los ou secá-los | especular (na bolsa de valores, etc.) | exagerar; dar publicidade exagerada; a fim de ampliar a influência, por meio de publicidade repetida e exagerada na mídia | demitir; despedir}
  \end{phonetics}
\end{entry}

\begin{entry}{爬}{8}{⽖}
  \begin{phonetics}{爬}{pa2}[][HSK 2]
    \definition{v.}{rastejar; arrastar-se; engatinhar | escalar; trepar; subir com dificuldade | sentar-se; levantar-se; levantar-se da posição deitada ou sentada}
  \end{phonetics}
\end{entry}

\begin{entry}{爬上}{8,3}{⽖、⼀}
  \begin{phonetics}{爬上}{pa2shang4}
    \definition{v.}{escalar}
  \end{phonetics}
\end{entry}

\begin{entry}{爬山}{8,3}{⽖、⼭}
  \begin{phonetics}{爬山}{pa2shan1}[][HSK 2]
    \definition{v.+compl.}{escalar uma montanha;}
  \end{phonetics}
\end{entry}

\begin{entry}{爬升}{8,4}{⽖、⼗}
  \begin{phonetics}{爬升}{pa2sheng1}
    \definition{v.}{ascender | ganhar promoção | subir (números de vendas, etc.) | aumentar}
  \end{phonetics}
\end{entry}

\begin{entry}{爬行}{8,6}{⽖、⾏}
  \begin{phonetics}{爬行}{pa2xing2}
    \definition{v.}{rastejar | arrastar | engatinhar}
  \end{phonetics}
\end{entry}

\begin{entry}{爬杆}{8,7}{⽖、⽊}
  \begin{phonetics}{爬杆}{pa2gan1}
    \definition{s.}{escalada em poste}
    \definition{v.}{escalar um poste}
  \end{phonetics}
\end{entry}

\begin{entry}{爬竿}{8,9}{⽖、⽵}
  \begin{phonetics}{爬竿}{pa2gan1}
    \definition{s.}{poste de escalada | escalada em poste (como ginástica ou ato de circo)}
  \end{phonetics}
\end{entry}

\begin{entry}{爬梳}{8,11}{⽖、⽊}
  \begin{phonetics}{爬梳}{pa2shu1}
    \definition{v.}{vasculhar (documentos históricos, etc.) | desvendar}
  \end{phonetics}
\end{entry}

\begin{entry}{爬犁}{8,11}{⽖、⽜}
  \begin{phonetics}{爬犁}{pa2li2}
    \definition{s.}{trenó}
  \seealsoref{扒犁}{pa2li2}
  \end{phonetics}
\end{entry}

\begin{entry}{爬墙}{8,14}{⽖、⼟}
  \begin{phonetics}{爬墙}{pa2qiang2}
    \definition{v.}{escalar uma parede}
  \end{phonetics}
\end{entry}

\begin{entry}{爸}{8}{⽗}
  \begin{phonetics}{爸}{ba4}[][HSK 1]
    \definition[个,位]{s.}{(informal) pai}
  \seealsoref{爸爸}{ba4ba5}
  \end{phonetics}
\end{entry}

\begin{entry}{爸妈}{8,6}{⽗、⼥}
  \begin{phonetics}{爸妈}{ba4ma1}
    \definition{s.}{pai e mãe}
  \end{phonetics}
\end{entry}

\begin{entry}{爸爸}{8,8}{⽗、⽗}
  \begin{phonetics}{爸爸}{ba4ba5}[][HSK 1]
    \definition[个,位,名,群]{s.}{(informal) pai; papai; papa}
  \seealsoref{爸}{ba4}
  \end{phonetics}
\end{entry}

\begin{entry}{版}{8}{⽚}
  \begin{phonetics}{版}{ban3}[][HSK 5]
    \definition{clas.}{usado como uma palavra de medida para materiais impressos (por exemplo, livros, jornais, edições)}
    \definition{s.}{chapa, placa ou bloco de impressão | edição (livros impressos) | página (de um jornal) | moldes ou fromas de construção}
  \end{phonetics}
\end{entry}

\begin{entry}{牦}{8}{⽜}
  \begin{phonetics}{牦}{mao2}
    \definition[头]{s.}{iaque; boi da Tartária}
  \end{phonetics}
\end{entry}

\begin{entry}{牦牛}{8,4}{⽜、⽜}
  \begin{phonetics}{牦牛}{mao2niu2}
    \definition{s.}{iaque}
  \end{phonetics}
\end{entry}

\begin{entry}{物}{8}{⽜}
  \begin{phonetics}{物}{wu4}
    \definition{s.}{coisa; matéria; objeto | mundo exterior distinto de si mesmo; outras pessoas; refere-se a outras pessoas além de si mesmo ou ao ambiente em relação a si mesmo | essência; conteúdo; substância | criatura; criação}
  \end{phonetics}
\end{entry}

\begin{entry}{物业}{8,5}{⽜、⼀}
  \begin{phonetics}{物业}{wu4ye4}[][HSK 5]
    \definition[处]{s.}{propriedade; gestão de propriedades; gestão patrimonial; administração de imóveis | empresa de administração de imóveis; empresa de gestão imobiliária; empresa de administração de bens imóveis}
  \end{phonetics}
\end{entry}

\begin{entry}{物价}{8,6}{⽜、⼈}
  \begin{phonetics}{物价}{wu4 jia4}[][HSK 5]
    \definition[个]{s.}{preços das commodities; preços das matérias-primas; preço das mercadorias}
  \end{phonetics}
\end{entry}

\begin{entry}{物质}{8,8}{⽜、⾙}
  \begin{phonetics}{物质}{wu4zhi4}[][HSK 5]
    \definition[个]{s.}{matéria; substância; algo que existe além do espírito, que pode ser visto, tocado, cheirado ou detectado por instrumentos científicos | material; meios de subsistência; coisas que permitem às pessoas viver ou viver melhor, como comida, roupas, casas, dinheiro, etc.}
  \end{phonetics}
\end{entry}

\begin{entry}{物理}{8,11}{⽜、⽟}
  \begin{phonetics}{物理}{wu4li3}
    \definition{s.}{física (disciplina)}
  \end{phonetics}
\end{entry}

\begin{entry}{狒}{8}{⽝}
  \begin{phonetics}{狒}{fei4}
    \definition{s.}{babuíno (uma espécie de macaco)}
  \end{phonetics}
\end{entry}

\begin{entry}{狒狒}{8,8}{⽝、⽝}
  \begin{phonetics}{狒狒}{fei4fei4}
    \definition{s.}{babuíno}
  \end{phonetics}
\end{entry}

\begin{entry}{狗}{8}{⽝}
  \begin{phonetics}{狗}{gou3}[][HSK 2]
    \definition[条,只,群]{s.}{cão; cachorro | palavrão usado para se referir a pessoas más ou seus capangas}
  \end{phonetics}
\end{entry}

\begin{entry}{玩}{8}{⽟}
  \begin{phonetics}{玩}{wan2}
    \definition{s.}{brinquedo | algo usado para diversão}
    \definition{v.}{divertir-se | manter algo para entretenimento | brincar com}
  \end{phonetics}
\end{entry}

\begin{entry}{玩儿}{8,2}{⽟、⼉}
  \begin{phonetics}{玩儿}{wan2r5}[][HSK 1]
    \definition{v.}{divertir-se; (entretenimento) relaxar ou experimentar alguma atividade}
  \end{phonetics}
\end{entry}

\begin{entry}{玩艺}{8,4}{⽟、⾋}
  \begin{phonetics}{玩艺}{wan2yi4}
    \variantof{玩意}
  \end{phonetics}
\end{entry}

\begin{entry}{玩伴}{8,7}{⽟、⼈}
  \begin{phonetics}{玩伴}{wan2ban4}
    \definition{s.}{parceiro de brincadeira}
  \end{phonetics}
\end{entry}

\begin{entry}{玩具}{8,8}{⽟、⼋}
  \begin{phonetics}{玩具}{wan2ju4}[][HSK 3]
    \definition[个,件,套]{s.}{brinquedo; coisas para brincar}
  \end{phonetics}
\end{entry}

\begin{entry}{玩具厂}{8,8,2}{⽟、⼋、⼚}
  \begin{phonetics}{玩具厂}{wan2ju4chang3}
    \definition{s.}{fábrica de brinquedos}
  \end{phonetics}
\end{entry}

\begin{entry}{玩具车}{8,8,4}{⽟、⼋、⾞}
  \begin{phonetics}{玩具车}{wan2ju4 che1}
    \definition{s.}{carrinho de brinquedo}
  \end{phonetics}
\end{entry}

\begin{entry}{玩味}{8,8}{⽟、⼝}
  \begin{phonetics}{玩味}{wan2wei4}
    \definition{v.}{ponderar sutilezas | ruminar (pensamentos)}
  \end{phonetics}
\end{entry}

\begin{entry}{玩者}{8,8}{⽟、⽼}
  \begin{phonetics}{玩者}{wan2zhe3}
    \definition{s.}{jogador}
  \end{phonetics}
\end{entry}

\begin{entry}{玩耍}{8,9}{⽟、⽽}
  \begin{phonetics}{玩耍}{wan2shua3}
    \definition{v.}{divertir-me | brincar (como as crianças fazem)}
  \end{phonetics}
\end{entry}

\begin{entry}{玩家}{8,10}{⽟、⼧}
  \begin{phonetics}{玩家}{wan2jia1}
    \definition{s.}{entusiasta (áudio, modelos de aviões, etc.) | jogador (de um jogo)}
  \end{phonetics}
\end{entry}

\begin{entry}{玩偶}{8,11}{⽟、⼈}
  \begin{phonetics}{玩偶}{wan2'ou3}
    \definition{s.}{estatueta de brinquedo | boneco de ação | bicho de pelúcia | boneca}
  \end{phonetics}
\end{entry}

\begin{entry}{玩遍}{8,12}{⽟、⾡}
  \begin{phonetics}{玩遍}{wan2bian4}
    \definition{v.}{passear (todo o país, toda a cidade, etc.) | visitar (um grande número de lugares)}
  \end{phonetics}
\end{entry}

\begin{entry}{玩意}{8,13}{⽟、⼼}
  \begin{phonetics}{玩意}{wan2yi4}
    \definition{s.}{ato | brinquedo | coisa | truque (em uma performance, show de palco, acrobacias, etc.)}
  \end{phonetics}
\end{entry}

\begin{entry}{环}{8}{⽟}
  \begin{phonetics}{环}{huan2}[][HSK 3]
    \definition*{s.}{sobrenome Huan}
    \definition{clas.}{usado para anéis}
    \definition[个,串]{s.}{anel; arco | elo; \emph{link}; passo; etapa | anel; objeto em forma de círculo | arredores}
    \definition{v.}{cercar; rodear; circular; circundar}
  \end{phonetics}
\end{entry}

\begin{entry}{环卫}{8,3}{⽟、⼙}
  \begin{phonetics}{环卫}{huan2wei4}
    \definition{s.}{limpeza pública | saneamento urbano | saneamento ambiental | abreviação de 环境卫生}
  \seealsoref{环境卫生}{huan2jing4wei4sheng1}
  \end{phonetics}
\end{entry}

\begin{entry}{环节}{8,5}{⽟、⾋}
  \begin{phonetics}{环节}{huan2jie2}[][HSK 5]
    \definition{s.}{\emph{link}; ligação; vínculo; uma das muitas coisas que estão inter-relacionadas | segmento; estrutura anelar de alguns animais inferiores}
  \end{phonetics}
\end{entry}

\begin{entry}{环保}{8,9}{⽟、⼈}
  \begin{phonetics}{环保}{huan2 bao3}[][HSK 3]
    \definition{adj.}{ecológico; benefício para o meio ambiente; não prejudica o meio ambiente}
    \definition{s.}{proteção ambiental}
  \end{phonetics}
\end{entry}

\begin{entry}{环境}{8,14}{⽟、⼟}
  \begin{phonetics}{环境}{huan2jing4}[][HSK 3]
    \definition[个]{s.}{ambiente; os arredores | arredores; circunstâncias; condições políticas, econômicas, culturais, etc., dentro de um determinado âmbito}
  \end{phonetics}
\end{entry}

\begin{entry}{环境卫生}{8,14,3,5}{⽟、⼟、⼙、⽣}
  \begin{phonetics}{环境卫生}{huan2jing4wei4sheng1}
    \definition{s.}{saneamento ambiental}
  \seealsoref{环卫}{huan2wei4}
  \end{phonetics}
\end{entry}

\begin{entry}{现}{8}{⾒}
  \begin{phonetics}{现}{xian4}
    \definition{adj.}{presente | atual}
    \definition{v.}{aparecer}
  \seealsoref{见}{xian4}
  \end{phonetics}
\end{entry}

\begin{entry}{现代}{8,5}{⾒、⼈}
  \begin{phonetics}{现代}{xian4dai4}[][HSK 3]
    \definition*{s.}{Hyundai, empresa sul-coreana}
    \definition{adj.}{moderno; contemporâneo; com características, estilo e conceitos modernos, refletindo a vanguarda, a moda e a inovação da atualidade}
    \definition{s.}{tempos modernos; era contemporânea; atualmente, na divisão cronológica da história da China, refere-se principalmente ao período desde o Movimento 4 de Maio até os dias atuais}
  \end{phonetics}
\end{entry}

\begin{entry}{现在}{8,6}{⾒、⼟}
  \begin{phonetics}{现在}{xian4zai4}[][HSK 1]
    \definition{adv.}{agora; no momento; atualmente; neste momento, quando se fala, às vezes inclui um período de tempo mais ou menos longo antes ou depois da fala (diferente de 过去 ou 将来)}
  \seealsoref{过去}{guo4 qu4}
  \seealsoref{将来}{jiang1lai2}
  \end{phonetics}
\end{entry}

\begin{entry}{现场}{8,6}{⾒、⼟}
  \begin{phonetics}{现场}{xian4chang3}[][HSK 3]
    \definition[个,处]{s.}{local onde ocorreu o acidente, incidente ou desastre| local; ponto; local onde se realizam diretamente atividades como produção, apresentações e competições}
  \end{phonetics}
\end{entry}

\begin{entry}{现有}{8,6}{⾒、⽉}
  \begin{phonetics}{现有}{xian4 you3}[][HSK 5]
    \definition{adj.}{agora disponível; existente}
    \definition{v.}{estar disponível agora; existir | (literário) ter em mãos; ter em posse}
  \end{phonetics}
\end{entry}

\begin{entry}{现抓}{8,7}{⾒、⼿}
  \begin{phonetics}{现抓}{xian4zhua1}
    \definition{v.}{improvisar}
  \end{phonetics}
\end{entry}

\begin{entry}{现状}{8,7}{⾒、⽝}
  \begin{phonetics}{现状}{xian4zhuang4}[][HSK 5]
    \definition{s.}{situação atual; situação atual}
  \end{phonetics}
\end{entry}

\begin{entry}{现实}{8,8}{⾒、⼧}
  \begin{phonetics}{现实}{xian4shi2}[][HSK 3]
    \definition{adj.}{real; efetivo; verdadeiro; de acordo com circunstâncias objetivas}
    \definition[个]{s.}{realidade; factualidade; coisas que existem objetivamente}
  \end{phonetics}
\end{entry}

\begin{entry}{现货}{8,8}{⾒、⾙}
  \begin{phonetics}{现货}{xian4huo4}
    \definition{s.}{produtos à vista}
  \end{phonetics}
\end{entry}

\begin{entry}{现货的}{8,8,8}{⾒、⾙、⽩}
  \begin{phonetics}{现货的}{xian4huo4 de5}
    \definition{s.}{produtos em estoque}
  \end{phonetics}
\end{entry}

\begin{entry}{现金}{8,8}{⾒、⾦}
  \begin{phonetics}{现金}{xian4jin1}[][HSK 3]
    \definition[笔]{s.}{dinheiro; dinheiro vivo; moeda que pode ser usada diretamente | reserva de dinheiro em um banco; o dinheiro guardado no cofre do banco}
  \end{phonetics}
\end{entry}

\begin{entry}{现做}{8,11}{⾒、⼈}
  \begin{phonetics}{现做}{xian4zuo4}
    \definition{adj.}{fresco}
    \definition{v.}{fazer (comida) no local}
  \end{phonetics}
\end{entry}

\begin{entry}{现象}{8,11}{⾒、⾗}
  \begin{phonetics}{现象}{xian4xiang4}[][HSK 3]
    \definition[个,种]{s.}{aparência (das coisas); fenômeno; a forma externa e as relações manifestadas pelas coisas em seu desenvolvimento e mudança}
  \end{phonetics}
\end{entry}

\begin{entry}{画}{8}{⽥}
  \begin{phonetics}{画}{hua4}[][HSK 2]
    \definition*{s.}{sobrenome Hua}
    \definition{clas.}{traços (de um caractere chinês)}
    \definition[张,幅]{s.}{desenho; pintura; imagem; figura desenhada | traço horizontal (em caracteres chineses)}
    \definition{v.}{desenhar; pintar | desenhar; marcar; assinar}
  \seealsoref{划}{hua4}
  \end{phonetics}
\end{entry}

\begin{entry}{画儿}{8,2}{⽥、⼉}
  \begin{phonetics}{画儿}{hua4r5}[][HSK 2]
    \definition[幅,张]{s.}{imagem; desenho; pintura; obra de arte pintada}
  \end{phonetics}
\end{entry}

\begin{entry}{画地为牢}{8,6,4,7}{⽥、⼟、⼂、⼧}
  \begin{phonetics}{画地为牢}{hua4di4wei2lao2}
    \definition{expr.}{(literalmente) ser confinado dentro de um círculo desenhado no chão | (figurativo) limitar-se a uma gama restrita de atividades}
  \end{phonetics}
\end{entry}

\begin{entry}{画面}{8,9}{⽥、⾯}
  \begin{phonetics}{画面}{hua4 mian4}[][HSK 5]
    \definition[个,幅]{s.}{quadro; aparência geral de uma imagem; imagem apresentada no quadro, na tela, etc.}
  \end{phonetics}
\end{entry}

\begin{entry}{画家}{8,10}{⽥、⼧}
  \begin{phonetics}{画家}{hua4 jia1}[][HSK 2]
    \definition[个,位,名,些]{s.}{pintor; pessoa com talento para pintura}
  \end{phonetics}
\end{entry}

\begin{entry}{畅}{8}{⽥}
  \begin{phonetics}{畅}{chang4}
    \definition*{s.}{sobrenome Chang}
    \definition{adj.}{suave; desimpedido; sem obstáculos; desobstruído | livre; desinibido}
  \end{phonetics}
\end{entry}

\begin{entry}{的}{8}{⽩}
  \begin{phonetics}{的}{de5}
    \definition{part.}{usado para indicar posse | formar uma frase nominal ou expressão nominal | substituir a pessoa ou coisa mencionada anteriormente | no final de uma frase declarativa, para dar ênfase; usado após o verbo predicativo, enfatiza o agente da ação, o tempo, o local, etc. | usado no final de uma frase declarativa, expressa afirmação, ênfase, certeza, etc. | indica que alguém obteve uma determinada posição ou status | usado com 是 para indicar predicado ou ênfase; indica que alguém é o objeto da ação | e assim por diante; e assim por diante; e similares; usado após palavras paralelas, significa 等等, 之类 | indica uma ação (o pronome é o objeto da ação); combinado com o verbo anterior, expressa uma ação, e o pronome é o objeto dessa ação}
  \seealsoref{等等}{deng3 deng3}
  \seealsoref{是}{shi4}
  \seealsoref{之类}{zhi1 lei4}
  \end{phonetics}
  \begin{phonetics}{的}{di1}
    \definition{s.}{abreviação de 的士: um táxi}
  \seealsoref{的士}{di1shi4}
  \end{phonetics}
  \begin{phonetics}{的}{di2}
    \definition{adv.}{verdadeiramente; exatamente; realmente}
  \end{phonetics}
  \begin{phonetics}{的}{di4}
    \definition{adj.}{alvo; centro do alvo}
  \end{phonetics}
\end{entry}

\begin{entry}{的士}{8,3}{⽩、⼠}
  \begin{phonetics}{的士}{di1shi4}
    \definition{s.}{(empréstimo linguístico) táxi}
  \end{phonetics}
\end{entry}

\begin{entry}{的时候}{8,7,10}{⽩、⽇、⼈}
  \begin{phonetics}{的时候}{de5 shi2hou4}
    \definition{part.}{naquele momento; quando; em; descreve o momento específico em que um evento ocorreu}
  \end{phonetics}
\end{entry}

\begin{entry}{的话}{8,8}{⽩、⾔}
  \begin{phonetics}{的话}{de5 hua4}[][HSK 2]
    \definition{part.}{se; caso; suponha que; partícula usada após uma frase hipotética para introduzir o texto seguinte}
  \end{phonetics}
\end{entry}

\begin{entry}{的确}{8,12}{⽩、⽯}
  \begin{phonetics}{的确}{di2que4}[][HSK 4]
    \definition{adv.}{realmente; de fato, ao expressar certeza sobre a situação}
  \end{phonetics}
\end{entry}

\begin{entry}{盲}{8}{⽬}
  \begin{phonetics}{盲}{mang2}
    \definition{adj.}{cego | incapaz de distinguir coisas | totalmente incompetente}
    \definition{adv.}{cegamente}
  \end{phonetics}
\end{entry}

\begin{entry}{盲目}{8,5}{⽬、⽬}
  \begin{phonetics}{盲目}{mang2mu4}
    \definition{adj.}{ignorante | sem compreensão}
    \definition{adv.}{cegamente}
    \definition{s.}{cego}
  \end{phonetics}
\end{entry}

\begin{entry}{直}{8}{⽬}
  \begin{phonetics}{直}{zhi2}[][HSK 3]
    \definition*{s.}{sobrenome Zhi}
    \definition{adj.}{reto (oposto a 弯,曲) | vertical; perpendicular (oposto a 横) | justo; íntegro; imparcial | franco; sincero; direto ao ponto | rígido; entorpecido | direto; em linha reta; rígido | ereto; perpendicular ao solo}
    \definition{adv.}{diretamente; sempre; reto | continuamente; constantemente | apenas; simplesmente | de ​​fato}
    \definition[条]{s.}{traço vertical (em caracteres chineses, 竖)}
    \definition{v.}{endireitar; tornar reto | alongar}
  \seealsoref{横}{heng2}
  \seealsoref{曲}{qu1}
  \seealsoref{竖}{shu4}
  \seealsoref{弯}{wan1}
  \end{phonetics}
\end{entry}

\begin{entry}{直译}{8,7}{⽬、⾔}
  \begin{phonetics}{直译}{zhi2yi4}
    \definition{s.}{tradução literal}
  \seealsoref{意译}{yi4yi4}
  \end{phonetics}
\end{entry}

\begin{entry}{直译器}{8,7,16}{⽬、⾔、⼝}
  \begin{phonetics}{直译器}{zhi2yi4qi4}
    \definition{s.}{(computação) interpretador}
  \end{phonetics}
\end{entry}

\begin{entry}{直到}{8,8}{⽬、⼑}
  \begin{phonetics}{直到}{zhi2 dao4}[][HSK 3]
    \definition{adv.}{até (geralmente se refere ao tempo); até que}
  \end{phonetics}
\end{entry}

\begin{entry}{直线}{8,8}{⽬、⽷}
  \begin{phonetics}{直线}{zhi2 xian4}[][HSK 5]
    \definition{adj.}{direto; retilíneo | íngreme; acentuada (subida ou descida)}
    \definition[条]{s.}{linha reta}
  \end{phonetics}
\end{entry}

\begin{entry}{直接}{8,11}{⽬、⼿}
  \begin{phonetics}{直接}{zhi2jie1}[][HSK 2]
    \definition{adj.}{direto (oposto: indireto 间接) | imediato}
  \seealsoref{间接}{jian4jie1}
  \end{phonetics}
\end{entry}

\begin{entry}{直播}{8,15}{⽬、⼿}
  \begin{phonetics}{直播}{zhi2bo1}[][HSK 3]
    \definition{v.}{transmissão ao vivo; transmitir diretamente, sem gravar}
  \end{phonetics}
\end{entry}

\begin{entry}{知了}{8,2}{⽮、⼅}
  \begin{phonetics}{知了}{zhi1liao3}
    \definition[通]{s.}{cigarra}
  \end{phonetics}
\end{entry}

\begin{entry}{知识}{8,7}{⽮、⾔}
  \begin{phonetics}{知识}{zhi1shi5}[][HSK 1]
    \definition[个,门,种]{s.}{conhecimento; conjunto de conhecimentos e experiências adquiridos pelas pessoas na prática de transformar o mundo | intelectual; refere-se à cultura acadêmica}
  \end{phonetics}
\end{entry}

\begin{entry}{知道}{8,12}{⽮、⾡}
  \begin{phonetics}{知道}{zhi1dao4}[][HSK 1]
    \definition{v.}{saber; perceber; estar ciente de; ter conhecimento dos fatos ou da razão; ser sensato}
  \end{phonetics}
\end{entry}

\begin{entry}{知道了}{8,12,2}{⽮、⾡、⼅}
  \begin{phonetics}{知道了}{zhi1dao4le5}
    \definition{interj.}{Entendi! | OK!}
  \end{phonetics}
\end{entry}

\begin{entry}{矿}{8}{⽯}
  \begin{phonetics}{矿}{kuang4}
    \definition[个,座]{s.}{depósito de minério | minério | mina}
  \end{phonetics}
\end{entry}

\begin{entry}{矿泉水}{8,9,4}{⽯、⽔、⽔}
  \begin{phonetics}{矿泉水}{kuang4quan2shui3}[][HSK 4]
    \definition[瓶,杯]{s.}{água mineral de nascente}
  \end{phonetics}
\end{entry}

\begin{entry}{码}{8}{⽯}
  \begin{phonetics}{码}{ma3}
    \definition{clas.}{refere-se a um assunto específico ou a uma categoria de assuntos; refere-se a uma coisa ou a uma classe de coisas | jarda; unidade de comprimento britânica e americana}
    \definition{s.}{um sinal ou objeto que indica número; código; símbolos ou ferramentas que indicam números}
    \definition{v.}{empilhar; acumular}
  \end{phonetics}
\end{entry}

\begin{entry}{码头}{8,5}{⽯、⼤}
  \begin{phonetics}{码头}{ma3tou2}[][HSK 5]
    \definition[个]{s.}{doca; cais; píer; molhe; edifícios à beira-mar ou à beira do rio destinados exclusivamente à atracação de embarcações, embarque e desembarque de passageiros e carga e descarga de mercadorias | cidade portuária; centro comercial e de transportes; refere-se a uma cidade comercial com transporte terrestre e marítimo bem desenvolvido.}
  \end{phonetics}
\end{entry}

\begin{entry}{祅}{8}{⽰}
  \begin{phonetics}{祅}{yao1}
    \definition{s.}{espírito maligno | \emph{goblin} | bruxaria}
    \variantof{妖}
  \end{phonetics}
\end{entry}

\begin{entry}{空}{8}{⽳}
  \begin{phonetics}{空}{kong1}[][HSK 3]
    \definition*{s.}{sobrenome Kong}
    \definition{adj.}{vazio; oco; nulo; não inclui nada; não contém nada ou não tem conteúdo; irrealista}
    \definition{adv.}{por nada; em vão; sem efeito}
    \definition{s.}{céu; ar | vazio; vazio do mundo dos sentidos}
  \end{phonetics}
  \begin{phonetics}{空}{kong4}[][HSK 4]
    \definition*{s.}{sobrenome Quan}
    \definition{adj.}{vazio; oco; nulo; que não contém nada; que não tem nada ou nenhum conteúdo; impraticável}
    \definition{adv.}{para nada; em vão; sem efeito}
    \definition{s.}{céu; ar | vazio; ausência do mundo dos sentidos}
  \end{phonetics}
\end{entry}

\begin{entry}{空儿}{8,2}{⽳、⼉}
  \begin{phonetics}{空儿}{kong4r5}[][HSK 3]
    \definition[个]{s.}{tempo livre; sem horário específico | sala; espaço (não utilizado); área ainda não utilizada}
    \definition{v.}{ter tempo livre}
  \end{phonetics}
\end{entry}

\begin{entry}{空中}{8,4}{⽳、⼁}
  \begin{phonetics}{空中}{kong1 zhong1}[][HSK 5]
    \definition{adj.}{aéreo; aerotransportado; refere-se à transmissão de sinais de rádio}
    \definition{s.}{no céu; no ar}
  \end{phonetics}
\end{entry}

\begin{entry}{空中小姐}{8,4,3,8}{⽳、⼁、⼩、⼥}
  \begin{phonetics}{空中小姐}{kong1zhong1xiao3jie3}
    \definition{s.}{aeromoça}
  \end{phonetics}
\end{entry}

\begin{entry}{空心菜}{8,4,11}{⽳、⼼、⾋}
  \begin{phonetics}{空心菜}{kong1xin1cai4}
    \definition{s.}{espinafre aquático | \emph{ong choy} | repolho do pântano | convolvulus aquático | glória-da-manhã aquática}
  \seealsoref{蕹菜}{weng4cai4}
  \end{phonetics}
\end{entry}

\begin{entry}{空气}{8,4}{⽳、⽓}
  \begin{phonetics}{空气}{kong1qi4}[][HSK 2]
    \definition[缕,股,份,阵]{s.}{ar; gases que compõe a atmosfera terrestre | atmosfera}
  \end{phonetics}
\end{entry}

\begin{entry}{空间}{8,7}{⽳、⾨}
  \begin{phonetics}{空间}{kong1jian1}[][HSK 4]
    \definition[个]{s.}{espaço; recinto; cômodo; espaço em branco; interespaço}
  \end{phonetics}
\end{entry}

\begin{entry}{空间站}{8,7,10}{⽳、⾨、⽴}
  \begin{phonetics}{空间站}{kong1jian1zhan4}
    \definition{s.}{estação espacial}
  \end{phonetics}
\end{entry}

\begin{entry}{空姐}{8,8}{⽳、⼥}
  \begin{phonetics}{空姐}{kong1jie3}
    \definition{s.}{aeromoça | comissária de bordo | abreviação de 空中小姐}
  \seealsoref{空中小姐}{kong1zhong1xiao3jie3}
  \end{phonetics}
\end{entry}

\begin{entry}{空调}{8,10}{⽳、⾔}
  \begin{phonetics}{空调}{kong1tiao2}[][HSK 3]
    \definition[台,个]{s.}{ar-condicionado;  condicionador de ar}
  \end{phonetics}
\end{entry}

\begin{entry}{线}{8}{⽷}
  \begin{phonetics}{线}{xian4}[][HSK 3]
    \definition{clas.}{usado para coisas abstratas, o número é limitado a ``一''}
    \definition[根,个]{s.}{fio; corda; arame; objetos finos e longos feitos de seda, algodão, metal, etc. | linha; figura formada pelo movimento arbitrário de um ponto| feito de fio de algodão | algo em forma de linha, fio, etc. | rota de transporte; linha | linha de demarcação; limite; zona de fronteira; zona de transição | beira; borda | linha ideológica e política | pista; fio}
  \end{phonetics}
\end{entry}

\begin{entry}{线香}{8,9}{⽷、⾹}
  \begin{phonetics}{线香}{xian4xiang1}
    \definition{s.}{bastão ou vareta de incenso}
  \end{phonetics}
\end{entry}

\begin{entry}{线索}{8,10}{⽷、⽷}
  \begin{phonetics}{线索}{xian4suo3}[][HSK 5]
    \definition[条,个]{s.}{pista; fio; metáfora para o desenvolvimento das coisas ou a maneira de explorar um problema | fio; linha; refere-se ao contexto de desenvolvimento do enredo em obras literárias}
  \end{phonetics}
\end{entry}

\begin{entry}{练}{8}{⽷}
  \begin{phonetics}{练}{lian4}[][HSK 2]
    \definition*{s.}{sobrenome Lian}
    \definition{adj.}{habilidoso; experiente; bem treinado}
    \definition{s.}{seda branca}
    \definition{v.}{tratar, amaciar e branquear a seda por meio de fervura; cozinhar seda crua ou tecidos de seda crua | treinar; praticar; exercitar}
  \end{phonetics}
\end{entry}

\begin{entry}{练习}{8,3}{⽷、⼄}
  \begin{phonetics}{练习}{lian4xi2}[][HSK 2]
    \definition[项,次]{s.}{exercício (em livros); tarefas ou exercícios organizados para consolidar os resultados da aprendizagem}
    \definition{v.}{praticar; exercitar; repitir várias vezes até ficar bem treinado}
  \end{phonetics}
\end{entry}

\begin{entry}{组}{8}{⽷}
  \begin{phonetics}{组}{zu3}[][HSK 2]
    \definition{clas.}{usado para conjuntos, séries, suítes, baterias}
    \definition[个]{s.}{grupo; uma unidade composta por um pequeno número de pessoas}
    \definition{v.}{formar; organizar; combinar pessoas ou coisas dispersas em um todo ou sistema}
  \end{phonetics}
\end{entry}

\begin{entry}{组长}{8,4}{⽷、⾧}
  \begin{phonetics}{组长}{zu3 zhang3}[][HSK 2]
    \definition[名,位,个]{s.}{líder de grupo; um supervisor de grupo}
  \end{phonetics}
\end{entry}

\begin{entry}{组合}{8,6}{⽷、⼝}
  \begin{phonetics}{组合}{zu3he2}[][HSK 3]
    \definition{s.}{associação; combinação; o todo organizado | combinação; retirar n elementos diferentes de m elementos e agrupá-los, independentemente da ordem, em que cada grupo contenha pelo menos um elemento diferente, o resultado obtido é chamado de combinação de n elementos de m}
    \definition{v.}{compor; constituir; formar}
  \end{phonetics}
\end{entry}

\begin{entry}{组成}{8,6}{⽷、⼽}
  \begin{phonetics}{组成}{zu3cheng2}[][HSK 2]
    \definition{v.}{formar; compor; inventar}
  \end{phonetics}
\end{entry}

\begin{entry}{组织}{8,8}{⽷、⽷}
  \begin{phonetics}{组织}{zu3zhi1}[][HSK 5]
    \definition{s.}{organização; um coletivo ou grupo estabelecido de acordo com determinados objetivos e princípios | sistema organizado; vários fatores interligados de determinada maneira, formando um sistema | tecer; a combinação de linhas horizontais e verticais nos têxteis | tecido; os seres humanos, os animais, as plantas e outros seres vivos são compostos por uma combinação de células com formas e funções semelhantes, que formam os tecidos; os tecidos são as unidades que compõem os diversos órgãos}
  \end{phonetics}
\end{entry}

\begin{entry}{细}{8}{⽷}
  \begin{phonetics}{细}{xi4}[][HSK 4]
    \definition{adj.}{fino; delgado; esguio; esbelto; em oposição a 粗 | fino; em partículas pequenas; grãos pequenos | fino e macio;  um sussuro | fino; requintado; delicado | cuidadoso; detalhado; meticuloso | ínfimo; minúsculo; insignificante; diminuto | jovem; pequeno}
  \seealsoref{粗}{cu1}
  \end{phonetics}
\end{entry}

\begin{entry}{细节}{8,5}{⽷、⾋}
  \begin{phonetics}{细节}{xi4jie2}[][HSK 4]
    \definition{s.}{detalhe; particularidade; aspectos secundários ou partes sutis de um enredo ou episódios secundários usados em uma obra literária para expressar o caráter de uma pessoa ou as características essenciais de uma coisa}
  \end{phonetics}
\end{entry}

\begin{entry}{细致}{8,10}{⽷、⾄}
  \begin{phonetics}{细致}{xi4zhi4}[][HSK 4]
    \definition{adj.}{meticuloso; cuidadoso; minucioso | intrincado; delicado}
  \end{phonetics}
\end{entry}

\begin{entry}{细菌战}{8,11,9}{⽷、⾋、⼽}
  \begin{phonetics}{细菌战}{xi4jun1zhan4}
    \definition{s.}{guerra biológica}
  \end{phonetics}
\end{entry}

\begin{entry}{织}{8}{⽷}
  \begin{phonetics}{织}{zhi1}
    \definition{v.}{tecer | tricotar}
  \end{phonetics}
\end{entry}

\begin{entry}{终于}{8,3}{⽷、⼆}
  \begin{phonetics}{终于}{zhong1yu2}[][HSK 3]
    \definition{adv.}{finalmente; por fim; eventualmente; no final; indica uma situação que surge após várias mudanças ou espera}
  \end{phonetics}
\end{entry}

\begin{entry}{终止}{8,4}{⽷、⽌}
  \begin{phonetics}{终止}{zhong1zhi3}[][HSK 5]
    \definition{v.}{parar; terminar | anular; encerrar; expirar; revogar}
  \end{phonetics}
\end{entry}

\begin{entry}{终究}{8,7}{⽷、⽳}
  \begin{phonetics}{终究}{zhong1jiu1}
    \definition{adv.}{afinal de contas; enfatiza que, não importa o que aconteça, a natureza das pessoas e das coisas não mudará e que as características básicas devem ser reconhecidas (tem o efeito de fortalecer o tom) |  no final; indica que um determinado resultado ocorrerá ou não, frequentemente usado em especulações, julgamentos etc. | afinal de contas; indica que, apesar do grande esforço ou da grande esperança, o resultado objetivo ainda é insatisfatório, geralmente com o significado de pesar ou pena | afinal de contas; indica que um resultado desejado finalmente aparece}
  \end{phonetics}
\end{entry}

\begin{entry}{终身}{8,7}{⽷、⾝}
  \begin{phonetics}{终身}{zhong1shen1}[][HSK 5]
    \definition{s.}{vida inteira; por toda a vida; por toda a vida}
  \end{phonetics}
\end{entry}

\begin{entry}{终点}{8,9}{⽷、⽕}
  \begin{phonetics}{终点}{zhong1dian3}[][HSK 5]
    \definition[个]{s.}{destino; ponto terminal; ponto de chegada; lugar onde uma jornada termina | final; refere-se especificamente ao local onde a corrida é interrompida}
  \end{phonetics}
\end{entry}

\begin{entry}{经}{8}{⽷}
  \begin{phonetics}{经}{jing1}
    \definition*{s.}{sobrenome Jing}
    \definition{s.}{livro sagrado | escritura | clássicos | longitude | menstruação | canal}
    \definition{v.}{passar | sofrer | suportar | deformar (têxtil)}
  \end{phonetics}
\end{entry}

\begin{entry}{经历}{8,4}{⽷、⼚}
  \begin{phonetics}{经历}{jing1li4}[][HSK 3]
    \definition[个,次,段,种]{s.}{experiência; coisas que você viu, fez ou sofreu pessoalmente}
    \definition{v.}{passar por; atravessar; ter visto, feito ou sofrido pessoalmente}
  \end{phonetics}
\end{entry}

\begin{entry}{经过}{8,6}{⽷、⾡}
  \begin{phonetics}{经过}{jing1guo4}[][HSK 2]
    \definition{prep.}{depois; através; como resultado de; passar por uma atividade ou evento que traz novas mudanças para pessoas ou coisas}
    \definition[个,段,番]{s.}{processo; curso; experiência}
    \definition{v.}{passar; atravessar; passar por; através de (local, tempo, ação, etc.)}
  \end{phonetics}
\end{entry}

\begin{entry}{经典}{8,8}{⽷、⼋}
  \begin{phonetics}{经典}{jing1dian3}[][HSK 4]
    \definition{adj.}{clássico; (escritos ou obras, etc.) que são típicos, autorizados}
    \definition{s.}{clássicos; escritos tradicionais e valiosos; os livros mais importantes e fundamentais da religião | escrituras; escritos de doutrinas religiosas}
  \end{phonetics}
\end{entry}

\begin{entry}{经济}{8,9}{⽷、⽔}
  \begin{phonetics}{经济}{jing1ji4}[][HSK 3]
    \definition{adj.}{econômico;  parcimonioso; descreve algo que custa pouco e rende muito; preço acessível}
    \definition{s.}{economia; a soma das relações de produção social em um determinado período histórico|econômico; de valor industrial ou econômico; refere-se à economia nacional; também se refere a um determinado setor da economia nacional | economia; refere-se às atividades econômicas, incluindo produção, circulação, distribuição e consumo, bem como atividades ou processos financeiros, de seguros, etc. | renda; situação financeira; refere-se à situação financeira de uma pessoa}
    \definition{v.}{governar o país e beneficiar o povo}
  \end{phonetics}
\end{entry}

\begin{entry}{经费}{8,9}{⽷、⾙}
  \begin{phonetics}{经费}{jing1fei4}[][HSK 5]
    \definition[笔]{s.}{fundos; desembolso; gastos | despesas; gastos}
  \end{phonetics}
\end{entry}

\begin{entry}{经验}{8,10}{⽷、⾺}
  \begin{phonetics}{经验}{jing1yan4}[][HSK 3]
    \definition[个,次,种]{s.}{experiência; conhecimento ou habilidades adquiridos através da prática}
    \definition{v.}{experimentar; passar por; ter visto, feito ou sofrido pessoalmente}
  \end{phonetics}
\end{entry}

\begin{entry}{经常}{8,11}{⽷、⼱}
  \begin{phonetics}{经常}{jing1chang2}[][HSK 2]
    \definition{adj.}{habitual; cotidiano; diário; do dia a dia}
    \definition{adv.}{frequentemente; regularmente; constantemente; com frequência; indica que a ação ocorre repetidamente}
  \end{phonetics}
\end{entry}

\begin{entry}{经理}{8,11}{⽷、⽟}
  \begin{phonetics}{经理}{jing1li3}[][HSK 2]
    \definition[个,位,名]{s.}{gerente; diretor; pessoas responsáveis pela gestão e administração de empresas ou corporações}
  \end{phonetics}
\end{entry}

\begin{entry}{经营}{8,11}{⽷、⾋}
  \begin{phonetics}{经营}{jing1ying2}[][HSK 3]
    \definition{v.}{executar; gerenciar; operar; envolver-se em; planejar e gerenciar (empresas, etc.) | gerenciar; refere-se a planos e organizações em geral}
  \end{phonetics}
\end{entry}

\begin{entry}{罔}{8}{⼌}
  \begin{phonetics}{罔}{wang3}
    \definition{v.}{enganar}
  \end{phonetics}
\end{entry}

\begin{entry}{罗}{8}{⽹}
  \begin{phonetics}{罗}{luo2}
    \definition*{s.}{sobrenome Luo}
    \definition{v.}{coletar | juntar | pegar | peneirar}
  \end{phonetics}
\end{entry}

\begin{entry}{者}{8}{⽼}
  \begin{phonetics}{者}{zhe3}[][HSK 3]
    \definition{part.}{significa 是 e é usado após palavras, frases e orações para indicar uma pausa}
    \definition{pron.}{usado para se referir à pessoa, coisa ou assunto que realiza uma ação ou possui um determinado atributo | pessoas; caras (Usado para se referir a alguém envolvido em uma determinada profissão, que acredita em uma determinada ideologia ou que tem uma forte tendência para algo) | usado após certos números ou palavras direcionais para se referir a coisas mencionadas anteriormente | significado semelhante a 这 (mais comum na linguagem coloquial antiga)}
    \definition{s.}{sobrenome Zhe}
  \seealsoref{是}{shi4}
  \seealsoref{这}{zhe4}
  \end{phonetics}
\end{entry}

\begin{entry}{肏}{8}{⼊}
  \begin{phonetics}{肏}{cao4}
    \definition{v.}{(vulgar) foder; palavras sujas usadas para insultar pessoas; refere-se à relação sexual masculina}
  \end{phonetics}
\end{entry}

\begin{entry}{肥}{8}{⾁}
  \begin{phonetics}{肥}{fei2}[][HSK 4]
    \definition{adj.}{gordo; gorduroso; contém muita gordura (o oposto de 瘦, geralmente não usado para descrever pessoas) | fértil; rico | solto; largo; folgado; (roupas, etc.) largas (em oposição a 瘦) | lucrativo; rendendo bons lucros}
    \definition{s.}{fertilizante; esterco}
    \definition{v.}{fertilizar; tornar fértil ou obeso | enriquecer com renda ilegal, ilícita}
  \seealsoref{瘦}{shou4}
  \end{phonetics}
\end{entry}

\begin{entry}{肩}{8}{⾁}
  \begin{phonetics}{肩}{jian1}[][HSK 5]
    \definition*{s.}{sobrenome Jian}
    \definition{s.}{ombro; torso}
    \definition{v.}{assumir; empreender; carregar; suportar; suportar um fardo}
  \end{phonetics}
\end{entry}

\begin{entry}{肩膀}{8,14}{⾁、⾁}
  \begin{phonetics}{肩膀}{jian1bang3}
    \definition{s.}{ombro}
  \end{phonetics}
\end{entry}

\begin{entry}{肯}{8}{⾁}
  \begin{phonetics}{肯}{ken3}
    \definition{s.}{carne presa ao osso}
    \definition{v.}{concordar; consentir}
    \definition{v.aux.}{estar disposto a; estar pronto para; para expressar vontade subjetiva; vontade de aceitar}
  \end{phonetics}
\end{entry}

\begin{entry}{肯定}{8,8}{⾁、⼧}
  \begin{phonetics}{肯定}{ken3ding4}[][HSK 5]
    \definition{adj.}{certo; definitivo; positivo; afirmativo | positivo; afirmativo; aceitável}
    \definition{adv.}{com certeza | certamente | definitivamente | afirmativo (resposta)}
    \definition{adv.}{certamente; definitivamente; sem dúvida; sem dúvida alguma}
    \definition{v.}{afirmar; aprovar; confirmar; considerar positivo; reconhecer a existência de algo ou sua autenticidade ou racionalidade (em oposição à 否定)}
  \seealsoref{否定}{fou3ding4}
  \end{phonetics}
\end{entry}

\begin{entry}{舍}{8}{⾆}
  \begin{phonetics}{舍}{she3}
    \definition{v.}{abandonar; desistir; descartar; jogar fora | dar esmola; dispensar caridade}
  \end{phonetics}
  \begin{phonetics}{舍}{she4}
    \definition*{s.}{sobrenome She}
    \definition{clas.}{uma unidade antiga de distância igual a 30 li, 里}
    \definition{pron.}{meu, uma palavra humilde usada para se referir aos parentes mais jovens ou de geração inferior}
    \definition{s.}{1. cabana; casa
2. minha casa; minha humilde morada
chiqueiro; galpão; curral de gado}
  \seealsoref{里}{li3}
  \end{phonetics}
\end{entry}

\begin{entry}{舍不得}{8,4,11}{⾆、⼀、⼻}
  \begin{phonetics}{舍不得}{she3bu5de5}[][HSK 5]
    \definition{v.}{não se pode abandonar ou deixar, não se quer usar ou descartar; detestar separar-me ou usar}
  \end{phonetics}
\end{entry}

\begin{entry}{舍得}{8,11}{⾆、⼻}
  \begin{phonetics}{舍得}{she3 de5}[][HSK 5]
    \definition{v.}{não guardar rancor; estar disposto a abrir mão de algo; estar disposto a gastar dinheiro, tempo, etc.; estar disposto a abrir mão de pessoas, oportunidades, coisas, etc. que são importantes para você}
  \end{phonetics}
\end{entry}

\begin{entry}{艰}{8}{⾉}
  \begin{phonetics}{艰}{jian1}
    \definition{adj.}{difícil; duro}
  \end{phonetics}
\end{entry}

\begin{entry}{艰苦}{8,8}{⾉、⾋}
  \begin{phonetics}{艰苦}{jian1ku3}[][HSK 5]
    \definition{adj.}{duro; resistente; árduo; difícil; condições de trabalho ou de vida ruins que tornam as pessoas miseráveis}
  \end{phonetics}
\end{entry}

\begin{entry}{艰难}{8,10}{⾉、⾫}
  \begin{phonetics}{艰难}{jian1nan2}[][HSK 5]
    \definition{adj.}{duro; árduo; difícil}
  \end{phonetics}
\end{entry}

\begin{entry}{若}{8}{⾋}
  \begin{phonetics}{若}{ruo4}
    \definition*{s.}{sobrenome Ruo}
    \definition{adv.}{como se; como se fosse; usado antes do verbo para indicar que o que foi dito é mais ou menos assim, equivalente a 好像}
    \definition{conj.}{se; usado na primeira parte de uma frase composta, expressa uma relação hipotética, equivalente a 如果}
    \definition{pron.}{você; referir-se ao interlocutor como 你 ou 你的}
    \definition{v.}{parecer}
  \seealsoref{好像}{hao3xiang4}
  \seealsoref{你}{ni3}
  \seealsoref{你的}{ni3 de5}
  \seealsoref{如果}{ru2guo3}
  \end{phonetics}
\end{entry}

\begin{entry}{苦}{8}{⾋}
  \begin{phonetics}{苦}{ku3}[][HSK 4]
    \definition{adj.}{amargo | difícil; doloroso; sofrido | desgastado; gasto demais}
    \definition{adv.}{meticulosamente; fazendo o máximo possível; de forma árdua; pacientemente}
    \definition{v.}{causar sofrimento a alguém; causar dificuldades a alguém | sofrer com; ser incomodado por; sentir-se angustiado com uma situação}
  \end{phonetics}
\end{entry}

\begin{entry}{苦瓜}{8,5}{⾋、⽠}
  \begin{phonetics}{苦瓜}{ku3gua1}
    \definition{s.}{melão amargo (cabaça amarga, pêra bálsamo, maçã bálsamo, pepino amargo)}
  \end{phonetics}
\end{entry}

\begin{entry}{英文}{8,4}{⾋、⽂}
  \begin{phonetics}{英文}{ying1 wen2}[][HSK 2]
    \definition{s.}{inglês, língua inglesa; a forma escrita do inglês}
  \end{phonetics}
\end{entry}

\begin{entry}{英国}{8,8}{⾋、⼞}
  \begin{phonetics}{英国}{ying1guo2}
    \definition*{s.}{Reino Unido}
  \end{phonetics}
\end{entry}

\begin{entry}{英国人}{8,8,2}{⾋、⼞、⼈}
  \begin{phonetics}{英国人}{ying1guo2ren2}
    \definition{s.}{inglês | pessoa ou povo do Reino Unido}
  \end{phonetics}
\end{entry}

\begin{entry}{英勇}{8,9}{⾋、⼒}
  \begin{phonetics}{英勇}{ying1yong3}[][HSK 4]
    \definition{adj.}{heroico; valente; bravo; corajoso; extraordinariamente corajoso}
  \end{phonetics}
\end{entry}

\begin{entry}{英语}{8,9}{⾋、⾔}
  \begin{phonetics}{英语}{ying1 yu3}[][HSK 2]
    \definition{s.}{inglês, língua inglesa}
  \end{phonetics}
\end{entry}

\begin{entry}{英雄}{8,12}{⾋、⾫}
  \begin{phonetics}{英雄}{ying1xiong2}
    \definition[个]{s.}{herói}
  \end{phonetics}
\end{entry}

\begin{entry}{苹}{8}{⾋}
  \begin{phonetics}{苹}{ping2}
    \definition[个]{s.}{uma espécie de artemísia | maçã | lentilha-d'água}
  \end{phonetics}
\end{entry}

\begin{entry}{苹果}{8,8}{⾋、⽊}
  \begin{phonetics}{苹果}{ping2guo3}[][HSK 3]
    \definition[个,斤,筐,箱,棵,种]{s.}{maçã}
  \end{phonetics}
\end{entry}

\begin{entry}{茄}{8}{⾋}
  \begin{phonetics}{茄}{jia1}
    \definition{s.}{caracter fonético usado em empréstimos linguísticos para o som "jia", embora 夹 seja mais comum}
  \seealsoref{夹}{jia1}
  \end{phonetics}
  \begin{phonetics}{茄}{qie2}
    \definition[只]{s.}{berinjela}
  \end{phonetics}
\end{entry}

\begin{entry}{茄子}{8,3}{⾋、⼦}
  \begin{phonetics}{茄子}{qie2zi5}
    \definition{s.}{berinjela chinesa | ``xis'' fonético (ao ser fotografado), equivale ao ``diga xis''}
  \end{phonetics}
\end{entry}

\begin{entry}{茅}{8}{⾋}
  \begin{phonetics}{茅}{mao2}
    \definition*{s.}{sobrenome Mao}
    \definition[座]{s.}{capim-cogon | planta semelhante ao capim-cogon (como palha)}
  \end{phonetics}
\end{entry}

\begin{entry}{茅厕}{8,8}{⾋、⼚}
  \begin{phonetics}{茅厕}{mao2ce4}
    \definition{s.}{(dialeto) latrina}
  \end{phonetics}
\end{entry}

\begin{entry}{虎}{8}{⾌}
  \begin{phonetics}{虎}{hu3}[][HSK 5]
    \definition*{s.}{sobrenome Hu}
    \definition{adj.}{corajoso; bravo; valente; vigoroso}
    \definition[只]{s.}{tigre}
    \definition{v.}{blefar; o mesmo que 唬 | parecer feroz; mostrar a aparência feroz de alguém}
  \seealsoref{唬}{hu3}
  \seealsoref{老虎}{lao3hu3}
  \end{phonetics}
\end{entry}

\begin{entry}{虎口}{8,3}{⾌、⼝}
  \begin{phonetics}{虎口}{hu3kou3}
    \definition{s.}{lugar perigoso | cova do tigre}
  \end{phonetics}
\end{entry}

\begin{entry}{虎虎}{8,8}{⾌、⾌}
  \begin{phonetics}{虎虎}{hu3hu3}
    \definition{adj.}{formidável | forte | vigoroso}
  \end{phonetics}
\end{entry}

\begin{entry}{虎鼬}{8,18}{⾌、⿏}
  \begin{phonetics}{虎鼬}{hu3you4}
    \definition{s.}{doninha}
  \end{phonetics}
\end{entry}

\begin{entry}{表}{8}{⾐}
  \begin{phonetics}{表}{biao3}[][HSK 2]
    \definition*{s.}{sobrenome Biao}
    \definition{s.}{exterior; superfície; externo | a relação entre os filhos ou netos de um irmão e uma irmã ou de irmãs | modelo; exemplo; padrão | memorial a um imperador; um tipo de petição da antiguidade, frequentemente usado para expressar intenções; mais tarde, também usado para expressar opiniões sobre eventos importantes | formulário; lista; gráfico; tabela | medidor; instrumento para medir uma determinada quantidade | relógio; um dispositivo para medir o tempo, menor que um relógio, que geralmente pode ser carregado no bolso | medidor de luz solar; antiga vara de madeira para medir o tempo através da sombra do sol | coluna usada antigamente para marcação}
    \definition{v.}{mostrar; expressar; expressar ideias, pensamentos, sentimentos, etc. | administrar medicamentos para aliviar o resfriado; na medicina tradicional chinesa refere-se ao uso de medicamentos para dissipar o frio e o vento que afetam o corpo}
  \end{phonetics}
\end{entry}

\begin{entry}{表白}{8,5}{⾐、⽩}
  \begin{phonetics}{表白}{biao3bai2}
    \definition{s.}{declaração | confissão}
    \definition{v.}{confessar a si mesmo | expressar | revelar pensamentos ou sentimentos de alguém}
  \end{phonetics}
\end{entry}

\begin{entry}{表示}{8,5}{⾐、⽰}
  \begin{phonetics}{表示}{biao3shi4}[][HSK 2]
    \definition{s.}{expressão; indicação}
    \definition{v.}{mostrar; expressar; indicar | significar | expressar pensamentos e sentimentos através de palavras, ações ou expressões faciais}
  \end{phonetics}
\end{entry}

\begin{entry}{表扬}{8,6}{⾐、⼿}
  \begin{phonetics}{表扬}{biao3yang2}[][HSK 4]
    \definition[次,种,份]{s.}{elogios públicos por boas ações}
    \definition{v.}{elogiar; louvar}
  \end{phonetics}
\end{entry}

\begin{entry}{表扬信}{8,6,9}{⾐、⼿、⼈}
  \begin{phonetics}{表扬信}{biao3yang2 xin4}
    \definition{s.}{carta de elogio | depoimento}
  \end{phonetics}
\end{entry}

\begin{entry}{表达}{8,6}{⾐、⾡}
  \begin{phonetics}{表达}{biao3da2}[][HSK 3]
    \definition{v.}{entregar; expressar; mostrar; manifestar; transmitir; comunicar; refere-se ao processo de transmitir pensamentos, sentimentos ou opiniões pessoais a outras pessoas por meio de linguagem, texto, ações, etc.}
  \end{phonetics}
\end{entry}

\begin{entry}{表明}{8,8}{⾐、⽇}
  \begin{phonetics}{表明}{biao3ming2}[][HSK 3]
    \definition{v.}{indicar; demonstrar; expressar; marcar; expressar claramente; expressar de forma clara}
  \end{phonetics}
\end{entry}

\begin{entry}{表现}{8,8}{⾐、⾒}
  \begin{phonetics}{表现}{biao3xian4}[][HSK 3]
    \definition[个,种,份]{s.}{desempenho; expressão; manifestação; comportamento; as ideias, o estilo, as qualidades, o nível ou as capacidades demonstrados em ação.}
    \definition{v.}{mostrar; expressar; exibir; manifestar; descrever; demonstrar algum tipo de pensamento, espírito, qualidade, sentimento ou habilidade, etc. | exibir-se; demonstrar de forma inadequada e intencional alguma habilidade, ponto forte ou vantagem.}
  \end{phonetics}
\end{entry}

\begin{entry}{表面}{8,9}{⾐、⾯}
  \begin{phonetics}{表面}{biao3mian4}[][HSK 3]
    \definition{s.}{superfície; face; exterior; aparência | aparência; superficialidade | mostrador (placa); mostrador do relógio | aparência; a aparência externa das coisas ou a parte não essencial delas}
  \end{phonetics}
\end{entry}

\begin{entry}{表格}{8,10}{⾐、⽊}
  \begin{phonetics}{表格}{biao3ge2}[][HSK 3]
    \definition[张,份,个]{s.}{tabela; formulário}
  \end{phonetics}
\end{entry}

\begin{entry}{表情}{8,11}{⾐、⼼}
  \begin{phonetics}{表情}{biao3qing2}[][HSK 4]
    \definition[个,种,幅]{s.}{expressão; expressão facial; expressão de pensamentos e sentimentos internos por meio de mudanças faciais ou de gestos}
    \definition{v.}{expressar pensamentos e sentimentos internos por meio de mudanças faciais ou de gestos}
  \end{phonetics}
\end{entry}

\begin{entry}{表演}{8,14}{⾐、⽔}
  \begin{phonetics}{表演}{biao3yan3}[][HSK 3]
    \definition[场]{s.}{performance; exposição; refere-se às atividades expressas pelos atores por meio da linguagem, voz, expressões faciais, instrumentos musicais ou movimentos}
    \definition{v.}{atuar; representar; interpretar | demonstrar; fazer demonstrações | fingir; agir de forma afetada; metáfora para fingir deliberadamente uma determinada atitude para enganar alguém}
  \end{phonetics}
\end{entry}

\begin{entry}{表演艺术}{8,14,4,5}{⾐、⽔、⾋、⽊}
  \begin{phonetics}{表演艺术}{biao3yan3 yi4shu4}
    \definition{s.}{arte performática}
  \end{phonetics}
\end{entry}

\begin{entry}{表演者}{8,14,8}{⾐、⽔、⽼}
  \begin{phonetics}{表演者}{biao3yan3 zhe3}
    \definition{s.}{ator}
  \end{phonetics}
\end{entry}

\begin{entry}{表演特技}{8,14,10,7}{⾐、⽔、⽜、⼿}
  \begin{phonetics}{表演特技}{biao3yan3 te4ji4}
    \definition{s.}{acrobacia | pirueta | façanha}
  \end{phonetics}
\end{entry}

\begin{entry}{表演游戏}{8,14,12,6}{⾐、⽔、⽔、⼽}
  \begin{phonetics}{表演游戏}{biao3yan3 you2xi4}
    \definition{s.}{exibição dramática}
  \end{phonetics}
\end{entry}

\begin{entry}{表演赛}{8,14,14}{⾐、⽔、⾙}
  \begin{phonetics}{表演赛}{biao3yan3sai4}
    \definition{s.}{partida ou jogo de exibição}
  \end{phonetics}
\end{entry}

\begin{entry}{衬}{8}{⾐}
  \begin{phonetics}{衬}{chen4}
    \definition[件,个]{s.}{forro}
    \definition{v.}{forrar; colocar algo embaixo | fornecer um pano de fundo para; destacar; servir como contraste para}
  \end{phonetics}
\end{entry}

\begin{entry}{衬衣}{8,6}{⾐、⾐}
  \begin{phonetics}{衬衣}{chen4 yi1}[][HSK 3]
    \definition[件,个]{s.}{camisa; também se refere a uma peça de roupa usada por baixo do casaco}
  \end{phonetics}
\end{entry}

\begin{entry}{衬衫}{8,8}{⾐、⾐}
  \begin{phonetics}{衬衫}{chen4shan1}[][HSK 3]
    \definition[件,个]{s.}{camisa; blusa; camisa ocidental usada por baixo}
  \end{phonetics}
\end{entry}

\begin{entry}{规}{8}{⾒}
  \begin{phonetics}{规}{gui1}
    \definition*{s.}{sobrenome Gui}
    \definition[个,种]{s.}{bússola | regulamentação; regra | (mecânica) medidor | compasso; ferramenta para desenhar círculos}
    \definition{v.}{admoestar; aconselhar; advertir | planejar; fazer planos}
  \end{phonetics}
\end{entry}

\begin{entry}{规划}{8,6}{⾒、⼑}
  \begin{phonetics}{规划}{gui1hua4}[][HSK 5]
    \definition{s.}{plano; projeto; planejamento; programa; programação; esquematização; plano de desenvolvimento de longo prazo mais abrangente |}
    \definition{v.}{planejar; programar;}
  \end{phonetics}
\end{entry}

\begin{entry}{规则}{8,6}{⾒、⼑}
  \begin{phonetics}{规则}{gui1ze2}[][HSK 4]
    \definition{adj.}{ordenado; regular; descreve a forma, estrutura, arranjo, etc., que se conformam a uma determinada maneira organizada}
    \definition{s.}{regra; regulamento; sistema ou código de conduta prescrito para observância comum | lei; norma}
  \end{phonetics}
\end{entry}

\begin{entry}{规定}{8,8}{⾒、⼧}
  \begin{phonetics}{规定}{gui1ding4}[][HSK 3]
    \definition[个,条,项,款]{s.}{regra; regulamento; estipulação; tomar decisões sobre a forma, o método, a quantidade ou a qualidade de algo}
    \definition{v.}{estipular; prover; prescrever; estabelecer requisitos ou restrições em termos de métodos, qualidade, quantidade, tempo, etc.}
  \end{phonetics}
\end{entry}

\begin{entry}{规律}{8,9}{⾒、⼻}
  \begin{phonetics}{规律}{gui1lv4}[][HSK 4]
    \definition{adj.}{estável; regular; coisas, comportamentos, fenômenos, etc. que ocorrem em um determinado momento}
    \definition{s.}{lei; padrão regular; conexão essencial e recorrente entre as coisas}
  \end{phonetics}
\end{entry}

\begin{entry}{规范}{8,9}{⾒、⾋}
  \begin{phonetics}{规范}{gui1fan4}[][HSK 3]
    \definition{adj.}{regular; normal; padrão; que atende às especificações; em conformidade com as normas}
    \definition{s.}{norma; padrão; diretriz}
    \definition{v.}{regular; padronizar; tornar conforme as normas}
  \end{phonetics}
\end{entry}

\begin{entry}{规模}{8,14}{⾒、⽊}
  \begin{phonetics}{规模}{gui1mo2}[][HSK 4]
    \definition[个]{s.}{escala; escopo; dimensões; padrão, forma ou escopo (de um empreendimento, instituição, projeto, movimento, etc.)}
  \end{phonetics}
\end{entry}

\begin{entry}{视}{8}{⾒}
  \begin{phonetics}{视}{shi4}
    \definition{v.}{olhar para | considerar; olhar para | inspecionar; observar}
  \end{phonetics}
\end{entry}

\begin{entry}{视为}{8,4}{⾒、⼂}
  \begin{phonetics}{视为}{shi4 wei2}[][HSK 5]
    \definition{v.}{considerar; ver como; considerar como; considerar ser; achar que é}
  \end{phonetics}
\end{entry}

\begin{entry}{视角}{8,7}{⾒、⾓}
  \begin{phonetics}{视角}{shi4jiao3}
    \definition{s.}{ângulo do qual se observa um objeto | (figurativo) perspectiva, ponto de vista, quadro de referência | (cinematografia) ângulo da câmera | (percepção visual) ângulo visual (o ângulo que um objeto visto subtende no olho) | (fotografia) ângulo de visão}
  \end{phonetics}
\end{entry}

\begin{entry}{视频}{8,13}{⾒、⾴}
  \begin{phonetics}{视频}{shi4pin2}[][HSK 5]
    \definition[个,段]{s.}{vídeo; videoclipe}
  \end{phonetics}
\end{entry}

\begin{entry}{试}{8}{⾔}
  \begin{phonetics}{试}{shi4}[][HSK 1]
    \definition{s.}{teste; exame; avaliação de conhecimentos ou habilidades através de métodos específicos}
    \definition{v.}{tentar; investigar resultados ou verificar a natureza, não se envolver formalmente (em determinada atividade)}
  \end{phonetics}
\end{entry}

\begin{entry}{试卷}{8,8}{⾔、⼙}
  \begin{phonetics}{试卷}{shi4juan4}[][HSK 4]
    \definition[分,张]{s.}{folha de teste; folha de exame; papel usado para escrever as respostas nos exames}
  \end{phonetics}
\end{entry}

\begin{entry}{试图}{8,8}{⾔、⼞}
  \begin{phonetics}{试图}{shi4tu2}[][HSK 5]
    \definition{v.}{tentar; pretender, fazer o possível para realizar algo}
  \end{phonetics}
\end{entry}

\begin{entry}{试验}{8,10}{⾔、⾺}
  \begin{phonetics}{试验}{shi4yan4}[][HSK 3]
    \definition{v.}{testar; fazer um teste; fazer um experimento; para examinar o efeito ou desempenho de algo, primeiro experimente em um laboratório ou em uma escala menor}
  \end{phonetics}
\end{entry}

\begin{entry}{试题}{8,15}{⾔、⾴}
  \begin{phonetics}{试题}{shi4 ti2}[][HSK 3]
    \definition[道]{s.}{questões de um exame}
  \end{phonetics}
\end{entry}

\begin{entry}{诗}{8}{⾔}
  \begin{phonetics}{诗}{shi1}[][HSK 4]
    \definition*{s.}{sobrenome Shi}
    \definition*{s.}{O Livro das Canções《诗经》}
    \definition{s.}{poesia; verso; poema}
  \seealsoref{诗经}{shi1jing1}
  \end{phonetics}
\end{entry}

\begin{entry}{诗人}{8,2}{⾔、⼈}
  \begin{phonetics}{诗人}{shi1 ren2}[][HSK 4]
    \definition{s.}{poeta; escritor de poesia}
  \end{phonetics}
\end{entry}

\begin{entry}{诗句}{8,5}{⾔、⼝}
  \begin{phonetics}{诗句}{shi1ju4}
    \definition[行]{s.}{verso | versículo}
  \end{phonetics}
\end{entry}

\begin{entry}{诗词}{8,7}{⾔、⾔}
  \begin{phonetics}{诗词}{shi1ci2}
    \definition{s.}{verso}
  \end{phonetics}
\end{entry}

\begin{entry}{诗经}{8,8}{⾔、⽷}
  \begin{phonetics}{诗经}{shi1jing1}
    \definition*{s.}{Shijing, o Livro das Canções, antiga coleção de poemas chineses e um dos Cinco Clássicos do Confucionismo}
  \end{phonetics}
\end{entry}

\begin{entry}{诗意}{8,13}{⾔、⼼}
  \begin{phonetics}{诗意}{shi1yi4}
    \definition{adj.}{poético}
    \definition{s.}{poesia}
  \end{phonetics}
\end{entry}

\begin{entry}{诗歌}{8,14}{⾔、⽋}
  \begin{phonetics}{诗歌}{shi1 ge1}[][HSK 5]
    \definition[本,首,段]{s.}{poesia; poemas e canções; refere-se a todos os tipos de poesia}
  \end{phonetics}
\end{entry}

\begin{entry}{诚}{8}{⾔}
  \begin{phonetics}{诚}{cheng2}
    \definition{adj.}{sincero; honesto; verdadeiro}
    \definition{adv.}{na verdade; realmente; de fato}
    \definition{s.}{sinceridade; genuinidade; seriedade}
  \end{phonetics}
\end{entry}

\begin{entry}{诚实}{8,8}{⾔、⼧}
  \begin{phonetics}{诚实}{cheng2shi2}[][HSK 4]
    \definition{adj.}{honesto; sincero e honesto, não hipócrita}
  \end{phonetics}
\end{entry}

\begin{entry}{诚实地}{8,8,6}{⾔、⼧、⼟}
  \begin{phonetics}{诚实地}{cheng2shi2 di4}
    \definition{adv.}{honestamente}
  \end{phonetics}
\end{entry}

\begin{entry}{诚信}{8,9}{⾔、⼈}
  \begin{phonetics}{诚信}{cheng2 xin4}[][HSK 4]
    \definition{adj.}{honesto e confiável}
    \definition[种]{s.}{fé; honestidade; padrão e princípio de comportamento: não contar mentiras, prometer aos outros o que eles podem fazer e ter a confiança dos outros}
  \end{phonetics}
\end{entry}

\begin{entry}{话}{8}{⾔}
  \begin{phonetics}{话}{hua4}[][HSK 1]
    \definition[句,段,番,种]{s.}{palavra; conversa; a voz que expressa os pensamentos quando falada, ou os caracteres que registram essa voz}
    \definition{v.}{falar sobre; falar a respeito}
  \end{phonetics}
\end{entry}

\begin{entry}{话剧}{8,10}{⾔、⼑}
  \begin{phonetics}{话剧}{hua4 ju4}[][HSK 3]
    \definition[场,幕,部,出,台]{s.}{drama moderno; peça de teatro; peça teatral representada através de diálogos e ações}
  \end{phonetics}
\end{entry}

\begin{entry}{话题}{8,15}{⾔、⾴}
  \begin{phonetics}{话题}{hua4ti2}[][HSK 3]
    \definition[个,种,项]{s.}{assunto de uma palestra; tópico de uma conversa; o foco da conversa}
  \end{phonetics}
\end{entry}

\begin{entry}{诟}{8}{⾔}
  \begin{phonetics}{诟}{gou4}
    \definition*{s.}{sobrenome Gou}
    \definition{s.}{vergonha; humilhação}
    \definition{v.}{insultar; xingar; falar de forma abusiva}
  \end{phonetics}
\end{entry}

\begin{entry}{诟骂}{8,9}{⾔、⾺}
  \begin{phonetics}{诟骂}{gou4ma4}
    \definition{v.}{abusar verbalmente | insultar | criticar}
  \end{phonetics}
\end{entry}

\begin{entry}{询问}{8,6}{⾔、⾨}
  \begin{phonetics}{询问}{xun2wen4}[][HSK 5]
    \definition{v.}{indagar; perguntar sobre; pedir conselho}
  \end{phonetics}
\end{entry}

\begin{entry}{该}{8}{⾔}
  \begin{phonetics}{该}{gai1}[][HSK 2]
    \definition{adj.}{completo; integral; abrangente; inclusivo; o mesmo que 赅}
    \definition{pron.}{isto; aquilo; o referido; o acima mencionado; indica a pessoa ou coisa mencionada acima, equivalente a 此, 这个, etc.}
    \definition{v.}{deveria ser; deveria ser assim | caber a alguém; ser a vez (ou dever) de alguém fazer algo | merecer; servir a alguém de direito; indica que algo deve ser feito | dever | deve; provavelmente irá; muito provavelmente; pode ser razoavelmente ou naturalmente esperado que; expressa uma conclusão lógica ou provável com base na razão ou na experiência}
    \definition{v.aux.}{usado em frases exclamativas, tem a função de reforçar o tom}
  \seealsoref{此}{ci3}
  \seealsoref{赅}{gai1}
  \seealsoref{这个}{zhe4ge5}
  \end{phonetics}
\end{entry}

\begin{entry}{详细}{8,8}{⾔、⽷}
  \begin{phonetics}{详细}{xiang2xi4}[][HSK 5]
    \definition{adj.}{explícito; detalhado; minucioso; circunstancial; meticuloso}
  \end{phonetics}
\end{entry}

\begin{entry}{责任}{8,6}{⾙、⼈}
  \begin{phonetics}{责任}{ze2ren4}[][HSK 3]
    \definition[个,种,份]{s.}{dever; responsabilidade; de acordo com a profissão, cargo, identidade, etc., as coisas que você deve fazer ou as tarefas que deve assumir | culpa; responsabilidade por uma falha ou erro; não ter feito o que era sua obrigação e, portanto, ser responsável pela falha}
  \end{phonetics}
\end{entry}

\begin{entry}{责怪}{8,8}{⾙、⼼}
  \begin{phonetics}{责怪}{ze2guai4}
    \definition{v.}{repreender | censurar}
  \end{phonetics}
\end{entry}

\begin{entry}{败}{8}{⾒}
  \begin{phonetics}{败}{bai4}[][HSK 4]
    \definition{adj.}{ruim; deteriorado; murcho; dilapidado; decadente}
    \definition{v.}{ser derrotado; perder (oposto a 胜) | derrotar; bater | falha (oposto a 成) | estragar; arruinar | decair; murchar | quebrar; neutralizar; dissipar}
  \seealsoref{成}{cheng2}
  \seealsoref{胜}{sheng4}
  \end{phonetics}
\end{entry}

\begin{entry}{货}{8}{⾙}
  \begin{phonetics}{货}{huo4}[][HSK 4]
    \definition{s.}{dinheiro; moeda | bens; mercadorias; \emph{commodity} | refere-se a uma pessoa com um certo mau caráter (usado como um insulto) | riqueza; fortuna; um termo geral para dinheiro, joias, tecidos, etc.}
    \definition{v.}{vender}
  \end{phonetics}
\end{entry}

\begin{entry}{货车}{8,4}{⾙、⾞}
  \begin{phonetics}{货车}{huo4che1}
    \definition{s.}{caminhão | van | vagão de carga}
  \end{phonetics}
\end{entry}

\begin{entry}{质量}{8,12}{⾙、⾥}
  \begin{phonetics}{质量}{zhi4liang4}[][HSK 4]
    \definition{s.}{qualidade; o quão bom ou ruim é o produto ou o trabalho}
  \end{phonetics}
\end{entry}

\begin{entry}{贪}{8}{⾙}
  \begin{phonetics}{贪}{tan1}
    \definition{adj.}{corrupto; venal | ganancioso; avarento; ambicioso}
    \definition{v.}{apropriar-se indevidamente; praticar corrupção; ser corrupto | ter um desejo insaciável por; ter um desejo voraz por | cobiçar; ansiar por; ser ganancioso por}
  \end{phonetics}
\end{entry}

\begin{entry}{贪婪}{8,11}{⾙、⼥}
  \begin{phonetics}{贪婪}{tan1lan2}
    \definition{adj.}{avaro | ambicioso | voraz | insaciável}
  \end{phonetics}
\end{entry}

\begin{entry}{贫}{8}{⾙}
  \begin{phonetics}{贫}{pin2}
    \definition{adj.}{pobre; empobrecido | inadequado; deficiente; insuficiente | tagarela; loquaz; falante; chato e irritante}
  \end{phonetics}
\end{entry}

\begin{entry}{贫民窟}{8,5,13}{⾙、⽒、⽳}
  \begin{phonetics}{贫民窟}{pin2min2ku1}
    \definition{s.}{favela}
  \end{phonetics}
\end{entry}

\begin{entry}{购}{8}{⾙}
  \begin{phonetics}{购}{gou4}
    \definition{v.}{comprar}
  \end{phonetics}
\end{entry}

\begin{entry}{购买}{8,6}{⾙、⼄}
  \begin{phonetics}{购买}{gou4 mai3}[][HSK 4]
    \definition{v.}{comprar; adquirir; usar dinheiro para obter itens}
  \end{phonetics}
\end{entry}

\begin{entry}{购物}{8,8}{⾙、⽜}
  \begin{phonetics}{购物}{gou4wu4}[][HSK 4]
    \definition{s.}{compras; itens comprados; \emph{shopping}}
  \end{phonetics}
\end{entry}

\begin{entry}{转}{8}{⾞}
  \begin{phonetics}{转}{zhuai3}
  \end{phonetics}
  \begin{phonetics}{转}{zhuan3}
    \definition{v.}{mudar; deslocar; transferir; virar; mudar de direção, posição, situação, circunstâncias, etc. | transmitir; transferir; passar adiante}
  \end{phonetics}
  \begin{phonetics}{转}{zhuan4}[][HSK 3]
    \definition{clas.}{usado para rotações (por minuto, por segundo, etc.): RPM}
    \definition{v.}{girar; rodar; revolver; movimento em torno de um centro | passear; dar uma volta}
  \end{phonetics}
\end{entry}

\begin{entry}{转化}{8,4}{⾞、⼔}
  \begin{phonetics}{转化}{zhuan3 hua4}[][HSK 5]
    \definition{v.}{mudar; transformar | inverter; converter}
  \end{phonetics}
\end{entry}

\begin{entry}{转让}{8,5}{⾞、⾔}
  \begin{phonetics}{转让}{zhuan3rang4}[][HSK 5]
    \definition{v.}{ceder; fazer a entrega; transferir a posse de; ceder seus bens ou direitos a outra pessoa}
  \end{phonetics}
\end{entry}

\begin{entry}{转产}{8,6}{⾞、⼇}
  \begin{phonetics}{转产}{zhuan3chan3}
    \definition{v.}{mudar a produção | mudar para novos produtos}
  \end{phonetics}
\end{entry}

\begin{entry}{转动}{8,6}{⾞、⼒}
  \begin{phonetics}{转动}{zhuan3 dong4}[][HSK 4]
    \definition{v.}{girar; rodar; dar voltas; torcer | dar a volta em algo}
  \end{phonetics}
  \begin{phonetics}{转动}{zhuan4 dong4}[][HSK 4]
    \definition{v.}{girar; correr; rolar; revolver; rotacionar; torcer}
  \end{phonetics}
\end{entry}

\begin{entry}{转向}{8,6}{⾞、⼝}
  \begin{phonetics}{转向}{zhuan3 xiang4}[][HSK 5]
    \definition{v.}{desviar; desviar-se; mudar a direção do avanço | mudar a posição política de alguém | mudar de direção; virar-se para (a outra parte)}
  \end{phonetics}
  \begin{phonetics}{转向}{zhuan4 xiang4}
    \definition{v.+compl.}{perder-se; perder o rumo; não consiguir distinguir a direção; estar perdido}
  \end{phonetics}
\end{entry}

\begin{entry}{转告}{8,7}{⾞、⼝}
  \begin{phonetics}{转告}{zhuan3gao4}[][HSK 4]
    \definition{v.}{passar adiante; comunicar; transmitir; ser instruído a dizer a outra parte o que uma pessoa diz, o que está acontecendo, etc.}
  \end{phonetics}
\end{entry}

\begin{entry}{转身}{8,7}{⾞、⾝}
  \begin{phonetics}{转身}{zhuan3 shen1}[][HSK 4]
    \definition{adv.}{em um instante; em um piscar de olhos}
    \definition{v.}{dar a volta; dar meia-volta; dar a volta por cima | virar; girar; refere-se a uma mudança de direção, localização, natureza, etc.}
  \end{phonetics}
\end{entry}

\begin{entry}{转变}{8,8}{⾞、⼜}
  \begin{phonetics}{转变}{zhuan3bian4}[][HSK 3]
    \definition{v.}{mudar; converter; transformar}
  \end{phonetics}
\end{entry}

\begin{entry}{转念}{8,8}{⾞、⼼}
  \begin{phonetics}{转念}{zhuan3nian4}
    \definition{v.}{ter dúvidas sobre algo | pensar melhor}
  \end{phonetics}
\end{entry}

\begin{entry}{转账}{8,8}{⾞、⾙}
  \begin{phonetics}{转账}{zhuan3zhang4}
    \definition{v.+compl.}{transferir entre contas | trazer à frente | incluir uma soma de dinheiro do balanço anterior no seguinte}
  \end{phonetics}
\end{entry}

\begin{entry}{转弯}{8,9}{⾞、⼸}
  \begin{phonetics}{转弯}{zhuan4 wan1}[][HSK 4]
    \definition{v.}{rodar; desviar; virar uma esquina; fazer uma curva; fazer uma curva de 180º}
  \end{phonetics}
\end{entry}

\begin{entry}{转换}{8,10}{⾞、⼿}
  \begin{phonetics}{转换}{zhuan3 huan4}[][HSK 5]
    \definition{v.}{mudar; trocar; converter; transformar; alterar}
  \end{phonetics}
\end{entry}

\begin{entry}{转递}{8,10}{⾞、⾡}
  \begin{phonetics}{转递}{zhuan3di4}
    \definition{v.}{passar | retransmitir}
  \end{phonetics}
\end{entry}

\begin{entry}{转悠}{8,11}{⾞、⼼}
  \begin{phonetics}{转悠}{zhuan4you5}
    \definition{v.}{aparecer repetidamente | rolar | passear por aí}
  \end{phonetics}
\end{entry}

\begin{entry}{转移}{8,11}{⾞、⽲}
  \begin{phonetics}{转移}{zhuan3yi2}[][HSK 4]
    \definition{v.}{deslocar; desviar; transferir; redirecionar; reposicionar; reorientar | mudar; transformar}
  \end{phonetics}
\end{entry}

\begin{entry}{转游}{8,12}{⾞、⽔}
  \begin{phonetics}{转游}{zhuan4you5}
    \variantof{转悠}
  \end{phonetics}
\end{entry}

\begin{entry}{轮}{8}{⾞}
  \begin{phonetics}{轮}{lun2}[][HSK 4]
    \definition{clas.}{para sol vermelho, lua brilhante, etc. | para rodadas | doze anos de idade (os doze ramos terrestres são usados para lembrar o gênero humano e cada doze anos de idade é um ciclo)}
    \definition{s.}{roda | anel; disco; objeto semelhante a uma roda | navio a vapor; barco a vapor}
    \definition{v.}{revezar; substituir um ao outro em sequência (para fazer algo)}
  \end{phonetics}
\end{entry}

\begin{entry}{轮子}{8,3}{⾞、⼦}
  \begin{phonetics}{轮子}{lun2 zi5}[][HSK 4]
    \definition[个]{s.}{roda; peças circulares de veículos ou máquinas com capacidade de rotação}
  \end{phonetics}
\end{entry}

\begin{entry}{轮回}{8,6}{⾞、⼞}
  \begin{phonetics}{轮回}{lun2hui2}
    \definition[个]{s.}{reencarnação (Budismo) | ciclo}
    \definition{v.}{reencarnar}
  \end{phonetics}
\end{entry}

\begin{entry}{轮船}{8,11}{⾞、⾈}
  \begin{phonetics}{轮船}{lun2chuan2}[][HSK 4]
    \definition[艘]{s.}{navio}
  \end{phonetics}
\end{entry}

\begin{entry}{轮椅}{8,12}{⾞、⽊}
  \begin{phonetics}{轮椅}{lun2 yi3}[][HSK 4]
    \definition{s.}{cadeira de rodas; dispositivo de assento especialmente projetado com rodas para pessoas com dificuldade de locomoção, que pode ser acionado por um disco de roda ou manivela operados manualmente}
  \end{phonetics}
\end{entry}

\begin{entry}{软}{8}{⾞}
  \begin{phonetics}{软}{ruan3}[][HSK 5]
    \definition*{s.}{sobrenome Ruan}
    \definition{adj.}{macio; flexível; maleável; maleável (oposto de 硬) | suave; brando; delicado | fraco; débil | de baixa qualidade, capacidade, etc. | facilmente movido (ou influenciado) | de maneira suave (ou gentil) | indulgente; tolerante | maleável; flexível | fácil de se emocionar ou abalar}
  \seealsoref{硬}{ying4}
  \end{phonetics}
\end{entry}

\begin{entry}{软件}{8,6}{⾞、⼈}
  \begin{phonetics}{软件}{ruan3jian4}[][HSK 5]
    \definition[款,个]{s.}{(computador) \emph{software}; programas de computador, procedimentos, regras e quaisquer arquivos, documentos e dados relacionados à operação do sistema de computador}
  \end{phonetics}
\end{entry}

\begin{entry}{轰}{8}{⾞}
  \begin{phonetics}{轰}{hong1}
    \definition{interj.}{(onomatopéia) Bum!; estrondo; refere-se aos ruídos altos feitos por trovões, fogo de artilharia, etc.}
    \definition{v.}{retumbar; bombardear; explodir | espantar; expulsar}
  \end{phonetics}
\end{entry}

\begin{entry}{轰鸣}{8,8}{⾞、⿃}
  \begin{phonetics}{轰鸣}{hong1ming2}
    \definition{s.}{bum (som de explosão) | estrondo}
  \end{phonetics}
\end{entry}

\begin{entry}{轰炸机}{8,9,6}{⾞、⽕、⽊}
  \begin{phonetics}{轰炸机}{hong1zha4ji1}
    \definition{s.}{bombardeiro (aeronave)}
  \end{phonetics}
\end{entry}

\begin{entry}{迫}{8}{⾡}
  \begin{phonetics}{迫}{po4}
    \definition{adj.}{urgente; premente}
    \definition{s.}{morteiro; artilharia}
    \definition{v.}{compelir; forçar; pressionar | aproximar-se; ir em direção a (ou perto de)}
  \end{phonetics}
\end{entry}

\begin{entry}{迫切}{8,4}{⾡、⼑}
  \begin{phonetics}{迫切}{po4qie4}[][HSK 4]
    \definition{adj.}{urgente; premente; muito ansiosamente, a ponto de ser difícil esperar}
  \end{phonetics}
\end{entry}

\begin{entry}{郁郁葱葱}{8,8,12,12}{⾢、⾢、⾋、⾋}
  \begin{phonetics}{郁郁葱葱}{yu4yu4cong1cong1}
    \definition{expr.}{verdejante e exuberante}
  \end{phonetics}
\end{entry}

\begin{entry}{郊}{8}{⾢}
  \begin{phonetics}{郊}{jiao1}
    \definition*{s.}{sobrenome Jiao}
    \definition{s.}{subúrbios; periferias; áreas ao redor da cidade}
  \end{phonetics}
\end{entry}

\begin{entry}{郊区}{8,4}{⾢、⼖}
  \begin{phonetics}{郊区}{jiao1qu1}[][HSK 5]
    \definition[个,片,块]{s.}{subúrbios; arredores; periferia; área ao redor da cidade que está administrativamente sob a jurisdição da cidade}
  \end{phonetics}
\end{entry}

\begin{entry}{采}{8}{⾤}
  \begin{phonetics}{采}{cai3}
    \definition*{s.}{sobrenome Cai}
    \definition{s.}{espírito; tez; cor e expressão facial | cores}
    \definition{v.}{escolher; arrancar; reunir; colher (flores, folhas, frutas) | minerar; extrair | reunir; coletar | adotar; pegar; selecionar}
  \end{phonetics}
  \begin{phonetics}{采}{cai4}
    \definition{s.}{atribuição a um nobre feudal; a terra (incluindo os escravos que cultivavam a terra) concedida pelos antigos príncipes aos nobres; também chamada de feudo}
  \end{phonetics}
\end{entry}

\begin{entry}{采用}{8,5}{⾤、⽤}
  \begin{phonetics}{采用}{cai3 yong4}[][HSK 3]
    \definition{v.}{selecionar e usar; adotar; considerar adequado e utilizar}
  \end{phonetics}
\end{entry}

\begin{entry}{采访}{8,6}{⾤、⾔}
  \begin{phonetics}{采访}{cai3fang3}[][HSK 4]
    \definition{s.}{cobertura; entrevista; coleta de notícias; entrevistas, pesquisas, gravações de áudio e vídeo, etc., com o objetivo de coletar os materiais necessários}
    \definition{v.}{cobrir; entrevistar; reunir novas informações}
  \end{phonetics}
\end{entry}

\begin{entry}{采取}{8,8}{⾤、⼜}
  \begin{phonetics}{采取}{cai3qu3}[][HSK 3]
    \definition{v.}{adotar; escolha da implementação (diretrizes, políticas, métodos, ações, etc.) | reunir; coletar; tomar; assumir}
  \end{phonetics}
\end{entry}

\begin{entry}{采购}{8,8}{⾤、⾙}
  \begin{phonetics}{采购}{cai3gou4}[][HSK 5]
    \definition{s.}{comprador; responsável pelas compras}
    \definition{v.}{adquirir; comprar; fazer compras para uma organização; fazer compras para uma empresa}
  \end{phonetics}
\end{entry}

\begin{entry}{金}{8}{⾦}[Kangxi 167]
  \begin{phonetics}{金}{jin1}[][HSK 3]
    \definition*{s.}{sobrenome Jin}
    \definition*{s.}{a Dinastia Jin (1115-1234)}
    \definition{adj.}{dourado | altamente respeitado; precioso. metáfora de nobreza}
    \definition[锭,块]{s.}{ouro | metal | dinheiro | instrumento antigo de percussão de metal}
  \end{phonetics}
\end{entry}

\begin{entry}{金子}{8,3}{⾦、⼦}
  \begin{phonetics}{金子}{jin1zi5}
    \definition{s.}{ouro; elemento metálico, símbolo Au (aurum) amarelo-avermelhado, macio, dúctil, quimicamente estável é um metal precioso, usado para fabricar dinheiro, ornamentos etc.}
  \end{phonetics}
\end{entry}

\begin{entry}{金刚石}{8,6,5}{⾦、⼑、⽯}
  \begin{phonetics}{金刚石}{jin1gang1shi2}
    \definition{s.}{diamante, também chamado de 钻石}
  \seealsoref{钻石}{zuan4shi2}
  \end{phonetics}
\end{entry}

\begin{entry}{金色}{8,6}{⾦、⾊}
  \begin{phonetics}{金色}{jin1 se4}
    \definition{s.}{cor ouro; dourado}
  \end{phonetics}
\end{entry}

\begin{entry}{金牌}{8,12}{⾦、⽚}
  \begin{phonetics}{金牌}{jin1 pai2}[][HSK 3]
    \definition[枚]{s.}{medalha de ouro; refere-se à medalha conquistada pelo campeão em uma competição esportiva | ficha de ouro; placa de ouro usada como símbolo}
  \end{phonetics}
\end{entry}

\begin{entry}{金融}{8,16}{⾦、⿀}
  \begin{phonetics}{金融}{jin1rong2}
    \definition{s.}{finança}
  \end{phonetics}
\end{entry}

\begin{entry}{钓}{8}{⾦}
  \begin{phonetics}{钓}{diao4}
    \definition{v.}{pescar com anzol e linha | buscar (fama e ganho pessoal) | fisgar; defraudar por meios desleais}
  \end{phonetics}
\end{entry}

\begin{entry}{钓鱼}{8,8}{⾦、⿂}
  \begin{phonetics}{钓鱼}{diao4yu2}
    \definition{v.}{pescar (com linha e anzol) | (figurativo) aprisionar}
  \end{phonetics}
\end{entry}

\begin{entry}{闸门}{8,3}{⾨、⾨}
  \begin{phonetics}{闸门}{zha2men2}
    \definition{s.}{eclusa | comporta}
  \end{phonetics}
\end{entry}

\begin{entry}{闹}{8}{⾾}
  \begin{phonetics}{闹}{nao4}[][HSK 4]
    \definition{adj.}{barulhento}
    \definition{v.}{fazer barulho; provocar problemas | dar vazão (à sua raiva, ressentimento, etc.) | sofrer de; ser incomodado por; ocorrer (um desastre ou coisa ruim) | fazer;  entrar em ação | agitar; perturbar | brincar; fazer bagunça}
  \end{phonetics}
\end{entry}

\begin{entry}{闹钟}{8,9}{⾾、⾦}
  \begin{phonetics}{闹钟}{nao4 zhong1}[][HSK 4]
    \definition[个,台,只]{s.}{despertador; relógios capazes de tocar alarmes em horários predeterminados}
  \end{phonetics}
\end{entry}

\begin{entry}{降}{8}{⾩}
  \begin{phonetics}{降}{jiang4}[][HSK 4]
    \definition*{s.}{sobrenome Jiang}
    \definition{v.}{cair; descer | diminuir; reduzir | nascer}
  \end{phonetics}
\end{entry}

\begin{entry}{降价}{8,6}{⾩、⼈}
  \begin{phonetics}{降价}{jiang4 jia4}[][HSK 4]
    \definition{v.}{ficar mais barato; cortar o preço; reduzir o preço}
  \end{phonetics}
\end{entry}

\begin{entry}{降低}{8,7}{⾩、⼈}
  \begin{phonetics}{降低}{jiang4di1}[][HSK 4]
    \definition{v.}{reduzir; cortar; diminuir; rebaixar; cair; abaixar}
  \end{phonetics}
\end{entry}

\begin{entry}{降温}{8,12}{⾩、⽔}
  \begin{phonetics}{降温}{jiang4 wen1}[][HSK 4]
    \definition{v.}{baixar a temperatura (como em uma oficina);  recusar | cair a temperatura | esfriar; resfriar; metáfora para um declínio no entusiasmo ou uma diminuição no ímpeto de algo}
  \end{phonetics}
\end{entry}

\begin{entry}{降落}{8,12}{⾩、⾋}
  \begin{phonetics}{降落}{jiang4luo4}[][HSK 4]
    \definition{v.}{aterrissar; descer; descer do céu}
  \end{phonetics}
\end{entry}

\begin{entry}{限制}{8,8}{⾩、⼑}
  \begin{phonetics}{限制}{xian4zhi4}[][HSK 4]
    \definition{s.}{limite; restrição; confinamento}
    \definition{v.}{limitar; adstringir; restringir; confinar; fechar em (sobre)}
  \end{phonetics}
\end{entry}

\begin{entry}{隶}{8}{⾪}[Kangxi 171]
  \begin{phonetics}{隶}{li4}
    \definition*{s.}{sobrenome Li}
    \definition{s.}{escravo; uma pessoa em servidão; uma pessoa humilde | um dos estilos antigos da caligrafia chinesa}
    \definition{v.}{estar subordinado a}
  \end{phonetics}
\end{entry}

\begin{entry}{雨}{8}{⾬}[Kangxi 173]
  \begin{phonetics}{雨}{yu3}[][HSK 1]
    \definition*{s.}{sobrenome Yu}
    \definition[场,阵,滴]{s.}{chuva; água que cai das nuvens para o solo}
  \end{phonetics}
  \begin{phonetics}{雨}{yu4}
    \definition{v.}{cair (chuva, neve, etc.) | precipitar | chover | molhar}
  \end{phonetics}
\end{entry}

\begin{entry}{雨水}{8,4}{⾬、⽔}
  \begin{phonetics}{雨水}{yu3 shui3}[][HSK 5]
    \definition{s.}{água da chuva; precipitação; chuva; água proveniente da chuva}
  \end{phonetics}
\end{entry}

\begin{entry}{雨伞}{8,6}{⾬、⼈}
  \begin{phonetics}{雨伞}{yu3san3}
    \definition[把]{s.}{guarda-chuva}
  \end{phonetics}
\end{entry}

\begin{entry}{雨衣}{8,6}{⾬、⾐}
  \begin{phonetics}{雨衣}{yu3yi1}
    \definition[件]{s.}{impermeável}
  \end{phonetics}
\end{entry}

\begin{entry}{雨蚀}{8,9}{⾬、⾷}
  \begin{phonetics}{雨蚀}{yu3shi2}
    \definition{s.}{erosão da chuva}
  \end{phonetics}
\end{entry}

\begin{entry}{雨靴}{8,13}{⾬、⾰}
  \begin{phonetics}{雨靴}{yu3xue1}
    \definition[双]{s.}{botas de chuva}
  \end{phonetics}
\end{entry}

\begin{entry}{青}{8}{⾭}
  \begin{phonetics}{青}{qing1}[][HSK 5]
    \definition*{s.}{sobrenome Qing}
    \definition*{s.}{abreviação de Província de Qinghai}
    \definition{adj.}{azul ou verde | preto | jovens (pessoas)}
    \definition{s.}{grama verde | colheitas jovens (não maduras) | tiras de bambu verde}
  \end{phonetics}
\end{entry}

\begin{entry}{青天}{8,4}{⾭、⼤}
  \begin{phonetics}{青天}{qing1tian1}
    \definition{s.}{céu claro, limpo ou azul}
  \end{phonetics}
\end{entry}

\begin{entry}{青少年}{8,4,6}{⾭、⼩、⼲}
  \begin{phonetics}{青少年}{qing1shao4nian2}[][HSK 2]
    \definition[位,名,个,些]{s.}{adolescentes}
  \end{phonetics}
\end{entry}

\begin{entry}{青玉米}{8,5,6}{⾭、⽟、⽶}
  \begin{phonetics}{青玉米}{qing1yu4mi3}
    \definition{s.}{milho verde}
  \end{phonetics}
\end{entry}

\begin{entry}{青年}{8,6}{⾭、⼲}
  \begin{phonetics}{青年}{qing1 nian2}[][HSK 2]
    \definition[个,位,名,些]{s.}{juventude; jovem; refere-se ao período entre os 15 e os 30 anos de idade.}
  \end{phonetics}
\end{entry}

\begin{entry}{青年节}{8,6,5}{⾭、⼲、⾋}
  \begin{phonetics}{青年节}{qing1nian2jie2}
    \definition*{s.}{Dia da Juventude (4 de maio)}
  \end{phonetics}
\end{entry}

\begin{entry}{青春}{8,9}{⾭、⽇}
  \begin{phonetics}{青春}{qing1chun1}[][HSK 4]
    \definition[个]{s.}{juventude; jovialidade}
  \end{phonetics}
\end{entry}

\begin{entry}{青菜}{8,11}{⾭、⾋}
  \begin{phonetics}{青菜}{qing1cai4}
    \definition{s.}{verduras}
  \end{phonetics}
\end{entry}

\begin{entry}{青铜}{8,11}{⾭、⾦}
  \begin{phonetics}{青铜}{qing1tong2}
    \definition{s.}{bronze (liga de cobre, 銅, e estanho, 锡)}
  \end{phonetics}
\end{entry}

\begin{entry}{青椒}{8,12}{⾭、⽊}
  \begin{phonetics}{青椒}{qing1jiao1}
    \definition{s.}{pimenta verde}
  \end{phonetics}
\end{entry}

\begin{entry}{青蛙}{8,12}{⾭、⾍}
  \begin{phonetics}{青蛙}{qing1wa1}
    \definition{adj.}{(gíria velha) cara feio}
    \definition[只]{s.}{sapo}
  \end{phonetics}
\end{entry}

\begin{entry}{非}{8}{⾮}[Kangxi 175]
  \begin{phonetics}{非}{fei1}[][HSK 4]
    \definition*{s.}{sobrenome Fei}
    \definition*{s.}{África, abreviação de 非洲}
    \definition{adv.}{Em resposta a 不, indica necessidade (deve)}
    \definition{pref.}{indicando negatividade ou exclusão}
    \definition{s.}{engano; erro}
    \definition{v.}{opor-se a; culpar; censurar | não estar em conformidade com; ser contrário a | não ser | ter que; simplesmente precisar (fazer algo)}
  \seealsoref{不}{bu4}
  \seealsoref{非洲}{fei1zhou1}
  \end{phonetics}
\end{entry}

\begin{entry}{非洲}{8,9}{⾮、⽔}
  \begin{phonetics}{非洲}{fei1zhou1}
    \definition*{s.}{África}
  \end{phonetics}
\end{entry}

\begin{entry}{非洲人}{8,9,2}{⾮、⽔、⼈}
  \begin{phonetics}{非洲人}{fei1zhou1ren2}
    \definition{s.}{africano | pessoa ou povo da África}
  \end{phonetics}
\end{entry}

\begin{entry}{非常}{8,11}{⾮、⼱}
  \begin{phonetics}{非常}{fei1chang2}[][HSK 1]
    \definition{adj.}{extraordinário; incomum; especial}
    \definition{adv.}{muito; extremamente; altamente}
  \end{phonetics}
\end{entry}

\begin{entry}{靣}{8}{⼀}
  \begin{phonetics}{靣}{mian4}
    \variantof{面}
  \end{phonetics}
\end{entry}

\begin{entry}{顶}{8}{⾴}
  \begin{phonetics}{顶}{ding3}[][HSK 4]
    \definition{adv.}{muito (linguagem falada); a maioria; extremamente; expressa o grau mais alto, equivalente a 最 e 极}
    \definition{clas.}{usado para coisas que têm um topo}
    \definition{prep.}{até}
    \definition{s.}{coroa da cabeça; parte mais alta do corpo ou objeto | topo; limite superior; ponto mais alto}
    \definition{v.}{carregar na cabeça; carregar em sua cabeça | empurrar (ou apoiar) para cima; empurrar por baixo (ou por trás) | dar cabeçadas; dar uma coronhada | sustentar; apoiar; suportar | resistir; ir contra; enfrentar | rebater; retorquir; responder de volta | cooperar; enfrentar; apoiar; dar suporte | igualar; ser equivalente a | substituir; tomar o lugar de | assumir o controle; transferir ou adquirir o direito de administrar um negócio ou alugar uma casa ou terreno}
  \seealsoref{极}{ji2}
  \seealsoref{最}{zui4}
  \end{phonetics}
\end{entry}

\begin{entry}{饱}{8}{⾷}
  \begin{phonetics}{饱}{bao3}[][HSK 2]
    \definition{adj.}{cheio; comer até ficar satisfeito | cheio; rechonchudo}
    \definition{adv.}{totalmente; completamente; plenamente}
    \definition{v.}{satisfazer}
  \end{phonetics}
\end{entry}

\begin{entry}{驻军}{8,6}{⾺、⼍}
  \begin{phonetics}{驻军}{zhu4jun1}
    \definition{s.}{guarnição}
    \definition{v.}{guarcener ou prover uma tropa}
  \end{phonetics}
\end{entry}

\begin{entry}{驾}{8}{⾺}
  \begin{phonetics}{驾}{jia4}
    \definition*{s.}{sobrenome Jia}
    \definition{s.}{carruagem do imperador; refere-se especificamente ao carro do imperador, referindo-se ao imperador | referindo-se a um veículo, usado como um termo respeitoso para uma pessoa}
    \definition{v.}{atrelar; puxar (uma carroça, etc.) | dirigir (um veículo); pilotar (um avião); velejar (um barco) | montar; cavalgar}
  \end{phonetics}
\end{entry}

\begin{entry}{驾驶}{8,8}{⾺、⾺}
  \begin{phonetics}{驾驶}{jia4shi3}[][HSK 5]
    \definition{v.}{dirigir; pilotar; conduzir; guiar; operar (um carro, navio, avião, trator, etc.) para fazê-lo mover}
  \end{phonetics}
\end{entry}

\begin{entry}{驾照}{8,13}{⾺、⽕}
  \begin{phonetics}{驾照}{jia4 zhao4}[][HSK 5]
    \definition[本,张]{s.}{carteira de motorista}
  \end{phonetics}
\end{entry}

\begin{entry}{鱼}{8}{⿂}[Kangxi 195]
  \begin{phonetics}{鱼}{yu2}[][HSK 2]
    \definition*{s.}{sobrenome Yu}
    \definition[条,种,尾]{s.}{peixe; um vertebrado que vive na água; geralmente possui um corpo achatado lateralmente, fusiforme e com muitas escamas; nada com as nadadeiras e respira com as brânquias; sua temperatura corporal varia de acordo com a temperatura externa; existem muitas espécies, a maioria das quais comestíveis | carne de peixe; peixe (como alimento)}
  \end{phonetics}
\end{entry}

\begin{entry}{鱼片}{8,4}{⿂、⽚}
  \begin{phonetics}{鱼片}{yu2pian4}
    \definition{s.}{fatia de peixe | filé de peixe}
  \end{phonetics}
\end{entry}

\begin{entry}{鱼汛}{8,6}{⿂、⽔}
  \begin{phonetics}{鱼汛}{yu2xun4}
    \variantof{渔汛}
  \end{phonetics}
\end{entry}

\begin{entry}{鱼网}{8,6}{⿂、⽹}
  \begin{phonetics}{鱼网}{yu2wang3}
    \variantof{渔网}
  \end{phonetics}
\end{entry}

\begin{entry}{鱼具}{8,8}{⿂、⼋}
  \begin{phonetics}{鱼具}{yu2ju4}
    \variantof{渔具}
  \end{phonetics}
\end{entry}

\begin{entry}{鱼香}{8,9}{⿂、⾹}
  \begin{phonetics}{鱼香}{yu2xiang1}
    \definition{s.}{um tempero da culinária chinesa que normalmente contém alho, cebolinha, gengibre, açúcar, sal, pimenta, etc. (Embora 鱼香 signifique literalmente ``fragrância de peixe'', não contém frutos do mar.)}
  \end{phonetics}
\end{entry}

\begin{entry}{鱼香肉丝}{8,9,6,5}{⿂、⾹、⾁、⼀}
  \begin{phonetics}{鱼香肉丝}{yu2xiang1rou4si1}
    \definition{s.}{tiras de carne de porco salteadas com molho picante (prato)}
  \seealsoref{鱼香}{yu2xiang1}
  \end{phonetics}
\end{entry}

\begin{entry}{鱼船}{8,11}{⿂、⾈}
  \begin{phonetics}{鱼船}{yu2chuan2}
    \definition{s.}{barco de pesca}
  \seealsoref{渔船}{yu2chuan2}
  \end{phonetics}
\end{entry}

\begin{entry}{鸣}{8}{⿃}
  \begin{phonetics}{鸣}{ming2}
    \definition{v.}{chorar (pássaros, animais e insetos) | fazer um som | dar voz (gratidão, queixas, etc.)}
  \end{phonetics}
\end{entry}

\begin{entry}{齿}{8}{⿒}[Kangxi 211]
  \begin{phonetics}{齿}{chi3}
    \definition[颗]{s.}{dente | uma parte de qualquer coisa semelhante a um dente; parte dentada de um objeto | idade (de uma pessoa); faixa etária}
    \definition{v.}{mencionar; falar de}
  \end{phonetics}
\end{entry}

\begin{entry}{齿儿}{8,2}{⿒、⼉}
  \begin{phonetics}{齿儿}{chi3r5}
    \definition{s.}{dentes}
  \end{phonetics}
\end{entry}

%%%%% EOF %%%%%


%%%
%%% 9画
%%%

\section*{9画}\addcontentsline{toc}{section}{9画}

\begin{entry}{临时}{9,7}[Radicais ⼁、⽇]
  \begin{phonetics}{临时}{lin2shi2}[][HSK 4]
    \definition{adj.}{temporário; provisório; por um breve período}
    \definition{adv.}{no momento em que algo acontece (quando as coisas dão errado)}
  \end{phonetics}
\end{entry}

\begin{entry}{举}{9}[Radical ⼂]
  \begin{phonetics}{举}{ju3}[][HSK 2]
    \definition{v.}{levantar | segurar | iniciar | começar | dar à luz a | eleger | escolher | citar| enumerar}
  \end{phonetics}
\end{entry}

\begin{entry}{举办}{9,4}[Radicais ⼂、⼒]
  \begin{phonetics}{举办}{ju3ban4}[][HSK 3]
    \definition{v.}{segurar; conduzir}
  \end{phonetics}
\end{entry}

\begin{entry}{举手}{9,4}[Radicais ⼂、⼿]
  \begin{phonetics}{举手}{ju3 shou3}[][HSK 2]
    \definition{v.}{levantar (colocar) a mão ou mãos}
  \end{phonetics}
\end{entry}

\begin{entry}{举行}{9,6}[Radicais ⼂、⾏]
  \begin{phonetics}{举行}{ju3xing2}[][HSK 2]
    \definition{v.}{realizar (uma reunião, cerimônia, etc.) | ter lugar}
  \end{phonetics}
\end{entry}

\begin{entry}{亭}{9}[Radical ⼇]
  \begin{phonetics}{亭}{ting2}
    \definition{s.}{pavilhão | cabine | quiosque}
  \end{phonetics}
\end{entry}

\begin{entry}{亮}{9}[Radical ⼇]
  \begin{phonetics}{亮}{liang4}[][HSK 2]
    \definition*{s.}{sobrenome Lian}
    \definition{adj.}{brilhante | alto e claro | retumbante | iluminado | aberto e claro}
    \definition{s.}{luz}
    \definition{v.}{iluminar | brilhar | elevar a voz | ressoar | revelar | mostrar | aparecer | exibir}
  \end{phonetics}
\end{entry}

\begin{entry}{亲}{9}[Radical ⼇]
  \begin{phonetics}{亲}{qin1}[][HSK 3]
    \definition{adj.}{parente próximo; relacionado por sangue; de ​​relação de sangue | querido; próximo; íntimo | em si mesmo; pessoalmente}
    \definition[位]{s.}{pais | parente | casal; casamento | noiva}
    \definition{v.}{beijar | (países, partidos, etc.) a favor de; apoiar; estar perto de}
  \end{phonetics}
  \begin{phonetics}{亲}{qing4}
    \definition{s.}{parentes por afinidade; parentes por casamento}
  \end{phonetics}
\end{entry}

\begin{entry}{亲人}{9,2}[Radicais ⼇、⼈]
  \begin{phonetics}{亲人}{qin1 ren2}[][HSK 3]
    \definition{s.}{um membro da família; os pais, o cônjuge, os filhos, etc. | queridos; entes queridos; aqueles queridos para alguém}
  \end{phonetics}
\end{entry}

\begin{entry}{亲切}{9,4}[Radicais ⼇、⼑]
  \begin{phonetics}{亲切}{qin1qie4}[][HSK 3]
    \definition{adj.}{gentil; cordial | próximo; íntimo}
  \end{phonetics}
\end{entry}

\begin{entry}{亲自}{9,6}[Radicais ⼇、⾃]
  \begin{phonetics}{亲自}{qin1zi4}[][HSK 3]
    \definition{adv.}{pessoalmente; em pessoa; si mesmo}
  \end{phonetics}
\end{entry}

\begin{entry}{亲爱}{9,10}[Radicais ⼇、⽖]
  \begin{phonetics}{亲爱}{qin1'ai4}[][HSK 4]
    \definition{adj.}{querido; amado; termo carinhoso que expressa intimidade e afeto}
  \end{phonetics}
\end{entry}

\begin{entry}{亲密}{9,11}[Radicais ⼇、⼧]
  \begin{phonetics}{亲密}{qin1mi4}[][HSK 4]
    \definition{adj.}{próximo; íntimo; relacionamento afetuoso e próximo}
  \end{phonetics}
\end{entry}

\begin{entry}{侵略}{9,11}[Radicais ⼈、⽥]
  \begin{phonetics}{侵略}{qin1lve4}
    \definition{s.}{invasão}
    \definition{v.}{invadir}
  \end{phonetics}
\end{entry}

\begin{entry}{便宜}{9,8}[Radicais ⼈、⼧]
  \begin{phonetics}{便宜}{pian2yi5}[][HSK 2]
    \definition{adj.}{barato}
    \definition{v.}{deixar alguém levemente de lado}
  \end{phonetics}
\end{entry}

\begin{entry}{促进}{9,7}[Radicais ⼈、⾡]
  \begin{phonetics}{促进}{cu4jin4}[][HSK 4]
    \definition{v.}{impulsionar; promover; avançar; incentivar o desenvolvimento}
  \end{phonetics}
\end{entry}

\begin{entry}{促使}{9,8}[Radicais ⼈、⼈]
  \begin{phonetics}{促使}{cu4shi3}[][HSK 4]
    \definition{v.}{incitar; estimular; impelir; causar; provocar uma mudança em alguém ou em algo}
  \end{phonetics}
\end{entry}

\begin{entry}{促销}{9,12}[Radicais ⼈、⾦]
  \begin{phonetics}{促销}{cu4 xiao1}[][HSK 4]
    \definition{v.}{promover vendas}
  \end{phonetics}
\end{entry}

\begin{entry}{俄}{9}[Radical ⼈]
  \begin{phonetics}{俄}{e2}
    \definition*{s.}{Rússia, abreviação de 俄罗斯}
  \seealsoref{俄罗斯}{e2luo2si1}
  \end{phonetics}
\end{entry}

\begin{entry}{俄罗斯}{9,8,12}[Radicais ⼈、⽹、⽄]
  \begin{phonetics}{俄罗斯}{e2luo2si1}
    \definition*{s.}{Rússia}
  \end{phonetics}
\end{entry}

\begin{entry}{俄罗斯人}{9,8,12,2}[Radicais ⼈、⽹、⽄、⼈]
  \begin{phonetics}{俄罗斯人}{e2luo2si1ren2}
    \definition{s.}{russo | pessoa ou povo da Rússia}
  \end{phonetics}
\end{entry}

\begin{entry}{保}{9}[Radical ⼈]
  \begin{phonetics}{保}{bao3}[][HSK 3]
    \definition*{s.}{sobrenome Bao}
    \definition{s.}{fiador
oficial responsável
sistema administrativo}
    \definition{v.}{defender | proteger |manter | preservar | conservar em boas condições | garantir | assegurar | ficar como fiador de alguém.}
  \end{phonetics}
\end{entry}

\begin{entry}{保存}{9,6}[Radicais ⼈、⼦]
  \begin{phonetics}{保存}{bao3cun2}[][HSK 3]
    \definition{v.}{conservar | preservar | (computação) salvar (um arquivo, etc.)}
  \end{phonetics}
\end{entry}

\begin{entry}{保守}{9,6}[Radicais ⼈、⼧]
  \begin{phonetics}{保守}{bao3shou3}[][HSK 4]
    \definition{adj.}{retrógrado; conservador; pensamentos e conceitos que são retrógrados e não conseguem acompanhar o desenvolvimento da situação}
    \definition{v.}{manter; guardar; evitar perder}
  \end{phonetics}
\end{entry}

\begin{entry}{保安}{9,6}[Radicais ⼈、⼧]
  \begin{phonetics}{保安}{bao3 an1}[][HSK 3]
    \definition{s.}{guarda de segurança}
    \definition{v.}{manter seguro | garantir a segurança}
  \end{phonetics}
\end{entry}

\begin{entry}{保护}{9,7}[Radicais ⼈、⼿]
  \begin{phonetics}{保护}{bao3hu4}[][HSK 3]
    \definition{s.}{proteção | salvaguarda}
    \definition{v.}{proteger | defender | salvaguardar}
  \end{phonetics}
\end{entry}

\begin{entry}{保护区}{9,7,4}[Radicais ⼈、⼿、⼖]
  \begin{phonetics}{保护区}{bao3hu4qu1}
    \definition{s.}{área protegida | área de conservação}
  \end{phonetics}
\end{entry}

\begin{entry}{保护主义}{9,7,5,3}[Radicais ⼈、⼿、⼂、⼂]
  \begin{phonetics}{保护主义}{bao3hu4zhu3yi4}
    \definition{s.}{protecionismo}
  \end{phonetics}
\end{entry}

\begin{entry}{保护色}{9,7,6}[Radicais ⼈、⼿、⾊]
  \begin{phonetics}{保护色}{bao3hu4se4}
    \definition{s.}{camuflagem}
  \end{phonetics}
\end{entry}

\begin{entry}{保护剂}{9,7,8}[Radicais ⼈、⼿、⼑]
  \begin{phonetics}{保护剂}{bao3hu4ji4}
    \definition{s.}{agente protetor}
  \end{phonetics}
\end{entry}

\begin{entry}{保护国}{9,7,8}[Radicais ⼈、⼿、⼞]
  \begin{phonetics}{保护国}{bao3hu4guo2}
    \definition{s.}{protetorado}
  \end{phonetics}
\end{entry}

\begin{entry}{保护性}{9,7,8}[Radicais ⼈、⼿、⼼]
  \begin{phonetics}{保护性}{bao3hu4xing4}
    \definition{s.}{proteção}
  \end{phonetics}
\end{entry}

\begin{entry}{保护物}{9,7,8}[Radicais ⼈、⼿、⽜]
  \begin{phonetics}{保护物}{bao3hu4 wu4}
    \definition{s.}{protetor}
  \end{phonetics}
\end{entry}

\begin{entry}{保护者}{9,7,8}[Radicais ⼈、⼿、⽼]
  \begin{phonetics}{保护者}{bao3hu4zhe3}
    \definition{s.}{protetor | segurador}
  \end{phonetics}
\end{entry}

\begin{entry}{保护神}{9,7,9}[Radicais ⼈、⼿、⽰]
  \begin{phonetics}{保护神}{bao3hu4shen2}
    \definition{s.}{anjo da guarda | santo patrono}
  \end{phonetics}
\end{entry}

\begin{entry}{保证}{9,7}[Radicais ⼈、⾔]
  \begin{phonetics}{保证}{bao3zheng4}[][HSK 3]
    \definition[个]{s.}{garantia}
    \definition{v.}{garantir}
  \end{phonetics}
\end{entry}

\begin{entry}{保持}{9,9}[Radicais ⼈、⼿]
  \begin{phonetics}{保持}{bao3chi2}[][HSK 3]
    \definition{v.}{manter | segurar | reter | preservar}
  \end{phonetics}
\end{entry}

\begin{entry}{保险}{9,9}[Radicais ⼈、⾩]
  \begin{phonetics}{保险}{bao3xian3}[][HSK 3]
    \definition[个]{adj./s.}{seguro}
    \definition{v.}{ter certeza | estar vinculado a}
  \end{phonetics}
\end{entry}

\begin{entry}{保留}{9,10}[Radicais ⼈、⽥]
  \begin{phonetics}{保留}{bao3liu2}[][HSK 3]
    \definition{v.}{reter | continuar a ter | segurar | reservar}
  \end{phonetics}
\end{entry}

\begin{entry}{保密}{9,11}[Radicais ⼈、⼧]
  \begin{phonetics}{保密}{bao3mi4}[][HSK 4]
    \definition{v.}{manter segredo; manter algo em segredo; manter a confidencialidade}
  \end{phonetics}
\end{entry}

\begin{entry}{信}{9}[Radical ⼈]
  \begin{phonetics}{信}{xin4}[][HSK 2,3]
    \definition*{s.}{sobrenome Xin}
    \definition{adj.}{verdade}
    \definition{adv.}{à vontade; ao acaso; sem plano}
    \definition[封,个,张]{s.}{carta; correio
mensagem; palavra; informação
sinal; evidência
confiança; fé
fusível
arsênico}
    \definition{v.}{acreditar; fazer um balanço; dar crédito | professar fé em; acreditar em}
  \end{phonetics}
\end{entry}

\begin{entry}{信心}{9,4}[Radicais ⼈、⼼]
  \begin{phonetics}{信心}{xin4xin1}[][HSK 2]
    \definition[个]{s.}{confiança | fé (em alguém ou algo)}
  \end{phonetics}
\end{entry}

\begin{entry}{信号}{9,5}[Radicais ⼈、⼝]
  \begin{phonetics}{信号}{xin4hao4}[][HSK 2]
    \definition[个]{s.}{sinal | ponte de sinalização}
  \end{phonetics}
\end{entry}

\begin{entry}{信用}{9,5}[Radicais ⼈、⽤]
  \begin{phonetics}{信用}{xin4yong4}
    \definition{s.}{crédito (comércio)}
  \end{phonetics}
\end{entry}

\begin{entry}{信用卡}{9,5,5}[Radicais ⼈、⽤、⼘]
  \begin{phonetics}{信用卡}{xin4yong4ka3}[][HSK 2]
    \definition[些]{s.}{cartão de crédito}
  \end{phonetics}
\end{entry}

\begin{entry}{信任}{9,6}[Radicais ⼈、⼈]
  \begin{phonetics}{信任}{xin4ren4}[][HSK 3]
    \definition[个]{s.}{confiança; certeza; convicção}
    \definition{v.}{confiar; ter confiança em}
  \end{phonetics}
\end{entry}

\begin{entry}{信访}{9,6}[Radicais ⼈、⾔]
  \begin{phonetics}{信访}{xin4fang3}
    \definition{s.}{carta de reclamação | carta de petição}
  \seealsoref{上访}{shang4fang3}
  \end{phonetics}
\end{entry}

\begin{entry}{信经}{9,8}[Radicais ⼈、⽷]
  \begin{phonetics}{信经}{xin4jing1}
    \definition[个]{s.}{crença | credo (seção da missa católica)}
  \end{phonetics}
\end{entry}

\begin{entry}{信封}{9,9}[Radicais ⼈、⼨]
  \begin{phonetics}{信封}{xin4feng1}[][HSK 3]
    \definition[个]{s.}{envelope de carta}
  \end{phonetics}
\end{entry}

\begin{entry}{信息}{9,10}[Radicais ⼈、⼼]
  \begin{phonetics}{信息}{xin4xi1}[][HSK 2]
    \definition[个,条]{s.}{notícias | informação | mensagem}
  \end{phonetics}
\end{entry}

\begin{entry}{俩}{9}[Radical ⼈]
  \begin{phonetics}{俩}{lia3}[][HSK 4]
    \definition{num.}{ambos; dois; contração de ``两个'' | alguns; vários; refere-se a um pequeno número}
  \end{phonetics}
\end{entry}

\begin{entry}{俩钱}{9,10}[Radicais ⼈、⾦]
  \begin{phonetics}{俩钱}{lia3qian2}
    \definition{s.}{uma pequena quantia de dinheiro}
  \end{phonetics}
\end{entry}

\begin{entry}{俭省}{9,9}[Radicais ⼈、⽬]
  \begin{phonetics}{俭省}{jian3sheng3}
    \definition{adj.}{econômico}
  \end{phonetics}
\end{entry}

\begin{entry}{修}{9}[Radical ⼈]
  \begin{phonetics}{修}{xiu1}[][HSK 3]
    \definition*{s.}{sobrenome Xiu}
    \definition{adj.}{comprido; alto e esbelto}
    \definition{s.}{revisionismo}
    \definition{v.}{embelezar; decorar
consertar; reparar; reformar
escrever; compilar
estudar; cultivar
construir; edificar
aparar; podar}
  \end{phonetics}
\end{entry}

\begin{entry}{修改}{9,7}[Radicais ⼈、⽁]
  \begin{phonetics}{修改}{xiu1gai3}[][HSK 3]
    \definition{v.}{revisar; alterar}
  \end{phonetics}
\end{entry}

\begin{entry}{修规}{9,8}[Radicais ⼈、⾒]
  \begin{phonetics}{修规}{xiu1gui1}
    \definition{s.}{plano de construção}
  \end{phonetics}
\end{entry}

\begin{entry}{养}{9}[Radical ⼋]
  \begin{phonetics}{养}{yang3}[][HSK 2]
    \definition{v.}{criar (animais ou filhos), plantar (flores), etc. | dar a luz}
  \end{phonetics}
\end{entry}

\begin{entry}{养分}{9,4}[Radicais ⼋、⼑]
  \begin{phonetics}{养分}{yang3fen4}
    \definition{s.}{nutriente}
  \end{phonetics}
\end{entry}

\begin{entry}{养料}{9,10}[Radicais ⼋、⽃]
  \begin{phonetics}{养料}{yang3liao4}
    \definition{s.}{nutrição}
  \end{phonetics}
\end{entry}

\begin{entry}{冒险}{9,9}[Radicais ⽇、⾩]
  \begin{phonetics}{冒险}{mao4xian3}
    \definition{adj.}{corajoso}
    \definition{s.}{risco | aventura}
    \definition{v.+compl.}{correr risco | arriscar-se | aventurar-se em}
  \end{phonetics}
\end{entry}

\begin{entry}{冠}{9}[Radical ⼍]
  \begin{phonetics}{冠}{guan1}
    \definition{s.}{chapéu | coroa | brasão | boné}
  \end{phonetics}
  \begin{phonetics}{冠}{guan4}
    \definition*{s.}{sobrenome Guan}
    \definition{v.}{colocar um chapéu | ser o primeiro | dublar}
  \end{phonetics}
\end{entry}

\begin{entry}{前}{9}[Radical ⼑]
  \begin{phonetics}{前}{qian2}[][HSK 1]
    \definition{adv.}{frente | em frente de | A.C. (Antes de~Cristo)}
  \seealsoref{公元}{gong1yuan2}
    \example{前293年}[293 a.C.]
  \end{phonetics}
\end{entry}

\begin{entry}{前天}{9,4}[Radicais ⼑、⼤]
  \begin{phonetics}{前天}{qian2tian1}[][HSK 1]
    \definition{adv.}{anteontem}
  \end{phonetics}
\end{entry}

\begin{entry}{前头}{9,5}[Radicais ⼑、⼤]
  \begin{phonetics}{前头}{qian2 tou5}[][HSK 4]
    \definition{s.}{à frente; na frente; adiante}
  \end{phonetics}
\end{entry}

\begin{entry}{前边}{9,5}[Radicais ⼑、⾡]
  \begin{phonetics}{前边}{qian2bian5}[][HSK 1]
    \definition{adv.}{à frente | da frente}
  \end{phonetics}
\end{entry}

\begin{entry}{前后}{9,6}[Radicais ⼑、⼝]
  \begin{phonetics}{前后}{qian2 hou4}[][HSK 3]
    \definition{s.}{em volta; sobre | do início ao fim | frente e verso}
  \end{phonetics}
\end{entry}

\begin{entry}{前年}{9,6}[Radicais ⼑、⼲]
  \begin{phonetics}{前年}{qian2 nian2}[][HSK 2]
    \definition{adv.}{há dois anos}
  \end{phonetics}
\end{entry}

\begin{entry}{前进}{9,7}[Radicais ⼑、⾡]
  \begin{phonetics}{前进}{qian2 jin4}[][HSK 3]
    \definition{v.}{marchar; avançar; para ir em frente; seguir em frente}
  \end{phonetics}
\end{entry}

\begin{entry}{前往}{9,8}[Radicais ⼑、⼻]
  \begin{phonetics}{前往}{qian2 wang3}[][HSK 3]
    \definition{v.}{ir para; prosseguir para; partir para}
  \end{phonetics}
\end{entry}

\begin{entry}{前面}{9,9}[Radicais ⼑、⾯]
  \begin{phonetics}{前面}{qian2mian4}[][HSK 3]
    \definition{s.}{frente | parte anterior; acima}
  \end{phonetics}
\end{entry}

\begin{entry}{前途}{9,10}[Radicais ⼑、⾡]
  \begin{phonetics}{前途}{qian2tu2}[][HSK 4]
    \definition[片,段,种]{s.}{futuro; perspectiva; prospecto; originalmente, refere-se à jornada à frente, mas, metaforicamente, refere-se ao futuro.}
  \end{phonetics}
\end{entry}

\begin{entry}{剑}{9}[Radical ⼑]
  \begin{phonetics}{剑}{jian4}
    \definition{clas.}{para golpes de uma espada}
    \definition[口,把]{s.}{espada de dois gumes}
  \end{phonetics}
\end{entry}

\begin{entry}{剑客}{9,9}[Radicais ⼑、⼧]
  \begin{phonetics}{剑客}{jian4ke4}
    \definition{s.}{espada | esgrimista, espadachim}
  \end{phonetics}
\end{entry}

\begin{entry}{勇士}{9,3}[Radicais ⼒、⼠]
  \begin{phonetics}{勇士}{yong3shi4}
    \definition{s.}{um guerreiro | uma pessoa corajosa}
  \end{phonetics}
\end{entry}

\begin{entry}{勇气}{9,4}[Radicais ⼒、⽓]
  \begin{phonetics}{勇气}{yong3qi4}
    \definition{adj.}{coragem | valor}
  \end{phonetics}
\end{entry}

\begin{entry}{勇敢}{9,11}[Radicais ⼒、⽁]
  \begin{phonetics}{勇敢}{yong3gan3}
    \definition{adj.}{bravo | corajoso}
  \end{phonetics}
\end{entry}

\begin{entry}{南}{9}[Radical ⼗]
  \begin{phonetics}{南}{nan2}[][HSK 1]
    \definition*{s.}{sobrenome Nan}
    \definition{s.}{sul}
  \end{phonetics}
\end{entry}

\begin{entry}{南方}{9,4}[Radicais ⼗、⽅]
  \begin{phonetics}{南方}{nan2 fang1}[][HSK 2]
    \definition{s.}{sul | o Sul | a parte sul do país}
  \end{phonetics}
\end{entry}

\begin{entry}{南边}{9,5}[Radicais ⼗、⾡]
  \begin{phonetics}{南边}{nan2bian5}[][HSK 1]
    \definition{adv.}{sul | lado sul | parte sul | ao sul de}
  \end{phonetics}
\end{entry}

\begin{entry}{南极}{9,7}[Radicais ⼗、⽊]
  \begin{phonetics}{南极}{nan2ji2}
    \definition*{s.}{Antártico | Pólo Sul}
    \definition{s.}{pólo sul magnético}
  \end{phonetics}
\end{entry}

\begin{entry}{南面}{9,9}[Radicais ⼗、⾯]
  \begin{phonetics}{南面}{nan2mian4}
    \definition{s.}{sul | lado sul}
  \end{phonetics}
\end{entry}

\begin{entry}{南部}{9,10}[Radicais ⼗、⾢]
  \begin{phonetics}{南部}{nan2 bu4}[][HSK 3]
    \definition{s.}{parte sul; sul | a parte sul}
  \end{phonetics}
\end{entry}

\begin{entry}{厘米}{9,6}[Radicais ⼚、⽶]
  \begin{phonetics}{厘米}{li2mi3}[][HSK 4]
    \definition{clas.}{centímetro; unidade de comprimento, símbolo cm, 1 metro é igual a 100 centímetros}
  \end{phonetics}
\end{entry}

\begin{entry}{厚}{9}[Radical ⼚]
  \begin{phonetics}{厚}{hou4}[][HSK 4]
    \definition*{s.}{sobrenome Hou}
    \definition{adj.}{grosso; espesso | profundo | bondoso; gentil; magnânimo | grande; generoso | rico ou forte em sabor}
    \definition{s.}{espessura; profundidade}
    \definition{v.}{favorecer; enfatizar}
  \end{phonetics}
\end{entry}

\begin{entry}{咱}{9}[Radical ⼝]
  \begin{phonetics}{咱}{zan2}[][HSK 2]
    \definition{pron.}{eu}
  \end{phonetics}
\end{entry}

\begin{entry}{咱们}{9,5}[Radicais ⼝、⼈]
  \begin{phonetics}{咱们}{zan2men5}[][HSK 2]
    \definition{pron.}{nós (incluindo o orador e a(s) pessoa(s) com quem se fala)}
  \end{phonetics}
\end{entry}

\begin{entry}{咱俩}{9,9}[Radicais ⼝、⼈]
  \begin{phonetics}{咱俩}{zan2lia3}
    \definition{pron.}{nós dois}
  \end{phonetics}
\end{entry}

\begin{entry}{咱家}{9,10}[Radicais ⼝、⼧]
  \begin{phonetics}{咱家}{za2jia1}
    \definition{pron.}{eu (frequentemente usado na literatura vernácula antiga) | me | mim | comigo}
  \end{phonetics}
\end{entry}

\begin{entry}{咳嗽}{9,14}[Radicais ⼝、⼝]
  \begin{phonetics}{咳嗽}{ke2sou5}
    \definition{v.}{ter tosse | tossir}
  \end{phonetics}
\end{entry}

\begin{entry}{咸}{9}[Radical ⼝]
  \begin{phonetics}{咸}{xian2}
    \definition*{s.}{sobrenome Xian}
    \definition{adj.}{salgado}
  \end{phonetics}
\end{entry}

\begin{entry}{咸水}{9,4}[Radicais ⼝、⽔]
  \begin{phonetics}{咸水}{xian2shui3}
    \definition{s.}{salmora | água salgada}
  \end{phonetics}
\end{entry}

\begin{entry}{咸肉}{9,6}[Radicais ⼝、⾁]
  \begin{phonetics}{咸肉}{xian2rou4}
    \definition{s.}{\emph{bacon} | carne curada com sal}
  \end{phonetics}
\end{entry}

\begin{entry}{咸鱼}{9,8}[Radicais ⼝、⿂]
  \begin{phonetics}{咸鱼}{xian2yu2}
    \definition{s.}{peixe salgado}
  \end{phonetics}
\end{entry}

\begin{entry}{咸涩}{9,10}[Radicais ⼝、⽔]
  \begin{phonetics}{咸涩}{xian2se4}
    \definition{s.}{ácido | salgado e amargo}
  \end{phonetics}
\end{entry}

\begin{entry}{咸盐}{9,10}[Radicais ⼝、⽫]
  \begin{phonetics}{咸盐}{xian2yan2}
    \definition{s.}{(coloquial) sal | sal de mesa}
  \end{phonetics}
\end{entry}

\begin{entry}{咸淡}{9,11}[Radicais ⼝、⽔]
  \begin{phonetics}{咸淡}{xian2dan4}
    \definition{s.}{água salobra | grau de salinidade | salgado e sem sal (sabores)}
  \end{phonetics}
\end{entry}

\begin{entry}{咸菜}{9,11}[Radicais ⼝、⾋]
  \begin{phonetics}{咸菜}{xian2cai4}
    \definition{s.}{legumes salgados | \emph{pickles}}
  \end{phonetics}
\end{entry}

\begin{entry}{品质}{9,8}[Radicais ⼝、⾙]
  \begin{phonetics}{品质}{pin3zhi4}[][HSK 4]
    \definition[个,种]{s.}{qualidade; caráter; natureza do pensamento, da compreensão, do caráter, etc., conforme expresso no comportamento, no estilo, etc. | qualidade (de produtos, mercadorias, etc.)}
  \end{phonetics}
\end{entry}

\begin{entry}{品德}{9,15}[Radicais ⼝、⼻]
  \begin{phonetics}{品德}{pin3de2}
    \definition{s.}{caráter moral | moralidade}
  \end{phonetics}
\end{entry}

\begin{entry}{哄}{9}[Radical ⼝]
  \begin{phonetics}{哄}{hong1}
    \definition{s.}{gargalhadas | risadas ruidosas | algazarra | rugido | clamor}
  \end{phonetics}
  \begin{phonetics}{哄}{hong3}
    \definition{v.}{enganar | persuadir | divertir (uma criança)}
  \end{phonetics}
  \begin{phonetics}{哄}{hong4}
    \definition{s.}{tumulto | agitação | perturbação}
  \end{phonetics}
\end{entry}

\begin{entry}{哇塞}{9,13}[Radicais ⼝、⼟]
  \begin{phonetics}{哇塞}{wa1sai1}
    \definition{interj.}{(gíria) Uau!}
  \end{phonetics}
\end{entry}

\begin{entry}{哇噻}{9,16}[Radicais ⼝、⼝]
  \begin{phonetics}{哇噻}{wa1sai1}
    \variantof{哇塞}
  \end{phonetics}
\end{entry}

\begin{entry}{哈马斯}{9,3,12}[Radicais ⼝、⾺、⽄]
  \begin{phonetics}{哈马斯}{ha1ma3si1}
    \definition*{s.}{Hamas (Grupo Palestino)}
  \end{phonetics}
\end{entry}

\begin{entry}{哈哈}{9,9}[Radicais ⼝、⼝]
  \begin{phonetics}{哈哈}{ha1 ha1}[][HSK 3]
    \definition{expr.}{(onomatopéia)  ha ha; o som de uma risada alta}
  \end{phonetics}
\end{entry}

\begin{entry}{响}{9}[Radical ⼝]
  \begin{phonetics}{响}{xiang3}[][HSK 2]
    \definition{adj.}{barulhento}
    \definition[声,阵]{s.}{som | barulho | eco}
    \definition{v.}{fazer um som | soar | tocar}
  \end{phonetics}
\end{entry}

\begin{entry}{哪}{9}[Radical ⼝]
  \begin{phonetics}{哪}{na3}[][HSK 1,4]
    \definition{adv.}{para expressar uma pergunta retórica}
    \definition{pron.}{qual?; o que? | qualquer; ser usado em um sentido geral}
  \end{phonetics}
  \begin{phonetics}{哪}{na5}
    \definition{part.}{usado depois de uma palavra com a terminação -n, é equivalente a ``啊''}
  \seealsoref{啊}{a5}
  \end{phonetics}
  \begin{phonetics}{哪}{nei3}
    \definition{part.}{qual? (interrogativo, seguido de classificador ou numeral-classificador)}
  \end{phonetics}
\end{entry}

\begin{entry}{哪儿}{9,2}[Radicais ⼝、⼉]
  \begin{phonetics}{哪儿}{na3r5}[][HSK 1]
    \definition{adv.}{onde?}
  \end{phonetics}
\end{entry}

\begin{entry}{哪里}{9,7}[Radicais ⼝、⾥]
  \begin{phonetics}{哪里}{na3 li3}[][HSK 1]
    \definition{adv.}{onde?}
  \end{phonetics}
\end{entry}

\begin{entry}{哪些}{9,8}[Radicais ⼝、⼆]
  \begin{phonetics}{哪些}{na3xie1}[][HSK 1]
    \definition{pron.}{quais?}
  \end{phonetics}
\end{entry}

\begin{entry}{哪国人}{9,8,2}[Radicais ⼝、⼞、⼈]
  \begin{phonetics}{哪国人}{na3 guo2ren2}
    \definition{expr.}{de qual país?}
  \end{phonetics}
\end{entry}

\begin{entry}{哪怕}{9,8}[Radicais ⼝、⼼]
  \begin{phonetics}{哪怕}{na3pa4}[][HSK 4]
    \definition{conj.}{mesmo; mesmo se; mesmo que; não importa o quão}
  \end{phonetics}
\end{entry}

\begin{entry}{垫子}{9,3}[Radicais ⼟、⼦]
  \begin{phonetics}{垫子}{dian4zi5}
    \definition{s.}{colchão | esteira | almofada}
  \end{phonetics}
\end{entry}

\begin{entry}{城}{9}[Radical ⼟]
  \begin{phonetics}{城}{cheng2}[][HSK 3]
    \definition*{s.}{sobrenome Cheng}
    \definition[座,道,个]{s.}{muralha da cidade; muro | cidade}
  \end{phonetics}
\end{entry}

\begin{entry}{城市}{9,5}[Radicais ⼟、⼱]
  \begin{phonetics}{城市}{cheng2shi4}[][HSK 3]
    \definition[个,座]{s.}{cidade}
  \end{phonetics}
\end{entry}

\begin{entry}{城度}{9,9}[Radicais ⼟、⼴]
  \begin{phonetics}{城度}{cheng2du4}[][HSK 3]
    \definition*{s.}{Cidade}
  \end{phonetics}
\end{entry}

\begin{entry}{城堡}{9,12}[Radicais ⼟、⼟]
  \begin{phonetics}{城堡}{cheng2bao3}
    \definition*{s.}{castelo | torre (peça de xadrez)}
  \end{phonetics}
\end{entry}

\begin{entry}{复习}{9,3}[Radicais ⼢、⼄]
  \begin{phonetics}{复习}{fu4xi2}[][HSK 2]
    \definition{s.}{revisão}
    \definition{v.}{rever | revisar}
  \end{phonetics}
\end{entry}

\begin{entry}{复印}{9,5}[Radicais ⼢、⼙]
  \begin{phonetics}{复印}{fu4yin4}[][HSK 3]
    \definition{v.}{fotografar; fotocopiar; duplicar}
  \end{phonetics}
\end{entry}

\begin{entry}{复杂}{9,6}[Radicais ⼢、⽊]
  \begin{phonetics}{复杂}{fu4za2}[][HSK 3]
    \definition{adj.}{complexo; complicado}
  \end{phonetics}
\end{entry}

\begin{entry}{复制}{9,8}[Radicais ⼢、⼑]
  \begin{phonetics}{复制}{fu4zhi4}[][HSK 4]
    \definition{v.}{copiar; duplicar; reproduzir; fazer uma cópia de; fazer uma cópia do original ou reproduzi-lo, reimprimi-lo ou copiá-lo em sua forma original (geralmente referindo-se a relíquias culturais ou obras de arte)}
  \end{phonetics}
\end{entry}

\begin{entry}{复刻}{9,8}[Radicais ⼢、⼑]
  \begin{phonetics}{复刻}{fu4ke4}
    \definition{v.}{reimprimir (um trabalho que esteve fora do catálogo) | reeditar (um disco de vinil, um CD, etc.) | replicar | recriar | (empréstimo linguístico) (computação) \emph{fork}}
  \end{phonetics}
\end{entry}

\begin{entry}{复活节}{9,9,5}[Radicais ⼢、⽔、⾋]
  \begin{phonetics}{复活节}{fu4huo2jie2}
    \definition*{s.}{Páscoa}
  \end{phonetics}
\end{entry}

\begin{entry}{奏效}{9,10}[Radicais ⼤、⽁]
  \begin{phonetics}{奏效}{zou4xiao4}
    \definition{v.}{mostrar resultados | ser eficaz}
  \end{phonetics}
\end{entry}

\begin{entry}{奖}{9}[Radical ⼤]
  \begin{phonetics}{奖}{jiang3}[][HSK 4]
    \definition[个,次]{s.}{prêmio; recompensa | elogio; loa}
    \definition{v.}{elogiar; recompensar; recomendar; incentivar}
  \end{phonetics}
\end{entry}

\begin{entry}{奖学金}{9,8,8}[Radicais ⼤、⼦、⾦]
  \begin{phonetics}{奖学金}{jiang3 xue2 jin1}[][HSK 4]
    \definition[笔]{s.}{bolsa de estudos; exposição; prêmios concedidos por escolas, organizações ou indivíduos a alunos com bom desempenho acadêmico}
  \end{phonetics}
\end{entry}

\begin{entry}{奖金}{9,8}[Radicais ⼤、⾦]
  \begin{phonetics}{奖金}{jiang3jin1}[][HSK 4]
    \definition[个,笔]{s.}{bônus; recompensa; prêmio; prêmio em dinheiro; dinheiro de recompensa, dinheiro dado às pessoas para incentivá-las ou elogiá-las por terem se saído bem em alguma coisa}
  \end{phonetics}
\end{entry}

\begin{entry}{姜}{9}[Radical ⼥]
  \begin{phonetics}{姜}{jiang1}
    \definition*{s.}{sobrenome Jiang}
    \definition{s.}{gengibre}
  \end{phonetics}
\end{entry}

\begin{entry}{孩子}{9,3}[Radicais ⼦、⼦]
  \begin{phonetics}{孩子}{hai2zi5}[][HSK 1]
    \definition{s.}{criança | filho}
  \end{phonetics}
\end{entry}

\begin{entry}{客人}{9,2}[Radicais ⼧、⼈]
  \begin{phonetics}{客人}{ke4ren2}[][HSK 2]
    \definition{s.}{visitante | convidado | cliente | passageiro | viajante}
  \end{phonetics}
\end{entry}

\begin{entry}{客厅}{9,4}[Radicais ⼧、⼚]
  \begin{phonetics}{客厅}{ke4ting1}
    \definition[间]{s.}{sala de estar | sala de visitas}
  \end{phonetics}
\end{entry}

\begin{entry}{客气}{9,4}[Radicais ⼧、⽓]
  \begin{phonetics}{客气}{ke4qi5}
    \definition{adj.}{cortês | delicado | modesto | educado}
    \definition{v.}{fazer cerimônia}
  \end{phonetics}
\end{entry}

\begin{entry}{客观}{9,6}[Radicais ⼧、⾒]
  \begin{phonetics}{客观}{ke4guan1}[][HSK 3]
    \definition{adj.}{objetivo; justo e razoável; imparcial}
    \definition{s.}{objetivo}
  \end{phonetics}
\end{entry}

\begin{entry}{宣布}{9,5}[Radicais ⼧、⼱]
  \begin{phonetics}{宣布}{xuan1bu4}[][HSK 3]
    \definition{v.}{declarar; proclamar; pronunciar; anunciar | anunciar oficialmente e publicamente as últimas decisões e situações a todos}
  \end{phonetics}
\end{entry}

\begin{entry}{宣传}{9,6}[Radicais ⼧、⼈]
  \begin{phonetics}{宣传}{xuan1chuan2}[][HSK 3]
    \definition{v.}{propagar; disseminar; conduzir propaganda | explicar às massas para que elas possam acreditar e agir de acordo}
  \end{phonetics}
\end{entry}

\begin{entry}{宣扬}{9,6}[Radicais ⼧、⼿]
  \begin{phonetics}{宣扬}{xuan1yang2}
    \definition{v.}{divulgar | anunciar | espalhar por toda parte}
  \end{phonetics}
\end{entry}

\begin{entry}{室}{9}[Radical ⼧]
  \begin{phonetics}{室}{shi4}[][HSK 3]
    \definition*{s.}{sobrenome Shi | Shi, uma das mansões lunares}
    \definition{s.}{sala; aposento; cômodo |  seção; escritório | esposa}
  \end{phonetics}
\end{entry}

\begin{entry}{宪制}{9,8}[Radicais ⼧、⼑]
  \begin{phonetics}{宪制}{xian4zhi4}
    \definition{adj.}{constitucional}
    \definition{s.}{sistema de governo constitucional}
  \end{phonetics}
\end{entry}

\begin{entry}{宪法法院}{9,8,8,9}[Radicais ⼧、⽔、⽔、⾩]
  \begin{phonetics}{宪法法院}{xian4fa3fa3yuan4}
    \definition{s.}{tribunal constitucional}
  \end{phonetics}
\end{entry}

\begin{entry}{宪政}{9,9}[Radicais ⼧、⽁]
  \begin{phonetics}{宪政}{xian4zheng4}
    \definition{s.}{governo constitucional}
  \end{phonetics}
\end{entry}

\begin{entry}{封}{9}[Radical ⼨]
  \begin{phonetics}{封}{feng1}[][HSK 2]
    \definition*{s.}{sobrenome Feng}
    \definition{clas.}{para objetos selados, especialmente cartas}
    \definition{v.}{conceder um título | conferir | conceder | selar}
  \end{phonetics}
\end{entry}

\begin{entry}{封口}{9,3}[Radicais ⼨、⼝]
  \begin{phonetics}{封口}{feng1kou3}
    \definition{v.}{selar | fechar | curar (uma ferida) | manter os lábios selados}
  \end{phonetics}
\end{entry}

\begin{entry}{封印}{9,5}[Radicais ⼨、⼙]
  \begin{phonetics}{封印}{feng1yin4}
    \definition{s.}{selo (em envelopes)}
  \end{phonetics}
\end{entry}

\begin{entry}{封闭}{9,6}[Radicais ⼨、⾨]
  \begin{phonetics}{封闭}{feng1bi4}[][HSK 4]
    \definition{adj.}{fechado; aqueles que não têm contato com o mundo exterior; aqueles que são muito conservadores (em seu pensamento) e não se comunicam com os outros}
    \definition{v.}{selar; fechar; lacrar; vedar; de modo a impedir a passagem, o uso ou a abertura}
  \end{phonetics}
\end{entry}

\begin{entry}{封冻}{9,7}[Radicais ⼨、⼎]
  \begin{phonetics}{封冻}{feng1dong4}
    \definition{v.}{congelar (água ou terra)}
  \end{phonetics}
\end{entry}

\begin{entry}{封底}{9,8}[Radicais ⼨、⼴]
  \begin{phonetics}{封底}{feng1di3}
    \definition{s.}{contracapa de um livro}
  \end{phonetics}
\end{entry}

\begin{entry}{封建}{9,8}[Radicais ⼨、⼵]
  \begin{phonetics}{封建}{feng1jian4}
    \definition{adj.}{feudal}
    \definition{s.}{feudalismo}
  \end{phonetics}
\end{entry}

\begin{entry}{封面}{9,9}[Radicais ⼨、⾯]
  \begin{phonetics}{封面}{feng1mian4}
    \definition{s.}{capa (de uma publicação) | sobrecapa}
  \end{phonetics}
\end{entry}

\begin{entry}{封斋}{9,10}[Radicais ⼨、⽂]
  \begin{phonetics}{封斋}{feng1zhai1}
    \definition*{s.}{Ramadã (Islã)}
  \end{phonetics}
\end{entry}

\begin{entry}{封盖}{9,11}[Radicais ⼨、⽫]
  \begin{phonetics}{封盖}{feng1gai4}
    \definition{s.}{boné | capa | selo}
    \definition{v.}{cobrir}
  \end{phonetics}
\end{entry}

\begin{entry}{将来}{9,7}[Radicais ⼨、⽊]
  \begin{phonetics}{将来}{jiang1lai2}[][HSK 3]
    \definition[个]{s.}{futuro}
  \end{phonetics}
\end{entry}

\begin{entry}{将近}{9,7}[Radicais ⼨、⾡]
  \begin{phonetics}{将近}{jiang1jin4}[][HSK 3]
    \definition{adv.}{quase}
  \end{phonetics}
\end{entry}

\begin{entry}{将要}{9,9}[Radicais ⼨、⾑]
  \begin{phonetics}{将要}{jiang1yao4}
    \definition{adv.}{vai | deve}
  \end{phonetics}
\end{entry}

\begin{entry}{屋子}{9,3}[Radicais ⼫、⼦]
  \begin{phonetics}{屋子}{wu1zi5}[][HSK 3]
    \definition[间,座,栋]{s.}{casa}
  \end{phonetics}
\end{entry}

\begin{entry}{屌丝}{9,5}[Radicais ⼫、⼀]
  \begin{phonetics}{屌丝}{diao3si1}
    \definition{adj.}{panaca | zé-ninguém | (gíria de \emph{Internet}) \emph{looser}}
  \end{phonetics}
\end{entry}

\begin{entry}{屎}{9}[Radical ⼫]
  \begin{phonetics}{屎}{shi3}
    \definition{s.}{fezes | excrementos | (forma ligada) secreção (do ouvido, olho, etc.)}
  \end{phonetics}
\end{entry}

\begin{entry}{差}{9}[Radical ⼯]
  \begin{phonetics}{差}{cha4}[][HSK 1]
    \definition{adv.}{ligeiramente | comparativamente | um pouco}
    \definition{s.}{differença | dissimilaridade | engano | equívoco}
  \end{phonetics}
\end{entry}

\begin{entry}{差不多}{9,4,6}[Radicais ⼯、⼀、⼣]
  \begin{phonetics}{差不多}{cha4bu5duo1}[][HSK 2]
    \definition{adj.}{mais ou menos}
    \definition{adv.}{quase perto}
  \end{phonetics}
\end{entry}

\begin{entry}{差点儿}{9,9,2}[Radicais ⼯、⽕、⼉]
  \begin{phonetics}{差点儿}{cha4dian3r5}
    \definition{adv.}{por pouco | por um triz | quase}
  \end{phonetics}
\end{entry}

\begin{entry}{帝国}{9,8}[Radicais ⼱、⼞]
  \begin{phonetics}{帝国}{di4guo2}
    \definition{adj.}{imperial}
    \definition{s.}{império}
  \end{phonetics}
\end{entry}

\begin{entry}{带}{9}[Radical ⼱]
  \begin{phonetics}{带}{dai4}[][HSK 2]
    \definition{v.}{levar | trazer}
  \end{phonetics}
\end{entry}

\begin{entry}{带动}{9,6}[Radicais ⼱、⼒]
  \begin{phonetics}{带动}{dai4 dong4}[][HSK 3]
    \definition{v.}{dirigir; ativar; fazer algo funcionar; acionar | liderar; trazer; estimular; motivar; atrair}
  \end{phonetics}
\end{entry}

\begin{entry}{带来}{9,7}[Radicais ⼱、⽊]
  \begin{phonetics}{带来}{dai4 lai2}[][HSK 2]
    \definition{v.}{trazer | (figurativo) provocar, produzir}
  \end{phonetics}
\end{entry}

\begin{entry}{带领}{9,11}[Radicais ⼱、⾴]
  \begin{phonetics}{带领}{dai4ling3}[][HSK 3]
    \definition{v.}{guiar | liderar}
  \end{phonetics}
\end{entry}

\begin{entry}{帮}{9}[Radical ⼱]
  \begin{phonetics}{帮}{bang1}[][HSK 1]
    \definition{clas.}{para alguém (como uma ajuda)}
    \definition{s.}{gangue | grupo | contratado (como trabalhador) | camada externa | festa | sociedade secreta}
    \definition{v.}{ajudar | apoiar}
  \end{phonetics}
\end{entry}

\begin{entry}{帮忙}{9,6}[Radicais ⼱、⼼]
  \begin{phonetics}{帮忙}{bang1 mang2}[][HSK 1]
    \definition{v.+compl.}{ajudar | dar uma mão | estender a mão | fazer um favor}
  \end{phonetics}
\end{entry}

\begin{entry}{帮佣}{9,7}[Radicais ⼱、⼈]
  \begin{phonetics}{帮佣}{bang1yong1}
    \definition{s.}{ajudante doméstico | servo}
  \end{phonetics}
\end{entry}

\begin{entry}{帮助}{9,7}[Radicais ⼱、⼒]
  \begin{phonetics}{帮助}{bang1zhu4}[][HSK 2]
    \definition[种]{s.}{ajuda | assistência}
    \definition{v.}{ajudar | dar assistência}
  \end{phonetics}
\end{entry}

\begin{entry}{帮教}{9,11}[Radicais ⼱、⽁]
  \begin{phonetics}{帮教}{bang1jiao4}
    \definition{v.}{orientar}
  \end{phonetics}
\end{entry}

\begin{entry}{度}{9}[Radical ⼴]
  \begin{phonetics}{度}{du4}[][HSK 2]
    \definition{clas.}{para temperatura, etc. | para eventos e ocorrências}
    \definition{s.}{grau (ângulo, temperatura, etc.) | kilowatt-hora}
  \end{phonetics}
  \begin{phonetics}{度}{duo2}
    \definition{v.}{estimar}
  \end{phonetics}
\end{entry}

\begin{entry}{度过}{9,6}[Radicais ⼴、⾡]
  \begin{phonetics}{度过}{du4guo4}[][HSK 4]
    \definition{s.}{passar o tempo; fazer o tempo desaparecer no trabalho, na vida, no lazer e no descanso}
  \end{phonetics}
\end{entry}

\begin{entry}{待遇}{9,12}[Radicais ⼻、⾡]
  \begin{phonetics}{待遇}{dai4yu4}[][HSK 4]
    \definition[种,项,份]{s.}{tratamento; refere-se a direitos, status social, etc. | salário; ordenado; remuneração}
  \end{phonetics}
\end{entry}

\begin{entry}{很}{9}[Radical ⼻]
  \begin{phonetics}{很}{hen3}[][HSK 1]
    \definition{adv.}{bastante | muito | terrivelmente | advérbio de grau}
  \end{phonetics}
\end{entry}

\begin{entry}{律师}{9,6}[Radicais ⼻、⼱]
  \begin{phonetics}{律师}{lv4shi1}[][HSK 4]
    \definition[名,个,位]{s.}{advogado; procurador; profissionais encarregados pelas partes ou nomeados pelo tribunal para auxiliar as partes no litígio, para comparecer ao tribunal para defesa e para tratar de assuntos jurídicos relacionados, de acordo com a lei}
  \end{phonetics}
\end{entry}

\begin{entry}{怎}{9}[Radical ⼼]
  \begin{phonetics}{怎}{zen3}
    \definition{adv.}{como}
  \end{phonetics}
\end{entry}

\begin{entry}{怎么}{9,3}[Radicais ⼼、⼃]
  \begin{phonetics}{怎么}{zen3me5}[][HSK 1]
    \definition{pron.}{como? | o que?}
  \end{phonetics}
\end{entry}

\begin{entry}{怎么了}{9,3,2}[Radicais ⼼、⼃、⼅]
  \begin{phonetics}{怎么了}{zen3me5le5}
    \definition{expr.}{O que aconteceu? | O que está acontecendo? | E aí?}
  \end{phonetics}
\end{entry}

\begin{entry}{怎么办}{9,3,4}[Radicais ⼼、⼃、⼒]
  \begin{phonetics}{怎么办}{zen3 me5 ban4}[][HSK 2]
    \definition{adv.}{o que fazer?}
  \end{phonetics}
\end{entry}

\begin{entry}{怎么回事}{9,3,6,8}[Radicais ⼼、⼃、⼞、⼅]
  \begin{phonetics}{怎么回事}{zen3me5hui2shi4}
    \definition{expr.}{O que aconteceu? | O que se passou?}
  \end{phonetics}
\end{entry}

\begin{entry}{怎么样}{9,3,10}[Radicais ⼼、⼃、⽊]
  \begin{phonetics}{怎么样}{zen3me5yang4}[][HSK 2]
    \definition{adv.}{como? | que tal?}
  \end{phonetics}
\end{entry}

\begin{entry}{怎么得了}{9,3,11,2}[Radicais ⼼、⼃、⼻、⼅]
  \begin{phonetics}{怎么得了}{zen3me5de2liao3}
    \definition{expr.}{Como isso pode ser? | Que bagunça horrível! | O que deve ser feito?}
  \end{phonetics}
\end{entry}

\begin{entry}{怎么搞的}{9,3,13,8}[Radicais ⼼、⼃、⼿、⽩]
  \begin{phonetics}{怎么搞的}{zen3me5gao3de5}
    \definition{expr.}{Como isso aconteceu? | O que deu errado? | E aí? | O que está errado?}
  \end{phonetics}
\end{entry}

\begin{entry}{怎样}{9,10}[Radicais ⼼、⽊]
  \begin{phonetics}{怎样}{zen3 yang4}[][HSK 2]
    \definition{pron.}{como | o que | de uma certa maneira | de qualquer maneira | não importa o quão}
  \end{phonetics}
\end{entry}

\begin{entry}{怒骂}{9,9}[Radicais ⼼、⾺]
  \begin{phonetics}{怒骂}{nu4ma4}
    \definition{v.}{praguejar de raiva}
  \end{phonetics}
\end{entry}

\begin{entry}{思想}{9,13}[Radicais ⼼、⼼]
  \begin{phonetics}{思想}{si1xiang3}[][HSK 3]
    \definition[个]{s.}{reflexão; pensamento; ideologia | ideia}
  \end{phonetics}
\end{entry}

\begin{entry}{急}{9}[Radical ⼼]
  \begin{phonetics}{急}{ji2}[][HSK 2]
    \definition{adj.}{impaciente |ansioso | irritado | aborrecido |violento | urgente | premente}
    \definition{s.}{urgência | emergência}
    \definition{v.}{preocupar | estar ansioso para ajudar}
  \end{phonetics}
\end{entry}

\begin{entry}{急忙}{9,6}[Radicais ⼼、⼼]
  \begin{phonetics}{急忙}{ji2mang2}[][HSK 4]
    \definition{adv.}{apressadamente; com pressa}
  \end{phonetics}
\end{entry}

\begin{entry}{急救}{9,11}[Radicais ⼼、⽁]
  \begin{phonetics}{急救}{ji2jiu4}
    \definition{s.}{primeiros socorros}
    \definition{v.}{dar tratamento de emergência}
  \end{phonetics}
\end{entry}

\begin{entry}{怹}{9}[Radical ⼼]
  \begin{phonetics}{怹}{tan1}
    \definition{pron.}{ele, ela (cortês, em oposição a 他)}
    \seeref{他}{ta1}
  \end{phonetics}
\end{entry}

\begin{entry}{总}{9}[Radical ⼼]
  \begin{phonetics}{总}{zong3}[][HSK 3]
    \definition{adj.}{total; geral; global | responsável (liderança)}
    \definition{adv.}{sempre; invariavelmente | de qualquer forma; afinal; eventualmente; mais cedo ou mais tarde | seguramente; provavelmente; certamente}
    \definition{v.}{resumir; juntar; reunir}
  \end{phonetics}
\end{entry}

\begin{entry}{总长}{9,4}[Radicais ⼼、⾧]
  \begin{phonetics}{总长}{zong3chang2}
    \definition{s.}{comprimento total}
  \end{phonetics}
\end{entry}

\begin{entry}{总务}{9,5}[Radicais ⼼、⼒]
  \begin{phonetics}{总务}{zong3wu4}
    \definition{s.}{divisão de assuntos gerais | assuntos gerais | pessoa responsável geral}
  \end{phonetics}
\end{entry}

\begin{entry}{总台}{9,5}[Radicais ⼼、⼝]
  \begin{phonetics}{总台}{zong3tai2}
    \definition{s.}{recepção | balcão de recepção}
  \end{phonetics}
\end{entry}

\begin{entry}{总价}{9,6}[Radicais ⼼、⼈]
  \begin{phonetics}{总价}{zong3jia4}
    \definition{s.}{preço total}
  \end{phonetics}
\end{entry}

\begin{entry}{总线}{9,8}[Radicais ⼼、⽷]
  \begin{phonetics}{总线}{zong3xian4}
    \definition{s.}{barramento (computador) | \emph{computer bus}}
  \end{phonetics}
\end{entry}

\begin{entry}{总是}{9,9}[Radicais ⼼、⽇]
  \begin{phonetics}{总是}{zong3shi4}[][HSK 3]
    \definition{adv.}{sempre; indica que algo está acontecendo por um período de tempo; um certo estado permanece inalterado
 | afinal; significa que não importa o que aconteça, haverá um resultado.}
  \end{phonetics}
\end{entry}

\begin{entry}{总结}{9,9}[Radicais ⼼、⽷]
  \begin{phonetics}{总结}{zong3jie2}[][HSK 3]
    \definition[个,份]{s.}{resumo; conclusão obtida}
    \definition{v.}{resumir; sumariar; analisar a experiência da pesquisa e tirar conclusões}
  \end{phonetics}
\end{entry}

\begin{entry}{总统}{9,9}[Radicais ⼼、⽷]
  \begin{phonetics}{总统}{zong3tong3}
    \definition*[个,位,名,届]{s.}{Presidente (de um país)}
  \end{phonetics}
\end{entry}

\begin{entry}{总值}{9,10}[Radicais ⼼、⼈]
  \begin{phonetics}{总值}{zong3zhi2}
    \definition{s.}{valor total}
  \end{phonetics}
\end{entry}

\begin{entry}{总站}{9,10}[Radicais ⼼、⽴]
  \begin{phonetics}{总站}{zong3zhan4}
    \definition{s.}{terminal}
  \end{phonetics}
\end{entry}

\begin{entry}{总得}{9,11}[Radicais ⼼、⼻]
  \begin{phonetics}{总得}{zong3dei3}
    \definition{adv.}{prestes a}
    \definition{v.}{dever | precisar}
  \end{phonetics}
\end{entry}

\begin{entry}{总理}{9,11}[Radicais ⼼、⽟]
  \begin{phonetics}{总理}{zong3li3}
    \definition*{s.}{Primeiro-Ministro}
  \end{phonetics}
\end{entry}

\begin{entry}{总督}{9,13}[Radicais ⼼、⽬]
  \begin{phonetics}{总督}{zong3du1}
    \definition*{s.}{Governador-Geral | Governador | Vice-Rei}
  \end{phonetics}
\end{entry}

\begin{entry}{恒星系}{9,9,7}[Radicais ⼼、⽇、⽷]
  \begin{phonetics}{恒星系}{heng2xing1xi4}
    \definition{s.}{sistema estelar | galáxia}
  \end{phonetics}
\end{entry}

\begin{entry}{T-恤}{9}[Radical ⼼]
  \begin{phonetics}{T-恤}{xu4}
    \definition{s.}{camiseta | pulôver | suéter}
  \end{phonetics}
\end{entry}

\begin{entry}{恨}{9}[Radical ⼼]
  \begin{phonetics}{恨}{hen4}
    \definition{s.}{ódio}
    \definition{v.}{odiar}
  \end{phonetics}
\end{entry}

\begin{entry}{恰}{9}[Radical ⼼]
  \begin{phonetics}{恰}{qia4}
    \definition{adv.}{exatamente | apenas}
  \end{phonetics}
\end{entry}

\begin{entry}{恰好}{9,6}[Radicais ⼼、⼥]
  \begin{phonetics}{恰好}{qia4hao3}
    \definition{adv.}{certo | por sorte | ao que parece | por sorte coincidência}
  \end{phonetics}
\end{entry}

\begin{entry}{恰到好处}{9,8,6,5}[Radicais ⼼、⼑、⼥、⼡]
  \begin{phonetics}{恰到好处}{qia4dao4hao3chu4}
    \definition{expr.}{é simplesmente perfeito | é simplesmente correto}
  \end{phonetics}
\end{entry}

\begin{entry}{战}{9}[Radical ⼽]
  \begin{phonetics}{战}{zhan4}
    \definition{s.}{luta | guerra | batalha}
    \definition{v.}{lutar}
  \end{phonetics}
\end{entry}

\begin{entry}{战士}{9,3}[Radicais ⼽、⼠]
  \begin{phonetics}{战士}{zhan4shi4}
    \definition[个]{s.}{lutador | soldado | guerreiro}
  \end{phonetics}
\end{entry}

\begin{entry}{战争}{9,6}[Radicais ⼽、⼑]
  \begin{phonetics}{战争}{zhan4zheng1}
    \definition[場,次]{s.}{guerra | conflito}
  \end{phonetics}
\end{entry}

\begin{entry}{括号}{9,5}[Radicais ⼿、⼝]
  \begin{phonetics}{括号}{kuo4 hao4}[][HSK 4]
    \definition{s.}{chaves, colchetes e parênteses (em fórmulas aritméticas ou algébricas, os símbolos que indicam a combinação e a ordem de vários números ou termos) | colchetes e parênteses usados como um tipo de sinal de pontuação para mostrar a parte explicativa de uma passagem em um texto}
  \end{phonetics}
\end{entry}

\begin{entry}{拼}{9}[Radical ⼿]
  \begin{phonetics}{拼}{pin1}
    \definition{v.}{soletrar | juntar | unir}
  \end{phonetics}
\end{entry}

\begin{entry}{拼命}{9,8}[Radicais ⼿、⼝]
  \begin{phonetics}{拼命}{pin1ming4}
    \definition{adv.}{com toda a força | desesperadamente}
    \definition{v.+compl.}{arriscar a vida de alguém | desafiar a morte | colocar-se em uma luta desesperada | fazer algo desesperadamente | exercer a maior força}
  \end{phonetics}
\end{entry}

\begin{entry}{拼音}{9,9}[Radicais ⼿、⾳]
  \begin{phonetics}{拼音}{pin1yin1}
    \definition{s.}{escrita fonética | pinyin (romanização chinesa)}
  \end{phonetics}
\end{entry}

\begin{entry}{持续}{9,11}[Radicais ⼿、⽷]
  \begin{phonetics}{持续}{chi2xu4}[][HSK 3]
    \definition{v.}{durar; continuar; sustentar}
  \end{phonetics}
\end{entry}

\begin{entry}{挂}{9}[Radical ⼿]
  \begin{phonetics}{挂}{gua4}[][HSK 3]
    \definition{clas.}{para conjuntos ou sequência de itens}
    \definition{v.}{pendurar; colocar; suspender | interromper chamada (telefônica) | colocar alguém em contato com; ligar; telefonar
pegar carona; ser pego | ter em mente; estar preocupado com | ser revestido com; ser coberto com | colocar em registro; registrar}
  \end{phonetics}
\end{entry}

\begin{entry}{挂号}{9,5}[Radicais ⼿、⼝]
  \begin{phonetics}{挂号}{gua4hao4}
    \definition{v.+compl.}{registrar-se (em um hospital, etc.) | enviar através de carta registrada}
  \end{phonetics}
\end{entry}

\begin{entry}{挂号信}{9,5,9}[Radicais ⼿、⼝、⼈]
  \begin{phonetics}{挂号信}{gua4hao4xin4}
    \definition{s.}{carta registrada}
  \end{phonetics}
\end{entry}

\begin{entry}{指}{9}[Radical ⼿]
  \begin{phonetics}{指}{zhi3}[][HSK 3]
    \definition*{s.}{sobrenome Zhi}
    \definition{clas.}{dígito; largura do dedo; a largura de um dedo é chamada de "一指", que é usado para medir profundidade, largura, etc.}
    \definition{s.}{dedo}
    \definition{v.}{apontar para | (pelo) eriçar | indicar; mostrar-se; apontar; demonstrar | referir-se a; dirigir-se a | confiar em; contar com; depender de | criticar; repreender}
  \end{phonetics}
\end{entry}

\begin{entry}{指出}{9,5}[Radicais ⼿、⼐]
  \begin{phonetics}{指出}{zhi3 chu1}[][HSK 3]
    \definition{v.}{apontar; indicar}
  \end{phonetics}
\end{entry}

\begin{entry}{指甲}{9,5}[Radicais ⼿、⽥]
  \begin{phonetics}{指甲}{zhi3jia5}
    \definition{s.}{unha da mão}
  \end{phonetics}
\end{entry}

\begin{entry}{指导}{9,6}[Radicais ⼿、⼨]
  \begin{phonetics}{指导}{zhi3dao3}[][HSK 3]
    \definition{s.}{guia; pessoa que faz trabalho de orientação}
    \definition{v.}{guiar; dirigir; instruir}
  \end{phonetics}
\end{entry}

\begin{entry}{指南针}{9,9,7}[Radicais ⼿、⼗、⾦]
  \begin{phonetics}{指南针}{zhi3nan2zhen1}
    \definition{s.}{bússola}
  \end{phonetics}
\end{entry}

\begin{entry}{指挥}{9,9}[Radicais ⼿、⼿]
  \begin{phonetics}{指挥}{zhi3hui1}
    \definition[个]{s.}{condutor (de uma orquestra)}
    \definition{v.}{conduzir | comandar | direcionar}
  \end{phonetics}
\end{entry}

\begin{entry}{按}{9}[Radical ⼿]
  \begin{phonetics}{按}{an4}[][HSK 3]
    \definition{v.}{pressionar | empurrar para baixo | deixar de lado | arquivar | restringir | controlar}
  \end{phonetics}
\end{entry}

\begin{entry}{按时}{9,7}[Radicais ⼿、⽇]
  \begin{phonetics}{按时}{an4shi2}[][HSK 4]
    \definition{adv.}{na hora; no horário; pontualmente; de acordo com o tempo estipulado}
  \end{phonetics}
\end{entry}

\begin{entry}{按照}{9,13}[Radicais ⼿、⽕]
  \begin{phonetics}{按照}{an4zhao4}[][HSK 3]
    \definition{prep.}{de acordo com | em conformidade com | à luz de | com base em}
  \end{phonetics}
\end{entry}

\begin{entry}{挑衅}{9,11}[Radicais ⼿、⾎]
  \begin{phonetics}{挑衅}{tiao3xin4}
    \definition{s.}{provocação}
    \definition{v.}{provocar}
  \end{phonetics}
\end{entry}

\begin{entry}{挖}{9}[Radical ⼿]
  \begin{phonetics}{挖}{wa1}
    \definition{v.}{cavar | escavar}
  \end{phonetics}
\end{entry}

\begin{entry}{挖掘机}{9,11,6}[Radicais ⼿、⼿、⽊]
  \begin{phonetics}{挖掘机}{wa1jue2ji1}
    \definition{s.}{escavadeira | escavador | escavadora | pá mecânica}
  \end{phonetics}
\end{entry}

\begin{entry}{挡风玻璃}{9,4,9,14}[Radicais ⼿、⾵、⽟、⽟]
  \begin{phonetics}{挡风玻璃}{dang3feng1bo1li5}
    \definition{s.}{parabrisa}
  \end{phonetics}
\end{entry}

\begin{entry}{挣}{9}[Radical ⼿]
  \begin{phonetics}{挣}{zheng4}
    \definition{v.}{ganhar dinheiro | esforçar-se para adquirir | lutar para se libertar}
  \end{phonetics}
\end{entry}

\begin{entry}{挣扎}{9,4}[Radicais ⼿、⼿]
  \begin{phonetics}{挣扎}{zheng1zha2}
    \definition{v.}{lutar}
  \end{phonetics}
\end{entry}

\begin{entry}{挣钱}{9,10}[Radicais ⼿、⾦]
  \begin{phonetics}{挣钱}{zheng4qian2}
    \definition{v.+compl.}{ganhar dinheiro}
  \end{phonetics}
\end{entry}

\begin{entry}{挣得}{9,11}[Radicais ⼿、⼻]
  \begin{phonetics}{挣得}{zheng4de2}
    \definition{v.}{ganhar renda ou dinheiro}
  \end{phonetics}
\end{entry}

\begin{entry}{挥汗如雨}{9,6,6,8}[Radicais ⼿、⽔、⼥、⾬]
  \begin{phonetics}{挥汗如雨}{hui1han4ru2yu3}
    \definition{s.}{suor derramado}
    \definition{v.}{pingar com suor}
  \end{phonetics}
\end{entry}

\begin{entry}{挺}{9}[Radical ⼿]
  \begin{phonetics}{挺}{ting3}[][HSK 2]
    \definition{adj.}{ereto | fora do comum | direto}
    \definition{adv.}{bastante, ou melhor, bonito | muito (coloquial)}
    \definition{clas.}{para metralhadoras}
    \definition{v.}{endireitar (fisicamente) | sobressair (uma parte do corpo) | dar suporte | resistir}
  \end{phonetics}
\end{entry}

\begin{entry}{挺尸}{9,3}[Radicais ⼿、⼫]
  \begin{phonetics}{挺尸}{ting3shi1}
    \definition{v.}{(coloquial) dormir | (literalmente) ficar deitado duro como um cadáver}
  \end{phonetics}
\end{entry}

\begin{entry}{挺立}{9,5}[Radicais ⼿、⽴]
  \begin{phonetics}{挺立}{ting3li4}
    \definition{v.}{ficar ereto | ficar de pé}
  \end{phonetics}
\end{entry}

\begin{entry}{挺好}{9,6}[Radicais ⼿、⼥]
  \begin{phonetics}{挺好}{ting3 hao3}[][HSK 2]
    \definition{adj.}{muito bom}
  \end{phonetics}
\end{entry}

\begin{entry}{挺过}{9,6}[Radicais ⼿、⾡]
  \begin{phonetics}{挺过}{ting3guo4}
    \definition{s.}{sobreviver}
  \end{phonetics}
\end{entry}

\begin{entry}{挺住}{9,7}[Radicais ⼿、⼈]
  \begin{phonetics}{挺住}{ting3zhu4}
    \definition{v.}{permanecer firme | manter-se firme (diante da adversidade ou da dor)}
  \end{phonetics}
\end{entry}

\begin{entry}{挺杆}{9,7}[Radicais ⼿、⽊]
  \begin{phonetics}{挺杆}{ting3gan3}
    \definition{s.}{tucho (peça de máquina)}
  \end{phonetics}
\end{entry}

\begin{entry}{挺身}{9,7}[Radicais ⼿、⾝]
  \begin{phonetics}{挺身}{ting3shen1}
    \definition{v.}{endireitar as costas}
  \end{phonetics}
\end{entry}

\begin{entry}{挺进}{9,7}[Radicais ⼿、⾡]
  \begin{phonetics}{挺进}{ting3jin4}
    \definition{s.}{progresso | avanço}
    \definition{v.}{progredir | avançar}
  \end{phonetics}
\end{entry}

\begin{entry}{挺拔}{9,8}[Radicais ⼿、⼿]
  \begin{phonetics}{挺拔}{ting3ba2}
    \definition{adj.}{alto e reto}
  \end{phonetics}
\end{entry}

\begin{entry}{挺腰}{9,13}[Radicais ⼿、⾁]
  \begin{phonetics}{挺腰}{ting3yao1}
    \definition{v.}{arquear as costas | endireitar as costas}
  \end{phonetics}
\end{entry}

\begin{entry}{政纲}{9,7}[Radicais ⽁、⽷]
  \begin{phonetics}{政纲}{zheng4gang1}
    \definition{s.}{programa ou plataforma política}
  \end{phonetics}
\end{entry}

\begin{entry}{政府}{9,8}[Radicais ⽁、⼴]
  \begin{phonetics}{政府}{zheng4fu3}
    \definition[个]{s.}{governo}
  \end{phonetics}
\end{entry}

\begin{entry}{政治局}{9,8,7}[Radicais ⽁、⽔、⼫]
  \begin{phonetics}{政治局}{zheng4zhi4ju2}
    \definition{s.}{o principal comitê de políticas de um partido comunista}
  \end{phonetics}
\end{entry}

\begin{entry}{故}{9}[Radical ⽁]
  \begin{phonetics}{故}{gu4}
    \definition{conj.}{por isso | portanto | então}
  \end{phonetics}
\end{entry}

\begin{entry}{故乡}{9,3}[Radicais ⽁、⼄]
  \begin{phonetics}{故乡}{gu4xiang1}[][HSK 3]
    \definition[个]{s.}{cidade natal; terra natal}
  \end{phonetics}
\end{entry}

\begin{entry}{故事}{9,8}[Radicais ⽁、⼅]
  \begin{phonetics}{故事}{gu4shi4}
    \definition{s.}{prática antiga}
  \end{phonetics}
  \begin{phonetics}{故事}{gu4shi5}[][HSK 2]
    \definition{s.}{narrativa | história | conto}
  \end{phonetics}
\end{entry}

\begin{entry}{故宫}{9,9}[Radicais ⽁、⼧]
  \begin{phonetics}{故宫}{gu4gong1}
    \definition*{s.}{Palácio Imperial | Cidade Proibida}
  \end{phonetics}
\end{entry}

\begin{entry}{故意}{9,13}[Radicais ⽁、⼼]
  \begin{phonetics}{故意}{gu4yi4}[][HSK 2]
    \definition{adv.}{intencionalmente | deliberadamente | propositalmente}
  \end{phonetics}
\end{entry}

\begin{entry}{既}{9}[Radical ⽆]
  \begin{phonetics}{既}{ji4}[][HSK 4]
    \definition*{s.}{sobrenome Ji}
    \definition{adv.}{já}
    \definition{conj.}{desde; como; agora que | assim como; e também; ambos\dots e\dots; usado em conjunto com advérbios como ``且、又、也'' para indicar uma combinação de ambas as situações}
  \seealsoref{且}{qie3}
  \seealsoref{也}{ye3}
  \seealsoref{又}{you4}
  \end{phonetics}
\end{entry}

\begin{entry}{既又}{9,2}[Radicais ⽆、⼜]
  \begin{phonetics}{既又}{ji4you4}
    \definition{conj.}{desde | como | agora isso | os dois e | assim como}
  \end{phonetics}
\end{entry}

\begin{entry}{既不……又不……}{9,4,2,4}[Radicais ⽆、⼀、⼜、⼀]
  \begin{phonetics}{既不……又不……}{ji4bu4 you4bu4}
    \definition{conj.}{nem mesmo\dots}
  \end{phonetics}
\end{entry}

\begin{entry}{既然}{9,12}[Radicais ⽆、⽕]
  \begin{phonetics}{既然}{ji4ran2}[][HSK 4]
    \definition{conj.}{como; desde; agora que; usado na primeira metade de uma frase, muitas vezes repetido na segunda metade pelos advérbios ``就、也、还'' para indicar que a premissa é primeiro declarada e depois inferida}
  \seealsoref{还}{hai2}
  \seealsoref{就}{jiu4}
  \seealsoref{也}{ye3}
  \end{phonetics}
\end{entry}

\begin{entry}{星火}{9,4}[Radicais ⽇、⽕]
  \begin{phonetics}{星火}{xing1huo3}
    \definition{s.}{trilha de meteoro (usada principalmente em expressões como 急如星火) | faísca}
  \end{phonetics}
\end{entry}

\begin{entry}{星辰}{9,7}[Radicais ⽇、⾠]
  \begin{phonetics}{星辰}{xing1chen2}
    \definition{s.}{estrelas}
  \end{phonetics}
\end{entry}

\begin{entry}{星表}{9,8}[Radicais ⽇、⾐]
  \begin{phonetics}{星表}{xing1biao3}
    \definition{s.}{catálogo de estrelas}
  \end{phonetics}
\end{entry}

\begin{entry}{星星}{9,9}[Radicais ⽇、⽇]
  \begin{phonetics}{星星}{xing1 xing5}[][HSK 2]
    \definition{s.}{estrela}
  \end{phonetics}
\end{entry}

\begin{entry}{星座}{9,10}[Radicais ⽇、⼴]
  \begin{phonetics}{星座}{xing1zuo4}
    \definition[张]{s.}{signo astrológico | constelação}
  \end{phonetics}
\end{entry}

\begin{entry}{星期}{9,12}[Radicais ⽇、⽉]
  \begin{phonetics}{星期}{xing1qi1}[][HSK 1]
    \definition[个]{s.}{semana}
  \end{phonetics}
\end{entry}

\begin{entry}{星期一}{9,12,1}[Radicais ⽇、⽉、⼀]
  \begin{phonetics}{星期一}{xing1qi1yi1}[][HSK 1]
    \definition{s.}{segunda-feira}
  \end{phonetics}
\end{entry}

\begin{entry}{星期二}{9,12,2}[Radicais ⽇、⽉、⼆]
  \begin{phonetics}{星期二}{xing1qi1'er4}[][HSK 1]
    \definition{s.}{terça-feira}
  \end{phonetics}
\end{entry}

\begin{entry}{星期三}{9,12,3}[Radicais ⽇、⽉、⼀]
  \begin{phonetics}{星期三}{xing1qi1san1}[][HSK 1]
    \definition{s.}{quarta-feira}
  \end{phonetics}
\end{entry}

\begin{entry}{星期五}{9,12,4}[Radicais ⽇、⽉、⼆]
  \begin{phonetics}{星期五}{xing1qi1wu3}[][HSK 1]
    \definition{s.}{sexta-feira}
  \end{phonetics}
\end{entry}

\begin{entry}{星期六}{9,12,4}[Radicais ⽇、⽉、⼋]
  \begin{phonetics}{星期六}{xing1qi1liu4}[][HSK 1]
    \definition{s.}{sábado}
  \end{phonetics}
\end{entry}

\begin{entry}{星期天}{9,12,4}[Radicais ⽇、⽉、⼤]
  \begin{phonetics}{星期天}{xing1qi1tian1}[][HSK 1]
    \definition{s.}{domingo}
  \seealsoref{星期日}{xing1qi1ri4}
  \end{phonetics}
\end{entry}

\begin{entry}{星期日}{9,12,4}[Radicais ⽇、⽉、⽇]
  \begin{phonetics}{星期日}{xing1qi1ri4}[][HSK 1]
    \definition{s.}{domingo}
  \seealsoref{星期天}{xing1qi1tian1}
  \end{phonetics}
\end{entry}

\begin{entry}{星期四}{9,12,5}[Radicais ⽇、⽉、⼞]
  \begin{phonetics}{星期四}{xing1qi1si4}[][HSK 1]
    \definition{s.}{quinta-feira}
  \end{phonetics}
\end{entry}

\begin{entry}{春}{9}[Radical ⽇]
  \begin{phonetics}{春}{chun1}
    \definition*{s.}{sobrenome Chun}
    \definition{s.}{primavera | amor | luxúria | vida | vitalidade}
  \end{phonetics}
\end{entry}

\begin{entry}{春天}{9,4}[Radicais ⽇、⼤]
  \begin{phonetics}{春天}{chun1 tian1}
    \definition[个]{s.}{primavera}
  \end{phonetics}
\end{entry}

\begin{entry}{春节}{9,5}[Radicais ⽇、⾋]
  \begin{phonetics}{春节}{chun1 jie2}[][HSK 2]
    \definition*{s.}{Festival da Primavera (Ano Novo Chinês)}
  \end{phonetics}
\end{entry}

\begin{entry}{春季}{9,8}[Radicais ⽇、⼦]
  \begin{phonetics}{春季}{chun1 ji4}[][HSK 4]
    \definition{s.}{primavera; primeiro trimestre do ano, que na China se refere ao período de três meses entre o início da primavera e o início do verão, e também se refere aos três meses do calendário lunar, a saber, o primeiro, o segundo e o terceiro meses}
  \end{phonetics}
\end{entry}

\begin{entry}{昨}{9}[Radical ⽇]
  \begin{phonetics}{昨}{zuo2}
    \definition{s.}{ontem}
  \end{phonetics}
\end{entry}

\begin{entry}{昨天}{9,4}[Radicais ⽇、⼤]
  \begin{phonetics}{昨天}{zuo2tian1}[][HSK 1]
    \definition{adv.}{ontem}
  \end{phonetics}
\end{entry}

\begin{entry}{昨日}{9,4}[Radicais ⽇、⽇]
  \begin{phonetics}{昨日}{zuo2ri4}
    \definition{adv.}{ontem}
  \end{phonetics}
\end{entry}

\begin{entry}{昨夜}{9,8}[Radicais ⽇、⼣]
  \begin{phonetics}{昨夜}{zuo2ye4}
    \definition{adv.}{noite passada}
  \end{phonetics}
\end{entry}

\begin{entry}{昨晚}{9,11}[Radicais ⽇、⽇]
  \begin{phonetics}{昨晚}{zuo2wan3}
    \definition{adv.}{noite passada | ontem à noite}
  \end{phonetics}
\end{entry}

\begin{entry}{是}{9}[Radical ⽇]
  \begin{phonetics}{是}{shi4}[][HSK 1]
    \definition{adj.}{correto | certo | verdadeiro | (reconhecimento respeitoso de um comando) muito bem}
    \definition{adv.}{(advérbio para afirmação enfática)}
    \definition{v.}{ser (somente seguido por substantivos)}
  \end{phonetics}
\end{entry}

\begin{entry}{是的}{9,8}[Radicais ⽇、⽩]
  \begin{phonetics}{是的}{shi4de5}
    \definition{adv.}{sim | está certo}
  \end{phonetics}
\end{entry}

\begin{entry}{显示}{9,5}[Radicais ⽇、⽰]
  \begin{phonetics}{显示}{xian3shi4}[][HSK 3]
    \definition{v.}{mostrar | exibir}
  \end{phonetics}
\end{entry}

\begin{entry}{显得}{9,11}[Radicais ⽇、⼻]
  \begin{phonetics}{显得}{xian3de5}[][HSK 3]
    \definition{v.}{parecer; aparecer}
  \end{phonetics}
\end{entry}

\begin{entry}{显然}{9,12}[Radicais ⽇、⽕]
  \begin{phonetics}{显然}{xian3ran2}[][HSK 3]
    \definition{adj.}{claro; evidente; óbvio}
    \definition{adv.}{claramente; evidentemente; obviamente}
  \end{phonetics}
\end{entry}

\begin{entry}{枯木}{9,4}[Radicais ⽊、⽊]
  \begin{phonetics}{枯木}{ku1mu4}
    \definition{s.}{árvore morta | madeira morta}
  \end{phonetics}
\end{entry}

\begin{entry}{架}{9}[Radical ⽊]
  \begin{phonetics}{架}{jia4}[][HSK 3]
    \definition{clas.}{para coisas com pilares ou componentes mecânicos | quadrado (usado para montanhas)}
    \definition{s.}{quadro; prateleira; suporte | briga; discussão}
    \definition{v.}{colocar para cima; erigir | afastar; resistir | suportar; ajudar | sequestrar; levar alguém embora à força}
  \end{phonetics}
\end{entry}

\begin{entry}{架式}{9,6}[Radicais ⽊、⼷]
  \begin{phonetics}{架式}{jia4shi5}
    \variantof{架势}
  \end{phonetics}
\end{entry}

\begin{entry}{架势}{9,8}[Radicais ⽊、⼒]
  \begin{phonetics}{架势}{jia4shi5}
    \definition{s.}{postura | atitude | posição (sobre um assunto, etc.)}
  \end{phonetics}
\end{entry}

\begin{entry}{柏树}{9,9}[Radicais ⽊、⽊]
  \begin{phonetics}{柏树}{bai3shu4}
    \definition{s.}{cipreste}
  \end{phonetics}
\end{entry}

\begin{entry}{某}{9}[Radical ⽊]
  \begin{phonetics}{某}{mou3}[][HSK 3]
    \definition{pron.}{um certo alguém ou coisa; algum | usado para substituir seu próprio nome}
  \end{phonetics}
\end{entry}

\begin{entry}{柔软}{9,8}[Radicais ⽊、⾞]
  \begin{phonetics}{柔软}{rou2ruan3}
    \definition{adj.}{macio | suave}
  \end{phonetics}
\end{entry}

\begin{entry}{柠檬}{9,17}[Radicais ⽊、⽊]
  \begin{phonetics}{柠檬}{ning2meng2}
    \definition{s.}{limão}
  \end{phonetics}
\end{entry}

\begin{entry}{查}{9}[Radical ⽊]
  \begin{phonetics}{查}{cha2}[][HSK 2]
    \definition{v.}{verificar | examinar | investigar |consultar}
  \end{phonetics}
  \begin{phonetics}{查}{zha1}
    \definition*{s.}{sobrenome Zha}
    \definition{s.}{espinheiro}
  \end{phonetics}
\end{entry}

\begin{entry}{柬埔寨}{9,10,14}[Radicais ⽊、⼟、⼧]
  \begin{phonetics}{柬埔寨}{jian3pu3zhai4}
    \definition*{s.}{Camboja}
  \end{phonetics}
\end{entry}

\begin{entry}{柳}{9}[Radical ⽊]
  \begin{phonetics}{柳}{liu3}
    \definition*{s.}{sobrenome Liu}
    \definition{s.}{salgueiro}
  \end{phonetics}
\end{entry}

\begin{entry}{柳橙汁}{9,16,5}[Radicais ⽊、⽊、⽔]
  \begin{phonetics}{柳橙汁}{liu3cheng2zhi1}
    \definition[瓶,杯,罐,盒]{s.}{suco de laranja}
  \seealsoref{橙汁}{cheng2zhi1}
  \seealsoref{橘子汁}{ju2zi5zhi1}
  \end{phonetics}
\end{entry}

\begin{entry}{标志}{9,7}[Radicais ⽊、⼼]
  \begin{phonetics}{标志}{biao1zhi4}[][HSK 4]
    \definition[个,种]{s.}{sinal; marca; logotipo; símbolo; emblema; marcações que caracterizam um objeto para facilitar a identificação}
    \definition{v.}{marcar; indicar; simbolizar; identificar}
  \end{phonetics}
\end{entry}

\begin{entry}{标准}{9,10}[Radicais ⽊、⼎]
  \begin{phonetics}{标准}{biao1zhun3}[][HSK 3]
    \definition{adj.}{criterioso | padronizado | normatizado}
    \definition[个]{s.}{critério | padrão (oficial) | norma}
  \end{phonetics}
\end{entry}

\begin{entry}{标题}{9,15}[Radicais ⽊、⾴]
  \begin{phonetics}{标题}{biao1ti2}[][HSK 3]
    \definition[个,条,篇]{s.}{título | manchete | cabeçalho}
  \end{phonetics}
\end{entry}

\begin{entry}{树}{9}[Radical ⽊]
  \begin{phonetics}{树}{shu4}[][HSK 1]
    \definition[棵]{s.}{árvore}
    \definition{v.}{cultivar}
  \end{phonetics}
\end{entry}

\begin{entry}{树木}{9,4}[Radicais ⽊、⽊]
  \begin{phonetics}{树木}{shu4mu4}
    \definition{s.}{árvore}
  \end{phonetics}
\end{entry}

\begin{entry}{树叶}{9,5}[Radicais ⽊、⼝]
  \begin{phonetics}{树叶}{shu4ye4}
    \definition{s.}{folhas de árvores}
  \end{phonetics}
\end{entry}

\begin{entry}{树莓}{9,10}[Radicais ⽊、⾋]
  \begin{phonetics}{树莓}{shu4mei2}
    \definition{s.}{framboesa}
  \end{phonetics}
\end{entry}

\begin{entry}{歪}{9}[Radical ⽌]
  \begin{phonetics}{歪}{wai1}
    \definition{adj.}{torto | tortuoso | nocivo}
  \end{phonetics}
\end{entry}

\begin{entry}{歪果仁}{9,8,4}[Radicais ⽌、⽊、⼈]
  \begin{phonetics}{歪果仁}{wai1guo3ren2}
    \definition{s.}{gíria na \emph{Internet} para 外国人}
    \seeref{外国人}{wai4guo2ren2}
  \end{phonetics}
\end{entry}

\begin{entry}{残疾人}{9,10,2}[Radicais ⽍、⽧、⼈]
  \begin{phonetics}{残疾人}{can2ji2ren2}
    \definition{s.}{pessoa com deficiência}
  \end{phonetics}
\end{entry}

\begin{entry}{残酷}{9,14}[Radicais ⽍、⾣]
  \begin{phonetics}{残酷}{can2ku4}
    \definition{adj.}{cruel}
    \definition{s.}{crueldade}
  \end{phonetics}
\end{entry}

\begin{entry}{段}{9}[Radical ⽎]
  \begin{phonetics}{段}{duan4}[][HSK 2]
    \definition*{s.}{sobrenome Duan}
    \definition{clas.}{para histórias, períodos de tempo, desenvolvimento de um tópico, etc.}
    \definition{s.}{parágrafo | seção | segmento | estágio (de um processo)}
  \end{phonetics}
\end{entry}

\begin{entry}{毒}{9}[Radical ⽏]
  \begin{phonetics}{毒}{du2}
    \definition{adj.}{venenoso | tóxico}
    \definition{s.}{veneno | tóxico}
    \definition{v.}{intoxicar}
  \end{phonetics}
\end{entry}

\begin{entry}{毒杀}{9,6}[Radicais ⽏、⽊]
  \begin{phonetics}{毒杀}{du2sha1}
    \definition{v.}{matar por envenenamento}
  \end{phonetics}
\end{entry}

\begin{entry}{毒物}{9,8}[Radicais ⽏、⽜]
  \begin{phonetics}{毒物}{du2wu4}
    \definition{s.}{substância venenosa | toxina}
  \end{phonetics}
\end{entry}

\begin{entry}{毒害}{9,10}[Radicais ⽏、⼧]
  \begin{phonetics}{毒害}{du2hai4}
    \definition{s.}{envenenamento}
    \definition{v.}{envenenar (prejudicar com uma substância tóxica) | envenenar (as mentes das pessoas)}
  \end{phonetics}
\end{entry}

\begin{entry}{毒蛇}{9,11}[Radicais ⽏、⾍]
  \begin{phonetics}{毒蛇}{du2she2}
    \definition{s.}{víbora | cobra venenosa}
  \end{phonetics}
\end{entry}

\begin{entry}{洋葱}{9,12}[Radicais ⽔、⾋]
  \begin{phonetics}{洋葱}{yang2cong1}
    \definition{s.}{cebola}
  \end{phonetics}
\end{entry}

\begin{entry}{洒水}{9,4}[Radicais ⽔、⽔]
  \begin{phonetics}{洒水}{sa3shui3}
    \definition{v.}{borrifar}
  \end{phonetics}
\end{entry}

\begin{entry}{洗}{9}[Radical ⽔]
  \begin{phonetics}{洗}{xi3}[][HSK 1]
    \definition{v.}{lavar | revelar (fotos) | tomar banho}
  \end{phonetics}
\end{entry}

\begin{entry}{洗手}{9,4}[Radicais ⽔、⼿]
  \begin{phonetics}{洗手}{xi3shou3}
    \definition{v.}{ir ao banheiro | lavar as mãos}
  \end{phonetics}
\end{entry}

\begin{entry}{洗手不干}{9,4,4,3}[Radicais ⽔、⼿、⼀、⼲]
  \begin{phonetics}{洗手不干}{xi3shou3bu2gan4}
    \definition{v.}{parar totalmente de fazer algo}
  \end{phonetics}
\end{entry}

\begin{entry}{洗手池}{9,4,6}[Radicais ⽔、⼿、⽔]
  \begin{phonetics}{洗手池}{xi3shou3chi2}
    \definition{s.}{pia de banheiro | lavatório}
  \seealsoref{洗手盆}{xi3shou3pen2}
  \end{phonetics}
\end{entry}

\begin{entry}{洗手间}{9,4,7}[Radicais ⽔、⼿、⾨]
  \begin{phonetics}{洗手间}{xi3shou3jian1}[][HSK 1]
    \definition{s.}{sanitário | toilette | banheiro}
  \end{phonetics}
\end{entry}

\begin{entry}{洗手乳}{9,4,8}[Radicais ⽔、⼿、⼄]
  \begin{phonetics}{洗手乳}{xi3shou3ru3}
    \definition{s.}{sabonete líquido para lavar as mãos}
  \seealsoref{洗手液}{xi3shou3ye4}
  \end{phonetics}
\end{entry}

\begin{entry}{洗手盆}{9,4,9}[Radicais ⽔、⼿、⽫]
  \begin{phonetics}{洗手盆}{xi3shou3pen2}
    \definition{s.}{pia de banheiro | lavatório}
  \seealsoref{洗手池}{xi3shou3chi2}
  \end{phonetics}
\end{entry}

\begin{entry}{洗手液}{9,4,11}[Radicais ⽔、⼿、⽔]
  \begin{phonetics}{洗手液}{xi3shou3ye4}
    \definition{s.}{sabonete líquido para lavar as mãos}
  \seealsoref{洗手乳}{xi3shou3ru3}
  \end{phonetics}
\end{entry}

\begin{entry}{洗礼}{9,5}[Radicais ⽔、⽰]
  \begin{phonetics}{洗礼}{xi3li3}
    \definition{s.}{batismo}
    \definition{v.}{batizar}
  \end{phonetics}
\end{entry}

\begin{entry}{洗衣机}{9,6,6}[Radicais ⽔、⾐、⽊]
  \begin{phonetics}{洗衣机}{xi3 yi1 ji1}[][HSK 2]
    \definition[台]{s.}{máquina de lavar roupa}
  \end{phonetics}
\end{entry}

\begin{entry}{洗劫}{9,7}[Radicais ⽔、⼒]
  \begin{phonetics}{洗劫}{xi3jie2}
    \definition{v.}{saquear | pilhar | roubar}
  \end{phonetics}
\end{entry}

\begin{entry}{洗净}{9,8}[Radicais ⽔、⼎]
  \begin{phonetics}{洗净}{xi3jing4}
    \definition{v.}{lavar (limpeza)}
  \end{phonetics}
\end{entry}

\begin{entry}{洗胃}{9,9}[Radicais ⽔、⾁]
  \begin{phonetics}{洗胃}{xi3wei4}
    \definition{s.}{(medicina) lavagem gástrica}
    \definition{v.}{ter o estômago lavado}
  \end{phonetics}
\end{entry}

\begin{entry}{洗涤}{9,10}[Radicais ⽔、⽔]
  \begin{phonetics}{洗涤}{xi3di2}
    \definition{s.}{enxágue | lava}
    \definition{v.}{enxaguar | lavar}
  \end{phonetics}
\end{entry}

\begin{entry}{洗涤间}{9,10,7}[Radicais ⽔、⽔、⾨]
  \begin{phonetics}{洗涤间}{xi3di2jian1}
    \definition{s.}{lavanderia}
  \end{phonetics}
\end{entry}

\begin{entry}{洗脱}{9,11}[Radicais ⽔、⾁]
  \begin{phonetics}{洗脱}{xi3tuo1}
    \definition{v.}{limpar | purgar | lavar}
  \end{phonetics}
\end{entry}

\begin{entry}{洗碗}{9,13}[Radicais ⽔、⽯]
  \begin{phonetics}{洗碗}{xi3wan3}
    \definition{v.}{lavar pratos}
  \end{phonetics}
\end{entry}

\begin{entry}{洗澡}{9,16}[Radicais ⽔、⽔]
  \begin{phonetics}{洗澡}{xi3zao3}[][HSK 2]
    \definition{v.+compl.}{tomar banho | duchar-se | lavar-se}
  \end{phonetics}
\end{entry}

\begin{entry}{洗澡间}{9,16,7}[Radicais ⽔、⽔、⾨]
  \begin{phonetics}{洗澡间}{xi3zao3jian1}
    \definition[间]{s.}{banheiro}
  \end{phonetics}
\end{entry}

\begin{entry}{洞穴}{9,5}[Radicais ⽔、⽳]
  \begin{phonetics}{洞穴}{dong4xue2}
    \definition{s.}{caverna}
  \end{phonetics}
\end{entry}

\begin{entry}{洪水}{9,4}[Radicais ⽔、⽔]
  \begin{phonetics}{洪水}{hong2shui3}
    \definition{s.}{enchente | inundação | dilúvio}
  \end{phonetics}
\end{entry}

\begin{entry}{洲}{9}[Radical ⽔]
  \begin{phonetics}{洲}{zhou1}
    \definition{s.}{continente | ilha em um rio}
  \end{phonetics}
\end{entry}

\begin{entry}{活}{9}[Radical ⽔]
  \begin{phonetics}{活}{huo2}[][HSK 3]
    \definition{adj.}{vivo; vivendo | vívido; animado; ativo | móvel; em movimento}
    \definition{adv.}{exatamente; simplesmente}
    \definition{s.}{trabalho | produto}
    \definition{v.}{viver | salvar (a vida de uma pessoa)}
  \end{phonetics}
\end{entry}

\begin{entry}{活力}{9,2}[Radicais ⽔、⼒]
  \begin{phonetics}{活力}{huo2li4}
    \definition{s.}{energia | vitalidade | vigor | força vital}
  \end{phonetics}
\end{entry}

\begin{entry}{活动}{9,6}[Radicais ⽔、⼒]
  \begin{phonetics}{活动}{huo2dong4}[][HSK 2]
    \definition[项,个]{s.}{atividade | evento | campanha}
    \definition{v.}{exercer | operar}
  \end{phonetics}
\end{entry}

\begin{entry}{活着}{9,11}[Radicais ⽔、⽬]
  \begin{phonetics}{活着}{huo2zhe5}
    \definition{adj.}{vivo}
  \end{phonetics}
\end{entry}

\begin{entry}{活路}{9,13}[Radicais ⽔、⾜]
  \begin{phonetics}{活路}{huo2lu4}
    \definition{s.}{maneira de sobreviver | meio de subsistência}
  \end{phonetics}
  \begin{phonetics}{活路}{huo2lu5}
    \definition{s.}{labor | trabalho físico}
  \end{phonetics}
\end{entry}

\begin{entry}{派}{9}[Radical ⽔]
  \begin{phonetics}{派}{pai4}[][HSK 3]
    \definition{adj.}{elegante; bonito}
    \definition{clas.}{para grupos, escolas de pensamento ou arte, etc. | para um discursos, atmosferas, cenas, etc.}
    \definition{s.}{panelinha; grupo exclusivo; facção | torta | estilo | afluente; braço de rio}
    \definition{v.}{enviar; despachar | alocar; repartir; distribuir}
  \end{phonetics}
\end{entry}

\begin{entry}{测}{9}[Radical ⽔]
  \begin{phonetics}{测}{ce4}[][HSK 4]
    \definition{v.}{pesquisar; sondar; medir | conjecturar; inferir}
  \end{phonetics}
\end{entry}

\begin{entry}{测试}{9,8}[Radicais ⽔、⾔]
  \begin{phonetics}{测试}{ce4 shi4}[][HSK 4]
    \definition[个]{s.}{exame; teste; medição do conhecimento humano, das habilidades ou do funcionamento de máquinas, ferramentas ou instrumentos}
    \definition{v.}{examinar | testar, medição do desempenho e da precisão de máquinas, instrumentos, aparelhos, etc.}
  \end{phonetics}
\end{entry}

\begin{entry}{测量}{9,12}[Radicais ⽔、⾥]
  \begin{phonetics}{测量}{ce4liang2}[][HSK 4]
    \definition{v.}{aferir; pesquisar; medir; determinar valores relevantes para espaço, tempo, temperatura, velocidade, função, etc.}
  \end{phonetics}
\end{entry}

\begin{entry}{浓}{9}[Radical ⽔]
  \begin{phonetics}{浓}{nong2}[][HSK 4]
    \definition{adj.}{denso; espesso; concentrado; um líquido ou gás que contém mais de um determinado ingrediente | grande; forte; profundo (de grau ou extensão) | profundo; (algumas cores) escuro}
  \end{phonetics}
\end{entry}

\begin{entry}{点}{9}[Radical ⽕]
  \begin{phonetics}{点}{dian3}[][HSK 1]
    \definition{clas.}{para itens | hora cheia}
    \definition{s.}{ponto | gota | mancha | horas | ponto (no espaço ou no tempo) | traço de ponto em caracteres chineses}
    \definition{v.}{desenhar um ponto | verificar uma lista | escolher | pedir (comida em um restaurante) | tocar brevemente | sugerir | acender | derramar um líquido gota a gota}
  \end{phonetics}
\end{entry}

\begin{entry}{点火}{9,4}[Radicais ⽕、⽕]
  \begin{phonetics}{点火}{dian3huo3}
    \definition{s.}{ignição}
    \definition{v.}{inflamar | acender um fogo | agitar | dar partida em um motor | (figurativo) provocar problemas}
  \end{phonetics}
\end{entry}

\begin{entry}{点头}{9,5}[Radicais ⽕、⼤]
  \begin{phonetics}{点头}{dian3 tou2}[][HSK 2]
    \definition{v.}{acenar com a cabeça}
  \end{phonetics}
\end{entry}

\begin{entry}{点名}{9,6}[Radicais ⽕、⼝]
  \begin{phonetics}{点名}{dian3 ming2}[][HSK 4]
    \definition{v.}{fazer a lista de chamada; manter o controle da presença de alguém; chamar nomes para controle de presença | mencionar alguém pelo nome}
  \end{phonetics}
\end{entry}

\begin{entry}{点燃}{9,16}[Radicais ⽕、⽕]
  \begin{phonetics}{点燃}{dian3ran2}
    \definition{v.}{inflamar | incendiar}
  \end{phonetics}
\end{entry}

\begin{entry}{独}{9}[Radical ⽝]
  \begin{phonetics}{独}{du2}
    \definition{adj.}{sozinho | solitário | solteiro}
    \definition{adv.}{apenas}
  \end{phonetics}
\end{entry}

\begin{entry}{独立}{9,5}[Radicais ⽝、⽴]
  \begin{phonetics}{独立}{du2li4}[][HSK 4]
    \definition{adj.}{independente; por conta própria | separado; respectivo; descreve algo que é separado e não está em contato com outra coisa}
    \definition{prep.}{independente de; separado de; não mais anexado à unidade original, mas uma unidade separada}
    \definition{v.}{ficar sozinho | alcançar a independência; tornar-se um país independente; liberdade de um Estado, regime ou organização contra interferência, controle e dominação por forças externas}
  \end{phonetics}
\end{entry}

\begin{entry}{独自}{9,6}[Radicais ⽝、⾃]
  \begin{phonetics}{独自}{du2 zi4}[][HSK 4]
    \definition{adj.}{sozinho; por si mesmo; por conta própria}
  \end{phonetics}
\end{entry}

\begin{entry}{独特}{9,10}[Radicais ⽝、⽜]
  \begin{phonetics}{独特}{du2te4}[][HSK 4]
    \definition{adj.}{único; distinto; original; especial}
  \end{phonetics}
\end{entry}

\begin{entry}{玻璃}{9,14}[Radicais ⽟、⽟]
  \begin{phonetics}{玻璃}{bo1li5}
    \definition[张,塊]{s.}{vidro | (gíria) homossexual masculino}
  \end{phonetics}
\end{entry}

\begin{entry}{珍贵}{9,9}[Radicais ⽟、⾙]
  \begin{phonetics}{珍贵}{zhen1gui4}
    \definition{adj.}{precioso}
  \end{phonetics}
\end{entry}

\begin{entry}{珍珠}{9,10}[Radicais ⽟、⽟]
  \begin{phonetics}{珍珠}{zhen1zhu1}
    \definition[颗]{s.}{pérola}
  \end{phonetics}
\end{entry}

\begin{entry}{甚而}{9,6}[Radicais ⽢、⽽]
  \begin{phonetics}{甚而}{shen4'er2}
    \definition{conj.}{(ir) tão longe quanto | tanto que}
  \end{phonetics}
\end{entry}

\begin{entry}{甚至}{9,6}[Radicais ⽢、⾄]
  \begin{phonetics}{甚至}{shen4zhi4}
    \definition{conj.}{(ir) tão longe quanto | tanto que | mesmo (na medida em que)}
  \end{phonetics}
\end{entry}

\begin{entry}{甚或}{9,8}[Radicais ⽢、⼽]
  \begin{phonetics}{甚或}{shen4huo4}
    \definition{conj.}{(ir) tão longe quanto | tanto que}
  \end{phonetics}
\end{entry}

\begin{entry}{甭}{9}[Radical ⽤]
  \begin{phonetics}{甭}{beng2}
    \definition{v.}{contração de 不用 | não precisar}
    \seeref{不用}{bu2 yong4}
  \end{phonetics}
\end{entry}

\begin{entry}{界碑}{9,13}[Radicais ⽥、⽯]
  \begin{phonetics}{界碑}{jie4bei1}
    \definition{s.}{marco de fronteira}
  \end{phonetics}
\end{entry}

\begin{entry}{疯狂}{9,7}[Radicais ⽧、⽝]
  \begin{phonetics}{疯狂}{feng1kuang2}
    \definition{adj.}{louco | frenético | selvagem}
  \end{phonetics}
\end{entry}

\begin{entry}{皆}{9}[Radical ⽩]
  \begin{phonetics}{皆}{jie1}
    \definition{adv.}{todos | em todos os casos}
  \end{phonetics}
\end{entry}

\begin{entry}{皇帝}{9,9}[Radicais ⽩、⼱]
  \begin{phonetics}{皇帝}{huang2di4}
    \definition[个]{s.}{imperador}
  \end{phonetics}
\end{entry}

\begin{entry}{盆}{9}[Radical ⽫]
  \begin{phonetics}{盆}{pen2}
    \definition[个]{s.}{panela | bacia | vaso de flores}
  \end{phonetics}
\end{entry}

\begin{entry}{盆友}{9,4}[Radicais ⽫、⼜]
  \begin{phonetics}{盆友}{pen2you3}
    \definition{s.}{(gíria na \emph{Internet}) amigo (trocadilho com 朋友)}
    \seeref{朋友}{peng2you5}
  \end{phonetics}
\end{entry}

\begin{entry}{相互}{9,4}[Radicais ⽬、⼆]
  \begin{phonetics}{相互}{xiang1 hu4}[][HSK 3]
    \definition{adj.}{mútuo; recíproco}
    \definition{adv.}{mutuamente; um ao outro}
  \end{phonetics}
\end{entry}

\begin{entry}{相比}{9,4}[Radicais ⽬、⽐]
  \begin{phonetics}{相比}{xiang1 bi3}[][HSK 3]
    \definition{v.}{combinar; comparar com |comparar uma coisa com outra, usar uma coisa como padrão para ver as características de outra coisa ou para obter um ponto de vista}
  \end{phonetics}
\end{entry}

\begin{entry}{相处}{9,5}[Radicais ⽬、⼡]
  \begin{phonetics}{相处}{xiang1chu3}
    \definition{v.}{entrar em contato (com alguém) | associar | interagir | se dar bem (bem, mal)}
  \end{phonetics}
\end{entry}

\begin{entry}{相似}{9,6}[Radicais ⽬、⼈]
  \begin{phonetics}{相似}{xiang1si4}[][HSK 3]
    \definition{v.}{assemelhar-se; ser semelhante; ser igual}
  \end{phonetics}
\end{entry}

\begin{entry}{相关}{9,6}[Radicais ⽬、⼋]
  \begin{phonetics}{相关}{xiang1guan1}[][HSK 3]
    \definition{v.}{mutuamente relacionados; inter-relacionados}
  \end{phonetics}
\end{entry}

\begin{entry}{相同}{9,6}[Radicais ⽬、⼝]
  \begin{phonetics}{相同}{xiang1tong2}[][HSK 2]
    \definition{adj.}{igual | idêntico | o mesmo}
  \end{phonetics}
\end{entry}

\begin{entry}{相当}{9,6}[Radicais ⽬、⼹]
  \begin{phonetics}{相当}{xiang1dang1}[][HSK 3]
    \definition{adj.}{adequado; ajustado; apropriado}
    \definition{adv.}{bastante; razoavelmente; consideravelmente}
    \definition{v.}{combinar; equilibrar; corresponder a; ser aproximadamente igual a; ser compatível com}
  \end{phonetics}
\end{entry}

\begin{entry}{相机}{9,6}[Radicais ⽬、⽊]
  \begin{phonetics}{相机}{xiang4 ji1}[][HSK 2]
    \definition[台,个]{s.}{câmera | máquina fotográfica}
    \definition{v.}{ficar atento a uma oportunidade}
  \end{phonetics}
\end{entry}

\begin{entry}{相宜}{9,8}[Radicais ⽬、⼧]
  \begin{phonetics}{相宜}{xiang1yi2}
    \definition{adj.}{adequado | apropriado}
    \definition{v.}{ser adequado ou apropriado}
  \end{phonetics}
\end{entry}

\begin{entry}{相亲}{9,9}[Radicais ⽬、⼇]
  \begin{phonetics}{相亲}{xiang1qin1}
    \definition{s.}{encontro às cegas | entrevista arranjada para avaliar a proposta de um parceiro de casamento | apegar-se profundamente um ao outro}
  \end{phonetics}
\end{entry}

\begin{entry}{相信}{9,9}[Radicais ⽬、⼈]
  \begin{phonetics}{相信}{xiang1xin4}[][HSK 2]
    \definition{v.}{acreditar | estar convencido | aceitar como verdadeiro}
  \end{phonetics}
\end{entry}

\begin{entry}{相思病}{9,9,10}[Radicais ⽬、⼼、⽧]
  \begin{phonetics}{相思病}{xiang1si1bing4}
    \definition{s.}{saudade de amor}
  \end{phonetics}
\end{entry}

\begin{entry}{相遇}{9,12}[Radicais ⽬、⾡]
  \begin{phonetics}{相遇}{xiang1yu4}
    \definition{v.}{encontrar (reunião, encontro, etc.)}
  \end{phonetics}
\end{entry}

\begin{entry}{相聚}{9,14}[Radicais ⽬、⽿]
  \begin{phonetics}{相聚}{xiang1ju4}
    \definition{v.}{reunir-se | montar}
  \end{phonetics}
\end{entry}

\begin{entry}{省}{9}[Radical ⽬]
  \begin{phonetics}{省}{sheng3}[][HSK 2]
    \definition{s.}{província | capital provincial}
    \definition{v.}{economizar | guardar | ser frugal | omitir | excluir | deixar de fora}
  \end{phonetics}
  \begin{phonetics}{省}{xing3}
    \definition[个]{s.}{governadoria}
    \definition{v.}{examinar minuciosamente | refletir (sobre a conduta de alguém) | realizar | fazer uma visita (aos pais ou idosos)}
  \end{phonetics}
\end{entry}

\begin{entry}{省力}{9,2}[Radicais ⽬、⼒]
  \begin{phonetics}{省力}{sheng3li4}
    \definition{v.}{economizar esforço ou trabalho}
  \end{phonetics}
\end{entry}

\begin{entry}{省心}{9,4}[Radicais ⽬、⼼]
  \begin{phonetics}{省心}{sheng3xin1}
    \definition{adj.}{despreocupado}
    \definition{v.}{ser poupado de preocupações | despreocupar-se}
  \end{phonetics}
\end{entry}

\begin{entry}{省长}{9,4}[Radicais ⽬、⾧]
  \begin{phonetics}{省长}{sheng3zhang3}
    \definition*{s.}{Governador | governador de uma província}
  \end{phonetics}
\end{entry}

\begin{entry}{省会}{9,6}[Radicais ⽬、⼈]
  \begin{phonetics}{省会}{sheng3hui4}
    \definition{s.}{capital da província}
  \end{phonetics}
\end{entry}

\begin{entry}{省却}{9,7}[Radicais ⽬、⼙]
  \begin{phonetics}{省却}{sheng3que4}
    \definition{v.}{livrar-se (para economizar espaço) | salvar}
  \end{phonetics}
\end{entry}

\begin{entry}{省俭}{9,9}[Radicais ⽬、⼈]
  \begin{phonetics}{省俭}{sheng3jian3}
    \definition{s.}{econômico | frugal}
    \definition{v.}{economizar}
  \end{phonetics}
\end{entry}

\begin{entry}{省城}{9,9}[Radicais ⽬、⼟]
  \begin{phonetics}{省城}{sheng3cheng2}
    \definition{s.}{capital da província}
  \end{phonetics}
\end{entry}

\begin{entry}{省悟}{9,10}[Radicais ⽬、⼼]
  \begin{phonetics}{省悟}{xing3wu4}
    \definition{v.}{voltar a si | constatar | ver a verdade | acordar para a realidade}
  \end{phonetics}
\end{entry}

\begin{entry}{省钱}{9,10}[Radicais ⽬、⾦]
  \begin{phonetics}{省钱}{sheng3qian2}
    \definition{v.}{economizar dinheiro}
  \end{phonetics}
\end{entry}

\begin{entry}{眉}{9}[Radical ⽬]
  \begin{phonetics}{眉}{mei2}
    \definition{s.}{sobrancelha | margem superior}
  \end{phonetics}
\end{entry}

\begin{entry}{眉毛}{9,4}[Radicais ⽬、⽑]
  \begin{phonetics}{眉毛}{mei2mao5}
    \definition[根]{s.}{sobrancelha}
  \end{phonetics}
\end{entry}

\begin{entry}{眉头}{9,5}[Radicais ⽬、⼤]
  \begin{phonetics}{眉头}{mei2tou2}
    \definition{s.}{testa}
  \end{phonetics}
\end{entry}

\begin{entry}{看}{9}[Radical ⽬]
  \begin{phonetics}{看}{kan1}
    \definition{v.}{cuidar | vigiar}
  \end{phonetics}
  \begin{phonetics}{看}{kan4}[][HSK 1]
    \definition{interj.}{Cuidado! (para um perigo)}
    \definition{part.}{(depois de um verbo) tentar}
    \definition{v.}{olhar | ver | assistir | ler | visitar (pessoas)}
  \end{phonetics}
\end{entry}

\begin{entry}{看上去}{9,3,5}[Radicais ⽬、⼀、⼛]
  \begin{phonetics}{看上去}{kan4 shang4 qu4}[][HSK 3]
    \definition{adv.}{parece que}
  \end{phonetics}
\end{entry}

\begin{entry}{看不起}{9,4,10}[Radicais ⽬、⼀、⾛]
  \begin{phonetics}{看不起}{kan4bu5qi3}[][HSK 4]
    \definition{v.}{desprezar; desdenhar; menosprezar; ter desprezo; olhar de cima para baixo}
  \end{phonetics}
\end{entry}

\begin{entry}{看见}{9,4}[Radicais ⽬、⾒]
  \begin{phonetics}{看见}{kan4 jian4}[][HSK 1]
    \definition{v.}{encontrar | enxergar | ver | avistar}
  \end{phonetics}
\end{entry}

\begin{entry}{看来}{9,7}[Radicais ⽬、⽊]
  \begin{phonetics}{看来}{kan4 lai2}[][HSK 4]
    \definition{adv.}{parecer; parecer como se (ou embora); refere-se a um julgamento aproximado; expressa um julgamento por observação}
    \definition{v.}{ser considerado; na visão de alguém; na opinião de alguém; expressar a ideia aproximada que o locutor tem da situação}
  \end{phonetics}
\end{entry}

\begin{entry}{看到}{9,8}[Radicais ⽬、⼑]
  \begin{phonetics}{看到}{kan4 dao4}[][HSK 1]
    \definition{v.}{ver}
  \end{phonetics}
\end{entry}

\begin{entry}{看法}{9,8}[Radicais ⽬、⽔]
  \begin{phonetics}{看法}{kan4fa3}[][HSK 2]
    \definition[个]{s.}{modo de olhar alguma coisa | ponto de vista | opinião}
  \end{phonetics}
\end{entry}

\begin{entry}{看病}{9,10}[Radicais ⽬、⽧]
  \begin{phonetics}{看病}{kan4 bing4}[][HSK 1]
    \definition{v.+compl.}{(médico) ver um paciente | (paciente) consultar (ver) um médico}
  \end{phonetics}
\end{entry}

\begin{entry}{看起来}{9,10,7}[Radicais ⽬、⾛、⽊]
  \begin{phonetics}{看起来}{kan4 qi3 lai5}[][HSK 3]
    \definition{v.}{parecer; parecer com}
  \end{phonetics}
\end{entry}

\begin{entry}{看望}{9,11}[Radicais ⽬、⽉]
  \begin{phonetics}{看望}{kan4wang4}[][HSK 4]
    \definition{v.}{ver; visitar; ligar; dar uma olhada; ir até os pais, idosos, professores ou amigos para cumprimentá-los}
  \end{phonetics}
\end{entry}

\begin{entry}{看淡}{9,11}[Radicais ⽬、⽔]
  \begin{phonetics}{看淡}{kan4dan4}
    \definition{v.}{considerar sem importância | ser indiferente a (fama, riqueza, etc.) | (de uma economia ou mercado) enfraquecer, ficar mais lento, diminuir a velocidade}
  \end{phonetics}
\end{entry}

\begin{entry}{砂}{9}[Radical ⽯]
  \begin{phonetics}{砂}{sha1}
    \variantof{沙}
  \end{phonetics}
\end{entry}

\begin{entry}{砍}{9}[Radical ⽯]
  \begin{phonetics}{砍}{kan3}
    \definition{v.}{cortar}
  \end{phonetics}
\end{entry}

\begin{entry}{砍刀}{9,2}[Radicais ⽯、⼑]
  \begin{phonetics}{砍刀}{kan3dao1}
    \definition{s.}{facão | machete}
  \end{phonetics}
\end{entry}

\begin{entry}{砍头}{9,5}[Radicais ⽯、⼤]
  \begin{phonetics}{砍头}{kan3tou2}
    \definition{v.}{decapitar}
  \end{phonetics}
\end{entry}

\begin{entry}{砍价}{9,6}[Radicais ⽯、⼈]
  \begin{phonetics}{砍价}{kan3jia4}
    \definition{v.}{barganhar | cortar ou derrubar um preço}
  \end{phonetics}
\end{entry}

\begin{entry}{砍伤}{9,6}[Radicais ⽯、⼈]
  \begin{phonetics}{砍伤}{kan3shang1}
    \definition{v.}{ferir com lâmina ou machado}
  \end{phonetics}
\end{entry}

\begin{entry}{砍杀}{9,6}[Radicais ⽯、⽊]
  \begin{phonetics}{砍杀}{kan3sha1}
    \definition{v.}{atacar com arma branca}
  \end{phonetics}
\end{entry}

\begin{entry}{砍死}{9,6}[Radicais ⽯、⽍]
  \begin{phonetics}{砍死}{kan3si3}
    \definition{v.}{matar com um machado}
  \end{phonetics}
\end{entry}

\begin{entry}{砍树}{9,9}[Radicais ⽯、⽊]
  \begin{phonetics}{砍树}{kan3shu4}
    \definition{v.}{derrubar árvores}
  \end{phonetics}
\end{entry}

\begin{entry}{砍掉}{9,11}[Radicais ⽯、⼿]
  \begin{phonetics}{砍掉}{kan3diao4}
    \definition{v.}{amputar}
  \end{phonetics}
\end{entry}

\begin{entry}{砍断}{9,11}[Radicais ⽯、⽄]
  \begin{phonetics}{砍断}{kan3duan4}
    \definition{v.}{cortar}
  \end{phonetics}
\end{entry}

\begin{entry}{砖}{9}[Radical ⽯]
  \begin{phonetics}{砖}{zhuan1}
    \definition[块]{s.}{tijolo}
  \end{phonetics}
\end{entry}

\begin{entry}{祖国}{9,8}[Radicais ⽰、⼞]
  \begin{phonetics}{祖国}{zu3guo2}
    \definition{s.}{pátria | terra natal}
  \end{phonetics}
\end{entry}

\begin{entry}{祝}{9}[Radical ⽰]
  \begin{phonetics}{祝}{zhu4}[][HSK 3]
    \definition*{s.}{sobrenome Zhu}
    \definition{v.}{expressar bons desejos; desejar; abençoar | orar aos deuses ou espíritos por bênçãos}
  \end{phonetics}
\end{entry}

\begin{entry}{祝好}{9,6}[Radicais ⽰、⼥]
  \begin{phonetics}{祝好}{zhu4hao3}
    \definition{expr.}{desejo-lhe tudo de melhor! (ao encerrar uma correspondência)}
  \end{phonetics}
\end{entry}

\begin{entry}{祝寿}{9,7}[Radicais ⽰、⼨]
  \begin{phonetics}{祝寿}{zhu4shou4}
    \definition{v.}{dar parabéns pelo aniversário (a uma pessoa idosa)}
  \end{phonetics}
\end{entry}

\begin{entry}{祝贺}{9,9}[Radicais ⽰、⾙]
  \begin{phonetics}{祝贺}{zhu4he4}
    \definition[个]{s.}{congratulações}
    \definition{v.}{congratular}
  \end{phonetics}
\end{entry}

\begin{entry}{祝酒}{9,10}[Radicais ⽰、⾣]
  \begin{phonetics}{祝酒}{zhu4jiu3}
    \definition{v.}{parabenizar e fazer um brinde | brindar}
  \end{phonetics}
\end{entry}

\begin{entry}{祝颂}{9,10}[Radicais ⽰、⾴]
  \begin{phonetics}{祝颂}{zhu4song4}
    \definition{v.}{expressar bons desejos}
  \end{phonetics}
\end{entry}

\begin{entry}{祝祷}{9,11}[Radicais ⽰、⽰]
  \begin{phonetics}{祝祷}{zhu4dao3}
    \definition{v.}{rezar | orar}
  \end{phonetics}
\end{entry}

\begin{entry}{祝谢}{9,12}[Radicais ⽰、⾔]
  \begin{phonetics}{祝谢}{zhu4xie4}
    \definition{v.}{agradecer | dar parabéns}
  \end{phonetics}
\end{entry}

\begin{entry}{祝福}{9,13}[Radicais ⽰、⽰]
  \begin{phonetics}{祝福}{zhu4fu2}
    \definition{s.}{bênçãos}
    \definition{v.}{desejar boa sorte a alguém}
  \end{phonetics}
\end{entry}

\begin{entry}{祝愿}{9,14}[Radicais ⽰、⽕]
  \begin{phonetics}{祝愿}{zhu4yuan4}
    \definition{v.}{desejar}
  \end{phonetics}
\end{entry}

\begin{entry}{神}{9}[Radical ⽰]
  \begin{phonetics}{神}{shen2}
    \definition*{s.}{Deus}
    \definition{s.}{deus | divindade}
  \end{phonetics}
\end{entry}

\begin{entry}{神奇}{9,8}[Radicais ⽰、⼤]
  \begin{phonetics}{神奇}{shen2qi2}
    \definition{adj.}{mágico | místico | milagroso}
    \definition{s.}{mágica | milagre}
  \end{phonetics}
\end{entry}

\begin{entry}{神明}{9,8}[Radicais ⽰、⽇]
  \begin{phonetics}{神明}{shen2ming2}
    \definition{s.}{divindades | deuses}
  \end{phonetics}
\end{entry}

\begin{entry}{神经}{9,8}[Radicais ⽰、⽷]
  \begin{phonetics}{神经}{shen2jing1}
    \definition{adj.}{desequilibrado | louco | insano}
    \definition{s.}{nervo}
  \end{phonetics}
\end{entry}

\begin{entry}{神经病学}{9,8,10,8}[Radicais ⽰、⽷、⽧、⼦]
  \begin{phonetics}{神经病学}{shen2jing1bing4xue2}
    \definition{s.}{neurologia}
  \end{phonetics}
\end{entry}

\begin{entry}{神经病的}{9,8,10,8}[Radicais ⽰、⽷、⽧、⽩]
  \begin{phonetics}{神经病的}{shen2jing1bing4de5}
    \definition{adj.}{neurótico}
  \end{phonetics}
\end{entry}

\begin{entry}{神话}{9,8}[Radicais ⽰、⾔]
  \begin{phonetics}{神话}{shen2hua4}
    \definition{s.}{lenda | conto de fadas | mito | mitologia}
  \end{phonetics}
\end{entry}

\begin{entry}{神兽}{9,11}[Radicais ⽰、⼋]
  \begin{phonetics}{神兽}{shen2shou4}
    \definition{s.}{animal mitológico | fera}
  \end{phonetics}
\end{entry}

\begin{entry}{神器}{9,16}[Radicais ⽰、⼝]
  \begin{phonetics}{神器}{shen2qi4}
    \definition{s.}{objeto mágico | objeto simbólico do poder imperial | arma fina | ferramenta muito útil}
  \end{phonetics}
\end{entry}

\begin{entry}{秋}{9}[Radical ⽲]
  \begin{phonetics}{秋}{qiu1}
    \definition*{s.}{sobrenome Qiu}
    \definition{s.}{outono | colheita}
  \end{phonetics}
\end{entry}

\begin{entry}{秋天}{9,4}[Radicais ⽲、⼤]
  \begin{phonetics}{秋天}{qiu1 tian1}[][HSK 2]
    \definition[个]{s.}{outono}
  \end{phonetics}
\end{entry}

\begin{entry}{秋季}{9,8}[Radicais ⽲、⼦]
  \begin{phonetics}{秋季}{qiu1 ji4}[][HSK 4]
    \definition[个]{s.}{outono; terceiro trimestre do ano, segundo o costume chinês, refere-se ao período de três meses entre o outono e o inverno, também se refere aos sétimo, oitavo e nono meses do calendário lunar}
  \end{phonetics}
\end{entry}

\begin{entry}{种}{9}[Radical ⽲]
  \begin{phonetics}{种}{zhong3}[][HSK 3]
    \definition*{s.}{sobrenome Zhong}
    \definition{clas.}{para tipos, espécies e gêneros}
    \definition{s.}{espécie | semente; estirpe; raça | entranhas; brio; coragem; espinha dorsal | tipo; variedade; indica tipo, usado para pessoas e qualquer coisa}
  \end{phonetics}
  \begin{phonetics}{种}{zhong4}
    \definition{v.}{plantar; semear; crescer; cultivar}
  \end{phonetics}
\end{entry}

\begin{entry}{种子}{9,3}[Radicais ⽲、⼦]
  \begin{phonetics}{种子}{zhong3zi5}[][HSK 3]
    \definition[颗,粒]{s.}{semente; um órgão exclusivo de certas plantas, geralmente composto de três partes: tegumento, embrião e endosperma, as sementes podem germinar e se tornar novas plantas sob certas condições | jogador cabeça de chave; na competição, quando é realizada a fase eliminatória, são escolhidos os jogadores mais fortes de cada equipe}
  \end{phonetics}
\end{entry}

\begin{entry}{种地}{9,6}[Radicais ⽲、⼟]
  \begin{phonetics}{种地}{zhong4di4}
    \definition{v.}{cultivar | trabalhar a terra}
  \end{phonetics}
\end{entry}

\begin{entry}{种种}{9,9}[Radicais ⽲、⽲]
  \begin{phonetics}{种种}{zhong3zhong3}
    \definition{adj.}{todos os tipos de}
  \end{phonetics}
\end{entry}

\begin{entry}{种族灭绝}{9,11,5,9}[Radicais ⽲、⽅、⽕、⽷]
  \begin{phonetics}{种族灭绝}{zhong3zu2mie4jue2}
    \definition{s.}{genocídio | extinção étnica}
  \end{phonetics}
\end{entry}

\begin{entry}{种麻}{9,11}[Radicais ⽲、⿇]
  \begin{phonetics}{种麻}{zhong3ma2}
    \definition{s.}{planta de cânhamo (feminina)}
  \end{phonetics}
\end{entry}

\begin{entry}{种薯}{9,16}[Radicais ⽲、⾋]
  \begin{phonetics}{种薯}{zhong3shu3}
    \definition{s.}{tubérculo semente}
  \end{phonetics}
\end{entry}

\begin{entry}{科}{9}[Radical ⽲]
  \begin{phonetics}{科}{ke1}[][HSK 2]
    \definition*{s.}{sobrenome Ke}
    \definition{s.}{um ramo de estudo acadêmico ou profissional |uma divisão ou subdivisão de uma unidade administrativa | família | instruções de palco no drama chinês clássico}
  \end{phonetics}
\end{entry}

\begin{entry}{科技}{9,7}[Radicais ⽲、⼿]
  \begin{phonetics}{科技}{ke1 ji4}[][HSK 3]
    \definition{s.}{ciência e tecnologia}
  \end{phonetics}
\end{entry}

\begin{entry}{科学}{9,8}[Radicais ⽲、⼦]
  \begin{phonetics}{科学}{ke1xue2}[][HSK 2]
    \definition{adj.}{científico}
    \definition[门]{s.}{ciência}
  \end{phonetics}
\end{entry}

\begin{entry}{科学家}{9,8,10}[Radicais ⽲、⼦、⼧]
  \begin{phonetics}{科学家}{ke1xue2jia1}
    \definition[个]{s.}{cientista}
  \end{phonetics}
\end{entry}

\begin{entry}{秒}{9}[Radical ⽲]
  \begin{phonetics}{秒}{miao3}
    \definition{adv.}{(coloquial) instantaneamente}
    \definition{s.}{segundo (unidade de tempo) | segundo (unidade de medida angular)}
  \end{phonetics}
\end{entry}

\begin{entry}{穿}{9}[Radical ⽳]
  \begin{phonetics}{穿}{chuan1}[][HSK 1]
    \definition{v.}{vestir}
  \end{phonetics}
\end{entry}

\begin{entry}{穿上}{9,3}[Radicais ⽳、⼀]
  \begin{phonetics}{穿上}{chuan1 shang4}[][HSK 4]
    \definition{v.}{vestir (roupas, etc.); colocar roupas}
  \end{phonetics}
\end{entry}

\begin{entry}{突出}{9,5}[Radicais ⽳、⼐]
  \begin{phonetics}{突出}{tu1chu1}[][HSK 3]
    \definition{adj.}{proeminente; excelente}
    \definition{v.}{romper | enfatizar; destacar; dar destaque a | sobressair; projetar-se; destacar-se}
  \end{phonetics}
\end{entry}

\begin{entry}{突然}{9,12}[Radicais ⽳、⽕]
  \begin{phonetics}{突然}{tu1ran2}[][HSK 3]
    \definition{adj.}{repentino; abrupto; inesperado}
    \definition{adv.}{de repente; abruptamente; inesperadamente}
  \end{phonetics}
\end{entry}

\begin{entry}{类}{9}[Radical ⽶]
  \begin{phonetics}{类}{lei4}[][HSK 3]
    \definition*{s.}{sobrenome Lei}
    \definition{s.}{classe; categoria; tipo; espécie}
    \definition{v.}{assemelhar-se a; ser semelhante a}
  \end{phonetics}
\end{entry}

\begin{entry}{类似}{9,6}[Radicais ⽶、⼈]
  \begin{phonetics}{类似}{lei4si4}[][HSK 3]
    \definition{adj.}{semelhante; análogo}
  \end{phonetics}
\end{entry}

\begin{entry}{类型}{9,9}[Radicais ⽶、⼟]
  \begin{phonetics}{类型}{lei4xing2}[][HSK 4]
    \definition[种,个]{s.}{tipo; espécie; categoria; tipos formados por coisas com características comuns}
  \end{phonetics}
\end{entry}

\begin{entry}{结}{9}[Radical ⽷]
  \begin{phonetics}{结}{jie1}
    \definition{v.}{dar (frutos); formar (sementes); produzir frutos ou sementes (uma planta)}
  \end{phonetics}
  \begin{phonetics}{结}{jie2}[][HSK 4]
    \definition*{s.}{sobrenome Jie}
    \definition{s.}{nó | declaração juramentada; garantia por escrito; documento que, antigamente, significava um reconhecimento de encerramento ou uma garantia de responsabilidade}
    \definition{v.}{amarrar; tricotar; dar nó; tecer | formar; forjar; cimentar; solidificar | resolver; concluir | combinar; formar um relacionamento}
  \end{phonetics}
\end{entry}

\begin{entry}{结合}{9,6}[Radicais ⽷、⼝]
  \begin{phonetics}{结合}{jie2he2}[][HSK 3]
    \definition{v.}{ligar; unir; combinar; integrar | casar-se; unir-se em matrimônio}
  \end{phonetics}
\end{entry}

\begin{entry}{结论}{9,6}[Radicais ⽷、⾔]
  \begin{phonetics}{结论}{jie2lun4}[][HSK 4]
    \definition[个]{s.}{conclusão; palavra final sobre uma pessoa ou coisa após investigação e pesquisa | veredito; julgamento deduzido de premissas também é chamado de conclusão}
  \end{phonetics}
\end{entry}

\begin{entry}{结局}{9,7}[Radicais ⽷、⼫]
  \begin{phonetics}{结局}{jie2ju2}
    \definition{s.}{conclusão | fim | final}
  \end{phonetics}
\end{entry}

\begin{entry}{结束}{9,7}[Radicais ⽷、⽊]
  \begin{phonetics}{结束}{jie2shu4}[][HSK 3]
    \definition{v.}{finalizar; fechar; terminar; concluir; encerrar}
  \end{phonetics}
\end{entry}

\begin{entry}{结束工作}{9,7,3,7}[Radicais ⽷、⽊、⼯、⼈]
  \begin{phonetics}{结束工作}{jie2shu4gong1zuo4}
    \definition{s.}{trabalho final}
    \definition{v.}{terminar o trabalho}
  \end{phonetics}
\end{entry}

\begin{entry}{结束区}{9,7,4}[Radicais ⽷、⽊、⼖]
  \begin{phonetics}{结束区}{jie2shu4 qu1}
    \definition{s.}{zona final}
  \end{phonetics}
\end{entry}

\begin{entry}{结束文本}{9,7,4,5}[Radicais ⽷、⽊、⽂、⽊]
  \begin{phonetics}{结束文本}{jie2shu4 wen2ben3}
    \definition{s.}{texto final}
  \end{phonetics}
\end{entry}

\begin{entry}{结束剂}{9,7,8}[Radicais ⽷、⽊、⼑]
  \begin{phonetics}{结束剂}{jie2shu4 ji4}
    \definition{s.}{finalizador}
  \end{phonetics}
\end{entry}

\begin{entry}{结束语}{9,7,9}[Radicais ⽷、⽊、⾔]
  \begin{phonetics}{结束语}{jie2shu4yu3}
    \definition{s.}{conclusões finais | considerações finais}
  \end{phonetics}
\end{entry}

\begin{entry}{结束辩论}{9,7,16,6}[Radicais ⽷、⽊、⾟、⾔]
  \begin{phonetics}{结束辩论}{jie2shu4 bian4 lun4}
    \definition{s.}{debate de encerramento}
  \end{phonetics}
\end{entry}

\begin{entry}{结社自由}{9,7,6,5}[Radicais ⽷、⽰、⾃、⽥]
  \begin{phonetics}{结社自由}{jie2she4zi4you2}
    \definition{s.}{(constitucional) liberdade de associação}
  \end{phonetics}
\end{entry}

\begin{entry}{结实}{9,8}[Radicais ⽷、⼧]
  \begin{phonetics}{结实}{jie1shi5}[][HSK 3]
    \definition{adj.}{sólido; resistente; durável | forte; resistente; robusto}
  \end{phonetics}
\end{entry}

\begin{entry}{结构}{9,8}[Radicais ⽷、⽊]
  \begin{phonetics}{结构}{jie2gou4}[][HSK 4]
    \definition[个,座]{s.}{estrutura; composição; construção; formação; constituição; tecido; forma; sistematização; mecânica; organização | arquitetura; estrutura; construção; construção de partes de edifícios com suporte de carga ou com carga externa | textura (geológico)}
  \end{phonetics}
\end{entry}

\begin{entry}{结果}{9,8}[Radicais ⽷、⽊]
  \begin{phonetics}{结果}{jie1guo3}
    \definition{v.}{dar frutos}
  \end{phonetics}
  \begin{phonetics}{结果}{jie2guo3}[][HSK 2]
    \definition{s.}{resultado | conclusão}
    \definition{v.}{despachar | matar}
  \end{phonetics}
\end{entry}

\begin{entry}{结婚}{9,11}[Radicais ⽷、⼥]
  \begin{phonetics}{结婚}{jie2hun1}[][HSK 3]
    \definition{v.+compl.}{casar; casar-se}
  \end{phonetics}
\end{entry}

\begin{entry}{结婚礼服}{9,11,5,8}[Radicais ⽷、⼥、⽰、⽉]
  \begin{phonetics}{结婚礼服}{jie2hun1 li3 fu2}
    \definition{s.}{vestido de casamento}
  \end{phonetics}
\end{entry}

\begin{entry}{给}{9}[Radical ⽷]
  \begin{phonetics}{给}{gei3}[][HSK 1]
    \definition{prep.}{a | para}
    \definition{v.}{dar | permitir | fazer alguma coisa (para alguém)}
  \end{phonetics}
  \begin{phonetics}{给}{ji3}
    \definition{v.}{fornecer | prover}
  \end{phonetics}
\end{entry}

\begin{entry}{给……打电话}{9,5,5,8}[Radicais ⽷、⼿、⽥、⾔]
  \begin{phonetics}{给……打电话}{gei3 da3 dian4 hua4}
    \definition{expr.}{telefonar para alguém}
    \seeref{打电话}{da3 dian4 hua4}
  \end{phonetics}
\end{entry}

\begin{entry}{绝不}{9,4}[Radicais ⽷、⼀]
  \begin{phonetics}{绝不}{jue2bu4}
    \definition{adv.}{definitivamente não | de forma alguma | sob nenhuma circunstância}
  \end{phonetics}
\end{entry}

\begin{entry}{绝对}{9,5}[Radicais ⽷、⼨]
  \begin{phonetics}{绝对}{jue2dui4}[][HSK 3]
    \definition{adj.}{absoluto; extremo}
    \definition{adv.}{absolutamente}
  \end{phonetics}
\end{entry}

\begin{entry}{绝招}{9,8}[Radicais ⽷、⼿]
  \begin{phonetics}{绝招}{jue2zhao1}
    \definition{s.}{habilidade única | movimento delicado inesperado (como último recurso) | golpe de mestre | golpe final}
  \end{phonetics}
\end{entry}

\begin{entry}{绝版}{9,8}[Radicais ⽷、⽚]
  \begin{phonetics}{绝版}{jue2ban3}
    \definition{adj.}{esgotado | fora de catálogo}
  \end{phonetics}
\end{entry}

\begin{entry}{罚}{9}[Radical ⽹]
  \begin{phonetics}{罚}{fa2}
    \definition{v.}{castigar | punir}
  \end{phonetics}
\end{entry}

\begin{entry}{罚款}{9,12}[Radicais ⽹、⽋]
  \begin{phonetics}{罚款}{fa2kuan3}
    \definition{s.}{multa (monetária) | pena}
    \definition{v.+compl.}{aplicar uma multa | multar}
  \end{phonetics}
\end{entry}

\begin{entry}{美}{9}[Radical ⽺]
  \begin{phonetics}{美}{mei3}[][HSK 3]
    \definition*{s.}{Abreviatura de América (美洲) | Abreviatura de Estados Unidos da América (美国)}
    \definition{adj.}{lindo; bonito; belo; atraente | satisfatório; bom; agradável}
    \definition{v.}{embelezar; enfeitar | orgulhar-se de; estar satisfeito consigo mesmo}
  \seealsoref{美国}{mei3guo2}
  \seealsoref{美洲}{mei3zhou1}
  \end{phonetics}
\end{entry}

\begin{entry}{美女}{9,3}[Radicais ⽺、⼥]
  \begin{phonetics}{美女}{mei3 nv3}[][HSK 4]
    \definition[个,位]{s.}{beldade; mulher bonita; uma jovem linda}
  \end{phonetics}
\end{entry}

\begin{entry}{美元}{9,4}[Radicais ⽺、⼉]
  \begin{phonetics}{美元}{mei3yuan2}[][HSK 3]
    \definition*[元,笔,沓]{s.}{Dólar Americano}
  \end{phonetics}
\end{entry}

\begin{entry}{美术}{9,5}[Radicais ⽺、⽊]
  \begin{phonetics}{美术}{mei3shu4}[][HSK 3]
    \definition[种]{s.}{arte; belas artes | pintura}
  \end{phonetics}
\end{entry}

\begin{entry}{美甲}{9,5}[Radicais ⽺、⽥]
  \begin{phonetics}{美甲}{mei3jia3}
    \definition{s.}{manicure e/ou pedicure}
  \end{phonetics}
\end{entry}

\begin{entry}{美好}{9,6}[Radicais ⽺、⼥]
  \begin{phonetics}{美好}{mei3 hao3}[][HSK 3]
    \definition{adj.}{bem; feliz; glorioso}
  \end{phonetics}
\end{entry}

\begin{entry}{美丽}{9,7}[Radicais ⽺、⼀]
  \begin{phonetics}{美丽}{mei3li4}[][HSK 3]
    \definition{adj.}{bonito; lindo}
  \end{phonetics}
\end{entry}

\begin{entry}{美味}{9,8}[Radicais ⽺、⼝]
  \begin{phonetics}{美味}{mei3wei4}
    \definition{adj.}{delicioso}
    \definition{s.}{comida deliciosa | delicadeza (\emph{delicacy})}
  \end{phonetics}
\end{entry}

\begin{entry}{美国}{9,8}[Radicais ⽺、⼞]
  \begin{phonetics}{美国}{mei3guo2}
    \definition*{s.}{Estados Unidos da América}
  \end{phonetics}
\end{entry}

\begin{entry}{美国人}{9,8,2}[Radicais ⽺、⼞、⼈]
  \begin{phonetics}{美国人}{mei3guo2ren2}
    \definition{s.}{americano | pessoa ou povo dos Estados Unidos da América}
  \end{phonetics}
\end{entry}

\begin{entry}{美学}{9,8}[Radicais ⽺、⼦]
  \begin{phonetics}{美学}{mei3xue2}
    \definition{s.}{estética}
  \end{phonetics}
\end{entry}

\begin{entry}{美金}{9,8}[Radicais ⽺、⾦]
  \begin{phonetics}{美金}{mei3 jin1}[][HSK 4]
    \definition{s.}{USD; dólar americano: a moeda local dos Estados Unidos}
  \end{phonetics}
\end{entry}

\begin{entry}{美洲}{9,9}[Radicais ⽺、⽔]
  \begin{phonetics}{美洲}{mei3zhou1}
    \definition*{s.}{América (incluindo Norte, Central e Sul)}
  \end{phonetics}
\end{entry}

\begin{entry}{美洲人}{9,9,2}[Radicais ⽺、⽔、⼈]
  \begin{phonetics}{美洲人}{mei3zhou1ren2}
    \definition{s.}{americano | pessoa ou povo do continente Americano}
  \end{phonetics}
\end{entry}

\begin{entry}{美食}{9,9}[Radicais ⽺、⾷]
  \begin{phonetics}{美食}{mei3 shi2}[][HSK 3]
    \definition[种,道,桌]{s.}{iguaria; comida deliciosa}
  \end{phonetics}
\end{entry}

\begin{entry}{耍}{9}[Radical ⽽]
  \begin{phonetics}{耍}{shua3}
    \definition{v.}{brincar com | empunhar | agir (legal, calmo, tranquilo, descolado, etc.) | exibir (uma habilidade, o temperamento de alguém, etc.)}
  \end{phonetics}
\end{entry}

\begin{entry}{耍赖}{9,13}[Radicais ⽽、⾙]
  \begin{phonetics}{耍赖}{shua3lai4}
    \definition{v.}{agir descaradamente | recusar -se a reconhecer que alguém perdeu o jogo ou fez uma promessa, etc. | agir como um idiota | agir como se algo nunca tivesse acontecido}
  \end{phonetics}
\end{entry}

\begin{entry}{耐心}{9,4}[Radicais ⽽、⼼]
  \begin{phonetics}{耐心}{nai4xin1}
    \definition{s.}{paciência}
    \definition{v.}{ser paciente}
  \end{phonetics}
\end{entry}

\begin{entry}{胃口}{9,3}[Radicais ⾁、⼝]
  \begin{phonetics}{胃口}{wei4kou3}
    \definition{s.}{apetite}
  \end{phonetics}
\end{entry}

\begin{entry}{胆小鬼}{9,3,9}[Radicais ⾁、⼩、⿁]
  \begin{phonetics}{胆小鬼}{dan3xiao3gui3}
    \definition{adj.}{covarde | medroso}
  \end{phonetics}
\end{entry}

\begin{entry}{背}{9}[Radical ⾁]
  \begin{phonetics}{背}{bei1}[][HSK 2]
    \definition{v.}{estar sobrecarregado | carregar nas costas ou no ombro}
  \end{phonetics}
  \begin{phonetics}{背}{bei4}[][HSK 3]
    \definition{adv.}{a parte de trás de um corpo ou objeto}
    \definition{s.}{costas | (gíria) azarado}
    \definition{v.}{esconder algo de | decorar | recitar de memória | virar as costas}
  \end{phonetics}
\end{entry}

\begin{entry}{背后}{9,6}[Radicais ⾁、⼝]
  \begin{phonetics}{背后}{bei4 hou4}[][HSK 3]
    \definition{s.}{parte de trás | traseira | nas costas de alguém}
  \end{phonetics}
\end{entry}

\begin{entry}{背景}{9,12}[Radicais ⾁、⽇]
  \begin{phonetics}{背景}{bei4jing3}[][HSK 4]
    \definition[种]{s.}{pano de fundo; fundo; cenário de teatro, filme ou drama de TV | fundo; cenário que permeia a imagem principal na tela | condições sociais; ambientes históricos (significativamente influentes para algo ou alguém) | poder que dá suporte a alguém}
  \end{phonetics}
\end{entry}

\begin{entry}{胖}{9}[Radical ⾁]
  \begin{phonetics}{胖}{pan2}
    \definition{adj.}{saudável}
  \end{phonetics}
  \begin{phonetics}{胖}{pang4}[][HSK 3]
    \definition{adj.}{gordo; robusto; rechonchudo}
  \end{phonetics}
\end{entry}

\begin{entry}{胖子}{9,3}[Radicais ⾁、⼦]
  \begin{phonetics}{胖子}{pang4 zi5}[][HSK 4]
    \definition{s.}{obeso; gordo; pessoa gorda}
  \end{phonetics}
\end{entry}

\begin{entry}{胚}{9}[Radical ⾁]
  \begin{phonetics}{胚}{pei1}
    \definition{s.}{embrião}
  \end{phonetics}
\end{entry}

\begin{entry}{胜}{9}[Radical ⾁]
  \begin{phonetics}{胜}{sheng4}[][HSK 3]
    \definition{adj.}{soberbo; maravilhoso; adorável}
    \definition[场]{s.}{vitória; sucesso | penteado de mulher}
    \definition{v.}{ganhar; derrotar; vencer; ter sucesso | superar; ser superior a; levar a melhor sobre | ser igual a; poder suportar}
  \end{phonetics}
\end{entry}

\begin{entry}{胜利}{9,7}[Radicais ⾁、⼑]
  \begin{phonetics}{胜利}{sheng4li4}[][HSK 3]
    \definition{adv.}{com sucesso; triunfantemente}
    \definition[场,个]{s.}{vitória; triunfo; sucesso}
    \definition{v.}{ganhar; vencer; triunfar; ter sucesso}
  \end{phonetics}
\end{entry}

\begin{entry}{胜算}{9,14}[Radicais ⾁、⽵]
  \begin{phonetics}{胜算}{sheng4suan4}
    \definition{s.}{probabilidade de sucesso | estratégia que garante o sucesso}
    \definition{v.}{ter certeza do sucesso}
  \end{phonetics}
\end{entry}

\begin{entry}{胡萝卜}{9,11,2}[Radicais ⾁、⾋、⼘]
  \begin{phonetics}{胡萝卜}{hu2luo2bo5}
    \definition{s.}{cenoura}
  \end{phonetics}
\end{entry}

\begin{entry}{范围}{9,7}[Radicais ⾋、⼞]
  \begin{phonetics}{范围}{fan4wei2}[][HSK 3]
    \definition[个]{s.}{escopo; limite; alcance}
    \definition{v.}{estabelecer limites para; limitar o escopo de}
  \end{phonetics}
\end{entry}

\begin{entry}{茶}{9}[Radical ⾋]
  \begin{phonetics}{茶}{cha2}[][HSK 1]
    \definition[杯,壶]{s.}{chá | pé (planta) de chá}
  \end{phonetics}
\end{entry}

\begin{entry}{茶叶}{9,5}[Radicais ⾋、⼝]
  \begin{phonetics}{茶叶}{cha2 ye4}[][HSK 4]
    \definition[盒,罐,包,片]{s.}{chá; folhas de chá; as folhas jovens da planta do chá que são processadas para produzir bebidas}
  \end{phonetics}
\end{entry}

\begin{entry}{草}{9}[Radical ⾋]
  \begin{phonetics}{草}{cao3}[][HSK 2]
    \definition[棵,撮,株,根]{s.}{erva | grama}
  \end{phonetics}
\end{entry}

\begin{entry}{草地}{9,6}[Radicais ⾋、⼟]
  \begin{phonetics}{草地}{cao3 di4}[][HSK 2]
    \definition[片]{s.}{relva | pastagem}
  \end{phonetics}
\end{entry}

\begin{entry}{草纸}{9,7}[Radicais ⾋、⽷]
  \begin{phonetics}{草纸}{cao3zhi3}
    \definition{s.}{papel pardo | pergaminho | papel de palha áspero | papel higiênico}
  \end{phonetics}
\end{entry}

\begin{entry}{草莓}{9,10}[Radicais ⾋、⾋]
  \begin{phonetics}{草莓}{cao3mei2}
    \definition[颗]{s.}{morango}
  \end{phonetics}
\end{entry}

\begin{entry}{荒芜}{9,7}[Radicais ⾋、⾋]
  \begin{phonetics}{荒芜}{huang1wu2}
    \definition{adj.}{estéril}
  \end{phonetics}
\end{entry}

\begin{entry}{荔枝}{9,8}[Radicais ⾋、⽊]
  \begin{phonetics}{荔枝}{li4zhi1}
    \definition{s.}{lichia}
  \end{phonetics}
\end{entry}

\begin{entry}{药}{9}[Radical ⾋]
  \begin{phonetics}{药}{yao4}[][HSK 2]
    \definition[种,服,味]{s.}{medicamento | remédio | droga}
  \end{phonetics}
\end{entry}

\begin{entry}{药丸}{9,3}[Radicais ⾋、⼂]
  \begin{phonetics}{药丸}{yao4wan2}
    \definition[粒]{s.}{pílula}
  \end{phonetics}
\end{entry}

\begin{entry}{药水}{9,4}[Radicais ⾋、⽔]
  \begin{phonetics}{药水}{yao4 shui3}[][HSK 2]
    \definition{s.}{remédio engarrafado | loção | medicamento em forma líquida}
  \end{phonetics}
\end{entry}

\begin{entry}{药片}{9,4}[Radicais ⾋、⽚]
  \begin{phonetics}{药片}{yao4 pian4}[][HSK 2]
    \definition[片]{s.}{uma pílula ou comprimido (remédio)}
  \end{phonetics}
\end{entry}

\begin{entry}{药补}{9,7}[Radicais ⾋、⾐]
  \begin{phonetics}{药补}{yao4bu3}
    \definition{s.}{suplemento dietético medicinal que ajuda a melhorar a saúde}
  \end{phonetics}
\end{entry}

\begin{entry}{药典}{9,8}[Radicais ⾋、⼋]
  \begin{phonetics}{药典}{yao4dian3}
    \definition{s.}{farmacopéia}
  \end{phonetics}
\end{entry}

\begin{entry}{药店}{9,8}[Radicais ⾋、⼴]
  \begin{phonetics}{药店}{yao4 dian4}[][HSK 2]
    \definition{s.}{farmácia | drogaria | loja de produtos químicos}
  \end{phonetics}
\end{entry}

\begin{entry}{药房}{9,8}[Radicais ⾋、⼾]
  \begin{phonetics}{药房}{yao4fang2}
    \definition{s.}{farmácia | drogaria}
  \end{phonetics}
\end{entry}

\begin{entry}{药品}{9,9}[Radicais ⾋、⼝]
  \begin{phonetics}{药品}{yao4pin3}
    \definition{s.}{medicamento | remédio | droga}
  \end{phonetics}
\end{entry}

\begin{entry}{药签}{9,13}[Radicais ⾋、⽵]
  \begin{phonetics}{药签}{yao4qian1}
    \definition{s.}{cotonete médico}
  \end{phonetics}
\end{entry}

\begin{entry}{药膳}{9,16}[Radicais ⾋、⾁]
  \begin{phonetics}{药膳}{yao4shan4}
    \definition{s.}{dieta medicinal}
  \end{phonetics}
\end{entry}

\begin{entry}{药罐}{9,23}[Radicais ⾋、⽸]
  \begin{phonetics}{药罐}{yao4guan4}
    \definition{s.}{frasco de remédio}
  \end{phonetics}
\end{entry}

\begin{entry}{虽}{9}[Radical ⾍]
  \begin{phonetics}{虽}{sui1}
    \definition{conj.}{no entanto | embora | mesmo se/embora}
  \end{phonetics}
\end{entry}

\begin{entry}{虽然}{9,12}[Radicais ⾍、⽕]
  \begin{phonetics}{虽然}{sui1 ran2}[][HSK 2]
    \definition{conj.}{embora (frequentemente usado correlativamente com 可是, 但是, etc); geralmente é usado no início de uma frase para indicar que o fato anterior foi reconhecido, mas não mudará o que acontecerá em seguida}
  \seealsoref{但是}{dan4 shi4}
  \seealsoref{可是}{ke3shi4}
  \end{phonetics}
\end{entry}

\begin{entry}{虾}{9}[Radical ⾍]
  \begin{phonetics}{虾}{xia1}
    \definition{s.}{camarão}
  \end{phonetics}
\end{entry}

\begin{entry}{蚂蚁}{9,9}[Radicais ⾍、⾍]
  \begin{phonetics}{蚂蚁}{ma3yi3}
    \definition{s.}{formiga}
  \end{phonetics}
\end{entry}

\begin{entry}{要}{9}[Radical ⾑]
  \begin{phonetics}{要}{yao1}[][HSK 1]
    \definition{v.}{(forma ligada) demandar, coagir}
  \end{phonetics}
  \begin{phonetics}{要}{yao4}
    \definition{adj.}{(forma ligada) importante}
    \definition{conj.}{se (o mesmo que  要是)}
    \definition{v.}{querer | precisar | pedir por | precisar de}
  \seealsoref{要是}{yao4shi5}
  \end{phonetics}
\end{entry}

\begin{entry}{要么……要么……}{9,3,9,3}[Radicais ⾑、⼃、⾑、⼃]
  \begin{phonetics}{要么……要么……}{yao4me5 yao4me5}
    \definition{conj.}{ou\dots ou\dots}
  \end{phonetics}
\end{entry}

\begin{entry}{要义}{9,3}[Radicais ⾑、⼂]
  \begin{phonetics}{要义}{yao4yi4}
    \definition{s.}{resumo | o essencial}
  \end{phonetics}
\end{entry}

\begin{entry}{要不}{9,4}[Radicais ⾑、⼀]
  \begin{phonetics}{要不}{yao4bu4}
    \definition{conj.}{de outra forma | se não | outro | ou}
  \end{phonetics}
\end{entry}

\begin{entry}{要不然}{9,4,12}[Radicais ⾑、⼀、⽕]
  \begin{phonetics}{要不然}{yao4bu4ran2}
    \definition{conj.}{de outra forma | se não | outro | ou}
  \end{phonetics}
\end{entry}

\begin{entry}{要好}{9,6}[Radicais ⾑、⼥]
  \begin{phonetics}{要好}{yao4hao3}
    \definition{v.}{ser amigos íntimos | estar em boas condições}
  \end{phonetics}
\end{entry}

\begin{entry}{要死}{9,6}[Radicais ⾑、⽍]
  \begin{phonetics}{要死}{yao4si3}
    \definition{adv.}{extremamente | muito}
  \end{phonetics}
\end{entry}

\begin{entry}{要求}{9,7}[Radicais ⾑、⽔]
  \begin{phonetics}{要求}{yao1qiu2}[][HSK 2]
    \definition[点]{s.}{requerimento}
    \definition{v.}{pedir | exigir | solicitar | fazer uma reivindicação}
  \end{phonetics}
\end{entry}

\begin{entry}{要挟}{9,9}[Radicais ⾑、⼿]
  \begin{phonetics}{要挟}{yao1xie2}
    \definition{v.}{chantagear | ameaçar}
  \end{phonetics}
\end{entry}

\begin{entry}{要是}{9,9}[Radicais ⾑、⽇]
  \begin{phonetics}{要是}{yao4shi5}[][HSK 3]
    \definition{conj.}{se; no caso; orações de conexão, expressando relações hipotéticas, equivalentes a "se", podem ser usadas com "então"}
  \end{phonetics}
\end{entry}

\begin{entry}{要是……的话}{9,9,8,8}[Radicais ⾑、⽇、⽩、⾔]
  \begin{phonetics}{要是……的话}{yao4shi5 de5hua4}[][HSK 2,3]
    \definition{conj.}{se\dots no caso de}
  \end{phonetics}
\end{entry}

\begin{entry}{要点}{9,9}[Radicais ⾑、⽕]
  \begin{phonetics}{要点}{yao4dian3}
    \definition{s.}{pontos principais | essencial}
  \end{phonetics}
\end{entry}

\begin{entry}{要谎}{9,11}[Radicais ⾑、⾔]
  \begin{phonetics}{要谎}{yao4huang3}
    \definition{v.}{pedir um preço enorme (como primeiro passo de negociação)}
  \end{phonetics}
\end{entry}

\begin{entry}{要强}{9,12}[Radicais ⾑、⼸]
  \begin{phonetics}{要强}{yao4qiang2}
    \definition{adj.}{ansioso para se destacar | ansioso para progredir na vida | obstinado}
  \end{phonetics}
\end{entry}

\begin{entry}{觉得}{9,11}[Radicais ⾒、⼻]
  \begin{phonetics}{觉得}{jue2de5}[][HSK 1]
    \definition{v.}{pensar que\dots | sentir que\dots | sentir (desconfortável, etc.)}
  \end{phonetics}
\end{entry}

\begin{entry}{语}{9}[Radical ⾔]
  \begin{phonetics}{语}{yu3}
    \definition{s.}{língua; linguagem | dito; provérbio; refere-se especialmente a coloquialismos, provérbios, expressões idiomáticas ou palavras de livros antigos | sinal; meio não linguístico de comunicar ideias ; ações ou sinais que substituem palavras para expressar significado | palavras; expressão; refere-se a uma palavra, frase ou sentença}
    \definition{v.}{dizer; falar | (de pássaros, insetos, etc.) gorjear; pipilar}
  \end{phonetics}
  \begin{phonetics}{语}{yu4}
    \definition{v.}{contar; informar}
  \end{phonetics}
\end{entry}

\begin{entry}{语气}{9,4}[Radicais ⾔、⽓]
  \begin{phonetics}{语气}{yu3qi4}
    \definition[个]{s.}{maneira de falar | tom}
  \end{phonetics}
\end{entry}

\begin{entry}{语言}{9,7}[Radicais ⾔、⾔]
  \begin{phonetics}{语言}{yu3yan2}[][HSK 2]
    \definition[门,种]{s.}{linguagem | língua}
  \end{phonetics}
\end{entry}

\begin{entry}{语言实验室}{9,7,8,10,9}[Radicais ⾔、⾔、⼧、⾺、⼧]
  \begin{phonetics}{语言实验室}{yu3yan2shi2yan4shi4}
    \definition{s.}{laboratório de línguas}
  \end{phonetics}
\end{entry}

\begin{entry}{语法}{9,8}[Radicais ⾔、⽔]
  \begin{phonetics}{语法}{yu3fa3}
    \definition{s.}{gramática}
  \end{phonetics}
\end{entry}

\begin{entry}{语法术语}{9,8,5,9}[Radicais ⾔、⽔、⽊、⾔]
  \begin{phonetics}{语法术语}{yu3fa3shu4yu3}
    \definition{s.}{termo gramatical}
  \end{phonetics}
\end{entry}

\begin{entry}{语调}{9,10}[Radicais ⾔、⾔]
  \begin{phonetics}{语调}{yu3diao4}
    \definition[个]{s.}{entonação}
  \end{phonetics}
\end{entry}

\begin{entry}{误点}{9,9}[Radicais ⾔、⽕]
  \begin{phonetics}{误点}{wu4dian3}
    \definition{v.+compl.}{atrasar | chegar tarde}
  \end{phonetics}
\end{entry}

\begin{entry}{诱人}{9,2}[Radicais ⾔、⼈]
  \begin{phonetics}{诱人}{you4ren2}
    \definition{adj.}{atraente | cativante}
  \end{phonetics}
\end{entry}

\begin{entry}{说}{9}[Radical ⾔]
  \begin{phonetics}{说}{shui4}
    \definition{v.}{persuadir}
  \end{phonetics}
  \begin{phonetics}{说}{shuo1}[][HSK 1]
    \definition{s.}{uma teoria (normalmente o último caractere, como em 日心说, teoria heliocêntrica)}
    \definition{v.}{falar | dizer | explicar | contar}
  \end{phonetics}
\end{entry}

\begin{entry}{说好}{9,6}[Radicais ⾔、⼥]
  \begin{phonetics}{说好}{shuo1hao3}
    \definition{v.}{chegar a um acordo | concluir negociações}
  \end{phonetics}
\end{entry}

\begin{entry}{说完}{9,7}[Radicais ⾔、⼧]
  \begin{phonetics}{说完}{shuo1-wan2}
    \definition{expr.}{acabar/terminar palavras}
  \end{phonetics}
\end{entry}

\begin{entry}{说明}{9,8}[Radicais ⾔、⽇]
  \begin{phonetics}{说明}{shuo1ming2}[][HSK 2]
    \definition[本,个]{s.}{legenda | instrução | explicação}
    \definition{v.}{mostrar | explicar | ilustrar | indicar | provar | demonstrar}
  \end{phonetics}
\end{entry}

\begin{entry}{说话}{9,8}[Radicais ⾔、⾔]
  \begin{phonetics}{说话}{shuo1 hua4}[][HSK 1]
    \definition{adv.}{imediatamente | em um minuto}
    \definition{v.}{falar | dizer | bater-papo | conversar | fofocar}
  \end{phonetics}
\end{entry}

\begin{entry}{说理}{9,11}[Radicais ⾔、⽟]
  \begin{phonetics}{说理}{shuo1li3}
    \definition{v.}{racionalizar | discutir logicamente}
  \end{phonetics}
\end{entry}

\begin{entry}{说谎}{9,11}[Radicais ⾔、⾔]
  \begin{phonetics}{说谎}{shuo1huang3}
    \definition{v.+compl.}{mentir | contar uma mentira}
  \end{phonetics}
\end{entry}

\begin{entry}{贵}{9}[Radical ⾙]
  \begin{phonetics}{贵}[⻉]{gui4}[中一⻉][HSK 1]
    \definition{adj.}{caro | nobre | precioso}
  \end{phonetics}
\end{entry}

\begin{entry}{贵姓}{9,8}[Radicais ⾙、⼥]
  \begin{phonetics}{贵姓}{gui4xing4}
    \definition{expr.}{qual seu sobrenome?}
  \end{phonetics}
\end{entry}

\begin{entry}{贸易}{9,8}[Radicais ⾙、⽇]
  \begin{phonetics}{贸易}{mao4yi4}
    \definition[个]{s.}{transação comercial}
    \definition{v.}{fazer uma transação comercial}
  \end{phonetics}
\end{entry}

\begin{entry}{费}{9}[Radical ⾙]
  \begin{phonetics}{费}{fei4}[][HSK 3]
    \definition*{s.}{Fei}
    \definition{s.}{taxa; despesa; encargo}
    \definition{v.}{custar; gastar; desperdiçar}
  \end{phonetics}
\end{entry}

\begin{entry}{费用}{9,5}[Radicais ⾙、⽤]
  \begin{phonetics}{费用}{fei4 yong4}[][HSK 3]
    \definition[笔,个]{s.}{custo; despesa; desembolso}
  \end{phonetics}
\end{entry}

\begin{entry}{贺}{9}[Radical ⾙]
  \begin{phonetics}{贺}{he4}
    \definition*{s.}{sobrenome He}
    \definition{v.}{parabenizar | congratular}
  \end{phonetics}
\end{entry}

\begin{entry}{轴承}{9,8}[Radicais ⾞、⼿]
  \begin{phonetics}{轴承}{zhou2cheng2}
    \definition{s.}{(mecânico) rolamento}
  \end{phonetics}
\end{entry}

\begin{entry}{轻}{9}[Radical ⾞]
  \begin{phonetics}{轻}{qing1}[][HSK 2]
    \definition{adj.}{leve | pequeno em número, grau, etc. | não importante | relaxado}
    \definition{adv.}{suavemente | levemente | precipitadamente}
    \definition{v.}{menosprezar}
  \end{phonetics}
\end{entry}

\begin{entry}{轻易}{9,8}[Radicais ⾞、⽇]
  \begin{phonetics}{轻易}{qing1yi4}[][HSK 4]
    \definition{adj.}{fácil; simples}
    \definition{adv.}{facilmente; prontamente | facilmente; precipitadamente; indica que uma ação é realizada casualmente, geralmente usado em frases negativas}
  \end{phonetics}
\end{entry}

\begin{entry}{轻松}{9,8}[Radicais ⾞、⽊]
  \begin{phonetics}{轻松}{qing1song1}[][HSK 4]
    \definition{adj.}{leve; relaxado; livre de fardos; não se sentir nervoso ou cansado}
    \definition{v.}{relaxar; levar as coisas menos a sério}
  \end{phonetics}
\end{entry}

\begin{entry}{迷}{9}[Radical ⾡]
  \begin{phonetics}{迷}{mi2}[][HSK 3]
    \definition*{s.}{sobrenome Mi}
    \definition{adj.}{perdido; confuso}
    \definition{s.}{fã; entusiasta; fanático}
    \definition{v.}{estar confuso; perder o rumo; se perder-se | ficar fascinado por; entregar-se a; ficar encantado com (por); ser louco por | confundir; desorientar; fascinar; encantar}
  \end{phonetics}
\end{entry}

\begin{entry}{迷人}{9,2}[Radicais ⾡、⼈]
  \begin{phonetics}{迷人}{mi2ren2}
    \definition{adj.}{fascinante | encantador | tentador}
  \end{phonetics}
\end{entry}

\begin{entry}{迷你}{9,7}[Radicais ⾡、⼈]
  \begin{phonetics}{迷你}{mi2ni3}
    \definition{adj.}{(empréstimo linguístico) mini, como em minissaia ou \emph{Mini Cooper}}
  \end{phonetics}
\end{entry}

\begin{entry}{迷宫}{9,9}[Radicais ⾡、⼧]
  \begin{phonetics}{迷宫}{mi2gong1}
    \definition{s.}{labirinto}
  \end{phonetics}
\end{entry}

\begin{entry}{迷恋}{9,10}[Radicais ⾡、⼼]
  \begin{phonetics}{迷恋}{mi2lian4}
    \definition{adj.}{obcecado}
    \definition{v.}{estar/ser apaixonado por | ficar encantado por | estar/ser obcecado por}
  \end{phonetics}
\end{entry}

\begin{entry}{迷路}{9,13}[Radicais ⾡、⾜]
  \begin{phonetics}{迷路}{mi2lu4}
    \definition{s.}{labirinto | ouvido interno}
    \definition{v.+compl.}{perder o caminho | perder-se | seguir pelo caminho errado | não conseguir encontrar o caminho}
  \end{phonetics}
\end{entry}

\begin{entry}{追}{9}[Radical ⾡]
  \begin{phonetics}{追}{zhui1}[][HSK 3]
    \definition*{s.}{sobrenome Zhui}
    \definition{v.}{perseguir; correr atrás; ir atrás de; alcançar | rastrear; investigar; chegar ao fundo de | ansiar por (depois); ir atrás; procurar | recordar; relembrar; lembrar | agir retroativamente; fazer postumamente}
  \end{phonetics}
\end{entry}

\begin{entry}{追赶}{9,10}[Radicais ⾡、⾛]
  \begin{phonetics}{追赶}{zhui1gan3}
    \definition{v.}{perseguir | acelerar | alcançar | ultrapassar}
  \end{phonetics}
\end{entry}

\begin{entry}{退}{9}[Radical ⾡]
  \begin{phonetics}{退}{tui4}[][HSK 3]
    \definition{v.}{recuar; mover-se para trás | fazer recuar; remover; retirar | desistir; retirar-se de | retroceder; refluir; declinar | desaparecer; desvanecer | devolver; retornar | cancelar; rescindir; romper}
  \end{phonetics}
\end{entry}

\begin{entry}{退出}{9,5}[Radicais ⾡、⼐]
  \begin{phonetics}{退出}{tui4 chu1}[][HSK 3]
    \definition{v.}{desistir; retirar-se; separar-se; retirar-se de}
  \end{phonetics}
\end{entry}

\begin{entry}{退休}{9,6}[Radicais ⾡、⼈]
  \begin{phonetics}{退休}{tui4xiu1}[][HSK 3]
    \definition{v.+compl.}{aposentar-se}
  \end{phonetics}
\end{entry}

\begin{entry}{送}{9}[Radical ⾡]
  \begin{phonetics}{送}{song4}[][HSK 1]
    \definition{v.}{distribuir | entregar | dar | oferecer (alguma coisa como presente) | enviar | remeter}
  \end{phonetics}
\end{entry}

\begin{entry}{送到}{9,8}[Radicais ⾡、⼑]
  \begin{phonetics}{送到}{song4 dao4}[][HSK 2]
    \definition{v.}{enviar para (lugar)}
  \end{phonetics}
\end{entry}

\begin{entry}{送给}{9,9}[Radicais ⾡、⽷]
  \begin{phonetics}{送给}{song4 gei3}[][HSK 2]
    \definition{v.}{dar a (alguém ou organização)}
  \end{phonetics}
\end{entry}

\begin{entry}{适用}{9,5}[Radicais ⾡、⽤]
  \begin{phonetics}{适用}{shi4 yong4}[][HSK 3]
    \definition{adj.}{adequado; aplicável}
    \definition{v.}{ser aplicável; ser adequado}
  \end{phonetics}
\end{entry}

\begin{entry}{适合}{9,6}[Radicais ⾡、⼝]
  \begin{phonetics}{适合}{shi4he2}[][HSK 3]
    \definition{v.}{servir (uma roupa); caber; se adequar}
  \end{phonetics}
\end{entry}

\begin{entry}{适应}{9,7}[Radicais ⾡、⼴]
  \begin{phonetics}{适应}{shi4ying4}[][HSK 3]
    \definition{v.}{ajustar-se; adequar-se; adaptar-se}
  \end{phonetics}
\end{entry}

\begin{entry}{逃}{9}[Radical ⾡]
  \begin{phonetics}{逃}{tao2}
    \definition{v.}{escapar | fugir}
  \end{phonetics}
\end{entry}

\begin{entry}{逆境}{9,14}[Radicais ⾡、⼟]
  \begin{phonetics}{逆境}{ni4jing4}
    \definition{s.}{adversidade | tribulação}
  \end{phonetics}
\end{entry}

\begin{entry}{选}{9}[Radical ⾡]
  \begin{phonetics}{选}{xuan3}[][HSK 2]
    \definition{s.}{seleções | antologia}
    \definition{v.}{selecionar | escolher | eleger}
  \end{phonetics}
\end{entry}

\begin{entry}{选手}{9,4}[Radicais ⾡、⼿]
  \begin{phonetics}{选手}{xuan3shou3}[][HSK 3]
    \definition[位]{s.}{jogador; competidor (selecionado); atleta selecionado para uma competição esportiva}
  \end{phonetics}
\end{entry}

\begin{entry}{选择}{9,8}[Radicais ⾡、⼿]
  \begin{phonetics}{选择}{xuan3ze2}
    \definition{s.}{escolha | opção | alternativa}
    \definition{v.}{selecionar | escolher}
  \end{phonetics}
\end{entry}

\begin{entry}{重}{9}[Radical ⾥]
  \begin{phonetics}{重}{chong2}
    \definition*{s.}{sobrenome Chong}
    \definition{adv.}{novamente; mais uma vez}
    \definition{clas.}{para camadas}
    \definition{v.}{repetir; duplicar}
  \end{phonetics}
  \begin{phonetics}{重}{zhong4}[][HSK 1,3]
    \definition{adj.}{pesado | profundo; sério | importante; momentoso | discreto; prudente | considerável em quantidade ou valor}
    \definition{adv.}{pesadamente; severamente}
    \definition{v.}{colocar (pôr) ênfase em; dar valor a; atribuir importância a}
  \end{phonetics}
\end{entry}

\begin{entry}{重大}{9,3}[Radicais ⾥、⼤]
  \begin{phonetics}{重大}{zhong4da4}[][HSK 3]
    \definition{adj.}{grande; importante; significativo; de grande importância}
  \end{phonetics}
\end{entry}

\begin{entry}{重阳节}{9,6,5}[Radicais ⾥、⾩、⾋]
  \begin{phonetics}{重阳节}{chong2yang2jie2}
    \definition*{s.}{Festa do Duplo Nove, Festival Yang, dia de subir aos lugares mais altos para evitar calamidades e dia do culto aos antepassados (9º dia do nono mês lunar)}
  \end{phonetics}
\end{entry}

\begin{entry}{重视}{9,8}[Radicais ⾥、⾒]
  \begin{phonetics}{重视}{zhong4shi4}[][HSK 2]
    \definition{v.}{atribuir valor a | dar peso a | atribuir importância a | prestar atenção a}
  \end{phonetics}
\end{entry}

\begin{entry}{重复}{9,9}[Radicais ⾥、⼢]
  \begin{phonetics}{重复}{chong2fu4}[][HSK 2]
    \definition{v.}{repetir | iterar | duplicar | reduplicar | fazer algo de novo}
  \end{phonetics}
\end{entry}

\begin{entry}{重点}{9,9}[Radicais ⾥、⽕]
  \begin{phonetics}{重点}{chong2dian3}
    \definition{adj./adv./s.}{nota-chave | ponto-chave | ponto focal | ênfase}
  \end{phonetics}
  \begin{phonetics}{重点}{zhong4dian3}[][HSK 2]
    \definition{s.}{nota-chave | ponto-chave | ponto focal | ênfase}
  \end{phonetics}
\end{entry}

\begin{entry}{重要}{9,9}[Radicais ⾥、⾑]
  \begin{phonetics}{重要}{zhong4yao4}[][HSK 1]
    \definition{adj.}{importante | significativo | principal}
  \end{phonetics}
\end{entry}

\begin{entry}{重重}{9,9}[Radicais ⾥、⾥]
  \begin{phonetics}{重重}{chong2chong2}
    \definition{adv.}{camada após camada | um após o outro}
  \end{phonetics}
  \begin{phonetics}{重重}{zhong4zhong4}
    \definition{adv.}{fortemente | severamente}
  \end{phonetics}
\end{entry}

\begin{entry}{重逢}{9,10}[Radicais ⾥、⾡]
  \begin{phonetics}{重逢}{chong2feng2}
    \definition{s.}{reunião}
    \definition{v.}{encontrar-se novamente | reunir-se}
  \end{phonetics}
\end{entry}

\begin{entry}{重量}{9,12}[Radicais ⾥、⾥]
  \begin{phonetics}{重量}{zhong4liang4}
    \definition[个]{s.}{peso}
  \end{phonetics}
\end{entry}

\begin{entry}{重新}{9,13}[Radicais ⾥、⽄]
  \begin{phonetics}{重新}{chong2xin1}[][HSK 2]
    \definition{adv.}{de novo | novamente}
  \end{phonetics}
\end{entry}

\begin{entry}{钟}{9}[Radical ⾦]
  \begin{phonetics}{钟}{zhong1}[][HSK 3]
    \definition*{s.}{sobrenome Zhong}
    \definition[顶,个,口]{s.}{sino; campainha; um chocalho feito de cobre ou ferro | relógio; temporizador | tempo; refere-se à hora, ao tempo | copo sem alça; xícara sem alça}
    \definition{v.}{concentrar (as afeições de alguém, etc.)}
  \end{phonetics}
\end{entry}

\begin{entry}{钟室}{9,9}[Radicais ⾦、⼧]
  \begin{phonetics}{钟室}{zhong1shi4}
    \definition{s.}{campanário | sala do relógio}
  \end{phonetics}
\end{entry}

\begin{entry}{钟罩}{9,13}[Radicais ⾦、⽹]
  \begin{phonetics}{钟罩}{zhong1zhao4}
    \definition{s.}{redoma | dossel de sino}
  \end{phonetics}
\end{entry}

\begin{entry}{钢}{9}[Radical ⾦]
  \begin{phonetics}{钢}{gang1}
    \definition{s.}{aço}
  \end{phonetics}
\end{entry}

\begin{entry}{钢丝}{9,5}[Radicais ⾦、⼀]
  \begin{phonetics}{钢丝}{gang1si1}
    \definition{s.}{cabo de aço | corda bamba}
  \end{phonetics}
\end{entry}

\begin{entry}{钢琴}{9,12}[Radicais ⾦、⽟]
  \begin{phonetics}{钢琴}{gang1qin2}
    \definition[架,台]{s.}{piano}
  \end{phonetics}
\end{entry}

\begin{entry}{钥匙}{9,11}[Radicais ⾦、⼔]
  \begin{phonetics}{钥匙}{yao4shi5}
    \definition[把]{s.}{chave}
  \end{phonetics}
\end{entry}

\begin{entry}{钥匙孔}{9,11,4}[Radicais ⾦、⼔、⼦]
  \begin{phonetics}{钥匙孔}{yao4shi5kong3}
    \definition{s.}{buraco da fechadura}
  \end{phonetics}
\end{entry}

\begin{entry}{钥匙卡}{9,11,5}[Radicais ⾦、⼔、⼘]
  \begin{phonetics}{钥匙卡}{yao4shi5ka3}
    \definition{s.}{cartão de acesso}
  \end{phonetics}
\end{entry}

\begin{entry}{钥匙洞孔}{9,11,9,4}[Radicais ⾦、⼔、⽔、⼦]
  \begin{phonetics}{钥匙洞孔}{yao4shi5dong4kong3}
    \definition{s.}{buraco da fechadura}
  \end{phonetics}
\end{entry}

\begin{entry}{钥匙圈}{9,11,11}[Radicais ⾦、⼔、⼞]
  \begin{phonetics}{钥匙圈}{yao4shi5quan1}
    \definition{s.}{chaveiro}
  \end{phonetics}
\end{entry}

\begin{entry}{钩}{9}[Radical ⾦]
  \begin{phonetics}{钩}{gou1}
    \definition{s.}{gancho | \emph{check mark} | \emph{tick}}
    \definition{v.}{enganchar | costurar}
  \end{phonetics}
\end{entry}

\begin{entry}{闻}{9}[Radical ⾨]
  \begin{phonetics}{闻}{wen2}[][HSK 2]
    \definition*{s.}{sobrenome Wen}
    \definition{s.}{notícias | reputação | fama}
    \definition{v.}{ouvir | cheirar | farejar}
  \end{phonetics}
\end{entry}

\begin{entry}{阁下}{9,3}[Radicais ⾨、⼀]
  \begin{phonetics}{阁下}{ge2xia4}
    \definition{pron.}{Sua Excelência | Sua Majestade | \emph{Sire}}
  \end{phonetics}
\end{entry}

\begin{entry}{院}{9}[Radical ⾩]
  \begin{phonetics}{院}{yuan4}[][HSK 2]
    \definition[个]{s.}{pátio | instituição}
  \end{phonetics}
\end{entry}

\begin{entry}{院子}{9,3}[Radicais ⾩、⼦]
  \begin{phonetics}{院子}{yuan4zi5}[][HSK 2]
    \definition[个]{s.}{pátio | jardim | quintal}
  \end{phonetics}
\end{entry}

\begin{entry}{院长}{9,4}[Radicais ⾩、⾧]
  \begin{phonetics}{院长}{yuan4zhang3}[][HSK 2]
    \definition[个]{s.}{presidente de um conselho | reitor | chefe de departamento | primeiro-ministro da República da China | presidente de uma universidade}
  \end{phonetics}
\end{entry}

\begin{entry}{除了}{9,2}[Radicais ⾩、⼅]
  \begin{phonetics}{除了}{chu2le5}[][HSK 3]
    \definition{prep.}{exceto; à parte | além disso; além de | ou \dots ou \dots}
  \end{phonetics}
\end{entry}

\begin{entry}{除非}{9,8}[Radicais ⾩、⾮]
  \begin{phonetics}{除非}{chu2fei1}
    \definition{conj.}{a menos que | somente se}
  \end{phonetics}
\end{entry}

\begin{entry}{面}{9}[Kangxi 176][Radical ⾯]
  \begin{phonetics}{面}{mian4}[][HSK 2]
    \definition{clas.}{para objetos com superfície plana como tambores, espelhos, bandeiras, etc.}
    \definition{s.}{farinha | massa | (gíria) (uma pessoa) ineficaz | face | superfície | lado | lado de fora}
  \end{phonetics}
\end{entry}

\begin{entry}{面包}{9,5}[Radicais ⾯、⼓]
  \begin{phonetics}{面包}{mian4bao1}[][HSK 1]
    \definition[个,片,袋,块]{s.}{pão}
    \example{我买八个面包了。}[Comprei oito pães.]
    \example{他在吃两片面包。}[Ele está comendo duas fatias de pão.]
    \example{我在家里带了一袋面包。}[Trouxe um saco de pão para casa.]
    \example{我拿了一块面包。}[Peguei um pedaço de pão.]
  \end{phonetics}
\end{entry}

\begin{entry}{面对}{9,5}[Radicais ⾯、⼨]
  \begin{phonetics}{面对}{mian4dui4}[][HSK 3]
    \definition{v.}{enfrentar; defrontar | confrontar (problema)}
  \end{phonetics}
\end{entry}

\begin{entry}{面对面}{9,5,9}[Radicais ⾯、⼨、⾯]
  \begin{phonetics}{面对面}{mian4dui4mian4}
    \definition{expr.}{cara a cara}
  \end{phonetics}
\end{entry}

\begin{entry}{面对面吃面}{9,5,9,6,9}[Radicais ⾯、⼨、⾯、⼝、⾯]
  \begin{phonetics}{面对面吃面}{mian4dui4mian4 chi1 mian4}
    \definition{expr.}{Comer macarrão cara a cara; indica que o seu estado atual, ou algumas das posições em que você está, ou algumas das coisas que você fez são muito claras}
  \end{phonetics}
\end{entry}

\begin{entry}{面团}{9,6}[Radicais ⾯、⼞]
  \begin{phonetics}{面团}{mian4tuan2}
    \definition{s.}{massa | pasta}
  \end{phonetics}
\end{entry}

\begin{entry}{面条}{9,7}[Radicais ⾯、⽊]
  \begin{phonetics}{面条}{mian4tiao2}
    \definition{s.}{macarrão | espaguete}
  \end{phonetics}
\end{entry}

\begin{entry}{面条儿}{9,7,2}[Radicais ⾯、⽊、⼉]
  \begin{phonetics}{面条儿}{mian4tiao2r5}[][HSK 1]
    \definition{s.}{macarrão | \emph{noodles}}
  \end{phonetics}
\end{entry}

\begin{entry}{面试}{9,8}[Radicais ⾯、⾔]
  \begin{phonetics}{面试}{mian4 shi4}[][HSK 4]
    \definition[次]{s.}{entrevista; audição}
  \end{phonetics}
\end{entry}

\begin{entry}{面临}{9,9}[Radicais ⾯、⼁]
  \begin{phonetics}{面临}{mian4lin2}[][HSK 4]
    \definition{v.}{ser confrontado com; encontrar (uma situação) na frente de}
  \end{phonetics}
\end{entry}

\begin{entry}{面前}{9,9}[Radicais ⾯、⼑]
  \begin{phonetics}{面前}{mian4 qian2}[][HSK 2]
    \definition{adv.}{antes | na frente de | na (frente) de}
  \end{phonetics}
\end{entry}

\begin{entry}{面积}{9,10}[Radicais ⾯、⽲]
  \begin{phonetics}{面积}{mian4ji1}[][HSK 3]
    \definition{s.}{área (de um andar, pedaço de terreno, etc.); área de uma superfície}
  \end{phonetics}
\end{entry}

\begin{entry}{韭菜}{9,11}[Radicais ⾲、⾋]
  \begin{phonetics}{韭菜}{jiu3cai4}
    \definition{s.}{cebolinha chinesa | (figurativo) investidores de varejo que perdem seu dinheiro para operadores mais experientes (ou seja, são ``colhidos'' como cebolinhas)}
  \end{phonetics}
\end{entry}

\begin{entry}{音乐}{9,5}[Radicais ⾳、⼃]
  \begin{phonetics}{音乐}{yin1yue4}[][HSK 2]
    \definition[张,曲,段]{s.}{música}
  \end{phonetics}
\end{entry}

\begin{entry}{音乐厅}{9,5,4}[Radicais ⾳、⼃、⼚]
  \begin{phonetics}{音乐厅}{yin1yue4ting1}
    \definition{s.}{auditório | teatro | \emph{concert hall}}
  \end{phonetics}
\end{entry}

\begin{entry}{音乐节}{9,5,5}[Radicais ⾳、⼃、⾋]
  \begin{phonetics}{音乐节}{yin1yue4jie2}
    \definition{s.}{festival de música}
  \end{phonetics}
\end{entry}

\begin{entry}{音乐会}{9,5,6}[Radicais ⾳、⼃、⼈]
  \begin{phonetics}{音乐会}{yin1 yue4 hui4}[][HSK 2]
    \definition[场]{s.}{concerto}
  \end{phonetics}
\end{entry}

\begin{entry}{音乐光碟}{9,5,6,14}[Radicais ⾳、⼃、⼉、⽯]
  \begin{phonetics}{音乐光碟}{yin1yue4guang1die2}
    \definition{s.}{CD de música}
  \end{phonetics}
\end{entry}

\begin{entry}{音乐学}{9,5,8}[Radicais ⾳、⼃、⼦]
  \begin{phonetics}{音乐学}{yin1yue4xue2}
    \definition{s.}{musicologia}
  \end{phonetics}
\end{entry}

\begin{entry}{音乐学院}{9,5,8,9}[Radicais ⾳、⼃、⼦、⾩]
  \begin{phonetics}{音乐学院}{yin1yue4xue2yuan4}
    \definition{s.}{conservatório | academia de música}
  \end{phonetics}
\end{entry}

\begin{entry}{音乐院}{9,5,9}[Radicais ⾳、⼃、⾩]
  \begin{phonetics}{音乐院}{yin1yue4yuan4}
    \definition{s.}{conservatório | instituto de música}
  \end{phonetics}
\end{entry}

\begin{entry}{音乐家}{9,5,10}[Radicais ⾳、⼃、⼧]
  \begin{phonetics}{音乐家}{yin1yue4jia1}
    \definition{s.}{músico}
  \end{phonetics}
\end{entry}

\begin{entry}{音节}{9,5}[Radicais ⾳、⾋]
  \begin{phonetics}{音节}{yin1 jie2}[][HSK 2]
    \definition{s.}{sílaba}
  \end{phonetics}
\end{entry}

\begin{entry}{顺}{9}[Radical ⾴]
  \begin{phonetics}{顺}{shun4}
    \definition{adj.}{correr bem | favorável}
  \end{phonetics}
\end{entry}

\begin{entry}{顺从}{9,4}[Radicais ⾴、⼈]
  \begin{phonetics}{顺从}{shun4cong2}
    \definition{v.}{obedecer | submeter-se}
  \end{phonetics}
\end{entry}

\begin{entry}{顺心}{9,4}[Radicais ⾴、⼼]
  \begin{phonetics}{顺心}{shun4xin1}
    \definition{adj.}{satisfatório | satisfeito}
  \end{phonetics}
\end{entry}

\begin{entry}{顺水}{9,4}[Radicais ⾴、⽔]
  \begin{phonetics}{顺水}{shun4shui3}
    \definition{v.}{ir com o fluxo}
  \end{phonetics}
\end{entry}

\begin{entry}{顺延}{9,6}[Radicais ⾴、⼵]
  \begin{phonetics}{顺延}{shun4yan2}
    \definition{v.}{adiar | procrastinar}
  \end{phonetics}
\end{entry}

\begin{entry}{顺当}{9,6}[Radicais ⾴、⼹]
  \begin{phonetics}{顺当}{shun4dang5}
    \definition{adv.}{suavemente}
  \end{phonetics}
\end{entry}

\begin{entry}{顺耳}{9,6}[Radicais ⾴、⽿]
  \begin{phonetics}{顺耳}{shun4'er3}
    \definition{adj.}{agradável ao ouvido}
  \end{phonetics}
\end{entry}

\begin{entry}{顺利}{9,7}[Radicais ⾴、⼑]
  \begin{phonetics}{顺利}{shun4li4}[][HSK 2]
    \definition{adv.}{suavemente | sem problemas}
  \end{phonetics}
\end{entry}

\begin{entry}{顺畅}{9,8}[Radicais ⾴、⽥]
  \begin{phonetics}{顺畅}{shun4chang4}
    \definition{adj.}{liso e sem obstáculos | fluente}
  \end{phonetics}
\end{entry}

\begin{entry}{顺便}{9,9}[Radicais ⾴、⼈]
  \begin{phonetics}{顺便}{shun4bian4}
    \definition{adv.}{convenientemente | de passagem | sem muito esforço extra}
  \end{phonetics}
\end{entry}

\begin{entry}{顺叙}{9,9}[Radicais ⾴、⼜]
  \begin{phonetics}{顺叙}{shun4xu4}
    \definition{s.}{narrativa cronológica}
  \end{phonetics}
\end{entry}

\begin{entry}{顺眼}{9,11}[Radicais ⾴、⽬]
  \begin{phonetics}{顺眼}{shun4yan3}
    \definition{adj.}{agradável aos olhos}
  \end{phonetics}
\end{entry}

\begin{entry}{顺境}{9,14}[Radicais ⾴、⼟]
  \begin{phonetics}{顺境}{shun4jing4}
    \definition{s.}{circunstâncias favoráveis}
  \end{phonetics}
\end{entry}

\begin{entry}{顺嘴}{9,16}[Radicais ⾴、⼝]
  \begin{phonetics}{顺嘴}{shun4zui3}
    \definition{v.}{deixar escapar (sem pensar) | ler suavemente (texto) | adequar-se  ao gosto (comida)}
  \end{phonetics}
\end{entry}

\begin{entry}{飒飒}{9,9}[Radicais ⾵、⾵]
  \begin{phonetics}{飒飒}{sa4sa4}
    \definition{s.}{o farfalhar | sussurro | murmúrio (do vento nas árvores, o mar, etc.)}
  \end{phonetics}
\end{entry}

\begin{entry}{食物}{9,8}[Radicais ⾷、⽜]
  \begin{phonetics}{食物}{shi2wu4}[][HSK 2]
    \definition[种]{s.}{comida}
  \end{phonetics}
\end{entry}

\begin{entry}{食品}{9,9}[Radicais ⾷、⼝]
  \begin{phonetics}{食品}{shi2 pin3}[][HSK 3]
    \definition[种]{s.}{comida; gêneros alimentícios; provisões}
  \end{phonetics}
\end{entry}

\begin{entry}{食堂}{9,11}[Radicais ⾷、⼟]
  \begin{phonetics}{食堂}{shi2tang2}
    \definition[个,间]{s.}{sala de jantar}
  \end{phonetics}
\end{entry}

\begin{entry}{饺子}{9,3}[Radicais ⾷、⼦]
  \begin{phonetics}{饺子}{jiao3zi5}[][HSK 2]
    \definition[个,只]{s.}{jiaozi | bolinhos chineses | bolinho de massa}
  \end{phonetics}
\end{entry}

\begin{entry}{饼}{9}[Radical ⾷]
  \begin{phonetics}{饼}{bing3}
    \definition[张]{s.}{panqueca | biscoito | torta}
  \end{phonetics}
\end{entry}

\begin{entry}{饼干}{9,3}[Radicais ⾷、⼲]
  \begin{phonetics}{饼干}{bing3gan1}
    \definition[片,块]{s.}{bolacha | biscoito}
  \end{phonetics}
\end{entry}

\begin{entry}{首先}{9,6}[Radicais ⾸、⼉]
  \begin{phonetics}{首先}{shou3xian1}[][HSK 3]
    \definition{adv.}{primeiramente; antes de todos os outros}
    \definition{conj.}{acima de tudo; primeiramente; em primeiro lugar}
  \end{phonetics}
\end{entry}

\begin{entry}{首相}{9,9}[Radicais ⾸、⽬]
  \begin{phonetics}{首相}{shou3xiang4}
    \definition*{s.}{Primeiro-Ministro (Japão, UK, etc.)}
  \end{phonetics}
\end{entry}

\begin{entry}{首席执行官}{9,10,6,6,8}[Radicais ⾸、⼱、⼿、⾏、⼧]
  \begin{phonetics}{首席执行官}{shou3xi2 zhi2xing2 guan1}
    \definition{s.}{\emph{chief executive officer}, CEO}
  \end{phonetics}
\end{entry}

\begin{entry}{首都}{9,10}[Radicais ⾸、⾢]
  \begin{phonetics}{首都}{shou3du1}[][HSK 3]
    \definition[个]{s.}{capital (cidade)}
  \end{phonetics}
\end{entry}

\begin{entry}{香}{9}[Kangxi 186][Radical ⾹]
  \begin{phonetics}{香}{xiang1}[][HSK 3]
    \definition*{s.}{sobrenome Xiang}
    \definition{adj.}{aromático; perfumado; fragrante; cheiroso | saboroso; saboroso; delicioso; apetitoso | com gosto; com bom apetite | (sono) profundo | popular; bem-vindo}
    \definition[束,根,炷]{s.}{especiaria; perfume; fragrância; aromatizante | incenso | relacionado a mulheres ou às próprias mulheres}
  \end{phonetics}
\end{entry}

\begin{entry}{香气}{9,4}[Radicais ⾹、⽓]
  \begin{phonetics}{香气}{xiang1qi4}
    \definition{s.}{fragrância | aroma | incenso}
  \end{phonetics}
\end{entry}

\begin{entry}{香皂}{9,7}[Radicais ⾹、⽩]
  \begin{phonetics}{香皂}{xiang1zao4}
    \definition{s.}{sabonete | sabonete perfumado}
  \end{phonetics}
\end{entry}

\begin{entry}{香肠}{9,7}[Radicais ⾹、⾁]
  \begin{phonetics}{香肠}{xiang1chang2}
    \definition[根]{s.}{salsicha}
  \end{phonetics}
\end{entry}

\begin{entry}{香味}{9,8}[Radicais ⾹、⼝]
  \begin{phonetics}{香味}{xiang1wei4}
    \definition[股]{s.}{fragrância | cheiro doce}
  \end{phonetics}
\end{entry}

\begin{entry}{香波}{9,8}[Radicais ⾹、⽔]
  \begin{phonetics}{香波}{xiang1bo1}
    \definition{s.}{xampu}
  \end{phonetics}
\end{entry}

\begin{entry}{香炉}{9,8}[Radicais ⾹、⽕]
  \begin{phonetics}{香炉}{xiang1lu2}
    \definition{s.}{incensário (para queimar incenso) | queimador de incenso | insensório, turíbulo}
  \end{phonetics}
\end{entry}

\begin{entry}{香烟}{9,10}[Radicais ⾹、⽕]
  \begin{phonetics}{香烟}{xiang1yan1}
    \definition[支,条]{s.}{cigarro | fumaça de incenso queimado}
  \end{phonetics}
\end{entry}

\begin{entry}{香艳}{9,10}[Radicais ⾹、⾊]
  \begin{phonetics}{香艳}{xiang1yan4}
    \definition{adj.}{atraente | erótico | romântico}
  \end{phonetics}
\end{entry}

\begin{entry}{香港}{9,12}[Radicais ⾹、⽔]
  \begin{phonetics}{香港}{xiang1gang3}
    \definition*{s.}{Hong Kong}
  \seealsoref{香港岛}{xiang1gang3 dao3}
  \end{phonetics}
\end{entry}

\begin{entry}{香港岛}{9,12,7}[Radicais ⾹、⽔、⼭]
  \begin{phonetics}{香港岛}{xiang1gang3 dao3}
    \definition*{s.}{Ilha de Hong Kong}
  \seealsoref{香港}{xiang1gang3}
  \end{phonetics}
\end{entry}

\begin{entry}{香槟酒}{9,14,10}[Radicais ⾹、⽊、⾣]
  \begin{phonetics}{香槟酒}{xiang1bin1jiu3}
    \definition[杯]{s.}{(empréstimo linguístico) \emph{champagne}}
  \end{phonetics}
\end{entry}

\begin{entry}{香蕈}{9,15}[Radicais ⾹、⾋]
  \begin{phonetics}{香蕈}{xiang1xun4}
    \definition{s.}{\emph{shiitake}, cogumelo comestível}
  \end{phonetics}
\end{entry}

\begin{entry}{香蕉}{9,15}[Radicais ⾹、⾋]
  \begin{phonetics}{香蕉}{xiang1jiao1}[][HSK 3]
    \definition[枝,根,个,把,串,束,弓]{s.}{banana}
  \end{phonetics}
\end{entry}

\begin{entry}{骂}{9}[Radical ⾺]
  \begin{phonetics}{骂}{ma4}
    \definition{v.}{insultar | maldizer | ralhar}
  \end{phonetics}
\end{entry}

\begin{entry}{骂名}{9,6}[Radicais ⾺、⼝]
  \begin{phonetics}{骂名}{ma4ming2}
    \definition{s.}{infâmia}
  \end{phonetics}
\end{entry}

\begin{entry}{骂街}{9,12}[Radicais ⾺、⾏]
  \begin{phonetics}{骂街}{ma4jie1}
    \definition{v.}{gritar abusos na rua}
  \end{phonetics}
\end{entry}

\begin{entry}{骆驼}{9,8}[Radicais ⾺、⾺]
  \begin{phonetics}{骆驼}{luo4tuo5}
    \definition[峰,匹,头]{s.}{camelo | (coloquial) cabeça-dura, idiota}
  \end{phonetics}
\end{entry}

\begin{entry}{骨}{9}[Kangxi 188][Radical ⾻]
  \begin{phonetics}{骨}{gu3}
    \definition{s.}{osso}
  \end{phonetics}
\end{entry}

\begin{entry}{骨头}{9,5}[Radicais ⾻、⼤]
  \begin{phonetics}{骨头}{gu3tou5}[][HSK 4]
    \definition[根,块]{s.}{osso; tecidos mais duros no corpo de uma pessoa ou de alguns animais que sustentam o corpo ou protegem os órgãos do corpo | caráter de uma pessoa; refere-se à qualidade do caráter de uma pessoa}
  \end{phonetics}
\end{entry}

\begin{entry}{鬼火}{9,4}[Radicais ⿁、⽕]
  \begin{phonetics}{鬼火}{gui3huo3}
    \definition{s.}{fogo-fátuo | boitatá | fogo corredor | fogo de santelmo}
  \end{phonetics}
\end{entry}

\begin{entry}{鬼怪}{9,8}[Radicais ⿁、⼼]
  \begin{phonetics}{鬼怪}{gui3guai4}
    \definition{s.}{\emph{hobgoblin} | bicho-papão | fantasma}
  \end{phonetics}
\end{entry}

%%%%% EOF %%%%%


%%%
%%% 10画
%%%
\section*{10画}\addcontentsline{toc}{section}{10画}

%%%%%%%%%% 乘 %%%%%%%%%%
\subsection*{乘}

\begin{Entry}{乘}{10}{⽲}
  \begin{Phonetics}{乘}{cheng2}[][HSK 5]
    \definition*{s.}{Sobrenome: Cheng}
    \definition{s.}{uma divisão principal das escolas budistas; uma seita ou doutrina do budismo}
    \definition{v.}{cavalgar; andar a cavalo; utilizar um veículo ou animal em vez de caminhar | aproveitar-se de; valer-se de; tirar vantagem de; tirar proveito de | multiplicar; realizar multiplicação | perseguir; caçar}
  \end{Phonetics}
  \begin{Phonetics}{乘}{sheng4}
    \definition{clas.}{usado para carruagens de guerra puxada por quatro cavalos}
    \definition{s.}{obras históricas; livros de história geral | antigamente, uma carruagem puxada por quatro cavalos}
  \end{Phonetics}
\end{Entry}

\begin{Entry}{乘人之危}{10,2,3,6}{⽲、⼈、⼂、⼙}
  \begin{Phonetics}{乘人之危}{cheng2ren2zhi1wei1}[][HSK 7-9]
    \definition{expr.}{tirar vantagem das dificuldades dos outros (posição precária; problema); capitalizar as dificuldades de alguém (desastres); fazer uso (utilizar) da situação precária em que alguém se encontra; tentar usar o dilema de alguém para\dots; aproveitar-se da angústia dos outros para prejudicá-los; atacar em um momento de crise}
  \end{Phonetics}
\end{Entry}

\begin{Entry}{乘车}{10,4}{⽲、⾞}
  \begin{Phonetics}{乘车}{cheng2 che1}[][HSK 5]
    \definition{v.}{montar; dirigir; conduzir; andar a cavalo, de moto, de bicicleta, etc.}
  \end{Phonetics}
\end{Entry}

\begin{Entry}{乘坐}{10,7}{⽲、⼟}
  \begin{Phonetics}{乘坐}{cheng2zuo4}[][HSK 5]
    \definition{v.}{pegar (um trem, ônibus, etc.); andar de (bicicleta, moto, etc.)}
  \end{Phonetics}
\end{Entry}

\begin{Entry}{乘客}{10,9}{⽲、⼧}
  \begin{Phonetics}{乘客}{cheng2 ke4}[][HSK 5]
    \definition[个,位,名]{s.}{passageiro; pessoas viajando de carro, navio ou avião}
  \end{Phonetics}
\end{Entry}

\begin{Entry}{乘客数}{10,9,13}{⽲、⼧、⽁}
  \begin{Phonetics}{乘客数}{cheng2ke4 shu4}
    \definition{s.}{número de passageiros}
  \end{Phonetics}
\end{Entry}

\begin{Entry}{乘积}{10,10}{⽲、⽲}
  \begin{Phonetics}{乘积}{cheng2ji1}
    \definition{s.}{(matemática) produto (resultado da multiplicação)}
  \end{Phonetics}
\end{Entry}

%%%%%%%%%% 俯 %%%%%%%%%%
\subsection*{俯}

\begin{Entry}{俯}{10}{⼈}
  \begin{Phonetics}{俯}{fu3}
    \definition{v.}{curvar (a cabeça), oposto a 仰 | inclinar-se | Obsoleto: (em documentos ou cartas oficiais) condescender com | curvar-se; fazer uma reverência}
  \seealsoref{仰}{yang3}
  \end{Phonetics}
\end{Entry}

\begin{Entry}{俯首}{10,9}{⼈、⾸}
  \begin{Phonetics}{俯首}{fu3shou3}[][HSK 7-9]
    \definition{v.}{abaixar a cabeça; curvar-se; inclinar-se}
  \end{Phonetics}
\end{Entry}

%%%%%%%%%% 俱 %%%%%%%%%%
\subsection*{俱}

\begin{Entry}{俱}{10}{⼈}
  \begin{Phonetics}{俱}{ju4}
    \definition{adv.}{(literário) tudo; completamente; inteiramente}
  \end{Phonetics}
\end{Entry}

\begin{Entry}{俱乐部}{10,5,10}{⼈、⼃、⾢}
  \begin{Phonetics}{俱乐部}{ju4le4bu4}[][HSK 5]
    \definition[个,家,间]{s.}{clube; grupos e locais para atividades sociais, políticas, literárias, recreativas e outras}
  \end{Phonetics}
\end{Entry}

%%%%%%%%%% 倂 %%%%%%%%%%
\subsection*{倂}

\begin{Entry}{倂}{10}{⼈}
  \begin{Phonetics}{倂}{bing4}
    \variantof{并}
  \end{Phonetics}
\end{Entry}

%%%%%%%%%% 倍 %%%%%%%%%%
\subsection*{倍}

\begin{Entry}{倍}{10}{⼈}
  \begin{Phonetics}{倍}{bei4}[][HSK 4]
    \definition{adv.}{ainda mais; especialmente | (antes de certos adjetivos) muito; particularmente; é pronunciado como um som erhua e é usado antes de certos adjetivos para expressar um alto grau de profundidade, equivalente a 非常 ou 特别}
    \definition{clas.}{vezes; usado após um numeral, significa que o valor anterior é multiplicado por este número}[增长了五倍。===Aumentou cinco vezes. | 二的三倍是六。===Três vezes dois é seis.]
  \seealsoref{非常}{fei1chang2}
  \seealsoref{特别}{te4bie2}
  \end{Phonetics}
\end{Entry}

%%%%%%%%%% 倒 %%%%%%%%%%
\subsection*{倒}

\begin{Entry}{倒}{10}{⼈}
  \begin{Phonetics}{倒}{dao3}[][HSK 2]
    \definition{v.}{cair; tombar | falhar; entrar em colapso | ficar rouco | mudar; trocar; transferir; converter | movimentar-se; manobrar | oferecer (casa, loja) para venda; vender mercadorias ou lojas a terceiros a um preço fixo | derrubar; derrubar com}
  \end{Phonetics}
  \begin{Phonetics}{倒}{dao4}[][HSK 2]
    \definition{adj.}{inverso; invertido; de cabeça para baixo}
    \definition{adv.}{mas; pelo contrário; expressa o contrário do esperado, equivalente a 反倒 | indicando que algo não é o que se pensa; indica que as coisas não são assim | usado para indicar uma transição ou concessão | transmitindo a sensação de ``urgência''; expressa pressa ou insistência, com um tom impaciente}
    \definition{v.}{ser inverso; estar invertido; estar de cabeça para baixo; inverter a posição original para cima e para baixo ou para a frente e para trás | recuar; virar de cabeça para baixo; fazer mover na direção oposta ou inverter | inclinar ou virar o recipiente para retirar o conteúdo; inclinar; derramar}
  \seealsoref{反倒}{fan3dao4}
  \end{Phonetics}
\end{Entry}

\begin{Entry}{倒下}{10,3}{⼈、⼀}
  \begin{Phonetics}{倒下}{dao3xia4}[][HSK 7-9]
    \definition{v.}{entrar em colapso | tombar}
  \end{Phonetics}
\end{Entry}

\begin{Entry}{倒计时}{10,4,7}{⼈、⾔、⽇}
  \begin{Phonetics}{倒计时}{dao4ji4shi2}[][HSK 7-9]
    \definition{s.}{contagem regressiva; contagem de tempo a partir de um determinado ponto no futuro até o presente, de mais para menos, até que o tempo chegue a zero; é frequentemente usado para expressar que um certo momento está se aproximando}
  \end{Phonetics}
\end{Entry}

\begin{Entry}{倒车}{10,4}{⼈、⾞}
  \begin{Phonetics}{倒车}{dao3che1}[][HSK 4]
    \definition{v.}{trocar de trem ou ônibus (no meio do caminho)}
  \end{Phonetics}
  \begin{Phonetics}{倒车}{dao4che1}[][HSK 4]
    \definition{v.}{dar marcha à ré (em um veículo)}
  \end{Phonetics}
\end{Entry}

\begin{Entry}{倒地}{10,6}{⼈、⼟}
  \begin{Phonetics}{倒地}{dao3di4}
    \definition{v.}{cair no chão}
  \end{Phonetics}
\end{Entry}

\begin{Entry}{倒血霉}{10,6,15}{⼈、⾎、⾬}
  \begin{Phonetics}{倒血霉}{dao3xue4mei2}
    \definition{v.}{ter muito azar (versão mais forte de 倒霉)}
  \seealsoref{倒霉}{dao3/mei2}
  \end{Phonetics}
\end{Entry}

\begin{Entry}{倒闭}{10,6}{⼈、⾨}
  \begin{Phonetics}{倒闭}{dao3bi4}[][HSK 4]
    \definition{v.}{fechar; ir à falência; entrar em liquidação; sair do negócio; (empresa, loja ou banco) deixar de operar devido ao baixo desempenho}
  \end{Phonetics}
\end{Entry}

\begin{Entry}{倒卖}{10,8}{⼈、⼗}
  \begin{Phonetics}{倒卖}{dao3mai4}[][HSK 7-9]
    \definition{v.}{revender com lucro}
  \end{Phonetics}
\end{Entry}

\begin{Entry}{倒是}{10,9}{⼈、⽇}
  \begin{Phonetics}{倒是}{dao4 shi4}[][HSK 5]
    \definition{adv.}{usado para indicar o oposto do que geralmente é verdade; ao contrário do senso comum; pelo contrário | usado para indicar o que é contrário aos fatos, com um toque de crítica; indica que as coisas não são assim (com um sentimento de culpa) | usado de algo inesperado; expressando surpresa | usado para indicar concessão | usado para indicar uma mudança de significado; indica um ponto de virada | usado para modificar ou suavizar uma declaração anterior; para suavizar o tom | usado para pressionar ou questionar alguém; para instar ou perguntar}
  \end{Phonetics}
\end{Entry}

\begin{Entry}{倒塌}{10,13}{⼈、⼟}
  \begin{Phonetics}{倒塌}{dao3ta1}[][HSK 7-9]
    \definition{v.}{colapsar; desabar}
  \end{Phonetics}
\end{Entry}

\begin{Entry}{倒数}{10,13}{⼈、⽁}
  \begin{Phonetics}{倒数}{dao4shu3}[][HSK 7-9]
    \definition{v.}{contar de trás para frente (contagem regressiva)}
  \end{Phonetics}
  \begin{Phonetics}{倒数}{dao4shu4}
    \definition{s.}{número inverso; Matemática: recíproco}
  \end{Phonetics}
\end{Entry}

\begin{Entry}{倒楣}{10,13}{⼈、⽊}
  \begin{Phonetics}{倒楣}{dao3mei2}
    \definition{adj.}{azarado; infeliz; tendo má sorte}
  \end{Phonetics}
\end{Entry}

\begin{Entry}{倒霉}{10,15}{⼈、⾬}
  \begin{Phonetics}{倒霉}{dao3/mei2}[][HSK 7-9]
    \definition{adj.}{azarado}
    \definition{s.}{azar; má sorte}
    \definition{v.+compl.}{cair em dias maus; cair em tempos difíceis; encontrar coisas desfavoráveis; ter má sorte}
  \seealsoref{倒血霉}{dao3xue4mei2}
  \end{Phonetics}
\end{Entry}

%%%%%%%%%% 倔 %%%%%%%%%%
\subsection*{倔}

\begin{Entry}{倔}{10}{⼈}
  \begin{Phonetics}{倔}{jue2}
    \definition{adj.}{rude; mal-humorado; abrupto; curto (uso limitado em 倔犟) | teimoso; direto e franco}
  \end{Phonetics}
  \begin{Phonetics}{倔}{jue4}[][HSK 7-9]
    \definition{adj.}{teimoso; direto; rude; grosseiro; de natureza direta, com uma atitude severa em relação aos outros}
  \end{Phonetics}
\end{Entry}

\begin{Entry}{倔强}{10,12}{⼈、⼸}
  \begin{Phonetics}{倔强}{jue2jiang4}[][HSK 7-9]
    \definition{adj.}{teimoso; rígido; inflexível; de personalidade forte e teimosa}
  \end{Phonetics}
\end{Entry}

%%%%%%%%%% 倘 %%%%%%%%%%
\subsection*{倘}

\begin{Entry}{倘}{10}{⼈}
  \begin{Phonetics}{倘}{chang2}
  \end{Phonetics}
  \begin{Phonetics}{倘}{tang3}
    \definition{conj.}{se; supondo; no caso}
  \end{Phonetics}
\end{Entry}

\begin{Entry}{倘使}{10,8}{⼈、⼈}
  \begin{Phonetics}{倘使}{tang3shi3}
    \definition{conj.}{se | supondo que | no caso}
  \end{Phonetics}
\end{Entry}

\begin{Entry}{倘或}{10,8}{⼈、⼽}
  \begin{Phonetics}{倘或}{tang3huo4}
    \definition{conj.}{se | supondo que | no caso}
  \end{Phonetics}
\end{Entry}

\begin{Entry}{倘若}{10,8}{⼈、⾋}
  \begin{Phonetics}{倘若}{tang3ruo4}
    \definition{conj.}{se | supondo que | no caso}
  \end{Phonetics}
\end{Entry}

%%%%%%%%%% 候 %%%%%%%%%%
\subsection*{候}

\begin{Entry}{候}{10}{⼈}
  \begin{Phonetics}{候}{hou4}
    \definition*{s.}{Sobrenome: Hou}
    \definition{s.}{tempo; estação | condição; estado | situação meteorológica | uma unidade tradicional de tempo no antigo calendário chinês; antigamente, cinco dias constituíam uma estação, o que ainda é usado na meteorologia hoje em dia}
    \definition{v.}{esperar; aguardar | perguntar depois | assistir; observar}
  \end{Phonetics}
\end{Entry}

\begin{Entry}{候选人}{10,9,2}{⼈、⾡、⼈}
  \begin{Phonetics}{候选人}{hou4xuan3ren2}[][HSK 7-9]
    \definition[个,名,位]{s.}{candidato}
  \end{Phonetics}
\end{Entry}

%%%%%%%%%% 借 %%%%%%%%%%
\subsection*{借}

\begin{Entry}{借}{10}{⼈}
  \begin{Phonetics}{借}{jie4}[][HSK 2]
    \definition{adv.}{por meio de}
    \definition{v.}{emprestar | pedir emprestado | usar como pretexto | aproveitar; tirar proveito (de uma oportunidade, etc.)}
  \end{Phonetics}
\end{Entry}

\begin{Entry}{借口}{10,3}{⼈、⼝}
  \begin{Phonetics}{借口}{jie4kou3}[][HSK 7-9]
    \definition[个,种]{s.}{desculpa; pretexto; razões falsas apresentadas para atingir um objetivo}
    \definition{v.}{usar como desculpa; usar sob o pretexto de; usar com a justificativa de; usar (algo) como motivo (que não seja um motivo real)}
  \end{Phonetics}
\end{Entry}

\begin{Entry}{借书证}{10,4,7}{⼈、⼄、⾔}
  \begin{Phonetics}{借书证}{jie4shu1zheng4}
    \definition{s.}{cartão da biblioteca; comprovante de solicitação}
  \seealsoref{借书证卡}{jie4shu1zheng4 ka3}
  \end{Phonetics}
\end{Entry}

\begin{Entry}{借书证卡}{10,4,7,5}{⼈、⼄、⾔、⼘}
  \begin{Phonetics}{借书证卡}{jie4shu1zheng4 ka3}
    \definition{s.}{cartão da biblioteca}
  \seealsoref{借书证}{jie4shu1zheng4}
  \end{Phonetics}
\end{Entry}

\begin{Entry}{借用}{10,5}{⼈、⽤}
  \begin{Phonetics}{借用}{jie4yong4}[][HSK 7-9]
    \definition{v.}{tomar emprestado; ter o empréstimo de | usar algo para outro propósito}
  \end{Phonetics}
\end{Entry}

\begin{Entry}{借助}{10,7}{⼈、⼒}
  \begin{Phonetics}{借助}{jie4zhu4}[][HSK 7-9]
    \definition{v.}{contar com a ajuda de; obter apoio de}
  \end{Phonetics}
\end{Entry}

\begin{Entry}{借条}{10,7}{⼈、⽊}
  \begin{Phonetics}{借条}{jie4tiao2}[][HSK 7-9]
    \definition{s.}{recibo de empréstimo; nota promissória}
  \seealsoref{借条儿}{jie4tiao2r5}
  \end{Phonetics}
\end{Entry}

\begin{Entry}{借条儿}{10,7,2}{⼈、⽊、⼉}
  \begin{Phonetics}{借条儿}{jie4tiao2r5}
    \definition{s.}{nota promissória}
  \end{Phonetics}
\end{Entry}

\begin{Entry}{借鉴}{10,13}{⼈、⾦}
  \begin{Phonetics}{借鉴}{jie4jian4}[][HSK 6]
    \definition{s.}{tirar lições de; aproveitar a experiência de; ganhar experiência e lições com o passado ou com as experiências de outras pessoas}
  \end{Phonetics}
\end{Entry}

%%%%%%%%%% 倡 %%%%%%%%%%
\subsection*{倡}

\begin{Entry}{倡}{10}{⼈}
  \begin{Phonetics}{倡}{chang4}
    \definition{v.}{iniciar; propor; defender | promover; assumir a liderança}
  \end{Phonetics}
\end{Entry}

\begin{Entry}{倡议}{10,5}{⼈、⾔}
  \begin{Phonetics}{倡议}{chang4yi4}[][HSK 7-9]
    \definition[项,条,次]{s.}{proposta; iniciativa; primeiras sugestões}
    \definition{v.}{propor; iniciar; defender}
  \end{Phonetics}
\end{Entry}

\begin{Entry}{倡导}{10,6}{⼈、⼨}
  \begin{Phonetics}{倡导}{chang4dao3}[][HSK 5]
    \definition{v.}{iniciar; propor; promover; defender; advogar}
  \end{Phonetics}
\end{Entry}

%%%%%%%%%% 债 %%%%%%%%%%
\subsection*{债}

\begin{Entry}{债}{10}{⼈}
  \begin{Phonetics}{债}{zhai4}[][HSK 6]
    \definition[笔]{s.}{dívida | empréstimo}
  \end{Phonetics}
\end{Entry}

%%%%%%%%%% 值 %%%%%%%%%%
\subsection*{值}

\begin{Entry}{值}{10}{⼈}
  \begin{Phonetics}{值}{zhi2}[][HSK 3]
    \definition{adj.}{significativo; valioso; digno de nota}
    \definition{prep.}{quando; introduz o momento em que algo acontece ou existe, equivalente a 当 ou 在}
    \definition{s.}{preço; valor | valor de um número, de uma variável}
    \definition{v.}{valer; custar; a mercadoria é adequada ao preço | ir de encontro; encontrar; cruzar | estar de serviço; ter sua vez em algo; assumir o cargo que lhe cabe | é a vez de (executar uma determinada função pública)}
  \seealsoref{当}{dang1}
  \seealsoref{在}{zai4}
  \end{Phonetics}
\end{Entry}

\begin{Entry}{值班}{10,10}{⼈、⽟}
  \begin{Phonetics}{值班}{zhi2ban1}[][HSK 5]
    \definition{v.}{estar em serviço ou plantão; trabalhar em um turno; (em rodízio) desempenhar funções durante um período de tempo determinado}
  \end{Phonetics}
\end{Entry}

\begin{Entry}{值得}{10,11}{⼈、⼻}
  \begin{Phonetics}{值得}{zhi2de5}[][HSK 3]
    \definition{adj.}{que tem valor; (fazer algo) é vantajoso, sem prejuízos}
    \definition{v.}{merecer; ter valor; significa que fazer isso terá bons resultados; que é valioso e significativo}
  \end{Phonetics}
\end{Entry}

%%%%%%%%%% 倾 %%%%%%%%%%
\subsection*{倾}

\begin{Entry}{倾}{10}{⼈}
  \begin{Phonetics}{倾}{qing1}
    \definition{s.}{desvio; tendência}
    \definition{v.}{inclinar; inclinar-se; dobrar-se | colapsar | virar e despejar; esvaziar | fazer tudo o que puder; usar todos os recursos | sobrecarregar; dominar; dominar | admirar | superar}
  \end{Phonetics}
\end{Entry}

\begin{Entry}{倾向}{10,6}{⼈、⼝}
  \begin{Phonetics}{倾向}{qing1xiang4}[][HSK 6]
    \definition{s.}{tendência; desvio; inclinação; direção do desenvolvimento}
    \definition{v.}{preferir; estar inclinado a; concordar com uma determinada opinião}
  \end{Phonetics}
\end{Entry}

\begin{Entry}{倾城}{10,9}{⼈、⼟}
  \begin{Phonetics}{倾城}{qing1cheng2}
    \definition{adj.}{sedutora (mulher)}
    \definition{adv.}{de todo o lugar | vindo de todos os lugares}
    \definition{v.}{arruinar e derrubar o estado}
  \end{Phonetics}
\end{Entry}

%%%%%%%%%% 健 %%%%%%%%%%
\subsection*{健}

\begin{Entry}{健}{10}{⼈}
  \begin{Phonetics}{健}{jian4}
    \definition{adj.}{forte; saudável; bem definido | ser forte em; ser bom em; apresentar um grau superior à média em determinado aspecto}
    \definition{v.}{fortalecer; endurecer; revigorar}
  \end{Phonetics}
\end{Entry}

\begin{Entry}{健全}{10,6}{⼈、⼊}
  \begin{Phonetics}{健全}{jian4quan2}[][HSK 5]
    \definition{adj.}{saudável; íntegro; capaz; apto; robusto e sem mácula | sólido; completo; perfeito}
    \definition{v.}{aperfeiçoar; melhorar; fortalecer; reforçar}
  \end{Phonetics}
\end{Entry}

\begin{Entry}{健壮}{10,6}{⼈、⼠}
  \begin{Phonetics}{健壮}{jian4zhuang4}[][HSK 7-9]
    \definition{adj.}{robusto; saudável e forte}
  \end{Phonetics}
\end{Entry}

\begin{Entry}{健身}{10,7}{⼈、⾝}
  \begin{Phonetics}{健身}{jian4/shen1}[][HSK 4]
    \definition{s.}{exercício físico | \emph{fitness}}
    \definition{v.+compl.}{exercitar-se; manter a forma; praticar um esporte, especialmente a ginástica, inclusive em aparelhos, para desenvolver força, flexibilidade, aumentar a resistência, melhorar a coordenação e o controle de todas as partes do corpo}
  \end{Phonetics}
\end{Entry}

\begin{Entry}{健美}{10,9}{⼈、⽺}
  \begin{Phonetics}{健美}{jian4mei3}[][HSK 7-9]
    \definition{adj.}{forte e bonito; vigoroso e gracioso; robusto e elegante}
    \definition[次]{s.}{fisiculturismo; exercícios que desenvolvem os músculos e o físico}
  \end{Phonetics}
\end{Entry}

\begin{Entry}{健康}{10,11}{⼈、⼴}
  \begin{Phonetics}{健康}{jian4kang1}[][HSK 2]
    \definition{adj.}{em forma; saudável; descreve que a pessoa está em ótimo estado físico ou mental, sem nenhum problema | sudável; tudo está normal, sem problemas | saudável; livre de doenças; bom para a saúde}
    \definition{s.}{saúde; físico; estado de saúde}
  \end{Phonetics}
\end{Entry}

%%%%%%%%%% 党 %%%%%%%%%%
\subsection*{党}

\begin{Entry}{党}{10}{⼉}
  \begin{Phonetics}{党}{dang3}[][HSK 6]
    \definition*{s.}{O Partido (Partido Comunista da China) | Sobrenome: Dang}
    \definition{s.}{partido político; partido | camarilha; facção; gangue | Obsoleto: parentes}
    \definition{v.}{ser parcial; tomar partido de}
  \end{Phonetics}
\end{Entry}

%%%%%%%%%% 兼 %%%%%%%%%%
\subsection*{兼}

\begin{Entry}{兼}{10}{⼋}
  \begin{Phonetics}{兼}{jian1}[][HSK 7-9]
    \definition*{s.}{Sobrenome: Jian}
    \definition{adj.}{duplo; dobrado; duplicado | simultâneo; concomitante}
    \definition{adv.}{simultaneamente; concomitivamente; envolve várias coisas ao mesmo tempo.}
    \definition{v.}{ocupar um cargo simultâneo | ter dois ou mais empregos simultaneamente; fazer várias coisas ao mesmo tempo ou possuir várias coisas | Literário: reunir; unir em um só; anexar}
  \end{Phonetics}
\end{Entry}

\begin{Entry}{兼任}{10,6}{⼋、⼈}
  \begin{Phonetics}{兼任}{jian1ren4}[][HSK 7-9]
    \definition{v.}{ocupar um cargo simultâneo; ter vários empregos ao mesmo tempo | realizar algo em tempo parcial; trabalhar em tempo parcial}
  \end{Phonetics}
\end{Entry}

\begin{Entry}{兼容}{10,10}{⼋、⼧}
  \begin{Phonetics}{兼容}{jian1rong2}[][HSK 7-9]
    \definition{v.}{abranger a todos; ser compatível; aceitar e acomodar simultaneamente coisas ou aspectos diferentes.}
  \end{Phonetics}
\end{Entry}

\begin{Entry}{兼顾}{10,10}{⼋、⾴}
  \begin{Phonetics}{兼顾}{jian1gu4}[][HSK 7-9]
    \definition{v.}{levar em consideração duas ou mais coisas; dar atenção a duas ou mais coisas}
  \end{Phonetics}
\end{Entry}

\begin{Entry}{兼职}{10,11}{⼋、⽿}
  \begin{Phonetics}{兼职}{jian1zhi2}[][HSK 7-9]
    \definition[份]{pron.}{vaga simultânea; emprego de meio período; cargos ocupados fora da função principal de emprego}
    \definition{v.}{ocupar dois ou mais cargos simultaneamente; exercer outras funções além do trabalho principal}
  \end{Phonetics}
\end{Entry}

%%%%%%%%%% 准 %%%%%%%%%%
\subsection*{准}

\begin{Entry}{准}{10}{⼎}
  \begin{Phonetics}{准}{zhun3}[][HSK 3]
    \definition{adj.}{exato; preciso; algo determinado a ser imutável | preciso; exato; correto | perto; parcialmente; quase; próximo de algo em termos de padrão}
    \definition{adv.}{definitivamente; certamente}
    \definition{pref.}{quasi-; para-}
    \definition{prep.}{de acordo com; baseado em}
    \definition{s.}{norma; padrão; critério | confiança certa; uma ideia definida, certeza, etc. (geralmente usada depois de 有 ou 没有)}
    \definition{v.}{autorizar; conceder; consentir; permitir}
  \seealsoref{没有}{mei2 you3}
  \seealsoref{有}{you3}
  \end{Phonetics}
\end{Entry}

\begin{Entry}{准时}{10,7}{⼎、⽇}
  \begin{Phonetics}{准时}{zhun3shi2}[][HSK 4]
    \definition{adj.}{pontual}
    \definition{adv.}{na hora certa; dentro do prazo; no horário especificado}
  \end{Phonetics}
\end{Entry}

\begin{Entry}{准备}{10,8}{⼎、⼡}
  \begin{Phonetics}{准备}{zhun3bei4}[][HSK 1]
    \definition{v.}{preparar; ficar pronto; planejar ou organizar com antecedência | pretender; planejar}
  \end{Phonetics}
\end{Entry}

\begin{Entry}{准确}{10,12}{⼎、⽯}
  \begin{Phonetics}{准确}{zhun3que4}[][HSK 2]
    \definition{adj.}{exato; preciso; acurado; os resultados da ação são completamente consistentes com os resultados reais ou esperados}
  \end{Phonetics}
\end{Entry}

%%%%%%%%%% 凉 %%%%%%%%%%
\subsection*{凉}

\begin{Entry}{凉}{10}{⼎}
  \begin{Phonetics}{凉}{liang2}[][HSK 2]
    \definition{adj.}{frio; gelado; ligeiramente fria (menos do que 冷) | sombrio; desolado; sem animação | desanimado; desapontado | usado para prevenir o calor e manter a temperatura amena; para proteção contra o calor}
    \definition{s.}{frio; refere-se a um ambiente fresco ou a uma brisa fresca}
  \seealsoref{冷}{leng3}
  \end{Phonetics}
  \begin{Phonetics}{凉}{liang4}
    \definition{v.}{deixar algo esfriar; deixar um objeto quente descansar por um tempo para que a temperatura diminua}
  \end{Phonetics}
\end{Entry}

\begin{Entry}{凉水}{10,4}{⼎、⽔}
  \begin{Phonetics}{凉水}{liang2 shui3}[][HSK 3]
    \definition{s.}{água fria; água não aquecida | água não fervida}
  \end{Phonetics}
\end{Entry}

\begin{Entry}{凉快}{10,7}{⼎、⼼}
  \begin{Phonetics}{凉快}{liang2kuai5}[][HSK 2]
    \definition{adj.}{fresco; refrescante}
    \definition{v.}{refrescar; refrescar-se; deixar o corpo fresco e revigorado}
  \end{Phonetics}
\end{Entry}

\begin{Entry}{凉爽}{10,11}{⼎、⽘}
  \begin{Phonetics}{凉爽}{liang2shuang3}[][HSK 7-9]
    \definition{adj.}{agradável e fresco; agradavelmente fresco; não está quente, é uma sensação agradável}
  \end{Phonetics}
\end{Entry}

\begin{Entry}{凉鞋}{10,15}{⼎、⾰}
  \begin{Phonetics}{凉鞋}{liang2 xie2}[][HSK 6]
    \definition[双,只]{s.}{sandália; alpargata; alpercata; alparca ; sapatos de verão com cabedal ventilado}
  \end{Phonetics}
\end{Entry}

%%%%%%%%%% 凌 %%%%%%%%%%
\subsection*{凌}

\begin{Entry}{凌}{10}{⼎}
  \begin{Phonetics}{凌}{ling2}
    \definition*{s.}{Sobrenome: Ling}
    \definition{s.}{gelo}
    \definition{v.}{insultar; invadir; violar; abusar | elevar-se alto; subir; ascender | abordar; aproximar-se}
  \end{Phonetics}
\end{Entry}

\begin{Entry}{凌晨}{10,11}{⼎、⽇}
  \begin{Phonetics}{凌晨}{ling2chen2}[][HSK 7-9]
    \definition[个]{s.}{antes do amanhecer; nas primeiras horas da manhã; com a aproximação do amanhecer}
  \end{Phonetics}
\end{Entry}

%%%%%%%%%% 剥 %%%%%%%%%%
\subsection*{剥}

\begin{Entry}{剥}{10}{⼑}
  \begin{Phonetics}{剥}{bao1}[][HSK 7-9]
    \definition{v.}{descascar; despelar; remover a casca ou pele externa}
  \end{Phonetics}
  \begin{Phonetics}{剥}{bo1}
    \definition{v.}{Dialeto: descascar; despelar; remover a casca ou pele externa | (pele, tinta, etc.) sair; descascar | privar; explorar}
  \end{Phonetics}
\end{Entry}

\begin{Entry}{剥夺}{10,6}{⼑、⼤}
  \begin{Phonetics}{剥夺}{bo1duo2}[][HSK 7-9]
    \definition{v.}{roubar algo de alguém; tirar algo de alguém à força | privar; privar por lei; cancelar de acordo com a lei}
  \end{Phonetics}
\end{Entry}

\begin{Entry}{剥削}{10,9}{⼑、⼑}
  \begin{Phonetics}{剥削}{bo1xue1}[][HSK 7-9]
    \definition{v.}{explorar; apropriar-se do trabalho ou dos frutos do trabalho de outrem sem remuneração}
  \end{Phonetics}
\end{Entry}

%%%%%%%%%% 剧 %%%%%%%%%%
\subsection*{剧}

\begin{Entry}{剧}{10}{⼑}
  \begin{Phonetics}{剧}{ju4}[][HSK 6]
    \definition*{s.}{Sobrenome: Ju}
    \definition{adj.}{agudo; grave; intenso; violento}
    \definition[部,个,种]{s.}{obra teatral; drama; peça; ópera}
  \end{Phonetics}
\end{Entry}

\begin{Entry}{剧本}{10,5}{⼑、⽊}
  \begin{Phonetics}{剧本}{ju4ben3}[][HSK 5]
    \definition[部,个]{s.}{cenário; roteiro (para drama, filme, etc.); gênero de obra literária que consiste em diálogos entre personagens (às vezes cantados) e indicações de palco}
  \end{Phonetics}
\end{Entry}

\begin{Entry}{剧目}{10,5}{⼑、⽬}
  \begin{Phonetics}{剧目}{ju4mu4}[][HSK 7-9]
    \definition{s.}{repertório; programa; lista de peças teatrais ou óperas}
  \end{Phonetics}
\end{Entry}

\begin{Entry}{剧团}{10,6}{⼑、⼞}
  \begin{Phonetics}{剧团}{ju4tuan2}[][HSK 7-9]
    \definition[家,个]{s.}{companhia teatral; grupo de ópera; trupe | grupo de teatro}
  \end{Phonetics}
\end{Entry}

\begin{Entry}{剧场}{10,6}{⼑、⼟}
  \begin{Phonetics}{剧场}{ju4 chang3}[][HSK 3]
    \definition[个,坐]{s.}{teatro; local para apresentações teatrais, musicais, etc.}
  \end{Phonetics}
\end{Entry}

\begin{Entry}{剧组}{10,8}{⼑、⽷}
  \begin{Phonetics}{剧组}{ju4zu3}[][HSK 7-9]
    \definition{s.}{equipe de produção; elenco e equipe técnica; um grupo composto por diretores, atores e membros da equipe com o objetivo de apresentar uma peça teatral ou filmar um filme ou série de televisão}
  \end{Phonetics}
\end{Entry}

\begin{Entry}{剧院}{10,9}{⼑、⾩}
  \begin{Phonetics}{剧院}{ju4yuan4}[][HSK 7-9]
    \definition[家,座]{s.}{teatro; casa de espetáculos | companhias teatrais maiores e de classe mais alta}
  \end{Phonetics}
\end{Entry}

\begin{Entry}{剧烈}{10,10}{⼑、⽕}
  \begin{Phonetics}{剧烈}{ju4lie4}[][HSK 7-9]
    \definition{adj.}{violento; agudo; severo; feroz; rápido e intenso}
  \end{Phonetics}
\end{Entry}

\begin{Entry}{剧情}{10,11}{⼑、⼼}
  \begin{Phonetics}{剧情}{ju4qing2}[][HSK 7-9]
    \definition[个,段]{s.}{a história; o enredo (de uma peça ou ópera)}
  \end{Phonetics}
\end{Entry}

%%%%%%%%%% 原 %%%%%%%%%%
\subsection*{原}

\begin{Entry}{原}{10}{⼚}
  \begin{Phonetics}{原}{yuan2}[][HSK 6]
    \definition*{s.}{Sobrenome: Yuan}
    \definition{adj.}{inicial; básico; primitivo | cru; bruto; não processado | virgem; primário; original; antigo; inalterado}
    \definition{adv.}{originalmente}
    \definition[项,条,片]{s.}{planície; país aberto; terreno plano e amplo | início; fonte; origem; aparência original | origem; a raiz ou o começo das coisas}
    \definition{v.}{desculpar; perdoar; tolerar; compreender | rastrear; sondar; investigar (a origem das coisas)}
  \end{Phonetics}
\end{Entry}

\begin{Entry}{原木}{10,4}{⼚、⽊}
  \begin{Phonetics}{原木}{yuan2mu4}
    \definition{s.}{registro | \emph{logs}}
  \end{Phonetics}
\end{Entry}

\begin{Entry}{原先}{10,6}{⼚、⼉}
  \begin{Phonetics}{原先}{yuan2xian1}[][HSK 5]
    \definition{adj.}{antigo; original}
    \definition{s.}{antigamente; no início; no passado; no começo}
  \end{Phonetics}
\end{Entry}

\begin{Entry}{原则}{10,6}{⼚、⼑}
  \begin{Phonetics}{原则}{yuan2ze2}[][HSK 4]
    \definition{adv.}{em geral; em princípio; refere-se a um aspecto geral; geralmente}
    \definition[个,条,项,点]{s.}{princípios; leis ou padrões pelos quais alguém fala ou age}
  \end{Phonetics}
\end{Entry}

\begin{Entry}{原因}{10,6}{⼚、⼞}
  \begin{Phonetics}{原因}{yuan2yin1}[][HSK 2]
    \definition[个,条,种,些]{s.}{causa; razão; motivo; as condições que fazem com que algo aconteça ou produzam um certo resultado}
  \end{Phonetics}
\end{Entry}

\begin{Entry}{原有}{10,6}{⼚、⽉}
  \begin{Phonetics}{原有}{yuan2 you3}[][HSK 5]
    \definition{v.}{já estar pronto, não é necessário fazer ou procurar nada; ser o original}
  \end{Phonetics}
\end{Entry}

\begin{Entry}{原色}{10,6}{⼚、⾊}
  \begin{Phonetics}{原色}{yuan2 se4}
    \definition{s.}{cor primária}
  \end{Phonetics}
\end{Entry}

\begin{Entry}{原告}{10,7}{⼚、⼝}
  \begin{Phonetics}{原告}{yuan2gao4}[][HSK 6]
    \definition{s.}{(em casos civis) autor; solicitante | (em casos criminais) promotor; acusador; reclamante (oposto a 被告)}
  \seealsoref{被告}{bei4gao4}
  \end{Phonetics}
\end{Entry}

\begin{Entry}{原来}{10,7}{⼚、⽊}
  \begin{Phonetics}{原来}{yuan2lai2}[][HSK 2]
    \definition{adj.}{original; anterior; em primeiro lugar; inicialmente; inalterado}
    \definition{adv.}{na verdade; de fato; como se vê; expressar compreensão repentina}
    \definition{s.}{a princípio; no passado; antigamente}
  \end{Phonetics}
\end{Entry}

\begin{Entry}{原始}{10,8}{⼚、⼥}
  \begin{Phonetics}{原始}{yuan2shi3}[][HSK 5]
    \definition{s.}{original; de primeira mão | primitivo; mais antigo; não desenvolvido; não civilizado}
  \end{Phonetics}
\end{Entry}

\begin{Entry}{原料}{10,10}{⼚、⽃}
  \begin{Phonetics}{原料}{yuan2liao4}[][HSK 4]
    \definition[种,个]{s.}{matéria-prima; refere-se a materiais que não foram processados e fabricados, como minérios para metalurgia e algodão para têxteis}
  \end{Phonetics}
\end{Entry}

\begin{Entry}{原谅}{10,10}{⼚、⾔}
  \begin{Phonetics}{原谅}{yuan2liang4}[][HSK 6]
    \definition{v.}{perdoar; perdoar a negligência, os erros ou as falhas das pessoas sem culpá-las ou puni-las}
  \end{Phonetics}
\end{Entry}

\begin{Entry}{原理}{10,11}{⼚、⽟}
  \begin{Phonetics}{原理}{yuan2li3}[][HSK 5]
    \definition[个,条]{s.}{princípio; axioma; teoria; teoria básica ou princípio científico de significado universal}
  \end{Phonetics}
\end{Entry}

%%%%%%%%%% 哥 %%%%%%%%%%
\subsection*{哥}

\begin{Entry}{哥}{10}{⼝}
  \begin{Phonetics}{哥}{ge1}[][HSK 1]
    \definition[个,位,名,些]{s.}{irmão mais velho | forma de se dirigir a um parente masculino mais velho de sua geração | irmão; termo amigável para se dirigir a conhecidos mais velhos do sexo masculino}
  \seealsoref{哥哥}{ge1 ge5}
  \end{Phonetics}
\end{Entry}

\begin{Entry}{哥们}{10,5}{⼝、⼈}
  \begin{Phonetics}{哥们}{ge1men5}
    \definition{expr.}{\emph{Brothers!}}
    \definition{s.}{(coloquial) cara | irmão (forma diminuta de tratamento entre homens)}
  \end{Phonetics}
\end{Entry}

\begin{Entry}{哥哥}{10,10}{⼝、⼝}
  \begin{Phonetics}{哥哥}{ge1 ge5}[][HSK 1]
    \definition[个,位]{s.}{irmão mais velho | primo}
  \end{Phonetics}
\end{Entry}

\begin{Entry}{哥斯拉}{10,12,8}{⼝、⽄、⼿}
  \begin{Phonetics}{哥斯拉}{ge1si1la1}
    \definition*{s.}{Godzilla}
  \seealsoref{酷斯拉}{ku4si1la1}
  \end{Phonetics}
\end{Entry}

%%%%%%%%%% 哦 %%%%%%%%%%
\subsection*{哦}

\begin{Entry}{哦}{10}{⼝}
  \begin{Phonetics}{哦}{e2}
    \definition{v.}{cantar suavemente (um poema)}
  \end{Phonetics}
  \begin{Phonetics}{哦}{o2}
    \definition{interj.}{Oh! (indicando dúvida ou surpresa)}
  \end{Phonetics}
  \begin{Phonetics}{哦}{o4}
    \definition{interj.}{Oh! (indicando que acabou de aprender algo)}
  \end{Phonetics}
  \begin{Phonetics}{哦}{o5}
    \definition{part.}{usada no final da frase para indicar que uma pessoa está afirmando um fato que a outra pessoa não sabe | usada no final da frase para transmitir informalidade, calor, simpatia ou intimidade}
  \end{Phonetics}
\end{Entry}

%%%%%%%%%% 哩 %%%%%%%%%%
\subsection*{哩}

\begin{Entry}{哩}{10}{⼝}
  \begin{Phonetics}{哩}{li3}
    \definition{clas.}{milha (unidade de comprimento igual a 1.609,344 m)}
  \end{Phonetics}
  \begin{Phonetics}{哩}{li5}
    \definition{part.}{(dialeto) final modal semelhante a 呢 ou 啦, usado em um tom definido, mas um tanto exagerado}
  \seealsoref{啦}{la5}
  \seealsoref{呢}{ne5}
  \end{Phonetics}
\end{Entry}

\begin{Entry}{哩哩啦啦}{10,10,11,11}{⼝、⼝、⼝、⼝}
  \begin{Phonetics}{哩哩啦啦}{li1 li1 la1 la1}
    \definition{adj.}{espalhado; disperso; disseminado; difuso; esporádico; aqui e ali}
  \end{Phonetics}
\end{Entry}

%%%%%%%%%% 哭 %%%%%%%%%%
\subsection*{哭}

\begin{Entry}{哭}{10}{⼝}
  \begin{Phonetics}{哭}{ku1}[][HSK 2]
    \definition{v.}{chorar; soluçar; lamentar-se; chorar de dor ou emoção}
  \end{Phonetics}
\end{Entry}

\begin{Entry}{哭泣}{10,8}{⼝、⽔}
  \begin{Phonetics}{哭泣}{ku1qi4}[][HSK 7-9]
    \definition{v.}{chorar; soluçar; chorar copiosamente}
  \end{Phonetics}
\end{Entry}

\begin{Entry}{哭笑不得}{10,10,4,11}{⼝、⽵、⼀、⼻}
  \begin{Phonetics}{哭笑不得}{ku1xiao4-bu4de2}[][HSK 7-9]
    \definition{expr.}{``Sem saber se ria ou chorava.''; a incapacidade de chorar ou rir descreve uma situação constrangedora em que a pessoa não sabe o que fazer}
  \end{Phonetics}
\end{Entry}

\begin{Entry}{哭墙}{10,14}{⼝、⼟}
  \begin{Phonetics}{哭墙}{ku1qiang2}
    \definition*{s.}{Muro das Lamentações (Jerusalém)}
  \end{Phonetics}
\end{Entry}

%%%%%%%%%% 哮 %%%%%%%%%%
\subsection*{哮}

\begin{Entry}{哮}{10}{⼝}
  \begin{Phonetics}{哮}{xiao4}
    \definition{s.}{respiração pesada; chiado}
    \definition{v.}{rugir; uivar}
  \end{Phonetics}
\end{Entry}

\begin{Entry}{哮喘}{10,12}{⼝、⼝}
  \begin{Phonetics}{哮喘}{xiao4chuan3}
    \definition{s.}{asma; sintomas de dispneia: os pacientes sentem que a respiração está muito difícil; pneumonia, insuficiência cardíaca, bronquite crônica e outras doenças causadas por espasmo da musculatura lisa respiratória frequentemente apresentam esse sintoma}
    \definition{v.}{sofrer de asma}
  \end{Phonetics}
\end{Entry}

%%%%%%%%%% 哲 %%%%%%%%%%
\subsection*{哲}

\begin{Entry}{哲}{10}{⼝}
  \begin{Phonetics}{哲}{zhe2}
    \definition{adj.}{sábio; sagaz}
    \definition[位,名,个]{s.}{pessoas sábias; sábio |sabedoria}
  \end{Phonetics}
\end{Entry}

\begin{Entry}{哲学}{10,8}{⼝、⼦}
  \begin{Phonetics}{哲学}{zhe2xue2}[][HSK 6]
    \definition{s.}{filosofia; é uma disciplina que explora questões fundamentais e conceitos básicos}
  \end{Phonetics}
\end{Entry}

\begin{Entry}{哲理}{10,11}{⼝、⽟}
  \begin{Phonetics}{哲理}{zhe2li3}
    \definition{s.}{filosofia | teoria filosófica}
  \end{Phonetics}
\end{Entry}

%%%%%%%%%% 哺 %%%%%%%%%%
\subsection*{哺}

\begin{Entry}{哺}{10}{⼝}
  \begin{Phonetics}{哺}{bu3}
    \definition{s.}{comida na boca; chorando por comida | alimentos de mastigação; mastigando comida}
    \definition{v.}{alimentar (um bebê); amamentar}
  \end{Phonetics}
\end{Entry}

\begin{Entry}{哺育}{10,8}{⼝、⾁}
  \begin{Phonetics}{哺育}{bu3yu4}[][HSK 7-9]
    \definition{v.}{alimentar | Figurativo: nutrir; fomentar | desenvolver}
  \end{Phonetics}
\end{Entry}

%%%%%%%%%% 哼 %%%%%%%%%%
\subsection*{哼}

\begin{Entry}{哼}{10}{⼝}
  \begin{Phonetics}{哼}{heng1}[][HSK 7-9]
    \definition{v.}{gemer; bufar | cantarolar}
  \end{Phonetics}
  \begin{Phonetics}{哼}{hng5}
    \definition{interj.}{Hmm; Humph; expressa insatisfação, desprezo, desdém ou indignação}
  \end{Phonetics}
\end{Entry}

%%%%%%%%%% 唇 %%%%%%%%%%
\subsection*{唇}

\begin{Entry}{唇}{10}{⼝}
  \begin{Phonetics}{唇}{chun2}
    \definition[片]{s.}{lábios}
  \end{Phonetics}
\end{Entry}

%%%%%%%%%% 唉 %%%%%%%%%%
\subsection*{唉}

\begin{Entry}{唉}{10}{⼝}
  \begin{Phonetics}{唉}{ai1}
    \definition{interj.}{Sim!; Certo!; Bem! | Ai de mim!; o som dos suspiros}
  \end{Phonetics}
  \begin{Phonetics}{唉}{ai4}[][HSK 7-9]
    \definition{interj.}{Oh!; Ah!; Bem!; interjeição que expressa tristeza ou arrependimento | Bem!; Argh!; usado para responder ou reconhecer}
  \end{Phonetics}
\end{Entry}

%%%%%%%%%% 唐 %%%%%%%%%%
\subsection*{唐}

\begin{Entry}{唐}{10}{⼝}
  \begin{Phonetics}{唐}{tang2}
    \definition*{s.}{Dinastia estabelecida pelo Imperador Yao, 尧, no período lendário da história chinesa | Dinastia Tang (618-907) | Dinastia Tang posterior (923-936), uma das cinco dinastias | Sobrenome: Tang}
    \definition{adj.}{exagerado; bombástico; orgulhoso | em vão; por nada}
  \seealsoref{尧}{yao2}
  \end{Phonetics}
\end{Entry}

\begin{Entry}{唐人街}{10,2,12}{⼝、⼈、⾏}
  \begin{Phonetics}{唐人街}{tang2ren2 jie1}
    \definition*[条,座]{s.}{Bairro Chinês; Chinatown; refere-se ao mercado de rua onde os chineses do exterior vivem e abrem muitas lojas com características chinesas}
  \seealsoref{中国城}{zhong1guo2cheng2}
  \end{Phonetics}
\end{Entry}

\begin{Entry}{唐初四大家}{10,7,5,3,10}{⼝、⾐、⼞、⼤、⼧}
  \begin{Phonetics}{唐初四大家}{tang2 chu1 si4 da4jia1}
    \definition*{s.}{Quatro grandes calígrafos do início da dinastia Tang; refere-se a Yu Shi'nan 虞世南, Ouyang Xun 欧阳询, Chu Suiliang 褚遂良 e Xue Ji 薛稷}
  \seealsoref{褚遂良}{chu3 sui4liang2}
  \seealsoref{欧阳询}{ou1yang2 xun2}
  \seealsoref{薛稷}{xue1 ji4}
  \seealsoref{虞世南}{yu2 shi4'nan2}
  \end{Phonetics}
\end{Entry}

%%%%%%%%%% 唠 %%%%%%%%%%
\subsection*{唠}

\begin{Entry}{唠}{10}{⼝}
  \begin{Phonetics}{唠}{lao2}
    \definition{v.}{conversar; falar sobre}
  \end{Phonetics}
  \begin{Phonetics}{唠}{lao4}
    \definition{v.}{Dialeto: conversar; falar sobre | fofocar}
  \end{Phonetics}
\end{Entry}

\begin{Entry}{唠叨}{10,5}{⼝、⼝}
  \begin{Phonetics}{唠叨}{lao2dao5}[][HSK 7-9]
    \definition{v.}{tagarelar; ser loquaz; falar indefinidamente; ser interminável}
  \end{Phonetics}
\end{Entry}

%%%%%%%%%% 唤 %%%%%%%%%%
\subsection*{唤}

\begin{Entry}{唤}{10}{⼝}
  \begin{Phonetics}{唤}{huan4}
    \definition{v.}{chamar; fazer um barulho alto para fazer a outra parte acordar, prestar atenção ou vir até você}
  \end{Phonetics}
\end{Entry}

\begin{Entry}{唤起}{10,10}{⼝、⾛}
  \begin{Phonetics}{唤起}{huan4qi3}[][HSK 7-9]
    \definition{v.}{despertar | chamar; evocar}
  \end{Phonetics}
\end{Entry}

%%%%%%%%%% 啊 %%%%%%%%%%
\subsection*{啊}

\begin{Entry}{啊}{10}{⼝}
  \begin{Phonetics}{啊}{a1}[][HSK 2]
    \definition{interj.}{Ah!; Oh!; expressar surpresa ou admiração}
  \end{Phonetics}
  \begin{Phonetics}{啊}{a2}[][HSK 2]
    \definition{interj.}{Eh?; Ei?; Que?; Por que?; expressar questionamento, dúvida ou solicitar opinião}
  \end{Phonetics}
  \begin{Phonetics}{啊}{a3}[][HSK 2]
    \definition{interj.}{Eh?; Meu!; E aí?; Que?; expressar surpresa e dúvida}
  \end{Phonetics}
  \begin{Phonetics}{啊}{a4}[][HSK 2]
    \definition{interj.}{Bem!; Sim!; expressa concordância, pronúncia mais curta | Oh!; Ah!; indica que compreendeu, com pronúncia mais longa | Oh!; expressa surpresa ou admiração, com pronúncia mais longa | Desgraça!; expressa tristeza ou pesar}
  \end{Phonetics}
  \begin{Phonetics}{啊}{a5}[][HSK 2,4]
    \definition{part.}{usado no final da frase para expressar admiração | usado no final da frase para expressar afirmação, justificativa, insistência, recomendação, etc. | usado no final da frase para indicar dúvida | usado para fazer uma pequena pausa na frase, chamando a atenção para o que vem a seguir | usado após os itens enumerados | usado após verbos repetitivos, indica um processo longo}
  \end{Phonetics}
\end{Entry}

\begin{Entry}{啊呀}{10,7}{⼝、⼝}
  \begin{Phonetics}{啊呀}{a1ya1}
    \definition{interj.}{Oh meu Deus! | interjeição de surpresa}
  \end{Phonetics}
\end{Entry}

\begin{Entry}{啊哟}{10,9}{⼝、⼝}
  \begin{Phonetics}{啊哟}{a1yo5}
    \definition{interj.}{Meu Deus! | Oh! | Ai! | interjeição de surpresa ou dor}
  \end{Phonetics}
\end{Entry}

%%%%%%%%%% 圆 %%%%%%%%%%
\subsection*{圆}

\begin{Entry}{圆}{10}{⼞}
  \begin{Phonetics}{圆}{yuan2}[][HSK 4]
    \definition*{s.}{Sobrenome: Yuan}
    \definition{adj.}{redondo; circular; esférico; arredondado | diplomático; satisfatório}
    \definition[个,轮]{s.}{círculo; circunferência | uma moeda de valor e peso fixos}
    \definition{v.}{tornar plausível; justificar; tornar completo; completar}
  \end{Phonetics}
\end{Entry}

\begin{Entry}{圆珠笔}{10,10,10}{⼞、⽟、⽵}
  \begin{Phonetics}{圆珠笔}{yuan2 zhu1 bi3}[][HSK 6]
    \definition[支,枝]{s.}{caneta esferográfica}
  \end{Phonetics}
\end{Entry}

\begin{Entry}{圆满}{10,13}{⼞、⽔}
  \begin{Phonetics}{圆满}{yuan2man3}[][HSK 4]
    \definition{adj.}{perfeito; satisfatório; sem defeitos}
  \end{Phonetics}
\end{Entry}

%%%%%%%%%% 埋 %%%%%%%%%%
\subsection*{埋}

\begin{Entry}{埋}{10}{⼟}
  \begin{Phonetics}{埋}{mai2}[][HSK 6]
    \definition{v.}{cobrir (com terra, neve, etc.); enterrar | esconder | enterrar (uma pessoa morta)}
  \end{Phonetics}
  \begin{Phonetics}{埋}{man2}
    \definition{part.}{caracter formador de palavras}
  \end{Phonetics}
\end{Entry}

\begin{Entry}{埋伏}{10,6}{⼟、⼈}
  \begin{Phonetics}{埋伏}{mai2fu2}
    \definition{s.}{emboscada}
    \definition{v.}{emboscar}
  \end{Phonetics}
\end{Entry}

%%%%%%%%%% 壶 %%%%%%%%%%
\subsection*{壶}

\begin{Entry}{壶}{10}{⼠}
  \begin{Phonetics}{壶}{hu2}[][HSK 6]
    \definition*{s.}{Sobrenome: Hu}
    \definition[个,把]{s.}{chaleira; panela | garrafa; frasco; recipiente para líquidos}
  \end{Phonetics}
\end{Entry}

%%%%%%%%%% 夏 %%%%%%%%%%
\subsection*{夏}

\begin{Entry}{夏}{10}{⼢}
  \begin{Phonetics}{夏}{xia4}
    \definition*{s.}{Dinastia Xia (2070-1600 a.C.) | China; refere-se à China | Sobrenome: Xia}
    \definition{s.}{verão}
  \end{Phonetics}
\end{Entry}

\begin{Entry}{夏天}{10,4}{⼢、⼤}
  \begin{Phonetics}{夏天}{xia4 tian1}[][HSK 2]
    \definition[个]{s.}{verão}
  \end{Phonetics}
\end{Entry}

\begin{Entry}{夏日}{10,4}{⼢、⽇}
  \begin{Phonetics}{夏日}{xia4ri4}
    \definition{s.}{horário de verão}
  \end{Phonetics}
\end{Entry}

\begin{Entry}{夏季}{10,8}{⼢、⼦}
  \begin{Phonetics}{夏季}{xia4 ji4}[][HSK 4]
    \definition[个]{s.}{verão; segundo trimestre do ano, habitualmente chamado na China de período de três meses, do início do verão ao início do outono, também chamado de ``quarto, quinto e sexto'' meses do calendário lunar}
  \end{Phonetics}
\end{Entry}

%%%%%%%%%% 套 %%%%%%%%%%
\subsection*{套}

\begin{Entry}{套}{10}{⼤}
  \begin{Phonetics}{套}{tao4}[][HSK 2]
    \definition{clas.}{usado para coisas agrupadas: conjuntos, coleções, séries, etc.}
    \definition{s.}{estojo; capa; bainha | local onde o rio ou a cordilheira faz uma curva (usado principalmente em nomes de lugares) | enchimento de algodão em roupas e edredons | arreios; corda para amarrar animais | nó; laço; um objeto circular feito com corda ou algo semelhante | cortersia; convenção; conversa fiada; métodos repetitivos | armadilha; truque; conspiração}
    \definition{v.}{sobrepor; interligar | deslizar sobre; cobrir por fora | atrelar; engatar; usar um cinto de segurança | copiar; imitar; seguir o modelo de | extrair; induzir a falar; persuadir alguém a revelar um segredo; induzir; provocar | tentar vencer; aproximar-se de; aproximar-se intencionalmente de outras pessoas para algum propósito | fazer a rosca de um parafuso; usar um macho de rosca ou uma chave de rosca para fazer roscas}
  \end{Phonetics}
\end{Entry}

\begin{Entry}{套问}{10,6}{⼤、⾨}
  \begin{Phonetics}{套问}{tao4wen4}
    \definition{s.}{retórica}
    \definition{v.}{descobrir por meio de questionamento indireto diplomático}
  \end{Phonetics}
\end{Entry}

\begin{Entry}{套餐}{10,16}{⼤、⾷}
  \begin{Phonetics}{套餐}{tao4 can1}[][HSK 4]
    \definition{s.}{combo; pacote de produtos; pacote de serviços; metaforicamente, bens ou projetos que são combinados e levados ao mercado | refeição preparada; pacotes de refeições completos}
  \end{Phonetics}
\end{Entry}

%%%%%%%%%% 娱 %%%%%%%%%%
\subsection*{娱}

\begin{Entry}{娱}{10}{⼥}
  \begin{Phonetics}{娱}{yu2}
    \definition{s.}{alegria; prazer; diversão; felicidade}
    \definition{v.}{dar prazer a; divertir; fazer feliz}
  \end{Phonetics}
\end{Entry}

\begin{Entry}{娱乐}{10,5}{⼥、⼃}
  \begin{Phonetics}{娱乐}{yu2le4}[][HSK 6]
    \definition[项]{s.}{entretenimento; diversão; recreação; passa-tempo; atividades recreativas, prazerosas e divertidas}
    \definition{v.}{recrear; divertir; distrair; entreter; passar o tempo}
  \end{Phonetics}
\end{Entry}

%%%%%%%%%% 孬 %%%%%%%%%%
\subsection*{孬}

\begin{Entry}{孬}{10}{⼥}
  \begin{Phonetics}{孬}{nao1}
    \definition{adj.}{ruim | covarde | Dialeto: não (é) bom (contração de 不 + 好)}
  \seealsoref{不}{bu4}
  \seealsoref{好}{hao3}
  \end{Phonetics}
\end{Entry}

%%%%%%%%%% 害 %%%%%%%%%%
\subsection*{害}

\begin{Entry}{害}{10}{⼧}
  \begin{Phonetics}{害}{hai4}[][HSK 5]
    \definition{adj.}{prejudicial; destrutivo; injurioso; nocivo}
    \definition{s.}{mal; maldade; dano; calamidade}
    \definition{v.}{prejudicar; fazer mal a; causar problemas a | matar; assassinar | sofrer de; contrair (uma doença) | sentir-se (envergonhado, com medo, etc.); despertar (um sentimento ou uma emoção)}
  \end{Phonetics}
\end{Entry}

\begin{Entry}{害虫}{10,6}{⼧、⾍}
  \begin{Phonetics}{害虫}{hai4chong2}[][HSK 7-9]
    \definition[种,只,个]{s.}{verme; bicho; inseto nocivo (ou destrutivo); praga (oposto a 益虫)}
  \seealsoref{益虫}{yi4chong2}
  \end{Phonetics}
\end{Entry}

\begin{Entry}{害怕}{10,8}{⼧、⼼}
  \begin{Phonetics}{害怕}{hai4pa4}[][HSK 3]
    \definition{v.}{estar assustado; ter medo; encontrar dificuldades, perigos, etc., e sentir-se inquieto ou nervoso}
  \end{Phonetics}
\end{Entry}

\begin{Entry}{害羞}{10,10}{⼧、⽺}
  \begin{Phonetics}{害羞}{hai4/xiu1}[][HSK 7-9]
    \definition{v.+compl.}{ser tímido; parecer tímido; tornar-se tímido}
  \end{Phonetics}
\end{Entry}

\begin{Entry}{害臊}{10,17}{⼧、⾁}
  \begin{Phonetics}{害臊}{hai4/sao4}[][HSK 7-9]
    \definition{v.+compl.}{sentir vergonha; ser tímido}
  \end{Phonetics}
\end{Entry}

%%%%%%%%%% 宴 %%%%%%%%%%
\subsection*{宴}

\begin{Entry}{宴}{10}{⼧}
  \begin{Phonetics}{宴}{yan4}
    \definition{adj.}{tranquilo e confortável}
    \definition[个,场]{s.}{festa; banquete | facilidade e conforto; felicidade; lazer}
    \definition{v.}{entreter em um banquete; festejar}
  \end{Phonetics}
\end{Entry}

\begin{Entry}{宴会}{10,6}{⼧、⼈}
  \begin{Phonetics}{宴会}{yan4hui4}[][HSK 6]
    \definition[个,场,次]{s.}{festa; banquete; jantar; uma reunião onde convidados e anfitriões bebem e comem juntos (referindo-se a uma ocasião mais solene)}
  \end{Phonetics}
\end{Entry}

%%%%%%%%%% 家 %%%%%%%%%%
\subsection*{家}

\begin{Entry}{家}{10}{⼧}
  \begin{Phonetics}{家}{jia1}[][HSK 1,2]
    \definition*{s.}{Sobrenome: Jia}
    \definition{adj.}{domado; domesticado; criado; alimentado | interno}
    \definition{clas.}{usado para famílias ou estabelecimentos comerciais; para uso doméstico; lojas; fábricas, etc.}
    \definition{pron.}{Educado: meu (irmã, tio, etc.)}
    \definition[个]{s.}{família; domicílio; clã | lar; casa; residência da família | pessoa ou família envolvida em um determinado comércio; pessoas que trabalham em determinada profissão ou que possuem determinada identidade | especialista em um determinado campo; pessoa que possui conhecimentos especializados ou se dedica a atividades específicas | escola de pensamento; rscola acadêmica | (em cartas de baralho, mah-jong etc.) festa; lado; refere-se a jogar xadrez ou cartas, em que uma das partes joga contra a outra | nacionalidade; referindo-se à etnia | membros da família; parentes; pessoas ou famílias com quem você tem algum tipo de relação | membro do mesmo clã; pessoas com o mesmo sobrenome}
    \definition{suf.}{sufixo substantivo para designar um especialista em alguma atividade, como um músico ou revolucionário, para designar uma profissão como em -eiro, -ista, por exemplo 科学家}
  \seealsoref{科学家}{ke1xue2jia1}
  \end{Phonetics}
\end{Entry}

\begin{Entry}{家人}{10,2}{⼧、⼈}
  \begin{Phonetics}{家人}{jia1 ren2}[][HSK 1]
    \definition{s.}{família (de alguém); membro da família; os membros de uma família}
  \end{Phonetics}
\end{Entry}

\begin{Entry}{家乡}{10,3}{⼧、⼄}
  \begin{Phonetics}{家乡}{jia1xiang1}[][HSK 3]
    \definition[片,座]{s.}{cidade natal; o lugar onde sua família vive há gerações}
  \end{Phonetics}
\end{Entry}

\begin{Entry}{家长}{10,4}{⼧、⾧}
  \begin{Phonetics}{家长}{jia1 zhang3}[][HSK 2]
    \definition[位,名,个]{s.}{pais; patriarca; tutor; guardião; refere-se aos pais ou outros responsáveis legais}
  \end{Phonetics}
\end{Entry}

\begin{Entry}{家务}{10,5}{⼧、⼒}
  \begin{Phonetics}{家务}{jia1wu4}[][HSK 4]
    \definition[堆,次,件]{s.}{trabalho doméstico; tarefas domésticas}
  \end{Phonetics}
\end{Entry}

\begin{Entry}{家用}{10,5}{⼧、⽤}
  \begin{Phonetics}{家用}{jia1yong4}[][HSK 7-9]
    \definition{adj.}{doméstico; para uso doméstico}
    \definition[本]{s.}{despesas familiares; dinheiro para manutenção da casa}
    \definition{v.}{ser usado em casa; para uso doméstico}
  \end{Phonetics}
\end{Entry}

\begin{Entry}{家用电器}{10,5,5,16}{⼧、⽤、⽥、⼝}
  \begin{Phonetics}{家用电器}{jia1yong4 dian4qi4}
    \definition{s.}{eletrodoméstico; refere-se a diversos aparelhos elétricos utilizados na vida doméstica e coletiva}
  \end{Phonetics}
\end{Entry}

\begin{Entry}{家电}{10,5}{⼧、⽥}
  \begin{Phonetics}{家电}{jia1 dian4}[][HSK 6]
    \definition[件,台]{s.}{eletrodomésticos, abreviação de 家用电器}
  \seealsoref{家用电器}{jia1yong4 dian4qi4}
  \end{Phonetics}
\end{Entry}

\begin{Entry}{家伙}{10,6}{⼧、⼈}
  \begin{Phonetics}{家伙}{jia1huo5}[][HSK 7-9]
    \definition[些,个,群,帮]{s.}{ferramenta; utensílio; arma; refere-se a ferramentas ou armas | cara; companheiro; refere-se a pessoas (com desprezo ou humor)  | gado; animal doméstico}
  \end{Phonetics}
\end{Entry}

\begin{Entry}{家园}{10,7}{⼧、⼞}
  \begin{Phonetics}{家园}{jia1 yuan2}[][HSK 6]
    \definition{s.}{casa; terra natal; um jardim em casa, geralmente referindo-se à cidade natal ou à família}
  \end{Phonetics}
\end{Entry}

\begin{Entry}{家里}{10,7}{⼧、⾥}
  \begin{Phonetics}{家里}{jia1 li3}[][HSK 1]
    \definition{s.}{(em) casa; (em sua) família | esposa}
  \end{Phonetics}
\end{Entry}

\begin{Entry}{家具}{10,8}{⼧、⼋}
  \begin{Phonetics}{家具}{jia1ju4}[][HSK 3]
    \definition[件,套,些,个]{s.}{móveis; mobiliário de casa; utensílios domésticos, incluem principalmente camas, mesas, cadeiras, armários, etc.}
  \end{Phonetics}
\end{Entry}

\begin{Entry}{家庭}{10,9}{⼧、⼴}
  \begin{Phonetics}{家庭}{jia1ting2}[][HSK 2]
    \definition[个,户]{s.}{família}
  \end{Phonetics}
\end{Entry}

\begin{Entry}{家政}{10,9}{⼧、⽁}
  \begin{Phonetics}{家政}{jia1zheng4}[][HSK 7-9]
    \definition{s.}{tarefas domésticas; trabalho de gestão doméstica}
  \end{Phonetics}
\end{Entry}

\begin{Entry}{家俱}{10,10}{⼧、⼈}
  \begin{Phonetics}{家俱}{jia1ju4}
    \definition{s.}{mobília}
  \end{Phonetics}
\end{Entry}

\begin{Entry}{家家户户}{10,10,4,4}{⼧、⼧、⼾、⼾}
  \begin{Phonetics}{家家户户}{jia1jia1hu4hu4}[][HSK 7-9]
    \definition{expr.}{cada família; cada lar}
  \end{Phonetics}
\end{Entry}

\begin{Entry}{家教}{10,11}{⼧、⽁}
  \begin{Phonetics}{家教}{jia1jiao4}[][HSK 7-9]
    \definition[个,名,位]{s.}{educação; educação familiar; disciplina doméstica; a educação de moral e etiqueta que os pais dão aos seus filhos |  tutor}
  \end{Phonetics}
\end{Entry}

\begin{Entry}{家族}{10,11}{⼧、⽅}
  \begin{Phonetics}{家族}{jia1zu2}[][HSK 7-9]
    \definition[个]{s.}{clã; família; uma organização social baseada em relações de sangue, incluindo várias gerações do mesmo sangue}
  \end{Phonetics}
\end{Entry}

\begin{Entry}{家喻户晓}{10,12,4,10}{⼧、⼝、⼾、⽇}
  \begin{Phonetics}{家喻户晓}{jia1yu4-hu4xiao3}[][HSK 7-9]
    \definition{expr.}{nome familiar | conhecido por todas as famílias; amplamente conhecido; conhecido por todos}[他是家喻户晓的演员。===Ele é um ator famoso.]
  \end{Phonetics}
\end{Entry}

\begin{Entry}{家属}{10,12}{⼧、⼫}
  \begin{Phonetics}{家属}{jia1shu3}[][HSK 3]
    \definition{s.}{membros da família; dependentes (familiares); os membros da família que não sejam o próprio chefe da família, ou seja, os membros da família que não sejam o próprio trabalhador}
  \end{Phonetics}
\end{Entry}

\begin{Entry}{家禽}{10,12}{⼧、⽱}
  \begin{Phonetics}{家禽}{jia1qin2}[][HSK 7-9]
    \definition{s.}{aves domésticas | ave; pássaro doméstico}
  \end{Phonetics}
\end{Entry}

\begin{Entry}{家境}{10,14}{⼧、⼟}
  \begin{Phonetics}{家境}{jia1jing4}[][HSK 7-9]
    \definition{s.}{situação financeira familiar; circunstâncias familiares}
  \end{Phonetics}
\end{Entry}

%%%%%%%%%% 容 %%%%%%%%%%
\subsection*{容}

\begin{Entry}{容}{10}{⼧}
  \begin{Phonetics}{容}{rong2}
    \definition*{s.}{Sobrenome: Rong}
    \definition{adv.}{talvez; provavelmente; possivelmente}
    \definition{s.}{expressão facial e tez | aparência; o estado ou condição das coisas}
    \definition{v.}{permitir; quando os outros querem fazer algo, deixe-os fazer | tolerar; ser capaz de aceitar pessoas ou coisas com as quais você não está satisfeito | conter (número de pessoas ou coisas que podem ser colocadas em um determinado espaço)}
  \end{Phonetics}
\end{Entry}

\begin{Entry}{容易}{10,8}{⼧、⽇}
  \begin{Phonetics}{容易}{rong2yi4}[][HSK 3]
    \definition{adj.}{fácil; simples; sem complicações | provável; passível; inclinado; indica uma alta probabilidade de algo acontecer}
  \end{Phonetics}
\end{Entry}

\begin{Entry}{容貌}{10,14}{⼧、⾘}
  \begin{Phonetics}{容貌}{rong2mao4}
    \definition{s.}{aparência | aspecto | características}
  \end{Phonetics}
\end{Entry}

%%%%%%%%%% 宽 %%%%%%%%%%
\subsection*{宽}

\begin{Entry}{宽}{10}{⼧}
  \begin{Phonetics}{宽}{kuan1}[][HSK 4]
    \definition*{s.}{Sobrenome: Kuan}
    \definition{adj.}{largo; amplo; espaçoso; grandes distâncias horizontais (em oposição a 窄) | leniente; generoso; indulgente | bem de vida; rico; confortável}
    \definition[米]{s.}{largura; amplitude}[桌子有一米宽。===A mesa tem um metro de largura.]
    \definition{v.}{relaxar; aliviar}
  \seealsoref{窄}{zhai3}
  \end{Phonetics}
\end{Entry}

\begin{Entry}{宽广}{10,3}{⼧、⼴}
  \begin{Phonetics}{宽广}{kuan1 guang3}[][HSK 4]
    \definition{adj.}{vasto; amplo; espaçoso; extenso}
  \end{Phonetics}
\end{Entry}

\begin{Entry}{宽泛}{10,7}{⼧、⽔}
  \begin{Phonetics}{宽泛}{kuan1fan4}[][HSK 7-9]
    \definition{adj.}{abrangente; (conteúdo, significado) abrange uma ampla gama de aspectos}
  \end{Phonetics}
\end{Entry}

\begin{Entry}{宽松}{10,8}{⼧、⽊}
  \begin{Phonetics}{宽松}{kuan1song1}[][HSK 7-9]
    \definition{adj.}{(roupas) folgado e confortável; espaçoso e sem aglomeração; relaxante e sem aperto | (estado mental, atmosfera, etc.) relaxado; aliviado; sem tensão; livre de preocupações; descontraído; não há tensão | Economia: abundante; que tem dinheiro suficiente para viver bem e sem problemas, mas sem ser excessivamente rico}
  \end{Phonetics}
\end{Entry}

\begin{Entry}{宽厚}{10,9}{⼧、⼚}
  \begin{Phonetics}{宽厚}{kuan1hou4}[][HSK 7-9]
    \definition{adj.}{largo e espesso; amplo e sólido | tolerante; gentil e generoso; tolerância e bondade | simples; sincero; (voz) profunda e ressonante}
  \end{Phonetics}
\end{Entry}

\begin{Entry}{宽度}{10,9}{⼧、⼴}
  \begin{Phonetics}{宽度}{kuan1 du4}[][HSK 5]
    \definition{s.}{largura; amplitude; duração; o grau de largura e estreiteza; a distância horizontal (no caso de um retângulo, a distância entre os dois lados mais longos)}
  \end{Phonetics}
\end{Entry}

\begin{Entry}{宽容}{10,10}{⼧、⼧}
  \begin{Phonetics}{宽容}{kuan1rong2}[][HSK 7-9]
    \definition{adj.}{tolerante; generoso e magnânimo, não mesquinho ou oportunista}
    \definition{v.}{tolerar; ter paciência com; ser tolerante com os outros, não guardar rancor nem insistir no assunto}
  \end{Phonetics}
\end{Entry}

\begin{Entry}{宽恕}{10,10}{⼧、⼼}
  \begin{Phonetics}{宽恕}{kuan1shu4}[][HSK 7-9]
    \definition{v.}{perdoar; desculpar; absolver}
  \end{Phonetics}
\end{Entry}

\begin{Entry}{宽敞}{10,12}{⼧、⽁}
  \begin{Phonetics}{宽敞}{kuan1chang5}[][HSK 7-9]
    \definition{adj.}{amplo; espaçoso; descreve um espaço ou área grande}
  \end{Phonetics}
\end{Entry}

\begin{Entry}{宽阔}{10,12}{⼧、⾨}
  \begin{Phonetics}{宽阔}{kuan1 kuo4}[][HSK 6]
    \definition{adj.}{amplo; largo; espaçoso | tolerante; mente aberta; descreve uma mente alegre e ampla}
  \end{Phonetics}
\end{Entry}

\begin{Entry}{宽影片}{10,15,4}{⼧、⼺、⽚}
  \begin{Phonetics}{宽影片}{kuan1ying3pian4}
    \definition{s.}{filme \emph{widescreen}}
  \end{Phonetics}
\end{Entry}

%%%%%%%%%% 宾 %%%%%%%%%%
\subsection*{宾}

\begin{Entry}{宾}{10}{⼧}
  \begin{Phonetics}{宾}{bin1}
    \definition*{s.}{Sobrenome: Bin}
    \definition[个,位,名,些]{s.}{convidado}
  \end{Phonetics}
\end{Entry}

\begin{Entry}{宾馆}{10,11}{⼧、⾷}
  \begin{Phonetics}{宾馆}{bin1guan3}[][HSK 5]
    \definition[家,个,座]{s.}{hotel; acomodações públicas para hóspedes}
  \end{Phonetics}
\end{Entry}

%%%%%%%%%% 射 %%%%%%%%%%
\subsection*{射}

\begin{Entry}{射}{10}{⼨}
  \begin{Phonetics}{射}{she4}[][HSK 5]
    \definition*{s.}{Sobrenome: She}
    \definition{v.}{atirar; disparar | descarregar em jato; jorrar | emitir (luz, calor, etc.) | irradiar | aludir a algo ou alguém; insinuar}
  \end{Phonetics}
\end{Entry}

\begin{Entry}{射击}{10,5}{⼨、⼐}
  \begin{Phonetics}{射击}{she4ji1}[][HSK 5]
    \definition{s.}{tiro; tiro ao alvo}
    \definition{v.}{disparar; atirar}
  \end{Phonetics}
\end{Entry}

%%%%%%%%%% 展 %%%%%%%%%%
\subsection*{展}

\begin{Entry}{展}{10}{⼫}
  \begin{Phonetics}{展}{zhan3}
    \definition*{s.}{Sobrenome: Zhan}
    \definition{s.}{exposição}
    \definition{v.}{abrir; espalhar; desdobrar | fazer bom uso; dar liberdade para | adiar; estender; prolongar | expandir | abrir; deixar ir | exibir; mostrar}
  \end{Phonetics}
\end{Entry}

\begin{Entry}{展开}{10,4}{⼫、⼶}
  \begin{Phonetics}{展开}{zhan3kai1}[][HSK 3]
    \definition{s.}{desenvolvimento; expansão; explosão; evolução}
    \definition{v.}{espalhar; desdobrar; abrir | lançar; desdobrar; desenvolver; realizar em grande escala | espalhar; desenrolar; amplificar; desenvolver; expandir; explodir; evoluir; alongar}
  \end{Phonetics}
\end{Entry}

\begin{Entry}{展示}{10,5}{⼫、⽰}
  \begin{Phonetics}{展示}{zhan3shi4}[][HSK 5]
    \definition{v.}{mostrar; revelar; pôr a nu; abrir diante de alguém; expor claramente; manifestar de forma evidente}
  \end{Phonetics}
\end{Entry}

\begin{Entry}{展现}{10,8}{⼫、⾒}
  \begin{Phonetics}{展现}{zhan3xian4}[][HSK 5]
    \definition{v.}{mostrar; surgir; manifestar}
  \end{Phonetics}
\end{Entry}

\begin{Entry}{展览}{10,9}{⼫、⾒}
  \begin{Phonetics}{展览}{zhan3lan3}[][HSK 5]
    \definition[个,次,场]{s.}{exposição; exibição; atividades expostas; itens expostos}
    \definition{v.}{mostrar; exibir; expor; expor algo para que as pessoas vejam}
  \end{Phonetics}
\end{Entry}

%%%%%%%%%% 峰 %%%%%%%%%%
\subsection*{峰}

\begin{Entry}{峰}{10}{⼭}
  \begin{Phonetics}{峰}{feng1}
    \definition{clas.}{usado para camelos}
    \definition{s.}{pico; cume; o pico proeminente de uma montanha | coisa parecida com um pico; coisas em forma de montanhas}
  \end{Phonetics}
\end{Entry}

\begin{Entry}{峰会}{10,6}{⼭、⼈}
  \begin{Phonetics}{峰会}{feng1 hui4}[][HSK 6]
    \definition{s.}{cúpula; reunião de cúpula}
  \end{Phonetics}
\end{Entry}

\begin{Entry}{峰回路转}{10,6,13,8}{⼭、⼞、⾜、⾞}
  \begin{Phonetics}{峰回路转}{feng1hui2-lu4zhuan3}[][HSK 7-9]
    \definition{expr.}{cume em meio a elevações circundantes e estradas sinuosas;  (estrada de montanha) torcendo e virando; a estrada da montanha serpenteia em torno de cada novo pico | boa (ou nova) reviravolta nos acontecimentos; uma oportunidade surgiu inesperadamente; as coisas tomaram um novo rumo}
  \end{Phonetics}
\end{Entry}

%%%%%%%%%% 席 %%%%%%%%%%
\subsection*{席}

\begin{Entry}{席}{10}{⼱}
  \begin{Phonetics}{席}{xi2}
    \definition*{s.}{Sobrenome: Xi}
    \definition[卷,张]{s.}{esteira | assento; lugar; caixa | assento (em uma assembleia legislativa) | festim; banquete; jantar}
  \end{Phonetics}
\end{Entry}

\begin{Entry}{席卷}{10,8}{⼱、⼙}
  \begin{Phonetics}{席卷}{xi2juan3}
    \definition{v.}{engolfar | varrer | levar tudo para fora}
  \end{Phonetics}
\end{Entry}

%%%%%%%%%% 座 %%%%%%%%%%
\subsection*{座}

\begin{Entry}{座}{10}{⼴}
  \begin{Phonetics}{座}{zuo4}[][HSK 2]
    \definition{clas.}{usado para montanhas, edifícios e objetos imóveis semelhantes}
    \definition{s.}{assento; lugar | suporte; pedestal; base | (astronomia) constalação | (antigo) forma de tratamento a altos funcionários |}
  \end{Phonetics}
\end{Entry}

\begin{Entry}{座子}{10,3}{⼴、⼦}
  \begin{Phonetics}{座子}{zuo4zi5}
    \definition{s.}{soquete | pedestal | sela}
  \end{Phonetics}
\end{Entry}

\begin{Entry}{座位}{10,7}{⼴、⼈}
  \begin{Phonetics}{座位}{zuo4wei4}[][HSK 2]
    \definition[个,排]{s.}{assento; lugar}
  \end{Phonetics}
\end{Entry}

\begin{Entry}{座标}{10,9}{⼴、⽊}
  \begin{Phonetics}{座标}{zuo4biao1}
    \variantof{坐标}
  \end{Phonetics}
\end{Entry}

\begin{Entry}{座谈会}{10,10,6}{⼴、⾔、⼈}
  \begin{Phonetics}{座谈会}{zuo4 tan2 hui4}[][HSK 6]
    \definition{s.}{fórum; simpósio; discussão informal | conferência | sessão de rap}
  \end{Phonetics}
\end{Entry}

%%%%%%%%%% 弱 %%%%%%%%%%
\subsection*{弱}

\begin{Entry}{弱}{10}{⼸}
  \begin{Phonetics}{弱}{ruo4}[][HSK 4]
    \definition{adj.}{fraco; debilitado | jovem | inferior; pior | colocado depois de uma fração ou decimal para indicar que é um pouco menor que esse número (oposto a 强)}
    \definition{v.}{perder (através da morte)}
  \seealsoref{强}{qiang2}
  \end{Phonetics}
\end{Entry}

%%%%%%%%%% 徐 %%%%%%%%%%
\subsection*{徐}

\begin{Entry}{徐}{10}{⼻}
  \begin{Phonetics}{徐}{xu2}
    \definition*{s.}{Sobrenome: Xu}
    \definition{adv.}{lentamente; suavemente}
  \end{Phonetics}
\end{Entry}

\begin{Entry}{徐徐}{10,10}{⼻、⼻}
  \begin{Phonetics}{徐徐}{xu2xu2}
    \definition{adv.}{lentamente; suavemente}
  \end{Phonetics}
\end{Entry}

%%%%%%%%%% 徒 %%%%%%%%%%
\subsection*{徒}

\begin{Entry}{徒}{10}{⼻}
  \begin{Phonetics}{徒}{tu2}
    \definition{adj.}{vazio; nu}
    \definition{adv.}{somente; meramente; apenas | a pé | em vão; sem sucesso; sem sucesso}
    \definition{s.}{aprendiz; aluno | seguidor; crente | (pejorativo) pessoas da mesma facção | (pejorativo) pessoa; companheiro | (prisão) pena; prisão; sentença | estudante}
    \definition{v.}{estar a pé | andar}
  \end{Phonetics}
\end{Entry}

\begin{Entry}{徒手}{10,4}{⼻、⼿}
  \begin{Phonetics}{徒手}{tu2shou3}
    \definition{adj.}{com as mãos vazias | desarmado | mão livre (desenho) | lutando mão-a-mão}
  \end{Phonetics}
\end{Entry}

\begin{Entry}{徒弟}{10,7}{⼻、⼸}
  \begin{Phonetics}{徒弟}{tu2di4}[][HSK 6]
    \definition[位,名,个]{s.}{discípulo; aprendiz; uma pessoa que aprende com um mestre; geralmente se refere a uma pessoa que aprende com um especialista}[他是我的徒弟。===Ele é meu aprendiz.]
  \end{Phonetics}
\end{Entry}

%%%%%%%%%% 恋 %%%%%%%%%%
\subsection*{恋}

\begin{Entry}{恋}{10}{⼼}
  \begin{Phonetics}{恋}{lian4}
    \definition*{s.}{Sobrenome: Lian}
    \definition{v.}{amor (romântico) | ansiar por; sentir-se apegado a | amar; apaixonar-se por | não querendo se separar de; sentir sua falta para sempre; não suportar ficar separado}
  \end{Phonetics}
\end{Entry}

\begin{Entry}{恋恋不舍}{10,10,4,8}{⼼、⼼、⼀、⾆}
  \begin{Phonetics}{恋恋不舍}{lian4lian4-bu4she4}[][HSK 7-9]
    \definition{expr.}{afastar-se de algo; relutar em se separar; sentir-se apegado a algo; descrevendo a relutância em partir}
  \end{Phonetics}
\end{Entry}

\begin{Entry}{恋爱}{10,10}{⼼、⽖}
  \begin{Phonetics}{恋爱}{lian4'ai4}[][HSK 5]
    \definition[个,场,段]{s.}{namoro; afeto; amor romântico; ações que demonstram o amor mútuo}
    \definition{v.}{amar; estar apaixonado}
  \end{Phonetics}
\end{Entry}

%%%%%%%%%% 恐 %%%%%%%%%%
\subsection*{恐}

\begin{Entry}{恐}{10}{⼼}
  \begin{Phonetics}{恐}{kong3}
    \definition{adv.}{talvez; provavelmente}
    \definition{v.}{temer; recear; ter medo de | ameaçar; aterrorizar; intimidar}
  \end{Phonetics}
\end{Entry}

\begin{Entry}{恐龙}{10,5}{⼼、⿓}
  \begin{Phonetics}{恐龙}{kong3long2}[][HSK 7-9]
    \definition[只,头]{s.}{dinossauro | garota feia (gíria da \emph{Internet}, ofensiva)}
  \end{Phonetics}
\end{Entry}

\begin{Entry}{恐吓}{10,6}{⼼、⼝}
  \begin{Phonetics}{恐吓}{kong3he4}[][HSK 7-9]
    \definition{v.}{ameaçar; assustar; intimidar; ameaçar alguém com palavras ou meios ameaçadores}
  \end{Phonetics}
\end{Entry}

\begin{Entry}{恐怕}{10,8}{⼼、⼼}
  \begin{Phonetics}{恐怕}{kong3pa4}[][HSK 3]
    \definition{adv.}{talvez; provavelmente; pode ser; expressa suposição; estimativa. | por medo de; expressar estimativa e preocupação}
    \definition{v.}{ter medo de; temer; recear}
  \end{Phonetics}
\end{Entry}

\begin{Entry}{恐怖}{10,8}{⼼、⼼}
  \begin{Phonetics}{恐怖}{kong3bu4}[][HSK 7-9]
    \definition[部]{adj.}{terrível; aterrador; horripilante; medo causado por ameaças à vida ou por presenciar violência ou derramamento de sangue | assustador; aterrorizante | terroristas; o comportamento ou os métodos utilizados são extremamente cruéis e perversos, causando choque e medo}
  \end{Phonetics}
\end{Entry}

\begin{Entry}{恐怖主义}{10,8,5,3}{⼼、⼼、⼂、⼂}
  \begin{Phonetics}{恐怖主义}{kong3bu4zhu3yi4}
    \definition{adj.}{terrorista}
    \definition{s.}{terrorismo}
  \end{Phonetics}
\end{Entry}

\begin{Entry}{恐惧}{10,11}{⼼、⼼}
  \begin{Phonetics}{恐惧}{kong3ju4}[][HSK 7-9]
    \definition{adj.}{assustado; com medo; muito assustado}
  \end{Phonetics}
\end{Entry}

\begin{Entry}{恐慌}{10,12}{⼼、⼼}
  \begin{Phonetics}{恐慌}{kong3huang1}[][HSK 7-9]
    \definition{adj.}{pânico; em pânico; pânico devido ao medo}
    \definition{s.}{pânico; medo}
  \end{Phonetics}
\end{Entry}

%%%%%%%%%% 恩 %%%%%%%%%%
\subsection*{恩}

\begin{Entry}{恩}{10}{⼼}
  \begin{Phonetics}{恩}{en1}
    \definition*{s.}{Sobrenome: En}
    \definition{s.}{bondade; favor; graça; gentileza}
  \end{Phonetics}
\end{Entry}

\begin{Entry}{恩人}{10,2}{⼼、⼈}
  \begin{Phonetics}{恩人}{en1 ren2}[][HSK 6]
    \definition{s.}{benfeitor; uma pessoa que ajudou significativamente alguém}
  \end{Phonetics}
\end{Entry}

\begin{Entry}{恩怨}{10,9}{⼼、⼼}
  \begin{Phonetics}{恩怨}{en1yuan4}[][HSK 7-9]
    \definition{s.}{sentimento de gratidão ou ressentimento (inimizade) | ressentimento; queixa; velhas contas}
  \end{Phonetics}
\end{Entry}

\begin{Entry}{恩情}{10,11}{⼼、⼼}
  \begin{Phonetics}{恩情}{en1qing2}[][HSK 7-9]
    \definition{s.}{amor; bondade; afeição profunda}
  \end{Phonetics}
\end{Entry}

\begin{Entry}{恩惠}{10,12}{⼼、⼼}
  \begin{Phonetics}{恩惠}{en1hui4}[][HSK 7-9]
    \definition[份]{s.}{favor; generosidade | bondade; graça; benefícios concedidos ou recebidos}
  \end{Phonetics}
\end{Entry}

\begin{Entry}{恩赐}{10,12}{⼼、⾙}
  \begin{Phonetics}{恩赐}{en1ci4}[][HSK 7-9]
    \definition{s.}{favor; caridade; esmola}
    \definition{v.}{conceder (favores, caridade, etc.); recompensar}
  \end{Phonetics}
\end{Entry}

%%%%%%%%%% 恭 %%%%%%%%%%
\subsection*{恭}

\begin{Entry}{恭}{10}{⼼}
  \begin{Phonetics}{恭}{gong1}
    \definition{adj.}{respeitoso; reverente | educado}
  \end{Phonetics}
\end{Entry}

\begin{Entry}{恭维}{10,11}{⼼、⽷}
  \begin{Phonetics}{恭维}{gong1wei2}[][HSK 7-9]
    \definition{v.}{bajular; elogiar}
  \end{Phonetics}
\end{Entry}

\begin{Entry}{恭喜}{10,12}{⼼、⼝}
  \begin{Phonetics}{恭喜}{gong1xi3}[][HSK 7-9]
    \definition{v.}{parabenizar; uma maneira educada de parabenizar alguém por seu feliz evento}
  \end{Phonetics}
\end{Entry}

%%%%%%%%%% 恳 %%%%%%%%%%
\subsection*{恳}

\begin{Entry}{恳}{10}{⼼}
  \begin{Phonetics}{恳}{ken3}
    \definition{adj.}{sério; sincero | cordial; honesto}
    \definition{v.}{pedir; suplicar; implorar; rogar}
  \end{Phonetics}
\end{Entry}

\begin{Entry}{恳求}{10,7}{⼼、⽔}
  \begin{Phonetics}{恳求}{ken3qiu2}[][HSK 7-9]
    \definition{v.}{implorar; suplicar; rogar; solicitar encarecidamente}
  \end{Phonetics}
\end{Entry}

%%%%%%%%%% 恶 %%%%%%%%%%
\subsection*{恶}

\begin{Entry}{恶}{10}{⼼}
  \begin{Phonetics}{恶}{e3}
    \definition{part.}{elementos formadores de palavras}
  \end{Phonetics}
  \begin{Phonetics}{恶}{e4}[][HSK 7-9]
    \definition{adj.}{feroz | ruim; maligno; perverso | vicioso | feio | grosseiro}
    \definition{s.}{mal; vício; crime (oposto a 善) | maldade; comportamento muito ruim; coisas criminosas}
  \seealsoref{善}{shan4}
  \end{Phonetics}
  \begin{Phonetics}{恶}{wu1}
    \definition{interj.}{Droga!; Ah não!; expressa surpresa}
    \definition{pron.}{como?; por que?; refere-se a um lugar ou coisa; expressa uma pergunta retórica; equivalente a 何 ou 怎么}
  \seealsoref{何}{he2}
  \seealsoref{怎么}{zen3me5}
  \end{Phonetics}
  \begin{Phonetics}{恶}{wu4}
    \definition{v.}{não gostar; odiar; detestar; repugnar}
  \end{Phonetics}
\end{Entry}

\begin{Entry}{恶化}{10,4}{⼼、⼔}
  \begin{Phonetics}{恶化}{e4hua4}[][HSK 7-9]
    \definition{v.}{piorar; deteriorar; exacerbar | piorar a situação}
  \end{Phonetics}
\end{Entry}

\begin{Entry}{恶心}{10,4}{⼼、⼼}
  \begin{Phonetics}{恶心}{e3xin5}[][HSK 4]
    \definition{adj.}{nauseante; repugnante}
    \definition{s.}{náusea; repugnância}
    \definition{v.}{repugnar; ser nauseante; sentir-se mal | envergonhar (deliberadamente)}
  \end{Phonetics}
  \begin{Phonetics}{恶心}{e4xin1}
    \definition{s.}{mau hábito; hábito vicioso; vício}
  \end{Phonetics}
\end{Entry}

\begin{Entry}{恶劣}{10,6}{⼼、⼒}
  \begin{Phonetics}{恶劣}{e4lie4}[][HSK 7-9]
    \definition{adj.}{mau; odioso; abominável; repugnante; desprezível; muito mau; muito ruim}
  \end{Phonetics}
\end{Entry}

\begin{Entry}{恶性}{10,8}{⼼、⼼}
  \begin{Phonetics}{恶性}{e4xing4}[][HSK 7-9]
    \definition{adj.}{maligno; pernicioso; vicioso (oposto a 良性) | produzindo o mal | rápido (declínio) | descontrolada (inflação) | vicioso (círculo) | perverso}
  \seealsoref{良性}{liang2xing4}
  \end{Phonetics}
\end{Entry}

\begin{Entry}{恶意}{10,13}{⼼、⼼}
  \begin{Phonetics}{恶意}{e4yi4}[][HSK 7-9]
    \definition[丝]{s.}{malícia; má vontade; más intenções}
  \end{Phonetics}
\end{Entry}

%%%%%%%%%% 悄 %%%%%%%%%%
\subsection*{悄}

\begin{Entry}{悄}{10}{⼼}
  \begin{Phonetics}{悄}{qiao1}
    \definition{adj.}{quieto; silencioso}
  \end{Phonetics}
  \begin{Phonetics}{悄}{qiao3}
    \definition{adj.}{quieto; silencioso | triste; preocupado; aflito}
  \end{Phonetics}
\end{Entry}

\begin{Entry}{悄悄}{10,10}{⼼、⼼}
  \begin{Phonetics}{悄悄}{qiao1qiao1}[][HSK 5]
    \definition{adv.}{silenciosamente; em silêncio; aos sussuros; sem som ou em voz baixa; com o mínimo de ruído possível}
  \end{Phonetics}
\end{Entry}

%%%%%%%%%% 悔 %%%%%%%%%%
\subsection*{悔}

\begin{Entry}{悔}{10}{⼼}
  \begin{Phonetics}{悔}{hui3}
    \definition{v.}{lamentar; arrepender-se}
  \end{Phonetics}
\end{Entry}

\begin{Entry}{悔恨}{10,9}{⼼、⼼}
  \begin{Phonetics}{悔恨}{hui3hen4}[][HSK 7-9]
    \definition{v.}{arrepender-se profundamente; estar amargamente arrependido}
  \end{Phonetics}
\end{Entry}

%%%%%%%%%% 扇 %%%%%%%%%%
\subsection*{扇}

\begin{Entry}{扇}{10}{⼾}
  \begin{Phonetics}{扇}{shan1}[][HSK 5]
    \definition{s.}{ventilar; agitar um leque para fazer o ar circular | dar um tapa; bater com a palma da mão | bater asas; esvoaçar | incitar; instigar; estimular; agitar}
  \end{Phonetics}
  \begin{Phonetics}{扇}{shan4}[][HSK 5]
    \definition{clas.}{usado para portas, janelas, etc.}
    \definition[把]{s.}{leque | folha; algo em forma de placa ou folha}
  \end{Phonetics}
\end{Entry}

\begin{Entry}{扇子}{10,3}{⼾、⼦}
  \begin{Phonetics}{扇子}{shan4zi5}[][HSK 5]
    \definition[把,个]{s.}{leque; abano; abanador; utensílios que produzem vento ao serem agitados}
  \end{Phonetics}
\end{Entry}

%%%%%%%%%% 拳 %%%%%%%%%%
\subsection*{拳}

\begin{Entry}{拳}{10}{⼿}
  \begin{Phonetics}{拳}{quan2}
    \definition*{s.}{Sobrenome: Quan}
    \definition[个,记,套]{s.}{punho | boxe; pugilismo}
    \definition{v.}{enrolar}
  \end{Phonetics}
\end{Entry}

\begin{Entry}{拳王}{10,4}{⼿、⽟}
  \begin{Phonetics}{拳王}{quan2wang2}
    \definition{s.}{pugilista | boxeador}
  \end{Phonetics}
\end{Entry}

\begin{Entry}{拳法}{10,8}{⼿、⽔}
  \begin{Phonetics}{拳法}{quan2fa3}
    \definition{s.}{boxe | luta}
  \end{Phonetics}
\end{Entry}

%%%%%%%%%% 拿 %%%%%%%%%%
\subsection*{拿}

\begin{Entry}{拿}{10}{⼿}
  \begin{Phonetics}{拿}{na2}[][HSK 1]
    \definition{part.}{usado da mesma forma que 把: para marcar o seguinte substantivo seguinte como objeto direto}
    \definition{prep.}{ferramentas, materiais, métodos, etc. utilizados para a introdução | os objetos que estão sendo manipulados para introdução}
    \definition{v.}{segurar; pegar; pegar ou mover objetos com as mãos ou de outra forma | apreender; capturar; prender; usar força bruta para capturar | ter certeza de; ser capaz de fazer; ter uma compreensão firme de | tornar as coisas difíceis para alguém; colocar alguém em uma situação difícil; obstruir; chantagear; coagir; causar dificuldades intencionalmente | fingir ou fazer (algum tipo de postura ou aparência) | ter certeza de; tomar uma decisão | obter; ganhar; receber}
  \end{Phonetics}
\end{Entry}

\begin{Entry}{拿出}{10,5}{⼿、⼐}
  \begin{Phonetics}{拿出}{na2 chu1}[][HSK 2]
    \definition{v.}{apresentar (evidências) | fornecer | apresentar (uma proposta) | oferecer; servir | retirar; tirar}
  \end{Phonetics}
\end{Entry}

\begin{Entry}{拿走}{10,7}{⼿、⾛}
  \begin{Phonetics}{拿走}{na2 zou3}[][HSK 6]
    \definition{v.}{tirar; remover}
  \end{Phonetics}
\end{Entry}

\begin{Entry}{拿到}{10,8}{⼿、⼑}
  \begin{Phonetics}{拿到}{na2 dao4}[][HSK 2]
    \definition{v.}{pegar; obter, conseguir}
  \end{Phonetics}
\end{Entry}

%%%%%%%%%% 挨 %%%%%%%%%%
\subsection*{挨}

\begin{Entry}{挨}{10}{⼿}
  \begin{Phonetics}{挨}{ai1}
    \definition{prep.}{por turnos; em sequência; indica sequencialmente}
    \definition{v.}{estar próximo de; estar (ou chegar) perto de; abordar}
  \end{Phonetics}
  \begin{Phonetics}{挨}{ai2}[][HSK 6]
    \definition{v.}{sofrer; suportar | arrastar-se; lutar para sobreviver (tempos difíceis); passar (tempo) com dificuldade | parar; atrasar; adiar; procrastinar}
  \end{Phonetics}
\end{Entry}

\begin{Entry}{挨打}{10,5}{⼿、⼿}
  \begin{Phonetics}{挨打}{ai2/da3}[][HSK 6]
    \definition{v.+compl.}{levar uma surra; ser atacado; ser espancado}
  \end{Phonetics}
\end{Entry}

\begin{Entry}{挨家挨户}{10,10,10,4}{⼿、⼧、⼿、⼾}
  \begin{Phonetics}{挨家挨户}{ai1jia1-ai1hu4}[][HSK 7-9]
    \definition{expr.}{ir de casa em casa, de porta em porta ; um após o outro}
  \end{Phonetics}
\end{Entry}

\begin{Entry}{挨着}{10,11}{⼿、⽬}
  \begin{Phonetics}{挨着}{ai1 zhe5}[][HSK 6]
    \definition{adv.}{ao lado de; perto de; imediatamente depois}
  \end{Phonetics}
\end{Entry}

%%%%%%%%%% 挫 %%%%%%%%%%
\subsection*{挫}

\begin{Entry}{挫}{10}{⼿}
  \begin{Phonetics}{挫}{cuo4}
    \definition{v.}{frustrar | diminuir; embotar; desinflar | pressionar para baixo; abaixar}
  \end{Phonetics}
\end{Entry}

\begin{Entry}{挫折}{10,7}{⼿、⼿}
  \begin{Phonetics}{挫折}{cuo4zhe2}[][HSK 7-9]
    \definition[个,次]{s.}{retrocesso; reversão; frustração | derrota, fracasso, insucesso}
    \definition{v.}{falhar; derrotar; fracassar}
  \end{Phonetics}
\end{Entry}

%%%%%%%%%% 振 %%%%%%%%%%
\subsection*{振}

\begin{Entry}{振}{10}{⼿}
  \begin{Phonetics}{振}{zhen4}
    \definition{v.}{sacudir; acenar; bater as asas; empunhar | vibrar | recompor-se; levantar-se; animar}
  \end{Phonetics}
\end{Entry}

\begin{Entry}{振动}{10,6}{⼿、⼒}
  \begin{Phonetics}{振动}{zhen4dong4}[][HSK 5]
    \definition{s.}{vibração}
    \definition{v.}{sacudir; balançar; tremer; roncar; tagarelar; vibrar; oscilar; a física se refere ao movimento contínuo de um objeto em torno de um determinado ponto no espaço, como o movimento de um pêndulo, um diapasão ou uma corda de violão}
  \end{Phonetics}
\end{Entry}

%%%%%%%%%% 捆 %%%%%%%%%%
\subsection*{捆}

\begin{Entry}{捆}{10}{⼿}
  \begin{Phonetics}{捆}{kun3}[][HSK 7-9]
    \definition{clas.}{feixe; maço; materiais usados ​​para amarrar}
    \definition{s.}{coisas que estão agrupadas}
    \definition{v.}{amarrar; prender; agrupar | amarrar; acorrentar; algemar | agrupar; enfardar}
  \seealsoref{捆儿}{kun3r5}
  \end{Phonetics}
\end{Entry}

\begin{Entry}{捆儿}{10,2}{⼿、⼉}
  \begin{Phonetics}{捆儿}{kun3r5}
    \definition{s.}{coisas que estão agrupadas}
  \seealsoref{捆}{kun3}
  \end{Phonetics}
\end{Entry}

%%%%%%%%%% 捉 %%%%%%%%%%
\subsection*{捉}

\begin{Entry}{捉}{10}{⼿}
  \begin{Phonetics}{捉}{zhuo1}[][HSK 6]
    \definition{v.}{agarrar; segurar; apreender | pegar; capturar; aprisionar}
  \end{Phonetics}
\end{Entry}

%%%%%%%%%% 捍 %%%%%%%%%%
\subsection*{捍}

\begin{Entry}{捍}{10}{⼿}
  \begin{Phonetics}{捍}{han4}
    \definition{v.}{defender; guardar | defender-se | afastar (um golpe) | resistir}
  \end{Phonetics}
\end{Entry}

\begin{Entry}{捍卫}{10,3}{⼿、⼙}
  \begin{Phonetics}{捍卫}{han4wei4}[][HSK 7-9]
    \definition{v.}{defender; guardar; proteger; defender-se pela força ou outros meios de ser violado ou prejudicado}
  \end{Phonetics}
\end{Entry}

%%%%%%%%%% 捐 %%%%%%%%%%
\subsection*{捐}

\begin{Entry}{捐}{10}{⼿}
  \begin{Phonetics}{捐}{juan1}[][HSK 6]
    \definition{s.}{imposto}
    \definition{v.}{renunciar; abandonar | contribuir; doar; assinar}
  \end{Phonetics}
\end{Entry}

\begin{Entry}{捐助}{10,7}{⼿、⼒}
  \begin{Phonetics}{捐助}{juan1 zhu4}[][HSK 6]
    \definition{v.}{oferecer (assistência financeira ou material); contribuir; doar}
  \end{Phonetics}
\end{Entry}

\begin{Entry}{捐款}{10,12}{⼿、⽋}
  \begin{Phonetics}{捐款}{juan1/kuan3}[][HSK 6]
    \definition[笔]{s.}{doação; contribuição (de dinheiro); valor doado}
    \definition{v.+compl.}{doar; contribuir com dinheiro}
  \end{Phonetics}
\end{Entry}

\begin{Entry}{捐献}{10,13}{⼿、⽝}
  \begin{Phonetics}{捐献}{juan1xian4}[][HSK 7-9]
    \definition{v.}{doar; apresentar; contribuir (para uma organização); doar bens ao (estado, a uma cooperativa, etc.)}
  \end{Phonetics}
\end{Entry}

\begin{Entry}{捐赠}{10,16}{⼿、⾙}
  \begin{Phonetics}{捐赠}{juan1 zeng4}[][HSK 6]
    \definition{v.}{apresentar; contribuir (como um presente); doar (itens para um país ou grupo)}
  \end{Phonetics}
\end{Entry}

%%%%%%%%%% 捕 %%%%%%%%%%
\subsection*{捕}

\begin{Entry}{捕}{10}{⼿}
  \begin{Phonetics}{捕}{bu3}[][HSK 6]
    \definition{v.}{pegar; apreender; prender}
  \end{Phonetics}
\end{Entry}

\begin{Entry}{捕捉}{10,10}{⼿、⼿}
  \begin{Phonetics}{捕捉}{bu3zhuo1}[][HSK 7-9]
    \definition{v.}{caçar; perseguir; pegar; capturar; apreender; pegar; fazer uma pessoa ou animal cair nas mãos; pode ser usado tanto para pessoas quanto para coisas; tem uma ampla gama de aplicações; usado tanto na linguagem falada quanto na escrita}
  \end{Phonetics}
\end{Entry}

%%%%%%%%%% 捞 %%%%%%%%%%
\subsection*{捞}

\begin{Entry}{捞}{10}{⼿}
  \begin{Phonetics}{捞}{lao1}[][HSK 7-9]
    \definition{v.}{arrastar para; pescar para; recolher; dragar (para fora); retirar algo da água ou de outros líquidos | ganhar; obter por meios ilícitos | sair andando com alguma coisa; puxar ou pegar casualmente}
  \end{Phonetics}
\end{Entry}

%%%%%%%%%% 损 %%%%%%%%%%
\subsection*{损}

\begin{Entry}{损}{10}{⼿}
  \begin{Phonetics}{损}{sun3}
    \definition{adj.}{sarcástico; cortante; de ​​língua afiada; maldoso; mau; cruel}
    \definition{v.}{diminuir; perder; reduzir | prejudicar; danificar; degradar; destruir; arruinar; destruir o estado original ou fazê-lo perder sua eficácia original | ser sarcástico; ser cáustico; ser cortante; ferir; insultar; usar palavras duras para zombar de alguém}
  \end{Phonetics}
\end{Entry}

\begin{Entry}{损失}{10,5}{⼿、⼤}
  \begin{Phonetics}{损失}{sun3shi1}[][HSK 5]
    \definition{s.}{perda; desperdício; algo que se consome ou se perde sem custo algum}
    \definition{v.}{perder; consumir ou perder}
  \end{Phonetics}
\end{Entry}

\begin{Entry}{损害}{10,10}{⼿、⼧}
  \begin{Phonetics}{损害}{sun3 hai4}[][HSK 5]
    \definition{v.}{prejudicar; danificar; ferir; causar danos; causar perdas}
  \end{Phonetics}
\end{Entry}

%%%%%%%%%% 捡 %%%%%%%%%%
\subsection*{捡}

\begin{Entry}{捡}{10}{⼿}
  \begin{Phonetics}{捡}{jian3}[][HSK 6]
    \definition{v.}{coletar; reunir; apanhar; pegar coisas do chão}
  \end{Phonetics}
\end{Entry}

%%%%%%%%%% 换 %%%%%%%%%%
\subsection*{换}

\begin{Entry}{换}{10}{⼿}
  \begin{Phonetics}{换}{huan4}[][HSK 2]
    \definition{v.}{negociar; trocar; permutar; dar algo a alguém e, ao mesmo tempo, obter algo dele em troca | mudar; transformar; substituir | trocar dinheiro (câmbio) | transfundir (sangue) | transplantar (um órgão)}
  \end{Phonetics}
\end{Entry}

\begin{Entry}{换成}{10,6}{⼿、⼽}
  \begin{Phonetics}{换成}{huan4cheng2}[][HSK 7-9]
    \definition{v.}{trocar (algo) por (outro); indica a substituição de um objeto, estado ou situação por outro}
  \end{Phonetics}
\end{Entry}

\begin{Entry}{换位}{10,7}{⼿、⼈}
  \begin{Phonetics}{换位}{huan4wei4}[][HSK 7-9]
    \definition{v.}{trocar posições; transpor | mudar de posição}
  \end{Phonetics}
\end{Entry}

\begin{Entry}{换言之}{10,7,3}{⼿、⾔、⼂}
  \begin{Phonetics}{换言之}{huan4yan2zhi1}[][HSK 7-9]
    \definition{adv.}{em outras palavras}
  \end{Phonetics}
\end{Entry}

\begin{Entry}{换取}{10,8}{⼿、⼜}
  \begin{Phonetics}{换取}{huan4qu3}[][HSK 7-9]
    \definition{v.}{trocar (ou escambo) algo por; obter em troca | trocar algo por; obter por troca}
  \end{Phonetics}
\end{Entry}

\begin{Entry}{换钱}{10,10}{⼿、⾦}
  \begin{Phonetics}{换钱}{huan4/qian2}
    \definition{v.+compl.}{trocar dinheiro (em pequenas valores ou em outra moeda) | trocar (mercadorias) por dinheiro | vender}
  \end{Phonetics}
\end{Entry}

%%%%%%%%%% 捣 %%%%%%%%%%
\subsection*{捣}

\begin{Entry}{捣}{10}{⼿}
  \begin{Phonetics}{捣}{dao3}
    \definition{v.}{bater com um pilão, etc.; bater; esmagar | assediar; perturbar | bater com um pedaço de pau}
  \end{Phonetics}
\end{Entry}

\begin{Entry}{捣乱}{10,7}{⼿、⼄}
  \begin{Phonetics}{捣乱}{dao3/luan4}[][HSK 7-9]
    \definition{v.+compl.}{causar problemas; criar uma perturbação; causar intencionalmente problemas para os outros; interromper | perturbar; interferir com; causar problemas intencionalmente}
  \end{Phonetics}
\end{Entry}

%%%%%%%%%% 效 %%%%%%%%%%
\subsection*{效}

\begin{Entry}{效}{10}{⽁}
  \begin{Phonetics}{效}{xiao4}
    \definition{s.}{efeito; função | eficiência; resultado}
    \definition{v.}{imitar; seguir o exemplo de | dedicar (a energia ou a vida de alguém) a; prestar (um serviço)}
  \end{Phonetics}
\end{Entry}

\begin{Entry}{效果}{10,8}{⽁、⽊}
  \begin{Phonetics}{效果}{xiao4guo3}[][HSK 3]
    \definition[种,个]{s.}{efeito; resultado | efeitos sonoros; vários sons ou fenômenos naturais criados para combinar com o enredo em dramas e filmes, como vento e chuva, tiros, fogo, neve, etc.}
  \end{Phonetics}
\end{Entry}

\begin{Entry}{效率}{10,11}{⽁、⽞}
  \begin{Phonetics}{效率}{xiao4lv4}[][HSK 4]
    \definition[种]{s.}{eficiência; produtividade; a quantidade de trabalho concluído por unidade de tempo}
  \end{Phonetics}
\end{Entry}

%%%%%%%%%% 敌 %%%%%%%%%%
\subsection*{敌}

\begin{Entry}{敌}{10}{⾆}
  \begin{Phonetics}{敌}{di2}
    \definition[个,名,位,种]{s.}{inimigo; adversário}
    \definition{v.}{opor-se; lutar; resistir; suportar | combinar; igualar}
  \end{Phonetics}
\end{Entry}

\begin{Entry}{敌人}{10,2}{⾆、⼈}
  \begin{Phonetics}{敌人}{di2ren2}[][HSK 4]
    \definition[群,伙,帮,个,队]{s.}{inimigo; pessoa hostil; parte hostil}
  \end{Phonetics}
\end{Entry}

%%%%%%%%%% 料 %%%%%%%%%%
\subsection*{料}

\begin{Entry}{料}{10}{⽃}
  \begin{Phonetics}{料}{liao4}[][HSK 6]
    \definition{clas.}{usado na medicina tradicional chinesa para preparar pílulas | unidade usada para calcular um pedaço de madeira, é a seção transversal em ambas as extremidades, que é de 1 pé (quadrado) com 7 pés de comprimento}
    \definition{s.}{material; coisa | (grão) alimento; forragem | artigos de vidro; vidros coloridos opacos | (para pílulas de medicina chinesa) prescrição}
    \definition{v.}{supor; esperar; antecipar | gerenciar; cuidar de | prever}
  \end{Phonetics}
\end{Entry}

\begin{Entry}{料到}{10,8}{⽃、⼑}
  \begin{Phonetics}{料到}{liao4dao4}[][HSK 7-9]
    \definition{v.}{prever; esperar; significa que as coisas estão se desenvolvendo conforme o esperado}
  \end{Phonetics}
\end{Entry}

\begin{Entry}{料理}{10,11}{⽃、⽟}
  \begin{Phonetics}{料理}{liao4li3}[][HSK 7-9]
    \definition[个,种]{s.}{prato; culinária; cozinha; refere-se a um certo estilo de culinária}
    \definition{v.}{organizar; gerir; cuidar de; dar atenção a; cuidar; lidar com isso | cozinhar; preparar alimentos}
  \end{Phonetics}
\end{Entry}

%%%%%%%%%% 旁 %%%%%%%%%%
\subsection*{旁}

\begin{Entry}{旁}{10}{⽅}
  \begin{Phonetics}{旁}{pang2}[][HSK 5]
    \definition{adj.}{outro | abundante; abrangente}
    \definition{s.}{lado | radical lateral de um caractere chinês}
  \end{Phonetics}
\end{Entry}

\begin{Entry}{旁边}{10,5}{⽅、⾡}
  \begin{Phonetics}{旁边}{pang2bian1}[][HSK 1]
    \definition{s.}{junto a; próximo de; ao lado}
  \end{Phonetics}
\end{Entry}

%%%%%%%%%% 旅 %%%%%%%%%%
\subsection*{旅}

\begin{Entry}{旅}{10}{⽅}
  \begin{Phonetics}{旅}{lv3}
    \definition{adv.}{juntos; conjuntamente}
    \definition[个]{s.}{brigada; unidade organizacional militar, abaixo do nível de divisão e acima do nível de regimento ou batalhão | força; tropas; geralmente se refere aos militares | viajante; passageiro; turista | viagem; jornada | pessoas}
    \definition{v.}{viajar; ficar longe de casa; ir para longe; morar longe de casa}
  \end{Phonetics}
\end{Entry}

\begin{Entry}{旅行}{10,6}{⽅、⾏}
  \begin{Phonetics}{旅行}{lv3xing2}[][HSK 2]
    \definition{v.}{viajar; passear; para tratar de assuntos ou passear, ir de um lugar para outro (geralmente se refere a distâncias longas)}
  \end{Phonetics}
\end{Entry}

\begin{Entry}{旅行社}{10,6,7}{⽅、⾏、⽰}
  \begin{Phonetics}{旅行社}{lv3 xing2 she4}[][HSK 3]
    \definition[家]{s.}{agência de viagens; agência especializada em serviços relacionados a viagens, que providencia hospedagem, transporte e outros serviços para viajantes}
  \end{Phonetics}
\end{Entry}

\begin{Entry}{旅店}{10,8}{⽅、⼴}
  \begin{Phonetics}{旅店}{lv3 dian5}[][HSK 6]
    \definition[家,个]{s.}{pousada; albergue; hotel}
  \end{Phonetics}
\end{Entry}

\begin{Entry}{旅客}{10,9}{⽅、⼧}
  \begin{Phonetics}{旅客}{lv3 ke4}[][HSK 2]
    \definition[名,位,个,些]{s.}{viajante; passageiro; as agências de transporte e turismo referem-se às pessoas que viajam}
  \end{Phonetics}
\end{Entry}

\begin{Entry}{旅途}{10,10}{⽅、⾡}
  \begin{Phonetics}{旅途}{lv3tu2}[][HSK 7-9]
    \definition{s.}{viagem; jornada; durante a viagem}
  \end{Phonetics}
\end{Entry}

\begin{Entry}{旅馆}{10,11}{⽅、⾷}
  \begin{Phonetics}{旅馆}{lv3 guan3}[][HSK 3]
    \definition[家,个,所]{s.}{pousada; hotel; local comercial destinado ao alojamento de viajantes}
  \end{Phonetics}
\end{Entry}

\begin{Entry}{旅游}{10,12}{⽅、⽔}
  \begin{Phonetics}{旅游}{lv3you2}[][HSK 2]
    \definition{v.}{viajar para outros lugares para passear e fazer turismo}
  \end{Phonetics}
\end{Entry}

\begin{Entry}{旅程}{10,12}{⽅、⽲}
  \begin{Phonetics}{旅程}{lv3cheng2}[][HSK 7-9]
    \definition[条,段]{s.}{rota; itinerário; viagem; distância percorrida; a jornada}
  \end{Phonetics}
\end{Entry}

%%%%%%%%%% 晃 %%%%%%%%%%
\subsection*{晃}

\begin{Entry}{晃}{10}{⽇}
  \begin{Phonetics}{晃}{huang3}[][HSK 7-9]
    \definition*{s.}{Sobrenome: Huang}
    \definition{adj.}{deslumbrante}
    \definition{v.}{passar rapidamente | deslumbrar; cegar}
  \end{Phonetics}
  \begin{Phonetics}{晃}{huang4}[][HSK 7-9]
    \definition{v.}{sacudir; balançar}
  \end{Phonetics}
\end{Entry}

\begin{Entry}{晃荡}{10,9}{⽇、⾋}
  \begin{Phonetics}{晃荡}{huang4dang5}[][HSK 7-9]
    \definition{v.}{balançar; sacudir | Coloquial: vagar; ficar ocioso; divagar | oscilar}
  \end{Phonetics}
\end{Entry}

%%%%%%%%%% 晋 %%%%%%%%%%
\subsection*{晋}

\begin{Entry}{晋}{10}{⽇}
  \begin{Phonetics}{晋}{jin4}
    \definition*{s.}{Estado da Dinastia Zhou (1046-256 a.C.), ocupando partes do que hoje são Shanxi, Shaanxi, Hebei e Henan |
Dinastia Jin Ocidental (265-316), Dinastia Jin Oriental (317-420) e Dinastia Jin Posterior (936-946) | Nome abreviado da província de Shanxi: 山西 | Sobrenome: Jin}
    \definition{v.}{avançar | promover}
  \seealsoref{山西}{shan1xi1}
  \end{Phonetics}
\end{Entry}

\begin{Entry}{晋升}{10,4}{⽇、⼗}
  \begin{Phonetics}{晋升}{jin4sheng1}[][HSK 7-9]
    \definition{v.}{elevar; promover (a um cargo superior)}
  \end{Phonetics}
\end{Entry}

%%%%%%%%%% 晒 %%%%%%%%%%
\subsection*{晒}

\begin{Entry}{晒}{10}{⽇}
  \begin{Phonetics}{晒}{shai4}[][HSK 4]
    \definition{v.}{(sol) brilhar sobre | aquecer-se; secar ao sol; colocar algo sob a luz do sol para secar | ignorar (alguém) | mostrar; divulgar o conteúdo de sua vida privada na Internet}
  \end{Phonetics}
\end{Entry}

\begin{Entry}{晒干}{10,3}{⽇、⼲}
  \begin{Phonetics}{晒干}{shai4gan1}
    \definition{v.}{secar ao sol}
  \end{Phonetics}
\end{Entry}

%%%%%%%%%% 晓 %%%%%%%%%%
\subsection*{晓}

\begin{Entry}{晓}{10}{⽇}
  \begin{Phonetics}{晓}{xiao3}
    \definition{s.}{amanhecer; alvorada}
    \definition{v.}{(um dia) amanhecer; romper | saber; deixar alguém saber; dizer}
  \end{Phonetics}
\end{Entry}

\begin{Entry}{晓得}{10,11}{⽇、⼻}
  \begin{Phonetics}{晓得}{xiao3 de2}[][HSK 6]
    \definition{v.}{saber; entender}[我不晓得他在哪里。===Não sei onde ele está.]
  \end{Phonetics}
\end{Entry}

%%%%%%%%%% 晕 %%%%%%%%%%
\subsection*{晕}

\begin{Entry}{晕}{10}{⽇}
  \begin{Phonetics}{晕}{yun1}[][HSK 6]
    \definition{adj.}{tonto; vertiginoso; confuso; sensação de que as coisas estão girando ao seu redor e, às vezes, sensação de que você vai cair}
    \definition{v.}{desmaiar; desfalecer}
  \end{Phonetics}
  \begin{Phonetics}{晕}{yun4}
    \definition{s.}{auréola; o círculo de luz formado pela refração da luz solar ou do luar através dos cristais de gelo nas nuvens | halo em torno de alguma cor ou luz; áreas desfocadas em torno de luz, sombra e cor}
    \definition{v.}{ficar tonto; desmaiar; desfalecer; sensação de tontura, como se os objetos ao seu redor estivessem girando e como se você estivesse prestes a cair}
  \end{Phonetics}
\end{Entry}

\begin{Entry}{晕车}{10,4}{⽇、⾞}
  \begin{Phonetics}{晕车}{yun4 che1}[][HSK 6]
    \definition{v.}{ter enjoo no carro; ter tontura e vômito ao andar de carro}
  \end{Phonetics}
\end{Entry}

%%%%%%%%%% 朗 %%%%%%%%%%
\subsection*{朗}

\begin{Entry}{朗}{10}{⽉}
  \begin{Phonetics}{朗}{lang3}
    \definition*{s.}{Sobrenome: Lang}
    \definition{adj.}{claro; brilhante | alto e claro (som)}
  \end{Phonetics}
\end{Entry}

\begin{Entry}{朗诵}{10,9}{⽉、⾔}
  \begin{Phonetics}{朗诵}{lang3song4}[][HSK 7-9]
    \definition{v.}{recitar; ler em voz alta com expressividade; ler poemas ou prosa em voz alta para expressar as emoções transmitidas pela obra}
  \end{Phonetics}
\end{Entry}

\begin{Entry}{朗读}{10,10}{⽉、⾔}
  \begin{Phonetics}{朗读}{lang3du2}[][HSK 5]
    \definition{v.}{ler em voz alta; recitar com voz clara e alta}
  \end{Phonetics}
\end{Entry}

%%%%%%%%%% 校 %%%%%%%%%%
\subsection*{校}

\begin{Entry}{校}{10}{⽊}
  \begin{Phonetics}{校}{jiao4}
    \definition{v.}{verificar | comparar | revisar}
  \end{Phonetics}
  \begin{Phonetics}{校}{xiao4}
    \definition[所]{s.}{oficial militar | escola}
  \end{Phonetics}
\end{Entry}

\begin{Entry}{校长}{10,4}{⽊、⾧}
  \begin{Phonetics}{校长}{xiao4zhang3}[][HSK 2]
    \definition[个,位,名]{s.}{diretor; presidente; reitor; o mais alto líder administrativo e empresarial de uma escola}
  \end{Phonetics}
\end{Entry}

\begin{Entry}{校园}{10,7}{⽊、⼞}
  \begin{Phonetics}{校园}{xiao4 yuan2}[][HSK 2]
    \definition[个]{s.}{campus; pátio da escola; refere-se a todos os terrenos e edifícios dentro da área escolar}
  \end{Phonetics}
\end{Entry}

\begin{Entry}{校服}{10,8}{⽊、⽉}
  \begin{Phonetics}{校服}{xiao4fu2}
    \definition{s.}{uniforme escolar}
  \end{Phonetics}
\end{Entry}

\begin{Entry}{校规}{10,8}{⽊、⾒}
  \begin{Phonetics}{校规}{xiao4gui1}
    \definition{s.}{regras e regulamentos escolares}
  \end{Phonetics}
\end{Entry}

\begin{Entry}{校监}{10,10}{⽊、⽫}
  \begin{Phonetics}{校监}{xiao4jian1}
    \definition{s.}{diretor | supervisor (de escola)}
  \end{Phonetics}
\end{Entry}

%%%%%%%%%% 样 %%%%%%%%%%
\subsection*{样}

\begin{Entry}{样}{10}{⽊}
  \begin{Phonetics}{样}{yang4}[][HSK 6]
    \definition{clas.}{usado para tipos de coisas}[这里有四样东西。===Há quatro coisas aqui.]
    \definition[个]{s.}{aparência; aspecto;  forma; aparência; a forma do objeto | modelo; amostra; padrão; coisas usadas como padrões | ar; maneira; aparência; a aparência ou expressão de uma pessoa | tendência; probabilidade; a situação ou tendência das coisas}
  \end{Phonetics}
\end{Entry}

\begin{Entry}{样儿}{10,2}{⽊、⼉}
  \begin{Phonetics}{样儿}{yang4r5}
    \definition{s.}{aparência | forma | modelo}
  \seealsoref{样子}{yang4zi5}
  \end{Phonetics}
\end{Entry}

\begin{Entry}{样子}{10,3}{⽊、⼦}
  \begin{Phonetics}{样子}{yang4zi5}[][HSK 2]
    \definition[个,种,副]{s.}{forma; aparência; estilo | ar; maneira; modalidade; estado | tendência; probabilidade; usado com 看 e 照 para expressar uma estimativa de uma tendência | modelo; amostra; padrão; uma pessoa ou coisa que pode ser usada como um padrão para as pessoas verificarem, seguirem ou aprenderem com ela}
  \seealsoref{看}{kan4}
  \seealsoref{样儿}{yang4r5}
  \seealsoref{照}{zhao4}
  \end{Phonetics}
\end{Entry}

\begin{Entry}{样品}{10,9}{⽊、⼝}
  \begin{Phonetics}{样品}{yang4pin3}
    \definition{s.}{amostra | espécime}
  \end{Phonetics}
\end{Entry}

\begin{Entry}{样样}{10,10}{⽊、⽊}
  \begin{Phonetics}{样样}{yang4yang4}
    \definition{adv.}{todos os tipos}
  \end{Phonetics}
\end{Entry}

\begin{Entry}{样章}{10,11}{⽊、⾳}
  \begin{Phonetics}{样章}{yang4zhang1}
    \definition{s.}{capítulo de amostra}
  \end{Phonetics}
\end{Entry}

%%%%%%%%%% 核 %%%%%%%%%%
\subsection*{核}

\begin{Entry}{核}{10}{⽊}
  \begin{Phonetics}{核}{he2}[][HSK 7-9]
    \definition{adj.}{Literário: verdadeiro; fiel}
    \definition{s.}{poço; pedra; caroço | núcleo | núcleo atômico}
    \definition{v.}{examinar; verificar}
  \end{Phonetics}
  \begin{Phonetics}{核}{hu2}
    \definition{s.}{semente; o mesmo que 核}
  \end{Phonetics}
\end{Entry}

\begin{Entry}{核心}{10,4}{⽊、⼼}
  \begin{Phonetics}{核心}{he2xin1}[][HSK 6]
    \definition[个]{s.}{núcleo; elite; coração; centro; parte principal (em termos de relacionamento entre as coisas)}
  \end{Phonetics}
\end{Entry}

\begin{Entry}{核对}{10,5}{⽊、⼨}
  \begin{Phonetics}{核对}{he2dui4}[][HSK 7-9]
    \definition{v.}{verificar; checar; verificar cuidadosamente (para ver se corresponde)}
  \end{Phonetics}
\end{Entry}

\begin{Entry}{核电站}{10,5,10}{⽊、⽥、⽴}
  \begin{Phonetics}{核电站}{he2dian4zhan4}[][HSK 7-9]
    \definition{s.}{usina nuclear; usina que utiliza energia nuclear para gerar eletricidade}
  \end{Phonetics}
\end{Entry}

\begin{Entry}{核实}{10,8}{⽊、⼧}
  \begin{Phonetics}{核实}{he2shi2}[][HSK 7-9]
    \definition{v.}{verificar; checar; verificar se é verdade}
  \end{Phonetics}
\end{Entry}

\begin{Entry}{核武器}{10,8,16}{⽊、⽌、⼝}
  \begin{Phonetics}{核武器}{he2wu3qi4}[][HSK 7-9]
    \definition[个]{s.}{arma nuclear}
  \end{Phonetics}
\end{Entry}

\begin{Entry}{核桃}{10,10}{⽊、⽊}
  \begin{Phonetics}{核桃}{he2tao5}[][HSK 7-9]
    \definition[颗,个,棵,顆]{s.}{noz | nogueira}
  \end{Phonetics}
\end{Entry}

\begin{Entry}{核能}{10,10}{⽊、⾁}
  \begin{Phonetics}{核能}{he2neng2}[][HSK 7-9]
    \definition{s.}{energia nuclear}
  \end{Phonetics}
\end{Entry}

%%%%%%%%%% 根 %%%%%%%%%%
\subsection*{根}

\begin{Entry}{根}{10}{⽊}
  \begin{Phonetics}{根}{gen1}[][HSK 4]
    \definition*{s.}{Sobrenome: Gen}
    \definition{adv.}{completamente; minuciosamente; radicalmente}
    \definition{clas.}{usado para objetos finos, alongados}
    \definition{s.}{raiz (de uma planta) | descendentes; posteridade; analogia com as gerações futuras | raiz (abreviação de raiz quadrada) | radical (química, refere-se a radicais carregados) | base; pé; raiz; parte inferior, base ou parte de um objeto que está presa a outra coisa | a parte de baixo das coisas; fonte; a origem  das coisas | base; fundamento}
  \end{Phonetics}
\end{Entry}

\begin{Entry}{根本}{10,5}{⽊、⽊}
  \begin{Phonetics}{根本}{gen1ben3}[][HSK 3]
    \definition{adj.}{básico; essencial; fundamental; importante; decisivo}
    \definition{adv.}{nunca; simplesmente; de forma alguma | radicalmente; completamente; nunca (mais usado em negativas)}
    \definition[个]{s.}{base; raiz; fundação; a origem, a base ou a parte mais importante das coisas}
  \end{Phonetics}
\end{Entry}

\begin{Entry}{根治}{10,8}{⽊、⽔}
  \begin{Phonetics}{根治}{gen1zhi4}[][HSK 7-9]
    \definition{v.}{efetuar uma cura radical; curar de uma vez por todas; colocar sob controle permanente; curar completamente (referindo-se à erradicação de pragas ou doenças)}
  \end{Phonetics}
\end{Entry}

\begin{Entry}{根基}{10,11}{⽊、⼟}
  \begin{Phonetics}{根基}{gen1ji1}[][HSK 7-9]
    \definition{s.}{base; fundação; alicerce; parte subterrânea de um edifício | recursos; propriedade acumulada ao longo de um longo período}
  \end{Phonetics}
\end{Entry}

\begin{Entry}{根据}{10,11}{⽊、⼿}
  \begin{Phonetics}{根据}{gen1ju4}[][HSK 4]
    \definition{prep.}{com base em; de acordo com; à luz de}
    \definition[个]{s.}{base; fundamentos; razão; fundo; alicerce}
    \definition{v.}{basear; usar algo como premissa para uma conclusão ou como base para uma ação verbal}
  \end{Phonetics}
\end{Entry}

\begin{Entry}{根深蒂固}{10,11,12,8}{⽊、⽔、⾋、⼞}
  \begin{Phonetics}{根深蒂固}{gen1shen1-di4gu4}[][HSK 7-9]
    \definition{expr.}{arraigado; inveterado; tornar-se profundamente enraizado em; profundamente enraizado; profundamente enraizado; profundamente enraizado e firmemente plantado -- bem fundado; ter uma base firme; ter raízes profundas e uma base firme; bem estabelecido; significa que a fundação é sólida e não se abala facilmente}
  \end{Phonetics}
\end{Entry}

\begin{Entry}{根源}{10,13}{⽊、⽔}
  \begin{Phonetics}{根源}{gen1yuan2}[][HSK 7-9]
    \definition{s.}{fonte; origem; raiz | raízes da grama; fonte; nascente; raiz; fundo}
    \definition{v.}{originar-se; provir de}
  \end{Phonetics}
\end{Entry}

%%%%%%%%%% 格 %%%%%%%%%%
\subsection*{格}

\begin{Entry}{格}{10}{⽊}
  \begin{Phonetics}{格}{ge1}
    \definition{s.}{Onomatopéia: estalo (som); riso zombeteiro}
  \end{Phonetics}
  \begin{Phonetics}{格}{ge2}[][HSK 7-9]
    \definition*{s.}{Sobrenome: Ge}
    \definition{s.}{quadrados formados por linhas cruzadas; quadriculado; grade | divisão (horizontal ou não); treliça | padrão; forma; formato; estilo | caso; as categorias morfológicas de substantivos, pronomes e adjetivos em algumas línguas}
    \definition{v.}{resistir; dificultar; obstruir; impedir | estudar cuidadosamente; investigar | lutar; bater}
  \end{Phonetics}
\end{Entry}

\begin{Entry}{格兰菜}{10,5,11}{⽊、⼋、⾋}
  \begin{Phonetics}{格兰菜}{ge2lan2cai4}
    \definition{s.}{brócolis chinês | couve chinesa | mostarda}
  \seealsoref{芥蓝}{gai4lan2}
  \end{Phonetics}
\end{Entry}

\begin{Entry}{格外}{10,5}{⽊、⼣}
  \begin{Phonetics}{格外}{ge2wai4}[][HSK 4]
    \definition{adv.}{especialmente; particularmente; ainda mais; indica mais do que a média | adicionalmente; indica adicional ou extra}
  \end{Phonetics}
\end{Entry}

\begin{Entry}{格式}{10,6}{⽊、⼷}
  \begin{Phonetics}{格式}{ge2shi5}[][HSK 7-9]
    \definition[种]{s.}{forma; estilo; \emph{layout}; padrão; formato; modo}
  \end{Phonetics}
\end{Entry}

\begin{Entry}{格局}{10,7}{⽊、⼫}
  \begin{Phonetics}{格局}{ge2ju2}[][HSK 7-9]
    \definition{s.}{padrão; configuração; estrutura; estilo; maneira; arranjo | a visão ou percepção de uma situação geral; a visão de uma pessoa, a altura e a profundidade da consideração do problema}
  \end{Phonetics}
\end{Entry}

\begin{Entry}{格格不入}{10,10,4,2}{⽊、⽊、⼀、⼊}
  \begin{Phonetics}{格格不入}{ge2ge2-bu2ru4}[][HSK 7-9]
    \definition{expr.}{incompatível com; fora de sintonia com; estranho; fora do seu elemento; como uma estaca quadrada em um buraco redondo; desarmônico}
  \end{Phonetics}
\end{Entry}

%%%%%%%%%% 栽 %%%%%%%%%%
\subsection*{栽}

\begin{Entry}{栽}{10}{⽊}
  \begin{Phonetics}{栽}{zai1}
    \definition{v.}{cultivar | plantar}
  \end{Phonetics}
\end{Entry}

\begin{Entry}{栽种}{10,9}{⽊、⽲}
  \begin{Phonetics}{栽种}{zai1zhong4}
    \definition{v.}{plantar}
  \end{Phonetics}
\end{Entry}

\begin{Entry}{栽倒}{10,10}{⽊、⼈}
  \begin{Phonetics}{栽倒}{zai1dao3}
    \definition{v.}{cair | sofrer uma queda}
  \end{Phonetics}
\end{Entry}

\begin{Entry}{栽赃}{10,10}{⽊、⾙}
  \begin{Phonetics}{栽赃}{zai1zang1}
    \definition{v.}{enquadrar alguém (plantar provas nele)}
  \end{Phonetics}
\end{Entry}

\begin{Entry}{栽培}{10,11}{⽊、⼟}
  \begin{Phonetics}{栽培}{zai1pei2}
    \definition{v.}{cultivar | educar | patrocinar | treinar}
  \end{Phonetics}
\end{Entry}

\begin{Entry}{栽培种}{10,11,9}{⽊、⼟、⽲}
  \begin{Phonetics}{栽培种}{zai1pei2 zhong3}
    \definition{s.}{espécies cultivadas}
  \end{Phonetics}
\end{Entry}

\begin{Entry}{栽植}{10,12}{⽊、⽊}
  \begin{Phonetics}{栽植}{zai1zhi2}
    \definition{v.}{plantar | transplantar}
  \end{Phonetics}
\end{Entry}

%%%%%%%%%% 桂 %%%%%%%%%%
\subsection*{桂}

\begin{Entry}{桂}{10}{⽊}
  \begin{Phonetics}{桂}{gui4}
    \definition*{s.}{outro nome para o rio Guijiang 桂江 (em Guangxi 广西) | outro nome para Guangxi 广西 (Região Autônoma de Zhuang) | Sobrenome: Gui}
    \definition[棵]{s.}{louro; loureiro | osmanthus de aroma doce | árvore de casca de cássia | canela; osmanthus}
  \seealsoref{广西}{guang3xi1}
  \seealsoref{桂江}{gui4jiang1}
  \end{Phonetics}
\end{Entry}

\begin{Entry}{桂江}{10,6}{⽊、⽔}
  \begin{Phonetics}{桂江}{gui4jiang1}
    \definition*{s.}{Rio Guijiang}
  \end{Phonetics}
\end{Entry}

\begin{Entry}{桂花}{10,7}{⽊、⾋}
  \begin{Phonetics}{桂花}{gui4hua1}[][HSK 7-9]
    \definition{s.}{jasmim do imperador; um arbusto perene ou pequena árvore, cujas flores também são chamadas de osmanthus, são muito perfumadas e podem ser usadas para extrair óleos aromáticos ou fazer especiarias. Variedades comuns incluem Jingui 金桂 (flores amarelo-alaranjadas), Dangui 丹桂 (flores vermelho-alaranjadas), Yingui 银桂 (flores branco-amareladas) e Sijigui 四季桂 (flores branco-amareladas).}
  \end{Phonetics}
\end{Entry}

%%%%%%%%%% 桃 %%%%%%%%%%
\subsection*{桃}

\begin{Entry}{桃}{10}{⽊}
  \begin{Phonetics}{桃}{tao2}[][HSK 5]
    \definition*{s.}{Sobrenome: Tao}
    \definition[个,箱,袋,斤,棵,种]{s.}{pêssego | em forma de pêssego | pessegueiro}
  \end{Phonetics}
\end{Entry}

\begin{Entry}{桃花}{10,7}{⽊、⾋}
  \begin{Phonetics}{桃花}{tao2 hua1}[][HSK 5]
    \definition[朵,枝,株]{s.}{Figurativo: caso amoroso | flor de pessegueiro}
  \end{Phonetics}
\end{Entry}

\begin{Entry}{桃树}{10,9}{⽊、⽊}
  \begin{Phonetics}{桃树}{tao2 shu4}[][HSK 5]
    \definition[棵,株]{s.}{pêssego (árvore) | pessegueiro; pêssegos}
  \end{Phonetics}
\end{Entry}

%%%%%%%%%% 框 %%%%%%%%%%
\subsection*{框}

\begin{Entry}{框}{10}{⽊}
  \begin{Phonetics}{框}{kuang4}[][HSK 7-9]
    \definition{s.}{moldura; estojo | caixa; bloco}
    \definition{v.}{Obsoleto: desenhar uma moldura ao redor; adicionar linhas ao redor do texto e das imagens | Obsoleto: restringir; confinar; conter; amarrar; colocar em uma camisa de força}
  \end{Phonetics}
\end{Entry}

\begin{Entry}{框架}{10,9}{⽊、⽊}
  \begin{Phonetics}{框架}{kuang4jia4}[][HSK 7-9]
    \definition[个,副,种,套]{s.}{moldura; estrutura; na construção civil, as estruturas são formadas por conexões como vigas e colunas | estrutura; estrutura básica de um sistema, texto, etc.; metáfora para a organização e estrutura das coisas}
  \end{Phonetics}
\end{Entry}

%%%%%%%%%% 案 %%%%%%%%%%
\subsection*{案}

\begin{Entry}{案}{10}{⽊}
  \begin{Phonetics}{案}{an4}
    \definition{s.}{mesa; escrivaninha; mesa longa | caso; caso de direito (legal) | registro; arquivo; arquivo de caso | um plano submetido para consideração; proposta; um documento que propõe planos, sugestões, métodos, etc.}
  \end{Phonetics}
\end{Entry}

\begin{Entry}{案件}{10,6}{⽊、⼈}
  \begin{Phonetics}{案件}{an4jian4}[][HSK 7-9]
    \definition[个,起,件,类]{s.}{caso; caso de direito; caso legal; contencioso e eventos ilegais}
  \end{Phonetics}
\end{Entry}

%%%%%%%%%% 桌 %%%%%%%%%%
\subsection*{桌}

\begin{Entry}{桌}{10}{⽊}
  \begin{Phonetics}{桌}{zhuo1}
    \definition{clas.}{usado para mesas de convidados em um banquete etc.}
    \definition{s.}{mesa}
  \end{Phonetics}
\end{Entry}

\begin{Entry}{桌子}{10,3}{⽊、⼦}
  \begin{Phonetics}{桌子}{zhuo1zi5}[][HSK 1]
    \definition[张,套]{s.}{mesa; escrivaninha; móveis, com uma superfície plana na parte superior e uma estrutura de suporte na parte inferior, para colocar objetos ou realizar atividades}
  \end{Phonetics}
\end{Entry}

\begin{Entry}{桌布}{10,5}{⽊、⼱}
  \begin{Phonetics}{桌布}{zhuo1bu4}
    \definition[条,块,张]{s.}{(computação) plano de fundo da área de trabalho | toalha de mesa | papel de parede}
  \end{Phonetics}
\end{Entry}

\begin{Entry}{桌机}{10,6}{⽊、⽊}
  \begin{Phonetics}{桌机}{zhuo1ji1}
    \definition{s.}{computador \emph{desktop}}
  \end{Phonetics}
\end{Entry}

\begin{Entry}{桌灯}{10,6}{⽊、⽕}
  \begin{Phonetics}{桌灯}{zhuo1deng1}
    \definition{s.}{luminária | lâmpada de mesa}
  \end{Phonetics}
\end{Entry}

\begin{Entry}{桌面}{10,9}{⽊、⾯}
  \begin{Phonetics}{桌面}{zhuo1mian4}
    \definition{s.}{área de trabalho | mesa}
  \end{Phonetics}
\end{Entry}

\begin{Entry}{桌球}{10,11}{⽊、⽟}
  \begin{Phonetics}{桌球}{zhuo1qiu2}
    \definition{s.}{bilhar | sinuca | mesa de ping-pong}
  \end{Phonetics}
\end{Entry}

\begin{Entry}{桌游}{10,12}{⽊、⽔}
  \begin{Phonetics}{桌游}{zhuo1you2}
    \definition{s.}{jogo de tabuleiro}
  \end{Phonetics}
\end{Entry}

%%%%%%%%%% 桑 %%%%%%%%%%
\subsection*{桑}

\begin{Entry}{桑}{10}{⽊}
  \begin{Phonetics}{桑}{sang1}
    \definition*{s.}{Sobrenome: Sang}
    \definition[棵]{s.}{amoreira}
  \end{Phonetics}
\end{Entry}

\begin{Entry}{桑巴舞}{10,4,14}{⽊、⼰、⾇}
  \begin{Phonetics}{桑巴舞}{sang1ba1wu3}
    \definition{s.}{samba}
  \end{Phonetics}
\end{Entry}

\begin{Entry}{桑树}{10,9}{⽊、⽊}
  \begin{Phonetics}{桑树}{sang1shu4}
    \definition{s.}{amoreira, suas folhas são utilizadas para alimentar bichos-da-seda}
  \end{Phonetics}
\end{Entry}

%%%%%%%%%% 档 %%%%%%%%%%
\subsection*{档}

\begin{Entry}{档}{10}{⽊}
  \begin{Phonetics}{档}{dang4}[][HSK 6]
    \definition{clas.}{festa; usado para eventos, shows}
    \definition{s.}{prateleiras (para arquivos); compartimentos para documentos | arquivos; arquivos | travessa (de uma mesa, etc.) | qualidade; nota}
  \end{Phonetics}
\end{Entry}

\begin{Entry}{档次}{10,6}{⽊、⽋}
  \begin{Phonetics}{档次}{dang4ci4}[][HSK 7-9]
    \definition{s.}{classe; grau; qualidade; nível; diferentes níveis divididos de acordo com certos padrões}
  \end{Phonetics}
\end{Entry}

\begin{Entry}{档案}{10,10}{⽊、⽊}
  \begin{Phonetics}{档案}{dang4'an4}[][HSK 6]
    \definition[份,个]{s.}{arquivos; registro; dossiê; arquivos e materiais armazenados de forma classificada para referência futura}
  \end{Phonetics}
\end{Entry}

%%%%%%%%%% 桥 %%%%%%%%%%
\subsection*{桥}

\begin{Entry}{桥}{10}{⽊}
  \begin{Phonetics}{桥}{qiao2}[][HSK 3]
    \definition*{s.}{Sobrenome: Qiao}
    \definition[座]{s.}{ponte; construção que atravessa a água conectando as duas margens}
  \end{Phonetics}
\end{Entry}

\begin{Entry}{桥梁}{10,11}{⽊、⽊}
  \begin{Phonetics}{桥梁}{qiao2liang2}[][HSK 6]
    \definition[座]{s.}{ponte; acesso; uma obra construída na superfície do rio, conectando as duas margens | ponte; metáfora para pessoas ou coisas que podem se comunicar}
  \end{Phonetics}
\end{Entry}

%%%%%%%%%% 桩 %%%%%%%%%%
\subsection*{桩}

\begin{Entry}{桩}{10}{⽊}
  \begin{Phonetics}{桩}{zhuang1}
    \definition{clas.}{para eventos, casos, transações, assuntos, etc.}
    \definition{s.}{toco | estaca | pilha}
  \end{Phonetics}
\end{Entry}

%%%%%%%%%% 欱 %%%%%%%%%%
\subsection*{欱}

\begin{Entry}{欱}{10}{⽋}
  \begin{Phonetics}{欱}{he1}
    \definition{v.}{beber | beber bebida alcoólica}
    \variantof{喝}
  \end{Phonetics}
\end{Entry}

%%%%%%%%%% 氧 %%%%%%%%%%
\subsection*{氧}

\begin{Entry}{氧}{10}{⽓}
  \begin{Phonetics}{氧}{yang3}
    \definition{s.}{oxigênio}
  \end{Phonetics}
\end{Entry}

\begin{Entry}{氧气}{10,4}{⽓、⽓}
  \begin{Phonetics}{氧气}{yang3qi4}[][HSK 6]
    \definition{s.}{oxigênio (O); gás oxigênio}
  \end{Phonetics}
\end{Entry}

%%%%%%%%%% 流 %%%%%%%%%%
\subsection*{流}

\begin{Entry}{流}{10}{⽔}
  \begin{Phonetics}{流}{liu2}[][HSK 2]
    \definition*{s.}{Sobrenome: Liu}
    \definition{adj.}{fluente; tão suave quanto a água corrente}
    \definition{clas.}{lúmen; abreviação de lumens, 流明}
    \definition[名,个]{s.}{corrente de água | corrente; algo que se assemelha a um fluxo de água | razão; taxa; classe; grau; ramificação; facção; hierarquia}
    \definition{v.}{(de líquido) fluir | vaguear; vagar; mover-se de um lugar para outro; movimentar-se sem direção fixa | espalhar; circular; transmitir; divulgar | degenerar; mudar para pior; tender (aspecto negativo) | banir; enviar para o exílio | correr (ou fluir) como líquido; refere-se à parte do rio após deixar sua nascente (em contraste com a 源)}
  \seealsoref{流明}{liu2ming2}
  \seealsoref{源}{yuan2}
  \end{Phonetics}
\end{Entry}

\begin{Entry}{流入}{10,2}{⽔、⼊}
  \begin{Phonetics}{流入}{liu2ru4}[][HSK 7-9]
    \definition{s.}{fluxo de entrada; afluxo}
    \definition{v.}{fluir (ou correr) em | cair em | deixar-se levar para dentro | fluir para dentro}
  \end{Phonetics}
\end{Entry}

\begin{Entry}{流水}{10,4}{⽔、⽔}
  \begin{Phonetics}{流水}{liu2shui3}[][HSK 7-9]
    \definition{s.}{água corrente (uma metáfora para coisas contínuas) | faturamento (nos negócios); isso se refere à receita de vendas de uma loja}
  \end{Phonetics}
\end{Entry}

\begin{Entry}{流失}{10,5}{⽔、⼤}
  \begin{Phonetics}{流失}{liu2shi1}[][HSK 7-9]
    \definition{v.}{escorrer; ser levado pela água; rochas, água e solo na natureza desaparecem por si mesmos ou são levados pela água e pelo vento | perder; esgotar; fluir para longe; perder algo útil e valioso | ir embora; perder funcionários; essa metáfora descreve pessoas que deixam sua região ou local de trabalho}
  \end{Phonetics}
\end{Entry}

\begin{Entry}{流传}{10,6}{⽔、⼈}
  \begin{Phonetics}{流传}{liu2chuan2}[][HSK 4]
    \definition[间]{v.}{espalhar; circular; passar adiante}
  \end{Phonetics}
\end{Entry}

\begin{Entry}{流动}{10,6}{⽔、⼒}
  \begin{Phonetics}{流动}{liu2 dong4}[][HSK 5]
    \definition{v.}{(água, ar, etc.) fluir; correr; circular | ir de um lugar para outro; estar em movimento; ser móvel (oposto a 固定)}
  \seealsoref{固定}{gu4ding4}
  \end{Phonetics}
\end{Entry}

\begin{Entry}{流向}{10,6}{⽔、⼝}
  \begin{Phonetics}{流向}{liu2xiang4}[][HSK 7-9]
    \definition{s.}{direção de uma corrente | direção do fluxo (de pessoal, mercadorias, etc.) | direção}
    \definition{v.}{fluir (ou correr, mover-se) em direção a}
  \end{Phonetics}
\end{Entry}

\begin{Entry}{流血}{10,6}{⽔、⾎}
  \begin{Phonetics}{流血}{liu2xie3}
    \definition{s.}{sangrar; estar sangrando; derramar sangue}
  \end{Phonetics}
  \begin{Phonetics}{流血}{liu2xue4}[][HSK 7-9]
    \definition{v.}{sangrar; derramar sangue; metaforicamente se refere a sacrifício ou ferimento}
  \end{Phonetics}
\end{Entry}

\begin{Entry}{流行}{10,6}{⽔、⾏}
  \begin{Phonetics}{流行}{liu2xing2}[][HSK 2]
    \definition{adj.}{popular; na moda; muito popular}
    \definition{v.}{ser popular; prevalecer; espalhar-se amplamente; divulgar amplamente}
  \end{Phonetics}
\end{Entry}

\begin{Entry}{流行性感冒}{10,6,8,13,9}{⽔、⾏、⼼、⼼、⽇}
  \begin{Phonetics}{流行性感冒}{liu2xing2 xing4 gan3mao4}
    \definition{s.}{gripe muito forte; influenza}
  \end{Phonetics}
\end{Entry}

\begin{Entry}{流利}{10,7}{⽔、⼑}
  \begin{Phonetics}{流利}{liu2li4}[][HSK 2]
    \definition{adj.}{fluente; suave; lúcido; falar e escrever com fluência e clareza | com fluência; sem dificuldades}
  \end{Phonetics}
\end{Entry}

\begin{Entry}{流明}{10,8}{⽔、⽇}
  \begin{Phonetics}{流明}{liu2ming2}
    \definition{s.}{(empréstimo linguístico) lúmen (unidade de fluxo luminoso)}
  \end{Phonetics}
\end{Entry}

\begin{Entry}{流氓}{10,8}{⽔、⽒}
  \begin{Phonetics}{流氓}{liu2mang2}[][HSK 7-9]
    \definition[群,帮,伙,个]{s.}{malandro; delinquente; arruaceiro; originalmente, o termo se referia a vagabundos desempregados; posteriormente, passou a se referir a pessoas que não se dedicam a um trabalho honesto e frequentemente cometem atos ilícitos | vandalismo; indecência; comportamento imoral; isso se refere a atos hediondos, como o assédio a mulheres}
  \end{Phonetics}
\end{Entry}

\begin{Entry}{流泪}{10,8}{⽔、⽔}
  \begin{Phonetics}{流泪}{liu2lei4}[][HSK 7-9]
    \definition{v.}{derramar lágrimas}
  \end{Phonetics}
\end{Entry}

\begin{Entry}{流畅}{10,8}{⽔、⽥}
  \begin{Phonetics}{流畅}{liu2chang4}[][HSK 7-9]
    \definition{adj.}{(escrita, fala, etc.) fluente; fácil e sem problema; suave}
  \end{Phonetics}
\end{Entry}

\begin{Entry}{流转}{10,8}{⽔、⾞}
  \begin{Phonetics}{流转}{liu2zhuan3}[][HSK 7-9]
    \definition{adj.}{(escritos, pessoas, etc.) suave; fluido; fluente; refere-se à poesia e à prosa que são fluentes e bem elaboradas}
    \definition{v.}{vagar; perambular; estar em movimento; flutuando e mudando de lugar; não fixo em um só lugar | (bens ou capital) circular; a movimentação de bens ou fundos durante o processo de circulação}
  \end{Phonetics}
\end{Entry}

\begin{Entry}{流星}{10,9}{⽔、⽇}
  \begin{Phonetics}{流星}{liu2xing1}
    \definition{s.}{meteoro | estrela cadente}
  \end{Phonetics}
\end{Entry}

\begin{Entry}{流浪}{10,10}{⽔、⽔}
  \begin{Phonetics}{流浪}{liu2lang4}[][HSK 7-9]
    \definition{v.}{vagar sem rumo; levar uma vida errante}
  \end{Phonetics}
\end{Entry}

\begin{Entry}{流通}{10,10}{⽔、⾡}
  \begin{Phonetics}{流通}{liu2tong1}[][HSK 5]
    \definition{v.}{(ar, dinheiro, mercadorias, etc.) fluir; circular}
  \end{Phonetics}
\end{Entry}

\begin{Entry}{流域}{10,11}{⽔、⼟}
  \begin{Phonetics}{流域}{liu2yu4}[][HSK 7-9]
    \definition[个]{s.}{vale fluvial (ou bacia hidrográfica); área de drenagem | bacia hidrográfica; bacia; área de captação; bacia de drenagem; área de alimentação; área de captação de água; vale; bacia fluvial}
  \end{Phonetics}
\end{Entry}

\begin{Entry}{流淌}{10,11}{⽔、⽔}
  \begin{Phonetics}{流淌}{liu2tang3}[][HSK 7-9]
    \definition{v.}{fluir; gotejar}
  \end{Phonetics}
\end{Entry}

\begin{Entry}{流程}{10,12}{⽔、⽲}
  \begin{Phonetics}{流程}{liu2cheng2}[][HSK 7-9]
    \definition{s.}{um caminho de fluxo; uma distância percorrida pela água; a distância do fluxo de água | fluxo de trabalho; um processo tecnológico; abreviação de fluxograma de processo, 工艺流程 | circuito; procedimentos para cada processo}
  \seealsoref{工艺流程}{gong1yi4 liu2cheng2}
  \end{Phonetics}
\end{Entry}

\begin{Entry}{流量}{10,12}{⽔、⾥}
  \begin{Phonetics}{流量}{liu2liang4}[][HSK 7-9]
    \definition[个]{s.}{(taxa de) fluxo; descarga; a quantidade de fluido que passa por uma seção transversal por unidade de tempo. geralmente é calculada em metros cúbicos por segundo | Internet: tráfego do site; tráfego de rede refere-se à quantidade de dados transmitidos por uma rede por unidade de tempo | fluxo/volume de tráfego; o número de pedestres, veículos, etc., que passam por um determinado local por unidade de tempo}
  \end{Phonetics}
\end{Entry}

\begin{Entry}{流感}{10,13}{⽔、⼼}
  \begin{Phonetics}{流感}{liu2 gan3}[][HSK 6]
    \definition{s.}{gripe; influenza; abreviação de 流行性感冒}
  \seealsoref{流行性感冒}{liu2xing2 xing4 gan3mao4}
  \end{Phonetics}
\end{Entry}

\begin{Entry}{流露}{10,21}{⽔、⾬}
  \begin{Phonetics}{流露}{liu2lu4}[][HSK 7-9]
    \definition{v.}{revelar; trair; mostrar involuntariamente (os próprios pensamentos ou sentimentos); (significado, emoção) revelar inconscientemente}
  \end{Phonetics}
\end{Entry}

%%%%%%%%%% 浙 %%%%%%%%%%
\subsection*{浙}

\begin{Entry}{浙}{10}{⽔}
  \begin{Phonetics}{浙}{zhe4}
    \definition{s.}{abreviação de província de Zhejiang,  浙江, no leste da China}
  \seealsoref{浙江}{zhe4jiang1}
  \end{Phonetics}
\end{Entry}

\begin{Entry}{浙江}{10,6}{⽔、⽔}
  \begin{Phonetics}{浙江}{zhe4jiang1}
    \definition*{s.}{Província de Zhejiang}
  \end{Phonetics}
\end{Entry}

%%%%%%%%%% 浩 %%%%%%%%%%
\subsection*{浩}

\begin{Entry}{浩}{10}{⽔}
  \begin{Phonetics}{浩}{hao4}
    \definition*{s.}{Sobrenome: Hao}
    \definition{adj.}{grande; vasto; grandioso; sem limites | um grande número; infinito}
  \end{Phonetics}
\end{Entry}

\begin{Entry}{浩劫}{10,7}{⽔、⼒}
  \begin{Phonetics}{浩劫}{hao4jie2}[][HSK 7-9]
    \definition[场,次]{s.}{grande calamidade; catástrofe | devastação; holocausto; flagelo}
  \end{Phonetics}
\end{Entry}

%%%%%%%%%% 浪 %%%%%%%%%%
\subsection*{浪}

\begin{Entry}{浪}{10}{⽔}
  \begin{Phonetics}{浪}{lang4}[][HSK 7-9]
    \definition*{s.}{Sobrenome: Lang}
    \definition{adj.}{desenfreado; perdulário}
    \definition{adv.}{livremente}
    \definition[朵,阵,波]{s.}{onda; vagalhão; rebentação | algo ondulatório | coisas ondulando como ondas}
    \definition{v.}{passear; divagar}
  \end{Phonetics}
\end{Entry}

\begin{Entry}{浪花}{10,7}{⽔、⾋}
  \begin{Phonetics}{浪花}{lang4hua1}
    \definition[朵]{s.}{\emph{spray} | \emph{spray} do oceano | (figurativo) acontecimentos de sua vida}
  \end{Phonetics}
\end{Entry}

\begin{Entry}{浪费}{10,9}{⽔、⾙}
  \begin{Phonetics}{浪费}{lang4fei4}[][HSK 3]
    \definition{adj.}{desperdiçado; extravagante; não econômico}
    \definition{v.}{desperdiçar; dissipar; esbanjar; ser extravagante; uso excessivo ou inadequado de bens, recursos humanos, tempo, etc.}
  \end{Phonetics}
\end{Entry}

\begin{Entry}{浪漫}{10,14}{⽔、⽔}
  \begin{Phonetics}{浪漫}{lang4man4}[][HSK 5]
    \definition{adj.}{romântico; poético | não convencional; boêmio; abandonado; libertino; devasso; comportar-se de maneira descuidada e descuidada (geralmente se referindo a relacionamentos entre pessoas) | irrealista; impraticável}
  \end{Phonetics}
\end{Entry}

%%%%%%%%%% 浮 %%%%%%%%%%
\subsection*{浮}

\begin{Entry}{浮}{10}{⽔}
  \begin{Phonetics}{浮}{fu2}[][HSK 6]
    \definition*{s.}{Sobrenome: Fu}
    \definition{adj.}{superficial; na superfície | móvel; removível | temporário; provisório | superficial e frívolo; volátil; impetuoso | oco; vazio; inflado | excessivo; excedente}
    \definition{v.}{flutuar (oposto a 沉) | (dialeto) nadar | flutuar; derivar; flutuar na superfície do líquido}
  \seealsoref{沉}{chen2}
  \end{Phonetics}
\end{Entry}

\begin{Entry}{浮力}{10,2}{⽔、⼒}
  \begin{Phonetics}{浮力}{fu2li4}[][HSK 7-9]
    \definition{s.}{flutuabilidade; a força de empuxo, a força ascendente exercida sobre um objeto em um fluido, é igual ao peso do fluido deslocado pelo objeto}
  \end{Phonetics}
\end{Entry}

\begin{Entry}{浮图}{10,8}{⽔、⼞}
  \begin{Phonetics}{浮图}{fu2tu2}
    \definition*{s.}{Termo alternativo para 佛陀}
    \variantof{浮屠}
  \seealsoref{佛陀}{fo2tuo2}
  \end{Phonetics}
\end{Entry}

\begin{Entry}{浮现}{10,8}{⽔、⾒}
  \begin{Phonetics}{浮现}{fu2xian4}[][HSK 7-9]
    \definition{v.}{(experiência passada) ressurgir; vir à mente | aparecer; apresentar; revelar}
  \end{Phonetics}
\end{Entry}

\begin{Entry}{浮屠}{10,11}{⽔、⼫}
  \begin{Phonetics}{浮屠}{fu2tu2}
    \definition*{s.}{Buda | Templo (Stupa) Budista (transliteração de Pali Thuo)}
  \end{Phonetics}
\end{Entry}

\begin{Entry}{浮躁}{10,20}{⽔、⾜}
  \begin{Phonetics}{浮躁}{fu2zao4}[][HSK 7-9]
    \definition{adj.}{inquieto; impetuoso; impaciente; frívolo e impaciente}
  \end{Phonetics}
\end{Entry}

%%%%%%%%%% 海 %%%%%%%%%%
\subsection*{海}

\begin{Entry}{海}{10}{⽔}
  \begin{Phonetics}{海}{hai3}[][HSK 2]
    \definition*{s.}{Sobrenome: Hai}
    \definition{adj.}{extragrande; de grande capacidade; descreve capacidade, tom de voz, etc.}
    \definition{adv.}{aleatoriamente; sem rumo; sem limites; sem restrições}
    \definition[片]{s.}{mar; grande lago; a parte do oceano próxima à costa, alguns grandes lagos também são chamados de mar | grande número de pessoas ou coisas reunidas; metáfora para muitas coisas semelhantes que formam um grande conjunto}
  \end{Phonetics}
\end{Entry}

\begin{Entry}{海内外}{10,4,5}{⽔、⼌、⼣}
  \begin{Phonetics}{海内外}{hai3 nei4wai4}[][HSK 7-9]
    \definition{s.}{em casa e no exterior | nacional e internacional}
  \end{Phonetics}
\end{Entry}

\begin{Entry}{海水}{10,4}{⽔、⽔}
  \begin{Phonetics}{海水}{hai3 shui3}[][HSK 4]
    \definition[把]{s.}{água do mar; salmoura}
  \end{Phonetics}
\end{Entry}

\begin{Entry}{海风}{10,4}{⽔、⾵}
  \begin{Phonetics}{海风}{hai3feng1}
    \definition{s.}{brisa do mar | vento que vem do mar}
  \end{Phonetics}
\end{Entry}

\begin{Entry}{海外}{10,5}{⽔、⼣}
  \begin{Phonetics}{海外}{hai3 wai4}[][HSK 6]
    \definition[次]{s.}{fora das fronteiras nacionais; no exterior}
  \end{Phonetics}
\end{Entry}

\begin{Entry}{海边}{10,5}{⽔、⾡}
  \begin{Phonetics}{海边}{hai3 bian1}[][HSK 2]
    \definition{s.}{praia; costa; litoral; orla marítima; a parte marginal do oceano e as grandes áreas de água salgada cercadas por terra firme, onde a terra e a água se encontram, formam a costa}
  \end{Phonetics}
\end{Entry}

\begin{Entry}{海关}{10,6}{⽔、⼋}
  \begin{Phonetics}{海关}{hai3guan1}[][HSK 3]
    \definition[个]{s.}{alfândega; órgão administrativo nacional, sua principal função é supervisionar e inspecionar os bens e meios de transporte que entram e saem do país, cobrar impostos alfandegários e reprimir o contrabando}
  \end{Phonetics}
\end{Entry}

\begin{Entry}{海军}{10,6}{⽔、⼍}
  \begin{Phonetics}{海军}{hai3 jun1}[][HSK 6]
    \definition[支,名,位,个]{s.}{marinha; o exército que luta no mar geralmente é composto por navios de superfície, submarinos, aviação naval, fuzileiros navais e outros ramos, além de diversas forças profissionais}
  \end{Phonetics}
\end{Entry}

\begin{Entry}{海报}{10,7}{⽔、⼿}
  \begin{Phonetics}{海报}{hai3 bao4}[][HSK 6]
    \definition[张,份,幅]{s.}{pôster; cartaz; cartazes anunciando apresentações culturais, exibições de filmes ou competições esportivas, etc.}
  \end{Phonetics}
\end{Entry}

\begin{Entry}{海运}{10,7}{⽔、⾡}
  \begin{Phonetics}{海运}{hai3yun4}[][HSK 7-9]
    \definition{s.}{transporte marítimo; transporte oceânico}
    \definition{v.}{transportar pelo mar}
  \end{Phonetics}
\end{Entry}

\begin{Entry}{海里}{10,7}{⽔、⾥}
  \begin{Phonetics}{海里}{hai3li3}
    \definition{s.}{milha náutica}
  \end{Phonetics}
\end{Entry}

\begin{Entry}{海岸}{10,8}{⽔、⼭}
  \begin{Phonetics}{海岸}{hai3'an4}[][HSK 7-9]
    \definition[段]{s.}{litoral; costa; praia}
  \end{Phonetics}
\end{Entry}

\begin{Entry}{海底}{10,8}{⽔、⼴}
  \begin{Phonetics}{海底}{hai3 di3}[][HSK 6]
    \definition{s.}{fundo do mar; fundo do oceano; solo oceânico}
  \end{Phonetics}
\end{Entry}

\begin{Entry}{海拔}{10,8}{⽔、⼿}
  \begin{Phonetics}{海拔}{hai3ba2}[][HSK 7-9]
    \definition{s.}{altitude; altura em relação ao nível médio do mar}
  \end{Phonetics}
\end{Entry}

\begin{Entry}{海峡}{10,9}{⽔、⼭}
  \begin{Phonetics}{海峡}{hai3xia2}[][HSK 7-9]
    \definition*{s.}{Estreito de Taiwan}
    \definition[个]{s.}{estreito; canal; um canal estreito que conecta dois oceanos entre duas massas de terra}
  \end{Phonetics}
\end{Entry}

\begin{Entry}{海洋}{10,9}{⽔、⽔}
  \begin{Phonetics}{海洋}{hai3yang2}[][HSK 6]
    \definition[片,个]{s.}{mar; oceano; um termo geral para os mares e oceanos que formam uma entidade contínua na superfície da Terra; também pode ser usado para descrever um grande número de coisas semelhantes}
  \end{Phonetics}
\end{Entry}

\begin{Entry}{海面}{10,9}{⽔、⾯}
  \begin{Phonetics}{海面}{hai3mian4}[][HSK 7-9]
    \definition{s.}{nível do mar | superfície do mar}
  \end{Phonetics}
\end{Entry}

\begin{Entry}{海鸥}{10,9}{⽔、⿃}
  \begin{Phonetics}{海鸥}{hai3'ou1}
    \definition{s.}{gaivota}
  \end{Phonetics}
\end{Entry}

\begin{Entry}{海浪}{10,10}{⽔、⽔}
  \begin{Phonetics}{海浪}{hai3 lang4}[][HSK 6]
    \definition{s.}{ondas do mar}
  \end{Phonetics}
\end{Entry}

\begin{Entry}{海啸}{10,11}{⽔、⼝}
  \begin{Phonetics}{海啸}{hai3xiao4}[][HSK 7-9]
    \definition{s.}{\emph{tsunami}; maremoto}
  \end{Phonetics}
\end{Entry}

\begin{Entry}{海域}{10,11}{⽔、⼟}
  \begin{Phonetics}{海域}{hai3yu4}[][HSK 7-9]
    \definition{s.}{área marítima; espaço marítimo; refere-se a uma determinada área do oceano (tanto acima quanto abaixo da água)}
  \end{Phonetics}
\end{Entry}

\begin{Entry}{海盗}{10,11}{⽔、⽫}
  \begin{Phonetics}{海盗}{hai3dao4}[][HSK 7-9]
    \definition{s.}{pirata; viajante do mar | bucaneiro; viking; pirata}
  \end{Phonetics}
\end{Entry}

\begin{Entry}{海绵}{10,11}{⽔、⽷}
  \begin{Phonetics}{海绵}{hai3mian2}[][HSK 7-9]
    \definition{s.}{espuma de borracha; espuma de plástico; esponja; um material poroso feito de borracha ou plástico que é elástico, como uma esponja | esponja marinha seca; poríferos; osso esponjoso; refere-se especificamente ao esqueleto queratinoso das esponjas}
  \end{Phonetics}
\end{Entry}

\begin{Entry}{海棠}{10,12}{⽔、⽊}
  \begin{Phonetics}{海棠}{hai3tang2}
    \definition{s.}{begônia}
  \end{Phonetics}
\end{Entry}

\begin{Entry}{海湾}{10,12}{⽔、⽔}
  \begin{Phonetics}{海湾}{hai3 wan1}[][HSK 6]
    \definition{s.}{baía; golfo | lago}
  \end{Phonetics}
\end{Entry}

\begin{Entry}{海量}{10,12}{⽔、⾥}
  \begin{Phonetics}{海量}{hai3liang4}[][HSK 7-9]
    \definition{s.}{magnanimidade | alta tolerância ao álcool | cargas de; uma grande quantidade de}
  \end{Phonetics}
\end{Entry}

\begin{Entry}{海滨}{10,13}{⽔、⽔}
  \begin{Phonetics}{海滨}{hai3bin1}[][HSK 7-9]
    \definition{s.}{praia; beira-mar; litoral; um lugar perto do mar}
  \end{Phonetics}
\end{Entry}

\begin{Entry}{海滩}{10,13}{⽔、⽔}
  \begin{Phonetics}{海滩}{hai3tan1}[][HSK 7-9]
    \definition[个,片]{s.}{praia; praia com declive suave em direção ao mar}
  \end{Phonetics}
\end{Entry}

\begin{Entry}{海鲜}{10,14}{⽔、⿂}
  \begin{Phonetics}{海鲜}{hai3xian1}[][HSK 4]
    \definition[种,份,桌,批,些]{s.}{frutos do mar; mariscos; peixes marinhos frescos, camarões, etc., para consumo |}
  \end{Phonetics}
\end{Entry}

\begin{Entry}{海藻}{10,19}{⽔、⾋}
  \begin{Phonetics}{海藻}{hai3zao3}[][HSK 7-9]
    \definition{s.}{alga marinha; planta marítima}
  \end{Phonetics}
\end{Entry}

%%%%%%%%%% 浸 %%%%%%%%%%
\subsection*{浸}

\begin{Entry}{浸}{10}{⽔}
  \begin{Phonetics}{浸}{jin4}
    \definition{adv.}{gradualmente; passo a passo; pouco a pouco | Literário: gradualmente; cada vez mais}
    \definition{v.}{deixar de molho; imergir; mergulhar | saturar}
  \end{Phonetics}
\end{Entry}

\begin{Entry}{浸泡}{10,8}{⽔、⽔}
  \begin{Phonetics}{浸泡}{jin4pao4}[][HSK 7-9]
    \definition{v.}{mergulhar; banhar; imergir; deixar de molho em líquido}
  \end{Phonetics}
\end{Entry}

%%%%%%%%%% 消 %%%%%%%%%%
\subsection*{消}

\begin{Entry}{消}{10}{⽔}
  \begin{Phonetics}{消}{xiao1}
    \definition{v.}{desaparecer | dissipar; remover; eliminar; fazer desaparecer | passar o tempo de forma descontraída (recreação) | precisar; tomar (necessidade, geralmente precedido por 不, 几, 何)}
  \seealsoref{不}{bu4}
  \seealsoref{何}{he2}
  \seealsoref{几}{ji3}
  \end{Phonetics}
\end{Entry}

\begin{Entry}{消化}{10,4}{⽔、⼔}
  \begin{Phonetics}{消化}{xiao1hua4}[][HSK 4]
    \definition{v.}{digerir (alimentos) | digerir (conhecimento); pensar e absorver; uma metáfora para a compreensão total de novos conhecimentos ou informações e a capacidade de transformá-los em algo que possa ser usado}
  \end{Phonetics}
\end{Entry}

\begin{Entry}{消失}{10,5}{⽔、⼤}
  \begin{Phonetics}{消失}{xiao1shi1}[][HSK 3]
    \definition{v.}{desaparecer; desvanecer; dissolver; dissipar; evaporar; sumir}
  \end{Phonetics}
\end{Entry}

\begin{Entry}{消灭}{10,5}{⽔、⽕}
  \begin{Phonetics}{消灭}{xiao1mie4}[][HSK 6]
    \definition{v.}{perecer; morrer; falecer; desaparecer | abolir; erradicar; eliminar; aniquilar; exterminar; acabar com; fazer com que não exista}
  \end{Phonetics}
\end{Entry}

\begin{Entry}{消防}{10,6}{⽔、⾩}
  \begin{Phonetics}{消防}{xiao1fang2}[][HSK 5]
    \definition{s.}{combate a incêncios; controle de incêndios}
  \end{Phonetics}
\end{Entry}

\begin{Entry}{消防员}{10,6,7}{⽔、⾩、⼝}
  \begin{Phonetics}{消防员}{xiao1fang2yuan2}
    \definition{s.}{bombeiro}
  \end{Phonetics}
\end{Entry}

\begin{Entry}{消极}{10,7}{⽔、⽊}
  \begin{Phonetics}{消极}{xiao1ji2}[][HSK 5]
    \definition{adj.}{negativo; oposto; adverso | passivo; inativo; sem ambição; sem iniciativa; desanimado; apático}
  \end{Phonetics}
\end{Entry}

\begin{Entry}{消毒}{10,9}{⽔、⽏}
  \begin{Phonetics}{消毒}{xiao1du2}[][HSK 5]
    \definition{v.}{desinfetar; esterilizar; matar os microrganismos causadores de doenças por meios físicos ou químicos}
  \end{Phonetics}
\end{Entry}

\begin{Entry}{消费}{10,9}{⽔、⾙}
  \begin{Phonetics}{消费}{xiao1fei4}[][HSK 3]
    \definition{v.}{gastar; consumir; consumir materiais para satisfazer as necessidades de produção ou de vida (geralmente refere-se ao consumo doméstico) | consumir (recursos naturais)}
  \end{Phonetics}
\end{Entry}

\begin{Entry}{消费者}{10,9,8}{⽔、⾙、⽼}
  \begin{Phonetics}{消费者}{xiao1 fei4 zhe3}[][HSK 5]
    \definition{s.}{consumidor; cliente; consumo; indivíduos membros da sociedade que compram e utilizam bens e serviços para consumo pessoal}
  \end{Phonetics}
\end{Entry}

\begin{Entry}{消除}{10,9}{⽔、⾩}
  \begin{Phonetics}{消除}{xiao1chu2}[][HSK 5]
    \definition{v.}{dissipar; eliminar; limpar; tornar algo inexistente; remover (algo desfavorável)}
  \end{Phonetics}
\end{Entry}

\begin{Entry}{消息}{10,10}{⽔、⼼}
  \begin{Phonetics}{消息}{xiao1xi5}[][HSK 3]
    \definition[个,条,篇,些]{s.}{notícias; informação; reportagem sobre pessoas ou situações | notícias; novidades;}
  \end{Phonetics}
\end{Entry}

\begin{Entry}{消耗}{10,10}{⽔、⽾}
  \begin{Phonetics}{消耗}{xiao1hao4}[][HSK 6]
    \definition{v.}{gastar; esgotar; consumir; usar; (espírito, força, coisas, etc.) diminuir gradualmente devido ao uso ou perda}
  \end{Phonetics}
\end{Entry}

%%%%%%%%%% 涉 %%%%%%%%%%
\subsection*{涉}

\begin{Entry}{涉}{10}{⽔}
  \begin{Phonetics}{涉}{she4}[][HSK 6]
    \definition*{s.}{Sobrenome: She}
    \definition{v.}{vadear; atravessar ou passar um rio ou um obstáculo | passar por; experimentar | envolver; implicar}
  \end{Phonetics}
\end{Entry}

\begin{Entry}{涉及}{10,3}{⽔、⼃}
  \begin{Phonetics}{涉及}{she4ji2}[][HSK 6]
    \definition{v.}{envolver; relacionar-se com; referir-se a; tocar em}
  \end{Phonetics}
\end{Entry}

%%%%%%%%%% 涝 %%%%%%%%%%
\subsection*{涝}

\begin{Entry}{涝}{10}{⽔}
  \begin{Phonetics}{涝}{lao4}[][HSK 7-9]
    \definition{adj.}{inundado; alagado}
    \definition{s.}{alagamento (de terras ou plantações); água acumulada nos campos devido às fortes chuvas}
    \definition{v.}{inundar; alagar}
  \end{Phonetics}
\end{Entry}

%%%%%%%%%% 涨 %%%%%%%%%%
\subsection*{涨}

\begin{Entry}{涨}{10}{⽔}
  \begin{Phonetics}{涨}{zhang3}[][HSK 5,6]
    \definition{v.}{subir; inchar; aumentar; elevar; melhorar}
  \end{Phonetics}
  \begin{Phonetics}{涨}{zhang4}
    \definition{v.}{inchar; ter o volume aumentado | ser inundado por uma torrente de sangue; ter uma dor de cabeça; ficar com o rosto vermelho de raiva | ser mais, maior, etc. do que o esperado}
  \end{Phonetics}
\end{Entry}

\begin{Entry}{涨价}{10,6}{⽔、⼈}
  \begin{Phonetics}{涨价}{zhang3/jia4}[][HSK 5]
    \definition{s.}{aumento de preços}
    \definition{v.+compl.}{(preços) subir; aumentar o preço}
  \end{Phonetics}
\end{Entry}

%%%%%%%%%% 烈 %%%%%%%%%%
\subsection*{烈}

\begin{Entry}{烈}{10}{⽕}
  \begin{Phonetics}{烈}{lie4}
    \definition*{s.}{Sobrenome: Lie}
    \definition{adj.}{forte; violento; intenso; feroz | justo; severo | firme; convicto; rigoroso}
    \definition{s.}{pessoa que morreu por uma causa justa | conquistas; façanhas | mártir sacrificando-se por uma causa justa}
  \end{Phonetics}
\end{Entry}

\begin{Entry}{烈士}{10,3}{⽕、⼠}
  \begin{Phonetics}{烈士}{lie4shi4}[][HSK 7-9]
    \definition[位,名]{s.}{mártir; pessoas que se sacrificaram pela causa da justiça | uma pessoa de grande empenho; na antiguidade, referia-se a uma pessoa que aspirava a alcançar grandes feitos}
  \end{Phonetics}
\end{Entry}

%%%%%%%%%% 烘 %%%%%%%%%%
\subsection*{烘}

\begin{Entry}{烘}{10}{⽕}
  \begin{Phonetics}{烘}{hong1}
    \definition{v.}{secar; assar; aquecer; usar fogo ou vapor para aquecer o corpo ou para cozinhar, aquecer ou secar algo | destacar}
  \end{Phonetics}
\end{Entry}

\begin{Entry}{烘干}{10,3}{⽕、⼲}
  \begin{Phonetics}{烘干}{hong1gan1}[][HSK 7-9]
    \definition{v.}{secar em fogo alto | secar ao lado ou sobre o fogo | assar; secar no forno}
  \end{Phonetics}
\end{Entry}

\begin{Entry}{烘托}{10,6}{⽕、⼿}
  \begin{Phonetics}{烘托}{hong1tuo1}[][HSK 7-9]
    \definition{v.}{adicionar sombreamento ao redor de um objeto para destacá-lo; um dos métodos de pintura chinesa, que utiliza tinta ou cores claras para pontilhar o contorno do objeto e torná-lo mais claro | destacar por contraste; colocar em nítido relevo; fazer com que se destaque}
  \end{Phonetics}
\end{Entry}

%%%%%%%%%% 烟 %%%%%%%%%%
\subsection*{烟}

\begin{Entry}{烟}{10}{⽕}
  \begin{Phonetics}{烟}{yan1}[][HSK 3]
    \definition[股,支,根,盒,包]{s.}{fumaça; gás produzido pela combustão de materiais, misturado com pequenas partículas não completamente queimadas | névoa; neblina | tabaco; planta de tabaco | fumo; cigarro; termo geral para cigarros, charutos, etc. | ópio | fuligem; fumaça de carvão}
    \definition{v.}{ficar irritado com a fumaça (os olhos lacrimejam ou não conseguem abrir)}
  \end{Phonetics}
\end{Entry}

\begin{Entry}{烟火}{10,4}{⽕、⽕}
  \begin{Phonetics}{烟火}{yan1huo3}
    \definition{s.}{fogo de artifício}
  \end{Phonetics}
\end{Entry}

\begin{Entry}{烟叶}{10,5}{⽕、⼝}
  \begin{Phonetics}{烟叶}{yan1ye4}
    \definition{s.}{folha de tabaco}
  \end{Phonetics}
\end{Entry}

\begin{Entry}{烟头}{10,5}{⽕、⼤}
  \begin{Phonetics}{烟头}{yan1tou2}
    \definition[根]{s.}{bituca de cigarro}
  \end{Phonetics}
\end{Entry}

\begin{Entry}{烟囱}{10,7}{⽕、⼞}
  \begin{Phonetics}{烟囱}{yan1cong1}
    \definition{s.}{chaminé}
  \end{Phonetics}
\end{Entry}

\begin{Entry}{烟花}{10,7}{⽕、⾋}
  \begin{Phonetics}{烟花}{yan1 hua1}[][HSK 6]
    \definition[场,朵]{s.}{fogos de artifício; uma coisa que emite faíscas de várias cores quando exposta à observação | prostituta; antigamente, referia-se a algo relacionado à prostituição}
  \end{Phonetics}
\end{Entry}

\begin{Entry}{烟雨}{10,8}{⽕、⾬}
  \begin{Phonetics}{烟雨}{yan1yu3}
    \definition{s.}{chuvisco | garoa}
  \end{Phonetics}
\end{Entry}

\begin{Entry}{烟草}{10,9}{⽕、⾋}
  \begin{Phonetics}{烟草}{yan1cao3}
    \definition{s.}{tabaco}
  \end{Phonetics}
\end{Entry}

%%%%%%%%%% 烤 %%%%%%%%%%
\subsection*{烤}

\begin{Entry}{烤}{10}{⽕}
  \begin{Phonetics}{烤}{kao3}
    \definition{v.}{assar | grelhar}
  \end{Phonetics}
\end{Entry}

\begin{Entry}{烤肉}{10,6}{⽕、⾁}
  \begin{Phonetics}{烤肉}{kao3 rou4}[][HSK 5]
    \definition[块,串,片,盘]{s.}{churrasco (literalmente carne assada)}
  \end{Phonetics}
\end{Entry}

\begin{Entry}{烤鸭}{10,10}{⽕、⿃}
  \begin{Phonetics}{烤鸭}{kao3ya1}[][HSK 5]
    \definition[只,盘]{s.}{pato assado; pato recheado e assado em um forno especial após ser abatido}
  \end{Phonetics}
\end{Entry}

%%%%%%%%%% 烦 %%%%%%%%%%
\subsection*{烦}

\begin{Entry}{烦}{10}{⽕}
  \begin{Phonetics}{烦}{fan2}[][HSK 4]
    \definition{adj.}{redundante e confuso | supérfluo e confuso; muito bagunçado}
    \definition{v.}{aborrecer | irritar; incomodar; estar cansado de; ficar irritado | incomodar; solicitar}
  \end{Phonetics}
\end{Entry}

\begin{Entry}{烦闷}{10,7}{⽕、⾨}
  \begin{Phonetics}{烦闷}{fan2men4}[][HSK 7-9]
    \definition{adj.}{infeliz; deprimido; mal-humorado | desconfortável}
  \end{Phonetics}
\end{Entry}

\begin{Entry}{烦恼}{10,9}{⽕、⼼}
  \begin{Phonetics}{烦恼}{fan2nao3}[][HSK 7-9]
    \definition{adj.}{irritado; preocupado; incomodado}
    \definition[个,种,些]{s.}{aborrecimento; coisas que te incomodam}
  \end{Phonetics}
\end{Entry}

\begin{Entry}{烦躁}{10,20}{⽕、⾜}
  \begin{Phonetics}{烦躁}{fan2zao4}[][HSK 7-9]
    \definition{adj.}{inquieto; agitado; irritável}
  \end{Phonetics}
\end{Entry}

%%%%%%%%%% 烧 %%%%%%%%%%
\subsection*{烧}

\begin{Entry}{烧}{10}{⽕}
  \begin{Phonetics}{烧}{shao1}[][HSK 4]
    \definition[次]{s.}{febre; temperatura corporal mais alta do que o normal}
    \definition{v.}{queimar; pegar fogo | cozinhar; aquecer; assar | guisar depois de fritar ou fritar depois de guisar | assar; grelhar os ingredientes dos alimentos diretamente sobre o fogo | ter febre; estar com febre | danificar (matar ou murchar) as plantas pelo uso excessivo (ou inadequado) de fertilizantes | tornar-se arrogante ou presunçoso; metáfora de estar em uma boa posição e se deixar levar}
  \end{Phonetics}
\end{Entry}

\begin{Entry}{烧烤}{10,10}{⽕、⽕}
  \begin{Phonetics}{烧烤}{shao1kao3}
    \definition{s.}{churrasco}
    \definition{v.}{assar}
  \end{Phonetics}
\end{Entry}

%%%%%%%%%% 热 %%%%%%%%%%
\subsection*{热}

\begin{Entry}{热}{10}{⽕}
  \begin{Phonetics}{热}{re4}[][HSK 1]
    \definition{adj.}{quente; temperatura elevada | ardente; caloroso; profundamente afetuoso | ansioso; invejoso; descreve inveja e desejo de possuir algo | térmico; altamente radioativo | popular; muito procurado; muito apreciado por muitas pessoas}
    \definition{s.}{calor; energia liberada pelo movimento irregular das moléculas dentro de um objeto | febre; febre alta causada por doença | moda passageira; mania; febre}
    \definition{v.}{aquecer (geralmente se refere a alimentos)}
  \end{Phonetics}
\end{Entry}

\begin{Entry}{热门}{10,3}{⽕、⾨}
  \begin{Phonetics}{热门}{re4men2}[][HSK 5]
    \definition{adj.}{popular; durante um período de tempo, foi algo que interessava a todos}
    \definition{s.}{algo que desperta o interesse popular; metáfora para algo que está na moda e recebe a atenção de todos (em contraste com 冷门)}
  \seealsoref{冷门}{leng3men2}
  \end{Phonetics}
\end{Entry}

\begin{Entry}{热心}{10,4}{⽕、⼼}
  \begin{Phonetics}{热心}{re4xin1}[][HSK 4]
    \definition{adj.}{ardente; sincero; entusiasmado; afetuoso; apaixonado; interessado}
    \definition{v.}{ser entusiasmado com alguma coisa}
  \end{Phonetics}
\end{Entry}

\begin{Entry}{热水}{10,4}{⽕、⽔}
  \begin{Phonetics}{热水}{re4 shui3}[][HSK 6]
    \definition{s.}{água quente; água em temperatura mais alta}
  \end{Phonetics}
\end{Entry}

\begin{Entry}{热水器}{10,4,16}{⽕、⽔、⼝}
  \begin{Phonetics}{热水器}{re4 shui3 qi4}[][HSK 6]
    \definition[台]{s.}{aquecedor de água; aparelhos que aquecem água usando eletricidade, gás natural, gás liquefeito de petróleo ou energia solar}
  \end{Phonetics}
\end{Entry}

\begin{Entry}{热血沸腾}{10,6,8,13}{⽕、⾎、⽔、⾁}
  \begin{Phonetics}{热血沸腾}{re4xue4-fei4teng2}
    \definition{expr.}{estar animado; ter o sangue correndo}
  \end{Phonetics}
\end{Entry}

\begin{Entry}{热泪盈眶}{10,8,9,11}{⽕、⽔、⽫、⽬}
  \begin{Phonetics}{热泪盈眶}{re4lei4ying2kuang4}
    \definition{expr.}{olhos cheios de lágrimas de emoção | extremamente emocionado}
  \end{Phonetics}
\end{Entry}

\begin{Entry}{热线}{10,8}{⽕、⽷}
  \begin{Phonetics}{热线}{re4 xian4}[][HSK 6]
    \definition[条]{s.}{raio infravermelho | linha direta; \emph{hot line}; uma linha telefônica ou telegráfica direta; uma linha para um ponto de acesso | rota quente (ou movimentada, popular) | raio de calor}
  \end{Phonetics}
\end{Entry}

\begin{Entry}{热闹}{10,8}{⽕、⾾}
  \begin{Phonetics}{热闹}{re4nao5}[][HSK 4]
    \definition{adj.}{animado; agitado; movimentado com barulho e excitação; descreve uma cena animada com uma atmosfera calorosa}
    \definition{s.}{uma vista emocionante; uma cena de agitação e excitação; atmosfera acolhedora}
    \definition{v.}{animar; divertir-se}
  \end{Phonetics}
\end{Entry}

\begin{Entry}{热点}{10,9}{⽕、⽕}
  \begin{Phonetics}{热点}{re4 dian3}[][HSK 6]
    \definition{s.}{ponto de acesso; \emph{hotspot}}
  \end{Phonetics}
\end{Entry}

\begin{Entry}{热烈}{10,10}{⽕、⽕}
  \begin{Phonetics}{热烈}{re4lie4}[][HSK 3]
    \definition{adj.}{caloroso; fervoroso; ardente; entusiasmado; excitado}
  \end{Phonetics}
\end{Entry}

\begin{Entry}{热爱}{10,10}{⽕、⽖}
  \begin{Phonetics}{热爱}{re4'ai4}[][HSK 3]
    \definition{v.}{amar ardentemente; amar de coração; ter amor profundo por; amar apaixonadamente}
  \end{Phonetics}
\end{Entry}

\begin{Entry}{热情}{10,11}{⽕、⼼}
  \begin{Phonetics}{热情}{re4qing2}[][HSK 2]
    \definition{adj.}{caloroso; fervoroso; entusiasmado; cordial; descreve sentimentos calorosos por alguém}
    \definition{s.}{entusiasmo; ardor; devoção; calor humano; zelo; sentimentos calorosos}
  \end{Phonetics}
\end{Entry}

\begin{Entry}{热量}{10,12}{⽕、⾥}
  \begin{Phonetics}{热量}{re4 liang4}[][HSK 5]
    \definition{s.}{calor; quantidade de calor; calorias; em física, refere-se à energia transferida entre objetos com temperaturas diferentes, do objeto com temperatura mais alta para o objeto com temperatura mais baixa}
  \end{Phonetics}
\end{Entry}

%%%%%%%%%% 爱 %%%%%%%%%%
\subsection*{爱}

\begin{Entry}{爱}{10}{⽖}
  \begin{Phonetics}{爱}{ai4}[][HSK 1]
    \definition*{s.}{Sobrenome: Ai}
    \definition[个]{s.}{amor; afeição profunda; preocupação profunda; especialmente amor entre pessoas}[爱是理解和包容。===O amor é compreensão e tolerância.]
    \definition{v.}{amar; ter sentimentos profundos por pessoas ou coisas | gostar; gostar de; estar interessado em |  cuidar; valorizar; ter em alta estima; cuidar bem de | estar apto a; ter o hábito de}[他们深深爱着对方。===Eles se amam profundamente. | 我爱我的家人。===Eu amo minha família. | 我爱旅行。===Eu adoro viajar.]
  \end{Phonetics}
\end{Entry}

\begin{Entry}{爱人}{10,2}{⽖、⼈}
  \begin{Phonetics}{爱人}{ai4 ren5}[][HSK 2]
    \definition[个]{s.}{amante; \emph{dollbaby}; namorado(a) | marido ou esposa; mais usado em ocasiões formais}[这是我的爱人。===Este é o meu/minha esposo/companheiro. | 她是我一生的爱人。===Ela é o amor da minha vida. | 请携带爱人出席晚宴。===Por favor, traga seu cônjuge para o jantar.]
  \end{Phonetics}
\end{Entry}

\begin{Entry}{爱上}{10,3}{⽖、⼀}
  \begin{Phonetics}{爱上}{ai4shang4}
    \definition{v.}{perder o coração por; apaixonar-se por}[他在旅行时爱上了一位法国女孩。===Ele se apaixonou por uma garota francesa durante a viagem.  | 来到杭州后,我爱上了龙井茶。===Depois de chegar em Hangzhou, me apaixonei pelo chá Longjing. | 我从来没想过自己会爱上健身。===Eu nunca imaginei que iria me apaixonar por academia.]
  \end{Phonetics}
\end{Entry}

\begin{Entry}{爱不释手}{10,4,12,4}{⽖、⼀、⾤、⼿}
  \begin{Phonetics}{爱不释手}{ai4bu2shi4shou3}[][HSK 7-9]
    \definition{expr.}{``Não consigo parar de ler.''; ``Amo tanto que não consigo deixar passar.''; gostar (amar) algo tanto que não se pode suportar separar-se dele}
  \end{Phonetics}
\end{Entry}

\begin{Entry}{爱心}{10,4}{⽖、⼼}
  \begin{Phonetics}{爱心}{ai4xin1}[][HSK 3]
    \definition[片]{s.}{amor; carinho; compaixão; um sentimento de preocupação e carinho por outras pessoas ou animais}
  \end{Phonetics}
\end{Entry}

\begin{Entry}{爱好}{10,6}{⽖、⼥}
  \begin{Phonetics}{爱好}{ai4 hao4}[][HSK 1]
    \definition[个,种]{s.}{passatempo; interesse; \emph{hobby}; sentimentos de interesse especial ou afeição por algo | 爱好 é mais usado para atividades regulares (esportes, música), enquanto 喜欢 é para preferências gerais}[他的爱好是收集邮票。===Seu hobby era colecionar selos.  | 我的爱好是读书和旅行。===Meus hobbies são ler e viajar.]
    \definition{v.}{estar interessado em; ter prazer em; ter um forte interesse em algo; ter sentimentos profundos por alguém ou algo}
  \seealsoref{喜欢}{xi3huan5}
  \end{Phonetics}
\end{Entry}

\begin{Entry}{爱好者}{10,6,8}{⽖、⼥、⽼}
  \begin{Phonetics}{爱好者}{ai4 hao4 zhe3}
    \definition{s.}{hobbista; amador; entusiasta; fã; amante (de arte, esportes, etc.)}[他是一位摄影爱好者。===Ele é um entusiasta de fotografia. | 她是位潜水爱好者,经常去东南亚潜水。===Ela é uma mergulhadora amadora e frequentemente mergulha no Sudeste Asiático.  | 我们为书法爱好者创建了一个微信群。===Criamos um grupo no WeChat para amantes de caligrafia.]
  \end{Phonetics}
\end{Entry}

\begin{Entry}{爱抚}{10,7}{⽖、⼿}
  \begin{Phonetics}{爱抚}{ai4fu3}
    \definition{s.}{carinho; carícia}
    \definition{v.}{acariciar; afagar; cuidar (com ternura)}[他轻轻爱抚她的头发。===Ele afagou suavemente o cabelo dela. | 母亲爱抚婴儿的脸颊。===A mãe acaricia a bochecha do bebê. | 她爱抚着小猫的耳朵。===Ela acariciou as orelhas do gatinho.]
  \end{Phonetics}
\end{Entry}

\begin{Entry}{爱护}{10,7}{⽖、⼿}
  \begin{Phonetics}{爱护}{ai4hu4}[][HSK 4]
    \definition{v.}{acalentar; valorizar; salvaguardar; cuidar bem de}[全社会都应爱护老年人。===Toda a sociedade deve tratar os idosos com cuidado e respeito. | 请爱护公园里的小动物。===Por favor, tratem os animais do parque com cuidado.]
  \end{Phonetics}
\end{Entry}

\begin{Entry}{爱国}{10,8}{⽖、⼞}
  \begin{Phonetics}{爱国}{ai4 guo2}[][HSK 4]
    \definition{adj.}{patriótico; patriotismo}[爱国是每个公民的责任。===O patriotismo é o dever de todo cidadão. | 这部电影讲述了英雄的爱国故事。===Este filme conta a história patriótica de um herói.]
    \definition{v.}{ser patriota; amar o seu país}
  \end{Phonetics}
\end{Entry}

\begin{Entry}{爱面子}{10,9,3}{⽖、⾯、⼦}
  \begin{Phonetics}{爱面子}{ai4/mian4zi5}[][HSK 7-9]
    \definition{v.+compl.}{estar preocupado em salvar a face; ser vigilante em relação à reputação; ser sensível ao próprio orgulho; valorizar minha própria dignidade e ter medo que os outros me desprezem}[他爱面子,怕别人笑话他。===Ele se importa com sua reputação e tem medo que os outros riam dele.]
  \end{Phonetics}
\end{Entry}

\begin{Entry}{爱爱}{10,10}{⽖、⽖}
  \begin{Phonetics}{爱爱}{ai4'ai5}
    \definition{v.}{Coloquial: fazer amor ou relações íntimas | pode ser usado como um apelido entre casais, transmitindo ternura | pode soar vulgar se usado em contextos inadequados}[他们俩刚结婚,天天都想爱爱。===Eles acabaram de se casar e querem fazer amor todo dia. | 爱爱,你今天好漂亮!===Amor, você está linda hoje!]
  \end{Phonetics}
\end{Entry}

\begin{Entry}{爱情}{10,11}{⽖、⼼}
  \begin{Phonetics}{爱情}{ai4qing2}[][HSK 2]
    \definition{s.}{amor (entre pessoas); afeição}[爱情是盲目的。===O amor é cego. | 爱情如同玫瑰,美丽却带刺。===O amor é como uma rosa, bela mas com espinhos.  | 这首歌讲述了破碎的爱情故事。===Esta música conta uma história de amor fracassado.]
  \end{Phonetics}
\end{Entry}

\begin{Entry}{爱惜}{10,11}{⽖、⼼}
  \begin{Phonetics}{爱惜}{ai4xi1}[][HSK 7-9]
    \definition{v.}{valorizar; prezar; estimar; usar com moderação; não desperdiçar}
  \end{Phonetics}
\end{Entry}

\begin{Entry}{爱理不理}{10,11,4,11}{⽖、⽟、⼀、⽟}
  \begin{Phonetics}{爱理不理}{ai4li3-bu4li3}[][HSK 7-9]
    \definition{expr.}{frio e indiferente; distante}
  \end{Phonetics}
\end{Entry}

\begin{Entry}{爱戴}{10,17}{⽖、⼽}
  \begin{Phonetics}{爱戴}{ai4dai4}
    \definition{v.}{reverenciar; adorar; amar profundamente e respeitar (a líderes, celebridades, etc.)}
  \end{Phonetics}
\end{Entry}

%%%%%%%%%% 爹 %%%%%%%%%%
\subsection*{爹}

\begin{Entry}{爹}{10}{⽗}
  \begin{Phonetics}{爹}{die1}[][HSK 7-9]
    \definition[个]{s.}{Coloquial: pai; papai | velho pai; um título respeitoso para homens idosos em algumas áreas}
  \end{Phonetics}
\end{Entry}

%%%%%%%%%% 特 %%%%%%%%%%
\subsection*{特}

\begin{Entry}{特}{10}{⽜}
  \begin{Phonetics}{特}{te4}[][HSK 6]
    \definition{adj.}{especial; incomum; particular; excepcional; diferente do geral | especial; solteiro; solitário}
    \definition{adv.}{muito; extremamente | especialmente; para um propósito especial |mas; somente}
    \definition{clas.}{TEX; abreviação para unidades de medida como TEX; a unidade de medida TEX indica a espessura de um fio têxtil através do seu peso}
    \definition{s.}{espião; agente secreto}
  \end{Phonetics}
\end{Entry}

\begin{Entry}{特大}{10,3}{⽜、⼤}
  \begin{Phonetics}{特大}{te4 da4}[][HSK 6]
    \definition{adj.}{especialmente (excepcionalmente) grande; o mais}
  \end{Phonetics}
\end{Entry}

\begin{Entry}{特价}{10,6}{⽜、⼈}
  \begin{Phonetics}{特价}{te4 jia4}[][HSK 4]
    \definition{s.}{oferta especial; preço de barganha; preço especial reduzido}
  \end{Phonetics}
\end{Entry}

\begin{Entry}{特地}{10,6}{⽜、⼟}
  \begin{Phonetics}{特地}{te4 di4}[][HSK 6]
    \definition{adv.}{especialmente; propositalmente; para um propósito especial}
  \end{Phonetics}
\end{Entry}

\begin{Entry}{特有}{10,6}{⽜、⽉}
  \begin{Phonetics}{特有}{te4 you3}[][HSK 5]
    \definition{adj.}{específico; peculiar; característico; único; exclusivo; especial}
  \end{Phonetics}
\end{Entry}

\begin{Entry}{特色}{10,6}{⽜、⾊}
  \begin{Phonetics}{特色}{te4se4}[][HSK 3]
    \definition{s.}{característica; característica distintiva; a cor única, estilo, etc. de um objeto}
  \end{Phonetics}
\end{Entry}

\begin{Entry}{特别}{10,7}{⽜、⼑}
  \begin{Phonetics}{特别}{te4bie2}[][HSK 2]
    \definition{adj.}{especial; particular; fora do comum; diferente dos outros, com características próprias}
    \definition{adv.}{especialmente; particularmente | ainda mais; em particular; frequentemente usado com 是 | especialmente; deliberadamente; para um propósito específico}
  \seealsoref{是}{shi4}
  \end{Phonetics}
\end{Entry}

\begin{Entry}{特别快车}{10,7,7,4}{⽜、⼑、⼼、⾞}
  \begin{Phonetics}{特别快车}{te4bie2 kuai4che1}
    \definition{s.}{trem expresso; expresso; expresso especial; refere-se a trens de passageiros que param em menos estações e têm menor tempo de viagem do que trens expressos diretos}
  \end{Phonetics}
\end{Entry}

\begin{Entry}{特快}{10,7}{⽜、⼼}
  \begin{Phonetics}{特快}{te4 kuai4}[][HSK 6]
    \definition{adj.}{expresso (trem, entrega etc.)}
    \definition{s.}{trem expresso (opp. 普快); abreviação de 特别快车}
  \seealsoref{特别快车}{te4bie2 kuai4che1}
  \end{Phonetics}
\end{Entry}

\begin{Entry}{特技}{10,7}{⽜、⼿}
  \begin{Phonetics}{特技}{te4ji4}
    \definition{s.}{efeito especial | dublê}
  \end{Phonetics}
\end{Entry}

\begin{Entry}{特定}{10,8}{⽜、⼧}
  \begin{Phonetics}{特定}{te4ding4}[][HSK 5]
    \definition{adj.}{específico; especialmente designado | dado; especificado; específico (pessoa, hora, lugar, local, ambiente, etc.)}
  \end{Phonetics}
\end{Entry}

\begin{Entry}{特征}{10,8}{⽜、⼻}
  \begin{Phonetics}{特征}{te4zheng1}[][HSK 4]
    \definition[个,种]{s.}{característica; aparência ou o fenômeno característico de uma pessoa ou coisa que pode ser visto de fora}
  \end{Phonetics}
\end{Entry}

\begin{Entry}{特性}{10,8}{⽜、⼼}
  \begin{Phonetics}{特性}{te4 xing4}[][HSK 5]
    \definition[种,个]{s.}{propriedade específica (ou característica) | característica; sabores | propriedade}
  \end{Phonetics}
\end{Entry}

\begin{Entry}{特点}{10,9}{⽜、⽕}
  \begin{Phonetics}{特点}{te4dian3}[][HSK 2]
    \definition[个,大]{s.}{característica; peculiaridade; traço distintivo; a singularidade de uma pessoa ou coisa}
  \end{Phonetics}
\end{Entry}

\begin{Entry}{特殊}{10,10}{⽜、⽍}
  \begin{Phonetics}{特殊}{te4shu1}[][HSK 4]
    \definition{adj.}{especial; particular; peculiar; excepcional; incomum}
  \end{Phonetics}
\end{Entry}

\begin{Entry}{特意}{10,13}{⽜、⼼}
  \begin{Phonetics}{特意}{te4yi4}[][HSK 6]
    \definition{adv.}{especialmente; para um propósito especial}
  \end{Phonetics}
\end{Entry}

%%%%%%%%%% 牺 %%%%%%%%%%
\subsection*{牺}

\begin{Entry}{牺}{10}{⽜}
  \begin{Phonetics}{牺}{xi1}
    \definition{s.}{um animal de cor uniforme para sacrifício; sacrifício; gado com pelagem pura usado para sacrifício}
  \end{Phonetics}
\end{Entry}

\begin{Entry}{牺牲}{10,9}{⽜、⽜}
  \begin{Phonetics}{牺牲}{xi1sheng1}[][HSK 6]
    \definition[份]{s.}{sacrifício; um animal abatido para sacrifício; refere-se ao sacrifício da própria vida ou dos próprios interesses por um propósito justo, ou refere-se ao preço pago por um determinado propósito}
    \definition{v.}{sacrificar-se; morrer como mártir; dar a própria vida; sacrificar sua vida pela justiça | sacrificar; desistir; fazer algo às custas de; geralmente se refere a pagar um preço ou sofrer danos por alguém ou algo}
  \end{Phonetics}
\end{Entry}

%%%%%%%%%% 狼 %%%%%%%%%%
\subsection*{狼}

\begin{Entry}{狼}{10}{⽝}
  \begin{Phonetics}{狼}{lang2}[][HSK 7-9]
    \definition*{s.}{Sírius (estrela) | Sobrenome: Lang}
    \definition[只,匹,群,条]{s.}{lobo}
  \end{Phonetics}
\end{Entry}

\begin{Entry}{狼狈}{10,7}{⽝、⽝}
  \begin{Phonetics}{狼狈}{lang2bei4}[][HSK 7-9]
    \definition{adj.}{em uma posição difícil; em um canto apertado; descreve um estado de angústia ou constrangimento}
  \end{Phonetics}
\end{Entry}

%%%%%%%%%% 猃 %%%%%%%%%%
\subsection*{猃}

\begin{Entry}{猃}{10}{⽝}
  \begin{Phonetics}{猃}{xian3}
    \definition{s.}{(arcaico) um tipo de cão com focinho longo}
  \end{Phonetics}
\end{Entry}

\begin{Entry}{猃狁}{10,7}{⽝、⽝}
  \begin{Phonetics}{猃狁}{xian3yun3}
    \definition*{s.}{Termo da dinastia Zhou para uma tribo nômade do norte mais tarde chamou o Xiongnu (匈奴) nas dinastias Qin e Han}
  \seealsoref{匈奴}{xiong1nu2}
  \end{Phonetics}
\end{Entry}

%%%%%%%%%% 珠 %%%%%%%%%%
\subsection*{珠}

\begin{Entry}{珠}{10}{⽟}
  \begin{Phonetics}{珠}{zhu1}
    \definition[粒,颗]{s.}{pérola | conta (de colar, ábaco, etc.) | coisa parecida com uma bola (como um globo ocular)}
  \end{Phonetics}
\end{Entry}

\begin{Entry}{珠子}{10,3}{⽟、⼦}
  \begin{Phonetics}{珠子}{zhu1zi5}
    \definition[粒,颗]{s.}{pérola | contas}
  \end{Phonetics}
\end{Entry}

\begin{Entry}{珠宝}{10,8}{⽟、⼧}
  \begin{Phonetics}{珠宝}{zhu1 bao3}[][HSK 6]
    \definition[串]{s.}{joias; pérolas; um termo geral para pérolas, pedras preciosas e outros ornamentos}
  \end{Phonetics}
\end{Entry}

%%%%%%%%%% 班 %%%%%%%%%%
\subsection*{班}

\begin{Entry}{班}{10}{⽟}
  \begin{Phonetics}{班}{ban1}[][HSK 1]
    \definition*{s.}{Sobrenome: Ban}
    \definition{adj.}{regular; programado; executado regularmente; com horários fixos (meios de transporte)}
    \definition{clas.}{um grupo de; uma classe de; usado para pessoas | meios de transporte com horários fixos}
    \definition[个]{s.}{equipe; turma; organização estruturada | dever; turno; período de trabalho dentro de um dia | equipe; esquadrão; unidade básica das forças armadas | nome usado antigamente para designar uma companhia teatral}
    \definition{v.}{mover; implantar; implementar}
  \end{Phonetics}
\end{Entry}

\begin{Entry}{班长}{10,4}{⽟、⾧}
  \begin{Phonetics}{班长}{ban1 zhang3}[][HSK 2]
    \definition[个,位,名]{s.}{monitor de turma; líder de equipe; alunos responsáveis nas turmas da escola | líder de esquadrão; responsável por uma turma de soldados, geralmente com patente de sargento}
  \end{Phonetics}
\end{Entry}

\begin{Entry}{班级}{10,6}{⽟、⽷}
  \begin{Phonetics}{班级}{ban1 ji2}[][HSK 3]
    \definition[个]{s.}{classe; série (na escola); o nome geral para as séries e turmas da escola}
  \end{Phonetics}
\end{Entry}

%%%%%%%%%% 瓶 %%%%%%%%%%
\subsection*{瓶}

\begin{Entry}{瓶}{10}{⽡}
  \begin{Phonetics}{瓶}{ping2}[][HSK 2]
    \definition*{s.}{Sobrenome: Ping}
    \definition{clas.}{usado para coisas que são engarrafadas; quantidade contida em um frasco, vaso, garrafa}
    \definition[个]{s.}{jarra; vaso; frasco; garrafa}
  \end{Phonetics}
\end{Entry}

\begin{Entry}{瓶子}{10,3}{⽡、⼦}
  \begin{Phonetics}{瓶子}{ping2zi5}[][HSK 2]
    \definition[个,只,种]{s.}{garrafa; recipientes com gargalo feitos de cerâmica, vidro, plástico, etc., geralmente em forma cilíndrica}
  \end{Phonetics}
\end{Entry}

\begin{Entry}{瓶盖}{10,11}{⽡、⽫}
  \begin{Phonetics}{瓶盖}{ping2gai4}
    \definition{s.}{tampa de garrafa}
  \end{Phonetics}
\end{Entry}

\begin{Entry}{瓶装}{10,12}{⽡、⾐}
  \begin{Phonetics}{瓶装}{ping2zhuang1}
    \definition{adj.}{engarrafado}
  \end{Phonetics}
\end{Entry}

%%%%%%%%%% 瓷 %%%%%%%%%%
\subsection*{瓷}

\begin{Entry}{瓷}{10}{⽡}
  \begin{Phonetics}{瓷}{ci2}[][HSK 7-9]
    \definition{adj.}{Dialeto: (relação) próxima; íntima}
    \definition{s.}{artigos de porcelana}
  \end{Phonetics}
\end{Entry}

\begin{Entry}{瓷器}{10,16}{⽡、⼝}
  \begin{Phonetics}{瓷器}{ci2qi4}[][HSK 7-9]
    \definition[件,种]{s.}{porcelana; louça; utensílios feitos de argila de porcelana, feldspato, quartzo, etc.}
  \end{Phonetics}
\end{Entry}

%%%%%%%%%% 留 %%%%%%%%%%
\subsection*{留}

\begin{Entry}{留}{10}{⽥}
  \begin{Phonetics}{留}{liu2}[][HSK 2]
    \definition*{s.}{Sobrenome: Liu}
    \definition{v.}{ficar; permanecer; parar em um determinado local ou posição; não se afastar | estudar no exterior (geralmente seguido pelo nome de um país com uma sílaba) | pedir a alguém para ficar; manter alguém onde está | concentrar-se em; concentrar a atenção em algo | manter; guardar; reservar; não joger fora | acumular; deixar crescer | aceitar; receber | transmitir (legado); deixar para trás}
  \end{Phonetics}
\end{Entry}

\begin{Entry}{留下}{10,3}{⽥、⼀}
  \begin{Phonetics}{留下}{liu2 xia4}[][HSK 2]
    \definition{v.}{deixar; parar em algum lugar}
  \end{Phonetics}
\end{Entry}

\begin{Entry}{留心}{10,4}{⽥、⼼}
  \begin{Phonetics}{留心}{liu2/xin1}[][HSK 7-9]
    \definition{v.+compl.}{cuidar; ser atencioso | ficar atento; estar atento (a)}
  \end{Phonetics}
\end{Entry}

\begin{Entry}{留言}{10,7}{⽥、⾔}
  \begin{Phonetics}{留言}{liu2 yan2}[][HSK 6]
    \definition[条]{s.}{mensagem; recado}
    \definition{v.}{deixar uma mensagem; deixar seus comentários}
  \end{Phonetics}
\end{Entry}

\begin{Entry}{留学}{10,8}{⽥、⼦}
  \begin{Phonetics}{留学}{liu2xue2}[][HSK 3]
    \definition{v.}{estudar no exterior; permanecer no estrangeiro para estudar ou pesquisar}
  \end{Phonetics}
\end{Entry}

\begin{Entry}{留学生}{10,8,5}{⽥、⼦、⽣}
  \begin{Phonetics}{留学生}{liu2 xue2 sheng1}[][HSK 2]
    \definition[个,位,名,批,帮]{s.}{estudante estrangeiro; estudante que retornou; estudante que estuda no exterior}
  \end{Phonetics}
\end{Entry}

\begin{Entry}{留念}{10,8}{⽥、⼼}
  \begin{Phonetics}{留念}{liu2nian4}[][HSK 7-9]
    \definition{v.}{aceitar como lembrança; guardar como lembrança (frequentemente usado como presente de despedida)}
  \end{Phonetics}
\end{Entry}

\begin{Entry}{留神}{10,9}{⽥、⽰}
  \begin{Phonetics}{留神}{liu2/shen2}[][HSK 7-9]
    \definition{v.+compl.}{ser cuidadoso; tomar cuidado; ficar atento; manter os olhos bem abertos; ficar de olho na situação; ficar atento às condições climáticas; estar atento a}
  \end{Phonetics}
\end{Entry}

\begin{Entry}{留恋}{10,10}{⽥、⼼}
  \begin{Phonetics}{留恋}{liu2lian4}[][HSK 7-9]
    \definition{v.}{relutante em partir; odiar ter que ir | indisposto a desistir | recordar com carinho}
  \end{Phonetics}
\end{Entry}

\begin{Entry}{留意}{10,13}{⽥、⼼}
  \begin{Phonetics}{留意}{liu2/yi4}[][HSK 7-9]
    \definition{v.+compl.}{ter cuidado; ficar atento; manter os olhos abertos; prestar atenção}
  \end{Phonetics}
\end{Entry}

%%%%%%%%%% 畜 %%%%%%%%%%
\subsection*{畜}

\begin{Entry}{畜}{10}{⽥}
  \begin{Phonetics}{畜}{chu4}
    \definition*{s.}{Sobrenome: Chu}
    \definition{s.}{animal doméstico; gado; bestas, principalmente referindo-se ao gado}
  \end{Phonetics}
  \begin{Phonetics}{畜}{xu4}
    \definition{v.}{criar (animais domésticos)}
  \end{Phonetics}
\end{Entry}

%%%%%%%%%% 疼 %%%%%%%%%%
\subsection*{疼}

\begin{Entry}{疼}{10}{⽧}
  \begin{Phonetics}{疼}{teng2}[][HSK 2]
    \definition{adj.}{dolorido; doído; sensação de extremo desconforto causada por ferimentos, doenças, etc.}
    \definition{v.}{ferir; machucar | adorar; amar profundamente; gostar muito; cuidar}
  \end{Phonetics}
\end{Entry}

\begin{Entry}{疼痛}{10,12}{⽧、⽧}
  \begin{Phonetics}{疼痛}{teng2 tong4}[][HSK 6]
    \definition[阵,种]{s.}{dor; sofrimento; ferimento; descreve a sensação de dor causada por lesão ou doença}
  \end{Phonetics}
\end{Entry}

%%%%%%%%%% 疾 %%%%%%%%%%
\subsection*{疾}

\begin{Entry}{疾}{10}{⽧}
  \begin{Phonetics}{疾}{ji2}
    \definition*{s.}{Sobrenome: Ji}
    \definition{s.}{doença; enfermidade; moléstia; padecimento | sofrimento; dor; dificuldade; mazela}
  \end{Phonetics}
\end{Entry}

\begin{Entry}{疾病}{10,10}{⽧、⽧}
  \begin{Phonetics}{疾病}{ji2bing4}[][HSK 6]
    \definition[种]{s.}{doença; enfermidade; termo geral para doença}
  \end{Phonetics}
\end{Entry}

%%%%%%%%%% 病 %%%%%%%%%%
\subsection*{病}

\begin{Entry}{病}{10}{⽧}
  \begin{Phonetics}{病}{bing4}[][HSK 1]
    \definition[种]{s.}{doença; enfermidade | doença; males | falha; defeito; desvantagem; erro}
    \definition{v.}{adoecer; ficar doente | ferir; causar danos a | angustiar; desaprovar}
  \end{Phonetics}
\end{Entry}

\begin{Entry}{病人}{10,2}{⽧、⼈}
  \begin{Phonetics}{病人}{bing4 ren2}[][HSK 1]
    \definition[个,位]{s.}{doente; paciente; pessoas doentes; pessoas em tratamento}
  \end{Phonetics}
\end{Entry}

\begin{Entry}{病床}{10,7}{⽧、⼴}
  \begin{Phonetics}{病床}{bing4chuang2}[][HSK 7-9]
    \definition[号,张]{s.}{cama de hospital | leito de doente}
  \end{Phonetics}
\end{Entry}

\begin{Entry}{病房}{10,8}{⽧、⼾}
  \begin{Phonetics}{病房}{bing4 fang2}[][HSK 6]
    \definition[个,间]{s.}{enfermaria de um hospital; quartos onde ficam os pacientes em hospitais e onde vivem em casas de repouso}
  \end{Phonetics}
\end{Entry}

\begin{Entry}{病毒}{10,9}{⽧、⽏}
  \begin{Phonetics}{病毒}{bing4du2}[][HSK 5]
    \definition[种,株,类]{s.}{vírus; patógenos que são menores que os germes e visíveis somente com um microscópio eletrônico | Computação: vírus de computador}
  \end{Phonetics}
\end{Entry}

\begin{Entry}{病症}{10,10}{⽧、⽧}
  \begin{Phonetics}{病症}{bing4zheng4}[][HSK 7-9]
    \definition[种]{s.}{doença; enfermidade}
  \end{Phonetics}
\end{Entry}

\begin{Entry}{病情}{10,11}{⽧、⼼}
  \begin{Phonetics}{病情}{bing4 qing2}[][HSK 6]
    \definition{s.}{estado de uma doença; condição do paciente; mudanças na doença}
  \end{Phonetics}
\end{Entry}

%%%%%%%%%% 症 %%%%%%%%%%
\subsection*{症}

\begin{Entry}{症}{10}{⽧}
  \begin{Phonetics}{症}{zheng1}
    \definition{s.}{doença; enfermidade | (figurativo) ponto de atrito | tumor abdominal | obstrução intestinal}
  \end{Phonetics}
  \begin{Phonetics}{症}{zheng4}
    \definition{s.}{doença; enfermidade}
  \end{Phonetics}
\end{Entry}

\begin{Entry}{症状}{10,7}{⽧、⽝}
  \begin{Phonetics}{症状}{zheng4zhuang4}[][HSK 6]
    \definition[种,些]{s.}{sintoma; estado anormal de um organismo devido a uma doença, como tosse, febre, etc.}
  \end{Phonetics}
\end{Entry}

%%%%%%%%%% 益 %%%%%%%%%%
\subsection*{益}

\begin{Entry}{益}{10}{⽫}
  \begin{Phonetics}{益}{yi4}
    \definition*{s.}{Sobrenome: Yi}
    \definition{adj.}{benéfico}
    \definition{adv.}{Literário: mais; cada vez mais}[空气污染问题日益严重。===O problema da poluição do ar está se tornando cada vez mais sério.]
    \definition{s.}{benefício; lucro; vantagem}
    \definition{v.}{aumentar}
  \end{Phonetics}
\end{Entry}

\begin{Entry}{益虫}{10,6}{⽫、⾍}
  \begin{Phonetics}{益虫}{yi4chong2}
    \definition{s.}{inseto benéfico (oposto a 害虫)}
  \seealsoref{害虫}{hai4chong2}
  \end{Phonetics}
\end{Entry}

%%%%%%%%%% 盏 %%%%%%%%%%
\subsection*{盏}

\begin{Entry}{盏}{10}{⽫}
  \begin{Phonetics}{盏}{zhan3}
    \definition{clas.}{usado para lâmpadas, iluminação}[一盏煤油灯。===Uma lamparina de querosene.]
    \definition{s.}{copo pequeno}
  \end{Phonetics}
\end{Entry}

%%%%%%%%%% 盐 %%%%%%%%%%
\subsection*{盐}

\begin{Entry}{盐}{10}{⽫}
  \begin{Phonetics}{盐}{yan2}[][HSK 4]
    \definition[袋,勺,把,包,粒]{s.}{sal (de cozinha) | Química: sal (produto formado pela neutralização de um ácido por uma base)}
  \end{Phonetics}
\end{Entry}

%%%%%%%%%% 监 %%%%%%%%%%
\subsection*{监}

\begin{Entry}{监}{10}{⽫}
  \begin{Phonetics}{监}{jian1}
    \definition{s.}{prisão; cadeia}
    \definition{v.}{supervisionar; inspecionar; observar}
  \end{Phonetics}
\end{Entry}

\begin{Entry}{监护}{10,7}{⽫、⼿}
  \begin{Phonetics}{监护}{jian1hu4}[][HSK 7-9]
    \definition{s.}{tutela}
    \definition{v.}{Lei: desempenhar as funções de tutor; agir como guardião; tutelar | Medicina: cuidar e zelar; vigiar}
  \end{Phonetics}
\end{Entry}

\begin{Entry}{监视}{10,8}{⽫、⾒}
  \begin{Phonetics}{监视}{jian1shi4}[][HSK 7-9]
    \definition{v.}{manter vigilância; ficar de olho em; observar atentamente}
  \end{Phonetics}
\end{Entry}

\begin{Entry}{监测}{10,9}{⽫、⽔}
  \begin{Phonetics}{监测}{jian1 ce4}[][HSK 6]
    \definition{v.}{monitorar; supervisionar e testar}
  \end{Phonetics}
\end{Entry}

\begin{Entry}{监狱}{10,9}{⽫、⽝}
  \begin{Phonetics}{监狱}{jian1yu4}[][HSK 7-9]
    \definition[个,所,座]{s.}{cadeia; prisão; instituições estatais responsáveis ​​pela aplicação de penas criminais; locais onde os presos são mantidos}
  \end{Phonetics}
\end{Entry}

\begin{Entry}{监控}{10,11}{⽫、⼿}
  \begin{Phonetics}{监控}{jian1kong4}[][HSK 7-9]
    \definition{s.}{monitor}
    \definition{v.}{monitorar}
  \end{Phonetics}
\end{Entry}

\begin{Entry}{监督}{10,13}{⽫、⽬}
  \begin{Phonetics}{监督}{jian1du1}[][HSK 6]
    \definition[个,位,名]{s.}{monitoramento; supervisão; pessoas que supervisionam}
    \definition{v.}{controlar; supervisionar; superintender; monitorar e supervisionar de perto}
  \end{Phonetics}
\end{Entry}

\begin{Entry}{监察}{10,14}{⽫、⼧}
  \begin{Phonetics}{监察}{jian1cha2}[][HSK 7-9]
    \definition{v.}{supervisionar; controlar}
  \end{Phonetics}
\end{Entry}

\begin{Entry}{监管}{10,14}{⽫、⽵}
  \begin{Phonetics}{监管}{jian1guan3}[][HSK 7-9]
    \definition{v.}{monitorar; supervisionar}
  \end{Phonetics}
\end{Entry}

%%%%%%%%%% 眞 %%%%%%%%%%
\subsection*{眞}

\begin{Entry}{眞}{10}{⽬}
  \begin{Phonetics}{眞}{zhen1}
    \variantof{真}
  \end{Phonetics}
\end{Entry}

%%%%%%%%%% 真 %%%%%%%%%%
\subsection*{真}

\begin{Entry}{真}{10}{⼗}
  \begin{Phonetics}{真}{zhen1}[][HSK 1]
    \definition*{s.}{Sobrenome: Zhen}
    \definition{adj.}{verdadeiro; real; genuíno (oposto de 假, 伪) | claro; inequívoco | genuíno; conforme os fatos objetivos (em oposição a 假 e 伪) | sincero}
    \definition{adv.}{realmente; verdadeiramente; de fato}
    \definition{s.}{escrita regular | retrato; imagem; cópia exata de algo | instintos naturais (ou caráter, disposição); natureza; qualidade inerente; origem | estado original; refere-se à forma original das coisas}
  \seealsoref{假}{jia4}
  \seealsoref{伪}{wei3}
  \end{Phonetics}
\end{Entry}

\begin{Entry}{真切}{10,4}{⼗、⼑}
  \begin{Phonetics}{真切}{zhen1qie4}
    \definition{adj.}{claro | distinto | honesto | sincero | vívido}
  \end{Phonetics}
\end{Entry}

\begin{Entry}{真心}{10,4}{⼗、⼼}
  \begin{Phonetics}{真心}{zhen1xin1}
    \definition{adj.}{sincero}
    \definition[片]{s.}{sinceridade}
  \end{Phonetics}
\end{Entry}

\begin{Entry}{真牛}{10,4}{⼗、⽜}
  \begin{Phonetics}{真牛}{zhen1niu2}
    \definition{adj.}{(gíria) muito legal, incrível}
  \end{Phonetics}
\end{Entry}

\begin{Entry}{真正}{10,5}{⼗、⽌}
  \begin{Phonetics}{真正}{zhen1zheng4}[][HSK 2]
    \definition{adj.}{verdadeiro; real; genuíno}
    \definition{adv.}{realmente; de fato; expressa afirmação de uma ação ou situação, equivalente a 确实}
  \seealsoref{确实}{que4shi2}
  \end{Phonetics}
\end{Entry}

\begin{Entry}{真声}{10,7}{⼗、⼠}
  \begin{Phonetics}{真声}{zhen1sheng1}
    \definition{s.}{voz modal; voz natural; voz verdadeira (oposto a 假声)}
  \seealsoref{假声}{jia3sheng1}
  \end{Phonetics}
\end{Entry}

\begin{Entry}{真实}{10,8}{⼗、⼧}
  \begin{Phonetics}{真实}{zhen1shi2}[][HSK 3]
    \definition{adj.}{verdadeiro; real; autêntico; de acordo com fatos objetivos}
  \end{Phonetics}
\end{Entry}

\begin{Entry}{真的}{10,8}{⼗、⽩}
  \begin{Phonetics}{真的}{zhen1 de5}[][HSK 1]
    \definition{adv.}{realmente; salientar que a situação existe realmente | verdadeiramente; realmente; existente na realidade; consistente com os fatos objetivos}
  \end{Phonetics}
\end{Entry}

\begin{Entry}{真诚}{10,8}{⼗、⾔}
  \begin{Phonetics}{真诚}{zhen1 cheng2}[][HSK 5]
    \definition{adj.}{verdadeiro; honesto; sério; sincero; genuíno; descreve uma pessoa que fala e age com sinceridade, de coração, fazendo com que os outros acreditem nela}
  \end{Phonetics}
\end{Entry}

\begin{Entry}{真相}{10,9}{⼗、⽬}
  \begin{Phonetics}{真相}{zhen1xiang4}[][HSK 5]
    \definition[个]{s.}{face; verdade; verdade nua e crua; a situação real; o estado real das coisas; a verdadeira situação}
  \end{Phonetics}
\end{Entry}

\begin{Entry}{真珠}{10,10}{⼗、⽟}
  \begin{Phonetics}{真珠}{zhen1zhu1}
    \variantof{珍珠}
  \end{Phonetics}
\end{Entry}

\begin{Entry}{真真}{10,10}{⼗、⼗}
  \begin{Phonetics}{真真}{zhen1zhen1}
    \definition{adv.}{genuinamente | realmente | escrupulosamente}
  \end{Phonetics}
\end{Entry}

\begin{Entry}{真理}{10,11}{⼗、⽟}
  \begin{Phonetics}{真理}{zhen1li3}[][HSK 5]
    \definition[条,个]{s.}{verdade; o reflexo correto das coisas objetivas e suas leis no cérebro humano}
  \end{Phonetics}
\end{Entry}

\begin{Entry}{真释}{10,12}{⼗、⾤}
  \begin{Phonetics}{真释}{zhen1shi4}
    \definition{s.}{razão genuína | explicação verdadeira}
  \end{Phonetics}
\end{Entry}

%%%%%%%%%% 破 %%%%%%%%%%
\subsection*{破}

\begin{Entry}{破}{10}{⽯}
  \begin{Phonetics}{破}{po4}[][HSK 3]
    \definition{adj.}{quebrado; danificado; rasgado; desgastado | insignificante; péssimo; medíocre}
    \definition{v.}{quebrar; danificar | dividir; cortar; separar | trocar (dinheiro) | livrar-se de; destruir; romper com | derrotar; capturar (uma cidade, etc.) | gastar dinheiro | revelar a verdade sobre; expor | mudar; romper; quebrar (regras, hábitos, ideias, etc.)}
  \end{Phonetics}
\end{Entry}

\begin{Entry}{破产}{10,6}{⽯、⼇}
  \begin{Phonetics}{破产}{po4/chan3}[][HSK 4]
    \definition{v.+compl.}{falir; ir à falência; tornar-se insolvente; entrar em liquidação; perder todo o patrimônio | falhar; fracassar; não dar em nada; figura de linguagem (geralmente com uma conotação depreciativa)}
  \end{Phonetics}
\end{Entry}

\begin{Entry}{破坏}{10,7}{⽯、⼟}
  \begin{Phonetics}{破坏}{po4huai4}[][HSK 3]
    \definition{v.}{demolir; naufragar; soçobrar; destruir; obliterar | quebrar; violar (um acordo, regulamento, etc.); não cumprir (disposições legais, regras, acordos, princípios, etc.) | prejudicar; perturbar; sabotar; causar grande dano; causar danos às coisas | reverter; mudar (um sistema social, costume, etc.) completamente ou violentamente | destruir; decompor; danificar o tecido ou a estrutura de um objeto}
  \end{Phonetics}
\end{Entry}

\begin{Entry}{破坏性}{10,7,8}{⽯、⼟、⼼}
  \begin{Phonetics}{破坏性}{po4huai4xing4}
    \definition{adj.}{destrutivo}
    \definition{s.}{poder destrutivo}
  \end{Phonetics}
\end{Entry}

%%%%%%%%%% 砸 %%%%%%%%%%
\subsection*{砸}

\begin{Entry}{砸}{10}{⽯}
  \begin{Phonetics}{砸}{za2}
    \definition{v.}{esmagar | bater | falhar | estragar}
  \end{Phonetics}
\end{Entry}

%%%%%%%%%% 离 %%%%%%%%%%
\subsection*{离}

\begin{Entry}{离}{10}{⼇}
  \begin{Phonetics}{离}{li2}[][HSK 2]
    \definition*{s.}{Um dos Oito Diagramas | Sobrenome: Li}
    \definition{prep.}{(ser longe) de\dots até\dots}
    \definition{v.}{partir; separar-se; afastar-se; estar longe de | prescindir; dispensar; ser independente de | mudar de; desviar-se de | mudar de; desviar-se de; trair; ser incompatível}
  \end{Phonetics}
\end{Entry}

\begin{Entry}{离不开}{10,4,4}{⼇、⼀、⼶}
  \begin{Phonetics}{离不开}{li2 bu4 kai1}[][HSK 4]
    \definition{v.}{não pode prescindir; ser inseparável de; não ser capaz de se separar ou deixar uma pessoa, coisa ou circunstância}
  \end{Phonetics}
\end{Entry}

\begin{Entry}{离开}{10,4}{⼇、⼶}
  \begin{Phonetics}{离开}{li2kai1}[][HSK 2]
    \definition{v.}{deixar; partir; desviar-se; separar-se das pessoas, dos lugares e das coisas}
  \end{Phonetics}
\end{Entry}

\begin{Entry}{离奇}{10,8}{⼇、⼤}
  \begin{Phonetics}{离奇}{li2qi2}[][HSK 7-9]
    \definition{adj.}{estranho; esquisito; fantástico; bizarro}
  \end{Phonetics}
\end{Entry}

\begin{Entry}{离婚}{10,11}{⼇、⼥}
  \begin{Phonetics}{离婚}{li2/hun1}[][HSK 3]
    \definition{v.+compl.}{divórciar; romper um casamento; obter o divórcio}
  \end{Phonetics}
\end{Entry}

\begin{Entry}{离职}{10,11}{⼇、⽿}
  \begin{Phonetics}{离职}{li2/zhi2}[][HSK 7-9]
    \definition{v.+compl.}{deixar o emprego temporariamente | demitir-se; deixar o cargo; abandonar o emprego}
  \end{Phonetics}
\end{Entry}

\begin{Entry}{离谱儿}{10,14,2}{⼇、⾔、⼉}
  \begin{Phonetics}{离谱儿}{li2/pu3r5}[][HSK 7-9]
    \definition{v.+compl.}{ir além do que é apropriado; estar fora de lugar; exagerar; estar muito longe do que é normal; falar ou agir de uma forma que não esteja em conformidade com os padrões geralmente aceitos}
  \end{Phonetics}
\end{Entry}

%%%%%%%%%% 秘 %%%%%%%%%%
\subsection*{秘}

\begin{Entry}{秘}{10}{⽲}
  \begin{Phonetics}{秘}{bi4}
    \definition*{s.}{Abreviação de Peru, 秘鲁 | Sobrenome: Bi}
  \seealsoref{秘鲁}{bi4lu3}
  \end{Phonetics}
  \begin{Phonetics}{秘}{mi4}
    \definition{adj.}{secreto; misterioso | raro; raramente visto; estranho}
    \definition{adv.}{secretamente; privadamente}
    \definition{s.}{secretário}
    \definition{v.}{manter algo em segredo; esconder algo; guardar segredos | bloquear; obstruir; ter dificuldade para defecar}
  \end{Phonetics}
\end{Entry}

\begin{Entry}{秘书}{10,4}{⽲、⼄}
  \begin{Phonetics}{秘书}{mi4shu1}[][HSK 4]
    \definition[个,位,名]{s.}{o cargo de secretário; funções de secretariado | secretário; pessoas encarregadas da correspondência e que auxiliam o chefe do órgão ou departamento na condução diária de seu trabalho}
  \end{Phonetics}
\end{Entry}

\begin{Entry}{秘书长}{10,4,4}{⽲、⼄、⾧}
  \begin{Phonetics}{秘书长}{mi4 shu1 zhang3}[][HSK 6]
    \definition{s.}{secretário-geral}
  \end{Phonetics}
\end{Entry}

\begin{Entry}{秘密}{10,11}{⽲、⼧}
  \begin{Phonetics}{秘密}{mi4mi4}[][HSK 4]
    \definition{adj.}{secreto}
    \definition[个,条,些]{s.}{segredo; algo secreto; coisas que você não quer que as pessoas saibam}
  \end{Phonetics}
\end{Entry}

\begin{Entry}{秘鲁}{10,12}{⽲、⿂}
  \begin{Phonetics}{秘鲁}{bi4lu3}
    \definition*{s.}{Peru}
  \end{Phonetics}
\end{Entry}

%%%%%%%%%% 租 %%%%%%%%%%
\subsection*{租}

\begin{Entry}{租}{10}{⽲}
  \begin{Phonetics}{租}{zu1}[][HSK 2]
    \definition{s.}{aluguel | imposto sobre a terra; tributação; (antigo) refere-se ao imposto predial}
    \definition{v.}{contratar; alugar; fretar | alugar; arrendar}
  \end{Phonetics}
\end{Entry}

\begin{Entry}{租用}{10,5}{⽲、⽤}
  \begin{Phonetics}{租用}{zu1yong4}
    \definition{v.}{contratar | alugar | alugar (algo de alguém)}
  \end{Phonetics}
\end{Entry}

\begin{Entry}{租让}{10,5}{⽲、⾔}
  \begin{Phonetics}{租让}{zu1rang4}
    \definition{v.}{alugar | alugar (a propriedade de alguém para outra pessoa)}
  \end{Phonetics}
\end{Entry}

\begin{Entry}{租约}{10,6}{⽲、⽷}
  \begin{Phonetics}{租约}{zu1yue1}
    \definition{s.}{aluguel}
  \end{Phonetics}
\end{Entry}

\begin{Entry}{租房}{10,8}{⽲、⼾}
  \begin{Phonetics}{租房}{zu1fang2}
    \definition{v.}{alugar um apartamento}
  \end{Phonetics}
\end{Entry}

\begin{Entry}{租金}{10,8}{⽲、⾦}
  \begin{Phonetics}{租金}{zu1 jin1}[][HSK 6]
    \definition[笔]{s.}{aluguel; aluguer; o custo do aluguel de terras, casas ou itens; a renda do aluguel de terras, casas ou itens}
  \seealsoref{租钱}{zu1qian5}
  \end{Phonetics}
\end{Entry}

\begin{Entry}{租赁}{10,10}{⽲、⾙}
  \begin{Phonetics}{租赁}{zu1lin4}
    \definition{v.}{contratar | alugar}
  \end{Phonetics}
\end{Entry}

\begin{Entry}{租钱}{10,10}{⽲、⾦}
  \begin{Phonetics}{租钱}{zu1qian5}
    \definition{s.}{aluguel}
  \seealsoref{租金}{zu1 jin1}
  \end{Phonetics}
\end{Entry}

\begin{Entry}{租船}{10,11}{⽲、⾈}
  \begin{Phonetics}{租船}{zu1chuan2}
    \definition{v.}{fretar um navio | alugar um navio}
  \end{Phonetics}
\end{Entry}

%%%%%%%%%% 秤 %%%%%%%%%%
\subsection*{秤}

\begin{Entry}{秤}{10}{⽲}
  \begin{Phonetics}{秤}{cheng4}[][HSK 7-9]
    \definition[把,杆,台]{s.}{balança; balança romana; um instrumento para medir o peso de um objeto}
  \end{Phonetics}
\end{Entry}

%%%%%%%%%% 积 %%%%%%%%%%
\subsection*{积}

\begin{Entry}{积}{10}{⽲}
  \begin{Phonetics}{积}{ji1}[][HSK 7-9]
    \definition{adj.}{de longa data; pendente há muito tempo | antiquíssimo; acumulado ao longo de um longo período de tempo}
    \definition{s.}{(medicina chinesa) indigestão (em bebês e crianças) | (matemática)  abreviação de produto, 乘积}
    \definition{v.}{acumular; juntar; amontoar; reunir; coletar}
  \seealsoref{乘积}{cheng2ji1}
  \end{Phonetics}
\end{Entry}

\begin{Entry}{积木}{10,4}{⽲、⽊}
  \begin{Phonetics}{积木}{ji1mu4}
    \definition{s.}{blocos de montar (brinquedo)}
  \end{Phonetics}
\end{Entry}

\begin{Entry}{积极}{10,7}{⽲、⽊}
  \begin{Phonetics}{积极}{ji1ji2}[][HSK 3]
    \definition{adj.}{ativo; descreve uma atitude proativa e esforçada | positivo; que tem um efeito positivo e ajuda no desenvolvimento das coisas}
  \end{Phonetics}
\end{Entry}

\begin{Entry}{积淀}{10,11}{⽲、⽔}
  \begin{Phonetics}{积淀}{ji1dian4}[][HSK 7-9]
    \definition{s.}{acumulação | depósitos acumulados ao longo de longos períodos | Figurativo: experiência valiosa, sabedoria acumulada}
    \definition{v.}{acumular}
  \end{Phonetics}
\end{Entry}

\begin{Entry}{积累}{10,11}{⽲、⽷}
  \begin{Phonetics}{积累}{ji1lei3}[][HSK 4]
    \definition{s.}{acúmulo; acumulação}
    \definition{v.}{acumular}
  \end{Phonetics}
\end{Entry}

\begin{Entry}{积蓄}{10,13}{⽲、⾋}
  \begin{Phonetics}{积蓄}{ji1xu4}[][HSK 7-9]
    \definition[笔]{s.}{poupança; economias}
    \definition{v.}{acumular; poupar; economizar}
  \end{Phonetics}
\end{Entry}

%%%%%%%%%% 称 %%%%%%%%%%
\subsection*{称}

\begin{Entry}{称}{10}{⽲}
  \begin{Phonetics}{称}{chen4}
    \definition{adj.}{ajustado; encaixado; adequado}
    \definition{v.}{ajustar; adequar; combinar; estar em conformidade com; ser adequado para | ter; possuir}
  \end{Phonetics}
  \begin{Phonetics}{称}{cheng1}[][HSK 2,5]
    \definition*{s.}{Sobrenome: Cheng}
    \definition{s.}{nome}
    \definition{v.}{chamar; ser chamado | dizer; declarar | elogiar; louvar; expressar afirmação ou elogio a pessoas ou coisas por meio de palavras | pesar; medir o peso | elevar; levantar; erguer | aplaudir; concordar; expressar suas opiniões ou sentimentos por meio de palavras ou ações | declarar-se como; declarar que é; reivindicar ser alguém em virtude do próprio poder}
  \end{Phonetics}
\end{Entry}

\begin{Entry}{称为}{10,4}{⽲、⼂}
  \begin{Phonetics}{称为}{cheng1 wei2}[][HSK 3]
    \definition{v.}{ser chamado de; ser conhecido como; denominar}
  \end{Phonetics}
\end{Entry}

\begin{Entry}{称号}{10,5}{⽲、⼝}
  \begin{Phonetics}{称号}{cheng1hao4}[][HSK 5]
    \definition{s.}{título; nome; designação; nome dado a alguém, a uma organização ou a alguma coisa (geralmente usado de forma honrosa)}
  \end{Phonetics}
\end{Entry}

\begin{Entry}{称作}{10,7}{⽲、⼈}
  \begin{Phonetics}{称作}{cheng1zuo4}[][HSK 7-9]
    \definition{v.}{ser chamado | ser conhecido como}
  \end{Phonetics}
\end{Entry}

\begin{Entry}{称呼}{10,8}{⽲、⼝}
  \begin{Phonetics}{称呼}{cheng1hu5}[][HSK 7-9]
    \definition{v.}{chamar; dirigir-se (a alguém)}[我称呼他为老师。===Eu o chamo de professor.]
  \end{Phonetics}
\end{Entry}

\begin{Entry}{称赞}{10,16}{⽲、⾙}
  \begin{Phonetics}{称赞}{cheng1zan4}[][HSK 4]
    \definition[句,声,番,次]{s.}{elogio; aclamação; louvor; avaliação positiva de um desempenho ou conquista}
    \definition{v.}{elogiar; aclamar; louvar; usar palavras para expressar um carinho pelas virtudes de uma pessoa ou coisa}
  \end{Phonetics}
\end{Entry}

%%%%%%%%%% 窄 %%%%%%%%%%
\subsection*{窄}

\begin{Entry}{窄}{10}{⽳}
  \begin{Phonetics}{窄}{zhai3}
    \definition{adj.}{estreito; pequena distância horizontal | mesquinho; estreito; (mente) não alegre; (capacidade) pequena | difícil; mal; falta de; (vida) não bem de vida}
  \end{Phonetics}
\end{Entry}

%%%%%%%%%% 站 %%%%%%%%%%
\subsection*{站}

\begin{Entry}{站}{10}{⽴}
  \begin{Phonetics}{站}{zhan4}[][HSK 1,2]
    \definition*{s.}{Sobrenome: Zhan}
    \definition{s.}{parada; estação; ponto de parada | central; estação; instituição criada para um determinado tipo de atividade | filial de uma empresa ou organização; local de trabalho criado para realizar uma determinada tarefa | \emph{website}; na rede de computadores, refere-se a um \emph{site}}
    \definition{v.}{ficar em pé; estar em pé | parar; interromper; fazer uma pausa}
  \end{Phonetics}
\end{Entry}

\begin{Entry}{站长}{10,4}{⽴、⾧}
  \begin{Phonetics}{站长}{zhan4zhang3}
    \definition{s.}{pessoa responsável pela estação de trem | chefe da estação | \emph{webmaster} | gerente de centro de voluntariado}
  \end{Phonetics}
\end{Entry}

\begin{Entry}{站台}{10,5}{⽴、⼝}
  \begin{Phonetics}{站台}{zhan4 tai2}[][HSK 6]
    \definition{s.}{plataforma (em uma estação ferroviária)}
  \end{Phonetics}
\end{Entry}

\begin{Entry}{站住}{10,7}{⽴、⼈}
  \begin{Phonetics}{站住}{zhan4 zhu4}[][HSK 2]
    \definition{v.}{parar; deter; parar enquanto se move | ficar firme nos pés; manter os pés; permanecer firme | manter-se firme; consolidar a posição de alguém; estabelecer-se em uma determinada unidade ou lugar | sustentar a opinião}
  \end{Phonetics}
\end{Entry}

\begin{Entry}{站姿}{10,9}{⽴、⼥}
  \begin{Phonetics}{站姿}{zhan4zi1}
    \definition{s.}{postura}
  \end{Phonetics}
\end{Entry}

\begin{Entry}{站点}{10,9}{⽴、⽕}
  \begin{Phonetics}{站点}{zhan4dian3}
    \definition{s.}{\emph{website}}
  \end{Phonetics}
\end{Entry}

%%%%%%%%%% 竞 %%%%%%%%%%
\subsection*{竞}

\begin{Entry}{竞}{10}{⽴}
  \begin{Phonetics}{竞}{jing4}
    \definition{adj.}{forte; poderoso}
    \definition{v.}{competir; contender; disputar | contestar}
  \end{Phonetics}
\end{Entry}

\begin{Entry}{竞争}{10,6}{⽴、⼑}
  \begin{Phonetics}{竞争}{jing4zheng1}[][HSK 5]
    \definition{v.}{competir; disputar; lutar; entre duas ou mais partes; em prol de seus próprios interesses; lutar pela vitória por meio de uma disputa de sua própria força contra outra}
  \end{Phonetics}
\end{Entry}

\begin{Entry}{竞技}{10,7}{⽴、⼿}
  \begin{Phonetics}{竞技}{jing4ji4}[][HSK 7-9]
    \definition{s.}{atletismo; provas de atletismo; esportes; pista e campo}
    \definition{v.}{competir; desafiar; geralmente referindo-se a competições atléticas}
  \end{Phonetics}
\end{Entry}

\begin{Entry}{竞相}{10,9}{⽴、⽬}
  \begin{Phonetics}{竞相}{jing4xiang1}[][HSK 7-9]
    \definition{adv.}{ansiosamente}
    \definition{s.}{competição}
    \definition{v.}{competir; disputar}
  \end{Phonetics}
\end{Entry}

\begin{Entry}{竞选}{10,9}{⽴、⾡}
  \begin{Phonetics}{竞选}{jing4xuan3}[][HSK 7-9]
    \definition{s.}{eleição; campanha eleitoral}
    \definition{v.}{participar de uma disputa eleitoral; fazer campanha para (um cargo); candidatar-se a}
  \end{Phonetics}
\end{Entry}

\begin{Entry}{竞赛}{10,14}{⽴、⾙}
  \begin{Phonetics}{竞赛}{jing4sai4}[][HSK 5]
    \definition[个]{s.}{concurso; competição; partida; corrida}
    \definition{v.}{correr; competir; competir uns com os outros por superioridade; em esportes, produção e outras atividades, para comparar competência, habilidade etc., usado principalmente na linguagem falada}
  \end{Phonetics}
\end{Entry}

%%%%%%%%%% 笋 %%%%%%%%%%
\subsection*{笋}

\begin{Entry}{笋}{10}{⽵}
  \begin{Phonetics}{笋}{sun3}
    \definition{s.}{broto de bambu}
  \end{Phonetics}
\end{Entry}

%%%%%%%%%% 笑 %%%%%%%%%%
\subsection*{笑}

\begin{Entry}{笑}{10}{⽵}
  \begin{Phonetics}{笑}{xiao4}[][HSK 1]
    \definition{adj.}{ridículo; engraçado; risível; hilário}
    \definition{v.}{sorrir; rir; mostrar expressão de alegria; emitir sons de alegria | ridicularizar; rir de; zombar}
  \end{Phonetics}
\end{Entry}

\begin{Entry}{笑声}{10,7}{⽵、⼠}
  \begin{Phonetics}{笑声}{xiao4 sheng1}[][HSK 6]
    \definition{s.}{riso; risada}
  \end{Phonetics}
\end{Entry}

\begin{Entry}{笑话}{10,8}{⽵、⾔}
  \begin{Phonetics}{笑话}{xiao4hua5}[][HSK 2]
    \definition[个]{s.}{piada; brincadeira; uma conversa ou história que faz as pessoas rirem; algo que as pessoas usam como piada}
    \definition{v.}{ridicularizar; zombar; rir de;}
  \end{Phonetics}
\end{Entry}

\begin{Entry}{笑话儿}{10,8,2}{⽵、⾔、⼉}
  \begin{Phonetics}{笑话儿}{xiao4 hua4r5}[][HSK 2]
    \definition{s.}{piada; brincadeira; gracejo}
  \end{Phonetics}
\end{Entry}

\begin{Entry}{笑容}{10,10}{⽵、⼧}
  \begin{Phonetics}{笑容}{xiao4 rong2}[][HSK 6]
    \definition[丝,抹,个]{s.}{sorriso; expressão sorridente; o olhar no rosto de alguém ao sorrir}
  \end{Phonetics}
\end{Entry}

\begin{Entry}{笑脸}{10,11}{⽵、⾁}
  \begin{Phonetics}{笑脸}{xiao4 lian3}[][HSK 6]
    \definition{s.}{\emph{smiley}; rosto sorridente (emoji)}
  \end{Phonetics}
\end{Entry}

%%%%%%%%%% 笔 %%%%%%%%%%
\subsection*{笔}

\begin{Entry}{笔}{10}{⽵}
  \begin{Phonetics}{笔}{bi3}[][HSK 2]
    \definition{clas.}{usado para grandes quantias de dinheiro, compras, negócios, propriedades, etc. | usado em caligrafia e pintura, etc.}
    \definition[支,枝]{s.}{caneta; lápis; pincel para escrever; ferramentas para escrever ou desenhar | técnica de escrita; caligrafia ou desenho | traço}
    \definition{v.}{escrever à mão}
  \end{Phonetics}
\end{Entry}

\begin{Entry}{笔记}{10,5}{⽵、⾔}
  \begin{Phonetics}{笔记}{bi3 ji4}[][HSK 2]
    \definition[篇,本,个]{s.}{notas; anotações feitas durante aulas, palestras e leituras | ensaios; esboços}
    \definition{v.}{tomar nota (por escrito)}
  \end{Phonetics}
\end{Entry}

\begin{Entry}{笔记本}{10,5,5}{⽵、⾔、⽊}
  \begin{Phonetics}{笔记本}{bi3ji4ben3}[][HSK 2]
    \definition[个,本]{s.}{caderno para anotações | \emph{laptop}; refere-se a um computador portátil}
    \definition{s.}{\emph{laptop}}
  \end{Phonetics}
\end{Entry}

\begin{Entry}{笔试}{10,8}{⽵、⾔}
  \begin{Phonetics}{笔试}{bi3 shi4}[][HSK 6]
    \definition{s.}{exame escrito; um tipo de exame que exige respostas escritas; diferente de 口试}
  \seealsoref{口试}{kou3 shi4}
  \end{Phonetics}
\end{Entry}

%%%%%%%%%% 粉 %%%%%%%%%%
\subsection*{粉}

\begin{Entry}{粉}{10}{⽶}
  \begin{Phonetics}{粉}{fen3}[][HSK 7-9]
    \definition{adj.}{branco | rosa}
    \definition{s.}{pó | cosméticos em pó | farinha de trigo | macarrão ou outro alimento feito de feijão, arroz, batata, amido de batata-doce, etc. | macarrão de arroz}
    \definition{v.}{virar pó | Dialeto: caiar}
  \end{Phonetics}
\end{Entry}

\begin{Entry}{粉丝}{10,5}{⽶、⼀}
  \begin{Phonetics}{粉丝}{fen3si1}[][HSK 7-9]
    \definition{s.}{(empréstimo linguístico) fã | entusiasta de alguém ou alguma coisa}
    \definition[个,群,位,名,些,批]{s.}{aletria de amido de feijão ou batata; aletria chinesa; macarrão de celofane ou macarrão de vidro (transparente) | Empréstimo linguístico: fã; refere-se a uma pessoa que é obcecada ou adora uma celebridade}
  \end{Phonetics}
\end{Entry}

\begin{Entry}{粉色}{10,6}{⽶、⾊}
  \begin{Phonetics}{粉色}{fen3 se4}
    \definition{s.}{cor-de-rosa}
  \end{Phonetics}
\end{Entry}

\begin{Entry}{粉碎}{10,13}{⽶、⽯}
  \begin{Phonetics}{粉碎}{fen3sui4}[][HSK 7-9]
    \definition{adj.}{pulverizado; quebrado em pedaços; descreve algo que está muito quebrado, quebrado em partículas muito pequenas}
    \definition{v.}{esmagar; transformar as coisas em partículas muito pequenas | esmagar; quebrar; estilhaçar; fazer com que a outra parte falhe ou seja completamente destruída}
  \end{Phonetics}
\end{Entry}

%%%%%%%%%% 素 %%%%%%%%%%
\subsection*{素}

\begin{Entry}{素}{10}{⽷}
  \begin{Phonetics}{素}{su4}
    \definition{adj.}{branco; de cor natural | simples; natural; singelo; de cor simples | nativo; original | normal; usual; geral}
    \definition{adv.}{geralmente; sempre; habitualmente}
    \definition{s.}{vegetais, frutas e outros alimentos (em oposição à 荤) | matéria-prima; matéria-prima básico; tecidos de seda naturais e não processados | elemento; os componentes básicos de algo}
  \seealsoref{荤}{hun1}
  \end{Phonetics}
\end{Entry}

\begin{Entry}{素质}{10,8}{⽷、⾙}
  \begin{Phonetics}{素质}{su4zhi4}[][HSK 6]
    \definition[个,种]{s.}{qualidade; características; caráter; o nível físico, moral, mental, intelectual e cultural de uma pessoa}
  \end{Phonetics}
\end{Entry}

%%%%%%%%%% 索 %%%%%%%%%%
\subsection*{索}

\begin{Entry}{索}{10}{⽷}
  \begin{Phonetics}{索}{suo3}
    \definition*{s.}{Sobrenome: Suo}
    \definition{adj.}{completamente sozinho; sozinho | maçante; insípido; sem significado}
    \definition[根]{s.}{corda; cabo; cordão; corrente | uma corda grande}
    \definition{v.}{(literário) pesquisar | exigir; pedir}
  \end{Phonetics}
\end{Entry}

\begin{Entry}{索性}{10,8}{⽷、⼼}
  \begin{Phonetics}{索性}{suo3xing4}
    \definition{adv.}{poderia muito bem | simplesmente | apenas}
  \end{Phonetics}
\end{Entry}

%%%%%%%%%% 紧 %%%%%%%%%%
\subsection*{紧}

\begin{Entry}{紧}{10}{⽷}
  \begin{Phonetics}{紧}{jin3}[][HSK 3]
    \definition{adj.}{tenso; apertado; o estado em que um objeto se encontra após ser submetido a uma grande força de tração ou pressão.| seguro; firme | cerrado; apertado | urgente; premente; tenso | rigoroso; rígido; severo | difícil; sem dinheiro}
    \definition{v.}{apertar; tornar mais apertado}
  \end{Phonetics}
\end{Entry}

\begin{Entry}{紧张}{10,7}{⽷、⼸}
  \begin{Phonetics}{紧张}{jin3zhang1}[][HSK 3]
    \definition{adj.}{nervoso; tenso; mentalmente em estado de alerta, excitado e inquieto | apertado; em falta; o que está disponível não satisfaz os requisitos| tenso; intenso; intenso ou urgente, causando tensão mental}
  \end{Phonetics}
\end{Entry}

\begin{Entry}{紧迫}{10,8}{⽷、⾡}
  \begin{Phonetics}{紧迫}{jin3po4}[][HSK 7-9]
    \definition{adj.}{urgente; premente; iminente; sem margem para manobras}
  \end{Phonetics}
\end{Entry}

\begin{Entry}{紧急}{10,9}{⽷、⼼}
  \begin{Phonetics}{紧急}{jin3ji2}[][HSK 3]
    \definition{adj./adj.}{urgente; premente; crítico}
  \end{Phonetics}
\end{Entry}

\begin{Entry}{紧紧}{10,10}{⽷、⽷}
  \begin{Phonetics}{紧紧}{jin3 jin3}[][HSK 5]
    \definition{adv.}{firmemente; estreitamente; apertadamente; prestar muita atenção (em algo)}
  \end{Phonetics}
\end{Entry}

\begin{Entry}{紧缺}{10,10}{⽷、⽸}
  \begin{Phonetics}{紧缺}{jin3que1}[][HSK 7-9]
    \definition{adj.}{em falta; extremamente necessário | escasso}
  \end{Phonetics}
\end{Entry}

\begin{Entry}{紧凑}{10,11}{⽷、⼎}
  \begin{Phonetics}{紧凑}{jin3cou4}[][HSK 7-9]
    \definition{adj.}{compacto; conciso; bem estruturado; rígido; sucinto}
  \end{Phonetics}
\end{Entry}

\begin{Entry}{紧密}{10,11}{⽷、⼧}
  \begin{Phonetics}{紧密}{jin3 mi4}[][HSK 4]
    \definition{adj.}{próximos; inseparáveis | incessante; rápido e intenso}
  \end{Phonetics}
\end{Entry}

\begin{Entry}{紧接着}{10,11,11}{⽷、⼿、⽬}
  \begin{Phonetics}{紧接着}{jin3 jie1zhe5}[][HSK 7-9]
    \definition{expr.}{imediatamente depois; uma coisa aconteceu após a outra}
  \end{Phonetics}
\end{Entry}

\begin{Entry}{紧缩}{10,14}{⽷、⽷}
  \begin{Phonetics}{紧缩}{jin3suo1}[][HSK 7-9]
    \definition{v.}{reduzir; cortar; desmantelar; encolher}
  \end{Phonetics}
\end{Entry}

%%%%%%%%%% 绣 %%%%%%%%%%
\subsection*{绣}

\begin{Entry}{绣}{10}{⽷}
  \begin{Phonetics}{绣}{xiu4}
    \definition{s.}{bordado}
    \definition{v.}{bordar}
  \end{Phonetics}
\end{Entry}

%%%%%%%%%% 继 %%%%%%%%%%
\subsection*{继}

\begin{Entry}{继}{10}{⽷}
  \begin{Phonetics}{继}{ji4}[][HSK 7-9]
    \definition{adv.}{então; depois}
    \definition{s.}{filhos; prole}
    \definition{v.}{continuar; ter sucesso; seguir}
  \end{Phonetics}
\end{Entry}

\begin{Entry}{继父}{10,4}{⽷、⽗}
  \begin{Phonetics}{继父}{ji4fu4}[][HSK 7-9]
    \definition{s.}{padrasto}
  \end{Phonetics}
\end{Entry}

\begin{Entry}{继母}{10,5}{⽷、⽏}
  \begin{Phonetics}{继母}{ji4mu3}[][HSK 7-9]
    \definition{s.}{madrasta}
  \end{Phonetics}
\end{Entry}

\begin{Entry}{继而}{10,6}{⽷、⽽}
  \begin{Phonetics}{继而}{ji4'er2}[][HSK 7-9]
    \definition{adv.}{então; depois; mais tarde}
  \end{Phonetics}
\end{Entry}

\begin{Entry}{继承}{10,8}{⽷、⼿}
  \begin{Phonetics}{继承}{ji4cheng2}[][HSK 5]
    \definition{v.}{herdar (o patrimônio de uma pessoa falecida, etc.) de acordo com a lei | continuar; geralmente se refere à aceitação do estilo, da cultura, do conhecimento, etc., daqueles que nos precederam | continuar; os descendentes continuam o trabalho deixado por seus antecessores.}
  \end{Phonetics}
\end{Entry}

\begin{Entry}{继续}{10,11}{⽷、⽷}
  \begin{Phonetics}{继续}{ji4xu4}[][HSK 3]
    \definition{s.}{continuação}
    \definition{v.}{continuar; prosseguir | prosseguir; continuar; seguir em frente (com); (atividades, eventos, etc.) continuar após uma pausa ou um determinado período de tempo}
  \end{Phonetics}
\end{Entry}

%%%%%%%%%% 缺 %%%%%%%%%%
\subsection*{缺}

\begin{Entry}{缺}{10}{⽸}
  \begin{Phonetics}{缺}{que1}[][HSK 3]
    \definition{adj.}{incompleto; imperfeito}
    \definition[种]{s.}{vaga; abertura; falta}
    \definition{v.}{estar com falta de; faltar | estar ausente}
  \end{Phonetics}
\end{Entry}

\begin{Entry}{缺乏}{10,4}{⽸、⼃}
  \begin{Phonetics}{缺乏}{que1fa2}[][HSK 5]
    \definition{v.}{faltar; estar em falta de; não ter ou não ter totalmente (algo que deveria possuir ou é desejaria possuir)}
  \end{Phonetics}
\end{Entry}

\begin{Entry}{缺少}{10,4}{⽸、⼩}
  \begin{Phonetics}{缺少}{que1shao3}[][HSK 3]
    \definition{v.}{falta; estar com falta de; estar em falta de; geralmente se refere à falta de pessoas ou coisas}
  \end{Phonetics}
\end{Entry}

\begin{Entry}{缺点}{10,9}{⽸、⽕}
  \begin{Phonetics}{缺点}{que1dian3}[][HSK 3]
    \definition[个,些]{s.}{desvantagem; deficiência; inconveniência; ponto fraco; uma deficiência ou imperfeição (em oposição a 优点)}
  \seealsoref{优点}{you1dian3}
  \end{Phonetics}
\end{Entry}

\begin{Entry}{缺陷}{10,10}{⽸、⾩}
  \begin{Phonetics}{缺陷}{que1xian4}[][HSK 6]
    \definition[个,处,项]{pron.}{defeito; falha; inconveniência; mancha; um lugar onde uma pessoa ou coisa está incompleta ou tem falhas porque algo está faltando}
  \end{Phonetics}
\end{Entry}

\begin{Entry}{缺勤}{10,13}{⽸、⼒}
  \begin{Phonetics}{缺勤}{que1/qin2}
    \definition{v.+compl.}{ausentar-se do dever (trabalho)}
  \end{Phonetics}
\end{Entry}

%%%%%%%%%% 罢 %%%%%%%%%%
\subsection*{罢}

\begin{Entry}{罢}{10}{⽹}
  \begin{Phonetics}{罢}{ba4}
    \definition{v.}{parar; cessar | revogar; destituir; encerrar | terminar | abandonar uma ideia; esqueçer sobre algo; deixar estar (passar)}
  \end{Phonetics}
  \begin{Phonetics}{罢}{ba5}
    \definition{part.}{partícula final, a mesma que 吧}
  \seealsoref{吧}{ba5}
  \end{Phonetics}
\end{Entry}

\begin{Entry}{罢了}{10,2}{⽹、⼅}
  \begin{Phonetics}{罢了}{ba4 le5}[][HSK 6]
    \definition{part.}{usado no final de uma frase, significa 仅此而已, geralmente seguido de 无非, 不过, 只是}
  \seealsoref{不过}{bu2guo4}
  \seealsoref{仅此而已}{jin3ci3'er2yi3}
  \seealsoref{无非}{wu2fei1}
  \seealsoref{只是}{zhi3 shi4}
  \end{Phonetics}
  \begin{Phonetics}{罢了}{ba4 liao3}
    \definition{part.}{uma partícula modal indicando (não se preocupe, ok)}
  \end{Phonetics}
\end{Entry}

\begin{Entry}{罢工}{10,3}{⽹、⼯}
  \begin{Phonetics}{罢工}{ba4gong1}[][HSK 6]
    \definition{v.}{parar de trabalhar; entrar em greve; abandonar o emprego}
  \end{Phonetics}
\end{Entry}

\begin{Entry}{罢休}{10,6}{⽹、⼈}
  \begin{Phonetics}{罢休}{ba4xiu1}[][HSK 7-9]
    \definition{v.}{parar; desistir; deixar o assunto de lado; parar de fazer algo, enfatizando a determinação de parar de fazê-lo}
  \end{Phonetics}
\end{Entry}

\begin{Entry}{罢免}{10,7}{⽹、⼉}
  \begin{Phonetics}{罢免}{ba4mian3}[][HSK 7-9]
    \definition{v.}{destituir; remover do cargo; demitir alguém do seu posto | destituir do cargo uma pessoa eleita pelo eleitorado ou por um órgão representativo}
  \end{Phonetics}
\end{Entry}

%%%%%%%%%% 翅 %%%%%%%%%%
\subsection*{翅}

\begin{Entry}{翅}{10}{⽻}
  \begin{Phonetics}{翅}{chi4}
    \definition[只]{s.}{asa | barbatana de tubarão | coisa parecida com uma asa}
  \end{Phonetics}
\end{Entry}

\begin{Entry}{翅膀}{10,14}{⽻、⾁}
  \begin{Phonetics}{翅膀}{chi4bang3}[][HSK 7-9]
    \definition[只,个,对]{s.}{asa; os órgãos de voo de animais como pássaros e insetos geralmente aparecem em pares | barbatana; aba; lâmina; a parte de algo que tem o formato ou age como uma asa}
  \end{Phonetics}
\end{Entry}

%%%%%%%%%% 耕 %%%%%%%%%%
\subsection*{耕}

\begin{Entry}{耕}{10}{⽾}
  \begin{Phonetics}{耕}{geng1}
    \definition{v.}{arar; cultivar | trabalhar; fazer | ganhar a vida}
  \end{Phonetics}
\end{Entry}

\begin{Entry}{耕地}{10,6}{⽾、⼟}
  \begin{Phonetics}{耕地}{geng1/di4}[][HSK 7-9]
    \definition[块,公顷]{s.}{terra cultivada; terra para cultivo}
    \definition{v.+compl.}{lavrar; arar}
  \end{Phonetics}
\end{Entry}

%%%%%%%%%% 耗 %%%%%%%%%%
\subsection*{耗}

\begin{Entry}{耗}{10}{⽾}
  \begin{Phonetics}{耗}{hao4}[][HSK 7-9]
    \definition{s.}{más notícias}[听到噩耗,他碎心裂胆。===Ele ficou arrasado ao ouvir as más notícias.]
    \definition{v.}{consumir; custar | perder tempo; procrastinar}
  \end{Phonetics}
\end{Entry}

\begin{Entry}{耗时}{10,7}{⽾、⽇}
  \begin{Phonetics}{耗时}{hao4shi2}[][HSK 7-9]
    \definition{adj.}{demorado; levar um período de ($x$ quantidade de tempo)}
  \end{Phonetics}
\end{Entry}

\begin{Entry}{耗费}{10,9}{⽾、⾙}
  \begin{Phonetics}{耗费}{hao4fei4}[][HSK 7-9]
    \definition{v.}{gastar; consumir; esgotar}
  \end{Phonetics}
\end{Entry}

%%%%%%%%%% 耻 %%%%%%%%%%
\subsection*{耻}

\begin{Entry}{耻}{10}{⽿}
  \begin{Phonetics}{耻}{chi3}
    \definition{s.}{vergonha; desgraça; humilhação}
    \definition{v.}{estar envergonhado de; considerar vergonhoso}
  \end{Phonetics}
\end{Entry}

\begin{Entry}{耻笑}{10,10}{⽿、⽵}
  \begin{Phonetics}{耻笑}{chi3xiao4}[][HSK 7-9]
    \definition{v.}{ridicularizar alguém; zombar; zombar de; rir de}
  \end{Phonetics}
\end{Entry}

\begin{Entry}{耻辱}{10,10}{⽿、⾠}
  \begin{Phonetics}{耻辱}{chi3ru3}[][HSK 7-9]
    \definition{s.}{vergonha; desgraça; humilhação; danos à reputação; incidente vergonhoso}
  \end{Phonetics}
\end{Entry}

%%%%%%%%%% 耽 %%%%%%%%%%
\subsection*{耽}

\begin{Entry}{耽}{10}{⽿}
  \begin{Phonetics}{耽}{dan1}
    \definition*{s.}{Sobrenome: Dan}
    \definition{v.}{atrasar | (literário) abandonar-se a; entregar-se a}
  \end{Phonetics}
\end{Entry}

\begin{Entry}{耽心}{10,4}{⽿、⼼}
  \begin{Phonetics}{耽心}{dan1xin1}
    \variantof{担心}
  \end{Phonetics}
\end{Entry}

\begin{Entry}{耽误}{10,9}{⽿、⾔}
  \begin{Phonetics}{耽误}{dan1wu5}[][HSK 7-9]
    \definition{v.}{atrasar; segurar; perder algo devido a atraso ou oportunidade perdida; perder (oportunidade)}
  \end{Phonetics}
\end{Entry}

\begin{Entry}{耽搁}{10,12}{⽿、⼿}
  \begin{Phonetics}{耽搁}{dan1ge5}[][HSK 7-9]
    \definition{v.}{ficar; fazer uma parada | atrasar | perder (uma oportunidade, um prazo)}
  \end{Phonetics}
\end{Entry}

%%%%%%%%%% 耿 %%%%%%%%%%
\subsection*{耿}

\begin{Entry}{耿}{10}{⽿}
  \begin{Phonetics}{耿}{geng3}
    \definition*{s.}{Sobrenome: Geng}
    \definition{adj.}{Literário: brilhante | honesto e justo; correto; íntegro | dedicado; leal}
  \end{Phonetics}
\end{Entry}

\begin{Entry}{耿直}{10,8}{⽿、⽬}
  \begin{Phonetics}{耿直}{geng3zhi2}[][HSK 7-9]
    \definition{adj.}{íntregro; franco; correto; honesto e franco}
  \end{Phonetics}
\end{Entry}

%%%%%%%%%% 胳 %%%%%%%%%%
\subsection*{胳}

\begin{Entry}{胳}{10}{⾁}
  \begin{Phonetics}{胳}{ga1}
    \definition{s.}{usado em 胳肢窝}
  \seealsoref{胳肢窝}{ga1 zhi1 wo1}
  \end{Phonetics}
  \begin{Phonetics}{胳}{ge1}
    \definition{s.}{axila; sovaco}
  \end{Phonetics}
  \begin{Phonetics}{胳}{ge2}
    \definition{v.}{usado em 胳肢}
  \seealsoref{胳肢}{ge2zhi5}
  \end{Phonetics}
\end{Entry}

\begin{Entry}{胳肢}{10,8}{⾁、⾁}
  \begin{Phonetics}{胳肢}{ge2zhi5}
    \definition{v.}{(dialeto) fazer cócegas}
  \end{Phonetics}
\end{Entry}

\begin{Entry}{胳肢窝}{10,8,12}{⾁、⾁、⽳}
  \begin{Phonetics}{胳肢窝}{ga1 zhi1 wo1}
    \definition{s.}{axila; sovaco; também escrito 夹肢窝}
  \seealsoref{夹肢窝}{jia1 zhi1 wo1}
  \end{Phonetics}
\end{Entry}

\begin{Entry}{胳膊}{10,14}{⾁、⾁}
  \begin{Phonetics}{胳膊}{ge1bo5}[][HSK 7-9]
    \definition[条,双,只]{s.}{braço; a área abaixo do ombro e acima do pulso}
  \end{Phonetics}
\end{Entry}

%%%%%%%%%% 胶 %%%%%%%%%%
\subsection*{胶}

\begin{Entry}{胶}{10}{⾁}
  \begin{Phonetics}{胶}{jiao1}
    \definition*{s.}{Sobrenome: Jiao}
    \definition{adj.}{pegajoso; viscoso; grudento}
    \definition{s.}{cola; goma; adesivo | borracha | gel; colóide}
    \definition{v.}{colar com cola | colar; grudar}
  \end{Phonetics}
\end{Entry}

\begin{Entry}{胶水}{10,4}{⾁、⽔}
  \begin{Phonetics}{胶水}{jiao1shui3}[][HSK 5]
    \definition[瓶]{s.}{cola; mucilagem; cola líquida}
  \end{Phonetics}
\end{Entry}

\begin{Entry}{胶片}{10,4}{⾁、⽚}
  \begin{Phonetics}{胶片}{jiao1pian4}[][HSK 7-9]
    \definition[卷]{s.}{filme; cartucho; filme fotográfico}
  \end{Phonetics}
\end{Entry}

\begin{Entry}{胶卷}{10,8}{⾁、⼙}
  \begin{Phonetics}{胶卷}{jiao1juan3}
    \definition{s.}{filme | rolo de filme}
  \end{Phonetics}
\end{Entry}

\begin{Entry}{胶带}{10,9}{⾁、⼱}
  \begin{Phonetics}{胶带}{jiao1 dai4}[][HSK 5]
    \definition[卷,条,段]{s.}{fita de embalagem transparente; fita adesiva | fita magnética de plástico; fita de gravação | fita emborrachada; cinta de borracha}
  \end{Phonetics}
\end{Entry}

\begin{Entry}{胶囊}{10,22}{⾁、⾐}
  \begin{Phonetics}{胶囊}{jiao1nang2}[][HSK 7-9]
    \definition{s.}{Medicina: cápsula; refere-se a uma cápsula de gelatina usada para encapsular medicamentos em pó ou granulados, facilitando a ingestão}
  \end{Phonetics}
\end{Entry}

%%%%%%%%%% 胸 %%%%%%%%%%
\subsection*{胸}

\begin{Entry}{胸}{10}{⾁}
  \begin{Phonetics}{胸}{xiong1}
    \definition{s.}{peito | tórax}
  \end{Phonetics}
\end{Entry}

\begin{Entry}{胸部}{10,10}{⾁、⾢}
  \begin{Phonetics}{胸部}{xiong1 bu4}[][HSK 4]
    \definition{s.}{peito; tórax; seios}
  \end{Phonetics}
\end{Entry}

%%%%%%%%%% 能 %%%%%%%%%%
\subsection*{能}

\begin{Entry}{能}{10}{⾁}
  \begin{Phonetics}{能}{neng2}[][HSK 1]
    \definition*{s.}{Sobrenome: Neng}
    \definition{adv.}{talvez}
    \definition{s.}{habilidade; capacidade; competência | potência; energia; em física, refere-se à energia}
    \definition{v.}{poder fazer; ser capaz de | ser possível | entre 不 \dots 不 para expressar obrigação, certeza ou grande probabilidade | poder; ter permissão para | ser bom em fazer algo | permitir}
  \end{Phonetics}
\end{Entry}

\begin{Entry}{能力}{10,2}{⾁、⼒}
  \begin{Phonetics}{能力}{neng2li4}[][HSK 3]
    \definition[个,种]{s.}{habilidade; capacidade; aptidão; as condições subjetivas para ser competente para uma tarefa}
  \end{Phonetics}
\end{Entry}

\begin{Entry}{能上能下}{10,3,10,3}{⾁、⼀、⾁、⼀}
  \begin{Phonetics}{能上能下}{neng2shang4neng2xia4}
    \definition{s.}{pronto para aceitar qualquer trabalho, alto ou baixo}
  \end{Phonetics}
\end{Entry}

\begin{Entry}{能干}{10,3}{⾁、⼲}
  \begin{Phonetics}{能干}{neng2gan4}[][HSK 4]
    \definition{adj.}{apto; capaz; competente}
  \end{Phonetics}
\end{Entry}

\begin{Entry}{能不能}{10,4,10}{⾁、⼀、⾁}
  \begin{Phonetics}{能不能}{neng2 bu4 neng2}[][HSK 3]
    \definition{adv.}{pode ou não pode\dots?}
  \end{Phonetics}
\end{Entry}

\begin{Entry}{能否}{10,7}{⾁、⼝}
  \begin{Phonetics}{能否}{neng2 fou3}[][HSK 6]
    \definition{adv.}{é possível; se ou não; pode ou não pode; Você consegue?; expressa dúvida, frequentemente usado em perguntas de sim ou não}
  \end{Phonetics}
\end{Entry}

\begin{Entry}{能够}{10,11}{⾁、⼣}
  \begin{Phonetics}{能够}{neng2 gou4}[][HSK 2]
    \definition{v.}{poder; ser capaz de; indica que possui uma determinada capacidade ou que atingiu um determinado nível de eficiência | poder; ser capaz de; indica que algo é permitido sob certas condições ou por motivos razoáveis}
  \end{Phonetics}
\end{Entry}

\begin{Entry}{能量}{10,12}{⾁、⾥}
  \begin{Phonetics}{能量}{neng2liang4}[][HSK 5]
    \definition[种]{s.}{energia; quantidade de energia; Uma grandeza física que mede a capacidade da matéria de realizar trabalho | capacidade; competências; capacidade e papel que uma pessoa pode desempenhar}
  \end{Phonetics}
\end{Entry}

%%%%%%%%%% 脂 %%%%%%%%%%
\subsection*{脂}

\begin{Entry}{脂}{10}{⾁}
  \begin{Phonetics}{脂}{zhi1}
    \definition*{s.}{Sobrenome: Zhi}
    \definition{s.}{gordura; graxa; sebo | (cosméticos) rouge | (cosméticos) baton; protetor labial}
  \end{Phonetics}
\end{Entry}

\begin{Entry}{脂麻}{10,11}{⾁、⿇}
  \begin{Phonetics}{脂麻}{zhi1ma5}
    \variantof{芝麻}
  \end{Phonetics}
\end{Entry}

%%%%%%%%%% 脆 %%%%%%%%%%
\subsection*{脆}

\begin{Entry}{脆}{10}{⾁}
  \begin{Phonetics}{脆}{cui4}[][HSK 5]
    \definition{adj.}{frágil; quebradiço (oposto a 韧) | crocante | (voz) clara; nítida | puro}
  \seealsoref{韧}{ren4}
  \end{Phonetics}
\end{Entry}

\begin{Entry}{脆弱}{10,10}{⾁、⼸}
  \begin{Phonetics}{脆弱}{cui4ruo4}[][HSK 7-9]
    \definition{adj.}{frágil; débil; fraco; incapaz de suportar contratempos}
  \end{Phonetics}
\end{Entry}

%%%%%%%%%% 脊 %%%%%%%%%%
\subsection*{脊}

\begin{Entry}{脊}{10}{⾁}
  \begin{Phonetics}{脊}{ji2}
    \definition{s.}{coluna vertebral (de humanos e animais) | espinha; costas; cumeeira; a parte de um objeto em forma de espinha}
  \end{Phonetics}
  \begin{Phonetics}{脊}{ji3}
    \definition{s.}{espinha dorsal; coluna vertebral | crista; cumeeira; espinhaço | vértebra}
  \end{Phonetics}
\end{Entry}

\begin{Entry}{脊梁}{10,11}{⾁、⽊}
  \begin{Phonetics}{脊梁}{ji3liang2}[][HSK 7-9]
    \definition{s.}{espinha dorsal | coluna vertebral}
  \end{Phonetics}
\end{Entry}

%%%%%%%%%% 脏 %%%%%%%%%%
\subsection*{脏}

\begin{Entry}{脏}{10}{⾁}
  \begin{Phonetics}{脏}{zang1}[][HSK 2]
    \definition{adj.}{sujo; imundo | imundo; metáfora para vulgaridade e obscenidade}
    \definition{v.}{tornar algo sujo ou impuro}
  \end{Phonetics}
  \begin{Phonetics}{脏}{zang4}
    \definition[处]{s.}{vísceras; órgãos internos do corpo, geralmente o coração, o fígado, o baço, os pulmões e os rins; um termo geral para órgãos nas cavidades torácica e abdominal de humanos ou animais | (anatomia) órgão; a medicina tradicional chinesa chama o coração, o fígado, o baço, os pulmões e os rins de órgãos internos}
  \end{Phonetics}
\end{Entry}

\begin{Entry}{脏土}{10,3}{⾁、⼟}
  \begin{Phonetics}{脏土}{zang1tu3}
    \definition{s.}{solo sujo | lama | lixo}
  \end{Phonetics}
\end{Entry}

\begin{Entry}{脏字}{10,6}{⾁、⼦}
  \begin{Phonetics}{脏字}{zang1zi4}
    \definition{s.}{obscenidade}
  \end{Phonetics}
\end{Entry}

\begin{Entry}{脏病}{10,10}{⾁、⽧}
  \begin{Phonetics}{脏病}{zang1bing4}
    \definition{s.}{doença venérea}
  \end{Phonetics}
\end{Entry}

\begin{Entry}{脏脏}{10,10}{⾁、⾁}
  \begin{Phonetics}{脏脏}{zang1zang1}
    \definition{adj.}{sujo}
  \end{Phonetics}
\end{Entry}

\begin{Entry}{脏煤}{10,13}{⾁、⽕}
  \begin{Phonetics}{脏煤}{zang1mei2}
    \definition{s.}{carvão sujo | sujeira (de uma mina de carvão)}
  \end{Phonetics}
\end{Entry}

\begin{Entry}{脏器}{10,16}{⾁、⼝}
  \begin{Phonetics}{脏器}{zang4qi4}
    \definition{s.}{órgãos internos}
  \end{Phonetics}
\end{Entry}

\begin{Entry}{脏辫}{10,17}{⾁、⾟}
  \begin{Phonetics}{脏辫}{zang1bian4}
    \definition{s.}{\emph{dreadlocks}}
  \end{Phonetics}
\end{Entry}

%%%%%%%%%% 脑 %%%%%%%%%%
\subsection*{脑}

\begin{Entry}{脑}{10}{⾁}
  \begin{Phonetics}{脑}{nao3}
    \definition{s.}{(fisiologia) cérebro | tofu;  substância branca semelhante ao cérebro ou à medula espinhal cerebral | cabeça | a essência de um objeto}
  \end{Phonetics}
\end{Entry}

\begin{Entry}{脑子}{10,3}{⾁、⼦}
  \begin{Phonetics}{脑子}{nao3 zi5}[][HSK 5]
    \definition[个]{s.}{cérebro | mente; cabeça; cérebro; inteligência; poder mental; refere-se à capacidade de pensar, memorizar, raciocinar, etc.; inteligência}
  \end{Phonetics}
\end{Entry}

\begin{Entry}{脑瓜}{10,5}{⾁、⽠}
  \begin{Phonetics}{脑瓜}{nao3gua1}
    \definition{s.}{crânio | cérebro | cabeça | mente | mentalidade | ideia}
  \seealsoref{脑瓜子}{nao3gua1zi5}
  \end{Phonetics}
\end{Entry}

\begin{Entry}{脑瓜子}{10,5,3}{⾁、⽠、⼦}
  \begin{Phonetics}{脑瓜子}{nao3gua1zi5}
    \definition{s.}{Coloquial: crânio; cérebro; cabeça; mente; mentalidade; ideia}
  \seealsoref{脑瓜}{nao3gua1}
  \end{Phonetics}
\end{Entry}

\begin{Entry}{脑袋}{10,11}{⾁、⾐}
  \begin{Phonetics}{脑袋}{nao3dai5}[][HSK 4]
    \definition[颗,个]{s.}{cabeça; a parte mais alta do corpo humano ou a parte mais alta de um animal que contém órgãos como a boca, o nariz, os olhos etc. | mente; cérebro; capacidade de pensar, lembrar, etc.}
  \end{Phonetics}
\end{Entry}

%%%%%%%%%% 臭 %%%%%%%%%%
\subsection*{臭}

\begin{Entry}{臭}{10}{⾃}
  \begin{Phonetics}{臭}{chou4}[][HSK 5]
    \definition{adj.}{sujo; malcheiroso; fedorento; contrário de 香 | repugnante; nojento; repulsivo | ruim; pobre; péssimo}
    \definition{adv.}{severamente; firmemente}
    \definition{v.}{falhar em detonar (bala)}
  \seealsoref{香}{xiang1}
  \end{Phonetics}
  \begin{Phonetics}{臭}{xiu4}
    \definition{s.}{odor; cheiro}
    \definition{v.}{cheirar; farejar; o mesmo que 嗅}
  \seealsoref{嗅}{xiu4}
  \end{Phonetics}
\end{Entry}

\begin{Entry}{臭气}{10,4}{⾃、⽓}
  \begin{Phonetics}{臭气}{chou4qi4}
    \definition{s.}{fedor}
  \end{Phonetics}
\end{Entry}

%%%%%%%%%% 致 %%%%%%%%%%
\subsection*{致}

\begin{Entry}{致}{10}{⾄}
  \begin{Phonetics}{致}{zhi4}
    \definition{adj.}{fino; delicado; meticuloso; preciso}
    \definition{s.}{interesse}
    \definition{v.}{enviar; estender; entregar; dar; mostrar (cortesia, afeto, etc.) à outra parte | concentrar-se; trabalhar para; dedicar (os próprios esforços, etc.); focar em um aspecto | causar; incorrer; convidar; levar a | alcançar}
  \end{Phonetics}
\end{Entry}

\begin{Entry}{致敬}{10,12}{⾄、⽁}
  \begin{Phonetics}{致敬}{zhi4jing4}
    \definition{v.}{saudar; prestar homenagem a; demonstrar respeito (homenagem) a}
  \end{Phonetics}
\end{Entry}

%%%%%%%%%% 航 %%%%%%%%%%
\subsection*{航}

\begin{Entry}{航}{10}{⾈}
  \begin{Phonetics}{航}{hang2}
    \definition*{s.}{Sobrenome: Hang}
    \definition[趟]{s.}{barco; navio}
    \definition{v.}{navegar (por água ou ar) | velejar}
  \end{Phonetics}
\end{Entry}

\begin{Entry}{航天}{10,4}{⾈、⼤}
  \begin{Phonetics}{航天}{hang2tian1}[][HSK 7-9]
    \definition{s.}{voo espacial; astronáutica}
    \definition{v.}{voar ou viajar no espaço}
  \end{Phonetics}
\end{Entry}

\begin{Entry}{航天员}{10,4,7}{⾈、⼤、⼝}
  \begin{Phonetics}{航天员}{hang2tian1yuan2}[][HSK 7-9]
    \definition[名,位,个]{s.}{astronauta}
  \end{Phonetics}
\end{Entry}

\begin{Entry}{航行}{10,6}{⾈、⾏}
  \begin{Phonetics}{航行}{hang2xing2}[][HSK 7-9]
    \definition{v.}{velejar; voar; navegar pela água, pelo ar}
  \end{Phonetics}
\end{Entry}

\begin{Entry}{航运}{10,7}{⾈、⾡}
  \begin{Phonetics}{航运}{hang2yun4}[][HSK 7-9]
    \definition{s.}{transporte hidroviário; transporte marítimo}
  \end{Phonetics}
\end{Entry}

\begin{Entry}{航空}{10,8}{⾈、⽳}
  \begin{Phonetics}{航空}{hang2kong1}[][HSK 4]
    \definition{s.}{viagem; aviação; refere-se ao voo de uma aeronave no ar}
  \end{Phonetics}
\end{Entry}

\begin{Entry}{航海}{10,10}{⾈、⽔}
  \begin{Phonetics}{航海}{hang2hai3}[][HSK 7-9]
    \definition{v.}{velejar; navegar}
  \end{Phonetics}
\end{Entry}

\begin{Entry}{航班}{10,10}{⾈、⽟}
  \begin{Phonetics}{航班}{hang2ban1}[][HSK 4]
    \definition[个,次]{s.}{número do voo; voo programado; o horário de um navio ou avião de passageiros}
  \end{Phonetics}
\end{Entry}

%%%%%%%%%% 般 %%%%%%%%%%
\subsection*{般}

\begin{Entry}{般}{10}{⾈}
  \begin{Phonetics}{般}{ban1}
    \definition{clas.}{tipo; classe; gênero; amostra}
    \definition{part.}{(o mesmo) que; como; semelhante}
  \end{Phonetics}
  \begin{Phonetics}{般}{bo1}
    \definition{s.}{utilizado em 般若}
  \seealsoref{般若}{bo1re3}
  \end{Phonetics}
  \begin{Phonetics}{般}{pan2}
    \definition{adj.}{feliz; bem-aventurado}
  \end{Phonetics}
\end{Entry}

\begin{Entry}{般乐}{10,5}{⾈、⼃}
  \begin{Phonetics}{般乐}{pan2le4}
    \definition{v.}{jogar | divertir-se}
  \end{Phonetics}
\end{Entry}

\begin{Entry}{般若}{10,8}{⾈、⾋}
  \begin{Phonetics}{般若}{bo1re3}
    \definition*{s.}{Prajña (sânscrito), \emph{insight} sobre a verdadeira natureza da realidade}
    \definition{s.}{budismo: sabedoria}
  \end{Phonetics}
\end{Entry}

%%%%%%%%%% 舱 %%%%%%%%%%
\subsection*{舱}

\begin{Entry}{舱}{10}{⾈}
  \begin{Phonetics}{舱}{cang1}[][HSK 7-9]
    \definition{s.}{cabine (de um avião ou navio) | módulo (de uma nave espacial) | espaço em um navio ou aeronave para transportar pessoas, carga ou máquinas}
  \end{Phonetics}
\end{Entry}

%%%%%%%%%% 荷 %%%%%%%%%%
\subsection*{荷}

\begin{Entry}{荷}{10}{⾋}
  \begin{Phonetics}{荷}{he2}
    \definition*{s.}{Países Baixos; Holanda, abreviação de 荷兰 | Sobrenome: He}
    \definition{s.}{lótus}
  \seealsoref{荷兰}{he2lan2}
  \end{Phonetics}
  \begin{Phonetics}{荷}{he4}
    \definition{s.}{fardo; responsabilidade}
    \definition{v.}{carregar no ombro ou nas costas | aceitar um favor, frequentemente usado em cartas para expressar cortesia}
  \end{Phonetics}
\end{Entry}

\begin{Entry}{荷兰}{10,5}{⾋、⼋}
  \begin{Phonetics}{荷兰}{he2lan2}
    \definition*{s.}{Países Baixos; Holanda}
  \end{Phonetics}
\end{Entry}

\begin{Entry}{荷花}{10,7}{⾋、⾋}
  \begin{Phonetics}{荷花}{he2hua1}[][HSK 7-9]
    \definition[朵,枝,片]{s.}{lótus; flor de lótus}
  \end{Phonetics}
\end{Entry}

%%%%%%%%%% 莎 %%%%%%%%%%
\subsection*{莎}

\begin{Entry}{莎}{10}{⾋}
  \begin{Phonetics}{莎}{sha1}
    \definition{s.}{em nomes pessoais e de lugares | cigarra | fonético "sha" usado na transliteração}
  \end{Phonetics}
  \begin{Phonetics}{莎}{suo1}
  \end{Phonetics}
\end{Entry}

\begin{Entry}{莎莎舞}{10,10,14}{⾋、⾋、⾇}
  \begin{Phonetics}{莎莎舞}{sha1sha1wu3}
    \definition{s.}{salsa (dança)}
  \end{Phonetics}
\end{Entry}

%%%%%%%%%% 莫 %%%%%%%%%%
\subsection*{莫}

\begin{Entry}{莫}{10}{⾋}
  \begin{Phonetics}{莫}{mo4}
    \definition*{s.}{Sobrenome: Mo}
    \definition{adv.}{não, frequentemente usado em frases imperativas | não; não pode | pode ser que; não pode ser que; é possível que}
    \definition{pron.}{nenhum; nada; ninguém; significa 没有谁 ou 没有哪一种东西}
  \seealsoref{没有哪一种东西}{mei2you3 na3 yi4 zhong3 dong1xi1}
  \seealsoref{没有谁}{mei2you3 shei2}
  \end{Phonetics}
\end{Entry}

\begin{Entry}{莫名其妙}{10,6,8,7}{⾋、⼝、⼋、⼥}
  \begin{Phonetics}{莫名其妙}{mo4ming2qi2miao4}
    \definition{adj.}{desconcertante | bizzaro | inexplicável | perplexo}
  \end{Phonetics}
\end{Entry}

\begin{Entry}{莫非}{10,8}{⾋、⾮}
  \begin{Phonetics}{莫非}{mo4fei1}
    \definition{expr.}{Não é mesmo?; é frequentemente usado com 不成}
    \definition{v.}{pode ser que; é possível que}
  \seealsoref{不成}{bu4 cheng2}
  \end{Phonetics}
\end{Entry}

%%%%%%%%%% 莲 %%%%%%%%%%
\subsection*{莲}

\begin{Entry}{莲}{10}{⾋}
  \begin{Phonetics}{莲}{lian2}
    \definition*{s.}{Sobrenome: Lian}
    \definition[粒]{s.}{lótus}
  \end{Phonetics}
\end{Entry}

\begin{Entry}{莲子}{10,3}{⾋、⼦}
  \begin{Phonetics}{莲子}{lian2zi3}[][HSK 7-9]
    \definition[颗,个]{s.}{semente de lótus; a semente de lótus tem formato oval, com um miolo verde no centro e polpa branco-leitosa; pode ser consumida e usada na medicina}
  \end{Phonetics}
\end{Entry}

\begin{Entry}{莲花}{10,7}{⾋、⾋}
  \begin{Phonetics}{莲花}{lian2hua1}
    \definition{s.}{flor de lótus | lírio aquático}
  \end{Phonetics}
\end{Entry}

\begin{Entry}{莲藕}{10,18}{⾋、⾋}
  \begin{Phonetics}{莲藕}{lian2'ou3}
    \definition{s.}{raiz de Lotus}
  \end{Phonetics}
\end{Entry}

%%%%%%%%%% 获 %%%%%%%%%%
\subsection*{获}

\begin{Entry}{获}{10}{⾋}
  \begin{Phonetics}{获}{huo4}[][HSK 4]
    \definition*{s.}{Sobrenome: Huo}
    \definition{v.}{capturar; pegar | obter; ganhar; colher | colher; ceifar}
  \end{Phonetics}
\end{Entry}

\begin{Entry}{获取}{10,8}{⾋、⼜}
  \begin{Phonetics}{获取}{huo4 qu3}[][HSK 4]
    \definition{v.}{adquirir; obter; ganhar; colher}
  \end{Phonetics}
\end{Entry}

\begin{Entry}{获奖}{10,9}{⾋、⼤}
  \begin{Phonetics}{获奖}{huo4 jiang3}[][HSK 4]
    \definition{v.}{ganhar prêmio; ser recompensado; ganhar um prêmio; receber um prêmio}
  \end{Phonetics}
\end{Entry}

\begin{Entry}{获胜}{10,9}{⾋、⾁}
  \begin{Phonetics}{获胜}{huo4sheng4}[][HSK 7-9]
    \definition{v.}{vencer; ser vitorioso; triunfar; alcançar a vitória}
  \end{Phonetics}
\end{Entry}

\begin{Entry}{获得}{10,11}{⾋、⼻}
  \begin{Phonetics}{获得}{huo4de2}[][HSK 4]
    \definition{v.}{adquirir; ganhar; obter; alcançar}
  \end{Phonetics}
\end{Entry}

\begin{Entry}{获悉}{10,11}{⾋、⼼}
  \begin{Phonetics}{获悉}{huo4xi1}[][HSK 7-9]
    \definition{v.}{saber (de um evento); receber notícias; ser informado}
  \end{Phonetics}
\end{Entry}

%%%%%%%%%% 蚊 %%%%%%%%%%
\subsection*{蚊}

\begin{Entry}{蚊}{10}{⾍}
  \begin{Phonetics}{蚊}{wen2}
    \definition{s.}{mosquito; pernilongo}
  \end{Phonetics}
\end{Entry}

\begin{Entry}{蚊子}{10,3}{⾍、⼦}
  \begin{Phonetics}{蚊子}{wen2zi5}
    \definition{s.}{pernilongo}
  \end{Phonetics}
\end{Entry}

\begin{Entry}{蚊香}{10,9}{⾍、⾹}
  \begin{Phonetics}{蚊香}{wen2xiang1}
    \definition{s.}{incenso ou espiral repelente de mosquitos}
  \end{Phonetics}
\end{Entry}

%%%%%%%%%% 蚕 %%%%%%%%%%
\subsection*{蚕}

\begin{Entry}{蚕}{10}{⾍}
  \begin{Phonetics}{蚕}{can2}
    \definition[只,条]{s.}{bicho-da-seda; um inseto que pode fiar seda e fazer casulos}
  \end{Phonetics}
\end{Entry}

\begin{Entry}{蚕纸}{10,7}{⾍、⽷}
  \begin{Phonetics}{蚕纸}{can2zhi3}
    \definition{s.}{papel onde o bicho-da-seda põe seus ovos}
  \end{Phonetics}
\end{Entry}

%%%%%%%%%% 蚝 %%%%%%%%%%
\subsection*{蚝}

\begin{Entry}{蚝}{10}{⾍}
  \begin{Phonetics}{蚝}{hao2}
    \definition[只]{s.}{ostra}
  \end{Phonetics}
\end{Entry}

%%%%%%%%%% 袖 %%%%%%%%%%
\subsection*{袖}

\begin{Entry}{袖}{10}{⾐}
  \begin{Phonetics}{袖}{xiu4}
    \definition{s.}{manga (de camisa, de camiseta, etc.)}
  \end{Phonetics}
\end{Entry}

\begin{Entry}{袖珍}{10,9}{⾐、⽟}
  \begin{Phonetics}{袖珍}{xiu4 zhen1}[][HSK 6]
    \definition{adj.}{do tamanho do bolso; de bolso (livro, agenda, etc.)}
  \end{Phonetics}
\end{Entry}

%%%%%%%%%% 袜 %%%%%%%%%%
\subsection*{袜}

\begin{Entry}{袜}{10}{⾐}
  \begin{Phonetics}{袜}{wa4}
    \definition[只,双,打]{s.}{meias; meias-calças}
  \end{Phonetics}
\end{Entry}

\begin{Entry}{袜子}{10,3}{⾐、⼦}
  \begin{Phonetics}{袜子}{wa4zi5}[][HSK 4]
    \definition[双,只,对]{s.}{meias; peúgas; meias-calças}
  \end{Phonetics}
\end{Entry}

%%%%%%%%%% 被 %%%%%%%%%%
\subsection*{被}

\begin{Entry}{被}{10}{⾐}
  \begin{Phonetics}{被}{bei4}[][HSK 3]
    \definition{part.}{usada antes de verbos para formar frases verbais passivas}
    \definition{prep.}{usado em uma estrutura passiva para introduzir o executor da ação ou apenas a ação | usado em frases para expressar passividade, com o sujeito sendo o objeto}
    \definition{s.}{colcha}
    \definition{v.}{cobrir; espalhar | sofrer}
  \end{Phonetics}
\end{Entry}

\begin{Entry}{被子}{10,3}{⾐、⼦}
  \begin{Phonetics}{被子}{bei4zi5}[][HSK 3]
    \definition[条,床]{s.}{colcha; cobertor; algo com que você se cobre quando dorme, geralmente feito de pano ou seda, com forro de pano e preenchido com algodão ou fio de seda}
  \end{Phonetics}
\end{Entry}

\begin{Entry}{被动}{10,6}{⾐、⼒}
  \begin{Phonetics}{被动}{bei4dong4}[][HSK 5]
    \definition{adj.}{passivo;  agir com base em um impulso externo (oposto de 主动) | passivo; impossibilidade de prosseguir como pretendido devido a resistência ou interferência}
  \seealsoref{主动}{zhu3dong4}
  \end{Phonetics}
\end{Entry}

\begin{Entry}{被告}{10,7}{⾐、⼝}
  \begin{Phonetics}{被告}{bei4gao4}[][HSK 6]
    \definition{s.}{réu; indiciado; acusado (oposto a 原告)}
  \seealsoref{原告}{yuan2gao4}
  \end{Phonetics}
\end{Entry}

\begin{Entry}{被单}{10,8}{⾐、⼗}
  \begin{Phonetics}{被单}{bei4dan1}
    \definition[床]{s.}{lençol (de cama) | envelope para uma colcha acolchoada}
  \end{Phonetics}
\end{Entry}

\begin{Entry}{被迫}{10,8}{⾐、⾡}
  \begin{Phonetics}{被迫}{bei4 po4}[][HSK 4]
    \definition{v.}{ser forçado; ser coagido; ser compelido; ser constrangido; ser forçado a fazer algo por força externa}
  \end{Phonetics}
\end{Entry}

\begin{Entry}{被套}{10,10}{⾐、⼤}
  \begin{Phonetics}{被套}{bei4tao4}
    \definition{s.}{capa de \emph{edredon}}
    \definition{v.}{ter dinheiro preso (em ações, imóveis, etc.)}
  \end{Phonetics}
\end{Entry}

\begin{Entry}{被捕}{10,10}{⾐、⼿}
  \begin{Phonetics}{被捕}{bei4bu3}[][HSK 7-9]
    \definition{v.}{ser preso; estar sob prisão}
  \end{Phonetics}
\end{Entry}

\begin{Entry}{被窝}{10,12}{⾐、⽳}
  \begin{Phonetics}{被窝}{bei4wo1}
    \definition{s.}{colcha}
  \end{Phonetics}
\end{Entry}

%%%%%%%%%% 请 %%%%%%%%%%
\subsection*{请}

\begin{Entry}{请}{10}{⾔}
  \begin{Phonetics}{请}{qing3}[][HSK 1]
    \definition*{s.}{Sobrenome: Qing}
    \definition{v.}{solicitar; perguntar | convidar; envolver | por favor; uma expressão educada usada quando você quer que alguém faça algo | comprar coisas sagradas para sacrifício, como incenso, velas, cavalos de papel e santuários de Buda; superstição se refere à compra de estátuas de Buda, santuários, etc. | entreter}
  \end{Phonetics}
\end{Entry}

\begin{Entry}{请问}{10,6}{⾔、⾨}
  \begin{Phonetics}{请问}{qing3 wen4}[][HSK 1]
    \definition{expr.}{Com licença, posso perguntar\dots? (para perguntar por qualquer coisa); uma maneira educada de pedir para alguém responder a uma pergunta}
  \end{Phonetics}
\end{Entry}

\begin{Entry}{请坐}{10,7}{⾔、⼟}
  \begin{Phonetics}{请坐}{qing3 zuo4}[][HSK 1]
    \definition{v.}{por favor, sente-se; convidar outras pessoas para sentar ou descansar}
  \end{Phonetics}
\end{Entry}

\begin{Entry}{请求}{10,7}{⾔、⽔}
  \begin{Phonetics}{请求}{qing3qiu2}[][HSK 2]
    \definition[个,次]{s.}{pedido; petição; solicitação; refere-se à exigência apresentada}
    \definition{v.}{pedir; solicitar; requerer; peticionar; fazer uma solicitação e pedir que a outra parte concorde com ela}
  \end{Phonetics}
\end{Entry}

\begin{Entry}{请进}{10,7}{⾔、⾡}
  \begin{Phonetics}{请进}{qing3 jin4}[][HSK 1]
    \definition{v.}{por favor entre; convidar alguém para um espaço ou lugar}
  \end{Phonetics}
\end{Entry}

\begin{Entry}{请客}{10,9}{⾔、⼧}
  \begin{Phonetics}{请客}{qing3/ke4}[][HSK 2]
    \definition{v.+compl.}{receber convidados; hospedar convidados | oferecer; convidar; pagar a conta; arcar com os custos; convidar alguém para comer, tomar chá, etc.}
  \end{Phonetics}
\end{Entry}

\begin{Entry}{请假}{10,11}{⾔、⼈}
  \begin{Phonetics}{请假}{qing3/jia4}[][HSK 1]
    \definition{v.+compl.}{pedir licença para sair; solicitar permissão para não trabalhar ou estudar por um determinado período de tempo devido a doença ou outros motivos}
  \end{Phonetics}
\end{Entry}

\begin{Entry}{请假条}{10,11,7}{⾔、⼈、⽊}
  \begin{Phonetics}{请假条}{qing3jia4tiao2}
    \definition{s.}{pedido de licença de ausência (do trabalho ou da escola)}
  \end{Phonetics}
\end{Entry}

\begin{Entry}{请教}{10,11}{⾔、⽁}
  \begin{Phonetics}{请教}{qing3jiao4}[][HSK 3]
    \definition{v.}{consultar; pedir conselho}
  \end{Phonetics}
\end{Entry}

%%%%%%%%%% 诸 %%%%%%%%%%
\subsection*{诸}

\begin{Entry}{诸}{10}{⾔}
  \begin{Phonetics}{诸}{zhu1}
    \definition*{s.}{Sobrenome: Zhu}
    \definition{adj.}{todos; cada; vários}
    \definition{prep.}{em; para; de}
  \end{Phonetics}
\end{Entry}

\begin{Entry}{诸位}{10,7}{⾔、⼈}
  \begin{Phonetics}{诸位}{zhu1wei4}[][HSK 6]
    \definition{pron.}{senhores; todos; todos vocês; senhoras e senhores; um termo educado que se refere a várias pessoas}
  \end{Phonetics}
\end{Entry}

%%%%%%%%%% 诺 %%%%%%%%%%
\subsection*{诺}

\begin{Entry}{诺}{10}{⾔}
  \begin{Phonetics}{诺}{nuo4}
    \definition*{s.}{Sobrenome: Nuo}
    \definition{interj.}{Sim!}
    \definition{v.}{prometer}
  \end{Phonetics}
\end{Entry}

\begin{Entry}{诺贝尔奖}{10,4,5,9}{⾔、⾙、⼩、⼤}
  \begin{Phonetics}{诺贝尔奖}{nuo4bei4'er3 jiang3}
    \definition*{s.}{Prêmio Nobel}
  \end{Phonetics}
\end{Entry}

\begin{Entry}{诺奖}{10,9}{⾔、⼤}
  \begin{Phonetics}{诺奖}{nuo4jiang3}
    \definition*{s.}{Prêmio Nobel, abreviação de 诺贝尔奖}
  \seealsoref{诺贝尔奖}{nuo4bei4'er3 jiang3}
  \end{Phonetics}
\end{Entry}

%%%%%%%%%% 读 %%%%%%%%%%
\subsection*{读}

\begin{Entry}{读}{10}{⾔}
  \begin{Phonetics}{读}{dou4}
    \definition{s.}{vírgula; uma breve pausa na leitura}
  \end{Phonetics}
  \begin{Phonetics}{读}{du2}[][HSK 1]
    \definition*{s.}{Sobrenome: Du}
    \definition{v.}{ler em voz alta | ler; ler o texto e compreendera seu significado | frequentar a escola; refere-se a ir à escola ou estudar | (computação) ler dados}
  \end{Phonetics}
\end{Entry}

\begin{Entry}{读书}{10,4}{⾔、⼄}
  \begin{Phonetics}{读书}{du2/shu1}[][HSK 1]
    \definition{v.+compl.}{ler; estudar | frequentar a escola}
  \end{Phonetics}
\end{Entry}

\begin{Entry}{读者}{10,8}{⾔、⽼}
  \begin{Phonetics}{读者}{du2 zhe3}[][HSK 3]
    \definition[个,位,名,些,群]{s.}{leitor; (para obras, autores, revistas, etc.) Pessoas que compram ou leem livros, revistas, artigos, jornais, etc.}
  \end{Phonetics}
\end{Entry}

\begin{Entry}{读音}{10,9}{⾔、⾳}
  \begin{Phonetics}{读音}{du2 yin1}[][HSK 2]
    \definition[种]{s.}{pronúncia}
  \end{Phonetics}
\end{Entry}

%%%%%%%%%% 诽 %%%%%%%%%%
\subsection*{诽}

\begin{Entry}{诽}{10}{⾔}
  \begin{Phonetics}{诽}{fei3}
    \definition{v.}{calúnia}
  \end{Phonetics}
\end{Entry}

\begin{Entry}{诽谤}{10,12}{⾔、⾔}
  \begin{Phonetics}{诽谤}{fei3bang4}[][HSK 7-9]
    \definition{v.}{difamar; caluniar; falar mal; espalhar boatos e caluniar os outros; falar mal dos outros e prejudicar sua reputação}
  \end{Phonetics}
\end{Entry}

%%%%%%%%%% 课 %%%%%%%%%%
\subsection*{课}

\begin{Entry}{课}{10}{⾔}
  \begin{Phonetics}{课}{ke4}[][HSK 1]
    \definition{clas.}{aula; lição; unidade de tempo de ensino; parágrafo do material didático}
    \definition[门,节]{s.}{classe; aula; ensino por etapas planejado | disciplina; curso | imposto; antiga referência a impostos | seção; departamentos de escritório criados no antigo governo}
    \definition{v.}{cobrar; impor; taxar}
  \end{Phonetics}
\end{Entry}

\begin{Entry}{课文}{10,4}{⾔、⽂}
  \begin{Phonetics}{课文}{ke4 wen2}[][HSK 1]
    \definition[篇,段]{s.}{texto (de uma lição); texto principal do livro didático (diferente das notas de rodapé, exercícios, etc.)}
  \end{Phonetics}
\end{Entry}

\begin{Entry}{课本}{10,5}{⾔、⽊}
  \begin{Phonetics}{课本}{ke4 ben3}[][HSK 1]
    \definition[本]{s.}{livro didático; livro-texto}
  \end{Phonetics}
\end{Entry}

\begin{Entry}{课堂}{10,11}{⾔、⼟}
  \begin{Phonetics}{课堂}{ke4 tang2}[][HSK 2]
    \definition[间,节,个]{s.}{sala de aula; local onde se realizam as aulas; local onde se realizam as atividades de ensino}
  \end{Phonetics}
\end{Entry}

\begin{Entry}{课程}{10,12}{⾔、⽲}
  \begin{Phonetics}{课程}{ke4cheng2}[][HSK 3]
    \definition[个,堂,节,门]{s.}{curso; currículo; as disciplinas e o programa letivo da escola}
  \end{Phonetics}
\end{Entry}

\begin{Entry}{课题}{10,15}{⾔、⾴}
  \begin{Phonetics}{课题}{ke4ti2}[][HSK 5]
    \definition[组]{s.}{uma questão para estudo ou discussão; principais questões a serem pesquisadas ou discutidas, ou assuntos importantes que precisam ser resolvidos com urgência | tarefa; problema; questões a serem resolvidas}
  \end{Phonetics}
\end{Entry}

%%%%%%%%%% 谁 %%%%%%%%%%
\subsection*{谁}

\begin{Entry}{谁}{10}{⾔}
  \begin{Phonetics}{谁}{shei2}[][HSK 1]
    \definition{pron.}{quem? | (em pergunta retórica) quem?; usado em perguntas retóricas, para indicar que não há ninguém | refere-se a pessoas que não têm certeza, incluindo aquelas que não sabem | alguém; qualquer pessoa; indica qualquer pessoa ou qualquer um | repetido em uma frase para se referir a uma pessoa | (repetido em duas frases) quem quer que seja; fazer com que o sujeito e o objeto se refiram a duas pessoas diferentes}
  \end{Phonetics}
  \begin{Phonetics}{谁}{shui2}[][HSK 1]
  \end{Phonetics}
\end{Entry}

%%%%%%%%%% 调 %%%%%%%%%%
\subsection*{调}

\begin{Entry}{调}{10}{⾔}
  \begin{Phonetics}{调}{diao4}[][HSK 3]
    \definition{s.}{sotaque; pronúncia | nota (musical) | melodia; música | tom; refere-se ao tom da fala, ou seja, a elevação e descida do tom das palavras | estilo; ambiente; estilo metafórico, talento, etc. | argumento; discurso}
    \definition{v.}{deslocar; mover; transferir; mover (pessoas, objetos, etc.) de um lugar para outro | examinar; investigar}
  \end{Phonetics}
  \begin{Phonetics}{调}{tiao2}[][HSK 3]
    \definition{adj.}{harmonioso; boa coordenação}
    \definition{v.}{misturar; ajustar; fazer o ajuste uniforme e apropriado | provocar; importunar; fazer pouco de | incitar; instigar; provocar; semear discórdia | mediar; trazer harmonia}
  \end{Phonetics}
\end{Entry}

\begin{Entry}{调皮}{10,5}{⾔、⽪}
  \begin{Phonetics}{调皮}{tiao2pi2}[][HSK 4]
    \definition{adj.}{travesso; malicioso; malandro | indisciplinado; desordeiro; indomável; astuto | inteligente e desonesto}
  \end{Phonetics}
\end{Entry}

\begin{Entry}{调节}{10,5}{⾔、⾋}
  \begin{Phonetics}{调节}{tiao2jie2}[][HSK 5]
    \definition{v.}{regular; ajustar; ajustar e controlar de várias maneiras para atender aos requisitos}
  \end{Phonetics}
\end{Entry}

\begin{Entry}{调动}{10,6}{⾔、⼒}
  \begin{Phonetics}{调动}{diao4dong4}[][HSK 5]
    \definition{v.}{mudar; transferir; pessoal, trabalho | mobilizar; despertar; pôr em jogo; melhorar (motivação, entusiasmo, etc.) por meio de alguns meios | reunir; manobrar; mover (tropas); mobilizar forças militares}
  \end{Phonetics}
\end{Entry}

\begin{Entry}{调度}{10,9}{⾔、⼴}
  \begin{Phonetics}{调度}{diao4du4}[][HSK 7-9]
    \definition[位,个]{s.}{despachante; pessoal responsável pelo despacho do trabalho}
    \definition{v.}{organizar e despachar; arranjar e despachar}
  \end{Phonetics}
\end{Entry}

\begin{Entry}{调律}{10,9}{⾔、⼻}
  \begin{Phonetics}{调律}{tiao2lv4}
    \definition{v.}{afinar (por exemplo, um piano)}
  \end{Phonetics}
\end{Entry}

\begin{Entry}{调查}{10,9}{⾔、⽊}
  \begin{Phonetics}{调查}{diao4cha2}[][HSK 3]
    \definition[项,个,份]{s.}{pesquisa; investigação; informações obtidas após perguntar a outras pessoas ou investigar}
    \definition{v.}{investigar; indagar; inquerir; examinar; realizar uma investigação (geralmente no local) para entender a situação}
  \end{Phonetics}
\end{Entry}

\begin{Entry}{调研}{10,9}{⾔、⽯}
  \begin{Phonetics}{调研}{diao4 yan2}[][HSK 6]
    \definition{v.}{pesquisar e estudar; investigar e pesquisar; pesquisar}
  \end{Phonetics}
\end{Entry}

\begin{Entry}{调解}{10,13}{⾔、⾓}
  \begin{Phonetics}{调解}{tiao2jie3}[][HSK 5]
    \definition{v.}{mediar; fazer as pazes; resolver conflitos através da persuasão}
  \end{Phonetics}
\end{Entry}

\begin{Entry}{调整}{10,16}{⾔、⽁}
  \begin{Phonetics}{调整}{tiao2zheng3}[][HSK 3]
    \definition{v.}{ajustar; revisar; regularizar; fazer as alterações apropriadas no estado original para se adaptar à nova situação}
  \end{Phonetics}
\end{Entry}

%%%%%%%%%% 谅 %%%%%%%%%%
\subsection*{谅}

\begin{Entry}{谅}{10}{⾔}
  \begin{Phonetics}{谅}{liang4}
    \definition*{s.}{Sobrenome: Liang}
    \definition{v.}{perdoar; compreender | supor; presumir | desculpar | pensar; acreditar; supor}
  \end{Phonetics}
\end{Entry}

\begin{Entry}{谅解}{10,13}{⾔、⾓}
  \begin{Phonetics}{谅解}{liang4jie3}[][HSK 7-9]
    \definition{v.}{compreender; levar em consideração; compreender e perdoar os outros após entender a situação real}
  \end{Phonetics}
\end{Entry}

%%%%%%%%%% 谈 %%%%%%%%%%
\subsection*{谈}

\begin{Entry}{谈}{10}{⾔}
  \begin{Phonetics}{谈}{tan2}[][HSK 3]
    \definition*{s.}{Sobrenome: Tan}
    \definition{s.}{o que é dito ou falado; discurso}
    \definition{v.}{falar; bater papo; discutir}
  \end{Phonetics}
\end{Entry}

\begin{Entry}{谈判}{10,7}{⾔、⼑}
  \begin{Phonetics}{谈判}{tan2pan4}[][HSK 3]
    \definition{v.}{negociar; manter conversações; para resolver um grande problema, as partes relevantes trocaram opiniões entre si, na esperança de encontrar uma solução com a qual todos pudessem concordar}
  \end{Phonetics}
\end{Entry}

\begin{Entry}{谈话}{10,8}{⾔、⾔}
  \begin{Phonetics}{谈话}{tan2/hua4}[][HSK 3]
    \definition[次]{s.}{declaração; opiniões (principalmente políticas) expressas na forma de conversas}
    \definition{v.+compl.}{conversar; discutir | falar; refere-se especificamente ao uso da conversa para entender a situação, fazer trabalho ideológico, etc. (usado principalmente por superiores para subordinados)}
  \end{Phonetics}
\end{Entry}

\begin{Entry}{谈恋爱}{10,10,10}{⾔、⼼、⽖}
  \begin{Phonetics}{谈恋爱}{tan2lian4'ai4}
    \definition{v.}{namorar | apaixonar-se}
  \end{Phonetics}
\end{Entry}

%%%%%%%%%% 豹 %%%%%%%%%%
\subsection*{豹}

\begin{Entry}{豹}{10}{⾘}
  \begin{Phonetics}{豹}{bao4}[][HSK 7-9]
    \definition*{s.}{Sobrenome: Bao}
    \definition[只]{s.}{leopardo; pantera | espécies de gato da montanha}
  \end{Phonetics}
\end{Entry}

\begin{Entry}{豹子}{10,3}{⾘、⼦}
  \begin{Phonetics}{豹子}{bao4zi5}
    \definition[头]{s.}{leopardo}
  \end{Phonetics}
\end{Entry}

%%%%%%%%%% 贿 %%%%%%%%%%
\subsection*{贿}

\begin{Entry}{贿}{10}{⾙}
  \begin{Phonetics}{贿}{hui4}
    \definition[行]{s.}{bens; riqueza; objetos de valor; propriedade | suborno | Literário: wealth}
    \definition{v.}{subornar}
  \end{Phonetics}
\end{Entry}

\begin{Entry}{贿赂}{10,10}{⾙、⾙}
  \begin{Phonetics}{贿赂}{hui4lu4}[][HSK 7-9]
    \definition[笔]{s.}{suborno}
    \definition{v.}{subornar; subornar outros com dinheiro}
  \end{Phonetics}
\end{Entry}

%%%%%%%%%% 资 %%%%%%%%%%
\subsection*{资}

\begin{Entry}{资}{10}{⾙}
  \begin{Phonetics}{资}{zi1}
    \definition{s.}{recursos | capital | dinheiro | despesa}
    \definition{v.}{fornecer | suprir}
  \end{Phonetics}
\end{Entry}

\begin{Entry}{资本}{10,5}{⾙、⽊}
  \begin{Phonetics}{资本}{zi1ben3}[][HSK 5]
    \definition{s.}{capital; meios de produção ou moeda utilizados para fins lucrativos | o que é capitalizado; algo usado em benefício próprio; metáfora para obter benefícios}
  \end{Phonetics}
\end{Entry}

\begin{Entry}{资产}{10,6}{⾙、⼇}
  \begin{Phonetics}{资产}{zi1chan3}[][HSK 5]
    \definition{s.}{propriedade; bens; patrimônio | capital; fundo de capital; recursos financeiros da empresa | ativos; na contabilidade, refere-se à utilização de fundos}
  \end{Phonetics}
\end{Entry}

\begin{Entry}{资助}{10,7}{⾙、⼒}
  \begin{Phonetics}{资助}{zi1zhu4}[][HSK 5]
    \definition{s.}{subsídio}
    \definition{v.}{subsidiar; patrocinar; ajudar financeiramente; ajudar com recursos financeiros}
  \end{Phonetics}
\end{Entry}

\begin{Entry}{资金}{10,8}{⾙、⾦}
  \begin{Phonetics}{资金}{zi1jin1}[][HSK 3]
    \definition[笔]{s.}{fundo; capital; capital necessário para atividades comerciais, etc.}
  \end{Phonetics}
\end{Entry}

\begin{Entry}{资料}{10,10}{⾙、⽃}
  \begin{Phonetics}{资料}{zi1liao4}[][HSK 4]
    \definition[份,堆,本,个]{s.}{dados; material; material informativo para referência ou para ser considerado confiável | material de produção; meios de subsistência; requisitos de produção ou subsistência}
  \end{Phonetics}
\end{Entry}

\begin{Entry}{资格}{10,10}{⾙、⽊}
  \begin{Phonetics}{资格}{zi1ge2}[][HSK 3]
    \definition{s.}{qualificação; condições e identidades necessárias para exercer uma determinada atividade | senioridade; identidade formada pelo tempo dedicado a um determinado trabalho ou atividade}
  \end{Phonetics}
\end{Entry}

\begin{Entry}{资源}{10,13}{⾙、⽔}
  \begin{Phonetics}{资源}{zi1yuan2}[][HSK 4]
    \definition{s.}{recurso; fontes naturais de meios de produção ou subsistência}
  \end{Phonetics}
\end{Entry}

%%%%%%%%%% 赅 %%%%%%%%%%
\subsection*{赅}

\begin{Entry}{赅}{10}{⾙}
  \begin{Phonetics}{赅}{gai1}
    \definition*{s.}{Sobrenome: Gai}
    \definition{adj.}{completo; integral; abrangente; inclusivo}
  \end{Phonetics}
\end{Entry}

%%%%%%%%%% 赶 %%%%%%%%%%
\subsection*{赶}

\begin{Entry}{赶}{10}{⾛}
  \begin{Phonetics}{赶}{gan3}[][HSK 3]
    \definition*{s.}{Sobrenome: Gan}
    \definition{prep.}{por; até; até que; até quando; introduzir o momento em que algo aconteceu, indicando que se espera até um determinado momento}
    \definition{v.}{ultrapassar; alcançar | perseguir; correr atrás; tentar alcançar; dar uma corrida; acelerar ou intensificar  | dirigir; conduzir | expulsar; afugentar; afastar | encontrar; deparar-se com; esbarrar em; acontecer; encontrar-se em (uma situação); aproveitar-se de (uma oportunidade) | ir para; participar (atividades com horário marcado)}
  \end{Phonetics}
\end{Entry}

\begin{Entry}{赶上}{10,3}{⾛、⼀}
  \begin{Phonetics}{赶上}{gan3 shang4}[][HSK 6]
    \definition{v.}{alcançar; manter o ritmo com; acompanhar alguém ou o padrão do planejador | chegar a tempo para; ter tempo suficiente; não ser tarde demais | encontrar; topar com; cruzar com; encontrar-se com; acontecer de encontrar; encontrar algo, em um determinado momento ou oportunidade}
  \end{Phonetics}
\end{Entry}

\begin{Entry}{赶不上}{10,4,3}{⾛、⼀、⼀}
  \begin{Phonetics}{赶不上}{gan3 bu5 shang4}[][HSK 6]
    \definition{v.}{ficar para trás; ser incapaz de alcançar; não conseguir alcançar; não conseguir acompanhar | ser tarde demais (para fazer algo); (não) existir tempo suficiente (para fazer algo) |  deixar de ter; ser incapaz de encontrar ou ter a chance de encontrar; não encontrar; não encontrar (boa oportunidade) | não poder ser comparado a}
  \end{Phonetics}
\end{Entry}

\begin{Entry}{赶忙}{10,6}{⾛、⼼}
  \begin{Phonetics}{赶忙}{gan3 mang2}[][HSK 6]
    \definition{adv.}{imediatamente; com pressa; às pressas; rapidamente}
  \end{Phonetics}
\end{Entry}

\begin{Entry}{赶早}{10,6}{⾛、⽇}
  \begin{Phonetics}{赶早}{gan3zao3}
    \definition{adv.}{o mais breve possível | na primeira oportunidade | antes que seja tarde | quanto antes melhor}
  \end{Phonetics}
\end{Entry}

\begin{Entry}{赶快}{10,7}{⾛、⼼}
  \begin{Phonetics}{赶快}{gan3kuai4}[][HSK 3]
    \definition{adv.}{rapidamente; imediatamente; aproveite o momento e acelere o ritmo}
  \end{Phonetics}
\end{Entry}

\begin{Entry}{赶走}{10,7}{⾛、⾛}
  \begin{Phonetics}{赶走}{gan3zou3}
    \definition{v.}{expulsar | voltar atrás}
  \end{Phonetics}
\end{Entry}

\begin{Entry}{赶到}{10,8}{⾛、⼑}
  \begin{Phonetics}{赶到}{gan3 dao4}[][HSK 3]
    \definition{v.}{correr (para algum lugar); apressar-se}
  \end{Phonetics}
\end{Entry}

\begin{Entry}{赶往}{10,8}{⾛、⼻}
  \begin{Phonetics}{赶往}{gan3wang3}[][HSK 7-9]
    \definition{v.}{apressar-se para (algum lugar)}
  \end{Phonetics}
\end{Entry}

\begin{Entry}{赶赴}{10,9}{⾛、⾛}
  \begin{Phonetics}{赶赴}{gan3fu4}[][HSK 7-9]
    \definition{v.}{apressar-se para; correr para | apressar-se}
  \end{Phonetics}
\end{Entry}

\begin{Entry}{赶紧}{10,10}{⾛、⽷}
  \begin{Phonetics}{赶紧}{gan3jin3}[][HSK 3]
    \definition{adv.}{apressadamente; precipitadamente; às pressas; significa agir imediatamente, sem demora}
  \end{Phonetics}
\end{Entry}

\begin{Entry}{赶脚}{10,11}{⾛、⾁}
  \begin{Phonetics}{赶脚}{gan3jiao3}
    \definition{v.}{transportar mercadorias para ganhar a vida (especialmente de burro) | trabalhar como carroceiro ou porteiro}
  \end{Phonetics}
\end{Entry}

\begin{Entry}{赶跑}{10,12}{⾛、⾜}
  \begin{Phonetics}{赶跑}{gan3pao3}
    \definition{v.}{afastar | forçar a saída | repelir}
  \end{Phonetics}
\end{Entry}

\begin{Entry}{赶集}{10,12}{⾛、⾫}
  \begin{Phonetics}{赶集}{gan3ji2}
    \definition{v.}{ir a uma feira | ir ao mercado}
  \end{Phonetics}
\end{Entry}

\begin{Entry}{赶路}{10,13}{⾛、⾜}
  \begin{Phonetics}{赶路}{gan3lu4}
    \definition{v.}{apressar a jornada | apressar-se}
  \end{Phonetics}
\end{Entry}

%%%%%%%%%% 起 %%%%%%%%%%
\subsection*{起}

\begin{Entry}{起}{10}{⾛}
  \begin{Phonetics}{起}{qi3}[][HSK 1]
    \definition{clas.}{caso; instância | lote; grupo}
    \definition{prep.}{de; colocado antes de uma palavra de tempo ou lugar, indica um ponto de partida | por; colocado antes de uma palavra de lugar, indica um lugar por onde passou}
    \definition{v.}{levantar-se; ficar de pé| iniciar; lançar; deicar a posição original | subir; ascender | aparecer; levantar; crescer (bolhas, protuberâncias, brotoeja) | puxar para cima; puxar para fora; tirar o que está guardado ou incorporado | crescer; aumentar | esboçar; elaborar | construir; montar; estabelecer | receber (comprovante) | começar; iniciar; combina com 从 e 由; indica quando, onde e quem começou | buscar; pegar; usado após um verbo, indica movimento para cima | indicar se alguém tem força suficiente ou não; usado após um verbo, indica que a força é suficiente ou insuficiente | indicar que a ação envolve alguém ou algo; equivalente a 及 ou 到 | começar; iniciar; usado depois de um verbo, indica o início de uma ação | juntar; implodir; (informal) usado depois de um verbo, para unir coisas ou fechá-las}
  \seealsoref{从}{cong2}
  \seealsoref{到}{dao4}
  \seealsoref{及}{ji2}
  \seealsoref{由}{you2}
  \end{Phonetics}
\end{Entry}

\begin{Entry}{起飞}{10,3}{⾛、⾶}
  \begin{Phonetics}{起飞}{qi3fei1}[][HSK 2]
    \definition{v.}{decolar; levantar voo | crescer rapidamente; decolar; disparar; metáfora para o rápido desenvolvimento de negócios, economia, etc.}
  \end{Phonetics}
\end{Entry}

\begin{Entry}{起床}{10,7}{⾛、⼴}
  \begin{Phonetics}{起床}{qi3/chuang2}[][HSK 1]
    \definition{v.+compl.}{levantar-se; sair da cama; acordar e sair da cama (geralmente pela manhã); levantar-se da posição sentada, deitada ou deitada de bruços, ou sentar-se a partir da posição deitada}
  \end{Phonetics}
\end{Entry}

\begin{Entry}{起来}{10,7}{⾛、⽊}
  \begin{Phonetics}{起来}{qi3/lai2}[][HSK 1]
    \definition{v.+compl.}{levantar-se; passar de posições como deitado, sentado ou ajoelhado para ficar em pé | levantar-se; sair da cama | levantar-se; revoltar-se; rebelar-se; refere-se a ascensão, surgimento, levantamento, etc.}
  \end{Phonetics}
  \begin{Phonetics}{起来}{qi3lai5}
    \definition{v.aux.}{usado depois de um verbo para indicar movimento ascendente}
  \end{Phonetics}
  \begin{Phonetics}{起来}{qi5lai2}
    \definition{v.}{descrever resultados, retratar comportamentos, transmitir movimento}
  \end{Phonetics}
\end{Entry}

\begin{Entry}{起诉}{10,7}{⾛、⾔}
  \begin{Phonetics}{起诉}{qi3 su4}[][HSK 6]
    \definition{v.}{processar; entrar com uma ação judicial}
  \end{Phonetics}
\end{Entry}

\begin{Entry}{起到}{10,8}{⾛、⼑}
  \begin{Phonetics}{起到}{qi3 dao4}[][HSK 5]
    \definition{v.}{ter (um efeito motivador, etc.); desempenhar (um papel estabilizador, etc.)}
  \end{Phonetics}
\end{Entry}

\begin{Entry}{起码}{10,8}{⾛、⽯}
  \begin{Phonetics}{起码}{qi3ma3}[][HSK 5]
    \definition{adj.}{mínimo; elementar; rudimentar}
    \definition{adv.}{mínimamente; pelo menos;}
  \end{Phonetics}
\end{Entry}

\begin{Entry}{起点}{10,9}{⾛、⽕}
  \begin{Phonetics}{起点}{qi3 dian3}[][HSK 6]
    \definition[个]{s.}{ponto de partida (para o tempo ou local do início de algo); o lugar ou hora de início | ponto de partida (para o nível ou base de algo feito inicialmente); refere-se especificamente ao ponto de partida designado em um evento de pista}
  \end{Phonetics}
\end{Entry}

\begin{Entry}{起跳}{10,13}{⾛、⾜}
  \begin{Phonetics}{起跳}{qi3tiao4}
    \definition{v.}{(atletismo) decolar (no início de um salto) | (de preço, salário, etc.) começar (de um determinado nível)}
  \end{Phonetics}
\end{Entry}

%%%%%%%%%% 轿 %%%%%%%%%%
\subsection*{轿}

\begin{Entry}{轿}{10}{⾞}
  \begin{Phonetics}{轿}{jiao4}
    \definition{s.}{liteira; palanquim; cadeira de arruar}
  \end{Phonetics}
\end{Entry}

\begin{Entry}{轿车}{10,4}{⾞、⾞}
  \begin{Phonetics}{轿车}{jiao4che1}[][HSK 7-9]
    \definition[辆]{s.}{carruagem (puxada por cavalos); carruagens puxadas por animais com cortinas cobrindo os compartimentos de passageiros antigamente | ônibus; carro; sedã; um carro relativamente luxuoso e confortável, com teto e assentos para passageiros}
  \end{Phonetics}
\end{Entry}

%%%%%%%%%% 较 %%%%%%%%%%
\subsection*{较}

\begin{Entry}{较}{10}{⾞}
  \begin{Phonetics}{较}{jiao4}[][HSK 3]
    \definition{adj.}{claro; óbvio; evidente}
    \definition{adv.}{comparativamente; relativamente; razoavelmente; bastante; bastante}
    \definition{prep.}{usado para comparar características e graus; introduzir o objeto de comparação; equivalente a 比}
    \definition{v.}{comparar | disputar}
  \seealsoref{比}{bi3}
  \end{Phonetics}
\end{Entry}

\begin{Entry}{较劲}{10,7}{⾞、⼒}
  \begin{Phonetics}{较劲}{jiao4/jin4}[][HSK 7-9]
    \definition{v.+compl.}{exigir esforço extra | adequar a própria força a (competição de força; disputa de habilidade) | colocar-se contra alguém; ir um contra o outro}
  \end{Phonetics}
\end{Entry}

\begin{Entry}{较量}{10,12}{⾞、⾥}
  \begin{Phonetics}{较量}{jiao4liang4}[][HSK 7-9]
    \definition{v.}{realizar uma competição; medir a própria força com; determinar quem é superior ou inferior através da competição, da luta ou de outros meios | regatear; discutir; disputar; calcular}
  \end{Phonetics}
\end{Entry}

%%%%%%%%%% 辱 %%%%%%%%%%
\subsection*{辱}

\begin{Entry}{辱}{10}{⾠}
  \begin{Phonetics}{辱}{ru3}
    \definition*{s.}{Sobrenome: Ru}
    \definition{s.}{desgraça; desonra (oposto a 荣)}
    \definition{v.}{trazer desgraça (ou humilhação) para | trazer desgraça; ser uma desgraça para | estar em dívida (com alguém por uma gentileza) | humilhar; insultar}
  \seealsoref{荣}{rong2}
  \end{Phonetics}
\end{Entry}

\begin{Entry}{辱骂}{10,9}{⾠、⾺}
  \begin{Phonetics}{辱骂}{ru3ma4}
    \definition{v.}{insultar | abusar}
  \end{Phonetics}
\end{Entry}

%%%%%%%%%% 透 %%%%%%%%%%
\subsection*{透}

\begin{Entry}{透}{10}{⾡}
  \begin{Phonetics}{透}{tou4}[][HSK 4]
    \definition{adv.}{totalmente; completamente; minuciosamente | profundamente; extremamente}
    \definition{v.}{penetrar; passar através de; infiltrar-se através de | revelar; deixar transparecer; contar secretamente |mostrar; aparecer}
  \end{Phonetics}
\end{Entry}

\begin{Entry}{透支}{10,4}{⾡、⽀}
  \begin{Phonetics}{透支}{tou4zhi1}
    \definition{v.}{cheque especial (bancário) | saque a descoberto}
  \end{Phonetics}
\end{Entry}

\begin{Entry}{透气}{10,4}{⾡、⽓}
  \begin{Phonetics}{透气}{tou4qi4}
    \definition{v.}{respirar (sobre tecido, etc.) | fluir livremente (sobre ar) | respirar ar fresco | ventilar}
  \end{Phonetics}
\end{Entry}

\begin{Entry}{透水}{10,4}{⾡、⽔}
  \begin{Phonetics}{透水}{tou4shui3}
    \definition{adj.}{permeável}
    \definition{s.}{vazamento de água}
  \end{Phonetics}
\end{Entry}

\begin{Entry}{透过}{10,6}{⾡、⾡}
  \begin{Phonetics}{透过}{tou4guo4}
    \definition{v.}{passar através | penetrar}
  \end{Phonetics}
\end{Entry}

\begin{Entry}{透彻}{10,7}{⾡、⼻}
  \begin{Phonetics}{透彻}{tou4che4}
    \definition{adj.}{minucioso | incisivo | penetrante}
  \end{Phonetics}
\end{Entry}

\begin{Entry}{透明}{10,8}{⾡、⽇}
  \begin{Phonetics}{透明}{tou4ming2}[][HSK 4]
    \definition{adj.}{transparente; diáfano; capaz de transmitir luz | evidente; transparente; situação ou assunto que seja aberto e não oculto | transparente; diáfano; indica pureza, ausência de impurezas}
  \end{Phonetics}
\end{Entry}

\begin{Entry}{透顶}{10,8}{⾡、⾴}
  \begin{Phonetics}{透顶}{tou4ding3}
    \definition{adv.}{completamente}
  \end{Phonetics}
\end{Entry}

\begin{Entry}{透亮}{10,9}{⾡、⼇}
  \begin{Phonetics}{透亮}{tou4liang4}
    \definition{adj.}{brilhante | claro como cristal}
  \end{Phonetics}
\end{Entry}

\begin{Entry}{透辟}{10,13}{⾡、⾟}
  \begin{Phonetics}{透辟}{tou4pi4}
    \definition{adj.}{incisivo | penetrante}
  \end{Phonetics}
\end{Entry}

\begin{Entry}{透澈}{10,15}{⾡、⽔}
  \begin{Phonetics}{透澈}{tou4che4}
    \variantof{透彻}
  \end{Phonetics}
\end{Entry}

\begin{Entry}{透露}{10,21}{⾡、⾬}
  \begin{Phonetics}{透露}{tou4lu4}[][HSK 6]
    \definition{v.}{vazar; revelar; expor; divulgar; contar deliberadamente um segredo a alguém; revelar um certo significado}
  \end{Phonetics}
\end{Entry}

%%%%%%%%%% 逐 %%%%%%%%%%
\subsection*{逐}

\begin{Entry}{逐}{10}{⾡}
  \begin{Phonetics}{逐}{zhu2}
    \definition{prep.}{um por um; um a um}[逐月===mês a mês]
    \definition{v.}{ir atrás de; perseguir | expulsar; banir | correr atrás; alcançar}
  \end{Phonetics}
\end{Entry}

\begin{Entry}{逐步}{10,7}{⾡、⽌}
  \begin{Phonetics}{逐步}{zhu2bu4}[][HSK 4]
    \definition{adv.}{gradualmente; passo a passo; progressivamente}
  \end{Phonetics}
\end{Entry}

\begin{Entry}{逐渐}{10,11}{⾡、⽔}
  \begin{Phonetics}{逐渐}{zhu2jian4}[][HSK 4]
    \definition{adv.}{gradualmente; aos poucos; por etapas; indica mudanças lentas e ordenadas no grau, na quantidade, etc.}
  \end{Phonetics}
\end{Entry}

%%%%%%%%%% 递 %%%%%%%%%%
\subsection*{递}

\begin{Entry}{递}{10}{⾡}
  \begin{Phonetics}{递}{di4}[][HSK 5]
    \definition{adv.}{na ordem correta; sucessivamente}
    \definition{v.}{entregar; passar; dar; transmitir}
  \end{Phonetics}
\end{Entry}

\begin{Entry}{递交}{10,6}{⾡、⼇}
  \begin{Phonetics}{递交}{di4jiao1}[][HSK 7-9]
    \definition{v.}{apresentar; submeter; entregar; entregar pessoalmente}
  \end{Phonetics}
\end{Entry}

\begin{Entry}{递给}{10,9}{⾡、⽷}
  \begin{Phonetics}{递给}{di4 gei3}[][HSK 5]
    \definition{v.}{entregar algo a alguém; passar itens ou coisas para outras pessoas}
  \end{Phonetics}
\end{Entry}

%%%%%%%%%% 途 %%%%%%%%%%
\subsection*{途}

\begin{Entry}{途}{10}{⾡}
  \begin{Phonetics}{途}{tu2}
    \definition[条]{s.}{caminho; estrada; rota | jornada; caminho}
  \end{Phonetics}
\end{Entry}

\begin{Entry}{途中}{10,4}{⾡、⼁}
  \begin{Phonetics}{途中}{tu2 zhong1}[][HSK 4]
    \definition[家]{adv.}{no caminho; ao longo do caminho}
  \end{Phonetics}
\end{Entry}

\begin{Entry}{途径}{10,8}{⾡、⼻}
  \begin{Phonetics}{途径}{tu2jing4}[][HSK 6]
    \definition[种,条,个]{s.}{caminho; canal; metaforicamente falando, uma maneira ou método de resolver um problema ou fazer algo}
  \end{Phonetics}
\end{Entry}

%%%%%%%%%% 逗 %%%%%%%%%%
\subsection*{逗}

\begin{Entry}{逗}{10}{⾡}
  \begin{Phonetics}{逗}{dou4}[][HSK 7-9]
    \definition{adj.}{engraçado; divertido}
    \definition{s.}{ligeira pausa na leitura; antigamente, referia-se ao lugar em um artigo onde o significado de uma frase não era completado e uma pausa era necessária durante a leitura}
    \definition{v.}{provocar; brincar com | divertir; provocar (risos, etc.) | ficar; parar}
  \end{Phonetics}
\end{Entry}

%%%%%%%%%% 通 %%%%%%%%%%
\subsection*{通}

\begin{Entry}{通}{10}{⾡}
  \begin{Phonetics}{通}{tong1}[][HSK 2]
    \definition*{s.}{Sobrenome: Tong}
    \definition{adj.}{lógico; coerente | geral; comum | tudo; inteiro | aberto; através de | total}
    \definition{clas.}{(antigo) usado para cartas, telegramas, documentos oficiais, etc.}
    \definition{s.}{autoridade; especialista}
    \definition{suf.}{especialista}
    \definition{v.}{abrir; atravessar | abrir ou limpar cutucando ou espetando | levar a; ir a | conectar; comunicar | notificar; informar | compreender; saber | cutucar; dar uma pancada | transmitir; conectar; interagir | dominar; compreender; entender}
  \end{Phonetics}
  \begin{Phonetics}{通}{tong4}
    \definition{clas.}{usado para uma atividade, tomada em sua totalidade (discurso de abuso, período de reprodução de música, bebedeira, etc.)}
  \end{Phonetics}
\end{Entry}

\begin{Entry}{通用}{10,5}{⾡、⽤}
  \begin{Phonetics}{通用}{tong1yong4}[][HSK 5]
    \definition[家]{adj.}{de uso comum; universal; (em um determinado âmbito) de uso generalizado | intercambiável; alguns caracteres chineses com grafia diferente, mas pronúncia igual, podem ser usados indistintamente (alguns limitados a um determinado significado)}
  \end{Phonetics}
\end{Entry}

\begin{Entry}{通讯}{10,5}{⾡、⾔}
  \begin{Phonetics}{通讯}{tong1xun4}[][HSK 6]
    \definition[个,种]{s.}{relatório; comunicação; boletim informativo; correspondência; reportagem; despacho de notícias; artigos que relatam fatos objetivos ou números típicos de forma detalhada e vívida}
    \definition{v.}{usar equipamentos de telecomunicações para transmitir mensagens}
  \end{Phonetics}
\end{Entry}

\begin{Entry}{通红}{10,6}{⾡、⽷}
  \begin{Phonetics}{通红}{tong1 hong2}[][HSK 6]
    \definition{adj.}{muito vermelho; vermelho por completo}
  \end{Phonetics}
\end{Entry}

\begin{Entry}{通行}{10,6}{⾡、⾏}
  \begin{Phonetics}{通行}{tong1 xing2}[][HSK 6]
    \definition{adj.}{atual; geral}
    \definition{v.}{passar (ou ir) através; passar por; atravessar | prevalecer; predominar; ser corrente | (pedestres, veículos, etc.) passar na linha de trânsito}
  \end{Phonetics}
\end{Entry}

\begin{Entry}{通观}{10,6}{⾡、⾒}
  \begin{Phonetics}{通观}{tong1guan1}
    \definition{v.}{ter uma visão geral de algo}
  \end{Phonetics}
\end{Entry}

\begin{Entry}{通过}{10,6}{⾡、⾡}
  \begin{Phonetics}{通过}{tong1guo4}[][HSK 2]
    \definition{prep.}{por; através de; por meio de; por meio de; meios, métodos, etc. para introduzir ações}
    \definition{v.}{atravessar; passar por; transitar | aprovar; adotar | solicitar o consentimento ou aprovação de}
  \end{Phonetics}
\end{Entry}

\begin{Entry}{通报}{10,7}{⾡、⼿}
  \begin{Phonetics}{通报}{tong1 bao4}[][HSK 6]
    \definition[份]{s.}{circular | boletim; jornal; publicação | sumário; notificação para informações gerais}
    \definition{v.}{circular um aviso (aviso por escrito) | notificar; dar informações com; compartilhar informações com}
  \end{Phonetics}
\end{Entry}

\begin{Entry}{通识}{10,7}{⾡、⾔}
  \begin{Phonetics}{通识}{tong1shi2}
    \definition{s.}{conhecimento comum | erudição | conhecimento geral | amplamente conhecido}
  \end{Phonetics}
\end{Entry}

\begin{Entry}{通知}{10,8}{⾡、⽮}
  \begin{Phonetics}{通知}{tong1zhi1}[][HSK 2]
    \definition[份,个,张]{s.}{aviso; circular; notificação por escrito ou verbal}
    \definition{v.}{aconselhar; notificar; informar; dar aviso prévio}
  \end{Phonetics}
\end{Entry}

\begin{Entry}{通知书}{10,8,4}{⾡、⽮、⼄}
  \begin{Phonetics}{通知书}{tong1 zhi1 shu1}[][HSK 4]
    \definition[份]{s.}{aviso; observação; notificação}
  \end{Phonetics}
\end{Entry}

\begin{Entry}{通话}{10,8}{⾡、⾔}
  \begin{Phonetics}{通话}{tong1 hua4}[][HSK 6]
    \definition{v.}{comunicar por telefone | conversar; comunicar; falar em uma língua que ambos possam entender}
  \end{Phonetics}
\end{Entry}

\begin{Entry}{通信}{10,9}{⾡、⼈}
  \begin{Phonetics}{通信}{tong1/xin4}[][HSK 3]
    \definition{v.+compl.}{corresponder; comunicar por carta; comunicar situações e informações escrevendo cartas | transmitir (ou transportar) mensagem; passar (ou transmitir) informação; usar ondas de rádio e outros sinais para transmitir texto, imagens, etc.}
  \end{Phonetics}
\end{Entry}

\begin{Entry}{通常}{10,11}{⾡、⼱}
  \begin{Phonetics}{通常}{tong1chang2}[][HSK 3]
    \definition{adj.}{usual; normal; geral}
    \definition{adv.}{habitualmente; usualmente; geralmente; ordinariamente}
  \end{Phonetics}
\end{Entry}

\begin{Entry}{通道}{10,12}{⾡、⾡}
  \begin{Phonetics}{通道}{tong1 dao4}[][HSK 6]
    \definition[条,个]{s.}{acesso; corredor; passagem; caminhos que levam ao exterior de teatros, minas, etc. | passagem; via pública}
  \end{Phonetics}
\end{Entry}

\begin{Entry}{通牒}{10,13}{⾡、⽚}
  \begin{Phonetics}{通牒}{tong1die2}
    \definition{s.}{nota diplomática}
  \end{Phonetics}
\end{Entry}

%%%%%%%%%% 逛 %%%%%%%%%%
\subsection*{逛}

\begin{Entry}{逛}{10}{⾡}
  \begin{Phonetics}{逛}{guang4}[][HSK 4]
    \definition{v.}{perambular; passear; vaguear}
  \end{Phonetics}
\end{Entry}

%%%%%%%%%% 逞 %%%%%%%%%%
\subsection*{逞}

\begin{Entry}{逞}{10}{⾡}
  \begin{Phonetics}{逞}{cheng3}
    \definition*{s.}{Sobrenome: Cheng}
    \definition{v.}{exibir-se; ostentar; gabar-se | executar (um plano maligno); ter sucesso (em um esquema) | saciar; satisfazer; dar rédea solta a; deliciar-se}
  \end{Phonetics}
\end{Entry}

\begin{Entry}{逞能}{10,10}{⾡、⾁}
  \begin{Phonetics}{逞能}{cheng3/neng2}[][HSK 7-9]
    \definition{v.+compl.}{exibir a própria habilidade (ou capacidade); exibir a própria capacidade | mostrar sua habilidade ou capacidade}
  \end{Phonetics}
\end{Entry}

\begin{Entry}{逞强}{10,12}{⾡、⼸}
  \begin{Phonetics}{逞强}{cheng3/qiang2}[][HSK 7-9]
    \definition{v.+compl.}{exibir-se; ser orgulhoso; ser teimoso; ostentar a própria superioridade}
  \end{Phonetics}
\end{Entry}

%%%%%%%%%% 速 %%%%%%%%%%
\subsection*{速}

\begin{Entry}{速}{10}{⾡}
  \begin{Phonetics}{速}{su4}
    \definition{adj.}{rápido; veloz}
    \definition{s.}{velocidade}
    \definition{v.aux.}{convidar}
  \end{Phonetics}
\end{Entry}

\begin{Entry}{速度}{10,9}{⾡、⼴}
  \begin{Phonetics}{速度}{su4du4}[][HSK 3]
    \definition[个,种]{s.}{velocidade; taxa; ritmo; andamento; uma quantidade física que indica a velocidade e a direção do movimento de um objeto, ou seja, a distância que um objeto percorre em uma direção por unidade de tempo | velocidade; rapidez; geralmente se refere ao grau de velocidade}
  \end{Phonetics}
\end{Entry}

%%%%%%%%%% 造 %%%%%%%%%%
\subsection*{造}

\begin{Entry}{造}{10}{⾡}
  \begin{Phonetics}{造}{zao4}[][HSK 3]
    \definition*{s.}{Sobrenome: Zao}
    \definition{clas.}{para colheitas ou número de colheitas de safras}
    \definition{s.}{uma das duas partes em um acordo legal ou um processo judicial | (dialeto) colheita; safra | realizações; conquistas}
    \definition{v.}{fazer; construir; criar; produzir | forjar; inventar | correr solto; bagunçar as coisas | expor sem restrições |  treinar; educar | fabricar | alcançar; atingir}
  \end{Phonetics}
\end{Entry}

\begin{Entry}{造成}{10,6}{⾡、⼽}
  \begin{Phonetics}{造成}{zao4cheng2}[][HSK 3]
    \definition{v.}{criar; dar origem a; provocar; causar (geralmente se refere a resultados negativos)}
  \end{Phonetics}
\end{Entry}

\begin{Entry}{造型}{10,9}{⾡、⼟}
  \begin{Phonetics}{造型}{zao4xing2}[][HSK 4]
    \definition[个,种]{s.}{molde; modelo; formato; forma; moldagem}
    \definition{v.}{modelar; moldar}
  \end{Phonetics}
\end{Entry}

%%%%%%%%%% 逢 %%%%%%%%%%
\subsection*{逢}

\begin{Entry}{逢}{10}{⾡}
  \begin{Phonetics}{逢}{feng2}[][HSK 7-9]
    \definition*{s.}{Sobrenome: Feng}
    \definition{v.}{encontrar; vir até; encontrar-se por acaso}
  \end{Phonetics}
\end{Entry}

%%%%%%%%%% 部 %%%%%%%%%%
\subsection*{部}

\begin{Entry}{部}{10}{⾢}
  \begin{Phonetics}{部}{bu4}[][HSK 3]
    \definition*{s.}{Sobrenome: Bu}
    \definition{clas.}{usado para obras de literatura, livros, filmes, etc.}
    \definition[根]{s.}{parte; seção | unidade; ministério; departamento; conselho | sede; matriz; quartel general | tropas; forças | divisão; região}
    \definition{v.}{comandar; liderar}
  \end{Phonetics}
\end{Entry}

\begin{Entry}{部下}{10,3}{⾢、⼀}
  \begin{Phonetics}{部下}{bu4xia4}
    \definition{s.}{subordinado | tropas sob comando de alguém}
  \end{Phonetics}
\end{Entry}

\begin{Entry}{部门}{10,3}{⾢、⾨}
  \begin{Phonetics}{部门}{bu4men2}[][HSK 3]
    \definition[个]{s.}{departamento; ramo; classe; seção; partes ou unidades que compõem um todo}
  \end{Phonetics}
\end{Entry}

\begin{Entry}{部分}{10,4}{⾢、⼑}
  \begin{Phonetics}{部分}{bu4fen5}[][HSK 2]
    \definition[个,些,快,份]{s.}{parte; seção; porção; parte do todo; alguns indivíduos dentro do todo | ramo; parte separada de um sistema ou entidade}
  \end{Phonetics}
\end{Entry}

\begin{Entry}{部长}{10,4}{⾢、⾧}
  \begin{Phonetics}{部长}{bu4 zhang3}[][HSK 3]
    \definition[个,位,名]{s.}{ministro; chefe de departamento; um alto funcionário do estado encarregado pelo chefe de estado ou chefe executivo do governo da gestão das atividades governamentais de um departamento | chefe de seção; líder tribal}
  \end{Phonetics}
\end{Entry}

\begin{Entry}{部队}{10,4}{⾢、⾩}
  \begin{Phonetics}{部队}{bu4 dui4}[][HSK 6]
    \definition[支,个]{s.}{militar; exército; forças armadas | tropas; refere-se a uma parte do exército}
  \end{Phonetics}
\end{Entry}

\begin{Entry}{部件}{10,6}{⾢、⼈}
  \begin{Phonetics}{部件}{bu4jian4}[][HSK 7-9]
    \definition[个]{s.}{peças; partes; componentes; um componente de uma máquina, montado a partir de várias partes | partes; componentes (para caracteres chineses); uma unidade de caracteres chineses composta por traços, por exemplo, 氵, 礻, 口 são todos componentes de caracteres chineses}
  \end{Phonetics}
\end{Entry}

\begin{Entry}{部位}{10,7}{⾢、⼈}
  \begin{Phonetics}{部位}{bu4wei4}[][HSK 5]
    \definition{s.}{lugar; posição (usado principalmente para o corpo humano)}
  \end{Phonetics}
\end{Entry}

\begin{Entry}{部族}{10,11}{⾢、⽅}
  \begin{Phonetics}{部族}{bu4zu2}
    \definition{adj.}{tribal}
    \definition{s.}{tribo}
  \end{Phonetics}
\end{Entry}

\begin{Entry}{部属}{10,12}{⾢、⼫}
  \begin{Phonetics}{部属}{bu4shu3}
    \definition{s.}{afiliado a um ministério | subordinado | tropas sob comando de alguém}
  \end{Phonetics}
\end{Entry}

\begin{Entry}{部署}{10,13}{⾢、⽹}
  \begin{Phonetics}{部署}{bu4shu3}[][HSK 7-9]
    \definition{v.}{organizar; implantar; dispor; organizar ou dispor de maneira planejada (usado principalmente em grandes aspectos)}
  \end{Phonetics}
\end{Entry}

%%%%%%%%%% 都 %%%%%%%%%%
\subsection*{都}

\begin{Entry}{都}{10}{⾢}
  \begin{Phonetics}{都}{dou1}[][HSK 1]
    \definition{adv.}{todos; representa a soma total | apenas por causa de; usado em conjunto com a palavra 是, explica o motivo | mesmo; até; indicativo de ênfase | já; significa 已经}
  \seealsoref{是}{shi4}
  \seealsoref{已经}{yi3jing1}
  \end{Phonetics}
  \begin{Phonetics}{都}{du1}
    \definition*{s.}{Sobrenome: Du}
    \definition[座]{s.}{capital | cidade grande; metrópole}
  \end{Phonetics}
\end{Entry}

\begin{Entry}{都市}{10,5}{⾢、⼱}
  \begin{Phonetics}{都市}{du1 shi4}[][HSK 6]
    \definition[个]{s.}{cidade grande; grandes cidades}
  \end{Phonetics}
\end{Entry}

\begin{Entry}{都会}{10,6}{⾢、⼈}
  \begin{Phonetics}{都会}{du1hui4}[][HSK 7-9]
    \definition{s.}{cidade; metrópole}
  \end{Phonetics}
\end{Entry}

%%%%%%%%%% 配 %%%%%%%%%%
\subsection*{配}

\begin{Entry}{配}{10}{⾣}
  \begin{Phonetics}{配}{pei4}[][HSK 3]
    \definition{adj.}{adequado; bem combinado}
    \definition{s.}{cônjuge (geralmente referindo-se a uma esposa)}
    \definition{v.}{unir-se em matrimônio | (animais) acasalar; copular | compor; combinar; mesclar; amalgamar; misturar | distribuir de forma planejada; repartir | encontrar algo para encaixar ou substituir outra coisa; compensar as partes faltantes de acordo com certos padrões | combinar; harmonizar com; estar em harmonia com | exilar; banir; nos tempos antigos, referia-se ao exílio de criminosos}
    \definition{v.aux.}{adequar-se a; merecer; ser qualificado; ser digno de}
  \end{Phonetics}
\end{Entry}

\begin{Entry}{配合}{10,6}{⾣、⼝}
  \begin{Phonetics}{配合}{pei4he2}[][HSK 3]
    \definition{v.}{cooperar; coordenar; todas as partes trabalham juntas para concluir tarefas comuns}
  \end{Phonetics}
\end{Entry}

\begin{Entry}{配备}{10,8}{⾣、⼡}
  \begin{Phonetics}{配备}{pei4bei4}[][HSK 5]
    \definition{s.}{equipamento; material; conjunto completo de utensílios, etc.}
    \definition{v.}{fornecer; alocar; equipar; distribuir conforme necessário | posicionar; dispor (tropas, etc.)}
  \end{Phonetics}
\end{Entry}

\begin{Entry}{配套}{10,10}{⾣、⼤}
  \begin{Phonetics}{配套}{pei4/tao4}[][HSK 5]
    \definition{v.+compl.}{formar um conjunto ou sistema completo; combinar vários elementos relacionados em um conjunto completo}
  \end{Phonetics}
\end{Entry}

\begin{Entry}{配置}{10,13}{⾣、⽹}
  \begin{Phonetics}{配置}{pei4 zhi4}[][HSK 6]
    \definition{s.}{configuração; refere-se especificamente à seleção e combinação de software e hardware em várias partes de computadores, carros, etc.}
    \definition{v.}{implantar; alocar; dispor (tropas, etc.); equipar e configurar}
  \end{Phonetics}
\end{Entry}

%%%%%%%%%% 酒 %%%%%%%%%%
\subsection*{酒}

\begin{Entry}{酒}{10}{⾣}
  \begin{Phonetics}{酒}{jiu3}[][HSK 2]
    \definition*{s.}{Sobrenome: Jiu}
    \definition[口,杯,瓶,罐,桶,缸]{s.}{bebida alcoólica; vinho; licor; bebidas destiladas}
  \end{Phonetics}
\end{Entry}

\begin{Entry}{酒水}{10,4}{⾣、⽔}
  \begin{Phonetics}{酒水}{jiu3 shui3}[][HSK 6]
    \definition{s.}{bebidas; bebidas e álcool | Dialeto: festa; banquete}
  \end{Phonetics}
\end{Entry}

\begin{Entry}{酒吧}{10,7}{⾣、⼝}
  \begin{Phonetics}{酒吧}{jiu3ba1}[][HSK 4]
    \definition[家,间]{s.}{bar; \emph{pub}; um local onde são vendidas bebidas alcoólicas e onde as pessoas podem beber e conversar, referindo-se principalmente a um restaurante ou hotel de estilo ocidental especializado na venda de bebidas alcoólicas.}
  \end{Phonetics}
\end{Entry}

\begin{Entry}{酒店}{10,8}{⾣、⼴}
  \begin{Phonetics}{酒店}{jiu3 dian4}[][HSK 2]
    \definition[家,个]{s.}{hotel; Estabelecimento comercial que oferece hospedagem e alimentação aos hóspedes | restaurante}
  \end{Phonetics}
\end{Entry}

\begin{Entry}{酒鬼}{10,9}{⾣、⿁}
  \begin{Phonetics}{酒鬼}{jiu3gui3}[][HSK 5]
    \definition[个]{s.}{bebedor de vinho; beberrão; ébrio | alcoólatra}
  \end{Phonetics}
\end{Entry}

\begin{Entry}{酒馆}{10,11}{⾣、⾷}
  \begin{Phonetics}{酒馆}{jiu3guan3}
    \definition{s.}{bar | taverna | adega}
  \end{Phonetics}
\end{Entry}

\begin{Entry}{酒楼}{10,13}{⾣、⽊}
  \begin{Phonetics}{酒楼}{jiu3lou2}[][HSK 7-9]
    \definition[座,家]{s.}{restaurante (em nomes de restaurantes)}[广东酒楼===Restaurante Guangdong]
  \end{Phonetics}
\end{Entry}

\begin{Entry}{酒精}{10,14}{⾣、⽶}
  \begin{Phonetics}{酒精}{jiu3jing1}[][HSK 7-9]
    \definition{s.}{álcool; álcool etílico; etanol}
  \end{Phonetics}
\end{Entry}

%%%%%%%%%% 钱 %%%%%%%%%%
\subsection*{钱}

\begin{Entry}{钱}{10}{⾦}
  \begin{Phonetics}{钱}{qian2}[][HSK 1]
    \definition*{s.}{Sobrenome: Qian}
    \definition{clas.}{qian, uma unidade de peso (=5 gramas) | qian, uma unidade de peso (um décimo de um tael 两)}
    \definition[笔]{s.}{dinheiro; riqueza; bens | moeda de cobre; dinheiro | objeto em forma de moeda de cobre | fundo; montante | dinheiro guardado ou gasto para algum fim específico (geralmente se refere a quantias significativas de dinheiro que entram e saem de órgãos públicos, organizações, etc.)}
  \seealsoref{两}{liang3}
  \end{Phonetics}
\end{Entry}

\begin{Entry}{钱包}{10,5}{⾦、⼓}
  \begin{Phonetics}{钱包}{qian2 bao1}[][HSK 1]
    \definition[个]{s.}{carteira; bolsa; bolsa de dinheiro}
  \end{Phonetics}
\end{Entry}

%%%%%%%%%% 钻 %%%%%%%%%%
\subsection*{钻}

\begin{Entry}{钻}{10}{⾦}
  \begin{Phonetics}{钻}{zuan1}
    \definition{v.}{furar; perfurar; girar um objeto pontiagudo para perfurar outro objeto | perfurar; entrar; penetrar; passar por | aprofundar-se; estudar intensivamente; fazer um estudo penetrante de | buscar ganho pessoal; tramar; refere-se a esquemas}
  \end{Phonetics}
  \begin{Phonetics}{钻}{zuan4}[][HSK 6]
    \definition[把]{s.}{broca; pua; sonda; existem muitos tipos de ferramentas para perfuração, incluindo manivela, elétrica e pneumática | joia; diamante}
    \definition{v.}{furar; perfurar;  girar um objeto pontiagudo para perfurar outro objeto}
  \end{Phonetics}
\end{Entry}

\begin{Entry}{钻石}{10,5}{⾦、⽯}
  \begin{Phonetics}{钻石}{zuan4shi2}
    \definition[颗]{s.}{diamante}
  \end{Phonetics}
\end{Entry}

\begin{Entry}{钻戒}{10,7}{⾦、⼽}
  \begin{Phonetics}{钻戒}{zuan4jie4}
    \definition[枚]{s.}{anel de diamante}
  \end{Phonetics}
\end{Entry}

%%%%%%%%%% 钿 %%%%%%%%%%
\subsection*{钿}

\begin{Entry}{钿}{10}{⾦}
  \begin{Phonetics}{钿}{dian4}
    \definition{s.}{ornamento incrustado antigo em forma de flor | enfeite de cabelo feminino com flores douradas | incrustação de madrepérola; um padrão incrustado com conchas de caracóis em madeira e laca}
    \definition{v.}{incrustar com ouro, prata, etc.}
  \end{Phonetics}
  \begin{Phonetics}{钿}{tian2}
    \definition{s.}{(dialeto) moeda | dinheiro; moeda | uma quantia de dinheiro}
  \end{Phonetics}
\end{Entry}

%%%%%%%%%% 铁 %%%%%%%%%%
\subsection*{铁}

\begin{Entry}{铁}{10}{⾦}
  \begin{Phonetics}{铁}{tie3}[][HSK 3]
    \definition*{s.}{Sobrenome: Tie}
    \definition{adj.}{duro; forte; sólido como ferro; metáfora para natureza dura; vontade forte | violento | inabalável; inalterável; determinado; metáfora para violência ou crueldade}
    \definition{s.}{ferro (Fe) | arma; armamento; refere-se a facas, armas de fogo, etc.}
    \definition{v.}{resolver; determinar}
  \end{Phonetics}
\end{Entry}

\begin{Entry}{铁轨}{10,6}{⾦、⾞}
  \begin{Phonetics}{铁轨}{tie3gui3}
    \definition[根]{s.}{trilho | trilho ferroviário}
  \end{Phonetics}
\end{Entry}

\begin{Entry}{铁路}{10,13}{⾦、⾜}
  \begin{Phonetics}{铁路}{tie3 lu4}[][HSK 3]
    \definition[条,公里]{s.}{ferrovia; estrada de ferro; uma estrada com trilhos de aço dispostos no leito da estrada para a circulação de trens}
  \end{Phonetics}
\end{Entry}

%%%%%%%%%% 铃 %%%%%%%%%%
\subsection*{铃}

\begin{Entry}{铃}{10}{⾦}
  \begin{Phonetics}{铃}{ling2}[][HSK 5]
    \definition[串,个]{s.}{sino; instrumento musical feito de metal | objetos em forma de sino | cápsula; botão; broto}
  \end{Phonetics}
\end{Entry}

\begin{Entry}{铃声}{10,7}{⾦、⼠}
  \begin{Phonetics}{铃声}{ling2 sheng1}[][HSK 5]
    \definition{s.}{o tilintar de sinos; o som de um sino tocando}
  \end{Phonetics}
\end{Entry}

%%%%%%%%%% 铅 %%%%%%%%%%
\subsection*{铅}

\begin{Entry}{铅}{10}{⾦}
  \begin{Phonetics}{铅}{qian1}
    \definition[根,盒]{s.}{chumbo (Pb) | grafite (em um lápis); grafite preta |}
  \end{Phonetics}
\end{Entry}

\begin{Entry}{铅笔}{10,10}{⾦、⽵}
  \begin{Phonetics}{铅笔}{qian1bi3}[][HSK 6]
    \definition[支,盒,种,枝,杆]{s.}{lápis; canetas com pontas de grafite ou argila pigmentada}
  \end{Phonetics}
\end{Entry}

%%%%%%%%%% 阅 %%%%%%%%%%
\subsection*{阅}

\begin{Entry}{阅}{10}{⾨}
  \begin{Phonetics}{阅}{yue4}
    \definition{v.}{ler; repassar; examinar | revisar; inspecionar | experimentar; passar por}
  \end{Phonetics}
\end{Entry}

\begin{Entry}{阅兵式}{10,7,6}{⾨、⼋、⼷}
  \begin{Phonetics}{阅兵式}{yue4bing1shi4}
    \definition{s.}{parada militar; desfile militar}
  \end{Phonetics}
\end{Entry}

\begin{Entry}{阅览室}{10,9,9}{⾨、⾒、⼧}
  \begin{Phonetics}{阅览室}{yue4 lan3 shi4}[][HSK 5]
    \definition[个,间]{s.}{sala de leitura; a biblioteca dispõe de salas para leitura e pesquisa, equipadas com mesas e cadeiras adequadas, livros, jornais, revistas, etc.}
  \end{Phonetics}
\end{Entry}

\begin{Entry}{阅读}{10,10}{⾨、⾔}
  \begin{Phonetics}{阅读}{yue4du2}[][HSK 4]
    \definition{v.}{ler; examinar; olhar (livros, jornais, etc.) e entender seu conteúdo}
  \end{Phonetics}
\end{Entry}

\begin{Entry}{阅读广度}{10,10,3,9}{⾨、⾔、⼴、⼴}
  \begin{Phonetics}{阅读广度}{yue4du2guang3du4}
    \definition{s.}{intervalo de leitura}
  \end{Phonetics}
\end{Entry}

\begin{Entry}{阅读时间}{10,10,7,7}{⾨、⾔、⽇、⾨}
  \begin{Phonetics}{阅读时间}{yue4 du2 shi2 jian1}
    \definition{s.}{tempo de leitura}
  \end{Phonetics}
\end{Entry}

\begin{Entry}{阅读理解}{10,10,11,13}{⾨、⾔、⽟、⾓}
  \begin{Phonetics}{阅读理解}{yue4du2li3jie3}
    \definition{s.}{compreensão de leitura}
  \end{Phonetics}
\end{Entry}

\begin{Entry}{阅读装置}{10,10,12,13}{⾨、⾔、⾐、⽹}
  \begin{Phonetics}{阅读装置}{yue4du2zhuang1zhi4}
    \definition{s.}{dispositivo de leitura (por exemplo, para códigos de barras, etiquetas RFID, etc.)}
  \end{Phonetics}
\end{Entry}

\begin{Entry}{阅读障碍}{10,10,13,13}{⾨、⾔、⾩、⽯}
  \begin{Phonetics}{阅读障碍}{yue4du2zhang4ai4}
    \definition{s.}{dislexia}
  \end{Phonetics}
\end{Entry}

\begin{Entry}{阅读器}{10,10,16}{⾨、⾔、⼝}
  \begin{Phonetics}{阅读器}{yue4du2qi4}
    \definition{s.}{leitor (\emph{software})}
  \end{Phonetics}
\end{Entry}

%%%%%%%%%% 陪 %%%%%%%%%%
\subsection*{陪}

\begin{Entry}{陪}{10}{⾩}
  \begin{Phonetics}{陪}{pei2}[][HSK 5]
    \definition{v.}{servir; acompanhar; cuidar; fazer companhia a alguém | auxiliar; ajudar}
  \end{Phonetics}
\end{Entry}

\begin{Entry}{陪同}{10,6}{⾩、⼝}
  \begin{Phonetics}{陪同}{pei2 tong2}[][HSK 6]
    \definition{v.}{acompanhar; acompanhar alguém para fazer uma atividade ou trabalhar junto}
  \end{Phonetics}
\end{Entry}

%%%%%%%%%% 陵 %%%%%%%%%%
\subsection*{陵}

\begin{Entry}{陵}{10}{⾩}
  \begin{Phonetics}{陵}{ling2}
    \definition*{s.}{Sobrenome: Ling}
    \definition{s.}{colina; monte | túmulo imperial; mausoléu}
    \definition{v.}{(literário) intimidar; violar}
  \end{Phonetics}
\end{Entry}

\begin{Entry}{陵园}{10,7}{⾩、⼞}
  \begin{Phonetics}{陵园}{ling2yuan2}
    \definition{s.}{cemitério}
  \end{Phonetics}
\end{Entry}

%%%%%%%%%% 陷 %%%%%%%%%%
\subsection*{陷}

\begin{Entry}{陷}{10}{⾩}
  \begin{Phonetics}{陷}{xian4}
    \definition[个]{s.}{armadilha; cilada | defeito | deficiência; desvantagem}
    \definition{v.}{ficar preso (ou atolado); enredar | afundar; desabar | acusar falsamente; incriminar; armar | (de uma cidade, etc.) ser capturado; cair | ser enquadrado; ser capturado}
  \end{Phonetics}
\end{Entry}

\begin{Entry}{陷入}{10,2}{⾩、⼊}
  \begin{Phonetics}{陷入}{xian4ru4}[][HSK 6]
    \definition{v.}{afundar em; cair em; cair em uma situação desfavorável | estar perdido em; estar profundamente em; estar imerso em; metaforicamente, estar profundamente imerso em (uma situação ou pensamento) | estar atolado (lama fofa, areia, etc.)}
  \end{Phonetics}
\end{Entry}

%%%%%%%%%% 难 %%%%%%%%%%
\subsection*{难}

\begin{Entry}{难}{10}{⾫}
  \begin{Phonetics}{难}{nan2}[][HSK 1]
    \definition{adj.}{difícil; duro; problemático (oposto a 易) | dificilmente possível; inevitável | ruim; desagradável | problemático; improvável}
    \definition{s.}{dificuldade}
    \definition{v.}{colocar alguém em uma situação difícil}
  \seealsoref{易}{yi4}
  \end{Phonetics}
  \begin{Phonetics}{难}{nan4}
    \definition{s.}{catástrofe; calamidade; desastre; adversidade; grande infortúnio}
    \definition{v.}{acusar; culpar}
  \end{Phonetics}
\end{Entry}

\begin{Entry}{难以}{10,4}{⾫、⼈}
  \begin{Phonetics}{难以}{nan2 yi3}[][HSK 5]
    \definition{adj.}{difícil; complicado}
  \end{Phonetics}
\end{Entry}

\begin{Entry}{难过}{10,6}{⾫、⾡}
  \begin{Phonetics}{难过}{nan2guo4}[][HSK 2]
    \definition{adj.}{triste; ruim; psicologicamente desconfortável | difícil; árduo}
  \end{Phonetics}
\end{Entry}

\begin{Entry}{难免}{10,7}{⾫、⼉}
  \begin{Phonetics}{难免}{nan4mian3}[][HSK 4]
    \definition{adj.}{inevitável; difícil de evitar}
  \end{Phonetics}
\end{Entry}

\begin{Entry}{难听}{10,7}{⾫、⼝}
  \begin{Phonetics}{难听}{nan2 ting1}[][HSK 2]
    \definition{adj.}{desagradável de ouvir | ofensivo; grosseiro; vulgar e desagradável | escandaloso; indigno}
  \end{Phonetics}
\end{Entry}

\begin{Entry}{难忘}{10,7}{⾫、⼼}
  \begin{Phonetics}{难忘}{nan2 wang4}[][HSK 6]
    \definition{adj.}{memorável; inesquecível}
  \end{Phonetics}
\end{Entry}

\begin{Entry}{难受}{10,8}{⾫、⼜}
  \begin{Phonetics}{难受}{nan2shou4}[][HSK 2]
    \definition{adj.}{sentir dor; sentir-se mal; sentir-se desconfortável | sentir-se mal; sentir-se infeliz; de mau humor; triste}
  \end{Phonetics}
\end{Entry}

\begin{Entry}{难度}{10,9}{⾫、⼴}
  \begin{Phonetics}{难度}{nan2 du4}[][HSK 3]
    \definition{s.}{dificuldade; grau de dificuldade}
  \end{Phonetics}
\end{Entry}

\begin{Entry}{难看}{10,9}{⾫、⽬}
  \begin{Phonetics}{难看}{nan2 kan4}[][HSK 2]
    \definition{adj.}{feio; desagradável à vista | vergonhoso; embaraçoso; desonroso; sem glória; sem dignidade}
  \end{Phonetics}
\end{Entry}

\begin{Entry}{难得}{10,11}{⾫、⼻}
  \begin{Phonetics}{难得}{nan2de2}[][HSK 5]
    \definition{adj.}{raro; difícil de encontrar; difícil de obter ou realizar, indicando que é valioso}
    \definition{adv.}{raramente; com pouca frequência}
  \end{Phonetics}
\end{Entry}

\begin{Entry}{难道}{10,12}{⾫、⾡}
  \begin{Phonetics}{难道}{nan2dao4}[][HSK 3]
    \definition{adv.}{certamente não significa que\dots?; é possível que\dots?; não me diga\dots; poderia ser que\dots?; usado em frases interrogativas para reforçar o tom interrogativo; frequentemente usado com palavras como "吗" e "不成".}
  \seealsoref{不成}{bu4 cheng2}
  \seealsoref{吗}{ma5}
  \end{Phonetics}
\end{Entry}

\begin{Entry}{难题}{10,15}{⾫、⾴}
  \begin{Phonetics}{难题}{nan2 ti2}[][HSK 2]
    \definition[个,道]{s.}{desafio; problema difícil; questão difícil; questões difíceis de responder ou resolver}
  \end{Phonetics}
\end{Entry}

%%%%%%%%%% 顽 %%%%%%%%%%
\subsection*{顽}

\begin{Entry}{顽}{10}{⾴}
  \begin{Phonetics}{顽}{wan2}
    \definition*{s.}{Sobrenome: Wan}
    \definition{adj.}{estúpido; denso; insensível | teimoso; obstinado; não é facilmente persuadido ou subjugado | travesso; pernicioso | cabeça dura; estúpido e ignorante}
    \definition{v.}{brincar; divertir-se; divertir-se | empregar; recorrer a | envolver-se em; tomar parte em}
  \end{Phonetics}
\end{Entry}

\begin{Entry}{顽皮}{10,5}{⾴、⽪}
  \begin{Phonetics}{顽皮}{wan2 pi2}[][HSK 6]
    \definition{adj.}{atrevido; travesso; arteiro; levado; (crianças, adolescentes, etc.) adoram brincar e causar problemas e não dão ouvidos a conselhos}
  \end{Phonetics}
\end{Entry}

\begin{Entry}{顽强}{10,12}{⾴、⼸}
  \begin{Phonetics}{顽强}{wan2qiang2}[][HSK 6]
    \definition{adj.}{firme; tenaz; indomável; forte; resistente}
  \end{Phonetics}
\end{Entry}

%%%%%%%%%% 顾 %%%%%%%%%%
\subsection*{顾}

\begin{Entry}{顾}{10}{⾴}
  \begin{Phonetics}{顾}{gu4}[][HSK 6]
    \definition*{s.}{Sobrenome: Gu}
    \definition{adv.}{em vez disso; pelo contrário; indica o oposto, equivalente a 却 ou 反而}
    \definition{conj.}{mas; no entanto}
    \definition{v.}{olhar para trás; olhar para; virar-se e olhar para | cuidar de; atender a; levar em conta ou consideração | visitar; chamar | sentir pena de}
  \seealsoref{反而}{fan3'er2}
  \seealsoref{却}{que4}
  \end{Phonetics}
\end{Entry}

\begin{Entry}{顾及}{10,3}{⾴、⼃}
  \begin{Phonetics}{顾及}{gu4ji2}[][HSK 7-9]
    \definition{v.}{atender a; levar em conta; dar consideração a; cuidar de; notar}
  \end{Phonetics}
\end{Entry}

\begin{Entry}{顾不上}{10,4,3}{⾴、⼀、⼀}
  \begin{Phonetics}{顾不上}{gu4bu5shang4}[][HSK 7-9]
    \definition{v.}{não conseguir; não conseguir atender; incapaz de cuidar de (fazer algo)}
  \end{Phonetics}
\end{Entry}

\begin{Entry}{顾不得}{10,4,11}{⾴、⼀、⼻}
  \begin{Phonetics}{顾不得}{gu4bu5de5}[][HSK 7-9]
    \definition{v.}{incapaz de mudar algo | incapaz de lidar com}
  \end{Phonetics}
\end{Entry}

\begin{Entry}{顾全大局}{10,6,3,7}{⾴、⼊、⼤、⼫}
  \begin{Phonetics}{顾全大局}{gu4quan2-da4ju2}[][HSK 7-9]
    \definition{expr.}{``Considere a situação geral.''; levar em conta os interesses do todo; considerar a situação como um todo; levar em consideração o panorama geral; trabalhar para o benefício de todos}
  \end{Phonetics}
\end{Entry}

\begin{Entry}{顾问}{10,6}{⾴、⾨}
  \begin{Phonetics}{顾问}{gu4wen4}[][HSK 5]
    \definition[个,位,名]{s.}{conselheiro; consultor; assessor; pessoas com conhecimento especializado ou experiência contratadas para prestar consultoria a organizações ou indivíduos}
  \end{Phonetics}
\end{Entry}

\begin{Entry}{顾客}{10,9}{⾴、⼧}
  \begin{Phonetics}{顾客}{gu4ke4}[][HSK 2]
    \definition[个,位,名,些]{s.}{cliente; comprador; consumidor; paciente}
  \end{Phonetics}
\end{Entry}

\begin{Entry}{顾虑}{10,10}{⾴、⾌}
  \begin{Phonetics}{顾虑}{gu4lv4}[][HSK 7-9]
    \definition[丝,点]{s.}{preocupação; escrúpulo; receio; apreensão}
    \definition{v.}{estar apreensivo (sobre as consequências da própria ação)}
  \end{Phonetics}
\end{Entry}

%%%%%%%%%% 顿 %%%%%%%%%%
\subsection*{顿}

\begin{Entry}{顿}{10}{⾴}
  \begin{Phonetics}{顿}{dun4}[][HSK 3]
    \definition*{s.}{Sobrenome: Dun}
    \definition{adj.}{cansado; fatigado}
    \definition{adv.}{de repente; imediatamente; indica que o tempo é curto, equivalente a 立刻}
    \definition{clas.}{usado para refeições | usado para surras, repreensões, castigos físicos, etc.}
    \definition{s.}{um lugar para ficar; acomodação e alimentação}
    \definition{v.}{pausar; parar; fazer uma pausa | pausar na escrita para reforçar o início ou o fim de um traço; ao escrever com pincel, pressione o pincel com força e pare um pouco sobre o papel | tocar o chão (com a cabeça) | bater o pé); chutar o chão ou bater no chão com um objeto | resolver; arranjar | montar acampamento; ficar temporariamente; parar para se hospedar; acampar}
  \seealsoref{立刻}{li4ke4}
  \end{Phonetics}
\end{Entry}

\begin{Entry}{顿时}{10,7}{⾴、⽇}
  \begin{Phonetics}{顿时}{dun4shi2}[][HSK 7-9]
    \definition{adv.}{de repente; imediatamente; repentinamente; indica que uma ação ou comportamento ocorre sob certas circunstâncias ou imediatamente após algo; usado principalmente na escrita; usado apenas para descrever eventos passados}
  \end{Phonetics}
\end{Entry}

%%%%%%%%%% 颁 %%%%%%%%%%
\subsection*{颁}

\begin{Entry}{颁}{10}{⾴}
  \begin{Phonetics}{颁}{ban1}
    \definition{v.}{promulgar; emitir; enviar | conceder ou conferir}
  \end{Phonetics}
\end{Entry}

\begin{Entry}{颁发}{10,5}{⾴、⼜}
  \begin{Phonetics}{颁发}{ban1fa1}[][HSK 7-9]
    \definition{v.}{promulgar; emitir (comandos, instruções, regulamentos, etc.) a um superior | premiar (prêmio, medalha, certificado, etc.)}
  \end{Phonetics}
\end{Entry}

\begin{Entry}{颁布}{10,5}{⾴、⼱}
  \begin{Phonetics}{颁布}{ban1bu4}[][HSK 7-9]
    \definition{v.}{promulgar; emitir; publicar; anunciar (leis, regulamentos, etc.), com um escopo de uso mais restrito do que 公布}
  \seealsoref{公布}{gong1bu4}
  \end{Phonetics}
\end{Entry}

\begin{Entry}{颁奖}{10,9}{⾴、⼤}
  \begin{Phonetics}{颁奖}{ban1/jiang3}[][HSK 7-9]
    \definition{v.+compl.}{conceder prêmios, bônus, certificados, etc.; distribuir prêmios, bônus, certificados, etc.}
  \end{Phonetics}
\end{Entry}

%%%%%%%%%% 预 %%%%%%%%%%
\subsection*{预}

\begin{Entry}{预}{10}{⾴}
  \begin{Phonetics}{预}{yu4}
    \definition{adv.}{antecipadamente}
    \definition{v.}{avançar | preparar}
  \end{Phonetics}
\end{Entry}

\begin{Entry}{预习}{10,3}{⾴、⼄}
  \begin{Phonetics}{预习}{yu4xi2}[][HSK 3]
    \definition{v.}{pré-visualizar; preparar uma lição; estudar antecipadamente as matérias que serão abordadas nas aulas}
  \end{Phonetics}
\end{Entry}

\begin{Entry}{预见}{10,4}{⾴、⾒}
  \begin{Phonetics}{预见}{yu4jian4}
    \definition{s.}{previsão; intuição; vislumbre}
    \definition{v.}{prever}
  \end{Phonetics}
\end{Entry}

\begin{Entry}{预计}{10,4}{⾴、⾔}
  \begin{Phonetics}{预计}{yu4 ji4}[][HSK 3]
    \definition{v.}{estimar; calcular com antecedência}
  \end{Phonetics}
\end{Entry}

\begin{Entry}{预订}{10,4}{⾴、⾔}
  \begin{Phonetics}{预订}{yu4ding4}[][HSK 4]
    \definition{v.}{reservar; fazer uma reserva}
  \end{Phonetics}
\end{Entry}

\begin{Entry}{预付}{10,5}{⾴、⼈}
  \begin{Phonetics}{预付}{yu4fu4}
    \definition{s.}{pré-pago}
    \definition{v.}{pagar antecipadamente}
  \end{Phonetics}
\end{Entry}

\begin{Entry}{预约}{10,6}{⾴、⽷}
  \begin{Phonetics}{预约}{yu4 yue1}[][HSK 6]
    \definition[个]{s.}{reserva}
    \definition{v.}{reservar; agendar; marcar compromisso; marcar uma consulta}
  \end{Phonetics}
\end{Entry}

\begin{Entry}{预防}{10,6}{⾴、⾩}
  \begin{Phonetics}{预防}{yu4fang2}[][HSK 3]
    \definition{v.}{prevenir; proteger-se contra; tomar precauções contra; preparar-se com antecedência para evitar que algo ruim aconteça}
  \end{Phonetics}
\end{Entry}

\begin{Entry}{预判}{10,7}{⾴、⼑}
  \begin{Phonetics}{预判}{yu4pan4}
    \definition{v.}{prever | antecipar}
  \end{Phonetics}
\end{Entry}

\begin{Entry}{预报}{10,7}{⾴、⼿}
  \begin{Phonetics}{预报}{yu4bao4}[][HSK 3]
    \definition[个,项]{s.}{boletim meteorológico; previsões meteorológicas antecipadas}
    \definition{v.}{prever (o tempo); relatar antes que algo aconteça, usado principalmente em relação ao clima, astronomia, desastres naturais, etc.}
  \end{Phonetics}
\end{Entry}

\begin{Entry}{预备}{10,8}{⾴、⼡}
  \begin{Phonetics}{预备}{yu4 bei4}[][HSK 5]
    \definition{v.}{preparar-se; ficar pronto}
  \end{Phonetics}
\end{Entry}

\begin{Entry}{预定}{10,8}{⾴、⼧}
  \begin{Phonetics}{预定}{yu4ding4}
    \definition{v.}{agendar com antecedência}
  \end{Phonetics}
\end{Entry}

\begin{Entry}{预购}{10,8}{⾴、⾙}
  \begin{Phonetics}{预购}{yu4gou4}
    \definition{s.}{compra antecipada}
    \definition{v.}{comprar antecipadamente}
  \end{Phonetics}
\end{Entry}

\begin{Entry}{预测}{10,9}{⾴、⽔}
  \begin{Phonetics}{预测}{yu4 ce4}[][HSK 4]
    \definition{v.}{prever; prognosticar; predizer}
  \end{Phonetics}
\end{Entry}

\begin{Entry}{预祝}{10,9}{⾴、⽰}
  \begin{Phonetics}{预祝}{yu4zhu4}
    \definition{v.}{parabenizar de antemão | oferecer os melhores votos para}
  \end{Phonetics}
\end{Entry}

\begin{Entry}{预览}{10,9}{⾴、⾒}
  \begin{Phonetics}{预览}{yu4lan3}
    \definition{s.}{visualização}
    \definition{v.}{visualizar}
  \end{Phonetics}
\end{Entry}

\begin{Entry}{预留}{10,10}{⾴、⽥}
  \begin{Phonetics}{预留}{yu4liu2}
    \definition{v.}{separar | reservar}
  \end{Phonetics}
\end{Entry}

\begin{Entry}{预配}{10,10}{⾴、⾣}
  \begin{Phonetics}{预配}{yu4pei4}
    \definition{s.}{pré-alocado | pré-cabeado}
    \definition{v.}{pré-alocar | pré-cabear}
  \end{Phonetics}
\end{Entry}

\begin{Entry}{预谋}{10,11}{⾴、⾔}
  \begin{Phonetics}{预谋}{yu4mou2}
    \definition{adj.}{premeditado}
    \definition{v.}{planejar algo com antecedência (especialmente um crime)}
  \end{Phonetics}
\end{Entry}

\begin{Entry}{预提}{10,12}{⾴、⼿}
  \begin{Phonetics}{预提}{yu4ti2}
    \definition{s.}{retenção}
    \definition{v.}{reter (imposto)}
  \end{Phonetics}
\end{Entry}

\begin{Entry}{预期}{10,12}{⾴、⽉}
  \begin{Phonetics}{预期}{yu4qi1}[][HSK 5]
    \definition{v.}{esperar; antecipar; imaginar; antecipar com expectativa}
  \end{Phonetics}
\end{Entry}

\begin{Entry}{预感}{10,13}{⾴、⼼}
  \begin{Phonetics}{预感}{yu4gan3}
    \definition{s.}{premonição}
    \definition{v.}{ter uma premonição}
  \end{Phonetics}
\end{Entry}

\begin{Entry}{预警}{10,19}{⾴、⾔}
  \begin{Phonetics}{预警}{yu4jing3}
    \definition{s.}{aviso | aviso antecipado}
  \end{Phonetics}
\end{Entry}

%%%%%%%%%% 饿 %%%%%%%%%%
\subsection*{饿}

\begin{Entry}{饿}{10}{⾷}
  \begin{Phonetics}{饿}{e4}[][HSK 1]
    \definition{adj.}{faminto}
    \definition{v.}{passar fome; causar fome}
  \end{Phonetics}
\end{Entry}

%%%%%%%%%% 骏 %%%%%%%%%%
\subsection*{骏}

\begin{Entry}{骏}{10}{⾺}
  \begin{Phonetics}{骏}{jun4}
    \definition{s.}{belo cavalo; corcel; animal de montaria}
  \end{Phonetics}
\end{Entry}

\begin{Entry}{骏马}{10,3}{⾺、⾺}
  \begin{Phonetics}{骏马}{jun4ma3}[][HSK 7-9]
    \definition[匹,群]{s.}{belo cavalo; corcel; animal de montaria}
  \end{Phonetics}
\end{Entry}

%%%%%%%%%% 高 %%%%%%%%%%
\subsection*{高}

\begin{Entry}{高}{10}{⾼}[Kangxi 189]
  \begin{Phonetics}{高}{gao1}[][HSK 1]
    \definition*{s.}{Sobrenome: Gao}
    \definition{adj.}{alto; elevado; grande distância de baixo para cima; longe do chão | barulhento | sofisticado; caro; de preço elevado; acima do valor real ou do preço de mercado | acima da média; de alto nível ou grau; acima do padrão geral ou da média; de nível superior}
    \definition{s.}{altura; altitude}
  \end{Phonetics}
\end{Entry}

\begin{Entry}{高于}{10,3}{⾼、⼆}
  \begin{Phonetics}{高于}{gao1 yu2}[][HSK 5]
    \definition{v.}{ser mais alto do que; sobrepujar}
  \end{Phonetics}
\end{Entry}

\begin{Entry}{高大}{10,3}{⾼、⼤}
  \begin{Phonetics}{高大}{gao1 da4}[][HSK 5]
    \definition{adj.}{alto e grande; alto | elevado; sublime; nobre}
  \end{Phonetics}
\end{Entry}

\begin{Entry}{高山}{10,3}{⾼、⼭}
  \begin{Phonetics}{高山}{gao1shan1}[][HSK 7-9]
    \definition[座]{s.}{alta montanha; alpes}
  \end{Phonetics}
\end{Entry}

\begin{Entry}{高中}{10,4}{⾼、⼁}
  \begin{Phonetics}{高中}{gao1 zhong1}[][HSK 2]
    \definition[所,个]{s.}{ensino médio; escola secundária de ensino médio}
  \end{Phonetics}
\end{Entry}

\begin{Entry}{高手}{10,4}{⾼、⼿}
  \begin{Phonetics}{高手}{gao1 shou3}[][HSK 6]
    \definition[位,个,名,些,群]{s.}{ás; mestre; especialista; \emph{expert}; uma pessoa com habilidades excepcionais}
  \end{Phonetics}
\end{Entry}

\begin{Entry}{高尔夫}{10,5,4}{⾼、⼩、⼤}
  \begin{Phonetics}{高尔夫}{gao1'er3fu1}
    \definition{s.}{(empréstimo linguístico) \emph{golf}}
  \end{Phonetics}
\end{Entry}

\begin{Entry}{高尔夫球}{10,5,4,11}{⾼、⼩、⼤、⽟}
  \begin{Phonetics}{高尔夫球}{gao1'er3fu1qiu2}[][HSK 7-9]
    \definition[个,只,场,些]{s.}{golfe | bola de golfe}
  \end{Phonetics}
\end{Entry}

\begin{Entry}{高价}{10,6}{⾼、⼈}
  \begin{Phonetics}{高价}{gao1 jia4}[][HSK 4]
    \definition{s.}{preço alto; bilhete caro; custo elevado; dispendioso}
  \end{Phonetics}
\end{Entry}

\begin{Entry}{高兴}{10,6}{⾼、⼋}
  \begin{Phonetics}{高兴}{gao1xing4}[][HSK 1]
    \definition{adj.}{contente; feliz; exultante; alegre; satisfeito; animado}
    \definition{v.}{estar contente; estar feliz; estar animado; estar de bom humor; fazer algo com alegria; gostar}
  \end{Phonetics}
\end{Entry}

\begin{Entry}{高压}{10,6}{⾼、⼚}
  \begin{Phonetics}{高压}{gao1ya1}[][HSK 7-9]
    \definition{s.}{Física, Meteorologia: alta pressão (oposto a 低压) | Eletricidade: alta tensão; alta voltagem (oposto a 低压) | Política: mão de ferro; arrogância | Medicina: pressão sistólica; pressão máxima | perseguição cruel; opressão extrema}
  \seealsoref{低压}{di1ya1}
  \end{Phonetics}
\end{Entry}

\begin{Entry}{高级}{10,6}{⾼、⽷}
  \begin{Phonetics}{高级}{gao1ji2}[][HSK 2]
    \definition{adj.}{sênior; de alto escalão; de alto nível; elevado; excelente; superior; estágio avançado | e alta qualidade; de primeira qualidade; avançado}
  \end{Phonetics}
\end{Entry}

\begin{Entry}{高考}{10,6}{⾼、⽼}
  \begin{Phonetics}{高考}{gao1 kao3}[][HSK 6]
    \definition[次,回,场]{s.}{vestibular; exame de admissão em instituições de ensino superior}
  \end{Phonetics}
\end{Entry}

\begin{Entry}{高血压}{10,6,6}{⾼、⾎、⼚}
  \begin{Phonetics}{高血压}{gao1xue4ya1}[][HSK 7-9]
    \definition{adj.}{hipertenso}
    \definition[点儿]{s.}{hipertenção; pressão alta}
  \end{Phonetics}
\end{Entry}

\begin{Entry}{高低}{10,7}{⾼、⼈}
  \begin{Phonetics}{高低}{gao1di1}[][HSK 7-9]
    \definition{adv.}{apenas; simplesmente; em qualquer caso; de qualquer forma; em qualquer conta; indica que não importa o que | finalmente; no final, depois de tudo}
    \definition{s.}{inclinação; nível; altura | diferença de grau; superioridade ou inferioridade relativa | discrição; senso de propriedade | (falar ou fazer coisas) medida; profundidade, leveza e peso}
  \end{Phonetics}
\end{Entry}

\begin{Entry}{高层}{10,7}{⾼、⼫}
  \begin{Phonetics}{高层}{gao1 ceng2}[][HSK 6]
    \definition{adj.}{(de um edifício) arranha-céu | (de posição oficial) alto nível}
    \definition{s.}{nível superior; piso, camada, etc. | arranha-céus; um prédio de apartamentos alto}
  \end{Phonetics}
\end{Entry}

\begin{Entry}{高技术}{10,7,5}{⾼、⼿、⽊}
  \begin{Phonetics}{高技术}{gao1 ji4 shu4}
    \definition{s.}{alta tecnologia; \emph{hight tech}}
  \seealsoref{高科技}{gao1 ke1 ji4}
  \end{Phonetics}
\end{Entry}

\begin{Entry}{高尚}{10,8}{⾼、⼩}
  \begin{Phonetics}{高尚}{gao1shang4}[][HSK 4]
    \definition{adj.}{nobre; elevado; descreve um alto padrão moral e uma boa qualidade de pensamento | significativo e não de mau gosto}
  \end{Phonetics}
\end{Entry}

\begin{Entry}{高昂}{10,8}{⾼、⽇}
  \begin{Phonetics}{高昂}{gao1'ang2}[][HSK 7-9]
    \definition{adj.}{alto; eufórico; exaltado | caro; exorbitante}
    \definition{v.}{manter erguida (a cabeça, etc.)}
  \end{Phonetics}
\end{Entry}

\begin{Entry}{高明}{10,8}{⾼、⽇}
  \begin{Phonetics}{高明}{gao1ming2}[][HSK 7-9]
    \definition{adj.}{sábio; brilhante; (percepção, habilidades) excelente}
    \definition{s.}{pessoa sábia; pessoa habilidosa}
  \end{Phonetics}
\end{Entry}

\begin{Entry}{高空}{10,8}{⾼、⽳}
  \begin{Phonetics}{高空}{gao1kong1}[][HSK 7-9]
    \definition{s.}{alta altitude; ar superior (oposto a 低空)}
  \seealsoref{低空}{di1kong1}
  \end{Phonetics}
\end{Entry}

\begin{Entry}{高度}{10,9}{⾼、⼴}
  \begin{Phonetics}{高度}{gao1 du4}[][HSK 5]
    \definition{adj.}{alto; elevado; avançado; alto grau | alta concentração; intenso}
    \definition[个]{s.}{altura; altitude; elevação; distância de baixo para cima; o grau e o nível em que as coisas se desenvolveram}
  \end{Phonetics}
\end{Entry}

\begin{Entry}{高科技}{10,9,7}{⾼、⽲、⼿}
  \begin{Phonetics}{高科技}{gao1 ke1 ji4}[][HSK 6]
    \definition[种,类]{s.}{alta tecnologia; \emph{high tech}}
  \seealsoref{高技术}{gao1 ji4 shu4}
  \end{Phonetics}
\end{Entry}

\begin{Entry}{高贵}{10,9}{⾼、⾙}
  \begin{Phonetics}{高贵}{gao1gui4}[][HSK 7-9]
    \definition{adj.}{(caráter pessoal) nobre; honrado; moralmente elevado; magnânimo | grandeza; extremamente valioso | elitista; altamente privilegiado; refere-se àqueles com status elevado e vida superior}
  \end{Phonetics}
\end{Entry}

\begin{Entry}{高原}{10,10}{⾼、⼚}
  \begin{Phonetics}{高原}{gao1 yuan2}[][HSK 5]
    \definition[片]{s.}{platô; terras altas; planalto | planalto continental}
  \end{Phonetics}
\end{Entry}

\begin{Entry}{高峰}{10,10}{⾼、⼭}
  \begin{Phonetics}{高峰}{gao1feng1}[][HSK 6]
    \definition[个,座]{s.}{cume; pináculo; pico da montanha | pico (de atividade, qualidade ou realização); uma metáfora para o ponto mais alto no desenvolvimento das coisas | cúpula; principais líderes; uma metáfora para o mais alto nível de liderança}
  \end{Phonetics}
\end{Entry}

\begin{Entry}{高峰期}{10,10,12}{⾼、⼭、⽉}
  \begin{Phonetics}{高峰期}{gao1feng1qi1}[][HSK 7-9]
    \definition[个]{s.}{período de pico; (de tráfego) horas de pico; o período em que ocorre com mais frequência ou se desenvolve mais próspero}
  \end{Phonetics}
\end{Entry}

\begin{Entry}{高效}{10,10}{⾼、⽁}
  \begin{Phonetics}{高效}{gao1xiao4}[][HSK 7-9]
    \definition{adj.}{altamente eficiente | eficiente | altamente eficaz}
  \end{Phonetics}
\end{Entry}

\begin{Entry}{高档}{10,10}{⾼、⽊}
  \begin{Phonetics}{高档}{gao1dang4}[][HSK 6]
    \definition{adj.}{grau superior; alta qualidade; alo grau; qualidade superior; boa qualidade, preço alto (produto)}
  \end{Phonetics}
\end{Entry}

\begin{Entry}{高涨}{10,10}{⾼、⽔}
  \begin{Phonetics}{高涨}{gao1zhang3}[][HSK 7-9]
    \definition{v.}{ascender; subir alto; avançar; (preços, sentimento, etc.) subir rapidamente (em oposição a 低落)}
  \seealsoref{低落}{di1luo4}
  \end{Phonetics}
\end{Entry}

\begin{Entry}{高调}{10,10}{⾼、⾔}
  \begin{Phonetics}{高调}{gao1diao4}[][HSK 7-9]
    \definition{adj.}{alto perfil; retaliação; significa agir de forma muito ostensiva e chamativa, deixando claro para todos; também pode significar opor-se deliberadamente aos outros, chegando até a provocar uma briga}
    \definition{s.}{tom elevado; palavras de alto som; sons mais agudos do que o normal ao cantar ou falar}
  \end{Phonetics}
\end{Entry}

\begin{Entry}{高速}{10,10}{⾼、⾡}
  \begin{Phonetics}{高速}{gao1 su4}[][HSK 3]
    \definition{adj.}{alta velocidade; veloz; rápido}
    \definition[条]{s.}{autoestrada; via expressa; rodovia}
  \end{Phonetics}
\end{Entry}

\begin{Entry}{高速公路}{10,10,4,13}{⾼、⾡、⼋、⾜}
  \begin{Phonetics}{高速公路}{gao1su4gong1lu4}[][HSK 3]
    \definition[条]{s.}{via expressa; rodovia; autoestrada; as rodovias destinadas exclusivamente ao tráfego de veículos em alta velocidade são retas e, quando cruzam outras vias, utilizam cruzamentos em nível}
  \end{Phonetics}
\end{Entry}

\begin{Entry}{高铁}{10,10}{⾼、⾦}
  \begin{Phonetics}{高铁}{gao1 tie3}[][HSK 4]
    \definition{s.}{trem de alta velocidade; trem bala}
  \end{Phonetics}
\end{Entry}

\begin{Entry}{高傲}{10,12}{⾼、⼈}
  \begin{Phonetics}{高傲}{gao1'ao4}[][HSK 7-9]
    \definition{adj.}{arrogante; altivo | orgulhoso; respeitoso; nobre}
  \end{Phonetics}
\end{Entry}

\begin{Entry}{高温}{10,12}{⾼、⽔}
  \begin{Phonetics}{高温}{gao1 wen1}[][HSK 5]
    \definition{s.}{alta temperatura (oposto a 低温); temperatura elevada; hipertermia; megatemperatura; inferno}
  \seealsoref{低温}{di1 wen1}
  \end{Phonetics}
\end{Entry}

\begin{Entry}{高等}{10,12}{⾼、⽵}
  \begin{Phonetics}{高等}{gao1 deng3}[][HSK 6]
    \definition{adj.}{superior; avançado (oposto a 低等) | alto nível}
  \seealsoref{低等}{di1 deng3}
  \end{Phonetics}
\end{Entry}

\begin{Entry}{高超}{10,12}{⾼、⾛}
  \begin{Phonetics}{高超}{gao1chao1}[][HSK 7-9]
    \definition{adj.}{soberbo; excelente; descreve um nível muito alto, excedendo a maioria dos níveis}
  \end{Phonetics}
\end{Entry}

\begin{Entry}{高雅}{10,12}{⾼、⾫}
  \begin{Phonetics}{高雅}{gao1ya3}[][HSK 7-9]
    \definition{adj.}{delicado | elegante}
    \definition{s.}{delicadeza; decoro | elegância}
  \end{Phonetics}
\end{Entry}

\begin{Entry}{高新技术}{10,13,7,5}{⾼、⽄、⼿、⽊}
  \begin{Phonetics}{高新技术}{gao1xin1-ji4shu4}[][HSK 7-9]
    \definition[项,门,套]{s.}{nova e alta tecnologia; \emph{high‐tech}}
  \end{Phonetics}
\end{Entry}

\begin{Entry}{高楼}{10,13}{⾼、⽊}
  \begin{Phonetics}{高楼}{gao1lou2}
    \definition[座]{s.}{edifício alto | edifício de muitos andares | arranha-céu}
  \end{Phonetics}
\end{Entry}

\begin{Entry}{高跟鞋}{10,13,15}{⾼、⾜、⾰}
  \begin{Phonetics}{高跟鞋}{gao1 gen1 xie2}[][HSK 5]
    \definition[双]{s.}{salto alto; sapatos de salto alto; sapato feminino com salto mais alto e mais distante do chão}
  \end{Phonetics}
\end{Entry}

\begin{Entry}{高龄}{10,13}{⾼、⿒}
  \begin{Phonetics}{高龄}{gao1ling2}[][HSK 7-9]
    \definition{adj.}{mais velho que o normal | avançado em anos}
    \definition{s.}{idade avançada; idade venerável}
  \end{Phonetics}
\end{Entry}

\begin{Entry}{高潮}{10,15}{⾼、⽔}
  \begin{Phonetics}{高潮}{gao1chao2}[][HSK 4]
    \definition[个,场]{s.}{maré alta; o nível mais alto da maré em um ciclo de maré | pico; aumento; maré alta; uma metáfora para o estágio mais próspero de desenvolvimento das coisas (diferente de 低潮) | (ficção, drama e filmes) clímax}
  \seealsoref{低潮}{di1chao2}
  \end{Phonetics}
\end{Entry}

\begin{Entry}{高额}{10,15}{⾼、⾴}
  \begin{Phonetics}{高额}{gao1'e2}[][HSK 7-9]
    \definition{s.}{quantidade enorme; cota grande | grande quantidade}
  \end{Phonetics}
\end{Entry}

%%%%%%%%%% 髟 %%%%%%%%%%
\subsection*{髟}

\begin{Entry}{髟}{10}{⾽}[Kangxi 190]
  \begin{Phonetics}{髟}{biao1}
    \definition{adj.}{(de cabelo) solto, caído}
  \end{Phonetics}
\end{Entry}

%%%%%%%%%% 鬯 %%%%%%%%%%
\subsection*{鬯}

\begin{Entry}{鬯}{10}{⾿}[Kangxi 192]
  \begin{Phonetics}{鬯}{chang4}
    \definition{adj.}{suave; desimpedido | livre; desinibido}
    \definition{s.}{um tipo de vinho usado em sacrifícios antigos | (antigo) estojo ou bolsa para arco | o mesmo que 畅}
  \seealsoref{畅}{chang4}
  \end{Phonetics}
\end{Entry}

%%%%%%%%%% 鬲 %%%%%%%%%%
\subsection*{鬲}

\begin{Entry}{鬲}{10}{⿀}[Kangxi 193]
  \begin{Phonetics}{鬲}{ge2}
    \definition{s.}{um antigo utensílio de cozinha semelhante a um caldeirão; uma grande panela de barro | utilizado em nomes geográficos ou pessoais}
  \end{Phonetics}
  \begin{Phonetics}{鬲}{li4}
    \definition{s.}{recipiente de cerâmica antigo com três pernas usado para cozinhar, com marcas de cordão na parte externa e pernas ocas}
  \end{Phonetics}
\end{Entry}

%%%%%%%%%% 鸭 %%%%%%%%%%
\subsection*{鸭}

\begin{Entry}{鸭}{10}{⿃}
  \begin{Phonetics}{鸭}{ya1}
    \definition[只]{s.}{pato | (gíria) prostituto}
  \end{Phonetics}
\end{Entry}

\begin{Entry}{鸭子}{10,3}{⿃、⼦}
  \begin{Phonetics}{鸭子}{ya1 zi5}[][HSK 5]
    \definition[只,群]{s.}{pato | Gíria: prostituto}
  \end{Phonetics}
\end{Entry}

%%%%%%%%%% 鸵 %%%%%%%%%%
\subsection*{鸵}

\begin{Entry}{鸵}{10}{⿃}
  \begin{Phonetics}{鸵}{tuo2}
    \definition[只]{s.}{avestruz}
  \end{Phonetics}
\end{Entry}

\begin{Entry}{鸵鸟}{10,5}{⿃、⿃}
  \begin{Phonetics}{鸵鸟}{tuo2niao3}
    \definition{s.}{avestruz}
  \end{Phonetics}
\end{Entry}

%%%%% EOF %%%%%


%%%
%%% 11画
%%%

\section*{11画}\addcontentsline{toc}{section}{11画}

\begin{entry}{假}{11}[Radical 人]
  \begin{phonetics}{假}{jia3}
    \definition{adj.}{falso | artificial}
    \definition{v.}{emprestar}
  \end{phonetics}
  \begin{phonetics}{假}{jia4}
    \definition{s.}{férias}
  \end{phonetics}
\end{entry}

\begin{entry}{假如}{11,6}
  \begin{phonetics}{假如}{jia3ru2}
    \definition{conj.}{se | supondo | em caso}
  \end{phonetics}
\end{entry}

\begin{entry}{假声}{11,7}
  \begin{phonetics}{假声}{jia3sheng1}
    \definition{s.}{falsete}
  \seealsoref{真声}{zhen1sheng1}
  \end{phonetics}
\end{entry}

\begin{entry}{假证件}{11,7,6}
  \begin{phonetics}{假证件}{jia3zheng4jian4}
    \definition{s.}{documentos falsos}
  \end{phonetics}
\end{entry}

\begin{entry}{假使}{11,8}
  \begin{phonetics}{假使}{jia3shi3}
    \definition{conj.}{se | supondo | em caso}
  \end{phonetics}
\end{entry}

\begin{entry}{假的}{11,8}
  \begin{phonetics}{假的}{jia3de5}
    \definition{adj.}{falso | substituto | simulado}
  \end{phonetics}
\end{entry}

\begin{entry}{偏偏}{11,11}
  \begin{phonetics}{偏偏}{pian1pian1}
    \definition{adv.}{voluntariamente | insistentemente | persistentemente | ao contrário da expectativa | infelizmente (indicando que alguma coisa aconteceu ao contrário do que se esperava) | teimosamente (indicando que algo é o oposto ao que seria normal ou razoável) | precisamente (indicando que alguém ou um grupo é escolhido)}
  \end{phonetics}
\end{entry}

\begin{entry}{做}{11}[Radical 人]
  \begin{phonetics}{做}{zuo4}
    \definition{v.}{fazer}
  \end{phonetics}
\end{entry}

\begin{entry}{做生活}{11,5,9}
  \begin{phonetics}{做生活}{zuo4sheng1huo2}
    \definition{v.}{fazer tabalhos manuais}
  \end{phonetics}
\end{entry}

\begin{entry}{做戏}{11,6}
  \begin{phonetics}{做戏}{zuo4xi4}
    \definition{v.}{atuar em uma peça | fazer uma peça}
  \end{phonetics}
\end{entry}

\begin{entry}{做作}{11,7}
  \begin{phonetics}{做作}{zuo4zuo5}
    \definition{adj.}{afetado | artificial}
  \end{phonetics}
\end{entry}

\begin{entry}{做饭}{11,7}
  \begin{phonetics}{做饭}{zuo4fan4}
    \definition{v.}{preparar uma refeição | cozinhar}
  \end{phonetics}
\end{entry}

\begin{entry}{做法}{11,8}
  \begin{phonetics}{做法}{zuo4fa3}
    \definition[个]{s.}{método para fazer | prática | receita | maneira de lidar com algo | método de trabalho}
  \end{phonetics}
\end{entry}

\begin{entry}{做活}{11,9}
  \begin{phonetics}{做活}{zuo4huo2}
    \definition{v.}{trabalhar para ganhar a vida (especialmente de mulher costureira)}
  \end{phonetics}
\end{entry}

\begin{entry}{做眼}{11,11}
  \begin{phonetics}{做眼}{zuo4yan3}
    \definition{v.}{agir como um guia | trabalhar como espião}
  \end{phonetics}
\end{entry}

\begin{entry}{停}{11}[Radical 人]
  \begin{phonetics}{停}{ting2}
    \definition{v.}{parar | estacionar (um carro)}
  \end{phonetics}
\end{entry}

\begin{entry}{停工}{11,3}
  \begin{phonetics}{停工}{ting2gong1}
    \definition{v.}{parar de trabalhar | parar a produção}
  \end{phonetics}
\end{entry}

\begin{entry}{停办}{11,4}
  \begin{phonetics}{停办}{ting2ban4}
    \definition{v.}{cancelar | sair do negócio | desligar | terminar}
  \end{phonetics}
\end{entry}

\begin{entry}{停止}{11,4}
  \begin{phonetics}{停止}{ting2zhi3}
    \definition{v.}{cessar | encerrar | parar}
  \end{phonetics}
\end{entry}

\begin{entry}{停火}{11,4}
  \begin{phonetics}{停火}{ting2huo3}
    \definition{s.}{cessar-fogo}
    \definition{v.+compl.}{cessar fogo}
  \end{phonetics}
\end{entry}

\begin{entry}{停车}{11,4}
  \begin{phonetics}{停车}{ting2che1}
    \definition{v.}{parar de trabalhar (uma máquina) | estacionar | parar (um veículo) | paralisar}
  \end{phonetics}
\end{entry}

\begin{entry}{停车场}{11,4,6}
  \begin{phonetics}{停车场}{ting2che1chang3}
    \definition{s.}{parque de estacionamento}
  \end{phonetics}
\end{entry}

\begin{entry}{停业}{11,5}
  \begin{phonetics}{停业}{ting2ye4}
    \definition{v.}{cessar a negociação (temporária ou permanentemente) | fechar}
  \end{phonetics}
\end{entry}

\begin{entry}{停用}{11,5}
  \begin{phonetics}{停用}{ting2yong4}
    \definition{v.}{desabilitar | descontinuar | parar de usar | suspender}
  \end{phonetics}
\end{entry}

\begin{entry}{停电}{11,5}
  \begin{phonetics}{停电}{ting2dian4}
    \definition{s.}{corte de energia}
    \definition{v.}{ter uma falha de energia}
  \end{phonetics}
\end{entry}

\begin{entry}{停当}{11,6}
  \begin{phonetics}{停当}{ting2dang5}
    \definition{adj.}{realizado | preparado | assentado}
  \end{phonetics}
\end{entry}

\begin{entry}{停息}{11,10}
  \begin{phonetics}{停息}{ting2xi1}
    \definition{v.}{cessar | parar}
  \end{phonetics}
\end{entry}

\begin{entry}{停留}{11,10}
  \begin{phonetics}{停留}{ting2liu2}
    \definition{v.}{ficar em algum lugar temporariamente | demorar | permanecer}
  \end{phonetics}
\end{entry}

\begin{entry}{停课}{11,10}
  \begin{phonetics}{停课}{ting2ke4}
    \definition{v.}{fechar (escola) | parar as aulas}
  \end{phonetics}
\end{entry}

\begin{entry}{停歇}{11,13}
  \begin{phonetics}{停歇}{ting2xie1}
    \definition{v.}{parar para descansar}
  \end{phonetics}
\end{entry}

\begin{entry}{偶然}{11,12}
  \begin{phonetics}{偶然}{ou3ran2}
    \definition{adv.}{por acaso | fortuitamente}
  \end{phonetics}
\end{entry}

\begin{entry}{偷}{11}[Radical 人]
  \begin{phonetics}{偷}{tou1}
    \definition{adv.}{furtivamente}
    \definition{v.}{furtar | roubar}
  \end{phonetics}
\end{entry}

\begin{entry}{偷安}{11,6}
  \begin{phonetics}{偷安}{tou1'an1}
    \definition{v.}{buscar facilidade temporária}
  \end{phonetics}
\end{entry}

\begin{entry}{偷听}{11,7}
  \begin{phonetics}{偷听}{tou1ting1}
    \definition{v.}{bisbilhotar; monitorar (secretamente)}
  \end{phonetics}
\end{entry}

\begin{entry}{偷窃}{11,9}
  \begin{phonetics}{偷窃}{tou1qie4}
    \definition{v.}{furtar | roubar}
  \end{phonetics}
\end{entry}

\begin{entry}{偷情}{11,11}
  \begin{phonetics}{偷情}{tou1qing2}
    \definition{v.}{manter um caso de amor clandestino}
  \end{phonetics}
\end{entry}

\begin{entry}{偷袭}{11,11}
  \begin{phonetics}{偷袭}{tou1xi2}
    \definition{s.}{ataque surpresa}
    \definition{v.}{montar um ataque furtivo | invadir}
  \end{phonetics}
\end{entry}

\begin{entry}{偷渡}{11,12}
  \begin{phonetics}{偷渡}{tou1du4}
    \definition{s.}{contrabando | imigração ilegal | clandestino (em um navio)}
    \definition{v.}{executar um bloqueio | roubar através da fronteira internacional}
  \end{phonetics}
\end{entry}

\begin{entry}{偷税}{11,12}
  \begin{phonetics}{偷税}{tou1shui4}
    \definition{s.}{evasão fiscal}
  \end{phonetics}
\end{entry}

\begin{entry}{偸}{11}
  \begin{phonetics}{偸}{tou1}
    \variantof{偷}
  \end{phonetics}
\end{entry}

\begin{entry}{副}{11}[Radical 刀]
  \begin{phonetics}{副}{fu4}
    \definition{clas.}{para pares, conjuntos de coisas e expressões faciais | para óculos, luvas, etc.}
  \end{phonetics}
\end{entry}

\begin{entry}{唱}{11}[Radical ⼝]
  \begin{phonetics}{唱}{chang4}
    \definition{v.}{cantar}
  \end{phonetics}
\end{entry}

\begin{entry}{唱歌}{11,14}
  \begin{phonetics}{唱歌}{chang4ge1}
    \definition{v.+compl.}{cantar}
  \end{phonetics}
\end{entry}

\begin{entry}{唾骂}{11,9}
  \begin{phonetics}{唾骂}{tuo4ma4}
    \definition{v.}{insultar | amaldiçoar}
  \end{phonetics}
\end{entry}

\begin{entry}{商店}{11,8}
  \begin{phonetics}{商店}{shang1dian4}
    \definition[家,个]{s.}{loja}
  \end{phonetics}
\end{entry}

\begin{entry}{商贸}{11,9}
  \begin{phonetics}{商贸}{shang1mao4}
    \definition{s.}{comércio}
  \end{phonetics}
\end{entry}

\begin{entry}{啤酒}{11,10}
  \begin{phonetics}{啤酒}{pi2jiu3}
    \definition[杯,瓶,罐,桶,缸]{s.}{(empréstimo linguístico) cerveja}
  \end{phonetics}
\end{entry}

\begin{entry}{啤酒馆}{11,10,11}
  \begin{phonetics}{啤酒馆}{pi2jiu3guan3}
    \definition{s.}{cervejaria}
  \end{phonetics}
\end{entry}

\begin{entry}{啥}{11}[Radical 口]
  \begin{phonetics}{啥}{sha2}
    \definition{adv.}{Equivalente a 什么 (dialeto), também pronunciado como \dpy{sha4}}
    \seeref{什么}{shen2me5}
  \end{phonetics}
  \begin{phonetics}{啥}{sha4}
    \definition{adv.}{Equivalente a 什么 (dialeto), também pronunciado como \dpy{sha2}}
  \end{phonetics}
\end{entry}

\begin{entry}{啵}{11}[Radical 口]
  \begin{phonetics}{啵}{bo1}
    \definition{s.}{(onomatopéia) borbulhar}
  \end{phonetics}
  \begin{phonetics}{啵}{bo5}
    \definition{part.}{partícula gramaticalmente equivalente a 吧}
  \seealsoref{吧}{ba5}
  \end{phonetics}
\end{entry}

\begin{entry}{圈粉}{11,10}
  \begin{phonetics}{圈粉}{quan1fen3}
    \definition{s.}{(neologismo, coloquial) ganhar alguém como fã, obter novos fãs}
  \end{phonetics}
\end{entry}

\begin{entry}{埦}{11}
  \begin{phonetics}{埦}{wan3}
    \variantof{碗}
  \end{phonetics}
\end{entry}

\begin{entry}{基本功}{11,5,5}
  \begin{phonetics}{基本功}{ji1ben3gong1}
    \definition{s.}{habilidades | fundamentos básicos}
  \end{phonetics}
\end{entry}

\begin{entry}{基本法}{11,5,8}
  \begin{phonetics}{基本法}{ji1ben3fa3}
    \definition{s.}{lei básica (constituição)}
  \end{phonetics}
\end{entry}

\begin{entry}{基因}{11,6}
  \begin{phonetics}{基因}{ji1yin1}
    \definition{s.}{gene}
  \end{phonetics}
\end{entry}

\begin{entry}{基督教}{11,13,11}
  \begin{phonetics}{基督教}{ji1du1jiao4}
    \definition*{s.}{Cristianismo | Cristão}
  \end{phonetics}
\end{entry}

\begin{entry}{堵车}{11,4}
  \begin{phonetics}{堵车}{du3che1}
    \definition{v.}{congestionar (trânsito)}
    \definition{v.+compl.}{congestionamento | engarrafamento (de trânsito)}
  \end{phonetics}
\end{entry}

\begin{entry}{够}{11}[Radical ⼣]
  \begin{phonetics}{够}{gou4}
    \definition{adj.}{suficiente}
    \definition{adv.}{(antes do adj.) realmente}
    \definition{v.}{bastar | chegar}
  \end{phonetics}
\end{entry}

\begin{entry}{够不着}{11,4,11}
  \begin{phonetics}{够不着}{gou4bu5zhao2}
    \definition{v.}{ser incapaz de alcançar}
  \end{phonetics}
\end{entry}

\begin{entry}{够本}{11,5}
  \begin{phonetics}{够本}{gou4ben3}
    \definition{v.}{empatar | fazer valer o dinheiro}
  \end{phonetics}
\end{entry}

\begin{entry}{够呛}{11,7}
  \begin{phonetics}{够呛}{gou4qiang4}
    \definition{adj.}{suficiente | terrível | insuportável | improvável}
  \end{phonetics}
\end{entry}

\begin{entry}{够味}{11,8}
  \begin{phonetics}{够味}{gou4wei4}
    \definition{adj.}{excelente | na medida}
  \end{phonetics}
\end{entry}

\begin{entry}{够戗}{11,8}
  \begin{phonetics}{够戗}{gou4qiang4}
    \variantof{够呛}
  \end{phonetics}
\end{entry}

\begin{entry}{够朋友}{11,8,4}
  \begin{phonetics}{够朋友}{gou4peng2you5}
    \definition{v.}{ser um amigo verdadeiro}
  \end{phonetics}
\end{entry}

\begin{entry}{够格}{11,10}
  \begin{phonetics}{够格}{gou4ge2}
    \definition{adj.}{apto | qualificado | apresentável}
  \end{phonetics}
\end{entry}

\begin{entry}{够得着}{11,11,11}
  \begin{phonetics}{够得着}{gou4de5zhao2}
    \definition{v.}{estar à altura | alcançar}
  \end{phonetics}
\end{entry}

\begin{entry}{婚礼}{11,5}
  \begin{phonetics}{婚礼}{hun1li3}
    \definition[场]{s.}{casamento | núpcias | cerimônia de casamento}
  \end{phonetics}
\end{entry}

\begin{entry}{宿舍}{11,8}
  \begin{phonetics}{宿舍}{su4she4}
    \definition[间]{s.}{dormitório | quarto de dormir | hostel}
  \end{phonetics}
\end{entry}

\begin{entry}{寂寞}{11,13}
  \begin{phonetics}{寂寞}{ji4mo4}
    \definition{adj.}{sozinho | solitário | (de um lugar) silencioso}
  \end{phonetics}
\end{entry}

\begin{entry}{寂寥}{11,14}
  \begin{phonetics}{寂寥}{ji4liao2}
    \definition{s.}{solidão | vasto e vazio | quieto e desolado (literário)}
  \end{phonetics}
\end{entry}

\begin{entry}{寄}{11}[Radical 宀]
  \begin{phonetics}{寄}{ji4}
    \definition{v.}{enviar | mandar}
  \end{phonetics}
\end{entry}

\begin{entry}{寄予}{11,4}
  \begin{phonetics}{寄予}{ji4yu3}
    \definition{v.}{expressar | colocar (esperança, importância, etc.) em | mostrar}
  \end{phonetics}
\end{entry}

\begin{entry}{寄生}{11,5}
  \begin{phonetics}{寄生}{ji4sheng1}
    \definition{s.}{parasita | parasitismo}
    \definition{v.}{viver tirando vantagem dos outros | viver dentro ou sobre outro organismo como um parasita}
  \end{phonetics}
\end{entry}

\begin{entry}{寄生生活}{11,5,5,9}
  \begin{phonetics}{寄生生活}{ji4sheng1sheng1huo2}
    \definition{s.}{parasitismo | vida parasitária}
  \end{phonetics}
\end{entry}

\begin{entry}{寄存}{11,6}
  \begin{phonetics}{寄存}{ji4cun2}
    \definition{v.}{depositar | deixar algo com alguém | armazenar}
  \end{phonetics}
\end{entry}

\begin{entry}{寄托}{11,6}
  \begin{phonetics}{寄托}{ji4tuo1}
    \definition{v.}{investir (sua esperança, energia, etc.) em algo | confiar (a alguém) | colocar (a esperança, a energia, etc.) em}
  \end{phonetics}
\end{entry}

\begin{entry}{寄卖}{11,8}
  \begin{phonetics}{寄卖}{ji4mai4}
    \definition{v.}{consignar para venda}
  \end{phonetics}
\end{entry}

\begin{entry}{寄居}{11,8}
  \begin{phonetics}{寄居}{ji4ju1}
    \definition{s.}{morar longe de casa}
  \end{phonetics}
\end{entry}

\begin{entry}{寄放}{11,8}
  \begin{phonetics}{寄放}{ji4fang4}
    \definition{v.}{deixar algo com alguém}
  \end{phonetics}
\end{entry}

\begin{entry}{寄养}{11,9}
  \begin{phonetics}{寄养}{ji4yang3}
    \definition{v.}{embarcar | promover | colocar sob os cuidados de alguém (uma criança, animal de estimação, etc.)}
  \end{phonetics}
\end{entry}

\begin{entry}{寄送}{11,9}
  \begin{phonetics}{寄送}{ji4song4}
    \definition{v.}{enviar | transmitir}
  \end{phonetics}
\end{entry}

\begin{entry}{寄递}{11,10}
  \begin{phonetics}{寄递}{ji4di4}
    \definition{s.}{entrega de correspondência}
  \end{phonetics}
\end{entry}

\begin{entry}{寄售}{11,11}
  \begin{phonetics}{寄售}{ji4shou4}
    \definition{v.}{venda em consignação}
  \end{phonetics}
\end{entry}

\begin{entry}{寄宿}{11,11}
  \begin{phonetics}{寄宿}{ji4su4}
    \definition{s.}{embarque}
    \definition{v.}{embarcar}
  \end{phonetics}
\end{entry}

\begin{entry}{寄望}{11,11}
  \begin{phonetics}{寄望}{ji4wang4}
    \definition{v.}{depositar esperanças em}
  \end{phonetics}
\end{entry}

\begin{entry}{密切}{11,4}
  \begin{phonetics}{密切}{mi4qie4}
    \definition{adj.}{perto | familiar | íntimo}
    \definition{v.}{promover laços estreitos (relacionamento) | prestar muita atenção}
  \end{phonetics}
\end{entry}

\begin{entry}{崇}{11}[Radical ⼭]
  \begin{phonetics}{崇}{chong2}
    \definition*{s.}{sobrenome Chong}
    \definition{adj.}{alto | sublime | elevado}
    \definition{v.}{estimar | adorar}
  \end{phonetics}
\end{entry}

\begin{entry}{崖}{11}[Radical 山]
  \begin{phonetics}{崖}{ya2}
    \definition{s.}{precipício | penhasco}
  \end{phonetics}
\end{entry}

\begin{entry}{崩}{11}[Radical 山]
  \begin{phonetics}{崩}{beng1}
    \definition{s.}{morte de rei ou imperador | desaparecimento}
    \definition{v.}{entrar em colapso | cair em ruínas}
  \end{phonetics}
\end{entry}

\begin{entry}{巢}{11}[Radical ⼮]
  \begin{phonetics}{巢}{chao2}
    \definition*{s.}{sobrenome Chao}
    \definition{s.}{ninho (de aves, etc.)}
  \end{phonetics}
\end{entry}

\begin{entry}{常}{11}[Radical 巾]
  \begin{phonetics}{常}{chang2}
    \definition*{s.}{sobrenome Chang}
    \definition{adv.}{muitas vezes | frequentemente}
  \end{phonetics}
\end{entry}

\begin{entry}{常问问题}{11,6,6,15}
  \begin{phonetics}{常问问题}{chang2wen4wen4ti2}
    \definition{s.}{FAQ; perguntas frequentes}
  \end{phonetics}
\end{entry}

\begin{entry}{常常}{11,11}
  \begin{phonetics}{常常}{chang2chang2}
    \definition{adv.}{frequentemente | com frequência}
  \end{phonetics}
\end{entry}

\begin{entry}{彩虹}{11,9}
  \begin{phonetics}{彩虹}{cai3hong2}
    \definition[道]{s.}{arco-íris}
  \end{phonetics}
\end{entry}

\begin{entry}{得}{11}[Radical 彳]
  \begin{phonetics}{得}{de2}
    \definition{v.}{obter | ganhar | pegar (uma doença)}
  \end{phonetics}
  \begin{phonetics}{得}{de5}
    \definition{part.}{(estrutural) ligando um verbo à frase seguinte indicando efeito, grau, possibilidade, etc.}
  \end{phonetics}
  \begin{phonetics}{得}{dei3}
    \definition{v.}{haver de | ter de}
  \end{phonetics}
\end{entry}

\begin{entry}{得了}{11,2}
  \begin{phonetics}{得了}{de2le5}
    \definition{expr.}{Tudo bem!; É o bastante!}
  \end{phonetics}
  \begin{phonetics}{得了}{de2liao3}
    \definition{adj.}{(enfaticamente, em perguntas retóricas) possível}
  \end{phonetics}
\end{entry}

\begin{entry}{得到}{11,8}
  \begin{phonetics}{得到}{de2dao4}
    \definition{v.}{obter | receber}
  \end{phonetics}
\end{entry}

\begin{entry}{得意}{11,13}
  \begin{phonetics}{得意}{de2yi4}
    \definition{adj.}{orgulhoso de si mesmo | satisfeito consigo mesmo | complacente}
    \definition{v.+compl.}{orgulhar-se de si mesmo; ter satisfação consigo mesmo; ser complacente}
  \end{phonetics}
\end{entry}

\begin{entry}{悉心}{11,4}
  \begin{phonetics}{悉心}{xi1xin1}
    \definition{adv.}{colocar o coração (e a alma) em algo | com muito cuidado}
  \end{phonetics}
\end{entry}

\begin{entry}{悉尼}{11,5}
  \begin{phonetics}{悉尼}{xi1ni2}
    \definition*{s.}{Sidney}
  \end{phonetics}
\end{entry}

\begin{entry}{悉数}{11,13}
  \begin{phonetics}{悉数}{xi1shu3}
    \definition{adv.}{enumerar em detalhes | explicar claramente}
  \end{phonetics}
  \begin{phonetics}{悉数}{xi1shu4}
    \definition{adv.}{todos | cada um | toda a soma}
  \end{phonetics}
\end{entry}

\begin{entry}{您}{11}[Radical 心]
  \begin{phonetics}{您}{nin2}
    \definition{pron.}{você (formal) | tu | te | ti | contigo}
    \seeref{你}{ni3}
  \end{phonetics}
\end{entry}

\begin{entry}{悬挂}{11,9}
  \begin{phonetics}{悬挂}{xuan2gua4}
    \definition{s.}{(veículo) suspensão}
    \definition{v.}{suspender}
  \end{phonetics}
\end{entry}

\begin{entry}{悬崖}{11,11}
  \begin{phonetics}{悬崖}{xuan2ya2}
    \definition{s.}{precipício | penhasco}
  \end{phonetics}
\end{entry}

\begin{entry}{情况}{11,7}
  \begin{phonetics}{情况}{qing2kuang4}
    \definition[个,种]{s.}{circunstância | situação | estado das coisas}
  \end{phonetics}
\end{entry}

\begin{entry}{情绪}{11,11}
  \begin{phonetics}{情绪}{qing2xu4}
    \definition[种]{s.}{humor | estado da mente | mau humor}
  \end{phonetics}
\end{entry}

\begin{entry}{情感}{11,13}
  \begin{phonetics}{情感}{qing2gan3}
    \definition{s.}{sentimento | emoção}
    \definition{v.}{mover-se (emocionalmente)}
  \end{phonetics}
\end{entry}

\begin{entry}{惊呆}{11,7}
  \begin{phonetics}{惊呆}{jing1dai1}
    \definition{adj.}{estupefato | chocado}
  \end{phonetics}
\end{entry}

\begin{entry}{惊喜}{11,12}
  \begin{phonetics}{惊喜}{jing1xi3}
    \definition{s.}{boa surpresa}
    \definition{v.}{ser agradavelmente surpreendido}
  \end{phonetics}
\end{entry}

\begin{entry}{惨}{11}[Radical 心]
  \begin{phonetics}{惨}{can3}
    \definition{adj.}{miserável | cruel | desumano | desastroso | trágico | sombrio}
  \end{phonetics}
\end{entry}

\begin{entry}{捷径}{11,8}
  \begin{phonetics}{捷径}{jie2jing4}
    \definition{s.}{atalho}
  \end{phonetics}
\end{entry}

\begin{entry}{掉}{11}[Radical 手]
  \begin{phonetics}{掉}{diao4}
    \definition{v.}{cair | deixar cair}
  \end{phonetics}
\end{entry}

\begin{entry}{掉队}{11,4}
  \begin{phonetics}{掉队}{diao4dui4}
    \definition{v.}{abandonar | ficar para trás}
  \end{phonetics}
\end{entry}

\begin{entry}{掉包}{11,5}
  \begin{phonetics}{掉包}{diao4bao1}
    \definition{v.}{vender uma falsificação pelo artigo genuíno | roubar o item valioso de alguém e substituí-lo por um item de aparência semelhante, mas sem valor}
  \end{phonetics}
\end{entry}

\begin{entry}{掉线}{11,8}
  \begin{phonetics}{掉线}{diao4xian4}
    \definition{v.}{desconectar-se (da \emph{Internet})}
  \end{phonetics}
\end{entry}

\begin{entry}{掉转}{11,8}
  \begin{phonetics}{掉转}{diao4zhuan3}
    \definition{v.}{dar a volta}
  \end{phonetics}
\end{entry}

\begin{entry}{掉落}{11,12}
  \begin{phonetics}{掉落}{diao4luo4}
    \definition{v.}{derrubar}
  \end{phonetics}
\end{entry}

\begin{entry}{掉膘}{11,15}
  \begin{phonetics}{掉膘}{diao4biao1}
    \definition{v.}{perder peso (gado)}
  \end{phonetics}
\end{entry}

\begin{entry}{排水}{11,4}
  \begin{phonetics}{排水}{pai2shui3}
    \definition{v.}{drenar}
  \end{phonetics}
\end{entry}

\begin{entry}{排队}{11,4}
  \begin{phonetics}{排队}{pai2dui4}
    \definition{v.+compl.}{formar uma fila | alinhar | listar | classificar}
  \end{phonetics}
\end{entry}

\begin{entry}{排球}{11,11}
  \begin{phonetics}{排球}{pai2qiu2}
    \definition[个]{s.}{voleibol}
  \end{phonetics}
\end{entry}

\begin{entry}{探亲}{11,9}
  \begin{phonetics}{探亲}{tan4qin1}
    \definition{v.+compl.}{ir para casa para visitar a família}
  \end{phonetics}
\end{entry}

\begin{entry}{接}{11}[Radical 手]
  \begin{phonetics}{接}{jie1}
    \definition{v.}{ir buscar (alguém) |  ir ao encontro de (alguém) | receber}
  \end{phonetics}
\end{entry}

\begin{entry}{接(电话)}{11,5,8}
  \begin{phonetics}{接(电话)}{jie1(dian4hua4)}
    \definition{v.}{atender (o telefone) | receber (uma ligação telefônica)}
  \end{phonetics}
\end{entry}

\begin{entry}{接待}{11,9}
  \begin{phonetics}{接待}{jie1dai4}
    \definition{v.}{receber (alguém) | acolher | recepcionar}
  \end{phonetics}
\end{entry}

\begin{entry}{接班人}{11,10,2}
  \begin{phonetics}{接班人}{jie1ban1ren2}
    \definition{s.}{sucessor}
  \end{phonetics}
\end{entry}

\begin{entry}{控制}{11,8}
  \begin{phonetics}{控制}{kong4zhi4}
    \definition{v.}{controlar}
  \end{phonetics}
\end{entry}

\begin{entry}{推介}{11,4}
  \begin{phonetics}{推介}{tui1jie4}
    \definition{s.}{promoção}
    \definition{v.}{promover | introduzir e recomendar}
  \end{phonetics}
\end{entry}

\begin{entry}{推迟}{11,7}
  \begin{phonetics}{推迟}{tui1chi2}
    \definition{v.}{adiar | deixar para mais tarde | tardar}
  \end{phonetics}
\end{entry}

\begin{entry}{敎}{11}
  \begin{phonetics}{敎}{jiao4}
    \variantof{教}
  \end{phonetics}
\end{entry}

\begin{entry}{救出}{11,5}
  \begin{phonetics}{救出}{jiu4chu1}
    \definition{v.}{resgatar | tirar do perigo}
  \end{phonetics}
\end{entry}

\begin{entry}{救护车}{11,7,4}
  \begin{phonetics}{救护车}{jiu4hu4che1}
    \definition[辆]{s.}{ambulância}
  \end{phonetics}
\end{entry}

\begin{entry}{救命}{11,8}
  \begin{phonetics}{救命}{jiu4ming4}
    \definition{interj.}{Socorro! | Salve-me!}
    \definition{v.+compl.}{salvar a vida de alguém}
  \end{phonetics}
\end{entry}

\begin{entry}{教}{11}[Radical 攴]
  \begin{phonetics}{教}{jiao1}
    \definition{v.}{ensinar | lecionar}
  \end{phonetics}
  \begin{phonetics}{教}{jiao4}
    \definition*{s.}{sobrenome Jiao}
    \definition{s.}{religião | ensinamento}
    \definition{v.}{causar | como fazer algo | contar (explicar como fazer algo)}
  \end{phonetics}
\end{entry}

\begin{entry}{教长}{11,4}
  \begin{phonetics}{教长}{jiao4zhang3}
    \definition{s.}{imã (Islã) | mulá}
  \end{phonetics}
\end{entry}

\begin{entry}{教会}{11,6}
  \begin{phonetics}{教会}{jiao1hui4}
    \definition{v.}{mostrar | ensinar}
  \end{phonetics}
  \begin{phonetics}{教会}{jiao4hui4}
    \definition{s.}{igreja cristã}
  \end{phonetics}
\end{entry}

\begin{entry}{教导}{11,6}
  \begin{phonetics}{教导}{jiao4dao3}
    \definition{s.}{instrução | orientação | ensino}
    \definition{v.}{instruir | orientar | ensinar}
  \end{phonetics}
\end{entry}

\begin{entry}{教师}{11,6}
  \begin{phonetics}{教师}{jiao4shi1}
    \definition[个]{s.}{professor | mestre}
  \end{phonetics}
\end{entry}

\begin{entry}{教学}{11,8}
  \begin{phonetics}{教学}{jiao1xue2}
    \definition{v.}{ensinar (como um professor)}
  \end{phonetics}
  \begin{phonetics}{教学}{jiao4xue2}
    \definition[次]{s.}{ensino | instrução}
  \end{phonetics}
\end{entry}

\begin{entry}{教学楼}{11,8,13}
  \begin{phonetics}{教学楼}{jiao4xue2lou2}
    \definition{s.}{edifício de salas de aula}
  \end{phonetics}
\end{entry}

\begin{entry}{教官}{11,8}
  \begin{phonetics}{教官}{jiao4guan1}
    \definition{s.}{instrutor militar}
  \end{phonetics}
\end{entry}

\begin{entry}{教练}{11,8}
  \begin{phonetics}{教练}{jiao4lian4}
    \definition[个,位,名]{s.}{instrutor | treinador (esportes)}
  \end{phonetics}
\end{entry}

\begin{entry}{教室}{11,9}
  \begin{phonetics}{教室}{jiao4shi4}
    \definition[间]{s.}{sala de aula}
  \end{phonetics}
\end{entry}

\begin{entry}{教堂}{11,11}
  \begin{phonetics}{教堂}{jiao4tang2}
    \definition[间]{s.}{igreja | capela}
  \end{phonetics}
\end{entry}

\begin{entry}{教授}{11,11}
  \begin{phonetics}{教授}{jiao4shou4}
    \definition[个,位]{s.}{professor (universitário)}
    \definition{v.}{instruir | palestrar sobre}
  \end{phonetics}
\end{entry}

\begin{entry}{敢情}{11,11}
  \begin{phonetics}{敢情}{gan3qing5}
    \definition{adv.}{claro | acontece que\dots}
  \end{phonetics}
\end{entry}

\begin{entry}{斜阳}{11,6}
  \begin{phonetics}{斜阳}{xie2yang2}
    \definition{s.}{sol poente}
  \end{phonetics}
\end{entry}

\begin{entry}{断交}{11,6}
  \begin{phonetics}{断交}{duan4jiao1}
    \definition{v.+compl.}{terminar uma amizade | romper relações diplomáticas}
  \end{phonetics}
\end{entry}

\begin{entry}{旋转}{11,8}
  \begin{phonetics}{旋转}{xuan2zhuan3}
    \definition{v.}{girar}
  \end{phonetics}
\end{entry}

\begin{entry}{族}{11}[Radical 方]
  \begin{phonetics}{族}{zu2}
    \definition{s.}{raça | nacionalidade | etnia | clã | por extensão, grupo social}
  \end{phonetics}
\end{entry}

\begin{entry}{旣}{11}
  \begin{phonetics}{旣}{ji4}
    \variantof{既}
  \end{phonetics}
\end{entry}

\begin{entry}{晚}{11}[Radical 日]
  \begin{phonetics}{晚}{wan3}
    \definition{adj.}{tarde | noite}
  \end{phonetics}
\end{entry}

\begin{entry}{晚上}{11,3}
  \begin{phonetics}{晚上}{wan3shang5}
    \definition{adv.}{noite | à noite}
  \end{phonetics}
\end{entry}

\begin{entry}{晚会}{11,6}
  \begin{phonetics}{晚会}{wan3hui4}
    \definition[个]{s.}{festa noturna}
  \end{phonetics}
\end{entry}

\begin{entry}{晚报}{11,7}
  \begin{phonetics}{晚报}{wan3bao4}
    \definition{s.}{jornal da noite}
  \end{phonetics}
\end{entry}

\begin{entry}{晚近}{11,7}
  \begin{phonetics}{晚近}{wan3jin4}
    \definition{adj.}{recente | mais recente no passado}
    \definition{adv.}{ultimamente | recentemente}
  \end{phonetics}
\end{entry}

\begin{entry}{晚饭}{11,7}
  \begin{phonetics}{晚饭}{wan3fan4}
    \definition[份,顿,次,餐]{s.}{jantar}
  \end{phonetics}
\end{entry}

\begin{entry}{晚育}{11,8}
  \begin{phonetics}{晚育}{wan3yu4}
    \definition{s.}{parto tardio}
    \definition{v.}{ter um filho mais tarde}
  \end{phonetics}
\end{entry}

\begin{entry}{晚点}{11,9}
  \begin{phonetics}{晚点}{wan3dian3}
    \definition{adj.}{atrasado}
    \definition{s.}{jantar leve}
  \end{phonetics}
\end{entry}

\begin{entry}{晚景}{11,12}
  \begin{phonetics}{晚景}{wan3jing3}
    \definition{s.}{circunstâncias dos anos de declínio de alguém | cena noturna}
  \end{phonetics}
\end{entry}

\begin{entry}{晚餐}{11,16}
  \begin{phonetics}{晚餐}{wan3can1}
    \definition[份,顿,次]{s.}{jantar | refeição noturna}
  \end{phonetics}
\end{entry}

\begin{entry}{梦}{11}[Radical 木]
  \begin{phonetics}{梦}{meng4}
    \definition[场,个]{s.}{sonho}
    \definition{v.}{sonhar}
  \end{phonetics}
\end{entry}

\begin{entry}{梯恩梯}{11,10,11}
  \begin{phonetics}{梯恩梯}{ti1'en1ti1}
    \definition{s.}{(empréstimo linguístico) TNT, trinitrotolueno}
  \end{phonetics}
\end{entry}

\begin{entry}{检查}{11,9}
  \begin{phonetics}{检查}{jian3cha2}
    \definition[次]{s.}{inspeção}
    \definition{v.}{examinar | inspecionar}
  \end{phonetics}
\end{entry}

\begin{entry}{欲}{11}[Radical 欠]
  \begin{phonetics}{欲}{yu4}
    \definition{adj.}{desejo | apetite | paixão | luxúria | ganância}
    \definition{v.}{desejar}
  \end{phonetics}
\end{entry}

\begin{entry}{毫不费力}{11,4,9,2}
  \begin{phonetics}{毫不费力}{hao2bu2fei4li4}
    \definition{adj.}{sem esforço | não gastando o menor esforço}
  \end{phonetics}
\end{entry}

\begin{entry}{毫米}{11,6}
  \begin{phonetics}{毫米}{hao2mi3}
    \definition{s.}{milímetro}
  \end{phonetics}
\end{entry}

\begin{entry}{液体}{11,7}
  \begin{phonetics}{液体}{ye4ti3}
    \definition{adj./s.}{líquido}
  \end{phonetics}
\end{entry}

\begin{entry}{涵}{11}[Radical 水]
  \begin{phonetics}{涵}{han2}
    \definition{s.}{bueiro | galeria}
    \definition{v.}{conter | incluir | entupir}
  \end{phonetics}
\end{entry}

\begin{entry}{淀}{11}[Radical 水]
  \begin{phonetics}{淀}{dian4}
    \definition{adj.}{pantanoso}
    \definition{s.}{lago raso | pântano}
    \definition{v.}{formar sedimentos | precipitar}
  \end{phonetics}
\end{entry}

\begin{entry}{淋}{11}[Radical 水]
  \begin{phonetics}{淋}{lin2}
    \definition{v.}{borrifar | pingar | derramar | encharcar}
  \end{phonetics}
  \begin{phonetics}{淋}{lin4}
    \definition{s.}{gonorréia}
    \definition{v.}{filtrar | coar | drenar}
  \end{phonetics}
\end{entry}

\begin{entry}{淤泥}{11,8}
  \begin{phonetics}{淤泥}{yu1ni2}
    \definition{s.}{lodo}
  \end{phonetics}
\end{entry}

\begin{entry}{深}{11}[Radical 水]
  \begin{phonetics}{深}{shen1}
    \definition{adj.}{profundo}
  \end{phonetics}
\end{entry}

\begin{entry}{深夜}{11,8}
  \begin{phonetics}{深夜}{shen1ye4}
    \definition{adv.}{tarde da noite}
  \end{phonetics}
\end{entry}

\begin{entry}{深厚}{11,9}
  \begin{phonetics}{深厚}{shen1hou4}
    \definition{adj.}{profundo}
  \end{phonetics}
\end{entry}

\begin{entry}{深深}{11,11}
  \begin{phonetics}{深深}{shen1shen1}
    \definition{adj.}{profundo}
    \definition{adv.}{profundamente}
  \end{phonetics}
\end{entry}

\begin{entry}{混乱}{11,7}
  \begin{phonetics}{混乱}{hun4luan4}
    \definition{adj.}{confuso | caótico | desordenado}
    \definition{s.}{caos}
  \end{phonetics}
\end{entry}

\begin{entry}{混饭}{11,7}
  \begin{phonetics}{混饭}{hun4fan4}
    \definition{v.+compl.}{trabalhar para viver}
  \end{phonetics}
\end{entry}

\begin{entry}{清}{11}[Radical 水]
  \begin{phonetics}{清}{qing1}
    \definition*{s.}{sobrenome Qing}
    \definition{adj.}{claro | limpo (água, etc.) | tranquilo | quieto | puro | não corrompido | distinto}
    \definition{v.}{limpar | resolver (contas)}
  \end{phonetics}
\end{entry}

\begin{entry}{清彻}{11,7}
  \begin{phonetics}{清彻}{qing1che4}
    \variantof{清澈}
  \end{phonetics}
\end{entry}

\begin{entry}{清明节}{11,8,5}
  \begin{phonetics}{清明节}{qing1ming2jie2}
    \definition*{s.}{Dia Qingming, Dia dos Finados (uma das 24~divisões do ano solar no calendário lunar chinês:~dia~4 ou 5~de~abril solar)}
  \end{phonetics}
\end{entry}

\begin{entry}{清凉}{11,10}
  \begin{phonetics}{清凉}{qing1liang2}
    \definition{adj.}{fresco | refrescante | (roupa) ousada, reveladora}
  \end{phonetics}
\end{entry}

\begin{entry}{清唱}{11,11}
  \begin{phonetics}{清唱}{qing1chang4}
    \definition{v.}{cantar à capela}
  \end{phonetics}
\end{entry}

\begin{entry}{清爽}{11,11}
  \begin{phonetics}{清爽}{qing1shuang3}
    \definition{adj.}{refrescante | relaxado}
  \end{phonetics}
\end{entry}

\begin{entry}{清理}{11,11}
  \begin{phonetics}{清理}{qing1li3}
    \definition{v.}{limpar | arrumar | descartar}
  \end{phonetics}
\end{entry}

\begin{entry}{清晰}{11,12}
  \begin{phonetics}{清晰}{qing1xi1}
    \definition{adj.}{claro | distinto}
  \end{phonetics}
\end{entry}

\begin{entry}{清楚}{11,13}
  \begin{phonetics}{清楚}{qing1chu5}
    \definition{adj.}{claro | límpido}
    \definition{v.}{ser claro sobre | entender completamente}
  \end{phonetics}
\end{entry}

\begin{entry}{清澈}{11,15}
  \begin{phonetics}{清澈}{qing1che4}
    \definition{adj.}{claro | límpido}
  \end{phonetics}
\end{entry}

\begin{entry}{渐渐}{11,11}
  \begin{phonetics}{渐渐}{jian4jian4}
    \definition{adv.}{pouco a pouco | passo a passo | progressivamente}
  \end{phonetics}
\end{entry}

\begin{entry}{渔}{11}[Radical 水]
  \begin{phonetics}{渔}{yu2}
    \definition[条]{s.}{pescador}
    \definition{v.}{pescar}
  \end{phonetics}
\end{entry}

\begin{entry}{渔夫}{11,4}
  \begin{phonetics}{渔夫}{yu2fu1}
    \definition{s.}{pescador}
  \end{phonetics}
\end{entry}

\begin{entry}{渔民}{11,5}
  \begin{phonetics}{渔民}{yu2min2}
    \definition{s.}{pescadores | povo pescador}
  \end{phonetics}
\end{entry}

\begin{entry}{渔场}{11,6}
  \begin{phonetics}{渔场}{yu2chang3}
    \definition{s.}{área de pesca}
  \end{phonetics}
\end{entry}

\begin{entry}{渔汛}{11,6}
  \begin{phonetics}{渔汛}{yu2xun4}
    \definition{s.}{temporada de pesca}
  \end{phonetics}
\end{entry}

\begin{entry}{渔网}{11,6}
  \begin{phonetics}{渔网}{yu2wang3}
    \definition{s.}{rede de pesca}
  \end{phonetics}
\end{entry}

\begin{entry}{渔具}{11,8}
  \begin{phonetics}{渔具}{yu2ju4}
    \definition{s.}{equipamento de pesca}
  \end{phonetics}
\end{entry}

\begin{entry}{渔轮}{11,8}
  \begin{phonetics}{渔轮}{yu2lun2}
    \definition{s.}{navio de pesca}
  \end{phonetics}
\end{entry}

\begin{entry}{渔捞}{11,10}
  \begin{phonetics}{渔捞}{yu2lao1}
    \definition{s.}{pesca (como atividade comercial)}
  \end{phonetics}
\end{entry}

\begin{entry}{渔猎}{11,11}
  \begin{phonetics}{渔猎}{yu2lie4}
    \definition{s.}{pesca e caça}
    \definition{v.}{saquear | pilhar}
  \end{phonetics}
\end{entry}

\begin{entry}{渔笼}{11,11}
  \begin{phonetics}{渔笼}{yu2long2}
    \definition{s.}{gaiola de pesca | armadilha de pesca}
  \end{phonetics}
\end{entry}

\begin{entry}{渔船}{11,11}
  \begin{phonetics}{渔船}{yu2chuan2}
    \definition[条]{s.}{barco de pesca}
  \seealsoref{鱼船}{yu2chuan2}
  \end{phonetics}
\end{entry}

\begin{entry}{渔船队}{11,11,4}
  \begin{phonetics}{渔船队}{yu2chuan2dui4}
    \definition{s.}{frota pesqueira}
  \end{phonetics}
\end{entry}

\begin{entry}{焊}{11}[Radical 火]
  \begin{phonetics}{焊}{han4}
    \definition{v.}{soldar}
  \end{phonetics}
\end{entry}

\begin{entry}{猎物}{11,8}
  \begin{phonetics}{猎物}{lie4wu4}
    \definition{s.}{presa (vítima de um predador)}
  \end{phonetics}
\end{entry}

\begin{entry}{猛}{11}[Radical 犬]
  \begin{phonetics}{猛}{meng3}
    \definition{adj.}{feroz | violento | corajoso | abrupto | (gíria) incrível}
    \definition{adv.}{de repente}
  \end{phonetics}
\end{entry}

\begin{entry}{猛然}{11,12}
  \begin{phonetics}{猛然}{meng3ran2}
    \definition{adv.}{de repente | abruptamente}
  \end{phonetics}
\end{entry}

\begin{entry}{猜}{11}[Radical 犬]
  \begin{phonetics}{猜}{cai1}
    \definition{v.}{advinhar}
  \end{phonetics}
\end{entry}

\begin{entry}{猪}{11}[Radical 犬]
  \begin{phonetics}{猪}{zhu1}
    \definition[口,头]{s.}{porco | suíno}
  \end{phonetics}
\end{entry}

\begin{entry}{猪头}{11,5}
  \begin{phonetics}{猪头}{zhu1tou2}
    \definition{s.}{tolo | idiota}
  \end{phonetics}
\end{entry}

\begin{entry}{猪柳}{11,9}
  \begin{phonetics}{猪柳}{zhu1liu3}
    \definition{s.}{filé de porco}
  \end{phonetics}
\end{entry}

\begin{entry}{猪笼}{11,11}
  \begin{phonetics}{猪笼}{zhu1long2}
    \definition{s.}{estrutura cilíndrica de bambu ou arame usada para restringir um porco durante o transporte}
  \end{phonetics}
\end{entry}

\begin{entry}{猪窠}{11,13}
  \begin{phonetics}{猪窠}{zhu1ke1}
    \definition{s.}{chiqueiro}
  \end{phonetics}
\end{entry}

\begin{entry}{猫}{11}[Radical 犬]
  \begin{phonetics}{猫}{mao1}
    \definition[只]{s.}{gato |  (empréstimo linguístico) (coloquial) MODEM}
    \definition{v.}{(dialeto) esconder-se}
  \end{phonetics}
  \begin{phonetics}{猫}{mao2}
    \definition{v.}{utilizado em 猫腰 \dpy{mao2yao1}}
    \seeref{猫腰}{mao2yao1}
  \end{phonetics}
\end{entry}

\begin{entry}{猫腰}{11,13}
  \begin{phonetics}{猫腰}{mao2yao1}
    \definition{v.}{curvar-se}
  \end{phonetics}
\end{entry}

\begin{entry}{猫熊}{11,14}
  \begin{phonetics}{猫熊}{mao1xiong2}
    \definition[把,只]{s.}{panda gigante}
  \seealsoref{熊猫}{xiong2mao1}
  \end{phonetics}
\end{entry}

\begin{entry}{球}{11}[Radical 玉]
  \begin{phonetics}{球}{qiu2}
    \definition[个]{s.}{bola | esfera | globo}
    \definition[场]{s.}{jogo | partida de bola}
  \end{phonetics}
\end{entry}

\begin{entry}{球拍}{11,8}
  \begin{phonetics}{球拍}{qiu2pai1}
    \definition{s.}{raquete}
  \end{phonetics}
\end{entry}

\begin{entry}{球迷}{11,9}
  \begin{phonetics}{球迷}{qiu2mi2}
    \definition[个]{s.}{fã (esportes de bola)}
  \end{phonetics}
\end{entry}

\begin{entry}{理发}{11,5}
  \begin{phonetics}{理发}{li3fa4}
    \definition{v.+compl.}{fazer um corte de cabelo | cortar o cabelo de alguém}
  \end{phonetics}
\end{entry}

\begin{entry}{理由}{11,5}
  \begin{phonetics}{理由}{li3you2}
    \definition[个]{s.}{razão | justificativa}
  \end{phonetics}
\end{entry}

\begin{entry}{甜}{11}[Radical 甘]
  \begin{phonetics}{甜}{tian2}
    \definition{adj.}{doce}
  \end{phonetics}
\end{entry}

\begin{entry}{甜心}{11,4}
  \begin{phonetics}{甜心}{tian2xin1}
    \definition{s.}{querido}
  \end{phonetics}
\end{entry}

\begin{entry}{甜头}{11,5}
  \begin{phonetics}{甜头}{tian2tou5}
    \definition{s.}{benefício | sabor doce (de poder, sucesso, etc.)}
  \end{phonetics}
\end{entry}

\begin{entry}{甜玉米}{11,5,6}
  \begin{phonetics}{甜玉米}{tian2 yu4mi3}
    \definition{s.}{milho doce}
  \end{phonetics}
\end{entry}

\begin{entry}{甜言}{11,7}
  \begin{phonetics}{甜言}{tian2yan2}
    \definition{s.}{boa conversa | palavras amáveis}
  \end{phonetics}
\end{entry}

\begin{entry}{甜品}{11,9}
  \begin{phonetics}{甜品}{tian2pin3}
    \definition{s.}{sobremesa}
  \end{phonetics}
\end{entry}

\begin{entry}{甜食}{11,9}
  \begin{phonetics}{甜食}{tian2shi2}
    \definition{s.}{doces | sobremesa}
  \end{phonetics}
\end{entry}

\begin{entry}{甜酒}{11,10}
  \begin{phonetics}{甜酒}{tian2jiu3}
    \definition{s.}{licor doce}
  \end{phonetics}
\end{entry}

\begin{entry}{甜甜圈}{11,11,11}
  \begin{phonetics}{甜甜圈}{tian2tian2quan1}
    \definition{s.}{rosquinha | \emph{doughnut}}
  \end{phonetics}
\end{entry}

\begin{entry}{甜菊}{11,11}
  \begin{phonetics}{甜菊}{tian2ju2}
    \definition{s.}{estévia, arbusto cujas folhas produzem um substituto para o açúcar}
  \end{phonetics}
\end{entry}

\begin{entry}{甜筒}{11,12}
  \begin{phonetics}{甜筒}{tian2tong3}
    \definition{s.}{sorvete de casquinha}
  \end{phonetics}
\end{entry}

\begin{entry}{甜稚}{11,13}
  \begin{phonetics}{甜稚}{tian2zhi4}
    \definition{s.}{doce e inocente}
  \end{phonetics}
\end{entry}

\begin{entry}{甜酸}{11,14}
  \begin{phonetics}{甜酸}{tian2suan1}
    \definition{adj.}{agridoce}
  \end{phonetics}
\end{entry}

\begin{entry}{略}{11}[Radical 田]
  \begin{phonetics}{略}{lve4}
    \definition{adv.}{ligeiramente | marginalmente | aproximadamente}
  \end{phonetics}
\end{entry}

\begin{entry}{略微}{11,13}
  \begin{phonetics}{略微}{lve4wei1}
    \definition{adv.}{ligeiramente | marginalmente | aproximadamente}
  \end{phonetics}
\end{entry}

\begin{entry}{盒}{11}[Radical ⽫]
  \begin{phonetics}{盒}{he2}
    \definition{clas.}{caixa pequena}
    \definition{s.}{caixa pequena | estojo}
  \end{phonetics}
\end{entry}

\begin{entry}{盘}{11}[Radical 皿]
  \begin{phonetics}{盘}{pan2}
    \definition{clas.}{para bobinas de fio | (de comida) pratos, serviços | para jogos de xadrez}
    \definition{s.}{tabuleiro | prato | bandeja | (computação) disco rígido}
    \definition{v.}{construir | checar | enrolar | examinar | transferir (propriedade)}
  \end{phonetics}
\end{entry}

\begin{entry}{盛宴}{11,10}
  \begin{phonetics}{盛宴}{sheng4yan4}
    \definition{s.}{celebração}
  \end{phonetics}
\end{entry}

\begin{entry}{眯}{11}[Radical 目]
  \begin{phonetics}{眯}{mi1}
    \definition{v.}{estreitar os olhos | esmagar | (dialeto) tirar uma soneca}
  \end{phonetics}
  \begin{phonetics}{眯}{mi2}
    \definition{v.}{cegar (como com poeira)}
  \end{phonetics}
\end{entry}

\begin{entry}{眼}{11}[Radical 目]
  \begin{phonetics}{眼}{yan3}
    \definition{clas.}{para grandes coisas ocas: poços, fogões, panelas, etc.}
    \definition[只,双]{s.}{ponto crucial (de um assunto) | olho | pequeno buraco}
  \end{phonetics}
\end{entry}

\begin{entry}{眼花缭乱}{11,7,15,7}
  \begin{phonetics}{眼花缭乱}{yan3hua1liao2luan4}
    \definition{v.}{ficar deslumbrado | deslumbrar}
  \end{phonetics}
\end{entry}

\begin{entry}{眼证}{11,7}
  \begin{phonetics}{眼证}{yan3zheng4}
    \definition{s.}{testemunha ocular}
  \end{phonetics}
\end{entry}

\begin{entry}{眼泪}{11,8}
  \begin{phonetics}{眼泪}{yan3lei4}
    \definition[滴]{s.}{choro | lágrimas}
  \end{phonetics}
\end{entry}

\begin{entry}{眼柄}{11,9}
  \begin{phonetics}{眼柄}{yan3bing3}
    \definition{s.}{pedúnculo ocular (de crustáceo, etc.)}
  \end{phonetics}
\end{entry}

\begin{entry}{眼袋}{11,11}
  \begin{phonetics}{眼袋}{yan3dai4}
    \definition{s.}{inchaço sob os olhos}
  \end{phonetics}
\end{entry}

\begin{entry}{眼睛}{11,13}
  \begin{phonetics}{眼睛}{yan3jing5}
    \definition[只,双]{s.}{olho(s)}
  \end{phonetics}
\end{entry}

\begin{entry}{眼镜}{11,16}
  \begin{phonetics}{眼镜}{yan3jing4}
    \definition[副]{s.}{óculos}
  \end{phonetics}
\end{entry}

\begin{entry}{着}{11}[Radical 目]
  \begin{phonetics}{着}{zhao1}
    \definition{interj.}{Tudo bem!}
    \definition{s.}{movimento (xadrez) | truque}
  \end{phonetics}
  \begin{phonetics}{着}{zhao2}
    \definition{v.}{ser afetado por | queimar | pegar fogo | entrar em contato com | sentir | tocar}
  \end{phonetics}
  \begin{phonetics}{着}{zhe5}
    \definition{part.}{indicando ação em andamento ou estado em andamento}
  \end{phonetics}
  \begin{phonetics}{着}{zhuo2}
    \definition{v.}{aplicar | contactar | usar | vestir (roupas)}
  \end{phonetics}
\end{entry}

\begin{entry}{着手}{11,4}
  \begin{phonetics}{着手}{zhuo2shou3}
    \definition{v.}{colocar a mão nisso | estabelecer | começar uma tarefa}
  \end{phonetics}
\end{entry}

\begin{entry}{着地}{11,6}
  \begin{phonetics}{着地}{zhao2di4}
    \definition{v.}{pousar | tocar o chão}
  \end{phonetics}
\end{entry}

\begin{entry}{着花}{11,7}
  \begin{phonetics}{着花}{zhao2hua1}
    \definition{v.}{florescer}
  \end{phonetics}
  \begin{phonetics}{着花}{zhuo2hua1}
    \definition{s.}{floração}
    \definition{v.}{florescer}
  \end{phonetics}
\end{entry}

\begin{entry}{着急}{11,9}
  \begin{phonetics}{着急}{zhao2ji2}
    \definition{adj.}{inquieto | ansioso}
    \definition{s.}{preocupação | ansiedade}
    \definition{v.+compl.}{preocupar-se | sentir-se ansioso | sentir uma sensação de urgência}
  \end{phonetics}
\end{entry}

\begin{entry}{着凉}{11,10}
  \begin{phonetics}{着凉}{zhao2liang2}
    \definition{v.}{pegar um resfriado}
  \end{phonetics}
\end{entry}

\begin{entry}{着眼}{11,11}
  \begin{phonetics}{着眼}{zhuo2yan3}
    \definition{v.}{ter seus olhos em (um objetivo) | ter algo em mente | concentrar-se}
  \end{phonetics}
\end{entry}

\begin{entry}{着装}{11,12}
  \begin{phonetics}{着装}{zhuo2zhuang1}
    \definition{s.}{roupa | vestimenta}
    \definition{v.}{vestir}
  \end{phonetics}
\end{entry}

\begin{entry}{着想}{11,13}
  \begin{phonetics}{着想}{zhuo2xiang3}
    \definition{v.}{considerar (as necessidades de outras pessoas) | pensar (para os outros)}
  \end{phonetics}
\end{entry}

\begin{entry}{着数}{11,13}
  \begin{phonetics}{着数}{zhao1shu4}
    \definition{s.}{estratégia | movimento (no xadrez, no palco, nas artes marciais) | esquema | truque}
  \end{phonetics}
\end{entry}

\begin{entry}{票}{11}[Radical 示]
  \begin{phonetics}{票}{piao4}
    \definition{clas.}{para grupos, lotes, transações comerciais}
    \definition[张]{s.}{performance amadora de ópera chinesa | cédula eleitoral | nota | bilhete | pessoa detida por resgate}
  \end{phonetics}
\end{entry}

\begin{entry}{章}{11}[Radical 音]
  \begin{phonetics}{章}{zhang1}
    \definition*{s.}{sobrenome Zhang}
    \definition{s.}{capítulo | seção | cláusula |  movimento (de sinfonia) | selo | crachá | regulamento}
  \end{phonetics}
\end{entry}

\begin{entry}{章鱼}{11,8}
  \begin{phonetics}{章鱼}{zhang1yu2}
    \definition{s.}{polvo | octópode}
  \end{phonetics}
\end{entry}

\begin{entry}{笛}{11}[Radical 竹]
  \begin{phonetics}{笛}{di2}
    \definition{s.}{flauta}
  \end{phonetics}
\end{entry}

\begin{entry}{符合}{11,6}
  \begin{phonetics}{符合}{fu2he2}
    \definition{conj.}{de acordo com | concordando com | contando com | alinhado com}
    \definition{v.}{concordar com | estar em conformidade com | corresponder com | gerenciar | lidar}
  \end{phonetics}
\end{entry}

\begin{entry}{笨蛋}{11,11}
  \begin{phonetics}{笨蛋}{ben4dan4}
    \definition{s.}{bobalhão | cabeça-oca | cabeça-dura}
    \definition{v.}{iludir | enganar}
  \end{phonetics}
\end{entry}

\begin{entry}{第}{11}[Radical 竹]
  \begin{phonetics}{第}{di4}
    \definition{num.}{prefixo para expressar números ordinais}
  \end{phonetics}
\end{entry}

\begin{entry}{笼}{11}[Radical 竹]
  \begin{phonetics}{笼}{long2}
    \definition{s.}{armação fechada de bambu, arame, etc. | jaula | gaiola}
  \end{phonetics}
  \begin{phonetics}{笼}{long3}
    \definition{v.}{envolver | cobrir}
  \end{phonetics}
\end{entry}

\begin{entry}{笼子}{11,3}
  \begin{phonetics}{笼子}{long2zi5}
    \definition{s.}{jaula | cesta | gaiola | recipiente}
  \end{phonetics}
  \begin{phonetics}{笼子}{long3zi5}
    \definition{s.}{caixa grande | porta-malas}
  \end{phonetics}
\end{entry}

\begin{entry}{粗心}{11,4}
  \begin{phonetics}{粗心}{cu1xin1}
    \definition{adj.}{descuido}
  \end{phonetics}
\end{entry}

\begin{entry}{粗心地做}{11,4,6,11}
  \begin{phonetics}{粗心地做}{cu1xin1 di4 zuo4}
    \definition{adj.}{feito descuidadamente}
  \end{phonetics}
\end{entry}

\begin{entry}{粗糙}{11,16}
  \begin{phonetics}{粗糙}{cu1cao1}
    \definition{adj.}{áspero | grosseiro}
  \end{phonetics}
\end{entry}

\begin{entry}{累}{11}[Radical 糸]
  \begin{phonetics}{累}{lei2}
    \definition*{s.}{sobrenome Lei}
    \definition{s.}{corda}
    \definition{v.}{amarrar | torcer}
  \end{phonetics}
  \begin{phonetics}{累}{lei3}
    \definition{adj.}{contínuo | repetido}
    \definition{v.}{acumular | envolver ou implicar}
  \end{phonetics}
  \begin{phonetics}{累}{lei4}
    \definition{adj.}{cansado | fatigado}
    \definition{v.}{forçar | desgastar | trabalhar duro}
  \end{phonetics}
\end{entry}

\begin{entry}{绰号}{11,5}
  \begin{phonetics}{绰号}{chuo4hao4}
    \definition{s.}{apelido}
  \end{phonetics}
\end{entry}

\begin{entry}{绳子}{11,3}
  \begin{phonetics}{绳子}{sheng2zi5}
    \definition[条]{s.}{corda | cordão}
  \end{phonetics}
\end{entry}

\begin{entry}{维吾尔}{11,7,5}
  \begin{phonetics}{维吾尔}{wei2wu2'er3}
    \definition*{s.}{Grupo étnico Uigur de Xinjiang}
  \end{phonetics}
\end{entry}

\begin{entry}{绷带}{11,9}
  \begin{phonetics}{绷带}{beng1dai4}
    \definition{s.}{curativo | (empréstimo linguístico) \emph{bandage}}
  \end{phonetics}
\end{entry}

\begin{entry}{绿}{11}[Radical 糸]
  \begin{phonetics}{绿}{lv4}
    \definition{adj.}{verde}
  \end{phonetics}
\end{entry}

\begin{entry}{绿色}{11,6}
  \begin{phonetics}{绿色}{lv4se4}
    \definition{s.}{cor verde}
  \end{phonetics}
\end{entry}

\begin{entry}{绿豆}{11,7}
  \begin{phonetics}{绿豆}{lv4dou4}
    \definition{s.}{vagens}
  \end{phonetics}
\end{entry}

\begin{entry}{绿豆芽}{11,7,7}
  \begin{phonetics}{绿豆芽}{lv4dou4 ya2}
    \definition{s.}{broto de feijão verde}
  \end{phonetics}
\end{entry}

\begin{entry}{聊天}{11,4}
  \begin{phonetics}{聊天}{liao2tian1}
    \definition{v.+compl.}{papear | bater papo}
  \end{phonetics}
\end{entry}

\begin{entry}{职业}{11,5}
  \begin{phonetics}{职业}{zhi2ye4}
    \definition{adj.}{profissional}
    \definition{s.}{ocupação | profissão | vocação}
  \end{phonetics}
\end{entry}

\begin{entry}{职员}{11,7}
  \begin{phonetics}{职员}{zhi2yuan2}
    \definition[个,位]{s.}{empregado | trabalhador de escritório | membro da equipe}
  \end{phonetics}
\end{entry}

\begin{entry}{脖子}{11,3}
  \begin{phonetics}{脖子}{bo2zi5}
    \definition[个]{s.}{pescoço}
  \end{phonetics}
\end{entry}

\begin{entry}{脚}{11}[Radical 肉]
  \begin{phonetics}{脚}{jiao3}
    \definition{clas.}{para chutes}
    \definition[双,只]{s.}{pé | base (de um objeto) | perna (de um animal ou objeto)}
  \end{phonetics}
  \begin{phonetics}{脚}{jue2}
    \variantof{角}
  \end{phonetics}
\end{entry}

\begin{entry}{脱毛}{11,4}
  \begin{phonetics}{脱毛}{tuo1mao2}
    \definition{s.}{depilação}
    \definition{v.}{perder cabelo ou penas | depilar | fazer a barba}
  \end{phonetics}
\end{entry}

\begin{entry}{脱险}{11,9}
  \begin{phonetics}{脱险}{tuo1xian3}
    \definition{v.}{sair do perigo}
  \end{phonetics}
\end{entry}

\begin{entry}{脸}{11}[Radical 肉]
  \begin{phonetics}{脸}{lian3}
    \definition[张,个]{s.}{cara | rosto | face}
  \end{phonetics}
\end{entry}

\begin{entry}{脸色}{11,6}
  \begin{phonetics}{脸色}{lian3se4}
    \definition{s.}{compleição; tez; face}
  \end{phonetics}
\end{entry}

\begin{entry}{船}{11}[Radical ⾈]
  \begin{phonetics}{船}{chuan2}
    \definition[条,艘,只]{s.}{barco | navio}
  \end{phonetics}
\end{entry}

\begin{entry}{菜}{11}[Radical 艸]
  \begin{phonetics}{菜}{cai4}
    \definition[棵]{s.}{hortaliça | verdura}
    \definition[样,道,盘]{s.}{prato (de comida)}
  \end{phonetics}
\end{entry}

\begin{entry}{菜刀}{11,2}
  \begin{phonetics}{菜刀}{cai4dao1}
    \definition[把]{s.}{faca de vegetais | faca de cozinha | cutelo}
  \end{phonetics}
\end{entry}

\begin{entry}{菜单}{11,8}
  \begin{phonetics}{菜单}{cai4dan1}
    \definition[份,张]{s.}{menu | cardápio}
  \end{phonetics}
\end{entry}

\begin{entry}{菠菜}{11,11}
  \begin{phonetics}{菠菜}{bo1cai4}
    \definition[棵]{s.}{espinafre}
  \end{phonetics}
\end{entry}

\begin{entry}{菱角}{11,7}
  \begin{phonetics}{菱角}{ling2jiao5}
    \definition{s.}{castanha d'água}
  \end{phonetics}
\end{entry}

\begin{entry}{虚伪}{11,6}
  \begin{phonetics}{虚伪}{xu1wei3}
    \definition{adj.}{falso | hipócrita | artificial}
  \end{phonetics}
\end{entry}

\begin{entry}{蛇}{11}[Radical 虫]
  \begin{phonetics}{蛇}{she2}
    \definition[条]{s.}{cobra | serpente}
  \end{phonetics}
\end{entry}

\begin{entry}{蛋}{11}[Radical 足]
  \begin{phonetics}{蛋}{dan4}
    \definition[个,打]{s.}{ovo | objeto de formato oval}
  \end{phonetics}
\end{entry}

\begin{entry}{蛋糕}{11,16}
  \begin{phonetics}{蛋糕}{dan4gao1}
    \definition[块,个]{s.}{bolo}
  \end{phonetics}
\end{entry}

\begin{entry}{袭击}{11,5}
  \begin{phonetics}{袭击}{xi2ji1}
    \definition{s.}{ataque (especialmente um ataque surpresa) | invasão}
    \definition{v.}{atacar}
  \end{phonetics}
\end{entry}

\begin{entry}{谎话}{11,8}
  \begin{phonetics}{谎话}{huang3hua4}
    \definition{s.}{mentira}
  \end{phonetics}
\end{entry}

\begin{entry}{谐}{11}[Radical 言]
  \begin{phonetics}{谐}{xie2}
    \definition{adj.}{harmonioso | humorístico}
  \end{phonetics}
\end{entry}

\begin{entry}{距离}{11,10}
  \begin{phonetics}{距离}{ju4li2}
    \definition[个]{s.}{distância}
    \definition{v.}{estar distante de}
  \end{phonetics}
\end{entry}

\begin{entry}{辆}{11}[Radical 車]
  \begin{phonetics}{辆}{liang4}
    \definition{clas.}{para automóveis, veículos, etc.}
  \end{phonetics}
\end{entry}

\begin{entry}{逮}{11}[Radical 辵]
  \begin{phonetics}{逮}{dai3}
    \definition{v.}{(coloquial) pegar, aproveitar, capturar}
  \end{phonetics}
  \begin{phonetics}{逮}{dai4}
    \definition{v.}{(literário) alcançar, usado em 逮捕}
  \seealsoref{逮捕}{dai4bu3}
  \end{phonetics}
\end{entry}

\begin{entry}{逮捕}{11,10}
  \begin{phonetics}{逮捕}{dai4bu3}
    \definition{v.}{prender | apreender | levar sob custódia}
  \end{phonetics}
\end{entry}

\begin{entry}{野}{11}[Radical 里]
  \begin{phonetics}{野}{ye3}
    \definition{adj.}{selvagem | rude}
    \definition{s.}{campo | espaço aberto | limite}
  \end{phonetics}
\end{entry}

\begin{entry}{野生}{11,5}
  \begin{phonetics}{野生}{ye3sheng1}
    \definition{adj.}{selvagem | não domesticado}
  \end{phonetics}
\end{entry}

\begin{entry}{铲车}{11,4}
  \begin{phonetics}{铲车}{chan3che1}
    \definition[台]{s.}{empilhadeira}
  \end{phonetics}
\end{entry}

\begin{entry}{银色}{11,6}
  \begin{phonetics}{银色}{yin2se4}
    \definition{s.}{prateado}
  \end{phonetics}
\end{entry}

\begin{entry}{银行}{11,6}
  \begin{phonetics}{银行}{yin2hang2}
    \definition[家,个]{s.}{banco | agência bancária}
  \end{phonetics}
\end{entry}

\begin{entry}{银河}{11,8}
  \begin{phonetics}{银河}{yin2he2}
    \definition*{s.}{Via Láctea}
  \seealsoref{银河系}{yin2he2xi4}
  \end{phonetics}
\end{entry}

\begin{entry}{银河系}{11,8,7}
  \begin{phonetics}{银河系}{yin2he2xi4}
    \definition*{s.}{Galáxia Via Láctea}
  \seealsoref{银河}{yin2he2}
  \end{phonetics}
\end{entry}

\begin{entry}{随处}{11,5}
  \begin{phonetics}{随处}{sui2chu4}
    \definition{adv.}{em qualquer lugar}
  \end{phonetics}
\end{entry}

\begin{entry}{随地}{11,6}
  \begin{phonetics}{随地}{sui2di4}
    \definition{adv.}{qualquer lugar | todo lugar}
  \end{phonetics}
\end{entry}

\begin{entry}{随机存取记忆体}{11,6,6,8,5,4,7}
  \begin{phonetics}{随机存取记忆体}{sui2ji1cun2qu3ji4yi4ti3}
    \definition{s.}{RAM (\emph{random access memory})}
  \seealsoref{内存}{nei4cun2}
  \seealsoref{随机存取存储器}{sui2ji1cun2qu3cun2chu3qi4}
  \end{phonetics}
\end{entry}

\begin{entry}{随机存取存储器}{11,6,6,8,6,12,16}
  \begin{phonetics}{随机存取存储器}{sui2ji1cun2qu3cun2chu3qi4}
    \definition{s.}{RAM (\emph{random access memory})}
  \seealsoref{内存}{nei4cun2}
  \seealsoref{随机存取记忆体}{sui2ji1cun2qu3ji4yi4ti3}
  \end{phonetics}
\end{entry}

\begin{entry}{随时}{11,7}
  \begin{phonetics}{随时}{sui2shi2}
    \definition{adv.}{a qualquer momento | sempre que necessário}
  \end{phonetics}
\end{entry}

\begin{entry}{随便}{11,9}
  \begin{phonetics}{随便}{sui2bian4}
    \definition{adj.}{à vontade | como queira | como desejar | casual | negligente | devasso}
    \definition{adv.}{aleatoriamente}
  \end{phonetics}
\end{entry}

\begin{entry}{雪}{11}[Radical 雨]
  \begin{phonetics}{雪}{xue3}
    \definition*{s.}{sobrenome Xue}
    \definition[场]{s.}{neve}
  \end{phonetics}
\end{entry}

\begin{entry}{雪人}{11,2}
  \begin{phonetics}{雪人}{xue3ren2}
    \definition{s.}{boneco de neve | \emph{Yeti}}
  \end{phonetics}
\end{entry}

\begin{entry}{雪山}{11,3}
  \begin{phonetics}{雪山}{xue3shan1}
    \definition{s.}{montanha coberta de neve}
  \end{phonetics}
\end{entry}

\begin{entry}{雪花}{11,7}
  \begin{phonetics}{雪花}{xue3hua1}
    \definition{s.}{floco de neve}
  \end{phonetics}
\end{entry}

\begin{entry}{雪板}{11,8}
  \begin{phonetics}{雪板}{xue3ban3}
    \definition{s.}{prancha de \emph{snowboard}}
    \definition{v.}{praticar \textit{snowboard}}
  \end{phonetics}
\end{entry}

\begin{entry}{雪葩}{11,12}
  \begin{phonetics}{雪葩}{xue3pa1}
    \definition{s.}{sorvete}
  \end{phonetics}
\end{entry}

\begin{entry}{雪鞋}{11,15}
  \begin{phonetics}{雪鞋}{xue3xie2}
    \definition[双]{s.}{sapatos de neve}
  \end{phonetics}
\end{entry}

\begin{entry}{雪糕}{11,16}
  \begin{phonetics}{雪糕}{xue3gao1}
    \definition{s.}{picolé}
  \end{phonetics}
\end{entry}

\begin{entry}{领}{11}[Radical 頁]
  \begin{phonetics}{领}{ling3}
    \definition{clas.}{para roupas, tapetes, telas, etc.}
    \definition{s.}{pescoço | colarinho}
    \definition{v.}{liderar | receber}
  \end{phonetics}
\end{entry}

\begin{entry}{领导}{11,6}
  \begin{phonetics}{领导}{ling3dao3}
    \definition[位,个]{s.}{líder | liderança}
    \definition{v.}{liderar}
  \end{phonetics}
\end{entry}

\begin{entry}{领情}{11,11}
  \begin{phonetics}{领情}{ling3qing2}
    \definition{v.+compl.}{sentir-se grato a alguém}
  \end{phonetics}
\end{entry}

\begin{entry}{颇}{11}[Radical 頁]
  \begin{phonetics}{颇}{po1}
    \definition*{s.}{sobrenome Po}
    \definition{adv.}{muito, bastante (linguagem escrita)}
  \end{phonetics}
\end{entry}

\begin{entry}{骑}{11}[Radical 馬]
  \begin{phonetics}{骑}{qi2}
    \definition{clas.}{para cavalos de sela}
    \definition{v.}{andar (cavalo, bicicleta, etc.) | sentar-se montado | montar}
  \end{phonetics}
\end{entry}

\begin{entry}{骑车}{11,4}
  \begin{phonetics}{骑车}{qi2che1}
    \definition{v.}{andar de bicicleta | pedalar}
  \end{phonetics}
\end{entry}

\begin{entry}{鸽子}{11,3}
  \begin{phonetics}{鸽子}{ge1zi5}
    \definition{s.}{pombo}
  \end{phonetics}
\end{entry}

\begin{entry}{鹿}{11}[Radical 鹿]
  \begin{phonetics}{鹿}{lu4}
    \definition{s.}{cervo | veado}
  \end{phonetics}
\end{entry}

\begin{entry}{麻将}{11,9}
  \begin{phonetics}{麻将}{ma2jiang4}
    \definition[副]{s.}{\emph{mahjong}}
  \end{phonetics}
\end{entry}

\begin{entry}{麻烦}{11,10}
  \begin{phonetics}{麻烦}{ma2fan5}
    \definition{adj.}{fastidioso | maçante | inconveniente | problemático}
    \definition{s.}{incômodo}
    \definition{v.}{incomodar alguém | colocar alguém em apuros}
  \end{phonetics}
\end{entry}

\begin{entry}{麻辣豆腐}{11,14,7,14}
  \begin{phonetics}{麻辣豆腐}{ma2la4 dou4fu5}
    \definition{s.}{tofú guisado em molho picante (prato)}
  \end{phonetics}
\end{entry}

\begin{entry}{黄}{11}[Radical ⻩][Kangxi 201]
  \begin{phonetics}{黄}{huang2}
    \definition*{s.}{sobrenome Huang ou Hwang}
    \definition{adj.}{amarelo | pornográfico}
  \end{phonetics}
\end{entry}

\begin{entry}{黄瓜}{11,5}
  \begin{phonetics}{黄瓜}{huang2gua1}
    \definition[条]{s.}{pepino}
  \end{phonetics}
\end{entry}

\begin{entry}{黄色}{11,6}
  \begin{phonetics}{黄色}{huang2se4}
    \definition{s.}{cor amarela}
  \end{phonetics}
\end{entry}

\begin{entry}{黄昏}{11,8}
  \begin{phonetics}{黄昏}{huang2hun1}
    \definition{s.}{anoitecer}
  \end{phonetics}
\end{entry}

\begin{entry}{黄油}{11,8}
  \begin{phonetics}{黄油}{huang2you2}
    \definition[盒]{s.}{manteiga}
  \end{phonetics}
\end{entry}

%%%%% EOF %%%%%


%%%
%%% 12画
%%%

\section*{12画}\addcontentsline{toc}{section}{12画}

\begin{entry}{傢具}{12,8}{⼈、⼋}
  \begin{phonetics}{傢具}{jia1ju4}
    \variantof{家具}
  \end{phonetics}
\end{entry}

\begin{entry}{博士}{12,3}{⼗、⼠}
  \begin{phonetics}{博士}{bo2shi4}[][HSK 5]
    \definition{s.}{doutorado; grau de doutor; nível mais alto de um diploma; também, uma pessoa que obteve esse diploma | doutor; antigo título honorífico para uma pessoa que é habilidosa em um determinado ofício ou especializada em uma determinada ocupação | doutor; autoridades que ensinavam as escrituras na China nos tempos antigos}
  \end{phonetics}
\end{entry}

\begin{entry}{博文}{12,4}{⼗、⽂}
  \begin{phonetics}{博文}{bo2wen2}
    \definition{s.}{artigo em um blog}
    \definition{v.}{escrever um artigo em um blog}
  \end{phonetics}
\end{entry}

\begin{entry}{博主}{12,5}{⼗、⼂}
  \begin{phonetics}{博主}{bo2zhu3}
    \definition{s.}{blogueiro}
  \end{phonetics}
\end{entry}

\begin{entry}{博物馆}{12,8,11}{⼗、⽜、⾷}
  \begin{phonetics}{博物馆}{bo2wu4guan3}[][HSK 5]
    \definition[个]{s.}{museu; locais para coleta, armazenamento, pesquisa, exibição e exposição de relíquias culturais ou espécimes relacionados à história, cultura, arte, ciências naturais, ciência e tecnologia, etc.}
  \end{phonetics}
\end{entry}

\begin{entry}{博客}{12,9}{⼗、⼧}
  \begin{phonetics}{博客}{bo2 ke4}[][HSK 5]
    \definition{s.}{\emph{blog}; página da Web ou site gerenciado por um indivíduo, geralmente composto por postagens organizadas da mais recente para a mais antiga | blogueiro; \emph{blogger}; pessoas que possuem ou escrevem \emph{blogs}}
  \end{phonetics}
\end{entry}

\begin{entry}{博览会}{12,9,6}{⼗、⾒、⼈}
  \begin{phonetics}{博览会}{bo2lan3hui4}[][HSK 5]
    \definition[次]{s.}{exposição; feira internacional; exposições de produtos em grande escala}
  \end{phonetics}
\end{entry}

\begin{entry}{厨房}{12,8}{⼚、⼾}
  \begin{phonetics}{厨房}{chu2fang2}[][HSK 5]
    \definition[间,个]{s.}{cozinha}
  \end{phonetics}
\end{entry}

\begin{entry}{喂}{12}{⼝}
  \begin{phonetics}{喂}{wei4}[][HSK 2,4]
    \definition{interj.}{Ei!, Olá!, para chamar atenção | Alô? (quando respondendo uma chamada telefônica, pronuncia-se como \dpy{wei2})}
    \definition{v.}{criar; alimentar (animais); dar comida a um animal |
alimentar (pessoas); colocar alimentos, medicamentos, etc. na boca de alguém}
  \end{phonetics}
\end{entry}

\begin{entry}{喂奶}{12,5}{⼝、⼥}
  \begin{phonetics}{喂奶}{wei4nai3}
    \definition{v.}{amamentar}
  \end{phonetics}
\end{entry}

\begin{entry}{喂母乳}{12,5,8}{⼝、⽏、⼄}
  \begin{phonetics}{喂母乳}{wei4mu3ru3}
    \definition{s.}{amamentação}
  \end{phonetics}
\end{entry}

\begin{entry}{喂养}{12,9}{⼝、⼋}
  \begin{phonetics}{喂养}{wei4yang3}
    \definition{v.}{alimentar (uma criança, animal doméstico, etc.) | manter | criar (um animal)}
  \end{phonetics}
\end{entry}

\begin{entry}{喂食}{12,9}{⼝、⾷}
  \begin{phonetics}{喂食}{wei4shi2}
    \definition{v.}{alimentar}
  \end{phonetics}
\end{entry}

\begin{entry}{喂哺}{12,10}{⼝、⼝}
  \begin{phonetics}{喂哺}{wei4bu3}
    \definition{v.}{alimentar (um bebê)}
  \end{phonetics}
\end{entry}

\begin{entry}{喂料}{12,10}{⼝、⽃}
  \begin{phonetics}{喂料}{wei4liao4}
    \definition{v.}{alimentar (também no sentido figurativo)}
  \end{phonetics}
\end{entry}

\begin{entry}{善于}{12,3}{⼝、⼆}
  \begin{phonetics}{善于}{shan4yu2}[][HSK 4]
    \definition{adv./v.}{ser bom em; ser hábil em}
  \end{phonetics}
\end{entry}

\begin{entry}{善良}{12,7}{⼝、⾉}
  \begin{phonetics}{善良}{shan4liang2}[][HSK 4]
    \definition{adj.}{de bom coração; bom e honesto; de bom coração e cheio de boa vontade}
  \end{phonetics}
\end{entry}

\begin{entry}{善意}{12,13}{⼝、⼼}
  \begin{phonetics}{善意}{shan4yi4}
    \definition{s.}{boa vontade | benevolência | bondade}
  \end{phonetics}
\end{entry}

\begin{entry}{喊}{12}{⼝}
  \begin{phonetics}{喊}{han3}[][HSK 2]
    \definition{clas.}{gritar | berrar | chamar (uma pessoa)}
  \end{phonetics}
\end{entry}

\begin{entry}{喔}{12}{⼝}
  \begin{phonetics}{喔}{o1}
    \definition{interj.}{Oh!, Entendi!, usado para indicar realização, compreensão}
  \end{phonetics}
\end{entry}

\begin{entry}{喜欢}{12,6}{⼝、⽋}
  \begin{phonetics}{喜欢}{xi3huan5}[][HSK 1]
    \definition{v.}{gostar}
  \end{phonetics}
\end{entry}

\begin{entry}{喜剧}{12,10}{⼝、⼑}
  \begin{phonetics}{喜剧}{xi3ju4}
    \definition[部,出]{s.}{uma comédia}
  \end{phonetics}
\end{entry}

\begin{entry}{喜爱}{12,10}{⼝、⽖}
  \begin{phonetics}{喜爱}{xi3 ai4}[][HSK 4]
    \definition{v.}{gostar; amar; ter afeição por; estar interessado em; ter uma queda ou sentir interesse por pessoas ou coisas}
  \end{phonetics}
\end{entry}

\begin{entry}{喝}{12}{⼝}
  \begin{phonetics}{喝}{he1}[][HSK 1]
    \definition{interj.}{Meu Deus!}
    \definition{v.}{beber}
  \end{phonetics}
  \begin{phonetics}{喝}{he4}
    \definition{v.}{gritar bem alto}
  \end{phonetics}
\end{entry}

\begin{entry}{喝彩}{12,11}{⼝、⼺}
  \begin{phonetics}{喝彩}{he4cai3}
    \definition{s.}{aclamar | torcer}
  \end{phonetics}
\end{entry}

\begin{entry}{喝醉}{12,15}{⼝、⾣}
  \begin{phonetics}{喝醉}{he1zui4}
    \definition{v.}{ficar bêbado}
  \end{phonetics}
\end{entry}

\begin{entry}{喻}{12}{⼝}
  \begin{phonetics}{喻}{yu4}
    \definition{s.}{analogia | símile | metáfora | alegoria}
    \definition{v.}{descrever algo como}
  \end{phonetics}
\end{entry}

\begin{entry}{堤坝}{12,7}{⼟、⼟}
  \begin{phonetics}{堤坝}{di1ba4}
    \definition{s.}{represa | dique | barragem}
  \end{phonetics}
\end{entry}

\begin{entry}{奥}{12}{⼤}
  \begin{phonetics}{奥}{ao4}
    \definition{adj.}{obscuro | misterioso}
  \end{phonetics}
\end{entry}

\begin{entry}{奥运}{12,7}{⼤、⾡}
  \begin{phonetics}{奥运}{ao4yun4}
    \definition*{s.}{Jogos Olímpicos, Olimpíadas, abreviação de 奥林匹克运动会}
  \seealsoref{奥林匹克运动会}{ao4lin2pi3ke4 yun4dong4hui4}
  \end{phonetics}
\end{entry}

\begin{entry}{奥运会}{12,7,6}{⼤、⾡、⼈}
  \begin{phonetics}{奥运会}{ao4yun4hui4}
    \definition*{s.}{Jogos Olímpicos, Olimpíadas, abreviação de 奥林匹克运动会}
  \seealsoref{奥林匹克运动会}{ao4lin2pi3ke4 yun4dong4hui4}
  \end{phonetics}
\end{entry}

\begin{entry}{奥林匹克运动会}{12,8,4,7,7,6,6}{⼤、⽊、⼖、⼗、⾡、⼒、⼈}
  \begin{phonetics}{奥林匹克运动会}{ao4lin2pi3ke4 yun4dong4hui4}
    \definition*{s.}{Jogos Olímpicos, Olimpíadas}
  \end{phonetics}
\end{entry}

\begin{entry}{奥特曼}{12,10,11}{⼤、⽜、⽈}
  \begin{phonetics}{奥特曼}{ao4te4man4}
    \definition*{s.}{\emph{Ultraman},  super-herói de ficção científica japonesa}
  \end{phonetics}
\end{entry}

\begin{entry}{媒体}{12,7}{⼥、⼈}
  \begin{phonetics}{媒体}{mei2ti3}[][HSK 3]
    \definition[家,个,种]{s.}{mídia; mídia de massa}
  \end{phonetics}
\end{entry}

\begin{entry}{嫂子}{12,3}{⼥、⼦}
  \begin{phonetics}{嫂子}{sao3zi5}
    \definition{s.}{esposa do irmão mais velho}
  \end{phonetics}
\end{entry}

\begin{entry}{富}{12}{⼧}
  \begin{phonetics}{富}{fu4}[][HSK 3]
    \definition*{s.}{sobrenome Fu}
    \definition{adj.}{rico; póspero | rico; abundante}
    \definition{s.}{fortuna; riqueza}
  \end{phonetics}
\end{entry}

\begin{entry}{寒冷}{12,7}{⼧、⼎}
  \begin{phonetics}{寒冷}{han2 leng3}[][HSK 4]
    \definition{adj.}{frio; frígido; gélido; gelado}
  \end{phonetics}
\end{entry}

\begin{entry}{寒假}{12,11}{⼧、⼈}
  \begin{phonetics}{寒假}{han2jia4}[][HSK 4]
    \definition[个]{s.}{férias de inverno (feriados); férias escolares no meio do inverno, em janeiro e fevereiro (na China)}
  \end{phonetics}
\end{entry}

\begin{entry}{寓意}{12,13}{⼧、⼼}
  \begin{phonetics}{寓意}{yu4yi4}
    \definition{s.}{moral (de uma história),  lição a ser aprendida, implicação, mensagem, significado metafórico}
  \end{phonetics}
\end{entry}

\begin{entry}{就}{12}{⼪}
  \begin{phonetics}{就}{jiu4}[][HSK 1]
    \definition{adv.}{exatamente | justamente}
    \definition{v.}{realizar | se envolver em | acompanhar (em alimentos) | aproveitar | avançar | empreender}
  \end{phonetics}
\end{entry}

\begin{entry}{就业}{12,5}{⼪、⼀}
  \begin{phonetics}{就业}{jiu4ye4}[][HSK 3]
    \definition{v.+compl.}{conseguir um emprego; obter emprego; assumir uma ocupação}
  \end{phonetics}
\end{entry}

\begin{entry}{就是}{12,9}{⼪、⽇}
  \begin{phonetics}{就是}{jiu4 shi4}[][HSK 3]
    \definition{adv.}{exatamente; precisamente | apenas; simplesmente | usado para indicar escolha}
    \definition{conj.}{ainda que}
    \definition{part.}{usado no final de uma frase para expressar afirmação}
  \end{phonetics}
\end{entry}

\begin{entry}{就要}{12,9}{⼪、⾑}
  \begin{phonetics}{就要}{jiu4 yao4}[][HSK 2]
    \definition{adv.}{estar prestes a | estar indo para | estar a ponto de}
  \end{phonetics}
\end{entry}

\begin{entry}{就职}{12,11}{⼪、⽿}
  \begin{phonetics}{就职}{jiu4zhi2}
    \definition{v.}{assumir o cargo | assumir um posto}
  \end{phonetics}
\end{entry}

\begin{entry}{属}{12}{⼫}
  \begin{phonetics}{属}{shu3}[][HSK 3]
    \definition{s.}{categoria
gênero
membros da família; dependentes}
    \definition{v.}{estar sob; subordinado a | pertencer a | nascer no ano de (um dos doze animais do zodíaco)}
  \end{phonetics}
  \begin{phonetics}{属}{zhu3}
    \definition{v.}{juntar; combinar | fixar (a mente) em; centrar (a atenção, etc.) em}
  \end{phonetics}
\end{entry}

\begin{entry}{属于}{12,3}{⼫、⼆}
  \begin{phonetics}{属于}{shu3yu2}[][HSK 3]
    \definition{v.}{pertencer a; fazer parte de; ser classificado como}
  \end{phonetics}
\end{entry}

\begin{entry}{屡次}{12,6}{⼫、⽋}
  \begin{phonetics}{屡次}{lv3ci4}
    \definition{adv.}{repetidamente | uma e outra vez | muitas vezes}
  \end{phonetics}
\end{entry}

\begin{entry}{帽子}{12,3}{⼱、⼦}
  \begin{phonetics}{帽子}{mao4zi5}[][HSK 4]
    \definition[顶,个,种]{s.}{boné; chapéu; capacete | etiqueta; rótulo; marca}
  \end{phonetics}
\end{entry}

\begin{entry}{幅}{12}{⼱}
  \begin{phonetics}{幅}{fu2}[][HSK 5]
    \definition{clas.}{para tecidos, telas de lã, pinturas, etc.}
    \definition{s.}{largura do tecido, seda, tweed, etc. | tamanho; largura; largura em termos gerais}
  \end{phonetics}
\end{entry}

\begin{entry}{幅度}{12,9}{⼱、⼴}
  \begin{phonetics}{幅度}{fu2du4}[][HSK 5]
    \definition{s.}{alcance; escopo; extensão; largura; largura da propagação de um objeto que vibra ou balança, uma metáfora para a magnitude de uma mudança em algo}
  \end{phonetics}
\end{entry}

\begin{entry}{强}{12}{⼸}
  \begin{phonetics}{强}{jiang4}
    \definition{adj.}{teimoso; inflexível}
  \end{phonetics}
  \begin{phonetics}{强}{qiang2}[][HSK 3]
    \definition*{s.}{sobrenome Qiang}
    \definition{adj.}{forte; poderoso | melhor; superior | mais; extra; adicional; um pouco mais que | resoluto; firme | decidido; resolvido | violento; impetuoso | alto padrão}
    \definition{v.}{fortalecer; tornar forte}
  \end{phonetics}
  \begin{phonetics}{强}{qiang3}
    \definition{v.}{fazer um esforço; esforçar-se}
  \end{phonetics}
\end{entry}

\begin{entry}{强大}{12,3}{⼸、⼤}
  \begin{phonetics}{强大}{qiang2 da4}[][HSK 3]
    \definition{adj.}{forte; poderoso; potente; possante}
  \end{phonetics}
\end{entry}

\begin{entry}{强烈}{12,10}{⼸、⽕}
  \begin{phonetics}{强烈}{qiang2lie4}[][HSK 3]
    \definition{adj.}{forte; intenso | violento; impetuoso | afiado; marcante}
  \end{phonetics}
\end{entry}

\begin{entry}{强调}{12,10}{⼸、⾔}
  \begin{phonetics}{强调}{qiang2diao4}[][HSK 3]
    \definition{v.}{salientar; sublinhar; enfatizar; dar ênfase a; vincar}
  \end{phonetics}
\end{entry}

\begin{entry}{悲伤}{12,6}{⽕、⼈}
  \begin{phonetics}{悲伤}{bei1 shang1}[][HSK 5]
    \definition{adj.}{triste; pesaroso}
  \end{phonetics}
\end{entry}

\begin{entry}{悲剧}{12,10}{⽕、⼑}
  \begin{phonetics}{悲剧}{bei1 ju4}[][HSK 5]
    \definition[部,出]{s.}{tragédia; drama trágico; uma das principais categorias de teatro, caracterizada basicamente pela representação do conflito irreconciliável entre o protagonista e a realidade e seu final trágico | tragédia; evento triste; metáfora para encontro infeliz}
  \end{phonetics}
\end{entry}

\begin{entry}{惑星}{12,9}{⼼、⽇}
  \begin{phonetics}{惑星}{huo4xing1}
    \definition{s.}{planeta}
  \seealsoref{行星}{xing2xing1}
  \end{phonetics}
\end{entry}

\begin{entry}{惩处}{12,5}{⼼、⼡}
  \begin{phonetics}{惩处}{cheng2chu3}
    \definition{v.}{administrar justiça | punir}
  \end{phonetics}
\end{entry}

\begin{entry}{惩罚}{12,9}{⼼、⽹}
  \begin{phonetics}{惩罚}{cheng2fa2}
    \definition{v.}{punir | penalizar}
  \end{phonetics}
\end{entry}

\begin{entry}{愉快}{12,7}{⼼、⼼}
  \begin{phonetics}{愉快}{yu2kuai4}
    \definition{adj.}{alegre | delicioso | prazeroso | agradável | feliz | encantado}
    \definition{adv.}{alegremente | agradavelmente}
  \end{phonetics}
\end{entry}

\begin{entry}{愤世嫉俗}{12,5,13,9}{⼼、⼀、⼥、⼈}
  \begin{phonetics}{愤世嫉俗}{fen4shi4ji2su2}
    \definition{v.}{ser cínico | ser amargurado}
  \end{phonetics}
\end{entry}

\begin{entry}{愤怒}{12,9}{⼼、⼼}
  \begin{phonetics}{愤怒}{fen4nu4}
    \definition{adj.}{zangado | indignado}
    \definition{s.}{ira}
  \end{phonetics}
\end{entry}

\begin{entry}{掌}{12}{⼿}
  \begin{phonetics}{掌}{zhang3}
    \definition{s.}{palma da mão | sola do pé | pata | ferradura}
    \definition{v.}{dar um tapa | segurar na mão | empunhar}
  \end{phonetics}
\end{entry}

\begin{entry}{掱}{12}{⼿}
  \begin{phonetics}{掱}{shou3}
    \variantof{手}
  \end{phonetics}
\end{entry}

\begin{entry}{揉}{12}{⼿}
  \begin{phonetics}{揉}{rou2}
    \definition{v.}{amassar | massagear | esfregar}
  \end{phonetics}
\end{entry}

\begin{entry}{揉碎}{12,13}{⼿、⽯}
  \begin{phonetics}{揉碎}{rou2sui4}
    \definition{v.}{esmagar | desintegrar-se em pedaços}
  \end{phonetics}
\end{entry}

\begin{entry}{提}{12}{⼿}
  \begin{phonetics}{提}{ti2}[][HSK 2]
    \definition*{s.}{sobrenome Ti}
    \definition{s.}{concha | traço ascendente (em caracteres chineses)}
    \definition{v.}{carregar (na mão com o braço para baixo) | levantar | elevar | promover | avançar | mudar para um momento anterior | mover uma data para a frente | trazer à tona | apresentar | extrair | tirar | trazer | entregar | mencionar | referir-se a}
  \end{phonetics}
\end{entry}

\begin{entry}{提及}{12,3}{⼿、⼃}
  \begin{phonetics}{提及}{ti2ji2}
    \definition{v.}{mencionar | levantar (um assunto) | chamar a atenção de alguém}
  \end{phonetics}
\end{entry}

\begin{entry}{提升}{12,4}{⼿、⼗}
  \begin{phonetics}{提升}{ti2sheng1}
    \definition{v.}{promover (para uma posição de classificação mais alta) | levantar | içar | (figurativo) elevar, levantar, melhorar}
  \end{phonetics}
\end{entry}

\begin{entry}{提出}{12,5}{⼿、⼐}
  \begin{phonetics}{提出}{ti2 chu1}[][HSK 2]
    \definition{v.}{levantar | propor | expor | apresentar}
  \end{phonetics}
\end{entry}

\begin{entry}{提问}{12,6}{⼿、⾨}
  \begin{phonetics}{提问}{ti2wen4}[][HSK 3]
    \definition{v.}{\emph{quiz}; fazer uma pergunta; colocar questões para}
  \end{phonetics}
\end{entry}

\begin{entry}{提供}{12,8}{⼿、⼈}
  \begin{phonetics}{提供}{ti2gong1}[][HSK 4]
    \definition{v.}{oferecer; fornecer; suprir; prover; proporcionar}
  \end{phonetics}
\end{entry}

\begin{entry}{提到}{12,8}{⼿、⼑}
  \begin{phonetics}{提到}{ti2 dao4}[][HSK 2]
    \definition{v.}{mencionar | referir-se a | levantar (assunto)}
  \end{phonetics}
\end{entry}

\begin{entry}{提前}{12,9}{⼿、⼑}
  \begin{phonetics}{提前}{ti2qian2}[][HSK 3]
    \definition{adv.}{antecipadamente}
    \definition{v.}{avançar; adiantar; mudar para uma data anterior; mover para a frente (uma data)}
  \end{phonetics}
\end{entry}

\begin{entry}{提高}{12,10}{⼿、⾼}
  \begin{phonetics}{提高}{ti2gao1}[][HSK 2]
    \definition{v.}{melhorar | aumentar | elevar}
  \end{phonetics}
\end{entry}

\begin{entry}{提醒}{12,16}{⼿、⾣}
  \begin{phonetics}{提醒}{ti2xing3}[][HSK 4]
    \definition{v.+compl.}{alertar; avisar; advertir; lembrar; apontar para ou chamar a atenção para}
  \end{phonetics}
\end{entry}

\begin{entry}{插}{12}{⼿}
  \begin{phonetics}{插}{cha1}[][HSK 5]
    \definition{v.}{enfiar; inserir; colocar, apertar, empurrar ou perfurar uma coisa fina ou delgada; mergulhar |interpor; inserir; colocar no meio}
  \end{phonetics}
\end{entry}

\begin{entry}{插手}{12,4}{⼿、⼿}
  \begin{phonetics}{插手}{cha1shou3}
    \definition{v.+compl.}{envolver-se em | dar uma mão | ter (tomar) uma mão | cutucar o nariz de alguém | intrometer-se}
  \end{phonetics}
\end{entry}

\begin{entry}{插话}{12,8}{⼿、⾔}
  \begin{phonetics}{插话}{cha1hua4}
    \definition{s.}{interrupção | digressão}
    \definition{v.+compl.}{interromper (a fala de alguém)}
  \end{phonetics}
\end{entry}

\begin{entry}{握手}{12,4}{⼿、⼿}
  \begin{phonetics}{握手}{wo4shou3}[][HSK 3]
    \definition{v.+compl.}{apertar as mãos}
  \end{phonetics}
\end{entry}

\begin{entry}{援助}{12,7}{⼿、⼒}
  \begin{phonetics}{援助}{yuan2zhu4}
    \definition{s.}{assistência}
    \definition{v.}{ajudar | apoiar | assistir}
  \end{phonetics}
\end{entry}

\begin{entry}{搁浅}{12,8}{⼿、⽔}
  \begin{phonetics}{搁浅}{ge1qian3}
    \definition{v.}{ficar encalhado (navio) | encalhar | (figurativo) encontrar dificuldades e parar}
  \end{phonetics}
\end{entry}

\begin{entry}{搓}{12}{⼿}
  \begin{phonetics}{搓}{cuo1}
    \definition{s.}{torção}
    \definition{v.}{esfregar ou rolar entre as mãos ou dedos | torcer}
  \end{phonetics}
\end{entry}

\begin{entry}{搭讪}{12,5}{⼿、⾔}
  \begin{phonetics}{搭讪}{da1shan4}
    \definition{v.}{bater em alguém | incitar uma conversa | começar a conversar para acabar com um silêncio constrangedor ou uma situação embaraçosa}
  \end{phonetics}
\end{entry}

\begin{entry}{搭配}{12,10}{⼿、⾣}
  \begin{phonetics}{搭配}{da1pei4}
    \definition{v.}{emparelhar | combinar | organizar em pares | adicionar alguém em um grupo}
  \end{phonetics}
\end{entry}

\begin{entry}{散}{12}{⽁}
  \begin{phonetics}{散}{san3}
    \definition{adj.}{disperso; fragmentado; não integrado}
    \definition{s.}{medicamento em forma de pó}
    \definition{v.}{divergir; espalhar-se; separar-se; soltar-se; não se manter unido;  desintegrar}
  \end{phonetics}
  \begin{phonetics}{散}{san4}
    \definition{v.}{quebrar; fragmentar; dispersar | dar; distribuir; disseminar; divulgar | dissipar; deixar sai  | terminar um acordo ou contrato; demitir}
  \end{phonetics}
\end{entry}

\begin{entry}{散心}{12,4}{⽁、⼼}
  \begin{phonetics}{散心}{san4xin1}
    \definition{v.+compl.}{aliviar o tédio | desfrutar de uma diversão | estar despreocupado}
  \end{phonetics}
\end{entry}

\begin{entry}{散步}{12,7}{⽁、⽌}
  \begin{phonetics}{散步}{san4bu4}[][HSK 3]
    \definition{v.+compl.}{dar uma volta; passear; dar uma caminhada}
  \end{phonetics}
\end{entry}

\begin{entry}{敬礼}{12,5}{⽁、⽰}
  \begin{phonetics}{敬礼}{jing4li3}
    \definition{s.}{saudação}
    \definition{v.}{saudar}
  \end{phonetics}
\end{entry}

\begin{entry}{斯巴达}{12,4,6}{⽄、⼰、⾡}
  \begin{phonetics}{斯巴达}{si1ba1da2}
    \definition*{s.}{Esparta}
  \end{phonetics}
\end{entry}

\begin{entry}{普及}{12,3}{⽇、⼃}
  \begin{phonetics}{普及}{pu3ji2}[][HSK 3]
    \definition{adj.}{popular; universal; onipresente}
    \definition{v.}{popularizar; disseminar; espalhar entre o povo}
  \end{phonetics}
\end{entry}

\begin{entry}{普通}{12,10}{⽇、⾡}
  \begin{phonetics}{普通}{pu3 tong1}[][HSK 2]
    \definition{adj.}{ordinário | comum | geral | médio}
  \end{phonetics}
\end{entry}

\begin{entry}{普通话}{12,10,8}{⽇、⾡、⾔}
  \begin{phonetics}{普通话}{pu3tong1hua4}[][HSK 2]
    \definition*{s.}{Mandarim (literalmente ``linguagem comum'') | Putonghua (fala comum da língua chinesa) | discurso comum}
  \end{phonetics}
\end{entry}

\begin{entry}{普遍}{12,12}{⽇、⾡}
  \begin{phonetics}{普遍}{pu3bian4}[][HSK 3]
    \definition{adj.}{geral; comum; universal; difundido}
  \end{phonetics}
\end{entry}

\begin{entry}{景色}{12,6}{⽇、⾊}
  \begin{phonetics}{景色}{jing3se4}[][HSK 3]
    \definition[片,幅,道,处]{s.}{vista; cena; cenário; paisagem}
  \end{phonetics}
\end{entry}

\begin{entry}{晴}{12}{⽇}
  \begin{phonetics}{晴}{qing2}[][HSK 2]
    \definition{adj.}{ensolarado | claro}
  \end{phonetics}
\end{entry}

\begin{entry}{晴天}{12,4}{⽇、⼤}
  \begin{phonetics}{晴天}{qing2 tian1}[][HSK 2]
    \definition[个]{s.}{dia ensolarado}
  \end{phonetics}
\end{entry}

\begin{entry}{智力}{12,2}{⽇、⼒}
  \begin{phonetics}{智力}{zhi4li4}[][HSK 4]
    \definition{s.}{inteligência; refere-se à capacidade de uma pessoa de conhecer e entender coisas objetivas e aplicar o conhecimento e a experiência para resolver problemas, incluindo memória, observação, imaginação, pensamento e julgamento}
  \end{phonetics}
\end{entry}

\begin{entry}{智能}{12,10}{⽇、⾁}
  \begin{phonetics}{智能}{zhi4neng2}[][HSK 4]
    \definition{adj.}{inteligente (telefone, sistema, etc.); descreve máquinas, equipamentos, tecnologia, etc. que foram processados com alta tecnologia e têm a capacidade de falar, pensar, calcular, resolver problemas, etc., como um ser humano}
    \definition{s.}{intelecto; a capacidade de aprender, agir, pensar, inventar, criar, resolver problemas, etc.}
  \end{phonetics}
\end{entry}

\begin{entry}{智商}{12,11}{⽇、⼝}
  \begin{phonetics}{智商}{zhi4shang1}
    \definition{s.}{quociente de inteligência, QI}
  \end{phonetics}
\end{entry}

\begin{entry}{智障}{12,13}{⽇、⾩}
  \begin{phonetics}{智障}{zhi4zhang4}
    \definition{adj./s.}{retardado}
  \end{phonetics}
\end{entry}

\begin{entry}{智慧}{12,15}{⽇、⼼}
  \begin{phonetics}{智慧}{zhi4hui4}
    \definition{s.}{sabedoria | inteligência}
  \end{phonetics}
\end{entry}

\begin{entry}{暑假}{12,11}{⽇、⼈}
  \begin{phonetics}{暑假}{shu3 jia4}[][HSK 4]
    \definition[个]{s.}{férias de verão; feriado de verão; férias escolares de verão, na China, durante o sétimo e o oitavo meses do calendário gregoriano}
  \end{phonetics}
\end{entry}

\begin{entry}{曾}{12}{⽈}
  \begin{phonetics}{曾}{ceng2}[][HSK 4]
    \definition{adv.}{uma vez; antigamente; há algum tempo; usado para indicar ação ou estado passado}
  \end{phonetics}
  \begin{phonetics}{曾}{zeng1}
    \definition*{s.}{sobrenome Zeng}
    \definition{s.}{relacionamento entre bisnetos e bisavós}
  \end{phonetics}
\end{entry}

\begin{entry}{曾经}{12,8}{⽈、⽷}
  \begin{phonetics}{曾经}{ceng2jing1}[][HSK 3]
    \definition{adv.}{uma vez; indica certos comportamentos ou situações}
  \end{phonetics}
\end{entry}

\begin{entry}{替}{12}{⽈}
  \begin{phonetics}{替}{ti4}[][HSK 4]
    \definition{prep.}{para; em nome de}
    \definition{s.}{decadência; declínio; enfraquecimento}
    \definition{v.}{substituir; substituir por; tomar o lugar de}
  \end{phonetics}
\end{entry}

\begin{entry}{替代}{12,5}{⽈、⼈}
  \begin{phonetics}{替代}{ti4 dai4}[][HSK 4]
    \definition{v.}{substituir; suplantar}
  \end{phonetics}
\end{entry}

\begin{entry}{最}{12}{⽈}
  \begin{phonetics}{最}{zui4}[][HSK 1]
    \definition{adv.}{o mais | o melhor | a coisa mais\dots | grau superlativo relativo de superioridade}
  \end{phonetics}
\end{entry}

\begin{entry}{最少}{12,4}{⽈、⼩}
  \begin{phonetics}{最少}{zui4shao3}
    \definition{adv.}{finalmente}
  \end{phonetics}
\end{entry}

\begin{entry}{最优}{12,6}{⽈、⼈}
  \begin{phonetics}{最优}{zui4you1}
    \definition{adj.}{ótimo}
  \end{phonetics}
\end{entry}

\begin{entry}{最先}{12,6}{⽈、⼉}
  \begin{phonetics}{最先}{zui4xian1}
    \definition{adv.}{o primeiro}
  \end{phonetics}
\end{entry}

\begin{entry}{最后}{12,6}{⽈、⼝}
  \begin{phonetics}{最后}{zui4hou4}[][HSK 1]
    \definition{adj.}{final | último}
    \definition{adv.}{finalmente}
  \end{phonetics}
\end{entry}

\begin{entry}{最多}{12,6}{⽈、⼣}
  \begin{phonetics}{最多}{zui4duo1}
    \definition{adv.}{no máximo | máximo}
  \end{phonetics}
\end{entry}

\begin{entry}{最好}{12,6}{⽈、⼥}
  \begin{phonetics}{最好}{zui4hao3}[][HSK 1]
    \definition{adv.}{ser melhor que}
    \definition{v.}{(você) estar melhor (faça o que sugerimos) | querer ser o melhor}
  \end{phonetics}
\end{entry}

\begin{entry}{最初}{12,7}{⽈、⾐}
  \begin{phonetics}{最初}{zui4chu1}[][HSK 4]
    \definition{adj.}{primordial; inicial; primeiro}
    \definition{adv.}{inicialmente; originalmente}
    \definition{s.}{o período mais antigo; início; começo}
  \end{phonetics}
\end{entry}

\begin{entry}{最近}{12,7}{⽈、⾡}
  \begin{phonetics}{最近}{zui4jin4}[][HSK 2]
    \definition{adv.}{ultimamente | recentemente}
  \end{phonetics}
\end{entry}

\begin{entry}{最远}{12,7}{⽈、⾡}
  \begin{phonetics}{最远}{zui4yuan3}
    \definition{adv.}{mais distante | mais longe}
  \end{phonetics}
\end{entry}

\begin{entry}{最佳}{12,8}{⽈、⼈}
  \begin{phonetics}{最佳}{zui4jia1}
    \definition{adj.}{melhor (atleta, filme etc) | ótimo}
  \end{phonetics}
\end{entry}

\begin{entry}{最终}{12,8}{⽈、⽷}
  \begin{phonetics}{最终}{zui4zhong1}
    \definition{adv.}{pelo menos | finalmente}
    \definition{s.}{final | ultimato}
  \end{phonetics}
\end{entry}

\begin{entry}{最高}{12,10}{⽈、⾼}
  \begin{phonetics}{最高}{zui4gao1}
    \definition{adj.}{altíssimo | supremo | mais alto}
  \end{phonetics}
\end{entry}

\begin{entry}{最善}{12,12}{⽈、⼝}
  \begin{phonetics}{最善}{zui4shan4}
    \definition{adj.}{ótimo | o melhor}
  \end{phonetics}
\end{entry}

\begin{entry}{最新}{12,13}{⽈、⽄}
  \begin{phonetics}{最新}{zui4xin1}
    \definition{adv.}{mais recente | mais novo}
  \end{phonetics}
\end{entry}

\begin{entry}{朝}{12}{⽉}
  \begin{phonetics}{朝}{chao2}[][HSK 3]
    \definition*{s.}{sobrenome Chao}
    \definition{prep.}{para; em direção a}
    \definition{s.}{tribunal; governo | dinastia | o reino de um imperador}
    \definition{v.}{ter uma audiência com (um rei, um imperador, etc.); fazer uma peregrinação a | encarar; olhar}
  \end{phonetics}
  \begin{phonetics}{朝}{zhao1}
    \definition{s.}{manhã cedo; manhã | dia}
  \end{phonetics}
\end{entry}

\begin{entry}{朝廷}{12,6}{⽉、⼵}
  \begin{phonetics}{朝廷}{chao2ting2}
    \definition{s.}{corte imperial | dinastia}
  \end{phonetics}
\end{entry}

\begin{entry}{朝鲜}{12,14}{⽉、⿂}
  \begin{phonetics}{朝鲜}{chao2xian3}
    \definition*{s.}{Coréia do Norte}
  \end{phonetics}
\end{entry}

\begin{entry}{期}{12}{⽉}
  \begin{phonetics}{期}{qi1}[][HSK 3]
    \definition{clas.}{questão; número; termo}
    \definition{s.}{tempo designado (programado) | um período de tempo; fase; estágio}
    \definition{v.}{marcar uma consulta | esperar; supor; imaginar}
  \end{phonetics}
\end{entry}

\begin{entry}{期中}{12,4}{⽉、⼁}
  \begin{phonetics}{期中}{qi1 zhong1}[][HSK 4]
    \definition{adj.}{provisório; interino; intermediário}
  \end{phonetics}
\end{entry}

\begin{entry}{期末}{12,5}{⽉、⽊}
  \begin{phonetics}{期末}{qi1 mo4}[][HSK 4]
    \definition{s.}{terminal; final do prazo; fim do período}
  \end{phonetics}
\end{entry}

\begin{entry}{期间}{12,7}{⽉、⾨}
  \begin{phonetics}{期间}{qi1jian1}[][HSK 4]
    \definition{s.}{prazo; tempo; período}
  \end{phonetics}
\end{entry}

\begin{entry}{期限}{12,8}{⽉、⾩}
  \begin{phonetics}{期限}{qi1xian4}[][HSK 4]
    \definition{s.}{prazo; limite de tempo; tempo alocado; período de tempo limitado, também o limite final do limite de tempo}
  \end{phonetics}
\end{entry}

\begin{entry}{期待}{12,9}{⽉、⼻}
  \begin{phonetics}{期待}{qi1dai4}[][HSK 4]
    \definition{v.}{aguardar; esperar; aguardar ansiosamente; ter em mente a realização de um determinado fim ou a ocorrência de uma determinada situação}
  \end{phonetics}
\end{entry}

\begin{entry}{棉}{12}{⽊}
  \begin{phonetics}{棉}{mian2}
    \definition{s.}{termo genérico para algodão ou paina | algodão | acolchoado ou estofado com algodão}
  \end{phonetics}
\end{entry}

\begin{entry}{棒}{12}{⽊}
  \begin{phonetics}{棒}{bang4}[][HSK 5]
    \definition{adj.}{bom; forte; excelente}
    \definition[根]{s.}{porrete; vara; bastão; cacete; haste}
  \end{phonetics}
\end{entry}

\begin{entry}{棒冰}{12,6}{⽊、⼎}
  \begin{phonetics}{棒冰}{bang4bing1}
    \definition{s.}{picolé}
  \end{phonetics}
\end{entry}

\begin{entry}{棒棒糖}{12,12,16}{⽊、⽊、⽶}
  \begin{phonetics}{棒棒糖}{bang4bang4tang2}
    \definition[根]{s.}{pirulito}
  \end{phonetics}
\end{entry}

\begin{entry}{棕褐色}{12,14,6}{⽊、⾐、⾊}
  \begin{phonetics}{棕褐色}{zong1he4 se4}
    \definition{s.}{cor sépia | bronzeado}
  \end{phonetics}
\end{entry}

\begin{entry}{森林}{12,8}{⽊、⽊}
  \begin{phonetics}{森林}{sen1lin2}[][HSK 4]
    \definition[片,座,处]{s.}{floresta; bosque; normalmente, refere-se a uma grande área de árvores em crescimento; na silvicultura, refere-se a um grande número de árvores que crescem em uma área razoavelmente grande de terra, juntamente com os animais e outras plantas}
  \end{phonetics}
\end{entry}

\begin{entry}{棵}{12}{⽊}
  \begin{phonetics}{棵}{ke1}[][HSK 4]
    \definition{clas.}{para plantas, árvores}
  \end{phonetics}
\end{entry}

\begin{entry}{棹}{12}{⽊}
  \begin{phonetics}{棹}{zhuo1}
    \variantof{桌}
  \end{phonetics}
\end{entry}

\begin{entry}{棺}{12}{⽊}
  \begin{phonetics}{棺}{guan1}
    \definition{s.}{caixão | esquife | ataúde}
  \end{phonetics}
\end{entry}

\begin{entry}{椅子}{12,3}{⽊、⼦}
  \begin{phonetics}{椅子}{yi3zi5}[][HSK 2]
    \definition[把,套]{s.}{cadeira}
  \end{phonetics}
\end{entry}

\begin{entry}{植物}{12,8}{⽊、⽜}
  \begin{phonetics}{植物}{zhi2wu4}[][HSK 4]
    \definition[种,株,盆,棵]{s.}{planta; vegetação; flora}
  \end{phonetics}
\end{entry}

\begin{entry}{椰汁}{12,5}{⽊、⽔}
  \begin{phonetics}{椰汁}{ye1zhi1}
    \definition{s.}{água de coco}
  \end{phonetics}
\end{entry}

\begin{entry}{款}{12}{⽋}
  \begin{phonetics}{款}{kuan3}
    \definition{clas.}{para versões ou modelos (de um produto)}
    \definition[笔,个]{s.}{montante de dinheiro | fundos | parágrafo | seção}
  \end{phonetics}
\end{entry}

\begin{entry}{殖}{12}{⽍}
  \begin{phonetics}{殖}{zhi2}
    \definition{v.}{crescer | reproduzir}
  \end{phonetics}
\end{entry}

\begin{entry}{渡过}{12,6}{⽔、⾡}
  \begin{phonetics}{渡过}{du4guo4}
    \definition{v.}{atravessar | passar por}
  \end{phonetics}
\end{entry}

\begin{entry}{温度}{12,9}{⽔、⼴}
  \begin{phonetics}{温度}{wen1du4}[][HSK 2]
    \definition[个]{s.}{temperatura}
  \end{phonetics}
\end{entry}

\begin{entry}{温度计}{12,9,4}{⽔、⼴、⾔}
  \begin{phonetics}{温度计}{wen1du4ji4}
    \definition{s.}{termógrafo | termômetro}
  \end{phonetics}
\end{entry}

\begin{entry}{温度表}{12,9,8}{⽔、⼴、⾐}
  \begin{phonetics}{温度表}{wen1du4biao3}
    \definition{s.}{termômetro}
  \end{phonetics}
\end{entry}

\begin{entry}{温度梯度}{12,9,11,9}{⽔、⼴、⽊、⼴}
  \begin{phonetics}{温度梯度}{wen1du4ti1du4}
    \definition{s.}{gradiente de temperatura}
  \end{phonetics}
\end{entry}

\begin{entry}{温柔}{12,9}{⽔、⽊}
  \begin{phonetics}{温柔}{wen1rou2}
    \definition{adj.}{gentil e suave | terno | doce (comumente usado para descrever uma menina ou mulher)}
  \end{phonetics}
\end{entry}

\begin{entry}{温暖}{12,13}{⽔、⽇}
  \begin{phonetics}{温暖}{wen1nuan3}[][HSK 3]
    \definition{adj.}{caloroso; gentil}
    \definition{v.}{aquecer (fazer você se sentir aquecido)}
  \end{phonetics}
\end{entry}

\begin{entry}{渴}{12}{⽔}
  \begin{phonetics}{渴}{ke3}[][HSK 1]
    \definition{adj.}{sedento}
  \end{phonetics}
\end{entry}

\begin{entry}{游}{12}{⽔}
  \begin{phonetics}{游}{you2}[][HSK 3]
    \definition*{s.}{sobrenome You}
    \definition{adj.}{itinerante; errante; não fixo; frequentemente em movimento}
    \definition{s.}{parte de um rio; uma seção do rio}
    \definition{v.}{nadar; pessoas ou animais se movendo na água | vagar por aí; vagar; viajar; passear | associar com (comunicação)}
  \end{phonetics}
\end{entry}

\begin{entry}{游戏}{12,6}{⽔、⼽}
  \begin{phonetics}{游戏}{you2xi4}[][HSK 3]
    \definition[场]{s.}{jogo; recreação; atividades recreativas, como esconde-esconde, adivinhação de enigmas de lanternas e algumas atividades esportivas informais, como bola recreativa, também são chamadas de jogos}
    \definition{v.}{jogar; fazer atividades relaxantes e prazerosas sozinho ou com outras pessoas}
  \end{phonetics}
\end{entry}

\begin{entry}{游泳}{12,8}{⽔、⽔}
  \begin{phonetics}{游泳}{you2yong3}[][HSK 3]
    \definition[次]{s.}{natação; refere-se ao esporte ou atividade de natação}
    \definition{v.+compl.}{nadar; pessoas ou animais nadando na água}
  \end{phonetics}
\end{entry}

\begin{entry}{游泳池}{12,8,6}{⽔、⽔、⽔}
  \begin{phonetics}{游泳池}{you2yong3chi2}
    \definition[场]{s.}{piscina}
  \seealsoref{泳池}{yong3chi2}
  \seealsoref{游泳馆}{you2yong3guan3}
  \end{phonetics}
\end{entry}

\begin{entry}{游泳衣}{12,8,6}{⽔、⽔、⾐}
  \begin{phonetics}{游泳衣}{you2yong3yi1}
    \definition{s.}{roupa de banho}
  \seealsoref{泳衣}{yong3yi1}
  \end{phonetics}
\end{entry}

\begin{entry}{游泳馆}{12,8,11}{⽔、⽔、⾷}
  \begin{phonetics}{游泳馆}{you2yong3guan3}
    \definition[场]{s.}{piscina}
  \seealsoref{泳池}{yong3chi2}
  \seealsoref{游泳池}{you2yong3chi2}
  \end{phonetics}
\end{entry}

\begin{entry}{游泳镜}{12,8,16}{⽔、⽔、⾦}
  \begin{phonetics}{游泳镜}{you2yong3jing4}
    \definition{s.}{óculos de natação}
  \end{phonetics}
\end{entry}

\begin{entry}{游客}{12,9}{⽔、⼧}
  \begin{phonetics}{游客}{you2 ke4}[][HSK 2]
    \definition{s.}{viajante | turista | (jogo online) jogador convidado}
  \end{phonetics}
\end{entry}

\begin{entry}{游艇}{12,12}{⽔、⾈}
  \begin{phonetics}{游艇}{you2ting3}
    \definition[只]{s.}{barcaça | iate}
  \end{phonetics}
\end{entry}

\begin{entry}{湖}{12}{⽔}
  \begin{phonetics}{湖}{hu2}[][HSK 2]
    \definition[个,片]{s.}{lago}
  \end{phonetics}
\end{entry}

\begin{entry}{湖南}{12,9}{⽔、⼗}
  \begin{phonetics}{湖南}{hu2nan2}
    \definition*{s.}{Hunan}
  \end{phonetics}
\end{entry}

\begin{entry}{湿}{12}{⽔}
  \begin{phonetics}{湿}{shi1}[][HSK 4]
    \definition{adj.}{molhado; úmido; algo com água ou com muita água dentro}
  \end{phonetics}
\end{entry}

\begin{entry}{滑}{12}{⽔}
  \begin{phonetics}{滑}{hua2}
    \definition*{s.}{sobrenome Hua}
    \definition{adj.}{deslizado}
    \definition{v.}{deslizar}
  \end{phonetics}
\end{entry}

\begin{entry}{滑雪}{12,11}{⽔、⾬}
  \begin{phonetics}{滑雪}{hua2xue3}
    \definition{v.+compl.}{esquiar | praticar esqui}
  \end{phonetics}
\end{entry}

\begin{entry}{焚香}{12,9}{⽕、⾹}
  \begin{phonetics}{焚香}{fen2xiang1}
    \definition{v.}{queimar incenso}
  \end{phonetics}
\end{entry}

\begin{entry}{焦虑}{12,10}{⽕、⾌}
  \begin{phonetics}{焦虑}{jiao1lv4}
    \definition{adj.}{ansioso | preocupado | apreensivo}
  \end{phonetics}
\end{entry}

\begin{entry}{然}{12}{⽕}
  \begin{phonetics}{然}{ran2}
    \definition{conj.}{mas | no entanto}
  \end{phonetics}
\end{entry}

\begin{entry}{然后}{12,6}{⽕、⼝}
  \begin{phonetics}{然后}{ran2hou4}[][HSK 2]
    \definition{conj.}{depois | logo | portanto}
  \end{phonetics}
\end{entry}

\begin{entry}{然而}{12,6}{⽕、⽽}
  \begin{phonetics}{然而}{ran2'er2}[][HSK 4]
    \definition{conj.}{ainda; mas; contudo; todavia; usado no início de uma frase para indicar uma transição; para indicar uma transição, geralmente é precedido por uma conjunção como ``虽然'' para indicar concessão}
  \seealsoref{虽然}{sui1 ran2}
  \end{phonetics}
\end{entry}

\begin{entry}{牌}{12}{⽚}
  \begin{phonetics}{牌}{pai2}[][HSK 4]
    \definition[块]{s.}{placa; tabuleta; quadro; placar | marca; marca registrada; marca comercial | cartas, dominó, etc. | a tonalidade de uma música}
  \end{phonetics}
\end{entry}

\begin{entry}{牌子}{12,3}{⽚、⼦}
  \begin{phonetics}{牌子}{pai2 zi5}[][HSK 3]
    \definition[个,种,块]{s.}{sinal; placa | marca; marca registrada}
  \end{phonetics}
\end{entry}

\begin{entry}{猩猩}{12,12}{⽝、⽝}
  \begin{phonetics}{猩猩}{xing1xing5}
    \definition{s.}{orangotango}
  \end{phonetics}
\end{entry}

\begin{entry}{猴子}{12,3}{⽝、⼦}
  \begin{phonetics}{猴子}{hou2zi5}
    \definition[只]{s.}{macaco}
  \end{phonetics}
\end{entry}

\begin{entry}{琴键}{12,13}{⽟、⾦}
  \begin{phonetics}{琴键}{qin2jian4}
    \definition{s.}{tecla de piano}
  \end{phonetics}
\end{entry}

\begin{entry}{甁}{12}{⽡}
  \begin{phonetics}{甁}{ping2}
    \variantof{瓶}
  \end{phonetics}
\end{entry}

\begin{entry}{番茄}{12,8}{⽥、⾋}
  \begin{phonetics}{番茄}{fan1qie2}
    \definition{s.}{tomate}
  \end{phonetics}
\end{entry}

\begin{entry}{痛}{12}{⽧}
  \begin{phonetics}{痛}{tong4}[][HSK 3]
    \definition*{s.}{sobrenome Tong}
    \definition{s.}{pesar; angústia; aflição; tristeza}
    \definition{v.}{doer; causar dor}
  \end{phonetics}
\end{entry}

\begin{entry}{痛快}{12,7}{⽧、⼼}
  \begin{phonetics}{痛快}{tong4kuai4}[][HSK 4]
    \definition{adj.}{encantado; alegre; muito feliz; confortável | franco; direto; simples e direto}
  \end{phonetics}
\end{entry}

\begin{entry}{痛苦}{12,8}{⽧、⾋}
  \begin{phonetics}{痛苦}{tong4ku3}[][HSK 3]
    \definition{adj.}{doloroso; angustiado}
    \definition[降,种]{s.}{dor; agonia; sofrimento}
  \end{phonetics}
\end{entry}

\begin{entry}{痛骂}{12,9}{⽧、⾺}
  \begin{phonetics}{痛骂}{tong4ma4}
    \definition{v.}{repreender severamente}
  \end{phonetics}
\end{entry}

\begin{entry}{痠}{12}{⽧}
  \begin{phonetics}{痠}{suan1}
    \definition{v.}{doer | estar dolorido}
    \variantof{酸}
  \end{phonetics}
\end{entry}

\begin{entry}{登}{12}{⽨}
  \begin{phonetics}{登}{deng1}[][HSK 4]
    \definition*{s.}{sobrenome Deng}
    \definition{v.}{subir; montar; escalar (uma altura) | publicar; registrar; inserir | ser colhidas e levadas para a eira | pressionar com o pé; pedalar; pisar | pisar em; pisar | calçar (calçados, etc.)}
  \end{phonetics}
\end{entry}

\begin{entry}{登山}{12,3}{⽨、⼭}
  \begin{phonetics}{登山}{deng1 shan1}[][HSK 4]
    \definition{s.}{escalar; fazer alpinismo; subir uma montanha}
  \end{phonetics}
\end{entry}

\begin{entry}{登记}{12,5}{⽨、⾔}
  \begin{phonetics}{登记}{deng1ji4}[][HSK 4]
    \definition{v.+compl.}{registrar-se; fazer o \emph{check-in} | registrar; reportar; informar; relatar por escrito a um superior ou autoridade relevante (usado principalmente para documentos legais)}
  \end{phonetics}
\end{entry}

\begin{entry}{登录}{12,8}{⽨、⼹}
  \begin{phonetics}{登录}{deng1lu4}[][HSK 4]
    \definition{v.}{fazer \emph{logon}; fazer \emph{login} | gravar; registrar; computadores eletrônicos e sua terminologia de rede, referindo-se ao acesso ao sistema operacional ou ao site a ser visitado}
  \end{phonetics}
\end{entry}

\begin{entry}{短}{12}{⽮}
  \begin{phonetics}{短}{duan3}[][HSK 2]
    \definition{adj.}{curto | breve}
  \end{phonetics}
\end{entry}

\begin{entry}{短少}{12,4}{⽮、⼩}
  \begin{phonetics}{短少}{duan3shao3}
    \definition{v.}{estar aquém do valor total}
  \end{phonetics}
\end{entry}

\begin{entry}{短处}{12,5}{⽮、⼡}
  \begin{phonetics}{短处}{duan3 chu4}[][HSK 3]
    \definition{s.}{deficiência; ponto fraco; defeito; fraqueza}
  \end{phonetics}
\end{entry}

\begin{entry}{短视}{12,8}{⽮、⾒}
  \begin{phonetics}{短视}{duan3shi4}
    \definition{adj.}{míope}
  \end{phonetics}
\end{entry}

\begin{entry}{短促}{12,9}{⽮、⼈}
  \begin{phonetics}{短促}{duan3cu4}
    \definition{adj.}{curto (tom de voz) | fugaz | ofegante (respiração) | curto no tempo}
  \end{phonetics}
\end{entry}

\begin{entry}{短信}{12,9}{⽮、⼈}
  \begin{phonetics}{短信}{duan3xin4}[][HSK 2]
    \definition{s.}{mensagem de texto}
  \end{phonetics}
\end{entry}

\begin{entry}{短缺}{12,10}{⽮、⽸}
  \begin{phonetics}{短缺}{duan3que1}
    \definition{s.}{escassez}
  \end{phonetics}
\end{entry}

\begin{entry}{短暂}{12,12}{⽮、⽇}
  \begin{phonetics}{短暂}{duan3zan4}
    \definition{adj.}{momentâneo | de curta duração}
  \end{phonetics}
\end{entry}

\begin{entry}{短期}{12,12}{⽮、⽉}
  \begin{phonetics}{短期}{duan3 qi1}[][HSK 3]
    \definition{adj.}{curto prazo}
    \definition[个]{s.}{curto período}
  \end{phonetics}
\end{entry}

\begin{entry}{短裤}{12,12}{⽮、⾐}
  \begin{phonetics}{短裤}{duan3 ku4}[][HSK 3]
    \definition[条]{s.}{calças curtas; calção; \emph{shorts}}
  \end{phonetics}
\end{entry}

\begin{entry}{短跑}{12,12}{⽮、⾜}
  \begin{phonetics}{短跑}{duan3pao3}
    \definition{s.}{corrida}
  \end{phonetics}
\end{entry}

\begin{entry}{硬件}{12,6}{⽯、⼈}
  \begin{phonetics}{硬件}{ying4jian4}
    \definition{s.}{\emph{hardware}}
  \end{phonetics}
\end{entry}

\begin{entry}{确}{12}{⽯}
  \begin{phonetics}{确}{que4}
    \definition{adj.}{autenticado | sólido | firme | real | verdadeiro}
  \end{phonetics}
\end{entry}

\begin{entry}{确认}{12,4}{⽯、⾔}
  \begin{phonetics}{确认}{que4ren4}[][HSK 4]
    \definition{v.}{afirmar; confirmar; reconhecer; confirmar explicitamente (fatos, princípios, etc.)}
  \end{phonetics}
\end{entry}

\begin{entry}{确定}{12,8}{⽯、⼧}
  \begin{phonetics}{确定}{que4ding4}[][HSK 3]
    \definition{adj.}{definido; certo}
    \definition{v.}{consertar; definir; determinar}
  \end{phonetics}
\end{entry}

\begin{entry}{确实}{12,8}{⽯、⼧}
  \begin{phonetics}{确实}{que4shi2}[][HSK 3]
    \definition{adj.}{verdadeiro; confiável}
    \definition{adv.}{verdadeiramente; realmente; de ​​fato}
  \end{phonetics}
\end{entry}

\begin{entry}{确保}{12,9}{⽯、⼈}
  \begin{phonetics}{确保}{que4bao3}[][HSK 3]
    \definition{v.}{assegurar; garantir}
  \end{phonetics}
\end{entry}

\begin{entry}{禅}{12}{⽰}
  \begin{phonetics}{禅}{chan2}
    \definition*{s.}{Zen}
    \definition{s.}{meditação (Budismo)}
  \end{phonetics}
  \begin{phonetics}{禅}{shan4}
    \definition{v.}{abdicar}
  \end{phonetics}
\end{entry}

\begin{entry}{禽}{12}{⽱}
  \begin{phonetics}{禽}{qin2}
    \definition*{s.}{sobrenome Qin}
    \definition[只]{s.}{aves; pássaros | termo genérico para aves e animais}
  \end{phonetics}
\end{entry}

\begin{entry}{程序}{12,7}{⽲、⼴}
  \begin{phonetics}{程序}{cheng2xu4}[][HSK 4]
    \definition[个,套,种]{s.}{ordem; curso; sequência; procedimento; ordem em que algo é feito; também, um determinado número de etapas em um trabalho | programa; conjunto de instruções de computador projetado em sequência para permitir que um computador execute uma ou mais operações}
  \end{phonetics}
\end{entry}

\begin{entry}{程序设计}{12,7,6,4}{⽲、⼴、⾔、⾔}
  \begin{phonetics}{程序设计}{cheng2xu4she4ji4}
    \definition{s.}{programação de computadores}
  \end{phonetics}
\end{entry}

\begin{entry}{程序库}{12,7,7}{⽲、⼴、⼴}
  \begin{phonetics}{程序库}{cheng2xu4ku4}
    \definition{s.}{biblioteca de funções e procedimentos para programas de computador}
  \end{phonetics}
\end{entry}

\begin{entry}{程度}{12,9}{⽲、⼴}
  \begin{phonetics}{程度}{cheng2du4}[][HSK 3]
    \definition[种]{s.}{nível; grau (de cultura, educação, aprendizagem, etc.) | extensão; grau}
  \end{phonetics}
\end{entry}

\begin{entry}{程控}{12,11}{⽲、⼿}
  \begin{phonetics}{程控}{cheng2kong4}
    \definition{s.}{programado | sob controle automático}
  \end{phonetics}
\end{entry}

\begin{entry}{稍}{12}{⽲}
  \begin{phonetics}{稍}{shao1}
    \definition{adv.}{um pouco | ligeiramente | em vez de}
  \end{phonetics}
\end{entry}

\begin{entry}{稍微}{12,13}{⽲、⼻}
  \begin{phonetics}{稍微}{shao1wei1}
    \definition{adv.}{um pouco}
  \end{phonetics}
\end{entry}

\begin{entry}{税}{12}{⽲}
  \begin{phonetics}{税}{shui4}
    \definition{s.}{taxas | impostos}
  \end{phonetics}
\end{entry}

\begin{entry}{窗子}{12,3}{⽳、⼦}
  \begin{phonetics}{窗子}{chuang1 zi5}[][HSK 4]
    \definition{s.}{janela}
  \end{phonetics}
\end{entry}

\begin{entry}{窗户}{12,4}{⽳、⼾}
  \begin{phonetics}{窗户}{chuang1hu5}[][HSK 4]
    \definition[个,扇,面,排]{s.}{janela; dispositivo de ventilação e transmissão de luz nas paredes}
  \end{phonetics}
\end{entry}

\begin{entry}{窗台}{12,5}{⽳、⼝}
  \begin{phonetics}{窗台}{chuang1 tai2}[][HSK 4]
    \definition{s.}{parapeito da janela; peitoril; parte plana de uma janela que segura a moldura}
  \end{phonetics}
\end{entry}

\begin{entry}{窗帘}{12,8}{⽳、⼱}
  \begin{phonetics}{窗帘}{chuang1lian2}[][HSK 5]
    \definition[副,幅,个,套,片,对]{s.}{cortinas para janelas}
  \end{phonetics}
\end{entry}

\begin{entry}{童年}{12,6}{⽴、⼲}
  \begin{phonetics}{童年}{tong2 nian2}[][HSK 4]
    \definition{s.}{infância; primeiros anos de vida}
  \end{phonetics}
\end{entry}

\begin{entry}{童话}{12,8}{⽴、⾔}
  \begin{phonetics}{童话}{tong2hua4}[][HSK 4]
    \definition[个,部]{s.}{conto de fadas; gênero de literatura infantil no qual as histórias adequadas para a diversão das crianças são escritas com muita imaginação, fantasia e exagero}
  \end{phonetics}
\end{entry}

\begin{entry}{等}{12}{⽵}
  \begin{phonetics}{等}{deng3}[][HSK 1,2]
    \definition{adj.}{igual}
    \definition{clas.}{para classe, grau, classificação | para tipo}
    \definition{prep.}{quando | até}
    \definition{v.}{esperar | aguardar}
  \end{phonetics}
\end{entry}

\begin{entry}{等于}{12,3}{⽵、⼆}
  \begin{phonetics}{等于}{deng3yu2}[][HSK 2]
    \definition{adv.}{igual a | equivalente a}
    \definition{v.}{equivaler a | ser equivalente a}
  \end{phonetics}
\end{entry}

\begin{entry}{等级}{12,6}{⽵、⽷}
  \begin{phonetics}{等级}{deng3ji2}[][HSK 5]
    \definition{s.}{grau; classificação; posição; distinções por qualidade, grau, status, etc. | estado social; estrato social; ordem e grau; grupos sociais desiguais em termos de status social e legal}
  \end{phonetics}
\end{entry}

\begin{entry}{等到}{12,8}{⽵、⼑}
  \begin{phonetics}{等到}{deng3 dao4}[][HSK 2]
    \definition{prep.}{pelo tempo | quando | espere até}
  \end{phonetics}
\end{entry}

\begin{entry}{等待}{12,9}{⽵、⼻}
  \begin{phonetics}{等待}{deng3dai4}[][HSK 3]
    \definition{v.}{esperar; aguardar}
  \end{phonetics}
\end{entry}

\begin{entry}{等候}{12,10}{⽵、⼈}
  \begin{phonetics}{等候}{deng3hou4}[][HSK 5]
    \definition{v.}{esperar; aguardar; expectar; usado principalmente para objetos específicos}
  \end{phonetics}
\end{entry}

\begin{entry}{筏}{12}{⽵}
  \begin{phonetics}{筏}{fa2}
    \definition{s.}{jangada (de troncos, bambus, etc.)}
  \end{phonetics}
\end{entry}

\begin{entry}{答}{12}{⽵}
  \begin{phonetics}{答}{da1}[][HSK 5]
    \definition{v.}{concordar; responder | responder; prestar atenção}
  \end{phonetics}
  \begin{phonetics}{答}{da2}[][HSK 5]
    \definition{v.}{responder; dar resposta a; responder a | retribuir; devolver (uma visita, etc.); retribuir um favor feito a alguém por outro; fazer o bem}
  \end{phonetics}
\end{entry}

\begin{entry}{答应}{12,7}{⽵、⼴}
  \begin{phonetics}{答应}{da1ying5}[][HSK 2]
    \definition{v.}{responder | concordar | prometer | cumprir com}
  \end{phonetics}
\end{entry}

\begin{entry}{答复}{12,9}{⽵、⼢}
  \begin{phonetics}{答复}{da2fu4}[][HSK 5]
    \definition[个]{s.}{resposta; respostas a perguntas ou solicitações}
    \definition{v.}{responder; dar uma resposta}
  \end{phonetics}
\end{entry}

\begin{entry}{答案}{12,10}{⽵、⽊}
  \begin{phonetics}{答案}{da2'an4}[][HSK 4]
    \definition[个]{s.}{chave; resposta; solução}
  \end{phonetics}
\end{entry}

\begin{entry}{策划}{12,6}{⽵、⼑}
  \begin{phonetics}{策划}{ce4hua4}
    \definition{s.}{planejador | produtor | plano}
    \definition{v.}{esquematizar | engenhar | planejar}
  \end{phonetics}
\end{entry}

\begin{entry}{粤语}{12,9}{⾔、⾔}
  \begin{phonetics}{粤语}{yue4yu3}
    \definition{s.}{cantonês | língua cantonesa}
  \end{phonetics}
\end{entry}

\begin{entry}{紫}{12}{⽷}
  \begin{phonetics}{紫}{zi3}
    \definition{adj.}{púrpura | violeta}
  \end{phonetics}
\end{entry}

\begin{entry}{紫色}{12,6}{⽷、⾊}
  \begin{phonetics}{紫色}{zi3 se4}
    \definition{s.}{cor púrpura | cor violeta}
  \end{phonetics}
\end{entry}

\begin{entry}{絫}{12}{⽷}
  \begin{phonetics}{絫}{lei3}
    \variantof{累}
  \end{phonetics}
\end{entry}

\begin{entry}{缓解}{12,13}{⽶、⾓}
  \begin{phonetics}{缓解}{huan3jie3}[][HSK 4]
    \definition{v.}{facilitar; aliviar; atenuar; amenizar; reduzir}
  \end{phonetics}
\end{entry}

\begin{entry}{编}{12}{⽷}
  \begin{phonetics}{编}{bian1}[][HSK 4]
    \definition*{s.}{sobrenome Bian}
    \definition{s.}{livro; volume; parte de um livro}
    \definition{v.}{tecer; trançar; entrançar | fazer uma lista; organizar em uma lista; organizar; agrupar | editar; compilar | compor; escrever | fabricar; inventar; fazer; preparar}
  \end{phonetics}
\end{entry}

\begin{entry}{编程}{12,12}{⽷、⽲}
  \begin{phonetics}{编程}{bian1cheng2}
    \definition{s.}{programa de computador}
    \definition{v.}{programar computador}
  \end{phonetics}
\end{entry}

\begin{entry}{编辑}{12,13}{⽷、⾞}
  \begin{phonetics}{编辑}{bian1ji2}[][HSK 5]
    \definition{v.}{editar; compilar; organizar e processar dados ou trabalhos existentes}
  \end{phonetics}
  \begin{phonetics}{编辑}{bian1ji5}[][HSK 5]
    \definition{s.}{editor; compilador; pessoa que organiza e processa dados ou trabalhos existentes}
  \end{phonetics}
\end{entry}

\begin{entry}{缘}{12}{⽷}
  \begin{phonetics}{缘}{yuan2}
    \definition{s.}{causa | razão | karma | destino | predestinação}
  \end{phonetics}
\end{entry}

\begin{entry}{缘分}{12,4}{⽷、⼑}
  \begin{phonetics}{缘分}{yuan2fen4}
    \definition{s.}{destino ou acaso que une as pessoas | afinidade ou relacionamento predestinado | destino (Budismo)}
  \end{phonetics}
\end{entry}

\begin{entry}{羡慕}{12,14}{⽺、⼼}
  \begin{phonetics}{羡慕}{xian4mu4}
    \definition{v.}{invejar | admirar}
  \end{phonetics}
\end{entry}

\begin{entry}{联合}{12,6}{⽿、⼝}
  \begin{phonetics}{联合}{lian2he2}[][HSK 3]
    \definition{adj.}{conjunto; unido; federal; combinado}
    \definition{s.}{aliado; união; aliança}
  \end{phonetics}
\end{entry}

\begin{entry}{联合会}{12,6,6}{⽿、⼝、⼈}
  \begin{phonetics}{联合会}{lian2he2hui4}
    \definition{s.}{federação}
  \end{phonetics}
\end{entry}

\begin{entry}{联合国}{12,6,8}{⽿、⼝、⼞}
  \begin{phonetics}{联合国}{lian2 he2 guo2}[][HSK 3]
    \definition*{s.}{Nações Unidas}
  \end{phonetics}
\end{entry}

\begin{entry}{联系}{12,7}{⽿、⽷}
  \begin{phonetics}{联系}{lian2xi4}[][HSK 3]
    \definition{s.}{relacionamento; conexão}
    \definition[个,种,层]{v.}{entrar em contato; contatar | organizar; entrar em contato com | relacionar; combinar; integrar}
  \end{phonetics}
\end{entry}

\begin{entry}{脾气}{12,4}{⾁、⽓}
  \begin{phonetics}{脾气}{pi2qi5}
    \definition{s.}{temperamento | humor | disposição | caráter}
  \end{phonetics}
\end{entry}

\begin{entry}{舒服}{12,8}{⾆、⽉}
  \begin{phonetics}{舒服}{shu1fu5}[][HSK 2]
    \definition{adj.}{estar confortável | bem disposto | sentir-se bem}
  \end{phonetics}
\end{entry}

\begin{entry}{舒适}{12,9}{⾆、⾡}
  \begin{phonetics}{舒适}{shu1shi4}[][HSK 4]
    \definition{adj.}{aconchegante; confortável; acolhedor; cômodo}
  \end{phonetics}
\end{entry}

\begin{entry}{落}{12}{⾋}
  \begin{phonetics}{落}{la4}
    \definition{v.}{deixar de fora; estar ausente | deixar para trás; esquecer de trazer | ficar para trás (ou cair)}
  \end{phonetics}
  \begin{phonetics}{落}{lao4}
    \definition{v.}{cair | descer | ficar; fazer escala; deixar para trás | obter; ter; receber}
  \end{phonetics}
  \begin{phonetics}{落}{luo4}[][HSK 4]
    \definition*{s.}{sobrenome Luo}
    \definition{s.}{paradeiro; lugar para ficar; local de descanso | assentamento; local de reunião | parte curta; área pequena; refere-se a um pequeno lugar ou área}
    \definition{v.}{cair | descer | baixar; deixar cair (ou descer) | afundar; declinar; cair; (figurativo) mudança da prosperidade para o declínio | ficar para trás; deixar para trás ou do lado de fora | ficar; parar; deixar para trás | cair em cima de; descansar com | obter; ter; receber | escrever; colocar a caneta no papel | cair em; entrar em}
  \end{phonetics}
\end{entry}

\begin{entry}{落日}{12,4}{⾋、⽇}
  \begin{phonetics}{落日}{luo4ri4}
    \definition{s.}{pôr do sol}
  \end{phonetics}
\end{entry}

\begin{entry}{落后}{12,6}{⾋、⼝}
  \begin{phonetics}{落后}{luo4hou4}[][HSK 3]
    \definition{adj.}{atrasado}
    \definition{s.}{atraso}
    \definition{v.}{ficar para trás | atrasar}
    \definition{v.}{atrasar-se; ficar para trás}
  \end{phonetics}
\end{entry}

\begin{entry}{落汤鸡}{12,6,7}{⾋、⽔、⿃}
  \begin{phonetics}{落汤鸡}{luo4tang1ji1}
    \definition{s.}{uma pessoa que parece encharcada e acamada| sofrimento profundo}
  \end{phonetics}
\end{entry}

\begin{entry}{葡}{12}{⾋}
  \begin{phonetics}{葡}{pu2}
    \definition*{s.}{Portugal, abreviação de 葡萄牙}
    \seeref{葡萄牙}{pu2tao2ya2}
  \end{phonetics}
\end{entry}

\begin{entry}{葡文}{12,4}{⾋、⽂}
  \begin{phonetics}{葡文}{pu2wen2}
    \definition{s.}{português, língua portuguesa}
    \seeref{葡萄牙文}{pu2tao2ya2wen2}
  \end{phonetics}
\end{entry}

\begin{entry}{葡汉词典}{12,5,7,8}{⾋、⽔、⾔、⼋}
  \begin{phonetics}{葡汉词典}{pu2-han4 ci2dian3}
    \definition{s.}{dicionário português-chinês}
  \seealsoref{汉葡词典}{han4-pu2 ci2dian3}
  \end{phonetics}
\end{entry}

\begin{entry}{葡语}{12,9}{⾋、⾔}
  \begin{phonetics}{葡语}{pu2yu3}
    \definition{s.}{português, língua portuguesa}
    \seeref{葡萄牙语}{pu2tao2ya2yu3}
  \end{phonetics}
\end{entry}

\begin{entry}{葡萄}{12,11}{⾋、⾋}
  \begin{phonetics}{葡萄}{pu2tao5}
    \definition{s.}{uva}
  \end{phonetics}
\end{entry}

\begin{entry}{葡萄牙}{12,11,4}{⾋、⾋、⽛}
  \begin{phonetics}{葡萄牙}{pu2tao2ya2}
    \definition{s.}{Portugal}
    \seeref{葡}{pu2}
  \end{phonetics}
\end{entry}

\begin{entry}{葡萄牙文}{12,11,4,4}{⾋、⾋、⽛、⽂}
  \begin{phonetics}{葡萄牙文}{pu2tao2ya2wen2}
    \definition{s.}{português, língua portuguesa}
    \seeref{葡文}{pu2wen2}
  \end{phonetics}
\end{entry}

\begin{entry}{葡萄牙语}{12,11,4,9}{⾋、⾋、⽛、⾔}
  \begin{phonetics}{葡萄牙语}{pu2tao2ya2yu3}
    \definition{s.}{português, língua portuguesa}
    \seeref{葡语}{pu2yu3}
  \end{phonetics}
\end{entry}

\begin{entry}{葫芦}{12,7}{⾋、⾋}
  \begin{phonetics}{葫芦}{hu2lu5}
    \definition{adj.}{confuso}
    \definition{s.}{cabaça | termo genérico para bloco e equipamento (ou partes dele)}
  \end{phonetics}
\end{entry}

\begin{entry}{葬}{12}{⾋}
  \begin{phonetics}{葬}{zang4}
    \definition{v.}{enterrar (os mortos) | sepultar}
  \end{phonetics}
\end{entry}

\begin{entry}{葱}{12}{⾋}
  \begin{phonetics}{葱}{cong1}
    \definition{s.}{cebolinha}
  \end{phonetics}
\end{entry}

\begin{entry}{葵花}{12,7}{⾋、⾋}
  \begin{phonetics}{葵花}{kui2hua1}
    \definition{s.}{girassol (flor)}
  \end{phonetics}
\end{entry}

\begin{entry}{街}{12}{⾏}
  \begin{phonetics}{街}{jie1}[][HSK 2]
    \definition[条]{s.}{rua}
  \end{phonetics}
\end{entry}

\begin{entry}{街道}{12,12}{⾏、⾡}
  \begin{phonetics}{街道}{jie1dao4}[][HSK 4]
    \definition[条]{s.}{caminho; rua; estrada; via pública com casas em ambos os lados, relativamente larga | escritório do subdistrito; tipo de organização responsável por gerenciar determinados aspectos da rua}
  \end{phonetics}
\end{entry}

\begin{entry}{街舞}{12,14}{⾏、⾇}
  \begin{phonetics}{街舞}{jie1wu3}
    \definition{s.}{dança de rua, \emph{street dance} (por exemplo, \emph{breakdance})}
  \end{phonetics}
\end{entry}

\begin{entry}{裁}{12}{⾐}
  \begin{phonetics}{裁}{cai2}
    \definition{s.}{decisão | julgamento}
    \definition{v.}{recortar (tecido de uma roupa) | cortar | aparar | reduzir | diminuir | cortar pessoal de uma equipe}
  \end{phonetics}
\end{entry}

\begin{entry}{裁判}{12,7}{⾐、⼑}
  \begin{phonetics}{裁判}{cai2pan4}[][HSK 5]
    \definition[个,位,名]{s.}{árbitro; juiz; pessoa que desempenha funções de arbitragem em esportes e outras competições}
    \definition{v.}{arbitrar; atuar como árbitro; em esportes e outras atividades competitivas, julgar o desempenho dos atletas, vitórias e derrotas, classificações e problemas que ocorrem durante a competição de acordo com as regras da competição | julgar; refere-se a um terceiro que faz um julgamento quando surge uma disputa entre duas partes}
  \end{phonetics}
\end{entry}

\begin{entry}{装}{12}{⾐}
  \begin{phonetics}{装}{zhuang1}[][HSK 2]
    \definition{s.}{adorno | roupa | traje (de um ator em uma peça)}
    \definition{v.}{adornar | vestir | desepenhar um papel | fingir | instalar | consertar | embrulhar (algo em um saco) | empacotar}
  \end{phonetics}
\end{entry}

\begin{entry}{装扮}{12,7}{⾐、⼿}
  \begin{phonetics}{装扮}{zhuang1ban4}
    \definition{v.}{enfeitar | decorar | disfarçar-me | vestir-se}
  \end{phonetics}
\end{entry}

\begin{entry}{装备}{12,8}{⾐、⼡}
  \begin{phonetics}{装备}{zhuang1bei4}
    \definition{s.}{equipamento}
    \definition{v.}{equipar}
  \end{phonetics}
\end{entry}

\begin{entry}{装修}{12,9}{⾐、⼈}
  \begin{phonetics}{装修}{zhuang1 xiu1}[][HSK 4]
    \definition{v.}{equipar; renovar; decorar (equipar uma sala ou prédio com equipamentos ou decorações)}
  \end{phonetics}
\end{entry}

\begin{entry}{装配}{12,10}{⾐、⾣}
  \begin{phonetics}{装配}{zhuang1pei4}
    \definition{v.}{montar | encaixar}
  \end{phonetics}
\end{entry}

\begin{entry}{装置}{12,13}{⾐、⽹}
  \begin{phonetics}{装置}{zhuang1 zhi4}[][HSK 4]
    \definition{s.}{dispositivo; equipamento; máquinas, instrumentos ou outros equipamentos de construção mais complexa e com alguma função independente}
    \definition{v.}{instalar; ajustar; configurar; equipar; montar}
  \end{phonetics}
\end{entry}

\begin{entry}{裙子}{12,3}{⾐、⼦}
  \begin{phonetics}{裙子}{qun2zi5}[][HSK 3]
    \definition[条,件]{s.}{saia (peça de vestuário)}
  \end{phonetics}
\end{entry}

\begin{entry}{裤子}{12,3}{⾐、⼦}
  \begin{phonetics}{裤子}{ku4zi5}[][HSK 3]
    \definition[条]{s.}{calças}
  \end{phonetics}
\end{entry}

\begin{entry}{詈骂}{12,9}{⾔、⾺}
  \begin{phonetics}{詈骂}{li4ma4}
    \definition{v.}{xingar | abusar}
  \end{phonetics}
\end{entry}

\begin{entry}{谢天谢地}{12,4,12,6}{⾔、⼤、⾔、⼟}
  \begin{phonetics}{谢天谢地}{xie4tian1xie4di4}
    \definition{expr.}{agradecer a Deus | agradecer aos céus}
  \end{phonetics}
\end{entry}

\begin{entry}{谢世}{12,5}{⾔、⼀}
  \begin{phonetics}{谢世}{xie4shi4}
    \definition{v.}{morrer | falecer}
  \end{phonetics}
\end{entry}

\begin{entry}{谢恩}{12,10}{⾔、⼼}
  \begin{phonetics}{谢恩}{xie4'en1}
    \definition{v.}{agradecer a alguém pelo favor (especialmente imperador ou oficial superior)}
  \end{phonetics}
\end{entry}

\begin{entry}{谢病}{12,10}{⾔、⽧}
  \begin{phonetics}{谢病}{xie4bing4}
    \definition{v.}{desculpar-se por causa de doença}
  \end{phonetics}
\end{entry}

\begin{entry}{谢媒}{12,12}{⾔、⼥}
  \begin{phonetics}{谢媒}{xie4mei2}
    \definition{v.}{agradecer ao casamenteiro}
  \end{phonetics}
\end{entry}

\begin{entry}{谢谢}{12,12}{⾔、⾔}
  \begin{phonetics}{谢谢}{xie4xie5}[][HSK 1]
    \definition{interj.}{Obrigado!}
    \definition{v.}{agradecer}
  \end{phonetics}
\end{entry}

\begin{entry}{谢意}{12,13}{⾔、⼼}
  \begin{phonetics}{谢意}{xie4yi4}
    \definition{s.}{gratidão}
  \end{phonetics}
\end{entry}

\begin{entry}{貂}{12}{⾘}
  \begin{phonetics}{貂}{diao1}
    \definition{s.}{marta | fuinha}
  \end{phonetics}
\end{entry}

\begin{entry}{赏}{12}{⾙}
  \begin{phonetics}{赏}{shang3}[][HSK 4]
    \definition*{s.}{sobrenome Shang}
    \definition{s.}{recompensa; prêmio}
    \definition{v.}{conceder (outorgar) uma recompensa; recompensar; premiar | admirar; desfrutar; apreciar; valorizar}
  \end{phonetics}
\end{entry}

\begin{entry}{赏心悦目}{12,4,10,5}{⾙、⼼、⼼、⽬}
  \begin{phonetics}{赏心悦目}{shang3xin1yue4mu4}
    \definition{expr.}{``Aquece o coração e encanta os olhos.''}
  \end{phonetics}
\end{entry}

\begin{entry}{赏赐}{12,12}{⾙、⾙}
  \begin{phonetics}{赏赐}{shang3ci4}
    \definition{s.}{recompensa | prêmio}
    \definition{v.}{recompensar | premiar}
  \end{phonetics}
\end{entry}

\begin{entry}{赔钱}{12,10}{⾙、⾦}
  \begin{phonetics}{赔钱}{pei2qian2}
    \definition{v.+compl.}{perder dinheiro | pagar pelos danos}
  \end{phonetics}
\end{entry}

\begin{entry}{超市}{12,5}{⾛、⼱}
  \begin{phonetics}{超市}{chao1shi4}[][HSK 2]
    \definition[家]{s.}{supermercado}
  \end{phonetics}
\end{entry}

\begin{entry}{超级}{12,6}{⾛、⽷}
  \begin{phonetics}{超级}{chao1ji2}[][HSK 3]
    \definition{adj.}{super}
    \definition{pref.}{super-; ultra-; hiper-}
  \end{phonetics}
\end{entry}

\begin{entry}{超过}{12,6}{⾛、⾡}
  \begin{phonetics}{超过}{chao1guo4}[][HSK 2]
    \definition{v.}{passar | ultrapassar (alguém ou algo) | exceder | ser mais do que | estar acima de (um padrão)}
  \end{phonetics}
\end{entry}

\begin{entry}{超声}{12,7}{⾛、⼠}
  \begin{phonetics}{超声}{chao1sheng1}
    \definition{adj.}{ultrasônico}
    \definition{s.}{ultrasom}
  \end{phonetics}
\end{entry}

\begin{entry}{超越}{12,12}{⾛、⾛}
  \begin{phonetics}{超越}{chao1yue4}[][HSK 5]
    \definition{v.}{ultrapassar; superar; passar por cima; transcender}
  \end{phonetics}
\end{entry}

\begin{entry}{越}{12}{⾛}
  \begin{phonetics}{越}{yue4}[][HSK 2]
    \definition{adv.}{quanto mais\dots mais}
    \definition{v.}{subir | exceder | superar}
  \end{phonetics}
\end{entry}

\begin{entry}{越来越……}{12,7,12}{⾛、⽊、⾛}
  \begin{phonetics}{越来越……}{yue4lai2yue4}[][HSK 2]
    \definition{adv.}{cada vez mais\dots}
  \end{phonetics}
\end{entry}

\begin{entry}{越……越……}{12,12}{⾛、⾛}
  \begin{phonetics}{越……越……}{yue4 yue4}[][HSK 2]
    \definition{expr.}{quanto mais\dots tanto mais\dots}
  \end{phonetics}
\end{entry}

\begin{entry}{越障}{12,13}{⾛、⾩}
  \begin{phonetics}{越障}{yue4zhang4}
    \definition{s.}{curso com obstáculos para treinamento de tropas}
    \definition{v.}{superar obstáculos}
  \end{phonetics}
\end{entry}

\begin{entry}{越境}{12,14}{⾛、⼟}
  \begin{phonetics}{越境}{yue4jing4}
    \definition{v.}{cruzar uma fronteira (geralmente ilegalmente) | entrar ou sair furtivamente de um país}
  \end{phonetics}
\end{entry}

\begin{entry}{趋势}{12,8}{⾛、⼒}
  \begin{phonetics}{趋势}{qu1shi4}[][HSK 4]
    \definition{s.}{tendência; tendência; direção; impulso das coisas que se movem em uma direção ou outra}
  \end{phonetics}
\end{entry}

\begin{entry}{跑}{12}{⾜}
  \begin{phonetics}{跑}{pao2}
    \definition{v.}{(de um animal) dar patadas (no chão)}
  \end{phonetics}
  \begin{phonetics}{跑}{pao3}[][HSK 1]
    \definition{v.}{vazar ou evaporar (sobre um gás ou líquido) | escapar | correr | correr (em tarefas, etc.) | fugir}
  \end{phonetics}
\end{entry}

\begin{entry}{跑马}{12,3}{⾜、⾺}
  \begin{phonetics}{跑马}{pao3ma3}
    \definition{s.}{corrida de cavalos}
    \definition{v.}{andar a cavalo em ritmo acelerado}
  \end{phonetics}
\end{entry}

\begin{entry}{跑步}{12,7}{⾜、⽌}
  \begin{phonetics}{跑步}{pao3bu4}[][HSK 3]
    \definition{s.}{corrida}
    \definition{v.+compl.}{correr; trotar}
  \end{phonetics}
\end{entry}

\begin{entry}{跑肚}{12,7}{⾜、⾁}
  \begin{phonetics}{跑肚}{pao3du4}
    \definition{v.}{(coloquial) ter diarréia}
  \end{phonetics}
\end{entry}

\begin{entry}{跑调}{12,10}{⾜、⾔}
  \begin{phonetics}{跑调}{pao3diao4}
    \definition{v.}{(coloquial) estar fora do tom ou desafinado (enquanto canta)}
  \end{phonetics}
\end{entry}

\begin{entry}{跑掉}{12,11}{⾜、⼿}
  \begin{phonetics}{跑掉}{pao3diao4}
    \definition{v.}{fugir}
  \end{phonetics}
\end{entry}

\begin{entry}{跑腿}{12,13}{⾜、⾁}
  \begin{phonetics}{跑腿}{pao3tui3}
    \definition{v.}{realizar tarefas}
  \end{phonetics}
\end{entry}

\begin{entry}{跑酷}{12,14}{⾜、⾣}
  \begin{phonetics}{跑酷}{pao3ku4}
    \definition*{s.}{(empréstimo linguístico) \emph{Parkour}}
  \end{phonetics}
\end{entry}

\begin{entry}{跑题}{12,15}{⾜、⾴}
  \begin{phonetics}{跑题}{pao3ti2}
    \definition{v.}{divagar | fugir do assunto | tergiversar}
  \end{phonetics}
\end{entry}

\begin{entry}{辈}{12}{⾞}
  \begin{phonetics}{辈}{bei4}[][HSK 5]
    \definition{s.}{geração da família | semelhante; círculo familiar; pessoas de um determinado tipo | vida útil; tempo de vida}
  \end{phonetics}
\end{entry}

\begin{entry}{遇}{12}{⾡}
  \begin{phonetics}{遇}{yu4}[][HSK 4]
    \definition*{s.}{sobrenome Yu}
    \definition{s.}{chance; oportunidade}
    \definition{v.}{encontrar; deparar-se com; encontrar-se | tratar; receber}
  \end{phonetics}
\end{entry}

\begin{entry}{遇见}{12,4}{⾡、⾒}
  \begin{phonetics}{遇见}{yu4 jian4}[][HSK 4]
    \definition{v.}{encontrar; deparar-se com}
  \end{phonetics}
\end{entry}

\begin{entry}{遇到}{12,8}{⾡、⼑}
  \begin{phonetics}{遇到}{yu4dao4}[][HSK 4]
    \definition{v.}{esbarrar em; encontrar; deparar-se com; conhecer alguém ou algo (inesperado)}
  \end{phonetics}
\end{entry}

\begin{entry}{遍}{12}{⾡}
  \begin{phonetics}{遍}{bian4}[][HSK 2]
    \definition{adv.}{em todos os lugares | por toda parte}
    \definition{clas.}{para a repetição de ações de leitura, fala ou escrita}
  \end{phonetics}
\end{entry}

\begin{entry}{道}{12}{⾡}
  \begin{phonetics}{道}{dao4}[][HSK 2]
    \definition*{s.}{Taoism | Taoist}
    \definition*{s.}{sobrenome Dao}
    \definition{s.}{estrada | caminho | rota | caminho | canal | curso | maneira | método | moral | moralidade | doutrina | corpo de ensinamentos morais | o Caminho da Natureza que não pode receber um nome | princípio | seita supersticiosa | linha | trato | habilidade}
  \end{phonetics}
\end{entry}

\begin{entry}{道理}{12,11}{⾡、⽟}
  \begin{phonetics}{道理}{dao4li5}[][HSK 2]
    \definition[个]{s.}{razão | argumento | sentido | princípio | base | justificativa}
  \end{phonetics}
\end{entry}

\begin{entry}{道路}{12,13}{⾡、⾜}
  \begin{phonetics}{道路}{dao4 lu4}[][HSK 2]
    \definition{s.}{estrada | caminho | processo}
  \end{phonetics}
\end{entry}

\begin{entry}{道歉}{12,14}{⾡、⽋}
  \begin{phonetics}{道歉}{dao4qian4}
    \definition{v.+compl.}{desculpar-se | fazer um pedido de desculpas}
  \end{phonetics}
\end{entry}

\begin{entry}{道德}{12,15}{⾡、⼻}
  \begin{phonetics}{道德}{dao4de2}[][HSK 5]
    \definition{adj.}{moral; descreve uma pessoa ou comportamento que atende aos requisitos morais; mais usado em situações negativas}
    \definition{s.}{moral; ética; moralidade; regras e normas para que as pessoas vivam juntas e se comportem em comum}
  \end{phonetics}
\end{entry}

\begin{entry}{遗产}{12,6}{⾡、⼇}
  \begin{phonetics}{遗产}{yi2chan3}[][HSK 4]
    \definition[笔,份]{s.}{legado; herança; patrimônio; propriedade deixada pelo falecido | patrimônio; riqueza cultural ou riqueza material transmitida pela história}
  \end{phonetics}
\end{entry}

\begin{entry}{遗传}{12,6}{⾡、⼈}
  \begin{phonetics}{遗传}{yi2chuan2}[][HSK 4]
    \definition{v.}{herdar, descender, transmitir, passar adiante}
  \end{phonetics}
\end{entry}

\begin{entry}{遗男}{12,7}{⾡、⽥}
  \begin{phonetics}{遗男}{yi2nan2}
    \definition{s.}{órfão | filho póstumo}
  \end{phonetics}
\end{entry}

\begin{entry}{遗迹}{12,9}{⾡、⾡}
  \begin{phonetics}{遗迹}{yi2ji4}
    \definition{s.}{vestígios históricos | remanescente | vestígio}
  \end{phonetics}
\end{entry}

\begin{entry}{遗案}{12,10}{⾡、⽊}
  \begin{phonetics}{遗案}{yi2'an4}
    \definition{s.}{(lei) caso não resolvido}
  \end{phonetics}
\end{entry}

\begin{entry}{遗落}{12,12}{⾡、⾋}
  \begin{phonetics}{遗落}{yi2luo4}
    \definition{v.}{esquecer | deixar para trás (inadvertidamente) | deixar de fora | omitir}
  \end{phonetics}
\end{entry}

\begin{entry}{遗嘱}{12,15}{⾡、⼝}
  \begin{phonetics}{遗嘱}{yi2zhu3}
    \definition{s.}{testamento}
  \end{phonetics}
\end{entry}

\begin{entry}{遗骸}{12,15}{⾡、⾻}
  \begin{phonetics}{遗骸}{yi2hai2}
    \definition{v.}{restos mortais}
  \end{phonetics}
\end{entry}

\begin{entry}{遗憾}{12,16}{⾡、⼼}
  \begin{phonetics}{遗憾}{yi2han4}
    \definition{v.}{ter pena de | lamentar}
  \end{phonetics}
\end{entry}

\begin{entry}{酢}{12}{⾣}
  \begin{phonetics}{酢}{cu4}
    \variantof{醋}
  \end{phonetics}
  \begin{phonetics}{酢}{zuo4}
    \definition{v.}{brindar o anfitrião com vinho}
  \end{phonetics}
\end{entry}

\begin{entry}{量}{12}{⾥}
  \begin{phonetics}{量}{liang2}[][HSK 4]
    \definition{v.}{medir | estimar; dimensionar}
  \end{phonetics}
  \begin{phonetics}{量}{liang4}
    \definition{s.}{instrumento de medida; antigamente, o termo se referia a objetos como baldes e litros, que medem o volume | capacidade (para tolerância ou ingestão de alimentos ou bebidas); refere-se ao limite do que pode ser acomodado | quantidade; valor; volume; número}
    \definition{v.}{estimar; medir; pesar}
  \end{phonetics}
\end{entry}

\begin{entry}{铺}{12}{⾦}
  \begin{phonetics}{铺}{pu1}
    \definition{v.}{espalhar | exibir | montar}
  \end{phonetics}
  \begin{phonetics}{铺}{pu4}
    \definition{s.}{cama de tábua | lugar para dormir | loja | depósito}
  \end{phonetics}
\end{entry}

\begin{entry}{铺垫}{12,9}{⾦、⼟}
  \begin{phonetics}{铺垫}{pu1dian4}
    \definition{s.}{cobre leito | colcha | roupa de cama}
    \definition{v.}{pavimentar}
  \end{phonetics}
\end{entry}

\begin{entry}{销售}{12,11}{⾦、⼝}
  \begin{phonetics}{销售}{xiao1shou4}[][HSK 4]
    \definition{v.}{vender; comercializar}
  \end{phonetics}
\end{entry}

\begin{entry}{锅}{12}{⾦}
  \begin{phonetics}{锅}{guo1}
    \definition[口,只]{s.}{panela | frigideira | \emph{wok} | caldeirão | coisa em forma de pote}
  \end{phonetics}
\end{entry}

\begin{entry}{隔}{12}{⾩}
  \begin{phonetics}{隔}{ge2}[][HSK 4]
    \definition{adj.}{seguinte; vizinho}
    \definition{v.}{dividir; separar; bloquear; obstruir | estar a uma distância de, após ou em um intervalo de}
  \end{phonetics}
\end{entry}

\begin{entry}{隔开}{12,4}{⾩、⼶}
  \begin{phonetics}{隔开}{ge2 kai1}[][HSK 4]
    \definition{v.}{separar; manter separado; barricar; separar completamente duas pessoas (ou coisas) ou duas partes de uma coisa que estão intimamente unidas}
  \end{phonetics}
\end{entry}

\begin{entry}{隔壁}{12,16}{⾩、⼟}
  \begin{phonetics}{隔壁}{ge2bi4}[][HSK 5]
    \definition{s.}{vizinho; casas ou pessoas vizinhas | septo; distante (socialmente distante) | anteparo; partição}
  \end{phonetics}
\end{entry}

\begin{entry}{集中}{12,4}{⾫、⼁}
  \begin{phonetics}{集中}{ji2zhong1}[][HSK 3]
    \definition{adj.}{centralizado; concentrado}
    \definition{v.}{concentrar; juntar}
  \end{phonetics}
\end{entry}

\begin{entry}{集合}{12,6}{⾫、⼝}
  \begin{phonetics}{集合}{ji2he2}[][HSK 4]
    \definition{v.}{reunir-se; juntar-se | reunir, juntar, convocar}
  \end{phonetics}
\end{entry}

\begin{entry}{集团}{12,6}{⾫、⼞}
  \begin{phonetics}{集团}{ji2tuan2}
    \definition{s.}{grupo | bloco | corporação | conglomerado}
  \end{phonetics}
\end{entry}

\begin{entry}{集体}{12,7}{⾫、⼈}
  \begin{phonetics}{集体}{ji2ti3}[][HSK 3]
    \definition{s.}{coletivo; comunidade; grupo; equipe}
  \end{phonetics}
\end{entry}

\begin{entry}{韩国}{12,8}{⾱、⼞}
  \begin{phonetics}{韩国}{han2guo2}
    \definition*{s.}{Coréia do Sul}
  \end{phonetics}
\end{entry}

\begin{entry}{韩国人}{12,8,2}{⾱、⼞、⼈}
  \begin{phonetics}{韩国人}{han2guo2ren2}
    \definition{s.}{coreano | pessoa ou povo da Coréia}
  \end{phonetics}
\end{entry}

\begin{entry}{骚乱}{12,7}{⾺、⼄}
  \begin{phonetics}{骚乱}{sao1luan4}
    \definition{s.}{rebelião | perturbação | tumulto}
    \definition{v.}{criar um distúrbio}
  \end{phonetics}
\end{entry}

\begin{entry}{黍}{12}{⿉}[Kangxi 202]
  \begin{phonetics}{黍}{shu3}
    \definition{s.}{painço}
  \end{phonetics}
\end{entry}

\begin{entry}{黑}{12}{⿊}[Kangxi 203]
  \begin{phonetics}{黑}{hei1}[][HSK 2]
    \definition{adj.}{preto | escuro | ilegal | secreto | sombrio | sinistro}
    \definition{v.}{esconder (algo) | difamar | (empréstimo linguístico) (computador) hackear}
  \end{phonetics}
\end{entry}

\begin{entry}{黑色}{12,6}{⿊、⾊}
  \begin{phonetics}{黑色}{hei1 se4}[][HSK 2]
    \definition{s.}{cor preta}
  \end{phonetics}
\end{entry}

\begin{entry}{黑板}{12,8}{⿊、⽊}
  \begin{phonetics}{黑板}{hei1ban3}[][HSK 2]
    \definition[块,个]{s.}{quadro negro}
  \end{phonetics}
\end{entry}

\begin{entry}{黑客}{12,9}{⿊、⼧}
  \begin{phonetics}{黑客}{hei1ke4}
    \definition{s.}{(empréstimo linguístico) (computação) \emph{hacker}}
  \end{phonetics}
\end{entry}

\begin{entry}{黑暗}{12,13}{⿊、⽇}
  \begin{phonetics}{黑暗}{hei1 an4}[][HSK 4]
    \definition{adj.}{escuro; sombrio; sem luz | maligno; corrupto; reacionário}
  \end{phonetics}
\end{entry}

\begin{entry}{黹}{12}{⿋}[Kangxi 204]
  \begin{phonetics}{黹}{zhi3}
    \definition{v.}{costurar; bordar}
  \end{phonetics}
\end{entry}

\begin{entry}{鼎}{12}{⿍}[Kangxi 206]
  \begin{phonetics}{鼎}{ding3}
    \definition{adj.}{importante; significativo}
    \definition{adv.}{exatamente quando; exatamente o momento para}
    \definition[尊]{s.}{um antigo recipiente de cozinha com duas alças e três ou quatro pernas | pote; caldeirão}
  \end{phonetics}
\end{entry}

%%%%% EOF %%%%%


%%%
%%% 13画
%%%

\section*{13画}\addcontentsline{toc}{section}{13画}

\begin{entry}{傻}{13}{⼈}
  \begin{phonetics}{傻}{sha3}[][HSK 5]
    \definition{adj.}{estúpido; confuso; burro; idiota; inflexível}
  \end{phonetics}
\end{entry}

\begin{entry}{傻瓜}{13,5}{⼈、⽠}
  \begin{phonetics}{傻瓜}{sha3gua1}
    \definition{adj.}{tolo | burro | simplório | idiota}
    \definition{v.}{enganar | iludir | lograr}
  \end{phonetics}
\end{entry}

\begin{entry}{傻眼}{13,11}{⼈、⽬}
  \begin{phonetics}{傻眼}{sha3yan3}
    \definition{adj.}{estupefato | atordoado}
  \end{phonetics}
\end{entry}

\begin{entry}{像}{13}{⼈}
  \begin{phonetics}{像}{xiang4}[][HSK 2]
    \definition{s.}{imagem | retrato | aparência}
    \definition{v.}{assemelhar-se | ser como}
  \end{phonetics}
\end{entry}

\begin{entry}{勤奋}{13,8}{⼒、⼤}
  \begin{phonetics}{勤奋}{qin2fen4}[][HSK 5]
    \definition{adj.}{diligente; assíduo; trabalhador; descreve alguém que se esforça continuamente nos estudos ou no trabalho}
  \end{phonetics}
\end{entry}

\begin{entry}{嗄}{13}{⼝}
  \begin{phonetics}{嗄}{a2}
    \definition{adj.}{rouco}
    \variantof{啊}
  \end{phonetics}
\end{entry}

\begin{entry}{嗅}{13}{⼝}
  \begin{phonetics}{嗅}{xiu4}
    \definition{v.}{cheirar; farejar; identificar odores pelo nariz}
  \end{phonetics}
\end{entry}

\begin{entry}{嗡嗡}{13,13}{⼝、⼝}
  \begin{phonetics}{嗡嗡}{weng1weng1}
    \definition{s.}{zumbido}
    \definition{v.}{zumbir}
  \end{phonetics}
\end{entry}

\begin{entry}{嘟}{13}{⼝}
  \begin{phonetics}{嘟}{du1}
    \definition{s.}{buzina | bip}
    \definition{v.}{fazer beicinho}
  \end{phonetics}
\end{entry}

\begin{entry}{塑料}{13,10}{⼟、⽃}
  \begin{phonetics}{塑料}{su4 liao4}[][HSK 4]
    \definition[块,种]{s.}{plástico; compostos de polímeros feitos de resinas naturais ou sintéticas como componente principal}
  \end{phonetics}
\end{entry}

\begin{entry}{塑料袋}{13,10,11}{⼟、⽃、⾐}
  \begin{phonetics}{塑料袋}{su4liao4dai4}[][HSK 4]
    \definition{s.}{saco plástico; sacola de plástico}
  \end{phonetics}
\end{entry}

\begin{entry}{填}{13}{⼟}
  \begin{phonetics}{填}{tian2}
    \definition{v.}{encher; rechear | reabastecer; suplementar; complementar | preencher; escrever dados em uma caixa (em um questionário ou formulário da \emph{Web})}
  \end{phonetics}
\end{entry}

\begin{entry}{填空}{13,8}{⼟、⽳}
  \begin{phonetics}{填空}{tian2kong4}[][HSK 4]
    \definition{v.}{preencher o espaço em branco (por exemplo, em um teste)}
  \end{phonetics}
\end{entry}

\begin{entry}{嫉妒}{13,7}{⼥、⼥}
  \begin{phonetics}{嫉妒}{ji2du4}
    \definition{v.}{estar com ciúmes de | invejar}
  \end{phonetics}
\end{entry}

\begin{entry}{幕}{13}{⼱}
  \begin{phonetics}{幕}{mu4}
    \definition{s.}{cortina ou tela | dossel ou tenda | quartel de um general | ato (de uma peça)}
  \end{phonetics}
\end{entry}

\begin{entry}{彀}{13}{⼸}
  \begin{phonetics}{彀}{gou4}
    \definition{s.}{calcance de um arco e flecha}
    \definition{v.}{puxar um arco ao máximo}
  \end{phonetics}
\end{entry}

\begin{entry}{微风}{13,4}{⼻、⾵}
  \begin{phonetics}{微风}{wei1feng1}
    \definition{s.}{brisa | vento leve}
  \end{phonetics}
\end{entry}

\begin{entry}{微软}{13,8}{⼻、⾞}
  \begin{phonetics}{微软}{wei1ruan3}
    \definition*{s.}{\emph{Microsoft Corporation}}
  \end{phonetics}
\end{entry}

\begin{entry}{微信}{13,9}{⼻、⼈}
  \begin{phonetics}{微信}{wei1 xin4}[][HSK 4]
    \definition*{s.}{WeChat; aplicativo gratuito lançado pela Tencent em 21 de janeiro de 2011 para fornecer serviços de mensagens instantâneas para terminais inteligentes}
  \end{phonetics}
\end{entry}

\begin{entry}{微型}{13,9}{⼻、⼟}
  \begin{phonetics}{微型}{wei1xing2}
    \definition{pref.}{micro-}
    \definition{s.}{miniatura}
  \end{phonetics}
\end{entry}

\begin{entry}{微笑}{13,10}{⼻、⽵}
  \begin{phonetics}{微笑}{wei1xiao4}[][HSK 4]
    \definition[个,丝]{s.}{sorriso;}
    \definition{v.}{sorrir}
  \end{phonetics}
\end{entry}

\begin{entry}{微博}{13,12}{⼻、⼗}
  \begin{phonetics}{微博}{wei1 bo2}[][HSK 5]
    \definition*{s.}{Weibo (um aplicativo de mídia social chinês)}
    \definition[条]{s.}{\emph{microblog}}
  \end{phonetics}
\end{entry}

\begin{entry}{想}{13}{⼼}
  \begin{phonetics}{想}{xiang3}[][HSK 1]
    \definition{v.}{pensar; ponderar; refletir | supor; contar; considerar; pensar; estimar | querer; gostaria de; sentir vontade (de fazer algo) | lembrar com saudade; sentir falta}
  \end{phonetics}
\end{entry}

\begin{entry}{想到}{13,8}{⼼、⼑}
  \begin{phonetics}{想到}{xiang3 dao4}[][HSK 2]
    \definition{v.}{pensar em | trazer à mente | ter no coração}
  \end{phonetics}
\end{entry}

\begin{entry}{想念}{13,8}{⼼、⼼}
  \begin{phonetics}{想念}{xiang3nian4}[][HSK 4]
    \definition{v.}{sentir falta; pensar em; lembrar com carinho; ficar doente por; desejar ver novamente; lembrar com saudade}
  \end{phonetics}
\end{entry}

\begin{entry}{想法}{13,8}{⼼、⽔}
  \begin{phonetics}{想法}{xiang3 fa3}[][HSK 2]
    \definition[个]{s.}{noção | opinião | jeito de pensar}
    \definition{s.}{maneira de pensar | opinião | noção}
    \definition{v.}{pensar em uma maneira (de fazer algo)}
  \end{phonetics}
\end{entry}

\begin{entry}{想起}{13,10}{⼼、⾛}
  \begin{phonetics}{想起}{xiang3 qi3}[][HSK 2]
    \definition{v.}{recordar | lembrar | pensar em | trazer à mente | cruzar pelos pensamentos de alguém | passar pelo pensamento de alguém}
  \end{phonetics}
\end{entry}

\begin{entry}{想象}{13,11}{⼼、⾗}
  \begin{phonetics}{想象}{xiang3xiang4}[][HSK 4]
    \definition[个]{s.}{imaginação; refere-se ao processo mental de processamento e transformação de representações armazenadas na mente para formar novas imagens}
    \definition{v.}{imaginar; vislumbrar; visualizar; refere-se a ter uma imagem concreta de algo que não está na frente dos olhos}
  \end{phonetics}
\end{entry}

\begin{entry}{想想看}{13,13,9}{⼼、⼼、⽬}
  \begin{phonetics}{想想看}{xiang3xiang3kan4}
    \definition{v.}{pensar sobre isso}
  \end{phonetics}
\end{entry}

\begin{entry}{愁}{13}{⼼}
  \begin{phonetics}{愁}{chou2}[][HSK 5]
    \definition{adj.}{triste; pesaroso; angustiado; desconsolado}
    \definition{s.}{pesar; sofrimento; dor; tristeza}
    \definition{v.}{preocupar-se; estar preocuoado; ficar ansioso; sentir ansiedade}
  \end{phonetics}
\end{entry}

\begin{entry}{愈}{13}{⼼}
  \begin{phonetics}{愈}{yu4}
    \definition{adv.}{mais e mais | ainda mais}
    \definition{v.}{recuperar | curar}
  \end{phonetics}
\end{entry}

\begin{entry}{意义}{13,3}{⼼、⼂}
  \begin{phonetics}{意义}{yi4yi4}[][HSK 3]
    \definition[个]{s.}{sentido; significado; significado expresso por palavras ou outros sinais; significado indicado por comportamento ou aquisição |valor; efeito; significância; impacto}
  \end{phonetics}
\end{entry}

\begin{entry}{意见}{13,4}{⼼、⾒}
  \begin{phonetics}{意见}{yi4jian4}[][HSK 2]
    \definition[点,条]{s.}{reclamação | ideia | objeção | opinião | sugestão}
  \end{phonetics}
\end{entry}

\begin{entry}{意外}{13,5}{⼼、⼣}
  \begin{phonetics}{意外}{yi4wai4}[][HSK 3]
    \definition{adj.}{inesperado; imprevisto}
    \definition{adv.}{acidentalmente}
    \definition[个]{s.}{acidente; infortúnio; infortúnio inesperado}
  \end{phonetics}
\end{entry}

\begin{entry}{意志}{13,7}{⼼、⼼}
  \begin{phonetics}{意志}{yi4zhi4}[][HSK 5]
    \definition[个,股]{s.}{vontade; determinação; desejo; força de vontade}
  \end{phonetics}
\end{entry}

\begin{entry}{意识}{13,7}{⼼、⾔}
  \begin{phonetics}{意识}{yi4shi2}[][HSK 5]
    \definition{s.}{consciência}
    \definition{s.}{consciência; percepção; grau de reconhecimento e importância atribuído a uma determinada questão}
    \definition{v.}{perceber; despertar para; estar ciente de; sentir, descobrir o que antes não se sentia ou não se descobria; geralmente é usado junto com 到}
  \seealsoref{到}{dao4}
  \end{phonetics}
\end{entry}

\begin{entry}{意译}{13,7}{⼼、⾔}
  \begin{phonetics}{意译}{yi4yi4}
    \definition{s.}{tradução livre | significado (de expressão estrangeira) | paráfrase | tradução do significado (em oposição à tradução literal)}
  \seealsoref{直译}{zhi2yi4}
  \end{phonetics}
\end{entry}

\begin{entry}{意味着}{13,8,11}{⼼、⼝、⽬}
  \begin{phonetics}{意味着}{yi4wei4zhe5}[][HSK 5]
    \definition{v.}{significar; subentender}
  \end{phonetics}
\end{entry}

\begin{entry}{意思}{13,9}{⼼、⼼}
  \begin{phonetics}{意思}{yi4si5}[][HSK 2]
    \definition[个]{s.}{interesse}
  \end{phonetics}
\end{entry}

\begin{entry}{意指}{13,9}{⼼、⼿}
  \begin{phonetics}{意指}{yi4zhi3}
    \definition{v.}{implicar | significar}
  \end{phonetics}
\end{entry}

\begin{entry}{感兴趣}{13,6,15}{⼼、⼋、⾛}
  \begin{phonetics}{感兴趣}{gan3xing4qu4}[][HSK 4]
    \definition{v.}{estar interessado}
  \seealsoref{对……感兴趣}{dui4 gan3xing4qu4}
  \end{phonetics}
\end{entry}

\begin{entry}{感动}{13,6}{⼼、⼒}
  \begin{phonetics}{感动}{gan3dong4}[][HSK 2]
    \definition{v.}{mover (alguém) | tocar (alguém emocionalmente)}
  \end{phonetics}
\end{entry}

\begin{entry}{感到}{13,8}{⼼、⼑}
  \begin{phonetics}{感到}{gan3 dao4}[][HSK 2]
    \definition{v.}{sentir; achar; perceber}
  \end{phonetics}
\end{entry}

\begin{entry}{感受}{13,8}{⼼、⼜}
  \begin{phonetics}{感受}{gan3shou4}[][HSK 3]
    \definition{s.}{percepção ; sentimento; experiência}
    \definition{v.}{sentir; sentir (através dos sentidos); experimentar}
  \end{phonetics}
\end{entry}

\begin{entry}{感冒}{13,9}{⼼、⽇}
  \begin{phonetics}{感冒}{gan3mao4}[][HSK 3]
    \definition{adj.}{interessado}
    \definition[场,次]{s.}{resfriado; resfriado comum; gripe}
    \definition{v.}{pegar (ter) um resfriado}
  \end{phonetics}
\end{entry}

\begin{entry}{感染}{13,9}{⼼、⽊}
  \begin{phonetics}{感染}{gan3ran3}
    \definition{s.}{infecção}
    \definition{v.}{infectar | (figurativo) influenciar}
  \end{phonetics}
\end{entry}

\begin{entry}{感觉}{13,9}{⼼、⾒}
  \begin{phonetics}{感觉}{gan3jue2}[][HSK 2]
    \definition[个]{s.}{sentimento; sensação; percepção sensorial;}
    \definition{v.}{sentir; perceber; tomar consciência; sentir no coração, acreditar}
  \end{phonetics}
\end{entry}

\begin{entry}{感情}{13,11}{⼼、⼼}
  \begin{phonetics}{感情}{gan3qing2}[][HSK 3]
    \definition[份,个,种]{s.}{emoção; sentimento | amor; afeição; apego}
  \end{phonetics}
\end{entry}

\begin{entry}{感谢}{13,12}{⼼、⾔}
  \begin{phonetics}{感谢}{gan3xie4}[][HSK 2]
    \definition{v.}{agradecer; ser grato; expressar gratidão com palavras ou ações}
  \end{phonetics}
\end{entry}

\begin{entry}{感想}{13,13}{⼼、⼼}
  \begin{phonetics}{感想}{gan3xiang3}[][HSK 5]
    \definition[个,条]{s.}{pensamentos; impressões; reflexões; resposta do pensamento decorrente da exposição ao mundo exterior}
  \end{phonetics}
\end{entry}

\begin{entry}{搞}{13}{⼿}
  \begin{phonetics}{搞}{gao3}[][HSK 5]
    \definition{v.}{fazer; realizar; estar envolvido em; engajar-se em um estudo, fazer algo em relação a, etc. | fazer; produzir; gerar; trabalhar | iniciar; estabelecer; organizar; configurar | consertar (mudar) alguém; fazer alguém sofrer | obter; assegurar; agarrar |  (seguido de um complemento) fazer com que se torne; produzir um determinado efeito ou resultado}
  \end{phonetics}
\end{entry}

\begin{entry}{搞好}{13,6}{⼿、⼥}
  \begin{phonetics}{搞好}{gao3 hao3}[][HSK 5]
    \definition{v.}{fazer um bom trabalho; fazer bem; suar; tornar submisso, tornar útil, por meio de solicitações e presentes amigáveis; amolecer}
  \end{phonetics}
\end{entry}

\begin{entry}{搞乱}{13,7}{⼿、⼄}
  \begin{phonetics}{搞乱}{gao3luan4}
    \definition{v.}{estragar | confundir | bagunçar}
  \end{phonetics}
\end{entry}

\begin{entry}{搞定}{13,8}{⼿、⼧}
  \begin{phonetics}{搞定}{gao3ding4}
    \definition{v.}{consertar | resolver}
  \end{phonetics}
\end{entry}

\begin{entry}{搞鬼}{13,9}{⼿、⿁}
  \begin{phonetics}{搞鬼}{gao3gui3}
    \definition{v.}{fazer travessuras | fazer truques}
  \end{phonetics}
\end{entry}

\begin{entry}{搞笑}{13,10}{⼿、⽵}
  \begin{phonetics}{搞笑}{gao3xiao4}
    \definition{adj.}{engraçado | hilário}
    \definition{v.}{fazer as pessoas rirem}
  \end{phonetics}
\end{entry}

\begin{entry}{搞通}{13,10}{⼿、⾡}
  \begin{phonetics}{搞通}{gao3tong1}
    \definition{v.}{entender algo}
  \end{phonetics}
\end{entry}

\begin{entry}{搞钱}{13,10}{⼿、⾦}
  \begin{phonetics}{搞钱}{gao3qian2}
    \definition{v.}{fazer dinheiro | acumular dinheiro}
  \end{phonetics}
\end{entry}

\begin{entry}{搞混}{13,11}{⼿、⽔}
  \begin{phonetics}{搞混}{gao3hun4}
    \definition{v.}{confundir}
  \end{phonetics}
\end{entry}

\begin{entry}{搞错}{13,13}{⼿、⾦}
  \begin{phonetics}{搞错}{gao3cuo4}
    \definition{v.}{cometer um erro}
  \end{phonetics}
\end{entry}

\begin{entry}{搬}{13}{⼿}
  \begin{phonetics}{搬}{ban1}[][HSK 3]
    \definition{v.}{copiar indiscriminadamente | mover-se (ou seja, mudar-se) | mover-se (algo relativamente pesado ou volumoso) | mudar | mudar-se}
  \end{phonetics}
\end{entry}

\begin{entry}{搬口}{13,3}{⼿、⼝}
  \begin{phonetics}{搬口}{ban1kou3}
    \definition{v.}{tagarelar | (idioma) transmitir histórias | semear dissensão | contar histórias}
  \end{phonetics}
\end{entry}

\begin{entry}{搬动}{13,6}{⼿、⼒}
  \begin{phonetics}{搬动}{ban1dong4}
    \definition{v.}{mover-se (alguma coisa) | se mudar}
  \end{phonetics}
\end{entry}

\begin{entry}{搬弄}{13,7}{⼿、⼶}
  \begin{phonetics}{搬弄}{ban1nong4}
    \definition{v.}{causar problemas | mexer com alguém | mostrar (o que se pode fazer)}
  \end{phonetics}
\end{entry}

\begin{entry}{搬走}{13,7}{⼿、⾛}
  \begin{phonetics}{搬走}{ban1zou3}
    \definition{v.}{carregar}
  \end{phonetics}
\end{entry}

\begin{entry}{搬运}{13,7}{⼿、⾡}
  \begin{phonetics}{搬运}{ban1yun4}
    \definition{s.}{frete | transporte}
    \definition{v.}{carregar | transportar}
  \end{phonetics}
\end{entry}

\begin{entry}{搬家}{13,10}{⼿、⼧}
  \begin{phonetics}{搬家}{ban1jia1}[][HSK 3]
    \definition{s.}{mudança}
    \definition{v.+compl.}{mudar-se de casa}
  \end{phonetics}
\end{entry}

\begin{entry}{摄氏}{13,4}{⼿、⽒}
  \begin{phonetics}{摄氏}{she4shi4}
    \definition{s.}{graus Celsius (°C), centígrado}
  \end{phonetics}
\end{entry}

\begin{entry}{摄像}{13,13}{⼿、⼈}
  \begin{phonetics}{摄像}{she4 xiang4}[][HSK 5]
    \definition{v.}{gravar; filmar; filmar com câmera; fazer uma gravação de vídeo (com uma câmera de vídeo ou TV)}
  \end{phonetics}
\end{entry}

\begin{entry}{摄像机}{13,13,6}{⼿、⼈、⽊}
  \begin{phonetics}{摄像机}{she4 xiang4 ji1}[][HSK 5]
    \definition[个,部]{s.}{câmera de vídeo; dispositivo que pode ser usado para converter imagens captadas em sinais de imagem de televisão}
  \end{phonetics}
\end{entry}

\begin{entry}{摄影}{13,15}{⼿、⼺}
  \begin{phonetics}{摄影}{she4ying3}[][HSK 5]
    \definition{s.}{fotografia}
    \definition{v.}{fotografar; tirar uma foto; tirar fotos ou filmar}
  \end{phonetics}
\end{entry}

\begin{entry}{摄影师}{13,15,6}{⼿、⼺、⼱}
  \begin{phonetics}{摄影师}{she4 ying3 shi1}[][HSK 5]
    \definition[个]{s.}{fotógrafo; cinegrafista; operador de câmera; técnico de fotografia em estúdio fotográfico}
  \end{phonetics}
\end{entry}

\begin{entry}{摆}{13}{⼿}
  \begin{phonetics}{摆}{bai3}[][HSK 4]
    \definition*{s.}{sobrenome Bai}
    \definition*{s.}{Festival de Ganbai; uma reunião realizada nas áreas Dai durante festivais religiosos, para celebrar uma boa colheita ou para trocar materiais; geralmente se refere a uma reunião em massa}
    \definition{s.}{pêndulo; dispositivo mecânico que controla a frequência de oscilação em relógios e instrumentos |  a bainha inferior de um vestido, jaqueta ou saia}
    \definition{v.}{colocar; posicionar; organizar | assumir; mostrar intencionalmente | balançar; ondular; balançar para frente e para trás | revelar; listar; afirmar claramente | dizer; falar; declarar | libertar-se; livrar-se}
  \end{phonetics}
\end{entry}

\begin{entry}{摆手}{13,4}{⼿、⼿}
  \begin{phonetics}{摆手}{bai3shou3}
    \definition{v.+compl.}{gesticular com a mão (acenando, acenando adeus, etc.) | balançar os braços | acenar com as mãos}
  \end{phonetics}
\end{entry}

\begin{entry}{摆动}{13,6}{⼿、⼒}
  \begin{phonetics}{摆动}{bai3 dong4}[][HSK 4]
    \definition{v.}{balançar; balançar para frente e para trás; oscilar; vibrar}
  \end{phonetics}
\end{entry}

\begin{entry}{摆烂}{13,9}{⼿、⽕}
  \begin{phonetics}{摆烂}{bai3lan4}
    \definition{v.}{(neologismo, gíria) parar de lutar (especialmente quando se sabe que não pode ter sucesso) | deixar tudo ir para o inferno}
  \end{phonetics}
\end{entry}

\begin{entry}{摆脱}{13,11}{⼿、⾁}
  \begin{phonetics}{摆脱}{bai3tuo1}[][HSK 4]
    \definition{v.}{sacudir; rejeitar; romper com; libertar-se (ou desembaraçar-se) de; livrar-se de dificuldades, escravidão, controle, etc.}
  \end{phonetics}
\end{entry}

\begin{entry}{摇}{13}{⼿}
  \begin{phonetics}{摇}{yao2}[][HSK 4]
    \definition{v.}{chacoalhar; ondular; balançar; fazer com que um objeto se mova para frente e para trás | agitar algo | sacudir; chacoalhar; agitar algo para que se mova}
  \end{phonetics}
\end{entry}

\begin{entry}{摇头}{13,5}{⼿、⼤}
  \begin{phonetics}{摇头}{yao2tou2}[][HSK 5]
    \definition{v.+compl.}{sacudir; balançar a cabeça; balançar a cabeça para a esquerda e para a direita, indicando negação, desacordo ou impedimento}
  \end{phonetics}
\end{entry}

\begin{entry}{摇晃}{13,10}{⼿、⽇}
  \begin{phonetics}{摇晃}{yao2huang4}
    \definition{v.}{sacudir | agitar | balançar | chacoalhar}
  \end{phonetics}
\end{entry}

\begin{entry}{摸}{13}{⼿}
  \begin{phonetics}{摸}{mo1}[][HSK 4]
    \definition{v.}{sentir; acariciar; tocar; tocar (um objeto) levemente com a mão e depois removê-lo ou mover a mão suavemente sobre a superfície do objeto | sentir para; tatear para; sentir algo com as mãos | descobrir; sentir; sondar; explorar; tentar fazer ou entender | sentir o caminho; tatear no escuro; andar por estradas que você não consegue reconhecer | furtar; roubar}
  \end{phonetics}
\end{entry}

\begin{entry}{数}{13}{⽁}
  \begin{phonetics}{数}{shu3}[][HSK 2]
    \definition{v.}{contar
ser considerado excepcionalmente (bom, ruim, etc.)
enumerar; listar}
  \end{phonetics}
  \begin{phonetics}{数}{shu4}
    \definition{num.}{vários | alguns}
    \definition{s.}{número | figura | destino}
  \end{phonetics}
  \begin{phonetics}{数}{shuo4}
    \definition{adv.}{frequentemente | repetidamente}
  \end{phonetics}
\end{entry}

\begin{entry}{数目}{13,5}{⽁、⽬}
  \begin{phonetics}{数目}{shu4 mu4}[][HSK 5]
    \definition{s.}{número; quantidade; quantidade de algo expressa em uma determinada medida padrão (como unidades de medida, etc.)}
  \end{phonetics}
\end{entry}

\begin{entry}{数字}{13,6}{⽁、⼦}
  \begin{phonetics}{数字}{shu4zi4}[][HSK 2]
    \definition{adj.}{digital}
    \definition[个]{s.}{dígito | figura | número | numeral | quantidade | montante}
  \end{phonetics}
\end{entry}

\begin{entry}{数学}{13,8}{⽁、⼦}
  \begin{phonetics}{数学}{shu4xue2}
    \definition{s.}{matemática (disciplina)}
  \end{phonetics}
\end{entry}

\begin{entry}{数码}{13,8}{⽁、⽯}
  \begin{phonetics}{数码}{shu4ma3}[][HSK 4]
    \definition{s.}{dígito; numeral; algarismo | número; quantidade (usado principalmente na linguagem falada)}
    \definition{v.}{digitalizar}
  \end{phonetics}
\end{entry}

\begin{entry}{数据}{13,11}{⽁、⼿}
  \begin{phonetics}{数据}{shu4ju4}[][HSK 4]
    \definition[些,个]{s.}{dados; valores com base nos quais são realizadas estatísticas, cálculos, pesquisas científicas ou projetos técnicos}
  \end{phonetics}
\end{entry}

\begin{entry}{数量}{13,12}{⽁、⾥}
  \begin{phonetics}{数量}{shu4liang4}[][HSK 3]
    \definition[个]{s.}{quantidade; quantum; quantia; magnitude; número}
  \end{phonetics}
\end{entry}

\begin{entry}{新}{13}{⽄}
  \begin{phonetics}{新}{xin1}[][HSK 1]
    \definition*{s.}{sobrenome Xin}
    \definition*{s.}{abreviação de Xinjiang (新疆)}
    \definition*{s.}{abreviação de Singapura (新加坡)}
    \definition{adj.}{novo; fresco; inovador; atualizado; aparecer ou ser experimentado pela primeira vez | nunca utilizado; novo; não foi usado ou foi usado por pouco tempo | recém-casado}
    \definition{adv.}{recém; recentemente; há pouco tempo}
    \definition{pref.}{(química) meso-}
    \definition{v.}{atualizar; renovar}
  \seealsoref{新加坡}{xin1jia1po1}
  \seealsoref{新疆}{xin1jiang1}
  \end{phonetics}
\end{entry}

\begin{entry}{新加坡}{13,5,8}{⽄、⼒、⼟}
  \begin{phonetics}{新加坡}{xin1jia1po1}
    \definition*{s.}{Singapura}
  \end{phonetics}
\end{entry}

\begin{entry}{新年}{13,6}{⽄、⼲}
  \begin{phonetics}{新年}{xin1 nian2}[][HSK 1]
    \definition*[个]{s.}{Ano Novo}
  \end{phonetics}
\end{entry}

\begin{entry}{新郎}{13,8}{⽄、⾢}
  \begin{phonetics}{新郎}{xin1lang2}[][HSK 4]
    \definition[位,个]{s.}{noivo; homens no momento do casamento}
  \end{phonetics}
\end{entry}

\begin{entry}{新型}{13,9}{⽄、⼟}
  \begin{phonetics}{新型}{xin1 xing2}[][HSK 4]
    \definition[种]{s.}{ultimo modelo; novo tipo; novo padrão; novo estilo}
  \end{phonetics}
\end{entry}

\begin{entry}{新闻}{13,9}{⽄、⾨}
  \begin{phonetics}{新闻}{xin1wen2}[][HSK 2]
    \definition[条,个]{s.}{notícia}
  \end{phonetics}
\end{entry}

\begin{entry}{新娘}{13,10}{⽄、⼥}
  \begin{phonetics}{新娘}{xin1niang2}[][HSK 4]
    \definition[位,个]{s.}{noiva; a mulher no momento do casamento}
  \seealsoref{新娘子}{xin1niang2zi5}
  \end{phonetics}
\end{entry}

\begin{entry}{新娘子}{13,10,3}{⽄、⼥、⼦}
  \begin{phonetics}{新娘子}{xin1niang2zi5}
    \definition{s.}{noiva}
  \seealsoref{新娘}{xin1niang2}
  \end{phonetics}
\end{entry}

\begin{entry}{新娘服装}{13,10,8,12}{⽄、⼥、⽉、⾐}
  \begin{phonetics}{新娘服装}{xin1niang2 fu2zhuang1}
    \definition{s.}{roupas de noiva}
  \end{phonetics}
\end{entry}

\begin{entry}{新鲜}{13,14}{⽄、⿂}
  \begin{phonetics}{新鲜}{xin1xian1}
    \definition{adj.}{fresco (experiência, alimento, etc.)}
    \definition{s.}{frescor}
  \end{phonetics}
\end{entry}

\begin{entry}{新疆}{13,19}{⽄、⼸}
  \begin{phonetics}{新疆}{xin1jiang1}
    \definition*{s.}{Xinjiang}
  \end{phonetics}
\end{entry}

\begin{entry}{新疆维吾尔自治区}{13,19,11,7,5,6,8,4}{⽄、⼸、⽷、⼝、⼩、⾃、⽔、⼖}
  \begin{phonetics}{新疆维吾尔自治区}{xin1jiang1 wei2wu2'er3 zi4zhi4qu1}
    \definition*{s.}{Região Autônoma Uigur de Xinjiang}
  \end{phonetics}
\end{entry}

\begin{entry}{暖}{13}{⽇}
  \begin{phonetics}{暖}{nuan3}[][HSK 5]
    \definition{adj.}{caloroso; cordial}
    \definition{v.}{aquecer; esquentar; aquecer algo ou aquecer o corpo}
  \end{phonetics}
\end{entry}

\begin{entry}{暖气}{13,4}{⽇、⽓}
  \begin{phonetics}{暖气}{nuan3qi4}[][HSK 4]
    \definition[个]{s.}{aquecedor; aquecimento; aquecimento central}
  \end{phonetics}
\end{entry}

\begin{entry}{暖和}{13,8}{⽇、⼝}
  \begin{phonetics}{暖和}{nuan3huo5}[][HSK 3]
    \definition{adj.}{morno; agradável e quente}
    \definition{v.}{aquecer}
  \end{phonetics}
\end{entry}

\begin{entry}{暗}{13}{⽇}
  \begin{phonetics}{暗}{an4}[][HSK 4]
    \definition{adj.}{escuro; opaco; sem graça; pouca luz | escondido; secreto; não revelado | pouco claro; nebuloso; vago; confuso | subterrâneo}
    \definition{adv.}{secretamente | no escuro}
  \end{phonetics}
\end{entry}

\begin{entry}{暗示}{13,5}{⽇、⽰}
  \begin{phonetics}{暗示}{an4shi4}[][HSK 4]
    \definition[个]{s.}{sugestão; insinuação; intimação; (psicologia) refere-se ao uso de palavras, gestos, expressões, etc. para fazer as pessoas aceitarem involuntariamente uma determinada opinião ou fazerem algo}
    \definition{v.}{dar uma dica; sugerir secretamente; indicar algo a alguém usando outras palavras, expressões faciais ou gestos sem dizer em voz alta}
  \end{phonetics}
\end{entry}

\begin{entry}{暗香}{13,9}{⽇、⾹}
  \begin{phonetics}{暗香}{an4xiang1}
    \definition{s.}{fragrância sutil}
  \end{phonetics}
\end{entry}

\begin{entry}{暗恋}{13,10}{⽇、⼼}
  \begin{phonetics}{暗恋}{an4lian4}
    \definition{s.}{amor secreto}
    \definition{v.}{estar secretamente apaixonado por}
  \end{phonetics}
\end{entry}

\begin{entry}{楼}{13}{⽊}
  \begin{phonetics}{楼}{lou2}[][HSK 1]
    \definition*{s.}{sobrenome Lou}
    \definition{clas.}{andar, piso}
    \definition[层,座,栋]{s.}{um prédio com muitos andares | piso; andar | superestrutura; uma estrutura com um convés superior; um andar adicional construído sobre uma casa ou outro edifício | nome usado para certas lojas ou locais de entretenimento | arco ornamental; certas construções decorativas altas com passagens por baixo}
  \end{phonetics}
\end{entry}

\begin{entry}{楼上}{13,3}{⽊、⼀}
  \begin{phonetics}{楼上}{lou2 shang4}[][HSK 1]
    \definition{s.}{no andar de cima | autor anterior em um tópico do fórum; em plataformas como fóruns na internet, refere-se à pessoa que se manifesta antes de você.}
  \end{phonetics}
\end{entry}

\begin{entry}{楼下}{13,3}{⽊、⼀}
  \begin{phonetics}{楼下}{lou2 xia4}[][HSK 1]
    \definition{s.}{no andar de baixo}
  \end{phonetics}
\end{entry}

\begin{entry}{楼梯}{13,11}{⽊、⽊}
  \begin{phonetics}{楼梯}{lou2 ti1}[][HSK 4]
    \definition[个]{s.}{escada; escadaria; degraus no meio de dois andares para permitir que as pessoas subam ou desçam as escadas}
  \end{phonetics}
\end{entry}

\begin{entry}{概念}{13,8}{⽊、⼼}
  \begin{phonetics}{概念}{gai4nian4}[][HSK 3]
    \definition[个]{s.}{ideia; noção; conceito; concepção}
  \end{phonetics}
\end{entry}

\begin{entry}{概括}{13,9}{⽊、⼿}
  \begin{phonetics}{概括}{gai4kuo4}[][HSK 4]
    \definition{adj.}{genérico; simples e claro, captando o conteúdo principal}
    \definition{s.}{generalização}
    \definition{v.}{generalizar; resumir}
  \end{phonetics}
\end{entry}

\begin{entry}{歇}{13}{⽋}
  \begin{phonetics}{歇}{xie1}[][HSK 5]
    \definition*{s.}{sobrenome Xie}
    \definition{s.}{um pouco de tempo}
    \definition{v.}{descansar; fazer uma pausa | parar (o trabalho); encerrar o expediente | dormir; ir para a cama}
  \end{phonetics}
\end{entry}

\begin{entry}{源}{13}{⽔}
  \begin{phonetics}{源}{yuan2}
    \definition*{s.}{sobrenome Yuan}
    \definition{s.}{nascente (de um rio); fonte | fonte; origem; causa}
    \definition{v.}{originar-se; provir de}
  \end{phonetics}
\end{entry}

\begin{entry}{滔天}{13,4}{⽔、⼤}
  \begin{phonetics}{滔天}{tao1tian1}
    \definition{adj.}{(ondas, raiva, desastres, crimes, etc.) imponente, avassalador, imenso}
  \end{phonetics}
\end{entry}

\begin{entry}{滚}{13}{⽔}
  \begin{phonetics}{滚}{gun3}[][HSK 5]
    \definition*{s.}{sobrenome Gun}
    \definition{adj.}{rolante | fervente | precipitado; torrencial}
    \definition{adv.}{muito; em um grau elevado}
    \definition{v.}{rolar; girar; virar | escapar; fugir; ir embora | ferver | amarrar; aparar; fazer bainha}
  \end{phonetics}
\end{entry}

\begin{entry}{滚轮}{13,8}{⽔、⾞}
  \begin{phonetics}{滚轮}{gun3lun2}
    \definition{s.}{pneu | dial rotativo | roda de rolagem (\emph{scroll})  (mouse de computador)}
  \end{phonetics}
\end{entry}

\begin{entry}{滚滚}{13,13}{⽔、⽔}
  \begin{phonetics}{滚滚}{gun3gun3}
    \definition*{s.}{Apelido para um panda}
    \definition{v.}{mover-se | rolar | fluir continuamente}
  \end{phonetics}
\end{entry}

\begin{entry}{满}{13}{⽔}
  \begin{phonetics}{满}{man3}[][HSK 2]
    \definition*{s.}{sobrenome Man}
    \definition*{s.}{etnia Manchu}
    \definition{adj.}{cheio; repleto; lotado; totalmente cheio; atingindo o limite da capacidade | tudo; inteiro; completo | presunçoso; complacente; orgulhoso}
    \definition{adv.}{muito; um tanto; bastante | completamente; inteiramente; perfeitamente}
    \definition{v.}{encher | sentir-se satisfeito; sentir que já é o suficiente | expirar; atingir o limite; atingir um determinado prazo ou limite}
  \end{phonetics}
\end{entry}

\begin{entry}{满分}{13,4}{⽔、⼑}
  \begin{phonetics}{满分}{man3fen1}
    \definition{s.}{pontuação completa}
  \end{phonetics}
\end{entry}

\begin{entry}{满足}{13,7}{⽔、⾜}
  \begin{phonetics}{满足}{man3zu2}[][HSK 3]
    \definition{v.}{estar satisfeito; contentar-se | satisfazer; causar satisfação; contentar}
  \end{phonetics}
\end{entry}

\begin{entry}{满意}{13,13}{⽔、⼼}
  \begin{phonetics}{满意}{man3yi4}[][HSK 2]
    \definition{adj.}{satisfeito; contente; gratificado}
    \definition{v.}{estar satisfeito; sentir-se contente; satisfazer os seus desejos; estar de acordo com os seus desejos}
  \end{phonetics}
\end{entry}

\begin{entry}{满满}{13,13}{⽔、⽔}
  \begin{phonetics}{满满}{man3man3}
    \definition{adj.}{completo | densamente empacotado}
  \end{phonetics}
\end{entry}

\begin{entry}{煎}{13}{⽕}
  \begin{phonetics}{煎}{jian1}
    \definition{v.}{fritar | refogar}
  \end{phonetics}
\end{entry}

\begin{entry}{煎饼}{13,9}{⽕、⾷}
  \begin{phonetics}{煎饼}{jian1bing3}
    \definition[张]{s.}{jianbing, crepe chinês | panqueca}
  \end{phonetics}
\end{entry}

\begin{entry}{煎蛋}{13,11}{⽕、⾍}
  \begin{phonetics}{煎蛋}{jian1dan4}
    \definition{s.}{ovos fritos}
  \end{phonetics}
\end{entry}

\begin{entry}{煤}{13}{⽕}
  \begin{phonetics}{煤}{mei2}[][HSK 5]
    \definition[吨,堆,块]{s.}{carvão; carvão vegetal; minério sólido preto}
  \end{phonetics}
\end{entry}

\begin{entry}{煤气}{13,4}{⽕、⽓}
  \begin{phonetics}{煤气}{mei2 qi4}[][HSK 5]
    \definition[把]{s.}{gás; gás de carvão; gás obtido a partir do processamento do carvão não tem cor nem odor, é tóxico e pode ser queimado ou utilizado como matéria-prima na indústria química | envenenamento por monóxido de carbono}
  \end{phonetics}
\end{entry}

\begin{entry}{照}{13}{⽕}
  \begin{phonetics}{照}{zhao4}[][HSK 3]
    \definition{adv.}{de acordo com; indica agir de acordo com o original ou um certo padrão}
    \definition{prep.}{em direção a; na direção de | de acordo com; conforme}
    \definition{s.}{imagem; fotografia | permissão; licença | brilho; iluminação}
    \definition{v.}{brilhar; acender; iluminar | refletir; espelhar; olhar para sua própria imagem em um espelho, etc. | filmar; fotografar; tirar uma foto (fotografia) | cuidar de; tomar conta de | notificar | contrastar | entender}
  \end{phonetics}
\end{entry}

\begin{entry}{照片}{13,4}{⽕、⽚}
  \begin{phonetics}{照片}{zhao4pian4}[][HSK 2]
    \definition[张,套,幅]{s.}{fotografia | foto}
  \end{phonetics}
\end{entry}

\begin{entry}{照片子}{13,4,3}{⽕、⽚、⼦}
  \begin{phonetics}{照片子}{zhao4pian4zi5}
    \definition{v.}{tirar um raio X}
  \end{phonetics}
\end{entry}

\begin{entry}{照片底版}{13,4,8,8}{⽕、⽚、⼴、⽚}
  \begin{phonetics}{照片底版}{zhao4pian4di3ban3}
    \definition{s.}{placa fotográfica}
  \end{phonetics}
\end{entry}

\begin{entry}{照亮}{13,9}{⽕、⼇}
  \begin{phonetics}{照亮}{zhao4liang4}
    \definition{s.}{iluminação}
    \definition{v.}{iluminar}
  \end{phonetics}
\end{entry}

\begin{entry}{照相}{13,9}{⽕、⽬}
  \begin{phonetics}{照相}{zhao4 xiang4}[][HSK 2]
    \definition{v.+compl.}{tirar fotografia}
  \end{phonetics}
\end{entry}

\begin{entry}{照相机}{13,9,6}{⽕、⽬、⽊}
  \begin{phonetics}{照相机}{zhao4xiang4ji1}
    \definition[个,架,部,台,只]{s.}{câmera/máquina fotográfica}
  \end{phonetics}
\end{entry}

\begin{entry}{照准}{13,10}{⽕、⼎}
  \begin{phonetics}{照准}{zhao4zhun3}
    \definition{s.}{solicitação concedida (uso formal em documento antigo)}
    \definition{v.}{mirar (arma)}
  \end{phonetics}
\end{entry}

\begin{entry}{照顾}{13,10}{⽕、⾴}
  \begin{phonetics}{照顾}{zhao4gu4}[][HSK 2]
    \definition{v.}{cuidar de | atender a | oferecer tratamento preferencial | (de um cliente) patrocinar | fazer compras em | dar consideração a | mostrar consideração por | levar em conta | fazer concessões para}
  \end{phonetics}
\end{entry}

\begin{entry}{照骗}{13,12}{⽕、⾺}
  \begin{phonetics}{照骗}{zhao4pian4}
    \definition{s.}{imagem alterada digitalmente; ``photoshopada''}
  \end{phonetics}
\end{entry}

\begin{entry}{照像}{13,13}{⽕、⼈}
  \begin{phonetics}{照像}{zhao4 xiang4}
    \variantof{照相}
  \end{phonetics}
\end{entry}

\begin{entry}{照像机}{13,13,6}{⽕、⼈、⽊}
  \begin{phonetics}{照像机}{zhao4xiang4ji1}
    \variantof{照相机}
  \end{phonetics}
\end{entry}

\begin{entry}{献}{13}{⽝}
  \begin{phonetics}{献}{xian4}[][HSK 5]
    \definition{v.}{oferecer; apresentar; dedicar; doar | mostrar; apresentar; exibir | exibir-se; mostrar-se para que os outros vejam}
  \end{phonetics}
\end{entry}

\begin{entry}{瑜伽}{13,7}{⽟、⼈}
  \begin{phonetics}{瑜伽}{yu2jia1}
    \definition*{s.}{Ioga}
  \end{phonetics}
\end{entry}

\begin{entry}{瑜珈}{13,9}{⽟、⽟}
  \begin{phonetics}{瑜珈}{yu2jia1}
    \variantof{瑜伽}
  \end{phonetics}
\end{entry}

\begin{entry}{睡}{13}{⽬}
  \begin{phonetics}{睡}{shui4}[][HSK 1]
    \definition{v.}{dormir | deitar-se}
  \end{phonetics}
\end{entry}

\begin{entry}{睡衣}{13,6}{⽬、⾐}
  \begin{phonetics}{睡衣}{shui4yi1}
    \definition{s.}{pijamas | roupas de dormir}
  \end{phonetics}
\end{entry}

\begin{entry}{睡觉}{13,9}{⽬、⾒}
  \begin{phonetics}{睡觉}{shui4jiao4}[][HSK 1]
    \definition{v.+compl.}{dormir; ir para a cama; entrar em estado de sono}
  \end{phonetics}
\end{entry}

\begin{entry}{睡眠}{13,10}{⽬、⽬}
  \begin{phonetics}{睡眠}{shui4 mian2}[][HSK 5]
    \definition{s.}{sono; \emph{somnus}; sonolência}
  \end{phonetics}
\end{entry}

\begin{entry}{睡着}{13,11}{⽬、⽬}
  \begin{phonetics}{睡着}{shui4 zhao2}[][HSK 4]
    \definition{v.}{dormir; adormecer; cair no sono}
  \end{phonetics}
\end{entry}

\begin{entry}{睡懒觉}{13,16,9}{⽬、⼼、⾒}
  \begin{phonetics}{睡懒觉}{shui4lan3jiao4}
    \definition{v.}{levantar-se tarde | passar o tempo a dormir}
  \end{phonetics}
\end{entry}

\begin{entry}{矮}{13}{⽮}
  \begin{phonetics}{矮}{ai3}[][HSK 4]
    \definition{adj.}{baixo em estatura, dimensão, grau ou ranque | curto (em comprimento)}[他比我矮。(Ele é mais baixo que eu.) | 这栋楼很矮,只有三层。(Esse prédio é baixo, tem só três andares.) | 她虽然矮,但是跑得很快!(Ela pode ser baixinha, mas corre muito rápido!)]
  \end{phonetics}
\end{entry}

\begin{entry}{矮人}{13,2}{⽮、⼈}
  \begin{phonetics}{矮人}{ai3ren2}
    \definition{s.}{anão; pessoa de baixa estatura (indivíduo) | homúnculo; figuras criadas artificialmente pelos alquimistas em frascos de destilação | nanismo}[他虽然是矮人,但很有力气。(Embora ele seja baixo, é muito forte.) | 北欧神话中的矮人是技艺高超的工匠。(Na mitologia nórdica, os anões são artesãos habilidosos.) | 他因为身高被嘲笑为‘矮人’,这让他很伤心。(Ele foi zombado por ser chamado de ‘anão’ devido à sua altura, o que o magoou.)]
  \end{phonetics}
\end{entry}

\begin{entry}{矮子}{13,3}{⽮、⼦}
  \begin{phonetics}{矮子}{ai3zi5}
    \definition{s.}{pessoa baixa; anão; baixinho}[白雪公主和七个小矮子住在森林里。(Branca de Neve e os sete anões vivem na floresta.) | 用`矮子'称呼他人是不礼貌的。(Chamar alguém de ``baixinho'' é falta de educação.)]
  \end{phonetics}
\end{entry}

\begin{entry}{矮小}{13,3}{⽮、⼩}
  \begin{phonetics}{矮小}{ai3 xiao3}[][HSK 4]
    \definition{adj.}{subdimensionado; curto e pequeno; baixo e pequeno | quando usado para pessoas, pode soar depreciativo se não for em contexto neutro ou afetuoso}[这位矮小的老人是村里的智者。(Este idoso baixinho é o sábio da vila.) | 这种矮小的灌木适合盆栽。(Este tipo de arbusto pequeno é ideal para vasos.) | 山脚下有一片矮小的房屋,显得格外宁静。(Ao pé da montanha, havia casas baixas que transmitiam uma tranquilidade única.)]
  \end{phonetics}
\end{entry}

\begin{entry}{矮林}{13,8}{⽮、⽊}
  \begin{phonetics}{矮林}{ai3lin2}
    \definition{s.}{mata rasteira | bosque baixo}[这片矮林里有很多野兔和鸟类。(Neste bosque baixo há muitos coelhos selvagens e pássaros.) | 山坡上长满了矮林,远看像绿色的地毯。(A encosta está coberta de mata rasteira, que de longe parece um tapete verde.)]
  \end{phonetics}
\end{entry}

\begin{entry}{矮星}{13,9}{⽮、⽇}
  \begin{phonetics}{矮星}{ai3xing1}
    \definition{s.}{estrela anã}[白矮星是恒星演化的最终阶段之一。(Anãs brancas são um dos estágios finais da evolução estelar.)]
  \end{phonetics}
\end{entry}

\begin{entry}{矮树}{13,9}{⽮、⽊}
  \begin{phonetics}{矮树}{ai3shu4}
    \definition{s.}{arbusto | árvore pequena, baixa}[矮树比高树更容易修剪。(Árvores baixas são mais fáceis de podar do que árvores altas.) | 我们种了些矮树作为花园的边界。(Plantamos alguns arbustos como cerca natural do jardim.)]
  \end{phonetics}
\end{entry}

\begin{entry}{矮胖}{13,9}{⽮、⾁}
  \begin{phonetics}{矮胖}{ai3pang4}
    \definition{adj.}{atarracado; gorducho; rechonchudo; roliço; baixo e robusto | chamar alguém diretamente de 矮胖 pode ser ofensivo}[我家猫矮胖矮胖的,像个毛球。(Meu gato é baixinho e gordinho, parece uma bolinha de pelo.)]
  \end{phonetics}
\end{entry}

\begin{entry}{矮凳}{13,14}{⽮、⼏}
  \begin{phonetics}{矮凳}{ai3deng4}
    \definition{s.}{banquinho baixo | banqueta}[这个矮凳是木制的,很结实。(Este banquinho é de madeira e bem resistente.)]
  \end{phonetics}
\end{entry}

\begin{entry}{碍事}{13,8}{⽯、⼅}
  \begin{phonetics}{碍事}{ai4shi4}
    \definition{s.}{(usualmente em frases negativas) sem consequência, não importa}
    \definition{v.+compl.}{estar no caminho | ser um obstáculo}
  \end{phonetics}
\end{entry}

\begin{entry}{碎}{13}{⽯}
  \begin{phonetics}{碎}{sui4}[][HSK 5]
    \definition*{s.}{sobrenome Sui}
    \definition{adj.}{quebrado; fragmentado | tagarela; falante}
    \definition{v.}{(transitivo ou intransitivo) quebrar em pedaços; esmagar}
  \end{phonetics}
\end{entry}

\begin{entry}{碗}{13}{⽯}
  \begin{phonetics}{碗}{wan3}[][HSK 2]
    \definition{clas.}{tigelas}
    \definition[只,个]{s.}{tigela}
  \end{phonetics}
\end{entry}

\begin{entry}{碗子}{13,3}{⽯、⼦}
  \begin{phonetics}{碗子}{wan3zi5}
    \definition{s.}{tigela}
  \end{phonetics}
\end{entry}

\begin{entry}{碗柜}{13,8}{⽯、⽊}
  \begin{phonetics}{碗柜}{wan3gui4}
    \definition{s.}{armário}
  \end{phonetics}
\end{entry}

\begin{entry}{碰}{13}{⽯}
  \begin{phonetics}{碰}{peng4}[][HSK 2]
    \definition{v.}{tocar | bater | encontrar | correr para | tentar a sorte | arriscar | encontrar para discutir}
  \end{phonetics}
\end{entry}

\begin{entry}{碰见}{13,4}{⽯、⾒}
  \begin{phonetics}{碰见}{peng4 jian4}[][HSK 2]
    \definition{v.}{reunir-se | encontrar}
  \end{phonetics}
\end{entry}

\begin{entry}{碰头}{13,5}{⽯、⼤}
  \begin{phonetics}{碰头}{peng4tou2}
    \definition{s.}{colisão | conflito}
    \definition{v.}{colidir}
    \definition{v.+compl.}{conhecer e discutir | juntar ideias | ver-se}
  \end{phonetics}
\end{entry}

\begin{entry}{碰运气}{13,7,4}{⽯、⾡、⽓}
  \begin{phonetics}{碰运气}{peng4yun4qi5}
    \definition{v.}{deixar algo ao acaso | tentar a sorte}
  \end{phonetics}
\end{entry}

\begin{entry}{碰到}{13,8}{⽯、⼑}
  \begin{phonetics}{碰到}{peng4 dao4}[][HSK 2]
    \definition{v.}{encontrar (com) | esbarrar em | deparar-se com}
  \end{phonetics}
\end{entry}

\begin{entry}{禁止}{13,4}{⽰、⽌}
  \begin{phonetics}{禁止}{jin4zhi3}[][HSK 4]
    \definition{v.}{banir; proibir; interditar}
  \end{phonetics}
\end{entry}

\begin{entry}{福}{13}{⽰}
  \begin{phonetics}{福}{fu2}[][HSK 3]
    \definition*{s.}{sobrenome Fu}
    \definition{s.}{benção; felicidade; boa sorte; boa fortuna}
    \definition{v.}{curvar-se; reverenciar}
  \end{phonetics}
\end{entry}

\begin{entry}{福克斯}{13,7,12}{⽰、⼗、⽄}
  \begin{phonetics}{福克斯}{fu2ke4si1}
    \definition*{s.}{Fox (empresa de mídia) | Focus (automóvel fabricado pela Ford)}
  \end{phonetics}
\end{entry}

\begin{entry}{福利}{13,7}{⽰、⼑}
  \begin{phonetics}{福利}{fu2li4}[][HSK 5]
    \definition{s.}{bem-estar; benefícios materiais}
    \definition{v.}{melhorar suas condições de vida; facilitar a vida}
  \end{phonetics}
\end{entry}

\begin{entry}{福泽}{13,8}{⽰、⽔}
  \begin{phonetics}{福泽}{fu2ze2}
    \definition{s.}{boa sorte}
  \end{phonetics}
\end{entry}

\begin{entry}{筷子}{13,3}{⽵、⼦}
  \begin{phonetics}{筷子}{kuai4zi5}[][HSK 2]
    \definition[根,双,副,把,对]{s.}{pauzinhos; \emph{chopsticks}; dois bastôes finos feitos de bambu, madeira, metal ou outro material, usados para segurar comida ou outros objetos}
  \end{phonetics}
\end{entry}

\begin{entry}{签}{13}{⽵}
  \begin{phonetics}{签}{qian1}[][HSK 5]
    \definition{s.}{tiras de bambu usadas para adivinhação ou sorteio; pPequenas tiras de bambu ou varas finas com caracteres e símbolos gravados, usadas para adivinhação, jogos de azar ou como fichas para contagem, etc. | etiqueta; adesivo; pequena tira usada como marca | um pedaço fino e pontiagudo de bambu ou madeira; pequeno bastão pontiagudo}
    \definition{v.}{assinar; autografar; escrever o nome, palavras ou fazer marcas em documentos ou recibos | fazer comentários breves em um documento; escrever brevemente (pontos principais ou opiniões) | (em costura) alinhavar; costura grosseira}
  \end{phonetics}
\end{entry}

\begin{entry}{签订}{13,4}{⽵、⾔}
  \begin{phonetics}{签订}{qian1 ding4}[][HSK 5]
    \definition{v.}{concluir e assinar (um tratado, etc.)}
  \end{phonetics}
\end{entry}

\begin{entry}{签名}{13,6}{⽵、⼝}
  \begin{phonetics}{签名}{qian1 ming2}[][HSK 5]
    \definition[个,次]{s.}{assinatura; autógrafo}
    \definition{v.+compl.}{assinar o próprio nome; autografar; escrever seu nome para indicar concordância, apoio ou homenagem, etc.}
  \end{phonetics}
\end{entry}

\begin{entry}{签字}{13,6}{⽵、⼦}
  \begin{phonetics}{签字}{qian1 zi4}[][HSK 5]
    \definition{v.}{assinar; colocar a assinatura; escrever seu nome à mão em documentos, recibos, etc., para demonstrar responsabilidade}
  \end{phonetics}
\end{entry}

\begin{entry}{签约}{13,6}{⽵、⽷}
  \begin{phonetics}{签约}{qian1 yue1}[][HSK 5]
    \definition{v.}{assinar um contrato; assinar contratos e tratados, frequentemente utilizado no trabalho e em cooperações comerciais}
  \end{phonetics}
\end{entry}

\begin{entry}{签证}{13,7}{⽵、⾔}
  \begin{phonetics}{签证}{qian1zheng4}[][HSK 5]
    \definition[张,个]{s.}{visto; visto de entrada em um país}
  \end{phonetics}
\end{entry}

\begin{entry}{简历}{13,4}{⽵、⼚}
  \begin{phonetics}{简历}{jian3li4}[][HSK 4]
    \definition[个,份]{s.}{currículo; \emph{curriculum vitae} (CV); notas biográficas}
  \end{phonetics}
\end{entry}

\begin{entry}{简单}{13,8}{⽵、⼗}
  \begin{phonetics}{简单}{jian3dan1}[][HSK 3]
    \definition{adj.}{simples; descomplicado | comum; lugar-comum | casual; simplificado}
  \end{phonetics}
\end{entry}

\begin{entry}{简直}{13,8}{⽵、⽬}
  \begin{phonetics}{简直}{jian3zhi2}[][HSK 3]
    \definition{adv.}{simplesmente; em tudo; virtualmente}
  \end{phonetics}
\end{entry}

\begin{entry}{粮食}{13,9}{⽶、⾷}
  \begin{phonetics}{粮食}{liang2shi5}[][HSK 4]
    \definition[种,斤,吨,袋]{s.}{alimentos; grãos; termo geral para os vários tipos de arroz, feijão, etc. que podem ser consumidos}
  \end{phonetics}
\end{entry}

\begin{entry}{缝纫}{13,6}{⽷、⽷}
  \begin{phonetics}{缝纫}{feng2ren4}
    \definition{v.}{costurar}
  \end{phonetics}
\end{entry}

\begin{entry}{缝纫机}{13,6,6}{⽷、⽷、⽊}
  \begin{phonetics}{缝纫机}{feng2ren4ji1}
    \definition[架]{s.}{máquina de costura}
  \end{phonetics}
\end{entry}

\begin{entry}{罪犯}{13,5}{⽹、⽝}
  \begin{phonetics}{罪犯}{zui4fan4}
    \definition{s.}{criminoso}
  \end{phonetics}
\end{entry}

\begin{entry}{罪行}{13,6}{⽹、⾏}
  \begin{phonetics}{罪行}{zui4xing2}
    \definition{s.}{crime | ofensa}
  \end{phonetics}
\end{entry}

\begin{entry}{置疑}{13,14}{⽹、⽦}
  \begin{phonetics}{置疑}{zhi4yi2}
    \definition{v.}{duvidar}
  \end{phonetics}
\end{entry}

\begin{entry}{群}{13}{⽺}
  \begin{phonetics}{群}{qun2}[][HSK 3]
    \definition{clas.}{grupo; rebanho; manada}
    \definition{s.}{multidão; grupo}
  \end{phonetics}
\end{entry}

\begin{entry}{群山}{13,3}{⽺、⼭}
  \begin{phonetics}{群山}{qun2shan1}
    \definition{s.}{montanhas | uma cadeia de colinas}
  \end{phonetics}
\end{entry}

\begin{entry}{群众}{13,6}{⽺、⼈}
  \begin{phonetics}{群众}{qun2zhong4}[][HSK 5]
    \definition[个,名,位]{s.}{as massas; refere-se ao povo em geral | não filiado; apartidário; refere-se a pessoas que não são membros do Partido Comunista Chinês nem da Liga da Juventude Comunista |
alguém que não ocupa uma posição de liderança}
  \end{phonetics}
\end{entry}

\begin{entry}{群体}{13,7}{⽺、⼈}
  \begin{phonetics}{群体}{qun2 ti3}[][HSK 5]
    \definition{s.}{colônia; um conjunto composto por muitos indivíduos da mesma espécie que estão fisicamente conectados, exemplos incluem corais entre os animais e certas algas entre as plantas | grupos; refere-se, de maneira geral, ao conjunto formado por muitos indivíduos interligados que compartilham características essenciais em comum}
  \end{phonetics}
\end{entry}

\begin{entry}{肆}{13}{⾀}
  \begin{phonetics}{肆}{si4}
    \definition*{s.}{sobrenome Si}
    \definition{adj.}{arbitrário; desenfreado; sem limites; descuidado; imprudente}
    \definition{num.}{quatro (usado para o numeral 四 em cheques, etc., para evitar erros ou alterações)}
    \definition{s.}{loja}
  \seealsoref{四}{si4}
  \end{phonetics}
\end{entry}

\begin{entry}{腰}{13}{⾁}
  \begin{phonetics}{腰}{yao1}[][HSK 4]
    \definition*{s.}{sobrenome Yao}
    \definition[个]{s.}{cintura; região lombar | cós | bolso | parte do meio das coisas | lombo}
  \end{phonetics}
\end{entry}

\begin{entry}{腰包}{13,5}{⾁、⼓}
  \begin{phonetics}{腰包}{yao1bao1}
    \definition{s.}{pochete | bolso}
  \end{phonetics}
\end{entry}

\begin{entry}{腰椎}{13,12}{⾁、⽊}
  \begin{phonetics}{腰椎}{yao1zhui1}
    \definition{s.}{vértebra lombar (espinha dorsal inferior)}
  \end{phonetics}
\end{entry}

\begin{entry}{腿}{13}{⾁}
  \begin{phonetics}{腿}{tui3}[][HSK 2]
    \definition[条]{s.}{perna | osso do quadril}
  \end{phonetics}
\end{entry}

\begin{entry}{腿号}{13,5}{⾁、⼝}
  \begin{phonetics}{腿号}{tui3hao4}
    \definition{s.}{anilha numerada (por exemplo, usada para identificar pássaros)}
  \seealsoref{腿号箍}{tui3hao4gu1}
  \end{phonetics}
\end{entry}

\begin{entry}{腿号箍}{13,5,14}{⾁、⼝、⽵}
  \begin{phonetics}{腿号箍}{tui3hao4gu1}
    \definition{s.}{anilha numerada (por exemplo, usada para identificar pássaros)}
  \seealsoref{腿号}{tui3hao4}
  \end{phonetics}
\end{entry}

\begin{entry}{艁}{13}{⾈}
  \begin{phonetics}{艁}{zao4}
    \variantof{造}
  \end{phonetics}
\end{entry}

\begin{entry}{蒙面}{13,9}{⾋、⾯}
  \begin{phonetics}{蒙面}{meng2mian4}
    \definition{adj.}{descarado | desavergonhado | mascarado}
    \definition{v.}{cobrir o rosto | usar uma máscara}
  \end{phonetics}
\end{entry}

\begin{entry}{蓝}{13}{⾋}
  \begin{phonetics}{蓝}{lan2}[][HSK 2]
    \definition*{s.}{sobrenome Lan}
    \definition{adj.}{azul}
    \definition{s.}{planta índigo; anil | plantas azuis; refere-se a certas plantas que podem ser usadas como corante azul ou certas plantas cujas folhas são azul-esverdeadas}
  \end{phonetics}
\end{entry}

\begin{entry}{蓝色}{13,6}{⾋、⾊}
  \begin{phonetics}{蓝色}{lan2 se4}[][HSK 2]
    \definition[抹,片,缕,团,块]{s.}{cor azul}
  \end{phonetics}
\end{entry}

\begin{entry}{解开}{13,4}{⾓、⼶}
  \begin{phonetics}{解开}{jie3 kai1}[][HSK 3]
    \definition{v.}{desatar; desfazer; desamarrar; desabotoar}
  \end{phonetics}
\end{entry}

\begin{entry}{解决}{13,6}{⾓、⼎}
  \begin{phonetics}{解决}{jie3jue2}[][HSK 3]
    \definition{v.}{solucionar; resolver; liquidar | acabar com; descartar}
  \end{phonetics}
\end{entry}

\begin{entry}{解压}{13,6}{⾓、⼚}
  \begin{phonetics}{解压}{jie3ya1}
    \definition{v.}{aliviar o estresse | (computação) descomprimir}
  \end{phonetics}
\end{entry}

\begin{entry}{解放}{13,8}{⾓、⽅}
  \begin{phonetics}{解放}{jie3fang4}[][HSK 5]
    \definition{v.}{libertar; emancipar; eliminar as restrições para permitir o desenvolvimento da liberdade}
  \end{phonetics}
\end{entry}

\begin{entry}{解除}{13,9}{⾓、⾩}
  \begin{phonetics}{解除}{jie3chu2}[][HSK 5]
    \definition{v.}{remover; aliviar; livrar-se de; eliminar}
  \end{phonetics}
\end{entry}

\begin{entry}{解救}{13,11}{⾓、⽁}
  \begin{phonetics}{解救}{jie3jiu4}
    \definition{v.}{resgatar | ajudar a sair de dificuldades | salvar a situação}
  \end{phonetics}
\end{entry}

\begin{entry}{解释}{13,12}{⾓、⾤}
  \begin{phonetics}{解释}{jie3shi4}[][HSK 4]
    \definition{v.}{explicar; expor; interpretar | analisar; explicaro significado, razões, justificativas, etc.}
  \end{phonetics}
\end{entry}

\begin{entry}{解雇}{13,12}{⾓、⾫}
  \begin{phonetics}{解雇}{jie3gu4}
    \definition{v.}{demitir}
  \end{phonetics}
\end{entry}

\begin{entry}{谩骂}{13,9}{⾔、⾺}
  \begin{phonetics}{谩骂}{man4ma4}
    \definition{v.}{ridicularizar | abusar}
  \end{phonetics}
\end{entry}

\begin{entry}{赖}{13}{⾙}
  \begin{phonetics}{赖}{lai4}
    \definition*{s.}{sobrenome Lai}
    \definition{v.}{depender | aguentar em um lugar | renegar (promessa) | isolar-se | culpar | colocar a culpa em}
  \end{phonetics}
\end{entry}

\begin{entry}{跟}{13}{⾜}
  \begin{phonetics}{跟}{gen1}[][HSK 1]
    \definition{conj.}{e; expressa uma relação de união; 和}
    \definition{prep.}{com; Introduzir objetos relacionados à mesma ação, equivalente a 同 | para; em direção a | de; introduzir o objeto de comparação; equivalente a 从, 由 | como; objetos que causam comparações e semelhanças}
    \definition[个]{s.}{calcanhar; parte posterior do pé ou parte posterior do sapato ou meia |
base (de um objeto)}
    \definition{v.}{seguir; acompanhar; seguir imediatamente na mesma direção | (de uma mulher) estar casada com; casar-se com alguém}
  \seealsoref{从}{cong2}
  \seealsoref{和}{he2}
  \seealsoref{同}{tong2}
  \seealsoref{由}{you2}
  \end{phonetics}
\end{entry}

\begin{entry}{跟前}{13,9}{⾜、⼑}
  \begin{phonetics}{跟前}{gen1qian2}[][HSK 5]
    \definition{s.}{próximo; perto de; na frente de; (na ou para) a presença de alguém | o tempo imediatamente anterior a algum evento; tempo que se aproxima}
  \end{phonetics}
  \begin{phonetics}{跟前}{gen1qian5}
    \definition{v.}{(dos filhos de alguém) viver com alguém (exclusivamente com relação à presença ou ausência de crianças)}
  \end{phonetics}
\end{entry}

\begin{entry}{跟随}{13,11}{⾜、⾩}
  \begin{phonetics}{跟随}{gen1sui2}[][HSK 5]
    \definition{s.}{seguidor; usado para se referir a alguém que seguiu}
    \definition{v.}{seguir; ir atrás; acompanhar}
  \end{phonetics}
\end{entry}

\begin{entry}{跪拜}{13,9}{⾜、⼿}
  \begin{phonetics}{跪拜}{gui4bai4}
    \definition{v.}{prostrar-se | ajoelhar-se e adorar}
  \end{phonetics}
\end{entry}

\begin{entry}{路}{13}{⾜}
  \begin{phonetics}{路}{lu4}[][HSK 1]
    \definition*{s.}{sobrenome Lu}
    \definition{clas.}{tipo; classe | linha; coluna; usado para um grupo de pessoas ou uma equipe; para organizar em ordem}
    \definition[条]{s.}{estrada; caminho; via | viagem; jornada; distância | maneira; meios | sequência; linha; lógica | região; distrito | rota | classe; classificação; grau | linha; fileira}
  \end{phonetics}
\end{entry}

\begin{entry}{路上}{13,3}{⾜、⼀}
  \begin{phonetics}{路上}{lu4 shang5}[][HSK 1]
    \definition{s.}{na estrada | a caminho; na rota; em processo de mudança de um lugar para outro}
  \end{phonetics}
\end{entry}

\begin{entry}{路口}{13,3}{⾜、⼝}
  \begin{phonetics}{路口}{lu4 kou3}[][HSK 1]
    \definition[个]{s.}{cruzamento; intersecção; onde as estradas se encontram}
  \end{phonetics}
\end{entry}

\begin{entry}{路边}{13,5}{⾜、⾡}
  \begin{phonetics}{路边}{lu4 bian1}[][HSK 2]
    \definition{s.}{calçada; beira da estrada; margem da rua}
  \end{phonetics}
\end{entry}

\begin{entry}{路线}{13,8}{⾜、⽷}
  \begin{phonetics}{路线}{lu4 xian4}[][HSK 3]
    \definition[条]{s.}{rota; caminho; linha | linha; diretriz (de política, ideologia, campo de trabalho)}
  \end{phonetics}
\end{entry}

\begin{entry}{跳}{13}{⾜}
  \begin{phonetics}{跳}{tiao4}[][HSK 3]
    \definition{v.}{pular; saltar; quicar | mover para cima e para baixo; pulsar; palpitar; contrair-se | pular; saltar por cima}
  \end{phonetics}
\end{entry}

\begin{entry}{跳水}{13,4}{⾜、⽔}
  \begin{phonetics}{跳水}{tiao4shui3}
    \definition{s.}{mergulho esportivo}
    \definition{v.}{mergulhar (na água) | cometer suicídio pulando na água | (figurativo, preços das ações, etc.) cair dramaticamente}
  \end{phonetics}
\end{entry}

\begin{entry}{跳电}{13,5}{⾜、⽥}
  \begin{phonetics}{跳电}{tiao4dian4}
    \definition{v.}{desarmar (um disjuntor ou interruptor)}
  \end{phonetics}
\end{entry}

\begin{entry}{跳伞}{13,6}{⾜、⼈}
  \begin{phonetics}{跳伞}{tiao4san3}
    \definition{s.}{paraquedas}
    \definition{v.}{saltar de paraquedas}
  \end{phonetics}
\end{entry}

\begin{entry}{跳远}{13,7}{⾜、⾡}
  \begin{phonetics}{跳远}{tiao4 yuan3}[][HSK 3]
    \definition{s.}{salto em distância (atletismo)}
  \end{phonetics}
\end{entry}

\begin{entry}{跳挡}{13,9}{⾜、⼿}
  \begin{phonetics}{跳挡}{tiao4dang3}
    \definition{v.}{pular marcha (de um carro) | perder a marcha}
  \end{phonetics}
\end{entry}

\begin{entry}{跳蚤}{13,9}{⾜、⾍}
  \begin{phonetics}{跳蚤}{tiao4zao5}
    \definition{s.}{pulga}
  \end{phonetics}
\end{entry}

\begin{entry}{跳高}{13,10}{⾜、⾼}
  \begin{phonetics}{跳高}{tiao4 gao1}[][HSK 3]
    \definition{s.}{salto em altura (atletismo)}
  \end{phonetics}
\end{entry}

\begin{entry}{跳绳}{13,11}{⾜、⽷}
  \begin{phonetics}{跳绳}{tiao4sheng2}
    \definition{v.}{pular corda}
  \end{phonetics}
\end{entry}

\begin{entry}{跳跳糖}{13,13,16}{⾜、⾜、⽶}
  \begin{phonetics}{跳跳糖}{tiao4tiao4tang2}
    \definition{s.}{\emph{Pop Rocks}, \emph{popping candy}}
  \end{phonetics}
\end{entry}

\begin{entry}{跳频}{13,13}{⾜、⾴}
  \begin{phonetics}{跳频}{tiao4pin2}
    \definition{s.}{FHSS, \emph{Frequency-Hopping Spread Spectrum}, método de transmissão de sinais de rádio}
  \end{phonetics}
\end{entry}

\begin{entry}{跳舞}{13,14}{⾜、⾇}
  \begin{phonetics}{跳舞}{tiao4wu3}[][HSK 3]
    \definition{v.+compl.}{dançar (como performance)}
  \end{phonetics}
\end{entry}

\begin{entry}{躲}{13}{⾝}
  \begin{phonetics}{躲}{duo3}[][HSK 5]
    \definition{v.}{esconder (a si mesmo); ocultar (a si mesmo); esconder-se | evitar; esquivar-se}
  \end{phonetics}
\end{entry}

\begin{entry}{躲闪}{13,5}{⾝、⾨}
  \begin{phonetics}{躲闪}{duo3shan3}
    \definition{v.}{desviar | evadir | esquivar (para fora do caminho)}
  \end{phonetics}
\end{entry}

\begin{entry}{输}{13}{⾞}
  \begin{phonetics}{输}{shu1}[][HSK 3]
    \definition{v.}{transportar; transmitir | contribuir com dinheiro; doar | perder; ser batido; ser derrotado}
  \end{phonetics}
\end{entry}

\begin{entry}{输入}{13,2}{⾞、⼊}
  \begin{phonetics}{输入}{shu1ru4}[][HSK 3]
    \definition{v.}{introduzir; importar  (de fora para dentro) | inserir informações, programas, dados, sinais, etc. em uma máquina}
  \end{phonetics}
\end{entry}

\begin{entry}{输出}{13,5}{⾞、⼐}
  \begin{phonetics}{输出}{shu1 chu1}[][HSK 5]
    \definition{v.}{exportar (de dentro para fora); transportar (de dentro) para fora | exportar; vender ou distribuir no exterior ou fora do país | emitir informações, programas, dados, sinais, etc. a partir de uma máquina; enviar por uma determinada instituição ou dispositivo (energia, sinal, etc.)}
  \end{phonetics}
\end{entry}

\begin{entry}{辞典}{13,8}{⾟、⼋}
  \begin{phonetics}{辞典}{ci2 dian3}[][HSK 5]
    \definition[本,部]{s.}{dicionário; coleção de termos especializados ou enciclopédicos, organizados em uma determinada ordem e explicados, para fins de referência}
    \variantof{词典}
  \end{phonetics}
\end{entry}

\begin{entry}{辞职}{13,11}{⾟、⽿}
  \begin{phonetics}{辞职}{ci2zhi2}[][HSK 5]
    \definition{v.+compl.}{renunciar; deixar o cargo; entregar a renúncia; pedir para ser dispensado de suas funções}
  \end{phonetics}
\end{entry}

\begin{entry}{遛狗}{13,8}{⾡、⽝}
  \begin{phonetics}{遛狗}{liu4gou3}
    \definition{v.+compl.}{passear com um cachorro}
  \end{phonetics}
\end{entry}

\begin{entry}{遥控}{13,11}{⾡、⼿}
  \begin{phonetics}{遥控}{yao2kong4}
    \definition{s.}{controle remoto}
    \definition{v.}{dirigir operações de um local remoto | controlar remotamente}
  \end{phonetics}
\end{entry}

\begin{entry}{酬劳}{13,7}{⾣、⼒}
  \begin{phonetics}{酬劳}{chou2lao2}
    \definition{s.}{recompensa}
  \end{phonetics}
\end{entry}

\begin{entry}{酱}{13}{⾣}
  \begin{phonetics}{酱}{jiang4}
    \definition{s.}{pasta grossa de soja fermentada | marinada em pasta de soja | pasta | geléia}
  \end{phonetics}
\end{entry}

\begin{entry}{错}{13}{⾦}
  \begin{phonetics}{错}{cuo4}[][HSK 1]
    \definition*{s.}{sobrenome Cuo}
    \definition{adj.}{errado; equivocado; errôneo | (na negativa) nada ruim; muito bom | entrelaçado e recortado; intrincado; complexo | ruim; pobre; péssimo (usado apenas em negativas)}
    \definition{s.}{falha; demérito | erro; engano | (arcaico) pedra de amolar para polir jade}
    \definition{v.}{estar entrelaçado e serrilhado; ser intrincado | moer; esfregar | abrir caminho; sair do caminho | alternar; escalonar | estar fora de alinhamento | deslocar | evitar; fazer com que não se encontre ou não entre em conflito | polir; polir pedras preciosas | (literário) incrustar ou revestir com ouro, prata, etc. | interseccionar; cruzar; entrecruzar}
  \end{phonetics}
\end{entry}

\begin{entry}{错误}{13,9}{⾦、⾔}
  \begin{phonetics}{错误}{cuo4wu4}[][HSK 3]
    \definition{adj.}{equivocado; errado; errôneo}
    \definition[个,次]{s.}{engano; erro; erro grosseiro; falha}
  \end{phonetics}
\end{entry}

\begin{entry}{锤}{13}{⾦}
  \begin{phonetics}{锤}{chui2}
    \definition{s.}{martelo | marreta}
    \definition{s.}{pesos (por exemplo, de uma balança)}
    \definition{v.}{marterlar para dar forma | atacar com um martelo}
  \end{phonetics}
\end{entry}

\begin{entry}{锦上添花}{13,3,11,7}{⾦、⼀、⽔、⾋}
  \begin{phonetics}{锦上添花}{jin3shang4tian1hua1}
    \definition{expr.}{A cereja do bolo | (literalmente) adicione flores ao brocato}
    \definition{v.}{dar a alguém esplendor adicional | fornecer o toque final}
  \end{phonetics}
\end{entry}

\begin{entry}{键}{13}{⾦}
  \begin{phonetics}{键}{jian4}[][HSK 5]
    \definition[个]{s.}{pino (para máquinas) | tecla (de uma máquina de escrever, piano, etc.) | chave | etapa crucial}
  \end{phonetics}
\end{entry}

\begin{entry}{键盘}{13,11}{⾦、⽫}
  \begin{phonetics}{键盘}{jian4pan2}[][HSK 5]
    \definition[个]{s.}{braço; teclado; cravo; painel de teclas; porta-chaves}
  \end{phonetics}
\end{entry}

\begin{entry}{零下}{13,3}{⾬、⼀}
  \begin{phonetics}{零下}{ling2 xia4}[][HSK 2]
    \definition{s.}{abaixo de zero; negativo}
  \end{phonetics}
\end{entry}

\begin{entry}{零食}{13,9}{⾬、⾷}
  \begin{phonetics}{零食}{ling2shi2}[][HSK 4]
    \definition[包,袋,盒,箱,堆]{s.}{lanches; refrescos; petiscos entre as refeições; alimentação esporádica, além das refeições normais}
  \end{phonetics}
\end{entry}

\begin{entry}{零散}{13,12}{⾬、⽁}
  \begin{phonetics}{零散}{ling2san3}
    \definition{adj.}{espalhado; disperso}
  \end{phonetics}
\end{entry}

\begin{entry}{零/〇}{13,13}{⾬、⾬}
  \begin{phonetics}{零/〇}{ling2 ling2}[][HSK 1]
    \definition*{s.}{sobrenome Ling}
    \definition{adj.}{ímpar; dispersos; fragmentados (em oposição a 整)}
    \definition{num.}{zero; 0; representa um número menor que qualquer número positivo e maior que qualquer número negativo; representa a ausência de quantidade | zero grau no termômetro | usado para indicar qualidade, comprimento, tempo, idade, etc. Entre dois dígitos, indica que a quantidade da unidade mais alta é acompanhada pela quantidade da unidade mais baixa | sinal de zero (0); nulo; espaço em branco para indicar números em caracteres chineses maiúsculos}
    \definition{s.}{fragmento; fração; lote ímpar; um número fracionário que não é suficiente para uma determinada unidade; um ponto decimal diferente de um inteiro}
    \definition{v.}{(de chuva, lágrimas, etc.) cair | murchar e cair}
  \seealsoref{整}{zheng3}
  \end{phonetics}
\end{entry}

\begin{entry}{雷电}{13,5}{⾬、⽥}
  \begin{phonetics}{雷电}{lei2dian4}
    \definition{s.}{trovão e relâmpago; raio}
  \end{phonetics}
\end{entry}

\begin{entry}{雷亚尔}{13,6,5}{⾬、⼆、⼩}
  \begin{phonetics}{雷亚尔}{lei2ya4'er3}
    \definition*{s.}{Real Brasileiro}
  \end{phonetics}
\end{entry}

\begin{entry}{雾气}{13,4}{⾬、⽓}
  \begin{phonetics}{雾气}{wu4qi4}
    \definition{s.}{nevoeiro | névoa | vapor}
  \end{phonetics}
\end{entry}

\begin{entry}{颐和园}{13,8,7}{⾴、⼝、⼞}
  \begin{phonetics}{颐和园}{yi2he2yuan2}
    \definition*{s.}{Palácio de Verão}
  \end{phonetics}
\end{entry}

\begin{entry}{频道}{13,12}{⾴、⾡}
  \begin{phonetics}{频道}{pin2dao4}[][HSK 5]
    \definition[个]{s.}{canal; canal de frequência; televisão e rádio, os sinais de som e imagem ocupam um determinado canal de frequência}
  \end{phonetics}
\end{entry}

\begin{entry}{频繁}{13,17}{⾴、⽷}
  \begin{phonetics}{频繁}{pin2fan2}[][HSK 5]
    \definition{adj.}{frequentemente}
    \definition{adj.}{frequente}
  \end{phonetics}
\end{entry}

\begin{entry}{魂}{13}{⿁}
  \begin{phonetics}{魂}{hun2}
    \definition{s.}{alma | espírito | alma imortal (que pode ser separada do corpo)}
  \end{phonetics}
\end{entry}

\begin{entry}{鼓}{13}{⿎}
  \begin{phonetics}{鼓}{gu3}[][HSK 5]
    \definition*{s.}{sobrenome Gu}
    \definition{adj.}{abaulado; inchado; saliente; protuberante}
    \definition{clas.}{unidades antigas de cronometragem noturna; vigílias da noite}
    \definition{s.}{tambor; instrumento de percussão |
coisas semelhantes a tambores; formato, som e função semelhantes aos de um tambor |}
    \definition{v.}{soar; bater; golpear; fazer um objeto soar | ventilar; soprar com fole | agitar; despertar; ativar; incitar; revigorar | bater asas | aumentar; fazer beicinho}
  \end{phonetics}
\end{entry}

\begin{entry}{鼓励}{13,7}{⿎、⼒}
  \begin{phonetics}{鼓励}{gu3li4}[][HSK 5]
    \definition{v.}{incitar; encorajar; provocar e incentivar}
  \end{phonetics}
\end{entry}

\begin{entry}{鼓掌}{13,12}{⿎、⼿}
  \begin{phonetics}{鼓掌}{gu3zhang3}[][HSK 5]
    \definition{v.+compl.}{aplaudir; bater palmas, principalmente para expressar felicidade, aprovação ou boas-vindas}
  \end{phonetics}
\end{entry}

\begin{entry}{鼠}{13}{⿏}[Kangxi 208]
  \begin{phonetics}{鼠}{shu3}[][HSK 5]
    \definition[只]{s.}{rato; camundongo}
  \end{phonetics}
\end{entry}

\begin{entry}{鼠标}{13,9}{⿏、⽊}
  \begin{phonetics}{鼠标}{shu3biao1}[][HSK 5]
    \definition[个]{s.}{\emph{mouse} (de computador); dispositivo de entrada externo para computadores, usado para controlar o movimento do cursor na tela do computador, selecionar objetos de operação, executar vários comandos, etc.}
  \end{phonetics}
\end{entry}

%%%%% EOF %%%%%


%%%
%%% 14画
%%%

\section*{14画}\addcontentsline{toc}{section}{14画}

\begin{Entry}{㮸}{14}{⽊}
  \begin{Phonetics}{㮸}{song4}
    \variantof{送}
  \end{Phonetics}
\end{Entry}

\begin{Entry}{僧}{14}{⼈}
  \begin{Phonetics}{僧}{seng1}
    \definition*{s.}{Sobrenome Seng}
    \definition[位,名,个]{s.}{monge Budista, abreviação de 僧伽}
  \seealsoref{僧伽}{seng1qie2}
  \end{Phonetics}
\end{Entry}

\begin{Entry}{僧伽}{14,7}{⼈、⼈}
  \begin{Phonetics}{僧伽}{seng1qie2}
    \definition{s.}{sangha ou sanga (Budismo) | a comunidade monástica | monge}
  \end{Phonetics}
\end{Entry}

\begin{Entry}{僮}{14}{⼈}
  \begin{Phonetics}{僮}{tong2}
    \definition*{s.}{Sobrenome Tong}
  \end{Phonetics}
  \begin{Phonetics}{僮}{zhuang4}
    \variantof{壮}
  \end{Phonetics}
\end{Entry}

\begin{Entry}{凳}{14}{⼏}
  \begin{Phonetics}{凳}{deng4}
    \definition[条]{s.}{banco; banqueta}
  \end{Phonetics}
\end{Entry}

\begin{Entry}{凳子}{14,3}{⼏、⼦}
  \begin{Phonetics}{凳子}{deng4zi5}[][HSK 7-9]
    \definition[把,条,个]{s.}{banco; banqueta; um móvel que tem pernas para sentar, mas não tem encosto}
  \end{Phonetics}
\end{Entry}

\begin{Entry}{嘉}{14}{⼝}
  \begin{Phonetics}{嘉}{jia1}
    \definition*{s.}{Sobrenome Jia}
    \definition{adj.}{bom; ótimo | auspicioso | excelente}
    \definition{v.}{elogiar; recomendar}
    \definition{v.}{elogiar}
  \end{Phonetics}
\end{Entry}

\begin{Entry}{嘉年华}{14,6,6}{⼝、⼲、⼗}
  \begin{Phonetics}{嘉年华}{jia1nian2hua2}
    \definition{s.}{(empréstimo linguístico) carnaval}
  \end{Phonetics}
\end{Entry}

\begin{Entry}{嘉宾}{14,10}{⼝、⼧}
  \begin{Phonetics}{嘉宾}{jia1bin1}[][HSK 6]
    \definition[个,位,名,些]{s.}{convidado}
  \end{Phonetics}
\end{Entry}

\begin{Entry}{嘛}{14}{⼝}
  \begin{Phonetics}{嘛}{ma5}[][HSK 6]
    \definition{part.}{usado no final de uma declaração para expressar que é claro que é verdade que é óbvio | usado no final de uma frase imperativa para expressar expectativa ou dissuasão | usado em uma frase para indicar uma pausa e chamar a atenção da outra pessoa}
  \end{Phonetics}
\end{Entry}

\begin{Entry}{墙}{14}{⼟}
  \begin{Phonetics}{墙}{qiang2}[][HSK 2]
    \definition[面,堵,道]{s.}{parede; barreira ou perímetro construído com tijolos, pedras, etc. | qualquer coisa com a forma ou função de uma parede; a parte de um objeto que funciona como parede ou divisória}
    \definition{v.}{(gíria) bloquear (um website) (usado geralmente na voz passiva: 被墙)}
  \end{Phonetics}
\end{Entry}

\begin{Entry}{墙纸}{14,7}{⼟、⽷}
  \begin{Phonetics}{墙纸}{qiang2zhi3}
    \definition{s.}{papel de parede}
  \end{Phonetics}
\end{Entry}

\begin{Entry}{墙壁}{14,16}{⼟、⼟}
  \begin{Phonetics}{墙壁}{qiang2 bi4}[][HSK 5]
    \definition[面,堵,道]{s.}{parede; barreira ou perímetro construído com tijolos, pedras ou terra}
  \end{Phonetics}
\end{Entry}

\begin{Entry}{墬}{14}{⼟}
  \begin{Phonetics}{墬}{di4}
    \variantof{地}
  \end{Phonetics}
\end{Entry}

\begin{Entry}{嫦}{14}{⼥}
  \begin{Phonetics}{嫦}{chang2}
    \definition{s.}{uma beleza lendária que voou para a lua | a dama da lua}
  \end{Phonetics}
\end{Entry}

\begin{Entry}{嫦娥}{14,10}{⼥、⼥}
  \begin{Phonetics}{嫦娥}{chang2'e2}[][HSK 7-9]
    \definition*{s.}{Chang'e, a dama da lua (mitologia chinesa); uma fada que voou do mundo humano para o Palácio da Lua na mitologia}
  \end{Phonetics}
\end{Entry}

\begin{Entry}{孵}{14}{⼦}
  \begin{Phonetics}{孵}{fu1}
    \definition{v.}{chocar; incubar; (pássaros) sentar em ovos}
  \end{Phonetics}
\end{Entry}

\begin{Entry}{孵化}{14,4}{⼦、⼔}
  \begin{Phonetics}{孵化}{fu1hua4}[][HSK 7-9]
    \definition{v.}{chocar; incubar | incubar; metaforicamente, cultivar e desenvolver coisas novas (agora se refere principalmente ao suporte a empresas de alta tecnologia recém-criadas)}
  \end{Phonetics}
\end{Entry}

\begin{Entry}{察}{14}{⼧}
  \begin{Phonetics}{察}{cha2}
    \definition*{s.}{Sobrenome Cha}
    \definition{v.}{examinar; investigar; escrutinar | observar; olhar atentamente; investigar}
  \end{Phonetics}
\end{Entry}

\begin{Entry}{察看}{14,9}{⼧、⽬}
  \begin{Phonetics}{察看}{cha2kan4}[][HSK 7-9]
    \definition{v.}{observar; olhar atentamente; inspecionar}
  \end{Phonetics}
\end{Entry}

\begin{Entry}{察觉}{14,9}{⼧、⾒}
  \begin{Phonetics}{察觉}{cha2jue2}[][HSK 7-9]
    \definition{v.}{detectar; perceber; estar ciente de; estar consciente de; descobrir; ver}
  \end{Phonetics}
\end{Entry}

\begin{Entry}{寡}{14}{⼧}
  \begin{Phonetics}{寡}{gua3}
    \definition{adj.}{poucos; escassos (oposto a 众, 多)  | insípido; sem sabor | pouco; escasso | insípido; sem graça}
    \definition{pron.}{eu; título autoproclamado de um antigo monarca}
    \definition{s.}{viúva | viuvez; a natureza ou estado de uma mulher viúva que vive sozinha}
  \seealsoref{多}{duo1}
  \seealsoref{众}{zhong4}
  \end{Phonetics}
\end{Entry}

\begin{Entry}{寡妇}{14,6}{⼧、⼥}
  \begin{Phonetics}{寡妇}{gua3fu5}[][HSK 7-9]
    \definition[个]{s.}{viúva; uma mulher cujo marido morreu}
  \end{Phonetics}
\end{Entry}

\begin{Entry}{寨}{14}{⼧}
  \begin{Phonetics}{寨}{zhai4}
    \definition{s.}{fortaleza | paliçada | acampamento | vila (paliçada)}
  \end{Phonetics}
\end{Entry}

\begin{Entry}{弊}{14}{⼶}
  \begin{Phonetics}{弊}{bi4}
    \definition{s.}{fraude; abuso; negligência médica | desvantagem; falha; defeito; dano (oposto a 利) | trapaça; fraude, engano e falsificação}
  \seealsoref{利}{li4}
  \end{Phonetics}
\end{Entry}

\begin{Entry}{弊病}{14,10}{⼶、⽧}
  \begin{Phonetics}{弊病}{bi4bing4}[][HSK 7-9]
    \definition{s.}{doença; mal; negligência | incinveniente; desvantagem; problemas com coisas}
  \end{Phonetics}
\end{Entry}

\begin{Entry}{弊端}{14,14}{⼶、⽴}
  \begin{Phonetics}{弊端}{bi4duan1}[][HSK 7-9]
    \definition{s.}{abuso; negligência; prática corrupta; danos ao interesse público devido a uma lacuna no trabalho}
  \end{Phonetics}
\end{Entry}

\begin{Entry}{愿}{14}{⽕}
  \begin{Phonetics}{愿}{yuan4}[][HSK 5]
    \definition{adj.}{honesto e prudente}
    \definition{s.}{esperança; desejo; vontade; a ideia de alcançar algum objetivo no futuro | voto (feito perante o Buda ou um deus); o desejo de retribuição feito ao rezar para os deuses e Buda}
    \definition{v.}{estar disposto; estar pronto; de bom grado, concordar porque está de acordo com seus desejos | ter esperança; desejar; qerer alcançar algum desejo}
  \end{Phonetics}
\end{Entry}

\begin{Entry}{愿望}{14,11}{⽕、⽉}
  \begin{Phonetics}{愿望}{yuan4wang4}[][HSK 3]
    \definition[个,种]{s.}{desejo; aspiração; a ideia de alcançar algum objetivo no futuro.}
  \end{Phonetics}
\end{Entry}

\begin{Entry}{愿意}{14,13}{⽕、⼼}
  \begin{Phonetics}{愿意}{yuan4yi4}[][HSK 2]
    \definition{v.}{estar disposto; estar pronto | desejar; ter esperança}
  \end{Phonetics}
\end{Entry}

\begin{Entry}{慢}{14}{⼼}
  \begin{Phonetics}{慢}{man4}[][HSK 1]
    \definition*{s.}{Sobrenome Man}
    \definition{adj.}{lento; devagar; baixa velocidade; longa duração (em oposição a 快) | rude; arrogante; sem educação com as pessoas | frouxo; lento}
    \definition{adv.}{lentamente}
  \seealsoref{快}{kuai4}
  \end{Phonetics}
\end{Entry}

\begin{Entry}{慢车}{14,4}{⼼、⾞}
  \begin{Phonetics}{慢车}{man4 che1}[][HSK 6]
    \definition{s.}{trem lento com muitas paradas (oposto a 快车) | ônibus ou trem local; parada do trem}
  \seealsoref{快车}{kuai4 che1}
  \end{Phonetics}
\end{Entry}

\begin{Entry}{慢动作}{14,6,7}{⼼、⼒、⼈}
  \begin{Phonetics}{慢动作}{man4dong4zuo4}
    \definition{s.}{(cinema) câmera lenta}
  \end{Phonetics}
\end{Entry}

\begin{Entry}{慢慢}{14,14}{⼼、⼼}
  \begin{Phonetics}{慢慢}{man4 man4}[][HSK 3]
    \definition{adv.}{lentamente; vagarosamente; gradualmente | lentamente; vagarosamente; gradualmente; depois de um longo período de tempo}
  \end{Phonetics}
\end{Entry}

\begin{Entry}{截}{14}{⼽}
  \begin{Phonetics}{截}{jie2}
    \definition{clas.}{seção; pedaço; comprimento}
    \definition{prep.}{por (um tempo especificado); até}
    \definition{v.}{cortar; romper | parar; verificar; interromper; interceptar}
  \end{Phonetics}
\end{Entry}

\begin{Entry}{截止}{14,4}{⼽、⽌}
  \begin{Phonetics}{截止}{jie2zhi3}[][HSK 6]
    \definition{adv.}{até (um certo limite de tempo); por (um tempo especificado)}
  \end{Phonetics}
\end{Entry}

\begin{Entry}{截至}{14,6}{⼽、⾄}
  \begin{Phonetics}{截至}{jie2zhi4}[][HSK 6]
    \definition{adv.}{a partir de; até (um certo limite de tempo); por (um tempo especificado)}
  \end{Phonetics}
\end{Entry}

\begin{Entry}{摔}{14}{⼿}
  \begin{Phonetics}{摔}{shuai1}[][HSK 5]
    \definition{v.}{cair; tropeçar; perder o equilíbrio | mergulhar; precipitar-se; cair de uma altura elevada | quebrar; fazer cair e quebrar | lançar; atirar; arremessar; joguar coisas com força e para baixo | bater; golpear; bater com força para que o que está grudado cair}
  \end{Phonetics}
\end{Entry}

\begin{Entry}{摔倒}{14,10}{⼿、⼈}
  \begin{Phonetics}{摔倒}{shuai1dao3}[][HSK 5]
    \definition{v.}{cair; tropeçar; perder o equilíbrio e cair}
  \end{Phonetics}
\end{Entry}

\begin{Entry}{摘}{14}{⼿}
  \begin{Phonetics}{摘}{zhai1}[][HSK 5]
    \definition{v.}{pegar; arrancar; tirar; colher (flores, frutos, folhas de plantas); retirar (coisas que estão sendo usadas ou penduradas) | selecionar; fazer extrações de | pedir dinheiro emprestado em caso de necessidade urgente | vencer; ganhar; alcançar; obter}
  \end{Phonetics}
\end{Entry}

\begin{Entry}{摧}{14}{⼿}
  \begin{Phonetics}{摧}{cui1}
    \definition{v.}{quebrar; destruir}
  \end{Phonetics}
\end{Entry}

\begin{Entry}{摧毁}{14,13}{⼿、⽎}
  \begin{Phonetics}{摧毁}{cui1hui3}[][HSK 7-9]
    \definition{v.}{destruir; esmagar; nocautear; destruir com grande força}
  \end{Phonetics}
\end{Entry}

\begin{Entry}{敲}{14}{⽁}
  \begin{Phonetics}{敲}{qiao1}[][HSK 5]
    \definition{v.}{bater; dar uma pancada; golpear | explorar alguém; cobrar a mais; extorquir; chantagear | lembrar; criticar; alertar; advertir}
  \end{Phonetics}
\end{Entry}

\begin{Entry}{敲门}{14,3}{⽁、⾨}
  \begin{Phonetics}{敲门}{qiao1 men2}[][HSK 5]
    \definition{v.}{bater na porta}
  \end{Phonetics}
\end{Entry}

\begin{Entry}{斡}{14}{⽃}
  \begin{Phonetics}{斡}{wo4}
    \definition{v.}{virar-se}
  \end{Phonetics}
\end{Entry}

\begin{Entry}{斡旋}{14,11}{⽃、⽅}
  \begin{Phonetics}{斡旋}{wo4xuan2}
    \definition{v.}{mediar (um conflito, etc.)}
  \end{Phonetics}
\end{Entry}

\begin{Entry}{旗}{14}{⽅}
  \begin{Phonetics}{旗}{qi2}
    \definition[面]{s.}{bandeira}
  \end{Phonetics}
\end{Entry}

\begin{Entry}{榜}{14}{⽊}
  \begin{Phonetics}{榜}{bang3}
    \definition[块]{s.}{lista publicada de nomes | Literário: placa horizontal inscrita | aviso; anúncio; proclamação antiga}
  \end{Phonetics}
\end{Entry}

\begin{Entry}{榜首}{14,9}{⽊、⾸}
  \begin{Phonetics}{榜首}{bang3shou3}[][HSK 7-9]
    \definition{s.}{cabeça da lista de candidatos aprovados; primeiro lugar em um concurso, etc. | topo da lista}
  \end{Phonetics}
\end{Entry}

\begin{Entry}{榜样}{14,10}{⽊、⽊}
  \begin{Phonetics}{榜样}{bang3yang4}[][HSK 7-9]
    \definition[个,位]{s.}{exemplo; modelo; padrão; pessoas ou coisas boas que valem a pena aprender, usado principalmente na linguagem falada}
  \end{Phonetics}
\end{Entry}

\begin{Entry}{槃}{14}{⽊}
  \begin{Phonetics}{槃}{pan2}
    \variantof{盘}
  \end{Phonetics}
\end{Entry}

\begin{Entry}{模}{14}{⽊}
  \begin{Phonetics}{模}{mo2}
    \definition{s.}{padrão | modelo; exemplo | modelo (pessoa) | exame simulado | módulo}
    \definition{v.}{imitar | copiar; emular}
  \end{Phonetics}
  \begin{Phonetics}{模}{mu2}
    \definition*{s.}{Sobrenome Mu}
    \definition{s.}{molde; padrão; matriz}
  \end{Phonetics}
\end{Entry}

\begin{Entry}{模仿}{14,6}{⽊、⼈}
  \begin{Phonetics}{模仿}{mo2fang3}[][HSK 5]
    \definition{v.}{copiar; imitar; aprender a fazer algo seguindo um modelo pronto}
  \end{Phonetics}
\end{Entry}

\begin{Entry}{模式}{14,6}{⽊、⼷}
  \begin{Phonetics}{模式}{mo2shi4}[][HSK 5]
    \definition{s.}{modelo; modo; padrão; a forma padrão de algo ou o modelo padrão que as pessoas podem seguir}
  \end{Phonetics}
\end{Entry}

\begin{Entry}{模具}{14,8}{⽊、⼋}
  \begin{Phonetics}{模具}{mu2ju4}
    \definition{s.}{molde | matriz | padrão}
  \end{Phonetics}
\end{Entry}

\begin{Entry}{模型}{14,9}{⽊、⼟}
  \begin{Phonetics}{模型}{mo2xing2}[][HSK 4]
    \definition[个]{s.}{modelo; padrão; itens feitos em escala com base em objetos ou desenhos | molde; padrão; molde para fundir máquinas, objetos, etc.}
  \end{Phonetics}
\end{Entry}

\begin{Entry}{模范}{14,9}{⽊、⾋}
  \begin{Phonetics}{模范}{mo2fan4}[][HSK 5]
    \definition{adj.}{exemplar}
    \definition{s.}{modelo; exemplo excelente; pessoa exemplar; coisa exemplar; pessoas ou coisas exemplares que servem de modelo}
  \end{Phonetics}
\end{Entry}

\begin{Entry}{模样}{14,10}{⽊、⽊}
  \begin{Phonetics}{模样}{mu2yang4}[][HSK 5]
    \definition[副,种]{s.}{aparência; a aparência ou o estilo de vestir de uma pessoa | indicando uma estimativa aproximada de tempo ou idade; expressão de estimativas relativas a tempo, idade, etc. | tendência; situação; inclinação}
  \end{Phonetics}
\end{Entry}

\begin{Entry}{模特儿}{14,10,2}{⽊、⽜、⼉}
  \begin{Phonetics}{模特儿}{mo2 te4r5}[][HSK 4]
    \definition[个,名,位]{s.}{modelo (pessoa que posa para um fotógrafo ou pintor ou escultor); objeto de representação ou referência usado por artistas para esboços e esculturas, como o corpo humano, objetos, modelos etc.; também se refere aos arquétipos que os estudiosos da literatura usam para retratar seus personagens | modelo (uma pessoa que usa roupas para exibir modas); pessoa ou manequim usado para exibir estilos de roupas}
  \end{Phonetics}
\end{Entry}

\begin{Entry}{模糊}{14,15}{⽊、⽶}
  \begin{Phonetics}{模糊}{mo2hu5}[][HSK 5]
    \definition{adj.}{vago; confuso; indistinto}
    \definition{v.}{confundir; desorientar}
  \end{Phonetics}
\end{Entry}

\begin{Entry}{歌}{14}{⽋}
  \begin{Phonetics}{歌}{ge1}[][HSK 1]
    \definition[首,支,段]{s.}{canção; poesia cantável}
    \definition{v.}{cantar; entoar | louvar; exaltar; cantar louvores a}
  \end{Phonetics}
\end{Entry}

\begin{Entry}{歌手}{14,4}{⽋、⼿}
  \begin{Phonetics}{歌手}{ge1 shou3}[][HSK 3]
    \definition[个,位,名]{s.}{cantor; vocalista; pessoa com talento para cantar}
  \end{Phonetics}
\end{Entry}

\begin{Entry}{歌曲}{14,6}{⽋、⽈}
  \begin{Phonetics}{歌曲}{ge1 qu3}[][HSK 5]
    \definition[首,支]{s.}{música; obra para as pessoas cantarem, uma combinação de poesia e música}
  \end{Phonetics}
\end{Entry}

\begin{Entry}{歌声}{14,7}{⽋、⼠}
  \begin{Phonetics}{歌声}{ge1 sheng1}[][HSK 3]
    \definition{s.}{canto; voz cantada; som do canto}
  \end{Phonetics}
\end{Entry}

\begin{Entry}{歌词}{14,7}{⽋、⾔}
  \begin{Phonetics}{歌词}{ge1 ci2}[][HSK 6]
    \definition{s.}{letra da música; libreto}
  \end{Phonetics}
\end{Entry}

\begin{Entry}{歌咏}{14,8}{⽋、⼝}
  \begin{Phonetics}{歌咏}{ge1yong3}[][HSK 7-9]
    \definition{v.}{cantar; cantar canções}
  \end{Phonetics}
\end{Entry}

\begin{Entry}{歌星}{14,9}{⽋、⽇}
  \begin{Phonetics}{歌星}{ge1 xing1}[][HSK 6]
    \definition[位,名]{s.}{cantor famoso; estrela da música}
  \end{Phonetics}
\end{Entry}

\begin{Entry}{歌迷}{14,9}{⽋、⾡}
  \begin{Phonetics}{歌迷}{ge1 mi2}
    \definition{s.}{fã de um cantor; pessoas que gostam de ouvir música ou cantar e ficam fascinadas por isso}
  \end{Phonetics}
\end{Entry}

\begin{Entry}{歌剧}{14,10}{⽋、⼑}
  \begin{Phonetics}{歌剧}{ge1ju4}[][HSK 7-9]
    \definition[场,出]{s.}{ópera | ópera ocidental; um drama que integra poesia, música, dança e outras artes, tendo o canto como principal característica}
  \end{Phonetics}
\end{Entry}

\begin{Entry}{歌颂}{14,10}{⽋、⾴}
  \begin{Phonetics}{歌颂}{ge1song4}[][HSK 7-9]
    \definition{v.}{cantar louvores de; exaltar; elogiar; elogio com poesia, geralmente se refere a elogiar com palavras, etc.}
  \end{Phonetics}
\end{Entry}

\begin{Entry}{歌唱}{14,11}{⽋、⼝}
  \begin{Phonetics}{歌唱}{ge1 chang4}[][HSK 6]
    \definition{v.}{cantar | cantar em louvor de; louvor através de cânticos, recitações, etc.}
  \end{Phonetics}
\end{Entry}

\begin{Entry}{歌舞}{14,14}{⽋、⾇}
  \begin{Phonetics}{歌舞}{ge1wu3}[][HSK 7-9]
    \definition{s.}{canto e dança}
  \end{Phonetics}
\end{Entry}

\begin{Entry}{滴}{14}{⽔}
  \begin{Phonetics}{滴}{di1}[][HSK 6]
    \definition{clas.}{gota; quantificador para "gotejamento"}
    \definition{s.}{uma gota}
    \definition{v.}{pingar}
  \end{Phonetics}
\end{Entry}

\begin{Entry}{漂}{14}{⽔}
  \begin{Phonetics}{漂}{piao1}
    \definition{v.}{flutuar | estar a deriva}
  \end{Phonetics}
  \begin{Phonetics}{漂}{piao3}
    \definition{v.}{alvejar | branquear}
  \end{Phonetics}
  \begin{Phonetics}{漂}{piao4}
    \definition{adj.}{bonita; usado em 漂亮}
    \definition{v.}{falhar; terminar em fracasso}[这笔投资的钱全都漂了。===Todo o dinheiro desse investimento foi perdido.]
  \seealsoref{漂亮}{piao4liang5}
  \end{Phonetics}
\end{Entry}

\begin{Entry}{漂亮}{14,9}{⽔、⼇}
  \begin{Phonetics}{漂亮}{piao4liang5}[][HSK 2]
    \definition{adj.}{bonito; lindo; atraente; de boa aparência; esteticamente agradável | excelente; notável | não pode ser utilizado para descrever homens}
  \end{Phonetics}
\end{Entry}

\begin{Entry}{漂流}{14,10}{⽔、⽔}
  \begin{Phonetics}{漂流}{piao1liu2}
    \definition{s.}{\emph{rafting}}
    \definition{v.}{ser levado pela correnteza | flutuar ao longo ou sobre}
  \end{Phonetics}
\end{Entry}

\begin{Entry}{漏}{14}{⽔}
  \begin{Phonetics}{漏}{lou4}[][HSK 5]
    \definition{s.}{relógio de água; ampulheta | falha; ponto fraco | gonorreia; a medicina tradicional chinesa refere-se a certas doenças que causam secreção de pus, sangue e muco | unidade de tempo medida por um relógio de água durante a noite}
    \definition{v.}{(líquido, gás, etc.) pingar; vazar; escorrer; cair (de um buraco ou fenda) | vazar; deixar escapar; divulgar | perder; deixar de fora por engano | vazar; o objeto tem poros e pode vazar coisas | há uma fuga de ar}
  \end{Phonetics}
\end{Entry}

\begin{Entry}{漏电}{14,5}{⽔、⽥}
  \begin{Phonetics}{漏电}{lou4dian4}
    \definition{v.}{vazar eletricidade}
  \end{Phonetics}
\end{Entry}

\begin{Entry}{漏洞}{14,9}{⽔、⽔}
  \begin{Phonetics}{漏洞}{lou4 dong4}[][HSK 5]
    \definition[个,点]{s.}{vazamento; rachadura; lacunas ou buracos desnecessários que permitem que coisas vazem | falha; defeito; lacuna; (fala, ação, método, etc.) imperfeições}
  \end{Phonetics}
\end{Entry}

\begin{Entry}{演}{14}{⽔}
  \begin{Phonetics}{演}{yan3}[][HSK 3]
    \definition{v.}{desenvolver; evoluir | deduzir; elaborar | exercitar; praticar | representar; atuar; encenar | desempenhar}
  \end{Phonetics}
\end{Entry}

\begin{Entry}{演出}{14,5}{⽔、⼐}
  \begin{Phonetics}{演出}{yan3chu1}[][HSK 3]
    \definition[场,次]{s.}{show; concerto; performance}
    \definition{v.}{apresentar; representar; fazer um show; apresentar peças teatrais, danças, artes cênicas, acrobacias, etc. para o público apreciar}
  \end{Phonetics}
\end{Entry}

\begin{Entry}{演讲}{14,6}{⽔、⾔}
  \begin{Phonetics}{演讲}{yan3jiang3}[][HSK 4]
    \definition[场,次]{s.}{palestra; discurso; ato ou a atividade de apresentar ou expressar ideias, opiniões ou informações oralmente em público ou diante de um público}
    \definition{v.}{dar uma palestra; fazer um discurso; informar o público sobre uma determinada área de conhecimento ou opinião sobre um determinado assunto}
  \end{Phonetics}
\end{Entry}

\begin{Entry}{演员}{14,7}{⽔、⼝}
  \begin{Phonetics}{演员}{yan3yuan2}[][HSK 3]
    \definition[个,位,名]{s.}{ator; artista; pessoas que participam de apresentações teatrais, cinematográficas, de dança, de artes cênicas, de acrobacias, etc.}
  \end{Phonetics}
\end{Entry}

\begin{Entry}{演奏}{14,9}{⽔、⼤}
  \begin{Phonetics}{演奏}{yan3zou4}[][HSK 6]
    \definition{v.}{tocar um instrumento musical; fazer uma apresentação instrumental}
  \end{Phonetics}
\end{Entry}

\begin{Entry}{演唱}{14,11}{⽔、⼝}
  \begin{Phonetics}{演唱}{yan3 chang4}[][HSK 3]
    \definition{v.}{cantar em uma performance; apresentar canções, óperas, peças teatrais, etc.}
  \end{Phonetics}
\end{Entry}

\begin{Entry}{演唱会}{14,11,6}{⽔、⼝、⼈}
  \begin{Phonetics}{演唱会}{yan3 chang4 hui4}[][HSK 3]
    \definition[个,场]{s.}{recital vocal; concerto vocal; uma forma de apresentação centrada no canto, acompanhada por movimentos de dança simples}
  \end{Phonetics}
\end{Entry}

\begin{Entry}{漫}{14}{⽔}
  \begin{Phonetics}{漫}{man4}
    \definition{adj.}{livre; irrestrito; casual | longo; extenso | em todos os lugares; por toda parte | aleatório; irrestrito; livre}
    \definition{adv.}{não}
    \definition{v.}{transbordar; inundar | estar em todo lugar}
  \end{Phonetics}
\end{Entry}

\begin{Entry}{漫长}{14,4}{⽔、⾧}
  \begin{Phonetics}{漫长}{man4chang2}[][HSK 5]
    \definition{adj.}{muito longo; interminável; (tempo, espaço) dura muito tempo}
  \end{Phonetics}
\end{Entry}

\begin{Entry}{漫画}{14,8}{⽔、⽥}
  \begin{Phonetics}{漫画}{man4hua4}[][HSK 5]
    \definition[幅,本,张,套]{s.}{desenho animado; caricatura; \emph{cartoon}}
  \end{Phonetics}
\end{Entry}

\begin{Entry}{漫骂}{14,9}{⽔、⾺}
  \begin{Phonetics}{漫骂}{man4ma4}
    \variantof{谩骂}
  \end{Phonetics}
\end{Entry}

\begin{Entry}{煽}{14}{⽕}
  \begin{Phonetics}{煽}{shan1}
    \definition{v.}{abanar (fogo); agitar um leque ou outra folha | incitar; instigar; agitar | vangloriar-se de; esbanjar prêmios em}
  \end{Phonetics}
\end{Entry}

\begin{Entry}{煽动}{14,6}{⽕、⼒}
  \begin{Phonetics}{煽动}{shan1dong4}
    \definition{v.}{instigar; incitar; agitar; atiçar | incitar; açoitar; conduzir; chicotear}
  \end{Phonetics}
\end{Entry}

\begin{Entry}{熊}{14}{⽕}
  \begin{Phonetics}{熊}{xiong2}[][HSK 5]
    \definition*{s.}{Sobrenome Xiong}
    \definition[头,只]{s.}{urso}
    \definition{v.}{repreender; censurar}
  \end{Phonetics}
\end{Entry}

\begin{Entry}{熊猫}{14,11}{⽕、⽝}
  \begin{Phonetics}{熊猫}{xiong2mao1}
    \definition[把,只]{s.}{panda gigante}
  \seealsoref{猫熊}{mao1xiong2}
  \end{Phonetics}
\end{Entry}

\begin{Entry}{熏}{14}{⽕}
  \begin{Phonetics}{熏}{xun1}
    \definition{v.}{expor à fumaça ou vapores; fumigar | tratar (carne, peixe, etc.) com fumaça; defumar | tornar perfumado com incenso, etc. | sufocar (asfixia e envenenamento por gás)}
  \end{Phonetics}
\end{Entry}

\begin{Entry}{熏香}{14,9}{⽕、⾹}
  \begin{Phonetics}{熏香}{xun1xiang1}
    \definition{s.}{incenso}
  \end{Phonetics}
\end{Entry}

\begin{Entry}{熬}{14}{⽕}
  \begin{Phonetics}{熬}{ao1}
    \definition{v.}{ensopar; ferver; cozinhar em água}
  \end{Phonetics}
  \begin{Phonetics}{熬}{ao2}[][HSK 7-9]
    \definition{v.}{ferver; ensopar; fazer uma decocção; cozinhar em fogo baixo por muito tempo | preparar; infundir; extrair a essência por fervura longa | resistir; suportar (angústia, tempos difíceis, etc.)}
  \end{Phonetics}
\end{Entry}

\begin{Entry}{熬夜}{14,8}{⽕、⼣}
  \begin{Phonetics}{熬夜}{ao2/ye4}[][HSK 7-9]
    \definition{v.+compl.}{ficar acordado a noite toda ou até tarde da noite}
  \end{Phonetics}
\end{Entry}

\begin{Entry}{疑}{14}{⽦}
  \begin{Phonetics}{疑}{yi2}
    \definition{adj.}{duvidoso; incerto}
    \definition{v.}{duvidar; desacreditar; suspeitar}
  \end{Phonetics}
\end{Entry}

\begin{Entry}{疑问}{14,6}{⽦、⾨}
  \begin{Phonetics}{疑问}{yi2wen4}[][HSK 4]
    \definition[个,些]{s.}{dúvida; consulta; pergunta; questionamento; coisas que não podem ser determinadas ou explicadas}
  \end{Phonetics}
\end{Entry}

\begin{Entry}{瘦}{14}{⽧}
  \begin{Phonetics}{瘦}{shou4}[][HSK 5]
    \definition{adj.}{magro; esquelético (oposto de 胖, 肥) | magro (oposto de 肥) | apertado (oposto de 肥) | infértil; pobre | esquelético; pouca gordura; pouca carne (em oposição a 或 ou 肥) | (roupas, sapatos, meias, etc.) apertado (em oposição a 肥) |magra; (carne comestível) com baixo teor de gordura (em oposição a 肥)}
    \definition{v.}{perder peso}
  \seealsoref{肥}{fei2}
  \seealsoref{或}{huo4}
  \seealsoref{胖}{pang4}
  \end{Phonetics}
\end{Entry}

\begin{Entry}{瞅}{14}{⽬}
  \begin{Phonetics}{瞅}{chou3}[][HSK 7-9]
    \definition{v.}{Dialeto: olhar para}[让我瞅瞅。===Deixe-me dar uma olhada.]
  \end{Phonetics}
\end{Entry}

\begin{Entry}{碧}{14}{⽯}
  \begin{Phonetics}{碧}{bi4}
    \definition*{s.}{Sobrenome Bi}
    \definition{adj.}{verde claro | azul claro | azul; verde-azulado; esverdeado; azul-celeste. turquesa}
    \definition{s.}{Literário: jade verde | safira}
  \end{Phonetics}
\end{Entry}

\begin{Entry}{碧绿}{14,11}{⽯、⽷}
  \begin{Phonetics}{碧绿}{bi4lv4}[][HSK 7-9]
    \definition{adj.}{verde jade; verde esmeralda; descreve um verde muito brilhante e profundo}
  \end{Phonetics}
\end{Entry}

\begin{Entry}{碳}{14}{⽯}
  \begin{Phonetics}{碳}{tan4}
    \definition{s.}{carbono (elemento químico)}
  \end{Phonetics}
\end{Entry}

\begin{Entry}{碳足迹}{14,7,9}{⽯、⾜、⾡}
  \begin{Phonetics}{碳足迹}{tan4 zu2ji4}
    \definition{s.}{pegada de carbono}
  \end{Phonetics}
\end{Entry}

\begin{Entry}{磁}{14}{⽯}
  \begin{Phonetics}{磁}{ci2}
    \definition[块]{s.}{porcelana | (física) magnetismo; propriedade de atrair ferro, níquel, etc. | (dialeto)  (de relação) próximo; íntimo}
  \end{Phonetics}
\end{Entry}

\begin{Entry}{磁卡}{14,5}{⽯、⼘}
  \begin{Phonetics}{磁卡}{ci2ka3}[][HSK 7-9]
    \definition[张]{s.}{cartão magnético}
  \end{Phonetics}
\end{Entry}

\begin{Entry}{磁带}{14,9}{⽯、⼱}
  \begin{Phonetics}{磁带}{ci2dai4}[][HSK 7-9]
    \definition[盘,盒,卷]{s.}{fita; fita magnética; cassete; uma fita plástica tratada com material magnético que pode gravar som ou imagens}
  \end{Phonetics}
\end{Entry}

\begin{Entry}{磁铁}{14,10}{⽯、⾦}
  \begin{Phonetics}{磁铁}{ci2tie3}
    \definition{s.}{imã | magneto}
  \seealsoref{吸铁石}{xi1tie3shi2}
  \end{Phonetics}
\end{Entry}

\begin{Entry}{磁盘}{14,11}{⽯、⽫}
  \begin{Phonetics}{磁盘}{ci2pan2}[][HSK 7-9]
    \definition{s.}{Computação: disco; disquete; um disco é um dispositivo de armazenamento que usa tecnologia de gravação magnética para armazenar dados}
  \end{Phonetics}
\end{Entry}

\begin{Entry}{磋}{14}{⽯}
  \begin{Phonetics}{磋}{cuo1}
    \definition{v.}{moer e polir marfim (significado original) | Figurativo: consultar; trocar opiniões | moer; polir}
  \end{Phonetics}
\end{Entry}

\begin{Entry}{磋商}{14,11}{⽯、⼝}
  \begin{Phonetics}{磋商}{cuo1shang1}[][HSK 7-9]
    \definition{v.}{consultar; negociar; trocar pontos de vista; discutir repetidamente; discutir cuidadosamente}
  \end{Phonetics}
\end{Entry}

\begin{Entry}{稳}{14}{⽲}
  \begin{Phonetics}{稳}{wen3}[][HSK 4]
    \definition{adj.}{constante; estável; firme | estável; estático; sedado | seguro; confiável; certo}
    \definition{adv.}{certamente; com certeza; seguramente; sem dúvida}
    \definition{v.}{estabilizar, manter estável; acalmar}
  \end{Phonetics}
\end{Entry}

\begin{Entry}{稳定}{14,8}{⽲、⼧}
  \begin{Phonetics}{稳定}{wen3ding4}[][HSK 4]
    \definition{adj.}{estável; firme; descreve uma natureza, um estado, etc. relativamente fixo; não muda significativamente}
    \definition{v.}{manter estável; estabilizar}
  \end{Phonetics}
\end{Entry}

\begin{Entry}{端}{14}{⽴}
  \begin{Phonetics}{端}{duan1}[][HSK 6]
    \definition*{s.}{Sobrenome Duan}
    \definition{adj.}{adequado; próprio | reto; correto}
    \definition{s.}{fim; extremidade | começo | item; ponto; pista, projeto ou aspecto | causa; razão | problema; incidente; coisas (geralmente se refere a coisas ruins, como acidentes, disputas, etc.)}
    \definition{v.}{carregar; segurar algo nivelado com ambas as mãos; segurar algo horizontalmente | erradicar; eliminar; acabar com; remover completamente; varrer | dar ares de superioridade | revelar}
  \end{Phonetics}
\end{Entry}

\begin{Entry}{端午节}{14,4,5}{⽴、⼗、⾋}
  \begin{Phonetics}{端午节}{duan1wu3jie2}[][HSK 6]
    \definition*[个]{s.}{Festa do Duplo Cinco, Festival dos Barcos-Dragão (5º~dia do quinto mês lunar)}
  \end{Phonetics}
\end{Entry}

\begin{Entry}{端正}{14,5}{⽴、⽌}
  \begin{Phonetics}{端正}{duan1zheng4}[][HSK 7-9]
    \definition{adj.}{apropriado; correto; não torto ou inclinado | ereto; integridade; decência}
    \definition{v.}{corrigir; fazer o certo}
  \end{Phonetics}
\end{Entry}

\begin{Entry}{算}{14}{⽵}
  \begin{Phonetics}{算}{suan4}[][HSK 2]
    \definition{adv.}{finalmente; por fim; no final; significa que, após um longo período de tempo ou muitas dificuldades, finalmente se alcançou o objetivo, equivalente a 总算}
    \definition{v.}{calcular; estimar; computar | contar; incluir | planejar; calcular; projetar | pensar; supor; especular | considerar; considerar como; contar como; reconhecer como | (aritmética) contar; ter peso | deixe estar; deixe passar; seguido por 了: desistir, não se importar mais}
  \seealsoref{了}{le5}
  \seealsoref{总算}{zong3suan4}
  \end{Phonetics}
\end{Entry}

\begin{Entry}{算了}{14,2}{⽵、⼅}
  \begin{Phonetics}{算了}{suan4 le5}[][HSK 6]
    \definition{part.}{deixe estar; deixe passar; usado no final de uma frase para expressar imperativo, término, etc.}
    \definition{v.}{deixar;  deixe estar; deixe passar; esquecer isso; não querer continuar; é usado para persuadir os outros ou para expressar que posso aceitar a situação atual, para encerrar o assunto ou assunto atual, ou para dizer "esqueça"}
  \end{Phonetics}
\end{Entry}

\begin{Entry}{算命}{14,8}{⽵、⼝}
  \begin{Phonetics}{算命}{suan4ming4}
    \definition{s.}{cartomante}
    \definition{v.}{ler a sorte | fazer advinhações}
  \end{Phonetics}
\end{Entry}

\begin{Entry}{算是}{14,9}{⽵、⽇}
  \begin{Phonetics}{算是}{suan4 shi4}[][HSK 6]
    \definition{adv.}{finalmente; por fim; depois de muito tempo, o objetivo foi finalmente alcançado}
    \definition{v.}{contar como; pensar que; ser considerado}
  \end{Phonetics}
\end{Entry}

\begin{Entry}{管}{14}{⽵}
  \begin{Phonetics}{管}{guan3}[][HSK 3]
    \definition*{s.}{Guan, um estado da dinastia Zhou | Sobrenome Guan}
    \definition{adj.}{estreito; restrito; limitado; pequeno}
    \definition{clas.}{usado para objetos cilíndricos longos e finos}
    \definition{conj.}{não importa (quem, o quê, como, etc.)}
    \definition{prep.}{função semelhante a 把, usada especificamente em conjunto com 叫}
    \definition[根,条,排]{s.}{cano; tubo | instrumento musical de sopro | válvula; tubo | duto; canal; vasos}
    \definition{v.}{administrar; dirigir; controlar; cuidar; ser responsável por | ter jurisdição sobre; administrar | disciplinar (crianças ou alunos) | preocupar-se com; importar-se com; incomodar-se com; intervir | fornecer; garantir | supervisionar | governar | submeter alguém a disciplina | assumir; arcar com | incomodar; interferir | assegurar; garantir}
  \seealsoref{把}{ba3}
  \seealsoref{叫}{jiao4}
  \end{Phonetics}
\end{Entry}

\begin{Entry}{管子}{14,3}{⽵、⼦}
  \begin{Phonetics}{管子}{guan3zi5}[][HSK 7-9]
    \definition*{s.}{Guanzi ou Guan Zhong 管仲 (-645 a.C.), famoso político de Qi (齐国) do período da Primavera e do Outono | Guanzi, livro clássico contendo escritos de Guan Zhong e sua escola}
  \seealsoref{管仲}{guan3 zhong4}
  \seealsoref{齐国}{qi2 guo2}
  \end{Phonetics}
\end{Entry}

\begin{Entry}{管……叫……}{14,5}{⽵、⼝}
  \begin{Phonetics}{管……叫……}{guan3 jiao4}
    \definition{expr.}{chamar alguém (ou algo) de alguém (ou algo)}
  \end{Phonetics}
\end{Entry}

\begin{Entry}{管用}{14,5}{⽵、⽤}
  \begin{Phonetics}{管用}{guan3yong4}[][HSK 7-9]
    \definition{adj.}{eficaz; funcional}
  \end{Phonetics}
\end{Entry}

\begin{Entry}{管仲}{14,6}{⽵、⼈}
  \begin{Phonetics}{管仲}{guan3 zhong4}
    \definition*{s.}{1. Guan Zhong (-645 aC), famoso político do Qi (齐国) do período da Primavera e Outono |}
    \definition*{s.}{uma visão restrita através de um tubo de bambu | conhecido como tubo de Guangzi 管子}
  \seealsoref{管子}{guan3zi5}
  \seealsoref{齐国}{qi2 guo2}
  \end{Phonetics}
\end{Entry}

\begin{Entry}{管家}{14,10}{⽵、⼧}
  \begin{Phonetics}{管家}{guan3jia1}[][HSK 7-9]
    \definition[个]{s.}{mordomo; antigamente, referia-se a alguém que administrava os negócios de uma família rica | governanta; alguém que gerencia as tarefas domésticas | gerente; governanta; uma pessoa que administra bens ou negócios familiares ou coletivos}
    \definition{v.}{administrar uma casa}
  \end{Phonetics}
\end{Entry}

\begin{Entry}{管教}{14,11}{⽵、⽁}
  \begin{Phonetics}{管教}{guan3jiao4}[][HSK 7-9]
    \definition{adv.}{Dialeto: certamente; seguramente}
    \definition{v.}{corrigir; disciplinar alguém júnior | responsabilizar-se por | ensinar}
  \end{Phonetics}
\end{Entry}

\begin{Entry}{管理}{14,11}{⽵、⽟}
  \begin{Phonetics}{管理}{guan3li3}[][HSK 3]
    \definition{v.}{gerenciar; executar; administrar; governar; estar encarregado de; responsável por garantir o bom andamento de uma determinada tarefa | controlar; gerenciar; fazer com que pessoas e animais obedeçam ou se comportem de maneira ordeira | cuidar; zelar por; proteger; cuidar, organizar coisas}
  \end{Phonetics}
\end{Entry}

\begin{Entry}{管理费}{14,11,9}{⽵、⽟、⾙}
  \begin{Phonetics}{管理费}{guan3li3fei4}[][HSK 7-9]
    \definition{s.}{despesas de gestão; custos de administração | taxa de administração}
  \end{Phonetics}
\end{Entry}

\begin{Entry}{管道}{14,12}{⽵、⾡}
  \begin{Phonetics}{管道}{guan3 dao4}[][HSK 6]
    \definition[根,千米,公里]{s.}{oleoduto; canal; túnel; tubulação; um tubo feito de metal ou outro material usado para transportar ou descarregar fluidos (como vapor, gás, óleo, água, etc.) | caminho; canal; abordagem}
  \end{Phonetics}
\end{Entry}

\begin{Entry}{管辖}{14,14}{⽵、⾞}
  \begin{Phonetics}{管辖}{guan3xia2}[][HSK 7-9]
    \definition{v.}{gerenciar; governar (pessoal, assuntos, áreas, casos, etc.)}
  \end{Phonetics}
\end{Entry}

\begin{Entry}{精}{14}{⽶}
  \begin{Phonetics}{精}{jing1}[][HSK 6]
    \definition{adv.}{muito; extremamente; antes de certos adjetivos, significa 十分 ou 非常}
    \definition{s.}{refinado; escolhido; escolha; purificado ou selecionado | perfeito; excelente; melhor | fino (em oposição a 粗); preciso; meticuloso | inteligente; astuto; esperto | habilidoso; versado; proficiente | extrato; essência; essência refinada ou selecionada; extraída | energia; espírito | semente; esperma; sêmen | \emph{goblin}; espírito; elfo; demônio}
  \seealsoref{粗}{cu1}
  \seealsoref{非常}{fei1chang2}
  \seealsoref{十分}{shi2fen1}
  \end{Phonetics}
\end{Entry}

\begin{Entry}{精力}{14,2}{⽶、⼒}
  \begin{Phonetics}{精力}{jing1li4}[][HSK 4]
    \definition[些]{s.}{energia; vigor; força mental e física}
  \end{Phonetics}
\end{Entry}

\begin{Entry}{精灵}{14,7}{⽶、⽕}
  \begin{Phonetics}{精灵}{jing1ling2}
    \definition{s.}{espírito | fada | elfo | duende | gênio}
  \end{Phonetics}
\end{Entry}

\begin{Entry}{精品}{14,9}{⽶、⼝}
  \begin{Phonetics}{精品}{jing1pin3}[][HSK 6]
    \definition[个]{s.}{belas obras (de arte); objetos de arte | produtos de qualidade; artigos de excelente qualidade; produto \emph{premium}}
  \end{Phonetics}
\end{Entry}

\begin{Entry}{精神}{14,9}{⽶、⽰}
  \begin{Phonetics}{精神}{jing1shen2}[][HSK 3]
    \definition[种,个,类,股]{s.}{espírito; mente; estado mental; refere-se à consciência, às atividades mentais e ao estado psicológico geral de uma pessoa | substância; espírito; essência; propósito; significado principal}
  \end{Phonetics}
  \begin{Phonetics}{精神}{jing1shen5}[][HSK 3]
    \definition{adj.}{animado; espirituoso; vigoroso; descreve uma pessoa como cheia de energia | muito bonito; boa aparência, bom físico}
    \definition[种,个,类,股]{s.}{impulso; vigor; vitalidade}
  \end{Phonetics}
\end{Entry}

\begin{Entry}{精美}{14,9}{⽶、⽺}
  \begin{Phonetics}{精美}{jing1 mei3}[][HSK 6]
    \definition{adj.}{elegante; requintado}
  \end{Phonetics}
\end{Entry}

\begin{Entry}{精致}{14,10}{⽶、⾄}
  \begin{Phonetics}{精致}{jing1zhi4}
    \definition{adj.}{delicado | exótico | refinado}
  \end{Phonetics}
\end{Entry}

\begin{Entry}{精彩}{14,11}{⽶、⼺}
  \begin{Phonetics}{精彩}{jing1cai3}[][HSK 3]
    \definition{adj.}{brilhante; esplêndido; maravilhoso}
  \end{Phonetics}
\end{Entry}

\begin{Entry}{缩}{14}{⽷}
  \begin{Phonetics}{缩}{suo1}
    \definition*{s.}{Sobrenome Suo}
    \definition{v.}{contrair; encolher | recuar; retirar-se | economizar}
  \end{Phonetics}
\end{Entry}

\begin{Entry}{缩小}{14,3}{⽷、⼩}
  \begin{Phonetics}{缩小}{suo1 xiao3}[][HSK 4]
    \definition{v.}{reduzir, estreitar, encolher;  tornar menor (em oposição a 放大)}
  \seealsoref{放大}{fang4da4}
  \end{Phonetics}
\end{Entry}

\begin{Entry}{缩短}{14,12}{⽷、⽮}
  \begin{Phonetics}{缩短}{suo1duan3}[][HSK 4]
    \definition{v.}{encurtar; reduzir; diminuir}
  \end{Phonetics}
\end{Entry}

\begin{Entry}{缩影卡片}{14,15,5,4}{⽷、⼺、⼘、⽚}
  \begin{Phonetics}{缩影卡片}{suo1ying3 ka3pian4}
    \definition{s.}{cartão em miniatura}
  \end{Phonetics}
\end{Entry}

\begin{Entry}{翠}{14}{⽻}
  \begin{Phonetics}{翠}{cui4}
    \definition{adj.}{verde; verde esmeralda}
    \definition{s.}{martim-pescador | jadeíte; jade}
  \end{Phonetics}
\end{Entry}

\begin{Entry}{翠绿}{14,11}{⽻、⽷}
  \begin{Phonetics}{翠绿}{cui4lv4}[][HSK 7-9]
    \definition{adj.}{verde esmeralda; verde jade}
  \end{Phonetics}
\end{Entry}

\begin{Entry}{聚}{14}{⽿}
  \begin{Phonetics}{聚}{ju4}[][HSK 4]
    \definition*{s.}{Sobrenome Ju}
    \definition{v.}{reunir-se; juntar-se}
  \end{Phonetics}
\end{Entry}

\begin{Entry}{聚会}{14,6}{⽿、⼈}
  \begin{Phonetics}{聚会}{ju4hui4}[][HSK 4]
    \definition[个,次]{s.}{reunião; encontro; confraternização; festa}
    \definition{v.}{encontrar-se; reunir-se}
  \end{Phonetics}
\end{Entry}

\begin{Entry}{聚散}{14,12}{⽿、⽁}
  \begin{Phonetics}{聚散}{ju4san4}
    \definition{s.}{juntos e separados | agregação e dissipação}
  \end{Phonetics}
\end{Entry}

\begin{Entry}{腐}{14}{⾁}
  \begin{Phonetics}{腐}{fu3}
    \definition{adj.}{podre; obsoleto; corrupto | corroído; pútrido}
    \definition{s.}{tofu}
    \definition{v.}{apodrecer; corroer; estragar; decair}
  \end{Phonetics}
\end{Entry}

\begin{Entry}{腐化}{14,4}{⾁、⼔}
  \begin{Phonetics}{腐化}{fu3hua4}[][HSK 7-9]
    \definition{adj.}{degenerado; corrupto, dissoluto ou depravado; desmoralizado; decadente}
    \definition{v.}{decompor; apodrecer; tornar-se pútrido | quebrar; corroer}
  \end{Phonetics}
\end{Entry}

\begin{Entry}{腐朽}{14,6}{⾁、⽊}
  \begin{Phonetics}{腐朽}{fu3xiu3}[][HSK 7-9]
    \definition{adj.}{decaído; decadente; degenerado; uma metáfora para as ideias ultrapassadas das pessoas ou para a moral social corrupta}
    \definition{v.}{apodrecer; decair; apodrecimento e deterioração da madeira e outros materiais fibrosos}
  \end{Phonetics}
\end{Entry}

\begin{Entry}{腐败}{14,8}{⾁、⾒}
  \begin{Phonetics}{腐败}{fu3bai4}[][HSK 7-9]
    \definition{adj.}{(ideias) corrupto; decadente; (pensamento) obsoleto; (comportamento) degenerado | (sistema, organização, instituição, medida, etc.) corrupto}
    \definition{s.}{deterioração; podridão}
    \definition{v.}{apodrecer; decair}
  \end{Phonetics}
\end{Entry}

\begin{Entry}{腐烂}{14,9}{⾁、⽕}
  \begin{Phonetics}{腐烂}{fu3lan4}[][HSK 7-9]
    \definition{adj.}{corrupto; depravado | (pensamentos) obsoletos; (comportamento) degenerado}
    \definition{v.}{apodrecer; decompor; tornar-se pútrido}
  \end{Phonetics}
\end{Entry}

\begin{Entry}{腐蚀}{14,9}{⾁、⾷}
  \begin{Phonetics}{腐蚀}{fu3shi2}[][HSK 7-9]
    \definition{v.}{corroer; destruir gradualmente um objeto por meio de reações químicas | corroer; corromper (pensamentos e comportamentos)}
  \end{Phonetics}
\end{Entry}

\begin{Entry}{膜}{14}{⾁}
  \begin{Phonetics}{膜}{mo2}[][HSK 6]
    \definition[张]{s.}{membrana | filme; revestimento fino}
  \end{Phonetics}
\end{Entry}

\begin{Entry}{膜拜}{14,9}{⾁、⼿}
  \begin{Phonetics}{膜拜}{mo2bai4}
    \definition{v.}{ajoelhar-se e se curvar com as mãos unidas no nível da testa | ter ou mostrar sentimentos fortes de respeito e admiração por um deus}
  \end{Phonetics}
\end{Entry}

\begin{Entry}{舞}{14}{⾇}
  \begin{Phonetics}{舞}{wu3}[][HSK 5]
    \definition[支,段,个]{s.}{dança | palco; metáfora do domínio das atividades sociais}
    \definition{v.}{mover-se como numa dança | dançar com algo nas mãos; brincar com | florescer; empunhar; brandir | esvoaçar | fazer malabarismos; brincar com}
  \end{Phonetics}
\end{Entry}

\begin{Entry}{舞厅}{14,4}{⾇、⼚}
  \begin{Phonetics}{舞厅}{wu3ting1}
    \definition[间]{s.}{salão de dança | salão de baile}
  \end{Phonetics}
\end{Entry}

\begin{Entry}{舞厅舞}{14,4,14}{⾇、⼚、⾇}
  \begin{Phonetics}{舞厅舞}{wu3ting1wu3}
    \definition{s.}{dança de salão}
  \end{Phonetics}
\end{Entry}

\begin{Entry}{舞台}{14,5}{⾇、⼝}
  \begin{Phonetics}{舞台}{wu3 tai2}[][HSK 3]
    \definition[个]{s.}{palco; plataforma elevada usada exclusivamente para apresentações artísticas, geralmente localizada na parte frontal de teatros e auditórios | palco; metáfora do campo das atividades sociais}
  \end{Phonetics}
\end{Entry}

\begin{Entry}{舞会}{14,6}{⾇、⼈}
  \begin{Phonetics}{舞会}{wu3hui4}
    \definition{s.}{baile}
  \end{Phonetics}
\end{Entry}

\begin{Entry}{舞会舞}{14,6,14}{⾇、⼈、⾇}
  \begin{Phonetics}{舞会舞}{wu3hui4wu3}
    \definition{s.}{baile}
  \end{Phonetics}
\end{Entry}

\begin{Entry}{舞抃}{14,7}{⾇、⼿}
  \begin{Phonetics}{舞抃}{wu3bian4}
    \definition{s.}{dançar por prazer}
  \end{Phonetics}
\end{Entry}

\begin{Entry}{舞蹈}{14,17}{⾇、⾜}
  \begin{Phonetics}{舞蹈}{wu3dao3}[][HSK 6]
    \definition[段,支,场,个]{s.}{dança; uma forma de arte que usa movimentos rítmicos como principal meio de expressão, podendo expressar a vida, os pensamentos e os sentimentos das pessoas, geralmente acompanhada de música}
    \definition{v.}{dançar}
  \end{Phonetics}
\end{Entry}

\begin{Entry}{蔓}{14}{⾋}
  \begin{Phonetics}{蔓}{man2}
    \definition{s.}{couve-chinesa | nabo}
  \end{Phonetics}
  \begin{Phonetics}{蔓}{man4}
    \definition{s.}{uma videira com gavinhas; caule fino que não consegue ficar em pé}
    \definition{v.}{rastejar; espalhar; estender}
  \end{Phonetics}
  \begin{Phonetics}{蔓}{wan4}
    \definition*{s.}{Sobrenome Wan}
    \definition{s.}{uma videira com gavinhas; caule fino que não consegue ficar em pé}
  \end{Phonetics}
\end{Entry}

\begin{Entry}{蔓草}{14,9}{⾋、⾋}
  \begin{Phonetics}{蔓草}{man4cao3}
    \definition{s.}{videira | trepadeira}
  \end{Phonetics}
\end{Entry}

\begin{Entry}{蜘}{14}{⾍}
  \begin{Phonetics}{蜘}{zhi1}
    \definition[只]{s.}{aranha}
  \seealsoref{蜘蛛}{zhi1zhu1}
  \end{Phonetics}
\end{Entry}

\begin{Entry}{蜘蛛}{14,12}{⾍、⾍}
  \begin{Phonetics}{蜘蛛}{zhi1zhu1}
    \definition{s.}{aranha}
  \end{Phonetics}
\end{Entry}

\begin{Entry}{蜘蛛网}{14,12,6}{⾍、⾍、⽹}
  \begin{Phonetics}{蜘蛛网}{zhi1zhu1 wang3}
    \definition{s.}{teia de aranha}
  \end{Phonetics}
\end{Entry}

\begin{Entry}{蜜}{14}{⾍}
  \begin{Phonetics}{蜜}{mi4}
    \definition{adj.}{melado; doce}
    \definition{s.}{mel | semelhante ao mel | coisas parecidas com mel; melaço}
  \end{Phonetics}
\end{Entry}

\begin{Entry}{蜜桃}{14,10}{⾍、⽊}
  \begin{Phonetics}{蜜桃}{mi4tao2}
    \definition{s.}{pêssego suculento}
  \end{Phonetics}
\end{Entry}

\begin{Entry}{蜡}{14}{⾍}
  \begin{Phonetics}{蜡}{la4}
    \definition{s.}{cera; óleos produzidos por animais, minerais ou plantas | vela}
  \end{Phonetics}
\end{Entry}

\begin{Entry}{蜡烛}{14,10}{⾍、⽕}
  \begin{Phonetics}{蜡烛}{la4zhu2}
    \definition[根,支]{s.}{vela | círio | peça, geralmente de cera, que possui um pavio e se utiliza para iluminar}
  \end{Phonetics}
\end{Entry}

\begin{Entry}{蜥}{14}{⾍}
  \begin{Phonetics}{蜥}{xi1}
    \definition{s.}{lagarto}
  \end{Phonetics}
\end{Entry}

\begin{Entry}{蜥易}{14,8}{⾍、⽇}
  \begin{Phonetics}{蜥易}{xi1yi4}
    \variantof{蜥蜴}
  \end{Phonetics}
\end{Entry}

\begin{Entry}{蜥蜴}{14,14}{⾍、⾍}
  \begin{Phonetics}{蜥蜴}{xi1yi4}
    \definition{s.}{lagarto}
  \end{Phonetics}
\end{Entry}

\begin{Entry}{蜻}{14}{⾍}
  \begin{Phonetics}{蜻}{qing1}
    \definition[只]{s.}{libélula, 蜻蜓}
  \seealsoref{蜻蜓}{qing1ting2}
  \end{Phonetics}
\end{Entry}

\begin{Entry}{蜻蜓}{14,12}{⾍、⾍}
  \begin{Phonetics}{蜻蜓}{qing1ting2}
    \definition{s.}{libélula}
  \end{Phonetics}
\end{Entry}

\begin{Entry}{蜻蝏}{14,15}{⾍、⾍}
  \begin{Phonetics}{蜻蝏}{qing1ting2}
    \variantof{蜻蜓}
  \end{Phonetics}
\end{Entry}

\begin{Entry}{蝉}{14}{⾍}
  \begin{Phonetics}{蝉}{chan2}
    \definition[只,个]{s.}{cigarra}
  \seealsoref{知了}{zhi1liao3}
  \end{Phonetics}
\end{Entry}

\begin{Entry}{裹}{14}{⾐}
  \begin{Phonetics}{裹}{guo3}[][HSK 7-9]
    \definition{s.}{pacote; encomenda}
    \definition{v.}{amarrar; embrulhar; envolver | levar embora; varrer com violência | Dialeto: sugar (leite) | pressionar a servir; fugir com (algo)}
  \end{Phonetics}
\end{Entry}

\begin{Entry}{褐}{14}{⾐}
  \begin{Phonetics}{褐}{he4}
    \definition{adj.}{marrom; castanho; pardo}
    \definition{s.}{pano de cânhamo grosso}
  \end{Phonetics}
\end{Entry}

\begin{Entry}{褐色}{14,6}{⾐、⾊}
  \begin{Phonetics}{褐色}{he4 se4}
    \definition{s.}{cor marrom}
  \end{Phonetics}
\end{Entry}

\begin{Entry}{褡}{14}{⾐}
  \begin{Phonetics}{褡}{da1}
    \definition{s.}{bolsa; malote; algibeira | jaqueta sem mangas}
  \end{Phonetics}
\end{Entry}

\begin{Entry}{豪}{14}{⾗}
  \begin{Phonetics}{豪}{hao2}
    \definition*{s.}{Sobrenome Hao}
    \definition{adj.}{direto; irrestrito; ousado | despótico; intimidador | rico e poderoso}
    \definition{s.}{pessoa com poderes ou dons extraordinários}
  \end{Phonetics}
\end{Entry}

\begin{Entry}{豪华}{14,6}{⾗、⼗}
  \begin{Phonetics}{豪华}{hao2hua2}[][HSK 7-9]
    \definition{adj.}{luxo; luxuoso; (edifício, equipamento ou decoração) magnífico; muito lindo}
  \end{Phonetics}
\end{Entry}

\begin{Entry}{赚}{14}{⾙}
  \begin{Phonetics}{赚}{zhuan4}[][HSK 6]
    \definition{s.}{lucro}
    \definition{v.}{ganhar (dinheiro); obter lucro com o negócio (em oposição a 赔)}
  \seealsoref{赔}{pei2}
  \end{Phonetics}
\end{Entry}

\begin{Entry}{赚钱}{14,10}{⾙、⾦}
  \begin{Phonetics}{赚钱}{zhuan4 qian2}[][HSK 6]
    \definition{v.}{ganhar dinheiro; obter lucro ou recompensa}
  \end{Phonetics}
\end{Entry}

\begin{Entry}{赛}{14}{⾙}
  \begin{Phonetics}{赛}{sai4}[][HSK 6]
    \definition*{s.}{Sobrenome Sai}
    \definition{s.}{jogo; partida; competição | sacrifício; cerimônia de sacrifício; antigamente, sacrifícios eram feitos para agradecer aos deuses por suas dádivas}
    \definition{v.}{ter uma competição (comparando alto e baixo, forte e fraco) | superar; ser comparável a; comparar com}
  \end{Phonetics}
\end{Entry}

\begin{Entry}{赛车}{14,4}{⾙、⾞}
  \begin{Phonetics}{赛车}{sai4che1}
    \definition{s.}{corrida de automóvel | corrida de bicicleta | carro de corrida}
  \end{Phonetics}
\end{Entry}

\begin{Entry}{赛场}{14,6}{⾙、⼟}
  \begin{Phonetics}{赛场}{sai4 chang3}[][HSK 6]
    \definition{s.}{local de competição; arena; ringue; terreno | campo (para competição de atletismo) | pista de corrida}
  \end{Phonetics}
\end{Entry}

\begin{Entry}{辗}{14}{⾞}
  \begin{Phonetics}{辗}{zhan3}
    \definition{v.}{(arcaico) virar | (arcaico) rolar para o lado | (arcaico) virar a metade}
  \end{Phonetics}
\end{Entry}

\begin{Entry}{辣}{14}{⾟}
  \begin{Phonetics}{辣}{la4}[][HSK 4]
    \definition{adj.}{apimentado; picante; pungente; quente | cruel; implacável; venenoso; vicioso}
    \definition{v.}{queimar; picar; formigar; ter uma irritação picante (boca, nariz ou olhos)}
  \end{Phonetics}
\end{Entry}

\begin{Entry}{遭}{14}{⾡}
  \begin{Phonetics}{遭}{zao1}
    \definition{clas.}{tempo; vez; ocasião | rodadas}
    \definition{v.}{encontrar-se com (desastre, infortúnio, etc.); sofrer}
  \end{Phonetics}
\end{Entry}

\begin{Entry}{遭到}{14,8}{⾡、⼑}
  \begin{Phonetics}{遭到}{zao1 dao4}[][HSK 6]
    \definition{v.}{sofrer; ser rejeitado; receber crítica; significa sofrer infortúnio ou dano}[我们遭到意外事故。===Nós sofremos um acidente.]
  \end{Phonetics}
\end{Entry}

\begin{Entry}{遭受}{14,8}{⾡、⼜}
  \begin{Phonetics}{遭受}{zao1shou4}[][HSK 6]
    \definition{v.}{sofrer; aguentar; ser submetido a; encontrar ou vivenciar coisas dolorosas que você não quer que aconteçam}
  \end{Phonetics}
\end{Entry}

\begin{Entry}{遭遇}{14,12}{⾡、⾡}
  \begin{Phonetics}{遭遇}{zao1yu4}[][HSK 6]
    \definition[场,次,种,段]{s.}{sorte (difícil); experiência (amarga); encontrando coisas ruins}
    \definition{v.}{encontrar; encontrar-se com; esbarrar em; encontros inesperados com pessoas ou coisas que não são boas para você}
  \end{Phonetics}
\end{Entry}

\begin{Entry}{酷}{14}{⾣}
  \begin{Phonetics}{酷}{ku4}[][HSK 6]
    \definition{adj.}{cruel; opressivo | feroz; escaldante | brutal | \emph{cool} (empréstimo linguístico); legal; excelente; moderno; ótimo | elegante e sóbrio; gracioso e severo}
    \definition{adv.}{muito; extremamente}
  \end{Phonetics}
\end{Entry}

\begin{Entry}{酷斯拉}{14,12,8}{⾣、⽄、⼿}
  \begin{Phonetics}{酷斯拉}{ku4si1la1}
    \definition*{s.}{Godzilla. do Japonês Gojira, ゴジラ}
  \seealsoref{哥斯拉}{ge1si1la1}
  \end{Phonetics}
\end{Entry}

\begin{Entry}{酸}{14}{⾣}
  \begin{Phonetics}{酸}{suan1}[][HSK 4]
    \definition{adj.}{azedo; ácido | aflito; angustiado; doente do coração | pedante; descreve uma pessoa que finge ser culta e também descreve uma pessoa que é muito inflexível com suas próprias ideias e não está disposta a mudá-las para atender às exigências da época, é usado principalmente para satirizar intelectuais que fingem ser capazes de escrever poemas e artigos | ciumento; invejoso; sentimentos desconfortáveis porque outra pessoa é melhor do que você e, em geral, também apresenta comportamento hostil}
    \definition{s.}{ácido; produto químico que tem um sabor ácido quando misturado com água}
    \definition{v.}{estar dolorido (devido à fadiga ou doença); descreve a sensação de não ter força muscular e um pouco de dor por estar doente ou muito cansado}
  \end{Phonetics}
\end{Entry}

\begin{Entry}{酸奶}{14,5}{⾣、⼥}
  \begin{Phonetics}{酸奶}{suan1 nai3}[][HSK 4]
    \definition[瓶,杯,盒,袋]{s.}{iogurte; produto lácteo fermentado por bactérias de ácido láctico}
  \end{Phonetics}
\end{Entry}

\begin{Entry}{酸甜苦辣}{14,11,8,14}{⾣、⽢、⾋、⾟}
  \begin{Phonetics}{酸甜苦辣}{suan1 tian2 ku3 la4}[][HSK 5]
    \definition{expr.}{os altos e baixos da vida; as experiências agridoces da vida; os aspectos doces, azedos, amargos e picantes da vida; refere-se a todos os tipos de sabores, como metáfora para experiências diversas, como felicidade, sofrimento, etc. | azedo, doce, amargo, picante --- alegrias e tristezas da vida}
  \end{Phonetics}
\end{Entry}

\begin{Entry}{酸辣汤}{14,14,6}{⾣、⾟、⽔}
  \begin{Phonetics}{酸辣汤}{suan1la4tang1}
    \definition{s.}{sopa avinagrada e picante (prato)}
  \end{Phonetics}
\end{Entry}

\begin{Entry}{锺}{14}{⾦}
  \begin{Phonetics}{锺}{zhong1}
    \variantof{钟}
  \end{Phonetics}
\end{Entry}

\begin{Entry}{锻}{14}{⾦}
  \begin{Phonetics}{锻}{duan4}
    \definition{v.}{forjar; moldar}
  \end{Phonetics}
\end{Entry}

\begin{Entry}{锻炼}{14,9}{⾦、⽕}
  \begin{Phonetics}{锻炼}{duan4lian4}[][HSK 4]
    \definition{v.}{exercitar-se; fazer (ou fazer) exercícios; submeter-se a treinamento físico; fortalecer o corpo por meio do esporte | fortalecer; endurecer; aprimorar as habilidades de trabalho e de vida por meio de trabalho e outras atividades | forjar ou moldar metal para torná-lo mais refinado; refere-se à transformação de materiais metálicos em objetos de determinada forma e tamanho por meio de aquecimento, batimento, prensagem etc.}
  \end{Phonetics}
\end{Entry}

\begin{Entry}{镀}{14}{⾦}
  \begin{Phonetics}{镀}{du4}
    \definition{v.}{cobrir ou revestir (com um metal)}
  \end{Phonetics}
\end{Entry}

\begin{Entry}{镀金}{14,8}{⾦、⾦}
  \begin{Phonetics}{镀金}{du4jin1}
    \definition{v.}{banhar a ouro | dourar | (figurativo) fazer algo muito comum parecer especial}
  \end{Phonetics}
\end{Entry}

\begin{Entry}{隧}{14}{⾩}
  \begin{Phonetics}{隧}{sui4}
    \definition{s.}{túnel; passagem subterrânea | estrada | subúrbios; áreas suburbanas}
    \definition{v.}{virar}
  \end{Phonetics}
\end{Entry}

\begin{Entry}{隧道}{14,12}{⾩、⾡}
  \begin{Phonetics}{隧道}{sui4dao4}
    \definition{s.}{túnel}
  \end{Phonetics}
\end{Entry}

\begin{Entry}{需}{14}{⾬}
  \begin{Phonetics}{需}{xu1}
    \definition*{s.}{Sobrenome Xu}
    \definition{s.}{necessidades; bens de primeira necessidade}
    \definition{v.}{precisar; querer; exigir}
  \end{Phonetics}
\end{Entry}

\begin{Entry}{需求}{14,7}{⾬、⽔}
  \begin{Phonetics}{需求}{xu1qiu2}[][HSK 3]
    \definition[种]{s.}{necessidades; demanda; exigência; solicitações decorrentes de necessidades}
  \end{Phonetics}
\end{Entry}

\begin{Entry}{需要}{14,9}{⾬、⾑}
  \begin{Phonetics}{需要}{xu1yao4}[][HSK 3]
    \definition[种]{s.}{necessidade; desejo ou exigência em relação a algo}
    \definition{v.}{precisar; querer; exigir; demandar; solicitar}
  \end{Phonetics}
\end{Entry}

\begin{Entry}{静}{14}{⾭}
  \begin{Phonetics}{静}{jing4}[][HSK 3]
    \definition*{s.}{Sobrenome Jing}
    \definition{adj.}{tranquilo;  sossegado; calmo; imóvel | silencioso; quieto; sem emitir nenhum som | calmo, sereno; serenidade; (interior) paz}
    \definition{v.}{acalmar-se; aquietar-se; tranquilizar (o coração)}
  \end{Phonetics}
\end{Entry}

\begin{Entry}{颗}{14}{⾴}
  \begin{Phonetics}{颗}{ke1}[][HSK 5]
    \definition{clas.}{usado para grãos, pérolas, dentes, corações, satelites, pequenas esferas, etc.}
    \definition{s.}{grão; partícula; pequenas coisas redondas}
  \end{Phonetics}
\end{Entry}

\begin{Entry}{馒}{14}{⾷}
  \begin{Phonetics}{馒}{man2}
    \definition{s.}{pão cozido no vapor}
  \end{Phonetics}
\end{Entry}

\begin{Entry}{馒头}{14,5}{⾷、⼤}
  \begin{Phonetics}{馒头}{man2tou5}[][HSK 6]
    \definition[个,锅,屉,筐]{s.}{pão cozido no vapor; um alimento cozido no vapor feito de farinha fermentada, geralmente redondo na parte superior e plano na parte inferior, sem recheio}
  \end{Phonetics}
\end{Entry}

\begin{Entry}{魅}{14}{⿁}
  \begin{Phonetics}{魅}{mei4}
    \definition{s.}{espírito maligno; demônio | \emph{goblin}; trasgo; gnomo; duende maléfico}
    \definition{v.}{atormentar; cativar}
  \end{Phonetics}
\end{Entry}

\begin{Entry}{魅力}{14,2}{⿁、⼒}
  \begin{Phonetics}{魅力}{mei4li4}
    \definition{s.}{charme | fascínio | glamour | carisma}
  \end{Phonetics}
\end{Entry}

\begin{Entry}{鲜}{14}{⿂}
  \begin{Phonetics}{鲜}{xian1}[][HSK 4]
    \definition*{s.}{Sobrenome Xian}
    \definition{adj.}{fresco; novo; fresco (experiência, comida etc.) |brilhante; de cores vivas | saboroso; delicioso | exuberante; luxuriante}
    \definition{s.}{aves e animais recém-abatidos; vegetais recém-colhidos; frutas, etc. | alimentos aquáticos; geralmente, peixes vivos, camarões, etc., para alimentação}
  \end{Phonetics}
  \begin{Phonetics}{鲜}{xian3}
    \definition{adj.}{raro; pouco; pequeno}
    \definition{adv.}{raramente}
  \end{Phonetics}
\end{Entry}

\begin{Entry}{鲜花}{14,7}{⿂、⾋}
  \begin{Phonetics}{鲜花}{xian1 hua1}[][HSK 4]
    \definition[朵,束,支]{s.}{flor; flores frescas; flores bonitas e frescas}
  \end{Phonetics}
\end{Entry}

\begin{Entry}{鲜明}{14,8}{⿂、⽇}
  \begin{Phonetics}{鲜明}{xian1ming2}[][HSK 4]
    \definition{adj.}{brilhante (cor) | distinto; bem definido; nítido; claro; característico}
  \end{Phonetics}
\end{Entry}

\begin{Entry}{鲜艳}{14,10}{⿂、⾊}
  \begin{Phonetics}{鲜艳}{xian1yan4}[][HSK 5]
    \definition{adj.}{de cores alegres; de cores brilhantes}
  \end{Phonetics}
\end{Entry}

\begin{Entry}{鼻}{14}{⿐}[Kangxi 209]
  \begin{Phonetics}{鼻}{bi2}
    \definition{s.}{nariz}
  \end{Phonetics}
\end{Entry}

\begin{Entry}{鼻子}{14,3}{⿐、⼦}
  \begin{Phonetics}{鼻子}{bi2zi5}[][HSK 5]
    \definition[个,只]{s.}{nariz; órgão da face, responsável pela respiração e pelo olfato}
  \end{Phonetics}
\end{Entry}

\begin{Entry}{鼻涕}{14,10}{⿐、⽔}
  \begin{Phonetics}{鼻涕}{bi2ti4}[][HSK 7-9]
    \definition[些,点]{s.}{ranho; muco nasal; secreção nasal; fluido secretado pela mucosa nasal}
  \end{Phonetics}
\end{Entry}

%%%%% EOF %%%%%


%%%
%%% 15画
%%%
\section*{15画}\addcontentsline{toc}{section}{15画}\addcontentsline{loh}{figure}{\#\#\#\# 15画}

%%%%%%%%%% 僵 %%%%%%%%%%
\subsection*{僵}\addcontentsline{loh}{figure}{僵}

\begin{Entry}{僵}{15}{⼈}
  \begin{Phonetics}{僵}{jiang1}[][HSK 7-9]
    \definition{adj.}{rígido; paralisado; congelado | rígido; austero; duro | Dialeto: em impasse; tenso}
    \definition{v.}{Dialeto: parar de sorrir; ficar com uma expressão séria}
  \end{Phonetics}
\end{Entry}

\begin{Entry}{僵化}{15,4}{⼈、⼔}
  \begin{Phonetics}{僵化}{jiang1hua4}[][HSK 7-9]
    \definition{v.}{tornar-se rígido; ossificar; tornar-se estereotipado; parar de se desenvolver; petrificar; inativar}
  \end{Phonetics}
\end{Entry}

\begin{Entry}{僵局}{15,7}{⼈、⼫}
  \begin{Phonetics}{僵局}{jiang1ju2}[][HSK 7-9]
    \definition[个,种]{s.}{impasse; beco sem saída; a questão é difícil de resolver e o progresso está paralisado}
  \end{Phonetics}
\end{Entry}

%%%%%%%%%% 嘱 %%%%%%%%%%
\subsection*{嘱}\addcontentsline{loh}{figure}{嘱}

\begin{Entry}{嘱}{15}{⼝}
  \begin{Phonetics}{嘱}{zhu3}
    \definition{v.}{juntar-se | implorar | incitar}
  \end{Phonetics}
\end{Entry}

\begin{Entry}{嘱托}{15,6}{⼝、⼿}
  \begin{Phonetics}{嘱托}{zhu3tuo1}
    \definition{v.}{confiar uma tarefa a alguém}
  \end{Phonetics}
\end{Entry}

\begin{Entry}{嘱咐}{15,8}{⼝、⼝}
  \begin{Phonetics}{嘱咐}{zhu3fu5}
    \definition{v.}{ordenar | dizer | exortar}
  \end{Phonetics}
\end{Entry}

%%%%%%%%%% 嘲 %%%%%%%%%%
\subsection*{嘲}\addcontentsline{loh}{figure}{嘲}

\begin{Entry}{嘲}{15}{⼝}
  \begin{Phonetics}{嘲}{chao2}
    \definition{v.}{ridicularizar; zombar; fazer piada de}
  \end{Phonetics}
  \begin{Phonetics}{嘲}{zhao1}
    \definition{s.}{Onomatopéia: barulho clamoroso feito por várias pessoas falando ou cantando, ou por instrumentos musicais, ou pássaros cantando; descreve um som caótico e fragmentado}
  \end{Phonetics}
\end{Entry}

\begin{Entry}{嘲弄}{15,7}{⼝、⼶}
  \begin{Phonetics}{嘲弄}{chao2nong4}[][HSK 7-9]
    \definition{v.}{zombar; zombar de}
  \end{Phonetics}
\end{Entry}

\begin{Entry}{嘲笑}{15,10}{⼝、⽵}
  \begin{Phonetics}{嘲笑}{chao2xiao4}[][HSK 7-9]
    \definition{v.}{ridicularizar; zombar; rir de; zombar de; fazer graça de; usar palavras para zombar de alguém}
  \end{Phonetics}
\end{Entry}

%%%%%%%%%% 嘹 %%%%%%%%%%
\subsection*{嘹}\addcontentsline{loh}{figure}{嘹}

\begin{Entry}{嘹}{15}{⼝}
  \begin{Phonetics}{嘹}{liao2}
    \definition{adj.}{(som) alto e claro | som claro | grito (de guindastes, etc.)}
  \end{Phonetics}
\end{Entry}

\begin{Entry}{嘹亮}{15,9}{⼝、⼇}
  \begin{Phonetics}{嘹亮}{liao2liang4}
    \definition{adj.}{ressonante; alto e claro}
  \end{Phonetics}
\end{Entry}

%%%%%%%%%% 嘿 %%%%%%%%%%
\subsection*{嘿}\addcontentsline{loh}{figure}{嘿}

\begin{Entry}{嘿}{15}{⼝}
  \begin{Phonetics}{嘿}{hei1}[][HSK 7-9]
    \definition{interj.}{Ei!; indicando uma saudação ou chamar a atenção | expressando orgulho ou satisfação | expressando espanto, surpresa}
  \end{Phonetics}
  \begin{Phonetics}{嘿}{mo4}
    \definition{adj.}{quieto; silencioso; tácito}
  \end{Phonetics}
\end{Entry}

%%%%%%%%%% 噎 %%%%%%%%%%
\subsection*{噎}\addcontentsline{loh}{figure}{噎}

\begin{Entry}{噎}{15}{⼝}
  \begin{Phonetics}{噎}{ye1}
    \definition{v.}{engasgar | sufocar}
  \end{Phonetics}
\end{Entry}

%%%%%%%%%% 增 %%%%%%%%%%
\subsection*{增}\addcontentsline{loh}{figure}{增}

\begin{Entry}{增}{15}{⼟}
  \begin{Phonetics}{增}{zeng1}[][HSK 5]
    \definition*{s.}{Sobrenome: Zeng}
    \definition{v.}{aumentar; ganhar; adicionar}
  \end{Phonetics}
\end{Entry}

\begin{Entry}{增大}{15,3}{⼟、⼤}
  \begin{Phonetics}{增大}{zeng1 da4}[][HSK 5]
    \definition{v.}{ampliar; expandir; estender | amplificar}
  \end{Phonetics}
\end{Entry}

\begin{Entry}{增长}{15,4}{⼟、⾧}
  \begin{Phonetics}{增长}{zeng1 zhang3}[][HSK 3]
    \definition{v.}{subir; crescer; aumentar; melhorar a partir da base existente}
  \end{Phonetics}
\end{Entry}

\begin{Entry}{增加}{15,5}{⼟、⼒}
  \begin{Phonetics}{增加}{zeng1jia1}[][HSK 3]
    \definition{v.}{adicionar; aumentar; incrementar; adicionar mais ao que já existe}
  \end{Phonetics}
\end{Entry}

\begin{Entry}{增产}{15,6}{⼟、⼇}
  \begin{Phonetics}{增产}{zeng1/chan3}[][HSK 5]
    \definition{v.+compl.}{aumentar a produção}
  \end{Phonetics}
\end{Entry}

\begin{Entry}{增多}{15,6}{⼟、⼣}
  \begin{Phonetics}{增多}{zeng1 duo1}[][HSK 5]
    \definition{v.}{aumentar; crescer em número ou quantidade}
  \end{Phonetics}
\end{Entry}

\begin{Entry}{增进}{15,7}{⼟、⾡}
  \begin{Phonetics}{增进}{zeng1 jin4}[][HSK 6]
    \definition{v.}{melhorar; promover; aprofundar}
  \end{Phonetics}
\end{Entry}

\begin{Entry}{增值}{15,10}{⼟、⼈}
  \begin{Phonetics}{增值}{zeng1 zhi2}[][HSK 6]
    \definition{s.}{aumento de valor; apreciação; incremento | valor agregado}
  \end{Phonetics}
\end{Entry}

\begin{Entry}{增速}{15,10}{⼟、⾡}
  \begin{Phonetics}{增速}{zeng1su4}
    \definition{s.}{Economia: taxa de crescimento}
    \definition{v.}{acelerar; aumentar a velocidade}
  \end{Phonetics}
\end{Entry}

\begin{Entry}{增强}{15,12}{⼟、⼸}
  \begin{Phonetics}{增强}{zeng1 qiang2}[][HSK 5]
    \definition{v.}{impulsionar; aprimorar; aumentar; fortalecer; tornar mais forte ou mais poderoso}
  \end{Phonetics}
\end{Entry}

%%%%%%%%%% 墨 %%%%%%%%%%
\subsection*{墨}\addcontentsline{loh}{figure}{墨}

\begin{Entry}{墨}{15}{⿊}
  \begin{Phonetics}{墨}{mo4}[][HSK 7-9]
    \definition*{s.}{Escola Moísta; Moísmo | México, abreviação de 墨西哥}
    \definition{adj.}{preto; escuro como breu | corrupto | escuro}
    \definition{s.}{tinta chinesa; bastão de tinta | pigmento; tinta | caligrafia ou pintura | aprendizagem; alfabetização | marcador de linha de carpinteiro; marcador de tinta | tatuar o rosto (um castigo); uma punição na China antiga | corrupção; peculato; fraude}
  \seealsoref{墨西哥}{mo4xi1ge1}
  \end{Phonetics}
\end{Entry}

\begin{Entry}{墨水}{15,4}{⿊、⽔}
  \begin{Phonetics}{墨水}{mo4 shui3}[][HSK 6]
    \definition[瓶]{s.}{tinta chinesa preparada; tinta (para caneta-tinteiro) | aprendizagem; alfabetização; uma metáfora para o conhecimento ou a capacidade de ler e escrever}
  \end{Phonetics}
\end{Entry}

\begin{Entry}{墨西哥}{15,6,10}{⿊、⾑、⼝}
  \begin{Phonetics}{墨西哥}{mo4xi1ge1}
    \definition*{s.}{México; Planalto no México}
  \end{Phonetics}
\end{Entry}

\begin{Entry}{墨镜}{15,16}{⿊、⾦}
  \begin{Phonetics}{墨镜}{mo4jing4}
    \definition[只,双,副]{s.}{óculos escuros}
  \end{Phonetics}
\end{Entry}

%%%%%%%%%% 履 %%%%%%%%%%
\subsection*{履}\addcontentsline{loh}{figure}{履}

\begin{Entry}{履}{15}{⼫}
  \begin{Phonetics}{履}{lv3}
    \definition{s.}{sapato | pegada}
    \definition{v.}{pisar em; caminhar sobre | executar; cumprir; honrar; completar}
  \end{Phonetics}
\end{Entry}

\begin{Entry}{履行}{15,6}{⼫、⾏}
  \begin{Phonetics}{履行}{lv3xing2}[][HSK 7-9]
    \definition{v.}{cumprir; executar; realizar; a execução de contratos, acordos, promessas, responsabilidades, etc.}
  \end{Phonetics}
\end{Entry}

%%%%%%%%%% 影 %%%%%%%%%%
\subsection*{影}\addcontentsline{loh}{figure}{影}

\begin{Entry}{影}{15}{⼺}
  \begin{Phonetics}{影}{ying3}
    \definition*{s.}{Sobrenome: Ying}
    \definition{s.}{sombra | reflexão; imagem | traço; sinal; impressão vaga | fotografia; imagem | filme | jogo de sombras; pantomima de sombra}
    \definition{v.}{(dialeto) esconder; ocultar | copiar; rastrear | fotocopiar}
  \end{Phonetics}
\end{Entry}

\begin{Entry}{影子}{15,3}{⼺、⼦}
  \begin{Phonetics}{影子}{ying3zi5}[][HSK 4]
    \definition[个,片]{s.}{sombra; imagem projetada por um objeto, etc., que bloqueia a luz | reflexão; reflexo; imagem de um objeto, etc., conforme aparece em um refletor, como um espelho, uma superfície de água, etc. | sinal; vestígio; vaga impressão}
  \end{Phonetics}
\end{Entry}

\begin{Entry}{影片}{15,4}{⼺、⽚}
  \begin{Phonetics}{影片}{ying3 pian4}[][HSK 2]
    \definition[部,盘,盒,卷]{s.}{filme; imagem | filme; película usada para reproduzir filmes}
  \end{Phonetics}
\end{Entry}

\begin{Entry}{影视}{15,8}{⼺、⾒}
  \begin{Phonetics}{影视}{ying3 shi4}[][HSK 3]
    \definition{s.}{cinema e televisão combinados; denominação conjunta para cinema e TV}
  \end{Phonetics}
\end{Entry}

\begin{Entry}{影响}{15,9}{⼺、⼝}
  \begin{Phonetics}{影响}{ying3xiang3}[][HSK 2]
    \definition{s.}{efeito; influência; efeitos sobre pessoas ou coisas}
    \definition{v.}{afetar; influenciar; influência sobre os pensamentos ou ações dos outros}
  \end{Phonetics}
\end{Entry}

\begin{Entry}{影响力}{15,9,2}{⼺、⼝、⼒}
  \begin{Phonetics}{影响力}{ying3 xiang3 li4}[][HSK 6]
    \definition{s.}{impacto | influência}
  \end{Phonetics}
\end{Entry}

\begin{Entry}{影星}{15,9}{⼺、⽇}
  \begin{Phonetics}{影星}{ying3 xing1}[][HSK 6]
    \definition{s.}{estrela de cinema}
  \end{Phonetics}
\end{Entry}

\begin{Entry}{影迷}{15,9}{⼺、⾡}
  \begin{Phonetics}{影迷}{ying3 mi2}[][HSK 6]
    \definition[个,名,位]{s.}{fã de cinema; entusiasta de cinema; pessoas viciadas em assistir filmes}
  \end{Phonetics}
\end{Entry}

\begin{Entry}{影像}{15,13}{⼺、⼈}
  \begin{Phonetics}{影像}{ying3xiang4}
    \definition{s.}{imagem}
  \end{Phonetics}
\end{Entry}

%%%%%%%%%% 德 %%%%%%%%%%
\subsection*{德}\addcontentsline{loh}{figure}{德}

\begin{Entry}{德}{15}{⼻}
  \begin{Phonetics}{德}{de2}[][HSK 7-9]
    \definition*{s.}{Alemanha, abreviação de 德国 | Sobrenome: De}
    \definition{s.}{virtude; moral; caráter moral; moralidade; conduta; qualidades políticas | coração; mente; pensamentos | bondade; favor; graça}
  \seealsoref{德国}{de2guo2}
  \end{Phonetics}
\end{Entry}

\begin{Entry}{德国}{15,8}{⼻、⼞}
  \begin{Phonetics}{德国}{de2guo2}
    \definition*{s.}{Alemanha}
  \end{Phonetics}
\end{Entry}

\begin{Entry}{德国人}{15,8,2}{⼻、⼞、⼈}
  \begin{Phonetics}{德国人}{de2guo2ren2}
    \definition{s.}{alemão | pessoa ou povo da Alemanha}
  \end{Phonetics}
\end{Entry}

%%%%%%%%%% 慰 %%%%%%%%%%
\subsection*{慰}\addcontentsline{loh}{figure}{慰}

\begin{Entry}{慰}{15}{⼼}
  \begin{Phonetics}{慰}{wei4}
    \definition{adj.}{aliviado; em paz; confortável}
    \definition{v.}{consolar; confortar | ser (ficar) aliviado}
  \end{Phonetics}
\end{Entry}

\begin{Entry}{慰问}{15,6}{⼼、⾨}
  \begin{Phonetics}{慰问}{wei4wen4}[][HSK 5]
    \definition{v.}{visitar; consolar; expressar simpatia por; confortar e cumprimentar com palavras e presentes;  enfatizar o conforto e o cumprimento, frequentemente usado por superiores para subordinados}
  \end{Phonetics}
\end{Entry}

%%%%%%%%%% 憋 %%%%%%%%%%
\subsection*{憋}\addcontentsline{loh}{figure}{憋}

\begin{Entry}{憋}{15}{⼼}
  \begin{Phonetics}{憋}{bie1}[][HSK 7-9]
    \definition{adj.}{sufocado; oprimido}
    \definition{v.}{suprimir; conter | Dialeto: obrigar | Dialeto: ponderar; contemplar | Dialeto: ficar de olho em | Dialeto: destruir (por pressão interna) | calar a boca; inibir; bloquear | sufocar; abafar}
  \end{Phonetics}
\end{Entry}

%%%%%%%%%% 憧 %%%%%%%%%%
\subsection*{憧}\addcontentsline{loh}{figure}{憧}

\begin{Entry}{憧}{15}{⼼}
  \begin{Phonetics}{憧}{chong1}
    \definition{adj.}{irresoluto; indeciso | estúpido; imbecil; confuso}
  \end{Phonetics}
\end{Entry}

\begin{Entry}{憧憬}{15,15}{⼼、⼼}
  \begin{Phonetics}{憧憬}{chong1jing3}
    \definition{v.}{ansiar por | esperar por}
  \end{Phonetics}
\end{Entry}

%%%%%%%%%% 懂 %%%%%%%%%%
\subsection*{懂}\addcontentsline{loh}{figure}{懂}

\begin{Entry}{懂}{15}{⼼}
  \begin{Phonetics}{懂}{dong3}[][HSK 2]
    \definition*{s.}{Sobrenome: Dong}
    \definition{v.}{compreender; entender}
  \end{Phonetics}
\end{Entry}

\begin{Entry}{懂事}{15,8}{⼼、⼅}
  \begin{Phonetics}{懂事}{dong3shi4}[][HSK 7-9]
    \definition{adj.}{sensato; inteligente; muito compreensivo da natureza e da razão humana}
  \end{Phonetics}
\end{Entry}

\begin{Entry}{懂得}{15,11}{⼼、⼻}
  \begin{Phonetics}{懂得}{dong3 de5}[][HSK 2]
    \definition{v.}{saber (significado, prática, etc.); compreender; entender}
  \end{Phonetics}
\end{Entry}

%%%%%%%%%% 摩 %%%%%%%%%%
\subsection*{摩}\addcontentsline{loh}{figure}{摩}

\begin{Entry}{摩}{15}{⼿}
  \begin{Phonetics}{摩}{mo2}
    \definition{v.}{esfregar; raspar; tocar | refletir; estudar | afagar}
  \end{Phonetics}
\end{Entry}

\begin{Entry}{摩托}{15,6}{⼿、⼿}
  \begin{Phonetics}{摩托}{mo2 tuo1}[][HSK 5]
    \definition[辆]{s.}{Empréstimo linguístico: motor; motor de combustão interna | Empréstimo linguístico: motocicleta, abreviação de 摩托车}
  \seealsoref{摩托车}{mo2tuo1che1}
  \end{Phonetics}
\end{Entry}

\begin{Entry}{摩托车}{15,6,4}{⼿、⼿、⾞}
  \begin{Phonetics}{摩托车}{mo2tuo1che1}
    \definition[辆,部]{s.}{(empréstimo linguístico) motocicleta}
  \end{Phonetics}
\end{Entry}

\begin{Entry}{摩擦}{15,17}{⼿、⼿}
  \begin{Phonetics}{摩擦}{mo2ca1}[][HSK 5]
    \definition{s.}{atrito; desacordo; conflito (entre duas partes); a ação de impedir o movimento relativo entre dois objetos em contato, produzida na superfície de contato | atrito; metáfora para o conflito entre as duas partes}
    \definition{v.}{esfregar}
  \end{Phonetics}
\end{Entry}

%%%%%%%%%% 撑 %%%%%%%%%%
\subsection*{撑}\addcontentsline{loh}{figure}{撑}

\begin{Entry}{撑}{15}{⼿}
  \begin{Phonetics}{撑}{cheng1}[][HSK 6]
    \definition{s.}{suporte; escora;  apoio; esteio}
    \definition{v.}{sustentar; apoiar; resistir a | empurrar (ou mover) com uma vara; usar um mastro para empurrar a margem ou o leito do rio para fazer o barco avançar | manter; manter-se atualizado | abrir; desdobrar; expandir (um objeto contraído) | encher até estourar (inchaço devido a excesso de comida ou alimentação excessiva)}
  \end{Phonetics}
\end{Entry}

%%%%%%%%%% 撒 %%%%%%%%%%
\subsection*{撒}\addcontentsline{loh}{figure}{撒}

\begin{Entry}{撒}{15}{⼿}
  \begin{Phonetics}{撒}{sa1}
    \definition{v.}{lançar; deixar ir; deixar sair; liberar | livrar-se de todas as restrições; deixar-se levar; tentar usá-lo ou exibi-lo o máximo possível}
  \end{Phonetics}
\end{Entry}

\begin{Entry}{撒旦}{15,5}{⼿、⽇}
  \begin{Phonetics}{撒旦}{sa1dan4}
    \definition*{s.}{Satã}
  \end{Phonetics}
\end{Entry}

\begin{Entry}{撒旦主义}{15,5,5,3}{⼿、⽇、⼂、⼂}
  \begin{Phonetics}{撒旦主义}{sa1dan4 zhu3yi4}
    \definition*{s.}{Satanismo}
  \end{Phonetics}
\end{Entry}

\begin{Entry}{撒但}{15,7}{⼿、⼈}
  \begin{Phonetics}{撒但}{sa1dan4}
    \variantof{撒旦}
  \end{Phonetics}
\end{Entry}

%%%%%%%%%% 撞 %%%%%%%%%%
\subsection*{撞}\addcontentsline{loh}{figure}{撞}

\begin{Entry}{撞}{15}{⼿}
  \begin{Phonetics}{撞}{zhuang4}[][HSK 5]
    \definition{v.}{chocar-se contra; chocar-se com; bater; colidir | encontrar-se por acaso; esbarrar em; deparar-se com | apressar; correr; empurrar | aproveitar a chance | esbarrar de repente em |  encontrar | confiar em; tentar | agir precipitadamente; invadir}
  \end{Phonetics}
\end{Entry}

\begin{Entry}{撞车}{15,4}{⼿、⾞}
  \begin{Phonetics}{撞车}{zhuang4/che1}
    \definition{v.+compl.}{(figurativo) colidir (opiniões, cronogramas, etc.) | ser o mesmo (assunto) | colidir (com outro veículo)}
  \end{Phonetics}
\end{Entry}

\begin{Entry}{撞运气}{15,7,4}{⼿、⾡、⽓}
  \begin{Phonetics}{撞运气}{zhuang4yun4qi5}
    \definition{v.}{confiar no destino | tentar a sorte}
  \end{Phonetics}
\end{Entry}

%%%%%%%%%% 撤 %%%%%%%%%%
\subsection*{撤}\addcontentsline{loh}{figure}{撤}

\begin{Entry}{撤}{15}{⼿}
  \begin{Phonetics}{撤}{che4}[][HSK 7-9]
    \definition{v.}{remover, tirar | demitir; liberar | retirar-se; evacuar}
  \end{Phonetics}
\end{Entry}

\begin{Entry}{撤换}{15,10}{⼿、⼿}
  \begin{Phonetics}{撤换}{che4huan4}[][HSK 7-9]
    \definition{v.}{demitir e substituir (alguém); revogar; substituir (alguém ou alguma coisa)}
  \end{Phonetics}
\end{Entry}

\begin{Entry}{撤离}{15,10}{⼿、⼇}
  \begin{Phonetics}{撤离}{che4 li2}[][HSK 6]
    \definition{v.}{retirar-se de; deixar; evacuar}
  \end{Phonetics}
\end{Entry}

\begin{Entry}{撤销}{15,12}{⼿、⾦}
  \begin{Phonetics}{撤销}{che4xiao1}[][HSK 6]
    \definition{v.}{cancelar; rescindir; revogar; remover}
  \end{Phonetics}
\end{Entry}

%%%%%%%%%% 播 %%%%%%%%%%
\subsection*{播}\addcontentsline{loh}{figure}{播}

\begin{Entry}{播}{15}{⼿}
  \begin{Phonetics}{播}{bo1}[][HSK 6]
    \definition{v.}{espalhar; transmitir | semear | mover-se; migrar; ir para o exílio}
  \end{Phonetics}
\end{Entry}

\begin{Entry}{播出}{15,5}{⼿、⼐}
  \begin{Phonetics}{播出}{bo1 chu1}[][HSK 3]
    \definition{v.}{radiodifundir; transmitir; estar no ar; transmitir via rádio e televisão}
  \end{Phonetics}
\end{Entry}

\begin{Entry}{播放}{15,8}{⼿、⽅}
  \begin{Phonetics}{播放}{bo1fang4}[][HSK 3]
    \definition{v.}{ir ao ar; transmitir por rádio | mostrar; exibir; transmitir (um programa de TV)}
  \end{Phonetics}
\end{Entry}

\begin{Entry}{播音}{15,9}{⼿、⾳}
  \begin{Phonetics}{播音}{bo1/yin1}
    \definition{s.}{transmissão}
    \definition{v.+compl.}{transmitir}
  \end{Phonetics}
\end{Entry}

%%%%%%%%%% 擒 %%%%%%%%%%
\subsection*{擒}\addcontentsline{loh}{figure}{擒}

\begin{Entry}{擒}{15}{⼿}
  \begin{Phonetics}{擒}{qin2}
    \definition{v.}{capturar; pegar; apreender}
  \end{Phonetics}
\end{Entry}

\begin{Entry}{擒获}{15,10}{⼿、⾋}
  \begin{Phonetics}{擒获}{qin2huo4}
    \definition{v.}{apreender | capturar}
  \end{Phonetics}
\end{Entry}

%%%%%%%%%% 敷 %%%%%%%%%%
\subsection*{敷}\addcontentsline{loh}{figure}{敷}

\begin{Entry}{敷}{15}{⽁}
  \begin{Phonetics}{敷}{fu1}[][HSK 7-9]
    \definition*{s.}{Sobrenome: Fu}
    \definition{v.}{aplicar (pó, pomada, etc.) | espalhar; dispor | ser suficiente para | espalhar-se}
  \end{Phonetics}
\end{Entry}

%%%%%%%%%% 暴 %%%%%%%%%%
\subsection*{暴}\addcontentsline{loh}{figure}{暴}

\begin{Entry}{暴}{15}{⽇}
  \begin{Phonetics}{暴}{bao4}
    \definition*{s.}{Sobrenome: Bao}
    \definition{adj.}{repentino e violento | cruel; selvagem; feroz | temperamental | severo e tirânico; brutal | irritável; irascível; impaciente}
    \definition{adv.}{de repente e ferozmente}
    \definition{s.}{violência; ferocidade}
    \definition{v.}{sobressair; destacar-se; inchar | expor; transmitir | desperdiçar; arruinar; estragar}
  \end{Phonetics}
\end{Entry}

\begin{Entry}{暴力}{15,2}{⽇、⼒}
  \begin{Phonetics}{暴力}{bao4li4}[][HSK 6]
    \definition{s.}{violência; força (usada em tempos de conflito); poder de coerção}
  \end{Phonetics}
\end{Entry}

\begin{Entry}{暴风雨}{15,4,8}{⽇、⾵、⾬}
  \begin{Phonetics}{暴风雨}{bao4 feng1 yu3}[][HSK 6]
    \definition{s.}{tempestade; tormenta; temporal; borrasca; vento e chuva fortes e violentos}
  \end{Phonetics}
\end{Entry}

\begin{Entry}{暴风骤雨}{15,4,17,8}{⽇、⾵、⾺、⾬}
  \begin{Phonetics}{暴风骤雨}{bao4feng1-zhou4yu3}[][HSK 7-9]
    \definition{expr.}{tempestade violenta; furacão; tempestade | vento violento e tempestade de chuva}
  \end{Phonetics}
\end{Entry}

\begin{Entry}{暴行}{15,6}{⽇、⾏}
  \begin{Phonetics}{暴行}{bao4xing2}
    \definition{s.}{ato selvagem | atrocidade | indignação}
  \end{Phonetics}
\end{Entry}

\begin{Entry}{暴乱}{15,7}{⽇、⼄}
  \begin{Phonetics}{暴乱}{bao4luan4}
    \definition{s.}{rebelião | revolta | tumulto}
  \end{Phonetics}
\end{Entry}

\begin{Entry}{暴利}{15,7}{⽇、⼑}
  \begin{Phonetics}{暴利}{bao4li4}[][HSK 7-9]
    \definition{s.}{lucros enormes repentinos | lucros exorbitantes; lucros extravagantes; lucros excessivos}
  \end{Phonetics}
\end{Entry}

\begin{Entry}{暴雨}{15,8}{⽇、⾬}
  \begin{Phonetics}{暴雨}{bao4yu3}[][HSK 6]
    \definition[场,次,阵]{s.}{tempestade; chuva torrencial; chuva forte com precipitação intensa; em meteorologia, refere-se a chuvas de 16 mm ou mais em uma hora ou 50 mm ou mais em 24 horas}
  \end{Phonetics}
\end{Entry}

\begin{Entry}{暴躁}{15,20}{⽇、⾜}
  \begin{Phonetics}{暴躁}{bao4zao4}[][HSK 7-9]
    \definition{adj.}{irascível; febril; irritável; temperamental; descreve uma pessoa que é impaciente, não consegue controlar suas emoções e fica com raiva facilmente}
  \end{Phonetics}
\end{Entry}

\begin{Entry}{暴露}{15,21}{⽇、⾬}
  \begin{Phonetics}{暴露}{bao4lu4}[][HSK 6]
    \definition{adj.}{reveladoras (roupas inadequadas que expõem muito o corpo)}
    \definition{v.}{expor; desnudar; revelar; tornar público algo oculto}
  \end{Phonetics}
\end{Entry}

%%%%%%%%%% 槽 %%%%%%%%%%
\subsection*{槽}\addcontentsline{loh}{figure}{槽}

\begin{Entry}{槽}{15}{⽊}
  \begin{Phonetics}{槽}{cao2}[][HSK 7-9]
    \definition{clas.}{usado para portas | usado para porcos}
    \definition[个,道]{s.}{cocho | sulco; entalhe | canal | manjedoura (para água, ração animal, vinho, cuba); um recipiente para alimentar o gado, geralmente é retangular, alto em todos os lados e côncavo no meio, como uma caixa sem tampa | tanque de fermentação; cuba de vinho; geralmente se refere a certos utensílios com lados altos e côncavos no meio | leito do rio; fossa; refere-se a certos cursos d'água ou valas com lados altos e um meio côncavo | ranhura; fenda; uma depressão semelhante a um sulco em um objeto}
  \end{Phonetics}
\end{Entry}

%%%%%%%%%% 横 %%%%%%%%%%
\subsection*{横}\addcontentsline{loh}{figure}{横}

\begin{Entry}{横}{15}{⽊}
  \begin{Phonetics}{横}{heng2}[][HSK 6]
    \definition{adj.}{horizontal; transversal; paralelo ao plano horizontal (oposto de 竖 e 直) | em ângulo reto com; direção esquerda-direita (em oposição à 竖, 直 ou 纵) | e leste a oeste ou de oeste a leste; direção leste-oeste (oposta a 纵) | desenfreado; turbulento | violento; feroz; irracional}
    \definition{adv.}{de qualquer forma; em qualquer caso | provavelmente; muito provavelmente}
    \definition{s.}{traço horizontal (em caracteres chineses)}
    \definition{v.}{deitar-se transversalmente; estar de lado | colocar algo transversalmente (ou horizontalmente)}
  \seealsoref{竖}{shu4}
  \seealsoref{直}{zhi2}
  \seealsoref{纵}{zong4}
  \end{Phonetics}
  \begin{Phonetics}{横}{heng4}[][HSK 7-9]
    \definition{adj.}{chocante e irracional; inesperado}
  \end{Phonetics}
\end{Entry}

\begin{Entry}{横七竖八}{15,2,9,2}{⽊、⼀、⽴、⼋}
  \begin{Phonetics}{横七竖八}{heng2qi1-shu4ba1}[][HSK 7-9]
    \definition{expr.}{em desordem; em seis e sete; desorganizado}
  \end{Phonetics}
\end{Entry}

\begin{Entry}{横向}{15,6}{⽊、⼝}
  \begin{Phonetics}{横向}{heng2xiang4}[][HSK 7-9]
    \definition{adj.}{horizontal; transversal (oposto a 竖向,纵向) | lateral | ortogonal | perpendicular}
  \seealsoref{竖向}{shu4xiang4}
  \seealsoref{纵向}{zong4xiang4}
  \end{Phonetics}
\end{Entry}

\begin{Entry}{横竖}{15,9}{⽊、⽴}
  \begin{Phonetics}{横竖}{heng2shu5}
    \definition{adv.}{de qualquer forma; em qualquer maneira; isso significa que não importa o que aconteça, o resultado ou a conclusão não mudará; equivale a 反正}
  \seealsoref{反正}{fan3zheng4}
  \end{Phonetics}
\end{Entry}

%%%%%%%%%% 樱 %%%%%%%%%%
\subsection*{樱}\addcontentsline{loh}{figure}{樱}

\begin{Entry}{樱}{15}{⽊}
  \begin{Phonetics}{樱}{ying1}
    \definition[个,棵,朵]{s.}{cereja | cerejeira oriental; flores de cerejeira}
  \end{Phonetics}
\end{Entry}

\begin{Entry}{樱桃}{15,10}{⽊、⽊}
  \begin{Phonetics}{樱桃}{ying1tao2}
    \definition{s.}{cereja}
  \end{Phonetics}
\end{Entry}

%%%%%%%%%% 橄 %%%%%%%%%%
\subsection*{橄}\addcontentsline{loh}{figure}{橄}

\begin{Entry}{橄}{15}{⽊}
  \begin{Phonetics}{橄}{gan3}
    \definition*{s.}{Sobrenome: Gan}
  \end{Phonetics}
\end{Entry}

\begin{Entry}{橄榄球}{15,13,11}{⽊、⽊、⽟}
  \begin{Phonetics}{橄榄球}{gan3lan3qiu2}
    \definition{s.}{futebol jogado com bola oval (rúgbi, futebol americano, regras australianas, etc.)}
  \end{Phonetics}
\end{Entry}

%%%%%%%%%% 潜 %%%%%%%%%%
\subsection*{潜}\addcontentsline{loh}{figure}{潜}

\begin{Entry}{潜}{15}{⽔}
  \begin{Phonetics}{潜}{qian2}
    \definition*{s.}{Sobrenome: Qian}
    \definition{adj.}{latente; oculto}
    \definition{adv.}{furtivamente; secretamente; às escondidas}
    \definition{v.}{ir para debaixo d'água; esconder-se debaixo d'água; mergulhar | esconder | vadear (atravessar) na água | enterrar | fugir de casa}
  \end{Phonetics}
\end{Entry}

\begin{Entry}{潜力}{15,2}{⽔、⼒}
  \begin{Phonetics}{潜力}{qian2li4}[][HSK 6]
    \definition{s.}{potencial; potencialidade; capacidade latente; as habilidades e possibilidades de desenvolvimento que as pessoas e as coisas ainda não demonstraram}
  \end{Phonetics}
\end{Entry}

\begin{Entry}{潜在}{15,6}{⽔、⼟}
  \begin{Phonetics}{潜在}{qian2zai4}
    \definition{adj.}{oculto | latente}
    \definition{s.}{potencial}
  \end{Phonetics}
\end{Entry}

%%%%%%%%%% 潦 %%%%%%%%%%
\subsection*{潦}\addcontentsline{loh}{figure}{潦}

\begin{Entry}{潦}{15}{⽔}
  \begin{Phonetics}{潦}{lao3}
    \definition{s.}{Literário: água da chuva; chuva forte | Literário: poças nas estradas | Literário: inundado}
  \end{Phonetics}
  \begin{Phonetics}{潦}{liao2}
    \definition{s.}{rabisco; garrancho; garatuja; caligrafia desleixada ou descuidada}
  \end{Phonetics}
\end{Entry}

\begin{Entry}{潦草}{15,9}{⽔、⾋}
  \begin{Phonetics}{潦草}{liao2cao3}[][HSK 7-9]
    \definition{adj.}{ilegível; apressado e descuidado (na caligrafia) | desleixado; descuidado; descuidado e pouco sério ao realizar as coisas}
  \end{Phonetics}
\end{Entry}

%%%%%%%%%% 潮 %%%%%%%%%%
\subsection*{潮}\addcontentsline{loh}{figure}{潮}

\begin{Entry}{潮}{15}{⽔}
  \begin{Phonetics}{潮}{chao2}[][HSK 4]
    \definition{adj.}{úmido; molhado | inferior; de qualidade ruim | inferior; não muito habilidoso}
    \definition{s.}{maré; água da maré | surto; corrente; maré; uma metáfora para mudanças sociais em grande escala ou para os altos e baixos de um movimento (social)}
    \definition{s.}{Chaozhou, uma cidade na província de Guangdong}
  \end{Phonetics}
\end{Entry}

\begin{Entry}{潮流}{15,10}{⽔、⽔}
  \begin{Phonetics}{潮流}{chao2liu2}[][HSK 4]
    \definition[种,股,个]{s.}{maré; corrente de maré; movimento da água devido às marés | tendência; analogia com mudanças sociais ou tendências de desenvolvimento}
  \end{Phonetics}
\end{Entry}

\begin{Entry}{潮绣}{15,10}{⽔、⽷}
  \begin{Phonetics}{潮绣}{chao2xiu4}
    \definition*{s.}{Bordado Chaozhou}
  \end{Phonetics}
\end{Entry}

\begin{Entry}{潮湿}{15,12}{⽔、⽔}
  \begin{Phonetics}{潮湿}{chao2shi1}[][HSK 4]
    \definition{adj.}{molhado; úmido; umedecido; que contém mais água do que o normal}
  \end{Phonetics}
\end{Entry}

%%%%%%%%%% 澄 %%%%%%%%%%
\subsection*{澄}\addcontentsline{loh}{figure}{澄}

\begin{Entry}{澄}{15}{⽔}
  \begin{Phonetics}{澄}{cheng2}
    \definition*{s.}{Sobrenome: Cheng}
    \definition{adj.}{claro; transparente}
    \definition{v.}{esclarecer; purificar}
  \end{Phonetics}
  \begin{Phonetics}{澄}{deng4}
    \definition{adj.}{(água, ar, etc.) claro; transparente; límpido}
    \definition{v.}{esclarecer; aclarar | sedimentar; fazer com que impurezas em um líquido afundem}
  \end{Phonetics}
\end{Entry}

\begin{Entry}{澄清}{15,11}{⽔、⽔}
  \begin{Phonetics}{澄清}{cheng2qing1}[][HSK 7-9]
    \definition{adj.}{claro; transparente}
    \definition{v.}{esclarecer; deixar claro; entender | purificar; limpar; esclarecer a turbidez, uma metáfora para esclarecer uma situação caótica}
  \end{Phonetics}
\end{Entry}

%%%%%%%%%% 澳 %%%%%%%%%%
\subsection*{澳}\addcontentsline{loh}{figure}{澳}

\begin{Entry}{澳}{15}{⽔}
  \begin{Phonetics}{澳}{ao4}
    \definition*{s.}{Abreviação de Austrália, 澳大利亚 | Sobrenome: Ao}
    \definition{s.}{baía; uma entrada do mar; um lugar curvo na costa onde os barcos podem ser atracados, frequentemente usado em nomes de lugares}
  \seealsoref{澳大利亚}{ao4da4li4ya4}
  \end{Phonetics}
\end{Entry}

\begin{Entry}{澳大利亚}{15,3,7,6}{⽔、⼤、⼑、⼆}
  \begin{Phonetics}{澳大利亚}{ao4da4li4ya4}
    \definition*{s.}{Austrália}
  \end{Phonetics}
\end{Entry}

%%%%%%%%%% 熟 %%%%%%%%%%
\subsection*{熟}\addcontentsline{loh}{figure}{熟}

\begin{Entry}{熟}{15}{⽕}
  \begin{Phonetics}{熟}{shu2}[][HSK 2]
    \definition{adj.}{maduro (frutos) | pronto; cozido | processado, fabricado ou exercitado | familiar, bem conhecido; conhecido por ser comum ou frequentemente utilizado | habilidoso;  (trabalho, tecnologia) experiente; não é novato | profundo; sólido}
  \end{Phonetics}
\end{Entry}

\begin{Entry}{熟人}{15,2}{⽕、⼈}
  \begin{Phonetics}{熟人}{shu2 ren2}[][HSK 3]
    \definition[位,名,个,些]{s.}{amigo; conhecido; pessoas que se conhecem há muito tempo; pessoas que são muito familiares}
  \end{Phonetics}
\end{Entry}

\begin{Entry}{熟练}{15,8}{⽕、⽷}
  \begin{Phonetics}{熟练}{shu2lian4}[][HSK 4]
    \definition{adj.}{especializado; proficiente; qualificado; habilidoso}
  \end{Phonetics}
\end{Entry}

\begin{Entry}{熟悉}{15,11}{⽕、⼼}
  \begin{Phonetics}{熟悉}{shu2xi1}[][HSK 5]
    \definition{adj.}{familiarizado com; não ser estranho}
    \definition{v.}{estar familiarizado com; saber claramente que | conhecer bem algo ou alguém; compreender e dominar (a situação) através da observação ou da experiência}
  \end{Phonetics}
\end{Entry}

%%%%%%%%%% 獞 %%%%%%%%%%
\subsection*{獞}\addcontentsline{loh}{figure}{獞}

\begin{Entry}{獞}{15}{⽝}
  \begin{Phonetics}{獞}{tong2}
    \definition{s.}{nome de uma variedade de cão | tribos selvagens no sul da China}
  \end{Phonetics}
  \begin{Phonetics}{獞}{zhuang4}
    \variantof{壮}
  \end{Phonetics}
\end{Entry}

%%%%%%%%%% 瞒 %%%%%%%%%%
\subsection*{瞒}\addcontentsline{loh}{figure}{瞒}

\begin{Entry}{瞒}{15}{⽬}
  \begin{Phonetics}{瞒}{man2}[][HSK 7-9]
    \definition*{s.}{Sobrenome: Man}
    \definition{v.}{ocultar a verdade de; esconder; esconder a verdade de alguém}
  \end{Phonetics}
\end{Entry}

%%%%%%%%%% 碾 %%%%%%%%%%
\subsection*{碾}\addcontentsline{loh}{figure}{碾}

\begin{Entry}{碾}{15}{⽯}
  \begin{Phonetics}{碾}{nian3}
    \definition[台,个]{s.}{rolo e mó; rolo de pedra | rolo compressor}
    \definition{v.}{moer ou descascar com um rolo; esmagar | (literário) cortar e polir (jade, vidro, etc.) | achatar | pisar; pisotear, 轧}
  \seealsoref{辗}{zhan3}
  \end{Phonetics}
\end{Entry}

\begin{Entry}{碾碎}{15,13}{⽯、⽯}
  \begin{Phonetics}{碾碎}{nian3sui4}
    \definition{v.}{pulverizar | esmagar}
  \end{Phonetics}
\end{Entry}

%%%%%%%%%% 磅 %%%%%%%%%%
\subsection*{磅}\addcontentsline{loh}{figure}{磅}

\begin{Entry}{磅}{15}{⽯}
  \begin{Phonetics}{磅}{bang4}[][HSK 7-9]
    \definition{clas.}{libra | Tipografia: pt, ponto (tamanho de letra, por exemplo: 10pt)}
    \definition{s.}{escalas}
    \definition{v.}{pesar com uma balança}
  \end{Phonetics}
  \begin{Phonetics}{磅}{pang2}
    \definition{adj.}{majestoso; abundante; cheio de energia; magnífico}
  \end{Phonetics}
\end{Entry}

%%%%%%%%%% 磕 %%%%%%%%%%
\subsection*{磕}\addcontentsline{loh}{figure}{磕}

\begin{Entry}{磕}{15}{⽯}
  \begin{Phonetics}{磕}{ke1}[][HSK 7-9]
    \definition{v.}{bater (com força em algo); bater em algo duro | derrubar algo de um recipiente, vaso, etc.}
  \end{Phonetics}
\end{Entry}

%%%%%%%%%% 稻 %%%%%%%%%%
\subsection*{稻}\addcontentsline{loh}{figure}{稻}

\begin{Entry}{稻}{15}{⽲}
  \begin{Phonetics}{稻}{dao4}
    \definition{s.}{arroz; arroz com casca}
  \end{Phonetics}
\end{Entry}

\begin{Entry}{稻草}{15,9}{⽲、⾋}
  \begin{Phonetics}{稻草}{dao4cao3}[][HSK 7-9]
    \definition[捆,根,抱,束]{s.}{palha de arroz (pode ser usada para fazer cordas ou esteiras de palha, para fazer papel, ou para ser usada como ração, combustível, etc.)}
  \end{Phonetics}
\end{Entry}

%%%%%%%%%% 稿 %%%%%%%%%%
\subsection*{稿}\addcontentsline{loh}{figure}{稿}

\begin{Entry}{稿}{15}{⽲}
  \begin{Phonetics}{稿}{gao3}
    \definition[篇]{s.}{(significado original) talo de grão; palha | rascunho; esboço; manuscrito | texto original}
  \end{Phonetics}
\end{Entry}

\begin{Entry}{稿子}{15,3}{⽲、⼦}
  \begin{Phonetics}{稿子}{gao3 zi5}[][HSK 6]
    \definition[篇,份,堆,叠]{s.}{rascunho; esboço; rascunhos de poemas, ensaios, desenhos, etc. | rascunho; manuscrito; poemas escritos | ideia; plano; plano preliminar ou conceito de trabalho}
  \end{Phonetics}
\end{Entry}

\begin{Entry}{稿纸}{15,7}{⽲、⽷}
  \begin{Phonetics}{稿纸}{gao3zhi3}
    \definition{s.}{rascunho | manuscrito}
  \end{Phonetics}
\end{Entry}

%%%%%%%%%% 箭 %%%%%%%%%%
\subsection*{箭}\addcontentsline{loh}{figure}{箭}

\begin{Entry}{箭}{15}{⽵}
  \begin{Phonetics}{箭}{jian4}[][HSK 6]
    \definition[支]{s.}{seta | distância percorrida por uma flecha}
  \end{Phonetics}
\end{Entry}

%%%%%%%%%% 箱 %%%%%%%%%%
\subsection*{箱}\addcontentsline{loh}{figure}{箱}

\begin{Entry}{箱}{15}{⾋}
  \begin{Phonetics}{箱}{xiang1}[][HSK 4]
    \definition{s.}{caixa; estojo; baú | qualquer coisa no formato de caixa}
  \end{Phonetics}
\end{Entry}

\begin{Entry}{箱子}{15,3}{⾋、⼦}
  \begin{Phonetics}{箱子}{xiang1 zi5}[][HSK 4]
    \definition[个,只]{s.}{baú; caixa; estojo; maleta; pasta executiva}
  \end{Phonetics}
\end{Entry}

%%%%%%%%%% 篇 %%%%%%%%%%
\subsection*{篇}\addcontentsline{loh}{figure}{篇}

\begin{Entry}{篇}{15}{⽵}
  \begin{Phonetics}{篇}{pian1}[][HSK 2]
    \definition*{s.}{Sobrenome: Pian}
    \definition{clas.}{usado para folhas de papel, páginas de livros, artigos, etc.}
    \definition{s.}{um pedaço de escrita | folha (de papel, etc.) | (para papel, folhas de livros, artigos, etc.) folha; página; pedaço}
  \end{Phonetics}
\end{Entry}

%%%%%%%%%% 糆 %%%%%%%%%%
\subsection*{糆}\addcontentsline{loh}{figure}{糆}

\begin{Entry}{糆}{15}{⽶}
  \begin{Phonetics}{糆}{mian4}
    \variantof{面}
  \end{Phonetics}
\end{Entry}

%%%%%%%%%% 糊 %%%%%%%%%%
\subsection*{糊}\addcontentsline{loh}{figure}{糊}

\begin{Entry}{糊}{15}{⽶}
  \begin{Phonetics}{糊}{hu1}
    \definition{v.}{colar; untar; usar uma pasta mais espessa para revestir costuras, furos ou superfícies planas}
  \end{Phonetics}
  \begin{Phonetics}{糊}{hu2}[][HSK 7-9]
    \definition{adj.}{queimado}
    \definition{s.}{mingau; pasta; papa}
    \definition{v.}{colar com pasta; colar | (comida) ser queimado}
  \end{Phonetics}
  \begin{Phonetics}{糊}{hu4}
    \definition{s.}{pasta; comida que parece mingau}
  \end{Phonetics}
\end{Entry}

\begin{Entry}{糊里糊涂}{15,7,15,10}{⽶、⾥、⽶、⽔}
  \begin{Phonetics}{糊里糊涂}{hu2 li5 hu2tu5}
    \definition{adj.}{desnorteado | perturbado}
  \end{Phonetics}
\end{Entry}

\begin{Entry}{糊涂}{15,10}{⽶、⽔}
  \begin{Phonetics}{糊涂}{hu2tu5}[][HSK 7-9]
    \definition{adj.}{confuso; perplexo; desnorteado; com compreensão pouco clara ou confusa das coisas | confuso; com conteúdo confuso}
  \end{Phonetics}
\end{Entry}

%%%%%%%%%% 聪 %%%%%%%%%%
\subsection*{聪}\addcontentsline{loh}{figure}{聪}

\begin{Entry}{聪}{15}{⽿}
  \begin{Phonetics}{聪}{cong1}
    \definition{adj.}{audição aguçada | brilhante; inteligente; esperto | perspicaz}
    \definition{s.}{(literário) faculdades auditivas}
  \end{Phonetics}
\end{Entry}

\begin{Entry}{聪明}{15,8}{⽿、⽇}
  \begin{Phonetics}{聪明}{cong1ming5}[][HSK 5]
    \definition{adj.}{brilhante; esperto; inteligente; intelecto bem desenvolvido com boa memória e capacidade de compreensão}
  \end{Phonetics}
\end{Entry}

\begin{Entry}{聪慧}{15,15}{⽿、⼼}
  \begin{Phonetics}{聪慧}{cong1hui4}
    \definition{adj.}{inteligente | brilhante}
  \end{Phonetics}
\end{Entry}

%%%%%%%%%% 蔬 %%%%%%%%%%
\subsection*{蔬}\addcontentsline{loh}{figure}{蔬}

\begin{Entry}{蔬}{15}{⾋}
  \begin{Phonetics}{蔬}{shu1}
    \definition{s.}{vegetais}
  \end{Phonetics}
\end{Entry}

\begin{Entry}{蔬菜}{15,11}{⾋、⾋}
  \begin{Phonetics}{蔬菜}{shu1cai4}[][HSK 5]
    \definition[样,种]{s.}{verduras; legumes; vegetais; ervas que podem ser usadas na culinária}
  \end{Phonetics}
\end{Entry}

%%%%%%%%%% 蕃 %%%%%%%%%%
\subsection*{蕃}\addcontentsline{loh}{figure}{蕃}

\begin{Entry}{蕃}{15}{⾋}
  \begin{Phonetics}{蕃}{bo1}
    \definition[种]{s.}{estrangeiros}
  \end{Phonetics}
  \begin{Phonetics}{蕃}{fan1}
    \definition[种]{s.}{estrangeiros; aborígenes}
  \end{Phonetics}
  \begin{Phonetics}{蕃}{fan2}
    \definition{adj.}{exuberante; próspero}
    \definition{v.}{multiplicar; proliferar}
  \end{Phonetics}
\end{Entry}

\begin{Entry}{蕃茄}{15,8}{⾋、⾋}
  \begin{Phonetics}{蕃茄}{fan1 qie2}
    \variantof{番茄}
  \end{Phonetics}
\end{Entry}

%%%%%%%%%% 蝌 %%%%%%%%%%
\subsection*{蝌}\addcontentsline{loh}{figure}{蝌}

\begin{Entry}{蝌}{15}{⾍}
  \begin{Phonetics}{蝌}{ke1}
    \definition[只]{s.}{girino}
  \end{Phonetics}
\end{Entry}

\begin{Entry}{蝌蚪}{15,10}{⾍、⾍}
  \begin{Phonetics}{蝌蚪}{ke1dou3}
    \definition{s.}{girino}
  \end{Phonetics}
\end{Entry}

%%%%%%%%%% 蝲 %%%%%%%%%%
\subsection*{蝲}\addcontentsline{loh}{figure}{蝲}

\begin{Entry}{蝲}{15}{⾍}
  \begin{Phonetics}{蝲}{la4}
    \definition{s.}{lagostim de água doce}
  \seealsoref{蝲蛄}{la4gu3}
  \end{Phonetics}
\end{Entry}

\begin{Entry}{蝲蛄}{15,11}{⾍、⾍}
  \begin{Phonetics}{蝲蛄}{la4gu3}
    \definition{s.}{lagostim; lagostim de água doce}
  \end{Phonetics}
\end{Entry}

\begin{Entry}{蝲蝲蛄}{15,15,11}{⾍、⾍、⾍}
  \begin{Phonetics}{蝲蝲蛄}{la4la4gu3}
    \definition{s.}{grilo toupeira}
  \end{Phonetics}
\end{Entry}

%%%%%%%%%% 蝴 %%%%%%%%%%
\subsection*{蝴}\addcontentsline{loh}{figure}{蝴}

\begin{Entry}{蝴}{15}{⾍}
  \begin{Phonetics}{蝴}{hu2}
    \definition[对]{s.}{borboleta}
  \end{Phonetics}
\end{Entry}

\begin{Entry}{蝴蝶}{15,15}{⾍、⾍}
  \begin{Phonetics}{蝴蝶}{hu2die2}
    \definition[只]{s.}{borboleta}
  \end{Phonetics}
\end{Entry}

%%%%%%%%%% 豌 %%%%%%%%%%
\subsection*{豌}\addcontentsline{loh}{figure}{豌}

\begin{Entry}{豌}{15}{⾖}
  \begin{Phonetics}{豌}{wan1}
    \definition[粒]{s.}{ervilhas}
  \end{Phonetics}
\end{Entry}

\begin{Entry}{豌豆}{15,7}{⾖、⾖}
  \begin{Phonetics}{豌豆}{wan1dou4}
    \definition{s.}{ervilha}
  \end{Phonetics}
\end{Entry}

%%%%%%%%%% 豫 %%%%%%%%%%
\subsection*{豫}\addcontentsline{loh}{figure}{豫}

\begin{Entry}{豫}{15}{⾗}
  \begin{Phonetics}{豫}{yu4}
    \definition*{s.}{Província de Henan, abreviatura de 河南}
    \definition{adj.}{satisfeito; encantado | anterior; preliminar; preparatório}
    \definition{adv.}{com antecedência; antecipadamente}
    \definition{v.}{viver com facilidade e conforto | participar de}
  \seealsoref{河南}{he2nan2}
  \seealsoref{预}{yu4}
  \end{Phonetics}
\end{Entry}

%%%%%%%%%% 趟 %%%%%%%%%%
\subsection*{趟}\addcontentsline{loh}{figure}{趟}

\begin{Entry}{趟}{15}{⾛}
  \begin{Phonetics}{趟}{tang1}
    \definition{v.}{atravessar; andar na grama ou onde não haja caminho | usar arados, capinadores, etc. para virar o solo e remover ervas daninhas | vadear; atravessar a vau; caminhar por águas rasas}[我们趟水去那小岛。===Nós vadeamos até a ilha.]
  \end{Phonetics}
  \begin{Phonetics}{趟}{tang4}[][HSK 6]
    \definition{clas.}{usado para o número de vezes de viagens de ida e volta |  usado para coisas dispostas em fileiras ou tiras | usado para a programação de veículos, navios, etc. que circulam em uma determinada ordem | usado em conjuntos de movimentos de artes marciais}
    \definition{s.}{marcha; procissão; jornada; viagem}
  \end{Phonetics}
\end{Entry}

%%%%%%%%%% 踏 %%%%%%%%%%
\subsection*{踏}\addcontentsline{loh}{figure}{踏}

\begin{Entry}{踏}{15}{⾜}
  \begin{Phonetics}{踏}{ta1}
    \definition{part.}{Caracter formador de palavras}
  \end{Phonetics}
  \begin{Phonetics}{踏}{ta4}[][HSK 6]
    \definition{v.}{por os pés em; pisar em; esmagar com o pé | fazer uma investigação ou levantamento no local}
  \end{Phonetics}
\end{Entry}

\begin{Entry}{踏实}{15,8}{⾜、⼧}
  \begin{Phonetics}{踏实}{ta1shi5}[][HSK 6]
    \definition{adj.}{confiável; sério; estável e seguro; descreve uma atitude séria em relação ao trabalho ou estudo | à vontade; livre de ansiedade; descreve uma mente ou sentimento estável, sem qualquer preocupação ou ansiedade}
  \end{Phonetics}
\end{Entry}

\begin{Entry}{踏板}{15,8}{⾜、⽊}
  \begin{Phonetics}{踏板}{ta4ban3}
    \definition{s.}{pedal (em um carro, em um piano, etc.) |  apoio para os pés | estribo}
  \end{Phonetics}
\end{Entry}

%%%%%%%%%% 踢 %%%%%%%%%%
\subsection*{踢}\addcontentsline{loh}{figure}{踢}

\begin{Entry}{踢}{15}{⾜}
  \begin{Phonetics}{踢}{ti1}[][HSK 6]
    \definition{v.}{chutar | jogar (por exemplo, futebol)}
  \end{Phonetics}
\end{Entry}

\begin{Entry}{踢蹋舞}{15,17,14}{⾜、⾜、⾇}
  \begin{Phonetics}{踢蹋舞}{ti1ta4wu3}
    \definition{s.}{sapateado | passo de dança}
  \end{Phonetics}
\end{Entry}

\begin{Entry}{踢爆}{15,19}{⾜、⽕}
  \begin{Phonetics}{踢爆}{ti1bao4}
    \definition{v.}{expor | revelar}
  \end{Phonetics}
\end{Entry}

%%%%%%%%%% 踩 %%%%%%%%%%
\subsection*{踩}\addcontentsline{loh}{figure}{踩}

\begin{Entry}{踩}{15}{⾜}
  \begin{Phonetics}{踩}{cai3}[][HSK 6]
    \definition{v.}{pisar; pisotear | pisar; metáfora: depreciar ou estragar | rastrear; antigamente significava rastrear (bandidos) ou investigar (casos)}
  \end{Phonetics}
\end{Entry}

%%%%%%%%%% 躺 %%%%%%%%%%
\subsection*{躺}\addcontentsline{loh}{figure}{躺}

\begin{Entry}{躺}{15}{⾝}
  \begin{Phonetics}{躺}{tang3}[][HSK 4]
    \definition{v.}{deitar; reclinar; cair no chão ou sobre um objeto}
  \end{Phonetics}
\end{Entry}

%%%%%%%%%% 遵 %%%%%%%%%%
\subsection*{遵}\addcontentsline{loh}{figure}{遵}

\begin{Entry}{遵}{15}{⾡}
  \begin{Phonetics}{遵}{zun1}
    \definition{v.}{cumprir; obedecer; observar; seguir}
  \end{Phonetics}
\end{Entry}

\begin{Entry}{遵守}{15,6}{⾡、⼧}
  \begin{Phonetics}{遵守}{zun1shou3}[][HSK 5]
    \definition{v.}{obedecer; observar; cumprir; respeitar; atuar de acordo com as regras; não infringir}
  \end{Phonetics}
\end{Entry}

%%%%%%%%%% 醇 %%%%%%%%%%
\subsection*{醇}\addcontentsline{loh}{figure}{醇}

\begin{Entry}{醇}{15}{⾣}
  \begin{Phonetics}{醇}{chun2}
    \definition{adj.}{Literário: puro; puro e suave; não misturado}
    \definition{s.}{Literário: vinho suave; bom vinho ; Química: álcool}
  \end{Phonetics}
\end{Entry}

\begin{Entry}{醇厚}{15,9}{⾣、⼚}
  \begin{Phonetics}{醇厚}{chun2hou4}[][HSK 7-9]
    \definition{adj.}{suave; rico; cheiro e sabor puros e ricos | puro e honesto; simples e gentil}
  \end{Phonetics}
\end{Entry}

%%%%%%%%%% 醉 %%%%%%%%%%
\subsection*{醉}\addcontentsline{loh}{figure}{醉}

\begin{Entry}{醉}{15}{⾣}
  \begin{Phonetics}{醉}{zui4}[][HSK 5]
    \definition{v.}{embriagar-se; ficar bêbado; intoxicar-se; beber em excesso e perder o controle | (de certos alimentos) ser embebido em licor; ser mergulhado em vinho; marinar (alimentos) em vinho | entregar-se a; ser viciado em; gostar demais, a ponto de chegar à obsessão}
  \end{Phonetics}
\end{Entry}

%%%%%%%%%% 醋 %%%%%%%%%%
\subsection*{醋}\addcontentsline{loh}{figure}{醋}

\begin{Entry}{醋}{15}{⾣}
  \begin{Phonetics}{醋}{cu4}[][HSK 6]
    \definition[瓶,坛,碟,碗]{s.}{(condimento) vinagre | ciúme (como em caso de amor); uma metáfora para o ciúme, referindo-se principalmente aos relacionamentos entre pessoas}
  \end{Phonetics}
\end{Entry}

%%%%%%%%%% 镇 %%%%%%%%%%
\subsection*{镇}\addcontentsline{loh}{figure}{镇}

\begin{Entry}{镇}{15}{⾦}
  \begin{Phonetics}{镇}{zhen4}[][HSK 6]
    \definition{adj.}{inteiro; indica um período inteiro de tempo}
    \definition{adv.}{frequentemente; muitas vezes}
    \definition{s.}{posto de guarnição | cidade; divisão administrativa | centro comercial}
    \definition{v.}{suprimir; segurar; manter pressionado |  acalmar-se; recompor-se; estabilizar | guardar; guarnecer; fortalecer; usar a força para manter a estabilidade | resfriar com gelo; esfriar em água fria | acalmar; suprimir; dissuadir | suprimir pela força; sancionar}
  \end{Phonetics}
\end{Entry}

%%%%%%%%%% 震 %%%%%%%%%%
\subsection*{震}\addcontentsline{loh}{figure}{震}

\begin{Entry}{震}{15}{⾬}
  \begin{Phonetics}{震}{zhen4}
    \definition*{s.}{Zhen, um dos Oito Trigramas que representa o trovão | Sobrenome: Zhen}
    \definition{adj.}{(coloquial) muito animado; profundamente surpreso; chocado}
    \definition{s.}{vibração; trepidação; tremor; abalo | terremoto; refere-se especificamente a terremotos | trovão; relâmpago}
    \definition{v.}{sacudir; chocar; vibrar; estremecer | ficar muito animado; ficar profundamente surpreso; ficar chocado | superar; vencer}
  \end{Phonetics}
\end{Entry}

\begin{Entry}{震惊}{15,11}{⾬、⼼}
  \begin{Phonetics}{震惊}{zhen4jing1}[][HSK 5]
    \definition{adj.}{chocado; atordoado; espantado; atônito}
    \definition{v.}{chocar; surpreender; espantar}
  \end{Phonetics}
\end{Entry}

\begin{Entry}{震撼}{15,16}{⾬、⼿}
  \begin{Phonetics}{震撼}{zhen4han4}
    \definition{v.}{sacudir | chocar | atordoar}
  \end{Phonetics}
\end{Entry}

%%%%%%%%%% 靠 %%%%%%%%%%
\subsection*{靠}\addcontentsline{loh}{figure}{靠}

\begin{Entry}{靠}{15}{⾮}
  \begin{Phonetics}{靠}{kao4}[][HSK 2]
    \definition{prep.}{manter (em); aproximar-se (de); ao longo de | por; graças a; com base em; de acordo com}
    \definition{s.}{armadura de palco (feita de seda bordada); armadura usada pelos generais militares antigos nas peças teatrais}
    \definition{v.}{inclinar-se; sentado ou em pé, deixar parte do peso do corpo ser suportado por outra pessoa ou objeto (pessoa) | encostar-se (em); apoiar-se ou levantar-se com a ajuda de alguma coisa | aproximar-se; estar perto de | confiar em; depender de | confiar}
  \end{Phonetics}
\end{Entry}

\begin{Entry}{靠近}{15,7}{⾮、⾡}
  \begin{Phonetics}{靠近}{kao4 jin4}[][HSK 5]
    \definition{adv.}{próximo; perto de; ao lado de}
    \definition{v.}{aproximar-se; chegar perto; avançar em direção a um determinado objetivo de modo que a distância fique cada vez menor}
  \end{Phonetics}
\end{Entry}

\begin{Entry}{靠拢}{15,8}{⾮、⼿}
  \begin{Phonetics}{靠拢}{kao4long3}[][HSK 7-9]
    \definition{v.}{aproximar-se de; encostar-se; reunir-se; aconchegar-se}
  \end{Phonetics}
\end{Entry}

%%%%%%%%%% 鞋 %%%%%%%%%%
\subsection*{鞋}\addcontentsline{loh}{figure}{鞋}

\begin{Entry}{鞋}{15}{⾰}
  \begin{Phonetics}{鞋}{xie2}[][HSK 2]
    \definition[双,只]{s.}{sapatos; usado nos pés; algo que toca o chão ao caminhar; sem cano alto}
  \end{Phonetics}
\end{Entry}

%%%%%%%%%% 题 %%%%%%%%%%
\subsection*{题}\addcontentsline{loh}{figure}{题}

\begin{Entry}{题}{15}{⾴}
  \begin{Phonetics}{题}{ti2}[][HSK 2]
    \definition*{s.}{Sobrenome: Ti}
    \definition[个,道]{s.}{tópico; título; assunto; problema; frases que indicam o conteúdo de poemas ou discursos | questão; questões que devem ser respondidas durante os exercícios ou exames | antigamente, referia-se à testa}
    \definition{v.}{inscrever; escrever; assinar}
  \end{Phonetics}
\end{Entry}

\begin{Entry}{题目}{15,5}{⾴、⽬}
  \begin{Phonetics}{题目}{ti2mu4}[][HSK 3]
    \definition[个,道]{s.}{título; assunto; tópico; o título de um poema ou discurso | quebra-cabeça; problema de exercício; questões a serem respondidas em exercícios ou provas}
  \end{Phonetics}
\end{Entry}

\begin{Entry}{题材}{15,7}{⾴、⽊}
  \begin{Phonetics}{题材}{ti2cai2}[][HSK 5]
    \definition{s.}{tema; assunto; material que compõe as obras literárias e artísticas, ou seja, os eventos ou fenômenos da vida descritos concretamente nas obras}
  \end{Phonetics}
\end{Entry}

%%%%%%%%%% 颜 %%%%%%%%%%
\subsection*{颜}\addcontentsline{loh}{figure}{颜}

\begin{Entry}{颜}{15}{⾴}
  \begin{Phonetics}{颜}{yan2}
    \definition*{s.}{Sobrenome: Yan}
    \definition{s.}{rosto; semblante; expressão facial | rosto; prestígio; dignidade | cor}
  \end{Phonetics}
\end{Entry}

\begin{Entry}{颜色}{15,6}{⾴、⾊}
  \begin{Phonetics}{颜色}{yan2 se4}[][HSK 2]
    \definition[个,种]{s.}{cor; a sensação visual de um objeto é uma impressão diferente produzida pelas diferentes quantidades de luz absorvidas e refletidas pelo objeto | tez; semblante; aparência; geralmente se refere à aparência de uma garota | olhar severo no rosto como um aviso; um olhar ou ação que faz os outros parecerem particularmente ferozes | a expressão mostrada no rosto}
  \end{Phonetics}
\end{Entry}

%%%%%%%%%% 额 %%%%%%%%%%
\subsection*{额}\addcontentsline{loh}{figure}{额}

\begin{Entry}{额}{15}{⾴}
  \begin{Phonetics}{额}{e2}
    \definition*{s.}{Sobrenome: E}
    \definition[块]{s.}{testa; a área abaixo do cabelo e acima das sobrancelhas em humanos; a área aproximadamente equivalente na cabeça de alguns animais | uma tábua horizontal; placa horizontal inscrita; uma placa pendurada no lintel de uma porta ou na parede | um número específico (ou quantidade); limite superior de número; número limitado | a parte superior de algo}
  \end{Phonetics}
\end{Entry}

\begin{Entry}{额外}{15,5}{⾴、⼣}
  \begin{Phonetics}{额外}{e2wai4}[][HSK 7-9]
    \definition{adj.}{extra; adicional; excede a quantidade ou intervalo prescrito}
  \end{Phonetics}
\end{Entry}

%%%%%%%%%% 飘 %%%%%%%%%%
\subsection*{飘}\addcontentsline{loh}{figure}{飘}

\begin{Entry}{飘}{15}{⾵}
  \begin{Phonetics}{飘}{piao1}
    \definition{adj.}{complacente | frívolo | fraco | instável | bambo | cambaleante}
    \definition{v.}{flutuar (no ar) | esvoaçar | tremular}
  \end{Phonetics}
\end{Entry}

%%%%%%%%%% 鲨 %%%%%%%%%%
\subsection*{鲨}\addcontentsline{loh}{figure}{鲨}

\begin{Entry}{鲨}{15}{⿂}
  \begin{Phonetics}{鲨}{sha1}
    \definition[只,条]{s.}{tubarão}
  \end{Phonetics}
\end{Entry}

\begin{Entry}{鲨鱼}{15,8}{⿂、⿂}
  \begin{Phonetics}{鲨鱼}{sha1yu2}
    \definition{s.}{tubarão}
  \end{Phonetics}
\end{Entry}

%%%%%%%%%% 鹤 %%%%%%%%%%
\subsection*{鹤}\addcontentsline{loh}{figure}{鹤}

\begin{Entry}{鹤}{15}{⿃}
  \begin{Phonetics}{鹤}{he4}
    \definition*{s.}{Sobrenome: He}
    \definition[只]{s.}{grou (ave)}
  \end{Phonetics}
\end{Entry}

\begin{Entry}{鹤立鸡群}{15,5,7,13}{⿃、⽴、⿃、⽺}
  \begin{Phonetics}{鹤立鸡群}{he4li4ji1qun2}[][HSK 7-9]
    \definition{expr.}{destaque-se da multidão; manifestamente superior; muito acima do comum; como um guindaste em pé entre galinhas --- fique de pé acima dos outros}
  \end{Phonetics}
\end{Entry}

%%%%%%%%%% 麫 %%%%%%%%%%
\subsection*{麫}\addcontentsline{loh}{figure}{麫}

\begin{Entry}{麫}{15}{⿆}
  \begin{Phonetics}{麫}{mian4}
    \variantof{面}
  \end{Phonetics}
\end{Entry}

%%%%%%%%%% 黎 %%%%%%%%%%
\subsection*{黎}\addcontentsline{loh}{figure}{黎}

\begin{Entry}{黎}{15}{⿉}
  \begin{Phonetics}{黎}{li2}
    \definition*{s.}{Etnia Li, uma das minorias nacionais da província de Hainan | Sobrenome: Li}
    \definition{adj.}{Literário: preto; escuro | Literário: numeroso}
    \definition{s.}{multidão; as massas; a população}
  \end{Phonetics}
\end{Entry}

\begin{Entry}{黎明}{15,8}{⿉、⽇}
  \begin{Phonetics}{黎明}{li2ming2}[][HSK 7-9]
    \definition[个]{s.}{amanhecer; alvorecer; quando está prestes a amanhecer ou logo após o amanhecer}
  \end{Phonetics}
\end{Entry}

%%%%% EOF %%%%%


%%%
%%% 16画
%%%

\section*{16画}\addcontentsline{toc}{section}{16画}

\begin{entry}{儒}{16}{⼈}
  \begin{phonetics}{儒}{ru2}
    \definition*{s.}{Confucionismo; Confucionista | Sobrenome Ru}
    \definition{s.}{(antigo) erudito; homem culto}
  \end{phonetics}
\end{entry}

\begin{entry}{儒教}{16,11}{⼈、⽁}
  \begin{phonetics}{儒教}{ru2jiao4}
    \definition*{s.}{Confucionismo}
  \end{phonetics}
\end{entry}

\begin{entry}{嘴}{16}{⼝}
  \begin{phonetics}{嘴}{zui3}[][HSK 2]
    \definition[张]{s.}{boca; boca humana ou animal | qualquer coisa com formato ou função semelhante a uma boca | fala | comida}
  \end{phonetics}
\end{entry}

\begin{entry}{嘴巴}{16,4}{⼝、⼰}
  \begin{phonetics}{嘴巴}{zui3 ba5}[][HSK 4]
    \definition[张]{s.}{boca}
  \end{phonetics}
\end{entry}

\begin{entry}{嘴巴子}{16,4,3}{⼝、⼰、⼦}
  \begin{phonetics}{嘴巴子}{zui3ba5zi5}
    \definition{s.}{tapa | bofetada}
  \end{phonetics}
\end{entry}

\begin{entry}{器}{16}{⼝}
  \begin{phonetics}{器}{qi4}
    \definition[台]{s.}{dispositivo | ferramenta | utensílio}
  \end{phonetics}
\end{entry}

\begin{entry}{器官}{16,8}{⼝、⼧}
  \begin{phonetics}{器官}{qi4guan1}[][HSK 4]
    \definition[个]{s.}{órgão; aparelho; parte de um organismo que consiste em vários tipos de tecidos celulares que podem desempenhar uma função fisiológica separada}
  \end{phonetics}
\end{entry}

\begin{entry}{壁}{16}{⼟}
  \begin{phonetics}{壁}{bi4}
    \definition*{s.}{Bi, a décima quarta das vinte e oito constelações em que a esfera celeste foi dividida, consistindo em duas estrelas em linha reta, uma em Pégaso e a outra em Andrômeda | A Estrela Bìxìu, uma das Vinte e Oito Mansões da astronomia tradicional chinesa}
    \definition[道]{s.}{parede | superfície plana como uma parede | penhasco | muralha; parapeito | barreira}
  \end{phonetics}
\end{entry}

\begin{entry}{壁纸}{16,7}{⼟、⽷}
  \begin{phonetics}{壁纸}{bi4zhi3}
    \definition{s.}{papel de parede; papel colado em paredes internas para decoração ou proteção, com diversos tipos e cores}
  \end{phonetics}
\end{entry}

\begin{entry}{壁虎}{16,8}{⼟、⾌}
  \begin{phonetics}{壁虎}{bi4hu3}
    \definition{s.}{lagartixa}
  \end{phonetics}
\end{entry}

\begin{entry}{懒}{16}{⼼}
  \begin{phonetics}{懒}{lan3}[][HSK 6]
    \definition{adj.}{indolente; preguiçoso (oposto de 勤) | lento; lânguido | ocioso; preguiçoso}
  \seealsoref{勤}{qin2}
  \end{phonetics}
\end{entry}

\begin{entry}{懒人}{16,2}{⼼、⼈}
  \begin{phonetics}{懒人}{lan3ren2}
    \definition{s.}{pessoa preguiçosa}
  \end{phonetics}
\end{entry}

\begin{entry}{懒汉}{16,5}{⼼、⽔}
  \begin{phonetics}{懒汉}{lan3han4}
    \definition{s.}{sujeito ocioso | vagabundo | preguiçosos}
  \end{phonetics}
\end{entry}

\begin{entry}{懒虫}{16,6}{⼼、⾍}
  \begin{phonetics}{懒虫}{lan3chong2}
    \definition{s.}{desleixado ocioso | (insulto) sujeito preguiçoso}
  \end{phonetics}
\end{entry}

\begin{entry}{懒怠}{16,9}{⼼、⼼}
  \begin{phonetics}{懒怠}{lan3dai4}
    \definition{s.}{preguiça}
  \end{phonetics}
\end{entry}

\begin{entry}{懒鬼}{16,9}{⼼、⿁}
  \begin{phonetics}{懒鬼}{lan3gui3}
    \definition{s.}{cara preguiçoso}
  \end{phonetics}
\end{entry}

\begin{entry}{懒得}{16,11}{⼼、⼻}
  \begin{phonetics}{懒得}{lan3de5}
    \definition{adv.}{demasiado preguiçoso}
    \definition{v.}{não sentir vontade (de fazer algo)}
  \end{phonetics}
\end{entry}

\begin{entry}{懒惰}{16,12}{⼼、⼼}
  \begin{phonetics}{懒惰}{lan3duo4}
    \definition{adj.}{preguiçoso}
  \end{phonetics}
\end{entry}

\begin{entry}{懒散}{16,12}{⼼、⽁}
  \begin{phonetics}{懒散}{lan3san3}
    \definition{adj.}{inativo | indolente | preguiçoso | negligente}
  \end{phonetics}
\end{entry}

\begin{entry}{懒腰}{16,13}{⼼、⾁}
  \begin{phonetics}{懒腰}{lan3yao1}
    \definition[个]{s.}{alongamento (do corpo)}
  \end{phonetics}
\end{entry}

\begin{entry}{撼}{16}{⼿}
  \begin{phonetics}{撼}{han4}
    \definition{v.}{agitar; sacudir}
  \end{phonetics}
\end{entry}

\begin{entry}{擅}{16}{⼿}
  \begin{phonetics}{擅}{shan4}
    \definition{adv.}{sem autorização; arbitrariamente | fazer algo por conta própria}
    \definition{v.}{ser bom em; ser especialista em | arrogar-se a si mesmo; fazer algo por conta própria | reivindicar arbitrariamente; ir além do escopo e ajir arbitrariamente}
  \end{phonetics}
\end{entry}

\begin{entry}{擅自}{16,6}{⼿、⾃}
  \begin{phonetics}{擅自}{shan4zi4}
    \definition{adv.}{sem permissão ou autorização | por iniciativa própria}
  \end{phonetics}
\end{entry}

\begin{entry}{操}{16}{⼿}
  \begin{phonetics}{操}{cao1}
    \definition*{s.}{Sobrenome Cao}
    \definition[节,套]{s.}{exercício; ginástica | conduta; comportamento; moralidade, a moral e o código de conduta que as pessoas seguem}
    \definition{v.}{segurar; agarrar; segurar na mão | fazer algo; envolver-se em | falar (uma língua ou dialeto) | treinar (tropas); exercitar (corpo); praticar ou treinar de acordo com uma determinada forma ou postura | dirigir; manusear}
  \end{phonetics}
\end{entry}

\begin{entry}{操心}{16,4}{⼿、⼼}
  \begin{phonetics}{操心}{cao1xin1}
    \definition{v.+compl.}{preocupar-se com}
  \end{phonetics}
\end{entry}

\begin{entry}{操场}{16,6}{⼿、⼟}
  \begin{phonetics}{操场}{cao1chang3}[][HSK 4]
    \definition[个]{s.}{\emph{playground}; campo esportivo; locais para exercícios físicos ou exercícios militares}
  \end{phonetics}
\end{entry}

\begin{entry}{操作}{16,7}{⼿、⼈}
  \begin{phonetics}{操作}{cao1zuo4}[][HSK 4]
    \definition{s.}{operação}
    \definition{v.}{operar; seguir os requisitos e procedimentos prescritos| implementar; realizar; executar; refere-se à implementação concreta (planos, medidas, etc.)}
  \end{phonetics}
\end{entry}

\begin{entry}{操纵}{16,7}{⼿、⽷}
  \begin{phonetics}{操纵}{cao1zong4}[][HSK 6]
    \definition{v.}{operar; controlar (uma máquina, instrumento, etc.) | manipular; controlar secretamente; assumir o controle de (uma pessoa, organização, situação, etc.)}
  \end{phonetics}
\end{entry}

\begin{entry}{整}{16}{⽁}
  \begin{phonetics}{整}{zheng3}[][HSK 3]
    \definition*{s.}{Sobrenome Zheng}
    \definition{adj.}{cheio; integral; inteiro; completo; sem defeitos | limpo; arrumado; organizado; em boa ordem | redondo (não é uma fração)}
    \definition{s.}{número inteiro (não fracionário)}
    \definition{v.}{retificar; corrigir; pôr em ordem | consertar; renovar; reparar | corrigir; punir; causar sofrimento;  fazer alguém sofrer | fazer; realizar; trabalhar; em algumas regiões, significa 做, 搞}
  \seealsoref{搞}{gao3}
  \seealsoref{做}{zuo4}
  \end{phonetics}
\end{entry}

\begin{entry}{整个}{16,3}{⽁、⼈}
  \begin{phonetics}{整个}{zheng3ge4}[][HSK 3]
    \definition{adj.}{total; inteiro; completo}
  \end{phonetics}
\end{entry}

\begin{entry}{整天}{16,4}{⽁、⼤}
  \begin{phonetics}{整天}{zheng3 tian1}[][HSK 3]
    \definition{s.}{o dia inteiro; o dia todo; durante todo o dia; de manhã à noite}
  \end{phonetics}
\end{entry}

\begin{entry}{整齐}{16,6}{⽁、⿑}
  \begin{phonetics}{整齐}{zheng3qi2}[][HSK 3]
    \definition{adj.}{arrumado; organizado; em boa ordem | uniforme; regular; tamanho, comprimento, grau, etc. são relativamente consistentes | usado para descrever que todas as coisas necessárias estão prontas}
    \definition{v.}{estar em boas condições; manter a ordem e a organização}
  \end{phonetics}
\end{entry}

\begin{entry}{整体}{16,7}{⽁、⼈}
  \begin{phonetics}{整体}{zheng3ti3}[][HSK 3]
    \definition[个]{s.}{um todo; totalidade}
  \end{phonetics}
\end{entry}

\begin{entry}{整理}{16,11}{⽁、⽟}
  \begin{phonetics}{整理}{zheng3li3}[][HSK 3]
    \definition{v.}{organizar; reorganizar; classificar; ordenar; colocar em ordem}
  \end{phonetics}
\end{entry}

\begin{entry}{整整}{16,16}{⽁、⽁}
  \begin{phonetics}{整整}{zheng3 zheng3}[][HSK 3]
    \definition{adv.}{inteiramente; completamente; solidamente; continuamente}
  \end{phonetics}
\end{entry}

\begin{entry}{橘}{16}{⽊}
  \begin{phonetics}{橘}{ju2}
    \definition[只,棵]{s.}{tangerina}
  \end{phonetics}
\end{entry}

\begin{entry}{橘子汁}{16,3,5}{⽊、⼦、⽔}
  \begin{phonetics}{橘子汁}{ju2zi5zhi1}
    \definition[瓶,杯,罐,盒]{s.}{suco de laranja}
  \seealsoref{橙汁}{cheng2zhi1}
  \seealsoref{柳橙汁}{liu3cheng2zhi1}
  \end{phonetics}
\end{entry}

\begin{entry}{橙}{16}{⽊}
  \begin{phonetics}{橙}{cheng2}
    \definition{s.}{laranja; fruta da laranjeira | laranjeira; pé de laranja | cor laranja}
  \end{phonetics}
\end{entry}

\begin{entry}{橙汁}{16,5}{⽊、⽔}
  \begin{phonetics}{橙汁}{cheng2zhi1}
    \definition[瓶,杯,罐,盒]{s.}{suco de laranja}
  \seealsoref{橘子汁}{ju2zi5zhi1}
  \seealsoref{柳橙汁}{liu3cheng2zhi1}
  \end{phonetics}
\end{entry}

\begin{entry}{橙色}{16,6}{⽊、⾊}
  \begin{phonetics}{橙色}{cheng2 se4}
    \definition{s.}{cor de laranja}
  \end{phonetics}
\end{entry}

\begin{entry}{激}{16}{⽔}
  \begin{phonetics}{激}{ji1}
    \definition*{s.}{Sobrenome Ji}
    \definition{adj.}{afiado; feroz; violento | vívido}
    \definition{adv.}{bruscamente; ferozmente; violentamente}
    \definition{s.}{o impacto de ondas fortes contra a costa}
    \definition{v.}{bater; avançar; correr | despertar; estimular; incitar; excitar | ficar doente por se molhar | esfriar (colocando água gelada, etc.)}
  \end{phonetics}
\end{entry}

\begin{entry}{激动}{16,6}{⽔、⼒}
  \begin{phonetics}{激动}{ji1dong4}[][HSK 4]
    \definition{adj.}{animado; entusiasmado; empolgado}
    \definition{v.}{agitar; excitar; tornar fortes os sentimentos de alguém}
  \end{phonetics}
\end{entry}

\begin{entry}{激烈}{16,10}{⽔、⽕}
  \begin{phonetics}{激烈}{ji1lie4}[][HSK 4]
    \definition{adj.}{agudo; afiado; feroz; violento; intenso}
  \end{phonetics}
\end{entry}

\begin{entry}{激情}{16,11}{⽔、⼼}
  \begin{phonetics}{激情}{ji1qing2}[][HSK 6]
    \definition{s.}{paixão; emoções fortes e explosivas, como êxtase, raiva, etc.}
  \end{phonetics}
\end{entry}

\begin{entry}{燃}{16}{⽕}
  \begin{phonetics}{燃}{ran2}
    \definition{v.}{queimar | acender; inflamar}
  \end{phonetics}
\end{entry}

\begin{entry}{燃料}{16,10}{⽕、⽃}
  \begin{phonetics}{燃料}{ran2 liao4}[][HSK 4]
    \definition{s.}{combustível; carburante; substâncias que podem gerar calor e energia luminosa quando queimadas podem ser divididas em três tipos de acordo com sua forma: combustível sólido (como carvão, carvão vegetal, madeira), combustível líquido (como gasolina, querosene) e combustível gasoso (como gás de carvão, biogás); também se refere a substâncias que podem gerar energia nuclear, como urânio, plutônio, etc.}
  \end{phonetics}
\end{entry}

\begin{entry}{燃烧}{16,10}{⽕、⽕}
  \begin{phonetics}{燃烧}{ran2shao1}[][HSK 4]
    \definition{s.}{combustão | flama}
    \definition{v.}{queimar; acender | arder; inflamar; ferver; metáfora para as emoções de uma pessoa serem muito fortes, como um fogo ardente}
  \end{phonetics}
\end{entry}

\begin{entry}{犟}{16}{⽜}
  \begin{phonetics}{犟}{jiang4}
    \variantof{强}
  \end{phonetics}
\end{entry}

\begin{entry}{磨}{16}{⽯}
  \begin{phonetics}{磨}{mo2}[][HSK 6]
    \definition{v.}{esfregar; desgastar | moer; refletir; polir | desgastar; esgotar; cansar; exaurir | incomodar; causar problemas | destruir; obliterar; extinguir-se | ficar ocioso; perder tempo; perder tempo; procrastinar}
  \end{phonetics}
  \begin{phonetics}{磨}{mo4}
    \definition[盘]{s.}{mó (pedra pesada e redonda para moinho)}
    \definition{v.}{moer; esfarelar; triturar | virar; inverter a marcha}
  \end{phonetics}
\end{entry}

\begin{entry}{磨菇}{16,11}{⽯、⾋}
  \begin{phonetics}{磨菇}{mo2gu5}
    \variantof{蘑菇}
  \end{phonetics}
\end{entry}

\begin{entry}{篮}{16}{⽵}
  \begin{phonetics}{篮}{lan2}
    \definition[个]{s.}{cesto | o anel de ferro e a rede na cesta de basquete}
  \end{phonetics}
\end{entry}

\begin{entry}{篮球}{16,11}{⽵、⽟}
  \begin{phonetics}{篮球}{lan2qiu2}[][HSK 2]
    \definition[个,只]{s.}{basquetebol | bola de basquete; refere-se à bola utilizada no basquetebol}
  \end{phonetics}
\end{entry}

\begin{entry}{糕}{16}{⽶}
  \begin{phonetics}{糕}{gao1}
    \definition{s.}{bolo; alimentos feitos de farinha de arroz, farinha de trigo, etc.}
  \end{phonetics}
\end{entry}

\begin{entry}{糕点}{16,9}{⽶、⽕}
  \begin{phonetics}{糕点}{gao1dian3}
    \definition{s.}{bolos | pastéis}
  \end{phonetics}
\end{entry}

\begin{entry}{糕点师}{16,9,6}{⽶、⽕、⼱}
  \begin{phonetics}{糕点师}{gao1dian3 shi1}
    \definition{s.}{confeiteiro}
  \end{phonetics}
\end{entry}

\begin{entry}{糕点店}{16,9,8}{⽶、⽕、⼴}
  \begin{phonetics}{糕点店}{gao1dian3 dian4}
    \definition{s.}{confeitaria}
  \end{phonetics}
\end{entry}

\begin{entry}{糖}{16}{⽶}
  \begin{phonetics}{糖}{tang2}[][HSK 3]
    \definition[包,斤,勺,袋,块]{s.}{açúcar; um tipo de açúcar; um tipo de composto orgânico, que pode ser dividido em três tipos: monossacarídeos, dissacarídeos e polissacarídeos; é a principal substância que produz energia térmica no corpo humano, como glicose, sacarose, lactose, amido, etc. | açúcar; açúcar comestível; termo geral para açúcar | doces; balas | carboidrato; algo doce e calórico}
  \end{phonetics}
\end{entry}

\begin{entry}{糖醋鱼}{16,15,8}{⽶、⾣、⿂}
  \begin{phonetics}{糖醋鱼}{tang2cu4yu2}
    \definition{s.}{peixe guisado em molho agridoce (prato)}
  \end{phonetics}
\end{entry}

\begin{entry}{膨}{16}{⾁}
  \begin{phonetics}{膨}{peng2}
    \definition{v.}{inchar; inflar | expandir; aumentar o comprimento ou o volume de um objeto}
  \end{phonetics}
\end{entry}

\begin{entry}{膨胀}{16,8}{⾁、⾁}
  \begin{phonetics}{膨胀}{peng2zhang4}
    \definition{v.}{expandir | inflar | inchar}
  \end{phonetics}
\end{entry}

\begin{entry}{蕹}{16}{⾋}
  \begin{phonetics}{蕹}{weng4}
    \definition{s.}{espinafre-d’água ou \emph{ong choy}, usado como vegetal no sul da China e no sudeste da Ásia}
  \end{phonetics}
\end{entry}

\begin{entry}{蕹菜}{16,11}{⾋、⾋}
  \begin{phonetics}{蕹菜}{weng4cai4}
    \definition{s.}{espinafre aquático | \emph{ong choy} | repolho do pântano | convolvulus aquático | glória-da-manhã aquática}
  \seealsoref{空心菜}{kong1xin1cai4}
  \end{phonetics}
\end{entry}

\begin{entry}{薄}{16}{⾋}
  \begin{phonetics}{薄}{bao2}[][HSK 4]
    \definition{adj.}{fino; frágil | frio; indiferente; carente de calor | leve; fraco | pobre; infértil}
  \end{phonetics}
  \begin{phonetics}{薄}{bo2}
    \definition*{s.}{Sobrenome Bo}
    \definition{adj.}{pequeno; leve; magro | mau; cruel; mesquinho | frívolo; fútil; não solene | fraco; frágil}
    \definition{v.}{desprezar; tratar com desprezo; menosprezar | aproximar-se}
  \end{phonetics}
  \begin{phonetics}{薄}{bo4}
    \definition{s.}{menta; uma erva perene com aroma refrescante nos caules e folhas}
  \end{phonetics}
\end{entry}

\begin{entry}{薄弱}{16,10}{⾋、⼸}
  \begin{phonetics}{薄弱}{bo2ruo4}[][HSK 5]
    \definition{adj.}{fraco; frágil}
  \end{phonetics}
\end{entry}

\begin{entry}{薯}{16}{⾋}
  \begin{phonetics}{薯}{shu3}
    \definition{s.}{batata | inhame}
  \end{phonetics}
\end{entry}

\begin{entry}{薯片}{16,4}{⾋、⽚}
  \begin{phonetics}{薯片}{shu3 pian4}[][HSK 6]
    \definition{s.}{batatas fritas (\emph{chips}); batatas fritas crocantes ; flocos finos feitos de batatas}
  \end{phonetics}
\end{entry}

\begin{entry}{薯条}{16,7}{⾋、⽊}
  \begin{phonetics}{薯条}{shu3 tiao2}[][HSK 6]
    \definition{s.}{batatas fritas (palito)}
  \end{phonetics}
\end{entry}

\begin{entry}{融}{16}{⿀}
  \begin{phonetics}{融}{rong2}
    \definition*{s.}{Sobrenome Rong}
    \definition{adj.}{permanente; longo prazo; duradouro | muito brilhante | circulante; corrente}
    \definition{s.}{fogo | plena luz do dia}
    \definition{v.}{derreter; descongelar | misturar; fundir; estar em harmonia | circular (dinheiro, etc.)}
  \end{phonetics}
\end{entry}

\begin{entry}{融入}{16,2}{⿀、⼊}
  \begin{phonetics}{融入}{rong2 ru4}[][HSK 6]
    \definition{v.}{integrar em; juntar-se, integrar-se ao grupo | misturar-se; enfatizar a mistura e a combinação com o ambiente circundante para se tornar harmonioso e consistente | encher com (um certo sentimento); imbuir com (uma certa qualidade); preparar (chá, ervas, etc.); imergir; infundir (drogas, etc.)}
  \end{phonetics}
\end{entry}

\begin{entry}{融合}{16,6}{⿀、⼝}
  \begin{phonetics}{融合}{rong2 he2}[][HSK 6]
    \definition{v.}{fundir; mesclar; misturar; combinar várias coisas diferentes em uma}
  \end{phonetics}
\end{entry}

\begin{entry}{衡}{16}{⾏}
  \begin{phonetics}{衡}{heng2}
    \definition*{s.}{Sobrenome Heng}
    \definition[个]{s.}{braço graduado de uma balança | balança; aparelho de pesagem}
    \definition{v.}{pesar; medir; julgar}
  \end{phonetics}
\end{entry}

\begin{entry}{衡量}{16,12}{⾏、⾥}
  \begin{phonetics}{衡量}{heng2 liang2}[][HSK 6]
    \definition{v.}{pesar; medir; comparar; avaliar | considerar; pensar sobre; deliberar}
  \end{phonetics}
\end{entry}

\begin{entry}{赞}{16}{⾙}
  \begin{phonetics}{赞}{zan4}
    \definition{v.}{patrocinar | apoiar | elogiar | (gíria na \emph{Internet}) para curtir (uma postagem \emph{on-line})}
  \end{phonetics}
\end{entry}

\begin{entry}{赞成}{16,6}{⾙、⼽}
  \begin{phonetics}{赞成}{zan4cheng2}[][HSK 4]
    \definition{v.}{endossar; favorecer; aprovar; concordar com; concordar ou apoiar as ideias, os planos, as propostas ou o comportamento de outra pessoa}
  \end{phonetics}
\end{entry}

\begin{entry}{赞扬}{16,6}{⾙、⼿}
  \begin{phonetics}{赞扬}{zan4yang2}
    \definition{v.}{elogiar | aprovar | demonstrar aprovação}
  \end{phonetics}
\end{entry}

\begin{entry}{赞助}{16,7}{⾙、⼒}
  \begin{phonetics}{赞助}{zan4zhu4}[][HSK 4]
    \definition{s.}{patrocinador}
    \definition{v.}{apoiar; patrocinar; concordar e ajudar (refere-se principalmente a oferecer dinheiro para ajudar)}
  \end{phonetics}
\end{entry}

\begin{entry}{赞赏}{16,12}{⾙、⾙}
  \begin{phonetics}{赞赏}{zan4 shang3}[][HSK 4]
    \definition{v.}{admirar; apreciar; valorizar}
  \end{phonetics}
\end{entry}

\begin{entry}{赠}{16}{⾙}
  \begin{phonetics}{赠}{zeng4}[][HSK 5]
    \definition{v.}{dar um presente; presentear com um brinde}
  \end{phonetics}
\end{entry}

\begin{entry}{赠送}{16,9}{⾙、⾡}
  \begin{phonetics}{赠送}{zeng4song4}[][HSK 5]
    \definition{v.}{dar; dar de presente; dar algo de graça a alguém}
  \end{phonetics}
\end{entry}

\begin{entry}{辩}{16}{⾟}
  \begin{phonetics}{辩}{bian4}
    \definition{v.}{argumentar; disputar; debater}
  \end{phonetics}
\end{entry}

\begin{entry}{辩论}{16,6}{⾟、⾔}
  \begin{phonetics}{辩论}{bian4lun4}[][HSK 4]
    \definition[场,次]{s.}{debate; argumento; a atividade comportamental em si de argumentar ou refutar diferentes pontos de vista ou afirmações, ou uma ocasião ou situação em que tal argumentação ou refutação é feita}
    \definition{v.}{debater; obter um entendimento unificado ou correto, ambos os lados usam linguagem, palavras etc. para explicar seus pontos de vista, apontar os erros ou as contradições do outro lado}
  \end{phonetics}
\end{entry}

\begin{entry}{避}{16}{⾌}
  \begin{phonetics}{避}{bi4}[][HSK 4]
    \definition{v.}{evitar; evadir; esquivar-se; buscar abrigo; fugir | impedir; manter afastado; repelir; previnir}
  \end{phonetics}
\end{entry}

\begin{entry}{避免}{16,7}{⾌、⼉}
  \begin{phonetics}{避免}{bi4mian3}[][HSK 4]
    \definition{v.}{evitar; desviar; abster-se de; tentar não fazer com que algo aconteça; prevenir; tentar impedir (que algo ruim aconteça) com antecedência}
  \end{phonetics}
\end{entry}

\begin{entry}{邀}{16}{⾡}
  \begin{phonetics}{邀}{yao1}
    \definition{v.}{convidar; requerer | (literário)  buscar aprovação; pedir permissão | interceptar}
  \end{phonetics}
\end{entry}

\begin{entry}{邀请}{16,10}{⾡、⾔}
  \begin{phonetics}{邀请}{yao1qing3}[][HSK 5]
    \definition[份,个]{s.}{convite}
    \definition{v.}{convidar; solicitar; convidar pessoas para irem à sua casa ou a um local combinado}
  \end{phonetics}
\end{entry}

\begin{entry}{醒}{16}{⾣}
  \begin{phonetics}{醒}{xing3}[][HSK 4]
    \definition{adj.}{impressionante; notável; admirável; atraente; chamativo}
    \definition{v.}{ficar sóbrio; voltar a si; recuperar a consciência; retornar à normalidade após intoxicação, anestesia ou coma | despertar; estar acordado | ter a mente clara; mover a consciência da confusão para a compreensão | vir a entender; tornar-se ciente de; tomar consciência de}
  \end{phonetics}
\end{entry}

\begin{entry}{镖}{16}{⾦}
  \begin{phonetics}{镖}{biao1}
    \definition{s.}{dardo | arma de arremesso | mercadorias enviadas sob a proteção de uma escolta armada}
  \end{phonetics}
\end{entry}

\begin{entry}{镜}{16}{⾦}
  \begin{phonetics}{镜}{jing4}
    \definition*{s.}{Sobrenome Jing}
    \definition[面,副]{s.}{espelho | lente; vidro; dispositivos para auxiliar a visão ou conduzir experimentos ópticos}
    \definition{v.}{espelhar | perceber | usar como referência}
  \end{phonetics}
\end{entry}

\begin{entry}{镜子}{16,3}{⾦、⼦}
  \begin{phonetics}{镜子}{jing4zi5}[][HSK 4]
    \definition[面,个]{s.}{espelho; instrumento de reflexão de imagem liso e plano, antigamente esmerilhado a partir de um disco grosso de cobre fundido, atualmente feito de vidro plano revestido de prata ou alumínio | óculos; óculos de grau}
  \end{phonetics}
\end{entry}

\begin{entry}{镜头}{16,5}{⾦、⼤}
  \begin{phonetics}{镜头}{jing4tou2}[][HSK 4]
    \definition[个]{s.}{lente de câmera; objetiva; combinação de várias lentes, usada para formar uma imagem | foto; cena}
  \end{phonetics}
\end{entry}

\begin{entry}{雕}{16}{⾫}
  \begin{phonetics}{雕}{diao1}
    \definition*{s.}{Sobrenome Diao}
    \definition{s.}{abutre; águia | escultura ou obras esculpidas}
    \definition{v.}{esculpir; gravar}
  \end{phonetics}
\end{entry}

\begin{entry}{雕刻}{16,8}{⾫、⼑}
  \begin{phonetics}{雕刻}{diao1ke4}
    \definition{s.}{escultura}
    \definition{v.}{esculpir | gravar}
  \end{phonetics}
\end{entry}

\begin{entry}{餐}{16}{⾷}
  \begin{phonetics}{餐}{can1}[][HSK 6]
    \definition{clas.}{comer; fazer uma refeição}
    \definition{clas.}{usado para refeições}
    \definition{s.}{comida; refeição}
  \end{phonetics}
\end{entry}

\begin{entry}{餐厅}{16,4}{⾷、⼚}
  \begin{phonetics}{餐厅}{can1ting1}[][HSK 5]
    \definition[家]{s.}{restaurante; refeitório em um hotel | cantina, refeitório; também é chamado de 食堂}
    \definition[间]{s.}{sala de jantar}
  \seealsoref{食堂}{shi2 tang2}
  \end{phonetics}
\end{entry}

\begin{entry}{餐饮}{16,7}{⾷、⾷}
  \begin{phonetics}{餐饮}{can1 yin3}[][HSK 5]
    \definition{s.}{comidas e bebidas; refere-se a atividades de bufê em restaurantes e hotéis}
  \end{phonetics}
\end{entry}

\begin{entry}{餐馆}{16,11}{⾷、⾷}
  \begin{phonetics}{餐馆}{can1 guan3}[][HSK 5]
    \definition[家]{s.}{restaurante;}
  \end{phonetics}
\end{entry}

\begin{entry}{鲸}{16}{⿂}
  \begin{phonetics}{鲸}{jing1}
    \definition[头,只,条]{s.}{baleia; cetáceo}
  \end{phonetics}
\end{entry}

\begin{entry}{鲸鱼}{16,8}{⿂、⿂}
  \begin{phonetics}{鲸鱼}{jing1yu2}
    \definition{s.}{baleia}
  \end{phonetics}
\end{entry}

\begin{entry}{鲸鲨}{16,15}{⿂、⿂}
  \begin{phonetics}{鲸鲨}{jing1sha1}
    \definition{s.}{tubarão baleia}
  \end{phonetics}
\end{entry}

\begin{entry}{鹦}{16}{⿃}
  \begin{phonetics}{鹦}{ying1}
    \definition[只]{s.}{papagaio}
  \end{phonetics}
\end{entry}

\begin{entry}{鹦鹉}{16,13}{⿃、⿃}
  \begin{phonetics}{鹦鹉}{ying1wu3}
    \definition{s.}{papagaio (ave)}
  \end{phonetics}
\end{entry}

\begin{entry}{鹾}{16}{⿄}
  \begin{phonetics}{鹾}{cuo2}
    \definition{adj.}{salgado}
    \definition{s.}{sal}
  \end{phonetics}
\end{entry}

\begin{entry}{默}{16}{⿊}
  \begin{phonetics}{默}{mo4}
    \definition*{s.}{Sobrenome Mo}
    \definition{adj.}{taciturno; reservado | silencioso}
    \definition{v.}{escrever de memória}
  \end{phonetics}
\end{entry}

\begin{entry}{默契}{16,9}{⿊、⼤}
  \begin{phonetics}{默契}{mo4qi4}
    \definition{adj.}{(de membros da equipe) bem coordenados}
    \definition{s.}{entendimento tácito | entendimento mútuo | conectado em um nível mútuo profundo | (de membros da equipe) bem coordenados}
  \end{phonetics}
\end{entry}

\begin{entry}{默默}{16,16}{⿊、⿊}
  \begin{phonetics}{默默}{mo4mo4}[][HSK 4]
    \definition{adj.}{mudo; silencioso}
    \definition{adv.}{silenciosamente}
  \end{phonetics}
\end{entry}

%%%%% EOF %%%%%


%%%
%%% 17画
%%%

\section*{17画}\addcontentsline{toc}{section}{17画}

\begin{Entry}{戴}{17}{⼽}
  \begin{Phonetics}{戴}{dai4}[][HSK 4]
    \definition*{s.}{Sobrenome Dai}
    \definition{v.}{usar/vestir (óculos, gravata, relógio de pulso, luvas); colocar objetos em sua cabeça, rosto, pescoço, peito, braços etc. | honrar; respeitar;}
  \end{Phonetics}
\end{Entry}

\begin{Entry}{擦}{17}{⼿}
  \begin{Phonetics}{擦}{ca1}[][HSK 4]
    \definition{v.}{enxugar; esfregar; apagar; limpar; limpar esfregando com um pano, toalha de mão, etc. | espalhar sobre; colocar sobre | passar raspando | ralar (em pedaços); ralar frutas em um ralador para fazer fios finos}
  \end{Phonetics}
\end{Entry}

\begin{Entry}{擦拭}{17,9}{⼿、⼿}
  \begin{Phonetics}{擦拭}{ca1shi4}
    \definition{v.}{limpar com um pano}
  \end{Phonetics}
\end{Entry}

\begin{Entry}{癌}{17}{⽧}
  \begin{Phonetics}{癌}{ai2}[][HSK 7,8,9]
    \definition{s.}{câncer; carcinoma; tumor maligno}
  \end{Phonetics}
\end{Entry}

\begin{Entry}{癌症}{17,10}{⽧、⽧}
  \begin{Phonetics}{癌症}{ai2zheng4}[][HSK 7,8,9]
    \definition[种]{s.}{câncer; tumores malignos no corpo}
  \end{Phonetics}
\end{Entry}

\begin{Entry}{瞧}{17}{⽬}
  \begin{Phonetics}{瞧}{qiao2}[][HSK 5]
    \definition{v.}{ver; olhar | tratar; diagnosticar e tratar | ver; visitar; fazer uma visita}
  \end{Phonetics}
\end{Entry}

\begin{Entry}{窾}{17}{⽳}
  \begin{Phonetics}{窾}{cuan4}
    \definition{adj.}{vazio | seco | destituído; pobre}
    \definition{s.}{buraco | lei}
    \definition{v.}{esconder}
  \end{Phonetics}
  \begin{Phonetics}{窾}{kuan3}
    \definition{adj.}{oco}
    \definition{s.}{rachadura; cavidade | (onomatopéia) água batendo na rocha}
    \definition{v.}{escavar um buraco}
  \end{Phonetics}
\end{Entry}

\begin{Entry}{糟}{17}{⽶}
  \begin{Phonetics}{糟}{zao1}[][HSK 5]
    \definition{adj.}{pobre; apodrecido; deteriorado | estragado; em uma bagunça; em um estado miserável (terrível) | (situação ou circunstância) ruim; desfavorável}
    \definition{s.}{resíduos de destilação de bebidas alcoólicas; resíduos do processo de fermentação do vinho}
    \definition{v.}{marinar alimentos em vinho ou mosto | desperdiçar; estragar; destruir}
  \end{Phonetics}
\end{Entry}

\begin{Entry}{糟糕}{17,16}{⽶、⽶}
  \begin{Phonetics}{糟糕}{zao1gao1}[][HSK 5]
    \definition{adj.}{(corpo, situação, etc.) muito ruim, péssimo}
    \definition{interj.}{que terrível; que má sorte; muito ruim}
  \end{Phonetics}
\end{Entry}

\begin{Entry}{繁}{17}{⽷}
  \begin{Phonetics}{繁}{fan2}
    \definition{adj.}{em grande número; numerosos; múltiplos (oposto a 简) | em grande número; numerosos; complexos; complicado}
    \definition{v.}{propagar; multiplicar}
  \seealsoref{简}{jian3}
  \end{Phonetics}
\end{Entry}

\begin{Entry}{繁荣}{17,9}{⽷、⾋}
  \begin{Phonetics}{繁荣}{fan2rong2}[][HSK 5]
    \definition{adj.}{florescente; próspero}
    \definition{v.}{promover; prosperar}
  \end{Phonetics}
\end{Entry}

\begin{Entry}{繁殖}{17,12}{⽷、⽍}
  \begin{Phonetics}{繁殖}{fan2zhi2}[][HSK 6]
    \definition{v.}{criar; reproduzir; propagar; multiplicar; os organismos produzem novos indivíduos}
  \end{Phonetics}
\end{Entry}

\begin{Entry}{藏}{17}{⾋}
  \begin{Phonetics}{藏}{cang2}[][HSK 6]
    \definition*{s.}{Sobrenome Cang}
    \definition{v.}{esconder; ocultar; esconder da vista | armazenar; coletar; colocar de lado}
  \end{Phonetics}
  \begin{Phonetics}{藏}{zang4}
    \definition*{s.}{Escrituras budistas ou taoístas; um termo geral para clássicos budistas ou taoístas | Região Autônoma do Tibete, 西藏}
    \definition{s.}{depósito; local de armazenamento; armazém; local onde grandes quantidades de coisas são armazenadas | os tibetanos, 藏族; grupo étnico Zang (ou tibetano)}
  \seealsoref{西藏}{xi1zang4}
  \seealsoref{藏族}{zang4zu2}
  \end{Phonetics}
\end{Entry}

\begin{Entry}{藏族}{17,11}{⾋、⽅}
  \begin{Phonetics}{藏族}{zang4zu2}
    \definition*{s.}{Etnia Zang (ou tibetana); Os Zangs (ou tibetanos) , distribuídos pela Região Autônoma do Tibete e pelas províncias de Qinghai, Sichuan, Gansu e Yunnan}
  \end{Phonetics}
\end{Entry}

\begin{Entry}{螺}{17}{⾍}
  \begin{Phonetics}{螺}{luo2}
    \definition{s.}{concha em espiral | caracol | búzio}
  \end{Phonetics}
\end{Entry}

\begin{Entry}{螺丝}{17,5}{⾍、⼀}
  \begin{Phonetics}{螺丝}{luo2si1}
    \definition{s.}{parafuso}
  \end{Phonetics}
\end{Entry}

\begin{Entry}{赢}{17}{⾙}
  \begin{Phonetics}{赢}{ying2}[][HSK 3]
    \definition{v.}{vencer; derrotar | ganhar; lucrar}
  \end{Phonetics}
\end{Entry}

\begin{Entry}{赢得}{17,11}{⾙、⼻}
  \begin{Phonetics}{赢得}{ying2 de2}[][HSK 4]
    \definition{v.}{ganhar; obter; conquistar; assegurar; garantir}
  \end{Phonetics}
\end{Entry}

\begin{Entry}{辫}{17}{⾟}
  \begin{Phonetics}{辫}{bian4}
    \definition{s.}{trança; rabo de cavalo | para coisas como uma trança}
  \end{Phonetics}
\end{Entry}

\begin{Entry}{辫子}{17,3}{⾟、⼦}
  \begin{Phonetics}{辫子}{bian4zi5}
    \definition[根,条]{s.}{trança | um erro ou falha que pode ser explorado por um oponente | alça}
  \end{Phonetics}
\end{Entry}

\begin{Entry}{邉}{17}{⾡}
  \begin{Phonetics}{邉}{bian1}
    \variantof{边}
  \end{Phonetics}
\end{Entry}

\begin{Entry}{霜}{17}{⾬}
  \begin{Phonetics}{霜}{shuang1}
    \definition{s.}{geada | pó branco ou creme espalhado por uma superfície | glacê | creme de pele}
  \end{Phonetics}
\end{Entry}

\begin{Entry}{鳄}{17}{⿂}
  \begin{Phonetics}{鳄}{e4}
    \definition{s.}{crocodilo;  jacaré}
  \end{Phonetics}
\end{Entry}

\begin{Entry}{鳄鱼}{17,8}{⿂、⿂}
  \begin{Phonetics}{鳄鱼}{e4yu2}
    \definition[条]{s.}{jacaré | crocodilo}
  \end{Phonetics}
\end{Entry}

\begin{Entry}{龠}{17}{⿕}[Kangxi 214]
  \begin{Phonetics}{龠}{yue4}
    \definition{clas.}{yue, uma unidade de medida seca para grãos (= 0,5 decilitro);}
    \definition{s.}{uma flauta curta antiga}
  \end{Phonetics}
\end{Entry}

%%%%% EOF %%%%%


%%%
%%% 18画
%%%

\section*{18画}\addcontentsline{toc}{section}{18画}

\begin{entry}{嚣张}{18,7}
  \begin{phonetics}{嚣张}{xiao1zhang1}
    \definition{adj.}{desenfreado | arrogante | agressivo}
  \end{phonetics}
\end{entry}

\begin{entry}{懵懂}{18,15}
  \begin{phonetics}{懵懂}{meng3dong3}
    \definition{adj.}{confuso | ignorante}
  \end{phonetics}
\end{entry}

\begin{entry}{毉}{18}
  \begin{phonetics}{毉}{yi1}
    \variantof{医}
  \end{phonetics}
\end{entry}

\begin{entry}{瀑布}{18,5}
  \begin{phonetics}{瀑布}{pu4bu4}
    \definition{s.}{queda de água | cachoeira | cascata | catarata}
  \end{phonetics}
\end{entry}

\begin{entry}{翻过}{18,6}
  \begin{phonetics}{翻过}{fan1guo4}
    \definition{v.}{virar |  transformar}
  \end{phonetics}
\end{entry}

\begin{entry}{翻译}{18,7}
  \begin{phonetics}{翻译}{fan1yi4}
    \definition[个,位,名]{s.}{tradução | tradutor | interpretação | intérprete}
    \definition{v.}{traduzir; interpretar}
  \end{phonetics}
\end{entry}

\begin{entry}{翻脸}{18,11}
  \begin{phonetics}{翻脸}{fan1lian3}
    \definition{v.+compl.}{brigar com alguém | tornar-se hostil}
  \end{phonetics}
\end{entry}

\begin{entry}{覆盆子}{18,9,3}
  \begin{phonetics}{覆盆子}{fu4pen2zi5}
    \definition{s.}{framboesa}
  \end{phonetics}
\end{entry}

\begin{entry}{蹦极}{18,7}
  \begin{phonetics}{蹦极}{beng4ji2}
    \definition{s.}{\emph{bungee jumping}}
  \end{phonetics}
\end{entry}

%%%%% EOF %%%%%


%%%
%%% 19画
%%%

\section*{19画}\addcontentsline{toc}{section}{19画}

\begin{Entry}{攀}{19}{⼿}
  \begin{Phonetics}{攀}{pan1}
    \definition{v.}{escalar; escalar | buscar conexões em altos cargos | envolver; implicar | agarrar; agarrar-se; segurar-se a}
  \end{Phonetics}
\end{Entry}

\begin{Entry}{攀岩}{19,8}{⼿、⼭}
  \begin{Phonetics}{攀岩}{pan1yan2}
    \definition{s.}{alpinista}
    \definition{v.}{escalar uma montanha}
  \end{Phonetics}
\end{Entry}

\begin{Entry}{攀爬}{19,8}{⼿、⽖}
  \begin{Phonetics}{攀爬}{pan1pa2}
    \definition{v.}{escalar}
  \end{Phonetics}
\end{Entry}

\begin{Entry}{爆}{19}{⽕}
  \begin{Phonetics}{爆}{bao4}[][HSK 6]
    \definition{v.}{explodir; estourar | fritar rapidamente; ferver rapidamente | aparecer (ou ocorrer) inesperadamente}
  \end{Phonetics}
\end{Entry}

\begin{Entry}{爆发}{19,5}{⽕、⼜}
  \begin{Phonetics}{爆发}{bao4fa1}[][HSK 6]
    \definition{v.}{entrar em erupção; explodir | estourar; irromper; ocorrer de forma repentina e violenta}
  \end{Phonetics}
\end{Entry}

\begin{Entry}{爆米花}{19,6,7}{⽕、⽶、⾋}
  \begin{Phonetics}{爆米花}{bao4mi3hua1}
    \definition{s.}{pipoca (de milho) | pipoca de arroz}
  \end{Phonetics}
\end{Entry}

\begin{Entry}{爆炸}{19,9}{⽕、⽕}
  \begin{Phonetics}{爆炸}{bao4zha4}[][HSK 6]
    \definition{s.}{explosão}
    \definition{v.}{explodir; explodir; detonar | aumentar bruscamente em um curto espaço de tempo (de quantidade)}
  \end{Phonetics}
\end{Entry}

\begin{Entry}{聼}{19}{⼼}
  \begin{Phonetics}{聼}{ting1}
    \variantof{听}
  \end{Phonetics}
\end{Entry}

\begin{Entry}{蘑}{19}{⾋}
  \begin{Phonetics}{蘑}{mo2}
    \definition{s.}{cogumelo}
  \end{Phonetics}
\end{Entry}

\begin{Entry}{蘑菇}{19,11}{⾋、⾋}
  \begin{Phonetics}{蘑菇}{mo2gu5}
    \definition{s.}{cogumelo}
    \definition{v.}{mandriar | embromar | amofinar | incomodar alguém com solicitações ou interrupções frequentes ou persistentes}
  \end{Phonetics}
\end{Entry}

\begin{Entry}{警}{19}{⾔}
  \begin{Phonetics}{警}{jing3}
    \definition{s.}{policial}
    \definition{v.}{alertar | avisar}
  \end{Phonetics}
\end{Entry}

\begin{Entry}{警告}{19,7}{⾔、⼝}
  \begin{Phonetics}{警告}{jing3gao4}[][HSK 5]
    \definition[个]{s.}{advertência (como medida disciplinar); uma forma de punição}
    \definition{v.}{avisar; advertir; admoestar}
  \end{Phonetics}
\end{Entry}

\begin{Entry}{警官}{19,8}{⾔、⼧}
  \begin{Phonetics}{警官}{jing3guan1}
    \definition{s.}{polícia | policial}
  \end{Phonetics}
\end{Entry}

\begin{Entry}{警察}{19,14}{⾔、⼧}
  \begin{Phonetics}{警察}{jing3cha2}[][HSK 3]
    \definition[个,位,名,群,队]{s.}{polícia; policial; oficial de polícia; as forças armadas que mantêm a segurança social do país são uma parte importante do aparato estatal; também se refere aos membros dessas forças armadas}
  \end{Phonetics}
\end{Entry}

\begin{Entry}{蹲}{19}{⾜}
  \begin{Phonetics}{蹲}{dun1}[][HSK 6]
    \definition{v.}{agachamento sobre os calcanhares; dobrar as pernas o máximo possível, como se estivesse sentado, mas não deixar as nádegas tocarem o chão | ficar; metáfora para ficar ocioso em casa}
  \end{Phonetics}
\end{Entry}

\begin{Entry}{蹲下}{19,3}{⾜、⼀}
  \begin{Phonetics}{蹲下}{dun1xia4}
    \definition{v.}{agachar | agachar-se}
  \end{Phonetics}
\end{Entry}

%%%%% EOF %%%%%


%%%
%%% 20画
%%%

\section*{20画}\addcontentsline{toc}{section}{20画}

\begin{Entry}{嚼}{20}{⼝}
  \begin{Phonetics}{嚼}{jiao2}[][HSK 7-9]
    \definition{v.}{mastigar; mascar; limitado para uso em 过屠门而大嚼}
  \seealsoref{过屠门而大嚼}{guo4 tu2men2 er2 da4 jiao2}
  \end{Phonetics}
  \begin{Phonetics}{嚼}{jiao4}
    \definition{v.}{mascar; ruminar}
  \end{Phonetics}
  \begin{Phonetics}{嚼}{jue2}
    \definition{v.}{mastigar; morder; mastigar completamente; é usado em algumas palavras compostas e expressões idiomáticas; usado em 咀嚼}
  \seealsoref{咀嚼}{ju3jue2}
  \end{Phonetics}
\end{Entry}

\begin{Entry}{壤}{20}{⼟}
  \begin{Phonetics}{壤}{rang3}
    \definition{s.}{solo | terra | (literário) a terra (em contraste com o céu 天)}
  \end{Phonetics}
\end{Entry}

\begin{Entry}{灌}{20}{⽔}
  \begin{Phonetics}{灌}{guan4}[][HSK 7-9]
    \definition*{s.}{Sobrenome Guan}
    \definition{s.}{arbusto; aglomerados de árvores baixas | irrigação}
    \definition{v.}{irrigar (rega e irrigação do solo) | encher; despejar; injetar | gravar; refere-se à gravação (música)}
  \end{Phonetics}
\end{Entry}

\begin{Entry}{灌溉}{20,12}{⽔、⽔}
  \begin{Phonetics}{灌溉}{guan4gai4}[][HSK 7-9]
    \definition{v.}{regar; irrigar}
  \end{Phonetics}
\end{Entry}

\begin{Entry}{灌输}{20,13}{⽔、⾞}
  \begin{Phonetics}{灌输}{guan4shu1}[][HSK 7-9]
    \definition{v.}{implantar; incutir em; inculcar; imbuir com (ideias, conhecimento); transmitir (ideias, conhecimento, etc.) | canalizar água; despejar água em; direcionar a água para onde ela é necessária}
  \end{Phonetics}
\end{Entry}

\begin{Entry}{譬}{20}{⾔}
  \begin{Phonetics}{譬}{pi4}
    \definition{s.}{exemplo; analogia; metáfora}
    \definition{v.}{dar um exemplo; fazer uma analogia}
  \end{Phonetics}
\end{Entry}

\begin{Entry}{譬如}{20,6}{⾔、⼥}
  \begin{Phonetics}{譬如}{pi4ru2}
    \definition{conj.}{por exemplo | como}
  \end{Phonetics}
\end{Entry}

\begin{Entry}{魔}{20}{⿁}
  \begin{Phonetics}{魔}{mo2}
    \definition{adj.}{místico; misterioso; mágico}
    \definition{s.}{espírito maligno; demônio; diabo; monstro | mágico; místico}
  \end{Phonetics}
\end{Entry}

\begin{Entry}{魔头}{20,5}{⿁、⼤}
  \begin{Phonetics}{魔头}{mo2tou2}
    \definition{s.}{monstro | diabo}
  \end{Phonetics}
\end{Entry}

\begin{Entry}{魔术}{20,5}{⿁、⽊}
  \begin{Phonetics}{魔术}{mo2shu4}
    \definition{s.}{magia}
  \end{Phonetics}
\end{Entry}

%%%%% EOF %%%%%


%%%
%%% 21画
%%%

\section*{21画}\addcontentsline{toc}{section}{21画}

\begin{entry}{露}{21}{⾬}
  \begin{phonetics}{露}{lou4}[][HSK 6]
    \definition{v.}{mostrar; apresentar (uma certa emoção ou olhar no rosto) | mostrar; aparentar; fazer algo visível; as pessoas podem ver}
  \end{phonetics}
  \begin{phonetics}{露}{lu4}[][HSK 6]
    \definition{adj.}{fora de uma casa, tenda, etc., sem cobertura}
    \definition{s.}{orvalho; gotas de água condensadas | xarope; suco de fruta; bebida destilada de flores, folhas ou frutos}
    \definition{v.}{revelar; expor; mostrar; trair}
  \end{phonetics}
\end{entry}

\begin{entry}{露珠}{21,10}{⾬、⽟}
  \begin{phonetics}{露珠}{lu4zhu1}
    \definition{s.}{orvalho}
  \end{phonetics}
\end{entry}

\begin{entry}{霸}{21}{⾬}
  \begin{phonetics}{霸}{ba4}
    \definition*{s.}{Sobrenome Ba}
    \definition{adj.}{arrogante; dominador; tirânico}
    \definition{s.}{líder dos senhores feudais; suserano | tirano; déspota; valentão; \emph{bully} | poder hegemônico; hegemonismo; hegemonia | chefe dos príncipes feudais; líder da antiga aliança feudal}
    \definition{v.}{dominar; tiranizar; governar (ocupar) pela força}
  \end{phonetics}
\end{entry}

\begin{entry}{霸权}{21,6}{⾬、⽊}
  \begin{phonetics}{霸权}{ba4quan2}
    \definition{s.}{hegemonia | supremacia}
  \end{phonetics}
\end{entry}

\begin{entry}{鷄}{21}{⿃}
  \begin{phonetics}{鷄}{ji1}
    \variantof{鸡}
  \end{phonetics}
\end{entry}

%%%%% EOF %%%%%


%%%%
%%% 22画
%%%
\section*{22画}\addcontentsline{toc}{section}{22画}\addcontentsline{loh}{figure}{\#\#\#\# 22画}

%%%%%%%%%% 镶 %%%%%%%%%%
\subsection*{镶}\addcontentsline{loh}{figure}{镶}

\begin{Entry}{镶}{22}{⾦}
  \begin{Phonetics}{镶}{xiang1}
    \definition{v.}{para incrustar; cravar; montar | marginar; orlar; rodear; decorar; colocar uma moldura | inserir; integrar}
  \end{Phonetics}
\end{Entry}

%%%%% EOF %%%%%


%%%%
%%% 23画
%%%
\section*{23画}\addcontentsline{toc}{section}{23画}

%%%%%%%%%% 罐 %%%%%%%%%%
\subsection*{罐}

\begin{Entry}{罐}{23}{⽸}
  \begin{Phonetics}{罐}{guan4}[][HSK 7-9]
    \definition{clas.}{lata; jarra; gavetas e recipientes de água feitos de cerâmica ou metal}[我买了一罐可乐。===Comprei uma lata de Coca-Cola.]
    \definition{s.}{lata; jarra; jarro; pote; tanque | cuba de carvão; vagão de caçamba para carregamento de carvão em minas de carvão}
  \end{Phonetics}
\end{Entry}

\begin{Entry}{罐头}{23,5}{⽸、⼤}
  \begin{Phonetics}{罐头}{guan4tou5}[][HSK 7-9]
    \definition[个,盒,瓶]{s.}{lata; jarra | enlatado; comida enlatada é a abreviação de 罐头食品, que é processada e embalada em latas de ferro seladas ou garrafas de vidro, e pode ser armazenada por um longo tempo}
  \seealsoref{罐头食品}{guan4tou2 shi2pin3}
  \end{Phonetics}
\end{Entry}

\begin{Entry}{罐头食品}{23,5,9,9}{⽸、⼤、⾷、⼝}
  \begin{Phonetics}{罐头食品}{guan4tou2 shi2pin3}
    \definition{s.}{alimentos enlatados; produtos enlatados}
  \end{Phonetics}
\end{Entry}

%%%%% EOF %%%%%


%\input{groups_by_strokes/024.tex}
%\input{groups_by_strokes/025.tex}
%\input{groups_by_strokes/026.tex}
\end{multicols}

\clearpage
\pagestyle{plain}
\chapter{Termos Gramaticais Chineses}
\begin{tabular}{lll}
substantivo/nome       & \textbf{s.}        & 名词 \\
palavra de lugar       & \textbf{p.d.l.}    & 处所词 \\
palavra de localização & \textbf{p.l.}      & 方位词 \\
palavra de tempo       & \textbf{p.t.}      & 时间词 \\
verbo                  & \textbf{v.}        & 动词 \\
verbo direcional       & \textbf{v.d.}      & 趣向\hspace{1em}动词 \\
verbo optativo         & \textbf{v.o.}      & 能缘\hspace{1em}动词 \\
adjetivo               & \textbf{adj.}      & 形容词 \\
numeral                & \textbf{num.}      & 数词 \\
palavra classificadora & \textbf{p.c.}      & 两量词 \\
pronome                & \textbf{pron.}     & 代词 \\
interrogativo          & \textbf{interr.}   & 疑问词 \\
advérbio               & \textbf{adv.}      & 副词 \\
preposição             & \textbf{prep.}     & 介词 \\
conjunção              & \textbf{conj.}     & 连词 \\
partícula              & \textbf{part.}     & 助词 \\
sujeito                & \textbf{suj.}      & 主语 \\
objeto                 & \textbf{obj.}      & 宾语 \\
atributo               & \textbf{atrib.}    & 定语 \\
adjunto adverbial      & \textbf{a.adv.}    & 状语 \\
complemento            & \textbf{compl.}    & 补语 \\
verbo+complemento      & \textbf{v.+compl.} & 动宾式\hspace{1em}离合词 \\
expressão idiomática   & \textbf{expr.}     & \\
interjeição            & \textbf{interj.}   & \\
\end{tabular}


\clearpage
\pagestyle{plain}
\chapter{Radicais Kangxi}
\chapter{Radicais Chineses}

\begin{multicols}{3}
\begin{tabular}{rllll}
\hline
  Nº & Radical & Variante & Tradução & Pinyin \\
\hline
  1  & 一 && um           & \pinyin{yi1}         \\
  2  & 丨 && linha        & \pinyin{shu4}        \\
  3  & 丶 && ponto        & \pinyin{dian3}       \\
  4  & 丿 &乀,乁 & golpear & \pinyin{pie3}       \\
  5  & 乙 &乚,乛 & segundo & \pinyin{yi3}         \\
  6  & 亅 && gancho       & \pinyin{gou1}        \\
  7  & 二 && dois         & \pinyin{er4}         \\
  8  & 亠 && membro       & \pinyin{tou2}        \\
  9  & 人 &亻 & homem     & \pinyin{ren2}        \\
 10  & 儿 && pernas       & \pinyin{er2}         \\
 11  & 入 && entra        & \pinyin{ru4}         \\
 12  & 八 &丷 & oito      & \pinyin{ba1}         \\
 13  & 冂 && caixa de baixo & \pinyin{jiong3}    \\
 14  & 冖 && sobre        & \pinyin{mi4}         \\
 15  & 冫 && gelo         & \pinyin{bing1}       \\
 16  & 几 && mesa         & \pinyin{ji1},\pinyin{ji3} \\
 17  & 凵 && caixa aberta & \pinyin{qu3}         \\
 18  & 刀 &刂 & faca      & \pinyin{dao1}        \\
 19  & 力 && poder        & \pinyin{li4}         \\
 20  & 勹 && embrulho     & \pinyin{bao1}        \\
 21  & 匕 && colher       & \pinyin{bi3}         \\
 22  & 匚 && caixa aberta & \pinyin{fang1}       \\
 23  & 匸 && esconderijo anexo & \pinyin{xi3}    \\
 24  & 十 && dez          & \pinyin{shi2}        \\
 25  & 卜 && místico      & \pinyin{bu3}         \\
 26  & 卩 && foca         & \pinyin{jie2}        \\
 27  & 厂 && penhasco     & \pinyin{han4}        \\
 28  & 厶 && privado      & \pinyin{si1}         \\
 29  & 又 && novamente    & \pinyin{you4}        \\
 30  & 口 && boca         & \pinyin{kou3}        \\
 31  & 囗 && lugar        & \pinyin{wei2}        \\
 32  & 土 && Terra        & \pinyin{tu3}         \\
 33  & 士 && guerreiro    & \pinyin{shi4}        \\
 34  & 夂 && ir           & \pinyin{zhi1}        \\
 35  & 夊 && devagar      & \pinyin{sui1}        \\
 36  & 夕 && tarde        & \pinyin{xi1}         \\
 37  & 大 && grande       & \pinyin{da4}         \\
 38  & 女 && mulher       & \pinyin{nv3}         \\
 39  & 子 && criança      & \pinyin{zi3}         \\
 40  & 宀 && cobertura    & \pinyin{mian2}       \\
 41  & 寸 && polegada     & \pinyin{cun4}        \\
 42  & 小 && pequeno      & \pinyin{xiao3}       \\
 43  & 尢 &尣 & coxo      & \pinyin{you2}        \\
 44  & 尸 && cadáver      & \pinyin{shi1}        \\
 45  & 屮 && brotar       & \pinyin{che4}        \\
 46  & 山 && montanha     & \pinyin{shan1}       \\
 47  & 川 &巛,巜& rio     & \pinyin{chuan1}      \\
 48  & 工 && trabalho     & \pinyin{gong1}       \\
 49  & 己 && a si mesmo   & \pinyin{ji3}         \\
 50  & 巾 && turbante     & \pinyin{jin1}        \\
 51  & 干 && seco         & \pinyin{gan1}        \\
 52  & 幺 && fio curto    & \pinyin{yao1}        \\
 53  & 广 && vasto        & \pinyin{guang3}      \\
 54  & 廴 && passo longo  & \pinyin{yin3}        \\
 55  & 廾 && duas mãos    & \pinyin{gong3}       \\
 56  & 弋 && atirar flecha & \pinyin{yi4}        \\
 57  & 弓 && arco         & \pinyin{gong1}       \\
 58  & 彐 &彑 & focinho   & \pinyin{ji4}         \\
 59  & 彡 && cerdas       & \pinyin{shan1}       \\
 60  & 彳 && dupla        & \pinyin{chi4}        \\
 61  & 心 &忄& coração    & \pinyin{xin1}        \\
 62  & 戈 && lança        & \pinyin{ge1}         \\
 63  & 户 && por          & \pinyin{hu4}         \\
 64  & 手 &扌& mão        & \pinyin{shou3}       \\
 65  & 支 && ramo         & \pinyin{zhi1}        \\
 66  & 攴 &攵 & batida    & \pinyin{pu1}         \\
 67  & 文 && escrita      & \pinyin{wen2}        \\
 68  & 斗 && mergulhador  & \pinyin{dou3}        \\
 69  & 斤 && eixo         & \pinyin{jin1}        \\
 70  & 方 && quadrado     & \pinyin{fang1}       \\
 71  & 无 && não          & \pinyin{wu2}         \\
 72  & 日 && sol          & \pinyin{ri4}         \\
 73  & 曰 && dizer        & \pinyin{yue1}        \\
 74  & 月 && lua          & \pinyin{yue4}        \\
 75  & 木 && árvore       & \pinyin{mu4}         \\
 76  & 欠 && falta        & \pinyin{qian4}       \\
 77  & 止 && parar        & \pinyin{zhi3}        \\
 78  & 歹 && morte        & \pinyin{dai3}        \\
 79  & 殳 && arma         & \pinyin{shu1}        \\
 80  & 母 && mãe          & \pinyin{mu3}         \\
 81  & 比 && comparar     & \pinyin{bi3}         \\
 82  & 毛 && pelo & \pinyin{mao2}        \\
 83  & 氏 && clã & \pinyin{shi4}        \\
 84  & 气 && ar & \pinyin{qi4}         \\
 85  & 水 &氵 & água & \pinyin{shui3}       \\
 86  & 火 &灬 & fogo & \pinyin{huo3}        \\
 87  & 爪 &爫 & garra & \pinyin{zhao3}       \\
 88  & 父 && pai & \pinyin{fu4}         \\
 89  & 爻 && linha & \pinyin{yao2}        \\
 90  & 爿 &丬 & meio tronco & \pinyin{pan2}      \\
 91  & 片 && fatia & \pinyin{pian4}       \\
 92  & 牙 && dente & \pinyin{ya2}         \\
 93  & 牛 &牜 & vaca & \pinyin{niu2}        \\
 94  & 犬 &犭 & cão & \pinyin{quan3}       \\
 95  & 玄 && profundo & \pinyin{xuan2}       \\
 96  & 玉 &王 & jade & \pinyin{yu4}         \\
 97  & 瓜 && melão & \pinyin{gua1}         \\
 98  & 瓦 && telha & \pinyin{wa3}         \\
 99  & 甘 && doce & \pinyin{gan1}         \\
100  & 生 && vida & \pinyin{sheng1}         \\
101  & 用 && usar & \pinyin{yong4}         \\
102  & 田 && campo & \pinyin{tian2}         \\
103  & 疋 && roupa & \pinyin{pi3} \\
104  & 疒 && doença & \pinyin{ne4} \\
105  & 癶 && pegadas & \pinyin{bo1} \\
106  & 白 && branco & \pinyin{bai2} \\
107  & 皮 && pele & \pinyin{pi2} \\
108  & 皿 && prato & \pinyin{min3} \\
109  & 目 && olho & \pinyin{mu4} \\
110  & 矛 && lança & \pinyin{mao2} \\
111  & 矢 && seta & \pinyin{shi3} \\
112  & 石 && pedra & \pinyin{shi2} \\
113  & 示 &礻 & espírito & \pinyin{shi4} \\
114  & 禸 && rastrear & \pinyin{rou2} \\
115  & 禾 && grão & \pinyin{he2} \\
116  & 穴 && caverna & \pinyin{xue2} \\
117  & 立 && ficar em pé & \pinyin{li4} \\
118  & 竹 &⺮ & bambu & \pinyin{zhu2} \\
119  & 米 && arroz & \pinyin{mi3} \\
120  & 糸 &纟& seda & \pinyin{mi4} \\
121  & 缶 && pote & \pinyin{fou3} \\
122  & 网 &罒 & rede & \pinyin{wang3} \\
123  & 羊 && ovelha & \pinyin{yang2} \\
124  & 羽 && pena & \pinyin{yu3} \\
125  & 老 && velho & \pinyin{lao3} \\
126  & 而 && e & \pinyin{er2} \\
127  & 耒 && arado & \pinyin{lei3} \\
128  & 耳 && orelha & \pinyin{er3} \\
129  & 聿 && escova & \pinyin{yu4} \\
130  & 肉 && carne & \pinyin{rou4} \\
131  & 臣 && ministro & \pinyin{chen2} \\
132  & 自 && auto- & \pinyin{zi4} \\
133  & 至 && chegar & \pinyin{zhi4} \\
134  & 臼 && argamassa & \pinyin{jiu4} \\
135  & 舌 && língua & \pinyin{she2} \\
136  & 舛 && opor & \pinyin{chuan3} \\
137  & 舟 && barco & \pinyin{zhou1} \\
138  & 艮 && pausa & \pinyin{gen3} \\
139  & 色 && cor & \pinyin{se4} \\
140  & 艸 &艹 & grama & \pinyin{cao3} \\
141  & 虍 && tigre & \pinyin{hu1} \\
142  & 虫 && inseto & \pinyin{chong2} \\
143  & 血 && sangue & \pinyin{xue4} \\
144  & 行 && andar & \pinyin{xing2} \\
145  & 衣 &衤 & roupa & \pinyin{yi1} \\
146  & 襾 &覀 & oeste & \pinyin{ya4} \\
147  & 見 &见 & ver & \pinyin{jian4} \\
148  & 角 && chifre & \pinyin{jiao3} \\
149  & 言 &讠 & palavra & \pinyin{yan2} \\
150  & 谷 && vale & \pinyin{gu3} \\
151  & 豆 && grão & \pinyin{dou4} \\
152  & 豕 && porco & \pinyin{shi3} \\
153  & 豸 && texugo & \pinyin{zhi4} \\
154  & 貝 &贝 & concha & \pinyin{bei4} \\
155  & 赤 && vermelho & \pinyin{chi4} \\
156  & 走 && andar & \pinyin{zou3} \\
157  & 足 &⻊ & pé & \pinyin{zu2} \\
158  & 身 && corpo & \pinyin{shen1} \\
159  & 車 &车 & carro & \pinyin{che1} \\
160  & 辛 && amargo & \pinyin{xin1} \\
161  & 辰 && manhã & \pinyin{chen2} \\
162  & 辵 &辶 & caminhar & \pinyin{chuo4} \\
163  & 邑 &阝 & cidade & \pinyin{yi4} \\
164  & 酉 && vinho & \pinyin{you3} \\
165  & 釆 && distinto & \pinyin{bian4} \\
166  & 里 && aldeia & \pinyin{li3} \\
167  & 金 && ouro & \pinyin{jin1} \\
168  & 長 &长 & longo & \pinyin{zhang3} \\
169  & 門 &门 & portão & \pinyin{men2} \\
170  & 阜 &阝 & monte & \pinyin{fu4} \\
171  & 隶 && escravo & \pinyin{li4} \\
172  & 隹 && pássaro de cauda curta & \pinyin{zhui1} \\
173  & 雨 && chuva & \pinyin{yu3} \\
174  & 青 && azul & \pinyin{qing1} \\
175  & 非 && errado & \pinyin{fei1} \\
176  & 面 && face & \pinyin{mian4} \\
177  & 革 && couro & \pinyin{ge2} \\
178  & 韋 &韦 & couro tingido & \pinyin{wei2} \\
179  & 韭 && parecia & \pinyin{jiu3} \\
180  & 音 && som & \pinyin{yin1} \\
181  & 頁 &页 & folha & \pinyin{ye4} \\
182  & 風 &风 & vento & \pinyin{feng1} \\
183  & 飛 &飞 & mosca & \pinyin{fei1} \\
184  & 食 &饣,飠 & alimento & \pinyin{shi2} \\
185  & 首 && cabeça & \pinyin{shou3} \\
186  & 香 && perfume & \pinyin{xiang1} \\
187  & 馬 &马 & cavalo & \pinyin{ma3} \\
188  & 骨 && osso & \pinyin{gu3} \\
189  & 高 && alto & \pinyin{gao1} \\
190  & 髟 && cabelo & \pinyin{biao1} \\
191  & 鬥 && luta & \pinyin{dou4} \\
192  & 鬯 && vinho & \pinyin{chang4} \\
193  & 鬲 && separado & \pinyin{ge2} \\
194  & 鬼 && fantasma & \pinyin{gui3} \\
195  & 魚 &鱼 & peixe & \pinyin{yu2} \\
196  & 鳥 &鸟 & pássaro & \pinyin{niao3} \\
197  & 鹵 && sal & \pinyin{lu3} \\
198  & 鹿 && veado & \pinyin{lu4} \\
199  & 麥 &麦 & trigo & \pinyin{mai4} \\
200  & 麻 && cânhamo & \pinyin{ma2} \\
201  & 黃 && amarelo & \pinyin{huang4} \\
202  & 黍 && nação & \pinyin{shu3} \\
203  & 黑 && preto & \pinyin{hei1} \\
204  & 黹 && costura & \pinyin{zhi3} \\
205  & 黽 &黾 & rã & \pinyin{mian3} \\
206  & 鼎 && tripé & \pinyin{ding3} \\
207  & 鼓 && tambor & \pinyin{gu3} \\
208  & 鼠 &鼡 & rato & \pinyin{shu3} \\
209  & 鼻 && nariz & \pinyin{bi2} \\
210  & 齊 &齐 & até & \pinyin{qi2} \\
211  & 齒 &齿 & dente & \pinyin{chi3} \\
212  & 龍 &龙 & dragão & \pinyin{long2} \\
213  & 龜 &龟 & tartaruga & \pinyin{gui1} \\
214  & 龠 && flauta & \pinyin{yue4} \\
\end{tabular}
\end{multicols}


\printindex[sradical]

\end{document}

%%%%% EOF %%%%
