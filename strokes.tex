
%%%%%%%%%%%%%%%%%%%%%%%%%%%%%%%%%%%%%%%%%
% LuaLaTex
%
% Dicionário Chinês -> Português
% Autor: Luiz Eduardo Roncato Cordeiro
%
% Licença:
% CC BY-NC-SA 3.0 (http://creativecommons.org/licenses/by-nc-sa/3.0/)
%%%%%%%%%%%%%%%%%%%%%%%%%%%%%%%%%%%%%%%%%

\documentclass[a4paper,9pt,twoside,openright,book]{memoir}

\usepackage[brazilian]{babel}
\usepackage{fontspec}
\usepackage[dvipsnames]{xcolor}
\usepackage{imakeidx}
\usepackage[inline]{enumitem}
\usepackage{zhnumber}
\usepackage{tikz}
\usepackage[hyperindex]{hyperref}
\usepackage{pifont}
\usepackage{xstring}
\usepackage{xifthen}
\usepackage{tabularray}
\usepackage[most]{tcolorbox}
\usepackage{luacode}
\usepackage{parskip}
\usepackage{stackengine}

% Meus Comandos
%
% Dictionary functions
%

% Até agora o melhor estilo para capítulos
\chapterstyle{verville}

% Ajuste das margens                                                                                                     
\setlrmarginsandblock{3cm}{2cm}{*}
\setulmarginsandblock{2cm}{*}{1}
\checkandfixthelayout

% Linguagem principal
\setmainlanguage[variant=brazilian]{portuguese}
\setotherlanguages{chinese,english}
\hyphenation{
post-gresql
or-a-cle
mi-cro-soft
}


% Ajuste das fontes... No Tofu do Google
\setmainfont[
    Ligatures=TeX,
    BoldFont={NotoSerifCJKsc-SemiBold},
    BoldSlantedFont={NotoSerifCJKsc-SemiBold},
    AutoFakeSlant=0.25,
    SlantedFeatures={FakeSlant=0.25},
    BoldSlantedFeatures={FakeSlant=0.25}]
    {NotoSerifCJKsc-ExtraLight}
\setsansfont[Ligatures=TeX]{NotoSansCJKsc-DemiLight}
\setmonofont[Ligatures=TeX]{NotoSansMonoCJKsc-Regular}

% Sumário
\makeatletter
\renewcommand{\@pnumwidth}{2em} 
\renewcommand{\@tocrmarg}{4em}
\makeatother
\renewcommand\cftbeforechapterskip{5pt plus 1pt}

% Ajustes de espaçamento
\setlength{\parindent}{0cm}
\setlength{\parskip}{-1.5mm}                                                                                             
\setlength{\columnsep}{.8em}
\setlength{\columnseprule}{0.1mm}

% Headers & Footers
\setheadfoot{14pt}{28pt}
\makeevenfoot{plain}{\thepage}{汉葡词典}{}
\makeoddfoot{plain}{}{汉葡词典}{\thepage}
\makefootrule{plain}{\textwidth}{\normalrulethickness}{4pt}
\makepagestyle{dicionario}
\makeevenhead{dicionario}{\rightmark}{}{\leftmark}
\makeoddhead{dicionario}{\rightmark}{}{\leftmark}
\makeevenfoot{dicionario}{\thepage}{汉葡词典}{}
\makeoddfoot{dicionario}{}{汉葡词典}{\thepage}
\makeheadrule{dicionario}{\textwidth}{\normalrulethickness}

% Seções têm uma caixa arredondada em volta do nome, para o A-Z dos pinyin.
\newcommand{\boxedsec}[1]{%
    \begin{tcolorbox}[%
        nobeforeafter,
        colframe=black,%
        colback=black!5!white,%
        boxrule=2pt,%
        leftrule=3mm,%
        left=0mm,%
        right=0mm,%
        top=0mm,%
        bottom=0mm]
    \Large\bfseries#1
    \end{tcolorbox}
}
\setsecheadstyle{\boxedsec}
\setbeforesecskip{2sp}
\setaftersecskip{1sp}
\newcommand{\sectionbreak}{\phantomsection}%this is global effect

% Comandos
\newbool{f_veja}
\newbool{f_exemplo}
\renewcommand\stacktype{S}
\renewcommand\stackalignment{l}
\setstackgap{S}{2.5pt}

% Programa para converter pinyins numéricos para pinyins com acentos
\directlua{dofile "include/tex-sx-pinyin-tonemarks.lua"}

%% wrap converter in a TeX macro
\protected\def\pinyin#1{%
    %% switch to appropriate hyphenation pattern goes here
    \directlua{packagedata.pinyintones.convert ([==[#1]==])}%
}

% comando \&
\DeclareRobustCommand{\&}%
{
    \ifdim\fontdimen1\font>0pt
        \textsl{\symbol{`\&}}%
    \else
        \symbol{`\&}%
    \fi
}

% comando \dul{texto} --- underline
\NewDocumentCommand{\dul}{m}{\underline{#1}}

% comandos \dictpinyin{pin1yin1} e \dpy{pin1yin1}
\NewDocumentCommand{\dictpinyin}{m}{\guillemotleft\pinyin{#1}\guillemotright} 
\NewDocumentCommand{\dpy}{m}%
{%
    \StrSubstitute{#1}{r5}{r}[\rA]%
    \StrSubstitute{\rA}{v}{ü}[\rB]%
    \StrSubstitute{\rB}{V}{Ü}[\rC]%
    \edef\py{\dictpinyin{\rC}}%
    \mbox{}\py
}

% enumerate para as definições do dicionário
\NewDocumentCommand{\dictenumerate}{>{\SplitList{|}}m}
{%
    \begin{enumerate*}[left=0pt,mode=unboxed,font=\bfseries]
        \ProcessList{#1}{\insertitem}
    \end{enumerate*}
}
\NewDocumentCommand{\insertitem}{m}{\item #1}

% Lista Veja
\newenvironment{listaveja}%
{\list{}% empty label
    {
        \setlength{\topsep}{0ex}
        \setlength{\itemsep}{0ex}
        \setlength{\leftmargin}{0ex}
        \setlength{\labelsep}{0ex}
        \setlength{\parsep}{0pt}
        \setlength{\partopsep}{0pt}
        \setlength{\rightmargin}{0ex}
        \setlength{\listparindent}{0em}
        \setlength{\itemindent}{0ex}
        \setlength{\labelwidth}{0ex}
    }%
}%
{\endlist}
\newcommand{\vejaitem}[2]{\item[\addstackgap{\stackunder{#1}{\tiny\dpy{#2}}}]}

\newcommand\vejalst{}
\newcommand\exemplolst{}

\listadd{\vejalst}{}% Initialize list
\listadd{\exemplolst}{}% Initialize list

%%% EOF

%%%%%%%%%%%%%%%%%%%%%%%%%%%%%%%%%%%%%%%%%%%%%%%%%%%%%%%%%%%%%%%%%%%%%%%%%%%%%%%
%%%%%%%%%%%%%%%%%%%%%%%%%%%%%%%%%%%%%%%%%%%%%%%%%%%%%%%%%%%%%%%%%%%%%%%%%%%%%%%
%%%%%                                                                     %%%%%
%%%%% genericcmd.tex:                                                     %%%%%
%%%%% Arquivo com as definições da Capa do Dicionário                     %%%%%
%%%%%                                                                     %%%%%
%%%%%%%%%%%%%%%%%%%%%%%%%%%%%%%%%%%%%%%%%%%%%%%%%%%%%%%%%%%%%%%%%%%%%%%%%%%%%%%
%%%%%%%%%%%%%%%%%%%%%%%%%%%%%%%%%%%%%%%%%%%%%%%%%%%%%%%%%%%%%%%%%%%%%%%%%%%%%%%

\ExplSyntaxOn

%%% Cria listas especializadas (seelist e seealsolist)
\newlist{seelist}{itemize}{1}
\newlist{seealsolist}{itemize}{1}

\setlist[seelist]{label={},topsep=0pt,nosep,noitemsep,leftmargin=\parindent}
\setlist[seealsolist]{label={},topsep=0pt,nosep,noitemsep,leftmargin=\parindent}

%%% Cria e inicializa a lista "\seerefl", "Veja"
\newcommand\seerefl{}
\listadd{\seerefl}{}% Inicializa a lista

%%% Cria e inicializa a lista "\seealsorefl", "Veja também"
\newcommand\seealsorefl{}
\listadd{\seealsorefl}{}% Inicializa a lista

%%% Comando "\seeitem", adiciona um item "Veja" ou "Veja também" na lista,
%%% com os pinyins abaixo dos caracteres
\newcommand{\seeitem}[2]{#1\ \dpy{#2}\ (p.~\pageref{#1:#2})}

%%% Comando "\definition", gera o texto da definição
\NewDocumentCommand{\definition}{sommo}
 {%
  \IfBooleanTF{#1}%
   {% Substantivo Próprio
    {\small\ding{108}}\ (\textit{Substantivo\ Próprio})\IfValueT{#2}{\ [clas.:~#2]}{\ \dictenumerate{#4}}\par
   }%
   {%
    {\small\ding{108}}\ (\textit{#3})\IfValueT{#2}{\ [clas.: #2]}{\ \dictenumerate{#4}}\par
   }%
  \IfValueT{#5}%
   {%
    \IfSubStr{#5}{|}{\textbf{Exemplos: }}{\textbf{Exemplo: }}\dictexamples{\l_hanzi_tl}{#5}
   }%
 }

%%% Comando "Variante de"
\NewDocumentCommand{\variantof}{m}
 {
  {\small\ding{108}}\ Variante\ de\ #1\ (p.~\pageref{#1:\l_pinyin_tl})\par
 }

%%% Comando "Veja"
\NewDocumentCommand{\seeref}{mm}
 {%
  \booltrue{f_see}
  \listgadd{\seerefl}{#1:#2}
 }

%%% Comando "Veja também"
\NewDocumentCommand{\seealsoref}{mm}
 {%
  \booltrue{f_seealso}
  \listgadd{\seealsorefl}{#1:#2}
 }

\ExplSyntaxOff

%%%%% EOF %%%%%

%%%%%                         %%%%%
%%%%% Dictionary environments %%%%%
%%%%%                         %%%%%

\ExplSyntaxOn

%%%%% "entry" environment
\NewDocumentEnvironment{entry}{mmooo}%
 {%
  \leavevmode%
  \markboth{#1{\tiny(#2画)}}{#1{\tiny(#2画)}}
  \tl_set:Nn \l_hanzi_tl {#1}
  \tl_set:Nn \l_strokes_tl {#2}
  \begin{minipage}[t][][t]{.49\textwidth}
   \vspace{2sp}
   \begin{tcolorbox}[size=title,colframe=black,colback=white,boxrule=1pt,toprule=2pt,left=0mm,right=0mm,top=0mm,bottom=0mm]
    {\LARGE#1}\hfill\textsuperscript{\tiny(#2画)}\\
    \IfValueT{#3}{\mbox{}\hfill{\tiny#3}}{}%
    \IfValueT{#4}{\mbox{}\hfill{\tiny#4}}{}%
    \IfValueT{#5}{\mbox{}\hfill{\tiny#5}}{}
   \end{tcolorbox}
 }%
 {%
  \end{minipage}
 }

%%%%% "entry*" environment
\NewDocumentEnvironment{entry*}{mmooo}%
 {%
  \leavevmode%
  \markboth{#1{\tiny(#2画)}}{#1{\tiny(#2画)}}
  \tl_set:Nn \l_hanzi_tl {#1}
  \tl_set:Nn \l_strokes_tl {#2}
  \begin{minipage}[t][][t]{.49\textwidth}
   \vspace{2sp}
   \begin{tcolorbox}[size=title,colframe=black,colback=white,boxrule=1pt,toprule=2pt,left=0mm,right=0mm,top=0mm,bottom=0mm]
    \mbox{}\hfill\textsuperscript{\tiny(#2画)}\\
    {\LARGE#1}\\
    \IfValueT{#3}{\mbox{}\hfill{\tiny#3}}{}%
    \IfValueT{#4}{\mbox{}\hfill{\tiny#4}}{}%
    \IfValueT{#5}{\mbox{}\hfill{\tiny#5}}{}
   \end{tcolorbox}
 }%
 {%
  \end{minipage}
 }

%%%%% "phonetics" environment
\NewDocumentEnvironment{phonetics}{mO{}mO{}}%
 {%
  \tl_set:Nn \l_pinyin_tl {#3}
  \boolfalse{f_example} \renewcommand\examplel{} \listadd{\examplel}{}% Initialize list
  \boolfalse{f_see} \renewcommand\seerefl{} \listadd{\seerefl}{}% Initialize list
  \boolfalse{f_seealso} \renewcommand\seealsorefl{} \listadd{\seealsorefl}{}% Initialize list
  \label{#1:#3}
  \index[sradical]{\l_hanzi_tl@\l_hanzi_tl \dpy{#3}}
  \ding{93}\ #2\ \dpy{#3}\ #4\ \ding{93}\\
 }
 {%  
  \ifbool{f_example}%
   {% Process "examples"
    \RenewDocumentCommand\do{m}%                                                                                                
     {%
      \IfSubStr{##1}{manual::::}%
       {% Manual underline
        \StrBehind{##1}{manual::::}[\sM]%
        \IfSubStr{\sM}{::::}%
         {% With translation
          \StrCut{\sM}{::::}{\sE}{\sT}%
          \mbox{}\enskip\sE\\
          \mbox{}\enskip$\hookrightarrow$\ \sT\\
         }%
         {% Without translation
          \mbox{}\enskip\sM\\
         }%
       }%
       {%
        \IfSubStr{##1}{::::}%
         {% With translation
          \StrCut{##1}{::::}{\sE}{\sT}%
          \mbox{}\enskip\StrSubstitute{\sE}{\l_hanzi_tl}{\underline{\l_hanzi_tl}}\\
          \mbox{}\enskip$\hookrightarrow$\ \sT\\
         }
         {%
          \IfSubStr{##1}{ERRO}{##1}%
           {% There aren't entry words in text
            \mbox{}\enskip\StrSubstitute{##1}{\l_hanzi_tl}{\underline{\l_hanzi_tl}}\\
           }%
         }%
       }%
     }%
    \textbf{Exemplos:}\\
    \dolistloop{\examplel}
   }{}%
  \ifbool{f_see}%
   {% Process "see" references
    \RenewDocumentCommand\do{>{\SplitArgument{1}{:}}m}{\item \seeitem ##1}
    \textbf{Veja:\ }%
    \begin{itemize*}[label={}, itemjoin={{,\ }}, itemjoin*={{\ e~}}]
     \dolistloop{\seerefl}
    \end{itemize*}
   }{}%
  \ifbool{f_seealso}%
   {% Process "seealso" references
    \RenewDocumentCommand\do{>{\SplitArgument{1}{:}}m}{\item \seeitem ##1}
    \textbf{Veja\ também:\ }%
    \begin{itemize*}[label={}, itemjoin={{,\ }}, itemjoin*={{\ e~}}]
     \dolistloop{\seealsorefl}
    \end{itemize*}
   }{}%
 }

\ExplSyntaxOff

%%%%% EOF %%%%%


% Ajustes do PDF
\hypersetup{
  linktoc=page,
  colorlinks=true,
  urlcolor=blue,
  linkcolor=blue,
  citecolor=blue,
  pdftitle={汉葡词典 - Dicionário Chinês-Português},
  pdfsubject={Dicionário Chinês-Português -- Ordenado por Número de Traços},
  pdfauthor={罗学凯, Luiz Eduardo Roncato Cordeiro},
  pdfkeywords={dicionário, chinês, português, instituto confúcio}
}

%%%
%%% Documento começa aqui!
%%%

\begin{document}
\addfontfeatures{CharacterWidth=Proportional}
\let\clearforchapter\par % saves some space

\begin{titlingpage}
  \raggedleft
  \rule{1pt}{\textheight}
  \hspace{0.1\textwidth}
  \parbox[b]{0.75\textwidth}{
    \vspace{0.05\textheight}
    {\HUGE\bfseries 汉葡词典}\\[2\baselineskip] % Title
    {\Large\textsc{Dicionário Chinês-Português}\\%
     \large\textsc{\zhtoday}}\\% Date
    [4\baselineskip]
    {\Large\textsc{罗学凯}\\%
     \small AKA Luiz Eduardo Roncato Cordeiro}\\% Author
    \vspace{0.5\textheight}\\%
    {Aluno do Instituto Confúcio na UNESP}\\[\baselineskip] % Publisher?
  }
  \newpage
  \raggedright
  \setlength{\parindent}{0pt}
  \setlength{\parskip}{\baselineskip}
  \mbox{}
  \vfill
  \footnotesize
  \textcopyright{} 2024-2025 por Luiz Eduardo Roncato Cordeiro, está licenciado sob CC BY-NC-SA 4.0\\
  \begin{itemize}
    \item Para visualizar uma cópia desta licença, visite:\\ \url{http://creativecommons.org/licenses/by-nc-sa/4.0/}
    \item Este trabalho ainda está em andamento e o ``código fonte'' está localizado em:\\ \url{https://github.com/lercordeiro/dicionario_chines_portugues}
    \item A última versão compilada também pode ser encontrada em:\\ \url{https://ler.cordeiro.nom.br/}
  \end{itemize}
%  \begin{tabular}{ll}
%    First edition: & T.B.D. \\
%  \end{tabular}
\end{titlingpage}


\clearpage
\pagestyle{empty}
\tableofcontents

\clearpage
\pagestyle{empty}
\chapter{汉葡词典}

%%%%%%%%%%%%%%%%%%%%%%%%
%
% https://en.wikipedia.org/wiki/Chinese_character_orders
%
%%%%%%%%%%%%%%%%%%%%%%%%

Dicionário Chinês-Português ordenado primeiro pelo número de traços,
depois pela ordem do caracter na tabela UTF-8.  As definições são
agrupadas e ordenadas pelo pinyin em cada verbete.

\clearpage
\pagestyle{dictionary}
\twocolumn
%%%
%%% 1画
%%%

\section*{1画}\addcontentsline{toc}{section}{1画}

\begin{entry}{一}{1}{⼀}[Kangxi 1]
  \begin{phonetics}{一}{yi1}[(quando usado sozinho)][HSK 1]
    \definition{adv.}{uma vez; assim que; indica que duas ações ocorreram em um intervalo de tempo muito curto, uma terminando e a outra começando imediatamente em seguida | indica que primeiro se realiza uma ação e, em seguida, o resultado dessa ação  | indica uma ação única, indicando que a ação é muito curta ou apenas uma tentativa}
    \definition{num.}{um; 1 | pronunciado como \dpy{yao1} quando dito número a número | igual; refere-se ao mesmo ou igual | inteiro; todo; por toda parte | exclusivo ou único | refere-se a algo específico | também; caso contrário; referindo-se a outro ou mais um}
    \definition{part.}{antes de certas palavras para dar ênfase}
    \definition{prep.}{cada; por; toda vez}
    \definition{s.}{uma nota da escala em Gongchepu (工尺谱), correspondente ao 17 na notação musical numerada}
  \seealsoref{工尺谱}{gong1 che3 pu3}
  \end{phonetics}
  \begin{phonetics}{一}{yi2}[(antes de quarto tom)][HSK 1]
    \definition{num.}{um; 1 | um (artigo)}
  \end{phonetics}
  \begin{phonetics}{一}{yi4}[][HSK 1]
    \definition{adv.}{uma vez | assim que | ao}
    \definition{num.}{um; 1 | um (artigo)}
  \end{phonetics}
\end{entry}

\begin{entry}{一下}{1,3}{⼀、⼀}
  \begin{phonetics}{一下}{yi2xia4}
    \definition{adv.}{em um curto tempo | rapidamente}
  \end{phonetics}
\end{entry}

\begin{entry}{一下儿}{1,3,2}{⼀、⼀、⼉}
  \begin{phonetics}{一下儿}{yi2 xia4r5}[][HSK 1,5]
    \definition{s.}{um tempo; um momento}
  \end{phonetics}
\end{entry}

\begin{entry}{一下子}{1,3,3}{⼀、⼀、⼦}
  \begin{phonetics}{一下子}{yi2 xia4 zi5}[][HSK 5]
    \definition{adv.}{tudo de uma vez; de repente; em pouco tempo; em um curto espaço de tempo}
  \end{phonetics}
\end{entry}

\begin{entry}{一个样}{1,3,10}{⼀、⼈、⽊}
  \begin{phonetics}{一个样}{yi2ge5yang4}
    \definition{adj.}{igual | mesmo}
  \seealsoref{一样}{yi2yang4}
  \end{phonetics}
\end{entry}

\begin{entry}{一口气}{1,3,4}{⼀、⼝、⽓}
  \begin{phonetics}{一口气}{yi4 kou3 qi4}[][HSK 5]
    \definition{adv.}{em um só fôlego; sem pausa}
  \end{phonetics}
\end{entry}

\begin{entry}{一切}{1,4}{⼀、⼑}
  \begin{phonetics}{一切}{yi2qie4}[][HSK 3]
    \definition{pron.}{tudo; todo; todas as coisas}
  \end{phonetics}
\end{entry}

\begin{entry}{一方面}{1,4,9}{⼀、⽅、⾯}
  \begin{phonetics}{一方面}{yi4 fang1 mian4}[][HSK 3]
    \definition{s.}{um lado; um dos dois aspectos opostos ou um lado de algo que está relacionado a outro}
  \end{phonetics}
\end{entry}

\begin{entry}{一方面……,一方面……}{1,4,9,1,4,9}{⼀、⽅、⾯、⼀、⽅、⾯}
  \begin{phonetics}{一方面……,一方面……}{yi4 fang1 mian4 yi4 fang1 mian4}[][HSK 3]
    \definition{conj.}{por um lado\dots, por outro lado\dots; conecta duas orações paralelas (devem ser usadas juntas)}
  \end{phonetics}
\end{entry}

\begin{entry}{一半}{1,5}{⼀、⼗}
  \begin{phonetics}{一半}{yi2ban4}[][HSK 1]
    \definition{num.}{metade; em parte; uma metade}
  \end{phonetics}
\end{entry}

\begin{entry}{一句话}{1,5,8}{⼀、⼝、⾔}
  \begin{phonetics}{一句话}{yi2 ju4 hua4}[][HSK 5]
    \definition{s.}{em resumo; em uma palavra; expressar um conteúdo complexo de forma sucinta | trabalho fácil; fácil de fazer; descrever uma tarefa ou trabalho como muito simples e fácil de realizar}
  \end{phonetics}
\end{entry}

\begin{entry}{一旦}{1,5}{⼀、⽇}
  \begin{phonetics}{一旦}{yi2dan4}[][HSK 5]
    \definition{adv.}{uma vez; no caso; agora que | de repente; uma vez}
    \definition{s.}{em um único dia; em um tempo muito curto;}
  \end{phonetics}
\end{entry}

\begin{entry}{一生}{1,5}{⼀、⽣}
  \begin{phonetics}{一生}{yi4 sheng1}[][HSK 2]
    \definition{s.}{toda a vida | ao longo da vida | a vida de alguém}
  \end{phonetics}
\end{entry}

\begin{entry}{一边}{1,5}{⼀、⾡}
  \begin{phonetics}{一边}{yi4bian1}[][HSK 1]
    \definition{adj.}{igual; idêntico; da mesma forma}
    \definition{adv.}{enquanto; ao mesmo tempo; simultaneamente; indica que uma ação ocorre simultaneamente a outra ação}
    \definition{s.}{lado; um lado; um aspecto | ambos os lados; ao lado de}
  \end{phonetics}
\end{entry}

\begin{entry}{一会儿}{1,6,2}{⼀、⼈、⼉}
  \begin{phonetics}{一会儿}{yi2 hui4r5}[][HSK 1,2]
    \definition{adv.}{agora\dots agora\dots; um momento\dots o próximo\dots; usado antes de dois antônimos, indica a alternância de situações}
    \definition{s.}{um pouquinho de tempo; muito pouco tempo}
  \end{phonetics}
\end{entry}

\begin{entry}{一共}{1,6}{⼀、⼋}
  \begin{phonetics}{一共}{yi2gong4}[][HSK 2]
    \definition{adv.}{completamente | no total | no todo | em suma}
  \end{phonetics}
\end{entry}

\begin{entry}{一再}{1,6}{⼀、⼌}
  \begin{phonetics}{一再}{yi2zai4}[][HSK 4]
    \definition{adv.}{repetidamente; de novo e de novo; repetidas vezes}
  \end{phonetics}
\end{entry}

\begin{entry}{一同}{1,6}{⼀、⼝}
  \begin{phonetics}{一同}{yi4tong2}
    \definition{adv.}{juntos, ao mesmo tempo}
  \end{phonetics}
\end{entry}

\begin{entry}{一向}{1,6}{⼀、⼝}
  \begin{phonetics}{一向}{yi2xiang4}[][HSK 5]
    \definition{adv.}{desde o início; indica do passado até o presente}
  \end{phonetics}
\end{entry}

\begin{entry}{一行}{1,6}{⼀、⾏}
  \begin{phonetics}{一行}{yi1xing2}
    \definition{s.}{festa | delegação}
  \end{phonetics}
\end{entry}

\begin{entry}{一齐}{1,6}{⼀、⿑}
  \begin{phonetics}{一齐}{yi4qi2}
    \definition{adv.}{tudo ao mesmo tempo | em uníssono | junto}
  \end{phonetics}
\end{entry}

\begin{entry}{一块}{1,7}{⼀、⼟}
  \begin{phonetics}{一块}{yi2kuai4}
    \definition{adv.}{(principalmente mandarim) juntos}
  \end{phonetics}
\end{entry}

\begin{entry}{一块儿}{1,7,2}{⼀、⼟、⼉}
  \begin{phonetics}{一块儿}{yi2 kuai4r5}[][HSK 1]
    \definition{adv.}{juntos; em conjunto}
    \definition{s.}{no mesmo lugar; no mesmo local}
  \end{phonetics}
\end{entry}

\begin{entry}{一时}{1,7}{⼀、⽇}
  \begin{phonetics}{一时}{yi4shi2}
    \definition{adv.}{por pouco tempo | por um tempo | temporariamente | momentaneamente | uma vez | de tempos em tempos | ocasionalmente}
  \end{phonetics}
\end{entry}

\begin{entry}{一身}{1,7}{⼀、⾝}
  \begin{phonetics}{一身}{yi4 shen1}[][HSK 5]
    \definition{s.}{o corpo inteiro; em todo o corpo | um terno; (um conjunto completo de) roupas | sozinho; uma única pessoa; relativo a uma única pessoa}
  \end{phonetics}
\end{entry}

\begin{entry}{一些}{1,8}{⼀、⼆}
  \begin{phonetics}{一些}{yi4 xie1}[][HSK 1]
    \definition{clas.}{alguns; um número de; quantidade indeterminada | um pouco; uma pequena quantidade | mais de um; mais de uma vez; indica mais de um ou mais de uma vez, etc. | uma ligeira mudança no grau, intensidade; usado após certos verbos, adjetivos, etc., para indicar uma quantidade muito pequena}
    \definition{pron.}{uns; alguns}
  \end{phonetics}
\end{entry}

\begin{entry}{一定}{1,8}{⼀、⼧}
  \begin{phonetics}{一定}{yi2ding4}[][HSK 2]
    \definition{adv.}{certamente | definitivamente}
  \end{phonetics}
\end{entry}

\begin{entry}{一直}{1,8}{⼀、⽬}
  \begin{phonetics}{一直}{yi4zhi2}[][HSK 2]
    \definition{adv.}{diretamente | sempre em frente | o tempo todo | sempre | constantemente}
  \end{phonetics}
\end{entry}

\begin{entry}{一带}{1,9}{⼀、⼱}
  \begin{phonetics}{一带}{yi2 dai4}[][HSK 5]
    \definition{s.}{a área em torno de um determinado local; refere-se a um determinado local e suas proximidades.}
  \end{phonetics}
\end{entry}

\begin{entry}{一律}{1,9}{⼀、⼻}
  \begin{phonetics}{一律}{yi2lv4}[][HSK 4]
    \definition{adj.}{igual; semelhante; uniforme; parecido; idêntico}
    \definition{adv.}{todos; tudo; sem exceção; enfatiza que todos devem ser assim, sem exceção, e é usado principalmente em regulamentos ou requisitos}
  \end{phonetics}
\end{entry}

\begin{entry}{一战}{1,9}{⼀、⼽}
  \begin{phonetics}{一战}{yi2zhan4}
    \definition*{s.}{Primeira Guerra Mundial}
  \end{phonetics}
\end{entry}

\begin{entry}{一点儿}{1,9,2}{⼀、⽕、⼉}
  \begin{phonetics}{一点儿}{yi4dian3r5}[][HSK 1]
    \definition{adv.}{um pouco; uma pitada; uma gota; uma amostra; uma pequena quantidade; ({adj.} + (一)点儿, 一点儿 + {s.} ou 有 + (一)点儿 + {s.})}
  \end{phonetics}
\end{entry}

\begin{entry}{一点点}{1,9,9}{⼀、⽕、⽕}
  \begin{phonetics}{一点点}{yi4 dian3 dian3}[][HSK 2]
    \definition{adj.}{um pouco}
  \end{phonetics}
\end{entry}

\begin{entry}{一样}{1,10}{⼀、⽊}
  \begin{phonetics}{一样}{yi2yang4}[][HSK 1]
    \definition{adj.}{o mesmo; igualmente; semelhante; tão\dots quanto\dots}
    \definition{part.}{na mesma medida; anexado a verbos ou palavras nominais, indica uma comparação ou semelhança, equivalente a 似的}
  \seealsoref{似的}{shi4de5}
  \end{phonetics}
\end{entry}

\begin{entry}{一流}{1,10}{⼀、⽔}
  \begin{phonetics}{一流}{yi4liu2}[][HSK 5]
    \definition{adj.}{clássico; de primeira linha; de primeira classe; o melhor}
    \definition[些]{s.}{tipo; mesmo tipo; da mesma classe; da mesma categoria; uma categoria}
  \end{phonetics}
\end{entry}

\begin{entry}{一致}{1,10}{⼀、⾄}
  \begin{phonetics}{一致}{yi2zhi4}[][HSK 4]
    \definition{adj.}{equado; idêntico; uniforme; unânime; nenhuma diferença (de opinião ou ação)}
    \definition{adv.}{juntos; em conjunto}
  \end{phonetics}
\end{entry}

\begin{entry}{一般}{1,10}{⼀、⾈}
  \begin{phonetics}{一般}{yi4ban1}[][HSK 2]
    \definition{adj.}{geral | comum | normal}
    \definition{adv.}{normalmente}
  \end{phonetics}
\end{entry}

\begin{entry}{一般来说}{1,10,7,9}{⼀、⾈、⽊、⾔}
  \begin{phonetics}{一般来说}{yi4 ban1 lai2 shuo1}[][HSK 4]
    \definition{expr.}{de modo geral; na média; no caso usual; a declaração usual}
  \end{phonetics}
\end{entry}

\begin{entry}{一起}{1,10}{⼀、⾛}
  \begin{phonetics}{一起}{yi4qi3}[][HSK 1]
    \definition{adv.}{juntos; em companhia; indica o mesmo local, ao mesmo tempo que se faz algo | no total; em todos; no conjunto}
    \definition{s.}{no mesmo lugar}
  \end{phonetics}
\end{entry}

\begin{entry}{一部分}{1,10,4}{⼀、⾢、⼑}
  \begin{phonetics}{一部分}{yi2 bu4 fen4}[][HSK 2]
    \definition{adv.}{parcialmente}
    \definition{num.}{parte | porção | seção | fração}
    \definition[把]{s.}{parcial}
  \end{phonetics}
\end{entry}

\begin{entry}{一……就……}{1,12}{⼀、⼪}
  \begin{phonetics}{一……就……}{yi1 jiu4}
    \definition{expr.}{logo que |  uma vez que}
  \end{phonetics}
\end{entry}

\begin{entry}{一辈子}{1,12,3}{⼀、⾞、⼦}
  \begin{phonetics}{一辈子}{yi2bei4zi5}[][HSK 5]
    \definition{s.}{uma vida inteira; vida inteira; toda a vida; durante toda a vida; enquanto se vive; todo o tempo entre o nascimento e a morte}
  \end{phonetics}
\end{entry}

\begin{entry}{一道}{1,12}{⼀、⾡}
  \begin{phonetics}{一道}{yi2dao4}
    \definition{adv.}{juntos | ao lado}
  \end{phonetics}
\end{entry}

\begin{entry}{一路}{1,13}{⼀、⾜}
  \begin{phonetics}{一路}{yi2 lu4}[][HSK 5]
    \definition{adv.}{o tempo todo; persistentemente; continuamente | juntos; sem parar; continuamente}
    \definition{s.}{o mesmo caminho; a mesma rota; ao longo de toda a viagem, ao longo do caminho | do mesmo tipo; da mesma categoria}
  \end{phonetics}
\end{entry}

\begin{entry}{一路平安}{1,13,5,6}{⼀、⾜、⼲、⼧}
  \begin{phonetics}{一路平安}{yi2 lu4 ping2 an1}[][HSK 2]
    \definition{expr.}{Boa viagem!}
    \definition{v.}{ter uma viagem agradável}
  \end{phonetics}
\end{entry}

\begin{entry}{一路顺风}{1,13,9,4}{⼀、⾜、⾴、⾵}
  \begin{phonetics}{一路顺风}{yi2 lu4 shun4 feng1}[][HSK 2]
    \definition{expr.}{ter uma viagem agradável}
  \end{phonetics}
\end{entry}

\begin{entry}{乙}{1}{⼄}
  \begin{phonetics}{乙}{yi3}[][HSK 5]
    \definition*{s.}{sobrenome Yi}
    \definition*{s.}{o segundo lugar do Tian Gan}
    \definition{num.}{segundo}
    \definition{s.}{uma nota da escala em Gongchepu (工尺谱); nível superior na música tradicional chinesa}
  \seealsoref{工尺谱}{gong1 che3 pu3}
  \end{phonetics}
\end{entry}

%%%%% EOF %%%%%


%%%
%%% 2画
%%%

\section*{2画}\addcontentsline{toc}{section}{2画}

\begin{entry}{七}{2}{⼀}
  \begin{phonetics}{七}{qi1}[][HSK 1]
    \definition*{s.}{Sobrenome Qi}
    \definition{num.}{sete; 7}
    \definition{s.}{antigamente, os mortos eram homenageados a cada sete dias, chamados de 七, até o quadragésimo nono dia, num total de sete 七}
  \end{phonetics}
\end{entry}

\begin{entry}{七夕}{2,3}{⼀、⼣}
  \begin{phonetics}{七夕}{qi1xi1}
    \definition*{s.}{Dia dos Namorados Chinês, quando o vaqueiro e a tecelã (牛郎织女) têm permissão para se reunirem anualmente | Festival das Meninas | Festival Duplo Sete, noite do sétimo mês lunar}
  \seealsoref{牛郎织女}{niu2lang2zhi1nv3}
  \end{phonetics}
\end{entry}

\begin{entry}{九}{2}{⼄}
  \begin{phonetics}{九}{jiu3}[][HSK 1]
    \definition*{s.}{Sobrenome Jiu}
    \definition{adj.}{muitos; numerosos; indica várias vezes ou a maioria das vezes}
    \definition{num.}{nove; 9}
    \definition{s.}{cada um dos nove períodos de nove dias começando no dia seguinte ao solstício de inverno}
  \end{phonetics}
\end{entry}

\begin{entry}{了}{2}{⼅}
  \begin{phonetics}{了}{le5}[][HSK 1,3]
    \definition{part.}{usada após verbos ou adjetivos para indicar a conclusão de uma ação, em um momento no passado ou antes do início de outra ação, ou uma ação esperada ou presumida | usada para indicar uma mudança de situação ou estado, seja real ou prevista | comandos ou solicitações em resposta a uma situação alterada; usada para xpressar urgência ou dissuadir | usada para indicar que algo chegou ao extremo; usada no final da frase ou em pausas no meio da frase, para expressar um tom de exclamação}
  \end{phonetics}
  \begin{phonetics}{了}{liao3}
    \definition*{s.}{Sobrenome Liao}
    \definition{adv.}{inteiramente; um pouco; totalmente (mais usado em negativas)}
    \definition{v.}{terminar; concluir; encerrar; cumprir; eliminar; resolver | compreender; saber; perceber; saber claramente | expressar possibilidade ou impossibilidade; usado com 得 ou 不 após o verbo, indica possibilidade ou impossibilidade}
  \seealsoref{不}{bu4}
  \seealsoref{得}{de5}
  \end{phonetics}
\end{entry}

\begin{entry}{了不起}{2,4,10}{⼅、⼀、⾛}
  \begin{phonetics}{了不起}{liao3bu5qi3}[][HSK 4]
    \definition{adj.}{incrível; fantástico; extraordinário | sério; grave}
  \end{phonetics}
\end{entry}

\begin{entry}{了解}{2,13}{⼅、⾓}
  \begin{phonetics}{了解}{liao3jie3}[][HSK 4]
    \definition{v.}{entender; compreender | investigar; indagar sobre}
  \end{phonetics}
\end{entry}

\begin{entry}{二}{2}{⼆}[Kangxi 7]
  \begin{phonetics}{二}{er4}[][HSK 1]
    \definition{adj.}{diferente; refere-se a duas coisas ou coisas diferentes | bobo; pateta; tolo; sem inteligência | desleal; infiel; indiferente; sem determinação}
    \definition{num.}{dois; 2}
  \end{phonetics}
\end{entry}

\begin{entry}{二手}{2,4}{⼆、⼿}
  \begin{phonetics}{二手}{er4 shou3}[][HSK 4]
    \definition{adj.}{usado; de segunda mão; refere-se especificamente a usados e revendidos}
  \end{phonetics}
\end{entry}

\begin{entry}{二战}{2,9}{⼆、⼽}
  \begin{phonetics}{二战}{er4zhan4}
    \definition*{s.}{Segunda Guerra Mundial}
  \end{phonetics}
\end{entry}

\begin{entry}{二胡}{2,9}{⼆、⾁}
  \begin{phonetics}{二胡}{er4hu2}
    \definition{s.}{erhu; um instrumento de arco de duas cordas com um registro mais baixo que o 京胡; um tipo de 胡琴, a caixa de som é feita de bambu, madeira, etc., coberta com pele de cobra, etc., tem duas cordas e o tom é baixo e suave}
  \seealsoref{胡琴}{hu2qin2}
  \seealsoref{京胡}{jing1hu2}
  \end{phonetics}
\end{entry}

\begin{entry}{二维码}{2,11,8}{⼆、⽷、⽯}
  \begin{phonetics}{二维码}{er4 wei2 ma3}[][HSK 5]
    \definition{s.}{\emph{QR code}}
  \end{phonetics}
\end{entry}

\begin{entry}{人}{2}{⼈}[Kangxi 9]
  \begin{phonetics}{人}{ren2}[][HSK 1]
    \definition*{s.}{Sobrenome Ren}
    \definition[个,名,位]{s.}{homem; pessoa; pessoas; ser humano | todos; cada um; todo mundo | adulto; crescido | uma pessoa envolvida em uma atividade específica | pessoas; outras pessoas | caráter; personalidade; qualidade, caráter ou reputação de uma pessoa | como alguém se sente; estado de saúde de alguém | mão de obra; força de trabalho}
  \end{phonetics}
\end{entry}

\begin{entry}{人力}{2,2}{⼈、⼒}
  \begin{phonetics}{人力}{ren2 li4}[][HSK 5]
    \definition{s.}{mão de obra; trabalho manual; força de trabalho}
  \end{phonetics}
\end{entry}

\begin{entry}{人力车}{2,2,4}{⼈、⼒、⾞}
  \begin{phonetics}{人力车}{ren2 li4 che1}
    \definition{s.}{veículo de duas rodas puxado ou empurrado por um homem (oposto a 兽力车 e 机动车) | Datado: riquixá | uma carroça puxada ou empurrada por humanos}
  \seealsoref{机动车}{ji1 dong4 che1}
  \seealsoref{兽力车}{shou4 li4 che1}
  \end{phonetics}
\end{entry}

\begin{entry}{人口}{2,3}{⼈、⼝}
  \begin{phonetics}{人口}{ren2kou3}[][HSK 2]
    \definition[个,群]{s.}{população; o número total de pessoas que vivem em uma determinada região durante um determinado período de tempo | número de membros da família; o número total de pessoas em uma família | pessoas; público; população; referência geral a pessoas | rumores do povo; referindo-se à opinião pública}
  \end{phonetics}
\end{entry}

\begin{entry}{人士}{2,3}{⼈、⼠}
  \begin{phonetics}{人士}{ren2shi4}[][HSK 5]
    \definition{s.}{pessoa; figura; personalidade; figura pública; pessoas com certa influência social}
  \end{phonetics}
\end{entry}

\begin{entry}{人工}{2,3}{⼈、⼯}
  \begin{phonetics}{人工}{ren2gong1}[][HSK 3]
    \definition{adj.}{feito pelo homem; artificial (oposto a 天然)}
    \definition[个]{s.}{trabalho manual; trabalho feito à mão | mão de obra; homem-dia; uma unidade de cálculo da quantidade de trabalho realizado}
  \seealsoref{天然}{tian1 ran2}
  \end{phonetics}
\end{entry}

\begin{entry}{人工智能}{2,3,12,10}{⼈、⼯、⽇、⾁}
  \begin{phonetics}{人工智能}{ren2 gong1 zhi4 neng2}
    \definition*{s.}{Inteligência Artificial (IA)}
  \end{phonetics}
\end{entry}

\begin{entry}{人才}{2,3}{⼈、⼿}
  \begin{phonetics}{人才}{ren2cai2}[][HSK 3]
    \definition{adj.}{aparência bonita, elegante}
    \definition[个,些,位]{s.}{talento; pessoal qualificado; pessoa com capacidade; uma pessoa com capacidade e integridade política; uma pessoa com talentos especiais | aparência bonita; refere-se à aparência; especialmente à aparência bonita}
  \end{phonetics}
\end{entry}

\begin{entry}{人们}{2,5}{⼈、⼈}
  \begin{phonetics}{人们}{ren2 men5}[][HSK 2]
    \definition{s.}{homens; pessoas; o público; referindo-se a muitas pessoas; todos}
  \end{phonetics}
\end{entry}

\begin{entry}{人民}{2,5}{⼈、⽒}
  \begin{phonetics}{人民}{ren2 min2}[][HSK 3]
    \definition[群,批,个,国]{s.}{o povo; refere-se a um certo tipo de pessoas; membros básicos da sociedade com as massas trabalhadoras como o corpo principal}
  \end{phonetics}
\end{entry}

\begin{entry}{人民币}{2,5,4}{⼈、⽒、⼱}
  \begin{phonetics}{人民币}{ren2min2bi4}[][HSK 3]
    \definition*[块,张,元]{s.}{Renminbi (RMB); Yuan Chinês (CYN); nome da moeda chinesa}
  \end{phonetics}
\end{entry}

\begin{entry}{人生}{2,5}{⼈、⽣}
  \begin{phonetics}{人生}{ren2sheng1}[][HSK 3]
    \definition{s.}{vida; sobrevivência e vida humana}
  \end{phonetics}
\end{entry}

\begin{entry}{人权}{2,6}{⼈、⽊}
  \begin{phonetics}{人权}{ren2quan2}[][HSK 6]
    \definition{s.}{direitos humanos}[最基本的人权是生存权。___O direito humano mais básico é o direito à vida.]
  \seealsoref{人权法}{ren2quan2fa3}
  \end{phonetics}
\end{entry}

\begin{entry}{人权法}{2,6,8}{⼈、⽊、⽔}
  \begin{phonetics}{人权法}{ren2quan2fa3}
    \definition*{s.}{Direitos Humanos}
  \seealsoref{人权}{ren2quan2}
  \end{phonetics}
\end{entry}

\begin{entry}{人行道}{2,6,12}{⼈、⾏、⾡}
  \begin{phonetics}{人行道}{ren2xing2dao4}
    \definition{s.}{calçada}
  \end{phonetics}
\end{entry}

\begin{entry}{人员}{2,7}{⼈、⼝}
  \begin{phonetics}{人员}{ren2yuan2}[][HSK 3]
    \definition[个,位,名]{s.}{funcionários ; uma pessoa que ocupa uma determinada posição| pessoal; membros de um grupo}
  \end{phonetics}
\end{entry}

\begin{entry}{人材}{2,7}{⼈、⽊}
  \begin{phonetics}{人材}{ren2cai2}
    \variantof{人才}
  \end{phonetics}
\end{entry}

\begin{entry}{人间}{2,7}{⼈、⾨}
  \begin{phonetics}{人间}{ren2jian1}[][HSK 5]
    \definition{s.}{o mundo humano | o Mundo; a Terra}
  \end{phonetics}
\end{entry}

\begin{entry}{人物}{2,8}{⼈、⽜}
  \begin{phonetics}{人物}{ren2wu4}[][HSK 5]
    \definition[个,位,名]{s.}{personagem; personagens criados em obras literárias e artísticas | figura; personalidade; homem influente; refere-se a pessoas com grande talento e status; também se refere a pessoas com certas características ou que são representativas em algum aspecto | pintura figurativa; um tipo de pintura tradicional chinesa com personagens como tema}
  \end{phonetics}
\end{entry}

\begin{entry}{人鱼}{2,8}{⼈、⿂}
  \begin{phonetics}{人鱼}{ren2yu2}
    \definition{s.}{sereia | peixe-boi | salamandra gigante}
  \end{phonetics}
\end{entry}

\begin{entry}{人类}{2,9}{⼈、⽶}
  \begin{phonetics}{人类}{ren2lei4}[][HSK 3]
    \definition[种]{s.}{humano; humanidade; raça humana; um termo geral para pessoas}
  \end{phonetics}
\end{entry}

\begin{entry}{人家}{2,10}{⼈、⼧}
  \begin{phonetics}{人家}{ren2jia1}[][HSK 4]
    \definition[对]{s.}{lar; família; família do noivo; casa do futuro marido}
  \end{phonetics}
  \begin{phonetics}{人家}{ren2jia5}
    \definition{pron.}{outros; uma pessoa ou pessoas diferentes do falante ou ouvinte; refere-se a alguém diferente de si mesmo ou de outra pessoa | certa pessoa ou pessoas (a pessoa ou pessoas mencionadas em um contexto próximo, aproximadamente equivalente ao pronome de terceira pessoa);  refere-se a uma pessoa ou algumas pessoas, com significado semelhante a 他 | eu; mim (usado retoricamente no lugar do primeiro pronome pessoal, muitas vezes expressando descontentamento de forma jocosa; geralmente usado quando se fala com pessoas próximas, para significar 自己, usado principamente por meninas)}
  \seealsoref{他}{ta1}
  \seealsoref{自己}{zi4ji3}
  \end{phonetics}
\end{entry}

\begin{entry}{人海}{2,10}{⼈、⽔}
  \begin{phonetics}{人海}{ren2hai3}
    \definition{s.}{uma multidão | um mar de pessoas}
  \end{phonetics}
\end{entry}

\begin{entry}{人道}{2,12}{⼈、⾡}
  \begin{phonetics}{人道}{ren2dao4}
    \definition{s.}{solidariedade humana | humanitarismo | humano | a ``maneira humana'', um dos estágios do ciclo de reencarnação (budismo) | relação sexual}
  \end{phonetics}
\end{entry}

\begin{entry}{人像}{2,13}{⼈、⼈}
  \begin{phonetics}{人像}{ren2xiang4}
    \definition{s.}{``retrato'' de uma pessoa (esboço, foto, escultura, etc.)}
  \end{phonetics}
\end{entry}

\begin{entry}{人数}{2,13}{⼈、⽁}
  \begin{phonetics}{人数}{ren2 shu4}[][HSK 2]
    \definition{s.}{número de pessoas; significa o número total de pessoas, uma quantidade de pessoas; normalmente, usa-se números para fazer estatísticas específicas, mas às vezes também se usa um intervalo aproximado para fazer estimativas}
  \end{phonetics}
\end{entry}

\begin{entry}{人群}{2,13}{⼈、⽺}
  \begin{phonetics}{人群}{ren2 qun2}[][HSK 3]
    \definition[个,类]{s.}{multidão; ajuntamento; torpel; aglomeração; um grupo de pessoas}
  \end{phonetics}
\end{entry}

\begin{entry}{儿}{2}{⼉}[Kangxi 10]
  \begin{phonetics}{儿}{er2}
    \definition{adj.}{macho}
    \definition{s.}{criança | jovem; juventude | filho}
    \definition{suf.}{adicionado a substantivos para expressar pequenez  | adicionado a verbos, adjetivos e classificadores para formar substantivos | adicionado a substantivos para formar substantivos com significados diferentes | sufixos de alguns verbos | anexado após adjetivos duplicados}
  \end{phonetics}
  \begin{phonetics}{儿}{r5}
    \definition{suf.}{sufixo diminutivo não silábico | final retroflexo, pronunciado como ``r'' | adicionado a substantivos para expressar pequenez  | adicionado a verbos, adjetivos e classificadores para formar substantivos | adicionado a substantivos para formar substantivos com significados diferentes | sufixos de alguns verbos | anexado após adjetivos duplicados}
  \end{phonetics}
\end{entry}

\begin{entry}{儿女}{2,3}{⼉、⼥}
  \begin{phonetics}{儿女}{er2 nv3}[][HSK 5]
    \definition{s.}{crianças; filhos e filhas | homem e mulher jovens (apaixonados)}
  \end{phonetics}
\end{entry}

\begin{entry}{儿子}{2,3}{⼉、⼦}
  \begin{phonetics}{儿子}{er2zi5}[][HSK 1]
    \definition[个]{s.}{filho}
  \seealsoref{女儿}{nv3'er2}
  \end{phonetics}
\end{entry}

\begin{entry}{儿科}{2,9}{⼉、⽲}
  \begin{phonetics}{儿科}{er2 ke1}[][HSK 6]
    \definition{s.}{(departamento de) pediatria | pediatria; o ramo da medicina que trata do desenvolvimento, cuidado e doença das crianças}
  \end{phonetics}
\end{entry}

\begin{entry}{儿童}{2,12}{⼉、⽴}
  \begin{phonetics}{儿童}{er2tong2}[][HSK 4]
    \definition[个,群]{s.}{criança; menor de idade (mais jovem do que 少年)}
  \seealsoref{少年}{shao4 nian2}
  \end{phonetics}
\end{entry}

\begin{entry}{儿媳}{2,13}{⼉、⼥}
  \begin{phonetics}{儿媳}{er2xi2}
    \definition{s.}{esposa do filho}
  \end{phonetics}
\end{entry}

\begin{entry}{入}{2}{⼊}[Kangxi 11]
  \begin{phonetics}{入}{ru4}[][HSK 6]
    \definition{s.}{renda | tom de entrada}
    \definition{v.}{entrar; entrar (oposto a 出) | juntar-se; ser admitido em; tornar-se membro de | conformar-se com; concordar com | alcançar; atingir; entrar em (um certo nível ou estado) | fazer entrar; fazer algo entrar; fazer entrada}
  \seealsoref{出}{chu1}
  \end{phonetics}
\end{entry}

\begin{entry}{入乡随俗}{2,3,11,9}{⼊、⼄、⾩、⼈}
  \begin{phonetics}{入乡随俗}{ru4xiang1-sui2su2}
    \definition{expr.}{Em roma, faça como os romanos!}
  \end{phonetics}
\end{entry}

\begin{entry}{入口}{2,3}{⼊、⼝}
  \begin{phonetics}{入口}{ru4kou3}[][HSK 2]
    \definition[个]{s.}{entrada; entrada em locais, edifícios, estradas, etc., através de portões ou portas}
    \definition{v.}{entrar na boca | importar; mercadorias estrangeiras importadas, às vezes também se refere a mercadorias de outras regiões importadas para esta região}
  \end{phonetics}
\end{entry}

\begin{entry}{入门}{2,3}{⼊、⾨}
  \begin{phonetics}{入门}{ru4 men2}[][HSK 5]
    \definition{s.}{(geralmente em títulos de livros) curso básico; manual introdutório | ABC; guia; refere-se a leituras básicas; conhecimentos básicos}
    \definition{v.+compl.}{ultrapassar o limiar; aprender os rudimentos de um assunto | aprender o ABC de; ser introduzido a um assunto; aprender o básico}
  \end{phonetics}
\end{entry}

\begin{entry}{入学}{2,8}{⼊、⼦}
  \begin{phonetics}{入学}{ru4 xue2}[][HSK 6]
    \definition{v.+compl.}{(de uma criança) começar a escola; começar a escola primária | entrar em uma escola; matricular-se em uma escola}
  \end{phonetics}
\end{entry}

\begin{entry}{入党}{2,10}{⼊、⼉}
  \begin{phonetics}{入党}{ru4dang3}
    \definition{v.}{ingressar em um partido político (especialmente o partido comunista)}
  \end{phonetics}
\end{entry}

\begin{entry}{入境}{2,14}{⼊、⼟}
  \begin{phonetics}{入境}{ru4jing4}
    \definition{s.}{imigração}
    \definition{v.+compl.}{entrar em um país | imigrar}
  \end{phonetics}
\end{entry}

\begin{entry}{八}{2}{⼋}[Kangxi 12]
  \begin{phonetics}{八}{ba1}[][HSK 1]
    \definition{num.}{oito; 8}
  \end{phonetics}
\end{entry}

\begin{entry}{八八六}{2,2,4}{⼋、⼋、⼋}
  \begin{phonetics}{八八六}{ba1 ba1 liu4}
    \definition{expr.}{\emph{Bye bye!}, em salas de bate-papo e mensagens de texto}
  \end{phonetics}
\end{entry}

\begin{entry}{几}{2}{⼏}[Kangxi 16]
  \begin{phonetics}{几}{ji1}
    \definition{adv.}{quase; praticamente}
    \definition{s.}{uma mesa pequena}
  \end{phonetics}
  \begin{phonetics}{几}{ji3}[][HSK 1]
    \definition{adv.}{quanto?, usado para perguntar sobre quantidade e tempo}
    \definition{num.}{alguns; vários; poucos; indica um número indeterminado maior que um e menor que dez}
  \end{phonetics}
\end{entry}

\begin{entry}{几乎}{2,5}{⼏、⼃}
  \begin{phonetics}{几乎}{ji1hu1}[][HSK 4]
    \definition{adv.}{quase; praticamente; próximo a | perto de; quase; à beira de}
  \end{phonetics}
\end{entry}

\begin{entry}{几何}{2,7}{⼏、⼈}
  \begin{phonetics}{几何}{ji3he2}
    \definition{s.}{geometria}
  \end{phonetics}
\end{entry}

\begin{entry}{几率}{2,11}{⼏、⽞}
  \begin{phonetics}{几率}{ji1lv4}
    \definition{s.}{probabilidade; um evento pode ou não ocorrer sob as mesmas condições, a grandeza que indica a possibilidade de ocorrência é chamada de probabilidade}
  \end{phonetics}
\end{entry}

\begin{entry}{刀}{2}{⼑}[Kangxi 18]
  \begin{phonetics}{刀}{dao1}[][HSK 3]
    \definition*{s.}{Sobrenome Dao}
    \definition{clas.}{unidade de medida para papel, geralmente cem folhas por pacote}
    \definition[把,口]{s.}{faca; espada; armas antigas, referindo-se a ferramentas para cortar, retalhar, raspar, golpear e fatiar, geralmente feitas de ferro e aço | ferramenta; ferramenta de corte; lâminas para tornos; fresas (ferramentas; ferramentas de ferro para máquinas) | algo com a forma de uma faca}
  \end{phonetics}
\end{entry}

\begin{entry}{力}{2}{⼒}[Kangxi 19]
  \begin{phonetics}{力}{li4}[][HSK 3]
    \definition*{s.}{Sobrenome Li}
    \definition{adj.}{forte; eficiente; capaz | forte; poderoso; referência geral à função das coisas}
    \definition{adv.}{energicamente; arduamente; vigorosamente; com todo o esforço; com toda a dedicação}
    \definition{s.}{força; energia; poder; (física) refere-se à ação de alterar o estado de movimento ou a forma de um objeto |poder; força; habilidade; capacidade; funções dos órgãos do corpo humano | força física; resistência física}
  \end{phonetics}
\end{entry}

\begin{entry}{力气}{2,4}{⼒、⽓}
  \begin{phonetics}{力气}{li4qi5}[][HSK 4]
    \definition[点,把]{s.}{força física | esforço}
  \end{phonetics}
\end{entry}

\begin{entry}{力量}{2,12}{⼒、⾥}
  \begin{phonetics}{力量}{li4liang5}[][HSK 3]
    \definition[出]{s.}{força física; força espiritual | habilidade; capacidade | eficácia; efeito | força (pessoa ou grupo que tem muito poder ou influência); referência a uma pessoa ou grupo que pode desempenhar um papel importante}
  \end{phonetics}
\end{entry}

\begin{entry}{十}{2}{⼗}[Kangxi 24]
  \begin{phonetics}{十}{shi2}[][HSK 1]
    \definition*{s.}{Sobrenome Shi}
    \definition{num.}{dez; 10 | dezena | completo; no topo; máximo; referindo-se a algo que atingiu o ápice da perfeição ou plenitude | um monte de; indica que há muitos}
  \end{phonetics}
\end{entry}

\begin{entry}{十分}{2,4}{⼗、⼑}
  \begin{phonetics}{十分}{shi2fen1}[][HSK 2]
    \definition{adv.}{muito; totalmente; completamente; extremamente; indica um nível muito alto}
  \end{phonetics}
\end{entry}

\begin{entry}{十足}{2,7}{⼗、⾜}
  \begin{phonetics}{十足}{shi2zu2}[][HSK 5]
    \definition{adj.}{puro e simples; apenas este componente ou esta característica é muito evidente | 100\%; completo; total; muito satisfatório; muito adequado}
  \end{phonetics}
\end{entry}

\begin{entry}{厂}{2}{⼚}[Kangxi 27]
  \begin{phonetics}{厂}{an1}
    \definition{s.}{usado principalmente em nomes pessoais}[他名中有个厂字。___O nome dele contém a palavra `An'.]
  \end{phonetics}
  \begin{phonetics}{厂}{chang3}[][HSK 3]
    \definition[家,间]{s.}{fábrica; moinho; planta; obra | pátio; depósito; refere-se a um estabelecimento comercial com um amplo espaço para armazenamento de mercadorias e processamento}
  \end{phonetics}
  \begin{phonetics}{厂}{han3}
    \definition[家,间]{s.}{radical ``penhasco'' em caracteres chineses (radical Kangxi 27)}
  \end{phonetics}
\end{entry}

\begin{entry}{厂长}{2,4}{⼚、⾧}
  \begin{phonetics}{厂长}{chang3 zhang3}[][HSK 5]
    \definition[任,位,个]{s.}{diretor de fábrica; gerente de fábrica; líder responsável pela produção, pela vida e por todos os outros assuntos de toda a fábrica}
  \end{phonetics}
\end{entry}

\begin{entry}{厂商}{2,11}{⼚、⼝}
  \begin{phonetics}{厂商}{chang3 shang1}[][HSK 6]
    \definition[家,个]{s.}{empresa; fornecedor; fábrica; negócio; fabricante; uma unidade que produz e vende produtos; uma pessoa que administra uma fábrica}
  \end{phonetics}
\end{entry}

\begin{entry}{又}{2}{⼜}[Kangxi 29]
  \begin{phonetics}{又}{you4}[][HSK 2]
    \definition{adv.}{indica repetição ou continuação | indica que várias situações ou propriedades existem simultaneamente | indica um nível mais profundo de significado | indica adicionar zero a números inteiros | indica duas coisas contraditórias | indica um ponto de virada, significando 可是 | usado em frases negativas ou perguntas retóricas para fortalecer o tom | além disso; indica informações adicionais ou suplementares}
  \seealsoref{可是}{ke3shi4}
  \end{phonetics}
\end{entry}

\begin{entry}{又一次}{2,1,6}{⼜、⼀、⽋}
  \begin{phonetics}{又一次}{you4yi2ci4}
    \definition{adv.}{outra vez | mais uma vez | de novo}
  \end{phonetics}
\end{entry}

\begin{entry}{又及}{2,3}{⼜、⼃}
  \begin{phonetics}{又及}{you4ji2}
    \definition{s.}{P.S., \emph{postscript}}
  \end{phonetics}
\end{entry}

\begin{entry}{又名}{2,6}{⼜、⼝}
  \begin{phonetics}{又名}{you4ming2}
    \definition{s.}{também conhecido como | nome alternativo}
  \end{phonetics}
\end{entry}

\begin{entry}{又称}{2,10}{⼜、⽲}
  \begin{phonetics}{又称}{you4cheng1}
    \definition{s.}{também conhecido como}
  \end{phonetics}
\end{entry}

%%%%% EOF %%%%%


%%%
%%% 3画
%%%

\section*{3画}\addcontentsline{toc}{section}{3画}

\begin{entry}{万}{3}{⼀}
  \begin{phonetics}{万}{wan4}[][HSK 2]
    \definition*{s.}{sobrenome Wan}
    \definition{adj.}{um grande número}
    \definition{num.}{dez mil; 10.000; 1.0000}
  \end{phonetics}
\end{entry}

\begin{entry}{万一}{3,1}{⼀、⼀}
  \begin{phonetics}{万一}{wan4yi1}[][HSK 4]
    \definition{conj.}{por via das dúvidas; se por acaso; só por precaução; expressa uma suposição muito improvável (usado para coisas desagradáveis)}
    \definition{num.}{um décimo milionésimo; uma porcentagem muito pequena}
    \definition{s.}{contingência; eventualidade; contingências muito improváveis}
  \end{phonetics}
\end{entry}

\begin{entry}{万万}{3,3}{⼀、⼀}
  \begin{phonetics}{万万}{wan4wan4}
    \definition{adv.}{absolutamente | totalmente}
  \end{phonetics}
\end{entry}

\begin{entry}{万圣节}{3,5,5}{⼀、⼟、⾋}
  \begin{phonetics}{万圣节}{wan4sheng4jie2}
    \definition*{s.}{Dia de Todos os Santos}
  \seealsoref{万圣节前夕}{wan4sheng4jie2qian2xi1}
  \end{phonetics}
\end{entry}

\begin{entry}{万圣节前夕}{3,5,5,9,3}{⼀、⼟、⾋、⼑、⼣}
  \begin{phonetics}{万圣节前夕}{wan4sheng4jie2qian2xi1}
    \definition*{s.}{Véspera do Dia de Todos os Santos | \emph{Halloween}}
  \seealsoref{万圣节}{wan4sheng4jie2}
  \end{phonetics}
\end{entry}

\begin{entry}{丈夫}{3,4}{⼀、⼤}
  \begin{phonetics}{丈夫}{zhang4fu5}[][HSK 4]
    \definition[个,位,名]{s.}{marido; esposo}
  \end{phonetics}
\end{entry}

\begin{entry}{三}{3}{⼀}
  \begin{phonetics}{三}{san1}[][HSK 1]
    \definition*{s.}{sobrenome San}
    \definition{num.}{três; 3}
  \end{phonetics}
\end{entry}

\begin{entry}{三角}{3,7}{⼀、⾓}
  \begin{phonetics}{三角}{san1jiao3}
    \definition{s.}{triângulo}
  \end{phonetics}
\end{entry}

\begin{entry}{三角恋爱}{3,7,10,10}{⼀、⾓、⼼、⽖}
  \begin{phonetics}{三角恋爱}{san1jiao3lian4'ai4}
    \definition{s.}{triângulo amoroso}
  \end{phonetics}
\end{entry}

\begin{entry}{三明治}{3,8,8}{⼀、⽇、⽔}
  \begin{phonetics}{三明治}{san1ming2zhi4}
    \definition{s.}{(empréstimo linguístico) sanduíche}
  \end{phonetics}
\end{entry}

\begin{entry}{三轮车}{3,8,4}{⼀、⾞、⾞}
  \begin{phonetics}{三轮车}{san1lun2che1}
    \definition{s.}{triciclo}
  \end{phonetics}
\end{entry}

\begin{entry}{上}{3}{⼀}
  \begin{phonetics}{上}{shang4}[][HSK 1]
    \definition{adv.}{acima | em cima | sobre}
    \definition{v.}{subir | entrar em | frequentar (aula ou universidade)}
  \end{phonetics}
\end{entry}

\begin{entry}{上个月}{3,3,4}{⼀、⼈、⽉}
  \begin{phonetics}{上个月}{shang4 ge4 yue4}[][HSK 4]
    \definition{s.}{mês passado; refere-se à hora de um mês atrás, ou seja, o mês passado mais próximo da hora atual}
  \end{phonetics}
\end{entry}

\begin{entry}{上门}{3,3}{⼀、⾨}
  \begin{phonetics}{上门}{shang4 men2}[][HSK 4]
    \definition{v.}{chamar; visitar; aparecer; ir ou vir para ver alguém; ir até a porta; ir até a casa de alguém | trancar a porta; fechar a porta durante a noite | casar-se e morar com a família da noiva}
  \end{phonetics}
\end{entry}

\begin{entry}{上升}{3,4}{⼀、⼗}
  \begin{phonetics}{上升}{shang4 sheng1}[][HSK 3]
    \definition{v.}{elevar; subir; mover-se para cima}
  \end{phonetics}
\end{entry}

\begin{entry}{上午}{3,4}{⼀、⼗}
  \begin{phonetics}{上午}{shang4wu3}[][HSK 1]
    \definition{adv.}{manhã | de manhã | período antes do meio-dia}
  \end{phonetics}
\end{entry}

\begin{entry}{上车}{3,4}{⼀、⾞}
  \begin{phonetics}{上车}{shang4 che1}[][HSK 1]
    \definition{v.}{entrar (em ônibus, trem, carro, etc.)}
  \end{phonetics}
\end{entry}

\begin{entry}{上去}{3,5}{⼀、⼛}
  \begin{phonetics}{上去}{shang4 qu4}[][HSK 3]
    \definition{v.}{subir (a partir da minha localização) | ascender a um lugar (ou estado) considerado mais elevado (ou acima)}
  \end{phonetics}
\end{entry}

\begin{entry}{上古}{3,5}{⼀、⼝}
  \begin{phonetics}{上古}{shang4gu3}
    \definition{s.}{o passado distante | tempos antigos | antiguidade}
  \end{phonetics}
\end{entry}

\begin{entry}{上边}{3,5}{⼀、⾡}
  \begin{phonetics}{上边}{shang4bian5}[][HSK 1]
    \definition{adv.}{acima de | parte de cima | por cima}
  \end{phonetics}
\end{entry}

\begin{entry}{上当}{3,6}{⼀、⼹}
  \begin{phonetics}{上当}{shang4dang4}
    \definition{v.+compl.}{ser enganado | morder uma isca | ser manipulado | ser joguete nas mãos de alguém}
  \end{phonetics}
\end{entry}

\begin{entry}{上次}{3,6}{⼀、⽋}
  \begin{phonetics}{上次}{shang4 ci4}[][HSK 1]
    \definition{adv.}{última vez}
  \end{phonetics}
\end{entry}

\begin{entry}{上网}{3,6}{⼀、⽹}
  \begin{phonetics}{上网}{shang4 wang3}[][HSK 1]
    \definition{v.}{conectar à \emph{Internet} | fazer \emph{upload} | ficar \emph{online}}
  \end{phonetics}
\end{entry}

\begin{entry}{上衣}{3,6}{⼀、⾐}
  \begin{phonetics}{上衣}{shang4 yi1}[][HSK 3]
    \definition{s.}{jaqueta; vestimenta externa superior}
  \end{phonetics}
\end{entry}

\begin{entry}{上访}{3,6}{⼀、⾔}
  \begin{phonetics}{上访}{shang4fang3}
    \definition{v.}{buscar uma audiência com superiores (especialmente funcionários do governo) para fazer uma petição por algo}
  \end{phonetics}
\end{entry}

\begin{entry}{上声}{3,7}{⼀、⼠}
  \begin{phonetics}{上声}{shang3sheng1}
    \definition{s.}{tom descendente e ascendente | terceiro tom no mandarim moderno}
  \end{phonetics}
\end{entry}

\begin{entry}{上来}{3,7}{⼀、⽊}
  \begin{phonetics}{上来}{shang4 lai2}[][HSK 3]
    \definition{v.}{subir (para a minha localização) | estar no começo | vir à tona | usado depois de um verbo para indicar sucesso em fazer algo}
  \end{phonetics}
\end{entry}

\begin{entry}{上周}{3,8}{⼀、⼝}
  \begin{phonetics}{上周}{shang4 zhou1}[][HSK 2]
    \definition{s.}{semana passada}
  \end{phonetics}
\end{entry}

\begin{entry}{上坡路}{3,8,13}{⼀、⼟、⾜}
  \begin{phonetics}{上坡路}{shang4po1lu4}
    \definition{s.}{aclive | progresso | (fig.) tendência ascendente}
  \end{phonetics}
\end{entry}

\begin{entry}{上学}{3,8}{⼀、⼦}
  \begin{phonetics}{上学}{shang4 xue2}[][HSK 1]
    \definition{v.}{ir à escola | frequentar a escola | estar na escola | iniciar as aulas}
  \end{phonetics}
\end{entry}

\begin{entry}{上询}{3,8}{⼀、⾔}
  \begin{phonetics}{上询}{shang4 xun2}
    \definition{adv.}{primeira dezena do mês}
  \end{phonetics}
\end{entry}

\begin{entry}{上面}{3,9}{⼀、⾯}
  \begin{phonetics}{上面}{shang4 mian4}[][HSK 3]
    \definition{s.}{uma posição mais alta que algo; uma posição acima/acima de algo | superfície do objeto | aspecto | a parte acima mencionada | autoridades superiores | os mais velhos; a geração mais velha da família}
  \end{phonetics}
\end{entry}

\begin{entry}{上海}{3,10}{⼀、⽔}
  \begin{phonetics}{上海}{shang4hai3}
    \definition*{s.}{Shangai (Xangai)}
  \end{phonetics}
\end{entry}

\begin{entry}{上班}{3,10}{⼀、⽟}
  \begin{phonetics}{上班}{shang4 ban1}[][HSK 1]
    \definition{v.+compl.}{ir para o trabalho | ir para o emprego | estar de plantão}
  \end{phonetics}
\end{entry}

\begin{entry}{上课}{3,10}{⼀、⾔}
  \begin{phonetics}{上课}{shang4 ke4}[][HSK 1]
    \definition{v.}{assistir à aula | ir para a aula | ir dar uma aula}
  \end{phonetics}
\end{entry}

\begin{entry}{上楼}{3,13}{⼀、⽊}
  \begin{phonetics}{上楼}{shang4 lou2}[][HSK 4]
    \definition{v.}{subir as escadas; ir para o andar de cima}
  \end{phonetics}
\end{entry}

\begin{entry}{上演}{3,14}{⼀、⽔}
  \begin{phonetics}{上演}{shang4yan3}
    \definition{s.}{exibição | encenação}
    \definition{v.}{exibir (um filme) | encenar (uma peça)}
  \end{phonetics}
\end{entry}

\begin{entry}{下}{3}{⼀}
  \begin{phonetics}{下}{xia4}[][HSK 1,2]
    \definition{adv.}{abaixo | em baixo de}
    \definition{clas.}{para número de vezes para ações}
    \definition{v.}{descer | chegar a (uma decisão, conclusão, etc.) | recusar}
  \end{phonetics}
\end{entry}

\begin{entry}{下个月}{3,3,4}{⼀、⼈、⽉}
  \begin{phonetics}{下个月}{xia4 ge4 yue4}[][HSK 4]
    \definition{s.}{próximo mês; mês que vem; refere-se ao próximo mês do mês atual}
  \end{phonetics}
\end{entry}

\begin{entry}{下午}{3,4}{⼀、⼗}
  \begin{phonetics}{下午}{xia4wu3}[][HSK 1]
    \definition{adv.}{tarde | à tarde | período logo após o meio-dia}
  \end{phonetics}
\end{entry}

\begin{entry}{下午茶}{3,4,9}{⼀、⼗、⾋}
  \begin{phonetics}{下午茶}{xia4wu3cha2}
    \definition{s.}{chá da tarde (normalmente chás com doces)}
  \end{phonetics}
\end{entry}

\begin{entry}{下巴}{3,4}{⼀、⼰}
  \begin{phonetics}{下巴}{xia4ba5}
    \definition[个]{s.}{queixo}
  \end{phonetics}
\end{entry}

\begin{entry}{下水道}{3,4,12}{⼀、⽔、⾡}
  \begin{phonetics}{下水道}{xia4shui3dao4}
    \definition{s.}{esgoto}
  \end{phonetics}
\end{entry}

\begin{entry}{下车}{3,4}{⼀、⾞}
  \begin{phonetics}{下车}{xia4 che1}[][HSK 1]
    \definition{v.}{descer ou sair (de ônibus, carro, etc.)}
  \end{phonetics}
\end{entry}

\begin{entry}{下去}{3,5}{⼀、⼛}
  \begin{phonetics}{下去}{xia4 qu4}[][HSK 3]
    \definition{part.}{usado depois de verbos para indicar de alto a baixo | usado depois de um verbo para indicar continuação}
    \definition{v.}{descer (a partir da minha localização)
continuar
obter; crescer; tornar-se}
  \end{phonetics}
\end{entry}

\begin{entry}{下边}{3,5}{⼀、⾡}
  \begin{phonetics}{下边}{xia4bian5}[][HSK 1]
    \definition{adv.}{em baixo | abaixo | parte de baixo}
  \end{phonetics}
\end{entry}

\begin{entry}{下旬}{3,6}{⼀、⽇}
  \begin{phonetics}{下旬}{xia4xun2}
    \definition{adv.}{última dezena do mês}
  \end{phonetics}
\end{entry}

\begin{entry}{下次}{3,6}{⼀、⽋}
  \begin{phonetics}{下次}{xia4 ci4}[][HSK 1]
    \definition{s.}{próxima vez}
  \end{phonetics}
\end{entry}

\begin{entry}{下来}{3,7}{⼀、⽊}
  \begin{phonetics}{下来}{xia4 lai5}[][HSK 3]
    \definition{part.}{usado depois de um verbo para indicar que uma ação ou comportamento está se movendo em direção ao falante ou que a ação está continuando ou sendo concluída | usado depois de um adjetivo para indicar que um certo estado começou a aparecer e continuará a se desenvolver.}
    \definition{v.}{descer (para a minha localização) | (colheitas/frutas/vegetais, etc.) ser colhido; estar maduro o suficiente para ser colhido | (período de tempo) acabar; passar; chegar ao fim}
  \end{phonetics}
\end{entry}

\begin{entry}{下周}{3,8}{⼀、⼝}
  \begin{phonetics}{下周}{xia4 zhou1}[][HSK 2]
    \definition{s.}{próxima semana}
  \end{phonetics}
\end{entry}

\begin{entry}{下线}{3,8}{⼀、⽷}
  \begin{phonetics}{下线}{xia4xian4}
    \definition{v.}{ficar \emph{offline} | (um produto) sair da linha de montagem | pessoa abaixo de si em um esquema de pirâmide}
  \end{phonetics}
\end{entry}

\begin{entry}{下降}{3,8}{⼀、⾩}
  \begin{phonetics}{下降}{xia4 jiang4}[][HSK 4]
    \definition{v.}{cair; despencar; declinar; descer; diminuir; ir para baixo}
  \end{phonetics}
\end{entry}

\begin{entry}{下雨}{3,8}{⼀、⾬}
  \begin{phonetics}{下雨}{xia4 yu3}[][HSK 1]
    \definition{v.+compl.}{chover}
  \end{phonetics}
\end{entry}

\begin{entry}{下面}{3,9}{⼀、⾯}
  \begin{phonetics}{下面}{xia4 mian4}[][HSK 3]
    \definition{s.}{em baixo; abaixo; parte de baixo | próximo; seguinte | subordinado; o nível inferior; homens nos níveis inferiores}
    \definition{v.}{cozinhar macarrão}
  \end{phonetics}
\end{entry}

\begin{entry}{下海}{3,10}{⼀、⽔}
  \begin{phonetics}{下海}{xia4hai3}
    \definition{v.+compl.}{ir para o mar; (barco) deixar o porto e iniciar uma jornada | ir pescar no mar | tornar-se ator profissional}
  \end{phonetics}
\end{entry}

\begin{entry}{下班}{3,10}{⼀、⽟}
  \begin{phonetics}{下班}{xia4 ban1}[][HSK 1]
    \definition{v.+compl.}{sair do trabalho}
  \end{phonetics}
\end{entry}

\begin{entry}{下课}{3,10}{⼀、⾔}
  \begin{phonetics}{下课}{xia4 ke4}[][HSK 1]
    \definition{v.+compl.}{acabar a aula | terminar a aula}
  \end{phonetics}
\end{entry}

\begin{entry}{下载}{3,10}{⼀、⾞}
  \begin{phonetics}{下载}{xia4zai3}[][HSK 4]
    \definition{v.}{\emph{download}; baixar; salvar informações da \emph{Web} em um dispositivo, como um computador}
  \end{phonetics}
\end{entry}

\begin{entry}{下蛋}{3,11}{⼀、⾍}
  \begin{phonetics}{下蛋}{xia4dan4}
    \definition{v.}{botar ovos}
  \end{phonetics}
\end{entry}

\begin{entry}{下雪}{3,11}{⼀、⾬}
  \begin{phonetics}{下雪}{xia4 xue3}[][HSK 2]
    \definition[场,次]{s.}{neve}
    \definition{v.+compl.}{nevar}
  \end{phonetics}
\end{entry}

\begin{entry}{下崽}{3,12}{⼀、⼭}
  \begin{phonetics}{下崽}{xia4zai3}
    \definition{v.}{dar à luz (animais) | parir}
  \end{phonetics}
\end{entry}

\begin{entry}{下楼}{3,13}{⼀、⽊}
  \begin{phonetics}{下楼}{xia4 lou2}[][HSK 4]
    \definition{v.}{descer as escadas}
  \end{phonetics}
\end{entry}

\begin{entry}{与}{3}{⼀}
  \begin{phonetics}{与}{yu3}
    \definition{conj.}{e, com}
  \end{phonetics}
  \begin{phonetics}{与}{yu4}
    \definition{v.}{fazer parte de}
  \end{phonetics}
\end{entry}

\begin{entry}{与其}{3,8}{⼀、⼋}
  \begin{phonetics}{与其}{yu3qi2}
    \definition{conj.}{mais do que}
  \end{phonetics}
\end{entry}

\begin{entry}{与其……不如……}{3,8,4,6}{⼀、⼋、⼀、⼥}
  \begin{phonetics}{与其……不如……}{yu3qi2 bu4ru2}
    \definition{conj.}{ao invés de\dots melhor que\dots}
  \end{phonetics}
\end{entry}

\begin{entry}{与其……宁可……}{3,8,5,5}{⼀、⼋、⼧、⼝}
  \begin{phonetics}{与其……宁可……}{yu3qi2 ning4ke3}
    \definition{conj.}{ao invés de\dots melhor que\dots}
  \end{phonetics}
\end{entry}

\begin{entry}{个}{3}{⼈}
  \begin{phonetics}{个}{ge3}
    \definition{pron.}{usado em 自个儿}
    \seeref{自个儿}{zi4ge3r5}
  \end{phonetics}
  \begin{phonetics}{个}{ge4}[][HSK 1]
    \definition{clas.}{para objetos e pessoas em geral}
    \definition{pron.}{isto | aquilo}
    \definition{s.}{indivíduo | tamanho}
  \end{phonetics}
\end{entry}

\begin{entry}{个人}{3,2}{⼈、⼈}
  \begin{phonetics}{个人}{ge4ren2}[][HSK 3]
    \definition{pron.}{pessoal; si mesmo}
    \definition[个]{s.}{indivíduo}
  \end{phonetics}
\end{entry}

\begin{entry}{个儿}{3,2}{⼈、⼉}
  \begin{phonetics}{个儿}{ge4r5}[][HSK 5]
    \definition{s.}{tamanho; altura; estatura; tamanho do corpo ou do objeto |
pessoas ou coisas consideradas isoladamente; referir-se a uma pessoa ou coisa individualmente}
  \end{phonetics}
\end{entry}

\begin{entry}{个子}{3,3}{⼈、⼦}
  \begin{phonetics}{个子}{ge4zi5}[][HSK 2]
    \definition{s.}{altura | estatura}
  \end{phonetics}
\end{entry}

\begin{entry}{个体}{3,7}{⼈、⼈}
  \begin{phonetics}{个体}{ge4ti3}[][HSK 4]
    \definition{s.}{pessoa ou organismo individual}
  \end{phonetics}
\end{entry}

\begin{entry}{个别}{3,7}{⼈、⼑}
  \begin{phonetics}{个别}{ge4bie2}[][HSK 4]
    \definition{adj.}{muito poucos; excepcionais}
    \definition{adv.}{separadamente; individualmente; isoladamente}
  \end{phonetics}
\end{entry}

\begin{entry}{个性}{3,8}{⼈、⼼}
  \begin{phonetics}{个性}{ge4xing4}[][HSK 3]
    \definition{s.}{caráter individual; individualidade; personalidade}
  \end{phonetics}
\end{entry}

\begin{entry}{久}{3}{⼃}
  \begin{phonetics}{久}{jiu3}[][HSK 3]
    \definition{adj.}{por muito tempo | duração de tempo especificada}
  \end{phonetics}
\end{entry}

\begin{entry}{义务}{3,5}{⼂、⼒}
  \begin{phonetics}{义务}{yi4wu4}[][HSK 4]
    \definition{s.}{dever; obrigação; responsabilidades perante a lei, em oposição a ``权利''.}
  \seealsoref{权利}{quan2li4}
  \end{phonetics}
\end{entry}

\begin{entry}{之一}{3,1}{⼂、⼀}
  \begin{phonetics}{之一}{zhi1 yi1}[][HSK 4]
    \definition{s.}{um de (algo); pertence a um ou a todo um grupo de coisas com as mesmas características}
  \end{phonetics}
\end{entry}

\begin{entry}{之外}{3,5}{⼂、⼣}
  \begin{phonetics}{之外}{zhi1wai4}
    \definition{adv.}{lado de fora}
  \end{phonetics}
\end{entry}

\begin{entry}{之后}{3,6}{⼂、⼝}
  \begin{phonetics}{之后}{zhi1 hou4}[][HSK 4]
    \definition{adv.}{mais tarde; posteriormente; depois; desde então; para indicar que é depois de um determinado tempo ou de uma determinada coisa, ``以后'' é usado com frequência na linguagem falada; às vezes, também pode indicar que é depois de um determinado lugar ou local,  ``后面'' é usado com frequência na linguagem falada}
  \seealsoref{后面}{hou4mian4}
  \seealsoref{以后}{yi3 hou4}
  \end{phonetics}
\end{entry}

\begin{entry}{之间}{3,7}{⼂、⾨}
  \begin{phonetics}{之间}{zhi1 jian1}[][HSK 4]
    \definition{prep.}{(depois de um substantivo) entre; dentro de duas delimitações de tempo, local ou quantitativas | colocado após certos verbos ou advérbios de duas sílabas para indicar um curto período de tempo}
  \end{phonetics}
\end{entry}

\begin{entry}{之前}{3,9}{⼂、⼑}
  \begin{phonetics}{之前}{zhi1 qian2}[][HSK 4]
    \definition{adv.}{(referindo-se ao tempo) antes, antes de, atrás | (referindo-se ao local físico) na frente de | (usado independentemente) no passado, antes disso}
  \end{phonetics}
\end{entry}

\begin{entry}{也}{3}{⼄}
  \begin{phonetics}{也}{ye3}[][HSK 1]
    \definition*{s.}{sobrenome Ye}
    \definition{adv.}{também | (em frases negativas) nem, tampouco}
  \end{phonetics}
\end{entry}

\begin{entry}{也有今天}{3,6,4,4}{⼄、⽉、⼈、⼤}
  \begin{phonetics}{也有今天}{ye3you3jin1tian1}
    \definition{expr.}{obter apenas o que merece | todo cachorro tem seu dia | obter a sua parte (coisas boas ou ruins) | servir alguém bem}
  \end{phonetics}
\end{entry}

\begin{entry}{也许}{3,6}{⼄、⾔}
  \begin{phonetics}{也许}{ye3xu3}[][HSK 2]
    \definition{adv.}{possivelmente | talvez}
  \end{phonetics}
\end{entry}

\begin{entry}{也就是}{3,12,9}{⼄、⼪、⽇}
  \begin{phonetics}{也就是}{ye3jiu4shi4}
    \definition{adv.}{i.e., isso é | ou seja}
  \end{phonetics}
\end{entry}

\begin{entry}{也就是说}{3,12,9,9}{⼄、⼪、⽇、⾔}
  \begin{phonetics}{也就是说}{ye3jiu4shi4shuo1}
    \definition{adv.}{em outras palavras | então | isto é | por isso}
  \end{phonetics}
\end{entry}

\begin{entry}{习惯}{3,11}{⼄、⼼}
  \begin{phonetics}{习惯}{xi2guan4}[][HSK 2]
    \definition[个]{s.}{hábito | costume | prática usual}
    \definition{v.}{ser acostumado a | ter o hábito de}
  \end{phonetics}
\end{entry}

\begin{entry}{乡巴佬}{3,4,8}{⼄、⼰、⼈}
  \begin{phonetics}{乡巴佬}{xiang1ba1lao3}
    \definition{s.}{aldeão | caipira}
  \end{phonetics}
\end{entry}

\begin{entry}{乡村}{3,7}{⼄、⽊}
  \begin{phonetics}{乡村}{xiang1cun1}
    \definition{adj.}{rural | rústico}
    \definition{s.}{vila | campo}
  \end{phonetics}
\end{entry}

\begin{entry}{于}{3}{⼆}
  \begin{phonetics}{于}{yu2}
    \definition*{s.}{sobrenome Yu}
    \definition{prep.}{indica tempo, local, extensão, etc. | indica a direção da ação | usada após um verbo para indicar doação, entrega, etc. | relacionamento do objeto ou da entidade introduzida | indica o ponto inicial ou o ponto de partida | indica comparação}
  \end{phonetics}
\end{entry}

\begin{entry}{于是}{3,9}{⼆、⽇}
  \begin{phonetics}{于是}{yu2shi4}[][HSK 4]
    \definition{conj.}{então; portanto; consequentemente; como resultado; indica que o último segue o primeiro e que o último é frequentemente causado pelo primeiro}
  \end{phonetics}
\end{entry}

\begin{entry}{亿}{3}{⼈}
  \begin{phonetics}{亿}{yi4}[][HSK 2]
    \definition{num.}{cem milhões; 100.000.000; 1.0000.0000}
  \end{phonetics}
\end{entry}

\begin{entry}{千}{3}{⼗}
  \begin{phonetics}{千}{qian1}[][HSK 2]
    \definition{num.}{mil; 1.000; 1000}
  \end{phonetics}
\end{entry}

\begin{entry}{千万}{3,3}{⼗、⼀}
  \begin{phonetics}{千万}{qian1wan4}[][HSK 3]
    \definition{adv.}{(usado para indicar desejos fortes) por todos os meios; sob quaisquer circunstâncias}
    \definition{num.}{dez milhões; milhões e milhões}
  \end{phonetics}
\end{entry}

\begin{entry}{千千万万}{3,3,3,3}{⼗、⼗、⼀、⼀}
  \begin{phonetics}{千千万万}{qian1qian1wan4wan4}
    \definition{num.}{inumerável | números incontáveis | milhares e milhares}
  \end{phonetics}
\end{entry}

\begin{entry}{千古}{3,5}{⼗、⼝}
  \begin{phonetics}{千古}{qian1gu3}
    \definition{adv.}{por toda a eternidade | em todas as idades}
    \definition{s.}{eternidade (usada em um dístico elegíaco, coroa de flores, etc., dedicada aos mortos)}
  \end{phonetics}
\end{entry}

\begin{entry}{千年}{3,6}{⼗、⼲}
  \begin{phonetics}{千年}{qian1nian2}
    \definition{s.}{milênio}
  \end{phonetics}
\end{entry}

\begin{entry}{千克}{3,7}{⼗、⼗}
  \begin{phonetics}{千克}{qian1 ke4}[][HSK 2]
    \definition{clas.}{kg | quilo | quilograma}
  \end{phonetics}
\end{entry}

\begin{entry}{卫生}{3,5}{⼙、⽣}
  \begin{phonetics}{卫生}{wei4 sheng1}[][HSK 3]
    \definition{adj.}{bom para a saúde; higiênico}
    \definition{s.}{higiene; saneamento}
  \end{phonetics}
\end{entry}

\begin{entry}{卫生巾}{3,5,3}{⼙、⽣、⼱}
  \begin{phonetics}{卫生巾}{wei4sheng1jin1}
    \definition{s.}{absorvente higiênico}
  \end{phonetics}
\end{entry}

\begin{entry}{卫生厅}{3,5,4}{⼙、⽣、⼚}
  \begin{phonetics}{卫生厅}{wei4sheng1ting1}
    \definition*{s.}{Departamento de Saúde (da província)}
  \end{phonetics}
\end{entry}

\begin{entry}{卫生防疫}{3,5,6,9}{⼙、⽣、⾩、⽧}
  \begin{phonetics}{卫生防疫}{wei4sheng1 fang2yi4}
    \definition{s.}{prevenção contra a epidemia}
  \end{phonetics}
\end{entry}

\begin{entry}{卫生局}{3,5,7}{⼙、⽣、⼫}
  \begin{phonetics}{卫生局}{wei4sheng1ju2}
    \definition*{s.}{Departamento de Saúde | Escritório de Saúde}
  \end{phonetics}
\end{entry}

\begin{entry}{卫生纸}{3,5,7}{⼙、⽣、⽷}
  \begin{phonetics}{卫生纸}{wei4sheng1zhi3}
    \definition{s.}{papel higiênico}
  \end{phonetics}
\end{entry}

\begin{entry}{卫生间}{3,5,7}{⼙、⽣、⾨}
  \begin{phonetics}{卫生间}{wei4sheng1jian1}[][HSK 3]
    \definition[间,个]{s.}{banheiro; sanitário; \emph{toilette}}
  \end{phonetics}
\end{entry}

\begin{entry}{卫生套}{3,5,10}{⼙、⽣、⼤}
  \begin{phonetics}{卫生套}{wei4sheng1tao4}
    \definition[只]{s.}{preservativo | camisinha}
  \end{phonetics}
\end{entry}

\begin{entry}{卫生部}{3,5,10}{⼙、⽣、⾢}
  \begin{phonetics}{卫生部}{wei4sheng1bu4}
    \definition*{s.}{Ministério da Saúde}
  \end{phonetics}
\end{entry}

\begin{entry}{卫生球}{3,5,11}{⼙、⽣、⽟}
  \begin{phonetics}{卫生球}{wei4sheng1qiu2}
    \definition{s.}{naftalina}
  \end{phonetics}
\end{entry}

\begin{entry}{卫生棉}{3,5,12}{⼙、⽣、⽊}
  \begin{phonetics}{卫生棉}{wei4sheng1mian2}
    \definition{s.}{absorvente | algodão absorvente esterilizado (usado para curativos ou limpeza de feridas) | absorvente tampão}
  \end{phonetics}
\end{entry}

\begin{entry}{卫生署}{3,5,13}{⼙、⽣、⽹}
  \begin{phonetics}{卫生署}{wei4sheng1shu3}
    \definition*{s.}{Agência de Saúde (ou Escritório, ou Departamento)}
  \end{phonetics}
\end{entry}

\begin{entry}{叉}{3}{⼜}
  \begin{phonetics}{叉}{cha1}[][HSK 5]
    \definition{s.}{garfo; forquilha | símbolo de cruz, ``×''}
    \definition{v.}{trabalhar com um garfo; garfar; pegar coisas com um garfo}
  \end{phonetics}
  \begin{phonetics}{叉}{cha2}
    \definition{v.}{bloquear; emperrar; congestionar}
  \end{phonetics}
  \begin{phonetics}{叉}{cha3}
    \definition{v.}{separar de modo a formar uma bifurcação; bifurcar}
  \end{phonetics}
\end{entry}

\begin{entry}{叉子}{3,3}{⼜、⼦}
  \begin{phonetics}{叉子}{cha1zi5}[][HSK 5]
    \definition[把]{s.}{garfo; ferramenta com mais de duas pontas em uma extremidade | tridente; forquilha; ferramentas de agricultura antigas}
  \end{phonetics}
\end{entry}

\begin{entry}{及}{3}{⼃}
  \begin{phonetics}{及}{ji2}
    \definition{conj.}{e | bem como}
  \end{phonetics}
\end{entry}

\begin{entry}{及时}{3,7}{⼃、⽇}
  \begin{phonetics}{及时}{ji2shi2}[][HSK 3]
    \definition{adj.}{oportuno; a tempo; sazonal}
    \definition{adv.}{prontamente; sem demora}
  \end{phonetics}
\end{entry}

\begin{entry}{及格}{3,10}{⼃、⽊}
  \begin{phonetics}{及格}{ji2ge2}[][HSK 4]
    \definition{v.+compl.}{passar; passar em um teste, exame, etc.}
  \end{phonetics}
\end{entry}

\begin{entry}{口}{3}{⼝}[Kangxi 30]
  \begin{phonetics}{口}{kou3}[][HSK 1]
    \definition{clas.}{para coisas com bocas (pessoas, animais domésticos, canhões, etc.) | para mordidas ou bocados}
    \definition{s.}{boca}
  \end{phonetics}
\end{entry}

\begin{entry}{口语}{3,9}{⼝、⾔}
  \begin{phonetics}{口语}{kou3 yu3}[][HSK 4]
    \definition[门]{s.}{linguagem oral; linguagem falada; linguagem coloquial; linguagem usada em conversas}
  \end{phonetics}
\end{entry}

\begin{entry}{口音}{3,9}{⼝、⾳}
  \begin{phonetics}{口音}{kou3yin1}
    \definition{s.}{sons da fala oral (linguística)}
  \end{phonetics}
  \begin{phonetics}{口音}{kou3yin5}
    \definition{s.}{sotaque | voz}
  \end{phonetics}
\end{entry}

\begin{entry}{口香糖}{3,9,16}{⼝、⾹、⽶}
  \begin{phonetics}{口香糖}{kou3xiang1tang2}
    \definition{s.}{goma de mascar | chiclete}
  \end{phonetics}
\end{entry}

\begin{entry}{口袋}{3,11}{⼝、⾐}
  \begin{phonetics}{口袋}{kou3dai4}[][HSK 4]
    \definition[个]{s.}{bolso | saco; sacola; artigos de tecido ou couro}
  \end{phonetics}
\end{entry}

\begin{entry}{口袋妖怪}{3,11,7,8}{⼝、⾐、⼥、⼼}
  \begin{phonetics}{口袋妖怪}{kou3dai4 yao1guai4}
    \definition*{s.}{\emph{Pokémon}}
  \end{phonetics}
\end{entry}

\begin{entry}{土}{3}{⼟}[Kangxi 32]
  \begin{phonetics}{土}{tu3}[][HSK 3]
    \definition*{s.}{sobrenome Tu}
    \definition{adj.}{local; nativo | folclórico; popular; indígena | fora de moda; antiquado; inculto; rústico}
    \definition{s.}{solo; terra | terreno; chão}
  \end{phonetics}
\end{entry}

\begin{entry}{土地}{3,6}{⼟、⼟}
  \begin{phonetics}{土地}{tu3di4}[][HSK 4]
    \definition[片,块]{s.}{terra; solo; chão; superfície terrestre da Terra usada para cultivar, construir edifícios e viver | território}
  \end{phonetics}
  \begin{phonetics}{土地}{tu3di5}
    \definition{s.}{deus da audeia; deus local; \emph{genius loci} deidade protetora de um local; (superstição) refere-se ao deus da terra que governa uma pequena área}
  \end{phonetics}
\end{entry}

\begin{entry}{土豆}{3,7}{⼟、⾖}
  \begin{phonetics}{土豆}{tu3dou4}
    \definition[个,颗]{s.}{batata}
  \end{phonetics}
\end{entry}

\begin{entry}{土豆泥}{3,7,8}{⼟、⾖、⽔}
  \begin{phonetics}{土豆泥}{tu3dou4ni2}
    \definition{s.}{purê de batatas}
  \end{phonetics}
\end{entry}

\begin{entry}{土鸡}{3,7}{⼟、⿃}
  \begin{phonetics}{土鸡}{tu3ji1}
    \definition{s.}{galinha caipira}
  \end{phonetics}
\end{entry}

\begin{entry}{士兵}{3,7}{⼠、⼋}
  \begin{phonetics}{士兵}{shi4bing1}[][HSK 4]
    \definition[名,个]{s.}{soldado; militar; termo coletivo para oficiais não comissionados e soldados; os membros mais jovens do exército}
  \end{phonetics}
\end{entry}

\begin{entry}{夕阳}{3,6}{⼣、⾩}
  \begin{phonetics}{夕阳}{xi1yang2}
    \definition{s.}{pôr do sol}
  \seealsoref{日出}{ri4chu1}
  \end{phonetics}
\end{entry}

\begin{entry}{大}{3}{⼤}[Kangxi 37]
  \begin{phonetics}{大}{da4}[][HSK 1]
    \definition{adj.}{grande | enorme | maior | largo | profundo | mais velho (que) | mais antigo | mais velho | muito}
    \definition{s.}{(dialeto) pai | irmão mais velho ou mais novo do pai}
  \end{phonetics}
  \begin{phonetics}{大}{dai4}
    \definition{s.}{usado em 大夫 \dpy{dai4fu5}: médico, doutor}
    \seeref{大夫}{dai4fu5}
  \end{phonetics}
\end{entry}

\begin{entry}{大人}{3,2}{⼤、⼈}
  \begin{phonetics}{大人}{da4 ren2}[][HSK 2]
    \definition{s.}{adulto}
  \end{phonetics}
\end{entry}

\begin{entry}{大于}{3,3}{⼤、⼆}
  \begin{phonetics}{大于}{da4 yu2}[][HSK 5]
    \definition{v.}{ser maior, mais numeroso, mais importante, etc. do que}
  \end{phonetics}
\end{entry}

\begin{entry}{大口}{3,3}{⼤、⼝}
  \begin{phonetics}{大口}{da4kou3}
    \definition{s.}{grande bocado (de comida, bebida, fumo, etc.)}
  \end{phonetics}
\end{entry}

\begin{entry}{大大}{3,3}{⼤、⼤}
  \begin{phonetics}{大大}{da4 da4}[][HSK 2]
    \definition{adv.}{muito; enormemente}
  \end{phonetics}
\end{entry}

\begin{entry}{大小}{3,3}{⼤、⼩}
  \begin{phonetics}{大小}{da4 xiao3}[][HSK 2]
    \definition{adv.}{no mínimo}
    \definition[家]{s.}{tamanho | grau de antiguidade | adultos e crianças | grande ou pequeno}
  \end{phonetics}
\end{entry}

\begin{entry}{大门}{3,3}{⼤、⾨}
  \begin{phonetics}{大门}{da4 men2}[][HSK 2]
    \definition{s.}{portão | entrada}
  \end{phonetics}
\end{entry}

\begin{entry}{大马}{3,3}{⼤、⾺}
  \begin{phonetics}{大马}{da4ma3}
    \definition*{s.}{Malásia}
  \end{phonetics}
\end{entry}

\begin{entry}{大厅}{3,4}{⼤、⼚}
  \begin{phonetics}{大厅}{da4 ting1}[][HSK 5]
    \definition{s.}{\emph{hall}; saguão, uma sala grande para reuniões ou atividades em um edifício de grande porte}
  \end{phonetics}
\end{entry}

\begin{entry}{大夫}{3,4}{⼤、⼤}
  \begin{phonetics}{大夫}{da4fu1}
    \definition{s.}{oficial sênior (na China Imperial)}
  \end{phonetics}
  \begin{phonetics}{大夫}{dai4fu5}[][HSK 3]
    \definition{s.}{médico, doutor}
  \end{phonetics}
\end{entry}

\begin{entry}{大巴}{3,4}{⼤、⼰}
  \begin{phonetics}{大巴}{da4 ba1}[][HSK 4]
    \definition{s.}{ônibus}
  \end{phonetics}
\end{entry}

\begin{entry}{大方}{3,4}{⼤、⽅}
  \begin{phonetics}{大方}{da4fang5}[][HSK 4]
    \definition{adj.}{generoso | não afetado; natural e equilibrado |  de bom gosto}
  \end{phonetics}
\end{entry}

\begin{entry}{大众}{3,6}{⼤、⼈}
  \begin{phonetics}{大众}{da4 zhong4}[][HSK 4]
    \definition{s.}{massas; população; pessoas comuns; público em geral}
  \end{phonetics}
\end{entry}

\begin{entry}{大伙儿}{3,6,2}{⼤、⼈、⼉}
  \begin{phonetics}{大伙儿}{da4huo3r5}[][HSK 5]
    \definition{pron.}{todos nós; todos vocês; todo mundo; todos | equivalente a ``大家''}
  \seealsoref{大家}{da4jia1}
  \end{phonetics}
\end{entry}

\begin{entry}{大会}{3,6}{⼤、⼈}
  \begin{phonetics}{大会}{da4 hui4}[][HSK 4]
    \definition{s.}{sessão plenária; reunião geral de membros; reuniões convocadas por partidos políticos socialistas | reunião de massa; comício de massa}
  \end{phonetics}
\end{entry}

\begin{entry}{大全}{3,6}{⼤、⼊}
  \begin{phonetics}{大全}{da4quan2}
    \definition{s.}{coleção abrangente}
  \end{phonetics}
\end{entry}

\begin{entry}{大后天}{3,6,4}{⼤、⼝、⼤}
  \begin{phonetics}{大后天}{da4 hou4 tian1}
    \definition{adv.}{daqui a três dias}
  \end{phonetics}
\end{entry}

\begin{entry}{大多}{3,6}{⼤、⼣}
  \begin{phonetics}{大多}{da4 duo1}[][HSK 4]
    \definition{adv.}{majoritariamente; em sua maior parte; em sua maioria; em grande parte}
  \end{phonetics}
\end{entry}

\begin{entry}{大多数}{3,6,13}{⼤、⼣、⽁}
  \begin{phonetics}{大多数}{da4 duo1 shu4}[][HSK 2]
    \definition{s.}{a grande maioria | a vasta maioria | a maior parte}
  \end{phonetics}
\end{entry}

\begin{entry}{大妈}{3,6}{⼤、⼥}
  \begin{phonetics}{大妈}{da4 ma1}[][HSK 4]
    \definition{s.}{tia; esposa do irmão mais velho do pai | tia (homenagem às mulheres idosas)}
  \end{phonetics}
\end{entry}

\begin{entry}{大戏}{3,6}{⼤、⼽}
  \begin{phonetics}{大戏}{da4xi4}
    \definition*{s.}{Drama, Ópera Chinesa}
  \end{phonetics}
\end{entry}

\begin{entry}{大爷}{3,6}{⼤、⽗}
  \begin{phonetics}{大爷}{da4 ye5}[][HSK 4]
    \definition{s.}{irmão mais velho do pai; tio | tio (homenagem aos homens mais velhos)}
  \end{phonetics}
\end{entry}

\begin{entry}{大约}{3,6}{⼤、⽷}
  \begin{phonetics}{大约}{da4yue1}[][HSK 3]
    \definition{adv.}{aproximadamente; sobre | provavelmente}
  \end{phonetics}
\end{entry}

\begin{entry}{大自然}{3,6,12}{⼤、⾃、⽕}
  \begin{phonetics}{大自然}{da4 zi4 ran2}[][HSK 2]
    \definition{s.}{natureza}
  \end{phonetics}
\end{entry}

\begin{entry}{大衣}{3,6}{⼤、⾐}
  \begin{phonetics}{大衣}{da4 yi1}[][HSK 2]
    \definition{s.}{sobretudo}
  \end{phonetics}
\end{entry}

\begin{entry}{大声}{3,7}{⼤、⼠}
  \begin{phonetics}{大声}{da4 sheng1}[][HSK 2]
    \definition{adj.}{alto volume | em voz alta}
  \end{phonetics}
\end{entry}

\begin{entry}{大纲}{3,7}{⼤、⽷}
  \begin{phonetics}{大纲}{da4 gang1}[][HSK 5]
    \definition{s.}{esboço; compêndio; programa de estudos; resumo; fundamentos da organização sistemática de conteúdos (livros, discursos, programas, etc.)}
  \end{phonetics}
\end{entry}

\begin{entry}{大豆}{3,7}{⼤、⾖}
  \begin{phonetics}{大豆}{da4dou4}
    \definition{s.}{soja}
  \end{phonetics}
\end{entry}

\begin{entry}{大陆}{3,7}{⼤、⾩}
  \begin{phonetics}{大陆}{da4 lu4}[][HSK 4]
    \definition*{s.}{China continental; refere-se especificamente à vasta porção terrestre do território da China}
    \definition[个,块]{s.}{terra firme; continente; vasta extensão de terra}
  \end{phonetics}
\end{entry}

\begin{entry}{大事}{3,8}{⼤、⼅}
  \begin{phonetics}{大事}{da4 shi4}[][HSK 5]
    \definition[件,桩]{s.}{grande evento; grande acontecimento; assunto importante; grande questão; algo importante |
situação geral | em grande escala; em grande estilo; em grande parte}
  \end{phonetics}
\end{entry}

\begin{entry}{大使馆}{3,8,11}{⼤、⼈、⾷}
  \begin{phonetics}{大使馆}{da4shi3guan3}[][HSK 3]
    \definition[座,个]{s.}{embaixada}
  \end{phonetics}
\end{entry}

\begin{entry}{大姐}{3,8}{⼤、⼥}
  \begin{phonetics}{大姐}{da4 jie3}[][HSK 4]
    \definition[个]{s.}{irmã mais velha (também um termo educado para se dirigir a uma garota ou mulher um pouco mais velha do que a pessoa que fala)}
  \end{phonetics}
\end{entry}

\begin{entry}{大学}{3,8}{⼤、⼦}
  \begin{phonetics}{大学}{da4 xue2}[][HSK 1]
    \definition[所]{s.}{faculdade | universidade}
  \end{phonetics}
\end{entry}

\begin{entry}{大学生}{3,8,5}{⼤、⼦、⽣}
  \begin{phonetics}{大学生}{da4 xue2 sheng1}[][HSK 1]
    \definition{s.}{estudante universitário}
  \end{phonetics}
\end{entry}

\begin{entry}{大规模}{3,8,14}{⼤、⾒、⽊}
  \begin{phonetics}{大规模}{da4 gui1 mo2}[][HSK 4]
    \definition{adj.}{em larga escala; extensivo; maciço; massa}
    \definition{adj.}{em larga escala; extensivo; maciço; massivo}
  \end{phonetics}
\end{entry}

\begin{entry}{大雨}{3,8}{⼤、⾬}
  \begin{phonetics}{大雨}{da4yu3}
    \definition[场]{s.}{chuva pesada, forte}
  \end{phonetics}
\end{entry}

\begin{entry}{大前天}{3,9,4}{⼤、⼑、⼤}
  \begin{phonetics}{大前天}{da4qian2tian1}
    \definition{adv.}{três dias atrás}
  \end{phonetics}
\end{entry}

\begin{entry}{大型}{3,9}{⼤、⼟}
  \begin{phonetics}{大型}{da4xing2}[][HSK 4]
    \definition{adj.}{grande; em larga escala; tamanho e volume grandes | larga escala (importante e influente)}
  \end{phonetics}
\end{entry}

\begin{entry}{大奖赛}{3,9,14}{⼤、⼤、⾙}
  \begin{phonetics}{大奖赛}{da4 jiang3 sai4}[][HSK 5]
    \definition{s.}{grande competição; grande prêmio; \emph{grand prix}}
  \end{phonetics}
\end{entry}

\begin{entry}{大战}{3,9}{⼤、⼽}
  \begin{phonetics}{大战}{da4zhan4}
    \definition{s.}{guerra}
    \definition{v.}{guerrear | lutar em uma guerra}
  \end{phonetics}
\end{entry}

\begin{entry}{大洋洲}{3,9,9}{⼤、⽔、⽔}
  \begin{phonetics}{大洋洲}{da4yang2zhou1}
    \definition*{s.}{Oceania}
  \end{phonetics}
\end{entry}

\begin{entry}{大神}{3,9}{⼤、⽰}
  \begin{phonetics}{大神}{da4shen2}
    \definition{s.}{deidade | (gíria da Internet) guru | \emph{expert} | gênio}
  \end{phonetics}
\end{entry}

\begin{entry}{大胆}{3,9}{⼤、⾁}
  \begin{phonetics}{大胆}{da4 dan3}[][HSK 5]
    \definition{adj.}{ousado; atrevido; audacioso; corajoso; destemido}
  \end{phonetics}
\end{entry}

\begin{entry}{大哥}{3,10}{⼤、⼝}
  \begin{phonetics}{大哥}{da4 ge1}[][HSK 4]
    \definition{s.}{irmão mais velho | \emph{big brother} (endereço educado para um homem da mesma idade que você) | líder de gangue; pessoa mais poderosa em uma organização que realiza atividades ilegais na sociedade}
  \end{phonetics}
\end{entry}

\begin{entry}{大家}{3,10}{⼤、⼧}
  \begin{phonetics}{大家}{da4jia1}[][HSK 2]
    \definition{pron.}{todos}
  \end{phonetics}
\end{entry}

\begin{entry}{大海}{3,10}{⼤、⽔}
  \begin{phonetics}{大海}{da4 hai3}[][HSK 2]
    \definition{s.}{mar | oceano}
  \end{phonetics}
\end{entry}

\begin{entry}{大脑}{3,10}{⼤、⾁}
  \begin{phonetics}{大脑}{da4 nao3}[][HSK 5]
    \definition{s.}{cérebro; encéfalo}
  \end{phonetics}
\end{entry}

\begin{entry}{大致}{3,10}{⼤、⾄}
  \begin{phonetics}{大致}{da4zhi4}[][HSK 5]
    \definition{adj.}{geral; no todo}
    \definition{adv.}{grosso modo; aproximadamente; mais ou menos; indica uma estimativa aproximada da situação}
  \end{phonetics}
\end{entry}

\begin{entry}{大部分}{3,10,4}{⼤、⾢、⼑}
  \begin{phonetics}{大部分}{da4 bu4 fen4}[][HSK 2]
    \definition{s.}{a maioria | a maior parte}
  \end{phonetics}
\end{entry}

\begin{entry}{大都}{3,10}{⼤、⾢}
  \begin{phonetics}{大都}{da4 dou1}[][HSK 5]
  \end{phonetics}
  \begin{phonetics}{大都}{da4 du1}
    \definition{adv.}{em sua maior parte; na maior parte; indica que a maioria das pessoas ou coisas em um determinado intervalo tem a mesma natureza e características | também pronunciado como \dpy{da4dou1} na língua falada}
  \end{phonetics}
\end{entry}

\begin{entry}{大象}{3,11}{⼤、⾗}
  \begin{phonetics}{大象}{da4xiang4}[][HSK 5]
    \definition[只,头,群,个]{s.}{elefante}
  \end{phonetics}
\end{entry}

\begin{entry}{大猩猩}{3,12,12}{⼤、⽝、⽝}
  \begin{phonetics}{大猩猩}{da4xing1xing5}
    \definition{s.}{gorila}
  \end{phonetics}
\end{entry}

\begin{entry}{大量}{3,12}{⼤、⾥}
  \begin{phonetics}{大量}{da4 liang4}[][HSK 2]
    \definition{adj.}{numeroso | em massa | grande em número ou quantidade | generoso | magnânimo}
  \end{phonetics}
\end{entry}

\begin{entry}{大楼}{3,13}{⼤、⽊}
  \begin{phonetics}{大楼}{da4 lou2}[][HSK 4]
    \definition[座,幢]{s.}{edifício; mansão; edifício de vários andares disponível para uso residencial e comercial}
  \end{phonetics}
\end{entry}

\begin{entry}{大概}{3,13}{⼤、⽊}
  \begin{phonetics}{大概}{da4gai4}[][HSK 3]
    \definition{adj.}{geral; grosseiro; aproximado}
    \definition{adv.}{sobre; provavelmente
geralmente; brevemente}
    \definition{s.}{ideia geral; esboço geral}
  \end{phonetics}
\end{entry}

\begin{entry}{大腿}{3,13}{⼤、⾁}
  \begin{phonetics}{大腿}{da4tui3}
    \definition{s.}{coxa}
  \end{phonetics}
\end{entry}

\begin{entry}{大蒜}{3,13}{⼤、⾋}
  \begin{phonetics}{大蒜}{da4suan4}
    \definition[瓣,头]{s.}{alho}
  \end{phonetics}
\end{entry}

\begin{entry}{大熊猫}{3,14,11}{⼤、⽕、⽝}
  \begin{phonetics}{大熊猫}{da4 xiong2 mao1}[][HSK 5]
    \definition{s.}{panda gigante}
  \end{phonetics}
\end{entry}

\begin{entry}{大赛}{3,14}{⼤、⾙}
  \begin{phonetics}{大赛}{da4sai4}
    \definition{s.}{grande concurso, competição}
  \end{phonetics}
\end{entry}

\begin{entry}{女}{3}{⼥}[Kangxi 38]
  \begin{phonetics}{女}{nv3}[][HSK 1]
    \definition{adj.}{feminino}
  \end{phonetics}
\end{entry}

\begin{entry}{女人}{3,2}{⼥、⼈}
  \begin{phonetics}{女人}{nv3ren2}[][HSK 1]
    \definition[个,位]{s.}{mulher}
  \end{phonetics}
\end{entry}

\begin{entry}{女儿}{3,2}{⼥、⼉}
  \begin{phonetics}{女儿}{nv3'er2}[][HSK 1]
    \definition{s.}{filha}
  \seealsoref{儿子}{er2zi5}
  \end{phonetics}
\end{entry}

\begin{entry}{女士}{3,3}{⼥、⼠}
  \begin{phonetics}{女士}{nv3shi4}[][HSK 4]
    \definition{pron.}{Sra.; Senhorita; Senhora; título honorífico para mulheres (agora usado em contextos diplomáticos)}
    \definition[位,个]{s.}{senhora; madame}
  \end{phonetics}
\end{entry}

\begin{entry}{女子}{3,3}{⼥、⼦}
  \begin{phonetics}{女子}{nv3 zi3}[][HSK 3]
    \definition[位]{s.}{mulher; feminino}
  \end{phonetics}
\end{entry}

\begin{entry}{女王}{3,4}{⼥、⽟}
  \begin{phonetics}{女王}{nv3wang2}
    \definition{s.}{rainha}
  \end{phonetics}
\end{entry}

\begin{entry}{女生}{3,5}{⼥、⽣}
  \begin{phonetics}{女生}{nv3sheng1}[][HSK 1]
    \definition[个]{s.}{aluna | estudante so sexo feminino}
  \end{phonetics}
\end{entry}

\begin{entry}{女朋友}{3,8,4}{⼥、⽉、⼜}
  \begin{phonetics}{女朋友}{nv3peng2you5}[][HSK 1]
    \definition{s.}{namorada}
  \end{phonetics}
\end{entry}

\begin{entry}{女孩}{3,9}{⼥、⼦}
  \begin{phonetics}{女孩}{nv3hai2}
    \definition{s.}{menina | garota}
  \end{phonetics}
\end{entry}

\begin{entry}{女孩儿}{3,9,2}{⼥、⼦、⼉}
  \begin{phonetics}{女孩儿}{nv3hai2r5}[][HSK 1]
  \end{phonetics}
\end{entry}

\begin{entry}{女婿}{3,12}{⼥、⼥}
  \begin{phonetics}{女婿}{nv3xu5}
    \definition{s.}{marido da filha}
  \end{phonetics}
\end{entry}

\begin{entry}{子}{3}{⼦}
  \begin{phonetics}{子}{zi3}
    \definition{adj.}{jovem | pequeno | tenro}
    \definition{clas.}{para objetos finos que podem ser pinçados com os dedos}
    \definition{pron.}{você}
    \definition{s.}{filho | pessoa | antigo título de respeito para um homem culto ou virtuoso | semente | ovo; ova | coisas pequenas e duras | moeda de cobre; cobre | o quarto título da classificação dos cinco títulos feudais de nobreza; visconde}
  \end{phonetics}
  \begin{phonetics}{子}{zi5}[][HSK 1]
    \definition{clas.}{sufixos de palavras de medida individuais}
    \definition{suf.}{sufixo para substantivos}
  \end{phonetics}
\end{entry}

\begin{entry}{子女}{3,3}{⼦、⼥}
  \begin{phonetics}{子女}{zi3 nv3}[][HSK 3]
    \definition{s.}{crianças; descendência; filhos e filhas}
  \end{phonetics}
\end{entry}

\begin{entry}{子弹}{3,11}{⼦、⼸}
  \begin{phonetics}{子弹}{zi3dan4}
    \definition[粒,颗,发]{s.}{bala (de revólver)}
  \end{phonetics}
\end{entry}

\begin{entry}{寸}{3}{⼨}
  \begin{phonetics}{寸}{cun4}[][HSK 5]
    \definition*{s.}{sobrenome Cun}
    \definition{adj.}{muito pouco; muito curto; pequeno}
    \definition{clas.}{cun, uma unidade de comprimento (=13 decímetros)}
  \end{phonetics}
\end{entry}

\begin{entry}{小}{3}{⼩}[Kangxi 42]
  \begin{phonetics}{小}{xiao3}[][HSK 1,2]
    \definition{adj.}{pequeno | jovem}
  \end{phonetics}
\end{entry}

\begin{entry}{小小}{3,3}{⼩、⼩}
  \begin{phonetics}{小小}{xiao3xiao3}
    \definition{adj.}{muito pequeno}
  \end{phonetics}
\end{entry}

\begin{entry}{小区}{3,4}{⼩、⼖}
  \begin{phonetics}{小区}{xiao3qu1}
    \definition{s.}{conjunto habitacional, comunidade, bairro | célula (telecomunicações)}
  \end{phonetics}
\end{entry}

\begin{entry}{小心}{3,4}{⼩、⼼}
  \begin{phonetics}{小心}{xiao3xin1}[][HSK 2]
    \definition{adj.}{cuidado}
  \end{phonetics}
\end{entry}

\begin{entry}{小气鬼}{3,4,9}{⼩、⽓、⿁}
  \begin{phonetics}{小气鬼}{xiao3qi4gui3}
    \definition{adj.}{avarento | mão-de-vaca | miserável | pão-duro}
  \end{phonetics}
\end{entry}

\begin{entry}{小白菜}{3,5,11}{⼩、⽩、⾋}
  \begin{phonetics}{小白菜}{xiao3bai2cai4}
    \definition[棵]{s.}{\emph{bok choy} | couve chinesa}
  \end{phonetics}
\end{entry}

\begin{entry}{小众}{3,6}{⼩、⼈}
  \begin{phonetics}{小众}{xiao3zhong4}
    \definition{s.}{minoria da população | nicho (mercado, etc.)}
  \end{phonetics}
\end{entry}

\begin{entry}{小伙子}{3,6,3}{⼩、⼈、⼦}
  \begin{phonetics}{小伙子}{xiao3huo3zi5}[][HSK 4]
    \definition[个]{s.}{rapaz jovem; jovem colega}
  \end{phonetics}
\end{entry}

\begin{entry}{小吃}{3,6}{⼩、⼝}
  \begin{phonetics}{小吃}{xiao3chi1}[][HSK 4]
    \definition{s.}{lanche; petiscos; comida com especialidades locais, não muito para uma porção | prato frio; prato feito; cortes de frios na culinária ocidental | pratos pequenos e baratos; pratos simples em restaurantes com porções pequenas e preços baixos}
  \end{phonetics}
\end{entry}

\begin{entry}{小声}{3,7}{⼩、⼠}
  \begin{phonetics}{小声}{xiao3 sheng1}[][HSK 2]
    \definition{v.}{falar em voz baixa | sussurar}
  \end{phonetics}
\end{entry}

\begin{entry}{小时}{3,7}{⼩、⽇}
  \begin{phonetics}{小时}{xiao3shi2}[][HSK 1]
    \definition{adv.}{hora | para horas}
    \definition[个]{s.}{hora}
  \end{phonetics}
\end{entry}

\begin{entry}{小时候}{3,7,10}{⼩、⽇、⼈}
  \begin{phonetics}{小时候}{xiao3 shi2 hou5}[][HSK 2]
    \definition{s.}{na infância | quando alguém era jovem}
  \end{phonetics}
\end{entry}

\begin{entry}{小姐}{3,8}{⼩、⼥}
  \begin{phonetics}{小姐}{xiao3jie5}[][HSK 1]
    \definition[个,位]{s.}{senhorita | jovem senhora | (gíria) prostituta}
  \end{phonetics}
\end{entry}

\begin{entry}{小学}{3,8}{⼩、⼦}
  \begin{phonetics}{小学}{xiao3xue2}[][HSK 1]
    \definition{s.}{escola ensino fundamental}
  \end{phonetics}
\end{entry}

\begin{entry}{小学生}{3,8,5}{⼩、⼦、⽣}
  \begin{phonetics}{小学生}{xiao3xue2sheng1}[][HSK 1]
    \definition{s.}{aluno, estudante de escola primária}
  \end{phonetics}
\end{entry}

\begin{entry}{小朋友}{3,8,4}{⼩、⽉、⼜}
  \begin{phonetics}{小朋友}{xiao3peng2you3}[][HSK 1]
    \definition{s.}{criança | [termo de tratamento usado por um adulto para uma criança] amiguinho}
  \end{phonetics}
\end{entry}

\begin{entry}{小狗}{3,8}{⼩、⽝}
  \begin{phonetics}{小狗}{xiao3 gou3}
    \definition{s.}{filhote de cachorro}
  \end{phonetics}
\end{entry}

\begin{entry}{小组}{3,8}{⼩、⽷}
  \begin{phonetics}{小组}{xiao3 zu3}[][HSK 2]
    \definition[个]{s.}{grupo}
  \end{phonetics}
\end{entry}

\begin{entry}{小型}{3,9}{⼩、⼟}
  \begin{phonetics}{小型}{xiao3 xing2}[][HSK 4]
    \definition{adj.}{de tamanho pequeno; em pequena escala; miniatura; tipo pequeno; tamanho de bolso; tipo compacto}
    \definition{s.}{(Mediterrâneo) escunas, pequenos veleiros de pesca ou turismo | pequeno \emph{rover} lunar (duas pessoas)}
  \end{phonetics}
\end{entry}

\begin{entry}{小孩儿}{3,9,2}{⼩、⼦、⼉}
  \begin{phonetics}{小孩儿}{xiao3hai2r5}[][HSK 1]
    \definition[个]{s.}{criança | bebê}
  \end{phonetics}
\end{entry}

\begin{entry}{小屋}{3,9}{⼩、⼫}
  \begin{phonetics}{小屋}{xiao3wu1}
    \definition{s.}{cabana | chalé | cabine}
  \end{phonetics}
\end{entry}

\begin{entry}{小树}{3,9}{⼩、⽊}
  \begin{phonetics}{小树}{xiao3shu4}
    \definition[棵]{s.}{muda | arbusto | árvore pequena}
  \end{phonetics}
\end{entry}

\begin{entry}{小洋白菜}{3,9,5,11}{⼩、⽔、⽩、⾋}
  \begin{phonetics}{小洋白菜}{xiao3 yang2bai2cai4}
    \definition{s.}{couve de bruxelas}
  \end{phonetics}
\end{entry}

\begin{entry}{小说}{3,9}{⼩、⾔}
  \begin{phonetics}{小说}{xiao3shuo1}[][HSK 2]
    \definition[本,部]{s.}{romance | ficção}
  \end{phonetics}
\end{entry}

\begin{entry}{小腿}{3,13}{⼩、⾁}
  \begin{phonetics}{小腿}{xiao3tui3}
    \definition{s.}{perna (do joelho ao calcanhar) | haste}
  \end{phonetics}
\end{entry}

\begin{entry}{山}{3}{⼭}[Kangxi 46]
  \begin{phonetics}{山}{shan1}[][HSK 1]
    \definition*{s.}{sobrenome Shan}
    \definition[座]{s.}{montanha | monte | qualquer coisa que se assemelhe a uma montanha}
  \end{phonetics}
\end{entry}

\begin{entry}{山区}{3,4}{⼭、⼖}
  \begin{phonetics}{山区}{shan1qu1}
    \definition[个]{s.}{área montanhosa | montanhas}
  \end{phonetics}
\end{entry}

\begin{entry}{山东}{3,5}{⼭、⼀}
  \begin{phonetics}{山东}{shan1dong1}
    \definition*{s.}{Shandong}
  \end{phonetics}
\end{entry}

\begin{entry}{山羊}{3,6}{⼭、⽺}
  \begin{phonetics}{山羊}{shan1yang2}
    \definition{s.}{cabra | (ginástica) cavalo de salto de pequeno porte}
  \end{phonetics}
\end{entry}

\begin{entry}{山体}{3,7}{⼭、⼈}
  \begin{phonetics}{山体}{shan1ti3}
    \definition{s.}{forma de uma montanha}
  \end{phonetics}
\end{entry}

\begin{entry}{山谷}{3,7}{⼭、⾕}
  \begin{phonetics}{山谷}{shan1gu3}
    \definition{s.}{vale | ravina}
  \end{phonetics}
\end{entry}

\begin{entry}{山顶}{3,8}{⼭、⾴}
  \begin{phonetics}{山顶}{shan1ding3}
    \definition{s.}{cume da montanha}
  \end{phonetics}
\end{entry}

\begin{entry}{山寨}{3,14}{⼭、⼧}
  \begin{phonetics}{山寨}{shan1zhai4}
    \definition{s.}{fortaleza fortificada da vila | fortaleza da montanha (especialmente de bandidos) | falsificação | imitação | (fig.) pechincha}
  \end{phonetics}
\end{entry}

\begin{entry}{工}{3}{⼯}[Kangxi 48]
  \begin{phonetics}{工}{gong1}
    \definition{s.}{trabalho | trabalhador | habilidade | profissão | comércio | ofício}
  \end{phonetics}
\end{entry}

\begin{entry}{工人}{3,2}{⼯、⼈}
  \begin{phonetics}{工人}{gong1ren2}[][HSK 1]
    \definition{s.}{trabalhador | operário | mão de obra de fábrica}
  \end{phonetics}
\end{entry}

\begin{entry}{工厂}{3,2}{⼯、⼚}
  \begin{phonetics}{工厂}{gong1chang3}[][HSK 3]
    \definition[家,座,个]{s.}{fábrica; moinho; planta; obras}
  \end{phonetics}
\end{entry}

\begin{entry}{工夫}{3,4}{⼯、⼤}
  \begin{phonetics}{工夫}{gong1 fu1}
    \definition[个]{s.}{tempo | tempo livre; lazer}
  \end{phonetics}
  \begin{phonetics}{工夫}{gong1 fu5}[][HSK 3]
    \definition{s.}{(um período de) tempo | tempo livre}
  \end{phonetics}
\end{entry}

\begin{entry}{工尺谱}{3,4,14}{⼯、⼫、⾔}
  \begin{phonetics}{工尺谱}{gong1 che3 pu3}
    \definition{s.}{notação musical tradicional chinesa que usa caracteres chineses para representar notas musicais}
  \end{phonetics}
\end{entry}

\begin{entry}{工艺}{3,4}{⼯、⾋}
  \begin{phonetics}{工艺}{gong1yi4}
    \definition{s.}{artesanato}
  \end{phonetics}
\end{entry}

\begin{entry}{工艺品}{3,4,9}{⼯、⾋、⼝}
  \begin{phonetics}{工艺品}{gong1yi4pin3}
    \definition[个]{s.}{artigo de artesanato | trabalho manual}
  \end{phonetics}
\end{entry}

\begin{entry}{工业}{3,5}{⼯、⼀}
  \begin{phonetics}{工业}{gong1ye4}[][HSK 3]
    \definition{s.}{indústria}
  \end{phonetics}
\end{entry}

\begin{entry}{工作}{3,7}{⼯、⼈}
  \begin{phonetics}{工作}{gong1zuo4}[][HSK 1]
    \definition[个,份,项]{s.}{trabalho | tarefa}
    \definition{v.}{trabalhar | operar (uma máquina)}
  \end{phonetics}
\end{entry}

\begin{entry}{工具}{3,8}{⼯、⼋}
  \begin{phonetics}{工具}{gong1ju4}[][HSK 3]
    \definition[个]{s.}{ferramenta; implemento | ferramenta; meio; instrumento}
  \end{phonetics}
\end{entry}

\begin{entry}{工资}{3,10}{⼯、⾙}
  \begin{phonetics}{工资}{gong1zi1}[][HSK 3]
    \definition[份,个,年,月,天]{s.}{pagamento; salário}
  \end{phonetics}
\end{entry}

\begin{entry}{工程}{3,12}{⼯、⽲}
  \begin{phonetics}{工程}{gong1 cheng2}[][HSK 4]
    \definition[个,项]{s.}{projeto; programa; trabalhos que utilizam equipamentos grandes e complexos, como projetos de reconstrução urbana e projetos de cestas de alimentos, etc. | engenharia; departamentos de produção e manufatura usam equipamentos grandes e complexos para realizar seu trabalho}
  \end{phonetics}
\end{entry}

\begin{entry}{工程师}{3,12,6}{⼯、⽲、⼱}
  \begin{phonetics}{工程师}{gong1cheng2shi1}[][HSK 3]
    \definition[个,名]{s.}{engenheiro}
  \end{phonetics}
\end{entry}

\begin{entry}{工龄}{3,13}{⼯、⿒}
  \begin{phonetics}{工龄}{gong1ling2}
    \definition{s.}{tempo de serviço | senioridade}
  \end{phonetics}
\end{entry}

\begin{entry}{已}{3}{⼰}
  \begin{phonetics}{已}{yi3}[][HSK 3]
    \definition{adv.}{já | depois; mais tarde; depois de um tempo}
  \end{phonetics}
\end{entry}

\begin{entry}{已久}{3,3}{⼰、⼃}
  \begin{phonetics}{已久}{yi3jiu3}
    \definition{adv.}{já faz muito tempo}
  \end{phonetics}
\end{entry}

\begin{entry}{已灭}{3,5}{⼰、⽕}
  \begin{phonetics}{已灭}{yi3mie4}
    \definition{adj.}{extinto}
  \end{phonetics}
\end{entry}

\begin{entry}{已知}{3,8}{⼰、⽮}
  \begin{phonetics}{已知}{yi3zhi1}
    \definition{v.}{conhecido (ter ciência)}
  \end{phonetics}
\end{entry}

\begin{entry}{已经}{3,8}{⼰、⽷}
  \begin{phonetics}{已经}{yi3jing1}[][HSK 2]
    \definition{adv.}{já}
  \end{phonetics}
\end{entry}

\begin{entry}{已故}{3,9}{⼰、⽁}
  \begin{phonetics}{已故}{yi3gu4}
    \definition{adj.}{morto | atrasado}
  \end{phonetics}
\end{entry}

\begin{entry}{已婚}{3,11}{⼰、⼥}
  \begin{phonetics}{已婚}{yi3hun1}
    \definition{adj.}{casado}
  \end{phonetics}
\end{entry}

\begin{entry}{已然}{3,12}{⼰、⽕}
  \begin{phonetics}{已然}{yi3ran2}
    \definition{adv.}{já | já ser assim}
  \end{phonetics}
\end{entry}

\begin{entry}{干}{3}{⼲}
  \begin{phonetics}{干}{gan1}[][HSK 1]
    \definition*{s.}{sobrenome Gan}
    \definition{v.}{preocupar | ignorar | interferir}
  \end{phonetics}
  \begin{phonetics}{干}{gan4}[][HSK 1]
    \definition{v.}{fazer | gerenciar | trabalhar | (gíria) matar | (vulgar) foder}
  \end{phonetics}
\end{entry}

\begin{entry}{干与}{3,3}{⼲、⼀}
  \begin{phonetics}{干与}{gan1yu4}
    \variantof{干预}
  \end{phonetics}
\end{entry}

\begin{entry}{干什么}{3,4,3}{⼲、⼈、⼃}
  \begin{phonetics}{干什么}{gan4 shen2 me5}[][HSK 1]
    \definition{v.}{o que fazer? | o que está fazendo?}
  \end{phonetics}
\end{entry}

\begin{entry}{干吗}{3,6}{⼲、⼝}
  \begin{phonetics}{干吗}{gan4 ma2}[][HSK 3]
    \definition{pron.}{por que?}
    \definition{v.}{o que fazer?}
  \end{phonetics}
\end{entry}

\begin{entry}{干你屁事}{3,7,7,8}{⼲、⼈、⼫、⼅}
  \begin{phonetics}{干你屁事}{gan1 ni3 pi4shi4}
    \definition{interj.}{Foda-se!}
  \end{phonetics}
\end{entry}

\begin{entry}{干扰}{3,7}{⼲、⼿}
  \begin{phonetics}{干扰}{gan1rao3}[][HSK 5]
    \definition{v.}{perturbar; incomodar | interferir; interromper o funcionamento adequado de equipamentos eletrônicos com sinais eletrônicos dispersos}
  \end{phonetics}
\end{entry}

\begin{entry}{干净}{3,8}{⼲、⼎}
  \begin{phonetics}{干净}{gan1jing4}[][HSK 1]
    \definition{adj.}{limpo | arrumado}
  \end{phonetics}
\end{entry}

\begin{entry}{干杯}{3,8}{⼲、⽊}
  \begin{phonetics}{干杯}{gan1bei1}[][HSK 2]
    \definition{interj.}{Saúde!}
    \definition{v.+compl.}{fazer um brinde | brindar até a última gota}
  \end{phonetics}
\end{entry}

\begin{entry}{干活}{3,9}{⼲、⽔}
  \begin{phonetics}{干活}{gan4huo2}
    \definition{v.+compl.}{trabalhar | trabalhar em um emprego}
  \end{phonetics}
\end{entry}

\begin{entry}{干活儿}{3,9,2}{⼲、⽔、⼉}
  \begin{phonetics}{干活儿}{gan4huo2r5}[][HSK 2]
    \definition{v.}{trabalhar em um emprego}
  \end{phonetics}
\end{entry}

\begin{entry}{干脆}{3,10}{⼲、⾁}
  \begin{phonetics}{干脆}{gan1cui4}[][HSK 5]
    \definition{adj.}{claro; direto; (falando, fazendo coisas) sem hesitação; atitude clara}
    \definition{adv.}{justamente; diretamente; sem maiores considerações}
  \end{phonetics}
\end{entry}

\begin{entry}{干预}{3,10}{⼲、⾴}
  \begin{phonetics}{干预}{gan1yu4}[][HSK 5]
    \definition{s.}{intromissão; intervenção}
    \definition{v.}{intrometer-se; intervir; interpor-se;}
  \end{phonetics}
\end{entry}

\begin{entry}{广大}{3,3}{⼴、⼤}
  \begin{phonetics}{广大}{guang3da4}[][HSK 3]
    \definition{adj.}{muito difundido | (uma área ou espaço) vasto; extenso; em grande escala | numeroso}
  \end{phonetics}
\end{entry}

\begin{entry}{广东}{3,5}{⼴、⼀}
  \begin{phonetics}{广东}{guang3dong1}
    \definition*{s.}{Guangdong}
  \end{phonetics}
\end{entry}

\begin{entry}{广场}{3,6}{⼴、⼟}
  \begin{phonetics}{广场}{guang3chang3}[][HSK 2]
    \definition{s.}{praça | praça pública | esplanada}
  \end{phonetics}
\end{entry}

\begin{entry}{广场舞}{3,6,14}{⼴、⼟、⾇}
  \begin{phonetics}{广场舞}{guang3chang3wu3}
    \definition{s.}{quadrilha, uma rotina de exercícios tocada com música em quadrados públicos, parques e praças, popular especialmente entre mulheres de meia-idade e aposentados na China}
  \end{phonetics}
\end{entry}

\begin{entry}{广告}{3,7}{⼴、⼝}
  \begin{phonetics}{广告}{guang3gao4}[][HSK 2]
    \definition[项]{s.}{publicidade | anúncio publicitário}
  \end{phonetics}
\end{entry}

\begin{entry}{广播}{3,15}{⼴、⼿}
  \begin{phonetics}{广播}{guang3bo1}[][HSK 3]
    \definition[个]{s.}{programa de rádio; transmissão (de rádio)}
    \definition{v.}{transmitir; estar no ar | espalhar-se amplamente; ser conhecido em toda parte}
  \end{phonetics}
\end{entry}

\begin{entry}{才}{3}{⼿}
  \begin{phonetics}{才}{cai2}[][HSK 2,4]
    \definition*{s.}{sobrenome Cai}
    \definition{adv.}{indica que algo aconteceu há pouco tempo, agora mesmo | indica que algo acontece ou termina tarde | indica que algo só acontece sob certas condições, ou por um motivo ou propósito específico, seguido do que acontece depois, geralmente é precedida por palavras como “somente”, “deve”, “porque” ou “devido a” | em comparação, indica uma pequena quantidade, poucas ocorrências, pouca habilidade, etc.; meramente | indica ênfase no que está sendo dito, e o caractere “呢” é frequentemente usado no final da frase}
    \definition{conj.}{apenas quando}
    \definition{s.}{capacidade; talento; dom | pessoa capacitada}
  \seealsoref{呢}{ne5}
  \end{phonetics}
\end{entry}

\begin{entry}{才华}{3,6}{⼿、⼗}
  \begin{phonetics}{才华}{cai2hua2}
    \definition[份]{s.}{talento}
  \end{phonetics}
\end{entry}

\begin{entry}{才能}{3,10}{⼿、⾁}
  \begin{phonetics}{才能}{cai2 neng2}[][HSK 3]
    \definition[间]{s.}{talento | habilidade | dom | capacidade}
  \end{phonetics}
\end{entry}

\begin{entry}{才略}{3,11}{⼿、⽥}
  \begin{phonetics}{才略}{cai2lve4}
    \definition{s.}{habilidade e sagacidade}
  \end{phonetics}
\end{entry}

\begin{entry}{门}{3}{⾨}[Kangxi 169]
  \begin{phonetics}{门}{men2}[][HSK 1]
    \definition*{s.}{sobrenome Men}
    \definition{clas.}{para canhão | para lição de casa, tecnologia, etc.}
    \definition{s.}{porta | portão | entrada; saída | interruptor | válvula |maneira | método | acesso | família | casa | escola (de pensamento) | seita (religiosa) | ramo de estudo | categoria; classe | filo}
  \end{phonetics}
\end{entry}

\begin{entry}{门口}{3,3}{⾨、⼝}
  \begin{phonetics}{门口}{men2kou3}[][HSK 1]
    \definition[个]{s.}{porta | portão}
  \end{phonetics}
\end{entry}

\begin{entry}{门票}{3,11}{⾨、⽰}
  \begin{phonetics}{门票}{men2piao4}[][HSK 1]
    \definition{s.}{bilhete de entrada | bilhete de admissão}
  \end{phonetics}
\end{entry}

\begin{entry}{飞}{3}{⾶}[Kangxi 183]
  \begin{phonetics}{飞}{fei1}[][HSK 1]
    \definition*{s.}{sobrenome Fei}
    \definition{adj.}{inesperado | acidental | infundado | sem fundamento}
    \definition{adv.}{rapidamente}
    \definition{s.}{roda livre de uma bicicleta}
    \definition{v.}{voar | esvoaçar | flutuar no ar | volatilizar}
  \end{phonetics}
\end{entry}

\begin{entry}{飞机}{3,6}{⾶、⽊}
  \begin{phonetics}{飞机}{fei1ji1}[][HSK 1]
    \definition[架]{s.}{avião}
  \end{phonetics}
\end{entry}

\begin{entry}{飞机票}{3,6,11}{⾶、⽊、⽰}
  \begin{phonetics}{飞机票}{fei1ji1 piao4}
    \definition[张]{s.}{bilhete de avião}
  \seealsoref{机票}{ji1 piao4}
  \end{phonetics}
\end{entry}

\begin{entry}{飞行}{3,6}{⾶、⾏}
  \begin{phonetics}{飞行}{fei1 xing2}[][HSK 3]
    \definition{s.}{voo | aviação}
    \definition{v.}{voar; fazer um voo | (aviões, foguetes, etc.) voar no ar}
  \end{phonetics}
\end{entry}

\begin{entry}{飞船}{3,11}{⾶、⾈}
  \begin{phonetics}{飞船}{fei1chuan2}
    \definition{s.}{espaçonave | dirigível | aeronave}
  \end{phonetics}
\end{entry}

\begin{entry}{飞碟}{3,14}{⾶、⽯}
  \begin{phonetics}{飞碟}{fei1die2}
    \definition{s.}{disco-voador, OVNI, \emph{UFO} | \emph{frisbee}}
  \end{phonetics}
\end{entry}

\begin{entry}{马}{3}{⾺}[Kangxi 187]
  \begin{phonetics}{马}{ma3}[][HSK 3]
    \definition*{s.}{sobrenome Ma}
    \definition{adj.}{grande}
    \definition[匹]{s.}{cavalo | a peça do cavalo no xadrez chinês}
  \end{phonetics}
\end{entry}

\begin{entry}{马上}{3,3}{⾺、⼀}
  \begin{phonetics}{马上}{ma3shang4}[][HSK 1]
    \definition{adv.}{já | imediatamente | de imediato | sem demora}
  \end{phonetics}
\end{entry}

\begin{entry}{马马虎虎}{3,3,8,8}{⾺、⾺、⾌、⾌}
  \begin{phonetics}{马马虎虎}{ma3ma3hu3hu3}
    \definition{adj.}{descuidado | casual | tolerável | vago | mais ou menos}
  \end{phonetics}
\end{entry}

\begin{entry}{马耳他}{3,6,5}{⾺、⽿、⼈}
  \begin{phonetics}{马耳他}{ma3'er3ta1}
    \definition*{s.}{Malta}
  \end{phonetics}
\end{entry}

\begin{entry}{马克思列宁主义}{3,7,9,6,5,5,3}{⾺、⼗、⼼、⼑、⼧、⼂、⼂}
  \begin{phonetics}{马克思列宁主义}{ma3ke4si1 lie4ning2zhu3yi4}
    \definition*{s.}{Marxismo-Leninismo}
  \end{phonetics}
\end{entry}

\begin{entry}{马尾}{3,7}{⾺、⼫}
  \begin{phonetics}{马尾}{ma3wei3}
    \definition{s.}{(penteado) rabo de cavalo | cauda de cavalo}
  \end{phonetics}
\end{entry}

\begin{entry}{马路}{3,13}{⾺、⾜}
  \begin{phonetics}{马路}{ma3lu4}[][HSK 1]
    \definition[条]{s.}{rua | estrada}
  \end{phonetics}
\end{entry}

%%%%% EOF %%%%%


%%%
%%% 4画
%%%

\section*{4画}\addcontentsline{toc}{section}{4画}

\begin{entry}{不}{4}[Radical 一]
  \begin{phonetics}{不}{bu2}[(antes de quarto tom)]
    \definition{adv.}{não}
    \definition{pref.}{prefixo negativo}
  \end{phonetics}
  \begin{phonetics}{不}{bu4}
    \definition{adv.}{não}
    \definition{pref.}{prefixo negativo}
  \end{phonetics}
  \begin{phonetics}{不}{bu5}
    \definition{adv.}{não (em expressões ``v.+不+v.'')}
  \end{phonetics}
\end{entry}

\begin{entry}{不大离}{4,3,10}
  \begin{phonetics}{不大离}{bu2da4li2}
    \definition{adj.}{bem perto | quase certo | nada mal}
  \end{phonetics}
\end{entry}

\begin{entry}{不公}{4,4}
  \begin{phonetics}{不公}{bu4gong1}
    \definition{adj.}{injusto}
  \end{phonetics}
\end{entry}

\begin{entry}{不日}{4,4}
  \begin{phonetics}{不日}{bu2ri4}
    \definition{adv.}{em alguns dias}
  \end{phonetics}
\end{entry}

\begin{entry}{不止}{4,4}
  \begin{phonetics}{不止}{bu4zhi3}
    \definition{adv.}{incessantemente | sem fim | mais que | não limitado a}
  \end{phonetics}
\end{entry}

\begin{entry}{不可避免}{4,5,16,7}
  \begin{phonetics}{不可避免}{bu4ke3bi4mian3}
    \definition{adj./adv.}{inevitável}
  \end{phonetics}
\end{entry}

\begin{entry}{不用}{4,5}
  \begin{phonetics}{不用}{bu2yong4}
    \definition{v.}{não precisar}
    \seeref{甭}{beng2}
  \end{phonetics}
\end{entry}

\begin{entry}{不同}{4,6}
  \begin{phonetics}{不同}{bu4tong2}
    \definition{adj.}{diferente | distinto}
  \end{phonetics}
\end{entry}

\begin{entry}{不成话}{4,6,8}
  \begin{phonetics}{不成话}{bu4cheng2hua4}
    \definition{expr.}{sem razão | demasiado irracionável}
    \seeref{不是话}{bu2shi4hua4}
    \seeref{不像话}{bu2xiang4hua4}
  \end{phonetics}
\end{entry}

\begin{entry}{不论……也……}{4,6,3}
  \begin{phonetics}{不论……也……}{bu2lun4 ye3}
    \definition{conj.}{não apenas\dots, (o que, quem, como, etc.), \dots}
  \end{phonetics}
\end{entry}

\begin{entry}{不论……都……}{4,6,10}
  \begin{phonetics}{不论……都……}{bu2lun4 dou1}
    \definition{conj.}{não apenas\dots, (o que, quem, como, etc.), \dots}
  \end{phonetics}
\end{entry}

\begin{entry}{不过}{4,6}
  \begin{phonetics}{不过}{bu2guo4}
    \definition{conj.}{mas | contudo | no entanto}
  \end{phonetics}
\end{entry}

\begin{entry}{不但}{4,7}
  \begin{phonetics}{不但}{bu2dan4}
    \definition{conj.}{não somente}
  \end{phonetics}
\end{entry}

\begin{entry}{不但……而且……}{4,7,6,5}
  \begin{phonetics}{不但……而且……}{bu2dan4 er2qie3}
    \definition{conj.}{não só\dots mas também\dots}
  \end{phonetics}
\end{entry}

\begin{entry}{不到}{4,8}
  \begin{phonetics}{不到}{bu2dao4}
    \definition{adj.}{insuficiente}
    \definition{adv.}{menos que}
    \definition{v.}{não chegar}
  \end{phonetics}
\end{entry}

\begin{entry}{不注意}{4,8,13}
  \begin{phonetics}{不注意}{bu2zhu4yi4}
    \definition{adj.}{impensado | distraído}
    \definition{s.}{descuido | distração}
  \end{phonetics}
\end{entry}

\begin{entry}{不客气}{4,9,4}
  \begin{phonetics}{不客气}{bu2ke4qi5}
    \definition{adj.}{indelicado | rude | brusco}
    \definition{expr.}{de nada | não há de que | não mencione isso}
  \end{phonetics}
\end{entry}

\begin{entry}{不是话}{4,9,8}
  \begin{phonetics}{不是话}{bu2shi4hua4}
    \definition{expr.}{sem razão | demasiado irracionável}
    \seeref{不像话}{bu2xiang4hua4}
    \seeref{不成话}{bu4cheng2hua4}
  \end{phonetics}
\end{entry}

\begin{entry}{不要}{4,9}
  \begin{phonetics}{不要}{bu2yao4}
    \definition{adv.}{nada de (pedir a alguém para não fazer) | não}
  \end{phonetics}
\end{entry}

\begin{entry}{不断}{4,11}
  \begin{phonetics}{不断}{bu2duan4}
    \definition{adv.}{continuamente | sem fim}
  \end{phonetics}
\end{entry}

\begin{entry}{不像话}{4,13,8}
  \begin{phonetics}{不像话}{bu2xiang4hua4}
    \definition{expr.}{sem razão | demasiado irracionável}
    \seeref{不是话}{bu2shi4hua4}
    \seeref{不成话}{bu4cheng2hua4}
  \end{phonetics}
\end{entry}

\begin{entry}{不错}{4,13}
  \begin{phonetics}{不错}{bu2cuo4}
    \definition{adj.}{correto | não (é) mau | bastante bom | certo}
  \end{phonetics}
\end{entry}

\begin{entry}{不管……也……}{4,14,3}
  \begin{phonetics}{不管……也……}{bu4guan3 ye3}
    \definition{conj.}{não apenas\dots, (o que, quem, como, etc.), \dots}
  \end{phonetics}
\end{entry}

\begin{entry}{不管……都……}{4,14,10}
  \begin{phonetics}{不管……都……}{bu4guan3 dou1}
    \definition{conj.}{não apenas\dots, (o que, quem, como, etc.), \dots}
  \end{phonetics}
\end{entry}

\begin{entry}{专业}{4,5}
  \begin{phonetics}{专业}{zhuan1ye4}
    \definition[门,个]{s.}{área de atuação | especialidade}
  \end{phonetics}
\end{entry}

\begin{entry}{专业人士}{4,5,2,3}
  \begin{phonetics}{专业人士}{zhuan1ye4ren2shi4}
    \definition{s.}{profissional}
  \end{phonetics}
\end{entry}

\begin{entry}{专业人才}{4,5,2,3}
  \begin{phonetics}{专业人才}{zhuan1ye4ren2cai2}
    \definition{s.}{especialista (em uma área)}
  \end{phonetics}
\end{entry}

\begin{entry}{专业化}{4,5,4}
  \begin{phonetics}{专业化}{zhuan1ye4hua4}
    \definition{s.}{especialização}
  \end{phonetics}
\end{entry}

\begin{entry}{专业户}{4,5,4}
  \begin{phonetics}{专业户}{zhuan1ye4hu4}
    \definition{s.}{indústria caseira | empresa familiar produzindo um produto especial}
  \end{phonetics}
\end{entry}

\begin{entry}{专业性}{4,5,8}
  \begin{phonetics}{专业性}{zhuan1ye4xing4}
    \definition{s.}{profissionalismo | expertise}
  \end{phonetics}
\end{entry}

\begin{entry}{专业教育}{4,5,11,8}
  \begin{phonetics}{专业教育}{zhuan1ye4jiao4yu4}
    \definition{s.}{educação especializada | escola técnica}
  \end{phonetics}
\end{entry}

\begin{entry}{中午}{4,4}
  \begin{phonetics}{中午}{zhong1wu3}
    \definition[个]{s.}{meio-dia}
  \end{phonetics}
\end{entry}

\begin{entry}{中文}{4,4}
  \begin{phonetics}{中文}{zhong1wen2}
    \definition{s.}{chinês, língua chinesa}
  \end{phonetics}
\end{entry}

\begin{entry}{中东}{4,5}
  \begin{phonetics}{中东}{zhong1dong1}
    \definition*{s.}{Oriente Médio}
  \end{phonetics}
\end{entry}

\begin{entry}{中央情报局}{4,5,11,7,7}
  \begin{phonetics}{中央情报局}{zhong1yang1 qing2bao4ju2}
    \definition*{s.}{Agência Central de Inteligência dos EUA, CIA}
  \end{phonetics}
\end{entry}

\begin{entry}{中间}{4,7}
  \begin{phonetics}{中间}{zhong1jian1}
    \definition{adv.}{central | centro | no meio}
  \end{phonetics}
\end{entry}

\begin{entry}{中国}{4,8}
  \begin{phonetics}{中国}{zhong1guo2}
    \definition*{s.}{China}
  \end{phonetics}
\end{entry}

\begin{entry}{中国人}{4,8,2}
  \begin{phonetics}{中国人}{zhong1guo2ren2}
    \definition{s.}{chinês | pessoa ou povo da China}
  \end{phonetics}
\end{entry}

\begin{entry}{中国城}{4,8,9}
  \begin{phonetics}{中国城}{zhong1guo2cheng2}
    \definition*{s.}{Bairro Chinês, \emph{Chinatown}}
    \seeref{唐人街}{tang2ren2 jie1}
  \end{phonetics}
\end{entry}

\begin{entry}{中国科学院}{4,8,9,8,9}
  \begin{phonetics}{中国科学院}{zhong1guo2 ke1xue2yuan4}
    \definition*{s.}{Academia Chinesa de Ciências}
  \end{phonetics}
\end{entry}

\begin{entry}{中国通}{4,8,10}
  \begin{phonetics}{中国通}{zhong1guo2tong1}
    \definition*{s.}{Conhecedor da China, especialista em tudo sobre a China}
  \end{phonetics}
\end{entry}

\begin{entry}{中学}{4,8}
  \begin{phonetics}{中学}{zhong1xue2}
    \definition[个]{s.}{escola ensino médio}
  \end{phonetics}
\end{entry}

\begin{entry}{中学生}{4,8,5}
  \begin{phonetics}{中学生}{zhong1xue2sheng1}
    \definition{s.}{estudante da escola ensino médio}
  \end{phonetics}
\end{entry}

\begin{entry}{中性}{4,8}
  \begin{phonetics}{中性}{zhong1xing4}
    \definition{adj.}{neutro}
  \end{phonetics}
\end{entry}

\begin{entry}{中询}{4,8}
  \begin{phonetics}{中询}{zhong1 xun2}
    \definition{adv.}{segunda dezena do mês | meio do mês | em meados do mês}
  \end{phonetics}
\end{entry}

\begin{entry}{中秋节}{4,9,5}
  \begin{phonetics}{中秋节}{zhong1qiu1jie2}
    \definition*{s.}{Festival do Meio-Outono | Festival do Bolo Lunar (15º dia do oitavo mês lunar)}
  \end{phonetics}
\end{entry}

\begin{entry}{中药}{4,9}
  \begin{phonetics}{中药}{zhong1yao4}
    \definition[服,种]{s.}{medicina tradicional chinesa}
  \end{phonetics}
\end{entry}

\begin{entry}{中情局}{4,11,7}
  \begin{phonetics}{中情局}{zhong1qing2ju2}
    \definition*{s.}{Agência Central de Inteligência dos EUA, CIA (abreviação de 中央情报局)}
    \seeref{中央情报局}{zhong1yang1 qing2bao4ju2}
  \end{phonetics}
\end{entry}

\begin{entry}{中意}{4,13}
  \begin{phonetics}{中意}{zhong4yi4}
    \definition{s.}{ser do seu agrado | começar a gostar muito de algo ou de alguém}
  \end{phonetics}
\end{entry}

\begin{entry}{丰收}{4,6}
  \begin{phonetics}{丰收}{feng1shou1}
    \definition{s.}{colheita abundante}
  \end{phonetics}
\end{entry}

\begin{entry}{为}{4}[Radical 丶]
  \begin{phonetics}{为}{wei2}
    \definition{prep.}{como (na capacidade de) | por (na voz passiva)}
    \definition{v.}{tomar algo como | agir como | servir como | comportar-se como | tornar-se}
  \end{phonetics}
  \begin{phonetics}{为}{wei4}
    \definition{prep.}{para | porque}
  \end{phonetics}
\end{entry}

\begin{entry}{为什么}{4,4,3}
  \begin{phonetics}{为什么}{wei4shen2me5}
    \definition{adv.}{por que?}
  \end{phonetics}
\end{entry}

\begin{entry}{乌克兰}{4,7,5}
  \begin{phonetics}{乌克兰}{wu1ke4lan2}
    \definition*{s.}{Ucrânia}
  \end{phonetics}
\end{entry}

\begin{entry}{乌龟}{4,7}
  \begin{phonetics}{乌龟}{wu1gui1}
    \definition{s.}{tartaruga}
  \end{phonetics}
\end{entry}

\begin{entry}{书}{4}[Radical 乙]
  \begin{phonetics}{书}{shu1}
    \definition[本,册,部]{s.}{livro | carta | documento}
  \end{phonetics}
\end{entry}

\begin{entry}{书记}{4,5}
  \begin{phonetics}{书记}{shu1ji5}
    \definition{s.}{secretário (chefe de um ramo de um partido socialista ou comunista) | atendente | balconista | escriturário}
  \end{phonetics}
\end{entry}

\begin{entry}{书店}{4,8}
  \begin{phonetics}{书店}{shu1dian4}
    \definition[家]{s.}{livraria}
  \end{phonetics}
\end{entry}

\begin{entry}{云}{4}[Radical 二]
  \begin{phonetics}{云}{yun2}
    \definition*{s.}{sobrenome Yun}
    \definition[朵]{s.}{nuvem}
  \end{phonetics}
\end{entry}

\begin{entry}{云云}{4,4}
  \begin{phonetics}{云云}{yun2yun2}
    \definition{adv.}{e assim por diante | assim e assim}
  \end{phonetics}
\end{entry}

\begin{entry}{云南}{4,9}
  \begin{phonetics}{云南}{yun2nan2}
    \definition*{s.}{Yunnan}
  \end{phonetics}
\end{entry}

\begin{entry}{云端}{4,14}
  \begin{phonetics}{云端}{yun2duan1}
    \definition{s.}{alto nas nuvens | (computação) a nuvem}
  \end{phonetics}
\end{entry}

\begin{entry}{互}{4}[Radical ⼆]
  \begin{phonetics}{互}{hu4}
    \definition{adj.}{mútuo | recíproco}
  \end{phonetics}
\end{entry}

\begin{entry}{互动}{4,6}
  \begin{phonetics}{互动}{hu4dong4}
    \definition{s.}{interativo}
    \definition{v.}{interagir}
  \end{phonetics}
\end{entry}

\begin{entry}{互利}{4,7}
  \begin{phonetics}{互利}{hu4li4}
    \definition{s.}{benefício mútuo}
  \end{phonetics}
\end{entry}

\begin{entry}{互相}{4,9}
  \begin{phonetics}{互相}{hu4xiang1}
    \definition{adv.}{mutuamente | um ao outro}
  \end{phonetics}
\end{entry}

\begin{entry}{五}{4}[Radical 二]
  \begin{phonetics}{五}{wu3}
    \definition{num.}{cinco; 5}
  \end{phonetics}
\end{entry}

\begin{entry}{五五}{4,4}
  \begin{phonetics}{五五}{wu3wu3}
    \definition{num.}{50-50}
    \definition{s.}{igual (partilha, parceria, etc.)}
  \end{phonetics}
\end{entry}

\begin{entry}{五体投地}{4,7,7,6}
  \begin{phonetics}{五体投地}{wu3ti3tou2di4}
    \definition{expr.}{prostrar-se em admiração | adular alguém}
  \end{phonetics}
\end{entry}

\begin{entry}{井}{4}[Radical 二][Kangxi 7]
  \begin{phonetics}{井}{jing3}
    \definition{adj.}{puro | ordenado}
    \definition[口]{s.}{poço}
  \end{phonetics}
\end{entry}

\begin{entry}{什么}{4,3}
  \begin{phonetics}{什么}{shen2me5}
    \definition{pron.}{que? | o que?}
    \definition{pron.}{algo | qualquer coisa}
  \end{phonetics}
\end{entry}

\begin{entry}{什么时候}{4,3,7,10}
  \begin{phonetics}{什么时候}{shen2me5shi2hou5}
    \definition{adv.}{quando? | a que horas?}
  \end{phonetics}
\end{entry}

\begin{entry}{仅}{4}[Radical 人]
  \begin{phonetics}{仅}{jin3}
    \definition{adv.}{apenas | meramente}
  \end{phonetics}
\end{entry}

\begin{entry}{仅仅}{4,4}
  \begin{phonetics}{仅仅}{jin3jin3}
    \definition{adv.}{meramente | somente | apenas}
  \end{phonetics}
\end{entry}

\begin{entry}{今天}{4,4}
  \begin{phonetics}{今天}{jin1tian1}
    \definition{adv.}{hoje | no presente | agora}
  \end{phonetics}
\end{entry}

\begin{entry}{今年}{4,6}
  \begin{phonetics}{今年}{jin1nian2}
    \definition{adv.}{este ano}
  \end{phonetics}
\end{entry}

\begin{entry}{介绍}{4,8}
  \begin{phonetics}{介绍}{jie4shao4}
    \definition{s.}{introdução | apresentação}
    \definition{v.}{fazer uma apresentação | apresentar (alguém para alguém) | apresentar (alguém para um emprego, etc.)}
  \end{phonetics}
\end{entry}

\begin{entry}{仍然}{4,12}
  \begin{phonetics}{仍然}{reng2ran2}
    \definition{adv.}{ainda}
  \end{phonetics}
\end{entry}

\begin{entry}{从}{4}[Radical ⼈]
  \begin{phonetics}{从}{cong2}
    \definition*{s.}{sobrenome Cong}
    \definition{prep.}{de | desde | a partir de}
  \end{phonetics}
\end{entry}

\begin{entry}{从不}{4,4}
  \begin{phonetics}{从不}{cong2bu4}
    \definition{adv.}{nunca}
  \end{phonetics}
\end{entry}

\begin{entry}{从未}{4,5}
  \begin{phonetics}{从未}{cong2wei4}
    \definition{adv.}{nunca}
  \end{phonetics}
\end{entry}

\begin{entry}{从而}{4,6}
  \begin{phonetics}{从而}{cong2'er2}
    \definition{conj.}{assim | desse modo}
  \end{phonetics}
\end{entry}

\begin{entry}{从来}{4,7}
  \begin{phonetics}{从来}{cong2lai2}
    \definition{adv.}{do passado até o presente | o tempo todo | sempre | nunca (se usado em uma sentença negativa)}
  \end{phonetics}
\end{entry}

\begin{entry}{以及}{4,3}
  \begin{phonetics}{以及}{yi3ji2}
    \definition{conj.}{assim como | juntamente como}
  \end{phonetics}
\end{entry}

\begin{entry}{以为}{4,4}
  \begin{phonetics}{以为}{yi3wei2}
    \definition{v.}{pensar, ou seja, considerar que\dots (geralmente há uma implicação de que a noção está errada --- exceto ao expressar a própria opnião atual)}
  \end{phonetics}
\end{entry}

\begin{entry}{以后}{4,6}
  \begin{phonetics}{以后}{yi3hou4}
    \definition{adv.}{depois de | depois | após}
  \end{phonetics}
\end{entry}

\begin{entry}{以此}{4,6}
  \begin{phonetics}{以此}{yi3ci3}
    \definition{adv.}{devido a esta | deste modo | por isso | com isso}
  \end{phonetics}
\end{entry}

\begin{entry}{以至}{4,6}
  \begin{phonetics}{以至}{yi3zhi4}
    \definition{adv.}{até}
    \definition{conj.}{a tal ponto que\dots}
  \seealsoref{以至于}{yi3zhi4yu2}
  \end{phonetics}
\end{entry}

\begin{entry}{以至于}{4,6,3}
  \begin{phonetics}{以至于}{yi3zhi4yu2}
    \definition{adv.}{até}
    \definition{conj.}{na medida em que\dots}
  \seealsoref{以至}{yi3zhi4}
  \end{phonetics}
\end{entry}

\begin{entry}{以色列}{4,6,6}
  \begin{phonetics}{以色列}{yi3se4lie4}
    \definition*{s.}{Israel}
  \end{phonetics}
\end{entry}

\begin{entry}{以免}{4,7}
  \begin{phonetics}{以免}{yi3mian3}
    \definition{conj.}{para evitar isso}
  \end{phonetics}
\end{entry}

\begin{entry}{以来}{4,7}
  \begin{phonetics}{以来}{yi3lai2}
    \definition{prep.}{desde (um evento anterior)}
  \end{phonetics}
\end{entry}

\begin{entry}{以求}{4,7}
  \begin{phonetics}{以求}{yi3qiu2}
    \definition{conj.}{a fim de}
  \end{phonetics}
\end{entry}

\begin{entry}{以便}{4,9}
  \begin{phonetics}{以便}{yi3bian4}
    \definition{conj.}{a fim de | para que | assim como}
  \end{phonetics}
\end{entry}

\begin{entry}{以前}{4,9}
  \begin{phonetics}{以前}{yi3qian2}
    \definition{adv.}{antes de | antes}
  \end{phonetics}
\end{entry}

\begin{entry}{以期}{4,12}
  \begin{phonetics}{以期}{yi3qi1}
    \definition{v.}{tentando | esperando | esperando por}
  \end{phonetics}
\end{entry}

\begin{entry}{元}{4}[Radical 儿]
  \begin{phonetics}{元}{yuan2}
    \definition*{s.}{sobrenome Yuan | Dinastia Yuan (1279-1368)}
    \definition{clas.}{unidade monetária da China}
  \end{phonetics}
\end{entry}

\begin{entry}{元气}{4,4}
  \begin{phonetics}{元气}{yuan2qi4}
    \definition{s.}{força | vigor | vitalidade | energial vital}
  \end{phonetics}
\end{entry}

\begin{entry}{元旦}{4,5}
  \begin{phonetics}{元旦}{yuan2dan4}
    \definition*{s.}{Dia de Ano Novo (1 de janeiro)}
  \end{phonetics}
\end{entry}

\begin{entry}{元来}{4,7}
  \begin{phonetics}{元来}{yuan2lai2}
    \variantof{原来}
  \end{phonetics}
\end{entry}

\begin{entry}{元夜}{4,8}
  \begin{phonetics}{元夜}{yuan2ye4}
    \definition*{s.}{Festival das Lanternas}
  \seealsoref{元宵}{yuan2xiao1}
  \seealsoref{元宵节}{yuan2xiao1jie2}
  \end{phonetics}
\end{entry}

\begin{entry}{元宵}{4,10}
  \begin{phonetics}{元宵}{yuan2xiao1}
    \definition*{s.}{Festival das Lanternas}
  \seealsoref{元宵节}{yuan2xiao1jie2}
  \seealsoref{元夜}{yuan2ye4}
  \end{phonetics}
\end{entry}

\begin{entry}{元宵节}{4,10,5}
  \begin{phonetics}{元宵节}{yuan2xiao1jie2}
    \definition*{s.}{Festival das Lanternas (15º~dia do primeiro mês lunar)}
  \seealsoref{元宵}{yuan2xiao1}
  \seealsoref{元夜}{yuan2ye4}
  \end{phonetics}
\end{entry}

\begin{entry}{公元}{4,4}
  \begin{phonetics}{公元}{gong1yuan2}
    \definition{s.}{D.C. (Depois de~Cristo)}
  \seealsoref{前}{qian2}
    \example{公元293年}[293 d.C.]
  \end{phonetics}
\end{entry}

\begin{entry}{公开}{4,4}
  \begin{phonetics}{公开}{gong1kai1}
    \definition{s.}{aberto | público}
    \definition{v.}{tornar público | liberar}
  \end{phonetics}
\end{entry}

\begin{entry}{公车}{4,4}
  \begin{phonetics}{公车}{gong1che1}
    \definition{s.}{abreviação de~公共汽车, ônibus}
    \seeref{公共汽车}{gong1gong4qi4che1}
  \end{phonetics}
\end{entry}

\begin{entry}{公司}{4,5}
  \begin{phonetics}{公司}{gong1si1}
    \definition[家]{s.}{empresa | companhia | corporação | firma}
  \end{phonetics}
\end{entry}

\begin{entry}{公司治理}{4,5,8,11}
  \begin{phonetics}{公司治理}{gong1si1zhi4li3}
    \definition{s.}{governança corporativa}
  \end{phonetics}
\end{entry}

\begin{entry}{公用电话}{4,5,5,8}
  \begin{phonetics}{公用电话}{gong1yong4dian4hua4}
    \definition[部]{s.}{telefone público}
  \end{phonetics}
\end{entry}

\begin{entry}{公共汽车}{4,6,7,4}
  \begin{phonetics}{公共汽车}{gong1gong4qi4che1}
    \definition[辆,班]{s.}{ônibus}
    \seeref{公车}{gong1che1}
  \end{phonetics}
\end{entry}

\begin{entry}{公克}{4,7}
  \begin{phonetics}{公克}{gong1ke4}
    \definition{s.}{grama (medida de peso)}
  \end{phonetics}
\end{entry}

\begin{entry}{公园}{4,7}
  \begin{phonetics}{公园}{gong1yuan2}
    \definition[座]{s.}{parque (para recreação pública)}
  \end{phonetics}
\end{entry}

\begin{entry}{公里}{4,7}
  \begin{phonetics}{公里}{gong1li3}
    \definition{s.}{quilômetro}
  \end{phonetics}
\end{entry}

\begin{entry}{公寓}{4,12}
  \begin{phonetics}{公寓}{gong1yu4}
    \definition[套]{s.}{prédio de apartamentos | pensão}
  \end{phonetics}
\end{entry}

\begin{entry}{六}{4}[Radical 八]
  \begin{phonetics}{六}{liu4}
    \definition{num.}{seis; 6}
  \end{phonetics}
\end{entry}

\begin{entry}{内存}{4,6}
  \begin{phonetics}{内存}{nei4cun2}
    \definition{s.}{armazenamento interno | memória do computador | RAM (\emph{random access memory})}
  \seealsoref{随机存取存储器}{sui2ji1cun2qu3cun2chu3qi4}
  \seealsoref{随机存取记忆体}{sui2ji1cun2qu3ji4yi4ti3}
  \end{phonetics}
\end{entry}

\begin{entry}{内省}{4,9}
  \begin{phonetics}{内省}{nei4xing3}
    \definition{s.}{introspecção}
    \definition{v.}{refletir sobre si mesmo}
  \end{phonetics}
\end{entry}

\begin{entry}{内燃机}{4,16,6}
  \begin{phonetics}{内燃机}{nei4ran2ji1}
    \definition{s.}{motor de combustão interna}
  \end{phonetics}
\end{entry}

\begin{entry}{凤凰}{4,11}
  \begin{phonetics}{凤凰}{feng4huang2}
    \definition{s.}{fênix}
  \end{phonetics}
\end{entry}

\begin{entry}{分}{4}
  \begin{phonetics}{分}{fen1}
    \definition{s.}{parte ou subdivisão | fração | um décimo (de certas unidades) | unidade de comprimento equivalente a 0,33cm | minuto (unidade de tempo) | minuto (unidade de medida angular) | um ponto (em esportes e jogos) | 0,01 yuan (unidade de dinheiro)}
    \definition{v.}{dividir | separar | distribuir | atribuir | distinguir (bom e mau)}
  \end{phonetics}
  \begin{phonetics}{分}{fen4}
    \definition{s.}{parte | ingrediente | componente}
  \end{phonetics}
\end{entry}

\begin{entry}{分子}{4,3}
  \begin{phonetics}{分子}{fen1zi3}
    \definition{s.}{molécula | (matemática) numerador de uma fração}
  \end{phonetics}
  \begin{phonetics}{分子}{fen4zi3}
    \definition{s.}{membros de uma classe ou grupo | elementos políticos (como intelectuais ou extremistas)}
  \end{phonetics}
\end{entry}

\begin{entry}{分公司}{4,4,5}
  \begin{phonetics}{分公司}{fen1gong1si1}
    \definition{s.}{sucursal | filial de companhia}
  \end{phonetics}
\end{entry}

\begin{entry}{分手}{4,4}
  \begin{phonetics}{分手}{fen1shou3}
    \definition{v.+compl.}{separar | separar-se do companheiro | dizer adeus}
  \end{phonetics}
\end{entry}

\begin{entry}{分钟}{4,9}
  \begin{phonetics}{分钟}{fen1zhong1}
    \definition{s.}{minuto (usado em intervalos de tempo)}
  \end{phonetics}
\end{entry}

\begin{entry}{分量}{4,12}
  \begin{phonetics}{分量}{fen1liang4}
    \definition{s.}{componente vetorial}
  \end{phonetics}
  \begin{phonetics}{分量}{fen4liang4}
    \definition{s.}{tamanho da porção (comida)}
  \end{phonetics}
  \begin{phonetics}{分量}{fen4liang5}
    \definition{s.}{quantidade | peso | medida}
  \end{phonetics}
\end{entry}

\begin{entry}{切割}{4,12}
  \begin{phonetics}{切割}{qie1ge1}
    \definition{v.}{cortar}
  \end{phonetics}
\end{entry}

\begin{entry}{办}{4}[Radical 力]
  \begin{phonetics}{办}{ban4}
    \definition{v.}{lidar com | lidar | gerenciar | configurar}
  \end{phonetics}
\end{entry}

\begin{entry}{办公}{4,4}
  \begin{phonetics}{办公}{ban4gong1}
    \definition{v.+compl.}{lidar com negócios oficiais | trabalhar (especialmente em um escritório)}
  \end{phonetics}
\end{entry}

\begin{entry}{办公室}{4,4,9}
  \begin{phonetics}{办公室}{ban4gong1shi4}
    \definition[间]{s.}{gabinete | escritório}
  \end{phonetics}
\end{entry}

\begin{entry}{办法}{4,8}
  \begin{phonetics}{办法}{ban4fa3}
    \definition[条,个]{s.}{meio (de se fazer alguma coisa) | método | medida}
  \end{phonetics}
\end{entry}

\begin{entry}{勾}{4}[Radical ⼓]
  \begin{phonetics}{勾}{gou1}
    \definition*{s.}{sobrenome Gou}
    \definition{v.}{atrair | excitar | marcar | atacar | delinear | conspirar}
    \variantof{钩}
  \end{phonetics}
  \begin{phonetics}{勾}{gou4}
    \definition{s.}{usado em 勾当}
    \seeref{勾当}{gou4dang4}
  \end{phonetics}
\end{entry}

\begin{entry}{勾当}{4,6}
  \begin{phonetics}{勾当}{gou4dang4}
    \definition{s.}{negócio obscuro}
  \end{phonetics}
\end{entry}

\begin{entry}{化}{4}[Radical 匕]
  \begin{phonetics}{化}{hua1}
    \variantof{花}
  \end{phonetics}
\end{entry}

\begin{entry}{化学}{4,8}
  \begin{phonetics}{化学}{hua4xue2}
    \definition{s.}{química (disciplina)}
  \end{phonetics}
\end{entry}

\begin{entry}{区}{4}[Radical 匸]
  \begin{phonetics}{区}{ou1}
    \definition*{s.}{sobrenome Ou}
  \end{phonetics}
  \begin{phonetics}{区}{qu1}
    \definition[个]{s.}{área | região | distrito}
  \end{phonetics}
\end{entry}

\begin{entry}{区域}{4,11}
  \begin{phonetics}{区域}{qu1yu4}
    \definition{s.}{área | região | distrito}
  \end{phonetics}
\end{entry}

\begin{entry}{升起}{4,10}
  \begin{phonetics}{升起}{sheng1qi3}
    \definition{v.}{levantar | içar | subir}
  \end{phonetics}
\end{entry}

\begin{entry}{午}{4}[Radical 十]
  \begin{phonetics}{午}{wu3}
    \definition{s.}{período entre 11h00 e 13h00, meio-dia}
  \end{phonetics}
\end{entry}

\begin{entry}{午休}{4,6}
  \begin{phonetics}{午休}{wu3xiu1}
    \definition{s.}{pausa para almoço | cochilo na hora do almoço | intervalo do meio-dia}
  \end{phonetics}
\end{entry}

\begin{entry}{午后}{4,6}
  \begin{phonetics}{午后}{wu3hou4}
    \definition{s.}{tarde | período da tarde}
  \end{phonetics}
\end{entry}

\begin{entry}{午饭}{4,7}
  \begin{phonetics}{午饭}{wu3fan4}
    \definition[份,顿,次,餐]{s.}{almoço}
  \seealsoref{午餐}{wu3can1}
  \end{phonetics}
\end{entry}

\begin{entry}{午夜}{4,8}
  \begin{phonetics}{午夜}{wu3ye4}
    \definition{s.}{meia-noite}
  \end{phonetics}
\end{entry}

\begin{entry}{午前}{4,9}
  \begin{phonetics}{午前}{wu3qian2}
    \definition{s.}{\emph{A.M.} | manhã | período da manhã}
  \end{phonetics}
\end{entry}

\begin{entry}{午宴}{4,10}
  \begin{phonetics}{午宴}{wu3yan4}
    \definition{s.}{banquete de almoço}
  \end{phonetics}
\end{entry}

\begin{entry}{午睡}{4,13}
  \begin{phonetics}{午睡}{wu3shui4}
    \definition{s.}{siesta}
    \definition{v.}{tirar uma soneca}
  \end{phonetics}
\end{entry}

\begin{entry}{午餐}{4,16}
  \begin{phonetics}{午餐}{wu3can1}
    \definition[份,顿,次]{s.}{almoço}
  \seealsoref{午饭}{wu3fan4}
  \end{phonetics}
\end{entry}

\begin{entry}{历史}{4,5}
  \begin{phonetics}{历史}{li4shi3}
    \definition[门,段]{s.}{história}
  \end{phonetics}
\end{entry}

\begin{entry}{友好}{4,6}
  \begin{phonetics}{友好}{you3hao3}
    \definition{adj.}{amigável}
    \definition{s.}{amigo próximo, íntimo}
  \end{phonetics}
\end{entry}

\begin{entry}{双}{4}[Radical 又]
  \begin{phonetics}{双}{shuang1}
    \definition*{s.}{sobrenome Shuang}
    \definition{s.}{dobro | par | dupla | ambos | número par}
  \end{phonetics}
\end{entry}

\begin{entry}{双方同意}{4,4,6,13}
  \begin{phonetics}{双方同意}{shuang1fang1tong2yi4}
    \definition{s.}{acordo bilateral}
  \end{phonetics}
\end{entry}

\begin{entry}{双打}{4,5}
  \begin{phonetics}{双打}{shuang1da3}
    \definition[场]{s.}{duplas (em esportes)}
  \end{phonetics}
\end{entry}

\begin{entry}{双层床}{4,7,7}
  \begin{phonetics}{双层床}{shuang1ceng2chuang2}
    \definition{s.}{beliche}
  \end{phonetics}
\end{entry}

\begin{entry}{反对}{4,5}
  \begin{phonetics}{反对}{fan3dui4}
    \definition{v.}{contrariar | opor-se | lutar contra}
  \end{phonetics}
\end{entry}

\begin{entry}{反对派}{4,5,9}
  \begin{phonetics}{反对派}{fan3dui4pai4}
    \definition{s.}{facção de oposição}
  \end{phonetics}
\end{entry}

\begin{entry}{反对党}{4,5,10}
  \begin{phonetics}{反对党}{fan3dui4dang3}
    \definition{s.}{partido de oposição}
  \end{phonetics}
\end{entry}

\begin{entry}{反对票}{4,5,11}
  \begin{phonetics}{反对票}{fan3dui4piao4}
    \definition{s.}{voto dissidente}
  \end{phonetics}
\end{entry}

\begin{entry}{反正}{4,5}
  \begin{phonetics}{反正}{fan3zheng4}
    \definition{adv.}{de qualquer maneira | em qualquer caso | aconteça o que acontecer}
  \end{phonetics}
\end{entry}

\begin{entry}{反应}{4,7}
  \begin{phonetics}{反应}{fan3ying4}
    \definition[个]{s.}{reação | resposta | reação química}
    \definition{v.}{reagir | responder}
  \end{phonetics}
\end{entry}

\begin{entry}{反复}{4,9}
  \begin{phonetics}{反复}{fan3fu4}
    \definition{adv.}{de novo e de novo | repetidamente}
  \end{phonetics}
\end{entry}

\begin{entry}{反省}{4,9}
  \begin{phonetics}{反省}{fan3xing3}
    \definition{v.}{examinar a consciência | questionar-se | refletir sobre si mesmo | sondar a alma}
  \end{phonetics}
\end{entry}

\begin{entry}{天}{4}[Radical 大]
  \begin{phonetics}{天}{tian1}
    \definition{s.}{dia | céu | paraíso}
  \end{phonetics}
\end{entry}

\begin{entry}{天下}{4,3}
  \begin{phonetics}{天下}{tian1xia4}
    \definition{s.}{terra sob o céu | o mundo todo | toda a China | reino}
  \end{phonetics}
\end{entry}

\begin{entry}{天才}{4,3}
  \begin{phonetics}{天才}{tian1cai2}
    \definition{adj.}{talentoso | superdotado | genial}
    \definition{s.}{talento | dom | gênio}
  \end{phonetics}
\end{entry}

\begin{entry}{天公}{4,4}
  \begin{phonetics}{天公}{tian1gong1}
    \definition{s.}{céu, paraíso | senhor do céu}
  \end{phonetics}
\end{entry}

\begin{entry}{天天}{4,4}
  \begin{phonetics}{天天}{tian1tian1}
    \definition{adv.}{todo dia}
  \end{phonetics}
\end{entry}

\begin{entry}{天气}{4,4}
  \begin{phonetics}{天气}{tian1qi4}
    \definition{s.}{clima, tempo}
  \end{phonetics}
\end{entry}

\begin{entry}{天花板}{4,7,8}
  \begin{phonetics}{天花板}{tian1hua1ban3}
    \definition{s.}{teto}
  \end{phonetics}
\end{entry}

\begin{entry}{天使}{4,8}
  \begin{phonetics}{天使}{tian1shi3}
    \definition{s.}{anjo}
  \end{phonetics}
\end{entry}

\begin{entry}{天择}{4,8}
  \begin{phonetics}{天择}{tian1ze2}
    \definition{s.}{seleção natural}
  \end{phonetics}
\end{entry}

\begin{entry}{天柱}{4,9}
  \begin{phonetics}{天柱}{tian1zhu4}
    \definition{s.}{pilar celestial, que sustenta o céu}
  \end{phonetics}
\end{entry}

\begin{entry}{天堂}{4,11}
  \begin{phonetics}{天堂}{tian1tang2}
    \definition{s.}{paraíso, céu}
  \end{phonetics}
\end{entry}

\begin{entry}{天然}{4,12}
  \begin{phonetics}{天然}{tian1ran2}
    \definition{adj.}{natural}
  \end{phonetics}
\end{entry}

\begin{entry}{天鹅}{4,12}
  \begin{phonetics}{天鹅}{tian1'e2}
    \definition{s.}{cisne}
  \end{phonetics}
\end{entry}

\begin{entry}{太}{4}[Radical 大]
  \begin{phonetics}{太}{tai4}
    \definition{adv.}{excessivamente | demais | muito}
  \end{phonetics}
\end{entry}

\begin{entry}{太太}{4,4}
  \begin{phonetics}{太太}{tai4tai5}
    \definition[个,位]{s.}{esposa | madame| mulher casada}
  \end{phonetics}
\end{entry}

\begin{entry}{太平洋}{4,5,9}
  \begin{phonetics}{太平洋}{tai4ping2 yang2}
    \definition*{s.}{Oceano Pacífico}
  \end{phonetics}
\end{entry}

\begin{entry}{太阳}{4,6}
  \begin{phonetics}{太阳}{tai4yang5}
    \definition[个]{s.}{sol | abreviação de 太阳穴}
    \seeref{太阳穴}{tai4yang2xue2}
  \end{phonetics}
\end{entry}

\begin{entry}{太阳日}{4,6,4}
  \begin{phonetics}{太阳日}{tai4yang2ri4}
    \definition{s.}{dia solar}
  \end{phonetics}
\end{entry}

\begin{entry}{太阳风}{4,6,4}
  \begin{phonetics}{太阳风}{tai4yang2feng1}
    \definition{s.}{vento solar}
  \end{phonetics}
\end{entry}

\begin{entry}{太阳穴}{4,6,5}
  \begin{phonetics}{太阳穴}{tai4yang2xue2}
    \definition{s.}{têmpora (nas laterais da cabeça humana)}
  \end{phonetics}
\end{entry}

\begin{entry}{太阳灯}{4,6,6}
  \begin{phonetics}{太阳灯}{tai4yang2deng1}
    \definition{s.}{lâmpada solar (com células fotovoltaicas)}
  \end{phonetics}
\end{entry}

\begin{entry}{太阳雨}{4,6,8}
  \begin{phonetics}{太阳雨}{tai4yang2yu3}
    \definition{s.}{banho de sol}
  \end{phonetics}
\end{entry}

\begin{entry}{太阳窗}{4,6,12}
  \begin{phonetics}{太阳窗}{tai4yang2chuang1}
    \definition{s.}{teto solar (de veículos)}
  \end{phonetics}
\end{entry}

\begin{entry}{太阳镜}{4,6,16}
  \begin{phonetics}{太阳镜}{tai4yang2jing4}
    \definition{s.}{óculos de sol}
  \end{phonetics}
\end{entry}

\begin{entry}{太阳翼}{4,6,17}
  \begin{phonetics}{太阳翼}{tai4yang2yi4}
    \definition{s.}{painel solar}
  \end{phonetics}
\end{entry}

\begin{entry}{太极拳}{4,7,10}
  \begin{phonetics}{太极拳}{tai4ji2quan2}
    \definition*{s.}{Tai Chi Chuan, Taiji, T'aichi ou T'aichichuan; forma tradicional de exercício físico ou relaxamento}
  \end{phonetics}
\end{entry}

\begin{entry}{太空}{4,8}
  \begin{phonetics}{太空}{tai4kong1}
    \definition{s.}{espaço sideral | espaço exterior}
  \end{phonetics}
\end{entry}

\begin{entry}{夫妻}{4,8}
  \begin{phonetics}{夫妻}{fu1qi1}
    \definition{s.}{casal | marido e eposa}
  \end{phonetics}
\end{entry}

\begin{entry}{孔}{4}[Radical 子]
  \begin{phonetics}{孔}{kong3}
    \definition*{s.}{sobrenome Kong}
    \definition{clas.}{para habitações em cavernas}
    \definition[个]{s.}{buraco}
  \end{phonetics}
\end{entry}

\begin{entry}{孔子}{4,3}
  \begin{phonetics}{孔子}{kong3zi3}
    \definition*{s.}{Confúcio (551-479 aC), pensador e filósofo social chinês}
  \seealsoref{孔夫子}{kong3fu1zi3}
  \end{phonetics}
\end{entry}

\begin{entry}{孔子学院}{4,3,8,9}
  \begin{phonetics}{孔子学院}{kong3zi3 xue2yuan4}
    \definition*{s.}{Instituto Confúcio, organização estabelecida internacionalmente pela República Popular da China, que promove a língua e a cultura chinesas}
  \end{phonetics}
\end{entry}

\begin{entry}{孔夫子}{4,4,3}
  \begin{phonetics}{孔夫子}{kong3fu1zi3}
    \definition*{s.}{Confúcio (551-479 aC), pensador e filósofo social chinês}
  \seealsoref{孔子}{kong3zi3}
  \end{phonetics}
\end{entry}

\begin{entry}{孔雀}{4,11}
  \begin{phonetics}{孔雀}{kong3que4}
    \definition{s.}{pavão}
  \end{phonetics}
\end{entry}

\begin{entry}{少}{4}[Radical 小]
  \begin{phonetics}{少}{shao3}
    \definition{adj.}{pouco, poucos}
    \definition{v.}{sentir falta | faltar | parar (de fazer algo)}
  \end{phonetics}
  \begin{phonetics}{少}{shao4}
    \definition{s.}{jovem}
  \end{phonetics}
\end{entry}

\begin{entry}{尤其}{4,8}
  \begin{phonetics}{尤其}{you2qi2}
    \definition{adv.}{especialmente | particularmente}
  \end{phonetics}
\end{entry}

\begin{entry}{巴西}{4,6}
  \begin{phonetics}{巴西}{ba1xi1}
    \definition*{s.}{Brasil}
  \end{phonetics}
\end{entry}

\begin{entry}{巴西人}{4,6,2}
  \begin{phonetics}{巴西人}{ba1xi1ren2}
    \definition[个,位]{s.}{brasileiro | pessoa ou povo do Brasil}
    \example{他是巴西人。}[Ele é brasileiro.]
  \end{phonetics}
\end{entry}

\begin{entry}{巴西战舞}{4,6,9,14}
  \begin{phonetics}{巴西战舞}{ba1xi1zhan4wu3}
    \definition{s.}{capoeira}
  \end{phonetics}
\end{entry}

\begin{entry}{巴勒斯坦}{4,11,12,8}
  \begin{phonetics}{巴勒斯坦}{ba1le4si1tan3}
    \definition*{s.}{Palestina}
  \end{phonetics}
\end{entry}

\begin{entry}{幻觉}{4,9}
  \begin{phonetics}{幻觉}{huan4jue2}
    \definition{s.}{ilusão | alucinação}
  \end{phonetics}
\end{entry}

\begin{entry}{开}{4}[Radical 廾]
  \begin{phonetics}{开}{kai1}
    \definition{clas.}{quilate (ouro)}
    \definition{v.}{abrir | ligar | dirigir | iniciar (alguma coisa) | começar | ferver | escrever  (uma receita, cheque, fatura, etc.) | operar (um veículo) | abreviação de Kelvin 开尔文}
    \seeref{开尔文}{kai1'er3wen2}
  \end{phonetics}
\end{entry}

\begin{entry}{开口}{4,3}
  \begin{phonetics}{开口}{kai1kou3}
    \definition{v.}{abrir a boca de alguém | começar a falar}
  \end{phonetics}
\end{entry}

\begin{entry}{开心}{4,4}
  \begin{phonetics}{开心}{kai1xin1}
    \definition{v.}{sentir-se feliz | regozijar-se | divertir-se | tirar sarro de alguém}
  \end{phonetics}
\end{entry}

\begin{entry}{开车}{4,4}
  \begin{phonetics}{开车}{kai1che1}
    \definition{v.+compl.}{conduzir | dirigir}
  \end{phonetics}
\end{entry}

\begin{entry}{开发区}{4,5,4}
  \begin{phonetics}{开发区}{kai1fa1qu1}
    \definition{s.}{zona de desenvolvimento}
  \end{phonetics}
\end{entry}

\begin{entry}{开头}{4,5}
  \begin{phonetics}{开头}{kai1tou2}
    \definition{s.}{início | começo}
    \definition{v.+compl.}{iniciar | começar | fazer um começo}
  \end{phonetics}
\end{entry}

\begin{entry}{开尔文}{4,5,4}
  \begin{phonetics}{开尔文}{kai1'er3wen2}
    \definition{s.}{Kelvin, temperatura absoluta | K, escala de temperatura}
  \end{phonetics}
\end{entry}

\begin{entry}{开会}{4,6}
  \begin{phonetics}{开会}{kai1hui4}
    \definition{v.+compl.}{realizar uma reunião | ter uma reunião | participar de uma reunião (conferência)}
  \end{phonetics}
\end{entry}

\begin{entry}{开启}{4,7}
  \begin{phonetics}{开启}{kai1qi3}
    \definition{v.}{abrir | iniciar | (computação) ativar}
  \end{phonetics}
\end{entry}

\begin{entry}{开花}{4,7}
  \begin{phonetics}{开花}{kai1hua1}
    \definition{v.}{florescer | (fig.) explodir, abrir-se | (fig.) explodir de alegria | (fig.) começar a existir de repente em todos os lugares}
  \end{phonetics}
\end{entry}

\begin{entry}{开夜车}{4,8,4}
  \begin{phonetics}{开夜车}{kai1ye4che1}
    \definition{expr.}{trabalho noturno | (literalmente) ``conduzir carro à noite''}
  \end{phonetics}
\end{entry}

\begin{entry}{开始}{4,8}
  \begin{phonetics}{开始}{kai1shi3}
    \definition{adv.}{inicial}
    \definition[个]{s.}{começo | início}
    \definition{v.}{começar | iniciar}
  \end{phonetics}
\end{entry}

\begin{entry}{开锁}{4,12}
  \begin{phonetics}{开锁}{kai1suo3}
    \definition{v.}{desbloquear | destravar}
  \end{phonetics}
\end{entry}

\begin{entry}{引擎}{4,16}
  \begin{phonetics}{引擎}{yin3qing2}
    \definition[台]{s.}{motor | (empréstimo linguístico) \emph{engine}}
  \end{phonetics}
\end{entry}

\begin{entry}{心中}{4,4}
  \begin{phonetics}{心中}{xin1zhong1}
    \definition{adv.}{nos pensamentos | no coração}
    \definition{s.}{ponto central}
  \end{phonetics}
\end{entry}

\begin{entry}{心机}{4,6}
  \begin{phonetics}{心机}{xin1ji1}
    \definition{s.}{pensamento | esquema}
  \end{phonetics}
\end{entry}

\begin{entry}{心声}{4,7}
  \begin{phonetics}{心声}{xin1sheng1}
    \definition{s.}{desejo sincero | voz interior | aspiração}
  \end{phonetics}
\end{entry}

\begin{entry}{心疼}{4,10}
  \begin{phonetics}{心疼}{xin1teng2}
    \definition{adj.}{angustiado}
    \definition{v.}{sentir pena de alguém | arrepender-se | ressentir-se | ficar angustiado}
  \end{phonetics}
\end{entry}

\begin{entry}{手}{4}[Radical 手][Kangxi 64]
  \begin{phonetics}{手}{shou3}
    \definition{adj.}{conveniente}
    \definition{clas.}{de habilidade}
    \definition[双,只]{s.}{mão | pessoa envolvida em certos tipos de trabalho | pessoa qualificada para certos tipos de trabalho}
    \definition{v.}{segurar (formal)}
  \end{phonetics}
\end{entry}

\begin{entry}{手工}{4,3}
  \begin{phonetics}{手工}{shou3gong1}
    \definition{s.}{trabalho manual | artesanato}
  \end{phonetics}
\end{entry}

\begin{entry}{手工艺人}{4,3,4,2}
  \begin{phonetics}{手工艺人}{shou3gong1 yi4ren2}
    \definition{s.}{artesão}
  \end{phonetics}
\end{entry}

\begin{entry}{手边}{4,5}
  \begin{phonetics}{手边}{shou3bian1}
    \definition{adv.}{à mão | na mão}
  \end{phonetics}
\end{entry}

\begin{entry}{手机}{4,6}
  \begin{phonetics}{手机}{shou3ji1}
    \definition[部,支]{s.}{telefone celular ou móvel}
  \end{phonetics}
\end{entry}

\begin{entry}{手刹}{4,8}
  \begin{phonetics}{手刹}{shou3sha1}
    \definition{s.}{freio de mão}
  \end{phonetics}
\end{entry}

\begin{entry}{手指}{4,9}
  \begin{phonetics}{手指}{shou3zhi3}
    \definition[个,只]{s.}{dedo}
  \end{phonetics}
\end{entry}

\begin{entry}{手臂}{4,17}
  \begin{phonetics}{手臂}{shou3bi4}
    \definition{s.}{braço}
  \end{phonetics}
\end{entry}

\begin{entry}{支}{4}[Radical 支][Kangxi 65]
  \begin{phonetics}{支}{zhi1}
    \definition*{s.}{sobrenome Zhi}
    \definition{clas.}{para varetas como canetas e armas | para divisões do exército e para canções ou composições}
    \definition{v.}{sacar dinheiro | erguer | criar | suportar | sustentar}
  \end{phonetics}
\end{entry}

\begin{entry}{支支吾吾}{4,4,7,7}
  \begin{phonetics}{支支吾吾}{zhi1zhi1wu2wu2}
    \definition{v.}{falhar | murmurar | paralisar | gaguejar}
  \end{phonetics}
\end{entry}

\begin{entry}{支应}{4,7}
  \begin{phonetics}{支应}{zhi1ying4}
    \definition{v.}{lidar com | fornecer}
  \end{phonetics}
\end{entry}

\begin{entry}{支承}{4,8}
  \begin{phonetics}{支承}{zhi1cheng2}
    \definition{v.}{suportar o peso de (um edifício) | suportar}
  \end{phonetics}
\end{entry}

\begin{entry}{支持}{4,9}
  \begin{phonetics}{支持}{zhi1chi2}
    \definition[个]{s.}{apoio | suporte}
    \definition{v.}{apoiar | ser a favor de | suportar}
  \end{phonetics}
\end{entry}

\begin{entry}{支根}{4,10}
  \begin{phonetics}{支根}{zhi1gen1}
    \definition{s.}{raiz ramificada | raízes de apoio | radícula}
  \end{phonetics}
\end{entry}

\begin{entry}{支票}{4,11}
  \begin{phonetics}{支票}{zhi1piao4}
    \definition[本]{s.}{cheque (banco)}
  \end{phonetics}
\end{entry}

\begin{entry}{文化}{4,4}
  \begin{phonetics}{文化}{wen2hua4}
    \definition[个,种]{s.}{cultura | civilização}
  \end{phonetics}
\end{entry}

\begin{entry}{文化水平}{4,4,4,5}
  \begin{phonetics}{文化水平}{wen2hua4 shui3ping2}
    \definition{s.}{nível educacional}
  \end{phonetics}
\end{entry}

\begin{entry}{文化史}{4,4,5}
  \begin{phonetics}{文化史}{wen2hua4shi3}
    \definition*{s.}{História Cultural}
  \end{phonetics}
\end{entry}

\begin{entry}{文化层}{4,4,7}
  \begin{phonetics}{文化层}{wen2hua4ceng2}
    \definition{s.}{nível de cultura (em sítio arqueológico)}
  \end{phonetics}
\end{entry}

\begin{entry}{文化宫}{4,4,9}
  \begin{phonetics}{文化宫}{wen2hua4gong1}
    \definition{s.}{palácio cultural}
  \end{phonetics}
\end{entry}

\begin{entry}{文化热}{4,4,10}
  \begin{phonetics}{文化热}{wen2hua4re4}
    \definition{s.}{mania cultural | febre cultural}
  \end{phonetics}
\end{entry}

\begin{entry}{文化圈}{4,4,11}
  \begin{phonetics}{文化圈}{wen2hua4quan1}
    \definition{s.}{esfera de influência cultural}
  \end{phonetics}
\end{entry}

\begin{entry}{文化障碍}{4,4,13,13}
  \begin{phonetics}{文化障碍}{wen2hua4zhang4'ai4}
    \definition{s.}{barreira cultural}
  \end{phonetics}
\end{entry}

\begin{entry}{文学系}{4,8,7}
  \begin{phonetics}{文学系}{wen2xue2 xi4}
    \definition*{s.}{Faculdade de Letras}
  \end{phonetics}
\end{entry}

\begin{entry}{文明}{4,8}
  \begin{phonetics}{文明}{wen2ming2}
    \definition{adj.}{civilizado}
    \definition[个]{s.}{civilização | cultura}
  \end{phonetics}
\end{entry}

\begin{entry}{方言}{4,7}
  \begin{phonetics}{方言}{fang1yan2}
    \definition*{s.}{o primeiro dicionário de dialeto chinês, editado por Yang Xiong 扬雄 no século I, contendo mais de 9.000 caracteres}
    \definition{s.}{dialeto}
  \seealsoref{扬雄}{yang2xiong2}
  \end{phonetics}
\end{entry}

\begin{entry}{方法}{4,8}
  \begin{phonetics}{方法}{fang1fa3}
    \definition[个]{s.}{método | meio}
  \end{phonetics}
\end{entry}

\begin{entry}{方便}{4,9}
  \begin{phonetics}{方便}{fang1bian4}
    \definition{adj.}{conveniente | adequado}
    \definition{v.}{facilitar, facilitar as coisas | ter dinheiro de sobra | (eufemismo) aliviar-se}
  \end{phonetics}
\end{entry}

\begin{entry}{方案}{4,10}
  \begin{phonetics}{方案}{fang1'an4}
    \definition[个,套]{s.}{plano | programa (para uma ação, etc.) | proposta | proposta de projeto de lei}
  \end{phonetics}
\end{entry}

\begin{entry}{无}{4}[Radical 无][Kangxi 71]
  \begin{phonetics}{无}{wu2}
    \definition{adv.}{não ter algo | não há\dots}
  \end{phonetics}
\end{entry}

\begin{entry}{无人}{4,2}
  \begin{phonetics}{无人}{wu2ren2}
    \definition{adj.}{não tripulado | desabitado}
  \end{phonetics}
\end{entry}

\begin{entry}{无人机}{4,2,6}
  \begin{phonetics}{无人机}{wu2ren2ji1}
    \definition{s.}{\emph{drone} | veículo aéreo não tripulado}
  \end{phonetics}
\end{entry}

\begin{entry}{无论……也……}{4,6,3}
  \begin{phonetics}{无论……也……}{wu2lun4 ye3}
    \definition{conj.}{não apenas\dots, (o que, quem, como, etc.), \dots}
  \end{phonetics}
\end{entry}

\begin{entry}{无视}{4,8}
  \begin{phonetics}{无视}{wu2shi4}
    \definition{v.}{ignorar | desconsiderar}
  \end{phonetics}
\end{entry}

\begin{entry}{无故}{4,9}
  \begin{phonetics}{无故}{wu2gu4}
    \definition{adv.}{sem causa ou razão | sem motivo}
  \end{phonetics}
\end{entry}

\begin{entry}{无骨}{4,9}
  \begin{phonetics}{无骨}{wu2 gu3}
    \definition{adj.}{desossado}
  \end{phonetics}
\end{entry}

\begin{entry}{无敌}{4,10}
  \begin{phonetics}{无敌}{wu2di2}
    \definition{adj.}{invencível | inigualável}
  \end{phonetics}
\end{entry}

\begin{entry}{无氧}{4,10}
  \begin{phonetics}{无氧}{wu2yang3}
    \definition{adj.}{anaeróbico}
  \end{phonetics}
\end{entry}

\begin{entry}{日}{4}[Radical 日][Kangxi 72]
  \begin{phonetics}{日}{ri4}
    \definition*{s.}{Japão, abreviação de~日本}
    \definition{clas.}{dia (mais usado em escrita) | data, dia do mês}
    \seeref{日本}{ri4ben3}
  \end{phonetics}
\end{entry}

\begin{entry}{日子}{4,3}
  \begin{phonetics}{日子}{ri4zi5}
    \definition{s.}{dia | uma data (calendário) | dias de vida de alguém}
  \end{phonetics}
\end{entry}

\begin{entry}{日出}{4,5}
  \begin{phonetics}{日出}{ri4chu1}
    \definition{s.}{nascer do sol}
  \seealsoref{夕阳}{xi1yang2}
  \end{phonetics}
\end{entry}

\begin{entry}{日本}{4,5}
  \begin{phonetics}{日本}{ri4ben3}
    \definition*{s.}{Japão}
  \end{phonetics}
\end{entry}

\begin{entry}{日本人}{4,5,2}
  \begin{phonetics}{日本人}{ri4ben3ren2}
    \definition{s.}{japonês | pessoa ou povo do Japão}
  \end{phonetics}
\end{entry}

\begin{entry}{日光灯}{4,6,6}
  \begin{phonetics}{日光灯}{ri4guang1deng1}
    \definition{s.}{lâmpada fluorescente}
  \end{phonetics}
\end{entry}

\begin{entry}{日常}{4,11}
  \begin{phonetics}{日常}{ri4chang2}
    \definition{adv.}{diariamente | dia-a-dia | todo dia}
  \end{phonetics}
\end{entry}

\begin{entry}{月}{4}[Radical 月][Kangxi 74]
  \begin{phonetics}{月}{yue4}
    \definition[个,轮]{s.}{mês}
  \end{phonetics}
\end{entry}

\begin{entry}{月月}{4,4}
  \begin{phonetics}{月月}{yue4yue4}
    \definition{adv.}{todo mês}
  \end{phonetics}
\end{entry}

\begin{entry}{月径}{4,8}
  \begin{phonetics}{月径}{yue4jing4}
    \definition{s.}{diâmetro da lua | diâmetro da órbita da lua | caminho iluminado pela lua}
  \end{phonetics}
\end{entry}

\begin{entry}{月亮}{4,9}
  \begin{phonetics}{月亮}{yue4liang5}
    \definition{s.}{lua}
  \end{phonetics}
\end{entry}

\begin{entry}{月相}{4,9}
  \begin{phonetics}{月相}{yue4xiang4}
    \definition{s.}{fases da lua, a saber: lua nova 朔, lua crescente 上弦, lua cheia 望 e lua minguante 下弦}
  \end{phonetics}
\end{entry}

\begin{entry}{月饼}{4,9}
  \begin{phonetics}{月饼}{yue4bing3}
    \definition[张]{s.}{bolo da lua}
  \end{phonetics}
\end{entry}

\begin{entry}{月球}{4,11}
  \begin{phonetics}{月球}{yue4qiu2}
    \definition{s.}{a lua}
  \end{phonetics}
\end{entry}

\begin{entry}{月壤}{4,20}
  \begin{phonetics}{月壤}{yue4rang3}
    \definition{s.}{solo lunar}
  \end{phonetics}
\end{entry}

\begin{entry}{木头}{4,5}
  \begin{phonetics}{木头}{mu4tou5}
    \definition{adj.}{estúpido | cabeça-dura}
    \definition[块,根]{s.}{tronco (de madeira)}
  \end{phonetics}
\end{entry}

\begin{entry}{木偶}{4,11}
  \begin{phonetics}{木偶}{mu4'ou3}
    \definition{s.}{fantoche, marionete}
  \end{phonetics}
\end{entry}

\begin{entry}{歹徒}{4,10}
  \begin{phonetics}{歹徒}{dai3tu2}
    \definition{s.}{malfeitor | gangster | bandido}
  \end{phonetics}
\end{entry}

\begin{entry}{比}{4}[Radical 匕][Kangxi 81]
  \begin{phonetics}{比}{bi3}
    \definition*{s.}{Bélgica, abreviação de 比利时}
    \definition{part.}{partícula usada para comparação (superioridade)}
    \definition{prep.}{que | do que | (seguido por um substantivo e adjetivo) mais \{adj.\} do que \{s.\}}
    \definition{s.}{razão (taxa)}
    \definition{v.}{comparar | contrastar | gesticular (com as mãos)}
    \seeref{比利时}{bi3li4shi2}
  \end{phonetics}
\end{entry}

\begin{entry}{比亚迪}{4,6,8}
  \begin{phonetics}{比亚迪}{bi3ya4di2}
    \definition*{s.}{Montadora BYD}
  \end{phonetics}
\end{entry}

\begin{entry}{比如}{4,6}
  \begin{phonetics}{比如}{bi3ru2}
    \definition{conj.}{por exemplo | como}
  \end{phonetics}
\end{entry}

\begin{entry}{比利时}{4,7,7}
  \begin{phonetics}{比利时}{bi3li4shi2}
    \definition*{s.}{Bélgica}
  \end{phonetics}
\end{entry}

\begin{entry}{比拼}{4,9}
  \begin{phonetics}{比拼}{bi3pin1}
    \definition{s.}{concurso}
    \definition{v.}{competir ferozmente}
  \end{phonetics}
\end{entry}

\begin{entry}{比较}{4,10}
  \begin{phonetics}{比较}{bi3jiao4}
    \definition{adv.}{comparativamente | relativamente}
    \definition{s.}{comparação}
    \definition{v.}{comparar}
  \end{phonetics}
\end{entry}

\begin{entry}{比萨饼}{4,11,9}
  \begin{phonetics}{比萨饼}{bi3sa4bing3}
    \definition[张]{s.}{pizza}
  \end{phonetics}
\end{entry}

\begin{entry}{比赛}{4,14}
  \begin{phonetics}{比赛}{bi3sai4}
    \definition[场,次]{s.}{competição | concurso}
    \definition{v.}{competir}
  \end{phonetics}
\end{entry}

\begin{entry}{毛}{4}[Radical 毛][Kangxi 82]
  \begin{phonetics}{毛}{mao2}
    \definition*{s.}{sobrenome Mao}
    \definition{clas.}{1 mao corresponde a 10 centavos}
  \end{phonetics}
\end{entry}

\begin{entry}{气质}{4,8}
  \begin{phonetics}{气质}{qi4zhi4}
    \definition{s.}{traços de personalidade, temperamento, disposição | aura, ar, sentimento, \emph{vibe} | refinamento, sofisticação, classe}
  \end{phonetics}
\end{entry}

\begin{entry}{气球}{4,11}
  \begin{phonetics}{气球}{qi4qiu2}
    \definition{s.}{balão}
  \end{phonetics}
\end{entry}

\begin{entry}{气温}{4,12}
  \begin{phonetics}{气温}{qi4wen1}
    \definition[个]{s.}{temperatura do ar}
  \end{phonetics}
\end{entry}

\begin{entry}{水}{4}[Radical 水][Kangxi 85]
  \begin{phonetics}{水}{shui3}
    \definition*{s.}{sobrenome Shui}
    \definition{clas.}{para número de lavagens}
    \definition{s.}{água | líquido | encargos ou receitas adicionais}
  \end{phonetics}
\end{entry}

\begin{entry}{水平}{4,5}
  \begin{phonetics}{水平}{shui3ping2}
    \definition{s.}{nível (de realização, etc.) | padrão | nível horizontal}
  \end{phonetics}
\end{entry}

\begin{entry}{水平以下}{4,5,4,3}
  \begin{phonetics}{水平以下}{shui3ping2 yi3xia4}
    \definition{s.}{sub-nível}
  \end{phonetics}
\end{entry}

\begin{entry}{水平尺}{4,5,4}
  \begin{phonetics}{水平尺}{shui3ping2chi3}
    \definition{s.}{nível espiritual}
  \end{phonetics}
\end{entry}

\begin{entry}{水平仪}{4,5,5}
  \begin{phonetics}{水平仪}{shui3ping2yi2}
    \definition{s.}{nível (dispositivo para determinar horizontal) | nível espiritual | nível de topógrafo}
  \end{phonetics}
\end{entry}

\begin{entry}{水平视差}{4,5,8,9}
  \begin{phonetics}{水平视差}{shui3ping2 shi4cha1}
    \definition{s.}{paralaxe horizontal}
  \end{phonetics}
\end{entry}

\begin{entry}{水平度}{4,5,9}
  \begin{phonetics}{水平度}{shui3ping2 du4}
    \definition{s.}{nivelamento}
  \end{phonetics}
\end{entry}

\begin{entry}{水平轴}{4,5,9}
  \begin{phonetics}{水平轴}{shui3ping2zhou2}
    \definition{s.}{eixo horizontal}
  \end{phonetics}
\end{entry}

\begin{entry}{水平面}{4,5,9}
  \begin{phonetics}{水平面}{shui3ping2mian4}
    \definition{s.}{plano horizontal | nível-da-água | superfície horizontal}
  \end{phonetics}
\end{entry}

\begin{entry}{水边}{4,5}
  \begin{phonetics}{水边}{shui3bian1}
    \definition{s.}{beira d'água | beira-mar | costa (de mar, lago ou rio)}
  \end{phonetics}
\end{entry}

\begin{entry}{水污染}{4,6,9}
  \begin{phonetics}{水污染}{shui3wu1ran3}
    \definition{s.}{poluição da água}
  \end{phonetics}
\end{entry}

\begin{entry}{水灵}{4,7}
  \begin{phonetics}{水灵}{shui3ling2}
    \definition{adj.}{cheio de vida (sobre uma pessoa, etc.) | úmido e brilhante (sobre os olhos) | fresco (sobre frutas, etc.) | brilhante | aparência saudável}
  \end{phonetics}
\end{entry}

\begin{entry}{水果}{4,8}
  \begin{phonetics}{水果}{shui3guo3}
    \definition[个]{s.}{fruta}
  \end{phonetics}
\end{entry}

\begin{entry}{水波}{4,8}
  \begin{phonetics}{水波}{shui3bo1}
    \definition{s.}{ondulação (na água) | onda}
  \end{phonetics}
\end{entry}

\begin{entry}{水饺}{4,9}
  \begin{phonetics}{水饺}{shui3jiao3}
    \definition{s.}{\emph{dumplings} | pastéis chineses cozidos}
  \end{phonetics}
\end{entry}

\begin{entry}{水瓶}{4,10}
  \begin{phonetics}{水瓶}{shui3 ping2}
    \definition{s.}{garrada de água}
  \end{phonetics}
\end{entry}

\begin{entry}{水培}{4,11}
  \begin{phonetics}{水培}{shui3pei2}
    \definition{v.}{cultivar plantas hidroponicamente}
  \end{phonetics}
\end{entry}

\begin{entry}{水豚}{4,11}
  \begin{phonetics}{水豚}{shui3tun2}
    \definition{s.}{capivara}
  \end{phonetics}
\end{entry}

\begin{entry}{水路}{4,13}
  \begin{phonetics}{水路}{shui3lu4}
    \definition{s.}{hidrovia}
  \end{phonetics}
\end{entry}

\begin{entry}{水槽}{4,15}
  \begin{phonetics}{水槽}{shui3cao2}
    \definition{s.}{pia (de cozinha)}
  \end{phonetics}
\end{entry}

\begin{entry}{火}{4}[Radical 火][Kangxi 86]
  \begin{phonetics}{火}{huo3}
    \definition*{s.}{sobrenome Huo}
    \definition{adj.}{urgente | ardente ou flamejante | quente (popular)}
    \definition{clas.}{para unidades militares (antigo)}
    \definition{s.}{fogo | munição | calor interno (medicina chinesa)}
  \end{phonetics}
\end{entry}

\begin{entry}{火车}{4,4}
  \begin{phonetics}{火车}{huo3che1}
    \definition[列,节,班,趟]{s.}{trem | comboio}
  \end{phonetics}
\end{entry}

\begin{entry}{火车司机}{4,4,5,6}
  \begin{phonetics}{火车司机}{huo3che1 si1ji1}
    \definition{s.}{maquinista de trem}
  \end{phonetics}
\end{entry}

\begin{entry}{火柴}{4,10}
  \begin{phonetics}{火柴}{huo3chai2}
    \definition[根,盒]{s.}{fósforo (palito de fósforo)}
  \end{phonetics}
\end{entry}

\begin{entry}{火海}{4,10}
  \begin{phonetics}{火海}{huo3hai3}
    \definition{s.}{um mar de chamas}
  \end{phonetics}
\end{entry}

\begin{entry}{父母亲}{4,5,9}
  \begin{phonetics}{父母亲}{fu4mu3qin1}
    \definition{s.}{pais}
  \end{phonetics}
\end{entry}

\begin{entry}{父亲}{4,9}
  \begin{phonetics}{父亲}{fu4qin1}
    \definition[个]{s.}{pai}
  \end{phonetics}
\end{entry}

\begin{entry}{片}{4}[Radical 片][Kangxi 91]
  \begin{phonetics}{片}{pian4}
    \definition{adj.}{parcial | incompleto | que só tem um lado}
    \definition{clas.}{para CDs, filmes, DVDs, etc. | para fatias, comprimidos, extensão de terra, área de água | usado com numeral~一:~para  cenário, cena, sentimento, atmosfera, som etc.}
    \definition{s.}{uma fatia | floco | filme | pedaço fino}
    \definition{v.}{fatiar | esculpir fino}
  \end{phonetics}
\end{entry}

\begin{entry}{牙}{4}[Radical 牙][Kangxi 92]
  \begin{phonetics}{牙}{ya2}
    \definition[颗]{s.}{dente | marfim}
  \end{phonetics}
\end{entry}

\begin{entry}{牙行}{4,6}
  \begin{phonetics}{牙行}{ya2hang2}
    \definition{s.}{corretor | \emph{broker}}
  \end{phonetics}
\end{entry}

\begin{entry}{牙医}{4,7}
  \begin{phonetics}{牙医}{ya2yi1}
    \definition{s.}{dentista}
  \end{phonetics}
\end{entry}

\begin{entry}{牙刷}{4,8}
  \begin{phonetics}{牙刷}{ya2shua1}
    \definition[把]{s.}{escova de dentes}
  \end{phonetics}
\end{entry}

\begin{entry}{牙线}{4,8}
  \begin{phonetics}{牙线}{ya2xian4}
    \definition[条]{s.}{fio dental}
  \end{phonetics}
\end{entry}

\begin{entry}{牙齿}{4,8}
  \begin{phonetics}{牙齿}{ya2chi3}
    \definition{adv.}{dental}
    \definition[颗]{s.}{dente}
  \end{phonetics}
\end{entry}

\begin{entry}{牙膏}{4,14}
  \begin{phonetics}{牙膏}{ya2gao1}
    \definition[管]{s.}{pasta de dente}
  \end{phonetics}
\end{entry}

\begin{entry}{牛}{4}[Radical 牛][Kangxi 93]
  \begin{phonetics}{牛}{niu2}
    \definition*{s.}{sobrenome Niu}
    \definition[条,头]{s.}{boi | touro | vaca | (gíria) incrível}
  \end{phonetics}
\end{entry}

\begin{entry}{牛人}{4,2}
  \begin{phonetics}{牛人}{niu2ren2}
    \definition{s.}{(coloquial) o cara | verdadeiro especialista | \emph{badass}}
  \end{phonetics}
\end{entry}

\begin{entry}{牛仔裤}{4,5,12}
  \begin{phonetics}{牛仔裤}{niu2zai3ku4}
    \definition[条]{s.}{calça de ganga, jeans}
  \end{phonetics}
\end{entry}

\begin{entry}{牛奶}{4,5}
  \begin{phonetics}{牛奶}{niu2nai3}
    \definition[瓶,杯]{s.}{leite de vaca}
  \end{phonetics}
\end{entry}

\begin{entry}{牛肉}{4,6}
  \begin{phonetics}{牛肉}{niu2rou4}
    \definition{s.}{carne de vaca | bife}
  \end{phonetics}
\end{entry}

\begin{entry}{牛郎织女}{4,8,8,3}
  \begin{phonetics}{牛郎织女}{niu2lang2zhi1nv3}
    \definition*{s.}{Vaqueiro e Tecelã (personagens de contos folclóricos) | amantes separados | Altair e Vega (estrelas)}
  \end{phonetics}
\end{entry}

\begin{entry}{牛顿}{4,10}
  \begin{phonetics}{牛顿}{niu2dun4}
    \definition*{s.}{Newton (nome) | newton (N, unidade de força do SI)}
  \end{phonetics}
\end{entry}

\begin{entry}{犬}{4}[Radical 犬][Kangxi 94]
  \begin{phonetics}{犬}{quan3}
    \definition{s.}{cachorro}
  \end{phonetics}
\end{entry}

\begin{entry}{王}{4}[Radical 玉]
  \begin{phonetics}{王}{wang2}
    \definition*{s.}{sobrenome Wang}
    \definition{adj.}{grande | ótimo}
    \definition{s.}{rei ou monarca | melhor ou mais forte do seu tipo}
  \end{phonetics}
  \begin{phonetics}{王}{wang4}
    \definition{v.}{(literário) (um monarca) reinar (um reino)}
  \end{phonetics}
\end{entry}

\begin{entry}{王五}{4,4}
  \begin{phonetics}{王五}{wang2wu3}
    \definition{s.}{Wang Wu | Zé Ninguém | nome para uma pessoa não especificada, 3 de 3}
  \seealsoref{李四}{li3si4}
  \seealsoref{张三}{zhang1san1}
  \end{phonetics}
\end{entry}

\begin{entry}{王朝}{4,12}
  \begin{phonetics}{王朝}{wang2chao2}
    \definition{s.}{dinastia}
  \end{phonetics}
\end{entry}

\begin{entry}{瓦}{4}[Radical 瓦][Kangxi 98]
  \begin{phonetics}{瓦}{wa3}
    \definition{s.}{telha | abreviação de 瓦特}
    \seeref{瓦特}{wa3te4}
  \end{phonetics}
\end{entry}

\begin{entry}{瓦努阿图}{4,7,7,8}
  \begin{phonetics}{瓦努阿图}{wa3nu3'a1tu2}
    \definition*{s.}{Vanuatu, país do sudoeste do Oceano Pacífico}
  \end{phonetics}
\end{entry}

\begin{entry}{瓦特}{4,10}
  \begin{phonetics}{瓦特}{wa3te4}
    \definition{s.}{(empréstimo linguístico) watt | medida de potência}
  \end{phonetics}
\end{entry}

\begin{entry}{艺人}{4,2}
  \begin{phonetics}{艺人}{yi4ren2}
    \definition{s.}{artista | ator}
  \end{phonetics}
\end{entry}

\begin{entry}{见}{4}[Radical 見]
  \begin{phonetics}{见}{jian4}
    \definition{s.}{opinião, visão}
    \definition{v.}{ver | entrevistar | encontrar alguém | parecer (ser alguma coisa)}
  \end{phonetics}
  \begin{phonetics}{见}{xian4}
    \definition{v.}{aparecer | também escrito como 现}
    \seeref{现}{xian4}
  \end{phonetics}
\end{entry}

\begin{entry}{见面}{4,9}
  \begin{phonetics}{见面}{jian4mian4}
    \definition{v.+compl.}{encontrar-se com alguém | ver alguém face-a-face}
  \end{phonetics}
\end{entry}

\begin{entry}{计划}{4,6}
  \begin{phonetics}{计划}{ji4hua4}
    \definition[个,项]{s.}{plano | projeto | programa}
    \definition{v.}{planejar | mapear}
  \end{phonetics}
\end{entry}

\begin{entry}{认识}{4,7}
  \begin{phonetics}{认识}{ren4shi5}
    \definition{s.}{conhecimento | saber | entendimento}
    \definition{v.}{estar familiarizado com | conhecer alguém | saber | reconhecer}
  \end{phonetics}
\end{entry}

\begin{entry}{认真}{4,10}
  \begin{phonetics}{认真}{ren4zhen1}
    \definition{adj.}{sério | consciencioso}
    \definition{adv.}{seriamente}
    \definition{v.}{levar a sério}
  \end{phonetics}
\end{entry}

\begin{entry}{车}{4}[Radical 車][Kangxi 159]
  \begin{phonetics}{车}{che1}
    \definition*{s.}{sobrenome Che}
    \definition[辆]{s.}{carro | veículo | viatura}
  \end{phonetics}
  \begin{phonetics}{车}{ju1}
    \definition{s.}{(arcaico) carruagem de guerra | torre (no xadrez)}
  \end{phonetics}
\end{entry}

\begin{entry}{车子}{4,3}
  \begin{phonetics}{车子}{che1zi5}
    \definition{s.}{qualquer veículo (carro, bicicleta, caminhão, etc)}
  \end{phonetics}
\end{entry}

\begin{entry}{车水马龙}{4,4,3,5}
  \begin{phonetics}{车水马龙}{che1shui3-ma3long2}
    \definition{expr.}{tráfego engarrafado | engarrafamento | (literalmente) ``fluxo interminável de cavalos e carruagens''}
  \end{phonetics}
\end{entry}

\begin{entry}{车主}{4,5}
  \begin{phonetics}{车主}{che1zhu3}
    \definition{s.}{proprietário do carro}
  \end{phonetics}
\end{entry}

\begin{entry}{车次}{4,6}
  \begin{phonetics}{车次}{che1ci4}
    \definition{s.}{número do trem}
  \end{phonetics}
\end{entry}

\begin{entry}{车库}{4,7}
  \begin{phonetics}{车库}{che1ku4}
    \definition{s.}{garagem}
  \end{phonetics}
\end{entry}

\begin{entry}{车站}{4,10}
  \begin{phonetics}{车站}{che1zhan4}
    \definition[处,个]{s.}{estação | ponto de ônibus}
  \end{phonetics}
\end{entry}

\begin{entry}{车牌}{4,12}
  \begin{phonetics}{车牌}{che1pai2}
    \definition{s.}{matrícula | placa de carro}
  \end{phonetics}
\end{entry}

\begin{entry}{长}{4}[Radical 長][Kangxi 168]
  \begin{phonetics}{长}{chang2}
    \definition{adj.}{comprido | longo}
  \end{phonetics}
  \begin{phonetics}{长}{zhang3}
    \definition{s.}{chefe | ancião}
    \definition{v.}{crescer | desenvolver | aumentar | melhorar}
  \end{phonetics}
\end{entry}

\begin{entry}{长城}{4,9}
  \begin{phonetics}{长城}{chang2cheng2}
    \definition*{s.}{Grande Muralha}
  \end{phonetics}
\end{entry}

\begin{entry}{长颈鹿}{4,11,11}
  \begin{phonetics}{长颈鹿}{chang2jing3lu4}
    \definition[只]{s.}{girafa}
  \end{phonetics}
\end{entry}

\begin{entry}{队}{4}[Radical 阜]
  \begin{phonetics}{队}{dui4}
    \definition[个]{s.}{esquadrão | equipe | grupo}
  \end{phonetics}
\end{entry}

\begin{entry}{队友}{4,4}
  \begin{phonetics}{队友}{dui4you3}
    \definition{s.}{companheiro de equipe}
  \end{phonetics}
\end{entry}

\begin{entry}{风}{4}[Radical 風][Kangxi 182]
  \begin{phonetics}{风}{feng1}
    \definition[阵,丝]{s.}{vento}
  \end{phonetics}
\end{entry}

\begin{entry}{风扇}{4,10}
  \begin{phonetics}{风扇}{feng1shan4}
    \definition{s.}{ventilador elétrico}
  \end{phonetics}
\end{entry}

\begin{entry}{风景}{4,12}
  \begin{phonetics}{风景}{feng1jing3}
    \definition{s.}{cenário | paisagem}
  \end{phonetics}
\end{entry}

\begin{entry}{风筝}{4,12}
  \begin{phonetics}{风筝}{feng1zheng5}
    \definition{s.}{pipa | papagaio | pandorga}
  \end{phonetics}
\end{entry}

%%%%% EOF %%%%%


%%%
%%% 5画
%%%

\section*{5画}\addcontentsline{toc}{section}{5画}

\begin{Entry}{且}{5}{⼀}
  \begin{Phonetics}{且}{qie3}
    \definition*{s.}{Sobrenome Qie}
    \definition{adv.}{apenas; por enquanto | por um longo tempo}
    \definition{conj.}{mesmo; até; até mesmo; usado na primeira cláusula de uma frase complexa para expressar concessão, equivalente a 尚且 | ambos\dots e\dots; conecta adjetivos ou verbos para expressar relacionamento paralelo, equivalente a 而且 e 又……又……}
  \seealsoref{而且}{er2 qie3}
  \seealsoref{尚且}{shang4 qie3}
  \seealsoref{又……又……}{you4 you4}
  \end{Phonetics}
\end{Entry}

\begin{Entry}{世}{5}{⼀}
  \begin{Phonetics}{世}{shi4}
    \definition*{s.}{Sobrenome Shi}
    \definition{s.}{vida; tempo de vida; vida humana | geração; geração após geração | idade; era | o mundo; sociedade | (geologia) época, abaixo de ``período''}
  \end{Phonetics}
\end{Entry}

\begin{Entry}{世代}{5,5}{⼀、⼈}
  \begin{Phonetics}{世代}{shi4dai4}
    \definition{adv.}{por muitas gerações, eras}
    \definition{s.}{geração | era}
  \end{Phonetics}
\end{Entry}

\begin{Entry}{世纪}{5,6}{⼀、⽷}
  \begin{Phonetics}{世纪}{shi4ji4}[][HSK 3]
    \definition[个,段]{s.}{século; uma unidade para calcular anos, cem anos é um século}
  \end{Phonetics}
\end{Entry}

\begin{Entry}{世界}{5,9}{⼀、⽥}
  \begin{Phonetics}{世界}{shi4jie4}[][HSK 3]
    \definition[个,片,种]{s.}{mundo; todos os lugares da Terra | a soma da natureza e da sociedade humana; refere-se à soma de toda a existência objetiva na natureza e na sociedade humana | campo; refere-se a uma determinada área ou campo | o universo sem limites; costumava ser um termo budista, mas agora também se refere ao mundo natural ilimitado e à sociedade humana | situação social; a situação ou atmosfera social de um determinado período}
  \end{Phonetics}
\end{Entry}

\begin{Entry}{世界杯}{5,9,8}{⼀、⽥、⽊}
  \begin{Phonetics}{世界杯}{shi4jie4bei1}[][HSK 3]
    \definition*{s.}{Copa do Mundo; Troféu da Copa do Mundo}
  \end{Phonetics}
\end{Entry}

\begin{Entry}{世锦赛}{5,13,14}{⼀、⾦、⾙}
  \begin{Phonetics}{世锦赛}{shi4jin3sai4}
    \definition*{s.}{Campeonato Mundial}
  \end{Phonetics}
\end{Entry}

\begin{Entry}{丘}{5}{⼀}
  \begin{Phonetics}{丘}{qiu1}
    \definition*{s.}{Sobrenome Qiu}
    \definition[个]{s.}{monte; outeiro | (literário) sepultura}
  \end{Phonetics}
\end{Entry}

\begin{Entry}{丘陵}{5,10}{⼀、⾩}
  \begin{Phonetics}{丘陵}{qiu1ling2}
    \definition{s.}{colinas}
  \end{Phonetics}
\end{Entry}

\begin{Entry}{业}{5}{⼀}
  \begin{Phonetics}{业}{ye4}
    \definition*{s.}{Sobrenome Ye}
    \definition{adv.}{já; indica que a ação foi concluída, equivalente a 已经}
    \definition{s.}{comércio; indústria; ramo de negócios | emprego; ocupação; profissão | curso de estudo | causa; empreendimento | propriedade | carma; o budismo se refere a todas as ações, palavras e pensamentos humanos como carma, que são chamados de carma corporal, carma da fala e carma mental; o carma inclui aspectos bons e ruins, geralmente referindo-se ao destino ou ao pecado}
    \definition{v.}{envolver-se em; exercer uma determinada profissão}
  \seealsoref{已经}{yi3jing1}
  \end{Phonetics}
\end{Entry}

\begin{Entry}{业务}{5,5}{⼀、⼒}
  \begin{Phonetics}{业务}{ye4wu4}[][HSK 5]
    \definition[项,笔,个,类,种]{s.}{negócios; trabalho vocacional; trabalho profissional}
  \end{Phonetics}
\end{Entry}

\begin{Entry}{业余}{5,7}{⼀、⼈}
  \begin{Phonetics}{业余}{ye4yu2}[][HSK 4]
    \definition{adj.}{tempo livre; depois do expediente; fora do horário de trabalho | amador; não profissional}
  \end{Phonetics}
\end{Entry}

\begin{Entry}{东}{5}{⼀}
  \begin{Phonetics}{东}{dong1}[][HSK 1]
    \definition*{s.}{Sobrenome Dong}
    \definition{s.}{leste; uma das quatro direções básicas, o lado onde o sol nasce | proprietário; dono | anfitrião (antigamente, o anfitrião ficava a leste e os convidados a oeste)}
  \end{Phonetics}
\end{Entry}

\begin{Entry}{东方}{5,4}{⼀、⽅}
  \begin{Phonetics}{东方}{dong1 fang1}[][HSK 2]
    \definition*{s.}{Sobrenome Dongfang}
    \definition{s.}{leste | oriente; o leste; o Oriente}
  \end{Phonetics}
\end{Entry}

\begin{Entry}{东方学院}{5,4,8,9}{⼀、⽅、⼦、⾩}
  \begin{Phonetics}{东方学院}{dong1fang1 xue2yuan4}
    \definition*{s.}{Instituto Oriental}
  \end{Phonetics}
\end{Entry}

\begin{Entry}{东北}{5,5}{⼀、⼔}
  \begin{Phonetics}{东北}{dong1 bei3}[][HSK 2]
    \definition*{s.}{Nordeste da China; O Nordeste | Manchúria}
    \definition{s.}{nordeste}
  \end{Phonetics}
\end{Entry}

\begin{Entry}{东半球}{5,5,11}{⼀、⼗、⽟}
  \begin{Phonetics}{东半球}{dong1ban4qiu2}
    \definition*{s.}{Hemisfério Oriental}
  \end{Phonetics}
\end{Entry}

\begin{Entry}{东边}{5,5}{⼀、⾡}
  \begin{Phonetics}{东边}{dong1 bian5}[][HSK 1]
    \definition{s.}{leste; o lado leste; refere-se à fronteira oriental}
  \end{Phonetics}
\end{Entry}

\begin{Entry}{东西}{5,6}{⼀、⾑}
  \begin{Phonetics}{东西}{dong1xi1}
    \definition{s.}{leste e oeste | de leste a oeste; a distância de um lugar de leste a oeste}
  \end{Phonetics}
  \begin{Phonetics}{东西}{dong1xi5}[][HSK 1]
    \definition[个,件]{s.}{coisa; refere-se a todos os tipos de coisas concretas ou abstratas | coisa; criatura; refere-se especificamente a pessoas ou coisas que causam repulsa ou simpatia}
  \end{Phonetics}
\end{Entry}

\begin{Entry}{东南}{5,9}{⼀、⼗}
  \begin{Phonetics}{东南}{dong1 nan2}[][HSK 2]
    \definition*{s.}{Sudeste da China; O Sudeste; refere-se à região costeira sudeste da China, incluindo as províncias e cidades de Xangai, Jiangsu, Zhejiang, Fujian, Taiwan, etc.}
    \definition{s.}{sudeste}
  \end{Phonetics}
\end{Entry}

\begin{Entry}{东面}{5,9}{⼀、⾯}
  \begin{Phonetics}{东面}{dong1mian4}
    \definition{s.}{lado leste (de algo)}
  \end{Phonetics}
\end{Entry}

\begin{Entry}{东部}{5,10}{⼀、⾢}
  \begin{Phonetics}{东部}{dong1 bu4}[][HSK 3]
    \definition{s.}{o leste; parte oriental; a parte oriental de uma determinada região}
  \end{Phonetics}
\end{Entry}

\begin{Entry}{丝}{5}{⼀}
  \begin{Phonetics}{丝}{si1}
    \definition{clas.}{si, uma unidade de peso (=0,0005 gramas) | usado para expressar a aparência ou expressão de uma pessoa | um décimo de milésimo de certas unidades de medida (medida de comprimento) | usado para representar coisas abstratas}
    \definition[些,种,类,跟,缕]{s.}{seda | uma coisa semelhante a um fio; itens semelhantes à seda | cordas; instrumentos de corda}
  \end{Phonetics}
\end{Entry}

\begin{Entry}{主}{5}{⼂}
  \begin{Phonetics}{主}{zhu3}
    \definition*{s.}{Deus; Senhor; o nome do Deus em que se acredita o cristianismo, o judaísmo, etc.}
    \definition{adj.}{principal; primário; o mais básico; o mais importante | de si mesmo; por vontade própria; próprio; do próprio}
    \definition[位,名,个]{s.}{anfitrião; alguém que convida e recebe convidados (oposto de 宾 e 客) | mestre; dono; uma pessoa que possui poder ou propriedade; uma pessoa em posição dominante | pessoa ou parte interessada | decisão; opinião; visão definitiva | placa espiritual (ou memorial)}
    \definition{v.}{dirigir; administrar; assumir o comando de; presidir; assumir a responsabilidade primária | decidir; reivindicar | significar; indicar; prever um certo resultado}
  \seealsoref{宾}{bin1}
  \seealsoref{客}{ke4}
  \end{Phonetics}
\end{Entry}

\begin{Entry}{主人}{5,2}{⼂、⼈}
  \begin{Phonetics}{主人}{zhu3ren2}[][HSK 2]
    \definition[个,位]{s.}{mestre; uma pessoa que empregava tutores, contadores, etc. antigamente; uma pessoa que empregava empregados domésticos | anfitrião; Aaguém que entretém convidados (em oposição a 客人) | proprietário; uma pessoa que possui um certo tipo de bens ou poder}
  \seealsoref{客人}{ke4ren2}
  \end{Phonetics}
\end{Entry}

\begin{Entry}{主义}{5,3}{⼂、⼂}
  \begin{Phonetics}{主义}{zhu3yi4}
    \definition[种]{s.}{doutrina; um determinado sistema social ou sistema político e econômico | estilo de pensamento; um certo ponto de vista ou estilo | ideologia; teorias e doutrinas sistemáticas sobre a natureza, a sociedade humana, etc.}
    \definition{suf.}{-ismo}
  \end{Phonetics}
\end{Entry}

\begin{Entry}{主办}{5,4}{⼂、⼒}
  \begin{Phonetics}{主办}{zhu3ban4}[][HSK 5]
    \definition{v.}{manter; hospedar; dirigir; patrocinar}
  \end{Phonetics}
\end{Entry}

\begin{Entry}{主任}{5,6}{⼂、⼈}
  \begin{Phonetics}{主任}{zhu3ren4}[][HSK 3]
    \definition[个,位,名]{s.}{chefe; diretor; presidente; o principal responsável por um departamento ou instituição}
  \end{Phonetics}
\end{Entry}

\begin{Entry}{主动}{5,6}{⼂、⼒}
  \begin{Phonetics}{主动}{zhu3dong4}[][HSK 3]
    \definition{adj.}{ativo; positivo; agir sem esperar por um impulso externo (em oposição a 被动) | iniciativo; capaz de impulsionar as coisas por vontade própria; capaz de criar uma situação favorável e fazer as coisas acontecerem de acordo com suas próprias intenções (em oposição a 被动)}
  \seealsoref{被动}{bei4dong4}
  \end{Phonetics}
\end{Entry}

\begin{Entry}{主导}{5,6}{⼂、⼨}
  \begin{Phonetics}{主导}{zhu3dao3}[][HSK 5]
    \definition{adj.}{líder; dominante; guiado; principais e guias para que as coisas se desenvolvam em uma determinada direção}
    \definition{s.}{fator principal (ou orientador)}
  \end{Phonetics}
\end{Entry}

\begin{Entry}{主观}{5,6}{⼂、⾒}
  \begin{Phonetics}{主观}{zhu3guan1}[][HSK 5]
    \definition{adj.}{subjetivo; não com base nas condições reais, mas com base nos próprios desejos | subjetivo; filosoficamente, refere-se à consciência e aos aspectos espirituais dos seres humanos}
  \end{Phonetics}
\end{Entry}

\begin{Entry}{主体}{5,7}{⼂、⼈}
  \begin{Phonetics}{主体}{zhu3 ti3}[][HSK 5]
    \definition[个,些,种,群]{s.}{corpo principal; parte principal; parte principal; esteio; a parte principal das coisas | Filosofia: sujeito}
  \end{Phonetics}
\end{Entry}

\begin{Entry}{主张}{5,7}{⼂、⼸}
  \begin{Phonetics}{主张}{zhu3zhang1}[][HSK 3]
    \definition[个,项,些,种]{s.}{vista; posição; proposição}
    \definition{v.}{defender; apoiar; manter; representar; ter uma opinião sobre como agir, fazer uma sugestão}
  \end{Phonetics}
\end{Entry}

\begin{Entry}{主角}{5,7}{⼂、⾓}
  \begin{Phonetics}{主角}{zhu3 jue2}[][HSK 6]
    \definition[个,位,名]{s.}{liderança; papel principal; protagonista; um papel importante em uma peça, filme, etc.; um ator que desempenha um papel importante | (figurado) algo que tem grande influência em uma determinada área; refere-se ao personagem principal}
  \end{Phonetics}
\end{Entry}

\begin{Entry}{主持}{5,9}{⼂、⼿}
  \begin{Phonetics}{主持}{zhu3chi2}[][HSK 3]
    \definition[位,名]{s.}{anfitrião; a pessoa responsável por administrar e lidar com uma determinada atividade}
    \definition{v.}{dirigir; administrar; assumir o comando; encarregar-se de; ser responsável por gerenciar, organizar uma determinada atividade ou lidar com um determinado assunto | defender; apoiar; preservar; manter}
  \end{Phonetics}
\end{Entry}

\begin{Entry}{主持人}{5,9,2}{⼂、⼿、⼈}
  \begin{Phonetics}{主持人}{zhu3 chi2 ren2}[][HSK 6]
    \definition[个,位]{s.}{anfitrião; âncora; apresentador}
  \end{Phonetics}
\end{Entry}

\begin{Entry}{主要}{5,9}{⼂、⾑}
  \begin{Phonetics}{主要}{zhu3yao4}[][HSK 2]
    \definition{adj.}{principal; chefe; o mais importante na questão; o decisivo | principal; núcleo; a raiz ou parte mais importante de algo}
  \end{Phonetics}
\end{Entry}

\begin{Entry}{主席}{5,10}{⼂、⼱}
  \begin{Phonetics}{主席}{zhu3xi2}[][HSK 4]
    \definition*{s.}{Presidente (da China)}
    \definition[个,位,名]{s.}{presidente, \emph{chairman} (de uma reunião) | chefe; presidente (de uma organização ou estado)}
  \end{Phonetics}
\end{Entry}

\begin{Entry}{主席台}{5,10,5}{⼂、⼱、⼝}
  \begin{Phonetics}{主席台}{zhu3xi2tai2}
    \definition[个]{s.}{plataforma | tribuna}
  \end{Phonetics}
\end{Entry}

\begin{Entry}{主席团}{5,10,6}{⼂、⼱、⼞}
  \begin{Phonetics}{主席团}{zhu3xi2tuan2}
    \definition{s.}{presídio}
  \end{Phonetics}
\end{Entry}

\begin{Entry}{主流}{5,10}{⼂、⽔}
  \begin{Phonetics}{主流}{zhu3liu2}[][HSK 6]
    \definition{s.}{corrente principal; corrente mãe; convencional | tendência principal; aspecto essencial ou principal; falando metaforicamente, os principais aspectos do desenvolvimento das coisas}
  \end{Phonetics}
\end{Entry}

\begin{Entry}{主意}{5,13}{⼂、⼼}
  \begin{Phonetics}{主意}{zhu3yi5}[][HSK 3]
    \definition[个,种]{s.}{ideia; plano; decisão; método}
  \end{Phonetics}
\end{Entry}

\begin{Entry}{主管}{5,14}{⼂、⽵}
  \begin{Phonetics}{主管}{zhu3guan3}[][HSK 5]
    \definition[位,名,个,些]{s.}{pessoa responsável, como supervisor, gerente, diretor, etc.}
    \definition{v.}{estar encarregado de; ser responsável por; ser o principal responsável pela gestão de um trabalho; assumir a responsabilidade primária pela gestão (um certo aspecto)}
  \end{Phonetics}
\end{Entry}

\begin{Entry}{主题}{5,15}{⼂、⾴}
  \begin{Phonetics}{主题}{zhu3ti2}[][HSK 4]
    \definition[个]{s.}{tema; assunto; motivo; lema; ideias básicas expressas em toda a obra de literatura e arte por meio de imagens artísticas concretas | pontos/conteúdos principais; referência geral ao conteúdo principal de artigos, discursos, conferências, etc.}
  \end{Phonetics}
\end{Entry}

\begin{Entry}{乐}{5}{⼃}
  \begin{Phonetics}{乐}{le4}[][HSK 3]
    \definition*{s.}{Sobrenome Le}
    \definition{adj.}{feliz; contente; rejubilante; animado; bem disposto}
    \definition{s.}{prazer; diversão; felicidade}
    \definition{v.}{desfrutar; ficar feliz em; amar; encontrar prazer em | rir; divertir-se}
  \end{Phonetics}
  \begin{Phonetics}{乐}{yue4}
    \definition*{s.}{Sobrenome Yue}
    \definition{s.}{música}
  \end{Phonetics}
\end{Entry}

\begin{Entry}{乐队}{5,4}{⼃、⾩}
  \begin{Phonetics}{乐队}{yue4 dui4}[][HSK 3]
    \definition[支,个]{s.}{orquestra; banda; um grupo composto por muitas pessoas que tocam diferentes instrumentos musicais}
  \end{Phonetics}
\end{Entry}

\begin{Entry}{乐曲}{5,6}{⼃、⽈}
  \begin{Phonetics}{乐曲}{yue4 qu3}[][HSK 6]
    \definition[支,首,段]{s.}{música; composição musical}
  \end{Phonetics}
\end{Entry}

\begin{Entry}{乐观}{5,6}{⼃、⾒}
  \begin{Phonetics}{乐观}{le4guan1}[][HSK 3]
    \definition{adj.}{esperançoso; otimista; confiante; espírito alegre, confiante no futuro (oposto a 悲观)}
  \seealsoref{悲观}{bei1guan1}
  \end{Phonetics}
\end{Entry}

\begin{Entry}{乐园}{5,7}{⼃、⼞}
  \begin{Phonetics}{乐园}{le4yuan2}
    \definition{s.}{paraíso}
  \end{Phonetics}
\end{Entry}

\begin{Entry}{乐高}{5,10}{⼃、⾼}
  \begin{Phonetics}{乐高}{le4gao1}
    \definition*{s.}{Lego (brinquedo)}
  \end{Phonetics}
\end{Entry}

\begin{Entry}{乐趣}{5,15}{⼃、⾛}
  \begin{Phonetics}{乐趣}{le4qu4}[][HSK 4]
    \definition[个,种,些,点]{s.}{alegria; deleite; prazer; implicação de fazer alguém se sentir feliz; um humor de preferência}
  \end{Phonetics}
\end{Entry}

\begin{Entry}{仔}{5}{⼈}
  \begin{Phonetics}{仔}{zi1}
    \definition{s./v.}{usado em 仔肩}
  \seealsoref{仔肩}{zi1jian1}
  \end{Phonetics}
  \begin{Phonetics}{仔}{zi3}
    \definition{adj.}{jovem | cuidadoso; pequeno; fino}
  \end{Phonetics}
\end{Entry}

\begin{Entry}{仔细}{5,8}{⼈、⽷}
  \begin{Phonetics}{仔细}{zi3xi4}[][HSK 5]
    \definition{adj.}{cuidadoso; atencioso; descreve alguém que é cuidadoso e meticuloso ao fazer as coisas; não é descuidado | frugal; econômico; descreve o uso moderado de dinheiro ou bens, sem desperdício}
    \definition{v.}{ter cuidado; prestar atenção; ter muito cuidado e evitar que aconteçam coisas ruins}
  \end{Phonetics}
\end{Entry}

\begin{Entry}{仔肩}{5,8}{⼈、⾁}
  \begin{Phonetics}{仔肩}{zi1jian1}
    \definition{s.}{encargos oficiais (ou responsabilidades)}
    \definition{v.}{assumir a responsabilidade por algo}
  \end{Phonetics}
\end{Entry}

\begin{Entry}{他}{5}{⼈}
  \begin{Phonetics}{他}{ta1}[][HSK 1]
    \definition{pron.}{ele | outro; referindo-se a outro; diferente | usado após o verbo, indica referência vaga | alguém; todos; usado em conjunto com 你, significa qualquer pessoa ou muitas pessoas | em outro lugar; outro lugar}
  \seealsoref{你}{ni3}
  \seealsoref{怹}{tan1}
  \end{Phonetics}
\end{Entry}

\begin{Entry}{他们}{5,5}{⼈、⼈}
  \begin{Phonetics}{他们}{ta1men5}[][HSK 1]
    \definition{pron.}{eles}
  \end{Phonetics}
\end{Entry}

\begin{Entry}{他们的}{5,5,8}{⼈、⼈、⽩}
  \begin{Phonetics}{他们的}{ta1men5 de5}
    \definition{pron.}{deles}
  \end{Phonetics}
\end{Entry}

\begin{Entry}{他妈的}{5,6,8}{⼈、⼥、⽩}
  \begin{Phonetics}{他妈的}{ta1ma1de5}
    \definition{interj.}{Dane-se! | Foda-se!}
  \end{Phonetics}
\end{Entry}

\begin{Entry}{他的}{5,8}{⼈、⽩}
  \begin{Phonetics}{他的}{ta1 de5}
    \definition{pron.}{dele}
  \end{Phonetics}
\end{Entry}

\begin{Entry}{付}{5}{⼈}
  \begin{Phonetics}{付}{fu4}[][HSK 3]
    \definition*{s.}{Sobrenome Fu}
    \definition{clas.}{usado para pares ou conjuntos de coisas | usado para expressões faciais}
    \definition{v.}{comprometer-se com; entregar (ou transferir) para | pagar; refere-se especificamente a dar dinheiro}
  \end{Phonetics}
\end{Entry}

\begin{Entry}{付出}{5,5}{⼈、⼐}
  \begin{Phonetics}{付出}{fu4 chu1}[][HSK 4]
    \definition{v.}{pagar; gastar; entregar (dinheiro, consideração, etc.)}
  \end{Phonetics}
\end{Entry}

\begin{Entry}{付款}{5,12}{⼈、⽋}
  \begin{Phonetics}{付款}{fu4kuan3}
    \definition{s.}{pagamento}
    \definition{v.+compl.}{pagar uma quantia em dinheiro}
  \end{Phonetics}
\end{Entry}

\begin{Entry}{仙}{5}{⼈}
  \begin{Phonetics}{仙}{xian1}
    \definition{s.}{imortal}
  \end{Phonetics}
\end{Entry}

\begin{Entry}{代}{5}{⼈}
  \begin{Phonetics}{代}{dai4}[][HSK 3]
    \definition*{s.}{Sobrenome Dai}
    \definition{s.}{dinastia | geração; hierarquia familiar | era; o segundo nível da divisão geológica é o período, acima do qual está a era e abaixo do qual está o período, por exemplo, o Paleozóico, o Mesozóico e o Cenozóico pertencem à era Phanerozoico | período histórico; época}
    \definition{v.}{tomar o lugar de; estar no lugar de | agir em nome de; exercer}
  \end{Phonetics}
\end{Entry}

\begin{Entry}{代价}{5,6}{⼈、⼈}
  \begin{Phonetics}{代价}{dai4jia4}[][HSK 5]
    \definition[种,个]{s.}{preço; material, energia gasta ou sacrifícios feitos para atingir um objetivo | custo; preço; dinheiro pago para obter algo}
  \end{Phonetics}
\end{Entry}

\begin{Entry}{代言}{5,7}{⼈、⾔}
  \begin{Phonetics}{代言}{dai4yan2}
    \definition{v.}{ser um porta-voz | ser um embaixador (para uma marca) | endossar}
  \end{Phonetics}
\end{Entry}

\begin{Entry}{代表}{5,8}{⼈、⾐}
  \begin{Phonetics}{代表}{dai4biao3}[][HSK 3]
    \definition[位,名,个,些]{s.}{deputado; delegado; representante; pessoas eleitas para representar eleitores ou expressar opiniões, ou pessoas encarregadas ou designadas para representar indivíduos, grupos ou governos ou expressar opiniões | representante oficial; pessoas ou coisas que refletem as características comuns de um grupo específico}
    \definition{v.}{representar; defender | usar pessoas ou coisas para expressar um significado ou conceito específico}
  \end{Phonetics}
\end{Entry}

\begin{Entry}{代表团}{5,8,6}{⼈、⾐、⼞}
  \begin{Phonetics}{代表团}{dai4 biao3 tuan2}[][HSK 3]
    \definition[个]{s.}{delegação; contingente; um grupo temporário de grande dimensão formado para participar de uma determinada atividade em nome de um país, governo ou outra organização social}
  \end{Phonetics}
\end{Entry}

\begin{Entry}{代称}{5,10}{⼈、⽲}
  \begin{Phonetics}{代称}{dai4cheng1}
    \definition{s.}{nome alternativo | antonomásia}
    \definition{v.}{referir-se a algo ou alguém por outro nome}
  \end{Phonetics}
\end{Entry}

\begin{Entry}{代理}{5,11}{⼈、⽟}
  \begin{Phonetics}{代理}{dai4li3}[][HSK 5]
    \definition{v.}{agir em nome de alguém em uma posição de responsabilidade; substituir alguém | agir como procurador; agir como agente; ser encarregado pelas partes de realizar atividades e conduzir assuntos em seu nome dentro do escopo de sua autorização}
  \end{Phonetics}
\end{Entry}

\begin{Entry}{代替}{5,12}{⼈、⽈}
  \begin{Phonetics}{代替}{dai4ti4}[][HSK 4]
    \definition{v.}{substituir; substituir por; tomar o lugar de}
  \end{Phonetics}
\end{Entry}

\begin{Entry}{令}{5}{⼈}
  \begin{Phonetics}{令}{ling2}
    \definition*{s.}{Antigo nome geográfico, na região atual de Linyi, província de Shanxi | Sobrenome Ling}
  \end{Phonetics}
  \begin{Phonetics}{令}{ling3}
    \definition{clas.}{resma (de papel); unidade de medida de papel: 500 folhas inteiras de papel original produzidas mecanicamente equivalem a 1 resma}
  \end{Phonetics}
  \begin{Phonetics}{令}{ling4}[][HSK 5]
    \definition{adj.}{bom; excelente | termos de cortesia usados para se referir aos familiares e parentes da outra pessoa}
    \definition{s.}{ordem; decreto; comando; ordem emitida pela autoridade superior | um título oficial; administradores de certos departamentos governamentais na antiguidade | temporada; estação; clima e fenologia de uma determinada estação | poema-canção; letra curta}
    \definition{v.}{ordenar; comandar | fazer com que alguém; fazer com que; permitir que}
  \end{Phonetics}
\end{Entry}

\begin{Entry}{令人}{5,2}{⼈、⼈}
  \begin{Phonetics}{令人}{ling4ren2}
    \definition{v.}{causar alguém (a fazer alguma coisa) | fazer alguém ficar zangado, encantado, etc.}
  \end{Phonetics}
\end{Entry}

\begin{Entry}{仪}{5}{⼈}
  \begin{Phonetics}{仪}{yi2}
    \definition*{s.}{Sobrenome Yi}
    \definition{s.}{aparência; porte | cerimônia; rito | presente; dádiva | aparelho; instrumento}
    \definition{v.}{(literário) admirar; ansiar por}
  \end{Phonetics}
\end{Entry}

\begin{Entry}{仪式}{5,6}{⼈、⼷}
  \begin{Phonetics}{仪式}{yi2shi4}[][HSK 6]
    \definition{s.}{rito; cerimônia; procedimento e formato da cerimônia}
  \end{Phonetics}
\end{Entry}

\begin{Entry}{仪器}{5,16}{⼈、⼝}
  \begin{Phonetics}{仪器}{yi2qi4}[][HSK 6]
    \definition[台]{s.}{aparelho; instrumento; ferramentas ou equipamentos utilizados para observação, medição, inspeção, etc. em pesquisas ou experimentos científicos; geralmente, são relativamente precisos e padronizados}
  \end{Phonetics}
\end{Entry}

\begin{Entry}{们}{5}{⼈}
  \begin{Phonetics}{们}{men5}[][HSK 1]
    \definition{suf.}{usado após pronomes ou substantivos que se referem a pessoas para indicar pluralidade}
  \end{Phonetics}
\end{Entry}

\begin{Entry}{兄}{5}{⼉}
  \begin{Phonetics}{兄}{xiong1}
    \definition{s.}{irmão mais velho | parente mais velho do sexo masculino da mesma geração | uma forma cortês de tratamento entre amigos homens; um título respeitoso para amigos homens}
  \end{Phonetics}
\end{Entry}

\begin{Entry}{兄弟}{5,7}{⼉、⼸}
  \begin{Phonetics}{兄弟}{xiong1di4}[][HSK 4]
    \definition{adj.}{fraternal}
    \definition{pron.}{eu, me (termo de uso humilde por homens em discurso público)}
    \definition[个,位]{s.}{irmãos; irmão}
  \end{Phonetics}
\end{Entry}

\begin{Entry}{兰}{5}{⼋}
  \begin{Phonetics}{兰}{lan2}
    \definition*{s.}{Sobrenome Lan}
    \definition{s.}{orquídea | lírio magnólia}
  \end{Phonetics}
\end{Entry}

\begin{Entry}{兰州}{5,6}{⼋、⼮}
  \begin{Phonetics}{兰州}{lan2zhou1}
    \definition*{s.}{Lanzhou. capital da província de Gansu, 甘肃}
  \seealsoref{甘肃}{gan1su4}
  \end{Phonetics}
\end{Entry}

\begin{Entry}{兰花}{5,7}{⼋、⾋}
  \begin{Phonetics}{兰花}{lan2hua1}
    \definition{s.}{orquídea}
  \end{Phonetics}
\end{Entry}

\begin{Entry}{册}{5}{⼌}
  \begin{Phonetics}{册}{ce4}[][HSK 5]
    \definition{clas.}{usado para cópias de livros}
    \definition{s.}{volume; livro | cópia; volume | ordem imperial para conferir um título}
    \definition{v.}{conferir um título}
  \end{Phonetics}
\end{Entry}

\begin{Entry}{写}{5}{⼍}
  \begin{Phonetics}{写}{xie3}[][HSK 1]
    \definition{v.}{escrever | compor; escrever (como autor, repórter, etc.) | descrever; retratar | pintar; desenhar | expressar a imagem das coisas através da linguagem e da escrita | desenhar (pintura)}
  \end{Phonetics}
\end{Entry}

\begin{Entry}{写字}{5,6}{⼍、⼦}
  \begin{Phonetics}{写字}{xie3zi4}
    \definition{v.}{escrever (à mão) | praticar caligrafia}
  \end{Phonetics}
\end{Entry}

\begin{Entry}{写字台}{5,6,5}{⼍、⼦、⼝}
  \begin{Phonetics}{写字台}{xie3 zi4 tai2}[][HSK 6]
    \definition[个,张]{s.}{escrivaninha; secretária; escrivaninha de escrever; uma mesa retangular usada para escrever e trabalhar, com gavetas e algumas com pequenos armários}
  \end{Phonetics}
\end{Entry}

\begin{Entry}{写字匠}{5,6,6}{⼍、⼦、⼕}
  \begin{Phonetics}{写字匠}{xie3zi4 jiang4}
    \definition{s.}{calígrafo}
  \end{Phonetics}
\end{Entry}

\begin{Entry}{写字楼}{5,6,13}{⼍、⼦、⽊}
  \begin{Phonetics}{写字楼}{xie3 zi4 lou2}[][HSK 6]
    \definition{s.}{prédio de escritórios}
  \end{Phonetics}
\end{Entry}

\begin{Entry}{写作}{5,7}{⼍、⼈}
  \begin{Phonetics}{写作}{xie3zuo4}[][HSK 3]
    \definition{v.}{escrever artigos; escrever livros, etc.; também se refere especificamente à criação de obras literárias}
  \end{Phonetics}
\end{Entry}

\begin{Entry}{写真}{5,10}{⼍、⼗}
  \begin{Phonetics}{写真}{xie3zhen1}
    \definition{s.}{retrato}
    \definition{v.}{descrever algo com precisão}
  \end{Phonetics}
\end{Entry}

\begin{Entry}{写意}{5,13}{⼍、⼼}
  \begin{Phonetics}{写意}{xie3yi4}
    \definition{s.}{estilo de pintura chinesa à mão livre, caracterizado por traços ousados em vez de detalhes precisos}
    \definition{v.}{sugerir (em vez de descrever em detalhes)}
  \end{Phonetics}
  \begin{Phonetics}{写意}{xie4yi4}
    \definition{adj.}{confortável | agradável | relaxado}
  \end{Phonetics}
\end{Entry}

\begin{Entry}{写照}{5,13}{⼍、⽕}
  \begin{Phonetics}{写照}{xie3zhao4}
    \definition{s.}{retrato}
  \end{Phonetics}
\end{Entry}

\begin{Entry}{冬}{5}{⼎}
  \begin{Phonetics}{冬}{dong1}
    \definition*{s.}{Sobrenome Dong}
    \definition{s.}{inverno}
    \definition{s.}{onomatopéia: som de um tambor batendo, batendo na porta, etc.}
  \end{Phonetics}
\end{Entry}

\begin{Entry}{冬天}{5,4}{⼎、⼤}
  \begin{Phonetics}{冬天}{dong1 tian1}[][HSK 2]
    \definition[个]{s.}{inverno; a quarta estação do ano, na China, geralmente se refere aos três meses entre outubro e dezembro do calendário lunar}
  \end{Phonetics}
\end{Entry}

\begin{Entry}{冬瓜}{5,5}{⼎、⽠}
  \begin{Phonetics}{冬瓜}{dong1gua1}
    \definition{s.}{melão de inverno}
  \end{Phonetics}
\end{Entry}

\begin{Entry}{冬季}{5,8}{⼎、⼦}
  \begin{Phonetics}{冬季}{dong1 ji4}[][HSK 4]
    \definition[个,次,种]{s.}{inverno; o quarto trimestre do ano, habitualmente referido na China como o período de três meses entre o início do inverno e o início da primavera, e também referido como ``décimo, décimo primeiro e décimo segundo'' meses do calendário lunar}
  \end{Phonetics}
\end{Entry}

\begin{Entry}{出}{5}{⼐}
  \begin{Phonetics}{出}{chu1}[][HSK 1]
    \definition{clas.}{usado para dramas, peças, óperas, etc.}
    \definition{v.}{deixar; sair (ir); de dentro para fora | vir; chegar | exceder; ir além | emitir; levar para fora | produzir; despejar | surgir; acontecer; ter lugar | publicar; divulgar | ventilar; emitir; descarregar | aparecer; revelar | gastar; pagar}
  \end{Phonetics}
\end{Entry}

\begin{Entry}{出入}{5,2}{⼐、⼊}
  \begin{Phonetics}{出入}{chu1 ru4}[][HSK 6]
    \definition{s.}{discrepância; divergência; inconsistência; diferença}
    \definition{v.}{entrar e sair; sair e entrar}
  \end{Phonetics}
\end{Entry}

\begin{Entry}{出于}{5,3}{⼐、⼆}
  \begin{Phonetics}{出于}{chu1 yu2}[][HSK 5]
    \definition{prep.}{de; fora de; por causa de; em função de; de um certo ponto de vista}
    \definition{v.}{iniciar a partir de; originar-se de; prosseguir a partir de}
  \end{Phonetics}
\end{Entry}

\begin{Entry}{出口}{5,3}{⼐、⼝}
  \begin{Phonetics}{出口}{chu1kou3}[][HSK 2,4]
    \definition[个]{s.}{saída; porta ou passagem que dá acesso ao exterior}
    \definition{v.+compl.}{falar; proferir; manifestar-se | exportar mercadorias do país ou da região para venda no exterior ou em outro lugar | deixar o porto (de um navio)}
  \end{Phonetics}
\end{Entry}

\begin{Entry}{出门}{5,3}{⼐、⾨}
  \begin{Phonetics}{出门}{chu1 men2}[][HSK 2]
    \definition{v.+compl.}{sair | sair de casa; estar longe de casa; viajar para longe de casa | casar-se}
  \end{Phonetics}
\end{Entry}

\begin{Entry}{出击}{5,5}{⼐、⼐}
  \begin{Phonetics}{出击}{chu1ji1}
    \definition{v.}{atacar}
  \end{Phonetics}
\end{Entry}

\begin{Entry}{出去}{5,5}{⼐、⼛}
  \begin{Phonetics}{出去}{chu1 qu4}[][HSK 1]
    \definition{v.}{sair; ir para fora;  (a partir da minha localização)}
  \end{Phonetics}
\end{Entry}

\begin{Entry}{出发}{5,5}{⼐、⼜}
  \begin{Phonetics}{出发}{chu1fa1}[][HSK 2]
    \definition{v.}{sair; partir; ir embora; deixar; sair do lugar onde se está e ir para outro lugar | começar a partir de; partir de; considerar ou tratar uma questão a partir de um determinado ponto de vista}
  \end{Phonetics}
\end{Entry}

\begin{Entry}{出台}{5,5}{⼐、⼝}
  \begin{Phonetics}{出台}{chu1 tai2}[][HSK 6]
    \definition{v.}{aparecer no palco; entrar atores no palco | aparecer publicamente; metáfora (política, plano, programa, etc.) lançar oficialmente}
  \end{Phonetics}
\end{Entry}

\begin{Entry}{出生}{5,5}{⼐、⽣}
  \begin{Phonetics}{出生}{chu1sheng1}[][HSK 2]
    \definition{v.}{nascer}
  \end{Phonetics}
\end{Entry}

\begin{Entry}{出动}{5,6}{⼐、⼒}
  \begin{Phonetics}{出动}{chu1 dong4}[][HSK 6]
    \definition{v.}{partir; começar; passear em equipe | chamar; enviar; despachar; enviar forças militares | entrar em ação; manifestar-se; tomar uma atitude}
  \end{Phonetics}
\end{Entry}

\begin{Entry}{出名}{5,6}{⼐、⼝}
  \begin{Phonetics}{出名}{chu1 ming2}[][HSK 6]
    \definition{adj.}{famoso; bem conhecido; renomado}
    \definition{v.}{tornar-se famoso (ou conhecido)  | emprestar o próprio nome (para uma ocasião ou empresa); usar o nome de}
  \end{Phonetics}
\end{Entry}

\begin{Entry}{出场}{5,6}{⼐、⼟}
  \begin{Phonetics}{出场}{chu1 chang3}[][HSK 6]
    \definition{v.}{entrar no palco; aparecer em cena; entrar atores no palco (performance) | entrar na arena ou campo de esportes; entrar atletas no estádio (para participar de uma apresentação ou competição)}
  \end{Phonetics}
\end{Entry}

\begin{Entry}{出汗}{5,6}{⼐、⽔}
  \begin{Phonetics}{出汗}{chu1 han4}[][HSK 5]
    \definition{v.}{suar; transpirar}
  \end{Phonetics}
\end{Entry}

\begin{Entry}{出色}{5,6}{⼐、⾊}
  \begin{Phonetics}{出色}{chu1se4}[][HSK 4]
    \definition{adj.}{esplêndido; extraordinário; notável; excepcionalmente bom; acima da média}
  \end{Phonetics}
\end{Entry}

\begin{Entry}{出行}{5,6}{⼐、⾏}
  \begin{Phonetics}{出行}{chu1 xing2}[][HSK 6]
    \definition{v.}{viajar; sair}
  \end{Phonetics}
\end{Entry}

\begin{Entry}{出访}{5,6}{⼐、⾔}
  \begin{Phonetics}{出访}{chu1 fang3}[][HSK 6]
    \definition{v.}{ir ao exterior em visita oficial | ir visitar em caráter oficial ou para investigação}
  \end{Phonetics}
\end{Entry}

\begin{Entry}{出来}{5,7}{⼐、⽊}
  \begin{Phonetics}{出来}{chu1 lai2}[][HSK 1]
    \definition{v.}{emergir; sair; (para a minha direção) |  surgir; aparecer; emergir | concluir ou algo acontecer}
  \end{Phonetics}
\end{Entry}

\begin{Entry}{出事}{5,8}{⼐、⼅}
  \begin{Phonetics}{出事}{chu1 shi4}[][HSK 6]
    \definition{v.}{sofrer um acidente; ocorrer um acidente ou incidente}
  \end{Phonetics}
\end{Entry}

\begin{Entry}{出国}{5,8}{⼐、⼞}
  \begin{Phonetics}{出国}{chu1 guo2}[][HSK 2]
    \definition{v.+compl.}{ir para o exterior; deixar a terra natal; viajar para o exterior}
  \end{Phonetics}
\end{Entry}

\begin{Entry}{出版}{5,8}{⼐、⽚}
  \begin{Phonetics}{出版}{chu1ban3}[][HSK 5]
    \definition{v.}{aparecer; publicar; sair; sair da imprensa}
  \end{Phonetics}
\end{Entry}

\begin{Entry}{出版社}{5,8,7}{⼐、⽚、⽰}
  \begin{Phonetics}{出版社}{chu1ban3she4}
    \definition{s.}{editora}
  \end{Phonetics}
\end{Entry}

\begin{Entry}{出现}{5,8}{⼐、⾒}
  \begin{Phonetics}{出现}{chu1xian4}[][HSK 2]
    \definition{v.}{aparecer; surgir; emergir; crescer; revelar}
  \end{Phonetics}
\end{Entry}

\begin{Entry}{出差}{5,9}{⼐、⼯}
  \begin{Phonetics}{出差}{chu1chai1}[][HSK 5]
    \definition{v.+compl.}{fazer uma viagem de negócios | assumir tarefas de curto prazo em transporte, construção, etc.}
  \end{Phonetics}
\end{Entry}

\begin{Entry}{出院}{5,9}{⼐、⾩}
  \begin{Phonetics}{出院}{chu1 yuan4}[][HSK 2]
    \definition{v.}{sair do hospital; estar fora do hospital; receber alta do hospital}
  \end{Phonetics}
\end{Entry}

\begin{Entry}{出面}{5,9}{⼐、⾯}
  \begin{Phonetics}{出面}{chu1 mian4}[][HSK 6]
    \definition{v.}{comparecer pessoalmente; agir em sua própria capacidade (ou em nome de alguém) | agir em sua própria capacidade (em nome de uma organização); apresentar-se; fazer algo individualmente ou coletivamente}
  \end{Phonetics}
\end{Entry}

\begin{Entry}{出家}{5,10}{⼐、⼧}
  \begin{Phonetics}{出家}{chu1 jia1}
    \definition{v.}{renunciar à família (para se tornar monge ou monja) (oposto a 在家) | tornar-se monge, monja ou sacerdote taoísta}
  \seealsoref{在家}{zai4 jia1}
  \end{Phonetics}
\end{Entry}

\begin{Entry}{出席}{5,10}{⼐、⼱}
  \begin{Phonetics}{出席}{chu1xi2}[][HSK 4]
    \definition{v.}{comparecer; estar presente; participar de reuniões com o direito de falar e votar; juntar-se a uma organização ou atividade}
  \end{Phonetics}
\end{Entry}

\begin{Entry}{出租}{5,10}{⼐、⽲}
  \begin{Phonetics}{出租}{chu1 zu1}[][HSK 2]
    \definition{s.}{taxi; abreviação de 出租车}
    \definition{v.}{alugar; arrendar; receber dinheiro de outras pessoas para permitir que elas utilizem algo (como uma casa, um carro, livros, etc.) por um determinado período de tempo}
  \seealsoref{出租车}{chu1zu1che1}
  \end{Phonetics}
\end{Entry}

\begin{Entry}{出租车}{5,10,4}{⼐、⽲、⾞}
  \begin{Phonetics}{出租车}{chu1zu1che1}[][HSK 2]
    \definition[辆]{s.}{táxi; carro de aluguel; veículos de transporte urbano disponíveis para aluguel, com cobrança por quilometragem ou tempo}
  \seealsoref{出租汽车}{chu1zu1qi4che1}
  \end{Phonetics}
\end{Entry}

\begin{Entry}{出租司机}{5,10,5,6}{⼐、⽲、⼝、⽊}
  \begin{Phonetics}{出租司机}{chu1zu1si1ji1}
    \definition{s.}{motorista de táxi}
  \end{Phonetics}
\end{Entry}

\begin{Entry}{出租汽车}{5,10,7,4}{⼐、⽲、⽔、⾞}
  \begin{Phonetics}{出租汽车}{chu1zu1qi4che1}
    \definition[辆]{s.}{táxi}
  \seealsoref{出租车}{chu1zu1che1}
  \end{Phonetics}
\end{Entry}

\begin{Entry}{出站}{5,10}{⼐、⽴}
  \begin{Phonetics}{出站}{chu1 zhan4}
    \definition{s.}{saída da estação}
  \end{Phonetics}
\end{Entry}

\begin{Entry}{出售}{5,11}{⼐、⼝}
  \begin{Phonetics}{出售}{chu1 shou4}[][HSK 4]
    \definition{v.}{vender; oferecer para venda}
  \end{Phonetics}
\end{Entry}

\begin{Entry}{出路}{5,13}{⼐、⾜}
  \begin{Phonetics}{出路}{chu1lu4}[][HSK 6]
    \definition[条,个]{s.}{saída; futuro; uma maneira de manter a sobrevivência ou progredir intermitentemente; também pode se referir ao futuro | saída; formas de vender produtos | saída; um caminho que leva para fora ou para frente}
  \end{Phonetics}
\end{Entry}

\begin{Entry}{功}{5}{⼒}
  \begin{Phonetics}{功}{gong1}
    \definition*{s.}{Sobrenome Gong}
    \definition[次,大]{s.}{mérito; façanha; serviço meritório (ação) | resultado; eficácia; realização | habilidade; habilidade técnica; tecnologia e qualificação técnica | trabalho; uma força faz com que um objeto se desloque uma certa distância na direção da força}
  \end{Phonetics}
\end{Entry}

\begin{Entry}{功夫}{5,4}{⼒、⼤}
  \begin{Phonetics}{功夫}{gong1fu5}[][HSK 3]
    \definition*{s.}{Gongfu (Kung Fu), arte marcial}
    \definition[番]{s.}{habilidade; destreza; conhecimento | luta acrobática; habilidade em artes marciais | esforço; tempo e energia}
  \end{Phonetics}
\end{Entry}

\begin{Entry}{功臣}{5,6}{⼒、⾂}
  \begin{Phonetics}{功臣}{gong1chen2}
    \definition{s.}{oficial meritório | pessoa que presta serviço excepcional, herói | (fig.) alguém que desempenha um papel vital}
  \end{Phonetics}
\end{Entry}

\begin{Entry}{功能}{5,10}{⼒、⾁}
  \begin{Phonetics}{功能}{gong1neng2}[][HSK 3]
    \definition[种,项]{s.}{função; os efeitos positivos produzidos por coisas ou métodos}
  \end{Phonetics}
\end{Entry}

\begin{Entry}{功课}{5,10}{⼒、⾔}
  \begin{Phonetics}{功课}{gong1 ke4}[][HSK 3]
    \definition[份,门]{s.}{trabalho escolar; dever de casa; refere-se aos trabalhos de casa atribuídos pelos professores aos alunos| tarefa; lições; lição escolar | preparações; preparação necessária antes de fazer algo}
  \end{Phonetics}
\end{Entry}

\begin{Entry}{加}{5}{⼒}
  \begin{Phonetics}{加}{jia1}[][HSK 2]
    \definition*{s.}{Canadá, abreviação de 加拿大 | Sobrenome Jia}
    \definition{v.}{adicionar; somar | aumentar; incrementar; aumentar a quantidade ou o grau em relação ao original | inserir; adicionar; anexar; adicionar o que não existe; colocar no lugar | acrescentar; indica a realização de uma determinada ação | colocar uma coisa em cima da outra | impor ou aplicar algo a outra pessoa; atribuir um determinado comportamento a outra pessoa}
  \seealsoref{加拿大}{jia1na2da4}
  \end{Phonetics}
\end{Entry}

\begin{Entry}{加入}{5,2}{⼒、⼊}
  \begin{Phonetics}{加入}{jia1ru4}[][HSK 4]
    \definition{v.}{juntar-se; unir-se; aderir a; tornar-se um membro de uma organização, grupo | adicionar; colocar em}
  \end{Phonetics}
\end{Entry}

\begin{Entry}{加上}{5,3}{⼒、⼀}
  \begin{Phonetics}{加上}{jia1 shang4}[][HSK 5]
    \definition{conj.}{além disso; em adição}
    \definition{v.}{adicionar; acrescentar; dar; aumentar}
  \end{Phonetics}
\end{Entry}

\begin{Entry}{加工}{5,3}{⼒、⼯}
  \begin{Phonetics}{加工}{jia1gong1}[][HSK 3]
    \definition{s.}{processo | trabalho (de uma máquina)}
    \definition{v.}{processar; realizar diversos trabalhos em matérias-primas e produtos semiacabados (como alterar dimensões, formas, propriedades, aumentar a precisão, pureza, etc.) para que atendam aos requisitos especificados | melhorar; polir; refere-se a todos os tipos de trabalho que tornam o produto final mais perfeito e refinado}
  \end{Phonetics}
\end{Entry}

\begin{Entry}{加以}{5,4}{⼒、⼈}
  \begin{Phonetics}{加以}{jia1 yi3}[][HSK 5]
    \definition{conj.}{além disso; em adição; indica outras razões ou condições}
    \definition{v.aux.}{usado na frente de palavras dissilábicas para indicar como um objeto mencionado deve ser tratado ou descartado | usado antes de um verbo polifônico ou de um substantivo formado a partir de um verbo para indicar como tratar ou lidar com o que foi mencionado anteriormente}
  \end{Phonetics}
\end{Entry}

\begin{Entry}{加快}{5,7}{⼒、⼼}
  \begin{Phonetics}{加快}{jia1 kuai4}[][HSK 3]
    \definition{v.}{acelerar; aumentar a velocidade; agilizar}
  \end{Phonetics}
\end{Entry}

\begin{Entry}{加油}{5,8}{⼒、⽔}
  \begin{Phonetics}{加油}{jia1 you2}[][HSK 2]
    \definition{v.+compl.}{abastecer com óleo; reabastecer; adicionar combustível ou óleo lubrificante | fazer um esforço extra; dar o máximo; (Vamos lá!) metáfora para se esforçar ainda mais}
  \end{Phonetics}
\end{Entry}

\begin{Entry}{加油工}{5,8,3}{⼒、⽔、⼯}
  \begin{Phonetics}{加油工}{jia1 you2 gong1}[][HSK 6]
    \definition{s.}{frentista}
  \end{Phonetics}
\end{Entry}

\begin{Entry}{加油站}{5,8,10}{⼒、⽔、⽴}
  \begin{Phonetics}{加油站}{jia1you2zhan4}[][HSK 4]
    \definition[个,座,家]{s.}{posto de gasolina; posto de combustível; postos de abastecimento para venda a varejo de gasolina e óleo para carros e outros veículos motorizados}
  \end{Phonetics}
\end{Entry}

\begin{Entry}{加拿大}{5,10,3}{⼒、⼿、⼤}
  \begin{Phonetics}{加拿大}{jia1na2da4}
    \definition{s.}{Canadá}
  \end{Phonetics}
\end{Entry}

\begin{Entry}{加拿大人}{5,10,3,2}{⼒、⼿、⼤、⼈}
  \begin{Phonetics}{加拿大人}{jia1na2da4ren2}
    \definition{s.}{canadense | pessoa ou povo do Canadá}
  \end{Phonetics}
\end{Entry}

\begin{Entry}{加热}{5,10}{⼒、⽕}
  \begin{Phonetics}{加热}{jia1 re4}[][HSK 5]
    \definition{v.}{aquecer; esquentar; aumentar a temperatura de um objeto}
  \end{Phonetics}
\end{Entry}

\begin{Entry}{加班}{5,10}{⼒、⽟}
  \begin{Phonetics}{加班}{jia1ban1}[][HSK 4]
    \definition{v.+compl.}{fazer horas extras; trabalhar horas extras; aumentar o horário de trabalho ou os turnos além do limite de tempo prescrito}
  \end{Phonetics}
\end{Entry}

\begin{Entry}{加速}{5,10}{⼒、⾡}
  \begin{Phonetics}{加速}{jia1 su4}[][HSK 5]
    \definition{v.}{acelerar; agilizar}
  \end{Phonetics}
\end{Entry}

\begin{Entry}{加速度}{5,10,9}{⼒、⾡、⼴}
  \begin{Phonetics}{加速度}{jia1su4du4}
    \definition{s.}{aceleração}
  \end{Phonetics}
\end{Entry}

\begin{Entry}{加强}{5,12}{⼒、⼸}
  \begin{Phonetics}{加强}{jia1 qiang2}[][HSK 3]
    \definition{v.}{fortalecer; engrandecer; reforçar; tornar mais forte ou mais eficaz}
  \end{Phonetics}
\end{Entry}

\begin{Entry}{加盟}{5,13}{⼒、⽫}
  \begin{Phonetics}{加盟}{jia1 meng2}[][HSK 6]
    \definition{v.}{aliar-se a; filiar-se a um sindicato; juntar-se a um grupo ou organização}
  \end{Phonetics}
\end{Entry}

\begin{Entry}{务}{5}{⼒}
  \begin{Phonetics}{务}{wu4}
    \definition*{s.}{Sobrenome Wu}
    \definition{s.}{caso; negócio | usado em nomes de lugares}
    \definition{v.}{engajar-se em; dedicar seus esforços a | procurar; perseguir; ir atrás | estar envolvido em; dedicar-se a; envolver-se em; comprometer-se com | deve; deveria; ter certeza de}
  \end{Phonetics}
\end{Entry}

\begin{Entry}{务必}{5,5}{⼒、⼼}
  \begin{Phonetics}{务必}{wu4bi4}
    \definition{adv.}{deve; ter certeza de; necessariamente; usado principalmente em frases afirmativas}
  \end{Phonetics}
\end{Entry}

\begin{Entry}{务实}{5,8}{⼒、⼧}
  \begin{Phonetics}{务实}{wu4shi2}
    \definition{adj.}{pragmático; prático; pé-no-chão}
    \definition{v.}{lidar com assuntos concretos; discutir e estudar questões específicas; envolver-se em trabalho específico}
  \end{Phonetics}
\end{Entry}

\begin{Entry}{包}{5}{⼓}
  \begin{Phonetics}{包}{bao1}[][HSK 1]
    \definition*{s.}{Sobrenome Bao}
    \definition{clas.}{pacote; embalagem; embrulho; usado para coisas empacotadas}
    \definition[个,只]{s.}{feixe; pacote; encomenda; algo embrulhado | saco; sacola; saco para guardar coisas | caroço; inchaço; protuberância; inchaço ou protuberância no corpo ou em objetos | tenda; tenda com cúpula feita de feltro}
    \definition{v.}{embrulhar; envolver com papel, tecido, etc. | cercar; rodear; envolver; envelopar | incluir; conter | realizar todo o processo; assumir toda a responsabilidade | assegurar; garantir | contratar; reservar; fretar; comprar ou alugar tudo; acordar uso exclusivo}
  \end{Phonetics}
\end{Entry}

\begin{Entry}{包子}{5,3}{⼓、⼦}
  \begin{Phonetics}{包子}{bao1 zi5}[][HSK 1]
    \definition[个]{s.}{pão recheado cozido no vapor; alimentos, com recheio de vegetais, carne ou açúcar, etc., com massa levedada como invólucro, embrulhados e cozidos no vapor}
  \end{Phonetics}
\end{Entry}

\begin{Entry}{包干}{5,3}{⼓、⼲}
  \begin{Phonetics}{包干}{bao1gan1}
    \definition{s.}{tarefa alocada}
    \definition{v.}{ter a responsabilidade total sobre um trabalho}
  \end{Phonetics}
\end{Entry}

\begin{Entry}{包办}{5,4}{⼓、⼒}
  \begin{Phonetics}{包办}{bao1ban4}
    \definition{v.}{cuidar de tudo que diz respeito a um trabalho | comandar todo o espetáculo; monopolizar tudo | assumir tudo; manter tudo em suas próprias mãos}
  \end{Phonetics}
\end{Entry}

\begin{Entry}{包含}{5,7}{⼓、⼝}
  \begin{Phonetics}{包含}{bao1han2}[][HSK 4]
    \definition{v.}{conter; implicar; incluir; conter dentro, resumir, enfatizar o que está contido dentro, focar em relações internas, muitas vezes coisas abstratas}
  \end{Phonetics}
\end{Entry}

\begin{Entry}{包围}{5,7}{⼓、⼞}
  \begin{Phonetics}{包围}{bao1wei2}[][HSK 5]
    \definition{v.}{circundar; cercar; rodear}
  \end{Phonetics}
\end{Entry}

\begin{Entry}{包括}{5,9}{⼓、⼿}
  \begin{Phonetics}{包括}{bao1kuo4}[][HSK 4]
    \definition{v.}{incluir; compreender; consistir em; conter, conter dentro, resumir junto, enfatizar a listagem de todas as partes, ou a citação de uma parte delas, que podem ser coisas abstratas ou concretas}
  \end{Phonetics}
\end{Entry}

\begin{Entry}{包容}{5,10}{⼓、⼧}
  \begin{Phonetics}{包容}{bao1rong2}
    \definition{adj.}{inclusivo}
    \definition{v.}{perdoar | mostrar tolerância | conter | segurar}
  \end{Phonetics}
\end{Entry}

\begin{Entry}{包租}{5,10}{⼓、⽲}
  \begin{Phonetics}{包租}{bao1zu1}
    \definition{s.}{aluguel fixo para terras agrícolas}
    \definition{v.}{fretar | alugar | alugar um terreno ou uma casa para subarrendar}
  \end{Phonetics}
\end{Entry}

\begin{Entry}{包装}{5,12}{⼓、⾐}
  \begin{Phonetics}{包装}{bao1zhuang1}[][HSK 5]
    \definition[个,款]{s.}{embalagem; materiais usados para embalar produtos, como papel, sacolas, garrafas ou caixas}
    \definition{v.}{embalar; embrulhar; empacotar | aumentar a fama e o apelo de alguém ou algo por meio de publicidade | tornar alguém ou algo mais comercialmente viável ou atraente por meio de embelezamento ou publicidade}
  \end{Phonetics}
\end{Entry}

\begin{Entry}{包裹}{5,14}{⼓、⾐}
  \begin{Phonetics}{包裹}{bao1guo3}[][HSK 4]
    \definition[个,件]{s.}{pacote; embrulho}
    \definition{v.}{embrulhar; amarrar; enrolar coisas em pano ou outra coisa}
  \end{Phonetics}
\end{Entry}

\begin{Entry}{匆}{5}{⼓}
  \begin{Phonetics}{匆}{cong1}
    \definition{adj.}{apressado}
    \definition{adv.}{apressadamente}
  \end{Phonetics}
\end{Entry}

\begin{Entry}{匆匆}{5,5}{⼓、⼓}
  \begin{Phonetics}{匆匆}{cong1cong1}
    \definition{adv.}{apressadamente}
  \end{Phonetics}
\end{Entry}

\begin{Entry}{北}{5}{⼔}
  \begin{Phonetics}{北}{bei3}[][HSK 1]
    \definition*{s.}{Norte (os países desenvolvidos) | Sobrenome Bei}
    \definition{s.}{norte; uma das quatro direções básicas, a esquerda quando se está de frente para o sol pela manhã (oposta ao 南)}
    \definition{v.}{ser derrotado}
  \seealsoref{南}{nan2}
  \end{Phonetics}
\end{Entry}

\begin{Entry}{北大西洋公约组织}{5,3,6,9,4,6,8,8}{⼔、⼤、⾑、⽔、⼋、⽷、⽷、⽷}
  \begin{Phonetics}{北大西洋公约组织}{bei3 da4xi1 yang2 gong1 yue1 zu3zhi1}
    \definition*{s.}{Organização do Tratado do Atlântico Norte, OTAN}
  \end{Phonetics}
\end{Entry}

\begin{Entry}{北方}{5,4}{⼔、⽅}
  \begin{Phonetics}{北方}{bei3fang1}[][HSK 2]
    \definition{s.}{norte; indicando a direção norte | o Norte; a parte norte da China, especialmente a área ao norte do rio Huang He}
  \end{Phonetics}
\end{Entry}

\begin{Entry}{北边}{5,5}{⼔、⾡}
  \begin{Phonetics}{北边}{bei3 bian1}[][HSK 1]
    \definition{s.}{norte; o lado norte}
  \end{Phonetics}
\end{Entry}

\begin{Entry}{北约}{5,6}{⼔、⽷}
  \begin{Phonetics}{北约}{bei3yue1}
    \definition*{s.}{OTAN, Organização do Tratado do Atlântico Norte; Abreviação de 北大西洋公约组织}
  \seealsoref{北大西洋公约组织}{bei3 da4xi1 yang2 gong1 yue1 zu3zhi1}
  \end{Phonetics}
\end{Entry}

\begin{Entry}{北极}{5,7}{⼔、⽊}
  \begin{Phonetics}{北极}{bei3ji2}[][HSK 5]
    \definition*{s.}{Polo Norte; Polo Ártico}
    \definition{s.}{polo norte magnético; o ponto mais setentrional da Terra, também se refere à região mais setentrional da Terra}
  \end{Phonetics}
\end{Entry}

\begin{Entry}{北京}{5,8}{⼔、⼇}
  \begin{Phonetics}{北京}{bei3 jing1}[][HSK 1]
    \definition*{s.}{Pequim (Beijing), Capital da República Popular da China | Capital da China, localizada no nordeste do país, fundada em 700 a.C., a cidade é um importante centro comercial, industrial e cultural}
  \end{Phonetics}
\end{Entry}

\begin{Entry}{北面}{5,9}{⼔、⾯}
  \begin{Phonetics}{北面}{bei3mian4}
    \definition{s.}{norte; o lado norte}
  \end{Phonetics}
\end{Entry}

\begin{Entry}{北部}{5,10}{⼔、⾢}
  \begin{Phonetics}{北部}{bei3 bu4}[][HSK 3]
    \definition{s.}{parte norte de uma região ou país}
  \end{Phonetics}
\end{Entry}

\begin{Entry}{半}{5}{⼗}
  \begin{Phonetics}{半}{ban4}[][HSK 1]
    \definition{adv.}{parcialmente; usado antes de verbos ou adjetivos para indicar incompletude}
    \definition{num.}{(depois de um número) ``e meio'' | meio; metade | na metade; no meio | muito pouco; o mínimo}
  \end{Phonetics}
\end{Entry}

\begin{Entry}{半天}{5,4}{⼗、⼤}
  \begin{Phonetics}{半天}{ban4 tian1}[][HSK 1]
    \definition{s.}{metade do dia; metade do dia dividida pelo meio-dia | um longo tempo; bastante tempo; refere-se a um período de tempo relativamente longo (com um tom exagerado)}
  \end{Phonetics}
\end{Entry}

\begin{Entry}{半决赛}{5,6,14}{⼗、⼎、⾙}
  \begin{Phonetics}{半决赛}{ban4 jue2 sai4}[][HSK 6]
    \definition{s.}{semifinais}
  \end{Phonetics}
\end{Entry}

\begin{Entry}{半年}{5,6}{⼗、⼲}
  \begin{Phonetics}{半年}{ban4 nian2}[][HSK 1]
    \definition{s.}{meio ano}
  \end{Phonetics}
\end{Entry}

\begin{Entry}{半夜}{5,8}{⼗、⼣}
  \begin{Phonetics}{半夜}{ban4 ye4}[][HSK 2]
    \definition{s.}{no meio da noite; metade da noite | por volta da meia-noite, também se refere à madrugada}
  \end{Phonetics}
\end{Entry}

\begin{Entry}{半音}{5,9}{⼗、⾳}
  \begin{Phonetics}{半音}{ban4yin1}
    \definition{s.}{semitom; na música, uma oitava é dividida em doze notas e o intervalo entre duas notas adjacentes é chamado de semitom}
  \end{Phonetics}
\end{Entry}

\begin{Entry}{半球}{5,11}{⼗、⽟}
  \begin{Phonetics}{半球}{ban4qiu2}
    \definition{s.}{hemisfério}
  \end{Phonetics}
\end{Entry}

\begin{Entry}{占}{5}{⼘}
  \begin{Phonetics}{占}{zhan1}
    \definition*{s.}{Sobrenome Zhan}
    \definition{v.}{praticar adivinhação; antigamente, as pessoas usavam cascos de tartaruga e mil-folhas para prever boa ou má sorte; mais tarde, a palavra passou a se referir à previsão de boa ou má sorte por vários meios}
  \end{Phonetics}
  \begin{Phonetics}{占}{zhan4}[][HSK 2]
    \definition{v.}{tomar; apreender; ocupar; obter e possuir (terra, lugar, etc.) pela força ou outros meios impróprios | manter; inventar; constituir; explicar; estar em (uma certa posição); pertencer a (uma certa situação) | usar; ocupar; tomar; possuir}
  \end{Phonetics}
\end{Entry}

\begin{Entry}{占有}{5,6}{⼘、⽉}
  \begin{Phonetics}{占有}{zhan4 you3}[][HSK 5]
    \definition{v.}{possuir; ter; ocupar e possuir | manter; ocupar; estar em (uma determinada posição) | possuir; deter; ter; dominar}
  \end{Phonetics}
\end{Entry}

\begin{Entry}{占据}{5,11}{⼘、⼿}
  \begin{Phonetics}{占据}{zhan4 ju4}[][HSK 6]
    \definition{v.}{segurar; ocupar; assumir; tomar ou ocupar à força (uma região, lugar, etc.)}
  \end{Phonetics}
\end{Entry}

\begin{Entry}{占领}{5,11}{⼘、⾴}
  \begin{Phonetics}{占领}{zhan4ling3}[][HSK 5]
    \definition{v.}{manter; tomar; ocupar; capturar; conquistar (posições ou territórios) com forças armadas | ocupar; capturar; possuir}
  \end{Phonetics}
\end{Entry}

\begin{Entry}{卡}{5}{⼘}
  \begin{Phonetics}{卡}{ka3}[][HSK 2]
    \definition{clas.}{calorias (cal)}
    \definition[张,片]{s.}{cartão; documento semelhante a um cartão | cassete; dispositivo tipo compartimento para colocar fitas cassete no gravador | caminhão}
  \end{Phonetics}
  \begin{Phonetics}{卡}{qia3}
    \definition*{s.}{Sobrenome Qia}
    \definition[张,片]{s.}{clipe; prendedor; pinça; utensílio para prender objetos | posto de controle; posto de guarda ou posto de controle localizado em vias de comunicação importantes ou em locais com terreno acidentado}
    \definition{v.}{encravar; ficar preso; impedir de se mover | parar; controlar; impedir | pressionar firmemente com a palma da mão}
  \end{Phonetics}
\end{Entry}

\begin{Entry}{卡片}{5,4}{⼘、⽚}
  \begin{Phonetics}{卡片}{ka3pian4}
    \definition{s.}{cartão}
  \end{Phonetics}
\end{Entry}

\begin{Entry}{卡片游戏}{5,4,12,6}{⼘、⽚、⽔、⼽}
  \begin{Phonetics}{卡片游戏}{ka3pian4 you2xi4}
    \definition{s.}{carta de baralho}
  \end{Phonetics}
\end{Entry}

\begin{Entry}{卡车司机}{5,4,5,6}{⼘、⾞、⼝、⽊}
  \begin{Phonetics}{卡车司机}{ka3che1 si1ji1}
    \definition{s.}{motorista de caminhão}
  \end{Phonetics}
\end{Entry}

\begin{Entry}{卡通}{5,10}{⼘、⾡}
  \begin{Phonetics}{卡通}{ka3tong1}
    \definition{s.}{(empréstimo linguístico) \emph{cartoon}}
  \end{Phonetics}
\end{Entry}

\begin{Entry}{卢}{5}{⼘}
  \begin{Phonetics}{卢}{lu2}
    \definition*{s.}{Luxemburgo, abreviação de 卢森堡 | Sobrenome Lu}
    \definition{s.}{Aarcaico: preta (cor)}
  \seealsoref{卢森堡}{lu2sen1bao3}
  \end{Phonetics}
\end{Entry}

\begin{Entry}{卢旺达}{5,8,6}{⼘、⽇、⾡}
  \begin{Phonetics}{卢旺达}{lu2wang4da2}
    \definition*{s.}{Ruanda}
  \end{Phonetics}
\end{Entry}

\begin{Entry}{卢森堡}{5,12,12}{⼘、⽊、⼟}
  \begin{Phonetics}{卢森堡}{lu2sen1bao3}
    \definition*{s.}{Luxemburgo}
  \end{Phonetics}
\end{Entry}

\begin{Entry}{印}{5}{⼙}
  \begin{Phonetics}{印}{yin4}[][HSK 6]
    \definition*{s.}{Sobrenome Yin}
    \definition[个,枚,道,条]{s.}{selo; lacre; carimbo; estampilha | marca; estampa; impressão}
    \definition{v.}{imprimir; gravar | corresponder; conformar; estar em conformidade com}
  \end{Phonetics}
\end{Entry}

\begin{Entry}{印刷}{5,8}{⼙、⼑}
  \begin{Phonetics}{印刷}{yin4shua1}[][HSK 5]
    \definition{v.}{imprimir; imprimir textos, imagens, etc. em papel}
  \end{Phonetics}
\end{Entry}

\begin{Entry}{印象}{5,11}{⼙、⾗}
  \begin{Phonetics}{印象}{yin4xiang4}[][HSK 3]
    \definition[种]{s.}{impressão; marca; ideia; os vestígios deixados por coisas objetivas na mente das pessoas}
  \end{Phonetics}
\end{Entry}

\begin{Entry}{厉}{5}{⼚}
  \begin{Phonetics}{厉}{li4}
    \definition*{s.}{Sobrenome Li}
    \definition{adj.}{rigoroso; estrito | severo; sombrio; sério}
  \end{Phonetics}
\end{Entry}

\begin{Entry}{厉害}{5,10}{⼚、⼧}
  \begin{Phonetics}{厉害}{li4hai5}[][HSK 5]
    \definition{adj.}{feroz; severo; descreve uma situação como sendo muito grave | severo; duro; descreve uma pessoa que é exigente com os outros, muito severa, muitas vezes deixando os outros um pouco assustados | incrível; talentoso; impressionante; usado para avaliar a capacidade de uma pessoa ou algo que ela fez que é notável | aterrorizante; assustador; descreve animais ferozes e assustadores}
  \end{Phonetics}
\end{Entry}

\begin{Entry}{厺}{5}{⼤}
  \begin{Phonetics}{厺}{qu4}
    \variantof{去}
  \end{Phonetics}
\end{Entry}

\begin{Entry}{去}{5}{⼛}
  \begin{Phonetics}{去}{qu4}[][HSK 1]
    \definition{adj.}{passado; último; refere-se ao tempo passado (um ano)}
    \definition{adv.}{muito; extremamente; usado depois de adjetivos como 大, 多 e 远, significa 极 ou 非常}
    \definition{s.}{tom descendente, um dos quatro tons do chinês clássico e o quarto tom na pronúncia padrão do chinês moderno}
    \definition{v.}{ir; partir; sair | estar separado de | perder | remover; livrar-se de | ir (a algum lugar) para fazer algo; sair do local onde o interlocutor se encontra para outro lugar (oposto a 来) | ir para; estar indo para (fazer algo lá); usado antes de outro verbo para indicar fazer algo | desempenhar o papel de; representar o papel de; interpretar papéis em óperas | enviar; fazer ir; despachar}
    \definition{v.aux.}{usado entre uma frase verbal (ou frase preposicional) e um verbo para indicar que o primeiro é um método ou atitude e o último é um propósito | usado depois de um verbo para indicar que a ação está longe da localização do falante}
  \seealsoref{大}{da4}
  \seealsoref{多}{duo1}
  \seealsoref{非常}{fei1chang2}
  \seealsoref{极}{ji2}
  \seealsoref{来}{lai2}
  \seealsoref{远}{yuan3}
  \end{Phonetics}
\end{Entry}

\begin{Entry}{去世}{5,5}{⼛、⼀}
  \begin{Phonetics}{去世}{qu4shi4}[][HSK 3]
    \definition{v.}{(usado apenas para adultos, com conotações solenes) morrer; falecer; deixar este mundo}
  \end{Phonetics}
\end{Entry}

\begin{Entry}{去年}{5,6}{⼛、⼲}
  \begin{Phonetics}{去年}{qu4nian2}[][HSK 1]
    \definition{s.}{ano passado}
  \end{Phonetics}
\end{Entry}

\begin{Entry}{去死}{5,6}{⼛、⽍}
  \begin{Phonetics}{去死}{qu4si3}
    \definition{interj.}{Caia morto! | Vá para o Inferno!}
  \end{Phonetics}
\end{Entry}

\begin{Entry}{去掉}{5,11}{⼛、⼿}
  \begin{Phonetics}{去掉}{qu4 diao4}[][HSK 6]
    \definition{v.}{livrar-se de; tirar; acabar com; abandonar; erradicar}
  \end{Phonetics}
\end{Entry}

\begin{Entry}{发}{5}{⼜}
  \begin{Phonetics}{发}{fa1}[][HSK 2]
    \definition*{s.}{Sobrenome Fa}
    \definition{clas.}{bala, usada para munições e cartuchos}
    \definition{v.}{distribuir; enviar; entregar | emitir; disparar; lançar; descarregar | produzir; gerar; criar (dar origem a) | proferir; emitir; expressar | expandir; desenvolver | prosperar; prosperidade graças à aquisição de bens materiais | crescer ou expandir quando fermentado ou embebido | difundir; dispersar; espalhar | expor; descobrir; revelar | transformar-se; tornar-se; entrar em um determinado estado | demonstrar seus sentimentos; expressar (sentimentos) | sentir; ter um sentimento | começar; estabelecer | fazer com que se faça; iniciar um empreendimento; começar a agir; provocar uma ação}
  \end{Phonetics}
  \begin{Phonetics}{发}{fa4}
    \definition*{s.}{Sobrenome Fa}
    \definition[件]{s.}{cabelo}
  \end{Phonetics}
\end{Entry}

\begin{Entry}{发出}{5,5}{⼜、⼐}
  \begin{Phonetics}{发出}{fa1 chu1}[][HSK 3]
    \definition{v.}{fazer; produzir; deixar sair; ocorrer (som, dúvida, etc.) | emitir; anunciar; publicar; divulgar (ordens, instruções) | enviar (mercadorias, cartas, etc.); partir (veículos, etc.) | emitir; exalar (cheiro, calor, etc.)}
  \end{Phonetics}
\end{Entry}

\begin{Entry}{发布}{5,5}{⼜、⼱}
  \begin{Phonetics}{发布}{fa1bu4}[][HSK 5]
    \definition{v.}{emitir; publicar; liberar; anunciar; fazer ordens públicas, anúncios, notícias, etc.}
  \end{Phonetics}
\end{Entry}

\begin{Entry}{发生}{5,5}{⼜、⽣}
  \begin{Phonetics}{发生}{fa1sheng1}[][HSK 3]
    \definition{v.}{ocorrer; acontecer; tomar lugar; surgir algo que não existia antes}
  \end{Phonetics}
\end{Entry}

\begin{Entry}{发电}{5,5}{⼜、⽥}
  \begin{Phonetics}{发电}{fa1 dian4}[][HSK 6]
    \definition{s.}{geração de energia elétrica; produção de eletricidade; fornecimento de energia}
    \definition{v.}{gerar eletricidade (ou energia elétrica) | enviar um telegrama}
  \end{Phonetics}
\end{Entry}

\begin{Entry}{发动}{5,6}{⼜、⼒}
  \begin{Phonetics}{发动}{fa1dong4}[][HSK 3]
    \definition{v.}{iniciar; começar; lançar | chamar à ação; mobilizar; despertar | ligar o motor; dar a partida; dar o pontapé inicial (motor de combustão interna) | estimular; colocar em ação}
  \end{Phonetics}
\end{Entry}

\begin{Entry}{发动机}{5,6,6}{⼜、⼒、⽊}
  \begin{Phonetics}{发动机}{fa1dong4ji1}
    \definition[台]{s.}{motor}
  \end{Phonetics}
\end{Entry}

\begin{Entry}{发行}{5,6}{⼜、⾏}
  \begin{Phonetics}{发行}{fa1xing2}[][HSK 5]
    \definition{v.}{emitir; liberar; publicar; emitir ou vender de publicações recém-impressas, moeda, selos, etc.}
  \end{Phonetics}
\end{Entry}

\begin{Entry}{发达}{5,6}{⼜、⾡}
  \begin{Phonetics}{发达}{fa1da2}[][HSK 3]
    \definition{adj.}{desenvolvido; florescente; (coisas) Já estão bem desenvolvidas; (negócios) prosperam}
    \definition{v.}{desenvolver; promover; florescer; a pessoa tem um bom desempenho profissional e é muito bem-sucedida}
  \end{Phonetics}
\end{Entry}

\begin{Entry}{发抖}{5,7}{⼜、⼿}
  \begin{Phonetics}{发抖}{fa1dou3}
    \definition{v.}{tremer | sacudir | estremecer}
  \end{Phonetics}
\end{Entry}

\begin{Entry}{发言}{5,7}{⼜、⾔}
  \begin{Phonetics}{发言}{fa1yan2}[][HSK 3]
    \definition[个]{s.}{discurso; declaração; palestra; opiniões publicadas}
    \definition{v.+compl.}{falar; fazer uma declaração (discurso); expressar opinião (geralmente em reuniões)}
  \end{Phonetics}
\end{Entry}

\begin{Entry}{发言人}{5,7,2}{⼜、⾔、⼈}
  \begin{Phonetics}{发言人}{fa1 yan2 ren2}[][HSK 6]
    \definition{s.}{porta-voz}
  \end{Phonetics}
\end{Entry}

\begin{Entry}{发财}{5,7}{⼜、⾙}
  \begin{Phonetics}{发财}{fa1cai2}
    \definition{v.+compl.}{ficar rico | fazer fortuna}
  \end{Phonetics}
\end{Entry}

\begin{Entry}{发放}{5,8}{⼜、⽅}
  \begin{Phonetics}{发放}{fa1 fang4}[][HSK 6]
    \definition{v.}{conceder; estender; fornecer; (governo, organização) distribuir dinheiro ou suprimentos para os necessitados | emitir; enviar}
  \end{Phonetics}
\end{Entry}

\begin{Entry}{发明}{5,8}{⼜、⽇}
  \begin{Phonetics}{发明}{fa1ming2}[][HSK 3]
    \definition[个,项,种]{s.}{invenção; novos produtos ou métodos inventados}
    \definition{v.}{inventar; pesquisa que cria (novos produtos ou novos métodos) | expor; explicar; explicação criativa}
  \end{Phonetics}
\end{Entry}

\begin{Entry}{发明者}{5,8,8}{⼜、⽇、⽼}
  \begin{Phonetics}{发明者}{fa1ming2zhe3}
    \definition{s.}{inventor}
  \end{Phonetics}
\end{Entry}

\begin{Entry}{发泄}{5,8}{⼜、⽔}
  \begin{Phonetics}{发泄}{fa1xie4}
    \definition{v.}{soltar; abreviar; dar vazão a; desabafar emoções ou desejos}
  \end{Phonetics}
\end{Entry}

\begin{Entry}{发炎}{5,8}{⼜、⽕}
  \begin{Phonetics}{发炎}{fa1yan2}[][HSK 6]
    \definition{s.}{inflamação}
    \definition{v.}{irritar; inflamar; reação complexa de organismos a fatores patogênicos, como microrganismos, substâncias químicas e estímulos físicos; os sintomas sistêmicos incluem aumento da temperatura corporal, alterações na composição do sangue, vermelhidão local, inchaço, febre, dor, etc.}
  \end{Phonetics}
\end{Entry}

\begin{Entry}{发现}{5,8}{⼜、⾒}
  \begin{Phonetics}{发现}{fa1xian4}[][HSK 2]
    \definition[个,项]{s.}{descoberta; achado}
    \definition{v.}{encontrar; descobrir; detectar; identificar; através de pesquisa, exploração, etc., ver ou encontrar coisas ou leis que os antepassados não viram | descobrir; perceber; perceber; notar; estar ciente de}
  \end{Phonetics}
\end{Entry}

\begin{Entry}{发现者}{5,8,8}{⼜、⾒、⽼}
  \begin{Phonetics}{发现者}{fa1xian4 zhe3}
    \definition{s.}{descobridor}
  \end{Phonetics}
\end{Entry}

\begin{Entry}{发表}{5,8}{⼜、⾐}
  \begin{Phonetics}{发表}{fa1biao3}[][HSK 3]
    \definition{v.}{publicar; entregar; emitir; expressar; anunciar; expressar (opiniões) ou divulgar (assuntos) ao público, verbalmente ou por escrito | publicar em jornais (artigos, etc.)}
  \end{Phonetics}
\end{Entry}

\begin{Entry}{发型}{5,9}{⼜、⼟}
  \begin{Phonetics}{发型}{fa4xing2}
    \definition{s.}{penteado}
  \end{Phonetics}
\end{Entry}

\begin{Entry}{发怒}{5,9}{⼜、⼼}
  \begin{Phonetics}{发怒}{fa1 nu4}[][HSK 6]
    \definition{v.}{ficar com raiva; explodir; perder a paciência | entrar em fúria | entrar em fúria (paixão)}
  \end{Phonetics}
\end{Entry}

\begin{Entry}{发挥}{5,9}{⼜、⼿}
  \begin{Phonetics}{发挥}{fa1hui1}[][HSK 4]
    \definition{v.}{colocar em jogo; dar jogo a; dar espaço a; dar rédea solta a; revelar a natureza ou a capacidade interior | expressar; desenvolver (uma ideia, um tema, etc.); elaborar; fazer valer o ponto ou o motivo}
  \end{Phonetics}
\end{Entry}

\begin{Entry}{发觉}{5,9}{⼜、⾒}
  \begin{Phonetics}{发觉}{fa1jue2}[][HSK 5]
    \definition{v.}{vir a saber; estar ciente (de); perceber; tornar-se consciente | encontrar; detectar; perceber; descobrir}
  \end{Phonetics}
\end{Entry}

\begin{Entry}{发送}{5,9}{⼜、⾡}
  \begin{Phonetics}{发送}{fa1 song4}[][HSK 3]
    \definition{v.}{enviar; despachar | transmitir (rádio)}
  \end{Phonetics}
\end{Entry}

\begin{Entry}{发音}{5,9}{⼜、⾳}
  \begin{Phonetics}{发音}{fa1yin1}
    \definition{s.}{pronúncia}
    \definition{v.}{pronunciar}
  \end{Phonetics}
\end{Entry}

\begin{Entry}{发射}{5,10}{⼜、⼨}
  \begin{Phonetics}{发射}{fa1she4}[][HSK 5]
    \definition{v.}{subir; disparar; lançar; irradiar; projetar; descarregar; enviar algo (como uma bala, um projétil, um satélite, etc.) de um dispositivo em uma velocidade muito alta}
  \end{Phonetics}
\end{Entry}

\begin{Entry}{发展}{5,10}{⼜、⼫}
  \begin{Phonetics}{发展}{fa1zhan3}[][HSK 3]
    \definition{v.}{crescer; expandir; avançar; desenvolver; a mudança das coisas de pequeno para grande, de simples para complexo, de inferior para superior | recrutar; admitir expandir (organização, escala, etc.)}
  \end{Phonetics}
\end{Entry}

\begin{Entry}{发烧}{5,10}{⼜、⽕}
  \begin{Phonetics}{发烧}{fa1shao1}[][HSK 4]
    \definition{v.}{ter febre; a temperatura corporal normal de uma pessoa é de cerca de 37ºC; se exceder 37,5ºC, é febre}
  \end{Phonetics}
\end{Entry}

\begin{Entry}{发病}{5,10}{⼜、⽧}
  \begin{Phonetics}{发病}{fa1 bing4}[][HSK 6]
    \definition{v.}{(de uma doença) avanço | patogênese; morbidade | surto (de uma doença)}
  \end{Phonetics}
\end{Entry}

\begin{Entry}{发起}{5,10}{⼜、⾛}
  \begin{Phonetics}{发起}{fa1 qi3}[][HSK 6]
    \definition{s.}{iniciador; patrocinador}
    \definition{v.}{iniciar; patrocinar; começar; lançar}
  \end{Phonetics}
\end{Entry}

\begin{Entry}{发票}{5,11}{⼜、⽰}
  \begin{Phonetics}{发票}{fa1piao4}[][HSK 4]
    \definition[张,种]{s.}{conta; recibo; fatura; recibos emitidos por lojas ou outros escritórios de cobrança}
  \end{Phonetics}
\end{Entry}

\begin{Entry}{发愁}{5,13}{⼜、⼼}
  \begin{Phonetics}{发愁}{fa1chou2}
    \definition{v.+compl.}{preocupar-se | ficar ansioso | ficar triste}
  \end{Phonetics}
\end{Entry}

\begin{Entry}{发簪}{5,18}{⼜、⽵}
  \begin{Phonetics}{发簪}{fa4zan1}
    \definition{s.}{grampo de cabelo}
  \end{Phonetics}
\end{Entry}

\begin{Entry}{古}{5}{⼝}
  \begin{Phonetics}{古}{gu3}[][HSK 3]
    \definition*{s.}{Cuba, abreviação de 古巴 | Sobrenome Gu}
    \definition{adj.}{antigo; milenar; ancestral; secular | simples e sincero | velho | arcaico}
    \definition{pref.}{(distante no tempo; antigo; primitivo) paleo-; arqueo-}
    \definition{s.}{tempos antigos (oposto a 今) | antiguidade; ancestralidade | livros ou ortodoxias dos sábios antigos, a tradição do Tao | uma forma de poesia pré-Tang}
  \seealsoref{古巴}{gu3ba1}
  \seealsoref{今}{jin1}
  \end{Phonetics}
\end{Entry}

\begin{Entry}{古人}{5,2}{⼝、⼈}
  \begin{Phonetics}{古人}{gu3ren2}
    \definition{s.}{pessoas dos tempos antigos | os antigos | espécies humanas extintas, como \emph{Homo erectus} ou \emph{Homo neanderthalensis} | (literário) pessoa falecida}
  \end{Phonetics}
\end{Entry}

\begin{Entry}{古巴}{5,4}{⼝、⼰}
  \begin{Phonetics}{古巴}{gu3ba1}
    \definition*{s.}{Cuba}
  \end{Phonetics}
\end{Entry}

\begin{Entry}{古代}{5,5}{⼝、⼈}
  \begin{Phonetics}{古代}{gu3dai4}[][HSK 3]
    \definition{s.}{tempos antigos; o passado é um período muito distante do presente (diferentemente de 近代 e 现代); na periodização histórica chinesa, geralmente se refere ao período anterior a meados do século XIX | sociedade antiga; sociedade primitiva; refere-se especificamente à era da sociedade escravista (às vezes também inclui a era comunal primitiva) |antigamente; tempos antigos; no passado}
  \seealsoref{近代}{jin4dai4}
  \seealsoref{现代}{xian4dai4}
  \end{Phonetics}
\end{Entry}

\begin{Entry}{古老}{5,6}{⼝、⽼}
  \begin{Phonetics}{古老}{gu3 lao3}[][HSK 5]
    \definition{adj.}{antigo; antiquado; histórico}
  \end{Phonetics}
\end{Entry}

\begin{Entry}{古典}{5,8}{⼝、⼋}
  \begin{Phonetics}{古典}{gu3dian3}[][HSK 6]
    \definition{adj.}{clássico; descreve uma obra ou coisa como tendo características tradicionais ou exemplares}
    \definition{s.}{os clássicos}
  \end{Phonetics}
\end{Entry}

\begin{Entry}{古城}{5,9}{⼝、⼟}
  \begin{Phonetics}{古城}{gu3cheng2}
    \definition{s.}{cidade antiga}
  \end{Phonetics}
\end{Entry}

\begin{Entry}{古铜色}{5,11,6}{⼝、⾦、⾊}
  \begin{Phonetics}{古铜色}{gu3tong2 se4}
    \definition{s.}{cor bronze}
  \end{Phonetics}
\end{Entry}

\begin{Entry}{古装}{5,12}{⼝、⾐}
  \begin{Phonetics}{古装}{gu3 zhuang1}
    \definition[套]{s.}{traje antigo; roupas tradicionais; roupas de estilo antigo}
  \end{Phonetics}
\end{Entry}

\begin{Entry}{句}{5}{⼝}
  \begin{Phonetics}{句}{gou4}
    \variantof{勾}
  \end{Phonetics}
  \begin{Phonetics}{句}{ju4}[][HSK 2]
    \definition{clas.}{para sentenças, frases ou linhas de versos}
    \definition{s.}{frase; sentença}
  \end{Phonetics}
\end{Entry}

\begin{Entry}{句子}{5,3}{⼝、⼦}
  \begin{Phonetics}{句子}{ju4zi5}[][HSK 2]
    \definition[个,句]{s.}{sentença; uma unidade linguística composta por palavras ou frases que expressa um significado completo}
  \end{Phonetics}
\end{Entry}

\begin{Entry}{另}{5}{⼝}
  \begin{Phonetics}{另}{ling4}[][HSK 6]
    \definition*{s.}{Sobrenome Ling}
    \definition{adv.}{além disso; indica que está fora do escopo da declaração | no lugar de; em vez de}
    \definition{pron.}{(com substantivos) outro; diferente; refere-se a pessoas ou coisas fora do escopo do que é dito}
  \end{Phonetics}
\end{Entry}

\begin{Entry}{另一方面}{5,1,4,9}{⼝、⼀、⽅、⾯}
  \begin{Phonetics}{另一方面}{ling4 yi4 fang1 mian4}[][HSK 3]
    \definition{adv./conj.}{outro aspecto | por outro lado; por sua vez; em contrapartida}
  \end{Phonetics}
\end{Entry}

\begin{Entry}{另外}{5,5}{⼝、⼣}
  \begin{Phonetics}{另外}{ling4wai4}[][HSK 3]
    \definition{adv.}{além disso; em adição; ademais; além do mais; além de que; além do que já foi dito}
    \definition{conj.}{além disso; usada entre duas ou mais frases, indica algo além do que foi mencionado anteriormente}
    \definition{pron.}{outro; além das pessoas ou coisas mencionadas anteriormente}
  \end{Phonetics}
\end{Entry}

\begin{Entry}{只}{5}{⼝}
  \begin{Phonetics}{只}{zhi1}[][HSK 3]
    \definition{adj.}{solteiro; solitário; único; muito raro}
    \definition{clas.}{usado para um de um par | usado para animais pequenos (pássaros, gatos, cães, etc.) | usado para certos utensílios, aparelhos | usado para navios}
  \end{Phonetics}
  \begin{Phonetics}{只}{zhi3}[][HSK 2]
    \definition{adv.}{somente; apenas; meramente | simplesmente; usado para limitar o escopo, indicando que não há nada além disso, equivalente a 仅仅}
  \seealsoref{仅仅}{jin3 jin3}
  \end{Phonetics}
\end{Entry}

\begin{Entry}{只不过}{5,4,6}{⼝、⼀、⾡}
  \begin{Phonetics}{只不过}{zhi3 bu2 guo4}[][HSK 5]
    \definition{adv.}{somente; apenas; meramente; não mais do que}
  \end{Phonetics}
\end{Entry}

\begin{Entry}{只见}{5,4}{⼝、⾒}
  \begin{Phonetics}{只见}{zhi3 jian4}[][HSK 5]
    \definition{v.}{somente ver; ver; só vi, e de repente percebi uma certa situação}
  \end{Phonetics}
\end{Entry}

\begin{Entry}{只好}{5,6}{⼝、⼥}
  \begin{Phonetics}{只好}{zhi3hao3}[][HSK 3]
    \definition{v.}{ter que; ser forçado a; não ter escolha a não ser; significa que só pode ser assim, não há outra opção}
  \end{Phonetics}
\end{Entry}

\begin{Entry}{只有}{5,6}{⼝、⽉}
  \begin{Phonetics}{只有}{zhi3 you3}[][HSK 3]
    \definition{adv.}{somente; tem que; forçado a}
    \definition{conj.}{somente se; conecta frases, expressa condições necessárias, geralmente corresponde a 才 e 方}
  \seealsoref{才}{cai2}
  \seealsoref{方}{fang1}
  \end{Phonetics}
\end{Entry}

\begin{Entry}{只有……才……}{5,6,3}{⼝、⽉、⼿}
  \begin{Phonetics}{只有……才……}{zhi3you3 cai2}
    \definition{conj.}{só se\dots então\dots}
  \end{Phonetics}
\end{Entry}

\begin{Entry}{只身}{5,7}{⼝、⾝}
  \begin{Phonetics}{只身}{zhi1shen1}
    \definition{adv.}{sozinho | por si só}
  \end{Phonetics}
\end{Entry}

\begin{Entry}{只怕}{5,8}{⼝、⼼}
  \begin{Phonetics}{只怕}{zhi3pa4}
    \definition{adv.}{receio que\dots | talvez | muito provavelmente}
  \end{Phonetics}
\end{Entry}

\begin{Entry}{只是}{5,9}{⼝、⽇}
  \begin{Phonetics}{只是}{zhi3 shi4}[][HSK 3]
    \definition{adv.}{somente; meramente; apenas; expressa ênfase limitada a uma determinada situação ou âmbito}
    \definition{conj.}{somente; mas; exceto que; conecta frases, indicando uma ligeira transição, equivalente a 不过}
  \seealsoref{不过}{bu2guo4}
  \end{Phonetics}
\end{Entry}

\begin{Entry}{只要}{5,9}{⼝、⾑}
  \begin{Phonetics}{只要}{zhi3yao4}[][HSK 2]
    \definition{conj.}{desde que; se apenas; contanto que; indica condições necessárias (就 ou 可 são frequentemente usados depois)}
  \seealsoref{便}{bian4}
  \seealsoref{就}{jiu4}
  \end{Phonetics}
\end{Entry}

\begin{Entry}{只要……就……}{5,9,12}{⼝、⾑、⼪}
  \begin{Phonetics}{只要……就……}{zhi3yao4 jiu4}
    \definition{conj.}{contanto que/desde que/se somente\dots, então\dots}
  \end{Phonetics}
\end{Entry}

\begin{Entry}{只消}{5,10}{⼝、⽔}
  \begin{Phonetics}{只消}{zhi3xiao1}
    \definition{conj.}{desde que}
  \end{Phonetics}
\end{Entry}

\begin{Entry}{只能}{5,10}{⼝、⾁}
  \begin{Phonetics}{只能}{zhi3 neng2}[][HSK 2]
    \definition{adv.}{só pode; obrigado a fazer algo; isso significa que devido à limitação da capacidade pessoal ou às condições objetivas, não há outra escolha senão esta}
  \end{Phonetics}
\end{Entry}

\begin{Entry}{只读}{5,10}{⼝、⾔}
  \begin{Phonetics}{只读}{zhi3du2}
    \definition{s.}{somente leitura (computação) | \emph{read-only}}
  \end{Phonetics}
\end{Entry}

\begin{Entry}{只顾}{5,10}{⼝、⾴}
  \begin{Phonetics}{只顾}{zhi3 gu4}[][HSK 6]
    \definition{adv.}{meramente; simplesmente; apenas se importa com; indica que a atenção está focada em apenas um aspecto}
    \definition{v.}{considerar apenas uma coisa}
  \end{Phonetics}
\end{Entry}

\begin{Entry}{只得}{5,11}{⼝、⼻}
  \begin{Phonetics}{只得}{zhi3 de5}[][HSK 6]
    \definition{v.}{não ter alternativa senão; obrigado a; ter que; ser obrigado a}
  \end{Phonetics}
\end{Entry}

\begin{Entry}{只管}{5,14}{⼝、⽵}
  \begin{Phonetics}{只管}{zhi3 guan3}[][HSK 6]
    \definition{adv.}{por todos os meios; expressa incentivo para que os outros façam algo com confiança, sem se preocuparem com outras coisas | apenas; simplesmente; significa fazer uma coisa com seriedade, sem se preocupar com outras coisas}
  \end{Phonetics}
\end{Entry}

\begin{Entry}{叫}{5}{⼝}
  \begin{Phonetics}{叫}{jiao4}[][HSK 1,3]
    \definition{adj.}{macho (animal)}
    \definition{prep.}{usado em frases passivas; introduz o agente da ação; equivalente a 被 | combinado com 看, 说; usado para expressar suas ideias e pontos de vista}
    \definition{v.}{chorar; gritar; berrar | nomear; chamar | chamar; chamar a atenção | cumprimentar; saudar; dizer olá | pedir; ordenar; licitar | permitir; concordar com algo; concordar em fazer algo | contratar; encomendar; comprar o que você precisa}
  \seealsoref{被}{bei4}
  \seealsoref{看}{kan4}
  \seealsoref{说}{shuo1}
  \end{Phonetics}
\end{Entry}

\begin{Entry}{叫作}{5,7}{⼝、⼈}
  \begin{Phonetics}{叫作}{jiao4 zuo4}[][HSK 2]
    \definition{v.}{ser chamado de; ser conhecido como}
  \end{Phonetics}
\end{Entry}

\begin{Entry}{召}{5}{⼝}
  \begin{Phonetics}{召}{shao4}
    \definition*{s.}{Sobrenome Shao}
    \definition{s.}{(frequentemente em nomes de lugares mongóis) templo; mosteiro}
    \definition{v.}{convocar; intimar; invocar}
  \end{Phonetics}
  \begin{Phonetics}{召}{zhao4}
    \definition*{s.}{Sobrenome Zhao}
    \definition{s.}{templo}
    \definition{v.}{chamar; intimar; convocar; invocar}
  \end{Phonetics}
\end{Entry}

\begin{Entry}{召开}{5,4}{⼝、⼶}
  \begin{Phonetics}{召开}{zhao4kai1}[][HSK 4]
    \definition{v.}{convocar; chamar pessoas para uma reunião; realizar (uma reunião)}
  \end{Phonetics}
\end{Entry}

\begin{Entry}{叮}{5}{⼝}
  \begin{Phonetics}{叮}{ding1}
    \definition{v.}{picar; ferroar | dizer ou perguntar novamente para ter certeza; verificar; insistir; certificar-se | sondar; perseguir}
  \end{Phonetics}
\end{Entry}

\begin{Entry}{叮嘱}{5,15}{⼝、⼝}
  \begin{Phonetics}{叮嘱}{ding1zhu3}
    \definition{v.}{exortar | avisar | insistir de novo e de novo}
  \end{Phonetics}
\end{Entry}

\begin{Entry}{可}{5}{⼝}
  \begin{Phonetics}{可}{ke3}[][HSK 5]
    \definition*{s.}{Sobrenome Ke}
    \definition{adv.}{indica ênfase | indica o fortalecimento de perguntas retóricas | indica um tom de questionamento mais forte | sobre; a respeito de}
    \definition{conj.}{mas; ainda}
    \definition{v.}{aprovar; concordar com | poder; permitir; ser capaz de | precisar (fazer); valer a pena (fazer); merecer | ajustar; adequar | estar pronto para; estar disposto a; pretender}
  \end{Phonetics}
  \begin{Phonetics}{可}{ke4}
    \definition{s.}{governante supremo de uma tribo nômade do norte; Khan (可汗), título do governante supremo dos antigos grupos étnicos xianbei, turco, uigur e mongol}
  \seealsoref{可汗}{ke4han2}
  \end{Phonetics}
\end{Entry}

\begin{Entry}{可口可乐}{5,3,5,5}{⼝、⼝、⼝、⼃}
  \begin{Phonetics}{可口可乐}{ke3kou3ke3le4}
    \definition*{s.}{Empréstimo linguístico: Coca-Cola}
  \end{Phonetics}
\end{Entry}

\begin{Entry}{可以}{5,4}{⼝、⼈}
  \begin{Phonetics}{可以}{ke3yi3}[][HSK 2]
    \definition{adj.}{aceitável; nada mal; muito bom | impressionante; espantoso; tremendo}
    \definition{v.}{poder; ter condições, capacidade e tempo para fazer algo ou ter alguma utilidade | permitir; poder | valer a pena fazer; considerar que vale a pena, recomendar fazer algo}
  \end{Phonetics}
\end{Entry}

\begin{Entry}{可见}{5,4}{⼝、⾒}
  \begin{Phonetics}{可见}{ke3jian4}[][HSK 4]
    \definition{adj.}{visível; concebível; algo que é óbvio ou evidente}
    \definition{conj.}{isso mostra; isto prova; é, portanto, claro (ou evidente, óbvio) que}
    \definition{v.}{ser ou estar visível ; ser ou estar claro}
  \end{Phonetics}
\end{Entry}

\begin{Entry}{可乐}{5,5}{⼝、⼃}
  \begin{Phonetics}{可乐}{ke3 le4}[][HSK 3]
    \definition*[罐,杯,瓶,听,口]{s.}{\emph{coke}; coca; coca-cola}
    \definition{adj.}{engraçado; divertido; risível}
  \end{Phonetics}
\end{Entry}

\begin{Entry}{可卡因}{5,5,6}{⼝、⼘、⼞}
  \begin{Phonetics}{可卡因}{ke3ka3yin1}
    \definition{s.}{(empréstimo linguístico) cocaína}
  \end{Phonetics}
\end{Entry}

\begin{Entry}{可汗}{5,6}{⼝、⽔}
  \begin{Phonetics}{可汗}{ke4han2}
    \definition{s.}{khan (empréstimo linguístico); cham}
  \end{Phonetics}
\end{Entry}

\begin{Entry}{可怕}{5,8}{⼝、⼼}
  \begin{Phonetics}{可怕}{ke3pa4}[][HSK 2]
    \definition{adj.}{assustador; terrível; hediondo; medonho; horrível; aterrorizante}
    \definition{adv.}{terrivelmente}
  \end{Phonetics}
\end{Entry}

\begin{Entry}{可怜}{5,8}{⼝、⼼}
  \begin{Phonetics}{可怜}{ke3lian2}[][HSK 5]
    \definition{adj.}{pobre; lamentável; lastimável | miserável (de quantidade ou qualidade); descreve um número pequeno ou um lugar tão pequeno que não vale a pena falar sobre ele}
    \definition{v.}{ter pena; ter piedade de; ter simpatia por pessoas que tiveram coisas muito ruins acontecendo com elas}
  \end{Phonetics}
\end{Entry}

\begin{Entry}{可是}{5,9}{⼝、⽇}
  \begin{Phonetics}{可是}{ke3shi4}[][HSK 2]
    \definition{adv.}{de fato (usado para dar ênfase), equivalente a 的确}
    \definition{conj.}{mas; no entanto; contudo; conecta frases, expressa uma relação de transição, equivalente a 但是}
  \seealsoref{但是}{dan4 shi4}
  \seealsoref{的确}{di2que4}
  \end{Phonetics}
\end{Entry}

\begin{Entry}{可爱}{5,10}{⼝、⽖}
  \begin{Phonetics}{可爱}{ke3'ai4}[][HSK 2]
    \definition{adj.}{adorável; simpático; encantador | bonitinho; adorável | amado; querido; encantador; cativante; relacionamento próximo, sentimentos profundos | fofo; bonito}
  \end{Phonetics}
\end{Entry}

\begin{Entry}{可能}{5,10}{⼝、⾁}
  \begin{Phonetics}{可能}{ke3neng2}[][HSK 2]
    \definition{adj.}{possível}
    \definition{adv.}{possivelmente}
    \definition[种]{s.}{possibilidade; tendências ou oportunidades que podem se tornar realidade}
  \end{Phonetics}
\end{Entry}

\begin{Entry}{可惜}{5,11}{⼝、⼼}
  \begin{Phonetics}{可惜}{ke3xi1}[][HSK 5]
    \definition{adj.}{é uma pena; é muito ruim; é lamentável}
    \definition{adv.}{infelizmente}
  \end{Phonetics}
\end{Entry}

\begin{Entry}{可编程}{5,12,12}{⼝、⽷、⽲}
  \begin{Phonetics}{可编程}{ke3bian1cheng2}
    \definition{adj.}{programável}
  \end{Phonetics}
\end{Entry}

\begin{Entry}{可靠}{5,15}{⼝、⾮}
  \begin{Phonetics}{可靠}{ke3kao4}[][HSK 3]
    \definition{adj.}{confiável; digno de confiança | verdadeiro; autêntico; descrever notícias e outras informações como verdadeiras, de modo que as pessoas possam acreditar nelas}
  \end{Phonetics}
\end{Entry}

\begin{Entry*}{可擦写可编程只读存储器}{5,17,5,5,12,12,5,10,6,12,16}{⼝、⼿、⼍、⼝、⽷、⽲、⼝、⾔、⼦、⼈、⼝}
  \begin{Phonetics}{可擦写可编程只读存储器}{ke3 ca1 xie3 ke3 bian1cheng2 zhi1 du2 cun2chu3qi4}
    \definition{s.}{EPROM (\emph{erasable programmable read-only memory})}
  \end{Phonetics}
\end{Entry*}

\begin{Entry}{台}{5}{⼝}
  \begin{Phonetics}{台}{tai2}[][HSK 3]
    \definition*{s.}{Sobrenome Tai}
    \definition{clas.}{usado para certas máquinas, aparelhos, instrumentos, etc | usado para uma performance completa, como drama, música e dança}
    \definition{s.}{torre | plataforma; palco | suporte; pedestal | qualquer coisa em forma de plataforma ou palco | mesa; escrivaninha | estação de transmissão; refere-se a estações de rádio | um serviço telefônico especial; refere-se à estação telefônica | ``seu'' (um termo respeitoso usado antigamente para se dirigir a alguém) | tufão}
  \end{Phonetics}
\end{Entry}

\begin{Entry}{台上}{5,3}{⼝、⼀}
  \begin{Phonetics}{台上}{tai2 shang4}[][HSK 4]
    \definition{s.}{no palco}
  \end{Phonetics}
\end{Entry}

\begin{Entry}{台下}{5,3}{⼝、⼀}
  \begin{Phonetics}{台下}{tai2xia4}
    \definition{s.}{platéia | fora do palco}
  \end{Phonetics}
\end{Entry}

\begin{Entry}{台风}{5,4}{⼝、⾵}
  \begin{Phonetics}{台风}{tai2feng1}[][HSK 5]
    \definition[场,阵,级]{s.}{tufão; classificação de um ciclone tropical ocorrido no oeste do Pacífico Norte | postura; presença de palco; comportamento ou estilo que os atores demonstram no palco}
  \end{Phonetics}
\end{Entry}

\begin{Entry}{台灯}{5,6}{⼝、⽕}
  \begin{Phonetics}{台灯}{tai2 deng1}[][HSK 6]
    \definition[个,盏]{s.}{luminária de mesa; luminária de leitura; uma luminária com base para uso sobre uma mesa}
  \end{Phonetics}
\end{Entry}

\begin{Entry}{台阶}{5,6}{⼝、⾩}
  \begin{Phonetics}{台阶}{tai2jie1}[][HSK 4]
    \definition[个,级]{s.}{escada; escadaria | passos; metáfora para uma maneira ou oportunidade de evitar constrangimentos causados ​​por um impasse | nova fase; novo nível; novo patamar; metáfora para novas conquistas ou novos patamares alcançados no estudo ou no trabalho}
  \end{Phonetics}
\end{Entry}

\begin{Entry}{右}{5}{⼝}
  \begin{Phonetics}{右}{you4}[][HSK 1]
    \definition*{s.}{Sobrenome You}
    \definition{adj.}{conservador; reacionário}
    \definition{s.}{a direita; o lado direito | oeste; na antiguidade, referia-se especificamente à direção oeste (com base na orientação para o sul) | o lado direito como o lado de precedência; posição ou nível mais elevado (os antigos costumavam considerar a direita como mais respeitável)}
    \definition{v.}{favorecer; apoiar; reverenciar}
  \end{Phonetics}
\end{Entry}

\begin{Entry}{右手}{5,4}{⼝、⼿}
  \begin{Phonetics}{右手}{you4shou3}
    \definition{s.}{mão direita | lado direito}
  \end{Phonetics}
\end{Entry}

\begin{Entry}{右边}{5,5}{⼝、⾡}
  \begin{Phonetics}{右边}{you4bian5}[][HSK 1]
    \definition{s.}{a direita; o lado direito; do lado direito}
  \end{Phonetics}
\end{Entry}

\begin{Entry}{右侧}{5,8}{⼝、⼈}
  \begin{Phonetics}{右侧}{you4ce4}
    \definition{s.}{lateral direita | lado direito}
  \end{Phonetics}
\end{Entry}

\begin{Entry}{右转}{5,8}{⼝、⾞}
  \begin{Phonetics}{右转}{you4zhuan3}
    \definition{v.}{virar à direita}
  \end{Phonetics}
\end{Entry}

\begin{Entry}{右面}{5,9}{⼝、⾯}
  \begin{Phonetics}{右面}{you4mian4}
    \definition{s.}{lado direito}
  \end{Phonetics}
\end{Entry}

\begin{Entry}{右倾}{5,10}{⼝、⼈}
  \begin{Phonetics}{右倾}{you4qing1}
    \definition{adj.}{conservador | reacionário}
  \end{Phonetics}
\end{Entry}

\begin{Entry}{右袒}{5,10}{⼝、⾐}
  \begin{Phonetics}{右袒}{you4tan3}
    \definition{v.}{ser tendencioso | ser parcial | favorecer um lado | tomar partido}
  \end{Phonetics}
\end{Entry}

\begin{Entry}{叶}{5}{⼝}
  \begin{Phonetics}{叶}{ye4}
    \definition*{s.}{Sobrenome Ye}
    \definition[枝]{s.}{folha; folhagem | coisa parecida com uma folha | página; folha | parte de um período histórico; segmentos de período mais longos | lóbulo; lóbulos do cérebro, pulmões e fígado}
  \end{Phonetics}
\end{Entry}

\begin{Entry}{叶子}{5,3}{⼝、⼦}
  \begin{Phonetics}{叶子}{ye4zi5}[][HSK 4]
    \definition[片]{s.}{folha; termo genérico para as folhas de uma planta}
  \end{Phonetics}
\end{Entry}

\begin{Entry}{号}{5}{⼝}
  \begin{Phonetics}{号}{hao2}
    \definition{v.}{uivar; gritar; gritar em voz alta e prolongada | lamentar; chorar alto | uivar; (vento) assobiar, assoviar}
  \end{Phonetics}
  \begin{Phonetics}{号}{hao4}[][HSK 1]
    \definition{clas.}{usado para o número de pessoas |  tipo; espécie; classificação | usado para pessoas ou negócios; número de vezes utilizado para transações}
    \definition[把]{s.}{nome | nome presumido; nome alternativo; pseudônimo; apelido | casa de negócios; loja | marca; sinal; sinalização | número | data | ordem; no exército, as ordens são transmitidas verbalmente ou por meio de clarins | qualquer instrumento de sopro e latão; trombeta usada no exército ou em bandas | qualquer coisa usada como buzina | chamada de corneta; qualquer chamada feita em uma corneta; usar um apito para emitir um som com um significado específico | pessoa em uma condição especial; pessoas que se encontram em uma situação especial}
    \definition{suf.}{sufixo de navio}
    \definition{v.}{marcar; fazer uma marca | sentir; colocar a mão no pulso do paciente e avaliar a situação através do fluxo sanguíneo}
  \end{Phonetics}
\end{Entry}

\begin{Entry}{号召}{5,5}{⼝、⼝}
  \begin{Phonetics}{号召}{hao4zhao4}[][HSK 5]
    \definition{s.}{chamado; apelo; desejo ou pedido solene (de um governo, partido político, organização etc.) para que as massas façam algo}
    \definition{v.}{chamar;  (governo, partido político, organização, etc.) fazer um pedido solene às massas para que façam algo, na esperança de que todos se esforcem para alcançá-lo}
  \end{Phonetics}
\end{Entry}

\begin{Entry}{号角}{5,7}{⼝、⾓}
  \begin{Phonetics}{号角}{hao4jiao3}
    \definition{s.}{corneta | trombeta}
  \end{Phonetics}
\end{Entry}

\begin{Entry}{号码}{5,8}{⼝、⽯}
  \begin{Phonetics}{号码}{hao4ma3}[][HSK 4]
    \definition[个,组,串]{s.}{número}
  \end{Phonetics}
\end{Entry}

\begin{Entry}{司}{5}{⼝}
  \begin{Phonetics}{司}{si1}
    \definition*{s.}{Sobrenome Si}
    \definition{s.}{departamento (sob um ministério); um departamento dentro de uma agência de nível ministerial}
    \definition{v.}{assumir o comando de; atender; administrar; operar; gerenciar}
  \end{Phonetics}
\end{Entry}

\begin{Entry}{司长}{5,4}{⼝、⾧}
  \begin{Phonetics}{司长}{si1 zhang3}[][HSK 6]
    \definition[位,名]{s.}{diretor-geral | chefe de gabinete}
  \end{Phonetics}
\end{Entry}

\begin{Entry}{司机}{5,6}{⼝、⽊}
  \begin{Phonetics}{司机}{si1ji1}[][HSK 2]
    \definition[个,名,位]{s.}{motorista; motorista particular; chofer; motoristas de veículos de transporte público, como trens, ônibus e bondes}
  \end{Phonetics}
\end{Entry}

\begin{Entry}{叹}{5}{⼝}
  \begin{Phonetics}{叹}{tan4}
    \definition{v.}{suspirar | exclamar com admiração; aclamar; louvar |recitar com cadência; entoar cântico; entoar}
  \end{Phonetics}
\end{Entry}

\begin{Entry}{叹气}{5,4}{⼝、⽓}
  \begin{Phonetics}{叹气}{tan4qi4}[][HSK 6]
    \definition{v.}{suspirar; soltar um suspiro; soltar um longo suspiro e fazer um som devido à insatisfação ou desamparo}
  \end{Phonetics}
\end{Entry}

\begin{Entry}{囘}{5}{⼞}
  \begin{Phonetics}{囘}{hui2}
    \variantof{回}
  \end{Phonetics}
\end{Entry}

\begin{Entry}{四}{5}{⼞}
  \begin{Phonetics}{四}{si4}[][HSK 1]
    \definition*{s.}{Sobrenome Si}
    \definition{num.}{quatro; 4}
    \definition{s.}{uma nota da escala em Gongchepu (工尺谱), correspondente ao 6 na notação musical numerada}
  \seealsoref{工尺谱}{gong1 che3 pu3}
  \end{Phonetics}
\end{Entry}

\begin{Entry}{四川}{5,3}{⼞、⼮}
  \begin{Phonetics}{四川}{si4chuan1}
    \definition*{s.}{Sichuan}
  \end{Phonetics}
\end{Entry}

\begin{Entry}{四处}{5,5}{⼞、⼡}
  \begin{Phonetics}{四处}{si4 chu4}[][HSK 6]
    \definition{adv.}{em volta; ao redor; em todos os lugares; em todas as direções}
  \end{Phonetics}
\end{Entry}

\begin{Entry}{四周}{5,8}{⼞、⼝}
  \begin{Phonetics}{四周}{si4 zhou1}[][HSK 5]
    \definition{s.}{ao redor; por todos os lados; a parte que circunda o centro}
  \end{Phonetics}
\end{Entry}

\begin{Entry}{四季分明}{5,8,4,8}{⼞、⼦、⼑、⽇}
  \begin{Phonetics}{四季分明}{si4ji4-fen1ming2}
    \definition{expr.}{as quatro estações são muito distintas}
  \end{Phonetics}
\end{Entry}

\begin{Entry}{四季如春}{5,8,6,9}{⼞、⼦、⼥、⽇}
  \begin{Phonetics}{四季如春}{si4ji4-ru2chun1}
    \definition{expr.}{é primavera todo o ano | clima favorável durante todo o ano | quatro estações como a primavera}
  \end{Phonetics}
\end{Entry}

\begin{Entry}{圣}{5}{⼟}
  \begin{Phonetics}{圣}{sheng4}
    \definition*{s.}{Sobrenome Sheng}
    \definition{adj.}{santo; sagrado | imperial}
    \definition{s.}{santo; sábio | imperador | o maior mestre de uma determinada arte ou habilidade}
  \end{Phonetics}
\end{Entry}

\begin{Entry}{圣地}{5,6}{⼟、⼟}
  \begin{Phonetics}{圣地}{sheng4di4}
    \definition{s.}{terra santa (de uma religião) | lugar sagrado | santuário | cidade santa (como Jerusalém, Meca, etc.) | centro de interesse histórico}
  \end{Phonetics}
\end{Entry}

\begin{Entry}{圣诞节}{5,8,5}{⼟、⾔、⾋}
  \begin{Phonetics}{圣诞节}{sheng4 dan4 jie2}[][HSK 6]
    \definition*{s.}{Natal; Nascimento de Jesus Cristo em 25 de dezembro}
  \end{Phonetics}
\end{Entry}

\begin{Entry}{处}{5}{⼡}
  \begin{Phonetics}{处}{chu3}[][HSK 4]
    \definition*{s.}{Sobrenome Chu}
    \definition{v.}{morar; habitar; viver em um lugar | dar-se bem (com alguém); relacionar-se; interagir | estar situado em; estar em uma determinada condição; estar em (um lugar, período ou ocasião) | gerenciar; manejar; lidar com | punir; sentenciar; tomar medidas disciplinares contra (alguém)}
  \end{Phonetics}
  \begin{Phonetics}{处}{chu4}
    \definition{clas.}{usado para lugares ou para ocorrências ou atividades em lugares diferentes}
    \definition{s.}{lugar; local; instalação; dependência | parte; ponto; aspecto ou parte de um objeto | escritório; departamento; nomes de determinados órgãos, organizações ou unidades em órgãos por empresa}
  \end{Phonetics}
\end{Entry}

\begin{Entry}{处于}{5,3}{⼡、⼆}
  \begin{Phonetics}{处于}{chu3 yu2}[][HSK 4]
    \definition{v.}{estar em (uma condição, estado)}
  \end{Phonetics}
\end{Entry}

\begin{Entry}{处分}{5,4}{⼡、⼑}
  \begin{Phonetics}{处分}{chu3fen4}[][HSK 5]
    \definition{s.}{punição; castigo; refere-se a uma decisão de impor uma penalidade ou uma disposição}
    \definition{v.}{punir; tomar medidas disciplinares contra; fornecer algum tratamento ou disposição para aqueles que cometeram erros ou falhas}
  \end{Phonetics}
\end{Entry}

\begin{Entry}{处长}{5,4}{⼡、⾧}
  \begin{Phonetics}{处长}{chu4 zhang3}[][HSK 6]
    \definition{s.}{chefe de um departamento (ou escritório); chefe de seção}
  \end{Phonetics}
\end{Entry}

\begin{Entry}{处处}{5,5}{⼡、⼡}
  \begin{Phonetics}{处处}{chu4 chu4}[][HSK 6]
    \definition{adv.}{em todos os lugares; em todos os aspectos}
  \end{Phonetics}
\end{Entry}

\begin{Entry}{处在}{5,6}{⼡、⼟}
  \begin{Phonetics}{处在}{chu3 zai4}[][HSK 5]
    \definition{v.}{estar situado em; encontrar-se em; estar em (algum estado, posição ou condição)}
  \end{Phonetics}
\end{Entry}

\begin{Entry}{处罚}{5,9}{⼡、⽹}
  \begin{Phonetics}{处罚}{chu3 fa2}[][HSK 5]
    \definition[个,次,种]{s.}{punição; castigo; penalidade}
    \definition{v.}{punir; disciplinar; castigar; advertir o transgressor ou infrator sobre perdas políticas ou financeiras}
  \end{Phonetics}
\end{Entry}

\begin{Entry}{处理}{5,11}{⼡、⽟}
  \begin{Phonetics}{处理}{chu3li3}[][HSK 3]
    \definition{s.}{manuseio; descarte}
    \definition{v.}{lidar com; dispor de; organizar; resolver | resolver; punir; lidar | vender a preços reduzidos; liquidar | lidar com; processar; processar algo de uma maneira ou método específico; processar uma peça de trabalho ou produto de uma maneira específica para que a peça de trabalho ou produto obtenha o desempenho necessário}
  \end{Phonetics}
\end{Entry}

\begin{Entry}{外}{5}{⼣}
  \begin{Phonetics}{外}{wai4}[][HSK 1]
    \definition{adj.}{outro (que não o próprio) | não íntimo; não intimamente relacionado | não oficial | exterior; externo; do lado de fora | outros; referindo-se a um local fora da sua localização atual | do lado da mãe, da irmã ou da filha; referir-se a parentes do lado materno, irmãs ou filhas | informal; não oficial}
    \definition{adv.}{adicionalmente; além disso | para fora; para o exterior; fora | extra; além disso}
    \definition{s.}{fora; externo; exterior (oposto a 内, 里) | outro local; outro lugar | estrangeiro; país estrangeiro | lado externo | parentes de sua mãe, irmãs ou filhas}
  \seealsoref{里}{li3}
  \seealsoref{内}{nei4}
  \end{Phonetics}
\end{Entry}

\begin{Entry}{外公}{5,4}{⼣、⼋}
  \begin{Phonetics}{外公}{wai4gong1}
    \definition{s.}{avô materno}
  \end{Phonetics}
\end{Entry}

\begin{Entry}{外币}{5,4}{⼣、⼱}
  \begin{Phonetics}{外币}{wai4 bi4}[][HSK 6]
    \definition[种]{s.}{moeda estrangeira}
  \end{Phonetics}
\end{Entry}

\begin{Entry}{外文}{5,4}{⼣、⽂}
  \begin{Phonetics}{外文}{wai4 wen2}[][HSK 3]
    \definition[种,门]{s.}{língua ou escrita estrangeira}
  \end{Phonetics}
\end{Entry}

\begin{Entry}{外水}{5,4}{⼣、⽔}
  \begin{Phonetics}{外水}{wai4shui3}
    \definition{s.}{renda extra}
  \end{Phonetics}
\end{Entry}

\begin{Entry}{外出}{5,5}{⼣、⼐}
  \begin{Phonetics}{外出}{wai4 chu1}[][HSK 6]
    \definition{v.}{sair, especialmente para ir a outro lugar a negócios}
  \end{Phonetics}
\end{Entry}

\begin{Entry}{外号}{5,5}{⼣、⼝}
  \begin{Phonetics}{外号}{wai4hao4}
    \definition{s.}{apelido}
  \end{Phonetics}
\end{Entry}

\begin{Entry}{外头}{5,5}{⼣、⼤}
  \begin{Phonetics}{外头}{wai4 tou5}[][HSK 6]
    \definition{s.}{Coloquial: fora; ao ar livre (oposto a 里头)}
  \seealsoref{里头}{li3 tou5}
  \end{Phonetics}
\end{Entry}

\begin{Entry}{外汇}{5,5}{⼣、⽔}
  \begin{Phonetics}{外汇}{wai4 hui4}[][HSK 4]
    \definition{s.}{câmbio estrangeiro; moeda estrangeira; moedas estrangeiras e títulos, como cheques, letras de câmbio, notas promissórias, etc., conversíveis em moedas estrangeiras, usados na compensação do comércio internacional}
  \end{Phonetics}
\end{Entry}

\begin{Entry}{外边}{5,5}{⼣、⾡}
  \begin{Phonetics}{外边}{wai4 bian5}[][HSK 1]
    \definition{s.}{fora; exterior; externo; além de um determinado limite | local diferente de onde se vive ou trabalha; referindo-se a lugares distantes | exterior; externo; superfície}
  \end{Phonetics}
\end{Entry}

\begin{Entry}{外交}{5,6}{⼣、⼇}
  \begin{Phonetics}{外交}{wai4jiao1}[][HSK 3]
    \definition[个]{s.}{diplomacia; relações exteriores; atividades de um país nas relações internacionais, como participar de organizações e conferências internacionais, trocar enviados com outros países, conduzir negociações, assinar tratados e acordos, etc.}
  \end{Phonetics}
\end{Entry}

\begin{Entry}{外交官}{5,6,8}{⼣、⼇、⼧}
  \begin{Phonetics}{外交官}{wai4 jiao1 guan1}[][HSK 4]
    \definition[位,名]{s.}{diplomata}
  \end{Phonetics}
\end{Entry}

\begin{Entry}{外协}{5,6}{⼣、⼗}
  \begin{Phonetics}{外协}{wai4xie2}
    \definition{s.}{terceirização | pessoas que julgam os outros pela aparência}
  \seealsoref{外貌协会}{wai4mao4xie2hui4}
  \end{Phonetics}
\end{Entry}

\begin{Entry}{外地}{5,6}{⼣、⼟}
  \begin{Phonetics}{外地}{wai4 di4}[][HSK 2]
    \definition{s.}{não local; outros lugares; locais fora da área local}
  \end{Phonetics}
\end{Entry}

\begin{Entry}{外孙}{5,6}{⼣、⼦}
  \begin{Phonetics}{外孙}{wai4sun1}
    \definition{s.}{filho da filha}
  \end{Phonetics}
\end{Entry}

\begin{Entry}{外孙女}{5,6,3}{⼣、⼦、⼥}
  \begin{Phonetics}{外孙女}{wai4sun1nv3}
    \definition{s.}{filha da filha}
  \end{Phonetics}
\end{Entry}

\begin{Entry}{外衣}{5,6}{⼣、⾐}
  \begin{Phonetics}{外衣}{wai4 yi1}[][HSK 6]
    \definition[件]{s.}{casaco; jaqueta; colete; sobreveste; envoltório; roupa externa (ou vestimenta); capa externa; vestido externo | semblante; aparência; feição}
  \end{Phonetics}
\end{Entry}

\begin{Entry}{外观}{5,6}{⼣、⾒}
  \begin{Phonetics}{外观}{wai4 guan1}[][HSK 6]
    \definition{s.}{aspecto; semblante; aparência; aparência exterior; a aparência de um objeto}
  \end{Phonetics}
\end{Entry}

\begin{Entry}{外围}{5,7}{⼣、⼞}
  \begin{Phonetics}{外围}{wai4wei2}
    \definition{adv.}{arredores}
  \end{Phonetics}
\end{Entry}

\begin{Entry}{外来}{5,7}{⼣、⽊}
  \begin{Phonetics}{外来}{wai4 lai2}[][HSK 6]
    \definition{adj.}{de fora; externo; estrangeiro}
  \end{Phonetics}
\end{Entry}

\begin{Entry}{外事}{5,8}{⼣、⼅}
  \begin{Phonetics}{外事}{wai4shi4}
    \definition{s.}{assuntos ou relações exteriores}
  \end{Phonetics}
\end{Entry}

\begin{Entry}{外卖}{5,8}{⼣、⼗}
  \begin{Phonetics}{外卖}{wai4 mai4}[][HSK 2]
    \definition[份,单,盒]{s.}{comida para viagem; levar para viagem}
    \definition{v.}{entregar; oferecer; refere-se à ação do comerciante entregar alimentos no local especificado pelo cliente}
  \end{Phonetics}
\end{Entry}

\begin{Entry}{外国}{5,8}{⼣、⼞}
  \begin{Phonetics}{外国}{wai4 guo2}[][HSK 1]
    \definition[个]{s.}{país estrangeiro}
  \end{Phonetics}
\end{Entry}

\begin{Entry}{外国人}{5,8,2}{⼣、⼞、⼈}
  \begin{Phonetics}{外国人}{wai4 guo2 ren2}
    \definition[个]{s.}{estrangeiro | alienígena}
  \end{Phonetics}
\end{Entry}

\begin{Entry}{外界}{5,9}{⼣、⽥}
  \begin{Phonetics}{外界}{wai4jie4}[][HSK 5]
    \definition{s.}{o exterior; o mundo externo; área fora de um determinado âmbito; sociedade externa}
  \end{Phonetics}
\end{Entry}

\begin{Entry}{外科}{5,9}{⼣、⽲}
  \begin{Phonetics}{外科}{wai4 ke1}[][HSK 6]
    \definition[名]{s.}{cirurgia; departamento cirúrgico; um departamento em uma instituição médica que usa principalmente cirurgia para tratar doenças internas e externas}
  \end{Phonetics}
\end{Entry}

\begin{Entry}{外语}{5,9}{⼣、⾔}
  \begin{Phonetics}{外语}{wai4 yu3}[][HSK 1]
    \definition[种,门]{s.}{língua estrangeira}
  \end{Phonetics}
\end{Entry}

\begin{Entry}{外贸}{5,9}{⼣、⾙}
  \begin{Phonetics}{外贸}{wai4mao4}
    \definition{s.}{comércio exterior}
  \end{Phonetics}
\end{Entry}

\begin{Entry}{外面}{5,9}{⼣、⾯}
  \begin{Phonetics}{外面}{wai4 mian4}[][HSK 3]
    \definition{s.}{o lado de fora; fora de um certo intervalo | exterior; aparência externa; a superfície de um objeto}
  \end{Phonetics}
\end{Entry}

\begin{Entry}{外套}{5,10}{⼣、⼤}
  \begin{Phonetics}{外套}{wai4 tao4}[][HSK 4]
    \definition[件,套,个]{s.}{casaco; jaqueta; paletó; sobretudo}
  \end{Phonetics}
\end{Entry}

\begin{Entry}{外海}{5,10}{⼣、⽔}
  \begin{Phonetics}{外海}{wai4hai3}
    \definition{s.}{mar aberto}
  \end{Phonetics}
\end{Entry}

\begin{Entry}{外积}{5,10}{⼣、⽲}
  \begin{Phonetics}{外积}{wai4ji1}
    \definition{s.}{produto exterior | (matemática) o produto vetorial de dois vetores}
  \end{Phonetics}
\end{Entry}

\begin{Entry}{外资}{5,10}{⼣、⾙}
  \begin{Phonetics}{外资}{wai4 zi1}[][HSK 6]
    \definition{s.}{capital estrangeiro (oposto a 内资); investimento estrangeiro; fundos estrangeiros; capital investido por países estrangeiros}
  \seealsoref{内资}{nei4 zi1}
  \end{Phonetics}
\end{Entry}

\begin{Entry}{外部}{5,10}{⼣、⾢}
  \begin{Phonetics}{外部}{wai4 bu4}[][HSK 6]
    \definition{s.}{fora; externo; fora de um certo intervalo | exterior; superfície}
  \end{Phonetics}
\end{Entry}

\begin{Entry}{外婆}{5,11}{⼣、⼥}
  \begin{Phonetics}{外婆}{wai4po2}
    \definition{s.}{avó materna}
  \end{Phonetics}
\end{Entry}

\begin{Entry}{外插}{5,12}{⼣、⼿}
  \begin{Phonetics}{外插}{wai4cha1}
    \definition{v.}{extrapolar | (computação) conectar (um dispositivo periférico, etc.)}
  \end{Phonetics}
\end{Entry}

\begin{Entry}{外貌协会}{5,14,6,6}{⼣、⾘、⼗、⼈}
  \begin{Phonetics}{外貌协会}{wai4mao4xie2hui4}
    \definition{s.}{``o clube da boa aparência'': pessoas que dão grande importância à aparência de uma pessoa}
  \seealsoref{外协}{wai4xie2}
  \end{Phonetics}
\end{Entry}

\begin{Entry}{失}{5}{⼤}
  \begin{Phonetics}{失}{shi1}
    \definition{s.}{deslize; erro; defeito; acidente}
    \definition{v.}{perder (oposto de 得) | perder; deixar escapar | não agir de acordo com; negligenciar; violar | perder o controle de | errar; cometer um deslize; apresentar defeito em | não consiguir encontrar | não conseguir atingir o objetivo | desviar-se do normal | quebrar (uma promessa); voltar atrás (na palavra dada) | não conseguir obter | se perder}
  \seealsoref{得}{de2}
  \end{Phonetics}
\end{Entry}

\begin{Entry}{失业}{5,5}{⼤、⼀}
  \begin{Phonetics}{失业}{shi1ye4}[][HSK 4]
    \definition{v.}{não ter emprego; estar desempregado; estar sem trabalho; refere-se àqueles que estão dentro da idade legal para trabalhar, têm capacidade para trabalhar, estão desempregados e querem encontrar um emprego, mas não conseguem; embora se envolvam em certos trabalhos sociais, sua remuneração é menor do que o padrão mínimo de vida urbano local e são considerados desempregados}
  \end{Phonetics}
\end{Entry}

\begin{Entry}{失去}{5,5}{⼤、⼛}
  \begin{Phonetics}{失去}{shi1qu4}[][HSK 3]
    \definition{v.}{perder}
  \end{Phonetics}
\end{Entry}

\begin{Entry}{失败}{5,8}{⼤、⾒}
  \begin{Phonetics}{失败}{shi1bai4}[][HSK 4]
    \definition{adj.}{insatisfatório; a maneira como as coisas aconteceram deixou muito a desejar; o resultado final deixou muito a desejar}
    \definition{v.}{perder; ser derrotado; não vencer em uma guerra ou competição | falhar; fracassar; não dar em nada; falhar em atingir um objetivo ou meta desejada (trabalho, carreira, etc.)}
  \end{Phonetics}
\end{Entry}

\begin{Entry}{失误}{5,9}{⼤、⾔}
  \begin{Phonetics}{失误}{shi1wu4}[][HSK 5]
    \definition[个]{s.}{erro; engano; equívoco; erros causados por negligência ou medidas inadequadas}
    \definition{v.}{cometer um erro; cometer um equívoco}
  \end{Phonetics}
\end{Entry}

\begin{Entry}{失眠}{5,10}{⼤、⽬}
  \begin{Phonetics}{失眠}{shi1mian2}
    \definition{s.}{insônia}
    \definition{v.}{ter insônia}
  \end{Phonetics}
\end{Entry}

\begin{Entry}{失望}{5,11}{⼤、⽉}
  \begin{Phonetics}{失望}{shi1wang4}[][HSK 4]
    \definition{adj.}{desapontado; decepcionado}
    \definition{v.}{ficar desapontado; ficar decepcionado; estar desapontado; sentir-se sem esperança; perder a confiança}
  \end{Phonetics}
\end{Entry}

\begin{Entry}{失落}{5,12}{⼤、⾋}
  \begin{Phonetics}{失落}{shi1luo4}
    \definition{s.}{frustração | decepção | perda}
    \definition{v.}{perder (algo) | cair (algo) | sentir uma sensação de perda}
  \end{Phonetics}
\end{Entry}

\begin{Entry}{失意}{5,13}{⼤、⼼}
  \begin{Phonetics}{失意}{shi1yi4}
    \definition{adj.}{desapontado | frustrado}
  \end{Phonetics}
\end{Entry}

\begin{Entry}{头}{5}{⼤}
  \begin{Phonetics}{头}{tou2}[][HSK 2,3]
    \definition{adj.}{(antes de um numeral) primeiro | (antes de 年 ou 天) último; anterior}
    \definition{clas.}{usado para suínos ou gado (animais de criação) | usado para cabeças de alho ou coisas com formato de cabeça}
    \definition{num.}{primeiro}
    \definition{prep.}{antes de; perto de; introduz o tempo de uma ação, equivalente a  在……之前 ou 临近 | (entre dois algarismos, indicando um número aproximado) cerca de}
    \definition[个,颗]{s.}{cabeça; a parte do corpo humano ou animal que possui órgãos como boca, nariz, olhos e ouvidos | cabelo ou penteado | topo; fim; a parte superior ou final de um objeto | começo ou fim; o ponto inicial ou final de algo | fim; remanescente; os restos de algo | cabeça; chefe; líder | lado; aspecto}
  \seealsoref{临近}{lin2jin4}
  \seealsoref{年}{nian2}
  \seealsoref{天}{tian1}
  \seealsoref{在}{zai4}
  \seealsoref{之前}{zhi1 qian2}
  \end{Phonetics}
  \begin{Phonetics}{头}{tou5}
    \definition{suf.}{adicionado após componentes nominais comuns | adicionado após o componente verbal, forma um substantivo abstrato, geralmente indicando que vale a pena realizar essa ação | adicionado após um componente adjetival, forma um substantivo, geralmente indicando algo abstrato | adicionado após o componente substantivo que indica a direção}
  \end{Phonetics}
\end{Entry}

\begin{Entry}{头发}{5,5}{⼤、⼜}
  \begin{Phonetics}{头发}{tou2fa5}[][HSK 2]
    \definition[根,缕,头]{s.}{cabelo}
  \end{Phonetics}
\end{Entry}

\begin{Entry}{头号}{5,5}{⼤、⼝}
  \begin{Phonetics}{头号}{tou2hao4}
    \definition{adj.}{primeira classe | número um | \emph{top rank}}
  \end{Phonetics}
\end{Entry}

\begin{Entry}{头头}{5,5}{⼤、⼤}
  \begin{Phonetics}{头头}{tou2tou2}
    \definition{s.}{chefe | o cabeça}
  \end{Phonetics}
\end{Entry}

\begin{Entry}{头疼}{5,10}{⼤、⽧}
  \begin{Phonetics}{头疼}{tou2 teng2}[][HSK 6]
    \definition{s.}{dor de cabeça}
    \definition{v.}{estar preocupado ou incomodado por alguém ou algo}
  \end{Phonetics}
\end{Entry}

\begin{Entry}{头脑}{5,10}{⼤、⾁}
  \begin{Phonetics}{头脑}{tou2 nao3}[][HSK 3]
    \definition{s.}{inteligência; mente | pista; tópicos principais | chefe; líder; capitão}
  \end{Phonetics}
\end{Entry}

\begin{Entry}{头脑风暴}{5,10,4,15}{⼤、⾁、⾵、⽇}
  \begin{Phonetics}{头脑风暴}{tou2nao3feng1bao4}
    \definition{s.}{\emph{brainstorm}}
  \end{Phonetics}
\end{Entry}

\begin{Entry}{头像}{5,13}{⼤、⼈}
  \begin{Phonetics}{头像}{tou2xiang4}
    \definition{s.}{retrato | busto | avatar | imagem de perfil (computação)}
  \end{Phonetics}
\end{Entry}

\begin{Entry}{奶}{5}{⼥}
  \begin{Phonetics}{奶}{nai3}[][HSK 1]
    \definition{adj.}{bebê; infância; infantil}
    \definition[杯,滴,瓶,只,桶]{s.}{seios; mama | leite; produtos lácteos}
    \definition{v.}{amamentar; mamar}
  \end{Phonetics}
\end{Entry}

\begin{Entry}{奶牛}{5,4}{⼥、⽜}
  \begin{Phonetics}{奶牛}{nai3 niu2}[][HSK 6]
    \definition{s.}{vaca leiteira (ou leiteira); vaca}
  \end{Phonetics}
\end{Entry}

\begin{Entry}{奶奶}{5,5}{⼥、⼥}
  \begin{Phonetics}{奶奶}{nai3nai5}[][HSK 1]
    \definition[位]{s.}{avó (paterna) | vovó; avó; mulheres mais velhas | jovem senhora da casa}
  \end{Phonetics}
\end{Entry}

\begin{Entry}{奶茶}{5,9}{⼥、⾋}
  \begin{Phonetics}{奶茶}{nai3 cha2}[][HSK 3]
    \definition[杯]{s.}{chá com leite; chá com leite de vaca ou de ovelha}
  \end{Phonetics}
\end{Entry}

\begin{Entry}{奶粉}{5,10}{⼥、⽶}
  \begin{Phonetics}{奶粉}{nai3 fen3}[][HSK 6]
    \definition[袋,桶,罐,勺]{s.}{leite em pó}
  \end{Phonetics}
\end{Entry}

\begin{Entry}{宁}{5}{⼧}
  \begin{Phonetics}{宁}{ning2}
    \definition*{s.}{Região Autônoma de Ningxia Hui, abreviação de 宁夏回族自治区 | outro nome para Nanquim, 南京 | Sobrenome Ning}
    \definition{adj.}{calmo, pacífico, sereno | saudável}
    \definition{v.}{Literário: fazer uma visita (aos pais ou aos mais velhos); | Literário: pacificar; apaziguar}
  \seealsoref{南京}{nan2jing1}
  \seealsoref{宁夏回族自治区}{ning2xia4 hui2zu2 zi4zhi4qu1}
  \end{Phonetics}
  \begin{Phonetics}{宁}{ning4}
    \definition{conj.}{mais\dots do que\dots, melhor\dots do que\dots}
  \end{Phonetics}
\end{Entry}

\begin{Entry}{宁可}{5,5}{⼧、⼝}
  \begin{Phonetics}{宁可}{ning4ke3}
    \definition{conj.}{mais\dots do que\dots | melhor\dots do que\dots}
  \end{Phonetics}
\end{Entry}

\begin{Entry}{宁可……也不……}{5,5,3,4}{⼧、⼝、⼄、⼀}
  \begin{Phonetics}{宁可……也不……}{ning4ke3 ye3bu4}
    \definition{conj.}{preferiria\dots do que\dots}
  \end{Phonetics}
\end{Entry}

\begin{Entry}{宁可……也要……}{5,5,3,9}{⼧、⼝、⼄、⾑}
  \begin{Phonetics}{宁可……也要……}{ning4ke3 ye3yao4}
    \definition{conj.}{mesmo que tenhamos que\dots nós iremos\dots}
  \end{Phonetics}
\end{Entry}

\begin{Entry}{宁肯}{5,8}{⼧、⾁}
  \begin{Phonetics}{宁肯}{ning4ken3}
    \definition{conj.}{mais\dots do que\dots, melhor\dots do que\dots}
  \end{Phonetics}
\end{Entry}

\begin{Entry}{宁夏回族自治区}{5,10,6,11,6,8,4}{⼧、⼢、⼞、⽅、⾃、⽔、⼖}
  \begin{Phonetics}{宁夏回族自治区}{ning2xia4 hui2zu2 zi4zhi4qu1}
    \definition*{s.}{Região Autônoma de Ningxia Hui}
  \end{Phonetics}
\end{Entry}

\begin{Entry}{宁愿}{5,14}{⼧、⽕}
  \begin{Phonetics}{宁愿}{ning4yuan4}
    \definition{conj.}{mais\dots do que\dots, melhor\dots do que\dots}
  \end{Phonetics}
\end{Entry}

\begin{Entry}{宁静}{5,14}{⼧、⾭}
  \begin{Phonetics}{宁静}{ning2 jing4}[][HSK 4]
    \definition{adj.}{calmo; tranquilo; pacífico}
  \end{Phonetics}
\end{Entry}

\begin{Entry}{它}{5}{⼧}
  \begin{Phonetics}{它}{ta1}[][HSK 2]
    \definition*{s.}{Sobrenome Ta}
    \definition{pron.}{ele; referência a algo além da pessoa (para objetos inanimados) | ele; usado após o verbo, indica referência vaga}
  \end{Phonetics}
\end{Entry}

\begin{Entry}{它们}{5,5}{⼧、⼈}
  \begin{Phonetics}{它们}{ta1 men5}[][HSK 2]
    \definition{pron.}{eles; usado para se referir a mais de uma coisa não humana; geralmente se refere a animais, objetos ou conceitos abstratos}
  \end{Phonetics}
\end{Entry}

\begin{Entry}{对}{5}{⼨}
  \begin{Phonetics}{对}{dui4}[][HSK 1,2]
    \definition{adj.}{certo; correto; em conformidade com determinados padrões | oposto; contrário}
    \definition{adv.}{mutuamente; cara a cara}
    \definition{clas.}{usado para pessoas ou coisas que formam pares; casais}
    \definition{prep.}{o que diz respeito a; relativo a; com relação a; introduz o objeto da ação}
    \definition[幅]{s.}{dístico; refere-se a um par de versos | par; parceiro; pessoas ou coisas que se complementam}
    \definition{v.}{responder; dar uma resposta | tratar; lidar com; combater | ser treinado para; ser direcionado para; enfrentar | colocar (duas coisas) em contato; encaixar uma na outra; combinar ou cooperar entre si | comparar; verificar; identificar; comparar e verificar se estão de acordo | definir; ajustar; ajustar para atender a determinados requisitos | misturar (refere-se principalmente a líquidos); adicionar | dividir ao meio; dividir em duas partes iguais | combinar; concordar; dar-se bem; harmonizar-se}
  \end{Phonetics}
\end{Entry}

\begin{Entry}{对于}{5,3}{⼨、⼆}
  \begin{Phonetics}{对于}{dui4yu2}[][HSK 4]
    \definition{prep.}{para; relativo a; no que diz respeito a; a respeito de}
  \end{Phonetics}
\end{Entry}

\begin{Entry}{对不起}{5,4,10}{⼨、⼀、⾛}
  \begin{Phonetics}{对不起}{dui4bu5qi3}[][HSK 1]
    \definition{interj.}{Desculpe! | Desculpe-me! | Perdoe-me! | Com licença?}
    \definition{v.}{desculpar; pedir desculpas; perdoar}
  \end{Phonetics}
\end{Entry}

\begin{Entry}{对手}{5,4}{⼨、⼿}
  \begin{Phonetics}{对手}{dui4shou3}[][HSK 3]
    \definition[个,名,位,对]{s.}{oponente; adversário na competição | igual; correspondente; refere-se especificamente ao adversário em uma competição em que as habilidades e o nível são praticamente iguais}
  \end{Phonetics}
\end{Entry}

\begin{Entry}{对方}{5,4}{⼨、⽅}
  \begin{Phonetics}{对方}{dui4fang1}[][HSK 3]
    \definition{s.}{outro lado; lado oposto; outra parte; a parte contrária ao sujeito da ação ou outras pessoas envolvidas em um determinado evento ou situação}
  \end{Phonetics}
\end{Entry}

\begin{Entry}{对比}{5,4}{⼨、⽐}
  \begin{Phonetics}{对比}{dui4bi3}[][HSK 4]
    \definition{s.}{razão; proporção | contraste; comparação; diferenças ou lacunas encontradas após comparação}
    \definition{v.}{contrastar; comparar}
  \end{Phonetics}
\end{Entry}

\begin{Entry}{对付}{5,5}{⼨、⼈}
  \begin{Phonetics}{对付}{dui4fu5}[][HSK 4]
    \definition{adj.}{em bons termos; estar em termos agradáveis ​​(frequentemente usado em negativas); dialeto usado para descrever duas pessoas que têm um bom relacionamento e se dão bem, frequentemente usado para negar}
    \definition{v.}{enfrentar; tratar; lidar com | fazer acontecer; (informal) fazer algo que você não quer fazer; aceitar algo que você não gosta}
  \end{Phonetics}
\end{Entry}

\begin{Entry}{对外}{5,5}{⼨、⼣}
  \begin{Phonetics}{对外}{dui4 wai4}[][HSK 6]
    \definition{adj.}{externo; para fora | estrangeiro; no exterior}
  \end{Phonetics}
\end{Entry}

\begin{Entry}{对立}{5,5}{⼨、⽴}
  \begin{Phonetics}{对立}{dui4li4}[][HSK 5]
    \definition{v.}{opor-se; contrastar; filosoficamente, refere-se a duas coisas ou dois aspectos da mesma coisa que se contradizem, se excluem ou entram em conflito entre si | opor-se; ser antagônico a}
  \end{Phonetics}
\end{Entry}

\begin{Entry}{对……有兴趣}{5,6,6,15}{⼨、⽉、⼋、⾛}
  \begin{Phonetics}{对……有兴趣}{dui4 you3xing4qu4}
    \definition{expr.}{estar interessado em\dots; ter interesse em\dots; interessar-se por\dots}
  \seealsoref{对……感兴趣}{dui4 gan3xing4qu4}
  \end{Phonetics}
\end{Entry}

\begin{Entry}{对应}{5,7}{⼨、⼴}
  \begin{Phonetics}{对应}{dui4ying4}[][HSK 5]
    \definition{adj.}{homólogo; correspondente}
    \definition{v.}{corresponder; ser equivalente a}
  \end{Phonetics}
\end{Entry}

\begin{Entry}{对抗}{5,7}{⼨、⼿}
  \begin{Phonetics}{对抗}{dui4kang4}[][HSK 6]
    \definition{v.}{antagonizar; confrontar | resistir; opor-se; contra-atacar}
  \end{Phonetics}
\end{Entry}

\begin{Entry}{对话}{5,8}{⼨、⾔}
  \begin{Phonetics}{对话}{dui4hua4}[][HSK 2]
    \definition[段,番,个]{s.}{diálogo; conversa; refere-se especificamente a diálogos entre personagens em obras literárias, como peças de teatro e romances}
    \definition{v.}{conversar com; comunicar-se com | manter um diálogo; conversar uns com os outros}
  \end{Phonetics}
\end{Entry}

\begin{Entry}{对待}{5,9}{⼨、⼻}
  \begin{Phonetics}{对待}{dui4dai4}[][HSK 3]
    \definition{v.}{tratar; abordar; manusear; estar em uma posição relacionada ou comparada a outra; expressar uma certa atitude ou agir de determinada maneira em relação a pessoas ou coisas}
  \end{Phonetics}
\end{Entry}

\begin{Entry}{对……说}{5,9}{⼨、⾔}
  \begin{Phonetics}{对……说}{dui4 shuo5}
    \definition{v.}{dizer a alguém}
  \end{Phonetics}
\end{Entry}

\begin{Entry}{对面}{5,9}{⼨、⾯}
  \begin{Phonetics}{对面}{dui4mian4}[][HSK 2]
    \definition{adv.}{cara a cara}
    \definition[面]{s.}{lado oposto; o outro lado; os nomes dados às duas margens opostas de ruas, rios, etc. | bem na frente; diretamente à frente}
  \end{Phonetics}
\end{Entry}

\begin{Entry}{对得起}{5,11,10}{⼨、⼻、⾛}
  \begin{Phonetics}{对得起}{dui4de5qi3}
    \definition{v.}{não decepcionar alguém | tratar alguém de maneira justa | ser digno de}
  \end{Phonetics}
\end{Entry}

\begin{Entry}{对象}{5,11}{⼨、⾗}
  \begin{Phonetics}{对象}{dui4xiang4}[][HSK 3]
    \definition[个,位]{s.}{alvo; objeto; a pessoa ou coisa que serve como objetivo ao agir ou pensar | parceiro; namorado; namorada; refere-se especificamente à pessoa amada}
  \end{Phonetics}
\end{Entry}

\begin{Entry}{对……感兴趣}{5,13,6,15}{⼨、⼼、⼋、⾛}
  \begin{Phonetics}{对……感兴趣}{dui4 gan3xing4qu4}
    \definition{expr.}{estar interessado em\dots; ter interesse em\dots; interessar-se por\dots}
  \seealsoref{对……有兴趣}{dui4 you3xing4qu4}
  \end{Phonetics}
\end{Entry}

\begin{Entry}{对……熟悉}{5,15,11}{⼨、⽕、⼼}
  \begin{Phonetics}{对……熟悉}{dui4 shu2xi1}
    \definition{expr.}{estar familiarizado com\dots}
  \end{Phonetics}
\end{Entry}

\begin{Entry}{左}{5}{⼯}
  \begin{Phonetics}{左}{zuo3}[][HSK 1]
    \definition*{s.}{Sobrenome Zuo}
    \definition{adj.}{estranho; herético; não ortodoxo | errado; incorreto | diferente; contrário; oposto | progressista; revolucionário; politicamente e ideologicamente progressista; radical}
    \definition{s.}{a esquerda; o lado esquerdo | leste; na antiguidade, referia-se especificamente à direção leste (com base na orientação para o sul) | a esquerda; ala esquerda; refere-se a uma posição inferior (na antiguidade, a direita era considerada superior e a esquerda, inferior)}
    \definition{v.}{assistir; auxiliar}
  \end{Phonetics}
\end{Entry}

\begin{Entry}{左右}{5,5}{⼯、⼝}
  \begin{Phonetics}{左右}{zuo3you4}[][HSK 3]
    \definition{s.}{os lados esquerdo e direito; esquerda e direita, também indicam os arredores | atendentes; acompanhantes; as pessoas que o acompanham | aproximadamente; mais ou menos; por aí; usado após números para indicar uma estimativa, com o mesmo significado de 上下}
    \definition{v.}{controlar; manipular; influenciar; dominar}
  \seealsoref{上下}{shang4 xia4}
  \end{Phonetics}
\end{Entry}

\begin{Entry}{左边}{5,5}{⼯、⾡}
  \begin{Phonetics}{左边}{zuo3bian5}[][HSK 1]
    \definition{s.}{esquerda; o lado esquerdo}
  \end{Phonetics}
\end{Entry}

\begin{Entry}{左派}{5,9}{⼯、⽔}
  \begin{Phonetics}{左派}{zuo3pai4}
    \definition{s.}{(política) esquerda | esquerdista}
  \end{Phonetics}
\end{Entry}

\begin{Entry}{左面}{5,9}{⼯、⾯}
  \begin{Phonetics}{左面}{zuo3mian4}
    \definition{s.}{esquerda | lado esquerdo}
  \end{Phonetics}
\end{Entry}

\begin{Entry}{左倾}{5,10}{⼯、⼈}
  \begin{Phonetics}{左倾}{zuo3qing1}
    \definition{s.}{esquerdista | progressivo}
  \end{Phonetics}
\end{Entry}

\begin{Entry}{左袒}{5,10}{⼯、⾐}
  \begin{Phonetics}{左袒}{zuo3tan3}
    \definition{v.}{ser tendencioso | ser parcial para | favorecer um lado | tomar partido com}
  \end{Phonetics}
\end{Entry}

\begin{Entry}{左舷}{5,11}{⼯、⾈}
  \begin{Phonetics}{左舷}{zuo3xian2}
    \definition{s.}{porto (lado de um navio)}
  \end{Phonetics}
\end{Entry}

\begin{Entry}{左翼}{5,17}{⼯、⽻}
  \begin{Phonetics}{左翼}{zuo3yi4}
    \definition{s.}{esquerda (política)}
  \end{Phonetics}
\end{Entry}

\begin{Entry}{巧}{5}{⼯}
  \begin{Phonetics}{巧}{qiao3}[][HSK 3]
    \definition{adj.}{habilidoso; engenhoso; esperto | oportuno; coincidente; fortuito | astuto; enganoso; enganador; traiçoeiro; ardiloso | (de mão, língua) hábil; loquaz}
    \definition{s.}{(tecnologia, artesanato) habilidade; destreza}
  \end{Phonetics}
\end{Entry}

\begin{Entry}{巧合}{5,6}{⼯、⼝}
  \begin{Phonetics}{巧合}{qiao3he2}
    \definition{s.}{coincidência; (coisas) coincidentes ou idênticas}
  \end{Phonetics}
\end{Entry}

\begin{Entry}{巧克力}{5,7,2}{⼯、⼗、⼒}
  \begin{Phonetics}{巧克力}{qiao3ke4li4}[][HSK 4]
    \definition[块,颗,盒,包]{s.}{Empréstimo linguístico: chocolate; alimentos feitos com cacau em pó como principal matéria-prima, açúcar e especiarias}
  \end{Phonetics}
\end{Entry}

\begin{Entry}{巧妙}{5,7}{⼯、⼥}
  \begin{Phonetics}{巧妙}{qiao3miao4}[][HSK 6]
    \definition{adj.}{inteligente; engenhoso; (método ou técnica, etc.) inteligente, além do comum}
  \end{Phonetics}
\end{Entry}

\begin{Entry}{市}{5}{⼱}
  \begin{Phonetics}{市}{shi4}[][HSK 2]
    \definition{s.}{mercado; lugar onde se concentra o comércio | cidade; município; áreas densamente povoadas, com indústrias, comércio e cultura desenvolvidos | relativo ao sistema tradicional chinês de pesos e medidas; unidades administrativas, incluindo cidades sob jurisdição direta e cidades sob jurisdição provincial (ou autônoma) | unidade padrão de mercado; pertencente ao sistema municipal (unidades de medida) | preço de transação no mercado}
    \definition{v.}{comprar ou vender; fazer transações}
  \end{Phonetics}
\end{Entry}

\begin{Entry}{市中心}{5,4,4}{⼱、⼁、⼼}
  \begin{Phonetics}{市中心}{shi4zhong1xin1}
    \definition{s.}{centro da cidade}
  \end{Phonetics}
\end{Entry}

\begin{Entry}{市区}{5,4}{⼱、⼖}
  \begin{Phonetics}{市区}{shi4 qu1}[][HSK 4]
    \definition[个]{s.}{\emph{downtown}; centro da cidade; distrito urbano; áreas que ficam dentro dos limites da cidade e geralmente têm uma alta concentração de população e estoque de moradias}
  \end{Phonetics}
\end{Entry}

\begin{Entry}{市升}{5,4}{⼱、⼗}
  \begin{Phonetics}{市升}{shi4sheng1}
    \definition{clas.}{sheng; uma unidade tradicional de volume, equivalente a 1 litro ou 1,76 \emph{pints} ou 0,22 galão}
  \end{Phonetics}
\end{Entry}

\begin{Entry}{市尺}{5,4}{⼱、⼫}
  \begin{Phonetics}{市尺}{shi4 chi3}
    \definition{clas.}{chi, uma unidade tradicional de comprimento, equivalente a 0,333 metros ou 1,094 pés}
  \end{Phonetics}
\end{Entry}

\begin{Entry}{市斤}{5,4}{⼱、⽄}
  \begin{Phonetics}{市斤}{shi4jin1}
    \definition{clas.}{jin, uma unidade tradicional de peso, cada uma contendo 10 liang (市两)  e equivalente a 0,5 quilogramas ou 1,102 libras}
  \seealsoref{市两}{shi4liang3}
  \end{Phonetics}
\end{Entry}

\begin{Entry}{市长}{5,4}{⼱、⾧}
  \begin{Phonetics}{市长}{shi4 zhang3}[][HSK 2]
    \definition[个,位,名]{s.}{prefeito; chefe administrativo responsável pela administração de uma cidade}
  \end{Phonetics}
\end{Entry}

\begin{Entry}{市民}{5,5}{⼱、⽒}
  \begin{Phonetics}{市民}{shi4 min2}[][HSK 6]
    \definition[位,名]{s.}{habitantes da cidade; residente da cidade; moradores da cidade | cidadão; refere-se especificamente aos artesãos e comerciantes de pequeno e médio porte nas cidades da sociedade feudal tardia}
  \end{Phonetics}
\end{Entry}

\begin{Entry}{市场}{5,6}{⼱、⼟}
  \begin{Phonetics}{市场}{shi4chang3}[][HSK 3]
    \definition[家]{s.}{mercado (também no abstrato); um lugar fixo onde as pessoas compram e vendem coisas juntas | área de \emph{marketing}; região onde o produto é vendido | âmbito de influência (figurado); uma metáfora para o escopo e o grau em que uma determinada ideia ou comportamento é aceito por outros}
  \end{Phonetics}
\end{Entry}

\begin{Entry}{市两}{5,7}{⼱、⼀}
  \begin{Phonetics}{市两}{shi4liang3}
    \definition{clas.}{liang, uma unidade tradicional de peso, igual a 0,1 jin (市斤), e equivalente a 50 gramas ou 1,763 onças}
  \seealsoref{市斤}{shi4jin1}
  \end{Phonetics}
\end{Entry}

\begin{Entry}{市亩}{5,7}{⼱、⼇}
  \begin{Phonetics}{市亩}{shi4mu3}
    \definition{clas.}{mu, uma unidade tradicional de área, igual a 60 zhang quadrados (平方市丈) e equivalente a 6,667 ares ou 0,165 acre}
  \seealsoref{平方市丈}{ping2fang1 shi4 zhang4}
  \end{Phonetics}
\end{Entry}

\begin{Entry}{布}{5}{⼱}
  \begin{Phonetics}{布}{bu4}[][HSK 3]
    \definition*{s.}{Sobrenome Bu}
    \definition[块,幅,匹]{s.}{tecido; tecido de algodão; algodão, linho ou fibras sintéticas tecidas, que podem ser utilizadas como material para confecção de roupas ou outros objetos | uma moeda antiga | algo parecido com um pano}
    \definition{v.}{declarar; anunciar; publicar; proclamar | divulgar; espalhar por toda parte; difundir amplamente | implantar; dispor; organizar}
  \end{Phonetics}
\end{Entry}

\begin{Entry}{布谷鸟}{5,7,5}{⼱、⾕、⿃}
  \begin{Phonetics}{布谷鸟}{bu4gu3niao3}
    \definition{s.}{cuco (pássaro)}
  \seealsoref{杜鹃}{du4juan1}
  \seealsoref{杜鹃鸟}{du4juan1niao3}
  \seealsoref{杜宇}{du4yu3}
  \end{Phonetics}
\end{Entry}

\begin{Entry}{布满}{5,13}{⼱、⽔}
  \begin{Phonetics}{布满}{bu4 man3}[][HSK 6]
    \definition{v.}{abundar em; estar cheio de; espalhar-se e preencher um certo espaço}
  \end{Phonetics}
\end{Entry}

\begin{Entry}{布置}{5,13}{⼱、⽹}
  \begin{Phonetics}{布置}{bu4zhi4}[][HSK 4]
    \definition{v.}{arrumar; organizar; decorar; colocar adequadamente objetos ou paisagismo, conforme necessário | designar; tomar providências para; dar instruções sobre; organizar trabalho, atividades, etc.}
  \end{Phonetics}
\end{Entry}

\begin{Entry}{布署}{5,13}{⼱、⽹}
  \begin{Phonetics}{布署}{bu4shu3}
    \variantof{部署}
  \end{Phonetics}
\end{Entry}

\begin{Entry}{帅}{5}{⼱}
  \begin{Phonetics}{帅}{shuai4}[][HSK 4]
    \definition*{s.}{Sobrenome Shuai}
    \definition{adj.}{bonito; arrojado; elegante; inteligente}
    \definition{interj.}{Legal!}
    \definition[位,名,个,些]{s.}{comandante em chefe; o mais alto comandante do exército | comandante em chefe, a peça principal no xadrez chinês}
  \end{Phonetics}
\end{Entry}

\begin{Entry}{帅哥}{5,10}{⼱、⼝}
  \begin{Phonetics}{帅哥}{shuai4 ge1}[][HSK 4]
    \definition[个,位,名,些]{s.}{rapaz bonito; um garoto que é bonito e atraente na aparência}
  \end{Phonetics}
\end{Entry}

\begin{Entry}{平}{5}{⼲}
  \begin{Phonetics}{平}{ping2}[][HSK 2]
    \definition*{s.}{Sobrenome Ping}
    \definition{adj.}{plano; nivelado; uniforme; liso | igual; justo | mesma pontuação; empatado | médio; comum | silencioso; tranquilo | no mesmo nível; altura igual; sem diferença | imparcial; médio; equitativo | calmo; estável; tranquilo | comum;  frequente}
    \definition{s.}{no mesmo nível; em pé de igualdade com; igual | tom nivelado, um dos quatro tons do chinês clássico}
    \definition{v.}{tornar nivelado ou uniforme; nivelar | reprimir; suprimir | acalmar; tornar pacífico; silenciar (acalmar); conter a raiva | estar no mesmo nível | acalmar; amenizar; controlar a raiva}
  \end{Phonetics}
\end{Entry}

\begin{Entry}{平凡}{5,3}{⼲、⼏}
  \begin{Phonetics}{平凡}{ping2fan2}[][HSK 6]
    \definition{adj.}{comum; ordinário; normal; não surpreendente}
  \end{Phonetics}
\end{Entry}

\begin{Entry}{平方}{5,4}{⼲、⽅}
  \begin{Phonetics}{平方}{ping2fang1}[][HSK 4]
    \definition{s.}{Matemática: segunda potência (de uma quantidade); quadrado | metro quadrado (m²)}[那间房有十二平方。===O quarto tem doze metros quadrados.]
  \end{Phonetics}
\end{Entry}

\begin{Entry}{平方市丈}{5,4,5,3}{⼲、⽅、⼱、⼀}
  \begin{Phonetics}{平方市丈}{ping2fang1 shi4 zhang4}
    \definition{clas.}{pés quadrados}
  \end{Phonetics}
\end{Entry}

\begin{Entry}{平方米}{5,4,6}{⼲、⽅、⽶}
  \begin{Phonetics}{平方米}{ping2 fang1 mi3}[][HSK 6]
    \definition{s.}{metro quadrado; a unidade legal de medida de área, 1 metro quadrado é igual a 10.000 centímetros quadrados}
  \end{Phonetics}
  \begin{Phonetics}{平方米}{ping2fang1 mi3}
    \definition{clas.}{unidade de medida de área, 1 metro quadrado equivale a 10.000 centímetros quadrados}
  \end{Phonetics}
\end{Entry}

\begin{Entry}{平台}{5,5}{⼲、⼝}
  \begin{Phonetics}{平台}{ping2 tai2}[][HSK 6]
    \definition[个]{s.}{casa com telhado plano rebocado | terraço | plataforma móvel; metaforicamente, refere-se às áreas, oportunidades, ambientes, espaços, etc. que fornecem suporte e garantia para algo | plataforma; um sistema em um computador eletrônico que consiste em software e hardware básicos; tal sistema pode suportar a execução de programas aplicativos e softwares aplicativos podem ser desenvolvidos nesse sistema | plataforma; lugar; falando metaforicamente, o mesmo nível ou grau}
  \end{Phonetics}
\end{Entry}

\begin{Entry}{平地}{5,6}{⼲、⼟}
  \begin{Phonetics}{平地}{ping2di4}
    \definition{v.}{nivelar a terra | aplanar}
  \end{Phonetics}
\end{Entry}

\begin{Entry}{平安}{5,6}{⼲、⼧}
  \begin{Phonetics}{平安}{ping2'an1}[][HSK 2]
    \definition{s.}{seguro; bem; sem contratempos; sem acidentes; são e salvo}
  \end{Phonetics}
\end{Entry}

\begin{Entry}{平均}{5,7}{⼲、⼟}
  \begin{Phonetics}{平均}{ping2jun1}[][HSK 4]
    \definition{adj.}{igual; médio}
    \definition{s.}{média}
    \definition{v.}{calcular a média de um conjunto de números}
  \end{Phonetics}
\end{Entry}

\begin{Entry}{平时}{5,7}{⼲、⽇}
  \begin{Phonetics}{平时}{ping2shi2}[][HSK 2]
    \definition{s.}{em tempos normais; em tempos comuns | em tempo de paz; refere-se a períodos normais}
  \end{Phonetics}
\end{Entry}

\begin{Entry}{平坦}{5,8}{⼲、⼟}
  \begin{Phonetics}{平坦}{ping2tan3}[][HSK 5]
    \definition{adj.}{plano; uniforme; nivelado; liso; sem elevações ou depressões (referindo-se principalmente ao relevo)}
  \end{Phonetics}
\end{Entry}

\begin{Entry}{平原}{5,10}{⼲、⼚}
  \begin{Phonetics}{平原}{ping2yuan2}[][HSK 5]
    \definition[片,个]{s.}{campo; planície; terreno plano e extenso}
  \end{Phonetics}
\end{Entry}

\begin{Entry}{平常}{5,11}{⼲、⼱}
  \begin{Phonetics}{平常}{ping2chang2}[][HSK 2]
    \definition{adj.}{comum; normal; ordinário; nada de especial}
    \definition{adv.}{normalmente; geralmente; como regra geral}
  \end{Phonetics}
\end{Entry}

\begin{Entry}{平等}{5,12}{⼲、⽵}
  \begin{Phonetics}{平等}{ping2deng3}[][HSK 2]
    \definition{adj.}{igual; igualdade; refere-se ao fato de as pessoas gozarem de tratamento igualitário nos aspectos sociais, políticos, econômicos e jurídicos}
  \end{Phonetics}
\end{Entry}

\begin{Entry}{平稳}{5,14}{⼲、⽲}
  \begin{Phonetics}{平稳}{ping2 wen3}[][HSK 4]
    \definition{adj.}{firme; estável; suave e constante; sem oscilações ou flutuações}
  \end{Phonetics}
\end{Entry}

\begin{Entry}{平静}{5,14}{⼲、⾭}
  \begin{Phonetics}{平静}{ping2jing4}[][HSK 4]
    \definition{adj.}{(humor, ambiente, etc.) calmo; quieto; pacífico; tranquilo}
  \end{Phonetics}
\end{Entry}

\begin{Entry}{平衡}{5,16}{⼲、⾏}
  \begin{Phonetics}{平衡}{ping2 heng2}[][HSK 6]
    \definition{adj.}{balanceado; equilibrado; os aspectos opostos são iguais ou compensados ​​em quantidade ou qualidade | equilibrado; várias forças atuam sobre um objeto com magnitude igual e direções opostas para manter o objeto estável}
    \definition{v.}{equilibrar; trazer ou manter em equilíbrio; tornar as coisas ou alimentos iguais em quantidade, qualidade ou força}
  \end{Phonetics}
\end{Entry}

\begin{Entry}{幼}{5}{⼳}
  \begin{Phonetics}{幼}{you4}
    \definition{adj.}{jovem; menor de idade (oposto a 老)}
    \definition{s.}{crianças; os jovens}
  \seealsoref{老}{lao3}
  \end{Phonetics}
\end{Entry}

\begin{Entry}{幼儿园}{5,2,7}{⼳、⼉、⼞}
  \begin{Phonetics}{幼儿园}{you4'er2yuan2}[][HSK 4]
    \definition[家,所]{s.}{jardim de infância; escola maternal; escola infantil; instituição para a educação de crianças pequenas}
  \end{Phonetics}
\end{Entry}

\begin{Entry}{弘}{5}{⼸}
  \begin{Phonetics}{弘}{hong2}
    \definition{adj.}{grande; grandioso; magnífico}
    \definition{s.}{Sobrenome Hong}
    \definition{v.}{ampliar; expandir | promover}
  \end{Phonetics}
\end{Entry}

\begin{Entry}{归}{5}{⼹}
  \begin{Phonetics}{归}{gui1}[][HSK 4]
    \definition*{s.}{Sobrenome Gui}
    \definition{s.}{divisão no ábaco com divisor de um dígito}
    \definition{v.}{retornar; voltar para; voltar (ou ir) | devolver algo a; dar de volta a | convergir; juntar-se | encarregar alguém de algo | atribuir a; pertencer a}
    \definition{v.aux.}{usado entre dois verbos idênticos, indicando que a ação não levou ao resultado correspondente}
  \end{Phonetics}
\end{Entry}

\begin{Entry}{必}{5}{⼼}
  \begin{Phonetics}{必}{bi4}[][HSK 5]
    \definition{adv.}{certamente; necessariamente; indica que algo é certo ou que alguém acredita que esteja correto | deve; tem que}
  \end{Phonetics}
\end{Entry}

\begin{Entry}{必定}{5,8}{⼼、⼧}
  \begin{Phonetics}{必定}{bi4ding4}
    \definition{adv.}{sem falta | certamente | com certeza | definitivamente | inevitavelmente | com determinação}
    \definition{v.}{estar vinculado a | ter certeza de}
  \end{Phonetics}
\end{Entry}

\begin{Entry}{必修}{5,9}{⼼、⼈}
  \begin{Phonetics}{必修}{bi4 xiu1}[][HSK 6]
    \definition{adj.}{(de um curso acadêmico) obrigatório; compulsório; mandatório; obrigatório estudar de acordo com os regulamentos (em oposição a 选修)}
  \seealsoref{选修}{xuan3 xiu1}
  \end{Phonetics}
\end{Entry}

\begin{Entry}{必将}{5,9}{⼼、⼨}
  \begin{Phonetics}{必将}{bi4 jiang1}[][HSK 6]
    \definition{adv.}{certamente; certamente irá; usado para expressar inevitabilidade (ou necessidade)}
  \end{Phonetics}
\end{Entry}

\begin{Entry}{必要}{5,9}{⼼、⾑}
  \begin{Phonetics}{必要}{bi4yao4}[][HSK 3]
    \definition{adj.}{necessário; essencial; indispensável}
    \definition[个,些]{s.}{necessidade; características indispensáveis}
  \end{Phonetics}
\end{Entry}

\begin{Entry}{必须}{5,9}{⼼、⾴}
  \begin{Phonetics}{必须}{bi4xu1}[][HSK 2]
    \definition{adv.}{necessariamente; obrigatoriamente; indica a necessidade lógica e emocional | deve; tem que; é obrigado a}
  \end{Phonetics}
\end{Entry}

\begin{Entry}{必然}{5,12}{⼼、⽕}
  \begin{Phonetics}{必然}{bi4ran2}[][HSK 3]
    \definition{adj.}{certo; inevitável; necessário; definido e inalterável; imutável}
    \definition{adv.}{inevitavelmente}
    \definition{s.}{necessidade; em filosofia, refere-se às leis objetivas do desenvolvimento que não são influenciadas pela vontade humana}
  \end{Phonetics}
\end{Entry}

\begin{Entry}{必需}{5,14}{⼼、⾬}
  \begin{Phonetics}{必需}{bi4 xu1}[][HSK 5]
    \definition{adj.}{essencial; indispensável}
    \definition{v.}{ser essencial; ser indispensável}
  \end{Phonetics}
\end{Entry}

\begin{Entry}{扑}{5}{⼿}
  \begin{Phonetics}{扑}{pu1}[][HSK 6]
    \definition{s.}{sopro; refere-se a gases, fragrâncias, cinzas, areia, etc. que se apresentam | espanador}
    \definition{v.}{atacar; lançar-se sobre; correr para frente com toda a sua força e, de repente, jogar todo o seu corpo em um objeto | dedicar; dedicar todas as energias a uma causa; colocar toda a sua energia em (trabalho, carreira, etc.) | bater asas; esvoaçar | inclinar-se}
  \end{Phonetics}
\end{Entry}

\begin{Entry}{扑克}{5,7}{⼿、⼗}
  \begin{Phonetics}{扑克}{pu1ke4}
    \definition{s.}{(empréstimo linguístico) (jogo) \emph{poker}  | baralho}
  \end{Phonetics}
\end{Entry}

\begin{Entry}{扒}{5}{⼿}
  \begin{Phonetics}{扒}{ba1}
    \definition{v.}{segurar; agarrar-se a | cavar; varrer; puxar para baixo | empurrar para o lado | despir-se; tirar}
  \end{Phonetics}
  \begin{Phonetics}{扒}{pa2}
    \definition{v.}{reunir; juntar; reunir ou espalhar coisas com as mãos ou com um ancinho | roubar; furtar | arranhar; coçar com as mãos | cozinhar; refogar; cozinhar os alimentos em fogo baixo}
  \end{Phonetics}
\end{Entry}

\begin{Entry}{扒犁}{5,11}{⼿、⽜}
  \begin{Phonetics}{扒犁}{pa2li2}
    \definition{s.}{Dialeto: trenó; arado}
  \seealsoref{爬犁}{pa2li2}
  \end{Phonetics}
\end{Entry}

\begin{Entry}{打}{5}{⼿}
  \begin{Phonetics}{打}{da2}
    \definition{clas./s.}{(empréstimo linguístico) dúzia}
  \end{Phonetics}
  \begin{Phonetics}{打}{da3}[][HSK 1,4,5]
    \definition{prep.}{de; desde; ponto de partida que indica lugar, tempo ou extensão; indica rotas e locais percorridos | devido a; origem da introdução de coisas novas}
    \definition{v.}{golpear; acertar; bater | quebrar; esmagar | lutar; atacar; espancar | entrar com uma ação judicial; negociar; fazer representações | construir; edificar | fabricar (em uma ferraria); forjar | misturar; mexer; bater | amarrar; embalar | tricotar; tecer | desenhar; pintar; deixar uma marca; imprimir | abrir; perfurar; cavar | içar; levantar | enviar; despachar; projetar | emitir ou receber (um certificado, etc.) | remover; livrar-se de | colher; tirar; retirar | comprar | capturar; caçar | reunir; coletar; colher; recolher através de ações como cortar e podar | estimar; calcular; contar; determinar | fazer; envolver-se em | jogar algum tipo de jogo | expressar certos movimentos corporais | adotar; usar; adotar uma determinada abordagem | pegar (um táxi) | indicar a melhora de seu estado mental; melhorar o estado mental}
  \end{Phonetics}
\end{Entry}

\begin{Entry}{打工}{5,3}{⼿、⼯}
  \begin{Phonetics}{打工}{da3gong1}[][HSK 2]
    \definition{v.}{contratar para trabalhar; trabalhar em tempo parcial; realizar trabalho manual (para alguém, geralmente temporariamente)}
  \end{Phonetics}
\end{Entry}

\begin{Entry}{打工人}{5,3,2}{⼿、⼯、⼈}
  \begin{Phonetics}{打工人}{da3gong1ren2}
    \definition{s.}{trabalhador}
  \end{Phonetics}
\end{Entry}

\begin{Entry}{打开}{5,4}{⼿、⼶}
  \begin{Phonetics}{打开}{da3 kai1}[][HSK 1]
    \definition{v.}{abrir; desdobrar; desenrolar | descobrir; revelar; desvendar | ativar; ligar; ligar o circuito | romper | abrir-se; espalhar-se; expandir; ampliar | abrir; iniciar o funcionamento do software, etc.}
  \end{Phonetics}
\end{Entry}

\begin{Entry}{打车}{5,4}{⼿、⾞}
  \begin{Phonetics}{打车}{da3 che1}[][HSK 1]
    \definition{v.}{pegar um táxi; chamar um táxi; dar sinal para um táxi}
  \end{Phonetics}
\end{Entry}

\begin{Entry}{打击}{5,5}{⼿、⼐}
  \begin{Phonetics}{打击}{da3ji1}[][HSK 5]
    \definition{v.}{golpear; atacar; reprimir; atacar para frustrar; machucar | bater; bater (em um tambor, etc.); golpear ou bater em algo}
  \end{Phonetics}
\end{Entry}

\begin{Entry}{打包}{5,5}{⼿、⼓}
  \begin{Phonetics}{打包}{da3bao1}[][HSK 5]
    \definition{v.}{levar a comida embora; levar para viagem; refere-se especificamente a comer em um restaurante e levar as sobras em uma caixa, sacola ou outro recipiente | embalar; empacotar | desembalar; desempacotar}
  \end{Phonetics}
\end{Entry}

\begin{Entry}{打印}{5,5}{⼿、⼙}
  \begin{Phonetics}{打印}{da3yin4}[][HSK 2]
    \definition{v.}{imprimir; imprimir em papel ou outro suporte de gravação, como uma impressora}
  \end{Phonetics}
\end{Entry}

\begin{Entry}{打印机}{5,5,6}{⼿、⼙、⽊}
  \begin{Phonetics}{打印机}{da3 yin4 ji1}[][HSK 6]
    \definition[个,部,台]{s.}{impressora; uma máquina de escrever controlada por um microcomputador, sem teclado, que converte códigos de caracteres em caracteres e os imprime}
  \end{Phonetics}
\end{Entry}

\begin{Entry}{打发}{5,5}{⼿、⼜}
  \begin{Phonetics}{打发}{da3 fa5}[][HSK 6]
    \definition{v.}{enviar; despachar | dispensar; mandar embora | passar o tempo; matar o tempo}
  \end{Phonetics}
\end{Entry}

\begin{Entry}{打电话}{5,5,8}{⼿、⽥、⾔}
  \begin{Phonetics}{打电话}{da3 dian4 hua4}[][HSK 1]
    \definition{v.}{telefonar; fazer uma chamada telefônica; dar um telefonema}
  \seealsoref{给……打电话}{gei3 da3 dian4 hua4}
  \end{Phonetics}
\end{Entry}

\begin{Entry}{打动}{5,6}{⼿、⼒}
  \begin{Phonetics}{打动}{da3 dong4}[][HSK 6]
    \definition{v.}{mover; tocar}[这番话打动了她的心。===Essas palavras tocaram seu coração.]
  \end{Phonetics}
\end{Entry}

\begin{Entry}{打压}{5,6}{⼿、⼚}
  \begin{Phonetics}{打压}{da3ya1}
    \definition{v.}{reprimir | derrotar}
  \end{Phonetics}
\end{Entry}

\begin{Entry}{打扫}{5,6}{⼿、⼿}
  \begin{Phonetics}{打扫}{da3sao3}[][HSK 4]
    \definition{v.}{varrer; limpar; varrer para limpar}
  \end{Phonetics}
\end{Entry}

\begin{Entry}{打听}{5,7}{⼿、⼝}
  \begin{Phonetics}{打听}{da3ting5}[][HSK 3]
    \definition{v.}{perguntar sobre; indagar sobre; obter uma linha sobre}
  \end{Phonetics}
\end{Entry}

\begin{Entry}{打屁股}{5,7,8}{⼿、⼫、⾁}
  \begin{Phonetics}{打屁股}{da3pi4gu5}
    \definition{v.}{dar um tapa no bumbum de alguém}
  \end{Phonetics}
\end{Entry}

\begin{Entry}{打扮}{5,7}{⼿、⼿}
  \begin{Phonetics}{打扮}{da3ban5}[][HSK 5]
    \definition{s.}{estilo de se vestir; o modo de se vestir; as roupas que se usa}
    \definition{v.}{vestir-se bem; maquiar-se; dar uma boa aparência e vestir-se bem; adornar}
  \end{Phonetics}
\end{Entry}

\begin{Entry}{打扰}{5,7}{⼿、⼿}
  \begin{Phonetics}{打扰}{da3rao3}[][HSK 5]
    \definition{v.}{perturbar; incomodar; interferir no trabalho normal, na vida ou no que as outras pessoas estão fazendo, etc. | usado para expressar um pedido de desculpas por ajuda; gratidão por ajuda; hospitalidade recebida}
  \end{Phonetics}
\end{Entry}

\begin{Entry}{打折}{5,7}{⼿、⼿}
  \begin{Phonetics}{打折}{da3zhe2}[][HSK 4]
    \definition{v.+compl.}{dar desconto; dar um desconto; vender produtos a um preço reduzido em uma determinada porcentagem do preço original; metáfora para não cumprir 100\% do que foi originalmente padronizado ou prometido}
  \end{Phonetics}
\end{Entry}

\begin{Entry}{打针}{5,7}{⼿、⾦}
  \begin{Phonetics}{打针}{da3zhen1}[][HSK 4]
    \definition{v.+compl.}{dar ou receber uma injeção; injetar um medicamento líquido em um organismo com uma seringa}
  \end{Phonetics}
\end{Entry}

\begin{Entry}{打官司}{5,8,5}{⼿、⼧、⼝}
  \begin{Phonetics}{打官司}{da3guan1si5}[][HSK 6]
    \definition{v.+compl.}{ir ao tribunal (ou à lei); envolver-se em um processo judicial}
  \end{Phonetics}
\end{Entry}

\begin{Entry}{打的}{5,8}{⼿、⽩}
  \begin{Phonetics}{打的}{da3di1}
    \definition{v.+compl.}{(coloquial) pegar um táxi | ir de táxi}
  \end{Phonetics}
\end{Entry}

\begin{Entry}{打败}{5,8}{⼿、⾒}
  \begin{Phonetics}{打败}{da3 bai4}[][HSK 4]
    \definition{v.}{derrotar; vencer; piorar | sofrer uma derrota; ser derrotado}
  \end{Phonetics}
\end{Entry}

\begin{Entry}{打架}{5,9}{⼿、⽊}
  \begin{Phonetics}{打架}{da3jia4}[][HSK 5]
    \definition{v.+compl.}{brigar; discutir; entrar em conflito | contradizer; conflitar; ser inconsistente}
  \end{Phonetics}
\end{Entry}

\begin{Entry}{打结}{5,9}{⼿、⽷}
  \begin{Phonetics}{打结}{da3jie2}
    \definition{v.}{dar um nó | amarrar}
  \end{Phonetics}
\end{Entry}

\begin{Entry}{打骂}{5,9}{⼿、⾺}
  \begin{Phonetics}{打骂}{da3ma4}
    \definition{v.}{bater e repreender}
  \end{Phonetics}
\end{Entry}

\begin{Entry}{打破}{5,10}{⼿、⽯}
  \begin{Phonetics}{打破}{da3 po4}[][HSK 3]
    \definition{v.}{quebrar; esmagar; quebrar recordes, regras ou restrições existentes, etc.}
  \end{Phonetics}
\end{Entry}

\begin{Entry}{打造}{5,10}{⼿、⾡}
  \begin{Phonetics}{打造}{da3 zao4}[][HSK 6]
    \definition{v.}{forjar (trabalhar em metal); fabricar (principalmente objetos de metal) | fazer; criar; construir; desenvolver}
  \end{Phonetics}
\end{Entry}

\begin{Entry}{打断}{5,11}{⼿、⽄}
  \begin{Phonetics}{打断}{da3 duan4}[][HSK 6]
    \definition{v.}{interromper uma atividade (fala; pensamento ou ação) | fraturar (osso do corpo)  com força | arrombar; bater com força para quebrar}
  \end{Phonetics}
\end{Entry}

\begin{Entry}{打猎}{5,11}{⼿、⽝}
  \begin{Phonetics}{打猎}{da3lie4}
    \definition{v.}{ir caçar}
  \end{Phonetics}
\end{Entry}

\begin{Entry}{打球}{5,11}{⼿、⽟}
  \begin{Phonetics}{打球}{da3 qiu2}[][HSK 1]
    \definition{v.}{jogar bola (com as mãos) | jogar (basquetebol, handbol, etc.) | jogar um jogo de bola}
  \end{Phonetics}
\end{Entry}

\begin{Entry}{打搅}{5,12}{⼿、⼿}
  \begin{Phonetics}{打搅}{da3jiao3}
    \definition{v.}{perturbar | incomodar}
  \end{Phonetics}
\end{Entry}

\begin{Entry}{打牌}{5,12}{⼿、⽚}
  \begin{Phonetics}{打牌}{da3 pai2}[][HSK 6]
    \definition{v.}{jogar cartas, usar cartas para entretenimento ou jogos de azar}
  \end{Phonetics}
\end{Entry}

\begin{Entry}{打雷}{5,13}{⼿、⾬}
  \begin{Phonetics}{打雷}{da3 lei2}[][HSK 4]
    \definition{v.}{trovejar; produzir ruídos altos quando as nuvens descarregam eletricidade}
  \end{Phonetics}
\end{Entry}

\begin{Entry}{打算}{5,14}{⼿、⽵}
  \begin{Phonetics}{打算}{da3suan4}[][HSK 2]
    \definition[个,项]{s.}{plano; intenção; consideração; cálculo; ideias sobre a direção e os métodos da ação; pensamentos}
    \definition{v.}{pretender; planejar; calcular; considerar com antecedência}
  \end{Phonetics}
\end{Entry}

\begin{Entry}{打瞌睡}{5,15,13}{⼿、⽬、⽬}
  \begin{Phonetics}{打瞌睡}{da3ke1shui4}
    \definition{v.}{cochilar}
  \end{Phonetics}
\end{Entry}

\begin{Entry}{打磨}{5,16}{⼿、⽯}
  \begin{Phonetics}{打磨}{da3mo2}
    \definition{v.}{polir | fazer brilhar}
  \end{Phonetics}
\end{Entry}

\begin{Entry}{扔}{5}{⼿}
  \begin{Phonetics}{扔}{reng1}[][HSK 5]
    \definition{v.}{arremessar; lançar; atirar; jogar | esquecer; jogar fora; descartar | colocar casualmente; deixar as pessoas ou as coisas de lado, não se importar}
  \end{Phonetics}
\end{Entry}

\begin{Entry}{扔下}{5,3}{⼿、⼀}
  \begin{Phonetics}{扔下}{reng1xia4}
    \definition{v.}{lançar (uma bomba) | derrubar}
  \end{Phonetics}
\end{Entry}

\begin{Entry}{扔弃}{5,7}{⼿、⼶}
  \begin{Phonetics}{扔弃}{reng1qi4}
    \definition{v.}{abandonar | descartar | jogar fora}
  \end{Phonetics}
\end{Entry}

\begin{Entry}{扔掉}{5,11}{⼿、⼿}
  \begin{Phonetics}{扔掉}{reng1diao4}
    \definition{v.}{jogar fora}
  \end{Phonetics}
\end{Entry}

\begin{Entry}{斥}{5}{⽄}
  \begin{Phonetics}{斥}{chi4}
    \definition*{s.}{Sobrenome Chi}
    \definition{adj.}{(do solo) salino; alcalino}
    \definition{s.}{terra impregnada de sal, portanto estéril}
    \definition{v.}{repreender; censurar; denunciar; reprimir | repelir; excluir; expulsar | fornecer; prover | (literário) abrir; expandir | culpar; reprovar | estender; ampliar | (datado) reconhecer; detectar}
  \end{Phonetics}
\end{Entry}

\begin{Entry}{斥骂}{5,9}{⽄、⾺}
  \begin{Phonetics}{斥骂}{chi4ma4}
    \definition{v.}{repreender}
  \end{Phonetics}
\end{Entry}

\begin{Entry}{旧}{5}{⽇}
  \begin{Phonetics}{旧}{jiu4}[][HSK 3]
    \definition{adj.}{passado; antigo; velho; ultrapassado (em oposição a 新)| usado; desgastado; velho; descolorido ou deformado devido ao uso prolongado ou ao tempo | antigo; único; que já existiu; anterior}
    \definition{s.}{velha amizade; velho amigo}
  \seealsoref{新}{xin1}
  \end{Phonetics}
\end{Entry}

\begin{Entry}{未}{5}{⽊}
  \begin{Phonetics}{未}{wei4}
    \definition*{s.}{Sobrenome Wei}
    \definition{adv.}{Literário: não tem; não fez; (oposto a 已) | Literário: não}
    \definition{part.}{ou não; no final das perguntas, indicando dúvida}[今可以言未?===Posso falar agora?]
    \definition{s.}{wei (oitavo dos doze Ramos Terrestres)}
  \seealsoref{已}{yi3}
  \end{Phonetics}
\end{Entry}

\begin{Entry}{未必}{5,5}{⽊、⼼}
  \begin{Phonetics}{未必}{wei4bi4}[][HSK 4]
    \definition{adv.}{não tenho certeza; talvez não; não necessariamente}
  \end{Phonetics}
\end{Entry}

\begin{Entry}{未来}{5,7}{⽊、⽊}
  \begin{Phonetics}{未来}{wei4lai2}[][HSK 4]
    \definition{adj.}{próximo (refere-se ao tempo)}
    \definition[个,段,种]{s.}{futuro; o amanhã}
  \end{Phonetics}
\end{Entry}

\begin{Entry}{末}{5}{⽊}
  \begin{Phonetics}{末}{mo4}[][HSK 4]
    \definition{adj.}{último; final}
    \definition{s.}{ponta; terminal; extremidade; o final de algo | não essenciais; detalhes secundários | fim; final | pó; poeira | um papel na ópera tradicional}
  \end{Phonetics}
\end{Entry}

\begin{Entry}{本}{5}{⽊}
  \begin{Phonetics}{本}{ben3}[][HSK 1,6]
    \definition*{s.}{Sobrenome Ben}
    \definition{adj.}{original; inerente | principal; central}
    \definition{adv.}{originalmente}
    \definition{clas.}{para livros, dicionários, periódicos, arquivos, etc. | para vídeos de uma determinada duração | para peças de teatro, ópera}
    \definition{prep.}{de acordo com; em consonância com; em conformidade com; equivalentes a 依照 e 按照}
    \definition{pron.}{nativo; próprio; refere-se ao próprio interlocutor ou ao grupo, instituição, empresa, local, etc. ao qual o interlocutor pertence | isto; atual; presente}
    \definition[个]{s.}{caule ou raiz de plantas | base; origem; fundamento; fundação;  alicerce | capital; capital social | livro; caderno; livreto | edição; versão | cópia; roteiro; manuscrito | memorial do trono; na era feudal, referia-se a um documento oficial}
    \definition{v.}{seguir; basear-se em; estar de acordo com}
  \seealsoref{按照}{an4zhao4}
  \seealsoref{依照}{yi1 zhao4}
  \end{Phonetics}
\end{Entry}

\begin{Entry}{本人}{5,2}{⽊、⼈}
  \begin{Phonetics}{本人}{ben3ren2}[][HSK 5]
    \definition{pron.}{eu (mim, mim mesmo); o orador refere-se a si mesmo | a si mesmo; em pessoa; refere-se à própria pessoa ou à pessoa mencionada anteriormente}
  \end{Phonetics}
\end{Entry}

\begin{Entry}{本土}{5,3}{⽊、⼟}
  \begin{Phonetics}{本土}{ben3 tu3}[][HSK 6]
    \definition{s.}{território metropolitano; pátria-mãe; refere-se ao território do país | país (ou terra) natal de alguém; nativo; cidade natal; local original de crescimento}
  \end{Phonetics}
\end{Entry}

\begin{Entry}{本子}{5,3}{⽊、⼦}
  \begin{Phonetics}{本子}{ben3 zi5}[][HSK 1]
    \definition[个,本]{s.}{livro; caderno | edição | impressão | licença; certificado de competência emitido por uma instituição especializada, obtido após aprovação no exame | \emph{script}; roteiro}
  \end{Phonetics}
\end{Entry}

\begin{Entry}{本地}{5,6}{⽊、⼟}
  \begin{Phonetics}{本地}{ben3 di4}[][HSK 6]
    \definition{s.}{local; nativo; localidade; a área onde as pessoas e as coisas estão localizadas; uma área específica referida em uma narrativa}
  \end{Phonetics}
\end{Entry}

\begin{Entry}{本来}{5,7}{⽊、⽊}
  \begin{Phonetics}{本来}{ben3lai2}[][HSK 3]
    \definition{adj.}{original}
    \definition{adv.}{anteriormente; originalmente; indica que antes disso | claro; em primeiro lugar; como deveria ser; indica que algo é natural ou óbvio}
  \end{Phonetics}
\end{Entry}

\begin{Entry}{本身}{5,7}{⽊、⾝}
  \begin{Phonetics}{本身}{ben3shen1}[][HSK 6]
    \definition{pron.}{próprio; em si mesmo; refere-se à pessoa, unidade ou coisa em si}
  \end{Phonetics}
\end{Entry}

\begin{Entry}{本事}{5,8}{⽊、⼅}
  \begin{Phonetics}{本事}{ben3shi4}
    \definition{s.}{habilidade; aptidão; capacidade; competência; refere-se às habilidades, capacidades ou talentos que uma pessoa possui em determinada área | habilidade; aptidão; capacidade; competência; a capacidade e os meios necessários para atingir um determinado objetivo ou concluir uma determinada tarefa | status; poder; posição; autoridade; refere-se à identidade, posição ou poder de uma pessoa.}
  \end{Phonetics}
  \begin{Phonetics}{本事}{ben3shi5}[][HSK 3]
    \definition{s.}{habilidade; capacidade; talento; aptidão}
  \end{Phonetics}
\end{Entry}

\begin{Entry}{本质}{5,8}{⽊、⾙}
  \begin{Phonetics}{本质}{ben3zhi4}[][HSK 6]
    \definition{s.}{essência; natureza; caráter inato; qualidade intrínseca; refere-se aos atributos fundamentais inerentes às próprias coisas, que desempenham um papel decisivo na natureza, condição e desenvolvimento das coisas (distinguido de 现象)}
  \seealsoref{现象}{xian4xiang4}
  \end{Phonetics}
\end{Entry}

\begin{Entry}{本金}{5,8}{⽊、⾦}
  \begin{Phonetics}{本金}{ben3 jin1}
    \definition{s.}{capital; capital para a operação do comércio e da indústria; capital para a operação de negócios | valor principal; dinheiro retirado ao depositar ou tomar emprestado (diferente de 利息)}
  \seealsoref{利息}{li4xi1}
  \end{Phonetics}
\end{Entry}

\begin{Entry}{本科}{5,9}{⽊、⽲}
  \begin{Phonetics}{本科}{ben3ke1}[][HSK 4]
    \definition{s.}{graduação; bacharelado; o curso básico de uma universidade ou faculdade}
  \end{Phonetics}
\end{Entry}

\begin{Entry}{本领}{5,11}{⽊、⾴}
  \begin{Phonetics}{本领}{ben3 ling3}[][HSK 3]
    \definition[项,个,种]{s.}{habilidade; capacidade; faculdade; poder; destreza; talento}
  \end{Phonetics}
\end{Entry}

\begin{Entry}{本期}{5,12}{⽊、⽉}
  \begin{Phonetics}{本期}{ben3 qi1}[][HSK 6]
    \definition{adv.}{o período atual | este prazo (geralmente em finanças)}
  \end{Phonetics}
\end{Entry}

\begin{Entry}{术}{5}{⽊}
  \begin{Phonetics}{术}{shu4}
    \definition*{s.}{Sobrenome Shu}
    \definition{s.}{arte; habilidade; técnica; tecnologia; acadêmico | método; tática; estratégia}
  \end{Phonetics}
  \begin{Phonetics}{术}{zhu2}
    \definition{s.}{vários gêneros de flores da família Asteraceae (margaridas e crisântemos)}
  \end{Phonetics}
\end{Entry}

\begin{Entry}{术科}{5,9}{⽊、⽲}
  \begin{Phonetics}{术科}{shu4ke1}
    \definition{s.}{cursos técnicos oferecidos em treinamento militar ou físico (oposto a 学科)}
  \seealsoref{学科}{xue2 ke1}
  \end{Phonetics}
\end{Entry}

\begin{Entry}{正}{5}{⽌}
  \begin{Phonetics}{正}{zheng1}
    \definition{s.}{o primeiro mês do ano lunar; a primeira lua}
  \end{Phonetics}
  \begin{Phonetics}{正}{zheng4}[][HSK 1,3]
    \definition*{s.}{Sobrenome Zheng}
    \definition{adj.}{reto; ereto; vertical | principal; posicionado no meio | direito; anverso | honesto; íntegro; justo | puro; sem mistura (de cor ou sabor) | regular; padronizado; de acordo com a lei; correto | chefe; comandante; diretor | regular; as laterais e os ângulos do gráfico têm comprimentos e tamanhos iguais | positivo; (matemática), significa maior que zero; (física) significa perda de elétrons (oposto de 负) | exato; preciso; usado para indicar tempo, refere-se ao momento exato ou ao ponto médio de um período}
    \definition{adv.}{apenas; certo; exatamente; precisamente | agora mesmo; neste momento; indica a continuidade de uma ação ou a permanência de um estado}
    \definition{v.}{definir (colocar) corretamente; alinhar; endireitar | ajustar; corrigir; retificar}
  \seealsoref{负}{fu4}
  \end{Phonetics}
\end{Entry}

\begin{Entry}{正义}{5,3}{⽌、⼂}
  \begin{Phonetics}{正义}{zheng4yi4}[][HSK 5]
    \definition{adj.}{justo; íntegro}
    \definition{s.}{justiça; o que é certo; o que é benéfico para o povo | (frequentemente em títulos de livros) interpretação ortodoxa ou retificada (de textos antigos)}
  \end{Phonetics}
\end{Entry}

\begin{Entry}{正正}{5,5}{⽌、⽌}
  \begin{Phonetics}{正正}{zheng4zheng4}
    \definition{adv.}{na hora certa | ordenadamente}
  \end{Phonetics}
\end{Entry}

\begin{Entry}{正在}{5,6}{⽌、⼟}
  \begin{Phonetics}{正在}{zheng4zai4}[][HSK 1]
    \definition{adv.}{em processo de; em andamento; indica que uma ação está em andamento ou que uma situação está em curso.}
    \definition{v.}{estar a + {v.inf.} | estar + {v.ger.}}
  \end{Phonetics}
\end{Entry}

\begin{Entry}{正好}{5,6}{⽌、⼥}
  \begin{Phonetics}{正好}{zheng4hao3}[][HSK 2]
    \definition{adj.}{na hora certa; na hora certa; o suficiente}
    \definition{adv.}{acontecer com; chance de; como acontece}
  \end{Phonetics}
\end{Entry}

\begin{Entry}{正如}{5,6}{⽌、⼥}
  \begin{Phonetics}{正如}{zheng4 ru2}[][HSK 5]
    \definition{adv.}{exatamente como; assim como}
  \end{Phonetics}
\end{Entry}

\begin{Entry}{正式}{5,6}{⽌、⼷}
  \begin{Phonetics}{正式}{zheng4shi4}[][HSK 3]
    \definition{adj.}{formal; oficial; descreve uma atmosfera séria, atitudes ou comportamentos que não são fáceis ou descontraídos | formal; oficial; descreve o cumprimento de determinados trâmites e procedimentos}
  \end{Phonetics}
\end{Entry}

\begin{Entry}{正当}{5,6}{⽌、⼹}
  \begin{Phonetics}{正当}{zheng4dang1}[][HSK 6]
    \definition{adj./adv.}{exatamente quando; exatamente o momento para}
  \end{Phonetics}
  \begin{Phonetics}{正当}{zheng4dang4}[][HSK 6]
    \definition{adv.}{exatamente quando; exatamente o momento para; está em (um certo período ou estágio)}
  \end{Phonetics}
\end{Entry}

\begin{Entry}{正宗}{5,8}{⽌、⼧}
  \begin{Phonetics}{正宗}{zheng4zong1}
    \definition{adj.}{autêntico | genuíno | \emph{old school} | (fig.) tradicional}
  \end{Phonetics}
\end{Entry}

\begin{Entry}{正版}{5,8}{⽌、⽚}
  \begin{Phonetics}{正版}{zheng4 ban3}[][HSK 5]
    \definition{s.}{versão genuína; versão autorizada; versão publicada e distribuída oficialmente por uma editora legal (em contraste com a 盗版)}
  \seealsoref{盗版}{dao4 ban3}
  \end{Phonetics}
\end{Entry}

\begin{Entry}{正规}{5,8}{⽌、⾒}
  \begin{Phonetics}{正规}{zheng4gui1}[][HSK 5]
    \definition{adj.}{normal; regular; padrão; está em conformidade com padrões formalmente definidos ou geralmente reconhecidos}
  \end{Phonetics}
\end{Entry}

\begin{Entry}{正是}{5,9}{⽌、⽇}
  \begin{Phonetics}{正是}{zheng4 shi4}[][HSK 2]
    \definition{v.}{ser precisamente; ser exatamente}
  \end{Phonetics}
\end{Entry}

\begin{Entry}{正常}{5,11}{⽌、⼱}
  \begin{Phonetics}{正常}{zheng4chang2}[][HSK 2]
    \definition{adj.}{normal; regular; conforma-se com regras ou circunstâncias gerais}
  \end{Phonetics}
\end{Entry}

\begin{Entry}{正确}{5,12}{⽌、⽯}
  \begin{Phonetics}{正确}{zheng4que4}[][HSK 2]
    \definition{adj.}{correto; certo; próprio; conforma-se com fatos, razão ou algum padrão geralmente aceito}
  \end{Phonetics}
\end{Entry}

\begin{Entry}{母}{5}{⽏}[Kangxi 80]
  \begin{Phonetics}{母}{mu3}[][HSK 6]
    \definition*{s.}{Sobrenome Mu}
    \definition{adj.}{fêmea}
    \definition[位,名,个,些]{s.}{mãe | fêmea (animal) (oposto a 公) | origem; pais | parentes idosas; geralmente se refere a mulheres idosas | côncavo | fonte; algo que tem a capacidade ou função de produzir outras coisas}
  \seealsoref{公}{gong1}
  \end{Phonetics}
\end{Entry}

\begin{Entry}{母女}{5,3}{⽏、⼥}
  \begin{Phonetics}{母女}{mu3 nv3}[][HSK 6]
    \definition{s.}{mãe e filha}
  \end{Phonetics}
\end{Entry}

\begin{Entry}{母子}{5,3}{⽏、⼦}
  \begin{Phonetics}{母子}{mu3 zi3}[][HSK 6]
    \definition{s.}{mãe e filho}
  \end{Phonetics}
\end{Entry}

\begin{Entry}{母鸡}{5,7}{⽏、⿃}
  \begin{Phonetics}{母鸡}{mu3ji1}[][HSK 6]
    \definition{s.}{galinha}
  \end{Phonetics}
\end{Entry}

\begin{Entry}{母亲}{5,9}{⽏、⼇}
  \begin{Phonetics}{母亲}{mu3qin1}[][HSK 3]
    \definition[位,名,个,些]{s.}{mãe}
  \end{Phonetics}
\end{Entry}

\begin{Entry}{母语}{5,9}{⽏、⾔}
  \begin{Phonetics}{母语}{mu3yu3}
    \definition{s.}{língua materna | língua nativa}
  \end{Phonetics}
\end{Entry}

\begin{Entry}{民}{5}{⽒}
  \begin{Phonetics}{民}{min2}
    \definition*{s.}{Sobrenome Min}
    \definition{adj.}{folclórico ; civil (não militar)}
    \definition{s.}{pessoa | membro de um grupo étnico | uma pessoa de uma determinada ocupação | do povo; folclore | civil; cidadão | o povo | um membro de uma nacionalidade}
  \end{Phonetics}
\end{Entry}

\begin{Entry}{民工}{5,3}{⽒、⼯}
  \begin{Phonetics}{民工}{min2 gong1}[][HSK 6]
    \definition{s.}{trabalhador trabalhando em um projeto público | trabalhador temporário alistado em um projeto público | agricultor que trabalha em empregos temporários na cidade | trabalhador migrante}
  \end{Phonetics}
\end{Entry}

\begin{Entry}{民主}{5,5}{⽒、⼂}
  \begin{Phonetics}{民主}{min2zhu3}[][HSK 6]
    \definition{adj.}{democrático; em consonância com os princípios democráticos}
    \definition[个]{s.}{democracia; direitos democráticos; refere-se ao direito do povo de participar da vida política e dos assuntos do Estado e de expressar livremente suas opiniões}
  \end{Phonetics}
\end{Entry}

\begin{Entry}{民众}{5,6}{⽒、⼈}
  \begin{Phonetics}{民众}{min2zhong4}
    \definition{s.}{a população | as massas | as pessoas comuns}
  \end{Phonetics}
\end{Entry}

\begin{Entry}{民间}{5,7}{⽒、⾨}
  \begin{Phonetics}{民间}{min2jian1}[][HSK 3]
    \definition{s.}{entre o povo | não governamental; de pessoa para pessoa}
  \end{Phonetics}
\end{Entry}

\begin{Entry}{民族}{5,11}{⽒、⽅}
  \begin{Phonetics}{民族}{min2zu2}[][HSK 3]
    \definition[个]{s.}{nação; uma comunidade estável formada ao longo da história pela humanidade, com uma língua comum, uma região comum, uma vida econômica comum e uma mentalidade comum expressa em uma cultura comum | grupo étnico; refere-se, de maneira geral, às comunidades formadas ao longo da história por pessoas em diferentes estágios de desenvolvimento social}
  \end{Phonetics}
\end{Entry}

\begin{Entry}{民意}{5,13}{⽒、⼼}
  \begin{Phonetics}{民意}{min2 yi4}[][HSK 6]
    \definition{s.}{vontade do povo; vontade popular | opinião pública}
  \end{Phonetics}
\end{Entry}

\begin{Entry}{民歌}{5,14}{⽒、⽋}
  \begin{Phonetics}{民歌}{min2 ge1}[][HSK 6]
    \definition[支,首]{s.}{canção folclórica; os nomes dos autores das canções transmitidas oralmente são muitas vezes desconhecidos}
  \end{Phonetics}
\end{Entry}

\begin{Entry}{民警}{5,19}{⽒、⾔}
  \begin{Phonetics}{民警}{min2 jing3}[][HSK 6]
    \definition{s.}{polícia; policial}
  \end{Phonetics}
\end{Entry}

\begin{Entry}{永}{5}{⽔}
  \begin{Phonetics}{永}{yong3}
    \definition*{s.}{Sobrenome Yong}
    \definition{adj.}{sempre; para sempre; perpetuamente}
    \definition{adv.}{para sempre; significa um tempo muito longo sem fim, o que equivale a 永远}
  \seealsoref{永远}{yong3yuan3}
  \end{Phonetics}
\end{Entry}

\begin{Entry}{永不}{5,4}{⽔、⼀}
  \begin{Phonetics}{永不}{yong3bu4}
    \definition{adv.}{nunca}
  \end{Phonetics}
\end{Entry}

\begin{Entry}{永远}{5,7}{⽔、⾡}
  \begin{Phonetics}{永远}{yong3yuan3}[][HSK 2]
    \definition{adv.}{sempre; para sempre; Indica um longo período de tempo sem fim}
    \definition{s.}{eternidade; um futuro que nunca acaba}
  \end{Phonetics}
\end{Entry}

\begin{Entry}{汇}{5}{⽔}
  \begin{Phonetics}{汇}{hui4}[][HSK 4]
    \definition{s.}{montagem; coleção; coisas coletadas}
    \definition{v.}{convergir | reunir; coletar | remeter | trocar (câmbio de moedas)}
  \end{Phonetics}
\end{Entry}

\begin{Entry}{汇报}{5,7}{⽔、⼿}
  \begin{Phonetics}{汇报}{hui4bao4}[][HSK 4]
    \definition[份,次]{s.}{relatório; referindo-se ao conteúdo de declarações escritas ou orais feitas a um superior ou pessoa relevante para apresentar uma situação ou refletir um problema}
    \definition{v.}{relatar; fazer um relato de}
  \end{Phonetics}
\end{Entry}

\begin{Entry}{汇率}{5,11}{⽔、⽞}
  \begin{Phonetics}{汇率}{hui4lv4}[][HSK 4]
    \definition[个,种]{s.}{taxa de câmbio; relação entre a moeda de um país e a de outro}
  \end{Phonetics}
\end{Entry}

\begin{Entry}{汇款}{5,12}{⽔、⽋}
  \begin{Phonetics}{汇款}{hui4 kuan3}[][HSK 5]
    \definition[笔,个]{s.}{remessa; dinheiro enviado ou recebido}
    \definition{v.+compl.}{remeter dinheiro; fazer uma remessa; enviar dinheiro}
  \end{Phonetics}
\end{Entry}

\begin{Entry}{汉}{5}{⽔}
  \begin{Phonetics}{汉}{han4}
    \definition*{s.}{Dinastia Han (206 a.C.-220 d.C.)  | Astronomia: A Via Láctea | Sobrenome Han}
    \definition{s.}{grupo étnico Han | chinês (língua) | homem}
  \end{Phonetics}
\end{Entry}

\begin{Entry}{汉字}{5,6}{⽔、⼦}
  \begin{Phonetics}{汉字}{han4 zi4}[][HSK 1]
    \definition[个]{s.}{caractere chinês; ideograma chinês; sinograma; com pouquíssimas exceções, os caracteres chineses representam uma sílaba cada um}
  \end{Phonetics}
\end{Entry}

\begin{Entry}{汉服}{5,8}{⽔、⽉}
  \begin{Phonetics}{汉服}{han4fu2}
    \definition{s.}{vestido chinês tradicional Han}
  \end{Phonetics}
\end{Entry}

\begin{Entry}{汉语}{5,9}{⽔、⾔}
  \begin{Phonetics}{汉语}{han4yu3}[][HSK 1]
    \definition[门]{s.}{língua chinesa, mandarim}
  \end{Phonetics}
\end{Entry}

\begin{Entry}{汉堡王}{5,12,4}{⽔、⼟、⽟}
  \begin{Phonetics}{汉堡王}{han4bao3wang2}
    \definition*{s.}{Burguer King, restaurante de \emph{fast-food}}
  \end{Phonetics}
\end{Entry}

\begin{Entry}{汉堡包}{5,12,5}{⽔、⼟、⼓}
  \begin{Phonetics}{汉堡包}{han4bao3bao1}
    \definition[个]{s.}{hambúrguer}
  \end{Phonetics}
\end{Entry}

\begin{Entry}{汉葡词典}{5,12,7,8}{⽔、⾋、⾔、⼋}
  \begin{Phonetics}{汉葡词典}{han4-pu2 ci2dian3}
    \definition[部,本]{s.}{dicionário chinês-português}
  \seealsoref{葡汉词典}{pu2-han4 ci2dian3}
  \end{Phonetics}
\end{Entry}

\begin{Entry}{灭}{5}{⽕}
  \begin{Phonetics}{灭}{mie4}[][HSK 6]
    \definition{v.}{extinguir-se | extinguir; apagar; desligar | afogar; inundar; submergir | perecer; destruir | exterminar; apagar; acabar com; tornar inexistente}
  \end{Phonetics}
\end{Entry}

\begin{Entry}{灭火}{5,4}{⽕、⽕}
  \begin{Phonetics}{灭火}{mie4huo3}
    \definition{s.}{combate a incêndios}
    \definition{v.}{extinguir um incêndio}
  \end{Phonetics}
\end{Entry}

\begin{Entry}{犯}{5}{⽝}
  \begin{Phonetics}{犯}{fan4}[][HSK 6]
    \definition{s.}{criminoso}
    \definition{v.}{ofender; violar; ir contra | atacar; violar; trabalhar contra | fazer; ocorrer | voltar a; ter uma recorrência de; recair; retornar a (velhos hábitos)}
  \end{Phonetics}
\end{Entry}

\begin{Entry}{犯法}{5,8}{⽝、⽔}
  \begin{Phonetics}{犯法}{fan4fa3}
    \definition{v.}{violar (quebrar) a lei}
  \end{Phonetics}
\end{Entry}

\begin{Entry}{犯规}{5,8}{⽝、⾒}
  \begin{Phonetics}{犯规}{fan4 gui1}[][HSK 6]
    \definition{v.}{quebrar as regras; violar regras | Esporte: cometer uma falta contra}
  \end{Phonetics}
\end{Entry}

\begin{Entry}{犯罪}{5,13}{⽝、⽹}
  \begin{Phonetics}{犯罪}{fan4 zui4}[][HSK 6]
    \definition{v.+compl.}{cometer  um crime}
  \end{Phonetics}
\end{Entry}

\begin{Entry}{玄}{5}{⽞}[Kangxi 95]
  \begin{Phonetics}{玄}{xuan2}
    \definition*{s.}{Sobrenome Xuan}
    \definition{adj.}{preto; escuro | profundo; abstruso; escondido | não confiável; irrealista; não confiável}
  \end{Phonetics}
\end{Entry}

\begin{Entry}{玄学}{5,8}{⽞、⼦}
  \begin{Phonetics}{玄学}{xuan2xue2}
    \definition{s.}{Escola Philosófica Wei e Jin amalgamando os ideais daoísta e confucionistas | tradução da metafísica (形而上学) | Datado: metafísica}
  \seealsoref{形而上学}{xing2'er2shang4xue2}
  \end{Phonetics}
\end{Entry}

\begin{Entry}{玉}{5}{⽟}[Kangxi 96]
  \begin{Phonetics}{玉}{yu4}[][HSK 4]
    \definition*{s.}{Sobrenome Yu}
    \definition{adj.}{(pessoa, especialmente uma mulher) pura; justa; bonita; bela | cristalino, branco e belo como o jade | (vida) rica; luxuosa}
    \definition{pron.}{seu; um termo de respeito, usado para honrar o corpo, as ações ou as coisas associadas à outra pessoa}
    \definition[块,种]{s.}{jade}
  \end{Phonetics}
\end{Entry}

\begin{Entry}{玉米}{5,6}{⽟、⽶}
  \begin{Phonetics}{玉米}{yu4mi3}[][HSK 4]
    \definition[根,粒,棵,片]{s.}{milho}
  \end{Phonetics}
\end{Entry}

\begin{Entry}{玉米片}{5,6,4}{⽟、⽶、⽚}
  \begin{Phonetics}{玉米片}{yu4mi3pian4}
    \definition{s.}{flocos de milho | chips de tortilha}
  \end{Phonetics}
\end{Entry}

\begin{Entry}{玉米花}{5,6,7}{⽟、⽶、⾋}
  \begin{Phonetics}{玉米花}{yu4mi3hua1}
    \definition{s.}{pipoca}
  \end{Phonetics}
\end{Entry}

\begin{Entry}{玉米面}{5,6,9}{⽟、⽶、⾯}
  \begin{Phonetics}{玉米面}{yu4mi3mian4}
    \definition{s.}{fubá | farinha de milho}
  \end{Phonetics}
\end{Entry}

\begin{Entry}{玉米饼}{5,6,9}{⽟、⽶、⾷}
  \begin{Phonetics}{玉米饼}{yu4mi3bing3}
    \definition{s.}{tortilha mexicana | bolo de milho}
  \end{Phonetics}
\end{Entry}

\begin{Entry}{玉米笋}{5,6,10}{⽟、⽶、⽵}
  \begin{Phonetics}{玉米笋}{yu4mi3 sun3}
    \definition{s.}{broto de milho}
  \end{Phonetics}
\end{Entry}

\begin{Entry}{玉米粉}{5,6,10}{⽟、⽶、⽶}
  \begin{Phonetics}{玉米粉}{yu4mi3fen3}
    \definition{s.}{amido de milho | farinha de milho}
  \end{Phonetics}
\end{Entry}

\begin{Entry}{玉米糁}{5,6,14}{⽟、⽶、⽶}
  \begin{Phonetics}{玉米糁}{yu4mi3 san3}
    \definition{s.}{grãos de milho}
  \end{Phonetics}
\end{Entry}

\begin{Entry}{玉米糕}{5,6,16}{⽟、⽶、⽶}
  \begin{Phonetics}{玉米糕}{yu4mi3gao1}
    \definition{s.}{bolo de milho | polenta}
  \end{Phonetics}
\end{Entry}

\begin{Entry}{瓜}{5}{⽠}[Kangxi 97]
  \begin{Phonetics}{瓜}{gua1}[][HSK 4]
    \definition*{s.}{Sobrenome Gua}
    \definition[个]{s.}{qualquer tipo de melão ou cabaça | companheiro (termo depreciativo para uma pessoa)}
    \definition{v.}{fofocar}
  \end{Phonetics}
\end{Entry}

\begin{Entry}{甘}{5}{⽢}[Kangxi 99]
  \begin{Phonetics}{甘}{gan1}
    \definition*{s.}{Província de Gansu, abreviação de 甘肃 | Sobrenome Gan}
    \definition{adj.}{doce; agradável; satisfatório}
    \definition{v.}{estar disposto a; estar contente ou satisfeito com}
  \seealsoref{甘肃}{gan1su4}
  \end{Phonetics}
\end{Entry}

\begin{Entry}{甘心}{5,4}{⽢、⼼}
  \begin{Phonetics}{甘心}{gan1xin1}
    \definition{v.}{estar disposto a | resignar-se a}
  \end{Phonetics}
\end{Entry}

\begin{Entry}{甘肃}{5,8}{⽢、⾀}
  \begin{Phonetics}{甘肃}{gan1su4}
    \definition*{s.}{Província de Gansu}
  \end{Phonetics}
\end{Entry}

\begin{Entry}{甘薯}{5,16}{⽢、⾋}
  \begin{Phonetics}{甘薯}{gan1shu3}
    \definition{s.}{batata doce}
  \end{Phonetics}
\end{Entry}

\begin{Entry}{生}{5}{⽣}[Kangxi 100]
  \begin{Phonetics}{生}{sheng1}[][HSK 2,3]
    \definition*{s.}{Sobrenome Sheng}
    \definition{adj.}{vivo; vital | verde; não maduro | cru; não cozido; mal cozido | bruto; não refinado; não processado | estranho; desconhecido; não familiarizado | rígido; mecânico; forçado}
    \definition{adv.}{muito; usado antes de certas palavras que expressam emoções e sentimentos | verdadeiramente; realmente; forçosamente}
    \definition{s.}{vida | meio de subsistência | aluno; estudante | estudioso; antigamente chamados de eruditos | o tipo de personagem masculino na ópera de Pequim, etc.}
    \definition{suf.}{certos sufixos substantivos que se referem a pessoas (学生) | sufixos de certos advérbios (好生)}
    \definition{v.}{dar à luz; ter um filho | nascer | crescer; cultivar | viver; existir; sobreviver | favorecer; gerar; ocorrer | acender (uma fogueira); fazer o combustível queimar}
  \seealsoref{好生}{hao3sheng1}
  \seealsoref{学生}{xue2sheng5}
  \end{Phonetics}
\end{Entry}

\begin{Entry}{生日}{5,4}{⽣、⽇}
  \begin{Phonetics}{生日}{sheng1ri4}[][HSK 1]
    \definition[个,次]{s.}{aniversário; dia de nascimento, também se refere ao dia em que se completa um ano de idade a cada ano}
  \end{Phonetics}
\end{Entry}

\begin{Entry}{生气}{5,4}{⽣、⽓}
  \begin{Phonetics}{生气}{sheng1 qi4}[][HSK 1]
    \definition{s.}{vitalidade; vigor; energia da vida}
    \definition{v.+compl.}{ficar com raiva; ficar ofendido; ficar zangado; encontrar algo que não é do seu agrado e sentir-se descontente}
  \end{Phonetics}
\end{Entry}

\begin{Entry}{生长}{5,4}{⽣、⾧}
  \begin{Phonetics}{生长}{sheng1zhang3}[][HSK 3]
    \definition{v.}{cresçer; sob certas condições de vida, o volume e o peso dos organismos aumentam gradualmente | nascer e crescer}
  \end{Phonetics}
\end{Entry}

\begin{Entry}{生产}{5,6}{⽣、⼇}
  \begin{Phonetics}{生产}{sheng1chan3}[][HSK 3]
    \definition{v.}{produzir; fabricar; utilizar ferramentas para mudar o objeto de trabalho e criar meios de produção e meios de subsistência | dar à luz uma criança; ter filhos}
  \end{Phonetics}
\end{Entry}

\begin{Entry}{生动}{5,6}{⽣、⼒}
  \begin{Phonetics}{生动}{sheng1dong4}[][HSK 3]
    \definition{adj.}{vívido; animado; descreve a linguagem e as formas de expressão como sendo ativas e em movimento}
  \end{Phonetics}
\end{Entry}

\begin{Entry}{生存}{5,6}{⽣、⼦}
  \begin{Phonetics}{生存}{sheng1cun2}[][HSK 3]
    \definition{v.}{viver; sobreviver; subsistir; manter a vida; estar vivo}
  \end{Phonetics}
\end{Entry}

\begin{Entry}{生成}{5,6}{⽣、⼽}
  \begin{Phonetics}{生成}{sheng1 cheng2}[][HSK 5]
    \definition{v.}{formar; gerar; produzir | ter por natureza; nascer com}
  \end{Phonetics}
\end{Entry}

\begin{Entry}{生词}{5,7}{⽣、⾔}
  \begin{Phonetics}{生词}{sheng1 ci2}[][HSK 2]
    \definition[个,组,堆,条]{s.}{nova palavra; palavras que não aprendi, não conheço ou não entendo}
  \end{Phonetics}
\end{Entry}

\begin{Entry}{生命}{5,8}{⽣、⼝}
  \begin{Phonetics}{生命}{sheng1ming4}[][HSK 3]
    \definition{s.}{vida; não envolve apenas a existência e as atividades dos organismos, mas também inclui experiências de vida humana, valores e elementos-chave da sobrevivência e do desenvolvimento de várias coisas}
  \end{Phonetics}
\end{Entry}

\begin{Entry}{生态}{5,8}{⽣、⼼}
  \begin{Phonetics}{生态}{sheng1tai4}
    \definition{adj.}{ecológico}
    \definition{s.}{ecologia}
  \end{Phonetics}
\end{Entry}

\begin{Entry}{生物}{5,8}{⽣、⽜}
  \begin{Phonetics}{生物}{sheng1wu4}
    \definition{adj.}{biológico}
    \definition{s.}{biologia (disciplina) | organismo | ser vivo}
  \end{Phonetics}
\end{Entry}

\begin{Entry}{生的}{5,8}{⽣、⽩}
  \begin{Phonetics}{生的}{sheng1de5}
    \definition{conj.}{para evitar isso | para que\dots não\dots}
  \end{Phonetics}
\end{Entry}

\begin{Entry}{生鱼片}{5,8,4}{⽣、⿂、⽚}
  \begin{Phonetics}{生鱼片}{sheng1yu2pian4}
    \definition{s.}{fatias de peixe cru, \emph{sashimi}}
  \end{Phonetics}
\end{Entry}

\begin{Entry}{生活}{5,9}{⽣、⽔}
  \begin{Phonetics}{生活}{sheng1huo2}[][HSK 2]
    \definition[个,段,种]{s.}{vida; subsistência; as diversas atividades realizadas por pessoas ou seres vivos para sobreviver e se desenvolver | estilo de vida; condições de vida; situação em termos de vestuário, alimentação, habitação e transporte | trabalho (principalmente nas áreas industrial, agrícola e artesanal)}
    \definition{v.}{viver; realizar várias atividades | sobreviver}
  \end{Phonetics}
\end{Entry}

\begin{Entry}{生活垃圾}{5,9,8,6}{⽣、⽔、⼟、⼟}
  \begin{Phonetics}{生活垃圾}{sheng1huo2la1ji1}
    \definition{s.}{lixo doméstico}
  \end{Phonetics}
\end{Entry}

\begin{Entry}{生活型}{5,9,9}{⽣、⽔、⼟}
  \begin{Phonetics}{生活型}{sheng1huo2 xing2}
    \definition{s.}{forma de vida}
  \end{Phonetics}
\end{Entry}

\begin{Entry}{生活费}{5,9,9}{⽣、⽔、⾙}
  \begin{Phonetics}{生活费}{sheng1 huo2 fei4}[][HSK 6]
    \definition{s.}{subsídio; despesas de subsistência; despesas necessárias para manter a vida diária}
  \end{Phonetics}
\end{Entry}

\begin{Entry}{生病}{5,10}{⽣、⽧}
  \begin{Phonetics}{生病}{sheng1bing4}[][HSK 1]
    \definition{v.}{adoecer; ficar doente; ficar mal; contrair uma doença}
  \end{Phonetics}
\end{Entry}

\begin{Entry}{生理}{5,11}{⽣、⽟}
  \begin{Phonetics}{生理}{sheng1li3}
    \definition{adj.}{fisiológico}
    \definition{s.}{fisiologia}
  \end{Phonetics}
\end{Entry}

\begin{Entry}{生菜}{5,11}{⽣、⾋}
  \begin{Phonetics}{生菜}{sheng1cai4}
    \definition{s.}{alface}
  \end{Phonetics}
\end{Entry}

\begin{Entry}{生意}{5,13}{⽣、⼼}
  \begin{Phonetics}{生意}{sheng1yi4}
    \definition[笔,种,次]{s.}{tendência a crescer; vitalidade; vigor; energia}
  \end{Phonetics}
  \begin{Phonetics}{生意}{sheng1yi5}[][HSK 3]
    \definition[笔,种,次]{s.}{comércio, compra e venda; negócios; indústria; colegas do mesmo setor}
  \end{Phonetics}
\end{Entry}

\begin{Entry}{用}{5}{⽤}[Kangxi 101]
  \begin{Phonetics}{用}{yong4}[][HSK 1]
    \definition*{s.}{Sobrenome Yong}
    \definition{conj.}{portanto; por isso; assim sendo; razões para a introdução, equivalentes a 因}
    \definition{prep.}{com; ação de introduzir ferramentas, meios, etc. utilizados ou empregados}
    \definition{s.}{despesas; gastos; custos | uso; utilidade; eficácia}
    \definition{v.}{usar; aplicar; empregar | necessitar (normalmente na forma negativa) | respeitosamente: comer; beber}
  \seealsoref{因}{yin1}
  \end{Phonetics}
\end{Entry}

\begin{Entry}{用于}{5,3}{⽤、⼆}
  \begin{Phonetics}{用于}{yong4 yu2}[][HSK 5]
    \definition{v.}{usar para; ser usado para; usar em}
  \end{Phonetics}
\end{Entry}

\begin{Entry}{用不着}{5,4,11}{⽤、⼀、⽬}
  \begin{Phonetics}{用不着}{yong4 bu4 zhao2}[][HSK 5]
    \definition{v.}{não precisar; não ter utilidade para; não haver necessidade de}
  \end{Phonetics}
\end{Entry}

\begin{Entry}{用心}{5,4}{⽤、⼼}
  \begin{Phonetics}{用心}{yong4 xin1}[][HSK 6]
    \definition{adj.}{diligente; atento; com atenção concentrada}
    \definition{s.}{motivo; intenção; o verdadeiro propósito ou razão para fazer algo}
  \end{Phonetics}
\end{Entry}

\begin{Entry}{用户}{5,4}{⽤、⼾}
  \begin{Phonetics}{用户}{yong4hu4}[][HSK 5]
    \definition[个,位,名]{s.}{usuário; consumidor; entidades e indivíduos que utilizam determinados equipamentos públicos ou bens de consumo}
  \end{Phonetics}
\end{Entry}

\begin{Entry}{用处}{5,5}{⽤、⼡}
  \begin{Phonetics}{用处}{yong4 chu3}[][HSK 6]
    \definition[个]{s.}{uso; usabilidade; utilidade}
  \end{Phonetics}
\end{Entry}

\begin{Entry}{用来}{5,7}{⽤、⽊}
  \begin{Phonetics}{用来}{yong4 lai2}[][HSK 5]
    \definition{v.}{ser usado para; depender (dele) ou usar (ele) para atingir algum objetivo}
  \end{Phonetics}
\end{Entry}

\begin{Entry}{用法}{5,8}{⽤、⽔}
  \begin{Phonetics}{用法}{yong4 fa3}[][HSK 6]
    \definition[种,个]{s.}{uso; emprego; a maneira de usar}
  \end{Phonetics}
\end{Entry}

\begin{Entry}{用品}{5,9}{⽤、⼝}
  \begin{Phonetics}{用品}{yong4 pin3}[][HSK 6]
    \definition[批,件,种]{s.}{suprimentos; artigos para uso; itens para usar}
  \end{Phonetics}
\end{Entry}

\begin{Entry}{用料}{5,10}{⽤、⽃}
  \begin{Phonetics}{用料}{yong4liao4}
    \definition{s.}{ingredientes | materiais}
  \end{Phonetics}
\end{Entry}

\begin{Entry}{用途}{5,10}{⽤、⾡}
  \begin{Phonetics}{用途}{yong4tu2}[][HSK 4]
    \definition[个,种]{s.}{uso; aplicação; aspectos ou escopo da aplicação}
  \end{Phonetics}
\end{Entry}

\begin{Entry}{用得着}{5,11,11}{⽤、⼻、⽬}
  \begin{Phonetics}{用得着}{yong4 de5 zhao2}[][HSK 6]
    \definition{adj.}{útil; necessário}
    \definition{v.}{precisar; achar algo útil | ter necessidade de;  ser necessário; valer a pena}
  \end{Phonetics}
\end{Entry}

\begin{Entry}{田}{5}{⽥}[Kangxi 102]
  \begin{Phonetics}{田}{tian2}[][HSK 6]
    \definition*{s.}{Sobrenome Tian}
    \definition[亩,块,片]{s.}{campo; terra; terra de cultivo | área aberta rica em algum produto natural; campo}
    \definition{v.}{(arcaico) caçar}
  \end{Phonetics}
\end{Entry}

\begin{Entry}{田园}{5,7}{⽥、⼞}
  \begin{Phonetics}{田园}{tian2yuan2}
    \definition{adj.}{bucólico}
    \definition{s.}{campo | interior | rural}
  \end{Phonetics}
\end{Entry}

\begin{Entry}{田径}{5,8}{⽥、⼻}
  \begin{Phonetics}{田径}{tian2jing4}[][HSK 6]
    \definition{s.}{Esporte: atletismo}[他参加了这次的田径赛。===Ele participou da competição de atletismo.]
  \end{Phonetics}
\end{Entry}

\begin{Entry}{由}{5}{⽥}
  \begin{Phonetics}{由}{you2}[][HSK 3]
    \definition*{s.}{Sobrenome You}
    \definition{prep.}{por causa de; devido a | por; indica que algo deve ser feito por alguém | indica confiança em; indica dependência em | de; indica o ponto de partida | por; através de}
    \definition[个]{s.}{causa; razão; motivo}
    \definition{v.}{atravessar; passar por; seguir o caminho de | obedecer; seguir}
  \end{Phonetics}
\end{Entry}

\begin{Entry}{由于}{5,3}{⽥、⼆}
  \begin{Phonetics}{由于}{you2yu2}[][HSK 3]
    \definition{conj.}{porque; uma vez que; visto que;  usado no início da frase anterior, indica a razão, e a frase seguinte indica o resultado}
    \definition{prep.}{devido a; graças a; por causa de; em virtude de; como resultado de; introduzir a causa da ocorrência de eventos, ações, etc.}
  \end{Phonetics}
\end{Entry}

\begin{Entry}{由此}{5,6}{⽥、⽌}
  \begin{Phonetics}{由此}{you2 ci3}[][HSK 5]
    \definition{adv.}{assim; por meio disto; disto; daí; por causa disto; portanto; daqui; de agora em diante}
  \end{Phonetics}
\end{Entry}

\begin{Entry}{甲}{5}{⽥}
  \begin{Phonetics}{甲}{jia3}[][HSK 5]
    \definition*{s.}{Sobrenome Jia}
    \definition{s.}{alfa; primeiro lugar; o primeiro dos caules celestiais, geralmente usado para indicar o primeiro em ordem ou classificação | concha; carapaça; crustáceos | unha; crostas queratinosas nos dedos das mãos e dos pés | armadura; equipamento de proteção feito de metal | Datado: unidade de administração civil composta por 10 residências | uma palavra substituta para uma pessoa ou coisa indefinida; usado como pronome}
    \definition{v.}{ocupar o primeiro lugar; ser melhor do que}
  \end{Phonetics}
\end{Entry}

\begin{Entry}{甲骨文}{5,9,4}{⽥、⾻、⽂}
  \begin{Phonetics}{甲骨文}{jia3gu3wen2}
    \definition{s.}{escrituras de oráculos | inscrições em ossos de oráculos (forma original de escritura chinesa)}
  \end{Phonetics}
\end{Entry}

\begin{Entry}{申}{5}{⽥}
  \begin{Phonetics}{申}{shen1}
    \definition*{s.}{O nono dos doze Ramos Terrestres | Outro nome para Xangai, 上海 | Sobrenome Shen}
    \definition{v.}{declarar; explicar; expressar}
  \seealsoref{上海}{shang4hai3}
  \end{Phonetics}
\end{Entry}

\begin{Entry}{申请}{5,10}{⽥、⾔}
  \begin{Phonetics}{申请}{shen1qing3}[][HSK 4]
    \definition[份,批,项]{s.}{a solicitação para; o requerimento para; um pedido para ser visto pelos superiores ou departamentos relevantes}
    \definition{v.}{solicitar; apresentar uma solicitação; apresentar os motivos e fazer o pedido aos superiores ou aos departamentos competentes}
  \end{Phonetics}
\end{Entry}

\begin{Entry}{电}{5}{⽥}
  \begin{Phonetics}{电}{dian4}[][HSK 1]
    \definition*{s.}{Sobrenome Dian}
    \definition{s.}{eletricidade; energia elétrica | telegrama | relâmpago}
    \definition{v.}{dar ou receber um choque elétrico | enviar telegrama, telefonar ou enviar fax}
  \end{Phonetics}
\end{Entry}

\begin{Entry}{电力}{5,2}{⽥、⼒}
  \begin{Phonetics}{电力}{dian4 li4}[][HSK 6]
    \definition{s.}{energia elétrica; fornecimento de energia elétrica | energia elétrica | eletricidade}
  \end{Phonetics}
\end{Entry}

\begin{Entry}{电子}{5,3}{⽥、⼦}
  \begin{Phonetics}{电子}{dian4zi3}
    \definition{s.}{eletrônico | elétron}
  \end{Phonetics}
\end{Entry}

\begin{Entry}{电子名片}{5,3,6,4}{⽥、⼦、⼝、⽚}
  \begin{Phonetics}{电子名片}{dian4zi3 ming2pian4}
    \definition{s.}{cartão de visita eletrônico}
  \end{Phonetics}
\end{Entry}

\begin{Entry}{电子邮件}{5,3,7,6}{⽥、⼦、⾢、⼈}
  \begin{Phonetics}{电子邮件}{dian4zi3you2jian4}[][HSK 3]
    \definition[封,份,个,条]{s.}{correio eletrônico; \emph{e-mail}}
  \seealsoref{电邮}{dian4you2}
  \end{Phonetics}
\end{Entry}

\begin{Entry}{电子版}{5,3,8}{⽥、⼦、⽚}
  \begin{Phonetics}{电子版}{dian4 zi3 ban3}[][HSK 5]
    \definition[个]{s.}{edição eletrônica}
  \end{Phonetics}
\end{Entry}

\begin{Entry}{电车}{5,4}{⽥、⾞}
  \begin{Phonetics}{电车}{dian4 che1}[][HSK 6]
    \definition[辆,班,趟,路]{s.}{bonde; veículos de transporte público urbano movidos por linhas aéreas e acionados por motores de tração}
  \end{Phonetics}
\end{Entry}

\begin{Entry}{电车司机}{5,4,5,6}{⽥、⾞、⼝、⽊}
  \begin{Phonetics}{电车司机}{dian4che1 si1ji1}
    \definition{s.}{motorista de bonde}
  \end{Phonetics}
\end{Entry}

\begin{Entry}{电台}{5,5}{⽥、⼝}
  \begin{Phonetics}{电台}{dian4 tai2}[][HSK 3]
    \definition[个,家]{s.}{transceptor; transmissor-receptor | aparelho de rádio; estação de rádio; estação de transmissão}
  \end{Phonetics}
\end{Entry}

\begin{Entry}{电冰箱}{5,6,15}{⽥、⼎、⾋}
  \begin{Phonetics}{电冰箱}{dian4bing1xiang1}
    \definition[台]{s.}{frigorífico | refrigerador}
  \end{Phonetics}
\end{Entry}

\begin{Entry}{电动}{5,6}{⽥、⼒}
  \begin{Phonetics}{电动}{dian4 dong4}[][HSK 6]
    \definition{adj.}{motorizado; acionado por energia elétrica; operado por energia elétrica elétrico}
  \end{Phonetics}
\end{Entry}

\begin{Entry}{电动车}{5,6,4}{⽥、⼒、⾞}
  \begin{Phonetics}{电动车}{dian4 dong4 che1}[][HSK 4]
    \definition{s.}{veículo elétrico (\emph{scooter}, bicicleta, carro, etc.)}
  \end{Phonetics}
\end{Entry}

\begin{Entry}{电池}{5,6}{⽥、⽔}
  \begin{Phonetics}{电池}{dian4chi2}[][HSK 5]
    \definition[节,块,组,个]{s.}{célula; bateria}
  \end{Phonetics}
\end{Entry}

\begin{Entry}{电灯}{5,6}{⽥、⽕}
  \begin{Phonetics}{电灯}{dian4 deng1}[][HSK 4]
    \definition[盏,个]{s.}{luz elétrica; lâmpada elétrica; lâmpadas que usam eletricidade como fonte de energia}
  \end{Phonetics}
\end{Entry}

\begin{Entry}{电灯泡}{5,6,8}{⽥、⽕、⽔}
  \begin{Phonetics}{电灯泡}{dian4deng1pao4}
    \definition{s.}{lâmpada elétrica | (gíria) terceiro convidado indesejado}
  \end{Phonetics}
\end{Entry}

\begin{Entry}{电邮}{5,7}{⽥、⾢}
  \begin{Phonetics}{电邮}{dian4you2}
    \definition{s.}{correio eletrônico, \emph{e-mail} | abreviação de~电子邮件}
  \seealsoref{电子邮件}{dian4zi3you2jian4}
  \end{Phonetics}
\end{Entry}

\begin{Entry}{电饭锅}{5,7,12}{⽥、⾷、⾦}
  \begin{Phonetics}{电饭锅}{dian4 fan4 guo1}[][HSK 5]
    \definition[台,个]{s.}{panela elétrica de arroz}
  \end{Phonetics}
\end{Entry}

\begin{Entry}{电视}{5,8}{⽥、⾒}
  \begin{Phonetics}{电视}{dian4shi4}[][HSK 1]
    \definition[部,台,个]{s.}{televisão; TV; televisor}
  \end{Phonetics}
\end{Entry}

\begin{Entry}{电视台}{5,8,5}{⽥、⾒、⼝}
  \begin{Phonetics}{电视台}{dian4 shi4 tai2}[][HSK 3]
    \definition[家,座,个]{s.}{canal de TV; estação de televisão; locais e instituições que transmitem programas de televisão}
  \end{Phonetics}
\end{Entry}

\begin{Entry}{电视机}{5,8,6}{⽥、⾒、⽊}
  \begin{Phonetics}{电视机}{dian4 shi4 ji1}[][HSK 1]
    \definition[个,台]{s.}{aparelho de TV; receptor de televisão; receptor de imagem; televisor; aparelho de televisão}
  \end{Phonetics}
\end{Entry}

\begin{Entry}{电视剧}{5,8,10}{⽥、⾒、⼑}
  \begin{Phonetics}{电视剧}{dian4 shi4 ju4}[][HSK 3]
    \definition[部,集,个]{s.}{série de TV; drama de TV; novela; drama escrito e gravado para transmissão pela televisão}
  \end{Phonetics}
\end{Entry}

\begin{Entry}{电话}{5,8}{⽥、⾔}
  \begin{Phonetics}{电话}{dian4 hua4}[][HSK 1]
    \definition[部]{s.}{telefone; aparelho telefônico; telefonia}
    \definition[通]{s.}{chamada telefônica; telefonema}
  \end{Phonetics}
\end{Entry}

\begin{Entry}{电脑}{5,10}{⽥、⾁}
  \begin{Phonetics}{电脑}{dian4nao3}[][HSK 1]
    \definition[个,台]{s.}{computador eletrônico}
  \end{Phonetics}
\end{Entry}

\begin{Entry}{电脑语言}{5,10,9,7}{⽥、⾁、⾔、⾔}
  \begin{Phonetics}{电脑语言}{dian4nao3yu3yan2}
    \definition{s.}{linguagem de programação | linguagem de computador}
  \end{Phonetics}
\end{Entry}

\begin{Entry}{电梯}{5,11}{⽥、⽊}
  \begin{Phonetics}{电梯}{dian4ti1}[][HSK 4]
    \definition[部,台,架]{s.}{elevador}
  \end{Phonetics}
\end{Entry}

\begin{Entry}{电梯司机}{5,11,5,6}{⽥、⽊、⼝、⽊}
  \begin{Phonetics}{电梯司机}{dian4ti1 si1ji1}
    \definition{s.}{ascensorista}
  \end{Phonetics}
\end{Entry}

\begin{Entry}{电源}{5,13}{⽥、⽔}
  \begin{Phonetics}{电源}{dian4yuan2}[][HSK 4]
    \definition[台,个,套]{s.}{fonte de alimentação; fonte de energia; fonte de energia elétrica; dispositivo que fornece energia elétrica a um aparelho, como uma bateria, um gerador, etc.}
  \end{Phonetics}
\end{Entry}

\begin{Entry}{电影}{5,15}{⽥、⼺}
  \begin{Phonetics}{电影}{dian4ying3}[][HSK 1]
    \definition[部,片,幕,场]{s.}{filme; longa-metragem; cinema}
  \end{Phonetics}
\end{Entry}

\begin{Entry}{电影艺术}{5,15,4,5}{⽥、⼺、⾋、⽊}
  \begin{Phonetics}{电影艺术}{dian4ying3 yi4shu4}
    \definition{s.}{arte cinematográfica}
  \end{Phonetics}
\end{Entry}

\begin{Entry}{电影术}{5,15,5}{⽥、⼺、⽊}
  \begin{Phonetics}{电影术}{dian4ying3 shu4}
    \definition{s.}{cinematografia}
  \end{Phonetics}
\end{Entry}

\begin{Entry}{电影节}{5,15,5}{⽥、⼺、⾋}
  \begin{Phonetics}{电影节}{dian4ying3jie2}
    \definition{s.}{festival de cinema}
  \end{Phonetics}
\end{Entry}

\begin{Entry}{电影奖}{5,15,9}{⽥、⼺、⼤}
  \begin{Phonetics}{电影奖}{dian4ying3jiang3}
    \definition{s.}{premiações de cinema}
  \end{Phonetics}
\end{Entry}

\begin{Entry}{电影界}{5,15,9}{⽥、⼺、⽥}
  \begin{Phonetics}{电影界}{dian4ying3jie4}
    \definition{s.}{indústria cinematográfica}
  \end{Phonetics}
\end{Entry}

\begin{Entry}{电影院}{5,15,9}{⽥、⼺、⾩}
  \begin{Phonetics}{电影院}{dian4 ying3 yuan4}[][HSK 1]
    \definition[家,座,个]{s.}{cinema; sala de cinema; teatro; salão de cinema; local comercial dedicado à exibição de filmes}
  \end{Phonetics}
\end{Entry}

\begin{Entry}{电影音乐}{5,15,9,5}{⽥、⼺、⾳、⼃}
  \begin{Phonetics}{电影音乐}{dian4ying3 yin1yue4}
    \definition{s.}{música cinematográfica}
  \end{Phonetics}
\end{Entry}

\begin{Entry}{电影票}{5,15,11}{⽥、⼺、⽰}
  \begin{Phonetics}{电影票}{dian4ying3piao4}
    \definition{s.}{ingresso de filme}
  \end{Phonetics}
\end{Entry}

\begin{Entry}{电器}{5,16}{⽥、⼝}
  \begin{Phonetics}{电器}{dian4 qi4}[][HSK 6]
    \definition[件,种]{s.}{dispositivo elétrico; cargas em circuitos e dispositivos usados ​​para controlar, regular ou proteger circuitos, motores, etc.; como alto-falantes; interruptores; resistores; fusíveis, etc. | eletrodomésticos ou aparelhos elétricos domésticos; refere-se a eletrodomésticos, como televisores, gravadores, geladeiras, máquinas de lavar, etc.}
  \end{Phonetics}
\end{Entry}

\begin{Entry}{白}{5}{⽩}[Kangxi 106]
  \begin{Phonetics}{白}{bai2}[][HSK 1,3]
    \definition*{s.}{Sobrenome Bai}
    \definition{adj.}{branco | claro; entendível; compreendível | puro; claro; simples; sem mistura; em branco | branco (como símbolo de reação) | escrito incorretamente ou pronunciado incorretamente}
    \definition{adv.}{em vão; sem propósito; sem resultados | gratuito; sem custos}
    \definition{s.}{parte falada em ópera, etc.; frases de peças de teatro, etc. | dialeto local | funeral}
    \definition{v.}{explicar; apresentar; esclarecer; declarar | branquear | olhar para as pessoas com o branco dos olhos (olhar vazio, de desaprovação); olhar para alguém com desdém}
  \end{Phonetics}
\end{Entry}

\begin{Entry}{白天}{5,4}{⽩、⼤}
  \begin{Phonetics}{白天}{bai2 tian1}[][HSK 1]
    \definition{adv.}{dia;  de dia}
    \definition[个]{s.}{dia; horário diurno; durante o dia}
  \end{Phonetics}
\end{Entry}

\begin{Entry}{白色}{5,6}{⽩、⾊}
  \begin{Phonetics}{白色}{bai2 se4}[][HSK 2]
    \definition{s.}{a cor branca}
  \end{Phonetics}
\end{Entry}

\begin{Entry}{白苋}{5,7}{⽩、⾋}
  \begin{Phonetics}{白苋}{bai2xian4}
    \definition{s.}{amaranto branco | brotos e folhas tenras de espinafre chinês usados como alimento}
  \end{Phonetics}
\end{Entry}

\begin{Entry}{白拣}{5,8}{⽩、⼿}
  \begin{Phonetics}{白拣}{bai2jian3}
    \definition{s.}{uma escolha barata}
    \definition{v.}{escolher algo que não custa nada}
  \end{Phonetics}
\end{Entry}

\begin{Entry}{白酒}{5,10}{⽩、⾣}
  \begin{Phonetics}{白酒}{bai2 jiu3}[][HSK 5]
    \definition[瓶,杯,壶]{s.}{aguardente branca; aguardente (geralmente destilada de sorgo ou milho); bebidas destiladas tradicionais chinesas, feitas de sorgo, milho, etc., transparentes e incolores, com alto teor alcoólico}
  \end{Phonetics}
\end{Entry}

\begin{Entry}{白菜}{5,11}{⽩、⾋}
  \begin{Phonetics}{白菜}{bai2 cai4}[][HSK 3]
    \definition[棵,种]{s.}{couve chinesa | \emph{pak choi}, um tipo de couve}
  \end{Phonetics}
\end{Entry}

\begin{Entry}{白萝卜}{5,11,2}{⽩、⾋、⼘}
  \begin{Phonetics}{白萝卜}{bai2luo2bo5}
    \definition{s.}{rabanete branco | \emph{daikon}}
  \end{Phonetics}
\end{Entry}

\begin{Entry}{白蛋白}{5,11,5}{⽩、⾍、⽩}
  \begin{Phonetics}{白蛋白}{bai2dan4bai2}
    \definition{s.}{albumina}
  \end{Phonetics}
\end{Entry}

\begin{Entry}{白领}{5,11}{⽩、⾴}
  \begin{Phonetics}{白领}{bai2 ling3}[][HSK 6]
    \definition[个,名,位,些]{s.}{colarinho branco; trabalhador de colarinho branco; refere-se a funcionários cujo trabalho principal envolve trabalho intelectual, são conhecidos por suas roupas elegantes, colarinhos e camisas brancas; atualmente é frequentemente usado para se referir àqueles que trabalham em cargos de gestão ou técnicos em empresas e ganham salários relativamente altos}
  \end{Phonetics}
\end{Entry}

\begin{Entry}{白鹄}{5,12}{⽩、⿃}
  \begin{Phonetics}{白鹄}{bai2hu2}
    \definition{s.}{cisne branco}
  \end{Phonetics}
\end{Entry}

\begin{Entry}{白痴}{5,13}{⽩、⽧}
  \begin{Phonetics}{白痴}{bai2chi1}
    \definition{adj.}{idiota; uma pessoa que sofre de idiotice; frequentemente usado para menosprezar alguém que é incompetente ou incapaz de fazer as coisas}
    \definition{s.}{idiotice; uma doença caracterizada por retardo mental, demência, fala arrastada, movimentos lentos e até mesmo incapacidade de cuidar de si mesmo}
  \end{Phonetics}
\end{Entry}

\begin{Entry}{皮}{5}{⽪}[Kangxi 107]
  \begin{Phonetics}{皮}{pi2}[][HSK 3]
    \definition*{s.}{Sobrenome Pi}
    \definition{adj.}{macios e encharcados; não mais crocantes | malandro; travesso | apático; endurecido; indiferente devido a repetidas repreensões | pegajoso; tenaz; resiliente}
    \definition{pref.}{pico- (um trilhonésimo)}
    \definition[层,块,张,个]{s.}{pele; casca; uma camada de tecido na superfície dos organismos animais e vegetais | pele; couro; couro processado | capa; embalagem; a camada externa que envolve algo | superfície do objeto | folha; peça larga e plana (de algum material fino) | borracha}
  \end{Phonetics}
\end{Entry}

\begin{Entry}{皮下}{5,3}{⽪、⼀}
  \begin{Phonetics}{皮下}{pi2xia4}
    \definition{adj.}{(injeção) subcutâneo | sob a pele}
  \end{Phonetics}
\end{Entry}

\begin{Entry}{皮包}{5,5}{⽪、⼓}
  \begin{Phonetics}{皮包}{pi2 bao1}[][HSK 3]
    \definition[个,只,款]{s.}{bolsa; pasta; portfólio; bolsas de couro}
  \end{Phonetics}
\end{Entry}

\begin{Entry}{皮卡}{5,5}{⽪、⼘}
  \begin{Phonetics}{皮卡}{pi2ka3}
    \definition{s.}{(empréstimo linguístico) \emph{pick-up} | caminhonete}
  \end{Phonetics}
\end{Entry}

\begin{Entry}{皮卡丘}{5,5,5}{⽪、⼘、⼀}
  \begin{Phonetics}{皮卡丘}{pi2ka3qiu1}
    \definition*{s.}{Pikachu (Pokémon, 口袋妖怪)}
  \seealsoref{口袋妖怪}{kou3dai4 yao1guai4}
  \end{Phonetics}
\end{Entry}

\begin{Entry}{皮肤}{5,8}{⽪、⾁}
  \begin{Phonetics}{皮肤}{pi2fu1}[][HSK 5]
    \definition{adj.}{superficial}
    \definition[种,块,片,层]{s.}{pele; couro; derme}
  \end{Phonetics}
\end{Entry}

\begin{Entry}{皮球}{5,11}{⽪、⽟}
  \begin{Phonetics}{皮球}{pi2 qiu2}[][HSK 6]
    \definition{s.}{bola (feita de borracha, couro etc.)}
  \end{Phonetics}
\end{Entry}

\begin{Entry}{皮鞋}{5,15}{⽪、⾰}
  \begin{Phonetics}{皮鞋}{pi2xie2}[][HSK 5]
    \definition[双,只,款]{s.}{sapatos feitos de couro}
  \end{Phonetics}
\end{Entry}

\begin{Entry}{目}{5}{⽬}[Kangxi 109]
  \begin{Phonetics}{目}{mu4}
    \definition*{s.}{Sobrenome Mu}
    \definition{s.}{olho | item | (biologia) ordem | lista de coisas; catálogo; sumário | buraco em uma rede; malha (abertura)  | (de documentos, teses, etc.) nome; título | ponto; ponto de território, um termo do Go; refere-se à intersecção das linhas verticais e horizontais no tabuleiro, uma intersecção é chamada de 一目, \dpy{yi2 mu4}}
    \definition{v.}{(literário) olhar; considerar}
  \end{Phonetics}
\end{Entry}

\begin{Entry}{目光}{5,6}{⽬、⼉}
  \begin{Phonetics}{目光}{mu4guang1}[][HSK 5]
    \definition[道,束,种]{s.}{olhar fixo; a expressão e atitude reveladas pelos olhos | visão; vista; percepção visual; a linha imaginária formada entre os olhos e o objeto quando se olha para ele | perspicácia (capacidade de observar e reconhecer coisas); conhecimento adquirido através do contato com as coisas, capacidade de observar as coisas}
  \end{Phonetics}
\end{Entry}

\begin{Entry}{目的}{5,8}{⽬、⽩}
  \begin{Phonetics}{目的}{mu4di4}[][HSK 2]
    \definition[个,些,种]{s.}{objetivo; meta; alvo; finalidade; propósito; o lugar ou situação que se deseja alcançar; o resultado que se deseja obter; o centro do alvo}
  \end{Phonetics}
\end{Entry}

\begin{Entry}{目前}{5,9}{⽬、⼑}
  \begin{Phonetics}{目前}{mu4qian2}[][HSK 3]
    \definition{adv.}{agora; recentemente; no momento; no presente}
  \end{Phonetics}
\end{Entry}

\begin{Entry}{目标}{5,9}{⽬、⽊}
  \begin{Phonetics}{目标}{mu4biao1}[][HSK 3]
    \definition[个,项]{s.}{alvo; objetivo; objeto de tiro, ataque ou busca| objetivo; meta; destino; a situação ou padrão que se deseja alcançar}
  \end{Phonetics}
\end{Entry}

\begin{Entry}{矛}{5}{⽭}[Kangxi 110]
  \begin{Phonetics}{矛}{mao2}
    \definition{s.}{Arcaico: lança; lanceta}
  \end{Phonetics}
\end{Entry}

\begin{Entry}{矛盾}{5,9}{⽭、⽬}
  \begin{Phonetics}{矛盾}{mao2dun4}[][HSK 5]
    \definition{adj.}{contraditório; descreve pessoas ou coisas que se opõem ou se repelem mutuamente}
    \definition[对,个,种]{s.}{problema; contradição; discrepância; inconsistência | disputas e conflitos; relacionamento de oposição entre as duas partes devido a diferenças de opinião ou abordagem}
    \definition{v.}{opor-se; entrar em conflito; contradizer; nesta situação, apenas uma das opções está correta ou é verdadeira; não é possível que ambas estejam corretas ao mesmo tempo}
  \end{Phonetics}
\end{Entry}

\begin{Entry}{石}{5}{⽯}[Kangxi 112]
  \begin{Phonetics}{石}{dan4}
    \definition{clas.}{dan, uma unidade de medida seca para grãos; unidade de capacidade, 10 斗 é igual a 1 石}
  \seealsoref{斗}{dou4}
  \end{Phonetics}
  \begin{Phonetics}{石}{shi2}
    \definition*{s.}{Sobrenome Shi}
    \definition{s.}{pedra; rocha; o material duro que constitui a crosta terrestre é composto por uma coleção de minerais | inscrição em pedra; esculturas em pedra}
  \end{Phonetics}
\end{Entry}

\begin{Entry}{石头}{5,5}{⽯、⼤}
  \begin{Phonetics}{石头}{shi2tou5}[][HSK 3]
    \definition[块,堆,些]{s.}{rocha; pedra; uma substância muito dura que é o principal material da superfície da Terra}
  \end{Phonetics}
\end{Entry}

\begin{Entry}{石油}{5,8}{⽯、⽔}
  \begin{Phonetics}{石油}{shi2you2}[][HSK 3]
    \definition[桶,吨,升]{s.}{óleo; óleo fóssil; petróleo; um líquido inflamável extraído do solo, geralmente marrom escuro, preto ou verde escuro, do qual gasolina e outras substâncias podem ser obtidas}
  \end{Phonetics}
\end{Entry}

\begin{Entry}{示}{5}{⽰}[Kangxi 113]
  \begin{Phonetics}{示}{shi4}
    \definition*{s.}{Sobrenome Shi}
    \definition{s.}{(sua) carta  | missiva; instruções; palavras ou escritos para subordinados ou gerações mais jovens}
    \definition{v.}{mostrar; notificar; instruir | indicar; significar; mostrar ou apontar, fazer conhecido}
  \end{Phonetics}
\end{Entry}

\begin{Entry}{示范}{5,9}{⽰、⾋}
  \begin{Phonetics}{示范}{shi4fan4}[][HSK 5]
    \definition{v.}{demonstrar; dar o exemplo; criar um modelo que todos possam aprender}
  \end{Phonetics}
\end{Entry}

\begin{Entry}{礼}{5}{⽰}
  \begin{Phonetics}{礼}{li3}[][HSK 5]
    \definition*{s.}{Sobrenome Li}
    \definition[份]{s.}{observâncias cerimoniais em geral; cerimônia; rito | cortesia; etiqueta; boas maneiras | presente; oferta}
  \end{Phonetics}
\end{Entry}

\begin{Entry}{礼节}{5,5}{⽰、⾋}
  \begin{Phonetics}{礼节}{li3jie2}
    \definition{s.}{protocolo | cerimônia | etiqueta}
  \end{Phonetics}
\end{Entry}

\begin{Entry}{礼让}{5,5}{⽰、⾔}
  \begin{Phonetics}{礼让}{li3rang4}
    \definition{s.}{cortesia}
    \definition{v.}{mostrar consideração por (outros) | ceder a (outro veículo, etc.)}
  \end{Phonetics}
\end{Entry}

\begin{Entry}{礼物}{5,8}{⽰、⽜}
  \begin{Phonetics}{礼物}{li3wu4}[][HSK 2]
    \definition[份,件,个,分,些]{s.}{presente; lembrança; itens oferecidos como forma de respeito ou celebração, referindo-se de maneira geral a itens oferecidos como presente}
  \end{Phonetics}
\end{Entry}

\begin{Entry}{礼拜}{5,9}{⽰、⼿}
  \begin{Phonetics}{礼拜}{li3 bai4}[][HSK 5]
    \definition[个]{s.}{dia da semana; usado em conjunto com 一, 二, 三, 四, 五, 六, 日(或天, indica um dia específico da semana | semana; referência à semana | domingo}
    \definition{v.}{prestar homenagem aos deuses que veneram; rezar; orar}
  \end{Phonetics}
\end{Entry}

\begin{Entry}{礼堂}{5,11}{⽰、⼟}
  \begin{Phonetics}{礼堂}{li3 tang2}[][HSK 6]
    \definition[个,座,处]{s.}{auditórios; salão de assembleias; um salão para reuniões ou cerimônias}
  \end{Phonetics}
\end{Entry}

\begin{Entry}{礼貌}{5,14}{⽰、⾘}
  \begin{Phonetics}{礼貌}{li3mao4}[][HSK 5]
    \definition{adj.}{educado; descreve uma pessoa que fala e age respeitando os outros, sem arrogância, de acordo com as exigências das relações sociais}
    \definition{s.}{cortesia; educação; boas maneiras}
  \end{Phonetics}
\end{Entry}

\begin{Entry}{立}{5}{⽴}[Kangxi 117]
  \begin{Phonetics}{立}{li4}[][HSK 5]
    \definition{adj.}{ereto; vertical; na vertical}
    \definition{adv.}{imediatamente; instantaneamente}
    \definition{v.}{ficar em pé, com os pés no chão ou apoiados em algum objeto; o objeto deve estar na vertical | erguer; colocar (ou levantar) algo; colocar em pé | encontrar; criar; elaborar; formular; estabelecer | configurar; fundar; estabelecer | viver; existir | ascender ao trono; antigamente, referia-se à ascensão ao trono de um monarca | nomear; designar; antigamente, significava estabelecer uma determinada posição ou status}
  \end{Phonetics}
\end{Entry}

\begin{Entry}{立场}{5,6}{⽴、⼟}
  \begin{Phonetics}{立场}{li4chang3}[][HSK 5]
    \definition[个]{s.}{posição; postura; a posição e a atitude adotadas ao reconhecer e lidar com os problemas | ponto de vista; refere-se especificamente à atitude de reconhecer e lidar com questões a partir dos interesses de uma determinada classe, ou seja, a posição de classe}
  \end{Phonetics}
\end{Entry}

\begin{Entry}{立即}{5,7}{⽴、⼙}
  \begin{Phonetics}{立即}{li4ji2}[][HSK 4]
    \definition{adv.}{prontamente; imediatamente; de imediato}
  \end{Phonetics}
\end{Entry}

\begin{Entry}{立刻}{5,8}{⽴、⼑}
  \begin{Phonetics}{立刻}{li4ke4}[][HSK 3]
    \definition{adv.}{imediatamente; de ​​uma vez; indica que algo acontecerá imediatamente após um determinado momento}
  \end{Phonetics}
\end{Entry}

\begin{Entry}{立法}{5,8}{⽴、⽔}
  \begin{Phonetics}{立法}{li4fa3}
    \definition{s.}{legislação}
    \definition{v.}{promulgar leis | legislar}
  \end{Phonetics}
\end{Entry}

\begin{Entry}{纠}{5}{⽷}
  \begin{Phonetics}{纠}{jiu1}
    \definition*{s.}{Sobrenome Jiu}
    \definition{v.}{emaranhar | reunir-se | corrigir; retificar | supervisionar; superintender}
  \end{Phonetics}
\end{Entry}

\begin{Entry}{纠正}{5,5}{⽷、⽌}
  \begin{Phonetics}{纠正}{jiu1zheng4}[][HSK 6]
    \definition{v.}{fazer certo; corrigir (deficiências ou erros em pensamentos, ações, métodos, etc.)}
  \end{Phonetics}
\end{Entry}

\begin{Entry}{纠纷}{5,7}{⽷、⽷}
  \begin{Phonetics}{纠纷}{jiu1fen1}[][HSK 6]
    \definition[个,次]{s.}{questão; disputa; existem contradições ou conflitos de interesse entre as duas partes que precisam ser resolvidos}
  \end{Phonetics}
\end{Entry}

\begin{Entry}{纠葛}{5,12}{⽷、⾋}
  \begin{Phonetics}{纠葛}{jiu1ge2}
    \definition{s.}{emaranhado | disputa}
  \end{Phonetics}
\end{Entry}

\begin{Entry}{艾}{5}{⾋}
  \begin{Phonetics}{艾}{ai4}
    \definition*{s.}{Botânica: Artemísia chinesa (Artemisia argyi)}
    \definition*{s.}{Sobrenome Ai}
    \definition{adj.}{Literário: gracioso; bonito; lindo}
    \definition{s.}{artemísia; absinto; artemísia chinesa}
    \definition{v.}{Literário: parar; terminar}
  \end{Phonetics}
  \begin{Phonetics}{艾}{yi4}
    \definition{adj.}{estável}
    \definition{v.}{ser corrigido; estar corrigido}
  \end{Phonetics}
\end{Entry}

\begin{Entry}{艾滋病}{5,12,10}{⾋、⽔、⽧}
  \begin{Phonetics}{艾滋病}{ai4zi1bing4}[][HSK 7,8,9]
    \definition*{s.}{Síndrome da Imunodeficiência Adquirida (AIDS)}
  \end{Phonetics}
\end{Entry}

\begin{Entry}{节}{5}{⾋}
  \begin{Phonetics}{节}{jie1}
    \definition{adj.}{momento crucial; momento crítico; momento decisivo; metáfora para algo importante, decisivo ou oportuno}
  \end{Phonetics}
  \begin{Phonetics}{节}{jie2}[][HSK 2]
    \definition*{s.}{Sobrenome Jie}
    \definition{clas.}{nó (kn), velocidade de um barco | para seções, comprimentos}
    \definition[个]{s.}{junta; botão; nó; geralmente se refere à parte da grama ou caule da grama onde as folhas crescem ou à parte onde os galhos e troncos das plantas são conectados | parte; divisão; um trecho de algo interligado; uma parte do todo | festival; feriado; dia memorável; um período de tempo ou um dia com características específicas | item; assunto | castidade; integridade ética e moral | articulação; o local onde os ossos humanos ou animais se conectam | etiqueta; cerimonial | batida; ritmo | registro; documento utilizado na antiguidade para comprovar a identidade | estação do ano | sílaba}
    \definition{v.}{economizar; conservar; poupar | resumir; extrair; retirar uma parte do todo | controlar; restringir; moderar}
  \end{Phonetics}
\end{Entry}

\begin{Entry}{节日}{5,4}{⾋、⽇}
  \begin{Phonetics}{节日}{jie2ri4}[][HSK 2]
    \definition[个,种,类]{s.}{festival; feriado; dia de comemoração tradicional; dia comemorativo estabelecido por lei}
  \end{Phonetics}
\end{Entry}

\begin{Entry}{节目}{5,5}{⾋、⽬}
  \begin{Phonetics}{节目}{jie2mu4}[][HSK 2]
    \definition[个,场,项,台]{s.}{programa; item (em um programa); programas artísticos ou projetos transmitidos por rádios e televisões}
  \end{Phonetics}
\end{Entry}

\begin{Entry}{节约}{5,6}{⾋、⽷}
  \begin{Phonetics}{节约}{jie2yue1}[][HSK 3]
    \definition{adj.}{econômico; sem luxo}
    \definition{v.}{guardar; economizar; usar com moderação; economizar gastos desnecessários}
  \end{Phonetics}
\end{Entry}

\begin{Entry}{节奏}{5,9}{⾋、⼤}
  \begin{Phonetics}{节奏}{jie2zou4}[][HSK 6]
    \definition[个,种]{s.}{ritmo; o fenômeno da alternância regular de comprimento, força e fraqueza das notas na música | padrão regular; uma metáfora para um processo de ajuste adequado com tensão e relaxamento}
  \end{Phonetics}
\end{Entry}

\begin{Entry}{节省}{5,9}{⾋、⽬}
  \begin{Phonetics}{节省}{jie2sheng3}[][HSK 4]
    \definition{adj.}{econômico; parcimonioso}
    \definition{v.}{economizar; conservar; usar com moderação; reduzir; eliminar ou minimizar o esgotamento de itens potencialmente esgotáveis}
  \end{Phonetics}
\end{Entry}

\begin{Entry}{节能}{5,10}{⾋、⾁}
  \begin{Phonetics}{节能}{jie2 neng2}[][HSK 6]
    \definition{v.}{economizar no consumo de energia; conservar energia}
  \end{Phonetics}
\end{Entry}

\begin{Entry}{节假日}{5,11,4}{⾋、⼈、⽇}
  \begin{Phonetics}{节假日}{jie2 jia4 ri4}[][HSK 6]
    \definition[个]{s.}{feriados; festivais e feriados}
  \end{Phonetics}
\end{Entry}

\begin{Entry}{讨}{5}{⾔}
  \begin{Phonetics}{讨}{tao3}
    \definition{v.}{enviar forças armadas para suprimir; enviar uma expedição punitiva contra; enviar exército ou despachar tropas para suprimir ou atacar | denunciar; condenar; censurar | exigir; pedir; implorar por | casar (com uma mulher) | incorrer; convidar | discutir; estudar | provocar; cortejar}
  \end{Phonetics}
\end{Entry}

\begin{Entry}{讨生活}{5,5,9}{⾔、⽣、⽔}
  \begin{Phonetics}{讨生活}{tao3sheng1huo2}
    \definition{v.}{ganhar a vida}
  \end{Phonetics}
\end{Entry}

\begin{Entry}{讨厌}{5,6}{⾔、⼚}
  \begin{Phonetics}{讨厌}{tao3yan4}[][HSK 5]
    \definition{adj.}{desagradável; repugnante; repulsivo; irritante; incômodo}
    \definition{v.}{odiar; não gostar; sentir repulsa por}
  \end{Phonetics}
\end{Entry}

\begin{Entry}{讨论}{5,6}{⾔、⾔}
  \begin{Phonetics}{讨论}{tao3lun4}[][HSK 2]
    \definition{v.}{discutir; conversar sobre; trocar opiniões ou debater as questões levantadas}
  \end{Phonetics}
\end{Entry}

\begin{Entry}{让}{5}{⾔}
  \begin{Phonetics}{让}{rang4}[][HSK 2]
    \definition*{s.}{Sobrenome Rang}
    \definition{prep.}{em uma frase passiva para introduzir o executor da ação | de acordo com; em conformidade com; à luz de; com base em; usado para expressar a opinião subjetiva de alguém}
    \definition{v.}{ceder; recuar; render-se; desistir; admitir | convidar; oferecer | deixar; permitir; fazer | deixar alguém ter algo por um preço justo | ser inferior a; não ser tão bom quanto | ceder; afastar-se | expressar desejos | esquivar-se; evitar; fugir | Usado antes de 我们, indica uma ordem ou sugestão para que todos façam algo juntos}
  \seealsoref{我们}{wo3men5}
  \end{Phonetics}
\end{Entry}

\begin{Entry}{让步}{5,7}{⾔、⽌}
  \begin{Phonetics}{让步}{rang4bu4}
    \definition{v.+compl.}{fazer uma concessão | entregar | desistir | comprometer}
  \end{Phonetics}
\end{Entry}

\begin{Entry}{让座}{5,10}{⾔、⼴}
  \begin{Phonetics}{让座}{rang4 zuo4}[][HSK 6]
    \definition{v.}{oferecer seu lugar a alguém; ceder seu lugar a alguém | convidar os convidados para se sentarem}
  \end{Phonetics}
\end{Entry}

\begin{Entry}{训}{5}{⾔}
  \begin{Phonetics}{训}{xun4}
    \definition{s.}{instrução; ensinamento; ensino | padrão; modelo; exemplo; regra; diretriz | explicação ou interpretação crítica de um texto | treinamento; exercício}
    \definition{v.}{instruir; admoestar; dar uma palestra a alguém; ensinar | explicar; instruir; explicação do significado da palavra | treinar}
  \end{Phonetics}
\end{Entry}

\begin{Entry}{训诂}{5,7}{⾔、⾔}
  \begin{Phonetics}{训诂}{xun4gu3}
    \definition{s.}{estudos exegéticos (de textos antigos); exegese}
    \definition{v.}{explicação de palavras e frases em livros antigos | interpretar e elaborar glossários e comentários sobre textos clássicos}
  \end{Phonetics}
\end{Entry}

\begin{Entry}{训练}{5,8}{⾔、⽷}
  \begin{Phonetics}{训练}{xun4lian4}[][HSK 3]
    \definition{v.}{treinar; exercitar; planejar e executar de forma sistemática o desenvolvimento de habilidades ou competências específicas}
  \end{Phonetics}
\end{Entry}

\begin{Entry}{议}{5}{⾔}
  \begin{Phonetics}{议}{yi4}
    \definition[个,则,条]{s.}{opinião; visão}
    \definition{v.}{discutir; trocar pontos de vista sobre; conversar sobre | comentar; observar | fofocar; comentar}
  \end{Phonetics}
\end{Entry}

\begin{Entry}{议论}{5,6}{⾔、⾔}
  \begin{Phonetics}{议论}{yi4lun4}[][HSK 4]
    \definition[个]{s.}{comentário; discussão; opiniões ou pontos de vista sobre o que é bom ou ruim, certo ou errado em relação a pessoas ou coisas}
    \definition{v.}{discutir; comentar; falar sobre; expressar opiniões e trocar pontos de vista sobre o bom, o ruim, o certo e o errado de pessoas ou coisas}
  \end{Phonetics}
\end{Entry}

\begin{Entry}{议题}{5,15}{⾔、⾴}
  \begin{Phonetics}{议题}{yi4 ti2}[][HSK 6]
    \definition[项,个]{s.}{assunto; assunto em discussão; tópico para discussão}
  \end{Phonetics}
\end{Entry}

\begin{Entry}{记}{5}{⾔}
  \begin{Phonetics}{记}{ji4}[][HSK 1]
    \definition*{s.}{Sobrenome Ji}
    \definition{clas.}{tapas, palmadas, bofetadas, etc.; usado para indicar o número de vezes que uma determinada ação é realizada}
    \definition{s.}{assinatura; bloco de notas; livro ou artigo que registra fatos | insígnia; indicação; \& comercial; símbolo | marca de nascença; manchas escuras presentes na pele desde o nascimento}
    \definition{v.}{lembrar; ter em mente; guardar na memória; manter a imagem na mente | escrever (anotar); registrar; inscrever}
  \end{Phonetics}
\end{Entry}

\begin{Entry}{记忆}{5,4}{⾔、⼼}
  \begin{Phonetics}{记忆}{ji4yi4}[][HSK 5]
    \definition[段]{s.}{memória; manter em sua mente uma imagem do passado}
    \definition{v.}{recordar; lembrar; lembrar-se ou recordar alguém ou algo do passado}
  \end{Phonetics}
\end{Entry}

\begin{Entry}{记住}{5,7}{⾔、⼈}
  \begin{Phonetics}{记住}{ji4 zhu5}[][HSK 1]
    \definition{v.}{lembrar; aprender de cor; ter em mente; guardar na memória}
  \end{Phonetics}
\end{Entry}

\begin{Entry}{记录}{5,8}{⾔、⼹}
  \begin{Phonetics}{记录}{ji4lu4}[][HSK 3]
    \definition[份,名,位,个]{s.}{notas; registro | anotador; registrador; a pessoa que faz registros}
    \definition{v.}{tomar notas; registrar; escrever o que ouviu ou o que aconteceu; gravar o som ou a imagem com um gravador ou uma câmera de vídeo e transformar em algum tipo de obra}
  \end{Phonetics}
\end{Entry}

\begin{Entry}{记性}{5,8}{⾔、⼼}
  \begin{Phonetics}{记性}{ji4xing5}
    \definition{s.}{memória (habilidade em reter informações)}
  \end{Phonetics}
\end{Entry}

\begin{Entry}{记者}{5,8}{⾔、⽼}
  \begin{Phonetics}{记者}{ji4zhe3}[][HSK 3]
    \definition[群,名,位]{s.}{repórter; correspondente; jornalista; profissionais dedicados a entrevistar e reportar notícias para a mídia}
  \end{Phonetics}
\end{Entry}

\begin{Entry}{记载}{5,10}{⾔、⾞}
  \begin{Phonetics}{记载}{ji4zai3}[][HSK 4]
    \definition[段,种,条]{s.}{registro; conta; artigos e materiais que registram eventos}
    \definition{v.}{registrar; colocar por escrito}
  \end{Phonetics}
\end{Entry}

\begin{Entry}{记得}{5,11}{⾔、⼻}
  \begin{Phonetics}{记得}{ji4de5}[][HSK 1]
    \definition{v.}{lembrar; recordar; lembrar-se; não esquecer | manter algo em mente; (informal) não se esquecer de fazer algo, usado para lembrar}
  \end{Phonetics}
\end{Entry}

\begin{Entry}{边}{5}{⾡}
  \begin{Phonetics}{边}{bian1}[][HSK 2]
    \definition*{s.}{Sobrenome Bian}
    \definition{adv.}{dois ou mais 边 são usados separadamente antes de diferentes verbos, indicando que diferentes ações ocorrem simultaneamente}
    \definition[条,个]{s.}{lado (de uma figura geométrica) | borda; lado; margem; aba; rebordo | fronteira; limite | ao lado de; lugar próximo a; perto de um objeto; lateral | aro; aba; borda; decoração em forma de faixa incrustada ou pintada na borda de um objeto}
    \definition{suf.}{lado; anexado a palavras de localização monossilábicas, formando palavras de localização dissílabas}
  \end{Phonetics}
  \begin{Phonetics}{边}{bian5}
    \definition{suf.}{sufixo de uma palavra de localidade (lado); indica posição e direção, usado após palavras que indicam direção, como 上, 下, 前, 后, 左, 右}
  \end{Phonetics}
\end{Entry}

\begin{Entry}{边关}{5,6}{⾡、⼋}
  \begin{Phonetics}{边关}{bian1guan1}
    \definition{s.}{posto de fronteira | posição defensiva estratégica na fronteira}
  \end{Phonetics}
\end{Entry}

\begin{Entry}{边防}{5,6}{⾡、⾩}
  \begin{Phonetics}{边防}{bian1fang2}
    \definition{s.}{defesa da fronteira}
  \end{Phonetics}
\end{Entry}

\begin{Entry}{边缘}{5,12}{⾡、⽷}
  \begin{Phonetics}{边缘}{bian1yuan2}[][HSK 6]
    \definition{s.}{borda; beira; franja; uma área ou objeto próximo ao extremo |  borda; beira; algo está muito próximo de uma situação perigosa | interdisciplinar; relacionado a muitas coisas}
  \end{Phonetics}
\end{Entry}

\begin{Entry}{边境}{5,14}{⾡、⼟}
  \begin{Phonetics}{边境}{bian1jing4}[][HSK 5]
    \definition{s.}{fronteira; borda; perto da fronteira}
  \end{Phonetics}
\end{Entry}

\begin{Entry}{闪}{5}{⾨}
  \begin{Phonetics}{闪}{shan3}[][HSK 4]
    \definition*{s.}{Sobrenome Shan}
    \definition{s.}{relâmpago}
    \definition{v.}{esquivar-se; desviar; sair do caminho | torcer; distender | surgir de repente | cintilar; brilhar | deixar para trás; abandonar | (corpo) oscilar dramaticamente}
  \end{Phonetics}
\end{Entry}

\begin{Entry}{闪电}{5,5}{⾨、⽥}
  \begin{Phonetics}{闪电}{shan3dian4}[][HSK 4]
    \definition[道]{s.}{relâmpago; descargas elétricas entre nuvens ou entre nuvens e o solo}
  \seealsoref{雷电}{lei2dian4}
  \end{Phonetics}
\end{Entry}

\begin{Entry}{闪存盘}{5,6,11}{⾨、⼦、⽫}
  \begin{Phonetics}{闪存盘}{shan3cun2pan2}
    \definition{s.}{unidade de memória \emph{USB} | cartão de memória}
  \seealsoref{优盘}{you1pan2}
  \end{Phonetics}
\end{Entry}

\begin{Entry}{鸟}{5}{⿃}[Kangxi 196]
  \begin{Phonetics}{鸟}{diao3}
    \definition{s.}{(em romances tradicionais, como termo pejorativo) maldito; condenado; fudido; o mesmo que 屌}
  \seealsoref{屌}{diao3}
  \end{Phonetics}
  \begin{Phonetics}{鸟}{niao3}[][HSK 2]
    \definition*{s.}{Sobrenome Niao}
    \definition[只,群]{s.}{pássaro; ave}
  \end{Phonetics}
\end{Entry}

\begin{Entry}{鸟儿}{5,2}{⿃、⼉}
  \begin{Phonetics}{鸟儿}{niao3r5}
    \definition[只]{s.}{pássaro | ave}
  \end{Phonetics}
\end{Entry}

\begin{Entry}{龙}{5}{⿓}[Kangxi 212]
  \begin{Phonetics}{龙}{long2}[][HSK 3]
    \definition*{s.}{Sobrenome Long}
    \definition[条]{s.}{dragão; animal mítico e sobrenatural, com chifres, escamas, garras e bigodes, capaz de voar e mergulhar na água, provocar nuvens e chuva | dinossauro; um enorme réptil extinto; referência a certos répteis gigantes da antiguidade | do imperador; dragão como símbolo do imperador; usado na era feudal como símbolo do imperador; também se refere a coisas pertencentes ao imperador | em forma de dragão; com um desenho de dragão; refere-se a certos objetos que formam uma sequência semelhante a um dragão ou decorados com motivos de dragões}
  \end{Phonetics}
\end{Entry}

\begin{Entry}{龙山}{5,3}{⿓、⼭}
  \begin{Phonetics}{龙山}{long2shan1}
    \definition*{s.}{Longshan}
  \end{Phonetics}
\end{Entry}

\begin{Entry}{龙虾}{5,9}{⿓、⾍}
  \begin{Phonetics}{龙虾}{long2xia1}
    \definition{s.}{lagosta}
  \end{Phonetics}
\end{Entry}

%%%%% EOF %%%%%


%%%
%%% 6画
%%%

\section*{6画}\addcontentsline{toc}{section}{6画}

\begin{entry}{丢}{6}{⼛}
  \begin{phonetics}{丢}{diu1}[][HSK 5]
    \definition{v.}{perder; extraviar; estar ausente | lançar; atirar | colocar (deixar) de lado | deixar (para trás)}
  \end{phonetics}
\end{entry}

\begin{entry}{丢下}{6,3}{⼛、⼀}
  \begin{phonetics}{丢下}{diu1xia4}
    \definition{v.}{abandonar}
  \end{phonetics}
\end{entry}

\begin{entry}{丢开}{6,4}{⼛、⼶}
  \begin{phonetics}{丢开}{diu1kai1}
    \definition{v.}{jogar fora ou deixar de lado | esquecer por um tempo}
  \end{phonetics}
\end{entry}

\begin{entry}{丢失}{6,5}{⼛、⼤}
  \begin{phonetics}{丢失}{diu1shi1}
    \definition{v.}{perder}
  \end{phonetics}
\end{entry}

\begin{entry}{丢弃}{6,7}{⼛、⼶}
  \begin{phonetics}{丢弃}{diu1qi4}
    \definition{v.}{jogar fora | descartar}
  \end{phonetics}
\end{entry}

\begin{entry}{丢官}{6,8}{⼛、⼧}
  \begin{phonetics}{丢官}{diu1guan1}
    \definition{v.}{perder um cargo oficial}
  \end{phonetics}
\end{entry}

\begin{entry}{丢掉}{6,11}{⼛、⼿}
  \begin{phonetics}{丢掉}{diu1diao4}
    \definition{v.}{jogar fora | descartar |perder}
  \end{phonetics}
\end{entry}

\begin{entry}{丢脸}{6,11}{⼛、⾁}
  \begin{phonetics}{丢脸}{diu1lian3}
    \definition{adj.}{vergonhoso}
  \end{phonetics}
\end{entry}

\begin{entry}{乒}{6}{⼃}
  \begin{phonetics}{乒}{ping1}
    \definition{interj.}{(onomatopéia) estalo; estouro; estrondo | (onomatopéia)  ``ping''}
    \definition{s.}{(abreviação) tênis de mesa; pingue-pongue | (abreviação) bola de tênis de mesa; bola de pingue-pongue}
  \end{phonetics}
\end{entry}

\begin{entry}{乒乓球}{6,6,11}{⼃、⼃、⽟}
  \begin{phonetics}{乒乓球}{ping1pang1qiu2}
    \definition[个]{s.}{tênis de mesa |ping-pong}
  \end{phonetics}
\end{entry}

\begin{entry}{乓}{6}{⼃}
  \begin{phonetics}{乓}{pang1}
    \definition{interj.}{(onomatopéia) barulho repentino feito por tiros, uma porta batendo, coisas quebrando, etc.; estrondo; estouro; batida; colisão}
  \end{phonetics}
\end{entry}

\begin{entry}{买}{6}{⼄}
  \begin{phonetics}{买}{mai3}[][HSK 1]
    \definition*{s.}{Sobrenome Mai}
    \definition{v.}{comprar; adquirir | comprar; subornar; usar dinheiro ou outros meios para angariar apoio| pedir; obter; trocar dinheiro por coisas}
  \end{phonetics}
\end{entry}

\begin{entry}{买东西}{6,5,6}{⼄、⼀、⾑}
  \begin{phonetics}{买东西}{mai3dong1xi5}
    \definition{v.}{fazer compras}
  \end{phonetics}
\end{entry}

\begin{entry}{买卖}{6,8}{⼄、⼗}
  \begin{phonetics}{买卖}{mai3 mai4}[][HSK 5]
    \definition[种,笔]{s.}{negócio; compra e venda; transação | (privado) loja; armazém;}
  \end{phonetics}
\end{entry}

\begin{entry}{争}{6}{⼑}
  \begin{phonetics}{争}{zheng1}[][HSK 3]
    \definition*{s.}{Sobrenome Zheng}
    \definition{adv.}{como; por que}
    \definition{v.}{competir; disputar; lutar; esforçar-se para obter ou alcançar | discutir; argumentar; contestar; debater | faltar; estar em falta}
  \end{phonetics}
\end{entry}

\begin{entry}{争风吃醋}{6,4,6,15}{⼑、⾵、⼝、⾣}
  \begin{phonetics}{争风吃醋}{zheng1feng1chi1cu4}
    \definition{v.}{rivalizar com alguém pelo carinho de um homem ou mulher |estar com ciúmes de um rival em um caso de amor}
  \end{phonetics}
\end{entry}

\begin{entry}{争议}{6,5}{⼑、⾔}
  \begin{phonetics}{争议}{zheng1yi4}[][HSK 5]
    \definition{s.}{disputa; controvérsia; situações e questões em que há divergências de opinião}
    \definition{v.}{debater; discutir}
  \end{phonetics}
\end{entry}

\begin{entry}{争先}{6,6}{⼑、⼉}
  \begin{phonetics}{争先}{zheng1xian1}
    \definition{v.}{competir para ser o primeiro |contestar o primeiro lugar}
  \end{phonetics}
\end{entry}

\begin{entry}{争论}{6,6}{⼑、⾔}
  \begin{phonetics}{争论}{zheng1lun4}[][HSK 4]
    \definition{s.}{debate; discussão; argumentação; disputa}
    \definition{v.}{discutir; disputar; debater; argumentar; contestar}
  \end{phonetics}
\end{entry}

\begin{entry}{争取}{6,8}{⼑、⼜}
  \begin{phonetics}{争取}{zheng1qu3}[][HSK 3]
    \definition{v.}{lutar por; conquistar; vencer; se esforçar para conseguir}
  \end{phonetics}
\end{entry}

\begin{entry}{争霸}{6,21}{⼑、⾬}
  \begin{phonetics}{争霸}{zheng1ba4}
    \definition{s.}{hegemonia | uma luta de poder}
    \definition{v.}{disputar a hegemonia}
  \end{phonetics}
\end{entry}

\begin{entry}{亚}{6}{⼆}
  \begin{phonetics}{亚}{ya4}
    \definition*{s.}{Ásia, abreviação de 亚洲 | Sobrenome Ya}
    \definition{adj.}{inferior | abaixo do padrão | (química) de menor valência atômica}
    \definition{pref.}{sub-}
  \seealsoref{亚洲}{ya4zhou1}
  \end{phonetics}
\end{entry}

\begin{entry}{亚军}{6,6}{⼆、⼍}
  \begin{phonetics}{亚军}{ya4jun1}[][HSK 5]
    \definition{s.}{segundo lugar; vice-campeão; medalhista de prata}
  \end{phonetics}
\end{entry}

\begin{entry}{亚运会}{6,7,6}{⼆、⾡、⼈}
  \begin{phonetics}{亚运会}{ya4 yun4 hui4}[][HSK 4]
    \definition*{s.}{Jogos Asiáticos}
  \end{phonetics}
\end{entry}

\begin{entry}{亚细亚洲}{6,8,6,9}{⼆、⽷、⼆、⽔}
  \begin{phonetics}{亚细亚洲}{ya4xi4ya4zhou1}
    \definition*{s.}{Ásia}
  \end{phonetics}
\end{entry}

\begin{entry}{亚洲}{6,9}{⼆、⽔}
  \begin{phonetics}{亚洲}{ya4zhou1}
    \definition*{s.}{Ásia, abreviação de~亚细亚洲}
  \seealsoref{亚细亚洲}{ya4xi4ya4zhou1}
  \end{phonetics}
\end{entry}

\begin{entry}{亚洲人}{6,9,2}{⼆、⽔、⼈}
  \begin{phonetics}{亚洲人}{ya4zhou1ren2}
    \definition{s.}{asiático | pessoa ou povo da Ásia}
  \end{phonetics}
\end{entry}

\begin{entry}{亚热带}{6,10,9}{⼆、⽕、⼱}
  \begin{phonetics}{亚热带}{ya4re4dai4}
    \definition{s.}{zona ou clima subtropical; subtropical; semitropical}
  \end{phonetics}
\end{entry}

\begin{entry}{交}{6}{⼇}
  \begin{phonetics}{交}{jiao1}[][HSK 2]
    \definition*{s.}{Sobrenome Jiao}
    \definition{adv.}{mutuamente; recíprocamente; um ao outro | juntos; simultaneamente}
    \definition{s.}{amigo; conhecido; amizade; relacionamento | transação comercial; negócio; barganha | queda}
    \definition{v.}{entregar | (de lugares ou períodos de tempo) cruzar; encontrar; unir | chegar (a uma determinada hora ou estação); estabelecer-se; vir | cruzar; intersectar | associar-se a | ter relações sexuais | acasalar; reproduzir-se | transferir as coisas para as partes interessadas | unir (lugares ou períodos de tempo)}
  \end{phonetics}
\end{entry}

\begin{entry}{交叉}{6,3}{⼇、⼜}
  \begin{phonetics}{交叉}{jiao1cha1}
    \definition{v.}{cruzar | sobrepor}
  \end{phonetics}
\end{entry}

\begin{entry}{交叉口}{6,3,3}{⼇、⼜、⼝}
  \begin{phonetics}{交叉口}{jiao1cha1kou3}
    \definition{s.}{intersecção (rodovia)}
  \end{phonetics}
\end{entry}

\begin{entry}{交叉点}{6,3,9}{⼇、⼜、⽕}
  \begin{phonetics}{交叉点}{jiao1cha1dian3}
    \definition{s.}{encruzilhada | cruzamento | junção}
  \end{phonetics}
\end{entry}

\begin{entry}{交代}{6,5}{⼇、⼈}
  \begin{phonetics}{交代}{jiao1dai4}[][HSK 5]
    \definition{v.}{contar; entregar | ordenar; insistir; contar aos outros sobre suas intenções, instruções | contar; admitir}
  \end{phonetics}
\end{entry}

\begin{entry}{交运}{6,7}{⼇、⾡}
  \begin{phonetics}{交运}{jiao1yun4}
    \definition{v.}{despachar (bagagem em um aeroporto, etc.) | entregar para transporte}
  \end{phonetics}
\end{entry}

\begin{entry}{交际}{6,7}{⼇、⾩}
  \begin{phonetics}{交际}{jiao1ji4}[][HSK 4]
    \definition{s.}{contato; comunicação; relações sociais; contato interpessoal, socialização}
  \end{phonetics}
\end{entry}

\begin{entry}{交往}{6,8}{⼇、⼻}
  \begin{phonetics}{交往}{jiao1wang3}[][HSK 3]
    \definition{v.}{estar em contato com; associar-se a; interagir}
  \end{phonetics}
\end{entry}

\begin{entry}{交易}{6,8}{⼇、⽇}
  \begin{phonetics}{交易}{jiao1yi4}[][HSK 3]
    \definition[笔,桩,个,场]{s.}{negócio; comércio; transação comercial; transação; atividades de compra e venda de mercadorias}
    \definition{v.}{negociar; comprar e vender mercadorias}
  \end{phonetics}
\end{entry}

\begin{entry}{交朋友}{6,8,4}{⼇、⽉、⼜}
  \begin{phonetics}{交朋友}{jiao1 peng2 you3}[][HSK 2]
    \definition{v.}{fazer amizade com alguém; fazer amigos}
  \end{phonetics}
\end{entry}

\begin{entry}{交杯酒}{6,8,10}{⼇、⽊、⾣}
  \begin{phonetics}{交杯酒}{jiao1bei1jiu3}
    \definition{s.}{copo de vinho nupcial}
  \end{phonetics}
\end{entry}

\begin{entry}{交响}{6,9}{⼇、⼝}
  \begin{phonetics}{交响}{jiao1xiang3}
    \definition{s.}{sinfonia}
  \end{phonetics}
\end{entry}

\begin{entry}{交界}{6,9}{⼇、⽥}
  \begin{phonetics}{交界}{jiao1jie4}
    \definition{s.}{fronteira comum | limite comum | interface}
  \end{phonetics}
\end{entry}

\begin{entry}{交给}{6,9}{⼇、⽷}
  \begin{phonetics}{交给}{jiao1 gei3}[][HSK 2]
    \definition{v.}{entregar para | dar para}
  \end{phonetics}
\end{entry}

\begin{entry}{交费}{6,9}{⼇、⾙}
  \begin{phonetics}{交费}{jiao1 fei4}[][HSK 3]
    \definition{v.}{pagar taxas ou impostos; pagar uma taxa ou imposto}
  \end{phonetics}
\end{entry}

\begin{entry}{交换}{6,10}{⼇、⼿}
  \begin{phonetics}{交换}{jiao1huan4}[][HSK 4]
    \definition{v.}{trocar; permutar; comutar; intercambiar}
  \end{phonetics}
\end{entry}

\begin{entry}{交流}{6,10}{⼇、⽔}
  \begin{phonetics}{交流}{jiao1liu2}[][HSK 3]
    \definition{v.}{trocar; interagir; comunicar-se; compartilhar o que cada um tem com o outro}
  \end{phonetics}
\end{entry}

\begin{entry}{交班}{6,10}{⼇、⽟}
  \begin{phonetics}{交班}{jiao1ban1}
    \definition{v.}{passar para o próximo turno de trabalho}
  \end{phonetics}
\end{entry}

\begin{entry}{交通}{6,10}{⼇、⾡}
  \begin{phonetics}{交通}{jiao1tong1}[][HSK 2]
    \definition{s.}{tráfego | ligação; conexão | transporte; termo genérico para todos os tipos de transporte, como ferroviário e rodoviário}
    \definition{v.}{conspirar; fazer amizades; conchavar | estar conectado; estar ligado; estar vinculado | associar-se a; conspirar com}
  \end{phonetics}
\end{entry}

\begin{entry}{交通警察}{6,10,19,14}{⼇、⾡、⾔、⼧}
  \begin{phonetics}{交通警察}{jiao1tong1jing3cha2}
    \definition{s.}{policial de trânsito}
  \seealsoref{交警}{jiao1 jing3}
  \end{phonetics}
\end{entry}

\begin{entry}{交叠}{6,13}{⼇、⼜}
  \begin{phonetics}{交叠}{jiao1die2}
    \definition{s.}{sobreposição}
  \end{phonetics}
\end{entry}

\begin{entry}{交媾}{6,13}{⼇、⼥}
  \begin{phonetics}{交媾}{jiao1gou4}
    \definition{v.}{copular | ter relações sexuais}
  \end{phonetics}
\end{entry}

\begin{entry}{交警}{6,19}{⼇、⾔}
  \begin{phonetics}{交警}{jiao1 jing3}[][HSK 3]
    \definition{s.}{policial de trânsito, abreviação de 交通警察}
  \seealsoref{交通警察}{jiao1tong1jing3cha2}
  \end{phonetics}
\end{entry}

\begin{entry}{亦}{6}{⼇}
  \begin{phonetics}{亦}{yi4}
    \definition{adv.}{também | igualmente | apenas | embora | já}
  \end{phonetics}
\end{entry}

\begin{entry}{产}{6}{⼇}
  \begin{phonetics}{产}{chan3}
    \definition*{s.}{Sobrenome Chan}
    \definition{s.}{produto | propriedade; espólio | (abreviação) indústria}
    \definition{v.}{dar à luz; ser entregue a | produzir; render | separar um ser humano ou animal de sua mãe}
  \end{phonetics}
\end{entry}

\begin{entry}{产业}{6,5}{⼇、⼀}
  \begin{phonetics}{产业}{chan3ye4}[][HSK 5]
    \definition{s.}{patrimônio; propriedade; bens pessoais, como terrenos, casas, fábricas, etc. | indústria; refere-se especificamente à produção industrial moderna | setor; indústria; indústrias e setores da economia nacional}
  \end{phonetics}
\end{entry}

\begin{entry}{产生}{6,5}{⼇、⽣}
  \begin{phonetics}{产生}{chan3sheng1}[][HSK 3]
    \definition{v.}{produzir; evoluir; emergir; provocar; vir a ser; dar origem a; criar coisas novas e novos fenômenos a partir do que já existe}
  \end{phonetics}
\end{entry}

\begin{entry}{产后}{6,6}{⼇、⼝}
  \begin{phonetics}{产后}{chan3hou4}
    \definition{s.}{pós-parto}
  \end{phonetics}
\end{entry}

\begin{entry}{产品}{6,9}{⼇、⼝}
  \begin{phonetics}{产品}{chan3pin3}[][HSK 4]
    \definition[个,件,种,批,项,类]{s.}{produto; item produzido}
  \end{phonetics}
\end{entry}

\begin{entry}{产量}{6,12}{⼇、⾥}
  \begin{phonetics}{产量}{chan3 liang4}[][HSK 6]
    \definition{v.}{rendimento; produção; a quantidade de produção; a quantidade total de produtos produzidos em um determinado período de tempo}
  \end{phonetics}
\end{entry}

\begin{entry}{仰}{6}{⼈}
  \begin{phonetics}{仰}{yang3}[][HSK 6]
    \definition*{s.}{Sobrenome Yang}
    \definition{v.}{levantar (oposto a 俯) | virar para cima | admirar; respeitar | confiar em; depender de}
  \seealsoref{俯}{fu3}
  \end{phonetics}
\end{entry}

\begin{entry}{件}{6}{⼈}
  \begin{phonetics}{件}{jian4}[][HSK 2]
    \definition*{s.}{Sobrenome Jian}
    \definition{clas.}{item; peça; artigo; usado para coisas individuais}
    \definition{s.}{refere-se a coisas que podem ser contadas uma a uma | papel; carta; documento; correspondência}
  \end{phonetics}
\end{entry}

\begin{entry}{价}{6}{⼈}
  \begin{phonetics}{价}{jia4}[][HSK 5]
    \definition{s.}{preço | valor; (figurativo) valores (éticos, culturais etc.) | (química) valência}
  \end{phonetics}
\end{entry}

\begin{entry}{价值}{6,10}{⼈、⼈}
  \begin{phonetics}{价值}{jia4zhi2}[][HSK 3]
    \definition{s.}{valor; o trabalho social necessário condensado nos produtos | valor; importância; efeitos positivos}
  \end{phonetics}
\end{entry}

\begin{entry}{价格}{6,10}{⼈、⽊}
  \begin{phonetics}{价格}{jia4ge2}[][HSK 3]
    \definition[个,种]{s.}{preço; tarifa; o valor monetário da mercadoria}
  \end{phonetics}
\end{entry}

\begin{entry}{价钱}{6,10}{⼈、⾦}
  \begin{phonetics}{价钱}{jia4 qian2}[][HSK 3]
    \definition[个,种,笔]{s.}{preço}
  \end{phonetics}
\end{entry}

\begin{entry}{任}{6}{⼈}
  \begin{phonetics}{任}{ren4}[][HSK 3]
    \definition{clas.}{usado para o número de mandatos cumpridos em um cargo oficial}
    \definition{conj.}{não importa (como, o que, etc.); orações de conexão, ou usadas antes de pronomes interrogativos, para expressar incondicionalidade, equivalente a 不管 ou 无论}
    \definition{s.}{escritório; posto oficial; cargo | dever; fardo; responsabilidade}
    \definition{v.}{nomear; designar alguém para um cargo | assumir um emprego; assumir um posto; assumir uma posição | deixar; permitir; dar rédea solta a | suportar; empreender | ceder; permitir sem restrições; deixar (alguém) fazer o que quiser}
  \seealsoref{不管}{bu4guan3}
  \seealsoref{无论}{wu2lun4}
  \end{phonetics}
\end{entry}

\begin{entry}{任务}{6,5}{⼈、⼒}
  \begin{phonetics}{任务}{ren4wu5}[][HSK 3]
    \definition[项,个,种,些]{s.}{tarefa; dever; missão; designação; trabalho designado; responsabilidades designadas}
  \end{phonetics}
\end{entry}

\begin{entry}{任何}{6,7}{⼈、⼈}
  \begin{phonetics}{任何}{ren4he2}[][HSK 3]
    \definition{pron.}{qualquer; qualquer que seja; o que for; não importa o que}
  \end{phonetics}
\end{entry}

\begin{entry}{份}{6}{⼈}
  \begin{phonetics}{份}{fen4}
    \definition{clas.}{usado para emparelhar itens em grupos | usado para jornais, documentos, etc. | usado para partes de um todo | usado para aparência, estado, etc.}
    \definition{s.}{porção; parte | a unidade de divisão; usado após 省, 县, 年, 月,  indica a unidade de divisão | grau; extensão de algo}
  \seealsoref{年}{nian2}
  \seealsoref{省}{sheng3}
  \seealsoref{县}{xian4}
  \seealsoref{月}{yue4}
  \end{phonetics}
\end{entry}

\begin{entry}{企}{6}{⼈}
  \begin{phonetics}{企}{qi3}
    \definition{v.}{ficar na ponta dos pés | esperar ansiosamente por algo; ansiar por | planejar um projeto}
  \end{phonetics}
\end{entry}

\begin{entry}{企业}{6,5}{⼈、⼀}
  \begin{phonetics}{企业}{qi3ye4}[][HSK 4]
    \definition[家,个]{s.}{empresa; estabelecimento; empreendimento; negócio; setores envolvidos em atividades econômicas como produção, transporte, comércio, etc., como fábricas, minas, ferrovias, empresas comerciais, etc.}
  \end{phonetics}
\end{entry}

\begin{entry}{伊}{6}{⼈}
  \begin{phonetics}{伊}{yi1}
    \definition*{s.}{Iraque, abreviação de 伊拉克 | Irã,abreviação de  伊朗 | Sobrenome Yi}
    \definition{part.}{(chinês clássico) partícula introdutória sem significado específico}
    \definition{pron.}{(antigo) pronome de terceira pessoa do singular ("ele" ou "ela") | pronome de segunda pessoa do singular ("você") | que (precedendo um substantivo)}
  \seealsoref{伊拉克}{yi1la1ke4}
  \seealsoref{伊朗}{yi1lang3}
  \end{phonetics}
\end{entry}

\begin{entry}{伊马姆}{6,3,8}{⼈、⾺、⼥}
  \begin{phonetics}{伊马姆}{yi1ma3mu3}
    \definition*{s.}{Islã}
  \seealsoref{伊玛目}{yi1ma3mu4}
  \seealsoref{伊曼}{yi1man4}
  \seealsoref{伊斯兰}{yi1si1lan2}
  \end{phonetics}
\end{entry}

\begin{entry}{伊玛目}{6,7,5}{⼈、⽟、⽬}
  \begin{phonetics}{伊玛目}{yi1ma3mu4}
    \definition*{s.}{Islã}
  \seealsoref{伊马姆}{yi1ma3mu3}
  \seealsoref{伊曼}{yi1man4}
  \seealsoref{伊斯兰}{yi1si1lan2}
  \end{phonetics}
\end{entry}

\begin{entry}{伊拉克}{6,8,7}{⼈、⼿、⼗}
  \begin{phonetics}{伊拉克}{yi1la1ke4}
    \definition*{s.}{Iraque}
  \end{phonetics}
\end{entry}

\begin{entry}{伊朗}{6,10}{⼈、⽉}
  \begin{phonetics}{伊朗}{yi1lang3}
    \definition*{s.}{Irã}
  \end{phonetics}
\end{entry}

\begin{entry}{伊曼}{6,11}{⼈、⽈}
  \begin{phonetics}{伊曼}{yi1man4}
    \definition*{s.}{Islã}
  \seealsoref{伊马姆}{yi1ma3mu3}
  \seealsoref{伊玛目}{yi1ma3mu4}
  \seealsoref{伊斯兰}{yi1si1lan2}
  \end{phonetics}
\end{entry}

\begin{entry}{伊斯兰}{6,12,5}{⼈、⽄、⼋}
  \begin{phonetics}{伊斯兰}{yi1si1lan2}
    \definition*{s.}{Islã}
  \seealsoref{伊马姆}{yi1ma3mu3}
  \seealsoref{伊玛目}{yi1ma3mu4}
  \seealsoref{伊曼}{yi1man4}
  \end{phonetics}
\end{entry}

\begin{entry}{休}{6}{⼈}
  \begin{phonetics}{休}{xiu1}
    \definition{adj.}{feliz; alegre; festivo}
    \definition{adv.}{não; indica proibição ou dissuasão, equivalente a 别 ou 不要}
    \definition{s.}{fortuna e infortúnio; bom e mau}
    \definition{v.}{parar; cessar | descansar | abandonar a esposa e mandá-la para casa; antigamente, o marido mandava a esposa de volta para a casa dos pais e rompia o relacionamento conjugal}
  \seealsoref{别}{bie2}
  \seealsoref{不要}{bu2 yao4}
  \end{phonetics}
\end{entry}

\begin{entry}{休兵}{6,7}{⼈、⼋}
  \begin{phonetics}{休兵}{xiu1bing1}
    \definition{s.}{armistício}
    \definition{v.}{cessar fogo}
  \end{phonetics}
\end{entry}

\begin{entry}{休闲}{6,7}{⼈、⾨}
  \begin{phonetics}{休闲}{xiu1xian2}[][HSK 5]
    \definition{s.}{ócio; lazer; tempo livre}
    \definition{v.}{desfrutar do lazer; sair de férias; aproveitar o tempo livre; parar de trabalhar ou estudar, estar em um estado de lazer e descontração | ficar ocioso}
  \end{phonetics}
\end{entry}

\begin{entry}{休息}{6,10}{⼈、⼼}
  \begin{phonetics}{休息}{xiu1xi5}[][HSK 1]
    \definition{s.}{descanço}
    \definition{v.}{descansar; descansar um pouco; fazer uma pausa; interromper o trabalho, os estudos ou as atividades para recuperar as energias | dormir}
  \end{phonetics}
\end{entry}

\begin{entry}{休息室}{6,10,9}{⼈、⼼、⼧}
  \begin{phonetics}{休息室}{xiu1xi1shi4}
    \definition{s.}{saguão | salão}
  \end{phonetics}
\end{entry}

\begin{entry}{休假}{6,11}{⼈、⼈}
  \begin{phonetics}{休假}{xiu1 jia4}[][HSK 2]
    \definition{v.+compl.}{ter um feriado; tirar férias; sair de férias}
  \end{phonetics}
\end{entry}

\begin{entry}{休憩}{6,16}{⼈、⼼}
  \begin{phonetics}{休憩}{xiu1qi4}
    \definition{v.}{relaxar | descansar | dar um tempo}
  \end{phonetics}
\end{entry}

\begin{entry}{休整}{6,16}{⼈、⽁}
  \begin{phonetics}{休整}{xiu1zheng3}
    \definition{v.}{(militar) descansar e reorganizar}
  \end{phonetics}
\end{entry}

\begin{entry}{众}{6}{⼈}
  \begin{phonetics}{众}{zhong4}
    \definition*{s.}{Câmara dos Deputados, abreviação de 众议院}
    \definition{adj.}{numeroso}
    \definition{adv.}{muitos}
    \definition{s.}{multidão}
  \seealsoref{众议院}{zhong4yi4yuan4}
  \end{phonetics}
\end{entry}

\begin{entry}{众议院}{6,5,9}{⼈、⾔、⾩}
  \begin{phonetics}{众议院}{zhong4yi4yuan4}
    \definition*{s.}{Casa baixa da Assembléia Bicameral | Câmara dos Deputados}
  \end{phonetics}
\end{entry}

\begin{entry}{众多}{6,6}{⼈、⼣}
  \begin{phonetics}{众多}{zhong4 duo1}[][HSK 5]
    \definition{adj.}{muitos; numerosos; multitudinários}
  \end{phonetics}
\end{entry}

\begin{entry}{优}{6}{⼈}
  \begin{phonetics}{优}{you1}
    \definition{adj.}{excelente | superior}
  \end{phonetics}
\end{entry}

\begin{entry}{优于}{6,3}{⼈、⼆}
  \begin{phonetics}{优于}{you1yu2}
    \definition{v.}{superar}
  \end{phonetics}
\end{entry}

\begin{entry}{优先}{6,6}{⼈、⼉}
  \begin{phonetics}{优先}{you1xian1}[][HSK 5]
    \definition{adj.}{anterior; sênior; subjacente}
    \definition{v.}{ter prioridade; ter precedência; colocar-se à frente de outras pessoas ou assuntos}
  \end{phonetics}
\end{entry}

\begin{entry}{优伶}{6,7}{⼈、⼈}
  \begin{phonetics}{优伶}{you1ling2}
    \definition{s.}{ator}
  \end{phonetics}
\end{entry}

\begin{entry}{优秀}{6,7}{⼈、⽲}
  \begin{phonetics}{优秀}{you1xiu4}[][HSK 4]
    \definition{adj.}{esplêndido; excelente; extraordinário; excepcional; notável; descreve moral, qualidades, realizações, aprendizado, etc. muito bons.}
  \end{phonetics}
\end{entry}

\begin{entry}{优良}{6,7}{⼈、⾉}
  \begin{phonetics}{优良}{you1 liang2}[][HSK 4]
    \definition{adj.}{ótimo; bom; excelente; (variedade, qualidade, desempenho, estilo, etc.) muito bom}
  \end{phonetics}
\end{entry}

\begin{entry}{优势}{6,8}{⼈、⼒}
  \begin{phonetics}{优势}{you1shi4}[][HSK 3]
    \definition[种,个]{s.}{vantagem; superioridade; preponderância; posição dominante; uma situação favorável que permite superar o adversário}
  \end{phonetics}
\end{entry}

\begin{entry}{优质}{6,8}{⼈、⾙}
  \begin{phonetics}{优质}{you1zhi4}
    \definition{adj.}{excelente qualidade}
  \end{phonetics}
\end{entry}

\begin{entry}{优厚}{6,9}{⼈、⼚}
  \begin{phonetics}{优厚}{you1hou4}
    \definition{adj.}{generoso}
  \end{phonetics}
\end{entry}

\begin{entry}{优点}{6,9}{⼈、⽕}
  \begin{phonetics}{优点}{you1dian3}[][HSK 3]
    \definition[个,项,种,些]{s.}{mérito; virtude; ponto forte; vantagem (em oposição a 缺点)}
  \seealsoref{缺点}{que1dian3}
  \end{phonetics}
\end{entry}

\begin{entry}{优美}{6,9}{⼈、⽺}
  \begin{phonetics}{优美}{you1mei3}[][HSK 4]
    \definition{adj.}{fino; elegante; gracioso; bonito}
  \end{phonetics}
\end{entry}

\begin{entry}{优选}{6,9}{⼈、⾡}
  \begin{phonetics}{优选}{you1xuan3}
    \definition{v.}{otimizar}
  \end{phonetics}
\end{entry}

\begin{entry}{优格}{6,10}{⼈、⽊}
  \begin{phonetics}{优格}{you1ge2}
    \definition{s.}{iogurte}
  \end{phonetics}
\end{entry}

\begin{entry}{优盘}{6,11}{⼈、⽫}
  \begin{phonetics}{优盘}{you1pan2}
    \definition{s.}{unidade de memória USB}
  \seealsoref{闪存盘}{shan3cun2pan2}
  \end{phonetics}
\end{entry}

\begin{entry}{优惠}{6,12}{⼈、⼼}
  \begin{phonetics}{优惠}{you1hui4}[][HSK 5]
    \definition{adj.}{especial; pechincha; reduzido; com desconto | favorável; preferencial; melhores condições ou tratamento do que o normal, permitindo que as pessoas obtenham mais benefícios}
  \end{phonetics}
\end{entry}

\begin{entry}{优等}{6,12}{⼈、⽵}
  \begin{phonetics}{优等}{you1deng3}
    \definition{adj.}{excelente | de primeira linha | alta classe | da mais alta ordem, superior}
  \end{phonetics}
\end{entry}

\begin{entry}{优裕}{6,12}{⼈、⾐}
  \begin{phonetics}{优裕}{you1yu4}
    \definition{adj.}{abundante | bastante}
    \definition{s.}{abundância}
  \end{phonetics}
\end{entry}

\begin{entry}{伙}{6}{⼈}
  \begin{phonetics}{伙}{huo3}[][HSK 4]
    \definition{clas.}{grupo; multidão; banda}
    \definition{s.}{iguaria; alimentação; refeições | parceiro; companheiro | coletivo de colegas}
    \definition{v.}{combinar; unir}
  \end{phonetics}
\end{entry}

\begin{entry}{伙伴}{6,7}{⼈、⼈}
  \begin{phonetics}{伙伴}{huo3ban4}[][HSK 4]
    \definition[个,位,群]{s.}{parceiro; companheiro; antigo sistema militar de dez pessoas para uma fogueira, o chefe da fogueira, uma pessoa encarregada de cozinhar, com a fogueira é chamado de parceiro da fogueira, agora se refere à participação comum em uma determinada organização ou engajada em certas atividades}
  \end{phonetics}
\end{entry}

\begin{entry}{会}{6}{⼈}
  \begin{phonetics}{会}{hui4}[][HSK 1,2]
    \definition{adv.}{um momento}
    \definition{clas.}{momento; um curto período de tempo}
    \definition{s.}{reunião; festa; conferência; reunião com um objetivo específico | reunião; reunião no trabalho | feira do templo; festival religioso | associação; sociedade; sindicato; certas organizações públicas | oportunidade; ocasião; momento oportuno | cidade principal; capital; cidade central}
    \definition{suf.}{união; grupo; associação}
    \definition{v.}{ser provável que; ter certeza de; indica que é possível realizar (é possível responder à pergunta separadamente) |  poder; ser capaz de; significa saber como fazer ou ter a capacidade de fazer (geralmente se refere a coisas que precisam ser aprendidas) | saber; compreender; entender | encontrar; ver | reunir-se; reunir; agregar; juntar | destacar-se em; ser bom em; ser hábil em; indica proficiência | pagar (ou custear) contas}
  \end{phonetics}
  \begin{phonetics}{会}{kuai4}
    \definition[个,场,次]{s.}{contabilidade}
    \definition{v.}{computar; calcular; equilibrar uma conta}
  \end{phonetics}
\end{entry}

\begin{entry}{会计}{6,4}{⼈、⾔}
  \begin{phonetics}{会计}{kuai4ji4}[][HSK 4]
    \definition[个,位,名]{s.}{contabilidade | contador; contabilista; guarda-livros; pessoal que trabalha como contador}
  \end{phonetics}
\end{entry}

\begin{entry}{会议}{6,5}{⼈、⾔}
  \begin{phonetics}{会议}{hui4yi4}[][HSK 3]
    \definition[次,届,个,场]{s.}{reunião; conferência; reunião organizada pela organização relevante para ouvir opiniões, discutir questões e distribuir tarefas | conselho; congresso; um órgão ou organização permanente que discute e trata frequentemente assuntos importantes}
  \end{phonetics}
\end{entry}

\begin{entry}{会员}{6,7}{⼈、⼝}
  \begin{phonetics}{会员}{hui4 yuan2}[][HSK 3]
    \definition[位,名,个,些]{s.}{membro; associado; membros de certos grupos ou organizações}
  \end{phonetics}
\end{entry}

\begin{entry}{会首}{6,9}{⼈、⾸}
  \begin{phonetics}{会首}{hui4shou3}
    \definition{s.}{chefe de uma sociedade | patrocinador de uma organização}
  \end{phonetics}
\end{entry}

\begin{entry}{会谈}{6,10}{⼈、⾔}
  \begin{phonetics}{会谈}{hui4 tan2}[][HSK 5]
    \definition{v.}{manter conversações; comumente usado em assuntos internacionais ou atividades diplomáticas}
  \end{phonetics}
\end{entry}

\begin{entry}{伞}{6}{⼈}
  \begin{phonetics}{伞}{san3}[][HSK 4]
    \definition*{s.}{Sobrenome San}
    \definition[把]{s.}{guarda-chuva; proteção contra chuva ou sol | algo que tem o formato de um guarda-chuva}
  \end{phonetics}
\end{entry}

\begin{entry}{伟}{6}{⼈}
  \begin{phonetics}{伟}{wei3}
    \definition{adj.}{grande | ótimo}
  \end{phonetics}
\end{entry}

\begin{entry}{伟大}{6,3}{⼈、⼤}
  \begin{phonetics}{伟大}{wei3da4}[][HSK 3]
    \definition{adj.}{ótimo; excelente; extrovertido; descreve uma pessoa com moral e qualidades excelentes, habilidades e realizações excepcionais, que inspira grande respeito | ótimo; magnífico; descreve algo de grande importância, com impacto significativo, acima do normal, algo notável}
  \end{phonetics}
\end{entry}

\begin{entry}{传}{6}{⼈}
  \begin{phonetics}{传}{chuan2}[][HSK 3]
    \definition{v.}{passar; passar adiante | passar adiante; legar; passar de\dots para\dots; passar da geração anterior para a seguinte | transmitir (conhecimento, habilidade, etc.); comunicar; ensinar | espalhar; propagar | transmitir; conduzir; transferir | transmitir; expressar | convocar; dar ordem para chamar alguém | infectar; ser contagioso | enviar documentos por e-mail ou fax}
  \end{phonetics}
  \begin{phonetics}{传}{zhuan4}
    \definition{s.}{comentários sobre clássicos; obras que explicam as escrituras| biografia | romances sobre eventos históricos; obras que narram histórias históricas}
  \end{phonetics}
\end{entry}

\begin{entry}{传出}{6,5}{⼈、⼐}
  \begin{phonetics}{传出}{chuan2 chu1}[][HSK 6]
    \definition{adj.}{eferente (nervo)}
    \definition{v.}{disseminar | transmitir para fora}
  \end{phonetics}
\end{entry}

\begin{entry}{传达}{6,6}{⼈、⾡}
  \begin{phonetics}{传达}{chuan2da2}[][HSK 5]
    \definition{s.}{recepção e registro de chamadas em um estabelecimento público | zelador; recepcionista}
    \definition{v.}{passar adiante (informações, etc.); transmitir; retransmitir; comunicar}
  \end{phonetics}
\end{entry}

\begin{entry}{传来}{6,7}{⼈、⽊}
  \begin{phonetics}{传来}{chuan2 lai2}[][HSK 3]
    \definition{v.}{(um som) passar; transmitir de algum lugar para o local onde o locutor se encontra | (notícias) chegar; transmitir mensagens ou informações}
  \end{phonetics}
\end{entry}

\begin{entry}{传言}{6,7}{⼈、⾔}
  \begin{phonetics}{传言}{chuan2 yan2}[][HSK 6]
    \definition[个,种,些]{s.}{boato; rumor}
    \definition{v.}{passar uma mensagem | datado: fazer um discurso | falar; fazer uma declaração}
  \end{phonetics}
\end{entry}

\begin{entry}{传承}{6,8}{⼈、⼿}
  \begin{phonetics}{传承}{chuan2cheng2}
    \definition{s.}{herança | tradição continuada}
    \definition{v.}{transmitir (para as gerações futuras) | passar adiante (desde os tempos antigos)}
  \end{phonetics}
\end{entry}

\begin{entry}{传给}{6,9}{⼈、⽷}
  \begin{phonetics}{传给}{chuan2gei3}
    \definition{v.}{passar para | transferir para | entregar a}
  \end{phonetics}
\end{entry}

\begin{entry}{传统}{6,9}{⼈、⽷}
  \begin{phonetics}{传统}{chuan2tong3}[][HSK 4]
    \definition{adj.}{tradicional; histórico; transmitido de geração em geração | antiquado, conservador e fora de sintonia com os tempos}
    \definition[个]{s.}{tradição; costume; fatores sociais, como costumes, moral, ideias, estilos, artes, instituições etc., que são transmitidos de uma geração para outra e que são característicos da sociedade}
  \end{phonetics}
\end{entry}

\begin{entry}{传说}{6,9}{⼈、⾔}
  \begin{phonetics}{传说}{chuan2shuo1}[][HSK 3]
    \definition[个,种,段]{s.}{lenda; conto popular; folclore; coisas lendárias; especificamente, lendas populares}
    \definition{v.}{dizer que; ser dito; passar de boca em boca; transmitir oralmente, segundo a tradição}
  \end{phonetics}
\end{entry}

\begin{entry}{传真}{6,10}{⼈、⼗}
  \begin{phonetics}{传真}{chuan2zhen1}[][HSK 5]
    \definition[台,部,份]{s.}{\emph{FAX}, facsímile; texto, diagramas, fotografias, etc., transmitidos por aparelho de fax}
    \definition{v.}{enviar um fax}
  \end{phonetics}
\end{entry}

\begin{entry}{传递}{6,10}{⼈、⾡}
  \begin{phonetics}{传递}{chuan2 di4}[][HSK 5]
    \definition{v.}{transmitir; entregar; transferir; passar adiante}
  \end{phonetics}
\end{entry}

\begin{entry}{传媒}{6,12}{⼈、⼥}
  \begin{phonetics}{传媒}{chuan2 mei2}[][HSK 6]
    \definition{s.}{meios de comunicação de massa; mídia; jornais, rádio, televisão, \emph{Internet} e outras ferramentas de notícias | meio; veículo; vetor; o meio ou via de transmissão da doença}
  \end{phonetics}
\end{entry}

\begin{entry}{传输}{6,13}{⼈、⾞}
  \begin{phonetics}{传输}{chuan2 shu1}[][HSK 6]
    \definition{v.}{transmitir, transportar (energia, informação, etc.)}
  \end{phonetics}
\end{entry}

\begin{entry}{传播}{6,15}{⼈、⼿}
  \begin{phonetics}{传播}{chuan2bo1}[][HSK 3]
    \definition{v.}{espalhar; difundir; propagar; disseminar}
  \end{phonetics}
\end{entry}

\begin{entry}{伤}{6}{⼈}
  \begin{phonetics}{伤}{shang1}[][HSK 3]
    \definition*{s.}{Sobrenome Shang}
    \definition[处]{s.}{ferida; ferimento}
    \definition{v.}{ferir; machucar | ter os sentimentos feridos | estar angustiado | enjoar de algo; desenvolver aversão a algo | ser prejudicial a; entravar}
  \end{phonetics}
\end{entry}

\begin{entry}{伤心}{6,4}{⼈、⼼}
  \begin{phonetics}{伤心}{shang1xin1}[][HSK 3]
    \definition{v.+compl.}{estar triste; lamentar; estar com o coração partido; sentir-se triste por causa de infortúnio ou decepção}
  \end{phonetics}
\end{entry}

\begin{entry}{伤害}{6,10}{⼈、⼧}
  \begin{phonetics}{伤害}{shang1hai4}[][HSK 4]
    \definition{v.}{ferir; prejudicar; machucar; magoar; causar danos físicos ou mentais}
  \end{phonetics}
\end{entry}

\begin{entry}{伦}{6}{⼈}
  \begin{phonetics}{伦}{lun2}
    \definition*{s.}{Sobrenome Lun}
    \definition{s.}{relações humanas (especialmente como concebidas pela ética feudal) | lógica; ordem | par; correspondência; (mesma) classe | ética; relações humanas | sequência lógica; ordem | o mesmo tipo; semelhante; igual}
  \end{phonetics}
\end{entry}

\begin{entry}{伦敦}{6,12}{⼈、⽁}
  \begin{phonetics}{伦敦}{lun2dun1}
    \definition*{s.}{Londres}
  \end{phonetics}
\end{entry}

\begin{entry}{伪}{6}{⼈}
  \begin{phonetics}{伪}{wei3}
    \definition{adj.}{falso; falsificado | fantoche; colaboracionista; ilegal | forjado; falso}
    \definition{pref.}{pseudo-; quasi-; quase-}
  \end{phonetics}
\end{entry}

\begin{entry}{似}{6}{⼈}
  \begin{phonetics}{似}{shi4}
    \definition{v.}{ver; parecer}
  \end{phonetics}
  \begin{phonetics}{似}{si4}
    \definition*{s.}{Sobrenome Si}
    \definition{adv.}{parece; como se}
    \definition{v.}{ser semelhante; parecer-se com | parecer; aparecer | exceder}
  \end{phonetics}
\end{entry}

\begin{entry}{似乎}{6,5}{⼈、⼃}
  \begin{phonetics}{似乎}{si4hu1}[][HSK 4]
    \definition{adv.}{como se; aparentemente; se parece como}
  \end{phonetics}
\end{entry}

\begin{entry}{似的}{6,8}{⼈、⽩}
  \begin{phonetics}{似的}{shi4de5}[][HSK 4]
    \definition{part.}{como; como\dots como; como se (embora); usada após uma palavra ou frase para indicar uma semelhança com algo ou uma situação | usada para indicar alto grau}
  \end{phonetics}
\end{entry}

\begin{entry}{似曾相识}{6,12,9,7}{⼈、⽈、⽬、⾔}
  \begin{phonetics}{似曾相识}{si4ceng2xiang1shi2}
    \definition{s.}{\emph{déjà vu} (a experiência de ver exatamente a mesma situação pela segunda vez) | situação aparentemente familiar}
  \end{phonetics}
\end{entry}

\begin{entry}{充}{6}{⼉}
  \begin{phonetics}{充}{chong1}
    \definition*{s.}{Sobrenome Chong}
    \definition{adj.}{suficiente; completo; amplo; cheio}
    \definition{v.}{encher; carregar; atulhar | servir como; agir como | fingir ser; posar como; passar algo como}
  \end{phonetics}
\end{entry}

\begin{entry}{充分}{6,4}{⼉、⼑}
  \begin{phonetics}{充分}{chong1fen4}[][HSK 4]
    \definition{adj.}{cheio; amplo; abundante; suficiente; adequado}
    \definition{adv.}{totalmente; até o fim}
  \end{phonetics}
\end{entry}

\begin{entry}{充电}{6,5}{⼉、⽥}
  \begin{phonetics}{充电}{chong1 dian4}[][HSK 4]
    \definition{v.}{carregar (uma bateria); conectar uma fonte de alimentação CC aos terminais da bateria para recarregar a bateria | relaxar; passar o tempo livre; ``recarregar as baterias''; estudar para adquirir mais conhecimento; reabastecer (ou ampliar) o conhecimento; metaforicamente falando, para reabastecer a força física e a energia por meio do descanso e da recreação; também metaforicamente falando, para reabastecer novos conhecimentos e desenvolver novas habilidades por meio do reaprendizado}
  \end{phonetics}
\end{entry}

\begin{entry}{充电器}{6,5,16}{⼉、⽥、⼝}
  \begin{phonetics}{充电器}{chong1dian4qi4}[][HSK 4]
    \definition{s.}{carregador de bateria; dispositivo para alimentar uma bateria com energia, forçando uma corrente através dela}
  \end{phonetics}
\end{entry}

\begin{entry}{充足}{6,7}{⼉、⾜}
  \begin{phonetics}{充足}{chong1zu2}[][HSK 5]
    \definition{adj.}{bastante; adequado; suficiente}
  \end{phonetics}
\end{entry}

\begin{entry}{充满}{6,13}{⼉、⽔}
  \begin{phonetics}{充满}{chong1man3}[][HSK 3]
    \definition{v.}{preencher; encher; cobrir completamente| estar cheio de; estar repleto de; estar transbordando de; estar impregnado de}
  \end{phonetics}
\end{entry}

\begin{entry}{兆}{6}{⼉}
  \begin{phonetics}{兆}{zhao4}
    \definition{num.}{trilhão}
  \end{phonetics}
\end{entry}

\begin{entry}{先}{6}{⼉}
  \begin{phonetics}{先}{xian1}[][HSK 1]
    \definition*{s.}{Sobrenome Xian}
    \definition{adv.}{primeiro; antes; mais cedo; com antecedência | no momento; por enquanto; em um curto espaço de tempo; temporariamente}
    \definition{s.}{início; começo; em ordem cronológica ou de precedência | ancestral; geração mais velha; antepassado | tardio; falecido; morto (honrar os mortos)}
  \end{phonetics}
\end{entry}

\begin{entry}{先不先}{6,4,6}{⼉、⼀、⼉}
  \begin{phonetics}{先不先}{xian1bu4xian1}
    \definition{adv.}{(dialeto) antes de tudo | em primeiro lugar}
  \end{phonetics}
\end{entry}

\begin{entry}{先天}{6,4}{⼉、⼤}
  \begin{phonetics}{先天}{xian1tian1}
    \definition{adj.}{congênito | inato | natural}
    \definition{s.}{período embrionário}
  \end{phonetics}
\end{entry}

\begin{entry}{先生}{6,5}{⼉、⽣}
  \begin{phonetics}{先生}{xian1sheng5}[][HSK 1]
    \definition[个,位]{s.}{professor; títulos honoríficos para professores, médicos, etc. | marido; antigamente, referia-se ao marido de outra pessoa ou ao próprio marido (ambos com pronomes pessoais como determinantes) | médico; títulos usados para se referir aos médicos no passado | refere-se a pessoas cuja profissão envolve contar histórias, adivinhação, etc.; antigamente, era chamado de contador | senhor; \emph{sir}; título dado aos intelectuais}
  \end{phonetics}
\end{entry}

\begin{entry}{先后}{6,6}{⼉、⼝}
  \begin{phonetics}{先后}{xian1 hou4}[][HSK 5]
    \definition{adv.}{sucessivamente; um após o outro}
    \definition{s.}{prioridade; ordem; cedo ou tarde; primeiro e último}
  \end{phonetics}
\end{entry}

\begin{entry}{先有}{6,6}{⼉、⽉}
  \begin{phonetics}{先有}{xian1you3}
    \definition{adj.}{preexistente | anterior}
  \end{phonetics}
\end{entry}

\begin{entry}{先进}{6,7}{⼉、⾡}
  \begin{phonetics}{先进}{xian1jin4}[][HSK 3]
    \definition{adj.}{avançado; progressos rápidos e nível elevado, podendo servir de exemplo a seguir}
    \definition{s.}{indivíduos ou grupos avançados}
  \end{phonetics}
\end{entry}

\begin{entry}{先到先得}{6,8,6,11}{⼉、⼑、⼉、⼻}
  \begin{phonetics}{先到先得}{xian1dao4xian1de2}
    \definition{expr.}{primeiro a chegar | primeiro a ser servido}
  \end{phonetics}
\end{entry}

\begin{entry}{先前}{6,9}{⼉、⼑}
  \begin{phonetics}{先前}{xian1qian2}[][HSK 5]
    \definition[出]{s.}{antes; anteriormente; geralmente se refere ao passado ou a um certo tempo anterior}
  \end{phonetics}
\end{entry}

\begin{entry}{先烈}{6,10}{⼉、⽕}
  \begin{phonetics}{先烈}{xian1lie4}
    \definition{s.}{mártir}
  \end{phonetics}
\end{entry}

\begin{entry}{先验}{6,10}{⼉、⾺}
  \begin{phonetics}{先验}{xian1yan4}
    \definition{adj.}{(filosofia) a priori}
  \end{phonetics}
\end{entry}

\begin{entry}{先期}{6,12}{⼉、⽉}
  \begin{phonetics}{先期}{xian1qi1}
    \definition{adv.}{antecipadamente}
    \definition{s.}{prematuro | \emph{front-end}}
  \end{phonetics}
\end{entry}

\begin{entry}{光}{6}{⼉}
  \begin{phonetics}{光}{guang1}[][HSK 3]
    \definition*{s.}{Sobrenome Guang}
    \definition{adj.}{suave; liso; brilhante | esgotado; sem nada sobrando | brilhante}
    \definition{adv.}{somente; sozinho; meramente}
    \definition{s.}{luz; raio | cenário; paisagem | honra; glória; brilho | claridade | favor; graça | momento | corpo celeste; referindo-se especificamente a corpos celestes, como o sol, a lua e as estrelas}
    \definition{v.}{glorificar; recuperar; reconquistar | estar nu; expor}
  \end{phonetics}
\end{entry}

\begin{entry}{光污染}{6,6,9}{⼉、⽔、⽊}
  \begin{phonetics}{光污染}{guang1 wu1ran3}
    \definition{s.}{poluição luminosa}
  \end{phonetics}
\end{entry}

\begin{entry}{光明}{6,8}{⼉、⽇}
  \begin{phonetics}{光明}{guang1ming2}[][HSK 3]
    \definition{adj.}{brilhante; luminoso | sincero; ingênuo; metáfora da justiça e da esperança | justo; honesto; franco}
    \definition{s.}{luz}
  \end{phonetics}
\end{entry}

\begin{entry}{光线}{6,8}{⼉、⽷}
  \begin{phonetics}{光线}{guang1 xian4}[][HSK 5]
    \definition[条,道]{s.}{luz; feixe luminoso; raio de luz}
  \end{phonetics}
\end{entry}

\begin{entry}{光临}{6,9}{⼉、⼁}
  \begin{phonetics}{光临}{guang1lin2}[][HSK 4]
    \definition{v.}{honrar com sua presença, uma palavra de honra, usada para dizer que um convidado chegou}
  \end{phonetics}
\end{entry}

\begin{entry}{光荣}{6,9}{⼉、⾋}
  \begin{phonetics}{光荣}{guang1rong2}[][HSK 5]
    \definition{adj.}{honroso; honrado; glorioso; por fazer algo que é benéfico para o país ou para a coletividade e que é considerado por todos como digno de respeito ou elogio}
    \definition{s.}{honra; glória; crédito; sentimento de honra decorrente do fato de ser respeitado ou elogiado por fazer algo importante ou grandioso}
  \end{phonetics}
\end{entry}

\begin{entry}{光盘}{6,11}{⼉、⽫}
  \begin{phonetics}{光盘}{guang1pan2}[][HSK 4]
    \definition[片,张]{s.}{CD; disco compacto; um disco circular feito de plástico rígido composto que usa um laser para registrar e ler informações}
  \end{phonetics}
\end{entry}

\begin{entry}{光槃}{6,14}{⼉、⽊}
  \begin{phonetics}{光槃}{guang1pan2}
    \variantof{光盘}
  \end{phonetics}
\end{entry}

\begin{entry}{全}{6}{⼊}
  \begin{phonetics}{全}{quan2}[][HSK 2]
    \definition*{s.}{Sobrenome Quan}
    \definition{adj.}{completo; total; inteiro}
    \definition{adv.}{inteiramente; totalmente; completamente; significa 100\%; equivalente a 完全 ou 全然}
    \definition{v.}{manter intacto; tornar perfeito ou completo; completar}
  \seealsoref{全然}{quan2ran2}
  \seealsoref{完全}{wan2quan2}
  \end{phonetics}
\end{entry}

\begin{entry}{全世界}{6,5,9}{⼊、⼀、⽥}
  \begin{phonetics}{全世界}{quan2 shi4 jie4}[][HSK 5]
    \definition[种]{s.}{mundo inteiro; mundo todo | em todo o mundo}
  \end{phonetics}
\end{entry}

\begin{entry}{全场}{6,6}{⼊、⼟}
  \begin{phonetics}{全场}{quan2 chang3}[][HSK 3]
    \definition{s.}{toda a audiência; todos os presentes; todo o público}
  \end{phonetics}
\end{entry}

\begin{entry}{全年}{6,6}{⼊、⼲}
  \begin{phonetics}{全年}{quan2 nian2}[][HSK 2]
    \definition{s.}{ano inteiro | anual; todo ano}
  \end{phonetics}
\end{entry}

\begin{entry}{全体}{6,7}{⼊、⼈}
  \begin{phonetics}{全体}{quan2 ti3}[][HSK 2]
    \definition{s.}{(frequentemente referido a pessoas) todos; número total; todos | por todo o corpo | todos; inteiro; a soma de todas as partes; a soma de todos os indivíduos (geralmente se refere a pessoas)}
  \end{phonetics}
\end{entry}

\begin{entry}{全身}{6,7}{⼊、⾝}
  \begin{phonetics}{全身}{quan2 shen1}[][HSK 2]
    \definition{s.}{corpo inteiro; por todo o corpo; todo o corpo}
  \end{phonetics}
\end{entry}

\begin{entry}{全国}{6,8}{⼊、⼞}
  \begin{phonetics}{全国}{quan2 guo2}[][HSK 2]
    \definition{s.}{toda a nação (ou país); em todo o país; em todo o território nacional | toda a nação; todo o país}
  \end{phonetics}
\end{entry}

\begin{entry}{全面}{6,9}{⼊、⾯}
  \begin{phonetics}{全面}{quan2mian4}[][HSK 3]
    \definition{adj.}{geral; completo; abrangente; onipotente}
    \definition{s.}{todos os aspectos; cada aspecto}
  \seealsoref{片面}{pian4mian4}
  \end{phonetics}
\end{entry}

\begin{entry}{全家}{6,10}{⼊、⼧}
  \begin{phonetics}{全家}{quan2 jia1}[][HSK 2]
    \definition{s.}{toda a família; a família inteira}
  \end{phonetics}
\end{entry}

\begin{entry}{全称特命全权大使}{6,10,10,8,6,6,3,8}{⼊、⽲、⽜、⼝、⼊、⽊、⼤、⼈}
  \begin{phonetics}{全称特命全权大使}{quan2cheng1 te4ming4 quan2quan2 da4shi3}
    \definition*{s.}{Embaixador Extraordinário e Plenipotenciário}
  \end{phonetics}
\end{entry}

\begin{entry}{全部}{6,10}{⼊、⾢}
  \begin{phonetics}{全部}{quan2bu4}[][HSK 2]
    \definition{adv.}{tudo; total; inteiro; completo; aplica-se a todos, sem exceção}
    \definition{s.}{totalidade; total; integridade; a soma de todas as partes; o todo}
  \end{phonetics}
\end{entry}

\begin{entry}{全都}{6,10}{⼊、⾢}
  \begin{phonetics}{全都}{quan2 dou1}[][HSK 5]
    \definition{adv.}{tudo; todos; sem exceção}
  \end{phonetics}
\end{entry}

\begin{entry}{全都不}{6,10,4}{⼊、⾢、⼀}
  \begin{phonetics}{全都不}{quan2dou1 bu4}
    \definition{adj.}{nada; nenhum; nenhum deles; nada disso}
  \end{phonetics}
\end{entry}

\begin{entry}{全球}{6,11}{⼊、⽟}
  \begin{phonetics}{全球}{quan2 qiu2}[][HSK 3]
    \definition[门]{s.}{o mundo inteiro; a Terra inteira}
  \end{phonetics}
\end{entry}

\begin{entry}{全职}{6,11}{⼊、⽿}
  \begin{phonetics}{全职}{quan2zhi2}
    \definition{s.}{período integral | tempo inteiro | (trabalho) \emph{full-time}}
  \end{phonetics}
\end{entry}

\begin{entry}{全然}{6,12}{⼊、⽕}
  \begin{phonetics}{全然}{quan2ran2}
    \definition{adv.}{completamente; inteiramente}
  \end{phonetics}
\end{entry}

\begin{entry}{共}{6}{⼋}
  \begin{phonetics}{共}{gong4}[][HSK 4]
    \definition*{s.}{Partido Comunista, abreviação de 共产党 | Sobrenome Gong}
    \definition{adj.}{conjunto; mútuo; geral; comum; o mesmo para todos}
    \definition{adv.}{juntos; juntamente; conjuntamente | em sua totalidade; em todos}
    \definition{v.}{compartilhar com; empreender ou realizar em conjunto}
  \seealsoref{共产党}{gong4chan3dang3}
  \end{phonetics}
\end{entry}

\begin{entry}{共计}{6,4}{⼋、⾔}
  \begin{phonetics}{共计}{gong4ji4}[][HSK 5]
    \definition{s.}{total; total geral; agregado; montante}
    \definition{v.}{contar até; somar até; totalizar}
  \end{phonetics}
\end{entry}

\begin{entry}{共产}{6,6}{⼋、⼇}
  \begin{phonetics}{共产}{gong4chan3}
    \definition{adj.}{comunista}
    \definition{s.}{comunismo}
  \end{phonetics}
\end{entry}

\begin{entry}{共产党}{6,6,10}{⼋、⼇、⼉}
  \begin{phonetics}{共产党}{gong4chan3dang3}
    \definition*{s.}{Partido Comunista}
  \end{phonetics}
\end{entry}

\begin{entry}{共同}{6,6}{⼋、⼝}
  \begin{phonetics}{共同}{gong4tong2}[][HSK 3]
    \definition{adj.}{comum; compartilhado; colaborativo; todos têm}
    \definition{adv.}{juntos; conjuntamente; todos juntos (fazemos)}
  \end{phonetics}
\end{entry}

\begin{entry}{共同体}{6,6,7}{⼋、⼝、⼈}
  \begin{phonetics}{共同体}{gong4tong2ti3}
    \definition{s.}{comunidade}
  \end{phonetics}
\end{entry}

\begin{entry}{共有}{6,6}{⼋、⽉}
  \begin{phonetics}{共有}{gong4 you3}[][HSK 3]
    \definition{v.}{compartilhar; possuir (por todos); possuir ou desfrutar em conjunto}
  \end{phonetics}
\end{entry}

\begin{entry}{共享}{6,8}{⼋、⼇}
  \begin{phonetics}{共享}{gong4 xiang3}[][HSK 5]
    \definition{v.}{compartilhar; desfrutar juntos; aproveitar as coisas boas juntos}
  \end{phonetics}
\end{entry}

\begin{entry}{兲}{6}{⼋}
  \begin{phonetics}{兲}{tian1}
    \variantof{天}
  \end{phonetics}
\end{entry}

\begin{entry}{关}{6}{⼋}
  \begin{phonetics}{关}{guan1}[][HSK 1,4]
    \definition*{s.}{Sobrenome Guan}
    \definition{s.}{passagem; ponto de controle | alfândega; escritórios de cobrança de impostos para exportação e importação de mercadorias | ponto de inflexão ou barreira; ponto de virada ou dificuldade | momento crítico; mecanismo}
    \definition{v.}{fechar; encerrar; amarrar algo | fechar; trancar | encerrar; sair do mercado; falir | conceder ou sacar o pagamento de um salário | desligar | envolver; preocupar-se; conectar-se}
  \end{phonetics}
\end{entry}

\begin{entry}{关上}{6,3}{⼋、⼀}
  \begin{phonetics}{关上}{guan1 shang4}[][HSK 1]
    \definition{v.}{fechar (uma porta); fechar um objeto | desligar (luz, equipamento elétrico etc.); parar ou encerrar (uma atividade, situação, etc.)}
  \end{phonetics}
\end{entry}

\begin{entry}{关于}{6,3}{⼋、⼆}
  \begin{phonetics}{关于}{guan1yu2}[][HSK 4]
    \definition{prep.}{sobre; relativo a; pertencente a; uma questão de; com relação a}
  \end{phonetics}
\end{entry}

\begin{entry}{关心}{6,4}{⼋、⼼}
  \begin{phonetics}{关心}{guan1xin1}[][HSK 2]
    \definition{v.}{cuidar; preocupar-se com; manifestar interesse por; demonstrar solicitude por; (colocar uma pessoa ou coisa) sempre no coração; valorizar e cuidar}
  \end{phonetics}
\end{entry}

\begin{entry}{关机}{6,6}{⼋、⽊}
  \begin{phonetics}{关机}{guan1 ji1}[][HSK 2]
    \definition{v.}{encerrar; terminar; refere-se especificamente à conclusão das filmagens de um filme ou série de TV | desligar; desligar a fonte de alimentação; parar o funcionamento da máquina}
  \end{phonetics}
\end{entry}

\begin{entry}{关闭}{6,6}{⼋、⾨}
  \begin{phonetics}{关闭}{guan1bi4}[][HSK 4]
    \definition{v.}{fechar | (empresa) falir}
  \end{phonetics}
\end{entry}

\begin{entry}{关怀}{6,7}{⼋、⼼}
  \begin{phonetics}{关怀}{guan1huai2}[][HSK 5]
    \definition{v.}{mostrar cuidado amoroso por; mostrar solicitude por; cuidar, amar, apoiar ou ajudar os fracos ou grupos em dificuldade | geralmente usado para superiores para subordinados, anciãos para juniores ou organizações para indivíduos}
  \end{phonetics}
\end{entry}

\begin{entry}{关系}{6,7}{⼋、⽷}
  \begin{phonetics}{关系}{guan1xi5}[][HSK 3]
    \definition[个,种]{s.}{relações; conexões; relacionamento; a interligação entre pessoas ou coisas | consequência; impacto; significado a influência ou importância de algo; algo digno de nota (geralmente usado com 没有, 有). | causa; razão (geralmente usado com 由于 ou 因为); refere-se genericamente a causas, condições, etc. | credenciais que mostram filiação a uma organização; documento que comprova a existência de algum tipo de relação organizacional}
    \definition{v.}{preocupar; afetar; ter influência sobre; ter a ver com}
  \seealsoref{没有}{mei2 you3}
  \seealsoref{因为}{yin1wei4}
  \seealsoref{由于}{you2yu2}
  \seealsoref{有}{you3}
  \end{phonetics}
\end{entry}

\begin{entry}{关注}{6,8}{⼋、⽔}
  \begin{phonetics}{关注}{guan1 zhu4}[][HSK 3]
    \definition{v.}{prestar atenção em; seguir algo de perto; seguir (nas redes sociais)}
  \end{phonetics}
\end{entry}

\begin{entry}{关键}{6,13}{⼋、⾦}
  \begin{phonetics}{关键}{guan1jian4}[][HSK 5]
    \definition{adj.}{crucial; decisivo; importante; que pode determinar o curso e o resultado dos eventos}
    \definition[个]{s.}{chave; ponto crucial; aspectos ou condições mais importantes que determinam o desenvolvimento e o resultado de algo}
  \end{phonetics}
\end{entry}

\begin{entry}{兴}{6}{⼋}
  \begin{phonetics}{兴}{xing1}
    \definition*{s.}{Sobrenome Xing}
    \definition{adv.}{talvez (dialeto)}
    \definition{v.}{subir | florescer | tornar-se popular | começar | encorajar | levantar-se | (frequentemente usado em negativas) permitir (dialeto)}
  \end{phonetics}
  \begin{phonetics}{兴}{xing4}
    \definition{s.}{sentimento ou desejo de fazer algo | interesse em algo | excitação}
  \end{phonetics}
\end{entry}

\begin{entry}{兴奋}{6,8}{⼋、⼤}
  \begin{phonetics}{兴奋}{xing1fen4}[][HSK 4]
    \definition{adj.}{animado; excitante; empolgante;}
    \definition{s.}{excitação; empolgação}
    \definition{v.}{excitar; intoxicar}
  \end{phonetics}
\end{entry}

\begin{entry}{兴趣}{6,15}{⼋、⾛}
  \begin{phonetics}{兴趣}{xing4 qu4}[][HSK 4]
    \definition[个]{s.}{interesse (desejo de conhecer sobre alguma coisa ou coisa no qual está interessado) | \emph{hobby}}
  \end{phonetics}
\end{entry}

\begin{entry}{再}{6}{⼌}
  \begin{phonetics}{再}{zai4}[][HSK 1]
    \definition{adv.}{mais uma vez; além disso; ainda mais; indica a repetição ou continuação de uma mesma ação ou comportamento; refere-se principalmente a ações ou comportamentos não realizados ou contínuos | usado antes do adjetivo, indica intensificação, equivalente a 更 ou 更加 | (para uma ação adiada, precedida por uma expressão de tempo ou condição) então; somente então; depois de algo; indica que a ação ocorrerá após a conclusão de outra ação | além disso; indica um complemento, equivalente a 另外 ou 又 | próxima vez; indica que a ação ocorrerá após um determinado período de tempo | novamente; de novo}
  \seealsoref{更}{geng4}
  \seealsoref{更加}{geng4 jia1}
  \seealsoref{另外}{ling4wai4}
  \seealsoref{又}{you4}
  \end{phonetics}
\end{entry}

\begin{entry}{再三}{6,3}{⼌、⼀}
  \begin{phonetics}{再三}{zai4san1}[][HSK 4]
    \definition{adv.}{repetidamente; repetidas vezes; de novo e de novo}
  \end{phonetics}
\end{entry}

\begin{entry}{再也}{6,3}{⼌、⼄}
  \begin{phonetics}{再也}{zai4 ye3}[][HSK 5]
    \definition{adv.}{não mais; nunca mais; uma determinada situação ou ação nunca mais ocorrerá}
  \end{phonetics}
\end{entry}

\begin{entry}{再不}{6,4}{⼌、⼀}
  \begin{phonetics}{再不}{zai4bu4}
    \definition{adv.}{nunca mais}
  \end{phonetics}
\end{entry}

\begin{entry}{再见}{6,4}{⼌、⾒}
  \begin{phonetics}{再见}{zai4jian4}[][HSK 1]
    \definition{v.}{adeus; tchau; até logo; até mais; até mais tarde}
  \end{phonetics}
\end{entry}

\begin{entry}{再发}{6,5}{⼌、⼜}
  \begin{phonetics}{再发}{zai4fa1}
    \definition{v.}{reenviar}
  \end{phonetics}
\end{entry}

\begin{entry}{再生}{6,5}{⼌、⽣}
  \begin{phonetics}{再生}{zai4sheng1}
    \definition{s.}{reciclagem | regeneração}
    \definition{v.}{reciclar | renascer | regenerar}
  \end{phonetics}
\end{entry}

\begin{entry}{再次}{6,6}{⼌、⽋}
  \begin{phonetics}{再次}{zai4 ci4}[][HSK 5]
    \definition{adv.}{mais uma vez; uma segunda vez; outra vez}
  \end{phonetics}
\end{entry}

\begin{entry}{再审}{6,8}{⼌、⼧}
  \begin{phonetics}{再审}{zai4shen3}
    \definition{s.}{novo julgamento | revisão}
    \definition{v.}{ouvir um caso novamente}
  \end{phonetics}
\end{entry}

\begin{entry}{再者}{6,8}{⼌、⽼}
  \begin{phonetics}{再者}{zai4zhe3}
    \definition{conj.}{além do mais | além disso}
  \end{phonetics}
\end{entry}

\begin{entry}{再育}{6,8}{⼌、⾁}
  \begin{phonetics}{再育}{zai4yu4}
    \definition{v.}{aumentar | multiplicar | proliferar}
  \end{phonetics}
\end{entry}

\begin{entry}{再临}{6,9}{⼌、⼁}
  \begin{phonetics}{再临}{zai4lin2}
    \definition{v.}{vir de novo}
  \end{phonetics}
\end{entry}

\begin{entry}{再度}{6,9}{⼌、⼴}
  \begin{phonetics}{再度}{zai4du4}
    \definition{adv.}{outra vez | mais uma vez}
  \end{phonetics}
\end{entry}

\begin{entry}{再说}{6,9}{⼌、⾔}
  \begin{phonetics}{再说}{zai4shuo1}
    \definition{conj.}{além do mais | além disso | o que mais}
    \definition{v.}{adiar uma discussão para mais tarde | dizer novamente}
  \end{phonetics}
\end{entry}

\begin{entry}{再读}{6,10}{⼌、⾔}
  \begin{phonetics}{再读}{zai4du2}
    \definition{v.}{ler novamente | rever (uma lição, etc.)}
  \end{phonetics}
\end{entry}

\begin{entry}{军}{6}{⼍}
  \begin{phonetics}{军}{jun1}
    \definition*{s.}{Sobrenome Jun}
    \definition{s.}{forças armadas; exército; tropas | exército; contingente; muitas pessoas participando de uma atividade | exército; unidades militares}
  \end{phonetics}
\end{entry}

\begin{entry}{军人}{6,2}{⼍、⼈}
  \begin{phonetics}{军人}{jun1 ren2}[][HSK 5]
    \definition{s.}{soldado; militar; pessoal militar; pessoas com status militar; pessoas servindo nas forças armadas}
  \end{phonetics}
\end{entry}

\begin{entry}{军装}{6,12}{⼍、⾐}
  \begin{phonetics}{军装}{jun1zhuang1}
    \definition{s.}{uniforme militar}
  \end{phonetics}
\end{entry}

\begin{entry}{农}{6}{⼍}
  \begin{phonetics}{农}{nong2}
    \definition*{s.}{Sobrenome Nong}
    \definition{s.}{agricultura; criação de animais | camponês; fazendeiro}
  \end{phonetics}
\end{entry}

\begin{entry}{农业}{6,5}{⼍、⼀}
  \begin{phonetics}{农业}{nong2ye4}[][HSK 3]
    \definition{s.}{agricultura}
  \end{phonetics}
\end{entry}

\begin{entry}{农民}{6,5}{⼍、⽒}
  \begin{phonetics}{农民}{nong2min2}[][HSK 3]
    \definition[个,位,名,些]{s.}{fazendeiro; camponês; campesinato; trabalhadores que participam da produção agrícola há muito tempo}
  \end{phonetics}
\end{entry}

\begin{entry}{农产品}{6,6,9}{⼍、⼇、⼝}
  \begin{phonetics}{农产品}{nong2 chan3 pin3}[][HSK 5]
    \definition{s.}{produtos agrícolas}
  \end{phonetics}
\end{entry}

\begin{entry}{农村}{6,7}{⼍、⽊}
  \begin{phonetics}{农村}{nong2cun1}[][HSK 3]
    \definition{s.}{aldeia; campo; área rural; locais onde vivem os trabalhadores principalmente dedicados à produção agrícola}
  \end{phonetics}
\end{entry}

\begin{entry}{冰}{6}{⼎}
  \begin{phonetics}{冰}{bing1}[][HSK 4]
    \definition[块,层,些]{s.}{gelo; água em estado sólido |  algo parecido com gelo | (gíria) metanfetamina}
    \definition{v.}{colocar gelo; colocar gelo ao redor; colocar no gelo; resfriar objetos com gelo ou água fria | sentir frio}
  \end{phonetics}
\end{entry}

\begin{entry}{冰天雪地}{6,4,11,6}{⼎、⼤、⾬、⼟}
  \begin{phonetics}{冰天雪地}{bing1tian1-xue3di4}
    \definition{expr.}{um mundo de gelo e neve}
  \end{phonetics}
\end{entry}

\begin{entry}{冰球}{6,11}{⼎、⽟}
  \begin{phonetics}{冰球}{bing1qiu2}
    \definition[个]{s.}{hóquei no gelo | disco; a ``bola'' usada no hóquei no gelo}
  \end{phonetics}
\end{entry}

\begin{entry}{冰雪}{6,11}{⼎、⾬}
  \begin{phonetics}{冰雪}{bing1 xue3}[][HSK 4]
    \definition{adj.}{puro como gelo e neve; descreve uma pessoa como pura}
    \definition{s.}{gelo e neve}
  \end{phonetics}
\end{entry}

\begin{entry}{冰棍}{6,12}{⼎、⽊}
  \begin{phonetics}{冰棍}{bing1gun4}
    \definition[根]{s.}{picolé}
  \end{phonetics}
\end{entry}

\begin{entry}{冰箱}{6,15}{⼎、⾋}
  \begin{phonetics}{冰箱}{bing1xiang1}[][HSK 4]
    \definition[台,个]{s.}{geladeira; freezer; refrigerador; aparelhos para congelar alimentos ou medicamentos com gelo para mantê-los frios}
  \end{phonetics}
\end{entry}

\begin{entry}{冰激凌}{6,16,10}{⼎、⽔、⼎}
  \begin{phonetics}{冰激凌}{bing1ji1ling2}
    \definition{s.}{sorvete}
  \end{phonetics}
\end{entry}

\begin{entry}{冰糕}{6,16}{⼎、⽶}
  \begin{phonetics}{冰糕}{bing1gao1}
    \definition{s.}{sorvete | picolé}
  \end{phonetics}
\end{entry}

\begin{entry}{冲}{6}{⼎}
  \begin{phonetics}{冲}{chong1}[][HSK 4,6]
    \definition{s.}{via pública; local importante; via de passagem; via local importante | um trecho de planície em uma área montanhosa | (astronomia) oposição; os planetas externos orbitam até ficarem alinhados com a Terra e o Sol, e a Terra está no meio}
    \definition{v.}{atacar; apressar; correr; passar rapidamente; passar por um obstáculo | colidir; chocar; bater | despejar água fervente sobre | enxaguar; dar descarga; lavar | revelar (filme) | neutralizar a má sorte}
  \end{phonetics}
  \begin{phonetics}{冲}{chong4}
    \definition{adj.}{poderoso; com vigor; com muita força; vigoroso | forte; odor forte e pungente (olfato)}
    \definition{prep.}{de frente; em direção a | na força de; com base em; em virtude de}
    \definition{v.}{estampar (máquina de estamparia)}
  \end{phonetics}
\end{entry}

\begin{entry}{冲击}{6,5}{⼎、⼐}
  \begin{phonetics}{冲击}{chong1ji1}[][HSK 6]
    \definition{v.}{chicotear; bater | correr; voar; atacar; assaltar; atacar bravamente em direção a um alvo predeterminado | chocar; metáfora para interferência ou golpe sério}
  \end{phonetics}
\end{entry}

\begin{entry}{冲动}{6,6}{⼎、⼒}
  \begin{phonetics}{冲动}{chong1dong4}[][HSK 5]
    \definition{adj.}{impulsivo; impetuoso}
    \definition{s.}{impulso; impetuosidade; impulso de movimento; fenômeno psicológico no qual as emoções são particularmente fortes e o controle racional é fraco}
    \definition{v.}{ficar animado; ser impetuoso; agir por impulso}
  \end{phonetics}
\end{entry}

\begin{entry}{冲突}{6,9}{⼎、⽳}
  \begin{phonetics}{冲突}{chong1tu1}[][HSK 5]
    \definition{v.}{chocar-se; entrar em conflito; conflitar | contradizer; duas coisas opostas que interferem uma na outra}
  \end{phonetics}
\end{entry}

\begin{entry}{冲浪}{6,10}{⼎、⽔}
  \begin{phonetics}{冲浪}{chong1lang4}
    \definition{s.}{surfe}
    \definition{v.}{surfar}
  \end{phonetics}
\end{entry}

\begin{entry}{冲锋}{6,12}{⼎、⾦}
  \begin{phonetics}{冲锋}{chong1feng1}
    \definition{v.}{cobrar | tomar de assalto}
  \end{phonetics}
\end{entry}

\begin{entry}{决}{6}{⼎}
  \begin{phonetics}{决}{jue2}
    \definition{v.}{decidir; determinar | executar uma pessoa | (de um dique, etc.) romper; desabar}
  \end{phonetics}
\end{entry}

\begin{entry}{决不}{6,4}{⼎、⼀}
  \begin{phonetics}{决不}{jue2 bu4}[][HSK 5]
    \definition{adv.}{definitivamente não; certamente não; sob nenhuma circunstância; de forma alguma}
  \end{phonetics}
\end{entry}

\begin{entry}{决心}{6,4}{⼎、⼼}
  \begin{phonetics}{决心}{jue2xin1}[][HSK 3]
    \definition{s.}{resolução; determinação; determinação inabalável}
    \definition{v.}{secidir-se; decidir fazer algo e não vacilar nem mudar de ideia}
  \end{phonetics}
\end{entry}

\begin{entry}{决定}{6,8}{⼎、⼧}
  \begin{phonetics}{决定}{jue2ding4}[][HSK 3]
    \definition{adj.}{decisivo; as leis objetivas levam as coisas a se desenvolverem e mudarem em determinada direção}
    \definition[项,个]{s.}{decisão; resolução; assuntos decididos}
    \definition{v.}{decidir; determinar; algo se torna a base ou o pré-requisito para outra coisa; desempenha um papel dominante | decidir; resolver; tomar uma decisão; propor uma forma de agir}
  \end{phonetics}
\end{entry}

\begin{entry}{决赛}{6,14}{⼎、⾙}
  \begin{phonetics}{决赛}{jue2sai4}[][HSK 3]
    \definition[场]{s.}{finais (de uma competição); em competições esportivas, a última partida ou rodada disputada para determinar a classificação}
  \end{phonetics}
\end{entry}

\begin{entry}{划}{6}{⼑}
  \begin{phonetics}{划}{hua2}[][HSK 4]
    \definition{adj.}{rentável; vale (o esforço); compensa (fazer alguma coisa)}
    \definition{v.}{remar | ser vantajoso para alguém; ser uma pechincha | arranhar; cortar a superfície de; cortar em outra coisa com um objeto pontiagudo | arranhar; golpear;  esfregar uma coisa ou varrer sobre outra}
  \end{phonetics}
  \begin{phonetics}{划}{hua4}[][HSK 4]
    \definition*{s.}{Sobrenome Hua}
    \definition{s.}{traço de um caracter chinês}
    \definition{v.}{delimitar; diferenciar; delinear | transferir; ceder | planejar; programar | desenhar; marcar; delinear; fazer linhas ou escrever como marcadores com uma caneta ou objeto semelhante a uma caneta}
  \end{phonetics}
\end{entry}

\begin{entry}{划分}{6,4}{⼑、⼑}
  \begin{phonetics}{划分}{hua4fen1}[][HSK 5]
    \definition{v.}{dividir; particionar; reparticionar | diferenciar; encontrar aspectos diferentes}
  \end{phonetics}
\end{entry}

\begin{entry}{划船}{6,11}{⼑、⾈}
  \begin{phonetics}{划船}{hua2 chuan2}[][HSK 3]
    \definition[次,回]{s.}{remo (ato de remar); passeios de barco; a atividade ou esporte de “remar um barco com remos”}
    \definition{v.}{remar um barco; a ação ou comportamento de mover um barco na água usando remos}
  \end{phonetics}
\end{entry}

\begin{entry}{划艇}{6,12}{⼑、⾈}
  \begin{phonetics}{划艇}{hua2ting3}
    \definition{s.}{barco a remo}
  \end{phonetics}
\end{entry}

\begin{entry}{列}{6}{⼑}
  \begin{phonetics}{列}{lie4}[][HSK 4]
    \definition{v.}{organizar; formar uma linha; alinhar | listar; inserir em uma lista}
  \end{phonetics}
\end{entry}

\begin{entry}{列入}{6,2}{⼑、⼊}
  \begin{phonetics}{列入}{lie4 ru4}[][HSK 4]
    \definition{v.}{incluir em uma lista}
  \end{phonetics}
\end{entry}

\begin{entry}{列为}{6,4}{⼑、⼂}
  \begin{phonetics}{列为}{lie4 wei2}[][HSK 4]
    \definition{v.}{ser classificado como; ser listado como}
  \end{phonetics}
\end{entry}

\begin{entry}{列车}{6,4}{⼑、⾞}
  \begin{phonetics}{列车}{lie4che1}[][HSK 4]
    \definition{s.}{trem; trem em uma composição contínua, puxado por uma locomotiva e equipado com uma tripulação e marcações prescritas; geralmente um trem de passageiros}
  \end{phonetics}
\end{entry}

\begin{entry}{刘}{6}{⼑}
  \begin{phonetics}{刘}{liu2}
    \definition*{s.}{Sobrenome Liu}
    \definition{s.}{Clássico: um tipo de machado de batalha}
    \definition{v.}{matar}
  \end{phonetics}
\end{entry}

\begin{entry}{刚}{6}{⼑}
  \begin{phonetics}{刚}{gang1}[][HSK 2]
    \definition*{s.}{Sobrenome Gang}
    \definition{adj.}{duro; firme; rígido; forte; (personalidade, atitude) forte; (vontade) determinada}
    \definition{adv.}{apenas; exatamente; justamente | apenas; apenas por pouco; significa atingir um certo nível com dificuldade | apenas; há pouco tempo; indica que a ação ou situação ocorreu há pouco tempo | assim que; somente neste momento; aconteceu que; use a palavra 就 para indicar que duas coisas estão intimamente relacionadas}
  \seealsoref{就}{jiu4}
  \end{phonetics}
\end{entry}

\begin{entry}{刚才}{6,3}{⼑、⼿}
  \begin{phonetics}{刚才}{gang1cai2}[][HSK 2]
    \definition{s.}{agora mesmo; há pouco; há pouco tempo; referindo-se ao período recente que acabou de passar}
  \end{phonetics}
\end{entry}

\begin{entry}{刚刚}{6,6}{⼑、⼑}
  \begin{phonetics}{刚刚}{gang1 gang5}[][HSK 2]
    \definition{adv.}{apenas; somente; exatamente; refere-se a algo que é adequado em termos de grau, quantidade, tempo, etc., nem mais nem menos, nem cedo nem tarde, atingindo um estado satisfatório ou que atende exatamente às necessidades | agora mesmo; há pouco; há um momento atrás; referindo-se a um período de tempo muito curto no passado}
  \end{phonetics}
\end{entry}

\begin{entry}{创}{6}{⼑}
  \begin{phonetics}{创}{chuang1}
    \definition{s.}{ferimento; trauma}
  \end{phonetics}
  \begin{phonetics}{创}{chuang4}
    \definition{v.}{começar (fazer algo); alcançar (algo pela primeira vez); estabelecer; fazer pela primeira vez | estabelecer; fundar; criar; perceber algo novo, como um começo | ferir; machucar}
  \end{phonetics}
\end{entry}

\begin{entry}{创办}{6,4}{⼑、⼒}
  \begin{phonetics}{创办}{chuang4 ban4}[][HSK 6]
    \definition{v.}{estabelecer; montar; fundar}
  \end{phonetics}
\end{entry}

\begin{entry}{创业}{6,5}{⼑、⼀}
  \begin{phonetics}{创业}{chuang4ye4}[][HSK 3]
    \definition{s.}{empreendedorismo}
    \definition{v.}{começar um empreendimento; iniciar/fundar um negócio, uma empresa;}
  \end{phonetics}
\end{entry}

\begin{entry}{创立}{6,5}{⼑、⽴}
  \begin{phonetics}{创立}{chuang4li4}[][HSK 5]
    \definition{v.}{fundar; originar; estabelecer}
  \end{phonetics}
\end{entry}

\begin{entry}{创作}{6,7}{⼑、⼈}
  \begin{phonetics}{创作}{chuang4zuo4}[][HSK 3]
    \definition[个]{s.}{criação; trabalho criativo; obras literárias e artísticas}
    \definition{v.}{escrever; criar; produzir; compor; criar obras artísticas}
  \end{phonetics}
\end{entry}

\begin{entry}{创建}{6,8}{⼑、⼵}
  \begin{phonetics}{创建}{chuang4 jian4}[][HSK 6]
    \definition{v.}{fundar; estabelecer; montar}
  \end{phonetics}
\end{entry}

\begin{entry}{创造}{6,10}{⼑、⾡}
  \begin{phonetics}{创造}{chuang4zao4}[][HSK 3]
    \definition{s.}{criação; inovação; primeiro a concluir ou a alcançar resultados}
    \definition{v.}{criar; produzir; trazer à tona; fazer ou estabelecer pela primeira vez; referir-se de maneira geral a fazer ou estabelecer}
  \end{phonetics}
\end{entry}

\begin{entry}{创意}{6,13}{⼑、⼼}
  \begin{phonetics}{创意}{chuang4 yi4}[][HSK 6]
    \definition[个]{s.}{criatividade; originalidade; novidade; uma ideia original, conceito, etc.}
    \definition{v.}{inovar; criar um novo conceito, ideia, etc. | propor designs criativos, ideias, etc.}
  \end{phonetics}
\end{entry}

\begin{entry}{创新}{6,13}{⼑、⽄}
  \begin{phonetics}{创新}{chuang4xin1}[][HSK 3]
    \definition[个,种,次]{s.}{inovação; algo novo ou diferente, uma ideia}
    \definition{v.}{trazer novas ideias; inovar; abrir novos caminhos; criar ou fazer algo novo, diferente do que era antes}
  \end{phonetics}
\end{entry}

\begin{entry}{动}{6}{⼒}
  \begin{phonetics}{动}{dong4}[][HSK 1]
    \definition{adj.}{não estacionário; móvel; variável; mutável}
    \definition{adv.}{facilmente; frequentemente}
    \definition{s.}{ação; movimento}
    \definition{v.}{mover; mexer; (pessoas ou coisas) mudar a posição ou o estado original | agir; começar a agir; entrar em ação | alterar; mudar; alterar a posição ou o estado original | usar; utilizar; tornar ativo | despertar; tocar (o coração de alguém); provocar mudanças emocionais, reações | [geralmente na forma negativa] comer ou beber | emocionar; deixar emocionado}
  \end{phonetics}
\end{entry}

\begin{entry}{动人}{6,2}{⼒、⼈}
  \begin{phonetics}{动人}{dong4 ren2}[][HSK 3]
    \definition{adj.}{comovente; emocionante; tocante}
  \end{phonetics}
\end{entry}

\begin{entry}{动力}{6,2}{⼒、⼒}
  \begin{phonetics}{动力}{dong4li4}
    \definition[种,个]{s.}{poder; a força que faz com que as máquinas funcionem, por exemplo, energia elétrica, eólica, hidráulica, etc. | ímpeto; força motriz (ou propulsora); refere-se, de maneira geral, à força que impulsiona o desenvolvimento das coisas}
  \end{phonetics}
\end{entry}

\begin{entry}{动手}{6,4}{⼒、⼿}
  \begin{phonetics}{动手}{dong4shou3}[][HSK 5]
    \definition{v.+compl.}{iniciar o trabalho; começar a trabalhar | tocar; manusear; manipular | bater; levantar a mão (para bater); espancar}
  \end{phonetics}
\end{entry}

\begin{entry}{动机}{6,6}{⼒、⽊}
  \begin{phonetics}{动机}{dong4ji1}[][HSK 5]
    \definition[部]{s.}{motivo; razão; intenção; ideias que motivam as pessoas a se envolverem em determinados comportamentos}
  \end{phonetics}
\end{entry}

\begin{entry}{动作}{6,7}{⼒、⼈}
  \begin{phonetics}{动作}{dong4zuo4}[][HSK 1]
    \definition[个]{s.}{movimento; ação; atividade de todo o corpo ou parte do corpo}
    \definition{v.}{agir; começar a se mover; entrar em ação}
  \end{phonetics}
\end{entry}

\begin{entry}{动员}{6,7}{⼒、⼝}
  \begin{phonetics}{动员}{dong4yuan2}[][HSK 5]
    \definition{v.}{despertar; mobilizar; iniciar (para fazer algo ou participar de uma atividade) | mobilizar toda a nação; transferir dos setores militar, político e econômico para uma situação de guerra}
  \end{phonetics}
\end{entry}

\begin{entry}{动身}{6,7}{⼒、⾝}
  \begin{phonetics}{动身}{dong4shen1}
    \definition{v.+compl.}{fazer uma jornada | começar uma jornada | partir | partir em uma jornada | sair (para um lugar distante)}
  \end{phonetics}
\end{entry}

\begin{entry}{动态}{6,8}{⼒、⼼}
  \begin{phonetics}{动态}{dong4tai4}[][HSK 5]
    \definition{s.}{tendências; desenvolvimentos; tendência geral dos assuntos; causa provável de ação; curso dos acontecimentos | expressão; comportamento ativo | estado dinâmico; condição dinâmica; de ou em relação a um estado de movimento}
  \end{phonetics}
\end{entry}

\begin{entry}{动物}{6,8}{⼒、⽜}
  \begin{phonetics}{动物}{dong4wu4}[][HSK 2]
    \definition[个,只,群,种]{s.}{animal; uma grande classe de seres vivos, que se alimentam principalmente de matéria orgânica, possuem sistema nervoso, são sensíveis e capazes de se mover; refere-se a todos os tipos de coisas concretas ou abstratas}
  \end{phonetics}
\end{entry}

\begin{entry}{动物园}{6,8,7}{⼒、⽜、⼞}
  \begin{phonetics}{动物园}{dong4 wu4 yuan2}[][HSK 2]
    \definition[个,座,家]{s.}{jardim zoológico; zoo; parque que cria muitos tipos de animais (especialmente animais com valor científico ou raros na região) para exibição ao público}
  \end{phonetics}
\end{entry}

\begin{entry}{动画片}{6,8,4}{⼒、⽥、⽚}
  \begin{phonetics}{动画片}{dong4hua4pian4}[][HSK 4]
    \definition[部]{s.}{desenho animado; animações; filme de animação}
  \end{phonetics}
\end{entry}

\begin{entry}{动感}{6,13}{⼒、⼼}
  \begin{phonetics}{动感}{dong4gan3}
    \definition{adj.}{dinâmica | vívida}
    \definition{adv.}{dinamicamente}
    \definition{s.}{senso de movimento (geralmente em uma obra de arte estática)}
  \end{phonetics}
\end{entry}

\begin{entry}{动摇}{6,13}{⼒、⼿}
  \begin{phonetics}{动摇}{dong4 yao2}[][HSK 4]
    \definition{adj.}{instável}
    \definition{v.}{ondular; pairar; agitar; balançar; sacudir | hesitar; vacilar; esmorecer; abalar}
  \end{phonetics}
\end{entry}

\begin{entry}{动漫}{6,14}{⼒、⽔}
  \begin{phonetics}{动漫}{dong4man4}
    \definition{s.}{desenhos animados | quadrinhos | anime | mangá}
  \end{phonetics}
\end{entry}

\begin{entry}{匈}{6}{⼓}
  \begin{phonetics}{匈}{xiong1}
    \definition*{s.}{Hungria, abreviação de 匈牙利}
    \definition{s.}{peito; seio; tórax}
  \seealsoref{匈牙利}{xiong1ya2li4}
  \end{phonetics}
\end{entry}

\begin{entry}{匈牙利}{6,4,7}{⼓、⽛、⼑}
  \begin{phonetics}{匈牙利}{xiong1ya2li4}
    \definition*{s.}{Hungria}
  \end{phonetics}
\end{entry}

\begin{entry}{匈奴}{6,5}{⼓、⼥}
  \begin{phonetics}{匈奴}{xiong1nu2}
    \definition*{s.}{Xiongnu, um povo da estepe oriental que criou um império que floresceu na época das dinastias Qin e Han}
  \end{phonetics}
\end{entry}

\begin{entry}{匠}{6}{⼕}
  \begin{phonetics}{匠}{jiang4}
    \definition{s.}{artesão}
  \end{phonetics}
\end{entry}

\begin{entry}{华}{6}{⼗}
  \begin{phonetics}{华}{hua2}
    \definition*{s.}{China; refere-se à China (anteriormente conhecida como Huaxia, 华夏, mais tarde chamada de Zhonghua, 中华, ou simplesmente Hua, 华)}
    \definition{adj.}{esplêndido; magnífico | próspero; florescente | chamativo; extravagante; vaidoso | grisalho}
    \definition{s.}{corona; um halo colorido ao redor do sol ou da lua causado pela difração da luz através das nuvens | creme; melhor parte; a melhor parte das coisas | chinês; refere-se à nacionalidade Han (língua e escrita) | vezes; anos; refere-se a (bons) momentos | elixir; essência líquida; substâncias formadas pela sedimentação de minerais na água de nascente | Seu, palavra honorífica, usada para se referir a coisas relacionadas à outra pessoa}
  \seealsoref{华夏}{hua2xia4}
  \seealsoref{中华}{zhong1hua2}
  \end{phonetics}
  \begin{phonetics}{华}{hua4}
    \definition*{s.}{Huashan Mountain (na província de Shaanxi) | Sobrenome Hua}
  \end{phonetics}
\end{entry}

\begin{entry}{华人}{6,2}{⼗、⼈}
  \begin{phonetics}{华人}{hua2 ren2}[][HSK 3]
    \definition[名,位,个]{s.}{Chinês; chinês étnico | chineses no exterior; refere-se a cidadãos estrangeiros de ascendência chinesa que obtiveram a nacionalidade do país em que residem}
  \end{phonetics}
\end{entry}

\begin{entry}{华氏}{6,4}{⼗、⽒}
  \begin{phonetics}{华氏}{hua2shi4}
    \definition{s.}{graus Fahrenheit (°F)}
  \end{phonetics}
\end{entry}

\begin{entry}{华语}{6,9}{⼗、⾔}
  \begin{phonetics}{华语}{hua2 yu3}[][HSK 5]
    \definition*{s.}{Chinês (idioma)}
  \end{phonetics}
\end{entry}

\begin{entry}{华夏}{6,10}{⼗、⼢}
  \begin{phonetics}{华夏}{hua2xia4}
    \definition*{s.}{Huaxia, nome antigo da China | Catai}
  \end{phonetics}
\end{entry}

\begin{entry}{华盛顿}{6,11,10}{⼗、⽫、⾴}
  \begin{phonetics}{华盛顿}{hua2sheng4dun4}
    \definition*{s.}{Washington}
  \end{phonetics}
\end{entry}

\begin{entry}{华裔}{6,13}{⼗、⾐}
  \begin{phonetics}{华裔}{hua2yi4}
    \definition{s.}{descendente de chinês}
  \end{phonetics}
\end{entry}

\begin{entry}{协}{6}{⼗}
  \begin{phonetics}{协}{xie2}
    \definition*{s.}{Sobrenome Xie}
    \definition{adv.}{conjuntamente; coordenadamente; juntos}
    \definition{s.}{harmonioso}
    \definition{v.}{auxiliar; assistir; ajudar}
  \end{phonetics}
\end{entry}

\begin{entry}{协议}{6,5}{⼗、⾔}
  \begin{phonetics}{协议}{xie2yi4}[][HSK 5]
    \definition[项]{s.}{acordo; tratado; decisão conjunta alcançada através de negociação e consulta}
    \definition{v.}{concordar em}
  \end{phonetics}
\end{entry}

\begin{entry}{协议书}{6,5,4}{⼗、⾔、⼄}
  \begin{phonetics}{协议书}{xie2 yi4 shu1}[][HSK 5]
    \definition{s.}{contrato | protocolo}
  \end{phonetics}
\end{entry}

\begin{entry}{危}{6}{⼙}
  \begin{phonetics}{危}{wei1}
    \definition*{s.}{Wei, a décima segunda das vinte e oito constelações em que a esfera celeste foi dividida, consistindo de três estrelas em forma de triângulo obtuso, uma em Aquário e duas em Pégaso | Wei, uma das mansões lunares | Sobrenome Wei}
    \definition{adj.}{arriscado; inseguro; perigoso (oposto a 安) | estar gravemente doente; estar morrendo | alto; íngreme}
    \definition{s.}{perigo | cumeeira (de um telhado)}
    \definition{v.}{pôr em perigo; colocar em perigo; comprometer}
  \seealsoref{安}{an1}
  \end{phonetics}
\end{entry}

\begin{entry}{危急}{6,9}{⼙、⼼}
  \begin{phonetics}{危急}{wei1ji2}
    \definition{adj.}{crítico | desesperadora (situação)}
  \end{phonetics}
\end{entry}

\begin{entry}{危险}{6,9}{⼙、⾩}
  \begin{phonetics}{危险}{wei1xian3}[][HSK 3]
    \definition{adj.}{arriscado; perigoso}
  \end{phonetics}
\end{entry}

\begin{entry}{危害}{6,10}{⼙、⼧}
  \begin{phonetics}{危害}{wei1hai4}[][HSK 3]
    \definition[个,种]{s.}{prejuízo; perigo; dano}
    \definition{v.}{destruir; prejudicar; pôr em perigo; pôr em risco}
  \end{phonetics}
\end{entry}

\begin{entry}{危难}{6,10}{⼙、⾫}
  \begin{phonetics}{危难}{wei1nan4}
    \definition{s.}{calamidade}
  \end{phonetics}
\end{entry}

\begin{entry}{压}{6}{⼚}
  \begin{phonetics}{压}{ya1}[][HSK 3]
    \definition{v.}{pressionar; empurrar para baixo; segurar; pesar | acalmar emoções agitadas ou situações ruins; tranquilizar | intimidar; reprimir; exercer pressão sobre; usar poder, posição ou padrões morais para coagir ou restringir as pessoas, impedindo-as de se expressar, decidir ou se desenvolver livremente | aproximar-se; estar chegando perto | arquivar; deixar de lado | pressionar; metáfora para uma grande carga emocional e psicológica | superar; ultrapassar; voz, capacidade e presença mais fortes do que os outros | apostar em um determinado resultado ao jogar | pressionar; força na superfície de contato do objeto}
  \end{phonetics}
  \begin{phonetics}{压}{ya4}
    \definition{adv.}{fundamentalmente; nunca (usado principalmente em frases negativas)}
  \seealsoref{压根儿}{ya4 gen1r5}
  \end{phonetics}
\end{entry}

\begin{entry}{压力}{6,2}{⼚、⼒}
  \begin{phonetics}{压力}{ya1li4}[][HSK 3]
    \definition[份,个]{s.}{pressão; força atuando perpendicularmente à superfície de um objeto | pressão; força esmagadora; metáfora para a força que coage e intimida as pessoas (principalmente nos aspectos espirituais e psicológicos) | tensão; fardo; os encargos econômicos, psicológicos e espirituais impostos pelo mundo exterior}
  \end{phonetics}
\end{entry}

\begin{entry}{压岁钱}{6,6,10}{⼚、⼭、⾦}
  \begin{phonetics}{压岁钱}{ya1sui4qian2}
    \definition{s.}{dinheiro da sorte | dinheiro dado às crianças como presente no Ano Novo Chinês}
  \end{phonetics}
\end{entry}

\begin{entry}{压根儿}{6,10,2}{⼚、⽊、⼉}
  \begin{phonetics}{压根儿}{ya4 gen1r5}
    \definition{adv.}{fundamentalmente; nunca (usado principalmente em frases negativas)}
  \end{phonetics}
\end{entry}

\begin{entry}{压碎}{6,13}{⼚、⽯}
  \begin{phonetics}{压碎}{ya1sui4}
    \definition{v.}{esmagar em pedaços}
  \end{phonetics}
\end{entry}

\begin{entry}{压韵}{6,13}{⼚、⾳}
  \begin{phonetics}{压韵}{ya1yun4}
    \variantof{押韵}
  \end{phonetics}
\end{entry}

\begin{entry}{吃}{6}{⼝}
  \begin{phonetics}{吃}{chi1}[][HSK 1]
    \definition{s.}{alimentos; necessidades básicas}
    \definition{v.}{comer; pegar; fazer; colocar alimentos na boca, mastigar e engolir (incluindo sugar e beber) | viver; depender de algo para viver | aniquilar; eliminar (usado principalmente em jogos de guerra e jogos de tabuleiro) | esgotar; exaurir; ser um fardo; ser um esforço | absorver | sofrer; incorrer | entender; compreender | entrar um objeto em outro | expressar aceitação psicológica | fazer suas refeições; comer}
  \end{phonetics}
\end{entry}

\begin{entry}{吃力}{6,2}{⼝、⼒}
  \begin{phonetics}{吃力}{chi1li4}[][HSK 5]
    \definition{adj.}{suado; extenuante; trabalhoso; laborioso | cansado; fatigado}
  \end{phonetics}
\end{entry}

\begin{entry}{吃饭}{6,7}{⼝、⾷}
  \begin{phonetics}{吃饭}{chi1 fan4}[][HSK 1]
    \definition{v.+compl.}{comer; ter (comer) uma refeição | manter-se vivo;  ganhar a vida; refere-se à vida ou à sobrevivência em geral}
  \end{phonetics}
\end{entry}

\begin{entry}{吃屎}{6,9}{⼝、⼫}
  \begin{phonetics}{吃屎}{chi1 shi3}
    \definition{expr.}{Coma merda!}
  \end{phonetics}
\end{entry}

\begin{entry}{吃惊}{6,11}{⼝、⼼}
  \begin{phonetics}{吃惊}{chi1jing1}[][HSK 4]
    \definition{v.+compl.}{ficar assustado; ficar chocado; ficar espantado; pegar de surpresa; ficar assustado inesperadamente}
  \end{phonetics}
\end{entry}

\begin{entry}{各}{6}{⼝}
  \begin{phonetics}{各}{ge4}[][HSK 3]
    \definition{adv.}{de várias maneiras; de diversas formas; respectivamente; indica que algo é feito separadamente ou que possui uma determinada característica separadamente}
    \definition{pron.}{todo; todos; cada; refere-se a todos os indivíduos dentro de um determinado intervalo, equivalente a 每个}
  \seealsoref{每个}{mei3ge4}
  \end{phonetics}
\end{entry}

\begin{entry}{各个}{6,3}{⼝、⼈}
  \begin{phonetics}{各个}{ge4 ge4}[][HSK 4]
    \definition{adv./pron.}{cada | um a um; um após o outro}
  \end{phonetics}
\end{entry}

\begin{entry}{各地}{6,6}{⼝、⼟}
  \begin{phonetics}{各地}{ge4 di4}[][HSK 3]
    \definition{s.}{em todos os lugares; em vários locais}
  \end{phonetics}
\end{entry}

\begin{entry}{各自}{6,6}{⼝、⾃}
  \begin{phonetics}{各自}{ge4zi4}[][HSK 3]
    \definition{pron.}{por si mesmo; por conta própria; cada um por si | cada um; indica cada uma das partes envolvidas}
  \end{phonetics}
\end{entry}

\begin{entry}{各位}{6,7}{⼝、⼈}
  \begin{phonetics}{各位}{ge4 wei4}[][HSK 3]
    \definition{pron.}{todos; toda a gente; todo mundo | cada um}
  \end{phonetics}
\end{entry}

\begin{entry}{各种}{6,9}{⼝、⽲}
  \begin{phonetics}{各种}{ge4 zhong3}[][HSK 3]
    \definition{adv.}{todos os tipos; vários tipos}
  \end{phonetics}
\end{entry}

\begin{entry}{合}{6}{⼝}
  \begin{phonetics}{合}{he2}[][HSK 3]
    \definition{adj.}{todo; completo; inteiro}
    \definition{clas.}{usado para rodadas | 100ml | medida para grãos secos igual a um décimo de 升, ou um centésimo de 斗}
    \definition{s.}{casamento; união matrimonial | (astronomia) conjunção | nota da escala em Gongchepu (工尺谱), correspondente ao 5 na notação musical numerada}
    \definition{v.}{fechar | juntar; combinar (oposto de 分) | adequar-se; concordar; conformar-se a | ser igual a; somar | ser adequado}
  \seealsoref{斗}{dou4}
  \seealsoref{分}{fen1}
  \seealsoref{工尺谱}{gong1 che3 pu3}
  \seealsoref{升}{sheng1}
  \end{phonetics}
\end{entry}

\begin{entry}{合同}{6,6}{⼝、⼝}
  \begin{phonetics}{合同}{he2tong5}[][HSK 4]
    \definition[个,份]{s.}{contrato; acordo; uma disposição para observância mútua por duas ou mais partes na condução de um assunto com o objetivo de determinar seus respectivos direitos e obrigações.}
  \end{phonetics}
\end{entry}

\begin{entry}{合并}{6,6}{⼝、⼲}
  \begin{phonetics}{合并}{he2bing4}[][HSK 5]
    \definition{v.}{fundir; amalgamar; combinar várias coisas em uma coisa só | (doença) ser complicada por outra doença; uma doença levar a outra, ataques simultâneos (de várias doenças)}
  \end{phonetics}
\end{entry}

\begin{entry}{合成}{6,6}{⼝、⼽}
  \begin{phonetics}{合成}{he2cheng2}[][HSK 5]
    \definition{s.}{compor; integrar; combinar; misturar | sintetizar, reação química para transformar uma substância com uma composição simples em uma substância com uma composição complexa}
  \end{phonetics}
\end{entry}

\begin{entry}{合作}{6,7}{⼝、⼈}
  \begin{phonetics}{合作}{he2zuo4}[][HSK 3]
    \definition{v.}{cooperar; colaborar; trabalhar em conjunto; trabalhar em conjunto para realizar algo ou concluir uma tarefa}
  \end{phonetics}
\end{entry}

\begin{entry}{合法}{6,8}{⼝、⽔}
  \begin{phonetics}{合法}{he2fa3}[][HSK 3]
    \definition{adj.}{legal; legítimo; lícito;  justo; válido; em conformidade com as disposições legais}
  \end{phonetics}
\end{entry}

\begin{entry}{合宪性}{6,9,8}{⼝、⼧、⼼}
  \begin{phonetics}{合宪性}{he2xian4xing4}
    \definition{s.}{constitucionalismo}
  \end{phonetics}
\end{entry}

\begin{entry}{合适}{6,9}{⼝、⾡}
  \begin{phonetics}{合适}{he2shi4}[][HSK 2]
    \definition{adj.}{correto; adequado; apropriado; conveniente; em conformidade com a realidade ou com os requisitos objetivos}
  \end{phonetics}
\end{entry}

\begin{entry}{合格}{6,10}{⼝、⽊}
  \begin{phonetics}{合格}{he2ge2}[][HSK 3]
    \definition{adj.}{qualificado; dentro dos padrões; em conformidade com os requisitos ou normas}
  \end{phonetics}
\end{entry}

\begin{entry}{合资}{6,10}{⼝、⾙}
  \begin{phonetics}{合资}{he2zi1}
    \definition{s.}{\emph{joint-venture} com capitais mistos}
  \end{phonetics}
\end{entry}

\begin{entry}{合理}{6,11}{⼝、⽟}
  \begin{phonetics}{合理}{he2li3}[][HSK 3]
    \definition{adj.}{racional; razoável; equitativo; razoável ou lógico}
  \end{phonetics}
\end{entry}

\begin{entry}{吉}{6}{⼝}
  \begin{phonetics}{吉}{ji2}
    \definition*{s.}{Província de Jilin, abreviação de 吉林 | Sobrenome Ji}
    \definition{adj.}{sortudo; propício; auspicioso (oposto de 凶)}
  \seealsoref{吉林}{ji2lin2}
  \seealsoref{凶}{xiong1}
  \end{phonetics}
\end{entry}

\begin{entry}{吉他}{6,5}{⼝、⼈}
  \begin{phonetics}{吉他}{ji2ta1}
    \definition[把]{s.}{Empréstimo linguístico: guitarra}
  \end{phonetics}
\end{entry}

\begin{entry}{吉林}{6,8}{⼝、⽊}
  \begin{phonetics}{吉林}{ji2lin2}
    \definition*{s.}{Província de Jilin}
  \end{phonetics}
\end{entry}

\begin{entry}{吊}{6}{⼝}
  \begin{phonetics}{吊}{diao4}[][HSK 6]
    \definition{clas.}{uma sequência de 1.000 em dinheiro; antigamente, uma unidade monetária geralmente era composta por mil pequenas moedas de cobre}
    \definition{s.}{guindaste}
    \definition{v.}{pendurar; suspender | levantar ou abaixar com uma corda, etc. | colocar um forro de pele; adicionar revestimentos ou forros aos barris de couro para fazer roupas | revogar; retirar; recuperar documentos emitidos | lamentar; prestar homenagem os mortos ou oferecer condolências às famílias ou grupos que sofreram uma perda}
  \end{phonetics}
\end{entry}

\begin{entry}{同}{6}{⼝}
  \begin{phonetics}{同}{tong2}[][HSK 6]
    \definition{adj.}{como; igual; parecido; similar; o mesmo; sem diferença}
    \definition{adv.}{juntos; em comum; indica que diferentes atores realizam uma determinada ação juntos ou estão na mesma situação, o que equivale a 一同 ou 一起}
    \definition{v.}{ser o mesmo que}
  \seealsoref{一起}{yi4qi3}
  \seealsoref{一同}{yi4tong2}
  \end{phonetics}
  \begin{phonetics}{同}{tong4}
    \definition[条,处]{s.}{beco; rua estreita}
  \seealsoref{胡同}{hu2tong5}
  \end{phonetics}
\end{entry}

\begin{entry}{同伙}{6,6}{⼝、⼈}
  \begin{phonetics}{同伙}{tong2huo3}
    \definition[个]{s.}{cúmplice | colega}
  \end{phonetics}
\end{entry}

\begin{entry}{同时}{6,7}{⼝、⽇}
  \begin{phonetics}{同时}{tong2shi2}[][HSK 2]
    \definition{conj.}{além disso; além do mais; ainda mais; indica uma relação de equivalência, geralmente com um significado mais profundo}
    \definition{s.}{enquanto isso; ao mesmo tempo}
  \end{phonetics}
\end{entry}

\begin{entry}{同事}{6,8}{⼝、⼅}
  \begin{phonetics}{同事}{tong2shi4}[][HSK 2]
    \definition[个,位,名]{s.}{companheiro; colega; colega de trabalho; pessoas que trabalham juntas}
    \definition{v.}{trabalhar no mesmo lugar; trabalhar juntos; trabalhar na mesma unidade}
  \end{phonetics}
\end{entry}

\begin{entry}{同学}{6,8}{⼝、⼦}
  \begin{phonetics}{同学}{tong2xue2}[][HSK 1]
    \definition[位,个,些]{s.}{colega de escola; colega de turma; colega de estudos; pessoas que estudam na mesma escola}
  \end{phonetics}
\end{entry}

\begin{entry}{同性恋}{6,8,10}{⼝、⼼、⼼}
  \begin{phonetics}{同性恋}{tong2xing4lian4}
    \definition{s.}{homossexualidade | pessoa gay | amor gay}
  \end{phonetics}
\end{entry}

\begin{entry}{同屋}{6,9}{⼝、⼫}
  \begin{phonetics}{同屋}{tong2wu1}
    \definition[个]{s.}{companheiro de quarto | colega de quarto}
  \end{phonetics}
\end{entry}

\begin{entry}{同砚}{6,9}{⼝、⽯}
  \begin{phonetics}{同砚}{tong2yan4}
    \definition[位,个]{s.}{colega de classe | colega estudante}
  \end{phonetics}
\end{entry}

\begin{entry}{同样}{6,10}{⼝、⽊}
  \begin{phonetics}{同样}{tong2 yang4}[][HSK 2]
    \definition{adj.}{igual; semelhante; similar; idêntico; sem diferença}
  \end{phonetics}
\end{entry}

\begin{entry}{同流合污}{6,10,6,6}{⼝、⽔、⼝、⽔}
  \begin{phonetics}{同流合污}{tong2liu2he2wu1}
    \definition{expr.}{chafurdar na lama com alguém | seguir o mau exemplo dos outros}
  \end{phonetics}
\end{entry}

\begin{entry}{同情}{6,11}{⼝、⼼}
  \begin{phonetics}{同情}{tong2qing2}[][HSK 4]
    \definition{s.}{simpatia}
    \definition{v.}{simpatizar com; solidarizar-se; compadecer-se; ter empatia emocional pelo que os outros estão passando}
  \end{phonetics}
\end{entry}

\begin{entry}{同意}{6,13}{⼝、⼼}
  \begin{phonetics}{同意}{tong2yi4}[][HSK 3]
    \definition{v.}{concordar; consentir; aprovar; concordar com; dizer sim}
  \end{phonetics}
\end{entry}

\begin{entry}{名}{6}{⼝}
  \begin{phonetics}{名}{ming2}[][HSK 2]
    \definition*{s.}{Sobrenome Ming}
    \definition{adj.}{notável; famoso; conhecido; renomado}
    \definition{clas.}{usado para pessoas | usado para classificação por ordem}
    \definition{s.}{nome; denominação | desculpa; pretexto | fama; reputação}
    \definition{v.}{nome próprio (é) | expressar; descrever | possuir; tomar; ter}
  \end{phonetics}
\end{entry}

\begin{entry}{名人}{6,2}{⼝、⼈}
  \begin{phonetics}{名人}{ming2 ren2}[][HSK 4]
    \definition{s.}{celebridade; pessoa famosa}
  \end{phonetics}
\end{entry}

\begin{entry}{名片}{6,4}{⼝、⽚}
  \begin{phonetics}{名片}{ming2pian4}[][HSK 4]
    \definition[张,盒,叠]{s.}{cartão de visita; um pedaço de papel retangular com o nome, o cargo, o endereço etc. impressos}
  \end{phonetics}
\end{entry}

\begin{entry}{名字}{6,6}{⼝、⼦}
  \begin{phonetics}{名字}{ming2zi5}[][HSK 1]
    \definition[个]{s.}{nome; nome próprio | nome (de uma coisa)}
  \end{phonetics}
\end{entry}

\begin{entry}{名单}{6,8}{⼝、⼗}
  \begin{phonetics}{名单}{ming2 dan1}[][HSK 2]
    \definition[个,份]{s.}{lista com nomes de pessoas ou nomes de organizações}
  \end{phonetics}
\end{entry}

\begin{entry}{名称}{6,10}{⼝、⽲}
  \begin{phonetics}{名称}{ming2 cheng1}[][HSK 2]
    \definition[个,种]{s.}{nomes, apelidos e formas de se referir a pessoas ou coisas}
  \end{phonetics}
\end{entry}

\begin{entry}{名牌儿}{6,12,2}{⼝、⽚、⼉}
  \begin{phonetics}{名牌儿}{ming2 pai2r5}[][HSK 4]
    \definition{s.}{marca famosa}
  \end{phonetics}
\end{entry}

\begin{entry}{后}{6}{⼝}
  \begin{phonetics}{后}{hou4}[][HSK 1]
    \definition*{s.}{Sobrenome Hou}
    \definition{s.}{atrás; traseiro; a direção oposta àquela para a qual a pessoa está voltada; a direção oposta àquela para a qual a parte de trás de uma casa está voltada (o oposto de 前)  | depois; mais tarde no tempo; futuro (em oposição a 先 ou 前) | último | posteridade; descendência | rainha; imperatriz | governante; soberano; monarca antigo}
  \seealsoref{前}{qian2}
  \seealsoref{先}{xian1}
  \end{phonetics}
\end{entry}

\begin{entry}{后天}{6,4}{⼝、⼤}
  \begin{phonetics}{后天}{hou4 tian1}[][HSK 1]
    \definition{s.}{depois de amanhã; período em que uma pessoa ou animal vive e cresce sozinho após deixar o útero materno (em oposição a 先天)}
  \seealsoref{先天}{xian1tian1}
  \end{phonetics}
\end{entry}

\begin{entry}{后头}{6,5}{⼝、⼤}
  \begin{phonetics}{后头}{hou4 tou5}[][HSK 4]
    \definition{adv.}{posteriormente | atrás | mais tarde}
    \definition{s.}{a parte de trás | a parte traseira}
  \end{phonetics}
\end{entry}

\begin{entry}{后边}{6,5}{⼝、⾡}
  \begin{phonetics}{后边}{hou4 bian5}[][HSK 1]
    \definition{adv.}{costas; traseira; atrás}
  \end{phonetics}
\end{entry}

\begin{entry}{后年}{6,6}{⼝、⼲}
  \begin{phonetics}{后年}{hou4nian2}[][HSK 3]
    \definition{s.}{daqui a dois anos; no ano seguinte ao próximo ano}
  \end{phonetics}
\end{entry}

\begin{entry}{后来}{6,7}{⼝、⽊}
  \begin{phonetics}{后来}{hou4lai2}[][HSK 2]
    \definition{adv.}{mais tarde; depois; refere-se a um período posterior a um determinado momento no passado}
  \end{phonetics}
\end{entry}

\begin{entry}{后果}{6,8}{⼝、⽊}
  \begin{phonetics}{后果}{hou4guo3}[][HSK 3]
    \definition{s.}{consequência; resultado (geralmente negativo)}
  \end{phonetics}
\end{entry}

\begin{entry}{后面}{6,9}{⼝、⾯}
  \begin{phonetics}{后面}{hou4mian4}
    \definition{adv.}{parte de trás; retaguarda; atrás; a parte posterior do espaço ou localização | mais tarde; depois; no futuro; a parte posterior de um artigo ou discurso em relação ao que está sendo narrado no momento}
  \end{phonetics}
\end{entry}

\begin{entry}{后悔}{6,10}{⼝、⼼}
  \begin{phonetics}{后悔}{hou4hui3}[][HSK 5]
    \definition{v.}{lamentar; ter remorso; arrepender-se}
  \end{phonetics}
\end{entry}

\begin{entry}{吐}{6}{⼝}
  \begin{phonetics}{吐}{tu3}[][HSK 5]
    \definition{v.}{cuspir; sair pela boca | surgir ou aparecer pela boca ou por uma fenda | dizer; contar; falar abertamente}
  \end{phonetics}
  \begin{phonetics}{吐}{tu4}[][HSK 5]
    \definition{v.}{vomitar; sair pela boca | vomitar; expelir; metáfora para ser forçado a devolver bens usurpados}
  \end{phonetics}
\end{entry}

\begin{entry}{向}{6}{⼝}
  \begin{phonetics}{向}{xiang4}[][HSK 2]
    \definition*{s.}{Sobrenome Xiang}
    \definition{adv.}{sempre; o tempo todo}
    \definition{prep.}{em direção a; para}
    \definition{s.}{direção | a janela voltada para o norte}
    \definition{v.}{encarar; virar-se para | estar do lado de; ser parcial com; tomar o partido de alguém}
  \end{phonetics}
\end{entry}

\begin{entry}{向上}{6,3}{⼝、⼀}
  \begin{phonetics}{向上}{xiang4 shang4}[][HSK 5]
    \definition{v.}{mover-se; subir; ir para um lugar mais alto; ir para um lugar mais alto em relação a um determinado ponto; ir para um desenvolvimento mais alto que o atual | avançar; continuar se aperfeiçoar; subir na vida; desenvolver-se em direção ao progresso}
  \end{phonetics}
\end{entry}

\begin{entry}{向导}{6,6}{⼝、⼨}
  \begin{phonetics}{向导}{xiang4dao3}[][HSK 5]
    \definition{s.}{guia}
    \definition{v.}{agir como um guia; mostrar a alguém o caminho; levar alguém a algum lugar}
  \end{phonetics}
\end{entry}

\begin{entry}{向汪}{6,7}{⼝、⽔}
  \begin{phonetics}{向汪}{xiang4wang1}
    \definition{v.}{esperar que}
  \end{phonetics}
\end{entry}

\begin{entry}{向往}{6,8}{⼝、⼻}
  \begin{phonetics}{向往}{xiang4wang3}
    \definition{v.}{ansiar por | esperar ansiosamente por}
  \end{phonetics}
\end{entry}

\begin{entry}{向前}{6,9}{⼝、⼑}
  \begin{phonetics}{向前}{xiang4 qian2}[][HSK 5]
    \definition{adv.}{para frente; adiante;}
    \definition{v.}{avançar; ir em direção à frente; mover-se para frente; avançar um pouco mais}
  \end{phonetics}
\end{entry}

\begin{entry}{吓}{6}{⼝}
  \begin{phonetics}{吓}{xia4}[][HSK 5]
    \definition{interj.}{interjeição que demonstra espanto; Interjeição que expressa insatisfação}
    \definition{v.}{ameaçar; intimidar; usar ameaças ou meios coercitivos para intimidar ou assustar}
  \end{phonetics}
\end{entry}

\begin{entry}{吓人}{6,2}{⼝、⼈}
  \begin{phonetics}{吓人}{xia4ren2}
    \definition{adj.}{apavorante | assustador}
    \definition{v.+compl.}{assustar-se | tomar um susto}
  \end{phonetics}
\end{entry}

\begin{entry}{吗}{6}{⼝}
  \begin{phonetics}{吗}{ma2}
    \definition{adv.}{(coloquial) que?}
  \end{phonetics}
  \begin{phonetics}{吗}{ma3}
    \definition{s.}{usada em 吗啡, morfina}
  \seealsoref{吗啡}{ma3fei1}
  \end{phonetics}
  \begin{phonetics}{吗}{ma5}[][HSK 1]
    \definition{part.}{usado no final de uma pergunta | como uma pausa em uma frase antes de introduzir o ponto principal | usado no final de uma pergunta retórica}
  \end{phonetics}
\end{entry}

\begin{entry}{吗啡}{6,11}{⼝、⼝}
  \begin{phonetics}{吗啡}{ma3fei1}
    \definition{s.}{morfina (empréstimo linguístico)}
  \end{phonetics}
\end{entry}

\begin{entry}{吸}{6}{⼝}
  \begin{phonetics}{吸}{xi1}[][HSK 4]
    \definition{v.}{inalar; inspirar; aspirar; itroduzir líquidos, gases, etc. no corpo | absorver; sugar | atrair; atrair para si mesmo; atrair (interesse, investimento etc.)}
  \end{phonetics}
\end{entry}

\begin{entry}{吸引}{6,4}{⼝、⼸}
  \begin{phonetics}{吸引}{xi1yin3}[][HSK 4]
    \definition{v.}{atrair; apelar para; chamar a atenção de outros objetos, forças ou pessoas para si mesmo}
  \end{phonetics}
\end{entry}

\begin{entry}{吸收}{6,6}{⼝、⽁}
  \begin{phonetics}{吸收}{xi1shou1}[][HSK 4]
    \definition{v.}{imbuir; absorver; assimilar; sugar;  chupar; (animais, plantas, etc.) extrair material de fora dos tecidos para o interior dos tecidos | absorver; chupar;  sugar alguma substância de fora para dentro | recrutar; alistar; inscrever-se; matricular-se; admitir; (organizações ou coletivos) aceitar novos membros | absorver; aproveitar e usar a experiência, o conhecimento, o dinheiro e outras coisas valiosas de outras pessoas | absorver; diminuir, atenuar ou eliminar determinados efeitos ou fenômenos}
  \end{phonetics}
\end{entry}

\begin{entry}{吸烟}{6,10}{⼝、⽕}
  \begin{phonetics}{吸烟}{xi1yan1}[][HSK 4]
    \definition{v.+compl.}{fumar}
  \end{phonetics}
\end{entry}

\begin{entry}{吸铁石}{6,10,5}{⼝、⾦、⽯}
  \begin{phonetics}{吸铁石}{xi1tie3shi2}
    \definition{s.}{imã | magneto}
  \seealsoref{磁铁}{ci2tie3}
  \end{phonetics}
\end{entry}

\begin{entry}{吸管}{6,14}{⼝、⽵}
  \begin{phonetics}{吸管}{xi1 guan3}[][HSK 4]
    \definition[根,个]{s.}{tubo de sucção; sugador; canudo (para beber); refere-se ao tubo fino usado para sugar bebidas | conta-gotas; pipeta; cateter para bombeamento de líquidos usando pressão de ar}
  \end{phonetics}
\end{entry}

\begin{entry}{回}{6}{⼞}
  \begin{phonetics}{回}{hui2}[][HSK 1,2]
    \definition*{s.}{Sobrenome Hui}
    \definition*{s.}{Etnia Hui (mulçumanos chineses)}
    \definition{clas.}{usado para coisas, ações, número de vezes |  um trecho de um conto; um capítulo de um romance em capítulos | seção ou capítulo (de um livro clássico)}
    \definition{v.}{circular; enrolar | retornar; voltar; voltar ao lugar de origem | dar meia-volta | responder; contestar | relatar; reportar; responder}
  \end{phonetics}
\end{entry}

\begin{entry}{回忆}{6,4}{⼞、⼼}
  \begin{phonetics}{回忆}{hui2yi4}[][HSK 5]
    \definition[个,段]{s.}{memória; lembrança de eventos ou experiências passadas}
    \definition{v.}{lembrar; recordar}
  \end{phonetics}
\end{entry}

\begin{entry}{回去}{6,5}{⼞、⼛}
  \begin{phonetics}{回去}{hui2 qu4}[][HSK 1]
    \definition{v.}{retornar; voltar; estar de volta ; (a partir da minha localização)}
  \end{phonetics}
\end{entry}

\begin{entry}{回头}{6,5}{⼞、⼤}
  \begin{phonetics}{回头}{hui2 tou2}[][HSK 5]
    \definition{adv.}{mais tarde; depois de um tempo}
    \definition{conj.}{ou então; usado no início da segunda metade de uma frase para indicar o que acontecerá se você não fizer o que fez na primeira metade da frase}
    \definition{v.}{dar a meia-volta; virar a cabeça; virar a cabeça para trás | retornar; voltar | arrepender-se; corrigir seu caminho; reconhecer e corrigir erros}
  \end{phonetics}
\end{entry}

\begin{entry}{回收}{6,6}{⼞、⽁}
  \begin{phonetics}{回收}{hui2shou1}[][HSK 5]
    \definition{v.}{reciclar; reciclar itens (geralmente resíduos ou produtos antigos) para reutilização | recuperar; recolher; recuperar o que foi emitido ou demitido}
  \end{phonetics}
\end{entry}

\begin{entry}{回报}{6,7}{⼞、⼿}
  \begin{phonetics}{回报}{hui2bao4}[][HSK 5]
    \definition{s.}{recompensa; pagamento; benefícios recebidos como resultado de assistência, esforço ou afeto | retornos; benefícios recebidos por meio de investimentos}
    \definition{v.}{pagar de volta; beneficiar pessoas ou organizações que os ajudaram ou cuidaram deles de alguma forma}
  \end{phonetics}
\end{entry}

\begin{entry}{回来}{6,7}{⼞、⽊}
  \begin{phonetics}{回来}{hui2 lai5}[][HSK 1]
    \definition{v.}{voltar; regressar (para a minha localização) | retornar; usado após um verbo, significa ``vir ao lugar original''}
  \end{phonetics}
\end{entry}

\begin{entry}{回到}{6,8}{⼞、⼑}
  \begin{phonetics}{回到}{hui2 dao4}[][HSK 1]
    \definition{v.}{retornar para; voltar e chegar (ao lugar onde estava originalmente); (após uma mudança nas circunstâncias) retornar ao estado original}
  \end{phonetics}
\end{entry}

\begin{entry}{回国}{6,8}{⼞、⼞}
  \begin{phonetics}{回国}{hui2 guo2}[][HSK 2]
    \definition{v.}{regressar ao seu país (terra natal); referindo-se a voltar do exterior}
  \end{phonetics}
\end{entry}

\begin{entry}{回信}{6,9}{⼞、⼈}
  \begin{phonetics}{回信}{hui2 xin4}[][HSK 5]
    \definition[封]{s.}{uma carta em resposta; uma mensagem verbal em resposta}
    \definition{v.+compl.}{escrever em resposta; escrever de volta; responder uma carta; responder verbalmente uma mensagem}
  \end{phonetics}
\end{entry}

\begin{entry}{回复}{6,9}{⼞、⼢}
  \begin{phonetics}{回复}{hui2 fu4}[][HSK 4]
    \definition{v.}{responder (a uma carta) | retornar ao estado normal; restaurar algo ao seu estado original}
  \end{phonetics}
\end{entry}

\begin{entry}{回家}{6,10}{⼞、⼧}
  \begin{phonetics}{回家}{hui2 jia1}[][HSK 1]
    \definition{v.}{ir (voltar) para casa; estar em casa; voltar para casa}
  \end{phonetics}
\end{entry}

\begin{entry}{回顾}{6,10}{⼞、⾴}
  \begin{phonetics}{回顾}{hui2gu4}[][HSK 5]
    \definition{v.}{olhar para trás | revisar; fazer uma retrospectiva; olhar para trás, pensar no passado}
  \end{phonetics}
\end{entry}

\begin{entry}{回旋}{6,11}{⼞、⽅}
  \begin{phonetics}{回旋}{hui2xuan2}
    \definition{v.}{circular | rodar | dar a volta}
  \end{phonetics}
\end{entry}

\begin{entry}{回答}{6,12}{⼞、⽵}
  \begin{phonetics}{回答}{hui2da2}[][HSK 1]
    \definition[个]{s.}{resposta}
    \definition{v.}{responder; explicar a questão; expressar opinião sobre a solicitação}
  \end{phonetics}
\end{entry}

\begin{entry}{回避}{6,16}{⼞、⾌}
  \begin{phonetics}{回避}{hui2bi4}
    \definition{v.}{fugir (de um problema); em direito, refere-se especificamente à não participação nos procedimentos de um caso de um oficial de justiça, etc., que tenha interesse no caso ou nas partes do caso | esquivar-se; evadir-se; evitar (encontrar alguém)}
  \end{phonetics}
\end{entry}

\begin{entry}{因}{6}{⼞}
  \begin{phonetics}{因}{yin1}[][HSK 6]
    \definition*{s.}{Sobrenome Yin}
    \definition{conj.}{porque; orações de conexão, indicando relações de causa e efeito}
    \definition{prep.}{com base em; à luz de; de acordo com; a introdução da ação comportamental equivale a 按照 ou 根据}
    \definition{s.}{causa; motivo; condições em que algo ocorre ou causa um determinado resultado (em oposição a 果)}
    \definition{v.}{seguir; continuar; fazer como sempre fez | estar em conformidade com; estar de acordo com; depender; contar com}
  \seealsoref{按照}{an4zhao4}
  \seealsoref{根据}{gen1ju4}
  \seealsoref{果}{guo3}
  \end{phonetics}
\end{entry}

\begin{entry}{因为}{6,4}{⼞、⼂}
  \begin{phonetics}{因为}{yin1wei4}[][HSK 2]
    \definition{conj.}{porque; indica o motivo e a frase seguinte indica o resultado}
    \definition{prep.}{por causa de; por conta de; indica razão ou justificativa}
  \end{phonetics}
\end{entry}

\begin{entry}{因为……所以……}{6,4,8,4}{⼞、⼂、⼾、⼈}
  \begin{phonetics}{因为……所以……}{yin1wei4 suo3yi3}[][HSK 2]
    \definition{conj.}{porque\dots portanto\dots}
  \end{phonetics}
\end{entry}

\begin{entry}{因此}{6,6}{⼞、⽌}
  \begin{phonetics}{因此}{yin1ci3}[][HSK 3]
    \definition{conj.}{assim; portanto; consequentemente}
  \end{phonetics}
\end{entry}

\begin{entry}{因此就}{6,6,12}{⼞、⽌、⼪}
  \begin{phonetics}{因此就}{yin1ci3 jiu4}
    \definition{conj.}{portanto}
  \end{phonetics}
\end{entry}

\begin{entry}{因而}{6,6}{⼞、⽽}
  \begin{phonetics}{因而}{yin1'er2}[][HSK 5]
    \definition{conj.}{; como resultado; com o resultado que; conecta frases, indicando relação de causa e efeito}
  \end{phonetics}
\end{entry}

\begin{entry}{团}{6}{⼞}
  \begin{phonetics}{团}{tuan2}[][HSK 3]
    \definition*{s.}{Liga da Juventude Comunista da China; Liga}
    \definition{adj.}{redondo; circular}
    \definition{clas.}{usado para algo em forma de bola}
    \definition[个]{s.}{bolinho de massa; comida em forma de bola feita de arroz ou farinha | algo em forma de bola | grupo; corpo; sociedade; organização; um grupo envolvido em um determinado trabalho ou atividade | regimento; unidade organizacional militar, geralmente abaixo do nível de divisão e acima do nível de batalhão}
    \definition{v.}{enrolar algo para formar uma bola; rolar | reunir; unir; conglomerar}
  \end{phonetics}
\end{entry}

\begin{entry}{团长}{6,4}{⼞、⾧}
  \begin{phonetics}{团长}{tuan2 zhang3}[][HSK 5]
    \definition{s.}{comandante do regimento | chefe (ou presidente) de uma delegação, trupe, etc. | líder de uma delegação}
  \end{phonetics}
\end{entry}

\begin{entry}{团队}{6,4}{⼞、⾩}
  \begin{phonetics}{团队}{tuan2dui4}
    \definition{s.}{equipe}
  \end{phonetics}
\end{entry}

\begin{entry}{团体}{6,7}{⼞、⼈}
  \begin{phonetics}{团体}{tuan2ti3}[][HSK 3]
    \definition[种,个]{s.}{equipe; grupo; organização; um grupo de pessoas com objetivos e interesses comuns}
  \end{phonetics}
\end{entry}

\begin{entry}{团结}{6,9}{⼞、⽷}
  \begin{phonetics}{团结}{tuan2jie2}[][HSK 3]
    \definition{adj.}{unido; amigável; harmonioso; relação harmoniosa e coexistência harmoniosa}
    \definition{v.}{unir; reunir}
  \end{phonetics}
\end{entry}

\begin{entry}{在}{6}{⼟}
  \begin{phonetics}{在}{zai4}[][HSK 1]
    \definition{adv.}{em processo de; em curso de}
    \definition{prep.}{em; no (um lugar ou momento); indica tempo, local, âmbito, etc.}
    \definition{v.}{existir; estar vivo | estar em; estar no; estar em (um lugar); indica a localização de pessoas ou coisas | permanecer; ficar | depender de; residir em; repousar com | ingressar ou pertencer a uma organização; ser membro de uma organização}
  \end{phonetics}
\end{entry}

\begin{entry}{在下}{6,3}{⼟、⼀}
  \begin{phonetics}{在下}{zai4xia4}
    \definition{pron.}{eu mesmo (humildemente)}
  \end{phonetics}
\end{entry}

\begin{entry}{在于}{6,3}{⼟、⼆}
  \begin{phonetics}{在于}{zai4yu2}[][HSK 4]
    \definition{v.}{ser responsável por; caber a;  ser da competência de;  apontar a essência das coisas, ou do que elas se tratam | depender de; ser determinado por;  ser devido a (um determinado atributo)/(de um assunto a ser determinado)}
  \end{phonetics}
\end{entry}

\begin{entry}{在内}{6,4}{⼟、⼌}
  \begin{phonetics}{在内}{zai4 nei4}[][HSK 5]
    \definition{adj.}{incluido}
    \definition{adv.}{dentro; internamente; entre eles}
    \definition{v.}{ser incluído}
  \end{phonetics}
\end{entry}

\begin{entry}{在乎}{6,5}{⼟、⼃}
  \begin{phonetics}{在乎}{zai4hu5}[][HSK 4]
    \definition{v.}{preocupar-se; preocupar-se com; levar a sério | ser responsável por; caber ao; ser da competência de}
  \end{phonetics}
\end{entry}

\begin{entry}{在地}{6,6}{⼟、⼟}
  \begin{phonetics}{在地}{zai4di4}
    \definition{s.}{local}
  \end{phonetics}
\end{entry}

\begin{entry}{在场}{6,6}{⼟、⼟}
  \begin{phonetics}{在场}{zai4 chang3}[][HSK 5]
    \definition{v.}{estar presente; estar no local; estar em cena; estar presente onde as coisas estão acontecendo}
  \end{phonetics}
\end{entry}

\begin{entry}{在此}{6,6}{⼟、⽌}
  \begin{phonetics}{在此}{zai4ci3}
    \definition{adv.}{aqui}
  \end{phonetics}
\end{entry}

\begin{entry}{在行}{6,6}{⼟、⾏}
  \begin{phonetics}{在行}{zai4hang2}
    \definition{v.}{ser adepto de algo | ser um especialista em um comércio ou profissão}
  \end{phonetics}
\end{entry}

\begin{entry}{在线}{6,8}{⼟、⽷}
  \begin{phonetics}{在线}{zai4xian4}
    \definition{s.}{\emph{online}}
  \end{phonetics}
\end{entry}

\begin{entry}{在家}{6,10}{⼟、⼧}
  \begin{phonetics}{在家}{zai4 jia1}[][HSK 1]
    \definition{v.}{estar em; estar em casa; estar no local de trabalho ou alojamento; sem sair de casa | continuar sendo um leigo; permanecer leigo; para monges, freiras, taoístas e outros que 出家, as pessoas comuns são consideradas leigas}
  \seealsoref{出家}{chu1 jia1}
  \end{phonetics}
\end{entry}

\begin{entry}{在教}{6,11}{⼟、⽁}
  \begin{phonetics}{在教}{zai4jiao4}
    \definition{v.}{ser um crente (em uma religião)}
  \end{phonetics}
\end{entry}

\begin{entry}{在意}{6,13}{⼟、⼼}
  \begin{phonetics}{在意}{zai4yi4}
    \definition{v.+compl.}{preocupar-se | importar-se | levar a sério}
  \end{phonetics}
\end{entry}

\begin{entry}{地}{6}{⼟}
  \begin{phonetics}{地}{de5}[][HSK 1]
    \definition{part.}{(estrutural) utilizada antes de um verbo ou adjetivo, ligando-o ao adjunto adverbial modificador precedente}
  \end{phonetics}
  \begin{phonetics}{地}{di4}[][HSK 1]
    \definition*{s.}{A Terra | Sobrenome Di}
    \definition[块,片]{s.}{terra; solo | campos | chão; piso | posição; situação | contexto; base | distância percorrida (medida em 里 ou paradas 站) | indicando estado de espírito | território | lugar; local | parte do espaço | distância}
  \end{phonetics}
\end{entry}

\begin{entry}{地上}{6,3}{⼟、⼀}
  \begin{phonetics}{地上}{di4 shang5}[][HSK 1]
    \definition{adv.}{no chão; no solo; em terra}
  \end{phonetics}
\end{entry}

\begin{entry}{地下}{6,3}{⼟、⼀}
  \begin{phonetics}{地下}{di4 xia4}[][HSK 4]
    \definition{s.}{subterrâneo | secreta (atividade) | recursos ocultos}
  \end{phonetics}
\end{entry}

\begin{entry}{地下室}{6,3,9}{⼟、⼀、⼧}
  \begin{phonetics}{地下室}{di4xia4shi4}
    \definition{s.}{subterrâneo | porão}
  \end{phonetics}
\end{entry}

\begin{entry}{地区}{6,4}{⼟、⼖}
  \begin{phonetics}{地区}{di4qu1}[][HSK 3]
    \definition[个,片]{s.}{área; distrito; região; um lugar maior | prefeitura; unidade administrativa | latitudes; localidade; lado | em determinadas circunstâncias, algumas regiões administrativas locais da China, como Hong Kong e Macau, participam individualmente em algumas atividades internacionais}
    \definition{suf.}{como sufixo do nome da cidade, significa prefeitura ou condado}
  \end{phonetics}
\end{entry}

\begin{entry}{地方}{6,4}{⼟、⽅}
  \begin{phonetics}{地方}{di4fang1}
    \definition[个]{s.}{distrito; localidade;  em oposição a 中央, o número total de unidades administrativas em todos os níveis abaixo do centro | governo local e população; refere-se a outros setores que não o militar}
  \seealsoref{中央}{zhong1yang1}
  \end{phonetics}
  \begin{phonetics}{地方}{di4fang5}[][HSK 1,4]
    \definition[个,处,块]{s.}{lugar; cômodo; área; refere-se a um espaço específico | parte}
  \end{phonetics}
\end{entry}

\begin{entry}{地位}{6,7}{⼟、⼈}
  \begin{phonetics}{地位}{di4wei4}[][HSK 4]
    \definition{s.}{lugar; status; posição; posição da pessoa ou do grupo nas relações sociais | lugar; posição (ocupada por uma pessoa ou coisa); espaço ocupado por uma pessoa ou coisa}
  \end{phonetics}
\end{entry}

\begin{entry}{地址}{6,7}{⼟、⼟}
  \begin{phonetics}{地址}{di4zhi3}[][HSK 4]
    \definition[个]{s.}{endereço; local de residência ou correspondência}
  \end{phonetics}
\end{entry}

\begin{entry}{地形}{6,7}{⼟、⼺}
  \begin{phonetics}{地形}{di4 xing2}[][HSK 5]
    \definition{s.}{topografia; forma do terreno; relevo; disposição do terreno; característica do relevo; característica da superfície; terreno}
  \end{phonetics}
\end{entry}

\begin{entry}{地图}{6,8}{⼟、⼞}
  \begin{phonetics}{地图}{di4tu2}[][HSK 1]
    \definition[张,本]{s.}{mapa; mapa que mostra a distribuição de coisas e fenômenos na superfície da Terra, com símbolos e textos, e às vezes também com cores}
  \end{phonetics}
\end{entry}

\begin{entry}{地带}{6,9}{⼟、⼱}
  \begin{phonetics}{地带}{di4 dai4}[][HSK 5]
    \definition[个]{s.}{distrito; região; zona; área de uma determinada natureza ou extensão}
  \end{phonetics}
\end{entry}

\begin{entry}{地点}{6,9}{⼟、⽕}
  \begin{phonetics}{地点}{di4dian3}[][HSK 1]
    \definition[个]{s.}{lugar; local; região; localização}
  \end{phonetics}
\end{entry}

\begin{entry}{地狱}{6,9}{⼟、⽝}
  \begin{phonetics}{地狱}{di4yu4}
    \definition*{s.}{Budismo: Naraka}
    \definition{adj.}{infernal}
    \definition{s.}{inferno | submundo}
  \end{phonetics}
\end{entry}

\begin{entry}{地砖}{6,9}{⼟、⽯}
  \begin{phonetics}{地砖}{di4zhuan1}
    \definition{s.}{ladrilho de piso}
  \end{phonetics}
\end{entry}

\begin{entry}{地面}{6,9}{⼟、⾯}
  \begin{phonetics}{地面}{di4 mian4}[][HSK 4]
    \definition{s.}{a superfície da Terra | térreo; piso; camada de material colocada no chão dentro e ao redor dos edifícios | localidade; chão | região; território; principalmente áreas administrativas}
  \end{phonetics}
\end{entry}

\begin{entry}{地核}{6,10}{⼟、⽊}
  \begin{phonetics}{地核}{di4he2}
    \definition{s.}{(geologia) núcleo da Terra}
  \end{phonetics}
\end{entry}

\begin{entry}{地铁}{6,10}{⼟、⾦}
  \begin{phonetics}{地铁}{di4tie3}[][HSK 2]
    \definition[条,班,列,趟]{s.}{metrô; trem subterrâneo; também se refere ao vagão do metrô}
  \end{phonetics}
\end{entry}

\begin{entry}{地铁站}{6,10,10}{⼟、⾦、⽴}
  \begin{phonetics}{地铁站}{di4 tie3 zhan4}[][HSK 2]
    \definition[个,座]{s.}{estação de metrô}
  \end{phonetics}
\end{entry}

\begin{entry}{地球}{6,11}{⼟、⽟}
  \begin{phonetics}{地球}{di4qiu2}[][HSK 2]
    \definition[个]{s.}{o planeta Terra}
  \end{phonetics}
\end{entry}

\begin{entry}{地理}{6,11}{⼟、⽟}
  \begin{phonetics}{地理}{di4li3}
    \definition{s.}{geografia}
  \end{phonetics}
\end{entry}

\begin{entry}{地震}{6,15}{⼟、⾬}
  \begin{phonetics}{地震}{di4zhen4}[][HSK 5]
    \definition[场,次,级]{s.}{sismo; terremoto; tremor de terra; vibrações na crosta terrestre}
    \definition{v.}{sacudir com vibrações sísmicas}
  \end{phonetics}
\end{entry}

\begin{entry}{场}{6}{⼟}
  \begin{phonetics}{场}{chang2}
    \definition{clas.}{usado para descrever o desenrolar dos acontecimentos}
    \definition{s.}{eira; espaço aberto e plano; um terreno plano, geralmente usado para secar grãos e moer cereais | mercado; feira rural}
  \end{phonetics}
  \begin{phonetics}{场}{chang3}[][HSK 2]
    \definition*{s.}{Sobrenome Chang}
    \definition{clas.}{usado para atividades culturais, recreativas e esportivas | usado para pequenos trechos de uma peça}
    \definition{s.}{um local amplo utilizado para um fim específico | palco; campo | cena | (física) campo (por exemplo: campo manético) | (para atividades recreativas, esportivas ou outras) | um lugar onde as pessoas se reúnem | fazenda; quinta | abertura; encerramento; refere-se ao processo completo de uma apresentação ou competição | local; ponto; o local onde ocorreu o incidente}
  \end{phonetics}
\end{entry}

\begin{entry}{场合}{6,6}{⼟、⼝}
  \begin{phonetics}{场合}{chang3he2}[][HSK 3]
    \definition[个,些,种,类]{s.}{ocasião; situação; um certo tempo, lugar ou situação}
  \end{phonetics}
\end{entry}

\begin{entry}{场地}{6,6}{⼟、⼟}
  \begin{phonetics}{场地}{chang3 di4}[][HSK 6]
    \definition[片,块,个]{s.}{área; pátio; espaço; lugar; quadra; campo; um lugar onde construções ou atividades são realizadas}
  \end{phonetics}
\end{entry}

\begin{entry}{场所}{6,8}{⼟、⼾}
  \begin{phonetics}{场所}{chang3suo3}[][HSK 3]
    \definition{s.}{lugar; sítio; arena; local da atividade}
  \end{phonetics}
\end{entry}

\begin{entry}{场面}{6,9}{⼟、⾯}
  \begin{phonetics}{场面}{chang3mian4}[][HSK 5]
    \definition[个,种,番]{s.}{espetáculo; cena (em teatro, ficção, etc.); uma cena em uma produção teatral, cinematográfica ou televisiva que consiste em um cenário, música e personagens | cena; ocasião; literatura narrativa que consiste em situações da vida em que os personagens se relacionam entre si em determinadas ocasiões | orquestra ou instrumentos de acompanhamento (em ópera); refere-se às pessoas e aos instrumentos musicais que acompanham a apresentação de uma ópera, divididos em dois tipos: música de sopro e cordas é uma cena cultural, e gongos e tambores são uma cena marcial | situação; referência geral a uma situação em um determinado contexto | frente; fachada; aparência; espetáculo superficial}
  \end{phonetics}
\end{entry}

\begin{entry}{场馆}{6,11}{⼟、⾷}
  \begin{phonetics}{场馆}{chang3 guan3}[][HSK 6]
    \definition{s.}{ginásios e estádios | arena | local esportivo}
  \end{phonetics}
\end{entry}

\begin{entry}{场景}{6,12}{⼟、⽇}
  \begin{phonetics}{场景}{chang3 jing3}[][HSK 6]
    \definition[个,幕,种]{s.}{espetáculo; cena (em drama, ficção, etc.); refere-se a cenas de drama, cinema, televisão e obras literárias | cena; visão; circunstâncias; cenas e situações}
  \end{phonetics}
\end{entry}

\begin{entry}{多}{6}{⼣}
  \begin{phonetics}{多}{duo1}[][HSK 1,2]
    \definition*{s.}{Sobrenome Duo}
    \definition{adj.}{grande quantidade (oposto de 少, 寡) | excessivo; desnecessário | excessivo; em demasia; indica um grande grau de diferença | mais do que o número correto ou necessário; em excesso}
    \definition{adv.}{acima de um valor especificado; e mais | em que medida; usado em frases interrogativas para indagar sobre grau ou quantidade, equivalente a 多么 | uma extensão não especificada; usado em frases exclamativas para expressar um alto grau, equivalente a 多么 | quase; significa que a maior parte do intervalo é assim | mais;  sobre; ímpar; usado depois de um quantificador para indicar uma fração}
    \definition{num.}{(após um número) ímpar}
    \definition{pref.}{multi- | poli-}
    \definition{v.}{ter (uma quantidade específica) a mais ou a mais (oposto a 少) | ter algo em abundância  | (em perguntas) até que ponto | (em exclamações) até que ponto | ter mais}
  \seealsoref{多么}{duo1me5}
  \seealsoref{寡}{gua3}
  \seealsoref{少}{shao3}
  \end{phonetics}
\end{entry}

\begin{entry}{多久}{6,3}{⼣、⼃}
  \begin{phonetics}{多久}{duo1 jiu3}[][HSK 2]
    \definition{pron.}{quanto tempo?; quanto tempo; perguntar quanto tempo leva}
  \end{phonetics}
\end{entry}

\begin{entry}{多么}{6,3}{⼣、⼃}
  \begin{phonetics}{多么}{duo1me5}[][HSK 2]
    \definition{adv.}{(em exclamações) como; o quê; em que medida; usado em frases exclamativas, indica um grau muito alto | em grau indeterminado; usado em frases declarativas, indica um grau mais profundo | como (usado em uma frase interrogativa para perguntar sobre grau ou número)}
  \end{phonetics}
\end{entry}

\begin{entry}{多大}{6,3}{⼣、⼤}
  \begin{phonetics}{多大}{duo1da4}
    \definition{adj.}{quantos anos? | que idade? | quão grande?}
  \end{phonetics}
\end{entry}

\begin{entry}{多云}{6,4}{⼣、⼆}
  \begin{phonetics}{多云}{duo1 yun2}[][HSK 2]
    \definition{adj.}{céu nublado; em meteorologia, refere-se a condições atmosféricas em que a cobertura de nuvens médias e baixas ocupa entre 40\% e 70\% da área do céu, ou a cobertura de nuvens altas ocupa entre 60\% e 100\% da área do céu}
  \end{phonetics}
\end{entry}

\begin{entry}{多少}{6,4}{⼣、⼩}
  \begin{phonetics}{多少}{duo1shao3}
    \definition{adv.}{um pouco; mais ou menos; até certo ponto}
    \definition{s.}{número; quantidade; volume}
  \end{phonetics}
  \begin{phonetics}{多少}{duo1shao5}[][HSK 1]
    \definition{adv.}{quantos?; quanto?; usado em perguntas para perguntar sobre quantidade | expressar uma quantidade ou número não especificado; quantidade indefinida}
  \end{phonetics}
\end{entry}

\begin{entry}{多年}{6,6}{⼣、⼲}
  \begin{phonetics}{多年}{duo1 nian2}[][HSK 4]
    \definition{adv.}{por muitos anos; durante muitos anos}
  \end{phonetics}
\end{entry}

\begin{entry}{多次}{6,6}{⼣、⽋}
  \begin{phonetics}{多次}{duo1 ci4}[][HSK 4]
    \definition{adv.}{muitas vezes; de vez em quando; repetidamente; em muitas ocasiões}
  \end{phonetics}
\end{entry}

\begin{entry}{多咱}{6,9}{⼣、⼝}
  \begin{phonetics}{多咱}{duo1 zan5}
    \definition{adv.}{que horas?; quando?}
  \end{phonetics}
\end{entry}

\begin{entry}{多种}{6,9}{⼣、⽲}
  \begin{phonetics}{多种}{duo1 zhong3}[][HSK 4]
    \definition{adj.}{diverso; vários tipos de; múltiplo; diversificado}
  \end{phonetics}
\end{entry}

\begin{entry}{多重}{6,9}{⼣、⾥}
  \begin{phonetics}{多重}{duo1chong2}
    \definition{pref.}{multi (facetado, cultural, étnico, etc.)}
  \end{phonetics}
\end{entry}

\begin{entry}{多样}{6,10}{⼣、⽊}
  \begin{phonetics}{多样}{duo1 yang4}[][HSK 4]
    \definition{adj.}{diversos; variados; diversificado}
    \definition{s.}{diversidade}
  \end{phonetics}
\end{entry}

\begin{entry}{多数}{6,13}{⼣、⽁}
  \begin{phonetics}{多数}{duo1 shu4}[][HSK 2]
    \definition{adj.}{maioria; a maioria; plural}
    \definition{pref.}{pluri-}
  \end{phonetics}
\end{entry}

\begin{entry}{夹}{6}{⼤}
  \begin{phonetics}{夹}{ga1}
    \definition{s.}{axila; sovaco; atualmente, costuma-se escrever 胳肢窝}
  \seealsoref{胳肢窝}{ga1 zhi1 wo1}
  \end{phonetics}
  \begin{phonetics}{夹}{jia1}[][HSK 5]
    \definition{s.}{clipe, grampo, pasta, etc.}
    \definition{v.}{colocar no meio; pressionar de ambos os lados; aplicar força ou ação ao mesmo objeto de ambos os lados ao mesmo tempo | misturar; mesclar; intercalar}
  \end{phonetics}
  \begin{phonetics}{夹}{jia2}
    \definition{adj.}{forrado; com camada dupla; duas camadas (roupas, colchas, etc.) | pinçado; voz deliberadamente engraçada}
  \end{phonetics}
\end{entry}

\begin{entry}{夹肢窝}{6,8,12}{⼤、⾁、⽳}
  \begin{phonetics}{夹肢窝}{jia1 zhi1 wo1}
    \definition{s.}{axila; sovaco; também escrito como 胳肢窝}
  \seealsoref{胳肢窝}{ga1 zhi1 wo1}
  \end{phonetics}
\end{entry}

\begin{entry}{夺}{6}{⼤}
  \begin{phonetics}{夺}{duo2}[][HSK 6]
    \definition{v.}{tomar à força; apreender; arrancar; roubar | forçar a passagem; empurrar para abrir | lutar por; competir por; esforçar-se por; obter primeiro | privar; perder | perder; tirar | decidir; tomar uma decisão | omitir (palavra em um texto)}
  \end{phonetics}
\end{entry}

\begin{entry}{夺冠}{6,9}{⼤、⼍}
  \begin{phonetics}{夺冠}{duo2guan4}
    \definition{v.}{apoderar-se da coroa | (fig.) ganhar um campeonato | ganhar a medalha de ouro}
  \end{phonetics}
\end{entry}

\begin{entry}{奸}{6}{⼥}
  \begin{phonetics}{奸}{jian1}
    \definition{adj.}{perverso; maligno; traiçoeiro; malicioso}
    \definition{s.}{traidor; espião | pessoa perversa; pessoa traiçoeira | relações sexuais ilícitas; comportamento sexual impróprio}
    \definition{v.}{ter relações sexuais ilícitas}
  \end{phonetics}
\end{entry}

\begin{entry}{奸夫}{6,4}{⼥、⼤}
  \begin{phonetics}{奸夫}{jian1fu1}
    \definition{s.}{homem adúltero}
  \end{phonetics}
\end{entry}

\begin{entry}{她}{6}{⼥}
  \begin{phonetics}{她}{ta1}[][HSK 1]
    \definition{pron.}{ela | ela; referir-se a coisas que se ama ou aprecia, como a pátria, a bandeira nacional, etc.}
  \end{phonetics}
\end{entry}

\begin{entry}{她们}{6,5}{⼥、⼈}
  \begin{phonetics}{她们}{ta1men5}[][HSK 1]
    \definition{pron.}{elas; referindo-se a várias mulheres: em textos escritos, use 她们 quando todas as pessoas forem mulheres e 他们 quando houver homens e mulheres}
  \seealsoref{他们}{ta1men5}
  \end{phonetics}
\end{entry}

\begin{entry}{她们的}{6,5,8}{⼥、⼈、⽩}
  \begin{phonetics}{她们的}{ta1men5 de5}
    \definition{pron.}{delas}
  \end{phonetics}
\end{entry}

\begin{entry}{她的}{6,8}{⼥、⽩}
  \begin{phonetics}{她的}{ta1 de5}
    \definition{pron.}{dela}
  \end{phonetics}
\end{entry}

\begin{entry}{好}{6}{⼥}
  \begin{phonetics}{好}{hao3}[][HSK 1,2,4]
    \definition{adj.}{bom; ótimo; agradável; vantajoso; satisfatório | amigável; gentil; amistoso; amável | saudável; bem | pronto; concluído; usado após um verbo para indicar conclusão ou perfeição | fácil (de fazer); conveniente; responsável (por)}
    \definition{adv.}{muito; bastante; tão; usado na frente de uma palavra de quantidade ou uma palavra de tempo para indicar muito ou por muito tempo | em que medida; como; usado antes de adjetivos e verbos para indicar profundidade e com exclamação}
    \definition{interj.}{O.K.; tudo bem; aprovação, acordo ou encerramento | (no início de uma frase ou oração) expressa concordância (ou desaprovação, surpresa, etc.)}
    \definition{prep.}{de modo a; para que}
    \definition{s.}{referindo-se a palavras de elogio ou aplauso | saudações; cumprimentos}
    \definition{suf.}{sufixo que indica conclusão ou prontidão | depois de um pronome significa ``olá''}
    \definition{v.}{deve; precisa; tem que; deveria | apaixonar-se}
  \end{phonetics}
  \begin{phonetics}{好}{hao4}
    \definition*{s.}{Sobrenome Hao}
    \definition{adv.}{algo que acontece com frequência, que é fácil de acontecer}
    \definition{v.}{gostar; amar; ter afeição por}
  \end{phonetics}
\end{entry}

\begin{entry}{好人}{6,2}{⼥、⼈}
  \begin{phonetics}{好人}{hao3 ren2}[][HSK 2]
    \definition[个,位,名]{s.}{pessoa boa (ou excelente) (oposto de 坏人) | pessoa saudável | pessoa gentil que tenta se dar bem com todos (muitas vezes em detrimento dos princípios)}
  \seealsoref{坏人}{huai4 ren2}
  \end{phonetics}
\end{entry}

\begin{entry}{好久}{6,3}{⼥、⼃}
  \begin{phonetics}{好久}{hao3jiu3}[][HSK 2]
    \definition{adv.}{por muito tempo | por eras (no passado)}
  \end{phonetics}
\end{entry}

\begin{entry}{好友}{6,4}{⼥、⼜}
  \begin{phonetics}{好友}{hao3you3}[][HSK 4]
    \definition[位,个]{s.}{bom amigo; amigo próximo}
  \end{phonetics}
\end{entry}

\begin{entry}{好心}{6,4}{⼥、⼼}
  \begin{phonetics}{好心}{hao3xin1}
    \definition{s.}{bondade | boas intenções}
  \end{phonetics}
\end{entry}

\begin{entry}{好处}{6,5}{⼥、⼡}
  \begin{phonetics}{好处}{hao3chu4}[][HSK 2]
    \definition[个]{s.}{bom; benefício; vantagem; fatores favoráveis a pessoas ou coisas | ganho; lucro; algo que não se deveria receber, dado por outra pessoa ou obtido através de uma oportunidade; geralmente tem conotação pejorativa}
  \end{phonetics}
\end{entry}

\begin{entry}{好汉}{6,5}{⼥、⽔}
  \begin{phonetics}{好汉}{hao3han4}
    \definition[条]{s.}{herói | pessoa forte e corajosa}
  \end{phonetics}
\end{entry}

\begin{entry}{好生}{6,5}{⼥、⽣}
  \begin{phonetics}{好生}{hao3sheng1}
    \definition{adv.}{bastante; extremamente | cuidadosamente; apropriadamente}
  \end{phonetics}
\end{entry}

\begin{entry}{好用}{6,5}{⼥、⽤}
  \begin{phonetics}{好用}{hao3yong4}
    \definition{adj.}{fácil de usar | adequado ao uso}
  \end{phonetics}
\end{entry}

\begin{entry}{好吃}{6,6}{⼥、⼝}
  \begin{phonetics}{好吃}{hao3chi1}[][HSK 1]
    \definition{adj.}{bom; saboroso; delicioso; descreve o sabor agradável de algo, que as pessoas gostam de comer}
  \end{phonetics}
  \begin{phonetics}{好吃}{hao4chi1}
    \definition{v.}{ser guloso; gostar de comer boa comida}
  \end{phonetics}
\end{entry}

\begin{entry}{好多}{6,6}{⼥、⼣}
  \begin{phonetics}{好多}{hao3 duo1}[][HSK 2]
    \definition{adj.}{muitos; uma boa quantidade; uma grande quantidade; uma quantidade enorme}
    \definition{pron.}{quantos?; quanto?; frequentemente usado para perguntar sobre quantidade}
  \end{phonetics}
\end{entry}

\begin{entry}{好好}{6,6}{⼥、⼥}
  \begin{phonetics}{好好}{hao3 hao3}[][HSK 3]
    \definition{adj.}{realmente bom/bem; em perfeitas condições; quando tudo está bem}
    \definition{adv.}{diretamente; seriamente; cuidadosamente; com todo o empenho; ao máximo}
  \end{phonetics}
\end{entry}

\begin{entry}{好听}{6,7}{⼥、⼝}
  \begin{phonetics}{好听}{hao3 ting1}[][HSK 1]
    \definition{adj.}{agradável de ouvir (de som ou voz) | bom; palatável; satisfatório (de palavras)  | decente; honrado (de ações, etc.); descreve uma coisa que parece prestigiosa | interessante; descreve palavras, histórias e outras coisas interessantes}
  \end{phonetics}
\end{entry}

\begin{entry}{好运}{6,7}{⼥、⾡}
  \begin{phonetics}{好运}{hao3 yun4}[][HSK 5]
    \definition{s.}{boa sorte, fortuna ou oportunidade}
  \end{phonetics}
\end{entry}

\begin{entry}{好事}{6,8}{⼥、⼅}
  \begin{phonetics}{好事}{hao3 shi4}[][HSK 2]
    \definition[个,件]{s.}{boa ação; gentileza | (antigo) obra de caridade | acontecimento feliz; evento festivo}
  \end{phonetics}
  \begin{phonetics}{好事}{hao4 shi4}
    \definition[个,件]{s.}{intrometido; gostar de se meter na vida dos outros}
  \end{phonetics}
\end{entry}

\begin{entry}{好奇}{6,8}{⼥、⼤}
  \begin{phonetics}{好奇}{hao4qi2}[][HSK 3]
    \definition{adj.}{curioso; curiosidade e interesse por coisas não conhecidas}
    \definition{s.}{curiosidade}
    \definition{v.}{ser ou estar curioso}
  \end{phonetics}
\end{entry}

\begin{entry}{好学}{6,8}{⼥、⼦}
  \begin{phonetics}{好学}{hao3xue2}
    \definition{adj.}{fácil de aprender}
  \end{phonetics}
  \begin{phonetics}{好学}{hao4xue2}
    \definition{s.}{estudioso | erudito}
  \end{phonetics}
\end{entry}

\begin{entry}{好玩儿}{6,8,2}{⼥、⽟、⼉}
  \begin{phonetics}{好玩儿}{hao3 wan2r5}[][HSK 1]
    \definition{adj.}{divertido; interessante; capaz de despertar interesse}
  \end{phonetics}
\end{entry}

\begin{entry}{好看}{6,9}{⼥、⽬}
  \begin{phonetics}{好看}{hao3 kan4}[][HSK 1]
    \definition{adj.}{de boa aparência; agradável; bonito | interessante; descreve o enredo ou conteúdo de filmes, romances, performances, etc., como sendo cativante, agradável ou apreciável}
  \end{phonetics}
\end{entry}

\begin{entry}{好象}{6,11}{⼥、⾗}
  \begin{phonetics}{好象}{hao3xiang4}
    \variantof{好像}
  \end{phonetics}
\end{entry}

\begin{entry}{好像}{6,13}{⼥、⼈}
  \begin{phonetics}{好像}{hao3xiang4}[][HSK 2]
    \definition{adv.}{como se; um pouco parecido; como se fosse}
    \definition{v.}{parecer; ser como; parecer-se com}
  \end{phonetics}
\end{entry}

\begin{entry}{如}{6}{⼥}
  \begin{phonetics}{如}{ru2}[][HSK 6]
    \definition{adv.}{por exemplo; tal como; como}
    \definition{conj.}{se; no caso (de); no caso de; como se; como}
    \definition{prep.}{em conformidade com; de acordo com}
    \definition{v.}{estar em conformidade (ou de acordo) com | (geralmente no negativo) pode ser comparado com; ser comparável a; ser tão bom quanto | superar; exceder | (literário) ir para}
  \end{phonetics}
\end{entry}

\begin{entry}{如下}{6,3}{⼥、⼀}
  \begin{phonetics}{如下}{ru2 xia4}[][HSK 5]
    \definition{adv.}{como descrito ou listado abaixo; conforme segue; conforme abaixo}
  \end{phonetics}
\end{entry}

\begin{entry}{如今}{6,4}{⼥、⼈}
  \begin{phonetics}{如今}{ru2jin1}[][HSK 4]
    \definition{s.}{agora; hoje em dia; atualmente; no presente}
  \end{phonetics}
\end{entry}

\begin{entry}{如同}{6,6}{⼥、⼝}
  \begin{phonetics}{如同}{ru2 tong2}[][HSK 5]
    \definition{v.}{parecer que. usado principalmente em metáforas}
  \end{phonetics}
\end{entry}

\begin{entry}{如此}{6,6}{⼥、⽌}
  \begin{phonetics}{如此}{ru2 ci3}[][HSK 5]
    \definition{adv.}{assim; tal; dessa forma; dessa maneira; refere-se a uma situação mencionada anteriormente, equivalente a 这样}
  \seealsoref{这样}{zhe4 yang4}
  \end{phonetics}
\end{entry}

\begin{entry}{如何}{6,7}{⼥、⼈}
  \begin{phonetics}{如何}{ru2he2}[][HSK 3]
    \definition{pron.}{como?; o que?; usado para perguntar como resolver um problema | como?; o que?; usado para perguntar sobre a situação ou obter a opinião de outras pessoas}
  \end{phonetics}
\end{entry}

\begin{entry}{如果}{6,8}{⼥、⽊}
  \begin{phonetics}{如果}{ru2guo3}[][HSK 2]
    \definition{conj.}{se; no caso de; na eventualidade de; supondo que; para expressar suposições, pode-se usar 要是 na linguagem falada.}
  \seealsoref{要是}{yao4shi5}
  \end{phonetics}
\end{entry}

\begin{entry}{如画}{6,8}{⼥、⽥}
  \begin{phonetics}{如画}{ru2hua4}
    \definition{adj.}{pitoresco}
  \end{phonetics}
\end{entry}

\begin{entry}{妆}{6}{⼥}
  \begin{phonetics}{妆}{zhuang1}
    \definition{s.}{maquiagem | adorno | enxoval | maquiagem e figurino de palco}
    \definition{v.}{maquiar-se | enfeitar-se}
  \end{phonetics}
\end{entry}

\begin{entry}{妆扮}{6,7}{⼥、⼿}
  \begin{phonetics}{妆扮}{zhuang1ban4}
    \variantof{装扮}
  \end{phonetics}
\end{entry}

\begin{entry}{妈}{6}{⼥}
  \begin{phonetics}{妈}{ma1}[][HSK 1]
    \definition[个,位]{s.}{mãe; mamãe | uma forma de tratamento para uma mulher casada uma geração mais velha | (antigo) uma forma de tratamento para uma empregada doméstica de meia-idade ou velha}
  \seealsoref{妈妈}{ma1 ma5}
  \end{phonetics}
\end{entry}

\begin{entry}{妈妈}{6,6}{⼥、⼥}
  \begin{phonetics}{妈妈}{ma1 ma5}[][HSK 1]
    \definition[个,位]{s.}{mamãe; mãe | uma forma de chamar uma mulher de meia-idade; títulos de respeito para mulheres mais velhas}
  \end{phonetics}
\end{entry}

\begin{entry}{字}{6}{⼦}
  \begin{phonetics}{字}{zi4}[][HSK 1]
    \definition[个]{s.}{palavra; caractere; texto | pronúncia (de uma palavra ou caractere); som do caractere | tipo de impressão; estilo de caligrafia; forma de um caractere escrito ou impresso; refere-se às diferentes formas dos caracteres chineses; também se refere às diferentes escolas de caligrafia | escritas; obras de caligrafia | recibo; compromisso por escrito; documento | nome de estilo masculino adotado aos vinte anos de idade | sobrenome | um número indicado num contador elétrico, contador de água, etc.; registrar dos números dos medidores de consumo de água e eletricidade}
    \definition{v.}{ficar noiva (nos tempos antigos)}
  \end{phonetics}
\end{entry}

\begin{entry}{字母}{6,5}{⼦、⽏}
  \begin{phonetics}{字母}{zi4mu3}[][HSK 4]
    \definition[个]{s.}{letra; letras de um alfabeto | caractere que representa uma consoante inicial (em fonologia)}
  \end{phonetics}
\end{entry}

\begin{entry}{字字珠玉}{6,6,10,5}{⼦、⼦、⽟、⽟}
  \begin{phonetics}{字字珠玉}{zi4zi4zhu1yu4}
    \definition{expr.}{cada palavra é uma jóia}
    \definition{s.}{escrita magnífica}
  \end{phonetics}
\end{entry}

\begin{entry}{字典}{6,8}{⼦、⼋}
  \begin{phonetics}{字典}{zi4 dian3}[][HSK 2]
    \definition[本,册,部]{s.}{dicionário de caracteres chineses (contendo verbetes de caracteres únicos, em contraste com 词典 que contém verbetes para palavras com um ou mais caracteres)}
  \seealsoref{词典}{ci2dian3}
  \end{phonetics}
\end{entry}

\begin{entry}{字眼}{6,11}{⼦、⽬}
  \begin{phonetics}{字眼}{zi4yan3}
    \definition[个]{s.}{palavras | redação}
  \end{phonetics}
\end{entry}

\begin{entry}{字脚}{6,11}{⼦、⾁}
  \begin{phonetics}{字脚}{zi4jiao3}
    \definition[典]{s.}{gancho no final da pincelada | serifa}
  \end{phonetics}
\end{entry}

\begin{entry}{存}{6}{⼦}
  \begin{phonetics}{存}{cun2}[][HSK 3]
    \definition{v.}{existir; viver; sobreviver | armazenar; manter | acumular; coletar | depositar | sair com; verificar | reservar; reter | permanecer em equilíbrio; estar em estoque | estimar; abrigar}
  \end{phonetics}
\end{entry}

\begin{entry}{存在}{6,6}{⼦、⼟}
  \begin{phonetics}{存在}{cun2zai4}[][HSK 3]
    \definition{s.}{existência; ser; ente; o mundo objetivo, que não depende da consciência humana para mudar, ou seja, a matéria}
    \definition{v.}{existir; ser; as coisas ocupam continuamente o tempo e o espaço; na verdade, ainda não desapareceram}
  \end{phonetics}
\end{entry}

\begin{entry}{存款}{6,12}{⼦、⽋}
  \begin{phonetics}{存款}{cun2 kuan3}[][HSK 5]
    \definition[笔]{s.}{depósito; poupança bancária}
    \definition{v.}{depositar dinheiro; colocar dinheiro no banco}
  \end{phonetics}
\end{entry}

\begin{entry}{孙}{6}{⼦}
  \begin{phonetics}{孙}{sun1}
    \definition*{s.}{Sobrenome Sun}
    \definition{s.}{neto; neta | gerações abaixo da do neto | parentes pertencentes à geração do neto | segundo crescimento das plantas}
  \end{phonetics}
\end{entry}

\begin{entry}{孙女}{6,3}{⼦、⼥}
  \begin{phonetics}{孙女}{sun1nv3}[][HSK 4]
    \definition{s.}{filha do filho; neta}
  \end{phonetics}
\end{entry}

\begin{entry}{孙子}{6,3}{⼦、⼦}
  \begin{phonetics}{孙子}{sun1zi3}
    \definition*{s.}{Sun Tzu, também conhecido por Sun Wu, 孙武, general, estrategista e filósofo autor do ``Arte da Guerra'', 《孙子兵法》}
  \seealsoref{孙武}{sun1wu3}
  \seealsoref{孙子兵法}{sun1zi3 bing1fa3}
  \end{phonetics}
  \begin{phonetics}{孙子}{sun1zi5}[][HSK 4]
    \definition{s.}{filho do filho; neto}
  \end{phonetics}
\end{entry}

\begin{entry}{孙子兵法}{6,3,7,8}{⼦、⼦、⼋、⽔}
  \begin{phonetics}{孙子兵法}{sun1zi3 bing1fa3}
    \definition*{s.}{``Arte da Guerra'', escrito por Sun Tzu, 孫子}
  \seealsoref{孙武}{sun1wu3}
  \seealsoref{孙子}{sun1zi3}
  \end{phonetics}
\end{entry}

\begin{entry}{孙武}{6,8}{⼦、⽌}
  \begin{phonetics}{孙武}{sun1wu3}
    \definition*{s.}{Sun Wu, também conhecido por Sun Tzu, 孙子, general, estrategista e filósofo autor do ``Arte da Guerra'', 《孙子兵法》}
  \seealsoref{孙子}{sun1zi3}
  \seealsoref{孙子兵法}{sun1zi3 bing1fa3}
  \end{phonetics}
\end{entry}

\begin{entry}{宇}{6}{⼧}
  \begin{phonetics}{宇}{yu3}
    \definition*{s.}{Sobrenome Yu}
    \definition[座,栋]{s.}{beirais; calha; casa | espaço; universo; mundo | postura; porte}
  \end{phonetics}
\end{entry}

\begin{entry}{宇宙}{6,8}{⼧、⼧}
  \begin{phonetics}{宇宙}{yu3zhou4}
    \definition{s.}{universo | cosmos}
  \end{phonetics}
\end{entry}

\begin{entry}{宇航员}{6,10,7}{⼧、⾈、⼝}
  \begin{phonetics}{宇航员}{yu3hang2yuan2}
    \definition{s.}{astronauta}
  \end{phonetics}
\end{entry}

\begin{entry}{守}{6}{⼧}
  \begin{phonetics}{守}{shou3}[][HSK 4]
    \definition*{s.}{Sobrenome Shou}
    \definition{adv.}{próximo; perto de; perto de algum lugar em posição, perto de algum lugar}
    \definition{v.}{guardar; defender; estar presente para cuidar; não ir embora | manter vigilância; defender do ataque do oponente em uma luta ou confronto | observar; cumprir; respeitar; fazer as coisas como elas devem ser feitas | manter, observar a integridade; honrar a palavra de alguém; manter a palavra de alguém}
  \end{phonetics}
\end{entry}

\begin{entry}{守门员}{6,3,7}{⼧、⾨、⼝}
  \begin{phonetics}{守门员}{shou3men2yuan2}
    \definition{s.}{goleiro}
  \end{phonetics}
\end{entry}

\begin{entry}{安}{6}{⼧}
  \begin{phonetics}{安}{an1}[][HSK 4]
    \definition*{s.}{Sobrenome An}
    \definition{adj.}{pacífico; quieto; tranquilo; calmo | seguro; protegido (oposto a 危) | com boa saúde | em paz; bem}
    \definition{adv.}{pacificamente; silenciosamente | com segurança; em segurança | em perguntas retóricas: como?}
    \definition{pron.}{usado como pronome interrogativo, como em 哪里,怎么; 谁,何,如何}
    \definition{s.}{segurança; proteção; paz | ampère; abreviação de ampère, 安培}
    \definition{v.}{tranquilizar (a mente de alguém); acalmar | contentar-se; ficar satisfeito | colocar em uma posição adequada; encontrar um lugar para | instalar; consertar; encaixar; configurar | trazer (uma acusação contra alguém); dar (a alguém um apelido); reivindicar (crédito por algo) | abrigar (uma intenção) | acalmar; estabilizar | sentir-se satisfeito e à vontade}
  \seealsoref{安培}{an1pei2}
  \seealsoref{何}{he2}
  \seealsoref{哪里}{na3 li3}
  \seealsoref{如何}{ru2he2}
  \seealsoref{谁}{shei2}
  \seealsoref{危}{wei1}
  \seealsoref{怎么}{zen3me5}
  \end{phonetics}
\end{entry}

\begin{entry}{安全}{6,6}{⼧、⼊}
  \begin{phonetics}{安全}{an1quan2}[][HSK 2]
    \definition{adj.}{seguro; protegido; sem perigo; sem ameaças; sem acidentes}
    \definition{s.}{segurança; proteção; refere-se a um estado ou conceito, geralmente indicando ausência de ameaças ou perigo}
  \end{phonetics}
\end{entry}

\begin{entry}{安神}{6,9}{⼧、⽰}
  \begin{phonetics}{安神}{an1shen2}
    \definition{v.+compl.}{acalmar os nervos | aliviar a inquietação pela tranquilização da mente e do corpo}
  \end{phonetics}
\end{entry}

\begin{entry}{安家}{6,10}{⼧、⼧}
  \begin{phonetics}{安家}{an1jia1}
    \definition{v.+compl.}{montar uma casa | estabelecer-se}
  \end{phonetics}
\end{entry}

\begin{entry}{安培}{6,11}{⼧、⼟}
  \begin{phonetics}{安培}{an1pei2}
    \definition{clas.}{A; empréstimo linguístico: ampere; física: unidade de corrente elétrica}
  \end{phonetics}
\end{entry}

\begin{entry}{安排}{6,11}{⼧、⼿}
  \begin{phonetics}{安排}{an1pai2}[][HSK 3]
    \definition{s.}{plano; programação; organização; tabela do plano de atividades ou horários}
    \definition{v.}{organizar (assuntos) de acordo com a sequência ou regras; tratar as coisas de acordo com uma determinada ordem ou regras | atribuir tarefas a alguém; colocar as pessoas nos cargos de trabalho determinados, conforme planejado}
  \end{phonetics}
\end{entry}

\begin{entry}{安检}{6,11}{⼧、⽊}
  \begin{phonetics}{安检}{an1 jian3}[][HSK 6]
    \definition{s.}{verificação de segurança}
    \definition{v.}{realizar verificação de segurança}
  \end{phonetics}
\end{entry}

\begin{entry}{安装}{6,12}{⼧、⾐}
  \begin{phonetics}{安装}{an1zhuang1}[][HSK 3]
    \definition{v.}{instalar; consertar; configurar; fixar máquinas ou equipamentos (geralmente conjuntos) em um determinado local, de acordo com métodos e especificações específicos}
  \end{phonetics}
\end{entry}

\begin{entry}{安置}{6,13}{⼧、⽹}
  \begin{phonetics}{安置}{an1zhi4}[][HSK 4]
    \definition{v.}{providenciar; encontrar um lugar para; ajudar a estabelecer-se; colocar pessoas ou coisas em uma determinada posição ou organizá-las adequadamente}
  \end{phonetics}
\end{entry}

\begin{entry}{安静}{6,14}{⼧、⾭}
  \begin{phonetics}{安静}{an1jing4}[][HSK 2]
    \definition{adj.}{silencioso; tranquilo; sem som; sem barulho e sem algazarra}
  \end{phonetics}
\end{entry}

\begin{entry}{安慰}{6,15}{⼧、⼼}
  \begin{phonetics}{安慰}{an1wei4}[][HSK 5]
    \definition{adj.}{confortar; tranquilizar; consolar; apaziguar;}
    \definition[个]{s.}{conforto; consolo; comportamento que alivia a dor de alguém e o acalma com palavras ou gestos}
    \definition{v.}{confortar; consolar; acalmar e confortar; deixar a mente tranquila}
  \end{phonetics}
\end{entry}

\begin{entry}{寺}{6}{⼨}
  \begin{phonetics}{寺}{si4}[][HSK 6]
    \definition*{s.}{Sobrenome Si}
    \definition[座]{s.}{templo | (Islã) mesquita | (datado) ministério; agência governamental na China antiga}
  \end{phonetics}
\end{entry}

\begin{entry}{寺庙}{6,8}{⼨、⼴}
  \begin{phonetics}{寺庙}{si4miao4}
    \definition{s.}{templo | mosteiro | santuário}
  \end{phonetics}
\end{entry}

\begin{entry}{寻}{6}{⼨}
  \begin{phonetics}{寻}{xun2}
    \definition*{s.}{Sobrenome Xun}
    \definition{clas.}{uma unidade antiga de comprimento, igual a 8尺}
    \definition{v.}{procurar; pesquisar; buscar}
  \seealsoref{尺}{chi3}
  \end{phonetics}
\end{entry}

\begin{entry}{寻找}{6,7}{⼨、⼿}
  \begin{phonetics}{寻找}{xun2zhao3}[][HSK 4]
    \definition{v.}{buscar; procurar; pesquisar; encontrar, que pode ser usado tanto para coisas concretas quanto para coisas abstratas}
  \end{phonetics}
\end{entry}

\begin{entry}{寻求}{6,7}{⼨、⽔}
  \begin{phonetics}{寻求}{xun2 qiu2}[][HSK 5]
    \definition{v.}{procurar; perseguir; explorar; ir em busca de}
  \end{phonetics}
\end{entry}

\begin{entry}{导}{6}{⼨}
  \begin{phonetics}{导}{dao3}
    \definition[个,位,名,些]{s.}{guia turístico | diretor}
    \definition{v.}{liderar; guiar | conduzir; transmitir | ensinar; instruir; dar orientação a}
  \end{phonetics}
\end{entry}

\begin{entry}{导致}{6,10}{⼨、⾄}
  \begin{phonetics}{导致}{dao3zhi4}[][HSK 4]
    \definition{v.}{causar; levar a; dar origem a (um resultado ruim)}
  \end{phonetics}
\end{entry}

\begin{entry}{导弹}{6,11}{⼨、⼸}
  \begin{phonetics}{导弹}{dao3dan4}
    \definition[枚]{s.}{míssil (guiado)}
  \end{phonetics}
\end{entry}

\begin{entry}{导游}{6,12}{⼨、⽔}
  \begin{phonetics}{导游}{dao3you2}[][HSK 4]
    \definition[个,位,名]{s.}{guia turístico; pessoas que trabalham como guias turísticos}
    \definition{v.}{guiar; conduzir um passeio turístico}
  \end{phonetics}
\end{entry}

\begin{entry}{导演}{6,14}{⼨、⽔}
  \begin{phonetics}{导演}{dao3yan3}[][HSK 3]
    \definition[位,名,个]{s.}{diretor; pessoa que exerce a função de diretor}
    \definition{v.}{dirigir (um filme, peça, etc.); ensaio de peças teatrais ou filmagem de filmes e séries de TV; organização e orientação do trabalho de produção}
  \end{phonetics}
\end{entry}

\begin{entry}{尖}{6}{⼩}
  \begin{phonetics}{尖}{jian1}[][HSK 6]
    \definition{adj.}{pontiagudo; afilado; agudo | agudo; estridente; penetrante | mesquinho; pão-duro | mordaz; cáustico}
    \definition{s.}{ponto; ponta; topo | o melhor do seu tipo; a melhor escolha; a nata da safra; uma pessoa ou coisa notável}
    \definition{v.}{tornar (a voz, etc.) aguda; estridente}
  \end{phonetics}
\end{entry}

\begin{entry}{尧}{6}{⼪}
  \begin{phonetics}{尧}{yao2}
    \definition*{s.}{Yao, um monarca lendário da China antiga | Sobrenome Yao}
  \end{phonetics}
\end{entry}

\begin{entry}{尽}{6}{⼫}
  \begin{phonetics}{尽}{jin3}
    \definition{adv.}{na maior extensão possível | na extremidade mais distante de | usado antes de palavras que indicam direção, o mesmo que 最 | de agora em diante}
    \definition{prep.}{dentro dos limites de}
    \definition{v.}{dar prioridade a | deixar que certas pessoas ou coisas tenham precedência}
  \seealsoref{最}{zui4}
  \end{phonetics}
  \begin{phonetics}{尽}{jin4}[][HSK 6]
    \definition*{s.}{Sobrenome Jin}
    \definition{adj.}{exausto; acabado | ao máximo; ao limite | tudo; exaustivo}
    \definition{v.}{esgotar | tentar o seu melhor; fazer o melhor uso possível | morrer; falecer | terminar | chegar ao fim ao máximo; alcançar extremos}
  \end{phonetics}
\end{entry}

\begin{entry}{尽力}{6,2}{⼫、⼒}
  \begin{phonetics}{尽力}{jin4li4}[][HSK 4]
    \definition{v.+compl.}{esforçar-se ao máximo; esforçar-se ao máximo; usar toda a sua força; fazer algo com seu melhor esforço}
  \end{phonetics}
\end{entry}

\begin{entry}{尽可能}{6,5,10}{⼫、⼝、⾁}
  \begin{phonetics}{尽可能}{jin3 ke3 neng2}[][HSK 5]
    \definition{adv.}{na medida do possível; com o melhor de sua capacidade; tentar fazer algo, atingir um determinado nível ou extensão}
  \end{phonetics}
\end{entry}

\begin{entry}{尽快}{6,7}{⼫、⼼}
  \begin{phonetics}{尽快}{jin3kuai4}[][HSK 4]
    \definition{adv.}{com toda a velocidade; o mais rápido possível; o mais breve possível}
  \end{phonetics}
\end{entry}

\begin{entry}{尽量}{6,12}{⼫、⾥}
  \begin{phonetics}{尽量}{jin3liang4}[][HSK 3]
    \definition{adv.}{tanto quanto possível; da melhor maneira possível}
  \end{phonetics}
\end{entry}

\begin{entry}{尽管}{6,14}{⼫、⽵}
  \begin{phonetics}{尽管}{jin3guan3}[][HSK 5]
    \definition{adv.}{justo; livremente; faça o que quiser, não se preocupe, não há restrições de movimento ou comportamento}
    \definition{conj.}{no entanto; embora; apesar de ; normalmente usado no início de uma frase anterior para introduzir um fato, seguido de 但是, etc. para introduzir um resultado que o fato não deveria ter; às vezes, também pode ser usado no início de uma frase posterior.}
  \seealsoref{但是}{dan4 shi4}
  \end{phonetics}
\end{entry}

\begin{entry}{岁}{6}{⼭}
  \begin{phonetics}{岁}{sui4}[][HSK 1]
    \definition{clas.}{usado para anos (de idade)}
    \definition{s.}{ano (literário) | colheita do ano (literário) | idade | tempo (literário) | ano (de idade) | ano (para as colheitas)}
  \end{phonetics}
\end{entry}

\begin{entry}{岁月}{6,4}{⼭、⽉}
  \begin{phonetics}{岁月}{sui4yue4}[][HSK 5]
    \definition{s.}{anos; ano e mês; refere-se a tempo em geral}
  \end{phonetics}
\end{entry}

\begin{entry}{岂}{6}{⼭}
  \begin{phonetics}{岂}{qi3}
    \definition*{s.}{Sobrenome Qi}
    \definition{adv.}{expressa uma pergunta retórica, equivalente a 哪里, 怎么 e 难道}
  \seealsoref{哪里}{na3 li3}
  \seealsoref{难道}{nan2dao4}
  \seealsoref{怎么}{zen3me5}
  \end{phonetics}
\end{entry}

\begin{entry}{岂有此理}{6,6,6,11}{⼭、⽉、⽌、⽟}
  \begin{phonetics}{岂有此理}{qi3you3ci3li3}
    \definition{interj.}{Que exorbitante! | Absurdo! | Como isso pode ser assim? | Ridículo!}
  \end{phonetics}
\end{entry}

\begin{entry}{巡}{6}{⾡}
  \begin{phonetics}{巡}{xun2}
    \definition{clas.}{rodada de bebidas | usado para servir vinho a todos}
    \definition{v.}{patrulhar; fazer rondas; fazer uma excursão de inspeção}
  \end{phonetics}
\end{entry}

\begin{entry}{巡逻}{6,11}{⾡、⾡}
  \begin{phonetics}{巡逻}{xun2luo2}
    \definition{s.}{patrulha}
    \definition{v.}{patrulhar (polícia, exército ou marinha)}
  \end{phonetics}
\end{entry}

\begin{entry}{师}{6}{⼱}
  \begin{phonetics}{师}{shi1}
    \definition*{s.}{Sobrenome Shi}
    \definition{s.}{professor | mestre | especialista | modelo | divisão do exército}
    \definition{v.}{despachar tropas}
  \end{phonetics}
\end{entry}

\begin{entry}{师傅}{6,12}{⼱、⼈}
  \begin{phonetics}{师傅}{shi1fu5}[][HSK 5]
    \definition[个,位,名]{s.}{mestre; um trabalhador qualificado; título honorífico para pessoas habilidosas | mestre; professor (em certos ofícios); pessoas que ensinam técnicas em áreas como engenharia, comércio e teatro}
  \end{phonetics}
\end{entry}

\begin{entry}{年}{6}{⼲}
  \begin{phonetics}{年}{nian2}[][HSK 1]
    \definition*{s.}{Sobrenome Nian}
    \definition{clas.}{ano; usado para calcular o número de anos}
    \definition{s.}{ano | idade | um período (época) da história | colheita anual | Ano Novo | artigos para o dia de Ano Novo | um período da vida de uma pessoa; fases da vida humana divididas por idade}
  \end{phonetics}
\end{entry}

\begin{entry}{年代}{6,5}{⼲、⼈}
  \begin{phonetics}{年代}{nian2dai4}[][HSK 3]
    \definition[个]{s.}{idade; anos; tempo; um período de tempo com características distintas na história | uma década de um século; período de dez anos}
  \end{phonetics}
\end{entry}

\begin{entry}{年级}{6,6}{⼲、⽷}
  \begin{phonetics}{年级}{nian2ji2}[][HSK 2]
    \definition[个]{s.}{série; ano; níveis divididos de acordo com o tempo de estudo dos alunos na escola}
  \end{phonetics}
\end{entry}

\begin{entry}{年纪}{6,6}{⼲、⽷}
  \begin{phonetics}{年纪}{nian2ji4}[][HSK 3]
    \definition[把,个]{s.}{idade (de uma pessoa)}
  \end{phonetics}
\end{entry}

\begin{entry}{年初}{6,7}{⼲、⾐}
  \begin{phonetics}{年初}{nian2 chu1}[][HSK 3]
    \definition{s.}{o começo do ano; os primeiros dias do ano}
  \end{phonetics}
\end{entry}

\begin{entry}{年底}{6,8}{⼲、⼴}
  \begin{phonetics}{年底}{nian2 di3}[][HSK 3]
    \definition[个]{s.}{fim de ano; o fim do ano; geralmente os últimos dias de dezembro ou o fim do ano}
  \end{phonetics}
\end{entry}

\begin{entry}{年货}{6,8}{⼲、⾙}
  \begin{phonetics}{年货}{nian2huo4}
    \definition{s.}{mercadorias vendidas no Ano Novo Chinês}
  \end{phonetics}
\end{entry}

\begin{entry}{年前}{6,9}{⼲、⼑}
  \begin{phonetics}{年前}{nian2 qian2}[][HSK 5]
    \definition{s.}{antes do final do ano; antes do ano novo}
  \end{phonetics}
\end{entry}

\begin{entry}{年度}{6,9}{⼲、⼴}
  \begin{phonetics}{年度}{nian2du4}[][HSK 5]
    \definition{s.}{ano; de acordo com a natureza e as necessidades de um negócio, há um prazo de doze meses com data de início e término definidas}
  \end{phonetics}
\end{entry}

\begin{entry}{年轻}{6,9}{⼲、⾞}
  \begin{phonetics}{年轻}{nian2qing1}[][HSK 2]
    \definition{adj.}{jovem; não muito velho (geralmente se refere a pessoas entre 10 e 20 anos)}
  \end{phonetics}
\end{entry}

\begin{entry}{年龄}{6,13}{⼲、⿒}
  \begin{phonetics}{年龄}{nian2ling2}[][HSK 5]
    \definition[个]{s.}{idade; animais, plantas e outros seres vivos vivem e crescem no mundo durante um determinado número de anos}
  \end{phonetics}
\end{entry}

\begin{entry}{并}{6}{⼲}
  \begin{phonetics}{并}{bing4}[][HSK 3,4]
    \definition{adv.}{lado a lado; igualmente; simultaneamente | (usado para reforçar uma negação) na verdade; definitivamente | mesmo assim | (usado para reforçar uma negação) na verdade; de forma alguma | todos; indica o conjunto completo, equivalente a 全部}
    \definition{conj.}{e; além disso}
    \definition{v.}{combinar; fundir; incorporar | ficar (ou colocar) lado a lado | estar paralelo a | anexar; juntar}
  \seealsoref{全部}{quan2bu4}
  \end{phonetics}
\end{entry}

\begin{entry}{并且}{6,5}{⼲、⼀}
  \begin{phonetics}{并且}{bing4qie3}[][HSK 3]
    \definition{conj.}{e; bem como; usado entre verbos, adjetivos ou frases paralelas para indicar que várias ações são realizadas ao mesmo tempo ou que propriedades existem ao mesmo tempo | além disso; além do mais; ademais; usado na segunda metade de uma frase complexa para expressar um significado adicional}
  \end{phonetics}
\end{entry}

\begin{entry}{并排}{6,11}{⼲、⼿}
  \begin{phonetics}{并排}{bing4pai2}
    \definition{adv.}{lado a lado}
  \end{phonetics}
\end{entry}

\begin{entry}{庆}{6}{⼴}
  \begin{phonetics}{庆}{qing4}
    \definition*{s.}{Sobrenome Qing}
    \definition{s.}{celebração | ocasião para celebração; um aniversário que vale a pena comemorar}
    \definition{v.}{celebrar; felicitar; comemorar}
  \end{phonetics}
\end{entry}

\begin{entry}{庆祝}{6,9}{⼴、⽰}
  \begin{phonetics}{庆祝}{qing4zhu4}[][HSK 3]
    \definition{v.}{celebrar; comemorar; festejar; realizar atividades para comemorar ou celebrar festivais comuns e eventos felizes}
  \end{phonetics}
\end{entry}

\begin{entry}{延}{6}{⼵}
  \begin{phonetics}{延}{yan2}
    \definition*{s.}{Sobrenome Yan}
    \definition{v.}{prolongar; estender; alongar | adiar; atrasar | envolver (um professor, conselheiro, etc.); enviar para; convidar}
  \end{phonetics}
\end{entry}

\begin{entry}{延长}{6,4}{⼵、⾧}
  \begin{phonetics}{延长}{yan2chang2}[][HSK 4]
    \definition{v.}{estender; prolongar; alongar; aumentar o tempo, a distância ou a duração de algo específico}
  \end{phonetics}
\end{entry}

\begin{entry}{延伸}{6,7}{⼵、⼈}
  \begin{phonetics}{延伸}{yan2shen1}[][HSK 5]
    \definition{v.}{estender; esticar; alongar; estender-se}
  \end{phonetics}
\end{entry}

\begin{entry}{延续}{6,11}{⼵、⽷}
  \begin{phonetics}{延续}{yan2xu4}[][HSK 4]
    \definition{v.}{durar; continuar; prosseguir; continuar como antes; prolongar}
  \end{phonetics}
\end{entry}

\begin{entry}{延期}{6,12}{⼵、⽉}
  \begin{phonetics}{延期}{yan2qi1}[][HSK 4]
    \definition{v.+compl.}{atrasar; adiar; postergar}
  \end{phonetics}
\end{entry}

\begin{entry}{异}{6}{⼶}
  \begin{phonetics}{异}{yi4}
    \definition{adj.}{diferente | estranho; incomum; extraordinário; especial | outro}
    \definition{v.}{surpreender | separar; divorciar-se}
  \end{phonetics}
\end{entry}

\begin{entry}{异常}{6,11}{⼶、⼱}
  \begin{phonetics}{异常}{yi4chang2}
    \definition{adj.}{extraordinário | anormal}
    \definition{adv.}{extraordinariamente | excepcionalmente}
    \definition{s.}{anormalidade}
  \end{phonetics}
\end{entry}

\begin{entry}{式}{6}{⼷}
  \begin{phonetics}{式}{shi4}[][HSK 5]
    \definition*{s.}{Sobrenome Shi}
    \definition{s.}{tipo; estilo | forma; padrão | ritual; cerimônia | fórmula; conjunto de símbolos que expressam uma lei natural na ciência natural | humor; modo; categoria gramatical que expressa a atitude subjetiva do falante em relação ao que está sendo dito, como narrativa, imperativa e condicional}
  \end{phonetics}
\end{entry}

\begin{entry}{当}{6}{⼹}
  \begin{phonetics}{当}{dang1}[][HSK 2,6]
    \definition*{s.}{Sobrenome Dang}
    \definition{adj.}{igual; adequado; compatível}
    \definition{prep.}{na presença de alguém; na cara de alguém | exatamente em (um momento ou lugar); em algum momento, em algum lugar | na frente de alguém}
    \definition{s.}{topo; cume |uma lacuna no espaço ou no tempo; refere-se a um espaço ou intervalo de tempo}
    \definition{s.}{Onomatopéia: barulho metálico, som de um gongo ou sino}
    \definition{v.}{dever; ter que; dever ser | trabalhar como; servir como; ser; assumir; desempenhar a função de | suportar; aceitar; merecer | dirigir; gerenciar; estar no comando; ser responsável por;  presidir | conter; bloquear; segurar; reter; resistir}
  \end{phonetics}
  \begin{phonetics}{当}{dang4}
    \definition{adj.}{adequado; correto; apropriado | igual; o mesmo}
    \definition{pron.}{naquele mesmo (dia, etc.); refere-se ao momento em que algo aconteceu}
    \definition{s.}{algo penhorado; penhor; garantia; objetos físicos penhorados em casas de penhores}
    \definition{v.}{corresponder; ser igual a; combinar | tratar como; considerar como; tomar como | pensar que; achar que | penhorar; empréstimo com garantia real em uma loja de penhores}
  \end{phonetics}
\end{entry}

\begin{entry}{当中}{6,4}{⼹、⼁}
  \begin{phonetics}{当中}{dang1 zhong1}[][HSK 3]
    \definition{prep.}{no meio; no centro | entre; dentro}
  \end{phonetics}
\end{entry}

\begin{entry}{当天}{6,4}{⼹、⼤}
  \begin{phonetics}{当天}{dan1 tian1}[][HSK 6]
    \definition{s.}{no mesmo dia; naquele mesmo dia; refere-se ao dia em que algo aconteceu no passado}
  \end{phonetics}
\end{entry}

\begin{entry}{当代}{6,5}{⼹、⼈}
  \begin{phonetics}{当代}{dang1dai4}[][HSK 5]
    \definition{s.}{a era atual; a era contemporânea}
  \end{phonetics}
\end{entry}

\begin{entry}{当地}{6,6}{⼹、⼟}
  \begin{phonetics}{当地}{dang1di4}
    \definition{s.}{local; o lugar onde as pessoas e as coisas estão ou onde as coisas acontecem}
  \end{phonetics}
\end{entry}

\begin{entry}{当场}{6,6}{⼹、⼟}
  \begin{phonetics}{当场}{dang1chang3}[][HSK 5]
    \definition{adv.}{na hora; de imediato; na mesma hora}
  \end{phonetics}
\end{entry}

\begin{entry}{当年}{6,6}{⼹、⼲}
  \begin{phonetics}{当年}{dang1 nian2}[][HSK 5]
    \definition{adv.}{durante esse período; durante esse tempo | naquela época; naqueles dias | naqueles anos | nessa ocasião}
  \end{phonetics}
  \begin{phonetics}{当年}{dang4 nian2}
    \definition{s.}{no mesmo ano; naquele mesmo ano}
  \end{phonetics}
\end{entry}

\begin{entry}{当成}{6,6}{⼹、⼽}
  \begin{phonetics}{当成}{dan4 cheng2}[][HSK 6]
    \definition{v.}{considerar como; tratar como; tomar por}
  \end{phonetics}
\end{entry}

\begin{entry}{当作}{6,7}{⼹、⼈}
  \begin{phonetics}{当作}{dang4 zuo4}[][HSK 6]
    \definition{v.}{tratar como; considerar como}
  \end{phonetics}
\end{entry}

\begin{entry}{当初}{6,7}{⼹、⾐}
  \begin{phonetics}{当初}{dang1chu1}[][HSK 3]
    \definition{s.}{no começo; originalmente; no início; em primeiro lugar; refere-se a algo que aconteceu no passado, seja em geral ou especificamente}
  \end{phonetics}
\end{entry}

\begin{entry}{当时}{6,7}{⼹、⽇}
  \begin{phonetics}{当时}{dang1shi2}[][HSK 2]
    \definition{s.}{naquela época; aquela ocasião; aquela vez; refere-se a algo que aconteceu no passado}
    \definition{v.}{ser o momento adequado; acontecer no momento certo}
  \end{phonetics}
  \begin{phonetics}{当时}{dang4shi2}
    \definition{adv.}{(depois de fazer algo ou algo acontecer) imediatamente; de imediato; agora mesmo}
  \end{phonetics}
\end{entry}

\begin{entry}{当前}{6,9}{⼹、⼑}
  \begin{phonetics}{当前}{dang1qian2}[][HSK 5]
    \definition{s.}{presente; atual}
    \definition{v.}{estar diante de alguém; estar frente a frente com alguém; na frente de, geralmente refere-se a uma situação perigosa}
  \end{phonetics}
\end{entry}

\begin{entry}{当选}{6,9}{⼹、⾡}
  \begin{phonetics}{当选}{dang1xuan3}[][HSK 5]
    \definition{v.}{ser eleito}
  \end{phonetics}
\end{entry}

\begin{entry}{当然}{6,12}{⼹、⽕}
  \begin{phonetics}{当然}{dang1ran2}[][HSK 3]
    \definition{adj.}{natural; verdadeiro; espontâneo}
    \definition{adv.}{sem dúvida; certamente; claro}
  \end{phonetics}
\end{entry}

\begin{entry}{忙}{6}{⼼}
  \begin{phonetics}{忙}{mang2}[][HSK 1]
    \definition*{s.}{Sobrenome Mang}
    \definition{adj.}{ocupado; movimentado; totalmente ocupado; muitas coisas para fazer, sem tempo livre (oposto de 闲) | imperativo; ansioso; urgente}
    \definition{v.}{apressar-se; agitar-se; fazer algo com urgência e constantemente | trabalhar; fazer}
  \seealsoref{闲}{xian2}
  \end{phonetics}
\end{entry}

\begin{entry}{戏}{6}{⼽}
  \begin{phonetics}{戏}{xi4}[][HSK 5]
    \definition*{s.}{Sobrenome Xi}
    \definition[场,部,出,台]{s.}{drama; peça; espetáculo; \emph{show}}
    \definition{v.}{brincar; praticar esportes; jogar | zombar; brincar; provocar}
  \end{phonetics}
\end{entry}

\begin{entry}{戏弄}{6,7}{⼽、⼶}
  \begin{phonetics}{戏弄}{xi4nong4}
    \definition{v.}{zombar de | pregar peças | provocar}
  \end{phonetics}
\end{entry}

\begin{entry}{戏法}{6,8}{⼽、⽔}
  \begin{phonetics}{戏法}{xi4fa3}
    \definition{s.}{truque de mágica | prestidigitação}
  \end{phonetics}
\end{entry}

\begin{entry}{戏耍}{6,9}{⼽、⽽}
  \begin{phonetics}{戏耍}{xi4shua3}
    \definition{v.}{divertir-me | brincar com | provocar}
  \end{phonetics}
\end{entry}

\begin{entry}{戏院}{6,9}{⼽、⾩}
  \begin{phonetics}{戏院}{xi4yuan4}
    \definition{s.}{teatro}
  \end{phonetics}
\end{entry}

\begin{entry}{戏剧}{6,10}{⼽、⼑}
  \begin{phonetics}{戏剧}{xi4ju4}[][HSK 5]
    \definition{s.}{drama; peça; teatro | roteiro; peça; cenário}
  \end{phonetics}
\end{entry}

\begin{entry}{戏剧化地}{6,10,4,6}{⼽、⼑、⼔、⼟}
  \begin{phonetics}{戏剧化地}{xi4ju4hua4di4}
    \definition{adv.}{dramaticamente | teatralmente}
  \end{phonetics}
\end{entry}

\begin{entry}{戏剧性}{6,10,8}{⼽、⼑、⼼}
  \begin{phonetics}{戏剧性}{xi4ju4xing4}
    \definition{adj.}{dramático}
  \end{phonetics}
\end{entry}

\begin{entry}{戏剧家}{6,10,10}{⼽、⼑、⼧}
  \begin{phonetics}{戏剧家}{xi4ju4jia1}
    \definition{s.}{dramaturgo}
  \end{phonetics}
\end{entry}

\begin{entry}{戏剧效果}{6,10,10,8}{⼽、⼑、⽁、⽊}
  \begin{phonetics}{戏剧效果}{xi4ju4xiao4guo3}
    \definition{s.}{efeito dramático}
  \end{phonetics}
\end{entry}

\begin{entry}{戏剧般}{6,10,10}{⼽、⼑、⾈}
  \begin{phonetics}{戏剧般}{xi4ju4ban1}
    \definition{adj.}{melodramático}
  \end{phonetics}
\end{entry}

\begin{entry}{戏剧编剧}{6,10,12,10}{⼽、⼑、⽷、⼑}
  \begin{phonetics}{戏剧编剧}{xi4ju4bian1ju4}
    \definition{s.}{dramaturgo}
  \end{phonetics}
\end{entry}

\begin{entry}{戏剧演出}{6,10,14,5}{⼽、⼑、⽔、⼐}
  \begin{phonetics}{戏剧演出}{xi4ju4yan3chu1}
    \definition{s.}{performance dramática}
  \end{phonetics}
\end{entry}

\begin{entry}{戏谑}{6,11}{⼽、⾔}
  \begin{phonetics}{戏谑}{xi4xue4}
    \definition{v.}{brincar | fazer piadas | ridicularizar}
  \end{phonetics}
\end{entry}

\begin{entry}{成}{6}{⼽}
  \begin{phonetics}{成}{cheng2}[][HSK 2,6]
    \definition*{s.}{Sobrenome Cheng}
    \definition{adj.}{capaz; competente | totalmente crescido; totalmente desenvolvido; maduro | estabelecido; Já definido; pronto para uso | em números ou quantidades consideráveis; inteiro; suficiente: enfatiza a quantidade ou a duração}
    \definition{clas.}{um décimo}
    \definition{interj.}{O.K.; tudo bem}
    \definition{s.}{resultado; conquista}
    \definition{v.}{ter sucesso; conseguir; ser bem-sucedido | tornar-se; transformar-se | ajudar a completar; realizar}
  \end{phonetics}
\end{entry}

\begin{entry}{成人}{6,2}{⼽、⼈}
  \begin{phonetics}{成人}{cheng2ren2}[][HSK 4]
    \definition[个]{s.}{adulto; crescido; pessoa adulta}
    \definition{v.}{crescer; tornar-se adulto}
  \end{phonetics}
\end{entry}

\begin{entry}{成为}{6,4}{⼽、⼂}
  \begin{phonetics}{成为}{cheng2wei2}[][HSK 2]
    \definition{v.}{tornar-se; transformar-se; revelar-se; passar de uma situação, identidade ou estado para outro}
  \end{phonetics}
\end{entry}

\begin{entry}{成分}{6,4}{⼽、⼑}
  \begin{phonetics}{成分}{cheng2fen4}[][HSK 6]
    \definition[个,些,种]{s.}{composição; ingrediente; elemento; parte componente; as várias substâncias ou fatores que compõem as coisas | a condição de classe de alguém; a profissão ou a condição econômica de alguém; refere-se à classe à qual uma família pertence; à principal experiência ou ocupação anterior de uma pessoa}
  \end{phonetics}
\end{entry}

\begin{entry}{成长}{6,4}{⼽、⾧}
  \begin{phonetics}{成长}{cheng2zhang3}[][HSK 3]
    \definition{v.}{crescer; amadurecer; tornar-se adulto; o desenvolvimento de seres humanos, animais ou plantas desde a infância até a maturidade}
  \end{phonetics}
\end{entry}

\begin{entry}{成功}{6,5}{⼽、⼒}
  \begin{phonetics}{成功}{cheng2gong1}[][HSK 3]
    \definition{adj.}{bem-sucedido; frutífero}
    \definition[个,次]{s.}{sucesso}
    \definition{v.}{ter sucesso; obter os resultados esperados}
  \end{phonetics}
\end{entry}

\begin{entry}{成本}{6,5}{⼽、⽊}
  \begin{phonetics}{成本}{cheng2ben3}[][HSK 5]
    \definition{s.}{custo principal; custo; custo capitalizado; custo final; primeiro custo; custo próprio; custo de produção de um produto}
  \end{phonetics}
\end{entry}

\begin{entry}{成立}{6,5}{⼽、⽴}
  \begin{phonetics}{成立}{cheng2li4}[][HSK 3]
    \definition{v.}{fundar; estabelecer; criar; (organizações, instituições, etc.) começar a existir e a funcionar | ser válido; ser sustentável; fazer sentido; (teorias, pontos de vista, razões, etc.) fundamentados e válidos}
  \end{phonetics}
\end{entry}

\begin{entry}{成交}{6,6}{⼽、⼇}
  \begin{phonetics}{成交}{cheng2jiao1}[][HSK 5]
    \definition{v.+compl.}{fechar um acordo; fazer uma barganha; concluir uma transação}
  \end{phonetics}
\end{entry}

\begin{entry}{成吉思汗}{6,6,9,6}{⼽、⼝、⼼、⽔}
  \begin{phonetics}{成吉思汗}{cheng2ji2si1han2}
    \definition*{s.}{Genghis Khan (1162-1227), fundador e governante do Império Mongol}
  \end{phonetics}
\end{entry}

\begin{entry}{成色}{6,6}{⼽、⾊}
  \begin{phonetics}{成色}{cheng2se4}
    \definition{v.}{sair-se bem | ser bem sucedido}
  \end{phonetics}
\end{entry}

\begin{entry}{成员}{6,7}{⼽、⼝}
  \begin{phonetics}{成员}{cheng2yuan2}[][HSK 3]
    \definition[个,些,名,位]{s.}{membro; membros de um grupo ou família}
  \end{phonetics}
\end{entry}

\begin{entry}{成批}{6,7}{⼽、⼿}
  \begin{phonetics}{成批}{cheng2pi1}
    \definition{s.}{em lotes | a granel}
  \end{phonetics}
\end{entry}

\begin{entry}{成果}{6,8}{⼽、⽊}
  \begin{phonetics}{成果}{cheng2guo3}[][HSK 3]
    \definition[个]{s.}{realização; resultado; conquista; recompensas no trabalho ou na carreira}
  \end{phonetics}
\end{entry}

\begin{entry}{成品}{6,9}{⼽、⼝}
  \begin{phonetics}{成品}{cheng2 pin3}[][HSK 6]
    \definition[批]{s.}{produto final; produto acabado; produto processado e pronto para ser fornecido}
  \end{phonetics}
\end{entry}

\begin{entry}{成活}{6,9}{⼽、⽔}
  \begin{phonetics}{成活}{cheng2huo2}
    \definition{v.}{sobreviver}
  \end{phonetics}
\end{entry}

\begin{entry}{成语}{6,9}{⼽、⾔}
  \begin{phonetics}{成语}{cheng2yu3}[][HSK 5]
    \definition{s.}{expressão idiomática; frase de conjunto (frases de quatro caracteres em chinês, geralmente com alusões literárias)}
  \end{phonetics}
\end{entry}

\begin{entry}{成家}{6,10}{⼽、⼧}
  \begin{phonetics}{成家}{cheng2jia1}
    \definition{v.}{tornar-se um especialista reconhecido | estabelecer-se e casar-se (de um homem)}
  \end{phonetics}
\end{entry}

\begin{entry}{成效}{6,10}{⼽、⽁}
  \begin{phonetics}{成效}{cheng2xiao4}[][HSK 5]
    \definition{s.}{efeito; resultado}
  \end{phonetics}
\end{entry}

\begin{entry}{成都}{6,10}{⼽、⾢}
  \begin{phonetics}{成都}{cheng2du1}
    \definition*{s.}{Chengdu}
  \end{phonetics}
\end{entry}

\begin{entry}{成婚}{6,11}{⼽、⼥}
  \begin{phonetics}{成婚}{cheng2hun1}
    \definition{v.}{casar-se}
  \end{phonetics}
\end{entry}

\begin{entry}{成绩}{6,11}{⼽、⽷}
  \begin{phonetics}{成绩}{cheng2ji4}[][HSK 2]
    \definition[项,个]{s.}{realização; sucesso; resultado (de trabalho ou estudo); refere-se à pontuação obtida em exames e competições; classificação, também se refere aos resultados alcançados no trabalho}
  \end{phonetics}
\end{entry}

\begin{entry}{成就}{6,12}{⼽、⼪}
  \begin{phonetics}{成就}{cheng2jiu4}[][HSK 3]
    \definition[个,项]{s.}{realização; conquista; sucesso; realizações profissionais}
    \definition{v.}{realizar; alcançar; completar; concluir (carreira)}
  \end{phonetics}
\end{entry}

\begin{entry}{成熟}{6,15}{⼽、⽕}
  \begin{phonetics}{成熟}{cheng2shu2}[][HSK 3]
    \definition{adj.}{maduro; amadurecido; totalmente desenvolvido; descreve que as oportunidades, condições, etc. estão perfeitas e que não haverá nenhum problema}
    \definition{v.}{amadurecer; atingir a maturidade; estar totalmente desenvolvido; frutas e outros frutos totalmente maduros, referindo-se ao desenvolvimento completo de organismos vivos}
  \end{phonetics}
\end{entry}

\begin{entry}{成器}{6,16}{⼽、⼝}
  \begin{phonetics}{成器}{cheng2qi4}
    \definition{v.}{tornar-se uma pessoa digna de respeito | fazer algo de si mesmo}
  \end{phonetics}
\end{entry}

\begin{entry}{托}{6}{⼿}
  \begin{phonetics}{托}{tuo1}[][HSK 6]
    \definition{clas.}{torr, uma unidade de pressão, 1 torr é igual à pressão de 1 mmHg, ou 133,322 Pa}
    \definition{s.}{algo servindo como suporte | fantoche; cúmplice; pessoas que ajudam golpistas a enganar outras pessoas}
    \definition{v.}{segurar na palma; apoiar com a mão ou palma; suportar (um objeto) com um objeto ou com a palma da mão | destacar; servir como contraste | pedir; confiar | implorar; dar como pretexto | dever a; confiar em}
  \end{phonetics}
\end{entry}

\begin{entry}{扛}{6}{⼿}
  \begin{phonetics}{扛}{gang1}
    \definition{v.}{levantar com as duas mãos | carregar alguma coisa juntos (duas ou mais pessoas)}
  \end{phonetics}
  \begin{phonetics}{扛}{kang2}
    \definition{v.}{carregar objetos nos ombros |  suportar; aguentar | lidar; assumir}
  \end{phonetics}
\end{entry}

\begin{entry}{扣}{6}{⼿}
  \begin{phonetics}{扣}{kou4}[][HSK 6]
    \definition*{s.}{Sobrenome Kou}
    \definition{clas.}{giro; volta; uma volta de uma rosca}
    \definition[个,颗,粒]{s.}{nó | fivela; botão | círculo de rosca (em um parafuso)}
    \definition{v.}{fivela; abotoar; amarrar ou prender com um laço ou anel | colocar uma xícara, tigela etc. de cabeça para baixo; cobrir com uma xícara, tigela etc. invertida; colocar a boca do recipiente para baixo | deter; prender; levar sob custódia | cravar; esmagar (a bola); arremessar ou bater (em uma bola) com força de cima para baixo | atracar; deduzir; descontar; subtrair uma parte do valor original | puxar; pressionar | impor; marcar sem fundamento; acusar injustamente; impor ou atribuir (um crime ou má fama) a alguém}
  \end{phonetics}
\end{entry}

\begin{entry}{执}{6}{⼿}
  \begin{phonetics}{执}{zhi2}
    \definition*{s.}{Sobrenome Zhi}
    \definition[期]{s.}{reconhecimento por escrito | (literário) amigo íntimo (ou do peito)}
    \definition{v.}{segurar; agarrar; pegar; capturar | assumir o comando de; dirigir; gerenciar; controlar; administrar; exercer | manter (os próprios pontos de vista, etc.); persistir; persistir em; manter-se em; insistir em | realizar; executar; implementar}
  \end{phonetics}
\end{entry}

\begin{entry}{执行}{6,6}{⼿、⾏}
  \begin{phonetics}{执行}{zhi2xing2}[][HSK 5]
    \definition{v.}{executar; implementar; realizar}
  \end{phonetics}
\end{entry}

\begin{entry}{执着}{6,11}{⼿、⽬}
  \begin{phonetics}{执着}{zhi2zhuo2}
    \definition{s.}{(budismo) apego}
    \definition{v.}{estar fortemente apegado a | ser dedicado | apegar-se a}
  \end{phonetics}
\end{entry}

\begin{entry}{扩}{6}{⼿}
  \begin{phonetics}{扩}{kuo4}
    \definition{v.}{expandir; ampliar; estender; alargar}
  \end{phonetics}
\end{entry}

\begin{entry}{扩大}{6,3}{⼿、⼤}
  \begin{phonetics}{扩大}{kuo4da4}[][HSK 4]
    \definition{v.}{ampliar; expandir; estender; alargar}
  \end{phonetics}
\end{entry}

\begin{entry}{扩展}{6,10}{⼿、⼫}
  \begin{phonetics}{扩展}{kuo4 zhan3}[][HSK 4]
    \definition{v.}{esticar; expandir; estender; espalhar}
  \end{phonetics}
\end{entry}

\begin{entry}{扫}{6}{⼿}
  \begin{phonetics}{扫}{sao3}[][HSK 4]
    \definition{v.}{varrer; limpar | passar rapidamente ao longo ou sobre; varrer | juntar tudo}
  \end{phonetics}
  \begin{phonetics}{扫}{sao4}
  \seealsoref{扫帚}{sao4zhou5}
  \end{phonetics}
\end{entry}

\begin{entry}{扫兴}{6,6}{⼿、⼋}
  \begin{phonetics}{扫兴}{sao3xing4}
    \definition{v.+compl.}{sentir-se decepcionado | entristecer alguém}
  \end{phonetics}
\end{entry}

\begin{entry}{扫帚}{6,8}{⼿、⼱}
  \begin{phonetics}{扫帚}{sao4zhou5}
    \definition[把]{s.}{vassoura; ferramenta de varredura feita de varas de bambu, etc., maior que uma vassora}
  \end{phonetics}
\end{entry}

\begin{entry}{扬}{6}{⼿}
  \begin{phonetics}{扬}{yang2}
    \definition*{s.}{Yangzhou, abreviação de 扬州 | Sobrenome Yang}
    \definition{v.}{levantar | separar e espalhar; peneirar | espalhar; fazer conhecido}
  \seealsoref{扬州}{yang2zhou1}
  \end{phonetics}
\end{entry}

\begin{entry}{扬州}{6,6}{⼿、⼮}
  \begin{phonetics}{扬州}{yang2zhou1}
    \definition*{s.}{Yangzhou, uma cidade na província de Jiangsu}
  \end{phonetics}
\end{entry}

\begin{entry}{扬雄}{6,12}{⼿、⾫}
  \begin{phonetics}{扬雄}{yang2xiong2}
    \definition*{s.}{Yang Xiong (53 AC-18 DC), estudioso, poeta e lexicógrafo, autor do primeiro dicionário de dialeto chinês 方言}
  \seealsoref{方言}{fang1yan2}
  \end{phonetics}
\end{entry}

\begin{entry}{收}{6}{⽁}
  \begin{phonetics}{收}{shou1}[][HSK 2]
    \definition{expr.}{aos cuidados de (usado na linha de endereço após o nome)}
    \definition{v.}{recolocar; juntar; reunir e juntar coisas espalhadas ou dispersas | recolher; cobrar | ganhar; obter (benefícios econômicos) | colher; recolher; colher ou cortar frutas, legumes, cereais maduros, etc. | aceitar; receber; acolher | controlar; restringir; restringir, controlar os sentimentos ou ações, para voltar ao estado normal | finalizar; parar; concluir; encerrar | prender; deter; colocar sob custódia}
  \end{phonetics}
\end{entry}

\begin{entry}{收入}{6,2}{⽁、⼊}
  \begin{phonetics}{收入}{shou1ru4}[][HSK 2]
    \definition[笔,个]{s.}{renda; salário; dinheiro recebido}
    \definition{v.}{receber dinheiro | coletar; receber}
  \end{phonetics}
\end{entry}

\begin{entry}{收买}{6,6}{⽁、⼄}
  \begin{phonetics}{收买}{shou1mai3}
    \definition{v.}{subornar | comprar}
  \end{phonetics}
\end{entry}

\begin{entry}{收回}{6,6}{⽁、⼞}
  \begin{phonetics}{收回}{shou1 hui2}[][HSK 4]
    \definition{v.}{retomar; recuperar; relembrar; recordar; receber de volta o que foi enviado ou emprestado, ou o dinheiro que foi emprestado ou usado | sacar; retirar; recolher; rescindir; cancelar (uma opinião, ordem, etc.)}
  \end{phonetics}
\end{entry}

\begin{entry}{收听}{6,7}{⽁、⼝}
  \begin{phonetics}{收听}{shou1 ting1}[][HSK 3]
    \definition{v.}{ouvir (rádio)}
  \end{phonetics}
\end{entry}

\begin{entry}{收到}{6,8}{⽁、⼑}
  \begin{phonetics}{收到}{shou1 dao4}[][HSK 2]
    \definition{v.}{conseguir; obter; receber; alcançar}
  \end{phonetics}
\end{entry}

\begin{entry}{收购}{6,8}{⽁、⾙}
  \begin{phonetics}{收购}{shou1 gou4}[][HSK 5]
    \definition{v.}{comprar; adquirir; comprar muito em vários lugares | adquirir uma empresa; obter o controle efetivo de uma empresa por meio de dinheiro, transações de ações, etc.}
  \end{phonetics}
\end{entry}

\begin{entry}{收拾}{6,9}{⽁、⼿}
  \begin{phonetics}{收拾}{shou1shi5}[][HSK 5]
    \definition{v.}{arrumar; empacotar; limpar; organizar, policiar, restaurar a normalidade em situações adversas | consertar; reparar; restaurar algo que está danificado ao seu estado ou função original |  punir; punir alguém, geralmente com medidas mais severas | matar}
  \end{phonetics}
\end{entry}

\begin{entry}{收看}{6,9}{⽁、⽬}
  \begin{phonetics}{收看}{shou1 kan4}[][HSK 3]
    \definition{v.}{assistir (a um programa de TV)}
  \end{phonetics}
\end{entry}

\begin{entry}{收费}{6,9}{⽁、⾙}
  \begin{phonetics}{收费}{shou1 fei4}[][HSK 3]
    \definition{v.}{cobrar; cobrar taxas}
  \end{phonetics}
\end{entry}

\begin{entry}{收音机}{6,9,6}{⽁、⾳、⽊}
  \begin{phonetics}{收音机}{shou1yin1ji1}[][HSK 3]
    \definition[部,台]{s.}{rádio; sem fio; um termo geral para receptores de rádio}
  \end{phonetics}
\end{entry}

\begin{entry}{收益}{6,10}{⽁、⽫}
  \begin{phonetics}{收益}{shou1yi4}[][HSK 4]
    \definition{s.}{lucro; renda; benefício; ganhos; vantagens ou benefícios obtidos}
  \end{phonetics}
\end{entry}

\begin{entry}{收获}{6,10}{⽁、⾋}
  \begin{phonetics}{收获}{shou1huo4}[][HSK 4]
    \definition[次,番,份]{s.}{resultados; ganhos; metaforicamente falando, conhecimento, experiência, etc. obtidos em estudo ou trabalho; os resultados obtidos por meio de trabalho árduo | colheita; colheita de safras}
    \definition{v.}{colher; juntar as colheitas;}
  \end{phonetics}
\end{entry}

\begin{entry}{收据}{6,11}{⽁、⼿}
  \begin{phonetics}{收据}{shou1ju4}
    \definition[张]{s.}{recibo | \emph{voucher}}
  \end{phonetics}
\end{entry}

\begin{entry}{收敛}{6,11}{⽁、⽁}
  \begin{phonetics}{收敛}{shou1lian3}
    \definition{v.}{diminuir | desaparecer | fazer desaparecer | exercer restrição | conter (alegria, arrogância, etc.) | constringir | (matemática) convergir}
  \end{phonetics}
\end{entry}

\begin{entry}{收集}{6,12}{⽁、⾫}
  \begin{phonetics}{收集}{shou1 ji2}[][HSK 5]
    \definition{v.}{coletar; reunir; recolher}
  \end{phonetics}
\end{entry}

\begin{entry}{早}{6}{⽇}
  \begin{phonetics}{早}{zao3}[][HSK 1]
    \definition{adj.}{precoce; antes do previsto ou planejado; antes do tempo; antes de um determinado momento |}
    \definition{adv.}{há muito tempo; desde cedo; por muito tempo; há muito tempo atrás}
    \definition{interj.}{bom dia; saudações, usadas para cumprimentar uns aos outros ao se encontrarem pela manhã}
    \definition[个]{s.}{manhã}
  \end{phonetics}
\end{entry}

\begin{entry}{早上}{6,3}{⽇、⼀}
  \begin{phonetics}{早上}{zao3shang5}[][HSK 1]
    \definition[个]{s.}{de manhã cedo; madrugada; o período antes e depois do nascer do sol; geralmente, desde o amanhecer até às 8h ou 9h da manhã; às vezes também se refere ao período entre o amanhecer e o meio-dia}
  \end{phonetics}
\end{entry}

\begin{entry}{早亡}{6,3}{⽇、⼇}
  \begin{phonetics}{早亡}{zao3wang2}
    \definition[个]{s.}{morte prematura}
    \definition{v.}{morrer prematuramente}
  \end{phonetics}
\end{entry}

\begin{entry}{早已}{6,3}{⽇、⼰}
  \begin{phonetics}{早已}{zao3 yi3}[][HSK 3]
    \definition{adv.}{há muito tempo; por muito tempo | (dialeto) no passado}
  \end{phonetics}
\end{entry}

\begin{entry}{早车}{6,4}{⽇、⾞}
  \begin{phonetics}{早车}{zao3che1}
    \definition{s.}{trem matutino | ônibus matutino}
  \end{phonetics}
\end{entry}

\begin{entry}{早安}{6,6}{⽇、⼧}
  \begin{phonetics}{早安}{zao3'an1}
    \definition{interj.}{Bom dia!}
  \end{phonetics}
\end{entry}

\begin{entry}{早早儿}{6,6,2}{⽇、⽇、⼉}
  \begin{phonetics}{早早儿}{zao3zao3r5}
    \definition{adv.}{o mais cedo possível | o mais breve possível}
  \end{phonetics}
\end{entry}

\begin{entry}{早饭}{6,7}{⽇、⾷}
  \begin{phonetics}{早饭}{zao3 fan4}[][HSK 1]
    \definition[份,顿]{s.}{o café da manhã}
  \end{phonetics}
\end{entry}

\begin{entry}{早知}{6,8}{⽇、⽮}
  \begin{phonetics}{早知}{zao3zhi1}
    \definition{v.}{prever | se alguém soubesse antes, \dots}
  \end{phonetics}
\end{entry}

\begin{entry}{早前}{6,9}{⽇、⼑}
  \begin{phonetics}{早前}{zao3qian2}
    \definition{adv.}{previamente}
  \end{phonetics}
\end{entry}

\begin{entry}{早晚}{6,11}{⽇、⽇}
  \begin{phonetics}{早晚}{zao3 wan3}
    \definition{adv./s.}{manhã e noite | mais cedo ou mais tarde; cedo ou tarde | algum tempo no futuro; algum dia; em algum momento no futuro}
  \end{phonetics}
\end{entry}

\begin{entry}{早晨}{6,11}{⽇、⽇}
  \begin{phonetics}{早晨}{zao3 chen2}[][HSK 2]
    \definition[个,段,番]{s.}{manhã cedo; manhãzinha; o período do amanhecer às oito ou nove horas; às vezes, o período da meia-noite ao meio-dia}
  \end{phonetics}
\end{entry}

\begin{entry}{早就}{6,12}{⽇、⼪}
  \begin{phonetics}{早就}{zao3 jiu4}[][HSK 2]
    \definition{adv.}{já; há muito tempo; há muito tempo atrás}
  \end{phonetics}
\end{entry}

\begin{entry}{早期}{6,12}{⽇、⽉}
  \begin{phonetics}{早期}{zao3 qi1}[][HSK 5]
    \definition{s.}{prófase; estágio inicial; fase inicial; a fase inicial de uma determinada época, processo ou vida de uma pessoa}
  \end{phonetics}
\end{entry}

\begin{entry}{早餐}{6,16}{⽇、⾷}
  \begin{phonetics}{早餐}{zao3 can1}[][HSK 2]
    \definition[份,桌,顿]{s.}{café da manhã; desejum}
  \end{phonetics}
\end{entry}

\begin{entry}{曲}{6}{⽈}
  \begin{phonetics}{曲}{qu1}
    \definition*{s.}{Sobrenome Qu}
    \definition{adj.}{dobrado; curvado (oposto a 直) | errado; injustificável | torto}
    \definition{v.}{dobrar | torcer}
  \seealsoref{直}{zhi2}
  \end{phonetics}
\end{entry}

\begin{entry}{曲棍球}{6,12,11}{⽈、⽊、⽟}
  \begin{phonetics}{曲棍球}{qu1gun4qiu2}
    \definition{s.}{hóquei em campo}
  \end{phonetics}
\end{entry}

\begin{entry}{有}{6}{⽉}
  \begin{phonetics}{有}{you3}[][HSK 1]
    \definition*{s.}{Sobrenome You}
    \definition{pref.}{usado antes do nome de certas dinastias ou etnias}
    \definition{v.}{ter; possuir; indica posse ou propriedade | existe; há; indica que certas coisas existem em certos lugares | fazer uma estimativa ou uma comparação; expressar estimativa ou comparação | indicar ação; indica que algo aconteceu ou ocorreu | usado antes de substantivos abstratos, indica quantidade ou grandeza | em termos gerais, semelhante a 某; refere-se de maneira geral a algo semelhante | usado antes de pessoa, hora e lugar, indica a existência parcial | usado antes de certos verbos para formar uma expressão idiomática, indicando cortesia, polidez}
  \seealsoref{某}{mou3}
  \end{phonetics}
\end{entry}

\begin{entry}{有(一)些}{6,1,8}{⽉、⼀、⼆}
  \begin{phonetics}{有(一)些}{you3 (yi4) xie1}[][HSK 1]
    \definition{adv.}{em vez disso; em vez de; de certa forma}
    \definition{pron.}{de certa forma}
  \seealsoref{有些}{you3 xie1}
  \end{phonetics}
\end{entry}

\begin{entry}{有(一)点儿}{6,1,9,2}{⽉、⼀、⽕、⼉}
  \begin{phonetics}{有(一)点儿}{you3 yi4 dian3r5}[][HSK 2]
    \definition{adv.}{um pouco (有点儿 + {s.} ou {v. mental})}
  \seealsoref{有点儿}{you3 dian3r5}
  \end{phonetics}
\end{entry}

\begin{entry}{有人}{6,2}{⽉、⼈}
  \begin{phonetics}{有人}{you3 ren2}[][HSK 2]
    \definition{adj.}{ocupado (como no banheiro)}
    \definition{pron.}{qualquer um; alguém}
    \definition[所]{s.}{pessoas}
    \definition{v.}{ter alguém ali}
  \end{phonetics}
\end{entry}

\begin{entry}{有力}{6,2}{⽉、⼒}
  \begin{phonetics}{有力}{you3 li4}[][HSK 5]
    \definition{adj.}{forte; vigoroso; poderoso; energético}
  \end{phonetics}
\end{entry}

\begin{entry}{有用}{6,5}{⽉、⽤}
  \begin{phonetics}{有用}{you3yong4}[][HSK 1]
    \definition{adj.}{útil; prático; funcional}
  \end{phonetics}
\end{entry}

\begin{entry}{有名}{6,6}{⽉、⼝}
  \begin{phonetics}{有名}{you3ming2}[][HSK 1]
    \definition{adj.}{conhecido; famoso; célebre; nome conhecido por todos}
  \end{phonetics}
\end{entry}

\begin{entry}{有名无实}{6,6,4,8}{⽉、⼝、⽆、⼧}
  \begin{phonetics}{有名无实}{you3ming2wu2shi2}
    \definition{v.}{(literal) tem um nome, mas não tem realidade | existe apenas no nome}
  \end{phonetics}
\end{entry}

\begin{entry}{有利}{6,7}{⽉、⼑}
  \begin{phonetics}{有利}{you3li4}[][HSK 3]
    \definition{adj.}{benéfico; favorável; vantajoso}
  \end{phonetics}
\end{entry}

\begin{entry}{有利于}{6,7,3}{⽉、⼑、⼆}
  \begin{phonetics}{有利于}{you3 li4 yu2}[][HSK 5]
    \definition{prep.}{em favor de; fazer para; em benefício de; aproveitar}
  \end{phonetics}
\end{entry}

\begin{entry}{有劲儿}{6,7,2}{⽉、⼒、⼉}
  \begin{phonetics}{有劲儿}{you3 jin4er5}[][HSK 4]
    \definition{adj.}{forte; enérgico; energético}
  \end{phonetics}
\end{entry}

\begin{entry}{有劳}{6,7}{⽉、⼒}
  \begin{phonetics}{有劳}{you3lao2}
    \definition{v.}{posso incomodá-lo; desculpe incomodá-lo | (educado) obrigado pelo seu trabalho (usado ao pedir um favor ou após ter recebido um)}
  \end{phonetics}
\end{entry}

\begin{entry}{有时}{6,7}{⽉、⽇}
  \begin{phonetics}{有时}{you3 shi2}[][HSK 1]
    \definition{expr.}{às vezes; ocasionalmente; de vez em quando}
  \seealsoref{有的时候}{you3 de5 shi2 hou4}
  \seealsoref{有时候}{you3 shi2 hou5}
  \end{phonetics}
\end{entry}

\begin{entry}{有时……有时……}{6,7,6,7}{⽉、⽇、⽉、⽇}
  \begin{phonetics}{有时……有时……}{you3shi2 you3shi2}
    \definition{adv.}{às vezes\dots às vezes\dots}
  \end{phonetics}
\end{entry}

\begin{entry}{有时候}{6,7,10}{⽉、⽇、⼈}
  \begin{phonetics}{有时候}{you3 shi2 hou5}[][HSK 1]
    \definition{adv.}{às vezes; indica um momento incerto, mas não frequente}
  \seealsoref{有的时候}{you3 de5 shi2 hou4}
  \seealsoref{有时}{you3 shi2}
  \end{phonetics}
\end{entry}

\begin{entry}{有些}{6,8}{⽉、⼆}
  \begin{phonetics}{有些}{you3 xie1}[][HSK 1]
    \definition{adv.}{um pouco; bastante; ligeiramente}
    \definition{pron.}{uma parte; alguns}
    \definition{v.}{usado para indicar que há alguns, mas não muitos;}
  \seealsoref{有(一)些}{you3 (yi4) xie1}
  \end{phonetics}
\end{entry}

\begin{entry}{有的}{6,8}{⽉、⽩}
  \begin{phonetics}{有的}{you3 de5}[][HSK 1]
    \definition{pron.}{algum, alguns}
  \end{phonetics}
\end{entry}

\begin{entry}{有的时候}{6,8,7,10}{⽉、⽩、⽇、⼈}
  \begin{phonetics}{有的时候}{you3 de5 shi2 hou4}
    \definition{adv.}{às vezes; ocasionalmente}
  \seealsoref{有时}{you3 shi2}
  \seealsoref{有时候}{you3 shi2 hou5}
  \end{phonetics}
\end{entry}

\begin{entry}{有的是}{6,8,9}{⽉、⽩、⽇}
  \begin{phonetics}{有的是}{you3 de5 shi4}[][HSK 3]
    \definition{expr.}{ter em abundância; não faltar; enfatizar que há muitos}
  \end{phonetics}
\end{entry}

\begin{entry}{有空儿}{6,8,2}{⽉、⽳、⼉}
  \begin{phonetics}{有空儿}{you3 kong4r5}[][HSK 2]
    \definition{v.}{estar livre; ter tempo livre}
  \end{phonetics}
\end{entry}

\begin{entry}{有限}{6,8}{⽉、⾩}
  \begin{phonetics}{有限}{you3 xian4}[][HSK 4]
    \definition{adj.}{finito; limitado; restrito | indica baixo grau; indica pouco número; número baixo; nível baixo}
  \end{phonetics}
\end{entry}

\begin{entry}{有限公司}{6,8,4,5}{⽉、⾩、⼋、⼝}
  \begin{phonetics}{有限公司}{you3xian4gong1si1}
    \definition{s.}{companhia limitada | corporação}
  \end{phonetics}
\end{entry}

\begin{entry}{有毒}{6,9}{⽉、⽏}
  \begin{phonetics}{有毒}{you3 du2}[][HSK 5]
    \definition{adj.}{venenoso; tóxico; nocivo; geralmente é usada para descrever as propriedades nocivas à saúde de produtos químicos, plantas ou animais.}
  \end{phonetics}
\end{entry}

\begin{entry}{有点儿}{6,9,2}{⽉、⽕、⼉}
  \begin{phonetics}{有点儿}{you3 dian3r5}
    \definition{adv.}{um pouco; indica um grau inferior, equivalente a 稍微 (usado principalmente para coisas que são insatisfatórias)}
    \definition{v.}{há um pouco; tem (ou ser de) algum; existem alguns}
  \seealsoref{稍微}{shao1wei1}
  \seealsoref{有(一)点儿}{you3 yi4 dian3r5}
  \end{phonetics}
\end{entry}

\begin{entry}{有害}{6,10}{⽉、⼧}
  \begin{phonetics}{有害}{you3 hai4}[][HSK 5]
    \definition{adj.}{prejudicial; nocivo; danoso; que pode causar danos ou prejuízos a algo}
  \end{phonetics}
\end{entry}

\begin{entry}{有效}{6,10}{⽉、⽁}
  \begin{phonetics}{有效}{you3 xiao4}[][HSK 3]
    \definition{adj.}{válido; eficiente; eficaz; capaz de alcançar os objetivos esperados}
  \end{phonetics}
\end{entry}

\begin{entry}{有着}{6,11}{⽉、⽬}
  \begin{phonetics}{有着}{you3 zhe5}[][HSK 5]
    \definition{v.}{ter; possuir; haver; existir}
  \end{phonetics}
\end{entry}

\begin{entry}{有道理}{6,12,11}{⽉、⾡、⽟}
  \begin{phonetics}{有道理}{you3dao4li5}
    \definition{adj.}{razoável}
    \definition{v.}{fazer sentido}
  \end{phonetics}
\end{entry}

\begin{entry}{有意思}{6,13,9}{⽉、⼼、⼼}
  \begin{phonetics}{有意思}{you3 yi4 si5}[][HSK 2]
    \definition{adj.}{significativo; significativo e intrigante | interessante; agradável}
    \definition{v.}{ter interesse por; ser atraído sexualmente}
  \end{phonetics}
\end{entry}

\begin{entry}{有趣}{6,15}{⽉、⾛}
  \begin{phonetics}{有趣}{you3qu4}[][HSK 4]
    \definition{adj.}{interessante; fascinante; divertido; excitante; estimulante}
  \end{phonetics}
\end{entry}

\begin{entry}{朵}{6}{⽊}
  \begin{phonetics}{朵}{duo3}[][HSK 5]
    \definition*{s.}{Sobrenome Duo}
    \definition{clas.}{usado para flores, nuvens ou coisas que se assemelham a flores e nuvens}
  \end{phonetics}
\end{entry}

\begin{entry}{机}{6}{⽊}
  \begin{phonetics}{机}{ji1}
    \definition*{s.}{Sobrenome Ji}
    \definition{adj.}{flexível; perspicaz; destreza; agilidade}
    \definition[台]{s.}{máquina; motor | avião; aeronave; aeroplano; refere-se especificamente a aeronaves | ponto crucial; os fatores-chave para a ocorrência e mudança das coisas | chance; ocasião; oportunidade; um momento crítico ou oportuno para o desenvolvimento e mudança das coisas | organismo; funções vitais dos organismos | besta; mecanismo de disparo de flechas de madeira em uma besta antiga | assuntos importantes; assuntos extremamente importantes e confidenciais | ideia; intenção}
  \end{phonetics}
\end{entry}

\begin{entry}{机甲}{6,5}{⽊、⽥}
  \begin{phonetics}{机甲}{ji1jia3}
    \definition{s.}{\emph{mecha} (robôs operados pelo homem em mangá japonês)}
  \end{phonetics}
\end{entry}

\begin{entry}{机会}{6,6}{⽊、⼈}
  \begin{phonetics}{机会}{ji1hui4}[][HSK 2]
    \definition[个,次,种,些]{s.}{chance; oportunidade; momento favorável raro}
  \end{phonetics}
\end{entry}

\begin{entry}{机场}{6,6}{⽊、⼟}
  \begin{phonetics}{机场}{ji1chang3}[][HSK 1]
    \definition[个,家,处,座]{s.}{aeródromo; campo de aviação; aeroporto; campo de voo}
  \end{phonetics}
\end{entry}

\begin{entry}{机制}{6,8}{⽊、⼑}
  \begin{phonetics}{机制}{ji1 zhi4}[][HSK 5]
    \definition{s.}{mecanismo; processado por máquina; feito por máquina}
  \end{phonetics}
\end{entry}

\begin{entry}{机构}{6,8}{⽊、⽊}
  \begin{phonetics}{机构}{ji1gou4}[][HSK 4]
    \definition[所]{s.}{órgão; organização; instituição; instalações; aparelhamento; configuração | mecanismo; funcionamento interno de uma máquina ou unidade | estrutura interna de uma organização}
  \end{phonetics}
\end{entry}

\begin{entry}{机械}{6,11}{⽊、⽊}
  \begin{phonetics}{机械}{ji1xie4}
    \definition{s.}{máquina | maquinaria | mecânica}
  \end{phonetics}
\end{entry}

\begin{entry}{机票}{6,11}{⽊、⽰}
  \begin{phonetics}{机票}{ji1 piao4}[][HSK 1]
    \definition[张]{s.}{passagem aérea; passagem de avião}
  \seealsoref{飞机票}{fei1ji1 piao4}
  \end{phonetics}
\end{entry}

\begin{entry}{机遇}{6,12}{⽊、⾡}
  \begin{phonetics}{机遇}{ji1yu4}[][HSK 4]
    \definition[个]{s.}{chance; oportunidade; circunstâncias favoráveis}
  \end{phonetics}
\end{entry}

\begin{entry}{机器}{6,16}{⽊、⼝}
  \begin{phonetics}{机器}{ji1qi4}[][HSK 3]
    \definition[台,部,个]{s.}{máquina; maquinário; motor; dispositivos e máquinas que são montados a partir de peças, podem funcionar, transformar energia ou produzir trabalho útil podem ser usados como ferramentas de produção, reduzindo a intensidade do trabalho humano e aumentando a produtividade | aparato; sistema político e econômico}
  \end{phonetics}
\end{entry}

\begin{entry}{机器人}{6,16,2}{⽊、⼝、⼈}
  \begin{phonetics}{机器人}{ji1 qi4 ren2}[][HSK 5]
    \definition[个]{s.}{androide; golem | pessoa mecânica | robô}
  \end{phonetics}
\end{entry}

\begin{entry}{杀}{6}{⽊}
  \begin{phonetics}{杀}{sha1}[][HSK 5]
    \definition{adv.}{em extremo; excessivamente; usado após um verbo, indica grau intenso}
    \definition{v.}{matar; abater; esquartejar | lutar; entrar em batalha | enfraquecer; reduzir; diminuir | decolar; neutralizar}
  \end{phonetics}
\end{entry}

\begin{entry}{杀气}{6,4}{⽊、⽓}
  \begin{phonetics}{杀气}{sha1qi4}
    \definition{s.}{espírito assassino | aura de morte}
    \definition{v.}{desabafar a raiva de alguém}
  \end{phonetics}
\end{entry}

\begin{entry}{杀毒}{6,9}{⽊、⽏}
  \begin{phonetics}{杀毒}{sha1 du2}[][HSK 5]
    \definition{s.}{(computação) antivírus}
    \definition{v.}{esterilizar; desinfetar | (computação) eliminar um vírus}
  \end{phonetics}
\end{entry}

\begin{entry}{杂}{6}{⽊}
  \begin{phonetics}{杂}{za2}[][HSK 6]
    \definition{adj.}{diversos; misto; misturados | extra; irregular | variado}
    \definition{v.}{misturar}
  \end{phonetics}
\end{entry}

\begin{entry}{杂志}{6,7}{⽊、⼼}
  \begin{phonetics}{杂志}{za2zhi4}[][HSK 3]
    \definition[本,期,份,种]{s.}{jornal; revista; publicação}
  \end{phonetics}
\end{entry}

\begin{entry}{杂志社}{6,7,7}{⽊、⼼、⽰}
  \begin{phonetics}{杂志社}{za2zhi4she4}
    \definition{s.}{editora de revista}
  \end{phonetics}
\end{entry}

\begin{entry}{杂技}{6,7}{⽊、⼿}
  \begin{phonetics}{杂技}{za2ji4}
    \definition[场]{s.}{acrobacia}
  \end{phonetics}
\end{entry}

\begin{entry}{权}{6}{⽊}
  \begin{phonetics}{权}{quan2}[][HSK 6]
    \definition*{s.}{Sobrenome Quan}
    \definition{adv.}{provisoriamente; por enquanto}
    \definition{s.}{Lliterário: contrapeso; peso deslizante de uma balança romana | poder; autoridade | direito | posição vantajosa | conveniência}
    \definition{v.}{pesar; medir o peso}
  \end{phonetics}
\end{entry}

\begin{entry}{权利}{6,7}{⽊、⼑}
  \begin{phonetics}{权利}{quan2li4}[][HSK 4]
    \definition[项,种,个,条,份]{s.}{direito; interesse; os poderes e benefícios (em oposição a 义务) exercidos por um cidadão ou pessoa jurídica de acordo com a lei}
  \seealsoref{义务}{yi4wu4}
  \end{phonetics}
\end{entry}

\begin{entry}{次}{6}{⽋}
  \begin{phonetics}{次}{ci4}[][HSK 1,4]
    \definition*{s.}{Sobrenome Ci}
    \definition{adj.}{de segunda categoria; de qualidade inferior}
    \definition{clas.}{usado para coisas ou ações que podem ser repetidas}
    \definition{num.}{segundo; próximo}
    \definition{pref.}{(química) hipo-, radical ácido ou composto contendo dois átomos de oxigênio a menos}
    \definition{s.}{ordem; sequência; classificação | local de parada em uma viagem; escala}
  \end{phonetics}
\end{entry}

\begin{entry}{次数}{6,13}{⽋、⽁}
  \begin{phonetics}{次数}{ci4 shu4}[][HSK 6]
    \definition{s.}{frequência; número de vezes; o número de vezes que uma ação ou evento é repetido}
  \end{phonetics}
\end{entry}

\begin{entry}{欢}{6}{⽋}
  \begin{phonetics}{欢}{huan1}
    \definition*{s.}{Sobrenome Huan}
    \definition{adj.}{alegre; feliz; jubilante | vigoroso; energético; em pleno andamento; com grande impulso}
    \definition{s.}{amante; querida; um apelido usado por mulheres nos tempos antigos para se referir aos seus amantes; agora, geralmente se refere a alguém de quem você gosta ou com quem tem um relacionamento romântico}
  \end{phonetics}
\end{entry}

\begin{entry}{欢乐}{6,5}{⽋、⼃}
  \begin{phonetics}{欢乐}{huan1le4}[][HSK 3]
    \definition{adj.}{feliz; alegre; felicidade (geralmente coletiva)}
  \end{phonetics}
\end{entry}

\begin{entry}{欢快}{6,7}{⽋、⼼}
  \begin{phonetics}{欢快}{huan1kuai4}
    \definition{adj.}{feliz e sem ansiedade | vívido}
  \end{phonetics}
\end{entry}

\begin{entry}{欢迎}{6,7}{⽋、⾡}
  \begin{phonetics}{欢迎}{huan1ying2}[][HSK 2]
    \definition{adj.}{bem-vindo}
    \definition{v.}{dar as boas-vindas; cumprimentar; receber com alegria | dar as boas-vindas; receber favoravelmente (bem)}
  \end{phonetics}
\end{entry}

\begin{entry}{此}{6}{⽌}
  \begin{phonetics}{此}{ci3}[][HSK 4]
    \definition*{s.}{Sobrenome Ci}
    \definition{pron.}{esse; essa; isso; este; esta; isto; indica ou se refere a uma pessoa ou coisa que está mais próxima, equivalente a 这 ou 这个 (em oposição a 彼) | aqui e agora; refere-se a um tempo ou lugar recente, equivalente a 这会儿 ou 这里}
  \seealsoref{彼}{bi3}
  \seealsoref{这}{zhe4}
  \seealsoref{这会儿}{zhe4 hui4r5}
  \seealsoref{这里}{zhe4 li3}
  \seealsoref{这个}{zhe4ge5}
  \end{phonetics}
\end{entry}

\begin{entry}{此处}{6,5}{⽌、⼡}
  \begin{phonetics}{此处}{ci3 chu4}[][HSK 6]
    \definition{pron.}{este lugar; aqui (literário)}
  \end{phonetics}
\end{entry}

\begin{entry}{此外}{6,5}{⽌、⼣}
  \begin{phonetics}{此外}{ci3wai4}[][HSK 4]
    \definition{conj.}{além disso; em adição; além das coisas ou situações mencionadas acima}
  \end{phonetics}
\end{entry}

\begin{entry}{此后}{6,6}{⽌、⼝}
  \begin{phonetics}{此后}{ci3 hou4}[][HSK 5]
    \definition{s.}{daqui em diante; doravante; depois disso; após isso; de agora em diante}
  \end{phonetics}
\end{entry}

\begin{entry}{此次}{6,6}{⽌、⽋}
  \begin{phonetics}{此次}{ci3 ci4}[][HSK 6]
    \definition{adv.}{desta vez; refere-se a um ponto específico no tempo ou período de tempo}
  \end{phonetics}
\end{entry}

\begin{entry}{此时}{6,7}{⽌、⽇}
  \begin{phonetics}{此时}{ci3 shi2}[][HSK 5]
    \definition{s.}{agora; no presente; agora mesmo; neste momento; por enquanto}
  \end{phonetics}
\end{entry}

\begin{entry}{此事}{6,8}{⽌、⼅}
  \begin{phonetics}{此事}{ci3 shi4}[][HSK 6]
    \definition{s.}{matéria; assunto}
  \end{phonetics}
\end{entry}

\begin{entry}{此刻}{6,8}{⽌、⼑}
  \begin{phonetics}{此刻}{ci3 ke4}[][HSK 5]
    \definition{s.}{agora; no momento; exatamente agora; neste momento}
  \end{phonetics}
\end{entry}

\begin{entry}{此前}{6,9}{⽌、⼑}
  \begin{phonetics}{此前}{ci3 qian2}[][HSK 6]
    \definition{adv.}{literário: antes; anteriormente | literário: antes disso}
  \end{phonetics}
\end{entry}

\begin{entry}{此致}{6,10}{⽌、⾄}
  \begin{phonetics}{此致}{ci3 zhi4}[][HSK 6]
    \definition{expr.}{Atenciosamente; Sinceramente; Com os melhores votos; usada no final de uma carta ou correspondência oficial}
  \end{phonetics}
\end{entry}

\begin{entry}{死}{6}{⽍}
  \begin{phonetics}{死}{si3}[][HSK 3]
    \definition{adj.}{até a morte | implacável; mortal | fixo; rígido; inflexível | intransitável; fechado | (expressando raiva, reclamação, etc., às vezes jocosamente) maldito}
    \definition{adv.}{(frequentemente no negativo) teimosamente; inflexivelmente}
    \definition{v.}{morrer; estar morto (oposto a 生 e 活)}
  \seealsoref{活}{huo2}
  \seealsoref{生}{sheng1}
  \end{phonetics}
\end{entry}

\begin{entry}{死亡}{6,3}{⽍、⼇}
  \begin{phonetics}{死亡}{si3wang2}
    \definition{s.}{morte}
    \definition{v.}{morrer}
  \end{phonetics}
\end{entry}

\begin{entry}{毕}{6}{⽐}
  \begin{phonetics}{毕}{bi4}
    \definition*{s.}{Bi, uma das mansões lunares; a décima nona das vinte e oito constelações em que a esfera celeste foi dividida, consistindo de oito estrelas, seis em Híades e duas em Touro | Sobrenome Bi}
    \definition{adv.}{tudo; completamente; totalmente}
    \definition{v.}{terminar; realizar; concluir  | completar; terminar}
  \end{phonetics}
\end{entry}

\begin{entry}{毕业}{6,5}{⽐、⼀}
  \begin{phonetics}{毕业}{bi4ye4}[][HSK 4]
    \definition{s.}{formatura}
    \definition{v.+compl.}{formar-se}
  \end{phonetics}
\end{entry}

\begin{entry}{毕业生}{6,5,5}{⽐、⼀、⽣}
  \begin{phonetics}{毕业生}{bi4 ye4 sheng1}[][HSK 4]
    \definition[个]{s.}{diplomado; graduado; bacharel; pessoa que recebeu um diploma, grau ou certificado}
  \end{phonetics}
\end{entry}

\begin{entry}{毕竟}{6,11}{⽐、⾳}
  \begin{phonetics}{毕竟}{bi4jing4}[][HSK 5]
    \definition{adv.}{afinal de contas; quando tudo estiver dito e feito; em última análise; indica um resultado que não pode ser alterado, enfatizando que se trata de uma causa ou fato que precisa ser enfocado para referência | significa 到底, 究竟, 终究, indicando a conclusão final alcançada}
  \seealsoref{到底}{dao4di3}
  \seealsoref{究竟}{jiu1jing4}
  \seealsoref{终究}{zhong1jiu1}
  \end{phonetics}
\end{entry}

\begin{entry}{汗}{6}{⽔}
  \begin{phonetics}{汗}{han2}
    \definition*{s.}{Abreviação de Khan}[他是成吉思汗。___Ele é Genghis Khan.]
  \end{phonetics}
  \begin{phonetics}{汗}{han4}[][HSK 5]
    \definition{s.}{suor; transpiração; perspiração}
  \end{phonetics}
\end{entry}

\begin{entry}{汗水}{6,4}{⽔、⽔}
  \begin{phonetics}{汗水}{han4shui3}
    \definition*{s.}{Rio Han (Hanshui)}
    \definition{s.}{suor | transpiração}
  \end{phonetics}
\end{entry}

\begin{entry}{汗液}{6,11}{⽔、⽔}
  \begin{phonetics}{汗液}{han4ye4}
    \definition{s.}{suor}
  \end{phonetics}
\end{entry}

\begin{entry}{汗腺}{6,13}{⽔、⾁}
  \begin{phonetics}{汗腺}{han4xian4}
    \definition{s.}{glândula sudorípara}
  \end{phonetics}
\end{entry}

\begin{entry}{江}{6}{⽔}
  \begin{phonetics}{江}{jiang1}[][HSK 4]
    \definition*{s.}{Rio Changjiang | Sobrenome Jiang}
    \definition[条,道]{s.}{rio grande}
  \end{phonetics}
\end{entry}

\begin{entry}{江水}{6,4}{⽔、⽔}
  \begin{phonetics}{江水}{jiang1shui3}
    \definition{s.}{água do rio}
  \end{phonetics}
\end{entry}

\begin{entry}{江西}{6,6}{⽔、⾑}
  \begin{phonetics}{江西}{jiang1xi1}
    \definition*{s.}{Jiangxi}
  \end{phonetics}
\end{entry}

\begin{entry}{江苏}{6,7}{⽔、⾋}
  \begin{phonetics}{江苏}{jiang1su1}
    \definition*{s.}{Província de Jiangsu}
  \end{phonetics}
\end{entry}

\begin{entry}{江南水乡}{6,9,4,3}{⽔、⼗、⽔、⼄}
  \begin{phonetics}{江南水乡}{jiang1nan2shui3xiang1}
    \definition*{s.}{Vila Aquática de Jiangnan | Cidades Aquáticas}
  \end{phonetics}
\end{entry}

\begin{entry}{池}{6}{⽔}
  \begin{phonetics}{池}{chi2}
    \definition*{s.}{Sobrenome Chi}
    \definition[个,片]{s.}{piscina; lagoa | qualquer espaço fechado com laterais elevadas | baias (em um teatro); a parte frontal do salão principal do teatro | fosso}
  \end{phonetics}
\end{entry}

\begin{entry}{池子}{6,3}{⽔、⼦}
  \begin{phonetics}{池子}{chi2 zi5}[][HSK 5]
    \definition{s.}{lago; lagoa; viveiro | piscina; piscina do balneário | (antigo) arquibancada (primeiras fileiras em um teatro) | pista de dança de um salão de baile}
  \end{phonetics}
\end{entry}

\begin{entry}{污}{6}{⽔}
  \begin{phonetics}{污}{wu1}
    \definition{adj.}{sujo; imundo; imundo | corrupto}
    \definition{s.}{sujeira; imundície | esgoto; água suja; coisas sujas}
    \definition{v.}{contaminar; sujar | manchar}
  \end{phonetics}
\end{entry}

\begin{entry}{污水}{6,4}{⽔、⽔}
  \begin{phonetics}{污水}{wu1shui3}[][HSK 5]
    \definition{s.}{água suja (ou poluída, residual); esgoto; lodo | efluente; drenagem; água suja; água poluída; água residual}
  \end{phonetics}
\end{entry}

\begin{entry}{污染}{6,9}{⽔、⽊}
  \begin{phonetics}{污染}{wu1ran3}[][HSK 5]
    \definition{s.}{poluição}
    \definition{v.}{poluir; contaminar com substâncias nocivas e prejudiciais; refere-se especificamente à destruição do ambiente natural causada por substâncias nocivas, tais como gases, líquidos e resíduos emitidos por indústrias, minas, veículos, etc. | contaminar; metáfora de que pensamentos prejudiciais causam efeitos negativos nas pessoas}
  \end{phonetics}
\end{entry}

\begin{entry}{污染区}{6,9,4}{⽔、⽊、⼖}
  \begin{phonetics}{污染区}{wu1ran3qu1}
    \definition{s.}{área contaminada}
  \end{phonetics}
\end{entry}

\begin{entry}{污染物}{6,9,8}{⽔、⽊、⽜}
  \begin{phonetics}{污染物}{wu1ran3wu4}
    \definition{s.}{poluente}
  \seealsoref{污染物质}{wu1ran3 wu4zhi4}
  \end{phonetics}
\end{entry}

\begin{entry}{污染物质}{6,9,8,8}{⽔、⽊、⽜、⾙}
  \begin{phonetics}{污染物质}{wu1ran3 wu4zhi4}
    \definition{s.}{poluente}
  \seealsoref{污染物}{wu1ran3wu4}
  \end{phonetics}
\end{entry}

\begin{entry}{汤}{6}{⽔}
  \begin{phonetics}{汤}{shang1}
    \definition{s.}{correnteza forte}
  \end{phonetics}
  \begin{phonetics}{汤}{tang1}[][HSK 3]
    \definition*{s.}{Sobrenome Tang}
    \definition[勺,碗,杯,锅]{s.}{água quente; água fervente | fontes termais | água utilizada para ferver algo| sopa; caldo | uma preparação líquida de ervas medicinais; decocção}
  \end{phonetics}
\end{entry}

\begin{entry}{灯}{6}{⽕}
  \begin{phonetics}{灯}{deng1}[][HSK 2]
    \definition*{s.}{Sobrenome Deng}
    \definition[盏,个]{s.}{lâmpada; luz; lanterna; dispositivo luminoso, usado principalmente para iluminação | queimador; um aparelho que brilha e aquece como uma lâmpada e pode ser usado para aquecer | tubo; válvula; o nome popular dado aos tubos eletrônicos com formato semelhante a lâmpadas encontrados em aparelhos antigos, como rádios}
  \end{phonetics}
\end{entry}

\begin{entry}{灯丝}{6,5}{⽕、⼀}
  \begin{phonetics}{灯丝}{deng1si1}
    \definition{s.}{filamento (de uma lâmpada)}
  \end{phonetics}
\end{entry}

\begin{entry}{灯号}{6,5}{⽕、⼝}
  \begin{phonetics}{灯号}{deng1hao4}
    \definition{s.}{sinal luminoso | luz indicadora}
  \end{phonetics}
\end{entry}

\begin{entry}{灯光}{6,6}{⽕、⼉}
  \begin{phonetics}{灯光}{deng1 guang1}[][HSK 4]
    \definition{s.}{iluminação; luminosidade da lâmpada | luminação (palco); equipamento de iluminação para palco ou estúdio}
  \end{phonetics}
\end{entry}

\begin{entry}{灯泡}{6,8}{⽕、⽔}
  \begin{phonetics}{灯泡}{deng1pao4}
    \definition[个]{s.}{lâmpada | (gíria) terceiro indesejado estragando encontro de casal}
  \seealsoref{电灯泡}{dian4deng1pao4}
  \end{phonetics}
\end{entry}

\begin{entry}{灯标}{6,9}{⽕、⽊}
  \begin{phonetics}{灯标}{deng1biao1}
    \definition{s.}{farol | luz de farol}
  \end{phonetics}
\end{entry}

\begin{entry}{灰}{6}{⽕}
  \begin{phonetics}{灰}{hui1}
    \definition{adj.}{cinza (cor) | desanimado; desencorajado; deprimido}
    \definition[把,堆]{s.}{cinzas; pó que sobra após a queima de um objeto | pó; poeira; substância em pó | cal; argamassa (de cal)}
  \end{phonetics}
\end{entry}

\begin{entry}{灰色}{6,6}{⽕、⾊}
  \begin{phonetics}{灰色}{hui1 se4}[][HSK 5]
    \definition{adj.}{obscuro; ambíguo | sombrio; pessimista}
    \definition[个]{s.}{cor cinza; acinzentado}
  \end{phonetics}
\end{entry}

\begin{entry}{爷}{6}{⽗}
  \begin{phonetics}{爷}{ye2}
    \definition[个,位,名,些]{s.}{(dialeto) pai | (dialeto) avô | (uma forma respeitosa de se dirigir a um homem idoso) tio | (uma forma de se dirigir a um oficial ou homem rico) senhor; mestre; lorde; o antigo nome para burocratas, pessoas ricas, etc. | deus; forma de tratamento de um adorador para um deus}
  \end{phonetics}
\end{entry}

\begin{entry}{爷爷}{6,6}{⽗、⽗}
  \begin{phonetics}{爷爷}{ye2ye5}[][HSK 1]
    \definition[个,位]{s.}{avô (paterno)}
  \end{phonetics}
\end{entry}

\begin{entry}{百}{6}{⽩}
  \begin{phonetics}{百}{bai3}[][HSK 1]
    \definition{adj.}{todos; todos os tipos de; multifacetados; numerosos}
    \definition{adv.}{muito; sempre}
    \definition{num.}{cem; 100}
  \end{phonetics}
\end{entry}

\begin{entry}{百分}{6,4}{⽩、⼑}
  \begin{phonetics}{百分}{bai3fen1}
    \definition{s.}{por cento | nota máxima; pontuação máxima; 100 pontos (em um sistema de classificação de cem pontos) | um jogo específico; um jogo de pôquer}
  \end{phonetics}
\end{entry}

\begin{entry}{百分点}{6,4,9}{⽩、⼑、⽕}
  \begin{phonetics}{百分点}{bai3 fen1 dian3}[][HSK 6]
    \definition[个]{s.}{ponto percentual; em estatística, um por cento é chamado de ponto percentual}
  \end{phonetics}
\end{entry}

\begin{entry}{百货}{6,8}{⽩、⾙}
  \begin{phonetics}{百货}{bai3 huo4}[][HSK 4]
    \definition{s.}{mercadorias em geral; loja de departamentos; um termo geral para bens que incluem principalmente roupas, utensílios e necessidades diárias gerais}
  \end{phonetics}
\end{entry}

\begin{entry}{百般}{6,10}{⽩、⾈}
  \begin{phonetics}{百般}{bai3ban1}
    \definition{adv.}{de todas as maneiras possíveis | por todos os meios}
  \end{phonetics}
\end{entry}

\begin{entry}{竹}{6}{⽵}
  \begin{phonetics}{竹}{zhu2}
    \definition[根]{s.}{bambu | instrumento de sopro | tira de bambu}
  \end{phonetics}
\end{entry}

\begin{entry}{竹子}{6,3}{⽵、⼦}
  \begin{phonetics}{竹子}{zhu2zi5}[][HSK 5]
    \definition[块,株]{s.}{bambu; nome genérico para os tipos de bambu}
  \end{phonetics}
\end{entry}

\begin{entry}{竹马}{6,3}{⽵、⾺}
  \begin{phonetics}{竹马}{zhu2ma3}
    \definition{s.}{cavalo de bambu | vara de bambu usada como cavalo de brinquedo}
  \end{phonetics}
\end{entry}

\begin{entry}{竹排}{6,11}{⽵、⼿}
  \begin{phonetics}{竹排}{zhu2pai2}
    \definition{s.}{jangada de bambu}
  \end{phonetics}
\end{entry}

\begin{entry}{竹编}{6,12}{⽵、⽷}
  \begin{phonetics}{竹编}{zhu2bian1}
    \definition{s.}{vime | tecelagem de bambu}
  \end{phonetics}
\end{entry}

\begin{entry}{米}{6}{⽶}[Kangxi 119]
  \begin{phonetics}{米}{mi3}[][HSK 2,3]
    \definition*{s.}{Sobrenome Mi}
    \definition{clas.}{m, metro; unidade principal de comprimento do sistema métrico}
    \definition[粒,斤]{s.}{arroz | sementes descascadas; refere-se a sementes comestíveis descascadas ou sem casca | qualquer coisa que se assemelhe a um grão de arroz}
  \end{phonetics}
\end{entry}

\begin{entry}{米饭}{6,7}{⽶、⾷}
  \begin{phonetics}{米饭}{mi3fan4}[][HSK 1]
    \definition{s.}{arroz (cozido)}
  \end{phonetics}
\end{entry}

\begin{entry}{红}{6}{⽷}
  \begin{phonetics}{红}{hong2}[][HSK 2]
    \definition*{s.}{Sobrenome Hong}
    \definition{adj.}{vermelho | popular; bem-sucedido; símbolo de sucesso ou valorização | vermelho; revolucionário; símbolo da revolução | festivo; símbolo de alegria}
    \definition{s.}{tecido vermelho, bandeirinhas, etc. usados em ocasiões festivas | bônus; dividendo}
  \end{phonetics}
\end{entry}

\begin{entry}{红包}{6,5}{⽷、⼓}
  \begin{phonetics}{红包}{hong2 bao1}[][HSK 4]
    \definition[个]{s.}{saco de papel vermelho ou envelope contendo dinheiro como presente, gorjeta ou bônus | suborno; propina}
  \end{phonetics}
\end{entry}

\begin{entry}{红色}{6,6}{⽷、⾊}
  \begin{phonetics}{红色}{hong2 se4}[][HSK 2]
    \definition{adj.}{vermelho; revolucionário; símbolo da revolução ou da consciência política elevada}
    \definition{s.}{cor vermelha}
  \end{phonetics}
\end{entry}

\begin{entry}{红宝石}{6,8,5}{⽷、⼧、⽯}
  \begin{phonetics}{红宝石}{hong2bao3shi2}
    \definition{s.}{rubi}
  \end{phonetics}
\end{entry}

\begin{entry}{红线}{6,8}{⽷、⽷}
  \begin{phonetics}{红线}{hong2xian4}
    \definition{s.}{linha vermelha}
  \end{phonetics}
\end{entry}

\begin{entry}{红茶}{6,9}{⽷、⾋}
  \begin{phonetics}{红茶}{hong2 cha2}[][HSK 3]
    \definition[杯,壶,斤,种]{s.}{chá preto; chá acabado produzido através de fermentação completa}
  \end{phonetics}
\end{entry}

\begin{entry}{红烧}{6,10}{⽷、⽕}
  \begin{phonetics}{红烧}{hong2shao1}
    \definition{s.}{guisado em molho de soja (prato)}
  \end{phonetics}
\end{entry}

\begin{entry}{红酒}{6,10}{⽷、⾣}
  \begin{phonetics}{红酒}{hong2 jiu3}[][HSK 3]
    \definition[瓶,杯,壶,斤,箱]{s.}{vinho tinto}
  \end{phonetics}
\end{entry}

\begin{entry}{红绿灯}{6,11,6}{⽷、⽷、⽕}
  \begin{phonetics}{红绿灯}{hong2lv4deng1}
    \definition[个]{s.}{semáforo | sinal de trânsito}
  \end{phonetics}
\end{entry}

\begin{entry}{红薯}{6,16}{⽷、⾋}
  \begin{phonetics}{红薯}{hong2shu3}
    \definition{s.}{batata doce}
  \end{phonetics}
\end{entry}

\begin{entry}{约}{6}{⽷}
  \begin{phonetics}{约}{yao1}
    \definition{adj.}{econômico; frugal | simples; breve | indistinto}
    \definition{adv.}{cerca de; ao redor; aproximadamente}
    \definition{s.}{pacto; acordo; nomeação; coisa prometida}
    \definition{v.}{marcar uma consulta; organizar | perguntar ou convidar com antecedência | restringir; conter | reduzir (fração aproximada)}
  \end{phonetics}
  \begin{phonetics}{约}{yue1}[][HSK 3]
    \definition*{s.}{Sobrenome Yue}
    \definition{adj.}{econômico; frugal | simples; breve; resumido | indistinto; confuso}
    \definition{adv.}{cerca de; ao redor; aproximadamente}
    \definition{s.}{pacto; acordo; nomeação; o que foi combinado}
    \definition{v.}{combinar; propor ou discutir antecipadamente (o que deve ser respeitado por todos) | convidar com antecedência | restringir; conter | reduzir (fração aproximada)}
  \end{phonetics}
\end{entry}

\begin{entry}{约会}{6,6}{⽷、⼈}
  \begin{phonetics}{约会}{yue1hui4}[][HSK 4]
    \definition[个,次]{s.}{data; compromisso; engajamento; reunião pré-agendada}
    \definition{v.}{marcar uma reunião; marcar um encontro;}
  \end{phonetics}
\end{entry}

\begin{entry}{约束}{6,7}{⽷、⽊}
  \begin{phonetics}{约束}{yue1shu4}[][HSK 5]
    \definition{adj.}{amarrado}
    \definition{s.}{restrição; constrangimento; engajamento}
    \definition{v.}{amarrar; prender; reprimir; restringir; manter dentro de si}
  \end{phonetics}
\end{entry}

\begin{entry}{级}{6}{⽷}
  \begin{phonetics}{级}{ji2}[][HSK 2]
    \definition{clas.}{usado para degraus, escadas, pisos de torres, etc.}
    \definition[个,种]{s.}{nível; classificação; grau; classe | série; turma; qualquer uma das divisões anuais de um curso escolar | degrau}
  \end{phonetics}
\end{entry}

\begin{entry}{纪}{6}{⽷}
  \begin{phonetics}{纪}{ji3}
    \definition*{s.}{Sobrenome Ji}
    \definition{s.}{disciplina | um período de doze anos (na China antiga); um período de anos | (geologia) subdivisão de uma era geológica; período}
    \definition{v.}{colocar por escrito; registrar; mesmo significado de 记, usado principalmente em 记录, 纪年, 纪元, 纪传, etc. | classificar (fios de seda)}
  \seealsoref{记}{ji4}
  \seealsoref{纪传}{ji4 zhuan4}
  \seealsoref{记录}{ji4lu4}
  \seealsoref{纪年}{ji4nian2}
  \seealsoref{纪元}{ji4yuan2}
  \end{phonetics}
  \begin{phonetics}{纪}{ji4}
    \definition*{s.}{Sobrenome Ji}
    \definition{s.}{disciplina | idade; época | (geologia) período | um período de doze anos (na China antiga); um período de anos | (geologia) subdivisão de uma era geológica}
    \definition{v.}{colocar por escrito; registrar | registrar, mesmo significado de 记, usado principalmente em 记录, 纪年, 纪元, 纪传, etc. | classificar (fios de seda)}
  \seealsoref{记}{ji4}
  \seealsoref{纪传}{ji4 zhuan4}
  \seealsoref{记录}{ji4lu4}
  \seealsoref{纪年}{ji4nian2}
  \seealsoref{纪元}{ji4yuan2}
  \end{phonetics}
\end{entry}

\begin{entry}{纪元}{6,4}{⽷、⼉}
  \begin{phonetics}{纪元}{ji4yuan2}
    \definition{s.}{o início de uma era (por exemplo, o reinado de um imperador) | época; era}
  \end{phonetics}
\end{entry}

\begin{entry}{纪传}{6,6}{⽷、⼈}
  \begin{phonetics}{纪传}{ji4 zhuan4}
    \definition{s.}{crônica; biografia}
  \end{phonetics}
\end{entry}

\begin{entry}{纪传体}{6,6,7}{⽷、⼈、⼈}
  \begin{phonetics}{纪传体}{ji4 zhuan4 ti3}
    \definition{s.}{história apresentada em uma série de biografias | gênero histórico baseado em biografia}
  \end{phonetics}
\end{entry}

\begin{entry}{纪年}{6,6}{⽷、⼲}
  \begin{phonetics}{纪年}{ji4nian2}
    \definition{s.}{cronologia; uma maneira de numerar os anos | registro cronológico de eventos; anais; um dos gêneros de livros históricos é organizar fatos históricos em ordem cronológica}
  \end{phonetics}
\end{entry}

\begin{entry}{纪录}{6,8}{⽷、⼹}
  \begin{phonetics}{纪录}{ji4lu4}[][HSK 3]
    \definition[项,个]{s.}{recorde (esportes); o número mais alto ou mais baixo registrado em um determinado período de tempo}
  \end{phonetics}
\end{entry}

\begin{entry}{纪念}{6,8}{⽷、⼼}
  \begin{phonetics}{纪念}{ji4nian4}[][HSK 3]
    \definition[个,次]{s.}{lembrança; recordação; usado para representar uma lembrança (objeto)}
    \definition{v.}{comemorar; expressar saudade por pessoas ou coisas através de objetos ou ações}
  \end{phonetics}
\end{entry}

\begin{entry}{纪律}{6,9}{⽷、⼻}
  \begin{phonetics}{纪律}{ji4lv4}[][HSK 4]
    \definition{s.}{disciplina; código de conduta que cada membro da vida coletiva deve observar}
  \end{phonetics}
\end{entry}

\begin{entry}{网}{6}{⽹}[Kangxi 122]
  \begin{phonetics}{网}{wang3}[][HSK 2]
    \definition[张]{s.}{rede; um dispositivo feito de corda ou barbante para capturar peixes e pássaros | algo que parece uma rede | rede; uma rede de organizações; um sistema}
    \definition{v.}{pegar com uma rede | cobrir como com uma rede}
  \end{phonetics}
\end{entry}

\begin{entry}{网上}{6,3}{⽹、⼀}
  \begin{phonetics}{网上}{wang3 shang4}[][HSK 1]
    \definition{s.}{\emph{online}; refere-se a acessar a Internet através de um computador ou celular para pesquisar e consultar informações na rede}
  \end{phonetics}
\end{entry}

\begin{entry}{网上银行}{6,3,11,6}{⽹、⼀、⾦、⾏}
  \begin{phonetics}{网上银行}{wang3shang4yin2hang2}
    \definition[个]{s.}{banco \emph{online} | acesso a operações bancárias via \emph{Internet}}
  \seealsoref{网银}{wang3yin2}
  \end{phonetics}
\end{entry}

\begin{entry}{网友}{6,4}{⽹、⼜}
  \begin{phonetics}{网友}{wang3 you3}[][HSK 1]
    \definition{s.}{internauta; usuário da \emph{Internet}; amigos que se conhecem pela Internet; também usado como forma de tratamento entre internautas}
  \end{phonetics}
\end{entry}

\begin{entry}{网址}{6,7}{⽹、⼟}
  \begin{phonetics}{网址}{wang3 zhi3}[][HSK 4]
    \definition{s.}{\emph{website}; endereço da \emph{web}; endereço de um \emph{site} na \emph{Internet}, que os usuários podem acessar, consultar e obter recursos de informações nesse \emph{site} clicando nele}
  \end{phonetics}
\end{entry}

\begin{entry}{网际网络}{6,7,6,9}{⽹、⾩、⽹、⽷}
  \begin{phonetics}{网际网络}{wang3ji4wang3luo4}
    \definition*{s.}{Internet}
  \seealsoref{互联网}{hu4lian2wang3}
  \seealsoref{网际网路}{wang3ji4wang3lu4}
  \seealsoref{网路}{wang3lu4}
  \end{phonetics}
\end{entry}

\begin{entry}{网际网路}{6,7,6,13}{⽹、⾩、⽹、⾜}
  \begin{phonetics}{网际网路}{wang3ji4wang3lu4}
    \definition*{s.}{Internet}
  \seealsoref{互联网}{hu4lian2wang3}
  \seealsoref{网际网络}{wang3ji4wang3luo4}
  \seealsoref{网路}{wang3lu4}
  \end{phonetics}
\end{entry}

\begin{entry}{网络}{6,9}{⽹、⽷}
  \begin{phonetics}{网络}{wang3luo4}[][HSK 4]
    \definition{s.}{rede; um sistema que consiste em ramificações interconectadas; em um sistema elétrico, um circuito ou parte de um circuito que consiste em vários elementos que permitem a transmissão de sinais elétricos de acordo com determinados requisitos | rede; rede de computadores}
  \end{phonetics}
\end{entry}

\begin{entry}{网站}{6,10}{⽹、⽴}
  \begin{phonetics}{网站}{wang3zhan4}[][HSK 2]
    \definition[个,家]{s.}{\emph{web}; \emph{website}; um site virtual na Internet para uma organização ou indivíduo, geralmente consistindo em uma página inicial e muitas páginas da web}
  \end{phonetics}
\end{entry}

\begin{entry}{网罟}{6,10}{⽹、⽹}
  \begin{phonetics}{网罟}{wang3gu3}
    \definition{s.}{(fig.) a rede da justiça | rede usada para capturar peixes (ou outros animais, como pássaros)}
  \end{phonetics}
\end{entry}

\begin{entry}{网球}{6,11}{⽹、⽟}
  \begin{phonetics}{网球}{wang3qiu2}[][HSK 2]
    \definition[个,颗,些]{s.}{tênis (esporte) | bola de tênis}
  \end{phonetics}
\end{entry}

\begin{entry}{网银}{6,11}{⽹、⾦}
  \begin{phonetics}{网银}{wang3yin2}
    \definition{s.}{banco \emph{online} | acesso a operações bancárias via \emph{Internet}}
  \seealsoref{网上银行}{wang3shang4yin2hang2}
  \end{phonetics}
\end{entry}

\begin{entry}{网路}{6,13}{⽹、⾜}
  \begin{phonetics}{网路}{wang3lu4}
    \definition{s.}{\emph{Internet}}
  \seealsoref{互联网}{hu4lian2wang3}
  \seealsoref{网际网路}{wang3ji4wang3lu4}
  \seealsoref{网际网络}{wang3ji4wang3luo4}
  \end{phonetics}
\end{entry}

\begin{entry}{羊}{6}{⽺}[Kangxi 123]
  \begin{phonetics}{羊}{yang2}[][HSK 3]
    \definition*{s.}{Sobrenome Yang}
    \definition[只,头,群]{s.}{carneiro; ovelha; bode; cabra; antílope}
  \end{phonetics}
\end{entry}

\begin{entry}{羽}{6}{⽻}
  \begin{phonetics}{羽}{yu3}
    \definition*{s.}{Sobrenome Yu}
    \definition{s.}{pena; pluma | asas (de pássaros ou insetos) | uma nota da antiga escala chinesa de cinco tons, correspondente a 6 na notação musical numerada}
  \end{phonetics}
\end{entry}

\begin{entry}{羽毛}{6,4}{⽻、⽑}
  \begin{phonetics}{羽毛}{yu3mao2}
    \definition{s.}{pena | plumagem | pluma}
  \end{phonetics}
\end{entry}

\begin{entry}{羽毛笔}{6,4,10}{⽻、⽑、⽵}
  \begin{phonetics}{羽毛笔}{yu3mao2bi3}
    \definition{s.}{caneta de pena}
  \end{phonetics}
\end{entry}

\begin{entry}{羽毛球}{6,4,11}{⽻、⽑、⽟}
  \begin{phonetics}{羽毛球}{yu3mao2qiu2}[][HSK 5]
    \definition[只]{s.}{\emph{badminton}; esporte com bola, as regras e equipamentos são bastante semelhantes ao tênis | peteca}
  \end{phonetics}
\end{entry}

\begin{entry}{羽林}{6,8}{⽻、⽊}
  \begin{phonetics}{羽林}{yu3lin2}
    \definition{s.}{escolta armada}
  \end{phonetics}
\end{entry}

\begin{entry}{羽冠}{6,9}{⽻、⼍}
  \begin{phonetics}{羽冠}{yu3guan1}
    \definition{s.}{crista emplumada (de pássaro)}
  \end{phonetics}
\end{entry}

\begin{entry}{羽绒服}{6,9,8}{⽻、⽷、⽉}
  \begin{phonetics}{羽绒服}{yu3rong2fu2}[][HSK 5]
    \definition[件]{s.}{jaqueta de plumas; peça de vestuário com enchimento de plumas; casaco cujo interior é preenchido com penas de pato ou ganso}
  \end{phonetics}
\end{entry}

\begin{entry}{羽流}{6,10}{⽻、⽔}
  \begin{phonetics}{羽流}{yu3liu2}
    \definition{s.}{pluma}
  \end{phonetics}
\end{entry}

\begin{entry}{老}{6}{⽼}[Kangxi 125]
  \begin{phonetics}{老}{lao3}[][HSK 1,2]
    \definition*{s.}{Sobrenome Lao}
    \definition{adj.}{velho; envelhecido; idade avançada | antigo; de longa data; existe há muito tempo | antigo; desatualizado; obsoleto; ultrapassado  | antigo; tradicional; original | coberto de vegetação; os vegetais cresceram além do período ideal para serem consumidos | resistente; endurecido; alimentos muito cozidos | escuro; profundo; (sobre cores) | último nascido; o mais novo | veterano; experiente; sofisticado}
    \definition{adv.}{longo; por muito tempo | sempre (fazendo algo) | muito}
    \definition{pref.}{usado para designar pessoas, ordem de classificação, certos nomes de animais e plantas}
    \definition{s.}{idosos; pessoas mais velhas | ancião; sênior; um título respeitoso para pessoas mais velhas}
    \definition{v.}{envelhecer | morrer; referindo-se à morte de um idoso}
  \end{phonetics}
\end{entry}

\begin{entry}{老人}{6,2}{⽼、⼈}
  \begin{phonetics}{老人}{lao3 ren2}[][HSK 1]
    \definition[位]{s.}{homem ou mulher de idade avançada; o idoso; o velho}
  \end{phonetics}
\end{entry}

\begin{entry}{老人家}{6,2,10}{⽼、⼈、⼧}
  \begin{phonetics}{老人家}{lao3 ren2 jia1}
    \definition{s.}{senhor ancião | madame | senhora | termo educado para mulher ou homem velho}
  \end{phonetics}
\end{entry}

\begin{entry}{老公}{6,4}{⽼、⼋}
  \begin{phonetics}{老公}{lao3 gong1}[][HSK 4]
    \definition[个]{s.}{marido; esposo}
  \end{phonetics}
\end{entry}

\begin{entry}{老太太}{6,4,4}{⽼、⼤、⼤}
  \begin{phonetics}{老太太}{lao3 tai4 tai5}[][HSK 3]
    \definition[位,名,个]{s.}{velha senhora; (em tratamento direto)Venerável Senhora; uma maneira respeitosa de chamar uma senhora idosa; título honorífico para mulheres idosas | (forma de tratamento) sua velha mãe; minha velha mãe, avó ou sogra; referindo-se à própria mãe, à mãe do outro ou à mãe de outra pessoa, à sogra ou à sogra política}
  \end{phonetics}
\end{entry}

\begin{entry}{老头儿}{6,5,2}{⽼、⼤、⼉}
  \begin{phonetics}{老头儿}{lao3 tou2r5}[][HSK 3]
    \definition{s.}{(coloquial) (com um tom de intimidade) velho; velho amigo}
  \seealsoref{老头子}{lao3 tou2zi5}
  \end{phonetics}
\end{entry}

\begin{entry}{老头子}{6,5,3}{⽼、⼤、⼦}
  \begin{phonetics}{老头子}{lao3 tou2zi5}
    \definition{s.}{velho antiquado (ou velho rabugento) | (referindo-se ao marido idoso) meu velho | chefe de uma sociedade secreta | (coloquial) velho; velho rabugento}
  \seealsoref{老头儿}{lao3 tou2r5}
  \end{phonetics}
\end{entry}

\begin{entry}{老师}{6,6}{⽼、⼱}
  \begin{phonetics}{老师}{lao3shi1}[][HSK 1]
    \definition[个,位]{s.}{professor; título honorífico para professores; refere-se, de maneira geral, a pessoas que transmitem cultura e tecnologia ou que são dignas de admiração em termos de ideias, moralidade e conhecimentos profissionais}
  \end{phonetics}
\end{entry}

\begin{entry}{老年}{6,6}{⽼、⼲}
  \begin{phonetics}{老年}{lao3 nian2}[][HSK 2]
    \definition[个]{s.}{idoso; velhice; idade acima de 60 ou 70 anos}
  \end{phonetics}
\end{entry}

\begin{entry}{老百姓}{6,6,8}{⽼、⽩、⼥}
  \begin{phonetics}{老百姓}{lao3bai3xing4}[][HSK 3]
    \definition[些]{s.}{povo; civis; pessoas comuns; residentes (em contraste com militares e funcionários públicos)}
  \end{phonetics}
\end{entry}

\begin{entry}{老兵}{6,7}{⽼、⼋}
  \begin{phonetics}{老兵}{lao3bing1}
    \definition{s.}{velho soldado | veterano de guerra | veterano (alguém que tem muita experiência em algum domínio)}
  \end{phonetics}
\end{entry}

\begin{entry}{老实}{6,8}{⽼、⼧}
  \begin{phonetics}{老实}{lao3shi5}[][HSK 4]
    \definition{adj.}{franco; sincero; honesto | bom; bem-comportado | ingênuo; simplório; meio bobo; facilmente enganado; eufemismo para pouco inteligente}
  \end{phonetics}
\end{entry}

\begin{entry}{老朋友}{6,8,4}{⽼、⽉、⼜}
  \begin{phonetics}{老朋友}{lao3 peng2 you3}[][HSK 2]
    \definition[个,位,名]{s.}{velho amigo; refere-se a amigos que conhecemos há muito tempo e com quem temos uma relação íntima}
  \end{phonetics}
\end{entry}

\begin{entry}{老板}{6,8}{⽼、⽊}
  \begin{phonetics}{老板}{lao3ban3}[][HSK 3]
    \definition[个,位]{s.}{chefe; dono; líder; refere-se ao gerente de uma empresa comercial ou industrial | antigo título honorífico dado a atores famosos de ópera ou atores que também eram diretores de companhias de ópera}
  \end{phonetics}
\end{entry}

\begin{entry}{老虎}{6,8}{⽼、⾌}
  \begin{phonetics}{老虎}{lao3hu3}
    \definition[只]{s.}{tigre}
  \seealsoref{虎}{hu3}
  \end{phonetics}
\end{entry}

\begin{entry}{老是}{6,9}{⽼、⽇}
  \begin{phonetics}{老是}{lao3 shi4}[][HSK 2]
    \definition{adv.}{sempre; indica que a ação continua ou que o estado permanece inalterado, equivalente a 一直}
  \seealsoref{一直}{yi4zhi2}
  \end{phonetics}
\end{entry}

\begin{entry}{老家}{6,10}{⽼、⼧}
  \begin{phonetics}{老家}{lao3 jia1}[][HSK 4]
    \definition{s.}{cidade natal; local de origem | lugar nativo; refere-se às gerações anteriores da família ou ao local onde a pessoa nasceu ou viveu}
  \end{phonetics}
\end{entry}

\begin{entry}{老婆}{6,11}{⽼、⼥}
  \begin{phonetics}{老婆}{lao3po2}[][HSK 4]
    \definition[个]{s.}{esposa}
  \end{phonetics}
\end{entry}

\begin{entry}{考}{6}{⽼}
  \begin{phonetics}{考}{kao3}[][HSK 1]
    \definition*{s.}{Sobrenome Kao}
    \definition{adj.}{antigo; velho; com idade avançada}
    \definition{s.}{o pai falecido de alguém}
    \definition{v.}{examinar; dar (fazer) um exame, teste ou questionário | verificar; inspecionar | estudar; verificar; investigar | perguntar; testar; fazer perguntas para que o outro responda, a fim de testar suas habilidades em determinada área}
  \end{phonetics}
\end{entry}

\begin{entry}{考生}{6,5}{⽼、⽣}
  \begin{phonetics}{考生}{kao3 sheng1}[][HSK 2]
    \definition{s.}{candidato a exame; alunos inscritos para o exame de admissão}
  \end{phonetics}
\end{entry}

\begin{entry}{考试}{6,8}{⽼、⾔}
  \begin{phonetics}{考试}{kao3shi4}[][HSK 1]
    \definition[次]{s.}{teste; exame; prova; atividades realizadas para verificar conhecimentos ou habilidades}
    \definition{v.+compl.}{testar; avaliar; avaliar conhecimentos e habilidades por meio de perguntas escritas ou orais.}
  \end{phonetics}
\end{entry}

\begin{entry}{考核}{6,10}{⽼、⽊}
  \begin{phonetics}{考核}{kao3he2}[][HSK 5]
    \definition{v.}{examinar; checar; avaliar; avaliar (a proficiência de alguém)}
  \end{phonetics}
\end{entry}

\begin{entry}{考虑}{6,10}{⽼、⾌}
  \begin{phonetics}{考虑}{kao3lv4}[][HSK 4]
    \definition{v.}{considerar; refletir sobre; levar em conta}
  \end{phonetics}
\end{entry}

\begin{entry}{考验}{6,10}{⽼、⾺}
  \begin{phonetics}{考验}{kao3yan4}[][HSK 3]
    \definition[场,个,种]{s.}{teste; julgamento; atividade realizada para verificar se as habilidades, ideias, moral e qualidades de uma pessoa atendem aos requisitos}
    \definition{v.}{testar; testar as capacidades, ideias, moral e qualidades de uma pessoa através de situações, ações ou ambientes difíceis, para verificar se elas atendem aos requisitos}
  \end{phonetics}
\end{entry}

\begin{entry}{考察}{6,14}{⽼、⼧}
  \begin{phonetics}{考察}{kao3cha2}[][HSK 4]
    \definition{v.}{inspecionar; investigar; observar e estudar}
  \end{phonetics}
\end{entry}

\begin{entry}{而}{6}{⽽}[Kangxi 126]
  \begin{phonetics}{而}{er2}[][HSK 4]
    \definition{conj.}{e (coordenação) | e ainda (restrição) | conexão de componentes com continuidade semântica | conecxão de componentes afirmativos e negativos que se complementam | conexão de componentes com significados opostos para indicar um contraste |  conexão de componentes de causa e efeito no raciocínio | significa “chegar” ou “alcançar” | conexão de componentes que indicam tempo ou modo ao verbo | inserido entre o sujeito e o predicado, significa 如果}
  \seealsoref{如果}{ru2guo3}
  \end{phonetics}
\end{entry}

\begin{entry}{而且}{6,5}{⽽、⼀}
  \begin{phonetics}{而且}{er2 qie3}[][HSK 2]
    \definition{conj.}{e também; indica igualdade | e isso; não só\dots mas (também); indica um passo adiante}
  \end{phonetics}
\end{entry}

\begin{entry}{而况}{6,7}{⽽、⼎}
  \begin{phonetics}{而况}{er2kuang4}
    \definition{conj.}{além disso | além do mais}
  \end{phonetics}
\end{entry}

\begin{entry}{而是}{6,9}{⽽、⽇}
  \begin{phonetics}{而是}{er2 shi4}[][HSK 4]
    \definition{conj.}{mas; em vez disso; geralmente usada em conjunto com 不是 para formar o correlativo 不是……而是, indicando uma relação paralela}
  \seealsoref{不是……而是}{bu4shi4 er2 shi4}
  \end{phonetics}
\end{entry}

\begin{entry}{耳}{6}{⽿}
  \begin{phonetics}{耳}{er3}
    \definition*{s.}{Sobrenome Er}
    \definition{part.}{(clássico) somente; apenas}
    \definition{s.}{orelha | coisa parecida com uma orelha | em ambos os lados; lado | orelha de um utensílio}
  \end{phonetics}
\end{entry}

\begin{entry}{耳朵}{6,6}{⽿、⽊}
  \begin{phonetics}{耳朵}{er3duo5}[][HSK 5]
    \definition[双,只,个,对]{s.}{orelha; ouvido; órgão da audição e do equilíbrio}
  \end{phonetics}
\end{entry}

\begin{entry}{耳机}{6,6}{⽿、⽊}
  \begin{phonetics}{耳机}{er3 ji1}[][HSK 4]
    \definition[副,个,对]{s.}{fone de ouvido; receptor (de telefone); dispositivos que permitem que uma pessoa ouça sons sozinha, como ouvir música, histórias, chamadas telefônicas etc., usados na cabeça ou inseridos nos ouvidos}
  \end{phonetics}
\end{entry}

\begin{entry}{肉}{6}{⾁}[Kangxi 130]
  \begin{phonetics}{肉}{rou4}[][HSK 1]
    \definition{adj.}{não crocante; mole | lento (em movimento); preguiçoso | carnal; erótico}
    \definition[块]{s.}{carne (especialmente carne de porco) | carne | polpa (da fruta)}
  \end{phonetics}
\end{entry}

\begin{entry}{肉桂}{6,10}{⾁、⽊}
  \begin{phonetics}{肉桂}{rou4gui4}
    \definition{s.}{canela}
  \seealsoref{官桂}{guan1gui4}
  \end{phonetics}
\end{entry}

\begin{entry}{肌}{6}{⾁}
  \begin{phonetics}{肌}{ji1}
    \definition[块,片]{s.}{músculo; carne | pele;}
  \end{phonetics}
\end{entry}

\begin{entry}{肌肉}{6,6}{⾁、⾁}
  \begin{phonetics}{肌肉}{ji1rou4}[][HSK 5]
    \definition[身,块]{s.}{músculo; um dos tecidos básicos dos músculos humanos e de alguns animais, composto principalmente de células musculares fibrosas, pode se contrair, é o movimento do corpo e o corpo de digestão, respiração, circulação, excreção e outros processos fisiológicos da fonte de energia; pode ser dividido em três tipos: músculo liso, músculo esquelético e músculo cardíaco}
  \end{phonetics}
\end{entry}

\begin{entry}{自}{6}{⾃}[Kangxi 132]
  \begin{phonetics}{自}{zi4}[][HSK 4]
    \definition*{s.}{Sobrenome Zi}
    \definition{adv.}{certamente; com certeza; é claro; naturalmente}
    \definition{prep.}{de; desde; a partir de; apresenta o ponto de partida, a fonte ou o horário de início do comportamento da ação, equivalente a 从 e 由}
    \definition{pron.}{si mesmo; próprio | próprio; indica que a ação é iniciada por e direcionada a si mesmo | por si mesmo; indica que a ação é autoiniciada e não é causada por uma força externa}
    \definition{v.}{iniciar}
  \seealsoref{从}{cong2}
  \seealsoref{由}{you2}
  \end{phonetics}
\end{entry}

\begin{entry}{自个儿}{6,3,2}{⾃、⼈、⼉}
  \begin{phonetics}{自个儿}{zi4ge3r5}
    \definition{pron.}{(dialeto) a si mesmo, por si mesmo}
  \end{phonetics}
\end{entry}

\begin{entry}{自己}{6,3}{⾃、⼰}
  \begin{phonetics}{自己}{zi4ji3}[][HSK 2]
    \definition{pron.}{a si próprio; a si mesmo; refere-se ao substantivo ou pronome precedente (enfatiza principalmente que não é devido a forças externas)}
  \end{phonetics}
\end{entry}

\begin{entry}{自己动手}{6,3,6,4}{⾃、⼰、⼒、⼿}
  \begin{phonetics}{自己动手}{zi4ji3dong4shou3}
    \definition{v.}{fazer (algo) sozinho | ajudar-se a}
  \end{phonetics}
\end{entry}

\begin{entry}{自从}{6,4}{⾃、⼈}
  \begin{phonetics}{自从}{zi4cong2}[][HSK 3]
    \definition{prep.}{de; desde; a partir de; referir-se a um momento ou evento específico no passado}
  \end{phonetics}
\end{entry}

\begin{entry}{自主}{6,5}{⾃、⼂}
  \begin{phonetics}{自主}{zi4zhu3}[][HSK 3]
    \definition{v.}{agir por conta própria; decidir por si mesmo; manter a iniciativa em suas próprias mãos; tomar suas próprias decisões}
  \end{phonetics}
\end{entry}

\begin{entry}{自由}{6,5}{⾃、⽥}
  \begin{phonetics}{自由}{zi4you2}[][HSK 2]
    \definition{adj.}{livre; irrestrito}
    \definition[个]{s.}{liberdade; o direito de agir de acordo com a própria vontade dentro do âmbito da lei | liberdade; filosoficamente, liberdade é definida como o processo de as pessoas reconhecerem as leis que governam o desenvolvimento das coisas e aplicá-las conscientemente na prática}
  \end{phonetics}
\end{entry}

\begin{entry}{自由泳}{6,5,8}{⾃、⽥、⽔}
  \begin{phonetics}{自由泳}{zi4you2yong3}
    \definition{s.}{natação de estilo livre}
  \end{phonetics}
\end{entry}

\begin{entry}{自动}{6,6}{⾃、⼒}
  \begin{phonetics}{自动}{zi4dong4}[][HSK 3]
    \definition{adj.}{automático; auto-atuante; uso de dispositivos mecânicos, elétricos, etc, para funcionar automaticamente, sem necessidade de controle humano}
    \definition{adv.}{voluntariamente; por vontade própria; por iniciativa própria | automaticamente; espontaneamente; refere-se a movimentos, mudanças, etc., que não são causados pela ação humana, mas sim pelo próprio objeto}
  \end{phonetics}
\end{entry}

\begin{entry}{自动化}{6,6,4}{⾃、⼒、⼔}
  \begin{phonetics}{自动化}{zi4dong4hua4}
    \definition{s.}{automação}
  \end{phonetics}
\end{entry}

\begin{entry}{自杀}{6,6}{⾃、⽊}
  \begin{phonetics}{自杀}{zi4 sha1}[][HSK 5]
    \definition{s.}{suicídio; auto-assassinato; auto-sacrifício}
    \definition{v.}{cometer suicídio; tentar suicídio; matar-se}
  \end{phonetics}
\end{entry}

\begin{entry}{自行车}{6,6,4}{⾃、⾏、⾞}
  \begin{phonetics}{自行车}{zi4xing2che1}[][HSK 2]
    \definition[辆]{s.}{bicicleta; um veículo de duas rodas que é impulsionado para a frente com os pedais}
  \end{phonetics}
\end{entry}

\begin{entry}{自行车架}{6,6,4,9}{⾃、⾏、⾞、⽊}
  \begin{phonetics}{自行车架}{zi4xing2che1jia4}
    \definition{s.}{suporte para bicicleta | bicicletário}
  \end{phonetics}
\end{entry}

\begin{entry}{自行车馆}{6,6,4,11}{⾃、⾏、⾞、⾷}
  \begin{phonetics}{自行车馆}{zi4xing2che1guan3}
    \definition{s.}{estádio de ciclismo | velódromo}
  \end{phonetics}
\end{entry}

\begin{entry}{自行车赛}{6,6,4,14}{⾃、⾏、⾞、⾙}
  \begin{phonetics}{自行车赛}{zi4xing2che1sai4}
    \definition{s.}{corrida de bicicleta}
  \end{phonetics}
\end{entry}

\begin{entry}{自我}{6,7}{⾃、⼽}
  \begin{phonetics}{自我}{zi4wo3}
    \definition{pref.}{auto}
    \definition{pron.}{a si mesmo | eu próprio | (psicologia) ego}
  \end{phonetics}
\end{entry}

\begin{entry}{自我介绍}{6,7,4,8}{⾃、⼽、⼈、⽷}
  \begin{phonetics}{自我介绍}{zi4wo3jie4shao4}
    \definition{s.}{defesa pessoal | auto-defesa}
  \end{phonetics}
\end{entry}

\begin{entry}{自我安慰}{6,7,6,15}{⾃、⼽、⼧、⼼}
  \begin{phonetics}{自我安慰}{zi4wo3'an1wei4}
    \definition{v.}{confortar-se | consolar-se | tranquilizar-se}
  \end{phonetics}
\end{entry}

\begin{entry}{自我防卫}{6,7,6,3}{⾃、⼽、⾩、⼙}
  \begin{phonetics}{自我防卫}{zi4wo3fang2wei4}
    \definition{s.}{defesa pessoal | auto-defesa}
  \end{phonetics}
\end{entry}

\begin{entry}{自我吹嘘}{6,7,7,14}{⾃、⼽、⼝、⼝}
  \begin{phonetics}{自我吹嘘}{zi4wo3chui1xu1}
    \definition{expr.}{tocar a própria buzina}
  \end{phonetics}
\end{entry}

\begin{entry}{自我批评}{6,7,7,7}{⾃、⼽、⼿、⾔}
  \begin{phonetics}{自我批评}{zi4wo3pi1ping2}
    \definition{s.}{autocrítica}
  \end{phonetics}
\end{entry}

\begin{entry}{自我实现}{6,7,8,8}{⾃、⼽、⼧、⾒}
  \begin{phonetics}{自我实现}{zi4wo3shi2xian4}
    \definition{s.}{(psicologia) auto-realização}
  \end{phonetics}
\end{entry}

\begin{entry}{自我的人}{6,7,8,2}{⾃、⼽、⽩、⼈}
  \begin{phonetics}{自我的人}{zi4wo3de5ren2}
    \definition{s.}{(minha, sua) própria pessoa | (afirmar) a própria personalidade}
  \end{phonetics}
\end{entry}

\begin{entry}{自我保存}{6,7,9,6}{⾃、⼽、⼈、⼦}
  \begin{phonetics}{自我保存}{zi4wo3 bao3cun2}
    \definition{v.}{autopreservação}
  \end{phonetics}
\end{entry}

\begin{entry}{自我陶醉}{6,7,10,15}{⾃、⼽、⾩、⾣}
  \begin{phonetics}{自我陶醉}{zi4wo3tao2zui4}
    \definition{s.}{narcisista | auto-imbuído | satisfeito consigo mesmo}
  \end{phonetics}
\end{entry}

\begin{entry}{自我催眠}{6,7,13,10}{⾃、⼽、⼈、⽬}
  \begin{phonetics}{自我催眠}{zi4wo3cui1mian2}
    \definition{v.}{consolar-me | tranquilizar-me}
  \end{phonetics}
\end{entry}

\begin{entry}{自我意识}{6,7,13,7}{⾃、⼽、⼼、⾔}
  \begin{phonetics}{自我意识}{zi4wo3yi4shi2}
    \definition{s.}{autoapresentação}
    \definition{v.}{apresentar-se}
  \end{phonetics}
\end{entry}

\begin{entry}{自我解嘲}{6,7,13,15}{⾃、⼽、⾓、⼝}
  \begin{phonetics}{自我解嘲}{zi4wo3jie3chao2}
    \definition{s.}{referir-se às próprias fraquezas ou falhas com humor autodepreciativo}
  \end{phonetics}
\end{entry}

\begin{entry}{自来水}{6,7,4}{⾃、⽊、⽔}
  \begin{phonetics}{自来水}{zi4lai2shui3}
    \definition{s.}{água corrente | água da torneira}
  \end{phonetics}
\end{entry}

\begin{entry}{自身}{6,7}{⾃、⾝}
  \begin{phonetics}{自身}{zi4 shen1}[][HSK 3]
    \definition{pron.}{eu mesmo (enfatizando que não é outra pessoa ou outra coisa)}
  \end{phonetics}
\end{entry}

\begin{entry}{自责}{6,8}{⾃、⾙}
  \begin{phonetics}{自责}{zi4ze2}
    \definition{v.}{culpar-se}
  \end{phonetics}
\end{entry}

\begin{entry}{自信}{6,9}{⾃、⼈}
  \begin{phonetics}{自信}{zi4xin4}[][HSK 4]
    \definition{adj.}{confiante; descreve a crença em suas próprias habilidades, decisões, etc., tendo confiança em si mesmo}
    \definition[份,种]{s.}{autoconfiança; confiança em si mesmo}
    \definition{v.}{acreditar em si mesmo;}
  \end{phonetics}
\end{entry}

\begin{entry}{自觉}{6,9}{⾃、⾒}
  \begin{phonetics}{自觉}{zi4jue2}[][HSK 3]
    \definition{adj.}{autoconsciente; de ​​livre e espontânea vontade; controlar o próprio comportamento e agir por iniciativa própria}
    \definition{v.}{estar ciente de}
  \end{phonetics}
\end{entry}

\begin{entry}{自救}{6,11}{⾃、⽁}
  \begin{phonetics}{自救}{zi4jiu4}
    \definition{v.}{sair a si mesmo de problemas}
  \end{phonetics}
\end{entry}

\begin{entry}{自然}{6,12}{⾃、⽕}
  \begin{phonetics}{自然}{zi4ran2}[][HSK 3]
    \definition{adj.}{natural; no curso normal dos eventos; formado ou desenvolvido sem intervenção humana; algo que se desenvolve livremente}
    \definition{adv.}{naturalmente; definitivamente; certamente, isso significa que, de acordo com a lógica, deve ser assim}
    \definition{conj.}{usado para ligar duas frases, com a segunda introduzindo informações adicionais ou adversativas; indica explicação complementar ou uma mudança de significado}
    \definition{s.}{natureza; mundo natural; tudo o que não foi criado pelo ser humano}
  \end{phonetics}
\end{entry}

\begin{entry}{自愿}{6,14}{⾃、⽕}
  \begin{phonetics}{自愿}{zi4yuan4}[][HSK 5]
    \definition{adv.}{voluntariamente; por iniciativa própria; por vontade própria}
    \definition{s.}{voluntário}
  \end{phonetics}
\end{entry}

\begin{entry}{自豪}{6,14}{⾃、⾗}
  \begin{phonetics}{自豪}{zi4hao2}[][HSK 5]
    \definition{adj.}{orgulhar-se de; ter orgulho de; sentir-se honrado por possuir qualidades excelentes ou ter alcançado grandes conquistas, seja por si mesmo ou por um grupo ou indivíduo relacionado a si}
  \end{phonetics}
\end{entry}

\begin{entry}{自燃}{6,16}{⾃、⽕}
  \begin{phonetics}{自燃}{zi4ran2}
    \definition{s.}{combustão espontânea}
  \end{phonetics}
\end{entry}

\begin{entry}{至}{6}{⾄}
  \begin{phonetics}{至}{zhi4}[][HSK 5]
    \definition{adv.}{a maior parte; extremamente; indica o grau mais alto, equivalente a 极 ou 最}
    \definition{prep.}{para; até; chegar a um determinado ponto}
    \definition{s.}{extremo, máximo}
    \definition{v.}{chegar; alcançar}
  \seealsoref{极}{ji2}
  \seealsoref{最}{zui4}
  \end{phonetics}
\end{entry}

\begin{entry}{至于}{6,3}{⾄、⼆}
  \begin{phonetics}{至于}{zhi4yu2}
    \definition{conj.}{para | quanto a | a respeiro de}
  \end{phonetics}
\end{entry}

\begin{entry}{至今}{6,4}{⾄、⼈}
  \begin{phonetics}{至今}{zhi4jin1}[][HSK 3]
    \definition{adv.}{até agora; até o momento; até hoje}
  \end{phonetics}
\end{entry}

\begin{entry}{至少}{6,4}{⾄、⼩}
  \begin{phonetics}{至少}{zhi4shao3}[][HSK 3]
    \definition{adv.}{pelo menos; indica o limite mínimo}
  \end{phonetics}
\end{entry}

\begin{entry}{舌}{6}{⾆}[Kangxi 135]
  \begin{phonetics}{舌}{she2}
    \definition*{s.}{Sobrenome She}
    \definition[片,条]{s.}{língua (de um ser humano ou animal); glossa | algo em forma de língua | língua de sino; badalo}
  \end{phonetics}
\end{entry}

\begin{entry}{舌头}{6,5}{⾆、⼤}
  \begin{phonetics}{舌头}{she2tou5}
    \definition[个]{s.}{língua | soldado inimigo capturado com o propósito de extrair informações}
  \end{phonetics}
\end{entry}

\begin{entry}{色}{6}{⾊}[Kangxi 139]
  \begin{phonetics}{色}{se4}[][HSK 4]
    \definition[种]{s.}{cor | aparência; semblante; expressão | tipo; gênero; descrição | cena; cenário;  paisagem | qualidade (de metais preciosos, mercadorias, etc.) | aparência feminina; beleza feminina}
  \end{phonetics}
  \begin{phonetics}{色}{shai3}
    \definition[4]{s.}{cor}
  \end{phonetics}
\end{entry}

\begin{entry}{色狼}{6,10}{⾊、⽝}
  \begin{phonetics}{色狼}{se4lang2}
    \definition*{s.}{Sátiro}
    \definition{adj.}{depravado | tarado}
  \end{phonetics}
\end{entry}

\begin{entry}{色彩}{6,11}{⾊、⼺}
  \begin{phonetics}{色彩}{se4cai3}[][HSK 4]
    \definition[种,丝]{s.}{cor; matiz; tonalidade | cor; sabor; característica; metáfora para um determinado estado de espírito ou tendência de pensamento}
  \end{phonetics}
\end{entry}

\begin{entry}{芋}{6}{⾋}
  \begin{phonetics}{芋}{yu4}
    \definition*{s.}{Sobrenome Yu}
    \definition{s.}{taro; erva perene | tubérculos; geralmente se refere a batatas, etc.}
  \end{phonetics}
\end{entry}

\begin{entry}{芋头}{6,5}{⾋、⼤}
  \begin{phonetics}{芋头}{yu4tou5}
    \definition{s.}{taro, similar ao inhame e batata doce}
  \end{phonetics}
\end{entry}

\begin{entry}{芋头色}{6,5,6}{⾋、⼤、⾊}
  \begin{phonetics}{芋头色}{yu4tou5se4}
    \definition{s.}{cor lilás}
  \end{phonetics}
\end{entry}

\begin{entry}{芝}{6}{⾋}
  \begin{phonetics}{芝}{zhi1}
    \definition*{s.}{Sobrenome Zhi}
    \definition{s.}{Arcaico: fungo mágico, ganoderma brilhante | Arcaico: raiz de angélica dahuriana}
  \end{phonetics}
\end{entry}

\begin{entry}{芝麻}{6,11}{⾋、⿇}
  \begin{phonetics}{芝麻}{zhi1ma5}
    \definition{s.}{semente de gergelim}
  \end{phonetics}
\end{entry}

\begin{entry}{虫}{6}{⾍}[Kangxi 142]
  \begin{phonetics}{虫}{chong2}
    \definition[只,条]{s.}{inseto; verme | (pejorativo) pessoas que se comportam de forma desprezível | fã; viciado | forma inferior de vida animal, incluindo insetos, larvas de insetos, vermes e criaturas semelhantes | pessoa com uma característica indesejável específica}
  \end{phonetics}
\end{entry}

\begin{entry}{虫子}{6,3}{⾍、⼦}
  \begin{phonetics}{虫子}{chong2 zi5}[][HSK 4]
    \definition[条,只,种]{s.}{percevejo; besouro; inseto; verme; criaturas semelhantes a insetos}
  \end{phonetics}
\end{entry}

\begin{entry}{血}{6}{⾎}[Kangxi 143]
  \begin{phonetics}{血}{xie3}
  \end{phonetics}
  \begin{phonetics}{血}{xue4}[][HSK 3]
    \definition[滴,袋,口,毫升]{s.}{sangue | parente consanguíneo; com laços de parentesco | pessoa ativa e animada; metáfora para uma personalidade ou espírito forte e sincero | medicina tradicional chinesa refere-se à menstruação}
  \end{phonetics}
\end{entry}

\begin{entry}{血汗}{6,6}{⾎、⽔}
  \begin{phonetics}{血汗}{xue4han4}
    \definition{s.}{(fig.) suor e labuta, trabalho duro}
  \end{phonetics}
\end{entry}

\begin{entry}{行}{6}{⾏}
  \begin{phonetics}{行}{hang2}[][HSK 3]
    \definition{adj.}{temporário; improvisado | capaz; competente}
    \definition{adv.}{logo; em breve}
    \definition{clas.}{linha; fileira; coisas usadas para formar filas, linhas}
    \definition{s.}{comportamento; conduta | linha; fileira | empresa comercial; certas instituições comerciais | comércio; profissão; ramo de atividade | especialista; conhecedor; refere-se ao conhecimento e experiência em um determinado setor}
    \definition{v.}{ir; caminhar; viajar | estar atualizado; circular | fazer; executar; realizar | (antes de um verbo dissílabo, indicando a realização de alguma ação) | ficar bem; vai dar certo | (remédio) fazer efeito | classificar (entre irmãos e irmãs por ordem de idade)}
  \end{phonetics}
  \begin{phonetics}{行}{heng2}
    \definition{s.}{usado em 道行}
  \seealsoref{道行}{dao4 heng2}
  \end{phonetics}
  \begin{phonetics}{行}{xing2}[][HSK 1]
    \definition*{s.}{Sobrenome Xing}
    \definition{adj.}{de viajar; relacionado a viagens | temporário; improvisado; provisório | capaz; competente}
    \definition{adv.}{em breve}
    \definition{s.}{comportamento; conduta | caligrafia cursiva (na caligrafia chinesa); escrita cursiva}
    \definition{v.}{ir | fazer uma viagem | estar em voga; prevalecer; circular | fazer; executar; realizar; envolver-se em | estar tudo bem; O.K. | indica a realização de uma determinada atividade (usado principalmente antes de verbos dissilábicos) | (em medicina) fazer efeito}
  \end{phonetics}
\end{entry}

\begin{entry}{行人}{6,2}{⾏、⼈}
  \begin{phonetics}{行人}{xing2ren2}[][HSK 2]
    \definition[个]{s.}{pedestre; transeunte; viajante à pé; pessoas caminhando na estrada}
  \end{phonetics}
\end{entry}

\begin{entry}{行为}{6,4}{⾏、⼂}
  \begin{phonetics}{行为}{xing2wei2}[][HSK 2]
    \definition[个,种,类]{s.}{ação; comportamento; conduta; atividades que são controladas por pensamentos e manifestadas externamente}
  \end{phonetics}
\end{entry}

\begin{entry}{行凶}{6,4}{⾏、⼐}
  \begin{phonetics}{行凶}{xing2xiong1}
    \definition{v.+compl.}{cometer agressão física ou assassinato | fazer algo violento}
  \end{phonetics}
\end{entry}

\begin{entry}{行业}{6,5}{⾏、⼀}
  \begin{phonetics}{行业}{hang2ye4}[][HSK 4]
    \definition[种,个]{s.}{comércio; indústria; setor; profissão; categorias em negócios e indústria referem-se a ocupações em geral}
  \end{phonetics}
\end{entry}

\begin{entry}{行礼}{6,5}{⾏、⽰}
  \begin{phonetics}{行礼}{xing2li3}
    \definition{v.}{saudar | fazer saudação}
  \end{phonetics}
\end{entry}

\begin{entry}{行动}{6,6}{⾏、⼒}
  \begin{phonetics}{行动}{xing2dong4}[][HSK 2]
    \definition[次,场,项]{s.}{ação; operação; comportamento;}
    \definition{v.}{circular; mover-se; andar | agir; tomar medidas; atividades para atingir um determinado propósito}
  \end{phonetics}
\end{entry}

\begin{entry}{行李}{6,7}{⾏、⽊}
  \begin{phonetics}{行李}{xing2li5}[][HSK 3]
    \definition[点,个]{s.}{bagagem, malas, cestas de vime, etc. que você leva quando sai de casa}
  \end{phonetics}
\end{entry}

\begin{entry}{行进}{6,7}{⾏、⾡}
  \begin{phonetics}{行进}{xing2jin4}
    \definition{s.}{avançar | movimentar-se para frente}
  \end{phonetics}
\end{entry}

\begin{entry}{行驶}{6,8}{⾏、⾺}
  \begin{phonetics}{行驶}{xing2 shi3}[][HSK 5]
    \definition{v.}{ir; navegar; viajar (utilizando um veículo, navio, etc.);}
  \end{phonetics}
\end{entry}

\begin{entry}{行星}{6,9}{⾏、⽇}
  \begin{phonetics}{行星}{xing2xing1}
    \definition[颗]{s.}{planeta}
  \seealsoref{惑星}{huo4xing1}
  \end{phonetics}
\end{entry}

\begin{entry}{衣}{6}{⾐}[Kangxi 145]
  \begin{phonetics}{衣}{yi1}
    \definition[件]{s.}{roupa}
  \end{phonetics}
  \begin{phonetics}{衣}{yi4}
    \definition{v.}{vestir | vestir-se}
  \end{phonetics}
\end{entry}

\begin{entry}{衣甲}{6,5}{⾐、⽥}
  \begin{phonetics}{衣甲}{yi1jia3}
    \definition{s.}{armadura}
  \end{phonetics}
\end{entry}

\begin{entry}{衣服}{6,8}{⾐、⽉}
  \begin{phonetics}{衣服}{yi1fu5}[][HSK 1]
    \definition[套,件]{s.}{roupas; vestuário; algo que se veste para cobrir o corpo e se proteger do frio}
  \end{phonetics}
\end{entry}

\begin{entry}{衣柜}{6,8}{⾐、⽊}
  \begin{phonetics}{衣柜}{yi1gui4}
    \definition[个]{s.}{armário | guarda-roupa}
  \end{phonetics}
\end{entry}

\begin{entry}{衣架}{6,9}{⾐、⽊}
  \begin{phonetics}{衣架}{yi1 jia4}[][HSK 3]
    \definition[个,副,组]{s.}{cabideiro; móvel para pendurar roupas | estatura; figura; refere-se ao tipo físico de uma pessoa; estrutura corporal}
  \end{phonetics}
\end{entry}

\begin{entry}{西}{6}{⾑}
  \begin{phonetics}{西}{xi1}[][HSK 1]
    \definition*{s.}{Espanha, abreviatura de 西班牙 | Paraíso Ocidental | Sobrenome Xi}
    \definition{s.}{oeste; uma das quatro direções básicas, o lado onde o sol se põe (oposto ao 东) | ocidental; refere-se ao Ocidente (principalmente aos países europeus e americanos) | aqui e ali; em contraposição a 东, significa 到处 ou 零散, 没有次序}
  \seealsoref{到处}{dao4chu4}
  \seealsoref{东}{dong1}
  \seealsoref{零散}{ling2san3}
  \seealsoref{没有次序}{mei2you3 ci4xu4}
  \seealsoref{西班牙}{xi1ban1ya2}
  \end{phonetics}
\end{entry}

\begin{entry}{西文}{6,4}{⾑、⽂}
  \begin{phonetics}{西文}{xi1wen2}
    \definition{s.}{espanhol | língua espanhola}
  \seealsoref{西班牙文}{xi1ban1ya2wen2}
  \end{phonetics}
\end{entry}

\begin{entry}{西方}{6,4}{⾑、⽅}
  \begin{phonetics}{西方}{xi1 fang1}[][HSK 2]
    \definition{s.}{oeste | o Ocidente; o Oeste; países europeus e americanos | Paraíso Ocidental, termo budista}
  \end{phonetics}
\end{entry}

\begin{entry}{西兰花}{6,5,7}{⾑、⼋、⾋}
  \begin{phonetics}{西兰花}{xi1lan2hua1}
    \definition{s.}{brócolis}
  \end{phonetics}
\end{entry}

\begin{entry}{西北}{6,5}{⾑、⼔}
  \begin{phonetics}{西北}{xi1 bei3}[][HSK 2]
    \definition{s.}{noroeste | noroeste da China; o Noroeste}
  \end{phonetics}
\end{entry}

\begin{entry}{西半球}{6,5,11}{⾑、⼗、⽟}
  \begin{phonetics}{西半球}{xi1ban4qiu2}
    \definition{s.}{hemisfério oeste}
  \end{phonetics}
\end{entry}

\begin{entry}{西瓜}{6,5}{⾑、⽠}
  \begin{phonetics}{西瓜}{xi1gua1}[][HSK 4]
    \definition[个,颗,粒]{s.}{melancia; fruto que é uma baga de formato grande, globular ou oval, com muita polpa aguada e doce}
  \end{phonetics}
\end{entry}

\begin{entry}{西边}{6,5}{⾑、⾡}
  \begin{phonetics}{西边}{xi1bian1}[][HSK 1]
    \definition{s.}{lado oeste; (oeste) Uma das quatro direções principais; uma das direções cardeais, oposta ao 东方}
  \seealsoref{东方}{dong1 fang1}
  \end{phonetics}
\end{entry}

\begin{entry}{西安}{6,6}{⾑、⼧}
  \begin{phonetics}{西安}{xi1'an1}
    \definition*{s.}{Xi'an, Capital da Província de Shaanxi}
  \end{phonetics}
\end{entry}

\begin{entry}{西红柿}{6,6,9}{⾑、⽷、⽊}
  \begin{phonetics}{西红柿}{xi1hong2shi4}[][HSK 5]
    \definition[种,只]{s.}{tomate;}
  \end{phonetics}
\end{entry}

\begin{entry}{西西}{6,6}{⾑、⾑}
  \begin{phonetics}{西西}{xi1xi1}
    \definition{num.}{centímetro cúbico}
  \end{phonetics}
\end{entry}

\begin{entry}{西医}{6,7}{⾑、⼖}
  \begin{phonetics}{西医}{xi1 yi1}[][HSK 2]
    \definition[名,位]{s.}{medicina ocidental; medicina introduzida na China a partir da Europa e da América | um médico treinado em medicina ocidental}
  \end{phonetics}
\end{entry}

\begin{entry}{西南}{6,9}{⾑、⼗}
  \begin{phonetics}{西南}{xi1 nan2}[][HSK 2]
    \definition{s.}{sudoeste | o Sudoeste; Sudoeste da China}
  \end{phonetics}
\end{entry}

\begin{entry}{西药}{6,9}{⾑、⾋}
  \begin{phonetics}{西药}{xi1 yao4}
    \definition{s.}{medicina ocidental; refere-se aos medicamentos usados ​​na medicina ocidental, geralmente feitos por métodos sintéticos ou extraídos de produtos naturais, como comprimidos anti-inflamatórios, aspirina, tintura de iodo, penicilina, etc.}
  \end{phonetics}
\end{entry}

\begin{entry}{西语}{6,9}{⾑、⾔}
  \begin{phonetics}{西语}{xi1yu3}
    \definition{s.}{espanhol | língua espanhola}
  \seealsoref{西班牙语}{xi1ban1ya2yu3}
  \end{phonetics}
\end{entry}

\begin{entry}{西面}{6,9}{⾑、⾯}
  \begin{phonetics}{西面}{xi1mian4}
    \definition{s.}{oeste | lado oeste}
  \end{phonetics}
\end{entry}

\begin{entry}{西班牙}{6,10,4}{⾑、⽟、⽛}
  \begin{phonetics}{西班牙}{xi1ban1ya2}
    \definition*{s.}{Espanha}
  \end{phonetics}
\end{entry}

\begin{entry}{西班牙文}{6,10,4,4}{⾑、⽟、⽛、⽂}
  \begin{phonetics}{西班牙文}{xi1ban1ya2wen2}
    \definition{s.}{espanhol, língua espanhola}
  \seealsoref{西文}{xi1wen2}
  \end{phonetics}
\end{entry}

\begin{entry}{西班牙语}{6,10,4,9}{⾑、⽟、⽛、⾔}
  \begin{phonetics}{西班牙语}{xi1ban1ya2yu3}
    \definition{s.}{espanhol | língua espanhola}
  \seealsoref{西语}{xi1yu3}
  \end{phonetics}
\end{entry}

\begin{entry}{西部}{6,10}{⾑、⾢}
  \begin{phonetics}{西部}{xi1 bu4}[][HSK 3]
    \definition{s.}{(EUA) filme de faroeste; filme de \emph{cowboys}; um faroeste | filme da região ocidental (China) | parte ocidental; região oeste da China}
  \end{phonetics}
\end{entry}

\begin{entry}{西装}{6,12}{⾑、⾐}
  \begin{phonetics}{西装}{xi1 zhuang1}[][HSK 5]
    \definition[件,套,个]{s.}{terno; roupas de estilo ocidental; roupas ocidentais, divididas em masculinas e femininas}
  \end{phonetics}
\end{entry}

\begin{entry}{西蓝花}{6,13,7}{⾑、⾋、⾋}
  \begin{phonetics}{西蓝花}{xi1lan2hua1}
    \variantof{西兰花}
  \end{phonetics}
\end{entry}

\begin{entry}{西餐}{6,16}{⾑、⾷}
  \begin{phonetics}{西餐}{xi1 can1}[][HSK 2]
    \definition[份,顿,桌]{s.}{comida ocidental; comida de estilo ocidental, comida com garfo e faca (diferente da 中餐)}
  \seealsoref{中餐}{zhong1 can1}
  \end{phonetics}
\end{entry}

\begin{entry}{西藏}{6,17}{⾑、⾋}
  \begin{phonetics}{西藏}{xi1zang4}
    \definition*{s.}{Xizang; Região Autônoma do Tibete, 西藏自治区}
  \seealsoref{西藏自治区}{xi1zang4 zi4zhi4qu1}
  \end{phonetics}
\end{entry}

\begin{entry}{西藏自治区}{6,17,6,8,4}{⾑、⾋、⾃、⽔、⼖}
  \begin{phonetics}{西藏自治区}{xi1zang4 zi4zhi4qu1}
    \definition*{s.}{Região Autônoma do Tibete}
  \end{phonetics}
\end{entry}

\begin{entry}{观}{6}{⾒}
  \begin{phonetics}{观}{guan1}
    \definition*{s.}{Templo taoísta; ``Koon''}
    \definition{s.}{visão; vista | perspectiva; visão; conceito | aparência; perspectiva | alcance de visão | noção; ideia; conhecimento ou visão das coisas | ponto de vista; postura; uma visão de uma coisa}
    \definition{v.}{olhar para; assistir; observar | contemplar}
  \end{phonetics}
  \begin{phonetics}{观}{guan4}
    \definition*{s.}{Sobrenome Guan}
    \definition{s.}{mosteiro taoísta | torre de vigia do portão do palácio | plataforma}
  \end{phonetics}
\end{entry}

\begin{entry}{观众}{6,6}{⾒、⼈}
  \begin{phonetics}{观众}{guan1zhong4}[][HSK 3]
    \definition[位,名,批,个]{s.}{espectador; público; audiência; pessoas que assistem a espetáculos ou competições}
  \end{phonetics}
\end{entry}

\begin{entry}{观念}{6,8}{⾒、⼼}
  \begin{phonetics}{观念}{guan1nian4}[][HSK 3]
    \definition[种,个]{s.}{ideia; conceito; consciência ideológica}
  \end{phonetics}
\end{entry}

\begin{entry}{观点}{6,9}{⾒、⽕}
  \begin{phonetics}{观点}{guan1dian3}[][HSK 2]
    \definition[个,种]{s.}{ponto de vista; perspectiva; a visão ou atitude que se tem sobre algo a partir de uma determinada posição ou perspectiva | ponto de vista; perspectiva; a posição ou perspectiva adotada ao analisar uma questão}
  \end{phonetics}
\end{entry}

\begin{entry}{观看}{6,9}{⾒、⽬}
  \begin{phonetics}{观看}{guan1 kan4}[][HSK 3]
    \definition{v.}{assistir; ver propositadamente; observar}
  \end{phonetics}
\end{entry}

\begin{entry}{观察}{6,14}{⾒、⼧}
  \begin{phonetics}{观察}{guan1cha2}[][HSK 3]
    \definition{v.}{assistir; pesquisar; observar; examinar cuidadosamente coisas ou fenômenos}
  \end{phonetics}
\end{entry}

\begin{entry}{讲}{6}{⾔}
  \begin{phonetics}{讲}{jiang3}[][HSK 2]
    \definition[种]{s.}{palestra; discurso}
    \definition{v.}{contar; falar | explicar; transmitir oralmente; esclarecer | negociar; barganhar | ser exigente com; valorizar; dar importância}
  \end{phonetics}
\end{entry}

\begin{entry}{讲究}{6,7}{⾔、⽳}
  \begin{phonetics}{讲究}{jiang3jiu5}[][HSK 4]
    \definition{adj.}{requintado; elegante; de bom gosto; exigente com a vida e com outros aspectos, buscando alto nível, qualidade e detalhes}
    \definition{s.}{estudo cuidadoso; algo que merece atenção; elementos e aspectos que merecem atenção especial}
    \definition{v.}{dar ênfase a; ser específico sobre; prestar atenção a}
  \end{phonetics}
\end{entry}

\begin{entry}{讲话}{6,8}{⾔、⾔}
  \begin{phonetics}{讲话}{jiang3 hua4}[][HSK 2]
    \definition[个]{s.}{discurso; palestra | guia; introdução}
    \definition{v.}{falar; conversar; dirigir-se a alguém | criticar}
  \end{phonetics}
\end{entry}

\begin{entry}{讲述}{6,8}{⾔、⾡}
  \begin{phonetics}{讲述}{jiang3shu4}
    \definition{v.}{falar sobre | narrar | descrever}
  \end{phonetics}
\end{entry}

\begin{entry}{讲座}{6,10}{⾔、⼴}
  \begin{phonetics}{讲座}{jiang3zuo4}[][HSK 4]
    \definition[个]{s.}{palestra; um curso de palestras; a forma de instrução usada para ensinar um determinado assunto ou tópico, geralmente por meio de palestras ao vivo, seriados de rádio ou televisão ou seriados de jornal.}
  \end{phonetics}
\end{entry}

\begin{entry}{许}{6}{⾔}
  \begin{phonetics}{许}{xu3}
    \definition*{s.}{Sobrenome Xu}
    \definition{adv.}{um pouco | talvez}
    \definition{v.}{permitir | prometer | elogiar}
  \end{phonetics}
\end{entry}

\begin{entry}{许可}{6,5}{⾔、⼝}
  \begin{phonetics}{许可}{xu3ke3}[][HSK 5]
    \definition{v.}{permitir; autorizar}
  \end{phonetics}
\end{entry}

\begin{entry}{许多}{6,6}{⾔、⼣}
  \begin{phonetics}{许多}{xu3duo1}[][HSK 2]
    \definition{num.}{muitos; muito; numerosos; uma grande quantidade de}
  \end{phonetics}
\end{entry}

\begin{entry}{论}{6}{⾔}
  \begin{phonetics}{论}{lun2}
    \definition*{s.}{Os Analectos de Confúcio, registro dos ditos e feitos de Confúcio e seus discípulos}
  \end{phonetics}
  \begin{phonetics}{论}{lun4}
    \definition*{s.}{Sobrenome Lun}
    \definition{prep.}{por (uma certa unidade de medida) | de acordo com (um certo sistema ou princípio)}
    \definition{s.}{visão; opinião; declaração | (frequentemente em títulos) dissertação; ensaio; tratado | teoria; doutrina | ideia; palavras ou artigos que analisam e explicam coisas}
    \definition{v.}{discutir; falar sobre; discursar sobre; comentar | mencionar; considerar; falar de | decidir sobre; determinar | decidir sobre a natureza da culpa; punir | argumentar; analisar e explicar coisas | considerar; ponderar; medir; avaliar}
  \end{phonetics}
\end{entry}

\begin{entry}{论文}{6,4}{⾔、⽂}
  \begin{phonetics}{论文}{lun4wen2}[][HSK 4]
    \definition[篇]{s.}{tese; redação; artigo; artigo que discute ou examina uma questão}
  \end{phonetics}
\end{entry}

\begin{entry}{设}{6}{⾔}
  \begin{phonetics}{设}{she4}
    \definition*{s.}{Sobrenome She}
    \definition{conj.}{se; no caso | (matemática) dado; suponha; se}
    \definition{v.}{configurar; estabelecer; encontrar; colocar em prática}
  \end{phonetics}
\end{entry}

\begin{entry}{设计}{6,4}{⾔、⾔}
  \begin{phonetics}{设计}{she4ji4}[][HSK 3]
    \definition[份]{s.}{plano; esquema; refere-se a um plano de design ou a um projeto para um plano, etc.}
    \definition{v.}{planejar; projetar; formular métodos, desenhos, etc. com antecedência, de acordo com determinados requisitos de finalidade, antes de iniciar oficialmente um trabalho | arquitetar; idear; tramar; fazer um plano}
  \end{phonetics}
\end{entry}

\begin{entry}{设立}{6,5}{⾔、⽴}
  \begin{phonetics}{设立}{she4li4}[][HSK 3]
    \definition{v.}{fundar; estabelecer; começar}
  \end{phonetics}
\end{entry}

\begin{entry}{设备}{6,8}{⾔、⼡}
  \begin{phonetics}{设备}{she4bei4}[][HSK 3]
    \definition[台,套]{s.}{instalação; equipamento; montagem; um conjunto de edifícios ou equipamentos necessários para executar uma determinada tarefa ou suprir uma determinada necessidade}
  \end{phonetics}
\end{entry}

\begin{entry}{设施}{6,9}{⾔、⽅}
  \begin{phonetics}{设施}{she4shi1}[][HSK 4]
    \definition{s.}{facilidade; instalação; instituições, sistemas, organizações, edifícios, etc., estabelecidos para realizar um trabalho ou atender a uma necessidade}
  \end{phonetics}
\end{entry}

\begin{entry}{设想}{6,13}{⾔、⼼}
  \begin{phonetics}{设想}{she4xiang3}[][HSK 5]
    \definition[个,种]{s.}{plano provisório (ou ideia); (item, tipo) refere-se a algo hipotético ou imaginário}
    \definition{v.}{imaginar; prever; conceber; supor | ter consideração por}
  \end{phonetics}
\end{entry}

\begin{entry}{设置}{6,13}{⾔、⽹}
  \begin{phonetics}{设置}{she4zhi4}[][HSK 4]
    \definition{v.}{estabelecer; colocar em prática; estabelecer ou criar instituições, empregos, profissões ou códigos, etc. | encaixar; ajustar; instalar; configurar; colocar}
  \end{phonetics}
\end{entry}

\begin{entry}{访}{6}{⾔}
  \begin{phonetics}{访}{fang3}
    \definition{v.}{visitar; fazer uma visita; ligar para | procurar por meio de investigação ou busca; tentar obter; obter uma entrevista | entrevistar | investigar; procurar por meio de investigação (pesquisar)}
  \end{phonetics}
\end{entry}

\begin{entry}{访问}{6,6}{⾔、⾨}
  \begin{phonetics}{访问}{fang3wen4}[][HSK 3]
    \definition{v.}{visitar; ligar; entrevistar; visitar e conversar com um objetivo específico | visitar um \emph{site}}
  \end{phonetics}
\end{entry}

\begin{entry}{负}{6}{⾙}
  \begin{phonetics}{负}{fu4}[][HSK 6]
    \definition{adj.}{negativo; menor que zero | negativo; referindo-se ao que recebe elétrons (oposto a 正)}
    \definition{v.}{carregar; transportar nas costas ou nos ombros | suportar; assumir; encarar | confiar em; contar com; depender | sofrer | aproveitar; desfrutar | ter dívidas | trair; violar | perder; ser derrotado}
  \seealsoref{正}{zheng4}
  \end{phonetics}
\end{entry}

\begin{entry}{负担}{6,8}{⾙、⼿}
  \begin{phonetics}{负担}{fu4dan1}[][HSK 4]
    \definition{s.}{carga; fardo; frete; ônus; pressão ou responsabilidade, despesas, etc.}
    \definition{v.}{carregar; carregar (um fardo); assumir (responsabilidade, trabalho, despesas, etc.)}
  \end{phonetics}
\end{entry}

\begin{entry}{负责}{6,8}{⾙、⾙}
  \begin{phonetics}{负责}{fu4ze2}[][HSK 3]
    \definition{adj.}{consciencioso; ser sério e responsável}
    \definition{v.}{ser responsável por; estar encarregado de; assumir responsabilidades}
  \end{phonetics}
\end{entry}

\begin{entry}{负责人}{6,8,2}{⾙、⾙、⼈}
  \begin{phonetics}{负责人}{fu4 ze2 ren2}[][HSK 5]
    \definition{s.}{pessoa responsável; pessoa encarregada; pessoas com responsabilidades de liderança}
  \end{phonetics}
\end{entry}

\begin{entry}{赱}{6}{⼟}
  \begin{phonetics}{赱}{zou3}
    \variantof{走}
  \end{phonetics}
\end{entry}

\begin{entry}{达}{6}{⾡}
  \begin{phonetics}{达}{da2}
    \definition*{s.}{Sobrenome Da}
    \definition{adj.}{eminente; distinto; refere-se a um funcionário distinto; \emph{status} elevado | otimista; de mente aberta}
    \definition{v.}{prolongar | alcançar; atingir; equivaler a | entender completamente; compreender (assuntos) | expressar; comunicar}
  \end{phonetics}
\end{entry}

\begin{entry}{达成}{6,6}{⾡、⼽}
  \begin{phonetics}{达成}{da2cheng2}[][HSK 5]
    \definition{v.}{concluir; chegar (a um acordo); conseguir; obter (principalmente como resultado de uma negociação)}
  \end{phonetics}
\end{entry}

\begin{entry}{达到}{6,8}{⾡、⼑}
  \begin{phonetics}{达到}{da2dao4}[][HSK 3]
    \definition{v.}{alcançar; atender o padrão; atingir (refere-se principalmente a coisas abstratas ou graus); chegar a um determinado ponto ou grau}
  \end{phonetics}
\end{entry}

\begin{entry}{迅}{6}{⾡}
  \begin{phonetics}{迅}{xun4}
    \definition{adj.}{rápido; veloz}
    \definition{adv.}{rapidamente; velozmente}
  \end{phonetics}
\end{entry}

\begin{entry}{迅速}{6,10}{⾡、⾡}
  \begin{phonetics}{迅速}{xun4su4}[][HSK 4]
    \definition{adv.}{rapidamente; velozmente; prontamente}
  \end{phonetics}
\end{entry}

\begin{entry}{过}{6}{⾡}
  \begin{phonetics}{过}{guo1}
    \definition*{s.}{Sobrenome Guo}
  \end{phonetics}
  \begin{phonetics}{过}{guo4}[][HSK 1,2]
    \definition{adv.}{excessivamente; em excesso}
    \definition{clas.}{tempo; número de vezes usado para a ação}
    \definition{s.}{falha; erro; demérito; equívoco; negligência; (oposto a 功)}
    \definition{v.}{cruzar; passar; mudar-se de um lugar para outro; passar por | exceder; ir além; ultrapassar; usado após um adjetivo, significa ``mais do que'' | gastar (tempo); passar (tempo); exceder (um determinado limite ou limite) | celebrar; comemorar | mudar; transferir; transferir de um lado para o outro | passar por um processo; passar por; submeter a (algum tipo de tratamento) | visitar; fazer uma visita | falecer; morrer | infectar; ser contagioso; espalhar | exceder; ir além; usado após o verbo com o sufixo 得, significa ``superar'' ou ``passar'' | viver | revisar; examinar; usar os olhos para ver ou a mente para lembrar}
  \seealsoref{得}{de5}
  \seealsoref{功}{gong1}
  \end{phonetics}
  \begin{phonetics}{过}{guo5}
    \definition{part.}{usado depois de um verbo para indicar conclusão | usado depois de um verbo para indicar que uma ação ou mudança ocorreu | usado depois de um adjetivo para indicar que algo já teve uma certa qualidade ou estado e para compará-lo com o presente}
  \end{phonetics}
\end{entry}

\begin{entry}{过于}{6,3}{⾡、⼆}
  \begin{phonetics}{过于}{guo4yu2}[][HSK 5]
    \definition{adv.}{demais; indevidamente; excessivamente; advérbios de grau ou quantidade excessiva}
  \end{phonetics}
\end{entry}

\begin{entry}{过不惯}{6,4,11}{⾡、⼀、⼼}
  \begin{phonetics}{过不惯}{guo4 bu5 guan4}
    \definition{v.}{não se acostumar | não se habituar}
  \seealsoref{过惯}{guo4guan4}
  \end{phonetics}
\end{entry}

\begin{entry}{过分}{6,4}{⾡、⼑}
  \begin{phonetics}{过分}{guo4fen4}[][HSK 4]
    \definition{adj.}{excessivo; muito longe; demais; falar ou agir além dos limites ou graus adequados}
    \definition{adv.}{excessivamente; indevidamente; muito mesmo}
  \end{phonetics}
\end{entry}

\begin{entry}{过去}{6,5}{⾡、⼛}
  \begin{phonetics}{过去}{guo4 qu4}[][HSK 2,3]
    \definition{adv.}{(no) passado}
    \definition{s.}{o passado; refere-se a um período anterior; também se refere a coisas anteriores}
    \definition{v.}{atravessar; passar; sair do local onde o interlocutor se encontra e deslocar-se para outro local | acabar; passar; ficar para trás; indica que já passou por uma determinada fase | passar; indica que um determinado período ou situação já não existe mais | falecer | ir lá | passar por}
  \end{phonetics}
\end{entry}

\begin{entry}{过节}{6,5}{⾡、⾋}
  \begin{phonetics}{过节}{guo4jie2}
    \definition{v.+compl.}{celebrar festividades | comemorar um festival}
  \end{phonetics}
\end{entry}

\begin{entry}{过关}{6,6}{⾡、⼋}
  \begin{phonetics}{过关}{guo4guan1}
    \definition{v.+compl.}{passar uma barreira | passar por uma provação | passar em um teste | atingir um padrão | passar pela alfândega}
  \end{phonetics}
\end{entry}

\begin{entry}{过年}{6,6}{⾡、⼲}
  \begin{phonetics}{过年}{guo4 nian2}[][HSK 2]
    \definition{v.+compl.}{comemorar o Ano Novo; comemorar o Festival da Primavera; passar o Ano Novo; passar o Festival da Primavera; realizar atividades comemorativas durante o Ano Novo ou o Festival da Primavera}
  \end{phonetics}
\end{entry}

\begin{entry}{过来}{6,7}{⾡、⽊}
  \begin{phonetics}{过来}{guo4 lai2}[][HSK 2]
    \definition{v.}{vir até aqui | ser capaz de cuidar de | lidar com | administrar}
  \end{phonetics}
\end{entry}

\begin{entry}{过度}{6,9}{⾡、⼴}
  \begin{phonetics}{过度}{guo4du4}[][HSK 5]
    \definition{adj.}{excessivo; acima do limite; além do limite; além do que é apropriado}
  \end{phonetics}
\end{entry}

\begin{entry}{过惯}{6,11}{⾡、⼼}
  \begin{phonetics}{过惯}{guo4guan4}
    \definition{v.}{estar acostumado (a um certo estilo de vida, etc.)}
  \seealsoref{过不惯}{guo4 bu5 guan4}
  \end{phonetics}
\end{entry}

\begin{entry}{过敏}{6,11}{⾡、⽁}
  \begin{phonetics}{过敏}{guo4min3}[][HSK 5]
    \definition{adj.}{sensível; excessivamente sensível; resposta acima do normal; ceticismo excessivo}
    \definition{v.}{ser alérgico a}
  \end{phonetics}
\end{entry}

\begin{entry}{过期}{6,12}{⾡、⽉}
  \begin{phonetics}{过期}{guo4qi1}
    \definition{v.+compl.}{exceder a data | passar a data | expirar (passar a data de expiração)}
  \end{phonetics}
\end{entry}

\begin{entry}{过程}{6,12}{⾡、⽲}
  \begin{phonetics}{过程}{guo4cheng2}[][HSK 3]
    \definition[个,段]{s.}{curso dos eventos; processo; o processo pelo qual as coisas acontecem ou se desenvolvem.}
  \end{phonetics}
\end{entry}

\begin{entry}{过瘾}{6,16}{⾡、⽧}
  \begin{phonetics}{过瘾}{guo4yin3}
    \definition{adj.}{gratificante | imensamente agradável | satisfatório}
    \definition{v.+compl.}{satisfazer um desejo | se divertir com algo}
  \end{phonetics}
\end{entry}

\begin{entry}{那}{6}{⾢}
  \begin{phonetics}{那}{na1}
    \definition*{s.}{Sobrenome Na}
  \end{phonetics}
  \begin{phonetics}{那}{na3}
    \definition{adv.}{expressa negação em perguntas retóricas}
    \definition{pron.}{qual? | qualquer que seja; qualquer que; para expressar incerteza em uma declaração | variante de 哪}
  \seealsoref{哪}{na3}
  \end{phonetics}
  \begin{phonetics}{那}{na4}[][HSK 1,2]
    \definition{conj.}{então; nessa situação; nesse caso; o mesmo que 那么}
    \definition{pron.}{aquele; aquilo; indica pessoas ou coisas distantes | aquele; aquilo; expressa muitas coisas, sem se referir especificamente a uma pessoa ou coisa, e é frequentemente usado em conjunto com 这}
  \seealsoref{那么}{na4 me5}
  \seealsoref{这}{zhe4}
  \end{phonetics}
  \begin{phonetics}{那}{ne4}
    \definition{conj.}{então; nesse caso; o mesmo que 那么}
    \definition{pron.}{aquele; aquilo; pronúncia coloquial de 那 (\dpy{na4})}
  \seealsoref{那么}{na4 me5}
  \end{phonetics}
  \begin{phonetics}{那}{nei4}
    \definition{conj.}{então; o mesmo que 那么}
    \definition{pron.}{aquele; aquilo; A pronúncia coloquial de 那 (\dpy{na4})}
  \seealsoref{那么}{na4 me5}
  \end{phonetics}
  \begin{phonetics}{那}{nuo2}
    \definition*{s.}{Sobrenome Nuo}
  \end{phonetics}
\end{entry}

\begin{entry}{那儿}{6,2}{⾢、⼉}
  \begin{phonetics}{那儿}{na4r5}[][HSK 1]
    \definition{pron.}{lá; ali; naquele lugar | então; naquela época (usado após 打, 从 e 由)}
  \seealsoref{从}{cong2}
  \seealsoref{打}{da3}
  \seealsoref{由}{you2}
  \end{phonetics}
\end{entry}

\begin{entry}{那个}{6,3}{⾢、⼈}
  \begin{phonetics}{那个}{na4ge5}
    \definition{pron.}{aquele | usado antes de verbos e adjetivos para indicar exagero | para substituir o discurso direto inconveniente}
  \end{phonetics}
\end{entry}

\begin{entry}{那么}{6,3}{⾢、⼃}
  \begin{phonetics}{那么}{na4 me5}[][HSK 2]
    \definition{conj.}{então; nesse caso; afirmar o resultado esperado ou fazer um julgamento}
    \definition{pron.}{assim; dessa maneira; indica a natureza, o estado, a forma, o grau, etc. | assim; sobre; colocado antes do numeral, indica uma estimativa}
  \end{phonetics}
\end{entry}

\begin{entry}{那边}{6,5}{⾢、⾡}
  \begin{phonetics}{那边}{na4 bian5}[][HSK 1]
    \definition{pron.}{ali; acolá; aquele lado}
  \end{phonetics}
\end{entry}

\begin{entry}{那会儿}{6,6,2}{⾢、⼈、⼉}
  \begin{phonetics}{那会儿}{na4 hui4r5}[][HSK 2]
    \definition{pron.}{então; naquela época; refere-se ao passado ou ao futuro}
  \end{phonetics}
\end{entry}

\begin{entry}{那时}{6,7}{⾢、⽇}
  \begin{phonetics}{那时}{na4 shi2}[][HSK 2]
    \definition{pron.}{então; naquela época; naqueles dias; geralmente se refere a um período de tempo distante do presente}
  \seealsoref{那时候}{na4 shi2 hou5}
  \end{phonetics}
\end{entry}

\begin{entry}{那时候}{6,7,10}{⾢、⽇、⼈}
  \begin{phonetics}{那时候}{na4 shi2 hou5}[][HSK 2]
    \definition{adv.}{naquela hora; em algum momento no passado}
  \seealsoref{那时}{na4 shi2}
  \end{phonetics}
\end{entry}

\begin{entry}{那里}{6,7}{⾢、⾥}
  \begin{phonetics}{那里}{na4 li3}[][HSK 1]
    \definition{pron./s.}{lá; ali; aquele lugar; indica um lugar distante}
  \end{phonetics}
\end{entry}

\begin{entry}{那些}{6,8}{⾢、⼆}
  \begin{phonetics}{那些}{na4 xie1}[][HSK 1]
    \definition{pron.}{aqueles; indica duas ou mais pessoas ou coisas}
  \end{phonetics}
\end{entry}

\begin{entry}{那咱}{6,9}{⾢、⼝}
  \begin{phonetics}{那咱}{na4 zan5}
    \definition{s.}{(informal) naquela época; então | (antigo) naquela época}
  \end{phonetics}
\end{entry}

\begin{entry}{那样}{6,10}{⾢、⽊}
  \begin{phonetics}{那样}{na4 yang4}[][HSK 2]
    \definition{pron.}{assim; tal; desse tipo; desse gênero; dessa natureza; desse tipo; indica a natureza, o estado, a maneira, o grau ou refere-se a uma ação ou situação específica}
  \end{phonetics}
\end{entry}

\begin{entry}{那麽}{6,14}{⾢、⿇}
  \begin{phonetics}{那麽}{na4 me5}
    \variantof{那么}
  \end{phonetics}
\end{entry}

\begin{entry}{闭}{6}{⾨}
  \begin{phonetics}{闭}{bi4}[][HSK 6]
    \definition*{s.}{Sobrenome Bi}
    \definition{v.}{fechar; encerrar | bloquear; obstruir; parar}
  \end{phonetics}
\end{entry}

\begin{entry}{闭幕}{6,13}{⾨、⼱}
  \begin{phonetics}{闭幕}{bi4 mu4}[][HSK 5]
    \definition{v.+compl.}{fechar; concluir; (conferência, exposição, etc.) terminar | cair a cortina; abaixar a cortina; terminar a apresentação e a cortina se fechar em frente ao palco}
  \end{phonetics}
\end{entry}

\begin{entry}{闭幕式}{6,13,6}{⾨、⼱、⼷}
  \begin{phonetics}{闭幕式}{bi4 mu4 shi4}[][HSK 5]
    \definition{s.}{cerimônia de encerramento; cerimônia formal realizada no final de uma conferência ou exposição}
  \end{phonetics}
\end{entry}

\begin{entry}{闭嘴}{6,16}{⾨、⼝}
  \begin{phonetics}{闭嘴}{bi4zui3}
    \definition{expr.}{Cale-se!; Pare de falar!}
  \end{phonetics}
\end{entry}

\begin{entry}{问}{6}{⾨}
  \begin{phonetics}{问}{wen4}[][HSK 1]
    \definition*{s.}{Sobrenome Wen}
    \definition{prep.}{de; introduzir o objeto da ação, equivalente a 向 e 跟}
    \definition{v.}{perguntar; indagar; fazer com que as pessoas respondam ou esclareçam coisas que não sabem ou não têm certeza | perguntar (ou indagar) sobre | examinar; interrogar | intervir; responsabilizar; investigar | cuidar; preocupar-se; gerenciar; interferir}
  \seealsoref{跟}{gen1}
  \seealsoref{向}{xiang4}
  \end{phonetics}
\end{entry}

\begin{entry}{问市}{6,5}{⾨、⼱}
  \begin{phonetics}{问市}{wen4shi4}
    \definition{v.}{chegar ao mercado | bater o mercado | atingir o mercado}
  \end{phonetics}
\end{entry}

\begin{entry}{问安}{6,6}{⾨、⼧}
  \begin{phonetics}{问安}{wen4'an1}
    \definition{s.}{saudações}
    \definition{v.}{dar cumprimentos a | prestar homenagem}
  \end{phonetics}
\end{entry}

\begin{entry}{问卷}{6,8}{⾨、⼙}
  \begin{phonetics}{问卷}{wen4juan4}
    \definition[份]{s.}{questionário}
  \end{phonetics}
\end{entry}

\begin{entry}{问候}{6,10}{⾨、⼈}
  \begin{phonetics}{问候}{wen4hou4}[][HSK 4]
    \definition{s.}{homenagem | saudação}
    \definition{v.}{prestar homenagem; enviar uma saudação;  dar os respeitos (cumprimentos) a alguém | (fig.) (coloquial) fazer referência ofensiva a (alguém querido pela pessoa com quem se está falando)}
  \end{phonetics}
\end{entry}

\begin{entry}{问鼎}{6,12}{⾨、⿍}
  \begin{phonetics}{问鼎}{wen4ding3}
    \definition{v.}{visar (o primeiro lugar, etc.) | aspirar ao trono}
  \end{phonetics}
\end{entry}

\begin{entry}{问路}{6,13}{⾨、⾜}
  \begin{phonetics}{问路}{wen4 lu4}[][HSK 2]
    \definition{v.}{perguntar o caminho; pedir direções}
  \end{phonetics}
\end{entry}

\begin{entry}{问题}{6,15}{⾨、⾴}
  \begin{phonetics}{问题}{wen4ti2}[][HSK 2]
    \definition{adj.}{desqualificado; indesejável; anormal, não atende aos requisitos}
    \definition[个,种,类,串]{s.}{pergunta; problema; perguntas a serem respondidas | problema; questão; contradições que precisam ser estudadas e resolvidas | problema; acidente; incidente | chave; ponto crucial; pontos importantes}
  \end{phonetics}
\end{entry}

\begin{entry}{闯}{6}{⾨}
  \begin{phonetics}{闯}{chuang3}[][HSK 5]
    \definition*{s.}{Sobrenome Chuang}
    \definition{v.}{apressar-se; correr | moderar a si mesmo (lutando contra dificuldades e perigos); aventurar-se no mundo | incorrer; causar (um desastre, etc.)}
  \end{phonetics}
\end{entry}

\begin{entry}{防}{6}{⾩}
  \begin{phonetics}{防}{fang2}[][HSK 3]
    \definition*{s.}{Sobrenome Fang}
    \definition{s.}{defesa | dique; aterro | barragem; represa; estrutura para conter a água}
    \definition{v.}{proteger contra; prevenir contra; tomar precauções contra | defender-se contra}
  \end{phonetics}
\end{entry}

\begin{entry}{防止}{6,4}{⾩、⽌}
  \begin{phonetics}{防止}{fang2zhi3}[][HSK 3]
    \definition{v.}{evitar; prevenir; prevenir; proteger contra; preparar-se com antecedência para evitar que coisas ruins aconteçam}
  \end{phonetics}
\end{entry}

\begin{entry}{防护}{6,7}{⾩、⼿}
  \begin{phonetics}{防护}{fang2hu4}
    \definition{v.}{defender | proteger}
  \end{phonetics}
\end{entry}

\begin{entry}{防治}{6,8}{⾩、⽔}
  \begin{phonetics}{防治}{fang2zhi4}[][HSK 5]
    \definition{s.}{tratamento preventivo; prevenção e cura; profilaxia e tratamento}
  \end{phonetics}
\end{entry}

\begin{entry}{防晒}{6,10}{⾩、⽇}
  \begin{phonetics}{防晒}{fang2shai4}
    \definition{s.}{protetor solar}
  \end{phonetics}
\end{entry}

\begin{entry}{阳}{6}{⾩}
  \begin{phonetics}{阳}{yang2}
    \definition*{s.}{Yang, o princípio positivo de Yin e Yang}
    \definition{s.}{eletricidade: positivo | sol}
  \seealsoref{阴}{yin1}
  \seealsoref{阴阳}{yin1yang2}
  \end{phonetics}
\end{entry}

\begin{entry}{阳台}{6,5}{⾩、⼝}
  \begin{phonetics}{阳台}{yang2tai2}[][HSK 4]
    \definition{s.}{varanda; terraço; sacada; pequeno terraço do edifício com grades para se refrescar, tomar sol ou olhar o horizonte}
  \end{phonetics}
\end{entry}

\begin{entry}{阳光}{6,6}{⾩、⼉}
  \begin{phonetics}{阳光}{yang2guang1}[][HSK 3]
    \definition{adj.}{alegre; otimista; personalidade positiva e alegre; cheio de vitalidade juvenil | aberto; transparente; público; conduzido sob supervisão pública}
    \definition[缕,束,道]{s.}{luz do sol; raio de sol}
  \end{phonetics}
\end{entry}

\begin{entry}{阴}{6}{⾩}
  \begin{phonetics}{阴}{yin1}[][HSK 2]
    \definition*{s.}{Yin, o princípio negativo de Yin e Yang | A Lua; refere-se a Taiyin | Sobrenome Yin}
    \definition{adj.}{nublado; opaco; sombrio | escondido; secreto; não exposto | sinistro | do mundo inferior; dos fantasmas | Física: negativo; cátodo | nublado; mais de 80\% do céu estão cobertos por nuvens | em talhe-doce; rebaixado | (matéria) carregada negativamente}
    \definition[片]{s.}{sombra; lugar sombrio | partes íntimas (especialmente da mulher) | ao norte de uma colina ou ao sul de um rio | verso | entalhe}
  \seealsoref{阳}{yang2}
  \seealsoref{阴阳}{yin1yang2}
  \end{phonetics}
\end{entry}

\begin{entry}{阴天}{6,4}{⾩、⼤}
  \begin{phonetics}{阴天}{yin1 tian1}[][HSK 2]
    \definition[个]{s.}{nublado; céu nublado; dia nublado; uma condição climática em que 80\% do céu está coberto por nuvens e apenas um pouco de sol pode ser visto}
  \end{phonetics}
\end{entry}

\begin{entry}{阴阳}{6,6}{⾩、⾩}
  \begin{phonetics}{阴阳}{yin1yang2}
    \definition*{s.}{Yin e Yang}
  \seealsoref{阳}{yang2}
  \seealsoref{阴}{yin1}
  \end{phonetics}
\end{entry}

\begin{entry}{阵}{6}{⾩}
  \begin{phonetics}{阵}{zhen4}[][HSK 4]
    \definition{clas.}{passagens que expressam a passagem de eventos ou ações}
    \definition{s.}{matriz de batalha (formação); termo tático antigo para as fileiras ou formações de uma equipe de combate | \emph{front}; frente de batalha; posição | um período de tempo}
  \end{phonetics}
\end{entry}

\begin{entry}{阵地}{6,6}{⾩、⼟}
  \begin{phonetics}{阵地}{zhen4di4}
    \definition{s.}{posição (militar) | frente de batalha | \emph{front}}
  \end{phonetics}
\end{entry}

\begin{entry}{阶}{6}{⾩}
  \begin{phonetics}{阶}{jie1}
    \definition{s.}{degrau; escada; escadaria | classificação | escala | ordem | estágio}
  \end{phonetics}
\end{entry}

\begin{entry}{阶段}{6,9}{⾩、⽎}
  \begin{phonetics}{阶段}{jie1duan4}[][HSK 4]
    \definition{s.}{estágio; fase; período; bancada; gradação}
  \end{phonetics}
\end{entry}

\begin{entry}{页}{6}{⾴}[Kangxi 181]
  \begin{phonetics}{页}{ye4}[][HSK 1]
    \definition{clas.}{página; folha de papel; lâmina; antigamente, referia-se a uma folha de um livro encadernado; atualmente, refere-se a uma das faces de um livro impresso em ambos os lados}
    \definition{s.}{página; folha de papel; folhas soltas de um livro}
  \end{phonetics}
\end{entry}

\begin{entry}{齐}{6}{⿑}[Kangxi 210]
  \begin{phonetics}{齐}{qi2}[][HSK 3]
    \definition*{s.}{Qi, um estado da Dinastia Zhou | Dinastia Qi do Sul (479-502), uma das Dinastias do Sul | Dinastia Qi do Norte (550-577), uma das Dinastias do Norte | Sobrenome Qi}
    \definition{adj.}{arrumado; uniforme; regular; comprimento, tamanho, etc. são praticamente iguais; uniformes | semelhante; similar; da mesma forma; de acordo| tudo pronto; todos presentes; completo; perfeito}
    \definition{adv.}{juntos; simultaneamente; ao mesmo tempo}
    \definition{v.}{estar no mesmo nível que; alcançar o mesmo nível | estar nivelado em um ponto ou ao longo de uma linha; tornar consistente; harmonizar}
  \end{phonetics}
\end{entry}

\begin{entry}{齐全}{6,6}{⿑、⼊}
  \begin{phonetics}{齐全}{qi2quan2}[][HSK 5]
    \definition{adj.}{completo; tudo pronto}
  \end{phonetics}
\end{entry}

%%%%% EOF %%%%%


%%%
%%% 7画
%%%

\section*{7画}\addcontentsline{toc}{section}{7画}

\begin{entry}{两}{7}{⼀}
  \begin{phonetics}{两}{liang3}[][HSK 1,2]
    \definition*{s.}{sobrenome Liang}
    \definition{clas.}{liang, uma unidade de peso (=50 gramas)}
    \definition{num.}{dois (sempre usado antes de classificadores) | poucos; alguns; indica um número indeterminado}
    \definition{s.}{ambos (lados); qualquer (lado)}
  \end{phonetics}
\end{entry}

\begin{entry}{两边}{7,5}{⼀、⾡}
  \begin{phonetics}{两边}{liang3 bian1}[][HSK 4]
    \definition{s.}{ambos os lados; ambas as direções; ambos os lugares | ambas as partes; ambos os lados}
  \end{phonetics}
\end{entry}

\begin{entry}{两岸}{7,8}{⼀、⼭}
  \begin{phonetics}{两岸}{liang3 an4}[][HSK 5]
    \definition{s.}{ambos os lados; ambas as margens; ambas as costas; entre os dois lados do estreito; bilateral}
  \end{phonetics}
\end{entry}

\begin{entry}{两码事}{7,8,8}{⼀、⽯、⼅}
  \begin{phonetics}{两码事}{liang3ma3shi4}
    \definition{expr.}{duas coisas completamente diferentes}
  \end{phonetics}
\end{entry}

\begin{entry}{严}{7}{⼀}
  \begin{phonetics}{严}{yan2}[][HSK 4]
    \definition*{s.}{sobrenome Yan}
    \definition{adj.}{rígido; rigoroso; estrito; severo}
    \definition{s.}{pai; refere-se ao pai}
  \end{phonetics}
\end{entry}

\begin{entry}{严厉}{7,5}{⼀、⼚}
  \begin{phonetics}{严厉}{yan2li4}[][HSK 5]
    \definition{adj.}{severo; rigoroso}
  \end{phonetics}
\end{entry}

\begin{entry}{严肃}{7,8}{⼀、⾀}
  \begin{phonetics}{严肃}{yan2su4}[][HSK 5]
    \definition{adj.}{sério; solene; sincero; (expressão, atmosfera, etc.) faz as pessoas se sentirem admiradas e desconfortáveis | sóbrio; grave; sério; sincero}
    \definition{v.}{aplicar rigorosamente; fazer algo sério}
  \end{phonetics}
\end{entry}

\begin{entry}{严重}{7,9}{⼀、⾥}
  \begin{phonetics}{严重}{yan2zhong4}[][HSK 4]
    \definition{adj.}{sério; grave; crítico; severo}
  \end{phonetics}
\end{entry}

\begin{entry}{严重打伤}{7,9,5,6}{⼀、⾥、⼿、⼈}
  \begin{phonetics}{严重打伤}{yan2zhong4 da3 shang1}
    \definition{s.}{gravemente ferido}
  \end{phonetics}
\end{entry}

\begin{entry}{严重伤害}{7,9,6,10}{⼀、⾥、⼈、⼧}
  \begin{phonetics}{严重伤害}{yan2zhong4 shang1hai4}
    \definition{s.}{ferimento grave}
  \end{phonetics}
\end{entry}

\begin{entry}{严重关切}{7,9,6,4}{⼀、⾥、⼋、⼑}
  \begin{phonetics}{严重关切}{yan2zhong4guan1qie4}
    \definition{s.}{preocupação séria}
  \end{phonetics}
\end{entry}

\begin{entry}{严重危害}{7,9,6,10}{⼀、⾥、⼙、⼧}
  \begin{phonetics}{严重危害}{yan2zhong4wei1hai4}
    \definition{s.}{danos graves}
  \end{phonetics}
\end{entry}

\begin{entry}{严重后果}{7,9,6,8}{⼀、⾥、⼝、⽊}
  \begin{phonetics}{严重后果}{yan2zhong4hou4guo3}
    \definition{s.}{consequências sérias | repercursões graves}
  \end{phonetics}
\end{entry}

\begin{entry}{严重地}{7,9,6}{⼀、⾥、⼟}
  \begin{phonetics}{严重地}{yan2zhong4 di4}
    \definition{adv.}{seriamente | gravemente}
  \end{phonetics}
\end{entry}

\begin{entry}{严重问题}{7,9,6,15}{⼀、⾥、⾨、⾴}
  \begin{phonetics}{严重问题}{yan2zhong4wen4ti2}
    \definition{s.}{problema sério}
  \end{phonetics}
\end{entry}

\begin{entry}{严重性}{7,9,8}{⼀、⾥、⼼}
  \begin{phonetics}{严重性}{yan2zhong4xing4}
    \definition{s.}{seriedade | gravidade}
  \end{phonetics}
\end{entry}

\begin{entry}{严重破坏}{7,9,10,7}{⼀、⾥、⽯、⼟}
  \begin{phonetics}{严重破坏}{yan2zhong4 po4huai4}
    \definition{s.}{destruição grave}
  \end{phonetics}
\end{entry}

\begin{entry}{严格}{7,10}{⼀、⽊}
  \begin{phonetics}{严格}{yan2ge2}[][HSK 4]
    \definition{adj.}{rígido; estrito; rigoroso; muito consciente e meticuloso na implementação de sistemas e no domínio de padrões}
    \definition{v.}{tornar (sistemas, provisões, etc.) rigorosos;}
  \end{phonetics}
\end{entry}

\begin{entry}{乱}{7}{⼄}
  \begin{phonetics}{乱}{luan4}[][HSK 3]
    \definition{adj.}{em desordem; em confusão; em desarrumação; sem ordem nem organização | em um estado mental confuso | (de uma sociedade) turbulento; agitado | (de relações sexuais) impróprio; promíscuo}
    \definition{adv.}{aleatoriamente; arbitrariamente; indiscriminadamente; sem restrições; à vontade}
    \definition{s.}{motim; agitação; tumulto; revolta; guerra; calamidade}
    \definition{v.}{confundir; embaralhar; misturar; causar desordem}
  \end{phonetics}
\end{entry}

\begin{entry}{亩}{7}{⼇}
  \begin{phonetics}{亩}{mu3}
    \definition{clas.}{usado para campos | unidade de área igual a um décimo quinto de um hectare}
  \end{phonetics}
\end{entry}

\begin{entry}{估}{7}{⼈}
  \begin{phonetics}{估}{gu1}
    \definition{v.}{estimar; avaliar; aferir}
  \end{phonetics}
  \begin{phonetics}{估}{gu4}
    \definition{adj.}{velho | roupas de segunda mão}
  \end{phonetics}
\end{entry}

\begin{entry}{估计}{7,4}{⼈、⾔}
  \begin{phonetics}{估计}{gu1ji4}[][HSK 5]
    \definition{v.}{fazer contas; estimar; calcular; julgar a natureza, quantidade, mudança, etc. de uma coisa em uma determinada situação | parecer; parecer como se; aparentar; fazer inferências aproximadas sobre a natureza, a quantidade e a mudança das coisas com base em determinadas circunstâncias}
  \end{phonetics}
\end{entry}

\begin{entry}{伲}{7}{⼈}
  \begin{phonetics}{伲}{ni4}
    \definition{pron.}{(dialeto) eu | meu | nosso | nós}
  \seealsoref{你}{ni3}
  \end{phonetics}
\end{entry}

\begin{entry}{伴}{7}{⼈}
  \begin{phonetics}{伴}{ban4}
    \definition[个,位]{s.}{companheiro; parceiro}
    \definition{v.}{acompanhar; estsar perto}[伴君如伴虎。___Acompanhar o rei é como acompanhar um tigre.]
  \end{phonetics}
\end{entry}

\begin{entry}{伴侣}{7,8}{⼈、⼈}
  \begin{phonetics}{伴侣}{ban4lv3}
    \definition{s.}{companheiro | parceiro}
  \end{phonetics}
\end{entry}

\begin{entry}{伸}{7}{⼈}
  \begin{phonetics}{伸}{shen1}[][HSK 5]
    \definition{v.}{alongar; esticar; estender}
  \end{phonetics}
\end{entry}

\begin{entry}{但}{7}{⼈}
  \begin{phonetics}{但}{dan4}[][HSK 2]
    \definition*{s.}{sobrenome Dan}
    \definition{adv.}{apenas; meramente; indica uma restrição ao âmbito da ação, equivalente a 只 ou 仅}
    \definition{conj.}{mas; ainda assim; mesmo assim; no entanto; contudo; usado na última oração, conecta duas orações, expressando uma relação de transição, equivalente a 可是 ou 不过}
  \seealsoref{不过}{bu2guo4}
  \seealsoref{仅}{jin3}
  \seealsoref{可是}{ke3shi4}
  \seealsoref{只}{zhi3}
  \end{phonetics}
\end{entry}

\begin{entry}{但是}{7,9}{⼈、⽇}
  \begin{phonetics}{但是}{dan4 shi4}[][HSK 2]
    \definition{conj.}{mas; contudo; no entanto; mesmo assim; usado na segunda parte da frase para indicar uma mudança, geralmente acompanhada de expressões como 虽然 ou 尽管}
  \seealsoref{尽管}{jin3guan3}
  \seealsoref{虽然}{sui1 ran2}
  \end{phonetics}
\end{entry}

\begin{entry}{位}{7}{⼈}
  \begin{phonetics}{位}{wei4}[][HSK 2]
    \definition*{s.}{sobrenome Wei}
    \definition{clas.}{usado para pessoas (com cortesia, respeito) | usado para bits binários}[十六位___16 bits]
    \definition{s.}{lugar; localização; o lugar onde ou onde alguém está localizado | posto; \emph{status}; posição; a posição de uma pessoa em uma determinada área da vida social | trono; refere-se especificamente ao status do imperador | lugar; dígito; a posição de cada dígito em um número}
  \end{phonetics}
\end{entry}

\begin{entry}{位于}{7,3}{⼈、⼆}
  \begin{phonetics}{位于}{wei4yu2}[][HSK 4]
    \definition{v.}{estar localizado; estar situado}
  \end{phonetics}
\end{entry}

\begin{entry}{位子}{7,3}{⼈、⼦}
  \begin{phonetics}{位子}{wei4zi5}
    \definition{s.}{lugar | assento}
  \end{phonetics}
\end{entry}

\begin{entry}{位居}{7,8}{⼈、⼫}
  \begin{phonetics}{位居}{wei4ju1}
    \definition{v.}{estar localizado em}
  \end{phonetics}
\end{entry}

\begin{entry}{位置}{7,13}{⼈、⽹}
  \begin{phonetics}{位置}{wei4zhi4}[][HSK 4]
    \definition[通,个]{s.}{assento; lugar; localização | lugar; posição; \emph{status} | posição (por exemplo: cargo no escritório)}
  \end{phonetics}
\end{entry}

\begin{entry}{低}{7}{⼈}
  \begin{phonetics}{低}{di1}[][HSK 2]
    \definition*{s.}{sobrenome Di}
    \definition{adj.}{baixo; distância pequena de baixo para cima; próximo ao solo | abaixo da média; abaixo do padrão geral | inferior (em grau); de nível inferior}
    \definition{v.}{deixar cair; pendurar; abaixar (a cabeça)}
  \end{phonetics}
\end{entry}

\begin{entry}{低于}{7,3}{⼈、⼆}
  \begin{phonetics}{低于}{di1 yu2}[][HSK 5]
    \definition{v.}{ser inferior a; algo ou fenômeno é, de alguma forma, inferior ou pior do que outra coisa}
  \end{phonetics}
\end{entry}

\begin{entry}{低潮}{7,15}{⼈、⽔}
  \begin{phonetics}{低潮}{di1chao2}
    \definition{s.}{maré baixa/vazante; o nível mais baixo da maré durante um ciclo de maré (distinto da 高潮) | vazante baixa; o ponto mais baixo; uma metáfora para o baixo estágio de desenvolvimento das coisas}
  \seealsoref{高潮}{gao1chao2}
  \end{phonetics}
\end{entry}

\begin{entry}{住}{7}{⼈}
  \begin{phonetics}{住}{zhu4}[][HSK 1]
    \definition{adv.}{firmemente; indica estabilidade ou firmeza}
    \definition{v.}{viver; residir; morar; ficar | parar; cessar | (após um verbo) com firmeza; até parar | hospedar; acomodar | parar; interromper | ser competente; ser qualificado; estar à altura; usado com 得 ou 不, indica que a força é suficiente (ou insuficiente)}
  \seealsoref{不}{bu4}
  \seealsoref{得}{de5}
  \end{phonetics}
\end{entry}

\begin{entry}{住处}{7,5}{⼈、⼡}
  \begin{phonetics}{住处}{zhu4chu4}
    \definition{s.}{morada | habitação | residência}
  \end{phonetics}
\end{entry}

\begin{entry}{住宅}{7,6}{⼈、⼧}
  \begin{phonetics}{住宅}{zhu4zhai2}
    \definition{s.}{residência}
  \end{phonetics}
\end{entry}

\begin{entry}{住房}{7,8}{⼈、⼾}
  \begin{phonetics}{住房}{zhu4fang2}[][HSK 2]
    \definition[套,处]{s.}{habitação; alojamento; casas para as pessoas morarem}
  \end{phonetics}
\end{entry}

\begin{entry}{住所}{7,8}{⼈、⼾}
  \begin{phonetics}{住所}{zhu4suo3}
    \definition[处]{s.}{morada | habitação | residência}
  \end{phonetics}
\end{entry}

\begin{entry}{住院}{7,9}{⼈、⾩}
  \begin{phonetics}{住院}{zhu4 yuan4}[][HSK 2]
    \definition{v.}{estar hospitalizado; estar no hospital; ser internado no hospital para tratamento}
  \end{phonetics}
\end{entry}

\begin{entry}{住嘴}{7,16}{⼈、⼝}
  \begin{phonetics}{住嘴}{zhu4zui3}
    \definition{interj.}{Cale-se!}
    \definition{v.}{calar | calar-se}
  \end{phonetics}
\end{entry}

\begin{entry}{体力}{7,2}{⼈、⼒}
  \begin{phonetics}{体力}{ti3 li4}[][HSK 5]
    \definition{s.}{força física; vigor físico (ou corporal); a força do corpo humano para sustentar suas próprias atividades}
  \end{phonetics}
\end{entry}

\begin{entry}{体内}{7,4}{⼈、⼌}
  \begin{phonetics}{体内}{ti3nei4}
    \definition{adj.}{dentro do corpo | \emph{in vivo} (versus \emph{in vitro} | interno a}
  \end{phonetics}
\end{entry}

\begin{entry}{体会}{7,6}{⼈、⼈}
  \begin{phonetics}{体会}{ti3hui4}[][HSK 3]
    \definition[个,些,种]{s.}{conhecimento; compreensão; experiência pessoal}
    \definition{v.}{perceber; saber (ou aprender) com a experiência}
  \end{phonetics}
\end{entry}

\begin{entry}{体现}{7,8}{⼈、⾒}
  \begin{phonetics}{体现}{ti3xian4}[][HSK 3]
    \definition{v.}{refletir; incorporar; encarnar; uma certa qualidade ou fenômeno se manifesta especificamente em uma determinada coisa}
  \end{phonetics}
\end{entry}

\begin{entry}{体育}{7,8}{⼈、⾁}
  \begin{phonetics}{体育}{ti3yu4}[][HSK 2]
    \definition{s.}{cultura física; treinamento físico; educação cuja principal tarefa é desenvolver a capacidade física e fortalecer a constituição física, alcançada através da participação em várias atividades esportivas | esportes; atividades esportivas; refere-se a esportes}
  \end{phonetics}
\end{entry}

\begin{entry}{体育场}{7,8,6}{⼈、⾁、⼟}
  \begin{phonetics}{体育场}{ti3 yu4 chang3}[][HSK 2]
    \definition[个,座]{s.}{estádio; campo esportivo; espaço ao ar livre para a prática de exercícios físicos ou competições esportivas}
  \end{phonetics}
\end{entry}

\begin{entry}{体育馆}{7,8,11}{⼈、⾁、⾷}
  \begin{phonetics}{体育馆}{ti3 yu4 guan3}[][HSK 2]
    \definition[个,座,家]{s.}{ginásio; locais esportivos ou competições em ambientes fechados geralmente têm arquibancadas fixas}
  \end{phonetics}
\end{entry}

\begin{entry}{体重}{7,9}{⼈、⾥}
  \begin{phonetics}{体重}{ti3 zhong4}[][HSK 4]
    \definition{s.}{peso corporal}
  \end{phonetics}
\end{entry}

\begin{entry}{体积}{7,10}{⼈、⽲}
  \begin{phonetics}{体积}{ti3ji1}[][HSK 5]
    \definition[个]{s.}{volume; quantidade; o tamanho do espaço ocupado pelo objeto}
  \end{phonetics}
\end{entry}

\begin{entry}{体验}{7,10}{⼈、⾺}
  \begin{phonetics}{体验}{ti3yan4}[][HSK 3]
    \definition[种]{s.}{experiência; a sensação adquirida pela experiência pessoal}
    \definition{v.}{aprender através da prática; aprender através da experiência pessoal; entender as coisas através da experiência pessoal}
  \end{phonetics}
\end{entry}

\begin{entry}{体检}{7,11}{⼈、⽊}
  \begin{phonetics}{体检}{ti3 jian3}[][HSK 4]
    \definition{s.}{exame clínico}
    \definition{v.}{fazer um exame médico}
  \end{phonetics}
\end{entry}

\begin{entry}{体操}{7,16}{⼈、⼿}
  \begin{phonetics}{体操}{ti3 cao1}[][HSK 4]
    \definition{s.}{ginástica; esportes, exercícios ou performances de vários movimentos, sem armas ou com o auxílio de determinados equipamentos}
  \end{phonetics}
\end{entry}

\begin{entry}{何}{7}{⼈}
  \begin{phonetics}{何}{he2}
    \definition*{s.}{sobrenome He}
    \definition{adv.}{expressa exclamação, equivalente a 多么}
    \definition{pron.}{O que?; Onde?; Por que? | expressa uma pergunta retórica, equivalente a 岂, 怎}
  \seealsoref{多么}{duo1me5}
  \seealsoref{岂}{qi3}
  \seealsoref{怎}{zen3}
  \end{phonetics}
\end{entry}

\begin{entry}{何不}{7,4}{⼈、⼀}
  \begin{phonetics}{何不}{he2bu4}
    \definition{adv.}{por que não?; use o tom interrogativo para expressar "deveria" ou "pode"}
  \end{phonetics}
\end{entry}

\begin{entry}{何况}{7,7}{⼈、⼎}
  \begin{phonetics}{何况}{he2kuang4}
    \definition{conj.}{além disso | muito menos}
  \end{phonetics}
\end{entry}

\begin{entry}{佛}{7}{⼈}
  \begin{phonetics}{佛}{fo2}
    \definition*{s.}{Buda, abreviação de 佛陀 | Budismo}
    \definition{s.}{imagem de Buda | budista | nome de Buda; escritura budista | uma pessoa que alcançou a perfeição na prática espiritual; budista real | estátua do Buda}
  \seealsoref{佛陀}{fo2tuo2}
  \end{phonetics}
  \begin{phonetics}{佛}{fu2}
    \definition{adv.}{aparentemente}
    \definition{s.}{ornamento da cabeça (feminino)}
  \end{phonetics}
\end{entry}

\begin{entry}{佛陀}{7,7}{⼈、⾩}
  \begin{phonetics}{佛陀}{fo2tuo2}
    \definition{s.}{Buda, um título para Sakyamuni ou uma pessoa que atingiu a iluminação | Buda, uma pessoa que atingiu a Budeidade, ou especificamente Siddhartha Gautama}
  \end{phonetics}
\end{entry}

\begin{entry}{作}{7}{⼈}
  \begin{phonetics}{作}{zuo1}
    \definition{adj.}{(gíria) incômodo}
    \definition{s.}{trabalhador | oficina | (pessoa) de alta manutenção}
  \end{phonetics}
  \begin{phonetics}{作}{zuo4}
    \definition{s.}{escritos ou obras}
    \definition{v.}{fazer | crescer | escrever ou compor | fingir | considerar como | sentir}
  \end{phonetics}
\end{entry}

\begin{entry}{作为}{7,4}{⼈、⼂}
  \begin{phonetics}{作为}{zuo4wei2}[][HSK 4]
    \definition{prep.}{como; na capacidade de; no caráter de; no papel de; em termos de uma certa identidade de uma pessoa ou de uma certa natureza de uma coisa}
    \definition{s.}{ato; ação; conduta; feito; comportamento | conquista; realização; especificamente, uma boa ação}
    \definition{v.}{considerar como; tomar por; olhar como; tratar como | realizar; fazer conquistas; deixar uma marca}
  \end{phonetics}
\end{entry}

\begin{entry}{作文}{7,4}{⼈、⽂}
  \begin{phonetics}{作文}{zuo4wen2}[][HSK 2]
    \definition[篇]{s.}{ensaio; composição; redação}
    \definition{v.+compl.}{(de alunos) escrever uma redação, artigo ou ensaio}
  \end{phonetics}
\end{entry}

\begin{entry}{作业}{7,5}{⼈、⼀}
  \begin{phonetics}{作业}{zuo4ye4}[][HSK 2]
    \definition[份,个]{s.}{tarefa escolar; tarefa de casa atribuída pelos professores aos alunos}
    \definition{v.}{trabalhar; executar tarefa}
  \end{phonetics}
\end{entry}

\begin{entry}{作出}{7,5}{⼈、⼐}
  \begin{phonetics}{作出}{zuo4 chu1}[][HSK 4]
    \definition{v.}{mostrar; tomar (decisões, conclusões, etc. por meio de consideração ou discussão); formar (uma conclusão, decisão, etc.) por meio de consideração ou discussão}
  \end{phonetics}
\end{entry}

\begin{entry}{作用}{7,5}{⼈、⽤}
  \begin{phonetics}{作用}{zuo4yong4}[][HSK 2]
    \definition[副]{s.}{efeito; ação; função; a influência sobre as coisas; o efeito; a utilidade}
    \definition{v.}{afetar; agir sobre; realizar atividades que têm algum impacto nas coisas}
  \end{phonetics}
\end{entry}

\begin{entry}{作者}{7,8}{⼈、⽼}
  \begin{phonetics}{作者}{zuo4zhe3}[][HSK 3]
    \definition[位,名,个]{s.}{autor; escritor; pessoas que escrevem artigos ou criam obras de arte}
  \end{phonetics}
\end{entry}

\begin{entry}{作品}{7,9}{⼈、⼝}
  \begin{phonetics}{作品}{zuo4pin3}[][HSK 3]
    \definition[个,部,篇,幅]{s.}{obra de arte; obras literárias e artísticas}
  \end{phonetics}
\end{entry}

\begin{entry}{作家}{7,10}{⼈、⼧}
  \begin{phonetics}{作家}{zuo4jia1}[][HSK 2]
    \definition[位,名,个,些]{s.}{escritor; autor; pessoas que alcançaram sucesso na criação literária}
  \end{phonetics}
\end{entry}

\begin{entry}{你}{7}{⼈}
  \begin{phonetics}{你}{ni3}[][HSK 1]
    \definition{pron.}{você (segunda pessoa do singular); refere-se à pessoa com quem se está conversando | (referindo-se a qualquer pessoa) você; um; qualquer um | com 我 ou 你 em estruturas paralelas para indicar várias ou muitas pessoas se comportando da mesma maneira}
  \seealsoref{您}{nin2}
  \seealsoref{我}{wo3}
  \end{phonetics}
\end{entry}

\begin{entry}{你们}{7,5}{⼈、⼈}
  \begin{phonetics}{你们}{ni3men5}[][HSK 1]
    \definition{pron.}{você (segunda pessoa do plural); refere-se a mais de uma pessoa ou a várias pessoas, incluindo a outra parte}
  \end{phonetics}
\end{entry}

\begin{entry}{你们的}{7,5,8}{⼈、⼈、⽩}
  \begin{phonetics}{你们的}{ni3men5 de5}
    \definition{pron.}{vossos}
  \end{phonetics}
\end{entry}

\begin{entry}{你好}{7,6}{⼈、⼥}
  \begin{phonetics}{你好}{ni3hao3}
    \definition{interj.}{Olá! | Oi!}
  \end{phonetics}
\end{entry}

\begin{entry}{你的}{7,8}{⼈、⽩}
  \begin{phonetics}{你的}{ni3 de5}
    \definition{pron.}{seu}
  \end{phonetics}
\end{entry}

\begin{entry}{克}{7}{⼗}
  \begin{phonetics}{克}{ke4}[][HSK 2]
    \definition*{s.}{sobrenome Ke}
    \definition{clas.}{grama (g) | unidade tibetana de volume ou medida seca (com capacidade para cerca de 25 jin,  斤, de cevada) | unidade tibetana de área de terra equivalente a cerca de 1 mu, 亩}
    \definition{v.}{poder; ser capaz de | tolerar; conter; restringir; suprimir| subjugar; capturar; conquistar (uma cidade, etc.) | digerir (alimentos) | reduzir; diminuir | definir um limite de tempo}
  \seealsoref{斤}{jin1}
  \seealsoref{亩}{mu3}
  \end{phonetics}
\end{entry}

\begin{entry}{克服}{7,8}{⼗、⽉}
  \begin{phonetics}{克服}{ke4fu2}[][HSK 3]
    \definition{v.}{sobrepujar; superar; conquistar; vencer com força de vontade e determinação (deficiências, erros, fenômenos negativos, condições desfavoráveis, etc.) | aguentar; suportar (dificuldades, inconveniências, etc.)}
  \end{phonetics}
\end{entry}

\begin{entry}{免费}{7,9}{⼉、⾙}
  \begin{phonetics}{免费}{mian3fei4}[][HSK 4]
    \definition{s.}{gratuito; sem custo}
    \definition{v.+compl.}{isentar de taxas; tonar grátis}
  \end{phonetics}
\end{entry}

\begin{entry}{免得}{7,11}{⼉、⼻}
  \begin{phonetics}{免得}{mian3de5}
    \definition{conj.}{de modo a não | para evitar | para que não}
  \end{phonetics}
\end{entry}

\begin{entry}{免税}{7,12}{⼉、⽲}
  \begin{phonetics}{免税}{mian3shui4}
    \definition{adj.}{isento de impostos (tributação)}
    \definition{s.}{livre de impostos | isenção de impostos}
    \definition{v.+compl.}{isentar impostos}
  \end{phonetics}
\end{entry}

\begin{entry}{兵}{7}{⼋}
  \begin{phonetics}{兵}{bing1}[][HSK 4]
    \definition[个,种]{s.}{armas; armamentos | soldado; pessoal militar | exército; tropas | soldado raso; membro mais jovem do exército | assuntos militares (estratégia) | peão, uma das peças do xadrez chinês}
  \end{phonetics}
\end{entry}

\begin{entry}{兵器}{7,16}{⼋、⼝}
  \begin{phonetics}{兵器}{bing1qi4}
    \definition{s.}{armas | armamento}
  \end{phonetics}
\end{entry}

\begin{entry}{况且}{7,5}{⼎、⼀}
  \begin{phonetics}{况且}{kuang4qie3}
    \definition{conj.}{além disso | além do mais}
  \end{phonetics}
\end{entry}

\begin{entry}{冷}{7}{⼎}
  \begin{phonetics}{冷}{leng3}[][HSK 1]
    \definition*{s.}{sobrenome Leng}
    \definition{adj.}{frio; baixa temperatura; sensação de frio | gelado; frio por natureza; sem entusiasmo; sem gentileza | desolado; pouco frequentado; quieto; sem agitação | negligenciado; indesejável; ignorado por todos | raro; estranho; incomum | feito em segredo; filmado de forma escondida; lançado secretamente}
    \definition{v.}{esfriar; resfriar | esfriar; congelar; tornar-se indiferente, apático | ignorar}
  \end{phonetics}
\end{entry}

\begin{entry}{冷门}{7,3}{⼎、⾨}
  \begin{phonetics}{冷门}{leng3men2}
    \definition{s.}{uma profissão, ofício ou ramo de aprendizagem que recebe pouca atenção | um vencedor inesperado; azarão}
  \end{phonetics}
\end{entry}

\begin{entry}{冷静}{7,14}{⼎、⾭}
  \begin{phonetics}{冷静}{leng3jing4}[][HSK 4]
    \definition{adj.}{calmo; descreve uma pessoa que consegue ficar atenta em uma situação importante ou de emergência e não toma decisões aleatórias por causa de seus sentimentos no momento | (lugar) tranquilo; quieto; deserto}
  \end{phonetics}
\end{entry}

\begin{entry}{冻}{7}{⼎}
  \begin{phonetics}{冻}{dong4}[][HSK 5]
    \definition*{s.}{sobrenome Dong}
    \definition{s.}{geleia; gelatina}
    \definition{v.}{congelar; ser congelado | ficar com frio ou sentir frio}
  \end{phonetics}
\end{entry}

\begin{entry}{初}{7}{⾐}
  \begin{phonetics}{初}{chu1}[][HSK 3]
    \definition*{s.}{sobrenome Chu}
    \definition{adj.}{primeiro (em ordem) | elementar; rudimentar | original}
    \definition{adv.}{pela primeira vez; apenas começando; indica que a ação está ocorrendo pela primeira vez ou acabou de começar}
    \definition{pref.}{anexado a alguns substantivos ou números cardinais, indicando o primeiro}
    \definition{s.}{no início de; na primeira parte de | o estágio júnior (pleno; sênior)}
  \end{phonetics}
\end{entry}

\begin{entry}{初中}{7,4}{⾐、⼁}
  \begin{phonetics}{初中}{chu1 zhong1}[][HSK 3]
    \definition[所,个]{s.}{ensino médio; ensino fundamental}
  \end{phonetics}
\end{entry}

\begin{entry}{初心}{7,4}{⾐、⼼}
  \begin{phonetics}{初心}{chu1xin1}
    \definition{s.}{intenção original de alguém, aspiração, etc. | (budismo) ``mente do iniciante'' (ter a mente aberta quando estudando um assunto como um iniciante no assunto teria)}
  \end{phonetics}
\end{entry}

\begin{entry}{初级}{7,6}{⾐、⽷}
  \begin{phonetics}{初级}{chu1ji2}[][HSK 3]
    \definition{adj.}{elementar; primário; júnior; inicial; o nível mais baixo; de baixa qualidade}
  \end{phonetics}
\end{entry}

\begin{entry}{初步}{7,7}{⾐、⽌}
  \begin{phonetics}{初步}{chu1bu4}[][HSK 3]
    \definition{adj.}{inicial; preliminar; imaturo, incompleto}
  \end{phonetics}
\end{entry}

\begin{entry}{初期}{7,12}{⾐、⽉}
  \begin{phonetics}{初期}{chu1 qi1}[][HSK 5]
    \definition{s.}{primórdio; estágio inicial; primeiros dias; estágio preliminar; período inicial}
  \end{phonetics}
\end{entry}

\begin{entry}{判断}{7,11}{⼑、⽄}
  \begin{phonetics}{判断}{pan4duan4}[][HSK 3]
    \definition[个,项]{s.}{julgamento; conclusões tiradas após reflexão e análise}
    \definition{v.}{julgar; decidir}
  \end{phonetics}
\end{entry}

\begin{entry}{利}{7}{⼑}
  \begin{phonetics}{利}{li4}
    \definition*{s.}{sobrenome Li}
    \definition{adj.}{afiado; cortante | favorável; conveniente; sem dificuldades; sem ou com poucas dificuldades}
    \definition{s.}{benefício; vantagem | lucro; ganhos; juros}
    \definition{v.}{beneficiar; tornar vantajoso}
  \end{phonetics}
\end{entry}

\begin{entry}{利用}{7,5}{⼑、⽤}
  \begin{phonetics}{利用}{li4yong4}[][HSK 3]
    \definition{v.}{usar; utilizar; fazer uso de; fazer com que algo ou alguém funcione bem| explorar; tirar vantagem de; usar meios para fazer com que pessoas ou coisas sirvam aos seus interesses}
  \end{phonetics}
\end{entry}

\begin{entry}{利息}{7,10}{⼑、⼼}
  \begin{phonetics}{利息}{li4xi1}[][HSK 4]
    \definition{s.}{acréscimo; juros; dinheiro recebido além do valor principal como resultado de depósitos ou empréstimos (diferenciado de 本金)}
  \seealsoref{本金}{ben3 jin1}
  \end{phonetics}
\end{entry}

\begin{entry}{利润}{7,10}{⼑、⽔}
  \begin{phonetics}{利润}{li4run4}[][HSK 5]
    \definition[笔]{s.}{lucro; o dinheiro ganho com atividades comerciais e industriais}
  \end{phonetics}
\end{entry}

\begin{entry}{利益}{7,10}{⼑、⽫}
  \begin{phonetics}{利益}{li4yi4}[][HSK 4]
    \definition[个,种]{s.}{ganho; lucro; juros; benefício}
  \end{phonetics}
\end{entry}

\begin{entry}{别}{7}{⼑}
  \begin{phonetics}{别}{bie2}[][HSK 1,4]
    \definition*{s.}{sobrenome Bie}
    \definition{adv.}{não; nada de (pedir a alguém para não fazer); é melhor não | talvez, usado em conjunto com a palavra 是 para indicar especulação}
    \definition{pron.}{outro; algum outro}
    \definition{s.}{distinção; diferença | classificação}
    \definition{v.}{deixar; partir; separar | diferenciar; distinguir; encontrar aspectos diferentes | fixar objetos com pinos | girar; virar | aderir; colar; preder}
  \seealsoref{是}{shi4}
  \end{phonetics}
  \begin{phonetics}{别}{bie4}
    \definition{v.}{fazer com que alguém mude seus hábitos, opiniões, etc. | mudar a opinião de alguém (usado principalmente em 别不过)}
  \seealsoref{别不过}{bie2 bu2guo4}
  \end{phonetics}
\end{entry}

\begin{entry}{别人}{7,2}{⼑、⼈}
  \begin{phonetics}{别人}{bie2 ren2}[][HSK 1]
    \definition{pron.}{outros; outras pessoas}
    \definition{s.}{outros; pessoas; outras pessoas; refere-se a alguém diferente de si mesmo}
  \end{phonetics}
\end{entry}

\begin{entry}{别不过}{7,4,6}{⼑、⼀、⾡}
  \begin{phonetics}{别不过}{bie2 bu2guo4}
    \definition{expr.}{Não, mas}
  \end{phonetics}
\end{entry}

\begin{entry}{别的}{7,8}{⼑、⽩}
  \begin{phonetics}{别的}{bie2 de5}[][HSK 1]
    \definition{pron.}{outro; o resto}
  \end{phonetics}
\end{entry}

\begin{entry}{别说}{7,9}{⼑、⾔}
  \begin{phonetics}{别说}{bie2shuo1}
    \definition{v.}{não falar de | não mencionar}
  \end{phonetics}
\end{entry}

\begin{entry}{助手}{7,4}{⼒、⼿}
  \begin{phonetics}{助手}{zhu4shou3}[][HSK 5]
    \definition[个]{s.}{ajudante; auxiliar; assistente; alguém que ajuda os outros com seu trabalho}
  \end{phonetics}
\end{entry}

\begin{entry}{助兴}{7,6}{⼒、⼋}
  \begin{phonetics}{助兴}{zhu4xing4}
    \definition{v.+compl.}{animar as coisas | juntar-se à diversão}
  \end{phonetics}
\end{entry}

\begin{entry}{助理}{7,11}{⼒、⽟}
  \begin{phonetics}{助理}{zhu4li3}[][HSK 5]
    \definition[个,名,位]{s.}{deputado; assistente; auxiliar do diretor responsável (geralmente usado em cargos) | ajudante; assistente; pessoa que auxilia o responsável a fazer as coisas}
  \end{phonetics}
\end{entry}

\begin{entry}{努力}{7,2}{⼒、⼒}
  \begin{phonetics}{努力}{nu3li4}[][HSK 2]
    \definition{adj.}{extenuante; árduo | diligente; trabalhador; quem faz as coisas com o máximo de capacidade ou esforço possível}
    \definition{s.}{esforço; tentativa; fazer o melhor possível}
    \definition{v.}{fazer grandes esforços; esforçar-se; empenhar-se | esforçar-se; usar toda a força possível}
  \end{phonetics}
\end{entry}

\begin{entry}{劳工同事}{7,3,6,8}{⼒、⼯、⼝、⼅}
  \begin{phonetics}{劳工同事}{lao2gong1 tong2shi4}
    \definition{s.}{colaborador | colega de trabalho}
  \end{phonetics}
\end{entry}

\begin{entry}{劳动}{7,6}{⼒、⼒}
  \begin{phonetics}{劳动}{lao2dong4}[][HSK 5]
    \definition[次]{s.}{trabalho; mão de obra; atividades intelectuais ou físicas que podem criar valor | trabalho físico; trabalho manual; referindo-se especificamente ao trabalho físico}
    \definition{v.}{realizar trabalho físico}
  \end{phonetics}
\end{entry}

\begin{entry}{医}{7}{⼖}
  \begin{phonetics}{医}{yi1}
    \definition{s.}{médico | medicina}
    \definition{v.}{curar | tratar}
  \end{phonetics}
\end{entry}

\begin{entry}{医生}{7,5}{⼖、⽣}
  \begin{phonetics}{医生}{yi1sheng1}[][HSK 1]
    \definition[位,个,名]{s.}{médico; clínico; pessoa que possui conhecimentos médicos e cuja profissão é tratar doenças}
  \end{phonetics}
\end{entry}

\begin{entry}{医疗}{7,7}{⼖、⽧}
  \begin{phonetics}{医疗}{yi1 liao2}[][HSK 4]
    \definition{s.}{tratamento médico; tratamento de doenças}
  \end{phonetics}
\end{entry}

\begin{entry}{医学}{7,8}{⼖、⼦}
  \begin{phonetics}{医学}{yi1 xue2}[][HSK 4]
    \definition{s.}{medicina; iatrologia; ciência médica; ciência da prevenção e do tratamento de doenças e da proteção e promoção da saúde humana}
  \end{phonetics}
\end{entry}

\begin{entry}{医院}{7,9}{⼖、⾩}
  \begin{phonetics}{医院}{yi1yuan4}[][HSK 1]
    \definition[家,所,个]{s.}{hospital; instituições que tratam e cuidam de pacientes, e também realizam exames de saúde, prevenção de doenças, etc.}
  \end{phonetics}
\end{entry}

\begin{entry}{即}{7}{⼙}
  \begin{phonetics}{即}{ji2}
    \definition{conj.}{e | até | mesmo se/embora}
  \end{phonetics}
\end{entry}

\begin{entry}{即使}{7,8}{⼙、⼈}
  \begin{phonetics}{即使}{ji2shi3}[][HSK 5]
    \definition{conj.}{mesmo; mesmo que; mesmo se; apesar de; expressando uma concessão hipotética}
  \end{phonetics}
\end{entry}

\begin{entry}{即或}{7,8}{⼙、⼽}
  \begin{phonetics}{即或}{ji2huo4}
    \definition{conj.}{mesmo se/embora}
  \end{phonetics}
\end{entry}

\begin{entry}{即若}{7,8}{⼙、⾋}
  \begin{phonetics}{即若}{ji2ruo4}
    \definition{conj.}{mesmo se/embora}
  \end{phonetics}
\end{entry}

\begin{entry}{即便}{7,9}{⼙、⼈}
  \begin{phonetics}{即便}{ji2bian4}
    \definition{conj.}{mesmo se/embora}
  \end{phonetics}
\end{entry}

\begin{entry}{即将}{7,9}{⼙、⼨}
  \begin{phonetics}{即将}{ji2jiang1}[][HSK 4]
    \definition{adv.}{em breve; estar prestes a; estar a ponto de}
  \end{phonetics}
\end{entry}

\begin{entry}{即是}{7,9}{⼙、⽇}
  \begin{phonetics}{即是}{ji2shi4}
    \definition{conj.}{aquilo é}
  \end{phonetics}
\end{entry}

\begin{entry}{却}{7}{⼙}
  \begin{phonetics}{却}{que4}[][HSK 4]
    \definition{adv.}{mas; contudo; no entanto; enquanto; indica um ponto de virada}
    \definition{v.}{recuar; retroceder | afastar; repelir; desencorajar | declinar; recusar; rejeitar}
    \definition{v.aux.}{usado depois de certos verbos para indicar a conclusão de uma ação}
  \end{phonetics}
\end{entry}

\begin{entry}{却是}{7,9}{⼙、⽇}
  \begin{phonetics}{却是}{que4shi4}
    \definition{conj.}{no entanto | realmente | o fato é\dots | mas isso é\dots}
  \end{phonetics}
\end{entry}

\begin{entry}{县}{7}{⼛}
  \begin{phonetics}{县}{xian4}[][HSK 4]
    \definition[个]{s.}{condado; unidade de divisão administrativa}
  \end{phonetics}
\end{entry}

\begin{entry}{君主立宪制}{7,5,5,9,8}{⼝、⼂、⽴、⼧、⼑}
  \begin{phonetics}{君主立宪制}{jun1zhu3li4xian4zhi4}
    \definition{s.}{monarquia constitucional}
  \end{phonetics}
\end{entry}

\begin{entry}{吟诗}{7,8}{⼝、⾔}
  \begin{phonetics}{吟诗}{yin2shi1}
    \definition{v.}{recitar poesia}
  \end{phonetics}
\end{entry}

\begin{entry}{否}{7}{⼝}
  \begin{phonetics}{否}{fou3}
    \definition{adv.}{não; expressa discordância, equivalente à palavra falada 不 | usado no final de uma pergunta para indicar investigação | 是否, 能否 e 可否 que significa respectivamente 是不是, 能不能 e 可不可}
    \definition{v.}{negar}
  \seealsoref{不}{bu4}
  \seealsoref{可}{ke3}
  \seealsoref{能}{neng2}
  \seealsoref{是}{shi4}
  \end{phonetics}
  \begin{phonetics}{否}{pi3}
    \definition{adj.}{ruim; maligno; perverso}
    \definition{v.}{censurar}
  \end{phonetics}
\end{entry}

\begin{entry}{否认}{7,4}{⼝、⾔}
  \begin{phonetics}{否认}{fou3ren4}[][HSK 3]
    \definition{v.}{negar; repudiar; não reconhecer}
  \end{phonetics}
\end{entry}

\begin{entry}{否则}{7,6}{⼝、⼑}
  \begin{phonetics}{否则}{fou3ze2}[][HSK 4]
    \definition{conj.}{senão; se não; ou então; se não for isso}
  \end{phonetics}
\end{entry}

\begin{entry}{否定}{7,8}{⼝、⼧}
  \begin{phonetics}{否定}{fou3ding4}[][HSK 3]
    \definition{adj.}{negativo; contrário}
    \definition{v.}{rejeitar; negar a existência ou a autenticidade de algo}
  \end{phonetics}
\end{entry}

\begin{entry}{吧}{7}{⼝}
  \begin{phonetics}{吧}{ba1}
    \definition{s.}{som de estalo, som crepitante |  abreviação de bar, 酒吧 | cibercafé; um local público que fornece computadores e serviços de \emph{Internet} onde as pessoas podem navegar, jogar, etc.}
    \definition{v.}{fumar; dar uma tragada (puxar) no cachimbo}
  \seealsoref{酒吧}{jiu3ba1}
  \end{phonetics}
  \begin{phonetics}{吧}{ba5}[][HSK 1]
    \definition{part.}{indica discussão, sugestão, solicitação ou comando no final de uma frase | indica concordância ou aprovação no final de uma frase | indica uma pergunta ou especulação no final de uma frase | indica incerteza no final de uma frase | em uma frase, indica uma pausa, carrega um tom hipotético, frequentemente apresenta um contraste e implica um dilema}
  \end{phonetics}
\end{entry}

\begin{entry}{吨}{7}{⼝}
  \begin{phonetics}{吨}{dun1}[][HSK 5]
    \definition{clas.}{tonelada}
  \end{phonetics}
\end{entry}

\begin{entry}{含}{7}{⼝}
  \begin{phonetics}{含}{han2}[][HSK 4]
    \definition{v.}{manter na boca (sem engolir ou cuspir) | conter; incluir | cuidar; acalentar; abrigar}
  \end{phonetics}
\end{entry}

\begin{entry}{含义}{7,3}{⼝、⼂}
  \begin{phonetics}{含义}{han2yi4}[][HSK 4]
    \definition[个,种,层]{s.}{sentido; mensagem; significado; implicação}
  \end{phonetics}
\end{entry}

\begin{entry}{含有}{7,6}{⼝、⽉}
  \begin{phonetics}{含有}{han2 you3}[][HSK 4]
    \definition{v.}{conter; ter; incluir}
  \end{phonetics}
\end{entry}

\begin{entry}{含金量}{7,8,12}{⼝、⾦、⾥}
  \begin{phonetics}{含金量}{han2jin1liang4}
    \definition{adj.}{conteúdo de ouro | (fig.) valioso}
  \end{phonetics}
\end{entry}

\begin{entry}{含量}{7,12}{⼝、⾥}
  \begin{phonetics}{含量}{han2 liang4}[][HSK 4]
    \definition{s.}{conteúdo; a quantidade de um componente contido em uma substância}
  \end{phonetics}
\end{entry}

\begin{entry}{听}{7}{⼝}
  \begin{phonetics}{听}{ting1}[][HSK 1]
    \definition{clas.}{latas; usado para bebidas e alimentos para levar consigo}
    \definition{s.}{lata; embalagem metálica; recipiente cilíndrico utilizado para armazenar bebidas, alimentos, etc.}
    \definition{v.}{ouvir; escutar | obedecer; dar ouvidos; aceitar | supervisionar; administrar; tratar (assuntos políticos); julgar (casos) | permitir; deixar ser; deixar fazer}
  \end{phonetics}
  \begin{phonetics}{听}{yin3}
    \definition[个]{s.}{lata; embalagem metálica}
  \end{phonetics}
\end{entry}

\begin{entry}{听力}{7,2}{⼝、⼒}
  \begin{phonetics}{听力}{ting1li4}[][HSK 3]
    \definition{s.}{audição; capacidade auditiva | compreensão auditiva (na aprendizagem de línguas)}
  \end{phonetics}
\end{entry}

\begin{entry}{听力理解}{7,2,11,13}{⼝、⼒、⽟、⾓}
  \begin{phonetics}{听力理解}{ting1li4li3jie3}
    \definition{s.}{compreensão auditiva}
  \end{phonetics}
\end{entry}

\begin{entry}{听小骨}{7,3,9}{⼝、⼩、⾻}
  \begin{phonetics}{听小骨}{ting1xiao3gu3}
    \definition{s.}{ossículos (do ouvido médio)}
  \seealsoref{听骨}{ting1gu3}
  \end{phonetics}
\end{entry}

\begin{entry}{听见}{7,4}{⼝、⾒}
  \begin{phonetics}{听见}{ting1 jian4}[][HSK 1]
    \definition{v.}{ouvir; conseguir ouvir}
  \end{phonetics}
\end{entry}

\begin{entry}{听写}{7,5}{⼝、⼍}
  \begin{phonetics}{听写}{ting1 xie3}[][HSK 1]
    \definition{s.}{ditado}
    \definition{v.}{ouvir e escrever}
  \end{phonetics}
\end{entry}

\begin{entry}{听众}{7,6}{⼝、⼈}
  \begin{phonetics}{听众}{ting1 zhong4}[][HSK 3]
    \definition[位,名,个]{s.}{audiência; ouvintes; pessoas que ouvem palestras, música ou transmissões}
  \end{phonetics}
\end{entry}

\begin{entry}{听会}{7,6}{⼝、⼈}
  \begin{phonetics}{听会}{ting1hui4}
    \definition{v.}{participar de uma reunião (e ouvir o que é discutido)}
  \end{phonetics}
\end{entry}

\begin{entry}{听戏}{7,6}{⼝、⼽}
  \begin{phonetics}{听戏}{ting1xi4}
    \definition{v.}{assistir a uma ópera | ver uma ópera}
  \end{phonetics}
\end{entry}

\begin{entry}{听讲}{7,6}{⼝、⾔}
  \begin{phonetics}{听讲}{ting1 jiang3}[][HSK 2]
    \definition{v.+compl.}{assistir a uma palestra; ouvir palestras ou discursos}
  \end{phonetics}
\end{entry}

\begin{entry}{听来}{7,7}{⼝、⽊}
  \begin{phonetics}{听来}{ting1lai2}
    \definition{v.}{ouvir de algum lugar | soar (antigo, estrangeiro, excitante, certo, etc.) | soar como se (ou seja, dar uma impressão ao ouvinte)}
  \end{phonetics}
\end{entry}

\begin{entry}{听凭}{7,8}{⼝、⼏}
  \begin{phonetics}{听凭}{ting1ping2}
    \definition{v.}{permitir (alguém a fazer o que desejar)}
  \end{phonetics}
\end{entry}

\begin{entry}{听到}{7,8}{⼝、⼑}
  \begin{phonetics}{听到}{ting1 dao4}[][HSK 1]
    \definition{v.}{ouvir, escutar; ouvir atentamente, escutar atentamente}
  \end{phonetics}
\end{entry}

\begin{entry}{听命}{7,8}{⼝、⼝}
  \begin{phonetics}{听命}{ting1ming4}
    \definition{v.}{obedecer ordens | receber ordens}
  \end{phonetics}
\end{entry}

\begin{entry}{听说}{7,9}{⼝、⾔}
  \begin{phonetics}{听说}{ting1 shuo1}[][HSK 2]
    \definition{v.}{ser informado; ouvir falar de; ouvir dizer | ouvir e falar}
  \end{phonetics}
\end{entry}

\begin{entry}{听骨}{7,9}{⼝、⾻}
  \begin{phonetics}{听骨}{ting1gu3}
    \definition{s.}{ossículos (do ouvido médio)}
  \seealsoref{听小骨}{ting1xiao3gu3}
  \end{phonetics}
\end{entry}

\begin{entry}{听断}{7,11}{⼝、⽄}
  \begin{phonetics}{听断}{ting1duan4}
    \definition{v.}{ouvir e decidir | julgar (ou seja, ouvir e julgar em um tribunal)}
  \end{phonetics}
\end{entry}

\begin{entry}{听随}{7,11}{⼝、⾩}
  \begin{phonetics}{听随}{ting1sui2}
    \definition{v.}{permitir | obedecer}
  \end{phonetics}
\end{entry}

\begin{entry}{启发}{7,5}{⼝、⼜}
  \begin{phonetics}{启发}{qi3fa1}[][HSK 5]
    \definition{s.}{iluminação; esclarecimento; fenômenos e princípios que levam as pessoas a refletir e a abrir suas mentes}
    \definition{v.}{despertar; inspirar; esclarecer; orientar, fazer com que compreendam}
  \end{phonetics}
\end{entry}

\begin{entry}{启动}{7,6}{⼝、⼒}
  \begin{phonetics}{启动}{qi3 dong4}[][HSK 5]
    \definition{v.}{ligar (uma máquina); acionar; ligar máquinas, equipamentos elétricos, etc., para começar a trabalhar | entrar em vigor; começar a vigorar e a ser implementados planos, projetos, documentos jurídicos, etc.}
  \end{phonetics}
\end{entry}

\begin{entry}{启事}{7,8}{⼝、⼅}
  \begin{phonetics}{启事}{qi3shi4}[][HSK 5]
    \definition{s.}{aviso; anúncio; texto publicado em jornais ou afixado em paredes com o objetivo de divulgar publicamente algo}
  \end{phonetics}
\end{entry}

\begin{entry}{吵}{7}{⼝}
  \begin{phonetics}{吵}{chao3}[][HSK 3]
    \definition{adj.}{barulhento; ruidoso e perturbador}
    \definition{v.}{brigar; discutir; disputar}
  \end{phonetics}
\end{entry}

\begin{entry}{吵架}{7,9}{⼝、⽊}
  \begin{phonetics}{吵架}{chao3jia4}[][HSK 3]
    \definition{v.+compl.}{brigar; discutir; ter uma discussão acalorada}
  \end{phonetics}
\end{entry}

\begin{entry}{吹}{7}{⼝}
  \begin{phonetics}{吹}{chui1}[][HSK 2]
    \definition{v.}{soprar; baforar | tocar (instrumentos de sopro) | (do vento) soprar | gabar-se; vangloriar-se | elogiar; louvar aos céus; adular | (relacionamento) romper; separar-se; (assunto) fracassar}
  \end{phonetics}
\end{entry}

\begin{entry}{吹牛}{7,4}{⼝、⽜}
  \begin{phonetics}{吹牛}{chui1niu2}
    \definition{v.+compl.}{ogulhar-se | gabar-se | destacar-se}
  \end{phonetics}
\end{entry}

\begin{entry}{吾}{7}{⼝}
  \begin{phonetics}{吾}{wu2}
    \definition*{s.}{sobrenome Wu}
    \definition{pron.}{eu | (antigo) meu}
  \end{phonetics}
\end{entry}

\begin{entry}{呀}{7}{⼝}
  \begin{phonetics}{呀}{ya5}[][HSK 4]
    \definition{part.}{usado no lugar de 啊 quando a palavra anterior termina com o som a, e, i, o ou ü}
  \seealsoref{啊}{a5}
  \end{phonetics}
\end{entry}

\begin{entry}{呆}{7}{⼝}
  \begin{phonetics}{呆}{dai1}[][HSK 5]
    \definition*{s.}{sobrenome Dai}
    \definition{adj.}{maçante; de raciocínio lento | em branco; de madeira; rígido; inflexível}
    \definition{v.}{ficar; permanecer}
  \end{phonetics}
\end{entry}

\begin{entry}{告}{7}{⼝}
  \begin{phonetics}{告}{gao4}
    \definition*{s.}{sobrenome Gao}
    \definition{s.}{anúncio; declaração; notificação}
    \definition{v.}{informar; contar; notificar; explicar aos outros | acusar; processar; relatar | pedir; requisitar; solicitar | dar a conhecer; mostrar | anunciar; declarar; proclamar}
  \end{phonetics}
\end{entry}

\begin{entry}{告别}{7,7}{⼝、⼑}
  \begin{phonetics}{告别}{gao4bie2}[][HSK 3]
    \definition{v.+compl.}{dizer adeus a; expressar a outros, por meio de palavras, que está prestes a partir | deixar; sair; partir de | prestar as últimas homenagens ao falecido}
  \end{phonetics}
\end{entry}

\begin{entry}{告诉}{7,7}{⼝、⾔}
  \begin{phonetics}{告诉}{gao4su4}
    \definition{v.}{dizer; informar (dar a conhecer); dizer aos outros, para que todos saibam}
  \end{phonetics}
  \begin{phonetics}{告诉}{gao4su5}[][HSK 1]
    \definition{v.}{dizer; informar (dar a conhecer)}
  \end{phonetics}
\end{entry}

\begin{entry}{告急}{7,9}{⼝、⼼}
  \begin{phonetics}{告急}{gao4ji2}
    \definition{v.}{estar em estado de emergência | relatar uma emergência | solicitar assistência de emergência}
  \end{phonetics}
\end{entry}

\begin{entry}{员}{7}{⼝}
  \begin{phonetics}{员}{yuan2}[][HSK 3]
    \definition{clas.}{para comandantes militares}
    \definition{s.}{uma pessoa envolvida em algum campo de atividade; refere-se a pessoas que trabalham ou estudam | membro; refere-se aos membros de um grupo ou organização | vizinhança}
  \end{phonetics}
\end{entry}

\begin{entry}{员工}{7,3}{⼝、⼯}
  \begin{phonetics}{员工}{yuan2gong1}[][HSK 3]
    \definition[位,名,个]{s.}{equipe; funcionário; trabalhador; pessoal}
  \end{phonetics}
\end{entry}

\begin{entry}{园林}{7,8}{⼞、⽊}
  \begin{phonetics}{园林}{yuan2lin2}[][HSK 5]
    \definition{s.}{parque; jardim; área paisagística com plantas e árvores para as pessoas apreciarem e descansarem.}
  \end{phonetics}
\end{entry}

\begin{entry}{囯}{7}{⼞}
  \begin{phonetics}{囯}{guo2}
    \definition*{s.}{sobrenome Guo}
    \definition{adj.}{do estado; nacional | do nosso país; Chinês | do país}
    \definition{s.}{país; nação; estado | o melhor da nação | o melhor; o mais bonito do país}
    \variantof{国}
  \end{phonetics}
\end{entry}

\begin{entry}{困}{7}{⼞}
  \begin{phonetics}{困}{kun4}[][HSK 3]
    \definition{adj.}{cansado; exausto; fatigado | difícil; complicado; difícil e penoso; pobre e miserável | sonolento; com sono; cansado, com vontade de dormir}
    \definition{v.}{ficar encalhado; estar em apuros; preso em dificuldades e sofrimentos ou limitado por circunstâncias e condições que não pode escapar | cercar; envolver; imobilizar; controlar dentro de um determinado limite | dormir}
  \end{phonetics}
\end{entry}

\begin{entry}{困扰}{7,7}{⼞、⼿}
  \begin{phonetics}{困扰}{kun4 rao3}[][HSK 5]
    \definition{v.}{perturbar; deixar perplexo; perseguir}
  \end{phonetics}
\end{entry}

\begin{entry}{困难}{7,10}{⼞、⾫}
  \begin{phonetics}{困难}{kun4nan5}[][HSK 3]
    \definition{adj.}{dificuldades financeiras; circunstâncias difíceis | complicado; complexo; difícil; árduo; a situação é complexa e há muitos obstáculos}
    \definition[种]{s.}{dificuldade; situação difícil; problemas ou situações difíceis de resolver no trabalho e na vida}
  \end{phonetics}
\end{entry}

\begin{entry}{围}{7}{⼞}
  \begin{phonetics}{围}{wei2}[][HSK 3]
    \definition*{s.}{sobrenome Wei}
    \definition{clas.}{o comprimento das duas mãos com os polegares e os dedos indicadores juntos ou dos dois braços juntos}
    \definition{s.}{em volta de tudo; ao redor}
    \definition{v.}{cercar; rodear; circundar; encurralar; cercar tudo, impedindo a passagem entre o interior e o exterior | envolver; contornar}
  \end{phonetics}
\end{entry}

\begin{entry}{围巾}{7,3}{⼞、⼱}
  \begin{phonetics}{围巾}{wei2jin1}[][HSK 4]
    \definition[条]{s.}{lenço; cachecol; echarpe; gravata; tiras longas de malha ou tecido usadas ao redor do pescoço para aquecimento, proteção do colarinho ou decoração}
  \end{phonetics}
\end{entry}

\begin{entry}{围绕}{7,9}{⼞、⽷}
  \begin{phonetics}{围绕}{wei2rao4}[][HSK 5]
    \definition{v.}{girar; circundar; dar voltas; girar em torno de algo; cercar | concentrar-se em; centrar-se em; centrar-se em uma questão ou evento (para realizar atividades)}
  \end{phonetics}
\end{entry}

\begin{entry}{坏}{7}{⼟}
  \begin{phonetics}{坏}{huai4}[][HSK 1]
    \definition{adj.}{ruim; prejudicial; insatisfatório; péssimo | mal; extremamente; indica um grau profundo, geralmente usado após verbos ou adjetivos que expressam estado psicológico | podre; estragado; impróprio; prejudicial ao uso}
    \definition[种]{s.}{ideia maligna; truque sujo; péssima ideia}
    \definition{v.}{estragar; destruir; corromper}
  \end{phonetics}
\end{entry}

\begin{entry}{坏人}{7,2}{⼟、⼈}
  \begin{phonetics}{坏人}{huai4 ren2}[][HSK 2]
    \definition[个,种]{s.}{malfeitor; canalha; pessoa má; pessoa de má qualidade; pessoa que faz coisas ruins}
  \end{phonetics}
\end{entry}

\begin{entry}{坏处}{7,5}{⼟、⼡}
  \begin{phonetics}{坏处}{huai4 chu4}[][HSK 2]
    \definition[个]{s.}{dano; prejuízo; desvantagem; fatores prejudiciais a pessoas ou coisas}
  \end{phonetics}
\end{entry}

\begin{entry}{坏蛋}{7,11}{⼟、⾍}
  \begin{phonetics}{坏蛋}{huai4dan4}
    \definition{s.}{bastardo | canalha | pessoa má}
  \end{phonetics}
\end{entry}

\begin{entry}{坐}{7}{⼟}
  \begin{phonetics}{坐}{zuo4}[][HSK 1]
    \definition*{s.}{sobrenome Zuo}
    \definition{adv.}{sem motivo algum; sem causa ou razão; sem motivo aparente}
    \definition{prep.}{porque; pelo fato de que; pela razão de que; pelo motivo de que}
    \definition{s.}{assento; lugar; posição}
    \definition{v.}{sentar; sentar-se; ocupar um lugar; colocar os glúteos sobre um objeto para apoiar o peso corporal | pegar; viajar de; pegar carona | ter as costas voltadas para | colocar (uma panela, chaleira, etc.) no fogo | recuo; coice (de rifles, armas, etc.)  | produzir frutos; formar sementes | ser punido; ser acusado de crime | contrair (ou ter) uma doença; sofrer de uma doença | (um edifício) afundar; ceder}
  \end{phonetics}
\end{entry}

\begin{entry}{坐下}{7,3}{⼟、⼀}
  \begin{phonetics}{坐下}{zuo4 xia5}[][HSK 1]
    \definition{v.}{sentar-se; tomar um assento; passar da posição em pé para a posição sentada}
  \end{phonetics}
\end{entry}

\begin{entry}{坐车}{7,4}{⼟、⾞}
  \begin{phonetics}{坐车}{zuo4che1}
    \definition{v.}{andar de carro, ônibus, trem, etc.}
  \end{phonetics}
\end{entry}

\begin{entry}{坐好}{7,6}{⼟、⼥}
  \begin{phonetics}{坐好}{zuo4hao3}
    \definition{v.}{sentar-se corretamente | sentar direito}
  \end{phonetics}
\end{entry}

\begin{entry}{坐享}{7,8}{⼟、⼇}
  \begin{phonetics}{坐享}{zuo4xiang3}
    \definition{v.}{curtir algo sem levantar um dedo}
  \end{phonetics}
\end{entry}

\begin{entry}{坐垫}{7,9}{⼟、⼟}
  \begin{phonetics}{坐垫}{zuo4dian4}
    \definition[块]{s.}{assento (motocicleta) | almofada}
  \end{phonetics}
\end{entry}

\begin{entry}{坐标}{7,9}{⼟、⽊}
  \begin{phonetics}{坐标}{zuo4biao1}
    \definition{s.}{coordenada (geometria)}
  \end{phonetics}
\end{entry}

\begin{entry}{坑}{7}{⼟}
  \begin{phonetics}{坑}{keng1}
    \definition{s.}{poço | depressão | túnel | buraco no chão}
    \definition{v.}{enganar | trapacear}
  \end{phonetics}
\end{entry}

\begin{entry}{坑人}{7,2}{⼟、⼈}
  \begin{phonetics}{坑人}{keng1ren2}
    \definition{v.+compl.}{trapacear alguém}
  \end{phonetics}
\end{entry}

\begin{entry}{块}{7}{⼟}
  \begin{phonetics}{块}{kuai4}[][HSK 1]
    \definition{clas.}{usado para coisas em pedaços | usado para coisas em pedaços ou em algumas formas de folhas | usado para moedas de prata ou notas de papel equivalentes a 圆}
    \definition{s.}{pedaço; pedaço (de terra); peça; algo que forma um pedaço ou massa}
  \seealsoref{圆}{yuan2}
  \end{phonetics}
\end{entry}

\begin{entry}{坚决}{7,6}{⼟、⼎}
  \begin{phonetics}{坚决}{jian1jue2}[][HSK 3]
    \definition{adj.}{firme; resoluto; (atitude, opinião, ação, etc.) determinado e inabalável}
  \end{phonetics}
\end{entry}

\begin{entry}{坚守}{7,6}{⼟、⼧}
  \begin{phonetics}{坚守}{jian1shou3}
    \definition{v.}{agarrar-se}
  \end{phonetics}
\end{entry}

\begin{entry}{坚固}{7,8}{⼟、⼞}
  \begin{phonetics}{坚固}{jian1gu4}[][HSK 4]
    \definition{adj.}{firme; sólido; robusto; forte; durável; firmemente unidos e inquebráveis}
  \end{phonetics}
\end{entry}

\begin{entry}{坚定}{7,8}{⼟、⼧}
  \begin{phonetics}{坚定}{jian1ding4}[][HSK 5]
    \definition{adj.}{firme; inabalável; inamovível; (posição, opinião, vontade, etc.) firme e estável, inabalável}
    \definition{v.}{fortalecer}
  \end{phonetics}
\end{entry}

\begin{entry}{坚持}{7,9}{⼟、⼿}
  \begin{phonetics}{坚持}{jian1chi2}[][HSK 3]
    \definition{v.}{persistir em; perseverar em; defender; insistir em; manter-se fiel a; aderir a; persistir com determinação e não desistir quando se depara com dificuldades | aderir a; insistir em; não alterar (os princípios, opiniões, pontos de vista originais, etc.)}
  \end{phonetics}
\end{entry}

\begin{entry}{坚强}{7,12}{⼟、⼸}
  \begin{phonetics}{坚强}{jian1qiang2}[][HSK 3]
    \definition{adj.}{forte; firme; convicto; (qualidades humanas, personalidade, determinação, etc.) firme e forte, não vacila diante das dificuldades}
    \definition{v.}{fortalecer; tornar forte; é a qualidade, a determinação, etc., que não vacilam}
  \end{phonetics}
\end{entry}

\begin{entry}{坠}{7}{⼟}
  \begin{phonetics}{坠}{zhui4}
    \definition{v.}{cair | pesar | fazer vergar com o peso}
  \end{phonetics}
\end{entry}

\begin{entry}{坠落}{7,12}{⼟、⾋}
  \begin{phonetics}{坠落}{zhui4luo4}
    \definition{v.}{cair}
  \end{phonetics}
\end{entry}

\begin{entry}{声}{7}{⼠}
  \begin{phonetics}{声}{sheng1}[][HSK 5]
    \definition{clas.}{indica o número de vezes que um som é emitido}
    \definition{s.}{som; voz | reputação | consoante inicial (de uma sílaba chinesa) | tom; tom de voz | informação; notícia}
    \definition{v.}{declarar; anunciar; emitir um som}
  \end{phonetics}
\end{entry}

\begin{entry}{声明}{7,8}{⼠、⽇}
  \begin{phonetics}{声明}{sheng1ming2}[][HSK 3]
    \definition[项,份]{s.}{declaração}
    \definition{v.}{declarar; anunciar; expressar publicamente a sua atitude ou dizer a verdade}
  \end{phonetics}
\end{entry}

\begin{entry}{声音}{7,9}{⼠、⾳}
  \begin{phonetics}{声音}{sheng1yin1}[][HSK 2]
    \definition[个,种]{s.}{som; voz; a percepção auditiva das ondas sonoras}
  \end{phonetics}
\end{entry}

\begin{entry}{壳}{7}{⼠}
  \begin{phonetics}{壳}{ke2}
    \definition{s.}{casca (de ovo, noz, caranguejo, etc.) | caixa | invólucro | alojamento (de uma máquina ou dispositivo)}
  \end{phonetics}
\end{entry}

\begin{entry}{妖}{7}{⼥}
  \begin{phonetics}{妖}{yao1}
    \definition{adj.}{enfeitiçante | encantador}
    \definition{s.}{\emph{goblin} | bruxa | diabo | monstro | fantasma | demônio}
  \end{phonetics}
\end{entry}

\begin{entry}{妙招}{7,8}{⼥、⼿}
  \begin{phonetics}{妙招}{miao4zhao1}
    \definition{adj.}{escorregadio}
    \definition{s.}{movimento inteligente | maneira inteligente de fazer algo}
  \end{phonetics}
\end{entry}

\begin{entry}{宋}{7}{⼧}
  \begin{phonetics}{宋}{song4}
    \definition*{s.}{sobrenome Song}
    \definition{s.}{Dinastia Song (960-1279) | Song das dinastias do sul (420-479)}
  \end{phonetics}
\end{entry}

\begin{entry}{完}{7}{⼧}
  \begin{phonetics}{完}{wan2}[][HSK 2]
    \definition*{s.}{sobrenome Wan}
    \definition{adj.}{inteiro; intacto; completo}
    \definition{v.}{acabar; terminar; completar | pagar | estar terminado; estar pronto para | esgotar; ser usado}
  \end{phonetics}
\end{entry}

\begin{entry}{完了}{7,2}{⼧、⼅}
  \begin{phonetics}{完了}{wan2 le5}[][HSK 5]
    \definition{v.}{acabar; terminar; concluir; chegar ao fim}
  \end{phonetics}
\end{entry}

\begin{entry}{完人}{7,2}{⼧、⼈}
  \begin{phonetics}{完人}{wan2ren2}
    \definition{s.}{pessoa perfeita}
  \end{phonetics}
\end{entry}

\begin{entry}{完全}{7,6}{⼧、⼊}
  \begin{phonetics}{完全}{wan2quan2}[][HSK 2]
    \definition{adj.}{inteiro; completo; não falta nada, está tudo completo}
    \definition{adv.}{completamente; representa tudo}
  \end{phonetics}
\end{entry}

\begin{entry}{完成}{7,6}{⼧、⼽}
  \begin{phonetics}{完成}{wan2cheng2}[][HSK 2]
    \definition{v.}{realizar; completar; terminar; cumprir; levar ao sucesso}
  \end{phonetics}
\end{entry}

\begin{entry}{完毕}{7,6}{⼧、⽐}
  \begin{phonetics}{完毕}{wan2bi4}
    \definition{v.}{completar | terminar | acabar}
  \end{phonetics}
\end{entry}

\begin{entry}{完完全全}{7,7,6,6}{⼧、⼧、⼊、⼊}
  \begin{phonetics}{完完全全}{wan2wan2quan2quan2}
    \definition{adv.}{completamente}
  \end{phonetics}
\end{entry}

\begin{entry}{完备}{7,8}{⼧、⼡}
  \begin{phonetics}{完备}{wan2bei4}
    \definition{adj.}{completo | impecável | perfeito}
    \definition{v.}{não deixar nada a desejar}
  \end{phonetics}
\end{entry}

\begin{entry}{完美}{7,9}{⼧、⽺}
  \begin{phonetics}{完美}{wan2mei3}[][HSK 3]
    \definition{adj.}{perfeito; impecável; consumado}
  \end{phonetics}
\end{entry}

\begin{entry}{完善}{7,12}{⼧、⼝}
  \begin{phonetics}{完善}{wan2shan4}[][HSK 3]
    \definition{adj.}{perfeito; consumado}
    \definition{v.}{refinar; melhorar; tornar perfeito}
  \end{phonetics}
\end{entry}

\begin{entry}{完税}{7,12}{⼧、⽲}
  \begin{phonetics}{完税}{wan2shui4}
    \definition{v.}{pagar imposto}
  \end{phonetics}
\end{entry}

\begin{entry}{完满}{7,13}{⼧、⽔}
  \begin{phonetics}{完满}{wan2man3}
    \definition{adj.}{satisfatório | bem-sucedido}
  \end{phonetics}
\end{entry}

\begin{entry}{完整}{7,16}{⼧、⽁}
  \begin{phonetics}{完整}{wan2zheng3}[][HSK 3]
    \definition{adj.}{intacto; inteiro; completo; integrado; nenhum dano ou mutilação}
  \end{phonetics}
\end{entry}

\begin{entry}{寿司}{7,5}{⼨、⼝}
  \begin{phonetics}{寿司}{shou4 si1}[][HSK 5]
    \definition[份]{s.}{\emph{sushi}; iguaria tradicional japonesa}
  \end{phonetics}
\end{entry}

\begin{entry}{尾巴}{7,4}{⼫、⼰}
  \begin{phonetics}{尾巴}{wei3ba5}[][HSK 4]
    \definition{s.}{cauda; projeções na extremidade do corpo de certos animais | parte semelhante a uma cauda; refere-se, em geral, ao final de algo | apêndice; anexo; adepto servil; pessoa que segue ou concorda com outra pessoa | (figura de linguagem) alguém que faz sombra a outro | fim; remanescente; parte restante (ou inacabada)}
  \end{phonetics}
\end{entry}

\begin{entry}{尿}{7}{⼫}
  \begin{phonetics}{尿}{niao4}
    \definition[泡]{s.}{urina}
    \definition{v.}{urinar}
  \end{phonetics}
  \begin{phonetics}{尿}{sui1}
    \definition{s.}{(coloquial) urina}
  \end{phonetics}
\end{entry}

\begin{entry}{局}{7}{⼫}
  \begin{phonetics}{局}{ju2}[][HSK 4]
    \definition{s.}{tabuleiro de xadrez | jogo; turno; \emph{set} | situação; estado das coisas | tolerância; grandeza ou pequenez da mente; grau de tolerância de uma pessoa em relação às outras | reunião de pessoas em festas | ardil; artidício; estratagema; armadilha | parte; porção; parcela | nome de determinadas lojas}
  \end{phonetics}
\end{entry}

\begin{entry}{局长}{7,4}{⼫、⾧}
  \begin{phonetics}{局长}{ju2 zhang3}[][HSK 5]
    \definition[位,个]{s.}{comissário; diretor; principais chefes de gabinete do governo}
  \end{phonetics}
\end{entry}

\begin{entry}{局面}{7,9}{⼫、⾯}
  \begin{phonetics}{局面}{ju2mian4}[][HSK 5]
    \definition[种]{s.}{aspecto; fase; situação; o estado das coisas em um período de tempo, em sua maior parte abstraído | escopo; escala}
  \end{phonetics}
\end{entry}

\begin{entry}{屁股}{7,8}{⼫、⾁}
  \begin{phonetics}{屁股}{pi4gu5}
    \definition{s.}{nádega | quadris}
  \end{phonetics}
\end{entry}

\begin{entry}{屁话}{7,8}{⼫、⾔}
  \begin{phonetics}{屁话}{pi4hua4}
    \definition{s.}{absurdo | tolice | besteira}
  \end{phonetics}
\end{entry}

\begin{entry}{层}{7}{⼫}
  \begin{phonetics}{层}{ceng2}[][HSK 2]
    \definition{clas.}{usado para coisas que se sobrepõem e se acumulam, como andares, camadas e estratos | usado para coisas que podem ser divididas em itens e etapas | usado para coisas que podem ser removidas ou apagadas da superfície de um objeto}
    \definition{s.}{camada; nível; coisas que se sobrepõem | nível; classificação; camada}
    \definition{v.}{sobrepor; empilhar camada sobre camada}
  \end{phonetics}
\end{entry}

\begin{entry}{层次}{7,6}{⼫、⽋}
  \begin{phonetics}{层次}{ceng2ci4}[][HSK 5]
    \definition{s.}{disposição ordenada do conteúdo (de um discurso ou texto) | nível ou estrutura administrativa; distinções entre a mesma coisa devido a diferenças de tamanho, altura, etc. | nível; níveis de afiliação}
  \end{phonetics}
\end{entry}

\begin{entry}{层层}{7,7}{⼫、⼫}
  \begin{phonetics}{层层}{ceng2ceng2}
    \definition{s.}{camada sobre camada}
  \end{phonetics}
\end{entry}

\begin{entry}{希望}{7,11}{⼱、⽉}
  \begin{phonetics}{希望}{xi1wang4}[][HSK 3]
    \definition[个,丝,点]{s.}{esperança; desejo; expectativa; a possibilidade de alcançar um determinado objetivo ou de ocorrer uma determinada situação ideal no futuro | aquilo em que a esperança é depositada; o objeto da esperança}
    \definition{v.}{ter esperança; desejar; esperar; pensar em alcançar algum objetivo ou que alguma situação ocorra}
  \end{phonetics}
\end{entry}

\begin{entry}{床}{7}{⼴}
  \begin{phonetics}{床}{chuang2}[][HSK 1]
    \definition{clas.}{usado para colchas, roupas de cama, etc.}
    \definition[张]{s.}{cama; sofá; móveis para dormir | algo com o formato de uma cama}
  \end{phonetics}
\end{entry}

\begin{entry}{库}{7}{⼴}
  \begin{phonetics}{库}{ku4}[][HSK 5]
    \definition{s.}{depósito; tesouraria; armazém; almoxarifado; edifícios e equipamentos para armazenamento de mercadorias | banco de dados}
  \end{phonetics}
\end{entry}

\begin{entry}{应}{7}{⼴}
  \begin{phonetics}{应}{ying1}[][HSK 4,5]
    \definition{v.}{ecoar; responder; responder a; responder às chamadas, saudações, perguntas, etc. de outras pessoas | conceder; cumprir | adequar; adaptar; responder a | lidar com; enfrentar; abordar | tornar-se realidade; ser cumprido}
  \end{phonetics}
\end{entry}

\begin{entry}{应对}{7,5}{⼴、⼨}
  \begin{phonetics}{应对}{ying4dui4}
    \definition{v.}{responder | manusear | lidar}
  \end{phonetics}
\end{entry}

\begin{entry}{应用}{7,5}{⼴、⽤}
  \begin{phonetics}{应用}{ying4yong4}[][HSK 3]
    \definition{adj.}{aplicado (na vida ou na produção); usado diretamente na vida ou na produção}
    \definition{v.}{usar; aplicar}
  \end{phonetics}
\end{entry}

\begin{entry}{应用程序}{7,5,12,7}{⼴、⽤、⽲、⼴}
  \begin{phonetics}{应用程序}{ying4yong4 cheng2xu4}
    \definition{s.}{programa aplicativo; principais categorias de \emph{software}}
  \end{phonetics}
\end{entry}

\begin{entry}{应用程序接口}{7,5,12,7,11,3}{⼴、⽤、⽲、⼴、⼿、⼝}
  \begin{phonetics}{应用程序接口}{ying4yong4cheng2xu4jie1kou3}
    \definition{s.}{API (\emph{application programming interface})}
  \seealsoref{应用程序编程接口}{ying4yong4cheng2xu4bian1cheng2jie1kou3}
  \end{phonetics}
\end{entry}

\begin{entry*}{应用程序编程接口}{7,5,12,7,12,12,11,3}{⼴、⽤、⽲、⼴、⽷、⽲、⼿、⼝}
  \begin{phonetics}{应用程序编程接口}{ying4yong4cheng2xu4bian1cheng2jie1kou3}
    \definition{s.}{API (\emph{application programming interface})}
  \seealsoref{应用程序接口}{ying4yong4cheng2xu4jie1kou3}
  \end{phonetics}
\end{entry*}

\begin{entry}{应当}{7,6}{⼴、⼹}
  \begin{phonetics}{应当}{ying1 dang1}[][HSK 3]
    \definition{v.}{dever}[学生们应当努力学习。___Os alunos devem se esforçar nos estudos.]
  \end{phonetics}
\end{entry}

\begin{entry}{应该}{7,8}{⼴、⾔}
  \begin{phonetics}{应该}{ying1gai1}[][HSK 2]
    \definition{v.}{deveria; deve ser assim | deveria; acho que deve ser esse o caso}
  \end{phonetics}
\end{entry}

\begin{entry}{弄}{7}{⼶}
  \begin{phonetics}{弄}{long4}
    \definition{s.}{rua estreita; beco; viela; travessa}
  \end{phonetics}
  \begin{phonetics}{弄}{nong4}[][HSK 2]
    \definition{v.}{fazer, realizar; tratar; organizar | obter; buscar; tentar conseguir; encontrar uma maneira de conseguir | brincar com; enganar | pregar uma peça; brincar; manipular | mexer com; perturbar}
  \end{phonetics}
\end{entry}

\begin{entry}{弟}{7}{⼸}
  \begin{phonetics}{弟}{di4}[][HSK 1]
    \definition*{s.}{sobrenome Di}
    \definition[个]{s.}{irmão mais novo | (entre amigos homens) eu | geralmente se refere a colegas do sexo masculino mais jovens na família ou entre parentes | forma humilde que os amigos usam para se referir uns aos outros, usada principalmente em correspondência}
  \end{phonetics}
\end{entry}

\begin{entry}{弟弟}{7,7}{⼸、⼸}
  \begin{phonetics}{弟弟}{di4 di5}[][HSK 1]
    \definition[个,位]{s.}{irmão mais novo | primo}
  \end{phonetics}
\end{entry}

\begin{entry}{弟妹}{7,8}{⼸、⼥}
  \begin{phonetics}{弟妹}{di4mei4}
    \definition{s.}{esposa do irmão mais novo}
  \end{phonetics}
\end{entry}

\begin{entry}{张}{7}{⼸}
  \begin{phonetics}{张}{zhang1}[][HSK 3]
    \definition*{s.}{sobrenome Zhang}
    \definition*{s.}{Zhang, uma das mansões lunares; uma das 28 constelações}
    \definition{adj.}{indulgente; desenfreado; devasso; libertino}
    \definition{clas.}{usado para papel, couro, etc. | usado para cama, mesa, etc. | usado para o rosto, boca, etc. | usado para arco}
    \definition{v.}{abrir; espalhar; esticar | expor; exibir | (de uma loja) iniciar atividades comerciais; abrir | olhar; contemplar | expandir; estender; ampliar; exagerar | fixar (uma corda de arco); encordoar (um instrumento musical); puxar a corda do arco}
  \end{phonetics}
\end{entry}

\begin{entry}{张三}{7,3}{⼸、⼀}
  \begin{phonetics}{张三}{zhang1san1}
    \definition*{s.}{Zhang San | Zé Ninguém | nome para uma pessoa não especificada, 1 de 3}
  \seealsoref{李四}{li3si4}
  \seealsoref{王五}{wang2wu3}
  \end{phonetics}
\end{entry}

\begin{entry}{张狂}{7,7}{⼸、⽝}
  \begin{phonetics}{张狂}{zhang1kuang2}
    \definition{adj.}{impetuoso | frenético | insolente}
  \end{phonetics}
\end{entry}

\begin{entry}{形式}{7,6}{⼺、⼷}
  \begin{phonetics}{形式}{xing2shi4}[][HSK 3]
    \definition[种,个]{s.}{forma; formato; modalidade; a aparência, estrutura ou estado das coisas, etc.}
  \end{phonetics}
\end{entry}

\begin{entry}{形成}{7,6}{⼺、⼽}
  \begin{phonetics}{形成}{xing2cheng2}[][HSK 3]
    \definition{v.}{moldar; formar; tomar forma; tornar-se algo ou surgir uma situação após mudanças e desenvolvimentos}
  \end{phonetics}
\end{entry}

\begin{entry}{形而上学}{7,6,3,8}{⼺、⽽、⼀、⼦}
  \begin{phonetics}{形而上学}{xing2'er2shang4xue2}
    \definition{s.}{metafísica}
  \end{phonetics}
\end{entry}

\begin{entry}{形状}{7,7}{⼺、⽝}
  \begin{phonetics}{形状}{xing2zhuang4}[][HSK 3]
    \definition[个,种]{s.}{forma; aparência ; aspecto; a aparência de um objeto ou figura, representada pela combinação de superfícies ou linhas externas}
  \end{phonetics}
\end{entry}

\begin{entry}{形势}{7,8}{⼺、⼒}
  \begin{phonetics}{形势}{xing2shi4}[][HSK 4]
    \definition[个]{s.}{terreno; características topográficas; situação geográfica, principalmente de uma perspectiva militar | situação; circunstâncias; a situação geral, a tendência de como as coisas estão se desenvolvendo e mudando | geralmente não é usado em situações pessoais}
  \end{phonetics}
\end{entry}

\begin{entry}{形态}{7,8}{⼺、⼼}
  \begin{phonetics}{形态}{xing2tai4}[][HSK 5]
    \definition{s.}{forma; forma como as coisas se apresentam | forma; padrão; postura | morfologia; forma; (gramática) refere-se às formas internas de mudança das palavras, incluindo a formação de palavras e as mudanças morfológicas}
  \end{phonetics}
\end{entry}

\begin{entry}{形容}{7,10}{⼺、⼧}
  \begin{phonetics}{形容}{xing2rong2}[][HSK 4]
    \definition{s.}{aparência; semblante}
    \definition{v.}{descrever}
  \end{phonetics}
\end{entry}

\begin{entry}{形象}{7,11}{⼺、⾗}
  \begin{phonetics}{形象}{xing2xiang4}[][HSK 3]
    \definition{adj.}{vívido; expressão concreta e vívida}
    \definition[个,种]{s.}{imagem; forma; figura; formas ou posturas específicas que podem despertar pensamentos ou emoções nas pessoas | imagem literária; imagem artística; pessoas ou coisas com características diferentes criadas na literatura, no cinema e em outras artes}
  \end{phonetics}
\end{entry}

\begin{entry}{彻}{7}{⼻}
  \begin{phonetics}{彻}{che4}
    \definition{adj.}{minucioso; completo; penetrante}
    \definition{adv.}{minuciosamente; profundamente}
  \end{phonetics}
\end{entry}

\begin{entry}{彻底}{7,8}{⼻、⼴}
  \begin{phonetics}{彻底}{che4di3}[][HSK 4]
    \definition{adj.}{minucioso; completo; exaustivo; profundo e completo; nada é deixado de fora}
  \end{phonetics}
\end{entry}

\begin{entry}{忍}{7}{⼼}
  \begin{phonetics}{忍}{ren3}[][HSK 5]
    \definition{v.}{suportar; aguentar; tolerar; aturar | ter coragem para; ser insensível o suficiente para; ser capaz de endurecer o coração e fazer coisas que não se devem fazer por uma questão de razão}
  \end{phonetics}
\end{entry}

\begin{entry}{忍不住}{7,4,7}{⼼、⼀、⼈}
  \begin{phonetics}{忍不住}{ren3bu5zhu4}[][HSK 5]
    \definition{v.}{incapaz de suportar; não conseguir evitar fazer algo; não conseguir se controlar}
  \end{phonetics}
\end{entry}

\begin{entry}{忍受}{7,8}{⼼、⼜}
  \begin{phonetics}{忍受}{ren3shou4}[][HSK 5]
    \definition{v.}{suportar; sofrer; aguentar; tolerar; suportar com dificuldade o sofrimento, as dificuldades e as adversidades da vida}
  \end{phonetics}
\end{entry}

\begin{entry}{忍耐}{7,9}{⼼、⽽}
  \begin{phonetics}{忍耐}{ren3nai4}
    \definition{s.}{paciência | resistência}
    \definition{v.}{suportar | resistir | exercer paciência}
  \end{phonetics}
\end{entry}

\begin{entry}{志愿}{7,14}{⼼、⽕}
  \begin{phonetics}{志愿}{zhi4 yuan4}[][HSK 3]
    \definition{s.}{desejo; ideal; aspiração; os ideais, desejos ou objetivos que se deseja realizar no coração}
    \definition{v.}{ser voluntário; ser proativo e disposto a realizar trabalhos sem remuneração ou com remuneração baixa, mas que possam ajudar outras pessoas}
  \end{phonetics}
\end{entry}

\begin{entry}{志愿书}{7,14,4}{⼼、⽕、⼄}
  \begin{phonetics}{志愿书}{zhi4yuan4shu1}
    \definition{s.}{formulário de inscrição; formulário de adesão; carta de intenções}
  \end{phonetics}
\end{entry}

\begin{entry}{志愿者}{7,14,8}{⼼、⽕、⽼}
  \begin{phonetics}{志愿者}{zhi4yuan4zhe3}[][HSK 3]
    \definition[名,位,个]{s.}{voluntário; pessoas que se voluntariam para prestar serviços em atividades sociais, grandes eventos esportivos, conferências, etc.}
  \end{phonetics}
\end{entry}

\begin{entry}{忘}{7}{⼼}
  \begin{phonetics}{忘}{wang4}[][HSK 1]
    \definition{v.}{esquecer | ignorar; negligenciar}
  \end{phonetics}
\end{entry}

\begin{entry}{忘本}{7,5}{⼼、⽊}
  \begin{phonetics}{忘本}{wang4ben3}
    \definition{v.}{esquecer as próprias raízes}
  \end{phonetics}
\end{entry}

\begin{entry}{忘记}{7,5}{⼼、⾔}
  \begin{phonetics}{忘记}{wang4ji4}[][HSK 1]
    \definition{v.}{esquecer | ignorar; negligenciar | sair da memória de alguém; não ser lembrado | descartar da mente; ignorar}
  \end{phonetics}
\end{entry}

\begin{entry}{忘却}{7,7}{⼼、⼙}
  \begin{phonetics}{忘却}{wang4que4}
    \definition{v.}{esquecer}
  \end{phonetics}
\end{entry}

\begin{entry}{忘怀}{7,7}{⼼、⼼}
  \begin{phonetics}{忘怀}{wang4huai2}
    \definition{v.}{esquecer}
  \end{phonetics}
\end{entry}

\begin{entry}{忘恩}{7,10}{⼼、⼼}
  \begin{phonetics}{忘恩}{wang4'en1}
    \definition{v.}{ser ingrato}
  \end{phonetics}
\end{entry}

\begin{entry}{忘掉}{7,11}{⼼、⼿}
  \begin{phonetics}{忘掉}{wang4diao4}
    \definition{v.}{esquecer}
  \end{phonetics}
\end{entry}

\begin{entry}{忘餐}{7,16}{⼼、⾷}
  \begin{phonetics}{忘餐}{wang4can1}
    \definition{v.}{esquecer as refeições}
  \end{phonetics}
\end{entry}

\begin{entry}{忧郁}{7,8}{⼼、⾢}
  \begin{phonetics}{忧郁}{you1yu4}
    \definition{adj.}{deprimido | melancólico | desanimado}
    \definition{s.}{depressão | melancolia}
  \end{phonetics}
\end{entry}

\begin{entry}{快}{7}{⼼}
  \begin{phonetics}{快}{kuai4}[][HSK 1]
    \definition*{s.}{sobrenome Kuai}
    \definition{adj.}{rápido; veloz (oposto a 慢) | apressado | perspicaz; ágil; inteligente; de ​​mente rápida | (de uma faca, espada, etc.) afiado (oposto a 钝) | direto; franco; sem rodeios | satisfeito; feliz; gratificado | rápido; veloz; alta velocidade; tempo de execução curto | satisfeito; feliz; contente | engenhoso; ágil | afiado; facas, tesouras, machados e outros objetos afiados | sincero}
    \definition{adv.}{em breve; antes de muito tempo; estar prestes a | rapidamente}
    \definition{s.}{policial; polícia | (antigo) oficial encarregado de efetuar prisões}
  \seealsoref{钝}{dun4}
  \seealsoref{慢}{man4}
  \end{phonetics}
\end{entry}

\begin{entry}{快乐}{7,5}{⼼、⼃}
  \begin{phonetics}{快乐}{kuai4le4}[][HSK 2]
    \definition{adj.}{feliz; alegre; animado; prazeiroso}
    \definition{s.}{felicidade | alegria}
  \end{phonetics}
\end{entry}

\begin{entry}{快活}{7,9}{⼼、⽔}
  \begin{phonetics}{快活}{kuai4huo5}[][HSK 5]
    \definition{adj.}{feliz; alegre; contente; animado}
  \end{phonetics}
\end{entry}

\begin{entry}{快点儿}{7,9,2}{⼼、⽕、⼉}
  \begin{phonetics}{快点儿}{kuai4 dian3r5}[][HSK 2]
    \definition{v.}{apressar-se}
  \end{phonetics}
\end{entry}

\begin{entry}{快要}{7,9}{⼼、⾑}
  \begin{phonetics}{快要}{kuai4 yao4}[][HSK 2]
    \definition{adv.}{estar prestes a; estar indo para; estar à beira de; em breve; em pouco tempo; indica que a situação está prestes a ocorrer}
  \end{phonetics}
\end{entry}

\begin{entry}{快递}{7,10}{⼼、⾡}
  \begin{phonetics}{快递}{kuai4 di4}[][HSK 4]
    \definition[个]{s.}{correio rápido; entrega expressa; entrega rápida}
    \definition{v.}{entregar (serviço de entrega rápida por transportadoras especializadas)}
  \end{phonetics}
\end{entry}

\begin{entry}{快速}{7,10}{⼼、⾡}
  \begin{phonetics}{快速}{kuai4 su4}[][HSK 3]
    \definition{adj.}{rápido; veloz; de alta velocidade; descreve o tempo curto gasto para caminhar, fazer algo, etc.}
  \end{phonetics}
\end{entry}

\begin{entry}{快餐}{7,16}{⼼、⾷}
  \begin{phonetics}{快餐}{kuai4 can1}[][HSK 2]
    \definition[份,顿]{s.}{pedido (comida) rápido; \emph{fast food}; refere-se a refeições simples preparadas com antecedência e que podem ser servidas rapidamente}
  \end{phonetics}
\end{entry}

\begin{entry}{怀旧}{7,5}{⼼、⽇}
  \begin{phonetics}{怀旧}{huai2jiu4}
    \definition{s.}{nostalgia}
    \definition{v.}{sentir-se nostálgico}
  \end{phonetics}
\end{entry}

\begin{entry}{怀念}{7,8}{⼼、⼼}
  \begin{phonetics}{怀念}{huai2nian4}[][HSK 4]
    \definition{v.}{pensar em; valorizar a memória de}
  \end{phonetics}
\end{entry}

\begin{entry}{怀疑}{7,14}{⼼、⽦}
  \begin{phonetics}{怀疑}{huai2yi2}[][HSK 4]
    \definition{v.}{duvidar; suspeitar | supor}
  \end{phonetics}
\end{entry}

\begin{entry}{我}{7}{⼽}
  \begin{phonetics}{我}{wo3}[][HSK 1]
    \definition{pron.}{eu; mim | um; qualquer um; usado para contrastar 他 e 我; refere-se a muitas pessoas em geral}
  \seealsoref{他}{ta1}
  \end{phonetics}
\end{entry}

\begin{entry}{我们}{7,5}{⼽、⼈}
  \begin{phonetics}{我们}{wo3men5}[][HSK 1]
    \definition{pron.}{nós; nos}
  \end{phonetics}
\end{entry}

\begin{entry}{我们的}{7,5,8}{⼽、⼈、⽩}
  \begin{phonetics}{我们的}{wo3men5 de5}
    \definition{pron.}{nosso, nossos}
  \end{phonetics}
\end{entry}

\begin{entry}{我去}{7,5}{⼽、⼛}
  \begin{phonetics}{我去}{wo3qu4}
    \definition{interj.}{(gíria) O que\dots!! | Oh meu Deus! | Isso é insano!}
  \end{phonetics}
\end{entry}

\begin{entry}{我的}{7,8}{⼽、⽩}
  \begin{phonetics}{我的}{wo3 de5}
    \definition{pron.}{meu, meus}
  \end{phonetics}
\end{entry}

\begin{entry}{戒}{7}{⼽}
  \begin{phonetics}{戒}{jie4}[][HSK 5]
    \definition[个]{s.}{advertência; exortação | disciplina monástica budista; preceitos budistas | anel (dedo)}
    \definition{v.}{proteger-se contra; estar preparado; estar atento | advertir; exortar; admoestar | abandonar; parar; desistir; desistir (de um hábito ruim)}
  \end{phonetics}
\end{entry}

\begin{entry}{扮}{7}{⼿}
  \begin{phonetics}{扮}{ban4}
    \definition{v.}{vestir-se como; desempenhar o papel de | maquiar-se; disfarçar-se como | (expressão facial) fazer cara de}
  \end{phonetics}
\end{entry}

\begin{entry}{扮演}{7,14}{⼿、⽔}
  \begin{phonetics}{扮演}{ban4yan3}[][HSK 5]
    \definition{v.}{desempenhar o papel de; ter um papel (em uma peça, etc.); atuar}
  \end{phonetics}
\end{entry}

\begin{entry}{扶}{7}{⼿}
  \begin{phonetics}{扶}{fu2}[][HSK 5]
    \definition*{s.}{sobrenome Fu}
    \definition{v.}{segurar; apoiar com a mão; segurar algo com o apoio das mãos para que ninguém, objeto ou pessoa caia | dar apoio a; ajudar uma pessoa deitada ou caída a se levantar com as mãos; endireitar um objeto caído com as mãos | ajudar; tirar de baixo}
  \end{phonetics}
\end{entry}

\begin{entry}{扶梯}{7,11}{⼿、⽊}
  \begin{phonetics}{扶梯}{fu2ti1}
    \definition{s.}{escada rolante}
  \end{phonetics}
\end{entry}

\begin{entry}{批}{7}{⼿}
  \begin{phonetics}{批}{pi1}[][HSK 4]
    \definition{adj.}{(compra ou venda) atacado; a granel; em grandes quantidades}
    \definition{clas.}{para mercadorias a granel, grande número de pessoas}
    \definition{s.}{fibras de algodão, linho, etc., prontas para serem estiradas e torcidas | anotação; comentário}
    \definition{v.}{escrever comentários ou críticas sobre documentos subordinados, textos de outras pessoas, tarefas etc. | refutar; criticar | dar um tapa}
  \end{phonetics}
\end{entry}

\begin{entry}{批评}{7,7}{⼿、⾔}
  \begin{phonetics}{批评}{pi1ping2}[][HSK 3]
    \definition{v.}{criticar; comentar sobre deficiências e erros | criticar; apontar vantagens e desvantagens; comentar sobre o que é bom e o que é ruim}
  \end{phonetics}
\end{entry}

\begin{entry}{批准}{7,10}{⼿、⼎}
  \begin{phonetics}{批准}{pi1zhun3}[][HSK 3]
    \definition{v.}{aprovar}
  \end{phonetics}
\end{entry}

\begin{entry}{找}{7}{⼿}
  \begin{phonetics}{找}{zhao3}[][HSK 1]
    \definition{v.}{procurar; tentar encontrar; buscar | querer ver; visitar; abordar; solicitar | dar troco | descobrir; esforçar-se para ver ou obter a pessoa ou coisa desejada | examinar; investigar; completar as partes que faltam | causar intencionalmente (um resultado indesejável, negativo)}
  \end{phonetics}
\end{entry}

\begin{entry}{找见}{7,4}{⼿、⾒}
  \begin{phonetics}{找见}{zhao3jian4}
    \definition{v.}{encontrar (algo que está procurando)}
  \end{phonetics}
\end{entry}

\begin{entry}{找出}{7,5}{⼿、⼐}
  \begin{phonetics}{找出}{zhao3 chu1}[][HSK 2]
    \definition{v.}{encontrar | procurar}
  \end{phonetics}
\end{entry}

\begin{entry}{找回}{7,6}{⼿、⼞}
  \begin{phonetics}{找回}{zhao3hui2}
    \definition{v.}{recuperar algo}
  \end{phonetics}
\end{entry}

\begin{entry}{找寻}{7,6}{⼿、⼨}
  \begin{phonetics}{找寻}{zhao3xun2}
    \definition{v.}{encontrar falhas | procurar | buscar}
  \end{phonetics}
\end{entry}

\begin{entry}{找事}{7,8}{⼿、⼅}
  \begin{phonetics}{找事}{zhao3shi4}
    \definition{v.}{procurar emprego | começar uma briga}
  \end{phonetics}
\end{entry}

\begin{entry}{找到}{7,8}{⼿、⼑}
  \begin{phonetics}{找到}{zhao3 dao4}[][HSK 1]
    \definition{v.}{encontrar; procurar; achar; encontar através de pesquisa, exploração, etc.;  ver ou encontrar coisas ou padrões que os antepassados não viram}
  \end{phonetics}
\end{entry}

\begin{entry}{找钱}{7,10}{⼿、⾦}
  \begin{phonetics}{找钱}{zhao3qian2}
    \definition{v.}{dar troco}
  \end{phonetics}
\end{entry}

\begin{entry}{找着}{7,11}{⼿、⽬}
  \begin{phonetics}{找着}{zhao3zhao2}
    \definition{v.}{encontrar}
  \end{phonetics}
\end{entry}

\begin{entry}{找遍}{7,12}{⼿、⾡}
  \begin{phonetics}{找遍}{zhao3bian4}
    \definition{v.}{pentear | pesquisar em todos os lugares}
  \end{phonetics}
\end{entry}

\begin{entry}{找零}{7,13}{⼿、⾬}
  \begin{phonetics}{找零}{zhao3ling2}
    \definition{v.}{trocar dinheiro | dar troco}
  \end{phonetics}
\end{entry}

\begin{entry}{找辙}{7,16}{⼿、⾞}
  \begin{phonetics}{找辙}{zhao3zhe2}
    \definition{v.}{procurar um pretexto}
  \end{phonetics}
\end{entry}

\begin{entry}{技巧}{7,5}{⼿、⼯}
  \begin{phonetics}{技巧}{ji4qiao3}[][HSK 4]
    \definition{s.}{habilidade; técnica; habilidades engenhosas expressas em artes, artesanato, esportes, etc.}
  \end{phonetics}
\end{entry}

\begin{entry}{技术}{7,5}{⼿、⽊}
  \begin{phonetics}{技术}{ji4shu4}[][HSK 3]
    \definition[种,门,项]{s.}{habilidade; técnica; tecnologia; a experiência e o conhecimento acumulados pelo ser humano no processo de utilização e transformação da natureza, e refletidos no trabalho produtivo, também se referem, de maneira geral, a outras habilidades operacionais}
  \end{phonetics}
\end{entry}

\begin{entry}{技俩}{7,9}{⼿、⼈}
  \begin{phonetics}{技俩}{ji4liang3}
    \definition{s.}{truque | estratagema | ardil | esquema | estratégia | tática}
  \end{phonetics}
\end{entry}

\begin{entry}{技能}{7,10}{⼿、⾁}
  \begin{phonetics}{技能}{ji4 neng2}[][HSK 5]
    \definition[种,项]{s.}{habilidade técnica; domínio de uma habilidade ou técnica; capacidade de adquirir e aplicar conhecimento}
  \end{phonetics}
\end{entry}

\begin{entry}{抄}{7}{⼿}
  \begin{phonetics}{抄}{chao1}[][HSK 4]
    \definition*{s.}{sobrenome Chao}
    \definition{v.}{copiar; transcrever | plagiar | registrar as leituras de um medidor | revistar e confiscar; fazer uma incursão em  | pegar um atalho | dobrar (os braços) | agarrar; pegar | ir (andar) embora com}
  \end{phonetics}
\end{entry}

\begin{entry}{抄写}{7,5}{⼿、⼍}
  \begin{phonetics}{抄写}{chao1 xie3}[][HSK 4]
    \definition{v.}{copiar; transcrever}
  \end{phonetics}
\end{entry}

\begin{entry}{抄表}{7,8}{⼿、⾐}
  \begin{phonetics}{抄表}{chao1 biao3}
    \definition{s.}{leitura do medidor}
  \end{phonetics}
\end{entry}

\begin{entry}{把}{7}{⼿}
  \begin{phonetics}{把}{ba3}[][HSK 3]
    \definition{adj.}{referindo-se à relação de irmandade}
    \definition{clas.}{usado antes de objetos com alças ou coisas para segurar | um punhado de; a quantidade que se pode pegar com uma mão | usado antes de coisas abstratas | usado em coisas feitas com as mãos | número de ações, coisas}
    \definition{part.}{adicionado após quantificadores como 百, 千, 万 e 里, 斤, 个, indica que a quantidade é próxima dessa unidade (não pode ser adicionado outro quantificador antes)}
    \definition{prep.}{fazer uma determinada alteração em um objeto; causar uma determinada mudança em um objeto | fazer com que os outros façam/sintam algo}
    \definition{s.}{alça; punho; a parte que se segura | feixe; molho; algo que se segura com as mãos ou se amarra em pequenos feixes}
    \definition{v.}{agarrar; segurar | segurar (um bebê enquanto ele urina) | controlar; dominar; monopolizar | encostar-se; apoiar-se | vigiar (locais importantes); observar; guardar | dar | usar algo como; considerar como; tratar como; conter o significado de 拿 | acorrentar; trancar}
  \seealsoref{百}{bai3}
  \seealsoref{个}{ge4}
  \seealsoref{斤}{jin1}
  \seealsoref{里}{li3}
  \seealsoref{拿}{na2}
  \seealsoref{千}{qian1}
  \seealsoref{万}{wan4}
  \end{phonetics}
  \begin{phonetics}{把}{ba4}
    \definition{s.}{punho; alça; empunhadura; parte do utensílio que é fácil de segurar com a mão |haste (de uma folha, flor ou fruto) | motivo de ridículo; alvo; comportamentos e declarações que servem de assunto para piadas}
  \end{phonetics}
\end{entry}

\begin{entry}{把风}{7,4}{⼿、⾵}
  \begin{phonetics}{把风}{ba3feng1}
    \definition{v.}{estar atento | vigiar (durante uma atividade clandestina)}
  \end{phonetics}
\end{entry}

\begin{entry}{把关}{7,6}{⼿、⼋}
  \begin{phonetics}{把关}{ba3guan1}
    \definition{v.}{verificar estritamente | examinar cuidadosamente para ver se algo é feito de acordo com um padrão fixo | fazer a verificação final | guardar uma passagem, fronteira}
  \end{phonetics}
\end{entry}

\begin{entry}{把守}{7,6}{⼿、⼧}
  \begin{phonetics}{把守}{ba3shou3}
    \definition{v.}{vigiar | guardar}
  \end{phonetics}
\end{entry}

\begin{entry}{把式}{7,6}{⼿、⼷}
  \begin{phonetics}{把式}{ba3shi4}
    \definition{s.}{pessoa qualificada em um comércio}
  \end{phonetics}
\end{entry}

\begin{entry}{把戏}{7,6}{⼿、⼽}
  \begin{phonetics}{把戏}{ba3xi4}
    \definition{s.}{acrobacia | malabarismo | truque barato}
  \end{phonetics}
\end{entry}

\begin{entry}{把玩}{7,8}{⼿、⽟}
  \begin{phonetics}{把玩}{ba3wan2}
    \definition{v.}{brincar com | mexer com}
  \end{phonetics}
\end{entry}

\begin{entry}{把持}{7,9}{⼿、⼿}
  \begin{phonetics}{把持}{ba3chi2}
    \definition{v.}{controlar | dominar | monopolizar}
  \end{phonetics}
\end{entry}

\begin{entry}{把柄}{7,9}{⼿、⽊}
  \begin{phonetics}{把柄}{ba3bing3}
    \definition{s.}{(figurativo) informações que podem ser usadas contra alguém}
  \end{phonetics}
\end{entry}

\begin{entry}{把脉}{7,9}{⼿、⾁}
  \begin{phonetics}{把脉}{ba3mai4}
    \definition{v.}{sentir ou tomar o pulso de alguém}
  \end{phonetics}
\end{entry}

\begin{entry}{把握}{7,12}{⼿、⼿}
  \begin{phonetics}{把握}{ba3wo4}[][HSK 3]
    \definition[的]{s.}{seguro; garantia; certeza; confiabilidade do sucesso}
    \definition{v.}{agarrar; segurar; apreender |  (algo abstrato) agarrar; segurar}
  \end{phonetics}
\end{entry}

\begin{entry}{把稳}{7,14}{⼿、⽲}
  \begin{phonetics}{把稳}{ba3wen3}
    \definition{adj.}{confiável}
  \end{phonetics}
\end{entry}

\begin{entry}{抓}{7}{⼿}
  \begin{phonetics}{抓}{zhua1}[][HSK 3]
    \definition{v.}{agarrar; segurar; obter; apreender; juntar os dedos para segurar o objeto na mão | riscar; arranhar; usar as unhas, objetos com dentes ou garras de animais para riscar a superfície de um objeto | apanhar; capturar; controlar pessoas ou animais; fazer com que pessoas ou animais caiam nas mãos de alguém | compreender; saber onde está o ponto principal ou a chave de uma questão ou problema | concentrar-se em algo; reforçar a força para fazer (alguma coisa), controlar (algum aspecto) | chamar a atenção de alguém; atrair a atenção}
  \end{phonetics}
\end{entry}

\begin{entry}{抓住}{7,7}{⼿、⼈}
  \begin{phonetics}{抓住}{zhua1 zhu4}[][HSK 3]
    \definition{v.}{prender; deter; capturar (pessoas ou animais) e ter sucesso | segurar; agarrar; apreender; agarrar algo para que não se mova}
  \end{phonetics}
\end{entry}

\begin{entry}{抓紧}{7,10}{⼿、⽷}
  \begin{phonetics}{抓紧}{zhua1jin3}[][HSK 4]
    \definition{v.}{agarrar com firmeza; segurar firme e não soltar | prestar muita atenção a}
  \end{phonetics}
\end{entry}

\begin{entry}{投}{7}{⼿}
  \begin{phonetics}{投}{tou2}[][HSK 4]
    \definition*{s.}{sobrenome Tou}
    \definition{pron.}{para; indica tempo, equivalente a 到, 临 | para; em direção a; indica orientação, direção, equivalente a 朝 ou 向}
    \definition{s.}{um jogo durante uma festa em que o vencedor era decidido pelo número de flechas lançadas em um pote distante | jogo de dados}
    \definition{v.}{lançar; arremessar; atirar | deixar cair; colocar em; lançar | mergulhar em; lançar-se em; pular dentro | lançar; projetar; sombrear | entregar; postar; enviar | ir até; ir para; buscar; juntar-se | sentir-se atraído por; adaptar-se a; concordar com; atender a}
  \seealsoref{朝}{chao2}
  \seealsoref{到}{dao4}
  \seealsoref{临}{lin2}
  \seealsoref{向}{xiang4}
  \end{phonetics}
\end{entry}

\begin{entry}{投入}{7,2}{⼿、⼊}
  \begin{phonetics}{投入}{tou2ru4}[][HSK 4]
    \definition{adj.}{sisudo; dedicado; devotado; absorto}
    \definition{s.}{investimento; insumo; refere-se à aplicação de recursos}
    \definition{v.}{lançar em; colocar em; jogar em; por em | entrar em uma situação; participar de | aplicar; investir; colocar fundos em}
  \end{phonetics}
\end{entry}

\begin{entry}{投诉}{7,7}{⼿、⾔}
  \begin{phonetics}{投诉}{tou2su4}[][HSK 4]
    \definition{v.}{reclamar; queixar-se; reclamar às autoridades ou pessoas envolvidas}
  \end{phonetics}
\end{entry}

\begin{entry}{投资}{7,10}{⼿、⾙}
  \begin{phonetics}{投资}{tou2zi1}[][HSK 4]
    \definition[次]{s.}{investimento}
    \definition{v.}{investir; aplicar dinheiro; investir dinheiro em negócios}
  \end{phonetics}
\end{entry}

\begin{entry}{投资人}{7,10,2}{⼿、⾙、⼈}
  \begin{phonetics}{投资人}{tou2zi1ren2}
    \definition{s.}{investidor}
  \seealsoref{投资家}{tou2zi1jia1}
  \seealsoref{投资者}{tou2zi1zhe3}
  \end{phonetics}
\end{entry}

\begin{entry}{投资风险}{7,10,4,9}{⼿、⾙、⾵、⾩}
  \begin{phonetics}{投资风险}{tou2zi1feng1xian3}
    \definition{s.}{risco de investimento}
  \end{phonetics}
\end{entry}

\begin{entry}{投资回报率}{7,10,6,7,11}{⼿、⾙、⼞、⼿、⽞}
  \begin{phonetics}{投资回报率}{tou2zi1hui2bao4lv4}
    \definition{s.}{retorno sobre o investimento (ROI)}
  \end{phonetics}
\end{entry}

\begin{entry}{投资者}{7,10,8}{⼿、⾙、⽼}
  \begin{phonetics}{投资者}{tou2zi1zhe3}
    \definition{s.}{investidor}
  \seealsoref{投资家}{tou2zi1jia1}
  \seealsoref{投资人}{tou2zi1ren2}
  \end{phonetics}
\end{entry}

\begin{entry}{投资家}{7,10,10}{⼿、⾙、⼧}
  \begin{phonetics}{投资家}{tou2zi1jia1}
    \definition{s.}{investidor}
  \seealsoref{投资人}{tou2zi1ren2}
  \seealsoref{投资者}{tou2zi1zhe3}
  \end{phonetics}
\end{entry}

\begin{entry}{投递}{7,10}{⼿、⾡}
  \begin{phonetics}{投递}{tou2di4}
    \definition{v.}{despachar | enviar}
  \end{phonetics}
\end{entry}

\begin{entry}{投票}{7,11}{⼿、⽰}
  \begin{phonetics}{投票}{tou2piao4}
    \definition{v.+compl.}{votar | depositar um voto}
  \end{phonetics}
\end{entry}

\begin{entry}{折}{7}{⼿}
  \begin{phonetics}{折}{she2}
    \definition{v.}{estalar; quebrar | perder dinheiro em um negócio}
  \end{phonetics}
  \begin{phonetics}{折}{zhe1}
    \definition{v.}{rolar; virar | despejar algo de um recipiente em outro; ficar despejando algo entre dois recipientes}
  \end{phonetics}
  \begin{phonetics}{折}{zhe2}[][HSK 4]
    \definition*{s.}{sobrenome Zhe}
    \definition{clas.}{uma passagem em um roteiro de ópera miscelânea de Yuan, aproximadamente equivalente a uma cena ou ato em uma ópera moderna}
    \definition[张,个,些]{s.}{fratura; quebra | abatimento; desconto | traços dos caracteres chineses que têm o formato de "𠃍" e "乚", etc. | pasta; livreto; \emph{folder}}
    \definition{v.}{estalar; quebrar; fazer quebrar | perder; sofrer a perda de | voltar para trás; mudar de direção; retornar |ser convencido; estar cheio de admiração | equivaler a; converter em | dobrar}
  \end{phonetics}
\end{entry}

\begin{entry}{折转}{7,8}{⼿、⾞}
  \begin{phonetics}{折转}{zhe2zhuan3}
    \definition{s.}{reflexo (ângulo)}
    \definition{v.}{voltar atrás}
  \end{phonetics}
\end{entry}

\begin{entry}{抢}{7}{⼿}
  \begin{phonetics}{抢}{qiang1}
    \definition{prep.}{contra; direção relativa inversa}
    \definition{v.}{bater; tocar}
  \end{phonetics}
  \begin{phonetics}{抢}{qiang3}[][HSK 5]
    \definition{v.}{roubar; saquear | agarrar; apanhar; arrebatar | disputar; lutar por; ser o primeiro; competir para ser o primeiro | correr; apressar-se; fazer uma incursão | raspar; arranhar; raspar ou esfregar uma camada da superfície de um objeto}
  \end{phonetics}
\end{entry}

\begin{entry}{抢掠}{7,11}{⼿、⼿}
  \begin{phonetics}{抢掠}{qiang3lve4}
    \definition{s.}{saque | pilhagem}
    \definition{v.}{saquear | pilhar}
  \end{phonetics}
\end{entry}

\begin{entry}{抢救}{7,11}{⼿、⽁}
  \begin{phonetics}{抢救}{qiang3jiu4}[][HSK 5]
    \definition{v.}{salvar; resgatar; prestar de socorro ou assistência rápidos em situações de emergência | salvar; tomar medidas rápidas para evitar ou minimizar perdas iminentes.}
  \end{phonetics}
\end{entry}

\begin{entry}{护士}{7,3}{⼿、⼠}
  \begin{phonetics}{护士}{hu4shi5}[][HSK 4]
    \definition[名,位]{s.}{enfermeiro; pessoas especializadas em enfermagem em hospitais ou instituições epidemiológicas}
  \end{phonetics}
\end{entry}

\begin{entry}{护照}{7,13}{⼿、⽕}
  \begin{phonetics}{护照}{hu4zhao4}[][HSK 2]
    \definition[本,个]{s.}{passaporte; documento emitido pela autoridade competente do país para comprovar a nacionalidade e a identidade dos cidadãos que viajam para o exterior}
  \end{phonetics}
\end{entry}

\begin{entry}{报}{7}{⼿}
  \begin{phonetics}{报}{bao4}[][HSK 3]
    \definition[份,张]{s.}{jornal | revista; periódico; referência a uma publicação específica | relatório; boletim; algo que transmite alguma informação | telegrama | julgamento; retribuição}
    \definition{v.}{relatar; declarar; anunciar; informar; comunicar | responder; retribuir; revidar | retribuir; recompensar | vingar-se; retaliar | relatar; condenar de acordo com a lei e reportar às autoridades superiores | enviar; submeter; especificamente, relatar ao superior}
  \end{phonetics}
\end{entry}

\begin{entry}{报名}{7,6}{⼿、⼝}
  \begin{phonetics}{报名}{bao4ming2}[][HSK 2]
    \definition{v.+compl.}{inscrever-se; alistar-se; registrar seu nome; cadastrar-se; matricular-se; informar seu nome à pessoa responsável, órgão, grupo etc., indicando que você deseja participar de alguma atividade ou organização}
  \end{phonetics}
\end{entry}

\begin{entry}{报告}{7,7}{⼿、⼝}
  \begin{phonetics}{报告}{bao4gao4}[][HSK 3]
    \definition[份,篇]{s.}{relatório; discurso; palestra; consultivo; declaração formal feita a superiores ou ao público}
    \definition{v.}{relatar; divulgar; informar; informar formalmente sobre um assunto ou opinião aos superiores ou ao público em geral}
  \end{phonetics}
\end{entry}

\begin{entry}{报纸}{7,7}{⼿、⽷}
  \begin{phonetics}{报纸}{bao4zhi3}[][HSK 2]
    \definition[分,期,张]{s.}{jornal; publicações periódicas cujo conteúdo principal é notícias, geralmente referem-se a jornais diários | papel jornal; um tipo de papel usado para imprimir jornais ou publicações em geral}
  \end{phonetics}
\end{entry}

\begin{entry}{报到}{7,8}{⼿、⼑}
  \begin{phonetics}{报到}{bao4dao4}[][HSK 3]
    \definition{v.+compl.}{apresentar-se ao serviço; fazer o check-in; registrar-se; assinar o livro de presença; informar à organização que você já chegou}
  \end{phonetics}
\end{entry}

\begin{entry}{报答}{7,12}{⼿、⽵}
  \begin{phonetics}{报答}{bao4da2}[][HSK 5]
    \definition{v.}{reembolsar; devolver; retribuir; pagar de volta; mostrar seu apreço de forma tangível}
  \end{phonetics}
\end{entry}

\begin{entry}{报道}{7,12}{⼿、⾡}
  \begin{phonetics}{报道}{bao4dao4}[][HSK 3]
    \definition[个,篇,分]{s.}{história; reportagem; comunicado de imprensa publicado por escrito ou transmitido pela rádio}
    \definition{v.}{cobrir; reportar (notícias); divulgar notícias ao público através de jornais, rádio, etc.}
  \end{phonetics}
\end{entry}

\begin{entry}{报酬}{7,13}{⼿、⾣}
  \begin{phonetics}{报酬}{bao4chou5}
    \definition{s.}{recompensa | remuneração}
  \end{phonetics}
\end{entry}

\begin{entry}{报警}{7,19}{⼿、⾔}
  \begin{phonetics}{报警}{bao4jing3}[][HSK 5]
    \definition{v.}{relatar (um incidente) à polícia; relatar uma situação crítica ou sinalizar uma emergência às autoridades competentes}
  \end{phonetics}
\end{entry}

\begin{entry}{拒绝}{7,9}{⼿、⽷}
  \begin{phonetics}{拒绝}{ju4jue2}[][HSK 5]
    \definition{v.}{recusar; rejeitar; declinar; não aceitar (pedidos, sugestões ou presentes)}
  \end{phonetics}
\end{entry}

\begin{entry}{改}{7}{⽁}
  \begin{phonetics}{改}{gai3}[][HSK 2]
    \definition{v.}{mudar; converter; transformar; alterar; substituir | alterar; revisar; aperfeiçoar; modificar | corrigir; retificar; remediar; consertar}
  \end{phonetics}
\end{entry}

\begin{entry}{改正}{7,5}{⽁、⽌}
  \begin{phonetics}{改正}{gai3 zheng4}[][HSK 4]
    \definition{v.}{corrigir; emendar; mudar o errado para o correto}
  \end{phonetics}
\end{entry}

\begin{entry}{改良}{7,7}{⽁、⾉}
  \begin{phonetics}{改良}{gai3liang2}
    \definition{v.}{melhorar (algo) | reformar (um sistema)}
  \end{phonetics}
\end{entry}

\begin{entry}{改进}{7,7}{⽁、⾡}
  \begin{phonetics}{改进}{gai3jin4}[][HSK 3]
    \definition[个,些]{s.}{melhoria}
    \definition{v.}{aprimorar; aperfeiçoar; melhorar; tornar melhor; mudar a situação antiga para melhorar | modificar (mudança mecânica)}
  \end{phonetics}
\end{entry}

\begin{entry}{改变}{7,8}{⽁、⼜}
  \begin{phonetics}{改变}{gai3bian4}[][HSK 2]
    \definition{v.}{mudar; alterar; transformar; converter; moldar; modificar | causar mudanças; alterar}
  \end{phonetics}
\end{entry}

\begin{entry}{改革}{7,9}{⽁、⾰}
  \begin{phonetics}{改革}{gai3ge2}[][HSK 5]
    \definition[项,次,种]{s.}{reforma; reformação; iniciativas para aprimorar a inovação}
    \definition{v.}{reformar; transformar as antigas partes irracionais das coisas em novas que possam ser adaptadas à situação objetiva}
  \end{phonetics}
\end{entry}

\begin{entry}{改造}{7,10}{⽁、⾡}
  \begin{phonetics}{改造}{gai3 zao4}[][HSK 3]
    \definition{v.}{transformar; renovar; modificar o original para melhor se adequar às necessidades; usado principalmente para coisas específicas | remodelar; mudar radicalmente o que é velho e ruim; criar algo novo e bom, para se adaptar às novas circunstâncias e necessidades; usado principalmente para coisas abstratas}
  \end{phonetics}
\end{entry}

\begin{entry}{改善}{7,12}{⽁、⼝}
  \begin{phonetics}{改善}{gai3shan4}[][HSK 4]
    \definition{v.}{melhorar; amenizar; mudar a situação original para torná-la melhor}
  \end{phonetics}
\end{entry}

\begin{entry}{改善关系}{7,12,6,7}{⽁、⼝、⼋、⽷}
  \begin{phonetics}{改善关系}{gai3shan4guan1xi5}
    \definition{v.}{melhorar a relação}
  \end{phonetics}
\end{entry}

\begin{entry}{改善通讯}{7,12,10,5}{⽁、⼝、⾡、⾔}
  \begin{phonetics}{改善通讯}{gai3shan4tong1xun4}
    \definition{v.}{melhorar a comunicação}
  \end{phonetics}
\end{entry}

\begin{entry}{时}{7}{⽇}
  \begin{phonetics}{时}{shi2}[][HSK 3]
    \definition*{s.}{sobrenome Shi}
    \definition{adj.}{atual; presente | temporário; oportuno}
    \definition{adv.}{de vez em quando; ocasionalmente; de ​​tempos em tempos; equivalente a 常常 ou 经常 | às vezes\dots às vezes\dots; dois caracteres 时 usados juntos são equivalentes a ``有时……有时……'' e ``一会儿……一会儿……''}
    \definition{clas.}{hora, cada uma das 24 partes iguais de um dia e uma noite; também usada como unidade legal de tempo}
    \definition{s.}{dias; tempos; longo período de tempo; refere-se a um período de tempo | tempo; tempo fixo; refere-se ao tempo especificado | hora; hora do dia | temporada | chance; oportunidade; momento oportuno | atual; presente | tempo verbal; uma categoria gramatical que utiliza certas formas gramaticais para indicar o momento em que uma ação ocorre; geralmente é dividida em presente, pretérito e futuro}
  \seealsoref{常常}{chang2 chang2}
  \seealsoref{经常}{jing1chang2}
  \seealsoref{一会儿……一会儿……}{yi1hui4r5 yi1hui4r5}
  \seealsoref{有时……有时……}{you3shi2 you3shi2}
  \end{phonetics}
\end{entry}

\begin{entry}{时代}{7,5}{⽇、⼈}
  \begin{phonetics}{时代}{shi2dai4}[][HSK 3]
    \definition[个]{s.}{idade; era; tempos; época; períodos e fases históricas divididas de acordo com condições econômicas, políticas, culturais e outras | um período na vida de alguém; uma fase na vida de uma pessoa}
  \end{phonetics}
\end{entry}

\begin{entry}{时光}{7,6}{⽇、⼉}
  \begin{phonetics}{时光}{shi2guang1}[][HSK 5]
    \definition[台]{s.}{tempo; passagem do tempo | dias; horas; anos; épocas; períodos}
  \end{phonetics}
\end{entry}

\begin{entry}{时机}{7,6}{⽇、⽊}
  \begin{phonetics}{时机}{shi2ji1}[][HSK 5]
    \definition{s.}{oportunidade; momento oportuno}
  \end{phonetics}
\end{entry}

\begin{entry}{时时}{7,7}{⽇、⽇}
  \begin{phonetics}{时时}{shi2shi2}
    \definition{adv.}{muitas vezes | constantemente}
  \end{phonetics}
\end{entry}

\begin{entry}{时间}{7,7}{⽇、⾨}
  \begin{phonetics}{时间}{shi2jian1}[][HSK 1]
    \definition[段]{s.}{tempo; refere-se à forma de existência do movimento da matéria, um sistema contínuo composto pelo passado, presente e futuro | tempo; período (duração); um período de tempo com início e fim | tempo (um ponto); em algum momento do tempo}
  \end{phonetics}
\end{entry}

\begin{entry}{时事}{7,8}{⽇、⼅}
  \begin{phonetics}{时事}{shi2shi4}[][HSK 5]
    \definition{s.}{acontecimentos atuais; assuntos atuais; eventos atuais | tendências atuais | como as coisas estão indo | a situação atual}
  \end{phonetics}
\end{entry}

\begin{entry}{时刻}{7,8}{⽇、⼑}
  \begin{phonetics}{时刻}{shi2ke4}[][HSK 3]
    \definition{adv.}{constantemente; sempre; a cada momento; frequentemente}
    \definition[个,段]{s.}{tempo; hora; momento; conjuntura; um ponto no tempo}
  \end{phonetics}
\end{entry}

\begin{entry}{时差}{7,9}{⽇、⼯}
  \begin{phonetics}{时差}{shi2cha1}
    \definition{s.}{diferença de tempo | \emph{jet lag}}
  \end{phonetics}
\end{entry}

\begin{entry}{时候}{7,10}{⽇、⼈}
  \begin{phonetics}{时候}{shi2hou5}[][HSK 1]
    \definition[个]{s.}{(um ponto no) tempo; momento; um determinado momento no tempo | (a duração do) tempo; um período de tempo com início e fim}
  \end{phonetics}
\end{entry}

\begin{entry}{时常}{7,11}{⽇、⼱}
  \begin{phonetics}{时常}{shi2chang2}[][HSK 5]
    \definition{adv.}{frequentemente; com frequência}
  \end{phonetics}
\end{entry}

\begin{entry}{旷野}{7,11}{⽇、⾥}
  \begin{phonetics}{旷野}{kuang4ye3}
    \definition{s.}{região selvagem}
  \end{phonetics}
\end{entry}

\begin{entry}{更}{7}{⽈}
  \begin{phonetics}{更}{geng1}
    \definition*{s.}{sobrenome Geng}
    \definition{clas.}{um dos cinco períodos de duas horas em que a noite era anteriormente dividida; vigília; antigamente, a noite era dividida em cinco turnos, cada um com aproximadamente duas horas de duração}
    \definition{v.}{alterar; substituir | experimentar}
  \end{phonetics}
  \begin{phonetics}{更}{geng4}[][HSK 2]
    \definition{adv.}{mais; ainda mais | além disso; além do mais; ainda mais}
  \end{phonetics}
\end{entry}

\begin{entry}{更加}{7,5}{⽈、⼒}
  \begin{phonetics}{更加}{geng4 jia1}[][HSK 3]
    \definition{adv.}{mais; ainda mais; em maior grau; indica um nível mais profundo ou um aumento ou diminuição quantitativa adicional}
  \end{phonetics}
\end{entry}

\begin{entry}{更换}{7,10}{⽈、⼿}
  \begin{phonetics}{更换}{geng1 huan4}[][HSK 5]
    \definition{v.}{alterar; mudar; substituir; comutar}
  \end{phonetics}
\end{entry}

\begin{entry}{更新}{7,13}{⽈、⽄}
  \begin{phonetics}{更新}{geng1xin1}[][HSK 5]
    \definition{v.}{renovar; atualizar; substituir; remover o antigo e substituir pelo novo}
  \end{phonetics}
\end{entry}

\begin{entry}{李}{7}{⽊}
  \begin{phonetics}{李}{li3}
    \definition*{s.}{sobrenome Li}
    \definition{s.}{ameixa}
  \end{phonetics}
\end{entry}

\begin{entry}{李子}{7,3}{⽊、⼦}
  \begin{phonetics}{李子}{li3zi5}
    \definition[个]{s.}{ameixa}
  \end{phonetics}
\end{entry}

\begin{entry}{李四}{7,5}{⽊、⼞}
  \begin{phonetics}{李四}{li3si4}
    \definition*{s.}{Li Si | Zé Ninguém | nome para uma pessoa não especificada, 2 de 3}
  \seealsoref{王五}{wang2wu3}
  \seealsoref{张三}{zhang1san1}
  \end{phonetics}
\end{entry}

\begin{entry}{材}{7}{⽊}
  \begin{phonetics}{材}{cai2}
    \definition[份]{s.}{madeira | material; geralmente se refere a coisas que podem ser transformadas diretamente em produtos acabados | material; materiais para escrita ou referência | pessoa capaz; pessoas talentosas | habilidade; talento; aptidão | caixão}
  \end{phonetics}
\end{entry}

\begin{entry}{材料}{7,10}{⽊、⽃}
  \begin{phonetics}{材料}{cai2liao4}[][HSK 4]
    \definition[份,个,种]{s.}{material; algo para fazer um produto acabado | material (figura de linguagem) | dados; material para estudo, pesquisa, etc.; conteúdo de uma obra}
  \end{phonetics}
\end{entry}

\begin{entry}{村}{7}{⽊}
  \begin{phonetics}{村}{cun1}[][HSK 3]
    \definition{adj.}{rústico; grosseiro}
    \definition[个,座]{s.}{aldeia; vila | área povoada de certo tipo}
  \end{phonetics}
\end{entry}

\begin{entry}{村儿}{7,2}{⽊、⼉}
  \begin{phonetics}{村儿}{cun1r5}
    \definition{s.}{vila; aldeia}
  \end{phonetics}
\end{entry}

\begin{entry}{杜}{7}{⽊}
  \begin{phonetics}{杜}{du4}
    \definition{s.}{pêra de folha de bétula}
    \definition{s.}{sobrenome Du}
    \definition{v.}{excluir; parar; impedir; bloquear}
  \end{phonetics}
\end{entry}

\begin{entry}{杜宇}{7,6}{⽊、⼧}
  \begin{phonetics}{杜宇}{du4yu3}
    \definition{s.}{cuco (pássaro)}
  \seealsoref{布谷鸟}{bu4gu3niao3}
  \seealsoref{杜鹃}{du4juan1}
  \seealsoref{杜鹃鸟}{du4juan1niao3}
  \end{phonetics}
\end{entry}

\begin{entry}{杜鹃}{7,12}{⽊、⿃}
  \begin{phonetics}{杜鹃}{du4juan1}
    \definition{s.}{cuco (pássaro)}
  \seealsoref{布谷鸟}{bu4gu3niao3}
  \seealsoref{杜鹃鸟}{du4juan1niao3}
  \seealsoref{杜宇}{du4yu3}
  \end{phonetics}
\end{entry}

\begin{entry}{杜鹃鸟}{7,12,5}{⽊、⿃、⿃}
  \begin{phonetics}{杜鹃鸟}{du4juan1niao3}
    \definition{s.}{cuco (pássaro)}
  \seealsoref{布谷鸟}{bu4gu3niao3}
  \seealsoref{杜鹃}{du4juan1}
  \seealsoref{杜宇}{du4yu3}
  \end{phonetics}
\end{entry}

\begin{entry}{束}{7}{⽊}
  \begin{phonetics}{束}{shu4}[][HSK 3]
    \definition*{s.}{sobrenome Shu}
    \definition{clas.}{usado para cachos, molhos, feixes, feixes de luz, etc.}
    \definition{s.}{monte; pacote; maço; feixe; cacho; coisas agrupadas ou reunidas em tiras}
    \definition{v.}{atar; amarrar; vincular | controlar; restringir}
  \end{phonetics}
\end{entry}

\begin{entry}{束腰}{7,13}{⽊、⾁}
  \begin{phonetics}{束腰}{shu4yao1}
    \definition{s.}{cinto | cinta | cinturão}
  \end{phonetics}
\end{entry}

\begin{entry}{杠}{7}{⽊}
  \begin{phonetics}{杠}{gang1}
    \definition{s.}{pequena ponte | mastro de bandeira}
  \end{phonetics}
  \begin{phonetics}{杠}{gang4}
    \definition{s.}{vara grossa | (esportes) barra | peça sobressalente em forma de haste; peça sobressalente em forma de haste usada para máquinas-ferramentas | varas robustas usadas para carregar um caixão | (em um texto) linha grossa desenhada ao lado ou abaixo das palavras como uma marca | (coloquial) padrão; critério}
    \definition{v.}{marcar com uma linha grossa | afiar (faca, navalha, etc.)}
  \end{phonetics}
\end{entry}

\begin{entry}{条}{7}{⽊}
  \begin{phonetics}{条}{tiao2}[][HSK 2]
    \definition*{s.}{sobrenome Tiao}
    \definition{clas.}{usado para objetos longos e finos; usado para sintetizar certas coisas longas e retangulares em quantidades fixas | usado para itemização | aplicado ao corpo humano}
    \definition{s.}{galho; galhos finos e longos | tira; faixa | item; artigo | ordem; método | nota; anotação em papel}
  \end{phonetics}
\end{entry}

\begin{entry}{条目}{7,5}{⽊、⽬}
  \begin{phonetics}{条目}{tiao2mu4}
    \definition{s.}{cláusulas e subcláusulas (em documento formal) | verbete (em um dicionário, enciclopédia, etc.)}
  \end{phonetics}
\end{entry}

\begin{entry}{条件}{7,6}{⽊、⼈}
  \begin{phonetics}{条件}{tiao2jian4}[][HSK 2]
    \definition[个,项,些]{s.}{condição; termo; fator; fatores que restringem a ocorrência, existência ou desenvolvimento das coisas | requisito; pré-requisito; qualificação; requisitos ou padrões estabelecidos para determinadas coisas | situação; estado; condição}
  \end{phonetics}
\end{entry}

\begin{entry}{条例}{7,8}{⽊、⼈}
  \begin{phonetics}{条例}{tiao2li4}
    \definition{s.}{código de conduta | ordenanças | regulamentos | regras | estatutos}
  \end{phonetics}
\end{entry}

\begin{entry}{条贯}{7,8}{⽊、⾙}
  \begin{phonetics}{条贯}{tiao2guan4}
    \definition{s.}{ordem | procedimentos | sequência | sistema}
  \end{phonetics}
\end{entry}

\begin{entry}{条幅}{7,12}{⽊、⼱}
  \begin{phonetics}{条幅}{tiao2fu2}
    \definition{s.}{faixa | banner | pergaminho de parede (para pintura ou caligrafia)}
  \end{phonetics}
\end{entry}

\begin{entry}{来}{7}{⽊}
  \begin{phonetics}{来}{lai2}[][HSK 1]
    \definition*{s.}{sobrenome Lai}
    \definition{part.}{usado após uma palavra numérica ou de quantidade; indica uma quantidade aproximada | usado depois de numerais como 一, 二, 三; para listar razões ou fatos, etc.}
    \definition{s.}{usado após uma expressão de tempo para indicar uma duração que vai do passado ao presente}
    \definition{v.}{vir; chegar; de outro lugar para o lugar onde o interlocutor se encontra | aparecer; acontecer; vir; (problemas, coisas, etc.) ocorrerem; surgirem | substitui um verbo com significado específico, indicando a realização de uma ação específica | estar indo para; usado antes de outro verbo, indica que algo será feito | vir para fazer algo; usado após outro verbo, indica que se vai fazer algo | usado para indicar um propósito; expressar o objetivo, fazer algo usando o método, a atitude ou a direção anteriores | usado com 得 ou 不 para indicar possibilidade, capacidade ou hábito}
  \seealsoref{不}{bu4}
  \seealsoref{得}{de5}
  \end{phonetics}
\end{entry}

\begin{entry}{来不及}{7,4,3}{⽊、⼀、⼃}
  \begin{phonetics}{来不及}{lai2bu5ji2}[][HSK 4]
    \definition{v.}{ser tarde demais; não ter tempo; não ter tempo suficiente (para fazer algo); não ser possível participar ou se atualizar devido a restrições de tempo}
  \end{phonetics}
\end{entry}

\begin{entry}{来自}{7,6}{⽊、⾃}
  \begin{phonetics}{来自}{lai2zi4}[][HSK 2]
    \definition{v.}{vir de (um local) | \emph{From:} (cabeçalho de \emph{e -mail})}
  \end{phonetics}
\end{entry}

\begin{entry}{来到}{7,8}{⽊、⼑}
  \begin{phonetics}{来到}{lai2 dao4}[][HSK 1]
    \definition{v.}{chegar; vir}
  \end{phonetics}
\end{entry}

\begin{entry}{来信}{7,9}{⽊、⼈}
  \begin{phonetics}{来信}{lai2 xin4}[][HSK 5]
    \definition{s.}{sua carta; carta recebida; carta ao interlocutor}
    \definition{v.}{enviar uma carta para aqui; enviar uma carta para o remetente}
  \end{phonetics}
\end{entry}

\begin{entry}{来得及}{7,11,3}{⽊、⼻、⼃}
  \begin{phonetics}{来得及}{lai2de5ji2}[][HSK 4]
    \definition{v.}{ainda ter tempo; ser capaz de fazê-lo; ser capaz de fazer algo a tempo; ainda ter tempo de chegar lá ou de se atualizar}
  \end{phonetics}
\end{entry}

\begin{entry}{来源}{7,13}{⽊、⽔}
  \begin{phonetics}{来源}{lai2yuan2}[][HSK 4]
    \definition{s.}{origem; causa; fonte; tabula rasa (ou seja, o lugar de onde as coisas vêm)}
    \definition{v.}{originar-se; surgir; ter origem; (algo) originar (seguido de 于)}
  \seealsoref{于}{yu2}
  \end{phonetics}
\end{entry}

\begin{entry}{极}{7}{⽊}
  \begin{phonetics}{极}{ji2}[][HSK 4]
    \definition*{s.}{sobrenome Ji}
    \definition{adj.}{máximo; extremo; final; supremo}
    \definition{adv.}{extremamente; excessivamente}
    \definition{s.}{o ponto máximo, mais alto; extremo; ápice; ponto culminante | pólo; as extremidades norte e sul da Terra; as extremidades de um ímã; a extremidade de uma fonte de alimentação ou de um aparelho elétrico onde a corrente entra ou sai do aparelho}
    \definition{v.}{chegar ao fim de; levar a extremos}
  \end{phonetics}
\end{entry}

\begin{entry}{……极了}{7,2}{⽊、⼅}
  \begin{phonetics}{……极了}{ji2le5}[][HSK 3]
    \definition{expr.}{extremamente; alto grau de expressão}
  \end{phonetics}
\end{entry}

\begin{entry}{极其}{7,8}{⽊、⼋}
  \begin{phonetics}{极其}{ji2qi2}[][HSK 4]
    \definition{adv.}{mais; extremamente; excessivamente}
  \end{phonetics}
\end{entry}

\begin{entry}{步}{7}{⽌}
  \begin{phonetics}{步}{bu4}[][HSK 3]
    \definition*{s.}{sobrenome Bu}
    \definition*{s.}{geralmente em nomes de lugares}[盐步___Yanbu, na província de Guangdong]
    \definition{clas.}{uma unidade antiga para medida de comprimento, equivalente a cinco 尺}
    \definition{s.}{passo; ritmo | etapa; passo | condição; situação; estado | cais; píer | porto; cidade portuária | (geralmente em nomes de lugares)}
    \definition{v.}{caminhar; ir a pé | seguir os passos de alguém | (dialeto) medir com passos | seguir; acompanhar | medir a distância com os passos}
  \seealsoref{尺}{chi3}
  \end{phonetics}
\end{entry}

\begin{entry}{步行}{7,6}{⽌、⾏}
  \begin{phonetics}{步行}{bu4 xing2}[][HSK 4]
    \definition{v.}{caminhar; ir a pé; andar a pé (diferente de andar de carro, a cavalo, etc.)}
  \end{phonetics}
\end{entry}

\begin{entry}{每}{7}{⽏}
  \begin{phonetics}{每}{mei3}[][HSK 3]
    \definition{adv.}{cada um; cada qual; indica qualquer uma das repetições ou um conjunto de repetições de um movimento}
    \definition{pron.}{cada; cada um; cada qual; refere-se a qualquer indivíduo do grupo, enfatizando as semelhanças entre os indivíduos}
  \end{phonetics}
\end{entry}

\begin{entry}{每个}{7,3}{⽏、⼈}
  \begin{phonetics}{每个}{mei3ge4}
    \definition{pron.}{cada; cada um}
  \end{phonetics}
\end{entry}

\begin{entry}{每个人}{7,3,2}{⽏、⼈、⼈}
  \begin{phonetics}{每个人}{mei3ge5ren2}
    \definition{pron.}{todo mundo | todos}
  \end{phonetics}
\end{entry}

\begin{entry}{每天}{7,4}{⽏、⼤}
  \begin{phonetics}{每天}{mei3tian1}
    \definition{adv.}{todo dia | cada dia}
  \end{phonetics}
\end{entry}

\begin{entry}{每次}{7,6}{⽏、⽋}
  \begin{phonetics}{每次}{mei3ci4}
    \definition{adv.}{toda vez | cada vez}
  \end{phonetics}
\end{entry}

\begin{entry}{求}{7}{⽔}
  \begin{phonetics}{求}{qiu2}[][HSK 2]
    \definition*{s.}{sobrenome Qiu}
    \definition{v.}{implorar; solicitar; suplicar; rogar | lutar por; buscar; investigar | tentar; procurar; tentar obter | demandar}
  \end{phonetics}
\end{entry}

\begin{entry}{汹涌}{7,10}{⽔、⽔}
  \begin{phonetics}{汹涌}{xiong1yong3}
    \definition{adj.}{turbulento}
    \definition{v.}{aumentar ou emergir violentamente (oceano, rio, lago, etc.)}
  \end{phonetics}
\end{entry}

\begin{entry}{汽水}{7,4}{⽔、⽔}
  \begin{phonetics}{汽水}{qi4 shui3}[][HSK 4]
    \definition[罐,瓶]{s.}{refrigerante; refrigerante gaseificado; bebida refrescante, feita com a pressão de dióxido de carbono para dissolver na água e adicionar açúcar, suco de frutas, especiarias etc.}
  \end{phonetics}
\end{entry}

\begin{entry}{汽车}{7,4}{⽔、⾞}
  \begin{phonetics}{汽车}{qi4 che1}[][HSK 1]
    \definition[辆,种,款]{s.}{automóvel; carro; veículo motorizado; veículo movido a motor de combustão interna, que circula principalmente em rodovias ou ruas, geralmente com quatro ou mais pneus de borracha, usado para transportar pessoas ou mercadorias}
  \end{phonetics}
\end{entry}

\begin{entry}{汽油}{7,8}{⽔、⽔}
  \begin{phonetics}{汽油}{qi4you2}[][HSK 4]
    \definition{s.}{gasolina; mistura líquida de hidrocarbonetos com volatilidade e combustibilidade, que é usada como combustível a partir do fracionamento ou craqueamento do petróleo}
  \end{phonetics}
\end{entry}

\begin{entry}{沉}{7}{⽔}
  \begin{phonetics}{沉}{chen2}[][HSK 4]
    \definition{adj.}{profundo | pesado | pesado (sentir-se pesado)}
    \definition{v.}{afundar; submergir; imergir | manter baixo; abaixar | descansar; parar}
  \end{phonetics}
\end{entry}

\begin{entry}{沉重}{7,9}{⽔、⾥}
  \begin{phonetics}{沉重}{chen2zhong4}[][HSK 4]
    \definition{adj.}{(pressão, fardo, etc.) muito pesado; profundo | sério; pesado; humor pouco animador; fardo pesado de pensamentos}
  \end{phonetics}
\end{entry}

\begin{entry}{沉默}{7,16}{⽔、⿊}
  \begin{phonetics}{沉默}{chen2mo4}[][HSK 4]
    \definition{adj.}{silencioso; reticente; taciturno; não comunicativo}
    \definition{v.}{silenciar; não falar por causa de alguma coisa}
  \end{phonetics}
\end{entry}

\begin{entry}{沙}{7}{⽔}
  \begin{phonetics}{沙}{sha1}
    \definition*{s.}{sobrenome Sha}
    \definition[粒]{s.}{areia | cascalho | grânulo | pó}
  \end{phonetics}
\end{entry}

\begin{entry}{沙子}{7,3}{⽔、⼦}
  \begin{phonetics}{沙子}{sha1 zi5}[][HSK 3]
    \definition[粒,把,堆,袋,车]{s.}{areia; grão; pequenas pedras | \emph{pellets}; grãos pequenos; coisas parecidas com areia}
  \end{phonetics}
\end{entry}

\begin{entry}{沙发}{7,5}{⽔、⼜}
  \begin{phonetics}{沙发}{sha1fa1}[][HSK 3]
    \definition[套,组,个,张]{s.}{sofá; assentos com molas ou espuma plástica espessa, etc., com apoios de braços em ambos os lados}
  \end{phonetics}
\end{entry}

\begin{entry}{沙鱼}{7,8}{⽔、⿂}
  \begin{phonetics}{沙鱼}{sha1yu2}
    \variantof{鲨鱼}
  \end{phonetics}
\end{entry}

\begin{entry}{沙特}{7,10}{⽔、⽜}
  \begin{phonetics}{沙特}{sha1te4}
    \definition*{s.}{Saudita | abreviação de 沙特阿拉伯}
  \seealsoref{沙特阿拉伯}{sha1te4 a1la1bo2}
  \end{phonetics}
\end{entry}

\begin{entry}{沙特阿拉伯}{7,10,7,8,7}{⽔、⽜、⾩、⼿、⼈}
  \begin{phonetics}{沙特阿拉伯}{sha1te4 a1la1bo2}
    \definition*{s.}{Arábia Saudita}
  \end{phonetics}
\end{entry}

\begin{entry}{沙漠}{7,13}{⽔、⽔}
  \begin{phonetics}{沙漠}{sha1mo4}[][HSK 5]
    \definition[个]{s.}{deserto; superfície totalmente coberta por areia, sem água corrente, clima seco e vegetação escassa}
  \end{phonetics}
\end{entry}

\begin{entry}{沟}{7}{⽔}
  \begin{phonetics}{沟}{gou1}[][HSK 5]
    \definition[条]{s.}{canal; vala; sarjeta; trincheira; cursos d'água ou fortificações escavados | ranhura; sulco raso; uma depressão que se assemelha a uma vala | ravina; barranco; cursos d'água}
  \end{phonetics}
\end{entry}

\begin{entry}{沟通}{7,10}{⽔、⾡}
  \begin{phonetics}{沟通}{gou1tong1}[][HSK 5]
    \definition{v.}{comunicar; comunicar-se para entender as ideias, opiniões, etc. | conectar; ligar; estabelecer um paralelo entre os dois}
  \end{phonetics}
\end{entry}

\begin{entry}{没}{7}{⽔}
  \begin{phonetics}{没}{mei2}[][HSK 1]
    \definition{adv.}{não; nunca; negar que uma ação ou situação tenha ocorrido, com o significado de 不曾}
    \definition{pref.}{não (prefixo negativo para verbos, traduzido para outras línguas com verbos no pretérito)}
    \definition{v.}{não possuir; não ter | não existe; não há | ninguém; usado antes de 谁, 什么, 哪个, significa 全都不 | não ser tão bom quanto; ser inferior a; não chega a; não é tão bom quanto | menor que; insuficiente}
  \seealsoref{不曾}{bu4 ceng2}
  \seealsoref{哪个}{na3ge5}
  \seealsoref{全都不}{quan2dou1 bu4}
  \seealsoref{谁}{shei2}
  \seealsoref{什么}{shen2me5}
  \end{phonetics}
  \begin{phonetics}{没}{mo4}
    \definition{adj.}{último; final}
    \definition{v.}{afundar na água; submergir | transbordar; subir além; exceder ou ultrapassar | esconder-se; desaparecer; sumir; ocultar-se | confiscar; expropriar | morrer}
    \variantof{没}
  \end{phonetics}
\end{entry}

\begin{entry}{没了}{7,2}{⽔、⼅}
  \begin{phonetics}{没了}{mei2le5}
    \definition{v.}{estar morto | deixar de existir}
  \end{phonetics}
\end{entry}

\begin{entry}{没什么}{7,4,3}{⽔、⼈、⼃}
  \begin{phonetics}{没什么}{mei2 shen2 me5}[][HSK 1]
    \definition{expr.}{não é nada; está tudo bem; não importa}
  \end{phonetics}
\end{entry}

\begin{entry}{没用}{7,5}{⽔、⽤}
  \begin{phonetics}{没用}{mei2 yong4}[][HSK 3]
    \definition{adj.}{inútil; imprestável; sem valor; sem préstimo; vão; que não serve para nada}
  \end{phonetics}
\end{entry}

\begin{entry}{没关系}{7,6,7}{⽔、⼋、⽷}
  \begin{phonetics}{没关系}{mei2guan1xi5}[][HSK 1]
    \definition{v.}{está tudo bem; não é nada; não importa; não se preocupe}
  \seealsoref{没有关系}{mei2you3guan1xi5}
  \end{phonetics}
\end{entry}

\begin{entry}{没有}{7,6}{⽔、⽉}
  \begin{phonetics}{没有}{mei2 you3}[][HSK 1]
    \definition{adv.}{ainda não; (usado com o pretérito) não; ação ou estado negativo ocorreu}
    \definition{v.}{não há; não tem; não existe}
  \end{phonetics}
\end{entry}

\begin{entry}{没有关系}{7,6,6,7}{⽔、⽉、⼋、⽷}
  \begin{phonetics}{没有关系}{mei2you3guan1xi5}
    \definition{v.}{não ter problema | não ter importância | não fazer mal}
  \seealsoref{没关系}{mei2guan1xi5}
  \end{phonetics}
\end{entry}

\begin{entry}{没有次序}{7,6,6,7}{⽔、⽉、⽋、⼴}
  \begin{phonetics}{没有次序}{mei2you3 ci4xu4}
    \definition{adj.}{sem ordem; nenhuma ordem}
  \end{phonetics}
\end{entry}

\begin{entry}{没有意思}{7,6,13,9}{⽔、⽉、⼼、⼼}
  \begin{phonetics}{没有意思}{mei2you3yi4si5}
    \definition{adj.}{tedioso | chato | sem interesse}
  \end{phonetics}
\end{entry}

\begin{entry}{没事儿}{7,8,2}{⽔、⼅、⼉}
  \begin{phonetics}{没事儿}{mei2 shi4r5}[][HSK 1]
    \definition{expr.}{fora de perigo; nada sério | não importa; não é nada; está tudo bem; não importa | está tudo bem; sem problemas; não se preocupe com isso; não é grande coisa; não há nada errado}
    \definition{v.}{não ter nada para fazer; ser livre; estar perdido | estar desempregado; estar sem trabalho | não ter responsabilidade}
  \end{phonetics}
\end{entry}

\begin{entry}{没法儿}{7,8,2}{⽔、⽔、⼉}
  \begin{phonetics}{没法儿}{mei2 fa3r5}[][HSK 4]
    \definition{adv.}{não pode; sem chance}
  \end{phonetics}
\end{entry}

\begin{entry}{没想到}{7,13,8}{⽔、⼼、⼑}
  \begin{phonetics}{没想到}{mei2 xiang3 dao4}[][HSK 4]
    \definition{expr.}{não esperava; inesperado}
  \end{phonetics}
\end{entry}

\begin{entry}{没错}{7,13}{⽔、⾦}
  \begin{phonetics}{没错}{mei2 cuo4}[][HSK 4]
    \definition{adv.}{está certo; é isso mesmo; não há como errar}
  \end{phonetics}
\end{entry}

\begin{entry}{灵感}{7,13}{⽕、⼼}
  \begin{phonetics}{灵感}{ling2gan3}
    \definition{s.}{inspiração | explosão de criatividade em empreendimento científico ou artístico}
  \end{phonetics}
\end{entry}

\begin{entry}{灵魂}{7,13}{⽕、⿁}
  \begin{phonetics}{灵魂}{ling2hun2}
    \definition{s.}{alma | espírito}
  \end{phonetics}
\end{entry}

\begin{entry}{灶台}{7,5}{⽕、⼝}
  \begin{phonetics}{灶台}{zao4tai2}
    \definition{s.}{fogão}
  \end{phonetics}
\end{entry}

\begin{entry}{灾}{7}{⽕}
  \begin{phonetics}{灾}{zai1}[][HSK 5]
    \definition[个,场]{s.}{calamidade; desastre | infortúnio pessoal; adversidade | azar}
  \end{phonetics}
\end{entry}

\begin{entry}{灾区}{7,4}{⽕、⼖}
  \begin{phonetics}{灾区}{zai1 qu1}[][HSK 5]
    \definition{s.}{área de desastre; área afetada por catástrofes}
  \end{phonetics}
\end{entry}

\begin{entry}{灾害}{7,10}{⽕、⼧}
  \begin{phonetics}{灾害}{zai1hai4}[][HSK 5]
    \definition[个]{s.}{desastre; calamidade; danos causados pela seca, inundações, pragas, granizo, guerras, etc.}
  \end{phonetics}
\end{entry}

\begin{entry}{灾难}{7,10}{⽕、⾫}
  \begin{phonetics}{灾难}{zai1nan4}[][HSK 5]
    \definition[场,次]{s.}{desastre; sofrimento; calamidade; catástrofe; danos e sofrimentos causados por desastres naturais ou guerras}
  \end{phonetics}
\end{entry}

\begin{entry}{状况}{7,7}{⽝、⼎}
  \begin{phonetics}{状况}{zhuang4kuang4}[][HSK 3]
    \definition[个,种]{s.}{estado; \emph{status}; situação; condição; estado de coisas; a aparência ou o estado em que as coisas se apresentam}
  \end{phonetics}
\end{entry}

\begin{entry}{状态}{7,8}{⽝、⼼}
  \begin{phonetics}{状态}{zhuang4tai4}[][HSK 3]
    \definition[种,个]{s.}{\emph{status}; estado; condição; situação; estado de coisas; a forma manifestada por pessoas ou coisas}
  \end{phonetics}
\end{entry}

\begin{entry}{犹豫}{7,15}{⽝、⾗}
  \begin{phonetics}{犹豫}{you2yu4}[][HSK 5]
    \definition{adj.}{hesitante; indeciso, incapaz de decidir ou agir}
    \definition{v.}{hesitar; ser indeciso}
  \end{phonetics}
\end{entry}

\begin{entry}{狂}{7}{⽝}
  \begin{phonetics}{狂}{kuang2}[][HSK 5]
    \definition*{s.}{sobrenome Kuang}
    \definition{adj.}{louco; maluco | violento; selvagem | selvagem; delirante; furioso; desenfreado; desinibido; sem restrições | arrogante; autoritário}
  \end{phonetics}
\end{entry}

\begin{entry}{狂欢节}{7,6,5}{⽝、⽋、⾋}
  \begin{phonetics}{狂欢节}{kuang2huan1jie2}
    \definition*{s.}{Carnaval}
  \end{phonetics}
\end{entry}

\begin{entry}{男}{7}{⽥}
  \begin{phonetics}{男}{nan2}[][HSK 1]
    \definition{adj.}{homem; macho; masculino (em oposição a 女)}
    \definition[个,位]{s.}{filho; menino | homem | barão (o mais baixo de cinco ordens de nobreza)}
  \seealsoref{女}{nv3}
  \end{phonetics}
\end{entry}

\begin{entry}{男人}{7,2}{⽥、⼈}
  \begin{phonetics}{男人}{nan2 ren2}[][HSK 1]
    \definition[个]{s.}{homem adulto; macho; cavalheiro | marido}
  \end{phonetics}
\end{entry}

\begin{entry}{男士}{7,3}{⽥、⼠}
  \begin{phonetics}{男士}{nan2 shi4}[][HSK 4]
    \definition{s.}{cavalheiro; \emph{gentleman}}
  \end{phonetics}
\end{entry}

\begin{entry}{男女}{7,3}{⽥、⼥}
  \begin{phonetics}{男女}{nan2 nv3}[][HSK 4]
    \definition{s.}{homens e mulheres; masculino e feminino}
  \end{phonetics}
\end{entry}

\begin{entry}{男子}{7,3}{⽥、⼦}
  \begin{phonetics}{男子}{nan2zi3}[][HSK 3]
    \definition[个,位]{s.}{uma pessoa do sexo masculino; um homem}
  \end{phonetics}
\end{entry}

\begin{entry}{男生}{7,5}{⽥、⽣}
  \begin{phonetics}{男生}{nan2 sheng1}[][HSK 1]
    \definition[个]{s.}{menino; estudante; estudante do sexo masculino; aluno do sexo masculino}
  \end{phonetics}
\end{entry}

\begin{entry}{男性}{7,8}{⽥、⼼}
  \begin{phonetics}{男性}{nan2 xing4}[][HSK 5]
    \definition{s.}{masculino; homem; masculinidade}
  \end{phonetics}
\end{entry}

\begin{entry}{男朋友}{7,8,4}{⽥、⽉、⼜}
  \begin{phonetics}{男朋友}{nan2 peng2 you5}[][HSK 1]
    \definition{s.}{namorado}
  \end{phonetics}
\end{entry}

\begin{entry}{男孩儿}{7,9,2}{⽥、⼦、⼉}
  \begin{phonetics}{男孩儿}{nan2hai2r5}[][HSK 1]
    \definition{s.}{menino; rapaz}
  \end{phonetics}
\end{entry}

\begin{entry}{疗养}{7,9}{⽧、⼋}
  \begin{phonetics}{疗养}{liao2 yang3}[][HSK 4]
    \definition{v.}{recuperar; convalescer; tratar pessoas com doenças crônicas ou debilitantes em instituições médicas especializadas com foco na recuperação}
  \end{phonetics}
\end{entry}

\begin{entry}{社}{7}{⽰}
  \begin{phonetics}{社}{she4}[][HSK 5]
    \definition[个]{s.}{agência; sociedade; órgão organizado; organização; comunidade | comuna popular | o deus da terra, sacrifícios a ele ou altares para tais sacrifícios; na antiguidade, o deus da terra, o local onde ele era venerado, o dia da veneração e o ritual eram chamados de 社 | agência de notícias |  imprensa}
  \end{phonetics}
\end{entry}

\begin{entry}{社区}{7,4}{⽰、⼖}
  \begin{phonetics}{社区}{she4qu1}[][HSK 5]
    \definition{s.}{bairro; comunidade residencial; bairros da cidade, divididos de acordo com a localização geográfica | distrito; comunidade (para pessoas da mesma classe social, etc.) ; lugar onde pessoas com características comuns, como classe social, vivem juntas}
  \end{phonetics}
\end{entry}

\begin{entry}{社会}{7,6}{⽰、⼈}
  \begin{phonetics}{社会}{she4hui4}[][HSK 3]
    \definition[个,种]{s.}{sociedade; em um determinado estágio do desenvolvimento histórico, a relação geral entre as pessoas nas atividades de produção | comunidade; geralmente se refere a um grupo de pessoas que estão conectadas por atividades comuns}
  \end{phonetics}
\end{entry}

\begin{entry}{私}{7}{⽲}
  \begin{phonetics}{私}{si1}
    \definition{adj.}{pessoal; privado (oposição a 公) | egoísta (oposto a 公) | secreto; privado | ilícito; ilegal}
    \definition{s.}{interesse privado (ou egoísta); motivo (ou ideia) egoísta (oposição a 公) | contrabando; mercadorias contrabandeadas | propriedade privada | interesses privados; ganho pessoal}
    \definition{s.}{sobrenome Si}
  \seealsoref{公}{gong1}
  \end{phonetics}
\end{entry}

\begin{entry}{私人}{7,2}{⽲、⼈}
  \begin{phonetics}{私人}{si1ren2}[][HSK 5]
    \definition{adj.}{privado; pertencente a um indivíduo ou exercido a título individual; não público | interpessoal}
    \definition[个]{s.}{algo privado; pessoas que se aproximam de você por motivos pessoais ou interesses próprios}
  \end{phonetics}
\end{entry}

\begin{entry}{私人诊所}{7,2,7,8}{⽲、⼈、⾔、⼾}
  \begin{phonetics}{私人诊所}{si1ren2 zhen3suo3}
    \definition[些]{s.}{clínica privada}
  \end{phonetics}
\end{entry}

\begin{entry}{私人信件}{7,2,9,6}{⽲、⼈、⼈、⼈}
  \begin{phonetics}{私人信件}{si1ren2 xin4jian4}
    \definition{s.}{carta pessoal}
  \end{phonetics}
\end{entry}

\begin{entry}{私人钥匙}{7,2,9,11}{⽲、⼈、⾦、⼔}
  \begin{phonetics}{私人钥匙}{si1ren2yao4shi5}
    \definition{s.}{(criptografia) chave privada}
  \end{phonetics}
\end{entry}

\begin{entry}{私生活}{7,5,9}{⽲、⽣、⽔}
  \begin{phonetics}{私生活}{si1sheng1huo2}
    \definition{s.}{vida privada}
  \end{phonetics}
\end{entry}

\begin{entry}{私自}{7,6}{⽲、⾃}
  \begin{phonetics}{私自}{si1zi4}
    \definition{adj.}{privado | pessoal}
    \definition{adv.}{secretamente | sem aprovação explícita}
  \end{phonetics}
\end{entry}

\begin{entry}{究竟}{7,11}{⽳、⾳}
  \begin{phonetics}{究竟}{jiu1jing4}[][HSK 4]
    \definition{adv.}{de fato; exatamente; usado em frases interrogativas para buscar | afinal de contas, no final; ênfase em fatos ou motivos}
    \definition{s.}{resultado; desfecho; a causa, o efeito ou a história completa do que aconteceu}
  \end{phonetics}
\end{entry}

\begin{entry}{穷}{7}{⽳}
  \begin{phonetics}{穷}{qiong2}[][HSK 4]
    \definition{adj.}{remoto; isolado; de difícil acesso | pobre; atingido pela pobreza | situação difícil, sem saída}
    \definition{adv.}{completamente | extremamente}
    \definition{v.}{exaurir; esgotar; consmir | ir até o fim; perseguir completamente perseguido; sondar profundamente | gastar}
  \end{phonetics}
\end{entry}

\begin{entry}{穷人}{7,2}{⽳、⼈}
  \begin{phonetics}{穷人}{qiong2 ren2}[][HSK 4]
    \definition{s.}{os pobres; pessoas pobres}
  \end{phonetics}
\end{entry}

\begin{entry}{系}{7}{⽷}
  \begin{phonetics}{系}{ji4}
    \definition{v.}{amarrar; prender; abotoar; dar um nó}
  \end{phonetics}
  \begin{phonetics}{系}{xi4}[][HSK 3,4]
    \definition*{s.}{sobrenome Xi}
    \definition{s.}{sistema; série | departamento; faculdade; unidades administrativas de ensino divididas por disciplina nas instituições de ensino superior}
    \definition{v.}{relacionar-se com; suportar; depender de | sentir-se ansioso; estar preocupado | amarrar; prender | ser; expressa julgamento, equivalente a 是}
  \seealsoref{是}{shi4}
  \end{phonetics}
\end{entry}

\begin{entry}{系囚}{7,5}{⽷、⼞}
  \begin{phonetics}{系囚}{xi4qiu2}
    \definition{s.}{prisioneiro}
  \end{phonetics}
\end{entry}

\begin{entry}{系列}{7,6}{⽷、⼑}
  \begin{phonetics}{系列}{xi4lie4}[][HSK 4]
    \definition{s.}{série; conjunto; conjunto de coisas relacionadas (matemática)}
  \end{phonetics}
\end{entry}

\begin{entry}{系统}{7,9}{⽷、⽷}
  \begin{phonetics}{系统}{xi4tong3}[][HSK 4]
    \definition{adj.}{sistemático; organizado}
    \definition[个]{s.}{sistema; relação de tipos semelhantes (ou seja, grupo de coisas semelhantes)}
  \end{phonetics}
\end{entry}

\begin{entry}{纯}{7}{⽷}
  \begin{phonetics}{纯}{chun2}[][HSK 4]
    \definition{adj.}{puro; não misturado; livre de impurezas | simples; puro e simples | habilidoso; proficiente; bem versado}
    \definition{adv.}{puramente; completamente; totalmente | genuinamente}
  \end{phonetics}
\end{entry}

\begin{entry}{纯净水}{7,8,4}{⽷、⼎、⽔}
  \begin{phonetics}{纯净水}{chun2 jing4 shui3}[][HSK 4]
    \definition{s.}{água purificada}
  \end{phonetics}
\end{entry}

\begin{entry}{纯真}{7,10}{⽷、⼗}
  \begin{phonetics}{纯真}{chun2zhen1}
    \definition{adj.}{inocente e não afetado | puro e não adulterado}
    \definition{s.}{inocência}
  \end{phonetics}
\end{entry}

\begin{entry}{纵}{7}{⽷}
  \begin{phonetics}{纵}{zong4}
    \definition{adj.}{de norte a sul; geograficamente norte-sul | longitudinal | vertical; horizontal; paralelo ao lado longo do objeto | amassado; com rugas}
    \definition{conj.}{embora; mesmo que}
    \definition{v.}{libertar; deixar ir | entregar-se a; deixar-se levar | pular; saltar}
  \end{phonetics}
\end{entry}

\begin{entry}{纷}{7}{⽷}
  \begin{phonetics}{纷}{fen1}
    \definition[场]{adj.}{confuso; emaranhado; desordenado | muitos e variados; profusos; numerosos}
  \end{phonetics}
\end{entry}

\begin{entry}{纷纷}{7,7}{⽷、⽷}
  \begin{phonetics}{纷纷}{fen1fen1}[][HSK 4]
    \definition{adj.}{numeroso e confuso; muitos e desordenados}
    \definition{adv.}{um após o outro; em sucessão; em rápida sucessão}
  \end{phonetics}
\end{entry}

\begin{entry}{纸}{7}{⽷}
  \begin{phonetics}{纸}{zhi3}[][HSK 2]
    \definition{clas.}{usado para documentos, cartas, etc.}
    \definition[张,沓]{s.}{papel; uma folha fina de material usada para escrever, pintar, imprimir, embalar, etc., feita principalmente de fibras vegetais | papel joss; papel de incenso; refere-se especificamente a itens supersticiosos, como papel-moeda}
  \end{phonetics}
\end{entry}

\begin{entry}{纸巾}{7,3}{⽷、⼱}
  \begin{phonetics}{纸巾}{zhi3jin1}
    \definition[张,包]{s.}{lenço | guardanapo | papel toalha}
  \end{phonetics}
\end{entry}

\begin{entry}{纸币}{7,4}{⽷、⼱}
  \begin{phonetics}{纸币}{zhi3bi4}
    \definition[张]{s.}{nota (dinheiro) | cédula}
  \end{phonetics}
\end{entry}

\begin{entry}{纸尿裤}{7,7,12}{⽷、⼫、⾐}
  \begin{phonetics}{纸尿裤}{zhi3niao4ku4}
    \definition{s.}{fralda descartável}
  \end{phonetics}
\end{entry}

\begin{entry}{纸张}{7,7}{⽷、⼸}
  \begin{phonetics}{纸张}{zhi3zhang1}
    \definition{s.}{papel}
  \end{phonetics}
\end{entry}

\begin{entry}{纸烟}{7,10}{⽷、⽕}
  \begin{phonetics}{纸烟}{zhi3yan1}
    \definition{s.}{cigarro}
  \end{phonetics}
\end{entry}

\begin{entry}{纹路}{7,13}{⽷、⾜}
  \begin{phonetics}{纹路}{wen2lu4}
    \definition{s.}{padrão de linhas | rugas | veias | veias (em mármore ou impressão digital) | grãos (em madeira, etc.)}
  \end{phonetics}
\end{entry}

\begin{entry}{肚}{7}{⾁}
  \begin{phonetics}{肚}{du3}
    \definition{s.}{tripas; entranhas}
  \end{phonetics}
  \begin{phonetics}{肚}{du4}
    \definition{s.}{barriga; abdômen; estômago | tolerância}
  \end{phonetics}
\end{entry}

\begin{entry}{肚子}{7,3}{⾁、⼦}
  \begin{phonetics}{肚子}{du4zi5}[][HSK 4]
    \definition[个,只]{s.}{abdômen; barriguinha; ventre; barriga}
  \end{phonetics}
\end{entry}

\begin{entry}{肠}{7}{⾁}
  \begin{phonetics}{肠}{chang2}[][HSK 5]
    \definition[根,段,片]{s.}{intestinos; parte do sistema digestivo | salsicha; linguiça; alimentos de tripas recheadas com carne, peixe, etc. | sentimentos; emoções; humor}
  \end{phonetics}
\end{entry}

\begin{entry}{良心}{7,4}{⾉、⼼}
  \begin{phonetics}{良心}{liang2xin1}
    \definition{s.}{consciência}
  \end{phonetics}
\end{entry}

\begin{entry}{良田}{7,5}{⾉、⽥}
  \begin{phonetics}{良田}{liang2tian2}
    \definition{s.}{terra agrícola boa | terra fértil}
  \end{phonetics}
\end{entry}

\begin{entry}{良好}{7,6}{⾉、⼥}
  \begin{phonetics}{良好}{liang2hao3}[][HSK 4]
    \definition{adj.}{bom; ótimo; bem}
  \end{phonetics}
\end{entry}

\begin{entry}{芥}{7}{⾋}
  \begin{phonetics}{芥}{gai4}
    \definition{s.}{mostarda}
  \seealsoref{芥蓝}{gai4lan2}
  \end{phonetics}
  \begin{phonetics}{芥}{jie4}
    \definition{s.}{mostarda}
  \end{phonetics}
\end{entry}

\begin{entry}{芥兰}{7,5}{⾋、⼋}
  \begin{phonetics}{芥兰}{gai4lan2}
    \variantof{芥蓝}
  \end{phonetics}
  \begin{phonetics}{芥兰}{jie4lan2}
    \definition{s.}{couve}
  \end{phonetics}
\end{entry}

\begin{entry}{芥蓝}{7,13}{⾋、⾋}
  \begin{phonetics}{芥蓝}{gai4lan2}
    \definition{s.}{brócolis chinês | couve chinesa | mostarda}
  \seealsoref{格兰菜}{ge2lan2cai4}
  \end{phonetics}
\end{entry}

\begin{entry}{芦笋}{7,10}{⾋、⽵}
  \begin{phonetics}{芦笋}{lu2sun3}
    \definition{s.}{aspargos}
  \end{phonetics}
\end{entry}

\begin{entry}{芯片}{7,4}{⾋、⽚}
  \begin{phonetics}{芯片}{xin1pian4}
    \definition{s.}{chip de computador | microchip}
  \end{phonetics}
\end{entry}

\begin{entry}{花}{7}{⾋}
  \begin{phonetics}{花}{hua1}[][HSK 1,2,4]
    \definition*{s.}{sobrenome Hua}
    \definition{adj.}{multicolorido; colorido | embaçado; obscuro; deslumbrado e confuso | extravagante; florido; vistoso}
    \definition[朵,支,束,把,盆,簇]{s.}{flor; órgãos de reprodução sexual de plantas com sementes | flor; planta ornamental |  qualquer coisa que se assemelhe a uma flor | fogos de artifício | padrão; design; design decorativo | flor; metáfora para a essência de uma causa | prostituta; cortesã; referindo-se a prostitutas ou a assuntos relacionados a prostitutas | algodão | varíola | ferimento; ferida; lesões traumáticas sofridas em combate}
    \definition{v.}{gastar; despender; consumir}
  \end{phonetics}
\end{entry}

\begin{entry}{花儿}{7,2}{⾋、⼉}
  \begin{phonetics}{花儿}{hua1r5}
    \definition[朵,支,束,把,盆,簇]{s.}{flor}
  \end{phonetics}
\end{entry}

\begin{entry}{花生}{7,5}{⾋、⽣}
  \begin{phonetics}{花生}{hua1sheng1}
    \definition[粒]{s.}{amendoim}
  \end{phonetics}
\end{entry}

\begin{entry}{花园}{7,7}{⾋、⼞}
  \begin{phonetics}{花园}{hua1 yuan2}[][HSK 2]
    \definition[个,座]{s.}{jardim; um local onde se plantam flores e árvores para passear e descansar}
  \end{phonetics}
\end{entry}

\begin{entry}{花店}{7,8}{⾋、⼴}
  \begin{phonetics}{花店}{hua1dian4}
    \definition{s.}{floricultura}
  \end{phonetics}
\end{entry}

\begin{entry}{花茶}{7,9}{⾋、⾋}
  \begin{phonetics}{花茶}{hua1cha2}
    \definition[杯,壶]{s.}{chá perfumado}
  \end{phonetics}
\end{entry}

\begin{entry}{花样游泳}{7,10,12,8}{⾋、⽊、⽔、⽔}
  \begin{phonetics}{花样游泳}{hua1yang4you2yong3}
    \definition{s.}{nado sincronizado}
  \end{phonetics}
\end{entry}

\begin{entry}{花椰菜}{7,12,11}{⾋、⽊、⾋}
  \begin{phonetics}{花椰菜}{hua1ye1cai4}
    \definition{s.}{couve-flor}
  \end{phonetics}
\end{entry}

\begin{entry}{芹菜}{7,11}{⾋、⾋}
  \begin{phonetics}{芹菜}{qin2cai4}
    \definition{s.}{salsão}
  \end{phonetics}
\end{entry}

\begin{entry}{苏格兰}{7,10,5}{⾋、⽊、⼋}
  \begin{phonetics}{苏格兰}{su1ge2lan2}
    \definition*{s.}{Escócia}
  \end{phonetics}
\end{entry}

\begin{entry}{补}{7}{⾐}
  \begin{phonetics}{补}{bu3}[][HSK 3]
    \definition*{s.}{sobrenome Bu}
    \definition{s.}{ajuda; uso; benefício; utilidade}
    \definition{v.}{reparar; consertar; remendar; adicionar materiais, consertar coisas quebradas | abastecer; encher; repor; adicionar suplemento; complementar; completar; preencher | nutrir}
  \end{phonetics}
\end{entry}

\begin{entry}{补充}{7,6}{⾐、⼉}
  \begin{phonetics}{补充}{bu3chong1}[][HSK 3]
    \definition{adj.}{adicional | suplementar}
    \definition{v.}{reabastecer; suplementar; complementar; aumentar uma parte quando houver insuficiência ou perda}
  \end{phonetics}
\end{entry}

\begin{entry}{补贴}{7,9}{⾐、⾙}
  \begin{phonetics}{补贴}{bu3tie1}[][HSK 5]
    \definition[笔,项,种,份]{s.}{subsídio; ajuda de custo; custos de indenização ou assistência concedida a empresas ou indivíduos pelo estado ou governo}
    \definition{v.}{subsidiar; compensar a falta de dinheiro ou coisas; refere-se principalmente à compensação financeira ou ajuda dada pelo estado ou governo a empresas ou indivíduos}
  \end{phonetics}
\end{entry}

\begin{entry}{补偿}{7,11}{⾐、⼈}
  \begin{phonetics}{补偿}{bu3chang2}[][HSK 5]
    \definition{v.}{compensar (perda, consumo); compensar (deficiências, diferenças)}
  \end{phonetics}
\end{entry}

\begin{entry}{角}{7}{⾓}
  \begin{phonetics}{角}{jiao3}[][HSK 2]
    \definition*{s.}{Jiao, uma das mansões lunares}
    \definition{clas.}{uma unidade monetária fracionária na China (=1/10 de um yuan ou 10 fen)}
    \definition[个,只,对]{s.}{chifre; o objeto duro que cresce na cabeça de bovinos, ovinos, veados, etc. | buzina; corneta; instrumentos musicais tocados no exército antigo | algo com a forma de um chifre | cabo; promontório; península | esquina; canto; a junção entre duas arestas de um objeto | ângulo}
  \end{phonetics}
  \begin{phonetics}{角}{jue2}
    \definition*{s.}{sobrenome Jue}
    \definition{s.}{papel (teatro)}
    \definition{v.}{competir}
  \end{phonetics}
\end{entry}

\begin{entry}{角色}{7,6}{⾓、⾊}
  \begin{phonetics}{角色}{jue2se4}[][HSK 4]
    \definition{s.}{papel; personagem em uma peça; personagem representado por um ator | papel; função; parte}
  \end{phonetics}
\end{entry}

\begin{entry}{角度}{7,9}{⾓、⼴}
  \begin{phonetics}{角度}{jiao3du4}[][HSK 2]
    \definition[个,种]{s.}{perspectiva; ponto de vista; o ponto de partida para ver as coisas | ângulo; o tamanho do ângulo; normalmente expresso em graus ou radianos}
  \end{phonetics}
\end{entry}

\begin{entry}{言论}{7,6}{⾔、⾔}
  \begin{phonetics}{言论}{yan2lun4}
    \definition{s.}{expressão de opinião |  visualizações | comentários | argumentos}
  \end{phonetics}
\end{entry}

\begin{entry}{言语}{7,9}{⾔、⾔}
  \begin{phonetics}{言语}{yan2 yu3}[][HSK 5]
    \definition{s.}{verbal; fala; linguagem falada; conversa; palavras}
  \end{phonetics}
\end{entry}

\begin{entry}{证}{7}{⾔}
  \begin{phonetics}{证}{zheng4}[][HSK 3]
    \definition{s.}{evidência; prova; testemunho | certificado; cartão | evidência; testemunha | doença; enfermidade}
    \definition{v.}{provar; demonstrar | verificar}
  \end{phonetics}
\end{entry}

\begin{entry}{证书}{7,4}{⾔、⼄}
  \begin{phonetics}{证书}{zheng4shu1}[][HSK 5]
    \definition[张,份,些]{s.}{certificado; documentos emitidos por instituições, grupos, etc., que comprovem experiência, nível, honras, poderes, etc.}
  \end{phonetics}
\end{entry}

\begin{entry}{证件}{7,6}{⾔、⼈}
  \begin{phonetics}{证件}{zheng4jian4}[][HSK 3]
    \definition[个,本,张,份]{s.}{documentos; credenciais; certificado; documentos que comprovem a identidade, experiência, etc., tais como carteira de estudante, carteira de trabalho, diploma de graduação, etc.}
  \end{phonetics}
\end{entry}

\begin{entry}{证实}{7,8}{⾔、⼧}
  \begin{phonetics}{证实}{zheng4shi2}[][HSK 5]
    \definition{v.}{verificar; afirmar; confirmar; corroborar; demonstrar; autenticar; provar que é verdadeiro}
  \end{phonetics}
\end{entry}

\begin{entry}{证明}{7,8}{⾔、⽇}
  \begin{phonetics}{证明}{zheng4ming2}[][HSK 3]
    \definition[个,份]{s.}{certificado; atestado; identificação; certificado ou carta de referência; documentos que comprovem identidade, experiência, etc., tais como carteira de estudante, carteira de trabalho, diploma de graduação, etc.}
    \definition{v.}{provar; testemunhar; sustentar; usar materiais confiáveis para demonstrar ou determinar a autenticidade de pessoas ou coisas}
  \end{phonetics}
\end{entry}

\begin{entry}{证据}{7,11}{⾔、⼿}
  \begin{phonetics}{证据}{zheng4ju4}[][HSK 3]
    \definition{s.}{prova; evidência; testemunho; fatos ou materiais que comprovam a veracidade de algo}
  \end{phonetics}
\end{entry}

\begin{entry}{评价}{7,6}{⾔、⼈}
  \begin{phonetics}{评价}{ping2jia4}[][HSK 3]
    \definition[个,项,条,份]{s.}{avaliação; apreciação; comentários ou opiniões de pessoas sobre alguém ou algo}
    \definition{v.}{estimar valor; avaliar valor}
  \end{phonetics}
\end{entry}

\begin{entry}{评论}{7,6}{⾔、⾔}
  \begin{phonetics}{评论}{ping2lun4}[][HSK 5]
    \definition[篇]{s.}{revisão; comentário; artigos ou comentários críticos}
    \definition{v.}{discutir; comentar sobre algo ou alguém}
  \end{phonetics}
\end{entry}

\begin{entry}{评估}{7,7}{⾔、⼈}
  \begin{phonetics}{评估}{ping2gu1}[][HSK 5]
    \definition{v.}{estimar; avaliar; apreciar; avaliar e estimar (coisas abstratas)}
  \end{phonetics}
\end{entry}

\begin{entry}{诅咒}{7,8}{⾔、⼝}
  \begin{phonetics}{诅咒}{zu3zhou4}
    \definition{v.}{amaldiçoar}
  \end{phonetics}
\end{entry}

\begin{entry}{诊断}{7,11}{⾔、⽄}
  \begin{phonetics}{诊断}{zhen3duan4}[][HSK 5]
    \definition{s.}{diagnóstico; diacrisis}
    \definition{v.}{diagnosticar; após examinar os sintomas do paciente, determinar a doença e seu desenvolvimento}
  \end{phonetics}
\end{entry}

\begin{entry}{词}{7}{⾔}
  \begin{phonetics}{词}{ci2}[][HSK 2]
    \definition[个,组,句,段,首]{s.}{palavra; termo; antigamente, referia-se a palavras vazias; atualmente, refere-se a palavras com forma fonética fixa e significado específico na língua; a menor unidade que pode ser usada de forma independente | discurso; declaração; linguagem; texto | ci (um tipo de poesia clássica chinesa, originária da dinastia Tang e plenamente desenvolvida na dinastia Song); gênero poético escrito de acordo com uma estrutura fixa, com versos de comprimentos variados | palavras; redação; refere-se genericamente ao teatro; a parte da letra cantada em harmonia com a melodia em canções e certas artes vocais}
  \end{phonetics}
\end{entry}

\begin{entry}{词汇}{7,5}{⾔、⽔}
  \begin{phonetics}{词汇}{ci2hui4}[][HSK 4]
    \definition[个,组,批,串,堆]{s.}{vocabulário; termo geral para palavras usadas em um idioma}
  \end{phonetics}
\end{entry}

\begin{entry}{词典}{7,8}{⾔、⼋}
  \begin{phonetics}{词典}{ci2dian3}[][HSK 2]
    \definition[本,部]{s.}{dicionário, livro de referência que reúne palavras e explicações para consulta}
  \seealsoref{字典}{zi4 dian3}
  \end{phonetics}
\end{entry}

\begin{entry}{词语}{7,9}{⾔、⾔}
  \begin{phonetics}{词语}{ci2yu3}[][HSK 2]
    \definition[个,租]{s.}{termo; palavra; expressão; conjunto de palavras e frases}
  \end{phonetics}
\end{entry}

\begin{entry}{谷}{7}{⾕}[Kangxi 150]
  \begin{phonetics}{谷}{gu3}
    \definition*{s.}{sobrenome Gu}
    \definition{adj.}{bom; gentil}
    \definition{s.}{vale; ravina; desfiladeiro; garganta; faixa estreita de terra com uma saída no meio de duas colinas ou dois platôs | arroz não descascado | salário de funcionário (na época feudal) | calha; cocho; canal | fossa sob o cerebelo (anatomia); valécula | dificuldade; dilema}
    \definition{v.}{criar (filhos) | crescer}
  \end{phonetics}
\end{entry}

\begin{entry}{豆}{7}{⾖}
  \begin{phonetics}{豆}{dou4}
    \definition*{s.}{sobrenome Dou}
    \definition{s.}{planta que produz vagens ou suas sementes | coisa em forma de feijão | leguminosas ou sementes de leguminosas; feijões; ervilhas | uma xícara ou tigela antiga com haste}
  \end{phonetics}
\end{entry}

\begin{entry}{豆角}{7,7}{⾖、⾓}
  \begin{phonetics}{豆角}{dou4jiao3}
    \definition{s.}{feijão verde}
  \end{phonetics}
\end{entry}

\begin{entry}{豆制品}{7,8,9}{⾖、⼑、⼝}
  \begin{phonetics}{豆制品}{dou4 zhi4 pin3}[][HSK 5]
    \definition{s.}{produtos de soja}
  \end{phonetics}
\end{entry}

\begin{entry}{豆荚}{7,9}{⾖、⾋}
  \begin{phonetics}{豆荚}{dou4jia2}
    \definition{s.}{vagem (de legumes)}
  \end{phonetics}
\end{entry}

\begin{entry}{豆腐}{7,14}{⾖、⾁}
  \begin{phonetics}{豆腐}{dou4fu5}[][HSK 4]
    \definition[块,盒,斤,盘,锅]{s.}{\emph{tofu}}
  \end{phonetics}
\end{entry}

\begin{entry}{财}{7}{⾙}
  \begin{phonetics}{财}{cai2}
    \definition[笔]{s.}{riqueza; dinheiro; fortuna | propriedade; objetos de valor; um termo geral para dinheiro e materiais}
  \end{phonetics}
\end{entry}

\begin{entry}{财产}{7,6}{⾙、⼇}
  \begin{phonetics}{财产}{cai2chan3}[][HSK 4]
    \definition{s.}{ativos; propriedade; pertences; refere-se à posse de riqueza material, como dinheiro, bens, casas, terras, etc.}
  \end{phonetics}
\end{entry}

\begin{entry}{财富}{7,12}{⾙、⼧}
  \begin{phonetics}{财富}{cai2fu4}[][HSK 4]
    \definition{s.}{riqueza; fortuna}
  \end{phonetics}
\end{entry}

\begin{entry}{赤}{7}{⾚}[Kangxi 155]
  \begin{phonetics}{赤}{chi4}
    \definition*{s.}{sobrenome Chi}
    \definition{adj.}{vermelho | um tipo de vermelho um pouco mais claro que o vermelhão | (história)  revolucionário; comunista | leal; sincero | nú; exposto}
    \definition{s.}{ouro puro}
  \end{phonetics}
\end{entry}

\begin{entry}{走}{7}{⾛}[Kangxi 156]
  \begin{phonetics}{走}{zou3}[][HSK 1]
    \definition{v.}{andar; caminhar | correr | mover; movimentar; deslocar | sair; partir; ir embora | visitar; fazer uma visita; (entre amigos e familiares) troca de visitas | passar por; atravessar; ultrapassar | vazar; revelar; divulgar | afastar-se do original; alterar ou perder a forma, o sabor, a cor, etc. originais}
  \end{phonetics}
\end{entry}

\begin{entry}{走开}{7,4}{⾛、⼶}
  \begin{phonetics}{走开}{zou3 kai1}[][HSK 2]
    \definition{v.}{ir embora; fugir; ir para outro lugar}
  \end{phonetics}
\end{entry}

\begin{entry}{走去}{7,5}{⾛、⼛}
  \begin{phonetics}{走去}{zou3qu4}
    \definition{v.}{caminhar até (para)}
  \end{phonetics}
\end{entry}

\begin{entry}{走过}{7,6}{⾛、⾡}
  \begin{phonetics}{走过}{zou3 guo4}[][HSK 2]
    \definition{v.}{passar por; perambular}
  \end{phonetics}
\end{entry}

\begin{entry}{走秀}{7,7}{⾛、⽲}
  \begin{phonetics}{走秀}{zou3xiu4}
    \definition{s.}{desfile de moda}
    \definition{v.}{andar na passarela (em um desfile de moda)}
  \end{phonetics}
\end{entry}

\begin{entry}{走进}{7,7}{⾛、⾡}
  \begin{phonetics}{走进}{zou3 jin4}[][HSK 2]
    \definition{v.}{entrar}
  \end{phonetics}
\end{entry}

\begin{entry}{走势}{7,8}{⾛、⼒}
  \begin{phonetics}{走势}{zou3shi4}
    \definition{s.}{caminho | tendência}
  \end{phonetics}
\end{entry}

\begin{entry}{走卒}{7,8}{⾛、⼗}
  \begin{phonetics}{走卒}{zou3zu2}
    \definition{s.}{lacaio (masculino) | peão (isto é, soldado de infantaria) | servo}
  \end{phonetics}
\end{entry}

\begin{entry}{走鬼}{7,9}{⾛、⿁}
  \begin{phonetics}{走鬼}{zou3gui3}
    \definition{s.}{vendedor ambulante sem licença}
  \end{phonetics}
\end{entry}

\begin{entry}{走索}{7,10}{⾛、⽷}
  \begin{phonetics}{走索}{zou3suo3}
    \definition{v.}{andar na corda bamba}
  \seealsoref{走绳}{zou3sheng2}
  \end{phonetics}
\end{entry}

\begin{entry}{走绳}{7,11}{⾛、⽷}
  \begin{phonetics}{走绳}{zou3sheng2}
    \definition{v.}{andar na corda bamba}
  \seealsoref{走索}{zou3suo3}
  \end{phonetics}
\end{entry}

\begin{entry}{走路}{7,13}{⾛、⾜}
  \begin{phonetics}{走路}{zou3 lu4}[][HSK 1]
    \definition{v.}{caminhar; ir a pé; andar em pé sobre a terra | sair; ir embora; partir}
  \end{phonetics}
\end{entry}

\begin{entry}{足}{7}{⾜}[Kangxi 157]
  \begin{phonetics}{足}{ju4}
    \definition{adj.}{excessivo}
  \end{phonetics}
  \begin{phonetics}{足}{zu2}
    \definition{adj.}{amplo}
    \definition{s.}{pé}
    \definition{v.}{ser suficiente}
  \end{phonetics}
\end{entry}

\begin{entry}{足月}{7,4}{⾜、⽉}
  \begin{phonetics}{足月}{zu2yue4}
    \definition{s.}{gestação completa}
  \end{phonetics}
\end{entry}

\begin{entry}{足足}{7,7}{⾜、⾜}
  \begin{phonetics}{足足}{zu2zu2}
    \definition{adv.}{tanto quanto | extremamente | completamente | não menos que}
  \end{phonetics}
\end{entry}

\begin{entry}{足够}{7,11}{⾜、⼣}
  \begin{phonetics}{足够}{zu2 gou4}[][HSK 3]
    \definition{adj.}{bastante; amplo; suficiente; atingir o nível adequado ou capaz de satisfazer as necessidades}
    \definition{v.}{satisfazer; ser suficiente; estar a contento}
  \end{phonetics}
\end{entry}

\begin{entry}{足球}{7,11}{⾜、⽟}
  \begin{phonetics}{足球}{zu2qiu2}[][HSK 3]
    \definition[个,只,颗,袋]{s.}{futebol | bola de futebol}
  \end{phonetics}
\end{entry}

\begin{entry}{足球队}{7,11,4}{⾜、⽟、⾩}
  \begin{phonetics}{足球队}{zu2qiu2dui4}
    \definition{s.}{time de futebol}
  \end{phonetics}
\end{entry}

\begin{entry}{足球协会}{7,11,6,6}{⾜、⽟、⼗、⼈}
  \begin{phonetics}{足球协会}{zu2qiu2xie2hui4}
    \definition*{s.}{Associação de Futebol}
  \end{phonetics}
\end{entry}

\begin{entry}{足球场}{7,11,6}{⾜、⽟、⼟}
  \begin{phonetics}{足球场}{zu2qiu2chang3}
    \definition{s.}{campo de futebol}
  \end{phonetics}
\end{entry}

\begin{entry}{足球迷}{7,11,9}{⾜、⽟、⾡}
  \begin{phonetics}{足球迷}{zu2qiu2mi2}
    \definition{s.}{fã (ou entusiasta) de futebol}
  \end{phonetics}
\end{entry}

\begin{entry}{足球赛}{7,11,14}{⾜、⽟、⾙}
  \begin{phonetics}{足球赛}{zu2qiu2sai4}
    \definition{s.}{competição de futebol | partida de futebol}
  \end{phonetics}
\end{entry}

\begin{entry}{身上}{7,3}{⾝、⼀}
  \begin{phonetics}{身上}{shen1 shang5}[][HSK 1]
    \definition{s.}{no corpo de alguém | em um;  com um}
  \end{phonetics}
\end{entry}

\begin{entry}{身亡}{7,3}{⾝、⼇}
  \begin{phonetics}{身亡}{shen1wang2}
    \definition{v.}{morrer}
  \end{phonetics}
\end{entry}

\begin{entry}{身边}{7,5}{⾝、⾡}
  \begin{phonetics}{身边}{shen1 bian1}[][HSK 2]
    \definition{adv.}{ao redor; ao lado de alguém; perto do corpo | carregar consigo (transportar); à mão}
  \end{phonetics}
\end{entry}

\begin{entry}{身份}{7,6}{⾝、⼈}
  \begin{phonetics}{身份}{shen1fen4}[][HSK 4]
    \definition[种]{s.}{status; capacidade; identidade; refere-se à origem, ao status e às qualificações de uma pessoa | dignidade; posição honrada; referência especial ao status respeitável}
  \end{phonetics}
\end{entry}

\begin{entry}{身份证}{7,6,7}{⾝、⼈、⾔}
  \begin{phonetics}{身份证}{shen1 fen4 zheng4}[][HSK 3]
    \definition[张]{s.}{ID; bilhete de identidade; carteira de identidade}
  \end{phonetics}
\end{entry}

\begin{entry}{身体}{7,7}{⾝、⼈}
  \begin{phonetics}{身体}{shen1ti3}[][HSK 1]
    \definition[具,个]{s.}{corpo | saúde; saúde das pessoas}
  \end{phonetics}
\end{entry}

\begin{entry}{身体乳}{7,7,8}{⾝、⼈、⼄}
  \begin{phonetics}{身体乳}{shen1ti3 ru3}
    \definition{s.}{loção corporal}
  \end{phonetics}
\end{entry}

\begin{entry}{身体能力}{7,7,10,2}{⾝、⼈、⾁、⼒}
  \begin{phonetics}{身体能力}{shen1ti3 neng2li4}
    \definition{s.}{habilidade física}
  \end{phonetics}
\end{entry}

\begin{entry}{身材}{7,7}{⾝、⽊}
  \begin{phonetics}{身材}{shen1cai2}[][HSK 4]
    \definition[副,种,个,具]{s.}{figura; estatura; altura e peso corporal}
  \end{phonetics}
\end{entry}

\begin{entry}{身高}{7,10}{⾝、⾼}
  \begin{phonetics}{身高}{shen1 gao1}[][HSK 4]
    \definition[个,种,段]{s.}{estatura; altura (de uma pessoa);}
  \end{phonetics}
\end{entry}

\begin{entry}{辛苦}{7,8}{⾟、⾋}
  \begin{phonetics}{辛苦}{xin1ku3}[][HSK 5]
    \definition{adj.}{difícil; trabalhoso; árduo; descreve muito trabalho, alta intensidade e pouco descanso}
    \definition{s.}{dificuldades}
    \definition{v.}{trabalhar duro; passar por grandes dificuldades; passar por dificuldades}
  \end{phonetics}
\end{entry}

\begin{entry}{迎接}{7,11}{⾡、⼿}
  \begin{phonetics}{迎接}{ying2jie1}[][HSK 3]
    \definition{v.}{conhecer; cumprimentar; felicitar; dar as boas-vindas | cumprimentar; felicitar; dar as boas-vindas; preparar-se; aguardar a chegada de um determinado momento ou evento}
  \end{phonetics}
\end{entry}

\begin{entry}{运}{7}{⾡}
  \begin{phonetics}{运}{yun4}[][HSK 5]
    \definition*{s.}{sobrenome Yun}
    \definition{s.}{sorte; destino; fortuna}
    \definition{v.}{mover; deslocar | transportar; levar | usar; empunhar; utilizar}
  \end{phonetics}
\end{entry}

\begin{entry}{运气}{7,4}{⾡、⽓}
  \begin{phonetics}{运气}{yun4qi5}[][HSK 4]
    \definition{adj.}{sortudo; afortunado}
    \definition{s.}{sorte; fortuna}
  \end{phonetics}
\end{entry}

\begin{entry}{运用}{7,5}{⾡、⽤}
  \begin{phonetics}{运用}{yun4yong4}[][HSK 4]
    \definition{v.}{usar; utilizar; manejar; aplicar; explorar as coisas de acordo com suas características}
  \end{phonetics}
\end{entry}

\begin{entry}{运动}{7,6}{⾡、⼒}
  \begin{phonetics}{运动}{yun4dong4}[][HSK 2]
    \definition[项,种,场,次]{s.}{esportes; atletismo; exercício; atividades esportivas | movimento; campanha (política); atividades de massa organizadas, intencionais e de alto nível na política, cultura, produção, etc. | movimento; refere-se a todas as mudanças}
    \definition{v.}{exercitar; fazer atividade física | mover-se; refere-se à mudança na posição de um objeto}
  \end{phonetics}
\end{entry}

\begin{entry}{运动会}{7,6,6}{⾡、⼒、⼈}
  \begin{phonetics}{运动会}{yun4 dong4 hui4}[][HSK 4]
    \definition[个]{s.}{jogos; encontro esportivo; dia de esportes; reunião atlética}
  \end{phonetics}
\end{entry}

\begin{entry}{运动场}{7,6,6}{⾡、⼒、⼟}
  \begin{phonetics}{运动场}{yun4dong4chang3}
    \definition{s.}{campo desportivo | campo de jogos}
  \end{phonetics}
\end{entry}

\begin{entry}{运动员}{7,6,7}{⾡、⼒、⼝}
  \begin{phonetics}{运动员}{yun4 dong4 yuan2}[][HSK 4]
    \definition[名,个]{s.}{jogador; atleta; esportista; pessoas que participam de competições esportivas}
  \end{phonetics}
\end{entry}

\begin{entry}{运动学}{7,6,8}{⾡、⼒、⼦}
  \begin{phonetics}{运动学}{yun4dong4xue2}
    \definition{s.}{cinemática}
  \end{phonetics}
\end{entry}

\begin{entry}{运动服}{7,6,8}{⾡、⼒、⽉}
  \begin{phonetics}{运动服}{yun4dong4fu2}
    \definition{s.}{roupa para prática de esporte}
  \end{phonetics}
\end{entry}

\begin{entry}{运动衫}{7,6,8}{⾡、⼒、⾐}
  \begin{phonetics}{运动衫}{yun4dong4shan1}
    \definition[件]{s.}{moletom | camisa esportiva}
  \end{phonetics}
\end{entry}

\begin{entry}{运动家}{7,6,10}{⾡、⼒、⼧}
  \begin{phonetics}{运动家}{yun4dong4jia1}
    \definition{s.}{ativista | atleta | esportista}
  \end{phonetics}
\end{entry}

\begin{entry}{运动病}{7,6,10}{⾡、⼒、⽧}
  \begin{phonetics}{运动病}{yun4dong4bing4}
    \definition{s.}{enjôo (movimento, carro, etc.)}
  \end{phonetics}
\end{entry}

\begin{entry}{运动鞋}{7,6,15}{⾡、⼒、⾰}
  \begin{phonetics}{运动鞋}{yun4dong4xie2}
    \definition{s.}{tênis | sapatos esportivos}
  \end{phonetics}
\end{entry}

\begin{entry}{运行}{7,6}{⾡、⾏}
  \begin{phonetics}{运行}{yun4xing2}[][HSK 5]
    \definition{v.}{(corpos celestes, etc.) mover-se ao longo do curso | (figurativo) funcionar, estar em operação | (serviço de trem, etc.) operar | (computador) executar um programa}
  \end{phonetics}
\end{entry}

\begin{entry}{运河}{7,8}{⾡、⽔}
  \begin{phonetics}{运河}{yun4he2}
    \definition{s.}{canal (em um rio)}
  \end{phonetics}
\end{entry}

\begin{entry}{运输}{7,13}{⾡、⾞}
  \begin{phonetics}{运输}{yun4shu1}[][HSK 3]
    \definition{v.}{enviar; transportar; transportar pessoas ou coisas de um lugar para outro usando carros, barcos, aviões, etc.}
  \end{phonetics}
\end{entry}

\begin{entry}{近}{7}{⾡}
  \begin{phonetics}{近}{jin4}[][HSK 2]
    \definition{adj.}{próximo; perto; distância espacial ou temporal curta (oposto de 远) | íntimo; intimamente relacionado; relação estreita | fácil de entender}
  \seealsoref{远}{yuan3}
  \end{phonetics}
\end{entry}

\begin{entry}{近代}{7,5}{⾡、⼈}
  \begin{phonetics}{近代}{jin4dai4}[][HSK 4]
    \definition{s.}{tempos modernos; era passada relativamente próxima à era moderna, geralmente referida na história chinesa como 1840 a 1919 | na história mundial, geralmente se refere à era capitalista}
  \end{phonetics}
\end{entry}

\begin{entry}{近来}{7,7}{⾡、⽊}
  \begin{phonetics}{近来}{jin4lai2}[][HSK 5]
    \definition{adv.}{ultimamente; recentemente; de ​​tarde; refere-se a um período de tempo entre o passado imediato e o presente}
  \end{phonetics}
\end{entry}

\begin{entry}{近期}{7,12}{⾡、⽉}
  \begin{phonetics}{近期}{jin4 qi1}[][HSK 3]
    \definition{adv.}{num futuro próximo; brevemente}
  \end{phonetics}
\end{entry}

\begin{entry}{返}{7}{⾡}
  \begin{phonetics}{返}{fan3}
    \definition{v.}{retornar; vir ou voltar}
  \end{phonetics}
\end{entry}

\begin{entry}{返回}{7,6}{⾡、⼞}
  \begin{phonetics}{返回}{fan3 hui2}[][HSK 5]
    \definition{v.}{retornar; ir (voltar); reverter; recorrer; retroceder; voltar para (o lugar original)}
  \end{phonetics}
\end{entry}

\begin{entry}{还}{7}{⾡}
  \begin{phonetics}{还}{hai2}[][HSK 1]
    \definition{adv.}{ainda; indica que a ação ou estado permanece inalterado, equivalente a 仍然 | também; além disso; em adição; indica que há um aumento ou suplemento além do escopo já indicado | ainda mais; usado com 比 para indicar que as características e o grau das coisas comparadas aumentaram, o que é equivalente a 更加 razoavelmente; medianamente; usado antes de um adjetivo, indica que algo atinge apenas o nível mínimo exigido | mesmo; usado na primeira parte da frase como complemento, e na segunda parte como conclusão, equivalente a 尚且 | que expressa realização ou descoberta; expressa surpresa por algo que não se esperava, mas que acabou acontecendo | tão cedo quanto; por um curto período de tempo; indica que já era assim há muito tempo | para dar ênfase; para reforçar o tom}
  \seealsoref{比}{bi3}
  \seealsoref{更加}{geng4 jia1}
  \seealsoref{仍然}{reng2ran2}
  \seealsoref{尚且}{shang4qie3}
  \end{phonetics}
  \begin{phonetics}{还}{huan2}[][HSK 1]
    \definition*{s.}{sobrenome Huan}
    \definition{v.}{voltar; retornar; voltar ao lugar original ou restaurar o estado original | retribuir; devolver; reembolsar; devolver o dinheiro ou os bens emprestados ao seu proprietário | dar ou fazer algo em troca; retribuir as ações dos outros}
  \end{phonetics}
\end{entry}

\begin{entry}{还有}{7,6}{⾡、⽉}
  \begin{phonetics}{还有}{hai2 you3}[][HSK 1]
    \definition{adv.}{também; ainda; além disso; então novamente; enfatizar as partes complementares, excedentes ou não mencionadas além do que já é conhecido}
  \end{phonetics}
\end{entry}

\begin{entry}{还是}{7,9}{⾡、⽇}
  \begin{phonetics}{还是}{hai2shi5}[][HSK 1]
    \definition{adv.}{ainda; ainda assim; não é a continuação de um determinado estado, fenômeno ou ação; o resultado é o mesmo de antes, sem mudanças  |que expressa uma preferência por uma alternativa; expressa comparação ou escolha feita após consideração cuidadosa, frequentemente usado para fazer sugestões | que expressa realização ou descoberta; indica que o resultado final foi inesperado}
    \definition{conj.}{ou (somente para frases interrogativas); indica várias opções, geralmente usado em perguntas | tudo; se; não importa; independentemente de; significa que, independentemente das mudanças que ocorram, o resultado permanecerá o mesmo}
  \end{phonetics}
\end{entry}

\begin{entry}{这}{7}{⾡}
  \begin{phonetics}{这}{zhe4}[][HSK 1]
    \definition{pron.}{este, isto; substitui pessoas ou coisas que estão mais próximas | agora; em vez de 这时候, tem o efeito de reforçar a ênfase}
  \seealsoref{这时候}{zhe4 shi2 hou5}
  \end{phonetics}
  \begin{phonetics}{这}{zhei4}
    \definition{pron.}{(coloquial) este}
  \end{phonetics}
\end{entry}

\begin{entry}{这儿}{7,2}{⾡、⼉}
  \begin{phonetics}{这儿}{zhe4r5}[][HSK 1]
    \definition{pron.}{aqui | agora; neste momento (utilizado apenas após 打, 从, 由)}
  \seealsoref{从}{cong2}
  \seealsoref{打}{da3}
  \seealsoref{由}{you2}
  \end{phonetics}
\end{entry}

\begin{entry}{这个}{7,3}{⾡、⼈}
  \begin{phonetics}{这个}{zhe4ge5}
    \definition{pron.}{isto; este | isso; em vez das coisas mencionadas anteriormente | assim; tal; usado antes de verbos e adjetivos, indica um grau muito profundo, com um sentido exagerado | usado junto com 那个 para indicar pessoas ou objetos indefinidos}
  \seealsoref{那个}{na4ge5}
  \end{phonetics}
\end{entry}

\begin{entry}{这么}{7,3}{⾡、⼃}
  \begin{phonetics}{这么}{zhe4 me5}[][HSK 2]
    \definition{pron.}{tal (usado para mostrar o grau) | então (usado para mostrar exagero e exclamação) | desta forma; assim; formas de expressar ações | tal; indica quantidade}
  \end{phonetics}
\end{entry}

\begin{entry}{这边}{7,5}{⾡、⾡}
  \begin{phonetics}{这边}{zhe4 bian1}[][HSK 1]
    \definition{pron.}{aqui; deste lado; refere-se a um lugar próximo}
  \end{phonetics}
\end{entry}

\begin{entry}{这会儿}{7,6,2}{⾡、⼈、⼉}
  \begin{phonetics}{这会儿}{zhe4 hui4r5}
    \definition{adv./pron./s.}{agora; no momento; no presente}
  \end{phonetics}
\end{entry}

\begin{entry}{这时}{7,7}{⾡、⽇}
  \begin{phonetics}{这时}{zhe4 shi2}[][HSK 2]
    \definition{adv.}{neste momento}
  \end{phonetics}
\end{entry}

\begin{entry}{这时候}{7,7,10}{⾡、⽇、⼈}
  \begin{phonetics}{这时候}{zhe4 shi2 hou5}[][HSK 2]
    \definition{adv.}{neste momento}
  \end{phonetics}
\end{entry}

\begin{entry}{这里}{7,7}{⾡、⾥}
  \begin{phonetics}{这里}{zhe4 li3}[][HSK 1]
    \definition{pron.}{aqui; pronomes demonstrativo, indicando locais próximos}
  \end{phonetics}
\end{entry}

\begin{entry}{这些}{7,8}{⾡、⼆}
  \begin{phonetics}{这些}{zhe4 xie1}[][HSK 1]
    \definition{pron.}{estes; pronome demonstrativo, que indicam duas ou mais pessoas ou coisas que estão próximas}
  \end{phonetics}
\end{entry}

\begin{entry}{这咱}{7,9}{⾡、⼝}
  \begin{phonetics}{这咱}{zhe4 zan5}
    \definition{s.}{agora; no momento; no presente | neste momento}
  \end{phonetics}
\end{entry}

\begin{entry}{这样}{7,10}{⾡、⽊}
  \begin{phonetics}{这样}{zhe4 yang4}[][HSK 2]
    \definition{pron.}{assim; tal; assim; deste jeito; pronome demonstrativo, que indica a natureza, estado, maneira, grau, etc.}
  \end{phonetics}
\end{entry}

\begin{entry}{这麽}{7,14}{⾡、⿇}
  \begin{phonetics}{这麽}{zhe4 me5}
    \variantof{这么}
  \end{phonetics}
\end{entry}

\begin{entry}{进}{7}{⾡}
  \begin{phonetics}{进}{jin4}[][HSK 1]
    \definition*{s.}{sobrenome Jin}
    \definition{clas.}{para seções em um edifício ou complexo residencial; qualquer uma das várias fileiras de casas em um complexo residencial de estilo antigo}
    \definition{s.}{(matemática) base de um sistema numérico}
    \definition{v.}{avançar; ir adiante; seguir em frente; (oposto a 退) | entrar; entrar em; entrar ou sair; (oposto a 出) | receber | comer; tomar; beber | submeter; apresentar | marcar um gol}
    \definition{v.aux.}{usado após um verbo, significa ``para dentro''}
  \seealsoref{出}{chu1}
  \seealsoref{退}{tui4}
  \end{phonetics}
\end{entry}

\begin{entry}{进一步}{7,1,7}{⾡、⼀、⽌}
  \begin{phonetics}{进一步}{jin4 yi2 bu4}[][HSK 3]
    \definition{adv.}{mais; dar um passo adiante; avançar um passo; indica que as coisas estão progredindo em um nível mais alto do que antes}
  \end{phonetics}
\end{entry}

\begin{entry}{进入}{7,2}{⾡、⼊}
  \begin{phonetics}{进入}{jin4 ru4}[][HSK 2]
    \definition{v.}{entrar; entrar em}
  \end{phonetics}
\end{entry}

\begin{entry}{进口}{7,3}{⾡、⼝}
  \begin{phonetics}{进口}{jin4kou3}[][HSK 4]
    \definition{adj.}{importado}
    \definition{s.}{importação; entrada de um edifício ou local, também chamada de 入口}
    \definition{v.+compl.}{importar; comprar ou transportar mercadorias de outro país ou região | entrar no porto; navegar em direção a um porto}
  \seealsoref{入口}{ru4kou3}
  \end{phonetics}
\end{entry}

\begin{entry}{进化}{7,4}{⾡、⼔}
  \begin{phonetics}{进化}{jin4hua4}[][HSK 5]
    \definition[个]{s.}{evolução; os organismos se desenvolvem e evoluem do simples para o complexo e de níveis baixos para altos}
    \definition{v.}{evoluir; um termo geral usado para descrever uma mudança gradual para melhor}
  \end{phonetics}
\end{entry}

\begin{entry}{进出口}{7,5,3}{⾡、⼐、⼝}
  \begin{phonetics}{进出口}{jin4chu1kou3}
    \definition{s.}{importação e exportação}
    \definition{v.}{importar e exportar}
  \end{phonetics}
\end{entry}

\begin{entry}{进去}{7,5}{⾡、⼛}
  \begin{phonetics}{进去}{jin4 qu4}[][HSK 1]
    \definition{v.}{entrar (a partir da minha localização)}
    \definition{v.aux.}{usado depois de um verbo, significa ``ir para dentro''; para um determinado intervalo ou período de tempo}
  \end{phonetics}
\end{entry}

\begin{entry}{进行}{7,6}{⾡、⾏}
  \begin{phonetics}{进行}{jin4xing2}[][HSK 2]
    \definition{v.}{continuar; estar em andamento; estar em progresso | fazer; conduzir; realizar; executar | marchar; avançar; prosseguir; estar em marcha}
  \end{phonetics}
\end{entry}

\begin{entry}{进行编程}{7,6,12,12}{⾡、⾏、⽷、⽲}
  \begin{phonetics}{进行编程}{jin4xing2bian1cheng2}
    \definition{s.}{programa de computador executável}
  \end{phonetics}
\end{entry}

\begin{entry}{进来}{7,7}{⾡、⽊}
  \begin{phonetics}{进来}{jin4 lai2}[][HSK 1]
    \definition{v.}{entrar (para a minha localização)}
  \end{phonetics}
\end{entry}

\begin{entry}{进步}{7,7}{⾡、⽌}
  \begin{phonetics}{进步}{jin4bu4}[][HSK 3]
    \definition{adj.}{progressivo; adequado às tendências da época; que impulsiona o desenvolvimento social (em oposição a 落后)}
    \definition{v.}{avançar; progredir; melhorar}
  \seealsoref{落后}{luo4hou4}
  \end{phonetics}
\end{entry}

\begin{entry}{进展}{7,10}{⾡、⼫}
  \begin{phonetics}{进展}{jin4zhan3}[][HSK 3]
    \definition{v.}{fazer progresso; progredir; avançar no desenvolvimento}
  \end{phonetics}
\end{entry}

\begin{entry}{远}{7}{⾡}
  \begin{phonetics}{远}{yuan3}[][HSK 1]
    \definition*{s.}{sobrenome Yuan}
    \definition{adj.}{distante (no tempo ou no espaço); longe; remoto; Longa distância espacial ou temporal (em oposição a 近) | (relações de parentesco) distante | com grande diferença}
    \definition{v.}{manter-se afastado de; não se aproximar}
  \seealsoref{近}{jin4}
  \end{phonetics}
\end{entry}

\begin{entry}{远天}{7,4}{⾡、⼤}
  \begin{phonetics}{远天}{yuan3tian1}
    \definition{s.}{paraíso | o céu distante}
  \end{phonetics}
\end{entry}

\begin{entry}{远方}{7,4}{⾡、⽅}
  \begin{phonetics}{远方}{yuan3fang1}
    \definition{s.}{longe | um local distante}
  \end{phonetics}
\end{entry}

\begin{entry}{远处}{7,5}{⾡、⼡}
  \begin{phonetics}{远处}{yuan3 chu4}[][HSK 5]
    \definition{s.}{distância; lugar distante}
  \end{phonetics}
\end{entry}

\begin{entry}{远远}{7,7}{⾡、⾡}
  \begin{phonetics}{远远}{yuan3yuan3}
    \definition{adv.}{de longe}
  \end{phonetics}
\end{entry}

\begin{entry}{远征}{7,8}{⾡、⼻}
  \begin{phonetics}{远征}{yuan3zheng1}
    \definition{s.}{uma expedição militar | marcha para regiões remotas}
  \end{phonetics}
\end{entry}

\begin{entry}{违反}{7,4}{⾡、⼜}
  \begin{phonetics}{违反}{wei2fan3}[][HSK 5]
    \definition{v.}{violar; transgredir; contrariar; não estar em conformidade (com as regras, regulamentos, etc.)}
  \end{phonetics}
\end{entry}

\begin{entry}{违法}{7,8}{⾡、⽔}
  \begin{phonetics}{违法}{wei2 fa3}[][HSK 5]
    \definition{v.}{ser ilegal; infringir a lei; violar a lei ou os regulamentos}
  \end{phonetics}
\end{entry}

\begin{entry}{违规}{7,8}{⾡、⾒}
  \begin{phonetics}{违规}{wei2 gui1}[][HSK 5]
    \definition{v.}{violar (regras); infringir as regras e regulamentos}
  \end{phonetics}
\end{entry}

\begin{entry}{违宪}{7,9}{⾡、⼧}
  \begin{phonetics}{违宪}{wei2xian4}
    \definition{adj.}{inconstitucional}
  \end{phonetics}
\end{entry}

\begin{entry}{连}{7}{⾡}
  \begin{phonetics}{连}{lian2}[][HSK 3]
    \definition*{s.}{sobrenome Lian}
    \definition{adv.}{em sucessão; um após o outro; repetidamente}
    \definition{prep.}{incluindo; incluido | até mesmo}
    \definition[个]{s.}{companhia; unidades organizacionais das forças armadas}
    \definition{v.}{ligar; juntar; conectar | envolver-se (em problemas); implicar; incriminar | costurar; coser}
  \end{phonetics}
\end{entry}

\begin{entry}{连忙}{7,6}{⾡、⼼}
  \begin{phonetics}{连忙}{lian2mang2}[][HSK 3]
    \definition{adv.}{imediatamente; de imediato; com pressa; apressadamente}
  \end{phonetics}
\end{entry}

\begin{entry}{连接}{7,11}{⾡、⼿}
  \begin{phonetics}{连接}{lian2 jie1}[][HSK 5]
    \definition[条]{s.}{conexão}
    \definition{v.}{ligar; unir; relacionar, conectar; anexar}
  \end{phonetics}
\end{entry}

\begin{entry}{连续}{7,11}{⾡、⽷}
  \begin{phonetics}{连续}{lian2xu4}[][HSK 3]
    \definition{adv.}{continuamente; sucessivamente; em uma fileira; um após o outro}
  \end{phonetics}
\end{entry}

\begin{entry}{连续剧}{7,11,10}{⾡、⽷、⼑}
  \begin{phonetics}{连续剧}{lian2 xu4 ju4}[][HSK 3]
    \definition[部,集]{s.}{série; novela; drama dividido em vários episódios, transmitido continuamente pela rádio ou televisão, com enredo contínuo}
  \end{phonetics}
\end{entry}

\begin{entry}{连锁反应}{7,12,4,7}{⾡、⾦、⼜、⼴}
  \begin{phonetics}{连锁反应}{lian2suo3fan3ying4}
    \definition{s.}{reação em cadeia}
  \end{phonetics}
\end{entry}

\begin{entry}{迟}{7}{⾡}
  \begin{phonetics}{迟}{chi2}[][HSK 5]
    \definition*{s.}{sobrenome Chi}
    \definition{adj.}{lento; tardio; demorado | atrasado | lento; obtuso}
  \end{phonetics}
\end{entry}

\begin{entry}{迟到}{7,8}{⾡、⼑}
  \begin{phonetics}{迟到}{chi2dao4}[][HSK 4]
    \definition{v.}{chegar atrasado; atrasar-se}
  \end{phonetics}
\end{entry}

\begin{entry}{邮包}{7,5}{⾢、⼓}
  \begin{phonetics}{邮包}{you2bao1}
    \definition{s.}{encomenda postal}
  \end{phonetics}
\end{entry}

\begin{entry}{邮市}{7,5}{⾢、⼱}
  \begin{phonetics}{邮市}{you2shi4}
    \definition{s.}{mercado postal}
  \end{phonetics}
\end{entry}

\begin{entry}{邮电}{7,5}{⾢、⽥}
  \begin{phonetics}{邮电}{you2dian4}
    \definition*{s.}{Correios e Telecomunicações}
  \end{phonetics}
\end{entry}

\begin{entry}{邮件}{7,6}{⾢、⼈}
  \begin{phonetics}{邮件}{you2 jian4}[][HSK 3]
    \definition[封,份,个,条]{s.}{correspondência; correio; assunto postal; termo que se refere a cartas, encomendas, etc., recebidos, transportados e entregues pelos correios | \emph{e-mail}; refere-se a e-mails, informações recebidas e enviadas através de caixas de correio eletrônico na \emph{Internet}, etc.}
  \end{phonetics}
\end{entry}

\begin{entry}{邮局}{7,7}{⾢、⼫}
  \begin{phonetics}{邮局}{you2ju2}[][HSK 4]
    \definition[家]{s.}{correio; agência dos correios; organizações que lidam com serviços postais}
  \end{phonetics}
\end{entry}

\begin{entry}{邮费}{7,9}{⾢、⾙}
  \begin{phonetics}{邮费}{you2fei4}
    \definition{s.}{postagem}
    \definition{v.}{postar}
  \end{phonetics}
\end{entry}

\begin{entry}{邮迷}{7,9}{⾢、⾡}
  \begin{phonetics}{邮迷}{you2mi2}
    \definition{s.}{filatelista | colecionador de selos}
  \end{phonetics}
\end{entry}

\begin{entry}{邮资}{7,10}{⾢、⾙}
  \begin{phonetics}{邮资}{you2zi1}
    \definition{s.}{postagem}
  \end{phonetics}
\end{entry}

\begin{entry}{邮递}{7,10}{⾢、⾡}
  \begin{phonetics}{邮递}{you2di4}
    \definition{v.}{enviar por correio}
  \end{phonetics}
\end{entry}

\begin{entry}{邮票}{7,11}{⾢、⽰}
  \begin{phonetics}{邮票}{you2 piao4}[][HSK 3]
    \definition[枚,张,套,版]{s.}{selo; selo postal; comprovante vendido pelos correios, usado para colar nas correspondências para indicar que o porte foi pago}
  \end{phonetics}
\end{entry}

\begin{entry}{邮箱}{7,15}{⾢、⾋}
  \begin{phonetics}{邮箱}{you2 xiang1}[][HSK 3]
    \definition{s.}{caixa de correio | \emph{mailbox}; refere-se ao endereço de \emph{e-mail}}
  \end{phonetics}
\end{entry}

\begin{entry}{邻居}{7,8}{⾢、⼫}
  \begin{phonetics}{邻居}{lin2ju1}[][HSK 5]
    \definition[个,位,家]{s.}{vizinho; pessoas ou famílias que moram muito perto}
  \end{phonetics}
\end{entry}

\begin{entry}{里}{7}{⾥}[Kangxi 166]
  \begin{phonetics}{里}{li3}[][HSK 1]
    \definition*{s.}{sobrenome Li}
    \definition{clas.}{li, uma unidade chinesa de comprimento (= 1/2 quilômetro)}
    \definition{s.}{forro; revestimento; interior; parte de trás do tecido | interno; dentro; no interior | vizinhança; vizinhos | cidade natal; local de origem}
  \end{phonetics}
\end{entry}

\begin{entry}{里头}{7,5}{⾥、⼤}
  \begin{phonetics}{里头}{li3 tou5}[][HSK 2]
    \definition{s.}{dentro}
  \end{phonetics}
\end{entry}

\begin{entry}{里边}{7,5}{⾥、⾡}
  \begin{phonetics}{里边}{li3 bian5}[][HSK 1]
    \definition{s.}{em; dentro; no interior}
  \end{phonetics}
\end{entry}

\begin{entry}{里面}{7,9}{⾥、⾯}
  \begin{phonetics}{里面}{li3 mian4}[][HSK 3]
    \definition{s.}{dentro; interior}
  \end{phonetics}
\end{entry}

\begin{entry}{里斯本}{7,12,5}{⾥、⽄、⽊}
  \begin{phonetics}{里斯本}{li3si1ben3}
    \definition*{s.}{Lisboa}
  \end{phonetics}
\end{entry}

\begin{entry}{里斯本大学}{7,12,5,3,8}{⾥、⽄、⽊、⼤、⼦}
  \begin{phonetics}{里斯本大学}{li3si1ben3 da4xue2}
    \definition*{s.}{Universidade de Lisboa}
  \end{phonetics}
\end{entry}

\begin{entry}{针}{7}{⾦}
  \begin{phonetics}{针}{zhen1}[][HSK 4]
    \definition*{s.}{sobrenome Zhen}
    \definition[根]{s.}{agulha; ferramentas para costura de roupas | objetos semelhantes a agulhas; algo longo e fino como uma agulha | injeção | ponto de costura | pontos de acupuntura na medicina chinesa}
  \end{phonetics}
\end{entry}

\begin{entry}{针对}{7,5}{⾦、⼨}
  \begin{phonetics}{针对}{zhen1dui4}[][HSK 4]
    \definition{prep.}{em conexão com; de acordo com; à luz de; introdução de objetos de comportamento com uma finalidade clara}
    \definition{v.}{contrariar; apontar para; ter como objetivo; ser direcionado contra; fazer algo especificamente sobre um problema ou uma pessoa}
  \end{phonetics}
\end{entry}

\begin{entry}{闲}{7}{⾨}
  \begin{phonetics}{闲}{xian2}[][HSK 5]
    \definition{adj.}{ocioso; não ocupado; desocupado; sem coisas para fazer; sem atividades; tempo livre | desocupado; (casa, objeto, etc.) não em uso; ocioso | não oficial; não sério; não relacionado ao negócio}
    \definition{s.}{lazer; tempo livre}
  \end{phonetics}
\end{entry}

\begin{entry}{间}{7}{⾨}
  \begin{phonetics}{间}{jian1}[][HSK 1]
    \definition{clas.}{a menor unidade de uma casa; a menor unidade habitacional; cômodo}
    \definition{s.}{espaço entre duas partes  | (em um) tempo ou espaço definido | sala; quarto | uma seção de uma sala ou o espaço lateral entre dois pares de pilares | com um tempo ou espaço definido}
  \end{phonetics}
  \begin{phonetics}{间}{jian4}
    \definition{s.}{espaço entre as duas partes; abertura; lacuna}
    \definition{v.}{separar | semear a discórdia | desbastar (mudas); podar; remover ou arrancar as mudas em excesso}
  \end{phonetics}
\end{entry}

\begin{entry}{间或}{7,8}{⾨、⼽}
  \begin{phonetics}{间或}{jian4huo4}
    \definition{adv.}{às vezes | ocasionalmente | de vez em quando}
  \end{phonetics}
\end{entry}

\begin{entry}{间接}{7,11}{⾨、⼿}
  \begin{phonetics}{间接}{jian4jie1}[][HSK 5]
    \definition{adj.}{indireto; de segunda mão; em oposição a 直接}
  \seealsoref{直接}{zhi2jie1}
  \end{phonetics}
\end{entry}

\begin{entry}{闷热}{7,10}{⾨、⽕}
  \begin{phonetics}{闷热}{men1re4}
    \definition{adj.}{abafado | quente e abafado | sufocantemente quente | quente e sensual}
  \end{phonetics}
\end{entry}

\begin{entry}{阻止}{7,4}{⾩、⽌}
  \begin{phonetics}{阻止}{zu3zhi3}[][HSK 4]
    \definition{v.}{parar; reter; conter; interromper; impedir o avanço; impedir o movimento; obstruir}
  \end{phonetics}
\end{entry}

\begin{entry}{阻击}{7,5}{⾩、⼐}
  \begin{phonetics}{阻击}{zu3ji1}
    \definition{v.}{verificar | parar}
  \end{phonetics}
\end{entry}

\begin{entry}{阻碍}{7,13}{⾩、⽯}
  \begin{phonetics}{阻碍}{zu3'ai4}[][HSK 5]
    \definition{s.}{obstáculo; impedimento; barreira}
    \definition{v.}{bloquear; impedir; obstruir; impedir o bom andamento ou desenvolvimento}
  \end{phonetics}
\end{entry}

\begin{entry}{阿}{7}{⾩}
  \begin{phonetics}{阿}{a1}
    \definition{pref.}{em dialetos do sul para formar termos carinhosos, antes de nomes de animais de estimação, sobrenomes monossilábicos ou números que denotam ordem de antiguidade em uma; anexado a 大, 二, 三,\dots\ para indicar classificação (e, às vezes, intimidade) | antes dos termos de parentesco; na frente de um sobrenome, de um nome próprio ou de um determinado título, com uma conotação de intimidade | em alguns contextos, pode soar infantil ou muito informal (por exemplo, chamar um colega de trabalho por ``阿 + Nome'' sem intimidade)}[阿妈___mamãe | 阿明 ___forma carinhosa de chamar alguém chamado Ming]
  \end{phonetics}
  \begin{phonetics}{阿}{e1}
    \definition*{s.}{sobrenome E}
    \definition*{s.}{Dong'e (um condado na província de Shandong)}
    \definition{s.}{grande monte (ou colina) | um lugar sinuoso (montanha, água, etc.)}
    \definition{v.}{bajular; satisfazer}
  \end{phonetics}
\end{entry}

\begin{entry}{阿姨}{7,9}{⾩、⼥}
  \begin{phonetics}{阿姨}{a1yi2}[][HSK 4]
    \definition[个,位]{s.}{tia; uma forma de tratamento para uma mulher da geração dos pais; dirigir-se a uma mulher que tem aproximadamente a mesma idade da sua mãe, geralmente não é parente | babá em uma família; professora em um jardim de infância | tia; irmã da mãe (mais comum no sul da China)}[阿姨,生日快乐!___Tia, feliz aniversário! | 阿姨,这个苹果多少钱一斤?___Tia/Senhora, quanto custa o quilo dessas maçãs? | 阿姨,我想喝水。___Tia/Babá, eu quero beber água.]
  \end{phonetics}
\end{entry}

\begin{entry}{阿哥}{7,10}{⾩、⼝}
  \begin{phonetics}{阿哥}{a1ge1}
    \definition{s.}{irmão mais velho (afetivo)}[阿哥,帮我拿一下书包!___Irmão, ajude-me com minha mochila escolar!]
  \end{phonetics}
\end{entry}

\begin{entry}{附}{7}{⾩}
  \begin{phonetics}{附}{fu4}
    \definition*{s.}{sobrenome Fu}
    \definition{v.}{adicionar; anexar; incluir | chegar perto de; estar perto de | depender de; confiar em; cumprir com | concordar com; anexar a; aderir a; cumprir com; depender de}
  \end{phonetics}
\end{entry}

\begin{entry}{附件}{7,6}{⾩、⼈}
  \begin{phonetics}{附件}{fu4jian4}[][HSK 5]
    \definition*{s.}{\emph{Adnexa Uteri} ; refere-se à genitália interna feminina que não seja o útero, as trompas de falópio e os ovários}
    \definition{s.}{apêndice; documentos que acompanham o documento principal | acessório; anexo; peças ou sobressalentes que não sejam peças principais de máquinas e equipamentos | anexo; documentos ou itens relevantes emitidos com o documento}
  \end{phonetics}
\end{entry}

\begin{entry}{附近}{7,7}{⾩、⾡}
  \begin{phonetics}{附近}{fu4jin4}[][HSK 4]
    \definition{adj.}{perto; vizinho}
    \definition{s.}{vizinhança; bairro}
  \end{phonetics}
\end{entry}

\begin{entry}{陆地}{7,6}{⾩、⼟}
  \begin{phonetics}{陆地}{lu4di4}[][HSK 4]
    \definition[块,片]{s.}{terra; terra seca (em oposição ao mar); superfície da Terra, excluindo os oceanos (e, às vezes, rios e lagos)}
  \end{phonetics}
\end{entry}

\begin{entry}{陆续}{7,11}{⾩、⽷}
  \begin{phonetics}{陆续}{lu4xu4}[][HSK 4]
    \definition{adv.}{sucessivamente; um após o outro; intermitentemente}
  \end{phonetics}
\end{entry}

\begin{entry}{陆路}{7,13}{⾩、⾜}
  \begin{phonetics}{陆路}{lu4lu4}
    \definition{s.}{rota terrestre}
  \end{phonetics}
\end{entry}

\begin{entry}{韧}{7}{⾱}
  \begin{phonetics}{韧}{ren4}
    \definition{adj.}{flexível, mas forte; tenaz; resistente (oposto a 脆) | resistente; macio e forte, não quebra facilmente (ao contrário de 脆)}
  \seealsoref{脆}{cui4}
  \end{phonetics}
\end{entry}

\begin{entry}{饭}{7}{⾷}
  \begin{phonetics}{饭}{fan4}[][HSK 1]
    \definition{s.}{(empréstimo linguístico) fã, devoto}
    \definition[顿,份,碗,口,锅]{s.}{cereais cozidos; grãos cozidos | refeição; alimentos consumidos diariamente em horários regulares | trabalho; meio de subsistência; meio de vida}
  \end{phonetics}
\end{entry}

\begin{entry}{饭店}{7,8}{⾷、⼴}
  \begin{phonetics}{饭店}{fan4dian4}[][HSK 1]
    \definition[家,个]{s.}{restaurante | hotel; hotel grande e bem equipado}
  \end{phonetics}
\end{entry}

\begin{entry}{饭馆}{7,11}{⾷、⾷}
  \begin{phonetics}{饭馆}{fan4 guan3}[][HSK 2]
    \definition[家,个]{s.}{restaurante; lanchonete}
  \end{phonetics}
\end{entry}

\begin{entry}{饮食}{7,9}{⾷、⾷}
  \begin{phonetics}{饮食}{yin3shi2}[][HSK 5]
    \definition{s.}{dieta; alimentos e bebidas}
    \definition{v.}{comer; beber}
  \end{phonetics}
\end{entry}

\begin{entry}{饮料}{7,10}{⾷、⽃}
  \begin{phonetics}{饮料}{yin3liao4}[][HSK 5]
    \definition[杯,瓶,种]{s.}{bebida; drinque; líquidos processados e fabricados para consumo, como vinho, chá, refrigerantes, suco de laranja, etc.}
  \end{phonetics}
\end{entry}

\begin{entry}{驱}{7}{⾺}
  \begin{phonetics}{驱}{qu1}
    \definition{v.}{expulsar | repelir}
  \end{phonetics}
\end{entry}

\begin{entry}{驴}{7}{⾺}
  \begin{phonetics}{驴}{lv2}
    \definition[头]{s.}{burro | asno | jumento | jegue}
  \end{phonetics}
\end{entry}

\begin{entry}{鸡}{7}{⿃}
  \begin{phonetics}{鸡}{ji1}[][HSK 2]
    \definition*{s.}{sobrenome Ji}
    \definition[只]{s.}{galo, galinha, frango | palavra ofensiva para uma mulher que ganha dinheiro fazendo sexo com um homem}
  \end{phonetics}
\end{entry}

\begin{entry}{鸡蛋}{7,11}{⿃、⾍}
  \begin{phonetics}{鸡蛋}{ji1dan4}[][HSK 1]
    \definition[个,枚,筐,箱,打]{s.}{ovo de galinha}
  \end{phonetics}
\end{entry}

\begin{entry}{麦当劳}{7,6,7}{⿆、⼹、⼒}
  \begin{phonetics}{麦当劳}{mai4dang1lao2}
    \definition*{s.}{McDonald's (empresa de \emph{fast-food})}
  \seealsoref{麦当劳叔叔}{mai4dang1lao2 shu1shu5}
  \end{phonetics}
\end{entry}

\begin{entry}{麦当劳叔叔}{7,6,7,8,8}{⿆、⼹、⼒、⼜、⼜}
  \begin{phonetics}{麦当劳叔叔}{mai4dang1lao2 shu1shu5}
    \definition*{s.}{Ronald McDonald}
  \seealsoref{麦当劳}{mai4dang1lao2}
  \end{phonetics}
\end{entry}

\begin{entry}{麦淇淋}{7,11,11}{⿆、⽔、⽔}
  \begin{phonetics}{麦淇淋}{mai4qi2lin2}
    \definition{s.}{(empréstimo linguístico) margarina}
  \end{phonetics}
\end{entry}

\begin{entry}{龟}{7}{⿔}
  \begin{phonetics}{龟}{gui1}
    \definition[只]{s.}{tartaruga; cágado}
  \end{phonetics}
\end{entry}

\begin{entry}{龟速}{7,10}{⿔、⾡}
  \begin{phonetics}{龟速}{gui1su4}
    \definition{adv.}{tão lento quanto uma tartaruga}
  \end{phonetics}
\end{entry}

%%%%% EOF %%%%%


%%%
%%% 8画
%%%

\section*{8画}\addcontentsline{toc}{section}{8画}

\begin{Entry}{丧}{8}{⼗}
  \begin{Phonetics}{丧}{sang1}
    \definition{adj.}{decepcionado; deprimido; desanimado}
    \definition{v.}{perder | desanimar; frustrar}
  \end{Phonetics}
  \begin{Phonetics}{丧}{sang4}
    \definition{adj.}{decepcionado | desanimado}
    \definition{v.}{estar enlutado (do cônjuge etc.) | morrer}
  \end{Phonetics}
\end{Entry}

\begin{Entry}{丧失}{8,5}{⼗、⼤}
  \begin{Phonetics}{丧失}{sang4shi1}[][HSK 6]
    \definition{v.}{perder (algo que se tem)}
  \end{Phonetics}
\end{Entry}

\begin{Entry}{丧钟}{8,9}{⼗、⾦}
  \begin{Phonetics}{丧钟}{sang1zhong1}
    \definition{s.}{sentença de morte}
  \end{Phonetics}
\end{Entry}

\begin{Entry}{乖}{8}{⼃}
  \begin{Phonetics}{乖}{guai1}[][HSK 7-9]
    \definition{adj.}{(uma criança) bem comportado; bom; obediente | inteligente; astuto; esperto | (caráter, comportamento, etc.) estranho; anormal; irracional}
    \definition{v.}{perverter; ser contrário à razão; ir contra | (caráter, comportamento, etc.) ser anormal; ser estranho}
  \end{Phonetics}
\end{Entry}

\begin{Entry}{乖巧}{8,5}{⼃、⼯}
  \begin{Phonetics}{乖巧}{guai1qiao3}[][HSK 7-9]
    \definition{adj.}{fofo; adorável; agradável; descreve crianças, pequenos animais, etc. como sendo obedientes, fofos e simpáticos | inteligente; engenhoso; descreve uma pessoa que sempre fala ou faz coisas de acordo com os desejos de outras pessoas e é querida por elas}
  \end{Phonetics}
\end{Entry}

\begin{Entry*}{乖乖}{8,8}{⼃、⼃}
  \begin{Phonetics}{乖乖}{guai1guai1}
    \definition{adj.}{bem-comportado; obediente}
    \definition{s.}{bebezinho; pequenino; querido; docinho (usado apenas para crianças)}
  \end{Phonetics}
  \begin{Phonetics}{乖乖}{guai1guai5}
    \definition{expr.}{Uau!; Nossa!; Meu Deus!; Oh meu Deus!}
  \end{Phonetics}
\end{Entry*}

\begin{Entry}{乳}{8}{⼄}
  \begin{Phonetics}{乳}{ru3}
    \definition{adj.}{recém-nascido (animal); lactente}
    \definition{s.}{mama; peito | leite (em geral) | qualquer líquido semelhante ao leite}
    \definition{v.}{dar à luz}
  \end{Phonetics}
\end{Entry}

\begin{Entry}{乳制品}{8,8,9}{⼄、⼑、⼝}
  \begin{Phonetics}{乳制品}{ru3 zhi4 pin3}[][HSK 6]
    \definition{s.}{produtos lácteos}
  \end{Phonetics}
\end{Entry}

\begin{Entry}{乳房}{8,8}{⼄、⼾}
  \begin{Phonetics}{乳房}{ru3fang2}
    \definition{s.}{seio | mama | úbere}
  \end{Phonetics}
\end{Entry}

\begin{Entry}{事}{8}{⼅}
  \begin{Phonetics}{事}{shi4}[][HSK 1]
    \definition[件,桩,回]{s.}{assunto; questão; coisa; negócio | problema; acidente | emprego; trabalho | responsabilidade; envolvimento | caso, coisa; o que aconteceu}
    \definition{v.}{servir; atender | estar envolvido em; dedicar-se a}
  \end{Phonetics}
\end{Entry}

\begin{Entry}{事儿}{8,2}{⼅、⼉}
  \begin{Phonetics}{事儿}{shi4r5}
    \definition[件,桩]{s.}{o emprego | negócio | afazeres | assunto que precisa ser resolvido | matéria}
  \end{Phonetics}
\end{Entry}

\begin{Entry}{事业}{8,5}{⼅、⼀}
  \begin{Phonetics}{事业}{shi4ye4}[][HSK 3]
    \definition[个]{s.}{causa; carreira; empreendimento; atividades regulares realizadas por pessoas com um determinado objetivo, escala e sistema que têm impacto no desenvolvimento social | instituição; instalações; unidade de trabalho apoiada financeiramente pelo governo; refere-se especificamente a empresas que não têm rendimentos de produção, são financiadas pelo Estado e não realizam contabilidade económica}
  \end{Phonetics}
\end{Entry}

\begin{Entry}{事件}{8,6}{⼅、⼈}
  \begin{Phonetics}{事件}{shi4jian4}[][HSK 3]
    \definition[个,件,次]{s.}{evento; incidente; grandes eventos na história ou na sociedade}
  \end{Phonetics}
\end{Entry}

\begin{Entry}{事先}{8,6}{⼅、⼉}
  \begin{Phonetics}{事先}{shi4xian1}[][HSK 4]
    \definition{adv.}{antes; de antemão; com antecedência; antecipadamente}
  \end{Phonetics}
\end{Entry}

\begin{Entry}{事后}{8,6}{⼅、⼝}
  \begin{Phonetics}{事后}{shi4 hou4}[][HSK 6]
    \definition{s.}{depois; depois do evento; após o incidente ocorrer ou o problema ser resolvido}
  \end{Phonetics}
\end{Entry}

\begin{Entry}{事实}{8,8}{⼅、⼧}
  \begin{Phonetics}{事实}{shi4shi2}[][HSK 3]
    \definition[个,件]{s.}{mito; lenda; uma narrativa sobre alguém ou algo que foi transmitida oralmente}
    \definition{v.}{dizer; contar; ser dito; contar a história}
  \end{Phonetics}
\end{Entry}

\begin{Entry}{事实上}{8,8,3}{⼅、⼧、⼀}
  \begin{Phonetics}{事实上}{shi4 shi2 shang4}[][HSK 3]
    \definition{adv.}{realmente; de fato; na realidade; na verdade; de fato}
  \end{Phonetics}
\end{Entry}

\begin{Entry}{事物}{8,8}{⼅、⽜}
  \begin{Phonetics}{事物}{shi4wu4}[][HSK 4]
    \definition[种,类,个]{s.}{coisa; objeto; todos os objetos e fenômenos que existem objetivamente}
  \end{Phonetics}
\end{Entry}

\begin{Entry}{事故}{8,9}{⼅、⽁}
  \begin{Phonetics}{事故}{shi4gu4}[][HSK 3]
    \definition[起,桩,次,场]{s.}{acidente; perdas ou desastres repentinos, muitas vezes relacionados ao transporte, produção, trabalho e segurança pessoal}
  \end{Phonetics}
\end{Entry}

\begin{Entry}{事情}{8,11}{⼅、⼼}
  \begin{Phonetics}{事情}{shi4qing5}[][HSK 2]
    \definition[件,个,些,种]{s.}{assunto; questão; coisa; negócio | erro; acidente; infortúnio | (coloquial) emprego; trabalho}
  \end{Phonetics}
\end{Entry}

\begin{Entry}{些}{8}{⼆}
  \begin{Phonetics}{些}{xie1}[][HSK 4]
    \definition{adv.}{um pouco; um pouco mais; usado após um adjetivo ou parte de um verbo para indicar uma pequena quantidade, equivalente a 一点儿}
    \definition{clas.}{alguns; um pouco; denota uma quantidade indefinida}
  \seealsoref{一点儿}{yi4dian3r5}
  \end{Phonetics}
\end{Entry}

\begin{Entry}{些许}{8,6}{⼆、⾔}
  \begin{Phonetics}{些许}{xie1xu3}
    \definition{num.}{um pouco}
  \end{Phonetics}
\end{Entry}

\begin{Entry}{享}{8}{⼇}
  \begin{Phonetics}{享}{xiang3}
    \definition{v.}{aproveitar}
  \end{Phonetics}
\end{Entry}

\begin{Entry}{享受}{8,8}{⼇、⼜}
  \begin{Phonetics}{享受}{xiang3shou4}[][HSK 5]
    \definition{v.}{aproveitar; desfrutar; estar satisfeito material ou espiritualmente}
  \end{Phonetics}
\end{Entry}

\begin{Entry}{京}{8}{⼇}
  \begin{Phonetics}{京}{jing1}
    \definition*{s.}{Pequim (Beijing), abreviação de 北京 | Sobrenome Jing}
    \definition{num.}{dez milhões (um numeral antigo); 10.000.000; 1000.0000}
    \definition{s.}{capital de um país}
  \seealsoref{北京}{bei3 jing1}
  \end{Phonetics}
\end{Entry}

\begin{Entry}{京二胡}{8,2,9}{⼇、⼆、⾁}
  \begin{Phonetics}{京二胡}{jing1'er4hu2}
    \definition{s.}{um tipo de violino chinês semelhante ao 二胡 de duas cordas, usado principalmente para acompanhamento do canto da ópera de Pequim | também chamado de 京胡 | jing'erhu, um violino de duas cordas, intermediário em tamanho e tom entre o 京胡 e o 二胡, usado para acompanhar a ópera chinesa}
  \seealsoref{二胡}{er4hu2}
  \seealsoref{京胡}{jing1hu2}
  \end{Phonetics}
\end{Entry}

\begin{Entry}{京胡}{8,9}{⼇、⾁}
  \begin{Phonetics}{京胡}{jing1hu2}
    \definition{s.}{jinghu, um instrumento de arco de duas cordas com registro agudo; violino da ópera de Pequim | também chamado de 京二胡 | jinghu, um 二胡 (violino de duas cordas) menor e mais agudo, usado para acompanhar a ópera chinesa}
  \seealsoref{二胡}{er4hu2}
  \seealsoref{胡琴}{hu2qin2}
  \seealsoref{京二胡}{jing1'er4hu2}
  \end{Phonetics}
\end{Entry}

\begin{Entry}{京剧}{8,10}{⼇、⼑}
  \begin{Phonetics}{京剧}{jing1ju4}[][HSK 3]
    \definition*[场,段]{s.}{Ópera de Pequim}
  \end{Phonetics}
\end{Entry}

\begin{Entry}{佩}{8}{⼈}
  \begin{Phonetics}{佩}{pei4}
    \definition{s.}{um ornamento usado como pingente amarrados em cintos nos tempos antigos}
    \definition{v.}{vestir (na cintura, etc.) | (arcaico) admirar | (arcaico) usar, especialmente uma pistola ou espada, na cintura}
  \end{Phonetics}
\end{Entry}

\begin{Entry}{佩服}{8,8}{⼈、⽉}
  \begin{Phonetics}{佩服}{pei4fu2}
    \definition{v.}{admirar}
  \end{Phonetics}
\end{Entry}

\begin{Entry}{使}{8}{⼈}
  \begin{Phonetics}{使}{shi3}[][HSK 3]
    \definition{conj.}{se; supondo; usado como a primeira cláusula de uma frase complexa; indica uma relação hipotética; equivalente a 假如}
    \definition{s.}{enviado; mensageiro; pessoas em uma missão}
    \definition{v.}{enviar; despachar; dizer a alguém para fazer algo | usar; empregar; aplicar | deixar; chamar; habilitar}
  \seealsoref{假如}{jia3ru2}
  \end{Phonetics}
\end{Entry}

\begin{Entry}{使用}{8,5}{⼈、⽤}
  \begin{Phonetics}{使用}{shi3yong4}[][HSK 2]
    \definition{v.}{usar; empregar; aplicar; fazer com que pessoas, equipamentos, fundos, etc. sirvam a um determinado propósito}
  \end{Phonetics}
\end{Entry}

\begin{Entry}{使劲}{8,7}{⼈、⼒}
  \begin{Phonetics}{使劲}{shi3/jin4}[][HSK 4]
    \definition{v.+compl.}{colocar energia; exercer toda a sua força | esforçar-se para ajudar; colocar energia para ajudar}
  \end{Phonetics}
\end{Entry}

\begin{Entry}{使得}{8,11}{⼈、⼻}
  \begin{Phonetics}{使得}{shi3 de5}[][HSK 5]
    \definition{v.}{ser utilizável; poder ser usado | ser viável; ser exequível; ser possível;  poder fazer | fazer; tornar; causar um determinado resultado (intenção, plano, coisa)}
  \end{Phonetics}
\end{Entry}

\begin{Entry}{例}{8}{⼈}
  \begin{Phonetics}{例}{li4}
    \definition{adj.}{regular; rotineiro}
    \definition{s.}{exemplo; instância | precedente | caso; instância | regras; estatutos; regulamentos}
    \definition{v.}{analogizar}
  \end{Phonetics}
\end{Entry}

\begin{Entry}{例子}{8,3}{⼈、⼦}
  \begin{Phonetics}{例子}{li4 zi5}[][HSK 2]
    \definition[个]{s.}{exemplo; algo usado para ajudar a explicar ou provar uma determinada situação ou afirmação}
  \end{Phonetics}
\end{Entry}

\begin{Entry}{例外}{8,5}{⼈、⼣}
  \begin{Phonetics}{例外}{li4wai4}[][HSK 5]
    \definition[个,种]{s.}{exceção; situações que não se enquadram nas regras gerais ou nas leis comuns}
    \definition{v.}{ser excepcional; ser uma exceção}
  \end{Phonetics}
\end{Entry}

\begin{Entry}{例如}{8,6}{⼈、⼥}
  \begin{Phonetics}{例如}{li4ru2}[][HSK 2]
    \definition{conj.}{por exemplo; tal como; como por exemplo; colocado antes do exemplo, indica que o exemplo vem a seguir}
  \end{Phonetics}
\end{Entry}

\begin{Entry}{供}{8}{⼈}
  \begin{Phonetics}{供}{gong1}[][HSK 7-9]
    \definition*{s.}{Sobrenome Gong}
    \definition{v.}{fornecer; alimentar |  fornecer algo (para uso ou conveniência de); fornecer algumas condições de exploração à outra parte}
  \end{Phonetics}
  \begin{Phonetics}{供}{gong4}
    \definition{s.}{oferendas | confissão}
    \definition{v.}{depositar (oferendas) | confessar}
  \end{Phonetics}
\end{Entry}

\begin{Entry}{供不应求}{8,4,7,7}{⼈、⼀、⼴、⽔}
  \begin{Phonetics}{供不应求}{gong1bu2ying4qiu2}[][HSK 7-9]
    \definition{expr.}{``A oferta fica aquém da demanda.'' ou ``A demanda excede a oferta.''}
  \end{Phonetics}
\end{Entry}

\begin{Entry}{供应}{8,7}{⼈、⼴}
  \begin{Phonetics}{供应}{gong1 ying4}[][HSK 4]
    \definition{v.}{fornecer; prover de}
  \end{Phonetics}
\end{Entry}

\begin{Entry}{供求}{8,7}{⼈、⽔}
  \begin{Phonetics}{供求}{gong1qiu2}[][HSK 7-9]
    \definition{s.}{Economia: oferta e procura (principalmente de commodities)}
  \end{Phonetics}
\end{Entry}

\begin{Entry}{供奉}{8,8}{⼈、⼤}
  \begin{Phonetics}{供奉}{gong4feng4}[][HSK 7-9]
    \definition{s.}{artista servindo ao imperador; uma pessoa que serve ao imperador com alguma habilidade; especialmente um ator que é convocado ao palácio para atuar}
    \definition{v.}{consagrar; consagrar e adorar; colocar incenso e velas em frente aos retratos ou tábuas de deuses, Budas ou ancestrais; colocar oferendas; mostrar respeito | prestar homenagem à corte imperial}
  \end{Phonetics}
\end{Entry}

\begin{Entry}{供给}{8,9}{⼈、⽷}
  \begin{Phonetics}{供给}{gong1ji3}[][HSK 6]
    \definition{s.}{fornecer; prover; fornecer produção e necessidades de vida, dinheiro, etc. para aqueles que precisam}
  \end{Phonetics}
\end{Entry}

\begin{Entry}{供暖}{8,13}{⼈、⽇}
  \begin{Phonetics}{供暖}{gong1nuan3}[][HSK 7-9]
    \definition{s.}{fornecimento de aquecimento}
    \definition{v.}{fornecer aquecimento}
  \end{Phonetics}
\end{Entry}

\begin{Entry}{依}{8}{⼈}
  \begin{Phonetics}{依}{yi1}
    \definition*{s.}{Sobrenome Yi}
    \definition{prep.}{de acordo com; à luz de; julgando por}
    \definition{v.}{depender de; ser dependente de; confiar em | cumprir; ouvir; ceder a | inclinar-se; descansar sobre (ou contra)}
  \end{Phonetics}
\end{Entry}

\begin{Entry}{依旧}{8,5}{⼈、⽇}
  \begin{Phonetics}{依旧}{yi1jiu4}[][HSK 5]
    \definition{adv.}{ainda; como antes; como sempre}
  \end{Phonetics}
\end{Entry}

\begin{Entry}{依次}{8,6}{⼈、⽋}
  \begin{Phonetics}{依次}{yi1 ci4}[][HSK 6]
    \definition{adv.}{sucessivamente; na ordem correta; em ordem}
  \end{Phonetics}
\end{Entry}

\begin{Entry}{依法}{8,8}{⼈、⽔}
  \begin{Phonetics}{依法}{yi1 fa3}[][HSK 5]
    \definition{adv.}{e acordo com regras (ou métodos) fixas | de acordo com a lei; por força da lei; em conformidade com as disposições legais}
  \end{Phonetics}
\end{Entry}

\begin{Entry}{依偎}{8,11}{⼈、⼈}
  \begin{Phonetics}{依偎}{yi1wei1}
    \definition{v.}{aninhar-se | aconchegar-se}
  \end{Phonetics}
\end{Entry}

\begin{Entry}{依据}{8,11}{⼈、⼿}
  \begin{Phonetics}{依据}{yi1ju4}[][HSK 5]
    \definition{prep.}{julgando por; de acordo com; à luz de; com base em; de acordo com; introduzir algo que possa servir como premissa ou base}
    \definition[个]{s.}{base; evidência; fundamento; base para tomar uma decisão ou realizar uma ação}
    \definition{v.}{basear-se em; confiar em; depdender de; usar algo como premissa ou base}
  \end{Phonetics}
\end{Entry}

\begin{Entry}{依然}{8,12}{⼈、⽕}
  \begin{Phonetics}{依然}{yi1ran2}[][HSK 4]
    \definition{adv.}{ainda; como antes}
    \definition{v.}{estar quieto; estar como antes; estar como o original, sem alterações}
  \end{Phonetics}
\end{Entry}

\begin{Entry}{依照}{8,13}{⼈、⽕}
  \begin{Phonetics}{依照}{yi1 zhao4}[][HSK 5]
    \definition{prep.}{de acordo com; à luz de; introduzir certos padrões para os eventos, o que equivale a 按照}
    \definition{v.}{seguir (com base em algo)}
  \seealsoref{按照}{an4zhao4}
  \end{Phonetics}
\end{Entry}

\begin{Entry}{依赖}{8,13}{⼈、⾙}
  \begin{Phonetics}{依赖}{yi1lai4}[][HSK 6]
    \definition{v.}{confiar em; ser dependente de; ser completamente dependente e inseparável | depender de; ser mutuamente dependentes e inseparáveis}
  \end{Phonetics}
\end{Entry}

\begin{Entry}{依靠}{8,15}{⼈、⾮}
  \begin{Phonetics}{依靠}{yi1kao4}[][HSK 4]
    \definition{s.}{apoio; suporte; algo em que se apoiar; alguém ou algo em quem você pode confiar}
    \definition{v.}{depender de; confiar em (alguém ou alguma coisa para atingir um determinado objetivo)}
  \end{Phonetics}
\end{Entry}

\begin{Entry}{侧}{8}{⼈}
  \begin{Phonetics}{侧}{ce4}[][HSK 6]
    \definition*{s.}{Sobrenome Ce}
    \definition{s.}{lado | inclinação}
    \definition{v.}{inclinar; inclinar-se}
  \end{Phonetics}
  \begin{Phonetics}{侧}{zhai1}
    \definition{adj.}{inclinado; torto}
  \end{Phonetics}
\end{Entry}

\begin{Entry}{侧重}{8,9}{⼈、⾥}
  \begin{Phonetics}{侧重}{ce4zhong4}[][HSK 7-9]
    \definition{v.}{enfatizar; focar em (um certo aspecto)}
  \end{Phonetics}
\end{Entry}

\begin{Entry}{侧面}{8,9}{⼈、⾯}
  \begin{Phonetics}{侧面}{ce4mian4}[][HSK 7-9]
    \definition[个]{s.}{lado; flanco | vias indiretas; canais informais; algum aspecto; outro aspecto}
  \end{Phonetics}
\end{Entry}

\begin{Entry}{兔}{8}{⼉}
  \begin{Phonetics}{兔}{tu4}[][HSK 5]
    \definition[只]{s.}{lebre; coelho}
  \end{Phonetics}
\end{Entry}

\begin{Entry}{兔子}{8,3}{⼉、⼦}
  \begin{Phonetics}{兔子}{tu4zi5}
    \definition[只]{s.}{coelho | lebre}
  \end{Phonetics}
\end{Entry}

\begin{Entry}{其}{8}{⼋}
  \begin{Phonetics}{其}{qi2}[][HSK 5]
    \definition*{s.}{Sobrenome Qi}
    \definition{adv.}{fazer uma suposição ou uma réplica | expressar comando, ordem}
    \definition{pron.}{dele (dela, deles, delas) | ele, ela, isso, eles; elas | isso; tal | isso (referindo-se a nenhuma pessoa ou coisa específica)}
    \definition{suf.}{sufixo de palavra, anexado ao advérbio}
  \end{Phonetics}
\end{Entry}

\begin{Entry}{其中}{8,4}{⼋、⼁}
  \begin{Phonetics}{其中}{qi2zhong1}[][HSK 2]
    \definition{pron.}{dentro; entre (os quais, eles, etc.); em (o qual, ele, etc.); nas pessoas ou coisas mencionadas anteriormente}
  \end{Phonetics}
\end{Entry}

\begin{Entry}{其他}{8,5}{⼋、⼈}
  \begin{Phonetics}{其他}{qi2ta1}[][HSK 2]
    \definition{pron.}{outra pessoa/outra coisa | outras coisas; outras pessoas; em substituição de outras pessoas ou coisas}
  \end{Phonetics}
\end{Entry}

\begin{Entry}{其次}{8,6}{⼋、⽋}
  \begin{Phonetics}{其次}{qi2ci4}[][HSK 3]
    \definition{adj.}{secundário}
    \definition{conj.}{próximo; então; em segundo lugar; mais tarde na ordem}
  \end{Phonetics}
\end{Entry}

\begin{Entry}{其余}{8,7}{⼋、⼈}
  \begin{Phonetics}{其余}{qi2yu2}[][HSK 4]
    \definition{pron.}{o resto; os outros; o restante}
  \end{Phonetics}
\end{Entry}

\begin{Entry}{其实}{8,8}{⼋、⼧}
  \begin{Phonetics}{其实}{qi2shi2}[][HSK 3]
    \definition{adv.}{na verdade; na realidade; a primeira parte é a situação aparente, e 其实 é usado para introduzir a situação real}
  \end{Phonetics}
\end{Entry}

\begin{Entry}{具}{8}{⼋}
  \begin{Phonetics}{具}{ju4}
    \definition*{s.}{Sobrenome Ju}
    \definition{clas.}{(literário) usado para caixões, cadáveres e certos objetos}
    \definition{s.}{utensílio; ferramenta; implemento | capacidade; habilidade}
    \definition{v.}{possuir; ter | fornecer; prover | declarar; enumerar}
  \end{Phonetics}
\end{Entry}

\begin{Entry}{具有}{8,6}{⼋、⽉}
  \begin{Phonetics}{具有}{ju4 you3}[][HSK 3]
    \definition{v.}{ter; possuir; ser provido de}
  \end{Phonetics}
\end{Entry}

\begin{Entry}{具体}{8,7}{⼋、⼈}
  \begin{Phonetics}{具体}{ju4ti3}[][HSK 3]
    \definition{adj.}{específico; particular | concreto; específico; mais detalhado; muito detalhado; muito claro | concreto; real; não é abstrato, tem uma forma definida; pode ser visto ou sentido}
    \definition{v.}{incorporar; objetivar; combinar teorias, princípios, padrões, etc. com pessoas ou coisas específicas}
  \end{Phonetics}
\end{Entry}

\begin{Entry}{具备}{8,8}{⼋、⼡}
  \begin{Phonetics}{具备}{ju4bei4}[][HSK 4]
    \definition{v.}{ter; possuir; ser provido de}
  \end{Phonetics}
\end{Entry}

\begin{Entry}{典}{8}{⼋}
  \begin{Phonetics}{典}{dian3}
    \definition{s.}{lei; cânone; padrão; sistema; regulamentos | trabalho padrão de bolsa de estudos; livros que podem servir como padrões ou especificações | alusão; citação literária | cerimônia; uma grande cerimônia (nos tempos antigos, a etiqueta era um dos sistemas importantes do estado) | modelo; normas; regras}
    \definition{v.}{estar no comando de | hipotecar; usar imóveis ou casas como garantia ao pedir dinheiro emprestado}
  \end{Phonetics}
\end{Entry}

\begin{Entry}{典礼}{8,5}{⼋、⽰}
  \begin{Phonetics}{典礼}{dian3li3}[][HSK 5]
    \definition[个,次,场]{s.}{cerimônia; celebração; comemoração}
  \end{Phonetics}
\end{Entry}

\begin{Entry}{典型}{8,9}{⼋、⼟}
  \begin{Phonetics}{典型}{dian3xing2}[][HSK 4]
    \definition{adj.}{típico; representativo}
    \definition[个,种]{s.}{modelo; caso típico; indivíduo ou evento representativo | personagens típicos; personalidades modelo (em obras literárias); personagens na literatura e na arte que refletem a natureza de uma determinada sociedade e têm uma personalidade distinta}
  \end{Phonetics}
\end{Entry}

\begin{Entry}{典范}{8,9}{⼋、⾋}
  \begin{Phonetics}{典范}{dian3fan4}[][HSK 7-9]
    \definition{s.}{modelo; exemplo; paradigma; uma pessoa ou coisa que pode ser usada como padrão para aprendizagem ou emulação}
  \end{Phonetics}
\end{Entry}

\begin{Entry}{净}{8}{⼎}
  \begin{Phonetics}{净}{jing4}[][HSK 6]
    \definition{adj.}{limpo | (depois de um verbo) terminado; sem nada sobrando | líquido | vazio; oco; nu}
    \definition{adv.}{todo; o tempo todo | somente; meramente; nada além de | inteiramente; indica puro e nada mais}
    \definition{s.}{o “rosto pintado”, comumente conhecido como Hualian, 花脸, um tipo de personagem da ópera de Pequim, etc.}
    \definition{v.}{tornar limpo | limpar; lavar; esfregar para limpar}
  \seealsoref{花脸}{hua1lian3}
  \end{Phonetics}
\end{Entry}

\begin{Entry}{凭}{8}{⼏}
  \begin{Phonetics}{凭}{ping2}[][HSK 5]
    \definition{conj.}{não importa (o que, como, etc.); conecta frases complexas condicionais para expressar incondicionalidade, equivalente a 任凭 ou 不论}
    \definition{prep.}{introduzir a ação ou o comportamento com base em algo; quando a frase nominal após 凭 é longa, pode-se adicionar 着 após 凭}
    \definition[张]{s.}{prova; evidência}
    \definition{v.}{apoiar-se; encostar-se | confiar em; depender de | basear-se em; tomar como base}
  \seealsoref{不论}{bu2 lun4}
  \seealsoref{任凭}{ren4 ping2}
  \seealsoref{着}{zhe5}
  \end{Phonetics}
\end{Entry}

\begin{Entry}{函}{8}{⼐}
  \begin{Phonetics}{函}{han2}
    \definition*{s.}{Sobrenome Han}
    \definition[封]{s.}{caixa; envelope; capa | carta}
  \end{Phonetics}
\end{Entry}

\begin{Entry}{函授}{8,11}{⼐、⼿}
  \begin{Phonetics}{函授}{han2shou4}[][HSK 7-9]
    \definition{v.}{ensinar por correspondência; utilizar principalmente tutoria por correspondência para ministrar cursos}
  \end{Phonetics}
\end{Entry}

\begin{Entry}{函数}{8,13}{⼐、⽁}
  \begin{Phonetics}{函数}{han2shu4}
    \definition[个]{s.}{Matemática: função; em um determinado processo, duas variáveis $​​x$ e $y$ têm um certo valor de $y$ correspondente a cada valor de $x$ dentro de um determinado intervalo, $y$ é uma função de $x$; essa relação geralmente é expressa como $y = f(x)$}
  \end{Phonetics}
\end{Entry}

\begin{Entry}{刮}{8}{⼑}
  \begin{Phonetics}{刮}{gua1}[][HSK 6]
    \definition{v.}{barbear; raspar; depilar | untar com (pasta, etc.)  | extorquir; pilhar; adquirir avidamente (propriedade) por vários meios | (do vento) soprar}
  \end{Phonetics}
\end{Entry}

\begin{Entry}{刮风}{8,4}{⼑、⾵}
  \begin{Phonetics}{刮风}{gua1/feng1}[][HSK 7-9]
    \definition{v.+compl.}{ventar; fazer vento; soprar (vento)}
  \end{Phonetics}
\end{Entry}

\begin{Entry}{到}{8}{⼑}
  \begin{Phonetics}{到}{dao4}[][HSK 1]
    \definition*{s.}{Sobrenome Dao}
    \definition{adj.}{atencioso}
    \definition{prep.}{a; até; para; indica o tempo em que a ação ou comportamento foi alcançado}
    \definition{v.}{ir para; partir para | chegar; alcançar; chegar a | como complemento de um verbo para mostrar o resultado de uma ação}
  \end{Phonetics}
\end{Entry}

\begin{Entry}{到处}{8,5}{⼑、⼡}
  \begin{Phonetics}{到处}{dao4chu4}[][HSK 2]
    \definition{adv.}{em todos os lugares; em todos os locais; por toda parte}
  \end{Phonetics}
\end{Entry}

\begin{Entry}{到头来}{8,5,7}{⼑、⼤、⽊}
  \begin{Phonetics}{到头来}{dao4tou2lai2}[][HSK 7-9]
    \definition{adv.}{finalmente; no fim; no final; resultado (usado principalmente em aspectos ruins)}
  \end{Phonetics}
\end{Entry}

\begin{Entry}{到达}{8,6}{⼑、⾡}
  \begin{Phonetics}{到达}{dao4da2}[][HSK 3]
    \definition{v.}{chegar (a um determinado local, a uma determinada fase); alcançar}
  \end{Phonetics}
\end{Entry}

\begin{Entry}{到位}{8,7}{⼑、⼈}
  \begin{Phonetics}{到位}{dao4/wei4}[][HSK 7-9]
    \definition{adj.}{bom; muito preciso; até um nível adequado/padrão/satisfatório}
    \definition{v.+compl.}{estar no lugar/posição; chegar ao local designado; alcançar o local especificado; atender aos requisitos especificados}
  \end{Phonetics}
\end{Entry}

\begin{Entry}{到来}{8,7}{⼑、⽊}
  \begin{Phonetics}{到来}{dao4 lai2}[][HSK 5]
    \definition{v.}{chegar; chegar aqui de outro lugar}
  \end{Phonetics}
\end{Entry}

\begin{Entry}{到底}{8,8}{⼑、⼴}
  \begin{Phonetics}{到底}{dao4di3}[][HSK 3]
    \definition{adv.}{na terra (usado em frases interrogativas para expressar a determinação de alguém em encontrar uma resposta definitiva) | afinal | finalmente; por fim; no fim; indica uma situação que finalmente se concretizou após várias mudanças ou reviravoltas}
  \end{Phonetics}
\end{Entry}

\begin{Entry}{到期}{8,12}{⼑、⽉}
  \begin{Phonetics}{到期}{dao4/qi1}[][HSK 6]
    \definition{v.+compl.}{expirar; amadurecer; tornar-se devido; tornar-se devido}
  \end{Phonetics}
\end{Entry}

\begin{Entry}{制}{8}{⼑}
  \begin{Phonetics}{制}{zhi4}
    \definition[套,项]{s.}{sistema | regras; regulamentos}
    \definition{v.}{formular; elaborar | fazer; fabricar | restringir; limitar; controlar; disciplinar}
  \end{Phonetics}
\end{Entry}

\begin{Entry}{制订}{8,4}{⼑、⾔}
  \begin{Phonetics}{制订}{zhi4 ding4}[][HSK 4]
    \definition{v.}{esboçar; formular; elaborar; mapear}
  \end{Phonetics}
\end{Entry}

\begin{Entry}{制成}{8,6}{⼑、⼽}
  \begin{Phonetics}{制成}{zhi4 cheng2}[][HSK 5]
    \definition{v.}{fabricar; ser feito de; produzir}
  \end{Phonetics}
\end{Entry}

\begin{Entry}{制约}{8,6}{⼑、⽷}
  \begin{Phonetics}{制约}{zhi4yue1}[][HSK 5]
    \definition{v.}{limitar; verificar; restringir; a existência e a mudança de uma coisa determinam a existência e a mudança de outra coisa}
  \end{Phonetics}
\end{Entry}

\begin{Entry}{制作}{8,7}{⼑、⼈}
  \begin{Phonetics}{制作}{zhi4zuo4}[][HSK 3]
    \definition{v.}{fazer; produzir; itens feitos com matérias-primas, geralmente pequenos e feitos à mão | fazer; produzir; criar gráficos, anúncios, filmes, jogos, etc., utilizando texto, imagens, sons, imagens, etc.}
  \end{Phonetics}
\end{Entry}

\begin{Entry}{制定}{8,8}{⼑、⼧}
  \begin{Phonetics}{制定}{zhi4ding4}[][HSK 3]
    \definition{v.}{rascunhar; formular; elaborar; estabelecer (leis, regulamentos, planos, etc.)}
  \end{Phonetics}
\end{Entry}

\begin{Entry}{制度}{8,9}{⼑、⼴}
  \begin{Phonetics}{制度}{zhi4du4}[][HSK 3]
    \definition[项,条,套,种]{s.}{regulamentação; regulamento; procedimentos operacionais ou diretrizes de conduta que todos devem seguir | sistema; o sistema político, econômico e cultural formado sob determinadas condições históricas}
  \end{Phonetics}
\end{Entry}

\begin{Entry}{制造}{8,10}{⼑、⾡}
  \begin{Phonetics}{制造}{zhi4zao4}[][HSK 3]
    \definition{v.}{fazer; produzir; manufaturar; transformar matérias-primas em produtos acabados | criar; agitar; criar artificialmente uma situação ou atmosfera desfavorável}
  \end{Phonetics}
\end{Entry}

\begin{Entry}{制裁}{8,12}{⼑、⾐}
  \begin{Phonetics}{制裁}{zhi4cai2}
    \definition{s.}{punição | sanção (inclusive econômica)}
    \definition{v.}{punir}
  \end{Phonetics}
\end{Entry}

\begin{Entry}{刷}{8}{⼑}
  \begin{Phonetics}{刷}{shua1}[][HSK 4]
    \definition{s.}{escova; pincel | (onomatopéia) farfalhar; descreve o som de uma passagem rápida}
    \definition{v.}{escovar; esfregar; remover com uma escova | borrar; colar; aplicar com um pincel | eliminar; remover; limpar | rolar; navegar; visualizar grandes quantidades de informações muito rapidamente em um curto período de tempo online ou em dispositivos móveis | deslizar (passar o cartão magnético)}
  \end{Phonetics}
  \begin{Phonetics}{刷}{shua4}
    \definition{adj.}{pálido ou branco-azulado}
    \definition{adv.}{bastante; completamente; extremamente; descreve movimentos ágeis}
  \end{Phonetics}
\end{Entry}

\begin{Entry}{刷子}{8,3}{⼑、⼦}
  \begin{Phonetics}{刷子}{shua1zi5}[][HSK 4]
    \definition[把,个]{s.}{escova; escovão; utensílio feito de lã, fio de plástico, fio de metal, etc., para remover sujeira ou aplicar óleo de unção, etc., geralmente longo ou oval, alguns com alças}
  \end{Phonetics}
\end{Entry}

\begin{Entry}{刷牙}{8,4}{⼑、⽛}
  \begin{Phonetics}{刷牙}{shua1ya2}[][HSK 4]
    \definition{s.}{escovar os dentes}
  \end{Phonetics}
\end{Entry}

\begin{Entry}{券}{8}{⼑}
  \begin{Phonetics}{券}{quan4}[][HSK 6]
    \definition[张]{s.}{certificado; bilhete; ingresso; uma conta ou pedaço de papel que serve como recibo}
  \end{Phonetics}
\end{Entry}

\begin{Entry}{刹}{8}{⼑}
  \begin{Phonetics}{刹}{cha4}
    \definition*{s.}{abreviação de Kshatara, 刹多罗, sânscrito ``ksetra''}
    \definition{s.}{mosteiro, templo ou santuário budista}
  \seealsoref{刹多罗}{sha1duo1luo2}
  \end{Phonetics}
  \begin{Phonetics}{刹}{sha1}
    \definition{v.}{acionar o(s) freio(s); frear; brecar}
  \end{Phonetics}
\end{Entry}

\begin{Entry}{刹多罗}{8,6,8}{⼑、⼣、⽹}
  \begin{Phonetics}{刹多罗}{sha1duo1luo2}
    \definition*{s.}{Kshatara, sânscrito ``ksetra''}
  \end{Phonetics}
\end{Entry}

\begin{Entry}{刺}{8}{⼑}
  \begin{Phonetics}{刺}{ci1}
    \definition{s.}{(onomatopéia) som de rasgo, fricção, etc.}
  \end{Phonetics}
  \begin{Phonetics}{刺}{ci4}[][HSK 4]
    \definition*{s.}{Sobrenome Ci}
    \definition{s.}{espinho; farpa; algo afiado como uma agulha | cartão de visita | saliências; projeções pequenas e pontiagudas na superfície de um objeto ou na pele de uma pessoa}
    \definition{v.}{esfaquear; perfurar | irritar; estimular | assassinar | espionar; detectar | criticar}
  \end{Phonetics}
\end{Entry}

\begin{Entry}{刺耳}{8,6}{⼑、⽿}
  \begin{Phonetics}{刺耳}{ci4'er3}[][HSK 7-9]
    \definition{adj.}{irritante (desagradável) para o ouvido; estridente; penetrante; áspero}
  \end{Phonetics}
\end{Entry}

\begin{Entry}{刺骨}{8,9}{⼑、⾻}
  \begin{Phonetics}{刺骨}{ci4gu3}[][HSK 7-9]
    \definition{adj.}{perfurante (até os ossos); cortante}
  \end{Phonetics}
\end{Entry}

\begin{Entry}{刺绣}{8,10}{⼑、⽷}
  \begin{Phonetics}{刺绣}{ci4xiu4}[][HSK 7-9]
    \definition{s.}{bordado; um artesanato popular tradicional que usa fios de seda coloridos para bordar padrões ou imagens em tecidos; produtos bordados}
    \definition{v.}{bordar; bordar padrões ou imagens em tecido usando fios de seda coloridos}
  \end{Phonetics}
\end{Entry}

\begin{Entry}{刺猬}{8,12}{⼑、⽝}
  \begin{Phonetics}{刺猬}{ci4wei5}
    \definition{s.}{porco-espinho | ouriço}
  \end{Phonetics}
\end{Entry}

\begin{Entry}{刺激}{8,16}{⼑、⽔}
  \begin{Phonetics}{刺激}{ci4ji1}[][HSK 4]
    \definition{adj.}{animado; entusiasmado; sensação de empolgação e nervosismo}
    \definition[个]{s.}{estímulo; estimulação; fortes efeitos físicos ou psicológicos}
    \definition{v.}{irritar; provocar; estimular | incentivar; estimular; incitar; (por algum meio) para mudar as coisas para melhor, para fazer coisas positivas}
  \end{Phonetics}
\end{Entry}

\begin{Entry}{刻}{8}{⼑}
  \begin{Phonetics}{刻}{ke4}[][HSK 2,5]
    \definition{adj.}{cruel; severo; rude; indelicado | no mais alto grau}
    \definition{clas.}{um quarto (de uma hora, 15min)}
    \definition[件]{s.}{quarto (de hora); momento}
    \definition{v.}{esculpir; inscrever; gravar; talhar com uma faca (padrões, texto, etc.) | definir um limite de tempo | imprimir (CD)}
  \end{Phonetics}
\end{Entry}

\begin{Entry}{刻画}{8,8}{⼑、⽥}
  \begin{Phonetics}{刻画}{ke4hua4}
    \definition{v.}{retratar | tirar um retrato}
  \end{Phonetics}
\end{Entry}

\begin{Entry}{刻钟}{8,9}{⼑、⾦}
  \begin{Phonetics}{刻钟}{ke4 zhong1}
    \definition{s.}{um quarto de hora}
  \end{Phonetics}
\end{Entry}

\begin{Entry}{势}{8}{⼒}
  \begin{Phonetics}{势}{shi4}
    \definition{s.}{poder; força; influência | momentum; tendência | aparência externa de um objeto natural; fenômenos ou situações naturais | situação; estado de coisas; circunstâncias | sinal; gesto | genitais masculinos}
  \end{Phonetics}
\end{Entry}

\begin{Entry}{势力}{8,2}{⼒、⼒}
  \begin{Phonetics}{势力}{shi4li4}[][HSK 5]
    \definition[股]{s.}{força; poder; influência; forças políticas, econômicas, militares, etc.}
  \end{Phonetics}
\end{Entry}

\begin{Entry}{卑}{8}{⼗}
  \begin{Phonetics}{卑}{bei1}
    \definition{adj.}{Literário: baixo | inferior; médio | Literário: modesto; humilde}
  \end{Phonetics}
\end{Entry}

\begin{Entry}{卑鄙}{8,13}{⼗、⾢}
  \begin{Phonetics}{卑鄙}{bei1bi3}[][HSK 7-9]
    \definition{adj.}{ruim; vulgar; vil; desprezível}
  \end{Phonetics}
\end{Entry}

\begin{Entry}{单}{8}{⼗}
  \begin{Phonetics}{单}{chan2}
    \definition{s.}{usado em 单于 \dpy{chan2yu2}}
  \seealsoref{单于}{chan2yu2}
  \end{Phonetics}
  \begin{Phonetics}{单}{dan1}[][HSK 4]
    \definition*{s.}{Sobrenome Dan}
    \definition{adj.}{sozinho; único | ímpar; número ímpar (oposto a 双) | simples; poucos projetos e tipos; estrutura e ideias simples | fino; fraco; frágil}
    \definition{adv.}{isoladamente; sozinho; indica que uma ação ou coisa está dentro de um escopo limitado e não é combinada com outras; equivale a 只 ou 仅}
    \definition[个]{s.}{lençol; um único pedaço grande de pano usado para cobrir | conta; lista; pedaços de papel para anotações detalhadas (geralmente folhas soltas)}
  \seealsoref{仅}{jin3}
  \seealsoref{双}{shuang1}
  \seealsoref{只}{zhi3}
  \end{Phonetics}
  \begin{Phonetics}{单}{shan4}
    \definition*{s.}{Sobrenome Shan}
    \definition{s.}{material de tecido de largura simples (dupla) | número singular (plural)}
  \end{Phonetics}
\end{Entry}

\begin{Entry}{单一}{8,1}{⼗、⼀}
  \begin{Phonetics}{单一}{dan1 yi1}[][HSK 5]
    \definition{adj.}{único; unitário; exclusivo}
  \end{Phonetics}
\end{Entry}

\begin{Entry}{单于}{8,3}{⼗、⼆}
  \begin{Phonetics}{单于}{chan2yu2}
    \definition{s.}{rei de Xiongnu (匈奴)}
  \seealsoref{匈奴}{xiong1nu2}
  \end{Phonetics}
\end{Entry}

\begin{Entry}{单元}{8,4}{⼗、⼉}
  \begin{Phonetics}{单元}{dan1yuan2}[][HSK 3]
    \definition[个,组,套]{s.}{unidade (de algo); um conjunto completo, com parágrafos e sistemas próprios, que forma uma unidade independente}
  \end{Phonetics}
\end{Entry}

\begin{Entry}{单方面}{8,4,9}{⼗、⽅、⾯}
  \begin{Phonetics}{单方面}{dan1fang1mian4}[][HSK 7-9]
    \definition{adj.}{unilateral}
    \definition{adv.}{unilateralmente}
  \end{Phonetics}
\end{Entry}

\begin{Entry}{单打}{8,5}{⼗、⼿}
  \begin{Phonetics}{单打}{dan1 da3}[][HSK 6]
    \definition[场,局,次]{s.}{Esporte: simples; competição um contra um}
  \end{Phonetics}
\end{Entry}

\begin{Entry}{单边}{8,5}{⼗、⾡}
  \begin{Phonetics}{单边}{dan1bian1}[][HSK 7-9]
    \definition{adj.}{unilateral}
  \end{Phonetics}
\end{Entry}

\begin{Entry}{单位}{8,7}{⼗、⼈}
  \begin{Phonetics}{单位}{dan1wei4}[][HSK 2]
    \definition[个,家]{s.}{unidade (como padrão de medida) | unidade (como uma organização, departamento, divisão, seção, etc.) | unidade (grupo de pessoas como um todo) | unidade de trabalho (local de trabalho, especialmente na República Popular da China antes da reforma econômica)}
  \end{Phonetics}
\end{Entry}

\begin{Entry}{单纯}{8,7}{⼗、⽷}
  \begin{Phonetics}{单纯}{dan1chun2}[][HSK 4]
    \definition{adj.}{puro; simples; descomplicado}
    \definition{adv.}{sozinho; puramente; meramente}
  \end{Phonetics}
\end{Entry}

\begin{Entry}{单身}{8,7}{⼗、⾝}
  \begin{Phonetics}{单身}{dan1shen1}[][HSK 7-9]
    \definition{s.}{solteiro}
  \end{Phonetics}
\end{Entry}

\begin{Entry}{单质}{8,8}{⼗、⾙}
  \begin{Phonetics}{单质}{dan1zhi4}
    \definition{s.}{substância simples (consistindo puramente de um elemento, como diamante, grafite, etc.)}
  \end{Phonetics}
\end{Entry}

\begin{Entry}{单独}{8,9}{⼗、⽝}
  \begin{Phonetics}{单独}{dan1du2}[][HSK 4]
    \definition{adv.}{solo; sozinho; por si mesmo; por conta própria}
  \end{Phonetics}
\end{Entry}

\begin{Entry}{单调}{8,10}{⼗、⾔}
  \begin{Phonetics}{单调}{dan1diao4}[][HSK 4]
    \definition{adj.}{maçante; monótono}
  \end{Phonetics}
\end{Entry}

\begin{Entry}{单脚滑行车}{8,11,12,6,4}{⼗、⾁、⽔、⾏、⾞}
  \begin{Phonetics}{单脚滑行车}{dan1jiao3hua2xing2che1}
    \definition{s.}{\emph{scooter}}
  \end{Phonetics}
\end{Entry}

\begin{Entry}{单薄}{8,16}{⼗、⾋}
  \begin{Phonetics}{单薄}{dan1bo2}[][HSK 7-9]
    \definition{adj.}{fino; pouco | frágil; magro e fraco | fino; frágil; insubstancial}
  \end{Phonetics}
\end{Entry}

\begin{Entry}{卖}{8}{⼗}
  \begin{Phonetics}{卖}{mai4}[][HSK 2]
    \definition*{s.}{Sobrenome Mai}
    \definition{clas.}{um prato (nos tempos antigos); antigamente, os restaurantes chamavam cada prato vendido de 一卖 (uma porção)}
    \definition{v.}{vender (oposto de 买) | trair (o próprio país ou amigos); alcançar objetivos pessoais à custa dos interesses do país, da nação e dos outros | não poupar esforços; esforçar-se ao máximo; tentar fazer o máximo possível | mostrar-se intencionalmente; exibir-se | vender o próprio trabalho; trabalhar em troca de dinheiro}
  \seealsoref{买}{mai3}
  \end{Phonetics}
\end{Entry}

\begin{Entry}{卧}{8}{⾂}
  \begin{Phonetics}{卧}{wo4}
    \definition{adj.}{para dormir}
    \definition{s.}{vagão-leito (ou carruagem); leito | beliche | quarto | Dialeto: pochê (ovos)}
    \definition{v.}{deitar | Dialeto: deitar um bebê | (animais ou pássaros) agachar-se; sentar-se; empoleirar-se | Figurativo: viver em reclusão}
  \end{Phonetics}
\end{Entry}

\begin{Entry}{卧车}{8,4}{⾂、⾞}
  \begin{Phonetics}{卧车}{wo4che1}
    \definition{s.}{um carro-leito | vagão-leito}
  \end{Phonetics}
\end{Entry}

\begin{Entry}{卧式}{8,6}{⾂、⼷}
  \begin{Phonetics}{卧式}{wo4shi4}
    \definition{adj.}{horizontal}
  \end{Phonetics}
\end{Entry}

\begin{Entry}{卧床}{8,7}{⾂、⼴}
  \begin{Phonetics}{卧床}{wo4chuang2}
    \definition{adj.}{acamado}
    \definition{s.}{cama}
    \definition{v.}{deitar na cama}
  \end{Phonetics}
\end{Entry}

\begin{Entry}{卧室}{8,9}{⾂、⼧}
  \begin{Phonetics}{卧室}{wo4shi4}[][HSK 5]
    \definition[间,个]{s.}{quarto de dormir; quarto de uma casa usado para dormir}
  \end{Phonetics}
\end{Entry}

\begin{Entry}{卧倒}{8,10}{⾂、⼈}
  \begin{Phonetics}{卧倒}{wo4dao3}
    \definition{v.}{cair no chão | deitar-se}
  \end{Phonetics}
\end{Entry}

\begin{Entry}{卧病}{8,10}{⾂、⽧}
  \begin{Phonetics}{卧病}{wo4bing4}
    \definition{s.}{acamado | doente na cama}
  \end{Phonetics}
\end{Entry}

\begin{Entry}{卧舱}{8,10}{⾂、⾈}
  \begin{Phonetics}{卧舱}{wo4cang1}
    \definition{s.}{cabine de dormir em um barco ou trem}
  \end{Phonetics}
\end{Entry}

\begin{Entry}{卧推}{8,11}{⾂、⼿}
  \begin{Phonetics}{卧推}{wo4tui1}
    \definition{s.}{supino}
  \end{Phonetics}
\end{Entry}

\begin{Entry}{卧铺}{8,12}{⾂、⾦}
  \begin{Phonetics}{卧铺}{wo4 pu4}[][HSK 6]
    \definition[个,排]{s.}{beliche para dormir; um beliche em um trem ou ônibus de longa distância}
  \end{Phonetics}
\end{Entry}

\begin{Entry}{卧榻}{8,14}{⾂、⽊}
  \begin{Phonetics}{卧榻}{wo4ta4}
    \definition{s.}{um sofá | uma cama estreita}
  \end{Phonetics}
\end{Entry}

\begin{Entry}{卷}{8}{⼙}
  \begin{Phonetics}{卷}{juan3}[][HSK 4]
    \definition{clas.}{usado para pequenas coisas enroladas (maço de papel dinheiro, carretel de filme, etc.) | usado para rolos, carretéis, bobinas, etc.}
    \definition[张]{s.}{rolo; carretel; bobina}
    \definition{v.}{enrolar; dobrar algo em um cilindro ou semicírculo | varrer; carregar; levar junto | envolver-se; participar}
  \end{Phonetics}
  \begin{Phonetics}{卷}{juan4}[][HSK 4]
    \definition{clas.}{usado para capítulos, seções ou volumes; fascículos}
    \definition[大,小]{s.}{livro; livros e pinturas que são enrolados para coleção; geralmente se refere a pinturas e caligrafia | papel de exame | arquivo; dossiê}
  \end{Phonetics}
\end{Entry}

\begin{Entry}{厕}{8}{⼚}
  \begin{Phonetics}{厕}{ce4}
    \definition[个,间]{s.}{latrina; fossa sanitária; (componente formador de palavras)}
  \seealsoref{茅厕}{mao2ce4}
  \end{Phonetics}
  \begin{Phonetics}{厕}{si5}
    \definition{s.}{componente formador de palavras | latrina; fossa sanitária}
  \seealsoref{茅厕}{mao2ce4}
  \end{Phonetics}
\end{Entry}

\begin{Entry}{厕纸}{8,7}{⼚、⽷}
  \begin{Phonetics}{厕纸}{ce4zhi3}
    \definition{s.}{papel higiênico}
  \end{Phonetics}
\end{Entry}

\begin{Entry}{厕所}{8,8}{⼚、⼾}
  \begin{Phonetics}{厕所}{ce4suo3}[][HSK 6]
    \definition[个,间]{s.}{banheiro; lavatório; sanitário; latrina; um lugar para as pessoas urinarem e defecarem}
  \end{Phonetics}
\end{Entry}

\begin{Entry}{参}{8}{⼛}
  \begin{Phonetics}{参}{can1}
    \definition{v.}{juntar-se; entrar; tomar parte em; participar | referir; consultar; comparar com outros materiais | ligar para prestar homenagem a; fazer uma visita |  (significado antigo) acusar um funcionário perante o imperador; relatar ou expor ao imperador | explorar e compreender (verdade, significado, etc.)}
  \end{Phonetics}
\end{Entry}

\begin{Entry}{参与}{8,3}{⼛、⼀}
  \begin{Phonetics}{参与}{can1yu4}[][HSK 4]
    \definition{v.}{participar de; tomar parte em; ter uma mão em; envolver-se em; participar (no planejamento, discussão e condução dos assuntos)}
  \end{Phonetics}
\end{Entry}

\begin{Entry}{参见}{8,4}{⼛、⾒}
  \begin{Phonetics}{参见}{can1jian4}[][HSK 7-9]
    \definition{v.}{(em referências) ver; ver também | consultar; visualizar | ler algo para referência | prestar homenagem a (um superior, etc.)}
  \end{Phonetics}
\end{Entry}

\begin{Entry}{参加}{8,5}{⼛、⼒}
  \begin{Phonetics}{参加}{can1jia1}[][HSK 2]
    \definition{v.}{aderir (a organizações); participar; participar (de atividades); participar de alguma organização ou atividade | dar (conselho, sugestão, etc.)}
  \end{Phonetics}
\end{Entry}

\begin{Entry}{参军}{8,6}{⼛、⼍}
  \begin{Phonetics}{参军}{can1/jun1}[][HSK 7-9]
    \definition{s.}{oficial do estado-maior militar; título oficial antigo}
    \definition{v.+compl.}{juntar-se ao exército; alistar-se}
  \end{Phonetics}
\end{Entry}

\begin{Entry}{参考}{8,6}{⼛、⽼}
  \begin{Phonetics}{参考}{can1kao3}[][HSK 4]
    \definition{v.}{consultar; referir-se a; acessar informações relevantes para estudo ou pesquisa | consultar; referir-se a; lidar com coisas, observar, ler, aprender e usar materiais relevantes}
  \end{Phonetics}
\end{Entry}

\begin{Entry}{参观}{8,6}{⼛、⾒}
  \begin{Phonetics}{参观}{can1guan1}[][HSK 2]
    \definition{v.}{visitar; dar uma olhada; observação no local (resultados do trabalho, carreira, instalações, locais históricos e pontos turísticos, etc.)}
  \end{Phonetics}
\end{Entry}

\begin{Entry}{参展}{8,10}{⼛、⼫}
  \begin{Phonetics}{参展}{can1 zhan3}[][HSK 6]
    \definition{v.}{expor ou participar de uma feira comercial, etc.}
  \end{Phonetics}
\end{Entry}

\begin{Entry}{参谋}{8,11}{⼛、⾔}
  \begin{Phonetics}{参谋}{can1mou2}[][HSK 7-9]
    \definition{s.}{oficial de estado-maior; pessoal militar envolvido em planejamento militar e outros assuntos |conselheiro; consultor}
    \definition{v.}{aconselhar; dar conselhos}
  \end{Phonetics}
\end{Entry}

\begin{Entry}{参照}{8,13}{⼛、⽕}
  \begin{Phonetics}{参照}{can1zhao4}[][HSK 7-9]
    \definition{v.}{consultar; referir-se a; referir-se e imitar (métodos, experiências, etc.)}
  \end{Phonetics}
\end{Entry}

\begin{Entry}{参赛}{8,14}{⼛、⾙}
  \begin{Phonetics}{参赛}{can1 sai4}[][HSK 6]
    \definition{v.}{participar de uma partida (ou competição); competir}
  \end{Phonetics}
\end{Entry}

\begin{Entry}{叔}{8}{⼜}
  \begin{Phonetics}{叔}{shu1}
    \definition*{s.}{Sobrenome Shu}
    \definition{s.}{irmão mais novo do pai; tio (por parte de pai)| irmão mais novo do marido | terceiro entre irmãos | tio | uma forma de tratamento para um homem um pouco mais jovem que o pai; tio | terceiro tio (de quatro irmãos) | primo mais novo da mãe}
  \end{Phonetics}
\end{Entry}

\begin{Entry}{叔叔}{8,8}{⼜、⼜}
  \begin{Phonetics}{叔叔}{shu1shu5}
    \definition[个,位,名]{s.}{tio; irmão mais novo do pai | tio, dirigindo-se a um homem da mesma geração que o pai e mais jovem em idade}
  \end{Phonetics}
\end{Entry}

\begin{Entry}{取}{8}{⼜}
  \begin{Phonetics}{取}{qu3}[][HSK 2]
    \definition{v.}{pegar; obter; buscar; pegar de um lugar; pegar nas mãos | visar; procurar; obter; provocar | adotar; assumir; escolher; selecionar}
  \end{Phonetics}
\end{Entry}

\begin{Entry}{取水}{8,4}{⼜、⽔}
  \begin{Phonetics}{取水}{qu3shui3}
    \definition{v.}{obter água (de um poço, etc.)}
  \end{Phonetics}
\end{Entry}

\begin{Entry}{取现}{8,8}{⼜、⾒}
  \begin{Phonetics}{取现}{qu3xian4}
    \definition{v.}{sacar dinheiro}
  \end{Phonetics}
\end{Entry}

\begin{Entry}{取胜}{8,9}{⼜、⾁}
  \begin{Phonetics}{取胜}{qu3sheng4}
    \definition{v.}{prevalecer sobre os oponentes | marcar uma vitória}
  \end{Phonetics}
\end{Entry}

\begin{Entry}{取悦}{8,10}{⼜、⼼}
  \begin{Phonetics}{取悦}{qu3yue4}
    \definition{v.}{tentar agradar}
  \end{Phonetics}
\end{Entry}

\begin{Entry}{取消}{8,10}{⼜、⽔}
  \begin{Phonetics}{取消}{qu3xiao1}[][HSK 3]
    \definition{v.}{cancelar; suspender; anular; abolir; revogar; rescindir; tornar o sistema original, regulamentos, qualificações, direitos, etc. inválidos}
  \end{Phonetics}
\end{Entry}

\begin{Entry}{取得}{8,11}{⼜、⼻}
  \begin{Phonetics}{取得}{qu3 de2}[][HSK 2]
    \definition{v.}{ganhar; adquirir; obter; ser o primeiro a conseguir}
  \end{Phonetics}
\end{Entry}

\begin{Entry}{取款}{8,12}{⼜、⽋}
  \begin{Phonetics}{取款}{qu3kuan3}[][HSK 6]
    \definition{v.}{sacar dinheiro (de um banco); retirar o dinheiro que você depositou (geralmente se refere a retirar dinheiro do banco)}
  \end{Phonetics}
\end{Entry}

\begin{Entry}{取款机}{8,12,6}{⼜、⽋、⽊}
  \begin{Phonetics}{取款机}{qu3 kuan3 ji1}[][HSK 6]
    \definition{s.}{ATM; caixa eletrônico; um caixa eletrônico é uma máquina que pode concluir automaticamente operações bancárias, como saques e consultas de saldo}
  \end{Phonetics}
\end{Entry}

\begin{Entry}{受}{8}{⼜}
  \begin{Phonetics}{受}{shou4}[][HSK 3]
    \definition{v.}{receber; aceitar | sofrer; ser submetido a | aguentar; suportar; tolerar | ser agradável}
  \end{Phonetics}
\end{Entry}

\begin{Entry}{受不了}{8,4,2}{⼜、⼀、⼅}
  \begin{Phonetics}{受不了}{shou4bu5liao3}[][HSK 4]
    \definition{v.}{ser insuportável; não poder suportar algo; não suportar algo}
  \end{Phonetics}
\end{Entry}

\begin{Entry}{受伤}{8,6}{⼜、⼈}
  \begin{Phonetics}{受伤}{shou4shang1}[][HSK 3]
    \definition{v.}{ser ferido; sofrer uma lesão}
  \end{Phonetics}
\end{Entry}

\begin{Entry}{受灾}{8,7}{⼜、⽕}
  \begin{Phonetics}{受灾}{shou4 zai1}[][HSK 5]
    \definition{v.}{ser atingido por um desastre natural (ou calamidade) | ser atingido por uma adversidade natural}
  \end{Phonetics}
\end{Entry}

\begin{Entry}{受到}{8,8}{⼜、⼑}
  \begin{Phonetics}{受到}{shou4dao4}[][HSK 2]
    \definition{v.}{receber; receber itens, mensagens, instruções, etc. fornecidos por outras pessoas}
  \end{Phonetics}
\end{Entry}

\begin{Entry}{受限}{8,8}{⼜、⾩}
  \begin{Phonetics}{受限}{shou4xian4}
    \definition{v.}{ser limitado | ser restrito | ser constrangido}
  \end{Phonetics}
\end{Entry}

\begin{Entry}{受得了}{8,11,2}{⼜、⼻、⼅}
  \begin{Phonetics}{受得了}{shou4de5liao3}
    \definition{v.}{suportar | aguentar}
  \end{Phonetics}
\end{Entry}

\begin{Entry}{变}{8}{⼜}
  \begin{Phonetics}{变}{bian4}[][HSK 2]
    \definition{adj.}{alterado; mutável; que pode mudar; que está mudando ou já mudou}
    \definition{s.}{uma reviravolta inesperada nos acontecimentos; mudanças significativas repentinas}
    \definition{v.}{mudar; tornar-se diferente; fazer mudanças | tornar-se; transformar-se; natureza, estado ou situação diferentes dos originais | alterar; mudar; transformar}
  \end{Phonetics}
\end{Entry}

\begin{Entry}{变为}{8,4}{⼜、⼂}
  \begin{Phonetics}{变为}{bian4 wei2}[][HSK 3]
    \definition{v.}{transformar-se em; tornar-se | mudar para}
  \end{Phonetics}
\end{Entry}

\begin{Entry}{变化}{8,4}{⼜、⼔}
  \begin{Phonetics}{变化}{bian4hua4}[][HSK 3]
    \definition[个]{s.}{mudança; variação; a nova situação após uma mudança em pessoas ou coisas}
    \definition{v.}{mudar;  variar}
  \end{Phonetics}
\end{Entry}

\begin{Entry}{变幻莫测}{8,4,10,9}{⼜、⼳、⾋、⽔}
  \begin{Phonetics}{变幻莫测}{bian4huan4-mo4ce4}[][HSK 7-9]
    \definition{expr.}{mutável; imprevisível | errático | mudar imprevisivelmente | traiçoeiro}
  \end{Phonetics}
\end{Entry}

\begin{Entry}{变心}{8,4}{⼜、⼼}
  \begin{Phonetics}{变心}{bian4/xin1}
    \definition{v.+compl.}{deixar de ser fiel}
  \end{Phonetics}
\end{Entry}

\begin{Entry}{变节}{8,5}{⼜、⾋}
  \begin{Phonetics}{变节}{bian4jie2}
    \definition{s.}{traição | deserção | vira-casaca}
    \definition{v.}{retratar-se e submeter-se; renunciar e render-se | mudar de lado politicamente}
  \end{Phonetics}
\end{Entry}

\begin{Entry}{变动}{8,6}{⼜、⼒}
  \begin{Phonetics}{变动}{bian4 dong4}[][HSK 5]
    \definition{s.}{mudança; alteração; oscilação; modificação; variação}
    \definition{v.}{mudar; alterar; oscilar; modificar; variar}
  \end{Phonetics}
\end{Entry}

\begin{Entry}{变异}{8,6}{⼜、⼶}
  \begin{Phonetics}{变异}{bian4yi4}[][HSK 7-9]
    \definition{s.}{variação; mutação; muta; diferenças nas características morfológicas e fisiológicas entre gerações da mesma espécie ou entre indivíduos da mesma geração}
    \definition{v.}{variar; mudar}
  \end{Phonetics}
\end{Entry}

\begin{Entry}{变成}{8,6}{⼜、⼽}
  \begin{Phonetics}{变成}{bian4 cheng2}[][HSK 2]
    \definition{v.}{crescer; tornar-se; fazer; desenvolver-se; revelar-se; resultar; acontecer; passar a ser; passar para; acumular-se; converter-se; transformar-se; transformar-se em; mudar-se em; transformação da situação ou condição anterior para a situação ou condição atual}
  \end{Phonetics}
\end{Entry}

\begin{Entry}{变迁}{8,6}{⼜、⾡}
  \begin{Phonetics}{变迁}{bian4qian1}[][HSK 7-9]
    \definition{s.}{mudanças; transição; vicissitudes; mudança em tendências ou condições; mudança de situação ou estágio}
  \end{Phonetics}
\end{Entry}

\begin{Entry}{变形}{8,7}{⼜、⼺}
  \begin{Phonetics}{变形}{bian4/xing2}[][HSK 6]
    \definition{v.+compl.}{deformar; ficar fora de forma | transformar; transformar-se em outras formas}
  \end{Phonetics}
\end{Entry}

\begin{Entry}{变更}{8,7}{⼜、⽈}
  \begin{Phonetics}{变更}{bian4 geng1}[][HSK 6]
    \definition{v.}{alterar; mudar; modificar}
  \end{Phonetics}
\end{Entry}

\begin{Entry}{变性}{8,8}{⼜、⼼}
  \begin{Phonetics}{变性}{bian4xing4}
    \definition{s.}{desnaturação | transexual}
    \definition{v.}{desnaturar | mudar de sexo}
  \end{Phonetics}
\end{Entry}

\begin{Entry}{变质}{8,8}{⼜、⾙}
  \begin{Phonetics}{变质}{bian4/zhi4}[][HSK 7-9]
    \definition{v.+compl.}{deteriorar-se; estragar-se | tornar-se moralmente degenerado}
  \end{Phonetics}
\end{Entry}

\begin{Entry}{变革}{8,9}{⼜、⾰}
  \begin{Phonetics}{变革}{bian4ge2}[][HSK 7-9]
    \definition{s.}{mudança; transformação; a natureza das coisas foi reformada}
    \definition{v.}{transformar; mudar (de sistemas sociais, políticas, etc.); mudar o antigo e inovar; mudar a essência das coisas (referindo-se principalmente aos sistemas sociais)}
  \end{Phonetics}
\end{Entry}

\begin{Entry}{变换}{8,10}{⼜、⼿}
  \begin{Phonetics}{变换}{bian4 huan4}[][HSK 6]
    \definition{v.}{variar; alternar; mudar a forma ou o conteúdo de algo de uma coisa para outra}
  \end{Phonetics}
\end{Entry}

\begin{Entry}{变装}{8,12}{⼜、⾐}
  \begin{Phonetics}{变装}{bian4zhuang1}
    \definition{v.}{trocar de roupa | vestir-se | vestir uma fantasia | disfarçar-se ou fantasiar-se de personagem real ou ficcional, \emph{cosplay} | travestir-se}
  \end{Phonetics}
\end{Entry}

\begin{Entry}{变数}{8,13}{⼜、⽁}
  \begin{Phonetics}{变数}{bian4shu4}
    \definition{s.}{(matemática) variável | fatores variáveis}
  \end{Phonetics}
\end{Entry}

\begin{Entry}{呢}{8}{⼝}
  \begin{Phonetics}{呢}{ne5}[][HSK 1]
    \definition{part.}{usada no final de frases interrogativas (especificamente perguntas, perguntas de escolha e perguntas retóricas) para indicar um tom interrogativo | usada no final de uma frase declarativa, indica que uma ação ou situação está em andamento | usada em frases para indicar uma pausa (muitas vezes em pares) | usada no final de uma frase declarativa para confirmar um fato e convencer o interlocutor (com um tom de indicação e exagero)}
  \end{Phonetics}
  \begin{Phonetics}{呢}{ni2}
    \definition{s.}{(tecido feito de) lã; tecido de lã (para roupas pesadas); tecido de lã pesada; revestimento ou roupa de lã}
  \end{Phonetics}
\end{Entry}

\begin{Entry}{周}{8}{⼝}
  \begin{Phonetics}{周}{zhou1}[][HSK 2]
    \definition*{s.}{Dinastia Zhou (1046-256 BC) | Dinastia Zhou do Norte (557-581), uma das Dinastias do Norte | Dinastia Zhou Posterior (951-960), uma das Cinco Dinastias | Sobrenome Zhou}
    \definition{adj.}{universal; inteiro; por toda parte | atencioso; pensativo; completo; minucioso}
    \definition{adv.}{semanalmente}
    \definition{clas.}{usado para rodadas, voltas}
    \definition{s.}{periferia; arredores; círculo | semana | ciclo}
    \definition{v.}{fazer um circuito; mover-se em um curso circular | ajudar alguém}
  \end{Phonetics}
\end{Entry}

\begin{Entry}{周末}{8,5}{⼝、⽊}
  \begin{Phonetics}{周末}{zhou1mo4}[][HSK 2]
    \definition[个]{s.}{final-de-semana}
  \end{Phonetics}
\end{Entry}

\begin{Entry}{周年}{8,6}{⼝、⼲}
  \begin{Phonetics}{周年}{zhou1nian2}[][HSK 2]
    \definition{s.}{aniversário}
  \end{Phonetics}
\end{Entry}

\begin{Entry}{周围}{8,7}{⼝、⼞}
  \begin{Phonetics}{周围}{zhou1wei2}[][HSK 3]
    \definition{s.}{ao redor; redondeza; vizinhança; a parte ao redor do centro}
  \end{Phonetics}
\end{Entry}

\begin{Entry}{周期}{8,12}{⼝、⽉}
  \begin{Phonetics}{周期}{zhou1qi1}[][HSK 5]
    \definition[个]{s.}{período; ciclo; no processo de mudança e movimento das coisas, certas características se repetem várias vezes, com um intervalo de tempo entre cada repetição | período; ciclo; refere-se a um processo em que certas características se repetem várias vezes, e o tempo decorrido entre duas ocorrências consecutivas | classificação dos elementos na tabela periódica}
  \end{Phonetics}
\end{Entry}

\begin{Entry}{味}{8}{⼝}
  \begin{Phonetics}{味}{wei4}
    \definition{clas.}{usado para ingredientes (de uma receita de medicina chinesa)}
    \definition{s.}{gosto; sabor | cheiro; odor | interesse; deleite | acepipe; \emph{delicacy} | significância; significado}
    \definition{v.}{distinguir (provar) o sabor de; saborear}
  \end{Phonetics}
\end{Entry}

\begin{Entry}{味儿}{8,2}{⼝、⼉}
  \begin{Phonetics}{味儿}{wei4r5}[][HSK 4]
    \definition{s.}{gosto; sabor; propriedade de uma substância que dá à língua uma determinada sensação de sabor | cheiro; odor; propriedade de uma substância que dá ao nariz um determinado sentido de cheiro | interesse; significado; deleite}
  \end{Phonetics}
\end{Entry}

\begin{Entry}{味道}{8,12}{⼝、⾡}
  \begin{Phonetics}{味道}{wei4dao5}[][HSK 2]
    \definition[个,股,种]{s.}{gosto; sabor | sensação; gosto; experiência | interesse; deleite | cheiro; odor}
  \end{Phonetics}
\end{Entry}

\begin{Entry}{呵}{8}{⼝}
  \begin{Phonetics}{呵}{a1}
    \variantof{啊}
  \end{Phonetics}
  \begin{Phonetics}{呵}{he1}
    \definition{interj.}{Meu Deus!| Ah!; Oh!}
    \definition{v.}{expirar (com a boca aberta) | repreender}
  \end{Phonetics}
\end{Entry}

\begin{Entry}{呵护}{8,7}{⼝、⼿}
  \begin{Phonetics}{呵护}{he1hu4}[][HSK 7-9]
    \definition{v.}{proteger; cuidar bem de}
  \end{Phonetics}
\end{Entry}

\begin{Entry}{呶}{8}{⼝}
  \begin{Phonetics}{呶}{nao2}
    \definition{interj.}{(onomatopéia) ruído alto e contínuo}
    \definition{v.}{(literário) gritar; clamar; falar ruidosamente}
  \seealsoref{努}{nu3}
  \end{Phonetics}
\end{Entry}

\begin{Entry}{呼}{8}{⼝}
  \begin{Phonetics}{呼}{hu1}
    \definition*{s.}{Sobrenome Hu}
    \definition{s.}{Onomatopéia: descreve o som do vento}
    \definition{v.}{expirar | gritar; clamar | chamar; ligar; ligar para alguém}
  \end{Phonetics}
\end{Entry}

\begin{Entry}{呼吸}{8,6}{⼝、⼝}
  \begin{Phonetics}{呼吸}{hu1xi1}[][HSK 4]
    \definition{s.}{um suspiro; metáfora para um período de tempo muito curto}
    \definition{v.}{respirar}
  \end{Phonetics}
\end{Entry}

\begin{Entry}{呼啦啦}{8,11,11}{⼝、⼝、⼝}
  \begin{Phonetics}{呼啦啦}{hu1 la1 la1}
    \definition{s.}{Onomatopéia: som de bater asas}
  \end{Phonetics}
\end{Entry}

\begin{Entry}{呼啸}{8,11}{⼝、⼝}
  \begin{Phonetics}{呼啸}{hu1xiao4}
    \definition{v.}{assobiar}
  \end{Phonetics}
\end{Entry}

\begin{Entry}{命}{8}{⼝}
  \begin{Phonetics}{命}{ming4}[][HSK 6]
    \definition[条]{s.}{vida | sorte; destino; fado | ordem; comando; instrução | atribuição de um nome, título etc.}
    \definition{v.}{ordenar; nomear | atribuir (um nome etc.)}
  \end{Phonetics}
\end{Entry}

\begin{Entry}{命令}{8,5}{⼝、⼈}
  \begin{Phonetics}{命令}{ming4ling4}[][HSK 5]
    \definition[条,项,道,个]{s.}{ordem; comando; instruções emitidas pelos superiores aos subordinados}
    \definition{v.}{ordenar; comandar}
  \end{Phonetics}
\end{Entry}

\begin{Entry}{命运}{8,7}{⼝、⾡}
  \begin{Phonetics}{命运}{ming4yun4}[][HSK 3]
    \definition[个]{s.}{tendência de desenvolvimento; tendência de futuro; metáfora para a direção e tendência do desenvolvimento e das mudanças | destino; sina; sorte; refere-se à vida e à morte, à riqueza e à pobreza e a todas as experiências da vida}
  \end{Phonetics}
\end{Entry}

\begin{Entry}{和}{8}{⼝}
  \begin{Phonetics}{和}{he2}[][HSK 1]
    \definition*{s.}{Sobrenome He}
    \definition{adj.}{gentil; suave; amável | harmonioso; em boas condições}
    \definition{conj.}{e (somente para palavras); unidos com}
    \definition{prep.}{relacionado com | para; com; indica correlação; comparação, etc.}
    \definition{s.}{soma; soma total | japonês; refere-se ao Japão}
    \definition{v.}{disputar; reconciliar; acabar com a guerra ou a disputa | empatar; (próxima edição ou torneio) sem vencedor}
  \end{Phonetics}
  \begin{Phonetics}{和}{he4}
    \definition{v.}{compor um poema em resposta (ao poema de alguém) usando a mesma sequência de rimas | juntar-se à cantoria; cantar junto com outros em harmonia}
  \end{Phonetics}
  \begin{Phonetics}{和}{hu2}
    \definition{v.}{completar um conjunto de Mahjong, 麻将, ou cartas de baralho}
  \seealsoref{麻将}{ma2jiang4}
  \end{Phonetics}
  \begin{Phonetics}{和}{huo2}
    \definition{v.}{combinar uma substância em pó (farinha, gesso, etc.) com água; adicionar líquido ao pó e mexer ou amassar até ficar pegajoso ou espesso}
  \end{Phonetics}
  \begin{Phonetics}{和}{huo4}
    \definition{clas.}{usado para enxágues de roupas | usado para fervuras de ervas medicinais}
    \definition{v.}{misturar (ingredientes); misturar pós ou grãos; misturar com água para obter uma consistência mais líquida}
  \end{Phonetics}
\end{Entry}

\begin{Entry}{和气}{8,4}{⼝、⽓}
  \begin{Phonetics}{和气}{he2qi5}[][HSK 7-9]
    \definition{adj.}{gentil; bondoso; educado | amigável; amável; harmonioso}
    \definition{s.}{amizade; relações harmoniosas; atmosfera harmoniosa; sentimentos harmoniosos}
  \end{Phonetics}
\end{Entry}

\begin{Entry}{和平}{8,5}{⼝、⼲}
  \begin{Phonetics}{和平}{he2ping2}[][HSK 3]
    \definition{adj.}{pacífico; moderado; não violento | pacífico; tranquilo; sereno}
    \definition{s.}{paz ;ausência de guerra}
  \end{Phonetics}
\end{Entry}

\begin{Entry}{和平共处}{8,5,6,5}{⼝、⼲、⼋、⼡}
  \begin{Phonetics}{和平共处}{he2ping2 gong4chu3}[][HSK 7-9]
    \definition{expr.}{coexistência pacífica de nações, sociedades etc.; refere-se a países com diferentes sistemas sociais que resolvem disputas pacificamente e desenvolvem laços econômicos e culturais com base na igualdade e no benefício mútuo}
  \end{Phonetics}
\end{Entry}

\begin{Entry}{和尚}{8,8}{⼝、⼩}
  \begin{Phonetics}{和尚}{he2shang5}[][HSK 7-9]
    \definition[个,名,位]{s.}{monge budista; refere-se aos monges budistas do sexo masculino que praticam o budismo}
  \end{Phonetics}
\end{Entry}

\begin{Entry}{和谐}{8,11}{⼝、⾔}
  \begin{Phonetics}{和谐}{he2xie2}[][HSK 6]
    \definition{adj.}{harmonioso; sem conflitos | em perfeita harmonia; ajuste adequado e simétrico}
    \definition{v.}{(eufemismo) censurar}
  \end{Phonetics}
\end{Entry}

\begin{Entry}{和睦}{8,13}{⼝、⽬}
  \begin{Phonetics}{和睦}{he2mu4}[][HSK 7-9]
    \definition{adj.}{harmonioso; amigável; amistoso}
    \definition{s.}{harmonia; concórdia}
  \end{Phonetics}
\end{Entry}

\begin{Entry}{和解}{8,13}{⼝、⾓}
  \begin{Phonetics}{和解}{he2jie3}[][HSK 7-9]
    \definition{v.}{reconciliar}
  \end{Phonetics}
\end{Entry}

\begin{Entry}{和蔼}{8,14}{⼝、⾋}
  \begin{Phonetics}{和蔼}{he2'ai3}[][HSK 7-9]
    \definition{adj.}{gentil; afável; amável}
  \end{Phonetics}
\end{Entry}

\begin{Entry}{咒}{8}{⼝}
  \begin{Phonetics}{咒}{zhou4}
    \definition[个,句]{s.}{encantamento; feitiço}
    \definition{v.}{amaldiçoar; condenar | maltratar; dizer que você espera que as pessoas não tenham sucesso}
  \end{Phonetics}
\end{Entry}

\begin{Entry}{咒骂}{8,9}{⼝、⾺}
  \begin{Phonetics}{咒骂}{zhou4ma4}
    \definition{v.}{xingar | amaldiçoar | execrar}
  \end{Phonetics}
\end{Entry}

\begin{Entry}{咖}{8}{⼝}
  \begin{Phonetics}{咖}{ka1}
    \definition[杯]{s.}{classe | café | graduação}
  \end{Phonetics}
\end{Entry}

\begin{Entry}{咖啡}{8,11}{⼝、⼝}
  \begin{Phonetics}{咖啡}{ka1fei1}[][HSK 3]
    \definition[杯,瓶,罐,壶,包,袋,盒]{s.}{(empréstimo linguístico) café}
  \end{Phonetics}
\end{Entry}

\begin{Entry}{咖啡色}{8,11,6}{⼝、⼝、⾊}
  \begin{Phonetics}{咖啡色}{ka1fei1 se4}
    \definition{s.}{cor café}
  \end{Phonetics}
\end{Entry}

\begin{Entry}{咖啡馆}{8,11,11}{⼝、⼝、⾷}
  \begin{Phonetics}{咖啡馆}{ka1fei1guan3}
    \definition[家]{s.}{cafeteria}
  \end{Phonetics}
\end{Entry}

\begin{Entry}{哎}{8}{⼝}
  \begin{Phonetics}{哎}{ai1}[][HSK 7-9]
    \definition{interj.}{Por que?; Ei!; Ai!; expressar surpresa ou insatisfação | Ei!; Cuidado!}
  \end{Phonetics}
\end{Entry}

\begin{Entry}{哎呀}{8,7}{⼝、⼝}
  \begin{Phonetics}{哎呀}{ai1ya1}[][HSK 7-9]
    \definition{interj.}{expressar surpresa ou espanto | expressar reclamação, impaciência, etc.}
  \end{Phonetics}
\end{Entry}

\begin{Entry}{固}{8}{⼞}
  \begin{Phonetics}{固}{gu4}
    \definition*{s.}{Sobrenome Gu}
    \definition{adj.}{sólido; firme; forte | duro; sólido | mal informado; superficial; ignorante}
    \definition{adv.}{firmemente; resolutamente | originalmente; em primeiro lugar | certamente; reconhecidamente; seguramente}
    \definition{conj.}{usado da mesma forma que 固然}
    \definition{v.}{solidificar; consolidar; fortalecer | defender; proteger}
  \seealsoref{固然}{gu4ran2}
  \end{Phonetics}
\end{Entry}

\begin{Entry}{固执}{8,6}{⼞、⼿}
  \begin{Phonetics}{固执}{gu4zhi5}[][HSK 7-9]
    \definition{adj.}{obstinado; teimoso; mantém suas próprias opiniões e não quer mudá-las, mesmo que estejam erradas}
  \end{Phonetics}
\end{Entry}

\begin{Entry}{固定}{8,8}{⼞、⼧}
  \begin{Phonetics}{固定}{gu4ding4}[][HSK 4]
    \definition{adj.}{fixo; regular; inalterado ou imóvel}
    \definition{v.}{consertar; tornar fixo, não mover novamente; colocar as coisas em ordem, não mudá-las novamente}
  \end{Phonetics}
\end{Entry}

\begin{Entry}{固然}{8,12}{⼞、⽕}
  \begin{Phonetics}{固然}{gu4ran2}[][HSK 7-9]
    \definition{conj.}{usado para introduzir uma cláusula adversativa admitindo primeiro um certo fato; quando usado na primeira metade de uma frase, a segunda metade geralmente tem 可是 ou 但是 para ecoá-lo, indicando que o fato A é reconhecido, mas o fato B não se torna inválido por causa do fato A | admitir um fato sem negar outro; indica o reconhecimento de um fato, levando a uma transição no texto seguinte; indica o reconhecimento do fato A e não nega o fato B}
  \seealsoref{但是}{dan4 shi4}
  \seealsoref{可是}{ke3shi4}
  \end{Phonetics}
\end{Entry}

\begin{Entry}{国}{8}{⼞}
  \begin{Phonetics}{国}{guo2}[][HSK 1]
    \definition*{s.}{Sobrenome Guo}
    \definition{adj.}{nacional; do estado; representante do país | o melhor de um país}
    \definition[个]{s.}{estado; nação; país}
  \end{Phonetics}
\end{Entry}

\begin{Entry}{国人}{8,2}{⼞、⼈}
  \begin{Phonetics}{国人}{guo2ren2}
    \definition{s.}{compatriota}
  \end{Phonetics}
\end{Entry}

\begin{Entry}{国土}{8,3}{⼞、⼟}
  \begin{Phonetics}{国土}{guo2tu3}[][HSK 7-9]
    \definition{s.}{terra; território; território nacional}
  \end{Phonetics}
\end{Entry}

\begin{Entry}{国内}{8,4}{⼞、⼌}
  \begin{Phonetics}{国内}{guo2 nei4}[][HSK 3]
    \definition{s.}{interno (a um país); doméstico; lar; dentro de um determinado país}
  \end{Phonetics}
\end{Entry}

\begin{Entry}{国王}{8,4}{⼞、⽟}
  \begin{Phonetics}{国王}{guo2wang2}[][HSK 6]
    \definition[位,名,个,些]{s.}{rei; soberanos; o governante supremo de algumas monarquias antigas; nos tempos modernos, refere-se ao chefe de estado de algumas monarquias}
  \end{Phonetics}
\end{Entry}

\begin{Entry}{国外}{8,5}{⼞、⼣}
  \begin{Phonetics}{国外}{guo2 wai4}[][HSK 1]
    \definition{adj.}{externo; no exterior; fora do país; outros lugares fora do país; geralmente chamados de exterior;  exterior não é o mesmo que estrangeiro}
  \end{Phonetics}
\end{Entry}

\begin{Entry}{国民}{8,5}{⼞、⽒}
  \begin{Phonetics}{国民}{guo2 min2}[][HSK 5]
    \definition{adj.}{nacional}
    \definition[个]{s.}{membro de uma nação; povo de uma nação}
  \end{Phonetics}
\end{Entry}

\begin{Entry}{国产}{8,6}{⼞、⼇}
  \begin{Phonetics}{国产}{guo2 chan3}[][HSK 6]
    \definition{adj.}{doméstico; feito na China; produzido internamente, especificamente na China}
  \end{Phonetics}
\end{Entry}

\begin{Entry}{国会}{8,6}{⼞、⼈}
  \begin{Phonetics}{国会}{guo2 hui4}[][HSK 6]
    \definition{s.}{parlamento; congresso}
  \end{Phonetics}
\end{Entry}

\begin{Entry}{国庆}{8,6}{⼞、⼴}
  \begin{Phonetics}{国庆}{guo2 qing4}[][HSK 3]
    \definition*{s.}{Dia Nacional, o dia em que um país comemora sua independência ou fundação}
  \end{Phonetics}
\end{Entry}

\begin{Entry}{国庆节}{8,6,5}{⼞、⼴、⾋}
  \begin{Phonetics}{国庆节}{guo2qing4jie2}
    \definition*{s.}{Dia Nacional (1~de~outubro)}
  \end{Phonetics}
\end{Entry}

\begin{Entry}{国有}{8,6}{⼞、⽉}
  \begin{Phonetics}{国有}{guo2you3}[][HSK 7-9]
    \definition{v.}{pertencer ao estado; ser nacionalizado}
  \end{Phonetics}
\end{Entry}

\begin{Entry}{国防}{8,6}{⼞、⾩}
  \begin{Phonetics}{国防}{guo2fang2}[][HSK 7-9]
    \definition{s.}{defesa nacional; as instalações humanas, materiais e militares que um país possui para defender sua soberania territorial e impedir invasões estrangeiras}
  \end{Phonetics}
\end{Entry}

\begin{Entry}{国际}{8,7}{⼞、⾩}
  \begin{Phonetics}{国际}{guo2ji4}[][HSK 2]
    \definition{adj.}{internacional; entre países; entre nações}
    \definition{s.}{internacional; o mundo; entre nações; entre países de todo o mundo}
  \end{Phonetics}
\end{Entry}

\begin{Entry}{国际儿童节}{8,7,2,12,5}{⼞、⾩、⼉、⽴、⾋}
  \begin{Phonetics}{国际儿童节}{guo2ji4 er2tong2jie2}
    \definition*{s.}{Dia Internacional das Crianças (1~de~junho)}
  \end{Phonetics}
\end{Entry}

\begin{Entry}{国际妇女节}{8,7,6,3,5}{⼞、⾩、⼥、⼥、⾋}
  \begin{Phonetics}{国际妇女节}{guo2ji4 fu4nv3jie2}
    \definition*{s.}{Dia Internacional das Mulheres (8~de~março)}
  \end{Phonetics}
\end{Entry}

\begin{Entry}{国际劳动节}{8,7,7,6,5}{⼞、⾩、⼒、⼒、⾋}
  \begin{Phonetics}{国际劳动节}{guo2ji4 lao2dong4 jie2}
    \definition*{s.}{Dia Internacional dos Trabalhadores (1~de~maio)}
  \end{Phonetics}
\end{Entry}

\begin{Entry}{国学}{8,8}{⼞、⼦}
  \begin{Phonetics}{国学}{guo2xue2}[][HSK 7-9]
    \definition*{s.}{Arcaico: O Colégio Imperial}
    \definition{s.}{estudos da cultura clássica chinesa (história, filosofia, literatura, língua, etc.) | cultura nacional chinesa | estudos da antiga civilização chinesa}
  \end{Phonetics}
\end{Entry}

\begin{Entry}{国宝}{8,8}{⼞、⼧}
  \begin{Phonetics}{国宝}{guo2bao3}[][HSK 7-9]
    \definition[件]{s.}{tesouro nacional}
  \end{Phonetics}
\end{Entry}

\begin{Entry}{国画}{8,8}{⼞、⽥}
  \begin{Phonetics}{国画}{guo2hua4}[][HSK 7-9]
    \definition[幅,张,卷]{s.}{pintura tradicional chinesa | arte chinesa | pintura nacional}
  \end{Phonetics}
\end{Entry}

\begin{Entry}{国语}{8,9}{⼞、⾔}
  \begin{Phonetics}{国语}{guo2yu3}
    \definition*{s.}{Língua Chinesa (Mandarim), enfatizando sua natureza nacional}
  \end{Phonetics}
\end{Entry}

\begin{Entry}{国家}{8,10}{⼞、⼧}
  \begin{Phonetics}{国家}{guo2jia1}[][HSK 1]
    \definition[个]{s.}{país; estado; nação; um lugar reconhecido internacionalmente e com soberania independente, incluindo as pessoas e as instituições administrativas desse lugar}
  \end{Phonetics}
\end{Entry}

\begin{Entry}{国宾馆}{8,10,11}{⼞、⼧、⾷}
  \begin{Phonetics}{国宾馆}{guo2bin1guan3}
    \definition{s.}{pousada estadual}
  \end{Phonetics}
\end{Entry}

\begin{Entry}{国情}{8,11}{⼞、⼼}
  \begin{Phonetics}{国情}{guo2qing2}[][HSK 7-9]
    \definition{s.}{condição (ou estado) do país; condições nacionais; as condições e características básicas da natureza social, política, economia, cultura etc. de um país também se referem especificamente às condições e características básicas de um país em um determinado período de tempo}
  \end{Phonetics}
\end{Entry}

\begin{Entry}{国旗}{8,14}{⼞、⽅}
  \begin{Phonetics}{国旗}{guo2qi2}
    \definition[面]{s.}{bandeira (de um país)}
  \end{Phonetics}
\end{Entry}

\begin{Entry}{国歌}{8,14}{⼞、⽋}
  \begin{Phonetics}{国歌}{guo2 ge1}[][HSK 6]
    \definition[首,支]{s.}{hino nacional; o hino nacional da China, oficialmente designado pelo estado como a música que representa o país, é "Marcha dos Voluntários"}
  \end{Phonetics}
\end{Entry}

\begin{Entry}{国徽}{8,17}{⼞、⼻}
  \begin{Phonetics}{国徽}{guo2hui1}[][HSK 7-9]
    \definition{s.}{emblema nacional; o emblema nacional da China, oficialmente designado pelo estado para representar o país, apresenta a Praça da Paz Celestial sob o céu brilhante de cinco estrelas, cercada por espigas de grãos e engrenagens}
  \end{Phonetics}
\end{Entry}

\begin{Entry}{国籍}{8,20}{⼞、⽵}
  \begin{Phonetics}{国籍}{guo2ji2}[][HSK 5]
    \definition[个]{s.}{nacionalidade; cidadania; refere-se à identidade de um indivíduo como pertencente a um Estado | identidade nacional (de um avião, navio, etc.)}
  \end{Phonetics}
\end{Entry}

\begin{Entry}{图}{8}{⼞}
  \begin{Phonetics}{图}{tu2}[][HSK 3]
    \definition*{s.}{Sobrenome Tu}
    \definition[张]{s.}{mapa; gráfico; imagem; desenho | plano; esquema; tentativa}
    \definition{v.}{procurar; perseguir; esperar obter| desenhar; retratar; pintar | imaginar; planejar; pensar; maquinar}
  \end{Phonetics}
\end{Entry}

\begin{Entry}{图书}{8,4}{⼞、⼄}
  \begin{Phonetics}{图书}{tu2 shu1}[][HSK 6]
    \definition{s.}{livros; um termo geral para publicações como livros e álbuns de imagens}[这些图书都可以借阅。===Esses livros estão disponíveis para empréstimo.]
  \end{Phonetics}
\end{Entry}

\begin{Entry}{图书馆}{8,4,11}{⼞、⼄、⾷}
  \begin{Phonetics}{图书馆}{tu2shu1guan3}[][HSK 1]
    \definition[个,家]{s.}{biblioteca; instituição que coleta, organiza e armazena livros e materiais para leitura e consulta}
  \end{Phonetics}
\end{Entry}

\begin{Entry}{图片}{8,4}{⼞、⽚}
  \begin{Phonetics}{图片}{tu2 pian4}[][HSK 2]
    \definition[张,幅]{s.}{imagem; fotografia; um termo geral para imagens, fotografias, decalques, etc. usados para ilustrar algo}
  \end{Phonetics}
\end{Entry}

\begin{Entry}{图画}{8,8}{⼞、⽥}
  \begin{Phonetics}{图画}{tu2 hua4}[][HSK 3]
    \definition[幅,张,套]{s.}{desenho; imagem; pintura}
  \end{Phonetics}
\end{Entry}

\begin{Entry}{图案}{8,10}{⼞、⽊}
  \begin{Phonetics}{图案}{tu2'an4}[][HSK 4]
    \definition[种,个]{s.}{padrão; desenho; padrões e gráficos usados para decoração de edifícios, tecidos, artes e artesanato, etc.}
  \end{Phonetics}
\end{Entry}

\begin{Entry}{坡}{8}{⼟}
  \begin{Phonetics}{坡}{po1}[][HSK 6]
    \definition{adj.}{inclinado}
    \definition{s.}{declive | encosta}
  \end{Phonetics}
\end{Entry}

\begin{Entry}{坦}{8}{⼟}
  \begin{Phonetics}{坦}{tan3}
    \definition*{s.}{Sobrenome Tan}
    \definition{adj.}{nivelado; suave; plano | calmo; composto | aberto; sincero; franco}
  \end{Phonetics}
\end{Entry}

\begin{Entry}{坦克}{8,7}{⼟、⼗}
  \begin{Phonetics}{坦克}{tan3ke4}
    \definition{s.}{(empréstimo linguístico) tanque (veículo militar)}
  \end{Phonetics}
\end{Entry}

\begin{Entry}{垂}{8}{⼠}
  \begin{Phonetics}{垂}{chui2}[][HSK 7-9]
    \definition{adv.}{perto de; uma palavra respeitosa usada para se referir às ações de outros (geralmente anciãos ou superiores) em relação a si mesmo.}
    \definition{v.}{cair; deixar cair; pendurar (objeto de cabeça para baixo)  | Literário: (mais velhos ou superiores) condescender; um termo respeitoso usado para se referir às ações de outros (geralmente anciãos ou superiores) em relação a si mesmo | transmitir; legar à posteridade; passar para as gerações posteriores}
  \end{Phonetics}
\end{Entry}

\begin{Entry}{垂头丧气}{8,5,8,4}{⼠、⼤、⼗、⽓}
  \begin{Phonetics}{垂头丧气}{chui2tou2-sang4qi4}[][HSK 7-9]
    \definition{v.}{estar desanimado; estar abatido; abaixar a cabeça com um olhar abatido}
  \end{Phonetics}
\end{Entry}

\begin{Entry}{垃}{8}{⼟}
  \begin{Phonetics}{垃}{la1}
    \definition[堆]{s.}{lixo}
  \end{Phonetics}
\end{Entry}

\begin{Entry}{垃圾}{8,6}{⼟、⼟}
  \begin{Phonetics}{垃圾}{la1 ji1}[][HSK 4]
    \definition{adj.}{lixo; inútil, ruim ou prejudicial}
    \definition[袋,桶,堆,车,片]{s.}{entulho; lixo; refugo; rejeito; resíduo; coisa inútil que é jogada fora; metáfora para alguém ou algo que perdeu seu valor ou serve a um propósito ruim}
  \end{Phonetics}
\end{Entry}

\begin{Entry}{垃圾工}{8,6,3}{⼟、⼟、⼯}
  \begin{Phonetics}{垃圾工}{la1ji1gong1}
    \definition{s.}{lixeiro | gari}
  \end{Phonetics}
\end{Entry}

\begin{Entry}{垃圾车}{8,6,4}{⼟、⼟、⾞}
  \begin{Phonetics}{垃圾车}{la1ji1che1}
    \definition{s.}{caminhão de lixo}
  \end{Phonetics}
\end{Entry}

\begin{Entry}{垃圾电邮}{8,6,5,7}{⼟、⼟、⽥、⾢}
  \begin{Phonetics}{垃圾电邮}{la1ji1 dian4you2}
    \definition{s.}{\emph{e-mail} de \emph{spam}}
  \seealsoref{垃圾邮件}{la1ji1 you2jian4}
  \end{Phonetics}
\end{Entry}

\begin{Entry}{垃圾邮件}{8,6,7,6}{⼟、⼟、⾢、⼈}
  \begin{Phonetics}{垃圾邮件}{la1ji1 you2jian4}
    \definition{s.}{\emph{spam}, \emph{e-mail} não solicitado}
  \seealsoref{垃圾电邮}{la1ji1 dian4you2}
  \end{Phonetics}
\end{Entry}

\begin{Entry}{垃圾食品}{8,6,9,9}{⼟、⼟、⾷、⼝}
  \begin{Phonetics}{垃圾食品}{la1ji1shi2pin3}
    \definition{s.}{\emph{junk food}}
  \end{Phonetics}
\end{Entry}

\begin{Entry}{垃圾堆}{8,6,11}{⼟、⼟、⼟}
  \begin{Phonetics}{垃圾堆}{la1ji1dui1}
    \definition{s.}{depósito de lixo}
  \end{Phonetics}
\end{Entry}

\begin{Entry}{垃圾筒}{8,6,12}{⼟、⼟、⽵}
  \begin{Phonetics}{垃圾筒}{la1ji1tong3}
    \definition{s.}{cesto de lixo}
  \end{Phonetics}
\end{Entry}

\begin{Entry}{垃圾箱}{8,6,15}{⼟、⼟、⾋}
  \begin{Phonetics}{垃圾箱}{la1ji1xiang1}
    \definition{s.}{cesto de lixo}
  \end{Phonetics}
\end{Entry}

\begin{Entry}{备}{8}{⼡}
  \begin{Phonetics}{备}{bei4}
    \definition*{s.}{Sobrenome Bei}
    \definition{adv.}{totalmente; de todas as maneiras possíveis | todos; tudo}
    \definition{s.}{equipamento}
    \definition{v.}{estar equipar com; ter; possuir | preparar; ficar pronto | providenciar (ou preparar) contra; tomar precauções contra}
  \end{Phonetics}
\end{Entry}

\begin{Entry}{备用}{8,5}{⼡、⽤}
  \begin{Phonetics}{备用}{bei4yong4}[][HSK 7-9]
    \definition{v.}{reservar; guardar algo para uso futuro}
  \end{Phonetics}
\end{Entry}

\begin{Entry}{备份}{8,6}{⼡、⼈}
  \begin{Phonetics}{备份}{bei4fen4}
    \definition{s.}{cópia de segurança | \emph{backup}}
  \end{Phonetics}
\end{Entry}

\begin{Entry}{备受}{8,8}{⼡、⼜}
  \begin{Phonetics}{备受}{bei4shou4}[][HSK 7-9]
    \definition{v.}{experimentar plenamente (o bem ou o mal)}
  \end{Phonetics}
\end{Entry}

\begin{Entry}{备胎}{8,9}{⼡、⾁}
  \begin{Phonetics}{备胎}{bei4tai1}
    \definition{s.}{pneu sobressalente | (gíria) substituto}
  \end{Phonetics}
\end{Entry}

\begin{Entry}{备课}{8,10}{⼡、⾔}
  \begin{Phonetics}{备课}{bei4/ke4}[][HSK 7-9]
    \definition{v.+compl.}{(professor) preparar aulas}
  \end{Phonetics}
\end{Entry}

\begin{Entry}{夜}{8}{⼣}
  \begin{Phonetics}{夜}{ye4}[][HSK 2]
    \definition{s.}{noite; tarde; noturno; o período do anoitecer ao amanhecer (em oposição a 日 ou 昼); em meteorologia, refere-se especificamente ao período das 20h do dia atual às 8h do dia seguinte}
  \seealsoref{日}{ri4}
  \seealsoref{昼}{zhou4}
  \end{Phonetics}
\end{Entry}

\begin{Entry}{夜生活}{8,5,9}{⼣、⽣、⽔}
  \begin{Phonetics}{夜生活}{ye4sheng1huo2}
    \definition{s.}{vida noturna}
  \end{Phonetics}
\end{Entry}

\begin{Entry}{夜鸟}{8,5}{⼣、⿃}
  \begin{Phonetics}{夜鸟}{ye4niao3}
    \definition{s.}{ave noturna}
  \end{Phonetics}
\end{Entry}

\begin{Entry}{夜场}{8,6}{⼣、⼟}
  \begin{Phonetics}{夜场}{ye4chang3}
    \definition{s.}{show noturno (em um teatro, etc.) | local de entretenimento noturno (bar, boate, discoteca, etc.)}
  \end{Phonetics}
\end{Entry}

\begin{Entry}{夜里}{8,7}{⼣、⾥}
  \begin{Phonetics}{夜里}{ye4li5}[][HSK 2]
    \definition{s.}{noturno; à noite; o período do anoitecer ao amanhecer}
  \end{Phonetics}
\end{Entry}

\begin{Entry}{夜间}{8,7}{⼣、⾨}
  \begin{Phonetics}{夜间}{ye4 jian1}[][HSK 5]
    \definition{s.}{noite; à noite; noturno; durante a noite}
  \end{Phonetics}
\end{Entry}

\begin{Entry}{夜夜}{8,8}{⼣、⼣}
  \begin{Phonetics}{夜夜}{ye4ye4}
    \definition{adv.}{toda noite}
  \end{Phonetics}
\end{Entry}

\begin{Entry}{夜店}{8,8}{⼣、⼴}
  \begin{Phonetics}{夜店}{ye4dian4}
    \definition{s.}{boate | \emph{nightclub}}
  \end{Phonetics}
\end{Entry}

\begin{Entry}{夜晚}{8,11}{⼣、⽇}
  \begin{Phonetics}{夜晚}{ye4wan3}
    \definition[个]{s.}{noite}
  \end{Phonetics}
\end{Entry}

\begin{Entry}{夜深人静}{8,11,2,14}{⼣、⽔、⼈、⾭}
  \begin{Phonetics}{夜深人静}{ye4shen1ren2jing4}
    \definition{expr.}{``Na calada da noite.''; ``No silêncio (ou silêncio) da noite.''}
  \end{Phonetics}
\end{Entry}

\begin{Entry}{夜幕}{8,13}{⼣、⼱}
  \begin{Phonetics}{夜幕}{ye4mu4}
    \definition{s.}{cortina da noite}
  \end{Phonetics}
\end{Entry}

\begin{Entry}{奇}{8}{⼤}
  \begin{Phonetics}{奇}{qi2}
    \definition{adj.}{ímpar (número); singular; solteiro; não em pares (ao contrário de 偶)}
    \definition{s.}{lotes ímpares; quantidade fracionária (acima daquela mencionada em um número redondo)}
  \seealsoref{偶}{ou3}
  \end{Phonetics}
\end{Entry}

\begin{Entry}{奇妙}{8,7}{⼤、⼥}
  \begin{Phonetics}{奇妙}{qi2miao4}[][HSK 6]
    \definition{adj.}{maravilhoso; milagroso; intrigante; muito inteligente e engenhoso (usado principalmente para descrever coisas interessantes e novas)}
  \end{Phonetics}
\end{Entry}

\begin{Entry}{奇怪}{8,8}{⼤、⼼}
  \begin{Phonetics}{奇怪}{qi2guai4}[][HSK 3]
    \definition{adj.}{estranho; diferente do habitual; raramente visto, até um pouco irracional | estranho; esquisito; a descrição é diferente do imaginado e é difícil de entender}
    \definition{v.}{ficar perplexo; maravilhar-se; sentir-se surpreso; sentir-se estranho; sentir-se incompreensível}
  \end{Phonetics}
\end{Entry}

\begin{Entry}{奇迹}{8,9}{⼤、⾡}
  \begin{Phonetics}{奇迹}{qi2ji4}
    \definition[个,种]{s.}{milagre; maravilha; coisas extraordinárias inimagináveis}
  \end{Phonetics}
\end{Entry}

\begin{Entry}{奉}{8}{⼤}
  \begin{Phonetics}{奉}{feng4}
    \definition*{s.}{Sobrenome Feng}
    \definition{v.}{Literário: dedicar ou presentear com respeito | receber (pedidos, instruções, etc.) | Literário: estimar; reverenciar | Litrário: acreditar em  | esperar; atender; servir}
  \end{Phonetics}
\end{Entry}

\begin{Entry}{奉献}{8,13}{⼤、⽝}
  \begin{Phonetics}{奉献}{feng4xian4}[][HSK 6]
    \definition{v.}{dedicar; oferecer como tributo; apresentar com todo respeito; entregar respeitosamente}
  \end{Phonetics}
\end{Entry}

\begin{Entry}{奋}{8}{⼤}
  \begin{Phonetics}{奋}{fen4}
    \definition{adv.}{energicamente; com força e espírito}
    \definition{v.}{esforçar-se; agir vigorosamente; preparar-se | levantar | aplicar energia; resolver; animar-se | acenar; sacudir; levantar}
  \end{Phonetics}
\end{Entry}

\begin{Entry}{奋力}{8,2}{⼤、⼒}
  \begin{Phonetics}{奋力}{fen4li4}[][HSK 7-9]
    \definition{v.}{fazer tudo o que puder; não poupar esforços}
  \end{Phonetics}
\end{Entry}

\begin{Entry}{奋斗}{8,4}{⼤、⽃}
  \begin{Phonetics}{奋斗}{fen4dou4}[][HSK 4]
    \definition{v.}{lutar; esforçar-se; batalhar; trabalhar duro para atingir um determinado objetivo}
  \end{Phonetics}
\end{Entry}

\begin{Entry}{奋勇}{8,9}{⼤、⼒}
  \begin{Phonetics}{奋勇}{fen4yong3}[][HSK 7-9]
    \definition{v.}{reunir toda a coragem e energia; criar coragem}
  \end{Phonetics}
\end{Entry}

\begin{Entry}{奋战}{8,9}{⼤、⼽}
  \begin{Phonetics}{奋战}{fen4zhan4}
    \definition{v.}{lutar bravamente | trabalhar duro}
  \end{Phonetics}
\end{Entry}

\begin{Entry}{奔}{8}{⼤}
  \begin{Phonetics}{奔}{ben1}
    \definition{v.}{correr rápido; correr com pressa | apressar | fugir; escapar | galopar | fugir; termo antigo para uma mulher que foge com um homem}
  \end{Phonetics}
  \begin{Phonetics}{奔}{ben4}[][HSK 7-9]
    \definition{prep.}{em direção a}
    \definition{v.}{ir direto em direção a; seguir em direção a; ir direto para o seu destino | aproximar-se; estar prestes a | estar ocupado correndo por aí; correr por algo}
  \end{Phonetics}
\end{Entry}

\begin{Entry}{奔驰}{8,6}{⼤、⾺}
  \begin{Phonetics}{奔驰}{ben1chi2}
    \definition*{s.}{Benz de Mercedes-Benz}
    \definition{v.}{acelerar; galopar; (carro, cavalo, etc.) mover-se ou correr rapidamente}
  \seealsoref{梅赛德斯-奔驰}{mei2sai4de2si1-ben1chi2}
  \end{Phonetics}
\end{Entry}

\begin{Entry}{奔波}{8,8}{⼤、⽔}
  \begin{Phonetics}{奔波}{ben1bo1}[][HSK 7-9]
    \definition{v.}{correr; estar ocupado correndo; correr para frente e para trás, com dificuldade e ocupado}
  \end{Phonetics}
\end{Entry}

\begin{Entry}{奔赴}{8,9}{⼤、⾛}
  \begin{Phonetics}{奔赴}{ben1fu4}[][HSK 7-9]
    \definition{v.}{correr para; apressar-se para; correr em direção a (um certo destino)}
  \end{Phonetics}
\end{Entry}

\begin{Entry}{奔跑}{8,12}{⼤、⾜}
  \begin{Phonetics}{奔跑}{ben1 pao3}[][HSK 6]
    \definition{v.}{correr; correr muito rápido, com uma gama de aplicações mais ampla do que 奔驰, usado principalmente na linguagem falada}
  \seealsoref{奔驰}{ben1chi2}
  \end{Phonetics}
\end{Entry}

\begin{Entry}{妹}{8}{⼥}
  \begin{Phonetics}{妹}{mei4}[][HSK 1]
    \definition*{s.}{Sobrenome Mei}
    \definition[个]{s.}{irmã mais nova | parente do sexo feminino da mesma geração | jovem garota; jovem mulher ou menina}
  \seealsoref{妹妹}{mei4 mei5}
  \end{Phonetics}
\end{Entry}

\begin{Entry}{妹夫}{8,4}{⼥、⼤}
  \begin{Phonetics}{妹夫}{mei4fu5}
    \definition{s.}{marido da irmã mais nova}
  \end{Phonetics}
\end{Entry}

\begin{Entry}{妹妹}{8,8}{⼥、⼥}
  \begin{Phonetics}{妹妹}{mei4 mei5}[][HSK 1]
    \definition[个]{s.}{irmã mais nova}
  \end{Phonetics}
\end{Entry}

\begin{Entry}{妻}{8}{⼥}
  \begin{Phonetics}{妻}{qi1}
    \definition{s.}{esposa}
  \end{Phonetics}
  \begin{Phonetics}{妻}{qi4}
    \definition{v.}{casar uma mulher com (alguém)}
  \end{Phonetics}
\end{Entry}

\begin{Entry}{妻子}{8,3}{⼥、⼦}
  \begin{Phonetics}{妻子}{qi1zi3}
    \definition[个]{s.}{esposa e filhos; (chinês antigo) refere-se a esposas, filhos e filhas}
  \end{Phonetics}
  \begin{Phonetics}{妻子}{qi1zi5}[][HSK 4]
    \definition[个]{s.}{esposa (não é usado como um termo carinhoso)}
  \end{Phonetics}
\end{Entry}

\begin{Entry}{始}{8}{⼥}
  \begin{Phonetics}{始}{shi3}
    \definition*{s.}{Sobrenome Shi}
    \definition{adv.}{somente então; não\dots até}
    \definition{s.}{começo; início}
    \definition{v.}{começar; iniciar}
  \end{Phonetics}
\end{Entry}

\begin{Entry}{始终}{8,8}{⼥、⽷}
  \begin{Phonetics}{始终}{shi3zhong1}[][HSK 3]
    \definition{adv.}{sempre; o tempo todo; durante todo; do começo ao fim; indica continuidade do início ao fim}
    \definition{s.}{todo o processo do começo ao fim}
  \end{Phonetics}
\end{Entry}

\begin{Entry}{姐}{8}{⼥}
  \begin{Phonetics}{姐}{jie3}[][HSK 1]
    \definition[个,位]{s.}{irmã mais velha; irmã | termo genérico para mulheres jovens | mulheres da mesma geração que são mais velhas do que você (geralmente não inclui aquelas que podem ser chamadas de cunhadas) | um título respeitoso para mulheres jovens profissionais em determinados cargos}
  \seealsoref{姐姐}{jie3 jie5}
  \end{Phonetics}
\end{Entry}

\begin{Entry}{姐夫}{8,4}{⼥、⼤}
  \begin{Phonetics}{姐夫}{jie3fu5}
    \definition{s.}{marido da irmã mais velha}
  \end{Phonetics}
\end{Entry}

\begin{Entry}{姐妹}{8,8}{⼥、⼥}
  \begin{Phonetics}{姐妹}{jie3 mei4}[][HSK 4]
    \definition[个]{s.}{irmãs}
  \end{Phonetics}
\end{Entry}

\begin{Entry}{姐姐}{8,8}{⼥、⼥}
  \begin{Phonetics}{姐姐}{jie3 jie5}[][HSK 1]
    \definition[个]{s.}{irmã mais velha}
  \end{Phonetics}
\end{Entry}

\begin{Entry}{姑}{8}{⼥}
  \begin{Phonetics}{姑}{gu1}
    \definition{adv.}{provisoriamente; por enquanto}
    \definition[个,位,名,些]{s.}{irmã do pai; tia | irmã do marido; cunhada | mãe do marido; sogra | freira; mulher que exerce uma ocupação religiosa | a irmã do pai de alguém | mulheres jovens (no campo)}
  \end{Phonetics}
\end{Entry}

\begin{Entry}{姑且}{8,5}{⼥、⼀}
  \begin{Phonetics}{姑且}{gu1qie3}
    \definition{adv.}{provisoriamente | por enquanto}
  \end{Phonetics}
\end{Entry}

\begin{Entry}{姑姑}{8,8}{⼥、⼥}
  \begin{Phonetics}{姑姑}{gu1gu5}[][HSK 6]
    \definition[个,位,名]{s.}{tia; tia paterna}
  \end{Phonetics}
\end{Entry}

\begin{Entry}{姑娘}{8,10}{⼥、⼥}
  \begin{Phonetics}{姑娘}{gu1niang5}[][HSK 3]
    \definition[位,名,个,些]{s.}{menina; jovem senhora; mulher solteira | filha}
  \end{Phonetics}
\end{Entry}

\begin{Entry}{姓}{8}{⼥}
  \begin{Phonetics}{姓}{xing4}[][HSK 2]
    \definition[个]{s.}{sobrenome; nome de família; um caractere que representa um sistema familiar, os chineses colocam o sobrenome em primeiro lugar e o nome em segundo}
    \definition{v.}{ter como sobrenome; tratar um ou mais caracteres como sobrenome}
  \end{Phonetics}
\end{Entry}

\begin{Entry}{姓氏}{8,4}{⼥、⽒}
  \begin{Phonetics}{姓氏}{xing4shi4}
    \definition{s.}{sobrenome}
  \end{Phonetics}
\end{Entry}

\begin{Entry}{姓名}{8,6}{⼥、⼝}
  \begin{Phonetics}{姓名}{xing4ming2}[][HSK 2]
    \definition{s.}{nome; nome completo; sobrenome e nome próprio}
  \end{Phonetics}
\end{Entry}

\begin{Entry}{委}{8}{⼥}
  \begin{Phonetics}{委}{wei1}
    \definition{adj./adv.}{o mesmo que 逶 em 逶迤 sinuoso, curvo}
  \seealsoref{逶}{wei1}
  \seealsoref{逶迤}{wei1yi2}
  \end{Phonetics}
  \begin{Phonetics}{委}{wei3}
    \definition*{s.}{Sobrenome Wei}
    \definition{adj.}{indireto; desviado | apático; abatido | sinuoso; tortuoso | desanimado; apático; sem inspiração}
    \definition{adv.}{realmente; certamente; na verdade}
    \definition{s.}{membro do comitê | comitê; comissão; conselho}
    \definition{v.}{confiar; nomear |  jogar fora; deixar de lado | culpar os outros | confiar | descartar; abandonar | mudar; empurrar | acumular}
  \end{Phonetics}
\end{Entry}

\begin{Entry}{委内瑞拉}{8,4,13,8}{⼥、⼌、⽟、⼿}
  \begin{Phonetics}{委内瑞拉}{wei3nei4rui4la1}
    \definition*{s.}{Venezuela}
  \end{Phonetics}
\end{Entry}

\begin{Entry}{委托}{8,6}{⼥、⼿}
  \begin{Phonetics}{委托}{wei3tuo1}[][HSK 5]
    \definition{v.}{confiar; confiar uma tarefa a outra pessoa ou instituição (para que seja realizada)}
  \end{Phonetics}
\end{Entry}

\begin{Entry}{季}{8}{⼦}
  \begin{Phonetics}{季}{ji4}[][HSK 4]
    \definition*{s.}{Sobrenome Ji}
    \definition{s.}{estação; o ano é dividido em quatro estações, primavera, verão, outono e inverno, e uma estação dura três meses | temporada | o fim de uma era | o último mês de uma temporada | o quarto ou mais novo entre irmãos; último na ordem de precedência}
  \end{Phonetics}
\end{Entry}

\begin{Entry}{季节}{8,5}{⼦、⾋}
  \begin{Phonetics}{季节}{ji4jie2}[][HSK 4]
    \definition[个]{s.}{estação (clima); um período característico do ano}
  \end{Phonetics}
\end{Entry}

\begin{Entry}{季度}{8,9}{⼦、⼴}
  \begin{Phonetics}{季度}{ji4du4}[][HSK 4]
    \definition[个]{s.}{trimestre; período de tempo trimestral}
  \end{Phonetics}
\end{Entry}

\begin{Entry}{孤}{8}{⼦}
  \begin{Phonetics}{孤}{gu1}
    \definition*{s.}{Sobrenome Gu}
    \definition{adj.}{sozinho; solitário; isolado}
    \definition{pron.}{eu; meu humilde eu (usado por príncipes feudais); título autoproclamado dos príncipes feudais}
    \definition[个,名,位]{s.}{órfão}
  \end{Phonetics}
\end{Entry}

\begin{Entry}{孤儿}{8,2}{⼦、⼉}
  \begin{Phonetics}{孤儿}{gu1 er2}[][HSK 6]
    \definition[个,名,位]{s.}{órfão; criança sem pais; crianças que perderam os pais}
  \end{Phonetics}
\end{Entry}

\begin{Entry}{孤立}{8,5}{⼦、⽴}
  \begin{Phonetics}{孤立}{gu1li4}[][HSK 7-9]
    \definition{adj.}{isolado; condenado ao ostracismo; descreve a falta de ajuda e simpatia}
    \definition{v.}{isolar; ostracizar; privar uma pessoa de ajuda, apoio e confiança}
  \end{Phonetics}
\end{Entry}

\begin{Entry}{孤单}{8,8}{⼦、⼗}
  \begin{Phonetics}{孤单}{gu1dan1}[][HSK 7-9]
    \definition{adj.}{sozinho; solitário | fraco; inadequado; descreve um pequeno número de pessoas e poder fraco}
  \end{Phonetics}
\end{Entry}

\begin{Entry}{孤陋寡闻}{8,8,14,9}{⼦、⾩、⼧、⾨}
  \begin{Phonetics}{孤陋寡闻}{gu1lou4-gua3wen2}[][HSK 7-9]
    \definition{expr.}{ignorante e mal informado | ignorante e inexperiente | mal informado e tacanho}
  \end{Phonetics}
\end{Entry}

\begin{Entry}{孤独}{8,9}{⼦、⽝}
  \begin{Phonetics}{孤独}{gu1du2}[][HSK 6]
    \definition{adj.}{sozinho; solitário}
  \end{Phonetics}
\end{Entry}

\begin{Entry}{孤零零}{8,13,13}{⼦、⾬、⾬}
  \begin{Phonetics}{孤零零}{gu1ling2ling2}[][HSK 7-9]
    \definition{adj.}{solitário; sozinho; completamente sozinho; sem apoio ou companhia}
  \end{Phonetics}
\end{Entry}

\begin{Entry}{学}{8}{⼦}
  \begin{Phonetics}{学}{xue2}[][HSK 1]
    \definition[所]{s.}{aprendizagem; conhecimento; sabedoria; erudição | objeto de estudo; ramo do conhecimento | escola; faculdade | teoria; doutrina}
    \definition{v.}{estudar; aprender | imitar; copiar}
  \end{Phonetics}
\end{Entry}

\begin{Entry}{学习}{8,3}{⼦、⼄}
  \begin{Phonetics}{学习}{xue2xi2}[][HSK 1]
    \definition{s.}{estudo}
    \definition{v.}{estudar; aprender; adquirir conhecimentos ou habilidades através da leitura, da audição, da pesquisa e da prática}
  \end{Phonetics}
\end{Entry}

\begin{Entry}{学分}{8,4}{⼦、⼑}
  \begin{Phonetics}{学分}{xue2fen1}[][HSK 4]
    \definition{s.}{créditos de um curso; uma unidade de medida do peso e do tempo do curso no ensino superior; cada curso vale um crédito para uma aula por semana durante um semestre; alunos devem concluir o número necessário de créditos para se formar}
  \end{Phonetics}
\end{Entry}

\begin{Entry}{学术}{8,5}{⼦、⽊}
  \begin{Phonetics}{学术}{xue2shu4}[][HSK 4]
    \definition[种]{s.}{aprendizagem; aprendizado; ciências; aprendizado sistemático e especializado}
  \end{Phonetics}
\end{Entry}

\begin{Entry}{学生}{8,5}{⼦、⽣}
  \begin{Phonetics}{学生}{xue2sheng5}[][HSK 1]
    \definition{s.}{aluno; estudante; pupilo}
  \end{Phonetics}
\end{Entry}

\begin{Entry}{学生证}{8,5,7}{⼦、⽣、⾔}
  \begin{Phonetics}{学生证}{xue2sheng5zheng4}
    \definition{s.}{cartão de identidade de estudante}
  \end{Phonetics}
\end{Entry}

\begin{Entry}{学会}{8,6}{⼦、⼈}
  \begin{Phonetics}{学会}{xue2 hui4}[][HSK 6]
    \definition[个]{s.}{sociedade; instituto; sociedade científica; um grupo acadêmico composto por pessoas que estudam um determinado assunto, como a Sociedade de Física, a Sociedade de Biologia, etc.}
    \definition{v.}{aprender; dominar; aprender e aplicar}
  \end{Phonetics}
\end{Entry}

\begin{Entry}{学好}{8,6}{⼦、⼥}
  \begin{Phonetics}{学好}{xue2hao3}
    \definition{v.}{seguir bons exemplos | aprender bem}
  \end{Phonetics}
\end{Entry}

\begin{Entry}{学年}{8,6}{⼦、⼲}
  \begin{Phonetics}{学年}{xue2 nian2}[][HSK 4]
    \definition{s.}{ano letivo; ano acadêmico}
  \end{Phonetics}
\end{Entry}

\begin{Entry}{学问}{8,6}{⼦、⾨}
  \begin{Phonetics}{学问}{xue2wen4}[][HSK 4]
    \definition[门,种,个,项]{s.}{aprendizado, conhecimento, erudição; a compreensão correta do mundo objetivo que alguém tem | conhecimento; aprendizado sistemático; conhecimento sistemático sobre algo ou uma ciência que pode ser aprendido em um livro ou em uma experiência prática}
  \end{Phonetics}
\end{Entry}

\begin{Entry}{学位}{8,7}{⼦、⼈}
  \begin{Phonetics}{学位}{xue2wei4}[][HSK 5]
    \definition[个]{s.}{grau; grau acadêmico; título concedido com base no nível acadêmico profissional, como doutorado, mestrado, etc.}
  \end{Phonetics}
\end{Entry}

\begin{Entry}{学员}{8,7}{⼦、⼝}
  \begin{Phonetics}{学员}{xue2 yuan2}[][HSK 6]
    \definition[位,名,批,个]{s.}{estudante; estagiário; geralmente se refere a pessoas que estudam em escolas ou cursos de treinamento diferentes de faculdades, escolas de ensino médio e escolas primárias}
  \end{Phonetics}
\end{Entry}

\begin{Entry}{学时}{8,7}{⼦、⽇}
  \begin{Phonetics}{学时}{xue2 shi2}[][HSK 4]
    \definition{s.}{hora-aula; hora de aula; período}
  \end{Phonetics}
\end{Entry}

\begin{Entry}{学者}{8,8}{⼦、⽼}
  \begin{Phonetics}{学者}{xue2 zhe3}[][HSK 5]
    \definition[位]{s.}{erudito; homem culto; pessoas que fazem pesquisas acadêmicas geralmente se referem àquelas que alcançaram certo sucesso acadêmico}
  \end{Phonetics}
\end{Entry}

\begin{Entry}{学科}{8,9}{⼦、⽲}
  \begin{Phonetics}{学科}{xue2 ke1}[][HSK 5]
    \definition[门,级]{s.}{ramo do aprendizado; disciplina | disciplina escolar; curso de estudo | cursos teóricos oferecidos em treinamento militar ou físico (oposto a 术科)  | disciplina acadêmica | curso | assunto; tema}
  \seealsoref{术科}{shu4ke1}
  \end{Phonetics}
\end{Entry}

\begin{Entry}{学费}{8,9}{⼦、⾙}
  \begin{Phonetics}{学费}{xue2 fei4}[][HSK 3]
    \definition[笔]{s.}{mensalidade (taxa); prêmio; taxas que os alunos devem pagar para estudar na escola, conforme estabelecido pela escola | preço pelo que se aprendeu ao custo do próprio bolso; a metáfora do preço a pagar para obter uma determinada experiência | custo; preço; todas as despesas necessárias durante o período de estudos do aluno}
  \end{Phonetics}
\end{Entry}

\begin{Entry}{学院}{8,9}{⼦、⾩}
  \begin{Phonetics}{学院}{xue2yuan4}[][HSK 1]
    \definition[个,所]{s.}{academia; instituto; um tipo de instituição de ensino superior que se concentra em uma determinada área de especialização, como faculdades de engenharia, faculdades de música, faculdades de educação, etc.}
  \end{Phonetics}
\end{Entry}

\begin{Entry}{学校}{8,10}{⼦、⽊}
  \begin{Phonetics}{学校}{xue2xiao4}[][HSK 1]
    \definition[所,个]{s.}{escola; instituição de ensino}
  \end{Phonetics}
\end{Entry}

\begin{Entry}{学期}{8,12}{⼦、⽉}
  \begin{Phonetics}{学期}{xue2qi1}[][HSK 2]
    \definition[个,段]{s.}{semestre; período escolar; um ano acadêmico é dividido em dois semestres, um semestre do início do outono até as férias de inverno e um semestre do início da primavera até as férias de verão}
  \end{Phonetics}
\end{Entry}

\begin{Entry}{宗}{8}{⼧}
  \begin{Phonetics}{宗}{zong1}
    \definition*{s.}{Sobrenome Zong}
    \definition{adj.}{do mesmo clã; da mesma família}
    \definition{clas.}{usado para matérias, cargas, etc.}
    \definition{s.}{ancestral; antepassado | clã; família | seita; facção; escola | objetivo principal; propósito | modelo; grande mestre | Datado: unidade administrativa no Tibete, equivalente a um condado | templo ancestral}
    \definition{v.}{(no trabalho acadêmico ou artístico) tomar como modelo; modelar-se em}
  \end{Phonetics}
\end{Entry}

\begin{Entry}{宗教}{8,11}{⼧、⽁}
  \begin{Phonetics}{宗教}{zong1jiao4}[][HSK 6]
    \definition[种]{s.}{religião; uma ideologia social é um reflexo ilusório do mundo objetivo, exigindo que as pessoas acreditem em Deus, no Xintoísmo, em espíritos, no carma, etc., e que depositem suas esperanças no chamado céu ou vida após a morte}
  \end{Phonetics}
\end{Entry}

\begin{Entry}{官}{8}{⼧}
  \begin{Phonetics}{官}{guan1}[][HSK 4]
    \definition*{s.}{Sobrenome Guan}
    \definition{adj.}{propriedade do governo; pertencente ao governo ou ao público | público}
    \definition[个,位,名,些]{s.}{funcionário do governo; oficial; servidor público; titular de cargo; funcionário público nomeado acima de um determinado nível | órgão (parte do tecido do corpo)}
  \end{Phonetics}
\end{Entry}

\begin{Entry}{官方}{8,4}{⼧、⽅}
  \begin{Phonetics}{官方}{guan1fang1}[][HSK 4]
    \definition{s.}{autoridade; (do ou pelo) governo | oficial (de uma organização ou instituição)}
  \end{Phonetics}
\end{Entry}

\begin{Entry}{官司}{8,5}{⼧、⼝}
  \begin{Phonetics}{官司}{guan1 si5}[][HSK 6]
    \definition[场,个]{s.}{ação judicial}
  \end{Phonetics}
\end{Entry}

\begin{Entry}{官吏}{8,6}{⼧、⼝}
  \begin{Phonetics}{官吏}{guan1li4}[][HSK 7-9]
    \definition{s.}{funcionários do governo | burocrata | oficial}
  \end{Phonetics}
\end{Entry}

\begin{Entry}{官兵}{8,7}{⼧、⼋}
  \begin{Phonetics}{官兵}{guan1bing1}[][HSK 7-9]
    \definition{s.}{oficiais e soldados | Datado: tropas governamentais}
  \end{Phonetics}
\end{Entry}

\begin{Entry}{官员}{8,7}{⼧、⼝}
  \begin{Phonetics}{官员}{guan1yuan2}[][HSK 7-9]
    \definition[名,位]{s.}{oficial; funcionários do governo de um determinado nível}
  \end{Phonetics}
\end{Entry}

\begin{Entry}{官桂}{8,10}{⼧、⽊}
  \begin{Phonetics}{官桂}{guan1gui4}
    \definition{s.}{canela; também escrito como 肉桂}
  \seealsoref{肉桂}{rou4gui4}
  \end{Phonetics}
\end{Entry}

\begin{Entry}{官僚}{8,14}{⼧、⼈}
  \begin{Phonetics}{官僚}{guan1liao2}[][HSK 7-9]
    \definition{s.}{burocrata | burocracia | oficial}
  \end{Phonetics}
\end{Entry}

\begin{Entry}{官僚主义}{8,14,5,3}{⼧、⼈、⼂、⼂}
  \begin{Phonetics}{官僚主义}{guan1liao2 zhu3yi4}[][HSK 7-9]
    \definition{s.}{burocracia; burocratismo}
  \end{Phonetics}
\end{Entry}

\begin{Entry}{定}{8}{⼧}
  \begin{Phonetics}{定}{ding4}[][HSK 4]
    \definition{adj.}{calmo; estável}
    \definition{adv.}{certamente; com certeza; definitivamente; espressa certeza ou necessidade}
    \definition{v.}{decidir; fixar; definir; determinar; ter certeza | acalmar; estabilizar; tornar estável | assinar (um jornal, etc.); reservar (assentos, ingressos, etc.); encomendar (mercadorias, etc.)}
  \end{Phonetics}
\end{Entry}

\begin{Entry}{定义}{8,3}{⼧、⼂}
  \begin{Phonetics}{定义}{ding4yi4}[][HSK 7-9]
    \definition[个,种]{s.}{definição; delimitação; uma descrição precisa e concisa das características essenciais de uma coisa ou da conotação e extensão de um conceito}
    \definition{v.}{definir}
  \end{Phonetics}
\end{Entry}

\begin{Entry}{定为}{8,4}{⼧、⼂}
  \begin{Phonetics}{定为}{ding4wei2}[][HSK 7-9]
    \definition{v.}{prescrever como; estar marcado para}
  \end{Phonetics}
\end{Entry}

\begin{Entry}{定心丸}{8,4,3}{⼧、⼼、⼂}
  \begin{Phonetics}{定心丸}{ding4xin1wan2}[][HSK 7-9]
    \definition{s.}{algo que tranquiliza a mente; alívio | algo capaz de tranquilizar a mente de alguém; algo que acalma os nervos; tranquiliza a mente (paz); palavras ou ações que podem acalmar pensamentos e emoções}
  \end{Phonetics}
\end{Entry}

\begin{Entry}{定价}{8,6}{⼧、⼈}
  \begin{Phonetics}{定价}{ding4 jia4}[][HSK 6]
    \definition{s.}{fixação de preços; preço especificado}
    \definition{v.}{fixar um preço | fazer um preço; definir um preço}
  \end{Phonetics}
\end{Entry}

\begin{Entry}{定向}{8,6}{⼧、⼝}
  \begin{Phonetics}{定向}{ding4xiang4}[][HSK 7-9]
    \definition{adj.}{direcional; orientado}
    \definition{s.}{orientação; sentido de orientação; orientação predeterminada}
    \definition{v.}{orientar}
  \end{Phonetics}
\end{Entry}

\begin{Entry}{定论}{8,6}{⼧、⾔}
  \begin{Phonetics}{定论}{ding4lun4}[][HSK 7-9]
    \definition{s.}{conclusão final | veredito final}
    \definition{v.}{concluir; chegar a uma conclusão (ou julgamento)}
  \end{Phonetics}
\end{Entry}

\begin{Entry}{定位}{8,7}{⼧、⼈}
  \begin{Phonetics}{定位}{ding4 wei4}[][HSK 6]
    \definition{s.}{posição; localização; posição medida ou definida}
    \definition{v.}{localizar; posicionar; orientar; avaliar algo; usar instrumentos para determinar a localização de objetos; definir o \emph{status} das coisas}
  \end{Phonetics}
\end{Entry}

\begin{Entry}{定时}{8,7}{⼧、⽇}
  \begin{Phonetics}{定时}{ding4 shi2}[][HSK 6]
    \definition{s.}{em um horário fixo; em intervalos regulares}
    \definition{v.}{cronometrar; fixar um tempo (para fazer algo)}
  \end{Phonetics}
\end{Entry}

\begin{Entry}{定居}{8,8}{⼧、⼫}
  \begin{Phonetics}{定居}{ding4/ju1}[][HSK 7-9]
    \definition{v.+compl.}{estabelecer-se; fixar residência; viver permanentemente em um determinado lugar}
  \end{Phonetics}
\end{Entry}

\begin{Entry}{定金}{8,8}{⼧、⾦}
  \begin{Phonetics}{定金}{ding4jin1}[][HSK 7-9]
    \definition{s.}{sinal; depósito; o mesmo que 订金}
  \seealsoref{订金}{ding4jin1}
  \end{Phonetics}
\end{Entry}

\begin{Entry}{定做}{8,11}{⼧、⼈}
  \begin{Phonetics}{定做}{ding4zuo4}[][HSK 7-9]
    \definition{v.}{ter algo feito sob encomenda (medida); feito sob medida; personalizar}
  \end{Phonetics}
\end{Entry}

\begin{Entry}{定期}{8,12}{⼧、⽉}
  \begin{Phonetics}{定期}{ding4qi1}[][HSK 3]
    \definition{adj.}{regular; periódico; em intervalos regulares; com prazo determinado; por tempo limitado}
    \definition{v.}{fixar (definir) uma data; determinar a data; confirmar a data}
  \end{Phonetics}
\end{Entry}

\begin{Entry}{宝}{8}{⼧}
  \begin{Phonetics}{宝}{bao3}[][HSK 4]
    \definition*{s.}{Sobrenome Bao}
    \definition{adj.}{antigo; precioso; estimado}
    \definition{pron.}{estimado; um termo educado usado para se referir à família, loja, etc. de alguém}
    \definition[个,件]{s.}{tesouro; objeto estimado; coisa preciosa | dinheiro; moeda; moeda antiga com furo quadrado no centro; moeda de prata}
  \end{Phonetics}
\end{Entry}

\begin{Entry}{宝贝}{8,4}{⼧、⾙}
  \begin{Phonetics}{宝贝}{bao3bei4}[][HSK 4]
    \definition{adj.}{excêntrico; estranho; imprestável; um termo depreciativo para uma pessoa incompetente ou ridícula}
    \definition[个,件]{s.}{tesouro; objeto estimado; coisa preciosa | querida; \emph{darling}; \emph{baby}; apelido para crianças}
  \end{Phonetics}
\end{Entry}

\begin{Entry}{宝石}{8,5}{⼧、⽯}
  \begin{Phonetics}{宝石}{bao3 shi2}[][HSK 4]
    \definition[颗,枚,块,粒]{s.}{gema; jóia; pedra preciosa; mineral precioso que tem um brilho lindo e uma dureza de mais de sete graus, não é afetado pela atmosfera ou por produtos químicos e pode ser usado como decoração, suporte de instrumentos ou abrasivos}
  \end{Phonetics}
\end{Entry}

\begin{Entry}{宝库}{8,7}{⼧、⼴}
  \begin{Phonetics}{宝库}{bao3ku4}[][HSK 7-9]
    \definition[座,个]{s.}{tesouro; casa de tesouro; um lugar onde coisas preciosas são armazenadas (frequentemente usado metaforicamente)}[图书馆是知识的宝库。===A biblioteca é um tesouro de conhecimento.]
  \end{Phonetics}
\end{Entry}

\begin{Entry}{宝宝}{8,8}{⼧、⼧}
  \begin{Phonetics}{宝宝}{bao3 bao5}[][HSK 4]
    \definition[个,位]{s.}{querida; \emph{darling}; \emph{baby}; apelido para crianças}
  \end{Phonetics}
\end{Entry}

\begin{Entry}{宝贵}{8,9}{⼧、⾙}
  \begin{Phonetics}{宝贵}{bao3gui4}[][HSK 4]
    \definition{adj.}{precioso; extremamente valioso, muito raro, pode ser usado para descrever coisas específicas, também pode ser usado para descrever coisas abstratas | valioso; como um tesouro}
  \end{Phonetics}
\end{Entry}

\begin{Entry}{宝藏}{8,17}{⼧、⾋}
  \begin{Phonetics}{宝藏}{bao3zang4}[][HSK 7-9]
    \definition[座,个]{s.}{depósitos preciosos (minerais); tesouros ou riquezas armazenadas, principalmente minerais}
  \end{Phonetics}
\end{Entry}

\begin{Entry}{实}{8}{⼧}
  \begin{Phonetics}{实}{shi2}
    \definition{adj.}{sólido; cheio por dentro; sem espaços vazios (oposto de 虚) | verdadeiro; real; atual; sincero | forte; eficaz; concreto; real}
    \definition{adv.}{verdadeiramente; realmente; de fato; originalmente}
    \definition{s.}{fato; realidade | semente; fruto}
    \definition{v.}{preencher}
  \seealsoref{虚}{xu1}
  \end{Phonetics}
\end{Entry}

\begin{Entry}{实力}{8,2}{⼧、⼒}
  \begin{Phonetics}{实力}{shi2li4}[][HSK 3]
    \definition{s.}{força real; geralmente se refere à força militar e econômica de um país, grupo ou indivíduo, e também se refere à capacidade de um indivíduo ou grupo em uma competição}
  \end{Phonetics}
\end{Entry}

\begin{Entry}{实习}{8,3}{⼧、⼄}
  \begin{Phonetics}{实习}{shi2xi2}[][HSK 2]
    \definition{s.}{estagiário; prática; estágio}
    \definition{v.}{aplicar e testar os conhecimentos teóricos aprendidos no trabalho prático, a fim de exercitar a capacidade profissional}
  \end{Phonetics}
\end{Entry}

\begin{Entry}{实用}{8,5}{⼧、⽤}
  \begin{Phonetics}{实用}{shi2yong4}[][HSK 4]
    \definition{adj.}{prático; pragmático; funcional; atende aos requisitos reais da aplicação}
    \definition{v.}{colocar em uso prático}
  \end{Phonetics}
\end{Entry}

\begin{Entry}{实在}{8,6}{⼧、⼟}
  \begin{Phonetics}{实在}{shi2zai4}[][HSK 2]
    \definition{adj.}{honesto; sincero | verdadeiro; honesto; realista; não é falso, não é enganador}
    \definition{adv.}{verdadeiramente; de fato; na verdade; usado para reforçar o tom afirmativo, enfatizando que a situação é realmente assim}
  \end{Phonetics}
\end{Entry}

\begin{Entry}{实行}{8,6}{⼧、⾏}
  \begin{Phonetics}{实行}{shi2xing2}[][HSK 3]
    \definition{v.}{praticar; implementar; executar; colocar em prática; realizar (programa, política, plano, etc.) por meio de ação}
  \end{Phonetics}
\end{Entry}

\begin{Entry}{实际}{8,7}{⼧、⾩}
  \begin{Phonetics}{实际}{shi2ji4}[][HSK 2]
    \definition{adj.}{real; efetivo; concreto; prático | factual; prático; realista; de acordo com os fatos}
    \definition{s.}{realidade; prática; coisas e situações que existem objetivamente}
  \end{Phonetics}
\end{Entry}

\begin{Entry}{实际上}{8,7,3}{⼧、⾩、⼀}
  \begin{Phonetics}{实际上}{shi2 ji4 shang4}[][HSK 3]
    \definition{adv.}{de fato; na verdade}
  \end{Phonetics}
\end{Entry}

\begin{Entry}{实现}{8,8}{⼧、⾒}
  \begin{Phonetics}{实现}{shi2xian4}[][HSK 2]
    \definition{v.}{alcançar; atingir; realizar; concretizar; tornar (ideais, planos, etc.) realidade}
  \end{Phonetics}
\end{Entry}

\begin{Entry}{实施}{8,9}{⼧、⽅}
  \begin{Phonetics}{实施}{shi2shi1}[][HSK 4]
    \definition{v.}{colocar em vigor; implementar (leis, políticas, etc.); executar; trazer (colocar) algo em vigor; fazer cumprir; colocar algo em (prática)}
  \end{Phonetics}
\end{Entry}

\begin{Entry}{实验}{8,10}{⼧、⾺}
  \begin{Phonetics}{实验}{shi2yan4}[][HSK 3]
    \definition[个,次]{s.}{teste; experimento; trabalho de laboratório}
    \definition{v.}{testar; experimentar; realizar uma operação ou se envolver em uma atividade para testar uma teoria ou hipótese científica}
  \end{Phonetics}
\end{Entry}

\begin{Entry}{实验室}{8,10,9}{⼧、⾺、⼧}
  \begin{Phonetics}{实验室}{shi2 yan4 shi4}[][HSK 3]
    \definition[个,间]{s.}{laboratório; salas especiais para experimentos científicos}
  \end{Phonetics}
\end{Entry}

\begin{Entry}{实惠}{8,12}{⼧、⼼}
  \begin{Phonetics}{实惠}{shi2hui4}[][HSK 5]
    \definition{adj.}{sólido; substancial; benefícios práticos}
    \definition{s.}{benefício material; benefícios tangíveis; benefícios reais}
  \end{Phonetics}
\end{Entry}

\begin{Entry}{实践}{8,12}{⼧、⾜}
  \begin{Phonetics}{实践}{shi2jian4}[][HSK 6]
    \definition{s.}{prática; filosoficamente, refere-se às ações conscientes das pessoas para transformar a natureza e a sociedade; as atividades de produção são as atividades práticas mais básicas e também incluem atividades políticas, experimentos científicos, educação cultural, etc.}
    \definition{v.}{praticar; realizar; implementar planos e intenções em ações específicas}
  \end{Phonetics}
\end{Entry}

\begin{Entry}{宠}{8}{⼧}
  \begin{Phonetics}{宠}{chong3}[][HSK 7-9]
    \definition*{s.}{Sobrenome Chong}
    \definition{v.}{mimar; estragar; conceder favor a | regalar; encontrar favor com alguém; estar nas boas graças de alguém}
  \end{Phonetics}
\end{Entry}

\begin{Entry}{宠物}{8,8}{⼧、⽜}
  \begin{Phonetics}{宠物}{chong3wu4}[][HSK 6]
    \definition[只]{s.}{animal de estimação; refere-se a pequenos animais criados na família}
  \end{Phonetics}
\end{Entry}

\begin{Entry}{宠爱}{8,10}{⼧、⽖}
  \begin{Phonetics}{宠爱}{chong3'ai4}[][HSK 7-9]
    \definition{v.}{mimar; fazer de alguém um animal de estimação}[爷爷总是宠爱他的孙子。===O avô sempre mima o neto.]
  \end{Phonetics}
\end{Entry}

\begin{Entry}{审}{8}{⼧}
  \begin{Phonetics}{审}{shen3}[][HSK 6]
    \definition*{s.}{Sobrenome Shen}
    \definition{adj.}{cuidadoso; detalhado; completo}
    \definition{adv.}{Literpario: realmente; de ​​fato; como esperado}
    \definition{v.}{examinar; analizar | julgar; interrogar | Literário: saber}
  \end{Phonetics}
\end{Entry}

\begin{Entry}{审查}{8,9}{⼧、⽊}
  \begin{Phonetics}{审查}{shen3cha2}[][HSK 6]
    \definition{v.}{examinar; investigar; verificar se algo está correto e apropriado (geralmente referindo-se a planos, propostas, escritos, qualificações pessoais, etc.); ler e avaliar (provas ou trabalhos de exame)}
  \end{Phonetics}
\end{Entry}

\begin{Entry}{尚}{8}{⼩}
  \begin{Phonetics}{尚}{shang4}
    \definition*{s.}{Sobrenome Shang}
    \definition{adv.}{ainda}
    \definition{s.}{costume predominante; refere-se à tendência predominante na sociedade; coisas que geralmente são admiradas pelas pessoas}
    \definition{v.}{valorizar; estimar; dar grande importância a}
  \end{Phonetics}
\end{Entry}

\begin{Entry}{尚且}{8,5}{⼩、⼀}
  \begin{Phonetics}{尚且}{shang4 qie3}
    \definition{conj.}{nem\dots; muito menos\dots; é usado antes do verbo da primeira oração de uma frase complexa para apresentar alguns exemplos óbvios para comparação, a segunda oração frequentemente usa 何况 ou 更 para ecoar e tirar conclusões inevitáveis ​​sobre exemplos semelhantes com diferentes graus de gravidade}
  \seealsoref{更}{geng4}
  \seealsoref{何况}{he2kuang4}
  \end{Phonetics}
\end{Entry}

\begin{Entry}{尚且……何况……}{8,5,7,7}{⼩、⼀、⼈、⼎}
  \begin{Phonetics}{尚且……何况……}{shang4qie3 he2kuang4}
    \definition{conj.}{ainda que\dots, \dots; além do mais\dots e muito menos\dots}
  \end{Phonetics}
\end{Entry}

\begin{Entry}{居}{8}{⼫}
  \begin{Phonetics}{居}{ju1}
    \definition*{s.}{Sobrenome Ju}
    \definition{s.}{residência; casa | restaurante (em nomes de restaurantes)}
    \definition{v.}{residir; morar; viver | ocupar uma determinada posição; ocupar (um lugar); estar (em uma determinada posição) | reivindicar; afirmar | armazenar; guardar | ficar parado; estar parado}
  \end{Phonetics}
\end{Entry}

\begin{Entry}{居民}{8,5}{⼫、⽒}
  \begin{Phonetics}{居民}{ju1min2}[][HSK 4]
    \definition[个,户,位]{s.}{residente; habitante; pessoas que estão fixas em um único lugar}
  \end{Phonetics}
\end{Entry}

\begin{Entry}{居住}{8,7}{⼫、⼈}
  \begin{Phonetics}{居住}{ju1zhu4}[][HSK 4]
    \definition{v.}{viver; residir; morar; habitar}
  \end{Phonetics}
\end{Entry}

\begin{Entry}{居然}{8,12}{⼫、⽕}
  \begin{Phonetics}{居然}{ju1ran2}[][HSK 5]
    \definition{adv.}{inesperadamente; para surpresa de alguém; além da expectativa (expressão idiomática)}
    \definition{v.}{ir tão longe a ponto de; ter a impudência de; ter o descaramento de}
  \end{Phonetics}
\end{Entry}

\begin{Entry}{屈}{8}{⼫}
  \begin{Phonetics}{屈}{qu1}
    \definition*{s.}{Sobrenome Qu}
    \definition[个]{s.}{injustiça; tratamento injusto | erro; queixa; injustiça}
    \definition{v.}{dobrar; curvar; encurvar | subjugar; submeter | tratar mal; tratar injustamente (ou deslealmente) | estar errado}
  \end{Phonetics}
\end{Entry}

\begin{Entry}{屈原}{8,10}{⼫、⼚}
  \begin{Phonetics}{屈原}{qu1yuan2}
    \definition*{s.}{Qu Yuan, poeta, é uma figura histórica famosa na cultura chinesa que viveu durante o Período dos Reinos Combatentes (340-278 a.C.).}
  \end{Phonetics}
\end{Entry}

\begin{Entry}{届}{8}{⼫}
  \begin{Phonetics}{届}{jie4}[][HSK 5]
    \definition{clas.}{sessões (de uma conferência); anos (de graduação); quantificador, ligeiramente equivalente a 次, usado para reuniões regulares ou turmas de formandos, etc.}
    \definition{v.}{vencer o prazo}
  \seealsoref{次}{ci4}
  \end{Phonetics}
\end{Entry}

\begin{Entry}{岭}{8}{⼭}
  \begin{Phonetics}{岭}{ling3}
    \definition{s.}{cordilheira}
  \end{Phonetics}
\end{Entry}

\begin{Entry}{岸}{8}{⼭}
  \begin{Phonetics}{岸}{an4}[][HSK 5]
    \definition{adj.}{arrogante; orgulhoso; grandioso (de maneira sombria ou condescendente)}
    \definition[条,道,段,面]{s.}{margem; costa; litoral; terreno à beira da água}
  \end{Phonetics}
\end{Entry}

\begin{Entry}{岸上}{8,3}{⼭、⼀}
  \begin{Phonetics}{岸上}{an4 shang4}[][HSK 5]
    \definition{s.}{em terra; costa; margem | na margem do rio; na beira do rio}
  \end{Phonetics}
\end{Entry}

\begin{Entry}{帘}{8}{⼱}
  \begin{Phonetics}{帘}{lian2}
    \definition[块,个]{s.}{bandeira em mastro sobre adega; bandeira como placa de loja | cortina; tela de bambu ou tecido; objetos para cobrir portas e janelas}
  \end{Phonetics}
\end{Entry}

\begin{Entry}{幷}{8}{⼲}
  \begin{Phonetics}{幷}{bing4}
    \variantof{并}
  \end{Phonetics}
\end{Entry}

\begin{Entry}{幸}{8}{⼲}
  \begin{Phonetics}{幸}{xing4}
    \definition*{s.}{Sobrenome Xing}
    \definition{adj.}{feliz}
    \definition{adv.}{afortunadamente; felizmente}
    \definition{s.}{felicidade}
    \definition{v.}{alegrar-se; sentir-se feliz e contente | favorecer; patrocinar | vir; chegar; antigamente, referia-se à chegada de um monarca a um determinado lugar}
  \end{Phonetics}
\end{Entry}

\begin{Entry}{幸亏}{8,3}{⼲、⼆}
  \begin{Phonetics}{幸亏}{xing4kui1}
    \definition{adv.}{felizmente}
  \end{Phonetics}
\end{Entry}

\begin{Entry}{幸运}{8,7}{⼲、⾡}
  \begin{Phonetics}{幸运}{xing4yun4}[][HSK 3]
    \definition{adj.}{sortudo; feliz; afortunado}
    \definition[个,点,丝]{s.}{boa sorte; boa fortuna}
  \end{Phonetics}
\end{Entry}

\begin{Entry}{幸运儿}{8,7,2}{⼲、⾡、⼉}
  \begin{Phonetics}{幸运儿}{xing4yun4'er2}
    \definition{s.}{pessoa de sorte}
  \end{Phonetics}
\end{Entry}

\begin{Entry}{幸运抽奖}{8,7,8,9}{⼲、⾡、⼿、⼤}
  \begin{Phonetics}{幸运抽奖}{xing4yun4chou1jiang3}
    \definition{s.}{loteria | sorteio}
  \end{Phonetics}
\end{Entry}

\begin{Entry}{幸福}{8,13}{⼲、⽰}
  \begin{Phonetics}{幸福}{xing4fu2}[][HSK 3]
    \definition{adj.}{feliz; a vida, a família e outras circunstâncias deixam as pessoas satisfeitas e felizes}
    \definition{s.}{felicidade; bem estar; sensação ou experiência satisfatória e feliz, etc.}
  \end{Phonetics}
\end{Entry}

\begin{Entry}{底}{8}{⼴}
  \begin{Phonetics}{底}{de5}
    \definition{part.}{usada após uma palavra ou frase que é usada como determinante para indicar subordinação à palavra central}
  \end{Phonetics}
  \begin{Phonetics}{底}{di3}[][HSK 4]
    \definition*{s.}{Sobrenome Di}
    \definition{pron.}{o que? |  isto; isso; aqui | assim; tal}
    \definition{s.}{base; fundo; parte inferior de um objeto | detalhes; o cerne da questão; base, fonte ou contexto de uma coisa | rascunho; cópia mantida como registro; rascunho que pode ser usado como base | final de um ano ou mês | chão; fundo; fundação | a última parte de algo}
  \end{Phonetics}
\end{Entry}

\begin{Entry}{底下}{8,3}{⼴、⼀}
  \begin{Phonetics}{底下}{di3 xia4}[][HSK 3]
    \definition{adv.}{em baixo; abaixo; sob | próximo; mais tarde; depois; daqui para a frente}
  \end{Phonetics}
\end{Entry}

\begin{Entry}{底子}{8,3}{⼴、⼦}
  \begin{Phonetics}{底子}{di3zi5}[][HSK 7-9]
    \definition{s.}{fundo; base; a parte mais baixa de um objeto | solo; base; fundo; fundação | rascunho ou esboço; um rascunho para servir de base | cópia mantida como registro; cópia de arquivo | remanescente | detalhes; prós e contras | Literário: configuração (o padrão base)}
  \end{Phonetics}
\end{Entry}

\begin{Entry}{底气}{8,4}{⼴、⽓}
  \begin{Phonetics}{底气}{di3qi4}
    \definition{s.}{capacidade pulmonar | ousadia | confiança | autoconfiança | vigor}
  \end{Phonetics}
\end{Entry}

\begin{Entry}{底层}{8,7}{⼴、⼫}
  \begin{Phonetics}{底层}{di3ceng2}[][HSK 7-9]
    \definition[个]{s.}{andar térreo | fundo; o degrau mais baixo; classe social mais baixa | porão | subcamada; camada de base; subcapa; substrato}
  \end{Phonetics}
\end{Entry}

\begin{Entry}{底线}{8,8}{⼴、⽷}
  \begin{Phonetics}{底线}{di3xian4}[][HSK 7-9]
    \definition{s.}{linha de base (em esportes); limites em ambas as extremidades de campos esportivos como futebol, basquete, vôlei e badminton | um mínimo; o limite mais baixo; um limite mínimo; a menor quantidade possível; refere-se às condições mínimas | um fantoche; um informante; um agente infiltrado; uma pessoa que se esconde dentro do inimigo para reunir informações ou conduzir outras atividades; um \emph{insider}}
  \end{Phonetics}
\end{Entry}

\begin{Entry}{底蕴}{8,15}{⼴、⾋}
  \begin{Phonetics}{底蕴}{di3yun4}[][HSK 7-9]
    \definition{s.}{detalhes; informações privilegiadas; história interna}
  \end{Phonetics}
\end{Entry}

\begin{Entry}{店}{8}{⼴}
  \begin{Phonetics}{店}{dian4}[][HSK 2]
    \definition[家,间,个]{s.}{loja; armazém; loja de venda de mercadorias | pousada; pequena pousada com instalações simples | usado para nomes de lugares}
  \end{Phonetics}
\end{Entry}

\begin{Entry}{店主}{8,5}{⼴、⼂}
  \begin{Phonetics}{店主}{dian4zhu3}
    \definition{s.}{lojista | dono de loja}
  \end{Phonetics}
\end{Entry}

\begin{Entry}{店员}{8,7}{⼴、⼝}
  \begin{Phonetics}{店员}{dian4yuan2}
    \definition{s.}{assistente de loja | balconista | vendedor}
  \end{Phonetics}
\end{Entry}

\begin{Entry}{废}{8}{⼴}
  \begin{Phonetics}{废}{fei4}[][HSK 7-9]
    \definition{adj.}{desperdíçado; inútil; fora de uso; inválido; tendo perdido sua função original | Literário: incapacitado; mutilado; aleijado; desabilitado}
    \definition{v.}{desistir; abandonar; abolir; revogar  | Coloquial: punir; bater em alguém | descartar; abandonar}
  \end{Phonetics}
\end{Entry}

\begin{Entry}{废物}{8,8}{⼴、⽜}
  \begin{Phonetics}{废物}{fei4wu4}[][HSK 7-9]
    \definition{s.}{lixo; material residual; coisas que perderam seu valor de uso original}
  \end{Phonetics}
  \begin{Phonetics}{废物}{fei4wu5}
    \definition{s.}{pessoa inútil; imprestável (insulto); uma metáfora para uma pessoa inútil (palavrão)}
  \end{Phonetics}
\end{Entry}

\begin{Entry}{废话}{8,8}{⼴、⾔}
  \begin{Phonetics}{废话}{fei4hua4}[][HSK 7-9]
    \definition{s.}{lixo; absurdo; palavras supérfluas; palavras redundantes e inúteis}
    \definition{v.}{falar bobagens; conversa fiada}
  \end{Phonetics}
\end{Entry}

\begin{Entry}{废品}{8,9}{⼴、⼝}
  \begin{Phonetics}{废品}{fei4pin3}[][HSK 7-9]
    \definition[件,吨,批,堆]{s.}{produto residual; rejeito; descarte; produtos não qualificados; produto descartado; sucata; refugo; material rejeitado}
  \end{Phonetics}
\end{Entry}

\begin{Entry}{废除}{8,9}{⼴、⾩}
  \begin{Phonetics}{废除}{fei4chu2}[][HSK 7-9]
    \definition{v.}{revogar; anular; cancelar; abolir (uma lei, sistema, tratado, etc.)}
  \end{Phonetics}
\end{Entry}

\begin{Entry}{废寝忘食}{8,13,7,9}{⼴、⼧、⼼、⾷}
  \begin{Phonetics}{废寝忘食}{fei4qin3-wang4shi2}[][HSK 7-9]
    \definition{expr.}{esquecer de comer e dormir; estar totalmente absorvido em}
  \end{Phonetics}
\end{Entry}

\begin{Entry}{废墟}{8,14}{⼴、⼟}
  \begin{Phonetics}{废墟}{fei4xu1}[][HSK 7-9]
    \definition[片,堆,个]{s.}{ruínas; terreno baldio; um lugar como uma cidade ou vila que ficou deserta e desolada após ser destruída ou sofrer um desastre natural}
  \end{Phonetics}
\end{Entry}

\begin{Entry}{建}{8}{⼵}
  \begin{Phonetics}{建}{jian4}[][HSK 3]
    \definition*{s.}{Província de Fujian | Rio Jian Jiang (na província de Fujian) | Sobrenome Jian}
    \definition{v.}{construir; construir; erigir | estabelecer; configurar; fundar | propor; defender; apresentar (suas próprias opiniões)}
  \end{Phonetics}
\end{Entry}

\begin{Entry}{建立}{8,5}{⼵、⽴}
  \begin{Phonetics}{建立}{jian4li4}[][HSK 3]
    \definition{v.}{estabelecer; construir; começar a construir | vir a ser; começar a surgir; começar a se formar}
  \end{Phonetics}
\end{Entry}

\begin{Entry}{建立者}{8,5,8}{⼵、⽴、⽼}
  \begin{Phonetics}{建立者}{jian4li4zhe3}
    \definition{s.}{fundador; construtor}
  \end{Phonetics}
\end{Entry}

\begin{Entry}{建议}{8,5}{⼵、⾔}
  \begin{Phonetics}{建议}{jian4yi4}[][HSK 3]
    \definition[个,点,条]{s.}{proposta; sugestão; recomendação; para que alguém ou alguma coisa evolua para melhor, para o coletivo; pontos de vista e opiniões apresentados pelos líderes, etc.}
    \definition{v.}{propor; sugerir; recomendar; em relação a determinada pessoa ou situação, apresentar seus pontos de vista e opiniões ao coletivo, aos líderes ou a indivíduos, para que as coisas evoluam para melhor}
  \end{Phonetics}
\end{Entry}

\begin{Entry}{建成}{8,6}{⼵、⼽}
  \begin{Phonetics}{建成}{jian4 cheng2}[][HSK 3]
    \definition{v.}{terminar a construção}
  \end{Phonetics}
\end{Entry}

\begin{Entry}{建设}{8,6}{⼵、⾔}
  \begin{Phonetics}{建设}{jian4she4}[][HSK 3]
    \definition{s.}{reconstrução; desenvolvimento; trabalhos relacionados com a construção}
    \definition{v.}{construir; edificar; (Estado ou coletividade) criar novos empreendimentos ou aumento de novas instalações}
  \end{Phonetics}
\end{Entry}

\begin{Entry}{建设性}{8,6,8}{⼵、⾔、⼼}
  \begin{Phonetics}{建设性}{jian4she4xing4}
    \definition{adj.}{construtivo}
    \definition{s.}{construtividade}
  \end{Phonetics}
\end{Entry}

\begin{Entry}{建设者}{8,6,8}{⼵、⾔、⽼}
  \begin{Phonetics}{建设者}{jian4she4zhe3}
    \definition{s.}{construtor}
  \end{Phonetics}
\end{Entry}

\begin{Entry}{建造}{8,10}{⼵、⾡}
  \begin{Phonetics}{建造}{jian4 zao4}[][HSK 5]
    \definition{v.}{construir; edificar}
  \end{Phonetics}
\end{Entry}

\begin{Entry}{建筑}{8,12}{⼵、⽵}
  \begin{Phonetics}{建筑}{jian4zhu4}[][HSK 5]
    \definition[座,幢,排]{s.}{construção; estrutura; edifício; prédio}
    \definition{v.}{construir; erguer; edificar; construir casas, estradas, pontes, etc.}
  \end{Phonetics}
\end{Entry}

\begin{Entry}{廻}{8}{⼵}
  \begin{Phonetics}{廻}{hui2}
    \variantof{回}
  \end{Phonetics}
\end{Entry}

\begin{Entry}{录}{8}{⼹}
  \begin{Phonetics}{录}{lu4}[][HSK 3]
    \definition{s.}{registro; cadastro; coleção; seleções}
    \definition{v.}{copiar; gravar; escrever; copiar; registrar | contratar; selecionar; empregar; adotar ou nomear | gravar em fita magnética}
  \end{Phonetics}
\end{Entry}

\begin{Entry}{录取}{8,8}{⼹、⼜}
  \begin{Phonetics}{录取}{lu4qu3}[][HSK 4]
    \definition{v.}{aceitar; admitir; recrutar; entrar; matricular (os aprovados no exame)}
  \end{Phonetics}
\end{Entry}

\begin{Entry}{录音}{8,9}{⼹、⾳}
  \begin{Phonetics}{录音}{lu4/yin1}[][HSK 3]
    \definition[段,个]{s.}{gravação de som; som gravado com equipamento especializado}
    \definition{v.+compl.}{gravar; converter o som em sinal elétrico e, em seguida, gravá-lo por meios mecânicos, ópticos ou eletromagnéticos}
  \end{Phonetics}
\end{Entry}

\begin{Entry}{录音机}{8,9,6}{⼹、⾳、⽊}
  \begin{Phonetics}{录音机}{lu4 yin1 ji1}[][HSK 6]
    \definition[台]{s.}{gravador de som; máquina de gravação (de fita)}
  \end{Phonetics}
\end{Entry}

\begin{Entry}{录像}{8,13}{⼹、⼈}
  \begin{Phonetics}{录像}{lu4/xiang4}[][HSK 6]
    \definition[段,个,些,盘]{s.}{vídeo; gravação; fita de vídeo; imagens gravadas com celulares, câmeras, etc.}
    \definition{v.+compl.}{gravar bídeo; gravar em fita de vídeo | usar celulares, câmeras e outros dispositivos para salvar registros de vídeo}
  \end{Phonetics}
\end{Entry}

\begin{Entry}{录像机}{8,13,6}{⼹、⼈、⽊}
  \begin{Phonetics}{录像机}{lu4xiang4ji1}
    \definition[台]{s.}{gravador de vídeo | VCR}
  \end{Phonetics}
\end{Entry}

\begin{Entry}{录像带}{8,13,9}{⼹、⼈、⼱}
  \begin{Phonetics}{录像带}{lu4xiang4dai4}
    \definition[盘]{s.}{video-cassete}
  \end{Phonetics}
\end{Entry}

\begin{Entry}{彼}{8}{⼻}
  \begin{Phonetics}{彼}{bi3}
    \definition{s.}{aquele; aquilo (oposto a 此) ; outro | a outra parte}
  \seealsoref{此}{ci3}
  \end{Phonetics}
\end{Entry}

\begin{Entry}{彼此}{8,6}{⼻、⽌}
  \begin{Phonetics}{彼此}{bi3ci3}[][HSK 5]
    \definition{pron.}{um ao outro; uns com os outros; este e aquele têm algum tipo de relacionamento; ambas as partes}
  \end{Phonetics}
\end{Entry}

\begin{Entry}{往}{8}{⼻}
  \begin{Phonetics}{往}{wang3}[][HSK 2]
    \definition{adj.}{passado; anterior}
    \definition{prep.}{para; em direção a; na direção de}
    \definition{v.}{ir}
  \end{Phonetics}
\end{Entry}

\begin{Entry}{往日}{8,4}{⼻、⽇}
  \begin{Phonetics}{往日}{wang3ri4}
    \definition{adv.}{dias passados}
    \definition{s.}{o passado}
  \end{Phonetics}
\end{Entry}

\begin{Entry}{往生}{8,5}{⼻、⽣}
  \begin{Phonetics}{往生}{wang3sheng1}
    \definition{v.}{renascer | morrer | (Budismo) viver no paraíso}
  \end{Phonetics}
\end{Entry}

\begin{Entry}{往后}{8,6}{⼻、⼝}
  \begin{Phonetics}{往后}{wang3 hou4}[][HSK 6]
    \definition{s.}{de agora em diante; mais tarde; no futuro | na parte traseira; na parte de trás | para trás; depois; à ré}
  \end{Phonetics}
\end{Entry}

\begin{Entry}{往年}{8,6}{⼻、⼲}
  \begin{Phonetics}{往年}{wang3 nian2}[][HSK 6]
    \definition{s.}{(em) anos anteriores}
  \end{Phonetics}
\end{Entry}

\begin{Entry}{往来}{8,7}{⼻、⽊}
  \begin{Phonetics}{往来}{wang3 lai2}[][HSK 6]
    \definition{s.}{contatos comerciais; relações comerciais; relações diplomáticas | negociações; visitas mútuas; comunicação}
    \definition{v.}{ir e vir | contatar; ter relações}
  \end{Phonetics}
\end{Entry}

\begin{Entry}{往返}{8,7}{⼻、⾡}
  \begin{Phonetics}{往返}{wang3fan3}
    \definition{s.}{ida e volta}
    \definition{v.}{ir e voltar | ir e vir}
  \end{Phonetics}
\end{Entry}

\begin{Entry}{往事}{8,8}{⼻、⼅}
  \begin{Phonetics}{往事}{wang3shi4}
    \definition{s.}{acontecimentos anteriores | eventos passados}
  \end{Phonetics}
\end{Entry}

\begin{Entry}{往例}{8,8}{⼻、⼈}
  \begin{Phonetics}{往例}{wang3li4}
    \definition{s.}{prática (habitual) do passado | precedente}
  \end{Phonetics}
\end{Entry}

\begin{Entry}{往往}{8,8}{⼻、⼻}
  \begin{Phonetics}{往往}{wang3wang3}[][HSK 3]
    \definition{adv.}{frequentemente; muitas vezes; mais frequentemente do que não; indica que uma situação existe ou ocorre com frequência}
  \end{Phonetics}
\end{Entry}

\begin{Entry}{往昔}{8,8}{⼻、⽇}
  \begin{Phonetics}{往昔}{wang3xi1}
    \definition{s.}{o passado}
  \end{Phonetics}
\end{Entry}

\begin{Entry}{往复}{8,9}{⼻、⼢}
  \begin{Phonetics}{往复}{wang3fu4}
    \definition{s.}{para trás e para frente (por exemplo, da ação do pistão ou da bomba)}
    \definition{v.}{ir e voltar | fazer uma viagem de volta}
  \end{Phonetics}
\end{Entry}

\begin{Entry}{往迹}{8,9}{⼻、⾡}
  \begin{Phonetics}{往迹}{wang3ji4}
    \definition{s.}{eventos passados}
  \end{Phonetics}
\end{Entry}

\begin{Entry}{往程}{8,12}{⼻、⽲}
  \begin{Phonetics}{往程}{wang3cheng2}
    \definition{s.}{saída (de uma viagem de ônibus ou trem, etc.)}
  \end{Phonetics}
\end{Entry}

\begin{Entry}{征}{8}{⼻}
  \begin{Phonetics}{征}{zheng1}
    \definition{s.}{prova; evidência | sinal; símbolo; presságio; sinais de manifestação; fenômeno}
    \definition{v.}{viajar; fazer uma jornada; pegar o caminho mais longo | iniciar uma campanha; fazer uma expedição punitiva | convocar; selecionar; recrutar | cobrar; impor; coletar | solicitar; pedir; procurar}
  \end{Phonetics}
\end{Entry}

\begin{Entry}{征求}{8,7}{⼻、⽔}
  \begin{Phonetics}{征求}{zheng1qiu2}[][HSK 4]
    \definition{v.}{procurar; buscar; solicitar; pedir abertamente opiniões, pontos de vista, etc.}
  \end{Phonetics}
\end{Entry}

\begin{Entry}{征服}{8,8}{⼻、⽉}
  \begin{Phonetics}{征服}{zheng1fu2}[][HSK 4]
    \definition{v.}{conquistar; cativar; usar a força para fazer a outra parte se submeter | subjugar; dominar; convencer as pessoas com poder infeccioso}
  \end{Phonetics}
\end{Entry}

\begin{Entry}{忠}{8}{⼼}
  \begin{Phonetics}{忠}{zhong1}
    \definition{adj.}{leal; fiel; devotado | honesto}
  \end{Phonetics}
\end{Entry}

\begin{Entry}{忠心}{8,4}{⼼、⼼}
  \begin{Phonetics}{忠心}{zhong1 xin1}[][HSK 6]
    \definition{s.}{lealdade; devoção; fidelidade}
  \end{Phonetics}
\end{Entry}

\begin{Entry}{念}{8}{⼼}
  \begin{Phonetics}{念}{nian4}[][HSK 3]
    \definition*{s.}{Sobrenome Nian}
    \definition{num.}{vinte; 20; capitalização do número 廿}
    \definition{s.}{ideia; pensamento; pensamentos ou intenções internas}
    \definition{v.}{ler em voz alta | estudar; frequentar a escola | considerar; levar em conta | sentir falta; pensar em; pensar sobre; pensar frequentemente sobre}
  \seealsoref{廿}{nian4}
  \end{Phonetics}
\end{Entry}

\begin{Entry}{忽}{8}{⼼}
  \begin{Phonetics}{忽}{hu1}
    \definition*{s.}{Sobrenome Hu}
    \definition{adv.}{agora\dots, agora\dots | de repente; subitamente}[天气忽冷忽热。===O clima está frio em um minuto e quente no outro.]
    \definition{v.}{negligenciar; ignorar; não prestar atenção; não levar a sério}
  \end{Phonetics}
\end{Entry}

\begin{Entry}{忽视}{8,8}{⼼、⾒}
  \begin{Phonetics}{忽视}{hu1shi4}[][HSK 4]
    \definition{v.}{ignorar; negligenciar; menosprezar; desprezar; dar de ombros}
  \end{Phonetics}
\end{Entry}

\begin{Entry}{忽略}{8,11}{⼼、⽥}
  \begin{Phonetics}{忽略}{hu1lve4}[][HSK 6]
    \definition{v.}{negligenciar; ignorar; não perceber}
  \end{Phonetics}
\end{Entry}

\begin{Entry}{忽然}{8,12}{⼼、⽕}
  \begin{Phonetics}{忽然}{hu1ran2}[][HSK 2]
    \definition{adv.}{repentinamente; de repente; sem aviso prévio; significa que algo aconteceu de forma rápida e inesperada}
  \end{Phonetics}
\end{Entry}

\begin{Entry}{态}{8}{⼼}
  \begin{Phonetics}{态}{tai4}
    \definition{s.}{forma; aparência; condição | (física) estado | (linguística) voz}[气态===estado gasoso | 被动态===voz passiva]
  \end{Phonetics}
\end{Entry}

\begin{Entry}{态度}{8,9}{⼼、⼴}
  \begin{Phonetics}{态度}{tai4du5}[][HSK 2]
    \definition[种,个]{s.}{maneira; comportamento; atitude; comportamento e expressão facial das pessoas | atitude; abordagem; opinião sobre o assunto e medidas tomadas}
  \end{Phonetics}
\end{Entry}

\begin{Entry}{怕}{8}{⼼}
  \begin{Phonetics}{怕}{pa4}[][HSK 2]
    \definition{adv.}{(expressando suposição, julgamento, estimativa, etc.) talvez; suponho; receio (que)}
    \definition{adv.}{por medo; talvez; suponho}
    \definition{v.}{temer; ter medo; recear; sentir medo, ficar nervoso | estar preocupado com; estar preocupado por (ou sobre); ter medo de que algo possa acontecer | ser afetado por; não conseguir suportar; não aguentar mais}
  \end{Phonetics}
\end{Entry}

\begin{Entry}{性}{8}{⼼}
  \begin{Phonetics}{性}{xing4}[][HSK 3]
    \definition[个]{s.}{natureza; caráter; personalidade | propriedade; qualidade; natureza e características das coisas | sexo; gênero | sexualidade; relacionado com a reprodução e a sexualidade | caráter; temperamento}
    \definition{suf.}{indica uma determinada propriedade ou característica de algo; segue um substantivo, verbo ou adjetivo, formando um substantivo abstrato ou um adjetivo que expressa uma propriedade}
  \end{Phonetics}
\end{Entry}

\begin{Entry}{性生活}{8,5,9}{⼼、⽣、⽔}
  \begin{Phonetics}{性生活}{xing4sheng1huo2}
    \definition{s.}{vida sexual}
  \end{Phonetics}
\end{Entry}

\begin{Entry}{性别}{8,7}{⼼、⼑}
  \begin{Phonetics}{性别}{xing4bie2}[][HSK 3]
    \definition[种]{s.}{sexo; gênero}
  \end{Phonetics}
\end{Entry}

\begin{Entry}{性质}{8,8}{⼼、⾙}
  \begin{Phonetics}{性质}{xing4zhi4}[][HSK 4]
    \definition[个,种,类]{s.}{natureza; qualidade; caráter; propriedade; propriedade fundamental que distingue uma coisa de outra}
  \end{Phonetics}
\end{Entry}

\begin{Entry}{性侵}{8,9}{⼼、⼈}
  \begin{Phonetics}{性侵}{xing4qin1}
    \definition{s.}{agressão sexual}
    \definition{v.}{agredir sexualmente}
  \end{Phonetics}
\end{Entry}

\begin{Entry}{性格}{8,10}{⼼、⽊}
  \begin{Phonetics}{性格}{xing4ge2}[][HSK 3]
    \definition[种,个]{s.}{caráter; temperamento; as características psicológicas manifestadas na atitude e no comportamento em relação às pessoas e às coisas}
  \end{Phonetics}
\end{Entry}

\begin{Entry}{性能}{8,10}{⼼、⾁}
  \begin{Phonetics}{性能}{xing4neng2}[][HSK 5]
    \definition{s.}{natureza; propriedade; desempenho; função (de uma máquina, etc.); grau de conformidade dos produtos mecânicos ou outros produtos industriais com os requisitos de projeto}
  \end{Phonetics}
\end{Entry}

\begin{Entry}{怪}{8}{⼼}
  \begin{Phonetics}{怪}{guai4}[][HSK 4,5]
    \definition*{s.}{Sobrenome Guai}
    \definition{adj.}{estranho; esquisito; desconcertante | peculiar; excêntrico; pitoresco; monstruoso; anormal; incomum}
    \definition{adv.}{bastante; muito}
    \definition{s.}{monstro; demônio | diabo; ser maligno}
    \definition{v.}{culpar | achar algo estranho; maravilhar-se com; ficar surpreso | repreender; culpar; reclamar}
  \end{Phonetics}
\end{Entry}

\begin{Entry}{怪不得}{8,4,11}{⼼、⼀、⼻}
  \begin{Phonetics}{怪不得}{guai4bu5de5}[][HSK 7-9]
    \definition{adv.}{não é de admirar; então é por isso; isso explica por que; isso significa que você entende o motivo e não acha mais uma situação estranha}
    \definition{v.}{não culpar; não acusar; não poder culpar, não se ofender}[你做错了,怪不得别人。===Você cometeu um erro, então não culpe os outros.]
  \end{Phonetics}
\end{Entry}

\begin{Entry}{怪异}{8,6}{⼼、⼶}
  \begin{Phonetics}{怪异}{guai4yi4}[][HSK 7-9]
    \definition{adj.}{monstruoso; estranho; incomum}
    \definition{s.}{fenômeno estranho; presságio; prodígio | monstruosidade}
  \end{Phonetics}
\end{Entry}

\begin{Entry}{怪物}{8,8}{⼼、⽜}
  \begin{Phonetics}{怪物}{guai4wu5}[][HSK 7-9]
    \definition{s.}{monstro; aberração; coisas imaginárias que parecem estranhas, mas têm habilidades especiais | pessoa excêntrica; pássaro estranho; uma pessoa com temperamento excêntrico}
  \end{Phonetics}
\end{Entry}

\begin{Entry}{怪兽}{8,11}{⼼、⼋}
  \begin{Phonetics}{怪兽}{guai4shou4}
    \definition{s.}{animal raro | animal mítico | monstro}
  \end{Phonetics}
\end{Entry}

\begin{Entry}{怪癖}{8,18}{⼼、⽧}
  \begin{Phonetics}{怪癖}{guai4pi3}
    \definition{adj.}{peculiar}
    \definition{s.}{excentricidade | peculiaridade | hobby estranho}
  \end{Phonetics}
\end{Entry}

\begin{Entry}{或}{8}{⼽}
  \begin{Phonetics}{或}{huo4}[][HSK 2]
    \definition{adv.}{talvez; possivelmente; provavelmente | (geralmente na forma negativa) um pouco; ligeiramente}
    \definition{conj.}{ou (indicando escolha); ou\dots ou\dots}
    \definition{pron.}{alguém; algumas pessoas; refere-se a alguém ou algo, equivalente a 有人 ou 有的}
  \seealsoref{有的}{you3 de5}
  \seealsoref{有人}{you3 ren2}
  \end{Phonetics}
\end{Entry}

\begin{Entry}{或许}{8,6}{⼽、⾔}
  \begin{Phonetics}{或许}{huo4xu3}[][HSK 4]
    \definition{adv.}{talvez; possivelmente; receio; não tenho certeza}
  \end{Phonetics}
\end{Entry}

\begin{Entry}{或者}{8,8}{⼽、⽼}
  \begin{Phonetics}{或者}{huo4zhe3}[][HSK 2]
    \definition{adv.}{talvez; possivelmente}
    \definition{conj.}{ou (usado em expressões afirmativas); ou\dots ou\dots; usado em frases narrativas para indicar uma relação de escolha | ou (usado para indicar equação); indica relação de equivalência, indicando que os objetos anterior e posterior são iguais}
  \end{Phonetics}
\end{Entry}

\begin{Entry}{或是}{8,9}{⼽、⽇}
  \begin{Phonetics}{或是}{huo4 shi4}[][HSK 5]
    \definition{adv.}{um ou outro; o outro}
    \definition{conj.}{ou; às vezes, é apenas uma de duas coisas}
  \end{Phonetics}
\end{Entry}

\begin{Entry}{房}{8}{⼾}
  \begin{Phonetics}{房}{fang2}
    \definition*{s.}{Fang, a quarta das vinte e oito constelações nas quais a esfera celeste foi dividida, consistindo de quatro estrelas quase em linha reta em Escorpião | Sobrenome Fang}
    \definition[幢,个,间]{s.}{casa; edifício | sala; quarto; câmara | estrutura semelhante a uma casa | um ramo de uma família extensa | loja; estoque | local de trabalho do artesão; oficina; moinho}
  \end{Phonetics}
\end{Entry}

\begin{Entry}{房子}{8,3}{⼾、⼦}
  \begin{Phonetics}{房子}{fang2 zi5}[][HSK 1]
    \definition[栋,幢,座,套,间]{s.}{casa; edifício; prédio}
  \end{Phonetics}
\end{Entry}

\begin{Entry}{房东}{8,5}{⼾、⼀}
  \begin{Phonetics}{房东}{fang2dong1}[][HSK 3]
    \definition[个,位,名]{s.}{dono;  proprietário; senhorio; pessoas que alugam ou emprestam imóveis (para os 房客 )}
  \seealsoref{房客}{fang2ke4}
  \end{Phonetics}
\end{Entry}

\begin{Entry}{房主}{8,5}{⼾、⼂}
  \begin{Phonetics}{房主}{fang2zhu3}
    \definition{s.}{proprietário | dono de um imóvel}
  \end{Phonetics}
\end{Entry}

\begin{Entry}{房价}{8,6}{⼾、⼈}
  \begin{Phonetics}{房价}{fang2 jia4}[][HSK 6]
    \definition{s.}{custo de moradia; tarifa de quarto | preço da casa}
  \end{Phonetics}
\end{Entry}

\begin{Entry}{房地产}{8,6,6}{⼾、⼟、⼇}
  \begin{Phonetics}{房地产}{fang2di4chan3}[][HSK 7-9]
    \definition{s.}{imóveis; um termo geral para imóveis e terrenos}
  \end{Phonetics}
\end{Entry}

\begin{Entry}{房间}{8,7}{⼾、⾨}
  \begin{Phonetics}{房间}{fang2jian1}[][HSK 1]
    \definition[个,间,套]{s.}{sala; câmara; escritório; apartamento; divisões internas da casa}
  \end{Phonetics}
\end{Entry}

\begin{Entry}{房客}{8,9}{⼾、⼧}
  \begin{Phonetics}{房客}{fang2ke4}[][HSK 3]
    \definition{s.}{inquilino (de um quarto ou casa); hóspede (oposto a 房东) | inquilino; hóspede; pessoas que alugam ou emprestam imóveis para moradia (para o 房东)}
  \seealsoref{房东}{fang2dong1}
  \end{Phonetics}
\end{Entry}

\begin{Entry}{房屋}{8,9}{⼾、⼫}
  \begin{Phonetics}{房屋}{fang2 wu1}[][HSK 3]
    \definition[间,所,套]{s.}{casas; habitação; edifícios}
  \end{Phonetics}
\end{Entry}

\begin{Entry}{房租}{8,10}{⼾、⽲}
  \begin{Phonetics}{房租}{fang2 zu1}[][HSK 3]
    \definition[笔]{s.}{aluguel}
  \end{Phonetics}
\end{Entry}

\begin{Entry}{所}{8}{⼾}
  \begin{Phonetics}{所}{suo3}[][HSK 3,6]
    \definition*{s.}{Sobrenome Suo}
    \definition{clas.}{usado para casas, etc.}
    \definition{part.}{usado com 为 ou 被 para indicar voz passiva | usado antes do verbo para formar um substantivo ou para qualificar um substantivo | usado antes do verbo na estrutura sujeito-predicado usada como complemento, indica que o termo central é o objeto}
    \definition{s.}{lugar | usado como nome de órgãos governamentais ou outros locais de trabalho}
  \seealsoref{被}{bei4}
  \seealsoref{为}{wei4}
  \end{Phonetics}
\end{Entry}

\begin{Entry}{所以}{8,4}{⼾、⼈}
  \begin{Phonetics}{所以}{suo3 yi3}[][HSK 2]
    \definition{conj.}{assim; portanto; como resultado; conecta frases, expressa resultados e costuma corresponder a expressões como 因为 e 由于}
    \definition[个]{s.}{motivo real; causa real; comportamento adequado}
  \seealsoref{因为}{yin1wei4}
  \seealsoref{由于}{you2yu2}
  \end{Phonetics}
\end{Entry}

\begin{Entry}{所长}{8,4}{⼾、⾧}
  \begin{Phonetics}{所长}{suo3 chang2}
    \definition{s.}{aquilo em que alguém é bom; o ponto forte de alguém; o forte de alguém}
  \end{Phonetics}
  \begin{Phonetics}{所长}{suo3 zhang3}[][HSK 3]
    \definition{s.}{chefe de um instituto, etc. | superintendente}
  \end{Phonetics}
\end{Entry}

\begin{Entry}{所在}{8,6}{⼾、⼟}
  \begin{Phonetics}{所在}{suo3 zai4}[][HSK 5]
    \definition[个]{s.}{lugar; local; localização | o lugar onde alguém ou algo está}
  \end{Phonetics}
\end{Entry}

\begin{Entry}{所有}{8,6}{⼾、⽉}
  \begin{Phonetics}{所有}{suo3you3}[][HSK 2]
    \definition{adj.}{todo | tudo}
    \definition{adj.}{tudo}
    \definition{s.}{bens; posses;}
    \definition{v.}{possuir; ter}
  \end{Phonetics}
\end{Entry}

\begin{Entry}{承}{8}{⼿}
  \begin{Phonetics}{承}{cheng2}
    \definition*{s.}{Sobrenome Cheng}
    \definition{v.}{suportar; segurar; carregar; sustentar | empreender; contratar (para fazer um trabalho) | estar em dívida (com alguém por uma gentileza); receber um favor | continuar; prosseguir | receber de cima (instruções, mandato)}
  \end{Phonetics}
\end{Entry}

\begin{Entry}{承办}{8,4}{⼿、⼒}
  \begin{Phonetics}{承办}{cheng2ban4}[][HSK 5]
    \definition{v.}{ocupar-se de; encarregar-se de; (pessoas, organizações, instituições) aceitar (atividades, reuniões, negócios, etc.)}
  \end{Phonetics}
\end{Entry}

\begin{Entry}{承认}{8,4}{⼿、⾔}
  \begin{Phonetics}{承认}{cheng2ren4}[][HSK 4]
    \definition{s.}{reconhecimento (diplomático, artístico, etc.)}
    \definition{v.}{admitir; reconhecer | dar reconhecimento diplomático; reconhecer}
  \end{Phonetics}
\end{Entry}

\begin{Entry}{承包}{8,5}{⼿、⼓}
  \begin{Phonetics}{承包}{cheng2bao1}[][HSK 7-9]
    \definition{v.}{contratar (com; para); aceitar projetos ou pedidos em massa, etc. e ser responsável por concluí-los}[他承包了这个工程。===Ele foi contratado para esse projeto.]
  \end{Phonetics}
\end{Entry}

\begin{Entry}{承受}{8,8}{⼿、⼜}
  \begin{Phonetics}{承受}{cheng2shou4}[][HSK 4]
    \definition{v.}{suportar; resistir; realizar (tarefas, dificuldades, pressões, etc.); submeter-se a (testes, etc.) | herdar}
  \end{Phonetics}
\end{Entry}

\begin{Entry}{承担}{8,8}{⼿、⼿}
  \begin{Phonetics}{承担}{cheng2dan1}[][HSK 4]
    \definition{v.}{suportar; empreender; assumir; tomar conta de algo}
  \end{Phonetics}
\end{Entry}

\begin{Entry}{承诺}{8,10}{⼿、⾔}
  \begin{Phonetics}{承诺}{cheng2nuo4}[][HSK 6]
    \definition[个,句,份]{s.}{juramento; promessa; compromisso}
    \definition{v.}{prometer fazer algo; prometer empreender; comprometer-se a fazer algo}
  \end{Phonetics}
\end{Entry}

\begin{Entry}{承载}{8,10}{⼿、⾞}
  \begin{Phonetics}{承载}{cheng2zai4}[][HSK 7-9]
    \definition{v.}{suportar o peso; segurar o objeto e suportar seu peso}[桥梁承载着巨大的重量。===A ponte suporta uma carga pesada.]
  \end{Phonetics}
\end{Entry}

\begin{Entry}{披}{8}{⼿}
  \begin{Phonetics}{披}{pi1}[][HSK 5]
    \definition{v.}{colocar sobre os ombros; enrolar em volta; cobrir ou colocar sobre os ombros | abrir; desenrolar; espalhar | abrir-se; rachar}
  \end{Phonetics}
\end{Entry}

\begin{Entry}{抬}{8}{⼿}
  \begin{Phonetics}{抬}{tai2}[][HSK 5]
    \definition{clas.}{usado para objetos que precisam ser carregados por pessoas quando transportados (por exemplo, móveis)}
    \definition{v.}{levantar; elevar; puxar para cima | (por duas ou mais pessoas) carregar; transportar; duas ou mais pessoas carregando algo com as mãos ou nos ombros | discutir, debater (geralmente sem sentido ou sem importância)}
  \end{Phonetics}
\end{Entry}

\begin{Entry}{抬头}{8,5}{⼿、⼤}
  \begin{Phonetics}{抬头}{tai2 tou2}[][HSK 5]
    \definition{s.}{(em recibos, contas, etc.) nome do comprador ou beneficiário, o local no documento onde o nome do beneficiário ou destinatário é escrito}
    \definition{v.}{levantar a cabeça}
  \end{Phonetics}
\end{Entry}

\begin{Entry}{抬杠}{8,7}{⼿、⽊}
  \begin{Phonetics}{抬杠}{tai2/gang4}
    \definition{v.+compl.}{brigar; discutir; discutir por discutir; discutir sobre o certo e o errado (geralmente sem princípios) | Arcaico: carregar um caixão em barras resistentes}
  \end{Phonetics}
\end{Entry}

\begin{Entry}{抱}{8}{⼿}
  \begin{Phonetics}{抱}{bao4}[][HSK 4]
    \definition*{s.}{Sobrenome Bao}
    \definition{clas.}{braçada; medida dos dois braços}
    \definition{v.}{carregar no peito; segurar com ambos os braços; abraçar | ter o primeiro filho ou neto | adotar um bebê ou criança | ficar juntos, unidos | encaixar ou servir perfeitamente (roupas e sapatos do tamanho certo) | estimar; nutrir; abrigar; ter em mente | continuar; sobrecarregar com | chocar ovos}
  \end{Phonetics}
\end{Entry}

\begin{Entry}{抱负}{8,6}{⼿、⾙}
  \begin{Phonetics}{抱负}{bao4fu4}[][HSK 7-9]
    \definition{s.}{aspiração; ambição; objetivo elevado; grandes intenções e determinação, frequentemente usados na linguagem escrita}
  \end{Phonetics}
\end{Entry}

\begin{Entry}{抱怨}{8,9}{⼿、⼼}
  \begin{Phonetics}{抱怨}{bao4yuan4}[][HSK 5]
    \definition{v.}{reclamar ou expressar descontentamento ou insatisfação; falar com os outros sobre pessoas ou coisas com as quais você não está satisfeito}
  \end{Phonetics}
\end{Entry}

\begin{Entry}{抱歉}{8,14}{⼿、⽋}
  \begin{Phonetics}{抱歉}{bao4qian4}[][HSK 6]
    \definition{adj.}{pesaroso; arrependido; sentir pena de alguém porque você causou perda, inconveniência ou não atendeu às suas necessidades}
  \end{Phonetics}
\end{Entry}

\begin{Entry}{抵}{8}{⼿}
  \begin{Phonetics}{抵}{di3}
    \definition{v.}{apoiar; sustentar | resistir; suportar | compensar; fazer o bem | hipotecar; dar como garantia; garantir | equilibrar; cancelar; compensar | ser igual a; corresponder | alcançar; chegar a | colidir; dar cabeçada (por animais com chifres)}
  \end{Phonetics}
\end{Entry}

\begin{Entry}{抵达}{8,6}{⼿、⾡}
  \begin{Phonetics}{抵达}{di3da2}[][HSK 6]
    \definition{v.}{chegar; alcançar}
  \end{Phonetics}
\end{Entry}

\begin{Entry}{抵抗}{8,7}{⼿、⼿}
  \begin{Phonetics}{抵抗}{di3kang4}[][HSK 6]
    \definition{s.}{resistência}
    \definition{v.}{resistir; usar ação para resistir ou parar o ataque da outra parte}
  \end{Phonetics}
\end{Entry}

\begin{Entry}{抵制}{8,8}{⼿、⼑}
  \begin{Phonetics}{抵制}{di3zhi4}[][HSK 7-9]
    \definition{v.}{resistir; boicotar; bloquear, prevenir e impedir que forças externas invadam ou causem danos}
  \end{Phonetics}
\end{Entry}

\begin{Entry}{抵押}{8,8}{⼿、⼿}
  \begin{Phonetics}{抵押}{di3ya1}[][HSK 7-9]
    \definition{s.}{hipoteca; segurança; garantia}
    \definition{v.}{hipotecar; manter em penhor; penhorar}
  \end{Phonetics}
\end{Entry}

\begin{Entry}{抵挡}{8,9}{⼿、⼿}
  \begin{Phonetics}{抵挡}{di3dang3}[][HSK 7-9]
    \definition{v.}{resistir; suportar; bloquear}
  \end{Phonetics}
\end{Entry}

\begin{Entry}{抵御}{8,12}{⼿、⼻}
  \begin{Phonetics}{抵御}{di3yu4}[][HSK 7-9]
    \definition{v.}{resistir; suportar; afastar}[我们要抵御外敌的侵略。===Devemos resistir à invasão estrangeira.]
  \end{Phonetics}
\end{Entry}

\begin{Entry}{抵销}{8,12}{⼿、⾦}
  \begin{Phonetics}{抵销}{di3xiao1}[][HSK 7-9]
    \definition{v.}{compensar}[三笔债务可以抵销。===As três dívidas podem ser compensadas.]
  \end{Phonetics}
\end{Entry}

\begin{Entry}{抵触}{8,13}{⼿、⾓}
  \begin{Phonetics}{抵触}{di3chu4}[][HSK 7-9]
    \definition{adj.}{conflitante; contraditório}
    \definition{s.}{conflito}
    \definition{v.}{entrar em conflito; contradizer}
  \end{Phonetics}
\end{Entry}

\begin{Entry}{抹}{8}{⼿}
  \begin{Phonetics}{抹}{ma1}
    \definition{v.}{esfregar; limpar | deslizar algo para fora; tirar}
  \end{Phonetics}
  \begin{Phonetics}{抹}{mo3}
    \definition{v.}{colocar; aplicar; untar; engessar | limpar | anular; apagar | (para nuvem, etc.) irradiar; raiar; riscar; traçar | riscar; cancelar; marcar; remover; excluir}
  \end{Phonetics}
  \begin{Phonetics}{抹}{mo4}
    \definition{v.}{rebocar; engessar; alisar a massa ou o gesso com uma espátula | virar; contornar; dar uma volta de perto}
  \end{Phonetics}
\end{Entry}

\begin{Entry}{抹泪}{8,8}{⼿、⽔}
  \begin{Phonetics}{抹泪}{mo3lei4}
    \definition{v.}{limpar as lágrimas | (figurativo) derramar lágrimas}
  \end{Phonetics}
\end{Entry}

\begin{Entry}{押}{8}{⼿}
  \begin{Phonetics}{押}{ya1}
    \definition*{s.}{Sobrenome Ya}
    \definition{s.}{assinatura; marca em vez de assinatura; nome assinado ou símbolo desenhado}
    \definition{v.}{dar como garantia; hipotecar; penhorar | deter; levar sob custódia | escoltar | assinar (um documento, contrato, etc.); colocar sua assinatura (ou marcar no lugar da assinatura)}
  \end{Phonetics}
\end{Entry}

\begin{Entry}{押后}{8,6}{⼿、⼝}
  \begin{Phonetics}{押后}{ya1hou4}
    \definition{v.}{encerrar | adiar}
  \end{Phonetics}
\end{Entry}

\begin{Entry}{押运}{8,7}{⼿、⾡}
  \begin{Phonetics}{押运}{ya1yun4}
    \definition{v.}{escoltar sob guarda | escoltar (bens ou fundos)}
  \end{Phonetics}
\end{Entry}

\begin{Entry}{押注}{8,8}{⼿、⽔}
  \begin{Phonetics}{押注}{ya1zhu4}
    \definition{v.}{apostar}
  \end{Phonetics}
\end{Entry}

\begin{Entry}{押金}{8,8}{⼿、⾦}
  \begin{Phonetics}{押金}{ya1jin1}[][HSK 5]
    \definition[笔,份,些]{s.}{caução; sinal; depósito; dinheiro como garantia}
  \end{Phonetics}
\end{Entry}

\begin{Entry}{押送}{8,9}{⼿、⾡}
  \begin{Phonetics}{押送}{ya1song4}
    \definition{v.}{enviar sob escolta | transportar um detido}
  \end{Phonetics}
\end{Entry}

\begin{Entry}{押租}{8,10}{⼿、⽲}
  \begin{Phonetics}{押租}{ya1zu1}
    \definition{s.}{depósito de aluguel}
  \end{Phonetics}
\end{Entry}

\begin{Entry}{押韵}{8,13}{⼿、⾳}
  \begin{Phonetics}{押韵}{ya1yun4}
    \definition{v.}{rimar}
  \end{Phonetics}
\end{Entry}

\begin{Entry}{抽}{8}{⼿}
  \begin{Phonetics}{抽}{chou1}[][HSK 4]
    \definition{v.}{retirar; tirar (do meio); retirar, puxar ou arrancar algo que está preso ou emaranhado em outra coisa | tirar, retirar (uma parte de um todo) | (certas plantas) começar a crescer, produzir | bombear | encolher; contrair | chicotear; açoitar; surrar | dirigir; conduzir | encontrar tempo; libertar-se; sair de alguma coisa}
  \end{Phonetics}
\end{Entry}

\begin{Entry}{抽屉}{8,8}{⼿、⼫}
  \begin{Phonetics}{抽屉}{chou1ti4}[][HSK 7-9]
    \definition[个,层,组]{s.}{gaveta}
  \end{Phonetics}
\end{Entry}

\begin{Entry}{抽奖}{8,9}{⼿、⼤}
  \begin{Phonetics}{抽奖}{chou1 jiang3}[][HSK 4]
    \definition{s.}{loteria; sorteio de loteria}
  \end{Phonetics}
\end{Entry}

\begin{Entry}{抽烟}{8,10}{⼿、⽕}
  \begin{Phonetics}{抽烟}{chou1/yan1}[][HSK 4]
    \definition{v.+compl.}{fumar (um cigarro ou um cachimbo)}
  \end{Phonetics}
\end{Entry}

\begin{Entry}{抽象}{8,11}{⼿、⾗}
  \begin{Phonetics}{抽象}{chou1xiang4}[][HSK 7-9]
    \definition{adj.}{abstrato}[抽象的艺术需要想象力。===A arte abstrata requer imaginação.]
    \definition{v.}{abstrair}[这个理论很难抽象。===Essa teoria é difícil de abstrair.]
  \end{Phonetics}
\end{Entry}

\begin{Entry}{抽签}{8,13}{⼿、⽵}
  \begin{Phonetics}{抽签}{chou1/qian1}[][HSK 7-9]
    \definition{v.+compl.}{tirar/lançar sorte; realizar/fazer um sorteio}[他们抽签决定胜者。===Eles fizeram um sorteio para decidir o vencedor.]
  \end{Phonetics}
\end{Entry}

\begin{Entry}{担}{8}{⼿}
  \begin{Phonetics}{担}{dan1}[][HSK 7-9]
    \definition{v.}{carregar em uma vara de ombro e baldes; carregar nos ombros | assumir; empreender; não ter medo de correr riscos}
  \end{Phonetics}
  \begin{Phonetics}{担}{dan4}[][HSK 7-9]
    \definition{clas.}{dan, uma unidade de peso (=50 quilogramas) ; 100 jin = 1 dan | usado em coisas usadas para transportar cargas}
    \definition{s.}{carga; fardo; cargas de mercadorias transportadas em uma vara de ombro por um mascate itinerante}
  \end{Phonetics}
\end{Entry}

\begin{Entry}{担子}{8,3}{⼿、⼦}
  \begin{Phonetics}{担子}{dan4zi5}[][HSK 7-9]
    \definition[副,个]{s.}{vara de transporte (ou de ombro) e as cargas sob ela; canga; carga; fardo | tarefa}
  \end{Phonetics}
\end{Entry}

\begin{Entry}{担心}{8,4}{⼿、⼼}
  \begin{Phonetics}{担心}{dan1xin1}[][HSK 4]
    \definition{v.}{preocupar-se; ficar ansioso; sentir-se desconfortável com algo}
  \end{Phonetics}
\end{Entry}

\begin{Entry}{担任}{8,6}{⼿、⼈}
  \begin{Phonetics}{担任}{dan1ren4}[][HSK 4]
    \definition{v.}{servir como; assumir o cargo de; ocupar o posto de; ocupar um determinado cargo ou emprego}
  \end{Phonetics}
\end{Entry}

\begin{Entry}{担当}{8,6}{⼿、⼹}
  \begin{Phonetics}{担当}{dan1dang1}[][HSK 7-9]
    \definition{v.}{aceitar e assumir responsabilidade; empreender (responsabilidade, trabalho, despesas)}
  \end{Phonetics}
\end{Entry}

\begin{Entry}{担负}{8,6}{⼿、⾙}
  \begin{Phonetics}{担负}{dan1fu4}[][HSK 7-9]
    \definition{v.}{suportar; carregar; assumir; ser encarregado de}
  \end{Phonetics}
\end{Entry}

\begin{Entry}{担忧}{8,7}{⼿、⼼}
  \begin{Phonetics}{担忧}{dan1 you1}[][HSK 6]
    \definition[项,条,套,种]{v.}{preocupar-se; estar ansioso}
  \end{Phonetics}
\end{Entry}

\begin{Entry}{担保}{8,9}{⼿、⼈}
  \begin{Phonetics}{担保}{dan1bao3}[][HSK 4]
    \definition{v.}{garantir; atestar; expressar responsabilidade e garantir que não haverá problemas ou que eles serão resolvidos}
  \end{Phonetics}
\end{Entry}

\begin{Entry}{拆}{8}{⼿}
  \begin{Phonetics}{拆}{chai1}[][HSK 5]
    \definition{v.}{rasgar; desmontar; separar o que está unido | derrubar; desmantelar; demolir; refere-se especificamente à demolição de edifícios}
  \end{Phonetics}
\end{Entry}

\begin{Entry}{拆迁}{8,6}{⼿、⾡}
  \begin{Phonetics}{拆迁}{chai1 qian1}[][HSK 6]
    \definition{v.}{demolir uma casa velha e realocar seus ocupantes em outro lugar; devido às necessidades de construção, unidades ou casas residenciais são demolidas e realocadas em outros lugares}
  \end{Phonetics}
\end{Entry}

\begin{Entry}{拆除}{8,9}{⼿、⾩}
  \begin{Phonetics}{拆除}{chai1 chu2}[][HSK 5]
    \definition{v.}{desmantelar; demolir; derrubar; remover (um edifício, etc.)}
  \end{Phonetics}
\end{Entry}

\begin{Entry}{拉}{8}{⼿}
  \begin{Phonetics}{拉}{la1}[][HSK 2]
    \definition{s.}{abreviação de América Latina, 拉丁美洲}
    \definition{v.}{puxar; arrastar; rebocar | transportar por veículo; rebocar | arrastar (ou puxar) para fora | mover (tropas para um lugar) | dar uma mãozinha; ajudar | arrastar para dentro; implicar; envolver | criar (criança) | atrair; conquistar; solicitar; angariar votos | bater-papo | organizar; preparar | ter dívidas; estar endividado | pressionar; recrutar à força | (no tênis, tênis de mesa, etc.) levantar (a bola) | tocar (certos instrumentos musicais); puxar uma parte do instrumento para que ele emita som | prolongar; espaçar | envolver-se em | (coloquial) esvaziar os intestinos | levantar, uma das técnicas do tênis de mesa | destruir; esmagar; quebrar}
  \seealsoref{拉丁美洲}{la1ding1 mei3zhou1}
  \end{Phonetics}
  \begin{Phonetics}{拉}{la4}
    \definition{s.}{usado em 拉拉蛄 \dpy{la4la4gu3}}
  \seealsoref{拉拉蛄}{la4la4gu3}
  \end{Phonetics}
\end{Entry}

\begin{Entry}{拉丁美洲}{8,2,9,9}{⼿、⼀、⽺、⽔}
  \begin{Phonetics}{拉丁美洲}{la1ding1 mei3zhou1}
    \definition*{s.}{América Latina, nome coletivo dos países da América Central e do Sul, devido ao fato de a maioria de seus habitantes ser descendente de povos latinos e de a língua falada ser do grupo latino}
  \end{Phonetics}
\end{Entry}

\begin{Entry}{拉开}{8,4}{⼿、⼶}
  \begin{Phonetics}{拉开}{la1 kai1}[][HSK 4]
    \definition{v.}{puxar para abrir; recuar| ampliar; espaçar; distanciar; afastar; separar}
  \end{Phonetics}
\end{Entry}

\begin{Entry}{拉布布}{8,5,5}{⼿、⼱、⼱}
  \begin{Phonetics}{拉布布}{la1bu4bu4}
    \definition*{s.}{Labubu}
  \end{Phonetics}
\end{Entry}

\begin{Entry}{拉拉队}{8,8,4}{⼿、⼿、⾩}
  \begin{Phonetics}{拉拉队}{la1la1dui4}
    \definition{s.}{claque | torcida}
  \end{Phonetics}
\end{Entry}

\begin{Entry}{拉拉蛄}{8,8,11}{⼿、⼿、⾍}
  \begin{Phonetics}{拉拉蛄}{la4la4gu3}
    \variantof{蝲蝲蛄}
  \end{Phonetics}
\end{Entry}

\begin{Entry}{拉萨}{8,11}{⼿、⾋}
  \begin{Phonetics}{拉萨}{la1sa4}
    \definition*{s.}{Lhasa, capital da Região Autônoma do Tibete, 西藏自治区}
  \seealsoref{西藏自治区}{xi1zang4 zi4zhi4qu1}
  \end{Phonetics}
\end{Entry}

\begin{Entry}{拊}{8}{⼿}
  \begin{Phonetics}{拊}{fu3}
    \definition{v.}{Literário: bater palmas; esbofetear; golpear}
  \end{Phonetics}
\end{Entry}

\begin{Entry}{拌}{8}{⼿}
  \begin{Phonetics}{拌}{ban4}[][HSK 7-9]
    \definition{v.}{misturar | mexer e misturar | discutir; brigar; ter uma discussão}
  \end{Phonetics}
\end{Entry}

\begin{Entry}{拍}{8}{⼿}
  \begin{Phonetics}{拍}{pai1}[][HSK 3]
    \definition[个,副,对]{s.}{bastão; raquete | batida; tempo; (música) uma unidade para medir a duração de uma nota musical}
    \definition{v.}{tirar (uma foto); usar uma câmera para capturar imagens de pessoas e objetos em filme | dar um tapinha; bater suavemente com as mãos ou ferramentas | bater asas | bater (ondas do mar) | enviar (um telegrama, etc.) | bajular}
  \end{Phonetics}
\end{Entry}

\begin{Entry}{拍马}{8,3}{⼿、⾺}
  \begin{Phonetics}{拍马}{pai1ma3}
    \definition{v.}{instigar um cavalo dando tapinhas em seu traseiro | lisonjear | bajular}
  \seealsoref{拍马屁}{pai1ma3pi4}
  \end{Phonetics}
\end{Entry}

\begin{Entry}{拍马屁}{8,3,7}{⼿、⾺、⼫}
  \begin{Phonetics}{拍马屁}{pai1ma3pi4}
    \definition{s.}{puxa-saco | bajulador}
    \definition{v.}{puxar o saco | bajular}
  \seealsoref{拍马}{pai1ma3}
  \end{Phonetics}
\end{Entry}

\begin{Entry}{拍摄}{8,13}{⼿、⼿}
  \begin{Phonetics}{拍摄}{pai1 she4}[][HSK 5]
    \definition{s.}{fotografar; tirar (uma foto); usar uma câmera fotográfica para capturar imagens de pessoas e objetos}
  \end{Phonetics}
\end{Entry}

\begin{Entry}{拍照}{8,13}{⼿、⽕}
  \begin{Phonetics}{拍照}{pai1/zhao4}[][HSK 4]
    \definition{v.+compl.}{fotografar; tirar uma foto}
  \end{Phonetics}
\end{Entry}

\begin{Entry}{拐}{8}{⼿}
  \begin{Phonetics}{拐}{guai3}[][HSK 6]
    \definition[支,根,副]{s.}{muleta; bengala; uma bengala com uma barra horizontal na parte superior, usada por pessoas com doenças ou deficiências nos membros inferiores para ajudá-las a caminhar |
sete; forma falada do numeral 七 | esquina; curva; canto}
    \definition{v.}{virar; girar; mudar de direção enquanto se move | enganar | mudar; transformar | mancar}
  \seealsoref{七}{qi1}
  \end{Phonetics}
\end{Entry}

\begin{Entry}{拐杖}{8,7}{⼿、⽊}
  \begin{Phonetics}{拐杖}{guai3zhang4}[][HSK 7-9]
    \definition[个,根,支,副]{s.}{muleta; bengala}
  \end{Phonetics}
\end{Entry}

\begin{Entry}{拐弯}{8,9}{⼿、⼸}
  \begin{Phonetics}{拐弯}{guai3/wan1}[][HSK 7-9]
    \definition[个]{s.}{esquina; curva; canto}
    \definition{v.}{virar; virar uma esquina; indica mudança de direção da viagem | dar meia-volta; seguir um novo curso; indica mudança de ideias, linguagem, etc.}
  \end{Phonetics}
\end{Entry}

\begin{Entry}{拔}{8}{⼿}
  \begin{Phonetics}{拔}{ba2}[][HSK 5]
    \definition{v.aux.}{puxar para cima; puxar para fora; arrastar para fora | extrair; sugar | escolher; selecionar | superar; destacar-se entre | apreender; capturar | esfriar na água; mergulhar algo em água fria para que esfrie}
  \end{Phonetics}
\end{Entry}

\begin{Entry}{拔尖}{8,6}{⼿、⼩}
  \begin{Phonetics}{拔尖}{ba2/jian1}
    \definition{adj.}{topo de linha | fora do comum | o melhor}
    \definition{v.+compl.}{empurrar-se para a frente | sentir que é superior aos outros}
  \end{Phonetics}
\end{Entry}

\begin{Entry}{拖}{8}{⼿}
  \begin{Phonetics}{拖}{tuo1}[][HSK 6]
    \definition{v.}{puxar; arrastar; transportar; puxar um objeto para movê-lo contra o solo ou outra superfície | esfregar; limpar o chão com uma ferramenta especial para esfregar | atrasar; prolongar; procrastinar; arrastar; coisas que deveriam ser feitas nunca são iniciadas ou concluídas; uma certa nota é prolongada por um longo tempo | atrasar; conter; segurar; restringir}
  \end{Phonetics}
\end{Entry}

\begin{Entry}{拖拉机}{8,8,6}{⼿、⼿、⽊}
  \begin{Phonetics}{拖拉机}{tuo1la1ji1}
    \definition[台]{s.}{trator}
  \end{Phonetics}
\end{Entry}

\begin{Entry}{拖鞋}{8,15}{⼿、⾰}
  \begin{Phonetics}{拖鞋}{tuo1 xie2}[][HSK 6]
    \definition[双,只]{s.}{chinelos; samdálias; babouche; sapatos sem cabedal geralmente são usados ​​em ambientes fechados}
  \end{Phonetics}
\end{Entry}

\begin{Entry}{招}{8}{⼿}
  \begin{Phonetics}{招}{zhao1}[][HSK 6]
    \definition*{s.}{Sobrenome Zhao}
    \definition{s.}{\emph{banner}; Faixas e outros itens costumavam ser pendurados nas entradas de hotéis, restaurantes ou lojas para atrair clientes | movimento; estratagema; artifício; meios ou táticas | movimentos de artes marciais}
    \definition{v.}{acenar; gestuar para alguém ver | alistar; inscrever; recrutar | incorrer; cortejar; atrair; provocar (um certo resultado ou reação) | provocar; tocar ou provocar a outra pessoa com palavras ou ações | confessar (culpa); assumir (culpa) | infectar; ser contagioso}
  \end{Phonetics}
\end{Entry}

\begin{Entry}{招手}{8,4}{⼿、⼿}
  \begin{Phonetics}{招手}{zhao1/shou3}[][HSK 5]
    \definition{v.+compl.}{acenar; chamar a atenção; levantar a mão e acenar com a palma, para indicar que a outra pessoa se aproxime ou para cumprimentá-la}
  \end{Phonetics}
\end{Entry}

\begin{Entry}{招生}{8,5}{⼿、⽣}
  \begin{Phonetics}{招生}{zhao1/sheng1}[][HSK 5]
    \definition{v.+compl.}{conseguir alunos; matricular novos alunos; recrutar novos alunos}
  \end{Phonetics}
\end{Entry}

\begin{Entry}{招呼}{8,8}{⼿、⼝}
  \begin{Phonetics}{招呼}{zhao1 hu5}[][HSK 4]
    \definition{v.}{chamar; chamar a atenção com palavras ou gestos | cumprimentar; saudar; cumprimentar ou despedir-se das pessoas com palavras ou gestos | pedir a alguém para fazer algo; fazer solicitações, pedir ajuda ou fazer coisas | receber e dar boas-vindas aos convidados}
  \end{Phonetics}
\end{Entry}

\begin{Entry}{招数}{8,13}{⼿、⽁}
  \begin{Phonetics}{招数}{zhao1shu4}
    \definition{s.}{estratégia | movimento (no xadrez, no palco, nas artes marciais) | esquema | truque}
  \end{Phonetics}
\end{Entry}

\begin{Entry}{招聘}{8,13}{⼿、⽿}
  \begin{Phonetics}{招聘}{zhao1pin4}[][HSK 6]
    \definition{v.}{contratar; procurar; recrutar; convidar candidatos para um emprego}
  \end{Phonetics}
\end{Entry}

\begin{Entry}{拥}{8}{⼿}
  \begin{Phonetics}{拥}{yong1}
    \definition{v.}{segurar nos braços; abraçar | reunir em volta; envolver em volta | aglomerar-se; enxamear | para apoiar | (literário) ter; possuir}
  \end{Phonetics}
\end{Entry}

\begin{Entry}{拥有}{8,6}{⼿、⽉}
  \begin{Phonetics}{拥有}{yong1you3}[][HSK 5]
    \definition{v.}{possuir; deter; ter (grande quantidade de terras, população, bens, etc.)}
  \end{Phonetics}
\end{Entry}

\begin{Entry}{拥抱}{8,8}{⼿、⼿}
  \begin{Phonetics}{拥抱}{yong1bao4}[][HSK 5]
    \definition[个,次]{s.}{abraço}
    \definition{v.}{abraçar; segurar em seus braços; abraçar para demonstrar afeto}
  \end{Phonetics}
\end{Entry}

\begin{Entry}{拧}{8}{⼿}
  \begin{Phonetics}{拧}{ning2}
    \definition{v.}{torcer | beliscar; torcer a pele com os dedos e virá-la com força}
  \end{Phonetics}
  \begin{Phonetics}{拧}{ning3}
    \definition{adj.}{errado; equivocado; de cabeça para baixo; oposto}
    \definition{v.}{torcer; parafusar | divergir; discordar; estar em desacordo}
  \end{Phonetics}
  \begin{Phonetics}{拧}{ning4}
    \definition{adj.}{teimoso}
  \end{Phonetics}
\end{Entry}

\begin{Entry}{拧开}{8,4}{⼿、⼶}
  \begin{Phonetics}{拧开}{ning3kai1}
    \definition{v.}{desaparafusar | desatarrachar | torcer (uma tampa) | abrir (uma torneira) | ligar (girando um botão) | girar (maçaneta da porta)}
  \end{Phonetics}
\end{Entry}

\begin{Entry}{拨}{8}{⼿}
  \begin{Phonetics}{拨}{bo1}[][HSK 7-9]
    \definition{clas.}{usado para agrupar pessoas; grupo; lote}
    \definition{v.}{mover (mexer) com a mão, o pé, o bastão, etc.; usar as mãos, os pés ou os bastões para mover objetos | atribuir; alocar; reservar | virar-se; inverter a marcha | dedilhar (uma corda de violão) com os dedos ou com um instrumento | chamar (alguém)}
  \end{Phonetics}
\end{Entry}

\begin{Entry}{拨及}{8,3}{⼿、⼃}
  \begin{Phonetics}{拨及}{bo1ji2}[][HSK 7-9]
    \definition{v.}{espalhar para; envolver; afetar}
  \end{Phonetics}
\end{Entry}

\begin{Entry}{拨打}{8,5}{⼿、⼿}
  \begin{Phonetics}{拨打}{bo1 da3}[][HSK 6]
    \definition{v.}{ligar; discar; de acordo com o número da chamada, discar o número no telefone ou pressionar as teclas numéricas para fazer uma chamada}
  \end{Phonetics}
\end{Entry}

\begin{Entry}{拨转}{8,8}{⼿、⾞}
  \begin{Phonetics}{拨转}{bo1zhuan3}
    \definition{v.}{transferir (fundos, etc.) | virar | dar a volta}
  \end{Phonetics}
\end{Entry}

\begin{Entry}{拨通}{8,10}{⼿、⾡}
  \begin{Phonetics}{拨通}{bo1/tong1}[][HSK 7-9]
    \definition{v.+compl.}{discar (os números de um telefone, etc.)}
  \end{Phonetics}
\end{Entry}

\begin{Entry}{拨款}{8,12}{⼿、⽋}
  \begin{Phonetics}{拨款}{bo1kuan3}[][HSK 7-9]
    \definition[项,笔]{s.}{dinheiro apropriado; apropriação; subsídio financeiro do estado; alocação de fundos; financiamento alocado}
    \definition{v.}{apropriar-se de dinheiro; alocar fundos}
  \end{Phonetics}
\end{Entry}

\begin{Entry}{放}{8}{⽅}
  \begin{Phonetics}{放}{fang4}[][HSK 1]
    \definition{v.}{deixar ir; libertar; soltar | ceder; deixar-se levar | levar para se alimentar; pastar | soltar; liberar (ou expelir) | exibir (um filme, etc.); reproduzir (um disco, etc.) | acender; inflamar | emprestar (dinheiro) com juros | tornar maior ou mais longo; soltar; abaixar | moderar (a atitude ou o comportamento de alguém) | (de flores) florescer; abrir | colocar; posicionar; deitar | fazer com que algo (ou alguém) caia no chão | deixar de lado; guardar (para uso futuro); conservar | (seguido por 着\dots 不\dots) permitir que algo permaneça (por fazer, por pegar, por usar, etc.) | adicionar; colocar | colocar em pastagem; soltar para caçar | deixar de lado; suspender; interromper | remover; aliviar; livrar-se; proteger; libertar | deixar-se levar; sem restrições; libertino | mandar embora; tirar o prisioneiro da prisão e deportá-lo para uma região remota | distribuir; emitir; lançar | atear fogo | expandir; ampliar; prolongar | reajustar-se até certo ponto; controlar suas ações, adotar uma determinada atitude, atingir um certo equilíbrio | derrubar}
  \end{Phonetics}
\end{Entry}

\begin{Entry}{放下}{8,3}{⽅、⼀}
  \begin{Phonetics}{放下}{fang4 xia4}[][HSK 2]
    \definition{v.}{deitar-se; colocar no chão| deixar ir; soltar; desistir; largar | colocar; acomodar; depositar}
  \end{Phonetics}
\end{Entry}

\begin{Entry}{放大}{8,3}{⽅、⼤}
  \begin{Phonetics}{放大}{fang4da4}[][HSK 5]
    \definition{v.}{amplificar; magnificar; aumentar; ampliar; aumentar o tamanho de imagens, textos, sons, etc.}
  \end{Phonetics}
\end{Entry}

\begin{Entry}{放飞}{8,3}{⽅、⾶}
  \begin{Phonetics}{放飞}{fang4fei1}
    \definition{s.}{deixar voar}
  \end{Phonetics}
\end{Entry}

\begin{Entry}{放心}{8,4}{⽅、⼼}
  \begin{Phonetics}{放心}{fang4xin1}[][HSK 2]
    \definition{adj.}{despreocupado}
    \definition{v.}{confiar; ter confiança em alguém; sentir-se aliviado; ficar tranquilo; ficar com a consciência tranquila}
  \end{Phonetics}
\end{Entry}

\begin{Entry}{放水}{8,4}{⽅、⽔}
  \begin{Phonetics}{放水}{fang4/shui3}[][HSK 7-9]
    \definition{v.+compl.}{ligar a água; deixar a água fluir, geralmente significa abrir a fonte de água ou fornecer uma determinada vazão de água | (reservatório, etc.) retirar água; drenar água de reservatórios, lagoas, etc. para irrigação ou outros fins | (em uma competição, etc.) facilitar as coisas para alguém; perder um jogo intencionalmente; deixar deliberadamente o adversário vencer facilmente durante uma partida}
  \end{Phonetics}
\end{Entry}

\begin{Entry}{放出}{8,5}{⽅、⼐}
  \begin{Phonetics}{放出}{fang4chu1}
    \definition{v.}{liberar | libertar}
  \end{Phonetics}
\end{Entry}

\begin{Entry}{放电}{8,5}{⽅、⽥}
  \begin{Phonetics}{放电}{fang4dian4}
    \definition{s.}{descarga elétrica}
  \end{Phonetics}
\end{Entry}

\begin{Entry}{放任}{8,6}{⽅、⼈}
  \begin{Phonetics}{放任}{fang4ren4}
    \definition{v.}{ignorar | saciar-se | deixar sozinho}
  \end{Phonetics}
\end{Entry}

\begin{Entry}{放过}{8,6}{⽅、⾡}
  \begin{Phonetics}{放过}{fang4guo4}[][HSK 7-9]
    \definition{v.}{deixar escapar; perder}
  \end{Phonetics}
\end{Entry}

\begin{Entry}{放弃}{8,7}{⽅、⼶}
  \begin{Phonetics}{放弃}{fang4qi4}[][HSK 5]
    \definition{v.}{desistir, abandonar; descartar (direitos originais, reivindicações, opiniões, etc.)}
  \end{Phonetics}
\end{Entry}

\begin{Entry}{放弃权利}{8,7,6,7}{⽅、⼶、⽊、⼑}
  \begin{Phonetics}{放弃权利}{fang4qi4 quan2li4}
    \definition{s.}{renúncia}
  \end{Phonetics}
\end{Entry}

\begin{Entry}{放弃者}{8,7,8}{⽅、⼶、⽼}
  \begin{Phonetics}{放弃者}{fang4qi4zhe3}
    \definition{s.}{desistente}
  \end{Phonetics}
\end{Entry}

\begin{Entry}{放纵}{8,7}{⽅、⽷}
  \begin{Phonetics}{放纵}{fang4zong4}[][HSK 7-9]
    \definition{adj.}{grosseiro; inculto; autoindulgente; indisciplinado}
    \definition{v.}{satisfazer; ser conivente com; bajular; deixar alguém fazer o que quer}
  \end{Phonetics}
\end{Entry}

\begin{Entry}{放走}{8,7}{⽅、⾛}
  \begin{Phonetics}{放走}{fang4zou3}
    \definition{v.}{permitir (uma pessoa ou um animal) ir | liberar | libertar}
  \end{Phonetics}
\end{Entry}

\begin{Entry}{放到}{8,8}{⽅、⼑}
  \begin{Phonetics}{放到}{fang4 dao4}[][HSK 3]
    \definition{v.}{colocar em; meter}
  \end{Phonetics}
\end{Entry}

\begin{Entry}{放学}{8,8}{⽅、⼦}
  \begin{Phonetics}{放学}{fang4/xue2}[][HSK 1]
    \definition{v.+compl.}{encerrar; sair da escola; as aulas terminaram; a escola acabou (por hoje); voltar para casa depois de um dia ou meio dia de aula}
  \end{Phonetics}
\end{Entry}

\begin{Entry}{放松}{8,8}{⽅、⽊}
  \begin{Phonetics}{放松}{fang4song1}[][HSK 4]
    \definition{v.}{relaxar; afrouxar; soltar; desprender}
  \end{Phonetics}
\end{Entry}

\begin{Entry}{放养}{8,9}{⽅、⼋}
  \begin{Phonetics}{放养}{fang4yang3}
    \definition{v.}{criar (gado, peixes, culturas, etc.) | crescer | criar}
  \end{Phonetics}
\end{Entry}

\begin{Entry}{放映}{8,9}{⽅、⽇}
  \begin{Phonetics}{放映}{fang4ying4}[][HSK 7-9]
    \definition{v.}{mostrar (um filme); exibir; projetar; usar um dispositivo de luz forte para iluminar a imagem de uma foto ou filme em uma tela ou parede}
  \end{Phonetics}
\end{Entry}

\begin{Entry}{放假}{8,11}{⽅、⼈}
  \begin{Phonetics}{放假}{fang4/jia4}[][HSK 1]
    \definition{v.}{tirar férias (ou feriado); ter um dia de folga}
    \definition{v.+compl.}{tirar férias (ou feriado); começar as férias; ter um dia de folga; estar de férias (feriado)}
  \end{Phonetics}
\end{Entry}

\begin{Entry}{放置}{8,13}{⽅、⽹}
  \begin{Phonetics}{放置}{fang4zhi4}[][HSK 7-9]
    \definition{v.}{colocar; deitar; deixar de lado}
  \end{Phonetics}
\end{Entry}

\begin{Entry}{放肆}{8,13}{⽅、⾀}
  \begin{Phonetics}{放肆}{fang4si4}[][HSK 7-9]
    \definition{adj.}{desenfreado; devasso; atrevido; descontrolado; descreve agir de forma imprudente e sem escrúpulos}
  \end{Phonetics}
\end{Entry}

\begin{Entry}{放鞭炮}{8,18,9}{⽅、⾰、⽕}
  \begin{Phonetics}{放鞭炮}{fang4bian1pao4}
    \definition{s.}{um conjunto de bombinhas ou traques}
  \end{Phonetics}
\end{Entry}

\begin{Entry}{斧}{8}{⽄}
  \begin{Phonetics}{斧}{fu3}
    \definition[把,只]{s.}{machado; machadinha | machado de batalha (um tipo de arma usada na China antiga)}
  \end{Phonetics}
\end{Entry}

\begin{Entry}{斧子}{8,3}{⽄、⼦}
  \begin{Phonetics}{斧子}{fu3zi5}[][HSK 7-9]
    \definition[把,个]{s.}{machado; machadinha}
  \end{Phonetics}
\end{Entry}

\begin{Entry}{斩}{8}{⽄}
  \begin{Phonetics}{斩}{zhan3}
    \definition*{s.}{Sobrenome Zhan}
    \definition{v.}{matar; cortar; picar | (dialeto) tosquiar; chantagear | decapitar}
  \end{Phonetics}
\end{Entry}

\begin{Entry}{斩获}{8,10}{⽄、⾋}
  \begin{Phonetics}{斩获}{zhan3huo4}
    \definition{v.}{matar ou capturar (em batalha) | (figurativo) (esportes) marcar (um gol), ganhar (uma medalha) | (figurativo) colher recompensas, obter ganhos}
  \end{Phonetics}
\end{Entry}

\begin{Entry}{旺}{8}{⽇}
  \begin{Phonetics}{旺}{wang4}
    \definition{adj.}{próspero; florescente; vigoroso | abundante; numeroso}
  \end{Phonetics}
\end{Entry}

\begin{Entry}{旺季}{8,8}{⽇、⼦}
  \begin{Phonetics}{旺季}{wang4ji4}
    \definition{s.}{alta temporada; período de pico; temporada movimentada; a estação em que um determinado produto é produzido em grandes quantidades ou quando os negócios estão crescendo (diferente de 淡季)}
  \seealsoref{淡季}{dan4ji4}
  \end{Phonetics}
\end{Entry}

\begin{Entry}{昂}{8}{⽇}
  \begin{Phonetics}{昂}{ang2}
    \definition{adj.}{alto; subindo}
    \definition{v.}{manter (a cabeça) erguida | elevar; levantar; olhar para cima}
  \end{Phonetics}
\end{Entry}

\begin{Entry}{昂贵}{8,9}{⽇、⾙}
  \begin{Phonetics}{昂贵}{ang2gui4}[][HSK 7-9]
    \definition{adj.}{caro; dispendioso; algo é muito caro, o preço é particularmente alto; metaforicamente, o custo de fazer algo é particularmente alto}
  \end{Phonetics}
\end{Entry}

\begin{Entry}{昌}{8}{⽇}
  \begin{Phonetics}{昌}{chang1}
    \definition*{s.}{Sobrenome Chang}
    \definition{adj.}{próspero; florescente | adequado; bom}
  \end{Phonetics}
\end{Entry}

\begin{Entry}{昌盛}{8,11}{⽇、⽫}
  \begin{Phonetics}{昌盛}{chang1 sheng4}[][HSK 6]
    \definition{adj.}{(país, nação, etc.) próspero; florescente}
  \end{Phonetics}
\end{Entry}

\begin{Entry}{明}{8}{⽇}
  \begin{Phonetics}{明}{ming2}
    \definition*{s.}{Dinastia Ming (1368-1644) | Sobrenome Ming}
    \definition{adj.}{claro; brilhante; brilhante | claro; distinto; de fácil entendimento | aberto; evidente; explícito; exposto | de ​​olhos aguçados; boa visão; visão nítida | honesto}
    \definition{adv.}{claramente; definitivamente; aparentemente; de fato}
    \definition{s.}{imediatamente a seguir no tempo; ao lado deste ano e hoje; visão}
    \definition{v.}{mostrar; revelar; tornar conhecido; deixar claro | entender; compreender}
  \end{Phonetics}
\end{Entry}

\begin{Entry}{明天}{8,4}{⽇、⼤}
  \begin{Phonetics}{明天}{ming2tian1}[][HSK 1]
    \definition{s.}{amanhã | futuro próximo}
  \end{Phonetics}
\end{Entry}

\begin{Entry}{明日}{8,4}{⽇、⽇}
  \begin{Phonetics}{明日}{ming2 ri4}[][HSK 6]
    \definition{s.}{amanhã}
  \seealsoref{明天}{ming2tian1}
  \end{Phonetics}
\end{Entry}

\begin{Entry}{明白}{8,5}{⽇、⽩}
  \begin{Phonetics}{明白}{ming2bai5}[][HSK 1]
    \definition{adj.}{claro; óbvio; evidente; inequívoco | sensato; razoável | aberto; franco; inequívoco; explícito}
    \definition{v.}{entender; compreender; saber}
  \end{Phonetics}
\end{Entry}

\begin{Entry}{明年}{8,6}{⽇、⼲}
  \begin{Phonetics}{明年}{ming2 nian2}[][HSK 1]
    \definition{s.}{próximo ano}
  \end{Phonetics}
\end{Entry}

\begin{Entry}{明明}{8,8}{⽇、⽇}
  \begin{Phonetics}{明明}{ming2ming2}[][HSK 5]
    \definition{adv.}{obviamente; claramente; sem dúvida; indica que o fenômeno ou princípio é evidente}
  \end{Phonetics}
\end{Entry}

\begin{Entry}{明码}{8,8}{⽇、⽯}
  \begin{Phonetics}{明码}{ming2ma3}
    \definition{s.}{código simples, em claro (oposto a 密码) | preço claramente marcado}
  \seealsoref{密码}{mi4ma3}
  \end{Phonetics}
\end{Entry}

\begin{Entry}{明亮}{8,9}{⽇、⼇}
  \begin{Phonetics}{明亮}{ming2 liang4}[][HSK 5]
    \definition{adj.}{claro; bem iluminado | brilhante; resplandecente | claro; simples; compreensível}
  \end{Phonetics}
\end{Entry}

\begin{Entry}{明星}{8,9}{⽇、⽇}
  \begin{Phonetics}{明星}{ming2xing1}[][HSK 2]
    \definition[个,位,颗,名]{s.}{estrela; ator, atleta, cantor famosos, etc. | talento de ponta; profissional de destaque; também é usado como metáfora para pessoas ou grupos que se destacam pelo seu bom desempenho ou excelência | estrela brilhante; estrela resplandecente; referindo-se a estrelas muito brilhantes}
  \end{Phonetics}
\end{Entry}

\begin{Entry}{明显}{8,9}{⽇、⽇}
  \begin{Phonetics}{明显}{ming2xian3}[][HSK 3]
    \definition{adj.}{claro; óbvio; distinto; claramente visível}
  \end{Phonetics}
\end{Entry}

\begin{Entry}{明珠}{8,10}{⽇、⽟}
  \begin{Phonetics}{明珠}{ming2zhu1}
    \definition{s.}{pérola | jóia (de grande valor)}
  \end{Phonetics}
\end{Entry}

\begin{Entry}{明确}{8,12}{⽇、⽯}
  \begin{Phonetics}{明确}{ming2que4}[][HSK 3]
    \definition{adj.}{claro; definido; específico}
    \definition{v.}{deixar claro; tornar definitivo; tornar um ponto de vista, uma tarefa, etc. claro, compreensível e definitivo}
  \end{Phonetics}
\end{Entry}

\begin{Entry}{昏}{8}{⽇}
  \begin{Phonetics}{昏}{hun1}
    \definition*{s.}{Sobrenome Hun}
    \definition{adj.}{escuro; fraco; embaçado | confuso; embaraçado; inconsciente}
    \definition{s.}{crepúsculo; tarde}
    \definition{v.}{perder a consciência; desmaiar}
  \end{Phonetics}
\end{Entry}

\begin{Entry}{易}{8}{⽇}
  \begin{Phonetics}{易}{yi4}
    \definition*{s.}{Sobrenome Yi}
    \definition{adj.}{fácil | amigável; pacífico}
    \definition{v.}{modificar; transformar | trocar | subestimar; desprezar}
  \end{Phonetics}
\end{Entry}

\begin{Entry}{昔}{8}{⽇}
  \begin{Phonetics}{昔}{xi1}
    \definition{s.}{tempos antigos; o passado; era uma vez}
  \end{Phonetics}
\end{Entry}

\begin{Entry}{昔日}{8,4}{⽇、⽇}
  \begin{Phonetics}{昔日}{xi1ri4}
    \definition{adj.}{passado}
  \end{Phonetics}
\end{Entry}

\begin{Entry}{朋}{8}{⽉}
  \begin{Phonetics}{朋}{peng2}
    \definition*{s.}{Sobrenome Peng}
    \definition{s.}{amigo}
    \definition{v.}{(literário) rivalizar; igualar; comparar | (literário) reunir-se em grupo; juntar-se em grupo}
  \end{Phonetics}
\end{Entry}

\begin{Entry}{朋友}{8,4}{⽉、⼜}
  \begin{Phonetics}{朋友}{peng2you5}[][HSK 1]
    \definition[个,位,帮,群]{s.}{amigo; pessoas que têm um bom relacionamento, uma boa relação, se entendem e se ajudam mutuamente | namorado; namorada}
  \end{Phonetics}
\end{Entry}

\begin{Entry}{服}{8}{⽉}
  \begin{Phonetics}{服}{fu2}[][HSK 6]
    \definition*{s.}{Sobrenome Fu}
    \definition{s.}{roupas | vestuário de luto; refere-se a roupas de luto}
    \definition{v.}{vestir (roupas) | tomar (remédio) | envolver-se em; servir | obedecer; ser convencido | convencer; persuadir | adaptar-se; acostumar-se a}
  \end{Phonetics}
  \begin{Phonetics}{服}{fu4}
    \definition{clas.}{usado para remédio: dose; usado na medicina tradicional chinesa}
  \end{Phonetics}
\end{Entry}

\begin{Entry}{服从}{8,4}{⽉、⼈}
  \begin{Phonetics}{服从}{fu2cong2}[][HSK 5]
    \definition{v.}{obedecer; submeter-se a; estar subordinado a}
  \end{Phonetics}
\end{Entry}

\begin{Entry}{服务}{8,5}{⽉、⼒}
  \begin{Phonetics}{服务}{fu2 wu4}[][HSK 2]
    \definition{v.}{prestar serviço a; estar a serviço de; servir; trabalhar para o benefício coletivo (ou de outras pessoas) ou para uma causa específica | trabalhar; servir}
  \end{Phonetics}
\end{Entry}

\begin{Entry}{服务员}{8,5,7}{⽉、⼒、⼝}
  \begin{Phonetics}{服务员}{fu2wu4yuan2}
    \definition{s.}{atendente | garçom | garçonete | pessoal de atendimento ao cliente}
  \end{Phonetics}
\end{Entry}

\begin{Entry}{服务器}{8,5,16}{⽉、⼒、⼝}
  \begin{Phonetics}{服务器}{fu2wu4qi4}[][HSK 7-9]
    \definition[个,台]{s.}{Computção: servidor; um dispositivo dedicado que fornece serviços aos usuários em uma rede eletrônica de computadores}
  \end{Phonetics}
\end{Entry}

\begin{Entry}{服用}{8,5}{⽉、⽤}
  \begin{Phonetics}{服用}{fu2yong4}[][HSK 7-9]
    \definition{v.}{tomar (remédio)}[他已开始服用这种药。===Ele começou a tomar o remédio.]
  \end{Phonetics}
\end{Entry}

\begin{Entry}{服饰}{8,8}{⽉、⾷}
  \begin{Phonetics}{服饰}{fu2shi4}[][HSK 7-9]
    \definition[套]{s.}{roupa; vestido; traje; vestimenta e adorno pessoal}
  \end{Phonetics}
\end{Entry}

\begin{Entry}{服装}{8,12}{⽉、⾐}
  \begin{Phonetics}{服装}{fu2zhuang1}[][HSK 3]
    \definition[套,件,身]{s.}{roupas; vestuário; trajes; termo genérico para roupas, sapatos e chapéus, geralmente referido especificamente a roupas}
  \end{Phonetics}
\end{Entry}

\begin{Entry}{杯}{8}{⽊}
  \begin{Phonetics}{杯}{bei1}[][HSK 1]
    \definition{clas.}{para certos recipientes de líquidos: copo, xícara, etc.}
    \definition[只,个]{s.}{copo; caneca; xícara | taça; troféu; prêmio em forma de taça}
  \end{Phonetics}
\end{Entry}

\begin{Entry}{杯子}{8,3}{⽊、⼦}
  \begin{Phonetics}{杯子}{bei1 zi5}[][HSK 1]
    \definition[个,只,种]{s.}{xícara; copo; recipiente para bebidas ou outros líquidos, geralmente cilíndrico ou com a parte inferior ligeiramente mais estreita, com capacidade geralmente pequena}
  \end{Phonetics}
\end{Entry}

\begin{Entry}{杯具}{8,8}{⽊、⼋}
  \begin{Phonetics}{杯具}{bei1ju4}
    \definition{s.}{parachoque | fiasco | (gíria) tragédia}
  \end{Phonetics}
\end{Entry}

\begin{Entry}{杰}{8}{⽊}
  \begin{Phonetics}{杰}{jie2}
    \definition{adj.}{notável; proeminente; fora do comum}
    \definition[位,名,个,些]{s.}{pessoa excepcional; herói; uma pessoa com talentos excepcionais}
  \end{Phonetics}
\end{Entry}

\begin{Entry}{杰出}{8,5}{⽊、⼐}
  \begin{Phonetics}{杰出}{jie2chu1}[][HSK 6]
    \definition{adj.}{notável; proeminente; (talento, realização) excepcional}
  \end{Phonetics}
\end{Entry}

\begin{Entry}{松}{8}{⽊}
  \begin{Phonetics}{松}{song1}[][HSK 4]
    \definition*{s.}{Sobrenome Song}
    \definition{adj.}{solto; frouxo; folgado | leve e crocante; macio | relaxado; confortável}
    \definition[棵]{s.}{pinheiro | fio de carne seca; carne moída seca; alimentos macios ou quebradiços}
    \definition{v.}{afrouxar; relaxar; abrandar | desamarrar; desatar; liberar}
  \end{Phonetics}
\end{Entry}

\begin{Entry}{松木}{8,4}{⽊、⽊}
  \begin{Phonetics}{松木}{song1mu4}
    \definition{s.}{pinheiro}
  \end{Phonetics}
\end{Entry}

\begin{Entry}{松树}{8,9}{⽊、⽊}
  \begin{Phonetics}{松树}{song1 shu4}[][HSK 4]
    \definition[棵]{s.}{pinheiro; conífera comum, geralmente com folhas longas e pontiagudas e cones lenhosos}
  \end{Phonetics}
\end{Entry}

\begin{Entry}{板}{8}{⽊}
  \begin{Phonetics}{板}{ban3}[][HSK 3]
    \definition{adj.}{rígido; não natural; inflexível}
    \definition[块,个]{s.}{tábua; placa; prato; objeto rígido em forma de placa | veneziana; persiana; refere-se especificamente aos painéis de portas de lojas | badalos (instrumento musical que marca o ritmo) | uma batida acentuada (ritmo) na música e na ópera tradicional | chefe}
    \definition{v.}{parecer sério | corrigir maus hábitos ou defeitos | ser rígido como uma tábua}
  \end{Phonetics}
\end{Entry}

\begin{Entry}{板块}{8,7}{⽊、⼟}
  \begin{Phonetics}{板块}{ban3kuai4}[][HSK 7-9]
    \definition[个]{s.}{placa tectônica; segmentos móveis da crosta terrestre | seção; uma metáfora para uma combinação de partes que têm algo em comum ou conectado}
  \end{Phonetics}
\end{Entry}

\begin{Entry}{构}{8}{⽊}
  \begin{Phonetics}{构}{gou4}
    \definition{s.}{composição literária}
    \definition{v.}{construir; formar; compor | fabricar; inventar | construir; erguer uma casa}
    \variantof{够}
  \end{Phonetics}
\end{Entry}

\begin{Entry}{构成}{8,6}{⽊、⼽}
  \begin{Phonetics}{构成}{gou4cheng2}[][HSK 4]
    \definition{s.}{parte; componente; composição; estrutura}
    \definition{v.}{formar; compor; constituir; compor; encaixar muitas partes para formar um todo | consistir; causar; formar (principalmente em termos jurídicos)}
  \end{Phonetics}
\end{Entry}

\begin{Entry}{构建}{8,8}{⽊、⼵}
  \begin{Phonetics}{构建}{gou4 jian4}[][HSK 6]
    \definition{v.}{estabelecer (usado principalmente para coisas abstratas); montar; instalar}
  \end{Phonetics}
\end{Entry}

\begin{Entry}{构思}{8,9}{⽊、⼼}
  \begin{Phonetics}{构思}{gou4si1}[][HSK 7-9]
    \definition{s.}{concepção (ideia); o resultado da concepção}
    \definition{v.}{elaborar o enredo de uma obra literária ou a composição de uma pintura; pensar bem antes de escrever artigos ou criar obras literárias}
  \end{Phonetics}
\end{Entry}

\begin{Entry}{构造}{8,10}{⽊、⾡}
  \begin{Phonetics}{构造}{gou4 zao4}[][HSK 4]
    \definition[种]{s.}{estrutura; construção; disposição, organização e inter-relação dos componentes}
    \definition{v.}{formar; construir}
  \end{Phonetics}
\end{Entry}

\begin{Entry}{构想}{8,13}{⽊、⼼}
  \begin{Phonetics}{构想}{gou4xiang3}[][HSK 7-9]
    \definition{s.}{ideia; concepção; ideias formadas}
    \definition[种,个]{v.}{pensar (em um plano, projeto, etc.); conceber; usar a mente ao escrever ou criar arte}
  \end{Phonetics}
\end{Entry}

\begin{Entry}{枕}{8}{⽊}
  \begin{Phonetics}{枕}{zhen3}
    \definition*{s.}{Sobrenome Zhen}
    \definition[个]{s.}{travesseiro; almofada | Mecânica: bloco}
    \definition{v.}{descansar a cabeça no travesseiro, almofada}
  \end{Phonetics}
\end{Entry}

\begin{Entry}{果}{8}{⽊}
  \begin{Phonetics}{果}{guo3}
    \definition*{s.}{Sobrenome Guo}
    \definition{adj.}{resoluto; determinado; sem exitação}
    \definition{adv.}{realmente; como esperado; com certeza; isso significa que as coisas são consistentes com as expectativas, equivalente a 果然}
    \definition{conj.}{se realmente; se de fato}
    \definition[个,些,种]{s.}{fruta; fruto da planta | resultado; consequência; o resultado final de um assunto (em oposição à 因)}
  \seealsoref{果然}{guo3ran2}
  \seealsoref{因}{yin1}
  \end{Phonetics}
\end{Entry}

\begin{Entry}{果子}{8,3}{⽊、⼦}
  \begin{Phonetics}{果子}{guo3zi5}
    \definition{s.}{fruta}
  \end{Phonetics}
\end{Entry}

\begin{Entry}{果汁}{8,5}{⽊、⽔}
  \begin{Phonetics}{果汁}{guo3zhi1}[][HSK 3]
    \definition[杯,瓶,种]{s.}{suco; suco de frutas frescas; também se refere a bebidas feitas com suco de frutas frescas}
  \end{Phonetics}
\end{Entry}

\begin{Entry}{果园}{8,7}{⽊、⼞}
  \begin{Phonetics}{果园}{guo3yuan2}[][HSK 7-9]
    \definition[个,座]{s.}{pomar; um jardim onde são plantadas árvores frutíferas}
  \end{Phonetics}
\end{Entry}

\begin{Entry}{果实}{8,8}{⽊、⼧}
  \begin{Phonetics}{果实}{guo3shi2}[][HSK 4]
    \definition[种]{s.}{fruta; o órgão que se desenvolve a partir do ovário ou com outras partes da flor após a fertilização da flor | ganhos; frutos;  uma metáfora para conquista ou recompensa por trabalho árduo}
  \end{Phonetics}
\end{Entry}

\begin{Entry}{果树}{8,9}{⽊、⽊}
  \begin{Phonetics}{果树}{guo3 shu4}[][HSK 6]
    \definition[棵,个,片]{s.}{árvore frutífera; árvores cujos frutos são principalmente comestíveis, como pessegueiros e macieiras}
  \end{Phonetics}
\end{Entry}

\begin{Entry}{果真}{8,10}{⽊、⼗}
  \begin{Phonetics}{果真}{guo3zhen1}[][HSK 7-9]
    \definition{adv.}{realmente; como esperado; com certeza}
    \definition{conj.}{se de fato; se realmente; se for o caso}[果真如此, 我就放心了。===Se for esse o caso, então ficarei aliviado.]
  \end{Phonetics}
\end{Entry}

\begin{Entry}{果断}{8,11}{⽊、⽄}
  \begin{Phonetics}{果断}{guo3duan4}[][HSK 7-9]
    \definition{adj.}{resoluto; decisivo; agir decisivamente sem hesitação}
  \end{Phonetics}
\end{Entry}

\begin{Entry}{果然}{8,12}{⽊、⽕}
  \begin{Phonetics}{果然}{guo3ran2}[][HSK 3]
    \definition{adv.}{realmente; como esperado; com certeza; indica que os fatos correspondem ao que foi dito ou esperado}
    \definition{conj.}{se realmente; se de fato; suponha que os fatos correspondam ao que foi dito ou esperado}
  \end{Phonetics}
\end{Entry}

\begin{Entry}{果酱}{8,13}{⽊、⾣}
  \begin{Phonetics}{果酱}{guo3 jiang4}[][HSK 6]
    \definition{s.}{geléia | compota ou doce (de frutas); fruta em conserva}
  \end{Phonetics}
\end{Entry}

\begin{Entry}{枝}{8}{⽊}
  \begin{Phonetics}{枝}{zhi1}[][HSK 6]
    \definition*{s.}{Sobrenome Zhi}
    \definition{clas.}{usado para flores com galhos, ramos | usado para objetos em forma de haste}
    \definition{s.}{ramo; galho}
  \end{Phonetics}
\end{Entry}

\begin{Entry}{枪}{8}{⽊}
  \begin{Phonetics}{枪}{qiang1}[][HSK 5]
    \definition*{s.}{Sobrenome Qiang}
    \definition[把,杆,支,挺]{s.}{lança | arma; rifle; arma de fogo | uma coisa em forma de arma | enxada; ferramenta para cavar a terra}
    \definition{v.}{escrever artigos ou responder perguntas para outras pessoas}
  \end{Phonetics}
\end{Entry}

\begin{Entry}{枫}{8}{⽊}
  \begin{Phonetics}{枫}{feng1}
    \definition[棵]{s.}{goma doce chinesa | bordo; \emph{maple}}
  \end{Phonetics}
\end{Entry}

\begin{Entry}{枫叶}{8,5}{⽊、⼝}
  \begin{Phonetics}{枫叶}{feng1ye4}
    \definition{s.}{folha de bordo (maple, tipo de árvore)}
  \end{Phonetics}
\end{Entry}

\begin{Entry}{柜}{8}{⽊}
  \begin{Phonetics}{柜}{gui4}
    \definition{s.}{baú; armário; gabinete | loja; balcão}
  \end{Phonetics}
  \begin{Phonetics}{柜}{ju3}
    \definition{s.}{faia; salgueiro}
  \end{Phonetics}
\end{Entry}

\begin{Entry}{柜子}{8,3}{⽊、⼦}
  \begin{Phonetics}{柜子}{gui4 zi5}[][HSK 5]
    \definition[个]{s.}{gabinete; armário; dispositivo para guardar roupas, documentos, livros, etc.}
  \end{Phonetics}
\end{Entry}

\begin{Entry}{柜台}{8,5}{⽊、⼝}
  \begin{Phonetics}{柜台}{gui4tai2}[][HSK 7-9]
    \definition[个,排,组]{s.}{bar; balcão; uma longa área semelhante a uma mesa em uma loja ou banco usada para vender mercadorias ou conduzir negócios}
  \end{Phonetics}
\end{Entry}

\begin{Entry}{欣}{8}{⽋}
  \begin{Phonetics}{欣}{xin1}
    \definition*{s.}{Sobrenome Xin}
    \definition{adj.}{alegre; feliz; contente}
  \end{Phonetics}
\end{Entry}

\begin{Entry}{欣赏}{8,12}{⽋、⾙}
  \begin{Phonetics}{欣赏}{xin1shang3}[][HSK 5]
    \definition{v.}{apreciar; admirar; valorizar; apreciar as coisas boas e descubrir o prazer que elas proporcionam | apreciar; gostar; considerar bom}
  \end{Phonetics}
\end{Entry}

\begin{Entry}{欧}{8}{⽋}
  \begin{Phonetics}{欧}{ou1}
    \definition*{s.}{Europa, abreviação de 欧洲 | Sobrenome Ou}
  \seealsoref{欧洲}{ou1zhou1}
  \end{Phonetics}
\end{Entry}

\begin{Entry}{欧阳询}{8,6,8}{⽋、⾩、⾔}
  \begin{Phonetics}{欧阳询}{ou1yang2 xun2}
    \definition*{s.}{Ouyang Xun (557-641), um dos quatro grandes calígrafos do início da dinastia Tang, 唐初四大家}
  \seealsoref{唐初四大家}{tang2 chu1 si4 da4jia1}
  \end{Phonetics}
\end{Entry}

\begin{Entry}{欧洲}{8,9}{⽋、⽔}
  \begin{Phonetics}{欧洲}{ou1zhou1}
    \definition*{s.}{Europa}
  \end{Phonetics}
\end{Entry}

\begin{Entry}{欧洲人}{8,9,2}{⽋、⽔、⼈}
  \begin{Phonetics}{欧洲人}{ou1zhou1ren2}
    \definition{s.}{europeu | pessoa ou povo da Europa}
  \end{Phonetics}
\end{Entry}

\begin{Entry}{欧洲共同体}{8,9,6,6,7}{⽋、⽔、⼋、⼝、⼈}
  \begin{Phonetics}{欧洲共同体}{ou1zhou1 gong4tong2ti3}
    \definition*{s.}{Comunidade Europeia}
  \end{Phonetics}
\end{Entry}

\begin{Entry}{欧盟}{8,13}{⽋、⽫}
  \begin{Phonetics}{欧盟}{ou1meng2}
    \definition*{s.}{União Europeia (EU)}
  \end{Phonetics}
\end{Entry}

\begin{Entry}{武}{8}{⽌}
  \begin{Phonetics}{武}{wu3}
    \definition*{s.}{Sobrenome Wu}
    \definition{adj.}{valente; corajoso}
    \definition{s.}{militar; atividades e comportamentos relacionados a habilidades militares e de combate (em oposição a 文) | arte marcial | passo; meio passo; pegadas}
  \seealsoref{文}{wen2}
  \end{Phonetics}
\end{Entry}

\begin{Entry}{武力}{8,2}{⽌、⼒}
  \begin{Phonetics}{武力}{wu3li4}
    \definition{s.}{forças armadas | militares}
  \end{Phonetics}
\end{Entry}

\begin{Entry}{武士}{8,3}{⽌、⼠}
  \begin{Phonetics}{武士}{wu3shi4}
    \definition{s.}{samurai | guerreiro}
  \end{Phonetics}
\end{Entry}

\begin{Entry}{武大戏}{8,3,6}{⽌、⼤、⼽}
  \begin{Phonetics}{武大戏}{wu3 da4xi4}
    \definition*{s.}{Drama de Luta Acrobática | Drama Wu}
  \end{Phonetics}
\end{Entry}

\begin{Entry}{武艺}{8,4}{⽌、⾋}
  \begin{Phonetics}{武艺}{wu3yi4}
    \definition{s.}{arte marcial | habilidade militar}
  \end{Phonetics}
\end{Entry}

\begin{Entry}{武术}{8,5}{⽌、⽊}
  \begin{Phonetics}{武术}{wu3shu4}[][HSK 3]
    \definition[种,套,门]{s.}{arte marcial; autodefesa; \emph{wushu}; um esporte tradicional chinês que utiliza técnicas com os punhos, pernas, pés ou armas como facas e espadas}
  \end{Phonetics}
\end{Entry}

\begin{Entry}{武官}{8,8}{⽌、⼧}
  \begin{Phonetics}{武官}{wu3guan1}
    \definition{s.}{oficial militar | adido militar}
  \end{Phonetics}
\end{Entry}

\begin{Entry}{武断}{8,11}{⽌、⽄}
  \begin{Phonetics}{武断}{wu3duan4}
    \definition{adj.}{arbitrário | dogmático | subjetivo}
  \end{Phonetics}
\end{Entry}

\begin{Entry}{武装}{8,12}{⽌、⾐}
  \begin{Phonetics}{武装}{wu3zhuang1}
    \definition{s.}{forças armadas | militar | arma}
    \definition{v.}{armar}
  \end{Phonetics}
\end{Entry}

\begin{Entry}{武器}{8,16}{⽌、⼝}
  \begin{Phonetics}{武器}{wu3qi4}[][HSK 3]
    \definition[批,个,件,种]{s.}{arma; equipamentos e dispositivos utilizados diretamente para matar inimigos ou destruir suas instalações defensivas e ofensivas | armas; armamento; metáfora usada como ferramenta de luta}
  \end{Phonetics}
\end{Entry}

\begin{Entry}{氛}{8}{⽓}
  \begin{Phonetics}{氛}{fen1}
    \definition{s.}{atmosfera; gás}
  \end{Phonetics}
\end{Entry}

\begin{Entry}{氛围}{8,7}{⽓、⼞}
  \begin{Phonetics}{氛围}{fen1wei2}[][HSK 7-9]
    \definition[种,片,股]{s.}{atmosfera; a atmosfera e o humor ao redor}
  \end{Phonetics}
\end{Entry}

\begin{Entry}{河}{8}{⽔}
  \begin{Phonetics}{河}{he2}[][HSK 2]
    \definition*{s.}{Astronomia: o sistema da Via Láctea | O Rio Amarelo; O Rio Huanghe | Sobrenome He}
    \definition[条,道]{s.}{rio; refere-se a grandes cursos de água}
  \end{Phonetics}
\end{Entry}

\begin{Entry}{河南}{8,9}{⽔、⼗}
  \begin{Phonetics}{河南}{he2nan2}
    \definition*{s.}{Província de Henan}
  \end{Phonetics}
\end{Entry}

\begin{Entry}{河流}{8,10}{⽔、⽔}
  \begin{Phonetics}{河流}{he2liu2}[][HSK 7-9]
    \definition[条]{s.}{rio; córrego; um termo geral para grandes fluxos naturais de água (como rios, etc.) na superfície da Terra}
  \end{Phonetics}
\end{Entry}

\begin{Entry}{河畔}{8,10}{⽔、⽥}
  \begin{Phonetics}{河畔}{he2pan4}[][HSK 7-9]
    \definition{s.}{planície fluvial | beira-rio}
  \end{Phonetics}
\end{Entry}

\begin{Entry}{河蚌}{8,10}{⽔、⾍}
  \begin{Phonetics}{河蚌}{he2bang4}
    \definition{s.}{mexilhões | bivalves cultivados em rios e lagos}
  \end{Phonetics}
\end{Entry}

\begin{Entry}{沸}{8}{⽔}
  \begin{Phonetics}{沸}{fei4}
    \definition{adj.}{fervente; borbulhante; em ebulição}
    \definition{v.}{ferver | borbulhar}
  \end{Phonetics}
\end{Entry}

\begin{Entry}{沸沸扬扬}{8,8,6,6}{⽔、⽔、⼿、⼿}
  \begin{Phonetics}{沸沸扬扬}{fei4fei4yang2yang2}[][HSK 7-9]
    \definition{expr.}{levantar uma confusão de críticas sobre; borbulhando de barulho; discutir animadamente; dar origem a muita discussão; em um rebuliço; ``Tão barulhento quanto água fervente.'', frequentemente usado para descrever muita discussão}
  \end{Phonetics}
\end{Entry}

\begin{Entry}{沸腾}{8,13}{⽔、⾁}
  \begin{Phonetics}{沸腾}{fei4teng2}[][HSK 7-9]
    \definition{v.}{ferver; vaporizar | fervilhar de excitação; uma metáfora para alto astral ou vozes barulhentas}
  \end{Phonetics}
\end{Entry}

\begin{Entry}{油}{8}{⽔}
  \begin{Phonetics}{油}{you2}[][HSK 2]
    \definition*{s.}{Sobrenome You}
    \definition{adj.}{oleoso; gorduroso}
    \definition[瓶,滴,层]{s.}{óleo; gordura; graxa; petróleo}
    \definition{v.}{aplicar óleo de tungue, verniz ou tinta | estar manchado ou sujo com óleo ou graxa | aplicar óleo de tungue ou tinta}
  \end{Phonetics}
\end{Entry}

\begin{Entry}{治}{8}{⽔}
  \begin{Phonetics}{治}{zhi4}[][HSK 4]
    \definition*{s.}{Sobrenome Zhi}
    \definition{adj.}{calmo e pacífico}
    \definition{s.}{sede de um antigo governo local}
    \definition{v.}{reger; administrar; governar; gerenciar; gerir | tratar (uma doença); curar; sarar | eliminar; controlar pragas | controlar (um rio); restaurar um curso d'água por meio de dragagem | punir; castigar | estudar; pesquisar; explorar}
  \end{Phonetics}
\end{Entry}

\begin{Entry}{治安}{8,6}{⽔、⼧}
  \begin{Phonetics}{治安}{zhi4'an1}[][HSK 5]
    \definition{s.}{ordem pública; segurança pública; ordem social estável}
  \end{Phonetics}
\end{Entry}

\begin{Entry}{治疗}{8,7}{⽔、⽧}
  \begin{Phonetics}{治疗}{zhi4liao2}[][HSK 4]
    \definition{s.}{diagnóstico; tratamento}
    \definition{v.}{tratar; curar; remediar; eliminar doenças por meio de medicamentos, cirurgia, etc.}
  \end{Phonetics}
\end{Entry}

\begin{Entry}{治病}{8,10}{⽔、⽧}
  \begin{Phonetics}{治病}{zhi4 bing4}[][HSK 6]
    \definition{v.}{tratar uma doença; eliminar doenças por meio de medicamentos, cirurgias, etc.}
  \end{Phonetics}
\end{Entry}

\begin{Entry}{治理}{8,11}{⽔、⽟}
  \begin{Phonetics}{治理}{zhi4li3}[][HSK 5]
    \definition{s.}{governo | governança}
    \definition{v.}{dirigir; gerenciar; governar; administrar | tratar; aproveitar; colocar sob controle; colocar em ordem}
  \end{Phonetics}
\end{Entry}

\begin{Entry}{治愈}{8,13}{⽔、⼼}
  \begin{Phonetics}{治愈}{zhi4yu4}
    \definition{v.}{curar | restaurar a saúde}
  \end{Phonetics}
\end{Entry}

\begin{Entry}{沽}{8}{⽔}
  \begin{Phonetics}{沽}{gu1}
    \definition*{s.}{Município de Tianjin; outro nome para Tianjin}
    \definition{v.}{comprar | vender}
  \end{Phonetics}
\end{Entry}

\begin{Entry}{沽名钓誉}{8,6,8,13}{⽔、⼝、⾦、⾔}
  \begin{Phonetics}{沽名钓誉}{gu1ming2-diao4yu4}[][HSK 7-9]
    \definition{expr.}{``Buscando fama e reputação.''; pescar fama e elogios; tentar alcançar a fama}
  \end{Phonetics}
\end{Entry}

\begin{Entry}{沿}{8}{⽔}
  \begin{Phonetics}{沿}{yan2}[][HSK 6]
    \definition{prep.}{ao longo}
    \definition{s.}{beira; borda; acabamento}
    \definition{v.}{seguir (uma tradição, padrão, etc.) | enfeitar (com fita, faixa, etc.)}
  \end{Phonetics}
\end{Entry}

\begin{Entry}{沿海}{8,10}{⽔、⽔}
  \begin{Phonetics}{沿海}{yan2hai3}[][HSK 6]
    \definition{s.}{costa; litoral; área ou região ao longo da costa}
  \end{Phonetics}
\end{Entry}

\begin{Entry}{沿着}{8,11}{⽔、⽬}
  \begin{Phonetics}{沿着}{yan2 zhe5}[][HSK 6]
    \definition{prep.}{ao longo (de uma determinada rota)}
  \end{Phonetics}
\end{Entry}

\begin{Entry}{泄}{8}{⽔}
  \begin{Phonetics}{泄}{xie4}
    \definition*{s.}{Sobrenome Xie}
    \definition{v.}{deixar sair (um fluido ou gás); descarregar; liberar | revelar (um segredo); vazar (notícias, segredos, etc.) | dar vazão a; desabafar}
  \end{Phonetics}
\end{Entry}

\begin{Entry}{泄气}{8,4}{⽔、⽓}
  \begin{Phonetics}{泄气}{xie4/qi4}
    \definition{adj.}{decepcionante | frustrante | patético}
    \definition{v.+compl.}{perder o coração | sentir-se desencorajado | ficar desanimado}
  \end{Phonetics}
\end{Entry}

\begin{Entry}{泄底}{8,8}{⽔、⼴}
  \begin{Phonetics}{泄底}{xie4di3}
    \definition{v.}{revelar ou expor o que está no fundo de algo | divulgar a história interna; vazar segredos}
  \end{Phonetics}
\end{Entry}

\begin{Entry}{泄洪}{8,9}{⽔、⽔}
  \begin{Phonetics}{泄洪}{xie4hong2}
    \definition{v.}{liberar água da enchente (descarga de inundação)}
  \end{Phonetics}
\end{Entry}

\begin{Entry}{泄愤}{8,12}{⽔、⼼}
  \begin{Phonetics}{泄愤}{xie4fen4}
    \definition{v.}{dar vazão à raiva}
  \end{Phonetics}
\end{Entry}

\begin{Entry}{泄露}{8,21}{⽔、⾬}
  \begin{Phonetics}{泄露}{xie4lou4}
    \definition{v.}{vazar; deixar escapar; divulgar; revelar (um segredo, etc.) | vazar; escapar; descarregar (um fluido ou gás)}
  \end{Phonetics}
\end{Entry}

\begin{Entry}{法}{8}{⽔}
  \begin{Phonetics}{法}{fa3}[][HSK 4]
    \definition*{s.}{Doutrina budista; o dharma | França, abreviação de 法国 | Sobrenome Fa}
    \definition{adj.}{(usado após advérbios negativos) legal; cumpridor da lei}
    \definition{clas.}{F; Farad, medida de capacitância}
    \definition{s.}{lei; termo geral para regras de comportamento estabelecidas ou endossadas pelo Estado | maneira; método; modo; meios | padrão; modelo | artes mágicas; feitiço}
    \definition{v.}{seguir; imitar; aprender (os pontos fortes dos outros) |}
  \seealsoref{法国}{fa3guo2}
  \end{Phonetics}
\end{Entry}

\begin{Entry}{法文}{8,4}{⽔、⽂}
  \begin{Phonetics}{法文}{fa3wen2}
    \definition[份]{s.}{françês, língua francesa}
  \end{Phonetics}
\end{Entry}

\begin{Entry}{法网}{8,6}{⽔、⽹}
  \begin{Phonetics}{法网}{fa3wang3}
    \definition*{s.}{Torneio de Roland Garros (French Open), torneio de tênis}
  \end{Phonetics}
\end{Entry}

\begin{Entry}{法制}{8,8}{⽔、⼑}
  \begin{Phonetics}{法制}{fa3 zhi4}[][HSK 5]
    \definition{s.}{legalidade; instituições jurídicas; sistema jurídico}
  \end{Phonetics}
\end{Entry}

\begin{Entry}{法国}{8,8}{⽔、⼞}
  \begin{Phonetics}{法国}{fa3guo2}
    \definition*{s.}{França}
  \end{Phonetics}
\end{Entry}

\begin{Entry}{法国人}{8,8,2}{⽔、⼞、⼈}
  \begin{Phonetics}{法国人}{fa3guo2ren2}
    \definition{s.}{francês | pessoa ou povo da França}
  \end{Phonetics}
\end{Entry}

\begin{Entry}{法官}{8,8}{⽔、⼧}
  \begin{Phonetics}{法官}{fa3 guan1}[][HSK 4]
    \definition[位,名,个,些]{s.}{juiz; justiça; termo genérico para um membro do judiciário em um tribunal de justiça}
  \end{Phonetics}
\end{Entry}

\begin{Entry}{法规}{8,8}{⽔、⾒}
  \begin{Phonetics}{法规}{fa3 gui1}[][HSK 5]
    \definition[部,项,条,套,个]{s.}{lei e regulamento; estatuto; termo geral para leis, decretos, regulamentos, regras, estatutos, etc.}
  \end{Phonetics}
\end{Entry}

\begin{Entry}{法庭}{8,9}{⽔、⼴}
  \begin{Phonetics}{法庭}{fa3 ting2}[][HSK 6]
    \definition{s.}{corte; tribunal | tribunal; um órgão estatal que exerce o poder judicial de forma independente}
  \end{Phonetics}
\end{Entry}

\begin{Entry}{法律}{8,9}{⽔、⼻}
  \begin{Phonetics}{法律}{fa3lv4}[][HSK 4]
    \definition[项,条,套,个]{s.}{lei; estatuto; regras de conduta formuladas pelo legislativo e cuja aplicação é garantida pelo poder estatal}
  \end{Phonetics}
\end{Entry}

\begin{Entry}{法语}{8,9}{⽔、⾔}
  \begin{Phonetics}{法语}{fa3 yu3}[][HSK 6]
    \definition[种,门,句,段]{s.}{françês, língua francesa}
  \end{Phonetics}
\end{Entry}

\begin{Entry}{法院}{8,9}{⽔、⾩}
  \begin{Phonetics}{法院}{fa3yuan4}[][HSK 4]
    \definition[所,座]{s.}{tribunal; corte; órgãos estatais que exercem poder judicial independente}
  \end{Phonetics}
\end{Entry}

\begin{Entry}{泡}{8}{⽔}
  \begin{Phonetics}{泡}{pao1}
    \definition{adj.}{esponjoso; oco e macio; não duro}
    \definition{clas.}{usado para fezes e urina}
    \definition[串,个]{s.}{algo fofo e macio | pequeno lago}
  \end{Phonetics}
  \begin{Phonetics}{泡}{pao4}[][HSK 6]
    \definition[串,个]{s.}{bolha | algo em forma de bolha}
    \definition{v.}{mergulhar; encharcar | despejar água fervente em (chá, sopa instantânea, etc.) | enrolar; demorar-se; ficar por aí | (coloquial) (de um homem) brincar no campo; brincar com uma mulher | perder tempo; matar o tempo deliberadamente}
  \end{Phonetics}
\end{Entry}

\begin{Entry}{波}{8}{⽔}
  \begin{Phonetics}{波}{bo1}
    \definition*{s.}{Polônia, abreviação de 波兰 | Sobrenome Bo}
    \definition{s.}{ondas, a superfície irregular da água em rios, lagos e oceanos | onda, o processo de propagação da vibração | mudanças inesperadas; uma reviravolta inesperada nos acontecimentos; metáfora para mudanças inesperadas nas coisas | olhos; metáfora do olhar errante}
  \seealsoref{波兰}{bo1lan2}
  \end{Phonetics}
\end{Entry}

\begin{Entry}{波兰}{8,5}{⽔、⼋}
  \begin{Phonetics}{波兰}{bo1lan2}
    \definition*{s.}{Polônia}
  \end{Phonetics}
\end{Entry}

\begin{Entry}{波动}{8,6}{⽔、⼒}
  \begin{Phonetics}{波动}{bo1 dong4}[][HSK 6]
    \definition{s.}{ondulação; flutuação; movimento de onda}
    \definition{v.}{ondular; flutuar}
  \end{Phonetics}
\end{Entry}

\begin{Entry}{波折}{8,7}{⽔、⼿}
  \begin{Phonetics}{波折}{bo1zhe2}[][HSK 7-9]
    \definition{s.}{reviravoltas; contratempo; as reviravoltas que ocorrem durante o curso das coisas, o que significa que você sofre dificuldades ou contratempos}
  \end{Phonetics}
\end{Entry}

\begin{Entry}{波音}{8,9}{⽔、⾳}
  \begin{Phonetics}{波音}{bo1yin1}
    \definition*{s.}{Boeing (empresa aeroespacial)}
    \definition{s.}{mordente (música)}
  \end{Phonetics}
\end{Entry}

\begin{Entry}{波浪}{8,10}{⽔、⽔}
  \begin{Phonetics}{波浪}{bo1lang4}[][HSK 6]
    \definition{s.}{onda; a superfície irregular da água nos rios, lagos e oceanos, geralmente se refere a águas menores e mais bonitas, frequentemente usada na linguagem falada}
  \end{Phonetics}
\end{Entry}

\begin{Entry}{波涛}{8,10}{⽔、⽔}
  \begin{Phonetics}{波涛}{bo1tao1}[][HSK 7-9]
    \definition{s.}{grandes (enormes) ondas; ondas de maré; ondas grandes costumam se referir a paisagens espetaculares ou emocionantes; são usadas tanto na linguagem falada quanto na escrita.}
  \end{Phonetics}
\end{Entry}

\begin{Entry}{波澜}{8,15}{⽔、⽔}
  \begin{Phonetics}{波澜}{bo1lan2}[][HSK 7-9]
    \definition[个,场,阵]{s.}{ondas grandes}
  \end{Phonetics}
\end{Entry}

\begin{Entry}{泥}{8}{⽔}
  \begin{Phonetics}{泥}{ni2}[][HSK 6]
    \definition*{s.}{Sobrenome Ni}
    \definition{s.}{lama; atoleiro | pasta ou polpa; amassado | qualquer matéria pastosa; purê de vegetais ou frutas}
  \end{Phonetics}
  \begin{Phonetics}{泥}{ni4}
    \definition{adj.}{fanático; teimoso; obstinado; cabeçudo}
    \definition{v.}{cobrir ou rebocar com gesso, massa de vidraceiro, etc.}
  \end{Phonetics}
\end{Entry}

\begin{Entry}{泥潭}{8,15}{⽔、⽔}
  \begin{Phonetics}{泥潭}{ni2tan2}
    \definition{s.}{atoleiro | lamaçal | charco | pântano}
  \end{Phonetics}
\end{Entry}

\begin{Entry}{注}{8}{⽔}
  \begin{Phonetics}{注}{zhu4}
    \definition{s.}{apostas (em jogos de azar) | notas (em um texto)}
    \definition{v.}{derramar; encher | concentrar-se em; fixar-se em; focar em  | anotar; explicar com notas | registrar; gravar | irrigar | dar exegese ou explicação}
  \end{Phonetics}
\end{Entry}

\begin{Entry}{注册}{8,5}{⽔、⼌}
  \begin{Phonetics}{注册}{zhu4ce4}[][HSK 5]
    \definition{v.}{inscrever-se; matricular-se; registrar-se; registrar-se junto à autoridade ou escola competente para obter status legal; refere-se especificamente ao usuário de uma determinada rede de computadores que insere o nome de usuário, senha, etc. na rede para obter permissão para usar a rede}
  \end{Phonetics}
\end{Entry}

\begin{Entry}{注册人}{8,5,2}{⽔、⼌、⼈}
  \begin{Phonetics}{注册人}{zhu4ce4ren2}
    \definition{s.}{registrante}
  \end{Phonetics}
\end{Entry}

\begin{Entry}{注册表}{8,5,8}{⽔、⼌、⾐}
  \begin{Phonetics}{注册表}{zhu4ce4biao3}
    \definition[份,个,张]{s.}{registro do Windows}
  \end{Phonetics}
\end{Entry}

\begin{Entry}{注册商标}{8,5,11,9}{⽔、⼌、⼝、⽊}
  \begin{Phonetics}{注册商标}{zhu4ce4 shang1biao1}
    \definition{s.}{marca registrada}
  \end{Phonetics}
\end{Entry}

\begin{Entry}{注视}{8,8}{⽔、⾒}
  \begin{Phonetics}{注视}{zhu4shi4}[][HSK 5]
    \definition{v.}{olhar atentamente para; observar atentamente}
  \end{Phonetics}
\end{Entry}

\begin{Entry}{注重}{8,9}{⽔、⾥}
  \begin{Phonetics}{注重}{zhu4zhong4}[][HSK 5]
    \definition{v.}{enfatizar; dar ênfase a; dar ênfase a; prestar atenção a; dar importância a}
  \end{Phonetics}
\end{Entry}

\begin{Entry}{注射}{8,10}{⽔、⼨}
  \begin{Phonetics}{注射}{zhu4she4}[][HSK 5]
    \definition{v.}{injetar; usar uma seringa para administrar medicamento líquido em um organismo}
  \end{Phonetics}
\end{Entry}

\begin{Entry}{注意}{8,13}{⽔、⼼}
  \begin{Phonetics}{注意}{zhu4yi4}[][HSK 3]
    \definition{v.}{prestar atenção; notar; ficar de olho; concentrar os pensamentos em um aspecto específico}
  \end{Phonetics}
\end{Entry}

\begin{Entry}{注意力}{8,13,2}{⽔、⼼、⼒}
  \begin{Phonetics}{注意力}{zhu4yi4li4}
    \definition{s.}{atenção}
  \end{Phonetics}
\end{Entry}

\begin{Entry}{注意力缺失症}{8,13,2,10,5,10}{⽔、⼼、⼒、⽸、⼤、⽧}
  \begin{Phonetics}{注意力缺失症}{zhu4yi4li4que1shi1zheng4}
    \definition{s.}{transtorno de déficit de atenção}
  \end{Phonetics}
\end{Entry}

\begin{Entry}{注意地}{8,13,6}{⽔、⼼、⼟}
  \begin{Phonetics}{注意地}{zhu4yi4di4}
    \definition{s.}{área de cuidado, de observação}
  \end{Phonetics}
\end{Entry}

\begin{Entry}{泪}{8}{⽔}
  \begin{Phonetics}{泪}{lei4}[][HSK 4]
    \definition[滴,行]{s.}{lágrima | algo semelhante a uma lágrima}
  \end{Phonetics}
\end{Entry}

\begin{Entry}{泪水}{8,4}{⽔、⽔}
  \begin{Phonetics}{泪水}{lei4 shui3}[][HSK 4]
    \definition[滴,行]{s.}{lágrima}
  \end{Phonetics}
\end{Entry}

\begin{Entry}{泳}{8}{⽔}
  \begin{Phonetics}{泳}{yong3}
    \definition{v.}{nadar}
  \end{Phonetics}
\end{Entry}

\begin{Entry}{泳池}{8,6}{⽔、⽔}
  \begin{Phonetics}{泳池}{yong3chi2}
    \definition{s.}{piscina}
  \seealsoref{游泳池}{you2 yong3 chi2}
  \seealsoref{游泳馆}{you2yong3guan3}
  \end{Phonetics}
\end{Entry}

\begin{Entry}{泳衣}{8,6}{⽔、⾐}
  \begin{Phonetics}{泳衣}{yong3yi1}
    \definition{s.}{roupa de banho | maiô}
  \seealsoref{游泳衣}{you2yong3yi1}
  \end{Phonetics}
\end{Entry}

\begin{Entry}{泼}{8}{⽔}
  \begin{Phonetics}{泼}{po1}[][HSK 5]
    \definition{adj.}{rude e irracional; mal-humorado | Dialeto: ousado e vigoroso; ousado e resoluto}
    \definition{v.}{espalhar; salpicar; derramar; derramar ou espalhar o líquido com força para fora}
  \end{Phonetics}
\end{Entry}

\begin{Entry}{浅}{8}{⽔}
  \begin{Phonetics}{浅}{jian1}
    \definition{adj.}{murmurando, fluindo suavemente, gorgolejando suavemente}
    \definition{s.}{Onomatopéia: som de água em movimento}
  \end{Phonetics}
  \begin{Phonetics}{浅}{qian3}[][HSK 4]
    \definition{adj.}{raso; superficial;  (em oposição a 深) | fácil; simples; redação, conteúdo, etc. simples e fáceis de entender | superficial; não é profundo em aprendizado, percepção e sabedoria | não próximo; não íntimo; sentimentos não profundos | (cor) claro; pálido;  cor pouco intensa; leve |experiência breve; duração de tempo breve | baixo grau; peso leve; nível baixo}
  \seealsoref{深}{shen1}
  \end{Phonetics}
\end{Entry}

\begin{Entry}{炎}{8}{⽕}
  \begin{Phonetics}{炎}{yan2}
    \definition{adj.}{escaldante; ardente}
    \definition{s.}{inflamação | poder; influência}
  \end{Phonetics}
\end{Entry}

\begin{Entry}{炎热}{8,10}{⽕、⽕}
  \begin{Phonetics}{炎热}{yan2re4}
    \definition{adj.}{extremamente quente | escaldante (clima)}
  \end{Phonetics}
\end{Entry}

\begin{Entry}{炒}{8}{⽕}
  \begin{Phonetics}{炒}{chao3}[][HSK 6]
    \definition{v.}{saltear; refogar; aquecer os alimentos em uma panela e mexer repetidamente para cozinhá-los ou secá-los | especular (na bolsa de valores, etc.) | exagerar; dar publicidade exagerada; a fim de ampliar a influência, por meio de publicidade repetida e exagerada na mídia | demitir; despedir}
  \end{Phonetics}
\end{Entry}

\begin{Entry}{炒作}{8,7}{⽕、⼈}
  \begin{Phonetics}{炒作}{chao3 zuo4}[][HSK 6]
    \definition{v.}{promover (na mídia); exagerar artificialmente e promover ou desvalorizar de forma inadequada | especular; comprar e vender frequentemente no mercado de negociação para obter lucros}
  \end{Phonetics}
\end{Entry}

\begin{Entry}{炒股}{8,8}{⽕、⾁}
  \begin{Phonetics}{炒股}{chao3/gu3}[][HSK 6]
    \definition{v.+compl.}{especular em ações; comprar e vender ações; jogar no mercado}
  \end{Phonetics}
\end{Entry}

\begin{Entry}{炖}{8}{⽕}
  \begin{Phonetics}{炖}{dun4}[][HSK 7-9]
    \definition{v.}{cozinhar | aquecer algo colocando o recipiente em água quente}
  \end{Phonetics}
\end{Entry}

\begin{Entry}{爬}{8}{⽖}
  \begin{Phonetics}{爬}{pa2}[][HSK 2]
    \definition{v.}{rastejar; arrastar-se; engatinhar | escalar; trepar; subir com dificuldade | sentar-se; levantar-se; levantar-se da posição deitada ou sentada}
  \end{Phonetics}
\end{Entry}

\begin{Entry}{爬上}{8,3}{⽖、⼀}
  \begin{Phonetics}{爬上}{pa2shang4}
    \definition{v.}{escalar}
  \end{Phonetics}
\end{Entry}

\begin{Entry}{爬山}{8,3}{⽖、⼭}
  \begin{Phonetics}{爬山}{pa2/shan1}[][HSK 2]
    \definition{v.+compl.}{escalar uma montanha;}
  \end{Phonetics}
\end{Entry}

\begin{Entry}{爬升}{8,4}{⽖、⼗}
  \begin{Phonetics}{爬升}{pa2sheng1}
    \definition{v.}{ascender | ganhar promoção | subir (números de vendas, etc.) | aumentar}
  \end{Phonetics}
\end{Entry}

\begin{Entry}{爬行}{8,6}{⽖、⾏}
  \begin{Phonetics}{爬行}{pa2xing2}
    \definition{v.}{rastejar | arrastar | engatinhar}
  \end{Phonetics}
\end{Entry}

\begin{Entry}{爬杆}{8,7}{⽖、⽊}
  \begin{Phonetics}{爬杆}{pa2gan1}
    \definition{s.}{escalada em poste}
    \definition{v.}{escalar um poste}
  \end{Phonetics}
\end{Entry}

\begin{Entry}{爬竿}{8,9}{⽖、⽵}
  \begin{Phonetics}{爬竿}{pa2gan1}
    \definition{s.}{poste de escalada | escalada em poste (como ginástica ou ato de circo)}
  \end{Phonetics}
\end{Entry}

\begin{Entry}{爬梳}{8,11}{⽖、⽊}
  \begin{Phonetics}{爬梳}{pa2shu1}
    \definition{v.}{vasculhar (documentos históricos, etc.) | desvendar}
  \end{Phonetics}
\end{Entry}

\begin{Entry}{爬犁}{8,11}{⽖、⽜}
  \begin{Phonetics}{爬犁}{pa2li2}
    \definition{s.}{trenó}
  \seealsoref{扒犁}{pa2li2}
  \end{Phonetics}
\end{Entry}

\begin{Entry}{爬墙}{8,14}{⽖、⼟}
  \begin{Phonetics}{爬墙}{pa2qiang2}
    \definition{v.}{escalar uma parede}
  \end{Phonetics}
\end{Entry}

\begin{Entry}{爸}{8}{⽗}
  \begin{Phonetics}{爸}{ba4}[][HSK 1]
    \definition[个,位]{s.}{(informal) pai}
  \seealsoref{爸爸}{ba4ba5}
  \end{Phonetics}
\end{Entry}

\begin{Entry}{爸妈}{8,6}{⽗、⼥}
  \begin{Phonetics}{爸妈}{ba4ma1}
    \definition{s.}{pai e mãe}
  \end{Phonetics}
\end{Entry}

\begin{Entry}{爸爸}{8,8}{⽗、⽗}
  \begin{Phonetics}{爸爸}{ba4ba5}[][HSK 1]
    \definition[个,位,名,群]{s.}{(informal) pai; papai; papa}
  \seealsoref{爸}{ba4}
  \end{Phonetics}
\end{Entry}

\begin{Entry}{版}{8}{⽚}
  \begin{Phonetics}{版}{ban3}[][HSK 5]
    \definition{clas.}{usado como uma palavra de medida para materiais impressos (por exemplo, livros, jornais, edições)}
    \definition{s.}{chapa, placa ou bloco de impressão | edição (livros impressos) | página (de um jornal) | moldes ou fromas de construção}
  \end{Phonetics}
\end{Entry}

\begin{Entry}{牦}{8}{⽜}
  \begin{Phonetics}{牦}{mao2}
    \definition[头]{s.}{iaque; boi da Tartária}
  \end{Phonetics}
\end{Entry}

\begin{Entry}{牦牛}{8,4}{⽜、⽜}
  \begin{Phonetics}{牦牛}{mao2niu2}
    \definition{s.}{iaque}
  \end{Phonetics}
\end{Entry}

\begin{Entry}{物}{8}{⽜}
  \begin{Phonetics}{物}{wu4}
    \definition{s.}{coisa; matéria; objeto | mundo exterior distinto de si mesmo; outras pessoas; refere-se a outras pessoas além de si mesmo ou ao ambiente em relação a si mesmo | essência; conteúdo; substância | criatura; criação}
  \end{Phonetics}
\end{Entry}

\begin{Entry}{物业}{8,5}{⽜、⼀}
  \begin{Phonetics}{物业}{wu4ye4}[][HSK 5]
    \definition[处]{s.}{propriedade; gestão de propriedades; gestão patrimonial; administração de imóveis | empresa de administração de imóveis; empresa de gestão imobiliária; empresa de administração de bens imóveis}
  \end{Phonetics}
\end{Entry}

\begin{Entry}{物价}{8,6}{⽜、⼈}
  \begin{Phonetics}{物价}{wu4 jia4}[][HSK 5]
    \definition[个]{s.}{preços das commodities; preços das matérias-primas; preço das mercadorias}
  \end{Phonetics}
\end{Entry}

\begin{Entry}{物质}{8,8}{⽜、⾙}
  \begin{Phonetics}{物质}{wu4zhi4}[][HSK 5]
    \definition[种,类,个]{s.}{matéria; substância; algo que existe além do espírito, que pode ser visto, tocado, cheirado ou detectado por instrumentos científicos | material; meios de subsistência; coisas que permitem às pessoas viver ou viver melhor, como comida, roupas, casas, dinheiro, etc.}
  \end{Phonetics}
\end{Entry}

\begin{Entry}{物品}{8,9}{⽜、⼝}
  \begin{Phonetics}{物品}{wu4 pin3}[][HSK 6]
    \definition[件,个]{s.}{artigos; itens; bens}
  \end{Phonetics}
\end{Entry}

\begin{Entry}{物理}{8,11}{⽜、⽟}
  \begin{Phonetics}{物理}{wu4li3}
    \definition{s.}{física (disciplina)}
  \end{Phonetics}
\end{Entry}

\begin{Entry}{狒}{8}{⽝}
  \begin{Phonetics}{狒}{fei4}
    \definition{s.}{babuíno (uma espécie de macaco)}
  \end{Phonetics}
\end{Entry}

\begin{Entry}{狒狒}{8,8}{⽝、⽝}
  \begin{Phonetics}{狒狒}{fei4fei4}
    \definition{s.}{babuíno}
  \end{Phonetics}
\end{Entry}

\begin{Entry}{狗}{8}{⽝}
  \begin{Phonetics}{狗}{gou3}[][HSK 2]
    \definition[条,只,群]{s.}{cão; cachorro | palavrão usado para se referir a pessoas más ou seus capangas}
  \end{Phonetics}
\end{Entry}

\begin{Entry}{玩}{8}{⽟}
  \begin{Phonetics}{玩}{wan2}
    \definition*{s.}{Sobrenome Wan}
    \definition{s.}{objeto de apreciação; coisas para assistir}
    \definition{v.}{(~儿) divertir-se; entreter-se; fazer atividades que te deixem feliz | jogar; praticar algum tipo de atividade cultural, de entretenimento ou esportiva | recorrer a; usar métodos e meios impróprios para atingir o objetivo | provocar; subestimar; tratar com uma atitude frívola; desprezar | desfrutar; apreciar; observar | (~儿) envolver-se em; tomar parte em; perseguir ou expressar deliberadamente um certo sentimento | ponderar; pensar cuidadosamente; apreciar}
  \end{Phonetics}
\end{Entry}

\begin{Entry}{玩儿}{8,2}{⽟、⼉}
  \begin{Phonetics}{玩儿}{wan2r5}[][HSK 1]
    \definition{v.}{divertir-se; (entretenimento) relaxar ou experimentar alguma atividade}
  \end{Phonetics}
\end{Entry}

\begin{Entry}{玩艺}{8,4}{⽟、⾋}
  \begin{Phonetics}{玩艺}{wan2yi4}
    \variantof{玩意}
  \end{Phonetics}
\end{Entry}

\begin{Entry}{玩伴}{8,7}{⽟、⼈}
  \begin{Phonetics}{玩伴}{wan2ban4}
    \definition{s.}{parceiro de brincadeira}
  \end{Phonetics}
\end{Entry}

\begin{Entry}{玩具}{8,8}{⽟、⼋}
  \begin{Phonetics}{玩具}{wan2ju4}[][HSK 3]
    \definition[个,件,套]{s.}{brinquedo; coisas para brincar}
  \end{Phonetics}
\end{Entry}

\begin{Entry}{玩具厂}{8,8,2}{⽟、⼋、⼚}
  \begin{Phonetics}{玩具厂}{wan2ju4chang3}
    \definition{s.}{fábrica de brinquedos}
  \end{Phonetics}
\end{Entry}

\begin{Entry}{玩具车}{8,8,4}{⽟、⼋、⾞}
  \begin{Phonetics}{玩具车}{wan2ju4 che1}
    \definition{s.}{carrinho de brinquedo}
  \end{Phonetics}
\end{Entry}

\begin{Entry}{玩味}{8,8}{⽟、⼝}
  \begin{Phonetics}{玩味}{wan2wei4}
    \definition{v.}{ponderar sutilezas | ruminar (pensamentos)}
  \end{Phonetics}
\end{Entry}

\begin{Entry}{玩者}{8,8}{⽟、⽼}
  \begin{Phonetics}{玩者}{wan2zhe3}
    \definition{s.}{jogador}
  \end{Phonetics}
\end{Entry}

\begin{Entry}{玩耍}{8,9}{⽟、⽽}
  \begin{Phonetics}{玩耍}{wan2shua3}
    \definition{v.}{divertir-me | brincar (como as crianças fazem)}
  \end{Phonetics}
\end{Entry}

\begin{Entry}{玩家}{8,10}{⽟、⼧}
  \begin{Phonetics}{玩家}{wan2jia1}
    \definition{s.}{entusiasta (áudio, modelos de aviões, etc.) | jogador (de um jogo)}
  \end{Phonetics}
\end{Entry}

\begin{Entry}{玩偶}{8,11}{⽟、⼈}
  \begin{Phonetics}{玩偶}{wan2'ou3}
    \definition{s.}{estatueta de brinquedo | boneco de ação | bicho de pelúcia | boneca}
  \end{Phonetics}
\end{Entry}

\begin{Entry}{玩遍}{8,12}{⽟、⾡}
  \begin{Phonetics}{玩遍}{wan2bian4}
    \definition{v.}{passear (todo o país, toda a cidade, etc.) | visitar (um grande número de lugares)}
  \end{Phonetics}
\end{Entry}

\begin{Entry}{玩意}{8,13}{⽟、⼼}
  \begin{Phonetics}{玩意}{wan2yi4}
    \definition{s.}{ato | brinquedo | coisa | truque (em uma performance, show de palco, acrobacias, etc.)}
  \end{Phonetics}
\end{Entry}

\begin{Entry}{环}{8}{⽟}
  \begin{Phonetics}{环}{huan2}[][HSK 3]
    \definition*{s.}{Sobrenome Huan}
    \definition{clas.}{usado para anéis}
    \definition[个,串]{s.}{anel; arco | elo; \emph{link}; passo; etapa | anel; objeto em forma de círculo | arredores}
    \definition{v.}{cercar; rodear; circular; circundar}
  \end{Phonetics}
\end{Entry}

\begin{Entry}{环卫}{8,3}{⽟、⼙}
  \begin{Phonetics}{环卫}{huan2wei4}
    \definition{s.}{limpeza pública; saneamento ambiental; saneamento geral; abreviação de 环境卫生 | Arcaico: guardas imperiais; guardas}
  \seealsoref{环境卫生}{huan2jing4wei4sheng1}
  \end{Phonetics}
\end{Entry}

\begin{Entry}{环节}{8,5}{⽟、⾋}
  \begin{Phonetics}{环节}{huan2jie2}[][HSK 5]
    \definition[个]{s.}{\emph{link}; ligação; vínculo; uma das muitas coisas que estão inter-relacionadas | segmento; estrutura anelar de alguns animais inferiores}
  \end{Phonetics}
\end{Entry}

\begin{Entry}{环保}{8,9}{⽟、⼈}
  \begin{Phonetics}{环保}{huan2 bao3}[][HSK 3]
    \definition{adj.}{ecológico; benefício para o meio ambiente; não prejudica o meio ambiente}
    \definition{s.}{proteção ambiental}
  \end{Phonetics}
\end{Entry}

\begin{Entry}{环境}{8,14}{⽟、⼟}
  \begin{Phonetics}{环境}{huan2jing4}[][HSK 3]
    \definition[个]{s.}{ambiente; os arredores | arredores; circunstâncias; condições políticas, econômicas, culturais, etc., dentro de um determinado âmbito}
  \end{Phonetics}
\end{Entry}

\begin{Entry}{环境卫生}{8,14,3,5}{⽟、⼟、⼙、⽣}
  \begin{Phonetics}{环境卫生}{huan2jing4wei4sheng1}
    \definition{s.}{saneamento ambiental}
  \seealsoref{环卫}{huan2wei4}
  \end{Phonetics}
\end{Entry}

\begin{Entry}{现}{8}{⾒}
  \begin{Phonetics}{现}{xian4}
    \definition{adj.}{(dinheiro, etc.) em mãos}
    \definition{adv.}{recente; de improviso; naquela época; temporariamente}
    \definition{s.}{presente; atual; existente | dinheiro; dinheiro de pronto}
    \definition{v.}{mostrar; revelar; aparecer; tornar-se visível}
  \seealsoref{见}{xian4}
  \end{Phonetics}
\end{Entry}

\begin{Entry}{现代}{8,5}{⾒、⼈}
  \begin{Phonetics}{现代}{xian4dai4}[][HSK 3]
    \definition*{s.}{Hyundai, empresa sul-coreana}
    \definition{adj.}{moderno; contemporâneo; com características, estilo e conceitos modernos, refletindo a vanguarda, a moda e a inovação da atualidade}
    \definition{s.}{tempos modernos; era contemporânea; atualmente, na divisão cronológica da história da China, refere-se principalmente ao período desde o Movimento 4 de Maio até os dias atuais}
  \end{Phonetics}
\end{Entry}

\begin{Entry}{现在}{8,6}{⾒、⼟}
  \begin{Phonetics}{现在}{xian4zai4}[][HSK 1]
    \definition{adv.}{agora; no momento; atualmente; neste momento, quando se fala, às vezes inclui um período de tempo mais ou menos longo antes ou depois da fala (diferente de 过去 ou 将来)}
  \seealsoref{过去}{guo4 qu4}
  \seealsoref{将来}{jiang1lai2}
  \end{Phonetics}
\end{Entry}

\begin{Entry}{现场}{8,6}{⾒、⼟}
  \begin{Phonetics}{现场}{xian4chang3}[][HSK 3]
    \definition[个,处]{s.}{local onde ocorreu o acidente, incidente ou desastre| local; ponto; local onde se realizam diretamente atividades como produção, apresentações e competições}
  \end{Phonetics}
\end{Entry}

\begin{Entry}{现有}{8,6}{⾒、⽉}
  \begin{Phonetics}{现有}{xian4 you3}[][HSK 5]
    \definition{adj.}{agora disponível; existente}
    \definition{v.}{estar disponível agora; existir | Literário: ter em mãos; ter em posse}
  \end{Phonetics}
\end{Entry}

\begin{Entry}{现抓}{8,7}{⾒、⼿}
  \begin{Phonetics}{现抓}{xian4zhua1}
    \definition{v.}{improvisar}
  \end{Phonetics}
\end{Entry}

\begin{Entry}{现状}{8,7}{⾒、⽝}
  \begin{Phonetics}{现状}{xian4zhuang4}[][HSK 5]
    \definition{s.}{situação atual}
  \end{Phonetics}
\end{Entry}

\begin{Entry}{现实}{8,8}{⾒、⼧}
  \begin{Phonetics}{现实}{xian4shi2}[][HSK 3]
    \definition{adj.}{real; efetivo; verdadeiro; de acordo com circunstâncias objetivas}
    \definition[个]{s.}{realidade; factualidade; coisas que existem objetivamente}
  \end{Phonetics}
\end{Entry}

\begin{Entry}{现货}{8,8}{⾒、⾙}
  \begin{Phonetics}{现货}{xian4huo4}
    \definition{s.}{produtos à vista}
  \end{Phonetics}
\end{Entry}

\begin{Entry}{现货的}{8,8,8}{⾒、⾙、⽩}
  \begin{Phonetics}{现货的}{xian4huo4 de5}
    \definition{s.}{produtos em estoque}
  \end{Phonetics}
\end{Entry}

\begin{Entry}{现金}{8,8}{⾒、⾦}
  \begin{Phonetics}{现金}{xian4jin1}[][HSK 3]
    \definition[笔]{s.}{dinheiro; dinheiro vivo; moeda que pode ser usada diretamente | reserva de dinheiro em um banco; o dinheiro guardado no cofre do banco}
  \end{Phonetics}
\end{Entry}

\begin{Entry}{现做}{8,11}{⾒、⼈}
  \begin{Phonetics}{现做}{xian4zuo4}
    \definition{adj.}{fresco}
    \definition{v.}{fazer (comida) no local}
  \end{Phonetics}
\end{Entry}

\begin{Entry}{现象}{8,11}{⾒、⾗}
  \begin{Phonetics}{现象}{xian4xiang4}[][HSK 3]
    \definition[个,种]{s.}{aparência (das coisas); fenômeno; a forma externa e as relações manifestadas pelas coisas em seu desenvolvimento e mudança}
  \end{Phonetics}
\end{Entry}

\begin{Entry}{画}{8}{⽥}
  \begin{Phonetics}{画}{hua4}[][HSK 2]
    \definition*{s.}{Sobrenome Hua}
    \definition{clas.}{traços (de um caractere chinês)}
    \definition[张,幅]{s.}{desenho; pintura; imagem; figura desenhada | traço horizontal (em caracteres chineses)}
    \definition{v.}{desenhar; pintar | desenhar; marcar; assinar}
  \seealsoref{划}{hua4}
  \end{Phonetics}
\end{Entry}

\begin{Entry}{画儿}{8,2}{⽥、⼉}
  \begin{Phonetics}{画儿}{hua4r5}[][HSK 2]
    \definition[幅,张]{s.}{imagem; desenho; pintura; obra de arte pintada}
  \end{Phonetics}
\end{Entry}

\begin{Entry}{画地为牢}{8,6,4,7}{⽥、⼟、⼂、⼧}
  \begin{Phonetics}{画地为牢}{hua4di4wei2lao2}
    \definition{expr.}{desenhar um círculo no chão para servir como uma prisão; restringir as atividades de alguém a uma área ou esfera designada; limitar; restringir | (literário) ser confinado dentro de um círculo desenhado no chão | (figurativo) limitar-se a uma gama restrita de atividades}
  \end{Phonetics}
\end{Entry}

\begin{Entry}{画面}{8,9}{⽥、⾯}
  \begin{Phonetics}{画面}{hua4 mian4}[][HSK 5]
    \definition[个,幅,帧]{s.}{quadro; aparência geral de uma imagem; imagem apresentada no quadro, na tela, etc.}
  \end{Phonetics}
\end{Entry}

\begin{Entry}{画家}{8,10}{⽥、⼧}
  \begin{Phonetics}{画家}{hua4 jia1}[][HSK 2]
    \definition[个,位,名,些]{s.}{pintor; pessoa com talento para pintura}
  \end{Phonetics}
\end{Entry}

\begin{Entry}{畅}{8}{⽥}
  \begin{Phonetics}{畅}{chang4}
    \definition*{s.}{Sobrenome Chang}
    \definition{adj.}{suave; desimpedido; sem obstáculos; desobstruído | livre; desinibido}
  \end{Phonetics}
\end{Entry}

\begin{Entry}{畅谈}{8,10}{⽥、⾔}
  \begin{Phonetics}{畅谈}{chang4tan2}[][HSK 7-9]
    \definition{v.}{falar livremente e com entusiasmo; falar livremente e com satisfação; falar com entusiasmo sobre}
  \end{Phonetics}
\end{Entry}

\begin{Entry}{畅销}{8,12}{⽥、⾦}
  \begin{Phonetics}{畅销}{chang4xiao1}[][HSK 7-9]
    \definition{adj.}{mais vendido; \emph{best-seller}}
    \definition{v.}{vender bem; ter grande procura; ter um mercado pronto; ser um \emph{best-seller}}
  \end{Phonetics}
\end{Entry}

\begin{Entry}{的}{8}{⽩}
  \begin{Phonetics}{的}{de5}
    \definition{part.}{usado para indicar posse | formar uma frase nominal ou expressão nominal | substituir a pessoa ou coisa mencionada anteriormente | no final de uma frase declarativa, para dar ênfase; usado após o verbo predicativo, enfatiza o agente da ação, o tempo, o local, etc. | usado no final de uma frase declarativa, expressa afirmação, ênfase, certeza, etc. | indica que alguém obteve uma determinada posição ou status | usado com 是 para indicar predicado ou ênfase; indica que alguém é o objeto da ação | e assim por diante; e assim por diante; e similares; usado após palavras paralelas, significa 等等, 之类 | indica uma ação (o pronome é o objeto da ação); combinado com o verbo anterior, expressa uma ação, e o pronome é o objeto dessa ação}
  \seealsoref{等等}{deng3 deng3}
  \seealsoref{是}{shi4}
  \seealsoref{之类}{zhi1 lei4}
  \end{Phonetics}
  \begin{Phonetics}{的}{di1}
    \definition{s.}{abreviação de 的士: um táxi}
  \seealsoref{的士}{di1shi4}
  \end{Phonetics}
  \begin{Phonetics}{的}{di2}
    \definition{adv.}{verdadeiramente; exatamente; realmente}
  \end{Phonetics}
  \begin{Phonetics}{的}{di4}
    \definition{adj.}{alvo; centro do alvo}
  \end{Phonetics}
\end{Entry}

\begin{Entry}{的士}{8,3}{⽩、⼠}
  \begin{Phonetics}{的士}{di1shi4}
    \definition{s.}{(empréstimo linguístico) táxi}
  \end{Phonetics}
\end{Entry}

\begin{Entry}{的时候}{8,7,10}{⽩、⽇、⼈}
  \begin{Phonetics}{的时候}{de5 shi2hou4}
    \definition{part.}{naquele momento; quando; em; descreve o momento específico em que um evento ocorreu}
  \end{Phonetics}
\end{Entry}

\begin{Entry}{的话}{8,8}{⽩、⾔}
  \begin{Phonetics}{的话}{de5 hua4}[][HSK 2]
    \definition{part.}{se; caso; suponha que; partícula usada após uma frase hipotética para introduzir o texto seguinte}
  \end{Phonetics}
\end{Entry}

\begin{Entry}{的确}{8,12}{⽩、⽯}
  \begin{Phonetics}{的确}{di2que4}[][HSK 4]
    \definition{adv.}{realmente; de fato, ao expressar certeza sobre a situação}
  \end{Phonetics}
\end{Entry}

\begin{Entry}{盲}{8}{⽬}
  \begin{Phonetics}{盲}{mang2}
    \definition{adj.}{cego | incapaz de distinguir coisas | totalmente incompetente}
    \definition{adv.}{cegamente}
  \end{Phonetics}
\end{Entry}

\begin{Entry}{盲人}{8,2}{⽬、⼈}
  \begin{Phonetics}{盲人}{mang2 ren2}[][HSK 6]
    \definition[个,位,名]{s.}{cego; pessoa cega; pessoas com deficiência visual}
  \end{Phonetics}
\end{Entry}

\begin{Entry}{盲目}{8,5}{⽬、⽬}
  \begin{Phonetics}{盲目}{mang2mu4}
    \definition{adj.}{ignorante | sem compreensão}
    \definition{adv.}{cegamente}
    \definition{s.}{cego}
  \end{Phonetics}
\end{Entry}

\begin{Entry}{直}{8}{⽬}
  \begin{Phonetics}{直}{zhi2}[][HSK 3]
    \definition*{s.}{Sobrenome Zhi}
    \definition{adj.}{reto (oposto a 弯,曲) | vertical; perpendicular (oposto a 横) | justo; íntegro; imparcial | franco; sincero; direto ao ponto | rígido; entorpecido | direto; em linha reta; rígido | ereto; perpendicular ao solo}
    \definition{adv.}{diretamente; sempre; reto | continuamente; constantemente | apenas; simplesmente | de ​​fato}
    \definition[条]{s.}{traço vertical (em caracteres chineses, 竖)}
    \definition{v.}{endireitar; tornar reto | alongar}
  \seealsoref{横}{heng2}
  \seealsoref{曲}{qu1}
  \seealsoref{竖}{shu4}
  \seealsoref{弯}{wan1}
  \end{Phonetics}
\end{Entry}

\begin{Entry}{直升机}{8,4,6}{⽬、⼗、⽊}
  \begin{Phonetics}{直升机}{zhi2 sheng1 ji1}[][HSK 6]
    \definition[架,台,个]{s.}{helicóptero; uma aeronave que pode decolar e pousar verticalmente, com uma hélice montada na parte superior da fuselagem que gira horizontalmente, permitindo que ela permaneça no ar e decole e pouse em uma pequena área}
  \end{Phonetics}
\end{Entry}

\begin{Entry}{直译}{8,7}{⽬、⾔}
  \begin{Phonetics}{直译}{zhi2yi4}
    \definition{s.}{tradução literal}
  \seealsoref{意译}{yi4yi4}
  \end{Phonetics}
\end{Entry}

\begin{Entry}{直译器}{8,7,16}{⽬、⾔、⼝}
  \begin{Phonetics}{直译器}{zhi2yi4qi4}
    \definition{s.}{(computação) interpretador}
  \end{Phonetics}
\end{Entry}

\begin{Entry}{直到}{8,8}{⽬、⼑}
  \begin{Phonetics}{直到}{zhi2 dao4}[][HSK 3]
    \definition{adv.}{até (geralmente se refere ao tempo); até que}
  \end{Phonetics}
\end{Entry}

\begin{Entry}{直线}{8,8}{⽬、⽷}
  \begin{Phonetics}{直线}{zhi2 xian4}[][HSK 5]
    \definition{adj.}{direto; retilíneo | íngreme; acentuada (subida ou descida)}
    \definition[条]{s.}{linha reta}
  \end{Phonetics}
\end{Entry}

\begin{Entry}{直接}{8,11}{⽬、⼿}
  \begin{Phonetics}{直接}{zhi2jie1}[][HSK 2]
    \definition{adj.}{direto (oposto: indireto 间接) | imediato}
  \seealsoref{间接}{jian4jie1}
  \end{Phonetics}
\end{Entry}

\begin{Entry}{直播}{8,15}{⽬、⼿}
  \begin{Phonetics}{直播}{zhi2bo1}[][HSK 3]
    \definition{v.}{transmissão ao vivo; transmitir diretamente, sem gravar}
  \end{Phonetics}
\end{Entry}

\begin{Entry}{知}{8}{⽮}
  \begin{Phonetics}{知}{zhi1}
    \definition{s.}{conhecimento | amigo íntimo; refere-se a um confidente}
    \definition{v.}{saber; entender; perceber; estar ciente de | contar; informar; notificar; tornar conhecido | administrar; estar encarregado de; supervisionar}
  \end{Phonetics}
\end{Entry}

\begin{Entry}{知了}{8,2}{⽮、⼅}
  \begin{Phonetics}{知了}{zhi1liao3}
    \definition[通]{s.}{cigarra}
  \end{Phonetics}
\end{Entry}

\begin{Entry}{知名}{8,6}{⽮、⼝}
  \begin{Phonetics}{知名}{zhi1 ming2}[][HSK 6]
    \definition{adj.}{notável; famoso; celebrado; bem conhecido}
  \end{Phonetics}
\end{Entry}

\begin{Entry}{知识}{8,7}{⽮、⾔}
  \begin{Phonetics}{知识}{zhi1shi5}[][HSK 1]
    \definition[个,门,种]{s.}{conhecimento; conjunto de conhecimentos e experiências adquiridos pelas pessoas na prática de transformar o mundo | intelectual; refere-se à cultura acadêmica}
  \end{Phonetics}
\end{Entry}

\begin{Entry}{知道}{8,12}{⽮、⾡}
  \begin{Phonetics}{知道}{zhi1dao4}[][HSK 1]
    \definition{v.}{saber; perceber; estar ciente de; ter conhecimento dos fatos ou da razão; ser sensato}
  \end{Phonetics}
\end{Entry}

\begin{Entry}{知道了}{8,12,2}{⽮、⾡、⼅}
  \begin{Phonetics}{知道了}{zhi1dao4le5}
    \definition{interj.}{Entendi! | OK!}
  \end{Phonetics}
\end{Entry}

\begin{Entry}{矿}{8}{⽯}
  \begin{Phonetics}{矿}{kuang4}[][HSK 6]
    \definition[个,座]{s.}{depósito de minério | minério | mina}
  \end{Phonetics}
\end{Entry}

\begin{Entry}{矿泉水}{8,9,4}{⽯、⽔、⽔}
  \begin{Phonetics}{矿泉水}{kuang4quan2shui3}[][HSK 4]
    \definition[瓶,杯,口]{s.}{água mineral de nascente}
  \end{Phonetics}
\end{Entry}

\begin{Entry}{码}{8}{⽯}
  \begin{Phonetics}{码}{ma3}
    \definition{clas.}{refere-se a um assunto específico ou a uma categoria de assuntos; refere-se a uma coisa ou a uma classe de coisas | jarda; unidade de comprimento britânica e americana}
    \definition{s.}{um sinal ou objeto que indica número; código; símbolos ou ferramentas que indicam números}
    \definition{v.}{empilhar; acumular}
  \end{Phonetics}
\end{Entry}

\begin{Entry}{码头}{8,5}{⽯、⼤}
  \begin{Phonetics}{码头}{ma3tou2}[][HSK 5]
    \definition[个,座]{s.}{doca; cais; píer; molhe; edifícios à beira-mar ou à beira do rio destinados exclusivamente à atracação de embarcações, embarque e desembarque de passageiros e carga e descarga de mercadorias | cidade portuária; centro comercial e de transportes; refere-se a uma cidade comercial com transporte terrestre e marítimo bem desenvolvido.}
  \end{Phonetics}
\end{Entry}

\begin{Entry}{祅}{8}{⽰}
  \begin{Phonetics}{祅}{yao1}
    \definition{s.}{espírito maligno | \emph{goblin} | bruxaria}
    \variantof{妖}
  \end{Phonetics}
\end{Entry}

\begin{Entry}{秉}{8}{⽲}
  \begin{Phonetics}{秉}{bing3}
    \definition{clas.}{unidade antiga de volume; 16 hu}
    \definition{s.}{Sobrenome Bing}
    \definition{v.}{Literário: segurar; agarrar | Literário: controlar; presidir; assumir o comando de}
  \seealsoref{斛}{hu2}
  \end{Phonetics}
\end{Entry}

\begin{Entry}{秉承}{8,8}{⽲、⼿}
  \begin{Phonetics}{秉承}{bing3cheng2}[][HSK 7-9]
    \definition{v.}{receber (ordens); receber (comandos); aceitar e seguir (uma ordem ou instrução)}
  \end{Phonetics}
\end{Entry}

\begin{Entry}{空}{8}{⽳}
  \begin{Phonetics}{空}{kong1}[][HSK 3]
    \definition*{s.}{Sobrenome Kong}
    \definition{adj.}{vazio; oco; nulo; não inclui nada; não contém nada ou não tem conteúdo; irrealista}
    \definition{adv.}{por nada; em vão; sem efeito}
    \definition{s.}{céu; ar | vazio; vazio do mundo dos sentidos}
  \end{Phonetics}
  \begin{Phonetics}{空}{kong4}[][HSK 4]
    \definition*{s.}{Sobrenome Kong}
    \definition{adj.}{vazio; oco; nulo; que não contém nada; que não tem nada ou nenhum conteúdo; impraticável}
    \definition{adv.}{para nada; em vão; sem efeito}
    \definition{s.}{céu; ar | vazio; ausência do mundo dos sentidos}
  \end{Phonetics}
\end{Entry}

\begin{Entry}{空儿}{8,2}{⽳、⼉}
  \begin{Phonetics}{空儿}{kong4r5}[][HSK 3]
    \definition[个]{s.}{tempo livre; sem horário específico | sala; espaço (não utilizado); área ainda não utilizada}
    \definition{v.}{ter tempo livre}
  \end{Phonetics}
\end{Entry}

\begin{Entry}{空中}{8,4}{⽳、⼁}
  \begin{Phonetics}{空中}{kong1 zhong1}[][HSK 5]
    \definition{adj.}{aéreo; aerotransportado; refere-se à transmissão de sinais de rádio}
    \definition{s.}{no céu; no ar}
  \end{Phonetics}
\end{Entry}

\begin{Entry}{空中小姐}{8,4,3,8}{⽳、⼁、⼩、⼥}
  \begin{Phonetics}{空中小姐}{kong1zhong1xiao3jie3}
    \definition{s.}{aeromoça}
  \end{Phonetics}
\end{Entry}

\begin{Entry}{空心菜}{8,4,11}{⽳、⼼、⾋}
  \begin{Phonetics}{空心菜}{kong1xin1cai4}
    \definition{s.}{espinafre aquático | \emph{ong choy} | repolho do pântano | convolvulus aquático | glória-da-manhã aquática}
  \seealsoref{蕹菜}{weng4cai4}
  \end{Phonetics}
\end{Entry}

\begin{Entry}{空气}{8,4}{⽳、⽓}
  \begin{Phonetics}{空气}{kong1qi4}[][HSK 2]
    \definition[缕,股,份,阵]{s.}{ar; gases que compõe a atmosfera terrestre | atmosfera}
  \end{Phonetics}
\end{Entry}

\begin{Entry}{空军}{8,6}{⽳、⼍}
  \begin{Phonetics}{空军}{kong1 jun1}[][HSK 6]
    \definition[名,位,个,支]{s.}{força aérea; um exército que luta no ar, geralmente composto por várias unidades de aviação e unidades terrestres da força aérea}
  \end{Phonetics}
\end{Entry}

\begin{Entry}{空间}{8,7}{⽳、⾨}
  \begin{Phonetics}{空间}{kong1jian1}[][HSK 4]
    \definition[个]{s.}{espaço; recinto; cômodo; espaço em branco; interespaço}
  \end{Phonetics}
\end{Entry}

\begin{Entry}{空间站}{8,7,10}{⽳、⾨、⽴}
  \begin{Phonetics}{空间站}{kong1jian1zhan4}
    \definition{s.}{estação espacial}
  \end{Phonetics}
\end{Entry}

\begin{Entry}{空姐}{8,8}{⽳、⼥}
  \begin{Phonetics}{空姐}{kong1jie3}
    \definition[名,位,个]{s.}{aeromoça; comissária de bordo; abreviação de 空中小姐}
  \seealsoref{空中小姐}{kong1zhong1xiao3jie3}
  \end{Phonetics}
\end{Entry}

\begin{Entry}{空调}{8,10}{⽳、⾔}
  \begin{Phonetics}{空调}{kong1tiao2}[][HSK 3]
    \definition[台,个]{s.}{ar-condicionado;  condicionador de ar}
  \end{Phonetics}
\end{Entry}

\begin{Entry}{线}{8}{⽷}
  \begin{Phonetics}{线}{xian4}[][HSK 3]
    \definition{clas.}{usado para coisas abstratas, o número é limitado a ``一''}
    \definition[根,个]{s.}{fio; corda; arame; objetos finos e longos feitos de seda, algodão, metal, etc. | linha; figura formada pelo movimento arbitrário de um ponto| feito de fio de algodão | algo em forma de linha, fio, etc. | rota de transporte; linha | linha de demarcação; limite; zona de fronteira; zona de transição | beira; borda | linha ideológica e política | pista; fio}
  \end{Phonetics}
\end{Entry}

\begin{Entry}{线香}{8,9}{⽷、⾹}
  \begin{Phonetics}{线香}{xian4xiang1}
    \definition{s.}{bastão ou vareta de incenso}
  \end{Phonetics}
\end{Entry}

\begin{Entry}{线索}{8,10}{⽷、⽷}
  \begin{Phonetics}{线索}{xian4suo3}[][HSK 5]
    \definition[条,个]{s.}{pista; fio; metáfora para o desenvolvimento das coisas ou a maneira de explorar um problema | fio; linha; refere-se ao contexto de desenvolvimento do enredo em obras literárias}
  \end{Phonetics}
\end{Entry}

\begin{Entry}{线路}{8,13}{⽷、⾜}
  \begin{Phonetics}{线路}{xian4 lu4}[][HSK 6]
    \definition[条]{s.}{linha; rota; as rotas que os veículos de transporte percorrem, etc., que as pessoas podem usar para chegar aos seus destinos | Eletricidade: linha; circuito; a rota da corrente elétrica}
  \end{Phonetics}
\end{Entry}

\begin{Entry}{练}{8}{⽷}
  \begin{Phonetics}{练}{lian4}[][HSK 2]
    \definition*{s.}{Sobrenome Lian}
    \definition{adj.}{habilidoso; experiente; bem treinado}
    \definition{s.}{seda branca}
    \definition{v.}{tratar, amaciar e branquear a seda por meio de fervura; cozinhar seda crua ou tecidos de seda crua | treinar; praticar; exercitar}
  \end{Phonetics}
\end{Entry}

\begin{Entry}{练习}{8,3}{⽷、⼄}
  \begin{Phonetics}{练习}{lian4xi2}[][HSK 2]
    \definition[项,次]{s.}{exercício (em livros); tarefas ou exercícios organizados para consolidar os resultados da aprendizagem}
    \definition{v.}{praticar; exercitar; repitir várias vezes até ficar bem treinado}
  \end{Phonetics}
\end{Entry}

\begin{Entry}{组}{8}{⽷}
  \begin{Phonetics}{组}{zu3}[][HSK 2]
    \definition{clas.}{usado para conjuntos, séries, suítes, baterias}
    \definition[个]{s.}{grupo; uma unidade composta por um pequeno número de pessoas}
    \definition{v.}{formar; organizar; combinar pessoas ou coisas dispersas em um todo ou sistema}
  \end{Phonetics}
\end{Entry}

\begin{Entry}{组长}{8,4}{⽷、⾧}
  \begin{Phonetics}{组长}{zu3 zhang3}[][HSK 2]
    \definition[名,位,个]{s.}{líder de grupo; um supervisor de grupo}
  \end{Phonetics}
\end{Entry}

\begin{Entry}{组合}{8,6}{⽷、⼝}
  \begin{Phonetics}{组合}{zu3he2}[][HSK 3]
    \definition{s.}{associação; combinação; o todo organizado | combinação; retirar n elementos diferentes de m elementos e agrupá-los, independentemente da ordem, em que cada grupo contenha pelo menos um elemento diferente, o resultado obtido é chamado de combinação de n elementos de m}
    \definition{v.}{compor; constituir; formar}
  \end{Phonetics}
\end{Entry}

\begin{Entry}{组成}{8,6}{⽷、⼽}
  \begin{Phonetics}{组成}{zu3cheng2}[][HSK 2]
    \definition{v.}{formar; compor; inventar}
  \end{Phonetics}
\end{Entry}

\begin{Entry}{组织}{8,8}{⽷、⽷}
  \begin{Phonetics}{组织}{zu3zhi1}[][HSK 5]
    \definition{s.}{organização; um coletivo ou grupo estabelecido de acordo com determinados objetivos e princípios | sistema organizado; vários fatores interligados de determinada maneira, formando um sistema | tecer; a combinação de linhas horizontais e verticais nos têxteis | tecido; os seres humanos, os animais, as plantas e outros seres vivos são compostos por uma combinação de células com formas e funções semelhantes, que formam os tecidos; os tecidos são as unidades que compõem os diversos órgãos}
  \end{Phonetics}
\end{Entry}

\begin{Entry}{细}{8}{⽷}
  \begin{Phonetics}{细}{xi4}[][HSK 4]
    \definition{adj.}{fino; delgado; esguio; esbelto; em oposição a 粗 | fino; em partículas pequenas; grãos pequenos | fino e macio;  um sussuro | fino; requintado; delicado | cuidadoso; detalhado; meticuloso | ínfimo; minúsculo; insignificante; diminuto | jovem; pequeno}
  \seealsoref{粗}{cu1}
  \end{Phonetics}
\end{Entry}

\begin{Entry}{细节}{8,5}{⽷、⾋}
  \begin{Phonetics}{细节}{xi4jie2}[][HSK 4]
    \definition[处]{s.}{detalhe; particularidade; aspectos secundários ou partes sutis de um enredo ou episódios secundários usados em uma obra literária para expressar o caráter de uma pessoa ou as características essenciais de uma coisa}
  \end{Phonetics}
\end{Entry}

\begin{Entry}{细胞}{8,9}{⽷、⾁}
  \begin{Phonetics}{细胞}{xi4bao1}[][HSK 6]
    \definition[个]{s.}{célula; unidade estrutural e funcional básica de um organismo, com uma variedade de formas, composta principalmente pelo núcleo, citoplasma e membrana celular; as plantas também possuem paredes celulares fora da membrana celular}
  \end{Phonetics}
\end{Entry}

\begin{Entry}{细致}{8,10}{⽷、⾄}
  \begin{Phonetics}{细致}{xi4zhi4}[][HSK 4]
    \definition{adj.}{meticuloso; cuidadoso; minucioso | intrincado; delicado}
  \end{Phonetics}
\end{Entry}

\begin{Entry}{细菌}{8,11}{⽷、⾋}
  \begin{Phonetics}{细菌}{xi4jun1}[][HSK 6]
    \definition[个]{s.}{germe; bactéria; um organismo muito pequeno, invisível aos olhos humanos}
  \end{Phonetics}
\end{Entry}

\begin{Entry}{细菌战}{8,11,9}{⽷、⾋、⼽}
  \begin{Phonetics}{细菌战}{xi4jun1zhan4}
    \definition{s.}{guerra biológica}
  \end{Phonetics}
\end{Entry}

\begin{Entry}{织}{8}{⽷}
  \begin{Phonetics}{织}{zhi1}[][HSK 6]
    \definition{v.}{tecer; fazer fios ou linhas cruzarem para fazer seda, tecido, lã, etc. | tricotar; usar agulhas para fazer fios ou linhas entrelaçados para confeccionar suéteres, meias, rendas, redes, etc. | sobrepor-se; entrelaçar-se; cruzar; entrelaçar}
  \end{Phonetics}
\end{Entry}

\begin{Entry}{终}{8}{⽷}
  \begin{Phonetics}{终}{zhong1}
    \definition*{s.}{Sobrenome Zhong}
    \definition{adj.}{tudo; todo; inteiro; o tempo todo do começo ao fim}
    \definition{adv.}{afinal; no final; eventualmente; finalmente}
    \definition{s.}{fim; término | tempo todo; período inteiro; o tempo todo | final | morte; refere-se à morte}
  \end{Phonetics}
\end{Entry}

\begin{Entry}{终于}{8,3}{⽷、⼆}
  \begin{Phonetics}{终于}{zhong1yu2}[][HSK 3]
    \definition{adv.}{finalmente; por fim; eventualmente; no final; indica uma situação que surge após várias mudanças ou espera}
  \end{Phonetics}
\end{Entry}

\begin{Entry}{终止}{8,4}{⽷、⽌}
  \begin{Phonetics}{终止}{zhong1zhi3}[][HSK 5]
    \definition{v.}{parar; terminar | anular; encerrar; expirar; revogar}
  \end{Phonetics}
\end{Entry}

\begin{Entry}{终究}{8,7}{⽷、⽳}
  \begin{Phonetics}{终究}{zhong1jiu1}
    \definition{adv.}{afinal de contas; enfatiza que, não importa o que aconteça, a natureza das pessoas e das coisas não mudará e que as características básicas devem ser reconhecidas (tem o efeito de fortalecer o tom) |  no final; indica que um determinado resultado ocorrerá ou não, frequentemente usado em especulações, julgamentos etc. | afinal de contas; indica que, apesar do grande esforço ou da grande esperança, o resultado objetivo ainda é insatisfatório, geralmente com o significado de pesar ou pena | afinal de contas; indica que um resultado desejado finalmente aparece}
  \end{Phonetics}
\end{Entry}

\begin{Entry}{终身}{8,7}{⽷、⾝}
  \begin{Phonetics}{终身}{zhong1shen1}[][HSK 5]
    \definition{s.}{vida inteira; por toda a vida; por toda a vida}
  \end{Phonetics}
\end{Entry}

\begin{Entry}{终点}{8,9}{⽷、⽕}
  \begin{Phonetics}{终点}{zhong1dian3}[][HSK 5]
    \definition[个]{s.}{destino; ponto terminal; ponto de chegada; lugar onde uma jornada termina | final; refere-se especificamente ao local onde a corrida é interrompida}
  \end{Phonetics}
\end{Entry}

\begin{Entry}{绍}{8}{⽷}
  \begin{Phonetics}{绍}{shao4}
    \definition*{s.}{Shaoxing, abreviação de 绍兴 | Sobrenome Shao}
    \definition{v.}{continuar; herdar}
  \seealsoref{绍兴}{shao4xing1}
  \end{Phonetics}
\end{Entry}

\begin{Entry}{绍兴}{8,6}{⽷、⼋}
  \begin{Phonetics}{绍兴}{shao4xing1}
    \definition*{s.}{Shaoxing, anteriormente conhecida como Kuaiji, é uma cidade de nível de prefeitura na província de Zhejiang, na China; é uma grande cidade localizada na parte centro-norte da província de Zhejiang}
  \end{Phonetics}
\end{Entry}

\begin{Entry}{经}{8}{⽷}
  \begin{Phonetics}{经}{jing1}
    \definition*{s.}{Sobrenome Jing}
    \definition{adj.}{constante; regular}
    \definition{prep.}{como resultado de; depois; através de}
    \definition{s.}{urdidura, os fios longitudinais de um tecido (oposto a 纬) | Medicina chinesa: canais principais e colaterais | Geografia: longitude (oposto a 纬) | escritura; sutra; cânone; clássico | menstruação}
    \definition{v.}{Literário: gerenciar; lidar com; envolver-se em | enforcar-se | suportar; ficar de pé; aguentar; resistir | passar por; sofrer; experimentar}
  \seealsoref{纬}{wei3}
  \end{Phonetics}
\end{Entry}

\begin{Entry}{经历}{8,4}{⽷、⼚}
  \begin{Phonetics}{经历}{jing1li4}[][HSK 3]
    \definition[个,次,段,种]{s.}{experiência; coisas que você viu, fez ou sofreu pessoalmente}
    \definition{v.}{passar por; atravessar; ter visto, feito ou sofrido pessoalmente}
  \end{Phonetics}
\end{Entry}

\begin{Entry}{经过}{8,6}{⽷、⾡}
  \begin{Phonetics}{经过}{jing1guo4}[][HSK 2]
    \definition{prep.}{depois; através; como resultado de; passar por uma atividade ou evento que traz novas mudanças para pessoas ou coisas}
    \definition[个,段,番]{s.}{processo; curso; experiência}
    \definition{v.}{passar; atravessar; passar por; através de (local, tempo, ação, etc.)}
  \end{Phonetics}
\end{Entry}

\begin{Entry}{经典}{8,8}{⽷、⼋}
  \begin{Phonetics}{经典}{jing1dian3}[][HSK 4]
    \definition{adj.}{clássico; (escritos ou obras, etc.) que são típicos, autorizados}
    \definition{s.}{clássicos; escritos tradicionais e valiosos; os livros mais importantes e fundamentais da religião | escrituras; escritos de doutrinas religiosas}
  \end{Phonetics}
\end{Entry}

\begin{Entry}{经济}{8,9}{⽷、⽔}
  \begin{Phonetics}{经济}{jing1ji4}[][HSK 3]
    \definition{adj.}{econômico;  parcimonioso; descreve algo que custa pouco e rende muito; preço acessível}
    \definition{s.}{economia; a soma das relações de produção social em um determinado período histórico|econômico; de valor industrial ou econômico; refere-se à economia nacional; também se refere a um determinado setor da economia nacional | economia; refere-se às atividades econômicas, incluindo produção, circulação, distribuição e consumo, bem como atividades ou processos financeiros, de seguros, etc. | renda; situação financeira; refere-se à situação financeira de uma pessoa}
    \definition{v.}{governar o país e beneficiar o povo}
  \end{Phonetics}
\end{Entry}

\begin{Entry}{经费}{8,9}{⽷、⾙}
  \begin{Phonetics}{经费}{jing1fei4}[][HSK 5]
    \definition[笔]{s.}{fundos; desembolso; gastos | despesas; gastos}
  \end{Phonetics}
\end{Entry}

\begin{Entry}{经验}{8,10}{⽷、⾺}
  \begin{Phonetics}{经验}{jing1yan4}[][HSK 3]
    \definition[个,次,种]{s.}{experiência; conhecimento ou habilidades adquiridos através da prática}
    \definition{v.}{experimentar; passar por; ter visto, feito ou sofrido pessoalmente}
  \end{Phonetics}
\end{Entry}

\begin{Entry}{经常}{8,11}{⽷、⼱}
  \begin{Phonetics}{经常}{jing1chang2}[][HSK 2]
    \definition{adj.}{habitual; cotidiano; diário; do dia a dia}
    \definition{adv.}{frequentemente; regularmente; constantemente; com frequência; indica que a ação ocorre repetidamente}
  \end{Phonetics}
\end{Entry}

\begin{Entry}{经理}{8,11}{⽷、⽟}
  \begin{Phonetics}{经理}{jing1li3}[][HSK 2]
    \definition[个,位,名]{s.}{gerente; diretor; pessoas responsáveis pela gestão e administração de empresas ou corporações}
  \end{Phonetics}
\end{Entry}

\begin{Entry}{经营}{8,11}{⽷、⾋}
  \begin{Phonetics}{经营}{jing1ying2}[][HSK 3]
    \definition{v.}{executar; gerenciar; operar; envolver-se em; planejar e gerenciar (empresas, etc.) | gerenciar; refere-se a planos e organizações em geral}
  \end{Phonetics}
\end{Entry}

\begin{Entry}{罔}{8}{⼌}
  \begin{Phonetics}{罔}{wang3}
    \definition{v.}{enganar}
  \end{Phonetics}
\end{Entry}

\begin{Entry}{罗}{8}{⽹}
  \begin{Phonetics}{罗}{luo2}
    \definition*{s.}{Sobrenome Luo}
    \definition{clas.}{uma grosa; uma bruta; doze dúzias; 144 unidades}
    \definition{s.}{uma rede para capturar pássaros | peneira; tela | uma espécie de gaze de seda}
    \definition{v.}{pegar pássaros com uma rede | espalhar; exibir; mostrar | coletar; reunir; recrutar | peneirar}
  \end{Phonetics}
\end{Entry}

\begin{Entry}{者}{8}{⽼}
  \begin{Phonetics}{者}{zhe3}[][HSK 3]
    \definition*{s.}{Sobrenome Zhe}
    \definition{part.}{significa 是 e é usado após palavras, frases e orações para indicar uma pausa}
    \definition{pron.}{usado para se referir à pessoa, coisa ou assunto que realiza uma ação ou possui um determinado atributo | pessoas; caras (Usado para se referir a alguém envolvido em uma determinada profissão, que acredita em uma determinada ideologia ou que tem uma forte tendência para algo) | usado após certos números ou palavras direcionais para se referir a coisas mencionadas anteriormente | significado semelhante a 这 (mais comum na linguagem coloquial antiga)}
  \seealsoref{是}{shi4}
  \seealsoref{这}{zhe4}
  \end{Phonetics}
\end{Entry}

\begin{Entry}{肏}{8}{⼊}
  \begin{Phonetics}{肏}{cao4}
    \definition{v.}{(vulgar) foder; palavras sujas usadas para insultar pessoas; refere-se à relação sexual masculina}
  \end{Phonetics}
\end{Entry}

\begin{Entry}{股}{8}{⾁}
  \begin{Phonetics}{股}{gu3}[][HSK 6]
    \definition*{s.}{Sobrenome Gu}
    \definition{clas.}{usado para coisas em tiras, longas e estreitas | usado para gás, odor, força, etc. | Pejorativo: usado para um grupo de pessoas}
    \definition{s.}{coxa; ancas | seção (de um escritório, empresa, etc.); unidades organizacionais em agências governamentais, empresas e grupos | fio; camada | uma das várias partes iguais de propriedade | ação; \emph{stock}; ação do capital social; uma parte igual de fundos ou propriedade | a perna mais longa de um triângulo retângulo}
  \end{Phonetics}
\end{Entry}

\begin{Entry}{股东}{8,5}{⾁、⼀}
  \begin{Phonetics}{股东}{gu3dong1}[][HSK 6]
    \definition[个,位,名,家]{s.}{acionista de uma sociedade anônima com direito a participar e votar nas assembleias gerais; refere-se também a investidores em outras empresas industriais e comerciais administradas por sociedades}
  \end{Phonetics}
\end{Entry}

\begin{Entry}{股市}{8,5}{⾁、⼱}
  \begin{Phonetics}{股市}{gu3shi4}[][HSK 7-9]
    \definition{s.}{mercado de ações; mercado de compra e venda de ações | cotações na bolsa de valores}
  \end{Phonetics}
\end{Entry}

\begin{Entry}{股民}{8,5}{⾁、⽒}
  \begin{Phonetics}{股民}{gu3min2}[][HSK 7-9]
    \definition{s.}{pessoa que compra e vende ações; acionista | corretor de ações | investidor em ações}
  \end{Phonetics}
\end{Entry}

\begin{Entry}{股份}{8,6}{⾁、⼈}
  \begin{Phonetics}{股份}{gu3fen4}[][HSK 7-9]
    \definition{s.}{ação; unidade de distribuição de capital de uma sociedade anônima ou de uma empresa cooperativa, com uma parcela igual do capital total}
  \end{Phonetics}
\end{Entry}

\begin{Entry}{股票}{8,11}{⾁、⽰}
  \begin{Phonetics}{股票}{gu3piao4}[][HSK 6]
    \definition[只,股]{s.}{ação; quotas; certificado de ações; título de capital; capital social; títulos utilizados para representar ações}
  \end{Phonetics}
\end{Entry}

\begin{Entry}{肥}{8}{⾁}
  \begin{Phonetics}{肥}{fei2}[][HSK 4]
    \definition{adj.}{gordo; gorduroso; contém muita gordura (o oposto de 瘦, geralmente não usado para descrever pessoas) | fértil; rico | solto; largo; folgado; (roupas, etc.) largas (em oposição a 瘦) | lucrativo; rendendo bons lucros}
    \definition{s.}{fertilizante; esterco}
    \definition{v.}{fertilizar; tornar fértil ou obeso | enriquecer com renda ilegal, ilícita}
  \seealsoref{瘦}{shou4}
  \end{Phonetics}
\end{Entry}

\begin{Entry}{肥沃}{8,7}{⾁、⽔}
  \begin{Phonetics}{肥沃}{fei2wo4}[][HSK 7-9]
    \definition{adj.}{fértil; rico (de solo); (terra) contém mais nutrientes e água adequados para o crescimento das plantas}
  \end{Phonetics}
\end{Entry}

\begin{Entry}{肥皂}{8,7}{⾁、⽩}
  \begin{Phonetics}{肥皂}{fei2zao4}[][HSK 7-9]
    \definition[块,条]{s.}{sabão; produtos químicos usados ​​para limpeza}
  \end{Phonetics}
\end{Entry}

\begin{Entry}{肥胖}{8,9}{⾁、⾁}
  \begin{Phonetics}{肥胖}{fei2pang4}[][HSK 7-9]
    \definition{adj.}{gordo; obeso; corpulento; excesso de gordura corporal}
  \end{Phonetics}
\end{Entry}

\begin{Entry}{肥料}{8,10}{⾁、⽃}
  \begin{Phonetics}{肥料}{fei2liao4}[][HSK 7-9]
    \definition[种,袋,把]{s.}{esterco; fertilizante}
  \end{Phonetics}
\end{Entry}

\begin{Entry}{肩}{8}{⾁}
  \begin{Phonetics}{肩}{jian1}[][HSK 5]
    \definition*{s.}{Sobrenome Jian}
    \definition{s.}{ombro; torso}
    \definition{v.}{assumir; empreender; carregar; suportar; suportar um fardo}
  \end{Phonetics}
\end{Entry}

\begin{Entry}{肩膀}{8,14}{⾁、⾁}
  \begin{Phonetics}{肩膀}{jian1bang3}
    \definition{s.}{ombro}
  \end{Phonetics}
\end{Entry}

\begin{Entry}{肯}{8}{⾁}
  \begin{Phonetics}{肯}{ken3}[][HSK 6]
    \definition{s.}{carne presa ao osso}
    \definition{v.}{concordar; consentir}
    \definition{v.aux.}{estar disposto a; estar pronto para; para expressar vontade subjetiva; vontade de aceitar}
  \end{Phonetics}
\end{Entry}

\begin{Entry}{肯定}{8,8}{⾁、⼧}
  \begin{Phonetics}{肯定}{ken3ding4}[][HSK 5]
    \definition{adj.}{certo; definitivo; positivo; afirmativo | positivo; afirmativo; aceitável}
    \definition{adv.}{certamente; definitivamente; sem dúvida; sem dúvida alguma}
    \definition{v.}{afirmar; aprovar; confirmar; considerar positivo; reconhecer a existência de algo ou sua autenticidade ou racionalidade (em oposição à 否定)}
  \seealsoref{否定}{fou3ding4}
  \end{Phonetics}
\end{Entry}

\begin{Entry}{肺}{8}{⾁}
  \begin{Phonetics}{肺}{fei4}[][HSK 6]
    \definition[叶]{s.}{pulmão | pulmões; órgãos respiratórios de humanos e animais superiores}
  \end{Phonetics}
\end{Entry}

\begin{Entry}{肿}{8}{⾁}
  \begin{Phonetics}{肿}{zhong3}[][HSK 6]
    \definition{s.}{inchaço; protuberância}
    \definition{v.}{inchar; estar inchado}
  \end{Phonetics}
\end{Entry}

\begin{Entry}{舍}{8}{⾆}
  \begin{Phonetics}{舍}{she3}
    \definition{v.}{abandonar; desistir; descartar; jogar fora | dar esmola; dispensar caridade}
  \end{Phonetics}
  \begin{Phonetics}{舍}{she4}
    \definition*{s.}{Sobrenome She}
    \definition{clas.}{uma unidade antiga de distância igual a 30 li, 里}
    \definition{pron.}{meu, uma palavra humilde usada para se referir aos parentes mais jovens ou de geração inferior}
    \definition{s.}{cabana; casa | minha casa; minha humilde morada | chiqueiro; galpão; curral de gado}
  \seealsoref{里}{li3}
  \end{Phonetics}
\end{Entry}

\begin{Entry}{舍不得}{8,4,11}{⾆、⼀、⼻}
  \begin{Phonetics}{舍不得}{she3bu5de5}[][HSK 5]
    \definition{v.}{não se pode abandonar ou deixar, não se quer usar ou descartar; detestar separar-me ou usar}
  \end{Phonetics}
\end{Entry}

\begin{Entry}{舍得}{8,11}{⾆、⼻}
  \begin{Phonetics}{舍得}{she3 de5}[][HSK 5]
    \definition{v.}{não guardar rancor; estar disposto a abrir mão de algo; estar disposto a gastar dinheiro, tempo, etc.; estar disposto a abrir mão de pessoas, oportunidades, coisas, etc. que são importantes para você}
  \end{Phonetics}
\end{Entry}

\begin{Entry}{艰}{8}{⾉}
  \begin{Phonetics}{艰}{jian1}
    \definition{adj.}{difícil; duro}
  \end{Phonetics}
\end{Entry}

\begin{Entry}{艰苦}{8,8}{⾉、⾋}
  \begin{Phonetics}{艰苦}{jian1ku3}[][HSK 5]
    \definition{adj.}{duro; resistente; árduo; difícil; condições de trabalho ou de vida ruins que tornam as pessoas miseráveis}
  \end{Phonetics}
\end{Entry}

\begin{Entry}{艰难}{8,10}{⾉、⾫}
  \begin{Phonetics}{艰难}{jian1nan2}[][HSK 5]
    \definition{adj.}{duro; árduo; difícil}
  \end{Phonetics}
\end{Entry}

\begin{Entry}{若}{8}{⾋}
  \begin{Phonetics}{若}{ruo4}[][HSK 6]
    \definition*{s.}{Sobrenome Ruo}
    \definition{adv.}{como se; como se fosse; usado antes do verbo para indicar que o que foi dito é mais ou menos assim, equivalente a 好像}
    \definition{conj.}{se; usado na primeira parte de uma frase composta, expressa uma relação hipotética, equivalente a 如果}
    \definition{pron.}{você; referir-se ao interlocutor como 你 ou 你的}
    \definition{v.}{parecer}
  \seealsoref{好像}{hao3xiang4}
  \seealsoref{你}{ni3}
  \seealsoref{你的}{ni3 de5}
  \seealsoref{如果}{ru2guo3}
  \end{Phonetics}
\end{Entry}

\begin{Entry}{苦}{8}{⾋}
  \begin{Phonetics}{苦}{ku3}[][HSK 4]
    \definition{adj.}{amargo; descreve um sabor parecido com o de melão amargo ou raiz de coptis (em oposição a 甘 ou 甜) | difícil; doloroso; sofrido}
    \definition{adv.}{meticulosamente; diligentemente; pacientemente}
    \definition{v.}{causar sofrimento a alguém; dificultar a vida de alguém; causar dor; tornar desconfortável | sofrer de; ser incomodado por; sentir-se angustiado com uma situação | estar desgastado; cortar demais; descrever a superação de um certo nível em algum aspecto}
  \seealsoref{甘}{gan1}
  \seealsoref{甜}{tian2}
  \end{Phonetics}
\end{Entry}

\begin{Entry}{苦瓜}{8,5}{⾋、⽠}
  \begin{Phonetics}{苦瓜}{ku3gua1}
    \definition{s.}{melão amargo (cabaça amarga, pêra bálsamo, maçã bálsamo, pepino amargo)}
  \end{Phonetics}
\end{Entry}

\begin{Entry}{英}{8}{⾋}
  \begin{Phonetics}{英}{ying1}
    \definition*{s.}{Reino Unido, abreviação de 英国 | Sobrenome Ying}
    \definition{s.}{flor | herói; pessoa excepcional | uma pessoa de talento ou sabedoria extraordinários}
  \seealsoref{英国}{ying1guo2}
  \end{Phonetics}
\end{Entry}

\begin{Entry}{英文}{8,4}{⾋、⽂}
  \begin{Phonetics}{英文}{ying1 wen2}[][HSK 2]
    \definition{s.}{inglês, língua inglesa; a forma escrita do inglês}
  \end{Phonetics}
\end{Entry}

\begin{Entry}{英国}{8,8}{⾋、⼞}
  \begin{Phonetics}{英国}{ying1guo2}
    \definition*{s.}{Reino Unido; Grã-Bretanha; Inglaterra}
  \end{Phonetics}
\end{Entry}

\begin{Entry}{英国人}{8,8,2}{⾋、⼞、⼈}
  \begin{Phonetics}{英国人}{ying1guo2ren2}
    \definition{s.}{inglês | pessoa ou povo do Reino Unido}
  \end{Phonetics}
\end{Entry}

\begin{Entry}{英明}{8,8}{⾋、⽇}
  \begin{Phonetics}{英明}{ying1ming2}
    \definition{adj.}{sábio; brilhante; excelente e sábio}
  \end{Phonetics}
\end{Entry}

\begin{Entry}{英勇}{8,9}{⾋、⼒}
  \begin{Phonetics}{英勇}{ying1yong3}[][HSK 4]
    \definition{adj.}{heroico; valente; bravo; corajoso; extraordinariamente corajoso}
  \end{Phonetics}
\end{Entry}

\begin{Entry}{英语}{8,9}{⾋、⾔}
  \begin{Phonetics}{英语}{ying1 yu3}[][HSK 2]
    \definition{s.}{inglês, língua inglesa}
  \end{Phonetics}
\end{Entry}

\begin{Entry}{英雄}{8,12}{⾋、⾫}
  \begin{Phonetics}{英雄}{ying1xiong2}[][HSK 6]
    \definition{adj.}{heróico}
    \definition[名,个,位]{s.}{herói; uma pessoa cujas habilidades e coragem superam as das pessoas comuns | herói; aqueles que não têm medo das dificuldades, dos perigos ou da morte e que lutam bravamente pelos interesses do povo, mesmo ao custo das suas próprias vidas}
  \end{Phonetics}
\end{Entry}

\begin{Entry}{苹}{8}{⾋}
  \begin{Phonetics}{苹}{ping2}
    \definition[个]{s.}{uma espécie de artemísia | maçã | lentilha-d'água}
  \end{Phonetics}
\end{Entry}

\begin{Entry}{苹果}{8,8}{⾋、⽊}
  \begin{Phonetics}{苹果}{ping2guo3}[][HSK 3]
    \definition[个,斤,筐,箱,棵,种]{s.}{maçã}
  \end{Phonetics}
\end{Entry}

\begin{Entry}{茄}{8}{⾋}
  \begin{Phonetics}{茄}{jia1}
    \definition{s.}{caracter fonético usado em empréstimos linguísticos para o som "jia", embora 夹 seja mais comum}
  \seealsoref{夹}{jia1}
  \end{Phonetics}
  \begin{Phonetics}{茄}{qie2}
    \definition[只]{s.}{berinjela}
  \end{Phonetics}
\end{Entry}

\begin{Entry}{茄子}{8,3}{⾋、⼦}
  \begin{Phonetics}{茄子}{qie2 zi5}[][HSK 6]
    \definition{interj.}{Onomatopéia: ``xis'' fonético (ao ser fotografado), equivale ao ``diga xis''}
    \definition[个,根]{s.}{berinjela (fruto e planta)}
  \end{Phonetics}
\end{Entry}

\begin{Entry}{茅}{8}{⾋}
  \begin{Phonetics}{茅}{mao2}
    \definition*{s.}{Sobrenome Mao}
    \definition[座]{s.}{capim-cogon | planta semelhante ao capim-cogon (como palha)}
  \end{Phonetics}
\end{Entry}

\begin{Entry}{茅厕}{8,8}{⾋、⼚}
  \begin{Phonetics}{茅厕}{mao2ce4}
    \definition{s.}{(dialeto) latrina}
  \end{Phonetics}
\end{Entry}

\begin{Entry}{虎}{8}{⾌}
  \begin{Phonetics}{虎}{hu3}[][HSK 5]
    \definition*{s.}{Sobrenome Hu}
    \definition{adj.}{corajoso; bravo; valente; vigoroso}
    \definition[只]{s.}{tigre}
    \definition{v.}{blefar; o mesmo que 唬 | parecer feroz; mostrar a aparência feroz de alguém}
  \seealsoref{唬}{hu3}
  \seealsoref{老虎}{lao3hu3}
  \end{Phonetics}
\end{Entry}

\begin{Entry}{虎口}{8,3}{⾌、⼝}
  \begin{Phonetics}{虎口}{hu3kou3}
    \definition{s.}{lugar perigoso | cova do tigre}
  \end{Phonetics}
\end{Entry}

\begin{Entry}{虎虎}{8,8}{⾌、⾌}
  \begin{Phonetics}{虎虎}{hu3hu3}
    \definition{adj.}{formidável | forte | vigoroso}
  \end{Phonetics}
\end{Entry}

\begin{Entry}{虎鼬}{8,18}{⾌、⿏}
  \begin{Phonetics}{虎鼬}{hu3you4}
    \definition{s.}{doninha}
  \end{Phonetics}
\end{Entry}

\begin{Entry}{表}{8}{⾐}
  \begin{Phonetics}{表}{biao3}[][HSK 2]
    \definition*{s.}{Sobrenome Biao}
    \definition{s.}{exterior; superfície; externo | a relação entre os filhos ou netos de um irmão e uma irmã ou de irmãs | modelo; exemplo; padrão | memorial a um imperador; um tipo de petição da antiguidade, frequentemente usado para expressar intenções; mais tarde, também usado para expressar opiniões sobre eventos importantes | formulário; lista; gráfico; tabela | medidor; instrumento para medir uma determinada quantidade | relógio; um dispositivo para medir o tempo, menor que um relógio, que geralmente pode ser carregado no bolso | medidor de luz solar; antiga vara de madeira para medir o tempo através da sombra do sol | coluna usada antigamente para marcação}
    \definition{v.}{mostrar; expressar; expressar ideias, pensamentos, sentimentos, etc. | administrar medicamentos para aliviar o resfriado; na medicina tradicional chinesa refere-se ao uso de medicamentos para dissipar o frio e o vento que afetam o corpo}
  \end{Phonetics}
\end{Entry}

\begin{Entry}{表白}{8,5}{⾐、⽩}
  \begin{Phonetics}{表白}{biao3bai2}[][HSK 7-9]
    \definition{v.}{justificar; explicar-se; expressar ou declarar claramente; explicar (as próprias intenções) aos outros}
  \end{Phonetics}
\end{Entry}

\begin{Entry}{表示}{8,5}{⾐、⽰}
  \begin{Phonetics}{表示}{biao3shi4}[][HSK 2]
    \definition{s.}{expressão; indicação}
    \definition{v.}{mostrar; expressar; indicar | significar | expressar pensamentos e sentimentos através de palavras, ações ou expressões faciais}
  \end{Phonetics}
\end{Entry}

\begin{Entry}{表决}{8,6}{⾐、⼎}
  \begin{Phonetics}{表决}{biao3jue2}[][HSK 7-9]
    \definition{v.}{votar; colocar em votação; decidir por votação}
  \end{Phonetics}
\end{Entry}

\begin{Entry}{表扬}{8,6}{⾐、⼿}
  \begin{Phonetics}{表扬}{biao3yang2}[][HSK 4]
    \definition{v.}{elogiar; louvar; elogiar publicamente as pessoas boas e as boas ações}
  \end{Phonetics}
\end{Entry}

\begin{Entry}{表扬信}{8,6,9}{⾐、⼿、⼈}
  \begin{Phonetics}{表扬信}{biao3yang2 xin4}
    \definition{s.}{carta de elogio; depoimento}
  \end{Phonetics}
\end{Entry}

\begin{Entry}{表达}{8,6}{⾐、⾡}
  \begin{Phonetics}{表达}{biao3da2}[][HSK 3]
    \definition{v.}{entregar; expressar; mostrar; manifestar; transmitir; comunicar; refere-se ao processo de transmitir pensamentos, sentimentos ou opiniões pessoais a outras pessoas por meio de linguagem, texto, ações, etc.}
  \end{Phonetics}
\end{Entry}

\begin{Entry}{表态}{8,8}{⾐、⼼}
  \begin{Phonetics}{表态}{biao3/tai4}[][HSK 7-9]
    \definition{v.+compl.}{tornar conhecida a sua posição; declarar onde se posiciona; comprometer-se; expressar claramente a atitude de alguém em relação a algo}
  \end{Phonetics}
\end{Entry}

\begin{Entry}{表明}{8,8}{⾐、⽇}
  \begin{Phonetics}{表明}{biao3ming2}[][HSK 3]
    \definition{v.}{indicar; demonstrar; expressar; marcar; expressar claramente; expressar de forma clara}
  \end{Phonetics}
\end{Entry}

\begin{Entry}{表现}{8,8}{⾐、⾒}
  \begin{Phonetics}{表现}{biao3xian4}[][HSK 3]
    \definition[个,种,份]{s.}{desempenho; expressão; manifestação; comportamento; as ideias, o estilo, as qualidades, o nível ou as capacidades demonstrados em ação.}
    \definition{v.}{mostrar; expressar; exibir; manifestar; descrever; demonstrar algum tipo de pensamento, espírito, qualidade, sentimento ou habilidade, etc. | exibir-se; demonstrar de forma inadequada e intencional alguma habilidade, ponto forte ou vantagem.}
  \end{Phonetics}
\end{Entry}

\begin{Entry}{表述}{8,8}{⾐、⾡}
  \begin{Phonetics}{表述}{biao3shu4}[][HSK 7-9]
    \definition{s.}{formulação; expressão}
    \definition{v.}{declarar; explicar; descrever em palavras ou texto}
  \end{Phonetics}
\end{Entry}

\begin{Entry}{表面}{8,9}{⾐、⾯}
  \begin{Phonetics}{表面}{biao3mian4}[][HSK 3]
    \definition{s.}{superfície; face; exterior; aparência | aparência; superficialidade | mostrador (placa); mostrador do relógio | aparência; a aparência externa das coisas ou a parte não essencial delas}
  \end{Phonetics}
\end{Entry}

\begin{Entry}{表面上}{8,9,3}{⾐、⾯、⼀}
  \begin{Phonetics}{表面上}{biao3 mian4 shang4}[][HSK 6]
    \definition{adj.}{superficial; ostensivo; aparente}
  \end{Phonetics}
\end{Entry}

\begin{Entry}{表格}{8,10}{⾐、⽊}
  \begin{Phonetics}{表格}{biao3ge2}[][HSK 3]
    \definition[张,份,个]{s.}{tabela; formulário}
  \end{Phonetics}
\end{Entry}

\begin{Entry}{表情}{8,11}{⾐、⼼}
  \begin{Phonetics}{表情}{biao3qing2}[][HSK 4]
    \definition[个,种,幅]{s.}{expressão; expressão facial; expressão de pensamentos e sentimentos internos por meio de mudanças faciais ou de gestos}
    \definition{v.}{expressar pensamentos e sentimentos internos por meio de mudanças faciais ou de gestos}
  \end{Phonetics}
\end{Entry}

\begin{Entry}{表率}{8,11}{⾐、⽞}
  \begin{Phonetics}{表率}{biao3shuai4}[][HSK 7-9]
    \definition{s.}{modelo; exemplo}
  \end{Phonetics}
\end{Entry}

\begin{Entry}{表彰}{8,14}{⾐、⼺}
  \begin{Phonetics}{表彰}{biao3zhang1}[][HSK 7-9]
    \definition{v.}{citar; honrar; elogiar}
  \end{Phonetics}
\end{Entry}

\begin{Entry}{表演}{8,14}{⾐、⽔}
  \begin{Phonetics}{表演}{biao3yan3}[][HSK 3]
    \definition[场]{s.}{performance; exposição; refere-se às atividades expressas pelos atores por meio da linguagem, voz, expressões faciais, instrumentos musicais ou movimentos}
    \definition{v.}{atuar; representar; interpretar | demonstrar; fazer demonstrações | fingir; agir de forma afetada; metáfora para fingir deliberadamente uma determinada atitude para enganar alguém}
  \end{Phonetics}
\end{Entry}

\begin{Entry}{表演艺术}{8,14,4,5}{⾐、⽔、⾋、⽊}
  \begin{Phonetics}{表演艺术}{biao3yan3 yi4shu4}
    \definition{s.}{arte performática}
  \end{Phonetics}
\end{Entry}

\begin{Entry}{表演者}{8,14,8}{⾐、⽔、⽼}
  \begin{Phonetics}{表演者}{biao3yan3 zhe3}
    \definition{s.}{artista; intérprete}
  \end{Phonetics}
\end{Entry}

\begin{Entry}{表演特技}{8,14,10,7}{⾐、⽔、⽜、⼿}
  \begin{Phonetics}{表演特技}{biao3yan3 te4ji4}
    \definition{s.}{acrobacia | pirueta | façanha}
  \end{Phonetics}
\end{Entry}

\begin{Entry}{表演游戏}{8,14,12,6}{⾐、⽔、⽔、⼽}
  \begin{Phonetics}{表演游戏}{biao3yan3 you2xi4}
    \definition{s.}{exibição dramática}
  \end{Phonetics}
\end{Entry}

\begin{Entry}{表演赛}{8,14,14}{⾐、⽔、⾙}
  \begin{Phonetics}{表演赛}{biao3yan3sai4}
    \definition{s.}{partida de exibição; jogo de exibição; uma competição realizada para celebração, comemoração, demonstração, publicidade, etc.}
  \end{Phonetics}
\end{Entry}

\begin{Entry}{衬}{8}{⾐}
  \begin{Phonetics}{衬}{chen4}
    \definition[件,个]{s.}{forro}
    \definition{v.}{forrar; colocar algo embaixo | fornecer um pano de fundo para; destacar; servir como contraste para}
  \end{Phonetics}
\end{Entry}

\begin{Entry}{衬托}{8,6}{⾐、⼿}
  \begin{Phonetics}{衬托}{chen4tuo1}[][HSK 7-9]
    \definition{v.}{destacar; acentuar}[红色衬托了她的笑容。===O vermelho acentuava seu sorriso.]
  \end{Phonetics}
\end{Entry}

\begin{Entry}{衬衣}{8,6}{⾐、⾐}
  \begin{Phonetics}{衬衣}{chen4 yi1}[][HSK 3]
    \definition[件,个]{s.}{camisa; também se refere a uma peça de roupa usada por baixo do casaco}
  \end{Phonetics}
\end{Entry}

\begin{Entry}{衬衫}{8,8}{⾐、⾐}
  \begin{Phonetics}{衬衫}{chen4shan1}[][HSK 3]
    \definition[件,个]{s.}{camisa; blusa; camisa ocidental usada por baixo}
  \end{Phonetics}
\end{Entry}

\begin{Entry}{规}{8}{⾒}
  \begin{Phonetics}{规}{gui1}
    \definition*{s.}{Sobrenome Gui}
    \definition[个,种]{s.}{bússola | regulamentação; regra | (mecânica) medidor | compasso; ferramenta para desenhar círculos}
    \definition{v.}{admoestar; aconselhar; advertir | planejar; fazer planos}
  \end{Phonetics}
\end{Entry}

\begin{Entry}{规划}{8,6}{⾒、⼑}
  \begin{Phonetics}{规划}{gui1hua4}[][HSK 5]
    \definition[个,项]{s.}{plano; projeto; planejamento; programa; programação; esquematização; plano de desenvolvimento de longo prazo mais abrangente}
    \definition{v.}{planejar; programar}
  \end{Phonetics}
\end{Entry}

\begin{Entry}{规则}{8,6}{⾒、⼑}
  \begin{Phonetics}{规则}{gui1ze2}[][HSK 4]
    \definition{adj.}{ordenado; regular; descreve a forma, estrutura, arranjo, etc., que se conformam a uma determinada maneira organizada}
    \definition{s.}{regra; regulamento; sistema ou código de conduta prescrito para observância comum | lei; norma}
  \end{Phonetics}
\end{Entry}

\begin{Entry}{规定}{8,8}{⾒、⼧}
  \begin{Phonetics}{规定}{gui1ding4}[][HSK 3]
    \definition[个,条,项,款]{s.}{regra; regulamento; estipulação; tomar decisões sobre a forma, o método, a quantidade ou a qualidade de algo}
    \definition{v.}{estipular; prover; prescrever; estabelecer requisitos ou restrições em termos de métodos, qualidade, quantidade, tempo, etc.}
  \end{Phonetics}
\end{Entry}

\begin{Entry}{规律}{8,9}{⾒、⼻}
  \begin{Phonetics}{规律}{gui1lv4}[][HSK 4]
    \definition{adj.}{estável; regular; coisas, comportamentos, fenômenos, etc. que ocorrem em um determinado momento}
    \definition{s.}{lei; padrão regular; conexão essencial e recorrente entre as coisas}
  \end{Phonetics}
\end{Entry}

\begin{Entry}{规矩}{8,9}{⾒、⽮}
  \begin{Phonetics}{规矩}{gui1ju5}[][HSK 7-9]
    \definition{adj.}{adequado; bem comportado; bem disciplinado; honesto e correto; de acordo com os padrões ou o senso comum}
    \definition[条,个,项]{s.}{regra; costume; prática estabelecida; certos padrões, regras ou costumes}
  \end{Phonetics}
\end{Entry}

\begin{Entry}{规范}{8,9}{⾒、⾋}
  \begin{Phonetics}{规范}{gui1fan4}[][HSK 3]
    \definition{adj.}{regular; normal; padrão; que atende às especificações; em conformidade com as normas}
    \definition{s.}{norma; padrão; diretriz}
    \definition{v.}{regular; padronizar; tornar conforme as normas}
  \end{Phonetics}
\end{Entry}

\begin{Entry}{规格}{8,10}{⾒、⽊}
  \begin{Phonetics}{规格}{gui1ge2}[][HSK 7-9]
    \definition[种]{s.}{normas; padrões; especificações; padrões de qualidade do produto, como determinados tamanho, peso, precisão, desempenho, etc. | formato; padrão; requisito; geralmente se refere a requisitos ou condições especificados}
  \end{Phonetics}
\end{Entry}

\begin{Entry}{规模}{8,14}{⾒、⽊}
  \begin{Phonetics}{规模}{gui1mo2}[][HSK 4]
    \definition[个,种]{s.}{escala; escopo; dimensões; padrão, forma ou escopo (de um empreendimento, instituição, projeto, movimento, etc.)}
  \end{Phonetics}
\end{Entry}

\begin{Entry}{视}{8}{⾒}
  \begin{Phonetics}{视}{shi4}
    \definition{v.}{olhar para | considerar; olhar para | inspecionar; observar}
  \end{Phonetics}
\end{Entry}

\begin{Entry}{视为}{8,4}{⾒、⼂}
  \begin{Phonetics}{视为}{shi4 wei2}[][HSK 5]
    \definition{v.}{considerar; ver como; considerar como; considerar ser; achar que é}
  \end{Phonetics}
\end{Entry}

\begin{Entry}{视角}{8,7}{⾒、⾓}
  \begin{Phonetics}{视角}{shi4jiao3}
    \definition{s.}{ângulo do qual se observa um objeto | (figurativo) perspectiva, ponto de vista, quadro de referência | (cinematografia) ângulo da câmera | (percepção visual) ângulo visual (o ângulo que um objeto visto subtende no olho) | (fotografia) ângulo de visão}
  \end{Phonetics}
\end{Entry}

\begin{Entry}{视频}{8,13}{⾒、⾴}
  \begin{Phonetics}{视频}{shi4pin2}[][HSK 5]
    \definition[个,段,条]{s.}{vídeo; videoclipe}
  \end{Phonetics}
\end{Entry}

\begin{Entry}{试}{8}{⾔}
  \begin{Phonetics}{试}{shi4}[][HSK 1]
    \definition{s.}{teste; exame; avaliação de conhecimentos ou habilidades através de métodos específicos}
    \definition{v.}{tentar; investigar resultados ou verificar a natureza, não se envolver formalmente (em determinada atividade)}
  \end{Phonetics}
\end{Entry}

\begin{Entry}{试卷}{8,8}{⾔、⼙}
  \begin{Phonetics}{试卷}{shi4juan4}[][HSK 4]
    \definition[分,张]{s.}{folha de teste; folha de exame; papel usado para escrever as respostas nos exames}
  \end{Phonetics}
\end{Entry}

\begin{Entry}{试图}{8,8}{⾔、⼞}
  \begin{Phonetics}{试图}{shi4tu2}[][HSK 5]
    \definition{v.}{tentar; pretender, fazer o possível para realizar algo}
  \end{Phonetics}
\end{Entry}

\begin{Entry}{试点}{8,9}{⾔、⽕}
  \begin{Phonetics}{试点}{shi4 dian3}[][HSK 6]
    \definition[个]{s.}{local onde um experimento é conduzido; unidade experimental; local de teste; um lugar para pequenos experimentos}
    \definition{v.}{experimentar; fazer experimentos; realizar testes em pontos selecionados; lançar um projeto piloto}
  \end{Phonetics}
\end{Entry}

\begin{Entry}{试验}{8,10}{⾔、⾺}
  \begin{Phonetics}{试验}{shi4yan4}[][HSK 3]
    \definition{v.}{testar; fazer um teste; fazer um experimento; para examinar o efeito ou desempenho de algo, primeiro experimente em um laboratório ou em uma escala menor}
  \end{Phonetics}
\end{Entry}

\begin{Entry}{试题}{8,15}{⾔、⾴}
  \begin{Phonetics}{试题}{shi4 ti2}[][HSK 3]
    \definition[道]{s.}{questões de um exame}
  \end{Phonetics}
\end{Entry}

\begin{Entry}{诗}{8}{⾔}
  \begin{Phonetics}{诗}{shi1}[][HSK 4]
    \definition[首,句,行]{s.}{poesia; verso; poema; um gênero literário que reflete a vida e expressa emoções por meio de uma linguagem rítmica e rimada}
  \seealsoref{诗经}{shi1jing1}
  \end{Phonetics}
\end{Entry}

\begin{Entry}{诗人}{8,2}{⾔、⼈}
  \begin{Phonetics}{诗人}{shi1 ren2}[][HSK 4]
    \definition[个,位,名,些]{s.}{poeta; escritor de poesia}
  \end{Phonetics}
\end{Entry}

\begin{Entry}{诗句}{8,5}{⾔、⼝}
  \begin{Phonetics}{诗句}{shi1ju4}
    \definition[行]{s.}{verso | versículo}
  \end{Phonetics}
\end{Entry}

\begin{Entry}{诗词}{8,7}{⾔、⾔}
  \begin{Phonetics}{诗词}{shi1ci2}
    \definition{s.}{verso}
  \end{Phonetics}
\end{Entry}

\begin{Entry}{诗经}{8,8}{⾔、⽷}
  \begin{Phonetics}{诗经}{shi1jing1}
    \definition*{s.}{Shijing, o Livro das Canções, antiga coleção de poemas chineses e um dos Cinco Clássicos do Confucionismo}
  \end{Phonetics}
\end{Entry}

\begin{Entry}{诗意}{8,13}{⾔、⼼}
  \begin{Phonetics}{诗意}{shi1yi4}
    \definition{adj.}{poético}
    \definition{s.}{poesia}
  \end{Phonetics}
\end{Entry}

\begin{Entry}{诗歌}{8,14}{⾔、⽋}
  \begin{Phonetics}{诗歌}{shi1 ge1}[][HSK 5]
    \definition[本,首,段]{s.}{poesia; poemas e canções; refere-se a todos os tipos de poesia}
  \end{Phonetics}
\end{Entry}

\begin{Entry}{诚}{8}{⾔}
  \begin{Phonetics}{诚}{cheng2}
    \definition{adj.}{sincero; honesto; verdadeiro}
    \definition{adv.}{na verdade; realmente; de fato}
    \definition{s.}{sinceridade; genuinidade; seriedade}
  \end{Phonetics}
\end{Entry}

\begin{Entry}{诚心诚意}{8,4,8,13}{⾔、⼼、⾔、⼼}
  \begin{Phonetics}{诚心诚意}{cheng2xin1-cheng2yi4}[][HSK 7-9]
    \definition{expr.}{sincero e sério; com toda a sinceridade; é uma expressão idiomática chinesa que vem da Biografia de Ma Yuan no Livro da Dinastia Han Posterior | genuíno; sincero}
  \end{Phonetics}
\end{Entry}

\begin{Entry}{诚实}{8,8}{⾔、⼧}
  \begin{Phonetics}{诚实}{cheng2shi2}[][HSK 4]
    \definition{adj.}{honesto; sincero e honesto, não hipócrita}
  \end{Phonetics}
\end{Entry}

\begin{Entry}{诚实地}{8,8,6}{⾔、⼧、⼟}
  \begin{Phonetics}{诚实地}{cheng2shi2 di4}
    \definition{adv.}{honestamente}
  \end{Phonetics}
\end{Entry}

\begin{Entry}{诚信}{8,9}{⾔、⼈}
  \begin{Phonetics}{诚信}{cheng2 xin4}[][HSK 4]
    \definition{adj.}{honesto e confiável}
    \definition[种]{s.}{fé; honestidade; padrão e princípio de comportamento: não contar mentiras, prometer aos outros o que eles podem fazer e ter a confiança dos outros}
  \end{Phonetics}
\end{Entry}

\begin{Entry}{诚恳}{8,10}{⾔、⼼}
  \begin{Phonetics}{诚恳}{cheng2ken3}[][HSK 7-9]
    \definition{adj.}{sincero; sério; a atitude é muito real e pé no chão}
  \end{Phonetics}
\end{Entry}

\begin{Entry}{诚挚}{8,10}{⾔、⼿}
  \begin{Phonetics}{诚挚}{cheng2zhi4}[][HSK 7-9]
    \definition{adj.}{sincero; cordial; honesto}
  \end{Phonetics}
\end{Entry}

\begin{Entry}{诚意}{8,13}{⾔、⼼}
  \begin{Phonetics}{诚意}{cheng2yi4}[][HSK 7-9]
    \definition{s.}{boa fé; sinceridade; intenções sinceras}
  \end{Phonetics}
\end{Entry}

\begin{Entry}{话}{8}{⾔}
  \begin{Phonetics}{话}{hua4}[][HSK 1]
    \definition[句,段,番,种]{s.}{palavra; conversa; a voz que expressa os pensamentos quando falada, ou os caracteres que registram essa voz}
    \definition{v.}{falar sobre; falar a respeito}
  \end{Phonetics}
\end{Entry}

\begin{Entry}{话剧}{8,10}{⾔、⼑}
  \begin{Phonetics}{话剧}{hua4 ju4}[][HSK 3]
    \definition[场,幕,部,出,台]{s.}{drama moderno; peça de teatro; peça teatral representada através de diálogos e ações}
  \end{Phonetics}
\end{Entry}

\begin{Entry}{话题}{8,15}{⾔、⾴}
  \begin{Phonetics}{话题}{hua4ti2}[][HSK 3]
    \definition[个,种,项]{s.}{assunto de uma palestra; tópico de uma conversa; o foco da conversa}
  \end{Phonetics}
\end{Entry}

\begin{Entry}{诞}{8}{⾔}
  \begin{Phonetics}{诞}{dan4}
    \definition{adj.}{absurdo; fantástico; irreal; irracional}
    \definition{adv.}{absurdamente; fantasticamente}
    \definition{s.}{aniversário de nascimento | nascimento}
    \definition{v.}{nascer | dar à luz}
  \end{Phonetics}
\end{Entry}

\begin{Entry}{诞生}{8,5}{⾔、⽣}
  \begin{Phonetics}{诞生}{dan4sheng1}[][HSK 6]
    \definition{v.}{nascer; vir a existir; uma pessoa nasce; também significa que algo novo surgiu e tem um impacto positivo na sociedade}
  \end{Phonetics}
\end{Entry}

\begin{Entry}{诞辰}{8,7}{⾔、⾠}
  \begin{Phonetics}{诞辰}{dan4chen2}[][HSK 7-9]
    \definition[周年]{s.}{aniversário (usado principalmente para pessoas respeitadas)}[9月28日是孔子诞辰日。===28 de setembro é o aniversário de Confúcio.]
  \end{Phonetics}
\end{Entry}

\begin{Entry}{诟}{8}{⾔}
  \begin{Phonetics}{诟}{gou4}
    \definition*{s.}{Sobrenome Gou}
    \definition{s.}{vergonha; humilhação}
    \definition{v.}{insultar; xingar; falar de forma abusiva}
  \end{Phonetics}
\end{Entry}

\begin{Entry}{诟骂}{8,9}{⾔、⾺}
  \begin{Phonetics}{诟骂}{gou4ma4}
    \definition{v.}{abusar verbalmente | insultar | criticar}
  \end{Phonetics}
\end{Entry}

\begin{Entry}{询}{8}{⾔}
  \begin{Phonetics}{询}{xun2}
    \definition{v.}{perguntar; indagar; reunir informações | consultar; buscar conselho}
  \end{Phonetics}
\end{Entry}

\begin{Entry}{询问}{8,6}{⾔、⾨}
  \begin{Phonetics}{询问}{xun2wen4}[][HSK 5]
    \definition{v.}{indagar; perguntar sobre; pedir conselho}
  \end{Phonetics}
\end{Entry}

\begin{Entry}{该}{8}{⾔}
  \begin{Phonetics}{该}{gai1}[][HSK 2,7-9]
    \definition{adj.}{completo; integral; abrangente; inclusivo; o mesmo que 赅}
    \definition{pron.}{isto; aquilo; o referido; o acima mencionado; indica a pessoa ou coisa mencionada acima, equivalente a 此, 这个, etc.}
    \definition{v.}{deveria ser; deveria ser assim | caber a alguém; ser a vez (ou dever) de alguém fazer algo | merecer; servir a alguém de direito; indica que algo deve ser feito | dever | deve; provavelmente irá; muito provavelmente; pode ser razoavelmente ou naturalmente esperado que; expressa uma conclusão lógica ou provável com base na razão ou na experiência}
    \definition{v.aux.}{usado em frases exclamativas, tem a função de reforçar o tom}
  \seealsoref{此}{ci3}
  \seealsoref{赅}{gai1}
  \seealsoref{这个}{zhe4ge5}
  \end{Phonetics}
\end{Entry}

\begin{Entry}{详}{8}{⾔}
  \begin{Phonetics}{详}{xiang2}
    \definition{adj.}{conhecido; reconhecido; saber claramente | detalhado; minucioso; pormenorizado (oposto a 略)}
    \definition{s.}{detalhes; particularidades}
    \definition{v.}{contar; explicar; elaborar | saber claramente}
  \seealsoref{略}{lve4}
  \end{Phonetics}
\end{Entry}

\begin{Entry}{详细}{8,8}{⾔、⽷}
  \begin{Phonetics}{详细}{xiang2xi4}[][HSK 5]
    \definition{adj.}{explícito; detalhado; minucioso; circunstancial; meticuloso}
  \end{Phonetics}
\end{Entry}

\begin{Entry}{诧}{8}{⾔}
  \begin{Phonetics}{诧}{cha4}
    \definition{v.}{ficar surpreso}
  \end{Phonetics}
\end{Entry}

\begin{Entry}{诧异}{8,6}{⾔、⼶}
  \begin{Phonetics}{诧异}{cha4yi4}[][HSK 7-9]
    \definition{v.}{ficar surpreso; ficar espantado}
  \end{Phonetics}
\end{Entry}

\begin{Entry}{责}{8}{⾙}
  \begin{Phonetics}{责}{ze2}
    \definition{s.}{dever; responsabilidade}
    \definition{v.}{exigir; requerer; exigir que algo seja feito ou que atenda a certos padrões | questionar atentamente; chamar alguém para prestar contas; interrogar| reprovar; culpar | punir}
  \end{Phonetics}
\end{Entry}

\begin{Entry}{责任}{8,6}{⾙、⼈}
  \begin{Phonetics}{责任}{ze2ren4}[][HSK 3]
    \definition[个,种,份]{s.}{dever; responsabilidade; de acordo com a profissão, cargo, identidade, etc., as coisas que você deve fazer ou as tarefas que deve assumir | culpa; responsabilidade por uma falha ou erro; não ter feito o que era sua obrigação e, portanto, ser responsável pela falha}
  \end{Phonetics}
\end{Entry}

\begin{Entry}{责怪}{8,8}{⾙、⼼}
  \begin{Phonetics}{责怪}{ze2guai4}
    \definition{v.}{repreender | censurar}
  \end{Phonetics}
\end{Entry}

\begin{Entry}{败}{8}{⾒}
  \begin{Phonetics}{败}{bai4}[][HSK 4]
    \definition{adj.}{ruim; deteriorado; murcho; dilapidado; decadente}
    \definition{v.}{ser derrotado; perder (oposto a 胜) | derrotar; bater | falha (oposto a 成) | estragar; arruinar | decair; murchar | quebrar; neutralizar; dissipar}
  \seealsoref{成}{cheng2}
  \seealsoref{胜}{sheng4}
  \end{Phonetics}
\end{Entry}

\begin{Entry}{账}{8}{⾙}
  \begin{Phonetics}{账}{zhang4}[][HSK 6]
    \definition[笔,本]{s.}{conta | livro de contas | dívida; conta | crédito (de dívidas)}
  \end{Phonetics}
\end{Entry}

\begin{Entry}{账户}{8,4}{⾙、⼾}
  \begin{Phonetics}{账户}{zhang4hu4}[][HSK 6]
    \definition[个]{s.}{conta; refere-se à classificação de vários usos de fundos, fontes e processos de rotatividade no livro contábil}
  \end{Phonetics}
\end{Entry}

\begin{Entry}{货}{8}{⾙}
  \begin{Phonetics}{货}{huo4}[][HSK 4]
    \definition{s.}{dinheiro; moeda | bens; mercadorias; \emph{commodity} | refere-se a uma pessoa com um certo mau caráter (usado como um insulto) | riqueza; fortuna; um termo geral para dinheiro, joias, tecidos, etc.}
    \definition{v.}{vender}
  \end{Phonetics}
\end{Entry}

\begin{Entry}{货车}{8,4}{⾙、⾞}
  \begin{Phonetics}{货车}{huo4che1}
    \definition{s.}{caminhão | van | vagão de carga}
  \end{Phonetics}
\end{Entry}

\begin{Entry}{质}{8}{⾙}
  \begin{Phonetics}{质}{zhi4}
    \definition*{s.}{Sobrenome Zhi}
    \definition{adj.}{simples; claro; sem adornos}
    \definition{s.}{natureza; caráter; essência; substância | qualidade | matéria; substância | segurança; penhor; garantia}
    \definition{v.}{penhorar | hipotecar | questionar; chamar à responsabilidade; acusar}
  \end{Phonetics}
\end{Entry}

\begin{Entry}{质量}{8,12}{⾙、⾥}
  \begin{Phonetics}{质量}{zhi4liang4}[][HSK 4]
    \definition{s.}{qualidade; o quão bom ou ruim é o produto ou o trabalho | Física: massa}
  \end{Phonetics}
\end{Entry}

\begin{Entry}{贩}{8}{⾙}
  \begin{Phonetics}{贩}{fan4}
    \definition[个]{s.}{comerciante; mascate; negociante; vendedor ambulante}
    \definition{v.}{(comerciantes) comprar para revender}
  \end{Phonetics}
\end{Entry}

\begin{Entry}{贩卖}{8,8}{⾙、⼗}
  \begin{Phonetics}{贩卖}{fan4mai4}[][HSK 7-9]
    \definition{v.}{vender; traficar}
  \end{Phonetics}
\end{Entry}

\begin{Entry}{贪}{8}{⾙}
  \begin{Phonetics}{贪}{tan1}
    \definition{adj.}{corrupto; venal | ganancioso; avarento; ambicioso}
    \definition{v.}{apropriar-se indevidamente; praticar corrupção; ser corrupto | ter um desejo insaciável por; ter um desejo voraz por | cobiçar; ansiar por; ser ganancioso por}
  \end{Phonetics}
\end{Entry}

\begin{Entry}{贪婪}{8,11}{⾙、⼥}
  \begin{Phonetics}{贪婪}{tan1lan2}
    \definition{adj.}{avaro | ambicioso | voraz | insaciável}
  \end{Phonetics}
\end{Entry}

\begin{Entry}{贫}{8}{⾙}
  \begin{Phonetics}{贫}{pin2}
    \definition{adj.}{pobre; empobrecido | inadequado; deficiente; insuficiente | tagarela; loquaz; falante; chato e irritante}
  \end{Phonetics}
\end{Entry}

\begin{Entry}{贫民窟}{8,5,13}{⾙、⽒、⽳}
  \begin{Phonetics}{贫民窟}{pin2min2ku1}
    \definition{s.}{favela}
  \end{Phonetics}
\end{Entry}

\begin{Entry}{贫困}{8,7}{⾙、⼞}
  \begin{Phonetics}{贫困}{pin2kun4}[][HSK 6]
    \definition{adj.}{pobre; indigente; necessitado; empobrecido; assolado pela pobreza; em circunstâncias difíceis}
  \end{Phonetics}
\end{Entry}

\begin{Entry}{贬}{8}{⾙}
  \begin{Phonetics}{贬}{bian3}
    \definition{adj.}{depreciativo; derrogativo; rebaixante}
    \definition{v.}{(nos tempos antigos) rebaixar de posição; (nos tempos modernos) diminuir de valor | reduzir valor; desvalorizar | censurar; menosprezar; depreciar; dar uma avaliação ruim | degradar; rebaixar; relegar}
  \end{Phonetics}
\end{Entry}

\begin{Entry}{贬值}{8,10}{⾙、⼈}
  \begin{Phonetics}{贬值}{bian3zhi2}[][HSK 7-9]
    \definition{v.}{depreciar; tornar-se desvalorizado; refere-se à diminuição do poder de compra do dinheiro | depreciar; geralmente se refere à diminuição do valor de algo | desvalorizar; reduzir o teor de ouro da moeda de um país ou reduzir a taxa de câmbio da moeda de um país em relação às moedas estrangeiras}
  \end{Phonetics}
\end{Entry}

\begin{Entry}{购}{8}{⾙}
  \begin{Phonetics}{购}{gou4}[][HSK 7-9]
    \definition{v.}{comprar}
  \end{Phonetics}
\end{Entry}

\begin{Entry}{购买}{8,6}{⾙、⼄}
  \begin{Phonetics}{购买}{gou4 mai3}[][HSK 4]
    \definition{v.}{comprar; adquirir; usar dinheiro para obter itens}
  \end{Phonetics}
\end{Entry}

\begin{Entry}{购物}{8,8}{⾙、⽜}
  \begin{Phonetics}{购物}{gou4wu4}[][HSK 4]
    \definition{s.}{compras; itens comprados; \emph{shopping}}
    \definition{v.}{ir às compras; fazer compras}
  \end{Phonetics}
\end{Entry}

\begin{Entry}{贯}{8}{⾙}
  \begin{Phonetics}{贯}{guan4}
    \definition*{s.}{Sobrenome Guan}
    \definition{clas.}{uma sequência de 1.000 em dinheiro; antigamente, o dinheiro era amarrado com cordas, e cada mil moedas era uma corda.}
    \definition{s.}{lugar nativo; local de nascimento; lugar do lar ancestral; lugar onde gerações viveram | Literário: exemplo; instância; caso; precedente | Arcaico: guan (uma corda de 1.000 moedas de cobre); corda para amarrar dinheiro nos tempos antigos}
    \definition{v.}{passar através de; perfurar; enfiar; penetrar | estar ligados entre si; seguir em linha contínua; estar conectado | Literário: comparecer}
  \end{Phonetics}
\end{Entry}

\begin{Entry}{贯彻}{8,7}{⾙、⼻}
  \begin{Phonetics}{贯彻}{guan4che4}[][HSK 7-9]
    \definition{v.}{executar; implementar; pôr em prática; realizar ou incorporar completamente (diretrizes, políticas, espírito, etc.)}
  \end{Phonetics}
\end{Entry}

\begin{Entry}{贯穿}{8,9}{⾙、⽳}
  \begin{Phonetics}{贯穿}{guan4chuan1}[][HSK 7-9]
    \definition{v.}{cruzar; conectar; penetrar; correr através; passar através | permear; estar cheio de}
  \end{Phonetics}
\end{Entry}

\begin{Entry}{贯通}{8,10}{⾙、⾡}
  \begin{Phonetics}{贯通}{guan4tong1}[][HSK 7-9]
    \definition{v.}{ter um conhecimento profundo de; ser bem versado (em); (acadêmico, ideológico, etc.) ter compreensão completa | ligar; encadear}
  \end{Phonetics}
\end{Entry}

\begin{Entry}{转}{8}{⾞}
  \begin{Phonetics}{转}{zhuai3}
  \end{Phonetics}
  \begin{Phonetics}{转}{zhuan3}
    \definition{v.}{mudar; deslocar; transferir; virar; mudar de direção, posição, situação, circunstâncias, etc. | transmitir; transferir; passar adiante}
  \end{Phonetics}
  \begin{Phonetics}{转}{zhuan4}[][HSK 3,6]
    \definition{clas.}{usado para rotações (por minuto, por segundo, etc.): RPM}
    \definition{v.}{girar; rodar; revolver; movimento em torno de um centro | passear; dar uma volta}
  \end{Phonetics}
\end{Entry}

\begin{Entry}{转化}{8,4}{⾞、⼔}
  \begin{Phonetics}{转化}{zhuan3 hua4}[][HSK 5]
    \definition{v.}{mudar; transformar | inverter; converter}
  \end{Phonetics}
\end{Entry}

\begin{Entry}{转让}{8,5}{⾞、⾔}
  \begin{Phonetics}{转让}{zhuan3rang4}[][HSK 5]
    \definition{v.}{ceder; fazer a entrega; transferir a posse de; ceder seus bens ou direitos a outra pessoa}
  \end{Phonetics}
\end{Entry}

\begin{Entry}{转产}{8,6}{⾞、⼇}
  \begin{Phonetics}{转产}{zhuan3chan3}
    \definition{v.}{mudar a produção | mudar para novos produtos}
  \end{Phonetics}
\end{Entry}

\begin{Entry}{转动}{8,6}{⾞、⼒}
  \begin{Phonetics}{转动}{zhuan3 dong4}[][HSK 4]
    \definition{v.}{girar; rodar; dar voltas; torcer | dar a volta em algo}
  \end{Phonetics}
  \begin{Phonetics}{转动}{zhuan4 dong4}[][HSK 6]
    \definition{s.}{tambor; roda}
    \definition{v.}{girar; correr; rolar; revolver; rotacionar; torcer}
  \end{Phonetics}
\end{Entry}

\begin{Entry}{转向}{8,6}{⾞、⼝}
  \begin{Phonetics}{转向}{zhuan3 xiang4}[][HSK 5]
    \definition{v.}{desviar; desviar-se; mudar a direção do avanço | mudar a posição política de alguém | mudar de direção; virar-se para (a outra parte)}
  \end{Phonetics}
  \begin{Phonetics}{转向}{zhuan4/xiang4}
    \definition{v.+compl.}{perder-se; perder o rumo; não consiguir distinguir a direção; estar perdido}
  \end{Phonetics}
\end{Entry}

\begin{Entry}{转告}{8,7}{⾞、⼝}
  \begin{Phonetics}{转告}{zhuan3gao4}[][HSK 4]
    \definition{v.}{passar adiante; comunicar; transmitir; ser instruído a dizer a outra parte o que uma pessoa diz, o que está acontecendo, etc.}
  \end{Phonetics}
\end{Entry}

\begin{Entry}{转身}{8,7}{⾞、⾝}
  \begin{Phonetics}{转身}{zhuan3 shen1}[][HSK 4]
    \definition{adv.}{em um instante; em um piscar de olhos}
    \definition{v.}{dar a volta; dar meia-volta; dar a volta por cima | virar; girar; refere-se a uma mudança de direção, localização, natureza, etc.}
  \end{Phonetics}
\end{Entry}

\begin{Entry}{转变}{8,8}{⾞、⼜}
  \begin{Phonetics}{转变}{zhuan3bian4}[][HSK 3]
    \definition{v.}{mudar; converter; transformar}
  \end{Phonetics}
\end{Entry}

\begin{Entry}{转念}{8,8}{⾞、⼼}
  \begin{Phonetics}{转念}{zhuan3nian4}
    \definition{v.}{ter dúvidas sobre algo | pensar melhor}
  \end{Phonetics}
\end{Entry}

\begin{Entry}{转账}{8,8}{⾞、⾙}
  \begin{Phonetics}{转账}{zhuan3/zhang4}
    \definition{v.+compl.}{transferir entre contas | trazer à frente | incluir uma soma de dinheiro do balanço anterior no seguinte}
  \end{Phonetics}
\end{Entry}

\begin{Entry}{转弯}{8,9}{⾞、⼸}
  \begin{Phonetics}{转弯}{zhuan4 wan1}[][HSK 4]
    \definition{s.}{esquina; curva}[小心急转弯。===Tenha cuidado em curvas fechadas.]
    \definition{v.}{rodar; desviar; virar uma esquina; fazer uma curva; fazer uma curva de 180º}
  \end{Phonetics}
\end{Entry}

\begin{Entry}{转换}{8,10}{⾞、⼿}
  \begin{Phonetics}{转换}{zhuan3 huan4}[][HSK 5]
    \definition{v.}{mudar; trocar; converter; transformar; alterar}
  \end{Phonetics}
\end{Entry}

\begin{Entry}{转递}{8,10}{⾞、⾡}
  \begin{Phonetics}{转递}{zhuan3di4}
    \definition{v.}{passar | retransmitir}
  \end{Phonetics}
\end{Entry}

\begin{Entry}{转悠}{8,11}{⾞、⼼}
  \begin{Phonetics}{转悠}{zhuan4you5}
    \definition{v.}{aparecer repetidamente | rolar | passear por aí}
  \end{Phonetics}
\end{Entry}

\begin{Entry}{转移}{8,11}{⾞、⽲}
  \begin{Phonetics}{转移}{zhuan3yi2}[][HSK 4]
    \definition{v.}{deslocar; desviar; transferir; redirecionar; reposicionar; reorientar | mudar; transformar}
  \end{Phonetics}
\end{Entry}

\begin{Entry}{转游}{8,12}{⾞、⽔}
  \begin{Phonetics}{转游}{zhuan4you5}
  \seealsoref{转悠}{zhuan4you5}
  \end{Phonetics}
\end{Entry}

\begin{Entry}{轮}{8}{⾞}
  \begin{Phonetics}{轮}{lun2}[][HSK 4]
    \definition{clas.}{usado para sol vermelho, lua brilhante, etc. | usado para rodadas | doze anos de idade (os doze ramos terrestres são usados para lembrar o gênero humano e cada doze anos de idade é um ciclo)}
    \definition{s.}{roda | anel; disco; objeto semelhante a uma roda | navio a vapor; barco a vapor}
    \definition{v.}{revezar; substituir um ao outro em sequência (para fazer algo)}
  \end{Phonetics}
\end{Entry}

\begin{Entry}{轮子}{8,3}{⾞、⼦}
  \begin{Phonetics}{轮子}{lun2 zi5}[][HSK 4]
    \definition[个,只]{s.}{roda; peças circulares de veículos ou máquinas com capacidade de rotação}
  \end{Phonetics}
\end{Entry}

\begin{Entry}{轮回}{8,6}{⾞、⼞}
  \begin{Phonetics}{轮回}{lun2hui2}
    \definition[个]{s.}{reencarnação (Budismo) | ciclo}
    \definition{v.}{reencarnar}
  \end{Phonetics}
\end{Entry}

\begin{Entry}{轮船}{8,11}{⾞、⾈}
  \begin{Phonetics}{轮船}{lun2chuan2}[][HSK 4]
    \definition[艘,班]{s.}{vapor; navio a vapor; barco a vapor}
  \end{Phonetics}
\end{Entry}

\begin{Entry}{轮椅}{8,12}{⾞、⽊}
  \begin{Phonetics}{轮椅}{lun2 yi3}[][HSK 4]
    \definition{s.}{cadeira de rodas; dispositivo de assento especialmente projetado com rodas para pessoas com dificuldade de locomoção, que pode ser acionado por um disco de roda ou manivela operados manualmente}
  \end{Phonetics}
\end{Entry}

\begin{Entry}{软}{8}{⾞}
  \begin{Phonetics}{软}{ruan3}[][HSK 5]
    \definition*{s.}{Sobrenome Ruan}
    \definition{adj.}{macio; flexível; maleável; maleável (oposto de 硬) | suave; brando; delicado | fraco; débil | de baixa qualidade, capacidade, etc. | facilmente movido (ou influenciado) | de maneira suave (ou gentil) | indulgente; tolerante | maleável; flexível | fácil de se emocionar ou abalar}
  \seealsoref{硬}{ying4}
  \end{Phonetics}
\end{Entry}

\begin{Entry}{软件}{8,6}{⾞、⼈}
  \begin{Phonetics}{软件}{ruan3jian4}[][HSK 5]
    \definition[款,个]{s.}{\emph{software}; programas de computador, procedimentos, regras e quaisquer arquivos, documentos e dados relacionados à operação do sistema de computador}
  \end{Phonetics}
\end{Entry}

\begin{Entry}{轰}{8}{⾞}
  \begin{Phonetics}{轰}{hong1}
    \definition{interj.}{(onomatopéia) Bum!; estrondo; refere-se aos ruídos altos feitos por trovões, fogo de artilharia, etc.}
    \definition{v.}{retumbar; bombardear; explodir | espantar; expulsar}
  \end{Phonetics}
\end{Entry}

\begin{Entry}{轰鸣}{8,8}{⾞、⿃}
  \begin{Phonetics}{轰鸣}{hong1ming2}
    \definition{s.}{bum (som de explosão) | estrondo}
  \end{Phonetics}
\end{Entry}

\begin{Entry}{轰炸机}{8,9,6}{⾞、⽕、⽊}
  \begin{Phonetics}{轰炸机}{hong1zha4ji1}
    \definition{s.}{bombardeiro (aeronave)}
  \end{Phonetics}
\end{Entry}

\begin{Entry}{迫}{8}{⾡}
  \begin{Phonetics}{迫}{po4}
    \definition{adj.}{urgente; premente}
    \definition{s.}{morteiro; artilharia}
    \definition{v.}{compelir; forçar; pressionar | aproximar-se; ir em direção a (ou perto de)}
  \end{Phonetics}
\end{Entry}

\begin{Entry}{迫切}{8,4}{⾡、⼑}
  \begin{Phonetics}{迫切}{po4qie4}[][HSK 4]
    \definition{adj.}{urgente; premente; muito ansiosamente, a ponto de ser difícil esperar}
  \end{Phonetics}
\end{Entry}

\begin{Entry}{迭}{8}{⾡}
  \begin{Phonetics}{迭}{die2}
    \definition*{s.}{Sobrenome Die}
    \definition{adv.}{repetidamente; de ​​novo e de novo | a tempo para}
    \definition{v.}{alternar; mudar; revezar-se; substituir}
  \end{Phonetics}
\end{Entry}

\begin{Entry}{迭起}{8,10}{⾡、⾛}
  \begin{Phonetics}{迭起}{die2qi3}[][HSK 7-9]
    \definition{v.}{ocorrer repetidamente; acontecer com frequência | surgir repetidamente}
  \end{Phonetics}
\end{Entry}

\begin{Entry}{郁}{8}{⾢}
  \begin{Phonetics}{郁}{yu4}
    \definition*{s.}{Sobrenome Yu}
    \definition{adj.}{fortemente perfumado | luxuriante; exuberante | sombrio; deprimido}
  \end{Phonetics}
\end{Entry}

\begin{Entry}{郁郁葱葱}{8,8,12,12}{⾢、⾢、⾋、⾋}
  \begin{Phonetics}{郁郁葱葱}{yu4yu4cong1cong1}
    \definition{adj.}{exuberante e verde}
    \definition{expr.}{verdejante e exuberante; uma profusão selvagem de vegetação; luxuriantemente verde; ela cresce mais verde e mais fresca}
  \end{Phonetics}
\end{Entry}

\begin{Entry}{郊}{8}{⾢}
  \begin{Phonetics}{郊}{jiao1}
    \definition*{s.}{Sobrenome Jiao}
    \definition{s.}{subúrbios; periferias; áreas ao redor da cidade}
  \end{Phonetics}
\end{Entry}

\begin{Entry}{郊区}{8,4}{⾢、⼖}
  \begin{Phonetics}{郊区}{jiao1 qu1}[][HSK 5]
    \definition[个,片,块]{s.}{subúrbios; arredores; periferia; área ao redor da cidade que está administrativamente sob a jurisdição da cidade}
  \end{Phonetics}
\end{Entry}

\begin{Entry}{采}{8}{⾤}
  \begin{Phonetics}{采}{cai3}[][HSK 7-9]
    \definition*{s.}{Sobrenome Cai}
    \definition{s.}{espírito; tez; cor e expressão facial | cores}
    \definition{v.}{escolher; arrancar; reunir; colher (flores, folhas, frutas) | minerar; extrair | reunir; coletar | adotar; pegar; selecionar}
  \end{Phonetics}
  \begin{Phonetics}{采}{cai4}
    \definition{s.}{atribuição a um nobre feudal; a terra (incluindo os escravos que cultivavam a terra) concedida pelos antigos príncipes aos nobres; também chamada de feudo}
  \end{Phonetics}
\end{Entry}

\begin{Entry}{采用}{8,5}{⾤、⽤}
  \begin{Phonetics}{采用}{cai3 yong4}[][HSK 3]
    \definition{v.}{selecionar e usar; adotar; considerar adequado e utilizar}
  \end{Phonetics}
\end{Entry}

\begin{Entry}{采访}{8,6}{⾤、⾔}
  \begin{Phonetics}{采访}{cai3fang3}[][HSK 4]
    \definition{s.}{cobertura; entrevista; coleta de notícias; entrevistas, pesquisas, gravações de áudio e vídeo, etc., com o objetivo de coletar os materiais necessários}
    \definition{v.}{cobrir; entrevistar; reunir novas informações}
  \end{Phonetics}
\end{Entry}

\begin{Entry}{采纳}{8,7}{⾤、⽷}
  \begin{Phonetics}{采纳}{cai3na4}[][HSK 6]
    \definition{v.}{aceitar; adotar; tomar (opiniões, sugestões, solicitações, etc.)}
  \end{Phonetics}
\end{Entry}

\begin{Entry}{采取}{8,8}{⾤、⼜}
  \begin{Phonetics}{采取}{cai3qu3}[][HSK 3]
    \definition{v.}{adotar; escolha da implementação (diretrizes, políticas, métodos, ações, etc.) | reunir; coletar; tomar; assumir}
  \end{Phonetics}
\end{Entry}

\begin{Entry}{采矿}{8,8}{⾤、⽯}
  \begin{Phonetics}{采矿}{cai3/kuang4}[][HSK 7-9]
    \definition{s.}{mina}
    \definition{v.+compl.}{minerar; extrair minerais}
  \end{Phonetics}
\end{Entry}

\begin{Entry}{采购}{8,8}{⾤、⾙}
  \begin{Phonetics}{采购}{cai3gou4}[][HSK 5]
    \definition[名]{s.}{comprador; responsável pelas compras}
    \definition{v.}{adquirir; comprar; fazer compras para uma organização; fazer compras para uma empresa}
  \end{Phonetics}
\end{Entry}

\begin{Entry}{采集}{8,12}{⾤、⾫}
  \begin{Phonetics}{采集}{cai3ji2}[][HSK 7-9]
    \definition{v.}{reunir; coletar}
  \end{Phonetics}
\end{Entry}

\begin{Entry}{金}{8}{⾦}[Kangxi 167]
  \begin{Phonetics}{金}{jin1}[][HSK 3]
    \definition*{s.}{Dinastia Jin (1115-1234) | Sobrenome Jin}
    \definition{adj.}{dourado | altamente respeitado; precioso. metáfora de nobreza}
    \definition[锭,块]{s.}{ouro | metal | dinheiro | instrumento antigo de percussão de metal}
  \end{Phonetics}
\end{Entry}

\begin{Entry}{金子}{8,3}{⾦、⼦}
  \begin{Phonetics}{金子}{jin1zi5}
    \definition{s.}{ouro; elemento metálico, símbolo Au (aurum) amarelo-avermelhado, macio, dúctil, quimicamente estável é um metal precioso, usado para fabricar dinheiro, ornamentos etc.}
  \end{Phonetics}
\end{Entry}

\begin{Entry}{金刚石}{8,6,5}{⾦、⼑、⽯}
  \begin{Phonetics}{金刚石}{jin1gang1shi2}
    \definition{s.}{diamante, também chamado de 钻石}[金刚石比什么金属都硬。===O diamante é mais duro que qualquer metal.]
  \seealsoref{钻石}{zuan4shi2}
  \end{Phonetics}
\end{Entry}

\begin{Entry}{金色}{8,6}{⾦、⾊}
  \begin{Phonetics}{金色}{jin1 se4}
    \definition{s.}{cor ouro; dourado}
  \end{Phonetics}
\end{Entry}

\begin{Entry}{金钱}{8,10}{⾦、⾦}
  \begin{Phonetics}{金钱}{jin1 qian2}[][HSK 6]
    \definition[沓,笔,堆]{s.}{dinheiro; moeda}
  \end{Phonetics}
\end{Entry}

\begin{Entry}{金牌}{8,12}{⾦、⽚}
  \begin{Phonetics}{金牌}{jin1 pai2}[][HSK 3]
    \definition[枚]{s.}{medalha de ouro; refere-se à medalha conquistada pelo campeão em uma competição esportiva | ficha de ouro; placa de ouro usada como símbolo}
  \end{Phonetics}
\end{Entry}

\begin{Entry}{金额}{8,15}{⾦、⾴}
  \begin{Phonetics}{金额}{jin1 e2}[][HSK 6]
    \definition[份,笔]{s.}{quantidade de dinheiro; soma de dinheiro}
  \end{Phonetics}
\end{Entry}

\begin{Entry}{金融}{8,16}{⾦、⿀}
  \begin{Phonetics}{金融}{jin1rong2}[][HSK 6]
    \definition{s.}{finanças; serviços bancários; refere-se a atividades econômicas como a emissão, circulação e retirada de moeda, a concessão e retirada de empréstimos, o depósito e retirada de depósitos e transações de câmbio}
  \end{Phonetics}
\end{Entry}

\begin{Entry}{钓}{8}{⾦}
  \begin{Phonetics}{钓}{diao4}
    \definition{v.}{pescar com anzol e linha | buscar (fama e ganho pessoal) | fisgar; defraudar por meios desleais}
  \end{Phonetics}
\end{Entry}

\begin{Entry}{钓鱼}{8,8}{⾦、⿂}
  \begin{Phonetics}{钓鱼}{diao4yu2}[][HSK 7-9]
    \definition{v.}{pescar; ir pescar; atividade de captura de peixes com equipamentos de pesca na beira da água, que é uma forma de lazer e entretenimento | Figurativo: aprisionar; Internet: 钓鱼 significa que alguém publica deliberadamente algo que pode causar controvérsia, raiva ou outras emoções fortes, a fim de atrair pessoas para responder e discutir}
  \end{Phonetics}
\end{Entry}

\begin{Entry}{闸}{8}{⾨}
  \begin{Phonetics}{闸}{zha2}
    \definition[个,道]{s.}{comporta; comporta | freio | (coloquial) interruptor}
    \definition{v.}{represar um córrego, rio, etc. | represar a água; parar a água}
  \end{Phonetics}
\end{Entry}

\begin{Entry}{闸门}{8,3}{⾨、⾨}
  \begin{Phonetics}{闸门}{zha2men2}
    \definition{s.}{eclusa | comporta}
  \end{Phonetics}
\end{Entry}

\begin{Entry}{闹}{8}{⾾}
  \begin{Phonetics}{闹}{nao4}[][HSK 4]
    \definition{adj.}{barulhento}
    \definition{v.}{fazer barulho; provocar problemas | dar vazão (à sua raiva, ressentimento, etc.) | sofrer de; ser incomodado por; ocorrer (um desastre ou coisa ruim) | fazer;  entrar em ação | agitar; perturbar | brincar; fazer bagunça}
  \end{Phonetics}
\end{Entry}

\begin{Entry}{闹钟}{8,9}{⾾、⾦}
  \begin{Phonetics}{闹钟}{nao4 zhong1}[][HSK 4]
    \definition[个,台,只,款]{s.}{despertador; relógios capazes de tocar alarmes em horários predeterminados}
  \end{Phonetics}
\end{Entry}

\begin{Entry}{降}{8}{⾩}
  \begin{Phonetics}{降}{jiang4}[][HSK 4]
    \definition*{s.}{Sobrenome Jiang}
    \definition{v.}{cair; descer; quedar-se (oposto de 升 ) | diminuir; reduzir; cair; abaixar | nascer}
  \seealsoref{升}{sheng1}
  \end{Phonetics}
\end{Entry}

\begin{Entry}{降价}{8,6}{⾩、⼈}
  \begin{Phonetics}{降价}{jiang4 jia4}[][HSK 4]
    \definition{v.}{ficar mais barato; cortar o preço; reduzir o preço}
  \end{Phonetics}
\end{Entry}

\begin{Entry}{降低}{8,7}{⾩、⼈}
  \begin{Phonetics}{降低}{jiang4di1}[][HSK 4]
    \definition{v.}{reduzir; cortar; diminuir; rebaixar; cair; abaixar}
  \end{Phonetics}
\end{Entry}

\begin{Entry}{降温}{8,12}{⾩、⽔}
  \begin{Phonetics}{降温}{jiang4 wen1}[][HSK 4]
    \definition{v.}{baixar a temperatura (como em uma oficina);  recusar | cair a temperatura | esfriar; resfriar; metáfora para um declínio no entusiasmo ou uma diminuição no ímpeto de algo}
  \end{Phonetics}
\end{Entry}

\begin{Entry}{降落}{8,12}{⾩、⾋}
  \begin{Phonetics}{降落}{jiang4luo4}[][HSK 4]
    \definition{v.}{aterrissar; descer; descer do céu}
  \end{Phonetics}
\end{Entry}

\begin{Entry}{限}{8}{⾩}
  \begin{Phonetics}{限}{xian4}
    \definition{s.}{limite | limiar}
    \definition{v.}{definir um limite; limitar; restringir}
  \end{Phonetics}
\end{Entry}

\begin{Entry}{限制}{8,8}{⾩、⼑}
  \begin{Phonetics}{限制}{xian4zhi4}[][HSK 4]
    \definition[些]{s.}{limite; restrição; confinamento}
    \definition{v.}{limitar; adstringir; restringir; confinar; fechar em (sobre)}
  \end{Phonetics}
\end{Entry}

\begin{Entry}{隶}{8}{⾪}[Kangxi 171]
  \begin{Phonetics}{隶}{li4}
    \definition*{s.}{Sobrenome Li}
    \definition{s.}{escravo; pessoa em servidão; pessoas escravizadas | Arcaico: corredor de cargo governamental na China feudal | um dos estilos antigos da caligrafia chinesa}
    \definition{v.}{estar subordinado a; estar afiliado a (ou com)}
  \end{Phonetics}
\end{Entry}

\begin{Entry}{雨}{8}{⾬}[Kangxi 173]
  \begin{Phonetics}{雨}{yu3}[][HSK 1]
    \definition*{s.}{Sobrenome Yu}
    \definition[场,阵,滴]{s.}{chuva; água que cai das nuvens para o solo}
  \end{Phonetics}
  \begin{Phonetics}{雨}{yu4}
    \definition{v.}{cair (chuva, neve, etc.) | precipitar | chover | molhar}
  \end{Phonetics}
\end{Entry}

\begin{Entry}{雨水}{8,4}{⾬、⽔}
  \begin{Phonetics}{雨水}{yu3 shui3}[][HSK 5]
    \definition{s.}{água da chuva; precipitação; chuva; água proveniente da chuva}
  \end{Phonetics}
\end{Entry}

\begin{Entry}{雨伞}{8,6}{⾬、⼈}
  \begin{Phonetics}{雨伞}{yu3san3}
    \definition[把]{s.}{guarda-chuva}
  \end{Phonetics}
\end{Entry}

\begin{Entry}{雨衣}{8,6}{⾬、⾐}
  \begin{Phonetics}{雨衣}{yu3 yi1}[][HSK 6]
    \definition[件,个]{s.}{capa de chuva; jaqueta impermeável; roupas impermeáveis}
  \end{Phonetics}
\end{Entry}

\begin{Entry}{雨蚀}{8,9}{⾬、⾷}
  \begin{Phonetics}{雨蚀}{yu3shi2}
    \definition{s.}{erosão da chuva}
  \end{Phonetics}
\end{Entry}

\begin{Entry}{雨靴}{8,13}{⾬、⾰}
  \begin{Phonetics}{雨靴}{yu3xue1}
    \definition[双]{s.}{botas de chuva}
  \end{Phonetics}
\end{Entry}

\begin{Entry}{青}{8}{⾭}[Kangxi 174]
  \begin{Phonetics}{青}{qing1}[][HSK 5]
    \definition*{s.}{Província de Qinghai, abreviação de 青海 | Sobrenome Qing}
    \definition{adj.}{azul ou verde | preto | jovens (pessoas)}
    \definition{s.}{grama verde | colheitas jovens (não maduras) | tiras de bambu verde}
  \seealsoref{青海}{qing1hai3}
  \end{Phonetics}
\end{Entry}

\begin{Entry}{青天}{8,4}{⾭、⼤}
  \begin{Phonetics}{青天}{qing1tian1}
    \definition{s.}{céu claro, limpo ou azul}
  \end{Phonetics}
\end{Entry}

\begin{Entry}{青少年}{8,4,6}{⾭、⼩、⼲}
  \begin{Phonetics}{青少年}{qing1shao4nian2}[][HSK 2]
    \definition[位,名,个,些]{s.}{adolescentes}
  \end{Phonetics}
\end{Entry}

\begin{Entry}{青玉米}{8,5,6}{⾭、⽟、⽶}
  \begin{Phonetics}{青玉米}{qing1yu4mi3}
    \definition{s.}{milho verde}
  \end{Phonetics}
\end{Entry}

\begin{Entry}{青年}{8,6}{⾭、⼲}
  \begin{Phonetics}{青年}{qing1 nian2}[][HSK 2]
    \definition[个,位,名,些]{s.}{juventude; jovem; refere-se ao período entre os 15 e os 30 anos de idade.}
  \end{Phonetics}
\end{Entry}

\begin{Entry}{青年节}{8,6,5}{⾭、⼲、⾋}
  \begin{Phonetics}{青年节}{qing1nian2jie2}
    \definition*{s.}{Dia da Juventude (4 de maio)}
  \end{Phonetics}
\end{Entry}

\begin{Entry}{青春}{8,9}{⾭、⽇}
  \begin{Phonetics}{青春}{qing1chun1}[][HSK 4]
    \definition[个]{s.}{juventude; jovialidade}
  \end{Phonetics}
\end{Entry}

\begin{Entry}{青海}{8,10}{⾭、⽔}
  \begin{Phonetics}{青海}{qing1hai3}
    \definition*{s.}{Província de Qinghai}
  \end{Phonetics}
\end{Entry}

\begin{Entry}{青菜}{8,11}{⾭、⾋}
  \begin{Phonetics}{青菜}{qing1cai4}
    \definition{s.}{verduras}
  \end{Phonetics}
\end{Entry}

\begin{Entry}{青铜}{8,11}{⾭、⾦}
  \begin{Phonetics}{青铜}{qing1tong2}
    \definition{s.}{bronze (liga de cobre, 銅, e estanho, 锡)}
  \end{Phonetics}
\end{Entry}

\begin{Entry}{青椒}{8,12}{⾭、⽊}
  \begin{Phonetics}{青椒}{qing1jiao1}
    \definition{s.}{pimenta verde}
  \end{Phonetics}
\end{Entry}

\begin{Entry}{青蛙}{8,12}{⾭、⾍}
  \begin{Phonetics}{青蛙}{qing1wa1}
    \definition{adj.}{(gíria velha) cara feio}
    \definition[只]{s.}{sapo}
  \end{Phonetics}
\end{Entry}

\begin{Entry}{非}{8}{⾮}[Kangxi 175]
  \begin{Phonetics}{非}{fei1}[][HSK 4]
    \definition*{s.}{África, abreviação de 非洲 | Sobrenome Fei}
    \definition{adv.}{Em resposta a 不, indica necessidade (deve)}
    \definition{pref.}{indicando negatividade ou exclusão}
    \definition{s.}{engano; erro}
    \definition{v.}{opor-se a; culpar; censurar | não estar em conformidade com; ser contrário a | não ser | ter que; simplesmente precisar (fazer algo)}
  \seealsoref{不}{bu4}
  \seealsoref{非洲}{fei1zhou1}
  \end{Phonetics}
\end{Entry}

\begin{Entry}{非凡}{8,3}{⾮、⼏}
  \begin{Phonetics}{非凡}{fei1fan2}[][HSK 7-9]
    \definition{adj.}{excepcional; extraordinário; incomum; mais do que o normal}
  \end{Phonetics}
\end{Entry}

\begin{Entry}{非法}{8,8}{⾮、⽔}
  \begin{Phonetics}{非法}{fei1fa3}[][HSK 7-9]
    \definition{adj.}{ilegal; ilícito; fora da lei}
  \end{Phonetics}
\end{Entry}

\begin{Entry}{非金属}{8,8,12}{⾮、⾦、⼫}
  \begin{Phonetics}{非金属}{fei1jin4shu3}[][HSK 7-9]
    \definition{s.}{Química: não metal; metalóide; com exceção do bromo, os elementos que geralmente não têm brilho metálico nem ductilidade e não conduzem facilmente eletricidade ou calor são gases ou sólidos à temperatura ambiente, como oxigênio, enxofre, nitrogênio, fósforo, etc.}
  \end{Phonetics}
\end{Entry}

\begin{Entry}{非洲}{8,9}{⾮、⽔}
  \begin{Phonetics}{非洲}{fei1zhou1}
    \definition*{s.}{África}
  \end{Phonetics}
\end{Entry}

\begin{Entry}{非洲人}{8,9,2}{⾮、⽔、⼈}
  \begin{Phonetics}{非洲人}{fei1zhou1ren2}
    \definition{s.}{africano | pessoa ou povo da África}
  \end{Phonetics}
\end{Entry}

\begin{Entry}{非常}{8,11}{⾮、⼱}
  \begin{Phonetics}{非常}{fei1chang2}[][HSK 1]
    \definition{adj.}{extraordinário; incomum; especial}
    \definition{adv.}{muito; extremamente; altamente}
  \end{Phonetics}
\end{Entry}

\begin{Entry}{非得}{8,11}{⾮、⼻}
  \begin{Phonetics}{非得}{fei1dei3}[][HSK 7-9]
    \definition{adv.}{(geralmente usado comcomitantemente com 不 ou 才) tem que; deve}[你非得服从命令不可。===Você deve obedecer às ordens.]
  \seealsoref{不}{bu4}
  \seealsoref{才}{cai2}
  \end{Phonetics}
\end{Entry}

\begin{Entry}{靣}{8}{⼀}[Kangxi 176]
  \begin{Phonetics}{靣}{mian4}
    \variantof{面}
  \end{Phonetics}
\end{Entry}

\begin{Entry}{顶}{8}{⾴}
  \begin{Phonetics}{顶}{ding3}[][HSK 4]
    \definition{adv.}{muito (linguagem falada); a maioria; extremamente; expressa o grau mais alto, equivalente a 最 e 极}
    \definition{clas.}{usado para coisas que têm um topo}
    \definition{prep.}{até}
    \definition{s.}{coroa da cabeça; parte mais alta do corpo ou objeto | topo; limite superior; ponto mais alto}
    \definition{v.}{carregar na cabeça; carregar em sua cabeça | empurrar (ou apoiar) para cima; empurrar por baixo (ou por trás) | dar cabeçadas; dar uma coronhada | sustentar; apoiar; suportar | resistir; ir contra; enfrentar | rebater; retorquir; responder de volta | cooperar; enfrentar; apoiar; dar suporte | igualar; ser equivalente a | substituir; tomar o lugar de | assumir o controle; transferir ou adquirir o direito de administrar um negócio ou alugar uma casa ou terreno}
  \seealsoref{极}{ji2}
  \seealsoref{最}{zui4}
  \end{Phonetics}
\end{Entry}

\begin{Entry}{顶多}{8,6}{⾴、⼣}
  \begin{Phonetics}{顶多}{ding3duo1}[][HSK 7-9]
    \definition{adv.}{na melhor das hipóteses; no máximo, na opinião do orador, o número real não será maior que o maior número estimado}
  \end{Phonetics}
\end{Entry}

\begin{Entry}{顶尖}{8,6}{⾴、⼩}
  \begin{Phonetics}{顶尖}{ding3jian1}[][HSK 7-9]
    \definition{adj.}{melhor; de primeira classe; de mais alto nível}
    \definition{s.}{centro; ápice | topo pontiagudo; ponta; pico; a parte mais alta e pontiaguda}
  \end{Phonetics}
\end{Entry}

\begin{Entry}{顶级}{8,6}{⾴、⽷}
  \begin{Phonetics}{顶级}{ding3ji2}[][HSK 7-9]
    \definition{adj.}{de primeira classe; de alta qualidade; de ponta}
  \end{Phonetics}
\end{Entry}

\begin{Entry}{饱}{8}{⾷}
  \begin{Phonetics}{饱}{bao3}[][HSK 2]
    \definition{adj.}{cheio; comer até ficar satisfeito | cheio; rechonchudo}
    \definition{adv.}{totalmente; completamente; plenamente}
    \definition{v.}{satisfazer}
  \end{Phonetics}
\end{Entry}

\begin{Entry}{饱和}{8,8}{⾷、⼝}
  \begin{Phonetics}{饱和}{bao3he2}[][HSK 7-9]
    \definition{v.}{estar saturado; a uma certa temperatura ou pressão, a quantidade de soluto contida na solução atinge seu limite máximo e não consegue mais se dissolver | estar saturado; metaforicamente, a quantidade de algo atinge um máximo dentro de um certo intervalo}
  \end{Phonetics}
\end{Entry}

\begin{Entry}{饱满}{8,13}{⾷、⽔}
  \begin{Phonetics}{饱满}{bao3man3}[][HSK 7-9]
    \definition{adj.}{cheio; rechonchudo; bem empilhado; preenchido | robusto; abundante; pleno; vigoroso}
  \end{Phonetics}
\end{Entry}

\begin{Entry}{驻}{8}{⾺}
  \begin{Phonetics}{驻}{zhu4}[][HSK 6]
    \definition{v.}{parar; ficar | estar estacionado; acampar; (tropas ou pessoal) viver no local onde desempenham suas funções; (organização) estar localizada em um determinado lugar}
  \end{Phonetics}
\end{Entry}

\begin{Entry}{驻军}{8,6}{⾺、⼍}
  \begin{Phonetics}{驻军}{zhu4jun1}
    \definition{s.}{guarnição}
    \definition{v.}{guarcener ou prover uma tropa}
  \end{Phonetics}
\end{Entry}

\begin{Entry}{驾}{8}{⾺}
  \begin{Phonetics}{驾}{jia4}
    \definition*{s.}{Sobrenome Jia}
    \definition{s.}{carruagem do imperador; refere-se especificamente ao carro do imperador, referindo-se ao imperador | referindo-se a um veículo, usado como um termo respeitoso para uma pessoa}
    \definition{v.}{atrelar; puxar (uma carroça, etc.) | dirigir (um veículo); pilotar (um avião); velejar (um barco) | montar; cavalgar}
  \end{Phonetics}
\end{Entry}

\begin{Entry}{驾驶}{8,8}{⾺、⾺}
  \begin{Phonetics}{驾驶}{jia4shi3}[][HSK 5]
    \definition{v.}{dirigir; pilotar; conduzir; guiar; operar (um carro, navio, avião, trator, etc.) para fazê-lo mover}
  \end{Phonetics}
\end{Entry}

\begin{Entry}{驾照}{8,13}{⾺、⽕}
  \begin{Phonetics}{驾照}{jia4 zhao4}[][HSK 5]
    \definition[本,张]{s.}{carteira de motorista}
  \end{Phonetics}
\end{Entry}

\begin{Entry}{鱼}{8}{⿂}[Kangxi 195]
  \begin{Phonetics}{鱼}{yu2}[][HSK 2]
    \definition*{s.}{Sobrenome Yu}
    \definition[条,种,尾]{s.}{peixe; um vertebrado que vive na água; geralmente possui um corpo achatado lateralmente, fusiforme e com muitas escamas; nada com as nadadeiras e respira com as brânquias; sua temperatura corporal varia de acordo com a temperatura externa; existem muitas espécies, a maioria das quais comestíveis | carne de peixe; peixe (como alimento)}
  \end{Phonetics}
\end{Entry}

\begin{Entry}{鱼片}{8,4}{⿂、⽚}
  \begin{Phonetics}{鱼片}{yu2pian4}
    \definition{s.}{fatia de peixe | filé de peixe}
  \end{Phonetics}
\end{Entry}

\begin{Entry}{鱼汛}{8,6}{⿂、⽔}
  \begin{Phonetics}{鱼汛}{yu2xun4}
    \variantof{渔汛}
  \end{Phonetics}
\end{Entry}

\begin{Entry}{鱼网}{8,6}{⿂、⽹}
  \begin{Phonetics}{鱼网}{yu2wang3}
    \variantof{渔网}
  \end{Phonetics}
\end{Entry}

\begin{Entry}{鱼具}{8,8}{⿂、⼋}
  \begin{Phonetics}{鱼具}{yu2ju4}
    \variantof{渔具}
  \end{Phonetics}
\end{Entry}

\begin{Entry}{鱼香}{8,9}{⿂、⾹}
  \begin{Phonetics}{鱼香}{yu2xiang1}
    \definition{s.}{um tempero da culinária chinesa que normalmente contém alho, cebolinha, gengibre, açúcar, sal, pimenta, etc. (Embora 鱼香 signifique literalmente ``fragrância de peixe'', não contém frutos do mar.)}
  \end{Phonetics}
\end{Entry}

\begin{Entry}{鱼香肉丝}{8,9,6,5}{⿂、⾹、⾁、⼀}
  \begin{Phonetics}{鱼香肉丝}{yu2xiang1rou4si1}
    \definition{s.}{tiras de carne de porco salteadas com molho picante (prato)}
  \seealsoref{鱼香}{yu2xiang1}
  \end{Phonetics}
\end{Entry}

\begin{Entry}{鱼船}{8,11}{⿂、⾈}
  \begin{Phonetics}{鱼船}{yu2chuan2}
    \definition{s.}{barco de pesca}
  \seealsoref{渔船}{yu2chuan2}
  \end{Phonetics}
\end{Entry}

\begin{Entry}{鸣}{8}{⿃}
  \begin{Phonetics}{鸣}{ming2}
    \definition{v.}{chorar (pássaros, animais e insetos) | fazer um som | dar voz (gratidão, queixas, etc.)}
  \end{Phonetics}
\end{Entry}

\begin{Entry}{齿}{8}{⿒}[Kangxi 211]
  \begin{Phonetics}{齿}{chi3}
    \definition[颗]{s.}{dente | uma parte de qualquer coisa semelhante a um dente; parte dentada de um objeto | idade (de uma pessoa); faixa etária}
    \definition{v.}{mencionar; falar de}
  \end{Phonetics}
\end{Entry}

\begin{Entry}{齿儿}{8,2}{⿒、⼉}
  \begin{Phonetics}{齿儿}{chi3r5}
    \definition{s.}{dentes}
  \end{Phonetics}
\end{Entry}

%%%%% EOF %%%%%


%%%
%%% 9画
%%%

\section*{9画}\addcontentsline{toc}{section}{9画}

\begin{entry}{临}{9}{⼁}
  \begin{phonetics}{临}{lin2}
    \definition*{s.}{sobrenome Lin}
    \definition{adv.}{pouco antes; prestes a; no ponto de}
    \definition{v.}{encarar; enfrentar; aproximar-se | chegar; estar presente | copiar (um modelo de caligrafia ou pintura); traçar sobre as palavras ou figuras | olhar de cima para baixo | ir de cima para baixo}
  \end{phonetics}
\end{entry}

\begin{entry}{临时}{9,7}{⼁、⽇}
  \begin{phonetics}{临时}{lin2shi2}[][HSK 4]
    \definition{adj.}{temporário; provisório; por um breve período}
    \definition{adv.}{no momento em que algo acontece (quando as coisas dão errado)}
  \end{phonetics}
\end{entry}

\begin{entry}{临近}{9,7}{⼁、⾡}
  \begin{phonetics}{临近}{lin2jin4}
    \definition{v.}{aproximar-se; estar perto de}
  \end{phonetics}
\end{entry}

\begin{entry}{举}{9}{⼂}
  \begin{phonetics}{举}{ju3}[][HSK 2]
    \definition*{s.}{sobrenome Ju}
    \definition{adj.}{inteiro; completo}
    \definition{s.}{ato; ação; movimento; comportamento | (nas dinastias Ming e Qing) candidato aprovado nos exames imperiais a nível provincial}
    \definition{v.}{levantar; erguer; sustentar | começar; iniciar; surgir | eleger; escolher; recomendar; selecionar | citar; enumerar; propor; revelar}
  \end{phonetics}
\end{entry}

\begin{entry}{举办}{9,4}{⼂、⼒}
  \begin{phonetics}{举办}{ju3ban4}[][HSK 3]
    \definition{v.}{conduzir; organizar; realizar}
  \end{phonetics}
\end{entry}

\begin{entry}{举手}{9,4}{⼂、⼿}
  \begin{phonetics}{举手}{ju3 shou3}[][HSK 2]
    \definition{v.}{levantar a mão ou as mãos; levantar a mão para sinalizar ou responder a uma pergunta}
  \end{phonetics}
\end{entry}

\begin{entry}{举动}{9,6}{⼂、⼒}
  \begin{phonetics}{举动}{ju3dong4}[][HSK 5]
    \definition{s.}{ato; atividade; movimento; ação}
  \end{phonetics}
\end{entry}

\begin{entry}{举行}{9,6}{⼂、⾏}
  \begin{phonetics}{举行}{ju3xing2}[][HSK 2]
    \definition{v.}{realizar (uma reunião, cerimônia, etc.); realizar (atividades formais ou solenes)}
  \end{phonetics}
\end{entry}

\begin{entry}{亭}{9}{⼇}
  \begin{phonetics}{亭}{ting2}
    \definition{s.}{pavilhão | cabine | quiosque}
  \end{phonetics}
\end{entry}

\begin{entry}{亮}{9}{⼇}
  \begin{phonetics}{亮}{liang4}[][HSK 2]
    \definition*{s.}{sobrenome Lian}
    \definition{adj.}{brilhante; claro | alto e claro; retumbante | esclarecido; aberto e claro}
    \definition{s.}{luz}
    \definition{v.}{iluminar; clarear; brilhar | elevar a voz; ressoar; tornar o som mais alto | revelar; mostrar; aparecer; exibir}
  \end{phonetics}
\end{entry}

\begin{entry}{亲}{9}{⼇}
  \begin{phonetics}{亲}{qin1}[][HSK 3]
    \definition{adj.}{parente próximo; relacionado por sangue; de ​​parentesco consanguíneo; parente consanguíneo mais próximo | querido; próximo; íntimo; relações próximas entre pessoas; sentimentos profundos (em oposição a 疏) | em si mesmo; pessoalmente}
    \definition[位]{s.}{pais; refere-se aos pais; também se refere apenas ao pai ou à mãe | parente; refere-se a pessoas que são relacionadas por sangue ou casamento| casal; casamento; refere-se ao casamento ou relacionamento conjugal | noiva; refere-se especificamente à noiva}
    \definition{v.}{beijar | (de países, partidos, etc.) a favor de; apoiar; estar perto de}
  \seealsoref{疏}{shu1}
  \end{phonetics}
  \begin{phonetics}{亲}{qing4}
    \definition{s.}{parentes por afinidade; parentes por casamento}
  \end{phonetics}
\end{entry}

\begin{entry}{亲人}{9,2}{⼇、⼈}
  \begin{phonetics}{亲人}{qin1 ren2}[][HSK 3]
    \definition[个,位]{s.}{um membro da família; os pais, o cônjuge, os filhos, etc.; refere-se a parentes ou cônjuges | queridos; entes queridos; aqueles queridos para alguém; uma metáfora para pessoas que têm um relacionamento próximo e sentimentos profundos}
  \end{phonetics}
\end{entry}

\begin{entry}{亲切}{9,4}{⼇、⼑}
  \begin{phonetics}{亲切}{qin1qie4}[][HSK 3]
    \definition{adj.}{gentil; cordial; cheio de sinceridade e cuidado, fazendo com que as pessoas se sintam acolhidas e acessíveis | próximo; íntimo; por familiaridade e afeição}
  \end{phonetics}
\end{entry}

\begin{entry}{亲自}{9,6}{⼇、⾃}
  \begin{phonetics}{亲自}{qin1zi4}[][HSK 3]
    \definition{adv.}{pessoalmente; em pessoa; si mesmo; fazer algo diretamente por si mesmo}
  \end{phonetics}
\end{entry}

\begin{entry}{亲爱}{9,10}{⼇、⽖}
  \begin{phonetics}{亲爱}{qin1'ai4}[][HSK 4]
    \definition{adj.}{querido; amado; termo carinhoso que expressa intimidade e afeto}
  \end{phonetics}
\end{entry}

\begin{entry}{亲密}{9,11}{⼇、⼧}
  \begin{phonetics}{亲密}{qin1mi4}[][HSK 4]
    \definition{adj.}{próximo; íntimo; relacionamento afetuoso e próximo}
  \end{phonetics}
\end{entry}

\begin{entry}{侵略}{9,11}{⼈、⽥}
  \begin{phonetics}{侵略}{qin1lve4}
    \definition{s.}{invasão}
    \definition{v.}{invadir}
  \end{phonetics}
\end{entry}

\begin{entry}{便}{9}{⼈}
  \begin{phonetics}{便}{bian4}
    \definition{adj.}{prático; conveniente | simples; comum; informal}
    \definition{adv.}{então; apenas no caso de; mesmo significado e uso de 就}
    \definition{conj.}{mesmo que; expressa uma concessão hipotética}
    \definition{s.}{facilidade; conveniência; o momento certo; a oportunidade | fezes ou urina}
    \definition{v.}{aliviar-se; excretar fezes e urina}
  \seealsoref{就}{jiu4}
  \end{phonetics}
  \begin{phonetics}{便}{pian2}
    \definition*{s.}{sobrenome Pian}
    \definition{adj.}{silencioso e confortável}
  \end{phonetics}
\end{entry}

\begin{entry}{便于}{9,3}{⼈、⼆}
  \begin{phonetics}{便于}{bian4yu2}[][HSK 5]
    \definition{v.}{ser fácil para; ser conveniente para (algo ou fazer algo)}
  \end{phonetics}
\end{entry}

\begin{entry}{便利}{9,7}{⼈、⼑}
  \begin{phonetics}{便利}{bian4li4}[][HSK 5]
    \definition{adj.}{fácil; conveniente;}
    \definition{s.}{facilidade; conveniência; coisas ou condições convenientes}
    \definition{v.}{facilitar; fornecer ajuda para que os outros se sintam confortáveis}
  \end{phonetics}
\end{entry}

\begin{entry}{便条}{9,7}{⼈、⽊}
  \begin{phonetics}{便条}{bian4tiao2}[][HSK 5]
    \definition[张,个]{s.}{nota ou mensagem informal; geralmente uma mensagem ou notificação}
  \end{phonetics}
\end{entry}

\begin{entry}{便宜}{9,8}{⼈、⼧}
  \begin{phonetics}{便宜}{bian4yi2}
    \definition{adj.}{prático; conveniente; adequado}
  \end{phonetics}
  \begin{phonetics}{便宜}{pian2yi5}[][HSK 2]
    \definition{adj.}{barato; acessível}
    \definition[个,份,件]{s.}{vantagem em algum aspecto | ganho; lucro; vantagem; benefício indevido}
    \definition{v.}{deixar alguém escapar impune; obter algum benefício}
  \end{phonetics}
\end{entry}

\begin{entry}{促}{9}{⼈}
  \begin{phonetics}{促}{cu4}
    \definition{adj.}{curto; apressado; urgente}
    \definition{v.}{urgir; promover | estar perto de; estar perto}
  \end{phonetics}
\end{entry}

\begin{entry}{促进}{9,7}{⼈、⾡}
  \begin{phonetics}{促进}{cu4jin4}[][HSK 4]
    \definition{v.}{impulsionar; promover; avançar; incentivar o desenvolvimento}
  \end{phonetics}
\end{entry}

\begin{entry}{促使}{9,8}{⼈、⼈}
  \begin{phonetics}{促使}{cu4shi3}[][HSK 4]
    \definition{v.}{incitar; estimular; impelir; causar; provocar uma mudança em alguém ou em algo}
  \end{phonetics}
\end{entry}

\begin{entry}{促销}{9,12}{⼈、⾦}
  \begin{phonetics}{促销}{cu4 xiao1}[][HSK 4]
    \definition{v.}{promover vendas}
  \end{phonetics}
\end{entry}

\begin{entry}{俄}{9}{⼈}
  \begin{phonetics}{俄}{e2}
    \definition*{s.}{Rússia, abreviação de 俄罗斯}
    \definition{adv.}{muito em breve; em breve; de repente}
  \seealsoref{俄罗斯}{e2luo2si1}
  \end{phonetics}
\end{entry}

\begin{entry}{俄罗斯}{9,8,12}{⼈、⽹、⽄}
  \begin{phonetics}{俄罗斯}{e2luo2si1}
    \definition*{s.}{Rússia}
  \end{phonetics}
\end{entry}

\begin{entry}{俄罗斯人}{9,8,12,2}{⼈、⽹、⽄、⼈}
  \begin{phonetics}{俄罗斯人}{e2luo2si1ren2}
    \definition{s.}{russo | pessoa ou povo da Rússia}
  \end{phonetics}
\end{entry}

\begin{entry}{保}{9}{⼈}
  \begin{phonetics}{保}{bao3}[][HSK 3]
    \definition*{s.}{sobrenome Bao}
    \definition{s.}{fiador; babá ou responsável pela guarda de crianças | oficial responsável; sistema administrativo; unidade administrativa do antigo registro civil}
    \definition{v.}{defender; proteger | manter; preservar; conservar em boas condições | assegurar; garantir | ser fiador de alguém}
  \end{phonetics}
\end{entry}

\begin{entry}{保卫}{9,3}{⼈、⼙}
  \begin{phonetics}{保卫}{bao3wei4}[][HSK 5]
    \definition{v.}{defender; proteger; salvaguardar}
  \end{phonetics}
\end{entry}

\begin{entry}{保存}{9,6}{⼈、⼦}
  \begin{phonetics}{保存}{bao3cun2}[][HSK 3]
    \definition{v.}{salvar; preservar; conservar; manter a existência com ênfase em que as coisas, as propriedades, os significados, os estilos, etc. não sofram perdas ou mudanças | (computação) salvar (um arquivo, etc.)}
  \end{phonetics}
\end{entry}

\begin{entry}{保守}{9,6}{⼈、⼧}
  \begin{phonetics}{保守}{bao3shou3}[][HSK 4]
    \definition{adj.}{retrógrado; conservador; pensamentos e conceitos que são retrógrados e não conseguem acompanhar o desenvolvimento da situação}
    \definition{v.}{manter; guardar; evitar perder}
  \end{phonetics}
\end{entry}

\begin{entry}{保安}{9,6}{⼈、⼧}
  \begin{phonetics}{保安}{bao3 an1}[][HSK 3]
    \definition[个,位,名]{s.}{guarda de segurança; segurança}
    \definition{v.}{proteger; manter em segurança; defender a segurança social | garantir a segurança; proteger a segurança dos trabalhadores e prevenir acidentes durante o processo de produção}
  \end{phonetics}
\end{entry}

\begin{entry}{保护}{9,7}{⼈、⼿}
  \begin{phonetics}{保护}{bao3hu4}[][HSK 3]
    \definition{v.}{proteger, guardar, cuidar; salvaguardar; cuidar ao máximo, para que não seja danificado, referindo-se principalmente a coisas concretas}
  \end{phonetics}
\end{entry}

\begin{entry}{保护区}{9,7,4}{⼈、⼿、⼖}
  \begin{phonetics}{保护区}{bao3hu4qu1}
    \definition{s.}{área protegida | área de conservação}
  \end{phonetics}
\end{entry}

\begin{entry}{保护主义}{9,7,5,3}{⼈、⼿、⼂、⼂}
  \begin{phonetics}{保护主义}{bao3hu4zhu3yi4}
    \definition{s.}{protecionismo}
  \end{phonetics}
\end{entry}

\begin{entry}{保护色}{9,7,6}{⼈、⼿、⾊}
  \begin{phonetics}{保护色}{bao3hu4se4}
    \definition{s.}{camuflagem}
  \end{phonetics}
\end{entry}

\begin{entry}{保护剂}{9,7,8}{⼈、⼿、⼑}
  \begin{phonetics}{保护剂}{bao3hu4ji4}
    \definition{s.}{agente protetor}
  \end{phonetics}
\end{entry}

\begin{entry}{保护国}{9,7,8}{⼈、⼿、⼞}
  \begin{phonetics}{保护国}{bao3hu4guo2}
    \definition{s.}{protetorado}
  \end{phonetics}
\end{entry}

\begin{entry}{保护性}{9,7,8}{⼈、⼿、⼼}
  \begin{phonetics}{保护性}{bao3hu4xing4}
    \definition{s.}{proteção}
  \end{phonetics}
\end{entry}

\begin{entry}{保护物}{9,7,8}{⼈、⼿、⽜}
  \begin{phonetics}{保护物}{bao3hu4 wu4}
    \definition{s.}{protetor}
  \end{phonetics}
\end{entry}

\begin{entry}{保护者}{9,7,8}{⼈、⼿、⽼}
  \begin{phonetics}{保护者}{bao3hu4zhe3}
    \definition{s.}{protetor | segurador}
  \end{phonetics}
\end{entry}

\begin{entry}{保护神}{9,7,9}{⼈、⼿、⽰}
  \begin{phonetics}{保护神}{bao3hu4shen2}
    \definition{s.}{anjo da guarda | santo patrono}
  \end{phonetics}
\end{entry}

\begin{entry}{保证}{9,7}{⼈、⾔}
  \begin{phonetics}{保证}{bao3zheng4}[][HSK 3]
    \definition[种,份]{s.}{compromisso; garantia; caução; aval; condições ou coisas que garantem a realização de algo}
    \definition{v.}{prometer; garantir; assegurar; certamente concluir algo; garantir que determinados padrões e requisitos sejam alcançados}
  \end{phonetics}
\end{entry}

\begin{entry}{保养}{9,9}{⼈、⼋}
  \begin{phonetics}{保养}{bao3yang3}[][HSK 5]
    \definition{v.}{preservar; cuidar bem (ou conservar) da saúde |  fazer manutenção; conservar; manter; manter em bom estado de conservação}
  \end{phonetics}
\end{entry}

\begin{entry}{保持}{9,9}{⼈、⼿}
  \begin{phonetics}{保持}{bao3chi2}[][HSK 3]
    \definition{v.}{manter; conservar; reter; preservar; manter um determinado estado, para que não desapareça ou não se altere}
  \end{phonetics}
\end{entry}

\begin{entry}{保险}{9,9}{⼈、⾩}
  \begin{phonetics}{保险}{bao3xian3}[][HSK 3]
    \definition{adj.}{seguro; pode ficar tranquilo}
    \definition[个,份,种]{s.}{seguro; um tipo de seguro comercial que garante que o segurado receba uma indenização em caso de prejuízo}
    \definition{v.}{ter certeza; estar obrigado a; garantir que algo aconteça (o que as pessoas desejam)}
  \end{phonetics}
\end{entry}

\begin{entry}{保留}{9,10}{⼈、⽥}
  \begin{phonetics}{保留}{bao3liu2}[][HSK 3]
    \definition{v.}{manter; continuar a ter; manter o estado original inalterado | conter; reter; deixar ficar; não tirar | reservar; colocar os direitos, opiniões, etc. de lado, não exercê-los ou expressá-los por enquanto}
  \end{phonetics}
\end{entry}

\begin{entry}{保密}{9,11}{⼈、⼧}
  \begin{phonetics}{保密}{bao3mi4}[][HSK 4]
    \definition{v.}{manter segredo; manter algo em segredo; manter a confidencialidade}
  \end{phonetics}
\end{entry}

\begin{entry}{信}{9}{⼈}
  \begin{phonetics}{信}{xin4}[][HSK 2,3]
    \definition*{s.}{sobrenome Xin}
    \definition{adj.}{verdade}
    \definition[封,个,张]{s.}{carta; correio | mensagem; notícia; informação | sinal; evidência | confiança; fé; crédito | detonador (de bombas, etc.) | arsênico}
    \definition{v.}{acreditar; fazer um balanço; dar crédito | deixar à vontade; deixar à mercê; deixar ao acaso | professar fé em; acreditar em}
  \end{phonetics}
\end{entry}

\begin{entry}{信心}{9,4}{⼈、⼼}
  \begin{phonetics}{信心}{xin4xin1}[][HSK 2]
    \definition[个]{s.}{confiança; fé (em alguém ou algo) ; a crença de que os desejos se tornarão realidade}
  \end{phonetics}
\end{entry}

\begin{entry}{信号}{9,5}{⼈、⼝}
  \begin{phonetics}{信号}{xin4hao4}[][HSK 2]
    \definition[个,道]{s.}{sinal; luz, ondas de rádio, som, movimento, etc. usados para transmitir mensagens ou comandos | ponte de sinalização; marcação para chamar a atenção, ajudar na identificação e na memória}
  \end{phonetics}
\end{entry}

\begin{entry}{信用}{9,5}{⼈、⽤}
  \begin{phonetics}{信用}{xin4yong4}
    \definition{s.}{crédito (comércio)}
  \end{phonetics}
\end{entry}

\begin{entry}{信用卡}{9,5,5}{⼈、⽤、⼘}
  \begin{phonetics}{信用卡}{xin4yong4ka3}[][HSK 2]
    \definition[张]{s.}{cartão de crédito; moeda eletrônica emitida por um banco ou outra instituição especializada para consumidores; os titulares do cartão podem usá-lo para sacar dinheiro ou fazer compras de acordo com os regulamentos}
  \end{phonetics}
\end{entry}

\begin{entry}{信任}{9,6}{⼈、⼈}
  \begin{phonetics}{信任}{xin4ren4}[][HSK 3]
    \definition{s.}{confiança; um estado mental positivo e conexão emocional}
    \definition{v.}{confiar; ter confiança em; acreditar e ousar confiar}
  \end{phonetics}
\end{entry}

\begin{entry}{信访}{9,6}{⼈、⾔}
  \begin{phonetics}{信访}{xin4fang3}
    \definition{s.}{carta de reclamação | carta de petição}
  \seealsoref{上访}{shang4fang3}
  \end{phonetics}
\end{entry}

\begin{entry}{信念}{9,8}{⼈、⼼}
  \begin{phonetics}{信念}{xin4nian4}[][HSK 5]
    \definition[个]{s.}{fé; crença; convicção; concepções consideradas corretas e acreditadas com convicção}
  \end{phonetics}
\end{entry}

\begin{entry}{信经}{9,8}{⼈、⽷}
  \begin{phonetics}{信经}{xin4jing1}
    \definition[个]{s.}{crença | credo (seção da missa católica)}
  \end{phonetics}
\end{entry}

\begin{entry}{信封}{9,9}{⼈、⼨}
  \begin{phonetics}{信封}{xin4feng1}[][HSK 3]
    \definition[个,封]{s.}{envelope para cartas}
  \end{phonetics}
\end{entry}

\begin{entry}{信息}{9,10}{⼈、⼼}
  \begin{phonetics}{信息}{xin4xi1}[][HSK 2]
    \definition[个,条,段,些]{s.}{notícias; informações; as últimas notícias sobre alguém ou alguma coisa | mensagem; informação; na teoria da informação, uma mensagem transmitida usando símbolos, cujo conteúdo é desconhecido pelo receptor}
  \end{phonetics}
\end{entry}

\begin{entry}{信箱}{9,15}{⼈、⾋}
  \begin{phonetics}{信箱}{xin4 xiang1}[][HSK 5]
    \definition{s.}{caixa de correio; caixa postal instalada pelos correios para que as pessoas possam depositar cartas | caixa postal; caixas com números, localizadas nos correios, que podem ser alugadas para receber correspondência; chamadas de caixas postais exclusivas}
  \end{phonetics}
\end{entry}

\begin{entry}{俩}{9}{⼈}
  \begin{phonetics}{俩}{lia3}[][HSK 4]
    \definition{num.}{ambos; dois; contração de 两个 | alguns; vários; refere-se a um pequeno número}
  \end{phonetics}
\end{entry}

\begin{entry}{俩钱}{9,10}{⼈、⾦}
  \begin{phonetics}{俩钱}{lia3qian2}
    \definition{s.}{uma pequena quantia de dinheiro}
  \end{phonetics}
\end{entry}

\begin{entry}{俭省}{9,9}{⼈、⽬}
  \begin{phonetics}{俭省}{jian3sheng3}
    \definition{adj.}{econômico}
  \end{phonetics}
\end{entry}

\begin{entry}{修}{9}{⼈}
  \begin{phonetics}{修}{xiu1}[][HSK 3]
    \definition*{s.}{sobrenome Xiu}
    \definition{adj.}{comprido; alto e esbelto}
    \definition{s.}{revisionismo}
    \definition{v.}{embelezar; decorar | consertar; reparar; reformar | escrever; redigir; compilar | estudar; cultivar; aprender e praticar para aperfeiçoar ou melhorar (o caráter e o conhecimento) | construir; edificar | cortar ou aparar, para deixar bonito e arrumado | dedicar-se à prática da religião}
  \end{phonetics}
\end{entry}

\begin{entry}{修改}{9,7}{⼈、⽁}
  \begin{phonetics}{修改}{xiu1gai3}[][HSK 3]
    \definition{v.}{revisar; retocar; corrigir erros e falhas em artigos, planos, etc.}
  \end{phonetics}
\end{entry}

\begin{entry}{修建}{9,8}{⼈、⼵}
  \begin{phonetics}{修建}{xiu1jian4}[][HSK 5]
    \definition{v.}{construir; erguer; animar; edificar; construir com tijolos, telhas, madeira, cimento, areia, etc.}
  \end{phonetics}
\end{entry}

\begin{entry}{修规}{9,8}{⼈、⾒}
  \begin{phonetics}{修规}{xiu1gui1}
    \definition{s.}{plano de construção}
  \end{phonetics}
\end{entry}

\begin{entry}{修养}{9,9}{⼈、⼋}
  \begin{phonetics}{修养}{xiu1yang3}[][HSK 5]
    \definition[种]{s.}{treinamento; domínio; realização; refere-se a um determinado nível em termos de teoria, conhecimento, arte, pensamento, etc. | auto-cultivo; refere-se à atitude e ao comportamento cultivados ao longo do tempo, em conformidade com as exigências sociais}
  \end{phonetics}
\end{entry}

\begin{entry}{修复}{9,9}{⼈、⼢}
  \begin{phonetics}{修复}{xiu1fu4}[][HSK 5]
    \definition{v.}{reparar; restaurar; renovar | reparar; melhorar e restaurar (o relacionamento)}
  \end{phonetics}
\end{entry}

\begin{entry}{修理}{9,11}{⼈、⽟}
  \begin{phonetics}{修理}{xiu1li3}[][HSK 4]
    \definition{v.}{consertar; reparar; restaurar algo danificado à sua forma ou função original | aparar; podar; cortar com tesouras e outras ferramentas para deixar árvores, flores, cabelos, etc. arrumados | culpar; punir; criticar ou punir uma pessoa para mostrar que ela está errada}
  \end{phonetics}
\end{entry}

\begin{entry}{养}{9}{⼋}
  \begin{phonetics}{养}{yang3}[][HSK 2]
    \definition*{s.}{sobrenome Yang}
    \definition{adj.}{adotivo; órfão; adotado; não biológico}
    \definition{s.}{qualidade; (caráter moral, desempenho acadêmico, etc.) boas qualidades}
    \definition{v.}{apoiar; prover; fornecer dinheiro e materiais necessários para viver | aumentar; manter; crescer; alimentar os animais e cuidar de suas vidas para que possam crescer | dar à luz | formar; adquirir; cultivar | descansar; curar; convalescer; recuperar a saúde | manter; manter em bom estado | deixar (o cabelo) crescer | ajudar; apoiar | cultivar (plantações ou flores)}
  \end{phonetics}
\end{entry}

\begin{entry}{养分}{9,4}{⼋、⼑}
  \begin{phonetics}{养分}{yang3fen4}
    \definition{s.}{nutriente}
  \end{phonetics}
\end{entry}

\begin{entry}{养成}{9,6}{⼋、⼽}
  \begin{phonetics}{养成}{yang3cheng2}[][HSK 4]
    \definition{v.}{cultivar; desenvolver; cultivar para formar; nutrir para crescer}
  \end{phonetics}
\end{entry}

\begin{entry}{养料}{9,10}{⼋、⽃}
  \begin{phonetics}{养料}{yang3liao4}
    \definition{s.}{nutrição}
  \end{phonetics}
\end{entry}

\begin{entry}{冒}{9}{⽇}
  \begin{phonetics}{冒}{mao4}[][HSK 5]
    \definition*{s.}{sobrenome Mao}
    \definition{adv.}{com ousadia; precipitadamente | fingidamente; falsamente; fraudulentamente}
    \definition{v.}{emitir; liberar; enviar (para cima) | arriscar; ser corajoso}
  \end{phonetics}
\end{entry}

\begin{entry}{冒险}{9,9}{⽇、⾩}
  \begin{phonetics}{冒险}{mao4xian3}
    \definition{adj.}{corajoso}
    \definition{s.}{risco | aventura}
    \definition{v.+compl.}{correr risco | arriscar-se | aventurar-se em}
  \end{phonetics}
\end{entry}

\begin{entry}{冠}{9}{⼍}
  \begin{phonetics}{冠}{guan1}
    \definition{s.}{chapéu | corona; coroa; copa | crista}
  \end{phonetics}
  \begin{phonetics}{冠}{guan4}
    \definition*{s.}{sobrenome Guan}
    \definition{s.}{primeiro lugar; o melhor; classificado em primeiro lugar}
    \definition{v.}{colocar um chapéu (boné) | preceder com (por); coroar com; adicionar um nome ou texto na frente}
  \end{phonetics}
\end{entry}

\begin{entry}{冠军}{9,6}{⼍、⼍}
  \begin{phonetics}{冠军}{guan4jun1}[][HSK 5]
    \definition[个]{s.}{campeão; medalhista de ouro; primeiro lugar em esportes e outras competições}
  \end{phonetics}
\end{entry}

\begin{entry}{前}{9}{⼑}
  \begin{phonetics}{前}{qian2}[][HSK 1]
    \definition*{s.}{sobrenome Qian}
    \definition{s.}{frente | futuro; perspectiva | atrás; antes; mais cedo do que uma coisa ou um momento | à frente; para a frente; na parte frontal (referindo-se ao espaço, em oposição a 后) | precedente; antes que algo aconteça | antigo; antigamente | topo; primeiro; primeiro na ordem | frente; campo de batalha | A.C. (Antes de~Cristo)}[前293年___293 a.C.]
    \definition{v.}{seguir em frente; ir em frente}
  \seealsoref{公元}{gong1yuan2}
  \seealsoref{后}{hou4}
  \end{phonetics}
\end{entry}

\begin{entry}{前天}{9,4}{⼑、⼤}
  \begin{phonetics}{前天}{qian2 tian1}[][HSK 1]
    \definition{adv.}{anteontem; dia anterior a ontem}
  \end{phonetics}
\end{entry}

\begin{entry}{前头}{9,5}{⼑、⼤}
  \begin{phonetics}{前头}{qian2 tou5}[][HSK 4]
    \definition{s.}{à frente; na frente; adiante}
  \end{phonetics}
\end{entry}

\begin{entry}{前边}{9,5}{⼑、⾡}
  \begin{phonetics}{前边}{qian2 bian5}[][HSK 1]
    \definition{adv.}{à frente; na frente}
  \end{phonetics}
\end{entry}

\begin{entry}{前后}{9,6}{⼑、⼝}
  \begin{phonetics}{前后}{qian2 hou4}[][HSK 3]
    \definition{s.}{em volta; sobre; um período de tempo ligeiramente anterior ou posterior a um horário específico| do início ao fim; refere-se ao período de tempo do início ao fim de algo | frente e verso; na frente e atrás de algo}
  \end{phonetics}
\end{entry}

\begin{entry}{前年}{9,6}{⼑、⼲}
  \begin{phonetics}{前年}{qian2 nian2}[][HSK 2]
    \definition{adv.}{há dois anos; dois anos atrás}
  \end{phonetics}
\end{entry}

\begin{entry}{前进}{9,7}{⼑、⾡}
  \begin{phonetics}{前进}{qian2 jin4}[][HSK 3]
    \definition{v.}{marchar; avançar; para ir em frente; seguir em frente; geralmente se refere ao desenvolvimento futuro}
  \end{phonetics}
\end{entry}

\begin{entry}{前往}{9,8}{⼑、⼻}
  \begin{phonetics}{前往}{qian2 wang3}[][HSK 3]
    \definition{v.}{ir para; prosseguir para; partir para; ir em frente}
  \end{phonetics}
\end{entry}

\begin{entry}{前面}{9,9}{⼑、⾯}
  \begin{phonetics}{前面}{qian2mian4}[][HSK 3]
    \definition{s.}{frente; a parte frontal do espaço ou posição | parte anterior; acima; a parte que vem primeiro na ordem; a parte de um artigo ou discurso que precede a narração atual}
  \end{phonetics}
\end{entry}

\begin{entry}{前途}{9,10}{⼑、⾡}
  \begin{phonetics}{前途}{qian2tu2}[][HSK 4]
    \definition[片,段,种]{s.}{futuro; perspectiva; prospecto; originalmente, refere-se à jornada à frente, mas, metaforicamente, refere-se ao futuro.}
  \end{phonetics}
\end{entry}

\begin{entry}{前提}{9,12}{⼑、⼿}
  \begin{phonetics}{前提}{qian2ti2}[][HSK 5]
    \definition[个,项]{s.}{premissa; pressuposto | pré-requisito; pressuposição; condições prévias para que algo aconteça ou se desenvolva}
  \end{phonetics}
\end{entry}

\begin{entry}{前景}{9,12}{⼑、⽇}
  \begin{phonetics}{前景}{qian2jing3}[][HSK 5]
    \definition{s.}{primeiro plano (de uma vista, imagem, foto, etc.); as imagens que parecem mais próximas do espectador em pinturas, palcos e telas | vista; perspectiva; prospecto; ponto de vista; situações que podem ocorrer no trabalho, na carreira, etc.}
  \end{phonetics}
\end{entry}

\begin{entry}{剑}{9}{⼑}
  \begin{phonetics}{剑}{jian4}
    \definition{clas.}{para golpes de uma espada}
    \definition[口,把]{s.}{espada de dois gumes}
  \end{phonetics}
\end{entry}

\begin{entry}{剑客}{9,9}{⼑、⼧}
  \begin{phonetics}{剑客}{jian4ke4}
    \definition{s.}{espada | esgrimista, espadachim}
  \end{phonetics}
\end{entry}

\begin{entry}{勇士}{9,3}{⼒、⼠}
  \begin{phonetics}{勇士}{yong3shi4}
    \definition{s.}{um guerreiro | uma pessoa corajosa}
  \end{phonetics}
\end{entry}

\begin{entry}{勇气}{9,4}{⼒、⽓}
  \begin{phonetics}{勇气}{yong3qi4}[][HSK 4]
    \definition[种,股]{s.}{coragem; arrojo; nervos; coragem para agir sem medo}
  \end{phonetics}
\end{entry}

\begin{entry}{勇敢}{9,11}{⼒、⽁}
  \begin{phonetics}{勇敢}{yong3gan3}[][HSK 4]
    \definition{adj.}{bravo; valente; galante; corajoso}
  \end{phonetics}
\end{entry}

\begin{entry}{南}{9}{⼗}
  \begin{phonetics}{南}{nan2}[][HSK 1]
    \definition*{s.}{sobrenome Nan}
    \definition{s.}{sul; uma das quatro direções básicas, o lado direito quando se está de frente para o sol pela manhã (oposto ao 北) | especificamente no sul da China}
  \seealsoref{北}{bei3}
  \end{phonetics}
\end{entry}

\begin{entry}{南方}{9,4}{⼗、⽅}
  \begin{phonetics}{南方}{nan2 fang1}[][HSK 2]
    \definition{s.}{sul; indica a direção sul | o sul; a região sul}
  \end{phonetics}
\end{entry}

\begin{entry}{南北}{9,5}{⼗、⼔}
  \begin{phonetics}{南北}{nan2 bei3}[][HSK 5]
    \definition{s.}{norte e sul | de norte a sul}
  \end{phonetics}
\end{entry}

\begin{entry}{南边}{9,5}{⼗、⾡}
  \begin{phonetics}{南边}{nan2 bian5}[][HSK 1]
    \definition{s.}{sul; lado sul}
  \end{phonetics}
\end{entry}

\begin{entry}{南极}{9,7}{⼗、⽊}
  \begin{phonetics}{南极}{nan2ji2}[][HSK 5]
    \definition*{s.}{Polo Sul; Polo Antártico | Polo sul magnético}
    \definition{s.}{pólo sul magnético}
  \end{phonetics}
\end{entry}

\begin{entry}{南面}{9,9}{⼗、⾯}
  \begin{phonetics}{南面}{nan2mian4}
    \definition{s.}{sul | lado sul}
  \end{phonetics}
\end{entry}

\begin{entry}{南部}{9,10}{⼗、⾢}
  \begin{phonetics}{南部}{nan2 bu4}[][HSK 3]
    \definition{s.}{parte sul; sul | a parte sul}
  \end{phonetics}
\end{entry}

\begin{entry}{厘米}{9,6}{⼚、⽶}
  \begin{phonetics}{厘米}{li2mi3}[][HSK 4]
    \definition{clas.}{centímetro; unidade de comprimento, símbolo cm, 1 metro é igual a 100 centímetros}
  \end{phonetics}
\end{entry}

\begin{entry}{厚}{9}{⼚}
  \begin{phonetics}{厚}{hou4}[][HSK 4]
    \definition*{s.}{sobrenome Hou}
    \definition{adj.}{grosso; espesso | profundo | bondoso; gentil; magnânimo | grande; generoso | rico ou forte em sabor}
    \definition{s.}{espessura; profundidade}
    \definition{v.}{favorecer; enfatizar}
  \end{phonetics}
\end{entry}

\begin{entry}{咬}{9}{⼝}
  \begin{phonetics}{咬}{yao3}[][HSK 5]
    \definition{v.}{morder; estalar; pressionar os dentes superiores e inferiores com força | latir | agarrar; morder | incriminar outra pessoa (geralmente inocente) quando culpada ou interrogada | pronunciar; articular; pronunciar corretamente | corroer (metais); irritar (a pele) | ser minucioso (com relação ao uso de palavras) | aproximar-se de; pressionar em direção a; avançar sobre}
  \end{phonetics}
\end{entry}

\begin{entry}{咱}{9}{⼝}
  \begin{phonetics}{咱}{za2}
  \end{phonetics}
  \begin{phonetics}{咱}{zan2}[][HSK 2]
    \definition{pron.}{nós; nos (incluindo tanto o falante quanto a pessoa ou pessoas às quais se dirige) | eu; mim |}
  \end{phonetics}
  \begin{phonetics}{咱}{zan5}
    \definition{adv.}{quando; agora; então; naquele momento; usado em 这咱, 那咱, 多咱, uma combinação das duas palavras 早晚}
  \seealsoref{多咱}{duo1 zan5}
  \seealsoref{那咱}{na4 zan5}
  \seealsoref{早晚}{zao3 wan3}
  \seealsoref{这咱}{zhe4 zan5}
  \end{phonetics}
\end{entry}

\begin{entry}{咱们}{9,5}{⼝、⼈}
  \begin{phonetics}{咱们}{zan2men5}[][HSK 2]
    \definition{pron.}{dirige-se tanto ao falante (eu, nós) quanto ao ouvinte (você, vocês) | eu; mim; refere-se ao próprio orador, eu}
  \end{phonetics}
\end{entry}

\begin{entry}{咱俩}{9,9}{⼝、⼈}
  \begin{phonetics}{咱俩}{zan2lia3}
    \definition{pron.}{nós dois}
  \end{phonetics}
\end{entry}

\begin{entry}{咱家}{9,10}{⼝、⼧}
  \begin{phonetics}{咱家}{za2jia1}
    \definition{pron.}{eu (frequentemente usado na literatura vernácula antiga) | me | mim | comigo}
  \end{phonetics}
\end{entry}

\begin{entry}{咳}{9}{⼝}
  \begin{phonetics}{咳}{hai1}
    \definition{interj.}{expressa tristeza, arrependimento ou espanto}
  \end{phonetics}
  \begin{phonetics}{咳}{ke2}[][HSK 5]
    \definition{v.}{tossir}
  \end{phonetics}
\end{entry}

\begin{entry}{咳嗽}{9,14}{⼝、⼝}
  \begin{phonetics}{咳嗽}{ke2sou5}
    \definition{v.}{ter tosse | tossir}
  \end{phonetics}
\end{entry}

\begin{entry}{咸}{9}{⼝}
  \begin{phonetics}{咸}{xian2}[][HSK 4]
    \definition*{s.}{sobrenome Xian}
    \definition{adj.}{salgado; em conserva; sabor salgado}
    \definition{adv.}{todos; indica a totalidade de um intervalo, equivalente a 全 e 都}
  \seealsoref{都}{dou1}
  \seealsoref{全}{quan2}
  \end{phonetics}
\end{entry}

\begin{entry}{咸水}{9,4}{⼝、⽔}
  \begin{phonetics}{咸水}{xian2shui3}
    \definition{s.}{salmora | água salgada}
  \end{phonetics}
\end{entry}

\begin{entry}{咸肉}{9,6}{⼝、⾁}
  \begin{phonetics}{咸肉}{xian2rou4}
    \definition{s.}{\emph{bacon} | carne curada com sal}
  \end{phonetics}
\end{entry}

\begin{entry}{咸鱼}{9,8}{⼝、⿂}
  \begin{phonetics}{咸鱼}{xian2yu2}
    \definition{s.}{peixe salgado}
  \end{phonetics}
\end{entry}

\begin{entry}{咸涩}{9,10}{⼝、⽔}
  \begin{phonetics}{咸涩}{xian2se4}
    \definition{s.}{ácido | salgado e amargo}
  \end{phonetics}
\end{entry}

\begin{entry}{咸盐}{9,10}{⼝、⽫}
  \begin{phonetics}{咸盐}{xian2yan2}
    \definition{s.}{(coloquial) sal | sal de mesa}
  \end{phonetics}
\end{entry}

\begin{entry}{咸淡}{9,11}{⼝、⽔}
  \begin{phonetics}{咸淡}{xian2dan4}
    \definition{s.}{água salobra | grau de salinidade | salgado e sem sal (sabores)}
  \end{phonetics}
\end{entry}

\begin{entry}{咸菜}{9,11}{⼝、⾋}
  \begin{phonetics}{咸菜}{xian2cai4}
    \definition{s.}{legumes salgados | \emph{pickles}}
  \end{phonetics}
\end{entry}

\begin{entry}{品}{9}{⼝}
  \begin{phonetics}{品}{pin3}[][HSK 5]
    \definition*{s.}{sobrenome Pin}
    \definition{s.}{artigo; produto | grau; classe; classificação; nível | caráter; qualidade | classificação; os graus dos funcionários públicos antigos, num total de nove graus}
    \definition{v.}{provar; saborear; degustar algo com discernimento | soprar; tocar (instrumentos de sopro) | avaliar; distinguir o bom do ruim}
  \end{phonetics}
\end{entry}

\begin{entry}{品质}{9,8}{⼝、⾙}
  \begin{phonetics}{品质}{pin3zhi4}[][HSK 4]
    \definition[个,种]{s.}{qualidade; caráter; natureza do pensamento, da compreensão, do caráter, etc., conforme expresso no comportamento, no estilo, etc. | qualidade (de produtos, mercadorias, etc.)}
  \end{phonetics}
\end{entry}

\begin{entry}{品种}{9,9}{⼝、⽲}
  \begin{phonetics}{品种}{pin3zhong3}[][HSK 5]
    \definition[个]{s.}{raça; linhagem; variedade; refere-se a um grupo de organismos com características genéticas comuns, formados por meio da seleção e cultivo artificiais de culturas, gado, aves, etc. | variedade; sortimento; referência geral ao tipo de item}
  \end{phonetics}
\end{entry}

\begin{entry}{品德}{9,15}{⼝、⼻}
  \begin{phonetics}{品德}{pin3de2}
    \definition{s.}{caráter moral | moralidade}
  \end{phonetics}
\end{entry}

\begin{entry}{哄}{9}{⼝}
  \begin{phonetics}{哄}{hong1}
    \definition{s.}{gargalhadas | risadas ruidosas | algazarra | rugido | clamor}
  \end{phonetics}
  \begin{phonetics}{哄}{hong3}
    \definition{v.}{enganar | persuadir | divertir (uma criança)}
  \end{phonetics}
  \begin{phonetics}{哄}{hong4}
    \definition{s.}{tumulto | agitação | perturbação}
  \end{phonetics}
\end{entry}

\begin{entry}{哇塞}{9,13}{⼝、⼟}
  \begin{phonetics}{哇塞}{wa1sai1}
    \definition{interj.}{(gíria) Uau!}
  \end{phonetics}
\end{entry}

\begin{entry}{哇噻}{9,16}{⼝、⼝}
  \begin{phonetics}{哇噻}{wa1sai1}
    \variantof{哇塞}
  \end{phonetics}
\end{entry}

\begin{entry}{哈}{9}{⼝}
  \begin{phonetics}{哈}{ha1}
    \definition{interj.}{(onomatopeia) ha; descreve o riso, usado principalmente em duplicata | indica orgulho ou satisfação, frequentemente usado de forma duplicada}
    \definition{v.}{soprar; expirar (com a boca aberta) | dobrar}
  \seealsoref{哈哈}{ha1 ha1}
  \end{phonetics}
  \begin{phonetics}{哈}{ha3}
    \definition*{s.}{sobrenome Ha}
    \definition{v.}{repreender}
  \end{phonetics}
\end{entry}

\begin{entry}{哈马斯}{9,3,12}{⼝、⾺、⽄}
  \begin{phonetics}{哈马斯}{ha1ma3si1}
    \definition*{s.}{Hamas (Grupo Palestino)}
  \end{phonetics}
\end{entry}

\begin{entry}{哈哈}{9,9}{⼝、⼝}
  \begin{phonetics}{哈哈}{ha1 ha1}[][HSK 3]
    \definition{expr.}{(onomatopéia)  ha ha; o som de uma gargalhada}
  \end{phonetics}
\end{entry}

\begin{entry}{响}{9}{⼝}
  \begin{phonetics}{响}{xiang3}[][HSK 2]
    \definition{adj.}{barulhento; ressonante}
    \definition[声,阵]{s.}{som; ruído; barulho | eco}
    \definition{v.}{tocar; soar; ressoar; fazer um som | soar; fazer algo emitir um som}
  \end{phonetics}
\end{entry}

\begin{entry}{哪}{9}{⼝}
  \begin{phonetics}{哪}{na3}[][HSK 1,4]
    \definition{adv.}{para expressar uma pergunta retórica, indicando que é impossível}
    \definition{pron.}{qual?; o que?; expressa a necessidade de determinar um entre várias pessoas ou coisas | qualquer; ser usado em um sentido geral | qual?; o que?; (usado sozinho, o mesmo que 什么, frequentemente usado de forma intercambiável com 什么) | qualquer; qualquer que seja; refere-se a qualquer um, geralmente seguido por 都 ou 也, ou usando dois 哪 antes e depois | qual (indica algo incerto)}
  \seealsoref{都}{dou1}
  \seealsoref{什么}{shen2me5}
  \seealsoref{也}{ye3}
  \end{phonetics}
  \begin{phonetics}{哪}{na5}
    \definition{part.}{usado depois de uma palavra com a terminação -n, é equivalente a 啊}
  \seealsoref{啊}{a5}
  \end{phonetics}
  \begin{phonetics}{哪}{nei3}
    \definition{part.}{qual? (interrogativo, seguido de classificador ou numeral-classificador)}
  \end{phonetics}
\end{entry}

\begin{entry}{哪儿}{9,2}{⼝、⼉}
  \begin{phonetics}{哪儿}{na3r5}[][HSK 1]
    \definition{adv.}{usado para perguntas retóricas, indicando negação}
    \definition{pron.}{onde? | onde quer que seja; em qualquer lugar | usado como uma resposta educada a um elogio}
  \end{phonetics}
\end{entry}

\begin{entry}{哪个}{9,3}{⼝、⼈}
  \begin{phonetics}{哪个}{na3ge5}
    \definition{pron.}{qual deles (pergunta sobre o objeto) | quem (perguntar a alguém ou indicar qualquer pessoa)}
  \end{phonetics}
\end{entry}

\begin{entry}{哪里}{9,7}{⼝、⾥}
  \begin{phonetics}{哪里}{na3 li3}[][HSK 1]
    \definition{adv.}{usado em perguntas retóricas para expressar um significado negativo}
    \definition{pron.}{onde?; em que lugar? | onde quer que seja; em qualquer lugar | usado como uma resposta educada a um elogio}
  \end{phonetics}
\end{entry}

\begin{entry}{哪些}{9,8}{⼝、⼆}
  \begin{phonetics}{哪些}{na3xie1}[][HSK 1]
    \definition{pron.}{quais?}
  \end{phonetics}
\end{entry}

\begin{entry}{哪国人}{9,8,2}{⼝、⼞、⼈}
  \begin{phonetics}{哪国人}{na3 guo2ren2}
    \definition{expr.}{de qual país?}
  \end{phonetics}
\end{entry}

\begin{entry}{哪怕}{9,8}{⼝、⼼}
  \begin{phonetics}{哪怕}{na3pa4}[][HSK 4]
    \definition{conj.}{mesmo; mesmo se; mesmo que; não importa o quão}
  \end{phonetics}
\end{entry}

\begin{entry}{型}{9}{⼟}
  \begin{phonetics}{型}{xing2}[][HSK 4]
    \definition{s.}{molde; modelo | modelo; tipo; padrão}
  \end{phonetics}
\end{entry}

\begin{entry}{型号}{9,5}{⼟、⼝}
  \begin{phonetics}{型号}{xing2 hao4}[][HSK 4]
    \definition[个,种]{s.}{modelo; tipo; refere-se ao desempenho, às especificações e ao tamanho de aeronaves, máquinas, implementos agrícolas, etc.}
  \end{phonetics}
\end{entry}

\begin{entry}{垫}{9}{⼟}
  \begin{phonetics}{垫}{dian4}
    \definition[个]{s.}{almofada}
    \definition{v.}{colocar algo sob; elevar ou nivelar; encher; preencher | pagar por alguém e esperar ser reembolsado mais tarde | colocar algo sob algo para elevá-lo ou nivelá-lo; usar algo para apoiar, espalhar ou forrar algo para torná-lo mais alto, mais grosso ou mais plano | preencher uma vaga; preencher uma lacuna}
  \end{phonetics}
\end{entry}

\begin{entry}{垫子}{9,3}{⼟、⼦}
  \begin{phonetics}{垫子}{dian4zi5}
    \definition{s.}{colchão | esteira | almofada}
  \end{phonetics}
\end{entry}

\begin{entry}{城}{9}{⼟}
  \begin{phonetics}{城}{cheng2}[][HSK 3]
    \definition*{s.}{sobrenome Cheng}
    \definition[座,道,个]{s.}{muralha da cidade; muralha | cidade | centro de um determinado tipo (por exemplo, negócios, entretenimento, etc.)}
  \end{phonetics}
\end{entry}

\begin{entry}{城市}{9,5}{⼟、⼱}
  \begin{phonetics}{城市}{cheng2shi4}[][HSK 3]
    \definition[个,座]{s.}{cidade; regiões com alta densidade populacional, comércio e indústria desenvolvidos e cuja população é predominantemente não agrícola são geralmente centros políticos, econômicos e culturais das regiões vizinhas}
  \end{phonetics}
\end{entry}

\begin{entry}{城里}{9,7}{⼟、⾥}
  \begin{phonetics}{城里}{cheng2 li3}[][HSK 5]
    \definition{s.}{na cidade; dentro da cidade; originalmente referia-se à área dentro das muralhas da cidade, agora refere-se principalmente à área urbana}
  \end{phonetics}
\end{entry}

\begin{entry}{城度}{9,9}{⼟、⼴}
  \begin{phonetics}{城度}{cheng2du4}[][HSK 3]
    \definition*{s.}{Cidade}
  \end{phonetics}
\end{entry}

\begin{entry}{城堡}{9,12}{⼟、⼟}
  \begin{phonetics}{城堡}{cheng2bao3}
    \definition[座,个]{s.}{forte; castelo; cidadela; uma pequena cidade com muralhas que facilitam a defesa}
  \end{phonetics}
\end{entry}

\begin{entry}{复}{9}{⼢}
  \begin{phonetics}{复}{fu4}
    \definition*{s.}{sobrenome Fu}
    \definition{adj.}{composto; complexo; nem um único; dois ou mais}
    \definition{adv.}{de novo; novamente; indica o reaparecimento de uma situação, equivalente a 再}
    \definition{s.}{jaqueta; roupas forradas}
    \definition{v.}{virar; virar-se | responder; retornar | recuperar; retornar a; restaurar | vingar | duplicar; repetir}
  \seealsoref{再}{zai4}
  \end{phonetics}
\end{entry}

\begin{entry}{复习}{9,3}{⼢、⼄}
  \begin{phonetics}{复习}{fu4xi2}[][HSK 2]
    \definition{s.}{revisão}
    \definition{v.}{revisar; corrigir (lições, etc.); repetir o que já aprendeu para consolidar o conhecimento}
  \end{phonetics}
\end{entry}

\begin{entry}{复印}{9,5}{⼢、⼙}
  \begin{phonetics}{复印}{fu4yin4}[][HSK 3]
    \definition{v.}{fotografar; fotocopiar; duplicar; sem passar pelo processo de impressão, obter uma cópia diretamente do original (geralmente referindo-se à cópia feita com uma copiadora)}
  \end{phonetics}
\end{entry}

\begin{entry}{复杂}{9,6}{⼢、⽊}
  \begin{phonetics}{复杂}{fu4za2}[][HSK 3]
    \definition{adj.}{complexo; complicado; em oposição a 单纯 e 简单}
  \seealsoref{单纯}{dan1chun2}
  \seealsoref{简单}{jian3dan1}
  \end{phonetics}
\end{entry}

\begin{entry}{复制}{9,8}{⼢、⼑}
  \begin{phonetics}{复制}{fu4zhi4}[][HSK 4]
    \definition{v.}{copiar; duplicar; reproduzir; fazer uma cópia de; fazer uma cópia do original ou reproduzi-lo, reimprimi-lo ou copiá-lo em sua forma original (geralmente referindo-se a relíquias culturais ou obras de arte)}
  \end{phonetics}
\end{entry}

\begin{entry}{复刻}{9,8}{⼢、⼑}
  \begin{phonetics}{复刻}{fu4ke4}
    \definition{v.}{reimprimir (um trabalho que esteve fora do catálogo) | reeditar (um disco de vinil, um CD, etc.) | replicar | recriar | (empréstimo linguístico) (computação) \emph{fork}}
  \end{phonetics}
\end{entry}

\begin{entry}{复活节}{9,9,5}{⼢、⽔、⾋}
  \begin{phonetics}{复活节}{fu4huo2jie2}
    \definition*{s.}{Páscoa}
  \end{phonetics}
\end{entry}

\begin{entry}{奏效}{9,10}{⼤、⽁}
  \begin{phonetics}{奏效}{zou4xiao4}
    \definition{v.}{mostrar resultados | ser eficaz}
  \end{phonetics}
\end{entry}

\begin{entry}{奖}{9}{⼤}
  \begin{phonetics}{奖}{jiang3}[][HSK 4]
    \definition[个,次]{s.}{prêmio; recompensa | elogio; loa}
    \definition{v.}{elogiar; recompensar; recomendar; incentivar}
  \end{phonetics}
\end{entry}

\begin{entry}{奖励}{9,7}{⼤、⼒}
  \begin{phonetics}{奖励}{jiang3li4}[][HSK 5]
    \definition{s.}{prêmio; recompensa; dinheiro ou honras dadas em troca de elogios ou incentivos}
    \definition{v.}{recompensar; incentivar; encorajar}
  \end{phonetics}
\end{entry}

\begin{entry}{奖学金}{9,8,8}{⼤、⼦、⾦}
  \begin{phonetics}{奖学金}{jiang3 xue2 jin1}[][HSK 4]
    \definition[笔]{s.}{bolsa de estudos; exposição; prêmios concedidos por escolas, organizações ou indivíduos a alunos com bom desempenho acadêmico}
  \end{phonetics}
\end{entry}

\begin{entry}{奖金}{9,8}{⼤、⾦}
  \begin{phonetics}{奖金}{jiang3jin1}[][HSK 4]
    \definition[个,笔]{s.}{bônus; recompensa; prêmio; prêmio em dinheiro; dinheiro de recompensa, dinheiro dado às pessoas para incentivá-las ou elogiá-las por terem se saído bem em alguma coisa}
  \end{phonetics}
\end{entry}

\begin{entry}{姜}{9}{⼥}
  \begin{phonetics}{姜}{jiang1}
    \definition*{s.}{sobrenome Jiang}
    \definition{s.}{gengibre}
  \end{phonetics}
\end{entry}

\begin{entry}{孩子}{9,3}{⼦、⼦}
  \begin{phonetics}{孩子}{hai2 zi5}[][HSK 1]
    \definition[个]{s.}{criança; crianças; pessoas com idade entre alguns anos ou na adolescência, geralmente com menos de 14 anos | crianças; filho ou filha}
  \end{phonetics}
\end{entry}

\begin{entry}{客人}{9,2}{⼧、⼈}
  \begin{phonetics}{客人}{ke4ren2}[][HSK 2]
    \definition[位,个,桌,拨,批]{s.}{visitante; convidado | cliente; passageiro; hóspede; viajante}
  \end{phonetics}
\end{entry}

\begin{entry}{客厅}{9,4}{⼧、⼚}
  \begin{phonetics}{客厅}{ke4ting1}[][HSK 5]
    \definition[间,个]{s.}{sala de estar; sala de visitas; sala para receber convidados}
  \end{phonetics}
\end{entry}

\begin{entry}{客户}{9,4}{⼧、⼾}
  \begin{phonetics}{客户}{ke4hu4}[][HSK 5]
    \definition{s.}{cliente; consumidor}
  \end{phonetics}
\end{entry}

\begin{entry}{客气}{9,4}{⼧、⽓}
  \begin{phonetics}{客气}{ke4qi5}[][HSK 5]
    \definition{adj.}{educado; modesto; cortês}
    \definition{v.}{ser educado; ser cortês; fazer comentários educados ou agir educadamente}
  \end{phonetics}
\end{entry}

\begin{entry}{客观}{9,6}{⼧、⾒}
  \begin{phonetics}{客观}{ke4guan1}[][HSK 3]
    \definition{adj.}{objetivo; justo e razoável; imparcial; com base na situação real, sem preconceitos pessoais}
    \definition{s.}{objetivo; existe fora da consciência, sem depender da consciência subjetiva}
  \end{phonetics}
\end{entry}

\begin{entry}{宣布}{9,5}{⼧、⼱}
  \begin{phonetics}{宣布}{xuan1bu4}[][HSK 3]
    \definition{v.}{declarar; proclamar; pronunciar; anunciar; informar oficialmente a todos sobre as últimas decisões e situações}
  \end{phonetics}
\end{entry}

\begin{entry}{宣传}{9,6}{⼧、⼈}
  \begin{phonetics}{宣传}{xuan1chuan2}[][HSK 3]
    \definition[个]{v.}{propagar; divulgar; fazer propaganda; explicar e esclarecer às pessoas, para que elas acreditem e sigam as ações}
  \end{phonetics}
\end{entry}

\begin{entry}{宣扬}{9,6}{⼧、⼿}
  \begin{phonetics}{宣扬}{xuan1yang2}
    \definition{v.}{divulgar | anunciar | espalhar por toda parte}
  \end{phonetics}
\end{entry}

\begin{entry}{室}{9}{⼧}
  \begin{phonetics}{室}{shi4}[][HSK 3]
    \definition*{s.}{sobrenome Shi}
    \definition*{s.}{Shi, a décima terceira das vinte e oito constelações da esfera celeste, composta por duas estrelas em linha reta na constelação de Pégaso}
    \definition{s.}{sala; quarto; casa | departamento; sala como unidade administrativa ou de trabalho; órgãos públicos, fábricas, escolas e outras unidades de trabalho internas | esposa; familiares ou esposa | família; clã | cavidade; órgão com forma semelhante a uma câmara}
  \end{phonetics}
\end{entry}

\begin{entry}{宪制}{9,8}{⼧、⼑}
  \begin{phonetics}{宪制}{xian4zhi4}
    \definition{adj.}{constitucional}
    \definition{s.}{sistema de governo constitucional}
  \end{phonetics}
\end{entry}

\begin{entry}{宪法法院}{9,8,8,9}{⼧、⽔、⽔、⾩}
  \begin{phonetics}{宪法法院}{xian4fa3fa3yuan4}
    \definition{s.}{tribunal constitucional}
  \end{phonetics}
\end{entry}

\begin{entry}{宪政}{9,9}{⼧、⽁}
  \begin{phonetics}{宪政}{xian4zheng4}
    \definition{s.}{governo constitucional}
  \end{phonetics}
\end{entry}

\begin{entry}{封}{9}{⼨}
  \begin{phonetics}{封}{feng1}[][HSK 2,5]
    \definition*{s.}{sobrenome Feng}
    \definition{clas.}{usado para objetos selados, especialmente cartas}
    \definition{s.}{feudalismo | embalagem; envelope | pacote}
    \definition{v.}{conferir (um título, território, etc.) a | selar | acender uma fogueira | fechar}
  \end{phonetics}
\end{entry}

\begin{entry}{封口}{9,3}{⼨、⼝}
  \begin{phonetics}{封口}{feng1kou3}
    \definition{v.}{selar | fechar | curar (uma ferida) | manter os lábios selados}
  \end{phonetics}
\end{entry}

\begin{entry}{封印}{9,5}{⼨、⼙}
  \begin{phonetics}{封印}{feng1yin4}
    \definition{s.}{selo (em envelopes)}
  \end{phonetics}
\end{entry}

\begin{entry}{封闭}{9,6}{⼨、⾨}
  \begin{phonetics}{封闭}{feng1bi4}[][HSK 4]
    \definition{adj.}{fechado; aqueles que não têm contato com o mundo exterior; aqueles que são muito conservadores (em seu pensamento) e não se comunicam com os outros}
    \definition{v.}{selar; fechar; lacrar; vedar; de modo a impedir a passagem, o uso ou a abertura}
  \end{phonetics}
\end{entry}

\begin{entry}{封冻}{9,7}{⼨、⼎}
  \begin{phonetics}{封冻}{feng1dong4}
    \definition{v.}{congelar (água ou terra)}
  \end{phonetics}
\end{entry}

\begin{entry}{封底}{9,8}{⼨、⼴}
  \begin{phonetics}{封底}{feng1di3}
    \definition{s.}{contracapa de um livro}
  \end{phonetics}
\end{entry}

\begin{entry}{封建}{9,8}{⼨、⼵}
  \begin{phonetics}{封建}{feng1jian4}
    \definition{adj.}{feudal}
    \definition{s.}{feudalismo}
  \end{phonetics}
\end{entry}

\begin{entry}{封面}{9,9}{⼨、⾯}
  \begin{phonetics}{封面}{feng1mian4}
    \definition{s.}{capa (de uma publicação) | sobrecapa}
  \end{phonetics}
\end{entry}

\begin{entry}{封斋}{9,10}{⼨、⽂}
  \begin{phonetics}{封斋}{feng1zhai1}
    \definition*{s.}{Ramadã (Islã)}
  \end{phonetics}
\end{entry}

\begin{entry}{封盖}{9,11}{⼨、⽫}
  \begin{phonetics}{封盖}{feng1gai4}
    \definition{s.}{boné | capa | selo}
    \definition{v.}{cobrir}
  \end{phonetics}
\end{entry}

\begin{entry}{将}{9}{⼨}
  \begin{phonetics}{将}{jiang1}[][HSK 5]
    \definition*{s.}{sobrenome Jiang}
    \definition{adv.}{estar indo para; parcialmente\dots parcialmente\dots}
    \definition{part.}{expressar uma direção, como 进来, 出去; usado no meio de verbos e complementos que indicam tendência, como 进来, 出去, etc.}
    \definition{prep.}{com; por meio de; por | usado da mesma forma que 把}
    \definition{v.}{fazer algo; lidar com (um assunto) | dar um cheque-mate | cuidar (da saúde) | incitar alguém a agir; desafiar; estimular | segurar; pegar | colocar; tirar | levar; trazer | dar suporte; dar apoio}
  \seealsoref{把}{ba3}
  \seealsoref{出去}{chu1 qu4}
  \seealsoref{进来}{jin4 lai2}
  \end{phonetics}
  \begin{phonetics}{将}{jiang4}
    \definition{s.}{general; nome do posto; abaixo de marechal de campo; acima de coronel}
    \definition{v.}{comandar; liderar}
  \end{phonetics}
  \begin{phonetics}{将}{qiang1}
    \definition{v.}{pedir; apelar para}
  \end{phonetics}
\end{entry}

\begin{entry}{将来}{9,7}{⼨、⽊}
  \begin{phonetics}{将来}{jiang1lai2}[][HSK 3]
    \definition[个]{s.}{no futuro (geralmente se refere a um período mais longo)}
  \end{phonetics}
\end{entry}

\begin{entry}{将近}{9,7}{⼨、⾡}
  \begin{phonetics}{将近}{jiang1jin4}[][HSK 3]
    \definition{adv.}{quase}
  \end{phonetics}
\end{entry}

\begin{entry}{将要}{9,9}{⼨、⾑}
  \begin{phonetics}{将要}{jiang1 yao4}[][HSK 5]
    \definition{adv.}{irá; deverá; estará prestes a; irá a; indica que um ato ou situação ocorre logo em seguida}
  \end{phonetics}
\end{entry}

\begin{entry}{尝}{9}{⼩}
  \begin{phonetics}{尝}{chang2}[][HSK 5]
    \definition{adv.}{alguma vez; uma vez}
    \definition{v.}{provar; experimentar o sabor de | provar; experimentar; conhecer | tentar; testar}
  \end{phonetics}
\end{entry}

\begin{entry}{尝试}{9,8}{⼩、⾔}
  \begin{phonetics}{尝试}{chang2shi4}[][HSK 5]
    \definition{v.}{tentar; provar; experimentar}
  \end{phonetics}
\end{entry}

\begin{entry}{屋}{9}{⼫}
  \begin{phonetics}{屋}{wu1}[][HSK 5]
    \definition[间,座]{s.}{casa | quarto}
  \end{phonetics}
\end{entry}

\begin{entry}{屋子}{9,3}{⼫、⼦}
  \begin{phonetics}{屋子}{wu1zi5}[][HSK 3]
    \definition[间,座,栋]{s.}{quarto; sala}
  \end{phonetics}
\end{entry}

\begin{entry}{屌}{9}{⼫}
  \begin{phonetics}{屌}{diao3}
    \definition{adj.}{(gíria) legal ou extraordinário}
    \definition{s.}{órgão genital masculino; pênis}
    \definition{v.}{(cantonês) foder}
  \end{phonetics}
\end{entry}

\begin{entry}{屌丝}{9,5}{⼫、⼀}
  \begin{phonetics}{屌丝}{diao3si1}
    \definition{adj.}{panaca | zé-ninguém | (gíria de \emph{Internet}) \emph{looser}}
  \end{phonetics}
\end{entry}

\begin{entry}{屎}{9}{⼫}
  \begin{phonetics}{屎}{shi3}
    \definition{s.}{fezes | excrementos | (forma ligada) secreção (do ouvido, olho, etc.)}
  \end{phonetics}
\end{entry}

\begin{entry}{差}{9}{⼯}
  \begin{phonetics}{差}{cha1}
    \definition{adj.}{diferente; diferente ou inconsistente com um determinado padrão}
    \definition{adv.}{ligeiramente; comparativamente; um pouco}
    \definition{s.}{diferença; resto após a subtração de dois números | erro; engano}
  \end{phonetics}
  \begin{phonetics}{差}{cha4}[][HSK 1]
    \definition{adj.}{não está de acordo com o padrão; pobre; ruim; inferior | errado; incorreto | mesmo significado de 差 \dpy{cha1}}
    \definition{v.}{faltar}
  \end{phonetics}
  \begin{phonetics}{差}{chai1}
    \definition{s.}{tarefa; trabalho; ser enviado para fazer algo; deveres oficiais; posição | corvéia; mensageiro ou oficial de justiça em um yamen feudal; (velho) refere-se a pessoas que são enviadas para fazer coisas}
    \definition{v.}{enviar uma mensagem; despachar; fnviar (para fazer algo)}
  \end{phonetics}
\end{entry}

\begin{entry}{差(一)点儿}{9,1,9,2}{⼯、⼀、⽕、⼉}
  \begin{phonetics}{差(一)点儿}{cha1yi4dian3r5}[][HSK 5]
    \definition{adv.}{quase; à beira de; praticamente; aproximadamente; significa que algo está perto de ser alcançado, mas não foi alcançado, ou algo foi alcançado, mas mal foi alcançado}
  \end{phonetics}
\end{entry}

\begin{entry}{差不多}{9,4,6}{⼯、⼀、⼣}
  \begin{phonetics}{差不多}{cha4bu5duo1}[][HSK 2]
    \definition{adj.}{semelhante; aproximadamente igual | não muito longe; quase certo (suficiente); basicamente, próximo dos padrões e requisitos; normal | prestes a (terminar; acabar); descreve que (algo) está quase acabando; (uma tarefa) está quase concluída}
    \definition{adv.}{quase; perto; indica proximidade}
  \end{phonetics}
\end{entry}

\begin{entry}{差别}{9,7}{⼯、⼑}
  \begin{phonetics}{差别}{cha1bie2}[][HSK 5]
    \definition{s.}{diferença; disparidade; dissimilaridade; distinção; não semelhança; diferenças na forma ou no conteúdo}
  \end{phonetics}
\end{entry}

\begin{entry}{差点儿}{9,9,2}{⼯、⽕、⼉}
  \begin{phonetics}{差点儿}{cha4dian3r5}
    \definition{adv.}{por pouco | por um triz | quase}
  \end{phonetics}
\end{entry}

\begin{entry}{差距}{9,11}{⼯、⾜}
  \begin{phonetics}{差距}{cha1ju4}[][HSK 5]
    \definition[个,些,段]{s.}{lacuna; disparidade; discrepância; diferença; grau de diferença entre as coisas, especialmente em termos de distância de algum padrão.}
  \end{phonetics}
\end{entry}

\begin{entry}{帝}{9}{⼱}
  \begin{phonetics}{帝}{di4}
    \definition*{s.}{Ser Supremo; Deus}
    \definition[位,名,个]{s.}{imperador | (abreviação) imperialismo}
  \end{phonetics}
\end{entry}

\begin{entry}{帝国}{9,8}{⼱、⼞}
  \begin{phonetics}{帝国}{di4guo2}
    \definition{adj.}{imperial}
    \definition{s.}{império}
  \end{phonetics}
\end{entry}

\begin{entry}{带}{9}{⼱}
  \begin{phonetics}{带}{dai4}[][HSK 2]
    \definition*{s.}{sobrenome Dai}
    \definition[根]{s.}{cinto; faixa; banda; fita; fita adesiva; algo parecido com uma fita | pneu | zona; área; faixa; cinturão; região; uma determinada área geográfica com determinadas características | leucorreia; corrimento branco; corrimento vaginal}
    \definition{v.}{levar; trazer; transportar | liderar; dirigir; conduzir; assumir | cuidar de crianças; criar filhos; educar | fazer uma coisa e, ao mesmo tempo, fazer outra coisa |suportar; conter | ter algo anexado, simultâneo | trazer consigo | carregar consigo | demonstrar; parecer | incluir; acrescentar}
  \end{phonetics}
\end{entry}

\begin{entry}{带动}{9,6}{⼱、⼒}
  \begin{phonetics}{带动}{dai4 dong4}[][HSK 3]
    \definition{v.}{dirigir; ativar; fazer algo funcionar; acionar | liderar; trazer; estimular; motivar; atrair; liderar o avanço; dar o exemplo e fazer com que os outros sigam o exemplo}
  \end{phonetics}
\end{entry}

\begin{entry}{带有}{9,6}{⼱、⽉}
  \begin{phonetics}{带有}{dai4 you3}[][HSK 5]
    \definition{v.}{ter; envolver; carregar}
  \end{phonetics}
\end{entry}

\begin{entry}{带来}{9,7}{⼱、⽊}
  \begin{phonetics}{带来}{dai4 lai2}[][HSK 2]
    \definition{v.}{provocar; produzir; causar}
  \end{phonetics}
\end{entry}

\begin{entry}{带领}{9,11}{⼱、⾴}
  \begin{phonetics}{带领}{dai4ling3}[][HSK 3]
    \definition{v.}{guiar, na frente, liderando | liderar e comandar}
  \end{phonetics}
\end{entry}

\begin{entry}{帮}{9}{⼱}
  \begin{phonetics}{帮}{bang1}[][HSK 1]
    \definition*{s.}{sobrenome Bang}
    \definition{clas.}{um grupo de; um bando de; uma gangue de; um grupo de pessoas}
    \definition{s.}{lateral; superior; partes ao lado ou ao redor do objeto | folha externa; parte mais grossa das folhas externas dos vegetais | gangue; banda; grupo; conglomerado}
    \definition{v.}{ajudar; assistir; auxiliar | trabalho; refere-se ao envolvimento em trabalho assalariado}
  \end{phonetics}
\end{entry}

\begin{entry}{帮忙}{9,6}{⼱、⼼}
  \begin{phonetics}{帮忙}{bang1 mang2}[][HSK 1]
    \definition{v.+compl.}{ajudar; dar uma mão; dar uma mãozinha; fazer um favor; fazer uma boa ação; ajudar os outros a fazer algo, referindo-se, de maneira geral, a oferecer ajuda quando alguém está com dificuldades}
  \end{phonetics}
\end{entry}

\begin{entry}{帮佣}{9,7}{⼱、⼈}
  \begin{phonetics}{帮佣}{bang1yong1}
    \definition{s.}{ajudante doméstico | servo}
  \end{phonetics}
\end{entry}

\begin{entry}{帮助}{9,7}{⼱、⼒}
  \begin{phonetics}{帮助}{bang1zhu4}[][HSK 2]
    \definition[个,次,回,份,种]{s.}{ajuda; auxílio; socorro; função de promoção ou auxílio}
    \definition{v.}{ajudar; assistir; apoiar; quando alguém está passando por dificuldades, oferecer apoio financeiro ou material, ou ainda apoio moral, dar conselhos, pensar em soluções, fazer coisas por essa pessoa, etc.}
  \end{phonetics}
\end{entry}

\begin{entry}{帮教}{9,11}{⼱、⽁}
  \begin{phonetics}{帮教}{bang1jiao4}
    \definition{v.}{orientar}
  \end{phonetics}
\end{entry}

\begin{entry}{幽默}{9,16}{⼳、⿊}
  \begin{phonetics}{幽默}{you1mo4}[][HSK 5]
    \definition{adj.}{humorístico; interessante ou engraçado, mas com um significado profundo}
    \definition{s.}{humor; lado engraçado; graça; características, temperamento, palavras ou comportamentos interessantes, engraçados ou significativos}
  \end{phonetics}
\end{entry}

\begin{entry}{度}{9}{⼴}
  \begin{phonetics}{度}{du4}[][HSK 2]
    \definition*{s.}{sobrenome Du}
    \definition{clas.}{grau; unidade de medida para ângulos, temperatura, etc. | quilowatt-hora (kWh) | usado para indicar a quantidade de álcool presente no vinho | usado para arcos e ângulos | usado para indicar o grau de curvatura da lente dos óculos ou o grau de miopia | tempo; número de vezes | usado para longitude e latitude, localização geográfica}
    \definition{s.}{medida linear; padrões e instrumentos para medir comprimentos | grau de intensidade; refere-se especificamente ao grau alcançado por uma determinada propriedade de uma coisa | limite; extensão; grau; quota | regras; código de conduta; diretrizes | tolerância; magnanimidade; refere-se especificamente ao grau de tolerância | maneira; temperamento; disposição; a personalidade ou aparência de uma pessoa | indicador de grau, nível alcançado por algo | tempo ou espaço limitado; um determinado período de tempo ou espaço}
    \definition{v.}{passar; atravessar; passar por cima | (em termos de tempo) passar; passar por | (de monges ou monjas budistas, ou sacerdotes taoístas) pregar; converter; proselitar}
  \end{phonetics}
  \begin{phonetics}{度}{duo2}
    \definition{v.}{supor; estimar; especular}
  \end{phonetics}
\end{entry}

\begin{entry}{度过}{9,6}{⼴、⾡}
  \begin{phonetics}{度过}{du4guo4}[][HSK 4]
    \definition{s.}{passar o tempo; fazer o tempo desaparecer no trabalho, na vida, no lazer e no descanso}
  \end{phonetics}
\end{entry}

\begin{entry}{弯}{9}{⼸}
  \begin{phonetics}{弯}{wan1}[][HSK 4]
    \definition{adj.}{curvo; dobrado; torto; flexível; tortuoso}
    \definition{s.}{curva; dobra}
    \definition{v.}{curvar; dobrar; flexionar}
  \end{phonetics}
\end{entry}

\begin{entry}{待}{9}{⼻}
  \begin{phonetics}{待}{dai1}[][HSK 5]
    \definition{v.}{ficar; permanecer | ir além (de um período de tempo)}
  \end{phonetics}
  \begin{phonetics}{待}{dai4}
    \definition*{s.}{sobrenome Dai}
    \definition{v.}{tratar; lidar com | entreter; receber (convidados) | aguardar; esperar por | precisar; necessitar | desejar; pretender; querer}
  \end{phonetics}
\end{entry}

\begin{entry}{待遇}{9,12}{⼻、⾡}
  \begin{phonetics}{待遇}{dai4yu4}[][HSK 4]
    \definition[种,项,份]{s.}{tratamento; refere-se a direitos, status social, etc. | salário; ordenado; remuneração}
  \end{phonetics}
\end{entry}

\begin{entry}{很}{9}{⼻}
  \begin{phonetics}{很}{hen3}[][HSK 1]
    \definition{adv.}{muito; bastante; terrivelmente; indica um grau bastante elevado; definitivo; o mais alto}
  \end{phonetics}
\end{entry}

\begin{entry}{律师}{9,6}{⼻、⼱}
  \begin{phonetics}{律师}{lv4shi1}[][HSK 4]
    \definition[名,个,位]{s.}{advogado; procurador; profissionais encarregados pelas partes ou nomeados pelo tribunal para auxiliar as partes no litígio, para comparecer ao tribunal para defesa e para tratar de assuntos jurídicos relacionados, de acordo com a lei}
  \end{phonetics}
\end{entry}

\begin{entry}{怎}{9}{⼼}
  \begin{phonetics}{怎}{zen3}
    \definition{adv.}{como}
  \end{phonetics}
\end{entry}

\begin{entry}{怎么}{9,3}{⼼、⼃}
  \begin{phonetics}{怎么}{zen3me5}[][HSK 1]
    \definition{pron.}{como?; o quê?; perguntas sobre natureza, situação, método, motivo, etc. | de qualquer maneira; não importa como; de uma certa maneira; referência geral à natureza, condição ou modo | que? (usado sozinho no início de uma frase para expressar surpresa) | usado após 不 e 没, indica um grau baixo e é uma forma mais educada de se expressar | usado em perguntas retóricas}
  \seealsoref{不}{bu4}
  \seealsoref{没}{mei2}
  \end{phonetics}
\end{entry}

\begin{entry}{怎么了}{9,3,2}{⼼、⼃、⼅}
  \begin{phonetics}{怎么了}{zen3me5le5}
    \definition{expr.}{O que aconteceu? | O que está acontecendo? | E aí?}
  \end{phonetics}
\end{entry}

\begin{entry}{怎么办}{9,3,4}{⼼、⼃、⼒}
  \begin{phonetics}{怎么办}{zen3 me5 ban4}[][HSK 2]
    \definition{adv.}{o que fazer?; o que deve ser feito?}
  \end{phonetics}
\end{entry}

\begin{entry}{怎么回事}{9,3,6,8}{⼼、⼃、⼞、⼅}
  \begin{phonetics}{怎么回事}{zen3me5hui2shi4}
    \definition{expr.}{O que aconteceu? | O que se passou?}
  \end{phonetics}
\end{entry}

\begin{entry}{怎么样}{9,3,10}{⼼、⼃、⽊}
  \begin{phonetics}{怎么样}{zen3me5yang4}[][HSK 2]
    \definition{adv.}{como?; o que?; como é?; como estão as coisas?; o que você acha?; pergunte sobre o método, natureza, situação, opinião, etc. | substitui uma ação ou situação não dita (usado apenas na forma negativa, mais eufemístico do que uma declaração direta); indaga sobre a natureza, condição, método, razão, etc.}
  \end{phonetics}
\end{entry}

\begin{entry}{怎么得了}{9,3,11,2}{⼼、⼃、⼻、⼅}
  \begin{phonetics}{怎么得了}{zen3me5de2liao3}
    \definition{expr.}{Como isso pode ser? | Que bagunça horrível! | O que deve ser feito?}
  \end{phonetics}
\end{entry}

\begin{entry}{怎么搞的}{9,3,13,8}{⼼、⼃、⼿、⽩}
  \begin{phonetics}{怎么搞的}{zen3me5gao3de5}
    \definition{expr.}{Como isso aconteceu? | O que deu errado? | E aí? | O que está errado?}
  \end{phonetics}
\end{entry}

\begin{entry}{怎样}{9,10}{⼼、⽊}
  \begin{phonetics}{怎样}{zen3 yang4}[][HSK 2]
    \definition{pron.}{como?; o que?; indagar sobre a natureza, condição ou método, etc. | como?; indica uma referência virtual | de uma certa maneira; de qualquer maneira; não importa como; indica qualquer | como?; usado como predicado, objeto ou complemento para indagar sobre uma situação}
  \end{phonetics}
\end{entry}

\begin{entry}{怒骂}{9,9}{⼼、⾺}
  \begin{phonetics}{怒骂}{nu4ma4}
    \definition{v.}{praguejar de raiva}
  \end{phonetics}
\end{entry}

\begin{entry}{思考}{9,6}{⼼、⽼}
  \begin{phonetics}{思考}{si1kao3}[][HSK 4]
    \definition{v.}{pensar; ponderar; considerar; deliberar; envolver-se em atividades de pensamento, como análise, síntese, julgamento, raciocínio e generalização}
  \end{phonetics}
\end{entry}

\begin{entry}{思维}{9,11}{⼼、⽷}
  \begin{phonetics}{思维}{si1wei2}[][HSK 5]
    \definition[种]{s.}{pensamento; reflexão; organizar e transformar os materiais obtidos através do conhecimento sensorial para formar conceitos, julgamentos e raciocínios}
    \definition{v.}{pensar;}
  \end{phonetics}
\end{entry}

\begin{entry}{思想}{9,13}{⼼、⼼}
  \begin{phonetics}{思想}{si1xiang3}[][HSK 3]
    \definition[个,种]{s.}{reflexão; pensamento; ideologia; a existência objetiva é refletida na consciência das pessoas por meio de atividades de pensamento, que pertencem à cognição racional | ideia; pensamento}
  \end{phonetics}
\end{entry}

\begin{entry}{急}{9}{⼼}
  \begin{phonetics}{急}{ji2}[][HSK 2]
    \definition{adj.}{impaciente; ansioso | irritado; aborrecido; incomodado | rápido e intenso (em oposição a 缓); veloz | urgente; premente}
    \definition{s.}{urgência; emergência; assunto urgente e grave}
    \definition{v.}{preocupar; deixar ansioso | estar ansioso para ajudar; tratar os problemas dos outros como se fossem urgentes e ajudar a resolvê-los imediatamente}
  \seealsoref{缓}{huan3}
  \end{phonetics}
\end{entry}

\begin{entry}{急忙}{9,6}{⼼、⼼}
  \begin{phonetics}{急忙}{ji2mang2}[][HSK 4]
    \definition{adv.}{apressadamente; com pressa}
  \end{phonetics}
\end{entry}

\begin{entry}{急救}{9,11}{⼼、⽁}
  \begin{phonetics}{急救}{ji2jiu4}
    \definition{s.}{primeiros socorros}
    \definition{v.}{dar tratamento de emergência}
  \end{phonetics}
\end{entry}

\begin{entry}{怨}{9}{⼼}
  \begin{phonetics}{怨}{yuan4}[][HSK 5]
    \definition{s.}{ressentimento; inimizade; rancor}
    \definition{v.}{culpar; reclamar}
  \end{phonetics}
\end{entry}

\begin{entry}{怹}{9}{⼼}
  \begin{phonetics}{怹}{tan1}
    \definition{pron.}{ele, ela (cortês, em oposição a 他)}
  \seealsoref{他}{ta1}
  \end{phonetics}
\end{entry}

\begin{entry}{总}{9}{⼼}
  \begin{phonetics}{总}{zong3}[][HSK 3]
    \definition{adj.}{total; geral; global | responsável (liderança)}
    \definition{adv.}{sempre; invariavelmente | de qualquer forma; afinal; eventualmente; mais cedo ou mais tarde; no fim das contas | certamente; provavelmente; com certeza; expressa estimativa; suposição; equivalente a 大概}
    \definition{v.}{reunir; resumir; juntar; compilar}
  \seealsoref{大概}{da4gai4}
  \end{phonetics}
\end{entry}

\begin{entry}{总之}{9,3}{⼼、⼂}
  \begin{phonetics}{总之}{zong3zhi1}[][HSK 4]
    \definition{conj.}{em uma palavra; em suma; em resumo; indica que a declaração seguinte é uma declaração geral}
  \end{phonetics}
\end{entry}

\begin{entry}{总长}{9,4}{⼼、⾧}
  \begin{phonetics}{总长}{zong3chang2}
    \definition{s.}{comprimento total}
  \end{phonetics}
\end{entry}

\begin{entry}{总务}{9,5}{⼼、⼒}
  \begin{phonetics}{总务}{zong3wu4}
    \definition{s.}{divisão de assuntos gerais | assuntos gerais | pessoa responsável geral}
  \end{phonetics}
\end{entry}

\begin{entry}{总台}{9,5}{⼼、⼝}
  \begin{phonetics}{总台}{zong3tai2}
    \definition{s.}{recepção | balcão de recepção}
  \end{phonetics}
\end{entry}

\begin{entry}{总价}{9,6}{⼼、⼈}
  \begin{phonetics}{总价}{zong3jia4}
    \definition{s.}{preço total}
  \end{phonetics}
\end{entry}

\begin{entry}{总共}{9,6}{⼼、⼋}
  \begin{phonetics}{总共}{zong3gong4}[][HSK 4]
    \definition{adv.}{em tudo; em todos; no total; completamente; totalmente; em conjunto}
  \end{phonetics}
\end{entry}

\begin{entry}{总体}{9,7}{⼼、⼈}
  \begin{phonetics}{总体}{zong3 ti3}[][HSK 5]
    \definition{s.}{total; geral; conjunto; totalidade; massa; população; o todo formado pela união de vários indivíduos; a totalidade das coisas}
  \end{phonetics}
\end{entry}

\begin{entry}{总线}{9,8}{⼼、⽷}
  \begin{phonetics}{总线}{zong3xian4}
    \definition{s.}{barramento (computador) | \emph{computer bus}}
  \end{phonetics}
\end{entry}

\begin{entry}{总是}{9,9}{⼼、⽇}
  \begin{phonetics}{总是}{zong3shi4}[][HSK 3]
    \definition{adv.}{sempre; indica como tem sido durante um determinado período de tempo; um determinado estado permanece inalterado | afinal; significa que, independentemente do que acontecer, haverá ou será um resultado}
  \end{phonetics}
\end{entry}

\begin{entry}{总结}{9,9}{⼼、⽷}
  \begin{phonetics}{总结}{zong3jie2}[][HSK 3]
    \definition[个,篇]{s.}{resumo; síntese; conclusão resumida}
    \definition{v.}{resumir; sumariar; sintetizar; analisar e estudar as experiências para chegar a conclusões}
  \end{phonetics}
\end{entry}

\begin{entry}{总统}{9,9}{⼼、⽷}
  \begin{phonetics}{总统}{zong3tong3}[][HSK 4]
    \definition*[个,位,名]{s.}{Presidente (de um país); Título dos líderes de determinadas repúblicas}
  \end{phonetics}
\end{entry}

\begin{entry}{总值}{9,10}{⼼、⼈}
  \begin{phonetics}{总值}{zong3zhi2}
    \definition{s.}{valor total}
  \end{phonetics}
\end{entry}

\begin{entry}{总站}{9,10}{⼼、⽴}
  \begin{phonetics}{总站}{zong3zhan4}
    \definition{s.}{terminal}
  \end{phonetics}
\end{entry}

\begin{entry}{总得}{9,11}{⼼、⼻}
  \begin{phonetics}{总得}{zong3dei3}
    \definition{adv.}{prestes a}
    \definition{v.}{dever | precisar}
  \end{phonetics}
\end{entry}

\begin{entry}{总理}{9,11}{⼼、⽟}
  \begin{phonetics}{总理}{zong3li3}[][HSK 4]
    \definition*[个,位,名]{s.}{Primeiro-Ministro do Conselho de Estado; Título do líder do Conselho de Estado da China | Título do chefe de governo em determinados países | Primeiro-Ministro; Título de líderes de determinados partidos políticos | Título dos chefes de determinadas instituições e empresas nos velhos tempos}
    \definition{v.}{assumir a responsabilidade total;}
  \end{phonetics}
\end{entry}

\begin{entry}{总裁}{9,12}{⼼、⾐}
  \begin{phonetics}{总裁}{zong3cai2}[][HSK 5]
    \definition[位,名]{s.}{presidente (de uma empresa); nomes de certos líderes de partidos políticos ou grandes empresas}
  \end{phonetics}
\end{entry}

\begin{entry}{总数}{9,13}{⼼、⽁}
  \begin{phonetics}{总数}{zong3 shu4}[][HSK 5]
    \definition{s.}{soma; total; totalidade; inventário; número total; soma total}
  \end{phonetics}
\end{entry}

\begin{entry}{总督}{9,13}{⼼、⽬}
  \begin{phonetics}{总督}{zong3du1}
    \definition*{s.}{Governador-Geral | Governador | Vice-Rei}
  \end{phonetics}
\end{entry}

\begin{entry}{总算}{9,14}{⼼、⽵}
  \begin{phonetics}{总算}{zong3suan4}[][HSK 5]
    \definition{adv.}{finalmente; por fim; indica que, após um longo período de tempo, um desejo finalmente se tornou realidade | suficiente; considerando tudo; no geral; considerando todos os aspectos; significa que, em geral, está tudo bem}
  \end{phonetics}
\end{entry}

\begin{entry}{恒星系}{9,9,7}{⼼、⽇、⽷}
  \begin{phonetics}{恒星系}{heng2xing1xi4}
    \definition{s.}{sistema estelar | galáxia}
  \end{phonetics}
\end{entry}

\begin{entry}{恢复}{9,9}{⼼、⼢}
  \begin{phonetics}{恢复}{hui1fu4}[][HSK 5]
    \definition{v.}{retomar; renovar; restaurar; voltar a | reviver; recuperar; reencontrar | restaurar; restabelecer; reabilitar; regenerar; ressurgir; restabelecer alguém em; recuperar o que foi perdido}
  \end{phonetics}
\end{entry}

\begin{entry}{T-恤}{9}{⼼}
  \begin{phonetics}{T-恤}{xu4}
    \definition{s.}{camiseta | pulôver | suéter}
  \end{phonetics}
\end{entry}

\begin{entry}{恨}{9}{⼼}
  \begin{phonetics}{恨}{hen4}[][HSK 5]
    \definition{s.}{ódio; resentimento}
    \definition{v.}{odiar}
  \end{phonetics}
\end{entry}

\begin{entry}{恰}{9}{⼼}
  \begin{phonetics}{恰}{qia4}
    \definition{adv.}{exatamente | apenas}
  \end{phonetics}
\end{entry}

\begin{entry}{恰好}{9,6}{⼼、⼥}
  \begin{phonetics}{恰好}{qia4hao3}
    \definition{adv.}{certo | por sorte | ao que parece | por sorte coincidência}
  \end{phonetics}
\end{entry}

\begin{entry}{恰到好处}{9,8,6,5}{⼼、⼑、⼥、⼡}
  \begin{phonetics}{恰到好处}{qia4dao4hao3chu4}
    \definition{expr.}{é simplesmente perfeito | é simplesmente correto}
  \end{phonetics}
\end{entry}

\begin{entry}{战}{9}{⼽}
  \begin{phonetics}{战}{zhan4}
    \definition{s.}{luta | guerra | batalha}
    \definition{v.}{lutar}
  \end{phonetics}
\end{entry}

\begin{entry}{战士}{9,3}{⼽、⼠}
  \begin{phonetics}{战士}{zhan4shi4}[][HSK 4]
    \definition[个]{s.}{soldado; membros mais jovens do exército | campeão; guerreiro; lutador; geralmente, uma pessoa que se engaja em alguma causa justa ou participa de alguma luta justa}
  \end{phonetics}
\end{entry}

\begin{entry}{战斗}{9,4}{⼽、⽃}
  \begin{phonetics}{战斗}{zhan4dou4}[][HSK 4]
    \definition[场,次]{s.}{luta; batalha; combate; ação; conflito armado entre as partes oponentes}
    \definition{v.}{lutar | trabalhar sob pressão}
  \end{phonetics}
\end{entry}

\begin{entry}{战争}{9,6}{⼽、⼑}
  \begin{phonetics}{战争}{zhan4zheng1}[][HSK 4]
    \definition[场,次]{s.}{guerra; conflito; luta armada entre povos, entre nações, entre classes ou entre grupos políticos}
  \end{phonetics}
\end{entry}

\begin{entry}{战胜}{9,9}{⼽、⾁}
  \begin{phonetics}{战胜}{zhan4 sheng4}[][HSK 4]
    \definition{v.}{derrotar; vencer; superar; triunfar sobre; metáfora para superar dificuldades e alcançar o sucesso}
  \end{phonetics}
\end{entry}

\begin{entry}{拜}{9}{⼿}
  \begin{phonetics}{拜}{bai4}
    \definition*{s.}{sobrenome Bai}
    \definition{adv.}{respeitosamente (usado na comunicação interpessoal);}
    \definition{v.}{fazer uma visita de cortesia | adorar; prestar homenagem | fazer uma chamada cerimonial | ligar; fazer uma visita | intitular alguém com cerimônia; conceder uma posição oficial ou um determinado título com certa etiqueta | estabelecer ou jurar formalmente relacionamentos}
  \end{phonetics}
\end{entry}

\begin{entry}{拜访}{9,6}{⼿、⾔}
  \begin{phonetics}{拜访}{bai4fang3}[][HSK 5]
    \definition{v.}{visitar; fazer uma visita (respeitosamente)}
  \end{phonetics}
\end{entry}

\begin{entry}{括号}{9,5}{⼿、⼝}
  \begin{phonetics}{括号}{kuo4 hao4}[][HSK 4]
    \definition{s.}{chaves, colchetes e parênteses (em fórmulas aritméticas ou algébricas, os símbolos que indicam a combinação e a ordem de vários números ou termos) | colchetes e parênteses usados como um tipo de sinal de pontuação para mostrar a parte explicativa de uma passagem em um texto}
  \end{phonetics}
\end{entry}

\begin{entry}{拼}{9}{⼿}
  \begin{phonetics}{拼}{pin1}[][HSK 5]
    \definition{v.}{montar; juntar as peças | dar tudo de si no trabalho; estar disposto a arriscar a vida (em lutas, no trabalho, etc.); fazer tudo o que for preciso; arriscar tudo}
  \end{phonetics}
\end{entry}

\begin{entry}{拼命}{9,8}{⼿、⼝}
  \begin{phonetics}{拼命}{pin1ming4}
    \definition{adv.}{com toda a força | desesperadamente}
    \definition{v.+compl.}{arriscar a vida de alguém | desafiar a morte | colocar-se em uma luta desesperada | fazer algo desesperadamente | exercer a maior força}
  \end{phonetics}
\end{entry}

\begin{entry}{拼音}{9,9}{⼿、⾳}
  \begin{phonetics}{拼音}{pin1yin1}
    \definition{s.}{escrita fonética | pinyin (romanização chinesa)}
  \end{phonetics}
\end{entry}

\begin{entry}{拾}{9}{⼿}
  \begin{phonetics}{拾}{shi2}[][HSK 5]
    \definition{num.}{dez (usado no lugar do numeral 十 em cheques, notas bancárias, etc., para evitar erros ou alterações)}
    \definition{v.}{pegar (do chão); recolher}
  \end{phonetics}
\end{entry}

\begin{entry}{持}{9}{⼿}
  \begin{phonetics}{持}{chi2}
    \definition{v.}{segurar; agarrar | opor; confrontar | apoiar; manter | gerenciar; supervisionar | sequestrar; agarrar (controlar; forçar)}
  \end{phonetics}
\end{entry}

\begin{entry}{持续}{9,11}{⼿、⽷}
  \begin{phonetics}{持续}{chi2xu4}[][HSK 3]
    \definition{v.}{durar; continuar; sustentar; manter a situação ou as condições como estão, sem alterações}
  \end{phonetics}
\end{entry}

\begin{entry}{挂}{9}{⼿}
  \begin{phonetics}{挂}{gua4}[][HSK 3]
    \definition{clas.}{usado principalmente para coisas que vêm em conjuntos ou séries}
    \definition{v.}{pendurar; colocar; suspender; usando cordas, ganchos, pregos e outros itens para prender objetos em um ou mais pontos específicos | interromper chamada (telefônica) | colocar alguém em contato com; ligar; telefonar; refere-se a ligar o telefone, bem como a fazer uma chamada | falhar; fracassar | colocar em registro; registrarpegar carona; ser pego | preocupar-se com | ser revestido com; ser coberto com | estar pendente; deixar algo sem solução}
  \end{phonetics}
\end{entry}

\begin{entry}{挂号}{9,5}{⼿、⼝}
  \begin{phonetics}{挂号}{gua4hao4}
    \definition{v.+compl.}{registrar-se (em um hospital, etc.) | enviar através de carta registrada}
  \end{phonetics}
\end{entry}

\begin{entry}{挂号信}{9,5,9}{⼿、⼝、⼈}
  \begin{phonetics}{挂号信}{gua4hao4xin4}
    \definition{s.}{carta registrada}
  \end{phonetics}
\end{entry}

\begin{entry}{指}{9}{⼿}
  \begin{phonetics}{指}{zhi3}[][HSK 3]
    \definition*{s.}{sobrenome Zhi}
    \definition{clas.}{dígito; largura do dedo; a largura de um dedo é chamada de 一指, que é usado para medir profundidade, largura, etc.}
    \definition{s.}{dedo}
    \definition{v.}{apontar para; indicar; usar o dedo ou a ponta de um objeto para apontar | (pelo) eriçar;  (cabelo) ficar em pé | indicar; mostrar; apontar; demonstrar | referir-se a; dirigir-se a | confiar em; contar com; depender de | criticar; repreender}
  \end{phonetics}
\end{entry}

\begin{entry}{指出}{9,5}{⼿、⼐}
  \begin{phonetics}{指出}{zhi3 chu1}[][HSK 3]
    \definition{v.}{apontar; indicar}
  \end{phonetics}
\end{entry}

\begin{entry}{指甲}{9,5}{⼿、⽥}
  \begin{phonetics}{指甲}{zhi3jia5}[][HSK 5]
    \definition{s.}{unha; unha de agulha; unha de dedo; camada córnea na ponta dos dedos}
  \end{phonetics}
\end{entry}

\begin{entry}{指示}{9,5}{⼿、⽰}
  \begin{phonetics}{指示}{zhi3shi4}[][HSK 5]
    \definition{s.}{diretriz; instruções; para orientar o trabalho, os superiores emitem opiniões verbais ou escritas aos subordinados}
    \definition{v.}{indicar; apontar; apontar para alguém | instruir; superiores emitem opiniões verbais ou escritas para orientar o trabalho dos subordinados}
  \end{phonetics}
\end{entry}

\begin{entry}{指导}{9,6}{⼿、⼨}
  \begin{phonetics}{指导}{zhi3dao3}[][HSK 3]
    \definition[位]{s.}{guia; diretor; pessoa que dá orientações}
    \definition{v.}{orientar; dirigir; instruir}
  \end{phonetics}
\end{entry}

\begin{entry}{指责}{9,8}{⼿、⾙}
  \begin{phonetics}{指责}{zhi3ze2}[][HSK 5]
    \definition{v.}{censurar; criticar; encontrar falhas; repreender}
  \end{phonetics}
\end{entry}

\begin{entry}{指南针}{9,9,7}{⼿、⼗、⾦}
  \begin{phonetics}{指南针}{zhi3nan2zhen1}
    \definition{s.}{bússola}
  \end{phonetics}
\end{entry}

\begin{entry}{指挥}{9,9}{⼿、⼿}
  \begin{phonetics}{指挥}{zhi3hui1}[][HSK 4]
    \definition[个]{s.}{diretor; comandante; despachante; operador | maestro; condutor; pessoa na frente de uma orquestra ou coro que dá instruções sobre como tocar ou cantar}
    \definition{v.}{dirigir; conduzir; comandar; direcionar; emitir ordens de agendamento}
  \end{phonetics}
\end{entry}

\begin{entry}{指标}{9,9}{⼿、⽊}
  \begin{phonetics}{指标}{zhi3biao1}[][HSK 5]
    \definition{s.}{meta; cota; norma; índice; objetivos a serem alcançados | alvo; índice; refletir os requisitos de desenvolvimento em determinados aspectos através de números absolutos ou percentagens de aumento ou diminuição, inclui indicadores quantitativos e qualitativos}
  \end{phonetics}
\end{entry}

\begin{entry}{按}{9}{⼿}
  \begin{phonetics}{按}{an4}[][HSK 3]
    \definition{prep.}{de acordo com; à luz de; com base em; em conformidade com}
    \definition{v.}{pressionar; empurrar para baixo; pressionar ou apertar com a mão ou os dedos | pôr de parte; deixar de lado; deixar para mais tarde | restringir; controlar; inibir; parar | verificar; consultar | comentar ou anotar (por um editor ou autor)}
  \end{phonetics}
\end{entry}

\begin{entry}{按时}{9,7}{⼿、⽇}
  \begin{phonetics}{按时}{an4shi2}[][HSK 4]
    \definition{adv.}{na hora; no horário; pontualmente; de acordo com o tempo estipulado}
  \end{phonetics}
\end{entry}

\begin{entry}{按照}{9,13}{⼿、⽕}
  \begin{phonetics}{按照}{an4zhao4}[][HSK 3]
    \definition{prep.}{de acordo com; em conformidade com; à luz de; com base em; apresentar os fundamentos ou critérios de julgamento para fazer algo}
  \end{phonetics}
\end{entry}

\begin{entry}{按摩}{9,15}{⼿、⼿}
  \begin{phonetics}{按摩}{an4mo2}[][HSK 5]
    \definition{s.}{massagem; empurrar, pressionar, beliscar e amassar o corpo de uma pessoa com as mãos para promover a circulação sanguínea, aumentar a resistência da pele e regular a função dos nervos}
  \end{phonetics}
\end{entry}

\begin{entry}{挑}{9}{⼿}
  \begin{phonetics}{挑}{tiao1}[][HSK 4]
    \definition{clas.}{para coisas que são escolhidas ou selecionadas | para coisas que podem ser usadas como palhetas}
    \definition{s.}{vara comprida com algo pendurado nas pontas; haste de transporte}
    \definition{v.}{escolher; selecionar | fazer picuinhas; ser hipercrítico; ser meticuloso; ser excessivamente rigoroso nos detalhes | carregar com uma haste de transporte; carregar no ombro; pendurar coisas nas pontas de varas longas e carregá-las em seus ombros}
  \end{phonetics}
  \begin{phonetics}{挑}{tiao3}[][HSK 4]
    \definition{s.}{um dos traços dos caracteres chineses; inclinado para cima da esquerda para a direita}
    \definition{v.}{levantar; elevar; erguer | levantar ou apoiar com uma extremidade de uma vara ou objeto semelhante; segurar ou apoiar com a ponta de uma vara etc. | causar conflitos deliberadamente; provocar deliberadamente um conflito | (método de bordado) usar uma agulha para levantar os fios de urdidura ou trama, com a agulha e a linha passando por baixo para formar padrões e desenhos}
  \end{phonetics}
\end{entry}

\begin{entry}{挑战}{9,9}{⼿、⼽}
  \begin{phonetics}{挑战}{tiao3zhan4}[][HSK 4]
    \definition{v.}{desafiar; deixar um oponente deliberadamente irritado e sair para lutar ou lutar consigo mesmo; estimular um oponente a lutar consigo mesmo}
  \end{phonetics}
\end{entry}

\begin{entry}{挑选}{9,9}{⼿、⾡}
  \begin{phonetics}{挑选}{tiao1 xuan3}[][HSK 4]
    \definition{v.}{escolher; optar; selecionar; escolher a pessoa ou coisa certa para o trabalho}
  \end{phonetics}
\end{entry}

\begin{entry}{挑衅}{9,11}{⼿、⾎}
  \begin{phonetics}{挑衅}{tiao3xin4}
    \definition{s.}{provocação}
    \definition{v.}{provocar}
  \end{phonetics}
\end{entry}

\begin{entry}{挖}{9}{⼿}
  \begin{phonetics}{挖}{wa1}
    \definition{v.}{cavar | escavar}
  \end{phonetics}
\end{entry}

\begin{entry}{挖掘机}{9,11,6}{⼿、⼿、⽊}
  \begin{phonetics}{挖掘机}{wa1jue2ji1}
    \definition{s.}{escavadeira | escavador | escavadora | pá mecânica}
  \end{phonetics}
\end{entry}

\begin{entry}{挡}{9}{⼿}
  \begin{phonetics}{挡}{dang3}[][HSK 5]
    \definition{s.}{persiana; veneziana; paralama; coisas para cobrir ou bloquear | caixa de câmbio (automóvel)}
    \definition{v.}{bloquear; resistir; manter afastado; afastar | cobrir; bloquear; atrapalhar}
  \end{phonetics}
  \begin{phonetics}{挡}{dang4}
    \definition{v.}{organizar}
  \end{phonetics}
\end{entry}

\begin{entry}{挡风玻璃}{9,4,9,14}{⼿、⾵、⽟、⽟}
  \begin{phonetics}{挡风玻璃}{dang3feng1bo1li5}
    \definition{s.}{parabrisa}
  \end{phonetics}
\end{entry}

\begin{entry}{挣}{9}{⼿}
  \begin{phonetics}{挣}{zheng4}[][HSK 5]
    \definition{v.}{empurrar e puxar; tentar se livrar; lutar para se libertar; esforçar-se para se libertar das amarras | ganhar; fazer; trabalhar para; trocar trabalho por}
  \end{phonetics}
\end{entry}

\begin{entry}{挣扎}{9,4}{⼿、⼿}
  \begin{phonetics}{挣扎}{zheng1zha2}
    \definition{v.}{lutar}
  \end{phonetics}
\end{entry}

\begin{entry}{挣钱}{9,10}{⼿、⾦}
  \begin{phonetics}{挣钱}{zheng4qian2}[][HSK 5]
    \definition{v.+compl.}{ganhar dinheiro; fazer dinheiro; lucrar; trabalhar para ganhar dinheiro}
  \end{phonetics}
\end{entry}

\begin{entry}{挣得}{9,11}{⼿、⼻}
  \begin{phonetics}{挣得}{zheng4de2}
    \definition{v.}{ganhar renda ou dinheiro}
  \end{phonetics}
\end{entry}

\begin{entry}{挤}{9}{⼿}
  \begin{phonetics}{挤}{ji3}[][HSK 5]
    \definition{adj.}{lotado; congestionado; descreve um grande número de pessoas ou coisas e muito pouco espaço}
    \definition{v.}{empacotar; amontoar; aglomerar | sacudir; empurrar contra; empurrar alguém ou algo para longe com seu corpo com toda a força que puder| pressionar; apertar; expulsar por pressão}
  \end{phonetics}
\end{entry}

\begin{entry}{挥汗如雨}{9,6,6,8}{⼿、⽔、⼥、⾬}
  \begin{phonetics}{挥汗如雨}{hui1han4ru2yu3}
    \definition{s.}{suor derramado}
    \definition{v.}{pingar com suor}
  \end{phonetics}
\end{entry}

\begin{entry}{挺}{9}{⼿}
  \begin{phonetics}{挺}{ting3}[][HSK 2,4]
    \definition{adj.}{rígido; ereto; vertical; reto | notável; destacado; distinto}
    \definition{adv.}{muito; bastante}
    \definition{clas.}{usado para metralhadoras}
    \definition{v.}{sobressair; endireitar-se; protrudir (protuberância ou saliência) | suportar; aguentar; resistir; perseverar}
  \end{phonetics}
\end{entry}

\begin{entry}{挺尸}{9,3}{⼿、⼫}
  \begin{phonetics}{挺尸}{ting3shi1}
    \definition{v.}{(coloquial) dormir | (literalmente) ficar deitado duro como um cadáver}
  \end{phonetics}
\end{entry}

\begin{entry}{挺立}{9,5}{⼿、⽴}
  \begin{phonetics}{挺立}{ting3li4}
    \definition{v.}{ficar ereto | ficar de pé}
  \end{phonetics}
\end{entry}

\begin{entry}{挺好}{9,6}{⼿、⼥}
  \begin{phonetics}{挺好}{ting3 hao3}[][HSK 2]
    \definition{adj.}{nada mal; surpreendentemente bom}
  \end{phonetics}
\end{entry}

\begin{entry}{挺过}{9,6}{⼿、⾡}
  \begin{phonetics}{挺过}{ting3guo4}
    \definition{s.}{sobreviver}
  \end{phonetics}
\end{entry}

\begin{entry}{挺住}{9,7}{⼿、⼈}
  \begin{phonetics}{挺住}{ting3zhu4}
    \definition{v.}{permanecer firme | manter-se firme (diante da adversidade ou da dor)}
  \end{phonetics}
\end{entry}

\begin{entry}{挺杆}{9,7}{⼿、⽊}
  \begin{phonetics}{挺杆}{ting3gan3}
    \definition{s.}{tucho (peça de máquina)}
  \end{phonetics}
\end{entry}

\begin{entry}{挺身}{9,7}{⼿、⾝}
  \begin{phonetics}{挺身}{ting3shen1}
    \definition{v.}{endireitar as costas}
  \end{phonetics}
\end{entry}

\begin{entry}{挺进}{9,7}{⼿、⾡}
  \begin{phonetics}{挺进}{ting3jin4}
    \definition{s.}{progresso | avanço}
    \definition{v.}{progredir | avançar}
  \end{phonetics}
\end{entry}

\begin{entry}{挺拔}{9,8}{⼿、⼿}
  \begin{phonetics}{挺拔}{ting3ba2}
    \definition{adj.}{alto e reto}
  \end{phonetics}
\end{entry}

\begin{entry}{挺腰}{9,13}{⼿、⾁}
  \begin{phonetics}{挺腰}{ting3yao1}
    \definition{v.}{arquear as costas | endireitar as costas}
  \end{phonetics}
\end{entry}

\begin{entry}{政纲}{9,7}{⽁、⽷}
  \begin{phonetics}{政纲}{zheng4gang1}
    \definition{s.}{programa ou plataforma política}
  \end{phonetics}
\end{entry}

\begin{entry}{政府}{9,8}{⽁、⼴}
  \begin{phonetics}{政府}{zheng4fu3}[][HSK 4]
    \definition[个]{s.}{governo;  órgãos executivos do poder do Estado, ou seja, órgãos administrativos do Estado, como o Conselho de Estado (Governo Popular Central) e os governos populares locais em todos os níveis na China}
  \end{phonetics}
\end{entry}

\begin{entry}{政治}{9,8}{⽁、⽔}
  \begin{phonetics}{政治}{zheng4zhi4}[][HSK 4]
    \definition{s.}{política; assuntos políticos; questões políticas}
  \end{phonetics}
\end{entry}

\begin{entry}{政治局}{9,8,7}{⽁、⽔、⼫}
  \begin{phonetics}{政治局}{zheng4zhi4ju2}
    \definition{s.}{o principal comitê de políticas de um partido comunista}
  \end{phonetics}
\end{entry}

\begin{entry}{故}{9}{⽁}
  \begin{phonetics}{故}{gu4}
    \definition*{s.}{sobrenome Gu}
    \definition{adj.}{velho; antigo}
    \definition{adv.}{propositalmente; intencionalmente; deliberadamente}
    \definition{conj.}{assim; portanto; consequentemente; pelo contrário}
    \definition{s.}{evento; incidente; acontecimento; acidente | causa; razão | amigo; conhecido | o velho; refere-se a coisas antigas e passadas}
    \definition{v.}{morrer}
  \end{phonetics}
\end{entry}

\begin{entry}{故乡}{9,3}{⽁、⼄}
  \begin{phonetics}{故乡}{gu4xiang1}[][HSK 3]
    \definition[个]{s.}{cidade natal; terra natal; local de nascimento ou onde viveu por muito tempo}
  \end{phonetics}
\end{entry}

\begin{entry}{故事}{9,8}{⽁、⼅}
  \begin{phonetics}{故事}{gu4shi5}[][HSK 2]
    \definition[个,段,篇,则]{s.}{história; conto; coisas reais ou fictícias usadas como objeto de narrativa, com coerência, atraentes e capazes de emocionar as pessoas | enredo; trama; enredo que consegue mostrar a personalidade dos personagens e refletir a ideia central da obra literária}
  \end{phonetics}
\end{entry}

\begin{entry}{故宫}{9,9}{⽁、⼧}
  \begin{phonetics}{故宫}{gu4gong1}
    \definition*{s.}{Palácio Imperial | Cidade Proibida}
  \end{phonetics}
\end{entry}

\begin{entry}{故意}{9,13}{⽁、⼼}
  \begin{phonetics}{故意}{gu4yi4}[][HSK 2]
    \definition{adv.}{deliberadamente; intencionalmente; não é por descuido, mas sim conscientemente (geralmente coisas que não se devem fazer ou que não são necessárias)}
    \definition{s.}{intenção; um tipo de mentalidade, uma pessoa sabe claramente que seus atos podem causar danos a outras pessoas ou trazer consequências negativas para a sociedade, mas mesmo assim não faz nada para impedir isso}
  \end{phonetics}
\end{entry}

\begin{entry}{既}{9}{⽆}
  \begin{phonetics}{既}{ji4}[][HSK 4]
    \definition*{s.}{sobrenome Ji}
    \definition{adv.}{já}
    \definition{conj.}{desde; como; agora que | assim como; e também; ambos\dots e\dots; usado em conjunto com advérbios como 且, 又, 也 para indicar uma combinação de ambas as situações}
  \seealsoref{且}{qie3}
  \seealsoref{也}{ye3}
  \seealsoref{又}{you4}
  \end{phonetics}
\end{entry}

\begin{entry}{既又}{9,2}{⽆、⼜}
  \begin{phonetics}{既又}{ji4you4}
    \definition{conj.}{desde | como | agora isso | os dois e | assim como}
  \end{phonetics}
\end{entry}

\begin{entry}{既不……又不……}{9,4,2,4}{⽆、⼀、⼜、⼀}
  \begin{phonetics}{既不……又不……}{ji4bu4 you4bu4}
    \definition{conj.}{nem mesmo\dots}
  \end{phonetics}
\end{entry}

\begin{entry}{既然}{9,12}{⽆、⽕}
  \begin{phonetics}{既然}{ji4ran2}[][HSK 4]
    \definition{conj.}{como; desde; agora que; usado na primeira metade de uma frase, muitas vezes repetido na segunda metade pelos advérbios 就, 也, 还 para indicar que a premissa é primeiro declarada e depois inferida}
  \seealsoref{还}{hai2}
  \seealsoref{就}{jiu4}
  \seealsoref{也}{ye3}
  \end{phonetics}
\end{entry}

\begin{entry}{星火}{9,4}{⽇、⽕}
  \begin{phonetics}{星火}{xing1huo3}
    \definition{s.}{trilha de meteoro (usada principalmente em expressões como 急如星火) | faísca}
  \end{phonetics}
\end{entry}

\begin{entry}{星辰}{9,7}{⽇、⾠}
  \begin{phonetics}{星辰}{xing1chen2}
    \definition{s.}{estrelas}
  \end{phonetics}
\end{entry}

\begin{entry}{星表}{9,8}{⽇、⾐}
  \begin{phonetics}{星表}{xing1biao3}
    \definition{s.}{catálogo de estrelas}
  \end{phonetics}
\end{entry}

\begin{entry}{星星}{9,9}{⽇、⽇}
  \begin{phonetics}{星星}{xing1 xing5}[][HSK 2]
    \definition[颗,群,片]{s.}{estrela; em astronomia, refere-se aos corpos celestes luminosos no universo, como as estrelas que brilham no céu noturno | estrela; uma metáfora para alguém ou algo que se destaca em um determinado campo e atrai atenção | objetos em forma de estrela}
  \end{phonetics}
\end{entry}

\begin{entry}{星座}{9,10}{⽇、⼴}
  \begin{phonetics}{星座}{xing1zuo4}
    \definition[张]{s.}{signo astrológico | constelação}
  \end{phonetics}
\end{entry}

\begin{entry}{星期}{9,12}{⽇、⽉}
  \begin{phonetics}{星期}{xing1qi1}[][HSK 1]
    \definition[个]{s.}{semana | dias da semana; usado em conjunto com 日, 一, 二, 三, 四, 五, 六, 天, indica um determinado dia da semana | abreviação de domingo}
  \seealsoref{星期二}{xing1 qi1 er4}
  \seealsoref{星期六}{xing1 qi1 liu4}
  \seealsoref{星期日}{xing1 qi1 ri4}
  \seealsoref{星期三}{xing1 qi1 san1}
  \seealsoref{星期四}{xing1 qi1 si4}
  \seealsoref{星期天}{xing1 qi1 tian1}
  \seealsoref{星期五}{xing1 qi1 wu3}
  \seealsoref{星期一}{xing1 qi1 yi1}
  \end{phonetics}
\end{entry}

\begin{entry}{星期一}{9,12,1}{⽇、⽉、⼀}
  \begin{phonetics}{星期一}{xing1 qi1 yi1}[][HSK 1]
    \definition{s.}{segunda-feira}
  \end{phonetics}
\end{entry}

\begin{entry}{星期二}{9,12,2}{⽇、⽉、⼆}
  \begin{phonetics}{星期二}{xing1 qi1 er4}[][HSK 1]
    \definition{s.}{terça-feira}
  \end{phonetics}
\end{entry}

\begin{entry}{星期三}{9,12,3}{⽇、⽉、⼀}
  \begin{phonetics}{星期三}{xing1 qi1 san1}[][HSK 1]
    \definition{s.}{quarta-feira}
  \end{phonetics}
\end{entry}

\begin{entry}{星期五}{9,12,4}{⽇、⽉、⼆}
  \begin{phonetics}{星期五}{xing1 qi1 wu3}[][HSK 1]
    \definition{s.}{sexta-feira}
  \end{phonetics}
\end{entry}

\begin{entry}{星期六}{9,12,4}{⽇、⽉、⼋}
  \begin{phonetics}{星期六}{xing1 qi1 liu4}[][HSK 1]
    \definition{s.}{sábado}
  \end{phonetics}
\end{entry}

\begin{entry}{星期天}{9,12,4}{⽇、⽉、⼤}
  \begin{phonetics}{星期天}{xing1 qi1 tian1}[][HSK 1]
    \definition{s.}{domingo}
  \seealsoref{星期日}{xing1 qi1 ri4}
  \end{phonetics}
\end{entry}

\begin{entry}{星期日}{9,12,4}{⽇、⽉、⽇}
  \begin{phonetics}{星期日}{xing1 qi1 ri4}[][HSK 1]
    \definition{s.}{domingo}
  \seealsoref{星期天}{xing1 qi1 tian1}
  \end{phonetics}
\end{entry}

\begin{entry}{星期四}{9,12,5}{⽇、⽉、⼞}
  \begin{phonetics}{星期四}{xing1 qi1 si4}[][HSK 1]
    \definition{s.}{quinta-feira}
  \end{phonetics}
\end{entry}

\begin{entry}{春}{9}{⽇}
  \begin{phonetics}{春}{chun1}
    \definition*{s.}{sobrenome Chun}
    \definition{s.}{primavera | amor; luxúria | vida; vitalidade}
  \end{phonetics}
\end{entry}

\begin{entry}{春天}{9,4}{⽇、⼤}
  \begin{phonetics}{春天}{chun1 tian1}
    \definition[个,段,季,番]{s.}{primavera; época da primavera | primavera; renascimento; uma atmosfera cheia de energia e esperança}
  \end{phonetics}
\end{entry}

\begin{entry}{春节}{9,5}{⽇、⾋}
  \begin{phonetics}{春节}{chun1 jie2}[][HSK 2]
    \definition*[个]{s.}{Festival da Primavera (Ano Novo Chinês); o primeiro dia do primeiro mês do calendário lunar, também se refere aos dias seguintes ao primeiro dia do primeiro mês}
  \end{phonetics}
\end{entry}

\begin{entry}{春季}{9,8}{⽇、⼦}
  \begin{phonetics}{春季}{chun1 ji4}[][HSK 4]
    \definition{s.}{primavera; primeiro trimestre do ano, que na China se refere ao período de três meses entre o início da primavera e o início do verão, e também se refere aos três meses do calendário lunar, a saber, o primeiro, o segundo e o terceiro meses}
  \end{phonetics}
\end{entry}

\begin{entry}{昨}{9}{⽇}
  \begin{phonetics}{昨}{zuo2}
    \definition{s.}{ontem}
  \end{phonetics}
\end{entry}

\begin{entry}{昨天}{9,4}{⽇、⼤}
  \begin{phonetics}{昨天}{zuo2tian1}[][HSK 1]
    \definition{s.}{ontem}
  \end{phonetics}
\end{entry}

\begin{entry}{昨日}{9,4}{⽇、⽇}
  \begin{phonetics}{昨日}{zuo2ri4}
    \definition{adv.}{ontem}
  \end{phonetics}
\end{entry}

\begin{entry}{昨夜}{9,8}{⽇、⼣}
  \begin{phonetics}{昨夜}{zuo2ye4}
    \definition{adv.}{noite passada}
  \end{phonetics}
\end{entry}

\begin{entry}{昨晚}{9,11}{⽇、⽇}
  \begin{phonetics}{昨晚}{zuo2wan3}
    \definition{adv.}{noite passada | ontem à noite}
  \end{phonetics}
\end{entry}

\begin{entry}{是}{9}{⽇}
  \begin{phonetics}{是}{shi4}[][HSK 1]
    \definition*{s.}{sobrenome Shi}
    \definition{adj.}{correto; certo | verdadeiro}
    \definition{adv.}{(expressar afirmação firme) de fato; realmente}
    \definition{pron.}{isso; isto |  todos; qualquer um; usado antes de substantivos, tem o significado de 凡是}
    \definition{s.}{assuntos (importantes); grandes planos}
    \definition{v.}{usado como “ser” antes de substantivos ou pronomes para identificar, descrever ou ampliar o sujeito; indica que duas coisas são iguais, ou que a segunda explica a primeira | usado entre duas palavras idênticas; relacionar duas palavras semelhantes |  (usado antes de substantivos) ser exatamente; ser corretamente; usado antes de substantivos, tem o significado de 适合 | elogiar; justificar | expressar afirmação ou concordância (frequentemente usado sozinho) | usado para escolher perguntas, perguntas sim/não ou perguntas retóricas | (usado no início de uma frase) enfatizar uma determinada parte de uma frase | usado em perguntas sim-não}
  \seealsoref{凡是}{fan2shi4}
  \seealsoref{适合}{shi4he2}
  \end{phonetics}
\end{entry}

\begin{entry}{是不是}{9,4,9}{⽇、⼀、⽇}
  \begin{phonetics}{是不是}{shi4 bu2 shi4}[][HSK 1]
    \definition{expr.}{sim ou não; é ou não é; se ou não; questões levantadas sobre a confirmação e a negação dos fatos}
  \end{phonetics}
\end{entry}

\begin{entry}{是否}{9,7}{⽇、⼝}
  \begin{phonetics}{是否}{shi4fou3}[][HSK 4]
    \definition{adv.}{se; se ou não}
  \end{phonetics}
\end{entry}

\begin{entry}{是的}{9,8}{⽇、⽩}
  \begin{phonetics}{是的}{shi4de5}
    \definition{adv.}{sim | está certo}
  \end{phonetics}
\end{entry}

\begin{entry}{昼}{9}{⽇}
  \begin{phonetics}{昼}{zhou4}
    \definition*{s.}{sobrenome Zhou}
    \definition{s.}{diurno; luz do dia; dia (oposição à 夜) | dia; o período do amanhecer ao anoitecer; diurno}
  \seealsoref{夜}{ye4}
  \end{phonetics}
\end{entry}

\begin{entry}{显}{9}{⽇}
  \begin{phonetics}{显}{xian3}[][HSK 5]
    \definition*{s.}{sobrenome Xian}
    \definition{adj.}{aparente; óbvio; perceptível | ilustre e influente | evidente; óbvio}
    \definition{v.}{mostrar; exibir; manifestar | aparecer; mostrar; revelar}
  \end{phonetics}
\end{entry}

\begin{entry}{显示}{9,5}{⽇、⽰}
  \begin{phonetics}{显示}{xian3shi4}[][HSK 3]
    \definition{v.}{mostrar; manifestar-se claramente| exibir; ostentar}
  \end{phonetics}
\end{entry}

\begin{entry}{显得}{9,11}{⽇、⼻}
  \begin{phonetics}{显得}{xian3de5}[][HSK 3]
    \definition{v.}{parecer; aparecer; manifestar (alguma situação)}
  \end{phonetics}
\end{entry}

\begin{entry}{显著}{9,11}{⽇、⽬}
  \begin{phonetics}{显著}{xian3zhu4}[][HSK 4]
    \definition{adj.}{notável; significativo; notável; extraordinário; muito óbvio; muito claramente demonstrado; muito facilmente visto ou sentido}
  \end{phonetics}
\end{entry}

\begin{entry}{显然}{9,12}{⽇、⽕}
  \begin{phonetics}{显然}{xian3ran2}[][HSK 3]
    \definition{adj.}{claro; evidente; óbvio; fatos, verdades e outras coisas que são fáceis de descobrir, perceber ou sentir claramente}
  \end{phonetics}
\end{entry}

\begin{entry}{枯木}{9,4}{⽊、⽊}
  \begin{phonetics}{枯木}{ku1mu4}
    \definition{s.}{árvore morta | madeira morta}
  \end{phonetics}
\end{entry}

\begin{entry}{架}{9}{⽊}
  \begin{phonetics}{架}{jia4}[][HSK 3]
    \definition{clas.}{usado para coisas com pilares ou componentes mecânicos | quadrado (usado para montanhas)}
    \definition{s.}{estrutura; organização do corpo humano ou das coisas | prateleira; estante; suporte; componentes que sustentam objetos ou utensílios para colocar objetos, etc.}
    \definition{v.}{colocar para cima; erigir | brigar; discutir | resistir; repelir; afastar | sequestrar; levar alguém à força}
  \end{phonetics}
\end{entry}

\begin{entry}{架式}{9,6}{⽊、⼷}
  \begin{phonetics}{架式}{jia4shi5}
    \variantof{架势}
  \end{phonetics}
\end{entry}

\begin{entry}{架势}{9,8}{⽊、⼒}
  \begin{phonetics}{架势}{jia4shi5}
    \definition{s.}{postura | atitude | posição (sobre um assunto, etc.)}
  \end{phonetics}
\end{entry}

\begin{entry}{柏}{9}{⽊}
  \begin{phonetics}{柏}{bai3}
  \seealsoref{柏树}{bai3shu4}
  \end{phonetics}
  \begin{phonetics}{柏}{bo2}
    \definition{s.}{cipreste | usado para transcrever nomes}[柏林,德国城市名。___Berlim, uma cidade alemã.]
  \end{phonetics}
  \begin{phonetics}{柏}{bo4}
    \definition{s.}{cedro; cipreste amarelo}
  \end{phonetics}
\end{entry}

\begin{entry}{柏林}{9,8}{⽊、⽊}
  \begin{phonetics}{柏林}{bo2lin2}
    \definition*{s.}{Berlim, capital da Alemanha}
  \end{phonetics}
\end{entry}

\begin{entry}{柏树}{9,9}{⽊、⽊}
  \begin{phonetics}{柏树}{bai3shu4}
    \definition{s.}{cipreste}
  \end{phonetics}
\end{entry}

\begin{entry}{某}{9}{⽊}
  \begin{phonetics}{某}{mou3}[][HSK 3]
    \definition{pron.}{alguém ou algo indefinido; refere-se a pessoas ou coisas incertas | referindo-se a si mesmo; em vez do seu próprio nome | alguns; certos; refere-se a uma pessoa ou coisa específica cujo nome não se sabe ou não se pode revelar | tal e tal; substituir o nome de outra pessoa (geralmente com um tom rude)}
  \end{phonetics}
\end{entry}

\begin{entry}{染}{9}{⽊}
  \begin{phonetics}{染}{ran3}[][HSK 5]
    \definition*{s.}{sobrenome Ran}
    \definition{s.}{soja fermentada e temperada em forma de pasta}
    \definition{v.}{tingir; pintar | pegar (uma doença); cair em (um mau hábito, etc.) | sujar; contaminar | pegar (contrair) (uma doença) | adquirir (um mau hábito, etc.); contaminar}
  \end{phonetics}
\end{entry}

\begin{entry}{柔软}{9,8}{⽊、⾞}
  \begin{phonetics}{柔软}{rou2ruan3}
    \definition{adj.}{macio | suave}
  \end{phonetics}
\end{entry}

\begin{entry}{柠檬}{9,17}{⽊、⽊}
  \begin{phonetics}{柠檬}{ning2meng2}
    \definition{s.}{limão}
  \end{phonetics}
\end{entry}

\begin{entry}{查}{9}{⽊}
  \begin{phonetics}{查}{cha2}[][HSK 2]
    \definition{v.}{examinar; verificar cuidadosamente | examinar; investigar; entender bem a situação | procurar; consultar; revisar (documentos bibliográficos)}
  \end{phonetics}
  \begin{phonetics}{查}{zha1}
    \definition*{s.}{sobrenome Zha}
    \definition{s.}{espinheiro-chinês}
  \end{phonetics}
\end{entry}

\begin{entry}{查询}{9,8}{⽊、⾔}
  \begin{phonetics}{查询}{cha2 xun2}[][HSK 5]
    \definition{v.}{indagar; inquirir; perguntar sobre}
  \end{phonetics}
\end{entry}

\begin{entry}{柬埔寨}{9,10,14}{⽊、⼟、⼧}
  \begin{phonetics}{柬埔寨}{jian3pu3zhai4}
    \definition*{s.}{Camboja}
  \end{phonetics}
\end{entry}

\begin{entry}{柳}{9}{⽊}
  \begin{phonetics}{柳}{liu3}
    \definition*{s.}{sobrenome Liu}
    \definition{s.}{salgueiro}
  \end{phonetics}
\end{entry}

\begin{entry}{柳橙汁}{9,16,5}{⽊、⽊、⽔}
  \begin{phonetics}{柳橙汁}{liu3cheng2zhi1}
    \definition[瓶,杯,罐,盒]{s.}{suco de laranja}
  \seealsoref{橙汁}{cheng2zhi1}
  \seealsoref{橘子汁}{ju2zi5zhi1}
  \end{phonetics}
\end{entry}

\begin{entry}{标}{9}{⽊}
  \begin{phonetics}{标}{biao1}
    \definition{clas.}{usado para equipes (o numeral é limitado a um, 一, o que é comum no chinês moderno)}
    \definition[个]{s.}{copa da árvore (significado original) | marca; sinal | padrão; cota | sinal externo; sintoma | prêmio; troféu | oferta; licitação comercial pública | a ponta de uma árvore | aparência externa; ramos ou superfícies | partes aéreas das plantas | rótulo; etiqueta; identificação; sinal | regimento na Dinastia Qing; uma das organizações militares no final da Dinastia Qing}
    \definition{v.}{colocar uma marca, etiqueta ou rótulo em; rotular | agrupar; formar equipe | marcar; expressar com palavras ou outras coisas |}
  \end{phonetics}
\end{entry}

\begin{entry}{标志}{9,7}{⽊、⼼}
  \begin{phonetics}{标志}{biao1zhi4}[][HSK 4]
    \definition[个,种]{s.}{sinal; marca; logotipo; símbolo; emblema; marcações que caracterizam um objeto para facilitar a identificação}
    \definition{v.}{marcar; indicar; simbolizar; identificar}
  \end{phonetics}
\end{entry}

\begin{entry}{标准}{9,10}{⽊、⼎}
  \begin{phonetics}{标准}{biao1zhun3}[][HSK 3]
    \definition{adj.}{padrão (que serve como ou está em conformidade com um padrão); em conformidade com os documentos e princípios regulamentares}
    \definition[个,条,项,种]{s.}{padrão; critério; critérios de avaliação das coisas}
  \end{phonetics}
\end{entry}

\begin{entry}{标题}{9,15}{⽊、⾴}
  \begin{phonetics}{标题}{biao1ti2}[][HSK 3]
    \definition[个,条,篇]{s.}{título; manchete; cabeçalho; resumo conciso do conteúdo da obra}
  \end{phonetics}
\end{entry}

\begin{entry}{树}{9}{⽊}
  \begin{phonetics}{树}{shu4}[][HSK 1]
    \definition*{s.}{sobrenome Shu}
    \definition[棵,株]{s.}{árvore; nome comum das plantas lenhosas}
    \definition{v.}{plantar; cultivar | configurar; manter; estabelecer}
  \end{phonetics}
\end{entry}

\begin{entry}{树木}{9,4}{⽊、⽊}
  \begin{phonetics}{树木}{shu4mu4}
    \definition{s.}{árvore}
  \end{phonetics}
\end{entry}

\begin{entry}{树叶}{9,5}{⽊、⼝}
  \begin{phonetics}{树叶}{shu4ye4}[][HSK 4]
    \definition[片,枚,堆]{s.}{folha; folhagem;}
  \end{phonetics}
\end{entry}

\begin{entry}{树林}{9,8}{⽊、⽊}
  \begin{phonetics}{树林}{shu4 lin2}[][HSK 4]
    \definition{s.}{bosque; muitas árvores que crescem em fragmentos, menores que as florestas}
  \end{phonetics}
\end{entry}

\begin{entry}{树莓}{9,10}{⽊、⾋}
  \begin{phonetics}{树莓}{shu4mei2}
    \definition{s.}{framboesa}
  \end{phonetics}
\end{entry}

\begin{entry}{歪}{9}{⽌}
  \begin{phonetics}{歪}{wai1}
    \definition{adj.}{torto | tortuoso | nocivo}
  \end{phonetics}
\end{entry}

\begin{entry}{歪果仁}{9,8,4}{⽌、⽊、⼈}
  \begin{phonetics}{歪果仁}{wai1 guo3 ren2}
    \definition{s.}{gíria na \emph{Internet} para estrangeiro (外国人)}
  \seealsoref{外国人}{wai4 guo2 ren2}
  \end{phonetics}
\end{entry}

\begin{entry}{残}{9}{⽍}
  \begin{phonetics}{残}{can2}
    \definition{adj.}{incompleto; fragmentário; deficiente | remanescente; restante | cruel; feroz | opressivo; selvagem; bárbaro}
    \definition{v.}{ferir; danificar | estragar; prejudicar; destruir}
  \end{phonetics}
\end{entry}

\begin{entry}{残疾人}{9,10,2}{⽍、⽧、⼈}
  \begin{phonetics}{残疾人}{can2ji2ren2}
    \definition{s.}{pessoa com deficiência}
  \end{phonetics}
\end{entry}

\begin{entry}{残酷}{9,14}{⽍、⾣}
  \begin{phonetics}{残酷}{can2ku4}
    \definition{adj.}{cruel}
    \definition{s.}{crueldade}
  \end{phonetics}
\end{entry}

\begin{entry}{段}{9}{⽎}
  \begin{phonetics}{段}{duan4}[][HSK 2]
    \definition*{s.}{sobrenome Duan}
    \definition{clas.}{parte; seção; segmento; usado para dividir objetos em várias partes | passagem; parágrafo; parte de algo que tem características de continuidade | seção; período; usado para uma certa distância no tempo ou no espaço}
    \definition{s.}{nível; dan (no judô, weiqi, etc.) | seção (como nível administrativo em uma mina ou fábrica) | parte; etapa; estágio}
    \definition{v.}{cortar; separar}
  \end{phonetics}
\end{entry}

\begin{entry}{毒}{9}{⽏}
  \begin{phonetics}{毒}{du2}[][HSK 5]
    \definition*{s.}{sobrenome Du}
    \definition{adj.}{veneno; toxina; propriedade ou substância prejudicial aos organismos vivos | droga; narcóticos | vírus; vírus de computador | influência venenosa}
    \definition{adj.}{venenoso; tóxico; envenenado | malicioso; cruel; feroz}
    \definition{v.}{matar com veneno; envenenar | envenenar (a mente de alguém)}
  \end{phonetics}
\end{entry}

\begin{entry}{毒杀}{9,6}{⽏、⽊}
  \begin{phonetics}{毒杀}{du2sha1}
    \definition{v.}{matar por envenenamento}
  \end{phonetics}
\end{entry}

\begin{entry}{毒物}{9,8}{⽏、⽜}
  \begin{phonetics}{毒物}{du2wu4}
    \definition{s.}{substância venenosa | toxina}
  \end{phonetics}
\end{entry}

\begin{entry}{毒害}{9,10}{⽏、⼧}
  \begin{phonetics}{毒害}{du2hai4}
    \definition{s.}{envenenamento}
    \definition{v.}{envenenar (prejudicar com uma substância tóxica) | envenenar (as mentes das pessoas)}
  \end{phonetics}
\end{entry}

\begin{entry}{毒蛇}{9,11}{⽏、⾍}
  \begin{phonetics}{毒蛇}{du2she2}
    \definition{s.}{víbora | cobra venenosa}
  \end{phonetics}
\end{entry}

\begin{entry}{泉}{9}{⽔}
  \begin{phonetics}{泉}{quan2}[][HSK 5]
    \definition*{s.}{sobrenome Quan}
    \definition[股,眼,汪]{s.}{fonte (de água mineral) | a nascente de um rio | termo antigo para moeda}
  \end{phonetics}
\end{entry}

\begin{entry}{洋葱}{9,12}{⽔、⾋}
  \begin{phonetics}{洋葱}{yang2cong1}
    \definition{s.}{cebola}
  \end{phonetics}
\end{entry}

\begin{entry}{洒}{9}{⽔}
  \begin{phonetics}{洒}{sa3}[][HSK 5]
    \definition{adj.}{natural e sem restrições; confortável (sem restrições)}
    \definition{v.}{derramar; espalhar; borrifar; salpicar; fazer com que (água ou outra coisa) caia de forma dispersa | derramar; cair de forma dispersa}
  \end{phonetics}
\end{entry}

\begin{entry}{洒水}{9,4}{⽔、⽔}
  \begin{phonetics}{洒水}{sa3shui3}
    \definition{v.}{borrifar}
  \end{phonetics}
\end{entry}

\begin{entry}{洗}{9}{⽔}
  \begin{phonetics}{洗}{xi3}[][HSK 1]
    \definition[个]{s.}{pequeno recipiente contendo água para enxaguar os pincéis de escrever | batismo}
    \definition{v.}{lavar; tomar banho; remover a sujeira do objeto com água ou outro solvente | batizar | eliminar; corrigir; reparar | saquear; matar e pilhar; matar ou roubar tudo, como se tivesse sido lavado | revelar filmes; imprimir fotos | apagar; limpar (uma gravação, etc.) | embaralhar (cartas, etc.)}
  \end{phonetics}
\end{entry}

\begin{entry}{洗手}{9,4}{⽔、⼿}
  \begin{phonetics}{洗手}{xi3shou3}
    \definition{v.}{ir ao banheiro | lavar as mãos}
  \end{phonetics}
\end{entry}

\begin{entry}{洗手不干}{9,4,4,3}{⽔、⼿、⼀、⼲}
  \begin{phonetics}{洗手不干}{xi3shou3bu2gan4}
    \definition{v.}{parar totalmente de fazer algo}
  \end{phonetics}
\end{entry}

\begin{entry}{洗手池}{9,4,6}{⽔、⼿、⽔}
  \begin{phonetics}{洗手池}{xi3shou3chi2}
    \definition{s.}{pia de banheiro | lavatório}
  \seealsoref{洗手盆}{xi3shou3pen2}
  \end{phonetics}
\end{entry}

\begin{entry}{洗手间}{9,4,7}{⽔、⼿、⾨}
  \begin{phonetics}{洗手间}{xi3shou3jian1}[][HSK 1]
    \definition[个]{s.}{banheiro; lavatório; lavabo}
  \end{phonetics}
\end{entry}

\begin{entry}{洗手乳}{9,4,8}{⽔、⼿、⼄}
  \begin{phonetics}{洗手乳}{xi3shou3ru3}
    \definition{s.}{sabonete líquido para lavar as mãos}
  \seealsoref{洗手液}{xi3shou3ye4}
  \end{phonetics}
\end{entry}

\begin{entry}{洗手盆}{9,4,9}{⽔、⼿、⽫}
  \begin{phonetics}{洗手盆}{xi3shou3pen2}
    \definition{s.}{pia de banheiro | lavatório}
  \seealsoref{洗手池}{xi3shou3chi2}
  \end{phonetics}
\end{entry}

\begin{entry}{洗手液}{9,4,11}{⽔、⼿、⽔}
  \begin{phonetics}{洗手液}{xi3shou3ye4}
    \definition{s.}{sabonete líquido para lavar as mãos}
  \seealsoref{洗手乳}{xi3shou3ru3}
  \end{phonetics}
\end{entry}

\begin{entry}{洗礼}{9,5}{⽔、⽰}
  \begin{phonetics}{洗礼}{xi3li3}
    \definition{s.}{batismo}
    \definition{v.}{batizar}
  \end{phonetics}
\end{entry}

\begin{entry}{洗衣机}{9,6,6}{⽔、⾐、⽊}
  \begin{phonetics}{洗衣机}{xi3 yi1 ji1}[][HSK 2]
    \definition[台]{s.}{máquina de lavar roupa; eletrodomésticos para lavagem automática ou semiautomática de roupas}
  \end{phonetics}
\end{entry}

\begin{entry}{洗劫}{9,7}{⽔、⼒}
  \begin{phonetics}{洗劫}{xi3jie2}
    \definition{v.}{saquear | pilhar | roubar}
  \end{phonetics}
\end{entry}

\begin{entry}{洗净}{9,8}{⽔、⼎}
  \begin{phonetics}{洗净}{xi3jing4}
    \definition{v.}{lavar (limpeza)}
  \end{phonetics}
\end{entry}

\begin{entry}{洗胃}{9,9}{⽔、⾁}
  \begin{phonetics}{洗胃}{xi3wei4}
    \definition{s.}{(medicina) lavagem gástrica}
    \definition{v.}{ter o estômago lavado}
  \end{phonetics}
\end{entry}

\begin{entry}{洗涤}{9,10}{⽔、⽔}
  \begin{phonetics}{洗涤}{xi3di2}
    \definition{s.}{enxágue | lava}
    \definition{v.}{enxaguar | lavar}
  \end{phonetics}
\end{entry}

\begin{entry}{洗涤间}{9,10,7}{⽔、⽔、⾨}
  \begin{phonetics}{洗涤间}{xi3di2jian1}
    \definition{s.}{lavanderia}
  \end{phonetics}
\end{entry}

\begin{entry}{洗脱}{9,11}{⽔、⾁}
  \begin{phonetics}{洗脱}{xi3tuo1}
    \definition{v.}{limpar | purgar | lavar}
  \end{phonetics}
\end{entry}

\begin{entry}{洗碗}{9,13}{⽔、⽯}
  \begin{phonetics}{洗碗}{xi3wan3}
    \definition{v.}{lavar pratos}
  \end{phonetics}
\end{entry}

\begin{entry}{洗澡}{9,16}{⽔、⽔}
  \begin{phonetics}{洗澡}{xi3zao3}[][HSK 2]
    \definition{v.+compl.}{tomar banho; tomar banho de chuveiro; lavar-se}
  \end{phonetics}
\end{entry}

\begin{entry}{洗澡间}{9,16,7}{⽔、⽔、⾨}
  \begin{phonetics}{洗澡间}{xi3zao3jian1}
    \definition[间]{s.}{banheiro}
  \end{phonetics}
\end{entry}

\begin{entry}{洞}{9}{⽔}
  \begin{phonetics}{洞}{dong4}[][HSK 5]
    \definition{adj.}{profundo; minucioso; claro; completo; abrangente}
    \definition{s.}{buraco; cavidade; orifício; furo; parte penetrante ou profundamente recuada de um objeto; uma caverna}
  \end{phonetics}
\end{entry}

\begin{entry}{洞穴}{9,5}{⽔、⽳}
  \begin{phonetics}{洞穴}{dong4xue2}
    \definition{s.}{caverna}
  \end{phonetics}
\end{entry}

\begin{entry}{洪水}{9,4}{⽔、⽔}
  \begin{phonetics}{洪水}{hong2shui3}
    \definition{s.}{enchente | inundação | dilúvio}
  \end{phonetics}
\end{entry}

\begin{entry}{洲}{9}{⽔}
  \begin{phonetics}{洲}{zhou1}
    \definition{s.}{continente | ilha em um rio}
  \end{phonetics}
\end{entry}

\begin{entry}{活}{9}{⽔}
  \begin{phonetics}{活}{huo2}[][HSK 3]
    \definition{adj.}{vivo; vivendo; indica que (alguma ação) foi realizada enquanto a pessoa ainda estava viva | vívido; animado; ativo | móvel; em movimento; ativo}
    \definition{adv.}{exatamente; simplesmente; expressa um grau elevado, equivalente a 真正 ou 简直}
    \definition{s.}{emprego; meios de subsistência; trabalho (geralmente refere-se a trabalho físico) | produto; algo fabricado}
    \definition{v.}{viver; ter vida; sobreviver (em oposição a 死) | salvar (a vida de uma pessoa); fazer sobreviver; manter a vida}
  \seealsoref{简直}{jian3zhi2}
  \seealsoref{死}{si3}
  \seealsoref{真正}{zhen1zheng4}
  \end{phonetics}
\end{entry}

\begin{entry}{活力}{9,2}{⽔、⼒}
  \begin{phonetics}{活力}{huo2li4}[][HSK 5]
    \definition{s.}{vigor; vitalidade; energia; muito forte, geralmente usado para descrever pessoas, cidades, empresas, economias, etc.}
  \end{phonetics}
\end{entry}

\begin{entry}{活动}{9,6}{⽔、⼒}
  \begin{phonetics}{活动}{huo2dong4}[][HSK 2]
    \definition{adj.}{móvel; flexível para alterações ou mudanças}
    \definition[些,个,种,类,次]{s.}{atividade; ação tomada com o objetivo de alcançar um determinado objetivo}
    \definition{v.}{fazer exercício; movimentar-se | usar influência pessoal; usar meios irregulares | mover-se}
  \end{phonetics}
\end{entry}

\begin{entry}{活泼}{9,8}{⽔、⽔}
  \begin{phonetics}{活泼}{huo2po1}[][HSK 5]
    \definition{adj.}{vívido; ativo; animado; brilhante; vivaz; cheio de vida | reativo; (química) significa que a substância é ativa e reage facilmente com outras substâncias}
  \end{phonetics}
\end{entry}

\begin{entry}{活着}{9,11}{⽔、⽬}
  \begin{phonetics}{活着}{huo2zhe5}
    \definition{adj.}{vivo}
  \end{phonetics}
\end{entry}

\begin{entry}{活路}{9,13}{⽔、⾜}
  \begin{phonetics}{活路}{huo2lu4}
    \definition{s.}{maneira de sobreviver | meio de subsistência}
  \end{phonetics}
  \begin{phonetics}{活路}{huo2lu5}
    \definition{s.}{labor | trabalho físico}
  \end{phonetics}
\end{entry}

\begin{entry}{派}{9}{⽔}
  \begin{phonetics}{派}{pai4}[][HSK 3]
    \definition{adj.}{elegante; bonito; imponente}
    \definition{clas.}{usado para grupos, escolas de pensamento ou arte, etc. | usado para um discursos, situações, cenas, etc.}
    \definition[个,块,种]{s.}{panelinha; facção; pessoas com ideias, visões e estilos semelhantes | torta; um alimento recheado comumente consumido pelos ocidentais, geralmente doce | maneira e ar; estilo ou comportamento | afluente; braço de rio}
    \definition{v.}{enviar; despachar; arranjar ou ordenar que uma pessoa faça algo; providenciar transporte | alocar; repartir; distribuir}
  \end{phonetics}
\end{entry}

\begin{entry}{测}{9}{⽔}
  \begin{phonetics}{测}{ce4}[][HSK 4]
    \definition{v.}{pesquisar; sondar; medir | conjecturar; advinhar}
  \end{phonetics}
\end{entry}

\begin{entry}{测试}{9,8}{⽔、⾔}
  \begin{phonetics}{测试}{ce4 shi4}[][HSK 4]
    \definition[个]{s.}{exame; teste; medição do conhecimento humano, das habilidades ou do funcionamento de máquinas, ferramentas ou instrumentos}
    \definition{v.}{examinar | testar, medição do desempenho e da precisão de máquinas, instrumentos, aparelhos, etc.}
  \end{phonetics}
\end{entry}

\begin{entry}{测量}{9,12}{⽔、⾥}
  \begin{phonetics}{测量}{ce4liang2}[][HSK 4]
    \definition{v.}{aferir; pesquisar; medir; determinar valores relevantes para espaço, tempo, temperatura, velocidade, função, etc.}
  \end{phonetics}
\end{entry}

\begin{entry}{浓}{9}{⽔}
  \begin{phonetics}{浓}{nong2}[][HSK 4]
    \definition{adj.}{denso; espesso; concentrado; um líquido ou gás que contém mais de um determinado ingrediente | grande; forte; profundo (de grau ou extensão) | profundo; (algumas cores) escuro}
  \end{phonetics}
\end{entry}

\begin{entry}{点}{9}{⽕}
  \begin{phonetics}{点}{dian3}[][HSK 1]
    \definition{clas.}{hora cheia | ponto, uma unidade de medida para tipos; antigamente, a contagem do tempo durante a noite era feita por turnos, sendo cada turno dividido em cinco pontos | quantidade ínfima; um pouco; um pouquinho; alguma coisa; indica uma pequena quantidade | usado para itens}
    \definition{s.}{gota (de líquido); (ponto) pequena gota de líquido | mancha; ponto; salpico; (um pouco) Um pequeno vestígio | (ponto) Traço de um caractere chinês, cuja forma é ``、''  | ponto; (matemática) refere-se a uma figura geométrica que não tem comprimento, largura ou altura, mas apenas uma posição | gongo, instrumento musical de metal | ponto decimal; refere-se ao ponto decimal, símbolo matemático que representa os números decimais | lugar específico | lanche leve; petisco | lugar; grau; sinalização de um determinado local ou grau | hora marcada; hora regulamentar | aspecto; característica; partes ou aspectos específicos de algo | ritmo; batida}
    \definition{v.}{andar na ponta dos pés | dar uma dica, sugestão | tocar levemente com o dedo, pincel ou vara; tocar muito brevemente; passar rapidamente | acenar; baixar ligeiramente a cabeça e levantar rapidamente | gotejar; fazer cair líquido | semear em buracos; plantar com um plantador | verificar um por um | colocar um ponto; usar caneta e outras ferramentas para adicionar ideias | sugerir; indicar; dar uma dica | decorar; realçar | selecionar; escolher; especificar o que é exigido | acender; queimar; inflamar | (pedido) comer uma pequena quantidade de comida para saciar a fome}
  \end{phonetics}
\end{entry}

\begin{entry}{点火}{9,4}{⽕、⽕}
  \begin{phonetics}{点火}{dian3huo3}
    \definition{s.}{ignição}
    \definition{v.}{inflamar | acender um fogo | agitar | dar partida em um motor | (figurativo) provocar problemas}
  \end{phonetics}
\end{entry}

\begin{entry}{点头}{9,5}{⽕、⼤}
  \begin{phonetics}{点头}{dian3 tou2}[][HSK 2]
    \definition{v.}{acenar com a cabeça; balançar a cabeça; mover ligeiramente a cabeça para baixo; indicar permissão, aprovação, compreensão ou saudação}
  \end{phonetics}
\end{entry}

\begin{entry}{点名}{9,6}{⽕、⼝}
  \begin{phonetics}{点名}{dian3 ming2}[][HSK 4]
    \definition{v.}{fazer a lista de chamada; manter o controle da presença de alguém; chamar nomes para controle de presença | mencionar alguém pelo nome}
  \end{phonetics}
\end{entry}

\begin{entry}{点燃}{9,16}{⽕、⽕}
  \begin{phonetics}{点燃}{dian3 ran2}[][HSK 5]
    \definition{v.}{acender; inflamar; acender uma fogueira, para iluminar}
  \end{phonetics}
\end{entry}

\begin{entry}{炼}{9}{⽕}
  \begin{phonetics}{炼}{lian4}
    \definition{v.}{fundir; refinar | temperar (um metal) com fogo | pesar a palavra; procurar a frase certa; polir | trabalhar; tornar uma substância pura ou resistente por aquecimento, etc. | polir; fazer as palavras bonitas e concisas}
  \end{phonetics}
\end{entry}

\begin{entry}{烂}{9}{⽕}
  \begin{phonetics}{烂}{lan4}[][HSK 5]
    \definition{adj.}{macio; pastoso; amassado | podre; deteriorado | quebrado; esfarrapado; gasto | desorganizado; indigno}
    \definition{adv.}{totalmente; extremamente; completamente; expressa um grau muito profundo}
    \definition{v.}{apodrecer; infeccionar; decompor-se}
  \end{phonetics}
\end{entry}

\begin{entry}{独}{9}{⽝}
  \begin{phonetics}{独}{du2}
    \definition*{s.}{sobrenome Du}
    \definition{adj.}{só; solteiro | (coloquial) distante | único; só}
    \definition{adv.}{unicamente; somente | sozinho; por si mesmo; em solidão}
    \definition{s.}{idosos sem descendência; os sem filhos}
  \end{phonetics}
\end{entry}

\begin{entry}{独立}{9,5}{⽝、⽴}
  \begin{phonetics}{独立}{du2li4}[][HSK 4]
    \definition{adj.}{independente; por conta própria | separado; respectivo; descreve algo que é separado e não está em contato com outra coisa}
    \definition{prep.}{independente de; separado de; não mais anexado à unidade original, mas uma unidade separada}
    \definition{v.}{ficar sozinho | alcançar a independência; tornar-se um país independente; liberdade de um Estado, regime ou organização contra interferência, controle e dominação por forças externas}
  \end{phonetics}
\end{entry}

\begin{entry}{独自}{9,6}{⽝、⾃}
  \begin{phonetics}{独自}{du2 zi4}[][HSK 4]
    \definition{adj.}{sozinho; por si mesmo; por conta própria}
  \end{phonetics}
\end{entry}

\begin{entry}{独特}{9,10}{⽝、⽜}
  \begin{phonetics}{独特}{du2te4}[][HSK 4]
    \definition{adj.}{único; distinto; original; especial}
  \end{phonetics}
\end{entry}

\begin{entry}{狭}{9}{⽝}
  \begin{phonetics}{狭}{xia2}
    \definition{adj.}{estreito}
  \end{phonetics}
\end{entry}

\begin{entry}{玻}{9}{⽟}
  \begin{phonetics}{玻}{bo1}
    \definition{s.}{vidro}
  \end{phonetics}
\end{entry}

\begin{entry}{玻璃}{9,14}{⽟、⽟}
  \begin{phonetics}{玻璃}{bo1li5}[][HSK 5]
    \definition[张,块]{s.}{vidro; corpo duro, quebradiço e transparente, geralmente feito de areia, calcário, carbonato de sódio, etc. | \emph{nylon}; plástico; refere-se a determinados plásticos que se assemelham ao vidro.}
  \end{phonetics}
\end{entry}

\begin{entry}{珍贵}{9,9}{⽟、⾙}
  \begin{phonetics}{珍贵}{zhen1gui4}[][HSK 5]
    \definition{adj.}{raro; valioso; precioso; de grande valor; profundo significado}
  \end{phonetics}
\end{entry}

\begin{entry}{珍珠}{9,10}{⽟、⽟}
  \begin{phonetics}{珍珠}{zhen1zhu1}[][HSK 5]
    \definition[颗,串]{s.}{pérola; grânulos redondos produzidos nas conchas de certos animais aquáticos, de cor branca, rosa, etc., bonitos e brilhantes, frequentemente usados como adornos}
  \end{phonetics}
\end{entry}

\begin{entry}{珍惜}{9,11}{⽟、⼼}
  \begin{phonetics}{珍惜}{zhen1xi1}[][HSK 5]
    \definition{v.}{valorizar; estimar; valorizar e evitar o desperdício}
  \end{phonetics}
\end{entry}

\begin{entry}{甚而}{9,6}{⽢、⽽}
  \begin{phonetics}{甚而}{shen4'er2}
    \definition{conj.}{(ir) tão longe quanto | tanto que}
  \end{phonetics}
\end{entry}

\begin{entry}{甚至}{9,6}{⽢、⾄}
  \begin{phonetics}{甚至}{shen4zhi4}[][HSK 4]
    \definition{conj.}{e até mesmo; nem mesmo; para apresentar uma situação típica e especial, para enfatizar a profundidade e a seriedade de uma situação}
  \end{phonetics}
\end{entry}

\begin{entry}{甚或}{9,8}{⽢、⼽}
  \begin{phonetics}{甚或}{shen4huo4}
    \definition{conj.}{(ir) tão longe quanto | tanto que}
  \end{phonetics}
\end{entry}

\begin{entry}{甭}{9}{⽤}
  \begin{phonetics}{甭}{beng2}
    \definition{adv.}{não; não precisa; não tem que; contração de 不用}
  \seealsoref{不用}{bu2 yong4}
  \end{phonetics}
\end{entry}

\begin{entry}{界碑}{9,13}{⽥、⽯}
  \begin{phonetics}{界碑}{jie4bei1}
    \definition{s.}{marco de fronteira}
  \end{phonetics}
\end{entry}

\begin{entry}{疯}{9}{⽧}
  \begin{phonetics}{疯}{feng1}[][HSK 5]
    \definition{adj.}{louco; insano | tolo; leviano | (de uma planta, safra de grãos, etc.) esguia; refere-se ao crescimento vigoroso das plantações, mas sem frutos | com todas as forças; fazer o máximo possível}
    \definition{v.}{jogar sem restrições}
  \end{phonetics}
\end{entry}

\begin{entry}{疯狂}{9,7}{⽧、⽝}
  \begin{phonetics}{疯狂}{feng1kuang2}[][HSK 5]
    \definition{adj.}{louco; insano; frenético; desenfreado}
  \end{phonetics}
\end{entry}

\begin{entry}{皆}{9}{⽩}
  \begin{phonetics}{皆}{jie1}
    \definition{adv.}{todos | em todos os casos}
  \end{phonetics}
\end{entry}

\begin{entry}{皇帝}{9,9}{⽩、⼱}
  \begin{phonetics}{皇帝}{huang2di4}
    \definition[个]{s.}{imperador}
  \end{phonetics}
\end{entry}

\begin{entry}{盆}{9}{⽫}
  \begin{phonetics}{盆}{pen2}[][HSK 5]
    \definition*{s.}{sobrenome Pen}
    \definition[个]{s.}{bacia; banheira; panela; utensílios para guardar ou lavar coisas}
  \end{phonetics}
\end{entry}

\begin{entry}{盆友}{9,4}{⽫、⼜}
  \begin{phonetics}{盆友}{pen2you3}
    \definition{s.}{(gíria na \emph{Internet}) amigo (trocadilho com 朋友)}
  \seealsoref{朋友}{peng2you5}
  \end{phonetics}
\end{entry}

\begin{entry}{相}{9}{⽬}
  \begin{phonetics}{相}{xiang1}
    \definition*{s.}{sobrenome Xiang}
    \definition{adv.}{uns aos outros; mutuamente | (para uma ação realizada por uma pessoa em relação a outra) | indica a ação de uma parte em relação à outra parte}
    \definition{s.}{qualidade; substância}
    \definition{v.}{ver por si mesmo (se algo ou algo é do seu agrado)}
  \end{phonetics}
  \begin{phonetics}{相}{xiang4}
    \definition*{s.}{sobrenome Xiang}
    \definition{s.}{aparência | postura; porte; postura sentada, em pé, etc. | (física) fase; refere-se a uma parte homogênea de uma substância com a mesma composição e as mesmas propriedades físicas e químicas | fotografia | primeiro-ministro (na China antiga) | ministro; títulos oficiais de certos países | fácies marinha (carvão) | elefante, uma das peças do xadrez chinês | recepcionista (pessoa que ajuda o anfitrião a receber o hóspede); antigamente, referia-se a alguém que ajudava o anfitrião a receber convidados}
    \definition{v.}{olhar e avaliar; observe a aparência das coisas; julgar sua qualidade | assistir; ajudar; auxiliar}
  \end{phonetics}
\end{entry}

\begin{entry}{相互}{9,4}{⽬、⼆}
  \begin{phonetics}{相互}{xiang1 hu4}[][HSK 3]
    \definition{adj.}{mútuo; recíproco; entre duas pessoas ou coisas}
    \definition{adv.}{mutuamente; um ao outro; tratamento recíproco}
  \end{phonetics}
\end{entry}

\begin{entry}{相反}{9,4}{⽬、⼜}
  \begin{phonetics}{相反}{xiang1fan3}[][HSK 4]
    \definition{adj.}{oposto; contrário; dois aspectos das coisas são contraditórios e mutuamente exclusivos}
    \definition{conj.}{pelo contrário; usado no início ou no meio de uma frase para indicar uma contradição de significado com o que foi dito anteriormente.}
  \end{phonetics}
\end{entry}

\begin{entry}{相比}{9,4}{⽬、⽐}
  \begin{phonetics}{相比}{xiang1 bi3}[][HSK 3]
    \definition{v.}{combinar; comparar com | comparar mutuamente, usar uma coisa como padrão, perceber as características de outra coisa ou obter uma opinião}
  \end{phonetics}
\end{entry}

\begin{entry}{相片}{9,4}{⽬、⽚}
  \begin{phonetics}{相片}{xiang4 pian4}[][HSK 4]
    \definition[张]{s.}{foto; fotografia; uma imagem de uma pessoa ou objeto feita pela exposição de papel fotográfico a um negativo fotográfico e, em seguida, revelando e fixando a imagem.}
  \end{phonetics}
\end{entry}

\begin{entry}{相处}{9,5}{⽬、⼡}
  \begin{phonetics}{相处}{xiang1chu3}[][HSK 4]
    \definition{v.}{dar-se bem; viver juntos; dar-se bem (uns com os outros); viver uns com os outros; entrar em contato uns com os outros, tratar uns aos outros}
  \end{phonetics}
\end{entry}

\begin{entry}{相似}{9,6}{⽬、⼈}
  \begin{phonetics}{相似}{xiang1si4}[][HSK 3]
    \definition{v.}{assemelhar-se; ser semelhante; ser parecido}
  \end{phonetics}
\end{entry}

\begin{entry}{相关}{9,6}{⽬、⼋}
  \begin{phonetics}{相关}{xiang1guan1}[][HSK 3]
    \definition{v.}{estar mutuamente relacionado; estar intimamente relacionado; estar inter-relacionado}
  \end{phonetics}
\end{entry}

\begin{entry}{相同}{9,6}{⽬、⼝}
  \begin{phonetics}{相同}{xiang1tong2}[][HSK 2]
    \definition{adj.}{semelhante; similar; igual; idêntico; o mesmo; consistentes entre si, sem diferença}
  \end{phonetics}
\end{entry}

\begin{entry}{相当}{9,6}{⽬、⼹}
  \begin{phonetics}{相当}{xiang1dang1}[][HSK 3]
    \definition{adj.}{adequado; apropriado}
    \definition{adv.}{bastante; razoavelmente; consideravelmente; indica um grau relativamente alto e profundo}
    \definition{v.}{combinar; equilibrar; corresponder a; ser aproximadamente igual a; ser proporcional a}
  \end{phonetics}
\end{entry}

\begin{entry}{相机}{9,6}{⽬、⽊}
  \begin{phonetics}{相机}{xiang4 ji1}[][HSK 2]
    \definition[台,部,架,个]{s.}{câmera; máquina fotográfica}
    \definition{v.}{ficar atento a uma oportunidade; procurar oportunidades}
  \end{phonetics}
\end{entry}

\begin{entry}{相声}{9,7}{⽬、⼠}
  \begin{phonetics}{相声}{xiang4sheng5}[][HSK 5]
    \definition[个,段]{s.}{conversa cruzada; diálogo cômico; forma de performance humorística, em que os atores usam piadas, canções e imitações para satirizar e elogiar}
  \end{phonetics}
\end{entry}

\begin{entry}{相应}{9,7}{⽬、⼴}
  \begin{phonetics}{相应}{xiang1ying4}[][HSK 5]
    \definition{v.}{corresponder}
  \end{phonetics}
\end{entry}

\begin{entry}{相宜}{9,8}{⽬、⼧}
  \begin{phonetics}{相宜}{xiang1yi2}
    \definition{adj.}{adequado | apropriado}
    \definition{v.}{ser adequado ou apropriado}
  \end{phonetics}
\end{entry}

\begin{entry}{相亲}{9,9}{⽬、⼇}
  \begin{phonetics}{相亲}{xiang1qin1}
    \definition{s.}{encontro às cegas | entrevista arranjada para avaliar a proposta de um parceiro de casamento | apegar-se profundamente um ao outro}
  \end{phonetics}
\end{entry}

\begin{entry}{相信}{9,9}{⽬、⼈}
  \begin{phonetics}{相信}{xiang1xin4}[][HSK 2]
    \definition{v.}{acreditar em; estar convencido de; ter fé em; acreditar que algo é certo ou verdadeiro sem dúvida}
  \end{phonetics}
\end{entry}

\begin{entry}{相思病}{9,9,10}{⽬、⼼、⽧}
  \begin{phonetics}{相思病}{xiang1si1bing4}
    \definition{s.}{saudade de amor}
  \end{phonetics}
\end{entry}

\begin{entry}{相等}{9,12}{⽬、⽵}
  \begin{phonetics}{相等}{xiang1deng3}[][HSK 5]
    \definition{v.}{ser igual a; possuir a mesma quantidade, peso, tamanho e grau}
  \end{phonetics}
\end{entry}

\begin{entry}{相遇}{9,12}{⽬、⾡}
  \begin{phonetics}{相遇}{xiang1yu4}
    \definition{v.}{encontrar (reunião, encontro, etc.)}
  \end{phonetics}
\end{entry}

\begin{entry}{相聚}{9,14}{⽬、⽿}
  \begin{phonetics}{相聚}{xiang1ju4}
    \definition{v.}{reunir-se | montar}
  \end{phonetics}
\end{entry}

\begin{entry}{省}{9}{⽬}
  \begin{phonetics}{省}{sheng3}[][HSK 2]
    \definition*{s.}{sobrenome Sheng}
    \definition{s.}{província; unidade administrativa, subordinada diretamente ao governo central | capital provincial; refere-se à capital da província, localização da administração provincial | abreviação (de palavras)}
    \definition{v.}{economizar; poupar; reduzir o consumo (em oposição a 费) | omitir; deixar de fora}
  \seealsoref{费}{fei4}
  \end{phonetics}
  \begin{phonetics}{省}{xing3}
    \definition{v.}{examinar-se criticamente; verificar (os próprios pensamentos, palavras e ações) | visitar (especialmente os pais ou pessoas mais velhas) | estar ciente; tornar-se consciente; compreender; tomar consciência | examinar minuciosamente; inspecionar; escrutinar}
  \end{phonetics}
\end{entry}

\begin{entry}{省力}{9,2}{⽬、⼒}
  \begin{phonetics}{省力}{sheng3li4}
    \definition{v.}{economizar esforço ou trabalho}
  \end{phonetics}
\end{entry}

\begin{entry}{省心}{9,4}{⽬、⼼}
  \begin{phonetics}{省心}{sheng3xin1}
    \definition{adj.}{despreocupado}
    \definition{v.}{ser poupado de preocupações | despreocupar-se}
  \end{phonetics}
\end{entry}

\begin{entry}{省长}{9,4}{⽬、⾧}
  \begin{phonetics}{省长}{sheng3zhang3}
    \definition*{s.}{Governador | governador de uma província}
  \end{phonetics}
\end{entry}

\begin{entry}{省会}{9,6}{⽬、⼈}
  \begin{phonetics}{省会}{sheng3hui4}
    \definition{s.}{capital da província}
  \end{phonetics}
\end{entry}

\begin{entry}{省却}{9,7}{⽬、⼙}
  \begin{phonetics}{省却}{sheng3que4}
    \definition{v.}{livrar-se (para economizar espaço) | salvar}
  \end{phonetics}
\end{entry}

\begin{entry}{省俭}{9,9}{⽬、⼈}
  \begin{phonetics}{省俭}{sheng3jian3}
    \definition{s.}{econômico | frugal}
    \definition{v.}{economizar}
  \end{phonetics}
\end{entry}

\begin{entry}{省城}{9,9}{⽬、⼟}
  \begin{phonetics}{省城}{sheng3cheng2}
    \definition{s.}{capital da província}
  \end{phonetics}
\end{entry}

\begin{entry}{省悟}{9,10}{⽬、⼼}
  \begin{phonetics}{省悟}{xing3wu4}
    \definition{v.}{voltar a si | constatar | ver a verdade | acordar para a realidade}
  \end{phonetics}
\end{entry}

\begin{entry}{省钱}{9,10}{⽬、⾦}
  \begin{phonetics}{省钱}{sheng3qian2}
    \definition{v.}{economizar dinheiro}
  \end{phonetics}
\end{entry}

\begin{entry}{眉}{9}{⽬}
  \begin{phonetics}{眉}{mei2}
    \definition{s.}{sobrancelha | margem superior}
  \end{phonetics}
\end{entry}

\begin{entry}{眉毛}{9,4}{⽬、⽑}
  \begin{phonetics}{眉毛}{mei2mao5}
    \definition[根]{s.}{sobrancelha}
  \end{phonetics}
\end{entry}

\begin{entry}{眉头}{9,5}{⽬、⼤}
  \begin{phonetics}{眉头}{mei2tou2}
    \definition{s.}{testa}
  \end{phonetics}
\end{entry}

\begin{entry}{看}{9}{⽬}
  \begin{phonetics}{看}{kan1}
    \definition{v.}{cuidar de; tomar conta de; cuidar de; proteger | manter sob vigilância}
  \end{phonetics}
  \begin{phonetics}{看}{kan4}[][HSK 1]
    \definition{interj.}{Cuidado! (para um perigo)}
    \definition{part.}{usado depois de outros verbos, tentar}
    \definition{v.}{ver; olhar para; observar; fazer contato visual com pessoas ou objetos | pensar; considerar; observar; julgar; observar e analisar | visitar; ver; fazer uma visita | olhar para; considerar; tratar | tratar (um paciente ou uma doença) | cuidar | ficar atento; ficar de olho | depender de; ser dependente de | ler}
  \end{phonetics}
\end{entry}

\begin{entry}{看上去}{9,3,5}{⽬、⼀、⼛}
  \begin{phonetics}{看上去}{kan4 shang4 qu4}[][HSK 3]
    \definition{adv.}{parece que}
  \end{phonetics}
\end{entry}

\begin{entry}{看不起}{9,4,10}{⽬、⼀、⾛}
  \begin{phonetics}{看不起}{kan4bu5qi3}[][HSK 4]
    \definition{v.}{desprezar; desdenhar; menosprezar; ter desprezo; olhar de cima para baixo}
  \end{phonetics}
\end{entry}

\begin{entry}{看见}{9,4}{⽬、⾒}
  \begin{phonetics}{看见}{kan4jian4}[][HSK 1]
    \definition{v.}{ver; avistar; ao olhar, descobrir alguém ou algo}
  \end{phonetics}
\end{entry}

\begin{entry}{看出}{9,5}{⽬、⼐}
  \begin{phonetics}{看出}{kan4 chu1}[][HSK 5]
    \definition{v.}{perceber; descobrir; estar ciente de; ver}
  \end{phonetics}
\end{entry}

\begin{entry}{看成}{9,6}{⽬、⼽}
  \begin{phonetics}{看成}{kan4 cheng2}[][HSK 5]
    \definition{v.}{olhar como; considerar como; tratar como; pensar como; ter como}
  \end{phonetics}
\end{entry}

\begin{entry}{看来}{9,7}{⽬、⽊}
  \begin{phonetics}{看来}{kan4 lai2}[][HSK 4]
    \definition{adv.}{parecer; parecer como se (ou embora); refere-se a um julgamento aproximado; expressa um julgamento por observação}
    \definition{v.}{ser considerado; na visão de alguém; na opinião de alguém; expressar a ideia aproximada que o locutor tem da situação}
  \end{phonetics}
\end{entry}

\begin{entry}{看到}{9,8}{⽬、⼑}
  \begin{phonetics}{看到}{kan4 dao4}[][HSK 1]
    \definition{v.}{ver; avistar}
  \end{phonetics}
\end{entry}

\begin{entry}{看法}{9,8}{⽬、⽔}
  \begin{phonetics}{看法}{kan4fa3}[][HSK 2]
    \definition[个,种,点]{s.}{opinião; perspectiva; (ponto de) vista; uma maneira de ver uma coisa | opinião desfavorável (ou crítica) sobre alguém}
  \end{phonetics}
\end{entry}

\begin{entry}{看待}{9,9}{⽬、⼻}
  \begin{phonetics}{看待}{kan4dai4}[][HSK 5]
    \definition{v.}{tratar; considerar; olhar com atenção; ter uma certa atitude ou visão em relação a alguém ou alguma coisa}
  \end{phonetics}
\end{entry}

\begin{entry}{看病}{9,10}{⽬、⽧}
  \begin{phonetics}{看病}{kan4 bing4}[][HSK 1]
    \definition{v.+compl.}{(de um médico) ver um paciente | (de um paciente) ver (consultar) um médico}
  \end{phonetics}
\end{entry}

\begin{entry}{看起来}{9,10,7}{⽬、⾛、⽊}
  \begin{phonetics}{看起来}{kan4 qi3 lai5}[][HSK 3]
    \definition{v.}{parecer; aparentar; dar a impressão de (ou como se)}
  \end{phonetics}
\end{entry}

\begin{entry}{看望}{9,11}{⽬、⽉}
  \begin{phonetics}{看望}{kan4wang4}[][HSK 4]
    \definition{v.}{ver; visitar; ligar; dar uma olhada; ir até os pais, idosos, professores ou amigos para cumprimentá-los}
  \end{phonetics}
\end{entry}

\begin{entry}{看淡}{9,11}{⽬、⽔}
  \begin{phonetics}{看淡}{kan4dan4}
    \definition{v.}{considerar sem importância | ser indiferente a (fama, riqueza, etc.) | (de uma economia ou mercado) enfraquecer, ficar mais lento, diminuir a velocidade}
  \end{phonetics}
\end{entry}

\begin{entry}{矜}{9}{⽭}
  \begin{phonetics}{矜}{jin1}
    \definition{adj.}{presunçoso; vaidoso | contido; reservado; determinado}
    \definition{v.}{ter pena; simpatizar com; compadecer-se}
  \end{phonetics}
\end{entry}

\begin{entry}{砂}{9}{⽯}
  \begin{phonetics}{砂}{sha1}
    \variantof{沙}
  \end{phonetics}
\end{entry}

\begin{entry}{砍}{9}{⽯}
  \begin{phonetics}{砍}{kan3}
    \definition{v.}{cortar}
  \end{phonetics}
\end{entry}

\begin{entry}{砍刀}{9,2}{⽯、⼑}
  \begin{phonetics}{砍刀}{kan3dao1}
    \definition{s.}{facão | machete}
  \end{phonetics}
\end{entry}

\begin{entry}{砍头}{9,5}{⽯、⼤}
  \begin{phonetics}{砍头}{kan3tou2}
    \definition{v.}{decapitar}
  \end{phonetics}
\end{entry}

\begin{entry}{砍价}{9,6}{⽯、⼈}
  \begin{phonetics}{砍价}{kan3jia4}
    \definition{v.}{barganhar | cortar ou derrubar um preço}
  \end{phonetics}
\end{entry}

\begin{entry}{砍伤}{9,6}{⽯、⼈}
  \begin{phonetics}{砍伤}{kan3shang1}
    \definition{v.}{ferir com lâmina ou machado}
  \end{phonetics}
\end{entry}

\begin{entry}{砍杀}{9,6}{⽯、⽊}
  \begin{phonetics}{砍杀}{kan3sha1}
    \definition{v.}{atacar com arma branca}
  \end{phonetics}
\end{entry}

\begin{entry}{砍死}{9,6}{⽯、⽍}
  \begin{phonetics}{砍死}{kan3si3}
    \definition{v.}{matar com um machado}
  \end{phonetics}
\end{entry}

\begin{entry}{砍树}{9,9}{⽯、⽊}
  \begin{phonetics}{砍树}{kan3shu4}
    \definition{v.}{derrubar árvores}
  \end{phonetics}
\end{entry}

\begin{entry}{砍掉}{9,11}{⽯、⼿}
  \begin{phonetics}{砍掉}{kan3diao4}
    \definition{v.}{amputar}
  \end{phonetics}
\end{entry}

\begin{entry}{砍断}{9,11}{⽯、⽄}
  \begin{phonetics}{砍断}{kan3duan4}
    \definition{v.}{cortar}
  \end{phonetics}
\end{entry}

\begin{entry}{研究}{9,7}{⽯、⽳}
  \begin{phonetics}{研究}{yan2jiu1}[][HSK 4]
    \definition{v.}{estudar; pesquisar | discutir; considerar}
  \end{phonetics}
\end{entry}

\begin{entry}{研究生}{9,7,5}{⽯、⽳、⽣}
  \begin{phonetics}{研究生}{yan2 jiu1 sheng1}[][HSK 4]
    \definition[位,名]{s.}{pós-graduado; estudante de pós-graduação}
  \end{phonetics}
\end{entry}

\begin{entry}{研究所}{9,7,8}{⽯、⽳、⼾}
  \begin{phonetics}{研究所}{yan2 jiu1 suo3}[][HSK 5]
    \definition[个]{s.}{instituto de pesquisa; instituição de pesquisa científica envolvida em pesquisas em um determinado campo}
  \end{phonetics}
\end{entry}

\begin{entry}{研制}{9,8}{⽯、⼑}
  \begin{phonetics}{研制}{yan2 zhi4}[][HSK 4]
    \definition{v.}{desenvolver; fabricar; produzir | triturar; (medicina chinesa) moer}
  \end{phonetics}
\end{entry}

\begin{entry}{砖}{9}{⽯}
  \begin{phonetics}{砖}{zhuan1}
    \definition[块]{s.}{tijolo}
  \end{phonetics}
\end{entry}

\begin{entry}{祖国}{9,8}{⽰、⼞}
  \begin{phonetics}{祖国}{zu3guo2}
    \definition{s.}{pátria | terra natal}
  \end{phonetics}
\end{entry}

\begin{entry}{祝}{9}{⽰}
  \begin{phonetics}{祝}{zhu4}[][HSK 3]
    \definition*{s.}{sobrenome Zhu}
    \definition{v.}{expressar bons votos; desejar; abençoar | rezar aos deuses ou espíritos para obter bênçãos}
  \end{phonetics}
\end{entry}

\begin{entry}{祝好}{9,6}{⽰、⼥}
  \begin{phonetics}{祝好}{zhu4hao3}
    \definition{expr.}{desejo-lhe tudo de melhor! (ao encerrar uma correspondência)}
  \end{phonetics}
\end{entry}

\begin{entry}{祝寿}{9,7}{⽰、⼨}
  \begin{phonetics}{祝寿}{zhu4shou4}
    \definition{v.}{dar parabéns pelo aniversário (a uma pessoa idosa)}
  \end{phonetics}
\end{entry}

\begin{entry}{祝贺}{9,9}{⽰、⾙}
  \begin{phonetics}{祝贺}{zhu4he4}[][HSK 5]
    \definition[个]{s.}{congratulações; felicitações}
    \definition{v.}{congratular; felicitar; parabenizar}
  \end{phonetics}
\end{entry}

\begin{entry}{祝酒}{9,10}{⽰、⾣}
  \begin{phonetics}{祝酒}{zhu4jiu3}
    \definition{v.}{parabenizar e fazer um brinde | brindar}
  \end{phonetics}
\end{entry}

\begin{entry}{祝颂}{9,10}{⽰、⾴}
  \begin{phonetics}{祝颂}{zhu4song4}
    \definition{v.}{expressar bons desejos}
  \end{phonetics}
\end{entry}

\begin{entry}{祝祷}{9,11}{⽰、⽰}
  \begin{phonetics}{祝祷}{zhu4dao3}
    \definition{v.}{rezar | orar}
  \end{phonetics}
\end{entry}

\begin{entry}{祝谢}{9,12}{⽰、⾔}
  \begin{phonetics}{祝谢}{zhu4xie4}
    \definition{v.}{agradecer | dar parabéns}
  \end{phonetics}
\end{entry}

\begin{entry}{祝福}{9,13}{⽰、⽰}
  \begin{phonetics}{祝福}{zhu4fu2}[][HSK 4]
    \definition[个]{s.}{bênção; benzedura; benzimento; originalmente, referia-se à oração para obter as bênçãos de Deus, mas, mais tarde, refere-se a desejar paz e felicidade às pessoas}
    \definition{v.}{desejar boa sorte a alguém}
  \end{phonetics}
\end{entry}

\begin{entry}{祝愿}{9,14}{⽰、⽕}
  \begin{phonetics}{祝愿}{zhu4yuan4}
    \definition{v.}{desejar}
  \end{phonetics}
\end{entry}

\begin{entry}{神}{9}{⽰}
  \begin{phonetics}{神}{shen2}[][HSK 5]
    \definition*{s.}{sobrenome Shen}
    \definition*{s.}{Deus}
    \definition{adj.}{inteligente; esperto | mágico; sobrenatural}
    \definition[个,位,尊]{s.}{divindade; deidade | espírito; mente; refere-se ao espírito, energia ou atenção de uma pessoa | olhar; expressão; expressões que refletem o estado interior}
  \end{phonetics}
\end{entry}

\begin{entry}{神奇}{9,8}{⽰、⼤}
  \begin{phonetics}{神奇}{shen2qi2}[][HSK 5]
    \definition{adj.}{mágico; peculiar; místico; milagroso; algo que parece muito novo; algo que ninguém imaginaria, mas que geralmente traz bons resultados}
    \definition{s.}{mágica; milagre}
  \end{phonetics}
\end{entry}

\begin{entry}{神明}{9,8}{⽰、⽇}
  \begin{phonetics}{神明}{shen2ming2}
    \definition{s.}{divindades | deuses}
  \end{phonetics}
\end{entry}

\begin{entry}{神经}{9,8}{⽰、⽷}
  \begin{phonetics}{神经}{shen2jing1}[][HSK 5]
    \definition{adj.}{excêntrico; estranho; peculiar; descreve anormalidade neurológica}
    \definition{s.}{nervo; um tipo de tecido presente no corpo humano ou animal que conecta o cérebro aos órgãos, transmitindo as sensações ao cérebro e as informações do cérebro aos órgãos}
  \end{phonetics}
\end{entry}

\begin{entry}{神经病学}{9,8,10,8}{⽰、⽷、⽧、⼦}
  \begin{phonetics}{神经病学}{shen2jing1bing4xue2}
    \definition{s.}{neurologia}
  \end{phonetics}
\end{entry}

\begin{entry}{神经病的}{9,8,10,8}{⽰、⽷、⽧、⽩}
  \begin{phonetics}{神经病的}{shen2jing1bing4de5}
    \definition{adj.}{neurótico}
  \end{phonetics}
\end{entry}

\begin{entry}{神话}{9,8}{⽰、⾔}
  \begin{phonetics}{神话}{shen2hua4}[][HSK 4]
    \definition[个]{s.}{mito; mitologia; conto de fadas; refere-se a deuses e deusas lendários e histórias de heróis antigos deificados | lorota; refere-se a alegações ridículas e infundadas}
  \end{phonetics}
\end{entry}

\begin{entry}{神秘}{9,10}{⽰、⽲}
  \begin{phonetics}{神秘}{shen2mi4}[][HSK 4]
    \definition{adj.}{místico; misterioso}
  \end{phonetics}
\end{entry}

\begin{entry}{神兽}{9,11}{⽰、⼋}
  \begin{phonetics}{神兽}{shen2shou4}
    \definition{s.}{animal mitológico | fera}
  \end{phonetics}
\end{entry}

\begin{entry}{神情}{9,11}{⽰、⼼}
  \begin{phonetics}{神情}{shen2 qing2}[][HSK 5]
    \definition{s.}{aparência; expressão; atividades internas reveladas no rosto das pessoas}
  \end{phonetics}
\end{entry}

\begin{entry}{神器}{9,16}{⽰、⼝}
  \begin{phonetics}{神器}{shen2qi4}
    \definition{s.}{objeto mágico | objeto simbólico do poder imperial | arma fina | ferramenta muito útil}
  \end{phonetics}
\end{entry}

\begin{entry}{秋}{9}{⽲}
  \begin{phonetics}{秋}{qiu1}
    \definition*{s.}{sobrenome Qiu}
    \definition{s.}{outono | colheita}
  \end{phonetics}
\end{entry}

\begin{entry}{秋天}{9,4}{⽲、⼤}
  \begin{phonetics}{秋天}{qiu1 tian1}[][HSK 2]
    \definition[个,段,季,番]{s.}{outono}
  \end{phonetics}
\end{entry}

\begin{entry}{秋季}{9,8}{⽲、⼦}
  \begin{phonetics}{秋季}{qiu1 ji4}[][HSK 4]
    \definition[个]{s.}{outono; terceiro trimestre do ano, segundo o costume chinês, refere-se ao período de três meses entre o outono e o inverno, também se refere aos sétimo, oitavo e nono meses do calendário lunar}
  \end{phonetics}
\end{entry}

\begin{entry}{种}{9}{⽲}
  \begin{phonetics}{种}{zhong3}[][HSK 3,4]
    \definition{clas.}{indica tipo, usado para pessoas e qualquer coisa}
    \definition{s.}{espécie | etnia | semente; estirpe; linhagem; material para reprodução biológica em cadeia | coragem; determinação; garra; força de caráter; refere-se à coragem ou determinação}
  \end{phonetics}
  \begin{phonetics}{种}{zhong4}
    \definition{v.}{semear; cultivar; plantar}
  \end{phonetics}
\end{entry}

\begin{entry}{种子}{9,3}{⽲、⼦}
  \begin{phonetics}{种子}{zhong3zi5}[][HSK 3]
    \definition[颗,粒,个,号]{s.}{semente; um órgão exclusivo de certas plantas, geralmente composto de três partes: tegumento, embrião e endosperma, as sementes podem germinar e se tornar novas plantas sob certas condições | jogador classificado; durante a competição, nas eliminatórias, os jogadores mais fortes de cada equipe são escalados}
  \end{phonetics}
\end{entry}

\begin{entry}{种地}{9,6}{⽲、⼟}
  \begin{phonetics}{种地}{zhong4di4}
    \definition{v.}{cultivar | trabalhar a terra}
  \end{phonetics}
\end{entry}

\begin{entry}{种种}{9,9}{⽲、⽲}
  \begin{phonetics}{种种}{zhong3zhong3}
    \definition{adj.}{todos os tipos de}
  \end{phonetics}
\end{entry}

\begin{entry}{种类}{9,9}{⽲、⽶}
  \begin{phonetics}{种类}{zhong3lei4}[][HSK 4]
    \definition{s.}{espécie; classe; tipo; variedade; categoria; classificação de alguma coisa de acordo com sua natureza e características}
  \end{phonetics}
\end{entry}

\begin{entry}{种族灭绝}{9,11,5,9}{⽲、⽅、⽕、⽷}
  \begin{phonetics}{种族灭绝}{zhong3zu2mie4jue2}
    \definition{s.}{genocídio | extinção étnica}
  \end{phonetics}
\end{entry}

\begin{entry}{种麻}{9,11}{⽲、⿇}
  \begin{phonetics}{种麻}{zhong3ma2}
    \definition{s.}{planta de cânhamo (feminina)}
  \end{phonetics}
\end{entry}

\begin{entry}{种植}{9,12}{⽲、⽊}
  \begin{phonetics}{种植}{zhong4zhi2}[][HSK 4]
    \definition{v.}{plantar; crescer; cultivar; enterrar as sementes de uma planta no solo; plantar as mudas de uma planta no solo}
  \end{phonetics}
\end{entry}

\begin{entry}{种薯}{9,16}{⽲、⾋}
  \begin{phonetics}{种薯}{zhong3shu3}
    \definition{s.}{tubérculo semente}
  \end{phonetics}
\end{entry}

\begin{entry}{科}{9}{⽲}
  \begin{phonetics}{科}{ke1}[][HSK 2]
    \definition*{s.}{sobrenome Ke}
    \definition{s.}{um ramo de estudo acadêmico ou profissional |uma divisão ou subdivisão de uma unidade administrativa | família | instruções de palco no drama chinês clássico; nos roteiros de peças clássicas, termos usados para indicar as ações dos personagens | nível; classificação; categoria | sessão de exames; refere-se às disciplinas, notas e anos das provas para a seleção de candidatos a cargos públicos militares e civis na antiguidade | tecnológico | assunto | lei; regulamento; decreto | penalidade; pena; punição | treinamento profissional ou formal; curso profissionalizante}
    \definition{v.}{proferir uma sentença (penal)}
  \end{phonetics}
\end{entry}

\begin{entry}{科技}{9,7}{⽲、⼿}
  \begin{phonetics}{科技}{ke1 ji4}[][HSK 3]
    \definition{s.}{ciência e tecnologia}
  \end{phonetics}
\end{entry}

\begin{entry}{科学}{9,8}{⽲、⼦}
  \begin{phonetics}{科学}{ke1xue2}[][HSK 2]
    \definition{adj.}{científico; em conformidade com as leis da ciência}
    \definition[门,个,种]{s.}{ciência; um conjunto de conhecimentos que reflete as leis objetivas da natureza, da sociedade, do pensamento, etc.}
  \end{phonetics}
\end{entry}

\begin{entry}{科学家}{9,8,10}{⽲、⼦、⼧}
  \begin{phonetics}{科学家}{ke1xue2jia1}
    \definition[位,名,个]{s.}{cientista; pessoas com realizações significativas no campo da pesquisa científica}
  \end{phonetics}
\end{entry}

\begin{entry}{秒}{9}{⽲}
  \begin{phonetics}{秒}{miao3}[][HSK 5]
    \definition{adv.}{(coloquial) instantaneamente}
    \definition{s.}{segundo (unidade de tempo) | segundo (unidade de medida angular)}
  \end{phonetics}
\end{entry}

\begin{entry}{穿}{9}{⽳}
  \begin{phonetics}{穿}{chuan1}[][HSK 1]
    \definition{adj.}{direto; através; usado após certos verbos, indica um estado de revelação completa}
    \definition{s.}{vestuário; roupas; refere-se a roupas, sapatos, meias, etc.}
    \definition{v.}{usar; vestir; estar vestido; ter \dots vestido;  vestir roupas, sapatos, meias, etc. | perfurar através de; penetrar; formar orifícios por meio de cinzéis, brocas ou pontas afiadas | enfiar; amarrar; usar cordas e fios para ligar coisas | passar por; atravessar; passar por; através de (buracos, fendas, espaços vazios, etc.)}
  \end{phonetics}
\end{entry}

\begin{entry}{穿上}{9,3}{⽳、⼀}
  \begin{phonetics}{穿上}{chuan1 shang4}[][HSK 4]
    \definition{v.}{vestir (roupas, etc.); colocar roupas}
  \end{phonetics}
\end{entry}

\begin{entry}{突出}{9,5}{⽳、⼐}
  \begin{phonetics}{突出}{tu1chu1}[][HSK 3]
    \definition{adj.}{proeminente; excelente; mais que a média}
    \definition{v.}{romper | enfatizar; destacar; dar destaque a | sobressair; projetar-se; destacar-se}
  \end{phonetics}
\end{entry}

\begin{entry}{突破}{9,10}{⽳、⽯}
  \begin{phonetics}{突破}{tu1po4}[][HSK 5]
    \definition{v.}{romper; fazer uma descoberta revolucionária; concentrar esforços em um único ponto de ataque, reunir o sucesso | quebrar (limite); superar (dificuldade); superar dificuldades; ultrapassar os números ou limites anteriores, superar recordes anteriores, etc.; romper com as limitações e restrições anteriores}
  \end{phonetics}
\end{entry}

\begin{entry}{突然}{9,12}{⽳、⽕}
  \begin{phonetics}{突然}{tu1ran2}[][HSK 3]
    \definition{adj.}{repentino; abrupto; inesperado}
    \definition{adv.}{de repente; abruptamente; inesperadamente; subitamente}
  \end{phonetics}
\end{entry}

\begin{entry}{竖}{9}{⽴}
  \begin{phonetics}{竖}{shu4}
    \definition*{s.}{sobrenome Shu}
    \definition{adj.}{vertical; ereto; perpendicular ao solo}
    \definition{s.}{traço vertical (em caracteres chineses) | empregados domésticos; jovens criados}
    \definition{v.}{colocar em pé; erguer; ficar de pé; colocar o objeto perpendicular ao solo}
  \end{phonetics}
\end{entry}

\begin{entry}{类}{9}{⽶}
  \begin{phonetics}{类}{lei4}[][HSK 3]
    \definition*{s.}{sobrenome Lei}
    \definition{clas.}{tipo; espécie; categoria usada para pessoas ou coisas}
    \definition{s.}{classe; categoria; tipo; variedade; a combinação de muitas coisas semelhantes ou iguais}
    \definition{v.}{assemelhar-se a; ser semelhante a}
  \end{phonetics}
\end{entry}

\begin{entry}{类似}{9,6}{⽶、⼈}
  \begin{phonetics}{类似}{lei4si4}[][HSK 3]
    \definition{adj.}{semelhante; análogo}
  \end{phonetics}
\end{entry}

\begin{entry}{类型}{9,9}{⽶、⼟}
  \begin{phonetics}{类型}{lei4xing2}[][HSK 4]
    \definition[种,个]{s.}{tipo; espécie; categoria; tipos formados por coisas com características comuns}
  \end{phonetics}
\end{entry}

\begin{entry}{结}{9}{⽷}
  \begin{phonetics}{结}{jie1}
    \definition{v.}{dar (frutos); formar (sementes); produzir frutos ou sementes (uma planta)}
  \end{phonetics}
  \begin{phonetics}{结}{jie2}[][HSK 4]
    \definition*{s.}{sobrenome Jie}
    \definition{s.}{nó | declaração juramentada; garantia por escrito; documento que, antigamente, significava um reconhecimento de encerramento ou uma garantia de responsabilidade}
    \definition{v.}{amarrar; tricotar; dar nó; tecer | formar; forjar; cimentar; solidificar | resolver; concluir | combinar; formar um relacionamento}
  \end{phonetics}
\end{entry}

\begin{entry}{结合}{9,6}{⽷、⼝}
  \begin{phonetics}{结合}{jie2he2}[][HSK 3]
    \definition{v.}{ligar; unir; combinar; integrar; formar uma relação estreita entre pessoas ou coisas | casar-se; unir-se em matrimônio; referir-se especificamente a casais}
  \end{phonetics}
\end{entry}

\begin{entry}{结论}{9,6}{⽷、⾔}
  \begin{phonetics}{结论}{jie2lun4}[][HSK 4]
    \definition[个]{s.}{conclusão; palavra final sobre uma pessoa ou coisa após investigação e pesquisa | veredito; julgamento deduzido de premissas também é chamado de conclusão}
  \end{phonetics}
\end{entry}

\begin{entry}{结局}{9,7}{⽷、⼫}
  \begin{phonetics}{结局}{jie2ju2}
    \definition{s.}{conclusão | fim | final}
  \end{phonetics}
\end{entry}

\begin{entry}{结束}{9,7}{⽷、⽊}
  \begin{phonetics}{结束}{jie2shu4}[][HSK 3]
    \definition{v.}{finalizar; fechar; terminar; concluir; encerrar; desenvolver ou avançar até a fase final, sem continuidade}
  \end{phonetics}
\end{entry}

\begin{entry}{结束工作}{9,7,3,7}{⽷、⽊、⼯、⼈}
  \begin{phonetics}{结束工作}{jie2shu4gong1zuo4}
    \definition{s.}{trabalho final}
    \definition{v.}{terminar o trabalho}
  \end{phonetics}
\end{entry}

\begin{entry}{结束区}{9,7,4}{⽷、⽊、⼖}
  \begin{phonetics}{结束区}{jie2shu4 qu1}
    \definition{s.}{zona final}
  \end{phonetics}
\end{entry}

\begin{entry}{结束文本}{9,7,4,5}{⽷、⽊、⽂、⽊}
  \begin{phonetics}{结束文本}{jie2shu4 wen2ben3}
    \definition{s.}{texto final}
  \end{phonetics}
\end{entry}

\begin{entry}{结束剂}{9,7,8}{⽷、⽊、⼑}
  \begin{phonetics}{结束剂}{jie2shu4 ji4}
    \definition{s.}{finalizador}
  \end{phonetics}
\end{entry}

\begin{entry}{结束语}{9,7,9}{⽷、⽊、⾔}
  \begin{phonetics}{结束语}{jie2shu4yu3}
    \definition{s.}{conclusões finais | considerações finais}
  \end{phonetics}
\end{entry}

\begin{entry}{结束辩论}{9,7,16,6}{⽷、⽊、⾟、⾔}
  \begin{phonetics}{结束辩论}{jie2shu4 bian4 lun4}
    \definition{s.}{debate de encerramento}
  \end{phonetics}
\end{entry}

\begin{entry}{结社自由}{9,7,6,5}{⽷、⽰、⾃、⽥}
  \begin{phonetics}{结社自由}{jie2she4zi4you2}
    \definition{s.}{(constitucional) liberdade de associação}
  \end{phonetics}
\end{entry}

\begin{entry}{结实}{9,8}{⽷、⼧}
  \begin{phonetics}{结实}{jie1shi5}[][HSK 3]
    \definition{adj.}{sólido; resistente; durável | forte; resistente; robusto}
  \end{phonetics}
\end{entry}

\begin{entry}{结构}{9,8}{⽷、⽊}
  \begin{phonetics}{结构}{jie2gou4}[][HSK 4]
    \definition[个,座]{s.}{estrutura; composição; construção; formação; constituição; tecido; forma; sistematização; mecânica; organização | arquitetura; estrutura; construção; construção de partes de edifícios com suporte de carga ou com carga externa | textura (geológico)}
  \end{phonetics}
\end{entry}

\begin{entry}{结果}{9,8}{⽷、⽊}
  \begin{phonetics}{结果}{jie1guo3}
    \definition{v.}{dar frutos}
  \end{phonetics}
  \begin{phonetics}{结果}{jie2guo3}[][HSK 2]
    \definition{conj.}{como resultado | no final}
    \definition{s.}{resultado | conclusão | consequência}
    \definition{v.}{despachar | matar}
  \end{phonetics}
\end{entry}

\begin{entry}{结婚}{9,11}{⽷、⼥}
  \begin{phonetics}{结婚}{jie2hun1}[][HSK 3]
    \definition{v.+compl.}{casar; casar-se; casar-se bem;}
  \end{phonetics}
\end{entry}

\begin{entry}{结婚礼服}{9,11,5,8}{⽷、⼥、⽰、⽉}
  \begin{phonetics}{结婚礼服}{jie2hun1 li3 fu2}
    \definition{s.}{vestido de casamento}
  \end{phonetics}
\end{entry}

\begin{entry}{绕}{9}{⽷}
  \begin{phonetics}{绕}{rao4}[][HSK 5]
    \definition*{s.}{sobrenome Rao}
    \definition{v.}{enrolar; bobinar; rebobinar | mover-se em círculo; girar; revolver | fazer um desvio; contornar; dar a volta | confundir; desorientar}
  \end{phonetics}
\end{entry}

\begin{entry}{给}{9}{⽷}
  \begin{phonetics}{给}{gei3}[][HSK 1]
    \definition{prep.}{por; expressa significado passivo; tem o mesmo significado que 被, 叫; pode ser seguido pelo agente da ação; o agente da ação também pode não aparecer na frase | para; a; seguido por quem se beneficia da ação; igual a 为 | em direção a; seguido pelo destinatário da ação; o mesmo que 向 | indica transmissão}
    \definition{v.}{dar; conceder; fazer com que a outra parte obtenha algo | passar; pagar; indicar que a outra pessoa faça algo | deixar; permitir que alguém faça algo; autorizar alguém a fazer algo}
    \definition{v.aux.}{usado antes de verbos predicativos que expressam passividade, disposição, etc., para reforçar o tom}
  \seealsoref{被}{bei4}
  \seealsoref{叫}{jiao4}
  \seealsoref{为}{wei4}
  \seealsoref{向}{xiang4}
  \end{phonetics}
  \begin{phonetics}{给}{ji3}
    \definition{adj.}{abundante; próspero; bem provido para}
    \definition{v.}{fornecer; prover}
  \end{phonetics}
\end{entry}

\begin{entry}{给……打电话}{9,5,5,8}{⽷、⼿、⽥、⾔}
  \begin{phonetics}{给……打电话}{gei3 da3 dian4 hua4}
    \definition{expr.}{telefonar para alguém}
  \seealsoref{打电话}{da3 dian4 hua4}
  \end{phonetics}
\end{entry}

\begin{entry}{绝不}{9,4}{⽷、⼀}
  \begin{phonetics}{绝不}{jue2bu4}
    \definition{adv.}{definitivamente não | de forma alguma | sob nenhuma circunstância}
  \end{phonetics}
\end{entry}

\begin{entry}{绝对}{9,5}{⽷、⼨}
  \begin{phonetics}{绝对}{jue2dui4}[][HSK 3]
    \definition{adj.}{absoluto; sem condições; sem restrições | absoluto; extremo; incompleto; sem margem para negociação ou alteração}
    \definition{adv.}{absolutamente; completamente; com certeza}
  \end{phonetics}
\end{entry}

\begin{entry}{绝招}{9,8}{⽷、⼿}
  \begin{phonetics}{绝招}{jue2zhao1}
    \definition{s.}{habilidade única | movimento delicado inesperado (como último recurso) | golpe de mestre | golpe final}
  \end{phonetics}
\end{entry}

\begin{entry}{绝版}{9,8}{⽷、⽚}
  \begin{phonetics}{绝版}{jue2ban3}
    \definition{adj.}{esgotado | fora de catálogo}
  \end{phonetics}
\end{entry}

\begin{entry}{绝望}{9,11}{⽷、⽉}
  \begin{phonetics}{绝望}{jue2 wang4}[][HSK 5]
    \definition{v.+compl.}{desesperar; desistir de toda esperança; perder toda esperança de}
  \end{phonetics}
\end{entry}

\begin{entry}{统一}{9,1}{⽷、⼀}
  \begin{phonetics}{统一}{tong3yi1}[][HSK 4]
    \definition{adj.}{unificado; unitário; centralizado; consistente}
    \definition{v.}{unificar; unir; integrar; padronizar}
  \end{phonetics}
\end{entry}

\begin{entry}{统计}{9,4}{⽷、⾔}
  \begin{phonetics}{统计}{tong3ji4}[][HSK 4]
    \definition{v.}{compilar estatísticas; refere-se à realização de trabalho estatístico, ou seja, coletar, reunir, analisar e extrapolar dados sobre um fenômeno | somar; adicionar; contar}
  \end{phonetics}
\end{entry}

\begin{entry}{罚}{9}{⽹}
  \begin{phonetics}{罚}{fa2}[][HSK 5]
    \definition{s.}{punição; penalidade}
    \definition{v.}{punir; penalizar; multar; confiscar}
  \end{phonetics}
\end{entry}

\begin{entry}{罚款}{9,12}{⽹、⽋}
  \begin{phonetics}{罚款}{fa2kuan3}[][HSK 5]
    \definition[笔,次,宗]{s.}{multa; penalidade; refere-se ao dinheiro pago por uma pessoa ou entidade de acordo com as disposições de um delito ou violação de contrato ou contrato}
    \definition{v.+compl.}{multar; penalizar; exigir, de acordo com os regulamentos, uma determinada quantia de dinheiro de uma pessoa ou entidade que tenha violado a lei ou descumprido um regulamento ou contrato}
  \end{phonetics}
\end{entry}

\begin{entry}{美}{9}{⽺}
  \begin{phonetics}{美}{mei3}[][HSK 3]
    \definition*{s.}{beleza (oposto de 丑)}
    \definition*{s.}{Abreviatura de América (美洲) | Abreviatura de Estados Unidos da América (美国) | As Américas (美洲)}
    \definition{adj.}{belo; bonito (oposto de 丑) | muito satisfatório; bom; agradável}
    \definition{v.}{embelezar; tornar mais bonito | estar satisfeito consigo mesmo; orgulhar-se; sentir-se presunçoso}
  \seealsoref{丑}{chou3}
  \seealsoref{美国}{mei3guo2}
  \seealsoref{美洲}{mei3zhou1}
  \end{phonetics}
\end{entry}

\begin{entry}{美女}{9,3}{⽺、⼥}
  \begin{phonetics}{美女}{mei3 nv3}[][HSK 4]
    \definition[个,位]{s.}{beldade; mulher bonita; uma jovem linda}
  \end{phonetics}
\end{entry}

\begin{entry}{美元}{9,4}{⽺、⼉}
  \begin{phonetics}{美元}{mei3yuan2}[][HSK 3]
    \definition*[元,笔,沓]{s.}{Dólar Americano; a moeda dos Estados Unidos}
  \end{phonetics}
\end{entry}

\begin{entry}{美术}{9,5}{⽺、⽊}
  \begin{phonetics}{美术}{mei3shu4}[][HSK 3]
    \definition[种]{s.}{arte; artes plásticas: arte que ocupa um determinado espaço, compõe imagens estéticas e permite que as pessoas apreciem visualmente, incluindo pintura, escultura, arquitetura, etc. | pintura; pintura tradicional chinesa}
  \end{phonetics}
\end{entry}

\begin{entry}{美甲}{9,5}{⽺、⽥}
  \begin{phonetics}{美甲}{mei3jia3}
    \definition{s.}{manicure e/ou pedicure}
  \end{phonetics}
\end{entry}

\begin{entry}{美好}{9,6}{⽺、⼥}
  \begin{phonetics}{美好}{mei3 hao3}[][HSK 3]
    \definition{adj.}{bem; feliz; glorioso; descreve a vida, os desejos, etc. como sendo muito bons e satisfatórios}
  \end{phonetics}
\end{entry}

\begin{entry}{美丽}{9,7}{⽺、⼀}
  \begin{phonetics}{美丽}{mei3li4}[][HSK 3]
    \definition{adj.}{bonito; lindo; capaz de proporcionar uma sensação de beleza}
  \end{phonetics}
\end{entry}

\begin{entry}{美味}{9,8}{⽺、⼝}
  \begin{phonetics}{美味}{mei3wei4}
    \definition{adj.}{delicioso}
    \definition{s.}{comida deliciosa | delicadeza (\emph{delicacy})}
  \end{phonetics}
\end{entry}

\begin{entry}{美国}{9,8}{⽺、⼞}
  \begin{phonetics}{美国}{mei3guo2}
    \definition*{s.}{Estados Unidos da América}
  \end{phonetics}
\end{entry}

\begin{entry}{美国人}{9,8,2}{⽺、⼞、⼈}
  \begin{phonetics}{美国人}{mei3guo2ren2}
    \definition{s.}{americano | pessoa ou povo dos Estados Unidos da América}
  \end{phonetics}
\end{entry}

\begin{entry}{美学}{9,8}{⽺、⼦}
  \begin{phonetics}{美学}{mei3xue2}
    \definition{s.}{estética}
  \end{phonetics}
\end{entry}

\begin{entry}{美金}{9,8}{⽺、⾦}
  \begin{phonetics}{美金}{mei3 jin1}[][HSK 4]
    \definition{s.}{USD; dólar americano: a moeda local dos Estados Unidos}
  \end{phonetics}
\end{entry}

\begin{entry}{美洲}{9,9}{⽺、⽔}
  \begin{phonetics}{美洲}{mei3zhou1}
    \definition*{s.}{América (incluindo Norte, Central e Sul)}
  \end{phonetics}
\end{entry}

\begin{entry}{美洲人}{9,9,2}{⽺、⽔、⼈}
  \begin{phonetics}{美洲人}{mei3zhou1ren2}
    \definition{s.}{americano | pessoa ou povo do continente Americano}
  \end{phonetics}
\end{entry}

\begin{entry}{美食}{9,9}{⽺、⾷}
  \begin{phonetics}{美食}{mei3 shi2}[][HSK 3]
    \definition[种,道,桌]{s.}{iguaria; (gastronomia) comida saborosa}
  \end{phonetics}
\end{entry}

\begin{entry}{耍}{9}{⽽}
  \begin{phonetics}{耍}{shua3}
    \definition{v.}{brincar com | empunhar | agir (legal, calmo, tranquilo, descolado, etc.) | exibir (uma habilidade, o temperamento de alguém, etc.)}
  \end{phonetics}
\end{entry}

\begin{entry}{耍赖}{9,13}{⽽、⾙}
  \begin{phonetics}{耍赖}{shua3lai4}
    \definition{v.}{agir descaradamente | recusar -se a reconhecer que alguém perdeu o jogo ou fez uma promessa, etc. | agir como um idiota | agir como se algo nunca tivesse acontecido}
  \end{phonetics}
\end{entry}

\begin{entry}{耐心}{9,4}{⽽、⼼}
  \begin{phonetics}{耐心}{nai4xin1}[][HSK 5]
    \definition{adj.}{paciente}
    \definition{s.}{paciência; não se incomoda com as dificuldades e tem um caráter tolerante}
    \definition{v.}{ser paciente}
  \end{phonetics}
\end{entry}

\begin{entry}{胃}{9}{⾁}
  \begin{phonetics}{胃}{wei4}[][HSK 5]
    \definition*{s.}{Wei, uma das mansões lunares; uma das vinte e oito constelações}
    \definition{s.}{estômago; parte do aparelho digestivo}
  \end{phonetics}
\end{entry}

\begin{entry}{胃口}{9,3}{⾁、⼝}
  \begin{phonetics}{胃口}{wei4kou3}
    \definition{s.}{apetite}
  \end{phonetics}
\end{entry}

\begin{entry}{胆}{9}{⾁}
  \begin{phonetics}{胆}{dan3}[][HSK 5]
    \definition[个,颗]{s.}{vesícula biliar | coragem; bravura | um recipiente interno semelhante a uma bexiga; algo que se encaixa dentro de um objeto e pode conter água, ar, etc.}
  \end{phonetics}
\end{entry}

\begin{entry}{胆小}{9,3}{⾁、⼩}
  \begin{phonetics}{胆小}{dan3 xiao3}[][HSK 5]
    \definition{adj.}{tímido; covarde}
  \end{phonetics}
\end{entry}

\begin{entry}{胆小鬼}{9,3,9}{⾁、⼩、⿁}
  \begin{phonetics}{胆小鬼}{dan3xiao3gui3}
    \definition{adj.}{covarde | medroso}
  \end{phonetics}
\end{entry}

\begin{entry}{背}{9}{⾁}
  \begin{phonetics}{背}{bei1}[][HSK 2]
    \definition{clas.}{carga; pacote; para transportar coisas nas costas}
    \definition{v.}{carregar nas costas | suportar; carregar}
  \end{phonetics}
  \begin{phonetics}{背}{bei4}[][HSK 3]
    \definition{adj.}{azarado | fora do caminho; um lugar muito distante do centro movimentado, onde poucas pessoas aparecem | deficiente auditivo}
    \definition{s.}{parte posterior do corpo; costas; coluna vertebral; parte do tronco entre os ombros e a região lombar | parte de trás de um objeto}
    \definition{v.}{afastar-se; virar as costas | decorar; memorizar; recitar de memória | esconder algo de; fazer algo em segredo | sair, ir embora; partir; abandonar | quebrar; violar; agir de forma contrária a}
  \end{phonetics}
\end{entry}

\begin{entry}{背包}{9,5}{⾁、⼓}
  \begin{phonetics}{背包}{bei1 bao1}[][HSK 5]
    \definition[个]{s.}{mochila; mochila de ataque; mochila de infantaria; pacotes de roupas carregados nas costas quando marcham}
  \end{phonetics}
\end{entry}

\begin{entry}{背后}{9,6}{⾁、⼝}
  \begin{phonetics}{背后}{bei4 hou4}[][HSK 3]
    \definition{s.}{parte posterior; parte de trás; traseira | pelas costas de alguém}
  \end{phonetics}
\end{entry}

\begin{entry}{背景}{9,12}{⾁、⽇}
  \begin{phonetics}{背景}{bei4jing3}[][HSK 4]
    \definition[种]{s.}{pano de fundo; fundo; cenário de teatro, filme ou drama de TV | fundo; cenário que permeia a imagem principal na tela | condições sociais; ambientes históricos (significativamente influentes para algo ou alguém) | poder que dá suporte a alguém}
  \end{phonetics}
\end{entry}

\begin{entry}{胖}{9}{⾁}
  \begin{phonetics}{胖}{pan2}
    \definition{adj.}{saudável}
  \end{phonetics}
  \begin{phonetics}{胖}{pang4}[][HSK 3]
    \definition{adj.}{gordo; robusto; rechonchudo; (corpo humano) com muita gordura ou carne (em oposição a 瘦)}
  \seealsoref{瘦}{shou4}
  \end{phonetics}
\end{entry}

\begin{entry}{胖子}{9,3}{⾁、⼦}
  \begin{phonetics}{胖子}{pang4 zi5}[][HSK 4]
    \definition{s.}{obeso; gordo; pessoa gorda}
  \end{phonetics}
\end{entry}

\begin{entry}{胚}{9}{⾁}
  \begin{phonetics}{胚}{pei1}
    \definition{s.}{embrião}
  \end{phonetics}
\end{entry}

\begin{entry}{胜}{9}{⾁}
  \begin{phonetics}{胜}{sheng4}[][HSK 3]
    \definition{adj.}{soberbo; maravilhoso; adorável}
    \definition[场]{s.}{vitória; sucesso | penteado de mulher; joias usadas pelas mulheres na antiguidade}
    \definition{v.}{vencer (oposto de 负, 败) | derrotar | (frequentemente seguido por 于, etc.) superar; ser superior a; levar a melhor sobre | vencer; ter sucesso; derrotar o adversário | ultrapassar; ser superior ao outro | suportar; ser capaz de suportar ou aguentar}
  \seealsoref{败}{bai4}
  \seealsoref{负}{fu4}
  \seealsoref{于}{yu2}
  \end{phonetics}
\end{entry}

\begin{entry}{胜负}{9,6}{⾁、⾙}
  \begin{phonetics}{胜负}{sheng4fu4}[][HSK 5]
    \definition{s.}{vitória ou derrota; sucesso ou fracasso}
  \end{phonetics}
\end{entry}

\begin{entry}{胜利}{9,7}{⾁、⼑}
  \begin{phonetics}{胜利}{sheng4li4}[][HSK 3]
    \definition{adv.}{com sucesso; triunfantemente; atingir o objetivo previsto}
    \definition{v.}{ganhar; vencer; triunfar; ter sucesso}
  \end{phonetics}
\end{entry}

\begin{entry}{胜算}{9,14}{⾁、⽵}
  \begin{phonetics}{胜算}{sheng4suan4}
    \definition{s.}{probabilidade de sucesso | estratégia que garante o sucesso}
    \definition{v.}{ter certeza do sucesso}
  \end{phonetics}
\end{entry}

\begin{entry}{胡子}{9,3}{⾁、⼦}
  \begin{phonetics}{胡子}{hu2 zi5}[][HSK 5]
    \definition[团,根,个]{s.}{barba; bigode | bandido; salteador}
  \end{phonetics}
\end{entry}

\begin{entry}{胡同儿}{9,6,2}{⾁、⼝、⼉}
  \begin{phonetics}{胡同儿}{hu2 tong4r5}[][HSK 5]
    \definition{s.}{beco; via; rua}
  \end{phonetics}
\end{entry}

\begin{entry}{胡萝卜}{9,11,2}{⾁、⾋、⼘}
  \begin{phonetics}{胡萝卜}{hu2luo2bo5}
    \definition{s.}{cenoura}
  \end{phonetics}
\end{entry}

\begin{entry}{舁}{9}{⾅}
  \begin{phonetics}{舁}{yu2}
    \definition{v.}{levantar; elevar | aumentar}
  \end{phonetics}
\end{entry}

\begin{entry}{范}{9}{⾋}
  \begin{phonetics}{范}{fan4}
    \definition*{s.}{sobrenome Fan}
    \definition{s.}{padrão; molde; matriz | modelo; exemplo; modelo a seguir | limites; escopo | restrição; limite}
  \end{phonetics}
\end{entry}

\begin{entry}{范围}{9,7}{⾋、⼞}
  \begin{phonetics}{范围}{fan4wei2}[][HSK 3]
    \definition[个]{s.}{escopo; limite; alcance}
    \definition{v.}{estabelecer limites para; limitar o escopo de}
  \end{phonetics}
\end{entry}

\begin{entry}{茶}{9}{⾋}
  \begin{phonetics}{茶}{cha2}[][HSK 1]
    \definition{adj.}{moreno; fulvo; amarelo-acastanhado}
    \definition[杯,壶]{s.}{chá (a bebida); bebida feita com folhas de chá | chá (a planta) | certos tipos de bebidas ou alimentos líquidos | árvore de chá-de-óleo | camélia}
  \end{phonetics}
\end{entry}

\begin{entry}{茶叶}{9,5}{⾋、⼝}
  \begin{phonetics}{茶叶}{cha2 ye4}[][HSK 4]
    \definition[盒,罐,包,片]{s.}{chá; folhas de chá; as folhas jovens da planta do chá que são processadas para produzir bebidas}
  \end{phonetics}
\end{entry}

\begin{entry}{草}{9}{⾋}
  \begin{phonetics}{草}{cao3}[][HSK 2]
    \definition*{s.}{sobrenome Cao}
    \definition{adj.}{descuidado; rude | rascunho; inicial | femea; na linguagem coloquial, refere-se a animais domésticos e aves fêmeas | precipitado; pouco cuidadoso | rascunho; não definitivo; preliminar; informal}
    \definition[种,棵,撮,株,根]{s.}{grama; gramado | palha | campo; zona rural; área selvagem | letra cursiva | letra cursiva (ou caligráfica) de um alfabeto fonético | rascunho | caligrafia cursiva; um tipo de escrita chinesa}
    \definition{v.}{esboçar; redigir}
  \end{phonetics}
\end{entry}

\begin{entry}{草地}{9,6}{⾋、⼟}
  \begin{phonetics}{草地}{cao3 di4}[][HSK 2]
    \definition[片,块]{s.}{prado; gramado; campo; pastagem ou grande área de terra plantada com pastagem | gramado; relvado; local com grama alta ou gramado}
  \end{phonetics}
\end{entry}

\begin{entry}{草纸}{9,7}{⾋、⽷}
  \begin{phonetics}{草纸}{cao3zhi3}
    \definition{s.}{papel pardo | pergaminho | papel de palha áspero | papel higiênico}
  \end{phonetics}
\end{entry}

\begin{entry}{草原}{9,10}{⾋、⼚}
  \begin{phonetics}{草原}{cao3 yuan2}[][HSK 5]
    \definition[片,个]{s.}{estepe; pradaria; grandes áreas de terra coberta de vegetação em áreas semiáridas, intercaladas com árvores tolerantes à seca}
  \end{phonetics}
\end{entry}

\begin{entry}{草莓}{9,10}{⾋、⾋}
  \begin{phonetics}{草莓}{cao3mei2}
    \definition[颗]{s.}{morango}
  \end{phonetics}
\end{entry}

\begin{entry}{荒芜}{9,7}{⾋、⾋}
  \begin{phonetics}{荒芜}{huang1wu2}
    \definition{adj.}{estéril}
  \end{phonetics}
\end{entry}

\begin{entry}{荔枝}{9,8}{⾋、⽊}
  \begin{phonetics}{荔枝}{li4zhi1}
    \definition{s.}{lichia}
  \end{phonetics}
\end{entry}

\begin{entry}{药}{9}{⾋}
  \begin{phonetics}{药}{yao4}[][HSK 2]
    \definition*{s.}{sobrenome Yao}
    \definition[片,粒,颗,瓶,服]{s.}{droga; loção; remédio; medicamento; substâncias que podem prevenir e tratar doenças, pragas ou melhorar funções corporais | certos produtos químicos com efeitos específicos}
    \definition{v.}{curar com remédios; tomar remédios para tratar doenças | matar com veneno; envenenar}
  \end{phonetics}
\end{entry}

\begin{entry}{药丸}{9,3}{⾋、⼂}
  \begin{phonetics}{药丸}{yao4wan2}
    \definition[粒]{s.}{pílula}
  \end{phonetics}
\end{entry}

\begin{entry}{药水}{9,4}{⾋、⽔}
  \begin{phonetics}{药水}{yao4 shui3}[][HSK 2]
    \definition*{s.}{Yaksu na Coreia do Norte, perto da fronteira com Liaoning e a província de Jilin}
    \definition{s.}{medicamento líquido; líquido medicinal | loção | remédio engarrafado | medicamento em forma líquida}
  \end{phonetics}
\end{entry}

\begin{entry}{药片}{9,4}{⾋、⽚}
  \begin{phonetics}{药片}{yao4 pian4}[][HSK 2]
    \definition[颗,片]{s.}{pílula; comprimido; preparações em comprimidos}
  \end{phonetics}
\end{entry}

\begin{entry}{药补}{9,7}{⾋、⾐}
  \begin{phonetics}{药补}{yao4bu3}
    \definition{s.}{suplemento dietético medicinal que ajuda a melhorar a saúde}
  \end{phonetics}
\end{entry}

\begin{entry}{药典}{9,8}{⾋、⼋}
  \begin{phonetics}{药典}{yao4dian3}
    \definition{s.}{farmacopéia}
  \end{phonetics}
\end{entry}

\begin{entry}{药店}{9,8}{⾋、⼴}
  \begin{phonetics}{药店}{yao4 dian4}[][HSK 2]
    \definition[家]{s.}{farmácia; drogaria; lojas que vendem medicamentos}
  \end{phonetics}
\end{entry}

\begin{entry}{药房}{9,8}{⾋、⼾}
  \begin{phonetics}{药房}{yao4fang2}
    \definition{s.}{farmácia | drogaria}
  \end{phonetics}
\end{entry}

\begin{entry}{药物}{9,8}{⾋、⽜}
  \begin{phonetics}{药物}{yao4 wu4}[][HSK 4]
    \definition{s.}{droga; medicamento; remédio; substâncias que controlam doenças, pragas, etc.}
  \end{phonetics}
\end{entry}

\begin{entry}{药品}{9,9}{⾋、⼝}
  \begin{phonetics}{药品}{yao4pin3}
    \definition{s.}{medicamento | remédio | droga}
  \end{phonetics}
\end{entry}

\begin{entry}{药签}{9,13}{⾋、⽵}
  \begin{phonetics}{药签}{yao4qian1}
    \definition{s.}{cotonete médico}
  \end{phonetics}
\end{entry}

\begin{entry}{药膳}{9,16}{⾋、⾁}
  \begin{phonetics}{药膳}{yao4shan4}
    \definition{s.}{dieta medicinal}
  \end{phonetics}
\end{entry}

\begin{entry}{药罐}{9,23}{⾋、⽸}
  \begin{phonetics}{药罐}{yao4guan4}
    \definition{s.}{frasco de remédio}
  \end{phonetics}
\end{entry}

\begin{entry}{虽}{9}{⾍}
  \begin{phonetics}{虽}{sui1}
    \definition{conj.}{no entanto | embora | mesmo se/embora}
  \end{phonetics}
\end{entry}

\begin{entry}{虽然}{9,12}{⾍、⽕}
  \begin{phonetics}{虽然}{sui1 ran2}[][HSK 2]
    \definition{conj.}{apesar de; embora (frequentemente usado correlativamente com 可是, 但是, etc); geralmente é usado no início de uma frase para indicar que o fato anterior foi reconhecido, mas não mudará o que acontecerá em seguida}
  \seealsoref{但是}{dan4 shi4}
  \seealsoref{可是}{ke3shi4}
  \end{phonetics}
\end{entry}

\begin{entry}{虾}{9}{⾍}
  \begin{phonetics}{虾}{xia1}
    \definition{s.}{camarão}
  \end{phonetics}
\end{entry}

\begin{entry}{蚂蚁}{9,9}{⾍、⾍}
  \begin{phonetics}{蚂蚁}{ma3yi3}
    \definition{s.}{formiga}
  \end{phonetics}
\end{entry}

\begin{entry}{要}{9}{⾑}
  \begin{phonetics}{要}{yao1}[][HSK 1]
    \definition*{s.}{sobrenome Yao}
    \definition{v.}{exigir; pedir; requerer; solicitar; buscar; insistir com base em algo em que se apoia | forçar; coagir; ameaçar}
  \end{phonetics}
  \begin{phonetics}{要}{yao4}[][HSK 4]
    \definition{adj.}{importante; essencial}
    \definition{conj.}{suponha; no caso; se, indicando um relacionamento hipotético | ou; ou\dots ou\dots}
    \definition{s.}{ponto principal; manchete; conteúdo importante}
    \definition{v.}{querer; desejar; pensar | querer; pedir; deseja; querer obter; querer manter | recuperar algo; dizer a alguém para guardar algo para você ou devolver | pedir (ou querer) que alguém faça algo; pedir a alguém para fazer algo, quando usado para conseguir que alguém faça algo, tem um tom de comando e pode ser indelicado | precisar; tomar; pegar | deve; deveria; é necessário (imperativo, essencial) que\dots | estar indo para | querer; ter um desejo por; expressar determinação ou desejo de fazer algo | poder; dever;  indica uma estimativa, usada para comparação}
  \seealsoref{要是}{yao4shi5}
  \end{phonetics}
\end{entry}

\begin{entry}{要么……要么……}{9,3,9,3}{⾑、⼃、⾑、⼃}
  \begin{phonetics}{要么……要么……}{yao4me5 yao4me5}
    \definition{conj.}{ou\dots ou\dots}
  \end{phonetics}
\end{entry}

\begin{entry}{要义}{9,3}{⾑、⼂}
  \begin{phonetics}{要义}{yao4yi4}
    \definition{s.}{resumo | o essencial}
  \end{phonetics}
\end{entry}

\begin{entry}{要不}{9,4}{⾑、⼀}
  \begin{phonetics}{要不}{yao4bu4}
    \definition{conj.}{de outra forma | se não | outro | ou}
  \end{phonetics}
\end{entry}

\begin{entry}{要不然}{9,4,12}{⾑、⼀、⽕}
  \begin{phonetics}{要不然}{yao4bu4ran2}
    \definition{conj.}{de outra forma | se não | outro | ou}
  \end{phonetics}
\end{entry}

\begin{entry}{要好}{9,6}{⾑、⼥}
  \begin{phonetics}{要好}{yao4hao3}
    \definition{v.}{ser amigos íntimos | estar em boas condições}
  \end{phonetics}
\end{entry}

\begin{entry}{要死}{9,6}{⾑、⽍}
  \begin{phonetics}{要死}{yao4si3}
    \definition{adv.}{extremamente | muito}
  \end{phonetics}
\end{entry}

\begin{entry}{要求}{9,7}{⾑、⽔}
  \begin{phonetics}{要求}{yao1qiu2}[][HSK 2]
    \definition[个,点]{s.}{exigência; demanda; reivindicação; desejos ou condições específicas propostas}
    \definition{v.}{pedir; exigir; exigir; reivindicar; apresentar desejos ou condições específicas, esperando que sejam satisfeitos ou realizados}
  \end{phonetics}
\end{entry}

\begin{entry}{要挟}{9,9}{⾑、⼿}
  \begin{phonetics}{要挟}{yao1xie2}
    \definition{v.}{chantagear | ameaçar}
  \end{phonetics}
\end{entry}

\begin{entry}{要是}{9,9}{⾑、⽇}
  \begin{phonetics}{要是}{yao4shi5}[][HSK 3]
    \definition{conj.}{se; suponha; no caso de; conecta frases, expressa uma relação hipotética, equivalente a 如果, e pode ser usado em conjunto com 的话}
  \seealsoref{的话}{de5 hua4}
  \seealsoref{如果}{ru2guo3}
  \end{phonetics}
\end{entry}

\begin{entry}{要是……的话}{9,9,8,8}{⾑、⽇、⽩、⾔}
  \begin{phonetics}{要是……的话}{yao4shi5 de5hua4}
    \definition{conj.}{se for assim\dots}
  \end{phonetics}
\end{entry}

\begin{entry}{要点}{9,9}{⾑、⽕}
  \begin{phonetics}{要点}{yao4dian3}
    \definition{s.}{pontos principais | essencial}
  \end{phonetics}
\end{entry}

\begin{entry}{要谎}{9,11}{⾑、⾔}
  \begin{phonetics}{要谎}{yao4huang3}
    \definition{v.}{pedir um preço enorme (como primeiro passo de negociação)}
  \end{phonetics}
\end{entry}

\begin{entry}{要强}{9,12}{⾑、⼸}
  \begin{phonetics}{要强}{yao4qiang2}
    \definition{adj.}{ansioso para se destacar | ansioso para progredir na vida | obstinado}
  \end{phonetics}
\end{entry}

\begin{entry}{觉得}{9,11}{⾒、⼻}
  \begin{phonetics}{觉得}{jue2de5}[][HSK 1]
    \definition{v.}{sentir; estar ciente; pressentir; causar uma sensação | pensar; sentir; encontrar; considerar (tom menos assertivo)}
  \end{phonetics}
\end{entry}

\begin{entry}{语}{9}{⾔}
  \begin{phonetics}{语}{yu3}
    \definition{s.}{língua; linguagem | dito; provérbio; refere-se especialmente a coloquialismos, provérbios, expressões idiomáticas ou palavras de livros antigos | sinal; meio não linguístico de comunicar ideias ; ações ou sinais que substituem palavras para expressar significado | palavras; expressão; refere-se a uma palavra, frase ou sentença}
    \definition{v.}{dizer; falar | (de pássaros, insetos, etc.) gorjear; pipilar}
  \end{phonetics}
  \begin{phonetics}{语}{yu4}
    \definition{v.}{contar; informar}
  \end{phonetics}
\end{entry}

\begin{entry}{语气}{9,4}{⾔、⽓}
  \begin{phonetics}{语气}{yu3qi4}
    \definition[个]{s.}{maneira de falar | tom}
  \end{phonetics}
\end{entry}

\begin{entry}{语言}{9,7}{⾔、⾔}
  \begin{phonetics}{语言}{yu3yan2}[][HSK 2]
    \definition[种,门]{s.}{linguagem; é uma ferramenta exclusiva dos humanos para expressar ideias e comunicar pensamentos; é um fenômeno social especial e consiste em um sistema específico de pronúncia, vocabulário e gramática | linguagem falada}
  \end{phonetics}
\end{entry}

\begin{entry}{语言实验室}{9,7,8,10,9}{⾔、⾔、⼧、⾺、⼧}
  \begin{phonetics}{语言实验室}{yu3yan2shi2yan4shi4}
    \definition{s.}{laboratório de línguas}
  \end{phonetics}
\end{entry}

\begin{entry}{语法}{9,8}{⾔、⽔}
  \begin{phonetics}{语法}{yu3fa3}[][HSK 4]
    \definition[个]{s.}{gramática; maneira como o idioma é estruturado, incluindo a formação e as variações de palavras, a organização de frases e sentenças | estudo da gramática; estudo das regras de estrutura linguística}
  \end{phonetics}
\end{entry}

\begin{entry}{语法术语}{9,8,5,9}{⾔、⽔、⽊、⾔}
  \begin{phonetics}{语法术语}{yu3fa3shu4yu3}
    \definition{s.}{termo gramatical}
  \end{phonetics}
\end{entry}

\begin{entry}{语音}{9,9}{⾔、⾳}
  \begin{phonetics}{语音}{yu3 yin1}[][HSK 4]
    \definition{s.}{voz; pronúncia; sons da fala; som de alguém falando | pronúncia; som do idioma}
  \end{phonetics}
\end{entry}

\begin{entry}{语调}{9,10}{⾔、⾔}
  \begin{phonetics}{语调}{yu3diao4}
    \definition[个]{s.}{entonação}
  \end{phonetics}
\end{entry}

\begin{entry}{误会}{9,6}{⾔、⼈}
  \begin{phonetics}{误会}{wu4hui4}
    \definition[场]{s.}{mal-entendido; desentendimentos ou conflitos decorrentes de mal-entendidos}
    \definition{v.}{entender mal; entender errado; interpretar mal; não entender; não entender corretamente o significado}
  \end{phonetics}
\end{entry}

\begin{entry}{误点}{9,9}{⾔、⽕}
  \begin{phonetics}{误点}{wu4dian3}
    \definition{v.+compl.}{atrasar | chegar tarde}
  \end{phonetics}
\end{entry}

\begin{entry}{误解}{9,13}{⾔、⾓}
  \begin{phonetics}{误解}{wu4jie3}[][HSK 5]
    \definition[种]{s.}{equívoco; mal-entendido; desentendimento}
    \definition{v.}{interpretar mal; interpretar erroneamente; não compreender corretamente}
  \end{phonetics}
\end{entry}

\begin{entry}{诱人}{9,2}{⾔、⼈}
  \begin{phonetics}{诱人}{you4ren2}
    \definition{adj.}{atraente | cativante}
  \end{phonetics}
\end{entry}

\begin{entry}{说}{9}{⾔}
  \begin{phonetics}{说}{shui4}
    \definition{v.}{persuadir}
  \end{phonetics}
  \begin{phonetics}{说}{shuo1}[][HSK 1]
    \definition{s.}{uma teoria (normalmente o último caractere, como em 日心说, teoria heliocêntrica); ensinamentos; doutrina}
    \definition{v.}{falar; conversar; dizer | explicar | repreender | atuar como casamenteiro | referir-se a; indicar | criticar; aconselhar | fazer uma combinação; conciliar; mediar | discutir; falar sobre; conversar sobre | uma forma de expressão linguística da arte cênica}
  \seealsoref{日心说}{ri4 xin1 shuo1}
  \end{phonetics}
\end{entry}

\begin{entry}{说不定}{9,4,8}{⾔、⼀、⼧}
  \begin{phonetics}{说不定}{shuo1bu5ding4}[][HSK 4]
    \definition{adv.}{talvez; indica uma estimativa, possivelmente, provavelmente}
    \definition{v.}{não ter certeza; não estar certo; ser impreciso}
  \end{phonetics}
\end{entry}

\begin{entry}{说好}{9,6}{⾔、⼥}
  \begin{phonetics}{说好}{shuo1hao3}
    \definition{v.}{chegar a um acordo | concluir negociações}
  \end{phonetics}
\end{entry}

\begin{entry}{说完}{9,7}{⾔、⼧}
  \begin{phonetics}{说完}{shuo1-wan2}
    \definition{expr.}{acabar/terminar palavras}
  \end{phonetics}
\end{entry}

\begin{entry}{说明}{9,8}{⾔、⽇}
  \begin{phonetics}{说明}{shuo1ming2}[][HSK 2]
    \definition[本,个]{s.}{legenda; instrução; explicação}
    \definition{v.}{mostrar; explicar; ilustrar | indicar; mostrar; provar; demonstrar; usar materiais confiáveis para demonstrar ou determinar a autenticidade de pessoas ou coisas}
  \end{phonetics}
\end{entry}

\begin{entry}{说服}{9,8}{⾔、⽉}
  \begin{phonetics}{说服}{shuo1fu2}[][HSK 4]
    \definition{v.}{persuadir; convencer; convencer a outra parte com palavras bem fundamentadas}
  \end{phonetics}
\end{entry}

\begin{entry}{说法}{9,8}{⾔、⽔}
  \begin{phonetics}{说法}{shuo1 fa3}[][HSK 5]
    \definition[种]{s.}{maneira de dizer uma coisa; palavras ou frases usadas para expressar significado | declaração; versão; argumento; opinião; ponto de vista | motivo; razão; motivos ou bases para a resolução do problema}
  \end{phonetics}
\end{entry}

\begin{entry}{说话}{9,8}{⾔、⾔}
  \begin{phonetics}{说话}{shuo1hua4}[][HSK 1]
    \definition{adv.}{imediatamente; em um minuto; refere-se ao tempo que leva para falar, indicando um período muito curto}
    \definition{v.}{falar; conversar; dizer; expressar o significado através da linguagem | conversar (conversa fiada); bater papo | fofocar; conversar; criticar; censurar}
  \end{phonetics}
\end{entry}

\begin{entry}{说理}{9,11}{⾔、⽟}
  \begin{phonetics}{说理}{shuo1li3}
    \definition{v.}{racionalizar | discutir logicamente}
  \end{phonetics}
\end{entry}

\begin{entry}{说谎}{9,11}{⾔、⾔}
  \begin{phonetics}{说谎}{shuo1huang3}
    \definition{v.+compl.}{mentir | contar uma mentira}
  \end{phonetics}
\end{entry}

\begin{entry}{贱}{9}{⾙}
  \begin{phonetics}{贱}{jian4}
    \definition*{s.}{sobrenome Jian}
    \definition{adj.}{baixo preço; barato (oposto a 贵) | humilde (oposto a 贵) | baixo; básico; desprezível | humilde; baixa posição social}
    \definition{pron.}{meu (autodepreciativo)}
  \seealsoref{贵}{gui4}
  \end{phonetics}
\end{entry}

\begin{entry}{贴}{9}{⾙}
  \begin{phonetics}{贴}{tie1}[][HSK 4]
    \definition{adj.}{submisso; obediente}
    \definition{clas.}{para uso em gessos, emplastros}
    \definition{s.}{subsídio; subvenção}
    \definition{v.}{grudar; colar | aninhar-se a; aconchegar-se a | subsidiar; ajudar financeiramente}
  \end{phonetics}
\end{entry}

\begin{entry}{贵}{9}{⾙}
  \begin{phonetics}{贵}{gui4}[][HSK 1]
    \definition*{s.}{sobrenome Gui}
    \definition*{s.}{abreviação de Província de Guizhou, 贵州}
    \definition{adj.}{caro; dispendioso; (oposto de 贱) | altamente valorizado; valioso | de alta patente; nobre (oposto de 贱) | caro; preço ou valor elevado (em oposição a 贱) | digno de ser valorizado ou apreciado | nobre; honrado; posição social elevada}
    \definition{pron.}{seu (honrado)}
  \seealsoref{贵州}{gui4zhou1}
  \seealsoref{贱}{jian4}
  \end{phonetics}
\end{entry}

\begin{entry}{贵州}{9,6}{⾙、⼮}
  \begin{phonetics}{贵州}{gui4zhou1}
    \definition*{s.}{Guizhou (Província)}
  \end{phonetics}
\end{entry}

\begin{entry}{贵姓}{9,8}{⾙、⼥}
  \begin{phonetics}{贵姓}{gui4xing4}
    \definition{expr.}{qual seu sobrenome?}
  \end{phonetics}
\end{entry}

\begin{entry}{贷}{9}{⾙}
  \begin{phonetics}{贷}{dai4}
    \definition[笔]{s.}{empréstimo; valor do empréstimo}
    \definition{v.}{pedir dinheiro emprestado ou emprestar dinheiro | fugir da responsabilidade | perdoar}
  \end{phonetics}
\end{entry}

\begin{entry}{贷款}{9,12}{⾙、⽋}
  \begin{phonetics}{贷款}{dai4kuan3}[][HSK 5]
    \definition[个,笔]{s.}{empréstimo; crédito;}
    \definition{v.}{fornecer um empréstimo; conceder um empréstimo; conceder crédito a; emprestar dinheiro para quem precisa}
  \end{phonetics}
\end{entry}

\begin{entry}{贸易}{9,8}{⾙、⽇}
  \begin{phonetics}{贸易}{mao4yi4}[][HSK 5]
    \definition[笔,宗,项]{s.}{comércio; troca; negócios; refere-se a atividades comerciais, como a troca de mercadorias}
    \definition{v.}{fazer uma transação comercial}
  \end{phonetics}
\end{entry}

\begin{entry}{费}{9}{⾙}
  \begin{phonetics}{费}{fei4}[][HSK 3]
    \definition*{s.}{sobrenome Fei}
    \definition{s.}{taxa; despesa; encargo}
    \definition{v.}{custar; gastar; despender | ser desperdiçador; consumir em excesso; gastar algo muito rapidamente; consumo excessivo (oposto a 省)}
  \seealsoref{省}{sheng3}
  \end{phonetics}
\end{entry}

\begin{entry}{费用}{9,5}{⾙、⽤}
  \begin{phonetics}{费用}{fei4 yong4}[][HSK 3]
    \definition[笔,个]{s.}{custo; despesa; desembolso}
  \end{phonetics}
\end{entry}

\begin{entry}{贺}{9}{⾙}
  \begin{phonetics}{贺}{he4}
    \definition*{s.}{sobrenome He}
    \definition{v.}{parabenizar | congratular}
  \end{phonetics}
\end{entry}

\begin{entry}{贺卡}{9,5}{⾙、⼘}
  \begin{phonetics}{贺卡}{he4 ka3}[][HSK 5]
    \definition[张]{s.}{cartão de felicitações; pedaço de papel para parabenizar amigos e parentes em seu casamento, aniversário ou festivais, geralmente impresso com palavras e desenhos de felicitações}
  \end{phonetics}
\end{entry}

\begin{entry}{轴承}{9,8}{⾞、⼿}
  \begin{phonetics}{轴承}{zhou2cheng2}
    \definition{s.}{(mecânico) rolamento}
  \end{phonetics}
\end{entry}

\begin{entry}{轻}{9}{⾞}
  \begin{phonetics}{轻}{qing1}[][HSK 2]
    \definition{adj.}{de pouco peso; leve (oposto de 重) | (de carga, equipamento, etc.) pequeno; simples | pequeno em número, grau, etc. | não sério; relaxante; leve | sem importância | suave; delicado | levianos, crédulos | leve; peso leve; densidade baixa | leve; descontraído; fácil | imprudente; descuidado | inconstante; frívolo}
    \definition{v.}{menosprezar; subestimar}
  \seealsoref{重}{zhong4}
  \end{phonetics}
\end{entry}

\begin{entry}{轻易}{9,8}{⾞、⽇}
  \begin{phonetics}{轻易}{qing1yi4}[][HSK 4]
    \definition{adj.}{fácil; simples}
    \definition{adv.}{facilmente; prontamente | facilmente; precipitadamente; indica que uma ação é realizada casualmente, geralmente usado em frases negativas}
  \end{phonetics}
\end{entry}

\begin{entry}{轻松}{9,8}{⾞、⽊}
  \begin{phonetics}{轻松}{qing1song1}[][HSK 4]
    \definition{adj.}{leve; relaxado; livre de fardos; não se sentir nervoso ou cansado}
    \definition{v.}{relaxar; levar as coisas menos a sério}
  \end{phonetics}
\end{entry}

\begin{entry}{迷}{9}{⾡}
  \begin{phonetics}{迷}{mi2}[][HSK 3]
    \definition[个]{s.}{fã; entusiasta; aficionado; pessoa que gosta excessivamente de algo}
    \definition{v.}{estar confuso; perder o rumo; se perder-se; perda da capacidade de discernimento e julgamento | ficar fascinado por; entregar-se a; ficar encantado com (por); ser louco por | confundir; desorientar; fascinar; encantar; tornar indistinto; deixar encantado e fascinado}
  \end{phonetics}
\end{entry}

\begin{entry}{迷人}{9,2}{⾡、⼈}
  \begin{phonetics}{迷人}{mi2ren2}[][HSK 5]
    \definition{adj.}{encantador; fascinante; sedutor; hipnotizante}
    \definition{v.}{confundir; intrigar; enganar}
  \end{phonetics}
\end{entry}

\begin{entry}{迷你}{9,7}{⾡、⼈}
  \begin{phonetics}{迷你}{mi2ni3}
    \definition{adj.}{(empréstimo linguístico) mini, como em minissaia ou \emph{Mini Cooper}}
  \end{phonetics}
\end{entry}

\begin{entry}{迷信}{9,9}{⾡、⼈}
  \begin{phonetics}{迷信}{mi2xin4}[][HSK 5]
    \definition{s.}{superstição; crença supersticiosa; fé cega; adoração cega; crença em deuses, espíritos e fantasmas}
    \definition{v.}{ter fé cega em; fazer um fetiche de}
  \end{phonetics}
\end{entry}

\begin{entry}{迷宫}{9,9}{⾡、⼧}
  \begin{phonetics}{迷宫}{mi2gong1}
    \definition{s.}{labirinto}
  \end{phonetics}
\end{entry}

\begin{entry}{迷恋}{9,10}{⾡、⼼}
  \begin{phonetics}{迷恋}{mi2lian4}
    \definition{adj.}{obcecado}
    \definition{v.}{estar/ser apaixonado por | ficar encantado por | estar/ser obcecado por}
  \end{phonetics}
\end{entry}

\begin{entry}{迷路}{9,13}{⾡、⾜}
  \begin{phonetics}{迷路}{mi2lu4}
    \definition{s.}{labirinto | ouvido interno}
    \definition{v.+compl.}{perder o caminho | perder-se | seguir pelo caminho errado | não conseguir encontrar o caminho}
  \end{phonetics}
\end{entry}

\begin{entry}{追}{9}{⾡}
  \begin{phonetics}{追}{zhui1}[][HSK 3]
    \definition{v.}{perseguir; correr atrás; seguir de perto | rastrear; investigar; chegar ao fundo de | procurar; ir atrás; esforçar-se para alcançar um determinado objetivo | recordar; relembrar | fazer depois do ocorrido; retrabalhar | cortejar (uma mulher)}
  \end{phonetics}
\end{entry}

\begin{entry}{追求}{9,7}{⾡、⽔}
  \begin{phonetics}{追求}{zhui1qiu2}[][HSK 4]
    \definition{s.}{perseguição (ações e metas positivas)}[她的追求是获得成功。___Sua meta é alcançar o sucesso.]
    \definition{v.}{buscar; aspirar; perseguir | cortejar, uma referência especial ao namoro}
  \end{phonetics}
\end{entry}

\begin{entry}{追赶}{9,10}{⾡、⾛}
  \begin{phonetics}{追赶}{zhui1gan3}
    \definition{v.}{perseguir | acelerar | alcançar | ultrapassar}
  \end{phonetics}
\end{entry}

\begin{entry}{退}{9}{⾡}
  \begin{phonetics}{退}{tui4}[][HSK 3]
    \definition{v.}{recuar; mover-se para trás  (oposto de 進) | remover; retirar; fazer recuar; mover para trás | desistir; retirar-se de | refluir; declinar; retroceder | aposentar-se; deixar o emprego por atingir a idade estipulada ou por problemas de saúde | retornar; reembolsar; devolver | romper; cancelar o que foi decidido}
  \seealsoref{进}{jin4}
  \end{phonetics}
\end{entry}

\begin{entry}{退出}{9,5}{⾡、⼐}
  \begin{phonetics}{退出}{tui4 chu1}[][HSK 3]
    \definition{v.}{desistir; retirar-se; separar-se; retirar-se de; abandonar o local ou outro lugar e parar de participar; abandonaar o grupo ou organização}
  \end{phonetics}
\end{entry}

\begin{entry}{退休}{9,6}{⾡、⼈}
  \begin{phonetics}{退休}{tui4xiu1}[][HSK 3]
    \definition{v.+compl.}{aposentar-se; os trabalhadores que deixarem o emprego por velhice ou invalidez causada pelo trabalho receberão as despesas de subsistência conforme o cronograma}
  \end{phonetics}
\end{entry}

\begin{entry}{送}{9}{⾡}
  \begin{phonetics}{送}{song4}[][HSK 1]
    \definition*{s.}{sobrenome Song}
    \definition{v.}{transportar; entregar | dar; dar como presente; presentear | acompanhar; despedir-se de alguém (ao sair); acompanhar a pessoa que está partindo até o destino ou caminhar um trecho com ela | escoltar}
  \end{phonetics}
\end{entry}

\begin{entry}{送到}{9,8}{⾡、⼑}
  \begin{phonetics}{送到}{song4 dao4}[][HSK 2]
    \definition{v.}{enviar para (lugar)}
  \end{phonetics}
\end{entry}

\begin{entry}{送给}{9,9}{⾡、⽷}
  \begin{phonetics}{送给}{song4 gei3}[][HSK 2]
    \definition{v.}{dar a (alguém ou organização); dar como algo gratuito; dar como presente}
  \end{phonetics}
\end{entry}

\begin{entry}{适用}{9,5}{⾡、⽤}
  \begin{phonetics}{适用}{shi4 yong4}[][HSK 3]
    \definition{adj.}{adequado; aplicável}
  \end{phonetics}
\end{entry}

\begin{entry}{适合}{9,6}{⾡、⼝}
  \begin{phonetics}{适合}{shi4he2}[][HSK 3]
    \definition{v.}{servir; caber; se adequar; atender às necessidades de uma determinada situação ou pessoa}
  \end{phonetics}
\end{entry}

\begin{entry}{适应}{9,7}{⾡、⼴}
  \begin{phonetics}{适应}{shi4ying4}[][HSK 3]
    \definition{v.}{ajustar-se; adequar-se; adaptar-se; fazer as alterações correspondentes para se adequar à medida que as condições mudam}
  \end{phonetics}
\end{entry}

\begin{entry}{逃}{9}{⾡}
  \begin{phonetics}{逃}{tao2}[][HSK 5]
    \definition{v.}{fugir; escapar; correr; dar no pé | evadir; esquivar-se; escapar}
  \end{phonetics}
\end{entry}

\begin{entry}{逃走}{9,7}{⾡、⾛}
  \begin{phonetics}{逃走}{tao2 zou3}[][HSK 5]
    \definition{v.}{escapar; afastar-se de pessoas, coisas ou lugares que não são bons para você ou que você não gosta}
  \end{phonetics}
\end{entry}

\begin{entry}{逃跑}{9,12}{⾡、⾜}
  \begin{phonetics}{逃跑}{tao2 pao3}[][HSK 5]
    \definition{v.}{fugir; escapar; correr; partir para fugir de um ambiente ou de coisas que não lhe são favoráveis}
  \end{phonetics}
\end{entry}

\begin{entry}{逆境}{9,14}{⾡、⼟}
  \begin{phonetics}{逆境}{ni4jing4}
    \definition{s.}{adversidade | tribulação}
  \end{phonetics}
\end{entry}

\begin{entry}{选}{9}{⾡}
  \begin{phonetics}{选}{xuan3}[][HSK 2]
    \definition{s.}{pessoa ou coisa selecionada | seleções; antologia; trabalhos selecionados e compilados}
    \definition{v.}{selecionar; escolher | eleger}
  \end{phonetics}
\end{entry}

\begin{entry}{选手}{9,4}{⾡、⼿}
  \begin{phonetics}{选手}{xuan3shou3}[][HSK 3]
    \definition[位,名,个,些]{s.}{jogador; (selecionado) competidor; atleta selecionado para uma competição esportiva; participantes selecionados entre um grande número de candidatos}
  \end{phonetics}
\end{entry}

\begin{entry}{选择}{9,8}{⾡、⼿}
  \begin{phonetics}{选择}{xuan3ze2}[][HSK 4]
    \definition[个,种,次]{s.}{escolha; opção; resultado da escolha; possibilidade de escolha}
    \definition{v.}{selecionar; escolher}
  \end{phonetics}
\end{entry}

\begin{entry}{选修}{9,9}{⾡、⼈}
  \begin{phonetics}{选修}{xuan3 xiu1}[][HSK 5]
    \definition{v.}{fazer um curso eletivo; selecionar os cursos a serem estudados entre os cursos disponíveis}
  \end{phonetics}
\end{entry}

\begin{entry}{重}{9}{⾥}
  \begin{phonetics}{重}{chong2}
    \definition*{s.}{sobrenome Chong}
    \definition{adv.}{novamente; mais uma vez}
    \definition{clas.}{usado para camadas}
    \definition{v.}{repetir; duplicar}
  \end{phonetics}
  \begin{phonetics}{重}{zhong4}[][HSK 1,3]
    \definition{adj.}{pesado; densidade elevada | profundo; sério; grau profundo | importante; significativo | discreto; prudente | considerável em quantidade ou valor}
    \definition[斤,公,斤,吨]{s.}{peso}
    \definition{v.}{colocar (colocar, pôr) ênfase em; dar valor a; atribuir importância a}
  \end{phonetics}
\end{entry}

\begin{entry}{重大}{9,3}{⾥、⼤}
  \begin{phonetics}{重大}{zhong4da4}[][HSK 3]
    \definition{adj.}{excelente; importante; significativo; de grande importância}
  \end{phonetics}
\end{entry}

\begin{entry}{重阳节}{9,6,5}{⾥、⾩、⾋}
  \begin{phonetics}{重阳节}{chong2yang2jie2}
    \definition*{s.}{Festa do Duplo Nove, Festival Yang, dia de subir aos lugares mais altos para evitar calamidades e dia do culto aos antepassados (9º dia do nono mês lunar)}
  \end{phonetics}
\end{entry}

\begin{entry}{重视}{9,8}{⾥、⾒}
  \begin{phonetics}{重视}{zhong4shi4}[][HSK 2]
    \definition{v.}{valorizar; dar peso a; atribuir importância a; prestar atenção a; considerar a virtude ou o talento de uma pessoa ou o papel de algo como importante e levá-lo a sério}
  \end{phonetics}
\end{entry}

\begin{entry}{重复}{9,9}{⾥、⼢}
  \begin{phonetics}{重复}{chong2fu4}[][HSK 2]
    \definition{v.}{repetir; iterar; duplicar; reduplicar | fazer algo novamente; repetir as mesmas palavras, fazer as mesmas coisas}
  \end{phonetics}
\end{entry}

\begin{entry}{重点}{9,9}{⾥、⽕}
  \begin{phonetics}{重点}{chong2dian3}
    \definition[个]{adj./adv./s.}{nota principal; ponto-chave; ponto focal; ênfase}
  \end{phonetics}
  \begin{phonetics}{重点}{zhong4dian3}[][HSK 2]
    \definition[个]{s.}{nota principal; ponto-chave; ponto}
  \end{phonetics}
\end{entry}

\begin{entry}{重要}{9,9}{⾥、⾑}
  \begin{phonetics}{重要}{zhong4yao4}[][HSK 1]
    \definition{adj.}{importante; significativo; relevante; de grande importância, função e impacto}
  \end{phonetics}
\end{entry}

\begin{entry}{重重}{9,9}{⾥、⾥}
  \begin{phonetics}{重重}{chong2chong2}
    \definition{adv.}{camada após camada | um após o outro}
  \end{phonetics}
  \begin{phonetics}{重重}{zhong4zhong4}
    \definition{adv.}{fortemente | severamente}
  \end{phonetics}
\end{entry}

\begin{entry}{重逢}{9,10}{⾥、⾡}
  \begin{phonetics}{重逢}{chong2feng2}
    \definition{s.}{reunião}
    \definition{v.}{encontrar-se novamente | reunir-se}
  \end{phonetics}
\end{entry}

\begin{entry}{重量}{9,12}{⾥、⾥}
  \begin{phonetics}{重量}{zhong4liang4}[][HSK 4]
    \definition[个]{s.}{peso; a magnitude da força da gravidade em um objeto}
  \end{phonetics}
\end{entry}

\begin{entry}{重新}{9,13}{⾥、⽄}
  \begin{phonetics}{重新}{chong2xin1}[][HSK 2]
    \definition{adv.}{novamente; de novo; significa repetir uma ação ou comportamento já realizado | indica que se deve começar do início (mudança de método ou conteúdo)}
  \end{phonetics}
\end{entry}

\begin{entry}{钝}{9}{⾦}
  \begin{phonetics}{钝}{dun4}
    \definition{adj.}{sem corte; opaco (oposto a 快, 利, 锐) | estúpido; sem noção | maçante}
  \seealsoref{快}{kuai4}
  \seealsoref{利}{li4}
  \seealsoref{锐}{rui4}
  \end{phonetics}
\end{entry}

\begin{entry}{钟}{9}{⾦}
  \begin{phonetics}{钟}{zhong1}[][HSK 3]
    \definition*{s.}{sobrenome Zhong}
    \definition[顶,个,口]{s.}{sino; campainha; um instrumento de percussão antigo, oco, feito de cobre ou ferro | relógio; um aparelho para medir o tempo que não se leva consigo | tempo medido em horas e minutos; referindo-se ao tempo ou momento| um recipiente antigo para guardar vinho, com barriga grande e gargalo pequeno | sino; refere-se especificamente aos sinos pendurados em templos ou outros locais, cujo som é usado para marcar as horas, alertar ou convocar pessoas}
    \definition{v.}{focar; concentrar (as afeições de alguém, etc.)}
  \end{phonetics}
\end{entry}

\begin{entry}{钟室}{9,9}{⾦、⼧}
  \begin{phonetics}{钟室}{zhong1shi4}
    \definition{s.}{campanário | sala do relógio}
  \end{phonetics}
\end{entry}

\begin{entry}{钟罩}{9,13}{⾦、⽹}
  \begin{phonetics}{钟罩}{zhong1zhao4}
    \definition{s.}{redoma | dossel de sino}
  \end{phonetics}
\end{entry}

\begin{entry}{钢}{9}{⾦}
  \begin{phonetics}{钢}{gang1}
    \definition[吨,块,根]{s.}{aço; liga de ferro e carbono}
  \end{phonetics}
\end{entry}

\begin{entry}{钢丝}{9,5}{⾦、⼀}
  \begin{phonetics}{钢丝}{gang1si1}
    \definition{s.}{cabo de aço | corda bamba}
  \end{phonetics}
\end{entry}

\begin{entry}{钢笔}{9,10}{⾦、⽵}
  \begin{phonetics}{钢笔}{gang1 bi3}[][HSK 5]
    \definition[支]{s.}{caneta-tinteiro; canetas com ponta metálica}
  \end{phonetics}
\end{entry}

\begin{entry}{钢琴}{9,12}{⾦、⽟}
  \begin{phonetics}{钢琴}{gang1qin2}[][HSK 5]
    \definition[架]{s.}{piano}
  \end{phonetics}
\end{entry}

\begin{entry}{钥匙}{9,11}{⾦、⼔}
  \begin{phonetics}{钥匙}{yao4shi5}
    \definition[把]{s.}{chave}
  \end{phonetics}
\end{entry}

\begin{entry}{钥匙孔}{9,11,4}{⾦、⼔、⼦}
  \begin{phonetics}{钥匙孔}{yao4shi5kong3}
    \definition{s.}{buraco da fechadura}
  \end{phonetics}
\end{entry}

\begin{entry}{钥匙卡}{9,11,5}{⾦、⼔、⼘}
  \begin{phonetics}{钥匙卡}{yao4shi5ka3}
    \definition{s.}{cartão de acesso}
  \end{phonetics}
\end{entry}

\begin{entry}{钥匙洞孔}{9,11,9,4}{⾦、⼔、⽔、⼦}
  \begin{phonetics}{钥匙洞孔}{yao4shi5dong4kong3}
    \definition{s.}{buraco da fechadura}
  \end{phonetics}
\end{entry}

\begin{entry}{钥匙圈}{9,11,11}{⾦、⼔、⼞}
  \begin{phonetics}{钥匙圈}{yao4shi5quan1}
    \definition{s.}{chaveiro}
  \end{phonetics}
\end{entry}

\begin{entry}{钩}{9}{⾦}
  \begin{phonetics}{钩}{gou1}
    \definition*{s.}{sobrenome Gou}
    \definition[只,个]{s.}{gancho | traço de gancho em caracteres chineses | marca de verificação; visto; \emph{tick}; \emph{check mark} | marca em forma de gancho | uma espada em forma de gancho | forma falada do numeral 9 em certas ocasiões}
    \definition{v.}{prender com um gancho; enganchar | fazer crochê | costurar com pontos grandes | costurar com pontos longos}
  \end{phonetics}
\end{entry}

\begin{entry}{闻}{9}{⾨}
  \begin{phonetics}{闻}{wen2}[][HSK 2]
    \definition*{s.}{sobrenome Wen}
    \definition{adj.}{bem conhecido; famoso}
    \definition{s.}{notícia; história | reputação | boato; rumor}
    \definition{v.}{cheirar | ouvir}
  \end{phonetics}
\end{entry}

\begin{entry}{阁}{9}{⾨}
  \begin{phonetics}{阁}{ge2}
    \definition{s.}{pavilhão (geralmente de dois andares) | gabinete (de um governo) | (datado) quarto da mulher; \emph{boudoir} | prateleira}
  \end{phonetics}
\end{entry}

\begin{entry}{阁下}{9,3}{⾨、⼀}
  \begin{phonetics}{阁下}{ge2xia4}
    \definition{pron.}{Sua Excelência | Sua Majestade | \emph{Sire}}
  \end{phonetics}
\end{entry}

\begin{entry}{院}{9}{⾩}
  \begin{phonetics}{院}{yuan4}[][HSK 2]
    \definition*{s.}{sobrenome Yuan}
    \definition[个]{s.}{pátio; quintal; complexo | designação para certos escritórios governamentais e locais públicos | faculdade; academia; instituto de ensino superior | hospital}
  \end{phonetics}
\end{entry}

\begin{entry}{院子}{9,3}{⾩、⼦}
  \begin{phonetics}{院子}{yuan4zi5}[][HSK 2]
    \definition[个,座,处]{s.}{quintal; pátio; o espaço aberto na frente ou atrás de uma casa cercado por muros ou cercas}
  \end{phonetics}
\end{entry}

\begin{entry}{院长}{9,4}{⾩、⾧}
  \begin{phonetics}{院长}{yuan4zhang3}[][HSK 2]
    \definition[个,位,名]{s.}{reitor; diretor; o mais alto funcionário de qualquer instituição ou escola pública ou privada}
  \end{phonetics}
\end{entry}

\begin{entry}{除}{9}{⾩}
  \begin{phonetics}{除}{chu2}
    \definition*{s.}{sobrenome Chu}
    \definition{prep.}{exceto; não incluído | além do mais}
    \definition{s.}{degraus de uma casa; degraus de uma porta; escadaria}
    \definition{v.}{remover; livrar-se de; eliminar; limpar | dividir; executar operação de divisão | nomear para o cargo}
  \end{phonetics}
\end{entry}

\begin{entry}{除了}{9,2}{⾩、⼅}
  \begin{phonetics}{除了}{chu2le5}[][HSK 3]
    \definition{prep.}{exceto; à parte; indica que o que foi dito não é levado em consideração | além disso; além de; usado em conjunto com 还, 也 e 只, indica que, além de algo, há ainda outra coisa | ou \dots ou \dots; usado em conjunto com 就是, significa "ou assim ou assado"}
  \seealsoref{还}{hai2}
  \seealsoref{就是}{jiu4 shi4}
  \seealsoref{也}{ye3}
  \seealsoref{只}{zhi3}
  \end{phonetics}
\end{entry}

\begin{entry}{除夕}{9,3}{⾩、⼣}
  \begin{phonetics}{除夕}{chu2xi1}[][HSK 5]
    \definition*{s.}{Véspera de Ano Novo Lunar; a noite do último dia do ano, também se refere ao último dia do ano}
  \end{phonetics}
\end{entry}

\begin{entry}{除非}{9,8}{⾩、⾮}
  \begin{phonetics}{除非}{chu2fei1}[][HSK 5]
    \definition{conj.}{a menos que; somente se; indica a única condição, equivalente a 只有, frequentemente combinada com 才, 否则, 不然, etc.}
  \seealsoref{不然}{bu4ran2}
  \seealsoref{才}{cai2}
  \seealsoref{否则}{fou3ze2}
  \seealsoref{只有}{zhi3 you3}
  \end{phonetics}
\end{entry}

\begin{entry}{面}{9}{⾯}[Kangxi 176]
  \begin{phonetics}{面}{mian4}[][HSK 2]
    \definition*{s.}{sobrenome Mian}
    \definition{adj.}{macio e farinhento; descreve algo que é muito macio ao comer | superficial}
    \definition{adv.}{diretamente; pessoalmente; na frente de alguém; cara a cara}
    \definition{clas.}{usado para objetos planos | usado para indicar o número de vezes que as pessoas se encontram}
    \definition[斤,两,碗]{s.}{face; parte frontal da cabeça; rosto | topo; superfície | capa; exterior; a parte externa de um objeto ou a face frontal de um tecido (em oposição à 里) | (matemática) superfície | cara; sentimento; emoção | geral; área total; abrangente; toda a região | lado; aspecto | escopo; escala; extensão; alcance; âmbito | farinha; farinha de trigo | pó; algo em pó | macarrão; \emph{noodle}}
    \definition{suf.}{sufixo para localização ou direção; anexado ao final de palavras que indicam localização, equivalente a 边}
    \definition{v.}{encarar algo | encontrar; revelar-se}
  \seealsoref{边}{bian1}
  \seealsoref{里}{li3}
  \end{phonetics}
\end{entry}

\begin{entry}{面子}{9,3}{⾯、⼦}
  \begin{phonetics}{面子}{mian4zi5}[][HSK 5]
    \definition{s.}{face; exterior; parte externa; superfície do objeto | imagem; reputação; prestígio; decência; vaidade superficial | sentimentos; sensibilidades | pó}
  \end{phonetics}
\end{entry}

\begin{entry}{面包}{9,5}{⾯、⼓}
  \begin{phonetics}{面包}{mian4bao1}[][HSK 1]
    \definition[个,片,袋,块]{s.}{pão}[我买八个面包了。___Comprei oito pães. | 他在吃两片面包。___Ele está comendo duas fatias de pão. | 我在家里带了一袋面包。___Trouxe um saco de pão para casa. | 我拿了一块面包。___Peguei um pedaço de pão.]
  \end{phonetics}
\end{entry}

\begin{entry}{面对}{9,5}{⾯、⼨}
  \begin{phonetics}{面对}{mian4dui4}[][HSK 3]
    \definition{v.}{enfrentar; defrontar; olhar para (uma pessoa ou um objeto específico) | confrontar (problema); problemas, dificuldades e outras questões que precisam ser resolvidas e que merecem atenção}
  \end{phonetics}
\end{entry}

\begin{entry}{面对面}{9,5,9}{⾯、⼨、⾯}
  \begin{phonetics}{面对面}{mian4dui4mian4}
    \definition{expr.}{cara a cara}
  \end{phonetics}
\end{entry}

\begin{entry}{面对面吃面}{9,5,9,6,9}{⾯、⼨、⾯、⼝、⾯}
  \begin{phonetics}{面对面吃面}{mian4dui4mian4 chi1 mian4}
    \definition{expr.}{Comer macarrão cara a cara; indica que o seu estado atual, ou algumas das posições em que você está, ou algumas das coisas que você fez são muito claras}
  \end{phonetics}
\end{entry}

\begin{entry}{面团}{9,6}{⾯、⼞}
  \begin{phonetics}{面团}{mian4tuan2}
    \definition{s.}{massa | pasta}
  \end{phonetics}
\end{entry}

\begin{entry}{面条}{9,7}{⾯、⽊}
  \begin{phonetics}{面条}{mian4tiao2}
    \definition{s.}{macarrão | espaguete}
  \end{phonetics}
\end{entry}

\begin{entry}{面条儿}{9,7,2}{⾯、⽊、⼉}
  \begin{phonetics}{面条儿}{mian4 tiao2r5}[][HSK 1]
    \definition{s.}{macarrão; \emph{noodles}}
  \end{phonetics}
\end{entry}

\begin{entry}{面试}{9,8}{⾯、⾔}
  \begin{phonetics}{面试}{mian4 shi4}[][HSK 4]
    \definition[次]{s.}{entrevista; audição}
  \end{phonetics}
\end{entry}

\begin{entry}{面临}{9,9}{⾯、⼁}
  \begin{phonetics}{面临}{mian4lin2}[][HSK 4]
    \definition{v.}{ser confrontado com; encontrar (uma situação) na frente de}
  \end{phonetics}
\end{entry}

\begin{entry}{面前}{9,9}{⾯、⼑}
  \begin{phonetics}{面前}{mian4 qian2}[][HSK 2]
    \definition{s.}{antes; na frente de; diante de}
  \end{phonetics}
\end{entry}

\begin{entry}{面积}{9,10}{⾯、⽲}
  \begin{phonetics}{面积}{mian4ji1}[][HSK 3]
    \definition{s.}{área (de um andar, pedaço de terreno, etc.); área de uma superfície; o tamanho de uma superfície plana ou da superfície de um objeto}
  \end{phonetics}
\end{entry}

\begin{entry}{面貌}{9,14}{⾯、⾘}
  \begin{phonetics}{面貌}{mian4mao4}[][HSK 5]
    \definition[种,个]{s.}{rosto; traços faciais; formato do rosto; aparência | aparência; aspecto; aparência (das coisas)}
  \end{phonetics}
\end{entry}

\begin{entry}{韭菜}{9,11}{⾲、⾋}
  \begin{phonetics}{韭菜}{jiu3cai4}
    \definition{s.}{cebolinha chinesa | (figurativo) investidores de varejo que perdem seu dinheiro para operadores mais experientes (ou seja, são ``colhidos'' como cebolinhas)}
  \end{phonetics}
\end{entry}

\begin{entry}{音乐}{9,5}{⾳、⼃}
  \begin{phonetics}{音乐}{yin1yue4}[][HSK 2]
    \definition[种,段,张,曲]{s.}{música; ramo da arte que cria imagens artísticas, expressa pensamentos e sentimentos e reflete a vida real por meio da melodia e do ritmo da música; geralmente é dividido em duas categorias: música vocal e música instrumental}
  \end{phonetics}
\end{entry}

\begin{entry}{音乐厅}{9,5,4}{⾳、⼃、⼚}
  \begin{phonetics}{音乐厅}{yin1yue4ting1}
    \definition{s.}{auditório | teatro | \emph{concert hall}}
  \end{phonetics}
\end{entry}

\begin{entry}{音乐节}{9,5,5}{⾳、⼃、⾋}
  \begin{phonetics}{音乐节}{yin1yue4jie2}
    \definition{s.}{festival de música}
  \end{phonetics}
\end{entry}

\begin{entry}{音乐会}{9,5,6}{⾳、⼃、⼈}
  \begin{phonetics}{音乐会}{yin1 yue4 hui4}[][HSK 2]
    \definition[场]{s.}{concerto; atividades de execução de obras musicais}
  \end{phonetics}
\end{entry}

\begin{entry}{音乐光碟}{9,5,6,14}{⾳、⼃、⼉、⽯}
  \begin{phonetics}{音乐光碟}{yin1yue4guang1die2}
    \definition{s.}{CD de música}
  \end{phonetics}
\end{entry}

\begin{entry}{音乐学}{9,5,8}{⾳、⼃、⼦}
  \begin{phonetics}{音乐学}{yin1yue4xue2}
    \definition{s.}{musicologia}
  \end{phonetics}
\end{entry}

\begin{entry}{音乐学院}{9,5,8,9}{⾳、⼃、⼦、⾩}
  \begin{phonetics}{音乐学院}{yin1yue4xue2yuan4}
    \definition{s.}{conservatório | academia de música}
  \end{phonetics}
\end{entry}

\begin{entry}{音乐院}{9,5,9}{⾳、⼃、⾩}
  \begin{phonetics}{音乐院}{yin1yue4yuan4}
    \definition{s.}{conservatório | instituto de música}
  \end{phonetics}
\end{entry}

\begin{entry}{音乐家}{9,5,10}{⾳、⼃、⼧}
  \begin{phonetics}{音乐家}{yin1yue4jia1}
    \definition{s.}{músico}
  \end{phonetics}
\end{entry}

\begin{entry}{音节}{9,5}{⾳、⾋}
  \begin{phonetics}{音节}{yin1 jie2}[][HSK 2]
    \definition{s.}{sílaba}
  \end{phonetics}
\end{entry}

\begin{entry}{项}{9}{⾴}
  \begin{phonetics}{项}{xiang4}[][HSK 4]
    \definition*{s.}{sobrenome Xiang}
    \definition{clas.}{para itens discriminados; taxonomia}
    \definition{s.}{nuca (do pescoço); a parte de trás do pescoço | soma (de dinheiro); fundos para fins especiais | termo; em álgebra, significa uma única fórmula que não é unida por um sinal de mais ou de menos | item}
  \end{phonetics}
\end{entry}

\begin{entry}{项目}{9,5}{⾴、⽬}
  \begin{phonetics}{项目}{xiang4mu4}[][HSK 4]
    \definition{s.}{evento | item; projeto; trabalhos de engenharia, acadêmicos, etc., de conteúdo específico}
  \end{phonetics}
\end{entry}

\begin{entry}{顺}{9}{⾴}
  \begin{phonetics}{顺}{shun4}
    \definition{adj.}{correr bem | favorável}
  \end{phonetics}
\end{entry}

\begin{entry}{顺从}{9,4}{⾴、⼈}
  \begin{phonetics}{顺从}{shun4cong2}
    \definition{v.}{obedecer | submeter-se}
  \end{phonetics}
\end{entry}

\begin{entry}{顺心}{9,4}{⾴、⼼}
  \begin{phonetics}{顺心}{shun4xin1}
    \definition{adj.}{satisfatório | satisfeito}
  \end{phonetics}
\end{entry}

\begin{entry}{顺水}{9,4}{⾴、⽔}
  \begin{phonetics}{顺水}{shun4shui3}
    \definition{v.}{ir com o fluxo}
  \end{phonetics}
\end{entry}

\begin{entry}{顺延}{9,6}{⾴、⼵}
  \begin{phonetics}{顺延}{shun4yan2}
    \definition{v.}{adiar | procrastinar}
  \end{phonetics}
\end{entry}

\begin{entry}{顺当}{9,6}{⾴、⼹}
  \begin{phonetics}{顺当}{shun4dang5}
    \definition{adv.}{suavemente}
  \end{phonetics}
\end{entry}

\begin{entry}{顺耳}{9,6}{⾴、⽿}
  \begin{phonetics}{顺耳}{shun4'er3}
    \definition{adj.}{agradável ao ouvido}
  \end{phonetics}
\end{entry}

\begin{entry}{顺利}{9,7}{⾴、⼑}
  \begin{phonetics}{顺利}{shun4li4}[][HSK 2]
    \definition{adj.}{sem problemas; com sucesso; sem dificuldades; sem contratempos; sem obstáculos; sem obstáculos ou dificuldades significativas no desempenho das tarefas}
  \end{phonetics}
\end{entry}

\begin{entry}{顺序}{9,7}{⾴、⼴}
  \begin{phonetics}{顺序}{shun4xu4}[][HSK 4]
    \definition{adv.}{por sua vez; na ordem correta; na devida ordem; na ordem adequada; na ordem apropriada}
    \definition[个]{s.}{ordem; sequência; sucessão; subsequência; sequência simples; ordem de prioridade}
  \end{phonetics}
\end{entry}

\begin{entry}{顺畅}{9,8}{⾴、⽥}
  \begin{phonetics}{顺畅}{shun4chang4}
    \definition{adj.}{liso e sem obstáculos | fluente}
  \end{phonetics}
\end{entry}

\begin{entry}{顺便}{9,9}{⾴、⼈}
  \begin{phonetics}{顺便}{shun4bian4}
    \definition{adv.}{convenientemente | de passagem | sem muito esforço extra}
  \end{phonetics}
\end{entry}

\begin{entry}{顺叙}{9,9}{⾴、⼜}
  \begin{phonetics}{顺叙}{shun4xu4}
    \definition{s.}{narrativa cronológica}
  \end{phonetics}
\end{entry}

\begin{entry}{顺眼}{9,11}{⾴、⽬}
  \begin{phonetics}{顺眼}{shun4yan3}
    \definition{adj.}{agradável aos olhos}
  \end{phonetics}
\end{entry}

\begin{entry}{顺境}{9,14}{⾴、⼟}
  \begin{phonetics}{顺境}{shun4jing4}
    \definition{s.}{circunstâncias favoráveis}
  \end{phonetics}
\end{entry}

\begin{entry}{顺嘴}{9,16}{⾴、⼝}
  \begin{phonetics}{顺嘴}{shun4zui3}
    \definition{v.}{deixar escapar (sem pensar) | ler suavemente (texto) | adequar-se  ao gosto (comida)}
  \end{phonetics}
\end{entry}

\begin{entry}{飒飒}{9,9}{⾵、⾵}
  \begin{phonetics}{飒飒}{sa4sa4}
    \definition{s.}{o farfalhar | sussurro | murmúrio (do vento nas árvores, o mar, etc.)}
  \end{phonetics}
\end{entry}

\begin{entry}{食物}{9,8}{⾷、⽜}
  \begin{phonetics}{食物}{shi2wu4}[][HSK 2]
    \definition[种]{s.}{comida; alimentos; comestíveis}
  \end{phonetics}
\end{entry}

\begin{entry}{食品}{9,9}{⾷、⼝}
  \begin{phonetics}{食品}{shi2 pin3}[][HSK 3]
    \definition[种]{s.}{comida; gêneros alimentícios; provisões; alimentos vendidos em lojas que passaram por algum processamento}
  \end{phonetics}
\end{entry}

\begin{entry}{食堂}{9,11}{⾷、⼟}
  \begin{phonetics}{食堂}{shi2 tang2}[][HSK 4]
    \definition[个,间]{s.}{cantina; refeitório}
  \end{phonetics}
\end{entry}

\begin{entry}{饺子}{9,3}{⾷、⼦}
  \begin{phonetics}{饺子}{jiao3zi5}[][HSK 2]
    \definition[个,盘,碗,锅]{s.}{jiaozi; bolinho chinês; bolinho de massa}
  \end{phonetics}
\end{entry}

\begin{entry}{饼}{9}{⾷}
  \begin{phonetics}{饼}{bing3}[][HSK 5]
    \definition[张]{s.}{um bolo redondo e plano; massa assada ou cozida no vapor | algo que tem o formato de um bolo; semelhante a uma torta}
  \end{phonetics}
\end{entry}

\begin{entry}{饼干}{9,3}{⾷、⼲}
  \begin{phonetics}{饼干}{bing3gan1}[][HSK 5]
    \definition[块,片,包,盒,袋]{s.}{biscoito; bolacha; \emph{cookie}; alimentos, pedaços pequenos e finos cozidos em farinha com açúcar, ovos, leite, etc.}
  \end{phonetics}
\end{entry}

\begin{entry}{首}{9}{⾸}[Kangxi 185]
  \begin{phonetics}{首}{shou3}[][HSK 4]
    \definition*{s.}{sobrenome Shou}
    \definition{adj.}{primeiro}
    \definition{adv.}{inicialmente; como o primeiro; em primeiro lugar}
    \definition{clas.}{para canções e poemas}
    \definition{s.}{cabeça | cabeça; chefe; líder | capital (cidade)}
    \definition{v.}{apresentar acusações contra alguém}
  \end{phonetics}
\end{entry}

\begin{entry}{首先}{9,6}{⾸、⼉}
  \begin{phonetics}{首先}{shou3xian1}[][HSK 3]
    \definition{adv.}{primeiramente; antes de todos os outros}
    \definition{conj.}{acima de tudo; primeiramente; em primeiro lugar}
  \end{phonetics}
\end{entry}

\begin{entry}{首相}{9,9}{⾸、⽬}
  \begin{phonetics}{首相}{shou3xiang4}
    \definition*{s.}{Primeiro-Ministro (Japão, UK, etc.)}
  \end{phonetics}
\end{entry}

\begin{entry}{首席执行官}{9,10,6,6,8}{⾸、⼱、⼿、⾏、⼧}
  \begin{phonetics}{首席执行官}{shou3xi2 zhi2xing2 guan1}
    \definition{s.}{\emph{chief executive officer}, CEO}
  \end{phonetics}
\end{entry}

\begin{entry}{首都}{9,10}{⾸、⾢}
  \begin{phonetics}{首都}{shou3du1}[][HSK 3]
    \definition[个,座]{s.}{capital (cidade); a sede do mais alto poder político do país e o centro político do país}
  \end{phonetics}
\end{entry}

\begin{entry}{香}{9}{⾹}[Kangxi 186]
  \begin{phonetics}{香}{xiang1}[][HSK 3]
    \definition*{s.}{sobrenome Xiang}
    \definition{adj.}{aromático; perfumado; fragrante; cheiroso; oposto a 臭 | saboroso; saboroso; delicioso; apetitoso | com gosto; com bom apetite | (sono) profundo; dormir confortavelmente e tranquilamente | popular; valorizado; apreciado}
    \definition[根,炷]{s.}{especiaria; perfume; fragrância; aromatizante; substância com aroma intenso | incenso; bastão de incenso; tiras finas feitas de serragem e especiarias, queimadas em rituais para honrar os antepassados ou deuses e budas, e também usadas para afastar odores desagradáveis ou mosquitos| antigamente, referia-se a coisas relacionadas com mulheres ou mulheres}
  \seealsoref{臭}{chou4}
  \end{phonetics}
\end{entry}

\begin{entry}{香气}{9,4}{⾹、⽓}
  \begin{phonetics}{香气}{xiang1qi4}
    \definition{s.}{fragrância | aroma | incenso}
  \end{phonetics}
\end{entry}

\begin{entry}{香皂}{9,7}{⾹、⽩}
  \begin{phonetics}{香皂}{xiang1zao4}
    \definition{s.}{sabonete | sabonete perfumado}
  \end{phonetics}
\end{entry}

\begin{entry}{香肠}{9,7}{⾹、⾁}
  \begin{phonetics}{香肠}{xiang1chang2}[][HSK 5]
    \definition[根]{s.}{salsicha; linguiça; alimento feito com intestino de porco, recheado com carne picada e temperos}
  \end{phonetics}
\end{entry}

\begin{entry}{香味}{9,8}{⾹、⼝}
  \begin{phonetics}{香味}{xiang1wei4}
    \definition[股]{s.}{fragrância | cheiro doce}
  \end{phonetics}
\end{entry}

\begin{entry}{香波}{9,8}{⾹、⽔}
  \begin{phonetics}{香波}{xiang1bo1}
    \definition{s.}{xampu}
  \end{phonetics}
\end{entry}

\begin{entry}{香炉}{9,8}{⾹、⽕}
  \begin{phonetics}{香炉}{xiang1lu2}
    \definition{s.}{incensário (para queimar incenso) | queimador de incenso | insensório, turíbulo}
  \end{phonetics}
\end{entry}

\begin{entry}{香烟}{9,10}{⾹、⽕}
  \begin{phonetics}{香烟}{xiang1yan1}
    \definition[支,条]{s.}{cigarro | fumaça de incenso queimado}
  \end{phonetics}
\end{entry}

\begin{entry}{香艳}{9,10}{⾹、⾊}
  \begin{phonetics}{香艳}{xiang1yan4}
    \definition{adj.}{atraente | erótico | romântico}
  \end{phonetics}
\end{entry}

\begin{entry}{香港}{9,12}{⾹、⽔}
  \begin{phonetics}{香港}{xiang1gang3}
    \definition*{s.}{Hong Kong}
  \seealsoref{香港岛}{xiang1gang3 dao3}
  \end{phonetics}
\end{entry}

\begin{entry}{香港岛}{9,12,7}{⾹、⽔、⼭}
  \begin{phonetics}{香港岛}{xiang1gang3 dao3}
    \definition*{s.}{Ilha de Hong Kong}
  \seealsoref{香港}{xiang1gang3}
  \end{phonetics}
\end{entry}

\begin{entry}{香槟酒}{9,14,10}{⾹、⽊、⾣}
  \begin{phonetics}{香槟酒}{xiang1bin1jiu3}
    \definition[杯]{s.}{(empréstimo linguístico) \emph{champagne}}
  \end{phonetics}
\end{entry}

\begin{entry}{香蕈}{9,15}{⾹、⾋}
  \begin{phonetics}{香蕈}{xiang1xun4}
    \definition{s.}{\emph{shiitake}, cogumelo comestível}
  \end{phonetics}
\end{entry}

\begin{entry}{香蕉}{9,15}{⾹、⾋}
  \begin{phonetics}{香蕉}{xiang1jiao1}[][HSK 3]
    \definition[枝,根,个,把,串,束,弓]{s.}{banana}
  \end{phonetics}
\end{entry}

\begin{entry}{骂}{9}{⾺}
  \begin{phonetics}{骂}{ma4}[][HSK 5]
    \definition{v.}{abusar; xingar; insultar; insultar alguém com palavras grosseiras ou maliciosas | repreender; censurar; condenar}
  \end{phonetics}
\end{entry}

\begin{entry}{骂名}{9,6}{⾺、⼝}
  \begin{phonetics}{骂名}{ma4ming2}
    \definition{s.}{infâmia}
  \end{phonetics}
\end{entry}

\begin{entry}{骂街}{9,12}{⾺、⾏}
  \begin{phonetics}{骂街}{ma4jie1}
    \definition{v.}{gritar abusos na rua}
  \end{phonetics}
\end{entry}

\begin{entry}{骆驼}{9,8}{⾺、⾺}
  \begin{phonetics}{骆驼}{luo4tuo5}
    \definition[峰,匹,头]{s.}{camelo | (coloquial) cabeça-dura, idiota}
  \end{phonetics}
\end{entry}

\begin{entry}{骨}{9}{⾻}[Kangxi 188]
  \begin{phonetics}{骨}{gu3}
    \definition*{s.}{sobrenome Gu}
    \definition[根,块]{s.}{osso | esqueleto; estrutura | caráter; espírito | cadáver; corpo}
  \end{phonetics}
\end{entry}

\begin{entry}{骨头}{9,5}{⾻、⼤}
  \begin{phonetics}{骨头}{gu3tou5}[][HSK 4]
    \definition[根,块]{s.}{osso; tecidos mais duros no corpo de uma pessoa ou de alguns animais que sustentam o corpo ou protegem os órgãos do corpo | caráter de uma pessoa; refere-se à qualidade do caráter de uma pessoa}
  \end{phonetics}
\end{entry}

\begin{entry}{鬼}{9}{⿁}[Kangxi 194]
  \begin{phonetics}{鬼}{gui3}[][HSK 5]
    \definition*{s.}{sobrenome Gui}
    \definition*{s.}{Gui, uma das mansões lunares | Gui, a vigésima terceira das vinte e oito constelações em que a esfera celeste foi dividida, consistindo de quatro estrelas em Câncer}
    \definition{adj.}{evasivo; furtivo; sub-reptício; ardiloso; enganoso, malicioso; obscuro | terrível; ruim; severo; vil | esperto; astuto; inteligente}
    \definition{s.}{espírito; fantasma; aparição; refere-se à alma de uma pessoa após a morte | usado para formar um termo de abuso para caráter ignóbil; refere-se a pessoas que têm maus hábitos ou cujo comportamento é repugnante | companheiro; pessoa que é considerada divertida}
  \end{phonetics}
\end{entry}

\begin{entry}{鬼火}{9,4}{⿁、⽕}
  \begin{phonetics}{鬼火}{gui3huo3}
    \definition{s.}{fogo-fátuo | boitatá | fogo corredor | fogo de santelmo}
  \end{phonetics}
\end{entry}

\begin{entry}{鬼怪}{9,8}{⿁、⼼}
  \begin{phonetics}{鬼怪}{gui3guai4}
    \definition{s.}{\emph{hobgoblin} | bicho-papão | fantasma}
  \end{phonetics}
\end{entry}

%%%%% EOF %%%%%


%%%
%%% 10画
%%%

\section*{10画}\addcontentsline{toc}{section}{10画}

\begin{entry}{乘客}{10,9}
  \begin{phonetics}{乘客}{cheng2ke4}
    \definition{s.}{passageiro}
  \end{phonetics}
\end{entry}

\begin{entry}{乘客数}{10,9,13}
  \begin{phonetics}{乘客数}{cheng2ke4 shu4}
    \definition{s.}{número de passageiros}
  \end{phonetics}
\end{entry}

\begin{entry}{倂}{10}[Radical 亻]
  \begin{phonetics}{倂}{bing4}
    \variantof{并}
  \end{phonetics}
\end{entry}

\begin{entry}{倒}{10}[Radical 人]
  \begin{phonetics}{倒}{dao3}[][HSK 2]
    \definition{v.}{cair no chão | deitar-se no chão | colapsar | ir à falência}
  \end{phonetics}
  \begin{phonetics}{倒}{dao4}[][HSK 2]
    \definition{adv.}{ao contrário da expectativa | ao contrário}
    \definition{v.}{inverter | colocar de cabeça para baixo ou de frente para trás | derramar | tombar}
  \end{phonetics}
\end{entry}

\begin{entry}{倒地}{10,6}
  \begin{phonetics}{倒地}{dao3di4}
    \definition{v.}{cair no chão}
  \end{phonetics}
\end{entry}

\begin{entry}{倒血霉}{10,6,15}
  \begin{phonetics}{倒血霉}{dao3xue4mei2}
    \definition{v.}{ter muito azar (versão mais forte de 倒霉)}
  \seealsoref{倒霉}{dao3mei2}
  \end{phonetics}
\end{entry}

\begin{entry}{倒楣}{10,13}
  \begin{phonetics}{倒楣}{dao3mei2}
    \variantof{倒霉}
  \end{phonetics}
\end{entry}

\begin{entry}{倒霉}{10,15}
  \begin{phonetics}{倒霉}{dao3mei2}
    \definition{adj.}{azarado}
    \definition{s.}{azar | má sorte}
    \definition{v.}{estar sem sorte | ter azar}
  \seealsoref{倒血霉}{dao3xue4mei2}
  \end{phonetics}
\end{entry}

\begin{entry}{倘使}{10,8}
  \begin{phonetics}{倘使}{tang3shi3}
    \definition{conj.}{se | supondo que | no caso}
  \end{phonetics}
\end{entry}

\begin{entry}{倘或}{10,8}
  \begin{phonetics}{倘或}{tang3huo4}
    \definition{conj.}{se | supondo que | no caso}
  \end{phonetics}
\end{entry}

\begin{entry}{倘若}{10,8}
  \begin{phonetics}{倘若}{tang3ruo4}
    \definition{conj.}{se | supondo que | no caso}
  \end{phonetics}
\end{entry}

\begin{entry}{借}{10}[Radical 人]
  \begin{phonetics}{借}{jie4}[][HSK 2]
    \definition{adv.}{por meio de}
    \definition{v.}{pedir emprestado | emprestar | aproveitar (uma oportunidade)}
  \end{phonetics}
\end{entry}

\begin{entry}{借书证}{10,4,7}
  \begin{phonetics}{借书证}{jie4shu1zheng4}
    \definition{s.}{cartão de biblioteca | (literalmente) cartão para pedir emprestado livros}
  \end{phonetics}
\end{entry}

\begin{entry}{倾城}{10,9}
  \begin{phonetics}{倾城}{qing1cheng2}
    \definition{adj.}{sedutora (mulher)}
    \definition{adv.}{de todo o lugar | vindo de todos os lugares}
    \definition{v.}{arruinar e derrubar o estado}
  \end{phonetics}
\end{entry}

\begin{entry}{健身}{10,7}
  \begin{phonetics}{健身}{jian4shen1}
    \definition{s.}{exercício físico | \emph{fitness}}
    \definition{v.}{exercitar-se | manter a forma}
  \end{phonetics}
\end{entry}

\begin{entry}{健康}{10,11}
  \begin{phonetics}{健康}{jian4kang1}[][HSK 2]
    \definition{adj.}{em forma | saudável | curado}
    \definition{s.}{saúde | físico}
  \end{phonetics}
\end{entry}

\begin{entry}{兼}{10}[Radical 八]
  \begin{phonetics}{兼}{jian1}
    \definition{conj.}{e (ocupando dois ou mais cargos (oficiais) ao mesmo tempo)}
  \end{phonetics}
\end{entry}

\begin{entry}{准}{10}[Radical 冫]
  \begin{phonetics}{准}{zhun3}
    \definition{adv.}{certamente | de acordo com | à luz de}
    \definition{v.}{permitir | conceder}
  \end{phonetics}
\end{entry}

\begin{entry}{准备}{10,8}
  \begin{phonetics}{准备}{zhun3bei4}[][HSK 1]
    \definition{v.}{preparar | ficar pronto | pretender | planejar}
  \end{phonetics}
\end{entry}

\begin{entry}{准确}{10,12}
  \begin{phonetics}{准确}{zhun3que4}[][HSK 2]
    \definition{adj.}{exato | preciso | acurado}
  \end{phonetics}
\end{entry}

\begin{entry}{凉}{10}[Radical 冫]
  \begin{phonetics}{凉}{liang2}
    \definition{adj.}{frio | legal}
  \end{phonetics}
  \begin{phonetics}{凉}{liang4}
    \definition{v.}{esfriar | tornar ou tornar-se frio | deixar esfriar pelo ar}
  \end{phonetics}
\end{entry}

\begin{entry}{凉快}{10,7}
  \begin{phonetics}{凉快}{liang2kuai5}[][HSK 2]
    \definition{adj.}{agradável e frio | agradavelmente fresco}
  \end{phonetics}
\end{entry}

\begin{entry}{凉鞋}{10,15}
  \begin{phonetics}{凉鞋}{liang2xie2}
    \definition{s.}{sandália | alpargata | alpercata | alparca}
  \end{phonetics}
\end{entry}

\begin{entry}{原木}{10,4}
  \begin{phonetics}{原木}{yuan2mu4}
    \definition{s.}{registro | \emph{logs}}
  \end{phonetics}
\end{entry}

\begin{entry}{原因}{10,6}
  \begin{phonetics}{原因}{yuan2yin1}[][HSK 2]
    \definition[个]{s.}{causa | razão | motivo}
  \end{phonetics}
\end{entry}

\begin{entry}{原色}{10,6}
  \begin{phonetics}{原色}{yuan2se4}
    \definition{s.}{cor primária}
  \end{phonetics}
\end{entry}

\begin{entry}{原来}{10,7}
  \begin{phonetics}{原来}{yuan2lai2}[][HSK 2]
    \definition{adv.}{originalmente | como se vê | na verdade}
    \definition{v.}{vir a ser}
  \end{phonetics}
\end{entry}

\begin{entry}{原理}{10,11}
  \begin{phonetics}{原理}{yuan2li3}
    \definition{s.}{princípio | teoria}
  \end{phonetics}
\end{entry}

\begin{entry}{哥}{10}[Radical 口]
  \begin{phonetics}{哥}{ge1}[][HSK 1]
    \definition{s.}{irmão mais velho}
  \seealsoref{哥哥}{ge1ge5}
  \end{phonetics}
\end{entry}

\begin{entry}{哥们}{10,5}
  \begin{phonetics}{哥们}{ge1men5}
    \definition{expr.}{\emph{Brothers!}}
    \definition{s.}{(coloquial) cara | irmão (forma diminuta de tratamento entre homens)}
  \end{phonetics}
\end{entry}

\begin{entry}{哥哥}{10,10}
  \begin{phonetics}{哥哥}{ge1ge5}[][HSK 1]
    \definition[个,位]{s.}{irmão mais velho}
  \end{phonetics}
\end{entry}

\begin{entry}{哥斯拉}{10,12,8}
  \begin{phonetics}{哥斯拉}{ge1si1la1}
    \definition*{s.}{Godzilla}
  \seealsoref{酷斯拉}{ku4si1la1}
  \end{phonetics}
\end{entry}

\begin{entry}{哦}{10}[Radical ⼝]
  \begin{phonetics}{哦}{e2}
    \definition{v.}{entoar cântico}
  \end{phonetics}
  \begin{phonetics}{哦}{o2}
    \definition{interj.}{Oh! (indicando dúvida ou surpresa)}
  \end{phonetics}
  \begin{phonetics}{哦}{o4}
    \definition{interj.}{Oh! (indicando que acabou de aprender algo)}
  \end{phonetics}
  \begin{phonetics}{哦}{o5}
    \definition{part.}{final da frase que transmite informalidade, calor, simpatia ou intimidade; também pode indicar que alguém está declarando um fato de que a outra pessoa não está ciente}
  \end{phonetics}
\end{entry}

\begin{entry}{哭}{10}[Radical 口]
  \begin{phonetics}{哭}{ku1}[][HSK 2]
    \definition{v.}{chorar}
  \end{phonetics}
\end{entry}

\begin{entry}{哭墙}{10,14}
  \begin{phonetics}{哭墙}{ku1qiang2}
    \definition*{s.}{Muro das Lamentações (Jerusalém)}
  \end{phonetics}
\end{entry}

\begin{entry}{哮喘}{10,12}
  \begin{phonetics}{哮喘}{xiao4chuan3}
    \definition{s.}{asma}
  \end{phonetics}
\end{entry}

\begin{entry}{哲理}{10,11}
  \begin{phonetics}{哲理}{zhe2li3}
    \definition{s.}{filosofia | teoria filosófica}
  \end{phonetics}
\end{entry}

\begin{entry}{唇}{10}[Radical ⼝]
  \begin{phonetics}{唇}{chun2}
    \definition{s.}{lábios}
  \end{phonetics}
\end{entry}

\begin{entry}{唐人街}{10,2,12}
  \begin{phonetics}{唐人街}{tang2ren2 jie1}
    \definition*{s.}{Bairro Chinês | \emph{Chinatown}}
  \seealsoref{中国城}{zhong1guo2cheng2}
  \end{phonetics}
\end{entry}

\begin{entry}{啊}{10}[Radical 口]
  \begin{phonetics}{啊}{a1}[][HSK 2]
    \definition{interj.}{Ah! | Oh! | interjeição de surpresa}
  \end{phonetics}
  \begin{phonetics}{啊}{a2}[][HSK 2]
    \definition{interj.}{Eh? | Que? | interjeição expressando dúvida ou exigindo resposta}
  \end{phonetics}
  \begin{phonetics}{啊}{a3}[][HSK 2]
    \definition{interj.}{Eh? | Meu! | E aí? | Que? | interjeição de surpresa ou dúvida}
  \end{phonetics}
  \begin{phonetics}{啊}{a4}[][HSK 2]
    \definition{interj.}{Ah! | OK! | Oh, é você! | Hum! | expressão de reconhecimento | interjeição de acordo}
  \end{phonetics}
  \begin{phonetics}{啊}{a5}[][HSK 2]
    \definition{adv.}{assim por diante}
    \definition{part.}{no final de sentença para expressar admiração | no final de sentença mostrando afirmação, aprovação, urgência, aconselhamento, etc. | no final de sentença para indicar uma pergunta | para pausar ligeiramente uma frase, chamando a atenção para as palavras seguintes | após cada um dos itens listados}
  \end{phonetics}
\end{entry}

\begin{entry}{啊呀}{10,7}
  \begin{phonetics}{啊呀}{a1ya1}
    \definition{interj.}{Oh meu Deus! | interjeição de surpresa}
  \end{phonetics}
\end{entry}

\begin{entry}{啊哟}{10,9}
  \begin{phonetics}{啊哟}{a1yo5}
    \definition{interj.}{Meu Deus! | Oh! | Ai! | interjeição de surpresa ou dor}
  \end{phonetics}
\end{entry}

\begin{entry}{埋伏}{10,6}
  \begin{phonetics}{埋伏}{mai2fu2}
    \definition{s.}{emboscada}
    \definition{v.}{emboscar}
  \end{phonetics}
\end{entry}

\begin{entry}{夏天}{10,4}
  \begin{phonetics}{夏天}{xia4 tian1}[][HSK 2]
    \definition[个]{s.}{verão}
  \end{phonetics}
\end{entry}

\begin{entry}{夏日}{10,4}
  \begin{phonetics}{夏日}{xia4ri4}
    \definition{s.}{horário de verão}
  \end{phonetics}
\end{entry}

\begin{entry}{套}{10}[Radical 大]
  \begin{phonetics}{套}{tao4}[][HSK 2]
    \definition{clas.}{para conjuntos, coleções}
    \definition{s.}{cobertura | fórmula | laço de corda}
    \definition{v.}{cobrir | envolver | intercalar | sobrepor}
  \end{phonetics}
\end{entry}

\begin{entry}{套问}{10,6}
  \begin{phonetics}{套问}{tao4wen4}
    \definition{s.}{retórica}
    \definition{v.}{descobrir por meio de questionamento indireto diplomático}
  \end{phonetics}
\end{entry}

\begin{entry}{孬}{10}[Radical 子]
  \begin{phonetics}{孬}{nao1}
    \definition{adj.}{(dialeto) não (é) bom (contração de 不+好)}
  \end{phonetics}
\end{entry}

\begin{entry}{害}{10}[Radical 宀]
  \begin{phonetics}{害}{hai4}
    \definition{s.}{dano | mal | calamidade}
    \definition{v.}{causar danos a | causar problemas para}
  \end{phonetics}
\end{entry}

\begin{entry}{害怕}{10,8}
  \begin{phonetics}{害怕}{hai4pa4}
    \definition{v.}{ter medo | ficar com medo | temer}
  \end{phonetics}
\end{entry}

\begin{entry}{害羞}{10,10}
  \begin{phonetics}{害羞}{hai4xiu1}
    \definition{adj.}{tímido | envergonhado}
  \end{phonetics}
\end{entry}

\begin{entry}{家}{10}[Radical 宀]
  \begin{phonetics}{家}{jia1}[][HSK 1]
    \definition{clas.}{para famílias ou empresas}
    \definition{pron.}{(educado) meu (irmã, tio, etc.)}
    \definition[个]{s.}{casa | família}
    \definition{suf.}{sufixo substantivo para designar um especialista em alguma atividade, como um músico ou revolucionário, para designar uma profissão como em -eiro, -ista}
  \end{phonetics}
\end{entry}

\begin{entry}{家人}{10,2}
  \begin{phonetics}{家人}{jia1ren2}[][HSK 1]
    \definition{s.}{(a) família | membro da família}
  \end{phonetics}
\end{entry}

\begin{entry}{家乡}{10,3}
  \begin{phonetics}{家乡}{jia1xiang1}
    \definition[个]{s.}{terra natal | cidade natal}
  \end{phonetics}
\end{entry}

\begin{entry}{家长}{10,4}
  \begin{phonetics}{家长}{jia1 zhang3}[][HSK 2]
    \definition[位,名,个]{s.}{pais | patriarca | guardião}
  \end{phonetics}
\end{entry}

\begin{entry}{家伙}{10,6}
  \begin{phonetics}{家伙}{jia1huo5}
    \definition{s.}{prato, implemento ou móvel doméstico | animal doméstico | (coloquial) o cara | indivíduo | arma}
  \end{phonetics}
\end{entry}

\begin{entry}{家里}{10,7}
  \begin{phonetics}{家里}{jia1 li3}[][HSK 1]
    \definition{adv.}{em casa}
  \end{phonetics}
\end{entry}

\begin{entry}{家具}{10,8}
  \begin{phonetics}{家具}{jia1ju4}
    \definition[件,套]{s.}{móveis | mobiliário}
  \end{phonetics}
\end{entry}

\begin{entry}{家庭}{10,9}
  \begin{phonetics}{家庭}{jia1ting2}[][HSK 2]
    \definition[个,户]{s.}{família}
  \end{phonetics}
\end{entry}

\begin{entry}{家俱}{10,10}
  \begin{phonetics}{家俱}{jia1ju4}
    \variantof{家具}
  \end{phonetics}
\end{entry}

\begin{entry}{容易}{10,8}
  \begin{phonetics}{容易}{rong2yi4}
    \definition{adj.}{fácil | responsável (por) | provável}
  \end{phonetics}
\end{entry}

\begin{entry}{容貌}{10,14}
  \begin{phonetics}{容貌}{rong2mao4}
    \definition{s.}{aparência | aspecto | características}
  \end{phonetics}
\end{entry}

\begin{entry}{宽影片}{10,15,4}
  \begin{phonetics}{宽影片}{kuan1ying3pian4}
    \definition{s.}{filme \emph{widescreen}}
  \end{phonetics}
\end{entry}

\begin{entry}{宾馆}{10,11}
  \begin{phonetics}{宾馆}{bin1guan3}
    \definition[个,家]{s.}{casa de hóspedes | hotel}
  \end{phonetics}
\end{entry}

\begin{entry}{射}{10}[Radical 寸]
  \begin{phonetics}{射}{she4}
    \definition{v.}{atirar | lançar}
  \end{phonetics}
\end{entry}

\begin{entry}{展示}{10,5}
  \begin{phonetics}{展示}{zhan3shi4}
    \definition{v.}{revelar | mostrar | exibir}
  \end{phonetics}
\end{entry}

\begin{entry}{席卷}{10,8}
  \begin{phonetics}{席卷}{xi2juan3}
    \definition{v.}{engolfar | varrer | levar tudo para fora}
  \end{phonetics}
\end{entry}

\begin{entry}{座}{10}[Radical 广]
  \begin{phonetics}{座}{zuo4}[][HSK 2]
    \definition{clas.}{frequentemente usado para objetos maiores ou fixos}
    \definition{s.}{assento | lugar | base | suporte | pedestal | constelação}
  \end{phonetics}
\end{entry}

\begin{entry}{座子}{10,3}
  \begin{phonetics}{座子}{zuo4zi5}
    \definition{s.}{soquete | pedestal | sela}
  \end{phonetics}
\end{entry}

\begin{entry}{座位}{10,7}
  \begin{phonetics}{座位}{zuo4wei4}[][HSK 2]
    \definition[个]{s.}{assento | lugar}
  \end{phonetics}
\end{entry}

\begin{entry}{座标}{10,9}
  \begin{phonetics}{座标}{zuo4biao1}
    \variantof{坐标}
  \end{phonetics}
\end{entry}

\begin{entry}{徒手}{10,4}
  \begin{phonetics}{徒手}{tu2shou3}
    \definition{adj.}{com as mãos vazias | desarmado | mão livre (desenho) | lutando mão-a-mão}
  \end{phonetics}
\end{entry}

\begin{entry}{恋爱}{10,10}
  \begin{phonetics}{恋爱}{lian4'ai4}
    \definition[个,场]{s.}{amor (romântico)}
    \definition{v.}{sentir-se profundamente apegado a}
  \end{phonetics}
\end{entry}

\begin{entry}{恐龙}{10,5}
  \begin{phonetics}{恐龙}{kong3long2}
    \definition[头,只]{s.}{dinossauro}
  \end{phonetics}
\end{entry}

\begin{entry}{恐怕}{10,8}
  \begin{phonetics}{恐怕}{kong3pa4}
    \definition{adv.}{talvez | possivelmente | provavelmente | (em sentido não tão bom)}
    \definition{v.}{temer}
  \end{phonetics}
\end{entry}

\begin{entry}{恐怖主义}{10,8,5,3}
  \begin{phonetics}{恐怖主义}{kong3bu4zhu3yi4}
    \definition{adj.}{terrorista}
    \definition{s.}{terrorismo}
  \end{phonetics}
\end{entry}

\begin{entry}{恩赐}{10,12}
  \begin{phonetics}{恩赐}{en1ci4}
    \definition{s.}{favor | caridade}
    \definition{v.}{conceder (favor, caridade)}
  \end{phonetics}
\end{entry}

\begin{entry}{恶心}{10,4}
  \begin{phonetics}{恶心}{e3xin1}
    \definition{adj.}{nauseante | repugnante}
    \definition{s.}{enjôo | náusea | repugnância}
    \definition{v.}{envergonhar (deliberadamente) | sentir-se doente}
  \end{phonetics}
  \begin{phonetics}{恶心}{e4xin1}
    \definition{s.}{mau hábito | hábito vicioso | vício}
  \end{phonetics}
\end{entry}

\begin{entry}{扇子}{10,3}
  \begin{phonetics}{扇子}{shan4zi5}
    \definition[把]{s.}{leque | abano | abanador}
  \end{phonetics}
\end{entry}

\begin{entry}{拳王}{10,4}
  \begin{phonetics}{拳王}{quan2wang2}
    \definition{s.}{pugilista | boxeador}
  \end{phonetics}
\end{entry}

\begin{entry}{拳法}{10,8}
  \begin{phonetics}{拳法}{quan2fa3}
    \definition{s.}{boxe | luta}
  \end{phonetics}
\end{entry}

\begin{entry}{拿}{10}[Radical 手]
  \begin{phonetics}{拿}{na2}[][HSK 1]
    \definition{part.}{usado da mesma forma que 把: para marcar o seguinte substantivo seguinte como objeto direto}
    \definition{v.}{segurar | tomar | pegar em}
  \end{phonetics}
\end{entry}

\begin{entry}{拿出}{10,5}
  \begin{phonetics}{拿出}{na2 chu1}[][HSK 2]
    \definition{v.}{apresentar (evidências) | prover | apresentar (uma proposta) | colocar para fora | retirar}
  \end{phonetics}
\end{entry}

\begin{entry}{拿到}{10,8}
  \begin{phonetics}{拿到}{na2 dao4}[][HSK 2]
    \definition{v.}{pegar | obter}
  \end{phonetics}
\end{entry}

\begin{entry}{挫折}{10,7}
  \begin{phonetics}{挫折}{cuo4zhe2}
    \definition{s.}{revés | reverso | derrota | frustração | decepção}
    \definition{v.}{frustrar | desencorajar | subjugar}
  \end{phonetics}
\end{entry}

\begin{entry}{捞}{10}[Radical 手]
  \begin{phonetics}{捞}{lao1}
    \definition{v.}{pescar | dragar}
  \end{phonetics}
\end{entry}

\begin{entry}{捡}{10}[Radical 手]
  \begin{phonetics}{捡}{jian3}
    \definition{v.}{apanhar | recolher | coletar}
  \end{phonetics}
\end{entry}

\begin{entry}{换}{10}[Radical 手]
  \begin{phonetics}{换}{huan4}[][HSK 2]
    \definition{v.}{mudar | trocar | substituir | converter (moedas)}
  \end{phonetics}
\end{entry}

\begin{entry}{换钱}{10,10}
  \begin{phonetics}{换钱}{huan4qian2}
    \definition{v.+compl.}{trocar dinheiro (em pequenas valores ou em outra moeda) | trocar (mercadorias) por dinheiro | vender}
  \end{phonetics}
\end{entry}

\begin{entry}{效果}{10,8}
  \begin{phonetics}{效果}{xiao4guo3}
    \definition{s.}{resultado | efeito | eficácia | (teatro/cinema) efeitos sonoros ou visuais}
  \end{phonetics}
\end{entry}

\begin{entry}{旁边}{10,5}
  \begin{phonetics}{旁边}{pang2bian1}[][HSK 1]
    \definition{adv.}{junto a | próximo de | ao lado}
  \end{phonetics}
\end{entry}

\begin{entry}{旅行}{10,6}
  \begin{phonetics}{旅行}{lv3xing2}[][HSK 2]
    \definition{v.}{viajar}
  \end{phonetics}
\end{entry}

\begin{entry}{旅客}{10,9}
  \begin{phonetics}{旅客}{lv3 ke4}[][HSK 2]
    \definition{s.}{viajante | turista}
  \end{phonetics}
\end{entry}

\begin{entry}{旅游}{10,12}
  \begin{phonetics}{旅游}{lv3you2}[][HSK 2]
    \definition[趟,次,个]{s.}{jornada | viagem}
    \definition{v.}{viajar}
  \end{phonetics}
\end{entry}

\begin{entry}{旅程}{10,12}
  \begin{phonetics}{旅程}{lv3cheng2}
    \definition{s.}{jornada | viagem}
  \end{phonetics}
\end{entry}

\begin{entry}{晒干}{10,3}
  \begin{phonetics}{晒干}{shai4gan1}
    \definition{v.}{secar ao sol}
  \end{phonetics}
\end{entry}

\begin{entry}{校}{10}[Radical 木]
  \begin{phonetics}{校}{jiao4}
    \definition{v.}{verificar | comparar | revisar}
  \end{phonetics}
  \begin{phonetics}{校}{xiao4}
    \definition[所]{s.}{oficial militar | escola}
  \end{phonetics}
\end{entry}

\begin{entry}{校长}{10,4}
  \begin{phonetics}{校长}{xiao4zhang3}[][HSK 2]
    \definition[个,位,名]{s.}{diretor de escola | reitor (universidade)}
  \end{phonetics}
\end{entry}

\begin{entry}{校园}{10,7}
  \begin{phonetics}{校园}{xiao4 yuan2}[][HSK 2]
    \definition{s.}{campus}
  \end{phonetics}
\end{entry}

\begin{entry}{校服}{10,8}
  \begin{phonetics}{校服}{xiao4fu2}
    \definition{s.}{uniforme escolar}
  \end{phonetics}
\end{entry}

\begin{entry}{校规}{10,8}
  \begin{phonetics}{校规}{xiao4gui1}
    \definition{s.}{regras e regulamentos escolares}
  \end{phonetics}
\end{entry}

\begin{entry}{校监}{10,10}
  \begin{phonetics}{校监}{xiao4jian1}
    \definition{s.}{diretor | supervisor (de escola)}
  \end{phonetics}
\end{entry}

\begin{entry}{样}{10}[Radical 木]
  \begin{phonetics}{样}{yang4}
    \definition{s.}{aparência | forma | modelo}
  \end{phonetics}
\end{entry}

\begin{entry}{样儿}{10,2}
  \begin{phonetics}{样儿}{yang4r5}
    \definition{s.}{aparência | forma | modelo}
  \seealsoref{样子}{yang4zi5}
  \end{phonetics}
\end{entry}

\begin{entry}{样子}{10,3}
  \begin{phonetics}{样子}{yang4zi5}[][HSK 2]
    \definition{s.}{aparência | forma | modelo}
  \seealsoref{样儿}{yang4r5}
  \end{phonetics}
\end{entry}

\begin{entry}{样品}{10,9}
  \begin{phonetics}{样品}{yang4pin3}
    \definition{s.}{amostra | espécime}
  \end{phonetics}
\end{entry}

\begin{entry}{样样}{10,10}
  \begin{phonetics}{样样}{yang4yang4}
    \definition{adv.}{todos os tipos}
  \end{phonetics}
\end{entry}

\begin{entry}{样章}{10,11}
  \begin{phonetics}{样章}{yang4zhang1}
    \definition{s.}{capítulo de amostra}
  \end{phonetics}
\end{entry}

\begin{entry}{核}{10}[Radical 木]
  \begin{phonetics}{核}{he2}
    \definition{adj.}{nuclear}
    \definition{s.}{poço | pedra | núcleo}
    \definition{v.}{examinar | checar | verificar}
  \end{phonetics}
\end{entry}

\begin{entry}{根本}{10,5}
  \begin{phonetics}{根本}{gen1ben3}
    \definition{adj.}{fundamental | básico}
    \definition{adv.}{simplesmente | absolutamente (não) | de jeito nenhum}
    \definition[个]{s.}{raiz}
  \end{phonetics}
\end{entry}

\begin{entry}{根据}{10,11}
  \begin{phonetics}{根据}{gen1ju4}
    \definition{prep.}{de acordo com}
    \definition[个]{s.}{base | fundação}
  \end{phonetics}
\end{entry}

\begin{entry}{格兰菜}{10,5,11}
  \begin{phonetics}{格兰菜}{ge2lan2cai4}
    \definition{s.}{brócolis chinês | couve chinesa | mostarda}
    \seeref{芥蓝}{gai4lan2}
  \end{phonetics}
\end{entry}

\begin{entry}{格外}{10,5}
  \begin{phonetics}{格外}{ge2wai4}
    \definition{adv.}{especialmente | particularmente | adicionalmente | de outra forma}
  \end{phonetics}
\end{entry}

\begin{entry}{栽}{10}[Radical 木]
  \begin{phonetics}{栽}{zai1}
    \definition{v.}{cultivar | plantar}
  \end{phonetics}
\end{entry}

\begin{entry}{栽种}{10,9}
  \begin{phonetics}{栽种}{zai1zhong4}
    \definition{v.}{plantar}
  \end{phonetics}
\end{entry}

\begin{entry}{栽倒}{10,10}
  \begin{phonetics}{栽倒}{zai1dao3}
    \definition{v.}{cair | sofrer uma queda}
  \end{phonetics}
\end{entry}

\begin{entry}{栽赃}{10,10}
  \begin{phonetics}{栽赃}{zai1zang1}
    \definition{v.}{enquadrar alguém (plantar provas nele)}
  \end{phonetics}
\end{entry}

\begin{entry}{栽培}{10,11}
  \begin{phonetics}{栽培}{zai1pei2}
    \definition{v.}{cultivar | educar | patrocinar | treinar}
  \end{phonetics}
\end{entry}

\begin{entry}{栽培种}{10,11,9}
  \begin{phonetics}{栽培种}{zai1pei2 zhong3}
    \definition{s.}{espécies cultivadas}
  \end{phonetics}
\end{entry}

\begin{entry}{栽植}{10,12}
  \begin{phonetics}{栽植}{zai1zhi2}
    \definition{v.}{plantar | transplantar}
  \end{phonetics}
\end{entry}

\begin{entry}{桃}{10}[Radical 木]
  \begin{phonetics}{桃}{tao2}
    \definition{s.}{pêssego}
  \end{phonetics}
\end{entry}

\begin{entry}{桌}{10}[Radical 木]
  \begin{phonetics}{桌}{zhuo1}
    \definition{clas.}{para mesas de convidados em um banquete etc.}
    \definition{s.}{mesa}
  \end{phonetics}
\end{entry}

\begin{entry}{桌子}{10,3}
  \begin{phonetics}{桌子}{zhuo1zi5}[][HSK 1]
    \definition[张,套]{s.}{mesa}
  \end{phonetics}
\end{entry}

\begin{entry}{桌布}{10,5}
  \begin{phonetics}{桌布}{zhuo1bu4}
    \definition[条,块,张]{s.}{(computação) plano de fundo da área de trabalho | toalha de mesa | papel de parede}
  \end{phonetics}
\end{entry}

\begin{entry}{桌机}{10,6}
  \begin{phonetics}{桌机}{zhuo1ji1}
    \definition{s.}{computador \emph{desktop}}
  \end{phonetics}
\end{entry}

\begin{entry}{桌灯}{10,6}
  \begin{phonetics}{桌灯}{zhuo1deng1}
    \definition{s.}{luminária | lâmpada de mesa}
  \end{phonetics}
\end{entry}

\begin{entry}{桌面}{10,9}
  \begin{phonetics}{桌面}{zhuo1mian4}
    \definition{s.}{área de trabalho | mesa}
  \end{phonetics}
\end{entry}

\begin{entry}{桌球}{10,11}
  \begin{phonetics}{桌球}{zhuo1qiu2}
    \definition{s.}{bilhar | sinuca | mesa de ping-pong}
  \end{phonetics}
\end{entry}

\begin{entry}{桌游}{10,12}
  \begin{phonetics}{桌游}{zhuo1you2}
    \definition{s.}{jogo de tabuleiro}
  \end{phonetics}
\end{entry}

\begin{entry}{桑}{10}[Radical 木]
  \begin{phonetics}{桑}{sang1}
    \definition*{s.}{sobrenome Sang}
    \definition{s.}{amoreira}
  \end{phonetics}
\end{entry}

\begin{entry}{桑巴舞}{10,4,14}
  \begin{phonetics}{桑巴舞}{sang1ba1wu3}
    \definition{s.}{samba}
  \end{phonetics}
\end{entry}

\begin{entry}{桑树}{10,9}
  \begin{phonetics}{桑树}{sang1shu4}
    \definition{s.}{amoreira, suas folhas são utilizadas para alimentar bichos-da-seda}
  \end{phonetics}
\end{entry}

\begin{entry}{桥}{10}[Radical 木]
  \begin{phonetics}{桥}{qiao2}
    \definition[座]{s.}{ponte}
  \end{phonetics}
\end{entry}

\begin{entry}{桩}{10}[Radical 木]
  \begin{phonetics}{桩}{zhuang1}
    \definition{clas.}{para eventos, casos, transações, assuntos, etc.}
    \definition{s.}{toco | estaca | pilha}
  \end{phonetics}
\end{entry}

\begin{entry}{欱}{10}[Radical 欠]
  \begin{phonetics}{欱}{he1}
    \variantof{喝}
  \end{phonetics}
\end{entry}

\begin{entry}{氧}{10}[Radical 气]
  \begin{phonetics}{氧}{yang3}
    \definition{s.}{oxigênio}
  \end{phonetics}
\end{entry}

\begin{entry}{流}{10}[Radical 水]
  \begin{phonetics}{流}{liu2}[][HSK 2]
    \definition[名,个]{s.}{fluxo de água | correnteza | córrego | algo que se assemelha a um fluxo de água | corrente | fluxo | classe | grau | taxa (de variação)}
    \definition{v.}{fluir
deriva; mover; vagar
espalhar
degenerar; mudar para pior
enviar para o exílio; banir}
  \end{phonetics}
\end{entry}

\begin{entry}{流水}{10,4}
  \begin{phonetics}{流水}{liu2shui3}
    \definition{s.}{água corrente | (negócio) rotatividade}
  \end{phonetics}
\end{entry}

\begin{entry}{流行}{10,6}
  \begin{phonetics}{流行}{liu2xing2}[][HSK 2]
    \definition{adj.}{(estilo de roupa, música, etc.) popular, na moda}
    \definition{v.}{(doença contagiosa, etc.) espalhar | propagar}
  \end{phonetics}
\end{entry}

\begin{entry}{流利}{10,7}
  \begin{phonetics}{流利}{liu2li4}[][HSK 2]
    \definition{adj.}{fluente (em uma língua)}
  \end{phonetics}
\end{entry}

\begin{entry}{流星}{10,9}
  \begin{phonetics}{流星}{liu2xing1}
    \definition{s.}{meteoro | estrela cadente}
  \end{phonetics}
\end{entry}

\begin{entry}{浙江}{10,6}
  \begin{phonetics}{浙江}{zhe4jiang1}
    \definition*{s.}{Zhejiang}
  \end{phonetics}
\end{entry}

\begin{entry}{浪花}{10,7}
  \begin{phonetics}{浪花}{lang4hua1}
    \definition[朵]{s.}{\emph{spray} | \emph{spray} do oceano | (figurativo) acontecimentos de sua vida}
  \end{phonetics}
\end{entry}

\begin{entry}{浪漫}{10,14}
  \begin{phonetics}{浪漫}{lang4man4}
    \definition{adj.}{romântico}
  \end{phonetics}
\end{entry}

\begin{entry}{浮力}{10,2}
  \begin{phonetics}{浮力}{fu2li4}
    \definition{s.}{flutuabilidade}
  \end{phonetics}
\end{entry}

\begin{entry}{浮图}{10,8}
  \begin{phonetics}{浮图}{fu2tu2}
    \definition*{s.}{Termo alternativo para 佛陀}
    \variantof{浮屠}
  \seealsoref{佛陀}{fo2tuo2}
  \end{phonetics}
\end{entry}

\begin{entry}{浮屠}{10,11}
  \begin{phonetics}{浮屠}{fu2tu2}
    \definition*{s.}{Buda | Templo (Stupa) Budista (transliteração de Pali Thuo)}
  \end{phonetics}
\end{entry}

\begin{entry}{海}{10}[Radical 水]
  \begin{phonetics}{海}{hai3}[][HSK 2]
    \definition*{s.}{sobrenome Hai}
    \definition[个,片]{s.}{mar | oceano}
  \end{phonetics}
\end{entry}

\begin{entry}{海水}{10,4}
  \begin{phonetics}{海水}{hai3shui3}
    \definition{s.}{água do mar}
  \end{phonetics}
\end{entry}

\begin{entry}{海风}{10,4}
  \begin{phonetics}{海风}{hai3feng1}
    \definition{s.}{brisa do mar | vento que vem do mar}
  \end{phonetics}
\end{entry}

\begin{entry}{海边}{10,5}
  \begin{phonetics}{海边}{hai3 bian1}[][HSK 2]
    \definition{s.}{costa marítima | litoral | beira-mar | praia}
  \end{phonetics}
\end{entry}

\begin{entry}{海里}{10,7}
  \begin{phonetics}{海里}{hai3li3}
    \definition{s.}{milha náutica}
  \end{phonetics}
\end{entry}

\begin{entry}{海底}{10,8}
  \begin{phonetics}{海底}{hai3di3}
    \definition{adj.}{submarino}
    \definition{s.}{fundo do mar | solo oceânico | fundo do oceano}
  \end{phonetics}
\end{entry}

\begin{entry}{海鸥}{10,9}
  \begin{phonetics}{海鸥}{hai3'ou1}
    \definition{s.}{gaivota}
  \end{phonetics}
\end{entry}

\begin{entry}{海浪}{10,10}
  \begin{phonetics}{海浪}{hai3lang4}
    \definition{s.}{ondas do mar}
  \end{phonetics}
\end{entry}

\begin{entry}{海绵}{10,11}
  \begin{phonetics}{海绵}{hai3mian2}
    \definition{s.}{(zoologia) esponja do mar | esponja (feita de poliéster ou celulose, etc.) | espuma de borracha}
  \end{phonetics}
\end{entry}

\begin{entry}{海棠}{10,12}
  \begin{phonetics}{海棠}{hai3tang2}
    \definition{s.}{begônia}
  \end{phonetics}
\end{entry}

\begin{entry}{消失}{10,5}
  \begin{phonetics}{消失}{xiao1shi1}
    \definition{v.}{desaparecer | desvanecer}
  \end{phonetics}
\end{entry}

\begin{entry}{消防}{10,6}
  \begin{phonetics}{消防}{xiao1fang2}
    \definition{s.}{combate a incêncios | controle de incêndios}
  \end{phonetics}
\end{entry}

\begin{entry}{消防员}{10,6,7}
  \begin{phonetics}{消防员}{xiao1fang2yuan2}
    \definition{s.}{bombeiro}
  \end{phonetics}
\end{entry}

\begin{entry}{涨价}{10,6}
  \begin{phonetics}{涨价}{zhang3jia4}
    \definition{s.}{aumento de preços}
    \definition{v.+compl.}{avaliar (em valor) | dar preço | aumentar o preço}
  \end{phonetics}
\end{entry}

\begin{entry}{烈士}{10,3}
  \begin{phonetics}{烈士}{lie4shi4}
    \definition{s.}{mártir}
  \end{phonetics}
\end{entry}

\begin{entry}{烟}{10}[Radical 火]
  \begin{phonetics}{烟}{yan1}
    \definition[根]{s.}{cigarro ou cachimbo}
    \definition[竜]{s.}{fumaça | névoa |  vapor}
    \definition{s.}{planta de tabaco}
    \definition{v.}{ficar irritado com a fumaça (olhos)}
  \end{phonetics}
\end{entry}

\begin{entry}{烟火}{10,4}
  \begin{phonetics}{烟火}{yan1huo3}
    \definition{s.}{fogo de artifício}
  \end{phonetics}
\end{entry}

\begin{entry}{烟叶}{10,5}
  \begin{phonetics}{烟叶}{yan1ye4}
    \definition{s.}{folha de tabaco}
  \end{phonetics}
\end{entry}

\begin{entry}{烟头}{10,5}
  \begin{phonetics}{烟头}{yan1tou2}
    \definition[根]{s.}{bituca de cigarro}
  \end{phonetics}
\end{entry}

\begin{entry}{烟囱}{10,7}
  \begin{phonetics}{烟囱}{yan1cong1}
    \definition{s.}{chaminé}
  \end{phonetics}
\end{entry}

\begin{entry}{烟花}{10,7}
  \begin{phonetics}{烟花}{yan1hua1}
    \definition{s.}{fogos de artifício}
  \end{phonetics}
\end{entry}

\begin{entry}{烟雨}{10,8}
  \begin{phonetics}{烟雨}{yan1yu3}
    \definition{s.}{chuvisco | garoa}
  \end{phonetics}
\end{entry}

\begin{entry}{烟草}{10,9}
  \begin{phonetics}{烟草}{yan1cao3}
    \definition{s.}{tabaco}
  \end{phonetics}
\end{entry}

\begin{entry}{烤}{10}[Radical 火]
  \begin{phonetics}{烤}{kao3}
    \definition{v.}{assar | grelhar}
  \end{phonetics}
\end{entry}

\begin{entry}{烤肉}{10,6}
  \begin{phonetics}{烤肉}{kao3rou4}
    \definition{s.}{churrasco}
  \end{phonetics}
\end{entry}

\begin{entry}{烧}{10}[Radical 火]
  \begin{phonetics}{烧}{shao1}
    \definition{s.}{febre}
    \definition{v.}{queimar | cozinhar | cozer | assar | aquecer | ferver (chá, água, etc.) | ter febre | (coloquial) deixar as coisas subirem à cabeça}
  \end{phonetics}
\end{entry}

\begin{entry}{烧烤}{10,10}
  \begin{phonetics}{烧烤}{shao1kao3}
    \definition{s.}{churrasco}
    \definition{v.}{assar}
  \end{phonetics}
\end{entry}

\begin{entry}{热}{10}[Radical 火]
  \begin{phonetics}{热}{re4}[][HSK 1]
    \definition{adj.}{quente (clima) | fervente | ardente | fervoroso}
    \definition{v.}{aquecer | ferver}
  \end{phonetics}
\end{entry}

\begin{entry}{热心}{10,4}
  \begin{phonetics}{热心}{re4xin1}
    \definition{adj.}{entusiasmado | ardente | zeloso}
  \end{phonetics}
\end{entry}

\begin{entry}{热血沸腾}{10,6,8,13}
  \begin{phonetics}{热血沸腾}{re4xue4fei4teng2}
    \definition{expr.}{ferver o sangue | apaixonar-se}
  \end{phonetics}
\end{entry}

\begin{entry}{热泪盈眶}{10,8,9,11}
  \begin{phonetics}{热泪盈眶}{re4lei4ying2kuang4}
    \definition{expr.}{olhos cheios de lágrimas de emoção | extremamente emocionado}
  \end{phonetics}
\end{entry}

\begin{entry}{热闹}{10,8}
  \begin{phonetics}{热闹}{re4nao5}
    \definition{adj.}{animado | movimentado com barulho e excitação}
  \end{phonetics}
\end{entry}

\begin{entry}{热爱}{10,10}
  \begin{phonetics}{热爱}{re4'ai4}
    \definition{v.}{amar ardentemente | adorar}
  \end{phonetics}
\end{entry}

\begin{entry}{热情}{10,11}
  \begin{phonetics}{热情}{re4qing2}[][HSK 2]
    \definition{adj.}{caloroso | fervoroso | entusiasmado}
    \definition{s.}{entusiasmo | ardor | devoção | calor | zelo}
  \end{phonetics}
\end{entry}

\begin{entry}{爱}{10}[Radical 爪]
  \begin{phonetics}{爱}{ai4}[][HSK 1]
    \definition*{s.}{sobrenome Ai}
    \definition[个]{s.}{amor | afeição}
    \definition{v.}{amar | ser afeiçoado a | ter interesse em (alguém) | cuidar bem de | gostar de (fazer algo) | estar inclinado (a fazer algo) | ter o hábito de (fazer algo) | tender a (acontecer)}
  \end{phonetics}
\end{entry}

\begin{entry}{爱人}{10,2}
  \begin{phonetics}{爱人}{ai4ren5}[][HSK 2]
    \definition[个]{s.}{esposa | amor, amada}
  \end{phonetics}
\end{entry}

\begin{entry}{爱上}{10,3}
  \begin{phonetics}{爱上}{ai4shang4}
    \definition{v.}{apaixonar-se por | ser gentil com}
  \end{phonetics}
\end{entry}

\begin{entry}{爱好}{10,6}
  \begin{phonetics}{爱好}{ai4hao4}[][HSK 1]
    \definition[个]{s.}{passatempo | interesse}
    \definition{v.}{ter prazer em | gostar de | ter algo como hobby | apetite por}
  \end{phonetics}
\end{entry}

\begin{entry}{爱好者}{10,6,8}
  \begin{phonetics}{爱好者}{ai4hao4zhe3}
    \definition{s.}{amador | entusiasta | fã | amante de arte, esportes, etc.}
  \end{phonetics}
\end{entry}

\begin{entry}{爱抚}{10,7}
  \begin{phonetics}{爱抚}{ai4fu3}
    \definition{s.}{cuidado afetuoso | carinho}
    \definition{v.}{acariciar | cuidar (com ternura)}
  \end{phonetics}
\end{entry}

\begin{entry}{爱国}{10,8}
  \begin{phonetics}{爱国}{ai4guo2}
    \definition{adj.}{patriótico}
    \definition{v.}{amar o país | ser patriota}
  \end{phonetics}
\end{entry}

\begin{entry}{爱爱}{10,10}
  \begin{phonetics}{爱爱}{ai4'ai5}
    \definition{v.}{(coloquial) fazer amor}
  \end{phonetics}
\end{entry}

\begin{entry}{爱情}{10,11}
  \begin{phonetics}{爱情}{ai4qing2}[][HSK 2]
    \definition{s.}{amor (entre pessoas) | afeição}
  \end{phonetics}
\end{entry}

\begin{entry}{特地}{10,6}
  \begin{phonetics}{特地}{te4di4}
    \definition{adv.}{especialmente | propositalmente}
  \end{phonetics}
\end{entry}

\begin{entry}{特别}{10,7}
  \begin{phonetics}{特别}{te4bie2}[][HSK 2]
    \definition{adj.}{especial | paricular | incomum}
    \definition{adv.}{especialmente | particularmente | propositalmente}
  \end{phonetics}
\end{entry}

\begin{entry}{特技}{10,7}
  \begin{phonetics}{特技}{te4ji4}
    \definition{s.}{efeito especial | dublê}
  \end{phonetics}
\end{entry}

\begin{entry}{特点}{10,9}
  \begin{phonetics}{特点}{te4dian3}[][HSK 2]
    \definition[个]{s.}{característica | peculiaridade | característica distintiva}
  \end{phonetics}
\end{entry}

\begin{entry}{牺牲}{10,9}
  \begin{phonetics}{牺牲}{xi1sheng1}
    \definition{s.}{abate de um animal como sacrifício}
    \definition{v.}{sacrificar a vida de alguém | sacrificar (algo de valor)}
  \end{phonetics}
\end{entry}

\begin{entry}{猃狁}{10,7}
  \begin{phonetics}{猃狁}{xian3yun3}
    \definition*{s.}{Termo da dinastia Zhou para uma tribo nômade do norte mais tarde chamou o Xiongnu (匈奴) nas dinastias Qin e Han}
  \seealsoref{匈奴}{xiong1nu2}
  \end{phonetics}
\end{entry}

\begin{entry}{珠子}{10,3}
  \begin{phonetics}{珠子}{zhu1zi5}
    \definition[粒,颗]{s.}{pérola | contas}
  \end{phonetics}
\end{entry}

\begin{entry}{班}{10}[Radical 玉]
  \begin{phonetics}{班}{ban1}[][HSK 1]
    \definition*{s.}{sobrenome Ban}
    \definition{clas.}{para grupos}
    \definition[个]{s.}{equipe| time | esquadrão | turno de trabalho | classificação}
  \end{phonetics}
\end{entry}

\begin{entry}{班长}{10,4}
  \begin{phonetics}{班长}{ban1 zhang3}[][HSK 2]
    \definition[个]{s.}{monitor de classe | líder de equipe | líder de esquadrão}
  \end{phonetics}
\end{entry}

\begin{entry}{瓶}{10}[Radical 瓦]
  \begin{phonetics}{瓶}{ping2}[][HSK 2]
    \definition{clas.}{para vinho ou líquidos}
    \definition[个]{s.}{garrafa | jarro| vaso}
  \end{phonetics}
\end{entry}

\begin{entry}{瓶子}{10,3}
  \begin{phonetics}{瓶子}{ping2zi5}[][HSK 2]
    \definition[个]{s.}{garrafa}
  \end{phonetics}
\end{entry}

\begin{entry}{瓶盖}{10,11}
  \begin{phonetics}{瓶盖}{ping2gai4}
    \definition{s.}{tampa de garrafa}
  \end{phonetics}
\end{entry}

\begin{entry}{瓶装}{10,12}
  \begin{phonetics}{瓶装}{ping2zhuang1}
    \definition{adj.}{engarrafado}
  \end{phonetics}
\end{entry}

\begin{entry}{瓷}{10}[Radical ⽡]
  \begin{phonetics}{瓷}{ci2}
    \definition{s.}{artigos de porcelana}
  \end{phonetics}
\end{entry}

\begin{entry}{留}{10}[Radical 田]
  \begin{phonetics}{留}{liu2}[][HSK 2]
    \definition{v.}{permanecer | ficar | pedir para alguém ficar | manter alguém onde ele está | concentrar-se em | reservar | manter | salvar | deixar crescer | crescer | vestir | aceitar | tomar | deixar para trás | estudar no exterior}
  \end{phonetics}
\end{entry}

\begin{entry}{留下}{10,3}
  \begin{phonetics}{留下}{liu2 xia4}[][HSK 2]
    \definition{v.}{deixar}
  \end{phonetics}
\end{entry}

\begin{entry}{留学}{10,8}
  \begin{phonetics}{留学}{liu2xue2}
    \definition{v.}{estudar no exterior}
  \end{phonetics}
\end{entry}

\begin{entry}{留学生}{10,8,5}
  \begin{phonetics}{留学生}{liu2 xue2 sheng1}[][HSK 2]
    \definition[个,位,名,批]{s.}{estudante estrangeiro | estudante estudando no exterior}
  \end{phonetics}
\end{entry}

\begin{entry}{留神}{10,9}
  \begin{phonetics}{留神}{liu2shen2}
    \definition{v.+compl.}{tomar cuidado | prestar atenção | manter os olhos abertos}
  \end{phonetics}
\end{entry}

\begin{entry}{畜}{10}[Radical ⽥]
  \begin{phonetics}{畜}{chu4}
    \definition{s.}{gado | animal domesticado | animal doméstico}
  \end{phonetics}
  \begin{phonetics}{畜}{xu4}
    \definition{v.}{criar (animais)}
  \end{phonetics}
\end{entry}

\begin{entry}{疼}{10}[Radical 疒]
  \begin{phonetics}{疼}{teng2}[][HSK 2]
    \definition{adj.}{dolorido | doído}
    \definition{v.}{doer | amar ternamente}
  \end{phonetics}
\end{entry}

\begin{entry}{病}{10}[Radical 疒]
  \begin{phonetics}{病}{bing4}[][HSK 1]
    \definition[场]{s.}{doença}
    \definition{v.}{adoecer | estar doente}
  \end{phonetics}
\end{entry}

\begin{entry}{病人}{10,2}
  \begin{phonetics}{病人}{bing4ren2}[][HSK 1]
    \definition{s.}{doente | paciente}
  \end{phonetics}
\end{entry}

\begin{entry}{盏}{10}[Radical 皿]
  \begin{phonetics}{盏}{zhan3}
    \definition{clas.}{para lâmpadas}
    \definition{s.}{copo pequeno}
  \end{phonetics}
\end{entry}

\begin{entry}{监狱}{10,9}
  \begin{phonetics}{监狱}{jian1yu4}
    \definition{s.}{prisão}
  \end{phonetics}
\end{entry}

\begin{entry}{眞}{10}[Radical 目]
  \begin{phonetics}{眞}{zhen1}
    \variantof{真}
  \end{phonetics}
\end{entry}

\begin{entry}{真}{10}[Radical 目]
  \begin{phonetics}{真}{zhen1}[][HSK 1]
    \definition{adj.}{genuíno}
    \definition{adv.}{que\dots tão\dots! | realmente}
  \end{phonetics}
\end{entry}

\begin{entry}{真切}{10,4}
  \begin{phonetics}{真切}{zhen1qie4}
    \definition{adj.}{claro | distinto | honesto | sincero | vívido}
  \end{phonetics}
\end{entry}

\begin{entry}{真心}{10,4}
  \begin{phonetics}{真心}{zhen1xin1}
    \definition{adj.}{sincero}
    \definition[片]{s.}{sinceridade}
  \end{phonetics}
\end{entry}

\begin{entry}{真牛}{10,4}
  \begin{phonetics}{真牛}{zhen1niu2}
    \definition{adj.}{(gíria) muito legal, incrível}
  \end{phonetics}
\end{entry}

\begin{entry}{真正}{10,5}
  \begin{phonetics}{真正}{zhen1zheng4}[][HSK 2]
    \definition{adj.}{verdadeiro | real | genuíno}
    \definition{adv.}{realmente | de ​​fato}
  \end{phonetics}
\end{entry}

\begin{entry}{真声}{10,7}
  \begin{phonetics}{真声}{zhen1sheng1}
    \definition{s.}{voz natural | voz verdadeira}
    \seeref{假声}{jia3sheng1}
  \end{phonetics}
\end{entry}

\begin{entry}{真的}{10,8}
  \begin{phonetics}{真的}{zhen1 de5}[][HSK 1]
    \definition{adv.}{realmente | verdadeiramente}
  \end{phonetics}
\end{entry}

\begin{entry}{真珠}{10,10}
  \begin{phonetics}{真珠}{zhen1zhu1}
    \variantof{珍珠}
  \end{phonetics}
\end{entry}

\begin{entry}{真真}{10,10}
  \begin{phonetics}{真真}{zhen1zhen1}
    \definition{adv.}{genuinamente | realmente | escrupulosamente}
  \end{phonetics}
\end{entry}

\begin{entry}{真理}{10,11}
  \begin{phonetics}{真理}{zhen1li3}
    \definition[个]{s.}{verdade}
  \end{phonetics}
\end{entry}

\begin{entry}{真释}{10,12}
  \begin{phonetics}{真释}{zhen1shi4}
    \definition{s.}{razão genuína | explicação verdadeira}
  \end{phonetics}
\end{entry}

\begin{entry}{破}{10}[Radical 石]
  \begin{phonetics}{破}{po4}
    \definition{adj.}{partido | quebrado | roto | nojento | esgotado | falido}
    \definition{v.}{romper com | quebrar, dividir ou clivar | capturar (uma cidade, etc.) | derrotar | destruir | expor a verdade de | se livrar}
  \end{phonetics}
\end{entry}

\begin{entry}{破产}{10,6}
  \begin{phonetics}{破产}{po4chan3}
    \definition{v.+compl.}{falir | quebrar | tornar-se insolvente | ficar empobrecido | cair | ser destruído | ser arruinado}
  \end{phonetics}
\end{entry}

\begin{entry}{破坏}{10,7}
  \begin{phonetics}{破坏}{po4huai4}
    \definition{s.}{destruição | dano}
    \definition{v.}{destruir | danificar}
  \end{phonetics}
\end{entry}

\begin{entry}{破坏性}{10,7,8}
  \begin{phonetics}{破坏性}{po4huai4xing4}
    \definition{adj.}{destrutivo}
    \definition{s.}{poder destrutivo}
  \end{phonetics}
\end{entry}

\begin{entry}{砸}{10}[Radical 石]
  \begin{phonetics}{砸}{za2}
    \definition{v.}{esmagar | bater | falhar | estragar}
  \end{phonetics}
\end{entry}

\begin{entry}{离}{10}[Radical 亠]
  \begin{phonetics}{离}{li2}[][HSK 2]
    \definition*{s.}{sobrenome Li}
    \definition{prep.}{(ser longe) de\dots até\dots}
    \definition{v.}{ficar longe de | deixar | separar-se de}
  \end{phonetics}
\end{entry}

\begin{entry}{离开}{10,4}
  \begin{phonetics}{离开}{li2kai1}[][HSK 2]
    \definition{v.}{partir| deixar}
  \end{phonetics}
\end{entry}

\begin{entry}{离婚}{10,11}
  \begin{phonetics}{离婚}{li2hun1}
    \definition{v.+compl.}{divorciar-se | teminar um casamento}
  \end{phonetics}
\end{entry}

\begin{entry}{租}{10}[Radical 禾]
  \begin{phonetics}{租}{zu1}[][HSK 2]
    \definition{s.}{imposto sobre propriedade urbana ou rural}
    \definition{v.}{alugar | tomar de aluguel}
  \end{phonetics}
\end{entry}

\begin{entry}{租用}{10,5}
  \begin{phonetics}{租用}{zu1yong4}
    \definition{v.}{contratar | alugar | alugar (algo de alguém)}
  \end{phonetics}
\end{entry}

\begin{entry}{租让}{10,5}
  \begin{phonetics}{租让}{zu1rang4}
    \definition{v.}{alugar | alugar (a propriedade de alguém para outra pessoa)}
  \end{phonetics}
\end{entry}

\begin{entry}{租约}{10,6}
  \begin{phonetics}{租约}{zu1yue1}
    \definition{s.}{aluguel}
  \end{phonetics}
\end{entry}

\begin{entry}{租房}{10,8}
  \begin{phonetics}{租房}{zu1fang2}
    \definition{v.}{alugar um apartamento}
  \end{phonetics}
\end{entry}

\begin{entry}{租金}{10,8}
  \begin{phonetics}{租金}{zu1jin1}
    \definition{s.}{aluguel}
    \seeref{租钱}{zu1qian5}
  \end{phonetics}
\end{entry}

\begin{entry}{租赁}{10,10}
  \begin{phonetics}{租赁}{zu1lin4}
    \definition{v.}{contratar | alugar}
  \end{phonetics}
\end{entry}

\begin{entry}{租钱}{10,10}
  \begin{phonetics}{租钱}{zu1qian5}
    \definition{s.}{aluguel}
    \seeref{租金}{zu1jin1}
  \end{phonetics}
\end{entry}

\begin{entry}{租船}{10,11}
  \begin{phonetics}{租船}{zu1chuan2}
    \definition{v.}{fretar um navio | alugar um navio}
  \end{phonetics}
\end{entry}

\begin{entry}{积木}{10,4}
  \begin{phonetics}{积木}{ji1mu4}
    \definition{s.}{blocos de montar (brinquedo)}
  \end{phonetics}
\end{entry}

\begin{entry}{称}{10}[Radical 禾]
  \begin{phonetics}{称}{chen4}
    \definition{v.}{ajustar | combinar}
  \end{phonetics}
  \begin{phonetics}{称}{cheng1}
    \definition*{s.}{sobrenome Cheng}
    \definition{s.}{nome}
    \definition{v.}{chamar | dizer | elogiar | louvar | pesar | levantar | começar}
  \end{phonetics}
\end{entry}

\begin{entry}{站}{10}[Radical 立]
  \begin{phonetics}{站}{zhan4}[][HSK 1]
    \definition{s.}{estação | ponto | parada}
  \end{phonetics}
\end{entry}

\begin{entry}{站长}{10,4}
  \begin{phonetics}{站长}{zhan4zhang3}
    \definition{s.}{pessoa responsável pela estação de trem | chefe da estação | \emph{webmaster} | gerente de centro de voluntariado}
  \end{phonetics}
\end{entry}

\begin{entry}{站台}{10,5}
  \begin{phonetics}{站台}{zhan4tai2}
    \definition{s.}{plataforma (em uma estação ferroviária)}
  \end{phonetics}
\end{entry}

\begin{entry}{站住}{10,7}
  \begin{phonetics}{站住}{zhan4 zhu4}[][HSK 2]
    \definition{v.}{parar | deter | ficar firme em pé | manter os pés firmes | manter a própria posição | consolidar a própria posição | reter água | ser sustentável}
  \end{phonetics}
\end{entry}

\begin{entry}{站姿}{10,9}
  \begin{phonetics}{站姿}{zhan4zi1}
    \definition{s.}{postura}
  \end{phonetics}
\end{entry}

\begin{entry}{站点}{10,9}
  \begin{phonetics}{站点}{zhan4dian3}
    \definition{s.}{\emph{website}}
  \end{phonetics}
\end{entry}

\begin{entry}{竞赛}{10,14}
  \begin{phonetics}{竞赛}{jing4sai4}
    \definition{s.}{concurso | competição | partida | corrida}
    \definition{v.}{competir | correr}
  \end{phonetics}
\end{entry}

\begin{entry}{笋}{10}[Radical 竹]
  \begin{phonetics}{笋}{sun3}
    \definition{s.}{broto de bambu}
  \end{phonetics}
\end{entry}

\begin{entry}{笑}{10}[Radical 竹]
  \begin{phonetics}{笑}{xiao4}[][HSK 1]
    \definition{v.}{sorrir | rir | rir de}
  \end{phonetics}
\end{entry}

\begin{entry}{笑话}{10,8}
  \begin{phonetics}{笑话}{xiao4hua5}[][HSK 2]
    \definition{adj.}{absurdo | ridículo}
    \definition[个]{s.}{piada | brincadeira}
    \definition{v.}{rir de algo | zombar | ridicularizar}
  \end{phonetics}
\end{entry}

\begin{entry}{笑话儿}{10,8,2}
  \begin{phonetics}{笑话儿}{xiao4 hua4r5}[][HSK 2]
    \definition{s.}{piada | gracejo}
  \end{phonetics}
\end{entry}

\begin{entry}{笑容}{10,10}
  \begin{phonetics}{笑容}{xiao4rong2}
    \definition[副]{s.}{sorriso | expressão sorridente}
  \end{phonetics}
\end{entry}

\begin{entry}{笔}{10}[Radical 竹]
  \begin{phonetics}{笔}{bi3}[][HSK 2]
    \definition{clas.}{para somas de dinheiro, negócios}
    \definition[支,枝]{s.}{caneta | lápis}
  \end{phonetics}
\end{entry}

\begin{entry}{笔记}{10,5}
  \begin{phonetics}{笔记}{bi3 ji4}[][HSK 2]
    \definition[篇,本,个]{s.}{notas | ensaios | esboços}
    \definition{v.}{tomar nota (por escrito)}
  \end{phonetics}
\end{entry}

\begin{entry}{笔记本}{10,5,5}
  \begin{phonetics}{笔记本}{bi3ji4ben3}[][HSK 2]
    \definition[本]{s.}{caderno}
    \definition{s.}{\emph{laptop}}
  \end{phonetics}
\end{entry}

\begin{entry}{粉}{10}[Radical ⽶]
  \begin{phonetics}{粉}{fen3}
    \definition{s.}{pó | pó cosmético facial | alimento preparado a partir de amido | macarrão feito de qualquer tipo de farinha}
    \definition{v.}{tornar algo em pó | ser um fã de}
  \end{phonetics}
\end{entry}

\begin{entry}{粉丝}{10,5}
  \begin{phonetics}{粉丝}{fen3si1}
    \definition{s.}{(empréstimo linguístico) fã | entusiasta de alguém ou alguma coisa}
    \definition[把]{s.}{aletria de amido de feijão | aletria chinesa | macarrão de celofane ou macarrão de vidro (transparente)}
  \end{phonetics}
\end{entry}

\begin{entry}{粉色}{10,6}
  \begin{phonetics}{粉色}{fen3se4}
    \definition{s.}{cor-de-rosa}
  \end{phonetics}
\end{entry}

\begin{entry}{索性}{10,8}
  \begin{phonetics}{索性}{suo3xing4}
    \definition{adv.}{poderia muito bem | simplesmente | apenas}
  \end{phonetics}
\end{entry}

\begin{entry}{紧急}{10,9}
  \begin{phonetics}{紧急}{jin3ji2}
    \definition{adj.}{urgente}
    \definition{s.}{emergência}
  \end{phonetics}
\end{entry}

\begin{entry}{绣}{10}[Radical 糸]
  \begin{phonetics}{绣}{xiu4}
    \definition{s.}{bordado}
    \definition{v.}{bordar}
  \end{phonetics}
\end{entry}

\begin{entry}{缺勤}{10,13}
  \begin{phonetics}{缺勤}{que1qin2}
    \definition{v.+compl.}{ausentar-se do dever (trabalho)}
  \end{phonetics}
\end{entry}

\begin{entry}{罢}{10}[Radical 网]
  \begin{phonetics}{罢}{ba4}
    \definition{v.}{parar | cessar | demitir | suspender | desistir | terminar}
  \end{phonetics}
  \begin{phonetics}{罢}{ba5}
    \definition{part.}{partícula final, a mesma que 吧}
  \seealsoref{吧}{ba5}
  \end{phonetics}
\end{entry}

\begin{entry}{耽心}{10,4}
  \begin{phonetics}{耽心}{dan1xin1}
    \variantof{担心}
  \end{phonetics}
\end{entry}

\begin{entry}{胶卷}{10,8}
  \begin{phonetics}{胶卷}{jiao1juan3}
    \definition{s.}{filme | rolo de filme}
  \end{phonetics}
\end{entry}

\begin{entry}{胸}{10}[Radical 肉]
  \begin{phonetics}{胸}{xiong1}
    \definition{s.}{peito | tórax}
  \end{phonetics}
\end{entry}

\begin{entry}{能}{10}[Radical 肉]
  \begin{phonetics}{能}{neng2}[][HSK 1]
    \definition*{s.}{sobrenome Neng}
    \definition{adv.}{talvez}
    \definition{s.}{(física)nenergia | habilidade}
    \definition{v.}{poder | ser capaz de}
  \end{phonetics}
\end{entry}

\begin{entry}{能上能下}{10,3,10,3}
  \begin{phonetics}{能上能下}{neng2shang4neng2xia4}
    \definition{s.}{pronto para aceitar qualquer trabalho, alto ou baixo}
  \end{phonetics}
\end{entry}

\begin{entry}{能干}{10,3}
  \begin{phonetics}{能干}{neng2gan4}
    \definition{adj.}{capaz | competente}
  \end{phonetics}
\end{entry}

\begin{entry}{能够}{10,11}
  \begin{phonetics}{能够}{neng2 gou4}[][HSK 2]
    \definition{v.}{ser capaz de}
  \end{phonetics}
\end{entry}

\begin{entry}{脂麻}{10,11}
  \begin{phonetics}{脂麻}{zhi1ma5}
    \variantof{芝麻}
  \end{phonetics}
\end{entry}

\begin{entry}{脏}{10}[Radical 肉]
  \begin{phonetics}{脏}{zang1}[][HSK 2]
    \definition{adj.}{sujo | imundo}
  \end{phonetics}
  \begin{phonetics}{脏}{zang4}[][HSK 0]
    \definition{s.}{órgão (anatomia) | víscera}
  \end{phonetics}
\end{entry}

\begin{entry}{脏土}{10,3}
  \begin{phonetics}{脏土}{zang1tu3}
    \definition{s.}{solo sujo | lama | lixo}
  \end{phonetics}
\end{entry}

\begin{entry}{脏字}{10,6}
  \begin{phonetics}{脏字}{zang1zi4}
    \definition{s.}{obscenidade}
  \end{phonetics}
\end{entry}

\begin{entry}{脏病}{10,10}
  \begin{phonetics}{脏病}{zang1bing4}
    \definition{s.}{doença venérea}
  \end{phonetics}
\end{entry}

\begin{entry}{脏脏}{10,10}
  \begin{phonetics}{脏脏}{zang1zang1}
    \definition{adj.}{sujo}
  \end{phonetics}
\end{entry}

\begin{entry}{脏煤}{10,13}
  \begin{phonetics}{脏煤}{zang1mei2}
    \definition{s.}{carvão sujo | sujeira (de uma mina de carvão)}
  \end{phonetics}
\end{entry}

\begin{entry}{脏器}{10,16}
  \begin{phonetics}{脏器}{zang4qi4}
    \definition{s.}{órgãos internos}
  \end{phonetics}
\end{entry}

\begin{entry}{脏辫}{10,17}
  \begin{phonetics}{脏辫}{zang1bian4}
    \definition{s.}{\emph{dreadlocks}}
  \end{phonetics}
\end{entry}

\begin{entry}{脑瓜}{10,5}
  \begin{phonetics}{脑瓜}{nao3gua1}
    \definition{s.}{crânio | cérebro | cabeça | mente | mentalidade | ideia}
  \seealsoref{脑瓜子}{nao3gua1zi5}
  \end{phonetics}
\end{entry}

\begin{entry}{脑瓜子}{10,5,3}
  \begin{phonetics}{脑瓜子}{nao3gua1zi5}
    \definition{s.}{crânio | cérebro | cabeça | mente | mentalidade | ideia}
  \seealsoref{脑瓜}{nao3gua1}
  \end{phonetics}
\end{entry}

\begin{entry}{脑袋}{10,11}
  \begin{phonetics}{脑袋}{nao3dai5}
    \definition[颗,个]{s.}{cabeça | crânio | cérebro | capacidade mental}
  \end{phonetics}
\end{entry}

\begin{entry}{臭}{10}[Radical ⾃]
  \begin{phonetics}{臭}{chou4}
    \definition{adj.}{fétido | repulsivo | repugnante | malcheiroso}
    \definition{s.}{fedor}
    \definition{v.}{feder}
  \end{phonetics}
  \begin{phonetics}{臭}{xiu4}
    \definition{s.}{olfato | cheiro ruim}
  \end{phonetics}
\end{entry}

\begin{entry}{臭气}{10,4}
  \begin{phonetics}{臭气}{chou4qi4}
    \definition{s.}{fedor}
  \end{phonetics}
\end{entry}

\begin{entry}{致敬}{10,12}
  \begin{phonetics}{致敬}{zhi4jing4}
    \definition{v.}{saudar | prestar respeitos a | prestar homenagem a}
  \end{phonetics}
\end{entry}

\begin{entry}{航天员}{10,4,7}
  \begin{phonetics}{航天员}{hang2tian1yuan2}
    \definition{s.}{astronauta}
  \end{phonetics}
\end{entry}

\begin{entry}{航班}{10,10}
  \begin{phonetics}{航班}{hang2ban1}
    \definition{s.}{voo | número de voo}
  \end{phonetics}
\end{entry}

\begin{entry}{般}{10}[Radical 舟]
  \begin{phonetics}{般}{ban1}
    \definition{s.}{espécie | tipo | classe | caminho | maneira}
  \end{phonetics}
  \begin{phonetics}{般}{bo1}
    \definition{s.}{utilizado em 般若 \dpy{bo1re3}}
    \seeref{般若}{bo1re3}
  \end{phonetics}
  \begin{phonetics}{般}{pan2}
    \definition{s.}{utilizado em 般乐 \dpy{pan2le4}}
    \seeref{般乐}{pan2le4}
  \end{phonetics}
\end{entry}

\begin{entry}{般乐}{10,5}
  \begin{phonetics}{般乐}{pan2le4}
    \definition{v.}{jogar | divertir-se}
  \end{phonetics}
\end{entry}

\begin{entry}{般若}{10,8}
  \begin{phonetics}{般若}{bo1re3}
    \definition*{s.}{Prajna (sânscrito), \emph{insight} sobre a verdadeira natureza da realidade | (Budismo) sabedoria}
  \end{phonetics}
\end{entry}

\begin{entry}{舱}{10}[Radical ⾈]
  \begin{phonetics}{舱}{cang1}
    \definition{s.}{cabine | porão (de carga) de um navio ou avião}
  \end{phonetics}
\end{entry}

\begin{entry}{荷}{10}[Radical 艸]
  \begin{phonetics}{荷}{he2}
    \definition{s.}{lótus}
  \end{phonetics}
  \begin{phonetics}{荷}{he4}
    \definition{s.}{carga | responsabilidade}
    \definition{v.}{carregar no ombro ou nas costas}
  \end{phonetics}
\end{entry}

\begin{entry}{荷花}{10,7}
  \begin{phonetics}{荷花}{he2hua1}
    \definition{s.}{lótus}
  \end{phonetics}
\end{entry}

\begin{entry}{莎莎舞}{10,10,14}
  \begin{phonetics}{莎莎舞}{sha1sha1wu3}
    \definition{s.}{salsa (dança)}
  \end{phonetics}
\end{entry}

\begin{entry}{莫名其妙}{10,6,8,7}
  \begin{phonetics}{莫名其妙}{mo4ming2qi2miao4}
    \definition{adj.}{desconcertante | bizzaro | inexplicável | perplexo}
  \end{phonetics}
\end{entry}

\begin{entry}{莲花}{10,7}
  \begin{phonetics}{莲花}{lian2hua1}
    \definition{s.}{flor de lótus | lírio aquático}
  \end{phonetics}
\end{entry}

\begin{entry}{莲藕}{10,18}
  \begin{phonetics}{莲藕}{lian2'ou3}
    \definition{s.}{raiz de Lotus}
  \end{phonetics}
\end{entry}

\begin{entry}{蚊子}{10,3}
  \begin{phonetics}{蚊子}{wen2zi5}
    \definition{s.}{pernilongo}
  \end{phonetics}
\end{entry}

\begin{entry}{蚊香}{10,9}
  \begin{phonetics}{蚊香}{wen2xiang1}
    \definition{s.}{incenso ou espiral repelente de mosquitos}
  \end{phonetics}
\end{entry}

\begin{entry}{蚕纸}{10,7}
  \begin{phonetics}{蚕纸}{can2zhi3}
    \definition{s.}{papel onde o bicho-da-seda põe seus ovos}
  \end{phonetics}
\end{entry}

\begin{entry}{蚝}{10}[Radical 虫]
  \begin{phonetics}{蚝}{hao2}
    \definition{s.}{ostra}
  \end{phonetics}
\end{entry}

\begin{entry}{袖}{10}[Radical 衣]
  \begin{phonetics}{袖}{xiu4}
    \definition{s.}{manga (de camisa, de camiseta, etc.)}
  \end{phonetics}
\end{entry}

\begin{entry}{被}{10}[Radical 衣]
  \begin{phonetics}{被}{bei4}
    \definition{prep.}{por}
  \end{phonetics}
\end{entry}

\begin{entry}{被子}{10,3}
  \begin{phonetics}{被子}{bei4zi5}
    \definition[床]{s.}{colcha}
  \end{phonetics}
\end{entry}

\begin{entry}{被动}{10,6}
  \begin{phonetics}{被动}{bei4dong4}
    \definition{adj.}{passivo}
  \end{phonetics}
\end{entry}

\begin{entry}{被告}{10,7}
  \begin{phonetics}{被告}{bei4gao4}
    \definition{s.}{réu}
  \end{phonetics}
\end{entry}

\begin{entry}{被单}{10,8}
  \begin{phonetics}{被单}{bei4dan1}
    \definition[床]{s.}{lençol}
  \end{phonetics}
\end{entry}

\begin{entry}{被迫}{10,8}
  \begin{phonetics}{被迫}{bei4po4}
    \definition{v.}{ser compelido | ser forçado}
  \end{phonetics}
\end{entry}

\begin{entry}{被套}{10,10}
  \begin{phonetics}{被套}{bei4tao4}
    \definition{s.}{capa de \emph{edredon}}
    \definition{v.}{ter dinheiro preso (em ações, imóveis, etc.)}
  \end{phonetics}
\end{entry}

\begin{entry}{被窝}{10,12}
  \begin{phonetics}{被窝}{bei4wo1}
    \definition{s.}{colcha}
  \end{phonetics}
\end{entry}

\begin{entry}{袮}{10}[Radical 衣]
  \begin{phonetics}{袮}{ni3}
    \definition{pron.}{Você, Tu (divindade)}
    \variantof{你}
  \end{phonetics}
\end{entry}

\begin{entry}{请}{10}[Radical 言]
  \begin{phonetics}{请}{qing3}[][HSK 1]
    \definition{v.}{por favor (fazer alguma coisa) | perguntar | convidar | solicitar}
  \end{phonetics}
\end{entry}

\begin{entry}{请问}{10,6}
  \begin{phonetics}{请问}{qing3wen4}[][HSK 1]
    \definition{expr.}{Com licença, posso perguntar\dots? (para perguntar por qualquer coisa)}
  \end{phonetics}
\end{entry}

\begin{entry}{请坐}{10,7}
  \begin{phonetics}{请坐}{qing3 zuo4}[][HSK 1]
    \definition{v.}{por favor, sente-se}
  \end{phonetics}
\end{entry}

\begin{entry}{请求}{10,7}
  \begin{phonetics}{请求}{qing3qiu2}[][HSK 2]
    \definition[个]{s.}{solicitação}
    \definition{v.}{solicitar | perguntar}
  \end{phonetics}
\end{entry}

\begin{entry}{请进}{10,7}
  \begin{phonetics}{请进}{qing3 jin4}[][HSK 1]
    \definition{v.}{por favor entre}
  \end{phonetics}
\end{entry}

\begin{entry}{请客}{10,9}
  \begin{phonetics}{请客}{qing3ke4}[][HSK 2]
    \definition{v.+compl.}{entreter os convidados | dar um jantar | convidar para jantar}
  \end{phonetics}
\end{entry}

\begin{entry}{请假}{10,11}
  \begin{phonetics}{请假}{qing3 jia4}[][HSK 1]
    \definition{v.+compl.}{pedir licença para sair}
  \end{phonetics}
\end{entry}

\begin{entry}{请假条}{10,11,7}
  \begin{phonetics}{请假条}{qing3jia4tiao2}
    \definition{s.}{pedido de licença de ausência (do trabalho ou da escola)}
  \end{phonetics}
\end{entry}

\begin{entry}{诺贝尔奖}{10,4,5,9}
  \begin{phonetics}{诺贝尔奖}{nuo4bei4'er3 jiang3}
    \definition*{s.}{Prêmio Nobel}
  \end{phonetics}
\end{entry}

\begin{entry}{诺奖}{10,9}
  \begin{phonetics}{诺奖}{nuo4jiang3}
    \definition*{s.}{Prêmio Nobel, abreviação de 诺贝尔奖}
    \seeref{诺贝尔奖}{nuo4bei4'er3 jiang3}
  \end{phonetics}
\end{entry}

\begin{entry}{读}{10}[Radical 言]
  \begin{phonetics}{读}{dou4}
    \definition{s.}{vírgula | frase marcada por pausa}
  \end{phonetics}
  \begin{phonetics}{读}{du2}
    \definition{v.}{ler em voz alta | ler | frequentar (escola) | estudar (uma matéria na escola) | pronunciar}
  \end{phonetics}
\end{entry}

\begin{entry}{读书}{10,4}
  \begin{phonetics}{读书}{du2 shu1}[][HSK 1]
    \definition{v.+compl.}{ler | estudar | frequentar a escola}
  \end{phonetics}
\end{entry}

\begin{entry}{读音}{10,9}
  \begin{phonetics}{读音}{du2 yin1}[][HSK 2]
    \definition{s.}{pronúncia}
  \end{phonetics}
\end{entry}

\begin{entry}{课}{10}[Radical 言]
  \begin{phonetics}{课}{ke4}[][HSK 1]
    \definition{s.}{aula | curso | lição | imposto | taxa |seção}
  \end{phonetics}
\end{entry}

\begin{entry}{课文}{10,4}
  \begin{phonetics}{课文}{ke4wen2}[][HSK 1]
    \definition{s.}{texto (de uma lição)}
  \end{phonetics}
\end{entry}

\begin{entry}{课本}{10,5}
  \begin{phonetics}{课本}{ke4ben3}[][HSK 1]
    \definition[本]{s.}{livro do aluno | manual}
  \end{phonetics}
\end{entry}

\begin{entry}{课堂}{10,11}
  \begin{phonetics}{课堂}{ke4 tang2}[][HSK 2]
    \definition[间]{s.}{sala de aula}
  \end{phonetics}
\end{entry}

\begin{entry}{谁}{10}[Radical 言]
  \begin{phonetics}{谁}{shei2}[][HSK 1]
    \definition{pron.}{quem?}
  \end{phonetics}
  \begin{phonetics}{谁}{shui2}[][HSK 1]
    \definition{pron.}{quem?}
  \end{phonetics}
\end{entry}

\begin{entry}{调律}{10,9}
  \begin{phonetics}{调律}{tiao2lv4}
    \definition{v.}{afinar (por exemplo, um piano)}
  \end{phonetics}
\end{entry}

\begin{entry}{谈话}{10,8}
  \begin{phonetics}{谈话}{tan2hua4}
    \definition[次]{s.}{conversa | fala | papo | declaração}
    \definition{v.+compl.}{conversar | falar | declarar}
  \end{phonetics}
\end{entry}

\begin{entry}{谈恋爱}{10,10,10}
  \begin{phonetics}{谈恋爱}{tan2lian4'ai4}
    \definition{v.}{namorar | apaixonar-se}
  \end{phonetics}
\end{entry}

\begin{entry}{豹子}{10,3}
  \begin{phonetics}{豹子}{bao4zi5}
    \definition[头]{s.}{leopardo}
  \end{phonetics}
\end{entry}

\begin{entry}{资}{10}[Radical 貝]
  \begin{phonetics}{资}{zi1}
    \definition{s.}{recursos | capital | dinheiro | despesa}
    \definition{v.}{fornecer | suprir}
  \end{phonetics}
\end{entry}

\begin{entry}{资助}{10,7}
  \begin{phonetics}{资助}{zi1zhu4}
    \definition{s.}{subsídio}
    \definition{v.}{subsidiar | fornecer ajuda financeira}
  \end{phonetics}
\end{entry}

\begin{entry}{赶}{10}[Radical ⾛]
  \begin{phonetics}{赶}{gan3}
    \definition{v.}{apressar | precipitar-se | conduzir (gado, etc.) | aproveitar (uma oportunidade)}
  \end{phonetics}
\end{entry}

\begin{entry}{赶上}{10,3}
  \begin{phonetics}{赶上}{gan3shang4}
    \definition{adv.}{a tempo para}
    \definition{v.}{alcançar | ultrapassar}
  \end{phonetics}
\end{entry}

\begin{entry}{赶忙}{10,6}
  \begin{phonetics}{赶忙}{gan3mang2}
    \definition{v.}{acelerar | apressar | se apressar}
  \end{phonetics}
\end{entry}

\begin{entry}{赶早}{10,6}
  \begin{phonetics}{赶早}{gan3zao3}
    \definition{adv.}{o mais breve possível | na primeira oportunidade | antes que seja tarde | quanto antes melhor}
  \end{phonetics}
\end{entry}

\begin{entry}{赶快}{10,7}
  \begin{phonetics}{赶快}{gan3kuai4}
    \definition{adv.}{imediatamente | de uma vez só}
  \end{phonetics}
\end{entry}

\begin{entry}{赶走}{10,7}
  \begin{phonetics}{赶走}{gan3zou3}
    \definition{v.}{expulsar | voltar atrás}
  \end{phonetics}
\end{entry}

\begin{entry}{赶到}{10,8}
  \begin{phonetics}{赶到}{gan3dao4}
    \definition{v.}{apressar-se (para algum lugar)}
  \end{phonetics}
\end{entry}

\begin{entry}{赶赴}{10,9}
  \begin{phonetics}{赶赴}{gan3fu4}
    \definition{v.}{apressar}
  \end{phonetics}
\end{entry}

\begin{entry}{赶紧}{10,10}
  \begin{phonetics}{赶紧}{gan3jin3}
    \definition{adv.}{apressadamente | sem demora}
  \end{phonetics}
\end{entry}

\begin{entry}{赶脚}{10,11}
  \begin{phonetics}{赶脚}{gan3jiao3}
    \definition{v.}{transportar mercadorias para ganhar a vida (especialmente de burro) | trabalhar como carroceiro ou porteiro}
  \end{phonetics}
\end{entry}

\begin{entry}{赶跑}{10,12}
  \begin{phonetics}{赶跑}{gan3pao3}
    \definition{v.}{afastar | forçar a saída | repelir}
  \end{phonetics}
\end{entry}

\begin{entry}{赶集}{10,12}
  \begin{phonetics}{赶集}{gan3ji2}
    \definition{v.}{ir a uma feira | ir ao mercado}
  \end{phonetics}
\end{entry}

\begin{entry}{赶路}{10,13}
  \begin{phonetics}{赶路}{gan3lu4}
    \definition{v.}{apressar a jornada | apressar-se}
  \end{phonetics}
\end{entry}

\begin{entry}{起}{10}[Radical 走]
  \begin{phonetics}{起}{qi3}[][HSK 1]
    \definition*{s.}{sobrenome Qi}
    \definition{clas.}{caso; instância | lote; grupo}
    \definition{v.}{levantar | levantar-se | extrair| remover | puxar | aparecer | crescer | construir | configurar | começar | iniciar}
  \end{phonetics}
\end{entry}

\begin{entry}{起飞}{10,3}
  \begin{phonetics}{起飞}{qi3fei1}[][HSK 2]
    \definition{v.}{decolar}
  \end{phonetics}
\end{entry}

\begin{entry}{起床}{10,7}
  \begin{phonetics}{起床}{qi3 chuang2}[][HSK 1]
    \definition{v.+compl.}{sair da cama | levantar-se}
  \end{phonetics}
\end{entry}

\begin{entry}{起来}{10,7}
  \begin{phonetics}{起来}{qi3 lai2}[][HSK 1]
    \definition{v.+compl.}{levantar-se}
  \end{phonetics}
\end{entry}

\begin{entry}{起跳}{10,13}
  \begin{phonetics}{起跳}{qi3tiao4}
    \definition{v.}{(atletismo) decolar (no início de um salto) | (de preço, salário, etc.) começar (de um determinado nível)}
  \end{phonetics}
\end{entry}

\begin{entry}{辱骂}{10,9}
  \begin{phonetics}{辱骂}{ru3ma4}
    \definition{v.}{insultar | abusar}
  \end{phonetics}
\end{entry}

\begin{entry}{透}{10}[Radical 辵]
  \begin{phonetics}{透}{tou4}
    \definition{adj.}{completo | total}
    \definition{adv.}{completamente | totalmente}
    \definition{v.}{aparecer | passar através | penetrar}
  \end{phonetics}
\end{entry}

\begin{entry}{透支}{10,4}
  \begin{phonetics}{透支}{tou4zhi1}
    \definition{v.}{cheque especial (bancário) | saque a descoberto}
  \end{phonetics}
\end{entry}

\begin{entry}{透气}{10,4}
  \begin{phonetics}{透气}{tou4qi4}
    \definition{v.}{respirar (sobre tecido, etc.) | fluir livremente (sobre ar) | respirar ar fresco | ventilar}
  \end{phonetics}
\end{entry}

\begin{entry}{透水}{10,4}
  \begin{phonetics}{透水}{tou4shui3}
    \definition{adj.}{permeável}
    \definition{s.}{vazamento de água}
  \end{phonetics}
\end{entry}

\begin{entry}{透过}{10,6}
  \begin{phonetics}{透过}{tou4guo4}
    \definition{v.}{passar através | penetrar}
  \end{phonetics}
\end{entry}

\begin{entry}{透彻}{10,7}
  \begin{phonetics}{透彻}{tou4che4}
    \definition{adj.}{minucioso | incisivo | penetrante}
  \end{phonetics}
\end{entry}

\begin{entry}{透明}{10,8}
  \begin{phonetics}{透明}{tou4ming2}
    \definition{adj.}{transparente | (figurativo) transparente, aberto a escrutínio}
  \end{phonetics}
\end{entry}

\begin{entry}{透顶}{10,8}
  \begin{phonetics}{透顶}{tou4ding3}
    \definition{adv.}{completamente}
  \end{phonetics}
\end{entry}

\begin{entry}{透亮}{10,9}
  \begin{phonetics}{透亮}{tou4liang4}
    \definition{adj.}{brilhante | claro como cristal}
  \end{phonetics}
\end{entry}

\begin{entry}{透辟}{10,13}
  \begin{phonetics}{透辟}{tou4pi4}
    \definition{adj.}{incisivo | penetrante}
  \end{phonetics}
\end{entry}

\begin{entry}{透澈}{10,15}
  \begin{phonetics}{透澈}{tou4che4}
    \variantof{透彻}
  \end{phonetics}
\end{entry}

\begin{entry}{透露}{10,21}
  \begin{phonetics}{透露}{tou4lu4}
    \definition{v.}{divulgar | vazar | revelar}
  \end{phonetics}
\end{entry}

\begin{entry}{逐步}{10,7}
  \begin{phonetics}{逐步}{zhu2bu4}
    \definition{adv.}{pouco a pouco; passo a passo; progressivamente}
  \end{phonetics}
\end{entry}

\begin{entry}{逐渐}{10,11}
  \begin{phonetics}{逐渐}{zhu2jian4}
    \definition{adv.}{pouco a pouco; passo a passo; progressivamente}
  \end{phonetics}
\end{entry}

\begin{entry}{通}{10}[Radical 辵]
  \begin{phonetics}{通}{tong1}[][HSK 2]
    \definition{clas.}{para cartas, telegramas, telefonemas, etc.}
    \definition{suf.}{especialista}
    \definition{v.}{ligar para | conseguir a ligação}
  \end{phonetics}
  \begin{phonetics}{通}{tong4}[][HSK 0]
    \definition{clas.}{para uma atividade, tomada em sua totalidade (discurso de abuso, período de reprodução de música, bebedeira, etc.)}
  \end{phonetics}
\end{entry}

\begin{entry}{通观}{10,6}
  \begin{phonetics}{通观}{tong1guan1}
    \definition{v.}{ter uma visão geral de algo}
  \end{phonetics}
\end{entry}

\begin{entry}{通过}{10,6}
  \begin{phonetics}{通过}{tong1guo4}[][HSK 2]
    \definition{adv.}{por meio de | através de | via}
    \definition{v.}{passar por | adotar (uma resolução), aprovar (legislação) | passar (em um teste)}
  \end{phonetics}
\end{entry}

\begin{entry}{通识}{10,7}
  \begin{phonetics}{通识}{tong1shi2}
    \definition{s.}{conhecimento comum | erudição | conhecimento geral | amplamente conhecido}
  \end{phonetics}
\end{entry}

\begin{entry}{通知}{10,8}
  \begin{phonetics}{通知}{tong1zhi1}[][HSK 2]
    \definition[份,个,张]{s.}{aviso | circular}
    \definition{v.}{aconselhar | notificar | informar | dar aviso}
  \end{phonetics}
\end{entry}

\begin{entry}{通牒}{10,13}
  \begin{phonetics}{通牒}{tong1die2}
    \definition{s.}{nota diplomática}
  \end{phonetics}
\end{entry}

\begin{entry}{造}{10}[Radical 辵]
  \begin{phonetics}{造}{zao4}
    \definition{clas.}{para colheitas, cultivos}
    \definition{v.}{criar | construir | fabricar | inventar}
  \end{phonetics}
\end{entry}

\begin{entry}{部}{10}[Radical 邑]
  \begin{phonetics}{部}{bu4}
    \definition{clas.}{para obras de literatura, filmes, máquinas etc.}
    \definition[根]{s.}{departamento | divisão | ministério | seção | parte | tropas}
  \end{phonetics}
\end{entry}

\begin{entry}{部下}{10,3}
  \begin{phonetics}{部下}{bu4xia4}
    \definition{s.}{subordinado | tropas sob comando de alguém}
  \end{phonetics}
\end{entry}

\begin{entry}{部门}{10,3}
  \begin{phonetics}{部门}{bu4men2}
    \definition[个]{s.}{filial | departamento | divisão | seção}
  \end{phonetics}
\end{entry}

\begin{entry}{部分}{10,4}
  \begin{phonetics}{部分}{bu4fen5}[][HSK 2]
    \definition[个]{s.}{parte | parte de | uma parte de | pedaço | secção}
  \end{phonetics}
\end{entry}

\begin{entry}{部队}{10,4}
  \begin{phonetics}{部队}{bu4dui4}
    \definition[个]{s.}{exército | forças armadas | tropas | unidades}
  \end{phonetics}
\end{entry}

\begin{entry}{部族}{10,11}
  \begin{phonetics}{部族}{bu4zu2}
    \definition{adj.}{tribal}
    \definition{s.}{tribo}
  \end{phonetics}
\end{entry}

\begin{entry}{部属}{10,12}
  \begin{phonetics}{部属}{bu4shu3}
    \definition{s.}{afiliado a um ministério | subordinado | tropas sob comando de alguém}
  \end{phonetics}
\end{entry}

\begin{entry}{部署}{10,13}
  \begin{phonetics}{部署}{bu4shu3}
    \definition{s.}{implantação}
    \definition{v.}{implantar}
  \end{phonetics}
\end{entry}

\begin{entry}{都}{10}[Radical 邑]
  \begin{phonetics}{都}{dou1}
    \definition{adv.}{todos | ambos | inteiramente | até | já (usado para dar ênfase) | (não) em tudo}
  \end{phonetics}
  \begin{phonetics}{都}{du1}
    \definition*{s.}{sobrenome Du}
    \definition{s.}{capital | metrópole}
  \end{phonetics}
\end{entry}

\begin{entry}{配}{10}[Radical 酉]
  \begin{phonetics}{配}{pei4}
    \definition{v.}{alocar | merecer | caber | juntar-se | compensar (uma receita) | combinar | acasalar | misturar}
  \end{phonetics}
\end{entry}

\begin{entry}{配合}{10,6}
  \begin{phonetics}{配合}{pei4he2}
    \definition{v.}{corresponder | ajustar | coordenar | agir em conjunto com | cooperar | combinar partes de uma máquina}
  \end{phonetics}
\end{entry}

\begin{entry}{酒}{10}[Radical 酉]
  \begin{phonetics}{酒}{jiu3}[][HSK 2]
    \definition[杯,瓶,罐,桶,缸]{s.}{bebida alcoólica | vinho (especialmente vinho de arroz) | aguardente | licor | espíritos}
  \end{phonetics}
\end{entry}

\begin{entry}{酒店}{10,8}
  \begin{phonetics}{酒店}{jiu3 dian4}[][HSK 2]
    \definition[家]{s.}{hotel | restaurante}
  \end{phonetics}
\end{entry}

\begin{entry}{酒鬼}{10,9}
  \begin{phonetics}{酒鬼}{jiu3gui3}
    \definition{adj.}{embriagado | ébrio}
    \definition{s.}{bêbado | alcoólatra | borracho}
  \end{phonetics}
\end{entry}

\begin{entry}{酒馆}{10,11}
  \begin{phonetics}{酒馆}{jiu3guan3}
    \definition{s.}{bar | taverna | adega}
  \end{phonetics}
\end{entry}

\begin{entry}{钱}{10}[Radical 金]
  \begin{phonetics}{钱}{qian2}[][HSK 1]
    \definition*{s.}{sobrenome Qian}
    \definition[笔]{s.}{moeda | dinheiro}
  \end{phonetics}
\end{entry}

\begin{entry}{钱包}{10,5}
  \begin{phonetics}{钱包}{qian2bao1}[][HSK 1]
    \definition{s.}{carteira | bolsa}
  \end{phonetics}
\end{entry}

\begin{entry}{钻石}{10,5}
  \begin{phonetics}{钻石}{zuan4shi2}
    \definition[颗]{s.}{diamante}
  \end{phonetics}
\end{entry}

\begin{entry}{钻戒}{10,7}
  \begin{phonetics}{钻戒}{zuan4jie4}
    \definition[只]{s.}{anel de diamante}
  \end{phonetics}
\end{entry}

\begin{entry}{钿}{10}[Radical 金]
  \begin{phonetics}{钿}{dian4}
    \definition{s.}{ornamento incrustado antigo em forma de flor}
    \definition{v.}{incrustar com ouro, prata, etc.}
  \end{phonetics}
  \begin{phonetics}{钿}{tian2}
    \definition{s.}{(dialeto) moeda, dinheiro}
  \end{phonetics}
\end{entry}

\begin{entry}{铁}{10}[Radical 金]
  \begin{phonetics}{铁}{tie3}
    \definition*{s.}{sobrenome Tie}
    \definition{adj.}{duro | forte | violento | inabalável | determinado | (gíria) apertado}
    \definition{s.}{ferro (metal) | arma}
  \end{phonetics}
\end{entry}

\begin{entry}{铁轨}{10,6}
  \begin{phonetics}{铁轨}{tie3gui3}
    \definition[根]{s.}{trilho | trilho ferroviário}
  \end{phonetics}
\end{entry}

\begin{entry}{铁路}{10,13}
  \begin{phonetics}{铁路}{tie3lu4}
    \definition[条]{s.}{ferrovia}
  \end{phonetics}
\end{entry}

\begin{entry}{阅兵式}{10,7,6}
  \begin{phonetics}{阅兵式}{yue4bing1shi4}
    \definition{s.}{parada militar}
  \end{phonetics}
\end{entry}

\begin{entry}{阅览室}{10,9,9}
  \begin{phonetics}{阅览室}{yue4lan3shi4}
    \definition[间]{s.}{sala de leitura}
  \end{phonetics}
\end{entry}

\begin{entry}{阅读}{10,10}
  \begin{phonetics}{阅读}{yue4du2}
    \definition{s.}{leitura}
    \definition{v.}{ler}
  \end{phonetics}
\end{entry}

\begin{entry}{阅读广度}{10,10,3,9}
  \begin{phonetics}{阅读广度}{yue4du2guang3du4}
    \definition{s.}{intervalo de leitura}
  \end{phonetics}
\end{entry}

\begin{entry}{阅读时间}{10,10,7,7}
  \begin{phonetics}{阅读时间}{yue4du2shi2jian1}
    \definition{s.}{tempo de leitura}
  \end{phonetics}
\end{entry}

\begin{entry}{阅读理解}{10,10,11,13}
  \begin{phonetics}{阅读理解}{yue4du2li3jie3}
    \definition{s.}{compreensão de leitura}
  \end{phonetics}
\end{entry}

\begin{entry}{阅读装置}{10,10,12,13}
  \begin{phonetics}{阅读装置}{yue4du2zhuang1zhi4}
    \definition{s.}{dispositivo de leitura (por exemplo, para códigos de barras, etiquetas RFID, etc.)}
  \end{phonetics}
\end{entry}

\begin{entry}{阅读障碍}{10,10,13,13}
  \begin{phonetics}{阅读障碍}{yue4du2zhang4ai4}
    \definition{s.}{dislexia}
  \end{phonetics}
\end{entry}

\begin{entry}{阅读器}{10,10,16}
  \begin{phonetics}{阅读器}{yue4du2qi4}
    \definition{s.}{leitor (\emph{software})}
  \end{phonetics}
\end{entry}

\begin{entry}{陪}{10}[Radical 阜]
  \begin{phonetics}{陪}{pei2}
    \definition{v.}{acompanhar | ajudar | fazer companhia a alguém}
  \end{phonetics}
\end{entry}

\begin{entry}{陵园}{10,7}
  \begin{phonetics}{陵园}{ling2yuan2}
    \definition{s.}{cemitério}
  \end{phonetics}
\end{entry}

\begin{entry}{陷入}{10,2}
  \begin{phonetics}{陷入}{xian4ru4}
    \definition{v.}{afundar | ser pego em | pousar (em uma situação)}
  \end{phonetics}
\end{entry}

\begin{entry}{难}{10}[Radical 隹]
  \begin{phonetics}{难}{nan2}
    \definition{adj.}{difícil}
    \definition{s.}{dificuldade}
  \end{phonetics}
  \begin{phonetics}{难}{nan4}
    \definition{s.}{desastre}
    \definition{v.}{repreender}
  \end{phonetics}
\end{entry}

\begin{entry}{难过}{10,6}
  \begin{phonetics}{难过}{nan2guo4}[][HSK 2]
    \definition{adj.}{triste | ruim | pesaroso | arrependido | difícil}
  \end{phonetics}
\end{entry}

\begin{entry}{难听}{10,7}
  \begin{phonetics}{难听}{nan2 ting1}[][HSK 2]
    \definition{adj.}{desagradável de ouvir | ofensivo | grosseiro | escandaloso}
  \end{phonetics}
\end{entry}

\begin{entry}{难受}{10,8}
  \begin{phonetics}{难受}{nan2shou4}[][HSK 2]
    \definition{adj.}{sofrer dor | sentir-se mal | desconfortável | sentir-se infeliz}
  \end{phonetics}
\end{entry}

\begin{entry}{难度}{10,9}
  \begin{phonetics}{难度}{nan2du4}
    \definition{s.}{grau de dificuldade}
  \end{phonetics}
\end{entry}

\begin{entry}{难看}{10,9}
  \begin{phonetics}{难看}{nan2 kan4}[][HSK 2]
    \definition{adj.}{feio | antiestético | vergonhoso | embaraçoso | vergonhoso}
  \end{phonetics}
\end{entry}

\begin{entry}{难道}{10,12}
  \begin{phonetics}{难道}{nan2dao4}
    \definition{adv.}{indica uma pergunta retórica | certamente não significa que\dots | é possível que\dots}
  \end{phonetics}
\end{entry}

\begin{entry}{难题}{10,15}
  \begin{phonetics}{难题}{nan2 ti2}[][HSK 2]
    \definition[出]{s.}{desafio | problema difícil | pergunta difícil}
  \end{phonetics}
\end{entry}

\begin{entry}{顽强}{10,12}
  \begin{phonetics}{顽强}{wan2qiang2}
    \definition{adj.}{persistente | tenaz | difícil de derrotar}
  \end{phonetics}
\end{entry}

\begin{entry}{顾客}{10,9}
  \begin{phonetics}{顾客}{gu4ke4}[][HSK 2]
    \definition[位]{s.}{cliente}
  \end{phonetics}
\end{entry}

\begin{entry}{顿}{10}[Radical 頁]
  \begin{phonetics}{顿}{dun4}
    \definition{clas.}{para refeições, espancamentos, repreensões, etc.: tempo, luta, feitiço, refeição}
    \definition{v.}{prostrar-se | pausar | bater (o pé)}
  \end{phonetics}
\end{entry}

\begin{entry}{预}{10}[Radical 頁]
  \begin{phonetics}{预}{yu4}
    \definition{adv.}{antecipadamente}
    \definition{v.}{avançar | preparar}
  \end{phonetics}
\end{entry}

\begin{entry}{预见}{10,4}
  \begin{phonetics}{预见}{yu4jian4}
    \definition{s.}{previsão; intuição; vislumbre}
    \definition{v.}{prever}
  \end{phonetics}
\end{entry}

\begin{entry}{预付}{10,5}
  \begin{phonetics}{预付}{yu4fu4}
    \definition{s.}{pré-pago}
    \definition{v.}{pagar antecipadamente}
  \end{phonetics}
\end{entry}

\begin{entry}{预约}{10,6}
  \begin{phonetics}{预约}{yu4yue1}
    \definition{s.}{reserva}
    \definition{v.}{agendar | marcar compromisso}
  \end{phonetics}
\end{entry}

\begin{entry}{预判}{10,7}
  \begin{phonetics}{预判}{yu4pan4}
    \definition{v.}{prever | antecipar}
  \end{phonetics}
\end{entry}

\begin{entry}{预报}{10,7}
  \begin{phonetics}{预报}{yu4bao4}
    \definition{s.}{previsão (meteorológica) | boletim meteorológico}
    \definition{v.}{prever (o tempo)}
  \end{phonetics}
\end{entry}

\begin{entry}{预定}{10,8}
  \begin{phonetics}{预定}{yu4ding4}
    \definition{v.}{agendar com antecedência}
  \end{phonetics}
\end{entry}

\begin{entry}{预购}{10,8}
  \begin{phonetics}{预购}{yu4gou4}
    \definition{s.}{compra antecipada}
    \definition{v.}{comprar antecipadamente}
  \end{phonetics}
\end{entry}

\begin{entry}{预祝}{10,9}
  \begin{phonetics}{预祝}{yu4zhu4}
    \definition{v.}{parabenizar de antemão | oferecer os melhores votos para}
  \end{phonetics}
\end{entry}

\begin{entry}{预览}{10,9}
  \begin{phonetics}{预览}{yu4lan3}
    \definition{s.}{visualização}
    \definition{v.}{visualizar}
  \end{phonetics}
\end{entry}

\begin{entry}{预留}{10,10}
  \begin{phonetics}{预留}{yu4liu2}
    \definition{v.}{separar | reservar}
  \end{phonetics}
\end{entry}

\begin{entry}{预配}{10,10}
  \begin{phonetics}{预配}{yu4pei4}
    \definition{s.}{pré-alocado | pré-cabeado}
    \definition{v.}{pré-alocar | pré-cabear}
  \end{phonetics}
\end{entry}

\begin{entry}{预谋}{10,11}
  \begin{phonetics}{预谋}{yu4mou2}
    \definition{adj.}{premeditado}
    \definition{v.}{planejar algo com antecedência (especialmente um crime)}
  \end{phonetics}
\end{entry}

\begin{entry}{预提}{10,12}
  \begin{phonetics}{预提}{yu4ti2}
    \definition{s.}{retenção}
    \definition{v.}{reter (imposto)}
  \end{phonetics}
\end{entry}

\begin{entry}{预感}{10,13}
  \begin{phonetics}{预感}{yu4gan3}
    \definition{s.}{premonição}
    \definition{v.}{ter uma premonição}
  \end{phonetics}
\end{entry}

\begin{entry}{预警}{10,19}
  \begin{phonetics}{预警}{yu4jing3}
    \definition{s.}{aviso | aviso antecipado}
  \end{phonetics}
\end{entry}

\begin{entry}{饿}{10}[Radical 食]
  \begin{phonetics}{饿}{e4}[][HSK 1]
    \definition{adj.}{faminto}
    \definition{s.}{fome}
    \definition{v.}{morrer de fome}
  \end{phonetics}
\end{entry}

\begin{entry}{高}{10}[Radical ⾼][Kangxi 189]
  \begin{phonetics}{高}{gao1}[][HSK 1]
    \definition*{s.}{sobrenome Gao}
    \definition{adj.}{alto | acima da média}
    \definition{pron.}{Seu (honorífico)}
  \end{phonetics}
\end{entry}

\begin{entry}{高中}{10,4}
  \begin{phonetics}{高中}{gao1 zhong1}[][HSK 2]
    \definition{s.}{escola secundária | escola de segundo grau}
  \end{phonetics}
\end{entry}

\begin{entry}{高手}{10,4}
  \begin{phonetics}{高手}{gao1shou3}
    \definition{s.}{\emph{expert} | mestre}
  \end{phonetics}
\end{entry}

\begin{entry}{高尔夫}{10,5,4}
  \begin{phonetics}{高尔夫}{gao1'er3fu1}
    \definition{s.}{(empréstimo linguístico) \emph{golf}}
  \end{phonetics}
\end{entry}

\begin{entry}{高兴}{10,6}
  \begin{phonetics}{高兴}{gao1xing4}[][HSK 1]
    \definition{adj.}{feliz | contente | disposto (a fazer alguma coisa) | de bom humor}
  \end{phonetics}
\end{entry}

\begin{entry}{高级}{10,6}
  \begin{phonetics}{高级}{gao1ji2}[][HSK 2]
    \definition{adj.}{sênior | alto escalão | alto nível | alto grau | grau superior | alta qualidade | avançado}
  \end{phonetics}
\end{entry}

\begin{entry}{高效}{10,10}
  \begin{phonetics}{高效}{gao1xiao4}
    \definition{adj.}{eficiente | altamente eficaz}
  \end{phonetics}
\end{entry}

\begin{entry}{高楼}{10,13}
  \begin{phonetics}{高楼}{gao1lou2}
    \definition[座]{s.}{edifício alto | edifício de muitos andares | arranha-céu}
  \end{phonetics}
\end{entry}

\begin{entry}{高跟鞋}{10,13,15}
  \begin{phonetics}{高跟鞋}{gao1gen1xie2}
    \definition{s.}{sapatos de salto alto}
  \end{phonetics}
\end{entry}

\begin{entry}{鸭}{10}[Radical 鳥]
  \begin{phonetics}{鸭}{ya1}
    \definition[只]{s.}{pato | (gíria) prostituto}
  \end{phonetics}
\end{entry}

\begin{entry}{鸭子}{10,3}
  \begin{phonetics}{鸭子}{ya1zi5}
    \definition[只]{s.}{pato | (gíria) prostituto}
  \end{phonetics}
\end{entry}

\begin{entry}{鸵鸟}{10,5}
  \begin{phonetics}{鸵鸟}{tuo2niao3}
    \definition{s.}{avestruz}
  \end{phonetics}
\end{entry}

%%%%% EOF %%%%%


%%%
%%% 11画
%%%

\section*{11画}\addcontentsline{toc}{section}{11画}

\begin{entry}{假}{11}{⼈}
  \begin{phonetics}{假}{jia3}[][HSK 2]
    \definition{adj.}{falso; artificial}
    \definition{conj.}{se; caso; no caso de; conecta frases, expressa relação hipotética, geralmente usada com 如, 若 e 使, equivalente a 如果}
    \definition[个,天]{s.}{falsificação; coisas falsas, irreais ou forjadas}
    \definition{v.}{emprestar | valer-se de; aproveitar; utilizar | supor; presumir; pressupor}
  \seealsoref{如}{ru2}
  \seealsoref{如果}{ru2guo3}
  \seealsoref{若}{ruo4}
  \seealsoref{使}{shi3}
  \end{phonetics}
  \begin{phonetics}{假}{jia4}
    \definition[个,天]{s.}{feriado; férias; período de suspensão temporária do trabalho ou dos estudos, legal ou aprovado | licença; afastamento temporário; período de licença temporária do trabalho ou dos estudos, após aprovação}
  \end{phonetics}
\end{entry}

\begin{entry}{假如}{11,6}{⼈、⼥}
  \begin{phonetics}{假如}{jia3ru2}[][HSK 4]
    \definition{conj.}{se; supondo; no caso}
  \end{phonetics}
\end{entry}

\begin{entry}{假声}{11,7}{⼈、⼠}
  \begin{phonetics}{假声}{jia3sheng1}
    \definition{s.}{falsete}
  \seealsoref{真声}{zhen1sheng1}
  \end{phonetics}
\end{entry}

\begin{entry}{假证件}{11,7,6}{⼈、⾔、⼈}
  \begin{phonetics}{假证件}{jia3zheng4jian4}
    \definition{s.}{documentos falsos}
  \end{phonetics}
\end{entry}

\begin{entry}{假使}{11,8}{⼈、⼈}
  \begin{phonetics}{假使}{jia3shi3}
    \definition{conj.}{se | supondo | em caso}
  \end{phonetics}
\end{entry}

\begin{entry}{假的}{11,8}{⼈、⽩}
  \begin{phonetics}{假的}{jia3de5}
    \definition{adj.}{falso | substituto | simulado}
  \end{phonetics}
\end{entry}

\begin{entry}{假期}{11,12}{⼈、⽉}
  \begin{phonetics}{假期}{jia4 qi1}[][HSK 2]
    \definition[个,段,次,种]{s.}{férias; feriados; período de licença}
  \end{phonetics}
\end{entry}

\begin{entry}{偏}{11}{⼈}
  \begin{phonetics}{偏}{pian1}[][HSK 6]
    \definition{adj.}{parcial; preconceituoso; injusto; focando apenas em um lado | torto; inclinado (oposto de 正) | não dominante; auxiliar | remoto; periférico; longe do centro; incomum}
    \definition{adv.}{intencionalmente; insistentemente; persistentemente; indica ir intencionalmente contra o senso comum ou a solicitação de outra pessoa}
    \definition{expr.}{uma expressão educada para indicar que alguém já tomou chá ou comeu}
    \definition{v.}{divergir; não ser igual a; ser diferente de; exceder ou ficar aquém dos padrões normais | desviar-se; afastar-se; sair na direção certa}
  \seealsoref{正}{zheng4}
  \end{phonetics}
\end{entry}

\begin{entry}{偏偏}{11,11}{⼈、⼈}
  \begin{phonetics}{偏偏}{pian1pian1}
    \definition{adv.}{voluntariamente | insistentemente | persistentemente | ao contrário da expectativa | infelizmente (indicando que alguma coisa aconteceu ao contrário do que se esperava) | teimosamente (indicando que algo é o oposto ao que seria normal ou razoável) | precisamente (indicando que alguém ou um grupo é escolhido)}
  \end{phonetics}
\end{entry}

\begin{entry}{做}{11}{⼈}
  \begin{phonetics}{做}{zuo4}[][HSK 1]
    \definition{v.}{fabricar; produzir; criar | escrever; compor | fazer; trabalhar em; dedicar-se a; exercer uma determinada profissão ou atividade | realizar uma festa em família; comemorar | ser; tornar-se; agir como; atuar como | ser usado como | formar ou estabelecer um relacionamento; conectar-se (em algum tipo de relação) | fingir (alguma coisa) | cozinhar; preparar}
  \end{phonetics}
\end{entry}

\begin{entry}{做生活}{11,5,9}{⼈、⽣、⽔}
  \begin{phonetics}{做生活}{zuo4sheng1huo2}
    \definition{v.}{fazer tabalhos manuais}
  \end{phonetics}
\end{entry}

\begin{entry}{做戏}{11,6}{⼈、⼽}
  \begin{phonetics}{做戏}{zuo4xi4}
    \definition{v.}{atuar em uma peça | fazer uma peça}
  \end{phonetics}
\end{entry}

\begin{entry}{做作}{11,7}{⼈、⼈}
  \begin{phonetics}{做作}{zuo4zuo5}
    \definition{adj.}{afetado | artificial}
  \end{phonetics}
\end{entry}

\begin{entry}{做饭}{11,7}{⼈、⾷}
  \begin{phonetics}{做饭}{zuo4 fan4}[][HSK 2]
    \definition{v.}{cozinhar; preparar uma refeição; cozinhar refeições e transformar alimentos crus em alimentos cozidos}
  \end{phonetics}
\end{entry}

\begin{entry}{做到}{11,8}{⼈、⼑}
  \begin{phonetics}{做到}{zuo4 dao4}[][HSK 2]
    \definition{v.}{alcançar; realizar; atingir um determinado objetivo; atingir um determinado padrão}
  \end{phonetics}
\end{entry}

\begin{entry}{做法}{11,8}{⼈、⽔}
  \begin{phonetics}{做法}{zuo4fa3}[][HSK 2]
    \definition[种,个]{s.}{método; maneira de fazer algo; métodos de lidar com coisas ou fazer coisas}
  \end{phonetics}
\end{entry}

\begin{entry}{做客}{11,9}{⼈、⼧}
  \begin{phonetics}{做客}{zuo4 ke4}[][HSK 3]
    \definition{v.}{visitar; ser um convidado; ser hóspede}
  \end{phonetics}
\end{entry}

\begin{entry}{做活}{11,9}{⼈、⽔}
  \begin{phonetics}{做活}{zuo4huo2}
    \definition{v.}{trabalhar para ganhar a vida (especialmente de mulher costureira)}
  \end{phonetics}
\end{entry}

\begin{entry}{做梦}{11,11}{⼈、⼣}
  \begin{phonetics}{做梦}{zuo4 meng4}[][HSK 4]
    \definition{s.}{sonho; ilusões e visões na consciência durante o sono}
    \definition{v.}{sonhar; ter um sonho | sonhar acordado, ter um sonho impossível (parábola de fantasias irrealistas)}[别​做​梦​了​,她​不​会​嫁​给​你​的​。___Pare de sonhar, ela não se casará com você.]
  \end{phonetics}
\end{entry}

\begin{entry}{做眼}{11,11}{⼈、⽬}
  \begin{phonetics}{做眼}{zuo4yan3}
    \definition{v.}{agir como um guia | trabalhar como espião}
  \end{phonetics}
\end{entry}

\begin{entry}{停}{11}{⼈}
  \begin{phonetics}{停}{ting2}[][HSK 2]
    \definition{adj.}{pronto; resolvido; bem organizado}
    \definition{clas.}{usado para partes (de um total); porções}
    \definition{v.}{parar; interromper; cessar; fazer uma pausa | permanecer; ficar; fazer uma parada (para descansar) | estacionar; ancorar; atracar}
  \end{phonetics}
\end{entry}

\begin{entry}{停下}{11,3}{⼈、⼀}
  \begin{phonetics}{停下}{ting2 xia4}[][HSK 4]
    \definition{v.}{encerrar; desligar; parar}
  \end{phonetics}
\end{entry}

\begin{entry}{停工}{11,3}{⼈、⼯}
  \begin{phonetics}{停工}{ting2gong1}
    \definition{v.}{parar de trabalhar | parar a produção}
  \end{phonetics}
\end{entry}

\begin{entry}{停办}{11,4}{⼈、⼒}
  \begin{phonetics}{停办}{ting2ban4}
    \definition{v.}{cancelar | sair do negócio | desligar | terminar}
  \end{phonetics}
\end{entry}

\begin{entry}{停止}{11,4}{⼈、⽌}
  \begin{phonetics}{停止}{ting2 zhi3}[][HSK 3]
    \definition{v.}{parar; suspender; cessar; cancelar}
  \end{phonetics}
\end{entry}

\begin{entry}{停火}{11,4}{⼈、⽕}
  \begin{phonetics}{停火}{ting2huo3}
    \definition{s.}{cessar-fogo}
    \definition{v.+compl.}{cessar fogo}
  \end{phonetics}
\end{entry}

\begin{entry}{停车}{11,4}{⼈、⾞}
  \begin{phonetics}{停车}{ting2 che1}[][HSK 2]
    \definition{v.}{(veículo) parar; frear | estacionar o veículo | parar; deixar de funcionar}
  \end{phonetics}
\end{entry}

\begin{entry}{停车场}{11,4,6}{⼈、⾞、⼟}
  \begin{phonetics}{停车场}{ting2 che1 chang3}[][HSK 2]
    \definition[个]{s.}{estacionamento; área de estacionamento; local para estacionamento de veículos}
  \end{phonetics}
\end{entry}

\begin{entry}{停业}{11,5}{⼈、⼀}
  \begin{phonetics}{停业}{ting2ye4}
    \definition{v.}{cessar a negociação (temporária ou permanentemente) | fechar}
  \end{phonetics}
\end{entry}

\begin{entry}{停用}{11,5}{⼈、⽤}
  \begin{phonetics}{停用}{ting2yong4}
    \definition{v.}{desabilitar | descontinuar | parar de usar | suspender}
  \end{phonetics}
\end{entry}

\begin{entry}{停电}{11,5}{⼈、⽥}
  \begin{phonetics}{停电}{ting2dian4}
    \definition{s.}{corte de energia}
    \definition{v.}{ter uma falha de energia}
  \end{phonetics}
\end{entry}

\begin{entry}{停当}{11,6}{⼈、⼹}
  \begin{phonetics}{停当}{ting2dang5}
    \definition{adj.}{realizado | preparado | assentado}
  \end{phonetics}
\end{entry}

\begin{entry}{停息}{11,10}{⼈、⼼}
  \begin{phonetics}{停息}{ting2xi1}
    \definition{v.}{cessar | parar}
  \end{phonetics}
\end{entry}

\begin{entry}{停留}{11,10}{⼈、⽥}
  \begin{phonetics}{停留}{ting2 liu2}[][HSK 5]
    \definition{v.}{permanecer; ficar por muito tempo; parar temporariamente em algum lugar, sem continuar avançando | permanecer; parar por um longo tempo; parar em um determinado estágio ou nível, sem evoluir}
  \end{phonetics}
\end{entry}

\begin{entry}{停课}{11,10}{⼈、⾔}
  \begin{phonetics}{停课}{ting2ke4}
    \definition{v.}{fechar (escola) | parar as aulas}
  \end{phonetics}
\end{entry}

\begin{entry}{停歇}{11,13}{⼈、⽋}
  \begin{phonetics}{停歇}{ting2xie1}
    \definition{v.}{parar para descansar}
  \end{phonetics}
\end{entry}

\begin{entry}{偶}{11}{⼈}
  \begin{phonetics}{偶}{ou3}
    \definition{adv.}{por acaso; por acidente; de vez em quando; ocasionalmente | par; número par; pareado (em oposição a 奇)}
    \definition{s.}{imagem; ídolo; figuras feitas de madeira, barro, etc. | companheiro; cônjuge; parceiro; refere-se a um casal ou a um dos casais}
  \seealsoref{奇}{qi2}
  \end{phonetics}
\end{entry}

\begin{entry}{偶尔}{11,5}{⼈、⼩}
  \begin{phonetics}{偶尔}{ou3'er3}[][HSK 5]
    \definition{adj.}{ocasional}
    \definition{adv.}{ocasionalmente; de vez em quando; às vezes}
  \end{phonetics}
\end{entry}

\begin{entry}{偶然}{11,12}{⼈、⽕}
  \begin{phonetics}{偶然}{ou3ran2}[][HSK 5]
    \definition{adj.}{acidental; ocasional}
    \definition{adv.}{por acaso; acidentalmente; sem querer; inesperadamente | ocasionalmente; de vez em quando; às vezes}
  \end{phonetics}
\end{entry}

\begin{entry}{偶像}{11,13}{⼈、⼈}
  \begin{phonetics}{偶像}{ou3xiang4}[][HSK 5]
    \definition{s.}{ídolo; pessoa amada pelas pessoas; refere-se a uma pessoa que é apreciada por todos e que, em certos aspectos, é digna de admiração e respeito}
  \end{phonetics}
\end{entry}

\begin{entry}{偷}{11}{⼈}
  \begin{phonetics}{偷}{tou1}[][HSK 5]
    \definition{adv.}{furtivamente; secretamente; às escondidas}
    \definition{s.}{ladrão; furtador}
    \definition{v.}{roubar; furtar; levar sem pagar; roubar os bens alheios às escondidas | encontrar (tempo) | deixar-se levar; viver apenas para o presente, sem se preocupar com o futuro}
  \end{phonetics}
\end{entry}

\begin{entry}{偷安}{11,6}{⼈、⼧}
  \begin{phonetics}{偷安}{tou1'an1}
    \definition{v.}{buscar facilidade temporária}
  \end{phonetics}
\end{entry}

\begin{entry}{偷听}{11,7}{⼈、⼝}
  \begin{phonetics}{偷听}{tou1ting1}
    \definition{v.}{bisbilhotar; monitorar (secretamente)}
  \end{phonetics}
\end{entry}

\begin{entry}{偷窃}{11,9}{⼈、⽳}
  \begin{phonetics}{偷窃}{tou1qie4}
    \definition{v.}{furtar | roubar}
  \end{phonetics}
\end{entry}

\begin{entry}{偷偷}{11,11}{⼈、⼈}
  \begin{phonetics}{偷偷}{tou1 tou1}[][HSK 5]
    \definition{adv.}{secretamente; dissimuladamente; furtivamente; às escondidas}
  \end{phonetics}
\end{entry}

\begin{entry}{偷情}{11,11}{⼈、⼼}
  \begin{phonetics}{偷情}{tou1qing2}
    \definition{v.}{manter um caso de amor clandestino}
  \end{phonetics}
\end{entry}

\begin{entry}{偷袭}{11,11}{⼈、⾐}
  \begin{phonetics}{偷袭}{tou1xi2}
    \definition{s.}{ataque surpresa}
    \definition{v.}{montar um ataque furtivo | invadir}
  \end{phonetics}
\end{entry}

\begin{entry}{偷渡}{11,12}{⼈、⽔}
  \begin{phonetics}{偷渡}{tou1du4}
    \definition{s.}{contrabando | imigração ilegal | clandestino (em um navio)}
    \definition{v.}{executar um bloqueio | roubar através da fronteira internacional}
  \end{phonetics}
\end{entry}

\begin{entry}{偷税}{11,12}{⼈、⽲}
  \begin{phonetics}{偷税}{tou1shui4}
    \definition{s.}{evasão fiscal}
  \end{phonetics}
\end{entry}

\begin{entry}{偸}{11}{⼈}
  \begin{phonetics}{偸}{tou1}
    \variantof{偷}
  \end{phonetics}
\end{entry}

\begin{entry}{减}{11}{⼎}
  \begin{phonetics}{减}{jian3}[][HSK 4]
    \definition*{s.}{Sobrenome Jian}
    \definition{v.}{subtrair; remover uma parte da quantidade original | reduzir; diminuir; cortar}
  \end{phonetics}
\end{entry}

\begin{entry}{减少}{11,4}{⼎、⼩}
  \begin{phonetics}{减少}{jian3shao3}[][HSK 4]
    \definition{v.}{cair; reduzir; diminuir; subtrair uma parte}
  \end{phonetics}
\end{entry}

\begin{entry}{减肥}{11,8}{⼎、⾁}
  \begin{phonetics}{减肥}{jian3fei2}[][HSK 4]
    \definition{v.+compl.}{perder peso; dieta, exercícios, medicamentos, massagem, cirurgia, etc., para reduzir o excesso de gordura corporal, de modo que o grau de obesidade seja reduzido}
  \end{phonetics}
\end{entry}

\begin{entry}{减轻}{11,9}{⼎、⾞}
  \begin{phonetics}{减轻}{jian3 qing1}[][HSK 5]
    \definition{v.}{aliviar; remeter; clarear; facilitar; mitigar}
  \end{phonetics}
\end{entry}

\begin{entry}{剪}{11}{⼑}
  \begin{phonetics}{剪}{jian3}[][HSK 5]
    \definition[把]{s.}{tesouras; tesouras de poda; cortadores | pinças; tenazes}
    \definition{v.}{cortar; aparar; tosquiar; cortar (com uma tesoura) | exterminar; eliminar; acabar com}
  \end{phonetics}
\end{entry}

\begin{entry}{剪刀}{11,2}{⼑、⼑}
  \begin{phonetics}{剪刀}{jian3dao1}[][HSK 5]
    \definition[把]{s.}{tesoura; instrumento de ferro para cortar tecido, papel, barbante, etc., com duas lâminas interligadas que podem ser abertas e fechadas}
  \end{phonetics}
\end{entry}

\begin{entry}{剪子}{11,3}{⼑、⼦}
  \begin{phonetics}{剪子}{jian3 zi5}[][HSK 5]
    \definition[把]{s.}{cortador; tosquiadeira | tesoura}
  \end{phonetics}
\end{entry}

\begin{entry}{副}{11}{⼑}
  \begin{phonetics}{副}{fu4}[][HSK 6]
    \definition{adj.}{segundo em exercício; deputado; auxiliar | subsidiário; incidental; secundário}
    \definition{clas.}{usado para conjuntos completos de itens; usado para \emph{kits} | usado para expressões faciais | usado para som ou voz}
    \definition{pref.}{vice-}
    \definition{s.}{assistente; ajudante; auxiliar; posição auxiliar; pessoa que ocupa uma posição auxiliar}
    \definition{v.}{ajustar; corresponder a; conformar-se a}
  \end{phonetics}
\end{entry}

\begin{entry}{副研}{11,9}{⼑、⽯}
  \begin{phonetics}{副研}{fu4yan2}
    \definition{s.}{pesquisador adjunto}
  \end{phonetics}
\end{entry}

\begin{entry}{唬}{11}{⼝}
  \begin{phonetics}{唬}{hu3}
    \definition{v.}{blefar, exagerar para assustar ou confundir}
  \end{phonetics}
\end{entry}

\begin{entry}{售}{11}{⼝}
  \begin{phonetics}{售}{shou4}
    \definition{v.}{vender | fazer (o plano, truque, etc.) funcionar; continuar (as intrigas) | realizar (intrigas)}
  \end{phonetics}
\end{entry}

\begin{entry}{售货员}{11,8,7}{⼝、⾙、⼝}
  \begin{phonetics}{售货员}{shou4huo4yuan2}[][HSK 4]
    \definition[个]{s.}{vendedor; balconista; assistente de loja; equipe que vende produtos em lojas}
  \end{phonetics}
\end{entry}

\begin{entry}{唯}{11}{⼝}
  \begin{phonetics}{唯}{wei2}
    \definition{adv.}{somente; sozinho | ainda; somente; exceto que}
  \end{phonetics}
  \begin{phonetics}{唯}{wei3}
    \definition{interj.}{Sim!; Yea!; significa uma palavra que indica acordo}
  \end{phonetics}
\end{entry}

\begin{entry}{唯一}{11,1}{⼝、⼀}
  \begin{phonetics}{唯一}{wei2yi1}[][HSK 5]
    \definition{adj.}{único; exclusivo; singular}
  \end{phonetics}
\end{entry}

\begin{entry}{唱}{11}{⼝}
  \begin{phonetics}{唱}{chang4}[][HSK 1]
    \definition*{s.}{Sobrenome Chang}
    \definition{s.}{uma música ou uma parte cantada de uma ópera chinesa; canções; letras de óperas tradicionais}
    \definition{v.}{cantar; seguir o ritmo da música | chorar; chamar; gritar, falar ou recitar em voz alta}
  \end{phonetics}
\end{entry}

\begin{entry}{唱片}{11,4}{⼝、⽚}
  \begin{phonetics}{唱片}{chang4 pian4}[][HSK 4]
    \definition[枚,张]{s.}{disco; disco feito de goma-laca, plástico, etc. com ranhuras em espiral na superfície para registrar alterações no som que podem reproduzir o som gravado em um fonógrafo}
  \end{phonetics}
\end{entry}

\begin{entry}{唱歌}{11,14}{⼝、⽋}
  \begin{phonetics}{唱歌}{chang4 ge1}[][HSK 1]
    \definition{v.+compl.}{cantar (uma música); emitir sons com entonação ritmada e melodiosa; emitir sons (musicais) com a boca; emitir sons de acordo com a melodia}
  \end{phonetics}
\end{entry}

\begin{entry}{唾}{11}{⼝}
  \begin{phonetics}{唾}{tuo4}
    \definition[口]{s.}{saliva; cuspe}
    \definition{v.}{cuspir (mostrar desprezo) | rejeitar}
  \end{phonetics}
\end{entry}

\begin{entry}{唾骂}{11,9}{⼝、⾺}
  \begin{phonetics}{唾骂}{tuo4ma4}
    \definition{v.}{insultar | amaldiçoar}
  \end{phonetics}
\end{entry}

\begin{entry}{商}{11}{⼝}
  \begin{phonetics}{商}{shang1}
    \definition*{s.}{Dinastia Shang (c. 1600-1046 a.C.) | Shang, nome da estrela da constelação do coração entre as vinte e oito constelações | Sobrenome Shang}
    \definition{s.}{comércio; negócio; a atividade econômica de compra e venda de mercadorias | comerciante; negociante; comerciante; empresário; pessoas que compram e vendem mercadorias | (matemática) quociente;  o resultado de uma operação de divisão em aritmética | uma nota da antiga escala chinesa de cinco tons, correspondente a 2 na notação musical numerada}
    \definition{v.}{discutir; consultar; trocar ideias}
  \end{phonetics}
\end{entry}

\begin{entry}{商人}{11,2}{⼝、⼈}
  \begin{phonetics}{商人}{shang1 ren2}[][HSK 2]
    \definition[位,名]{s.}{comerciante; mercador; empresário; homem de negócios; pessoas que trabalham com a distribuição de mercadorias}
  \end{phonetics}
\end{entry}

\begin{entry}{商业}{11,5}{⼝、⼀}
  \begin{phonetics}{商业}{shang1ye4}[][HSK 3]
    \definition[个,种]{s.}{barganha; negócio; comércio; atividade econômica que circula mercadorias por meio de compra e venda}
  \end{phonetics}
\end{entry}

\begin{entry}{商务}{11,5}{⼝、⼒}
  \begin{phonetics}{商务}{shang1wu4}[][HSK 4]
    \definition[种,类,项]{s.}{negócios; assuntos de negócios; assuntos comerciais}
  \end{phonetics}
\end{entry}

\begin{entry}{商场}{11,6}{⼝、⼟}
  \begin{phonetics}{商场}{shang1 chang3}[][HSK 1]
    \definition[家]{s.}{mercado; shopping center; loja de departamentos; loja de grande área com uma variedade completa de produtos | o mundo dos negócios; referindo-se ao mundo dos negócios | mercado; mercado composto por várias lojas reunidas em um ou vários edifícios interligados}
  \end{phonetics}
\end{entry}

\begin{entry}{商店}{11,8}{⼝、⼴}
  \begin{phonetics}{商店}{shang1dian4}[][HSK 1]
    \definition[间,家,个]{s.}{loja; armazém; local de venda de mercadorias em recinto fechado}
  \end{phonetics}
\end{entry}

\begin{entry}{商品}{11,9}{⼝、⼝}
  \begin{phonetics}{商品}{shang1pin3}[][HSK 3]
    \definition[种,个,件,批]{s.}{bens; mercadoria; \emph{merchande}; os produtos do trabalho produzidos para troca têm a dupla natureza de valor de uso e valor; as mercadorias incorporam diferentes relações de produção em diferentes sistemas sociais}
  \end{phonetics}
\end{entry}

\begin{entry}{商标}{11,9}{⼝、⽊}
  \begin{phonetics}{商标}{shang1biao1}[][HSK 5]
    \definition[个]{s.}{marca; marca registrada; \emph{trademark}; marca ou símbolo (desenho, padrão, texto, etc.) gravado ou impresso na superfície ou embalagem de um produto, para diferenciá-lo de outros produtos semelhantes}
  \end{phonetics}
\end{entry}

\begin{entry}{商贸}{11,9}{⼝、⾙}
  \begin{phonetics}{商贸}{shang1mao4}
    \definition{s.}{comércio}
  \end{phonetics}
\end{entry}

\begin{entry}{商量}{11,12}{⼝、⾥}
  \begin{phonetics}{商量}{shang1liang5}[][HSK 2]
    \definition{v.}{consultar; discutir; conversar sobre; discutir e trocar opiniões}
  \end{phonetics}
\end{entry}

\begin{entry}{啤}{11}{⼝}
  \begin{phonetics}{啤}{pi2}
    \definition{s.}{cerveja}
  \end{phonetics}
\end{entry}

\begin{entry}{啤酒}{11,10}{⼝、⾣}
  \begin{phonetics}{啤酒}{pi2jiu3}[][HSK 3]
    \definition[杯,瓶,罐,桶,缸]{s.}{(empréstimo linguístico) cerveja; uma bebida de baixo teor alcoólico feita de malte de cevada e lúpulo, com espuma e aroma especial}
  \end{phonetics}
\end{entry}

\begin{entry}{啤酒馆}{11,10,11}{⼝、⾣、⾷}
  \begin{phonetics}{啤酒馆}{pi2jiu3guan3}
    \definition{s.}{cervejaria}
  \end{phonetics}
\end{entry}

\begin{entry}{啥}{11}{⼝}
  \begin{phonetics}{啥}{sha2}
    \definition{adv.}{Equivalente a 什么 (dialeto)}
  \end{phonetics}
\end{entry}

\begin{entry}{啦}{11}{⼝}
  \begin{phonetics}{啦}{la1}
    \definition{s.}{(onomatoméia) som de canto, aplausos etc.; usado para palavras como 呼啦啦, 哗啦啦, 哩哩啦啦, etc.}
  \seealsoref{呼啦啦}{hu1 la1 la1}
  \seealsoref{哗啦啦}{hua1la1 la5}
  \seealsoref{哩哩啦啦}{li1 li1 la1 la1}
  \end{phonetics}
  \begin{phonetics}{啦}{la5}[][HSK 6]
    \definition{part.}{uma palavra composta de 了 e 啊, que tem o significado de ambos}
  \seealsoref{啊}{a5}
  \seealsoref{了}{le5}
  \end{phonetics}
\end{entry}

\begin{entry}{啵}{11}{⼝}
  \begin{phonetics}{啵}{bo1}
    \definition{part.}{denotando pedido, comando, etc.; o uso é semelhante ao de 吧, que é mais comum no vernáculo antigo}
    \definition{v.aux.}{indicando uma sugestão, pedido ou comando suave | indicando consentimento ou aprovação | em uma pergunta tendenciosa que pede a confirmação de uma suposição | indicando alguma dúvida na mente do falante | marcando uma pausa após suposições como alternativas}
  \seealsoref{吧}{ba5}
  \end{phonetics}
  \begin{phonetics}{啵}{bo5}
    \definition{part.}{partícula gramaticalmente equivalente a 吧}
  \seealsoref{吧}{ba5}
  \end{phonetics}
\end{entry}

\begin{entry}{圈}{11}{⼞}
  \begin{phonetics}{圈}{juan1}
    \definition{v.}{prender aves e animais de criação | prender; colocar na cadeia, prisão | confinar}
  \end{phonetics}
  \begin{phonetics}{圈}{juan4}
    \definition*{s.}{Sobrenome Juan}
    \definition{s.}{curral; local onde o gado ou as aves são mantidos, geralmente cercado ou murado, alguns com galpões}
  \end{phonetics}
  \begin{phonetics}{圈}{quan1}[][HSK 4]
    \definition[个]{s.}{anel; círculo; refere-se a algo em forma de anel | domínio; grupo; escopo; círculo(s)}
    \definition{v.}{cercar; rodear; circundar | marcar com um círculo}
  \end{phonetics}
\end{entry}

\begin{entry}{圈粉}{11,10}{⼞、⽶}
  \begin{phonetics}{圈粉}{quan1fen3}
    \definition{s.}{(neologismo, coloquial) ganhar alguém como fã, obter novos fãs}
  \end{phonetics}
\end{entry}

\begin{entry}{埦}{11}{⼟}
  \begin{phonetics}{埦}{wan3}
    \variantof{碗}
  \end{phonetics}
\end{entry}

\begin{entry}{培}{11}{⼟}
  \begin{phonetics}{培}{pei2}
    \definition{v.}{aterrar com terra; aterrar | fomentar; treinar | cultivar; crescer e desenvolver-se propositalmente}
  \end{phonetics}
\end{entry}

\begin{entry}{培训}{11,5}{⼟、⾔}
  \begin{phonetics}{培训}{pei2xun4}[][HSK 4]
    \definition{v.}{treinar (trabalhadores técnicos, quadros profissionais, etc.)}
  \end{phonetics}
\end{entry}

\begin{entry}{培训班}{11,5,10}{⼟、⾔、⽟}
  \begin{phonetics}{培训班}{pei2 xun4 ban1}[][HSK 4]
    \definition{s.}{aula de treinamento; curso de treinamento}
  \end{phonetics}
\end{entry}

\begin{entry}{培育}{11,8}{⼟、⾁}
  \begin{phonetics}{培育}{pei2yu4}[][HSK 4]
    \definition{v.}{criar; fomentar; educar; procriar; nutrir; cultivar}
  \end{phonetics}
\end{entry}

\begin{entry}{培养}{11,9}{⼟、⼋}
  \begin{phonetics}{培养}{pei2yang3}[][HSK 4]
    \definition{v.}{cultivar (plantas, microorganismos) | promover; treinar ou desenvolver; educar e treinar para um determinado propósito durante um longo período de tempo; fazer crescer | progredir gradualmente; desenvolver ou cultivar gradualmente (hábito, qualidade, caráter, emoção, estilo, interesse, habilidade, etc.)}
  \end{phonetics}
\end{entry}

\begin{entry}{基}{11}{⼟}
  \begin{phonetics}{基}{ji1}
    \definition{adj.}{chave; básico; primário; cardinal; fundamental}
    \definition{s.}{base; fundação | base; grupo; radical; (química) uma parte dos átomos contidos na molécula de um composto, quando considerada como uma unidade, é chamada de base}
  \end{phonetics}
\end{entry}

\begin{entry}{基本}{11,5}{⼟、⽊}
  \begin{phonetics}{基本}{ji1ben3}[][HSK 3]
    \definition{adj.}{básico; fundamental; elementar | principal}
    \definition{adv.}{basicamente; em geral; no geral; em termos gerais}
    \definition{s.}{fundação}
  \end{phonetics}
\end{entry}

\begin{entry}{基本上}{11,5,3}{⼟、⽊、⼀}
  \begin{phonetics}{基本上}{ji1 ben3 shang4}[][HSK 3]
    \definition{adv.}{basicamente; principalmente | em geral; de modo geral}
  \end{phonetics}
\end{entry}

\begin{entry}{基本功}{11,5,5}{⼟、⽊、⼒}
  \begin{phonetics}{基本功}{ji1ben3gong1}
    \definition{s.}{habilidades | fundamentos básicos}
  \end{phonetics}
\end{entry}

\begin{entry}{基本法}{11,5,8}{⼟、⽊、⽔}
  \begin{phonetics}{基本法}{ji1ben3fa3}
    \definition{s.}{lei básica (constituição)}
  \end{phonetics}
\end{entry}

\begin{entry}{基因}{11,6}{⼟、⼞}
  \begin{phonetics}{基因}{ji1yin1}
    \definition{s.}{gene}
  \end{phonetics}
\end{entry}

\begin{entry}{基地}{11,6}{⼟、⼟}
  \begin{phonetics}{基地}{ji1di4}[][HSK 5]
    \definition{s.}{base; como base para alguns negócios | base; um local dedicado à realização de um negócio}
  \end{phonetics}
\end{entry}

\begin{entry}{基金}{11,8}{⼟、⾦}
  \begin{phonetics}{基金}{ji1jin1}[][HSK 5]
    \definition[项,支,种,个]{s.}{fundo; fundos reservados ou destinados ao estabelecimento ou desenvolvimento de uma empresa}
  \end{phonetics}
\end{entry}

\begin{entry}{基础}{11,10}{⼟、⽯}
  \begin{phonetics}{基础}{ji1chu3}[][HSK 3]
    \definition[个,种,点,层]{s.}{base; fundamento; fundação; a essência ou o ponto de partida do desenvolvimento das coisas | básico; fundamental; refere-se às condições mínimas | fundação do edifício; base do edifício}
  \end{phonetics}
\end{entry}

\begin{entry}{基督教}{11,13,11}{⼟、⽬、⽁}
  \begin{phonetics}{基督教}{ji1du1jiao4}
    \definition*{s.}{Cristianismo | Cristão}
  \end{phonetics}
\end{entry}

\begin{entry}{堆}{11}{⼟}
  \begin{phonetics}{堆}{dui1}[][HSK 5]
    \definition{clas.}{amontoado; pilha; multidão; usado para pilhas de coisas}
    \definition{s.}{amontoado; pilha; empilhamento | (em nomes de lugares)  colina; monte| multidão de pessoas ou coisas}
    \definition{v.}{empilhar; amontoar; acumular; juntar; reunir}
  \end{phonetics}
\end{entry}

\begin{entry}{堵}{11}{⼟}
  \begin{phonetics}{堵}{du3}[][HSK 4]
    \definition*{s.}{Sobrenome Du}
    \definition{adj.}{asfixiado; abafado; sufocado; oprimido}
    \definition{clas.}{usado para paredes}
    \definition{s.}{parede}
    \definition{v.}{impedir; bloquear}
  \end{phonetics}
\end{entry}

\begin{entry}{堵车}{11,4}{⼟、⾞}
  \begin{phonetics}{堵车}{du3che1}[][HSK 4]
    \definition{v.}{congestionar (trânsito)}
    \definition{v.+compl.}{congestionamento; tráfego intenso; ficar congestionado (no tráfego); bloqueio de vias devido ao excesso de tráfego, etc.}
  \end{phonetics}
\end{entry}

\begin{entry}{够}{11}{⼣}
  \begin{phonetics}{够}{gou4}[][HSK 2]
    \definition{adj.}{suficiente; adequado; apropriado; atingir e ultrapassar um determinado limite, difícil de suportar}
    \definition{adv.}{suficientemente; o suficiente (para atingir um determinado nível); indica que atingiu um determinado padrão ou nível elevado}
    \definition{v.}{alcançar (algo, esticando-se); (usando membros, etc.) esticar-se para alcançar ou tocar em locais de difícil acesso | atingir (um padrão ou nível); satisfazer ou atingir a quantidade, os padrões, etc. necessários}
  \end{phonetics}
\end{entry}

\begin{entry}{够不着}{11,4,11}{⼣、⼀、⽬}
  \begin{phonetics}{够不着}{gou4bu5zhao2}
    \definition{v.}{ser incapaz de alcançar}
  \end{phonetics}
\end{entry}

\begin{entry}{够本}{11,5}{⼣、⽊}
  \begin{phonetics}{够本}{gou4ben3}
    \definition{v.}{empatar | fazer valer o dinheiro}
  \end{phonetics}
\end{entry}

\begin{entry}{够呛}{11,7}{⼣、⼝}
  \begin{phonetics}{够呛}{gou4qiang4}
    \definition{adj.}{suficiente | terrível | insuportável | improvável}
  \end{phonetics}
\end{entry}

\begin{entry}{够味}{11,8}{⼣、⼝}
  \begin{phonetics}{够味}{gou4wei4}
    \definition{adj.}{excelente | na medida}
  \end{phonetics}
\end{entry}

\begin{entry}{够戗}{11,8}{⼣、⼽}
  \begin{phonetics}{够戗}{gou4qiang4}
    \variantof{够呛}
  \end{phonetics}
\end{entry}

\begin{entry}{够朋友}{11,8,4}{⼣、⽉、⼜}
  \begin{phonetics}{够朋友}{gou4peng2you5}
    \definition{v.}{ser um amigo verdadeiro}
  \end{phonetics}
\end{entry}

\begin{entry}{够格}{11,10}{⼣、⽊}
  \begin{phonetics}{够格}{gou4ge2}
    \definition{adj.}{apto | qualificado | apresentável}
  \end{phonetics}
\end{entry}

\begin{entry}{够得着}{11,11,11}{⼣、⼻、⽬}
  \begin{phonetics}{够得着}{gou4de5zhao2}
    \definition{v.}{estar à altura | alcançar}
  \end{phonetics}
\end{entry}

\begin{entry}{婚}{11}{⼥}
  \begin{phonetics}{婚}{hun1}
    \definition{s.}{casamento}
    \definition{v.}{casar}
  \end{phonetics}
\end{entry}

\begin{entry}{婚礼}{11,5}{⼥、⽰}
  \begin{phonetics}{婚礼}{hun1li3}[][HSK 4]
    \definition[场]{s.}{casamento; núpcias; cerimônia de casamento}
  \end{phonetics}
\end{entry}

\begin{entry}{宿}{11}{⼧}
  \begin{phonetics}{宿}{su4}
    \definition*{s.}{Sobrenome Su}
    \definition{adj.}{de longa data; antigo; velho | veterano; velho; experiente}
    \definition{v.}{hospedar-se para passar a noite; passar a noite}
  \end{phonetics}
  \begin{phonetics}{宿}{xiu3}
    \definition{s.}{usado para calcular a noite}[谈了半宿。___Conversamos por metade da noite.]
  \end{phonetics}
  \begin{phonetics}{宿}{xiu4}
    \definition{s.}{(astronomia) um termo antigo para constelação}
  \end{phonetics}
\end{entry}

\begin{entry}{宿舍}{11,8}{⼧、⾆}
  \begin{phonetics}{宿舍}{su4she4}[][HSK 5]
    \definition[间,幢]{s.}{alojamento; dormitório; república; albergue; casas onde escolas, empresas, etc. acomodam seus alunos ou funcionários}
  \end{phonetics}
\end{entry}

\begin{entry}{寂}{11}{⼧}
  \begin{phonetics}{寂}{ji4}
    \definition{adj.}{quieto; parado; silencioso | solitário}
  \end{phonetics}
\end{entry}

\begin{entry}{寂寞}{11,13}{⼧、⼧}
  \begin{phonetics}{寂寞}{ji4mo4}
    \definition{adj.}{sozinho | solitário | (de um lugar) silencioso}
  \end{phonetics}
\end{entry}

\begin{entry}{寂寥}{11,14}{⼧、⼧}
  \begin{phonetics}{寂寥}{ji4liao2}
    \definition{s.}{solidão | vasto e vazio | quieto e desolado (literário)}
  \end{phonetics}
\end{entry}

\begin{entry}{寄}{11}{⼧}
  \begin{phonetics}{寄}{ji4}[][HSK 4]
    \definition{adj.}{adotado; fomentado; promovido}
    \definition{v.}{enviar; postar; remeter | confiar; depositar; colocar | depender de; apegar-se a}
  \end{phonetics}
\end{entry}

\begin{entry}{寄予}{11,4}{⼧、⼅}
  \begin{phonetics}{寄予}{ji4yu3}
    \definition{v.}{expressar | colocar (esperança, importância, etc.) em | mostrar}
  \end{phonetics}
\end{entry}

\begin{entry}{寄生}{11,5}{⼧、⽣}
  \begin{phonetics}{寄生}{ji4sheng1}
    \definition{s.}{parasita | parasitismo}
    \definition{v.}{viver tirando vantagem dos outros | viver dentro ou sobre outro organismo como um parasita}
  \end{phonetics}
\end{entry}

\begin{entry}{寄生生活}{11,5,5,9}{⼧、⽣、⽣、⽔}
  \begin{phonetics}{寄生生活}{ji4sheng1sheng1huo2}
    \definition{s.}{parasitismo | vida parasitária}
  \end{phonetics}
\end{entry}

\begin{entry}{寄存}{11,6}{⼧、⼦}
  \begin{phonetics}{寄存}{ji4cun2}
    \definition{v.}{depositar | deixar algo com alguém | armazenar}
  \end{phonetics}
\end{entry}

\begin{entry}{寄托}{11,6}{⼧、⼿}
  \begin{phonetics}{寄托}{ji4tuo1}
    \definition{v.}{investir (sua esperança, energia, etc.) em algo | confiar (a alguém) | colocar (a esperança, a energia, etc.) em}
  \end{phonetics}
\end{entry}

\begin{entry}{寄卖}{11,8}{⼧、⼗}
  \begin{phonetics}{寄卖}{ji4mai4}
    \definition{v.}{consignar para venda}
  \end{phonetics}
\end{entry}

\begin{entry}{寄居}{11,8}{⼧、⼫}
  \begin{phonetics}{寄居}{ji4ju1}
    \definition{s.}{morar longe de casa}
  \end{phonetics}
\end{entry}

\begin{entry}{寄放}{11,8}{⼧、⽅}
  \begin{phonetics}{寄放}{ji4fang4}
    \definition{v.}{deixar algo com alguém}
  \end{phonetics}
\end{entry}

\begin{entry}{寄养}{11,9}{⼧、⼋}
  \begin{phonetics}{寄养}{ji4yang3}
    \definition{v.}{embarcar | promover | colocar sob os cuidados de alguém (uma criança, animal de estimação, etc.)}
  \end{phonetics}
\end{entry}

\begin{entry}{寄送}{11,9}{⼧、⾡}
  \begin{phonetics}{寄送}{ji4song4}
    \definition{v.}{enviar | transmitir}
  \end{phonetics}
\end{entry}

\begin{entry}{寄递}{11,10}{⼧、⾡}
  \begin{phonetics}{寄递}{ji4di4}
    \definition{s.}{entrega de correspondência}
  \end{phonetics}
\end{entry}

\begin{entry}{寄售}{11,11}{⼧、⼝}
  \begin{phonetics}{寄售}{ji4shou4}
    \definition{v.}{venda em consignação}
  \end{phonetics}
\end{entry}

\begin{entry}{寄宿}{11,11}{⼧、⼧}
  \begin{phonetics}{寄宿}{ji4su4}
    \definition{s.}{embarque}
    \definition{v.}{embarcar}
  \end{phonetics}
\end{entry}

\begin{entry}{寄望}{11,11}{⼧、⽉}
  \begin{phonetics}{寄望}{ji4wang4}
    \definition{v.}{depositar esperanças em}
  \end{phonetics}
\end{entry}

\begin{entry}{密}{11}{⼧}
  \begin{phonetics}{密}{mi4}[][HSK 4]
    \definition*{s.}{Sobrenome Mi}
    \definition{adj.}{fechado; denso; espesso | íntimo; próximo; afetuoso | delicado; fino; cuidadoso; meticuloso}
    \definition{adv.}{secretamente}
    \definition{s.}{segredo | densidade}
  \end{phonetics}
\end{entry}

\begin{entry}{密切}{11,4}{⼧、⼑}
  \begin{phonetics}{密切}{mi4qie4}[][HSK 4]
    \definition{adj.}{próximo; íntimo; relacionamento próximo}
    \definition{adv.}{cuidadosamente; atentamente; intimamente}
    \definition{v.}{tornar-se próximo; tornar-se íntimo; conectar-se}
  \end{phonetics}
\end{entry}

\begin{entry}{密码}{11,8}{⼧、⽯}
  \begin{phonetics}{密码}{mi4ma3}[][HSK 4]
    \definition[个]{s.}{código; senha;}
  \end{phonetics}
\end{entry}

\begin{entry}{崇}{11}{⼭}
  \begin{phonetics}{崇}{chong2}
    \definition*{s.}{Sobrenome Chong}
    \definition{adj.}{alto; elevado; sublime}
    \definition{v.}{adorar; reverenciar; venerar; estimar | respeitar}
  \end{phonetics}
\end{entry}

\begin{entry}{崇拜}{11,9}{⼭、⼿}
  \begin{phonetics}{崇拜}{chong2bai4}[][HSK 6]
    \definition{v.}{adorar; idolatrar; venerar}
  \end{phonetics}
\end{entry}

\begin{entry}{崖}{11}{⼭}
  \begin{phonetics}{崖}{ya2}
    \definition{s.}{precipício | penhasco}
  \end{phonetics}
\end{entry}

\begin{entry}{崩}{11}{⼭}
  \begin{phonetics}{崩}{beng1}
    \definition{v.}{colapsar |  estourar; quebrar | atingir por explosão | matar atirando; atirar; executar | (de um imperador) morrer | rachar; romper | atingir | executar atirando}
  \end{phonetics}
\end{entry}

\begin{entry}{巢}{11}{⼮}
  \begin{phonetics}{巢}{chao2}
    \definition*{s.}{Sobrenome Chao}
    \definition[个]{s.}{ninho (de aves, insetos, etc.)}
  \end{phonetics}
\end{entry}

\begin{entry}{常}{11}{⼱}
  \begin{phonetics}{常}{chang2}[][HSK 1]
    \definition*{s.}{Sobrenome Chang}
    \definition{adj.}{normal; comum; ordinário; indica frequência, normalidade, universalidade | constante; invariável; imutável; permanente}
    \definition{adv.}{frequentemente; geralmente; com frequência;}
    \definition{s.}{normas; disciplina, ordem social e lei e ordem do Estado}
  \end{phonetics}
\end{entry}

\begin{entry}{常见}{11,4}{⼱、⾒}
  \begin{phonetics}{常见}{chang2 jian4}[][HSK 2]
    \definition{adj.}{comum; frequentemente visto}
  \end{phonetics}
\end{entry}

\begin{entry}{常用}{11,5}{⼱、⽤}
  \begin{phonetics}{常用}{chang2 yong4}[][HSK 2]
    \definition{adj.}{em uso comum; frequentemente utilizado}
  \end{phonetics}
\end{entry}

\begin{entry}{常年}{11,6}{⼱、⼲}
  \begin{phonetics}{常年}{chang2 nian2}[][HSK 6]
    \definition{adj.}{perene; anual}
    \definition{adv.}{ano após ano; ao longo do ano; durante todo o ano; longo prazo}
  \end{phonetics}
\end{entry}

\begin{entry}{常问问题}{11,6,6,15}{⼱、⾨、⾨、⾴}
  \begin{phonetics}{常问问题}{chang2wen4wen4ti2}
    \definition{s.}{FAQ; perguntas frequentes}
  \end{phonetics}
\end{entry}

\begin{entry}{常识}{11,7}{⼱、⾔}
  \begin{phonetics}{常识}{chang2shi2}[][HSK 4]
    \definition{s.}{senso comum; conhecimento geral; conhecimento que uma pessoa comum deve ter}
  \end{phonetics}
\end{entry}

\begin{entry}{常规}{11,8}{⼱、⾒}
  \begin{phonetics}{常规}{chang2 gui1}[][HSK 6]
    \definition[个,种]{s.}{convenção; prática comum; rotina | (medicina) rotina | regra; sulco}
  \end{phonetics}
\end{entry}

\begin{entry}{常常}{11,11}{⼱、⼱}
  \begin{phonetics}{常常}{chang2 chang2}[][HSK 1]
    \definition{adv.}{frequentemente; muitas vezes; geralmente; indica que a ação ocorreu várias vezes}
  \end{phonetics}
\end{entry}

\begin{entry}{庵}{11}{⼴}
  \begin{phonetics}{庵}{an1}
    \definition*{s.}{Sobrenome An}
    \definition[个,座]{s.}{cabana | convento de freiras; templos budistas, principalmente onde vivem as freiras}
  \end{phonetics}
\end{entry}

\begin{entry}{庶}{11}{⼴}
  \begin{phonetics}{庶}{shu4}
    \definition*{s.}{Sobrenome Shu}
    \definition{adj.}{multitudinário; numeroso}
    \definition{conj.}{para que; de ​​modo a}
    \definition{s.}{da ou pela concubina (diferentemente da esposa legal); no sistema patriarcal, refere-se ao ramo lateral da família}
  \end{phonetics}
\end{entry}

\begin{entry}{庶民}{11,5}{⼴、⽒}
  \begin{phonetics}{庶民}{shu4min2}
    \definition{s.}{(antigo) pessoas comuns | (antigo) plebeu; plebeus | (antigo) a multidão de pessoas comuns (na literatura erudita)}
  \end{phonetics}
\end{entry}

\begin{entry}{廊}{11}{⼴}
  \begin{phonetics}{廊}{lang2}
    \definition[个]{s.}{varanda; corredor}
  \end{phonetics}
\end{entry}

\begin{entry}{廊坊}{11,7}{⼴、⼟}
  \begin{phonetics}{廊坊}{lang2fang2}
    \definition*{s.}{Cidade de Langfang em Hebei}
  \end{phonetics}
\end{entry}

\begin{entry}{弹}{11}{⼸}
  \begin{phonetics}{弹}{tan2}[][HSK 5]
    \definition{s.}{bola; pelota; pequenas bolas disparadas com um estilingue | bomba; bala; explosivos que podem ser lançados ou arremessados, com poder destrutivo e letal}
  \end{phonetics}
\end{entry}

\begin{entry}{彩}{11}{⼺}
  \begin{phonetics}{彩}{cai3}
    \definition{s.}{cor | aplausos; vivas | variedade; brilho; esplendor | prêmio; loteria | sangue de uma ferida | habilidades especiais empregadas em mágica ou ópera para alcançar um efeito desejado | seda colorida | cores variadas | graça na arte; graciosidade | prêmio de loteria; ganhos | efeitos especiais no teatro chinês (simbolizando sangue, fogo, etc.)}
  \end{phonetics}
\end{entry}

\begin{entry}{彩色}{11,6}{⼺、⾊}
  \begin{phonetics}{彩色}{cai3 se4}[][HSK 3]
    \definition[个,种]{s.}{multicolorido; cor; várias cores}
  \end{phonetics}
\end{entry}

\begin{entry}{彩虹}{11,9}{⼺、⾍}
  \begin{phonetics}{彩虹}{cai3hong2}
    \definition[道]{s.}{arco-íris}
  \end{phonetics}
\end{entry}

\begin{entry}{彩票}{11,11}{⼺、⽰}
  \begin{phonetics}{彩票}{cai3piao4}[][HSK 5]
    \definition[张]{s.}{bilhete de loteria}
  \end{phonetics}
\end{entry}

\begin{entry}{彪}{11}{⾌}
  \begin{phonetics}{彪}{biao1}
    \definition*{s.}{Sobrenome Biao}
    \definition{adj.}{semelhante a um tigre (metáfora para estatura alta)}
    \definition{s.}{tigre jovem}
  \end{phonetics}
\end{entry}

\begin{entry}{得}{11}{⼻}
  \begin{phonetics}{得}{de2}[][HSK 2]
    \definition{adj.}{adequado; apropriado | satisfeito; complacente; orgulhoso de si mesmo}
    \definition{interj.}{usado para encerrar uma conversa para indicar concordância ou proibição | usado quando a situação não é a esperada, para expressar impotência}
    \definition{v.}{obter (em oposição a 失); conseguir; ganhar |  (de um cálculo) igual; resultar em | estar pronto; estar acabado | pegar; apanhar; contrair uma doença}
    \definition{v.aux.}{usado antes de outros verbos para expressar permissão | usado antes de outros verbos para indicar que é possível (usado principalmente na forma negativa) | usado em conversas para indicar que não há necessidade de dizer mais nada}
  \seealsoref{失}{shi1}
  \end{phonetics}
  \begin{phonetics}{得}{de5}[][HSK 2]
    \definition{part.}{depois de um verbo ou adjetivo para expressar possibilidade ou capacidade | entre um verbo e seu complemento para expressar possibilidade | ligando um verbo ou um adjetivo a um complemento que descreve a maneira ou o grau}
  \end{phonetics}
  \begin{phonetics}{得}{dei3}[][HSK 4]
    \definition{v.}{precisar; expressa uma necessidade lógica, factual ou subjetiva; deve; é necessário | ter de; ser obrigado a; indica uma necessidade de vontade ou de fato | certamente irá; expressa a inevitabilidade da especulação}
  \end{phonetics}
\end{entry}

\begin{entry}{得了}{11,2}{⼻、⼅}
  \begin{phonetics}{得了}{de2le5}[][HSK 5]
    \definition{expr.}{Tudo bem!; É o bastante!}
  \end{phonetics}
  \begin{phonetics}{得了}{de2liao3}
    \definition{adj.}{(enfaticamente, em perguntas retóricas) possível; indica que a situação é séria (usado principalmente em perguntas retóricas ou formas negativas)}
  \end{phonetics}
\end{entry}

\begin{entry}{得以}{11,4}{⼻、⼈}
  \begin{phonetics}{得以}{de2 yi3}[][HSK 5]
    \definition{v.}{ser capaz de; para que\dots possa (ou possa)\dots}
  \end{phonetics}
\end{entry}

\begin{entry}{得分}{11,4}{⼻、⼑}
  \begin{phonetics}{得分}{de2 fen1}[][HSK 3]
    \definition{s.}{pontuação; classificação; nota; pontuação obtida em jogos ou competições}
    \definition{v.}{fazer pontos; pontuar}
  \end{phonetics}
\end{entry}

\begin{entry}{得出}{11,5}{⼻、⼐}
  \begin{phonetics}{得出}{de2 chu1}[][HSK 2]
    \definition{v.}{chegar (a uma conclusão); obter (a um resultado); deduzir ou calcular (conclusão ou resultado)}
  \end{phonetics}
\end{entry}

\begin{entry}{得到}{11,8}{⼻、⼑}
  \begin{phonetics}{得到}{de2 dao4}[][HSK 1]
    \definition{v.}{obter; conseguir; ganhar; receber; possuir algo; adquirir}
  \end{phonetics}
\end{entry}

\begin{entry}{得意}{11,13}{⼻、⼼}
  \begin{phonetics}{得意}{de2yi4}[][HSK 4]
    \definition{adj.}{complacente; orgulhoso de si mesmo; satisfeito consigo mesmo}
    \definition{v.+compl.}{orgulhar-se de si mesmo; ter satisfação consigo mesmo; ser complacente}
  \end{phonetics}
\end{entry}

\begin{entry}{悉}{11}{⼼}
  \begin{phonetics}{悉}{xi1}
    \definition*{s.}{Sobrenome Xi}
    \definition{adj.}{tudo; inteiro; total | detalhado}
    \definition{v.}{saber; aprender; ser informado de}
  \end{phonetics}
\end{entry}

\begin{entry}{悉心}{11,4}{⼼、⼼}
  \begin{phonetics}{悉心}{xi1xin1}
    \definition{adv.}{colocar o coração (e a alma) em algo | com muito cuidado}
  \end{phonetics}
\end{entry}

\begin{entry}{悉尼}{11,5}{⼼、⼫}
  \begin{phonetics}{悉尼}{xi1ni2}
    \definition*{s.}{Sidney}
  \end{phonetics}
\end{entry}

\begin{entry}{悉数}{11,13}{⼼、⽁}
  \begin{phonetics}{悉数}{xi1shu3}
    \definition{adv.}{enumerar em detalhes | explicar claramente}
  \end{phonetics}
  \begin{phonetics}{悉数}{xi1shu4}
    \definition{adv.}{todos | cada um | toda a soma}
  \end{phonetics}
\end{entry}

\begin{entry}{您}{11}{⼼}
  \begin{phonetics}{您}{nin2}[][HSK 1]
    \definition{pron.}{você; a forma de tratamento respeitosa da segunda pessoa do singular 你}
  \seealsoref{你}{ni3}
  \end{phonetics}
\end{entry}

\begin{entry}{悬}{11}{⼼}
  \begin{phonetics}{悬}{xuan2}[][HSK 6]
    \definition{adj.}{pendente; não resolvido; sem nenhum resultado | distante; a distância é grande; a diferença é grande | (dialeto) perigoso}
    \definition{v.}{pendurar; suspender | levantar; elevar | sentir-se ansioso; ser solícito | imaginar}
  \end{phonetics}
\end{entry}

\begin{entry}{悬挂}{11,9}{⼼、⼿}
  \begin{phonetics}{悬挂}{xuan2gua4}
    \definition{s.}{(veículo) suspensão}
    \definition{v.}{suspender}
  \end{phonetics}
\end{entry}

\begin{entry}{悬崖}{11,11}{⼼、⼭}
  \begin{phonetics}{悬崖}{xuan2ya2}
    \definition{s.}{precipício | penhasco}
  \end{phonetics}
\end{entry}

\begin{entry}{情}{11}{⼼}
  \begin{phonetics}{情}{qing2}
    \definition{s.}{sentimento; afeição | amor; paixão | paixão sexual; luxúria | favor; gentileza | situação; circunstâncias; condição | razão; sentido | sensibilidades; sentimentos}
  \end{phonetics}
\end{entry}

\begin{entry}{情节}{11,5}{⼼、⾋}
  \begin{phonetics}{情节}{qing2jie2}[][HSK 5]
    \definition{s.}{enredo; trama; desenrolar específico dos acontecimentos | circunstância; detalhes do crime ou erro | enredo; roteiro; refere-se especificamente ao processo de desenvolvimento e evolução dos conflitos e contradições em obras literárias narrativas}
  \end{phonetics}
\end{entry}

\begin{entry}{情况}{11,7}{⼼、⼎}
  \begin{phonetics}{情况}{qing2kuang4}[][HSK 3]
    \definition[种,个,些]{s.}{condição; situação; circunstâncias; estado das coisas | mudanças notáveis e impactantes}
  \end{phonetics}
\end{entry}

\begin{entry}{情形}{11,7}{⼼、⼺}
  \begin{phonetics}{情形}{qing2xing2}[][HSK 5]
    \definition[个]{s.}{situação; condição; circunstâncias; estado de coisas; a situação específica das coisas}
  \end{phonetics}
\end{entry}

\begin{entry}{情绪}{11,11}{⼼、⽷}
  \begin{phonetics}{情绪}{qing2xu4}
    \definition[种]{s.}{humor | estado da mente | mau humor}
  \end{phonetics}
\end{entry}

\begin{entry}{情景}{11,12}{⼼、⽇}
  \begin{phonetics}{情景}{qing2jing3}[][HSK 4]
    \definition[个]{s.}{cena; vista; circunstâncias}
  \end{phonetics}
\end{entry}

\begin{entry}{情感}{11,13}{⼼、⼼}
  \begin{phonetics}{情感}{qing2 gan3}[][HSK 3]
    \definition[份]{s.}{emoção; sentimento | afeição; apego; reações psicológicas positivas ou negativas a estímulos externos, como gosto, raiva, tristeza, medo, amor, nojo, etc.}
  \end{phonetics}
\end{entry}

\begin{entry}{惊}{11}{⼼}
  \begin{phonetics}{惊}{jing1}
    \definition{v.}{assustar; ficar assustado; ficar nervoso devido a estímulo repentino; ficar com medo | surpreender; chocar; alarmar}
  \end{phonetics}
\end{entry}

\begin{entry}{惊呆}{11,7}{⼼、⼝}
  \begin{phonetics}{惊呆}{jing1dai1}
    \definition{adj.}{estupefato | chocado}
  \end{phonetics}
\end{entry}

\begin{entry}{惊喜}{11,12}{⼼、⼝}
  \begin{phonetics}{惊喜}{jing1xi3}
    \definition{s.}{boa surpresa}
    \definition{v.}{ser agradavelmente surpreendido}
  \end{phonetics}
\end{entry}

\begin{entry}{惨}{11}{⽕}
  \begin{phonetics}{惨}{can3}[][HSK 6]
    \definition{adj.}{miserável; trágico | cruel; brutal; implacável | desastroso; terrível; esmagador | lamentável; desaventurado | em um grau sério; grau grave; dano grave | selvagem; desumano; vicioso; cruel}
  \end{phonetics}
\end{entry}

\begin{entry}{据}{11}{⼿}
  \begin{phonetics}{据}{ju1}
    \definition{part.}{elemento formador de palavras}
  \seealsoref{拮据}{jie2ju1}
  \end{phonetics}
  \begin{phonetics}{据}{ju4}[][HSK 6]
    \definition*{s.}{Sobrenome Ju}
    \definition{prep.}{de acordo com; com base em}
    \definition{s.}{evidência; certificado; prova}
    \definition{v.}{ocupar; apreender | confiar em; depender de}
  \end{phonetics}
\end{entry}

\begin{entry}{据说}{11,9}{⼿、⾔}
  \begin{phonetics}{据说}{ju4shuo1}[][HSK 3]
    \definition{v.}{é o que dizem; é o que se diz}
  \end{phonetics}
\end{entry}

\begin{entry}{捷}{11}{⼿}
  \begin{phonetics}{捷}{jie2}
    \definition*{s.}{Sobrenome Jie}
    \definition{adj.}{rápido; ágil}
    \definition{s.}{vitória; triunfo; sucesso}
  \end{phonetics}
\end{entry}

\begin{entry}{捷径}{11,8}{⼿、⼻}
  \begin{phonetics}{捷径}{jie2jing4}
    \definition{s.}{atalho}
  \end{phonetics}
\end{entry}

\begin{entry}{掉}{11}{⼿}
  \begin{phonetics}{掉}{diao4}[][HSK 2]
    \definition{v.}{cair; soltar-se; desprender-se | ficar para trás | perder; desaparecer; omitir | diminuir; reduzir | balançar; abanar; oscilar | virar; voltar; retornar | alterar; trocar; intercambiar}
    \definition{v.aux.}{usado após certos verbos para indicar a conclusão de uma ação}
  \end{phonetics}
\end{entry}

\begin{entry}{掉队}{11,4}{⼿、⾩}
  \begin{phonetics}{掉队}{diao4dui4}
    \definition{v.}{abandonar | ficar para trás}
  \end{phonetics}
\end{entry}

\begin{entry}{掉包}{11,5}{⼿、⼓}
  \begin{phonetics}{掉包}{diao4bao1}
    \definition{v.}{vender uma falsificação pelo artigo genuíno | roubar o item valioso de alguém e substituí-lo por um item de aparência semelhante, mas sem valor}
  \end{phonetics}
\end{entry}

\begin{entry}{掉线}{11,8}{⼿、⽷}
  \begin{phonetics}{掉线}{diao4xian4}
    \definition{v.}{desconectar-se (da \emph{Internet})}
  \end{phonetics}
\end{entry}

\begin{entry}{掉转}{11,8}{⼿、⾞}
  \begin{phonetics}{掉转}{diao4zhuan3}
    \definition{v.}{dar a volta}
  \end{phonetics}
\end{entry}

\begin{entry}{掉落}{11,12}{⼿、⾋}
  \begin{phonetics}{掉落}{diao4luo4}
    \definition{v.}{derrubar}
  \end{phonetics}
\end{entry}

\begin{entry}{掉膘}{11,15}{⼿、⾁}
  \begin{phonetics}{掉膘}{diao4biao1}
    \definition{v.}{perder peso (gado)}
  \end{phonetics}
\end{entry}

\begin{entry}{掏}{11}{⼿}
  \begin{phonetics}{掏}{tao1}[][HSK 6]
    \definition{v.}{extrair; retirar; pescar | cavar (um buraco, etc.); escavar; retirar | (coloquial) roubar do bolso de alguém | tirar}
  \end{phonetics}
\end{entry}

\begin{entry}{排}{11}{⼿}
  \begin{phonetics}{排}{pai2}[][HSK 2,3]
    \definition{clas.}{usado para linhas, filas; coisas usadas para formar filas}
    \definition{s.}{linha; fileira; fileiras horizontais | pelotão; unidade militar, abaixo do nível de companhia, acima do nível de pelotão | jangada; balsa; um meio de transporte aquático feito de bambu e madeira unidos lado a lado; também se refere a bambu e madeira amarrados em fileiras para facilitar o transporte aquático | torta; bolo de carne; bolinho assado; comida cozida no vapor}
    \definition{v.}{organizar; alinhar; colocar em ordem; posicionar ou organizar em uma determinada ordem; ordenar | ensaiar | ejetar; excluir; dispensar; remover; eliminar | empurrar o obstáculo para fora do caminho}
  \end{phonetics}
\end{entry}

\begin{entry}{排水}{11,4}{⼿、⽔}
  \begin{phonetics}{排水}{pai2shui3}
    \definition{v.}{drenar}
  \end{phonetics}
\end{entry}

\begin{entry}{排队}{11,4}{⼿、⾩}
  \begin{phonetics}{排队}{pai2dui4}[][HSK 2]
    \definition{v.+compl.}{formar uma fila; alinhar-se; enfileirar-se; organizar em sequência | listar; classificar}
  \end{phonetics}
\end{entry}

\begin{entry}{排列}{11,6}{⼿、⼑}
  \begin{phonetics}{排列}{pai2lie4}[][HSK 4]
    \definition{v.}{classificar; colocar; variar; organizar; pôr em ordem}
  \end{phonetics}
\end{entry}

\begin{entry}{排名}{11,6}{⼿、⼝}
  \begin{phonetics}{排名}{pai2 ming2}[][HSK 3]
    \definition{s.}{classificação; resultado; organizado de acordo com determinados critérios}
  \end{phonetics}
\end{entry}

\begin{entry}{排除}{11,9}{⼿、⾩}
  \begin{phonetics}{排除}{pai2chu2}[][HSK 5]
    \definition{v.}{remover; superar; excluir; eliminar; livrar-se de}
  \end{phonetics}
\end{entry}

\begin{entry}{排球}{11,11}{⼿、⽟}
  \begin{phonetics}{排球}{pai2 qiu2}[][HSK 2]
    \definition[场,只,个]{s.}{voleibol; bola de voleibol}
  \end{phonetics}
\end{entry}

\begin{entry}{探}{11}{⼿}
  \begin{phonetics}{探}{tan4}
    \definition[个,位,名]{s.}{batedor; espião; detetive}
    \definition{v.}{tentar descobrir; explorar; soar | explorar; espionar | visitar; fazer uma visita em | se destacar | preocupar-se com; envolver-se em | ver; invocar}
  \end{phonetics}
\end{entry}

\begin{entry}{探亲}{11,9}{⼿、⼇}
  \begin{phonetics}{探亲}{tan4qin1}
    \definition{v.+compl.}{ir para casa para visitar a família}
  \end{phonetics}
\end{entry}

\begin{entry}{接}{11}{⼿}
  \begin{phonetics}{接}{jie1}[][HSK 2]
    \definition*{s.}{Sobrenome Jie}
    \definition{v.}{entrar em contato com; aproximar-se de | conectar; unir; juntar | continuar; prosseguir | assumir o controle; assumir o trabalho de outra pessoa e continuar a fazê-lo | pegar; agarrar; segurar ou sustentar com as mãos | receber; aceitar | encontrar; dar as boas-vindas}
  \end{phonetics}
\end{entry}

\begin{entry}{接下来}{11,3,7}{⼿、⼀、⽊}
  \begin{phonetics}{接下来}{jie1 xia4 lai2}[][HSK 2]
    \definition{expr.}{próximo; seguinte; indica uma sequência temporal subsequente}
  \end{phonetics}
\end{entry}

\begin{entry}{接(电话)}{11,5,8}{⼿、⽥、⾔}
  \begin{phonetics}{接(电话)}{jie1(dian4hua4)}
    \definition{v.}{atender (o telefone) | receber (uma ligação telefônica)}
  \end{phonetics}
\end{entry}

\begin{entry}{接近}{11,7}{⼿、⾡}
  \begin{phonetics}{接近}{jie1jin4}[][HSK 3]
    \definition{adj.}{perto; próximo; a diferença entre os dois é mínima}
    \definition{v.}{estar perto de; aproximar; aproximar-se}
  \end{phonetics}
\end{entry}

\begin{entry}{接连}{11,7}{⼿、⾡}
  \begin{phonetics}{接连}{jie1lian2}[][HSK 5]
    \definition{adv.}{no final; em sucessão; em uma fileira; um após o outro; seguindo o anterior; continuando}
  \end{phonetics}
\end{entry}

\begin{entry}{接到}{11,8}{⼿、⼑}
  \begin{phonetics}{接到}{jie1 dao4}[][HSK 2]
    \definition{v.}{receber (carta, etc.)}
  \end{phonetics}
\end{entry}

\begin{entry}{接受}{11,8}{⼿、⼜}
  \begin{phonetics}{接受}{jie1shou4}[][HSK 2]
    \definition{v.}{aceitar; não recusar (o que os outros oferecem) | concordar; não recusar (opiniões/sugestões/críticas/convites de outras pessoas, etc.)}
  \end{phonetics}
\end{entry}

\begin{entry}{接待}{11,9}{⼿、⼻}
  \begin{phonetics}{接待}{jie1dai4}[][HSK 3]
    \definition{v.}{receber (alguém); acolher; recepcionar; receber com cordialidade e generosidade}
  \end{phonetics}
\end{entry}

\begin{entry}{接班人}{11,10,2}{⼿、⽟、⼈}
  \begin{phonetics}{接班人}{jie1ban1ren2}
    \definition{s.}{sucessor}
  \end{phonetics}
\end{entry}

\begin{entry}{接着}{11,11}{⼿、⽬}
  \begin{phonetics}{接着}{jie1zhe5}[][HSK 2]
    \definition{adv.}{por sua vez; um após o outro; sucessivamente; conectado (à frase anterior); imediatamente após (a ação anterior)}
    \definition{v.}{seguir; prosseguir; continuar; seguir em frente; ficar ao lado | pegar com as mãos; apanhar}
  \end{phonetics}
\end{entry}

\begin{entry}{接触}{11,13}{⼿、⾓}
  \begin{phonetics}{接触}{jie1chu4}[][HSK 5]
    \definition{v.}{entrar em contato com | entrar em contato; tocar; interagir | engajar; o termo militar refere-se a fogo cruzado}
  \end{phonetics}
\end{entry}

\begin{entry}{控}{11}{⼿}
  \begin{phonetics}{控}{kong4}
    \definition{v.}{acusar; cobrar | controlar; dominar | manter (parte do corpo em uma determinada posição) sem apoio | virar (um recipiente) de cabeça para baixo para deixar o líquido escorrer}
  \end{phonetics}
\end{entry}

\begin{entry}{控制}{11,8}{⼿、⼑}
  \begin{phonetics}{控制}{kong4zhi4}[][HSK 5]
    \definition{v.}{controlar; restringir; dominar; fazer com que não ultrapasse um determinado limite | controlar; dominar; comandar; ocupar, fazer com que não se perca}
  \end{phonetics}
\end{entry}

\begin{entry}{推}{11}{⼿}
  \begin{phonetics}{推}{tui1}[][HSK 2]
    \definition{v.}{empurrar; dar um encontrão | girar um moinho ou uma pedra de amolar; moer | cortar; aparar | impulsionar; promover; avançar | inferir; deduzir | afastar; fugir; deslocar | adiar | eleger; escolher | ter em alta estima; elogiar muito | declinar | selecionar | elogiar muito}
  \end{phonetics}
\end{entry}

\begin{entry}{推广}{11,3}{⼿、⼴}
  \begin{phonetics}{推广}{tui1guang3}[][HSK 3]
    \definition{v.}{espalhar; estender; promover; popularizar; expandir o escopo de uso ou função de algo}
  \end{phonetics}
\end{entry}

\begin{entry}{推介}{11,4}{⼿、⼈}
  \begin{phonetics}{推介}{tui1jie4}
    \definition{s.}{promoção}
    \definition{v.}{promover | introduzir e recomendar}
  \end{phonetics}
\end{entry}

\begin{entry}{推开}{11,4}{⼿、⼶}
  \begin{phonetics}{推开}{tui1 kai1}[][HSK 3]
    \definition{v.}{declinar; rejeitar | empurrar para longe; aplicar força em uma determinada direção para mover uma pessoa ou objeto para longe de seu lugar original | empurrar para abrir (um portão, etc.); empurrar para fora para abrir algo que está fechado | estender; popularizar; promover para um alcance mais amplo e realizar em uma escala mais ampla}
  \end{phonetics}
\end{entry}

\begin{entry}{推动}{11,6}{⼿、⼒}
  \begin{phonetics}{推动}{tui1 dong4}[][HSK 3]
    \definition{v.}{promover; atuar; impulsionar; empurrar para a frente; dar ímpeto a; começar ou avançar algo (com alguma força); começar a trabalhar}
  \end{phonetics}
\end{entry}

\begin{entry}{推行}{11,6}{⼿、⾏}
  \begin{phonetics}{推行}{tui1 xing2}[][HSK 5]
    \definition{v.}{realizar; prosseguir; praticar | implementar; praticar; implementação generalizada; divulgar (experiências, métodos, etc.)}
  \end{phonetics}
\end{entry}

\begin{entry}{推进}{11,7}{⼿、⾡}
  \begin{phonetics}{推进}{tui1 jin4}[][HSK 3]
    \definition{v.}{avançar; empurrar; levar adiante; dar ímpeto a; promover o trabalho e fazê-lo avançar | empurrar; dirigir; avançar; seguir em frente; seguir em frente}
  \end{phonetics}
\end{entry}

\begin{entry}{推迟}{11,7}{⼿、⾡}
  \begin{phonetics}{推迟}{tui1chi2}[][HSK 4]
    \definition{v.}{adiar; postergar; tardar; deixar para mais tarde}
  \end{phonetics}
\end{entry}

\begin{entry}{推销}{11,12}{⼿、⾦}
  \begin{phonetics}{推销}{tui1xiao1}[][HSK 4]
    \definition{v.}{vender; comercializar; promover vendas; promover a comercialização de mercadorias}
  \end{phonetics}
\end{entry}

\begin{entry}{措}{11}{⼿}
  \begin{phonetics}{措}{cuo4}
    \definition{s.}{iniciativa; solução; medida}
    \definition{v.}{organizar; gerenciar; lidar | fazer planos; administrar; organizar}
  \end{phonetics}
\end{entry}

\begin{entry}{措施}{11,9}{⼿、⽅}
  \begin{phonetics}{措施}{cuo4shi1}[][HSK 4]
    \definition{s.}{medida; etapa; passo; abordagem adotada para lidar com as coisas}
  \end{phonetics}
\end{entry}

\begin{entry}{描}{11}{⼿}
  \begin{phonetics}{描}{miao2}
    \definition{v.}{traçar; copiar | retocar; retocar | traçar um desenho | retratar | esboçar}
  \end{phonetics}
\end{entry}

\begin{entry}{描写}{11,5}{⼿、⼍}
  \begin{phonetics}{描写}{miao2xie3}[][HSK 4]
    \definition{v.}{representar; retratar; descrever; usar a linguagem e as palavras para transmitir uma imagem concreta de uma pessoa, evento ou situação}
  \end{phonetics}
\end{entry}

\begin{entry}{描述}{11,8}{⼿、⾡}
  \begin{phonetics}{描述}{miao2 shu4}[][HSK 4]
    \definition[段,种]{s.}{descrição; trecho que descreve um evento ou uma cena}
    \definition{v.}{descrever; representar}
  \end{phonetics}
\end{entry}

\begin{entry}{敎}{11}{⽁}
  \begin{phonetics}{敎}{jiao4}
    \variantof{教}
  \end{phonetics}
\end{entry}

\begin{entry}{敏}{11}{⽁}
  \begin{phonetics}{敏}{min3}
    \definition*{s.}{Sobrenome Min}
    \definition{adj.}{rápido; ágil | perspicaz; inteligente; rápido | inteligente; esperto}
  \end{phonetics}
\end{entry}

\begin{entry}{敏感}{11,13}{⽁、⼼}
  \begin{phonetics}{敏感}{min3gan3}[][HSK 5]
    \definition{adj.}{sensível; descreve pessoas ou animais que rapidamente percebem mudanças ou estímulos externos | reativo; sensível; fácil de causar reações intensas}
  \end{phonetics}
\end{entry}

\begin{entry}{救}{11}{⽁}
  \begin{phonetics}{救}{jiu4}[][HSK 3]
    \definition*{s.}{Sobrenome Jiu}
    \definition{v.}{resgatar; salvar | salvar de; aliviar (angústia, etc.) | resgatar; livrar alguém de um desastre ou perigo | ajudar; aliviar; socorrer; livrar pessoas e coisas de desastres e perigos}
  \end{phonetics}
\end{entry}

\begin{entry}{救出}{11,5}{⽁、⼐}
  \begin{phonetics}{救出}{jiu4chu1}
    \definition{v.}{resgatar | tirar do perigo}
  \end{phonetics}
\end{entry}

\begin{entry}{救护车}{11,7,4}{⽁、⼿、⾞}
  \begin{phonetics}{救护车}{jiu4hu4che1}
    \definition[辆]{s.}{ambulância}
  \end{phonetics}
\end{entry}

\begin{entry}{救灾}{11,7}{⽁、⽕}
  \begin{phonetics}{救灾}{jiu4 zai1}[][HSK 5]
    \definition{v.}{ajudar as vítimas de desastres, aliviar o desastre; resgatar pessoas afetadas por desastres; recuperar danos causados por desastres}
  \end{phonetics}
\end{entry}

\begin{entry}{救命}{11,8}{⽁、⼝}
  \begin{phonetics}{救命}{jiu4ming4}
    \definition{interj.}{Socorro! | Salve-me!}
    \definition{v.+compl.}{salvar a vida de alguém}
  \end{phonetics}
\end{entry}

\begin{entry}{教}{11}{⽁}
  \begin{phonetics}{教}{jiao1}[][HSK 1]
    \definition*{s.}{Sobrenome Jiao}
    \definition{prep.}{em uma frase passiva para introduzir o executor da ação}
    \definition{s.}{religião | professor; referência à educação ou aos professores}
    \definition{v.}{ensinar; instruir |  pedir; ordenar; dizer | permitir; possibilitar}
  \end{phonetics}
  \begin{phonetics}{教}{jiao4}
    \definition*{s.}{Sobrenome Jiao}
    \definition{s.}{religião | ensinamento}
    \definition{v.}{causar | como fazer algo | contar (explicar como fazer algo)}
  \end{phonetics}
\end{entry}

\begin{entry}{教长}{11,4}{⽁、⾧}
  \begin{phonetics}{教长}{jiao4zhang3}
    \definition{s.}{imã (Islã) | mulá}
  \end{phonetics}
\end{entry}

\begin{entry}{教训}{11,5}{⽁、⾔}
  \begin{phonetics}{教训}{jiao4xun4}[][HSK 4]
    \definition{s.}{moral; lição}
    \definition{v.}{repreender; educar; ensinar uma lição a alguém; dar uma bronca em alguém; dar um sermão em alguém (por ter cometido um erro, etc.)}
  \end{phonetics}
\end{entry}

\begin{entry}{教会}{11,6}{⽁、⼈}
  \begin{phonetics}{教会}{jiao1hui4}
    \definition{v.}{mostrar | ensinar}
  \end{phonetics}
  \begin{phonetics}{教会}{jiao4hui4}
    \definition{s.}{igreja cristã}
  \end{phonetics}
\end{entry}

\begin{entry}{教导}{11,6}{⽁、⼨}
  \begin{phonetics}{教导}{jiao4dao3}
    \definition{s.}{instrução | orientação | ensino}
    \definition{v.}{instruir | orientar | ensinar}
  \end{phonetics}
\end{entry}

\begin{entry}{教师}{11,6}{⽁、⼱}
  \begin{phonetics}{教师}{jiao4 shi1}[][HSK 2]
    \definition[个,位,名]{s.}{professor; professor de escola}
  \end{phonetics}
\end{entry}

\begin{entry}{教材}{11,7}{⽁、⽊}
  \begin{phonetics}{教材}{jiao4cai2}[][HSK 3]
    \definition[本,套]{s.}{livro didático; materiais didáticos, incluindo livros didáticos, apostilas, materiais de referência, vídeos, imagens, etc.}
  \end{phonetics}
\end{entry}

\begin{entry}{教学}{11,8}{⽁、⼦}
  \begin{phonetics}{教学}{jiao4 xue2}[][HSK 2]
    \definition[个,门]{s.}{ensino; educação; o processo de transmissão de conhecimentos e habilidades}
  \end{phonetics}
\end{entry}

\begin{entry}{教学楼}{11,8,13}{⽁、⼦、⽊}
  \begin{phonetics}{教学楼}{jiao4 xue2 lou2}[][HSK 1]
    \definition{s.}{prédio da escola; bloco de ensino; edifícios utilizados para atividades educacionais, geralmente incluindo salas de aula, laboratórios, auditórios, etc.}
  \end{phonetics}
\end{entry}

\begin{entry}{教官}{11,8}{⽁、⼧}
  \begin{phonetics}{教官}{jiao4guan1}
    \definition{s.}{instrutor militar}
  \end{phonetics}
\end{entry}

\begin{entry}{教练}{11,8}{⽁、⽷}
  \begin{phonetics}{教练}{jiao4lian4}[][HSK 3]
    \definition[个,位,名]{s.}{instrutor; treinador (esportes); pessoas que trabalham como treinadores}
    \definition{v.}{treinar; treinar outras pessoas para dominarem uma determinada técnica (como esportes, dirigir carros, pilotar aviões, etc.)}
  \end{phonetics}
\end{entry}

\begin{entry}{教育}{11,8}{⽁、⾁}
  \begin{phonetics}{教育}{jiao4yu4}[][HSK 2]
    \definition{s.}{educação; refere-se a atividades sociais cujo objetivo direto é influenciar o desenvolvimento físico e mental das pessoas; refere-se principalmente ao processo de formação dos alunos nas escolas}
    \definition{v.}{ensinar; educar; inspirar, fazer compreender a razão}
  \end{phonetics}
\end{entry}

\begin{entry}{教室}{11,9}{⽁、⼧}
  \begin{phonetics}{教室}{jiao4shi4}[][HSK 2]
    \definition[间]{s.}{sala de aula}
  \end{phonetics}
\end{entry}

\begin{entry}{教堂}{11,11}{⽁、⼟}
  \begin{phonetics}{教堂}{jiao4tang2}
    \definition[间]{s.}{igreja | capela}
  \end{phonetics}
\end{entry}

\begin{entry}{教授}{11,11}{⽁、⼿}
  \begin{phonetics}{教授}{jiao4shou4}[][HSK 4]
    \definition[个,位]{s.}{professor (universitário)}
    \definition{v.}{ensinar; instruir; dar aulas; dar palestras}
  \end{phonetics}
\end{entry}

\begin{entry}{敢}{11}{⽁}
  \begin{phonetics}{敢}{gan3}[][HSK 3]
    \definition{adj.}{ousado; corajoso; audacioso; valente}
    \definition{adv.}{talvez; provavelmente}
    \definition{v.}{ser ousado o suficiente; atrever-se | ter confiança em; ter certeza; estar certo | aventurar-se; ter coragem de fazer algo | ser ousado; arriscar-se}
  \end{phonetics}
\end{entry}

\begin{entry}{敢情}{11,11}{⽁、⼼}
  \begin{phonetics}{敢情}{gan3qing5}
    \definition{adv.}{claro | acontece que\dots}
  \end{phonetics}
\end{entry}

\begin{entry}{斜}{11}{⽃}
  \begin{phonetics}{斜}{xie2}[][HSK 5]
    \definition{adj.}{oblíquo; inclinado | enviesado; chanfrado; diagonal; torto; nem paralelo nem perpendicular a um plano ou linha}
    \definition{v.}{virar de lado; inclinar}
  \end{phonetics}
\end{entry}

\begin{entry}{斜阳}{11,6}{⽃、⾩}
  \begin{phonetics}{斜阳}{xie2yang2}
    \definition{s.}{sol poente}
  \end{phonetics}
\end{entry}

\begin{entry}{断}{11}{⽄}
  \begin{phonetics}{断}{duan4}[][HSK 3]
    \definition*{s.}{Sobrenome Duan}
    \definition{adv.}{(geralmente na forma negativa) absolutamente; decididamente}
    \definition{v.}{quebrar; partir; (objetos longos) dividir em segmentos não conectados | parar; interromper; romper; isolar; fazer com que não se sucedam mais | desistir; abster-se de; parar de fumar, beber, etc. | julgar; decidir | interceptar}
  \end{phonetics}
\end{entry}

\begin{entry}{断交}{11,6}{⽄、⼇}
  \begin{phonetics}{断交}{duan4jiao1}
    \definition{v.+compl.}{terminar uma amizade | romper relações diplomáticas}
  \end{phonetics}
\end{entry}

\begin{entry}{旋}{11}{⽅}
  \begin{phonetics}{旋}{xuan2}
    \definition*{s.}{Sobrenome Xuan}
    \definition{adv.}{em breve; rapidamente}
    \definition{s.}{redemoinho; turbilhão; vórtice}
    \definition{v.}{girar; circular; rodar | retornar; voltar}
  \end{phonetics}
\end{entry}

\begin{entry}{旋转}{11,8}{⽅、⾞}
  \begin{phonetics}{旋转}{xuan2zhuan3}
    \definition{v.}{girar}
  \end{phonetics}
\end{entry}

\begin{entry}{族}{11}{⽅}
  \begin{phonetics}{族}{zu2}[][HSK 6]
    \definition{s.}{clã; família | uma pena de morte na China antiga, imposta ao infrator e a toda a sua família, ou mesmo às famílias de sua mãe e esposa; uma antiga forma de tortura |
etnia; nacionalidade | uma grande categoria de coisas que compartilham algum atributo comum}
  \end{phonetics}
\end{entry}

\begin{entry}{旣}{11}{⽆}
  \begin{phonetics}{旣}{ji4}
    \variantof{既}
  \end{phonetics}
\end{entry}

\begin{entry}{晚}{11}{⽇}
  \begin{phonetics}{晚}{wan3}[][HSK 1]
    \definition*{s.}{Sobrenome Wan}
    \definition{adj.}{tarde; tardio; passado o prazo acordado | júnior; mais jovem | mais tarde no tempo}
    \definition{s.}{noite; à noite; após o pôr do sol | últimos anos; última vida; um período posterior; refere-se especificamente à velhice de uma pessoa | pôr do sol; ao pôr do sol}
  \end{phonetics}
\end{entry}

\begin{entry}{晚上}{11,3}{⽇、⼀}
  \begin{phonetics}{晚上}{wan3shang5}[][HSK 1]
    \definition[个]{s.}{noite; o período entre o pôr do sol e a madrugada}
  \end{phonetics}
\end{entry}

\begin{entry}{晚会}{11,6}{⽇、⼈}
  \begin{phonetics}{晚会}{wan3hui4}[][HSK 2]
    \definition[场,个,次]{s.}{festa noturna; entretenimento noturno}
  \end{phonetics}
\end{entry}

\begin{entry}{晚安}{11,6}{⽇、⼧}
  \begin{phonetics}{晚安}{wan3'an1}[][HSK 2]
    \definition{expr.}{Tenha uma boa noite; uma frase educada usada para se despedir ou cumprimentar as pessoas à noite}
  \end{phonetics}
\end{entry}

\begin{entry}{晚报}{11,7}{⽇、⼿}
  \begin{phonetics}{晚报}{wan3 bao4}[][HSK 2]
    \definition[份,张]{s.}{jornal vespertino; um jornal publicado todas as tardes}
  \end{phonetics}
\end{entry}

\begin{entry}{晚近}{11,7}{⽇、⾡}
  \begin{phonetics}{晚近}{wan3jin4}
    \definition{adj.}{recente | mais recente no passado}
    \definition{adv.}{ultimamente | recentemente}
  \end{phonetics}
\end{entry}

\begin{entry}{晚饭}{11,7}{⽇、⾷}
  \begin{phonetics}{晚饭}{wan3 fan4}[][HSK 1]
    \definition[顿]{s.}{jantar}
  \end{phonetics}
\end{entry}

\begin{entry}{晚育}{11,8}{⽇、⾁}
  \begin{phonetics}{晚育}{wan3yu4}
    \definition{s.}{parto tardio}
    \definition{v.}{ter um filho mais tarde}
  \end{phonetics}
\end{entry}

\begin{entry}{晚点}{11,9}{⽇、⽕}
  \begin{phonetics}{晚点}{wan3 dian3}[][HSK 4]
    \definition{adj.}{atrasado}
    \definition{s.}{jantar leve}
    \definition{v.}{atrasar; retardar; adiar; (carro, navio, avião) partir, correr ou chegar mais tarde do que o horário especificado}
  \end{phonetics}
\end{entry}

\begin{entry}{晚景}{11,12}{⽇、⽇}
  \begin{phonetics}{晚景}{wan3jing3}
    \definition{s.}{circunstâncias dos anos de declínio de alguém | cena noturna}
  \end{phonetics}
\end{entry}

\begin{entry}{晚餐}{11,16}{⽇、⾷}
  \begin{phonetics}{晚餐}{wan3 can1}[][HSK 2]
    \definition[份,顿,次]{s.}{ceia; jantar}
  \end{phonetics}
\end{entry}

\begin{entry}{梅}{11}{⽊}
  \begin{phonetics}{梅}{mei2}
    \definition*{s.}{Sobrenome Mei}
    \definition{s.}{ameixa | flor de ameixa | ameixeira | estação chuvosa}
  \end{phonetics}
\end{entry}

\begin{entry}{梅赛德斯-奔驰}{11,14,15,12,8,6}{⽊、⾙、⼻、⽄、⼤、⾺}
  \begin{phonetics}{梅赛德斯-奔驰}{mei2sai4de2si1-ben1chi2}
    \definition*{s.}{Mercedes-Benz}
  \end{phonetics}
\end{entry}

\begin{entry}{梦}{11}{⼣}
  \begin{phonetics}{梦}{meng4}[][HSK 4]
    \definition*{s.}{Sobrenome Meng}
    \definition[个,场]{s.}{sonho; atividade de representação no cérebro durante o sono}
    \definition{v.}{sonhar; ter um sonho}
  \end{phonetics}
\end{entry}

\begin{entry}{梦见}{11,4}{⼣、⾒}
  \begin{phonetics}{梦见}{meng4 jian4}[][HSK 4]
    \definition{v.}{sonhar; sonhar com; ver em um sonho}
  \end{phonetics}
\end{entry}

\begin{entry}{梦想}{11,13}{⼣、⼼}
  \begin{phonetics}{梦想}{meng4xiang3}[][HSK 4]
    \definition[个]{s.}{sonhar; esperança vã; sonho inalcançável}
    \definition{v.}{sonhar; sonhar com carinho; desejar ardentemente}
  \end{phonetics}
\end{entry}

\begin{entry}{梨}{11}{⽊}
  \begin{phonetics}{梨}{li2}[][HSK 5]
    \definition*{s.}{Sobrenome Li}
    \definition[个,颗]{s.}{perira; árvore de pera | pera}
  \end{phonetics}
\end{entry}

\begin{entry}{梯}{11}{⽊}
  \begin{phonetics}{梯}{ti1}
    \definition*{s.}{Sobrenome Ti}
    \definition{adj.}{em forma de escada; em socalcos}
    \definition[个]{s.}{escada; degrau; socalco (são plataformas niveladas, semelhantes a degraus, cortadas em encostas de morros para permitir o cultivo agrícola e evitar a erosão do solo)}
  \end{phonetics}
\end{entry}

\begin{entry}{梯恩梯}{11,10,11}{⽊、⼼、⽊}
  \begin{phonetics}{梯恩梯}{ti1'en1ti1}
    \definition{s.}{(empréstimo linguístico) TNT, trinitrotolueno}
  \end{phonetics}
\end{entry}

\begin{entry}{检}{11}{⽊}
  \begin{phonetics}{检}{jian3}
    \definition*{s.}{Sobrenome Jian}
    \definition{v.}{verificar; inspecionar; examinar | conter-se; ter cuidado na conduta}
  \end{phonetics}
\end{entry}

\begin{entry}{检查}{11,9}{⽊、⽊}
  \begin{phonetics}{检查}{jian3cha2}[][HSK 2]
    \definition[份,个,次]{s.}{autocrítica; reconhecer e criticar os próprios erros verbais ou escritos}
    \definition{v.}{verificar; inspecionar; examinar; verificar cuidadosamente para descobrir o problema | criticar a si mesmo; identificar seus pontos fracos e erros, e criticar seu próprio comportamento}
  \end{phonetics}
\end{entry}

\begin{entry}{检测}{11,9}{⽊、⽔}
  \begin{phonetics}{检测}{jian3 ce4}[][HSK 4]
    \definition{v.}{testar; detectar; verificar}
  \end{phonetics}
\end{entry}

\begin{entry}{检验}{11,10}{⽊、⾺}
  \begin{phonetics}{检验}{jian3yan4}[][HSK 5]
    \definition{v.}{testar; examinar; inspecionar}
  \end{phonetics}
\end{entry}

\begin{entry}{欲}{11}{⽋}
  \begin{phonetics}{欲}{yu4}
    \definition{adj.}{desejo | apetite | paixão | luxúria | ganância}
    \definition{v.}{desejar}
  \end{phonetics}
\end{entry}

\begin{entry}{毫}{11}{⽊}
  \begin{phonetics}{毫}{hao2}
    \definition{adv.}{nem um pouco; absolutamente nenhum; completamente sem}
    \definition{clas.}{hao, uma unidade de comprimento igual a um milésimo de polegada ou 1/30 de milímetro | hao, uma unidade de peso igual a um milésimo de um centavo ou 0,005 grama |
uma fração minúscula; uma parte muito pequena}
    \definition{pref.}{mili-, usado com a unidade de uma quantidade física para representar um milésimo dessa quantidade}
    \definition{s.}{cabelo longo e fino | pincel de escrita | uma das duas ou três alças de uma balança para pendurar na mão do usuário | cerda; uma corda de mão em uma balança ou equilíbrio | fio de cabelo}
  \end{phonetics}
\end{entry}

\begin{entry}{毫不费力}{11,4,9,2}{⽊、⼀、⾙、⼒}
  \begin{phonetics}{毫不费力}{hao2bu2fei4li4}
    \definition{adj.}{sem esforço | não gastando o menor esforço}
  \end{phonetics}
\end{entry}

\begin{entry}{毫升}{11,4}{⽊、⼗}
  \begin{phonetics}{毫升}{hao2 sheng1}[][HSK 4]
    \definition{clas.}{mililitro; unidade de volume, milésimo de um litro (ml)}
  \end{phonetics}
\end{entry}

\begin{entry}{毫米}{11,6}{⽊、⽶}
  \begin{phonetics}{毫米}{hao2mi3}[][HSK 4]
    \definition{clas.}{milímetro; unidade legal de medida de comprimento, 1 mm equivale a 0,1 cm}
  \end{phonetics}
\end{entry}

\begin{entry}{液}{11}{⽔}
  \begin{phonetics}{液}{ye4}
    \definition{s.}{líquido; fluido; suco}
  \end{phonetics}
\end{entry}

\begin{entry}{液体}{11,7}{⽔、⼈}
  \begin{phonetics}{液体}{ye4ti3}
    \definition{adj./s.}{líquido}
  \end{phonetics}
\end{entry}

\begin{entry}{涵}{11}{⽔}
  \begin{phonetics}{涵}{han2}
    \definition{s.}{bueiro; galeria}
    \definition{v.}{conter; incorporar}
  \end{phonetics}
\end{entry}

\begin{entry}{淀}{11}{⽔}
  \begin{phonetics}{淀}{dian4}
    \definition{s.}{lago raso, frequentemente usado em nomes de lugares}
    \definition{v.}{formar sedimentos | sedimentar; precipitar}
  \end{phonetics}
\end{entry}

\begin{entry}{淋}{11}{⽔}
  \begin{phonetics}{淋}{lin2}
    \definition{v.}{borrifar | pingar | derramar | encharcar}
  \end{phonetics}
  \begin{phonetics}{淋}{lin4}
    \definition{s.}{gonorréia}
    \definition{v.}{filtrar | coar | drenar}
  \end{phonetics}
\end{entry}

\begin{entry}{淡}{11}{⽔}
  \begin{phonetics}{淡}{dan4}[][HSK 4]
    \definition*{s.}{Sobrenome Dan}
    \definition{adj.}{sem gosto; fraco; não tem sabor forte; não é salgado | leve; fraco; pálido | indiferente; frio; sem entusiasmo | frouxo; sem brilho | sem sentido; trivial | fino; leve}
  \end{phonetics}
\end{entry}

\begin{entry}{淤}{11}{⽔}
  \begin{phonetics}{淤}{yu1}
    \definition{adj.}{assoreado}
    \definition{s.}{lodo}
    \definition[出]{s.}{(medicina chinesa) estase de sangue}
    \definition{v.}{ficar assoreado; ficar sufocado com lodo | derramar; transbordar}
  \end{phonetics}
\end{entry}

\begin{entry}{淤泥}{11,8}{⽔、⽔}
  \begin{phonetics}{淤泥}{yu1ni2}
    \definition{s.}{lodo}
  \end{phonetics}
\end{entry}

\begin{entry}{深}{11}{⽔}
  \begin{phonetics}{深}{shen1}[][HSK 3]
    \definition*{s.}{Sobrenome Shen}
    \definition{adj.}{profundo | difícil; intenso; profundo | completo; penetrante; intenso; profundo | próximo; íntimo; afeição profunda; relacionamento próximo | escuro; profundo | tardio}
    \definition{adv.}{muito; grandemente; profundamente}
    \definition{s.}{profundidade}
  \seealsoref{浅}{qian3}
  \end{phonetics}
\end{entry}

\begin{entry}{深入}{11,2}{⽔、⼊}
  \begin{phonetics}{深入}{shen1 ru4}[][HSK 3]
    \definition{adj.}{profundo; completo}
    \definition{v.}{ir fundo em; penetrar em; penetrar o exterior; alcançar o interior ou o centro de algo}
  \end{phonetics}
\end{entry}

\begin{entry}{深处}{11,5}{⽔、⼡}
  \begin{phonetics}{深处}{shen1 chu4}[][HSK 5]
    \definition{s.}{profundidades; recantos; recessos | profundezas}
  \end{phonetics}
\end{entry}

\begin{entry}{深刻}{11,8}{⽔、⼑}
  \begin{phonetics}{深刻}{shen1ke4}[][HSK 3]
    \definition{adj.}{profundo; instenso; chegar à essência de um assunto ou problema}
  \end{phonetics}
\end{entry}

\begin{entry}{深夜}{11,8}{⽔、⼣}
  \begin{phonetics}{深夜}{shen1ye4}
    \definition{adv.}{tarde da noite}
  \end{phonetics}
\end{entry}

\begin{entry}{深厚}{11,9}{⽔、⼚}
  \begin{phonetics}{深厚}{shen1hou4}[][HSK 4]
    \definition{adj.}{profundo; sentimentos fortes | sólido; profundamente enraizado; fundação sólida}
  \end{phonetics}
\end{entry}

\begin{entry}{深度}{11,9}{⽔、⼴}
  \begin{phonetics}{深度}{shen1 du4}[][HSK 5]
    \definition{adj.}{(em grau ou extensão) profundo; sério; grave}
    \definition{s.}{profundidade; grau de profundidade; | profundidade; rigor; meticulosidade; grau de contato com a essência das coisas | estágio avançado (ou em deterioração) de desenvolvimento; grau de crescimento e desenvolvimento das coisas}
  \end{phonetics}
\end{entry}

\begin{entry}{深深}{11,11}{⽔、⽔}
  \begin{phonetics}{深深}{shen1shen1}
    \definition{adj.}{profundo}
    \definition{adv.}{profundamente}
  \end{phonetics}
\end{entry}

\begin{entry}{混}{11}{⽔}
  \begin{phonetics}{混}{hun4}[][HSK 6]
    \definition{adj.}{confuso; imundo; turvo; lamacento; impuro}
    \definition{adv.}{de forma imprudente; irresponsável; irrefletidamente}
    \definition{v.}{misturar; confundir; misturar verdadeiro e falso | passar por; esgueirar-se | vagar à deriva; arrastar-se; sobreviver de maneira superficial; contentar-se com | se dar bem com alguém}
  \end{phonetics}
\end{entry}

\begin{entry}{混乱}{11,7}{⽔、⼄}
  \begin{phonetics}{混乱}{hun4luan4}
    \definition{adj.}{confuso | caótico | desordenado}
    \definition{s.}{caos}
  \end{phonetics}
\end{entry}

\begin{entry}{混饭}{11,7}{⽔、⾷}
  \begin{phonetics}{混饭}{hun4fan4}
    \definition{v.+compl.}{trabalhar para viver}
  \end{phonetics}
\end{entry}

\begin{entry}{添}{11}{⽔}
  \begin{phonetics}{添}{tian1}[][HSK 6]
    \definition{v.}{adicionar; aumentar | dar à luz}
  \end{phonetics}
\end{entry}

\begin{entry}{清}{11}{⽔}
  \begin{phonetics}{清}{qing1}[][HSK 6]
    \definition*{s.}{Dinastia Qing (1644-1911) | Sobrenome Qing}
    \definition{adj.}{claro; não misturado; (líquido ou gasoso) puro e sem mistura (em oposição a 浊) | silencioso; quieto | justo e honesto | distinto; claro; esclarecido | simples; puro, sem qualquer adulteração ou combinação | limpo; puro}
    \definition{v.}{limpar; tornar limpo | resolver; esclarecer; pagar; liquidar | contar; inspecionar}
  \seealsoref{浊}{zhuo2}
  \end{phonetics}
\end{entry}

\begin{entry}{清彻}{11,7}{⽔、⼻}
  \begin{phonetics}{清彻}{qing1che4}
    \variantof{清澈}
  \end{phonetics}
\end{entry}

\begin{entry}{清明节}{11,8,5}{⽔、⽇、⾋}
  \begin{phonetics}{清明节}{qing1ming2jie2}
    \definition*{s.}{Dia Qingming, Dia dos Finados (uma das 24~divisões do ano solar no calendário lunar chinês:~dia~4 ou 5~de~abril solar)}
  \end{phonetics}
\end{entry}

\begin{entry}{清凉}{11,10}{⽔、⼎}
  \begin{phonetics}{清凉}{qing1liang2}
    \definition{adj.}{fresco | refrescante | (roupa) ousada, reveladora}
  \end{phonetics}
\end{entry}

\begin{entry}{清唱}{11,11}{⽔、⼝}
  \begin{phonetics}{清唱}{qing1chang4}
    \definition{v.}{cantar à capela}
  \end{phonetics}
\end{entry}

\begin{entry}{清晨}{11,11}{⽔、⽇}
  \begin{phonetics}{清晨}{qing1chen2}[][HSK 5]
    \definition{s.}{matinal; manhã cedo; geralmente se refere ao período do amanhecer até logo após o nascer do sol}
  \end{phonetics}
\end{entry}

\begin{entry}{清爽}{11,11}{⽔、⽘}
  \begin{phonetics}{清爽}{qing1shuang3}
    \definition{adj.}{refrescante | relaxado}
  \end{phonetics}
\end{entry}

\begin{entry}{清理}{11,11}{⽔、⽟}
  \begin{phonetics}{清理}{qing1li3}[][HSK 5]
    \definition{v.}{esclarecer; resolver; verificar; colocar em ordem; organizar tudo e jogar fora o que não for útil}
  \end{phonetics}
\end{entry}

\begin{entry}{清晰}{11,12}{⽔、⽇}
  \begin{phonetics}{清晰}{qing1xi1}
    \definition{adj.}{claro | distinto}
  \end{phonetics}
\end{entry}

\begin{entry}{清楚}{11,13}{⽔、⽊}
  \begin{phonetics}{清楚}{qing1chu5}[][HSK 2]
    \definition{adj.}{claro; distinto; compreensível; organizado; fácil de identificar e entender | plenamente consciente de; claro sobre}
    \definition{v.}{ter clareza sobre; compreender; ação que expressa compreensão e conhecimento}
  \end{phonetics}
\end{entry}

\begin{entry}{清澈}{11,15}{⽔、⽔}
  \begin{phonetics}{清澈}{qing1che4}
    \definition{adj.}{claro | límpido}
  \end{phonetics}
\end{entry}

\begin{entry}{清醒}{11,16}{⽔、⾣}
  \begin{phonetics}{清醒}{qing1xing3}[][HSK 4]
    \definition{adj.}{sóbrio; lúcido; totalmente acordado}
  \end{phonetics}
\end{entry}

\begin{entry}{渐}{11}{⽔}
  \begin{phonetics}{渐}{jian1}
    \definition{v.}{encharcar; ficar saturado com | fluir para}
  \end{phonetics}
  \begin{phonetics}{渐}{jian4}
    \definition{adv.}{gradualmente; por graus}
  \end{phonetics}
\end{entry}

\begin{entry}{渐渐}{11,11}{⽔、⽔}
  \begin{phonetics}{渐渐}{jian4 jian4}[][HSK 4]
    \definition{adv.}{gradualmente; pouco a pouco; passo a passo; indica um aumento ou diminuição gradual em grau ou quantidade}
  \end{phonetics}
\end{entry}

\begin{entry}{渔}{11}{⽔}
  \begin{phonetics}{渔}{yu2}
    \definition[条]{s.}{pescador}
    \definition{v.}{pescar}
  \end{phonetics}
\end{entry}

\begin{entry}{渔夫}{11,4}{⽔、⼤}
  \begin{phonetics}{渔夫}{yu2fu1}
    \definition{s.}{pescador}
  \end{phonetics}
\end{entry}

\begin{entry}{渔民}{11,5}{⽔、⽒}
  \begin{phonetics}{渔民}{yu2min2}
    \definition{s.}{pescadores | povo pescador}
  \end{phonetics}
\end{entry}

\begin{entry}{渔场}{11,6}{⽔、⼟}
  \begin{phonetics}{渔场}{yu2chang3}
    \definition{s.}{área de pesca}
  \end{phonetics}
\end{entry}

\begin{entry}{渔汛}{11,6}{⽔、⽔}
  \begin{phonetics}{渔汛}{yu2xun4}
    \definition{s.}{temporada de pesca}
  \end{phonetics}
\end{entry}

\begin{entry}{渔网}{11,6}{⽔、⽹}
  \begin{phonetics}{渔网}{yu2wang3}
    \definition{s.}{rede de pesca}
  \end{phonetics}
\end{entry}

\begin{entry}{渔具}{11,8}{⽔、⼋}
  \begin{phonetics}{渔具}{yu2ju4}
    \definition{s.}{equipamento de pesca}
  \end{phonetics}
\end{entry}

\begin{entry}{渔轮}{11,8}{⽔、⾞}
  \begin{phonetics}{渔轮}{yu2lun2}
    \definition{s.}{navio de pesca}
  \end{phonetics}
\end{entry}

\begin{entry}{渔捞}{11,10}{⽔、⼿}
  \begin{phonetics}{渔捞}{yu2lao1}
    \definition{s.}{pesca (como atividade comercial)}
  \end{phonetics}
\end{entry}

\begin{entry}{渔猎}{11,11}{⽔、⽝}
  \begin{phonetics}{渔猎}{yu2lie4}
    \definition{s.}{pesca e caça}
    \definition{v.}{saquear | pilhar}
  \end{phonetics}
\end{entry}

\begin{entry}{渔笼}{11,11}{⽔、⽵}
  \begin{phonetics}{渔笼}{yu2long2}
    \definition{s.}{gaiola de pesca | armadilha de pesca}
  \end{phonetics}
\end{entry}

\begin{entry}{渔船}{11,11}{⽔、⾈}
  \begin{phonetics}{渔船}{yu2chuan2}
    \definition[条]{s.}{barco de pesca}
  \seealsoref{鱼船}{yu2chuan2}
  \end{phonetics}
\end{entry}

\begin{entry}{渔船队}{11,11,4}{⽔、⾈、⾩}
  \begin{phonetics}{渔船队}{yu2chuan2dui4}
    \definition{s.}{frota pesqueira}
  \end{phonetics}
\end{entry}

\begin{entry}{焊}{11}{⽕}
  \begin{phonetics}{焊}{han4}
    \definition{v.}{soldar; usar metal fundido para reparar objetos de metal ou conectar peças de metal}
  \end{phonetics}
\end{entry}

\begin{entry}{爽}{11}{⽘}
  \begin{phonetics}{爽}{shuang3}[][HSK 6]
    \definition{adj.}{claro; nítido; brilhante | franco; de coração aberto; direto | relaxado; confortável}
    \definition{v.}{desviar; afastar | tornar confortável; ficar confortável}
  \end{phonetics}
\end{entry}

\begin{entry}{猎}{11}{⽝}
  \begin{phonetics}{猎}{lie4}
    \definition[个]{s.}{traje de caça}
    \definition{v.}{caçar | procurar; perseguir}
  \end{phonetics}
\end{entry}

\begin{entry}{猎物}{11,8}{⽝、⽜}
  \begin{phonetics}{猎物}{lie4wu4}
    \definition{s.}{presa (vítima de um predador)}
  \end{phonetics}
\end{entry}

\begin{entry}{猛}{11}{⽝}
  \begin{phonetics}{猛}{meng3}[][HSK 6]
    \definition*{s.}{Sobrenome Meng}
    \definition{adj.}{feroz; violento | enérgico; vigoroso | valente}
    \definition{adv.}{de repente; abruptamente | vigorosamente; com força repentina | (coloquial) ao contentamento do coração; de todo o coração | ferozmente; violentamente}
  \end{phonetics}
\end{entry}

\begin{entry}{猛然}{11,12}{⽝、⽕}
  \begin{phonetics}{猛然}{meng3ran2}
    \definition{adv.}{de repente; abruptamente; indica ação repentina e rápida}
  \end{phonetics}
\end{entry}

\begin{entry}{猜}{11}{⽝}
  \begin{phonetics}{猜}{cai1}[][HSK 5]
    \definition{v.}{adivinhar; conjecturar; especular | suspeitar; ser cauteloso com os outros; desconfiar dos outros}
  \end{phonetics}
\end{entry}

\begin{entry}{猜忌}{11,7}{⽝、⼼}
  \begin{phonetics}{猜忌}{cai1 ji4}
    \definition{v.}{ser desconfiado e invejoso | ser desconfiado e ciumento de}
  \end{phonetics}
\end{entry}

\begin{entry}{猜测}{11,9}{⽝、⽔}
  \begin{phonetics}{猜测}{cai1 ce4}[][HSK 5]
    \definition[个,种]{s.}{advinhação; conjectura; suposição; especulação}
    \definition{v.}{adivinhar; conjecturar; especular; estimar a partir da imaginação}
  \end{phonetics}
\end{entry}

\begin{entry}{猜想}{11,13}{⽝、⼼}
  \begin{phonetics}{猜想}{cai1 xiang3}
    \definition{s.}{suposição; conjectura; palpite; especulação}
    \definition{v.}{supor; adivinhar; suspeitar}
  \end{phonetics}
\end{entry}

\begin{entry}{猜疑}{11,14}{⽝、⽦}
  \begin{phonetics}{猜疑}{cai1 yi2}
    \definition{v.}{abrigar suspeitas; ser desconfiado; ter receios; levantar suspeitas do nada}
  \end{phonetics}
\end{entry}

\begin{entry}{猪}{11}{⽝}
  \begin{phonetics}{猪}{zhu1}[][HSK 3]
    \definition[头,只,口]{s.}{porco; suíno}
  \end{phonetics}
\end{entry}

\begin{entry}{猪头}{11,5}{⽝、⼤}
  \begin{phonetics}{猪头}{zhu1tou2}
    \definition{s.}{tolo | idiota}
  \end{phonetics}
\end{entry}

\begin{entry}{猪柳}{11,9}{⽝、⽊}
  \begin{phonetics}{猪柳}{zhu1liu3}
    \definition{s.}{filé de porco}
  \end{phonetics}
\end{entry}

\begin{entry}{猪笼}{11,11}{⽝、⽵}
  \begin{phonetics}{猪笼}{zhu1long2}
    \definition{s.}{estrutura cilíndrica de bambu ou arame usada para restringir um porco durante o transporte}
  \end{phonetics}
\end{entry}

\begin{entry}{猪窠}{11,13}{⽝、⽳}
  \begin{phonetics}{猪窠}{zhu1ke1}
    \definition{s.}{chiqueiro}
  \end{phonetics}
\end{entry}

\begin{entry}{猫}{11}{⽝}
  \begin{phonetics}{猫}{mao1}[][HSK 2]
    \definition[只,种,群,窝,个]{s.}{gato |  (empréstimo linguístico) MODEM}
    \definition{v.}{esconder-se; entrar em esconderijo | inclinar-se para a frente; curvar-se}
  \end{phonetics}
  \begin{phonetics}{猫}{mao2}
    \definition{v.}{utilizado em 猫腰 \dpy{mao2yao1}}
  \seealsoref{猫腰}{mao2yao1}
  \end{phonetics}
\end{entry}

\begin{entry}{猫腰}{11,13}{⽝、⾁}
  \begin{phonetics}{猫腰}{mao2yao1}
    \definition{v.}{curvar-se}
  \end{phonetics}
\end{entry}

\begin{entry}{猫熊}{11,14}{⽝、⽕}
  \begin{phonetics}{猫熊}{mao1xiong2}
    \definition[把,只]{s.}{panda gigante}
  \seealsoref{熊猫}{xiong2mao1}
  \end{phonetics}
\end{entry}

\begin{entry}{率}{11}{⽞}
  \begin{phonetics}{率}{lv4}
    \definition{s.}{taxa; razão; proporção; a relação proporcional entre duas grandezas relacionadas}
  \end{phonetics}
  \begin{phonetics}{率}{shuai4}
    \definition*{s.}{Sobrenome Shuai}
    \definition{adj.}{precipitado; não cuidadoso; não cauteloso | franco; direto | elegante; bonito; o mesmo que 帅}
    \definition{adv.}{geralmente; expressa uma estimativa incerta, equivalente a 大约 e 大抵}
    \definition{s.}{modelo; exemplo}
    \definition{v.}{liderar; comandar | obedecer; seguir}
  \seealsoref{大抵}{da4di3}
  \seealsoref{大约}{da4yue1}
  \seealsoref{帅}{shuai4}
  \end{phonetics}
\end{entry}

\begin{entry}{率先}{11,6}{⽞、⼉}
  \begin{phonetics}{率先}{shuai4 xian1}[][HSK 4]
    \definition{v.}{tomar a iniciativa de fazer algo; ser o primeiro a fazer algo; assumir a liderança}
  \end{phonetics}
\end{entry}

\begin{entry}{率领}{11,11}{⽞、⾴}
  \begin{phonetics}{率领}{shuai4ling3}[][HSK 5]
    \definition{v.}{liderar (equipe ou grupo); chefiar; comandar}
  \end{phonetics}
\end{entry}

\begin{entry}{球}{11}{⽟}
  \begin{phonetics}{球}{qiu2}[][HSK 1]
    \definition[个,颗,筐]{s.}{esfera; globo; equipamento de jogo antigo, objeto tridimensional circular, feito de couro, recheado com penas, para ser chutado com os pés ou batido com um bastão | qualquer coisa com formato de bola; algo esférico ou quase esférico | bola; refere-se a certos artigos esportivos (geralmente redondos e tridimensionais) | jogo; partida; referência a esportes com bola | o Globo; a Terra; referindo-se especificamente à Terra}
  \end{phonetics}
\end{entry}

\begin{entry}{球队}{11,4}{⽟、⾩}
  \begin{phonetics}{球队}{qiu2 dui4}[][HSK 2]
    \definition[个,支]{s.}{equipe (basquete, futebol, etc.); equipe de atletas formada para competições esportivas com bola, como times de basquete, futebol, etc.}
  \end{phonetics}
\end{entry}

\begin{entry}{球场}{11,6}{⽟、⼟}
  \begin{phonetics}{球场}{qiu2 chang3}[][HSK 2]
    \definition[个,座]{s.}{quadra; campo; terreno para jogos com bola; campos para a prática de esportes com bola, como basquete, futebol, tênis e vôlei, cuja forma, tamanho e equipamentos variam de acordo com as exigências de cada esporte}
  \end{phonetics}
\end{entry}

\begin{entry}{球拍}{11,8}{⽟、⼿}
  \begin{phonetics}{球拍}{qiu2pai1}
    \definition{s.}{raquete}
  \end{phonetics}
\end{entry}

\begin{entry}{球迷}{11,9}{⽟、⾡}
  \begin{phonetics}{球迷}{qiu2mi2}[][HSK 3]
    \definition[个,位,名,些]{s.}{fã (de esportes de bola); pessoas obcecadas por jogar ou assistir jogos de bola}
  \end{phonetics}
\end{entry}

\begin{entry}{球鞋}{11,15}{⽟、⾰}
  \begin{phonetics}{球鞋}{qiu2 xie2}[][HSK 2]
    \definition[双,只,款]{s.}{tênis de ginástica; tênis de tênis; tênis esportivos}
  \end{phonetics}
\end{entry}

\begin{entry}{理}{11}{⽟}
  \begin{phonetics}{理}{li3}[][HSK 6]
    \definition*{s.}{Sobrenome Li}
    \definition{s.}{textura; grão (em madeira, pele, etc.) | ordem; sequência | razão; lógica; verdade | ciências naturais (especialmente física)}
    \definition{v.}{gerenciar; executar | colocar em ordem; arrumar | (geralmente no negativo) prestar atenção a; fazer um gesto ou falar com | tratar | colocar em ordem; limpar | tomar conhecimento de; prestar atenção a; expressar uma atitude; expressar uma opinião}
  \end{phonetics}
\end{entry}

\begin{entry}{理发}{11,5}{⽟、⼜}
  \begin{phonetics}{理发}{li3fa4}[][HSK 3]
    \definition{v.+compl.}{cortar e aparar o cabelo; ter (dar) um corte de cabelo}
  \end{phonetics}
\end{entry}

\begin{entry}{理由}{11,5}{⽟、⽥}
  \begin{phonetics}{理由}{li3you2}[][HSK 3]
    \definition[个,条,种,堆]{s.}{razão; justificativa; fundamento; a razão pela qual as coisas são feitas desta ou daquela maneira}
  \end{phonetics}
\end{entry}

\begin{entry}{理论}{11,6}{⽟、⾔}
  \begin{phonetics}{理论}{li3lun4}[][HSK 3]
    \definition[套,个]{s.}{teoria; uma série de conclusões tiradas pelas pessoas sobre atividades naturais ou sociais}
    \definition{v.}{argumentar; raciocinar com alguém; discutir com outras pessoas sobre quem está certo ou errado}
  \end{phonetics}
\end{entry}

\begin{entry}{理想}{11,13}{⽟、⼼}
  \begin{phonetics}{理想}{li3xiang3}[][HSK 2]
    \definition{adj.}{ideal; perfeito | conforme o desejado; satisfatório}
    \definition{adv.}{idealmente}
    \definition[个,种]{s.}{ideal; sonho; aspiração}
  \end{phonetics}
\end{entry}

\begin{entry}{理解}{11,13}{⽟、⾓}
  \begin{phonetics}{理解}{li3jie3}[][HSK 3]
    \definition{v.}{entender; compreender; compreender o significado por trás de algo através da reflexão e do aprendizado | entender com empatia; achar que os outros não conseguem fazer determinada coisa e demonstrar compaixão, perdão e não crítica}
  \end{phonetics}
\end{entry}

\begin{entry}{甜}{11}{⽢}
  \begin{phonetics}{甜}{tian2}[][HSK 3]
    \definition{adj.}{doce; melado | agradável; confortável; fazer as pessoas se sentirem confortáveis e felizes | (sono) profundo | feliz; descreve o sentimento de felicidade}
  \end{phonetics}
\end{entry}

\begin{entry}{甜心}{11,4}{⽢、⼼}
  \begin{phonetics}{甜心}{tian2xin1}
    \definition{s.}{querido}
  \end{phonetics}
\end{entry}

\begin{entry}{甜头}{11,5}{⽢、⼤}
  \begin{phonetics}{甜头}{tian2tou5}
    \definition{s.}{benefício | sabor doce (de poder, sucesso, etc.)}
  \end{phonetics}
\end{entry}

\begin{entry}{甜玉米}{11,5,6}{⽢、⽟、⽶}
  \begin{phonetics}{甜玉米}{tian2 yu4mi3}
    \definition{s.}{milho doce}
  \end{phonetics}
\end{entry}

\begin{entry}{甜言}{11,7}{⽢、⾔}
  \begin{phonetics}{甜言}{tian2yan2}
    \definition{s.}{boa conversa | palavras amáveis}
  \end{phonetics}
\end{entry}

\begin{entry}{甜品}{11,9}{⽢、⼝}
  \begin{phonetics}{甜品}{tian2pin3}
    \definition{s.}{sobremesa}
  \end{phonetics}
\end{entry}

\begin{entry}{甜食}{11,9}{⽢、⾷}
  \begin{phonetics}{甜食}{tian2shi2}
    \definition{s.}{doces | sobremesa}
  \end{phonetics}
\end{entry}

\begin{entry}{甜酒}{11,10}{⽢、⾣}
  \begin{phonetics}{甜酒}{tian2jiu3}
    \definition{s.}{licor doce}
  \end{phonetics}
\end{entry}

\begin{entry}{甜甜圈}{11,11,11}{⽢、⽢、⼞}
  \begin{phonetics}{甜甜圈}{tian2tian2quan1}
    \definition{s.}{rosquinha | \emph{doughnut}}
  \end{phonetics}
\end{entry}

\begin{entry}{甜菊}{11,11}{⽢、⾋}
  \begin{phonetics}{甜菊}{tian2ju2}
    \definition{s.}{estévia, arbusto cujas folhas produzem um substituto para o açúcar}
  \end{phonetics}
\end{entry}

\begin{entry}{甜筒}{11,12}{⽢、⽵}
  \begin{phonetics}{甜筒}{tian2tong3}
    \definition{s.}{sorvete de casquinha}
  \end{phonetics}
\end{entry}

\begin{entry}{甜稚}{11,13}{⽢、⽲}
  \begin{phonetics}{甜稚}{tian2zhi4}
    \definition{s.}{doce e inocente}
  \end{phonetics}
\end{entry}

\begin{entry}{甜酸}{11,14}{⽢、⾣}
  \begin{phonetics}{甜酸}{tian2suan1}
    \definition{adj.}{agridoce}
  \end{phonetics}
\end{entry}

\begin{entry}{略}{11}{⽥}
  \begin{phonetics}{略}{lve4}
    \definition{adv.}{ligeiramente | marginalmente | aproximadamente}
  \end{phonetics}
\end{entry}

\begin{entry}{略微}{11,13}{⽥、⼻}
  \begin{phonetics}{略微}{lve4wei1}
    \definition{adv.}{ligeiramente | marginalmente | aproximadamente}
  \end{phonetics}
\end{entry}

\begin{entry}{盒}{11}{⽫}
  \begin{phonetics}{盒}{he2}[][HSK 5]
    \definition{clas.}{caixa (de pequena dimensão)}
    \definition[个]{s.}{caixa; estojo; recipiente; receptáculo}
  \end{phonetics}
\end{entry}

\begin{entry}{盒子}{11,3}{⽫、⼦}
  \begin{phonetics}{盒子}{he2zi5}[][HSK 5]
    \definition[个]{s.}{caixa; recipiente que têm tampas na parte superior e podem conter coisas dentro, geralmente é pequeno e plano}
  \end{phonetics}
\end{entry}

\begin{entry}{盒饭}{11,7}{⽫、⾷}
  \begin{phonetics}{盒饭}{he2 fan4}[][HSK 5]
    \definition{s.}{refeição embalada; marmita; \emph{fast-food} vendida em caixas}
  \end{phonetics}
\end{entry}

\begin{entry}{盖}{11}{⽫}
  \begin{phonetics}{盖}{gai4}[][HSK 4]
    \definition*{s.}{Sobrenome Gai}
    \definition{adj.}{excelente; soberbo; fantástico}
    \definition{adv.}{cerca de; ao redor; aproximadamente; expressa um julgamento especulativo sobre algo, ou uma explicação da causa, o que é equivalente a 大概 ou 原来}
    \definition{conj.}{para; porque; dando continuidade à frase anterior, afirmando a razão ou causa, com tom incerto}
    \definition{s.}{tampa; capa; cobertura; algo que cobre ou sela a parte superior de um objeto | carapaça; concha (de tartaruga, caranguejo, etc.); ossos em formato de crânio em certas partes do corpo humano; as conchas nas costas de certos animais | dossel; capota; toldo | nivelador (uma ferramenta agrícola usada para nivelar terras)}
    \definition{v.}{cobrir; proteger; colocar uma capa em; colocar uma tampa em um objeto | selar; afixar um selo em | superar; sobressair; sobrepujar; ultrapassar | construir; colocar para cima | esconder; ocultar; encobrir | nivelar o terreno com um nivelador (ferramenta agrícola)}
  \seealsoref{大概}{da4gai4}
  \seealsoref{原来}{yuan2lai2}
  \end{phonetics}
  \begin{phonetics}{盖}{ge3}
    \definition*{s.}{Sobrenome Ge}
  \end{phonetics}
\end{entry}

\begin{entry}{盗}{11}{⽫}
  \begin{phonetics}{盗}{dao4}
    \definition[个,伙,帮,窝]{s.}{ladrão; assaltante}
    \definition{v.}{roubar; saquear | usurpar; buscar ganho pessoal ou ganho por meios impróprios}
  \end{phonetics}
\end{entry}

\begin{entry}{盗版}{11,8}{⽫、⽚}
  \begin{phonetics}{盗版}{dao4ban3}
    \definition{s.}{cópia ilegal; cópia pirata; refere-se a livros, periódicos e produtos audiovisuais pirateados (diferentes dos 正版)}
    \definition{v.}{piratear; copiar ou vender ilegalmente; para obter lucros enormes, reimprimir ou copiar livros, periódicos ou produtos audiovisuais em grandes quantidades sem o consentimento do detentor dos direitos autorais}
  \seealsoref{正版}{zheng4 ban3}
  \end{phonetics}
\end{entry}

\begin{entry}{盘}{11}{⽫}
  \begin{phonetics}{盘}{pan2}[][HSK 4]
    \definition*{s.}{Sobrenome Pan}
    \definition{clas.}{para pratos, pedras de moer, etc. | para jogos de xadrez e de bola | para as coisas que estão entrelaçadas, emaranhadas}
    \definition{s.}{bandeja; tabuleiro | recipiente plano e raso, como uma bandeja, prato, travessa etc.  | preço atual; cotação de mercado; refere-se ao preço básico pelo qual as commodities são negociadas}
    \definition{v.}{enrolar; torcer; enrolar (para cima); entrelaçar; cercar | construir (assentando tijolos, pedras, etc.) | checar; examinar; interrogar; verificar um por um ou repetidamente (quantidade, situação, etc.) | transferir a propriedade de; passar para outra pessoa | carregar; transportar}
  \end{phonetics}
\end{entry}

\begin{entry}{盘子}{11,3}{⽫、⼦}
  \begin{phonetics}{盘子}{pan2zi5}[][HSK 4]
    \definition[个,叠,套,只]{s.}{prato; utensílio de fundo raso para guardar objetos, maior do que um pires, geralmente de formato redondo | situação de mercado; taxa de mercado; transação comercial}
  \end{phonetics}
\end{entry}

\begin{entry}{盛}{11}{⽫}
  \begin{phonetics}{盛}{cheng2}
    \definition{v.}{encher; encher com uma concha; colocar as coisas em recipientes; especialmente colocar alimentos em tigelas, pratos e outros recipientes | segurar; conter; acomodar}
  \end{phonetics}
  \begin{phonetics}{盛}{sheng4}
    \definition*{s.}{Sobrenome Sheng}
    \definition{adj.}{florescente; próspero | vigoroso; enérgico | grandioso; magnífico | abundante; profundo | popular; comum; difundido; universal | amplo; generoso; abundante; suficiente | ótimo}
    \definition{adv.}{muito; profundamente}
  \end{phonetics}
\end{entry}

\begin{entry}{盛宴}{11,10}{⽫、⼧}
  \begin{phonetics}{盛宴}{sheng4yan4}
    \definition{s.}{celebração}
  \end{phonetics}
\end{entry}

\begin{entry}{眯}{11}{⽬}
  \begin{phonetics}{眯}{mi1}
    \definition{v.}{estreitar os olhos | esmagar | (dialeto) tirar uma soneca}
  \end{phonetics}
  \begin{phonetics}{眯}{mi2}
    \definition{v.}{cegar (como com poeira)}
  \end{phonetics}
\end{entry}

\begin{entry}{眼}{11}{⽬}
  \begin{phonetics}{眼}{yan3}[][HSK 2]
    \definition{clas.}{usado para grandes coisas ocas: poços, fogões, panelas, etc.}
    \definition[双,只]{s.}{olho; o órgão visual dos humanos ou animais | abertura; pequeno furo; pequeno buraco | ponto-chave; refere-se ao ponto-chave das coisas | armadilha; um termo do jogo Go que se refere a um espaço vazio cercado pelas peças de um jogador, onde o outro jogador não pode colocar uma peça, a menos que haja circunstâncias especiais | uma batida sem acento na música tradicional chinesa}
  \end{phonetics}
\end{entry}

\begin{entry}{眼光}{11,6}{⽬、⼉}
  \begin{phonetics}{眼光}{yan3guang1}[][HSK 5]
    \definition{s.}{olho; visão | visão; percepção; previsão; capacidade de observar e identificar coisas | vista; ponto de vista}
  \end{phonetics}
\end{entry}

\begin{entry}{眼花缭乱}{11,7,15,7}{⽬、⾋、⽷、⼄}
  \begin{phonetics}{眼花缭乱}{yan3hua1liao2luan4}
    \definition{v.}{ficar deslumbrado | deslumbrar}
  \end{phonetics}
\end{entry}

\begin{entry}{眼证}{11,7}{⽬、⾔}
  \begin{phonetics}{眼证}{yan3zheng4}
    \definition{s.}{testemunha ocular}
  \end{phonetics}
\end{entry}

\begin{entry}{眼里}{11,7}{⽬、⾥}
  \begin{phonetics}{眼里}{yan3 li3}[][HSK 4]
    \definition{s.}{nos olhos de uma pessoa; dentro de sua visão}
  \end{phonetics}
\end{entry}

\begin{entry}{眼泪}{11,8}{⽬、⽔}
  \begin{phonetics}{眼泪}{yan3 lei4}[][HSK 4]
    \definition[滴,行]{s.}{lágrimas; termo genérico para lágrimas; fluido incolor e transparente secretado pelas glândulas lacrimais no olho, que serve para proteger o olho}
  \end{phonetics}
\end{entry}

\begin{entry}{眼前}{11,9}{⽬、⼑}
  \begin{phonetics}{眼前}{yan3 qian2}[][HSK 3]
    \definition{adv.}{agora; (no) momento}
    \definition{s.}{diante dos olhos; diante de | agora; (no) momento}
  \end{phonetics}
\end{entry}

\begin{entry}{眼柄}{11,9}{⽬、⽊}
  \begin{phonetics}{眼柄}{yan3bing3}
    \definition{s.}{pedúnculo ocular (de crustáceo, etc.)}
  \end{phonetics}
\end{entry}

\begin{entry}{眼袋}{11,11}{⽬、⾐}
  \begin{phonetics}{眼袋}{yan3dai4}
    \definition{s.}{inchaço sob os olhos}
  \end{phonetics}
\end{entry}

\begin{entry}{眼睛}{11,13}{⽬、⽬}
  \begin{phonetics}{眼睛}{yan3jing5}[][HSK 2]
    \definition[双,只]{s.}{olho(s)}
  \end{phonetics}
\end{entry}

\begin{entry}{眼镜}{11,16}{⽬、⾦}
  \begin{phonetics}{眼镜}{yan3jing4}[][HSK 4]
    \definition[副]{s.}{óculos; óculos de grau}
  \end{phonetics}
\end{entry}

\begin{entry}{着}{11}{⽬}
  \begin{phonetics}{着}{zhao1}
    \definition{interj.}{tudo bem; tudo certo; \emph{O.K.}}
    \definition{s.}{uma jogada no xadrez | truque; meio; artifício; manobra; estratégia}
    \definition{v.}{colocar dentro; guardar}
  \end{phonetics}
  \begin{phonetics}{着}{zhao2}
    \definition{v.}{tocar (contato físico) | sentir; ser afetado por | queimar; acender | adormecer; cair no sono | acertar em cheio; ter sucesso em; usado após o verbo, indica que o objetivo foi alcançado ou que houve um resultado}
  \end{phonetics}
  \begin{phonetics}{着}{zhe5}[][HSK 1,4]
    \definition{part.}{adicionar a um verbo ou adjetivo para indicar uma ação ou estado contínuo | em frases que começam com uma palavra que indica um lugar, acrescente ao verbo para indicar um estado resultante | em frases imperativas, usado após verbos ou adjetivos para dar ênfase | adicionado após certos verbos, transforma-se em preposição}
    \definition{s.}{um movimento no xadrez |movimento; estratégia; estratagema}
  \end{phonetics}
  \begin{phonetics}{着}{zhuo2}
    \definition{v.}{vestir (roupas); vestir-se | tocar; entrar em contato com; aproximar-se de; (contato físico) | enviar; despachar | expressão usada em documentos oficiais antigos, indicando um tom de ordem | aplicar; usar; adicionar; anexar}
  \end{phonetics}
\end{entry}

\begin{entry}{着手}{11,4}{⽬、⼿}
  \begin{phonetics}{着手}{zhuo2shou3}
    \definition{v.}{colocar a mão nisso | estabelecer | começar uma tarefa}
  \end{phonetics}
\end{entry}

\begin{entry}{着火}{11,4}{⽬、⽕}
  \begin{phonetics}{着火}{zhao2huo3}[][HSK 4]
    \definition{v.}{pegar fogo; estar em chamas}
  \end{phonetics}
\end{entry}

\begin{entry}{着地}{11,6}{⽬、⼟}
  \begin{phonetics}{着地}{zhao2di4}
    \definition{v.}{pousar | tocar o chão}
  \end{phonetics}
\end{entry}

\begin{entry}{着花}{11,7}{⽬、⾋}
  \begin{phonetics}{着花}{zhao2hua1}
    \definition{v.}{florescer}
  \end{phonetics}
  \begin{phonetics}{着花}{zhuo2hua1}
    \definition{s.}{floração}
    \definition{v.}{florescer}
  \end{phonetics}
\end{entry}

\begin{entry}{着急}{11,9}{⽬、⼼}
  \begin{phonetics}{着急}{zhao2ji2}[][HSK 4]
    \definition{adj.}{ansioso; preocupado |}
    \definition{s.}{preocupação; ansiedade}
    \definition{v.+compl.}{preocupar-se | sentir-se ansioso | sentir uma sensação de urgência}
  \end{phonetics}
\end{entry}

\begin{entry}{着凉}{11,10}{⽬、⼎}
  \begin{phonetics}{着凉}{zhao2liang2}
    \definition{v.}{pegar um resfriado}
  \end{phonetics}
\end{entry}

\begin{entry}{着眼}{11,11}{⽬、⽬}
  \begin{phonetics}{着眼}{zhuo2yan3}
    \definition{v.}{ter seus olhos em (um objetivo) | ter algo em mente | concentrar-se}
  \end{phonetics}
\end{entry}

\begin{entry}{着装}{11,12}{⽬、⾐}
  \begin{phonetics}{着装}{zhuo2zhuang1}
    \definition{s.}{roupa | vestimenta}
    \definition{v.}{vestir}
  \end{phonetics}
\end{entry}

\begin{entry}{着想}{11,13}{⽬、⼼}
  \begin{phonetics}{着想}{zhuo2xiang3}
    \definition{v.}{considerar (as necessidades de outras pessoas) | pensar (para os outros)}
  \end{phonetics}
\end{entry}

\begin{entry}{着数}{11,13}{⽬、⽁}
  \begin{phonetics}{着数}{zhao1shu4}
    \definition{s.}{estratégia | movimento (no xadrez, no palco, nas artes marciais) | esquema | truque}
  \end{phonetics}
\end{entry}

\begin{entry}{硕}{11}{⽯}
  \begin{phonetics}{硕}{shuo4}
    \definition*{s.}{Sobrenome Shuo}
    \definition{adj.}{grande; enorme}
    \definition{s.}{mestrado (MBA)}
  \end{phonetics}
\end{entry}

\begin{entry}{硕士}{11,3}{⽯、⼠}
  \begin{phonetics}{硕士}{shuo4shi4}[][HSK 5]
    \definition[个,位,名]{s.}{mestrado}
  \end{phonetics}
\end{entry}

\begin{entry}{票}{11}{⽰}
  \begin{phonetics}{票}{piao4}[][HSK 1]
    \definition{clas.}{para grupos, lotes, transações comerciais}
    \definition[张]{s.}{bilhete; passagem; ingresso | cédula | nota bancária; conta | pessoa mantida em cativeiro por sequestradores para obter resgate; refém | apresentação amadora (de ópera de Pequim, etc.); peças teatrais amadoras}
    \definition{v.}{atuar como amador (na ópera de Pequim)}
  \end{phonetics}
\end{entry}

\begin{entry}{票价}{11,6}{⽰、⼈}
  \begin{phonetics}{票价}{piao4 jia4}[][HSK 3]
    \definition[个]{s.}{o preço de um ingresso; taxa de admissão; taxa de entrada}
  \end{phonetics}
\end{entry}

\begin{entry}{祸}{11}{⽰}
  \begin{phonetics}{祸}{huo4}
    \definition[场]{s.}{infortúnio; desastre; calamidade (oposto de 福) | desgraça; catástrofe}
    \definition{v.}{trazer desastre; arruinar | causar problemas}
  \seealsoref{福}{fu2}
  \end{phonetics}
\end{entry}

\begin{entry}{移}{11}{⽲}
  \begin{phonetics}{移}{yi2}[][HSK 4]
    \definition*{s.}{Sobrenome Yi}
    \definition{v.}{mover; remover; deslocar; mudar | mudar; alterar}
  \end{phonetics}
\end{entry}

\begin{entry}{移民}{11,5}{⽲、⽒}
  \begin{phonetics}{移民}{yi2min2}[][HSK 4]
    \definition{s.}{emigrante; migrantes; aqueles que se mudam para um país ou estado estrangeiro para se estabelecer}
    \definition{v.}{migrar; imigrar}
  \end{phonetics}
\end{entry}

\begin{entry}{移动}{11,6}{⽲、⼒}
  \begin{phonetics}{移动}{yi2dong4}[][HSK 4]
    \definition{v.}{deslocar; mover; mudar}
  \end{phonetics}
\end{entry}

\begin{entry}{竟}{11}{⾳}
  \begin{phonetics}{竟}{jing4}
    \definition{adj.}{todo; por toda parte; do começo ao fim}
    \definition{adv.}{no final; eventualmente | na verdade; inesperadamente; significa algo inesperado, equivalente a 居然}
    \definition{v.}{terminar; completar | investigar}
  \seealsoref{居然}{ju1ran2}
  \end{phonetics}
\end{entry}

\begin{entry}{竟然}{11,12}{⾳、⽕}
  \begin{phonetics}{竟然}{jing4ran2}[][HSK 4]
    \definition{adv.}{de fato; inesperadamente; para surpresa de alguém; chegar ao ponto de; indica que algo é um pouco inesperado}
  \end{phonetics}
\end{entry}

\begin{entry}{章}{11}{⾳}
  \begin{phonetics}{章}{zhang1}[][HSK 6]
    \definition*{s.}{Sobrenome Zhang}
    \definition[枚,个,方]{s.}{(para um livro, carta, etc.) capítulo; seção | ordem | regras; regulamentos; constituição | item; cláusula | (arcaico) memorial ao imperador; memorial ao trono | (arcaico) figura; padrão decorativo | selo; carimbo | distintivo; insígnia; medalha | verso; trecho do poema | escrita literária}
  \end{phonetics}
\end{entry}

\begin{entry}{章鱼}{11,8}{⾳、⿂}
  \begin{phonetics}{章鱼}{zhang1yu2}
    \definition{s.}{polvo | octópode}
  \end{phonetics}
\end{entry}

\begin{entry}{笛}{11}{⽵}
  \begin{phonetics}{笛}{di2}
    \definition[只]{s.}{flauta de bambu | sirene; apito; buzina}
  \end{phonetics}
\end{entry}

\begin{entry}{符}{11}{⽵}
  \begin{phonetics}{符}{fu2}
    \definition*{s.}{Sobrenome Fu}
    \definition[个]{s.}{registro emitido por um governante para generais, enviados, etc., como credenciais na China antiga | símbolo; emblema | figuras mágicas desenhadas por sacerdotes taoístas para invocar ou expulsar espíritos e trazer boa ou má sorte | marca; sinal}
    \definition{v.}{(usado com 相 xiāng ou 不) coincidir com; concordar com | encaixar bem; combinar com; em conformidade com}
  \seealsoref{不}{bu4}
  \seealsoref{相}{xiang1}
  \end{phonetics}
\end{entry}

\begin{entry}{符号}{11,5}{⽵、⼝}
  \begin{phonetics}{符号}{fu2hao4}[][HSK 4]
    \definition[个]{s.}{marca; símbolo; sinais que marcam as coisas | insígnia; emblema; um símbolo usado no corpo para indicar posição, \emph{status}, etc.}
  \end{phonetics}
\end{entry}

\begin{entry}{符合}{11,6}{⽵、⼝}
  \begin{phonetics}{符合}{fu2he2}[][HSK 4]
    \definition{conj.}{de acordo com; concordando com; contando com; alinhado com}
    \definition{v.}{concordar com; estar em conformidade com; corresponder com | gerenciar; lidar}
  \end{phonetics}
\end{entry}

\begin{entry}{笨}{11}{⽵}
  \begin{phonetics}{笨}{ben4}[][HSK 4]
    \definition{adj.}{estúpido; sem graça; tolo; de pouca habilidade; sem inteligência | desajeitado; tosco; inflexível | incômodo; pesado; desajeitado; difícil de manejar; trabalhoso}
  \end{phonetics}
\end{entry}

\begin{entry}{笨蛋}{11,11}{⽵、⾍}
  \begin{phonetics}{笨蛋}{ben4dan4}
    \definition{s.}{bobalhão | cabeça-oca | cabeça-dura}
    \definition{v.}{iludir | enganar}
  \end{phonetics}
\end{entry}

\begin{entry}{第}{11}{⽵}
  \begin{phonetics}{第}{di4}[][HSK 1]
    \definition*{s.}{Sobrenome Di}
    \definition{adv.}{mas, apenas, somente; Indica que a ação não está sujeita a restrições ou condições; equivalente a 只管}
    \definition{conj.}{mas; contudo; orações de conexão; indicando uma relação de transição; equivalente a 但是}
    \definition{pref.}{palavra auxiliar para números ordinais; usado antes de números inteiros, indica ordem}
    \definition{s.}{diferentes notas dos candidatos aprovados nos exames imperiais | a residência de um alto funcionário; grandes residências dos burocratas da era feudal}
  \seealsoref{但是}{dan4 shi4}
  \seealsoref{只管}{zhi3 guan3}
  \end{phonetics}
\end{entry}

\begin{entry}{笼}{11}{⽵}
  \begin{phonetics}{笼}{long2}
    \definition{s.}{armação fechada de bambu, arame, etc. | jaula | gaiola}
  \end{phonetics}
  \begin{phonetics}{笼}{long3}
    \definition{v.}{envolver | cobrir}
  \end{phonetics}
\end{entry}

\begin{entry}{笼子}{11,3}{⽵、⼦}
  \begin{phonetics}{笼子}{long2zi5}
    \definition{s.}{jaula | cesta | gaiola | recipiente}
  \end{phonetics}
  \begin{phonetics}{笼子}{long3zi5}
    \definition{s.}{caixa grande | porta-malas}
  \end{phonetics}
\end{entry}

\begin{entry}{粗}{11}{⽶}
  \begin{phonetics}{粗}{cu1}[][HSK 4]
    \definition{adj.}{largo (em diâmetro); grosso | grosseiro; rude; áspero | áspero; rouco | descuidado; negligente | rude; sem refinamento; vulgar}
    \definition{adv.}{grosseiramente; vagamente}
  \end{phonetics}
\end{entry}

\begin{entry}{粗心}{11,4}{⽶、⼼}
  \begin{phonetics}{粗心}{cu1xin1}[][HSK 4]
    \definition{adj.}{descuidado; irrefletido; (fazer as coisas) de forma desleixada, sem cuidado}
  \end{phonetics}
\end{entry}

\begin{entry}{粗心地做}{11,4,6,11}{⽶、⼼、⼟、⼈}
  \begin{phonetics}{粗心地做}{cu1xin1 di4 zuo4}
    \definition{adj.}{feito descuidadamente}
  \end{phonetics}
\end{entry}

\begin{entry}{粗糙}{11,16}{⽶、⽶}
  \begin{phonetics}{粗糙}{cu1cao1}
    \definition{adj.}{áspero | grosseiro}
  \end{phonetics}
\end{entry}

\begin{entry}{累}{11}{⽷}
  \begin{phonetics}{累}{lei2}
    \definition*{s.}{Sobrenome Lei}
    \definition{adj.}{incômodo; complicado}
    \definition{s.}{corda; cordão | touro na época de acasalamento}
    \definition{v.}{amarrar; prender; atar | copular}
  \end{phonetics}
  \begin{phonetics}{累}{lei3}
    \definition*{s.}{Sobrenome Lei}
    \definition{adj.}{em andamento; repetido; contínuo}
    \definition{v.}{acumular; empilhar; colocar em cima de outro | envolver; implicar | construir empilhando tijolos, pedras, terra, etc.}
  \end{phonetics}
  \begin{phonetics}{累}{lei4}[][HSK 1]
    \definition{adj.}{cansado; exausto; fatigado}
    \definition{v.}{cansar; desgastar; fatigar; esgotar | labutar; trabalhar duro}
  \end{phonetics}
\end{entry}

\begin{entry}{绰}{11}{⽷}
  \begin{phonetics}{绰}{chuo4}
    \definition{adj.}{amplo; espaçoso | (do porte de uma menina) graciosa; flexível}
  \end{phonetics}
\end{entry}

\begin{entry}{绰号}{11,5}{⽷、⼝}
  \begin{phonetics}{绰号}{chuo4hao4}
    \definition{s.}{apelido}
  \end{phonetics}
\end{entry}

\begin{entry}{绳}{11}{⽷}
  \begin{phonetics}{绳}{sheng2}
    \definition*{s.}{Sobrenome Sheng}
    \definition[根]{s.}{corda; cordão; barbante | a linha no marcador de tinta de carpinteiro}
    \definition{v.}{restringir; corrigir; sancionar | medir | continuar}
  \end{phonetics}
\end{entry}

\begin{entry}{绳子}{11,3}{⽷、⼦}
  \begin{phonetics}{绳子}{sheng2zi5}
    \definition[条]{s.}{corda | cordão}
  \end{phonetics}
\end{entry}

\begin{entry}{维}{11}{⽷}
  \begin{phonetics}{维}{wei2}
    \definition*{s.}{Sobrenome Wei}
    \definition{s.}{pensamento | dimensão; conceitos básicos de geometria e teoria do espaço}
    \definition{v.}{ligar; amarrar; manter unido; conectar | manter; manter; salvaguardar; preservar}
  \end{phonetics}
\end{entry}

\begin{entry}{维吾尔}{11,7,5}{⽷、⼝、⼩}
  \begin{phonetics}{维吾尔}{wei2wu2'er3}
    \definition*{s.}{Etnia Uigur de Xinjiang}
  \end{phonetics}
\end{entry}

\begin{entry}{维护}{11,7}{⽷、⼿}
  \begin{phonetics}{维护}{wei2hu4}[][HSK 4]
    \definition{v.}{defender; proteger; manter; preservar}
  \end{phonetics}
\end{entry}

\begin{entry}{维修}{11,9}{⽷、⼈}
  \begin{phonetics}{维修}{wei2xiu1}[][HSK 4]
    \definition{v.}{prestar serviços; manter; reparar; manter em (bom) estado de conservação}
  \end{phonetics}
\end{entry}

\begin{entry}{维持}{11,9}{⽷、⼿}
  \begin{phonetics}{维持}{wei2chi2}[][HSK 4]
    \definition{v.}{manter; conservar; guardar; manter vivo}
  \end{phonetics}
\end{entry}

\begin{entry}{绷}{11}{⽷}
  \begin{phonetics}{绷}{beng1}
    \definition{s.}{estrutura de cama amarrada com cordas, tiras de vime, etc.}
    \definition{v.}{esticar (ou puxar) com força | saltar; quicar | alinhavar; fixar | (dialeto) conseguir fazer algo com dificuldade | (roupas) apertar | costurar ou alfinetar com parcimônia | (dialeto) fraudar; roubar dinheiro}
  \end{phonetics}
  \begin{phonetics}{绷}{beng3}
    \definition{v.}{mostrar uma cara sombria, tensa; parecer descontente | conter o próprio temperamento}
  \end{phonetics}
\end{entry}

\begin{entry}{绷带}{11,9}{⽷、⼱}
  \begin{phonetics}{绷带}{beng1dai4}
    \definition{s.}{curativo | (empréstimo linguístico) \emph{bandage}}
  \end{phonetics}
\end{entry}

\begin{entry}{综}{11}{⽷}
  \begin{phonetics}{综}{zeng4}
    \definition{s.}{liço; fuso; um dispositivo em um tear que separa os fios da urdidura em um padrão alternado para permitir a passagem da lançadeira}
  \end{phonetics}
  \begin{phonetics}{综}{zong1}
    \definition*{s.}{Sobrenome Zong}
    \definition{v.}{reunir; resumir | combinar; reunir}
  \end{phonetics}
\end{entry}

\begin{entry}{综合}{11,6}{⽷、⼝}
  \begin{phonetics}{综合}{zong1he2}[][HSK 4]
    \definition{s.}{síntese}
    \definition{v.}{sintetizar; resumir as partes de uma coisa em um todo unificado após análise (em oposição a 分析); reunir coisas de um tipo ou natureza diferente}
  \seealsoref{分析}{fen1xi1}
  \end{phonetics}
\end{entry}

\begin{entry}{绿}{11}{⽷}
  \begin{phonetics}{绿}{lv4}[][HSK 2]
    \definition*{s.}{Sobrenome Lü}
    \definition{adj.}{verde}
    \definition{v.}{tornar-se verde; ficar verde}
  \end{phonetics}
\end{entry}

\begin{entry}{绿色}{11,6}{⽷、⾊}
  \begin{phonetics}{绿色}{lv4 se4}[][HSK 2]
    \definition{adj.}{verde; ecológico; sem poluição; em conformidade com os requisitos ambientais}
    \definition{s.}{cor verde}
  \end{phonetics}
\end{entry}

\begin{entry}{绿豆}{11,7}{⽷、⾖}
  \begin{phonetics}{绿豆}{lv4dou4}
    \definition{s.}{vagens}
  \end{phonetics}
\end{entry}

\begin{entry}{绿豆芽}{11,7,7}{⽷、⾖、⾋}
  \begin{phonetics}{绿豆芽}{lv4dou4 ya2}
    \definition{s.}{broto de feijão verde}
  \end{phonetics}
\end{entry}

\begin{entry}{绿茶}{11,9}{⽷、⾋}
  \begin{phonetics}{绿茶}{lv4 cha2}[][HSK 3]
    \definition{s.}{chá verde; chá produzido apenas através dos processos de maturação, enrolamento (ou sem enrolamento) e secagem, sem passar por fermentação, com cor verde-claro}
  \end{phonetics}
\end{entry}

\begin{entry}{聊}{11}{⽿}
  \begin{phonetics}{聊}{liao2}[][HSK 6]
    \definition*{s.}{Sobrenome Liao}
    \definition{adv.}{apenas; meramente; provisoriamente; por enquanto | um pouco; ligeiramente}
    \definition{v.}{tagarelar; conversar; bater papo | confiar (ou depender, recorrer) a}
  \end{phonetics}
\end{entry}

\begin{entry}{聊天}{11,4}{⽿、⼤}
  \begin{phonetics}{聊天}{liao2tian1}
    \definition{v.+compl.}{papear | bater papo}
  \end{phonetics}
\end{entry}

\begin{entry}{职}{11}{⽿}
  \begin{phonetics}{职}{zhi2}
    \definition*{s.}{Sobrenome Zhi}
    \definition{prep.}{para; devido a; por causa de}
    \definition{prep.}{(datado) Eu (em relatórios oficiais aos superiores)}
    \definition{s.}{dever; trabalho | cargo; posto; função; responsabilidades; posição}
    \definition{v.}{gerenciar; dirigir | administrar}
  \end{phonetics}
\end{entry}

\begin{entry}{职工}{11,3}{⽿、⼯}
  \begin{phonetics}{职工}{zhi2 gong1}[][HSK 3]
    \definition[个,位,名,些]{s.}{pessoal; trabalhadores e funcionários administrativos}
  \end{phonetics}
\end{entry}

\begin{entry}{职业}{11,5}{⽿、⼀}
  \begin{phonetics}{职业}{zhi2ye4}[][HSK 3]
    \definition{adj.}{profissional; não amador}
    \definition[种,份,个]{s.}{ocupação; profissão; vocação; o trabalho que um indivíduo realiza na sociedade como sua principal fonte de subsistência}
  \end{phonetics}
\end{entry}

\begin{entry}{职务}{11,5}{⽿、⼒}
  \begin{phonetics}{职务}{zhi2wu4}[][HSK 5]
    \definition{s.}{cargo; posto; deveres; função; funções que devem ser desempenhadas de acordo com as especificações do cargo}
  \end{phonetics}
\end{entry}

\begin{entry}{职位}{11,7}{⽿、⼈}
  \begin{phonetics}{职位}{zhi2wei4}[][HSK 5]
    \definition[个]{s.}{posto; posição; cargo que exerce determinadas funções em órgãos ou entidades}
  \end{phonetics}
\end{entry}

\begin{entry}{职员}{11,7}{⽿、⼝}
  \begin{phonetics}{职员}{zhi2yuan2}
    \definition[个,位]{s.}{empregado | trabalhador de escritório | membro da equipe}
  \end{phonetics}
\end{entry}

\begin{entry}{职能}{11,10}{⽿、⾁}
  \begin{phonetics}{职能}{zhi2neng2}[][HSK 5]
    \definition{s.}{função; funções ou papéis que as organizações, instituições, etc. devem desempenhar}
  \end{phonetics}
\end{entry}

\begin{entry}{脖}{11}{⾁}
  \begin{phonetics}{脖}{bo2}
    \definition[个]{s.}{pescoço | em forma de pescoço | parte semelhante ao pescoço}
  \end{phonetics}
\end{entry}

\begin{entry}{脖子}{11,3}{⾁、⼦}
  \begin{phonetics}{脖子}{bo2zi5}
    \definition[个]{s.}{pescoço}
  \end{phonetics}
\end{entry}

\begin{entry}{脚}{11}{⾁}
  \begin{phonetics}{脚}{jiao3}[][HSK 2]
    \definition{clas.}{usado para chutes}
    \definition[只,双]{s.}{pé; a parte inferior das pernas de pessoas ou animais, que entra em contato com o solo | base; pé; a parte inferior do objeto | antigamente, referia-se ao trabalho físico de transporte de cargas | resíduos; sobras}
  \end{phonetics}
  \begin{phonetics}{脚}{jue2}
    \variantof{角}
  \end{phonetics}
\end{entry}

\begin{entry}{脚步}{11,7}{⾁、⽌}
  \begin{phonetics}{脚步}{jiao3 bu4}[][HSK 5]
    \definition{s.}{pé; passo; pisada; refere-se ao movimento das pernas ao caminhar | ritmo; passo; distância entre os pés dianteiros e traseiros ao caminhar}
  \end{phonetics}
\end{entry}

\begin{entry}{脱}{11}{⾁}
  \begin{phonetics}{脱}{tuo1}[][HSK 4]
    \definition{conj.}{se; no caso;}
    \definition{v.}{(cabelo, pele) soltar-se; desprender-se; cair | retirar peça de roupa do corpo | sair de; escapar de | perder (palavras) | livrar-se de algo}
  \end{phonetics}
\end{entry}

\begin{entry}{脱毛}{11,4}{⾁、⽑}
  \begin{phonetics}{脱毛}{tuo1mao2}
    \definition{s.}{depilação}
    \definition{v.}{perder cabelo ou penas | depilar | fazer a barba}
  \end{phonetics}
\end{entry}

\begin{entry}{脱险}{11,9}{⾁、⾩}
  \begin{phonetics}{脱险}{tuo1xian3}
    \definition{v.}{sair do perigo}
  \end{phonetics}
\end{entry}

\begin{entry}{脱离}{11,10}{⾁、⼇}
  \begin{phonetics}{脱离}{tuo1li2}[][HSK 5]
    \definition{v.}{separar-se; divorciar-se; afastar-se; sair (de um determinado ambiente ou situação); romper (uma determinada relação)}
  \end{phonetics}
\end{entry}

\begin{entry}{脸}{11}{⾁}
  \begin{phonetics}{脸}{lian3}[][HSK 2]
    \definition[张,个]{s.}{rosto (de pessoas ou animais); a parte frontal da cabeça, da testa ao queixo | parte frontal de algo | cara; autoestima; aparência | rosto; expressões faciais}
  \end{phonetics}
\end{entry}

\begin{entry}{脸色}{11,6}{⾁、⾊}
  \begin{phonetics}{脸色}{lian3 se4}[][HSK 5]
    \definition{s.}{aparência; tez; cor da pele | aparência; expressão facial | (indicando a condição física de alguém) aparência; cor}
  \end{phonetics}
\end{entry}

\begin{entry}{脸盆}{11,9}{⾁、⽫}
  \begin{phonetics}{脸盆}{lian3 pen2}[][HSK 5]
    \definition[个]{s.}{lavatório; bacia para lavar as mãos e o rosto}
  \end{phonetics}
\end{entry}

\begin{entry}{船}{11}{⾈}
  \begin{phonetics}{船}{chuan2}[][HSK 2]
    \definition*{s.}{Sobrenome Chuan}
    \definition[条,艘,叶,只]{s.}{barco; navio | embarcação; meio de transporte aquático, nome genérico para embarcações}
  \end{phonetics}
\end{entry}

\begin{entry}{船长}{11,4}{⾈、⾧}
  \begin{phonetics}{船长}{chuan2 zhang3}[][HSK 6]
    \definition{s.}{capitão do navio; mestre; marinheiro; comandante; o oficial chefe a bordo}
  \end{phonetics}
\end{entry}

\begin{entry}{船只}{11,5}{⾈、⼝}
  \begin{phonetics}{船只}{chuan2 zhi1}[][HSK 6]
    \definition[艘,条]{s.}{transporte marítimo; embarcação | navio; veleiro}
  \end{phonetics}
\end{entry}

\begin{entry}{船员}{11,7}{⾈、⼝}
  \begin{phonetics}{船员}{chuan2 yuan2}[][HSK 6]
    \definition[名,位,个]{s.}{tripulação (do navio) | membro da tripulação (do navio); marinheiro; marujo; barqueiro; velejador}
  \end{phonetics}
\end{entry}

\begin{entry}{菜}{11}{⾋}
  \begin{phonetics}{菜}{cai4}[][HSK 1]
    \definition*{s.}{Sobrenome Cai}
    \definition{adj.}{pouca habilidade; baixo nível; baixa capacidade}
    \definition[棵,个,道]{s.}{legumes; verduras; plantas que podem ser usadas como alimentos complementares | óleo de canola | prato; item ou prato do cardápio (seja de carne ou de vegetais)}
  \end{phonetics}
\end{entry}

\begin{entry}{菜刀}{11,2}{⾋、⼑}
  \begin{phonetics}{菜刀}{cai4dao1}
    \definition[把]{s.}{faca de vegetais | faca de cozinha | cutelo}
  \end{phonetics}
\end{entry}

\begin{entry}{菜单}{11,8}{⾋、⼗}
  \begin{phonetics}{菜单}{cai4dan1}[][HSK 2]
    \definition[个,分,张]{s.}{menu; lista de pratos | menu (para computadores); lista utilizada para selecionar várias operações diferentes}
  \end{phonetics}
\end{entry}

\begin{entry}{菠}{11}{⾋}
  \begin{phonetics}{菠}{bo1}
    \definition{s.}{espinafre}
  \end{phonetics}
\end{entry}

\begin{entry}{菠菜}{11,11}{⾋、⾋}
  \begin{phonetics}{菠菜}{bo1cai4}
    \definition[棵]{s.}{espinafre}
  \end{phonetics}
\end{entry}

\begin{entry}{菱}{11}{⾋}
  \begin{phonetics}{菱}{ling2}
    \definition{s.}{maruca; caltrop aquático; castanha d'água}
  \end{phonetics}
\end{entry}

\begin{entry}{菱角}{11,7}{⾋、⾓}
  \begin{phonetics}{菱角}{ling2jiao5}
    \definition{s.}{castanha d'água}
  \end{phonetics}
\end{entry}

\begin{entry}{营}{11}{⾋}
  \begin{phonetics}{营}{ying2}
    \definition*{s.}{Sobrenome Ying}
    \definition{s.}{acampamento; quartel; onde o exército está estacionado | batalhão; unidades militares}
    \definition{v.}{procurar | operar; executar; gerenciar}
  \end{phonetics}
\end{entry}

\begin{entry}{营业}{11,5}{⾋、⼀}
  \begin{phonetics}{营业}{ying2ye4}[][HSK 4]
    \definition{v.}{fazer negócios; estar aberto para negócios}
  \end{phonetics}
\end{entry}

\begin{entry}{营养}{11,9}{⾋、⼋}
  \begin{phonetics}{营养}{ying2yang3}[][HSK 3]
    \definition[种]{s.}{nutrição; alimentação; a função do organismo de absorver as substâncias necessárias do meio externo para manter atividades vitais, como crescimento e desenvolvimento | nutrição; alimentação; ato ou processo de fornecer nutrição}
  \end{phonetics}
\end{entry}

\begin{entry}{著}{11}{⽬}
  \begin{phonetics}{著}{zhu4}
    \definition{adj.}{marcado; excelente; óbvio}
    \definition{s.}{livro; trabalho | nativo; pessoa/povo indígena; refere-se a pessoas que se estabeleceram em um lugar por gerações}
    \definition{v.}{mostrar; provar; revelar | escrever}
  \end{phonetics}
\end{entry}

\begin{entry}{著名}{11,6}{⽬、⼝}
  \begin{phonetics}{著名}{zhu4ming2}[][HSK 4]
    \definition{adj.}{famoso; bem conhecido; célebre}
  \end{phonetics}
\end{entry}

\begin{entry}{著作}{11,7}{⽬、⼈}
  \begin{phonetics}{著作}{zhu4zuo4}[][HSK 4]
    \definition[部]{s.}{obra; livro; escritos}
    \definition{v.}{escrever; usar palavras para expressar opiniões, conhecimentos, ideias, sentimentos, etc.}
  \end{phonetics}
\end{entry}

\begin{entry}{虚}{11}{⾌}
  \begin{phonetics}{虚}{xu1}
    \definition*{s.}{Xu, a décima primeira das vinte e oito constelações em que a esfera celeste foi dividida, consistindo de duas estrelas em linha reta, uma em Aquário e a outra em Equuleus | Xu, uma das mansões lunares | Sobrenome Xu}
    \definition{adj.}{vazio; oco; desocupado | desconfiado; tímido | falso; nominal (oposto a 实) | humilde; modesto | fraco; com saúde debilitada | (física) virtual}
    \definition{adv.}{em vão}
    \definition{s.}{vazio; nulidade; anulação | resumo; teoria; princípios orientadores; ideologia política e outros aspectos}
    \definition{v.}{reservar espaço}
  \seealsoref{实}{shi2}
  \end{phonetics}
\end{entry}

\begin{entry}{虚心}{11,4}{⾌、⼼}
  \begin{phonetics}{虚心}{xu1xin1}[][HSK 5]
    \definition{adj.}{modesto; humilde; de mente aberta; não ser presunçoso, ser capaz de aceitar as opiniões dos outros}
  \end{phonetics}
\end{entry}

\begin{entry}{虚伪}{11,6}{⾌、⼈}
  \begin{phonetics}{虚伪}{xu1wei3}
    \definition{adj.}{falso | hipócrita | artificial}
  \end{phonetics}
\end{entry}

\begin{entry}{蛇}{11}{⾍}
  \begin{phonetics}{蛇}{she2}[][HSK 5]
    \definition[条]{s.}{cobra; serpente}
  \end{phonetics}
\end{entry}

\begin{entry}{蛋}{11}{⾍}
  \begin{phonetics}{蛋}{dan4}[][HSK 2]
    \definition[个,只]{s.}{ovo; ovos produzidos por aves, tartarugas, cobras, etc. | algo em forma de ovo | tolo; idiota; metáfora para pessoas com determinadas características (com conotação pejorativa) | se perder; colocado após certos verbos, forma um verbo com conotação pejorativa | testículos; em algumas regiões, refere-se aos testículos de certos animais ou pessoas}
  \end{phonetics}
\end{entry}

\begin{entry}{蛋糕}{11,16}{⾍、⽶}
  \begin{phonetics}{蛋糕}{dan4gao1}[][HSK 5]
    \definition[块,个]{s.}{bolo; bolo fofo feito de ovos e farinha com açúcar e óleo}
  \end{phonetics}
\end{entry}

\begin{entry}{袋}{11}{⾐}
  \begin{phonetics}{袋}{dai4}[][HSK 4]
    \definition{clas.}{usado para coisas que podem ser colocadas nos bolsos | usado para cigarros, narguilé ou tabaco seco}
    \definition[口]{s.}{saco; sacola; bolso; bolsa}
  \end{phonetics}
\end{entry}

\begin{entry}{袭}{11}{⾐}
  \begin{phonetics}{袭}{xi2}
    \definition*{s.}{Sobrenome Xi}
    \definition{clas.}{usado para conjuntos completos de roupas}
    \definition{v.}{fazer um ataque surpresa a; invadir | seguir o padrão de; continuar como antes; fazer o mesmo}
  \end{phonetics}
\end{entry}

\begin{entry}{袭击}{11,5}{⾐、⼐}
  \begin{phonetics}{袭击}{xi2ji1}
    \definition{s.}{ataque (especialmente um ataque surpresa) | invasão}
    \definition{v.}{atacar}
  \end{phonetics}
\end{entry}

\begin{entry}{谎}{11}{⾔}
  \begin{phonetics}{谎}{huang3}
    \definition[句]{s.}{mentira; falsidade}
    \definition{v.}{contar uma mentira; mentir}
  \end{phonetics}
\end{entry}

\begin{entry}{谎话}{11,8}{⾔、⾔}
  \begin{phonetics}{谎话}{huang3hua4}
    \definition{s.}{mentira}
  \end{phonetics}
\end{entry}

\begin{entry}{谐}{11}{⾔}
  \begin{phonetics}{谐}{xie2}
    \definition{adj.}{harmonioso | humorístico}
  \end{phonetics}
\end{entry}

\begin{entry}{象}{11}{⾗}
  \begin{phonetics}{象}{xiang4}
    \definition*{s.}{Sobrenome Xiang}
    \definition[头,群,个]{s.}{elefante | elefante, uma das peças do xadrez chinês | aparência; forma; imagem}
    \definition{v.}{imitar | latir}
  \end{phonetics}
\end{entry}

\begin{entry}{象征}{11,8}{⾗、⼻}
  \begin{phonetics}{象征}{xiang4zheng1}[][HSK 5]
    \definition[种]{s.}{símbolo; emblema; insígnia; \emph{token}; objeto concreto que simboliza um significado especial}
    \definition{v.}{simbolizar; significar; representar; expressar um significado especial através de algo concreto}
  \end{phonetics}
\end{entry}

\begin{entry}{象棋}{11,12}{⾗、⽊}
  \begin{phonetics}{象棋}{xiang4qi2}
    \definition[副]{s.}{xadrez chinês; um tipo de jogo de xadrez em que dois jogadores têm dezesseis peças cada: um general, dois soldados, dois elefantes, duas carruagens, dois cavalos, dois canhões e cinco soldados ; cada jogador joga de acordo com as regras e o vencedor é aquele que der o xeque no general do adversário}
  \end{phonetics}
\end{entry}

\begin{entry}{距}{11}{⾜}
  \begin{phonetics}{距}{ju4}
    \definition{s.}{distância | espora (de um galo, etc.)}
    \definition{v.}{estar separado (longe) de; estar distante de}
  \end{phonetics}
\end{entry}

\begin{entry}{距离}{11,10}{⾜、⼇}
  \begin{phonetics}{距离}{ju4li2}[][HSK 4]
    \definition[个]{s.}{distância}
    \definition{v.}{estar distante de}
  \end{phonetics}
\end{entry}

\begin{entry}{辅}{11}{⾞}
  \begin{phonetics}{辅}{fu3}
    \definition*{s.}{Sobrenome Fu}
    \definition{adj.}{subsidiário}
    \definition{s.}{barras laterais do carrinho atuando como proteção da roda; duas barras retas de madeira são adicionadas na parte externa da roda para prender o cubo | maçã do rosto | assistente oficial; títulos oficiais antigos | (literário) território que circunda a capital}
    \definition{v.}{auxiliar; complementar; suplementar | ajudar}
  \end{phonetics}
\end{entry}

\begin{entry}{辅助}{11,7}{⾞、⼒}
  \begin{phonetics}{辅助}{fu3zhu4}[][HSK 5]
    \definition{adj.}{auxiliar; suplementar; complementar;}
    \definition{v.}{auxiliar; ajudar; colocar os outros em primeiro lugar e dar-lhes alguma ajuda externa}
  \end{phonetics}
\end{entry}

\begin{entry}{辆}{11}{⾞}
  \begin{phonetics}{辆}{liang4}[][HSK 2]
    \definition{clas.}{usado para automóveis, veículos, etc.}
  \end{phonetics}
\end{entry}

\begin{entry}{逮}{11}{⾡}
  \begin{phonetics}{逮}{dai3}
    \definition{v.}{(coloquial) pegar, aproveitar, capturar}
  \end{phonetics}
  \begin{phonetics}{逮}{dai4}
    \definition*{s.}{Sobrenome Dai}
    \definition{v.}{alcançar | prender, usado em 逮捕}
  \seealsoref{逮捕}{dai4bu3}
  \end{phonetics}
\end{entry}

\begin{entry}{逮捕}{11,10}{⾡、⼿}
  \begin{phonetics}{逮捕}{dai4bu3}
    \definition{v.}{prender | apreender | levar sob custódia}
  \end{phonetics}
\end{entry}

\begin{entry}{逶}{11}{⾡}
  \begin{phonetics}{逶}{wei1}
    \definition{adj.}{sinuoso; tortuoso}
  \end{phonetics}
\end{entry}

\begin{entry}{逶迤}{11,8}{⾡、⾡}
  \begin{phonetics}{逶迤}{wei1yi2}
    \definition{adj.}{sinuoso; tortuoso; descreve a aparência sinuosa e contínua de estradas, montanhas, rios, etc.}
  \end{phonetics}
\end{entry}

\begin{entry}{逻}{11}{⾡}
  \begin{phonetics}{逻}{luo2}
    \definition{s.}{patrulha | (literário) a beira de um riacho de montanha}
    \definition{v.}{patrulhar; fazer rondas}
  \end{phonetics}
\end{entry}

\begin{entry}{逻辑}{11,13}{⾡、⾞}
  \begin{phonetics}{逻辑}{luo2ji5}[][HSK 5]
    \definition{s.}{lógica; lei objetiva; a objetividade das leis que regem o desenvolvimento das coisas | lógica; razão; regras para o pensamento | lógica como ciência do raciocínio, do pensamento; disciplina que estuda a lógica}
  \end{phonetics}
\end{entry}

\begin{entry}{野}{11}{⾥}
  \begin{phonetics}{野}{ye3}[][HSK 6]
    \definition*{s.}{Sobrenome Ye}
    \definition{adj.}{(de plantas ou animais) selvagem; incultivado; não domesticado; indomável (opp. 家) | rude; áspero | desenfreado; abandonado; indisciplinado | ilícito; sem licença}
    \definition{s.}{espaço aberto; o aberto | limite; fronteira | não está no poder; fora do cargo}
  \seealsoref{家}{jia1}
  \end{phonetics}
\end{entry}

\begin{entry}{野生}{11,5}{⾥、⽣}
  \begin{phonetics}{野生}{ye3sheng1}
    \definition{adj.}{selvagem | não domesticado}
  \end{phonetics}
\end{entry}

\begin{entry}{铲}{11}{⾦}
  \begin{phonetics}{铲}{chan3}
    \definition[个,把]{s.}{pá}
    \definition{v.}{trabalhar com uma pá (ou enxada) | levantar (mover) com uma pá}
  \end{phonetics}
\end{entry}

\begin{entry}{铲车}{11,4}{⾦、⾞}
  \begin{phonetics}{铲车}{chan3che1}
    \definition[台]{s.}{empilhadeira}
  \end{phonetics}
\end{entry}

\begin{entry}{银}{11}{⾦}
  \begin{phonetics}{银}{yin2}[][HSK 3]
    \definition*{s.}{Sobrenome Yin}
    \definition{adj.}{prateado; como a cor da prata}
    \definition[锭]{s.}{Ag, prata | refere-se a moeda ou a coisas relacionadas com moeda}
  \end{phonetics}
\end{entry}

\begin{entry}{银色}{11,6}{⾦、⾊}
  \begin{phonetics}{银色}{yin2 se4}
    \definition{s.}{cor prata; prateado}
  \end{phonetics}
\end{entry}

\begin{entry}{银行}{11,6}{⾦、⾏}
  \begin{phonetics}{银行}{yin2hang2}[][HSK 2]
    \definition[个,家,所]{s.}{banco; instituições financeiras que operam depósitos, empréstimos, câmbio, poupança e outros negócios}
  \end{phonetics}
\end{entry}

\begin{entry}{银行卡}{11,6,5}{⾦、⾏、⼘}
  \begin{phonetics}{银行卡}{yin2 hang2 ka3}[][HSK 2]
    \definition{s.}{cartão bancário; cartão ATM}
  \end{phonetics}
\end{entry}

\begin{entry}{银河}{11,8}{⾦、⽔}
  \begin{phonetics}{银河}{yin2he2}
    \definition*{s.}{Via Láctea}
  \seealsoref{银河系}{yin2he2xi4}
  \end{phonetics}
\end{entry}

\begin{entry}{银河系}{11,8,7}{⾦、⽔、⽷}
  \begin{phonetics}{银河系}{yin2he2xi4}
    \definition*{s.}{Galáxia Via Láctea}
  \seealsoref{银河}{yin2he2}
  \end{phonetics}
\end{entry}

\begin{entry}{银牌}{11,12}{⾦、⽚}
  \begin{phonetics}{银牌}{yin2 pai2}[][HSK 3]
    \definition[枚]{s.}{medalha de prata; um tipo de medalha, concedida ao segundo colocado}
  \end{phonetics}
\end{entry}

\begin{entry}{随}{11}{⾩}
  \begin{phonetics}{随}{sui2}[][HSK 3]
    \definition*{s.}{Sobrenome Sui}
    \definition{adv.}{fazer algo imediatamente assim que ocorre, sem demora ou hesitação; usado antes de dois verbos ou frases verbais para indicar que a última ação segue a anterior}
    \definition{prep.}{junto com (alguma outra ação) | apresentando as condições das quais a ação depende}
    \definition{v.}{seguir; vir (ou ir) junto com | concordar com; adaptar-se a | deixar (alguém fazer o que quiser) | (dialeto) parecer-se com; assemelhar-se a | seguir ou agir de acordo com a condição ou circunstância da qual a ação depende}
  \end{phonetics}
\end{entry}

\begin{entry}{随手}{11,4}{⾩、⼿}
  \begin{phonetics}{随手}{sui2shou3}[][HSK 4]
    \definition{adv.}{convenientemente; sem problemas adicionais; casualmente}
  \end{phonetics}
\end{entry}

\begin{entry}{随处}{11,5}{⾩、⼡}
  \begin{phonetics}{随处}{sui2chu4}
    \definition{adv.}{em qualquer lugar}
  \end{phonetics}
\end{entry}

\begin{entry}{随后}{11,6}{⾩、⼝}
  \begin{phonetics}{随后}{sui2 hou4}[][HSK 5]
    \definition{adv.}{logo em seguida; logo depois; indica que segue imediatamente após a ação ou situação anterior (geralmente usado em conjunto com 就)}
  \seealsoref{就}{jiu4}
  \end{phonetics}
\end{entry}

\begin{entry}{随地}{11,6}{⾩、⼟}
  \begin{phonetics}{随地}{sui2di4}
    \definition{adv.}{qualquer lugar | todo lugar}
  \end{phonetics}
\end{entry}

\begin{entry}{随机存取记忆体}{11,6,6,8,5,4,7}{⾩、⽊、⼦、⼜、⾔、⼼、⼈}
  \begin{phonetics}{随机存取记忆体}{sui2ji1cun2qu3ji4yi4ti3}
    \definition{s.}{RAM (\emph{random access memory})}
  \seealsoref{内存}{nei4cun2}
  \seealsoref{随机存取存储器}{sui2ji1cun2qu3cun2chu3qi4}
  \end{phonetics}
\end{entry}

\begin{entry}{随机存取存储器}{11,6,6,8,6,12,16}{⾩、⽊、⼦、⼜、⼦、⼈、⼝}
  \begin{phonetics}{随机存取存储器}{sui2ji1cun2qu3cun2chu3qi4}
    \definition{s.}{RAM (\emph{random access memory})}
  \seealsoref{内存}{nei4cun2}
  \seealsoref{随机存取记忆体}{sui2ji1cun2qu3ji4yi4ti3}
  \end{phonetics}
\end{entry}

\begin{entry}{随时}{11,7}{⾩、⽇}
  \begin{phonetics}{随时}{sui2shi2}[][HSK 2]
    \definition{adv.}{a qualquer momento; em todos os momentos}
  \end{phonetics}
\end{entry}

\begin{entry}{随便}{11,9}{⾩、⼈}
  \begin{phonetics}{随便}{sui2bian4}[][HSK 2]
    \definition{adj.}{relaxado; descontraído; sem restrições; sem limitações | aleatório; casual; descuidado; indiferente; distraído, não pensa bem antes de falar ou agir | casual; informal; não dá importância aos detalhes}
    \definition{conj.}{qualquer; qualquer que seja; não importa}
    \definition{v.}{deixar alguém à vontade}
  \end{phonetics}
\end{entry}

\begin{entry}{随着}{11,11}{⾩、⽬}
  \begin{phonetics}{随着}{sui2zhe5}[][HSK 5]
    \definition{prep.}{junto com; na esteira de; em sintonia com; usado no início da frase ou antes do verbo, indica as condições necessárias para que uma ação, comportamento ou evento ocorra}
  \end{phonetics}
\end{entry}

\begin{entry}{随意}{11,13}{⾩、⼼}
  \begin{phonetics}{随意}{sui2yi4}[][HSK 5]
    \definition{adj.}{aleatório; casual; à vontade; como se deseja}
  \end{phonetics}
\end{entry}

\begin{entry}{雪}{11}{⾬}
  \begin{phonetics}{雪}{xue3}[][HSK 2]
    \definition*{s.}{Sobrenome Xue}
    \definition[场,层]{s.}{neve | algo parecido com neve}
    \definition{v.}{limpar; enxugar; remover}
  \end{phonetics}
\end{entry}

\begin{entry}{雪人}{11,2}{⾬、⼈}
  \begin{phonetics}{雪人}{xue3ren2}
    \definition{s.}{boneco de neve | \emph{Yeti}}
  \end{phonetics}
\end{entry}

\begin{entry}{雪山}{11,3}{⾬、⼭}
  \begin{phonetics}{雪山}{xue3shan1}
    \definition{s.}{montanha coberta de neve}
  \end{phonetics}
\end{entry}

\begin{entry}{雪花}{11,7}{⾬、⾋}
  \begin{phonetics}{雪花}{xue3hua1}
    \definition{s.}{floco de neve}
  \end{phonetics}
\end{entry}

\begin{entry}{雪板}{11,8}{⾬、⽊}
  \begin{phonetics}{雪板}{xue3ban3}
    \definition{s.}{prancha de \emph{snowboard}}
    \definition{v.}{praticar \textit{snowboard}}
  \end{phonetics}
\end{entry}

\begin{entry}{雪葩}{11,12}{⾬、⾋}
  \begin{phonetics}{雪葩}{xue3pa1}
    \definition{s.}{sorvete}
  \end{phonetics}
\end{entry}

\begin{entry}{雪鞋}{11,15}{⾬、⾰}
  \begin{phonetics}{雪鞋}{xue3xie2}
    \definition[双]{s.}{sapatos de neve}
  \end{phonetics}
\end{entry}

\begin{entry}{雪糕}{11,16}{⾬、⽶}
  \begin{phonetics}{雪糕}{xue3gao1}
    \definition{s.}{picolé}
  \end{phonetics}
\end{entry}

\begin{entry}{领}{11}{⾴}
  \begin{phonetics}{领}{ling3}[][HSK 3]
    \definition{clas.}{usado para roupas, mantos, esteiras, tapetes, telas, etc.}
    \definition{s.}{pescoço; gargalo | gola; colarinho; faixa de pescoço | esboço; ponto principal; essência}
    \definition{v.}{conduzir; guiar; orientar | possuir; ser o possuidor de; ter jurisdição sobre | obter; conseguir; receber (o que foi distribuído) | aceitar; tomar |entender; compreender (o significado)}
  \end{phonetics}
\end{entry}

\begin{entry}{领先}{11,6}{⾴、⼉}
  \begin{phonetics}{领先}{ling3xian1}[][HSK 3]
    \definition{v.}{liderar; assumir a liderança; estar na liderança; (velocidade, desempenho, etc.) superar pessoas ou coisas semelhantes, estar na vanguarda}
  \end{phonetics}
\end{entry}

\begin{entry}{领导}{11,6}{⾴、⼨}
  \begin{phonetics}{领导}{ling3dao3}[][HSK 3]
    \definition[个,位,名,些]{s.}{líder; liderança; pessoa que ocupa uma posição de liderança}
    \definition{v.}{liderar; exercer liderança; (elogio) liderar, gerenciar outras pessoas;  trabalhar com outras pessoas ou avançar em direção a um objetivo}
  \end{phonetics}
\end{entry}

\begin{entry}{领带}{11,9}{⾴、⼱}
  \begin{phonetics}{领带}{ling3 dai4}[][HSK 5]
    \definition[条]{s.}{colar; gargantilha; gravata}
  \end{phonetics}
\end{entry}

\begin{entry}{领情}{11,11}{⾴、⼼}
  \begin{phonetics}{领情}{ling3qing2}
    \definition{v.+compl.}{sentir-se grato a alguém}
  \end{phonetics}
\end{entry}

\begin{entry}{颇}{11}{⽪}
  \begin{phonetics}{颇}{po1}
    \definition*{s.}{Sobrenome Po}
    \definition{adv.}{muito, bastante (linguagem escrita)}
  \end{phonetics}
\end{entry}

\begin{entry}{骑}{11}{⾺}
  \begin{phonetics}{骑}{qi2}[][HSK 2]
    \definition{s.}{cavalos ou outros animais para montaria | cavalaria; cavaleiro, também se refere genericamente a qualquer pessoa que monta a cavalo}
    \definition{v.}{montar (um animal ou bicicleta); sentar-se na parte de trás de | montar; abranger ambos os lados}
  \end{phonetics}
\end{entry}

\begin{entry}{骑车}{11,4}{⾺、⾞}
  \begin{phonetics}{骑车}{qi2 che1}[][HSK 2]
    \definition{v.}{andar de bicicleta; pedalar}
  \end{phonetics}
\end{entry}

\begin{entry}{鸽}{11}{⿃}
  \begin{phonetics}{鸽}{ge1}
    \definition[只]{s.}{pombo}[和平鸽。___Pomba da Paz.]
  \end{phonetics}
\end{entry}

\begin{entry}{鸽子}{11,3}{⿃、⼦}
  \begin{phonetics}{鸽子}{ge1zi5}
    \definition{s.}{pombo}
  \end{phonetics}
\end{entry}

\begin{entry}{鹿}{11}{⿅}[Kangxi 198]
  \begin{phonetics}{鹿}{lu4}
    \definition{s.}{cervo | veado}
  \end{phonetics}
\end{entry}

\begin{entry}{麻}{11}{⿇}
  \begin{phonetics}{麻}{ma2}
    \definition*{s.}{Sobrenome Ma}
    \definition{adj.}{áspero; grosseiro | marcado; manchado | espinhas; manchas ásperas; cicatrizes deixadas após a varíola}
    \definition[棵,株]{s.}{nome geral para cânhamo, linho, etc. | fibra de cânhamo, linho, etc. para têxteis | sésamo; gergelim | marcas de varíola; um rosto com marcas de varíola}
    \definition{v.}{anestesiar | corromper (a mente de alguém); envenenar}
  \end{phonetics}
\end{entry}

\begin{entry}{麻将}{11,9}{⿇、⼨}
  \begin{phonetics}{麻将}{ma2jiang4}
    \definition*[副]{s.}{Mahjong}
  \end{phonetics}
\end{entry}

\begin{entry}{麻烦}{11,10}{⿇、⽕}
  \begin{phonetics}{麻烦}{ma2fan5}[][HSK 3]
    \definition{adj.}{incômodo; inconveniente; complicado; trabalhoso; burocrático | incômodo; inconveniente; (a situação) é confusa e complicada}
    \definition[个,些,点,堆]{s.}{problema; inconveniência; assuntos complicados e difíceis de resolver}
    \definition{v.}{incomodar; perturbar; incomodar alguém; irritar; aborrecer; causar incômodo ou sobrecarregar outras pessoas}
  \end{phonetics}
\end{entry}

\begin{entry}{麻辣豆腐}{11,14,7,14}{⿇、⾟、⾖、⾁}
  \begin{phonetics}{麻辣豆腐}{ma2la4 dou4fu5}
    \definition{s.}{tofú guisado em molho picante (prato)}
  \end{phonetics}
\end{entry}

\begin{entry}{黄}{11}{⿈}[Kangxi 201]
  \begin{phonetics}{黄}{huang2}[][HSK 2]
    \definition*{s.}{Rio Huanghe, abreviação de 黄河 | Refere-se ao Imperador Amarelo, um imperador da mitologia chinesa antiga | Sobrenome Huang ou Hwang}
    \definition{adj.}{amarelo | obsceno; indecente; pornográfico; símbolo de corrupção e decadência, referindo-se especificamente à pornografia}
    \definition{s.}{gema; ovas de caranguejo; refere-se a certas coisas de cor amarela}
    \definition{v.}{fracassar; dar errado}
  \seealsoref{黄河}{huang2he2}
  \end{phonetics}
\end{entry}

\begin{entry}{黄瓜}{11,5}{⿈、⽠}
  \begin{phonetics}{黄瓜}{huang2 gua1}[][HSK 4]
    \definition[根,棵,株]{s.}{pepino}
  \end{phonetics}
\end{entry}

\begin{entry}{黄色}{11,6}{⿈、⾊}
  \begin{phonetics}{黄色}{huang2 se4}[][HSK 2]
    \definition{adj.}{decadente; obsceno; erótico; pornográfico; símbolo de corrupção e decadência, referindo-se especificamente à pornografia}
    \definition[种]{s.}{cor amarela}
  \end{phonetics}
\end{entry}

\begin{entry}{黄昏}{11,8}{⿈、⽇}
  \begin{phonetics}{黄昏}{huang2hun1}
    \definition{s.}{anoitecer}
  \end{phonetics}
\end{entry}

\begin{entry}{黄河}{11,8}{⿈、⽔}
  \begin{phonetics}{黄河}{huang2he2}
    \definition*{s.}{Rio Amarelo | Rio Huang He}
  \end{phonetics}
\end{entry}

\begin{entry}{黄油}{11,8}{⿈、⽔}
  \begin{phonetics}{黄油}{huang2you2}
    \definition[盒]{s.}{manteiga}
  \end{phonetics}
\end{entry}

\begin{entry}{黄金}{11,8}{⿈、⾦}
  \begin{phonetics}{黄金}{huang2jin1}[][HSK 4]
    \definition{adj.}{de primeira qualidade; dourado;}
    \definition[块,克,两]{s.}{ouro; \emph{aurum}; um tipo de metal, de cor amarela, mais precioso, abreviação de 金, frequentemente falado como 金子}
  \seealsoref{金}{jin1}
  \seealsoref{金子}{jin1zi5}
  \end{phonetics}
\end{entry}

%%%%% EOF %%%%%


%%%
%%% 12画
%%%

\section*{12画}\addcontentsline{toc}{section}{12画}

\begin{entry}{傍}{12}{⼈}
  \begin{phonetics}{傍}{bang4}
    \definition*{s.}{Sobrenome Bang}
    \definition{v.}{estar perto de (à distância); aproximar-se | estar perto de (no tempo) | depender de; confiar em}
  \end{phonetics}
\end{entry}

\begin{entry}{傍晚}{12,11}{⼈、⽇}
  \begin{phonetics}{傍晚}{bang4wan3}[][HSK 6]
    \definition[个]{s.}{ao entardecer; ao cair da noite; (tarde) refere-se ao momento em que se aproxima o anoitecer, frequentemente usado na linguagem escrita}
  \end{phonetics}
\end{entry}

\begin{entry}{傢}{12}{⼈}
  \begin{phonetics}{傢}{jia1}
    \definition{s.}{usado em 家伙  e 家俱}
    \variantof{家}
  \seealsoref{傢伙}{jia1huo5}
  \seealsoref{家俱}{jia1ju4}
  \end{phonetics}
\end{entry}

\begin{entry}{傢伙}{12,6}{⼈、⼈}
  \begin{phonetics}{傢伙}{jia1huo5}
    \variantof{家伙}
  \end{phonetics}
\end{entry}

\begin{entry}{傢俱}{12,10}{⼈、⼈}
  \begin{phonetics}{傢俱}{jia1ju4}
    \variantof{家俱}
  \end{phonetics}
\end{entry}

\begin{entry}{储}{12}{⼈}
  \begin{phonetics}{储}{chu3}
    \definition*{s.}{Sobrenome Chu}
    \definition{s.}{herdeiro de um trono | herdeiro}
    \definition{v.}{armazenar | guardar; manter (ter) em reserva}
  \end{phonetics}
\end{entry}

\begin{entry}{储存}{12,6}{⼈、⼦}
  \begin{phonetics}{储存}{chu3cun2}[][HSK 6]
    \definition{v.}{armazenar; depositar; colocar em; economizar dinheiro ou coisas que você não precisará em um futuro próximo}
  \end{phonetics}
\end{entry}

\begin{entry}{剩}{12}{⼑}
  \begin{phonetics}{剩}{sheng4}[][HSK 5]
    \definition*{s.}{Sobrenome Sheng}
    \definition{v.}{permanecer; ser deixado (para trás);}
  \end{phonetics}
\end{entry}

\begin{entry}{剩下}{12,3}{⼑、⼀}
  \begin{phonetics}{剩下}{sheng4 xia4}[][HSK 5]
    \definition{v.}{permanecer; ser deixado (para trás); consumir e utilizar, restando apenas os resíduos}
  \end{phonetics}
\end{entry}

\begin{entry}{博}{12}{⼗}
  \begin{phonetics}{博}{bo2}
    \definition*{s.}{Sobrenome Bo}
    \definition{adj.}{rico; abundante | erudito; bem informado | solto; grande | grande}
    \definition{s.}{doutor em filosofia; doutorado}
    \definition{v.}{ter um amplo conhecimento de; ser bem lido | ganhar; vencer | jogar}
  \end{phonetics}
\end{entry}

\begin{entry}{博士}{12,3}{⼗、⼠}
  \begin{phonetics}{博士}{bo2shi4}[][HSK 5]
    \definition{s.}{doutorado; grau de doutor; nível mais alto de um diploma; também, uma pessoa que obteve esse diploma | doutor; antigo título honorífico para uma pessoa que é habilidosa em um determinado ofício ou especializada em uma determinada ocupação | doutor; autoridades que ensinavam as escrituras na China nos tempos antigos}
  \end{phonetics}
\end{entry}

\begin{entry}{博文}{12,4}{⼗、⽂}
  \begin{phonetics}{博文}{bo2wen2}
    \definition{s.}{artigo em um blog}
    \definition{v.}{escrever um artigo em um blog}
  \end{phonetics}
\end{entry}

\begin{entry}{博主}{12,5}{⼗、⼂}
  \begin{phonetics}{博主}{bo2zhu3}
    \definition{s.}{blogueiro}
  \end{phonetics}
\end{entry}

\begin{entry}{博物馆}{12,8,11}{⼗、⽜、⾷}
  \begin{phonetics}{博物馆}{bo2wu4guan3}[][HSK 5]
    \definition[个]{s.}{museu; locais para coleta, armazenamento, pesquisa, exibição e exposição de relíquias culturais ou espécimes relacionados à história, cultura, arte, ciências naturais, ciência e tecnologia, etc.}
  \end{phonetics}
\end{entry}

\begin{entry}{博客}{12,9}{⼗、⼧}
  \begin{phonetics}{博客}{bo2 ke4}[][HSK 5]
    \definition{s.}{\emph{blog}; página da Web ou site gerenciado por um indivíduo, geralmente composto por postagens organizadas da mais recente para a mais antiga | blogueiro; \emph{blogger}; pessoas que possuem ou escrevem \emph{blogs}}
  \end{phonetics}
\end{entry}

\begin{entry}{博览会}{12,9,6}{⼗、⾒、⼈}
  \begin{phonetics}{博览会}{bo2lan3hui4}[][HSK 5]
    \definition[次]{s.}{exposição; feira internacional; exposições de produtos em grande escala}
  \end{phonetics}
\end{entry}

\begin{entry}{厨}{12}{⼚}
  \begin{phonetics}{厨}{chu2}
    \definition[个]{s.}{cozinha}
  \end{phonetics}
\end{entry}

\begin{entry}{厨师}{12,6}{⼚、⼱}
  \begin{phonetics}{厨师}{chu2 shi1}[][HSK 6]
    \definition[名,位,个]{s.}{chefe de cozinha; cozinheiro; alguém que é bom em cozinhar e faz disso uma profissão}
  \end{phonetics}
\end{entry}

\begin{entry}{厨房}{12,8}{⼚、⼾}
  \begin{phonetics}{厨房}{chu2fang2}[][HSK 5]
    \definition[间,个]{s.}{cozinha}
  \end{phonetics}
\end{entry}

\begin{entry}{喂}{12}{⼝}
  \begin{phonetics}{喂}{wei4}[][HSK 2,4]
    \definition{interj.}{Ei!, Olá!, para chamar atenção | Alô? (quando respondendo uma chamada telefônica, pronuncia-se como \dpy{wei2})}
    \definition{v.}{criar; alimentar (animais); dar comida a um animal | alimentar (pessoas); colocar alimentos, medicamentos, etc. na boca de alguém}
  \end{phonetics}
\end{entry}

\begin{entry}{喂奶}{12,5}{⼝、⼥}
  \begin{phonetics}{喂奶}{wei4nai3}
    \definition{v.}{amamentar}
  \end{phonetics}
\end{entry}

\begin{entry}{喂母乳}{12,5,8}{⼝、⽏、⼄}
  \begin{phonetics}{喂母乳}{wei4mu3ru3}
    \definition{s.}{amamentação}
  \end{phonetics}
\end{entry}

\begin{entry}{喂养}{12,9}{⼝、⼋}
  \begin{phonetics}{喂养}{wei4yang3}
    \definition{v.}{alimentar (uma criança, animal doméstico, etc.) | manter | criar (um animal)}
  \end{phonetics}
\end{entry}

\begin{entry}{喂食}{12,9}{⼝、⾷}
  \begin{phonetics}{喂食}{wei4shi2}
    \definition{v.}{alimentar}
  \end{phonetics}
\end{entry}

\begin{entry}{喂哺}{12,10}{⼝、⼝}
  \begin{phonetics}{喂哺}{wei4bu3}
    \definition{v.}{alimentar (um bebê)}
  \end{phonetics}
\end{entry}

\begin{entry}{喂料}{12,10}{⼝、⽃}
  \begin{phonetics}{喂料}{wei4liao4}
    \definition{v.}{alimentar (também no sentido figurativo)}
  \end{phonetics}
\end{entry}

\begin{entry}{善}{12}{⼝}
  \begin{phonetics}{善}{shan4}
    \definition*{s.}{Sobrenome Shan}
    \definition{adj.}{bom; bem | bom; satisfatório | gentil; amigável | familiar}
    \definition{adv.}{bom; bem}
    \definition{s.}{boa ação; ato benevolente; coisas boas (em oposição a 恶)}
    \definition{v.}{fazer sucesso; fazer bem; fazer acontecer | ser bom em; ser especialista (versado) em | ser apto a}
  \seealsoref{恶}{e4}
  \end{phonetics}
\end{entry}

\begin{entry}{善于}{12,3}{⼝、⼆}
  \begin{phonetics}{善于}{shan4yu2}[][HSK 4]
    \definition{adv./v.}{ser bom em; ser hábil em}
  \end{phonetics}
\end{entry}

\begin{entry}{善良}{12,7}{⼝、⾉}
  \begin{phonetics}{善良}{shan4liang2}[][HSK 4]
    \definition{adj.}{de bom coração; bom e honesto; de bom coração e cheio de boa vontade}
  \end{phonetics}
\end{entry}

\begin{entry}{善意}{12,13}{⼝、⼼}
  \begin{phonetics}{善意}{shan4yi4}
    \definition{s.}{boa vontade | benevolência | bondade}
  \end{phonetics}
\end{entry}

\begin{entry}{喊}{12}{⼝}
  \begin{phonetics}{喊}{han3}[][HSK 2]
    \definition{v.}{gritar; clamar; berrar | chamar (uma pessoa) | chamar; dirigir-se a}
  \end{phonetics}
\end{entry}

\begin{entry}{喔}{12}{⼝}
  \begin{phonetics}{喔}{o1}
    \definition{interj.}{Oh!, Entendi!, usado para indicar realização, compreensão}
  \end{phonetics}
\end{entry}

\begin{entry}{喜}{12}{⼝}
  \begin{phonetics}{喜}{xi3}
    \definition{adj.}{feliz; satisfeito; encantado}
    \definition[桩,件]{s.}{evento feliz (especialmente casamento); ocasião para celebração; algo para comemorar | gravidez | casamento ou coisas relacionadas a ele}
    \definition{v.}{gostar; fonte de; ter inclinação para | precisa; requer; combina melhor com; (um certo organismo) precisa ou é adequado para (um certo ambiente ou algo)}
  \end{phonetics}
\end{entry}

\begin{entry}{喜欢}{12,6}{⼝、⽋}
  \begin{phonetics}{喜欢}{xi3huan5}[][HSK 1]
    \definition{adj.}{feliz; encantado; exultante; cheio de alegria}
    \definition{v.}{gostar; amar; ter afeição por; estar interessado em; ter uma boa impressão ou interesse por alguém ou algo}
  \end{phonetics}
\end{entry}

\begin{entry}{喜剧}{12,10}{⼝、⼑}
  \begin{phonetics}{喜剧}{xi3 ju4}[][HSK 5]
    \definition[部,出]{s.}{comédia (oposto de 悲剧) | comédia; uma das principais categorias do teatro; usa o exagero para satirizar e ridicularizar o feio; fenômenos retrógrados; destaca as contradições inerentes a esses fenômenos e seu conflito com coisas saudáveis; costuma provocar risadas; o final geralmente é feliz}
  \seealsoref{悲剧}{bei1 ju4}
  \end{phonetics}
\end{entry}

\begin{entry}{喜爱}{12,10}{⼝、⽖}
  \begin{phonetics}{喜爱}{xi3 ai4}[][HSK 4]
    \definition{v.}{gostar; amar; ter afeição por; estar interessado em; ter uma queda ou sentir interesse por pessoas ou coisas}
  \end{phonetics}
\end{entry}

\begin{entry}{喝}{12}{⼝}
  \begin{phonetics}{喝}{he1}[][HSK 1]
    \definition{interj.}{Meu Deus!; Oh!; Ah!; Uau!}
    \definition{s.}{bebida; especificamente, vinho}
    \definition{v.}{beber; engolir líquidos ou alimentos líquidos | beber bebida alcoólica; referência específica ao consumo de álcool}
  \end{phonetics}
  \begin{phonetics}{喝}{he4}
    \definition{v.}{gritar bem alto}
  \end{phonetics}
\end{entry}

\begin{entry}{喝彩}{12,11}{⼝、⼺}
  \begin{phonetics}{喝彩}{he4cai3}
    \definition{s.}{aclamar | torcer}
  \end{phonetics}
\end{entry}

\begin{entry}{喝醉}{12,15}{⼝、⾣}
  \begin{phonetics}{喝醉}{he1zui4}
    \definition{v.}{ficar bêbado}
  \end{phonetics}
\end{entry}

\begin{entry}{喷}{12}{⼝}
  \begin{phonetics}{喷}{pen1}[][HSK 5]
    \definition{v.}{jorrar; esguichar; expelir sob pressão | borrifar; espalhar; pulverizar}
  \end{phonetics}
  \begin{phonetics}{喷}{pen4}
    \definition{s.}{na época; tempo no mercado; época em que frutas, peixes e camarões são comercializados em grande quantidade | colheita; número de vezes que floresceu e frutificou; número de vezes que foi colhido na maturação}
  \end{phonetics}
\end{entry}

\begin{entry}{喻}{12}{⼝}
  \begin{phonetics}{喻}{yu4}
    \definition{s.}{analogia | símile | metáfora | alegoria}
    \definition{v.}{descrever algo como}
  \end{phonetics}
\end{entry}

\begin{entry}{堤}{12}{⼟}
  \begin{phonetics}{堤}{di1}
    \definition[道,条]{s.}{dique; aterro}
  \end{phonetics}
\end{entry}

\begin{entry}{堤坝}{12,7}{⼟、⼟}
  \begin{phonetics}{堤坝}{di1ba4}
    \definition{s.}{represa | dique | barragem}
  \end{phonetics}
\end{entry}

\begin{entry}{塔}{12}{⼟}
  \begin{phonetics}{塔}{ta3}[][HSK 6]
    \definition*{s.}{Sobrenome Ta}
    \definition[个,座]{s.}{pagode budista; pagode | torre | (química) coluna; torre}[蒸馏塔___torre de destilação]
  \end{phonetics}
\end{entry}

\begin{entry}{奥}{12}{⼤}
  \begin{phonetics}{奥}{ao4}
    \definition*{s.}{Oersted, a unidade eletromagnética de intensidade do campo magnético; abreviação de 奥斯特 | Sobrenome Ao}
    \definition{adj.}{profundo e difícil de entender; abstruso | significado profundo, não é fácil de entender}
    \definition{s.}{canto secreto da casa; antigamente, referia-se ao canto sudoeste de uma casa e também, de modo geral, à profundidade de uma casa}
  \seealsoref{奥斯特}{ao4 si1 te4}
  \end{phonetics}
\end{entry}

\begin{entry}{奥运}{12,7}{⼤、⾡}
  \begin{phonetics}{奥运}{ao4yun4}
    \definition*{s.}{Jogos Olímpicos, Olimpíadas; Abreviação de 奥林匹克运动会}
  \seealsoref{奥林匹克运动会}{ao4lin2pi3ke4 yun4dong4hui4}
  \end{phonetics}
\end{entry}

\begin{entry}{奥运会}{12,7,6}{⼤、⾡、⼈}
  \begin{phonetics}{奥运会}{ao4yun4hui4}
    \definition*{s.}{Jogos Olímpicos, Olimpíadas; Abreviação de 奥林匹克运动会}
  \seealsoref{奥林匹克运动会}{ao4lin2pi3ke4 yun4dong4hui4}
  \end{phonetics}
\end{entry}

\begin{entry}{奥林匹克运动会}{12,8,4,7,7,6,6}{⼤、⽊、⼖、⼗、⾡、⼒、⼈}
  \begin{phonetics}{奥林匹克运动会}{ao4lin2pi3ke4 yun4dong4hui4}
    \definition*{s.}{Jogos Olímpicos, Olimpíadas}
  \end{phonetics}
\end{entry}

\begin{entry}{奥特曼}{12,10,11}{⼤、⽜、⽈}
  \begin{phonetics}{奥特曼}{ao4te4man4}
    \definition*{s.}{Ultraman,  super-herói de ficção científica japonesa}
  \end{phonetics}
\end{entry}

\begin{entry}{奥斯特}{12,12,10}{⼤、⽄、⽜}
  \begin{phonetics}{奥斯特}{ao4 si1 te4}
    \definition{s.}{Oersted}
  \end{phonetics}
\end{entry}

\begin{entry}{媒}{12}{⼥}
  \begin{phonetics}{媒}{mei2}
    \definition{s.}{casamenteiro; intermediário | intermediário; médio}
    \definition{v.}{fazer uma combinação}
  \end{phonetics}
\end{entry}

\begin{entry}{媒体}{12,7}{⼥、⼈}
  \begin{phonetics}{媒体}{mei2ti3}[][HSK 3]
    \definition[家,个,种]{s.}{mídia; mídia de massa; vários meios de comunicação e transmissão de informações, como televisão, rádio, jornais, etc.}
  \end{phonetics}
\end{entry}

\begin{entry}{嫂}{12}{⼥}
  \begin{phonetics}{嫂}{sao3}
    \definition[个,位,名,些]{s.}{esposa do irmão mais velho; cunhada | irmã (uma forma de tratamento para uma mulher casada, mais ou menos da mesma idade)}
  \end{phonetics}
\end{entry}

\begin{entry}{嫂子}{12,3}{⼥、⼦}
  \begin{phonetics}{嫂子}{sao3zi5}
    \definition{s.}{esposa do irmão mais velho}
  \end{phonetics}
\end{entry}

\begin{entry}{富}{12}{⼧}
  \begin{phonetics}{富}{fu4}[][HSK 3]
    \definition*{s.}{Sobrenome Fu}
    \definition{adj.}{rico; abastado; abundante; refere-se a ter muito dinheiro (oposto de 贫) | rico; abundante}
    \definition{v.}{tornar-se rico; enriquecer}
  \seealsoref{贫}{pin2}
  \end{phonetics}
\end{entry}

\begin{entry}{寒}{12}{⼧}
  \begin{phonetics}{寒}{han2}
    \definition*{s.}{Sobrenome Han}
    \definition{adj.}{frio | pobre; necessitado | (autodepreciativo) meu/minha humilde\dots | assustado; medroso | com medo; tremendo (de medo) | humilde}
    \definition{s.}{estação fria; inverno (oposto a 暑) | (medicina chinesa) sintomas causados por fatores frios}
  \seealsoref{暑}{shu3}
  \end{phonetics}
\end{entry}

\begin{entry}{寒冷}{12,7}{⼧、⼎}
  \begin{phonetics}{寒冷}{han2 leng3}[][HSK 4]
    \definition{adj.}{frio; frígido; gélido; gelado}
  \end{phonetics}
\end{entry}

\begin{entry}{寒假}{12,11}{⼧、⼈}
  \begin{phonetics}{寒假}{han2jia4}[][HSK 4]
    \definition[个]{s.}{férias de inverno (feriados); férias escolares no meio do inverno, em janeiro e fevereiro (na China)}
  \end{phonetics}
\end{entry}

\begin{entry}{寓}{12}{⼧}
  \begin{phonetics}{寓}{yu4}
    \definition[座,间,栋]{s.}{residência; morada}
    \definition{v.}{(literário) residir; viver | implicar; conter}
  \end{phonetics}
\end{entry}

\begin{entry}{寓意}{12,13}{⼧、⼼}
  \begin{phonetics}{寓意}{yu4yi4}
    \definition{s.}{moral (de uma história),  lição a ser aprendida, implicação, mensagem, significado metafórico}
  \end{phonetics}
\end{entry}

\begin{entry}{尊}{12}{⼨}
  \begin{phonetics}{尊}{zun1}
    \definition*{s.}{Sobrenome Zun}
    \definition{adj.}{sênior; de uma geração sênior; alto status ou antiguidade}
    \definition{clas.}{usado para estátuas, canhões, etc.}
    \definition{pron.}{seu; vossa; antigamente, referia-se a pessoas ou coisas relacionadas entre si}
    \definition{s.}{um tipo de recipiente para vinho usado nos tempos antigos}
    \definition{v.}{respeitar; reverenciar; venerar; honrar}
  \end{phonetics}
\end{entry}

\begin{entry}{尊重}{12,9}{⼨、⾥}
  \begin{phonetics}{尊重}{zun1zhong4}[][HSK 5]
    \definition{adj.}{sério; adequado; correto; (linguagem, comportamento) não ser descuidado; não ser leviano}
    \definition{v.}{respeitar; valorizar; estimar; tratar com educação; valorizar | tratar com seriedade; levar a sério e tratar com seriedade}
  \end{phonetics}
\end{entry}

\begin{entry}{尊敬}{12,12}{⼨、⽁}
  \begin{phonetics}{尊敬}{zun1jing4}[][HSK 5]
    \definition{adj.}{respeitoso; respeitável}
    \definition{v.}{respeitar; honrar; estimar}
  \end{phonetics}
\end{entry}

\begin{entry}{就}{12}{⼪}
  \begin{phonetics}{就}{jiu4}[][HSK 1]
    \definition{adv.}{de imediato; imediatamente; indica que algo ocorrerá em breve | tão cedo quanto; já; há muito tempo; indica que a ação ocorreu há muito tempo | assim que; logo depois; indica que os eventos se sucedem imediatamente | nesse caso; então; indica que, sob determinadas condições, ocorre naturalmente um determinado resultado | exatamente; precisamente; indica que é exatamente assim | apenas; meramente; somente | tantos quanto; enfatiza a quantidade | apenas; simplesmente; reforço da afirmação | colocado entre dois componentes idênticos, significa tolerância ou indiferença}
    \definition{prep.}{tirar proveito de alguém (algo); expressa condições, oportunidades, etc., equivalente a 趁 | quando se trata de alguém (algo); relativo a; com relação a; sobre; objeto ou escopo da introdução da ação |no local; introduz o local próximo ao qual a ação ocorreu}
    \definition{v.}{ser comido com; ir com; pratos, frutas, etc., acompanhados de alimentos básicos ou bebidas alcoólicas | aproximar-se; mover-se em direção a | ir para; assumir; empreender; envolver-se em; entrar em | realizar; fazer | tirar proveito de; acomodar-se a; adequar-se; encaixar-se | assumir; começar a entrar ou a exercer | seguir; acompanhar}
  \seealsoref{趁}{chen4}
  \end{phonetics}
\end{entry}

\begin{entry}{就业}{12,5}{⼪、⼀}
  \begin{phonetics}{就业}{jiu4ye4}[][HSK 3]
    \definition{v.+compl.}{conseguir um emprego; obter emprego; assumir uma ocupação; começar a trabalhar}
  \end{phonetics}
\end{entry}

\begin{entry}{就是}{12,9}{⼪、⽇}
  \begin{phonetics}{就是}{jiu4 shi4}[][HSK 3]
    \definition{adv.}{exatamente; precisamente; expressar concordância com a afirmação da outra pessoa ou confirmar que a afirmação da outra pessoa está correta | apenas; simplesmente; expressa afirmação, determinação ou ênfase, o significado específico deve ser determinado com base no contexto anterior ou posterior | usado para indicar escolha}
    \definition{conj.}{ainda que; mesmo que se reconheça que essa situação é verdadeira, a situação posterior não mudará}
    \definition{part.}{usado no final de uma frase para expressar afirmação}
  \end{phonetics}
\end{entry}

\begin{entry}{就要}{12,9}{⼪、⾑}
  \begin{phonetics}{就要}{jiu4 yao4}[][HSK 2]
    \definition{adv.}{estar prestes a; estar indo para; estar no ponto de}
  \end{phonetics}
\end{entry}

\begin{entry}{就职}{12,11}{⼪、⽿}
  \begin{phonetics}{就职}{jiu4zhi2}
    \definition{v.}{assumir o cargo | assumir um posto}
  \end{phonetics}
\end{entry}

\begin{entry}{属}{12}{⼫}
  \begin{phonetics}{属}{shu3}[][HSK 3]
    \definition{s.}{categoria | gênero | membros da família; dependentes; familiares; parentes}
    \definition{v.}{estar sob; subordinado a | pertencer a | nascer no ano de (um dos doze animais do zodíaco)}
  \end{phonetics}
  \begin{phonetics}{属}{zhu3}
    \definition{v.}{juntar; combinar | fixar (a mente) em; centrar (a atenção, etc.) em}
  \end{phonetics}
\end{entry}

\begin{entry}{属于}{12,3}{⼫、⼆}
  \begin{phonetics}{属于}{shu3yu2}[][HSK 3]
    \definition{v.}{pertencer a; fazer parte de; pertencer ou ser propriedade de uma determinada parte}
  \end{phonetics}
\end{entry}

\begin{entry}{屡}{12}{⼫}
  \begin{phonetics}{屡}{lv3}
    \definition{adv.}{uma e outra vez; repetidamente | frequentemente}
  \end{phonetics}
\end{entry}

\begin{entry}{屡次}{12,6}{⼫、⽋}
  \begin{phonetics}{屡次}{lv3ci4}
    \definition{adv.}{repetidamente | uma e outra vez | muitas vezes}
  \end{phonetics}
\end{entry}

\begin{entry}{帽}{12}{⼱}
  \begin{phonetics}{帽}{mao4}
    \definition[个,顶]{s.}{chapéu; boné | capa; uma coisa que cobre um objeto e tem a função ou formato de um chapéu | elmo; capacete}
  \end{phonetics}
\end{entry}

\begin{entry}{帽子}{12,3}{⼱、⼦}
  \begin{phonetics}{帽子}{mao4zi5}[][HSK 4]
    \definition[顶,个,种]{s.}{boné; chapéu; capacete | etiqueta; rótulo; marca}
  \end{phonetics}
\end{entry}

\begin{entry}{幅}{12}{⼱}
  \begin{phonetics}{幅}{fu2}[][HSK 5]
    \definition{clas.}{usado para tecidos, telas de lã, pinturas, etc.}
    \definition{s.}{largura do tecido, seda, tweed, etc. | tamanho; largura; geralmente se refere à largura}
  \end{phonetics}
\end{entry}

\begin{entry}{幅度}{12,9}{⼱、⼴}
  \begin{phonetics}{幅度}{fu2du4}[][HSK 5]
    \definition{s.}{alcance; escopo; extensão; largura; largura da propagação de um objeto que vibra ou balança, uma metáfora para a magnitude de uma mudança em algo}
  \end{phonetics}
\end{entry}

\begin{entry}{强}{12}{⼸}
  \begin{phonetics}{强}{jiang4}
    \definition{adj.}{teimoso; inflexível}
  \end{phonetics}
  \begin{phonetics}{强}{qiang2}[][HSK 3]
    \definition*{s.}{Sobrenome Qiang}
    \definition{adj.}{forte; poderoso  (em oposição a 弱) | melhor; superior | mais; extra; adicional; um pouco mais que; usado após uma fração ou decimal para indicar que é um pouco maior que o número | resoluto; firme | violento | alto padrão}
    \definition{v.}{fortalecer; tornar forte; tornar poderoso}
  \seealsoref{弱}{ruo4}
  \end{phonetics}
  \begin{phonetics}{强}{qiang3}
    \definition{v.}{fazer um esforço; esforçar-se}
  \end{phonetics}
\end{entry}

\begin{entry}{强大}{12,3}{⼸、⼤}
  \begin{phonetics}{强大}{qiang2 da4}[][HSK 3]
    \definition{adj.}{forte; poderoso; potente; possante; descreve força forte e grande poder}
  \end{phonetics}
\end{entry}

\begin{entry}{强迫}{12,8}{⼸、⾡}
  \begin{phonetics}{强迫}{qiang3po4}[][HSK 5]
    \definition{v.}{impelir; forçar; impor; compelir; aplicar pessão para obedecer}
  \end{phonetics}
\end{entry}

\begin{entry}{强度}{12,9}{⼸、⼴}
  \begin{phonetics}{强度}{qiang2 du4}[][HSK 5]
    \definition[个]{s.}{intensidade; força | magnitude; rigor; avidez}
  \end{phonetics}
\end{entry}

\begin{entry}{强烈}{12,10}{⼸、⽕}
  \begin{phonetics}{强烈}{qiang2lie4}[][HSK 3]
    \definition{adj.}{muito forte; intenso; poderoso | violento; impetuoso; nível muito alto; atitude muito firme, sem espaço para mudanças | afiado; marcante; mostrado em contraste; muito claro}
  \end{phonetics}
\end{entry}

\begin{entry}{强调}{12,10}{⼸、⾔}
  \begin{phonetics}{强调}{qiang2diao4}[][HSK 3]
    \definition{v.}{salientar; sublinhar; enfatizar; dar ênfase a; vincar}
  \end{phonetics}
\end{entry}

\begin{entry}{悲}{12}{⽕}
  \begin{phonetics}{悲}{bei1}
    \definition{adj.}{triste; pesaroso; melancólico | compassivo; misericordioso}
  \end{phonetics}
\end{entry}

\begin{entry}{悲伤}{12,6}{⽕、⼈}
  \begin{phonetics}{悲伤}{bei1 shang1}[][HSK 5]
    \definition{adj.}{triste; pesaroso}
  \end{phonetics}
\end{entry}

\begin{entry}{悲观}{12,6}{⽕、⾒}
  \begin{phonetics}{悲观}{bei1guan1}
    \definition{adj.}{pessimista; negativismo, falta de confiança no futuro (oposto a 乐观)}
  \seealsoref{乐观}{le4guan1}
  \end{phonetics}
\end{entry}

\begin{entry}{悲剧}{12,10}{⽕、⼑}
  \begin{phonetics}{悲剧}{bei1 ju4}[][HSK 5]
    \definition[部,出]{s.}{tragédia; drama trágico; uma das principais categorias de teatro, caracterizada basicamente pela representação do conflito irreconciliável entre o protagonista e a realidade e seu final trágico | tragédia; evento triste; metáfora para encontro infeliz}
  \end{phonetics}
\end{entry}

\begin{entry}{悲惨}{12,11}{⽕、⽕}
  \begin{phonetics}{悲惨}{bei1can3}[][HSK 6]
    \definition{adj.}{trágico; miserável; extremamente doloroso e triste}
  \end{phonetics}
\end{entry}

\begin{entry}{惑}{12}{⼼}
  \begin{phonetics}{惑}{huo4}
    \definition{v.}{ficar confuso; ficar perplexo | iludir; enganar; confundir}
  \end{phonetics}
\end{entry}

\begin{entry}{惑星}{12,9}{⼼、⽇}
  \begin{phonetics}{惑星}{huo4xing1}
    \definition{s.}{planeta}
  \seealsoref{行星}{xing2xing1}
  \end{phonetics}
\end{entry}

\begin{entry}{惩}{12}{⼼}
  \begin{phonetics}{惩}{cheng2}
    \definition{v.}{receber ou dar aviso | punir; penalizar}
  \end{phonetics}
\end{entry}

\begin{entry}{惩处}{12,5}{⼼、⼡}
  \begin{phonetics}{惩处}{cheng2chu3}
    \definition{v.}{administrar justiça | punir}
  \end{phonetics}
\end{entry}

\begin{entry}{惩罚}{12,9}{⼼、⽹}
  \begin{phonetics}{惩罚}{cheng2fa2}
    \definition{v.}{punir | penalizar}
  \end{phonetics}
\end{entry}

\begin{entry}{愉}{12}{⼼}
  \begin{phonetics}{愉}{yu2}
    \definition{adj.}{satisfeito; feliz; alegre}
  \end{phonetics}
\end{entry}

\begin{entry}{愉快}{12,7}{⼼、⼼}
  \begin{phonetics}{愉快}{yu2kuai4}
    \definition{adj.}{alegre | delicioso | prazeroso | agradável | feliz | encantado}
    \definition{adv.}{alegremente | agradavelmente}
  \end{phonetics}
\end{entry}

\begin{entry}{愤}{12}{⼼}
  \begin{phonetics}{愤}{fen4}
    \definition{s.}{raiva; indignação; ressentimento; exasperação}
    \definition{v.}{ressentir-se; ficar indignado; ficar com raiva}
  \end{phonetics}
\end{entry}

\begin{entry}{愤世嫉俗}{12,5,13,9}{⼼、⼀、⼥、⼈}
  \begin{phonetics}{愤世嫉俗}{fen4shi4ji2su2}
    \definition{v.}{ser cínico | ser amargurado}
  \end{phonetics}
\end{entry}

\begin{entry}{愤怒}{12,9}{⼼、⼼}
  \begin{phonetics}{愤怒}{fen4nu4}
    \definition{adj.}{zangado | indignado}
    \definition{s.}{ira}
  \end{phonetics}
\end{entry}

\begin{entry}{慌}{12}{⼼}
  \begin{phonetics}{慌}{huang1}[][HSK 5]
    \definition{adj.}{agitado; confuso; que inspira terror}
    \definition{v.}{ficar com medo; ficar nervoso}
  \end{phonetics}
\end{entry}

\begin{entry}{慌忙}{12,6}{⼼、⼼}
  \begin{phonetics}{慌忙}{huang1 mang2}[][HSK 5]
    \definition{adj.}{apressado; afobado; com muita pressa}
    \definition{adv.}{apressadamente}
  \end{phonetics}
\end{entry}

\begin{entry}{掌}{12}{⼿}
  \begin{phonetics}{掌}{zhang3}
    \definition{s.}{palma da mão | sola do pé | pata | ferradura}
    \definition{v.}{dar um tapa | segurar na mão | empunhar}
  \end{phonetics}
\end{entry}

\begin{entry}{掌握}{12,12}{⼿、⼿}
  \begin{phonetics}{掌握}{zhang3wo4}[][HSK 5]
    \definition{v.}{compreender; dominar; conhecer bem; compreender as coisas; ser capaz de dominar ou utilizar plenamente | segurar; controlar; ter em mãos; tomar nas mãos}
  \end{phonetics}
\end{entry}

\begin{entry}{掱}{12}{⼿}
  \begin{phonetics}{掱}{shou3}
    \variantof{手}
  \end{phonetics}
\end{entry}

\begin{entry}{揉}{12}{⼿}
  \begin{phonetics}{揉}{rou2}
    \definition{v.}{amassar | massagear | esfregar}
  \end{phonetics}
\end{entry}

\begin{entry}{揉碎}{12,13}{⼿、⽯}
  \begin{phonetics}{揉碎}{rou2sui4}
    \definition{v.}{esmagar | desintegrar-se em pedaços}
  \end{phonetics}
\end{entry}

\begin{entry}{提}{12}{⼿}
  \begin{phonetics}{提}{ti2}[][HSK 2]
    \definition*{s.}{Sobrenome Ti}
    \definition{s.}{concha; utensílio para servir óleo ou vinho | traço ascendente (em caracteres chineses)}
    \definition{v.}{carregar (na mão, com o braço para baixo) ; segurar com as mãos para baixo | elevar; levantar; promover | avançar; antecipar uma data; mudar para uma data anterior; adiar o prazo previsto | levantar; apresentar; indicar ou citar | extrair; retirar (tirar) | (prisioneiros) trazer; entregar | mencionar; referir-se a; abordar}
  \end{phonetics}
\end{entry}

\begin{entry}{提及}{12,3}{⼿、⼃}
  \begin{phonetics}{提及}{ti2ji2}
    \definition{v.}{mencionar | levantar (um assunto) | chamar a atenção de alguém}
  \end{phonetics}
\end{entry}

\begin{entry}{提升}{12,4}{⼿、⼗}
  \begin{phonetics}{提升}{ti2sheng1}
    \definition{v.}{promover (para uma posição de classificação mais alta) | levantar | içar | (figurativo) elevar, levantar, melhorar}
  \end{phonetics}
\end{entry}

\begin{entry}{提出}{12,5}{⼿、⼐}
  \begin{phonetics}{提出}{ti2 chu1}[][HSK 2]
    \definition{v.}{levantar; propor; apresentar; expressar seus desejos, ideias, sugestões, etc. por meio de palavras ou textos}
  \end{phonetics}
\end{entry}

\begin{entry}{提示}{12,5}{⼿、⽰}
  \begin{phonetics}{提示}{ti2shi4}[][HSK 5]
    \definition[个]{s.}{dica; lembrete; pistas ou informações fornecidas para chamar a atenção, fazer com que a outra pessoa pense ou compreenda}
    \definition{v.}{solicitar; lembrar; indicar; alertar; levantar questões que o outro não tenha pensado ou não tenha imaginado, para chamar a atenção dele}
  \end{phonetics}
\end{entry}

\begin{entry}{提问}{12,6}{⼿、⾨}
  \begin{phonetics}{提问}{ti2wen4}[][HSK 3]
    \definition{v.}{\emph{quiz}; fazer uma pergunta; colocar questões para}
  \end{phonetics}
\end{entry}

\begin{entry}{提供}{12,8}{⼿、⼈}
  \begin{phonetics}{提供}{ti2gong1}[][HSK 4]
    \definition{v.}{oferecer; fornecer; suprir; prover; proporcionar}
  \end{phonetics}
\end{entry}

\begin{entry}{提到}{12,8}{⼿、⼑}
  \begin{phonetics}{提到}{ti2 dao4}[][HSK 2]
    \definition{v.}{mencionar; referir-se a; levantar (assunto)}
  \end{phonetics}
\end{entry}

\begin{entry}{提前}{12,9}{⼿、⼑}
  \begin{phonetics}{提前}{ti2qian2}[][HSK 3]
    \definition{adv.}{antecipadamente; faça uma coisa antes de fazer outra}
    \definition{v.}{avançar; adiantar; mudar para uma data anterior; trazer para frente}
  \end{phonetics}
\end{entry}

\begin{entry}{提倡}{12,10}{⼿、⼈}
  \begin{phonetics}{提倡}{ti2chang4}[][HSK 5]
    \definition{v.}{promover; incentivar; recomendar; apresentar as vantagens de algo para incentivar as pessoas a usá-lo ou implementá-lo}
  \end{phonetics}
\end{entry}

\begin{entry}{提起}{12,10}{⼿、⾛}
  \begin{phonetics}{提起}{ti2 qi3}[][HSK 5]
    \definition{v.}{mencionar; falar sobre; abordar | levantar; despertar; estimular; revigorar; alegrar/animar | iniciar; instituir; propor | levantar; pegar}
  \end{phonetics}
\end{entry}

\begin{entry}{提高}{12,10}{⼿、⾼}
  \begin{phonetics}{提高}{ti2gao1}[][HSK 2]
    \definition{v.}{elevar; aprimorar; aumentar; melhorar a posição, o nível, a quantidade, a qualidade e outros aspectos em relação ao estado original}
  \end{phonetics}
\end{entry}

\begin{entry}{提醒}{12,16}{⼿、⾣}
  \begin{phonetics}{提醒}{ti2xing3}[][HSK 4]
    \definition{v.+compl.}{alertar; avisar; advertir; lembrar; apontar para ou chamar a atenção para}
  \end{phonetics}
\end{entry}

\begin{entry}{插}{12}{⼿}
  \begin{phonetics}{插}{cha1}[][HSK 5]
    \definition{v.}{perfurar; inserir | interpor; inserir; colocar no meio}
  \end{phonetics}
\end{entry}

\begin{entry}{插手}{12,4}{⼿、⼿}
  \begin{phonetics}{插手}{cha1shou3}
    \definition{v.+compl.}{envolver-se em | dar uma mão | ter (tomar) uma mão | cutucar o nariz de alguém | intrometer-se}
  \end{phonetics}
\end{entry}

\begin{entry}{插话}{12,8}{⼿、⾔}
  \begin{phonetics}{插话}{cha1hua4}
    \definition{s.}{interrupção | digressão}
    \definition{v.+compl.}{interromper (a fala de alguém)}
  \end{phonetics}
\end{entry}

\begin{entry}{握}{12}{⼿}
  \begin{phonetics}{握}{wo4}[][HSK 5]
    \definition{v.}{segurar; agarrar | agarrar; segurar; empunhar; controlar | pegar pela mão}
  \end{phonetics}
\end{entry}

\begin{entry}{握手}{12,4}{⼿、⼿}
  \begin{phonetics}{握手}{wo4shou3}[][HSK 3]
    \definition{v.+compl.}{apertar as mãos; dar um aperto de mão; estender a mão e apertar a mão do outro é uma forma de saudação ao se encontrar ou se despedir, e também é usado para expressar felicitações ou condolências}
  \end{phonetics}
\end{entry}

\begin{entry}{揭}{12}{⼿}
  \begin{phonetics}{揭}{jie1}[][HSK 6]
    \definition*{s.}{Sobrenome Jie}
    \definition{v.}{rasgar; arrancar; tirar | descobrir; levantar (a tampa, etc.) | expor; mostrar; trazer à luz | (literário) levantar; içar}
  \end{phonetics}
\end{entry}

\begin{entry}{援}{12}{⼿}
  \begin{phonetics}{援}{yuan2}
    \definition*{s.}{Sobrenome Yuan}
    \definition{v.}{puxar com a mão; segurar | citar; referenciar | ajudar; auxiliar; resgatar}
  \end{phonetics}
\end{entry}

\begin{entry}{援助}{12,7}{⼿、⼒}
  \begin{phonetics}{援助}{yuan2zhu4}
    \definition{s.}{assistência}
    \definition{v.}{ajudar | apoiar | assistir}
  \end{phonetics}
\end{entry}

\begin{entry}{搁}{12}{⼿}
  \begin{phonetics}{搁}{ge1}
    \definition{v.}{pôr; colocar | colocar à parte; deixar para trás; deixar para mais tarde| deixar de lado}
  \end{phonetics}
  \begin{phonetics}{搁}{ge2}
    \definition{v.}{suportar; resistir}
  \end{phonetics}
\end{entry}

\begin{entry}{搁浅}{12,8}{⼿、⽔}
  \begin{phonetics}{搁浅}{ge1qian3}
    \definition{v.}{ficar encalhado (navio) | encalhar | (figurativo) encontrar dificuldades e parar}
  \end{phonetics}
\end{entry}

\begin{entry}{搓}{12}{⼿}
  \begin{phonetics}{搓}{cuo1}
    \definition{s.}{torção}
    \definition{v.}{esfregar ou rolar entre as mãos ou dedos |  (no tênis, tênis de mesa, críquete, etc.) cortar | (de roupa, etc.) torcer}
  \end{phonetics}
\end{entry}

\begin{entry}{搜}{12}{⼿}
  \begin{phonetics}{搜}{sou1}[][HSK 5]
    \definition{v.}{procurar | pesquisar | coletar; reunir | revistar}
  \end{phonetics}
\end{entry}

\begin{entry}{搜索}{12,10}{⼿、⽷}
  \begin{phonetics}{搜索}{sou1suo3}[][HSK 5]
    \definition{v.}{procurar; caçar; explorar; pesquisar cuidadosamente; refere-se especificamente à busca militar para identificar situações suspeitas em determinada região, área marítima ou aérea}
  \end{phonetics}
\end{entry}

\begin{entry}{搭}{12}{⼿}
  \begin{phonetics}{搭}{da1}[][HSK 6]
    \definition{v.}{colocar em prática; construir | ficar pendurado; colocar para cima | entrar em contato; juntar-se | adicionar (mais pessoas, dinheiro, etc.) | levantar algo junto |
pegar (um navio, avião, etc.); viajar (ou ir) por}
    \variantof{褡}
  \end{phonetics}
\end{entry}

\begin{entry}{搭讪}{12,5}{⼿、⾔}
  \begin{phonetics}{搭讪}{da1shan4}
    \definition{v.}{bater em alguém | incitar uma conversa | começar a conversar para acabar com um silêncio constrangedor ou uma situação embaraçosa}
  \end{phonetics}
\end{entry}

\begin{entry}{搭档}{12,10}{⼿、⽊}
  \begin{phonetics}{搭档}{da1dang4}[][HSK 6]
    \definition[个,名,位]{s.}{parceiro; colega de trabalho}
    \definition{v.}{cooperar; trabalhar em conjunto; formar pares; colaborar; formar uma parceria}
  \end{phonetics}
\end{entry}

\begin{entry}{搭配}{12,10}{⼿、⾣}
  \begin{phonetics}{搭配}{da1pei4}[][HSK 6]
    \definition{v.}{emparelhar; organizar em pares ou grupos; organizar a distribuição de acordo com certos requisitos | encaixar; combinar}
  \end{phonetics}
\end{entry}

\begin{entry}{散}{12}{⽁}
  \begin{phonetics}{散}{san3}[][HSK 5]
    \definition{adj.}{disperso; fragmentado; não integrado}
    \definition{s.}{medicamento em forma de pó}
    \definition{v.}{divergir; espalhar-se; separar-se; soltar-se; não se manter unido;  desintegrar}
  \end{phonetics}
  \begin{phonetics}{散}{san4}
    \definition{v.}{quebrar; fragmentar; dispersar | dar; distribuir; disseminar; divulgar | dissipar; deixar sai  | terminar um acordo ou contrato; demitir}
  \end{phonetics}
\end{entry}

\begin{entry}{散心}{12,4}{⽁、⼼}
  \begin{phonetics}{散心}{san4xin1}
    \definition{v.+compl.}{aliviar o tédio | desfrutar de uma diversão | estar despreocupado}
  \end{phonetics}
\end{entry}

\begin{entry}{散文}{12,4}{⽁、⽂}
  \begin{phonetics}{散文}{san3wen2}[][HSK 5]
    \definition[个]{s.}{ensaio; prosa; gênero literário, na antiguidade, referia-se a textos em prosa, em oposição à poesia e à prosa paralela; atualmente, refere-se a obras literárias que não sejam poesia, teatro ou romance, incluindo ensaios, contos, crônicas, relatos de viagem, etc.}
  \end{phonetics}
\end{entry}

\begin{entry}{散步}{12,7}{⽁、⽌}
  \begin{phonetics}{散步}{san4bu4}[][HSK 3]
    \definition{v.+compl.}{dar uma volta; dar um passeio; dar uma caminhada}
  \end{phonetics}
\end{entry}

\begin{entry}{敬}{12}{⽁}
  \begin{phonetics}{敬}{jing4}
    \definition*{s.}{Sobrenome Jing}
    \definition{adj.}{respeitoso; reverente}
    \definition{adv.}{respeitosamente}
    \definition{v.}{respeitar; honrar; estimar | oferecer educadamente | envolver-se em; dedicar-se a}
  \end{phonetics}
\end{entry}

\begin{entry}{敬礼}{12,5}{⽁、⽰}
  \begin{phonetics}{敬礼}{jing4li3}
    \definition{s.}{saudação}
    \definition{v.}{saudar}
  \end{phonetics}
\end{entry}

\begin{entry}{斯}{12}{⽄}
  \begin{phonetics}{斯}{si1}
    \definition*{s.}{Sobrenome Si}
    \definition{adv.}{então; assim}
    \definition{pron.}{isto; aqui}
  \end{phonetics}
\end{entry}

\begin{entry}{斯巴达}{12,4,6}{⽄、⼰、⾡}
  \begin{phonetics}{斯巴达}{si1ba1da2}
    \definition*{s.}{Esparta}
  \end{phonetics}
\end{entry}

\begin{entry}{普}{12}{⽇}
  \begin{phonetics}{普}{pu3}
    \definition*{s.}{Sobrenome Pu}
    \definition{adj.}{geral; universal}
  \end{phonetics}
\end{entry}

\begin{entry}{普及}{12,3}{⽇、⼃}
  \begin{phonetics}{普及}{pu3ji2}[][HSK 3]
    \definition{adj.}{popular; universal; onipresente; amplamente compreendido, aceito ou utilizado}
    \definition[种]{v.}{popularizar; disseminar; espalhar entre as pessoas; promover amplamente o conhecimento, a educação, a tecnologia, etc. para popularizá-los}
  \end{phonetics}
\end{entry}

\begin{entry}{普通}{12,10}{⽇、⾡}
  \begin{phonetics}{普通}{pu3 tong1}[][HSK 2]
    \definition{adj.}{comum; normal; geral; médio; em geral, nada de especial, como a maioria das pessoas ou coisas}
  \end{phonetics}
\end{entry}

\begin{entry}{普通话}{12,10,8}{⽇、⾡、⾔}
  \begin{phonetics}{普通话}{pu3tong1hua4}[][HSK 2]
    \definition*{s.}{Mandarim (literalmente "linguagem comum") | Putonghua (fala comum da língua chinesa) | Língua oficial da China}
  \end{phonetics}
\end{entry}

\begin{entry}{普遍}{12,12}{⽇、⾡}
  \begin{phonetics}{普遍}{pu3bian4}[][HSK 3]
    \definition{adj.}{geral; comum; universal; difundido; a existência é muito ampla; tem semelhança}
  \end{phonetics}
\end{entry}

\begin{entry}{景}{12}{⽇}
  \begin{phonetics}{景}{jing3}[][HSK 6]
    \definition*{s.}{Sobrenome Jing}
    \definition{adj.}{grandioso; elevado; grande}
    \definition{s.}{vista; cenário; cena | situação; condição | cenário (de uma peça ou filme) | cena (de uma peça)}
    \definition{v.}{admirar; reverenciar; respeitar}
  \end{phonetics}
\end{entry}

\begin{entry}{景色}{12,6}{⽇、⾊}
  \begin{phonetics}{景色}{jing3se4}[][HSK 3]
    \definition[片,幅,道,处]{s.}{vista; cena; cenário; paisagem}
  \end{phonetics}
\end{entry}

\begin{entry}{景象}{12,11}{⽇、⾗}
  \begin{phonetics}{景象}{jing3 xiang4}[][HSK 5]
    \definition[个]{s.}{cena; visão; vista; quadro}
  \end{phonetics}
\end{entry}

\begin{entry}{晴}{12}{⽇}
  \begin{phonetics}{晴}{qing2}[][HSK 2]
    \definition{adj.}{ensolarado; bom; claro; não há nuvens no céu ou há poucas nuvens}
  \end{phonetics}
\end{entry}

\begin{entry}{晴天}{12,4}{⽇、⼤}
  \begin{phonetics}{晴天}{qing2 tian1}[][HSK 2]
    \definition[个]{s.}{dia ensolarado; tempo sem nuvens ou com poucas nuvens; em meteorologia, refere-se a um tempo em que a cobertura de nuvens no céu é inferior a 10\%}
  \end{phonetics}
\end{entry}

\begin{entry}{晴朗}{12,10}{⽇、⽉}
  \begin{phonetics}{晴朗}{qing2lang3}[][HSK 5]
    \definition{adj.}{bom; claro; ensolarado; céu limpo e sem nuvens}
  \end{phonetics}
\end{entry}

\begin{entry}{智}{12}{⽇}
  \begin{phonetics}{智}{zhi4}
    \definition*{s.}{Sobrenome Zhi}
    \definition{adj.}{engenhoso; sábio; inteligente; astuto}
    \definition{s.}{discernimento; engenhosidade; sagacidade | inteligência; conhecimento; sabedoria; percepção}
  \end{phonetics}
\end{entry}

\begin{entry}{智力}{12,2}{⽇、⼒}
  \begin{phonetics}{智力}{zhi4li4}[][HSK 4]
    \definition{s.}{inteligência; refere-se à capacidade de uma pessoa de conhecer e entender coisas objetivas e aplicar o conhecimento e a experiência para resolver problemas, incluindo memória, observação, imaginação, pensamento e julgamento}
  \end{phonetics}
\end{entry}

\begin{entry}{智能}{12,10}{⽇、⾁}
  \begin{phonetics}{智能}{zhi4neng2}[][HSK 4]
    \definition{adj.}{inteligente (telefone, sistema, etc.); descreve máquinas, equipamentos, tecnologia, etc. que foram processados com alta tecnologia e têm a capacidade de falar, pensar, calcular, resolver problemas, etc., como um ser humano}
    \definition{s.}{intelecto; a capacidade de aprender, agir, pensar, inventar, criar, resolver problemas, etc.}
  \end{phonetics}
\end{entry}

\begin{entry}{智商}{12,11}{⽇、⼝}
  \begin{phonetics}{智商}{zhi4shang1}
    \definition{s.}{quociente de inteligência, QI}
  \end{phonetics}
\end{entry}

\begin{entry}{智障}{12,13}{⽇、⾩}
  \begin{phonetics}{智障}{zhi4zhang4}
    \definition{adj./s.}{retardado}
  \end{phonetics}
\end{entry}

\begin{entry}{智慧}{12,15}{⽇、⼼}
  \begin{phonetics}{智慧}{zhi4hui4}
    \definition{s.}{sabedoria | inteligência}
  \end{phonetics}
\end{entry}

\begin{entry}{暂}{12}{⽇}
  \begin{phonetics}{暂}{zan4}
    \definition{adj.}{de curta duração (oposto a 久) | curto; momentâneo; pouco tempo}
    \definition{adv.}{temporariamente; por enquanto}
  \seealsoref{久}{jiu3}
  \end{phonetics}
\end{entry}

\begin{entry}{暂时}{12,7}{⽇、⽇}
  \begin{phonetics}{暂时}{zan4shi2}[][HSK 5]
    \definition{adj.}{transitório; temporário}
    \definition{adv.}{por enquanto; em pouco tempo}
  \end{phonetics}
\end{entry}

\begin{entry}{暂停}{12,11}{⽇、⼈}
  \begin{phonetics}{暂停}{zan4 ting2}[][HSK 5]
    \definition{s.}{suspensão temporária; refere-se especificamente à suspensão temporária de certas competições desportivas de acordo com as regras}
    \definition{v.}{pausar; suspender; esgotar o tempo}
  \end{phonetics}
\end{entry}

\begin{entry}{暑}{12}{⽇}
  \begin{phonetics}{暑}{shu3}
    \definition{adj.}{calor; clima quente; quente (em oposição a 寒)}
    \definition{s.}{verão}
  \seealsoref{寒}{han2}
  \end{phonetics}
\end{entry}

\begin{entry}{暑假}{12,11}{⽇、⼈}
  \begin{phonetics}{暑假}{shu3 jia4}[][HSK 4]
    \definition[个]{s.}{férias de verão; feriado de verão; férias escolares de verão, na China, durante o sétimo e o oitavo meses do calendário gregoriano}
  \end{phonetics}
\end{entry}

\begin{entry}{曾}{12}{⽈}
  \begin{phonetics}{曾}{ceng2}[][HSK 4]
    \definition{adv.}{indica que uma ação já aconteceu ou um estado já existiu}
  \end{phonetics}
  \begin{phonetics}{曾}{zeng1}
    \definition*{s.}{Sobrenome Zeng}
    \definition{s.}{relacionamento entre bisnetos e bisavós; (parentesco) duas gerações de diferença}
  \end{phonetics}
\end{entry}

\begin{entry}{曾经}{12,8}{⽈、⽷}
  \begin{phonetics}{曾经}{ceng2jing1}[][HSK 3]
    \definition{adv.}{uma vez; indica que houve algum comportamento ou situação}
  \end{phonetics}
\end{entry}

\begin{entry}{替}{12}{⽈}
  \begin{phonetics}{替}{ti4}[][HSK 4]
    \definition{prep.}{para; em nome de}
    \definition{s.}{decadência; declínio; enfraquecimento}
    \definition{v.}{substituir; substituir por; tomar o lugar de}
  \end{phonetics}
\end{entry}

\begin{entry}{替代}{12,5}{⽈、⼈}
  \begin{phonetics}{替代}{ti4 dai4}[][HSK 4]
    \definition{v.}{substituir; suplantar}
  \end{phonetics}
\end{entry}

\begin{entry}{最}{12}{⽈}
  \begin{phonetics}{最}{zui4}[][HSK 1]
    \definition{adv.}{(diante de um adjetivo ou verbo) o mais | (colocado antes de um substantivo de localidade ou de uma palavra que indica um lugar)  mais distante ou mais próximo de (um lugar) | mais; melhor; pior; primeiro; muito; menos; acima de tudo; indica que uma determinada característica excede todas as outras pessoas ou coisas do mesmo tipo}
    \definition{s.}{o máximo; o melhor (ou o mais alto, o maior, etc.)}
  \end{phonetics}
\end{entry}

\begin{entry}{最少}{12,4}{⽈、⼩}
  \begin{phonetics}{最少}{zui4shao3}
    \definition{adv.}{finalmente}
  \end{phonetics}
\end{entry}

\begin{entry}{最优}{12,6}{⽈、⼈}
  \begin{phonetics}{最优}{zui4you1}
    \definition{adj.}{ótimo}
  \end{phonetics}
\end{entry}

\begin{entry}{最先}{12,6}{⽈、⼉}
  \begin{phonetics}{最先}{zui4xian1}
    \definition{adv.}{o primeiro}
  \end{phonetics}
\end{entry}

\begin{entry}{最后}{12,6}{⽈、⼝}
  \begin{phonetics}{最后}{zui4hou4}[][HSK 1]
    \definition{s.}{último; final; definitivo; refere-se ao tempo, local, etc. que vem depois de outros tempos, locais, etc. na ordem sequencial}
  \end{phonetics}
\end{entry}

\begin{entry}{最多}{12,6}{⽈、⼣}
  \begin{phonetics}{最多}{zui4duo1}
    \definition{adv.}{no máximo | máximo}
  \end{phonetics}
\end{entry}

\begin{entry}{最好}{12,6}{⽈、⼥}
  \begin{phonetics}{最好}{zui4hao3}[][HSK 1]
    \definition{adj.}{melhor; de primeira qualidade; excelente}
    \definition{adv.}{seria melhor; seria o ideal; indica a escolha mais adequada entre várias possibilidades}
  \end{phonetics}
\end{entry}

\begin{entry}{最初}{12,7}{⽈、⾐}
  \begin{phonetics}{最初}{zui4chu1}[][HSK 4]
    \definition{adj.}{primordial; inicial; primeiro}
    \definition{adv.}{inicialmente; originalmente}
    \definition{s.}{o período mais antigo; início; começo}
  \end{phonetics}
\end{entry}

\begin{entry}{最近}{12,7}{⽈、⾡}
  \begin{phonetics}{最近}{zui4jin4}[][HSK 2]
    \definition{adj.}{mais próximo}
    \definition{s.}{recentemente; ultimamente; de tarde; refere-se aos dias antes ou logo depois de um discurso | em breve; no futuro próximo; o futuro próximo}
  \end{phonetics}
\end{entry}

\begin{entry}{最远}{12,7}{⽈、⾡}
  \begin{phonetics}{最远}{zui4yuan3}
    \definition{adv.}{mais distante | mais longe}
  \end{phonetics}
\end{entry}

\begin{entry}{最佳}{12,8}{⽈、⼈}
  \begin{phonetics}{最佳}{zui4jia1}
    \definition{adj.}{melhor (atleta, filme etc) | ótimo}
  \end{phonetics}
\end{entry}

\begin{entry}{最终}{12,8}{⽈、⽷}
  \begin{phonetics}{最终}{zui4zhong1}
    \definition{adv.}{pelo menos | finalmente}
    \definition{s.}{final | ultimato}
  \end{phonetics}
\end{entry}

\begin{entry}{最高}{12,10}{⽈、⾼}
  \begin{phonetics}{最高}{zui4gao1}
    \definition{adj.}{altíssimo | supremo | mais alto}
  \end{phonetics}
\end{entry}

\begin{entry}{最善}{12,12}{⽈、⼝}
  \begin{phonetics}{最善}{zui4shan4}
    \definition{adj.}{ótimo | o melhor}
  \end{phonetics}
\end{entry}

\begin{entry}{最新}{12,13}{⽈、⽄}
  \begin{phonetics}{最新}{zui4xin1}
    \definition{adv.}{mais recente | mais novo}
  \end{phonetics}
\end{entry}

\begin{entry}{朝}{12}{⽉}
  \begin{phonetics}{朝}{chao2}[][HSK 3]
    \definition*{s.}{Sobrenome Chao}
    \definition{prep.}{para; em direção a; a direção ou o objeto da ação introduzida, equivalente a 向 ou 对}
    \definition[个]{s.}{corte real; governo; assembleia realizada por um soberano; também se refere à posição no poder, em oposição ao 野 | dinastia, todo o período de governo transmitido de geração em geração por um determinado sobrenome imperial | reinado (de um soberano); o período de reinado de um determinado monarca}
    \definition{v.}{fazer uma peregrinação para; ter uma audiência com (um rei, um imperador, etc.) | estar voltado para; estar em frente a}
  \seealsoref{对}{dui4}
  \seealsoref{向}{xiang4}
  \seealsoref{野}{ye3}
  \end{phonetics}
  \begin{phonetics}{朝}{zhao1}
    \definition{s.}{manhã cedo; manhã | dia}
  \end{phonetics}
\end{entry}

\begin{entry}{朝廷}{12,6}{⽉、⼵}
  \begin{phonetics}{朝廷}{chao2ting2}
    \definition{s.}{corte imperial | dinastia}
  \end{phonetics}
\end{entry}

\begin{entry}{朝鲜}{12,14}{⽉、⿂}
  \begin{phonetics}{朝鲜}{chao2xian3}
    \definition*{s.}{Coréia do Norte}
  \end{phonetics}
\end{entry}

\begin{entry}{期}{12}{⽉}
  \begin{phonetics}{期}{qi1}[][HSK 3]
    \definition{clas.}{questão; número; termo; coisas usadas para parcelamento}
    \definition{s.}{um período de tempo; fase; estágio | horário agendado; data agendada | tempo designado (programado)}
    \definition{v.}{marcar uma consulta | esperar; aguardar | esperar; ter esperança}
  \end{phonetics}
\end{entry}

\begin{entry}{期中}{12,4}{⽉、⼁}
  \begin{phonetics}{期中}{qi1 zhong1}[][HSK 4]
    \definition{adj.}{provisório; interino; intermediário}
  \end{phonetics}
\end{entry}

\begin{entry}{期末}{12,5}{⽉、⽊}
  \begin{phonetics}{期末}{qi1 mo4}[][HSK 4]
    \definition{s.}{terminal; final do prazo; fim do período}
  \end{phonetics}
\end{entry}

\begin{entry}{期间}{12,7}{⽉、⾨}
  \begin{phonetics}{期间}{qi1jian1}[][HSK 4]
    \definition{s.}{prazo; tempo; período}
  \end{phonetics}
\end{entry}

\begin{entry}{期限}{12,8}{⽉、⾩}
  \begin{phonetics}{期限}{qi1xian4}[][HSK 4]
    \definition{s.}{prazo; limite de tempo; tempo alocado; período de tempo limitado, também o limite final do limite de tempo}
  \end{phonetics}
\end{entry}

\begin{entry}{期待}{12,9}{⽉、⼻}
  \begin{phonetics}{期待}{qi1dai4}[][HSK 4]
    \definition{v.}{aguardar; esperar; aguardar ansiosamente; ter em mente a realização de um determinado fim ou a ocorrência de uma determinada situação}
  \end{phonetics}
\end{entry}

\begin{entry}{期望}{12,11}{⽉、⽉}
  \begin{phonetics}{期望}{qi1wang4}[][HSK 5]
    \definition{s.}{esperança; expectativa}
    \definition{v.}{esperar; ter esperança}
  \end{phonetics}
\end{entry}

\begin{entry}{棉}{12}{⽊}
  \begin{phonetics}{棉}{mian2}
    \definition{adj.}{almofadado com algodão; acolchoado}
    \definition[些,种,类]{s.}{termo genérico para algodão ou paina | algodão | material semelhante ao algodão | acolchoado ou estofado de algodão}
  \end{phonetics}
\end{entry}

\begin{entry}{棒}{12}{⽊}
  \begin{phonetics}{棒}{bang4}[][HSK 5]
    \definition{adj.}{bom; forte; excelente}
    \definition[根]{s.}{porrete; bastão; cajado; clava}
  \end{phonetics}
\end{entry}

\begin{entry}{棒冰}{12,6}{⽊、⼎}
  \begin{phonetics}{棒冰}{bang4bing1}
    \definition{s.}{picolé}
  \end{phonetics}
\end{entry}

\begin{entry}{棒棒糖}{12,12,16}{⽊、⽊、⽶}
  \begin{phonetics}{棒棒糖}{bang4bang4tang2}
    \definition[根]{s.}{pirulito}
  \end{phonetics}
\end{entry}

\begin{entry}{棕}{12}{⽊}
  \begin{phonetics}{棕}{zong1}
    \definition{adj.}{marrom}
    \definition[个]{s.}{palmeira | fibra de palmeira; fibra de coco}
  \end{phonetics}
\end{entry}

\begin{entry}{棕褐色}{12,14,6}{⽊、⾐、⾊}
  \begin{phonetics}{棕褐色}{zong1he4 se4}
    \definition{s.}{cor sépia | bronzeado}
  \end{phonetics}
\end{entry}

\begin{entry}{森}{12}{⽊}
  \begin{phonetics}{森}{sen1}
    \definition{adj.}{cheio de árvores | multitudinário; em multidões | escuro; sombrio}
  \end{phonetics}
\end{entry}

\begin{entry}{森林}{12,8}{⽊、⽊}
  \begin{phonetics}{森林}{sen1lin2}[][HSK 4]
    \definition[片,座,处]{s.}{floresta; bosque; normalmente, refere-se a uma grande área de árvores em crescimento; na silvicultura, refere-se a um grande número de árvores que crescem em uma área razoavelmente grande de terra, juntamente com os animais e outras plantas}
  \end{phonetics}
\end{entry}

\begin{entry}{棵}{12}{⽊}
  \begin{phonetics}{棵}{ke1}[][HSK 4]
    \definition{clas.}{para plantas, árvores}
  \end{phonetics}
\end{entry}

\begin{entry}{棹}{12}{⽊}
  \begin{phonetics}{棹}{zhuo1}
    \variantof{桌}
  \end{phonetics}
\end{entry}

\begin{entry}{棺}{12}{⽊}
  \begin{phonetics}{棺}{guan1}
    \definition[副]{s.}{caixão; esquife; ataúde}
  \end{phonetics}
\end{entry}

\begin{entry}{椅}{12}{⽊}
  \begin{phonetics}{椅}{yi3}
    \definition*{s.}{Sobrenome Yi}
    \definition{s.}{cadeira}
  \end{phonetics}
\end{entry}

\begin{entry}{椅子}{12,3}{⽊、⼦}
  \begin{phonetics}{椅子}{yi3zi5}[][HSK 2]
    \definition[把,套,排]{s.}{cadeira; assentos com encosto, feitos principalmente de madeira, bambu, rattan, etc.; móveis com pernas, mas sem encosto para as pessoas se sentarem}
  \end{phonetics}
\end{entry}

\begin{entry}{植}{12}{⽊}
  \begin{phonetics}{植}{zhi2}
    \definition*{s.}{Sobrenome Zhi}
    \definition{s.}{flora; planta; vegetação}
    \definition{v.}{plantar; crescer; cultivar | configurar; estabelecer}
  \end{phonetics}
\end{entry}

\begin{entry}{植物}{12,8}{⽊、⽜}
  \begin{phonetics}{植物}{zhi2wu4}[][HSK 4]
    \definition[种,株,盆,棵]{s.}{planta; vegetação; flora}
  \end{phonetics}
\end{entry}

\begin{entry}{椰}{12}{⽊}
  \begin{phonetics}{椰}{ye1}
    \definition[只,棵]{s.}{coqueiro; coco}
  \end{phonetics}
\end{entry}

\begin{entry}{椰汁}{12,5}{⽊、⽔}
  \begin{phonetics}{椰汁}{ye1zhi1}
    \definition{s.}{água de coco}
  \end{phonetics}
\end{entry}

\begin{entry}{款}{12}{⽋}
  \begin{phonetics}{款}{kuan3}
    \definition{clas.}{para versões ou modelos (de um produto)}
    \definition[笔,个]{s.}{montante de dinheiro | fundos | parágrafo | seção}
  \end{phonetics}
\end{entry}

\begin{entry}{殖}{12}{⽍}
  \begin{phonetics}{殖}{zhi2}
    \definition{v.}{crescer | reproduzir}
  \end{phonetics}
\end{entry}

\begin{entry}{渡}{12}{⽔}
  \begin{phonetics}{渡}{du4}[][HSK 6]
    \definition{s.}{(usualmente em nomes de lugares) travessia de balsa}
    \definition{v.}{atravessar (um rio, o mar, etc.) | superar; sobreviver | transportar (pessoas, mercadorias, etc.) através}
  \end{phonetics}
\end{entry}

\begin{entry}{渡过}{12,6}{⽔、⾡}
  \begin{phonetics}{渡过}{du4guo4}
    \definition{v.}{atravessar | passar por}
  \end{phonetics}
\end{entry}

\begin{entry}{温}{12}{⽔}
  \begin{phonetics}{温}{wen1}
    \definition{adj.}{morno; quente; suave}
    \definition{s.}{temperatura | doenças transmissíveis agudas; praga}
    \definition{v.}{aquecer; reaquecer; aquecer ligeiramente | revisar; repassar}
  \end{phonetics}
\end{entry}

\begin{entry}{温和}{12,8}{⽔、⼝}
  \begin{phonetics}{温和}{wen1he2}[][HSK 5]
    \definition{adj.}{gentil; suave; moderado}
  \end{phonetics}
\end{entry}

\begin{entry}{温度}{12,9}{⽔、⼴}
  \begin{phonetics}{温度}{wen1du4}[][HSK 2]
    \definition[度,级,档,个]{s.}{temperatura}
  \end{phonetics}
\end{entry}

\begin{entry}{温度计}{12,9,4}{⽔、⼴、⾔}
  \begin{phonetics}{温度计}{wen1du4ji4}
    \definition{s.}{termógrafo | termômetro}
  \end{phonetics}
\end{entry}

\begin{entry}{温度表}{12,9,8}{⽔、⼴、⾐}
  \begin{phonetics}{温度表}{wen1du4biao3}
    \definition{s.}{termômetro}
  \end{phonetics}
\end{entry}

\begin{entry}{温度梯度}{12,9,11,9}{⽔、⼴、⽊、⼴}
  \begin{phonetics}{温度梯度}{wen1du4ti1du4}
    \definition{s.}{gradiente de temperatura}
  \end{phonetics}
\end{entry}

\begin{entry}{温柔}{12,9}{⽔、⽊}
  \begin{phonetics}{温柔}{wen1rou2}
    \definition{adj.}{gentil e suave | terno | doce (comumente usado para descrever uma menina ou mulher)}
  \end{phonetics}
\end{entry}

\begin{entry}{温暖}{12,13}{⽔、⽇}
  \begin{phonetics}{温暖}{wen1nuan3}[][HSK 3]
    \definition{adj.}{caloroso; gentil; amigável | caloroso; quente}
    \definition{v.}{aquecer; fazer com que se sinta calor}
  \end{phonetics}
\end{entry}

\begin{entry}{渴}{12}{⽔}
  \begin{phonetics}{渴}{ke3}[][HSK 1]
    \definition{adj.}{sedento}
    \definition{adv.}{ansiosamente; metáfora de urgência}
    \definition{v.}{desejar; ansiar por}
  \end{phonetics}
\end{entry}

\begin{entry}{渴望}{12,11}{⽔、⽉}
  \begin{phonetics}{渴望}{ke3wang4}[][HSK 5]
    \definition{v.}{aspirar; (ter sede, ansiar, desejar) por}
  \end{phonetics}
\end{entry}

\begin{entry}{游}{12}{⽔}
  \begin{phonetics}{游}{you2}[][HSK 3]
    \definition*{s.}{Sobrenome You}
    \definition{adj.}{itinerante; não fixo; que se move frequentemente}
    \definition{s.}{parte de um rio; uma seção do rio; trecho; bacia; curso}
    \definition{v.}{nadar | vagar por aí; caminhar; viajar; fazer turismo | associar com (comunicação) | vagar; passear; andar tranquilamente por todos os lugares}
  \end{phonetics}
\end{entry}

\begin{entry}{游戏}{12,6}{⽔、⼽}
  \begin{phonetics}{游戏}{you2xi4}[][HSK 3]
    \definition[场]{s.}{jogo; recreação; atividades recreativas, como esconde-esconde, adivinhar charadas, etc.; certas atividades esportivas não competitivas; jogos recreativos}
    \definition{v.}{jogar; fazer atividades divertidas e agradáveis, sozinho ou com outras pessoas}
  \end{phonetics}
\end{entry}

\begin{entry}{游泳}{12,8}{⽔、⽔}
  \begin{phonetics}{游泳}{you2yong3}[][HSK 3]
    \definition[次]{s.}{natação; refere-se ao esporte ou atividade de natação}
    \definition{v.+compl.}{nadar; pessoas ou animais nadando na água}
  \end{phonetics}
\end{entry}

\begin{entry}{游泳池}{12,8,6}{⽔、⽔、⽔}
  \begin{phonetics}{游泳池}{you2 yong3 chi2}[][HSK 5]
    \definition[场,个]{s.}{piscina; piscinas artificiais para natação, divididas em duas categorias: internas e externas}
  \seealsoref{泳池}{yong3chi2}
  \seealsoref{游泳馆}{you2yong3guan3}
  \end{phonetics}
\end{entry}

\begin{entry}{游泳衣}{12,8,6}{⽔、⽔、⾐}
  \begin{phonetics}{游泳衣}{you2yong3yi1}
    \definition{s.}{roupa de banho}
  \seealsoref{泳衣}{yong3yi1}
  \end{phonetics}
\end{entry}

\begin{entry}{游泳馆}{12,8,11}{⽔、⽔、⾷}
  \begin{phonetics}{游泳馆}{you2yong3guan3}
    \definition[场]{s.}{piscina}
  \seealsoref{泳池}{yong3chi2}
  \seealsoref{游泳池}{you2 yong3 chi2}
  \end{phonetics}
\end{entry}

\begin{entry}{游泳镜}{12,8,16}{⽔、⽔、⾦}
  \begin{phonetics}{游泳镜}{you2yong3jing4}
    \definition{s.}{óculos de natação}
  \end{phonetics}
\end{entry}

\begin{entry}{游客}{12,9}{⽔、⼧}
  \begin{phonetics}{游客}{you2 ke4}[][HSK 2]
    \definition[个,位,名,群]{s.}{visitante; turista | (jogo online) jogador convidado}
  \end{phonetics}
\end{entry}

\begin{entry}{游艇}{12,12}{⽔、⾈}
  \begin{phonetics}{游艇}{you2ting3}
    \definition[只]{s.}{barcaça | iate}
  \end{phonetics}
\end{entry}

\begin{entry}{湖}{12}{⽔}
  \begin{phonetics}{湖}{hu2}[][HSK 2]
    \definition*{s.}{Huzhou, abreviação de 湖州 | Um nome que se refere às províncias de Hunan, 湖南,  e Hubei, 湖北}
    \definition[个,片]{s.}{lago}
  \seealsoref{湖北}{hu2bei3}
  \seealsoref{湖南}{hu2nan2}
  \seealsoref{湖州}{hu2zhou1}
  \end{phonetics}
\end{entry}

\begin{entry}{湖北}{12,5}{⽔、⼔}
  \begin{phonetics}{湖北}{hu2bei3}
    \definition*{s.}{Província de Hubei (Hupeh), na China central}
  \end{phonetics}
\end{entry}

\begin{entry}{湖州}{12,6}{⽔、⼮}
  \begin{phonetics}{湖州}{hu2zhou1}
    \definition*{s.}{Cidade de Huzhou, em Zhejiang}
  \end{phonetics}
\end{entry}

\begin{entry}{湖南}{12,9}{⽔、⼗}
  \begin{phonetics}{湖南}{hu2nan2}
    \definition*{s.}{Província de Hunan}
  \end{phonetics}
\end{entry}

\begin{entry}{湿}{12}{⽔}
  \begin{phonetics}{湿}{shi1}[][HSK 4]
    \definition{adj.}{molhado; úmido; algo com água ou com muita água dentro}
  \end{phonetics}
\end{entry}

\begin{entry}{滑}{12}{⽔}
  \begin{phonetics}{滑}{hua2}[][HSK 5]
    \definition*{s.}{Sobrenome Hua}
    \definition{adj.}{escorregadio; liso; objetos com superfícies lisas e baixo atrito | astuto; ardiloso; escorregadio}
    \definition{v.}{escorregar; deslizar | se atrapalhar; se safar de algo}
  \end{phonetics}
\end{entry}

\begin{entry}{滑雪}{12,11}{⽔、⾬}
  \begin{phonetics}{滑雪}{hua2xue3}
    \definition{v.+compl.}{esquiar | praticar esqui}
  \end{phonetics}
\end{entry}

\begin{entry}{焚}{12}{⽕}
  \begin{phonetics}{焚}{fen2}
    \definition{v.}{queimar}
  \end{phonetics}
\end{entry}

\begin{entry}{焚香}{12,9}{⽕、⾹}
  \begin{phonetics}{焚香}{fen2xiang1}
    \definition{v.}{queimar incenso}
  \end{phonetics}
\end{entry}

\begin{entry}{焦}{12}{⽕}
  \begin{phonetics}{焦}{jiao1}
    \definition*{s.}{Sobrenome Jiao}
    \definition{adj.}{queimado; chamuscado; carbonizado | preocupado; ansioso}
    \definition{clas.}{J; Joule, abreviação}
    \definition{pref.}{(química) piro-}
    \definition{s.}{Metalurgia: coque}
  \end{phonetics}
\end{entry}

\begin{entry}{焦虑}{12,10}{⽕、⾌}
  \begin{phonetics}{焦虑}{jiao1lv4}
    \definition{adj.}{ansioso | preocupado | apreensivo}
  \end{phonetics}
\end{entry}

\begin{entry}{然}{12}{⽕}
  \begin{phonetics}{然}{ran2}
    \definition{conj.}{mas | no entanto}
  \end{phonetics}
\end{entry}

\begin{entry}{然后}{12,6}{⽕、⼝}
  \begin{phonetics}{然后}{ran2hou4}[][HSK 2]
    \definition{conj.}{então; depois disso; posteriormente; indica que algo segue após uma ação ou situação}
  \end{phonetics}
\end{entry}

\begin{entry}{然而}{12,6}{⽕、⽽}
  \begin{phonetics}{然而}{ran2'er2}[][HSK 4]
    \definition{conj.}{ainda; mas; contudo; todavia; usado no início de uma frase para indicar uma transição; para indicar uma transição, geralmente é precedido por uma conjunção como 虽然 para indicar concessão}
  \seealsoref{虽然}{sui1 ran2}
  \end{phonetics}
\end{entry}

\begin{entry}{煮}{12}{⽕}
  \begin{phonetics}{煮}{zhu3}[][HSK 6]
    \definition*{s.}{Sobrenome Zhu}
    \definition{v.}{ferver; cozinhar; aquecer alimentos ou outros itens em água}
  \end{phonetics}
\end{entry}

\begin{entry}{牌}{12}{⽚}
  \begin{phonetics}{牌}{pai2}[][HSK 4]
    \definition[块]{s.}{placa; tabuleta; quadro; placar | marca; marca registrada; marca comercial | cartas, dominó, etc. | a tonalidade de uma música}
  \end{phonetics}
\end{entry}

\begin{entry}{牌子}{12,3}{⽚、⼦}
  \begin{phonetics}{牌子}{pai2 zi5}[][HSK 3]
    \definition[个,种,块]{s.}{sinal; placa; placas feitas de madeira ou outros materiais, geralmente com texto nelas | marca; marca registrada; um nome especial dado por uma empresa ao seu próprio produto}
  \end{phonetics}
\end{entry}

\begin{entry}{猩}{12}{⽝}
  \begin{phonetics}{猩}{xing1}
    \definition[只]{s.}{orangotango}
  \end{phonetics}
\end{entry}

\begin{entry}{猩猩}{12,12}{⽝、⽝}
  \begin{phonetics}{猩猩}{xing1xing5}
    \definition{s.}{orangotango}
  \end{phonetics}
\end{entry}

\begin{entry}{猴}{12}{⽝}
  \begin{phonetics}{猴}{hou2}[][HSK 5]
    \definition{adj.}{esperto; inteligente; perspicaz | travesso (menino)}
    \definition[只,群]{s.}{macaco}
  \end{phonetics}
\end{entry}

\begin{entry}{猴子}{12,3}{⽝、⼦}
  \begin{phonetics}{猴子}{hou2zi5}
    \definition[只]{s.}{macaco}
  \end{phonetics}
\end{entry}

\begin{entry}{琴}{12}{⽟}
  \begin{phonetics}{琴}{qin2}[][HSK 5]
    \definition*{s.}{Sobrenome Qin}
    \definition[架,台]{s.}{cítara; qin; guqin (um instrumento de cordas dedilhadas com sete cordas, em alguns aspectos semelhante à cítara)  | nome genérico para certos instrumentos musicais}
  \end{phonetics}
\end{entry}

\begin{entry}{琴键}{12,13}{⽟、⾦}
  \begin{phonetics}{琴键}{qin2jian4}
    \definition{s.}{tecla de piano}
  \end{phonetics}
\end{entry}

\begin{entry}{甁}{12}{⽡}
  \begin{phonetics}{甁}{ping2}
    \variantof{瓶}
  \end{phonetics}
\end{entry}

\begin{entry}{番}{12}{⽥}
  \begin{phonetics}{番}{fan1}[][HSK 6]
    \definition{adj.}{estrangeiro; de tribos estrangeiras; estrangeiro ou alienígena}
    \definition{clas.}{usado para o número de vezes que uma ação é executada, equivalente a 回 ou 次 | usado para o tipo de coisas, equivalente a 种}
    \definition{s.}{estrangeiro; de tribos estrangeiras; (velho) refere-se a países estrangeiros ou raças estrangeiras | tomate; batata-doce | aborígenes; nativos; povos indígenas}
    \definition{v.}{revezar; rotacionar; substituir}
  \seealsoref{次}{ci4}
  \seealsoref{回}{hui2}
  \seealsoref{种}{zhong3}
  \end{phonetics}
\end{entry}

\begin{entry}{番茄}{12,8}{⽥、⾋}
  \begin{phonetics}{番茄}{fan1qie2}
    \definition{s.}{tomate}
  \end{phonetics}
\end{entry}

\begin{entry}{疏}{12}{⽦}
  \begin{phonetics}{疏}{shu1}
    \definition*{s.}{Sobrenome Shu}
    \definition{adj.}{fino; esparso; disperso (oposto a 密) | espalhado; disperso; difuso; a distância entre as coisas é grande; as lacunas entre as partes das coisas são grandes | distante; relacionamento distante; não próximo (de relações familiares ou sociais) | não familiarizado com; desconhecido | escasso; vazio}
    \definition{s.}{memorial; memorial ao trono; um texto em que um ministro na era feudal apresentava seus assuntos ao monarca em detalhes | comentário; anotações mais detalhadas de livros antigos do que 注}
    \definition{v.}{dragar (um rio, etc.) | negligenciar | dispersar; espalhar}
  \seealsoref{密}{mi4}
  \seealsoref{注}{zhu4}
  \end{phonetics}
\end{entry}

\begin{entry}{痛}{12}{⽧}
  \begin{phonetics}{痛}{tong4}[][HSK 3]
    \definition{adv.}{extremamente; profundamente; amargamente}
    \definition{s.}{dor; sofrimento | tristeza; pesar}
  \end{phonetics}
\end{entry}

\begin{entry}{痛快}{12,7}{⽧、⼼}
  \begin{phonetics}{痛快}{tong4kuai4}[][HSK 4]
    \definition{adj.}{encantado; alegre; muito feliz; confortável | franco; direto; simples e direto}
  \end{phonetics}
\end{entry}

\begin{entry}{痛苦}{12,8}{⽧、⾋}
  \begin{phonetics}{痛苦}{tong4ku3}[][HSK 3]
    \definition{adj.}{doloroso; angustiado; sentindo-se muito desconfortável física ou mentalmente}
    \definition[降,种]{s.}{dor; agonia; sofrimento; refere-se a um estado ou sentimento de extremo desconforto físico ou mental}
  \end{phonetics}
\end{entry}

\begin{entry}{痛骂}{12,9}{⽧、⾺}
  \begin{phonetics}{痛骂}{tong4ma4}
    \definition{v.}{repreender severamente}
  \end{phonetics}
\end{entry}

\begin{entry}{痠}{12}{⽧}
  \begin{phonetics}{痠}{suan1}
    \definition{v.}{doer | estar dolorido}
    \variantof{酸}
  \end{phonetics}
\end{entry}

\begin{entry}{登}{12}{⽨}
  \begin{phonetics}{登}{deng1}[][HSK 4]
    \definition{v.}{subir; montar; escalar (uma altura) | publicar; registrar; inserir | recolher e levar para a eira | pisar em; pisar | calçar (calçados ou calças) | partir; começar uma jornada; embarcar em uma jornada}
  \end{phonetics}
\end{entry}

\begin{entry}{登山}{12,3}{⽨、⼭}
  \begin{phonetics}{登山}{deng1 shan1}[][HSK 4]
    \definition{s.}{escalar; fazer alpinismo; subir uma montanha}
  \end{phonetics}
\end{entry}

\begin{entry}{登记}{12,5}{⽨、⾔}
  \begin{phonetics}{登记}{deng1ji4}[][HSK 4]
    \definition{v.+compl.}{registrar-se; fazer o \emph{check-in} | registrar; reportar; informar; relatar por escrito a um superior ou autoridade relevante (usado principalmente para documentos legais)}
  \end{phonetics}
\end{entry}

\begin{entry}{登录}{12,8}{⽨、⼹}
  \begin{phonetics}{登录}{deng1lu4}[][HSK 4]
    \definition{v.}{fazer \emph{logon}; fazer \emph{login} | gravar; registrar; computadores eletrônicos e sua terminologia de rede, referindo-se ao acesso ao sistema operacional ou ao site a ser visitado}
  \end{phonetics}
\end{entry}

\begin{entry}{短}{12}{⽮}
  \begin{phonetics}{短}{duan3}[][HSK 2]
    \definition{adj.}{curto; comprimento pequeno de uma extremidade à outra (em oposição a 长) | curto; breve; a distância entre o ponto inicial e o ponto final de um determinado período é pequena | raso; superficial}
    \definition{s.}{falha; defeito; ponto fraco; desvantagens | tonelada curta (EUA)}
    \definition{v.}{dever; carecer}
  \seealsoref{长}{zhang3}
  \end{phonetics}
\end{entry}

\begin{entry}{短少}{12,4}{⽮、⼩}
  \begin{phonetics}{短少}{duan3shao3}
    \definition{v.}{estar aquém do valor total}
  \end{phonetics}
\end{entry}

\begin{entry}{短处}{12,5}{⽮、⼡}
  \begin{phonetics}{短处}{duan3 chu4}[][HSK 3]
    \definition[个]{s.}{deficiência; ponto fraco; defeito; fraqueza}
  \end{phonetics}
\end{entry}

\begin{entry}{短视}{12,8}{⽮、⾒}
  \begin{phonetics}{短视}{duan3shi4}
    \definition{adj.}{míope}
  \end{phonetics}
\end{entry}

\begin{entry}{短促}{12,9}{⽮、⼈}
  \begin{phonetics}{短促}{duan3cu4}
    \definition{adj.}{curto (tom de voz) | fugaz | ofegante (respiração) | curto no tempo}
  \end{phonetics}
\end{entry}

\begin{entry}{短信}{12,9}{⽮、⼈}
  \begin{phonetics}{短信}{duan3xin4}[][HSK 2]
    \definition[条,个,封]{s.}{mensagem de texto; refere-se especificamente a mensagens de texto curtas, imagens, etc., enviadas ou recebidas por celular}
  \end{phonetics}
\end{entry}

\begin{entry}{短缺}{12,10}{⽮、⽸}
  \begin{phonetics}{短缺}{duan3que1}
    \definition{s.}{escassez}
  \end{phonetics}
\end{entry}

\begin{entry}{短暂}{12,12}{⽮、⽇}
  \begin{phonetics}{短暂}{duan3zan4}
    \definition{adj.}{momentâneo | de curta duração}
  \end{phonetics}
\end{entry}

\begin{entry}{短期}{12,12}{⽮、⽉}
  \begin{phonetics}{短期}{duan3 qi1}[][HSK 3]
    \definition{adj.}{de curta duração; de prazo curto}
    \definition[个]{s.}{curto prazo}
  \end{phonetics}
\end{entry}

\begin{entry}{短裤}{12,12}{⽮、⾐}
  \begin{phonetics}{短裤}{duan3 ku4}[][HSK 3]
    \definition[条]{s.}{calças curtas; calção; \emph{shorts}; calças com bainha acima do joelho}
  \end{phonetics}
\end{entry}

\begin{entry}{短跑}{12,12}{⽮、⾜}
  \begin{phonetics}{短跑}{duan3 pao3}
    \definition{s.}{corrida de curta distância; corrida rápida (oposto a 长跑)}
  \seealsoref{长跑}{chang2 pao3}
  \end{phonetics}
\end{entry}

\begin{entry}{硬}{12}{⽯}
  \begin{phonetics}{硬}{ying4}[][HSK 4,5]
    \definition{adj.}{duro; rígido; resistente;  objeto resistente e não se deforma facilmente quando submetido a forças externas (em oposição a 软) | firme; forte; resistente; obstinado; (vontade, atitude, etc.) inabalável, forte e poderoso | capaz (pessoa); boa (qualidade) | rígido; severo; sem flexibilidade | duro; rígido; rigoroso; imutável}
    \definition{adv.}{conseguir fazer algo com dificuldade; indica fazer algo à força, independentemente das circunstâncias}
  \seealsoref{软}{ruan3}
  \end{phonetics}
\end{entry}

\begin{entry}{硬件}{12,6}{⽯、⼈}
  \begin{phonetics}{硬件}{ying4jian4}[][HSK 5]
    \definition{s.}{\emph{hardware}; nome genérico dado aos vários elementos, componentes e dispositivos que constituem um computador | máquina, materiais; equipamento; referência a máquinas, equipamentos, materiais físicos, etc., utilizados nos processos de produção, pesquisa científica, gestão, etc.}
  \end{phonetics}
\end{entry}

\begin{entry}{确}{12}{⽯}
  \begin{phonetics}{确}{que4}
    \definition{adj.}{autenticado | sólido | firme | real | verdadeiro}
  \end{phonetics}
\end{entry}

\begin{entry}{确认}{12,4}{⽯、⾔}
  \begin{phonetics}{确认}{que4ren4}[][HSK 4]
    \definition{v.}{afirmar; confirmar; reconhecer; confirmar explicitamente (fatos, princípios, etc.)}
  \end{phonetics}
\end{entry}

\begin{entry}{确立}{12,5}{⽯、⽴}
  \begin{phonetics}{确立}{que4li4}[][HSK 5]
    \definition{v.}{estabelecer; criar; construir; estabelecer ou consolidar firmemente}
  \end{phonetics}
\end{entry}

\begin{entry}{确定}{12,8}{⽯、⼧}
  \begin{phonetics}{确定}{que4ding4}[][HSK 3]
    \definition{adj.}{definido; certo; claro}
    \definition{v.}{firmar; definir; determinar; tomar uma decisão clara e não mudar}
  \end{phonetics}
\end{entry}

\begin{entry}{确实}{12,8}{⽯、⼧}
  \begin{phonetics}{确实}{que4shi2}[][HSK 3]
    \definition{adj.}{verdadeiro; confiável; autêntico}
    \definition{adv.}{verdadeiramente; realmente; de ​​fato; afirmar a autenticidade de fatos objetivos}
  \end{phonetics}
\end{entry}

\begin{entry}{确保}{12,9}{⽯、⼈}
  \begin{phonetics}{确保}{que4bao3}[][HSK 3]
    \definition{v.}{assegurar; garantir; manter ou garantir com certeza}
  \end{phonetics}
\end{entry}

\begin{entry}{禅}{12}{⽰}
  \begin{phonetics}{禅}{chan2}
    \definition{s.}{contemplação prolongada e intensa; meditação profunda | budista; refere-se geralmente a coisas relacionadas ao budismo}
  \end{phonetics}
  \begin{phonetics}{禅}{shan4}
    \definition{v.}{abdicar e entregar a coroa a outra pessoa}
  \end{phonetics}
\end{entry}

\begin{entry}{禽}{12}{⽱}
  \begin{phonetics}{禽}{qin2}
    \definition*{s.}{Sobrenome Qin}
    \definition[只]{s.}{aves; pássaros | termo genérico para aves e animais}
  \end{phonetics}
\end{entry}

\begin{entry}{程}{12}{⽲}
  \begin{phonetics}{程}{cheng2}
    \definition{s.}{regra; regulamento; lei | ordem; procedimento | jornada; etapa de uma jornada; estrada; um trecho de estrada | distância percorrida ou movida por um objeto | programação | medição; termo geral para pesos e medidas}
  \end{phonetics}
\end{entry}

\begin{entry}{程序}{12,7}{⽲、⼴}
  \begin{phonetics}{程序}{cheng2xu4}[][HSK 4]
    \definition[个,套,种]{s.}{ordem; curso; sequência; procedimento; ordem em que algo é feito; também, um determinado número de etapas em um trabalho | programa; conjunto de instruções de computador projetado em sequência para permitir que um computador execute uma ou mais operações}
  \end{phonetics}
\end{entry}

\begin{entry}{程序设计}{12,7,6,4}{⽲、⼴、⾔、⾔}
  \begin{phonetics}{程序设计}{cheng2xu4she4ji4}
    \definition{s.}{programação de computadores}
  \end{phonetics}
\end{entry}

\begin{entry}{程序库}{12,7,7}{⽲、⼴、⼴}
  \begin{phonetics}{程序库}{cheng2xu4ku4}
    \definition{s.}{biblioteca de funções e procedimentos para programas de computador}
  \end{phonetics}
\end{entry}

\begin{entry}{程度}{12,9}{⽲、⼴}
  \begin{phonetics}{程度}{cheng2du4}[][HSK 3]
    \definition[种]{s.}{nível; grau (de cultura, educação, aprendizagem, etc.) | extensão; grau; a situação, o nível ou o estágio em que as coisas mudam}
  \end{phonetics}
\end{entry}

\begin{entry}{程控}{12,11}{⽲、⼿}
  \begin{phonetics}{程控}{cheng2kong4}
    \definition{s.}{programado | sob controle automático}
  \end{phonetics}
\end{entry}

\begin{entry}{稍}{12}{⽲}
  \begin{phonetics}{稍}{shao1}[][HSK 5]
    \definition{adv.}{ligeiramente; um pouco; um pouquinho}
  \end{phonetics}
\end{entry}

\begin{entry}{稍微}{12,13}{⽲、⼻}
  \begin{phonetics}{稍微}{shao1wei1}[][HSK 5]
    \definition{adv.}{um pouco; um pouquinho; uma ninharia; indica que a quantidade é pequena ou o grau é superficial}
  \end{phonetics}
\end{entry}

\begin{entry}{税}{12}{⽲}
  \begin{phonetics}{税}{shui4}[][HSK 6]
    \definition*{s.}{Sobrenome Shui}
    \definition{s.}{imposto; taxa; tarifa}
  \end{phonetics}
\end{entry}

\begin{entry}{窗}{12}{⽳}
  \begin{phonetics}{窗}{chuang1}
    \definition[扇,个]{s.}{janela}
  \end{phonetics}
\end{entry}

\begin{entry}{窗口}{12,3}{⽳、⼝}
  \begin{phonetics}{窗口}{chuang1 kou3}[][HSK 6]
    \definition[个,号]{s.}{janela; em frente à janela; perto da janela | janela; postigo; refere-se a uma abertura especial em forma de janela | janela; meio; intermediário; peça de exibição; campo de testes; uma metáfora para um lugar com muitas interações com o mundo exterior e através do qual o entendimento mútuo é alcançado |  janela; uma metáfora para um lugar que pode refletir ou exibir a totalidade ou parte de algo |  caixa de diálogo; uma caixa de operação quadrada para aplicativos ou arquivos que aparece na tela do computador}
  \end{phonetics}
\end{entry}

\begin{entry}{窗子}{12,3}{⽳、⼦}
  \begin{phonetics}{窗子}{chuang1 zi5}[][HSK 4]
    \definition{s.}{janela}
  \end{phonetics}
\end{entry}

\begin{entry}{窗户}{12,4}{⽳、⼾}
  \begin{phonetics}{窗户}{chuang1hu5}[][HSK 4]
    \definition[个,扇,面,排]{s.}{janela; dispositivo de ventilação e transmissão de luz nas paredes}
  \end{phonetics}
\end{entry}

\begin{entry}{窗台}{12,5}{⽳、⼝}
  \begin{phonetics}{窗台}{chuang1 tai2}[][HSK 4]
    \definition{s.}{parapeito da janela; peitoril; parte plana de uma janela que segura a moldura}
  \end{phonetics}
\end{entry}

\begin{entry}{窗帘}{12,8}{⽳、⼱}
  \begin{phonetics}{窗帘}{chuang1lian2}[][HSK 5]
    \definition[副,幅,个,套,片,对]{s.}{cortinas para janelas}
  \end{phonetics}
\end{entry}

\begin{entry}{童}{12}{⽴}
  \begin{phonetics}{童}{tong2}
    \definition*{s.}{Sobrenome Tong}
    \definition{adj.}{virgem; solteira | nu; careca | árido; estéril}
    \definition{s.}{criança | jovem servo; antigamente, referia-se a um servo menor de idade.}
  \end{phonetics}
\end{entry}

\begin{entry}{童年}{12,6}{⽴、⼲}
  \begin{phonetics}{童年}{tong2 nian2}[][HSK 4]
    \definition{s.}{infância; primeiros anos de vida}
  \end{phonetics}
\end{entry}

\begin{entry}{童话}{12,8}{⽴、⾔}
  \begin{phonetics}{童话}{tong2hua4}[][HSK 4]
    \definition[个,部]{s.}{conto de fadas; gênero de literatura infantil no qual as histórias adequadas para a diversão das crianças são escritas com muita imaginação, fantasia e exagero}
  \end{phonetics}
\end{entry}

\begin{entry}{等}{12}{⽵}
  \begin{phonetics}{等}{deng3}[][HSK 1,2]
    \definition*{s.}{Sobrenome Deng}
    \definition{adj.}{igual; na mesma medida ou quantidade}
    \definition{clas.}{usado para classe, grau, classificação | usado para tipo}
    \definition{part.}{e assim por diante; etc.; indica que a enumeração não está completa (pode ser usada repetidamente) | indica o fim de uma enumeração; após a enumeração, é usado para encerrar; geralmente é seguido pelo total dos itens anteriores}
    \definition{pron.}{usado após pronomes pessoais ou substantivos que se referem a pessoas; indica plural}
    \definition{s.}{classe; série; posição | equilíbrio; balança para pesar pequenas quantidades de objetos valiosos e ervas medicinais; atualmente, geralmente escrita como 戥}
    \definition{v.}{esperar; aguardar | esperar até}
  \end{phonetics}
\end{entry}

\begin{entry}{等于}{12,3}{⽵、⼆}
  \begin{phonetics}{等于}{deng3yu2}[][HSK 2]
    \definition{adv.}{igual a | equivalente a}
    \definition{v.}{equivaler a; ser equivalente a; ser quase igual a; não ter diferença}
  \end{phonetics}
\end{entry}

\begin{entry}{等级}{12,6}{⽵、⽷}
  \begin{phonetics}{等级}{deng3ji2}[][HSK 5]
    \definition{s.}{grau; classificação; posição; distinções por qualidade, grau, status, etc. | estado social; estrato social; ordem e grau; grupos sociais desiguais em termos de status social e legal}
  \end{phonetics}
\end{entry}

\begin{entry}{等到}{12,8}{⽵、⼑}
  \begin{phonetics}{等到}{deng3 dao4}[][HSK 2]
    \definition{prep.}{na hora; quando; expressão de condições temporais | esperar até; aguardar até}
  \end{phonetics}
\end{entry}

\begin{entry}{等待}{12,9}{⽵、⼻}
  \begin{phonetics}{等待}{deng3dai4}[][HSK 3]
    \definition{v.}{esperar; aguardar; não agir até que a pessoa, coisa ou situação desejada apareça}
  \end{phonetics}
\end{entry}

\begin{entry}{等候}{12,10}{⽵、⼈}
  \begin{phonetics}{等候}{deng3hou4}[][HSK 5]
    \definition{v.}{esperar; aguardar; expectar; usado principalmente para objetos específicos}
  \end{phonetics}
\end{entry}

\begin{entry}{等等}{12,12}{⽵、⽵}
  \begin{phonetics}{等等}{deng3 deng3}
    \definition{part.}{etc.; e assim por diante; usada depois de duas ou mais palavras paralelas para indicar que a lista não está completa}
  \end{phonetics}
\end{entry}

\begin{entry}{筏}{12}{⽵}
  \begin{phonetics}{筏}{fa2}
    \definition[条]{s.}{jangada (de troncos, bambus, etc.)}
  \end{phonetics}
\end{entry}

\begin{entry}{筒}{12}{⽵}
  \begin{phonetics}{筒}{tong3}
    \definition[个]{s.}{seção de bambu grosso; tubo grosso de bambu | objeto em forma de tubo largo | a parte em forma de tubo das roupas etc.}
  \end{phonetics}
\end{entry}

\begin{entry}{答}{12}{⽵}
  \begin{phonetics}{答}{da1}[][HSK 5]
    \definition{v.}{concordar; responder | responder; prestar atenção}
  \end{phonetics}
  \begin{phonetics}{答}{da2}[][HSK 5]
    \definition{v.}{responder; dar resposta a; responder a | retribuir; devolver (uma visita, etc.); retribuir um favor feito a alguém por outro; fazer o bem}
  \end{phonetics}
\end{entry}

\begin{entry}{答应}{12,7}{⽵、⼴}
  \begin{phonetics}{答应}{da1ying5}[][HSK 2]
    \definition{v.}{responder; retribuir; reagir; retrucar | concordar; prometer; cumprir}
  \end{phonetics}
\end{entry}

\begin{entry}{答复}{12,9}{⽵、⼢}
  \begin{phonetics}{答复}{da2fu4}[][HSK 5]
    \definition[个]{s.}{resposta; respostas a perguntas ou solicitações}
    \definition{v.}{responder; dar uma resposta}
  \end{phonetics}
\end{entry}

\begin{entry}{答案}{12,10}{⽵、⽊}
  \begin{phonetics}{答案}{da2'an4}[][HSK 4]
    \definition[个]{s.}{chave; resposta; solução}
  \end{phonetics}
\end{entry}

\begin{entry}{策}{12}{⽵}
  \begin{phonetics}{策}{ce4}
    \definition*{s.}{Sobrenome Ce}
    \definition[个,项,根]{s.}{plano; esquema | tiras de bambu ou madeira usadas para escrever na China antiga | questões sobre atualidades definidas para os exames imperiais | chicote de montaria antigo | um tipo de ensaio na China antiga; um estilo de escrita para exames antigos | estratégia; método}
    \definition{v.}{chicotear (um cavalo) com um chicote de montaria | incitar com um chicote de cavalo, espora}
  \end{phonetics}
\end{entry}

\begin{entry}{策划}{12,6}{⽵、⼑}
  \begin{phonetics}{策划}{ce4hua4}[][HSK 6]
    \definition{v.}{planejar; traçar; esquematizar; pensar repetidamente para elaborar um plano}
  \end{phonetics}
\end{entry}

\begin{entry}{策略}{12,11}{⽵、⽥}
  \begin{phonetics}{策略}{ce4lve4}[][HSK 6]
    \definition{adj.}{diplomático; (métodos) flexíveis sem sacrificar princípios}
    \definition[种,个,条,套]{s.}{tática; estratégia; política; para atingir determinadas tarefas estratégicas, o curso de ação e os métodos de luta são formulados de acordo com o desenvolvimento da situação}
  \end{phonetics}
\end{entry}

\begin{entry}{粤}{12}{⾔}
  \begin{phonetics}{粤}{yue4}
    \definition*{s.}{Outro nome para a Província de Guangdong, 广东}
  \seealsoref{广东}{guang3dong1}
  \end{phonetics}
\end{entry}

\begin{entry}{粤语}{12,9}{⾔、⾔}
  \begin{phonetics}{粤语}{yue4yu3}
    \definition{s.}{cantonês | língua cantonesa}
  \end{phonetics}
\end{entry}

\begin{entry}{粥}{12}{⽶}
  \begin{phonetics}{粥}{yu4}
    \definition{v.}{dar a luz; ter filhos}
  \end{phonetics}
  \begin{phonetics}{粥}{zhou1}[][HSK 6]
    \definition[碗,锅,口]{s.}{mingau; mingau de aveia; alimentos semilíquidos feitos de grãos ou grãos misturados com outras coisas}
  \end{phonetics}
\end{entry}

\begin{entry}{紫}{12}{⽷}
  \begin{phonetics}{紫}{zi3}[][HSK 5]
    \definition*{s.}{Sobrenome Zi}
    \definition{adj.}{roxo; púrpura; violeta; cor resultante da combinação do vermelho e do azul}
  \end{phonetics}
\end{entry}

\begin{entry}{紫色}{12,6}{⽷、⾊}
  \begin{phonetics}{紫色}{zi3 se4}
    \definition{s.}{cor púrpura | cor violeta}
  \end{phonetics}
\end{entry}

\begin{entry}{絫}{12}{⽷}
  \begin{phonetics}{絫}{lei3}
    \variantof{累}
  \end{phonetics}
\end{entry}

\begin{entry}{缓}{12}{⽶}
  \begin{phonetics}{缓}{huan3}
    \definition{adj.}{lento; sem pressa | sem tensão; relaxado}
    \definition{v.}{atrasar; adiar; protelar | recuperar; reviver; voltar a si}
  \end{phonetics}
\end{entry}

\begin{entry}{缓解}{12,13}{⽶、⾓}
  \begin{phonetics}{缓解}{huan3jie3}[][HSK 4]
    \definition{v.}{facilitar; aliviar; atenuar; amenizar; reduzir}
  \end{phonetics}
\end{entry}

\begin{entry}{编}{12}{⽷}
  \begin{phonetics}{编}{bian1}[][HSK 4]
    \definition*{s.}{Sobrenome Bian}
    \definition{s.}{livro; volume; parte de um livro | organização e pessoal; estabelecimento}
    \definition{v.}{tecer; trançar; entrançar | fazer uma lista; organizar em uma lista; organizar; agrupar | editar; compilar | compor; escrever | fabricar; inventar; fazer; preparar}
  \end{phonetics}
\end{entry}

\begin{entry}{编制}{12,8}{⽷、⼑}
  \begin{phonetics}{编制}{bian1 zhi4}[][HSK 6]
    \definition{s.}{estabelecimento; organização e pessoal; refere-se à estrutura organizacional de uma unidade, cotas de pessoal, alocação de tarefas, etc.}
    \definition{v.}{tecer; trançar; entrelaçar tiras de vime, salgueiro, bambu, etc. para fazer objetos | resolver; realizar; elaborar; fazer de acordo com os dados (procedimentos, planos, etc.)}
  \end{phonetics}
\end{entry}

\begin{entry}{编程}{12,12}{⽷、⽲}
  \begin{phonetics}{编程}{bian1cheng2}
    \definition{v.}{programar computador}
  \end{phonetics}
\end{entry}

\begin{entry}{编辑}{12,13}{⽷、⾞}
  \begin{phonetics}{编辑}{bian1ji2}[][HSK 5]
    \definition{v.}{editar; compilar; organizar e processar dados ou trabalhos existentes}
  \end{phonetics}
  \begin{phonetics}{编辑}{bian1ji5}[][HSK 5]
    \definition{s.}{editor; compilador; pessoa que organiza e processa dados ou trabalhos existentes}
  \end{phonetics}
\end{entry}

\begin{entry}{缘}{12}{⽷}
  \begin{phonetics}{缘}{yuan2}
    \definition{s.}{causa | razão | karma | destino | predestinação}
  \end{phonetics}
\end{entry}

\begin{entry}{缘分}{12,4}{⽷、⼑}
  \begin{phonetics}{缘分}{yuan2fen4}
    \definition{s.}{destino ou acaso que une as pessoas | afinidade ou relacionamento predestinado | destino (Budismo)}
  \end{phonetics}
\end{entry}

\begin{entry}{羡}{12}{⽺}
  \begin{phonetics}{羡}{xian4}
    \definition{v.}{admirar; invejar}
  \end{phonetics}
\end{entry}

\begin{entry}{羡慕}{12,14}{⽺、⼼}
  \begin{phonetics}{羡慕}{xian4mu4}
    \definition{v.}{invejar | admirar}
  \end{phonetics}
\end{entry}

\begin{entry}{联}{12}{⽿}
  \begin{phonetics}{联}{lian2}
    \definition{s.}{dísticos (antitéticos)}
    \definition{v.}{aliar-se a; unir-se; juntar-se a}
  \end{phonetics}
\end{entry}

\begin{entry}{联合}{12,6}{⽿、⼝}
  \begin{phonetics}{联合}{lian2he2}[][HSK 3]
    \definition{adj.}{conjunto; unido; federal; combinado}
    \definition{s.}{aliado; união; aliança; conectar-se ou unir-se para agir em conjunto}
  \end{phonetics}
\end{entry}

\begin{entry}{联合会}{12,6,6}{⽿、⼝、⼈}
  \begin{phonetics}{联合会}{lian2he2hui4}
    \definition{s.}{federação}
  \end{phonetics}
\end{entry}

\begin{entry}{联合国}{12,6,8}{⽿、⼝、⼞}
  \begin{phonetics}{联合国}{lian2 he2 guo2}[][HSK 3]
    \definition*{s.}{Nações Unidas; Organização internacional fundada em 1945, após o fim da Segunda Guerra Mundial, com sede em Nova Iorque, Estados Unidos ; as suas principais instituições são a Assembleia Geral, o Conselho de Segurança, o Conselho Econômico e Social e o Secretariado; de acordo com a Carta das Nações Unidas, os seus principais objetivos são manter a paz e a segurança internacionais, desenvolver relações amigáveis entre os países e promover a cooperação internacional nas áreas econômica e cultural}
  \end{phonetics}
\end{entry}

\begin{entry}{联系}{12,7}{⽿、⽷}
  \begin{phonetics}{联系}{lian2xi4}[][HSK 3]
    \definition[个,种,层]{s.}{relacionamento; relacionamento entre duas coisas}
    \definition{v.}{entrar em contato; contatar; comunicar-se com alguém por telefone, e-mail ou carta | agendar; entrar em contato com; estabelecer algum tipo de relação com a outra parte | relacionar; combinar; integrar}
  \end{phonetics}
\end{entry}

\begin{entry}{联络}{12,9}{⽿、⽷}
  \begin{phonetics}{联络}{lian2luo4}[][HSK 5]
    \definition{v.}{entrar em contato; comunicar-se; entrar em contato com}
  \end{phonetics}
\end{entry}

\begin{entry}{联想}{12,13}{⽿、⼼}
  \begin{phonetics}{联想}{lian2xiang3}[][HSK 5]
    \definition*{s.}{Lenovo (empresa)}
    \definition{v.}{associar-se a; estabelecer uma conexão mental; lembrar-se de algo; lembrar-se de outras pessoas ou coisas relacionadas devido a alguém ou algo; evocar outros conceitos relacionados devido a um determinado conceito |}
  \end{phonetics}
\end{entry}

\begin{entry}{脾}{12}{⾁}
  \begin{phonetics}{脾}{pi2}
    \definition{s.}{baço}
  \end{phonetics}
\end{entry}

\begin{entry}{脾气}{12,4}{⾁、⽓}
  \begin{phonetics}{脾气}{pi2qi5}[][HSK 5]
    \definition[发,个]{s.}{temperamento; disposição; referindo-se ao caráter de uma pessoa | mau humor; temperamento irascível}
  \end{phonetics}
\end{entry}

\begin{entry}{舒}{12}{⾆}
  \begin{phonetics}{舒}{shu1}
    \definition*{s.}{Sobrenome Shu}
    \definition{adj.}{lento; vagaroso; sem pressa | confortável; relaxado e feliz}
    \definition{v.}{esticar; desdobrar | alongar; relaxar}
  \end{phonetics}
\end{entry}

\begin{entry}{舒服}{12,8}{⾆、⽉}
  \begin{phonetics}{舒服}{shu1fu5}[][HSK 2]
    \definition{adj.}{confortável; sentir-se relaxado e feliz, tanto física quanto mentalmente}
  \end{phonetics}
\end{entry}

\begin{entry}{舒适}{12,9}{⾆、⾡}
  \begin{phonetics}{舒适}{shu1shi4}[][HSK 4]
    \definition{adj.}{aconchegante; confortável; acolhedor; cômodo}
  \end{phonetics}
\end{entry}

\begin{entry}{落}{12}{⾋}
  \begin{phonetics}{落}{la4}[][HSK 5]
    \definition{v.}{deixar de fora; estar ausente | deixar para trás; esquecer de trazer; deixar algo em algum lugar e esquecer de levar| ficar para trás (ou cair); não conseguir acompanhar}
  \end{phonetics}
  \begin{phonetics}{落}{lao4}
    \definition{v.}{cair; cair de uma altura elevada | se abaixar; descer; ir para baixo | permanecer; fazer uma parada; deixar para trás | obter; ter; receber}
  \end{phonetics}
  \begin{phonetics}{落}{luo4}[][HSK 4]
    \definition*{s.}{Sobrenome Luo}
    \definition{s.}{paradeiro; lugar para ficar; local de descanso | assentamento; local de reunião | parte curta; área pequena; refere-se a um pequeno lugar ou área}
    \definition{v.}{cair; cair de uma altura elevada | se abaixar; descer; ir para baixo | abaixar; deixar cair (ou descer); fazer descer | afundar; declinar; descer | ficar para trás; ficar para trás ou ficar de fora | permanecer; fazer uma parada; deixar para trás | cair sobre; repousar com | obter; ter; receber | anotar; escrever no papel | cair em; entrar em; ficar preso}
  \end{phonetics}
\end{entry}

\begin{entry}{落日}{12,4}{⾋、⽇}
  \begin{phonetics}{落日}{luo4ri4}
    \definition{s.}{pôr do sol}
  \end{phonetics}
\end{entry}

\begin{entry}{落后}{12,6}{⾋、⼝}
  \begin{phonetics}{落后}{luo4hou4}[][HSK 3]
    \definition{adj.}{atrasado; trabalho em atraso, nível de desenvolvimento ou grau de reconhecimento (em oposição a 进步)}
    \definition{v.}{ficar para trás; ficar atrasado; ficar para trás em relação aos outros durante o avanço ou o progresso do trabalho}
  \seealsoref{进步}{jin4bu4}
  \end{phonetics}
\end{entry}

\begin{entry}{落汤鸡}{12,6,7}{⾋、⽔、⿃}
  \begin{phonetics}{落汤鸡}{luo4tang1ji1}
    \definition{s.}{uma pessoa que parece encharcada e acamada| sofrimento profundo}
  \end{phonetics}
\end{entry}

\begin{entry}{落实}{12,8}{⾋、⼧}
  \begin{phonetics}{落实}{luo4shi2}[][HSK 5]
    \definition{adj.}{sentimento de tranquilidade; (humor) estável; seguro}
    \definition{v.}{implementar; ser praticável; tornar os planos, políticas, medidas, etc. específicos e compreensíveis, de modo a que possam ser realizados | implementar; colocar em prática; pôr em prática significa que os planos, políticas e medidas são específicos e claros, e podem ser realizados}
  \end{phonetics}
\end{entry}

\begin{entry}{葡}{12}{⾋}
  \begin{phonetics}{葡}{pu2}
    \definition*{s.}{Portugal, abreviação de 葡萄牙}
  \seealsoref{葡萄牙}{pu2tao2ya2}
  \end{phonetics}
\end{entry}

\begin{entry}{葡文}{12,4}{⾋、⽂}
  \begin{phonetics}{葡文}{pu2wen2}
    \definition{s.}{português, língua portuguesa}
  \seealsoref{葡萄牙文}{pu2tao2ya2wen2}
  \end{phonetics}
\end{entry}

\begin{entry}{葡汉词典}{12,5,7,8}{⾋、⽔、⾔、⼋}
  \begin{phonetics}{葡汉词典}{pu2-han4 ci2dian3}
    \definition{s.}{dicionário português-chinês}
  \seealsoref{汉葡词典}{han4-pu2 ci2dian3}
  \end{phonetics}
\end{entry}

\begin{entry}{葡语}{12,9}{⾋、⾔}
  \begin{phonetics}{葡语}{pu2yu3}
    \definition{s.}{português, língua portuguesa}
  \seealsoref{葡萄牙语}{pu2tao2ya2yu3}
  \end{phonetics}
\end{entry}

\begin{entry}{葡萄}{12,11}{⾋、⾋}
  \begin{phonetics}{葡萄}{pu2tao5}[][HSK 5]
    \definition[棵,串]{s.}{parreira | uva}
  \end{phonetics}
\end{entry}

\begin{entry}{葡萄牙}{12,11,4}{⾋、⾋、⽛}
  \begin{phonetics}{葡萄牙}{pu2tao2ya2}
    \definition{s.}{Portugal}
  \end{phonetics}
\end{entry}

\begin{entry}{葡萄牙文}{12,11,4,4}{⾋、⾋、⽛、⽂}
  \begin{phonetics}{葡萄牙文}{pu2tao2ya2wen2}
    \definition{s.}{português, língua portuguesa}
  \seealsoref{葡文}{pu2wen2}
  \end{phonetics}
\end{entry}

\begin{entry}{葡萄牙语}{12,11,4,9}{⾋、⾋、⽛、⾔}
  \begin{phonetics}{葡萄牙语}{pu2tao2ya2yu3}
    \definition{s.}{português, língua portuguesa}
  \seealsoref{葡语}{pu2yu3}
  \end{phonetics}
\end{entry}

\begin{entry}{葡萄酒}{12,11,10}{⾋、⾋、⾣}
  \begin{phonetics}{葡萄酒}{pu2 tao2 jiu3}[][HSK 5]
    \definition[瓶]{s.}{vinho (de uva)}
  \end{phonetics}
\end{entry}

\begin{entry}{葫}{12}{⾋}
  \begin{phonetics}{葫}{hu2}
    \definition{s.}{cabaça}
  \end{phonetics}
\end{entry}

\begin{entry}{葫芦}{12,7}{⾋、⾋}
  \begin{phonetics}{葫芦}{hu2lu5}
    \definition{adj.}{confuso}
    \definition{s.}{cabaça | termo genérico para bloco e equipamento (ou partes dele)}
  \end{phonetics}
\end{entry}

\begin{entry}{葬}{12}{⾋}
  \begin{phonetics}{葬}{zang4}
    \definition{v.}{enterrar (os mortos) | sepultar}
  \end{phonetics}
\end{entry}

\begin{entry}{葱}{12}{⾋}
  \begin{phonetics}{葱}{cong1}
    \definition{adj.}{verde; turquesa}
    \definition[根,把,捆]{s.}{cebola; cebolinha}
  \end{phonetics}
\end{entry}

\begin{entry}{葵}{12}{⾋}
  \begin{phonetics}{葵}{kui2}
    \definition*{s.}{Sobrenome Kui}
    \definition[朵]{s.}{certas ervas de flores grandes}
  \end{phonetics}
\end{entry}

\begin{entry}{葵花}{12,7}{⾋、⾋}
  \begin{phonetics}{葵花}{kui2hua1}
    \definition{s.}{girassol (flor)}
  \end{phonetics}
\end{entry}

\begin{entry}{街}{12}{⾏}
  \begin{phonetics}{街}{jie1}[][HSK 2]
    \definition[条]{s.}{rua; avenida com prédios dos dois lados | mercado; feira rural}
  \end{phonetics}
\end{entry}

\begin{entry}{街道}{12,12}{⾏、⾡}
  \begin{phonetics}{街道}{jie1dao4}[][HSK 4]
    \definition[条]{s.}{caminho; rua; estrada; via pública com casas em ambos os lados, relativamente larga | escritório do subdistrito; tipo de organização responsável por gerenciar determinados aspectos da rua}
  \end{phonetics}
\end{entry}

\begin{entry}{街舞}{12,14}{⾏、⾇}
  \begin{phonetics}{街舞}{jie1wu3}
    \definition{s.}{dança de rua, \emph{street dance} (por exemplo, \emph{breakdance})}
  \end{phonetics}
\end{entry}

\begin{entry}{裁}{12}{⾐}
  \begin{phonetics}{裁}{cai2}
    \definition{clas.}{divisão de papel de impressão de tamanho padrão}
    \definition{s.}{planejamento | tipo de escrita | planejamento mental; arranjo e seleção, usados principalmente na literatura e na arte | sanção; restrição | estilo; forma | (impressão) tamanho do corte de papel}
    \definition{v.}{cortar (papel, tecido, etc.) em partes | reduzir; cortar; dispensar | julgar; decidir | verificar; sancionar | cortar; eliminar; remover coisas desnecessárias ou redundantes | discernir; medir; julgar}
  \end{phonetics}
\end{entry}

\begin{entry}{裁判}{12,7}{⾐、⼑}
  \begin{phonetics}{裁判}{cai2pan4}[][HSK 5]
    \definition[个,位,名]{s.}{árbitro; juiz; pessoa que desempenha funções de arbitragem em esportes e outras competições}
    \definition{v.}{arbitrar; atuar como árbitro; em esportes e outras atividades competitivas, julgar o desempenho dos atletas, vitórias e derrotas, classificações e problemas que ocorrem durante a competição de acordo com as regras da competição | julgar; refere-se a um terceiro que faz um julgamento quando surge uma disputa entre duas partes}
  \end{phonetics}
\end{entry}

\begin{entry}{裂}{12}{⾐}
  \begin{phonetics}{裂}{lie4}[][HSK 6]
    \definition{s.}{entalhe; incisão; entalhes grandes e profundos nas bordas das folhas ou corolas | brecha; lacuna; rachadura; refere-se à rachadura ou divisão que aparece na superfície ou no interior de um objeto}
    \definition{v.}{dividir; rachar; rasgar | (figurativo) quebrar; esmagar; arruinar}
  \end{phonetics}
\end{entry}

\begin{entry}{装}{12}{⾐}
  \begin{phonetics}{装}{zhuang1}[][HSK 2]
    \definition*{s.}{Sobrenome Zhuang}
    \definition{s.}{vestido; traje; vestimenta; roupa | maquiagem e figurino de palco; maquiagem de ator}
    \definition{v.}{enfeitar; adornar; vestir; decorar; vestir-se; vestir-se bem | fingir; fazer de conta | segurar; embalar; carregar; colocar as coisas em recipientes; colocar as coisas no transporte | encaixar; instalar; equipar; aparelhar; montar | embalar; encaixotar; embrulhar produtos ou colocá-los em caixas, garrafas, etc.}
  \end{phonetics}
\end{entry}

\begin{entry}{装扮}{12,7}{⾐、⼿}
  \begin{phonetics}{装扮}{zhuang1ban4}
    \definition{v.}{enfeitar | decorar | disfarçar-me | vestir-se}
  \end{phonetics}
\end{entry}

\begin{entry}{装备}{12,8}{⾐、⼡}
  \begin{phonetics}{装备}{zhuang1bei4}
    \definition{s.}{equipamento}
    \definition{v.}{equipar}
  \end{phonetics}
\end{entry}

\begin{entry}{装饰}{12,8}{⾐、⾷}
  \begin{phonetics}{装饰}{zhuang1shi4}[][HSK 5]
    \definition[件,个]{s.}{decoração; acessórios decorativos}
    \definition{v.}{enfeitar; adornar; decorar; ornamentar; embelezar; destacar}
  \end{phonetics}
\end{entry}

\begin{entry}{装修}{12,9}{⾐、⼈}
  \begin{phonetics}{装修}{zhuang1 xiu1}[][HSK 4]
    \definition{v.}{equipar; renovar; decorar (equipar uma sala ou prédio com equipamentos ou decorações)}
  \end{phonetics}
\end{entry}

\begin{entry}{装配}{12,10}{⾐、⾣}
  \begin{phonetics}{装配}{zhuang1pei4}
    \definition{v.}{montar | encaixar}
  \end{phonetics}
\end{entry}

\begin{entry}{装置}{12,13}{⾐、⽹}
  \begin{phonetics}{装置}{zhuang1 zhi4}[][HSK 4]
    \definition{s.}{dispositivo; equipamento; máquinas, instrumentos ou outros equipamentos de construção mais complexa e com alguma função independente}
    \definition{v.}{instalar; ajustar; configurar; equipar; montar}
  \end{phonetics}
\end{entry}

\begin{entry}{裙}{12}{⾐}
  \begin{phonetics}{裙}{qun2}
    \definition[条]{s.}{saia | avental | algo como uma saia}
  \end{phonetics}
\end{entry}

\begin{entry}{裙子}{12,3}{⾐、⼦}
  \begin{phonetics}{裙子}{qun2zi5}[][HSK 3]
    \definition[条,件]{s.}{saia (peça de vestuário); uma vestimenta usada abaixo da cintura}
  \end{phonetics}
\end{entry}

\begin{entry}{裤}{12}{⾐}
  \begin{phonetics}{裤}{ku4}
    \definition[条]{s.}{calças}
  \end{phonetics}
\end{entry}

\begin{entry}{裤子}{12,3}{⾐、⼦}
  \begin{phonetics}{裤子}{ku4zi5}[][HSK 3]
    \definition[条]{s.}{calças; calções; roupas usadas abaixo da cintura, com cós, virilha e duas pernas}
  \end{phonetics}
\end{entry}

\begin{entry}{詈}{12}{⾔}
  \begin{phonetics}{詈}{li4}
    \definition{v.}{xingar; usar linguagem severa}
  \end{phonetics}
\end{entry}

\begin{entry}{詈骂}{12,9}{⾔、⾺}
  \begin{phonetics}{詈骂}{li4ma4}
    \definition{v.}{xingar | abusar}
  \end{phonetics}
\end{entry}

\begin{entry}{谢}{12}{⾔}
  \begin{phonetics}{谢}{xie4}
    \definition*{s.}{Sobrenome Xie}
    \definition{v.}{agradecer | desculpar-se; pedir desculpas; admitir a própria culpa | recusar; declinar; renunciar | murchar; perder de flores ou folhas}
  \end{phonetics}
\end{entry}

\begin{entry}{谢天谢地}{12,4,12,6}{⾔、⼤、⾔、⼟}
  \begin{phonetics}{谢天谢地}{xie4tian1xie4di4}
    \definition{expr.}{agradecer a Deus | agradecer aos céus}
  \end{phonetics}
\end{entry}

\begin{entry}{谢世}{12,5}{⾔、⼀}
  \begin{phonetics}{谢世}{xie4shi4}
    \definition{v.}{morrer | falecer}
  \end{phonetics}
\end{entry}

\begin{entry}{谢恩}{12,10}{⾔、⼼}
  \begin{phonetics}{谢恩}{xie4'en1}
    \definition{v.}{agradecer a alguém pelo favor (especialmente imperador ou oficial superior)}
  \end{phonetics}
\end{entry}

\begin{entry}{谢病}{12,10}{⾔、⽧}
  \begin{phonetics}{谢病}{xie4bing4}
    \definition{v.}{desculpar-se por causa de doença}
  \end{phonetics}
\end{entry}

\begin{entry}{谢媒}{12,12}{⾔、⼥}
  \begin{phonetics}{谢媒}{xie4mei2}
    \definition{v.}{agradecer ao casamenteiro}
  \end{phonetics}
\end{entry}

\begin{entry}{谢谢}{12,12}{⾔、⾔}
  \begin{phonetics}{谢谢}{xie4xie5}[][HSK 1]
    \definition{interj.}{Obrigado!}
    \definition{v.}{agradecer; agradecer a gentileza dos outros}
  \end{phonetics}
\end{entry}

\begin{entry}{谢意}{12,13}{⾔、⼼}
  \begin{phonetics}{谢意}{xie4yi4}
    \definition{s.}{gratidão}
  \end{phonetics}
\end{entry}

\begin{entry}{貂}{12}{⾘}
  \begin{phonetics}{貂}{diao1}
    \definition*{s.}{Sobrenome Diao}
    \definition[只]{s.}{marta; fuinha; arminho}
  \end{phonetics}
\end{entry}

\begin{entry}{赌}{12}{⾙}
  \begin{phonetics}{赌}{du3}[][HSK 6]
    \definition{v.}{jogar | apostar; geralmente se refere à luta pela vitória ou derrota}
  \end{phonetics}
\end{entry}

\begin{entry}{赏}{12}{⾙}
  \begin{phonetics}{赏}{shang3}[][HSK 4]
    \definition*{s.}{Sobrenome Shang}
    \definition{s.}{recompensa; prêmio}
    \definition{v.}{conceder (outorgar) uma recompensa; recompensar; premiar | admirar; desfrutar; apreciar; valorizar}
  \end{phonetics}
\end{entry}

\begin{entry}{赏心悦目}{12,4,10,5}{⾙、⼼、⼼、⽬}
  \begin{phonetics}{赏心悦目}{shang3xin1yue4mu4}
    \definition{expr.}{``Aquece o coração e encanta os olhos.''}
  \end{phonetics}
\end{entry}

\begin{entry}{赏赐}{12,12}{⾙、⾙}
  \begin{phonetics}{赏赐}{shang3ci4}
    \definition{s.}{recompensa | prêmio}
    \definition{v.}{recompensar | premiar}
  \end{phonetics}
\end{entry}

\begin{entry}{赔}{12}{⾙}
  \begin{phonetics}{赔}{pei2}[][HSK 5]
    \definition{v.}{compensar; pagar por; indenizar | sofrer uma perda; fazer negócios e perder dinheiro}
  \end{phonetics}
\end{entry}

\begin{entry}{赔钱}{12,10}{⾙、⾦}
  \begin{phonetics}{赔钱}{pei2qian2}
    \definition{v.+compl.}{perder dinheiro | compensar; compensar com dinheiro os prejuízos causados a terceiros}
  \end{phonetics}
\end{entry}

\begin{entry}{赔偿}{12,11}{⾙、⼈}
  \begin{phonetics}{赔偿}{pei2chang2}[][HSK 5]
    \definition{v.}{indenizar; compensar; pagar por; indenizar outras pessoas ou grupos por perdas causadas por suas próprias ações}
  \end{phonetics}
\end{entry}

\begin{entry}{趁}{12}{⾛}
  \begin{phonetics}{趁}{chen4}
    \definition{prep.}{aproveitar-se de; tirar vantagem de (tempo, oportunidade, etc.); indica o tempo e as condições de uso}
    \definition{v.}{ser rico em; possuir}
  \end{phonetics}
\end{entry}

\begin{entry}{超}{12}{⾛}
  \begin{phonetics}{超}{chao1}[][HSK 6]
    \definition{adj.}{super; extremamente; maior (ou menor) que o padrão geral}
    \definition{v.}{exceder; ultrapassar; vir para a frente por trás; prevalecer | transcender; ir além; não ser sujeito a certas restrições; ir além de um certo intervalo | exceder; superar; exceder o limite prescrito}
  \end{phonetics}
\end{entry}

\begin{entry}{超出}{12,5}{⾛、⼐}
  \begin{phonetics}{超出}{chao1 chu1}[][HSK 6]
    \definition{v.}{exceder; ultrapassar; ir além (de uma certa quantidade ou intervalo)}
  \end{phonetics}
\end{entry}

\begin{entry}{超市}{12,5}{⾛、⼱}
  \begin{phonetics}{超市}{chao1shi4}[][HSK 2]
    \definition[家]{s.}{supermercado; abreviação de 超级市场}
  \seealsoref{超级市场}{chao1 ji2 shi4 chang3}
  \end{phonetics}
\end{entry}

\begin{entry}{超级}{12,6}{⾛、⽷}
  \begin{phonetics}{超级}{chao1ji2}[][HSK 3]
    \definition{adj.}{super; além do nível geral}
    \definition{pref.}{super-; ultra-; hiper-}
  \end{phonetics}
\end{entry}

\begin{entry}{超级市场}{12,6,5,6}{⾛、⽷、⼱、⼟}
  \begin{phonetics}{超级市场}{chao1 ji2 shi4 chang3}
    \definition[个,间,所,家]{s.}{supermercado; hipermercado}
  \end{phonetics}
\end{entry}

\begin{entry}{超过}{12,6}{⾛、⾡}
  \begin{phonetics}{超过}{chao1guo4}[][HSK 2]
    \definition{v.}{ultrapassar; superar (algo ou alguém); passar de trás para a frente de alguém ou algo | exceder; ser mais do que; ultrapassar (um padrão)}
  \end{phonetics}
\end{entry}

\begin{entry}{超声}{12,7}{⾛、⼠}
  \begin{phonetics}{超声}{chao1sheng1}
    \definition{adj.}{ultrasônico}
    \definition{s.}{ultrasom}
  \end{phonetics}
\end{entry}

\begin{entry}{超越}{12,12}{⾛、⾛}
  \begin{phonetics}{超越}{chao1yue4}[][HSK 5]
    \definition{v.}{ultrapassar; superar; passar por cima; transcender}
  \end{phonetics}
\end{entry}

\begin{entry}{越}{12}{⾛}
  \begin{phonetics}{越}{yue4}[][HSK 2]
    \definition{adj.}{superior; excede ou ultrapassa o ordinário}
    \definition{adv.}{quanto mais\dots mais; sados juntos, eles formam o formato de "越……越……" para indicar que o grau de uma situação se torna mais sério à medida que se desenvolve; "成年……" para indicar que o grau de uma situação se torna mais sério à medida que o tempo passa}
    \definition{v.}{passar por cima; pular; cruzar | exceder; ultrapassar | estar em um tom alto; estar animado | saquear; pilhar; expoliar; apreender; roubar | passar; passar através; atravessar}
  \seealsoref{越来越……}{yue4 lai2 yue4}
  \seealsoref{越……越……}{yue4 yue4}
  \end{phonetics}
\end{entry}

\begin{entry}{越来越……}{12,7,12}{⾛、⽊、⾛}
  \begin{phonetics}{越来越……}{yue4 lai2 yue4}[][HSK 2]
    \definition{adv.}{cada vez mais\dots; isso significa que o grau de algo se aprofunda à medida que o tempo passa}
  \end{phonetics}
\end{entry}

\begin{entry}{越……越……}{12,12}{⾛、⾛}
  \begin{phonetics}{越……越……}{yue4 yue4}[][HSK 2]
    \definition{expr.}{quanto mais\dots tanto mais\dots}
  \end{phonetics}
\end{entry}

\begin{entry}{越障}{12,13}{⾛、⾩}
  \begin{phonetics}{越障}{yue4zhang4}
    \definition{s.}{curso com obstáculos para treinamento de tropas}
    \definition{v.}{superar obstáculos}
  \end{phonetics}
\end{entry}

\begin{entry}{越境}{12,14}{⾛、⼟}
  \begin{phonetics}{越境}{yue4jing4}
    \definition{v.}{cruzar uma fronteira (geralmente ilegalmente) | entrar ou sair furtivamente de um país}
  \end{phonetics}
\end{entry}

\begin{entry}{趋}{12}{⾛}
  \begin{phonetics}{趋}{qu1}
    \definition{v.}{apressar-se | tender para; tender a se tornar | (de um ganso, cobra, etc.) estalar a cabeça e morder as pessoas}
  \end{phonetics}
\end{entry}

\begin{entry}{趋势}{12,8}{⾛、⼒}
  \begin{phonetics}{趋势}{qu1shi4}[][HSK 4]
    \definition{s.}{tendência; tendência; direção; impulso das coisas que se movem em uma direção ou outra}
  \end{phonetics}
\end{entry}

\begin{entry}{跌}{12}{⾜}
  \begin{phonetics}{跌}{die1}[][HSK 6]
    \definition{s.}{(de um objeto, etc.) queda; tombo | (de preços, etc.) queda}
    \definition{v.}{cair; tombar; perder o equilíbrio e cair | cair (objetos caindo); descer | cair (queda de preços)}
  \end{phonetics}
\end{entry}

\begin{entry}{跑}{12}{⾜}
  \begin{phonetics}{跑}{pao2}
    \definition{v.}{(de animais) bater com a pata (no chão); (de animais) escavar o solo com suas garras ou cascos}
  \end{phonetics}
  \begin{phonetics}{跑}{pao3}[][HSK 1]
    \definition{v.}{correr; pessoas ou animais que se movem rapidamente para a frente com as pernas e os pés | caminhar; passear | fugir; escapar | correr de um lado para outro; fazer rondas; correr atrás de algo | de um líquido ou gás) vazar; evaporar | (como complemento de um verbo) fora; longe | participar de uma corrida}
  \end{phonetics}
\end{entry}

\begin{entry}{跑马}{12,3}{⾜、⾺}
  \begin{phonetics}{跑马}{pao3ma3}
    \definition{s.}{corrida de cavalos}
    \definition{v.}{andar a cavalo em ritmo acelerado}
  \end{phonetics}
\end{entry}

\begin{entry}{跑步}{12,7}{⾜、⽌}
  \begin{phonetics}{跑步}{pao3bu4}[][HSK 3]
    \definition{v.+compl.}{correr; trotar}
  \end{phonetics}
\end{entry}

\begin{entry}{跑肚}{12,7}{⾜、⾁}
  \begin{phonetics}{跑肚}{pao3du4}
    \definition{v.}{(coloquial) ter diarréia}
  \end{phonetics}
\end{entry}

\begin{entry}{跑调}{12,10}{⾜、⾔}
  \begin{phonetics}{跑调}{pao3diao4}
    \definition{v.}{(coloquial) estar fora do tom ou desafinado (enquanto canta)}
  \end{phonetics}
\end{entry}

\begin{entry}{跑掉}{12,11}{⾜、⼿}
  \begin{phonetics}{跑掉}{pao3diao4}
    \definition{v.}{fugir}
  \end{phonetics}
\end{entry}

\begin{entry}{跑腿}{12,13}{⾜、⾁}
  \begin{phonetics}{跑腿}{pao3tui3}
    \definition{v.}{realizar tarefas}
  \end{phonetics}
\end{entry}

\begin{entry}{跑酷}{12,14}{⾜、⾣}
  \begin{phonetics}{跑酷}{pao3ku4}
    \definition*{s.}{Eempréstimo linguístico: Parkour}
  \end{phonetics}
\end{entry}

\begin{entry}{跑题}{12,15}{⾜、⾴}
  \begin{phonetics}{跑题}{pao3ti2}
    \definition{v.}{divagar | fugir do assunto | tergiversar}
  \end{phonetics}
\end{entry}

\begin{entry}{辈}{12}{⾞}
  \begin{phonetics}{辈}{bei4}[][HSK 5]
    \definition*{s.}{Sobrenome Bei}
    \definition{s.}{pessoas de um certo tipo; semelhantes | geração; geração na família | duração da vida | círculo familiar}
  \end{phonetics}
\end{entry}

\begin{entry}{逼}{12}{⾡}
  \begin{phonetics}{逼}{bi1}[][HSK 6]
    \definition{adj.}{estreito}
    \definition{v.}{forçar; pressionar; compelir | extorquir; pressionar por | fechar em; pressionar em direção a; aproximar-se}
  \end{phonetics}
\end{entry}

\begin{entry}{遇}{12}{⾡}
  \begin{phonetics}{遇}{yu4}[][HSK 4]
    \definition*{s.}{Sobrenome Yu}
    \definition{s.}{chance; oportunidade}
    \definition{v.}{encontrar; deparar-se com; encontrar-se | tratar; receber}
  \end{phonetics}
\end{entry}

\begin{entry}{遇见}{12,4}{⾡、⾒}
  \begin{phonetics}{遇见}{yu4 jian4}[][HSK 4]
    \definition{v.}{encontrar; deparar-se com}
  \end{phonetics}
\end{entry}

\begin{entry}{遇到}{12,8}{⾡、⼑}
  \begin{phonetics}{遇到}{yu4dao4}[][HSK 4]
    \definition{v.}{esbarrar em; encontrar; deparar-se com; conhecer alguém ou algo (inesperado)}
  \end{phonetics}
\end{entry}

\begin{entry}{遍}{12}{⾡}
  \begin{phonetics}{遍}{bian4}[][HSK 2]
    \definition{adv.}{por toda parte; em toda parte; em todos os lugares}
    \definition{clas.}{usado para a repetição de ações de leitura, fala ou escrita}
  \end{phonetics}
\end{entry}

\begin{entry}{遍地}{12,6}{⾡、⼟}
  \begin{phonetics}{遍地}{bian4 di4}[][HSK 6]
    \definition{adv.}{em todos os lugares; em toda parte; por toda parte}
  \end{phonetics}
\end{entry}

\begin{entry}{道}{12}{⾡}
  \begin{phonetics}{道}{dao4}[][HSK 2]
    \definition*{s.}{Taoismo;  Taoista | Sobrenome Dao}
    \definition{clas.}{usado para pratos em refeições, etapas em um procedimento, etc. | usado para certos objetos longos e estreitos; tira | usado para portas, paredes, etc.; pesado | usado para comandos, títulos, etc.}
    \definition[条]{s.}{estrada; caminho; trilha | curso; canal; o caminho percorrido pelo fluxo da água | maneira; método; princípio; raciocínio | moral; moralidade | habilidade; técnica | doutrina; princípio; sistema de pensamento acadêmico ou religioso; origem de todas as coisas no universo | taoísta; taoísmo; pertencente ao taoísmo | seita supersticiosa; certas organizações reacionárias e supersticiosas | linha; traços finos e alongados | trato; os canais dentro do corpo}
    \definition{v.}{dizer; falar; expressar-se | pensar; supor; considerar; acreditar que}
  \end{phonetics}
\end{entry}

\begin{entry}{道行}{12,6}{⾡、⾏}
  \begin{phonetics}{道行}{dao4 heng2}
    \definition{s.}{realizações de um monge budista ou sacerdote taoísta | habilidades; capacidades; aptidões | (figurativo) habilidade | habilidades adquiridas através da prática religiosa}
  \end{phonetics}
\end{entry}

\begin{entry}{道理}{12,11}{⾡、⽟}
  \begin{phonetics}{道理}{dao4li5}[][HSK 2]
    \definition[个,种]{s.}{verdade; princípio; a lei das coisas | sentido; razão}
  \end{phonetics}
\end{entry}

\begin{entry}{道路}{12,13}{⾡、⾜}
  \begin{phonetics}{道路}{dao4 lu4}[][HSK 2]
    \definition[条,段]{s.}{estrada; caminho; os canais de comunicação entre os dois lugares, incluindo terrestres e aquáticos | caminho; processo; refere-se à vida, à existência (significado abstrato)}
  \end{phonetics}
\end{entry}

\begin{entry}{道歉}{12,14}{⾡、⽋}
  \begin{phonetics}{道歉}{dao4qian4}
    \definition{v.+compl.}{desculpar-se | fazer um pedido de desculpas}
  \end{phonetics}
\end{entry}

\begin{entry}{道德}{12,15}{⾡、⼻}
  \begin{phonetics}{道德}{dao4de2}[][HSK 5]
    \definition{adj.}{moral; descreve uma pessoa ou comportamento que atende aos requisitos morais; mais usado em situações negativas}
    \definition{s.}{moral; ética; moralidade; regras e normas para que as pessoas vivam juntas e se comportem em comum}
  \end{phonetics}
\end{entry}

\begin{entry}{遗}{12}{⾡}
  \begin{phonetics}{遗}{yi2}
    \definition*{s.}{Sobrenome Yi}
    \definition{s.}{descarga involuntária de urina, etc. | algo perdido}
    \definition{v.}{perder | omitir | deixar para trás; guardar; não dar | deixar para trás após a morte; legar; transmitir}
  \end{phonetics}
\end{entry}

\begin{entry}{遗产}{12,6}{⾡、⼇}
  \begin{phonetics}{遗产}{yi2chan3}[][HSK 4]
    \definition[笔,份]{s.}{legado; herança; patrimônio; propriedade deixada pelo falecido | patrimônio; riqueza cultural ou riqueza material transmitida pela história}
  \end{phonetics}
\end{entry}

\begin{entry}{遗传}{12,6}{⾡、⼈}
  \begin{phonetics}{遗传}{yi2chuan2}[][HSK 4]
    \definition{v.}{herdar, descender, transmitir, passar adiante}
  \end{phonetics}
\end{entry}

\begin{entry}{遗男}{12,7}{⾡、⽥}
  \begin{phonetics}{遗男}{yi2nan2}
    \definition{s.}{órfão | filho póstumo}
  \end{phonetics}
\end{entry}

\begin{entry}{遗迹}{12,9}{⾡、⾡}
  \begin{phonetics}{遗迹}{yi2ji4}
    \definition{s.}{vestígios históricos | remanescente | vestígio}
  \end{phonetics}
\end{entry}

\begin{entry}{遗案}{12,10}{⾡、⽊}
  \begin{phonetics}{遗案}{yi2'an4}
    \definition{s.}{(lei) caso não resolvido}
  \end{phonetics}
\end{entry}

\begin{entry}{遗落}{12,12}{⾡、⾋}
  \begin{phonetics}{遗落}{yi2luo4}
    \definition{v.}{esquecer | deixar para trás (inadvertidamente) | deixar de fora | omitir}
  \end{phonetics}
\end{entry}

\begin{entry}{遗嘱}{12,15}{⾡、⼝}
  \begin{phonetics}{遗嘱}{yi2zhu3}
    \definition{s.}{testamento}
  \end{phonetics}
\end{entry}

\begin{entry}{遗骸}{12,15}{⾡、⾻}
  \begin{phonetics}{遗骸}{yi2hai2}
    \definition{v.}{restos mortais}
  \end{phonetics}
\end{entry}

\begin{entry}{遗憾}{12,16}{⾡、⼼}
  \begin{phonetics}{遗憾}{yi2han4}
    \definition{v.}{ter pena de | lamentar}
  \end{phonetics}
\end{entry}

\begin{entry}{酢}{12}{⾣}
  \begin{phonetics}{酢}{cu4}
    \definition{s.}{vinagre | (figurativo) ciúme (como em um caso de amor)}
    \variantof{醋}
  \end{phonetics}
  \begin{phonetics}{酢}{zuo4}
    \definition{s.}{brinde ao anfitrião feito pelo convidado}
  \end{phonetics}
\end{entry}

\begin{entry}{量}{12}{⾥}
  \begin{phonetics}{量}{liang2}[][HSK 4]
    \definition{v.}{medir | estimar; dimensionar}
  \end{phonetics}
  \begin{phonetics}{量}{liang4}
    \definition{s.}{instrumento de medida; antigamente, o termo se referia a objetos como baldes e litros, que medem o volume | capacidade (para tolerância ou ingestão de alimentos ou bebidas); refere-se ao limite do que pode ser acomodado | quantidade; valor; volume; número}
    \definition{v.}{estimar; medir; pesar}
  \end{phonetics}
\end{entry}

\begin{entry}{铺}{12}{⾦}
  \begin{phonetics}{铺}{pu1}[][HSK 6]
    \definition{clas.}{usado para kang, etc.; kang, uma plataforma de alvenaria ou de barro em uma extremidade de um cômodo, aquecida no inverno por fogueiras embaixo e coberta com esteiras para dormir}
    \definition{v.}{espalhar; estender; desdobrar | colocar; pavimentar}
  \end{phonetics}
  \begin{phonetics}{铺}{pu4}
    \definition{s.}{pequena loja; depósito | uma cama feita de tábuas de madeira; geralmente se refere a uma cama | estação de correios; antiga estação de correios (usada principalmente em nomes de lugares)}
  \end{phonetics}
\end{entry}

\begin{entry}{铺垫}{12,9}{⾦、⼟}
  \begin{phonetics}{铺垫}{pu1dian4}
    \definition{s.}{cobre leito | colcha | roupa de cama}
    \definition{v.}{pavimentar}
  \end{phonetics}
\end{entry}

\begin{entry}{销}{12}{⾦}
  \begin{phonetics}{销}{xiao1}
    \definition*{s.}{Sobrenome Xiao}
    \definition{s.}{gasto; despesa | pino}
    \definition{v.}{derreter (metal) | cancelar; anular | vender; comercializar | aferrolhar; fixar; prender; pregar | fixar com um parafuso; parafusar | gastar (consumo) | inserir um pino}
  \end{phonetics}
\end{entry}

\begin{entry}{销售}{12,11}{⾦、⼝}
  \begin{phonetics}{销售}{xiao1shou4}[][HSK 4]
    \definition{v.}{vender; comercializar}
  \end{phonetics}
\end{entry}

\begin{entry}{锁}{12}{⾦}
  \begin{phonetics}{锁}{suo3}[][HSK 5]
    \definition[把]{s.}{fechadura; dispositivo que impede que as pessoas abram facilmente a parte que se abre e fecha | correntes; cadeado e correntes | qualquer coisa com a forma de um cadeado antigo}
    \definition{v.}{trancar; trancar com chave | costurar com ponto fixo | tricotar}
  \end{phonetics}
\end{entry}

\begin{entry}{锅}{12}{⾦}
  \begin{phonetics}{锅}{guo1}[][HSK 5]
    \definition[口,个,只]{s.}{panela; frigideira; utensílios de cozinha, redondos e côncavos, feitos principalmente de ferro, alumínio, etc. | parte que se parece com um pote em alguns objetos}
  \end{phonetics}
\end{entry}

\begin{entry}{锐}{12}{⾦}
  \begin{phonetics}{锐}{rui4}
    \definition*{s.}{Sobrenome Rui}
    \definition{adj.}{afiado; aguçado (oposto a 钝) | agudo; perspicaz | rápido; ágil; veloz}
    \definition{adv.}{rapidamente; de ​​repente}
    \definition{s.}{vigor; espírito de luta | armas afiadas}
  \seealsoref{钝}{dun4}
  \end{phonetics}
\end{entry}

\begin{entry}{阔}{12}{⾨}
  \begin{phonetics}{阔}{kuo4}[][HSK 6]
    \definition{adj.}{amplo; amplo; vasto | rico | longo, no sentido de ``há muito tempo'' | vazio; impraticável}
  \end{phonetics}
\end{entry}

\begin{entry}{隔}{12}{⾩}
  \begin{phonetics}{隔}{ge2}[][HSK 4]
    \definition{adj.}{seguinte; vizinho}
    \definition{v.}{separar; cortar; dividir; particionar | estar a uma distância de, após ou em um intervalo de | ficar de pé ou deitar entre}
  \end{phonetics}
\end{entry}

\begin{entry}{隔开}{12,4}{⾩、⼶}
  \begin{phonetics}{隔开}{ge2 kai1}[][HSK 4]
    \definition{v.}{separar; manter separado; barricar; separar completamente duas pessoas (ou coisas) ou duas partes de uma coisa que estão intimamente unidas}
  \end{phonetics}
\end{entry}

\begin{entry}{隔壁}{12,16}{⾩、⼟}
  \begin{phonetics}{隔壁}{ge2bi4}[][HSK 5]
    \definition{s.}{vizinho; casas ou pessoas vizinhas | septo; distante (socialmente distante) | anteparo; partição}
  \end{phonetics}
\end{entry}

\begin{entry}{雄}{12}{⾫}
  \begin{phonetics}{雄}{xiong2}
    \definition*{s.}{Sobrenome Xiong}
    \definition{adj.}{masculino | grandioso; imponente; audacioso | poderoso}
    \definition{s.}{uma pessoa ou país com grande poder e influência}
  \end{phonetics}
\end{entry}

\begin{entry}{雄伟}{12,6}{⾫、⼈}
  \begin{phonetics}{雄伟}{xiong2wei3}[][HSK 5]
    \definition{adj.}{magnífico; magnificente | imponente; magnífico}
  \end{phonetics}
\end{entry}

\begin{entry}{集}{12}{⾫}
  \begin{phonetics}{集}{ji2}[][HSK 6]
    \definition*{s.}{Sobrenome Ji}
    \definition{clas.}{parte; volume}
    \definition[个,本]{s.}{mercado; feira rural | coleção; conjunto; antologia | (matemática) conjunto}
    \definition{v.}{reunir; coletar; montar}
  \end{phonetics}
\end{entry}

\begin{entry}{集中}{12,4}{⾫、⼁}
  \begin{phonetics}{集中}{ji2zhong1}[][HSK 3]
    \definition{adj.}{centralizado; concentrado}
    \definition{v.}{concentrar; centralizar; focar; acumular; reunir (oposto de 分散) | reunir pessoas, coisas, forças, etc. dispersas; resumir opiniões, experiências, etc.}
  \seealsoref{分散}{fen1san4}
  \end{phonetics}
\end{entry}

\begin{entry}{集合}{12,6}{⾫、⼝}
  \begin{phonetics}{集合}{ji2he2}[][HSK 4]
    \definition{v.}{reunir-se; juntar-se | reunir, juntar, convocar}
  \end{phonetics}
\end{entry}

\begin{entry}{集团}{12,6}{⾫、⼞}
  \begin{phonetics}{集团}{ji2tuan2}[][HSK 5]
    \definition[个]{s.}{anel; bloco; grupo; panelinha; círculo; grupo organizado para agir em conjunto com um determinado objetivo | grupo; entidade econômica com uma direção de negócios especializada, liderada por uma grande empresa com forte poder econômico e alta visibilidade, e formada pela combinação ou fusão de empresas relacionadas}
  \end{phonetics}
\end{entry}

\begin{entry}{集体}{12,7}{⾫、⼈}
  \begin{phonetics}{集体}{ji2ti3}[][HSK 3]
    \definition{s.}{coletivo; comunidade; grupo; equipe; organizações ou grupos em que muitas pessoas trabalham, estudam e vivem juntas}
  \end{phonetics}
\end{entry}

\begin{entry}{韩}{12}{⾱}
  \begin{phonetics}{韩}{han2}
    \definition*{s.}{Um estado durante o Período dos Estados Combatentes nas atuais províncias centrais de Henan e sudeste de Shanxi | O nome de um estado feudal durante a dinastia Zhou, localizado no que hoje é o nordeste de Hejin, província de Shanxi | Coreia do Sul, abreviação de 韩国; República da Coreia (RC) | Sobrenome Han}
  \seealsoref{韩国}{han2guo2}
  \end{phonetics}
\end{entry}

\begin{entry}{韩国}{12,8}{⾱、⼞}
  \begin{phonetics}{韩国}{han2guo2}
    \definition*{s.}{Coréia do Sul; República da Coreia}
  \end{phonetics}
\end{entry}

\begin{entry}{韩国人}{12,8,2}{⾱、⼞、⼈}
  \begin{phonetics}{韩国人}{han2guo2ren2}
    \definition{s.}{coreano | pessoa ou povo da Coréia}
  \end{phonetics}
\end{entry}

\begin{entry}{骗}{12}{⾺}
  \begin{phonetics}{骗}{pian4}[][HSK 5]
    \definition{v.}{enganar; trapacear; iludir; ludibriar; usar mentiras ou meios fraudulentos para fazer alguém acreditar ou ser enganado | enganar; fraudar | montar (um cavalo); balançar (ou saltar) para a sela}
  \end{phonetics}
\end{entry}

\begin{entry}{骗子}{12,3}{⾺、⼦}
  \begin{phonetics}{骗子}{pian4 zi5}[][HSK 5]
    \definition[个]{s.}{trapaceiro; vigarista; fraudador; impostor; golpista; pessoa que obtém bens de forma fraudulenta}
  \end{phonetics}
\end{entry}

\begin{entry}{骚}{12}{⾺}
  \begin{phonetics}{骚}{sao1}
    \definition*{s.}{Abreviação de Li Sao (Encontrando a Tristeza), um poema do poeta e estadista do século IV a.C. Qu Yuan (屈原)}
    \definition{adj.}{coquete; (de uma mulher) lasciva | masculino (de alguns animais domésticos)}
    \definition{s.}{escritos literários; geralmente se refere à poesia | o cheiro de urina; mau cheiro}
    \definition{v.}{perturbar}
  \seealsoref{屈原}{qu1yuan2}
  \end{phonetics}
\end{entry}

\begin{entry}{骚乱}{12,7}{⾺、⼄}
  \begin{phonetics}{骚乱}{sao1luan4}
    \definition{s.}{rebelião | perturbação | tumulto}
    \definition{v.}{criar um distúrbio}
  \end{phonetics}
\end{entry}

\begin{entry}{黍}{12}{⿉}[Kangxi 202]
  \begin{phonetics}{黍}{shu3}
    \definition{s.}{painço}
  \end{phonetics}
\end{entry}

\begin{entry}{黑}{12}{⿊}[Kangxi 203]
  \begin{phonetics}{黑}{hei1}[][HSK 2]
    \definition*{s.}{Província de Heilongjiang, abreviação de 黑龙江 | Sobrenome Hei}
    \definition{adj.}{preto; cor semelhante à do carvão | escuro | obscuro; secreto | perverso; sinistro; ruim; cruel | reacionário}
    \definition{s.}{noite}
    \definition{v.}{fazer algo ilegalmente ou de forma desonesta; enganar; desviar dinheiro ilegalmente | invadir (uma rede, sites, computador, etc.)}
  \seealsoref{黑龙江}{hei1long2jiang1}
  \end{phonetics}
\end{entry}

\begin{entry}{黑龙江}{12,5,6}{⿊、⿓、⽔}
  \begin{phonetics}{黑龙江}{hei1long2jiang1}
    \definition*{s.}{Província de Heilongjiang | Rio Heilong Jiang;  Rio Amur (na Rússia)}
  \end{phonetics}
\end{entry}

\begin{entry}{黑色}{12,6}{⿊、⾊}
  \begin{phonetics}{黑色}{hei1 se4}[][HSK 2]
    \definition{adj.}{metafórico: suspeito, ilegal}
    \definition{s.}{cor preta}
  \end{phonetics}
\end{entry}

\begin{entry}{黑板}{12,8}{⿊、⽊}
  \begin{phonetics}{黑板}{hei1ban3}[][HSK 2]
    \definition[块,个]{s.}{quadro negro; quadro de giz; uma placa, na qual se pode escrever com giz}
  \end{phonetics}
\end{entry}

\begin{entry}{黑客}{12,9}{⿊、⼧}
  \begin{phonetics}{黑客}{hei1ke4}
    \definition{s.}{(empréstimo linguístico) (computação) \emph{hacker}}
  \end{phonetics}
\end{entry}

\begin{entry}{黑暗}{12,13}{⿊、⽇}
  \begin{phonetics}{黑暗}{hei1 an4}[][HSK 4]
    \definition{adj.}{escuro; sombrio; sem luz | maligno; corrupto; reacionário}
  \end{phonetics}
\end{entry}

\begin{entry}{黹}{12}{⿋}[Kangxi 204]
  \begin{phonetics}{黹}{zhi3}
    \definition{v.}{costurar; bordar}
  \end{phonetics}
\end{entry}

\begin{entry}{鼎}{12}{⿍}[Kangxi 206]
  \begin{phonetics}{鼎}{ding3}
    \definition{adj.}{grande; generoso | importante; grandioso}
    \definition{adv.}{exatamente quando; exatamente o momento para}
    \definition[尊]{s.}{um antigo recipiente de cozinha com duas alças e três ou quatro pernas | pote; caldeirão | poder do estado; o trono | como símbolo de dinastia; nos tempos antigos, era considerada uma ferramenta importante para estabelecer um país}
  \end{phonetics}
\end{entry}

%%%%% EOF %%%%%


%%%
%%% 13画
%%%

\section*{13画}\addcontentsline{toc}{section}{13画}

\begin{entry}{傻瓜}{13,5}[Radicais ⼈、⽠]
  \begin{phonetics}{傻瓜}{sha3gua1}
    \definition{adj.}{tolo | burro | simplório | idiota}
    \definition{v.}{enganar | iludir | lograr}
  \end{phonetics}
\end{entry}

\begin{entry}{傻眼}{13,11}[Radicais ⼈、⽬]
  \begin{phonetics}{傻眼}{sha3yan3}
    \definition{adj.}{estupefato | atordoado}
  \end{phonetics}
\end{entry}

\begin{entry}{像}{13}[Radical ⼈]
  \begin{phonetics}{像}{xiang4}[][HSK 2]
    \definition{s.}{imagem | retrato | aparência}
    \definition{v.}{assemelhar-se | ser como}
  \end{phonetics}
\end{entry}

\begin{entry}{嗄}{13}[Radical ⼝]
  \begin{phonetics}{嗄}{a2}
    \definition{adj.}{rouco}
    \variantof{啊}
  \end{phonetics}
\end{entry}

\begin{entry}{嗡嗡}{13,13}[Radicais ⼝、⼝]
  \begin{phonetics}{嗡嗡}{weng1weng1}
    \definition{s.}{zumbido}
    \definition{v.}{zumbir}
  \end{phonetics}
\end{entry}

\begin{entry}{嘟}{13}[Radical ⼝]
  \begin{phonetics}{嘟}{du1}
    \definition{s.}{buzina | bip}
    \definition{v.}{fazer beicinho}
  \end{phonetics}
\end{entry}

\begin{entry}{嫉妒}{13,7}[Radicais ⼥、⼥]
  \begin{phonetics}{嫉妒}{ji2du4}
    \definition{v.}{estar com ciúmes de | invejar}
  \end{phonetics}
\end{entry}

\begin{entry}{幕}{13}[Radical ⼱]
  \begin{phonetics}{幕}{mu4}
    \definition{s.}{cortina ou tela | dossel ou tenda | quartel de um general | ato (de uma peça)}
  \end{phonetics}
\end{entry}

\begin{entry}{彀}{13}[Radical ⼸]
  \begin{phonetics}{彀}{gou4}
    \definition{s.}{calcance de um arco e flecha}
    \definition{v.}{puxar um arco ao máximo}
  \end{phonetics}
\end{entry}

\begin{entry}{微风}{13,4}[Radicais ⼻、⾵]
  \begin{phonetics}{微风}{wei1feng1}
    \definition{s.}{brisa | vento leve}
  \end{phonetics}
\end{entry}

\begin{entry}{微软}{13,8}[Radicais ⼻、⾞]
  \begin{phonetics}{微软}{wei1ruan3}
    \definition*{s.}{\emph{Microsoft Corporation}}
  \end{phonetics}
\end{entry}

\begin{entry}{微型}{13,9}[Radicais ⼻、⼟]
  \begin{phonetics}{微型}{wei1xing2}
    \definition{pref.}{``micro''}
    \definition{s.}{miniatura}
  \end{phonetics}
\end{entry}

\begin{entry}{想}{13}[Radical ⼼]
  \begin{phonetics}{想}{xiang3}[][HSK 1]
    \definition{v.}{acreditar | sentir falta (sentir-se melancólico com a ausência de alguém ou algo) | supor | pensar | querer | desejar}
  \end{phonetics}
\end{entry}

\begin{entry}{想到}{13,8}[Radicais ⼼、⼑]
  \begin{phonetics}{想到}{xiang3 dao4}[][HSK 2]
    \definition{v.}{pensar em | trazer à mente | ter no coração}
  \end{phonetics}
\end{entry}

\begin{entry}{想念}{13,8}[Radicais ⼼、⼼]
  \begin{phonetics}{想念}{xiang3nian4}
    \definition{v.}{perder | sentir falta | lembrar com saudade}
  \end{phonetics}
\end{entry}

\begin{entry}{想法}{13,8}[Radicais ⼼、⽔]
  \begin{phonetics}{想法}{xiang3 fa3}[][HSK 2]
    \definition[个]{s.}{noção | opinião | jeito de pensar}
    \definition{s.}{maneira de pensar | opinião | noção}
    \definition{v.}{pensar em uma maneira (de fazer algo)}
  \end{phonetics}
\end{entry}

\begin{entry}{想起}{13,10}[Radicais ⼼、⾛]
  \begin{phonetics}{想起}{xiang3 qi3}[][HSK 2]
    \definition{v.}{recordar | lembrar | pensar em | trazer à mente | cruzar pelos pensamentos de alguém | passar pelo pensamento de alguém}
  \end{phonetics}
\end{entry}

\begin{entry}{想象}{13,11}[Radicais ⼼、⾗]
  \begin{phonetics}{想象}{xiang3xiang4}
    \definition{v.}{imaginar}
  \end{phonetics}
\end{entry}

\begin{entry}{想想看}{13,13,9}[Radicais ⼼、⼼、⽬]
  \begin{phonetics}{想想看}{xiang3xiang3kan4}
    \definition{v.}{pensar sobre isso}
  \end{phonetics}
\end{entry}

\begin{entry}{愈}{13}[Radical ⼼]
  \begin{phonetics}{愈}{yu4}
    \definition{adv.}{mais e mais | ainda mais}
    \definition{v.}{recuperar | curar}
  \end{phonetics}
\end{entry}

\begin{entry}{意义}{13,3}[Radicais ⼼、⼂]
  \begin{phonetics}{意义}{yi4yi4}[][HSK 3]
    \definition[个]{s.}{sentido; significado; significado expresso por palavras ou outros sinais; significado indicado por comportamento ou aquisição |valor; efeito; significância; impacto}
  \end{phonetics}
\end{entry}

\begin{entry}{意见}{13,4}[Radicais ⼼、⾒]
  \begin{phonetics}{意见}{yi4jian4}[][HSK 2]
    \definition[点,条]{s.}{reclamação | ideia | objeção | opinião | sugestão}
  \end{phonetics}
\end{entry}

\begin{entry}{意外}{13,5}[Radicais ⼼、⼣]
  \begin{phonetics}{意外}{yi4wai4}[][HSK 3]
    \definition{adj.}{inesperado; imprevisto}
    \definition{adv.}{acidentalmente}
    \definition[个]{s.}{acidente; infortúnio; infortúnio inesperado}
  \end{phonetics}
\end{entry}

\begin{entry}{意志}{13,7}[Radicais ⼼、⼼]
  \begin{phonetics}{意志}{yi4zhi4}
    \definition[个]{s.}{determinação | desejo | força de vontade}
  \end{phonetics}
\end{entry}

\begin{entry}{意识}{13,7}[Radicais ⼼、⾔]
  \begin{phonetics}{意识}{yi4shi2}
    \definition{s.}{consciência}
    \definition{v.}{(usualmente seguido de 到) estar ciente, constatar}
  \end{phonetics}
\end{entry}

\begin{entry}{意译}{13,7}[Radicais ⼼、⾔]
  \begin{phonetics}{意译}{yi4yi4}
    \definition{s.}{tradução livre | significado (de expressão estrangeira) | paráfrase | tradução do significado (em oposição à tradução literal)}
  \seealsoref{直译}{zhi2yi4}
  \end{phonetics}
\end{entry}

\begin{entry}{意思}{13,9}[Radicais ⼼、⼼]
  \begin{phonetics}{意思}{yi4si5}[][HSK 2]
    \definition[个]{s.}{interesse}
  \end{phonetics}
\end{entry}

\begin{entry}{意指}{13,9}[Radicais ⼼、⼿]
  \begin{phonetics}{意指}{yi4zhi3}
    \definition{v.}{implicar | significar}
  \end{phonetics}
\end{entry}

\begin{entry}{感动}{13,6}[Radicais ⼼、⼒]
  \begin{phonetics}{感动}{gan3dong4}[][HSK 2]
    \definition{v.}{mover (alguém) | tocar (alguém emocionalmente)}
  \end{phonetics}
\end{entry}

\begin{entry}{感到}{13,8}[Radicais ⼼、⼑]
  \begin{phonetics}{感到}{gan3 dao4}[][HSK 2]
    \definition{v.}{sentir | perceber}
  \end{phonetics}
\end{entry}

\begin{entry}{感受}{13,8}[Radicais ⼼、⼜]
  \begin{phonetics}{感受}{gan3shou4}[][HSK 3]
    \definition{s.}{percepção ; sentimento; experiência}
    \definition{v.}{sentir; sentir (através dos sentidos); experimentar}
  \end{phonetics}
\end{entry}

\begin{entry}{感冒}{13,9}[Radicais ⼼、⽇]
  \begin{phonetics}{感冒}{gan3mao4}[][HSK 3]
    \definition{adj.}{interessado}
    \definition[场,次]{s.}{resfriado; resfriado comum; gripe}
    \definition{v.}{pegar (ter) um resfriado}
  \end{phonetics}
\end{entry}

\begin{entry}{感染}{13,9}[Radicais ⼼、⽊]
  \begin{phonetics}{感染}{gan3ran3}
    \definition{s.}{infecção}
    \definition{v.}{infectar | (figurativo) influenciar}
  \end{phonetics}
\end{entry}

\begin{entry}{感觉}{13,9}[Radicais ⼼、⾒]
  \begin{phonetics}{感觉}{gan3jue2}[][HSK 2]
    \definition{s.}{sentimento | impressão | sensação}
    \definition{v.}{sentir | perceber}
  \end{phonetics}
\end{entry}

\begin{entry}{感情}{13,11}[Radicais ⼼、⼼]
  \begin{phonetics}{感情}{gan3qing2}[][HSK 3]
    \definition[份,个,种]{s.}{emoção; sentimento | amor; afeição; apego}
  \end{phonetics}
\end{entry}

\begin{entry}{感谢}{13,12}[Radicais ⼼、⾔]
  \begin{phonetics}{感谢}{gan3xie4}[][HSK 2]
    \definition{s.}{gratidão | agradecimento}
  \end{phonetics}
\end{entry}

\begin{entry}{搞}{13}[Radical ⼿]
  \begin{phonetics}{搞}{gao3}
    \definition{v.}{fazer}
  \end{phonetics}
\end{entry}

\begin{entry}{搞好}{13,6}[Radicais ⼿、⼥]
  \begin{phonetics}{搞好}{gao3hao3}
    \definition{v.}{fazer um ótimo trabalho}
  \end{phonetics}
\end{entry}

\begin{entry}{搞乱}{13,7}[Radicais ⼿、⼄]
  \begin{phonetics}{搞乱}{gao3luan4}
    \definition{v.}{estragar | confundir | bagunçar}
  \end{phonetics}
\end{entry}

\begin{entry}{搞定}{13,8}[Radicais ⼿、⼧]
  \begin{phonetics}{搞定}{gao3ding4}
    \definition{v.}{consertar | resolver}
  \end{phonetics}
\end{entry}

\begin{entry}{搞鬼}{13,9}[Radicais ⼿、⿁]
  \begin{phonetics}{搞鬼}{gao3gui3}
    \definition{v.}{fazer travessuras | fazer truques}
  \end{phonetics}
\end{entry}

\begin{entry}{搞笑}{13,10}[Radicais ⼿、⽵]
  \begin{phonetics}{搞笑}{gao3xiao4}
    \definition{adj.}{engraçado | hilário}
    \definition{v.}{fazer as pessoas rirem}
  \end{phonetics}
\end{entry}

\begin{entry}{搞通}{13,10}[Radicais ⼿、⾡]
  \begin{phonetics}{搞通}{gao3tong1}
    \definition{v.}{entender algo}
  \end{phonetics}
\end{entry}

\begin{entry}{搞钱}{13,10}[Radicais ⼿、⾦]
  \begin{phonetics}{搞钱}{gao3qian2}
    \definition{v.}{fazer dinheiro | acumular dinheiro}
  \end{phonetics}
\end{entry}

\begin{entry}{搞混}{13,11}[Radicais ⼿、⽔]
  \begin{phonetics}{搞混}{gao3hun4}
    \definition{v.}{confundir}
  \end{phonetics}
\end{entry}

\begin{entry}{搞错}{13,13}[Radicais ⼿、⾦]
  \begin{phonetics}{搞错}{gao3cuo4}
    \definition{v.}{cometer um erro}
  \end{phonetics}
\end{entry}

\begin{entry}{搬}{13}[Radical ⼿]
  \begin{phonetics}{搬}{ban1}[][HSK 3]
    \definition{v.}{copiar indiscriminadamente | mover-se (ou seja, mudar-se) | mover-se (algo relativamente pesado ou volumoso) | mudar | mudar-se}
  \end{phonetics}
\end{entry}

\begin{entry}{搬口}{13,3}[Radicais ⼿、⼝]
  \begin{phonetics}{搬口}{ban1kou3}
    \definition{v.}{tagarelar | (idioma) transmitir histórias | semear dissensão | contar histórias}
  \end{phonetics}
\end{entry}

\begin{entry}{搬动}{13,6}[Radicais ⼿、⼒]
  \begin{phonetics}{搬动}{ban1dong4}
    \definition{v.}{mover-se (alguma coisa) | se mudar}
  \end{phonetics}
\end{entry}

\begin{entry}{搬弄}{13,7}[Radicais ⼿、⼶]
  \begin{phonetics}{搬弄}{ban1nong4}
    \definition{v.}{causar problemas | mexer com alguém | mostrar (o que se pode fazer)}
  \end{phonetics}
\end{entry}

\begin{entry}{搬走}{13,7}[Radicais ⼿、⾛]
  \begin{phonetics}{搬走}{ban1zou3}
    \definition{v.}{carregar}
  \end{phonetics}
\end{entry}

\begin{entry}{搬运}{13,7}[Radicais ⼿、⾡]
  \begin{phonetics}{搬运}{ban1yun4}
    \definition{s.}{frete | transporte}
    \definition{v.}{carregar | transportar}
  \end{phonetics}
\end{entry}

\begin{entry}{搬家}{13,10}[Radicais ⼿、⼧]
  \begin{phonetics}{搬家}{ban1jia1}[][HSK 3]
    \definition{s.}{mudança}
    \definition{v.+compl.}{mudar-se de casa}
  \end{phonetics}
\end{entry}

\begin{entry}{摄氏}{13,4}[Radicais ⼿、⽒]
  \begin{phonetics}{摄氏}{she4shi4}
    \definition{s.}{graus Celsius (°C), centígrado}
  \end{phonetics}
\end{entry}

\begin{entry}{摆手}{13,4}[Radicais ⼿、⼿]
  \begin{phonetics}{摆手}{bai3shou3}
    \definition{v.+compl.}{gesticular com a mão (acenando, acenando adeus, etc.) | balançar os braços | acenar com as mãos}
  \end{phonetics}
\end{entry}

\begin{entry}{摆烂}{13,9}[Radicais ⼿、⽕]
  \begin{phonetics}{摆烂}{bai3lan4}
    \definition{v.}{(neologismo, gíria) parar de lutar (especialmente quando se sabe que não pode ter sucesso) | deixar tudo ir para o inferno}
  \end{phonetics}
\end{entry}

\begin{entry}{摇头}{13,5}[Radicais ⼿、⼤]
  \begin{phonetics}{摇头}{yao2tou2}
    \definition{v.+compl.}{balançar a cabeça de alguém}
  \end{phonetics}
\end{entry}

\begin{entry}{摇晃}{13,10}[Radicais ⼿、⽇]
  \begin{phonetics}{摇晃}{yao2huang4}
    \definition{v.}{sacudir | agitar | balançar | chacoalhar}
  \end{phonetics}
\end{entry}

\begin{entry}{数}{13}[Radical ⽁]
  \begin{phonetics}{数}{shu3}[][HSK 2]
    \definition{v.}{contar
ser considerado excepcionalmente (bom, ruim, etc.)
enumerar; listar}
  \end{phonetics}
  \begin{phonetics}{数}{shu4}
    \definition{num.}{vários | alguns}
    \definition{s.}{número | figura | destino}
  \end{phonetics}
  \begin{phonetics}{数}{shuo4}
    \definition{adv.}{frequentemente | repetidamente}
  \end{phonetics}
\end{entry}

\begin{entry}{数字}{13,6}[Radicais ⽁、⼦]
  \begin{phonetics}{数字}{shu4zi4}[][HSK 2]
    \definition{adj.}{digital}
    \definition[个]{s.}{dígito | figura | número | numeral | quantidade | montante}
  \end{phonetics}
\end{entry}

\begin{entry}{数学}{13,8}[Radicais ⽁、⼦]
  \begin{phonetics}{数学}{shu4xue2}
    \definition{s.}{matemática (disciplina)}
  \end{phonetics}
\end{entry}

\begin{entry}{数量}{13,12}[Radicais ⽁、⾥]
  \begin{phonetics}{数量}{shu4liang4}[][HSK 3]
    \definition[个]{s.}{quantidade; quantum; quantia; magnitude; número}
  \end{phonetics}
\end{entry}

\begin{entry}{新}{13}[Radical ⽄]
  \begin{phonetics}{新}{xin1}[][HSK 1]
    \definition*{s.}{sobrenome Xin | abreviação de Xinjiang (新疆) | abreviação de Singapura (新加坡)}
    \definition{adj.}{novo}
    \definition{adv.}{recentemente}
    \definition{pref.}{(química) ``meso''}
  \seealsoref{新加坡}{xin1jia1po1}
  \seealsoref{新疆}{xin1jiang1}
  \end{phonetics}
\end{entry}

\begin{entry}{新加坡}{13,5,8}[Radicais ⽄、⼒、⼟]
  \begin{phonetics}{新加坡}{xin1jia1po1}
    \definition*{s.}{Singapura}
  \end{phonetics}
\end{entry}

\begin{entry}{新年}{13,6}[Radicais ⽄、⼲]
  \begin{phonetics}{新年}{xin1nian2}[][HSK 1]
    \definition*[个]{s.}{Ano Novo}
  \end{phonetics}
\end{entry}

\begin{entry}{新闻}{13,9}[Radicais ⽄、⾨]
  \begin{phonetics}{新闻}{xin1wen2}[][HSK 2]
    \definition[条,个]{s.}{notícia}
  \end{phonetics}
\end{entry}

\begin{entry}{新娘}{13,10}[Radicais ⽄、⼥]
  \begin{phonetics}{新娘}{xin1niang2}
    \definition{s.}{noiva}
  \seealsoref{新娘子}{xin1niang2zi5}
  \end{phonetics}
\end{entry}

\begin{entry}{新娘子}{13,10,3}[Radicais ⽄、⼥、⼦]
  \begin{phonetics}{新娘子}{xin1niang2zi5}
    \definition{s.}{noiva}
  \seealsoref{新娘}{xin1niang2}
  \end{phonetics}
\end{entry}

\begin{entry}{新娘服装}{13,10,8,12}[Radicais ⽄、⼥、⽉、⾐]
  \begin{phonetics}{新娘服装}{xin1niang2 fu2zhuang1}
    \definition{s.}{roupas de noiva}
  \end{phonetics}
\end{entry}

\begin{entry}{新鲜}{13,14}[Radicais ⽄、⿂]
  \begin{phonetics}{新鲜}{xin1xian1}
    \definition{adj.}{fresco (experiência, alimento, etc.)}
    \definition{s.}{frescor}
  \end{phonetics}
\end{entry}

\begin{entry}{新疆}{13,19}[Radicais ⽄、⼸]
  \begin{phonetics}{新疆}{xin1jiang1}
    \definition*{s.}{Xinjiang}
  \end{phonetics}
\end{entry}

\begin{entry}{新疆维吾尔自治区}{13,19,11,7,5,6,8,4}[Radicais ⽄、⼸、⽷、⼝、⼩、⾃、⽔、⼖]
  \begin{phonetics}{新疆维吾尔自治区}{xin1jiang1 wei2wu2'er3 zi4zhi4qu1}
    \definition*{s.}{Região Autônoma Uigur de Xinjiang}
  \end{phonetics}
\end{entry}

\begin{entry}{暖}{13}[Radical ⽇]
  \begin{phonetics}{暖}{nuan3}
    \definition{adj.}{quente}
    \definition{v.}{esquentar}
  \end{phonetics}
\end{entry}

\begin{entry}{暖气}{13,4}[Radicais ⽇、⽓]
  \begin{phonetics}{暖气}{nuan3qi4}
    \definition{s.}{aquecimento central | aquecedor | ar quente}
  \end{phonetics}
\end{entry}

\begin{entry}{暖和}{13,8}[Radicais ⽇、⼝]
  \begin{phonetics}{暖和}{nuan3huo5}[][HSK 3]
    \definition{adj.}{morno; agradável e quente}
    \definition{v.}{aquecer}
  \end{phonetics}
\end{entry}

\begin{entry}{暗香}{13,9}[Radicais ⽇、⾹]
  \begin{phonetics}{暗香}{an4xiang1}
    \definition{s.}{fragrância sutil}
  \end{phonetics}
\end{entry}

\begin{entry}{暗恋}{13,10}[Radicais ⽇、⼼]
  \begin{phonetics}{暗恋}{an4lian4}
    \definition{s.}{amor secreto}
    \definition{v.}{estar secretamente apaixonado por}
  \end{phonetics}
\end{entry}

\begin{entry}{楼}{13}[Radical ⽊]
  \begin{phonetics}{楼}{lou2}[][HSK 1]
    \definition*{s.}{sobrenome Lou}
    \definition{clas.}{andar, piso}
    \definition[层,座,栋]{s.}{edifício | prédio | sobrado | casa com 2 ou mais andares}
  \end{phonetics}
\end{entry}

\begin{entry}{楼上}{13,3}[Radicais ⽊、⼀]
  \begin{phonetics}{楼上}{lou2 shang4}[][HSK 1]
    \definition{adv.}{no andar de cima | (gíria da Internet) post anterior em um fio de um fórum}
  \end{phonetics}
\end{entry}

\begin{entry}{楼下}{13,3}[Radicais ⽊、⼀]
  \begin{phonetics}{楼下}{lou2 xia4}[][HSK 1]
    \definition{adv.}{no andar de baixo}
  \end{phonetics}
\end{entry}

\begin{entry}{楼梯}{13,11}[Radicais ⽊、⽊]
  \begin{phonetics}{楼梯}{lou2ti1}
    \definition[个]{s.}{escada | escadaria}
  \end{phonetics}
\end{entry}

\begin{entry}{概念}{13,8}[Radicais ⽊、⼼]
  \begin{phonetics}{概念}{gai4nian4}[][HSK 3]
    \definition[个]{s.}{ideia; noção; conceito; concepção}
  \end{phonetics}
\end{entry}

\begin{entry}{滔天}{13,4}[Radicais ⽔、⼤]
  \begin{phonetics}{滔天}{tao1tian1}
    \definition{adj.}{(ondas, raiva, desastres, crimes, etc.) imponente, avassalador, imenso}
  \end{phonetics}
\end{entry}

\begin{entry}{滚轮}{13,8}[Radicais ⽔、⾞]
  \begin{phonetics}{滚轮}{gun3lun2}
    \definition{s.}{pneu | dial rotativo | roda de rolagem (\emph{scroll})  (mouse de computador)}
  \end{phonetics}
\end{entry}

\begin{entry}{滚滚}{13,13}[Radicais ⽔、⽔]
  \begin{phonetics}{滚滚}{gun3gun3}
    \definition*{s.}{Apelido para um panda}
    \definition{v.}{mover-se | rolar | fluir continuamente}
  \end{phonetics}
\end{entry}

\begin{entry}{满}{13}[Radical ⽔]
  \begin{phonetics}{满}{man3}[][HSK 2]
    \definition{adj.}{completo | preenchido | embalado | satisfeito | contente}
    \definition{adv.}{completamente | bastante}
    \definition{v.}{preencher | atingir o limite | satisfazer}
  \end{phonetics}
\end{entry}

\begin{entry}{满分}{13,4}[Radicais ⽔、⼑]
  \begin{phonetics}{满分}{man3fen1}
    \definition{s.}{pontuação completa}
  \end{phonetics}
\end{entry}

\begin{entry}{满足}{13,7}[Radicais ⽔、⾜]
  \begin{phonetics}{满足}{man3zu2}[][HSK 3]
    \definition{v.}{estar satisfeito; contentar-se | satisfazer; causar satisfação; contentar}
  \end{phonetics}
\end{entry}

\begin{entry}{满意}{13,13}[Radicais ⽔、⼼]
  \begin{phonetics}{满意}{man3yi4}[][HSK 2]
    \definition{adj.}{satisfatório}
  \end{phonetics}
\end{entry}

\begin{entry}{满满}{13,13}[Radicais ⽔、⽔]
  \begin{phonetics}{满满}{man3man3}
    \definition{adj.}{completo | densamente empacotado}
  \end{phonetics}
\end{entry}

\begin{entry}{煎}{13}[Radical ⽕]
  \begin{phonetics}{煎}{jian1}
    \definition{v.}{fritar | refogar}
  \end{phonetics}
\end{entry}

\begin{entry}{煎饼}{13,9}[Radicais ⽕、⾷]
  \begin{phonetics}{煎饼}{jian1bing3}
    \definition[张]{s.}{jianbing, crepe chinês | panqueca}
  \end{phonetics}
\end{entry}

\begin{entry}{煎蛋}{13,11}[Radicais ⽕、⾍]
  \begin{phonetics}{煎蛋}{jian1dan4}
    \definition{s.}{ovos fritos}
  \end{phonetics}
\end{entry}

\begin{entry}{照}{13}[Radical ⽕]
  \begin{phonetics}{照}{zhao4}[][HSK 3]
    \definition{adv.}{de acordo com; indica agir de acordo com o original ou um certo padrão}
    \definition{prep.}{em direção a; na direção de | de acordo com; conforme}
    \definition{s.}{imagem; fotografia | permissão; licença | brilho; iluminação}
    \definition{v.}{brilhar; acender; iluminar | refletir; espelhar; olhar para sua própria imagem em um espelho, etc. | filmar; fotografar; tirar uma foto (fotografia) | cuidar de; tomar conta de | notificar | contrastar | entender}
  \end{phonetics}
\end{entry}

\begin{entry}{照片}{13,4}[Radicais ⽕、⽚]
  \begin{phonetics}{照片}{zhao4pian4}[][HSK 2]
    \definition[张,套,幅]{s.}{fotografia | foto}
  \end{phonetics}
\end{entry}

\begin{entry}{照片子}{13,4,3}[Radicais ⽕、⽚、⼦]
  \begin{phonetics}{照片子}{zhao4pian4zi5}
    \definition{v.}{tirar um raio X}
  \end{phonetics}
\end{entry}

\begin{entry}{照片底版}{13,4,8,8}[Radicais ⽕、⽚、⼴、⽚]
  \begin{phonetics}{照片底版}{zhao4pian4di3ban3}
    \definition{s.}{placa fotográfica}
  \end{phonetics}
\end{entry}

\begin{entry}{照亮}{13,9}[Radicais ⽕、⼇]
  \begin{phonetics}{照亮}{zhao4liang4}
    \definition{s.}{iluminação}
    \definition{v.}{iluminar}
  \end{phonetics}
\end{entry}

\begin{entry}{照相}{13,9}[Radicais ⽕、⽬]
  \begin{phonetics}{照相}{zhao4 xiang4}[][HSK 2]
    \definition{v.+compl.}{tirar fotografia}
  \end{phonetics}
\end{entry}

\begin{entry}{照相机}{13,9,6}[Radicais ⽕、⽬、⽊]
  \begin{phonetics}{照相机}{zhao4xiang4ji1}
    \definition[个,架,部,台,只]{s.}{câmera/máquina fotográfica}
  \end{phonetics}
\end{entry}

\begin{entry}{照准}{13,10}[Radicais ⽕、⼎]
  \begin{phonetics}{照准}{zhao4zhun3}
    \definition{s.}{solicitação concedida (uso formal em documento antigo)}
    \definition{v.}{mirar (arma)}
  \end{phonetics}
\end{entry}

\begin{entry}{照顾}{13,10}[Radicais ⽕、⾴]
  \begin{phonetics}{照顾}{zhao4gu4}[][HSK 2]
    \definition{v.}{cuidar de | atender a | oferecer tratamento preferencial | (de um cliente) patrocinar | fazer compras em | dar consideração a | mostrar consideração por | levar em conta | fazer concessões para}
  \end{phonetics}
\end{entry}

\begin{entry}{照骗}{13,12}[Radicais ⽕、⾺]
  \begin{phonetics}{照骗}{zhao4pian4}
    \definition{s.}{imagem ``photoshopada''}
  \end{phonetics}
\end{entry}

\begin{entry}{照像}{13,13}[Radicais ⽕、⼈]
  \begin{phonetics}{照像}{zhao4 xiang4}
    \variantof{照相}
  \end{phonetics}
\end{entry}

\begin{entry}{照像机}{13,13,6}[Radicais ⽕、⼈、⽊]
  \begin{phonetics}{照像机}{zhao4xiang4ji1}
    \variantof{照相机}
  \end{phonetics}
\end{entry}

\begin{entry}{瑜伽}{13,7}[Radicais ⽟、⼈]
  \begin{phonetics}{瑜伽}{yu2jia1}
    \definition*{s.}{Ioga}
  \end{phonetics}
\end{entry}

\begin{entry}{瑜珈}{13,9}[Radicais ⽟、⽟]
  \begin{phonetics}{瑜珈}{yu2jia1}
    \variantof{瑜伽}
  \end{phonetics}
\end{entry}

\begin{entry}{睡}{13}[Radical ⽬]
  \begin{phonetics}{睡}{shui4}[][HSK 1]
    \definition{v.}{dormir}
  \end{phonetics}
\end{entry}

\begin{entry}{睡衣}{13,6}[Radicais ⽬、⾐]
  \begin{phonetics}{睡衣}{shui4yi1}
    \definition{s.}{pijamas | roupas de dormir}
  \end{phonetics}
\end{entry}

\begin{entry}{睡觉}{13,9}[Radicais ⽬、⾒]
  \begin{phonetics}{睡觉}{shui4jiao4}[][HSK 1]
    \definition{v.+compl.}{ir para a cama | dormir | deitar-se}
  \end{phonetics}
\end{entry}

\begin{entry}{睡懒觉}{13,16,9}[Radicais ⽬、⼼、⾒]
  \begin{phonetics}{睡懒觉}{shui4lan3jiao4}
    \definition{v.}{levantar-se tarde | passar o tempo a dormir}
  \end{phonetics}
\end{entry}

\begin{entry}{矮}{13}[Radical ⽮]
  \begin{phonetics}{矮}{ai3}
    \definition{adj.}{baixo em estatura, dimensão, grau ou ranque | curto (em comprimento)}
  \end{phonetics}
\end{entry}

\begin{entry}{矮人}{13,2}[Radicais ⽮、⼈]
  \begin{phonetics}{矮人}{ai3ren2}
    \definition{s.}{anão | homúnculo | nanismo}
  \end{phonetics}
\end{entry}

\begin{entry}{矮子}{13,3}[Radicais ⽮、⼦]
  \begin{phonetics}{矮子}{ai3zi5}
    \definition{s.}{pessoa baixa | anão}
  \end{phonetics}
\end{entry}

\begin{entry}{矮小}{13,3}[Radicais ⽮、⼩]
  \begin{phonetics}{矮小}{ai3xiao3}
    \definition{adj.}{baixo e pequeno | curto e pequeno | subdimensionado}
  \end{phonetics}
\end{entry}

\begin{entry}{矮林}{13,8}[Radicais ⽮、⽊]
  \begin{phonetics}{矮林}{ai3lin2}
    \definition{s.}{mato | mata}
  \end{phonetics}
\end{entry}

\begin{entry}{矮星}{13,9}[Radicais ⽮、⽇]
  \begin{phonetics}{矮星}{ai3xing1}
    \definition{s.}{estrela anã}
  \end{phonetics}
\end{entry}

\begin{entry}{矮树}{13,9}[Radicais ⽮、⽊]
  \begin{phonetics}{矮树}{ai3shu4}
    \definition{s.}{arbusto | árvore pequena}
  \end{phonetics}
\end{entry}

\begin{entry}{矮胖}{13,9}[Radicais ⽮、⾁]
  \begin{phonetics}{矮胖}{ai3pang4}
    \definition{adj.}{atarracado |  gorducho | rechonchudo | roliço | curto e robusto}
  \end{phonetics}
\end{entry}

\begin{entry}{矮凳}{13,14}[Radicais ⽮、⼏]
  \begin{phonetics}{矮凳}{ai3deng4}
    \definition{s.}{banquinho baixo | banqueta}
  \end{phonetics}
\end{entry}

\begin{entry}{碍事}{13,8}[Radicais ⽯、⼅]
  \begin{phonetics}{碍事}{ai4shi4}
    \definition{s.}{(usualmente em frases negativas) sem consequência, não importa}
    \definition{v.+compl.}{estar no caminho | ser um obstáculo}
  \end{phonetics}
\end{entry}

\begin{entry}{碎}{13}[Radical ⽯]
  \begin{phonetics}{碎}{sui4}
    \definition{adj.}{quebrado | fragmentado | espalhado | tagarela}
    \definition{v.}{(transitivo ou intransitivo) quebrar em pedaços, quebrar, desmoronar}
  \end{phonetics}
\end{entry}

\begin{entry}{碗}{13}[Radical ⽯]
  \begin{phonetics}{碗}{wan3}[][HSK 2]
    \definition{clas.}{tigelas}
    \definition[只,个]{s.}{tigela}
  \end{phonetics}
\end{entry}

\begin{entry}{碗子}{13,3}[Radicais ⽯、⼦]
  \begin{phonetics}{碗子}{wan3zi5}
    \definition{s.}{tigela}
  \end{phonetics}
\end{entry}

\begin{entry}{碗柜}{13,8}[Radicais ⽯、⽊]
  \begin{phonetics}{碗柜}{wan3gui4}
    \definition{s.}{armário}
  \end{phonetics}
\end{entry}

\begin{entry}{碰}{13}[Radical ⽯]
  \begin{phonetics}{碰}{peng4}[][HSK 2]
    \definition{v.}{tocar | bater | encontrar | correr para | tentar a sorte | arriscar | encontrar para discutir}
  \end{phonetics}
\end{entry}

\begin{entry}{碰见}{13,4}[Radicais ⽯、⾒]
  \begin{phonetics}{碰见}{peng4 jian4}[][HSK 2]
    \definition{v.}{reunir-se | encontrar}
  \end{phonetics}
\end{entry}

\begin{entry}{碰头}{13,5}[Radicais ⽯、⼤]
  \begin{phonetics}{碰头}{peng4tou2}
    \definition{s.}{colisão | conflito}
    \definition{v.}{colidir}
    \definition{v.+compl.}{conhecer e discutir | juntar ideias | ver-se}
  \end{phonetics}
\end{entry}

\begin{entry}{碰运气}{13,7,4}[Radicais ⽯、⾡、⽓]
  \begin{phonetics}{碰运气}{peng4yun4qi5}
    \definition{v.}{deixar algo ao acaso | tentar a sorte}
  \end{phonetics}
\end{entry}

\begin{entry}{碰到}{13,8}[Radicais ⽯、⼑]
  \begin{phonetics}{碰到}{peng4 dao4}[][HSK 2]
    \definition{v.}{encontrar (com) | esbarrar em | deparar-se com}
  \end{phonetics}
\end{entry}

\begin{entry}{福}{13}[Radical ⽰]
  \begin{phonetics}{福}{fu2}[][HSK 3]
    \definition*{s.}{sobrenome Fu}
    \definition{s.}{benção; felicidade; boa sorte; boa fortuna}
    \definition{v.}{curvar-se; reverenciar}
  \end{phonetics}
\end{entry}

\begin{entry}{福克斯}{13,7,12}[Radicais ⽰、⼗、⽄]
  \begin{phonetics}{福克斯}{fu2ke4si1}
    \definition*{s.}{Fox (empresa de mídia) | Focus (automóvel fabricado pela Ford)}
  \end{phonetics}
\end{entry}

\begin{entry}{福泽}{13,8}[Radicais ⽰、⽔]
  \begin{phonetics}{福泽}{fu2ze2}
    \definition{s.}{boa sorte}
  \end{phonetics}
\end{entry}

\begin{entry}{筷子}{13,3}[Radicais ⽵、⼦]
  \begin{phonetics}{筷子}{kuai4zi5}[][HSK 2]
    \definition[对,根,把,双]{s.}{pauzinhos | \emph{chopsticks}}
  \end{phonetics}
\end{entry}

\begin{entry}{签}{13}[Radical ⽵]
  \begin{phonetics}{签}{qian1}
    \definition{s.}{vara de bambu com inscrição (usada em adivinhação, jogos de azar, sorteios, etc.) | rótulo | pequena lasca de madeira | etiqueta}
    \definition{v.}{assinar}
  \end{phonetics}
\end{entry}

\begin{entry}{签名}{13,6}[Radicais ⽵、⼝]
  \begin{phonetics}{签名}{qian1ming2}
    \definition{s.}{assinatura}
    \definition{v.+compl.}{autografar | assinar}
  \end{phonetics}
\end{entry}

\begin{entry}{简单}{13,8}[Radicais ⽵、⼗]
  \begin{phonetics}{简单}{jian3dan1}[][HSK 3]
    \definition{adj.}{simples; descomplicado | comum; lugar-comum | casual; simplificado}
  \end{phonetics}
\end{entry}

\begin{entry}{简直}{13,8}[Radicais ⽵、⽬]
  \begin{phonetics}{简直}{jian3zhi2}[][HSK 3]
    \definition{adv.}{simplesmente; em tudo; virtualmente}
  \end{phonetics}
\end{entry}

\begin{entry}{缝纫}{13,6}[Radicais ⽷、⽷]
  \begin{phonetics}{缝纫}{feng2ren4}
    \definition{v.}{costurar}
  \end{phonetics}
\end{entry}

\begin{entry}{缝纫机}{13,6,6}[Radicais ⽷、⽷、⽊]
  \begin{phonetics}{缝纫机}{feng2ren4ji1}
    \definition[架]{s.}{máquina de costura}
  \end{phonetics}
\end{entry}

\begin{entry}{罪犯}{13,5}[Radicais ⽹、⽝]
  \begin{phonetics}{罪犯}{zui4fan4}
    \definition{s.}{criminoso}
  \end{phonetics}
\end{entry}

\begin{entry}{罪行}{13,6}[Radicais ⽹、⾏]
  \begin{phonetics}{罪行}{zui4xing2}
    \definition{s.}{crime | ofensa}
  \end{phonetics}
\end{entry}

\begin{entry}{置疑}{13,14}[Radicais ⽹、⽦]
  \begin{phonetics}{置疑}{zhi4yi2}
    \definition{v.}{duvidar}
  \end{phonetics}
\end{entry}

\begin{entry}{群}{13}[Radical ⽺]
  \begin{phonetics}{群}{qun2}[][HSK 3]
    \definition{clas.}{grupo; rebanho; manada}
    \definition{s.}{multidão; grupo}
  \end{phonetics}
\end{entry}

\begin{entry}{群山}{13,3}[Radicais ⽺、⼭]
  \begin{phonetics}{群山}{qun2shan1}
    \definition{s.}{montanhas | uma cadeia de colinas}
  \end{phonetics}
\end{entry}

\begin{entry}{腰}{13}[Radical ⾁]
  \begin{phonetics}{腰}{yao1}
    \definition{s.}{cintura}
  \end{phonetics}
\end{entry}

\begin{entry}{腰包}{13,5}[Radicais ⾁、⼓]
  \begin{phonetics}{腰包}{yao1bao1}
    \definition{s.}{pochete | bolso}
  \end{phonetics}
\end{entry}

\begin{entry}{腰椎}{13,12}[Radicais ⾁、⽊]
  \begin{phonetics}{腰椎}{yao1zhui1}
    \definition{s.}{vértebra lombar (espinha dorsal inferior)}
  \end{phonetics}
\end{entry}

\begin{entry}{腿}{13}[Radical ⾁]
  \begin{phonetics}{腿}{tui3}[][HSK 2]
    \definition[条]{s.}{perna | osso do quadril}
  \end{phonetics}
\end{entry}

\begin{entry}{腿号}{13,5}[Radicais ⾁、⼝]
  \begin{phonetics}{腿号}{tui3hao4}
    \definition{s.}{anilha numerada (por exemplo, usada para identificar pássaros)}
  \seealsoref{腿号箍}{tui3hao4gu1}
  \end{phonetics}
\end{entry}

\begin{entry}{腿号箍}{13,5,14}[Radicais ⾁、⼝、⽵]
  \begin{phonetics}{腿号箍}{tui3hao4gu1}
    \definition{s.}{anilha numerada (por exemplo, usada para identificar pássaros)}
  \seealsoref{腿号}{tui3hao4}
  \end{phonetics}
\end{entry}

\begin{entry}{艁}{13}[Radical ⾈]
  \begin{phonetics}{艁}{zao4}
    \variantof{造}
  \end{phonetics}
\end{entry}

\begin{entry}{蒙面}{13,9}[Radicais ⾋、⾯]
  \begin{phonetics}{蒙面}{meng2mian4}
    \definition{adj.}{descarado | desavergonhado | mascarado}
    \definition{v.}{cobrir o rosto | usar uma máscara}
  \end{phonetics}
\end{entry}

\begin{entry}{蓝}{13}[Radical ⾋]
  \begin{phonetics}{蓝}{lan2}[][HSK 2]
    \definition*{s.}{sobrenome Lan}
    \definition{adj.}{azul}
  \end{phonetics}
\end{entry}

\begin{entry}{蓝色}{13,6}[Radicais ⾋、⾊]
  \begin{phonetics}{蓝色}{lan2 se4}[][HSK 2]
    \definition{s.}{cor azul}
  \end{phonetics}
\end{entry}

\begin{entry}{解开}{13,4}[Radicais ⾓、⼶]
  \begin{phonetics}{解开}{jie3 kai1}[][HSK 3]
    \definition{v.}{desatar; desfazer; desamarrar; desabotoar}
  \end{phonetics}
\end{entry}

\begin{entry}{解决}{13,6}[Radicais ⾓、⼎]
  \begin{phonetics}{解决}{jie3jue2}[][HSK 3]
    \definition{v.}{solucionar; resolver; liquidar | acabar com; descartar}
  \end{phonetics}
\end{entry}

\begin{entry}{解压}{13,6}[Radicais ⾓、⼚]
  \begin{phonetics}{解压}{jie3ya1}
    \definition{v.}{aliviar o estresse | (computação) descomprimir}
  \end{phonetics}
\end{entry}

\begin{entry}{解救}{13,11}[Radicais ⾓、⽁]
  \begin{phonetics}{解救}{jie3jiu4}
    \definition{v.}{resgatar | ajudar a sair de dificuldades | salvar a situação}
  \end{phonetics}
\end{entry}

\begin{entry}{解释}{13,12}[Radicais ⾓、⾤]
  \begin{phonetics}{解释}{jie3shi4}
    \definition[个]{s.}{explicação}
    \definition{v.}{explicar | interpretar | resolver}
  \end{phonetics}
\end{entry}

\begin{entry}{解雇}{13,12}[Radicais ⾓、⾫]
  \begin{phonetics}{解雇}{jie3gu4}
    \definition{v.}{demitir}
  \end{phonetics}
\end{entry}

\begin{entry}{谩骂}{13,9}[Radicais ⾔、⾺]
  \begin{phonetics}{谩骂}{man4ma4}
    \definition{v.}{ridicularizar | abusar}
  \end{phonetics}
\end{entry}

\begin{entry}{赖}{13}[Radical ⾙]
  \begin{phonetics}{赖}{lai4}
    \definition*{s.}{sobrenome Lai}
    \definition{v.}{depender | aguentar em um lugar | renegar (promessa) | isolar-se | culpar | colocar a culpa em}
  \end{phonetics}
\end{entry}

\begin{entry}{跟}{13}[Radical ⾜]
  \begin{phonetics}{跟}{gen1}[][HSK 1]
    \definition{conj.}{e; com}
    \definition{prep.}{com}
    \definition{v.}{acompanhar junto | seguir de perto | ir com}
  \end{phonetics}
\end{entry}

\begin{entry}{跪拜}{13,9}[Radicais ⾜、⼿]
  \begin{phonetics}{跪拜}{gui4bai4}
    \definition{v.}{prostrar-se | ajoelhar-se e adorar}
  \end{phonetics}
\end{entry}

\begin{entry}{路}{13}[Radical ⾜]
  \begin{phonetics}{路}{lu4}[][HSK 1]
    \definition*{s.}{sobrenome Lu}
    \definition[条]{s.}{caminho | estrada | via | jornada | linha (ônibus, etc.) | rota}
  \end{phonetics}
\end{entry}

\begin{entry}{路上}{13,3}[Radicais ⾜、⼀]
  \begin{phonetics}{路上}{lu4shang5}[][HSK 1]
    \definition{adv.}{na estrada | no caminho | a caminho}
  \end{phonetics}
\end{entry}

\begin{entry}{路口}{13,3}[Radicais ⾜、⼝]
  \begin{phonetics}{路口}{lu4kou3}[][HSK 1]
    \definition{s.}{cruzamento | interseção (de estradas)}
  \end{phonetics}
\end{entry}

\begin{entry}{路边}{13,5}[Radicais ⾜、⾡]
  \begin{phonetics}{路边}{lu4 bian1}[][HSK 2]
    \definition{s.}{meio-fio | acostamento}
  \end{phonetics}
\end{entry}

\begin{entry}{路线}{13,8}[Radicais ⾜、⽷]
  \begin{phonetics}{路线}{lu4 xian4}[][HSK 3]
    \definition[条]{s.}{rota; caminho; linha | linha; diretriz (de política, ideologia, campo de trabalho)}
  \end{phonetics}
\end{entry}

\begin{entry}{跳}{13}[Radical ⾜]
  \begin{phonetics}{跳}{tiao4}[][HSK 3]
    \definition{v.}{pular; saltar; quicar | mover para cima e para baixo; pulsar; palpitar; contrair-se | pular; saltar por cima}
  \end{phonetics}
\end{entry}

\begin{entry}{跳水}{13,4}[Radicais ⾜、⽔]
  \begin{phonetics}{跳水}{tiao4shui3}
    \definition{s.}{mergulho esportivo}
    \definition{v.}{mergulhar (na água) | cometer suicídio pulando na água | (figurativo, preços das ações, etc.) cair dramaticamente}
  \end{phonetics}
\end{entry}

\begin{entry}{跳电}{13,5}[Radicais ⾜、⽥]
  \begin{phonetics}{跳电}{tiao4dian4}
    \definition{v.}{desarmar (um disjuntor ou interruptor)}
  \end{phonetics}
\end{entry}

\begin{entry}{跳伞}{13,6}[Radicais ⾜、⼈]
  \begin{phonetics}{跳伞}{tiao4san3}
    \definition{s.}{paraquedas}
    \definition{v.}{saltar de paraquedas}
  \end{phonetics}
\end{entry}

\begin{entry}{跳远}{13,7}[Radicais ⾜、⾡]
  \begin{phonetics}{跳远}{tiao4 yuan3}[][HSK 3]
    \definition{s.}{salto em distância (atletismo)}
  \end{phonetics}
\end{entry}

\begin{entry}{跳挡}{13,9}[Radicais ⾜、⼿]
  \begin{phonetics}{跳挡}{tiao4dang3}
    \definition{v.}{pular marcha (de um carro) | perder a marcha}
  \end{phonetics}
\end{entry}

\begin{entry}{跳蚤}{13,9}[Radicais ⾜、⾍]
  \begin{phonetics}{跳蚤}{tiao4zao5}
    \definition{s.}{pulga}
  \end{phonetics}
\end{entry}

\begin{entry}{跳高}{13,10}[Radicais ⾜、⾼]
  \begin{phonetics}{跳高}{tiao4 gao1}[][HSK 3]
    \definition{s.}{salto em altura (atletismo)}
  \end{phonetics}
\end{entry}

\begin{entry}{跳绳}{13,11}[Radicais ⾜、⽷]
  \begin{phonetics}{跳绳}{tiao4sheng2}
    \definition{v.}{pular corda}
  \end{phonetics}
\end{entry}

\begin{entry}{跳跳糖}{13,13,16}[Radicais ⾜、⾜、⽶]
  \begin{phonetics}{跳跳糖}{tiao4tiao4tang2}
    \definition{s.}{\emph{Pop Rocks}, \emph{popping candy}}
  \end{phonetics}
\end{entry}

\begin{entry}{跳频}{13,13}[Radicais ⾜、⾴]
  \begin{phonetics}{跳频}{tiao4pin2}
    \definition{s.}{FHSS, \emph{Frequency-Hopping Spread Spectrum}, método de transmissão de sinais de rádio}
  \end{phonetics}
\end{entry}

\begin{entry}{跳舞}{13,14}[Radicais ⾜、⾇]
  \begin{phonetics}{跳舞}{tiao4wu3}[][HSK 3]
    \definition{v.+compl.}{dançar (como performance)}
  \end{phonetics}
\end{entry}

\begin{entry}{躲}{13}[Radical ⾝]
  \begin{phonetics}{躲}{duo3}
    \definition{v.}{esconder | esquivar | evitar}
  \end{phonetics}
\end{entry}

\begin{entry}{躲闪}{13,5}[Radicais ⾝、⾨]
  \begin{phonetics}{躲闪}{duo3shan3}
    \definition{v.}{desviar | evadir | esquivar (para fora do caminho)}
  \end{phonetics}
\end{entry}

\begin{entry}{输}{13}[Radical ⾞]
  \begin{phonetics}{输}{shu1}[][HSK 3]
    \definition{v.}{transportar; transmitir | contribuir com dinheiro; doar | perder; ser batido; ser derrotado}
  \end{phonetics}
\end{entry}

\begin{entry}{输入}{13,2}[Radicais ⾞、⼊]
  \begin{phonetics}{输入}{shu1ru4}[][HSK 3]
    \definition{v.}{introduzir; importar  (de fora para dentro) | inserir informações, programas, dados, sinais, etc. em uma máquina}
  \end{phonetics}
\end{entry}

\begin{entry}{辞典}{13,8}[Radicais ⾟、⼋]
  \begin{phonetics}{辞典}{ci2dian3}
    \variantof{词典}
  \end{phonetics}
\end{entry}

\begin{entry}{遛狗}{13,8}[Radicais ⾡、⽝]
  \begin{phonetics}{遛狗}{liu4gou3}
    \definition{v.+compl.}{passear com um cachorro}
  \end{phonetics}
\end{entry}

\begin{entry}{遥控}{13,11}[Radicais ⾡、⼿]
  \begin{phonetics}{遥控}{yao2kong4}
    \definition{s.}{controle remoto}
    \definition{v.}{dirigir operações de um local remoto | controlar remotamente}
  \end{phonetics}
\end{entry}

\begin{entry}{酬劳}{13,7}[Radicais ⾣、⼒]
  \begin{phonetics}{酬劳}{chou2lao2}
    \definition{s.}{recompensa}
  \end{phonetics}
\end{entry}

\begin{entry}{酱}{13}[Radical ⾣]
  \begin{phonetics}{酱}{jiang4}
    \definition{s.}{pasta grossa de soja fermentada | marinada em pasta de soja | pasta | geléia}
  \end{phonetics}
\end{entry}

\begin{entry}{错}{13}[Radical ⾦]
  \begin{phonetics}{错}{cuo4}[][HSK 1]
    \definition*{s.}{sobrenome Cuo}
    \definition{adj.}{errado | enganado}
  \end{phonetics}
\end{entry}

\begin{entry}{错误}{13,9}[Radicais ⾦、⾔]
  \begin{phonetics}{错误}{cuo4wu4}[][HSK 3]
    \definition{adj.}{equivocado; errado; errôneo}
    \definition[个,次]{s.}{engano; erro; erro grosseiro; falha}
  \end{phonetics}
\end{entry}

\begin{entry}{锤}{13}[Radical ⾦]
  \begin{phonetics}{锤}{chui2}
    \definition{s.}{martelo | marreta}
    \definition{s.}{pesos (por exemplo, de uma balança)}
    \definition{v.}{marterlar para dar forma | atacar com um martelo}
  \end{phonetics}
\end{entry}

\begin{entry}{锦上添花}{13,3,11,7}[Radicais ⾦、⼀、⽔、⾋]
  \begin{phonetics}{锦上添花}{jin3shang4tian1hua1}
    \definition{expr.}{A cereja do bolo | (literalmente) adicione flores ao brocato}
    \definition{v.}{dar a alguém esplendor adicional | fornecer o toque final}
  \end{phonetics}
\end{entry}

\begin{entry}{键}{13}[Radical ⾦]
  \begin{phonetics}{键}{jian4}
    \definition{s.}{tecla (em um teclado de piano ou computador) | botão (em um mouse ou outro dispositivo) | ligação química | cavilha de roda | chaveta}
  \end{phonetics}
\end{entry}

\begin{entry}{零下}{13,3}[Radicais ⾬、⼀]
  \begin{phonetics}{零下}{ling2 xia4}[][HSK 2]
    \definition{s.}{abaixo de zero}
  \end{phonetics}
\end{entry}

\begin{entry}{零/〇}{13,13}[Radicais ⾬、⾬]
  \begin{phonetics}{零/〇}{ling2 ling2}[][HSK 1]
    \definition{adj.}{extra}
    \definition{num.}{zero; 0}
    \definition{s.}{(matemática) resto (após a divisão) | fração | nada}
  \end{phonetics}
\end{entry}

\begin{entry}{雷亚尔}{13,6,5}[Radicais ⾬、⼆、⼩]
  \begin{phonetics}{雷亚尔}{lei2ya4'er3}
    \definition*{s.}{Real Brasileiro}
  \end{phonetics}
\end{entry}

\begin{entry}{雾气}{13,4}[Radicais ⾬、⽓]
  \begin{phonetics}{雾气}{wu4qi4}
    \definition{s.}{nevoeiro | névoa | vapor}
  \end{phonetics}
\end{entry}

\begin{entry}{颐和园}{13,8,7}[Radicais ⾴、⼝、⼞]
  \begin{phonetics}{颐和园}{yi2he2yuan2}
    \definition*{s.}{Palácio de Verão}
  \end{phonetics}
\end{entry}

\begin{entry}{频道}{13,12}[Radicais ⾴、⾡]
  \begin{phonetics}{频道}{pin2dao4}
    \definition{s.}{frequência | (televisão) canal}
  \end{phonetics}
\end{entry}

\begin{entry}{魂}{13}[Radical ⿁]
  \begin{phonetics}{魂}{hun2}
    \definition{s.}{alma | espírito | alma imortal (que pode ser separada do corpo)}
  \end{phonetics}
\end{entry}

\begin{entry}{鼓掌}{13,12}[Radicais ⿎、⼿]
  \begin{phonetics}{鼓掌}{gu3zhang3}
    \definition{v.+compl.}{aplaudir | bater palmas}
  \end{phonetics}
\end{entry}

%%%%% EOF %%%%%


%%%
%%% 14画
%%%

\section*{14画}\addcontentsline{toc}{section}{14画}

\begin{entry}{㮸}{14}{⽊}
  \begin{phonetics}{㮸}{song4}
    \variantof{送}
  \end{phonetics}
\end{entry}

\begin{entry}{僧}{14}{⼈}
  \begin{phonetics}{僧}{seng1}
    \definition{s.}{monge Budista, abreviação de 僧伽}
  \seealsoref{僧伽}{seng1qie2}
  \end{phonetics}
\end{entry}

\begin{entry}{僧伽}{14,7}{⼈、⼈}
  \begin{phonetics}{僧伽}{seng1qie2}
    \definition{s.}{sangha ou sanga (Budismo) | a comunidade monástica | monge}
  \end{phonetics}
\end{entry}

\begin{entry}{嘉年华}{14,6,6}{⼝、⼲、⼗}
  \begin{phonetics}{嘉年华}{jia1nian2hua2}
    \definition{s.}{(empréstimo linguístico) carnaval}
  \end{phonetics}
\end{entry}

\begin{entry}{墙}{14}{⼟}
  \begin{phonetics}{墙}{qiang2}[][HSK 2]
    \definition[面,堵,道]{s.}{parede; barreira ou perímetro construído com tijolos, pedras, etc. | qualquer coisa com a forma ou função de uma parede; a parte de um objeto que funciona como parede ou divisória}
    \definition{v.}{(gíria) bloquear (um website) (usado geralmente na voz passiva: 被墙)}
  \end{phonetics}
\end{entry}

\begin{entry}{墙纸}{14,7}{⼟、⽷}
  \begin{phonetics}{墙纸}{qiang2zhi3}
    \definition{s.}{papel de parede}
  \end{phonetics}
\end{entry}

\begin{entry}{墙壁}{14,16}{⼟、⼟}
  \begin{phonetics}{墙壁}{qiang2 bi4}[][HSK 5]
    \definition[堵]{s.}{parede; barreira ou perímetro construído com tijolos, pedras ou terra}
  \end{phonetics}
\end{entry}

\begin{entry}{墬}{14}{⼟}
  \begin{phonetics}{墬}{di4}
    \variantof{地}
  \end{phonetics}
\end{entry}

\begin{entry}{寡}{14}{⼧}
  \begin{phonetics}{寡}{gua3}
    \definition{adj.}{poucos; escassos (oposto a 众, 多)  | insípido; sem sabor | pouco; escasso | insípido; sem graça}
  \seealsoref{多}{duo1}
  \seealsoref{众}{zhong4}
  \end{phonetics}
\end{entry}

\begin{entry}{寨}{14}{⼧}
  \begin{phonetics}{寨}{zhai4}
    \definition{s.}{fortaleza | paliçada | acampamento | vila (paliçada)}
  \end{phonetics}
\end{entry}

\begin{entry}{愿}{14}{⽕}
  \begin{phonetics}{愿}{yuan4}[][HSK 5]
    \definition{adj.}{honesto e prudente}
    \definition{s.}{esperança; desejo; vontade; a ideia de alcançar algum objetivo no futuro | voto (feito perante o Buda ou um deus); o desejo de retribuição feito ao rezar para os deuses e Buda}
    \definition{v.}{estar disposto; estar pronto; de bom grado, concordar porque está de acordo com seus desejos | ter esperança; desejar; qerer alcançar algum desejo}
  \end{phonetics}
\end{entry}

\begin{entry}{愿望}{14,11}{⽕、⽉}
  \begin{phonetics}{愿望}{yuan4wang4}[][HSK 3]
    \definition[个]{s.}{desejo; aspiração; a ideia de esperar atingir um determinado objetivo no futuro}
  \end{phonetics}
\end{entry}

\begin{entry}{愿意}{14,13}{⽕、⼼}
  \begin{phonetics}{愿意}{yuan4yi4}[][HSK 2]
    \definition{s.}{desejo | esperança}
    \definition{v.}{estar disposto | estar pronto}
  \end{phonetics}
\end{entry}

\begin{entry}{慢}{14}{⼼}
  \begin{phonetics}{慢}{man4}[][HSK 1]
    \definition*{s.}{sobrenome Man}
    \definition{adj.}{lento; devagar; baixa velocidade; longa duração (em oposição a 快) | rude; arrogante; sem educação com as pessoas | frouxo; lento}
    \definition{adv.}{lentamente}
  \seealsoref{快}{kuai4}
  \end{phonetics}
\end{entry}

\begin{entry}{慢动作}{14,6,7}{⼼、⼒、⼈}
  \begin{phonetics}{慢动作}{man4dong4zuo4}
    \definition{s.}{(cinema) câmera lenta}
  \end{phonetics}
\end{entry}

\begin{entry}{慢慢}{14,14}{⼼、⼼}
  \begin{phonetics}{慢慢}{man4 man4}[][HSK 3]
    \definition{adv.}{lentamente; vagarosamente; gradualmente}
  \end{phonetics}
\end{entry}

\begin{entry}{摔}{14}{⼿}
  \begin{phonetics}{摔}{shuai1}[][HSK 5]
    \definition{v.}{cair; tropeçar; perder o equilíbrio | mergulhar; precipitar-se; cair de uma altura elevada | quebrar; fazer cair e quebrar | lançar; atirar; arremessar; joguar coisas com força e para baixo | bater; golpear; bater com força para que o que está grudado cair}
  \end{phonetics}
\end{entry}

\begin{entry}{摔倒}{14,10}{⼿、⼈}
  \begin{phonetics}{摔倒}{shuai1dao3}[][HSK 5]
    \definition{v.}{cair; tropeçar; perder o equilíbrio e cair}
  \end{phonetics}
\end{entry}

\begin{entry}{摘}{14}{⼿}
  \begin{phonetics}{摘}{zhai1}[][HSK 5]
    \definition{v.}{pegar; arrancar; tirar; colher (flores, frutos, folhas de plantas); retirar (coisas que estão sendo usadas ou penduradas) | selecionar; fazer extrações de | pedir dinheiro emprestado em caso de necessidade urgente | vencer; ganhar; alcançar; obter}
  \end{phonetics}
\end{entry}

\begin{entry}{敲}{14}{⽁}
  \begin{phonetics}{敲}{qiao1}[][HSK 5]
    \definition{v.}{bater; dar uma pancada; golpear | explorar alguém; cobrar a mais; extorquir; chantagear | lembrar; criticar; alertar; advertir}
  \end{phonetics}
\end{entry}

\begin{entry}{敲门}{14,3}{⽁、⾨}
  \begin{phonetics}{敲门}{qiao1 men2}[][HSK 5]
    \definition{v.}{bater na porta}
  \end{phonetics}
\end{entry}

\begin{entry}{斡旋}{14,11}{⽃、⽅}
  \begin{phonetics}{斡旋}{wo4xuan2}
    \definition{v.}{mediar (um conflito, etc.)}
  \end{phonetics}
\end{entry}

\begin{entry}{旗}{14}{⽅}
  \begin{phonetics}{旗}{qi2}
    \definition[面]{s.}{bandeira}
  \end{phonetics}
\end{entry}

\begin{entry}{槃}{14}{⽊}
  \begin{phonetics}{槃}{pan2}
    \variantof{盘}
  \end{phonetics}
\end{entry}

\begin{entry}{模仿}{14,6}{⽊、⼈}
  \begin{phonetics}{模仿}{mo2fang3}[][HSK 5]
    \definition{v.}{copiar; imitar; aprender a fazer algo seguindo um modelo pronto}
  \end{phonetics}
\end{entry}

\begin{entry}{模式}{14,6}{⽊、⼷}
  \begin{phonetics}{模式}{mo2shi4}[][HSK 5]
    \definition{s.}{modelo; modo; padrão; a forma padrão de algo ou o modelo padrão que as pessoas podem seguir}
  \end{phonetics}
\end{entry}

\begin{entry}{模具}{14,8}{⽊、⼋}
  \begin{phonetics}{模具}{mu2ju4}
    \definition{s.}{molde | matriz | padrão}
  \end{phonetics}
\end{entry}

\begin{entry}{模型}{14,9}{⽊、⼟}
  \begin{phonetics}{模型}{mo2xing2}[][HSK 4]
    \definition[个]{s.}{modelo; padrão; itens feitos em escala com base em objetos ou desenhos | molde; padrão; molde para fundir máquinas, objetos, etc.}
  \end{phonetics}
\end{entry}

\begin{entry}{模范}{14,9}{⽊、⾋}
  \begin{phonetics}{模范}{mo2fan4}[][HSK 5]
    \definition{adj.}{exemplar}
    \definition{s.}{modelo; exemplo excelente; pessoa exemplar; coisa exemplar; pessoas ou coisas exemplares que servem de modelo}
  \end{phonetics}
\end{entry}

\begin{entry}{模样}{14,10}{⽊、⽊}
  \begin{phonetics}{模样}{mu2yang4}[][HSK 5]
    \definition[副,种]{s.}{aparência; a aparência ou o estilo de vestir de uma pessoa |
indicando uma estimativa aproximada de tempo ou idade; expressão de estimativas relativas a tempo, idade, etc. | tendência; situação; inclinação}
  \end{phonetics}
\end{entry}

\begin{entry}{模特儿}{14,10,2}{⽊、⽜、⼉}
  \begin{phonetics}{模特儿}{mo2 te4r5}[][HSK 4]
    \definition[个]{s.}{modelo (pessoa que posa para um fotógrafo ou pintor ou escultor); objeto de representação ou referência usado por artistas para esboços e esculturas, como o corpo humano, objetos, modelos etc.; também se refere aos arquétipos que os estudiosos da literatura usam para retratar seus personagens | modelo (uma pessoa que usa roupas para exibir modas); pessoa ou manequim usado para exibir estilos de roupas}
  \end{phonetics}
\end{entry}

\begin{entry}{模糊}{14,15}{⽊、⽶}
  \begin{phonetics}{模糊}{mo2hu5}[][HSK 5]
    \definition{adj.}{vago; confuso; indistinto}
    \definition{v.}{confundir; desorientar}
  \end{phonetics}
\end{entry}

\begin{entry}{歌}{14}{⽋}
  \begin{phonetics}{歌}{ge1}[][HSK 1]
    \definition[首,支,段]{s.}{canção; poesia cantável}
    \definition{v.}{cantar; entoar | louvar; exaltar; cantar louvores a}
  \end{phonetics}
\end{entry}

\begin{entry}{歌手}{14,4}{⽋、⼿}
  \begin{phonetics}{歌手}{ge1 shou3}[][HSK 3]
    \definition[个,位,名]{s.}{cantor; vocalista}
  \end{phonetics}
\end{entry}

\begin{entry}{歌曲}{14,6}{⽋、⽈}
  \begin{phonetics}{歌曲}{ge1 qu3}[][HSK 5]
    \definition{s.}{música; obra para as pessoas cantarem, uma combinação de poesia e música}
  \end{phonetics}
\end{entry}

\begin{entry}{歌声}{14,7}{⽋、⼠}
  \begin{phonetics}{歌声}{ge1 sheng1}[][HSK 3]
    \definition{s.}{voz cantada; som de canto}
  \end{phonetics}
\end{entry}

\begin{entry}{歌迷}{14,9}{⽋、⾡}
  \begin{phonetics}{歌迷}{ge1 mi2}
    \definition{s.}{fã de um cantor}
  \end{phonetics}
\end{entry}

\begin{entry}{滴}{14}{⽔}
  \begin{phonetics}{滴}{di1}
    \definition{s.}{uma gota}
    \definition{v.}{pingar}
  \end{phonetics}
\end{entry}

\begin{entry}{漂}{14}{⽔}
  \begin{phonetics}{漂}{piao1}
    \definition{v.}{flutuar | estar a deriva}
  \end{phonetics}
  \begin{phonetics}{漂}{piao3}
    \definition{v.}{alvejar | branquear}
  \end{phonetics}
  \begin{phonetics}{漂}{piao4}
    \definition{adj.}{usado em 漂亮}
  \seealsoref{漂亮}{piao4liang5}
  \end{phonetics}
\end{entry}

\begin{entry}{漂亮}{14,9}{⽔、⼇}
  \begin{phonetics}{漂亮}{piao4liang5}[][HSK 2]
    \definition{adj.}{bonito; lindo; atraente; de boa aparência; esteticamente agradável | excelente; notável | não pode ser utilizado para descrever homens}
  \end{phonetics}
\end{entry}

\begin{entry}{漂流}{14,10}{⽔、⽔}
  \begin{phonetics}{漂流}{piao1liu2}
    \definition{s.}{\emph{rafting}}
    \definition{v.}{ser levado pela correnteza | flutuar ao longo ou sobre}
  \end{phonetics}
\end{entry}

\begin{entry}{漏}{14}{⽔}
  \begin{phonetics}{漏}{lou4}[][HSK 5]
    \definition{s.}{relógio de água; ampulheta | falha; ponto fraco | gonorreia; a medicina tradicional chinesa refere-se a certas doenças que causam secreção de pus, sangue e muco | unidade de tempo medida por um relógio de água durante a noite}
    \definition{v.}{(líquido, gás, etc.) pingar; vazar; escorrer; cair (de um buraco ou fenda) | vazar; deixar escapar; divulgar | perder; deixar de fora por engano | vazar; o objeto tem poros e pode vazar coisas | há uma fuga de ar}
  \end{phonetics}
\end{entry}

\begin{entry}{漏电}{14,5}{⽔、⽥}
  \begin{phonetics}{漏电}{lou4dian4}
    \definition{v.}{vazar eletricidade}
  \end{phonetics}
\end{entry}

\begin{entry}{漏洞}{14,9}{⽔、⽔}
  \begin{phonetics}{漏洞}{lou4 dong4}[][HSK 5]
    \definition[个]{s.}{vazamento; rachadura; lacunas ou buracos desnecessários que permitem que coisas vazem | falha; defeito; lacuna; (fala, ação, método, etc.) imperfeições}
  \end{phonetics}
\end{entry}

\begin{entry}{演}{14}{⽔}
  \begin{phonetics}{演}{yan3}[][HSK 3]
    \definition{v.}{desenvolver; evoluir | deduzir; elaborar | exercitar; praticar | representar; atuar}
  \end{phonetics}
\end{entry}

\begin{entry}{演出}{14,5}{⽔、⼐}
  \begin{phonetics}{演出}{yan3chu1}[][HSK 3]
    \definition[场,次]{s.}{show; concerto; performance}
    \definition{v.}{apresentar; representar; fazer um show; apresentar peças de teatro, dança, arte popular, acrobacias, etc. para o público aproveitar}
  \end{phonetics}
\end{entry}

\begin{entry}{演讲}{14,6}{⽔、⾔}
  \begin{phonetics}{演讲}{yan3jiang3}[][HSK 4]
    \definition[场,次]{s.}{palestra; discurso; ato ou a atividade de apresentar ou expressar ideias, opiniões ou informações oralmente em público ou diante de um público}
    \definition{v.}{dar uma palestra; fazer um discurso; informar o público sobre uma determinada área de conhecimento ou opinião sobre um determinado assunto}
  \end{phonetics}
\end{entry}

\begin{entry}{演员}{14,7}{⽔、⼝}
  \begin{phonetics}{演员}{yan3yuan2}[][HSK 3]
    \definition[个,位,名]{s.}{ator; performer; participantes de teatro, cinema, dança, arte popular, acrobacia e outras apresentações}
  \end{phonetics}
\end{entry}

\begin{entry}{演唱}{14,11}{⽔、⼝}
  \begin{phonetics}{演唱}{yan3 chang4}[][HSK 3]
    \definition{v.}{cantar em uma performance; apresentar canções, óperas, dramas, etc.}
  \end{phonetics}
\end{entry}

\begin{entry}{演唱会}{14,11,6}{⽔、⼝、⼈}
  \begin{phonetics}{演唱会}{yan3 chang4 hui4}[][HSK 3]
    \definition[个,场]{s.}{concerto; recital vocal; concerto vocal}
  \end{phonetics}
\end{entry}

\begin{entry}{漫长}{14,4}{⽔、⾧}
  \begin{phonetics}{漫长}{man4chang2}[][HSK 5]
    \definition{adj.}{muito longo; interminável; (tempo, espaço) dura muito tempo}
  \end{phonetics}
\end{entry}

\begin{entry}{漫画}{14,8}{⽔、⽥}
  \begin{phonetics}{漫画}{man4hua4}[][HSK 5]
    \definition[幅,本,张,套]{s.}{desenho animado; caricatura}
  \end{phonetics}
\end{entry}

\begin{entry}{漫骂}{14,9}{⽔、⾺}
  \begin{phonetics}{漫骂}{man4ma4}
    \variantof{谩骂}
  \end{phonetics}
\end{entry}

\begin{entry}{熊}{14}{⽕}
  \begin{phonetics}{熊}{xiong2}[][HSK 5]
    \definition*{s.}{sobrenome Xiong}
    \definition[把]{s.}{urso}
    \definition{v.}{repreender; censurar;}
  \end{phonetics}
\end{entry}

\begin{entry}{熊猫}{14,11}{⽕、⽝}
  \begin{phonetics}{熊猫}{xiong2mao1}
    \definition[把,只]{s.}{panda gigante}
  \seealsoref{猫熊}{mao1xiong2}
  \end{phonetics}
\end{entry}

\begin{entry}{熏香}{14,9}{⽕、⾹}
  \begin{phonetics}{熏香}{xun1xiang1}
    \definition{s.}{incenso}
  \end{phonetics}
\end{entry}

\begin{entry}{疑问}{14,6}{⽦、⾨}
  \begin{phonetics}{疑问}{yi2wen4}[][HSK 4]
    \definition[个]{s.}{dúvida; consulta; pergunta; questionamento; coisas que não podem ser determinadas ou explicadas}
  \end{phonetics}
\end{entry}

\begin{entry}{瘦}{14}{⽧}
  \begin{phonetics}{瘦}{shou4}[][HSK 5]
    \definition{adj.}{magro; esquelético (oposto de 胖, 肥) | magro (oposto de 肥) | apertado (oposto de 肥) | infértil; pobre | esquelético; pouca gordura; pouca carne (em oposição a 或 ou 肥) | (roupas, sapatos, meias, etc.) apertado (em oposição a 肥) |magra; (carne comestível) com baixo teor de gordura (em oposição a 肥)}
    \definition{v.}{perder peso}
  \seealsoref{肥}{fei2}
  \seealsoref{或}{huo4}
  \seealsoref{胖}{pang4}
  \end{phonetics}
\end{entry}

\begin{entry}{碳}{14}{⽯}
  \begin{phonetics}{碳}{tan4}
    \definition{s.}{carbono (elemento químico)}
  \end{phonetics}
\end{entry}

\begin{entry}{磁带}{14,9}{⽯、⼱}
  \begin{phonetics}{磁带}{ci2dai4}
    \definition[盘,盒]{s.}{cassete | fita magnética}
  \end{phonetics}
\end{entry}

\begin{entry}{磁铁}{14,10}{⽯、⾦}
  \begin{phonetics}{磁铁}{ci2tie3}
    \definition{s.}{imã | magneto}
  \seealsoref{吸铁石}{xi1tie3shi2}
  \end{phonetics}
\end{entry}

\begin{entry}{磁盘}{14,11}{⽯、⽫}
  \begin{phonetics}{磁盘}{ci2pan2}
    \definition{s.}{disquete}
  \end{phonetics}
\end{entry}

\begin{entry}{稳}{14}{⽲}
  \begin{phonetics}{稳}{wen3}[][HSK 4]
    \definition{adj.}{constante; estável; firme | estável; estático; sedado | seguro; confiável; certo}
    \definition{adv.}{certamente; com certeza; seguramente; sem dúvida}
    \definition{v.}{estabilizar, manter estável}
  \end{phonetics}
\end{entry}

\begin{entry}{稳定}{14,8}{⽲、⼧}
  \begin{phonetics}{稳定}{wen3ding4}[][HSK 4]
    \definition{adj.}{estável; firme; descreve uma natureza, um estado, etc. relativamente fixo; não muda significativamente}
    \definition{s.}{estabilidade}
    \definition{v.}{manter estável; estabilizar; liquidar; resolver a situação}
  \end{phonetics}
\end{entry}

\begin{entry}{端午节}{14,4,5}{⽴、⼗、⾋}
  \begin{phonetics}{端午节}{duan1wu3jie2}
    \definition*{s.}{Festa do Duplo Cinco, Festival dos Barcos-Dragão (5º~dia do quinto mês lunar)}
  \end{phonetics}
\end{entry}

\begin{entry}{算}{14}{⽵}
  \begin{phonetics}{算}{suan4}[][HSK 2]
    \definition{adv.}{finalmente; por fim; no final; significa que, após um longo período de tempo ou muitas dificuldades, finalmente se alcançou o objetivo, equivalente a 总算}
    \definition{v.}{calcular; estimar; computar | contar; incluir | planejar; calcular; projetar | pensar; supor; especular | considerar; considerar como; contar como; reconhecer como | (aritmética) contar; ter peso | deixe estar; deixe passar; seguido por 了: desistir, não se importar mais}
  \seealsoref{了}{le5}
  \seealsoref{总算}{zong3suan4}
  \end{phonetics}
\end{entry}

\begin{entry}{算了}{14,2}{⽵、⼅}
  \begin{phonetics}{算了}{suan4le5}
    \definition{v.}{deixar | deixe estar | deixe passar | esqueça isso}
  \end{phonetics}
\end{entry}

\begin{entry}{算命}{14,8}{⽵、⼝}
  \begin{phonetics}{算命}{suan4ming4}
    \definition{s.}{cartomante}
    \definition{v.}{ler a sorte | fazer advinhações}
  \end{phonetics}
\end{entry}

\begin{entry}{管}{14}{⽵}
  \begin{phonetics}{管}{guan3}[][HSK 3]
    \definition*{s.}{sobrenome Guan}
    \definition{adj.}{estreito; restrito; limitado; pequeno}
    \definition{clas.}{para objetos cilíndricos finos}
    \definition{conj.}{não importa (o que, como, etc.)}
    \definition{prep.}{a função é semelhante a 把, usada especificamente em conjunto com 叫}
    \definition[根,条,排]{s.}{cano; tubo | instrumento musical de sopro | válvula; tubo | duto; canal; vasos}
    \definition{v.}{estar encarregado de; gerenciar; executar; supervisionar | administrar; governar | sujeitar alguém à disciplina | assumir; arcar | interferir; incomodar | garantir; assegurar; fornecer}
  \end{phonetics}
\end{entry}

\begin{entry}{管家}{14,10}{⽵、⼧}
  \begin{phonetics}{管家}{guan3jia1}
    \definition{s.}{mordomo | governanta}
    \definition{v.}{administrar uma casa}
  \end{phonetics}
\end{entry}

\begin{entry}{管理}{14,11}{⽵、⽟}
  \begin{phonetics}{管理}{guan3li3}[][HSK 3]
    \definition{v.}{gerenciar; executar; administrar; governar; estar encarregado de | controlar; gerenciar | cuidar de}
  \end{phonetics}
\end{entry}

\begin{entry}{精力}{14,2}{⽶、⼒}
  \begin{phonetics}{精力}{jing1li4}[][HSK 4]
    \definition[些]{s.}{energia; vigor; força mental e física}
  \end{phonetics}
\end{entry}

\begin{entry}{精灵}{14,7}{⽶、⽕}
  \begin{phonetics}{精灵}{jing1ling2}
    \definition{s.}{espírito | fada | elfo | duende | gênio}
  \end{phonetics}
\end{entry}

\begin{entry}{精品}{14,9}{⽶、⼝}
  \begin{phonetics}{精品}{jing1pin3}
    \definition{s.}{produtos de qualidade | produto premium | bom trabalho (de arte)}
  \end{phonetics}
\end{entry}

\begin{entry}{精神}{14,9}{⽶、⽰}
  \begin{phonetics}{精神}{jing1shen2}[][HSK 3]
    \definition[个]{s.}{espírito; mente; estado mental | substância; espírito; essência}
  \end{phonetics}
  \begin{phonetics}{精神}{jing1shen5}[][HSK 3]
    \definition{adj.}{animado; espirituoso; vigoroso
bonito}
    \definition[种,个,类,股]{s.}{impulso; vigor; vitalidade}
  \end{phonetics}
\end{entry}

\begin{entry}{精致}{14,10}{⽶、⾄}
  \begin{phonetics}{精致}{jing1zhi4}
    \definition{adj.}{delicado | exótico | refinado}
  \end{phonetics}
\end{entry}

\begin{entry}{精彩}{14,11}{⽶、⼺}
  \begin{phonetics}{精彩}{jing1cai3}[][HSK 3]
    \definition{adj.}{brilhante; esplêndido; maravilhoso}
  \end{phonetics}
\end{entry}

\begin{entry}{缩小}{14,3}{⽷、⼩}
  \begin{phonetics}{缩小}{suo1 xiao3}[][HSK 4]
    \definition{v.}{reduzir, estreitar, encolher;  tornar menor (em oposição a 放大)}
  \seealsoref{放大}{fang4da4}
  \end{phonetics}
\end{entry}

\begin{entry}{缩短}{14,12}{⽷、⽮}
  \begin{phonetics}{缩短}{suo1duan3}[][HSK 4]
    \definition{v.}{encurtar; reduzir; diminuir}
  \end{phonetics}
\end{entry}

\begin{entry}{缩影卡片}{14,15,5,4}{⽷、⼺、⼘、⽚}
  \begin{phonetics}{缩影卡片}{suo1ying3 ka3pian4}
    \definition{s.}{cartão em miniatura}
  \end{phonetics}
\end{entry}

\begin{entry}{聚}{14}{⽿}
  \begin{phonetics}{聚}{ju4}[][HSK 4]
    \definition{v.}{reunir-se; juntar-se}
  \end{phonetics}
\end{entry}

\begin{entry}{聚会}{14,6}{⽿、⼈}
  \begin{phonetics}{聚会}{ju4hui4}[][HSK 4]
    \definition[个,次]{s.}{reunião; encontro; confraternização; festa}
    \definition{v.}{encontrar-se; reunir-se}
  \end{phonetics}
\end{entry}

\begin{entry}{聚散}{14,12}{⽿、⽁}
  \begin{phonetics}{聚散}{ju4san4}
    \definition{s.}{juntos e separados | agregação e dissipação}
  \end{phonetics}
\end{entry}

\begin{entry}{膜拜}{14,9}{⾁、⼿}
  \begin{phonetics}{膜拜}{mo2bai4}
    \definition{v.}{ajoelhar-se e se curvar com as mãos unidas no nível da testa | ter ou mostrar sentimentos fortes de respeito e admiração por um deus}
  \end{phonetics}
\end{entry}

\begin{entry}{舞}{14}{⾇}
  \begin{phonetics}{舞}{wu3}[][HSK 5]
    \definition{s.}{dança | palco; metáfora do domínio das atividades sociais}
    \definition{v.}{mover-se como numa dança | dançar com algo nas mãos; brincar com | florescer; empunhar; brandir | esvoaçar | fazer malabarismos; brincar com}
  \end{phonetics}
\end{entry}

\begin{entry}{舞厅}{14,4}{⾇、⼚}
  \begin{phonetics}{舞厅}{wu3ting1}
    \definition[间]{s.}{salão de dança | salão de baile}
  \end{phonetics}
\end{entry}

\begin{entry}{舞厅舞}{14,4,14}{⾇、⼚、⾇}
  \begin{phonetics}{舞厅舞}{wu3ting1wu3}
    \definition{s.}{dança de salão}
  \end{phonetics}
\end{entry}

\begin{entry}{舞台}{14,5}{⾇、⼝}
  \begin{phonetics}{舞台}{wu3 tai2}[][HSK 3]
    \definition[个]{s.}{palco; arena}
  \end{phonetics}
\end{entry}

\begin{entry}{舞会}{14,6}{⾇、⼈}
  \begin{phonetics}{舞会}{wu3hui4}
    \definition{s.}{baile}
  \end{phonetics}
\end{entry}

\begin{entry}{舞会舞}{14,6,14}{⾇、⼈、⾇}
  \begin{phonetics}{舞会舞}{wu3hui4wu3}
    \definition{s.}{baile}
  \end{phonetics}
\end{entry}

\begin{entry}{舞抃}{14,7}{⾇、⼿}
  \begin{phonetics}{舞抃}{wu3bian4}
    \definition{s.}{dançar por prazer}
  \end{phonetics}
\end{entry}

\begin{entry}{舞蹈}{14,17}{⾇、⾜}
  \begin{phonetics}{舞蹈}{wu3dao3}
    \definition{s.}{dança (ato performático)}
  \end{phonetics}
\end{entry}

\begin{entry}{蔓草}{14,9}{⾋、⾋}
  \begin{phonetics}{蔓草}{man4cao3}
    \definition{s.}{videira | trepadeira}
  \end{phonetics}
\end{entry}

\begin{entry}{蜘蛛}{14,12}{⾍、⾍}
  \begin{phonetics}{蜘蛛}{zhi1zhu1}
    \definition{s.}{aranha}
  \end{phonetics}
\end{entry}

\begin{entry}{蜘蛛网}{14,12,6}{⾍、⾍、⽹}
  \begin{phonetics}{蜘蛛网}{zhi1zhu1wang3}
    \definition{s.}{teia de aranha}
  \end{phonetics}
\end{entry}

\begin{entry}{蜜桃}{14,10}{⾍、⽊}
  \begin{phonetics}{蜜桃}{mi4tao2}
    \definition{s.}{pêssego suculento}
  \end{phonetics}
\end{entry}

\begin{entry}{蜡烛}{14,10}{⾍、⽕}
  \begin{phonetics}{蜡烛}{la4zhu2}
    \definition[根,支]{s.}{vela | círio | peça, geralmente de cera, que possui um pavio e se utiliza para iluminar}
  \end{phonetics}
\end{entry}

\begin{entry}{蜥易}{14,8}{⾍、⽇}
  \begin{phonetics}{蜥易}{xi1yi4}
    \variantof{蜥蜴}
  \end{phonetics}
\end{entry}

\begin{entry}{蜥蜴}{14,14}{⾍、⾍}
  \begin{phonetics}{蜥蜴}{xi1yi4}
    \definition{s.}{lagarto}
  \end{phonetics}
\end{entry}

\begin{entry}{蜻蜓}{14,12}{⾍、⾍}
  \begin{phonetics}{蜻蜓}{qing1ting2}
    \definition{s.}{libélula}
  \end{phonetics}
\end{entry}

\begin{entry}{蜻蝏}{14,15}{⾍、⾍}
  \begin{phonetics}{蜻蝏}{qing1ting2}
    \variantof{蜻蜓}
  \end{phonetics}
\end{entry}

\begin{entry}{蝉}{14}{⾍}
  \begin{phonetics}{蝉}{chan2}
    \definition{s.}{cigarra}
  \end{phonetics}
\end{entry}

\begin{entry}{褐色}{14,6}{⾐、⾊}
  \begin{phonetics}{褐色}{he4 se4}
    \definition{s.}{cor marrom}
  \end{phonetics}
\end{entry}

\begin{entry}{豪华}{14,6}{⾗、⼗}
  \begin{phonetics}{豪华}{hao2hua2}
    \definition{adj.}{luxuoso}
  \end{phonetics}
\end{entry}

\begin{entry}{赛}{14}{⾙}
  \begin{phonetics}{赛}{sai4}
    \definition{s.}{competição}
    \definition{v.}{competir | superar | destacar-se}
  \end{phonetics}
\end{entry}

\begin{entry}{赛车}{14,4}{⾙、⾞}
  \begin{phonetics}{赛车}{sai4che1}
    \definition{s.}{corrida de automóvel | corrida de bicicleta | carro de corrida}
  \end{phonetics}
\end{entry}

\begin{entry}{辣}{14}{⾟}
  \begin{phonetics}{辣}{la4}[][HSK 4]
    \definition{adj.}{apimentado; picante; pungente; quente | cruel; implacável; venenoso; vicioso}
    \definition{v.}{queimar; picar; formigar; ter uma irritação picante (boca, nariz ou olhos)}
  \end{phonetics}
\end{entry}

\begin{entry}{遭到}{14,8}{⾡、⼑}
  \begin{phonetics}{遭到}{zao1dao4}
    \definition{v.}{sofrer | encontrar-se com (algo infeliz)}
  \end{phonetics}
\end{entry}

\begin{entry}{遭受}{14,8}{⾡、⼜}
  \begin{phonetics}{遭受}{zao1shou4}
    \definition{v.}{sofrer | suportar (perda, infornúnio)}
  \end{phonetics}
\end{entry}

\begin{entry}{遭遇}{14,12}{⾡、⾡}
  \begin{phonetics}{遭遇}{zao1yu4}
    \definition{s.}{experiência (amarga)}
    \definition{v.}{encontrar-se com (algo infeliz)}
  \end{phonetics}
\end{entry}

\begin{entry}{酷}{14}{⾣}
  \begin{phonetics}{酷}{ku4}
    \definition{adj.}{impiedoso | forte (por exemplo, vinho) | (empréstimo linguístico) legal, \emph{cool}}
  \end{phonetics}
\end{entry}

\begin{entry}{酷斯拉}{14,12,8}{⾣、⽄、⼿}
  \begin{phonetics}{酷斯拉}{ku4si1la1}
    \definition*{s.}{Godzilla (Japonês ゴジラ Gojira)}
  \seealsoref{哥斯拉}{ge1si1la1}
  \end{phonetics}
\end{entry}

\begin{entry}{酸}{14}{⾣}
  \begin{phonetics}{酸}{suan1}[][HSK 4]
    \definition{adj.}{azedo; ácido | aflito; angustiado; doente do coração | pedante; descreve uma pessoa que finge ser culta e também descreve uma pessoa que é muito inflexível com suas próprias ideias e não está disposta a mudá-las para atender às exigências da época, é usado principalmente para satirizar intelectuais que fingem ser capazes de escrever poemas e artigos | ciumento; invejoso; sentimentos desconfortáveis porque outra pessoa é melhor do que você e, em geral, também apresenta comportamento hostil}
    \definition{s.}{ácido; produto químico que tem um sabor ácido quando misturado com água}
    \definition{v.}{estar dolorido (devido à fadiga ou doença); descreve a sensação de não ter força muscular e um pouco de dor por estar doente ou muito cansado}
  \end{phonetics}
\end{entry}

\begin{entry}{酸奶}{14,5}{⾣、⼥}
  \begin{phonetics}{酸奶}{suan1 nai3}[][HSK 4]
    \definition[瓶,杯,盒,袋]{s.}{iogurte; produto lácteo fermentado por bactérias de ácido láctico}
  \end{phonetics}
\end{entry}

\begin{entry}{酸甜苦辣}{14,11,8,14}{⾣、⽢、⾋、⾟}
  \begin{phonetics}{酸甜苦辣}{suan1 tian2 ku3 la4}[][HSK 5]
    \definition{expr.}{os altos e baixos da vida; as experiências agridoces da vida; os aspectos doces, azedos, amargos e picantes da vida; refere-se a todos os tipos de sabores, como metáfora para experiências diversas, como felicidade, sofrimento, etc. | azedo, doce, amargo, picante — alegrias e tristezas da vida}
  \end{phonetics}
\end{entry}

\begin{entry}{酸辣汤}{14,14,6}{⾣、⾟、⽔}
  \begin{phonetics}{酸辣汤}{suan1la4tang1}
    \definition{s.}{sopa avinagrada e picante (prato)}
  \end{phonetics}
\end{entry}

\begin{entry}{锺}{14}{⾦}
  \begin{phonetics}{锺}{zhong1}
    \variantof{钟}
  \end{phonetics}
\end{entry}

\begin{entry}{锻炼}{14,9}{⾦、⽕}
  \begin{phonetics}{锻炼}{duan4lian4}[][HSK 4]
    \definition{v.}{exercitar-se; fazer (ou fazer) exercícios; submeter-se a treinamento físico; fortalecer o corpo por meio do esporte | fortalecer; endurecer; aprimorar as habilidades de trabalho e de vida por meio de trabalho e outras atividades | forjar ou moldar metal para torná-lo mais refinado; refere-se à transformação de materiais metálicos em objetos de determinada forma e tamanho por meio de aquecimento, batimento, prensagem etc.}
  \end{phonetics}
\end{entry}

\begin{entry}{镀金}{14,8}{⾦、⾦}
  \begin{phonetics}{镀金}{du4jin1}
    \definition{v.}{banhar a ouro | dourar | (figurativo) fazer algo muito comum parecer especial}
  \end{phonetics}
\end{entry}

\begin{entry}{隧道}{14,12}{⾩、⾡}
  \begin{phonetics}{隧道}{sui4dao4}
    \definition{s.}{túnel}
  \end{phonetics}
\end{entry}

\begin{entry}{需求}{14,7}{⾬、⽔}
  \begin{phonetics}{需求}{xu1qiu2}[][HSK 3]
    \definition{s.}{necessidades; demanda; requisito; requerimento; exigência | solicitações decorrentes de necessidades}
  \end{phonetics}
\end{entry}

\begin{entry}{需要}{14,9}{⾬、⾑}
  \begin{phonetics}{需要}{xu1yao4}[][HSK 3]
    \definition{s.}{necessidade | desejo ou solicitação de algo}
    \definition{v.}{precisar; querer; requerer; demandar}
  \end{phonetics}
\end{entry}

\begin{entry}{静}{14}{⾭}
  \begin{phonetics}{静}{jing4}[][HSK 3]
    \definition*{s.}{sobrenome Jing}
    \definition{adj.}{tranquilo;  sossegado; calmo; imóvel | silencioso; quieto}
  \end{phonetics}
\end{entry}

\begin{entry}{颗}{14}{⾴}
  \begin{phonetics}{颗}{ke1}[][HSK 5]
    \definition{clas.}{para grãos, pérolas, dentes, corações, satelites, pequenas esferas, etc.}
    \definition{s.}{grão; partícula; pequenas coisas redondas}
  \end{phonetics}
\end{entry}

\begin{entry}{馒头}{14,5}{⾷、⼤}
  \begin{phonetics}{馒头}{man2tou5}
    \definition{s.}{pão cozido no vapor}
  \end{phonetics}
\end{entry}

\begin{entry}{魅力}{14,2}{⿁、⼒}
  \begin{phonetics}{魅力}{mei4li4}
    \definition{s.}{charme | fascínio | glamour | carisma}
  \end{phonetics}
\end{entry}

\begin{entry}{鲜}{14}{⿂}
  \begin{phonetics}{鲜}{xian1}[][HSK 4]
    \definition*{s.}{sobrenome Xian}
    \definition{adj.}{fresco; novo; fresco (experiência, comida etc.) |brilhante; de cores vivas | saboroso; delicioso | exuberante; luxuriante}
    \definition{s.}{aves e animais recém-abatidos; vegetais recém-colhidos; frutas, etc. | alimentos aquáticos; geralmente, peixes vivos, camarões, etc., para alimentação}
  \end{phonetics}
  \begin{phonetics}{鲜}{xian3}
    \definition{adj.}{raro; pouco; pequeno;}
    \definition{adv.}{raramente}
  \end{phonetics}
\end{entry}

\begin{entry}{鲜花}{14,7}{⿂、⾋}
  \begin{phonetics}{鲜花}{xian1 hua1}[][HSK 4]
    \definition[朵,束,支,捧]{s.}{flor; flores frescas; flores bonitas e frescas}
  \end{phonetics}
\end{entry}

\begin{entry}{鲜明}{14,8}{⿂、⽇}
  \begin{phonetics}{鲜明}{xian1ming2}[][HSK 4]
    \definition{adj.}{brilhante (cor) | distinto; bem definido; nítido; claro; característico}
  \end{phonetics}
\end{entry}

\begin{entry}{鲜艳}{14,10}{⿂、⾊}
  \begin{phonetics}{鲜艳}{xian1yan4}[][HSK 5]
    \definition{adj.}{de cores alegres; de cores brilhantes}
  \end{phonetics}
\end{entry}

\begin{entry}{鼻子}{14,3}{⿐、⼦}
  \begin{phonetics}{鼻子}{bi2zi5}[][HSK 5]
    \definition[个,只]{s.}{nariz; órgão da face, responsável pela respiração e pelo olfato}
  \end{phonetics}
\end{entry}

%%%%% EOF %%%%%


%%%
%%% 15画
%%%

\section*{15画}\addcontentsline{toc}{section}{15画}

\begin{Entry}{嘱}{15}{⼝}
  \begin{Phonetics}{嘱}{zhu3}
    \definition{v.}{juntar-se | implorar | incitar}
  \end{Phonetics}
\end{Entry}

\begin{Entry}{嘱托}{15,6}{⼝、⼿}
  \begin{Phonetics}{嘱托}{zhu3tuo1}
    \definition{v.}{confiar uma tarefa a alguém}
  \end{Phonetics}
\end{Entry}

\begin{Entry}{嘱咐}{15,8}{⼝、⼝}
  \begin{Phonetics}{嘱咐}{zhu3fu5}
    \definition{v.}{ordenar | dizer | exortar}
  \end{Phonetics}
\end{Entry}

\begin{Entry}{嘲}{15}{⼝}
  \begin{Phonetics}{嘲}{chao2}
    \definition{v.}{ridicularizar; zombar; fazer piada de}
  \end{Phonetics}
  \begin{Phonetics}{嘲}{zhao1}
    \definition{s.}{Onomatopéia: barulho clamoroso feito por várias pessoas falando ou cantando, ou por instrumentos musicais, ou pássaros cantando; descreve um som caótico e fragmentado}
  \end{Phonetics}
\end{Entry}

\begin{Entry}{嘲弄}{15,7}{⼝、⼶}
  \begin{Phonetics}{嘲弄}{chao2nong4}[][HSK 7-9]
    \definition{v.}{zombar; zombar de}
  \end{Phonetics}
\end{Entry}

\begin{Entry}{嘲笑}{15,10}{⼝、⽵}
  \begin{Phonetics}{嘲笑}{chao2xiao4}[][HSK 7-9]
    \definition{v.}{ridicularizar; zombar; rir de; zombar de; fazer graça de; usar palavras para zombar de alguém}
  \end{Phonetics}
\end{Entry}

\begin{Entry}{嘹}{15}{⼝}
  \begin{Phonetics}{嘹}{liao2}
    \definition{adj.}{(som) alto e claro | som claro | grito (de guindastes, etc.)}
  \end{Phonetics}
\end{Entry}

\begin{Entry}{嘹亮}{15,9}{⼝、⼇}
  \begin{Phonetics}{嘹亮}{liao2liang4}
    \definition{adj.}{ressonante; alto e claro}
  \end{Phonetics}
\end{Entry}

\begin{Entry}{嘿}{15}{⼝}
  \begin{Phonetics}{嘿}{hei1}[][HSK 7-9]
    \definition{interj.}{Ei!; indicando uma saudação ou chamar a atenção | expressando orgulho ou satisfação | expressando espanto, surpresa}
  \end{Phonetics}
  \begin{Phonetics}{嘿}{mo4}
    \definition{adj.}{quieto; silencioso; tácito}
  \end{Phonetics}
\end{Entry}

\begin{Entry}{噎}{15}{⼝}
  \begin{Phonetics}{噎}{ye1}
    \definition{v.}{engasgar | sufocar}
  \end{Phonetics}
\end{Entry}

\begin{Entry}{增}{15}{⼟}
  \begin{Phonetics}{增}{zeng1}[][HSK 5]
    \definition*{s.}{Sobrenome Zeng}
    \definition{v.}{aumentar; ganhar; adicionar}
  \end{Phonetics}
\end{Entry}

\begin{Entry}{增大}{15,3}{⼟、⼤}
  \begin{Phonetics}{增大}{zeng1 da4}[][HSK 5]
    \definition{v.}{ampliar; expandir; estender | amplificar}
  \end{Phonetics}
\end{Entry}

\begin{Entry}{增长}{15,4}{⼟、⾧}
  \begin{Phonetics}{增长}{zeng1 zhang3}[][HSK 3]
    \definition{v.}{subir; crescer; aumentar; melhorar a partir da base existente}
  \end{Phonetics}
\end{Entry}

\begin{Entry}{增加}{15,5}{⼟、⼒}
  \begin{Phonetics}{增加}{zeng1jia1}[][HSK 3]
    \definition{v.}{adicionar; aumentar; incrementar; adicionar mais ao que já existe}
  \end{Phonetics}
\end{Entry}

\begin{Entry}{增产}{15,6}{⼟、⼇}
  \begin{Phonetics}{增产}{zeng1/chan3}[][HSK 5]
    \definition{v.+compl.}{aumentar a produção}
  \end{Phonetics}
\end{Entry}

\begin{Entry}{增多}{15,6}{⼟、⼣}
  \begin{Phonetics}{增多}{zeng1 duo1}[][HSK 5]
    \definition{v.}{aumentar; crescer em número ou quantidade}
  \end{Phonetics}
\end{Entry}

\begin{Entry}{增进}{15,7}{⼟、⾡}
  \begin{Phonetics}{增进}{zeng1 jin4}[][HSK 6]
    \definition{v.}{melhorar; promover; aprofundar}
  \end{Phonetics}
\end{Entry}

\begin{Entry}{增值}{15,10}{⼟、⼈}
  \begin{Phonetics}{增值}{zeng1 zhi2}[][HSK 6]
    \definition{s.}{aumento de valor; apreciação; incremento | valor agregado}
  \end{Phonetics}
\end{Entry}

\begin{Entry}{增速}{15,10}{⼟、⾡}
  \begin{Phonetics}{增速}{zeng1su4}
    \definition{s.}{(economia) taxa de crescimento}
    \definition{v.}{acelerar;}
  \end{Phonetics}
\end{Entry}

\begin{Entry}{增强}{15,12}{⼟、⼸}
  \begin{Phonetics}{增强}{zeng1 qiang2}[][HSK 5]
    \definition{v.}{impulsionar; aprimorar; aumentar; fortalecer; tornar mais forte ou mais poderoso}
  \end{Phonetics}
\end{Entry}

\begin{Entry}{墨}{15}{⿊}
  \begin{Phonetics}{墨}{mo4}
    \definition*{s.}{Escola Moísta; Moísmo | México, abreviação de 墨西哥}
    \definition{adj.}{preto; escuro como breu | corrupto | escuro}
    \definition{s.}{tinta chinesa; bastão de tinta | pigmento; tinta | caligrafia ou pintura | aprendizagem; alfabetização | marcador de linha de carpinteiro; marcador de tinta | tatuar o rosto (um castigo); uma punição na China antiga | corrupção; peculato; fraude}
  \seealsoref{墨西哥}{mo4xi1ge1}
  \end{Phonetics}
\end{Entry}

\begin{Entry}{墨水}{15,4}{⿊、⽔}
  \begin{Phonetics}{墨水}{mo4 shui3}[][HSK 6]
    \definition[瓶]{s.}{tinta chinesa preparada; tinta (para caneta-tinteiro) | aprendizagem; alfabetização; uma metáfora para o conhecimento ou a capacidade de ler e escrever}
  \end{Phonetics}
\end{Entry}

\begin{Entry}{墨西哥}{15,6,10}{⿊、⾑、⼝}
  \begin{Phonetics}{墨西哥}{mo4xi1ge1}
    \definition*{s.}{México; Planalto no México}
  \end{Phonetics}
\end{Entry}

\begin{Entry}{墨镜}{15,16}{⿊、⾦}
  \begin{Phonetics}{墨镜}{mo4jing4}
    \definition[只,双,副]{s.}{óculos escuros}
  \end{Phonetics}
\end{Entry}

\begin{Entry}{影}{15}{⼺}
  \begin{Phonetics}{影}{ying3}
    \definition*{s.}{Sobrenome Ying}
    \definition{s.}{sombra | reflexão; imagem | traço; sinal; impressão vaga | fotografia; imagem | filme | jogo de sombras; pantomima de sombra}
    \definition{v.}{(dialeto) esconder; ocultar | copiar; rastrear | fotocopiar}
  \end{Phonetics}
\end{Entry}

\begin{Entry}{影子}{15,3}{⼺、⼦}
  \begin{Phonetics}{影子}{ying3zi5}[][HSK 4]
    \definition[个,片]{s.}{sombra; imagem projetada por um objeto, etc., que bloqueia a luz | reflexão; reflexo; imagem de um objeto, etc., conforme aparece em um refletor, como um espelho, uma superfície de água, etc. | sinal; vestígio; vaga impressão}
  \end{Phonetics}
\end{Entry}

\begin{Entry}{影片}{15,4}{⼺、⽚}
  \begin{Phonetics}{影片}{ying3 pian4}[][HSK 2]
    \definition[部,盘,盒,卷]{s.}{filme; imagem | filme; película usada para reproduzir filmes}
  \end{Phonetics}
\end{Entry}

\begin{Entry}{影视}{15,8}{⼺、⾒}
  \begin{Phonetics}{影视}{ying3 shi4}[][HSK 3]
    \definition{s.}{cinema e televisão combinados; denominação conjunta para cinema e TV}
  \end{Phonetics}
\end{Entry}

\begin{Entry}{影响}{15,9}{⼺、⼝}
  \begin{Phonetics}{影响}{ying3xiang3}[][HSK 2]
    \definition{s.}{efeito; influência; efeitos sobre pessoas ou coisas}
    \definition{v.}{afetar; influenciar; influência sobre os pensamentos ou ações dos outros}
  \end{Phonetics}
\end{Entry}

\begin{Entry}{影响力}{15,9,2}{⼺、⼝、⼒}
  \begin{Phonetics}{影响力}{ying3 xiang3 li4}[][HSK 6]
    \definition{s.}{impacto | influência}
  \end{Phonetics}
\end{Entry}

\begin{Entry}{影星}{15,9}{⼺、⽇}
  \begin{Phonetics}{影星}{ying3 xing1}[][HSK 6]
    \definition{s.}{estrela de cinema}
  \end{Phonetics}
\end{Entry}

\begin{Entry}{影迷}{15,9}{⼺、⾡}
  \begin{Phonetics}{影迷}{ying3 mi2}[][HSK 6]
    \definition[个,名,位]{s.}{fã de cinema; entusiasta de cinema; pessoas viciadas em assistir filmes}
  \end{Phonetics}
\end{Entry}

\begin{Entry}{影像}{15,13}{⼺、⼈}
  \begin{Phonetics}{影像}{ying3xiang4}
    \definition{s.}{imagem}
  \end{Phonetics}
\end{Entry}

\begin{Entry}{德}{15}{⼻}
  \begin{Phonetics}{德}{de2}[][HSK 7-9]
    \definition*{s.}{Alemanha, abreviação de 德国 | Sobrenome De}
    \definition{s.}{virtude; moral; caráter moral; moralidade; conduta; qualidades políticas | coração; mente; pensamentos | bondade; favor; graça}
  \seealsoref{德国}{de2guo2}
  \end{Phonetics}
\end{Entry}

\begin{Entry}{德国}{15,8}{⼻、⼞}
  \begin{Phonetics}{德国}{de2guo2}
    \definition*{s.}{Alemanha}
  \end{Phonetics}
\end{Entry}

\begin{Entry}{德国人}{15,8,2}{⼻、⼞、⼈}
  \begin{Phonetics}{德国人}{de2guo2ren2}
    \definition{s.}{alemão | pessoa ou povo da Alemanha}
  \end{Phonetics}
\end{Entry}

\begin{Entry}{慰}{15}{⼼}
  \begin{Phonetics}{慰}{wei4}
    \definition{adj.}{aliviado; em paz; confortável}
    \definition{v.}{consolar; confortar | ser (ficar) aliviado}
  \end{Phonetics}
\end{Entry}

\begin{Entry}{慰问}{15,6}{⼼、⾨}
  \begin{Phonetics}{慰问}{wei4wen4}[][HSK 5]
    \definition{v.}{visitar; consolar; expressar simpatia por; confortar e cumprimentar com palavras e presentes;  enfatizar o conforto e o cumprimento, frequentemente usado por superiores para subordinados}
  \end{Phonetics}
\end{Entry}

\begin{Entry}{憋}{15}{⼼}
  \begin{Phonetics}{憋}{bie1}[][HSK 7-9]
    \definition{adj.}{sufocado; oprimido}
    \definition{v.}{suprimir; conter | Dialeto: obrigar | Dialeto: ponderar; contemplar | Dialeto: ficar de olho em | Dialeto: destruir (por pressão interna) | calar a boca; inibir; bloquear | sufocar; abafar}
  \end{Phonetics}
\end{Entry}

\begin{Entry}{憧}{15}{⼼}
  \begin{Phonetics}{憧}{chong1}
    \definition{adj.}{irresoluto; indeciso | estúpido; imbecil; confuso}
  \end{Phonetics}
\end{Entry}

\begin{Entry}{憧憬}{15,15}{⼼、⼼}
  \begin{Phonetics}{憧憬}{chong1jing3}
    \definition{v.}{ansiar por | esperar por}
  \end{Phonetics}
\end{Entry}

\begin{Entry}{懂}{15}{⼼}
  \begin{Phonetics}{懂}{dong3}[][HSK 2]
    \definition*{s.}{Sobrenome Dong}
    \definition{v.}{compreender; entender}
  \end{Phonetics}
\end{Entry}

\begin{Entry}{懂事}{15,8}{⼼、⼅}
  \begin{Phonetics}{懂事}{dong3shi4}[][HSK 7-9]
    \definition{adj.}{sensato; inteligente; muito compreensivo da natureza e da razão humana}
  \end{Phonetics}
\end{Entry}

\begin{Entry}{懂得}{15,11}{⼼、⼻}
  \begin{Phonetics}{懂得}{dong3 de5}[][HSK 2]
    \definition{v.}{saber (significado, prática, etc.); compreender; entender}
  \end{Phonetics}
\end{Entry}

\begin{Entry}{摩}{15}{⼿}
  \begin{Phonetics}{摩}{mo2}
    \definition{v.}{esfregar; raspar; tocar | refletir; estudar | afagar}
  \end{Phonetics}
\end{Entry}

\begin{Entry}{摩托}{15,6}{⼿、⼿}
  \begin{Phonetics}{摩托}{mo2 tuo1}[][HSK 5]
    \definition[辆]{s.}{Empréstimo linguístico: motor; motor de combustão interna | Empréstimo linguístico: motocicleta, abreviação de 摩托车}
  \seealsoref{摩托车}{mo2tuo1che1}
  \end{Phonetics}
\end{Entry}

\begin{Entry}{摩托车}{15,6,4}{⼿、⼿、⾞}
  \begin{Phonetics}{摩托车}{mo2tuo1che1}
    \definition[辆,部]{s.}{(empréstimo linguístico) motocicleta}
  \end{Phonetics}
\end{Entry}

\begin{Entry}{摩擦}{15,17}{⼿、⼿}
  \begin{Phonetics}{摩擦}{mo2ca1}[][HSK 5]
    \definition{s.}{atrito; desacordo; conflito (entre duas partes); a ação de impedir o movimento relativo entre dois objetos em contato, produzida na superfície de contato | atrito; metáfora para o conflito entre as duas partes}
    \definition{v.}{esfregar}
  \end{Phonetics}
\end{Entry}

\begin{Entry}{撑}{15}{⼿}
  \begin{Phonetics}{撑}{cheng1}[][HSK 6]
    \definition{s.}{suporte; escora;  apoio; esteio}
    \definition{v.}{sustentar; apoiar; resistir a | empurrar (ou mover) com uma vara; usar um mastro para empurrar a margem ou o leito do rio para fazer o barco avançar | manter; manter-se atualizado | abrir; desdobrar; expandir (um objeto contraído) | encher até estourar (inchaço devido a excesso de comida ou alimentação excessiva)}
  \end{Phonetics}
\end{Entry}

\begin{Entry}{撒}{15}{⼿}
  \begin{Phonetics}{撒}{sa1}
    \definition{v.}{lançar; deixar ir; deixar sair; liberar | livrar-se de todas as restrições; deixar-se levar; tentar usá-lo ou exibi-lo o máximo possível}
  \end{Phonetics}
\end{Entry}

\begin{Entry}{撒旦}{15,5}{⼿、⽇}
  \begin{Phonetics}{撒旦}{sa1dan4}
    \definition*{s.}{Satã}
  \end{Phonetics}
\end{Entry}

\begin{Entry}{撒旦主义}{15,5,5,3}{⼿、⽇、⼂、⼂}
  \begin{Phonetics}{撒旦主义}{sa1dan4 zhu3yi4}
    \definition*{s.}{Satanismo}
  \end{Phonetics}
\end{Entry}

\begin{Entry}{撒但}{15,7}{⼿、⼈}
  \begin{Phonetics}{撒但}{sa1dan4}
    \variantof{撒旦}
  \end{Phonetics}
\end{Entry}

\begin{Entry}{撞}{15}{⼿}
  \begin{Phonetics}{撞}{zhuang4}[][HSK 5]
    \definition{v.}{chocar-se contra; chocar-se com; bater; colidir | encontrar-se por acaso; esbarrar em; deparar-se com | apressar; correr; empurrar | aproveitar a chance | esbarrar de repente em |  encontrar | confiar em; tentar | agir precipitadamente; invadir}
  \end{Phonetics}
\end{Entry}

\begin{Entry}{撞车}{15,4}{⼿、⾞}
  \begin{Phonetics}{撞车}{zhuang4/che1}
    \definition{v.+compl.}{(figurativo) colidir (opiniões, cronogramas, etc.) | ser o mesmo (assunto) | colidir (com outro veículo)}
  \end{Phonetics}
\end{Entry}

\begin{Entry}{撞运气}{15,7,4}{⼿、⾡、⽓}
  \begin{Phonetics}{撞运气}{zhuang4yun4qi5}
    \definition{v.}{confiar no destino | tentar a sorte}
  \end{Phonetics}
\end{Entry}

\begin{Entry}{撤}{15}{⼿}
  \begin{Phonetics}{撤}{che4}[][HSK 7-9]
    \definition{v.}{remover, tirar | demitir; liberar | retirar-se; evacuar}
  \end{Phonetics}
\end{Entry}

\begin{Entry}{撤换}{15,10}{⼿、⼿}
  \begin{Phonetics}{撤换}{che4huan4}[][HSK 7-9]
    \definition{v.}{demitir e substituir (alguém); revogar; substituir (alguém ou alguma coisa)}
  \end{Phonetics}
\end{Entry}

\begin{Entry}{撤离}{15,10}{⼿、⼇}
  \begin{Phonetics}{撤离}{che4 li2}[][HSK 6]
    \definition{v.}{retirar-se de; deixar; evacuar}
  \end{Phonetics}
\end{Entry}

\begin{Entry}{撤销}{15,12}{⼿、⾦}
  \begin{Phonetics}{撤销}{che4xiao1}[][HSK 6]
    \definition{v.}{cancelar; rescindir; revogar; remover}
  \end{Phonetics}
\end{Entry}

\begin{Entry}{播}{15}{⼿}
  \begin{Phonetics}{播}{bo1}[][HSK 6]
    \definition{v.}{espalhar; transmitir | semear | mover-se; migrar; ir para o exílio}
  \end{Phonetics}
\end{Entry}

\begin{Entry}{播出}{15,5}{⼿、⼐}
  \begin{Phonetics}{播出}{bo1 chu1}[][HSK 3]
    \definition{v.}{radiodifundir; transmitir; estar no ar; transmitir via rádio e televisão}
  \end{Phonetics}
\end{Entry}

\begin{Entry}{播放}{15,8}{⼿、⽅}
  \begin{Phonetics}{播放}{bo1fang4}[][HSK 3]
    \definition{v.}{ir ao ar; transmitir por rádio | mostrar; exibir; transmitir (um programa de TV)}
  \end{Phonetics}
\end{Entry}

\begin{Entry}{播音}{15,9}{⼿、⾳}
  \begin{Phonetics}{播音}{bo1/yin1}
    \definition{s.}{transmissão}
    \definition{v.+compl.}{transmitir}
  \end{Phonetics}
\end{Entry}

\begin{Entry}{擒}{15}{⼿}
  \begin{Phonetics}{擒}{qin2}
    \definition{v.}{capturar; pegar; apreender}
  \end{Phonetics}
\end{Entry}

\begin{Entry}{擒获}{15,10}{⼿、⾋}
  \begin{Phonetics}{擒获}{qin2huo4}
    \definition{v.}{apreender | capturar}
  \end{Phonetics}
\end{Entry}

\begin{Entry}{敷}{15}{⽁}
  \begin{Phonetics}{敷}{fu1}[][HSK 7-9]
    \definition*{s.}{Sobrenome Fu}
    \definition{v.}{aplicar (pó, pomada, etc.) | espalhar; dispor | ser suficiente para | espalhar-se}
  \end{Phonetics}
\end{Entry}

\begin{Entry}{暴}{15}{⽇}
  \begin{Phonetics}{暴}{bao4}
    \definition*{s.}{Sobrenome Bao}
    \definition{adj.}{repentino e violento | cruel; selvagem; feroz | temperamental | severo e tirânico; brutal | irritável; irascível; impaciente}
    \definition{adv.}{de repente e ferozmente}
    \definition{s.}{violência; ferocidade}
    \definition{v.}{sobressair; destacar-se; inchar | expor; transmitir | desperdiçar; arruinar; estragar}
  \end{Phonetics}
\end{Entry}

\begin{Entry}{暴力}{15,2}{⽇、⼒}
  \begin{Phonetics}{暴力}{bao4li4}[][HSK 6]
    \definition{s.}{violência; força (usada em tempos de conflito); poder de coerção}
  \end{Phonetics}
\end{Entry}

\begin{Entry}{暴风雨}{15,4,8}{⽇、⾵、⾬}
  \begin{Phonetics}{暴风雨}{bao4 feng1 yu3}[][HSK 6]
    \definition{s.}{tempestade; tormenta; temporal; borrasca; vento e chuva fortes e violentos}
  \end{Phonetics}
\end{Entry}

\begin{Entry}{暴风骤雨}{15,4,17,8}{⽇、⾵、⾺、⾬}
  \begin{Phonetics}{暴风骤雨}{bao4feng1-zhou4yu3}[][HSK 7-9]
    \definition{expr.}{tempestade violenta; furacão; tempestade | vento violento e tempestade de chuva}
  \end{Phonetics}
\end{Entry}

\begin{Entry}{暴行}{15,6}{⽇、⾏}
  \begin{Phonetics}{暴行}{bao4xing2}
    \definition{s.}{ato selvagem | atrocidade | indignação}
  \end{Phonetics}
\end{Entry}

\begin{Entry}{暴乱}{15,7}{⽇、⼄}
  \begin{Phonetics}{暴乱}{bao4luan4}
    \definition{s.}{rebelião | revolta | tumulto}
  \end{Phonetics}
\end{Entry}

\begin{Entry}{暴利}{15,7}{⽇、⼑}
  \begin{Phonetics}{暴利}{bao4li4}[][HSK 7-9]
    \definition{s.}{lucros enormes repentinos | lucros exorbitantes; lucros extravagantes; lucros excessivos}
  \end{Phonetics}
\end{Entry}

\begin{Entry}{暴雨}{15,8}{⽇、⾬}
  \begin{Phonetics}{暴雨}{bao4yu3}[][HSK 6]
    \definition[场,次,阵]{s.}{tempestade; chuva torrencial; chuva forte com precipitação intensa; em meteorologia, refere-se a chuvas de 16 mm ou mais em uma hora ou 50 mm ou mais em 24 horas}
  \end{Phonetics}
\end{Entry}

\begin{Entry}{暴躁}{15,20}{⽇、⾜}
  \begin{Phonetics}{暴躁}{bao4zao4}[][HSK 7-9]
    \definition{adj.}{irascível; febril; irritável; temperamental; descreve uma pessoa que é impaciente, não consegue controlar suas emoções e fica com raiva facilmente}
  \end{Phonetics}
\end{Entry}

\begin{Entry}{暴露}{15,21}{⽇、⾬}
  \begin{Phonetics}{暴露}{bao4lu4}[][HSK 6]
    \definition{adj.}{reveladoras (roupas inadequadas que expõem muito o corpo)}
    \definition{v.}{expor; desnudar; revelar; tornar público algo oculto}
  \end{Phonetics}
\end{Entry}

\begin{Entry}{槽}{15}{⽊}
  \begin{Phonetics}{槽}{cao2}[][HSK 7-9]
    \definition{clas.}{usado para portas | usado para porcos}
    \definition[个,道]{s.}{cocho | sulco; entalhe | canal | manjedoura (para água, ração animal, vinho, cuba); um recipiente para alimentar o gado, geralmente é retangular, alto em todos os lados e côncavo no meio, como uma caixa sem tampa | tanque de fermentação; cuba de vinho; geralmente se refere a certos utensílios com lados altos e côncavos no meio | leito do rio; fossa; refere-se a certos cursos d'água ou valas com lados altos e um meio côncavo | ranhura; fenda; uma depressão semelhante a um sulco em um objeto}
  \end{Phonetics}
\end{Entry}

\begin{Entry}{横}{15}{⽊}
  \begin{Phonetics}{横}{heng2}[][HSK 6]
    \definition{adj.}{horizontal; transversal; paralelo ao plano horizontal (oposto de 竖 e 直) | em ângulo reto com; direção esquerda-direita (em oposição à 竖, 直 ou 纵) | e leste a oeste ou de oeste a leste; direção leste-oeste (oposta a 纵) | desenfreado; turbulento | violento; feroz; irracional}
    \definition{adv.}{de qualquer forma; em qualquer caso | provavelmente; muito provavelmente}
    \definition{s.}{traço horizontal (em caracteres chineses)}
    \definition{v.}{deitar-se transversalmente; estar de lado | colocar algo transversalmente (ou horizontalmente)}
  \seealsoref{竖}{shu4}
  \seealsoref{直}{zhi2}
  \seealsoref{纵}{zong4}
  \end{Phonetics}
  \begin{Phonetics}{横}{heng4}[][HSK 7-9]
    \definition{adj.}{chocante e irracional; inesperado}
  \end{Phonetics}
\end{Entry}

\begin{Entry}{横七竖八}{15,2,9,2}{⽊、⼀、⽴、⼋}
  \begin{Phonetics}{横七竖八}{heng2qi1-shu4ba1}[][HSK 7-9]
    \definition{expr.}{em desordem; em seis e sete; desorganizado}
  \end{Phonetics}
\end{Entry}

\begin{Entry}{横向}{15,6}{⽊、⼝}
  \begin{Phonetics}{横向}{heng2xiang4}[][HSK 7-9]
    \definition{adj.}{horizontal; transversal (oposto a 竖向,纵向) | lateral | ortogonal | perpendicular}
  \seealsoref{竖向}{shu4xiang4}
  \seealsoref{纵向}{zong4xiang4}
  \end{Phonetics}
\end{Entry}

\begin{Entry}{横竖}{15,9}{⽊、⽴}
  \begin{Phonetics}{横竖}{heng2shu5}
    \definition{adv.}{de qualquer forma; em qualquer maneira; isso significa que não importa o que aconteça, o resultado ou a conclusão não mudará; equivale a 反正}
  \seealsoref{反正}{fan3zheng4}
  \end{Phonetics}
\end{Entry}

\begin{Entry}{樱}{15}{⽊}
  \begin{Phonetics}{樱}{ying1}
    \definition[个,棵,朵]{s.}{cereja | cerejeira oriental; flores de cerejeira}
  \end{Phonetics}
\end{Entry}

\begin{Entry}{樱桃}{15,10}{⽊、⽊}
  \begin{Phonetics}{樱桃}{ying1tao2}
    \definition{s.}{cereja}
  \end{Phonetics}
\end{Entry}

\begin{Entry}{橄}{15}{⽊}
  \begin{Phonetics}{橄}{gan3}
    \definition*{s.}{Sobrenome Gan}
  \end{Phonetics}
\end{Entry}

\begin{Entry}{橄榄球}{15,13,11}{⽊、⽊、⽟}
  \begin{Phonetics}{橄榄球}{gan3lan3qiu2}
    \definition{s.}{futebol jogado com bola oval (rúgbi, futebol americano, regras australianas, etc.)}
  \end{Phonetics}
\end{Entry}

\begin{Entry}{潜}{15}{⽔}
  \begin{Phonetics}{潜}{qian2}
    \definition*{s.}{Sobrenome Qian}
    \definition{adj.}{latente; oculto}
    \definition{adv.}{furtivamente; secretamente; às escondidas}
    \definition{v.}{ir para debaixo d'água; esconder-se debaixo d'água; mergulhar | esconder | vadear (atravessar) na água | enterrar | fugir de casa}
  \end{Phonetics}
\end{Entry}

\begin{Entry}{潜力}{15,2}{⽔、⼒}
  \begin{Phonetics}{潜力}{qian2li4}[][HSK 6]
    \definition{s.}{potencial; potencialidade; capacidade latente; as habilidades e possibilidades de desenvolvimento que as pessoas e as coisas ainda não demonstraram}
  \end{Phonetics}
\end{Entry}

\begin{Entry}{潜在}{15,6}{⽔、⼟}
  \begin{Phonetics}{潜在}{qian2zai4}
    \definition{adj.}{oculto | latente}
    \definition{s.}{potencial}
  \end{Phonetics}
\end{Entry}

\begin{Entry}{潮}{15}{⽔}
  \begin{Phonetics}{潮}{chao2}[][HSK 4]
    \definition{adj.}{úmido; molhado | inferior; de qualidade ruim | inferior; não muito habilidoso}
    \definition{s.}{maré; água da maré | surto; corrente; maré; uma metáfora para mudanças sociais em grande escala ou para os altos e baixos de um movimento (social)}
    \definition{s.}{Chaozhou, uma cidade na província de Guangdong}
  \end{Phonetics}
\end{Entry}

\begin{Entry}{潮流}{15,10}{⽔、⽔}
  \begin{Phonetics}{潮流}{chao2liu2}[][HSK 4]
    \definition[种,股,个]{s.}{maré; corrente de maré; movimento da água devido às marés | tendência; analogia com mudanças sociais ou tendências de desenvolvimento}
  \end{Phonetics}
\end{Entry}

\begin{Entry}{潮绣}{15,10}{⽔、⽷}
  \begin{Phonetics}{潮绣}{chao2xiu4}
    \definition*{s.}{Bordado Chaozhou}
  \end{Phonetics}
\end{Entry}

\begin{Entry}{潮湿}{15,12}{⽔、⽔}
  \begin{Phonetics}{潮湿}{chao2shi1}[][HSK 4]
    \definition{adj.}{molhado; úmido; umedecido; que contém mais água do que o normal}
  \end{Phonetics}
\end{Entry}

\begin{Entry}{澄}{15}{⽔}
  \begin{Phonetics}{澄}{cheng2}
    \definition*{s.}{Sobrenome Cheng}
    \definition{adj.}{claro; transparente}
    \definition{v.}{esclarecer; purificar}
  \end{Phonetics}
  \begin{Phonetics}{澄}{deng4}
    \definition{adj.}{(água, ar, etc.) claro; transparente; límpido}
    \definition{v.}{esclarecer; aclarar | sedimentar; fazer com que impurezas em um líquido afundem}
  \end{Phonetics}
\end{Entry}

\begin{Entry}{澄清}{15,11}{⽔、⽔}
  \begin{Phonetics}{澄清}{cheng2qing1}[][HSK 7-9]
    \definition{adj.}{claro; transparente}
    \definition{v.}{esclarecer; deixar claro; entender | purificar; limpar; esclarecer a turbidez, uma metáfora para esclarecer uma situação caótica}
  \end{Phonetics}
\end{Entry}

\begin{Entry}{澳}{15}{⽔}
  \begin{Phonetics}{澳}{ao4}
    \definition*{s.}{Abreviação de Austrália, 澳大利亚 | Sobrenome Ao}
    \definition{s.}{baía; uma entrada do mar; um lugar curvo na costa onde os barcos podem ser atracados, frequentemente usado em nomes de lugares}
  \seealsoref{澳大利亚}{ao4da4li4ya4}
  \end{Phonetics}
\end{Entry}

\begin{Entry}{澳大利亚}{15,3,7,6}{⽔、⼤、⼑、⼆}
  \begin{Phonetics}{澳大利亚}{ao4da4li4ya4}
    \definition*{s.}{Austrália}
  \end{Phonetics}
\end{Entry}

\begin{Entry}{熟}{15}{⽕}
  \begin{Phonetics}{熟}{shu2}[][HSK 2]
    \definition{adj.}{maduro (frutos) | pronto; cozido | processado, fabricado ou exercitado | familiar, bem conhecido; conhecido por ser comum ou frequentemente utilizado | habilidoso;  (trabalho, tecnologia) experiente; não é novato | profundo; sólido}
  \end{Phonetics}
\end{Entry}

\begin{Entry}{熟人}{15,2}{⽕、⼈}
  \begin{Phonetics}{熟人}{shu2 ren2}[][HSK 3]
    \definition[位,名,个,些]{s.}{amigo; conhecido; pessoas que se conhecem há muito tempo; pessoas que são muito familiares}
  \end{Phonetics}
\end{Entry}

\begin{Entry}{熟练}{15,8}{⽕、⽷}
  \begin{Phonetics}{熟练}{shu2lian4}[][HSK 4]
    \definition{adj.}{especializado; proficiente; qualificado; habilidoso}
  \end{Phonetics}
\end{Entry}

\begin{Entry}{熟悉}{15,11}{⽕、⼼}
  \begin{Phonetics}{熟悉}{shu2xi1}[][HSK 5]
    \definition{adj.}{familiarizado com; não ser estranho}
    \definition{v.}{estar familiarizado com; saber claramente que | conhecer bem algo ou alguém; compreender e dominar (a situação) através da observação ou da experiência}
  \end{Phonetics}
\end{Entry}

\begin{Entry}{獞}{15}{⽝}
  \begin{Phonetics}{獞}{tong2}
    \definition{s.}{nome de uma variedade de cão | tribos selvagens no sul da China}
  \end{Phonetics}
  \begin{Phonetics}{獞}{zhuang4}
    \variantof{壮}
  \end{Phonetics}
\end{Entry}

\begin{Entry}{碾}{15}{⽯}
  \begin{Phonetics}{碾}{nian3}
    \definition[台,个]{s.}{rolo e mó; rolo de pedra | rolo compressor}
    \definition{v.}{moer ou descascar com um rolo; esmagar | (literário) cortar e polir (jade, vidro, etc.) | achatar | pisar; pisotear, 轧}
  \seealsoref{辗}{zhan3}
  \end{Phonetics}
\end{Entry}

\begin{Entry}{碾碎}{15,13}{⽯、⽯}
  \begin{Phonetics}{碾碎}{nian3sui4}
    \definition{v.}{pulverizar | esmagar}
  \end{Phonetics}
\end{Entry}

\begin{Entry}{磅}{15}{⽯}
  \begin{Phonetics}{磅}{bang4}[][HSK 7-9]
    \definition{clas.}{libra | Tipografia: pt, ponto (tamanho de letra, por exemplo: 10pt)}
    \definition{s.}{escalas}
    \definition{v.}{pesar com uma balança}
  \end{Phonetics}
  \begin{Phonetics}{磅}{pang2}
    \definition{adj.}{majestoso; abundante; cheio de energia; magnífico}
  \end{Phonetics}
\end{Entry}

\begin{Entry}{稻}{15}{⽲}
  \begin{Phonetics}{稻}{dao4}
    \definition{s.}{arroz; arroz com casca}
  \end{Phonetics}
\end{Entry}

\begin{Entry}{稻草}{15,9}{⽲、⾋}
  \begin{Phonetics}{稻草}{dao4cao3}[][HSK 7-9]
    \definition[捆,根,抱,束]{s.}{palha de arroz (pode ser usada para fazer cordas ou esteiras de palha, para fazer papel, ou para ser usada como ração, combustível, etc.)}
  \end{Phonetics}
\end{Entry}

\begin{Entry}{稿}{15}{⽲}
  \begin{Phonetics}{稿}{gao3}
    \definition[篇]{s.}{(significado original) talo de grão; palha | rascunho; esboço; manuscrito | texto original}
  \end{Phonetics}
\end{Entry}

\begin{Entry}{稿子}{15,3}{⽲、⼦}
  \begin{Phonetics}{稿子}{gao3 zi5}[][HSK 6]
    \definition[篇,份,堆,叠]{s.}{rascunho; esboço; rascunhos de poemas, ensaios, desenhos, etc. | rascunho; manuscrito; poemas escritos | ideia; plano; plano preliminar ou conceito de trabalho}
  \end{Phonetics}
\end{Entry}

\begin{Entry}{稿纸}{15,7}{⽲、⽷}
  \begin{Phonetics}{稿纸}{gao3zhi3}
    \definition{s.}{rascunho | manuscrito}
  \end{Phonetics}
\end{Entry}

\begin{Entry}{箭}{15}{⽵}
  \begin{Phonetics}{箭}{jian4}[][HSK 6]
    \definition[支]{s.}{seta | distância percorrida por uma flecha}
  \end{Phonetics}
\end{Entry}

\begin{Entry}{箱}{15}{⾋}
  \begin{Phonetics}{箱}{xiang1}[][HSK 4]
    \definition{s.}{caixa; estojo; baú | qualquer coisa no formato de caixa}
  \end{Phonetics}
\end{Entry}

\begin{Entry}{箱子}{15,3}{⾋、⼦}
  \begin{Phonetics}{箱子}{xiang1 zi5}[][HSK 4]
    \definition[个,只]{s.}{baú; caixa; estojo; maleta; pasta executiva}
  \end{Phonetics}
\end{Entry}

\begin{Entry}{篇}{15}{⽵}
  \begin{Phonetics}{篇}{pian1}[][HSK 2]
    \definition*{s.}{Sobrenome Pian}
    \definition{clas.}{usado para folhas de papel, páginas de livros, artigos, etc.}
    \definition{s.}{um pedaço de escrita | folha (de papel, etc.) | (para papel, folhas de livros, artigos, etc.) folha; página; pedaço}
  \end{Phonetics}
\end{Entry}

\begin{Entry}{糆}{15}{⽶}
  \begin{Phonetics}{糆}{mian4}
    \variantof{面}
  \end{Phonetics}
\end{Entry}

\begin{Entry}{糊}{15}{⽶}
  \begin{Phonetics}{糊}{hu1}
    \definition{v.}{colar; untar; usar uma pasta mais espessa para revestir costuras, furos ou superfícies planas}
  \end{Phonetics}
  \begin{Phonetics}{糊}{hu2}[][HSK 7-9]
    \definition{adj.}{queimado}
    \definition{s.}{mingau; pasta; papa}
    \definition{v.}{colar com pasta; colar | (comida) ser queimado}
  \end{Phonetics}
  \begin{Phonetics}{糊}{hu4}
    \definition{s.}{pasta; comida que parece mingau}
  \end{Phonetics}
\end{Entry}

\begin{Entry}{糊里糊涂}{15,7,15,10}{⽶、⾥、⽶、⽔}
  \begin{Phonetics}{糊里糊涂}{hu2 li5 hu2tu5}
    \definition{adj.}{desnorteado | perturbado}
  \end{Phonetics}
\end{Entry}

\begin{Entry}{糊涂}{15,10}{⽶、⽔}
  \begin{Phonetics}{糊涂}{hu2tu5}[][HSK 7-9]
    \definition{adj.}{confuso; perplexo; desnorteado; com compreensão pouco clara ou confusa das coisas | confuso; com conteúdo confuso}
  \end{Phonetics}
\end{Entry}

\begin{Entry}{聪}{15}{⽿}
  \begin{Phonetics}{聪}{cong1}
    \definition{adj.}{audição aguçada | brilhante; inteligente; esperto | perspicaz}
    \definition{s.}{(literário) faculdades auditivas}
  \end{Phonetics}
\end{Entry}

\begin{Entry}{聪明}{15,8}{⽿、⽇}
  \begin{Phonetics}{聪明}{cong1ming5}[][HSK 5]
    \definition{adj.}{brilhante; esperto; inteligente; intelecto bem desenvolvido com boa memória e capacidade de compreensão}
  \end{Phonetics}
\end{Entry}

\begin{Entry}{聪慧}{15,15}{⽿、⼼}
  \begin{Phonetics}{聪慧}{cong1hui4}
    \definition{adj.}{inteligente | brilhante}
  \end{Phonetics}
\end{Entry}

\begin{Entry}{蔬}{15}{⾋}
  \begin{Phonetics}{蔬}{shu1}
    \definition{s.}{vegetais}
  \end{Phonetics}
\end{Entry}

\begin{Entry}{蔬菜}{15,11}{⾋、⾋}
  \begin{Phonetics}{蔬菜}{shu1cai4}[][HSK 5]
    \definition[样,种]{s.}{verduras; legumes; vegetais; ervas que podem ser usadas na culinária}
  \end{Phonetics}
\end{Entry}

\begin{Entry}{蕃}{15}{⾋}
  \begin{Phonetics}{蕃}{bo1}
    \definition[种]{s.}{estrangeiros}
  \end{Phonetics}
  \begin{Phonetics}{蕃}{fan1}
    \definition[种]{s.}{estrangeiros; aborígenes}
  \end{Phonetics}
  \begin{Phonetics}{蕃}{fan2}
    \definition{adj.}{exuberante; próspero}
    \definition{v.}{multiplicar; proliferar}
  \end{Phonetics}
\end{Entry}

\begin{Entry}{蕃茄}{15,8}{⾋、⾋}
  \begin{Phonetics}{蕃茄}{fan1 qie2}
    \variantof{番茄}
  \end{Phonetics}
\end{Entry}

\begin{Entry}{蝌}{15}{⾍}
  \begin{Phonetics}{蝌}{ke1}
    \definition[只]{s.}{girino}
  \end{Phonetics}
\end{Entry}

\begin{Entry}{蝌蚪}{15,10}{⾍、⾍}
  \begin{Phonetics}{蝌蚪}{ke1dou3}
    \definition{s.}{girino}
  \end{Phonetics}
\end{Entry}

\begin{Entry}{蝲}{15}{⾍}
  \begin{Phonetics}{蝲}{la4}
    \definition{s.}{lagostim de água doce}
  \seealsoref{蝲蛄}{la4gu3}
  \end{Phonetics}
\end{Entry}

\begin{Entry}{蝲蛄}{15,11}{⾍、⾍}
  \begin{Phonetics}{蝲蛄}{la4gu3}
    \definition{s.}{lagostim; lagostim de água doce}
  \end{Phonetics}
\end{Entry}

\begin{Entry}{蝲蝲蛄}{15,15,11}{⾍、⾍、⾍}
  \begin{Phonetics}{蝲蝲蛄}{la4la4gu3}
    \definition{s.}{grilo toupeira}
  \end{Phonetics}
\end{Entry}

\begin{Entry}{蝴}{15}{⾍}
  \begin{Phonetics}{蝴}{hu2}
    \definition[对]{s.}{borboleta}
  \end{Phonetics}
\end{Entry}

\begin{Entry}{蝴蝶}{15,15}{⾍、⾍}
  \begin{Phonetics}{蝴蝶}{hu2die2}
    \definition[只]{s.}{borboleta}
  \end{Phonetics}
\end{Entry}

\begin{Entry}{豌}{15}{⾖}
  \begin{Phonetics}{豌}{wan1}
    \definition[粒]{s.}{ervilhas}
  \end{Phonetics}
\end{Entry}

\begin{Entry}{豌豆}{15,7}{⾖、⾖}
  \begin{Phonetics}{豌豆}{wan1dou4}
    \definition{s.}{ervilha}
  \end{Phonetics}
\end{Entry}

\begin{Entry}{豫}{15}{⾗}
  \begin{Phonetics}{豫}{yu4}
    \definition*{s.}{Província de Henan, abreviatura de 河南}
    \definition{adj.}{satisfeito; encantado | anterior; preliminar; preparatório}
    \definition{adv.}{com antecedência; antecipadamente}
    \definition{v.}{viver com facilidade e conforto | participar de}
  \seealsoref{河南}{he2nan2}
  \seealsoref{预}{yu4}
  \end{Phonetics}
\end{Entry}

\begin{Entry}{趟}{15}{⾛}
  \begin{Phonetics}{趟}{tang1}
    \definition{v.}{atravessar; andar na grama ou onde não haja caminho | usar arados, capinadores, etc. para virar o solo e remover ervas daninhas | vadear; atravessar a vau; caminhar por águas rasas}[我们趟水去那小岛。===Nós vadeamos até a ilha.]
  \end{Phonetics}
  \begin{Phonetics}{趟}{tang4}[][HSK 6]
    \definition{clas.}{usado para o número de vezes de viagens de ida e volta |  usado para coisas dispostas em fileiras ou tiras | usado para a programação de veículos, navios, etc. que circulam em uma determinada ordem | usado em conjuntos de movimentos de artes marciais}
    \definition{s.}{marcha; procissão; jornada; viagem}
  \end{Phonetics}
\end{Entry}

\begin{Entry}{踏}{15}{⾜}
  \begin{Phonetics}{踏}{ta1}
    \definition{part.}{Caracter formador de palavras}
  \end{Phonetics}
  \begin{Phonetics}{踏}{ta4}[][HSK 6]
    \definition{v.}{por os pés em; pisar em; esmagar com o pé | fazer uma investigação ou levantamento no local}
  \end{Phonetics}
\end{Entry}

\begin{Entry}{踏实}{15,8}{⾜、⼧}
  \begin{Phonetics}{踏实}{ta1shi5}[][HSK 6]
    \definition{adj.}{confiável; sério; estável e seguro; descreve uma atitude séria em relação ao trabalho ou estudo | à vontade; livre de ansiedade; descreve uma mente ou sentimento estável, sem qualquer preocupação ou ansiedade}
  \end{Phonetics}
\end{Entry}

\begin{Entry}{踏板}{15,8}{⾜、⽊}
  \begin{Phonetics}{踏板}{ta4ban3}
    \definition{s.}{pedal (em um carro, em um piano, etc.) |  apoio para os pés | estribo}
  \end{Phonetics}
\end{Entry}

\begin{Entry}{踢}{15}{⾜}
  \begin{Phonetics}{踢}{ti1}[][HSK 6]
    \definition{v.}{chutar | jogar (por exemplo, futebol)}
  \end{Phonetics}
\end{Entry}

\begin{Entry}{踢蹋舞}{15,17,14}{⾜、⾜、⾇}
  \begin{Phonetics}{踢蹋舞}{ti1ta4wu3}
    \definition{s.}{sapateado | passo de dança}
  \end{Phonetics}
\end{Entry}

\begin{Entry}{踢爆}{15,19}{⾜、⽕}
  \begin{Phonetics}{踢爆}{ti1bao4}
    \definition{v.}{expor | revelar}
  \end{Phonetics}
\end{Entry}

\begin{Entry}{踩}{15}{⾜}
  \begin{Phonetics}{踩}{cai3}[][HSK 6]
    \definition{v.}{pisar; pisotear | pisar; metáfora: depreciar ou estragar | rastrear; antigamente significava rastrear (bandidos) ou investigar (casos)}
  \end{Phonetics}
\end{Entry}

\begin{Entry}{躺}{15}{⾝}
  \begin{Phonetics}{躺}{tang3}[][HSK 4]
    \definition{v.}{deitar; reclinar; cair no chão ou sobre um objeto}
  \end{Phonetics}
\end{Entry}

\begin{Entry}{遵}{15}{⾡}
  \begin{Phonetics}{遵}{zun1}
    \definition{v.}{cumprir; obedecer; observar; seguir}
  \end{Phonetics}
\end{Entry}

\begin{Entry}{遵守}{15,6}{⾡、⼧}
  \begin{Phonetics}{遵守}{zun1shou3}[][HSK 5]
    \definition{v.}{obedecer; observar; cumprir; respeitar; atuar de acordo com as regras; não infringir}
  \end{Phonetics}
\end{Entry}

\begin{Entry}{醇}{15}{⾣}
  \begin{Phonetics}{醇}{chun2}
    \definition{adj.}{Literário: puro; puro e suave; não misturado}
    \definition{s.}{Literário: vinho suave; bom vinho ; Química: álcool}
  \end{Phonetics}
\end{Entry}

\begin{Entry}{醇厚}{15,9}{⾣、⼚}
  \begin{Phonetics}{醇厚}{chun2hou4}[][HSK 7-9]
    \definition{adj.}{suave; rico; cheiro e sabor puros e ricos | puro e honesto; simples e gentil}
  \end{Phonetics}
\end{Entry}

\begin{Entry}{醉}{15}{⾣}
  \begin{Phonetics}{醉}{zui4}[][HSK 5]
    \definition{v.}{embriagar-se; ficar bêbado; intoxicar-se; beber em excesso e perder o controle | (de certos alimentos) ser embebido em licor; ser mergulhado em vinho; marinar (alimentos) em vinho | entregar-se a; ser viciado em; gostar demais, a ponto de chegar à obsessão}
  \end{Phonetics}
\end{Entry}

\begin{Entry}{醋}{15}{⾣}
  \begin{Phonetics}{醋}{cu4}[][HSK 6]
    \definition[瓶,坛,碟,碗]{s.}{(condimento) vinagre | ciúme (como em caso de amor); uma metáfora para o ciúme, referindo-se principalmente aos relacionamentos entre pessoas}
  \end{Phonetics}
\end{Entry}

\begin{Entry}{镇}{15}{⾦}
  \begin{Phonetics}{镇}{zhen4}[][HSK 6]
    \definition{adj.}{inteiro; indica um período inteiro de tempo}
    \definition{adv.}{frequentemente; muitas vezes}
    \definition{s.}{posto de guarnição | cidade; divisão administrativa | centro comercial}
    \definition{v.}{suprimir; segurar; manter pressionado |  acalmar-se; recompor-se; estabilizar | guardar; guarnecer; fortalecer; usar a força para manter a estabilidade | resfriar com gelo; esfriar em água fria | acalmar; suprimir; dissuadir | suprimir pela força; sancionar}
  \end{Phonetics}
\end{Entry}

\begin{Entry}{震}{15}{⾬}
  \begin{Phonetics}{震}{zhen4}
    \definition*{s.}{Zhen, um dos Oito Trigramas que representa o trovão | Sobrenome Zhen}
    \definition{adj.}{(coloquial) muito animado; profundamente surpreso; chocado}
    \definition{s.}{vibração; trepidação; tremor; abalo | terremoto; refere-se especificamente a terremotos | trovão; relâmpago}
    \definition{v.}{sacudir; chocar; vibrar; estremecer | ficar muito animado; ficar profundamente surpreso; ficar chocado | superar; vencer}
  \end{Phonetics}
\end{Entry}

\begin{Entry}{震惊}{15,11}{⾬、⼼}
  \begin{Phonetics}{震惊}{zhen4jing1}[][HSK 5]
    \definition{adj.}{chocado; atordoado; espantado; atônito}
    \definition{v.}{chocar; surpreender; espantar}
  \end{Phonetics}
\end{Entry}

\begin{Entry}{震撼}{15,16}{⾬、⼿}
  \begin{Phonetics}{震撼}{zhen4han4}
    \definition{v.}{sacudir | chocar | atordoar}
  \end{Phonetics}
\end{Entry}

\begin{Entry}{靠}{15}{⾮}
  \begin{Phonetics}{靠}{kao4}[][HSK 2]
    \definition{prep.}{manter (em); aproximar-se (de); ao longo de | por; graças a; com base em; de acordo com}
    \definition{s.}{armadura de palco (feita de seda bordada); armadura usada pelos generais militares antigos nas peças teatrais}
    \definition{v.}{inclinar-se; sentado ou em pé, deixar parte do peso do corpo ser suportado por outra pessoa ou objeto (pessoa) | encostar-se (em); apoiar-se ou levantar-se com a ajuda de alguma coisa | aproximar-se; estar perto de | confiar em; depender de | confiar}
  \end{Phonetics}
\end{Entry}

\begin{Entry}{靠近}{15,7}{⾮、⾡}
  \begin{Phonetics}{靠近}{kao4 jin4}[][HSK 5]
    \definition{adv.}{próximo; perto de; ao lado de}
    \definition{v.}{aproximar-se; chegar perto; avançar em direção a um determinado objetivo de modo que a distância fique cada vez menor}
  \end{Phonetics}
\end{Entry}

\begin{Entry}{鞋}{15}{⾰}
  \begin{Phonetics}{鞋}{xie2}[][HSK 2]
    \definition[双,只]{s.}{sapatos; usado nos pés; algo que toca o chão ao caminhar; sem cano alto}
  \end{Phonetics}
\end{Entry}

\begin{Entry}{题}{15}{⾴}
  \begin{Phonetics}{题}{ti2}[][HSK 2]
    \definition*{s.}{Sobrenome Ti}
    \definition[个,道]{s.}{tópico; título; assunto; problema; frases que indicam o conteúdo de poemas ou discursos | questão; questões que devem ser respondidas durante os exercícios ou exames | antigamente, referia-se à testa}
    \definition{v.}{inscrever; escrever; assinar}
  \end{Phonetics}
\end{Entry}

\begin{Entry}{题目}{15,5}{⾴、⽬}
  \begin{Phonetics}{题目}{ti2mu4}[][HSK 3]
    \definition[个,道]{s.}{título; assunto; tópico; o título de um poema ou discurso | quebra-cabeça; problema de exercício; questões a serem respondidas em exercícios ou provas}
  \end{Phonetics}
\end{Entry}

\begin{Entry}{题材}{15,7}{⾴、⽊}
  \begin{Phonetics}{题材}{ti2cai2}[][HSK 5]
    \definition{s.}{tema; assunto; material que compõe as obras literárias e artísticas, ou seja, os eventos ou fenômenos da vida descritos concretamente nas obras}
  \end{Phonetics}
\end{Entry}

\begin{Entry}{颜}{15}{⾴}
  \begin{Phonetics}{颜}{yan2}
    \definition*{s.}{Sobrenome Yan}
    \definition{s.}{rosto; semblante; expressão facial | rosto; prestígio; dignidade | cor}
  \end{Phonetics}
\end{Entry}

\begin{Entry}{颜色}{15,6}{⾴、⾊}
  \begin{Phonetics}{颜色}{yan2 se4}[][HSK 2]
    \definition[个,种]{s.}{cor; a sensação visual de um objeto é uma impressão diferente produzida pelas diferentes quantidades de luz absorvidas e refletidas pelo objeto | tez; semblante; aparência; geralmente se refere à aparência de uma garota | olhar severo no rosto como um aviso; um olhar ou ação que faz os outros parecerem particularmente ferozes | a expressão mostrada no rosto}
  \end{Phonetics}
\end{Entry}

\begin{Entry}{额}{15}{⾴}
  \begin{Phonetics}{额}{e2}
    \definition*{s.}{Sobrenome E}
    \definition[块]{s.}{testa; a área abaixo do cabelo e acima das sobrancelhas em humanos; a área aproximadamente equivalente na cabeça de alguns animais | uma tábua horizontal; placa horizontal inscrita; uma placa pendurada no lintel de uma porta ou na parede | um número específico (ou quantidade); limite superior de número; número limitado | a parte superior de algo}
  \end{Phonetics}
\end{Entry}

\begin{Entry}{额外}{15,5}{⾴、⼣}
  \begin{Phonetics}{额外}{e2wai4}[][HSK 7-9]
    \definition{adj.}{extra; adicional; excede a quantidade ou intervalo prescrito}
  \end{Phonetics}
\end{Entry}

\begin{Entry}{飘}{15}{⾵}
  \begin{Phonetics}{飘}{piao1}
    \definition{adj.}{complacente | frívolo | fraco | instável | bambo | cambaleante}
    \definition{v.}{flutuar (no ar) | esvoaçar | tremular}
  \end{Phonetics}
\end{Entry}

\begin{Entry}{鲨}{15}{⿂}
  \begin{Phonetics}{鲨}{sha1}
    \definition[只,条]{s.}{tubarão}
  \end{Phonetics}
\end{Entry}

\begin{Entry}{鲨鱼}{15,8}{⿂、⿂}
  \begin{Phonetics}{鲨鱼}{sha1yu2}
    \definition{s.}{tubarão}
  \end{Phonetics}
\end{Entry}

\begin{Entry}{鹤}{15}{⿃}
  \begin{Phonetics}{鹤}{he4}
    \definition*{s.}{Sobrenome He}
    \definition[只]{s.}{grou (ave)}
  \end{Phonetics}
\end{Entry}

\begin{Entry}{鹤立鸡群}{15,5,7,13}{⿃、⽴、⿃、⽺}
  \begin{Phonetics}{鹤立鸡群}{he4li4ji1qun2}[][HSK 7-9]
    \definition{expr.}{destaque-se da multidão; manifestamente superior; muito acima do comum; como um guindaste em pé entre galinhas --- fique de pé acima dos outros}
  \end{Phonetics}
\end{Entry}

\begin{Entry}{麫}{15}{⿆}
  \begin{Phonetics}{麫}{mian4}
    \variantof{面}
  \end{Phonetics}
\end{Entry}

\begin{Entry}{黎}{15}{⿉}
  \begin{Phonetics}{黎}{li2}
    \definition*{s.}{Etnia Li, uma das minorias nacionais da província de Hainan | Sobrenome Li}
    \definition{adj.}{Literário: preto; escuro | Literário: numeroso}
    \definition{s.}{multidão; as massas; a população}
  \end{Phonetics}
\end{Entry}

%%%%% EOF %%%%%


%%%
%%% 16画
%%%

\section*{16画}\addcontentsline{toc}{section}{16画}

\begin{entry}{儒教}{16,11}[Radicais ⼈、⽁]
  \begin{phonetics}{儒教}{ru2jiao4}
    \definition*{s.}{Confucionismo}
  \end{phonetics}
\end{entry}

\begin{entry}{嘴}{16}[Radical ⼝]
  \begin{phonetics}{嘴}{zui3}[][HSK 2]
    \definition[张]{s.}{boca | qualquer coisa com formato ou função semelhante a uma boca}
    \definition{v.}{falar}
  \end{phonetics}
\end{entry}

\begin{entry}{嘴巴}{16,4}[Radicais ⼝、⼰]
  \begin{phonetics}{嘴巴}{zui3ba5}
    \definition[张]{s.}{boca}
    \definition[个]{s.}{bofetada na cara}
  \end{phonetics}
\end{entry}

\begin{entry}{嘴巴子}{16,4,3}[Radicais ⼝、⼰、⼦]
  \begin{phonetics}{嘴巴子}{zui3ba5zi5}
    \definition{s.}{tapa | bofetada}
  \end{phonetics}
\end{entry}

\begin{entry}{器}{16}[Radical ⼝]
  \begin{phonetics}{器}{qi4}
    \definition[台]{s.}{dispositivo | ferramenta | utensílio}
  \end{phonetics}
\end{entry}

\begin{entry}{壁纸}{16,7}[Radicais ⼟、⽷]
  \begin{phonetics}{壁纸}{bi4zhi3}
    \definition{s.}{papel de parede}
  \end{phonetics}
\end{entry}

\begin{entry}{壁虎}{16,8}[Radicais ⼟、⾌]
  \begin{phonetics}{壁虎}{bi4hu3}
    \definition{s.}{lagartixa}
  \end{phonetics}
\end{entry}

\begin{entry}{懒}{16}[Radical ⼼]
  \begin{phonetics}{懒}{lan3}
    \definition{adj.}{preguiçoso | indolente | vadio}
  \end{phonetics}
\end{entry}

\begin{entry}{懒人}{16,2}[Radicais ⼼、⼈]
  \begin{phonetics}{懒人}{lan3ren2}
    \definition{s.}{pessoa preguiçosa}
  \end{phonetics}
\end{entry}

\begin{entry}{懒汉}{16,5}[Radicais ⼼、⽔]
  \begin{phonetics}{懒汉}{lan3han4}
    \definition{s.}{sujeito ocioso | vagabundo | preguiçosos}
  \end{phonetics}
\end{entry}

\begin{entry}{懒虫}{16,6}[Radicais ⼼、⾍]
  \begin{phonetics}{懒虫}{lan3chong2}
    \definition{s.}{desleixado ocioso | (insulto) sujeito preguiçoso}
  \end{phonetics}
\end{entry}

\begin{entry}{懒怠}{16,9}[Radicais ⼼、⼼]
  \begin{phonetics}{懒怠}{lan3dai4}
    \definition{s.}{preguiça}
  \end{phonetics}
\end{entry}

\begin{entry}{懒鬼}{16,9}[Radicais ⼼、⿁]
  \begin{phonetics}{懒鬼}{lan3gui3}
    \definition{s.}{cara preguiçoso}
  \end{phonetics}
\end{entry}

\begin{entry}{懒得}{16,11}[Radicais ⼼、⼻]
  \begin{phonetics}{懒得}{lan3de5}
    \definition{adv.}{demasiado preguiçoso}
    \definition{v.}{não sentir vontade (de fazer algo)}
  \end{phonetics}
\end{entry}

\begin{entry}{懒惰}{16,12}[Radicais ⼼、⼼]
  \begin{phonetics}{懒惰}{lan3duo4}
    \definition{adj.}{preguiçoso}
  \end{phonetics}
\end{entry}

\begin{entry}{懒散}{16,12}[Radicais ⼼、⽁]
  \begin{phonetics}{懒散}{lan3san3}
    \definition{adj.}{inativo | indolente | preguiçoso | negligente}
  \end{phonetics}
\end{entry}

\begin{entry}{懒腰}{16,13}[Radicais ⼼、⾁]
  \begin{phonetics}{懒腰}{lan3yao1}
    \definition[个]{s.}{alongamento (do corpo)}
  \end{phonetics}
\end{entry}

\begin{entry}{撼}{16}[Radical ⼿]
  \begin{phonetics}{撼}{han4}
    \definition{v.}{sacudir | vibrar}
  \end{phonetics}
\end{entry}

\begin{entry}{擅自}{16,6}[Radicais ⼿、⾃]
  \begin{phonetics}{擅自}{shan4zi4}
    \definition{adv.}{sem permissão ou autorização | por iniciativa própria}
  \end{phonetics}
\end{entry}

\begin{entry}{操心}{16,4}[Radicais ⼿、⼼]
  \begin{phonetics}{操心}{cao1xin1}
    \definition{v.+compl.}{preocupar-se com}
  \end{phonetics}
\end{entry}

\begin{entry}{操场}{16,6}[Radicais ⼿、⼟]
  \begin{phonetics}{操场}{cao1chang3}[][HSK 4]
    \definition[个]{s.}{\emph{playground}; campo esportivo; locais para exercícios físicos ou exercícios militares}
  \end{phonetics}
\end{entry}

\begin{entry}{操作}{16,7}[Radicais ⼿、⼈]
  \begin{phonetics}{操作}{cao1zuo4}[][HSK 4]
    \definition{s.}{operação}
    \definition{v.}{operar; seguir os requisitos e procedimentos prescritos| implementar; realizar; executar; refere-se à implementação concreta (planos, medidas, etc.)}
  \end{phonetics}
\end{entry}

\begin{entry}{整}{16}[Radical ⽁]
  \begin{phonetics}{整}{zheng3}[][HSK 3]
    \definition*{s.}{sobrenome Zheng}
    \definition{adj.}{cheio; integral; inteiro; completo; sem defeitos | limpo; arrumado; em boa ordem | redondo (não é uma fração)}
    \definition{s.}{inteiro (número)}
    \definition{v.}{retificar; pôr em ordem | consertar; renovar; reparar |consertar; punir; fazer alguém sofrer | fazer; produzir; trabalhar}
  \end{phonetics}
\end{entry}

\begin{entry}{整个}{16,3}[Radicais ⽁、⼈]
  \begin{phonetics}{整个}{zheng3ge4}[][HSK 3]
    \definition{adj.}{total; inteiro; completo}
  \end{phonetics}
\end{entry}

\begin{entry}{整天}{16,4}[Radicais ⽁、⼤]
  \begin{phonetics}{整天}{zheng3 tian1}[][HSK 3]
    \definition{s.}{o dia todo; de manhã até a noite}
  \end{phonetics}
\end{entry}

\begin{entry}{整齐}{16,6}[Radicais ⽁、⿑]
  \begin{phonetics}{整齐}{zheng3qi2}[][HSK 3]
    \definition{adj.}{limpo; arrumado; em boa ordem | uniforme; regular; relativamente consistente em tamanho, comprimento, grau, etc. | usado para descrever que as coisas necessárias estão todas prontas}
    \definition{v.}{estar em boas condições; deixar as coisas organizadas e arrumadas}
  \end{phonetics}
\end{entry}

\begin{entry}{整体}{16,7}[Radicais ⽁、⼈]
  \begin{phonetics}{整体}{zheng3ti3}[][HSK 3]
    \definition[个]{s.}{todo; totalidade}
  \end{phonetics}
\end{entry}

\begin{entry}{整理}{16,11}[Radicais ⽁、⽟]
  \begin{phonetics}{整理}{zheng3li3}[][HSK 3]
    \definition{v.}{organizar; reorganizar; classificar; ordenar; colocar em ordem}
  \end{phonetics}
\end{entry}

\begin{entry}{整整}{16,16}[Radicais ⽁、⽁]
  \begin{phonetics}{整整}{zheng3 zheng3}[][HSK 3]
    \definition{adv.}{inteiramente; completamente; solidamente; continuamente}
  \end{phonetics}
\end{entry}

\begin{entry}{橘子汁}{16,3,5}[Radicais ⽊、⼦、⽔]
  \begin{phonetics}{橘子汁}{ju2zi5zhi1}
    \definition[瓶,杯,罐,盒]{s.}{suco de laranja}
  \seealsoref{橙汁}{cheng2zhi1}
  \seealsoref{柳橙汁}{liu3cheng2zhi1}
  \end{phonetics}
\end{entry}

\begin{entry}{橙汁}{16,5}[Radicais ⽊、⽔]
  \begin{phonetics}{橙汁}{cheng2zhi1}
    \definition[瓶,杯,罐,盒]{s.}{suco de laranja}
  \seealsoref{橘子汁}{ju2zi5zhi1}
  \seealsoref{柳橙汁}{liu3cheng2zhi1}
  \end{phonetics}
\end{entry}

\begin{entry}{橙色}{16,6}[Radicais ⽊、⾊]
  \begin{phonetics}{橙色}{cheng2 se4}
    \definition{s.}{cor de laranja}
  \end{phonetics}
\end{entry}

\begin{entry}{激动}{16,6}[Radicais ⽔、⼒]
  \begin{phonetics}{激动}{ji1dong4}
    \definition{v.}{excitar | mover-se emocionalmente | agitar (emoções)}
  \end{phonetics}
\end{entry}

\begin{entry}{燃烧}{16,10}[Radicais ⽕、⽕]
  \begin{phonetics}{燃烧}{ran2shao1}
    \definition{s.}{combustão | flama}
    \definition{v.}{queimar | acender}
  \end{phonetics}
\end{entry}

\begin{entry}{犟}{16}[Radical ⽜]
  \begin{phonetics}{犟}{jiang4}
    \variantof{强}
  \end{phonetics}
\end{entry}

\begin{entry}{磨}{16}[Radical ⽯]
  \begin{phonetics}{磨}{mo2}
    \definition{v.}{moer | polir | afiar | desgastar | esfregar}
  \end{phonetics}
  \begin{phonetics}{磨}{mo4}
    \definition{s.}{mó (pedra pesada e redonda para moinho)}
    \definition{v.}{moer}
  \end{phonetics}
\end{entry}

\begin{entry}{磨菇}{16,11}[Radicais ⽯、⾋]
  \begin{phonetics}{磨菇}{mo2gu5}
    \variantof{蘑菇}
  \end{phonetics}
\end{entry}

\begin{entry}{篮球}{16,11}[Radicais ⽵、⽟]
  \begin{phonetics}{篮球}{lan2qiu2}[][HSK 2]
    \definition[个,只]{s.}{basquetebol}
  \end{phonetics}
\end{entry}

\begin{entry}{糕点}{16,9}[Radicais ⽶、⽕]
  \begin{phonetics}{糕点}{gao1dian3}
    \definition{s.}{bolos | pastéis}
  \end{phonetics}
\end{entry}

\begin{entry}{糕点师}{16,9,6}[Radicais ⽶、⽕、⼱]
  \begin{phonetics}{糕点师}{gao1dian3 shi1}
    \definition{s.}{confeiteiro}
  \end{phonetics}
\end{entry}

\begin{entry}{糕点店}{16,9,8}[Radicais ⽶、⽕、⼴]
  \begin{phonetics}{糕点店}{gao1dian3 dian4}
    \definition{s.}{confeitaria}
  \end{phonetics}
\end{entry}

\begin{entry}{糖}{16}[Radical ⽶]
  \begin{phonetics}{糖}{tang2}[][HSK 3]
    \definition{adj.}{açucarado; em calda}
    \definition[包,斤,勺,袋,块]{s.}{açúcar | doce; bala; bombom}
  \end{phonetics}
\end{entry}

\begin{entry}{糖醋鱼}{16,15,8}[Radicais ⽶、⾣、⿂]
  \begin{phonetics}{糖醋鱼}{tang2cu4yu2}
    \definition{s.}{peixe guisado em molho agridoce (prato)}
  \end{phonetics}
\end{entry}

\begin{entry}{膨胀}{16,8}[Radicais ⾁、⾁]
  \begin{phonetics}{膨胀}{peng2zhang4}
    \definition{v.}{expandir | inflar | inchar}
  \end{phonetics}
\end{entry}

\begin{entry}{蕹菜}{16,11}[Radicais ⾋、⾋]
  \begin{phonetics}{蕹菜}{weng4cai4}
    \definition{s.}{espinafre aquático | \emph{ong choy} | repolho do pântano | convolvulus aquático | glória-da-manhã aquática}
  \seealsoref{空心菜}{kong1xin1cai4}
  \end{phonetics}
\end{entry}

\begin{entry}{薄}{16}[Radical ⾋]
  \begin{phonetics}{薄}{bao2}[][HSK 4]
    \definition{adj.}{fino; frágil; pouca espessura |  frio; indiferente; carente de calor; emocionalmente frio; não profundo | leve; fraco | pobre; infértil}
  \end{phonetics}
  \begin{phonetics}{薄}{bo2}
    \definition{adj.}{ligeiro; escasso; pequeno | mesquinho; pouco generoso; cruel | frívolo; fútil; leviano}
  \end{phonetics}
\end{entry}

\begin{entry}{薯}{16}[Radical ⾋]
  \begin{phonetics}{薯}{shu3}
    \definition{s.}{batata | inhame}
  \end{phonetics}
\end{entry}

\begin{entry}{赞}{16}[Radical ⾙]
  \begin{phonetics}{赞}{zan4}
    \definition{v.}{patrocinar | apoiar | elogiar | (gíria na \emph{Internet}) para curtir (uma postagem \emph{on-line})}
  \end{phonetics}
\end{entry}

\begin{entry}{赞扬}{16,6}[Radicais ⾙、⼿]
  \begin{phonetics}{赞扬}{zan4yang2}
    \definition{v.}{elogiar | aprovar | demonstrar aprovação}
  \end{phonetics}
\end{entry}

\begin{entry}{赞助}{16,7}[Radicais ⾙、⼒]
  \begin{phonetics}{赞助}{zan4zhu4}
    \definition{s.}{patrocinador}
    \definition{v.}{apoiar | auxiliar | patrocinar}
  \end{phonetics}
\end{entry}

\begin{entry}{辩论}{16,6}[Radicais ⾟、⾔]
  \begin{phonetics}{辩论}{bian4lun4}[][HSK 4]
    \definition[场,次]{s.}{debate; argumento; a atividade comportamental em si de argumentar ou refutar diferentes pontos de vista ou afirmações, ou uma ocasião ou situação em que tal argumentação ou refutação é feita}
    \definition{v.}{debater; obter um entendimento unificado ou correto, ambos os lados usam linguagem, palavras etc. para explicar seus pontos de vista, apontar os erros ou as contradições do outro lado}
  \end{phonetics}
\end{entry}

\begin{entry}{避}{16}[Radical ⾌]
  \begin{phonetics}{避}{bi4}[][HSK 4]
    \definition{v.}{evitar; evadir; esquivar-se; buscar abrigo; fugir | impedir; manter afastado; repelir; previnir}
  \end{phonetics}
\end{entry}

\begin{entry}{避免}{16,7}[Radicais ⾌、⼉]
  \begin{phonetics}{避免}{bi4mian3}[][HSK 4]
    \definition{v.}{evitar; desviar; abster-se de; tentar não fazer com que algo aconteça; prevenir; tentar impedir (que algo ruim aconteça) com antecedência}
  \end{phonetics}
\end{entry}

\begin{entry}{镖}{16}[Radical ⾦]
  \begin{phonetics}{镖}{biao1}
    \definition{s.}{dardo | arma de arremesso | mercadorias enviadas sob a proteção de uma escolta armada}
  \end{phonetics}
\end{entry}

\begin{entry}{雕刻}{16,8}[Radicais ⾫、⼑]
  \begin{phonetics}{雕刻}{diao1ke4}
    \definition{s.}{escultura}
    \definition{v.}{esculpir | gravar}
  \end{phonetics}
\end{entry}

\begin{entry}{餐厅}{16,4}[Radicais ⾷、⼚]
  \begin{phonetics}{餐厅}{can1ting1}
    \definition[家]{s.}{restaurante}
    \definition[间]{s.}{sala de jantar}
  \end{phonetics}
\end{entry}

\begin{entry}{鲸鱼}{16,8}[Radicais ⿂、⿂]
  \begin{phonetics}{鲸鱼}{jing1yu2}
    \definition{s.}{baleia}
  \end{phonetics}
\end{entry}

\begin{entry}{鲸鲨}{16,15}[Radicais ⿂、⿂]
  \begin{phonetics}{鲸鲨}{jing1sha1}
    \definition{s.}{tubarão baleia}
  \end{phonetics}
\end{entry}

\begin{entry}{鹦鹉}{16,13}[Radicais ⿃、⿃]
  \begin{phonetics}{鹦鹉}{ying1wu3}
    \definition{s.}{papagaio (ave)}
  \end{phonetics}
\end{entry}

\begin{entry}{默契}{16,9}[Radicais ⿊、⼤]
  \begin{phonetics}{默契}{mo4qi4}
    \definition{adj.}{(de membros da equipe) bem coordenados}
    \definition{s.}{entendimento tácito | entendimento mútuo | conectado em um nível mútuo profundo | (de membros da equipe) bem coordenados}
  \end{phonetics}
\end{entry}

%%%%% EOF %%%%%


%%%
%%% 17画
%%%

\section*{17画}\addcontentsline{toc}{section}{17画}

\begin{entry}{戴}{17}[Radical ⼽]
  \begin{phonetics}{戴}{dai4}
    \definition*{s.}{sobrenome Dai}
    \definition[条]{s.}{área | cinturão | região | zona}
    \definition{v.}{usar/vestir (óculos, gravata, relógio de pulso, luvas) | trazer}
  \end{phonetics}
\end{entry}

\begin{entry}{擦拭}{17,9}[Radicais ⼿、⼿]
  \begin{phonetics}{擦拭}{ca1shi4}
    \definition{v.}{limpar com um pano}
  \end{phonetics}
\end{entry}

\begin{entry}{瞧}{17}[Radical ⽬]
  \begin{phonetics}{瞧}{qiao2}
    \definition{v.}{olhar para | ver | ver (ir a um médico) | visitar}
  \end{phonetics}
\end{entry}

\begin{entry}{窾}{17}[Radical ⽳]
  \begin{phonetics}{窾}{cuan4}
    \definition{v.}{esconder}
  \end{phonetics}
  \begin{phonetics}{窾}{kuan3}
    \definition{adj.}{oco}
    \definition{s.}{rachadura | cavidade | (onomatopéia) água atingindo a rocha}
    \definition{v.}{escavar um buraco}
  \end{phonetics}
\end{entry}

\begin{entry}{糟糕}{17,16}[Radicais ⽶、⽶]
  \begin{phonetics}{糟糕}{zao1gao1}
    \definition{adj.}{muito mau | péssimo}
  \end{phonetics}
\end{entry}

\begin{entry}{螺}{17}[Radical ⾍]
  \begin{phonetics}{螺}{luo2}
    \definition{s.}{concha em espiral | caracol | búzio}
  \end{phonetics}
\end{entry}

\begin{entry}{螺丝}{17,5}[Radicais ⾍、⼀]
  \begin{phonetics}{螺丝}{luo2si1}
    \definition{s.}{parafuso}
  \end{phonetics}
\end{entry}

\begin{entry}{赢}{17}[Radical ⾙]
  \begin{phonetics}{赢}{ying2}[][HSK 3]
    \definition[个]{s.}{ganho (lucro)}
    \definition{v.}{ganhar; derrotar; superar; conquistar}
  \end{phonetics}
\end{entry}

\begin{entry}{辫子}{17,3}[Radicais ⾟、⼦]
  \begin{phonetics}{辫子}{bian4zi5}
    \definition[根,条]{s.}{trança | um erro ou falha que pode ser explorado por um oponente | alça}
  \end{phonetics}
\end{entry}

\begin{entry}{邉}{17}[Radical ⾡]
  \begin{phonetics}{邉}{bian1}
    \variantof{边}
  \end{phonetics}
\end{entry}

\begin{entry}{霜}{17}[Radical ⾬]
  \begin{phonetics}{霜}{shuang1}
    \definition{s.}{geada | pó branco ou creme espalhado por uma superfície | glacê | creme de pele}
  \end{phonetics}
\end{entry}

\begin{entry}{鳄鱼}{17,8}[Radicais ⿂、⿂]
  \begin{phonetics}{鳄鱼}{e4yu2}
    \definition[条]{s.}{jacaré | crocodilo}
  \end{phonetics}
\end{entry}

%%%%% EOF %%%%%


%%%
%%% 18画
%%%

\section*{18画}\addcontentsline{toc}{section}{18画}

\begin{Entry}{嚣}{18}{⼝}
  \begin{Phonetics}{嚣}{xiao1}
    \definition*{s.}{Sobrenome Xiao}
    \definition{adj.}{lazer}
    \definition{v.}{clamar; fazer barulho}
  \end{Phonetics}
\end{Entry}

\begin{Entry}{嚣张}{18,7}{⼝、⼸}
  \begin{Phonetics}{嚣张}{xiao1zhang1}
    \definition{adj.}{desenfreado | arrogante | agressivo}
  \end{Phonetics}
\end{Entry}

\begin{Entry}{懵}{18}{⼼}
  \begin{Phonetics}{懵}{meng3}
    \definition{adj.}{confuso; ignorante; irracional | inconsciente; entorpecido}
  \end{Phonetics}
\end{Entry}

\begin{Entry}{懵懂}{18,15}{⼼、⼼}
  \begin{Phonetics}{懵懂}{meng3dong3}
    \definition{adj.}{confuso | ignorante}
  \end{Phonetics}
\end{Entry}

\begin{Entry}{戳}{18}{⼽}
  \begin{Phonetics}{戳}{chuo1}[][HSK 7-9]
    \definition{s.}{selo; carimbo, abreviação de 戳记}
    \definition{v.}{cutucar; esfaquear | Dialeto: torcer; embotar | Dialeto: ficar em pé}
  \seealsoref{戳记}{chuo1ji4}
  \end{Phonetics}
\end{Entry}

\begin{Entry}{戳记}{18,5}{⼽、⾔}
  \begin{Phonetics}{戳记}{chuo1ji4}
    \definition{s.}{carimbo; selo}
  \end{Phonetics}
\end{Entry}

\begin{Entry}{毉}{18}{⼖}
  \begin{Phonetics}{毉}{yi1}
    \variantof{医}
  \end{Phonetics}
\end{Entry}

\begin{Entry}{瀑}{18}{⽔}
  \begin{Phonetics}{瀑}{bao4}
    \definition{s.}{chuva torrencial; tempestade}
  \end{Phonetics}
  \begin{Phonetics}{瀑}{pu4}
    \definition{s.}{cachoeira; catarata}
  \end{Phonetics}
\end{Entry}

\begin{Entry}{瀑布}{18,5}{⽔、⼱}
  \begin{Phonetics}{瀑布}{pu4bu4}
    \definition{s.}{queda de água | cachoeira | cascata | catarata}
  \end{Phonetics}
\end{Entry}

\begin{Entry}{翻}{18}{⽻}
  \begin{Phonetics}{翻}{fan1}[][HSK 4]
    \definition{v.}{virar; dar a volta; inverter; mudar de posição; torcer; reverter | vasculhar; procurar; pesquisar; mover objetos para localizar algo | reverter; retrair; retirar | passar por cima; ultrapassar; cruzar | multiplicar | traduzir; decodificar | romper-se; cair; desentender-se com alguém}
  \end{Phonetics}
\end{Entry}

\begin{Entry}{翻过}{18,6}{⽻、⾡}
  \begin{Phonetics}{翻过}{fan1guo4}
    \definition{v.}{virar |  transformar}
  \end{Phonetics}
\end{Entry}

\begin{Entry}{翻译}{18,7}{⽻、⾔}
  \begin{Phonetics}{翻译}{fan1yi4}[][HSK 4]
    \definition[个,位,名]{s.}{tradutor; intérprete; pessoas que fazem trabalhos de tradução}
    \definition{v.}{traduzir; interpretar; colocar o significado de palavras de um idioma em palavras de outro idioma (expressão idiomática); expressar um significado em outro idioma}
  \end{Phonetics}
\end{Entry}

\begin{Entry}{翻脸}{18,11}{⽻、⾁}
  \begin{Phonetics}{翻脸}{fan1/lian3}
    \definition{v.+compl.}{brigar com alguém | tornar-se hostil}
  \end{Phonetics}
\end{Entry}

\begin{Entry}{覆}{18}{⾑}
  \begin{Phonetics}{覆}{fu4}
    \definition{v.}{cobrir; encapar | derrubar; perturbar; virar de cabeça para baixo}
  \end{Phonetics}
\end{Entry}

\begin{Entry}{覆盆子}{18,9,3}{⾑、⽫、⼦}
  \begin{Phonetics}{覆盆子}{fu4pen2zi5}
    \definition{s.}{framboesa}
  \end{Phonetics}
\end{Entry}

\begin{Entry}{蹦}{18}{⾜}
  \begin{Phonetics}{蹦}{beng4}[][HSK 7-9]
    \definition{v.}{pular; saltar; quicar}
  \end{Phonetics}
\end{Entry}

\begin{Entry}{蹦极}{18,7}{⾜、⽊}
  \begin{Phonetics}{蹦极}{beng4ji2}
    \definition{s.}{\emph{bungee jumping}}
  \end{Phonetics}
\end{Entry}

\begin{Entry}{鞭}{18}{⾰}
  \begin{Phonetics}{鞭}{bian1}
    \definition[条]{s.}{chicote; oçoite; chibata | um bastão de ferro usado como arma na China antiga | algo parecido com um chicote | uma série de pequenos fogos de artifício | pênis de animal; refere-se ao pênis de certos mamíferos usado para fins medicinais ou comestíveis}
    \definition{v.}{açoitar; chicotear; flagelar}
  \end{Phonetics}
\end{Entry}

\begin{Entry}{鞭炮}{18,9}{⾰、⽕}
  \begin{Phonetics}{鞭炮}{bian1pao4}[][HSK 7-9]
    \definition[串,挂,盒,捆,箱,个]{s.}{\emph{maroon}, um tipo de foguete usado como alarme ou aviso; fogos de artifício; um termo geral para fogos de artifício grandes e pequenos}
  \end{Phonetics}
\end{Entry}

\begin{Entry}{鞭策}{18,12}{⾰、⽵}
  \begin{Phonetics}{鞭策}{bian1ce4}[][HSK 7-9]
    \definition{v.}{estimular; incitar; incentivar}
  \end{Phonetics}
\end{Entry}

\begin{Entry}{鼂}{18}{⿌}
  \begin{Phonetics}{鼂}{chao2}
    \definition*{s.}{Sobrenome Chao}
    \definition{s.}{tartaruga marinha}
  \end{Phonetics}
\end{Entry}

%%%%% EOF %%%%%


%%%
%%% 19画
%%%

\section*{19画}\addcontentsline{toc}{section}{19画}

\begin{Entry}{巅}{19}{⼭}
  \begin{Phonetics}{巅}{dian1}
    \definition[个]{s.}{pico da montanha; cume; topo da montanha}
  \end{Phonetics}
\end{Entry}

\begin{Entry}{巅峰}{19,10}{⼭、⼭}
  \begin{Phonetics}{巅峰}{dian1feng1}[][HSK 7-9]
    \definition{s.}{um cume; um pico de montanha}
  \end{Phonetics}
\end{Entry}

\begin{Entry}{攀}{19}{⼿}
  \begin{Phonetics}{攀}{pan1}
    \definition{v.}{escalar; escalar | buscar conexões em altos cargos | envolver; implicar | agarrar; agarrar-se; segurar-se a}
  \end{Phonetics}
\end{Entry}

\begin{Entry}{攀岩}{19,8}{⼿、⼭}
  \begin{Phonetics}{攀岩}{pan1yan2}
    \definition{s.}{alpinista}
    \definition{v.}{escalar uma montanha}
  \end{Phonetics}
\end{Entry}

\begin{Entry}{攀爬}{19,8}{⼿、⽖}
  \begin{Phonetics}{攀爬}{pan1pa2}
    \definition{v.}{escalar}
  \end{Phonetics}
\end{Entry}

\begin{Entry}{曝}{19}{⽇}
  \begin{Phonetics}{曝}{bao4}
    \definition{v.}{usado em  曝光}
  \seealsoref{曝光}{bao4/guang1}
  \end{Phonetics}
  \begin{Phonetics}{曝}{pu4}
    \definition{v.}{expor ao sol}
  \end{Phonetics}
\end{Entry}

\begin{Entry}{曝光}{19,6}{⽇、⼉}
  \begin{Phonetics}{曝光}{bao4/guang1}[][HSK 7-9]
    \definition{v.+compl.}{expor; sensibilizar filme fotográfico ou papel fotossensível para formar uma imagem latente | expor; tornar (algo ruim) público; metáfora para revelar coisas secretas (geralmente vergonhosas) ao mundo}
  \end{Phonetics}
\end{Entry}

\begin{Entry}{爆}{19}{⽕}
  \begin{Phonetics}{爆}{bao4}[][HSK 6]
    \definition{v.}{explodir; estourar | fritar rapidamente; ferver rapidamente | aparecer (ou ocorrer) inesperadamente}
  \end{Phonetics}
\end{Entry}

\begin{Entry}{爆发}{19,5}{⽕、⼜}
  \begin{Phonetics}{爆发}{bao4fa1}[][HSK 6]
    \definition{v.}{entrar em erupção; explodir | estourar; irromper; ocorrer de forma repentina e violenta}
  \end{Phonetics}
\end{Entry}

\begin{Entry}{爆竹}{19,6}{⽕、⽵}
  \begin{Phonetics}{爆竹}{bao4 zhu2}[][HSK 7-9]
    \definition[串,个]{s.}{fogo de artifício}
  \end{Phonetics}
\end{Entry}

\begin{Entry}{爆米花}{19,6,7}{⽕、⽶、⾋}
  \begin{Phonetics}{爆米花}{bao4mi3hua1}
    \definition{s.}{pipoca (de milho) | pipoca de arroz}
  \end{Phonetics}
\end{Entry}

\begin{Entry}{爆冷门}{19,7,3}{⽕、⼎、⾨}
  \begin{Phonetics}{爆冷门}{bao4 leng3men2}[][HSK 7-9]
    \definition{s.}{um avanço | uma reviravolta (especialmente nos esportes) | reviravolta inesperada dos acontecimentos}
    \definition{v.}{dar um golpe}
  \end{Phonetics}
\end{Entry}

\begin{Entry}{爆炸}{19,9}{⽕、⽕}
  \begin{Phonetics}{爆炸}{bao4zha4}[][HSK 6]
    \definition{s.}{explosão}
    \definition{v.}{explodir; explodir; detonar | aumentar bruscamente em um curto espaço de tempo (de quantidade)}
  \end{Phonetics}
\end{Entry}

\begin{Entry}{爆满}{19,13}{⽕、⽔}
  \begin{Phonetics}{爆满}{bao4man3}[][HSK 7-9]
    \definition{v.}{(teatro, cinema, estádio, etc.) lotar; ter casa cheia | estar lotado}
  \end{Phonetics}
\end{Entry}

\begin{Entry}{聼}{19}{⼼}
  \begin{Phonetics}{聼}{ting1}
    \variantof{听}
  \end{Phonetics}
\end{Entry}

\begin{Entry}{蘑}{19}{⾋}
  \begin{Phonetics}{蘑}{mo2}
    \definition{s.}{cogumelo}
  \end{Phonetics}
\end{Entry}

\begin{Entry}{蘑菇}{19,11}{⾋、⾋}
  \begin{Phonetics}{蘑菇}{mo2gu5}
    \definition{s.}{cogumelo}
    \definition{v.}{mandriar | embromar | amofinar | incomodar alguém com solicitações ou interrupções frequentes ou persistentes}
  \end{Phonetics}
\end{Entry}

\begin{Entry}{警}{19}{⾔}
  \begin{Phonetics}{警}{jing3}
    \definition{s.}{policial}
    \definition{v.}{alertar | avisar}
  \end{Phonetics}
\end{Entry}

\begin{Entry}{警告}{19,7}{⾔、⼝}
  \begin{Phonetics}{警告}{jing3gao4}[][HSK 5]
    \definition[个]{s.}{advertência (como medida disciplinar); uma forma de punição}
    \definition{v.}{avisar; advertir; admoestar}
  \end{Phonetics}
\end{Entry}

\begin{Entry}{警官}{19,8}{⾔、⼧}
  \begin{Phonetics}{警官}{jing3guan1}
    \definition[名]{s.}{polícia | policial}
  \end{Phonetics}
\end{Entry}

\begin{Entry}{警察}{19,14}{⾔、⼧}
  \begin{Phonetics}{警察}{jing3cha2}[][HSK 3]
    \definition[个,位,名,群,队]{s.}{polícia; policial; oficial de polícia; as forças armadas que mantêm a segurança social do país são uma parte importante do aparato estatal; também se refere aos membros dessas forças armadas}
  \end{Phonetics}
\end{Entry}

\begin{Entry}{蹬}{19}{⾜}
  \begin{Phonetics}{蹬}{deng1}[][HSK 7-9]
    \definition{v.}{pressionar com o pé; pisar; pisar em | Dialeto: calçar (sapatos ou calças); usar (sapatos) | Gíria: despejar (algo)}
  \end{Phonetics}
  \begin{Phonetics}{蹬}{deng4}
    \definition{s.}{lutar; ter dificuldade}
  \seealsoref{蹭蹬}{ceng4deng4}
  \end{Phonetics}
\end{Entry}

\begin{Entry}{蹭}{19}{⾜}
  \begin{Phonetics}{蹭}{ceng4}[][HSK 7-9]
    \definition{v.}{esfregar; raspar; arranhar | esfregar em algo e ficar manchado; ser manchado com; manchar por fricção | mover-se lentamente; demorar-se; arrastar-se | Dialeto: roubar}
  \end{Phonetics}
\end{Entry}

\begin{Entry}{蹭蹬}{19,19}{⾜、⾜}
  \begin{Phonetics}{蹭蹬}{ceng4deng4}
    \definition{interj.}{Droga!}
    \definition{v.}{enfrentar contratempos; estar sem sorte; ter má sorte}
  \end{Phonetics}
\end{Entry}

\begin{Entry}{蹲}{19}{⾜}
  \begin{Phonetics}{蹲}{dun1}[][HSK 6]
    \definition{v.}{agachamento sobre os calcanhares; dobrar as pernas o máximo possível, como se estivesse sentado, mas não deixar as nádegas tocarem o chão | ficar; metáfora para ficar ocioso em casa}
  \end{Phonetics}
\end{Entry}

\begin{Entry}{蹲下}{19,3}{⾜、⼀}
  \begin{Phonetics}{蹲下}{dun1xia4}
    \definition{v.}{agachar | agachar-se}
  \end{Phonetics}
\end{Entry}

\begin{Entry}{颤}{19}{⾴}
  \begin{Phonetics}{颤}{chan4}
    \definition{v.}{tremer; estremecer | vibrar; tremer; sacudir}
  \end{Phonetics}
\end{Entry}

\begin{Entry}{颤抖}{19,7}{⾴、⼿}
  \begin{Phonetics}{颤抖}{chan4dou3}[][HSK 7-9]
    \definition{v.}{tremer; estremecer; tremular; tiritar}
  \end{Phonetics}
\end{Entry}

%%%%% EOF %%%%%


%%%
%%% 20画
%%%

\section*{20画}\addcontentsline{toc}{section}{20画}

\begin{Entry}{壤}{20}{⼟}
  \begin{Phonetics}{壤}{rang3}
    \definition{s.}{solo | terra | (literário) a terra (em contraste com o céu 天)}
  \end{Phonetics}
\end{Entry}

\begin{Entry}{譬}{20}{⾔}
  \begin{Phonetics}{譬}{pi4}
    \definition{s.}{exemplo; analogia; metáfora}
    \definition{v.}{dar um exemplo; fazer uma analogia}
  \end{Phonetics}
\end{Entry}

\begin{Entry}{譬如}{20,6}{⾔、⼥}
  \begin{Phonetics}{譬如}{pi4ru2}
    \definition{conj.}{por exemplo | como}
  \end{Phonetics}
\end{Entry}

\begin{Entry}{魔}{20}{⿁}
  \begin{Phonetics}{魔}{mo2}
    \definition{adj.}{místico; misterioso; mágico}
    \definition{s.}{espírito maligno; demônio; diabo; monstro | mágico; místico}
  \end{Phonetics}
\end{Entry}

\begin{Entry}{魔头}{20,5}{⿁、⼤}
  \begin{Phonetics}{魔头}{mo2tou2}
    \definition{s.}{monstro | diabo}
  \end{Phonetics}
\end{Entry}

\begin{Entry}{魔术}{20,5}{⿁、⽊}
  \begin{Phonetics}{魔术}{mo2shu4}
    \definition{s.}{magia}
  \end{Phonetics}
\end{Entry}

%%%%% EOF %%%%%


%%%
%%% 21画
%%%

\section*{21画}\addcontentsline{toc}{section}{21画}

\begin{entry}{露珠}{21,10}{⾬、⽟}
  \begin{phonetics}{露珠}{lu4zhu1}
    \definition{s.}{orvalho}
  \end{phonetics}
\end{entry}

\begin{entry}{霸}{21}{⾬}
  \begin{phonetics}{霸}{ba4}
    \definition*{s.}{sobrenome Ba}
    \definition{adj.}{arrogante; dominador; tirânico}
    \definition{s.}{líder dos senhores feudais; suserano | tirano; déspota; valentão; \emph{bully} | poder hegemônico; hegemonismo; hegemonia | chefe dos príncipes feudais; líder da antiga aliança feudal}
    \definition{v.}{dominar; tiranizar; governar (ocupar) pela força}
  \end{phonetics}
\end{entry}

\begin{entry}{霸权}{21,6}{⾬、⽊}
  \begin{phonetics}{霸权}{ba4quan2}
    \definition{s.}{hegemonia | supremacia}
  \end{phonetics}
\end{entry}

\begin{entry}{鷄}{21}{⿃}
  \begin{phonetics}{鷄}{ji1}
    \variantof{鸡}
  \end{phonetics}
\end{entry}

%%%%% EOF %%%%%


%\input{groups_by_strokes/022.tex}
%%%%
%%% 23画
%%%

\section*{23画}\addcontentsline{toc}{section}{23画}

\begin{Entry}{罐}{23}{⽸}
  \begin{Phonetics}{罐}{guan4}[][HSK 7-9]
    \definition{clas.}{lata; jarra; gavetas e recipientes de água feitos de cerâmica ou metal}[我买了一罐可乐。===Comprei uma lata de Coca-Cola.]
    \definition{s.}{lata; jarra; jarro; pote; tanque | cuba de carvão; vagão de caçamba para carregamento de carvão em minas de carvão}
  \end{Phonetics}
\end{Entry}

\begin{Entry}{罐头}{23,5}{⽸、⼤}
  \begin{Phonetics}{罐头}{guan4tou5}[][HSK 7-9]
    \definition[个,盒,瓶]{s.}{lata; jarra | enlatado; comida enlatada é a abreviação de 罐头食品, que é processada e embalada em latas de ferro seladas ou garrafas de vidro, e pode ser armazenada por um longo tempo}
  \seealsoref{罐头食品}{guan4tou2 shi2pin3}
  \end{Phonetics}
\end{Entry}

\begin{Entry}{罐头食品}{23,5,9,9}{⽸、⼤、⾷、⼝}
  \begin{Phonetics}{罐头食品}{guan4tou2 shi2pin3}
    \definition{s.}{alimentos enlatados; produtos enlatados}
  \end{Phonetics}
\end{Entry}

%%%%% EOF %%%%%


%\input{groups_by_strokes/024.tex}
%\input{groups_by_strokes/025.tex}
%\input{groups_by_strokes/026.tex}
%\input{groups_by_strokes/027.tex}
%\input{groups_by_strokes/028.tex}
%\input{groups_by_strokes/029.tex}
%\input{groups_by_strokes/030.tex}
\onecolumn

\ifdraftdoc 
%%%
\else

\clearpage
\pagestyle{plain}
\chapter{Termos Gramaticais Chineses}
\begin{center}
\begin{tblr}[m]{lll}
substantivo/nome  & \textbf{s.}        & 名词                     \\
pronome           & \textbf{pron.}     & 代词                     \\
numeral           & \textbf{num.}      & 数词                     \\
classificador     & \textbf{clas.}     & 量词                     \\
verbo             & \textbf{v.}        & 动词                     \\
verbo auxiliar    & \textbf{v.aux.}    & 助动词                   \\
verbo+complemento & \textbf{v.+compl.} & 动宾式\hspace{1em}离合词 \\
adjetivo          & \textbf{adj.}      & 形容词                   \\
advérbio          & \textbf{adv.}      & 副词                     \\
preposição        & \textbf{prep.}     & 介词                     \\
conjunção         & \textbf{conj.}     & 连词                     \\
partícula         & \textbf{part.}     & 助词                     \\
interjeição       & \textbf{interj.}   & 叹词                     \\
prefixo           & \textbf{pref.}     & 前缀                     \\
sufixo            & \textbf{suf.}      & 后缀                     \\
expressão         & \textbf{expr.}     & 熟语                     \\
\end{tblr}
\end{center}


\clearpage
\pagestyle{plain}
\chapter{Classificadores Nominais}
\DefTblrTemplate{caption}{default}{}
\DefTblrTemplate{capcont}{default}{ \UseTblrTemplate{conthead-text}{default} }
\DefTblrTemplate{contfoot-text}{default}{Continua na próxima página.}
\DefTblrTemplate{conthead-text}{default}{(Continuação)}
\DefTblrTemplate{firsthead}{default}{ \UseTblrTemplate{caption}{default} }
\DefTblrTemplate{middlehead,lasthead}{default}{ \UseTblrTemplate{conthead}{default} }
\DefTblrTemplate{firstfoot,middlefoot}{default}{ \UseTblrTemplate{contfoot}{default} }
\DefTblrTemplate{lastfoot}{default}{ \UseTblrTemplate{note}{default} \UseTblrTemplate{remark}{default} }

\begin{longtblr}
{
  colspec = {|c|c|X|X|}, hlines,
  width = 1\linewidth,
  rowhead = 1, rowfoot = 0,
  row{1} = {font=\bfseries, fg=white, bg=black},
}
\textbf{Hanzi} & \textbf{Pinyin} & \textbf{Descrição} & \textbf{Exemplos}\\
    把 & \dpy{ba3}     & mão cheia --- objetos que podem ser segurados, objetos relativamente longos e planos & faca, tesoura, espada, chave, guarda-chuva, leque, escova de dentes, colher, garfo, martelo, cadeado, pistola, rifle, carne, punhado de arroz, punhado de areia, ramo de flores, punhado de sementes, ramo de pauzinhos, esqueleto, fogo, bule\\
    班 & \dpy{ban1}    & serviços programados (trens, aviões, etc), grupos de pessoas, uma classe como em alunos & \\
    包 & \dpy{bao1}    & pacote & doces, cigarros, açúcar, biscoitos\\
    杯 & \dpy{bei1}    & copo --- bebidas & chá, vinho, álcool, água, leite, suco de fruta, refrigerante\\
    本 & \dpy{ben3}    & volume --- material impresso encadernado & livro, revista, romance, escritura, dicionário, bloco de notas, livro didático\\
    笔 & \dpy{bi3}     & traços de caracteres; grandes quantidades de dinheiro & \\
    部 & \dpy{bu4}     & máquinas, veículos; produções & celular, telefone, carro, jogo, romance, filme/imagem em movimento, ópera, obra literária\\
    册 & \dpy{ce4}     & volumes de livros & \\
    层 & \dpy{ceng2}   & andar, piso; camada & andares (em um prédio); camada de poeira, bolo, tinta\\
    场 & \dpy{chang3}  & evento de curta duração; precipitação & espetáculo público, jogo, crise, guerra, catástrofe, uma doença, performance, jogo, chuva, queda de neve\\
    串 & \dpy{chuan4}  & conjuntos de números & telefone celular/número de celular\\
    床 & \dpy{chuang4} & cama & cobertores, lençóis\\
    次 & \dpy{ci4}     & tempo, repetições --- oportunidades, acidentes & \\
    出 & \dpy{chu1}    & atuação em uma peça & \\
    打 & \dpy{da2}     & dúzia & lápis, ovos\\
    贷 & \dpy{dai4}    & sacos ou bolsos cheios & açúcar, farinha, arroz\\
    道 & \dpy{dao4}    & projeções lineares (raios de luz, etc.); ordem dada por uma figura de autoridade; pergunta, memorando; curso (de comida); coisas longas e tortas (cume da montanha, relâmpago) & \\
    滴 & \dpy{di1}     & gotículas de água, sangue, outros fluidos semelhantes & \\
    点 & \dpy{dian3}   & ideias, sugestões; um pouco/algum (somente com 一) & \\
    碟 & \dpy{die2}    & pires (molho de soja) & \\
    顶 & \dpy{ding3}   & objetos com topo saliente, algo para colocar sobre a cabeça & chapéu, barraca, mosquiteiro\\
    栋 & \dpy{dong4}   & pilar (edifício menor, casa) & \\
    堵 & \dpy{du3}     & luminárias abrangentes (parede sem teto) & \\
    段 & \dpy{duan4}   & comprimento adjacente --- cabos, estradas, pedaço de giz, parte de uma música & \\
    对 & \dpy{dui4}    & casal, par combinado (para certas coisas), dísticos & casal, brincos, vasos\\
    顿 & \dpy{dun4}    & ações de curta duração & refeição, conflito, espancamento, briga, repreensão\\
    朵 & \dpy{duo3}    & coisas parecidas com flores & flor, nuvem, cogumelo\\
    发 & \dpy{fa1}     & coisas redondas & bala, munição\\
    方 & \dpy{fang4}   & coisas quadradas & pedra de tinta, bacon\\
    份 & \dpy{fen4}    & porções, documentos de várias páginas & porção de comida, jornal, emprego\\
    封 & \dpy{feng1}   & coisas que podem ser seladas & carta, correio, telegrama\\
    峰 & \dpy{feng1}   & animais com corcundas & camelo\\
    幅 & \dpy{fu2}     & objetos semelhantes a imagens & foto, pintura, desenho, banner, obra de arte, quadro, cartaz\\
    服 & \dpy{fu4}     & dose (de remédio)\\
    副 & \dpy{fu4}     & objetos que vêm em pares (luvas, óculos, etc.), baralhos, mahjong & \\
    个 & {\dpy{ge5}\\ \dpy{ge4}} & coisas individuais, pessoas, classificador de uso geral (o uso desse classificador em conjunto com qualquer substantivo é geralmente aceito se a pessoa não souber o classificador adequado) & pessoa, irmão mais velho, estudante, parente, modo de pensar, sugestão, pergunta, nação\\
    根 & \dpy{gen1}    & objetos finos e esguios; fios finos e flexíveis & agulha, pilar, banana, palito de massa frita, coxa de frango frita, picolé, pirulito, pauzinho, vela, incenso, cabelo, linha, barbante\\
    股 & \dpy{gu3}     & fluxos (de ar, cheiro, influência, etc.) & \\
    挂 & \dpy{gua4}    & coisas multi-componentes & cavalo e carroça \\
    罐 & \dpy{guan4}   & lata pequena a média & refrigerante, suco, comida, feijão, óleo, doce\\
    行 & \dpy{hang2}   & objetos que formam linhas (palavras, etc.) & \\
    盒 & \dpy{he2}     & caixa pequena & fita, comida, bolo, doces, chocolate, brinquedos, livros, cigarros, detergente, roupas\\
    户 & \dpy{hu4}     & famílias & \\
    伙 & \dpy{huo3}    & classificador geralmente depreciativo para bandos de pessoas, como gangues ou bandidos & \\
    家 & \dpy{jia1}    & reunião de pessoas, estabelecimentos & famílias, empresas, lojas, restaurantes\\
    架 & \dpy{jia4}    & maquinário, veículo & aeronave, avião, piano, máquinas, computadores\\
    间 & \dpy{jian1}   & quartos, espaços & quarto, dormitório, cozinha, sala de aula, casa, escola, empresa, capela\\
    件 & \dpy{jian4}   & assuntos, roupas (tops), móveis & roupa (top), camiseta, camisa, casaco, lençol, bagagem, presente, questão/matéria/coisa\\
    讲 & \dpy{jiang3}  & longos períodos de aula & \\
    节 & \dpy{jie2}    & seção (de bambu, período curto de aula) & \\
    届 & \dpy{jie4}    & sessões ou reuniões agendadas regularmente, grupos de anos em uma escola (por exemplo, Turma de 2025) & \\
    句 & \dpy{ju4}     & linhas de poemas, frases & \\
    棵 & \dpy{ke1}     & árvores ou outra flora semelhante & pinheiro, rosa\\
    颗 & \dpy{ke1}     & pequenos objetos, objetos que parecem pequenos & corações, pérolas, dentes, diamantes, sementes, estrelas distantes, planetas distantes\\
    课 & \dpy{ke4}     & lições em um texto & \\
    口 & \dpy{kou3}    & população de aldeias (número inferior a 100), familiares, poços, bocados de comida & \\
    块 & \dpy{kuai4}   & pedaço de forma irregular & terra, pedra, tofu, sabonete, pedaço/fatia de bolo, pão (não fatias), melancia, carne, queijo, pizza, chiclete, toalha de mesa, relógio de pulso, bloco de incenso\\
    类 & \dpy{lei4}    & objetos do mesmo tipo ou natureza & \\
    粒 & \dpy{li4}     & grão & um (único) amendoim, uva, arroz cru, semenete, doce, bala, chocolate\\
    辆 & \dpy{liang4}  & veículos com rodas (não trens) & automóvel, bicicleta, carro\\
    列 & \dpy{lie4}    & trens & \\
    轮 & \dpy{lun2}    & lua & \\
    枚 & \dpy{mei2}    & medalhas, pequenas coisas planas como selos, cascas de banana, anéis, distintivos, foguetes, mísseis & \\
    门 & \dpy{men2}    & objetos pertencentes a acadêmicos (cursos, disciplinas, etc.); artilharia (canhão) & \\
    面 & \dpy{mian4}   & objetos planos e lisos & espelho, bandeira, parede com telhado, escudo\\
    名 & \dpy{ming4}   & pessoas de alto escalão (médicos, advogados, políticos, realeza, etc.), membros; em linguagem formal pode ser utilizado para qualquer pessoa não necessariamente de alto escalão & \\
    排 & \dpy{pai2}    & linhas --- objetos agrupados em linhas & cadeiras, assentos, mesas, filas de pessoas\\
    盘 & \dpy{pan2}    & objetos planos ou bobinas & cassete de vídeo ou áudio, pratinho, bobina de incenso\\
    批 & \dpy{pi1}     & pessoas, bens, etc. & \\
    匹 & \dpy{pi3}     & cavalos e outras montarias; rolos/pedaços de panos & cavalo, lobo\\
    篇 & \dpy{pian1}   & escritos & papel, artigo, ensaio, relatório\\
    片 & \dpy{pian4}   & fatia - objetos finos, planos, às vezes irregulares & cartão, lábio, nuvem, praia, chiclete, fatia de pão, fatia de carne, biscoito, queijo, fatia de melancia, folha, pétala de flor, campo, lago, pílula (comprimido de remédio), DVD\\
    瓶 & \dpy{ping2}   & garrafa & álcool, água, óleo, cerveja, bebida, vinho, refrigerante, leite, shampoo\\
    期 & \dpy{qi1}     & revistas & \\
    群 & \dpy{qun2}    & grupo (incl. pessoas), rebanho, multidão, enxame, etc. & pessoas, rebanho de gado, bando de pássaros, bando de cães, enxame de mosquitos, colônia de abelhas/formigas\\
    任 & \dpy{ren4}    & mandato (presidente, senador, deputado, etc.) & \\
    扇 & \dpy{shan4}   & coisas que abrem e fecham com dobradiças & janela, porta\\
    首 & \dpy{shou1}   & coisas com versos & canção, poema \\
    束 & \dpy{shu4}    & cachos & flores, uvas \\
    双 & \dpy{shuang1} & par de objetos que naturalmente vêm em pares & pauzinhos, sapatos, luvas, olhos\\
    艘 & \dpy{sou1}    & navios & \\
    所 & \dpy{suo3}    & pequenos edifícios, instituições & universidade, casa independente\\
    台 & \dpy{tai2}    & objetos pesados, especialmente máquinas; apresentações & TV, computador, piano, aparelho, avião, trem, carro, elevador; apresentação de teatro, jogo\\
    躺 & \dpy{tang2}   & períodos de aulas, suítes de imóveis & aulas, lições, leituras\\
    趟 & \dpy{tang4}   & viagens, serviços de transportes programados & \\
    套 & \dpy{tao4}    & conjuntos & livros, revistas, colecionáveis, roupas, casas/apartamentos com vários cômodos, suítes, selos, móveis, quartos, ternos\\
    题 & \dpy{ti2}     & classificador de perguntas & \\
    条 & \dpy{tiao2}   & objetos longos, estreitos e flexíveis & peixe, cobra, dragão, minhoca, cachorro, cachecol, estrada, fita, rio, raiz, caule, corda, edredom, toalha, fio dental, calças, gravata, saia, sofá/banco, pessoa heróica, barco pequeno\\
    帖 & \dpy{tie4}    & bandagens adesivas & \\
    通 & \dpy{tong1}   & conversa, palestra & \\
    桶 & \dpy{tong3}   & jarro, balde, barril & jarro de leite, barril de óleo\\
    头 & \dpy{tou2}    & cabeça de gado & porco, vaca, bois, iaques, ovelhas, burros, mulas, leopardos, dinossauros\\
    团 & \dpy{tuan2}   & bola --- objetos redondos e enrolados & bola de lã, etc. \\
    碗 & \dpy{wan3}    & tigela & de arroz, de macarrão, de sopa\\
    位 & \dpy{wei4}    & classificador educado e respeitoso para pessoas & professor, cliente, colega\\
    项 & \dpy{xiang4}  & projetos & \\
    些 & \dpy{xie1}    & alguns & somente com 一,这,那,哪\\
    样 & \dpy{yang4}   & itens gerais de diferentes atributos & \\
    页 & \dpy{ye4}     & página & \\
    则 & \dpy{ze2}     & diário, registro do dia & \\
    扎 & \dpy{zha1}    & jarra, caneca --- usado em cantonês no lugar de 束 \dpy{shu4} (por exemplo, um pacote de flores) & bebidas como cerveja, refrigerante, suco, etc. (pint/jar: empréstimo linguístico do inglês, pode ser considerado informal ou gíria)\\
    盏 & \dpy{zhan3}   & luminárias (geralmente lâmpadas), bule de chá, etc. & \\
    站 & \dpy{zhan4}   & paradas (de ônibus ou trens) & \\
    张 & \dpy{zhang1}  & folha --- objetos planos ou de papel & papel, mesa, cama, cartão, sofá, CD/DVD, guardanapo, fotografia, ingresso, pintura, constelação, rosto, boca, arco\\
    阵 & \dpy{zhen4}   & rajada, estouro --- eventos com duração curtas & tempestades com raios, rajadas de vento, ocorrência de chuva\\
    支 & \dpy{zhi1}    & objetos bastante longos, semelhantes a bastões & caneta, lápis, pauzinho, canudo, bambu, rosa, rifle, flecha, lança, projétil de artilharia, míssil, cantigas\\
    只 & \dpy{zhi1}    & um de um par; animais & mão, dedo, olho, pé, cabeça, meia, luva, sapato, brinco, óculos; pássaro, frango, gato, tigre, cachorro, macaco, elefante, ovelha, rato, borboleta, rã, inseto\\
    种 & \dpy{zhong3}  & tipos & de coisas, de livros, de pessoas\\
    株 & \dpy{zhu1}    & árvores/plantas menores & arbusto, planta de arroz, planta de trigo\\
    幢 & \dpy{zhuang2} & edifício de vários andares & \\
    组 & \dpy{zu3}     & conjuntos, linhas, séries, baterias (militares) & \\
    座 & \dpy{zuo4}    & montanha, edifício & montanha, colina, estrutura, grande edifício, cidade, ponte, vila, arranha-céu, torre, templo, bloco de apartamentos\\
\end{longtblr}


\clearpage
\pagestyle{plain}
\chapter{Classificadores Verbais}
\DefTblrTemplate{caption}{default}{}
\DefTblrTemplate{capcont}{default}{ \UseTblrTemplate{conthead-text}{default} }
\DefTblrTemplate{contfoot-text}{default}{Continua na próxima página.}
\DefTblrTemplate{conthead-text}{default}{(Continuação)}
\DefTblrTemplate{firsthead}{default}{ \UseTblrTemplate{caption}{default} }
\DefTblrTemplate{middlehead,lasthead}{default}{ \UseTblrTemplate{conthead}{default} }
\DefTblrTemplate{firstfoot,middlefoot}{default}{ \UseTblrTemplate{contfoot}{default} }
\DefTblrTemplate{lastfoot}{default}{ \UseTblrTemplate{note}{default} \UseTblrTemplate{remark}{default} }

\begin{longtblr}
{
  colspec = {|c|c|X|X|}, hlines,
  width = 1\linewidth,
  rowhead = 1, rowfoot = 0,
  row{1} = {font=\bfseries, fg=white, bg=black},
}
\textbf{Hanzi} & \textbf{Pinyin} & \textbf{Descrição}\\
    遍 & \dpy{bian4}  & o número de vezes que uma ação foi concluída \\
    场 & \dpy{chang3} & a duralção de um evento ocorrendo dentro de outro evento\\
    次 & \dpy{ci4}    & vezes (ao contrário de 遍, 次 refere-se ao número de vezes, independente de ter sido concluído ou não)\\
    顿 & \dpy{dun4}   & ações sem repetição\\
    回 & \dpy{hui2}   & ocorrências (usado coloquialmente)\\
    声 & \dpy{sheng1} & gritos, expressões\\
    趟 & \dpy{tang4}  & viagens, visitas\\
    下 & \dpy{xia4}   & ações breves e frequentemente repentinas, muito mais comum em cantonês do quem em dialetos do norte\\
\end{longtblr}


\clearpage
\pagestyle{plain}
\chapter{Verbos Direcionais}
%%%%%%%%%%%%%%%%%%%%%%%%%%%%%%%%%%%%%%%%%%%%%%%%%%%%%%%%%%%%%%%%%%%%%%%%%%%%%%%
%%%%%%%%%%%%%%%%%%%%%%%%%%%%%%%%%%%%%%%%%%%%%%%%%%%%%%%%%%%%%%%%%%%%%%%%%%%%%%%
%%%%%                                                                     %%%%%
%%%%% verbos_direcionais.tex:                                             %%%%%
%%%%% Tabela com os verbos direcionais chineses.                          %%%%%
%%%%%                                                                     %%%%%
%%%%%%%%%%%%%%%%%%%%%%%%%%%%%%%%%%%%%%%%%%%%%%%%%%%%%%%%%%%%%%%%%%%%%%%%%%%%%%%
%%%%%%%%%%%%%%%%%%%%%%%%%%%%%%%%%%%%%%%%%%%%%%%%%%%%%%%%%%%%%%%%%%%%%%%%%%%%%%%

%%% Ajustes para a tabela
\DefTblrTemplate{caption}{default}{}
\DefTblrTemplate{capcont}{default}{ \UseTblrTemplate{conthead-text}{default} }
\DefTblrTemplate{contfoot-text}{default}{Continua na próxima página.}
\DefTblrTemplate{conthead-text}{default}{(Continuação)}
\DefTblrTemplate{firsthead}{default}{ \UseTblrTemplate{caption}{default} }
\DefTblrTemplate{middlehead,lasthead}{default}{ \UseTblrTemplate{conthead}{default} }
\DefTblrTemplate{firstfoot,middlefoot}{default}{ \UseTblrTemplate{contfoot}{default} }
\DefTblrTemplate{lastfoot}{default}{ \UseTblrTemplate{note}{default} \UseTblrTemplate{remark}{default} }

%%% Tabela
\begin{longtblr}
{
  colspec = {cccccccc},
  width = 1\linewidth,
  hlines = {white},
  vlines = {white},
  rowhead = 1, rowfoot = 0,
  row{1} = {font=\bfseries, bg=gray8, fg=black},
  column{1} = {font=\bfseries, bg=gray8, fg=black},
  cell{1}{1} = {bg=white},
  cell{2-Z}{2-Z} = {bg=gray9},
  cell{3}{8} = {bg=white},
}
 & {上\\ \normalsize descer} & {下\\ \normalsize subir} & {进\\ \normalsize entrar} & {出\\ \normalsize sair} & {回\\ \normalsize retornar} & {过\\ \normalsize atravessar} & {起\\ \normalsize levantar} \\
{来\\ \normalsize vir}  &  上来 &  下来 &  进来 &  出来 &  回来 &  过来 &  起来 \\
{去\\ \normalsize ir }  &  上去 &  下去 &  进去 &  出去 &  回去 &  过去 &  \\ 
\end{longtblr}

%%%%% EOF %%%%%


\clearpage
\pagestyle{plain}
\chapter{Locativos}
%%%%%%%%%%%%%%%%%%%%%%%%%%%%%%%%%%%%%%%%%%%%%%%%%%%%%%%%%%%%%%%%%%%%%%%%%%%%%%%
%%%%%%%%%%%%%%%%%%%%%%%%%%%%%%%%%%%%%%%%%%%%%%%%%%%%%%%%%%%%%%%%%%%%%%%%%%%%%%%
%%%%%                                                                     %%%%%
%%%%% locativos.tex:                                                      %%%%%
%%%%% Tabela com os locativos chineses                                    %%%%%
%%%%%                                                                     %%%%%
%%%%%%%%%%%%%%%%%%%%%%%%%%%%%%%%%%%%%%%%%%%%%%%%%%%%%%%%%%%%%%%%%%%%%%%%%%%%%%%
%%%%%%%%%%%%%%%%%%%%%%%%%%%%%%%%%%%%%%%%%%%%%%%%%%%%%%%%%%%%%%%%%%%%%%%%%%%%%%%

%%% Ajustes para a tabela
\DefTblrTemplate{caption}{default}{}
\DefTblrTemplate{capcont}{default}{ \UseTblrTemplate{conthead-text}{default} }
\DefTblrTemplate{contfoot-text}{default}{Continua na próxima página.}
\DefTblrTemplate{conthead-text}{default}{(Continuação)}
\DefTblrTemplate{firsthead}{default}{ \UseTblrTemplate{caption}{default} }
\DefTblrTemplate{middlehead,lasthead}{default}{ \UseTblrTemplate{conthead}{default} }
\DefTblrTemplate{firstfoot,middlefoot}{default}{ \UseTblrTemplate{contfoot}{default} }
\DefTblrTemplate{lastfoot}{default}{ \UseTblrTemplate{note}{default} \UseTblrTemplate{remark}{default} }

%%% Tabela
\begin{longtblr}
{
 colspec = {cccccc},
 width = 1\linewidth,
 hlines = {white},
 vlines = {white},
 rowhead = 1, rowfoot = 0,
 row{1} = {font=\bfseries, bg=gray8, fg=black},
 column{1} = {font=\bfseries, bg=gray8, fg=black},
 cell{1}{1} = {bg=white},
 cell{2-Z}{2-Z} = {bg=gray9},
 cell{6}{5-6} = {bg=white},
 cell{7}{2-4} = {bg=white},
 cell{9}{2-5} = {bg=white},
 cell{10}{3-6} = {bg=white},
 cell{11}{2-5} = {bg=white},
 cell{12}{4-6} = {bg=white},
 cell{13}{4-6} = {bg=white},
}
                                           & {边\\   \normalsize\dpy{bian1}}        & {面\\   \normalsize\dpy{mian4}}        & {头\\   \normalsize\dpy{tou5}}        & {以\\   \normalsize\dpy{yi3}}        & {之\\   \normalsize\dpy{zhi1}}        \\
{上\\ \normalsize\dpy{shang4}\\ sobre}     & {上边\\ \normalsize\dpy{shang4 bian1}} & {上面\\ \normalsize\dpy{shang4 mian4}} & {上头\\ \normalsize\dpy{shang4 tou5}} & {以上\\ \normalsize\dpy{yi3 shang4}} & {之上\\ \normalsize\dpy{zhi1 shang4}} \\
{下\\ \normalsize\dpy{xia4}\\ sob}         & {下边\\ \normalsize\dpy{xia4 bian1}}   & {下面\\ \normalsize\dpy{xia4 mian4}}   & {下头\\ \normalsize\dpy{xia4 tou5}}   & {以下\\ \normalsize\dpy{yi3 xia4}}   & {之下\\ \normalsize\dpy{zhi1 xia4}}   \\
{前\\ \normalsize\dpy{qian2}\\ na frente}  & {前边\\ \normalsize\dpy{qian2 bian1}}  & {前面\\ \normalsize\dpy{qian2 mian4}}  & {前头\\ \normalsize\dpy{qian2 tou5}}  & {以前\\ \normalsize\dpy{yi3 qian2}}  & {之前\\ \normalsize\dpy{zhi1 qian2}}  \\
{后\\ \normalsize\dpy{hou4}\\ atrás}       & {后边\\ \normalsize\dpy{hou4 bian1}}   & {后面\\ \normalsize\dpy{hou4 mian4}}   & {后头\\ \normalsize\dpy{hou4 tou5}}   & {以后\\ \normalsize\dpy{yi3 hou4}}   & {之后\\ \normalsize\dpy{zhi1 hou4}}   \\
{里\\ \normalsize\dpy{li3}\\ dentro}       & {里边\\ \normalsize\dpy{li3 bian1}}    & {里面\\ \normalsize\dpy{li3 mian4}}    & {里头\\ \normalsize\dpy{li3 tou5}}    &                                      &                                       \\
{内\\ \normalsize\dpy{nei4}\\ no interior} &                                        &                                        &                                       & {以内\\ \normalsize\dpy{yi3 nei4}}   & {之内\\ \normalsize\dpy{zhi1 nei4}}   \\
{外\\ \normalsize\dpy{wai4}\\ no exterior} & {外边\\ \normalsize\dpy{wai4 bian1}}   & {外面\\ \normalsize\dpy{wai4 mian4}}   & {外头\\ \normalsize\dpy{wai4 tou5}}   & {以外\\ \normalsize\dpy{yi3 wai4}}   & {之外\\ \normalsize\dpy{zhi1 wai4}}   \\
{间\\ \normalsize\dpy{jian1}\\ entre}      &                                        &                                        &                                       &                                      & {之间\\ \normalsize\dpy{zhi1 jian1}}  \\
{旁\\ \normalsize\dpy{pang2}\\ ao lado}    & {旁边\\ \normalsize\dpy{pang2 bian1}}  &                                        &                                       &                                      &                                       \\
{中\\ \normalsize\dpy{zhong1}\\ no meio}   &                                        &                                        &                                       &                                      & {之中\\ \normalsize\dpy{zhi1 zhong1}} \\
{左\\ \normalsize\dpy{zuo3}\\ à esquerda}  & {左边\\ \normalsize\dpy{zuo3 bian1}}   & {左面\\ \normalsize\dpy{zuo3 mian4}}   &                                       &                                      &                                       \\
{右\\ \normalsize\dpy{you4}\\ à direita}   & {右边\\ \normalsize\dpy{you4 bian1}}   & {右面\\ \normalsize\dpy{you4 mian4}}   &                                       &                                      &                                       \\
\pagebreak
{东\\ \normalsize\dpy{dong1}\\ no leste}   & {东边\\ \normalsize\dpy{dong1 bian1}}  & {东面\\ \normalsize\dpy{dong1 mian4}}  & {东头\\ \normalsize\dpy{dong1 tou5}}  & {以东\\ \normalsize\dpy{yi3 dong1}}  & {之东\\ \normalsize\dpy{zhi1 dong1}}  \\
{南\\ \normalsize\dpy{nan2}\\ no sul}      & {南边\\ \normalsize\dpy{nan2 bian1}}   & {南面\\ \normalsize\dpy{nan2 mian4}}   & {南头\\ \normalsize\dpy{nan2 tou5}}   & {以南\\ \normalsize\dpy{yi3 nan2}}   & {之南\\ \normalsize\dpy{zhi1 nan2}}   \\
{西\\ \normalsize\dpy{xi1}\\ no oeste}     & {西边\\ \normalsize\dpy{xi1 bian1}}    & {西面\\ \normalsize\dpy{xi1 mian4}}    & {西头\\ \normalsize\dpy{xi1 tou5}}    & {以西\\ \normalsize\dpy{yi3 xi1}}    & {之西\\ \normalsize\dpy{zhi1 xi1}}    \\
{北\\ \normalsize\dpy{bei3}\\ n norte}     & {北边\\ \normalsize\dpy{bei3 bian1}}   & {北面\\ \normalsize\dpy{bei3 mian4}}   & {北头\\ \normalsize\dpy{bei3 tou5}}   & {以北\\ \normalsize\dpy{yi3 bei3}}   & {之北\\ \normalsize\dpy{zhi1 bei3}}   \\
\end{longtblr}

%%%%% EOF %%%%%


\clearpage
\pagestyle{plain}
\chapter{Radicais Kangxi}
%%%%%%%%%%%%%%%%%%%%%%%%%%%%%%%%%%%%%%%%%%%%%%%%%%%%%%%%%%%%%%%%%%%%%%%%%%%%%%%
%%%%%%%%%%%%%%%%%%%%%%%%%%%%%%%%%%%%%%%%%%%%%%%%%%%%%%%%%%%%%%%%%%%%%%%%%%%%%%%
%%%%%                                                                     %%%%%
%%%%% radicais_kangxi.tex:                                                %%%%%
%%%%% Lista dos 214 radicais Kangxi utilizados nos caracteres chineses.   %%%%%
%%%%%                                                                     %%%%%
%%%%%%%%%%%%%%%%%%%%%%%%%%%%%%%%%%%%%%%%%%%%%%%%%%%%%%%%%%%%%%%%%%%%%%%%%%%%%%%
%%%%%%%%%%%%%%%%%%%%%%%%%%%%%%%%%%%%%%%%%%%%%%%%%%%%%%%%%%%%%%%%%%%%%%%%%%%%%%%

%%% Ajustes para a tabela
\DefTblrTemplate{caption}{default}{}
\DefTblrTemplate{capcont}{default}{ \UseTblrTemplate{conthead-text}{default} }
\DefTblrTemplate{contfoot-text}{default}{Continua na próxima página.}
\DefTblrTemplate{conthead-text}{default}{(Continuação)}
\DefTblrTemplate{firsthead}{default}{ \UseTblrTemplate{caption}{default} }
\DefTblrTemplate{middlehead,lasthead}{default}{ \UseTblrTemplate{conthead}{default} }
\DefTblrTemplate{firstfoot,middlefoot}{default}{ \UseTblrTemplate{contfoot}{default} }
\DefTblrTemplate{lastfoot}{default}{ \UseTblrTemplate{note}{default} \UseTblrTemplate{remark}{default} }

%%% Tabela
\begin{longtblr}
{
  colspec = {|r|ll|l|l|}, hlines,
  width = 1\linewidth,
  rowhead = 1, rowfoot = 0,
  row{1} = {font=\bfseries, fg=white, bg=black},
  row{2-Z} = {font=\normalfont},
}
\textbf{Nº} & \SetCell[c=2]{c}\textbf{Radical e\\Variantes} & 2-2 & \textbf{Tradução} & \textbf{Pinyin} \\
1 & 一 & & um & \dictpinyin{yi1} \\
2 & 丨 & & linha & \dictpinyin{shu4} \\
3 & 丶 & & ponto, indica um fim & \dictpinyin{dian3} \\
4 & 丿 & 乀,乁 & cortar, dobrar & \dictpinyin{pie3} \\
5 & 乙 & 乚、乛、⺄ & segundo, anzol & \dictpinyin{yi3} \\
6 & 亅 & & gancho & \dictpinyin{gou1} \\
7 & 二 & & dois & \dictpinyin{er4} \\
8 & 亠 & & tampa & \dictpinyin{tou2} \\
9 & 人 & 亻、𠆢 & pessoa & \dictpinyin{ren2} \\
10 & 儿 & & pernas & \dictpinyin{er2} \\
11 & 入 & & entrar, juntar-se & \dictpinyin{ru4} \\
12 & 八 & 丷 & oito & \dictpinyin{ba1} \\
13 & 冂 & & largo, exterior & \dictpinyin{jiong3} \\
14 & 冖 & & capa de pano & \dictpinyin{mi4} \\
15 & 冫 & & gelo & \dictpinyin{bing1} \\
16 & 几 & & mesa pequena & \dictpinyin{ji1},\dictpinyin{ji3} \\
17 & 凵 & & receptáculo, caixa aberta & \dictpinyin{qu3} \\
18 & 刀 & 刂、⺈ & faca & \dictpinyin{dao1} \\
19 & 力 & & poder, força & \dictpinyin{li4} \\
20 & 勹 & & invólucro & \dictpinyin{bao1} \\
21 & 匕 & & colher & \dictpinyin{bi3} \\
22 & 匚 & & caixa & \dictpinyin{fang1} \\
23 & 匸 & & compartimento oculto & \dictpinyin{xi3} \\
24 & 十 & & dez, completo, perfeito & \dictpinyin{shi2} \\
25 & 卜 & & advinhação, divinação & \dictpinyin{bu3} \\
26 & 卩 & 㔾 & foca & \dictpinyin{jie2} \\
27 & 厂 & & penhasco, precipício & \dictpinyin{han4} \\
28 & 厶 & & privado & \dictpinyin{si1} \\
29 & 又 & & mão direita, e, novamente & \dictpinyin{you4} \\
30 & 口 & & boca & \dictpinyin{kou3} \\
31 & 囗 & & compartimento, recinto & \dictpinyin{wei2} \\
32 & 土 & & terra & \dictpinyin{tu3} \\
33 & 士 & & acadêmico, bacharel & \dictpinyin{shi4} \\
34 & 夂 & & ir & \dictpinyin{zhi1} \\
35 & 夊 & & ir devagar & \dictpinyin{sui1} \\
36 & 夕 & & tarde, pôr do sol & \dictpinyin{xi1} \\
37 & 大 & & grande, muito & \dictpinyin{da4} \\
38 & 女 & & mulher, fêmea & \dictpinyin{nv3} \\
39 & 子 & & criança, semente & \dictpinyin{zi3} \\
40 & 宀 & & teto, telhado & \dictpinyin{mian2} \\
41 & 寸 & & polegar, polegada & \dictpinyin{cun4} \\
42 & 小 & ⺌、⺍ & pequeno, insignificante & \dictpinyin{xiao3} \\
43 & 尢 & 尣 & manco, coxo & \dictpinyin{you2} \\
44 & 尸 & & cadáver & \dictpinyin{shi1} \\
45 & 屮 & & brotar, germinar & \dictpinyin{che4} \\
46 & 山 & & montanha & \dictpinyin{shan1} \\
47 & 巛 & 川 & rio & \dictpinyin{chuan1} \\
48 & 工 & & trabalho & \dictpinyin{gong1} \\
49 & 己 & ⺒ & próprio, a si mesmo & \dictpinyin{ji3} \\
50 & 巾 & & turbante, cachecol & \dictpinyin{jin1} \\
51 & 干 & & oposto, seco & \dictpinyin{gan1} \\
52 & 幺 & 么 & baixo, minúsculo & \dictpinyin{yao1} \\
53 & 广 & & casa em um penhasco & \dictpinyin{guang3} \\
54 & 廴 & & passada longa & \dictpinyin{yin3} \\
55 & 廾 & & duas mãos, vinte, arco & \dictpinyin{gong3} \\
56 & 弋 & & tiro, flecha & \dictpinyin{yi4} \\
57 & 弓 & & arco & \dictpinyin{gong1} \\
58 & 彐 & 彑 & focinho de porco & \dictpinyin{ji4} \\
59 & 彡 & & cerda, barba & \dictpinyin{shan1} \\
60 & 彳 & & passo & \dictpinyin{chi4} \\
61 & 心 & 忄、⺗ & coração, mente & \dictpinyin{xin1} \\
62 & 戈 & & lança & \dictpinyin{ge1} \\
63 & 戶 & 户、戸 & porta, casa & \dictpinyin{hu4} \\
64 & 手 & 扌、龵 & mão & \dictpinyin{shou3} \\
65 & 支 & & ramo & \dictpinyin{zhi1} \\
66 & 攴 & 攵 & tocar, bater levemente & \dictpinyin{pu1} \\
67 & 文 & & escrita, literatura & \dictpinyin{wen2} \\
68 & 斗 & & objeto em forma de concha & \dictpinyin{dou3} \\
69 & 斤 & & machado & \dictpinyin{jin1} \\
70 & 方 & & quadrado & \dictpinyin{fang1} \\
71 & 无 & 旡 & não, nada, negativo & \dictpinyin{wu2} \\
72 & 日 & & sol, dia & \dictpinyin{ri4} \\
73 & 曰 & & dizer, falar & \dictpinyin{yue1} \\
74 & 月 & & lua, mês & \dictpinyin{yue4} \\
75 & 木 & & árvore & \dictpinyin{mu4} \\
76 & 欠 & & falta, não ter, hiato & \dictpinyin{qian4} \\
77 & 止 & & parar & \dictpinyin{zhi3} \\
78 & 歹 & 歺 & morte, decadência & \dictpinyin{dai3} \\
79 & 殳 & & arma, lança & \dictpinyin{shu1} \\
80 & 毋 & 母 & mãe, não faça & \dictpinyin{mu3} \\
81 & 比 & & comparar, competir & \dictpinyin{bi3} \\
82 & 毛 & & pelagem & \dictpinyin{mao2} \\
83 & 氏 & & clã, linhagem & \dictpinyin{shi4} \\
84 & 气 & & ar, vapor, respiração & \dictpinyin{qi4} \\
85 & 水 & 氵、氺 & água & \dictpinyin{shui3} \\
86 & 火 & 灬 & fogo & \dictpinyin{huo3} \\
87 & 爪 & 爫 & garrai, unha & \dictpinyin{zhao3} \\
88 & 父 & & pai, luz & \dictpinyin{fu4} \\
89 & 爻 & & duplo x, trigramas & \dictpinyin{yao2} \\
90 & 爿 & 丬 & metade de um tronco, madeira rachada & \dictpinyin{pan2} \\
91 & 片 & & fatia, filme & \dictpinyin{pian4} \\
92 & 牙 & & dente, presa & \dictpinyin{ya2} \\
93 & 牛 & 牜、⺧ & boi, vaca & \dictpinyin{niu2} \\
94 & 犬 & 犭 & cão & \dictpinyin{quan3} \\
95 & 玄 & & escuro, profundo & \dictpinyin{xuan2} \\
96 & 玉 & 王、玊 & jade & \dictpinyin{yu4} \\
97 & 瓜 & & melão & \dictpinyin{gua1} \\
98 & 瓦 & & telha & \dictpinyin{wa3} \\
99 & 甘 & & doce & \dictpinyin{gan1} \\
100 & 生 & & vida & \dictpinyin{sheng1} \\
101 & 用 & & usar & \dictpinyin{yong4} \\
102 & 田 & & campo, arrozal & \dictpinyin{tian2} \\
103 & 疋 & ⺪& pedaço de pano & \dictpinyin{pi3} \\
104 & 疒 & & doença & \dictpinyin{ne4} \\
105 & 癶 & & pegadas, pernas & \dictpinyin{bo1} \\
106 & 白 & & branco & \dictpinyin{bai2} \\
107 & 皮 & & pele, couro & \dictpinyin{pi2} \\
108 & 皿 & & prato & \dictpinyin{min3} \\
109 & 目 & ⺫ & olho & \dictpinyin{mu4} \\
110 & 矛 & & lança & \dictpinyin{mao2} \\
111 & 矢 & & seta, flecha & \dictpinyin{shi3} \\
112 & 石 & & pedra & \dictpinyin{shi2} \\
113 & 示 & 礻& espírito, ancestral, veneração & \dictpinyin{shi4} \\
114 & 禸 & & trilha & \dictpinyin{rou2} \\
115 & 禾 & & grão & \dictpinyin{he2} \\
116 & 穴 & & caverna & \dictpinyin{xue2} \\
117 & 立 & & ficar em péi, ereto & \dictpinyin{li4} \\
118 & 竹 & ⺮ & bambu & \dictpinyin{zhu2} \\
119 & 米 & & arroz & \dictpinyin{mi3} \\
120 & 糸 & 纟、糹 & seda & \dictpinyin{mi4} \\
121 & 缶 & & pote, jarra & \dictpinyin{fou3} \\
122 & 网 & ⺲、罓、⺳ & rede & \dictpinyin{wang3} \\
123 & 羊 & ⺶、⺷ & ovelha, cabra & \dictpinyin{yang2} \\
124 & 羽 & & pena & \dictpinyin{yu3} \\
125 & 老 & 耂 & velho & \dictpinyin{lao3} \\
126 & 而 & & e, mas & \dictpinyin{er2} \\
127 & 耒 & & arado & \dictpinyin{lei3} \\
128 & 耳 & & orelha & \dictpinyin{er3} \\
129 & 聿 & ⺺、⺻ & escova & \dictpinyin{yu4} \\
130 & 肉 & 月、⺼ & carne & \dictpinyin{rou4} \\
131 & 臣 & & ministro, oficial & \dictpinyin{chen2} \\
132 & 自 & & próprio, auto-- & \dictpinyin{zi4} \\
133 & 至 & & chegar & \dictpinyin{zhi4} \\
134 & 臼 & & argamassa, ligação & \dictpinyin{jiu4} \\
135 & 舌 & & língua & \dictpinyin{she2} \\
136 & 舛 & & opor & \dictpinyin{chuan3} \\
137 & 舟 & & barco & \dictpinyin{zhou1} \\
138 & 艮 & & parada, quietude & \dictpinyin{gen3} \\
139 & 色 & & cor, forma & \dictpinyin{se4} \\
140 & 艸 & ⺿ & grama & \dictpinyin{cao3} \\
141 & 虍 & & tigre & \dictpinyin{hu1} \\
142 & 虫 & & inseto, verme & \dictpinyin{chong2} \\
143 & 血 & & sangue & \dictpinyin{xue4} \\
144 & 行 & & andar, ir, fazer & \dictpinyin{xing2} \\
145 & 衣 & ⻂& roupa & \dictpinyin{yi1} \\
146 & 襾 & 西、覀 & capa, oeste & \dictpinyin{ya4} \\
147 & 見 & 见 & ver & \dictpinyin{jian4} \\
148 & 角 & ⻆、⻇ & chifre & \dictpinyin{jiao3} \\
149 & 言 & 讠、訁 & palavra, linguagem & \dictpinyin{yan2} \\
150 & 谷 & & vale & \dictpinyin{gu3} \\
151 & 豆 & & feijão, fava & \dictpinyin{dou4} \\
152 & 豕 & & porco & \dictpinyin{shi3} \\
153 & 豸 & & texugo, inseto sem pernas & \dictpinyin{zhi4} \\
154 & 貝 & 贝 & concha & \dictpinyin{bei4} \\
155 & 赤 & & vermelho, nu & \dictpinyin{chi4} \\
156 & 走 & & correr & \dictpinyin{zou3} \\
157 & 足 & ⻊& pé & \dictpinyin{zu2} \\
158 & 身 & & corpo & \dictpinyin{shen1} \\
159 & 車 & 车 & carroça, carro & \dictpinyin{che1} \\
160 & 辛 & & amargo & \dictpinyin{xin1} \\
161 & 辰 & & manhã & \dictpinyin{chen2} \\
162 & 辵 & ⻌、⻍、⻎ & caminhar & \dictpinyin{chuo4} \\
163 & 邑 & ⻏ & cidade & \dictpinyin{yi4} \\
164 & 酉 & & vinho, álcool & \dictpinyin{you3} \\
165 & 釆 & & distinto & \dictpinyin{bian4} \\
166 & 里 & & aldeia, vila & \dictpinyin{li3} \\
167 & 金 & 钅、釒 & ouro, metal & \dictpinyin{jin1} \\
168 & 長 & 长、镸 & longo, crescer & \dictpinyin{zhang3} \\
169 & 門 & 门 & portão, porta & \dictpinyin{men2} \\
170 & 阜 & ⻖ & montei, barragem & \dictpinyin{fu4} \\
171 & 隶 & & escravo & \dictpinyin{li4} \\
172 & 隹 & & pássaro de cauda curta & \dictpinyin{zhui1} \\
173 & 雨 & & chuva & \dictpinyin{yu3} \\
174 & 靑 & 青 & azul, verde ou preto & \dictpinyin{qing1} \\
175 & 非 & & errado & \dictpinyin{fei1} \\
176 & 面 & 靣 & face & \dictpinyin{mian4} \\
177 & 革 & & couro, couro cru & \dictpinyin{ge2} \\
178 & 韋 & 韦 & couro tingido & \dictpinyin{wei2} \\
179 & 韭 & & alho-poró & \dictpinyin{jiu3} \\
180 & 音 & & som & \dictpinyin{yin1} \\
181 & 頁 & 页 & folha, página & \dictpinyin{ye4} \\
182 & 風 & 风 & vento & \dictpinyin{feng1} \\
183 & 飛 & 飞 & voar & \dictpinyin{fei1} \\
184 & 食 & 饣、飠 & alimento, comer & \dictpinyin{shi2} \\
185 & 首 & & cabeça & \dictpinyin{shou3} \\
186 & 香 & & perfume, aroma & \dictpinyin{xiang1} \\
187 & 馬 & 马 & cavalo & \dictpinyin{ma3} \\
188 & 骨 & ⻣ & osso & \dictpinyin{gu3} \\
189 & 高 & 髙 & alto & \dictpinyin{gao1} \\
190 & 髟 & & cabelo & \dictpinyin{biao1} \\
191 & 鬥 & & luta & \dictpinyin{dou4} \\
192 & 鬯 & & vinho sacrificial & \dictpinyin{chang4} \\
193 & 鬲 & & caldeirão, tripé & \dictpinyin{ge2} \\
194 & 鬼 & & fantasma, demônio & \dictpinyin{gui3} \\
195 & 魚 & 鱼 & peixe & \dictpinyin{yu2} \\
196 & 鳥 & 鸟 & pássaro & \dictpinyin{niao3} \\
197 & 鹵 & 卤 & sal & \dictpinyin{lu3} \\
198 & 鹿 & & corça, veado & \dictpinyin{lu4} \\
199 & 麥 & 麦 & trigo & \dictpinyin{mai4} \\
200 & 麻 & & cânhamo, linho & \dictpinyin{ma2} \\
201 & 黃 & 黄 & amarelo & \dictpinyin{huang4} \\
202 & 黍 & & milhete, painço & \dictpinyin{shu3} \\
203 & 黑 & & preto & \dictpinyin{hei1} \\
204 & 黹 & & bordado & \dictpinyin{zhi3} \\
205 & 黽 & 黾 & sapo, anfíbio & \dictpinyin{mian3} \\
206 & 鼎 & & tripé de sacrifício, caldeirão de três pernas & \dictpinyin{ding3} \\
207 & 鼓 & & tambor & \dictpinyin{gu3} \\
208 & 鼠 & 鼡 & rato, camundongo & \dictpinyin{shu3} \\
209 & 鼻 & & nariz & \dictpinyin{bi2} \\
210 & 齊 & 齐、斉 & mesmo, uniformemente & \dictpinyin{qi2} \\
211 & 齒 & 齿 & dente & \dictpinyin{chi3} \\
212 & 龍 & 龙 & dragão & \dictpinyin{long2} \\
213 & 龜 & 龟 & tartaruga & \dictpinyin{gui1} \\
214 & 龠 &   & flauta & \dictpinyin{yue4} \\
\end{longtblr}

%%%%% EOF %%%%%


\fi

\end{document}

%%%%% EOF %%%%
