
%%%%%%%%%%%%%%%%%%%%%%%%%%%%%%%%%%%%%%%%%
% LuaLaTex
%
% Dicionário Chinês -> Português
% Autor: Luiz Eduardo Roncato Cordeiro
%
% Licença:
% CC BY-NC-SA 3.0 (http://creativecommons.org/licenses/by-nc-sa/3.0/)
%%%%%%%%%%%%%%%%%%%%%%%%%%%%%%%%%%%%%%%%%

\documentclass[a4paper,9pt,twoside,openany]{memoir}

\usepackage[brazilian]{babel}
\usepackage{fontspec}
\usepackage[dvipsnames]{xcolor}
\usepackage{multicol}
\usepackage{imakeidx}
\usepackage[inline]{enumitem}
\usepackage{zhnumber}
\usepackage{tikz}
\usepackage[hyperindex,hidelinks]{hyperref}
\usepackage{pifont}
\usepackage{xstring}
\usepackage{tabularray}
\usepackage[most]{tcolorbox}
\usepackage{luacode}
\usepackage{parskip}
\usepackage{stackengine}
\usepackage{import}

% Meus Comandos
\import{include/}{cmd.tex}
\import{include/}{genericcmd.tex}
\import{include/}{strokescmd.tex}

% Ajustes do PDF
\hypersetup{
  linktoc=page,
  colorlinks=true,
  urlcolor=blue,
  linkcolor=blue,
  citecolor=blue,
  pdftitle={汉葡词典 - Dicionário Chinês-Português},
  pdfsubject={Dicionário Chinês-Português -- Ordenado por Número de Traços},
  pdfauthor={罗学凯 AKA Luiz Eduardo Roncato Cordeiro},
  pdfkeywords={dicionário, chinês, português, instituto confúcio}
}

% Índices Remissivos
\makeindex[title=Índice Remissivo por Radical,intoc=true,columns=3,columnsep=15pt,columnseprule=true,noautomatic=true,name=sradical]
\indexsetup{level=\chapter*,toclevel=chapter,headers={\indexname}{\indexname}}

%%%
%%% Documento começa aqui!
%%%

\begin{document}
\addfontfeatures{CharacterWidth=Proportional}

\begin{titlingpage}
  \raggedleft
  \rule{1pt}{\textheight}
  \hspace{0.1\textwidth}
  \parbox[b]{0.75\textwidth}{
    \vspace{0.05\textheight}
    {\HUGE\bfseries 汉葡词典}\\[2\baselineskip] % Title
    {\Large\textsc{Dicionário Chinês-Português}\\%
     \large\textsc{\zhtoday}}\\% Date
    [4\baselineskip]
    {\Large\textsc{罗学凯}\\%
     \small AKA Luiz Eduardo Roncato Cordeiro}\\% Author
    \vspace{0.5\textheight}\\%
    {Aluno do Instituto Confúcio na UNESP}\\[\baselineskip] % Publisher?
  }
  \newpage
  \raggedright
  \setlength{\parindent}{0pt}
  \setlength{\parskip}{\baselineskip}
  \mbox{}
  \vfill
  \footnotesize
  \textcopyright{} 2024-2025 por Luiz Eduardo Roncato Cordeiro, está licenciado sob CC BY-NC-SA 4.0\\
  \begin{itemize}
    \item Para visualizar uma cópia desta licença, visite:\\ \url{http://creativecommons.org/licenses/by-nc-sa/4.0/}
    \item Este trabalho ainda está em andamento e o ``código fonte'' está localizado em:\\ \url{https://github.com/lercordeiro/dicionario_chines_portugues}
    \item A última versão compilada também pode ser encontrada em:\\ \url{https://ler.cordeiro.nom.br/}
  \end{itemize}
%  \begin{tabular}{ll}
%    First edition: & T.B.D. \\
%  \end{tabular}
\end{titlingpage}


\clearpage
\pagestyle{empty}
\tableofcontents

\clearpage
\pagestyle{empty}
\chapter{汉葡词典}

%%%%%%%%%%%%%%%%%%%%%%%%
%
% https://en.wikipedia.org/wiki/Chinese_character_orders
%
%%%%%%%%%%%%%%%%%%%%%%%%

Dicionário Chinês-Português ordenado primeiro pelo número de traços,
depois pela ordem do caracter na tabela UTF-8.  As definições são
agrupadas e ordenadas pelo pinyin em cada verbete.

\clearpage
\pagestyle{dicionario}
\begin{multicols}{2}
\import{groups_by_strokes/}{001.tex}
\import{groups_by_strokes/}{002.tex}
\import{groups_by_strokes/}{003.tex}
\import{groups_by_strokes/}{004.tex}
\import{groups_by_strokes/}{005.tex}
\import{groups_by_strokes/}{006.tex}
\import{groups_by_strokes/}{007.tex}
\import{groups_by_strokes/}{008.tex}
\import{groups_by_strokes/}{009.tex}
\import{groups_by_strokes/}{010.tex}
\import{groups_by_strokes/}{011.tex}
\import{groups_by_strokes/}{012.tex}
\import{groups_by_strokes/}{013.tex}
\import{groups_by_strokes/}{014.tex}
\import{groups_by_strokes/}{015.tex}
\import{groups_by_strokes/}{016.tex}
\import{groups_by_strokes/}{017.tex}
\import{groups_by_strokes/}{018.tex}
\import{groups_by_strokes/}{019.tex}
\import{groups_by_strokes/}{020.tex}
\import{groups_by_strokes/}{021.tex}
%\import{groups_by_strokes/}{022.tex}
%\import{groups_by_strokes/}{023.tex}
%\import{groups_by_strokes/}{024.tex}
%\import{groups_by_strokes/}{025.tex}
%\import{groups_by_strokes/}{026.tex}
%\import{groups_by_strokes/}{027.tex}
%\import{groups_by_strokes/}{028.tex}
%\import{groups_by_strokes/}{029.tex}
%\import{groups_by_strokes/}{030.tex}
\end{multicols}

\clearpage
\pagestyle{plain}
\chapter{Termos Gramaticais Chineses}
\begin{center}
\begin{tblr}[m]{lll}
substantivo/nome  & \textbf{s.}        & 名词                     \\
pronome           & \textbf{pron.}     & 代词                     \\
numeral           & \textbf{num.}      & 数词                     \\
classificador     & \textbf{clas.}     & 量词                     \\
verbo             & \textbf{v.}        & 动词                     \\
verbo auxiliar    & \textbf{v.aux.}    & 助动词                   \\
verbo+complemento & \textbf{v.+compl.} & 动宾式\hspace{1em}离合词 \\
adjetivo          & \textbf{adj.}      & 形容词                   \\
advérbio          & \textbf{adv.}      & 副词                     \\
preposição        & \textbf{prep.}     & 介词                     \\
conjunção         & \textbf{conj.}     & 连词                     \\
partícula         & \textbf{part.}     & 助词                     \\
interjeição       & \textbf{interj.}   & 叹词                     \\
prefixo           & \textbf{pref.}     & 前缀                     \\
sufixo            & \textbf{suf.}      & 后缀                     \\
expressão         & \textbf{expr.}     & 熟语                     \\
\end{tblr}
\end{center}


\clearpage
\pagestyle{plain}
\chapter{Classificadores Nominais}
\DefTblrTemplate{caption}{default}{}
\DefTblrTemplate{capcont}{default}{ \UseTblrTemplate{conthead-text}{default} }
\DefTblrTemplate{contfoot-text}{default}{Continua na próxima página.}
\DefTblrTemplate{conthead-text}{default}{(Continuação)}
\DefTblrTemplate{firsthead}{default}{ \UseTblrTemplate{caption}{default} }
\DefTblrTemplate{middlehead,lasthead}{default}{ \UseTblrTemplate{conthead}{default} }
\DefTblrTemplate{firstfoot,middlefoot}{default}{ \UseTblrTemplate{contfoot}{default} }
\DefTblrTemplate{lastfoot}{default}{ \UseTblrTemplate{note}{default} \UseTblrTemplate{remark}{default} }

\begin{longtblr}
{
  colspec = {|c|c|X|X|}, hlines,
  width = 1\linewidth,
  rowhead = 1, rowfoot = 0,
  row{1} = {font=\bfseries, fg=white, bg=black},
}
\textbf{Hanzi} & \textbf{Pinyin} & \textbf{Descrição} & \textbf{Exemplos}\\
    把 & \dpy{ba3}     & mão cheia --- objetos que podem ser segurados, objetos relativamente longos e planos & faca, tesoura, espada, chave, guarda-chuva, leque, escova de dentes, colher, garfo, martelo, cadeado, pistola, rifle, carne, punhado de arroz, punhado de areia, ramo de flores, punhado de sementes, ramo de pauzinhos, esqueleto, fogo, bule\\
    班 & \dpy{ban1}    & serviços programados (trens, aviões, etc), grupos de pessoas, uma classe como em alunos & \\
    包 & \dpy{bao1}    & pacote & doces, cigarros, açúcar, biscoitos\\
    杯 & \dpy{bei1}    & copo --- bebidas & chá, vinho, álcool, água, leite, suco de fruta, refrigerante\\
    本 & \dpy{ben3}    & volume --- material impresso encadernado & livro, revista, romance, escritura, dicionário, bloco de notas, livro didático\\
    笔 & \dpy{bi3}     & traços de caracteres; grandes quantidades de dinheiro & \\
    部 & \dpy{bu4}     & máquinas, veículos; produções & celular, telefone, carro, jogo, romance, filme/imagem em movimento, ópera, obra literária\\
    册 & \dpy{ce4}     & volumes de livros & \\
    层 & \dpy{ceng2}   & andar, piso; camada & andares (em um prédio); camada de poeira, bolo, tinta\\
    场 & \dpy{chang3}  & evento de curta duração; precipitação & espetáculo público, jogo, crise, guerra, catástrofe, uma doença, performance, jogo, chuva, queda de neve\\
    串 & \dpy{chuan4}  & conjuntos de números & telefone celular/número de celular\\
    床 & \dpy{chuang4} & cama & cobertores, lençóis\\
    次 & \dpy{ci4}     & tempo, repetições --- oportunidades, acidentes & \\
    出 & \dpy{chu1}    & atuação em uma peça & \\
    打 & \dpy{da2}     & dúzia & lápis, ovos\\
    贷 & \dpy{dai4}    & sacos ou bolsos cheios & açúcar, farinha, arroz\\
    道 & \dpy{dao4}    & projeções lineares (raios de luz, etc.); ordem dada por uma figura de autoridade; pergunta, memorando; curso (de comida); coisas longas e tortas (cume da montanha, relâmpago) & \\
    滴 & \dpy{di1}     & gotículas de água, sangue, outros fluidos semelhantes & \\
    点 & \dpy{dian3}   & ideias, sugestões; um pouco/algum (somente com 一) & \\
    碟 & \dpy{die2}    & pires (molho de soja) & \\
    顶 & \dpy{ding3}   & objetos com topo saliente, algo para colocar sobre a cabeça & chapéu, barraca, mosquiteiro\\
    栋 & \dpy{dong4}   & pilar (edifício menor, casa) & \\
    堵 & \dpy{du3}     & luminárias abrangentes (parede sem teto) & \\
    段 & \dpy{duan4}   & comprimento adjacente --- cabos, estradas, pedaço de giz, parte de uma música & \\
    对 & \dpy{dui4}    & casal, par combinado (para certas coisas), dísticos & casal, brincos, vasos\\
    顿 & \dpy{dun4}    & ações de curta duração & refeição, conflito, espancamento, briga, repreensão\\
    朵 & \dpy{duo3}    & coisas parecidas com flores & flor, nuvem, cogumelo\\
    发 & \dpy{fa1}     & coisas redondas & bala, munição\\
    方 & \dpy{fang4}   & coisas quadradas & pedra de tinta, bacon\\
    份 & \dpy{fen4}    & porções, documentos de várias páginas & porção de comida, jornal, emprego\\
    封 & \dpy{feng1}   & coisas que podem ser seladas & carta, correio, telegrama\\
    峰 & \dpy{feng1}   & animais com corcundas & camelo\\
    幅 & \dpy{fu2}     & objetos semelhantes a imagens & foto, pintura, desenho, banner, obra de arte, quadro, cartaz\\
    服 & \dpy{fu4}     & dose (de remédio)\\
    副 & \dpy{fu4}     & objetos que vêm em pares (luvas, óculos, etc.), baralhos, mahjong & \\
    个 & {\dpy{ge5}\\ \dpy{ge4}} & coisas individuais, pessoas, classificador de uso geral (o uso desse classificador em conjunto com qualquer substantivo é geralmente aceito se a pessoa não souber o classificador adequado) & pessoa, irmão mais velho, estudante, parente, modo de pensar, sugestão, pergunta, nação\\
    根 & \dpy{gen1}    & objetos finos e esguios; fios finos e flexíveis & agulha, pilar, banana, palito de massa frita, coxa de frango frita, picolé, pirulito, pauzinho, vela, incenso, cabelo, linha, barbante\\
    股 & \dpy{gu3}     & fluxos (de ar, cheiro, influência, etc.) & \\
    挂 & \dpy{gua4}    & coisas multi-componentes & cavalo e carroça \\
    罐 & \dpy{guan4}   & lata pequena a média & refrigerante, suco, comida, feijão, óleo, doce\\
    行 & \dpy{hang2}   & objetos que formam linhas (palavras, etc.) & \\
    盒 & \dpy{he2}     & caixa pequena & fita, comida, bolo, doces, chocolate, brinquedos, livros, cigarros, detergente, roupas\\
    户 & \dpy{hu4}     & famílias & \\
    伙 & \dpy{huo3}    & classificador geralmente depreciativo para bandos de pessoas, como gangues ou bandidos & \\
    家 & \dpy{jia1}    & reunião de pessoas, estabelecimentos & famílias, empresas, lojas, restaurantes\\
    架 & \dpy{jia4}    & maquinário, veículo & aeronave, avião, piano, máquinas, computadores\\
    间 & \dpy{jian1}   & quartos, espaços & quarto, dormitório, cozinha, sala de aula, casa, escola, empresa, capela\\
    件 & \dpy{jian4}   & assuntos, roupas (tops), móveis & roupa (top), camiseta, camisa, casaco, lençol, bagagem, presente, questão/matéria/coisa\\
    讲 & \dpy{jiang3}  & longos períodos de aula & \\
    节 & \dpy{jie2}    & seção (de bambu, período curto de aula) & \\
    届 & \dpy{jie4}    & sessões ou reuniões agendadas regularmente, grupos de anos em uma escola (por exemplo, Turma de 2025) & \\
    句 & \dpy{ju4}     & linhas de poemas, frases & \\
    棵 & \dpy{ke1}     & árvores ou outra flora semelhante & pinheiro, rosa\\
    颗 & \dpy{ke1}     & pequenos objetos, objetos que parecem pequenos & corações, pérolas, dentes, diamantes, sementes, estrelas distantes, planetas distantes\\
    课 & \dpy{ke4}     & lições em um texto & \\
    口 & \dpy{kou3}    & população de aldeias (número inferior a 100), familiares, poços, bocados de comida & \\
    块 & \dpy{kuai4}   & pedaço de forma irregular & terra, pedra, tofu, sabonete, pedaço/fatia de bolo, pão (não fatias), melancia, carne, queijo, pizza, chiclete, toalha de mesa, relógio de pulso, bloco de incenso\\
    类 & \dpy{lei4}    & objetos do mesmo tipo ou natureza & \\
    粒 & \dpy{li4}     & grão & um (único) amendoim, uva, arroz cru, semenete, doce, bala, chocolate\\
    辆 & \dpy{liang4}  & veículos com rodas (não trens) & automóvel, bicicleta, carro\\
    列 & \dpy{lie4}    & trens & \\
    轮 & \dpy{lun2}    & lua & \\
    枚 & \dpy{mei2}    & medalhas, pequenas coisas planas como selos, cascas de banana, anéis, distintivos, foguetes, mísseis & \\
    门 & \dpy{men2}    & objetos pertencentes a acadêmicos (cursos, disciplinas, etc.); artilharia (canhão) & \\
    面 & \dpy{mian4}   & objetos planos e lisos & espelho, bandeira, parede com telhado, escudo\\
    名 & \dpy{ming4}   & pessoas de alto escalão (médicos, advogados, políticos, realeza, etc.), membros; em linguagem formal pode ser utilizado para qualquer pessoa não necessariamente de alto escalão & \\
    排 & \dpy{pai2}    & linhas --- objetos agrupados em linhas & cadeiras, assentos, mesas, filas de pessoas\\
    盘 & \dpy{pan2}    & objetos planos ou bobinas & cassete de vídeo ou áudio, pratinho, bobina de incenso\\
    批 & \dpy{pi1}     & pessoas, bens, etc. & \\
    匹 & \dpy{pi3}     & cavalos e outras montarias; rolos/pedaços de panos & cavalo, lobo\\
    篇 & \dpy{pian1}   & escritos & papel, artigo, ensaio, relatório\\
    片 & \dpy{pian4}   & fatia - objetos finos, planos, às vezes irregulares & cartão, lábio, nuvem, praia, chiclete, fatia de pão, fatia de carne, biscoito, queijo, fatia de melancia, folha, pétala de flor, campo, lago, pílula (comprimido de remédio), DVD\\
    瓶 & \dpy{ping2}   & garrafa & álcool, água, óleo, cerveja, bebida, vinho, refrigerante, leite, shampoo\\
    期 & \dpy{qi1}     & revistas & \\
    群 & \dpy{qun2}    & grupo (incl. pessoas), rebanho, multidão, enxame, etc. & pessoas, rebanho de gado, bando de pássaros, bando de cães, enxame de mosquitos, colônia de abelhas/formigas\\
    任 & \dpy{ren4}    & mandato (presidente, senador, deputado, etc.) & \\
    扇 & \dpy{shan4}   & coisas que abrem e fecham com dobradiças & janela, porta\\
    首 & \dpy{shou1}   & coisas com versos & canção, poema \\
    束 & \dpy{shu4}    & cachos & flores, uvas \\
    双 & \dpy{shuang1} & par de objetos que naturalmente vêm em pares & pauzinhos, sapatos, luvas, olhos\\
    艘 & \dpy{sou1}    & navios & \\
    所 & \dpy{suo3}    & pequenos edifícios, instituições & universidade, casa independente\\
    台 & \dpy{tai2}    & objetos pesados, especialmente máquinas; apresentações & TV, computador, piano, aparelho, avião, trem, carro, elevador; apresentação de teatro, jogo\\
    躺 & \dpy{tang2}   & períodos de aulas, suítes de imóveis & aulas, lições, leituras\\
    趟 & \dpy{tang4}   & viagens, serviços de transportes programados & \\
    套 & \dpy{tao4}    & conjuntos & livros, revistas, colecionáveis, roupas, casas/apartamentos com vários cômodos, suítes, selos, móveis, quartos, ternos\\
    题 & \dpy{ti2}     & classificador de perguntas & \\
    条 & \dpy{tiao2}   & objetos longos, estreitos e flexíveis & peixe, cobra, dragão, minhoca, cachorro, cachecol, estrada, fita, rio, raiz, caule, corda, edredom, toalha, fio dental, calças, gravata, saia, sofá/banco, pessoa heróica, barco pequeno\\
    帖 & \dpy{tie4}    & bandagens adesivas & \\
    通 & \dpy{tong1}   & conversa, palestra & \\
    桶 & \dpy{tong3}   & jarro, balde, barril & jarro de leite, barril de óleo\\
    头 & \dpy{tou2}    & cabeça de gado & porco, vaca, bois, iaques, ovelhas, burros, mulas, leopardos, dinossauros\\
    团 & \dpy{tuan2}   & bola --- objetos redondos e enrolados & bola de lã, etc. \\
    碗 & \dpy{wan3}    & tigela & de arroz, de macarrão, de sopa\\
    位 & \dpy{wei4}    & classificador educado e respeitoso para pessoas & professor, cliente, colega\\
    项 & \dpy{xiang4}  & projetos & \\
    些 & \dpy{xie1}    & alguns & somente com 一,这,那,哪\\
    样 & \dpy{yang4}   & itens gerais de diferentes atributos & \\
    页 & \dpy{ye4}     & página & \\
    则 & \dpy{ze2}     & diário, registro do dia & \\
    扎 & \dpy{zha1}    & jarra, caneca --- usado em cantonês no lugar de 束 \dpy{shu4} (por exemplo, um pacote de flores) & bebidas como cerveja, refrigerante, suco, etc. (pint/jar: empréstimo linguístico do inglês, pode ser considerado informal ou gíria)\\
    盏 & \dpy{zhan3}   & luminárias (geralmente lâmpadas), bule de chá, etc. & \\
    站 & \dpy{zhan4}   & paradas (de ônibus ou trens) & \\
    张 & \dpy{zhang1}  & folha --- objetos planos ou de papel & papel, mesa, cama, cartão, sofá, CD/DVD, guardanapo, fotografia, ingresso, pintura, constelação, rosto, boca, arco\\
    阵 & \dpy{zhen4}   & rajada, estouro --- eventos com duração curtas & tempestades com raios, rajadas de vento, ocorrência de chuva\\
    支 & \dpy{zhi1}    & objetos bastante longos, semelhantes a bastões & caneta, lápis, pauzinho, canudo, bambu, rosa, rifle, flecha, lança, projétil de artilharia, míssil, cantigas\\
    只 & \dpy{zhi1}    & um de um par; animais & mão, dedo, olho, pé, cabeça, meia, luva, sapato, brinco, óculos; pássaro, frango, gato, tigre, cachorro, macaco, elefante, ovelha, rato, borboleta, rã, inseto\\
    种 & \dpy{zhong3}  & tipos & de coisas, de livros, de pessoas\\
    株 & \dpy{zhu1}    & árvores/plantas menores & arbusto, planta de arroz, planta de trigo\\
    幢 & \dpy{zhuang2} & edifício de vários andares & \\
    组 & \dpy{zu3}     & conjuntos, linhas, séries, baterias (militares) & \\
    座 & \dpy{zuo4}    & montanha, edifício & montanha, colina, estrutura, grande edifício, cidade, ponte, vila, arranha-céu, torre, templo, bloco de apartamentos\\
\end{longtblr}


\clearpage
\pagestyle{plain}
\chapter{Classificadores Verbais}
\DefTblrTemplate{caption}{default}{}
\DefTblrTemplate{capcont}{default}{ \UseTblrTemplate{conthead-text}{default} }
\DefTblrTemplate{contfoot-text}{default}{Continua na próxima página.}
\DefTblrTemplate{conthead-text}{default}{(Continuação)}
\DefTblrTemplate{firsthead}{default}{ \UseTblrTemplate{caption}{default} }
\DefTblrTemplate{middlehead,lasthead}{default}{ \UseTblrTemplate{conthead}{default} }
\DefTblrTemplate{firstfoot,middlefoot}{default}{ \UseTblrTemplate{contfoot}{default} }
\DefTblrTemplate{lastfoot}{default}{ \UseTblrTemplate{note}{default} \UseTblrTemplate{remark}{default} }

\begin{longtblr}
{
  colspec = {|c|c|X|X|}, hlines,
  width = 1\linewidth,
  rowhead = 1, rowfoot = 0,
  row{1} = {font=\bfseries, fg=white, bg=black},
}
\textbf{Hanzi} & \textbf{Pinyin} & \textbf{Descrição}\\
    遍 & \dpy{bian4}  & o número de vezes que uma ação foi concluída \\
    场 & \dpy{chang3} & a duralção de um evento ocorrendo dentro de outro evento\\
    次 & \dpy{ci4}    & vezes (ao contrário de 遍, 次 refere-se ao número de vezes, independente de ter sido concluído ou não)\\
    顿 & \dpy{dun4}   & ações sem repetição\\
    回 & \dpy{hui2}   & ocorrências (usado coloquialmente)\\
    声 & \dpy{sheng1} & gritos, expressões\\
    趟 & \dpy{tang4}  & viagens, visitas\\
    下 & \dpy{xia4}   & ações breves e frequentemente repentinas, muito mais comum em cantonês do quem em dialetos do norte\\
\end{longtblr}


\clearpage
\pagestyle{plain}
\chapter{Verbos Direcionais}
%%%%%%%%%%%%%%%%%%%%%%%%%%%%%%%%%%%%%%%%%%%%%%%%%%%%%%%%%%%%%%%%%%%%%%%%%%%%%%%
%%%%%%%%%%%%%%%%%%%%%%%%%%%%%%%%%%%%%%%%%%%%%%%%%%%%%%%%%%%%%%%%%%%%%%%%%%%%%%%
%%%%%                                                                     %%%%%
%%%%% verbos_direcionais.tex:                                             %%%%%
%%%%% Tabela com os verbos direcionais chineses.                          %%%%%
%%%%%                                                                     %%%%%
%%%%%%%%%%%%%%%%%%%%%%%%%%%%%%%%%%%%%%%%%%%%%%%%%%%%%%%%%%%%%%%%%%%%%%%%%%%%%%%
%%%%%%%%%%%%%%%%%%%%%%%%%%%%%%%%%%%%%%%%%%%%%%%%%%%%%%%%%%%%%%%%%%%%%%%%%%%%%%%

%%% Ajustes para a tabela
\DefTblrTemplate{caption}{default}{}
\DefTblrTemplate{capcont}{default}{ \UseTblrTemplate{conthead-text}{default} }
\DefTblrTemplate{contfoot-text}{default}{Continua na próxima página.}
\DefTblrTemplate{conthead-text}{default}{(Continuação)}
\DefTblrTemplate{firsthead}{default}{ \UseTblrTemplate{caption}{default} }
\DefTblrTemplate{middlehead,lasthead}{default}{ \UseTblrTemplate{conthead}{default} }
\DefTblrTemplate{firstfoot,middlefoot}{default}{ \UseTblrTemplate{contfoot}{default} }
\DefTblrTemplate{lastfoot}{default}{ \UseTblrTemplate{note}{default} \UseTblrTemplate{remark}{default} }

%%% Tabela
\begin{longtblr}
{
  colspec = {cccccccc},
  width = 1\linewidth,
  hlines = {white},
  vlines = {white},
  rowhead = 1, rowfoot = 0,
  row{1} = {font=\bfseries, bg=gray8, fg=black},
  column{1} = {font=\bfseries, bg=gray8, fg=black},
  cell{1}{1} = {bg=white},
  cell{2-Z}{2-Z} = {bg=gray9},
  cell{3}{8} = {bg=white},
}
 & {上\\ \normalsize descer} & {下\\ \normalsize subir} & {进\\ \normalsize entrar} & {出\\ \normalsize sair} & {回\\ \normalsize retornar} & {过\\ \normalsize atravessar} & {起\\ \normalsize levantar} \\
{来\\ \normalsize vir}  &  上来 &  下来 &  进来 &  出来 &  回来 &  过来 &  起来 \\
{去\\ \normalsize ir }  &  上去 &  下去 &  进去 &  出去 &  回去 &  过去 &  \\ 
\end{longtblr}

%%%%% EOF %%%%%


\clearpage
\pagestyle{plain}
\chapter{Locativos}
%%%%%%%%%%%%%%%%%%%%%%%%%%%%%%%%%%%%%%%%%%%%%%%%%%%%%%%%%%%%%%%%%%%%%%%%%%%%%%%
%%%%%%%%%%%%%%%%%%%%%%%%%%%%%%%%%%%%%%%%%%%%%%%%%%%%%%%%%%%%%%%%%%%%%%%%%%%%%%%
%%%%%                                                                     %%%%%
%%%%% locativos.tex:                                                      %%%%%
%%%%% Tabela com os locativos chineses                                    %%%%%
%%%%%                                                                     %%%%%
%%%%%%%%%%%%%%%%%%%%%%%%%%%%%%%%%%%%%%%%%%%%%%%%%%%%%%%%%%%%%%%%%%%%%%%%%%%%%%%
%%%%%%%%%%%%%%%%%%%%%%%%%%%%%%%%%%%%%%%%%%%%%%%%%%%%%%%%%%%%%%%%%%%%%%%%%%%%%%%

%%% Ajustes para a tabela
\DefTblrTemplate{caption}{default}{}
\DefTblrTemplate{capcont}{default}{ \UseTblrTemplate{conthead-text}{default} }
\DefTblrTemplate{contfoot-text}{default}{Continua na próxima página.}
\DefTblrTemplate{conthead-text}{default}{(Continuação)}
\DefTblrTemplate{firsthead}{default}{ \UseTblrTemplate{caption}{default} }
\DefTblrTemplate{middlehead,lasthead}{default}{ \UseTblrTemplate{conthead}{default} }
\DefTblrTemplate{firstfoot,middlefoot}{default}{ \UseTblrTemplate{contfoot}{default} }
\DefTblrTemplate{lastfoot}{default}{ \UseTblrTemplate{note}{default} \UseTblrTemplate{remark}{default} }

%%% Tabela
\begin{longtblr}
{
 colspec = {cccccc},
 width = 1\linewidth,
 hlines = {white},
 vlines = {white},
 rowhead = 1, rowfoot = 0,
 row{1} = {font=\bfseries, bg=gray8, fg=black},
 column{1} = {font=\bfseries, bg=gray8, fg=black},
 cell{1}{1} = {bg=white},
 cell{2-Z}{2-Z} = {bg=gray9},
 cell{6}{5-6} = {bg=white},
 cell{7}{2-4} = {bg=white},
 cell{9}{2-5} = {bg=white},
 cell{10}{3-6} = {bg=white},
 cell{11}{2-5} = {bg=white},
 cell{12}{4-6} = {bg=white},
 cell{13}{4-6} = {bg=white},
}
                                           & {边\\   \normalsize\dpy{bian1}}        & {面\\   \normalsize\dpy{mian4}}        & {头\\   \normalsize\dpy{tou5}}        & {以\\   \normalsize\dpy{yi3}}        & {之\\   \normalsize\dpy{zhi1}}        \\
{上\\ \normalsize\dpy{shang4}\\ sobre}     & {上边\\ \normalsize\dpy{shang4 bian1}} & {上面\\ \normalsize\dpy{shang4 mian4}} & {上头\\ \normalsize\dpy{shang4 tou5}} & {以上\\ \normalsize\dpy{yi3 shang4}} & {之上\\ \normalsize\dpy{zhi1 shang4}} \\
{下\\ \normalsize\dpy{xia4}\\ sob}         & {下边\\ \normalsize\dpy{xia4 bian1}}   & {下面\\ \normalsize\dpy{xia4 mian4}}   & {下头\\ \normalsize\dpy{xia4 tou5}}   & {以下\\ \normalsize\dpy{yi3 xia4}}   & {之下\\ \normalsize\dpy{zhi1 xia4}}   \\
{前\\ \normalsize\dpy{qian2}\\ na frente}  & {前边\\ \normalsize\dpy{qian2 bian1}}  & {前面\\ \normalsize\dpy{qian2 mian4}}  & {前头\\ \normalsize\dpy{qian2 tou5}}  & {以前\\ \normalsize\dpy{yi3 qian2}}  & {之前\\ \normalsize\dpy{zhi1 qian2}}  \\
{后\\ \normalsize\dpy{hou4}\\ atrás}       & {后边\\ \normalsize\dpy{hou4 bian1}}   & {后面\\ \normalsize\dpy{hou4 mian4}}   & {后头\\ \normalsize\dpy{hou4 tou5}}   & {以后\\ \normalsize\dpy{yi3 hou4}}   & {之后\\ \normalsize\dpy{zhi1 hou4}}   \\
{里\\ \normalsize\dpy{li3}\\ dentro}       & {里边\\ \normalsize\dpy{li3 bian1}}    & {里面\\ \normalsize\dpy{li3 mian4}}    & {里头\\ \normalsize\dpy{li3 tou5}}    &                                      &                                       \\
{内\\ \normalsize\dpy{nei4}\\ no interior} &                                        &                                        &                                       & {以内\\ \normalsize\dpy{yi3 nei4}}   & {之内\\ \normalsize\dpy{zhi1 nei4}}   \\
{外\\ \normalsize\dpy{wai4}\\ no exterior} & {外边\\ \normalsize\dpy{wai4 bian1}}   & {外面\\ \normalsize\dpy{wai4 mian4}}   & {外头\\ \normalsize\dpy{wai4 tou5}}   & {以外\\ \normalsize\dpy{yi3 wai4}}   & {之外\\ \normalsize\dpy{zhi1 wai4}}   \\
{间\\ \normalsize\dpy{jian1}\\ entre}      &                                        &                                        &                                       &                                      & {之间\\ \normalsize\dpy{zhi1 jian1}}  \\
{旁\\ \normalsize\dpy{pang2}\\ ao lado}    & {旁边\\ \normalsize\dpy{pang2 bian1}}  &                                        &                                       &                                      &                                       \\
{中\\ \normalsize\dpy{zhong1}\\ no meio}   &                                        &                                        &                                       &                                      & {之中\\ \normalsize\dpy{zhi1 zhong1}} \\
{左\\ \normalsize\dpy{zuo3}\\ à esquerda}  & {左边\\ \normalsize\dpy{zuo3 bian1}}   & {左面\\ \normalsize\dpy{zuo3 mian4}}   &                                       &                                      &                                       \\
{右\\ \normalsize\dpy{you4}\\ à direita}   & {右边\\ \normalsize\dpy{you4 bian1}}   & {右面\\ \normalsize\dpy{you4 mian4}}   &                                       &                                      &                                       \\
\pagebreak
{东\\ \normalsize\dpy{dong1}\\ no leste}   & {东边\\ \normalsize\dpy{dong1 bian1}}  & {东面\\ \normalsize\dpy{dong1 mian4}}  & {东头\\ \normalsize\dpy{dong1 tou5}}  & {以东\\ \normalsize\dpy{yi3 dong1}}  & {之东\\ \normalsize\dpy{zhi1 dong1}}  \\
{南\\ \normalsize\dpy{nan2}\\ no sul}      & {南边\\ \normalsize\dpy{nan2 bian1}}   & {南面\\ \normalsize\dpy{nan2 mian4}}   & {南头\\ \normalsize\dpy{nan2 tou5}}   & {以南\\ \normalsize\dpy{yi3 nan2}}   & {之南\\ \normalsize\dpy{zhi1 nan2}}   \\
{西\\ \normalsize\dpy{xi1}\\ no oeste}     & {西边\\ \normalsize\dpy{xi1 bian1}}    & {西面\\ \normalsize\dpy{xi1 mian4}}    & {西头\\ \normalsize\dpy{xi1 tou5}}    & {以西\\ \normalsize\dpy{yi3 xi1}}    & {之西\\ \normalsize\dpy{zhi1 xi1}}    \\
{北\\ \normalsize\dpy{bei3}\\ n norte}     & {北边\\ \normalsize\dpy{bei3 bian1}}   & {北面\\ \normalsize\dpy{bei3 mian4}}   & {北头\\ \normalsize\dpy{bei3 tou5}}   & {以北\\ \normalsize\dpy{yi3 bei3}}   & {之北\\ \normalsize\dpy{zhi1 bei3}}   \\
\end{longtblr}

%%%%% EOF %%%%%


\clearpage
\pagestyle{plain}
\chapter{Radicais Kangxi}
%%%%%%%%%%%%%%%%%%%%%%%%%%%%%%%%%%%%%%%%%%%%%%%%%%%%%%%%%%%%%%%%%%%%%%%%%%%%%%%
%%%%%%%%%%%%%%%%%%%%%%%%%%%%%%%%%%%%%%%%%%%%%%%%%%%%%%%%%%%%%%%%%%%%%%%%%%%%%%%
%%%%%                                                                     %%%%%
%%%%% radicais_kangxi.tex:                                                %%%%%
%%%%% Lista dos 214 radicais Kangxi utilizados nos caracteres chineses.   %%%%%
%%%%%                                                                     %%%%%
%%%%%%%%%%%%%%%%%%%%%%%%%%%%%%%%%%%%%%%%%%%%%%%%%%%%%%%%%%%%%%%%%%%%%%%%%%%%%%%
%%%%%%%%%%%%%%%%%%%%%%%%%%%%%%%%%%%%%%%%%%%%%%%%%%%%%%%%%%%%%%%%%%%%%%%%%%%%%%%

%%% Ajustes para a tabela
\DefTblrTemplate{caption}{default}{}
\DefTblrTemplate{capcont}{default}{ \UseTblrTemplate{conthead-text}{default} }
\DefTblrTemplate{contfoot-text}{default}{Continua na próxima página.}
\DefTblrTemplate{conthead-text}{default}{(Continuação)}
\DefTblrTemplate{firsthead}{default}{ \UseTblrTemplate{caption}{default} }
\DefTblrTemplate{middlehead,lasthead}{default}{ \UseTblrTemplate{conthead}{default} }
\DefTblrTemplate{firstfoot,middlefoot}{default}{ \UseTblrTemplate{contfoot}{default} }
\DefTblrTemplate{lastfoot}{default}{ \UseTblrTemplate{note}{default} \UseTblrTemplate{remark}{default} }

%%% Tabela
\begin{longtblr}
{
  colspec = {|r|ll|l|l|}, hlines,
  width = 1\linewidth,
  rowhead = 1, rowfoot = 0,
  row{1} = {font=\bfseries, fg=white, bg=black},
  row{2-Z} = {font=\normalfont},
}
\textbf{Nº} & \SetCell[c=2]{c}\textbf{Radical e\\Variantes} & 2-2 & \textbf{Tradução} & \textbf{Pinyin} \\
1 & 一 & & um & \dictpinyin{yi1} \\
2 & 丨 & & linha & \dictpinyin{shu4} \\
3 & 丶 & & ponto, indica um fim & \dictpinyin{dian3} \\
4 & 丿 & 乀,乁 & cortar, dobrar & \dictpinyin{pie3} \\
5 & 乙 & 乚、乛、⺄ & segundo, anzol & \dictpinyin{yi3} \\
6 & 亅 & & gancho & \dictpinyin{gou1} \\
7 & 二 & & dois & \dictpinyin{er4} \\
8 & 亠 & & tampa & \dictpinyin{tou2} \\
9 & 人 & 亻、𠆢 & pessoa & \dictpinyin{ren2} \\
10 & 儿 & & pernas & \dictpinyin{er2} \\
11 & 入 & & entrar, juntar-se & \dictpinyin{ru4} \\
12 & 八 & 丷 & oito & \dictpinyin{ba1} \\
13 & 冂 & & largo, exterior & \dictpinyin{jiong3} \\
14 & 冖 & & capa de pano & \dictpinyin{mi4} \\
15 & 冫 & & gelo & \dictpinyin{bing1} \\
16 & 几 & & mesa pequena & \dictpinyin{ji1},\dictpinyin{ji3} \\
17 & 凵 & & receptáculo, caixa aberta & \dictpinyin{qu3} \\
18 & 刀 & 刂、⺈ & faca & \dictpinyin{dao1} \\
19 & 力 & & poder, força & \dictpinyin{li4} \\
20 & 勹 & & invólucro & \dictpinyin{bao1} \\
21 & 匕 & & colher & \dictpinyin{bi3} \\
22 & 匚 & & caixa & \dictpinyin{fang1} \\
23 & 匸 & & compartimento oculto & \dictpinyin{xi3} \\
24 & 十 & & dez, completo, perfeito & \dictpinyin{shi2} \\
25 & 卜 & & advinhação, divinação & \dictpinyin{bu3} \\
26 & 卩 & 㔾 & foca & \dictpinyin{jie2} \\
27 & 厂 & & penhasco, precipício & \dictpinyin{han4} \\
28 & 厶 & & privado & \dictpinyin{si1} \\
29 & 又 & & mão direita, e, novamente & \dictpinyin{you4} \\
30 & 口 & & boca & \dictpinyin{kou3} \\
31 & 囗 & & compartimento, recinto & \dictpinyin{wei2} \\
32 & 土 & & terra & \dictpinyin{tu3} \\
33 & 士 & & acadêmico, bacharel & \dictpinyin{shi4} \\
34 & 夂 & & ir & \dictpinyin{zhi1} \\
35 & 夊 & & ir devagar & \dictpinyin{sui1} \\
36 & 夕 & & tarde, pôr do sol & \dictpinyin{xi1} \\
37 & 大 & & grande, muito & \dictpinyin{da4} \\
38 & 女 & & mulher, fêmea & \dictpinyin{nv3} \\
39 & 子 & & criança, semente & \dictpinyin{zi3} \\
40 & 宀 & & teto, telhado & \dictpinyin{mian2} \\
41 & 寸 & & polegar, polegada & \dictpinyin{cun4} \\
42 & 小 & ⺌、⺍ & pequeno, insignificante & \dictpinyin{xiao3} \\
43 & 尢 & 尣 & manco, coxo & \dictpinyin{you2} \\
44 & 尸 & & cadáver & \dictpinyin{shi1} \\
45 & 屮 & & brotar, germinar & \dictpinyin{che4} \\
46 & 山 & & montanha & \dictpinyin{shan1} \\
47 & 巛 & 川 & rio & \dictpinyin{chuan1} \\
48 & 工 & & trabalho & \dictpinyin{gong1} \\
49 & 己 & ⺒ & próprio, a si mesmo & \dictpinyin{ji3} \\
50 & 巾 & & turbante, cachecol & \dictpinyin{jin1} \\
51 & 干 & & oposto, seco & \dictpinyin{gan1} \\
52 & 幺 & 么 & baixo, minúsculo & \dictpinyin{yao1} \\
53 & 广 & & casa em um penhasco & \dictpinyin{guang3} \\
54 & 廴 & & passada longa & \dictpinyin{yin3} \\
55 & 廾 & & duas mãos, vinte, arco & \dictpinyin{gong3} \\
56 & 弋 & & tiro, flecha & \dictpinyin{yi4} \\
57 & 弓 & & arco & \dictpinyin{gong1} \\
58 & 彐 & 彑 & focinho de porco & \dictpinyin{ji4} \\
59 & 彡 & & cerda, barba & \dictpinyin{shan1} \\
60 & 彳 & & passo & \dictpinyin{chi4} \\
61 & 心 & 忄、⺗ & coração, mente & \dictpinyin{xin1} \\
62 & 戈 & & lança & \dictpinyin{ge1} \\
63 & 戶 & 户、戸 & porta, casa & \dictpinyin{hu4} \\
64 & 手 & 扌、龵 & mão & \dictpinyin{shou3} \\
65 & 支 & & ramo & \dictpinyin{zhi1} \\
66 & 攴 & 攵 & tocar, bater levemente & \dictpinyin{pu1} \\
67 & 文 & & escrita, literatura & \dictpinyin{wen2} \\
68 & 斗 & & objeto em forma de concha & \dictpinyin{dou3} \\
69 & 斤 & & machado & \dictpinyin{jin1} \\
70 & 方 & & quadrado & \dictpinyin{fang1} \\
71 & 无 & 旡 & não, nada, negativo & \dictpinyin{wu2} \\
72 & 日 & & sol, dia & \dictpinyin{ri4} \\
73 & 曰 & & dizer, falar & \dictpinyin{yue1} \\
74 & 月 & & lua, mês & \dictpinyin{yue4} \\
75 & 木 & & árvore & \dictpinyin{mu4} \\
76 & 欠 & & falta, não ter, hiato & \dictpinyin{qian4} \\
77 & 止 & & parar & \dictpinyin{zhi3} \\
78 & 歹 & 歺 & morte, decadência & \dictpinyin{dai3} \\
79 & 殳 & & arma, lança & \dictpinyin{shu1} \\
80 & 毋 & 母 & mãe, não faça & \dictpinyin{mu3} \\
81 & 比 & & comparar, competir & \dictpinyin{bi3} \\
82 & 毛 & & pelagem & \dictpinyin{mao2} \\
83 & 氏 & & clã, linhagem & \dictpinyin{shi4} \\
84 & 气 & & ar, vapor, respiração & \dictpinyin{qi4} \\
85 & 水 & 氵、氺 & água & \dictpinyin{shui3} \\
86 & 火 & 灬 & fogo & \dictpinyin{huo3} \\
87 & 爪 & 爫 & garrai, unha & \dictpinyin{zhao3} \\
88 & 父 & & pai, luz & \dictpinyin{fu4} \\
89 & 爻 & & duplo x, trigramas & \dictpinyin{yao2} \\
90 & 爿 & 丬 & metade de um tronco, madeira rachada & \dictpinyin{pan2} \\
91 & 片 & & fatia, filme & \dictpinyin{pian4} \\
92 & 牙 & & dente, presa & \dictpinyin{ya2} \\
93 & 牛 & 牜、⺧ & boi, vaca & \dictpinyin{niu2} \\
94 & 犬 & 犭 & cão & \dictpinyin{quan3} \\
95 & 玄 & & escuro, profundo & \dictpinyin{xuan2} \\
96 & 玉 & 王、玊 & jade & \dictpinyin{yu4} \\
97 & 瓜 & & melão & \dictpinyin{gua1} \\
98 & 瓦 & & telha & \dictpinyin{wa3} \\
99 & 甘 & & doce & \dictpinyin{gan1} \\
100 & 生 & & vida & \dictpinyin{sheng1} \\
101 & 用 & & usar & \dictpinyin{yong4} \\
102 & 田 & & campo, arrozal & \dictpinyin{tian2} \\
103 & 疋 & ⺪& pedaço de pano & \dictpinyin{pi3} \\
104 & 疒 & & doença & \dictpinyin{ne4} \\
105 & 癶 & & pegadas, pernas & \dictpinyin{bo1} \\
106 & 白 & & branco & \dictpinyin{bai2} \\
107 & 皮 & & pele, couro & \dictpinyin{pi2} \\
108 & 皿 & & prato & \dictpinyin{min3} \\
109 & 目 & ⺫ & olho & \dictpinyin{mu4} \\
110 & 矛 & & lança & \dictpinyin{mao2} \\
111 & 矢 & & seta, flecha & \dictpinyin{shi3} \\
112 & 石 & & pedra & \dictpinyin{shi2} \\
113 & 示 & 礻& espírito, ancestral, veneração & \dictpinyin{shi4} \\
114 & 禸 & & trilha & \dictpinyin{rou2} \\
115 & 禾 & & grão & \dictpinyin{he2} \\
116 & 穴 & & caverna & \dictpinyin{xue2} \\
117 & 立 & & ficar em péi, ereto & \dictpinyin{li4} \\
118 & 竹 & ⺮ & bambu & \dictpinyin{zhu2} \\
119 & 米 & & arroz & \dictpinyin{mi3} \\
120 & 糸 & 纟、糹 & seda & \dictpinyin{mi4} \\
121 & 缶 & & pote, jarra & \dictpinyin{fou3} \\
122 & 网 & ⺲、罓、⺳ & rede & \dictpinyin{wang3} \\
123 & 羊 & ⺶、⺷ & ovelha, cabra & \dictpinyin{yang2} \\
124 & 羽 & & pena & \dictpinyin{yu3} \\
125 & 老 & 耂 & velho & \dictpinyin{lao3} \\
126 & 而 & & e, mas & \dictpinyin{er2} \\
127 & 耒 & & arado & \dictpinyin{lei3} \\
128 & 耳 & & orelha & \dictpinyin{er3} \\
129 & 聿 & ⺺、⺻ & escova & \dictpinyin{yu4} \\
130 & 肉 & 月、⺼ & carne & \dictpinyin{rou4} \\
131 & 臣 & & ministro, oficial & \dictpinyin{chen2} \\
132 & 自 & & próprio, auto-- & \dictpinyin{zi4} \\
133 & 至 & & chegar & \dictpinyin{zhi4} \\
134 & 臼 & & argamassa, ligação & \dictpinyin{jiu4} \\
135 & 舌 & & língua & \dictpinyin{she2} \\
136 & 舛 & & opor & \dictpinyin{chuan3} \\
137 & 舟 & & barco & \dictpinyin{zhou1} \\
138 & 艮 & & parada, quietude & \dictpinyin{gen3} \\
139 & 色 & & cor, forma & \dictpinyin{se4} \\
140 & 艸 & ⺿ & grama & \dictpinyin{cao3} \\
141 & 虍 & & tigre & \dictpinyin{hu1} \\
142 & 虫 & & inseto, verme & \dictpinyin{chong2} \\
143 & 血 & & sangue & \dictpinyin{xue4} \\
144 & 行 & & andar, ir, fazer & \dictpinyin{xing2} \\
145 & 衣 & ⻂& roupa & \dictpinyin{yi1} \\
146 & 襾 & 西、覀 & capa, oeste & \dictpinyin{ya4} \\
147 & 見 & 见 & ver & \dictpinyin{jian4} \\
148 & 角 & ⻆、⻇ & chifre & \dictpinyin{jiao3} \\
149 & 言 & 讠、訁 & palavra, linguagem & \dictpinyin{yan2} \\
150 & 谷 & & vale & \dictpinyin{gu3} \\
151 & 豆 & & feijão, fava & \dictpinyin{dou4} \\
152 & 豕 & & porco & \dictpinyin{shi3} \\
153 & 豸 & & texugo, inseto sem pernas & \dictpinyin{zhi4} \\
154 & 貝 & 贝 & concha & \dictpinyin{bei4} \\
155 & 赤 & & vermelho, nu & \dictpinyin{chi4} \\
156 & 走 & & correr & \dictpinyin{zou3} \\
157 & 足 & ⻊& pé & \dictpinyin{zu2} \\
158 & 身 & & corpo & \dictpinyin{shen1} \\
159 & 車 & 车 & carroça, carro & \dictpinyin{che1} \\
160 & 辛 & & amargo & \dictpinyin{xin1} \\
161 & 辰 & & manhã & \dictpinyin{chen2} \\
162 & 辵 & ⻌、⻍、⻎ & caminhar & \dictpinyin{chuo4} \\
163 & 邑 & ⻏ & cidade & \dictpinyin{yi4} \\
164 & 酉 & & vinho, álcool & \dictpinyin{you3} \\
165 & 釆 & & distinto & \dictpinyin{bian4} \\
166 & 里 & & aldeia, vila & \dictpinyin{li3} \\
167 & 金 & 钅、釒 & ouro, metal & \dictpinyin{jin1} \\
168 & 長 & 长、镸 & longo, crescer & \dictpinyin{zhang3} \\
169 & 門 & 门 & portão, porta & \dictpinyin{men2} \\
170 & 阜 & ⻖ & montei, barragem & \dictpinyin{fu4} \\
171 & 隶 & & escravo & \dictpinyin{li4} \\
172 & 隹 & & pássaro de cauda curta & \dictpinyin{zhui1} \\
173 & 雨 & & chuva & \dictpinyin{yu3} \\
174 & 靑 & 青 & azul, verde ou preto & \dictpinyin{qing1} \\
175 & 非 & & errado & \dictpinyin{fei1} \\
176 & 面 & 靣 & face & \dictpinyin{mian4} \\
177 & 革 & & couro, couro cru & \dictpinyin{ge2} \\
178 & 韋 & 韦 & couro tingido & \dictpinyin{wei2} \\
179 & 韭 & & alho-poró & \dictpinyin{jiu3} \\
180 & 音 & & som & \dictpinyin{yin1} \\
181 & 頁 & 页 & folha, página & \dictpinyin{ye4} \\
182 & 風 & 风 & vento & \dictpinyin{feng1} \\
183 & 飛 & 飞 & voar & \dictpinyin{fei1} \\
184 & 食 & 饣、飠 & alimento, comer & \dictpinyin{shi2} \\
185 & 首 & & cabeça & \dictpinyin{shou3} \\
186 & 香 & & perfume, aroma & \dictpinyin{xiang1} \\
187 & 馬 & 马 & cavalo & \dictpinyin{ma3} \\
188 & 骨 & ⻣ & osso & \dictpinyin{gu3} \\
189 & 高 & 髙 & alto & \dictpinyin{gao1} \\
190 & 髟 & & cabelo & \dictpinyin{biao1} \\
191 & 鬥 & & luta & \dictpinyin{dou4} \\
192 & 鬯 & & vinho sacrificial & \dictpinyin{chang4} \\
193 & 鬲 & & caldeirão, tripé & \dictpinyin{ge2} \\
194 & 鬼 & & fantasma, demônio & \dictpinyin{gui3} \\
195 & 魚 & 鱼 & peixe & \dictpinyin{yu2} \\
196 & 鳥 & 鸟 & pássaro & \dictpinyin{niao3} \\
197 & 鹵 & 卤 & sal & \dictpinyin{lu3} \\
198 & 鹿 & & corça, veado & \dictpinyin{lu4} \\
199 & 麥 & 麦 & trigo & \dictpinyin{mai4} \\
200 & 麻 & & cânhamo, linho & \dictpinyin{ma2} \\
201 & 黃 & 黄 & amarelo & \dictpinyin{huang4} \\
202 & 黍 & & milhete, painço & \dictpinyin{shu3} \\
203 & 黑 & & preto & \dictpinyin{hei1} \\
204 & 黹 & & bordado & \dictpinyin{zhi3} \\
205 & 黽 & 黾 & sapo, anfíbio & \dictpinyin{mian3} \\
206 & 鼎 & & tripé de sacrifício, caldeirão de três pernas & \dictpinyin{ding3} \\
207 & 鼓 & & tambor & \dictpinyin{gu3} \\
208 & 鼠 & 鼡 & rato, camundongo & \dictpinyin{shu3} \\
209 & 鼻 & & nariz & \dictpinyin{bi2} \\
210 & 齊 & 齐、斉 & mesmo, uniformemente & \dictpinyin{qi2} \\
211 & 齒 & 齿 & dente & \dictpinyin{chi3} \\
212 & 龍 & 龙 & dragão & \dictpinyin{long2} \\
213 & 龜 & 龟 & tartaruga & \dictpinyin{gui1} \\
214 & 龠 &   & flauta & \dictpinyin{yue4} \\
\end{longtblr}

%%%%% EOF %%%%%


\printindex[sradical]

\end{document}

%%%%% EOF %%%%
