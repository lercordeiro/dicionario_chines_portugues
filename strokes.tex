
%%%%%%%%%%%%%%%%%%%%%%%%%%%%%%%%%%%%%%%%%
% LuaLaTex
%
% Dicionário Chinês -> Português
% Autor: Luiz Eduardo Roncato Cordeiro
%
% Licença:
% CC BY-NC-SA 3.0 (http://creativecommons.org/licenses/by-nc-sa/3.0/)
%%%%%%%%%%%%%%%%%%%%%%%%%%%%%%%%%%%%%%%%%

\documentclass[a4paper,9pt,twoside,openright,showtrims,final]{memoir}

\usepackage[brazilian,chinese-hans,provide=*]{babel}
\usepackage{fontspec}
\usepackage[dvipsnames]{xcolor}
\usepackage{imakeidx}
\usepackage[inline]{enumitem}
\usepackage{zhnumber}
\usepackage{tikz}
\usepackage[hyperindex]{hyperref}
\usepackage{pifont}
\usepackage{xstring}
\usepackage{xifthen}
\usepackage{tabularray}
\usepackage[most]{tcolorbox}
\usepackage{luacode}
\usepackage{multicol}

% Meus Comandos
%%%%%%%%%%%%%%%%%%%%%%%%%%%%%%%%%%%%%%%%%%%%%%%%%%%%%%%%%%%%%%%%%%%%%%%%%%%%%%%
%%%%%%%%%%%%%%%%%%%%%%%%%%%%%%%%%%%%%%%%%%%%%%%%%%%%%%%%%%%%%%%%%%%%%%%%%%%%%%%
%%%%%                                                                     %%%%%
%%%%% Funções e Ajustes dos Documentos do Dicionário                      %%%%%
%%%%%                                                                     %%%%%
%%%%%%%%%%%%%%%%%%%%%%%%%%%%%%%%%%%%%%%%%%%%%%%%%%%%%%%%%%%%%%%%%%%%%%%%%%%%%%%
%%%%%%%%%%%%%%%%%%%%%%%%%%%%%%%%%%%%%%%%%%%%%%%%%%%%%%%%%%%%%%%%%%%%%%%%%%%%%%%

%%% Espaçamento das linhas normal
\SingleSpacing

%%% Hyperref em modo 'draft' não gera os hiperlinks
\hypersetup{final}

%%% Ajustes da separação das colunas quando em modo texto de 2 colunas
\setlength{\columnsep}{1.2em}
\setlength{\columnseprule}{0.1mm}

%%% Estilo do capítulo, o melhor que encontrei
\chapterstyle{dash}

%%% Sem identação
\setlength{\parindent}{0cm}
\setlength{\parskip}{0.15\baselineskip}

%%% Ajuste das margens do documento
\setlrmarginsandblock{3cm}{2cm}{*}
\setulmarginsandblock{2cm}{*}{1}
\checkandfixthelayout

%%% Pra evitar viúvas e órfãs
\clubpenalty=10000
\widowpenalty=10000
\raggedbottom

%%% Usando a fonte NoTofu do Google.
\babelfont{rm}[
 Renderer=Harfbuzz,
 Ligatures=TeX,
 BoldFont={NotoSerifCJKsc-Bold},
 BoldSlantedFont={NotoSansCJKsc-Regular},
 AutoFakeSlant=0.25,
 SlantedFeatures={FakeSlant=0.25},
 BoldSlantedFeatures={FakeSlant=0.25}]{Noto Serif CJK SC}
\babelfont{sf}[Renderer=Harfbuzz,Ligatures=TeX]{Noto Sans CJK SC}
\babelfont{tt}[Renderer=Harfbuzz,Ligatures=TeX]{Noto Sans Mono CJK SC}

%%% Ajustes do MultiCol: parar com a indentação do primeiro parágrafo
%\AddToHook{env/multicols/begin}{\AddToHookNext{para/begin}{\OmitIndent}}

%%% Ajustes do Sumário
\setcounter{secnumdepth}{2}
\makeatletter
\renewcommand{\@pnumwidth}{2em} 
\renewcommand{\@tocrmarg}{4em}
\makeatother
\renewcommand\cftbeforechapterskip{5pt plus 1pt}

%%% Cria 'Lista de Hanzis'
\newcommand{\listlohname}{Primeiros Hanzis}
\newlistof{listoffirsthanzis}{loh}{\listlohname}

%%% Ajustes de Cabeçalhos e Rodapés
\setheadfoot{14pt}{28pt}

% Estilo "plain"
\makefootrule{plain}{\textwidth}{\normalrulethickness}{2pt}
\ifdraftdoc
 \makeevenfoot{plain}{\thepage}{汉葡词典}{Draft}
 \makeoddfoot{plain}{Draft}{汉葡词典}{\thepage}
\else
 \makeevenfoot{plain}{\thepage}{汉葡词典}{}
 \makeoddfoot{plain}{}{汉葡词典}{\thepage}
\fi

% Estilo "dictionary"
\makepagestyle{dictionary}
\makeheadrule{dictionary}{\textwidth}{\normalrulethickness}
\makefootrule{dictionary}{\textwidth}{\normalrulethickness}{2pt}
\ifdraftdoc
 \makeevenhead{dictionary}{\rightmark}{Draft}{\leftmark}
 \makeoddhead{dictionary}{\rightmark}{Draft}{\leftmark}
 \makeevenfoot{dictionary}{\thepage}{汉葡词典}{Draft}
 \makeoddfoot{dictionary}{Draft}{汉葡词典}{\thepage}
\else
 \makeevenhead{dictionary}{\rightmark}{}{\leftmark}
 \makeoddhead{dictionary}{\rightmark}{}{\leftmark}
 \makeevenfoot{dictionary}{\thepage}{汉葡词典}{}
 \makeoddfoot{dictionary}{}{汉葡词典}{\thepage}
\fi

\newcommand{\boxedsec}[1]
 {%
  \begin{tcolorbox}%
   [%
    enhanced,%
    nobeforeafter,%
    before={\noindent},%
    colframe=black,%
    colback=black!20!white,%
    boxrule=2pt,%
    leftrule=4mm,%
    left=0mm,%
    right=0mm,%
    top=0mm,%
    bottom=0mm%
   ]
   \hfill\LARGE\bfseries#1
  \end{tcolorbox}
 }
\setsecheadstyle{\boxedsec}
\newcommand{\sectionbreak}{\phantomsection}

\newcommand{\boxedsubsec}[1]
 {%
  \begin{tcolorbox}%
   [%
    enhanced,%
    nobeforeafter,%
    before={\noindent},%
    colframe=black,%
    colback=black!10!white,%
    boxrule=2pt,%
    leftrule=2mm,%
    left=0mm,%
    right=0mm,%
    top=0mm,%
    bottom=0mm%
   ]
   \hfill\Large#1
  \end{tcolorbox}
 }
\setsubsecheadstyle{\boxedsubsec}

%%% Estilo das caixas dos verbetes
\newtcolorbox{lightbox}%
 {%
  enhanced,%
  size=fbox,%
  colframe=black,%
  colback=white,%
  boxrule=1pt,%
  toprule=3pt,%
  left=0mm,%
  right=0mm,%
  top=0mm,%
  bottom=0mm,%
  middle=0mm,%
  nobeforeafter,%
  segmentation empty,%
  before={\noindent}%
 }
\newtcolorbox{darkbox}%
 {%
  enhanced,%
  size=fbox,%
  colframe=black,%
  colback=black!5!white,%
  boxrule=1pt,%
  toprule=3pt,%
  left=0mm,%
  right=0mm,%
  top=0mm,%
  bottom=0mm,%
  middle=0mm,%
  nobeforeafter,%
  segmentation empty,%
  before={\noindent}%
 }


%%% Variáveis tipo "bool" para dizer se tem ou não os campos
%%% "Veja", "Veja também", "Synonym"" e "Antonym"
%%% nas definições dos verbetes
\newbool{hassee}
\newbool{hasseealso}
\newbool{hassynonym}
\newbool{hasantonym}

%%% Converte os pinyins numéricos em pinyins com marcação de tom
\directlua{dofile "include/tex-sx-pinyin-tonemarks.lua"}

%%% Comandos genéricos usados no Dicionário

% Função "\pinyin" faz a conversão
\protected\def\pinyin#1{\directlua{packagedata.pinyintones.convert ([==[#1]==])}}

\ExplSyntaxOn

% Comando "\dictpinyin", coloca o pinyin entre «»
\NewDocumentCommand{\dictpinyin}{m}{\guillemotleft\pinyin{#1}\guillemotright} 

% Comando "\dpy", gera a string do pinyin utilizada no Dicionário
% Este comando realiza uma série de substituições antes
\NewDocumentCommand{\dpy}{m}%
 {%
  \StrSubstitute{#1}{5}{}[\result]%
  \StrSubstitute{\result}{v}{ü}[\result]%
  \StrSubstitute{\result}{V}{Ü}[\result]%
  \edef\py{\dictpinyin{\result}}%
  \mbox{}\py
 }

% Yin, Yang e Os Oito Trigramas
\newfontfamily\dejavusans{DejaVu Sans}
\DeclareRobustCommand{\Yin}{{\dejavusans\symbol{"268A}}}
\DeclareRobustCommand{\Yang}{{\dejavusans\symbol{"268B}}}
\DeclareRobustCommand{\TrigramHeaven}{{\dejavusans\symbol{"2630}}}
\DeclareRobustCommand{\TrigramLake}{{\dejavusans\symbol{"2631}}}
\DeclareRobustCommand{\TrigramFire}{{\dejavusans\symbol{"2632}}}
\DeclareRobustCommand{\TrigramThunder}{{\dejavusans\symbol{"2633}}}
\DeclareRobustCommand{\TrigramWind}{{\dejavusans\symbol{"2634}}}
\DeclareRobustCommand{\TrigramWater}{{\dejavusans\symbol{"2635}}}
\DeclareRobustCommand{\TrigramMountain}{{\dejavusans\symbol{"2636}}}
\DeclareRobustCommand{\TrigramEarth}{{\dejavusans\symbol{"2637}}}

% Comando "\&", insere o caracgter "&"
%\DeclareRobustCommand{\&}%
% {%
%  \ifdim\fontdimen1\font>0pt%
%   \textsl{\symbol{`\&}}%
%  \else%
%   \symbol{`\&}%
%  \fi%
% }

\NewDocumentCommand{\setvar}{mm}
 {
  % clear an existing variable or allocate a new one
  \tl_clear_new:c { g__youthdoo_var_#1_tl }
  % set to the stated value
  \tl_gset:cn { g__youthdoo_var_#1_tl } { #2 }
 }

\NewExpandableDocumentCommand{\usevar}{m}
 {
  % deliver the contents
  \tl_use:c { g__youthdoo_var_#1_tl }
 }

% Ambiente "enumerate" especial utilizado no dicionário, coloca as definições 
% do verbete em uma lista numerada em linha
\NewDocumentCommand{\dictenumerate}{>{\SplitList{|}}m}
 {%
  \begin{enumerate*}[nosep,label={\arabic*},left=0pt,mode=boxed,font=\bfseries]
   \ProcessList{#1}{\insertitem}
  \end{enumerate*}
 }
\NewDocumentCommand{\insertitem}{>{\TrimSpaces}m}{\item #1}

% Ambiente "enumerate" especial utilizado no dicionário, coloca os exemplos
% das definições do verbete em uma lista numerada em linha, utilizando
% algarismos romanos
\makeatletter
\NewDocumentCommand{\dictexamples}{m>{\SplitList{|}}m}
 {%
  \def\@theword{#1}% 
  \begin{sloppypar}
   \begin{enumerate}[nosep,label=\alph*),left=0pt,mode=boxed,font=\bfseries]
    \ProcessList{#2}{\insertexample}
   \end{enumerate}
  \end{sloppypar}
 }
\NewDocumentCommand{\insertexample}{>{\TrimSpaces}m}
 {%
   \IfSubStr{#1}{===}
   {% Com traducao
     \StrCut{#1}{===}\csH\csP%
     \StrSubstitute{\csH}{\@theword}{\underline{\@theword}}[\csHUL]%
     \item\foreignlanguage{chinese-hans}{\csHUL}\\*\textit{\footnotesize``\csP''}
   }
   {% Sem traducao
     \StrSubstitute{#1}{\@theword}{\underline{\@theword}}[\csHUL]%
     \item\foreignlanguage{chinese-hans}{\csHUL}
   }
 }  
\makeatother

%%% Cria listas especializadas (seelist, seealsolist, synonymlist e antonymlist)
\newlist{seelist}{enumerate}{1}
\newlist{seealsolist}{enumerate}{1}
\newlist{synonymlist}{enumerate}{1}
\newlist{antonymlist}{enumerate}{1}

\setlist[seelist]{label={\roman*)},topsep=0pt,nosep,noitemsep,font=\bfseries}
\setlist[seealsolist]{label={\roman*)},topsep=0pt,nosep,noitemsep,font=\bfseries}
\setlist[synonymlist]{label={\roman*)},topsep=0pt,nosep,noitemsep,font=\bfseries}
\setlist[antonymlist]{label={\roman*)},topsep=0pt,nosep,noitemsep,font=\bfseries}

%%% Cria e inicializa a lista "Veja", "Veja também", "Sinônimo(s)" e "Antônimo(s)"
\newcommand\seerefl{}
\newcommand\seealsorefl{}
\newcommand\synonymrefl{}
\newcommand\antonymrefl{}

%%% Comando "\areference", adiciona um item "Veja", "Veja também", "Antônimo(s)" ou "Sinônimo(s)" na lista,
%%% com os pinyins abaixo dos caracteres
\newcommand{\areference}[2]
 {
  \StrLen{#1}[\hlen]%
  \StrLen{#2}[\plen]%
  \ifnumcomp{\hlen+\plen}{>}{24}
   {%
    \foreignlanguage{chinese-hans}{#1}\ (pág.~\pageref{\l_label_tl : #1 : #2})\\*
    \dpy{#2}
   }
   {%
    \foreignlanguage{chinese-hans}{#1}\ \dpy{#2}\ (pág.~\pageref{\l_label_tl : #1 : #2})
   }
 }

%%% Comando "\definition", gera o texto da definição
\NewDocumentCommand{\definition}{sommo}
 {%
  \begin{midsloppypar}
   \IfBooleanTF{#1}%
    {% Substantivo Próprio
     {\small\ding{108}}\ (\textit{S.P.})\IfValueT{#2}{~[clas.:~#2]}{\ \dictenumerate{#4}}\par
    }%
    {%
     {\small\ding{108}}\ (\textit{#3})\IfValueT{#2}{~[clas.:~#2]}{\ \dictenumerate{#4}}\par
    }%
  \end{midsloppypar}
  \IfValueT{#5}%
   {%
    \IfSubStr{#5}{|}{\textbf{Exemplos:}}{\textbf{Exemplo:}}\dictexamples{\l_hanzi_tl}{#5}\par
   }%
 }

%%% Comando "Variante de"
\NewDocumentCommand{\variantof}{m}
 {
  {\small\ding{108}}\ Variante\ de\ \foreignlanguage{chinese-hans}{#1}\ (p.~\pageref{\l_label_tl : #1 : \l_pinyin_tl})\par
 }

%%% Comando "Veja"
\NewDocumentCommand{\seeref}{m}{\booltrue{hassee}\listgadd{\seerefl}{\l_hanzi_tl : #1}}

%%% Comando "Veja também"
\NewDocumentCommand{\seealsoref}{mm}{\booltrue{hasseealso}\listgadd{\seealsorefl}{#1 : #2}}

%%% Comando "Sinônimo(s)"
\NewDocumentCommand{\synonymref}{mm}{\booltrue{hassynonym}\listgadd{\synonymrefl}{#1 : #2}}

%%% Comando "Antônimo(s)"
\NewDocumentCommand{\antonymref}{mm}{\booltrue{hasantonym}\listgadd{\antonymrefl}{#1 : #2}}

%%% Ambiente "DictionaryEntries" para definir o início das entradas dos verbetes
\NewDocumentEnvironment{DictionaryEntries}{m}%
 {%
  \tl_set:Nn \l_label_tl {#1}
  \pagestyle{dictionary}
  \twocolumn
 }%
 {%
  \onecolumn
 }%

%%% Ambiente "EntryWithPhonetic", para os verbetes
\NewDocumentEnvironment{EntryWithPhonetic}{mO{}mO{}mmooo}%
 {%
  \leavevmode
  \markboth{#1{\tiny\dpy{#3}}}{#1{\tiny\dpy{#3}}}
  \tl_set:Nn \l_hanzi_tl {#1}
  \tl_set:Nn \l_pinyin_tl {#3}
  \tl_set:Nn \l_strokes_tl {#5}
  \boolfalse{hassee}\renewcommand\seerefl{}
  \boolfalse{hasseealso}\renewcommand\seealsorefl{}
  \boolfalse{hassynonym}\renewcommand\synonymrefl{}
  \boolfalse{hasantonym}\renewcommand\antonymrefl{}
  \label{\l_label_tl : #1 : #3}
  \StrLen{#1}[\hlen]%
  \StrLen{#3}[\plen]%
  \begin{lightbox}
   \ifnumcomp{\hlen}{>}{10}
    {%
     \mbox{}\hfill\textsuperscript{\tiny(#5画)}\\
     {\Large\foreignlanguage{chinese-hans}{#1}}
    }
    {%
     {\Large\foreignlanguage{chinese-hans}{#1}}\hfill\textsuperscript{\tiny(#5画)}
    }
   \tcblower
   \ifnumcomp{\plen}{>}{25}
    {%
     {\footnotesize#2\ \dpy{#3}\ #4}\\
    }
    {%
      {\footnotesize#2\ \dpy{#3}\ #4}
    }
   \IfValueT{#7}{\mbox{}\hfill{\tiny#7}}{}%
   \IfValueT{#8}{\mbox{}\hfill{\tiny#8}}{}%
   \IfValueT{#9}{\mbox{}\hfill{\tiny#9}}{}%
   \mbox{}\hfill\IfSubStr{#6}{、}{\tiny Radicais\ #6}{\tiny Radical\ #6}
  \end{lightbox}
 }%
 {%
  \ifbool{hassee}%
   {% Processa as referências "Veja"
    \RenewDocumentCommand\do{>{\SplitArgument{1}{:}}m}{\item \areference ##1}
    \textbf{Veja:}%
    \begin{seelist}
     \dolistloop{\seerefl}
    \end{seelist}
   }{}%
  \ifbool{hasseealso}%
   {% Processa as referências "Veja também"
    \RenewDocumentCommand\do{>{\SplitArgument{1}{:}}m}{\item \areference ##1}
    \textbf{Veja\ também:}%
    \begin{seealsolist}
     \dolistloop{\seealsorefl}
    \end{seealsolist}
   }{}%
  \ifbool{hassynonym}%
   {% Processa as referências "Sinônimo"
    \RenewDocumentCommand\do{>{\SplitArgument{1}{:}}m}{\item \areference ##1}
    \textbf{Sinônimo(s):}%
    \begin{synonymlist}
     \dolistloop{\synonymrefl}
    \end{synonymlist}
   }{}%
  \ifbool{hasantonym}%
   {% Processa as referências "Antônimo"
    \RenewDocumentCommand\do{>{\SplitArgument{1}{:}}m}{\item \areference ##1}
    \textbf{Antônimo(s):}%
    \begin{antonymlist}
     \dolistloop{\antonymrefl}
    \end{antonymlist}
   }{}%
 }

%%% Ambiente "Entry", para os verbetes
\NewDocumentEnvironment{Entry}{mmmooo}%
 {%
  \leavevmode
  \markboth{#1{\tiny(#2画)}}{#1{\tiny(#2画)}}
  \tl_set:Nn \l_hanzi_tl {#1}
  \tl_set:Nn \l_strokes_tl {#2}
  \StrLen{#1}[\hlen]%
  \begin{lightbox}
   \ifnumcomp{\hlen}{>}{10}
    {%
     \mbox{}\hfill\textsuperscript{\tiny(#2画)}\\
     {\Large\foreignlanguage{chinese-hans}{#1}}
    }
    {%
     {\LARGE#1}\hfill\textsuperscript{\tiny(#2画)}
    }
   \tcblower
   \IfValueT{#4}{\mbox{}\hfill{\tiny#4}}{}%
   \IfValueT{#5}{\mbox{}\hfill{\tiny#5}}{}%
   \IfValueT{#6}{\mbox{}\hfill{\tiny#6}}{}%
   \mbox{}\hfill\IfSubStr{#3}{,}{\tiny Radicais\ #3}{\tiny Radical\ #3}
  \end{lightbox}\par
 }{}

%%% Ambiente "Phonetics", para as diversas entradas de fonética da palavra
\NewDocumentEnvironment{Phonetics}{mO{}mO{}O{}}%
 {%
  \tl_set:Nn \l_pinyin_tl {#3}
  \boolfalse{hassee}\renewcommand\seerefl{}
  \boolfalse{hasseealso}\renewcommand\seealsorefl{}
  \boolfalse{hassynonym}\renewcommand\synonymrefl{}
  \boolfalse{hasantonym}\renewcommand\antonymrefl{}
  \label{\l_label_tl : #1 : #3}
   \ding{93}\ #2\ \dpy{#3}\ #4\ \ding{93}\hfill \textbf{#5}
 }%
 {%  
  \ifbool{hassee}%
   {% Processa as referências "Veja"
    \RenewDocumentCommand\do{>{\SplitArgument{1}{:}}m}{\item \areference ##1}
    \textbf{Veja:}%
    \begin{seelist}
     \dolistloop{\seerefl}
    \end{seelist}
   }{}%
  \ifbool{hasseealso}%
   {% Processa as referências "Veja também"
    \RenewDocumentCommand\do{>{\SplitArgument{1}{:}}m}{\item \areference ##1}
    \textbf{Veja\ também:}%
    \begin{seealsolist}
     \dolistloop{\seealsorefl}
    \end{seealsolist}
   }{}%
  \ifbool{hassynonym}%
   {% Processa as referências "Sinônimo"
    \RenewDocumentCommand\do{>{\SplitArgument{1}{:}}m}{\item \areference ##1}
    \textbf{Sinônimo(s):}%
    \begin{synonymlist}
     \dolistloop{\synonymrefl}
    \end{synonymlist}
  }{}%
  \ifbool{hasantonym}%
   {% Processa as referências "Antônimo"
    \RenewDocumentCommand\do{>{\SplitArgument{1}{:}}m}{\item \areference ##1}
    \textbf{Antônimo(s):}%
    \begin{antonymlist}
     \dolistloop{\antonymrefl}
    \end{antonymlist}
   }{}%
 }

\ExplSyntaxOff

%%%%% EOF %%%%%


% Definições de hifenações
\hyphenation{
post-gresql
or-a-cle
mi-cro-soft
}


% Ajustes do PDF
\hypersetup{
  linktoc=page,
  colorlinks=true,
  urlcolor=blue,
  linkcolor=blue,
  citecolor=blue,
  pdftitle={汉葡词典 - Dicionário Chinês-Português},
  pdfsubject={Dicionário Chinês-Português -- Ordenado por Número de Traços},
  pdfauthor={罗学凯, Luiz Eduardo Roncato Cordeiro},
  pdfkeywords={dicionário, chinês, português, instituto confúcio}
}

%%%
%%% Documento começa aqui!
%%%

\begin{document}
\addfontfeatures{CharacterWidth=Proportional}
\selectlanguage{brazilian} % Start in Brazilian

\input{title.tex}

\clearpage
\pagestyle{empty}
\chapter{Sumário}                                                           
\begin{KeepFromToc}
 \renewcommand{\contentsname}{}% Remove \tableofcontents' title/name
 \tableofcontents
\end{KeepFromToc}

\clearpage
\pagestyle{empty}
\chapter{汉葡词典}

%%%%%%%%%%%%%%%%%%%%%%%%
%
% https://en.wikipedia.org/wiki/Chinese_character_orders
%
%%%%%%%%%%%%%%%%%%%%%%%%

Dicionário Chinês-Português ordenado primeiro pelo número de traços,
depois pela ordem do caracter na tabela UTF-8.  As definições são
agrupadas e ordenadas pelo pinyin em cada verbete.

\clearpage
\pagestyle{dictionary}
\begin{DictionaryEntries}{strokes}
 %%%
%%% ∅画
%%%
\section*{∅画}\addcontentsline{toc}{section}{∅画}\addcontentsline{loh}{figure}{\#\#\#\# ∅画}

\begin{Entry}{T-恤}{∅,9}{∅,⼼}
  \begin{Phonetics}{T-恤}{t5 xu4}
    \definition{s.}{camiseta | pulôver | suéter}
  \end{Phonetics}
\end{Entry}

%%%%% EOF %%%%%


 %%%
%%% 1画
%%%
\section*{1画}\addcontentsline{toc}{section}{1画}\addcontentsline{loh}{figure}{\#\#\#\# 1画}

%%%%%%%%%% 一 %%%%%%%%%%
\subsection*{一}\addcontentsline{loh}{figure}{一}

\begin{Entry}{一}{1}{⼀}[Kangxi 1]
  \begin{Phonetics}{一}{yi1}[(quando usado sozinho)][HSK 1]
    \definition{adv.}{uma vez; assim que; indica que duas ações ocorreram em um intervalo de tempo muito curto, uma terminando e a outra começando imediatamente em seguida | indica que primeiro se realiza uma ação e, em seguida, o resultado dessa ação  | indica uma ação única, indicando que a ação é muito curta ou apenas uma tentativa}
    \definition{num.}{um; 1 | pronunciado como \dpy{yao1} quando dito número a número | igual; refere-se ao mesmo ou igual | inteiro; todo; por toda parte | exclusivo ou único | refere-se a algo específico | também; caso contrário; referindo-se a outro ou mais um}
    \definition{part.}{antes de certas palavras para dar ênfase}
    \definition{prep.}{cada; por; toda vez}
    \definition{s.}{uma nota da escala em Gongchepu (工尺谱), correspondente ao 17 na notação musical numerada}
  \seealsoref{工尺谱}{gong1 che3 pu3}
  \end{Phonetics}
  \begin{Phonetics}{一}{yi2}[(antes de quarto tom)][HSK 1]
    \definition{num.}{um; 1 | um (artigo)}
  \end{Phonetics}
  \begin{Phonetics}{一}{yi4}[][HSK 1]
    \definition{adv.}{uma vez | assim que | ao}
    \definition{num.}{um; 1 | um (artigo)}
  \end{Phonetics}
\end{Entry}

\begin{Entry}{一下}{1,3}{⼀,⼀}
  \begin{Phonetics}{一下}{yi2xia4}
    \definition{adv.}{em um curto tempo | rapidamente}
  \end{Phonetics}
\end{Entry}

\begin{Entry}{一下儿}{1,3,2}{⼀,⼀,⼉}
  \begin{Phonetics}{一下儿}{yi2 xia4r5}[][HSK 1,5]
    \definition{s.}{um tempo; um momento}
  \end{Phonetics}
\end{Entry}

\begin{Entry}{一下子}{1,3,3}{⼀,⼀,⼦}
  \begin{Phonetics}{一下子}{yi2 xia4 zi5}[][HSK 5]
    \definition{adv.}{tudo de uma vez; de repente; em pouco tempo; em um curto espaço de tempo}
  \end{Phonetics}
\end{Entry}

\begin{Entry}{一个样}{1,3,10}{⼀,⼈,⽊}
  \begin{Phonetics}{一个样}{yi2ge5yang4}
    \definition{s.}{o mesmo}
  \seealsoref{一样}{yi2yang4}
  \end{Phonetics}
\end{Entry}

\begin{Entry}{一口气}{1,3,4}{⼀,⼝,⽓}
  \begin{Phonetics}{一口气}{yi4 kou3 qi4}[][HSK 5]
    \definition{adv.}{em um só fôlego; sem pausa; fazer algo continuamente}
  \end{Phonetics}
\end{Entry}

\begin{Entry}{一切}{1,4}{⼀,⼑}
  \begin{Phonetics}{一切}{yi2qie4}[][HSK 3]
    \definition{pron.}{tudo; todo; todas as coisas}
  \end{Phonetics}
\end{Entry}

\begin{Entry}{一方面}{1,4,9}{⼀,⽅,⾯}
  \begin{Phonetics}{一方面}{yi4 fang1 mian4}[][HSK 3]
    \definition{s.}{um lado; um dos dois aspectos opostos ou um lado de algo que está relacionado a outro}
  \seealsoref{一方面…,一方面…}{yi4 fang1 mian4 yi4 fang1 mian4}
  \end{Phonetics}
\end{Entry}

\begin{Entry}{一方面…,一方面…}{1,4,9,1,4,9}{⼀,⽅,⾯,⼀,⽅,⾯}
  \begin{Phonetics}{一方面…,一方面…}{yi4 fang1 mian4 yi4 fang1 mian4}[][HSK 3]
    \definition{conj.}{por um lado\dots, por outro lado\dots; conecta duas orações paralelas (devem ser usadas juntas)}[\underline{一方面}觉得兴奋,\underline{一方面}又害怕。===Por um lado, sinto-me entusiasmado, mas, por outro, também sinto medo.]
  \end{Phonetics}
\end{Entry}

\begin{Entry}{一代}{1,5}{⼀,⼈}
  \begin{Phonetics}{一代}{yi2 dai4}[][HSK 6]
    \definition{s.}{uma dinastia | era; época atual | vida; geração; toda a vida de uma pessoa}
  \end{Phonetics}
\end{Entry}

\begin{Entry}{一半}{1,5}{⼀,⼗}
  \begin{Phonetics}{一半}{yi2ban4}[][HSK 1]
    \definition{num.}{metade; em parte; uma metade}
  \end{Phonetics}
\end{Entry}

\begin{Entry}{一句话}{1,5,8}{⼀,⼝,⾔}
  \begin{Phonetics}{一句话}{yi2 ju4 hua4}[][HSK 5]
    \definition{s.}{em resumo; em uma palavra; expressar um conteúdo complexo de forma sucinta | trabalho fácil; fácil de fazer; descrever uma tarefa ou trabalho como muito simples e fácil de realizar}
  \end{Phonetics}
\end{Entry}

\begin{Entry}{一旦}{1,5}{⼀,⽇}
  \begin{Phonetics}{一旦}{yi2dan4}[][HSK 5]
    \definition{adv.}{uma vez; no caso; agora que | de repente; uma vez}
    \definition{s.}{em um único dia; em um tempo muito curto;}
  \end{Phonetics}
\end{Entry}

\begin{Entry}{一生}{1,5}{⼀,⽣}
  \begin{Phonetics}{一生}{yi4 sheng1}[][HSK 2]
    \definition{s.}{vida inteira; toda a vida; ao longo da vida; todo o tempo desde o nascimento até a morte; às vezes exagerado para indicar um longo período de tempo no curso da vida}
  \end{Phonetics}
\end{Entry}

\begin{Entry}{一边}{1,5}{⼀,⾡}
  \begin{Phonetics}{一边}{yi4bian1}[][HSK 1]
    \definition{adj.}{igual; idêntico; da mesma forma}
    \definition{adv.}{enquanto; ao mesmo tempo; simultaneamente; indica que uma ação ocorre simultaneamente a outra ação}
    \definition{s.}{lado; um lado; um aspecto | ambos os lados; ao lado de}
  \end{Phonetics}
\end{Entry}

\begin{Entry}{一会儿}{1,6,2}{⼀,⼈,⼉}
  \begin{Phonetics}{一会儿}{yi2 hui4r5}[][HSK 1,2]
    \definition{adv.}{agora\dots agora\dots; um momento\dots o próximo\dots; usado antes de dois antônimos, indica a alternância de situações}
    \definition{s.}{um pouquinho de tempo; muito pouco tempo}
  \end{Phonetics}
\end{Entry}

\begin{Entry}{一会儿…一会儿…}{1,6,2,1,6,2}{⼀,⼈,⼉,⼀,⼈,⼉}
  \begin{Phonetics}{一会儿…一会儿…}{yi1hui4r5 yi1hui4r5}
    \definition{adv.}{um tempo\dots um tempo\dots}
  \end{Phonetics}
\end{Entry}

\begin{Entry}{一共}{1,6}{⼀,⼋}
  \begin{Phonetics}{一共}{yi2gong4}[][HSK 2]
    \definition{adv.}{completamente; em tudo; no todo}
  \end{Phonetics}
\end{Entry}

\begin{Entry}{一再}{1,6}{⼀,⼌}
  \begin{Phonetics}{一再}{yi2zai4}[][HSK 4]
    \definition{adv.}{repetidamente; de novo e de novo; repetidas vezes; uma e outra vez}
  \end{Phonetics}
\end{Entry}

\begin{Entry}{一同}{1,6}{⼀,⼝}
  \begin{Phonetics}{一同}{yi4tong2}[][HSK 6]
    \definition{adv.}{juntos; ao mesmo tempo e lugar}
  \end{Phonetics}
\end{Entry}

\begin{Entry}{一向}{1,6}{⼀,⼝}
  \begin{Phonetics}{一向}{yi2xiang4}[][HSK 5]
    \definition{adv.}{desde o início; indica do passado até o presente}
  \end{Phonetics}
\end{Entry}

\begin{Entry}{一次性}{1,6,8}{⼀,⽋,⼼}
  \begin{Phonetics}{一次性}{yi2 ci4 xing4}[][HSK 6]
    \definition{adj.}{único; uso único; descartável (produtos); apenas uma vez, sem necessidade ou necessidade de fazer novamente}
  \end{Phonetics}
\end{Entry}

\begin{Entry}{一行}{1,6}{⼀,⾏}
  \begin{Phonetics}{一行}{yi1 xing2}[][HSK 6]
    \definition{s.}{delegação; um grupo viajando junto; festa}
  \end{Phonetics}
\end{Entry}

\begin{Entry}{一齐}{1,6}{⼀,⿑}
  \begin{Phonetics}{一齐}{yi4 qi2}[][HSK 6]
    \definition{adv.}{juntos; em uníssono; simultaneamente; ao mesmo tempo; indica que diferentes sujeitos emitem simultaneamente o mesmo comportamento ou o mesmo sujeito emite vários comportamentos diferentes ao mesmo tempo}
  \end{Phonetics}
\end{Entry}

\begin{Entry}{一块}{1,7}{⼀,⼟}
  \begin{Phonetics}{一块}{yi2kuai4}
    \definition{adv.}{(principalmente mandarim) juntos}
  \end{Phonetics}
\end{Entry}

\begin{Entry}{一块儿}{1,7,2}{⼀,⼟,⼉}
  \begin{Phonetics}{一块儿}{yi2 kuai4r5}[][HSK 1]
    \definition{adv.}{juntos; em conjunto}
    \definition{s.}{no mesmo lugar; no mesmo local}
  \end{Phonetics}
\end{Entry}

\begin{Entry}{一时}{1,7}{⼀,⽇}
  \begin{Phonetics}{一时}{yi4 shi2}[][HSK 6]
    \definition{adv.}{por um curto período; temporário | (usado em pares) agora\dots, agora\dots; este momento\dots, e o próximo\dots; o mesmo que 时而}
    \definition{s.}{um período de tempo | um momento; um breve momento; um tempo muito curto}
  \seealsoref{时而}{shi2'er2}
  \seealsoref{一时…,一时…}{yi4 shi2 yi4 shi2}
  \end{Phonetics}
\end{Entry}

\begin{Entry}{一时…,一时…}{1,7,1,7}{⼀,⽇,⼀,⽇}
  \begin{Phonetics}{一时…,一时…}{yi4 shi2 yi4 shi2}[][HSK 6]
    \definition{adv.}{por um tempo\dots, por um tempo\dots}
  \seealsoref{一时}{yi4 shi2}
  \end{Phonetics}
\end{Entry}

\begin{Entry}{一身}{1,7}{⼀,⾝}
  \begin{Phonetics}{一身}{yi4 shen1}[][HSK 5]
    \definition{s.}{o corpo inteiro; em todo o corpo | um terno; (um conjunto completo de) roupas | sozinho; uma única pessoa; relativo a uma única pessoa}
  \end{Phonetics}
\end{Entry}

\begin{Entry}{一些}{1,8}{⼀,⼆}
  \begin{Phonetics}{一些}{yi4 xie1}[][HSK 1]
    \definition{clas.}{alguns; um número de; quantidade indeterminada | um pouco; uma pequena quantidade | mais de um; mais de uma vez; indica mais de um ou mais de uma vez, etc. | uma ligeira mudança no grau, intensidade; usado após certos verbos, adjetivos, etc., para indicar uma quantidade muito pequena}
    \definition{pron.}{uns; alguns}
  \end{Phonetics}
\end{Entry}

\begin{Entry}{一定}{1,8}{⼀,⼧}
  \begin{Phonetics}{一定}{yi2ding4}[][HSK 2]
    \definition{adj.}{certo; particular; tendo um certo nível de especificidade; (objeto, situação) determinado em um ou mais | devido; certo; sempre foi assim, não vai mudar | fixo; especificado; há requisitos claros quanto à maneira, método, quantidade, etc.}
    \definition{adv.}{certamente; necessariamente; expressando determinação ou certeza | certamente; indica especulação ou avaliação de que um evento ou situação definitivamente acontecerá ou realmente existirá}
  \end{Phonetics}
\end{Entry}

\begin{Entry}{一直}{1,8}{⼀,⽬}
  \begin{Phonetics}{一直}{yi4zhi2}[][HSK 2]
    \definition{adv.}{direto; indica que permanece inalterado em uma direção | sempre; continuamente; o tempo todo; o tempo todo; indica que a ação é sempre ininterrupta ou o estado é sempre inalterado | de um ponto a outro sem enfatizar nenhuma exceção}
  \end{Phonetics}
\end{Entry}

\begin{Entry}{一贯}{1,8}{⼀,⾙}
  \begin{Phonetics}{一贯}{yi2guan4}[][HSK 6]
    \definition{adj./adv.}{do começo ao fim; inabalável; consistente; persistente; o tempo todo}
  \end{Phonetics}
\end{Entry}

\begin{Entry}{一带}{1,9}{⼀,⼱}
  \begin{Phonetics}{一带}{yi2 dai4}[][HSK 5]
    \definition{s.}{a área em torno de um determinado local; refere-se a um determinado local e suas proximidades}
  \end{Phonetics}
\end{Entry}

\begin{Entry}{一律}{1,9}{⼀,⼻}
  \begin{Phonetics}{一律}{yi2lv4}[][HSK 4]
    \definition{adj.}{igual; semelhante; uniforme; parecido; idêntico}
    \definition{adv.}{todos; tudo; sem exceção; enfatiza que todos devem ser assim, sem exceção, e é usado principalmente em regulamentos ou requisitos}
  \end{Phonetics}
\end{Entry}

\begin{Entry}{一战}{1,9}{⼀,⼽}
  \begin{Phonetics}{一战}{yi2zhan4}
    \definition*{s.}{Primeira Guerra Mundial}
  \end{Phonetics}
\end{Entry}

\begin{Entry}{一点儿}{1,9,2}{⼀,⽕,⼉}
  \begin{Phonetics}{一点儿}{yi4dian3r5}[][HSK 1]
    \definition{adv.}{um pouco; uma pitada; uma gota; uma amostra; uma pequena quantidade; ({adj.} + (一)点儿, 一点儿 + {s.} ou 有 + (一)点儿 + {s.})}
  \end{Phonetics}
\end{Entry}

\begin{Entry}{一点点}{1,9,9}{⼀,⽕,⽕}
  \begin{Phonetics}{一点点}{yi4 dian3 dian3}[][HSK 2]
    \definition{adj.}{um pouco; muito pouco ou um pouquinho}
  \end{Phonetics}
\end{Entry}

\begin{Entry}{一样}{1,10}{⼀,⽊}
  \begin{Phonetics}{一样}{yi2yang4}[][HSK 1]
    \definition{adj.}{o mesmo; igualmente; semelhante; tão\dots quanto\dots}
    \definition{part.}{na mesma medida; anexado a verbos ou palavras nominais, indica uma comparação ou semelhança, equivalente a 似的}
  \seealsoref{似的}{shi4de5}
  \end{Phonetics}
\end{Entry}

\begin{Entry}{一流}{1,10}{⼀,⽔}
  \begin{Phonetics}{一流}{yi4liu2}[][HSK 5]
    \definition{adj.}{clássico; de primeira linha; de primeira classe; o melhor}
    \definition[些]{s.}{tipo; mesmo tipo; da mesma classe; da mesma categoria; uma categoria}
  \end{Phonetics}
\end{Entry}

\begin{Entry}{一致}{1,10}{⼀,⾄}
  \begin{Phonetics}{一致}{yi2zhi4}[][HSK 4]
    \definition{adj.}{equado; idêntico; uniforme; unânime; nenhuma diferença (de opinião ou ação)}
    \definition{adv.}{juntos; em conjunto}
  \end{Phonetics}
\end{Entry}

\begin{Entry}{一般}{1,10}{⼀,⾈}
  \begin{Phonetics}{一般}{yi4ban1}[][HSK 2]
    \definition{adj.}{o mesmo que; exatamente como | geral; ordinário; comum | médio; medíocre; o grau ou nível não é muito alto}
    \definition{adv.}{frequentemente; geralmente}
  \end{Phonetics}
\end{Entry}

\begin{Entry}{一般来说}{1,10,7,9}{⼀,⾈,⽊,⾔}
  \begin{Phonetics}{一般来说}{yi4 ban1 lai2 shuo1}[][HSK 4]
    \definition{expr.}{de modo geral; na média; no caso usual; a declaração usual}
  \end{Phonetics}
\end{Entry}

\begin{Entry}{一起}{1,10}{⼀,⾛}
  \begin{Phonetics}{一起}{yi4qi3}[][HSK 1]
    \definition{adv.}{juntos; em companhia; indica o mesmo local, ao mesmo tempo que se faz algo | no total; em todos; no conjunto}
    \definition{s.}{no mesmo lugar}
  \end{Phonetics}
\end{Entry}

\begin{Entry}{一部分}{1,10,4}{⼀,⾢,⼑}
  \begin{Phonetics}{一部分}{yi2 bu4 fen4}[][HSK 2]
    \definition{adj.}{parcial}
    \definition{adv.}{parcialmente}
    \definition{num.}{parte; porção; seção; fração}
  \end{Phonetics}
\end{Entry}

\begin{Entry}{一…就…}{1,12}{⼀,⼪}
  \begin{Phonetics}{一…就…}{yi1 jiu4}
    \definition{expr.}{logo que |  uma vez que}
  \end{Phonetics}
\end{Entry}

\begin{Entry}{一番}{1,12}{⼀,⽥}
  \begin{Phonetics}{一番}{yi4 fan1}[][HSK 6]
    \definition{adv.}{uma demonstração de, uma dose de, um pedaço de (conversa, investigação, pensamento)}
  \end{Phonetics}
\end{Entry}

\begin{Entry}{一辈子}{1,12,3}{⼀,⾞,⼦}
  \begin{Phonetics}{一辈子}{yi2bei4zi5}[][HSK 5]
    \definition{s.}{uma vida inteira; vida inteira; toda a vida; durante toda a vida; enquanto se vive; todo o tempo entre o nascimento e a morte}
  \end{Phonetics}
\end{Entry}

\begin{Entry}{一道}{1,12}{⼀,⾡}
  \begin{Phonetics}{一道}{yi2 dao4}[][HSK 6]
    \definition{adv.}{juntos; lado a lado; junto com}
  \end{Phonetics}
\end{Entry}

\begin{Entry}{一路}{1,13}{⼀,⾜}
  \begin{Phonetics}{一路}{yi2 lu4}[][HSK 5]
    \definition{adv.}{o tempo todo; persistentemente; continuamente | juntos; sem parar; continuamente}
    \definition{s.}{o mesmo caminho; a mesma rota; ao longo de toda a viagem, ao longo do caminho | do mesmo tipo; da mesma categoria}
  \end{Phonetics}
\end{Entry}

\begin{Entry}{一路上}{1,13,3}{⼀,⾜,⼀}
  \begin{Phonetics}{一路上}{yi2 lu4 shang4}[][HSK 6]
    \definition{s.}{ao longo do caminho; todo o caminho}
  \end{Phonetics}
\end{Entry}

\begin{Entry}{一路平安}{1,13,5,6}{⼀,⾜,⼲,⼧}
  \begin{Phonetics}{一路平安}{yi2 lu4 ping2 an1}[][HSK 2]
    \definition{expr.}{Boa viagem!; Tenha uma boa viagem!}
    \definition{v.}{ter uma viagem agradável}
  \end{Phonetics}
\end{Entry}

\begin{Entry}{一路顺风}{1,13,9,4}{⼀,⾜,⾴,⾵}
  \begin{Phonetics}{一路顺风}{yi2 lu4 shun4 feng1}[][HSK 2]
    \definition{expr.}{ter uma viagem agradável; toda a viagem foi segura e tranquila; é uma metáfora para cada etapa do processo de lidar com algo que ocorre sem problemas | Tenha uma boa viagem!; Boa viagem!}
  \end{Phonetics}
\end{Entry}

\begin{Entry}{一模一样}{1,14,1,10}{⼀,⽊,⼀,⽊}
  \begin{Phonetics}{一模一样}{yi4 mu2 yi2 yang4}[][HSK 6]
    \definition{expr.}{tão parecidos quanto duas ervilhas; ser exatamente iguais; muito parecido, a mesma aparência}
  \end{Phonetics}
\end{Entry}

%%%%%%%%%% 乙 %%%%%%%%%%
\subsection*{乙}\addcontentsline{loh}{figure}{乙}

\begin{Entry}{乙}{1}{⼄}[Kangxi 5]
  \begin{Phonetics}{乙}{yi3}[][HSK 5]
    \definition*{s.}{Sobrenome: Yi}
    \definition{num.}{segundo}
    \definition{s.}{o segundo lugar do Tian Gan | uma nota da escala em Gongchepu (工尺谱); nível superior na música tradicional chinesa}
  \seealsoref{工尺谱}{gong1 che3 pu3}
  \end{Phonetics}
\end{Entry}

%%%%% EOF %%%%%


 %%%
%%% 2画
%%%
\section*{2画}\addcontentsline{toc}{section}{2画}\addcontentsline{loh}{figure}{\#\#\#\# 2画}

%%%%%%%%%% 丁 %%%%%%%%%%
\subsection*{丁}\addcontentsline{loh}{figure}{丁}

\begin{Entry}{丁}{2}{⼀}
  \begin{Phonetics}{丁}{ding1}[][HSK 7-9]
    \definition*{s.}{O quarto dos Dez Troncos Celestiais | Sobrenome: Ding}
    \definition{s.}{homem; homem adulto | população; membro de uma família | uma pessoa envolvida em uma determinada ocupação; pessoas em certas profissões | cubos; pequenos cubos de carne ou vegetais}
    \definition{v.}{encontrar; encontrar-se com; esbarrar em}
  \end{Phonetics}
  \begin{Phonetics}{丁}{zheng1}
    \definition{s.}{Onomatopéia: som agudo e metálico (como o de cortar madeira, jogar xadrez ou tocar instrumentos musicais)}
  \end{Phonetics}
\end{Entry}

%%%%%%%%%% 七 %%%%%%%%%%
\subsection*{七}\addcontentsline{loh}{figure}{七}

\begin{Entry}{七}{2}{⼀}
  \begin{Phonetics}{七}{qi1}[][HSK 1]
    \definition*{s.}{Sobrenome: Qi}
    \definition{num.}{sete; 7}
    \definition{s.}{antigamente, os mortos eram homenageados a cada sete dias, chamados de 七, até o quadragésimo nono dia, num total de sete 七}
  \end{Phonetics}
\end{Entry}

\begin{Entry}{七夕}{2,3}{⼀,⼣}
  \begin{Phonetics}{七夕}{qi1xi1}
    \definition*{s.}{Dia dos Namorados Chinês, quando o vaqueiro e a tecelã (牛郎织女) têm permissão para se reunirem anualmente | Festival das Meninas | Festival Duplo Sete, noite do sétimo mês lunar}
  \seealsoref{牛郎织女}{niu2 lang2 zhi1nv3}
  \end{Phonetics}
\end{Entry}

\begin{Entry}{七嘴八舌}{2,16,2,6}{⼀,⼝,⼋,⾆}
  \begin{Phonetics}{七嘴八舌}{qi1zui3-ba1she2}[][HSK 7-9]
    \definition{expr.}{``Uma cacofonia de vozes.''; todos falando ao mesmo tempo; falando uns por cima dos outros; isso descreve uma situação em que muitas pessoas estão falando ao mesmo tempo, com opiniões conflitantes; também descreve alguém que é falante e fofoqueiro}
  \synonymref{沸沸扬扬}{fei4fei4yang2yang2}
  \end{Phonetics}
\end{Entry}

%%%%%%%%%% 乃 %%%%%%%%%%
\subsection*{乃}\addcontentsline{loh}{figure}{乃}

\begin{Entry}{乃}{2}{⼃}
  \begin{Phonetics}{乃}{nai3}[][HSK 7-9]
    \definition{adv.}{então; portanto | somente então}
    \definition{pron.}{você; seu}
    \definition{v.}{ser | ser realmente; ser de fato}[失败乃成功之母。===O fracasso é a mãe do sucesso.]
  \end{Phonetics}
\end{Entry}

\begin{Entry}{乃至}{2,6}{⼃,⾄}
  \begin{Phonetics}{乃至}{nai3zhi4}[][HSK 7-9]
    \definition{conj.}{até mesmo; usado para enfatizar que algo excede o alcance ou a extensão esperada}
  \synonymref{甚至}{shen4zhi4}
  \synonymref{以及}{yi3ji2}
  \synonymref{以至}{yi3zhi4}
  \end{Phonetics}
\end{Entry}

%%%%%%%%%% 九 %%%%%%%%%%
\subsection*{九}\addcontentsline{loh}{figure}{九}

\begin{Entry}{九}{2}{⼄}
  \begin{Phonetics}{九}{jiu3}[][HSK 1]
    \definition*{s.}{Sobrenome: Jiu}
    \definition{adj.}{muitos; numerosos; indica várias vezes ou a maioria das vezes}
    \definition{num.}{nove; 9}
    \definition{s.}{cada um dos nove períodos de nove dias começando no dia seguinte ao solstício de inverno}
  \end{Phonetics}
\end{Entry}

%%%%%%%%%% 了 %%%%%%%%%%
\subsection*{了}\addcontentsline{loh}{figure}{了}

\begin{Entry}{了}{2}{⼅}
  \begin{Phonetics}{了}{le5}[][HSK 1]
    \definition{part.}{usada após verbos ou adjetivos para indicar a conclusão de uma ação, em um momento no passado ou antes do início de outra ação, ou uma ação esperada ou presumida | usada para indicar uma mudança de situação ou estado, seja real ou prevista | comandos ou solicitações em resposta a uma situação alterada; usada para xpressar urgência ou dissuadir | usada para indicar que algo chegou ao extremo; usada no final da frase ou em pausas no meio da frase, para expressar um tom de exclamação}
  \end{Phonetics}
  \begin{Phonetics}{了}{liao3}[][HSK 3]
    \definition*{s.}{Sobrenome: Liao}
    \definition{adv.}{inteiramente; um pouco; totalmente (mais usado em negativas)}
    \definition{v.}{terminar; concluir; encerrar; cumprir; eliminar; resolver | compreender; saber; perceber; saber claramente | expressar possibilidade ou impossibilidade; usado com 得 ou 不 após o verbo, indica possibilidade ou impossibilidade}
  \seealsoref{不}{bu4}
  \seealsoref{得}{de5}
  \end{Phonetics}
\end{Entry}

\begin{Entry}{了不起}{2,4,10}{⼅,⼀,⾛}
  \begin{Phonetics}{了不起}{liao3bu5qi3}[][HSK 4]
    \definition{adj.}{incrível; fantástico; extraordinário | sério; grave}
  \synonymref{不得了}{bu4 de2liao3}
  \end{Phonetics}
\end{Entry}

\begin{Entry}{了却}{2,7}{⼅,⼙}
  \begin{Phonetics}{了却}{liao3que4}[][HSK 7-9]
    \definition{v.}{resolver; solucionar; concluir}
  \antonymref{结束}{jie2shu4}
  \antonymref{完了}{wan2le5}
  \antonymref{完了}{wan2liao3}
  \end{Phonetics}
\end{Entry}

\begin{Entry}{了结}{2,9}{⼅,⽷}
  \begin{Phonetics}{了结}{liao3jie2}[][HSK 7-9]
    \definition{v.}{terminar; concluir; pôr fim a; finalizar; resolver}
  \synonymref{结束}{jie2shu4}
  \synonymref{了却}{liao3que4}
  \synonymref{完毕}{wan2bi4}
  \synonymref{完了}{wan2le5}
  \synonymref{完了}{wan2liao3}
  \end{Phonetics}
\end{Entry}

\begin{Entry}{了解}{2,13}{⼅,⾓}
  \begin{Phonetics}{了解}{liao3jie3}[][HSK 4]
    \definition{v.}{entender; compreender | investigar; indagar sobre}
  \synonymref{打听}{da3ting5}
  \synonymref{分析}{fen1xi1}
  \synonymref{理解}{li3jie3}
  \synonymref{明白}{ming2bai5}
  \synonymref{清晰}{qing1xi1}
  \synonymref{清楚}{qing1chu5}
  \synonymref{认识}{ren4shi5}
  \synonymref{熟悉}{shu2xi5}
  \synonymref{体会}{ti3hui4}
  \synonymref{知道}{zhi1dao5}
  \antonymref{陌生}{mo4sheng1}
  \antonymref{生疏}{sheng1shu1}
  \end{Phonetics}
\end{Entry}

%%%%%%%%%% 二 %%%%%%%%%%
\subsection*{二}\addcontentsline{loh}{figure}{二}

\begin{Entry}{二}{2}{⼆}[Kangxi 7]
  \begin{Phonetics}{二}{er4}[][HSK 1]
    \definition{adj.}{diferente; refere"-se a duas coisas ou coisas diferentes | bobo; pateta; tolo; sem inteligência | desleal; infiel; indiferente; sem determinação}
    \definition{num.}{dois; 2}
  \synonymref{两}{liang3}
  \end{Phonetics}
\end{Entry}

\begin{Entry}{二手}{2,4}{⼆,⼿}
  \begin{Phonetics}{二手}{er4shou3}[][HSK 4]
    \definition{adj.}{usado; de segunda mão; refere"-se especificamente a usados e revendidos}
  \antonymref{新鲜}{xin1xian1}
  \end{Phonetics}
\end{Entry}

\begin{Entry}{二手车}{2,4,4}{⼆,⼿,⾞}
  \begin{Phonetics}{二手车}{er4shou3che1}[][HSK 7-9]
    \definition{s.}{carro usado; carro de segunda mão}
  \end{Phonetics}
\end{Entry}

\begin{Entry}{二战}{2,9}{⼆,⼽}
  \begin{Phonetics}{二战}{er4zhan4}
    \definition*{s.}{Segunda Guerra Mundial (1039-1945)}
  \end{Phonetics}
\end{Entry}

\begin{Entry}{二胡}{2,9}{⼆,⾁}
  \begin{Phonetics}{二胡}{er4hu2}
    \definition{s.}{erhu; um instrumento de arco de duas cordas com um registro mais baixo que o 京胡; um tipo de 胡琴, a caixa de som é feita de bambu, madeira, etc., coberta com pele de cobra, etc., tem duas cordas e o tom é baixo e suave}
  \seealsoref{胡琴}{hu2qin2}
  \seealsoref{京胡}{jing1hu2}
  \synonymref{吉他}{ji2ta1}
  \end{Phonetics}
\end{Entry}

\begin{Entry}{二氧化碳}{2,10,4,14}{⼆,⽓,⼔,⽯}
  \begin{Phonetics}{二氧化碳}{er4yang3hua4tan4}[][HSK 7-9]
    \definition{s.}{CO$_2$; dióxido de carbono; gás carbônico}
  \end{Phonetics}
\end{Entry}

\begin{Entry}{二维码}{2,11,8}{⼆,⽷,⽯}
  \begin{Phonetics}{二维码}{er4wei2ma3}[][HSK 5]
    \definition[个]{s.}{\emph{QR code}; um identificador gráfico que distribui formas geométricas específicas em um plano ou direção bidimensional de acordo com certas regras para expressar um conjunto de informações}
  \end{Phonetics}
\end{Entry}

%%%%%%%%%% 人 %%%%%%%%%%
\subsection*{人}\addcontentsline{loh}{figure}{人}

\begin{Entry}{人}{2}{⼈}[Kangxi 9]
  \begin{Phonetics}{人}{ren2}[][HSK 1]
    \definition*{s.}{Sobrenome: Ren}
    \definition[个,名,位]{s.}{homem; pessoa; pessoas; ser humano | todos; cada um; todo mundo | adulto; crescido | uma pessoa envolvida em uma atividade específica | pessoas; outras pessoas | caráter; personalidade; qualidade, caráter ou reputação de uma pessoa | como alguém se sente; estado de saúde de alguém | mão de obra; força de trabalho}
  \end{Phonetics}
\end{Entry}

\begin{Entry}{人力}{2,2}{⼈,⼒}
  \begin{Phonetics}{人力}{ren2li4}[][HSK 5]
    \definition{s.}{mão de obra; trabalho manual; força de trabalho}
  \end{Phonetics}
\end{Entry}

\begin{Entry}{人力车}{2,2,4}{⼈,⼒,⾞}
  \begin{Phonetics}{人力车}{ren2li4che1}
    \definition{s.}{veículo de duas rodas puxado ou empurrado por um homem | Obsoleto: riquixá | uma carroça puxada ou empurrada por humanos}
  \antonymref{机动车}{ji1dong4che1}
  \antonymref{兽力车}{shou4li4che1}
  \end{Phonetics}
\end{Entry}

\begin{Entry}{人口}{2,3}{⼈,⼝}
  \begin{Phonetics}{人口}{ren2kou3}[][HSK 2]
    \definition[个,群]{s.}{população; o número total de pessoas que vivem em uma determinada região durante um determinado período de tempo | número de membros da família; o número total de pessoas em uma família | pessoas; público; população; referência geral a pessoas | rumores do povo; referindo"-se à opinião pública}
  \end{Phonetics}
\end{Entry}

\begin{Entry}{人士}{2,3}{⼈,⼠}
  \begin{Phonetics}{人士}{ren2shi4}[][HSK 5]
    \definition{s.}{pessoa; figura; personalidade; figura pública; pessoas com certa influência social}
  \end{Phonetics}
\end{Entry}

\begin{Entry}{人工}{2,3}{⼈,⼯}
  \begin{Phonetics}{人工}{ren2gong1}[][HSK 3]
    \definition{adj.}{feito pelo homem; artificial}
    \definition[个]{s.}{trabalho manual; trabalho feito à mão | mão de obra; homem-dia; uma unidade de cálculo da quantidade de trabalho realizado}
  \antonymref{天然}{tian1ran2}
  \end{Phonetics}
\end{Entry}

\begin{Entry}{人工智能}{2,3,12,10}{⼈,⼯,⽇,⾁}
  \begin{Phonetics}{人工智能}{ren2gong1-zhi4neng2}[][HSK 7-9]
    \definition*{s.}{Inteligência Artificial (IA)}
  \end{Phonetics}
\end{Entry}

\begin{Entry}{人才}{2,3}{⼈,⼿}
  \begin{Phonetics}{人才}{ren2cai2}[][HSK 3]
    \definition{adj.}{aparência bonita, elegante}
    \definition[个,些,位]{s.}{talento; pessoal qualificado; pessoa com capacidade; uma pessoa com capacidade e integridade política; uma pessoa com talentos especiais | aparência bonita; refere"-se à aparência; especialmente à aparência bonita}
  \end{Phonetics}
\end{Entry}

\begin{Entry}{人为}{2,4}{⼈,⼂}
  \begin{Phonetics}{人为}{ren2wei2}[][HSK 7-9]
    \definition{adj.}{artificial; feito pelo homem; causado por pessoas (usado para descrever coisas desagradáveis)}
    \definition{v.}{fazer pelo homem; fazer esforço humano; fazer isso com força humana}
  \end{Phonetics}
\end{Entry}

\begin{Entry}{人手}{2,4}{⼈,⼿}
  \begin{Phonetics}{人手}{ren2shou3}[][HSK 7-9]
    \definition{s.}{mão de obra; mão; pessoas que fazem coisas}
  \end{Phonetics}
\end{Entry}

\begin{Entry}{人文}{2,4}{⼈,⽂}
  \begin{Phonetics}{人文}{ren2wen2}[][HSK 7-9]
    \definition{s.}{humanidades; atividades culturais na sociedade humana; originalmente referindo"-se à poesia, livros, ritos e música, posteriormente passou a se referir a vários fenômenos culturais na sociedade humana}
  \end{Phonetics}
\end{Entry}

\begin{Entry}{人气}{2,4}{⼈,⽓}
  \begin{Phonetics}{人气}{ren2qi4}[][HSK 7-9]
    \definition[股,点,些]{s.}{fama; humor; popularidade; sentimento público/aclamação/apoio/confiança/entusiasmo; o grau em que uma pessoa ou coisa é popular | atmosfera animada | qualidade e estilo excelentes; boa personalidade; refere"-se ao caráter de uma pessoa}
  \end{Phonetics}
\end{Entry}

\begin{Entry}{人们}{2,5}{⼈,⼈}
  \begin{Phonetics}{人们}{ren2men5}[][HSK 2]
    \definition{s.}{homens; pessoas; o público; referindo"-se a muitas pessoas; todos}
  \end{Phonetics}
\end{Entry}

\begin{Entry}{人民}{2,5}{⼈,⽒}
  \begin{Phonetics}{人民}{ren2min2}[][HSK 3]
    \definition[群,批,个,国]{s.}{o povo; refere"-se a um certo tipo de pessoas; membros básicos da sociedade com as massas trabalhadoras como o corpo principal}
  \end{Phonetics}
\end{Entry}

\begin{Entry}{人民币}{2,5,4}{⼈,⽒,⼱}
  \begin{Phonetics}{人民币}{ren2min2bi4}[][HSK 3]
    \definition*[块,张,元]{s.}{Renminbi (RMB); Yuan Chinês (CYN); nome da moeda chinesa}
  \end{Phonetics}
\end{Entry}

\begin{Entry}{人生}{2,5}{⼈,⽣}
  \begin{Phonetics}{人生}{ren2sheng1}[][HSK 3]
    \definition{s.}{vida; sobrevivência e vida humana}
  \end{Phonetics}
\end{Entry}

\begin{Entry}{人权}{2,6}{⼈,⽊}
  \begin{Phonetics}{人权}{ren2quan2}[][HSK 6]
    \definition{s.}{direitos humanos}[最基本的人权是生存权。===O direito humano mais básico é o direito à vida.]
  \seealsoref{人权法}{ren2quan2fa3}
  \end{Phonetics}
\end{Entry}

\begin{Entry}{人权法}{2,6,8}{⼈,⽊,⽔}
  \begin{Phonetics}{人权法}{ren2quan2fa3}
    \definition*{s.}{Direitos Humanos}
  \seealsoref{人权}{ren2quan2}
  \end{Phonetics}
\end{Entry}

\begin{Entry}{人次}{2,6}{⼈,⽋}
  \begin{Phonetics}{人次}{ren2ci4}[][HSK 7-9]
    \definition{clas.}{visitantes; utilizado para o número de total participantes em várias visitas}[参观展览的总共二十万人次。===A exposição atraiu um total de 200.000 visitantes.]
  \end{Phonetics}
\end{Entry}

\begin{Entry}{人行道}{2,6,12}{⼈,⾏,⾡}
  \begin{Phonetics}{人行道}{ren2xing2dao4}[][HSK 7-9]
    \definition{s.}{calçada; as calçadas em ambos os lados da rua são exclusivas para pedestres}
  \end{Phonetics}
\end{Entry}

\begin{Entry}{人体}{2,7}{⼈,⼈}
  \begin{Phonetics}{人体}{ren2ti3}[][HSK 7-9]
    \definition{s.}{corpo humano}
  \end{Phonetics}
\end{Entry}

\begin{Entry}{人员}{2,7}{⼈,⼝}
  \begin{Phonetics}{人员}{ren2yuan2}[][HSK 3]
    \definition[个,位,名]{s.}{funcionários ; uma pessoa que ocupa uma determinada posição | pessoal; membros de um grupo}
  \end{Phonetics}
\end{Entry}

\begin{Entry}{人均}{2,7}{⼈,⼟}
  \begin{Phonetics}{人均}{ren2jun1}[][HSK 7-9]
    \definition{adj.}{per capita (ou por pessoa, cabeça)}[人均收入今年有所增长。===A renda per capita aumentou este ano.]
  \end{Phonetics}
\end{Entry}

\begin{Entry}{人材}{2,7}{⼈,⽊}
  \begin{Phonetics}{人材}{ren2cai2}
    \variantof{人才}
  \end{Phonetics}
\end{Entry}

\begin{Entry}{人身}{2,7}{⼈,⾝}
  \begin{Phonetics}{人身}{ren2shen1}[][HSK 7-9]
    \definition[个]{s.}{corpo vivo de um ser humano; pessoa | corpo humano | pessoal}
  \end{Phonetics}
\end{Entry}

\begin{Entry}{人间}{2,7}{⼈,⾨}
  \begin{Phonetics}{人间}{ren2jian1}[][HSK 5]
    \definition{s.}{o mundo humano; o Mundo; a Terra}
  \end{Phonetics}
\end{Entry}

\begin{Entry}{人事}{2,8}{⼈,⼅}
  \begin{Phonetics}{人事}{ren2shi4}[][HSK 7-9]
    \definition{s.}{assuntos humanos; acontecimentos na vida humana; separação, reencontro, circunstâncias, sobrevivência e morte dos seres humanos | assuntos pessoais; questões relativas a alterações de pessoal dentro de uma unidade, como recrutamento, demissão, promoção, rebaixamento, recompensas e punições, treinamento e transferências | modos de vida; relações humanas e princípios | consciência do mundo exterior; o objeto da consciência humana | o que é humanamente possível; o que os humanos podem fazer | relações de recursos humanos; refere"-se às relações entre pessoas | pessoal}
  \end{Phonetics}
\end{Entry}

\begin{Entry}{人性}{2,8}{⼈,⼼}
  \begin{Phonetics}{人性}{ren2xing4}[][HSK 7-9]
    \definition{s.}{humanidade; natureza humana; as emoções e a razão normais que os seres humanos possuem}
  \end{Phonetics}
\end{Entry}

\begin{Entry}{人物}{2,8}{⼈,⽜}
  \begin{Phonetics}{人物}{ren2wu4}[][HSK 5]
    \definition[个,位,名]{s.}{personagem; personagens criados em obras literárias e artísticas | figura; personalidade; homem influente; refere"-se a pessoas com grande talento e status; também se refere a pessoas com certas características ou que são representativas em algum aspecto | pintura figurativa; um tipo de pintura tradicional chinesa com personagens como tema}
  \end{Phonetics}
\end{Entry}

\begin{Entry}{人质}{2,8}{⼈,⾙}
  \begin{Phonetics}{人质}{ren2zhi4}[][HSK 7-9]
    \definition[个,名]{s.}{refém; uma das partes detém os pertences da outra parte para obrigá-la a cumprir uma promessa ou aceitar uma condição}
  \end{Phonetics}
\end{Entry}

\begin{Entry}{人鱼}{2,8}{⼈,⿂}
  \begin{Phonetics}{人鱼}{ren2yu2}
    \definition{s.}{sereia | peixe-boi | salamandra gigante}
  \end{Phonetics}
\end{Entry}

\begin{Entry}{人品}{2,9}{⼈,⼝}
  \begin{Phonetics}{人品}{ren2pin3}[][HSK 7-9]
    \definition{s.}{caráter; força moral; integridade | Coloquial: aparência; porte; atitude}
  \end{Phonetics}
\end{Entry}

\begin{Entry}{人类}{2,9}{⼈,⽶}
  \begin{Phonetics}{人类}{ren2lei4}[][HSK 3]
    \definition[种]{s.}{humano; humanidade; raça humana; um termo geral para pessoas}
  \end{Phonetics}
\end{Entry}

\begin{Entry}{人选}{2,9}{⼈,⾡}
  \begin{Phonetics}{人选}{ren2xuan3}[][HSK 7-9]
    \definition{s.}{candidato; pessoa selecionada de acordo com determinados critérios}
  \end{Phonetics}
\end{Entry}

\begin{Entry}{人家}{2,10}{⼈,⼧}
  \begin{Phonetics}{人家}{ren2jia1}
    \definition[户,个]{s.}{lar; família; família do noivo; casa do futuro marido}
  \end{Phonetics}
  \begin{Phonetics}{人家}{ren2jia5}[][HSK 4]
    \definition{pron.}{outros; uma pessoa ou pessoas diferentes do falante ou ouvinte; refere"-se a alguém diferente de si mesmo ou de outra pessoa | certa pessoa ou pessoas (a pessoa ou pessoas mencionadas em um contexto próximo, aproximadamente equivalente ao pronome de terceira pessoa);  refere"-se a uma pessoa ou algumas pessoas, com significado semelhante a 他 | eu; mim (usado retoricamente no lugar do primeiro pronome pessoal, muitas vezes expressando descontentamento de forma jocosa; geralmente usado quando se fala com pessoas próximas, para significar 自己, usado principamente por meninas)}
  \seealsoref{他}{ta1}
  \seealsoref{自己}{zi4ji3}
  \end{Phonetics}
\end{Entry}

\begin{Entry}{人格}{2,10}{⼈,⽊}
  \begin{Phonetics}{人格}{ren2ge2}[][HSK 7-9]
    \definition{s.}{caráter; personalidade; individualidade; a soma do caráter, temperamento, habilidades e outras características de uma pessoa | qualidade moral; caráter moral pessoal | entidade jurídica; dignidade humana; as qualificações de uma pessoa para agir como sujeito de direitos e obrigações}
  \end{Phonetics}
\end{Entry}

\begin{Entry}{人海}{2,10}{⼈,⽔}
  \begin{Phonetics}{人海}{ren2hai3}
    \definition{s.}{uma multidão | um mar de pessoas}
  \end{Phonetics}
\end{Entry}

\begin{Entry}{人造}{2,10}{⼈,⾡}
  \begin{Phonetics}{人造}{ren2zao4}[][HSK 7-9]
    \definition{adj.}{feito pelo homem; artificial; imitação | sintético}
  \antonymref{天然}{tian1ran2}
  \end{Phonetics}
\end{Entry}

\begin{Entry}{人情}{2,11}{⼈,⼼}
  \begin{Phonetics}{人情}{ren2qing2}[][HSK 7-9]
    \definition{s.}{sensibilidades; compaixão humana; emoções humanas; as emoções que as pessoas deveriam ter em circunstâncias normais | sentimentos; sensibilidades; relacionamento | etiqueta; costume; cortesia e costumes nas interações interpessoais | bondade; favor | presente; dádiva; um presente oferecido para expressar um determinado sentimento}
  \end{Phonetics}
\end{Entry}

\begin{Entry}{人缘儿}{2,12,2}{⼈,⽷,⼉}
  \begin{Phonetics}{人缘儿}{ren2yuan2r5}[][HSK 7-9]
    \definition{s.}{relações com as pessoas; popularidade}
  \end{Phonetics}
\end{Entry}

\begin{Entry}{人道}{2,12}{⼈,⾡}
  \begin{Phonetics}{人道}{ren2dao4}[][HSK 7-9]
    \definition{s.}{solidariedade humana; humanitarismo | humano | Budismo: ``a maneira humana'', um dos estágios do ciclo de reencarnação | relação sexual}
  \end{Phonetics}
\end{Entry}

\begin{Entry}{人像}{2,13}{⼈,⼈}
  \begin{Phonetics}{人像}{ren2xiang4}
    \definition{s.}{``retrato'' de uma pessoa (esboço, foto, escultura, etc.)}
  \end{Phonetics}
\end{Entry}

\begin{Entry}{人数}{2,13}{⼈,⽁}
  \begin{Phonetics}{人数}{ren2shu4}[][HSK 2]
    \definition{s.}{número de pessoas; significa o número total de pessoas, uma quantidade de pessoas; normalmente, usa-se números para fazer estatísticas específicas, mas às vezes também se usa um intervalo aproximado para fazer estimativas}
  \end{Phonetics}
\end{Entry}

\begin{Entry}{人群}{2,13}{⼈,⽺}
  \begin{Phonetics}{人群}{ren2qun2}[][HSK 3]
    \definition[个,类]{s.}{multidão; ajuntamento; torpel; aglomeração; um grupo de pessoas}
  \end{Phonetics}
\end{Entry}

%%%%%%%%%% 儿 %%%%%%%%%%
\subsection*{儿}\addcontentsline{loh}{figure}{儿}

\begin{Entry}{儿}{2}{⼉}[Kangxi 10]
  \begin{Phonetics}{儿}{er2}
    \definition{adj.}{macho}
    \definition{s.}{criança | jovem; juventude | filho}
    \definition{suf.}{adicionado a substantivos para expressar pequenez  | adicionado a verbos, adjetivos e classificadores para formar substantivos | adicionado a substantivos para formar substantivos com significados diferentes | sufixos de alguns verbos | anexado após adjetivos duplicados}
  \end{Phonetics}
  \begin{Phonetics}{儿}{r5}
    \definition{suf.}{sufixo diminutivo não silábico | final retroflexo, pronunciado como ``r'' | adicionado a substantivos para expressar pequenez  | adicionado a verbos, adjetivos e classificadores para formar substantivos | adicionado a substantivos para formar substantivos com significados diferentes | sufixos de alguns verbos | anexado após adjetivos duplicados}
  \end{Phonetics}
\end{Entry}

\begin{Entry}{儿女}{2,3}{⼉,⼥}
  \begin{Phonetics}{儿女}{er2nv3}[][HSK 5]
    \definition{s.}{crianças; filhos e filhas | homem e mulher jovens (apaixonados)}
  \end{Phonetics}
\end{Entry}

\begin{Entry}{儿子}{2,3}{⼉,⼦}
  \begin{Phonetics}{儿子}{er2zi5}[][HSK 1]
    \definition[个]{s.}{filho}
  \seealsoref{女儿}{nv3'er2}
  \end{Phonetics}
\end{Entry}

\begin{Entry}{儿科}{2,9}{⼉,⽲}
  \begin{Phonetics}{儿科}{er2ke1}[][HSK 6]
    \definition{s.}{(departamento de) pediatria | pediatria; o ramo da medicina que trata do desenvolvimento, cuidado e doença das crianças}
  \end{Phonetics}
\end{Entry}

\begin{Entry}{儿童}{2,12}{⼉,⽴}
  \begin{Phonetics}{儿童}{er2tong2}[][HSK 4]
    \definition[个,群]{s.}{criança; menor de idade (mais jovem do que 少年)}
  \seealsoref{少年}{shao4nian2}
  \end{Phonetics}
\end{Entry}

\begin{Entry}{儿媳}{2,13}{⼉,⼥}
  \begin{Phonetics}{儿媳}{er2xi2}
    \definition{s.}{esposa do filho}
  \end{Phonetics}
\end{Entry}

%%%%%%%%%% 入 %%%%%%%%%%
\subsection*{入}\addcontentsline{loh}{figure}{入}

\begin{Entry}{入}{2}{⼊}[Kangxi 11]
  \begin{Phonetics}{入}{ru4}[][HSK 6]
    \definition{s.}{renda | tom de entrada}
    \definition{v.}{entrar; entrar | juntar-se; ser admitido em; tornar-se membro de | conformar-se com; concordar com | alcançar; atingir; entrar em (um certo nível ou estado) | fazer entrar; fazer algo entrar; fazer entrada}
  \antonymref{出}{chu1}
  \end{Phonetics}
\end{Entry}

\begin{Entry}{入乡随俗}{2,3,11,9}{⼊,⼄,⾩,⼈}
  \begin{Phonetics}{入乡随俗}{ru4xiang1-sui2su2}
    \definition{expr.}{``Em roma, faça como os romanos!''}
  \end{Phonetics}
\end{Entry}

\begin{Entry}{入口}{2,3}{⼊,⼝}
  \begin{Phonetics}{入口}{ru4/kou3}[][HSK 2]
    \definition[个]{s.}{entrada; entrada em locais, edifícios, estradas, etc., através de portões ou portas}
    \definition{v.+compl.}{entrar na boca | importar; mercadorias estrangeiras importadas, às vezes também se refere a mercadorias de outras regiões importadas para esta região}
  \end{Phonetics}
\end{Entry}

\begin{Entry}{入门}{2,3}{⼊,⾨}
  \begin{Phonetics}{入门}{ru4/men2}[][HSK 5]
    \definition{s.}{(geralmente em títulos de livros) curso básico; manual introdutório | ABC; guia; refere"-se a leituras básicas; conhecimentos básicos}
    \definition{v.+compl.}{ultrapassar o limiar; aprender os rudimentos de um assunto | aprender o ABC de; ser introduzido a um assunto; aprender o básico}
  \end{Phonetics}
\end{Entry}

\begin{Entry}{入手}{2,4}{⼊,⼿}
  \begin{Phonetics}{入手}{ru4shou3}
    \definition{v.}{começar com; proceder a partir de; tomar como ponto de partida | obter; apoderar-se  | começar; para dar início}
  \antonymref{出手}{chu1/shou3}
  \end{Phonetics}
\end{Entry}

\begin{Entry}{入场}{2,6}{⼊,⼟}
  \begin{Phonetics}{入场}{ru4/chang3}[][HSK 7-9]
    \definition{v.+compl.}{entrar; ser admitido; entrar no local}
  \end{Phonetics}
\end{Entry}

\begin{Entry}{入场券}{2,6,8}{⼊,⼟,⼑}
  \begin{Phonetics}{入场券}{ru4chang3quan4}[][HSK 7-9]
    \definition{s.}{bilhete (de entrada) | pré-requisito para atingir um objetivo; qualificação para entrar em uma partida | ingresso; admissões}
  \end{Phonetics}
\end{Entry}

\begin{Entry}{入学}{2,8}{⼊,⼦}
  \begin{Phonetics}{入学}{ru4/xue2}[][HSK 6]
    \definition{v.+compl.}{(uma criança) começar a escola; começar a escola primária | entrar em uma escola; matricular-se em uma escola}
  \end{Phonetics}
\end{Entry}

\begin{Entry}{入侵}{2,9}{⼊,⼈}
  \begin{Phonetics}{入侵}{ru4qin1}[][HSK 7-9]
    \definition{v.}{invadir; intrometer-se; fazer uma incursão; abrir caminho}
  \end{Phonetics}
\end{Entry}

\begin{Entry}{入选}{2,9}{⼊,⾡}
  \begin{Phonetics}{入选}{ru4xuan3}[][HSK 7-9]
    \definition{v.}{ser escolhido; ser selecionado}
  \end{Phonetics}
\end{Entry}

\begin{Entry}{入党}{2,10}{⼊,⼉}
  \begin{Phonetics}{入党}{ru4dang3}
    \definition{v.}{ingressar em um partido político (especialmente o partido comunista)}
  \end{Phonetics}
\end{Entry}

\begin{Entry}{入境}{2,14}{⼊,⼟}
  \begin{Phonetics}{入境}{ru4/jing4}[][HSK 7-9]
    \definition{v.+compl.}{entrar em um país; imigrar}
  \end{Phonetics}
\end{Entry}

%%%%%%%%%% 八 %%%%%%%%%%
\subsection*{八}\addcontentsline{loh}{figure}{八}

\begin{Entry}{八}{2}{⼋}[Kangxi 12]
  \begin{Phonetics}{八}{ba1}[][HSK 1]
    \definition{num.}{oito; 8}
  \end{Phonetics}
\end{Entry}

\begin{Entry}{八八六}{2,2,4}{⼋,⼋,⼋}
  \begin{Phonetics}{八八六}{ba1 ba1 liu4}
    \definition{expr.}{\emph{``Bye bye!''}, em salas de bate-papo e mensagens de texto}
  \end{Phonetics}
\end{Entry}

\begin{Entry}{八卦}{2,8}{⼋,⼘}
  \begin{Phonetics}{八卦}{ba1gua4}[][HSK 7-9]
    \definition*{s.}{Oito Trigramas; na China antiga, havia um conjunto de símbolos, os oito trigramas, que são chamados de Bagua: \TrigramHeaven\  (céu, paraíso), \TrigramLake\ (lago), \TrigramFire\ (fogo), \TrigramThunder\ (trovão), \TrigramWind\ (vento), \TrigramWater\ (água), \TrigramMountain\ (montanha) e \TrigramEarth\ (terra) onde \Yang\ representa Yang (阳) e \Yin\ representa Yin (阴); mais tarde, foi usado para prever sucesso ou fracasso, sorte ou infortúnio, etc.}
    \definition[个,条]{adj.}{fofoca}
    \definition{adj.}{fofoqueiro}
  \seealsoref{阳}{yang2}
  \seealsoref{阴}{yin1}
  \end{Phonetics}
\end{Entry}

%%%%%%%%%% 几 %%%%%%%%%%
\subsection*{几}\addcontentsline{loh}{figure}{几}

\begin{Entry}{几}{2}{⼏}[Kangxi 16]
  \begin{Phonetics}{几}{ji1}
    \definition{adv.}{quase; praticamente}
    \definition{s.}{uma mesa pequena}
  \end{Phonetics}
  \begin{Phonetics}{几}{ji3}[][HSK 1]
    \definition{adv.}{quanto?, usado para perguntar sobre quantidade e tempo}
    \definition{num.}{alguns; vários; poucos; indica um número indeterminado maior que um e menor que dez}
  \end{Phonetics}
\end{Entry}

\begin{Entry}{几乎}{2,5}{⼏,⼃}
  \begin{Phonetics}{几乎}{ji1hu1}[][HSK 4]
    \definition{adv.}{quase; praticamente; próximo a | perto de; quase; à beira de}
  \end{Phonetics}
\end{Entry}

\begin{Entry}{几何}{2,7}{⼏,⼈}
  \begin{Phonetics}{几何}{ji3he2}
    \definition{s.}{geometria}
  \end{Phonetics}
\end{Entry}

\begin{Entry}{几率}{2,11}{⼏,⽞}
  \begin{Phonetics}{几率}{ji1lv4}[][HSK 7-9]
    \definition{s.}{probabilidade; um evento pode ou não ocorrer sob as mesmas condições, a grandeza que indica a possibilidade de ocorrência é chamada de probabilidade}
  \end{Phonetics}
\end{Entry}

%%%%%%%%%% 刀 %%%%%%%%%%
\subsection*{刀}\addcontentsline{loh}{figure}{刀}

\begin{Entry}{刀}{2}{⼑}[Kangxi 18]
  \begin{Phonetics}{刀}{dao1}[][HSK 3]
    \definition*{s.}{Sobrenome: Dao}
    \definition{clas.}{unidade de medida para papel, geralmente cem folhas por pacote}
    \definition[把,口]{s.}{faca; espada; armas antigas, referindo"-se a ferramentas para cortar, retalhar, raspar, golpear e fatiar, geralmente feitas de ferro e aço | ferramenta; ferramenta de corte; lâminas para tornos; fresas (ferramentas; ferramentas de ferro para máquinas) | algo com a forma de uma faca}
  \end{Phonetics}
\end{Entry}

%%%%%%%%%% 刁 %%%%%%%%%%
\subsection*{刁}\addcontentsline{loh}{figure}{刁}

\begin{Entry}{刁}{2}{⼑}
  \begin{Phonetics}{刁}{diao1}
    \definition*{s.}{Sobrenome: Diao}
    \definition{adj.}{traiçoeiro; astuto | exigente; exigente com a comida; difícil}
    \definition{v.}{dificultar as coisas}
  \end{Phonetics}
\end{Entry}

\begin{Entry}{刁难}{2,10}{⼑,⾫}
  \begin{Phonetics}{刁难}{diao1nan4}[][HSK 7-9]
    \definition{v.}{criar dificuldades; tornar as coisas difíceis; dificultar deliberadamente as coisas para os outros}
  \end{Phonetics}
\end{Entry}

%%%%%%%%%% 力 %%%%%%%%%%
\subsection*{力}\addcontentsline{loh}{figure}{力}

\begin{Entry}{力}{2}{⼒}[Kangxi 19]
  \begin{Phonetics}{力}{li4}[][HSK 3,6]
    \definition*{s.}{Sobrenome: Li}
    \definition{adj.}{forte; eficiente; capaz | forte; poderoso; referência geral à função das coisas}
    \definition{adv.}{energicamente; arduamente; vigorosamente; com todo o esforço; com toda a dedicação}
    \definition{s.}{força; energia; poder; Física: refere"-se à ação de alterar o estado de movimento ou a forma de um objeto | poder; força; habilidade; capacidade; funções dos órgãos do corpo humano | força física; resistência física}
  \end{Phonetics}
\end{Entry}

\begin{Entry}{力不从心}{2,4,4,4}{⼒,⼀,⼈,⼼}
  \begin{Phonetics}{力不从心}{li4bu4cong2xin1}[][HSK 7-9]
    \definition{expr.}{desamparado; cuja capacidade fica aquém dos seus desejos; incapaz de fazer tanto quanto gostaria}
  \end{Phonetics}
\end{Entry}

\begin{Entry}{力气}{2,4}{⼒,⽓}
  \begin{Phonetics}{力气}{li4qi5}[][HSK 4]
    \definition[把]{s.}{força física; eficiência muscular; força | esforço}
  \end{Phonetics}
\end{Entry}

\begin{Entry}{力争}{2,6}{⼒,⼑}
  \begin{Phonetics}{力争}{li4zheng1}[][HSK 7-9]
    \definition{v.}{trabalhar arduamente para; fazer tudo o que estiver ao seu alcance para; fazer todo o possível para atingir um objetivo específico | argumentar veementemente; contestar vigorosamente; usar todas as suas forças para argumentar e debater}
  \end{Phonetics}
\end{Entry}

\begin{Entry}{力求}{2,7}{⼒,⽔}
  \begin{Phonetics}{力求}{li4qiu2}[][HSK 7-9]
    \definition{v.}{fazer o melhor possível para; buscar com todas as suas forças; esforçar-se para obter}
  \end{Phonetics}
\end{Entry}

\begin{Entry}{力所能及}{2,8,10,3}{⼒,⼾,⾁,⼃}
  \begin{Phonetics}{力所能及}{li4suo3neng2ji2}[][HSK 7-9]
    \definition{expr.}{da melhor forma possível; dentro das capacidades de alguém; refere"-se ao que as próprias habilidades e força podem realizar}
  \end{Phonetics}
\end{Entry}

\begin{Entry}{力度}{2,9}{⼒,⼴}
  \begin{Phonetics}{力度}{li4du4}[][HSK 7-9]
    \definition{s.}{força | intensidade; profundidade; potência | dinâmica (musical) | esforços | vigor}
  \end{Phonetics}
\end{Entry}

\begin{Entry}{力量}{2,12}{⼒,⾥}
  \begin{Phonetics}{力量}{li4liang5}[][HSK 3]
    \definition[出]{s.}{força física; força espiritual | habilidade; capacidade | eficácia; efeito | força (pessoa ou grupo que tem muito poder ou influência); referência a uma pessoa ou grupo que pode desempenhar um papel importante}
  \end{Phonetics}
\end{Entry}

%%%%%%%%%% 十 %%%%%%%%%%
\subsection*{十}\addcontentsline{loh}{figure}{十}

\begin{Entry}{十}{2}{⼗}[Kangxi 24]
  \begin{Phonetics}{十}{shi2}[][HSK 1]
    \definition*{s.}{Sobrenome: Shi}
    \definition{num.}{dez; 10 | dezena | completo; no topo; máximo; referindo"-se a algo que atingiu o ápice da perfeição ou plenitude | um monte de; indica que há muitos}
  \end{Phonetics}
\end{Entry}

\begin{Entry}{十分}{2,4}{⼗,⼑}
  \begin{Phonetics}{十分}{shi2fen1}[][HSK 2]
    \definition{adv.}{muito; totalmente; completamente; extremamente; indica um nível muito alto}
  \end{Phonetics}
\end{Entry}

\begin{Entry}{十字路口}{2,6,13,3}{⼗,⼦,⾜,⼝}
  \begin{Phonetics}{十字路口}{shi2zi4 lu4kou3}[][HSK 7-9]
    \definition{expr.}{encruzilhada; cruzamento; interseção; ponto de virada; uma encruzilhada, um lugar onde duas estradas se cruzam, é uma metáfora para uma situação em que se deve escolher um caminho a seguir em uma questão crucial}
  \end{Phonetics}
\end{Entry}

\begin{Entry}{十足}{2,7}{⼗,⾜}
  \begin{Phonetics}{十足}{shi2zu2}[][HSK 5]
    \definition{adj.}{puro e simples; apenas este componente ou esta característica é muito evidente | 100\%; completo; total; muito satisfatório; muito adequado}
  \end{Phonetics}
\end{Entry}

%%%%%%%%%% 厂 %%%%%%%%%%
\subsection*{厂}\addcontentsline{loh}{figure}{厂}

\begin{Entry}{厂}{2}{⼚}[Kangxi 27]
  \begin{Phonetics}{厂}{an1}
    \definition{s.}{usado principalmente em nomes pessoais}[他名中有个厂字。===O nome dele contém a palavra `An'.]
  \end{Phonetics}
  \begin{Phonetics}{厂}{chang3}[][HSK 3]
    \definition[家,间]{s.}{fábrica; moinho; planta; obra | pátio; depósito; refere"-se a um estabelecimento comercial com um amplo espaço para armazenamento de mercadorias e processamento}
  \end{Phonetics}
  \begin{Phonetics}{厂}{han3}
    \definition[家,间]{s.}{radical ``penhasco'' em caracteres chineses}
  \end{Phonetics}
\end{Entry}

\begin{Entry}{厂长}{2,4}{⼚,⾧}
  \begin{Phonetics}{厂长}{chang3zhang3}[][HSK 5]
    \definition[位,个,名]{s.}{diretor de fábrica; gerente de fábrica; líder responsável pela produção, pela vida e por todos os outros assuntos de toda a fábrica}
  \end{Phonetics}
\end{Entry}

\begin{Entry}{厂家}{2,10}{⼚,⼧}
  \begin{Phonetics}{厂家}{chang3jia1}[][HSK 7-9]
    \definition[个]{s.}{fábrica; refere"-se aos aspectos de fábrica ou fábrica}
  \end{Phonetics}
\end{Entry}

\begin{Entry}{厂商}{2,11}{⼚,⼝}
  \begin{Phonetics}{厂商}{chang3shang1}[][HSK 6]
    \definition[家,个]{s.}{empresa; fornecedor; fábrica; negócio; fabricante; uma unidade que produz e vende produtos; uma pessoa que administra uma fábrica}
  \end{Phonetics}
\end{Entry}

%%%%%%%%%% 又 %%%%%%%%%%
\subsection*{又}\addcontentsline{loh}{figure}{又}

\begin{Entry}{又}{2}{⼜}[Kangxi 29]
  \begin{Phonetics}{又}{you4}[][HSK 2]
    \definition{adv.}{indica repetição ou continuação | indica que várias situações ou propriedades existem simultaneamente | indica um nível mais profundo de significado | indica adicionar zero a números inteiros | indica duas coisas contraditórias | indica um ponto de virada, significando 可是 | usado em frases negativas ou perguntas retóricas para fortalecer o tom | além disso; indica informações adicionais ou suplementares}
  \seealsoref{可是}{ke3shi4}
  \end{Phonetics}
\end{Entry}

\begin{Entry}{又一次}{2,1,6}{⼜,⼀,⽋}
  \begin{Phonetics}{又一次}{you4yi2ci4}
    \definition{adv.}{outra vez | mais uma vez | de novo}
  \end{Phonetics}
\end{Entry}

\begin{Entry}{又……又……}{2,2}{⼜,⼜}
  \begin{Phonetics}{又……又……}{you4 you4}
    \definition{conj.}{\dots e\dots; tanto\dots como\dots; as duas palavras usadas depois de 又 não devem ter nenhuma conotação de contraste, ambas devem ser positivas ou negativas}[这件毛衣挺不错的,\underline{又}便宜\underline{又}漂亮。===Este suéter é muito bom, barato e bonito.]
  \end{Phonetics}
\end{Entry}

\begin{Entry}{又及}{2,3}{⼜,⼃}
  \begin{Phonetics}{又及}{you4ji2}
    \definition{s.}{P.S., \emph{postscript}}
  \end{Phonetics}
\end{Entry}

\begin{Entry}{又名}{2,6}{⼜,⼝}
  \begin{Phonetics}{又名}{you4ming2}
    \definition{s.}{também conhecido como | nome alternativo}
  \end{Phonetics}
\end{Entry}

\begin{Entry}{又称}{2,10}{⼜,⽲}
  \begin{Phonetics}{又称}{you4cheng1}
    \definition{s.}{também conhecido como}
  \end{Phonetics}
\end{Entry}

%%%%% EOF %%%%%


 %%%
%%% 3画
%%%
\section*{3画}\addcontentsline{toc}{section}{3画}\addcontentsline{loh}{figure}{\#\#\#\# 3画}

%%%%%%%%%% 万 %%%%%%%%%%
\subsection*{万}\addcontentsline{loh}{figure}{万}

\begin{Entry}{万}{3}{⼀}
  \begin{Phonetics}{万}{wan4}[][HSK 2]
    \definition*{s.}{Sobrenome: Wan}
    \definition{adv.}{absolutamente; indica um grau extremamente alto, equivalente a 完全, 绝对 e 极}
    \definition{num.}{dez mil; 10.000; 1.0000 | miríade; um número muito grande}
  \seealsoref{极}{ji2}
  \seealsoref{绝对}{jue2dui4}
  \seealsoref{完全}{wan2quan2}
  \end{Phonetics}
\end{Entry}

\begin{Entry}{万一}{3,1}{⼀,⼀}
  \begin{Phonetics}{万一}{wan4yi1}[][HSK 4]
    \definition{conj.}{por via das dúvidas; se por acaso; só por precaução; expressa uma suposição muito improvável (usado para coisas desagradáveis)}
    \definition{num.}{um décimo milionésimo; uma porcentagem muito pequena}
    \definition{s.}{contingência; eventualidade; contingências muito improváveis}
  \synonymref{如果}{ru2guo3}
  \synonymref{要是}{yao4shi5}
  \synonymref{一旦}{yi2dan4}
  \end{Phonetics}
\end{Entry}

\begin{Entry}{万万}{3,3}{⼀,⼀}
  \begin{Phonetics}{万万}{wan4wan4}[][HSK 7-9]
    \definition{adv.}{totalmente; absolutamente; em qualquer caso; não importa o que aconteça}
    \definition{num.}{cem milhões; 100.000.000; 1.0000.0000}
  \synonymref{绝对}{jue2dui4}
  \synonymref{千万}{qian1wan4}
  \synonymref{完全}{wan2quan2}
  \end{Phonetics}
\end{Entry}

\begin{Entry}{万分}{3,4}{⼀,⼑}
  \begin{Phonetics}{万分}{wan4fen1}[][HSK 7-9]
    \definition{adv.}{extremamente; muito}
  \synonymref{非常}{fei1chang2}
  \synonymref{极度}{ji2du4}
  \synonymref{极端}{ji2duan1}
  \synonymref{十分}{shi2fen1}
  \synonymref{特别}{te4bie2}
  \synonymref{异常}{yi4chang2}
  \end{Phonetics}
\end{Entry}

\begin{Entry}{万无一失}{3,4,1,5}{⼀,⽆,⼀,⼤}
  \begin{Phonetics}{万无一失}{wan4wu2-yi1shi1}[][HSK 7-9]
    \definition{expr.}{não há perigo de algo dar errado; esteja do lado seguro; \dots não pode falhar em nenhuma circunstância; garantir sucesso total; nenhum risco; não há chance de erro; (faça com que seja mais do que provável que) nada dê errado; perfeitamente seguro; infalível; as chances são de mil para uma, não falharemos}
  \end{Phonetics}
\end{Entry}

\begin{Entry}{万古长青}{3,5,4,8}{⼀,⼝,⾧,⾭}
  \begin{Phonetics}{万古长青}{wan4gu3-chang2qing1}[][HSK 7-9]
    \definition{expr.}{seja perene; seja sempre verde; perene e eterno; sempre vivo; florescer para sempre; durar para sempre; permanecer fresco para sempre; sempre será verde como pinheiros e ciprestes por milhares de gerações; uma metáfora para um espírito nobre ou uma amizade profunda que nunca desaparecerá}
  \end{Phonetics}
\end{Entry}

\begin{Entry}{万圣节}{3,5,5}{⼀,⼟,⾋}
  \begin{Phonetics}{万圣节}{wan4 sheng4 jie2}
    \definition*{s.}{Dia de Todos os Santos}
  \seealsoref{万圣节前夕}{wan4sheng4 jie2 qian2xi1}
  \end{Phonetics}
\end{Entry}

\begin{Entry}{万圣节前夕}{3,5,5,9,3}{⼀,⼟,⾋,⼑,⼣}
  \begin{Phonetics}{万圣节前夕}{wan4sheng4 jie2 qian2xi1}
    \definition*{s.}{Véspera do Dia de Todos os Santos | Halloween}
  \seealsoref{万圣节}{wan4 sheng4 jie2}
  \end{Phonetics}
\end{Entry}

\begin{Entry}{万物}{3,8}{⼀,⽜}
  \begin{Phonetics}{万物}{wan4wu4}
    \definition{s.}{toda a criação; todas as coisas na Terra; todos os seres vivos; tudo no universo}
  \synonymref{生物}{sheng1wu4}
  \end{Phonetics}
\end{Entry}

\begin{Entry}{万能}{3,10}{⼀,⾁}
  \begin{Phonetics}{万能}{wan4neng2}[][HSK 7-9]
    \definition{adj.}{onipotente; todo-poderoso | universal; multifacetado; de múltiplos usos}
  \synonymref{全能}{quan2neng2}
  \end{Phonetics}
\end{Entry}

\begin{Entry}{万福}{3,13}{⼀,⽰}
  \begin{Phonetics}{万福}{wan4fu2}
    \definition{s.}{(antigo) reverência feminina; reverência}
  \end{Phonetics}
\end{Entry}

%%%%%%%%%% 丈 %%%%%%%%%%
\subsection*{丈}\addcontentsline{loh}{figure}{丈}

\begin{Entry}{丈}{3}{⼀}
  \begin{Phonetics}{丈}{zhang4}
    \definition{clas.}{zhang, uma unidade tradicional de comprimento, igual a 10 市尺 e equivalente a 3,333 metros ou 3,65 jardas}
    \definition{s.}{zhang, uma unidade de comprimento (= 3,333\dots metros)}
    \definition{s.}{sênior; ancião | marido (em certos termos de parentesco) | tratamento respeitoso ao idoso na China antiga; um título respeitoso para homens idosos nos tempos antigos | uma forma de tratamento para certos parentes do sexo masculino por casamento}
  \seealsoref{市尺}{shi4 chi3}
  \end{Phonetics}
\end{Entry}

\begin{Entry}{丈夫}{3,4}{⼀,⼤}
  \begin{Phonetics}{丈夫}{zhang4fu5}[][HSK 4]
    \definition[个,位,名]{s.}{marido; esposo}
  \end{Phonetics}
\end{Entry}

%%%%%%%%%% 三 %%%%%%%%%%
\subsection*{三}\addcontentsline{loh}{figure}{三}

\begin{Entry}{三}{3}{⼀}
  \begin{Phonetics}{三}{san1}[][HSK 1]
    \definition*{s.}{Sobrenome: San}
    \definition{num.}{três; 3 | muitos; vários; mais de dois; referindo"-se a muitos ou à maioria | alguns; poucos; menos; não muitos}
  \end{Phonetics}
\end{Entry}

\begin{Entry}{三角}{3,7}{⼀,⾓}
  \begin{Phonetics}{三角}{san1jiao3}[][HSK 7-9]
    \definition{adj.}{tripartido; que constitui uma relação tripartite}
    \definition[个,些]{s.}{triângulo; coisas triangulares | trigonometria, abreviação de 三角学}
  \seealsoref{三角学}{san1jiao3 xue2}
  \end{Phonetics}
\end{Entry}

\begin{Entry}{三角学}{3,7,8}{⼀,⾓,⼦}
  \begin{Phonetics}{三角学}{san1jiao3 xue2}
    \definition{s.}{trigonometria; um ramo da matemática que estuda principalmente as funções trigonométricas e suas propriedades, bem como suas aplicações em geometria}[我三角学学得很好。===Sou muito bom em trigonometria.]
  \end{Phonetics}
\end{Entry}

\begin{Entry}{三角恋爱}{3,7,10,10}{⼀,⾓,⼼,⽖}
  \begin{Phonetics}{三角恋爱}{san1jiao3lian4'ai4}
    \definition[场]{s.}{triângulo amoroso | triângulo eterno}
  \end{Phonetics}
\end{Entry}

\begin{Entry}{三明治}{3,8,8}{⼀,⽇,⽔}
  \begin{Phonetics}{三明治}{san1ming2zhi4}[][HSK 6]
    \definition[个,些,块]{s.}{Empréstimo linguístico: sanduíche, \emph{sandwich}}
  \end{Phonetics}
\end{Entry}

\begin{Entry}{三轮车}{3,8,4}{⼀,⾞,⾞}
  \begin{Phonetics}{三轮车}{san1lun2che1}
    \definition{s.}{triciclo}
  \end{Phonetics}
\end{Entry}

\begin{Entry}{三维}{3,11}{⼀,⽷}
  \begin{Phonetics}{三维}{san1wei2}[][HSK 7-9]
    \definition{s.}{três dimensões; 3D; tridimensional}[我们生活在三维空间。===Vivemos em um espaço tridimensional.]
  \end{Phonetics}
\end{Entry}

\begin{Entry}{三番五次}{3,12,4,6}{⼀,⽥,⼆,⽋}
  \begin{Phonetics}{三番五次}{san1fan1-wu3ci4}[][HSK 7-9]
    \definition{expr.}{repetidamente; de novo e de novo; várias e várias vezes; diversas vezes}
  \synonymref{翻来覆去}{fan1lai2-fu4qu4}
  \synonymref{接二连三}{jie1'er4-lian2san1}
  \end{Phonetics}
\end{Entry}

%%%%%%%%%% 上 %%%%%%%%%%
\subsection*{上}\addcontentsline{loh}{figure}{上}

\begin{Entry}{上}{3}{⼀}
  \begin{Phonetics}{上}{shang3}
    \definition{s.}{tom descendente-ascendente; significa o segundo tom dos quatro tons do mandarim, e também se refere ao terceiro tom do mandarim padrão}
  \end{Phonetics}
  \begin{Phonetics}{上}{shang4}[][HSK 1]
    \definition{adj.}{mais recente; último; anterior; tempo ou a sequência anterior | superior; mais alto; melhor; indica uma posição elevada em termos de qualidade, nível, etc. | lugar elevado; posição superior}
    \definition{s.}{superior; acima; para cima; um lugar alto ou mais alto do que um determinado local | na superfície de um objeto; usado após um substantivo, indica a superfície de um objeto | indica estar dentro do escopo de algo; usado após um substantivo, indica que algo está dentro do âmbito de determinada coisa | indica um aspecto específico | antigamente, referia"-se ao imperador | usado após palavras que indicam idade, equivale a ``\dots 的时候'' | o primeiro nível da escala da música folclórica chinesa, usado como um símbolo de nota na notação musical, equivalente ao ``1'' na notação simplificada.}
    \definition{v.}{subir; montar; embarcar; entrar | ir para; partir para | estar ocupado (com trabalho, estudos, etc.) em um horário fixo; começar a trabalhar ou estudar na hora marcada, etc. | seguir em frente; prosseguir | encher; abastecer; servir; melhorar; aumentar | aparecer no palco; entrar | colocar algo em posição; ajustar; fixar; montar as duas partes de algo | aplicar; pintar; espalhar | ser registrado; ser publicado (em uma publicação) | atingir; ser suficiente (uma determinada quantidade ou grau) | submeter; enviar; apresentar; submeter à aprovação superior | ventilar; apertar; torcer | trazer; servir; colocar comida, pratos, chá e outras coisas na mesa para os convidados | indicar que uma ação tem um resultado | pesquisar na \emph{Internet} | emaranhar-se; ficar emaranhado; enredar-se}
    \definition{v.aux.}{usado após um verbo para indicar início e continuidade}
  \seealsoref{的时候}{de5 shi2hou4}
  \antonymref{下}{xia4}
  \end{Phonetics}
\end{Entry}

\begin{Entry}{上下}{3,3}{⼀,⼀}
  \begin{Phonetics}{上下}{shang4xia4}[][HSK 5]
    \definition{adv.}{para cima e para baixo}
    \definition[顶]{s.}{alto e baixo | de cima para baixo; para cima e para baixo | superioridade ou inferioridade relativa | (após números redondos) aproximadamente; mais ou menos; por aí | velhos e jovens; hierarquia em termos de cargo e posição social}
    \definition{v.}{subir ou descer | subir e descer; da alta para a baixa ou da baixa para a alta}
  \synonymref{高低}{gao1di1}
  \antonymref{左右}{zuo3you4}
  \end{Phonetics}
\end{Entry}

\begin{Entry}{上个月}{3,3,4}{⼀,⼈,⽉}
  \begin{Phonetics}{上个月}{shang4ge4yue4}[][HSK 4]
    \definition{s.}{mês passado; refere"-se à hora de um mês atrás, ou seja, o mês passado mais próximo da hora atual}
  \end{Phonetics}
\end{Entry}

\begin{Entry}{上门}{3,3}{⼀,⾨}
  \begin{Phonetics}{上门}{shang4 men2}[][HSK 4]
    \definition{v.}{chamar; visitar; aparecer; ir ou vir para ver alguém; ir até a porta; ir até a casa de alguém | trancar a porta; fechar a porta durante a noite | casar-se e morar com a família da noiva}
  \synonymref{拜访}{bai4fang3}
  \end{Phonetics}
\end{Entry}

\begin{Entry}{上升}{3,4}{⼀,⼗}
  \begin{Phonetics}{上升}{shang4sheng1}[][HSK 3]
    \definition{v.}{elevar; subir; mover-se para cima; mover de baixo para cima; aumentar em nível, grau, quantidade, etc.}
  \synonymref{上涨}{shang4zhang3}
  \antonymref{回升}{hui2sheng1}
  \antonymref{降落}{jiang4luo4}
  \antonymref{落下}{luo4xia4}
  \antonymref{下降}{xia4jiang4}
  \end{Phonetics}
\end{Entry}

\begin{Entry}{上午}{3,4}{⼀,⼗}
  \begin{Phonetics}{上午}{shang4wu3}[][HSK 1]
    \definition[个]{s.}{manhã; \emph{ante meridiem} (a.m.); geralmente refere"-se ao período entre a manhã e o meio"-dia}
  \antonymref{下午}{xia4wu3}
  \end{Phonetics}
\end{Entry}

\begin{Entry}{上方}{3,4}{⼀,⽅}
  \begin{Phonetics}{上方}{shang4fang1}[][HSK 7-9]
    \definition{s.}{acima; sobre; em cima de | superjacente}
  \seealsoref{下方}{xia4fang1}
  \end{Phonetics}
\end{Entry}

\begin{Entry}{上火}{3,4}{⼀,⽕}
  \begin{Phonetics}{上火}{shang4/huo3}[][HSK 7-9]
    \definition{v.+compl.}{ter dor de garganta | ter excesso de calor interno; na medicina tradicional chinesa, sintomas como prisão de ventre ou inflamação da mucosa nasal, da mucosa oral ou da conjuntiva são classificados como ``calor interno'' | ficar com raiva}
  \seealsoref{上火儿}{shang4huo3r5}
  \end{Phonetics}
\end{Entry}

\begin{Entry}{上火儿}{3,4,2}{⼀,⽕,⼉}
  \begin{Phonetics}{上火儿}{shang4huo3r5}
    \definition{v.}{Dialeto: ficar com raiva; explodir}
  \end{Phonetics}
\end{Entry}

\begin{Entry}{上车}{3,4}{⼀,⾞}
  \begin{Phonetics}{上车}{shang4che1}[][HSK 1]
    \definition{v.}{entrar; subir (em um ônibus, trem, carro etc.)}
  \antonymref{下车}{xia4che1}
  \end{Phonetics}
\end{Entry}

\begin{Entry}{上去}{3,5}{⼀,⼛}
  \begin{Phonetics}{上去}{shang4 qu5}[][HSK 3]
    \definition{v.}{subir (a partir da minha localização) | ascender a um lugar (ou estado) considerado mais elevado (ou acima); usado depois de um verbo para indicar movimento, de baixo para cima ou de perto para longe}
  \synonymref{上来}{shang4 lai5}
  \antonymref{下来}{xia4 lai5}
  \end{Phonetics}
\end{Entry}

\begin{Entry}{上古}{3,5}{⼀,⼝}
  \begin{Phonetics}{上古}{shang4gu3}
    \definition{s.}{tempos antigos; eras remotas | antiguidade | tempos históricos antigos | o passado distante}
  \antonymref{现代}{xian4dai4}
  \end{Phonetics}
\end{Entry}

\begin{Entry}{上台}{3,5}{⼀,⼝}
  \begin{Phonetics}{上台}{shang4 tai2}[][HSK 6]
    \definition{v.}{aparecer no palco; subir na plataforma; ir para o palco ou pódio | assumir o poder; chegar (subir) ao poder; começar a assumir papéis de liderança ou a ganhar algum tipo de poder}
  \synonymref{上任}{shang4/ren4}
  \end{Phonetics}
\end{Entry}

\begin{Entry}{上司}{3,5}{⼀,⼝}
  \begin{Phonetics}{上司}{shang4si5}[][HSK 7-9]
    \definition[位,名,个]{s.}{chefe; superior}
  \synonymref{上级}{shang4ji2}
  \antonymref{部属}{bu4shu3}
  \antonymref{部下}{bu4xia4}
  \end{Phonetics}
\end{Entry}

\begin{Entry}{上头}{3,5}{⼀,⼤}
  \begin{Phonetics}{上头}{shang4tou2}
    \definition{v.}{(álcool, amor etc.) subir à cabeça; (uma ideia, uma música etc.) entrar na cabeça de alguém; capturar a atenção de alguém | Obsoleto: (uma garota no dia do seu casamento) começar a usar o cabelo preso em um coque (em vez de uma trança)}
  \synonymref{方面}{fang1mian4}
  \synonymref{上面}{shang4mian5}
  \end{Phonetics}
  \begin{Phonetics}{上头}{shang4tou5}[][HSK 7-9]
    \definition{s.}{acima; em cima de; na superfície de; superior}
  \end{Phonetics}
\end{Entry}

\begin{Entry}{上市}{3,5}{⼀,⼱}
  \begin{Phonetics}{上市}{shang4 shi4}[][HSK 6]
    \definition{v.}{listar; abrir o capital; ser listado (na bolsa de valores) | estar na estação; estar (aparecer) no mercado | ir ao mercado (para fazer compras)}
  \end{Phonetics}
\end{Entry}

\begin{Entry}{上边}{3,5}{⼀,⾡}
  \begin{Phonetics}{上边}{shang4bian5}[][HSK 1]
    \definition{s.}{topo; acima; sobre; superior}
  \synonymref{上方}{shang4fang1}
  \antonymref{下边}{xia4bian5}
  \end{Phonetics}
\end{Entry}

\begin{Entry}{上任}{3,6}{⼀,⼈}
  \begin{Phonetics}{上任}{shang4/ren4}[][HSK 7-9]
    \definition[班]{s.}{predecessor; ex"-funcionário}
    \definition{v.+compl.}{assumir o cargo; ocupar um cargo oficial; refere"-se à posse de autoridades}
  \synonymref{出任}{chu1ren4}
  \synonymref{就职}{jiu4/zhi2}
  \synonymref{就任}{jiu4ren4}
  \synonymref{上台}{shang4 tai2}
  \antonymref{辞职}{ci2/zhi2}
  \antonymref{离职}{li2/zhi2}
  \end{Phonetics}
\end{Entry}

\begin{Entry}{上场}{3,6}{⼀,⼟}
  \begin{Phonetics}{上场}{shang4/chang3}[][HSK 7-9]
    \definition{v.+compl.}{Esporte: entrar na quadra (ou campo); participar de uma competição | aparecer no palco; subir no palco; entrar em cena}
  \antonymref{退场}{tui4chang3}
  \end{Phonetics}
\end{Entry}

\begin{Entry}{上当}{3,6}{⼀,⼹}
  \begin{Phonetics}{上当}{shang4/dang4}[][HSK 6]
    \definition{v.+compl.}{ser enganado; ser ludibriado; morder a isca; cair nas mãos de alguém}
  \synonymref{受骗}{shou4/pian4}
  \antonymref{精明}{jing1ming2}
  \antonymref{警惕}{jing3ti4}
  \end{Phonetics}
\end{Entry}

\begin{Entry}{上旬}{3,6}{⼀,⽇}
  \begin{Phonetics}{上旬}{shang4xun2}[][HSK 7-9]
    \definition{s.}{primeiro terço do mês; os dez dias do dia 1 ao dia 10 de cada mês}
  \seealsoref{中旬}{zhong1xun2}
  \antonymref{下旬}{xia4xun2}
  \end{Phonetics}
\end{Entry}

\begin{Entry}{上次}{3,6}{⼀,⽋}
  \begin{Phonetics}{上次}{shang4ci4}[][HSK 1]
    \definition{adv.}{última vez}
  \end{Phonetics}
\end{Entry}

\begin{Entry}{上级}{3,6}{⼀,⽷}
  \begin{Phonetics}{上级}{shang4ji2}[][HSK 5]
    \definition[个,位]{s.}{nível superior; organização ou pessoa em nível superior; organizações ou pessoas de nível superior dentro do mesmo sistema organizacional}
  \synonymref{部属}{bu4shu3}
  \synonymref{上司}{shang4si5}
  \antonymref{部下}{bu4xia4}
  \end{Phonetics}
\end{Entry}

\begin{Entry}{上网}{3,6}{⼀,⽹}
  \begin{Phonetics}{上网}{shang4/wang3}[][HSK 1]
    \definition{v.+compl.}{conectar"-se à \emph{Internet}; acessar a \emph{Internet}; entrar na \emph{Internet}; acessar a rede; refere"-se especificamente ao computador do usuário conectado à \emph{Internet} para pesquisar e consultar informações, etc.}
  \end{Phonetics}
\end{Entry}

\begin{Entry}{上衣}{3,6}{⼀,⾐}
  \begin{Phonetics}{上衣}{shang4yi1}[][HSK 3]
    \definition[件]{s.}{jaqueta; roupas para a parte superior do corpo}
  \end{Phonetics}
\end{Entry}

\begin{Entry}{上访}{3,6}{⼀,⾔}
  \begin{Phonetics}{上访}{shang4fang3}
    \definition{v.}{buscar uma audiência com superiores (especialmente funcionários do governo) para fazer uma petição por algo}
  \end{Phonetics}
\end{Entry}

\begin{Entry}{上声}{3,7}{⼀,⼠}
  \begin{Phonetics}{上声}{shang3sheng1}
    \definition{s.}{tom descendente e ascendente | terceiro tom no mandarim moderno}
  \end{Phonetics}
\end{Entry}

\begin{Entry}{上岗}{3,7}{⼀,⼭}
  \begin{Phonetics}{上岗}{shang4/gang3}[][HSK 7-9]
    \definition{v.+compl.}{estar em período probatório | começar a trabalhar; assumir um cargo}
  \end{Phonetics}
\end{Entry}

\begin{Entry}{上报}{3,7}{⼀,⼿}
  \begin{Phonetics}{上报}{shang4bao4}[][HSK 7-9]
    \definition{v.}{aparecer nos jornais; ser publicado | reportar a um órgão superior; reportar à liderança; reportar-se aos superiores}
  \synonymref{报到}{bao4/dao4}
  \end{Phonetics}
\end{Entry}

\begin{Entry}{上来}{3,7}{⼀,⽊}
  \begin{Phonetics}{上来}{shang4 lai5}[][HSK 3]
    \definition{v.}{subir (para a minha localização) | estar no começo; começar; iniciar | surgir; de um lugar baixo para um lugar alto (o interlocutor está em um lugar alto) | usado após o verbo, indica que algo foi concluído com sucesso}
  \synonymref{上去}{shang4 qu5}
  \antonymref{下去}{xia4 qu5}
  \end{Phonetics}
\end{Entry}

\begin{Entry}{上诉}{3,7}{⼀,⾔}
  \begin{Phonetics}{上诉}{shang4su4}[][HSK 7-9]
    \definition{s.}{apelação (para um tribunal superior)}
    \definition{v.}{apresentar um recurso; instaurar um recurso}
  \end{Phonetics}
\end{Entry}

\begin{Entry}{上周}{3,8}{⼀,⼝}
  \begin{Phonetics}{上周}{shang4zhou1}[][HSK 2]
    \definition{s.}{semana passada}
  \antonymref{下周}{xia4zhou1}
  \end{Phonetics}
\end{Entry}

\begin{Entry}{上坡路}{3,8,13}{⼀,⼟,⾜}
  \begin{Phonetics}{上坡路}{shang4po1lu4}
    \definition{s.}{aclive | progresso | (fig.) tendência ascendente}
  \end{Phonetics}
\end{Entry}

\begin{Entry}{上学}{3,8}{⼀,⼦}
  \begin{Phonetics}{上学}{shang4 xue2}[][HSK 1]
    \definition{v.}{ir à escola; frequentar a escola; estar na escola; ir à escola para estudar | começar a escola; começar a estudar no ensino fundamental}
  \synonymref{入学}{ru4/xue2}
  \synonymref{上课}{shang4/ke4}
  \antonymref{放学}{fang4/xue2}
  \end{Phonetics}
\end{Entry}

\begin{Entry}{上空}{3,8}{⼀,⽳}
  \begin{Phonetics}{上空}{shang4kong1}[][HSK 7-9]
    \definition{s.}{no céu; acima da cabeça; no alto, no ar}
  \end{Phonetics}
\end{Entry}

\begin{Entry}{上述}{3,8}{⼀,⾡}
  \begin{Phonetics}{上述}{shang4shu4}[][HSK 7-9]
    \definition{adj.}{mencionado anteriormente; supracitado; acima citado; conforme dito ou narrado acima}
  \synonymref{描述}{miao2shu4}
  \synonymref{摸索}{mo1suo3}
  \synonymref{试验}{shi4yan4}
  \end{Phonetics}
\end{Entry}

\begin{Entry}{上限}{3,8}{⼀,⾩}
  \begin{Phonetics}{上限}{shang4xian4}[][HSK 7-9]
    \definition[个]{s.}{teto; limite superior; refere"-se ao primeiro ou mais alto limite dentro de um determinado conjunto de limites}
  \antonymref{下限}{xia4xian4}
  \end{Phonetics}
\end{Entry}

\begin{Entry}{上帝}{3,9}{⼀,⼱}
  \begin{Phonetics}{上帝}{shang4di4}[][HSK 6]
    \definition*{s.}{Deus; O Deus Supremo no Cristianismo | O Imperador do Céu; um deus na antiga crença chinesa que pode controlar tudo no mundo}
    \definition[个]{s.}{(figurado) cliente; metáfora para consumidores}
  \end{Phonetics}
\end{Entry}

\begin{Entry}{上映}{3,9}{⼀,⽇}
  \begin{Phonetics}{上映}{shang4ying4}[][HSK 7-9]
    \definition{v.}{executar; exibir; mostrar (um filme); (novo filme) lançar para exibição}
  \synonymref{放映}{fang4ying4}
  \end{Phonetics}
\end{Entry}

\begin{Entry}{上面}{3,9}{⼀,⾯}
  \begin{Phonetics}{上面}{shang4mian5}[][HSK 3]
    \definition{s.}{uma posição mais alta que algo; uma posição acima/acima de algo | superfície do objeto | aspecto | a parte acima mencionada; a parte que vem primeiro na ordem; a parte de um artigo ou discurso que vem antes da presente | autoridades superiores | os mais velhos; a geração mais velha da família}
  \synonymref{方面}{fang1mian4}
  \synonymref{上方}{shang4fang1}
  \synonymref{上头}{shang4tou5}
  \antonymref{底下}{di3xia5}
  \antonymref{下面}{xia4mian5}
  \end{Phonetics}
\end{Entry}

\begin{Entry}{上流}{3,10}{⼀,⽔}
  \begin{Phonetics}{上流}{shang4liu2}[][HSK 7-9]
    \definition{adj.}{da classe alta; refinado | pertencente aos círculos superiores; anteriormente, referia"-se a pessoas de alto \emph{status} social}
    \definition{s.}{trecho superior (de um rio); a montante}
  \synonymref{崇高}{chong2gao1}
  \synonymref{高超}{gao1chao1}
  \synonymref{高贵}{gao1gui4}
  \synonymref{高尚}{gao1shang4}
  \end{Phonetics}
\end{Entry}

\begin{Entry}{上海}{3,10}{⼀,⽔}
  \begin{Phonetics}{上海}{shang4hai3}
    \definition*{s.}{Município de Xangai (Shanghai), centro-leste da China}
  \end{Phonetics}
\end{Entry}

\begin{Entry}{上涨}{3,10}{⼀,⽔}
  \begin{Phonetics}{上涨}{shang4zhang3}[][HSK 5]
    \definition{v.}{subir; ir para cima; ascender}
  \synonymref{高潮}{gao1chao2}
  \synonymref{高涨}{gao1zhang3}
  \synonymref{上升}{shang4sheng1}
  \antonymref{回落}{hui2luo4}
  \antonymref{下降}{xia4jiang4}
  \end{Phonetics}
\end{Entry}

\begin{Entry}{上班}{3,10}{⼀,⽟}
  \begin{Phonetics}{上班}{shang4/ban1}[][HSK 1]
    \definition{v.+compl.}{ir trabalhar; começar a trabalhar; estar de plantão; ir trabalhar no local de trabalho regular no horário especificado}
  \synonymref{加班}{jia1/ban1}
  \antonymref{下班}{xia4/ban1}
  \end{Phonetics}
\end{Entry}

\begin{Entry}{上班族}{3,10,11}{⼀,⽟,⽅}
  \begin{Phonetics}{上班族}{shang4 ban1 zu2}
    \definition[本]{s.}{trabalhadores de escritório (como grupo social)}
  \end{Phonetics}
\end{Entry}

\begin{Entry}{上课}{3,10}{⼀,⾔}
  \begin{Phonetics}{上课}{shang4/ke4}[][HSK 1]
    \definition{v.+compl.}{frequentar aulas; ir às aulas; dar uma aula}
  \synonymref{上学}{shang4 xue2}
  \antonymref{下课}{xia4/ke4}
  \end{Phonetics}
\end{Entry}

\begin{Entry}{上调}{3,10}{⼀,⾔}
  \begin{Phonetics}{上调}{shang4tiao2}[][HSK 7-9]
    \definition{v.}{transferir (alguém) para um cargo de nível superior | transferir bens, fundos, etc. para uma unidade de nível superior | ajustar para cima | aumentar (os preços)}
  \antonymref{下调}{xia4tiao2}
  \end{Phonetics}
\end{Entry}

\begin{Entry}{上期}{3,12}{⼀,⽉}
  \begin{Phonetics}{上期}{shang4 qi1}[][HSK 7-9]
    \definition{s.}{período anterior}
  \end{Phonetics}
\end{Entry}

\begin{Entry}{上游}{3,12}{⼀,⽔}
  \begin{Phonetics}{上游}{shang4you2}[][HSK 7-9]
    \definition{s.}{a montante; trecho superior de um rio; o trecho de um rio próximo à sua nascente; também se refere à área por onde esse trecho flui | posição avançada; metaforicamente, refere"-se a um \emph{status} ou nível avançado}
  \synonymref{领先}{ling3/xian1}
  \synonymref{先进}{xian1jin4}
  \antonymref{下游}{xia4you2}
  \end{Phonetics}
\end{Entry}

\begin{Entry}{上楼}{3,13}{⼀,⽊}
  \begin{Phonetics}{上楼}{shang4lou2}[][HSK 4]
    \definition{v.}{subir as escadas; ir para o andar de cima}
  \end{Phonetics}
\end{Entry}

\begin{Entry}{上演}{3,14}{⼀,⽔}
  \begin{Phonetics}{上演}{shang4yan3}[][HSK 6]
    \definition{s.}{exibição | encenação}
    \definition{v.}{exibir (um filme); encenar (uma peça); atuar; colocar no palco}
  \synonymref{演出}{yan3chu1}
  \end{Phonetics}
\end{Entry}

\begin{Entry}{上瘾}{3,16}{⼀,⽧}
  \begin{Phonetics}{上瘾}{shang4/yin3}[][HSK 7-9]
    \definition{v.+compl.}{ser viciado (em algo); adquirir o hábito (de fazer algo); gostar muito de algo, a ponto de não conseguir viver sem}
  \synonymref{沉迷}{chen2mi2}
  \synonymref{痴迷}{chi1mi2}
  \synonymref{陶醉}{tao2zui4}
  \end{Phonetics}
\end{Entry}

%%%%%%%%%% 下 %%%%%%%%%%
\subsection*{下}\addcontentsline{loh}{figure}{下}

\begin{Entry}{下}{3}{⼀}
  \begin{Phonetics}{下}{xia4}[][HSK 1,2]
    \definition{clas.}{número de vezes usado para a ação | volume de um contêiner; quantidade de objetos que cabem em um utensílio | usado depois de 两 e 几 para expressar habilidade, capacidade, destreza}
    \definition{s.}{abaixo | próximo; último; segundo; referindo"-se ao que está por vir ou ao que vem em seguida | mais baixo; inferior; de baixo nível ou grau | próximo; último; segundo; em ordem ou em ordem cronológica | indica pertencer a uma determinada faixa, situação, condição, etc. | indica uma determinada época ou estação | usado após um número para indicar posição ou direção | para baixo (após uma preposição) | sob (depois de um substantivo) | para baixo (antes de um verbo)}
    \definition{v.}{desembarcar; descer; sair | cair (chuva, neve, etc.) | enviar; emitir; entregar | ir para | sair; partir; retirar-se | lançar; colocar | descarregar; desmontar; tirar (fora) | formar (uma opinião, ideia, etc.); tomar decisões, fazer julgamentos, etc. | usar; aplicar | dar à luz (animais) | tomar; capturar; conquistar | ceder | terminar; deixar de lado; terminar o trabalho ou os estudos diários na hora prevista | para negação; ser inferior a; ser menor que}
  \seealsoref{几}{ji3}
  \seealsoref{两}{liang3}
  \antonymref{高}{gao1}
  \antonymref{上}{shang4}
  \end{Phonetics}
\end{Entry}

\begin{Entry}{下个月}{3,3,4}{⼀,⼈,⽉}
  \begin{Phonetics}{下个月}{xia4ge4yue4}[][HSK 4]
    \definition{s.}{próximo mês; mês que vem; refere"-se ao próximo mês do mês atual}
  \end{Phonetics}
\end{Entry}

\begin{Entry}{下午}{3,4}{⼀,⼗}
  \begin{Phonetics}{下午}{xia4wu3}[][HSK 1]
    \definition[个]{s.}{tarde; \emph{post meridiem} (p.m.); refere"-se ao período entre o meio"-dia e o pôr do sol}
  \antonymref{上午}{shang4wu3}
  \end{Phonetics}
\end{Entry}

\begin{Entry}{下午茶}{3,4,9}{⼀,⼗,⾋}
  \begin{Phonetics}{下午茶}{xia4wu3cha2}
    \definition[杯]{s.}{chá da tarde (normalmente chás com doces)}
  \end{Phonetics}
\end{Entry}

\begin{Entry}{下巴}{3,4}{⼀,⼰}
  \begin{Phonetics}{下巴}{xia4ba5}
    \definition[个]{s.}{queixo | mandíbula inferior}
  \end{Phonetics}
\end{Entry}

\begin{Entry}{下方}{3,4}{⼀,⽅}
  \begin{Phonetics}{下方}{xia4fang1}
    \definition{s.}{parte inferior | abaixo | embaixo | mundo dos mortais}
    \definition{v.}{descer ao mundo dos mortais (deuses)}
  \seealsoref{上方}{shang4fang1}
  \antonymref{上方}{shang4fang1}
  \end{Phonetics}
\end{Entry}

\begin{Entry}{下水道}{3,4,12}{⼀,⽔,⾡}
  \begin{Phonetics}{下水道}{xia4shui3dao4}
    \definition{s.}{esgoto}
  \end{Phonetics}
\end{Entry}

\begin{Entry}{下车}{3,4}{⼀,⾞}
  \begin{Phonetics}{下车}{xia4che1}[][HSK 1]
    \definition{v.}{descer ou sair de (um ônibus, trem, carro etc.)}
  \end{Phonetics}
\end{Entry}

\begin{Entry}{下去}{3,5}{⼀,⼛}
  \begin{Phonetics}{下去}{xia4 qu5}[][HSK 3]
    \definition{part.}{usado depois de verbos para indicar de alto a baixo | usado depois de um verbo para indicar continuação}
    \definition{v.}{descer; baixar (a partir da minha localização) | (após um verbo) continuar (fazendo algo); prosseguir | usado após o verbo, indica uma descida de um ponto alto para um ponto baixo | usado após o verbo, indica continuidade | usado após um adjetivo, indica que o grau continua aumentando}
  \synonymref{出来}{chu1 lai5}
  \synonymref{下来}{xia4 lai5}
  \antonymref{上来}{shang4 lai5}
  \end{Phonetics}
\end{Entry}

\begin{Entry}{下边}{3,5}{⼀,⾡}
  \begin{Phonetics}{下边}{xia4bian5}[][HSK 1]
    \definition{s.}{abaixo; sob; por baixo | próximo em ordem; seguinte | nível inferior; subordinado | a parte inferior}
  \antonymref{上边}{shang4bian5}
  \end{Phonetics}
\end{Entry}

\begin{Entry}{下旬}{3,6}{⼀,⽇}
  \begin{Phonetics}{下旬}{xia4xun2}
    \definition[月]{s.}{última dezena do mês; último período de dez dias de um mês; do dia 21 até o final de cada mês}
  \seealsoref{中旬}{zhong1xun2}
  \antonymref{上旬}{shang4xun2}
  \end{Phonetics}
\end{Entry}

\begin{Entry}{下次}{3,6}{⼀,⽋}
  \begin{Phonetics}{下次}{xia4ci4}[][HSK 1]
    \definition{s.}{na próxima vez; na próxima oportunidade ou no próximo evento}
  \synonymref{再次}{zai4ci4}
  \end{Phonetics}
\end{Entry}

\begin{Entry}{下来}{3,7}{⼀,⽊}
  \begin{Phonetics}{下来}{xia4 lai5}[][HSK 3]
    \definition{part.}{usado após o verbo, indica que a ação ou o comportamento se dirige para a posição do falante ou que a ação é contínua ou concluída | usado após um adjetivo, indica que uma determinada situação começou a ocorrer e continuará a se desenvolver}
    \definition{v.}{descer (para a minha localização) | (colheitas/frutas/vegetais, etc.) ser colhido; estar maduro o suficiente para ser colhido | (período de tempo) acabar; passar; chegar ao fim; indicar o fim de um período de tempo}
  \synonymref{下去}{xia4 qu5}
  \antonymref{上去}{shang4 qu5}
  \end{Phonetics}
\end{Entry}

\begin{Entry}{下周}{3,8}{⼀,⼝}
  \begin{Phonetics}{下周}{xia4zhou1}[][HSK 2]
    \definition{s.}{próxima semana}
  \antonymref{上周}{shang4zhou1}
  \end{Phonetics}
\end{Entry}

\begin{Entry}{下线}{3,8}{⼀,⽷}
  \begin{Phonetics}{下线}{xia4xian4}
    \definition{v.}{sair da sessão; desconectar-se da \emph{Internet}; refere"-se à suspensão temporária das atividades de comunicação online, geralmente significando interromper temporariamente o bate"-papo online ou os jogos online | sair da linha de produção; isso se refere a automóveis, eletrodomésticos, etc., que foram montados na linha de produção e estão prontos para sair da fábrica}
  \synonymref{离开}{li2/kai1}
  \antonymref{登陆}{deng1/lu4}
  \antonymref{在线}{zai4xian4}
  \end{Phonetics}
\end{Entry}

\begin{Entry}{下降}{3,8}{⼀,⾩}
  \begin{Phonetics}{下降}{xia4jiang4}[][HSK 4]
    \definition{v.}{cair; despencar; declinar; descer; diminuir; ir para baixo}
  \synonymref{低落}{di1luo4}
  \synonymref{降低}{jiang4di1}
  \synonymref{降落}{jiang4luo4}
  \antonymref{回升}{hui2sheng1}
  \antonymref{爬升}{pa2sheng1}
  \antonymref{起飞}{qi3fei1}
  \antonymref{上升}{shang4sheng1}
  \antonymref{上涨}{shang4zhang3}
  \antonymref{增加}{zeng1jia1}
  \antonymref{增长}{zeng1zhang3}
  \end{Phonetics}
\end{Entry}

\begin{Entry}{下限}{3,8}{⼀,⾩}
  \begin{Phonetics}{下限}{xia4xian4}
    \definition{s.}{limite mínimo ou mais recente permitido; limite inferior; limiar; mínimo prescrito; piso (nível)}
  \antonymref{上限}{shang4xian4}
  \end{Phonetics}
\end{Entry}

\begin{Entry}{下雨}{3,8}{⼀,⾬}
  \begin{Phonetics}{下雨}{xia4/yu3}[][HSK 1]
    \definition{v.+compl.}{chover}
  \synonymref{暴雨}{bao4yu3}
  \synonymref{大雨}{da4yu3}
  \end{Phonetics}
\end{Entry}

\begin{Entry}{下面}{3,9}{⼀,⾯}
  \begin{Phonetics}{下面}{xia4mian5}[][HSK 3]
    \definition{s.}{em baixo; abaixo; parte de baixo | próximo; seguinte; a parte posterior; a parte posterior de um artigo ou discurso em relação ao que está sendo narrado no momento | subordinado; o nível inferior; homens nos níveis inferiores | por baixo}
  \synonymref{底下}{di3xia5}
  \synonymref{上方}{shang4fang1}
  \antonymref{上面}{shang4mian5}
  \end{Phonetics}
\end{Entry}

\begin{Entry}{下海}{3,10}{⼀,⽔}
  \begin{Phonetics}{下海}{xia4/hai3}
    \definition{v.+compl.}{ir pescar no mar; pescar para ganhar a vida; ir para o mar | tornar"-se profissional; isso se refere a atores de ópera amadores que se tornam atores profissionais | deixar o emprego original e abrir o próprio negócio; refere"-se a pessoas que não eram originalmente donas de negócios, mas que passaram a empreender}
  \antonymref{登陆}{deng1/lu4}
  \end{Phonetics}
\end{Entry}

\begin{Entry}{下班}{3,10}{⼀,⽟}
  \begin{Phonetics}{下班}{xia4/ban1}[][HSK 1]
    \definition{v.+compl.}{sair do trabalho; bater ponto; terminar o trabalho na hora prevista e sair do local de trabalho}
  \antonymref{开工}{kai1/gong1}
  \antonymref{上班}{shang4/ban1}
  \end{Phonetics}
\end{Entry}

\begin{Entry}{下课}{3,10}{⼀,⾔}
  \begin{Phonetics}{下课}{xia4/ke4}[][HSK 1]
    \definition{v.+compl.}{terminar a aula; sair da aula}
  \antonymref{上课}{shang4/ke4}
  \end{Phonetics}
\end{Entry}

\begin{Entry}{下调}{3,10}{⼀,⾔}
  \begin{Phonetics}{下调}{xia4diao4}
    \definition{v.}{rebaixar; diminuir a regulamentação | passar para uma unidade inferior}
  \end{Phonetics}
  \begin{Phonetics}{下调}{xia4tiao2}
    \definition{v.}{regular para baixo; ajustar para baixo}
  \end{Phonetics}
\end{Entry}

\begin{Entry}{下载}{3,10}{⼀,⾞}
  \begin{Phonetics}{下载}{xia4zai3}[][HSK 4]
    \definition{v.}{\emph{download}; baixar; salvar informações da \emph{Web} em um dispositivo, como um computador}
  \antonymref{传输}{chuan2shu1}
  \end{Phonetics}
\end{Entry}

\begin{Entry}{下蛋}{3,11}{⼀,⾍}
  \begin{Phonetics}{下蛋}{xia4dan4}
    \definition{v.}{botar ovos}
  \end{Phonetics}
\end{Entry}

\begin{Entry}{下雪}{3,11}{⼀,⾬}
  \begin{Phonetics}{下雪}{xia4/xue3}[][HSK 2]
    \definition{v.+compl.}{nevar}
  \end{Phonetics}
\end{Entry}

\begin{Entry}{下崽}{3,12}{⼀,⼭}
  \begin{Phonetics}{下崽}{xia4zai3}
    \definition{v.}{dar à luz à filhotes; parir}
  \end{Phonetics}
\end{Entry}

\begin{Entry}{下游}{3,12}{⼀,⽔}
  \begin{Phonetics}{下游}{xia4you2}
    \definition{s.}{a jusante; rio abaixo; trechos inferiores; o trecho do rio próximo à sua foz | para trás; a posição inferior; referindo"-se metaforicamente a uma posição invertida}
  \antonymref{上游}{shang4you2}
  \end{Phonetics}
\end{Entry}

\begin{Entry}{下楼}{3,13}{⼀,⽊}
  \begin{Phonetics}{下楼}{xia4lou2}[][HSK 4]
    \definition{v.}{descer as escadas}
  \end{Phonetics}
\end{Entry}

%%%%%%%%%% 与 %%%%%%%%%%
\subsection*{与}\addcontentsline{loh}{figure}{与}

\begin{Entry}{与}{3}{⼀}
  \begin{Phonetics}{与}{yu2}
    \definition{part.}{não; o quê; hein}
    \variantof{欤}
  \end{Phonetics}
  \begin{Phonetics}{与}{yu3}[][HSK 6]
    \definition*{s.}{Sobrenome: Yu}
    \definition{conj.}{e; junto com}
    \definition{prep.}{com}
    \definition{v.}{dar; oferecer; conceder | conviver com; estar em bons termos com; socializar; ser amigável | ajudar; apoiar; patrocinar | Literário: esperar}
  \synonymref{给}{gei3}
  \synonymref{跟}{gen1}
  \synonymref{和}{he2}
  \antonymref{取}{qu3}
  \end{Phonetics}
  \begin{Phonetics}{与}{yu4}
    \definition{v.}{participar de; tomar parte em}
  \synonymref{参加}{can1jia1}
  \synonymref{出席}{chu1/xi2}
  \synonymref{加入}{jia1ru4}
  \synonymref{介入}{jie4ru4}
  \antonymref{旁观}{pang2guan1}
  \end{Phonetics}
\end{Entry}

\begin{Entry}{与其}{3,8}{⼀,⼋}
  \begin{Phonetics}{与其}{yu3qi2}
    \definition{conj.}{em vez de\dots; orações de ligação, indicando uma decisão tomada após comparação; 与其 é usado para o aspecto a ser abandonado, enquanto o aspecto a ser escolhido é frequentemente expresso por frases como 不如 ou 宁可}
  \seealsoref{不如}{bu4ru2}
  \seealsoref{宁可}{ning4ke3}
  \synonymref{如果}{ru2guo3}
  \synonymref{因为}{yin1wei5}
  \end{Phonetics}
\end{Entry}

\begin{Entry}{与其……不如……}{3,8,4,6}{⼀,⼋,⼀,⼥}
  \begin{Phonetics}{与其……不如……}{yu3qi2 bu4ru2}
    \definition{conj.}{em vez de\dots é melhor\dots}
  \end{Phonetics}
\end{Entry}

\begin{Entry}{与其……宁可……}{3,8,5,5}{⼀,⼋,⼧,⼝}
  \begin{Phonetics}{与其……宁可……}{yu3qi2 ning4ke3}
    \definition{conj.}{em vez de\dots eu prefereria\dots}
  \end{Phonetics}
\end{Entry}

%%%%%%%%%% 个 %%%%%%%%%%
\subsection*{个}\addcontentsline{loh}{figure}{个}

\begin{Entry}{个}{3}{⼈}
  \begin{Phonetics}{个}{ge3}
    \definition{pron.}{usado em 自个儿}
  \seealsoref{自个儿}{zi4ge3r5}
  \end{Phonetics}
  \begin{Phonetics}{个}{ge4}[][HSK 1]
    \definition{adj.}{individual}
    \definition{clas.}{usado antes de substantivos que não têm palavras de medida específicas | usado na frente do divisor; usado na frente do número aproximado | usado após verbos com objeto direto |  usado entre verbos e complementos}
    \definition{part.}{usado após pronomes demonstrativos | adicionado após certas palavras de tempo}
  \end{Phonetics}
\end{Entry}

\begin{Entry}{个人}{3,2}{⼈,⼈}
  \begin{Phonetics}{个人}{ge4ren2}[][HSK 3]
    \definition{pron.}{pessoal; si mesmo}
    \definition[个]{s.}{indivíduo; pessoa}
  \antonymref{集体}{ji2ti3}
  \antonymref{团体}{tuan2ti3}
  \end{Phonetics}
\end{Entry}

\begin{Entry}{个儿}{3,2}{⼈,⼉}
  \begin{Phonetics}{个儿}{ge4r5}[][HSK 5]
    \definition{s.}{tamanho; altura; estatura; tamanho do corpo ou do objeto | pessoas ou coisas consideradas isoladamente; referir-se a uma pessoa ou coisa individualmente}
  \end{Phonetics}
\end{Entry}

\begin{Entry}{个子}{3,3}{⼈,⼦}
  \begin{Phonetics}{个子}{ge4zi5}[][HSK 2]
    \definition[个,种,些]{s.}{altura; estatura; refere"-se ao tamanho do corpo humano e também ao tamanho do corpo dos animais}
  \synonymref{身高}{shen1gao1}
  \end{Phonetics}
\end{Entry}

\begin{Entry}{个头儿}{3,5,2}{⼈,⼤,⼉}
  \begin{Phonetics}{个头儿}{ge4tou2er5}[][HSK 7-9]
    \definition{s.}{tamanho; altura}
  \end{Phonetics}
\end{Entry}

\begin{Entry}{个体}{3,7}{⼈,⼈}
  \begin{Phonetics}{个体}{ge4ti3}[][HSK 4]
    \definition[个,位]{s.}{uma única pessoa ou organismo}
  \synonymref{个别}{ge4bie2}
  \synonymref{个人}{ge4ren2}
  \antonymref{集体}{ji2ti3}
  \antonymref{群体}{qun2ti3}
  \antonymref{系统}{xi4tong3}
  \antonymref{总体}{zong3ti3}
  \end{Phonetics}
\end{Entry}

\begin{Entry}{个别}{3,7}{⼈,⼑}
  \begin{Phonetics}{个别}{ge4bie2}[][HSK 4]
    \definition{adj.}{muito poucos; excepcionais}
    \definition{adv.}{separadamente; individualmente; isoladamente}
  \synonymref{部分}{bu4fen5}
  \synonymref{个人}{ge4ren2}
  \synonymref{个体}{ge4ti3}
  \synonymref{局部}{ju2bu4}
  \synonymref{片面}{pian4mian4}
  \synonymref{一部分}{yi2bu4fen5}
  \antonymref{多数}{duo1shu4}
  \antonymref{集体}{ji2ti3}
  \antonymref{普遍}{pu3bian4}
  \antonymref{普通}{pu3tong1}
  \antonymref{全都}{quan2dou1}
  \antonymref{全体}{quan2ti3}
  \antonymref{特殊}{te4shu1}
  \antonymref{一般}{yi4ban1}
  \end{Phonetics}
\end{Entry}

\begin{Entry}{个性}{3,8}{⼈,⼼}
  \begin{Phonetics}{个性}{ge4xing4}[][HSK 3]
    \definition[种,点儿]{s.}{individualidade; personalidade; caráter individual; as características relativamente fixas de uma pessoa, formadas sob determinadas condições sociais e influências educacionais | propriedade específica; caráter específico; a propriedade ou característica especial que distingue uma coisa de outras coisas}
  \synonymref{本性}{ben3xing4}
  \synonymref{脾气}{pi2qi5}
  \synonymref{特性}{te4xing4}
  \synonymref{天性}{tian1xing4}
  \synonymref{性格}{xing4ge2}
  \antonymref{共性}{gong4xing4}
  \end{Phonetics}
\end{Entry}

\begin{Entry}{个案}{3,10}{⼈,⽊}
  \begin{Phonetics}{个案}{ge4'an4}[][HSK 7-9]
    \definition[个,些]{s.}{caso individual (ou especial); caso; caso a caso}
  \end{Phonetics}
\end{Entry}

%%%%%%%%%% 丸 %%%%%%%%%%
\subsection*{丸}\addcontentsline{loh}{figure}{丸}

\begin{Entry}{丸}{3}{⼂}
  \begin{Phonetics}{丸}{wan2}[][HSK 7-9]
    \definition{clas.}{utilizado para medicamentos em comprimido}
    \definition[个]{s.}{bola; grânulo | comprimido; bolo (como em bolo alimentar); pílula}
  \seealsoref{丸儿}{wan2r5}
  \end{Phonetics}
\end{Entry}

\begin{Entry}{丸儿}{3,2}{⼂,⼉}
  \begin{Phonetics}{丸儿}{wan2r5}
    \definition{s.}{bola; grânulo}
  \end{Phonetics}
\end{Entry}

%%%%%%%%%% 久 %%%%%%%%%%
\subsection*{久}\addcontentsline{loh}{figure}{久}

\begin{Entry}{久}{3}{⼃}
  \begin{Phonetics}{久}{jiu3}[][HSK 3]
    \definition{adj.}{por muito tempo; longo período de tempo | duração de tempo especificada}
  \antonymref{暂}{zan4}
  \end{Phonetics}
\end{Entry}

\begin{Entry}{久仰}{3,6}{⼃,⼈}
  \begin{Phonetics}{久仰}{jiu3yang3}[][HSK 7-9]
    \definition{expr.}{``É um prazer conhecê-lo(a).''; ``Há muito tempo que desejo conhecê-lo(a).''; ``Há muito tempo que aguardo ansiosamente o nosso encontro.''; ``Já ouvi falar muito de você.''}
  \end{Phonetics}
\end{Entry}

\begin{Entry}{久违}{3,7}{⼃,⾡}
  \begin{Phonetics}{久违}{jiu3wei2}[][HSK 7-9]
    \definition{v.}{não ver há muito tempo; fazer muito tempo desde o último encontro; apenas um comentário educado, muito tempo sem ver}
  \synonymref{重逢}{chong2feng2}
  \antonymref{经常}{jing1chang2}
  \end{Phonetics}
\end{Entry}

%%%%%%%%%% 义 %%%%%%%%%%
\subsection*{义}\addcontentsline{loh}{figure}{义}

\begin{Entry}{义}{3}{⼂}
  \begin{Phonetics}{义}{yi4}
    \definition*{s.}{Sobrenome: Yi}
    \definition{adj.}{justo; equitativo | adotado; adotivo | juramentado | artificial; falso}
    \definition[个]{s.}{justiça; retidão | laços humanos; relacionamento | significado; importância}
  \end{Phonetics}
\end{Entry}

\begin{Entry}{义务}{3,5}{⼂,⼒}
  \begin{Phonetics}{义务}{yi4wu4}[][HSK 4]
    \definition{adj.}{voluntário; fornecer serviços ou ajuda a outros gratuitamente}
    \definition[项]{s.}{dever; obrigação; responsabilidades perante a lei | obrigação moral; responsabilidade moral}
  \synonymref{负担}{fu4dan1}
  \synonymref{任务}{ren4wu5}
  \synonymref{仔肩}{zi1jian1}
  \antonymref{权利}{quan2li4}
  \antonymref{权力}{quan2li4}
  \antonymref{权益}{quan2yi4}
  \antonymref{责任}{ze2ren4}
  \end{Phonetics}
\end{Entry}

%%%%%%%%%% 之 %%%%%%%%%%
\subsection*{之}\addcontentsline{loh}{figure}{之}

\begin{Entry}{之}{3}{⼂}
  \begin{Phonetics}{之}{zhi1}
    \definition*{s.}{Sobrenome: Zhi}
    \definition{part.}{entre um atributivo e a palavra que ele modifica; equivalente a 的 | usado entre o sujeito e o predicado, em estruturas sujeito-predicado, de modo a torná-lo nominalizado}
    \definition{pron.}{substituto de uma pessoa ou coisa, limitado a ser usado como um objeto; substituir a pessoa ou coisa mencionada anteriormente | isto; isso; não substitui uma pessoa ou coisa específica, mas serve apenas para complementar sílabas}
    \definition{v.}{ir; deixar}
  \seealsoref{的}{de5}
  \end{Phonetics}
\end{Entry}

\begin{Entry}{之一}{3,1}{⼂,⼀}
  \begin{Phonetics}{之一}{zhi1yi1}[][HSK 4]
    \definition[分]{s.}{um de (algo); pertence a um ou a todo um grupo de coisas com as mesmas características}
  \synonymref{之中}{zhi1zhong1}
  \end{Phonetics}
\end{Entry}

\begin{Entry}{之下}{3,3}{⼂,⼀}
  \begin{Phonetics}{之下}{zhi1xia4}[][HSK 5]
    \definition{s.}{usado para indicar algo abaixo de um determinado intervalo, posição, grau, etc.; indica um aspecto inferior em termos de alcance, posição, status, nível, Chengdu, etc. | usado para indicar as condições sob as quais algo acontece | usado para indicar o humor, estado em que alguém faz algo; expressa um determinado comportamento em um determinado estado de espírito ou situação}
  \synonymref{以下}{yi3xia4}
  \end{Phonetics}
\end{Entry}

\begin{Entry}{之中}{3,4}{⼂,⼁}
  \begin{Phonetics}{之中}{zhi1zhong1}[][HSK 5]
    \definition{prep.}{em; no meio de; entre}
  \synonymref{之一}{zhi1yi1}
  \end{Phonetics}
\end{Entry}

\begin{Entry}{之内}{3,4}{⼂,⼌}
  \begin{Phonetics}{之内}{zhi1nei4}[][HSK 5]
    \definition{adv.}{em; dentro de; indica dentro de um determinado intervalo, limite ou período de tempo, etc.}
  \synonymref{以内}{yi3nei4}
  \end{Phonetics}
\end{Entry}

\begin{Entry}{之外}{3,5}{⼂,⼣}
  \begin{Phonetics}{之外}{zhi1wai4}[][HSK 5]
    \definition{adv.}{lado de fora; exceto; além de; além disso; refere"-se a algo que excede um determinado limite}
  \synonymref{除外}{chu2wai4}
  \synonymref{以外}{yi3wai4}
  \end{Phonetics}
\end{Entry}

\begin{Entry}{之后}{3,6}{⼂,⼝}
  \begin{Phonetics}{之后}{zhi1hou4}[][HSK 4]
    \definition{s.}{mais tarde; posteriormente; depois; desde então; para indicar que é depois de um determinado tempo ou de uma determinada coisa, 以后 é usado com frequência na linguagem falada; às vezes, também pode indicar que é depois de um determinado lugar ou local,  后面 é usado com frequência na linguagem falada}
  \seealsoref{后面}{hou4mian5}
  \seealsoref{以后}{yi3hou4}
  \synonymref{后来}{hou4lai2}
  \synonymref{继而}{ji4'er2}
  \synonymref{接着}{jie1zhe5}
  \synonymref{其后}{qi2hou4}
  \synonymref{事后}{shi4hou4}
  \synonymref{以后}{yi3hou4}
  \antonymref{之前}{zhi1qian2}
  \end{Phonetics}
\end{Entry}

\begin{Entry}{之间}{3,7}{⼂,⾨}
  \begin{Phonetics}{之间}{zhi1jian1}[][HSK 4]
    \definition{s.}{(depois de um substantivo) entre; dentro de duas delimitações de tempo, local ou quantitativas | colocado após certos verbos ou advérbios de duas sílabas para indicar um curto período de tempo}
  \end{Phonetics}
\end{Entry}

\begin{Entry}{之前}{3,9}{⼂,⼑}
  \begin{Phonetics}{之前}{zhi1qian2}[][HSK 4]
    \definition{adv.}{(referindo"-se ao tempo) antes, antes de, atrás | (referindo"-se ao local físico) na frente de | (usado independentemente) no passado, antes disso}
  \synonymref{曾经}{ceng2jing1}
  \synonymref{此前}{ci3qian2}
  \synonymref{以前}{yi3qian2}
  \antonymref{正在}{zheng4zai4}
  \antonymref{之后}{zhi1hou4}
  \end{Phonetics}
\end{Entry}

\begin{Entry}{之类}{3,9}{⼂,⽶}
  \begin{Phonetics}{之类}{zhi1lei4}[][HSK 6]
    \definition{s.}{usado para dar exemplos (coisas do tipo, desse tipo, assim); uma categoria de pessoas ou coisas que compartilham as mesmas características das pessoas ou coisas mencionadas anteriormente}[我喜欢香蕉、苹果之类的水果。===Eu gosto de frutas como bananas e maçãs.]
  \end{Phonetics}
\end{Entry}

%%%%%%%%%% 乞 %%%%%%%%%%
\subsection*{乞}\addcontentsline{loh}{figure}{乞}

\begin{Entry}{乞}{3}{⼄}
  \begin{Phonetics}{乞}{qi3}
    \definition*{s.}{Sobrenome: Qi}
    \definition{v.}{implorar (por esmolas, etc.); suplicar}
  \end{Phonetics}
\end{Entry}

\begin{Entry}{乞丐}{3,4}{⼄,⼀}
  \begin{Phonetics}{乞丐}{qi3gai4}[][HSK 7-9]
    \definition[个,位,群]{s.}{mendigo; pessoas que não têm meios de subsistência e dependem exclusivamente da mendicância para conseguir comida e dinheiro para sobreviver}
  \synonymref{乞讨}{qi3tao3}
  \antonymref{富人}{fu4ren2}
  \end{Phonetics}
\end{Entry}

\begin{Entry}{乞讨}{3,5}{⼄,⾔}
  \begin{Phonetics}{乞讨}{qi3tao3}[][HSK 7-9]
    \definition{v.}{implorar; pedir dinheiro, pedir comida, etc.}
  \synonymref{乞丐}{qi3gai4}
  \antonymref{恩赐}{en1ci4}
  \end{Phonetics}
\end{Entry}

\begin{Entry}{乞求}{3,7}{⼄,⽔}
  \begin{Phonetics}{乞求}{qi3qiu2}[][HSK 7-9]
    \definition{v.}{implorar; suplicar; mendigar | cair de joelhos}
  \synonymref{哀求}{ai1qiu2}
  \synonymref{恳求}{ken3qiu2}
  \synonymref{请求}{qing3qiu2}
  \antonymref{恩赐}{en1ci4}
  \end{Phonetics}
\end{Entry}

%%%%%%%%%% 也 %%%%%%%%%%
\subsection*{也}\addcontentsline{loh}{figure}{也}

\begin{Entry}{也}{3}{⼄}
  \begin{Phonetics}{也}{ye3}[][HSK 1]
    \definition*{s.}{Sobrenome: Ye}
    \definition{adv.}{também; igualmente; assim como; da mesma forma; usado em frases simples, implica que é igual a outra coisa | assim como (expressar ênfase) | (expressar que as consequências são as mesmas) | também (expressar ufemismo; expressar um tom diplomático) | usado em frases compostas paralelas, indica que duas ou mais coisas têm algo em comum (pode ser usado em todas as frases ou apenas na última frase)}
    \definition{part.}{usado no meio de uma frase, destacando um elemento da frase sobre o qual deve ser feita uma afirmação | usado no final de uma frase, indicando uma explicação ou um julgamento; usado no final da frase, indica tom afirmativo e também pode reforçar o tom interrogativo, exclamativo ou imperativo}
  \synonymref{还}{hai2}
  \synonymref{亦}{yi4}
  \end{Phonetics}
\end{Entry}

\begin{Entry}{也好}{3,6}{⼄,⼥}
  \begin{Phonetics}{也好}{ye3hao3}[][HSK 5]
    \definition{part.}{pode não ser uma má ideia; também pode ser | (reduplicado) se\dots ou\dots; não importa se | pode não ser uma má ideia | se\dots ou\dots; usado em conjunto, significa que não está condicionado a uma determinada situação}
  \end{Phonetics}
\end{Entry}

\begin{Entry}{也有今天}{3,6,4,4}{⼄,⽉,⼈,⼤}
  \begin{Phonetics}{也有今天}{ye3you3jin1tian1}
    \definition{expr.}{obter apenas o que merece | todo cachorro tem seu dia | obter a sua parte (coisas boas ou ruins) | servir alguém bem}
  \end{Phonetics}
\end{Entry}

\begin{Entry}{也许}{3,6}{⼄,⾔}
  \begin{Phonetics}{也许}{ye3xu3}[][HSK 2]
    \definition{adv.}{talvez; provavelmente; estou com medo; para expressar incerteza; para expressar uma alta probabilidade}
  \synonymref{或许}{huo4xu3}
  \synonymref{或者}{huo4zhe3}
  \end{Phonetics}
\end{Entry}

\begin{Entry}{也就是}{3,12,9}{⼄,⼪,⽇}
  \begin{Phonetics}{也就是}{ye3jiu4shi4}
    \definition{adv.}{i.e., isso é | ou seja}
  \end{Phonetics}
\end{Entry}

\begin{Entry}{也就是说}{3,12,9,9}{⼄,⼪,⽇,⾔}
  \begin{Phonetics}{也就是说}{ye3jiu4shi4shuo1}
    \definition{adv.}{em outras palavras | então | isto é | por isso}
  \end{Phonetics}
\end{Entry}

%%%%%%%%%% 习 %%%%%%%%%%
\subsection*{习}\addcontentsline{loh}{figure}{习}

\begin{Entry}{习}{3}{⼄}
  \begin{Phonetics}{习}{xi2}
    \definition*{s.}{Sobrenome: Xi}
    \definition{s.}{hábito; costume; prática usual; um comportamento que se desenvolve inconscientemente por meio de ações repetidas ao longo de um longo período de tempo}
    \definition{v.}{revisar; praticar; exercitar | acostumado a; familiarizado com; familiarizado com algo por meio de contato frequente | estudar; aprender (pássaro)}
  \end{Phonetics}
\end{Entry}

\begin{Entry}{习尚}{3,8}{⼄,⼩}
  \begin{Phonetics}{习尚}{xi2shang4}
    \definition{s.}{prática comum; costume; prática usual}
  \synonymref{风气}{feng1qi4}
  \synonymref{风尚}{feng1shang4}
  \end{Phonetics}
\end{Entry}

\begin{Entry}{习惯}{3,11}{⼄,⼼}
  \begin{Phonetics}{习惯}{xi2guan4}[][HSK 2]
    \definition[个,种]{s.}{hábito; costume; prática usual; comportamentos, tendências ou tendências sociais que se desenvolvem gradualmente ao longo de um longo período de tempo e são difíceis de mudar}
    \definition{v.}{estar acostumado a; ter o hábito de}
  \end{Phonetics}
\end{Entry}

%%%%%%%%%% 乡 %%%%%%%%%%
\subsection*{乡}\addcontentsline{loh}{figure}{乡}

\begin{Entry}{乡}{3}{⼄}
  \begin{Phonetics}{乡}{xiang1}[][HSK 5]
    \definition[个,座,片]{s.}{país; campo; vilarejo; área rural | local de origem; vila ou cidade natal | município (uma unidade administrativa rural subordinada ao condado) | vila natal; cidade natal | terra ou local famoso por produzir algo}
  \antonymref{城}{cheng2}
  \end{Phonetics}
\end{Entry}

\begin{Entry}{乡巴佬}{3,4,8}{⼄,⼰,⼈}
  \begin{Phonetics}{乡巴佬}{xiang1ba1lao3}
    \definition{s.}{aldeão | caipira}
  \end{Phonetics}
\end{Entry}

\begin{Entry}{乡村}{3,7}{⼄,⽊}
  \begin{Phonetics}{乡村}{xiang1cun1}[][HSK 5]
    \definition{adj.}{rural | rústico}
    \definition[个]{s.}{vila; campo; área rural; principalmente envolvido na agricultura; áreas com distribuição populacional mais dispersa em relação às cidades}
  \synonymref{农村}{nong2cun1}
  \antonymref{城市}{cheng2shi4}
  \end{Phonetics}
\end{Entry}

%%%%%%%%%% 于 %%%%%%%%%%
\subsection*{于}\addcontentsline{loh}{figure}{于}

\begin{Entry}{于}{3}{⼆}
  \begin{Phonetics}{于}{yu2}[][HSK 6]
    \definition*{s.}{Sobrenome: Yu}
    \definition{prep.}{indica hora, lugar, alcance, etc. | indica a direção da ação | usado depois de um verbo para indicar dar, entregar, etc. | apresentar a relação do objeto ou entidade introduzida | indica o ponto de início ou de partida | indica comparação | indica passividade}
  \end{Phonetics}
\end{Entry}

\begin{Entry}{于是}{3,9}{⼆,⽇}
  \begin{Phonetics}{于是}{yu2shi4}[][HSK 4]
    \definition{conj.}{então; portanto; consequentemente; como resultado; indica que o último segue o primeiro e que o último é frequentemente causado pelo primeiro}
  \synonymref{所以}{suo3yi3}
  \synonymref{因而}{yin1'er2}
  \synonymref{因此}{yin1ci3}
  \antonymref{但是}{dan4shi4}
  \antonymref{然而}{ran2'er2}
  \end{Phonetics}
\end{Entry}

%%%%%%%%%% 亏 %%%%%%%%%%
\subsection*{亏}\addcontentsline{loh}{figure}{亏}

\begin{Entry}{亏}{3}{⼆}
  \begin{Phonetics}{亏}{kui1}[][HSK 5]
    \definition{adv.}{felizmente; por sorte; graças a | contrariamente, expressando sarcasmo}
    \definition{s.}{prejuízo; perda; déficit | perda; dano; ferida}
    \definition{v.}{perder dinheiro, etc.; ter um déficit; ter prejuízo | ter falta de; ser deficiente; carecer de | tratar injustamente; causar prejuízo; trair a confiança}
  \synonymref{缺}{que1}
  \synonymref{损}{sun3}
  \end{Phonetics}
\end{Entry}

\begin{Entry}{亏本}{3,5}{⼆,⽊}
  \begin{Phonetics}{亏本}{kui1/ben3}[][HSK 7-9]
    \definition{v.+compl.}{estar no vermelho; perder o capital; perder dinheiro (nos negócios)}
  \synonymref{亏损}{kui1sun3}
  \antonymref{赚钱}{zhuan4 qian2}
  \end{Phonetics}
\end{Entry}

\begin{Entry}{亏损}{3,10}{⼆,⼿}
  \begin{Phonetics}{亏损}{kui1sun3}[][HSK 7-9]
    \definition{v.}{perder; esgotar; ter déficit; as despesas excedem as receitas | desidratar; enfraquecer; o corpo está enfraquecido devido a danos ou falta de nutrição}
  \synonymref{不足}{bu4zu2}
  \synonymref{吃亏}{chi1/kui1}
  \synonymref{耗费}{hao4fei4}
  \synonymref{亏本}{kui1/ben3}
  \synonymref{丧失}{sang4shi1}
  \synonymref{损失}{sun3shi1}
  \synonymref{牺牲}{xi1sheng1}
  \end{Phonetics}
\end{Entry}

%%%%%%%%%% 亡 %%%%%%%%%%
\subsection*{亡}\addcontentsline{loh}{figure}{亡}

\begin{Entry}{亡}{3}{⼇}
  \begin{Phonetics}{亡}{wang2}
    \definition{adj.}{falecido}
    \definition{v.}{fugir; escapar | perder; ir embora; jogar fora | morrer; perecer; falecer | conquistar; subjugar | ser destruído; morrer}
  \synonymref{灭}{mie4}
  \synonymref{死}{si3}
  \synonymref{卒}{zu2}
  \antonymref{存}{cun2}
  \antonymref{兴}{xing1}
  \end{Phonetics}
\end{Entry}

\begin{Entry}{亡羊补牢}{3,6,7,7}{⼇,⽺,⾐,⼧}
  \begin{Phonetics}{亡羊补牢}{wang2yang2-bu3lao2}[][HSK 7-9]
    \definition{expr.}{consertar a situação antes que seja tarde demais; agir tardiamente após a ocorrência de um acidente; reparar o curral depois que uma ovelha se perde é uma metáfora para encontrar maneiras de fazer as pazes depois de sofrer uma perda, de modo a evitar sofrer perdas novamente no futuro}
  \end{Phonetics}
\end{Entry}

%%%%%%%%%% 亿 %%%%%%%%%%
\subsection*{亿}\addcontentsline{loh}{figure}{亿}

\begin{Entry}{亿}{3}{⼈}
  \begin{Phonetics}{亿}{yi4}[][HSK 2]
    \definition*{s.}{Sobrenome: Yi}
    \definition{num.}{cem milhões; 100.000.000; 1.0000.0000}
  \end{Phonetics}
\end{Entry}

%%%%%%%%%% 凡 %%%%%%%%%%
\subsection*{凡}\addcontentsline{loh}{figure}{凡}

\begin{Entry}{凡}{3}{⼏}
  \begin{Phonetics}{凡}{fan2}[][HSK 7-9]
    \definition*{s.}{Sobrenome: Fan}
    \definition{adj.}{comum; ordinário}
    \definition{adv.}{Literário: qualquer; todos; todo | Literário: em tudo; completamente}
    \definition{s.}{este mundo mortal; a terra | o mundo secular; refere"-se ao mundo humano | uma nota da escala em Gongchepu (工尺谱), correspondente a 4 na notação musical numerada | Literário: ideia geral; esboço}
  \seealsoref{工尺谱}{gong1 che3 pu3}
  \end{Phonetics}
\end{Entry}

\begin{Entry}{凡是}{3,9}{⼏,⽇}
  \begin{Phonetics}{凡是}{fan2shi4}[][HSK 6]
    \definition{adv.}{todos; qualquer; cada; resumir tudo dentro de um determinado âmbito}
  \end{Phonetics}
\end{Entry}

%%%%%%%%%% 勺 %%%%%%%%%%
\subsection*{勺}\addcontentsline{loh}{figure}{勺}

\begin{Entry}{勺}{3}{⼓}
  \begin{Phonetics}{勺}{shao2}[][HSK 6]
    \definition{clas.}{shao; uma unidade tradicional de volume, igual a 0,01 市升, e equivalente a 1 centilitro ou 0,018 \emph{pint}}
    \definition{s.}{colher; concha}
  \seealsoref{市升}{shi4sheng1}
  \end{Phonetics}
\end{Entry}

%%%%%%%%%% 千 %%%%%%%%%%
\subsection*{千}\addcontentsline{loh}{figure}{千}

\begin{Entry}{千}{3}{⼗}
  \begin{Phonetics}{千}{qian1}[][HSK 2]
    \definition*{s.}{Sobrenome: Qian}
    \definition{num.}{mil; 1.000; 1000 | a grande quantidade de; um grande número de}
  \end{Phonetics}
\end{Entry}

\begin{Entry}{千万}{3,3}{⼗,⼀}
  \begin{Phonetics}{千万}{qian1wan4}[][HSK 3]
    \definition{adv.}{(usado para indicar desejos fortes) por todos os meios; sob quaisquer circunstâncias; expressa uma exortação sincera, equivalente a 务必}
    \definition{num.}{dez milhões; 10.000.000; 1000.0000; milhões e milhões; um número aproximado, indicando um grande número}
  \seealsoref{务必}{wu4bi4}
  \end{Phonetics}
\end{Entry}

\begin{Entry}{千千万万}{3,3,3,3}{⼗,⼗,⼀,⼀}
  \begin{Phonetics}{千千万万}{qian1qian1wan4wan4}
    \definition{expr.}{inumerável; números incontáveis; milhares e milhares}
  \end{Phonetics}
\end{Entry}

\begin{Entry}{千方百计}{3,4,6,4}{⼗,⽅,⽩,⾔}
  \begin{Phonetics}{千方百计}{qian1fang1-bai3ji4}[][HSK 7-9]
    \definition{expr.}{por todos os meios; fazer tudo o que for possível; descreve alguém que esgotou todos os meios ou métodos}
  \end{Phonetics}
\end{Entry}

\begin{Entry}{千古}{3,5}{⼗,⼝}
  \begin{Phonetics}{千古}{qian1gu3}
    \definition{adv.}{por toda a eternidade | em todas as idades}
    \definition{s.}{eternidade (usada em um dístico elegíaco, coroa de flores, etc., dedicada aos mortos)}
  \end{Phonetics}
\end{Entry}

\begin{Entry}{千军万马}{3,6,3,3}{⼗,⼍,⼀,⾺}
  \begin{Phonetics}{千军万马}{qian1jun1-wan4ma3}[][HSK 7-9]
    \definition{expr.}{``Milhares de soldados.''; milhares e milhares de homens e cavalos; um exército poderoso; uma força imensa; todos os cavalos do rei e todos os homens do rei; exército magnífico com milhares de homens e cavalos; demonstração impressionante de força humana}
  \end{Phonetics}
\end{Entry}

\begin{Entry}{千年}{3,6}{⼗,⼲}
  \begin{Phonetics}{千年}{qian1nian2}
    \definition{s.}{milênio}
  \end{Phonetics}
\end{Entry}

\begin{Entry}{千克}{3,7}{⼗,⼗}
  \begin{Phonetics}{千克}{qian1ke4}[][HSK 2]
    \definition{clas.}{kg; quilo; quilograma; 1 quilograma equivale a 1.000 gramas, ou 2 jin (斤)}
  \seealsoref{斤}{jin1}
  \end{Phonetics}
\end{Entry}

\begin{Entry}{千变万化}{3,8,3,4}{⼗,⼜,⼀,⼔}
  \begin{Phonetics}{千变万化}{qian1bian4-wan4hua4}[][HSK 7-9]
    \definition{expr.}{``Sempre em mudança.''; as miríades de mudanças; mudança caleidoscópica; mudanças intermináveis; em constante transformação; ser infinito em variedade; mudanças infinitas; em constante mudança}
  \end{Phonetics}
\end{Entry}

\begin{Entry}{千钧一发}{3,9,1,5}{⼗,⾦,⼀,⼜}
  \begin{Phonetics}{千钧一发}{qian1jun1-yi1fa4}[][HSK 7-9]
    \definition{expr.}{``Por pouco não deu certo.''; cem pesos pendurados por um fio; em perigo iminente; uma questão de vida ou morte}
  \end{Phonetics}
\end{Entry}

\begin{Entry}{千家万户}{3,10,3,4}{⼗,⼧,⼀,⼾}
  \begin{Phonetics}{千家万户}{qian1jia1-wan4hu4}[][HSK 7-9]
    \definition{expr.}{``Milhares de famílias.''; inúmeras famílias; todas as famílias}
  \end{Phonetics}
\end{Entry}

%%%%%%%%%% 卫 %%%%%%%%%%
\subsection*{卫}\addcontentsline{loh}{figure}{卫}

\begin{Entry}{卫}{3}{⼙}
  \begin{Phonetics}{卫}{wei4}
    \definition*{s.}{Wei, um estado da Dinastia Zhou | Sobrenome: Wei}
    \definition{s.}{uma palavra usada no nome do lugar | outro nome para um burro}
    \definition{v.}{defender; guardar; proteger}
  \end{Phonetics}
\end{Entry}

\begin{Entry}{卫生}{3,5}{⼙,⽣}
  \begin{Phonetics}{卫生}{wei4sheng1}[][HSK 3]
    \definition{adj.}{bom para a saúde; higiênico; limpo; capaz de prevenir doenças e benéfico para a saúde}
    \definition{s.}{higiene; saneamento; situação limpa}
  \end{Phonetics}
\end{Entry}

\begin{Entry}{卫生巾}{3,5,3}{⼙,⽣,⼱}
  \begin{Phonetics}{卫生巾}{wei4sheng1jin1}
    \definition{s.}{absorvente higiênico}
  \end{Phonetics}
\end{Entry}

\begin{Entry}{卫生厅}{3,5,4}{⼙,⽣,⼚}
  \begin{Phonetics}{卫生厅}{wei4 sheng1 ting1}
    \definition*{s.}{Departamento de Saúde (da Província)}
  \end{Phonetics}
\end{Entry}

\begin{Entry}{卫生防疫}{3,5,6,9}{⼙,⽣,⾩,⽧}
  \begin{Phonetics}{卫生防疫}{wei4sheng1 fang2yi4}
    \definition{s.}{prevenção contra a epidemia}
  \end{Phonetics}
\end{Entry}

\begin{Entry}{卫生局}{3,5,7}{⼙,⽣,⼫}
  \begin{Phonetics}{卫生局}{wei4sheng1ju2}
    \definition*{s.}{Departamento de Saúde | Escritório de Saúde}
  \end{Phonetics}
\end{Entry}

\begin{Entry}{卫生纸}{3,5,7}{⼙,⽣,⽷}
  \begin{Phonetics}{卫生纸}{wei4sheng1zhi3}
    \definition{s.}{papel higiênico}
  \end{Phonetics}
\end{Entry}

\begin{Entry}{卫生间}{3,5,7}{⼙,⽣,⾨}
  \begin{Phonetics}{卫生间}{wei4sheng1jian1}[][HSK 3]
    \definition[间,个]{s.}{banheiro; sanitário; \emph{toilette}; quartos com instalações sanitárias em hotéis ou residências}
  \end{Phonetics}
\end{Entry}

\begin{Entry}{卫生套}{3,5,10}{⼙,⽣,⼤}
  \begin{Phonetics}{卫生套}{wei4sheng1tao4}
    \definition[只]{s.}{preservativo | camisinha}
  \end{Phonetics}
\end{Entry}

\begin{Entry}{卫生部}{3,5,10}{⼙,⽣,⾢}
  \begin{Phonetics}{卫生部}{wei4sheng1bu4}
    \definition*{s.}{Ministério da Saúde}
  \end{Phonetics}
\end{Entry}

\begin{Entry}{卫生球}{3,5,11}{⼙,⽣,⽟}
  \begin{Phonetics}{卫生球}{wei4sheng1qiu2}
    \definition{s.}{naftalina}
  \end{Phonetics}
\end{Entry}

\begin{Entry}{卫生棉}{3,5,12}{⼙,⽣,⽊}
  \begin{Phonetics}{卫生棉}{wei4sheng1mian2}
    \definition{s.}{absorvente | algodão absorvente esterilizado (usado para curativos ou limpeza de feridas) | absorvente tampão}
  \end{Phonetics}
\end{Entry}

\begin{Entry}{卫生署}{3,5,13}{⼙,⽣,⽹}
  \begin{Phonetics}{卫生署}{wei4sheng1shu3}
    \definition*{s.}{Agência de Saúde (ou Escritório, ou Departamento)}
  \end{Phonetics}
\end{Entry}

\begin{Entry}{卫视}{3,8}{⼙,⾒}
  \begin{Phonetics}{卫视}{wei4shi4}[][HSK 7-9]
    \definition[大,家]{s.}{televisão por satélite; abreviação de 卫星电视}
  \seealsoref{卫星电视}{wei4xing1 dian4shi4}
  \end{Phonetics}
\end{Entry}

\begin{Entry}{卫星}{3,9}{⼙,⽇}
  \begin{Phonetics}{卫星}{wei4xing1}[][HSK 5]
    \definition[个,颗]{s.}{satélite; lua; corpos celestes orbitando planetas | satélite artificial | algo que gira em torno de um centro}
  \end{Phonetics}
\end{Entry}

\begin{Entry}{卫星电视}{3,9,5,8}{⼙,⽇,⽥,⾒}
  \begin{Phonetics}{卫星电视}{wei4xing1 dian4shi4}
    \definition{s.}{TV por satélite; televisão por satélite}
  \end{Phonetics}
\end{Entry}

%%%%%%%%%% 叉 %%%%%%%%%%
\subsection*{叉}\addcontentsline{loh}{figure}{叉}

\begin{Entry}{叉}{3}{⼜}
  \begin{Phonetics}{叉}{cha1}[][HSK 5]
    \definition{s.}{garfo; forquilha | símbolo de cruz, ``×''}
    \definition{v.}{trabalhar com um garfo; garfar; pegar coisas com um garfo}
  \end{Phonetics}
  \begin{Phonetics}{叉}{cha2}
    \definition{v.}{bloquear; emperrar; congestionar}
  \end{Phonetics}
  \begin{Phonetics}{叉}{cha3}
    \definition{v.}{separar de modo a formar uma bifurcação; bifurcar}
  \end{Phonetics}
\end{Entry}

\begin{Entry}{叉子}{3,3}{⼜,⼦}
  \begin{Phonetics}{叉子}{cha1zi5}[][HSK 5]
    \definition[把,个]{s.}{garfo; ferramenta com mais de duas pontas em uma extremidade | tridente; forquilha; ferramentas de agricultura antigas}
  \end{Phonetics}
\end{Entry}

%%%%%%%%%% 及 %%%%%%%%%%
\subsection*{及}\addcontentsline{loh}{figure}{及}

\begin{Entry}{及}{3}{⼃}
  \begin{Phonetics}{及}{ji2}[][HSK 7-9]
    \definition*{s.}{Sobrenome: Ji}
    \definition{conj.}{e; bem como; conectando substantivos paralelos ou frases nominais}
    \definition{v.}{alcançar; chegar até | ser comparável a; alcançar (geralmente usado em termos negativos) | chegar a tempo para | estender-se a; cuidar de; envolver | dar}
  \end{Phonetics}
\end{Entry}

\begin{Entry}{及早}{3,6}{⼃,⽇}
  \begin{Phonetics}{及早}{ji2zao3}[][HSK 7-9]
    \definition{adv.}{o mais cedo possível; antes que seja tarde demais}
  \synonymref{趁早}{chen4zao3}
  \synonymref{赶早}{gan3zao3}
  \end{Phonetics}
\end{Entry}

\begin{Entry}{及时}{3,7}{⼃,⽇}
  \begin{Phonetics}{及时}{ji2shi2}[][HSK 3]
    \definition{adj.}{oportuno; na hora certa; adequado; na ocasião certa}
    \definition{adv.}{prontamente; sem demora; imediatamente}
  \synonymref{随时}{sui2shi2}
  \antonymref{耽误}{dan1wu5}
  \end{Phonetics}
\end{Entry}

\begin{Entry}{及其}{3,8}{⼃,⼋}
  \begin{Phonetics}{及其}{ji2 qi2}[][HSK 7-9]
    \definition{conj.}{(conjunção que liga dois substantivos) e seu\dots.; e seus\dots.; e dele\dots.; e dela\dots; usado para conectar duas ou mais coisas para indicar que elas são de igual importância ou existem da mesma maneira}[文化及其发展影响社会。===A cultura e seu desenvolvimento influenciam a sociedade.]
  \end{Phonetics}
\end{Entry}

\begin{Entry}{及格}{3,10}{⼃,⽊}
  \begin{Phonetics}{及格}{ji2/ge2}[][HSK 4]
    \definition{v.+compl.}{passar; passar em um teste, exame, etc.}
  \synonymref{合格}{he2ge2}
  \end{Phonetics}
\end{Entry}

%%%%%%%%%% 口 %%%%%%%%%%
\subsection*{口}\addcontentsline{loh}{figure}{口}

\begin{Entry}{口}{3}{⼝}[Kangxi 30]
  \begin{Phonetics}{口}{kou3}[][HSK 1]
    \definition*{s.}{Sobrenome: Kou}
    \definition{clas.}{usado para coisas com bocas (pessoas, animais domésticos, canhões, etc.) | usado para mordidas ou bocados | usado para idiomas}
    \definition{s.}{boca | borda; boca; o espaço externo ao recipiente | saída; entrada; local de entrada e saída | o gosto de alguém | corte; buraco; ferida |  a borda de uma faca; lâminas de facas, espadas, tesouras, etc. | a idade de um animal de tração | seção; departamento; sistema integrado de departamentos relacionados | conversa, discurso; pronunciamento; referência à fala | um portão da Grande Muralha (frequentemente usado em nomes de lugares)}
  \end{Phonetics}
\end{Entry}

\begin{Entry}{口子}{3,3}{⼝,⼦}
  \begin{Phonetics}{口子}{kou3zi5}[][HSK 7-9]
    \definition{clas.}{referente a pessoas}[你家有几口子?===Quantas pessoas há na sua família?]
    \definition{s.}{Coloquial: pessoa | marido ou esposa | abertura; buraco; corte; rasgo; uma grande lacuna; uma fenda | Figurativo: abertura; oportunidade}
  \end{Phonetics}
\end{Entry}

\begin{Entry}{口才}{3,3}{⼝,⼿}
  \begin{Phonetics}{口才}{kou3cai2}[][HSK 7-9]
    \definition{s.}{eloquência; persuasão; a capacidade de se expressar verbalmente; o talento para a fala}
  \end{Phonetics}
\end{Entry}

\begin{Entry}{口气}{3,4}{⼝,⽓}
  \begin{Phonetics}{口气}{kou3qi4}[][HSK 7-9]
    \definition{s.}{maneira de falar; a força e o ritmo do tom; a dinâmica do discurso | implicação; o que realmente se quer dizer; o significado não dito nas palavras | tom; nota; tom emocional na fala}
  \end{Phonetics}
\end{Entry}

\begin{Entry}{口水}{3,4}{⼝,⽔}
  \begin{Phonetics}{口水}{kou3shui3}[][HSK 7-9]
    \definition{s.}{saliva; baba; o termo geral para saliva}
  \end{Phonetics}
\end{Entry}

\begin{Entry}{口令}{3,5}{⼝,⼈}
  \begin{Phonetics}{口令}{kou3ling4}[][HSK 7-9]
    \definition[串]{s.}{palavra de comando | senha; palavra-chave; contra-senha | palavra; comando; lema; comando verbal | códigos verbais para identificar amigo ou inimigo}
  \end{Phonetics}
\end{Entry}

\begin{Entry}{口号}{3,5}{⼝,⼝}
  \begin{Phonetics}{口号}{kou3hao4}[][HSK 5]
    \definition[个,条,些]{s.}{\emph{slogan}; palavra de ordem; lema}
  \end{Phonetics}
\end{Entry}

\begin{Entry}{口头}{3,5}{⼝,⼤}
  \begin{Phonetics}{口头}{kou3tou2}[][HSK 7-9]
    \definition{s.}{oral; verbal; expressa-se através da fala | boca (referindo"-se à boca ao falar); lábios}
  \synonymref{表面}{biao3mian4}
  \antonymref{思想}{si1xiang3}
  \antonymref{行动}{xing2dong4}
  \antonymref{行为}{xing2wei2}
  \end{Phonetics}
\end{Entry}

\begin{Entry}{口吃}{3,6}{⼝,⼝}
  \begin{Phonetics}{口吃}{kou3chi1}[][HSK 7-9]
    \definition{s.}{gagueira; espasmofemia; balbucinato; mogilalia; battarismo; battarismo; iscnofonia; pselismo; o fenômeno de repetir palavras ou interromper frases ao falar é um defeito habitual de linguagem comumente conhecido como gagueira}
  \end{Phonetics}
\end{Entry}

\begin{Entry}{口吃病}{3,6,10}{⼝,⼝,⽧}
  \begin{Phonetics}{口吃病}{kou3chi1 bing4}
    \definition{s.}{doença da gagueira}
  \end{Phonetics}
\end{Entry}

\begin{Entry}{口味}{3,8}{⼝,⼝}
  \begin{Phonetics}{口味}{kou3wei4}[][HSK 7-9]
    \definition[个,种]{s.}{sabor da comida; o gosto da comida | o gosto de uma pessoa; preferência de cada um em termos de sabor | gostos; metáfora para interesses e hobbies pessoais}
  \end{Phonetics}
\end{Entry}

\begin{Entry}{口径}{3,8}{⼝,⼻}
  \begin{Phonetics}{口径}{kou3jing4}[][HSK 7-9]
    \definition{s.}{calibre; diâmetro; diâmetro da boca circular do vaso | requisitos; especificações; geralmente se refere às especificações, desempenho, etc., exigidos | declaração; pensamento; linha de ação ou fala; conteúdo metafórico da fala}
  \end{Phonetics}
\end{Entry}

\begin{Entry}{口试}{3,8}{⼝,⾔}
  \begin{Phonetics}{口试}{kou3shi4}[][HSK 6]
    \definition{s.}{exame oral (ou teste); um tipo de exame que exige que os candidatos respondam a perguntas oralmente}
    \definition{v.}{examinar oralmente}
  \antonymref{笔试}{bi3shi4}
  \end{Phonetics}
\end{Entry}

\begin{Entry}{口语}{3,9}{⼝,⾔}
  \begin{Phonetics}{口语}{kou3yu3}[][HSK 4]
    \definition[门]{s.}{linguagem oral; linguagem falada; linguagem coloquial; linguagem usada em conversas}
  \end{Phonetics}
\end{Entry}

\begin{Entry}{口音}{3,9}{⼝,⾳}
  \begin{Phonetics}{口音}{kou3yin1}[][HSK 7-9]
    \definition[种]{s.}{sotaques; sons da fala oral (linguística); os hábitos de fala de uma pessoa, especialmente seus hábitos de pronúncia, podem revelar sua origem linguística}
  \end{Phonetics}
  \begin{Phonetics}{口音}{kou3yin5}
    \definition[种]{s.}{voz | sotaque; dialeto}
  \end{Phonetics}
\end{Entry}

\begin{Entry}{口香糖}{3,9,16}{⼝,⾹,⽶}
  \begin{Phonetics}{口香糖}{kou3xiang1tang2}[][HSK 7-9]
    \definition{s.}{goma de mascar; chiclete; um tipo de doce feito da seiva viscosa que escorre do tronco da sapotilha (uma árvore perene cujos frutos têm o tamanho de peras e formato de coração, daí o nome), juntamente com açúcar e aromatizantes; é para ser mastigado apenas e não deve ser engolido}
  \end{Phonetics}
\end{Entry}

\begin{Entry}{口哨}{3,10}{⼝,⼝}
  \begin{Phonetics}{口哨}{kou3shao4}[][HSK 7-9]
    \definition{s.}{apito; assobio}
    \definition{v.}{assobiar}
  \end{Phonetics}
\end{Entry}

\begin{Entry}{口袋}{3,11}{⼝,⾐}
  \begin{Phonetics}{口袋}{kou3dai5}[][HSK 4]
    \definition[个,只]{s.}{bolso | saco; sacola; artigos de tecido ou couro}
  \end{Phonetics}
\end{Entry}

\begin{Entry}{口袋妖怪}{3,11,7,8}{⼝,⾐,⼥,⼼}
  \begin{Phonetics}{口袋妖怪}{kou3dai4 yao1guai4}
    \definition*{s.}{Pokémon (franquia de mídia japonesa)}
  \end{Phonetics}
\end{Entry}

\begin{Entry}{口腔}{3,12}{⼝,⾁}
  \begin{Phonetics}{口腔}{kou3qiang1}[][HSK 7-9]
    \definition{s.}{cavidade oral; a cavidade oral é um espaço oco composto pelos lábios, bochechas, palato duro e palato mole; contém órgãos como dentes, língua e glândulas salivares}
  \end{Phonetics}
\end{Entry}

\begin{Entry}{口感}{3,13}{⼝,⼼}
  \begin{Phonetics}{口感}{kou3gan3}[][HSK 7-9]
    \definition{s.}{textura (dos alimentos); sabor; sensação que o alimento proporciona na boca}
  \end{Phonetics}
\end{Entry}

\begin{Entry}{口碑}{3,13}{⼝,⽯}
  \begin{Phonetics}{口碑}{kou3bei1}[][HSK 7-9]
    \definition{s.}{elogio público; isso se refere à avaliação verbal que as pessoas fazem de alguém (antigamente, elogios a uma pessoa eram frequentemente gravados em tábuas de pedra)}
  \end{Phonetics}
\end{Entry}

\begin{Entry}{口罩}{3,13}{⼝,⽹}
  \begin{Phonetics}{口罩}{kou3zhao4}[][HSK 7-9]
    \definition[副]{s.}{máscara cirúrgica; produtos de higiene; feitos de gaze, etc.; usados sobre a boca e o nariz para evitar a entrada de poeira e germes}
  \end{Phonetics}
\end{Entry}

%%%%%%%%%% 土 %%%%%%%%%%
\subsection*{土}\addcontentsline{loh}{figure}{土}

\begin{Entry}{土}{3}{⼟}[Kangxi 32]
  \begin{Phonetics}{土}{tu3}[][HSK 3,6]
    \definition*{s.}{Sobrenome: Tu}
    \definition{adj.}{local; nativo; local com características regionais| caseiro; indígena; o que é tradicional no país; popular | não refinado; não esclarecido; não está na moda; não é popular}
    \definition[堆,捧,层]{s.}{solo; terra | terra; território | ópio bruto | cidade natal; terra natal; pátria}
  \end{Phonetics}
\end{Entry}

\begin{Entry}{土生土长}{3,5,3,4}{⼟,⽣,⼟,⾧}
  \begin{Phonetics}{土生土长}{tu3sheng1-tu3zhang3}[][HSK 7-9]
    \definition{expr.}{1. nativo; nascido e criado
Cultivado localmente}
  \end{Phonetics}
\end{Entry}

\begin{Entry}{土地}{3,6}{⼟,⼟}
  \begin{Phonetics}{土地}{tu3di4}[][HSK 4]
    \definition[片,块,顷]{s.}{terra; solo; chão; superfície terrestre da Terra usada para cultivar, construir edifícios e viver | território; território de um país}
  \end{Phonetics}
  \begin{Phonetics}{土地}{tu3di5}
    \definition[片,块,顷]{s.}{deus da audeia; deus local; \emph{genius loci} deidade protetora de um local; Superstição: refere"-se ao deus da terra que governa uma pequena área}
  \end{Phonetics}
\end{Entry}

\begin{Entry}{土豆}{3,7}{⼟,⾖}
  \begin{Phonetics}{土豆}{tu3dou4}[][HSK 5]
    \definition[颗,斤,个,棵]{s.}{batata; denominação comum da batata}
  \end{Phonetics}
\end{Entry}

\begin{Entry}{土豆泥}{3,7,8}{⼟,⾖,⽔}
  \begin{Phonetics}{土豆泥}{tu3dou4ni2}
    \definition{s.}{purê de batata}[她的土豆泥确实不错。===O purê de batatas dela estava realmente muito bom.]
  \end{Phonetics}
\end{Entry}

\begin{Entry}{土鸡}{3,7}{⼟,⿃}
  \begin{Phonetics}{土鸡}{tu3ji1}
    \definition{s.}{galinha caipira}
  \end{Phonetics}
\end{Entry}

\begin{Entry}{土匪}{3,10}{⼟,⼕}
  \begin{Phonetics}{土匪}{tu3fei3}[][HSK 7-9]
    \definition{s.}{bandido; salteador}
  \seealsoref{胡匪}{hu2fei3}
  \synonymref{强盗}{qiang2dao4}
  \end{Phonetics}
\end{Entry}

\begin{Entry}{土壤}{3,20}{⼟,⼟}
  \begin{Phonetics}{土壤}{tu3rang3}[][HSK 7-9]
    \definition{s.}{solo; uma camada solta de material na superfície terrestre, composta por diversos minerais granulares, matéria orgânica, água, ar, microrganismos, etc., que permite o crescimento de plantas}
  \synonymref{泥土}{ni2tu3}
  \end{Phonetics}
\end{Entry}

%%%%%%%%%% 士 %%%%%%%%%%
\subsection*{士}\addcontentsline{loh}{figure}{士}

\begin{Entry}{士}{3}{⼠}[Kangxi 33]
  \begin{Phonetics}{士}{shi4}
    \definition*{s.}{Sobrenome: Shi}
    \definition[位,名,个]{s.}{soldado; militar | oficial não comissionado; primeira classe de soldados | pessoa treinada em uma determinada área; algum tipo de técnico | pessoa (louvável) | bacharel (na China antiga) | classe social, entre os oficiais, 大夫, e o povo comum, 庶民 | estudioso | guarda-costas, uma das peças do xadrez chinês}
  \seealsoref{大夫}{da4fu1}
  \seealsoref{庶民}{shu4min2}
  \end{Phonetics}
\end{Entry}

\begin{Entry}{士气}{3,4}{⼠,⽓}
  \begin{Phonetics}{士气}{shi4qi4}[][HSK 7-9]
    \definition{s.}{moral; o espírito de luta do exército e, de forma mais ampla, o espírito de luta das massas}
  \synonymref{斗志}{dou4zhi4}
  \synonymref{气魄}{qi4po4}
  \synonymref{气势}{qi4shi4}
  \synonymref{勇气}{yong3qi4}
  \end{Phonetics}
\end{Entry}

\begin{Entry}{士兵}{3,7}{⼠,⼋}
  \begin{Phonetics}{士兵}{shi4bing1}[][HSK 4]
    \definition[个,名,位,批,群]{s.}{soldado; militar; termo coletivo para oficiais não comissionados e soldados; os membros mais jovens do exército}
  \end{Phonetics}
\end{Entry}

%%%%%%%%%% 夕 %%%%%%%%%%
\subsection*{夕}\addcontentsline{loh}{figure}{夕}

\begin{Entry}{夕}{3}{⼣}[Kangxi 36]
  \begin{Phonetics}{夕}{xi1}
    \definition*{s.}{Sobrenome: Xi}
    \definition{s.}{pôr do sol; crepúsculo | tarde; noite}
  \end{Phonetics}
\end{Entry}

\begin{Entry}{夕阳}{3,6}{⼣,⾩}
  \begin{Phonetics}{夕阳}{xi1yang2}
    \definition{s.}{pôr do sol}
  \seealsoref{日出}{ri4chu1}
  \end{Phonetics}
\end{Entry}

%%%%%%%%%% 大 %%%%%%%%%%
\subsection*{大}\addcontentsline{loh}{figure}{大}

\begin{Entry}{大}{3}{⼤}[Kangxi 37]
  \begin{Phonetics}{大}{da4}[][HSK 1]
    \definition*{s.}{Sobrenome: Da}
    \definition{adj.}{grande; amplo; grande em volume, área, etc. | mais velho; em primeiro lugar no ranking | tamanho; descreve o grau de grandeza | usado em certas épocas do ano, condições climáticas, feriados ou antes de um determinado momento, para enfatizar | o tempo mais distante; há muito tempo}
    \definition{adv.}{grandemente; totalmente; expressa um grau muito profundo | não muito; não frequentemente; usado após 不, indica um grau baixo ou poucas vezes}
    \definition{s.}{adulto; crescido; pessoas idosas | pai | irmão do pai de alguém; tio}
  \seealsoref{不}{bu4}
  \end{Phonetics}
  \begin{Phonetics}{大}{dai4}
    \definition{s.}{usado em 大夫: médico, doutor | usado em 大王: grande rei}
  \seealsoref{大夫}{dai4fu5}
  \seealsoref{大王}{dai4wang5}
  \end{Phonetics}
\end{Entry}

\begin{Entry}{大人}{3,2}{⼤,⼈}
  \begin{Phonetics}{大人}{da4ren2}[][HSK 2]
    \definition[个,位]{s.}{senhor; ilustre; sua excelência; antigo título honorífico para funcionários públicos | adulto; crescido; maduro;}
  \end{Phonetics}
\end{Entry}

\begin{Entry}{大力}{3,2}{⼤,⼒}
  \begin{Phonetics}{大力}{da4li4}[][HSK 6]
    \definition{adv.}{energicamente; vigorosamente; indica uso de grande força}
    \definition{s.}{grande força, poder}
  \end{Phonetics}
\end{Entry}

\begin{Entry}{大于}{3,3}{⼤,⼆}
  \begin{Phonetics}{大于}{da4yu2}[][HSK 5]
    \definition{v.}{ser maior, mais numeroso, mais importante, etc. do que}
  \end{Phonetics}
\end{Entry}

\begin{Entry}{大口}{3,3}{⼤,⼝}
  \begin{Phonetics}{大口}{da4kou3}
    \definition{s.}{grande bocado (de comida, bebida, fumo, etc.)}
  \end{Phonetics}
\end{Entry}

\begin{Entry}{大大}{3,3}{⼤,⼤}
  \begin{Phonetics}{大大}{da4da5}[][HSK 2]
    \definition{adv.}{grandemente; enormemente; enfatizar grande quantidade ou grau profundo}
  \end{Phonetics}
\end{Entry}

\begin{Entry}{大大咧咧}{3,3,9,9}{⼤,⼤,⼝,⼝}
  \begin{Phonetics}{大大咧咧}{da4da5lie1lie1}[][HSK 7-9]
    \definition{expr.}{despreocupado; descuidado; casual}
  \end{Phonetics}
\end{Entry}

\begin{Entry}{大小}{3,3}{⼤,⼩}
  \begin{Phonetics}{大小}{da4xiao3}[][HSK 2]
    \definition{adv.}{no mínimo; grande ou pequeno (geralmente pequeno), significa que ainda pode ser considerado}
    \definition[家]{s.}{tamanho; o grau de tamanho | ordem de senioridade; hierarquia | adultos e crianças | grande ou pequeno}
  \end{Phonetics}
\end{Entry}

\begin{Entry}{大门}{3,3}{⼤,⾨}
  \begin{Phonetics}{大门}{da4men2}[][HSK 2]
    \definition{s.}{portão; entrada; portão grande, referindo"-se especificamente ao portão principal de um edifício (como uma casa, pátio ou parque) que dá para a rua (em contraste com o segundo portão e as portas das várias divisões)}
  \end{Phonetics}
\end{Entry}

\begin{Entry}{大马}{3,3}{⼤,⾺}
  \begin{Phonetics}{大马}{da4ma3}
    \definition*{s.}{Malásia}
  \end{Phonetics}
\end{Entry}

\begin{Entry}{大公无私}{3,4,4,7}{⼤,⼋,⽆,⽲}
  \begin{Phonetics}{大公无私}{da4gong1-wu2si1}[][HSK 7-9]
    \definition{expr.}{altruísta; generoso; desinteressado | perfeitamente imparcial | nenhuma consideração pessoal (egoísta); não pensar em si mesmo; justo e altruísta; grande imparcialidade exclui consideração de si mesmo; desinteresse e altruísmo}
  \end{Phonetics}
\end{Entry}

\begin{Entry}{大厅}{3,4}{⼤,⼚}
  \begin{Phonetics}{大厅}{da4ting1}[][HSK 5]
    \definition{s.}{\emph{hall}; saguão, uma sala grande para reuniões ou atividades em um edifício de grande porte}
  \end{Phonetics}
\end{Entry}

\begin{Entry}{大夫}{3,4}{⼤,⼤}
  \begin{Phonetics}{大夫}{da4fu1}
    \definition[个,位,名]{s.}{oficial sênior (na China Imperial)}
  \end{Phonetics}
  \begin{Phonetics}{大夫}{dai4fu5}[][HSK 3]
    \definition[个,位,名]{s.}{médico, doutor}
  \end{Phonetics}
\end{Entry}

\begin{Entry}{大巴}{3,4}{⼤,⼰}
  \begin{Phonetics}{大巴}{da4ba1}[][HSK 4]
    \definition{s.}{ônibus}
  \end{Phonetics}
\end{Entry}

\begin{Entry}{大方}{3,4}{⼤,⽅}
  \begin{Phonetics}{大方}{da4fang1}
    \definition{s.}{generosidades; liberalidades | estudioso; pessoas com conhecimento especializado | um tipo de chá verde, produzido principalmente no Condado de Shexian, Província de Anhui, Condado de Chun'an, Província de Zhejiang, etc.}
  \end{Phonetics}
  \begin{Phonetics}{大方}{da4fang5}[][HSK 4]
    \definition{adj.}{generoso | não afetado; natural e equilibrado |  de bom gosto}
  \end{Phonetics}
\end{Entry}

\begin{Entry}{大气}{3,4}{⼤,⽓}
  \begin{Phonetics}{大气}{da4qi4}[][HSK 7-9]
    \definition{adj.}{generoso; magnânimo; refere"-se a um porte extraordinário, não convencional}
    \definition{s.}{ar; atmosfera | respiração pesada}
  \end{Phonetics}
\end{Entry}

\begin{Entry}{大片}{3,4}{⼤,⽚}
  \begin{Phonetics}{大片}{da4pian4}[][HSK 7-9]
    \definition{adj.}{grande; enorme; imenso}
    \definition{s.}{um filme de grande orçamento; um sucesso de bilheteria; \emph{block-buster}; refere"-se a um filme caro e bem feito | grande área; vasta extensão; ampla extensão}
  \end{Phonetics}
\end{Entry}

\begin{Entry}{大王}{3,4}{⼤,⽟}
  \begin{Phonetics}{大王}{da4wang2}
    \definition{s.}{rei; magnata | pessoa da mais alta classe ou habilidade em algo; ás | barões | pessoa com habilidade especializada em algo}
  \end{Phonetics}
  \begin{Phonetics}{大王}{dai4wang5}
    \definition{s.}{magnata; barões | barão ladrão (em ópera, histórias antigas)}
  \end{Phonetics}
\end{Entry}

\begin{Entry}{大队}{3,4}{⼤,⾩}
  \begin{Phonetics}{大队}{da4dui4}[][HSK 7-9]
    \definition{adj.}{grande (contingente de tropas, corpo de manifestantes, número de pessoas, etc.)}
    \definition{s.}{unidade militar correspondente ao batalhão ou regimento; grupo | História: brigada de produção (de uma comuna popular rural); brigada}
  \end{Phonetics}
\end{Entry}

\begin{Entry}{大包大揽}{3,5,3,12}{⼤,⼓,⼤,⼿}
  \begin{Phonetics}{大包大揽}{da4bao1-da4lan3}[][HSK 7-9]
    \definition{v.}{assumir o controle total}
  \end{Phonetics}
\end{Entry}

\begin{Entry}{大众}{3,6}{⼤,⼈}
  \begin{Phonetics}{大众}{da4zhong4}[][HSK 4]
    \definition{s.}{massas; população; pessoas comuns; público em geral}
  \end{Phonetics}
\end{Entry}

\begin{Entry}{大伙儿}{3,6,2}{⼤,⼈,⼉}
  \begin{Phonetics}{大伙儿}{da4huo3r5}[][HSK 5]
    \definition{pron.}{todos nós; todos vocês; todo mundo; todos; equivalente a 大家}
  \seealsoref{大家}{da4jia1}
  \end{Phonetics}
\end{Entry}

\begin{Entry}{大会}{3,6}{⼤,⼈}
  \begin{Phonetics}{大会}{da4hui4}[][HSK 4]
    \definition[场,次,个,届]{s.}{sessão plenária; reunião geral de membros; reuniões convocadas por partidos políticos socialistas | reunião de massa; comício de massa}
  \end{Phonetics}
\end{Entry}

\begin{Entry}{大全}{3,6}{⼤,⼊}
  \begin{Phonetics}{大全}{da4quan2}
    \definition{s.}{coleção abrangente}
  \end{Phonetics}
\end{Entry}

\begin{Entry}{大吃一惊}{3,6,1,11}{⼤,⼝,⼀,⼼}
  \begin{Phonetics}{大吃一惊}{da4chi1-yi1jing1}[][HSK 7-9]
    \definition{expr.}{ficar assustado com; ficar espantado com; ficar completamente surpreso; ficar muito surpreso; ficar completamente chocado; ser pego de surpresa; levar um choque; ter (levar) um susto; levar um susto com; ficar pasmo}
  \end{Phonetics}
\end{Entry}

\begin{Entry}{大同小异}{3,6,3,6}{⼤,⼝,⼩,⼶}
  \begin{Phonetics}{大同小异}{da4tong2-xiao3yi4}[][HSK 7-9]
    \definition{expr.}{muito parecidos, mas com pequenas diferenças; semelhante, exceto por pequenas diferenças; ser o mesmo em aspectos essenciais, embora diferindo em pontos menores; uma semelhança geral com pequenas (menores) diferenças; ser semelhante em todos os aspectos essenciais importantes, sendo as diferenças de natureza menor; ser em grande parte idêntico com apenas pequenas diferenças; diferir apenas em (em) pequenos pontos; essencialmente o mesmo, embora diferindo em pontos menores; na maior parte, eles são os mesmos; idênticos em questões importantes, embora com pequenas diferenças; principalmente semelhantes, exceto (por) pequenas diferenças; virtualmente os mesmos}
  \end{Phonetics}
\end{Entry}

\begin{Entry}{大名鼎鼎}{3,6,12,12}{⼤,⼝,⿍,⿍}
  \begin{Phonetics}{大名鼎鼎}{da4ming2-ding3ding3}[][HSK 7-9]
    \definition{expr.}{``O nome de alguém é conhecido em toda parte.''; ser muito famoso; ser amplamente conhecido; celebrado; desfrutar de um grande nome; bem conhecido}
  \end{Phonetics}
\end{Entry}

\begin{Entry}{大后天}{3,6,4}{⼤,⼝,⼤}
  \begin{Phonetics}{大后天}{da4 hou4 tian1}
    \definition{s.}{daqui a três dias}
  \end{Phonetics}
\end{Entry}

\begin{Entry}{大地}{3,6}{⼤,⼟}
  \begin{Phonetics}{大地}{da4di4}[][HSK 7-9]
    \definition*{s.}{A Terra}[大地正在遭受污染。===A Terra está sendo poluída.]
    \definition[片,块,方]{s.}{terreno espaçoso; terra}
  \end{Phonetics}
\end{Entry}

\begin{Entry}{大多}{3,6}{⼤,⼣}
  \begin{Phonetics}{大多}{da4duo1}[][HSK 4]
    \definition{adv.}{majoritariamente; em sua maior parte; em sua maioria; em grande parte}
  \end{Phonetics}
\end{Entry}

\begin{Entry}{大多数}{3,6,13}{⼤,⼣,⽁}
  \begin{Phonetics}{大多数}{da4duo1shu4}[][HSK 2]
    \definition{s.}{grande maioria; vasta maioria; a maior parte; mais da metade, um número significativo}
  \end{Phonetics}
\end{Entry}

\begin{Entry}{大妈}{3,6}{⼤,⼥}
  \begin{Phonetics}{大妈}{da4ma1}[][HSK 4]
    \definition[个,位]{s.}{tia; esposa do irmão mais velho do pai | tratamento respeitoso às mulheres idosas}
  \end{Phonetics}
\end{Entry}

\begin{Entry}{大师}{3,6}{⼤,⼱}
  \begin{Phonetics}{大师}{da4shi1}[][HSK 6]
    \definition*{s.}{Grande Mestre, título de cortesia usado para se dirigir a um monge budista}
    \definition{s.}{grande mestre; mestre; maestro; uma pessoa com realizações profundas}
  \end{Phonetics}
\end{Entry}

\begin{Entry}{大戏}{3,6}{⼤,⼽}
  \begin{Phonetics}{大戏}{da4xi4}
    \definition*{s.}{Drama, Ópera Chinesa}
  \end{Phonetics}
\end{Entry}

\begin{Entry}{大有可为}{3,6,5,4}{⼤,⽉,⼝,⼂}
  \begin{Phonetics}{大有可为}{da4you3-ke3wei2}[][HSK 7-9]
    \definition{expr.}{ter um futuro brilhante; valer a pena; poder realizar grandes coisas; ter perspectivas brilhantes; pode ser bem administrado; há amplo escopo para nossas habilidades em\dots; valer a pena fazer; ter perspectivas brilhantes}
  \end{Phonetics}
\end{Entry}

\begin{Entry}{大爷}{3,6}{⼤,⽗}
  \begin{Phonetics}{大爷}{da4ye2}
    \definition[个,位]{s.}{Coloquial: irmão mais velho do pai; tio | tratamento respeitoso para um homem mais velho}
  \end{Phonetics}
  \begin{Phonetics}{大爷}{da4ye5}[][HSK 4]
    \definition[个,位]{s.}{irmão mais velho do pai; tio | tio (homenagem aos homens mais velhos)}
  \end{Phonetics}
\end{Entry}

\begin{Entry}{大米}{3,6}{⼤,⽶}
  \begin{Phonetics}{大米}{da4mi3}[][HSK 6]
    \definition[颗,粒,斤,包,袋]{s.}{arroz; arroz descascado; arroz bom}
  \end{Phonetics}
\end{Entry}

\begin{Entry}{大约}{3,6}{⼤,⽷}
  \begin{Phonetics}{大约}{da4yue1}[][HSK 3]
    \definition{adv.}{aproximadamente; sobre; estimativa não muito precisa | provavelmente; expressar suposições sobre a situação}
  \end{Phonetics}
\end{Entry}

\begin{Entry}{大臣}{3,6}{⼤,⾂}
  \begin{Phonetics}{大臣}{da4chen2}[][HSK 7-9]
    \definition[位,个]{s.}{secretário; ministro (de uma monarquia); altos funcionários da monarquia}
  \end{Phonetics}
\end{Entry}

\begin{Entry}{大自然}{3,6,12}{⼤,⾃,⽕}
  \begin{Phonetics}{大自然}{da4zi4ran2}[][HSK 2]
    \definition{s.}{natureza}
  \end{Phonetics}
\end{Entry}

\begin{Entry}{大衣}{3,6}{⼤,⾐}
  \begin{Phonetics}{大衣}{da4yi1}[][HSK 2]
    \definition[件,个]{s.}{sobretudo; casaco; casaco ocidental mais comprido}
  \end{Phonetics}
\end{Entry}

\begin{Entry}{大伯子}{3,7,3}{⼤,⼈,⼦}
  \begin{Phonetics}{大伯子}{da4 bai3zi5}
    \definition{s.}{Coloquial: irmão mais velho do marido; cunhado}
  \end{Phonetics}
\end{Entry}

\begin{Entry}{大体}{3,7}{⼤,⼈}
  \begin{Phonetics}{大体}{da4ti3}[][HSK 7-9]
    \definition{adv.}{aproximadamente; em geral; mais ou menos}
    \definition{s.}{princípio primordial; interesse geral; um princípio basilar baseado na situação geral; um princípio relativo à situação geral}
  \end{Phonetics}
\end{Entry}

\begin{Entry}{大体上}{3,7,3}{⼤,⼈,⼀}
  \begin{Phonetics}{大体上}{da4ti3 shang4}[][HSK 7-9]
    \definition{adv.}{em geral; grosso modo; de um modo geral}
  \end{Phonetics}
\end{Entry}

\begin{Entry}{大声}{3,7}{⼤,⼠}
  \begin{Phonetics}{大声}{da4sheng1}[][HSK 2]
    \definition{adj.}{alto; volume alto; em voz alta}
  \end{Phonetics}
\end{Entry}

\begin{Entry}{大局}{3,7}{⼤,⼫}
  \begin{Phonetics}{大局}{da4ju2}[][HSK 7-9]
    \definition{s.}{situação geral (ou inteira); toda a situação}
  \end{Phonetics}
\end{Entry}

\begin{Entry}{大批}{3,7}{⼤,⼿}
  \begin{Phonetics}{大批}{da4pi1}[][HSK 6]
    \definition{num.}{grandes quantidades de; exércitos; inundações}[大批书籍被印刷出来。===Grandes quantidades de livros foram impressas.]
  \end{Phonetics}
\end{Entry}

\begin{Entry}{大纲}{3,7}{⼤,⽷}
  \begin{Phonetics}{大纲}{da4gang1}[][HSK 5]
    \definition{s.}{esboço; compêndio; programa de estudos; resumo; fundamentos da organização sistemática de conteúdos (livros, discursos, programas, etc.)}
  \end{Phonetics}
\end{Entry}

\begin{Entry}{大豆}{3,7}{⼤,⾖}
  \begin{Phonetics}{大豆}{da4dou4}
    \definition{s.}{soja}
  \end{Phonetics}
\end{Entry}

\begin{Entry}{大陆}{3,7}{⼤,⾩}
  \begin{Phonetics}{大陆}{da4lu4}[][HSK 4]
    \definition*{s.}{China continental; refere"-se especificamente à vasta porção terrestre do território da China}
    \definition[个,块]{s.}{terra firme; continente; vasta extensão de terra}
  \end{Phonetics}
\end{Entry}

\begin{Entry}{大事}{3,8}{⼤,⼅}
  \begin{Phonetics}{大事}{da4shi4}[][HSK 5]
    \definition{adv.}{em grande escala; em grande estilo; em grande parte}
    \definition[件,桩]{s.}{grande evento; grande acontecimento; assunto importante; grande questão; algo importante | situação geral}
  \end{Phonetics}
\end{Entry}

\begin{Entry}{大使}{3,8}{⼤,⼈}
  \begin{Phonetics}{大使}{da4shi3}[][HSK 6]
    \definition[位,任]{s.}{embaixador; o representante diplomático de mais alto nível enviado por um país a outro país}
  \seealsoref{全称特命全权大使}{quan2cheng1 te4ming4 quan2quan2 da4shi3}
  \end{Phonetics}
\end{Entry}

\begin{Entry}{大使馆}{3,8,11}{⼤,⼈,⾷}
  \begin{Phonetics}{大使馆}{da4shi3guan3}[][HSK 3]
    \definition[座,个]{s.}{embaixada; uma representação diplomática de um país em outro país, chefiada por um embaixador}
  \end{Phonetics}
\end{Entry}

\begin{Entry}{大姐}{3,8}{⼤,⼥}
  \begin{Phonetics}{大姐}{da4jie3}[][HSK 4]
    \definition[个,位]{s.}{irmã mais velha (também um termo educado para se dirigir a uma garota ou mulher um pouco mais velha do que a pessoa que fala)}
  \end{Phonetics}
\end{Entry}

\begin{Entry}{大学}{3,8}{⼤,⼦}
  \begin{Phonetics}{大学}{da4xue2}[][HSK 1]
    \definition[所,座]{s.}{universidade; faculdade; tipo de instituição de ensino superior que, na China, geralmente se refere a uma universidade abrangente}
  \end{Phonetics}
\end{Entry}

\begin{Entry}{大学生}{3,8,5}{⼤,⼦,⽣}
  \begin{Phonetics}{大学生}{da4xue2sheng1}[][HSK 1]
    \definition[名,个]{s.}{estudante universitário; estudante de faculdade; estudantes de graduação ou cursos técnicos em instituições de ensino superior}
  \end{Phonetics}
\end{Entry}

\begin{Entry}{大宗}{3,8}{⼤,⼧}
  \begin{Phonetics}{大宗}{da4zong1}[][HSK 7-9]
    \definition{adj.}{grande; volumoso}
    \definition{s.}{itens principais; produtos ou bens básicos}
  \end{Phonetics}
\end{Entry}

\begin{Entry}{大抵}{3,8}{⼤,⼿}
  \begin{Phonetics}{大抵}{da4di3}
    \definition{adv.}{no geral; de um modo geral; provavelmente; principalmente}
  \end{Phonetics}
\end{Entry}

\begin{Entry}{大规模}{3,8,14}{⼤,⾒,⽊}
  \begin{Phonetics}{大规模}{da4gui1mo2}[][HSK 4]
    \definition{adj.}{em larga escala; extensivo; maciço; massivo}
    \definition{adv.}{em larga escala; extensivo; maciço; massa}
  \end{Phonetics}
\end{Entry}

\begin{Entry}{大雨}{3,8}{⼤,⾬}
  \begin{Phonetics}{大雨}{da4yu3}
    \definition[场]{s.}{chuva pesada, forte}
  \end{Phonetics}
\end{Entry}

\begin{Entry}{大前天}{3,9,4}{⼤,⼑,⼤}
  \begin{Phonetics}{大前天}{da4qian2tian1}
    \definition{adv.}{três dias atrás}
  \end{Phonetics}
\end{Entry}

\begin{Entry}{大型}{3,9}{⼤,⼟}
  \begin{Phonetics}{大型}{da4xing2}[][HSK 4]
    \definition{adj.}{grande; em larga escala; tamanho e volume grandes | larga escala (importante e influente)}
  \end{Phonetics}
\end{Entry}

\begin{Entry}{大城}{3,9}{⼤,⼟}
  \begin{Phonetics}{大城}{da4cheng2}
    \definition*[个,座]{s.}{Condado de Dacheng em Langfang 廊坊, Hebei | Município de Tacheng no condado de Changhua | Condado de Changhua, Taiwan}
  \seealsoref{廊坊}{lang2fang2}
  \end{Phonetics}
\end{Entry}

\begin{Entry}{大奖赛}{3,9,14}{⼤,⼤,⾙}
  \begin{Phonetics}{大奖赛}{da4jiang3sai4}[][HSK 5]
    \definition{s.}{grande competição; grande prêmio; \emph{grand prix}}
  \end{Phonetics}
\end{Entry}

\begin{Entry}{大姨}{3,9}{⼤,⼥}
  \begin{Phonetics}{大姨}{da4yi2}
    \definition{s.}{irmã mais velha da mãe; tia | Coloquial: irmã mais velha da esposa | cunhada}
  \seealsoref{大姨儿}{da4yi2r5}
  \synonymref{阿姨}{a1yi2}
  \end{Phonetics}
\end{Entry}

\begin{Entry}{大姨儿}{3,9,2}{⼤,⼥,⼉}
  \begin{Phonetics}{大姨儿}{da4yi2r5}
    \definition{s.}{tia}
  \end{Phonetics}
\end{Entry}

\begin{Entry}{大战}{3,9}{⼤,⼽}
  \begin{Phonetics}{大战}{da4zhan4}
    \definition{s.}{guerra}
    \definition{v.}{guerrear | lutar em uma guerra}
  \end{Phonetics}
\end{Entry}

\begin{Entry}{大洋洲}{3,9,9}{⼤,⽔,⽔}
  \begin{Phonetics}{大洋洲}{da4yang2zhou1}
    \definition*{s.}{Oceania}
  \end{Phonetics}
\end{Entry}

\begin{Entry}{大神}{3,9}{⼤,⽰}
  \begin{Phonetics}{大神}{da4shen2}
    \definition{s.}{deidade | (gíria da Internet) guru | \emph{expert} | gênio}
  \end{Phonetics}
\end{Entry}

\begin{Entry}{大胆}{3,9}{⼤,⾁}
  \begin{Phonetics}{大胆}{da4dan3}[][HSK 5]
    \definition{adj.}{ousado; atrevido; audacioso; corajoso; destemido}
  \end{Phonetics}
\end{Entry}

\begin{Entry}{大选}{3,9}{⼤,⾡}
  \begin{Phonetics}{大选}{da4xuan3}[][HSK 7-9]
    \definition[次]{s.}{eleição geral; refere"-se à eleição de membros do parlamento ou presidente em alguns países}
  \end{Phonetics}
\end{Entry}

\begin{Entry}{大面积}{3,9,10}{⼤,⾯,⽲}
  \begin{Phonetics}{大面积}{da4 mian4ji1}[][HSK 7-9]
    \definition{s.}{grande área}
  \end{Phonetics}
\end{Entry}

\begin{Entry}{大哥}{3,10}{⼤,⼝}
  \begin{Phonetics}{大哥}{da4ge1}[][HSK 4]
    \definition{s.}{irmão mais velho | \emph{big brother}; tratamento educado para um homem da mesma idade que você | líder de gangue; pessoa mais poderosa em uma organização que realiza atividades ilegais na sociedade}
  \end{Phonetics}
\end{Entry}

\begin{Entry}{大娘}{3,10}{⼤,⼥}
  \begin{Phonetics}{大娘}{da4niang2}
    \definition[个,位]{s.}{a esposa do irmão mais velho do pai; tia | tia; termo respeitoso para se referir a uma mulher mais velha}
  \synonymref{阿姨}{a1yi2}
  \synonymref{大妈}{da4ma1}
  \end{Phonetics}
\end{Entry}

\begin{Entry}{大家}{3,10}{⼤,⼧}
  \begin{Phonetics}{大家}{da4jia1}[][HSK 2]
    \definition{pron.}{todos; toda a gente; refere"-se a todas as pessoas dentro de um determinado âmbito}
    \definition{s.}{grande mestre; autoridade; especialista renomado | família nobre; família rica e influente; família tradicional}
  \end{Phonetics}
\end{Entry}

\begin{Entry}{大家庭}{3,10,9}{⼤,⼧,⼴}
  \begin{Phonetics}{大家庭}{da4jia1ting2}[][HSK 7-9]
    \definition[个,些]{s.}{grande família; comunidade | uma grande família é frequentemente usada para descrever um grupo com muitos membros e harmonia interna}
  \end{Phonetics}
\end{Entry}

\begin{Entry}{大海}{3,10}{⼤,⽔}
  \begin{Phonetics}{大海}{da4hai3}[][HSK 2]
    \definition{s.}{o mar; o oceano; o mar aberto, ou seja, a parte do oceano que não está fechada entre cabos nem incluída em estreitos}
  \end{Phonetics}
\end{Entry}

\begin{Entry}{大笔}{3,10}{⼤,⽵}
  \begin{Phonetics}{大笔}{da4bi3}[][HSK 7-9]
    \definition{adj.}{grande quantidade (de dinheiro) | grande soma (de capital, dinheiro, etc.)}
    \definition{s.}{caneta | Cortês: sua escrita; sua caligrafia}
  \end{Phonetics}
\end{Entry}

\begin{Entry}{大脑}{3,10}{⼤,⾁}
  \begin{Phonetics}{大脑}{da4nao3}[][HSK 5]
    \definition{s.}{cérebro; encéfalo}
  \end{Phonetics}
\end{Entry}

\begin{Entry}{大致}{3,10}{⼤,⾄}
  \begin{Phonetics}{大致}{da4zhi4}[][HSK 5]
    \definition{adj.}{geral; no todo}
    \definition{adv.}{grosso modo; aproximadamente; mais ou menos; indica uma estimativa aproximada da situação}
  \end{Phonetics}
\end{Entry}

\begin{Entry}{大部分}{3,10,4}{⼤,⾢,⼑}
  \begin{Phonetics}{大部分}{da4bu4fen5}[][HSK 2]
    \definition[把]{s.}{a maioria; a maior parte; em grande parte; refere"-se a uma quantidade superior a metade do total}
  \end{Phonetics}
\end{Entry}

\begin{Entry}{大都}{3,10}{⼤,⾢}
  \begin{Phonetics}{大都}{da4dou1}
  \end{Phonetics}
  \begin{Phonetics}{大都}{da4du1}[][HSK 5]
    \definition*{s.}{Dadu, capital da China durante a Dinastia Yuan (1280-1368), atual Pequim}
    \definition{adv.}{em sua maior parte; na maior parte; indica que a maioria das pessoas ou coisas em um determinado intervalo tem a mesma natureza e características; também pronunciado como \dpy{da4dou1} na língua falada}
  \end{Phonetics}
\end{Entry}

\begin{Entry}{大惊小怪}{3,11,3,8}{⼤,⼼,⼩,⼼}
  \begin{Phonetics}{大惊小怪}{da4jing1-xiao3guai4}[][HSK 7-9]
    \definition{expr.}{ficar animado com uma coisa pequena; uma tempestade em um copo d'água; ficar surpreso com algo perfeitamente normal; ficar alarmado com algo perfeitamente normal; sentir-se surpreso; ficar alarmado com; ficar todo agitado por nada; grande alarme com um pequeno fantasma; fazer um alarido; fazer um alarido por nada; fazer um grande alarido sobre (algo); fazer um raro alarido sobre (algo); fazer muito alarido; fazer barulho por nada}
  \end{Phonetics}
\end{Entry}

\begin{Entry}{大象}{3,11}{⼤,⾗}
  \begin{Phonetics}{大象}{da4xiang4}[][HSK 5]
    \definition[只,头,群,个]{s.}{elefante}
  \end{Phonetics}
\end{Entry}

\begin{Entry}{大黄}{3,11}{⼤,⿈}
  \begin{Phonetics}{大黄}{da4huang2}
    \definition{s.}{ruibarbo chinês}
  \end{Phonetics}
\end{Entry}

\begin{Entry}{大厦}{3,12}{⼤,⼚}
  \begin{Phonetics}{大厦}{da4sha4}[][HSK 7-9]
    \definition[栋,座,撞]{s.}{prédio; edifício grande (frequentemente usado em nomes de grandes edifícios)}
  \end{Phonetics}
\end{Entry}

\begin{Entry}{大幅度}{3,12,9}{⼤,⼱,⼴}
  \begin{Phonetics}{大幅度}{da4 fu2du4}[][HSK 7-9]
    \definition{adv.}{drasticamente; substancial; por uma ampla margem; a extensão da mudança ou alteração é significativa}
  \end{Phonetics}
\end{Entry}

\begin{Entry}{大棚}{3,12}{⼤,⽊}
  \begin{Phonetics}{大棚}{da4peng2}[][HSK 7-9]
    \definition{s.}{grande abrigo em arco coberto com lona plástica usada na agricultura; grande estufa | estufa}
  \end{Phonetics}
\end{Entry}

\begin{Entry}{大款}{3,12}{⼤,⽋}
  \begin{Phonetics}{大款}{da4kuan3}[][HSK 7-9]
    \definition[个]{s.}{o rico; pessoa muito rica}
  \end{Phonetics}
\end{Entry}

\begin{Entry}{大猩猩}{3,12,12}{⼤,⽝,⽝}
  \begin{Phonetics}{大猩猩}{da4xing1xing5}
    \definition{s.}{gorila}
  \end{Phonetics}
\end{Entry}

\begin{Entry}{大腕儿}{3,12,2}{⼤,⾁,⼉}
  \begin{Phonetics}{大腕儿}{da4wan4r5}[][HSK 7-9]
    \definition[位]{s.}{grande nome, figurão; uma pessoa que tem grande habilidade, fama e influência em um determinado setor ou aspecto}
  \end{Phonetics}
\end{Entry}

\begin{Entry}{大街}{3,12}{⼤,⾏}
  \begin{Phonetics}{大街}{da4jie1}[][HSK 6]
    \definition[条,个]{s.}{avenida; rua; rua principal}
  \end{Phonetics}
\end{Entry}

\begin{Entry}{大街小巷}{3,12,3,9}{⼤,⾏,⼩,⼰}
  \begin{Phonetics}{大街小巷}{da4jie1-xiao3xiang4}[][HSK 7-9]
    \definition{expr.}{grandes ruas e pequenos becos; em todos os lugares da cidade}
  \end{Phonetics}
\end{Entry}

\begin{Entry}{大道}{3,12}{⼤,⾡}
  \begin{Phonetics}{大道}{da4dao4}[][HSK 6]
    \definition*{s.}{O Grande Tao; O Grande Caminho}
    \definition[条]{s.}{estrada principal | o caminho da justiça | avenida | rua principal}
  \end{Phonetics}
\end{Entry}

\begin{Entry}{大量}{3,12}{⼤,⾥}
  \begin{Phonetics}{大量}{da4liang4}[][HSK 2]
    \definition{adj.}{numeroso; em grande quantidade; grande em número ou quantidade | generoso; magnânimo; descreve uma pessoa que não fica zangada quando os outros cometem erros e que costuma perdoar os outros}
  \end{Phonetics}
\end{Entry}

\begin{Entry}{大雁}{3,12}{⼤,⾫}
  \begin{Phonetics}{大雁}{da4yan4}[][HSK 7-9]
    \definition[只,群,行]{s.}{ganso selvagem}
  \end{Phonetics}
\end{Entry}

\begin{Entry}{大意}{3,13}{⼤,⼼}
  \begin{Phonetics}{大意}{da4yi4}[][HSK 7-9]
    \definition{s.}{essência; teor; ideia geral; pontos principais; o significado principal}
  \end{Phonetics}
  \begin{Phonetics}{大意}{da4yi5}[][HSK 7-9]
    \definition{adj.}{descuidado; negligente; desatento}
  \end{Phonetics}
\end{Entry}

\begin{Entry}{大数据}{3,13,11}{⼤,⽁,⼿}
  \begin{Phonetics}{大数据}{da4shu4ju4}[][HSK 7-9]
    \definition*{s.}{\emph{Big Data}}
  \end{Phonetics}
\end{Entry}

\begin{Entry}{大楼}{3,13}{⼤,⽊}
  \begin{Phonetics}{大楼}{da4lou2}[][HSK 4]
    \definition[座,幢]{s.}{edifício; mansão; edifício de vários andares disponível para uso residencial e comercial}
  \end{Phonetics}
\end{Entry}

\begin{Entry}{大概}{3,13}{⼤,⽊}
  \begin{Phonetics}{大概}{da4gai4}[][HSK 3]
    \definition{adj.}{geral; grosseiro; aproximado; não é muito preciso ou muito detalhado}
    \definition{adv.}{sobre; provavelmente; estimativas ou suposições imprecisas sobre eventos, quantidades, tempo, localização, etc. | geralmente; brevemente; não muito seriamente, casualmente; não muito cuidadosamente}
    \definition{s.}{ideia geral; esboço geral; conteúdo geral ou situação}
  \end{Phonetics}
\end{Entry}

\begin{Entry}{大肆}{3,13}{⼤,⾀}
  \begin{Phonetics}{大肆}{da4si4}[][HSK 7-9]
    \definition{adv.}{desenfreadamente; violentamente; imprudentemente; sem restrições; sem escrúpulos (geralmente referindo"-se a fazer coisas ruins)}
  \end{Phonetics}
\end{Entry}

\begin{Entry}{大腿}{3,13}{⼤,⾁}
  \begin{Phonetics}{大腿}{da4tui3}
    \definition{s.}{coxa}
  \end{Phonetics}
\end{Entry}

\begin{Entry}{大蒜}{3,13}{⼤,⾋}
  \begin{Phonetics}{大蒜}{da4suan4}
    \definition[瓣,头]{s.}{alho}
  \end{Phonetics}
\end{Entry}

\begin{Entry}{大模大样}{3,14,3,10}{⼤,⽊,⼤,⽊}
  \begin{Phonetics}{大模大样}{da4mu2-da4yang4}[][HSK 7-9]
    \definition{expr.}{pomposamente; corajosamente; ostensivamente; equilibrado; autoconfiante}
  \end{Phonetics}
\end{Entry}

\begin{Entry}{大熊猫}{3,14,11}{⼤,⽕,⽝}
  \begin{Phonetics}{大熊猫}{da4xiong2mao1}[][HSK 5]
    \definition{s.}{panda gigante}
  \end{Phonetics}
\end{Entry}

\begin{Entry}{大赛}{3,14}{⼤,⾙}
  \begin{Phonetics}{大赛}{da4sai4}[][HSK 6]
    \definition{s.}{grande torneio; competição importante; um evento de grande porte e alto nível; um grande evento}
  \end{Phonetics}
\end{Entry}

%%%%%%%%%% 女 %%%%%%%%%%
\subsection*{女}\addcontentsline{loh}{figure}{女}

\begin{Entry}{女}{3}{⼥}[Kangxi 38]
  \begin{Phonetics}{女}{nv3}[][HSK 1]
    \definition{adj.}{mulher; feminino | fêmea (de certos animais)}
    \definition{s.}{menina; filha | nü, uma das mansões lunares | mulher}
  \antonymref{男}{nan2}
  \end{Phonetics}
\end{Entry}

\begin{Entry}{女人}{3,2}{⼥,⼈}
  \begin{Phonetics}{女人}{nv3ren5}[][HSK 1]
    \definition[个,位]{s.}{mulher adulta}
  \end{Phonetics}
\end{Entry}

\begin{Entry}{女儿}{3,2}{⼥,⼉}
  \begin{Phonetics}{女儿}{nv3'er2}[][HSK 1]
    \definition[个]{s.}{menina; filha}
  \seealsoref{儿子}{er2zi5}
  \end{Phonetics}
\end{Entry}

\begin{Entry}{女士}{3,3}{⼥,⼠}
  \begin{Phonetics}{女士}{nv3shi4}[][HSK 4]
    \definition{pron.}{Sra.; Senhorita; Senhora; título honorífico para mulheres (agora usado em contextos diplomáticos)}
    \definition[位,名,个,些]{s.}{senhora; madame}
  \end{Phonetics}
\end{Entry}

\begin{Entry}{女子}{3,3}{⼥,⼦}
  \begin{Phonetics}{女子}{nv3zi3}[][HSK 3]
    \definition[位,名,个]{s.}{mulher; feminino; pessoa do sexo feminino}
  \end{Phonetics}
\end{Entry}

\begin{Entry}{女王}{3,4}{⼥,⽟}
  \begin{Phonetics}{女王}{nv3wang2}
    \definition{s.}{rainha}
  \end{Phonetics}
\end{Entry}

\begin{Entry}{女生}{3,5}{⼥,⽣}
  \begin{Phonetics}{女生}{nv3sheng1}[][HSK 1]
    \definition[个]{s.}{estudante; aluna; estudante do sexo feminino | menina; jovem mulher}
  \end{Phonetics}
\end{Entry}

\begin{Entry}{女性}{3,8}{⼥,⼼}
  \begin{Phonetics}{女性}{nv3xing4}[][HSK 5]
    \definition[个,位,名]{s.}{mulher; feminino; feminilidade}
  \antonymref{男性}{nan2xing4}
  \end{Phonetics}
\end{Entry}

\begin{Entry}{女朋友}{3,8,4}{⼥,⽉,⼜}
  \begin{Phonetics}{女朋友}{nv3peng2you5}[][HSK 1]
    \definition{s.}{namorada}
  \end{Phonetics}
\end{Entry}

\begin{Entry}{女孩}{3,9}{⼥,⼦}
  \begin{Phonetics}{女孩}{nv3hai2}
    \definition{s.}{menina | garota}
  \end{Phonetics}
\end{Entry}

\begin{Entry}{女孩儿}{3,9,2}{⼥,⼦,⼉}
  \begin{Phonetics}{女孩儿}{nv3hai2r5}[][HSK 1]
    \definition{s.}{garota; menina; atualmente também se refere a mulher adolescente | filha}
  \end{Phonetics}
\end{Entry}

\begin{Entry}{女婿}{3,12}{⼥,⼥}
  \begin{Phonetics}{女婿}{nv3xu5}[][HSK 7-9]
    \definition[个,位]{s.}{genro; marido da filha | em algumas regiões, refere"-se ao marido}
  \end{Phonetics}
\end{Entry}

%%%%%%%%%% 子 %%%%%%%%%%
\subsection*{子}\addcontentsline{loh}{figure}{子}

\begin{Entry}{子}{3}{⼦}[Kangxi 39]
  \begin{Phonetics}{子}{zi3}
    \definition*{s.}{Sobrenome: Zi}
    \definition{adj.}{pequeno; jovem; tenro | subsidiário; subordinado; derivado}
    \definition{clas.}{usado para objetos finos que podem ser pinçados com os dedos}
    \definition{pron.}{você;  antigamente, era uma forma de tratamento respeitosa para se referir a outras pessoas, equivalente a 您}
    \definition[个,位,名]{s.}{filho, criança; antigamente, referia"-se aos filhos, mas atualmente refere"-se especificamente aos filhos homens | pessoa | antigo título de respeito para um homem culto ou virtuoso; na antiguidade, referia"-se especificamente a homens eruditos | visconde; o quarto posto na hierarquia dos cinco títulos feudais da nobreza | ovo | semente | coisas pequenas e duras; pequenos fragmentos ou grãos duros e sólidos | cobre; moeda de cobre | o primeiro dos doze ramos terrestres}
  \seealsoref{您}{nin2}
  \end{Phonetics}
  \begin{Phonetics}{子}{zi5}[][HSK 1]
    \definition{suf.}{sufixo para substantivos | sufixos de palavras de medida individuais; anexado a certas palavras classificadoras}
  \end{Phonetics}
\end{Entry}

\begin{Entry}{子女}{3,3}{⼦,⼥}
  \begin{Phonetics}{子女}{zi3nv3}[][HSK 3]
    \definition[个]{s.}{crianças; descendentes; filhos e filhas}
  \end{Phonetics}
\end{Entry}

\begin{Entry}{子弹}{3,11}{⼦,⼸}
  \begin{Phonetics}{子弹}{zi3dan4}[][HSK 5]
    \definition[粒,颗,发]{s.}{bala; cartucho; munição}
  \end{Phonetics}
\end{Entry}

%%%%%%%%%% 寸 %%%%%%%%%%
\subsection*{寸}\addcontentsline{loh}{figure}{寸}

\begin{Entry}{寸}{3}{⼨}[Kangxi 41]
  \begin{Phonetics}{寸}{cun4}[][HSK 5]
    \definition*{s.}{Sobrenome: Cun}
    \definition{adj.}{muito pouco; muito curto; pequeno | Dialeto: coincidência}
    \definition{clas.}{cun, uma unidade tradicional de comprimento, igual a 0,1 市尺 e equivalente a 3,333 centímetros ou 1,312 polegadas | cun, uma unidade de comprimento (=13 decímetros)}
  \seealsoref{市尺}{shi4 chi3}
  \end{Phonetics}
\end{Entry}

%%%%%%%%%% 小 %%%%%%%%%%
\subsection*{小}\addcontentsline{loh}{figure}{小}

\begin{Entry}{小}{3}{⼩}[Kangxi 42]
  \begin{Phonetics}{小}{xiao3}[][HSK 1,2]
    \definition*{s.}{Sobrenome: Xiao}
    \definition{adj.}{menor; pequeno; insignificante; pouco; volume, área, quantidade, intensidade, etc. não são grandes | jovem | expressões humildes, referindo"-se a si mesmo ou a pessoas ou coisas relacionadas a si mesmo | por um tempo; por um curto período; por um curto período de tempo | o mais novo; o último na ordem de antiguidade; em último lugar na classificação}
    \definition{pref.}{usado antes do sobrenome, nome, posição na família, etc.}
    \definition{s.}{os jovens; pessoas mais jovens | concubina}
  \end{Phonetics}
\end{Entry}

\begin{Entry}{小于}{3,3}{⼩,⼆}
  \begin{Phonetics}{小于}{xiao3yu2}[][HSK 6]
    \definition{prep.}{menor que; menos que; indica que um número ou quantidade é menor que outro}
  \end{Phonetics}
\end{Entry}

\begin{Entry}{小小}{3,3}{⼩,⼩}
  \begin{Phonetics}{小小}{xiao3xiao3}
    \definition{adj.}{muito pequeno}
  \end{Phonetics}
\end{Entry}

\begin{Entry}{小区}{3,4}{⼩,⼖}
  \begin{Phonetics}{小区}{xiao3qu1}
    \definition{s.}{conjunto habitacional, comunidade, bairro | célula (telecomunicações)}
  \end{Phonetics}
\end{Entry}

\begin{Entry}{小心}{3,4}{⼩,⼼}
  \begin{Phonetics}{小心}{xiao3xin5}[][HSK 2]
    \definition{adj.}{cuidadoso; atento; com cautela}
    \definition{v.}{ter cuidado; ser cauteloso; estar atento; tomar cuidado; prestar atenção}
  \end{Phonetics}
\end{Entry}

\begin{Entry}{小气鬼}{3,4,9}{⼩,⽓,⿁}
  \begin{Phonetics}{小气鬼}{xiao3qi4gui3}
    \definition{adj.}{avarento | mão-de-vaca | miserável | pão-duro}
  \end{Phonetics}
\end{Entry}

\begin{Entry}{小白菜}{3,5,11}{⼩,⽩,⾋}
  \begin{Phonetics}{小白菜}{xiao3bai2cai4}
    \definition[棵]{s.}{\emph{bok choy} | couve chinesa}
  \end{Phonetics}
\end{Entry}

\begin{Entry}{小众}{3,6}{⼩,⼈}
  \begin{Phonetics}{小众}{xiao3zhong4}
    \definition{s.}{minoria da população | nicho (mercado, etc.)}
  \end{Phonetics}
\end{Entry}

\begin{Entry}{小伙子}{3,6,3}{⼩,⼈,⼦}
  \begin{Phonetics}{小伙子}{xiao3huo3zi5}[][HSK 4]
    \definition[位]{s.}{rapaz jovem; jovem colega}
  \end{Phonetics}
\end{Entry}

\begin{Entry}{小吃}{3,6}{⼩,⼝}
  \begin{Phonetics}{小吃}{xiao3chi1}[][HSK 4]
    \definition[家]{s.}{lanche; petiscos; comida com especialidades locais, não muito para uma porção | prato frio; prato feito; cortes de frios na culinária ocidental | pratos pequenos e baratos; pratos simples em restaurantes com porções pequenas e preços baixos}
  \end{Phonetics}
\end{Entry}

\begin{Entry}{小声}{3,7}{⼩,⼠}
  \begin{Phonetics}{小声}{xiao3sheng1}[][HSK 2]
    \definition{v.}{falar em voz baixa; falar baixinho; sussurar}
  \end{Phonetics}
\end{Entry}

\begin{Entry}{小时}{3,7}{⼩,⽇}
  \begin{Phonetics}{小时}{xiao3shi2}[][HSK 1]
    \definition{clas.}{hora; unidade de medida legal do tempo, 1 hora equivale a 60 minutos, é 1/24 de um dia}
    \definition[个]{s.}{hora; refere"-se a um período de uma hora}
  \end{Phonetics}
\end{Entry}

\begin{Entry}{小时候}{3,7,10}{⼩,⽇,⼈}
  \begin{Phonetics}{小时候}{xiao3shi2hou5}[][HSK 2]
    \definition{s.}{na infância; quando alguém era jovem; refere"-se à infância}
  \end{Phonetics}
\end{Entry}

\begin{Entry}{小麦}{3,7}{⼩,⿆}
  \begin{Phonetics}{小麦}{xiao3mai4}[][HSK 6]
    \definition[粒,公斤,吨,棵]{s.}{trigo}
  \end{Phonetics}
\end{Entry}

\begin{Entry}{小姐}{3,8}{⼩,⼥}
  \begin{Phonetics}{小姐}{xiao3jie3}[][HSK 1]
    \definition[个,位]{s.}{jovem senhora; anteriormente, era assim que se referiam às filhas de famílias ricas. | senhorita; título honorífico para mulheres jovens | (gíria) prostituta}
  \end{Phonetics}
\end{Entry}

\begin{Entry}{小学}{3,8}{⼩,⼦}
  \begin{Phonetics}{小学}{xiao3xue2}[][HSK 1]
    \definition[个]{s.}{escola primária (ou fundamental); escolas que oferecem ensino fundamental básico | estudos filológicos; antigamente, referia"-se ao estudo da escrita, da fonética e da exegese}
  \end{Phonetics}
\end{Entry}

\begin{Entry}{小学生}{3,8,5}{⼩,⼦,⽣}
  \begin{Phonetics}{小学生}{xiao3 xue2sheng5}[][HSK 1]
    \definition{s.}{aluno; estudante; estudante do sexo masculino (男); estudante do sexo feminino (女) | um aluno mais novo (do que os outros da sua turma) | (dialeto) um menino pequeno}
  \seealsoref{男}{nan2}
  \seealsoref{女}{nv3}
  \end{Phonetics}
\end{Entry}

\begin{Entry}{小朋友}{3,8,4}{⼩,⽉,⼜}
  \begin{Phonetics}{小朋友}{xiao3peng2you3}[][HSK 1]
    \definition[个]{s.}{criança; crianças; refere"-se a crianças e adolescentes | (termo usado por um adulto para se dirigir a uma criança) amiguinho; menino (ou menina); termo carinhoso para se referir a crianças e adolescentes}
  \end{Phonetics}
\end{Entry}

\begin{Entry}{小狗}{3,8}{⼩,⽝}
  \begin{Phonetics}{小狗}{xiao3 gou3}
    \definition{s.}{filhote de cachorro}
  \end{Phonetics}
\end{Entry}

\begin{Entry}{小组}{3,8}{⼩,⽷}
  \begin{Phonetics}{小组}{xiao3zu3}[][HSK 2]
    \definition[个,名,位]{s.}{grupo; um pequeno grupo de pessoas}
  \end{Phonetics}
\end{Entry}

\begin{Entry}{小型}{3,9}{⼩,⼟}
  \begin{Phonetics}{小型}{xiao3xing2}[][HSK 4]
    \definition{adj.}{de tamanho pequeno; em pequena escala; miniatura; tipo pequeno; tamanho de bolso; tipo compacto}
    \definition{s.}{Mediterrâneo: escunas, pequenos veleiros de pesca ou turismo | pequeno \emph{rover} lunar (duas pessoas)}
  \end{Phonetics}
\end{Entry}

\begin{Entry}{小孩儿}{3,9,2}{⼩,⼦,⼉}
  \begin{Phonetics}{小孩儿}{xiao3hai2r5}[][HSK 1]
    \definition[个]{s.}{criança; bebê}
  \end{Phonetics}
\end{Entry}

\begin{Entry}{小屋}{3,9}{⼩,⼫}
  \begin{Phonetics}{小屋}{xiao3wu1}
    \definition{s.}{cabana | chalé | cabine}
  \end{Phonetics}
\end{Entry}

\begin{Entry}{小树}{3,9}{⼩,⽊}
  \begin{Phonetics}{小树}{xiao3shu4}
    \definition[棵]{s.}{muda | arbusto | árvore pequena}
  \end{Phonetics}
\end{Entry}

\begin{Entry}{小洋白菜}{3,9,5,11}{⼩,⽔,⽩,⾋}
  \begin{Phonetics}{小洋白菜}{xiao3 yang2bai2cai4}
    \definition{s.}{couve de bruxelas}
  \end{Phonetics}
\end{Entry}

\begin{Entry}{小说}{3,9}{⼩,⾔}
  \begin{Phonetics}{小说}{xiao3shuo1}[][HSK 2]
    \definition[本,部,篇,章]{s.}{história; romance; ficção; uma forma literária que reflete a vida social por meio da descrição de personagens, ambiente e enredo}
  \end{Phonetics}
\end{Entry}

\begin{Entry}{小费}{3,9}{⼩,⾙}
  \begin{Phonetics}{小费}{xiao3fei4}[][HSK 6]
    \definition[笔]{s.}{gorjeta; gratificação; dinheiro extra pago por clientes e viajantes a funcionários de serviços em setores de serviços, como hotéis e pousadas}
  \end{Phonetics}
\end{Entry}

\begin{Entry}{小偷儿}{3,11,2}{⼩,⼈,⼉}
  \begin{Phonetics}{小偷儿}{xiao3tou1er5}[][HSK 5]
    \definition{s.}{ladrão insignificante (ou furtivo); ladrãozinho | ladrão}
  \end{Phonetics}
\end{Entry}

\begin{Entry}{小腿}{3,13}{⼩,⾁}
  \begin{Phonetics}{小腿}{xiao3tui3}
    \definition{s.}{perna (do joelho ao calcanhar) | haste}
  \end{Phonetics}
\end{Entry}

%%%%%%%%%% 尸 %%%%%%%%%%
\subsection*{尸}\addcontentsline{loh}{figure}{尸}

\begin{Entry}{尸}{3}{⼫}[Kangxi 44]
  \begin{Phonetics}{尸}{shi1}
    \definition*{s.}{Sobrenome: Shi}
    \definition[具]{s.}{cadáver; corpo morto; restos mortais | Arcaico: pessoa que se sentava atrás do altar, representando o falecido durante a realização de ritos sacrificiais}
    \definition{v.}{manter um emprego sem fazer nada (como um cadáver) | dirigir; agir como responsável | dispor}
  \end{Phonetics}
\end{Entry}

\begin{Entry}{尸体}{3,7}{⼫,⼈}
  \begin{Phonetics}{尸体}{shi1ti3}[][HSK 7-9]
    \definition[具,个]{s.}{cadáver; restos mortais; corpo morto; os corpos de humanos e animais após a morte}
  \end{Phonetics}
\end{Entry}

%%%%%%%%%% 山 %%%%%%%%%%
\subsection*{山}\addcontentsline{loh}{figure}{山}

\begin{Entry}{山}{3}{⼭}[Kangxi 46]
  \begin{Phonetics}{山}{shan1}[][HSK 1]
    \definition*{s.}{Sobrenome: Shan}
    \definition[座]{s.}{colina; maciço; montanha | qualquer coisa que se assemelhe a uma montanha | arbustos nos quais os bichos"-da"-seda tecem seus casulos; referindo"-se a casulos de bicho"-da"-seda | eco; metáfora para um som muito alto}
  \end{Phonetics}
\end{Entry}

\begin{Entry}{山川}{3,3}{⼭,⼮}
  \begin{Phonetics}{山川}{shan1chuan1}[][HSK 7-9]
    \definition{s.}{montanhas e rios; paisagem}
  \end{Phonetics}
\end{Entry}

\begin{Entry}{山冈}{3,4}{⼭,⼌}
  \begin{Phonetics}{山冈}{shan1gang1}[][HSK 7-9]
    \definition[座]{s.}{colina baixa; pequeno morro}
  \end{Phonetics}
\end{Entry}

\begin{Entry}{山区}{3,4}{⼭,⼖}
  \begin{Phonetics}{山区}{shan1qu1}[][HSK 5]
    \definition[片]{s.}{área montanhosa; região montanhosa | colina; serra; montanha | distrito montanhoso}
  \end{Phonetics}
\end{Entry}

\begin{Entry}{山东}{3,5}{⼭,⼀}
  \begin{Phonetics}{山东}{shan1dong1}
    \definition*{s.}{Província de Shandong (Shantung) no nordeste da China}
  \end{Phonetics}
\end{Entry}

\begin{Entry}{山羊}{3,6}{⼭,⽺}
  \begin{Phonetics}{山羊}{shan1yang2}
    \definition{s.}{cabra | (ginástica) cavalo de salto de pequeno porte}
  \end{Phonetics}
\end{Entry}

\begin{Entry}{山西}{3,6}{⼭,⾑}
  \begin{Phonetics}{山西}{shan1xi1}
    \definition*{s.}{Província de Shanxi (Shansi) no norte da China entre Hebei e Shaanxi, abreviada como 晋 (Shansi)}
  \seealsoref{晋}{jin4}
  \end{Phonetics}
\end{Entry}

\begin{Entry}{山阴}{3,6}{⼭,⾩}
  \begin{Phonetics}{山阴}{shan1yin1}
    \definition*{s.}{Condado de Shanyin em Shuozhou, Shanxi}
    \definition{s.}{lado norte (ou sombreado) de uma montanha}
  \end{Phonetics}
\end{Entry}

\begin{Entry}{山体}{3,7}{⼭,⼈}
  \begin{Phonetics}{山体}{shan1ti3}
    \definition{s.}{forma de uma montanha}
  \end{Phonetics}
\end{Entry}

\begin{Entry}{山谷}{3,7}{⼭,⾕}
  \begin{Phonetics}{山谷}{shan1gu3}[][HSK 6]
    \definition[条,个]{s.}{vale; desfiladeiro; ravina; a área baixa e estreita entre duas montanhas geralmente tem riachos no meio}
  \end{Phonetics}
\end{Entry}

\begin{Entry}{山坡}{3,8}{⼭,⼟}
  \begin{Phonetics}{山坡}{shan1po1}[][HSK 6]
    \definition[个,座,片]{s.}{encosta; encosta da montanha; a inclinação entre o topo da montanha e o terreno plano}
  \end{Phonetics}
\end{Entry}

\begin{Entry}{山岭}{3,8}{⼭,⼭}
  \begin{Phonetics}{山岭}{shan1ling3}[][HSK 7-9]
    \definition[座,片,条]{s.}{cadeia de montanhas; cordilheira | crista; montanhas altas contínuas}
  \end{Phonetics}
\end{Entry}

\begin{Entry}{山顶}{3,8}{⼭,⾴}
  \begin{Phonetics}{山顶}{shan1ding3}[][HSK 7-9]
    \definition{s.}{topo de uma colina; cume de uma montanha; o pico de uma montanha; o topo da montanha}[他们距山顶还有100米远。===Eles ainda estavam a 100 metros do cume.]
  \end{Phonetics}
\end{Entry}

\begin{Entry}{山峰}{3,10}{⼭,⼭}
  \begin{Phonetics}{山峰}{shan1feng1}[][HSK 6]
    \definition[座,个]{s.}{pico (montanha); topo alto e pontudo da montanha}
  \end{Phonetics}
\end{Entry}

\begin{Entry}{山路}{3,13}{⼭,⾜}
  \begin{Phonetics}{山路}{shan1lu4}[][HSK 7-9]
    \definition{s.}{trilha de montanha | estrada de montanha}
  \end{Phonetics}
\end{Entry}

\begin{Entry}{山寨}{3,14}{⼭,⼧}
  \begin{Phonetics}{山寨}{shan1zhai4}[][HSK 7-9]
    \definition{s.}{fortaleza na montanha (especialmente de bandidos); aldeia fortificada na montanha; vila fortificada na colina | Humorístico: “bandido”; imitador | Figurativo: imitação (de produtos); falsificação}
  \end{Phonetics}
\end{Entry}

%%%%%%%%%% 川 %%%%%%%%%%
\subsection*{川}\addcontentsline{loh}{figure}{川}

\begin{Entry}{川}{3}{⼮}[Kangxi 47]
  \begin{Phonetics}{川}{chuan1}
    \definition*{s.}{Província de Sichuan, abreviação de 四川}
    \definition{s.}{rio; córrego | planície; terra plana}
  \seealsoref{四川}{si4chuan1}
  \end{Phonetics}
\end{Entry}

\begin{Entry}{川流不息}{3,10,4,10}{⼮,⽔,⼀,⼼}
  \begin{Phonetics}{川流不息}{chuan1liu2-bu4xi1}[][HSK 7-9]
    \definition{expr.}{``O fluxo nunca para de fluir.''; um fluxo contínuo (de); vem e vai em um fluxo sem fim; vem e vai o tempo todo; flui continuamente; continua continuamente; flui em um fluxo sem fim; flui sem cessar; flui perpetuamente; em um fluxo sem fim; sem fim; despeja em um fluxo sem fim; ``O fluxo flui incessantemente.''; descreve os pedestres, veículos, etc. indo e vindo tão continuamente quanto um fluxo de água}
  \end{Phonetics}
\end{Entry}

%%%%%%%%%% 工 %%%%%%%%%%
\subsection*{工}\addcontentsline{loh}{figure}{工}

\begin{Entry}{工}{3}{⼯}[Kangxi 48]
  \begin{Phonetics}{工}{gong1}
    \definition*{s.}{Sobrenome: Gong}
    \definition{adj.}{fino; requintado; delicado}
    \definition{s.}{trabalhador; operário; artesão | trabalho; labor; trabalho produtivo | projeto; construção; refere"-se à engenharia | indústria; refere"-se à indústria | homem"-dia; a quantidade de trabalho que um trabalhador faz em um dia | uma nota da escala em Gongchepu (工尺谱), correspondente a 3 na notação musical numerada | engenheiro; refere"-se a engenheiros}
    \definition{v.}{ser versado em; ser bom em | trabalhar em; agora geralmente escrito como 功}
  \seealsoref{功}{gong1}
  \seealsoref{工尺谱}{gong1 che3 pu3}
  \end{Phonetics}
\end{Entry}

\begin{Entry}{工人}{3,2}{⼯,⼈}
  \begin{Phonetics}{工人}{gong1ren5}[][HSK 1]
    \definition[个,名]{s.}{trabalhador; operário; mão de obra; trabalhadores braçais que vivem do salário}
  \end{Phonetics}
\end{Entry}

\begin{Entry}{工厂}{3,2}{⼯,⼚}
  \begin{Phonetics}{工厂}{gong1chang3}[][HSK 3]
    \definition[个,家,座,间]{s.}{fábrica; moinho; planta; unidades que realizam atividades de produção industrial diretamente, geralmente incluindo diferentes oficinas}
  \end{Phonetics}
\end{Entry}

\begin{Entry}{工夫}{3,4}{⼯,⼤}
  \begin{Phonetics}{工夫}{gong1fu1}
    \definition[个]{s.}{tempo | tempo livre; lazer}
  \end{Phonetics}
  \begin{Phonetics}{工夫}{gong1fu5}[][HSK 3]
    \definition[个]{s.}{(um período de) tempo; o tempo ou energia gastos para realizar uma tarefa | tempo livre}
  \end{Phonetics}
\end{Entry}

\begin{Entry}{工尺谱}{3,4,14}{⼯,⼫,⾔}
  \begin{Phonetics}{工尺谱}{gong1 che3 pu3}
    \definition*{s.}{Gongchepu, notação musical tradicional chinesa}
    \definition{s.}{notação musical tradicional chinesa que usa caracteres chineses para representar notas musicais}
  \end{Phonetics}
\end{Entry}

\begin{Entry}{工艺}{3,4}{⼯,⾋}
  \begin{Phonetics}{工艺}{gong1yi4}[][HSK 5]
    \definition{s.}{técnica; tecnologia; arte industrial; técnicas ou métodos de fabricação e processamento de produtos | artesanato; arte artesanal}
  \end{Phonetics}
\end{Entry}

\begin{Entry}{工艺品}{3,4,9}{⼯,⾋,⼝}
  \begin{Phonetics}{工艺品}{gong1 yi4 pin3}
    \definition[个,件]{s.}{trabalho manual; artesanato; habilidade manual; artigo artesanal; itens delicados produzidos com técnicas artesanais. Por exemplo, esculturas em jade, esmaltes Jingtailan, bordados, etc.}
  \end{Phonetics}
\end{Entry}

\begin{Entry}{工艺流程}{3,4,10,12}{⼯,⾋,⽔,⽲}
  \begin{Phonetics}{工艺流程}{gong1yi4 liu2cheng2}
    \definition{s.}{fluxograma do processo; fluxo do processo}
  \end{Phonetics}
\end{Entry}

\begin{Entry}{工业}{3,5}{⼯,⼀}
  \begin{Phonetics}{工业}{gong1ye4}[][HSK 3]
    \definition{s.}{indústria; utilização de recursos naturais; fabricação de meios de produção; meios de subsistência; ou processamento de produtos agrícolas, produtos semiacabados, etc.}
  \end{Phonetics}
\end{Entry}

\begin{Entry}{工会}{3,6}{⼯,⼈}
  \begin{Phonetics}{工会}{gong1hui4}[][HSK 7-9]
    \definition[个]{s.}{sindicato; sindicato trabalhista; organizações de massa criadas pelos trabalhadores para proteger os seus próprios interesses}
  \end{Phonetics}
\end{Entry}

\begin{Entry}{工地}{3,6}{⼯,⼟}
  \begin{Phonetics}{工地}{gong1di4}[][HSK 7-9]
    \definition{s.}{canteiro de obras; locais onde são realizadas construções, desenvolvimentos, produções, etc.}
  \end{Phonetics}
\end{Entry}

\begin{Entry}{工作}{3,7}{⼯,⼈}
  \begin{Phonetics}{工作}{gong1zuo4}[][HSK 1]
    \definition[份,个,分,项]{s.}{trabalho; emprego | dever; tarefa; negócio}
    \definition{v.}{trabalhar; operar (uma máquina); envolver-se em trabalho físico ou intelectual, também se refere de maneira geral a máquinas e ferramentas operadas por pessoas para realizar funções produtivas}
  \end{Phonetics}
\end{Entry}

\begin{Entry}{工作日}{3,7,4}{⼯,⼈,⽇}
  \begin{Phonetics}{工作日}{gong1zuo4ri4}[][HSK 5]
    \definition{s.}{dia de trabalho; dia útil; dias em que você deveria estar trabalhando de acordo com as regras | horas de trabalho por dia; horas do dia para fazer o trabalho necessário}
  \end{Phonetics}
\end{Entry}

\begin{Entry}{工作量}{3,7,12}{⼯,⼈,⾥}
  \begin{Phonetics}{工作量}{gong1zuo4liang4}[][HSK 7-9]
    \definition{s.}{quantidade de trabalho; volume de trabalho; carga de trabalho}
  \end{Phonetics}
\end{Entry}

\begin{Entry}{工序}{3,7}{⼯,⼴}
  \begin{Phonetics}{工序}{gong1xu4}[][HSK 7-9]
    \definition[道]{s.}{processo; procedimento de trabalho; sequência do processo de produção}
  \end{Phonetics}
\end{Entry}

\begin{Entry}{工具}{3,8}{⼯,⼋}
  \begin{Phonetics}{工具}{gong1ju4}[][HSK 3]
    \definition[个,件,套]{s.}{ferramenta; ferramentas utilizadas na produção | ferramenta; meio; instrumento; (metáfora) algo ou meio utilizado para atingir um determinado objetivo}
  \end{Phonetics}
\end{Entry}

\begin{Entry}{工科}{3,9}{⼯,⽲}
  \begin{Phonetics}{工科}{gong1ke1}[][HSK 7-9]
    \definition{s.}{curso de engenharia | engenharia como disciplina acadêmica; um termo geral para disciplinas de engenharia no ensino}
  \end{Phonetics}
\end{Entry}

\begin{Entry}{工资}{3,10}{⼯,⾙}
  \begin{Phonetics}{工资}{gong1zi1}[][HSK 3]
    \definition[份,笔,月,天]{s.}{pagamento; salário; remuneração; vencimentos; o pagamento em dinheiro ou em espécie feito ao trabalhador como remuneração pelo trabalho realizado}
  \end{Phonetics}
\end{Entry}

\begin{Entry}{工商}{3,11}{⼯,⼝}
  \begin{Phonetics}{工商}{gong1shang1}[][HSK 6]
    \definition{s.}{indústria e comércio; um termo combinado para indústria e comércio}
  \end{Phonetics}
\end{Entry}

\begin{Entry}{工商界}{3,11,9}{⼯,⼝,⽥}
  \begin{Phonetics}{工商界}{gong1shang1jie4}[][HSK 7-9]
    \definition{s.}{círculos industriais e comerciais; círculos de negócios | indústria | o mundo dos negócios}
  \end{Phonetics}
\end{Entry}

\begin{Entry}{工程}{3,12}{⼯,⽲}
  \begin{Phonetics}{工程}{gong1cheng2}[][HSK 4]
    \definition[个,项]{s.}{projeto; programa; trabalhos que utilizam equipamentos grandes e complexos, como projetos de reconstrução urbana e projetos de cestas de alimentos, etc. | engenharia; departamentos de produção e manufatura usam equipamentos grandes e complexos para realizar seu trabalho}
  \end{Phonetics}
\end{Entry}

\begin{Entry}{工程师}{3,12,6}{⼯,⽲,⼱}
  \begin{Phonetics}{工程师}{gong1cheng2shi1}[][HSK 3]
    \definition[个,位,名,些]{s.}{engenheiro; um dos cargos técnicos é o de especialista capaz de realizar de forma independente o projeto e a execução de uma tarefa técnica específica}
  \end{Phonetics}
\end{Entry}

\begin{Entry}{工龄}{3,13}{⼯,⿒}
  \begin{Phonetics}{工龄}{gong1ling2}
    \definition{s.}{tempo de serviço | senioridade}
  \end{Phonetics}
\end{Entry}

\begin{Entry}{工整}{3,16}{⼯,⽁}
  \begin{Phonetics}{工整}{gong1zheng3}[][HSK 7-9]
    \definition{adj.}{limpo; organizado; meticuloso e organizado; não desleixado | fino; requintado}
  \end{Phonetics}
\end{Entry}

%%%%%%%%%% 已 %%%%%%%%%%
\subsection*{已}\addcontentsline{loh}{figure}{已}

\begin{Entry}{已}{3}{⼰}
  \begin{Phonetics}{已}{yi3}[][HSK 3]
    \definition{adv.}{já | posteriormente; mais tarde; depois de algum tempo | demasiadamente; excessivamente}
    \definition{v.}{terminar; parar; cessar}
  \end{Phonetics}
\end{Entry}

\begin{Entry}{已久}{3,3}{⼰,⼃}
  \begin{Phonetics}{已久}{yi3jiu3}
    \definition{adv.}{já faz muito tempo}
  \end{Phonetics}
\end{Entry}

\begin{Entry}{已灭}{3,5}{⼰,⽕}
  \begin{Phonetics}{已灭}{yi3mie4}
    \definition{adj.}{extinto}
  \end{Phonetics}
\end{Entry}

\begin{Entry}{已知}{3,8}{⼰,⽮}
  \begin{Phonetics}{已知}{yi3zhi1}
    \definition{v.}{conhecido (ter ciência)}
  \end{Phonetics}
\end{Entry}

\begin{Entry}{已经}{3,8}{⼰,⽷}
  \begin{Phonetics}{已经}{yi3jing1}[][HSK 2]
    \definition{adv.}{já; indica que uma ação ou mudança foi concluída ou atingiu um determinado nível}
  \end{Phonetics}
\end{Entry}

\begin{Entry}{已故}{3,9}{⼰,⽁}
  \begin{Phonetics}{已故}{yi3gu4}
    \definition{adj.}{falecido}
  \end{Phonetics}
\end{Entry}

\begin{Entry}{已婚}{3,11}{⼰,⼥}
  \begin{Phonetics}{已婚}{yi3hun1}
    \definition{adj.}{casado}
  \end{Phonetics}
\end{Entry}

\begin{Entry}{已然}{3,12}{⼰,⽕}
  \begin{Phonetics}{已然}{yi3ran2}
    \definition{adv.}{já | já ser assim}
  \end{Phonetics}
\end{Entry}

%%%%%%%%%% 干 %%%%%%%%%%
\subsection*{干}\addcontentsline{loh}{figure}{干}

\begin{Entry}{干}{3}{⼲}[Kangxi 51]
  \begin{Phonetics}{干}{gan1}
    \definition*{s.}{Sobrenome: Gan}
    \definition{adj.}{seco | vazio; oco; seco | sem substância; vazio | de parentesco nominal; (parentes) não ligados por laços sanguíneos | sem água; (água) esgotada; completamente vazia | assumido como parente nominal; relação familiar reconhecida por adoção | rude; grosseiro; mal-educado; descreve alguém que fala de forma muito direta e rude (sem delicadeza).}
    \definition{adv.}{em vão; fútil; sem propósito; para nada; sem resultado | apenas; sem nada mais | inutilmente; sem uso, sem aproveitamento | superficialmente; significa que não há conteúdo, apenas forma}
    \definition{s.}{Arcaico: escudo | margem; ribeira; margem das águas | alimentos desidratados | abreviação para os dez troncos celestiais}
    \definition{v.}{ofender; afrontar | ter a ver com; estar relacionado com; estar implicado em; interferir com | (antiquado) buscar (cargo público, remuneração, etc.) | (dialeto) deixar alguém de fora; tratar alguém com indiferença; desprezar | assediar; perturbar; criar confusão; causar estragos; bagunçar | solicitar; procurar; buscar (cargo, salário, etc.) | beber até o fim | tratar com indiferença; ignorar}
  \seealsoref{干儿}{gan1 er2}
  \seealsoref{干儿}{gan1r5}
  \antonymref{湿}{shi1}
  \end{Phonetics}
  \begin{Phonetics}{干}{gan4}[][HSK 1]
    \definition{adj.}{capaz; competente; habilidoso}
    \definition{s.}{tronco; parte principal; corpo principal ou parte importante de algo | habilidade; capacidade; competência}
    \definition{v.}{fazer; trabalhar; cuidar; fazer coisas | ocupar o cargo de; estar envolvido em; assumir, exercer | lutar; golpear; esforçar-se}
  \end{Phonetics}
\end{Entry}

\begin{Entry}{干儿}{3,2}{⼲,⼉}
  \begin{Phonetics}{干儿}{gan1 er2}
    \definition{s.}{filho adotivo (adoção tradicional, ou seja, sem implicações legais)}
  \end{Phonetics}
  \begin{Phonetics}{干儿}{gan1r5}
    \definition{s.}{alimentos secos, desidratados}
  \end{Phonetics}
\end{Entry}

\begin{Entry}{干与}{3,3}{⼲,⼀}
  \begin{Phonetics}{干与}{gan1yu4}
    \variantof{干预}
  \end{Phonetics}
\end{Entry}

\begin{Entry}{干什么}{3,4,3}{⼲,⼈,⼃}
  \begin{Phonetics}{干什么}{gan4shen2me5}[][HSK 1]
    \definition{adv.}{o que fazer; o que ele está fazendo?; o que você está fazendo?; perguntar a razão ou o objetivo}
  \end{Phonetics}
\end{Entry}

\begin{Entry}{干戈}{3,4}{⼲,⼽}
  \begin{Phonetics}{干戈}{gan1ge1}[][HSK 7-9]
    \definition{s.}{armas de guerra; armas; guerra}[化干戈为玉帛。===Transforme guerra em paz.]
  \antonymref{玉帛}{yu4bo2}
  \end{Phonetics}
\end{Entry}

\begin{Entry}{干吗}{3,6}{⼲,⼝}
  \begin{Phonetics}{干吗}{gan4ma2}[][HSK 3]
    \definition{pron.}{por que?}
    \definition{v.}{o que fazer?}
  \end{Phonetics}
\end{Entry}

\begin{Entry}{干你屁事}{3,7,7,8}{⼲,⼈,⼫,⼅}
  \begin{Phonetics}{干你屁事}{gan1 ni3 pi4shi4}
    \definition{interj.}{`Foda-se!''}
  \end{Phonetics}
\end{Entry}

\begin{Entry}{干扰}{3,7}{⼲,⼿}
  \begin{Phonetics}{干扰}{gan1rao3}[][HSK 5]
    \definition{v.}{perturbar; incomodar | interferir; interromper o funcionamento adequado de equipamentos eletrônicos com sinais eletrônicos dispersos}
  \end{Phonetics}
\end{Entry}

\begin{Entry}{干旱}{3,7}{⼲,⽇}
  \begin{Phonetics}{干旱}{gan1han4}[][HSK 7-9]
    \definition{adj.}{árido; seco}
  \end{Phonetics}
\end{Entry}

\begin{Entry}{干事}{3,8}{⼲,⼅}
  \begin{Phonetics}{干事}{gan4shi5}[][HSK 7-9]
    \definition{s.}{secretário (ou funcionário administrativo) encarregado de algo | administrador | secretária executiva}
  \end{Phonetics}
\end{Entry}

\begin{Entry}{干净}{3,8}{⼲,⼎}
  \begin{Phonetics}{干净}{gan1jing4}[][HSK 1]
    \definition{adj.}{limpo; limpo e arrumado; sem poeira, impurezas, etc.}
    \definition{adv.}{completamente; totalmente; sem deixar nada para trás}
  \end{Phonetics}
\end{Entry}

\begin{Entry}{干杯}{3,8}{⼲,⽊}
  \begin{Phonetics}{干杯}{gan1/bei1}[][HSK 2]
    \definition{interj.}{`Saúde!''}
    \definition{v.+compl.}{fazer um brinde;  brindar até a última gota}
  \end{Phonetics}
\end{Entry}

\begin{Entry}{干活}{3,9}{⼲,⽔}
  \begin{Phonetics}{干活}{gan4/huo2}
    \definition{v.+compl.}{trabalhar | trabalhar em um emprego}
  \end{Phonetics}
\end{Entry}

\begin{Entry}{干活儿}{3,9,2}{⼲,⽔,⼉}
  \begin{Phonetics}{干活儿}{gan4huo2r5}[][HSK 2]
    \definition{v.}{trabalhar; gastar energia física ou mental para fazer algo, especialmente trabalho árduo ou esforçado.}
  \end{Phonetics}
\end{Entry}

\begin{Entry}{干涉}{3,10}{⼲,⽔}
  \begin{Phonetics}{干涉}{gan1she4}[][HSK 6]
    \definition{s.}{interferência; refere"-se ao ato ou comportamento de interferir nos assuntos dos outros}
    \definition{v.}{interferir; intervir; intrometer"-se; pedir ou impedir algo geralmente significa interferir quando não se deve}
  \end{Phonetics}
\end{Entry}

\begin{Entry}{干脆}{3,10}{⼲,⾁}
  \begin{Phonetics}{干脆}{gan1cui4}[][HSK 5]
    \definition{adj.}{claro; direto; (falar, fazer coisas) sem hesitação; atitude clara}
    \definition{adv.}{justamente; diretamente; sem maiores considerações}
  \end{Phonetics}
\end{Entry}

\begin{Entry}{干部}{3,10}{⼲,⾢}
  \begin{Phonetics}{干部}{gan4bu4}[][HSK 7-9]
    \definition[名,个,位]{s.}{quadro; funcionário público; funcionário do governo; servidores públicos (excluindo soldados e pessoal geral) em órgãos estatais, militares e organizações populares; refere"-se àqueles que desempenham determinado trabalho de liderança ou gestão | líder; pessoal que ocupa determinados cargos de liderança}
  \end{Phonetics}
\end{Entry}

\begin{Entry}{干预}{3,10}{⼲,⾴}
  \begin{Phonetics}{干预}{gan1yu4}[][HSK 5]
    \definition{s.}{intromissão; intervenção}
    \definition{v.}{intrometer-se; intervir; interpor-se;}
  \end{Phonetics}
\end{Entry}

\begin{Entry}{干燥}{3,17}{⼲,⽕}
  \begin{Phonetics}{干燥}{gan1zao4}[][HSK 7-9]
    \definition{adj.}{seco; árido; sem umidade ou muito pouca umidade | enfadonho; desinteressante; chato, sem graça}
  \end{Phonetics}
\end{Entry}

%%%%%%%%%% 广 %%%%%%%%%%
\subsection*{广}\addcontentsline{loh}{figure}{广}

\begin{Entry}{广}{3}{⼴}[Kangxi 53]
  \begin{Phonetics}{广}{an1}
    \definition{s.}{mais comum em nomes de pessoas; o mesmo que 庵}[广安是我的朋友。===An'an é meu amigo.]
  \seealsoref{庵}{an1}
  \end{Phonetics}
  \begin{Phonetics}{广}{guang3}[][HSK 5]
    \definition*{s.}{Sobrenome: Guang}
    \definition{adj.}{largo; vasto; amplo; extenso | numeroso | comum; universal}
    \definition{s.}{Guangdong, 广东, e Guangxi, 广州}
    \definition{v.}{expandir; espalhar; ampliar}
  \seealsoref{广东}{guang3dong1}
  \seealsoref{广州}{guang3zhou1}
  \antonymref{狭}{xia2}
  \end{Phonetics}
  \begin{Phonetics}{广}{yan3}
    \definition[家]{s.}{casa ou edifício construído contra ou ao longo da encosta de uma montanha ou penhasco}
  \end{Phonetics}
\end{Entry}

\begin{Entry}{广义}{3,3}{⼴,⼂}
  \begin{Phonetics}{广义}{guang3yi4}[][HSK 7-9]
    \definition*{s.}{Província de Quang Ngai; nome de lugar vietnamita, uma das províncias do Vietnã Central}
    \definition{s.}{sentido amplo; sentido geral; definição mais ampla}
  \end{Phonetics}
\end{Entry}

\begin{Entry}{广大}{3,3}{⼴,⼤}
  \begin{Phonetics}{广大}{guang3da4}[][HSK 3]
    \definition{adj.}{muito difundido; enorme (alcance, escala) | (uma área ou espaço) vasto; extenso; em grande escala; amplo (área, espaço) | numeroso; muitos (número de pessoas)}
  \end{Phonetics}
\end{Entry}

\begin{Entry}{广东}{3,5}{⼴,⼀}
  \begin{Phonetics}{广东}{guang3dong1}
    \definition*{s.}{Província de Guangdong}
  \seealsoref{粤}{yue4}
  \end{Phonetics}
\end{Entry}

\begin{Entry}{广场}{3,6}{⼴,⼟}
  \begin{Phonetics}{广场}{guang3chang3}[][HSK 2]
    \definition{s.}{praça; praça pública; esplanada; área ampla, especificamente uma área ampla na cidade}
  \end{Phonetics}
\end{Entry}

\begin{Entry}{广场舞}{3,6,14}{⼴,⼟,⾇}
  \begin{Phonetics}{广场舞}{guang3chang3wu3}
    \definition{s.}{quadrilha, uma rotina de exercícios tocada com música em quadrados públicos, parques e praças, popular especialmente entre mulheres de meia-idade e aposentados na China}
  \end{Phonetics}
\end{Entry}

\begin{Entry}{广州}{3,6}{⼴,⼮}
  \begin{Phonetics}{广州}{guang3zhou1}
    \definition*{s.}{Guangzhou, antigamente Cantão; Capital da Província de Guangdong}
  \end{Phonetics}
\end{Entry}

\begin{Entry}{广西}{3,6}{⼴,⾑}
  \begin{Phonetics}{广西}{guang3xi1}
    \definition*{s.}{Guangxi (Região Autônoma de Zhuang)}
  \seealsoref{壮}{zhuang4}
  \end{Phonetics}
\end{Entry}

\begin{Entry}{广告}{3,7}{⼴,⼝}
  \begin{Phonetics}{广告}{guang3gao4}[][HSK 2]
    \definition[则,条,段,项,个]{s.}{anúncio; propaganda; uma forma de divulgação ao público de produtos, serviços ou programas culturais e esportivos, geralmente realizada por meio de jornais, televisão, rádio, cartazes, etc.}
    \definition{v.}{anunciar; a ação ou ato de promover ou divulgar algo}
  \end{Phonetics}
\end{Entry}

\begin{Entry}{广泛}{3,7}{⼴,⽔}
  \begin{Phonetics}{广泛}{guang3fan4}[][HSK 5]
    \definition{adj.}{amplo; extenso; de grande alcance; disseminado; escopo e cobertura amplos}
  \end{Phonetics}
\end{Entry}

\begin{Entry}{广阔}{3,12}{⼴,⾨}
  \begin{Phonetics}{广阔}{guang3kuo4}[][HSK 6]
    \definition{adj.}{vasto; largo; amplo}
  \end{Phonetics}
\end{Entry}

\begin{Entry}{广播}{3,15}{⼴,⼿}
  \begin{Phonetics}{广播}{guang3bo1}[][HSK 3]
    \definition[个,次,段,则,条]{s.}{programa de rádio; transmissão (de rádio); refere"-se a programas transmitidos por estações de rádio ou televisão a cabo}
    \definition{v.}{transmitir; estar no ar | espalhar"-se amplamente; ser conhecido em toda parte; divulgar amplamente}
  \end{Phonetics}
\end{Entry}

%%%%%%%%%% 弓 %%%%%%%%%%
\subsection*{弓}\addcontentsline{loh}{figure}{弓}

\begin{Entry}{弓}{3}{⼸}[Kangxi 57]
  \begin{Phonetics}{弓}{gong1}[][HSK 7-9]
    \definition*{s.}{Sobrenome: Gong}
    \definition{clas.}{uma antiga unidade de comprimento para medir a terra, igual a cinco 尺}
    \definition[张]{s.}{arco | qualquer coisa em forma de arco | Obsoleto: ferramenta de madeira de medição de terreno; divisores de madeira para medição de terrenos; arco de medição; régua escalonada (1,5m)}
    \definition{v.}{dobrar; arquear; curvar; entortar}
  \seealsoref{尺}{chi3}
  \end{Phonetics}
\end{Entry}

%%%%%%%%%% 才 %%%%%%%%%%
\subsection*{才}\addcontentsline{loh}{figure}{才}

\begin{Entry}{才}{3}{⼿}
  \begin{Phonetics}{才}{cai2}[][HSK 2,4]
    \definition*{s.}{Sobrenome: Cai}
    \definition{adv.}{há pouco; agora mesmo | (precedido por uma expressão de tempo) não até | (precedido por uma expressão de razão ou condição) não a menos que; não até que; então e somente então; por nenhuma outra razão | (seguido por uma expressão numérica) apenas; indica um intervalo pequeno ou uma quantidade reduzida, equivalente a 仅仅 ou 只 | (em uma afirmação ou negação, enfatizando o que vem antes de 才, geralmente com 呢 no final da frase) na verdade; realmente | dica que algo acontece tarde ou termina tarde | (precedido por uma expressão de tempo) não até; indicando que não era assim, mas agora surgiu uma nova situação | (precedido por uma expressão de razão ou condição) a menos que; indica que só em determinadas condições e, em seguida, como | (expressa ênfase )}
    \definition{s.}{habilidade; talento; dom | pessoa competente | pessoas de um determinado tipo (frequentemente usado como sufixo) | dotação; talento; habilidade}
  \seealsoref{呢}{ne5}
  \end{Phonetics}
\end{Entry}

\begin{Entry}{才华}{3,6}{⼿,⼗}
  \begin{Phonetics}{才华}{cai2hua2}[][HSK 7-9]
    \definition[份]{s.}{talento literário; talento artístico; talentos que são exibidos externamente (principalmente nas artes)}
  \end{Phonetics}
\end{Entry}

\begin{Entry}{才能}{3,10}{⼿,⾁}
  \begin{Phonetics}{才能}{cai2neng2}[][HSK 3]
    \definition[间]{s.}{talento; habilidade; dom; capacidade; inteligência e habilidade}
  \end{Phonetics}
\end{Entry}

\begin{Entry}{才略}{3,11}{⼿,⽥}
  \begin{Phonetics}{才略}{cai2lve4}
    \definition{s.}{habilidade e sagacidade}
  \end{Phonetics}
\end{Entry}

%%%%%%%%%% 门 %%%%%%%%%%
\subsection*{门}\addcontentsline{loh}{figure}{门}

\begin{Entry}{门}{3}{⾨}[Kangxi 169]
  \begin{Phonetics}{门}{men2}[][HSK 1]
    \definition*{s.}{Sobrenome: Men}
    \definition{clas.}{para equipamentos de artilharia (por exemplo: canhões) | para trabalhos escolares, ciência e tecnologia, etc. | para idiomas | para casamentos | para parentes}
    \definition[个,把,道,扇]{s.}{entradas e saídas de edifícios, veículos, navios, aviões, etc. | válvula; interruptor; algo que funciona como um interruptor ou como uma porta | habilidade; método; acesso; maneira de fazer algo | família; ramo de uma família ou clã | seita (religiosa); escola (de pensamento); faculdades acadêmicas, ideológicas ou religiosas | classe; categoria; ramo de estudo; refere"-se à categoria geral de coisas | filo; segundo nível da classificação biológica | (computador) \emph{gate}; porta (lógica) | porta; portão; entrada; refere"-se a uma porta que pode ser aberta e fechada, instalada na entrada e saída | qualquer abertura; partes de objetos que podem ser abertas e fechadas | orifício no corpo humano; refere"-se especificamente aos orifícios do corpo humano | estudar com o mesmo professor; refere"-se especificamente ao professor ou mestre | posição em um jogo de apostas (em relação ao local onde se senta ou onde se faz uma aposta)}
  \end{Phonetics}
\end{Entry}

\begin{Entry}{门口}{3,3}{⾨,⼝}
  \begin{Phonetics}{门口}{men2kou3}[][HSK 1]
    \definition[个]{s.}{porta; portão; entrada; porta de entrada}
  \end{Phonetics}
\end{Entry}

\begin{Entry}{门当户对}{3,6,4,5}{⾨,⼹,⼾,⼨}
  \begin{Phonetics}{门当户对}{men2dang1-hu4dui4}[][HSK 7-9]
    \definition{expr.}{``Casar com alguém de posição social equivalente.''; compatibilidade social e econômica adequada (para casamento); (um possível parceiro para casamento) uma combinação adequada; as famílias são bem compatíveis em termos de status social}
  \end{Phonetics}
\end{Entry}

\begin{Entry}{门诊}{3,7}{⾨,⾔}
  \begin{Phonetics}{门诊}{men2zhen3}[][HSK 5]
    \definition{s.}{(no hospital) clínica ambulatorial; seção para pacientes ambulatoriais; local onde os médicos atendem pacientes que não estão internados no hospital}
  \end{Phonetics}
\end{Entry}

\begin{Entry}{门铃}{3,10}{⾨,⾦}
  \begin{Phonetics}{门铃}{men2ling2}[][HSK 7-9]
    \definition{s.}{campainha; uma campainha ou campainha elétrica para bater à porta}
  \end{Phonetics}
\end{Entry}

\begin{Entry}{门票}{3,11}{⾨,⽰}
  \begin{Phonetics}{门票}{men2piao4}[][HSK 1]
    \definition{s.}{bilhete de entrada; bilhete de admissão; ingressos para locais de turismo, entretenimento, etc.}
  \end{Phonetics}
\end{Entry}

\begin{Entry}{门道}{3,12}{⾨,⾡}
  \begin{Phonetics}{门道}{men2dao4}
    \definition{s.}{porta | portal}
  \end{Phonetics}
  \begin{Phonetics}{门道}{men2dao5}
    \definition{s.}{talento | a maneira de fazer algo}
  \end{Phonetics}
\end{Entry}

\begin{Entry}{门路}{3,13}{⾨,⾜}
  \begin{Phonetics}{门路}{men2lu5}[][HSK 7-9]
    \definition{s.}{maneira de fazer algo; jeito | conexões sociais (para conseguir empregos, etc.) | jeito; saber fazer; truque do ofício; o segredo para fazer as coisas acontecerem}
  \seealsoref{门道}{men2dao5}
  \end{Phonetics}
\end{Entry}

\begin{Entry}{门槛}{3,14}{⾨,⽊}
  \begin{Phonetics}{门槛}{men2kan3}[][HSK 7-9]
    \definition{s.}{soleira; batente da porta; a viga horizontal ou faixa de pedra na parte inferior do batente da porta, próxima ao chão, etc. | o requisito para entrar em um determinado campo; metaforicamente, refere"-se aos padrões ou condições para entrar em um determinado intervalo}
  \end{Phonetics}
\end{Entry}

%%%%%%%%%% 飞 %%%%%%%%%%
\subsection*{飞}\addcontentsline{loh}{figure}{飞}

\begin{Entry}{飞}{3}{⾶}[Kangxi 183]
  \begin{Phonetics}{飞}{fei1}[][HSK 1]
    \definition{adj.}{inesperado; acidental; surgido do nada}
    \definition{adv.}{rapidamente; velozmente}
    \definition{s.}{roda livre de uma bicicleta}
    \definition{v.}{voar; esvoaçar; (pássaros, insetos, etc.) voar pelo ar batendo as asas | voar; utilizar máquinas motorizadas para se deslocar no ar | voar; (objetos naturais) flutuar ou esvoaçar no ar | volatilizar; evaporar; um gás se dissipar no ar | ir muito rapidamente; movimentar-se rapidamente, como se estivesse voando}
  \end{Phonetics}
\end{Entry}

\begin{Entry}{飞机}{3,6}{⾶,⽊}
  \begin{Phonetics}{飞机}{fei1ji1}[][HSK 1]
    \definition[架,个]{s.}{avião; aeronave; aroplano}
  \end{Phonetics}
\end{Entry}

\begin{Entry}{飞机票}{3,6,11}{⾶,⽊,⽰}
  \begin{Phonetics}{飞机票}{fei1ji1piao4}
    \definition[张]{s.}{bilhete de avião; documento emitido mediante pagamento de passagem aérea, que autoriza o titular a viajar}
  \seealsoref{机票}{ji1piao4}
  \end{Phonetics}
\end{Entry}

\begin{Entry}{飞行}{3,6}{⾶,⾏}
  \begin{Phonetics}{飞行}{fei1xing2}[][HSK 3]
    \definition{s.}{voo; aviação}
    \definition{v.}{voar; fazer um voo; (aviões, foguetes, etc.) voar no ar}
  \end{Phonetics}
\end{Entry}

\begin{Entry}{飞行员}{3,6,7}{⾶,⾏,⼝}
  \begin{Phonetics}{飞行员}{fei1xing2yuan2}[][HSK 6]
    \definition[名,班]{s.}{piloto; aviador; pilotos de aeronaves}
  \end{Phonetics}
\end{Entry}

\begin{Entry}{飞往}{3,8}{⾶,⼻}
  \begin{Phonetics}{飞往}{fei1wang3}[][HSK 7-9]
    \definition{v.}{voar para}[飞机将径直飞往圣保罗。===O avião voará diretamente para São Paulo.]
  \end{Phonetics}
\end{Entry}

\begin{Entry}{飞速}{3,10}{⾶,⾡}
  \begin{Phonetics}{飞速}{fei1su4}[][HSK 7-9]
    \definition{adj.}{em velocidade máxima}
  \end{Phonetics}
\end{Entry}

\begin{Entry}{飞弹}{3,11}{⾶,⼸}
  \begin{Phonetics}{飞弹}{fei1dan4}
    \definition{s.}{míssil (guiado) | bala perdida; bala disparada aleatoriamente | avião de mísseis; bomba voadora; bombas com equipamento de voo automático | dardo}
  \end{Phonetics}
\end{Entry}

\begin{Entry}{飞船}{3,11}{⾶,⾈}
  \begin{Phonetics}{飞船}{fei1chuan2}[][HSK 6]
    \definition{s.}{nave espacial; espaçonave | dirigível; aerobarco}
  \end{Phonetics}
\end{Entry}

\begin{Entry}{飞跃}{3,11}{⾶,⾜}
  \begin{Phonetics}{飞跃}{fei1yue4}[][HSK 7-9]
    \definition{v.}{saltar; crescer rapidamente}
  \end{Phonetics}
\end{Entry}

\begin{Entry}{飞翔}{3,12}{⾶,⽻}
  \begin{Phonetics}{飞翔}{fei1xiang2}[][HSK 7-9]
    \definition{v.}{pairar; voar; circular no ar; voar em círculos}
  \end{Phonetics}
\end{Entry}

\begin{Entry}{飞碟}{3,14}{⾶,⽯}
  \begin{Phonetics}{飞碟}{fei1die2}
    \definition{s.}{disco-voador, OVNI, \emph{UFO} | \emph{frisbee}}
  \end{Phonetics}
\end{Entry}

%%%%%%%%%% 马 %%%%%%%%%%
\subsection*{马}\addcontentsline{loh}{figure}{马}

\begin{Entry}{马}{3}{⾺}[Kangxi 187]
  \begin{Phonetics}{马}{ma3}[][HSK 3]
    \definition*{s.}{Sobrenome: Ma}
    \definition{adj.}{grande; extenso; amplo}
    \definition[匹,头,只,群]{s.}{cavalo | a peça do cavalo no xadrez chinês}
  \end{Phonetics}
\end{Entry}

\begin{Entry}{马力}{3,2}{⾺,⼒}
  \begin{Phonetics}{马力}{ma3li4}[][HSK 7-9]
    \definition{clas.}{Física: cavalos de potência, cavalo-vapor (cv)}
  \end{Phonetics}
\end{Entry}

\begin{Entry}{马上}{3,3}{⾺,⼀}
  \begin{Phonetics}{马上}{ma3shang4}[][HSK 1]
    \definition{adv.}{imediatamente; de uma só vez; em um piscar de olhos | em breve; em um futuro próximo; em um curto espaço de tempo}
  \end{Phonetics}
\end{Entry}

\begin{Entry}{马马虎虎}{3,3,8,8}{⾺,⾺,⾌,⾌}
  \begin{Phonetics}{马马虎虎}{ma3ma3hu3hu3}
    \definition{adj.}{tolerável; aceitável; mais ou menos; razoável; não tão ruim | descuidado; casual; vago; descuidado e negligente na execução das tarefas}
  \end{Phonetics}
\end{Entry}

\begin{Entry}{马车}{3,4}{⾺,⾞}
  \begin{Phonetics}{马车}{ma3che1}[][HSK 6]
    \definition[辆]{s.}{carruagem (puxada por cavalo); carroça; charrete}
  \end{Phonetics}
\end{Entry}

\begin{Entry}{马列主义}{3,6,5,3}{⾺,⼑,⼂,⼂}
  \begin{Phonetics}{马列主义}{ma3 lie4 zhu3yi4}
    \definition*{s.}{Marxismo-Leninismo}
  \end{Phonetics}
\end{Entry}

\begin{Entry}{马后炮}{3,6,9}{⾺,⼝,⽕}
  \begin{Phonetics}{马后炮}{ma3hou4pao4}[][HSK 7-9]
    \definition{s.}{(termo do xadrez chinês) ação ou conselho tardio; esforço tardio; tarde demais | Figurativo: ação tardia | visão retrospectiva}
  \end{Phonetics}
\end{Entry}

\begin{Entry}{马戏}{3,6}{⾺,⼽}
  \begin{Phonetics}{马戏}{ma3xi4}[][HSK 7-9]
    \definition[场,次]{s.}{circo | espetáculo de circo (apresentação)}
  \end{Phonetics}
\end{Entry}

\begin{Entry}{马耳他}{3,6,5}{⾺,⽿,⼈}
  \begin{Phonetics}{马耳他}{ma3'er3ta1}
    \definition*{s.}{Malta}
  \end{Phonetics}
\end{Entry}

\begin{Entry}{马克思列宁主义}{3,7,9,6,5,5,3}{⾺,⼗,⼼,⼑,⼧,⼂,⼂}
  \begin{Phonetics}{马克思列宁主义}{ma3ke4si1 lie4ning2 zhu3yi4}
    \definition*{s.}{Marxismo-Leninismo}
  \end{Phonetics}
\end{Entry}

\begin{Entry}{马尾}{3,7}{⾺,⼫}
  \begin{Phonetics}{马尾}{ma3wei3}
    \definition{s.}{(penteado) rabo de cavalo | cauda de cavalo}
  \end{Phonetics}
\end{Entry}

\begin{Entry}{马虎}{3,8}{⾺,⾌}
  \begin{Phonetics}{马虎}{ma3hu5}[][HSK 7-9]
    \definition{adj.}{descuidado; casual | superficial; apressado; descuidado}
    \definition{v.}{encarar de forma leviana; fazer um trabalho malfeito}
  \end{Phonetics}
\end{Entry}

\begin{Entry}{马桶}{3,11}{⾺,⽊}
  \begin{Phonetics}{马桶}{ma3tong3}[][HSK 7-9]
    \definition[个,台]{s.}{vaso sanitário}[小心别把手机掉进马桶。===Tenha cuidado para não deixar seu celular cair no vaso sanitário.]
  \end{Phonetics}
\end{Entry}

\begin{Entry}{马路}{3,13}{⾺,⾜}
  \begin{Phonetics}{马路}{ma3lu4}[][HSK 1]
    \definition[条]{s.}{estrada; rua; avenida; estradas largas e planas para o tráfego de carros e cavalos nas cidades ou nos subúrbios}
  \end{Phonetics}
\end{Entry}

%%%%% EOF %%%%%


 %%%
%%% 4画
%%%

\section*{4画}\addcontentsline{toc}{section}{4画}

\begin{entry}{不}{4}[Radical 一]
  \begin{phonetics}{不}{bu2}[(antes de quarto tom)][1]
    \definition{adv.}{não}
    \definition{pref.}{prefixo negativo}
  \end{phonetics}
  \begin{phonetics}{不}{bu4}[][HSK 1]
    \definition{adv.}{não}
    \definition{pref.}{prefixo negativo}
  \end{phonetics}
  \begin{phonetics}{不}{bu5}[][HSK 1]
    \definition{adv.}{não (em expressões ``v.+不+v.'')}
  \end{phonetics}
\end{entry}

\begin{entry}{不一会儿}{4,1,6,2}
  \begin{phonetics}{不一会儿}{bu4 yi2 hui4r5}[][HSK 2]
    \definition{expr.}{em um momento | em pouco tempo |em breve}
  \end{phonetics}
\end{entry}

\begin{entry}{不一定}{4,1,8}
  \begin{phonetics}{不一定}{bu4 yi2 ding4}[][HSK 2]
    \definition{adv.}{talvez | incerto | não tenho certeza | não necessariamente}
  \end{phonetics}
\end{entry}

\begin{entry}{不久}{4,3}
  \begin{phonetics}{不久}{bu4 jiu3}[][HSK 2]
    \definition{adj.}{em breve | futuro próximo | logo depois | não muito depois | não muito tempo (antes ou depois de algo)}
  \end{phonetics}
\end{entry}

\begin{entry}{不大}{4,3}
  \begin{phonetics}{不大}{bu2 da4}[][HSK 1]
    \definition{adv.}{não muito | não frequentemente | raramente |dificilmente | escassamente}
  \end{phonetics}
\end{entry}

\begin{entry}{不大离}{4,3,10}
  \begin{phonetics}{不大离}{bu2da4li2}
    \definition{adj.}{bem perto | quase certo | nada mal}
  \end{phonetics}
\end{entry}

\begin{entry}{不公}{4,4}
  \begin{phonetics}{不公}{bu4gong1}
    \definition{adj.}{injusto}
  \end{phonetics}
\end{entry}

\begin{entry}{不太}{4,4}
  \begin{phonetics}{不太}{bu2 tai4}[][HSK 2]
    \definition{adv.}{não bastante | não muito}
  \end{phonetics}
\end{entry}

\begin{entry}{不少}{4,4}
  \begin{phonetics}{不少}{bu4 shao3}[][HSK 2]
    \definition{adj.}{muitos | bastante | não poucos}
  \end{phonetics}
\end{entry}

\begin{entry}{不日}{4,4}
  \begin{phonetics}{不日}{bu2ri4}
    \definition{adv.}{em alguns dias}
  \end{phonetics}
\end{entry}

\begin{entry}{不止}{4,4}
  \begin{phonetics}{不止}{bu4zhi3}
    \definition{adv.}{incessantemente | sem fim | mais que | não limitado a}
  \end{phonetics}
\end{entry}

\begin{entry}{不可避免}{4,5,16,7}
  \begin{phonetics}{不可避免}{bu4ke3bi4mian3}
    \definition{adj./adv.}{inevitável}
  \end{phonetics}
\end{entry}

\begin{entry}{不对}{4,5}
  \begin{phonetics}{不对}{bu2dui4}[][HSK 1]
    \definition{adj.}{incorreto | errado | anormal | estranho | estar em desacordo com | ser difícil de conviver}
  \end{phonetics}
\end{entry}

\begin{entry}{不用}{4,5}
  \begin{phonetics}{不用}{bu2yong4}[][HSK 1]
    \definition{v.}{não precisar}
    \seeref{甭}{beng2}
  \end{phonetics}
\end{entry}

\begin{entry}{不同}{4,6}
  \begin{phonetics}{不同}{bu4 tong2}
    \definition{adj.}{diferente | distinto}
  \end{phonetics}
\end{entry}

\begin{entry}{不好意思}{4,6,13,9}
  \begin{phonetics}{不好意思}{bu4 hao3 yi4 si5}[][HSK 2]
    \definition{adj.}{pedir desculpas (por incomodar alguém) | sentir-se envergonhado | achar isso embaraçoso}
  \end{phonetics}
\end{entry}

\begin{entry}{不如}{4,6}
  \begin{phonetics}{不如}{bu4ru2}[][HSK 2]
    \definition{conj.}{em vez de | melhor que | seria melhor}
    \definition{v.}{ser inferior a | não ser igual a | não ser tão bom quanto | não poder fazer melhor que}
  \end{phonetics}
\end{entry}

\begin{entry}{不成话}{4,6,8}
  \begin{phonetics}{不成话}{bu4cheng2hua4}
    \definition{expr.}{sem razão | demasiado irracionável}
    \seeref{不是话}{bu2shi4hua4}
    \seeref{不像话}{bu2xiang4hua4}
  \end{phonetics}
\end{entry}

\begin{entry}{不行}{4,6}
  \begin{phonetics}{不行}{bu4 xing2}[][HSK 2]
    \definition{adj.}{não funciona | não é bom}
    \definition{adv.}{profundamente | terrivelmente | extremamente}
    \definition{v.}{não fazer | não ser permitido | estar fora de questão | estar à beira da morte}
  \end{phonetics}
\end{entry}

\begin{entry}{不论……也……}{4,6,3}
  \begin{phonetics}{不论……也……}{bu2lun4 ye3}
    \definition{conj.}{não apenas\dots, (o que, quem, como, etc.), \dots}
  \end{phonetics}
\end{entry}

\begin{entry}{不论……都……}{4,6,10}
  \begin{phonetics}{不论……都……}{bu2lun4 dou1}
    \definition{conj.}{não apenas\dots, (o que, quem, como, etc.), \dots}
  \end{phonetics}
\end{entry}

\begin{entry}{不过}{4,6}
  \begin{phonetics}{不过}{bu2guo4}[][HSK 2]
    \definition{conj.}{mas | contudo | no entanto}
  \end{phonetics}
\end{entry}

\begin{entry}{不但}{4,7}
  \begin{phonetics}{不但}{bu2 dan4}[][HSK 2]
    \definition{conj.}{não somente}
  \end{phonetics}
\end{entry}

\begin{entry}{不但……而且……}{4,7,6,5}
  \begin{phonetics}{不但……而且……}{bu2 dan4 er2qie3}[][HSK 2]
    \definition{conj.}{não só\dots mas também\dots}
  \end{phonetics}
\end{entry}

\begin{entry}{不到}{4,8}
  \begin{phonetics}{不到}{bu2dao4}
    \definition{adj.}{insuficiente}
    \definition{adv.}{menos que}
    \definition{v.}{não chegar}
  \end{phonetics}
\end{entry}

\begin{entry}{不注意}{4,8,13}
  \begin{phonetics}{不注意}{bu2zhu4yi4}
    \definition{adj.}{impensado | distraído}
    \definition{s.}{descuido | distração}
  \end{phonetics}
\end{entry}

\begin{entry}{不客气}{4,9,4}
  \begin{phonetics}{不客气}{bu2 ke4qi5}[][HSK 1]
    \definition{adj.}{indelicado | rude | brusco}
    \definition{expr.}{de nada | não há de que | não mencione isso}
  \end{phonetics}
\end{entry}

\begin{entry}{不是话}{4,9,8}
  \begin{phonetics}{不是话}{bu2shi4hua4}
    \definition{expr.}{sem razão | demasiado irracionável}
    \seeref{不像话}{bu2xiang4hua4}
    \seeref{不成话}{bu4cheng2hua4}
  \end{phonetics}
\end{entry}

\begin{entry}{不要}{4,9}
  \begin{phonetics}{不要}{bu2 yao4}[][HSK 2]
    \definition{adv.}{nada de (pedir a alguém para não fazer) | não}
  \end{phonetics}
\end{entry}

\begin{entry}{不够}{4,11}
  \begin{phonetics}{不够}{bu2 gou4}[][HSK 2]
    \definition{adv.}{insuficiente}
    \definition{v.}{não ser suficiente}
  \end{phonetics}
\end{entry}

\begin{entry}{不断}{4,11}
  \begin{phonetics}{不断}{bu2duan4}
    \definition{adv.}{continuamente | sem fim}
  \end{phonetics}
\end{entry}

\begin{entry}{不像话}{4,13,8}
  \begin{phonetics}{不像话}{bu2xiang4hua4}
    \definition{expr.}{sem razão | demasiado irracionável}
    \seeref{不是话}{bu2shi4hua4}
    \seeref{不成话}{bu4cheng2hua4}
  \end{phonetics}
\end{entry}

\begin{entry}{不满}{4,13}
  \begin{phonetics}{不满}{bu4 man3}[][HSK 2]
    \definition{adj.}{ressentido | insatisfeito | descontente}
    \definition{v.}{estar descontente com |ser menor que}
  \end{phonetics}
\end{entry}

\begin{entry}{不错}{4,13}
  \begin{phonetics}{不错}{bu2 cuo4}[][HSK 2]
    \definition{adj.}{correto | não (é) mau | bastante bom | certo}
  \end{phonetics}
\end{entry}

\begin{entry}{不管……也……}{4,14,3}
  \begin{phonetics}{不管……也……}{bu4guan3 ye3}
    \definition{conj.}{não apenas\dots, (o que, quem, como, etc.), \dots}
  \end{phonetics}
\end{entry}

\begin{entry}{不管……都……}{4,14,10}
  \begin{phonetics}{不管……都……}{bu4guan3 dou1}
    \definition{conj.}{não apenas\dots, (o que, quem, como, etc.), \dots}
  \end{phonetics}
\end{entry}

\begin{entry}{专业}{4,5}
  \begin{phonetics}{专业}{zhuan1ye4}
    \definition[门,个]{s.}{área de atuação | especialidade}
  \end{phonetics}
\end{entry}

\begin{entry}{专业人士}{4,5,2,3}
  \begin{phonetics}{专业人士}{zhuan1ye4ren2shi4}
    \definition{s.}{profissional}
  \end{phonetics}
\end{entry}

\begin{entry}{专业人才}{4,5,2,3}
  \begin{phonetics}{专业人才}{zhuan1ye4ren2cai2}
    \definition{s.}{especialista (em uma área)}
  \end{phonetics}
\end{entry}

\begin{entry}{专业化}{4,5,4}
  \begin{phonetics}{专业化}{zhuan1ye4hua4}
    \definition{s.}{especialização}
  \end{phonetics}
\end{entry}

\begin{entry}{专业户}{4,5,4}
  \begin{phonetics}{专业户}{zhuan1ye4hu4}
    \definition{s.}{indústria caseira | empresa familiar produzindo um produto especial}
  \end{phonetics}
\end{entry}

\begin{entry}{专业性}{4,5,8}
  \begin{phonetics}{专业性}{zhuan1ye4xing4}
    \definition{s.}{profissionalismo | expertise}
  \end{phonetics}
\end{entry}

\begin{entry}{专业教育}{4,5,11,8}
  \begin{phonetics}{专业教育}{zhuan1ye4jiao4yu4}
    \definition{s.}{educação especializada | escola técnica}
  \end{phonetics}
\end{entry}

\begin{entry}{中}{4}[Radical 丨]
  \begin{phonetics}{中}{zhong1}
    \definition*{s.}{China}
    \definition*{s.}{sobrenome Zhong}
    \definition{s.}{centro | meio | médio | intermediário | média | meio caminho entre dois extremos | intermediador}
  \seealsoref{中国}{zhong1guo2}
  \end{phonetics}
  \begin{phonetics}{中}{zhong4}
    \definition{v.}{acertar | encaixar exatamente |ser atingido por | cair em | ser afetado por | sofrer | sustentar}
  \end{phonetics}
\end{entry}

\begin{entry}{中小学}{4,3,8}
  \begin{phonetics}{中小学}{zhong1 xiao3 xue2}[][HSK 2]
    \definition{s.}{escolas primárias e secundárias}
  \end{phonetics}
\end{entry}

\begin{entry}{中午}{4,4}
  \begin{phonetics}{中午}{zhong1wu3}[][HSK 1]
    \definition[个]{s.}{meio-dia}
  \end{phonetics}
\end{entry}

\begin{entry}{中心}{4,4}
  \begin{phonetics}{中心}{zhong1xin1}[][HSK 2]
    \definition[个]{s.}{núcleo | coração | meio | centro |chave}
  \end{phonetics}
\end{entry}

\begin{entry}{中文}{4,4}
  \begin{phonetics}{中文}{zhong1wen2}[][HSK 1]
    \definition{s.}{chinês, língua chinesa}
  \end{phonetics}
\end{entry}

\begin{entry}{中东}{4,5}
  \begin{phonetics}{中东}{zhong1dong1}
    \definition*{s.}{Oriente Médio}
  \end{phonetics}
\end{entry}

\begin{entry}{中央情报局}{4,5,11,7,7}
  \begin{phonetics}{中央情报局}{zhong1yang1 qing2bao4ju2}
    \definition*{s.}{Agência Central de Inteligência dos EUA, CIA}
  \end{phonetics}
\end{entry}

\begin{entry}{中年}{4,6}
  \begin{phonetics}{中年}{zhong1 nian2}[][HSK 2]
    \definition{s.}{meia-idade}
  \end{phonetics}
\end{entry}

\begin{entry}{中级}{4,6}
  \begin{phonetics}{中级}{zhong1 ji2}[][HSK 2]
    \definition{adj.}{nível médio | nível intermediário}
  \end{phonetics}
\end{entry}

\begin{entry}{中医}{4,7}
  \begin{phonetics}{中医}{zhong1 yi1}[][HSK 2]
    \definition{s.}{ciência médica tradicional chinesa | médico de medicina tradicional chinesa | praticante de medicina chinesa}
  \end{phonetics}
\end{entry}

\begin{entry}{中间}{4,7}
  \begin{phonetics}{中间}{zhong1jian1}[][HSK 1]
    \definition{adv.}{central | centro | no meio}
  \end{phonetics}
\end{entry}

\begin{entry}{中国}{4,8}
  \begin{phonetics}{中国}{zhong1guo2}[][HSK 1]
    \definition*{s.}{China}
  \end{phonetics}
\end{entry}

\begin{entry}{中国人}{4,8,2}
  \begin{phonetics}{中国人}{zhong1guo2ren2}
    \definition{s.}{chinês | pessoa ou povo da China}
  \end{phonetics}
\end{entry}

\begin{entry}{中国城}{4,8,9}
  \begin{phonetics}{中国城}{zhong1guo2cheng2}
    \definition*{s.}{Bairro Chinês, \emph{Chinatown}}
    \seeref{唐人街}{tang2ren2 jie1}
  \end{phonetics}
\end{entry}

\begin{entry}{中国科学院}{4,8,9,8,9}
  \begin{phonetics}{中国科学院}{zhong1guo2 ke1xue2yuan4}
    \definition*{s.}{Academia Chinesa de Ciências}
  \end{phonetics}
\end{entry}

\begin{entry}{中国通}{4,8,10}
  \begin{phonetics}{中国通}{zhong1guo2tong1}
    \definition*{s.}{Conhecedor da China, especialista em tudo sobre a China}
  \end{phonetics}
\end{entry}

\begin{entry}{中学}{4,8}
  \begin{phonetics}{中学}{zhong1xue2}[][HSK 1]
    \definition[个]{s.}{escola ensino médio}
  \end{phonetics}
\end{entry}

\begin{entry}{中学生}{4,8,5}
  \begin{phonetics}{中学生}{zhong1xue2sheng1}[][HSK 1]
    \definition{s.}{aluno, estudante de escola ensino médio}
  \end{phonetics}
\end{entry}

\begin{entry}{中性}{4,8}
  \begin{phonetics}{中性}{zhong1xing4}
    \definition{adj.}{neutro}
  \end{phonetics}
\end{entry}

\begin{entry}{中询}{4,8}
  \begin{phonetics}{中询}{zhong1 xun2}
    \definition{adv.}{segunda dezena do mês | meio do mês | em meados do mês}
  \end{phonetics}
\end{entry}

\begin{entry}{中秋节}{4,9,5}
  \begin{phonetics}{中秋节}{zhong1qiu1jie2}
    \definition*{s.}{Festival do Meio-Outono | Festival do Bolo Lunar (15º dia do oitavo mês lunar)}
  \end{phonetics}
\end{entry}

\begin{entry}{中药}{4,9}
  \begin{phonetics}{中药}{zhong1yao4}
    \definition[服,种]{s.}{medicina tradicional chinesa}
  \end{phonetics}
\end{entry}

\begin{entry}{中情局}{4,11,7}
  \begin{phonetics}{中情局}{zhong1qing2ju2}
    \definition*{s.}{Agência Central de Inteligência dos EUA, CIA (abreviação de 中央情报局)}
    \seeref{中央情报局}{zhong1yang1 qing2bao4ju2}
  \end{phonetics}
\end{entry}

\begin{entry}{中意}{4,13}
  \begin{phonetics}{中意}{zhong4yi4}
    \definition{s.}{ser do seu agrado | começar a gostar muito de algo ou de alguém}
  \end{phonetics}
\end{entry}

\begin{entry}{中餐}{4,16}
  \begin{phonetics}{中餐}{zhong1 can1}[][HSK 2]
    \definition[分,顿]{s.}{comida chinesa | almoço}
  \end{phonetics}
\end{entry}

\begin{entry}{丰收}{4,6}
  \begin{phonetics}{丰收}{feng1shou1}
    \definition{s.}{colheita abundante}
  \end{phonetics}
\end{entry}

\begin{entry}{为}{4}[Radical 丶]
  \begin{phonetics}{为}{wei2}[][HSK 0]
    \definition{prep.}{como (na capacidade de) | por (na voz passiva)}
    \definition{v.}{tomar algo como | agir como | servir como | comportar-se como | tornar-se}
  \end{phonetics}
  \begin{phonetics}{为}{wei4}[][HSK 2]
    \definition{prep.}{para | porque}
  \end{phonetics}
\end{entry}

\begin{entry}{为什么}{4,4,3}
  \begin{phonetics}{为什么}{wei4shen2me5}[][HSK 2]
    \definition{adv.}{por que?}
  \end{phonetics}
\end{entry}

\begin{entry}{乌克兰}{4,7,5}
  \begin{phonetics}{乌克兰}{wu1ke4lan2}
    \definition*{s.}{Ucrânia}
  \end{phonetics}
\end{entry}

\begin{entry}{乌龟}{4,7}
  \begin{phonetics}{乌龟}{wu1gui1}
    \definition{s.}{tartaruga}
  \end{phonetics}
\end{entry}

\begin{entry}{书}{4}[Radical 乙]
  \begin{phonetics}{书}{shu1}[][HSK 1]
    \definition[本,册,部]{s.}{livro | carta | documento}
  \end{phonetics}
\end{entry}

\begin{entry}{书包}{4,5}
  \begin{phonetics}{书包}{shu1bao1}[][HSK 1]
    \definition[个,款]{s.}{mochila escolar}
  \end{phonetics}
\end{entry}

\begin{entry}{书记}{4,5}
  \begin{phonetics}{书记}{shu1ji5}
    \definition{s.}{secretário (chefe de um ramo de um partido socialista ou comunista) | atendente | balconista | escriturário}
  \end{phonetics}
\end{entry}

\begin{entry}{书店}{4,8}
  \begin{phonetics}{书店}{shu1dian4}[][HSK 1]
    \definition[家]{s.}{livraria}
  \end{phonetics}
\end{entry}

\begin{entry}{云}{4}[Radical 二]
  \begin{phonetics}{云}{yun2}[][HSK 2]
    \definition*{s.}{sobrenome Yun}
    \definition[朵]{s.}{nuvem}
  \end{phonetics}
\end{entry}

\begin{entry}{云云}{4,4}
  \begin{phonetics}{云云}{yun2yun2}
    \definition{adv.}{e assim por diante | assim e assim}
  \end{phonetics}
\end{entry}

\begin{entry}{云南}{4,9}
  \begin{phonetics}{云南}{yun2nan2}
    \definition*{s.}{Yunnan}
  \end{phonetics}
\end{entry}

\begin{entry}{云端}{4,14}
  \begin{phonetics}{云端}{yun2duan1}
    \definition{s.}{alto nas nuvens | (computação) a nuvem}
  \end{phonetics}
\end{entry}

\begin{entry}{互}{4}[Radical ⼆]
  \begin{phonetics}{互}{hu4}
    \definition{adj.}{mútuo | recíproco}
  \end{phonetics}
\end{entry}

\begin{entry}{互动}{4,6}
  \begin{phonetics}{互动}{hu4dong4}
    \definition{s.}{interativo}
    \definition{v.}{interagir}
  \end{phonetics}
\end{entry}

\begin{entry}{互利}{4,7}
  \begin{phonetics}{互利}{hu4li4}
    \definition{s.}{benefício mútuo}
  \end{phonetics}
\end{entry}

\begin{entry}{互相}{4,9}
  \begin{phonetics}{互相}{hu4xiang1}
    \definition{adv.}{mutuamente | um ao outro}
  \end{phonetics}
\end{entry}

\begin{entry}{五}{4}[Radical 二]
  \begin{phonetics}{五}{wu3}[][HSK 1]
    \definition{num.}{cinco; 5}
  \end{phonetics}
\end{entry}

\begin{entry}{五五}{4,4}
  \begin{phonetics}{五五}{wu3wu3}
    \definition{num.}{50-50}
    \definition{s.}{igual (partilha, parceria, etc.)}
  \end{phonetics}
\end{entry}

\begin{entry}{五体投地}{4,7,7,6}
  \begin{phonetics}{五体投地}{wu3ti3tou2di4}
    \definition{expr.}{prostrar-se em admiração | adular alguém}
  \end{phonetics}
\end{entry}

\begin{entry}{井}{4}[Radical 二][Kangxi 7]
  \begin{phonetics}{井}{jing3}
    \definition{adj.}{puro | ordenado}
    \definition[口]{s.}{poço}
  \end{phonetics}
\end{entry}

\begin{entry}{什么}{4,3}
  \begin{phonetics}{什么}{shen2me5}[][HSK 1]
    \definition{pron.}{que? | o que?}
    \definition{pron.}{algo | qualquer coisa}
  \end{phonetics}
\end{entry}

\begin{entry}{什么时候}{4,3,7,10}
  \begin{phonetics}{什么时候}{shen2me5shi2hou5}
    \definition{adv.}{quando? | a que horas?}
  \end{phonetics}
\end{entry}

\begin{entry}{什么样}{4,3,10}
  \begin{phonetics}{什么样}{shen2 me5 yang4}[][HSK 2]
    \definition{pron.}{que tipo? | o quê? | que tipo?}
  \end{phonetics}
\end{entry}

\begin{entry}{仅}{4}[Radical 人]
  \begin{phonetics}{仅}{jin3}
    \definition{adv.}{apenas | meramente}
  \end{phonetics}
\end{entry}

\begin{entry}{仅仅}{4,4}
  \begin{phonetics}{仅仅}{jin3jin3}
    \definition{adv.}{meramente | somente | apenas}
  \end{phonetics}
\end{entry}

\begin{entry}{今天}{4,4}
  \begin{phonetics}{今天}{jin1tian1}[][HSK 1]
    \definition{adv.}{hoje | no presente | agora}
  \end{phonetics}
\end{entry}

\begin{entry}{今后}{4,6}
  \begin{phonetics}{今后}{jin1 hou4}[][HSK 2]
    \definition{s.}{de agora em diante | daqui em diante | no futuro}
  \end{phonetics}
\end{entry}

\begin{entry}{今年}{4,6}
  \begin{phonetics}{今年}{jin1nian2}[][HSK 1]
    \definition{adv.}{este ano}
  \end{phonetics}
\end{entry}

\begin{entry}{介绍}{4,8}
  \begin{phonetics}{介绍}{jie4shao4}[][HSK 1]
    \definition{s.}{introdução | apresentação}
    \definition{v.}{fazer uma apresentação | apresentar (alguém para alguém) | apresentar (alguém para um emprego, etc.)}
  \end{phonetics}
\end{entry}

\begin{entry}{仍然}{4,12}
  \begin{phonetics}{仍然}{reng2ran2}
    \definition{adv.}{ainda}
  \end{phonetics}
\end{entry}

\begin{entry}{从}{4}[Radical ⼈]
  \begin{phonetics}{从}{cong2}[][HSK 1]
    \definition*{s.}{sobrenome Cong}
    \definition{prep.}{de | desde | a partir de}
  \end{phonetics}
\end{entry}

\begin{entry}{从小}{4,3}
  \begin{phonetics}{从小}{cong2 xiao3}[][HSK 2]
    \definition{adv.}{desde a infância | desde muito jovem | quando criança}
  \end{phonetics}
\end{entry}

\begin{entry}{从不}{4,4}
  \begin{phonetics}{从不}{cong2bu4}
    \definition{adv.}{nunca}
  \end{phonetics}
\end{entry}

\begin{entry}{从未}{4,5}
  \begin{phonetics}{从未}{cong2wei4}
    \definition{adv.}{nunca}
  \end{phonetics}
\end{entry}

\begin{entry}{从而}{4,6}
  \begin{phonetics}{从而}{cong2'er2}
    \definition{conj.}{assim | desse modo}
  \end{phonetics}
\end{entry}

\begin{entry}{从来}{4,7}
  \begin{phonetics}{从来}{cong2lai2}
    \definition{adv.}{do passado até o presente | o tempo todo | sempre | nunca (se usado em uma sentença negativa)}
  \end{phonetics}
\end{entry}

\begin{entry}{以上}{4,3}
  \begin{phonetics}{以上}{yi3 shang4}[][HSK 2]
    \definition{s.}{mais que | sobre | acima | o acima | o precedente | o acima mencionado}
  \end{phonetics}
\end{entry}

\begin{entry}{以下}{4,3}
  \begin{phonetics}{以下}{yi3 xia4}[][HSK 2]
    \definition[所]{s.}{abaixo | sob | seguinte}
  \end{phonetics}
\end{entry}

\begin{entry}{以及}{4,3}
  \begin{phonetics}{以及}{yi3ji2}
    \definition{conj.}{assim como | juntamente como}
  \end{phonetics}
\end{entry}

\begin{entry}{以为}{4,4}
  \begin{phonetics}{以为}{yi3wei2}[][HSK 2]
    \definition{v.}{pensar, ou seja, considerar que\dots (geralmente há uma implicação de que a noção está errada --- exceto ao expressar a própria opnião atual)}
  \end{phonetics}
\end{entry}

\begin{entry}{以外}{4,5}
  \begin{phonetics}{以外}{yi3 wai4}[][HSK 2]
    \definition{s.}{além | exceto | fora | diferente de}
  \end{phonetics}
\end{entry}

\begin{entry}{以后}{4,6}
  \begin{phonetics}{以后}{yi3 hou4}[][HSK 2]
    \definition{adv.}{depois de | depois | após}
  \end{phonetics}
\end{entry}

\begin{entry}{以此}{4,6}
  \begin{phonetics}{以此}{yi3ci3}
    \definition{adv.}{devido a esta | deste modo | por isso | com isso}
  \end{phonetics}
\end{entry}

\begin{entry}{以至}{4,6}
  \begin{phonetics}{以至}{yi3zhi4}
    \definition{adv.}{até}
    \definition{conj.}{a tal ponto que\dots}
  \seealsoref{以至于}{yi3zhi4yu2}
  \end{phonetics}
\end{entry}

\begin{entry}{以至于}{4,6,3}
  \begin{phonetics}{以至于}{yi3zhi4yu2}
    \definition{adv.}{até}
    \definition{conj.}{na medida em que\dots}
  \seealsoref{以至}{yi3zhi4}
  \end{phonetics}
\end{entry}

\begin{entry}{以色列}{4,6,6}
  \begin{phonetics}{以色列}{yi3se4lie4}
    \definition*{s.}{Israel}
  \end{phonetics}
\end{entry}

\begin{entry}{以免}{4,7}
  \begin{phonetics}{以免}{yi3mian3}
    \definition{conj.}{para evitar isso}
  \end{phonetics}
\end{entry}

\begin{entry}{以来}{4,7}
  \begin{phonetics}{以来}{yi3lai2}
    \definition{prep.}{desde (um evento anterior)}
  \end{phonetics}
\end{entry}

\begin{entry}{以求}{4,7}
  \begin{phonetics}{以求}{yi3qiu2}
    \definition{conj.}{a fim de}
  \end{phonetics}
\end{entry}

\begin{entry}{以便}{4,9}
  \begin{phonetics}{以便}{yi3bian4}
    \definition{conj.}{a fim de | para que | assim como}
  \end{phonetics}
\end{entry}

\begin{entry}{以前}{4,9}
  \begin{phonetics}{以前}{yi3qian2}[][HSK 2]
    \definition{adv.}{antes de | antes}
  \end{phonetics}
\end{entry}

\begin{entry}{以期}{4,12}
  \begin{phonetics}{以期}{yi3qi1}
    \definition{v.}{tentando | esperando | esperando por}
  \end{phonetics}
\end{entry}

\begin{entry}{元}{4}[Radical 儿]
  \begin{phonetics}{元}{yuan2}[][HSK 1]
    \definition*{s.}{sobrenome Yuan | Dinastia Yuan (1279-1368)}
    \definition{clas.}{unidade monetária da China}
  \end{phonetics}
\end{entry}

\begin{entry}{元气}{4,4}
  \begin{phonetics}{元气}{yuan2qi4}
    \definition{s.}{força | vigor | vitalidade | energial vital}
  \end{phonetics}
\end{entry}

\begin{entry}{元旦}{4,5}
  \begin{phonetics}{元旦}{yuan2dan4}
    \definition*{s.}{Dia de Ano Novo (1 de janeiro)}
  \end{phonetics}
\end{entry}

\begin{entry}{元来}{4,7}
  \begin{phonetics}{元来}{yuan2lai2}
    \variantof{原来}
  \end{phonetics}
\end{entry}

\begin{entry}{元夜}{4,8}
  \begin{phonetics}{元夜}{yuan2ye4}
    \definition*{s.}{Festival das Lanternas}
  \seealsoref{元宵}{yuan2xiao1}
  \seealsoref{元宵节}{yuan2xiao1jie2}
  \end{phonetics}
\end{entry}

\begin{entry}{元宵}{4,10}
  \begin{phonetics}{元宵}{yuan2xiao1}
    \definition*{s.}{Festival das Lanternas}
  \seealsoref{元宵节}{yuan2xiao1jie2}
  \seealsoref{元夜}{yuan2ye4}
  \end{phonetics}
\end{entry}

\begin{entry}{元宵节}{4,10,5}
  \begin{phonetics}{元宵节}{yuan2xiao1jie2}
    \definition*{s.}{Festival das Lanternas (15º~dia do primeiro mês lunar)}
  \seealsoref{元宵}{yuan2xiao1}
  \seealsoref{元夜}{yuan2ye4}
  \end{phonetics}
\end{entry}

\begin{entry}{公元}{4,4}
  \begin{phonetics}{公元}{gong1yuan2}
    \definition{s.}{D.C. (Depois de~Cristo)}
  \seealsoref{前}{qian2}
    \example{公元293年}[293 d.C.]
  \end{phonetics}
\end{entry}

\begin{entry}{公开}{4,4}
  \begin{phonetics}{公开}{gong1kai1}
    \definition{s.}{aberto | público}
    \definition{v.}{tornar público | liberar}
  \end{phonetics}
\end{entry}

\begin{entry}{公斤}{4,4}
  \begin{phonetics}{公斤}{gong1jin1}[][HSK 2]
    \definition{clas.}{quilograma (kg)}
  \end{phonetics}
\end{entry}

\begin{entry}{公车}{4,4}
  \begin{phonetics}{公车}{gong1che1}
    \definition{s.}{abreviação de~公共汽车, ônibus}
    \seeref{公共汽车}{gong1gong4qi4che1}
  \end{phonetics}
\end{entry}

\begin{entry}{公司}{4,5}
  \begin{phonetics}{公司}{gong1si1}[][HSK 2]
    \definition[家]{s.}{empresa | companhia | corporação | firma}
  \end{phonetics}
\end{entry}

\begin{entry}{公司治理}{4,5,8,11}
  \begin{phonetics}{公司治理}{gong1si1zhi4li3}
    \definition{s.}{governança corporativa}
  \end{phonetics}
\end{entry}

\begin{entry}{公平}{4,5}
  \begin{phonetics}{公平}{gong1ping2}[][HSK 2]
    \definition{adj.}{justo | imparcial | equitativo}
  \end{phonetics}
\end{entry}

\begin{entry}{公用电话}{4,5,5,8}
  \begin{phonetics}{公用电话}{gong1yong4dian4hua4}
    \definition[部]{s.}{telefone público}
  \end{phonetics}
\end{entry}

\begin{entry}{公交车}{4,6,4}
  \begin{phonetics}{公交车}{gong1 jiao1 che1}[][HSK 2]
    \definition[辆]{s.}{ônibus urbano | veículo de transporte público}
  \end{phonetics}
\end{entry}

\begin{entry}{公共汽车}{4,6,7,4}
  \begin{phonetics}{公共汽车}{gong1gong4qi4che1}[][HSK 2]
    \definition[辆,班]{s.}{ônibus}
    \seeref{公车}{gong1che1}
  \end{phonetics}
\end{entry}

\begin{entry}{公克}{4,7}
  \begin{phonetics}{公克}{gong1ke4}
    \definition{s.}{grama (medida de peso)}
  \end{phonetics}
\end{entry}

\begin{entry}{公园}{4,7}
  \begin{phonetics}{公园}{gong1yuan2}[][HSK 2]
    \definition[座]{s.}{parque (para recreação pública)}
  \end{phonetics}
\end{entry}

\begin{entry}{公里}{4,7}
  \begin{phonetics}{公里}{gong1li3}[][HSK 2]
    \definition{s.}{quilômetro}
  \end{phonetics}
\end{entry}

\begin{entry}{公寓}{4,12}
  \begin{phonetics}{公寓}{gong1yu4}
    \definition[套]{s.}{prédio de apartamentos | pensão}
  \end{phonetics}
\end{entry}

\begin{entry}{公路}{4,13}
  \begin{phonetics}{公路}{gong1 lu4}[][HSK 2]
    \definition[条]{s.}{rodovia | via de trânsito | estrada | auto-estrada}
  \end{phonetics}
\end{entry}

\begin{entry}{六}{4}[Radical 八]
  \begin{phonetics}{六}{liu4}[][HSK 1]
    \definition{num.}{seis; 6}
  \end{phonetics}
\end{entry}

\begin{entry}{内存}{4,6}
  \begin{phonetics}{内存}{nei4cun2}
    \definition{s.}{armazenamento interno | memória do computador | RAM (\emph{random access memory})}
  \seealsoref{随机存取存储器}{sui2ji1cun2qu3cun2chu3qi4}
  \seealsoref{随机存取记忆体}{sui2ji1cun2qu3ji4yi4ti3}
  \end{phonetics}
\end{entry}

\begin{entry}{内省}{4,9}
  \begin{phonetics}{内省}{nei4xing3}
    \definition{s.}{introspecção}
    \definition{v.}{refletir sobre si mesmo}
  \end{phonetics}
\end{entry}

\begin{entry}{内燃机}{4,16,6}
  \begin{phonetics}{内燃机}{nei4ran2ji1}
    \definition{s.}{motor de combustão interna}
  \end{phonetics}
\end{entry}

\begin{entry}{凤凰}{4,11}
  \begin{phonetics}{凤凰}{feng4huang2}
    \definition{s.}{fênix}
  \end{phonetics}
\end{entry}

\begin{entry}{分}{4}[Radical 刀]
  \begin{phonetics}{分}{fen1}[][HSK 1]
    \definition{s.}{parte ou subdivisão | fração | um décimo (de certas unidades) | unidade de comprimento equivalente a 0,33cm | minuto (unidade de tempo) | minuto (unidade de medida angular) | um ponto (em esportes e jogos) | 0,01 yuan (unidade de dinheiro)}
    \definition{v.}{dividir | separar | distribuir | atribuir | distinguir (bom e mau)}
  \end{phonetics}
  \begin{phonetics}{分}{fen4}[][HSK 2]
    \definition{s.}{parte | ingrediente | componente}
  \end{phonetics}
\end{entry}

\begin{entry}{分子}{4,3}
  \begin{phonetics}{分子}{fen1zi3}
    \definition{s.}{molécula | (matemática) numerador de uma fração}
  \end{phonetics}
  \begin{phonetics}{分子}{fen4zi3}
    \definition{s.}{membros de uma classe ou grupo | elementos políticos (como intelectuais ou extremistas)}
  \end{phonetics}
\end{entry}

\begin{entry}{分公司}{4,4,5}
  \begin{phonetics}{分公司}{fen1gong1si1}
    \definition{s.}{sucursal | filial de companhia}
  \end{phonetics}
\end{entry}

\begin{entry}{分开}{4,4}
  \begin{phonetics}{分开}{fen1 kai1}[][HSK 2]
    \definition{v.+compl.}{separar | dividir | desacoplar | desempacotar | quebrar | desmembrar | romper | desfazer | desvincular | distribuir | separar de (em) | dividir ... de ... | separar de}
  \end{phonetics}
\end{entry}

\begin{entry}{分手}{4,4}
  \begin{phonetics}{分手}{fen1shou3}
    \definition{v.+compl.}{separar | separar-se do companheiro | dizer adeus}
  \end{phonetics}
\end{entry}

\begin{entry}{分钟}{4,9}
  \begin{phonetics}{分钟}{fen1zhong1}[][HSK 2]
    \definition{s.}{minuto (usado em intervalos de tempo)}
  \end{phonetics}
\end{entry}

\begin{entry}{分量}{4,12}
  \begin{phonetics}{分量}{fen1liang4}
    \definition{s.}{componente vetorial}
  \end{phonetics}
  \begin{phonetics}{分量}{fen4liang4}
    \definition{s.}{tamanho da porção (comida)}
  \end{phonetics}
  \begin{phonetics}{分量}{fen4liang5}
    \definition{s.}{quantidade | peso | medida}
  \end{phonetics}
\end{entry}

\begin{entry}{分数}{4,13}
  \begin{phonetics}{分数}{fen1 shu4}[][HSK 2]
    \definition{s.}{fração | número fracionário | marca | nota | ponto}
  \end{phonetics}
\end{entry}

\begin{entry}{切割}{4,12}
  \begin{phonetics}{切割}{qie1ge1}
    \definition{v.}{cortar}
  \end{phonetics}
\end{entry}

\begin{entry}{办}{4}[Radical 力]
  \begin{phonetics}{办}{ban4}[][HSK 2]
    \definition{v.}{lidar com | lidar | gerenciar | configurar}
  \end{phonetics}
\end{entry}

\begin{entry}{办公}{4,4}
  \begin{phonetics}{办公}{ban4gong1}
    \definition{v.+compl.}{lidar com negócios oficiais | trabalhar (especialmente em um escritório)}
  \end{phonetics}
\end{entry}

\begin{entry}{办公室}{4,4,9}
  \begin{phonetics}{办公室}{ban4gong1shi4}[][HSK 2]
    \definition[间]{s.}{gabinete | escritório}
  \end{phonetics}
\end{entry}

\begin{entry}{办法}{4,8}
  \begin{phonetics}{办法}{ban4fa3}[][HSK 2]
    \definition[条,个]{s.}{meio (de se fazer alguma coisa) | método | medida}
  \end{phonetics}
\end{entry}

\begin{entry}{勾}{4}[Radical ⼓]
  \begin{phonetics}{勾}{gou1}
    \definition*{s.}{sobrenome Gou}
    \definition{v.}{atrair | excitar | marcar | atacar | delinear | conspirar}
    \variantof{钩}
  \end{phonetics}
  \begin{phonetics}{勾}{gou4}
    \definition{s.}{usado em 勾当}
    \seeref{勾当}{gou4dang4}
  \end{phonetics}
\end{entry}

\begin{entry}{勾当}{4,6}
  \begin{phonetics}{勾当}{gou4dang4}
    \definition{s.}{negócio obscuro}
  \end{phonetics}
\end{entry}

\begin{entry}{化}{4}[Radical 匕]
  \begin{phonetics}{化}{hua1}
    \variantof{花}
  \end{phonetics}
\end{entry}

\begin{entry}{化学}{4,8}
  \begin{phonetics}{化学}{hua4xue2}
    \definition{s.}{química (disciplina)}
  \end{phonetics}
\end{entry}

\begin{entry}{区}{4}[Radical 匸]
  \begin{phonetics}{区}{ou1}
    \definition*{s.}{sobrenome Ou}
  \end{phonetics}
  \begin{phonetics}{区}{qu1}
    \definition[个]{s.}{área | região | distrito}
  \end{phonetics}
\end{entry}

\begin{entry}{区域}{4,11}
  \begin{phonetics}{区域}{qu1yu4}
    \definition{s.}{área | região | distrito}
  \end{phonetics}
\end{entry}

\begin{entry}{升起}{4,10}
  \begin{phonetics}{升起}{sheng1qi3}
    \definition{v.}{levantar | içar | subir}
  \end{phonetics}
\end{entry}

\begin{entry}{午}{4}[Radical 十]
  \begin{phonetics}{午}{wu3}
    \definition{s.}{período entre 11h00 e 13h00, meio-dia}
  \end{phonetics}
\end{entry}

\begin{entry}{午休}{4,6}
  \begin{phonetics}{午休}{wu3xiu1}
    \definition{s.}{pausa para almoço | cochilo na hora do almoço | intervalo do meio-dia}
  \end{phonetics}
\end{entry}

\begin{entry}{午后}{4,6}
  \begin{phonetics}{午后}{wu3hou4}
    \definition{s.}{tarde | período da tarde}
  \end{phonetics}
\end{entry}

\begin{entry}{午饭}{4,7}
  \begin{phonetics}{午饭}{wu3fan4}[][HSK 1]
    \definition[份,顿,次,餐]{s.}{almoço}
  \seealsoref{午餐}{wu3can1}
  \end{phonetics}
\end{entry}

\begin{entry}{午夜}{4,8}
  \begin{phonetics}{午夜}{wu3ye4}
    \definition{s.}{meia-noite}
  \end{phonetics}
\end{entry}

\begin{entry}{午前}{4,9}
  \begin{phonetics}{午前}{wu3qian2}
    \definition{s.}{\emph{A.M.} | manhã | período da manhã}
  \end{phonetics}
\end{entry}

\begin{entry}{午宴}{4,10}
  \begin{phonetics}{午宴}{wu3yan4}
    \definition{s.}{banquete de almoço}
  \end{phonetics}
\end{entry}

\begin{entry}{午睡}{4,13}
  \begin{phonetics}{午睡}{wu3 shui4}[][HSK 2]
    \definition{s.}{siesta}
    \definition{v.}{tirar uma soneca}
  \end{phonetics}
\end{entry}

\begin{entry}{午餐}{4,16}
  \begin{phonetics}{午餐}{wu3 can1}[][HSK 2]
    \definition[份,顿,次]{s.}{almoço}
  \seealsoref{午饭}{wu3fan4}
  \end{phonetics}
\end{entry}

\begin{entry}{历史}{4,5}
  \begin{phonetics}{历史}{li4shi3}
    \definition[门,段]{s.}{história}
  \end{phonetics}
\end{entry}

\begin{entry}{友好}{4,6}
  \begin{phonetics}{友好}{you3hao3}[][HSK 2]
    \definition{adj.}{amigável}
    \definition{s.}{amigo próximo, íntimo}
  \end{phonetics}
\end{entry}

\begin{entry}{双}{4}[Radical 又]
  \begin{phonetics}{双}{shuang1}
    \definition*{s.}{sobrenome Shuang}
    \definition{s.}{dobro | par | dupla | ambos | número par}
  \end{phonetics}
\end{entry}

\begin{entry}{双方同意}{4,4,6,13}
  \begin{phonetics}{双方同意}{shuang1fang1tong2yi4}
    \definition{s.}{acordo bilateral}
  \end{phonetics}
\end{entry}

\begin{entry}{双打}{4,5}
  \begin{phonetics}{双打}{shuang1da3}
    \definition[场]{s.}{duplas (em esportes)}
  \end{phonetics}
\end{entry}

\begin{entry}{双层床}{4,7,7}
  \begin{phonetics}{双层床}{shuang1ceng2chuang2}
    \definition{s.}{beliche}
  \end{phonetics}
\end{entry}

\begin{entry}{反对}{4,5}
  \begin{phonetics}{反对}{fan3dui4}
    \definition{v.}{contrariar | opor-se | lutar contra}
  \end{phonetics}
\end{entry}

\begin{entry}{反对派}{4,5,9}
  \begin{phonetics}{反对派}{fan3dui4pai4}
    \definition{s.}{facção de oposição}
  \end{phonetics}
\end{entry}

\begin{entry}{反对党}{4,5,10}
  \begin{phonetics}{反对党}{fan3dui4dang3}
    \definition{s.}{partido de oposição}
  \end{phonetics}
\end{entry}

\begin{entry}{反对票}{4,5,11}
  \begin{phonetics}{反对票}{fan3dui4piao4}
    \definition{s.}{voto dissidente}
  \end{phonetics}
\end{entry}

\begin{entry}{反正}{4,5}
  \begin{phonetics}{反正}{fan3zheng4}
    \definition{adv.}{de qualquer maneira | em qualquer caso | aconteça o que acontecer}
  \end{phonetics}
\end{entry}

\begin{entry}{反应}{4,7}
  \begin{phonetics}{反应}{fan3ying4}
    \definition[个]{s.}{reação | resposta | reação química}
    \definition{v.}{reagir | responder}
  \end{phonetics}
\end{entry}

\begin{entry}{反复}{4,9}
  \begin{phonetics}{反复}{fan3fu4}
    \definition{adv.}{de novo e de novo | repetidamente}
  \end{phonetics}
\end{entry}

\begin{entry}{反省}{4,9}
  \begin{phonetics}{反省}{fan3xing3}
    \definition{v.}{examinar a consciência | questionar-se | refletir sobre si mesmo | sondar a alma}
  \end{phonetics}
\end{entry}

\begin{entry}{天}{4}[Radical 大]
  \begin{phonetics}{天}{tian1}[][HSK 1]
    \definition{s.}{dia | céu | paraíso}
  \end{phonetics}
\end{entry}

\begin{entry}{天上}{4,3}
  \begin{phonetics}{天上}{tian1 shang4}[][HSK 2]
    \definition{s.}{o céu | paraíso}
  \end{phonetics}
\end{entry}

\begin{entry}{天下}{4,3}
  \begin{phonetics}{天下}{tian1xia4}
    \definition{s.}{terra sob o céu | o mundo todo | toda a China | reino}
  \end{phonetics}
\end{entry}

\begin{entry}{天才}{4,3}
  \begin{phonetics}{天才}{tian1cai2}
    \definition{adj.}{talentoso | superdotado | genial}
    \definition{s.}{talento | dom | gênio}
  \end{phonetics}
\end{entry}

\begin{entry}{天公}{4,4}
  \begin{phonetics}{天公}{tian1gong1}
    \definition{s.}{céu, paraíso | senhor do céu}
  \end{phonetics}
\end{entry}

\begin{entry}{天天}{4,4}
  \begin{phonetics}{天天}{tian1tian1}
    \definition{adv.}{todo dia}
  \end{phonetics}
\end{entry}

\begin{entry}{天气}{4,4}
  \begin{phonetics}{天气}{tian1qi4}[][HSK 1]
    \definition{s.}{clima, tempo}
  \end{phonetics}
\end{entry}

\begin{entry}{天花板}{4,7,8}
  \begin{phonetics}{天花板}{tian1hua1ban3}
    \definition{s.}{teto}
  \end{phonetics}
\end{entry}

\begin{entry}{天使}{4,8}
  \begin{phonetics}{天使}{tian1shi3}
    \definition{s.}{anjo}
  \end{phonetics}
\end{entry}

\begin{entry}{天择}{4,8}
  \begin{phonetics}{天择}{tian1ze2}
    \definition{s.}{seleção natural}
  \end{phonetics}
\end{entry}

\begin{entry}{天柱}{4,9}
  \begin{phonetics}{天柱}{tian1zhu4}
    \definition{s.}{pilar celestial, que sustenta o céu}
  \end{phonetics}
\end{entry}

\begin{entry}{天堂}{4,11}
  \begin{phonetics}{天堂}{tian1tang2}
    \definition{s.}{paraíso, céu}
  \end{phonetics}
\end{entry}

\begin{entry}{天然}{4,12}
  \begin{phonetics}{天然}{tian1ran2}
    \definition{adj.}{natural}
  \end{phonetics}
\end{entry}

\begin{entry}{天鹅}{4,12}
  \begin{phonetics}{天鹅}{tian1'e2}
    \definition{s.}{cisne}
  \end{phonetics}
\end{entry}

\begin{entry}{太}{4}[Radical 大]
  \begin{phonetics}{太}{tai4}[][HSK 1]
    \definition{adv.}{excessivamente | demais | muito}
  \end{phonetics}
\end{entry}

\begin{entry}{太太}{4,4}
  \begin{phonetics}{太太}{tai4tai5}[][HSK 2]
    \definition[个,位]{s.}{esposa | madame| mulher casada}
  \end{phonetics}
\end{entry}

\begin{entry}{太平洋}{4,5,9}
  \begin{phonetics}{太平洋}{tai4ping2 yang2}
    \definition*{s.}{Oceano Pacífico}
  \end{phonetics}
\end{entry}

\begin{entry}{太阳}{4,6}
  \begin{phonetics}{太阳}{tai4yang5}[][HSK 2]
    \definition[个]{s.}{sol | abreviação de 太阳穴}
    \seeref{太阳穴}{tai4yang2xue2}
  \end{phonetics}
\end{entry}

\begin{entry}{太阳日}{4,6,4}
  \begin{phonetics}{太阳日}{tai4yang2ri4}
    \definition{s.}{dia solar}
  \end{phonetics}
\end{entry}

\begin{entry}{太阳风}{4,6,4}
  \begin{phonetics}{太阳风}{tai4yang2feng1}
    \definition{s.}{vento solar}
  \end{phonetics}
\end{entry}

\begin{entry}{太阳穴}{4,6,5}
  \begin{phonetics}{太阳穴}{tai4yang2xue2}
    \definition{s.}{têmpora (nas laterais da cabeça humana)}
  \end{phonetics}
\end{entry}

\begin{entry}{太阳灯}{4,6,6}
  \begin{phonetics}{太阳灯}{tai4yang2deng1}
    \definition{s.}{lâmpada solar (com células fotovoltaicas)}
  \end{phonetics}
\end{entry}

\begin{entry}{太阳雨}{4,6,8}
  \begin{phonetics}{太阳雨}{tai4yang2yu3}
    \definition{s.}{banho de sol}
  \end{phonetics}
\end{entry}

\begin{entry}{太阳窗}{4,6,12}
  \begin{phonetics}{太阳窗}{tai4yang2chuang1}
    \definition{s.}{teto solar (de veículos)}
  \end{phonetics}
\end{entry}

\begin{entry}{太阳镜}{4,6,16}
  \begin{phonetics}{太阳镜}{tai4yang2jing4}
    \definition{s.}{óculos de sol}
  \end{phonetics}
\end{entry}

\begin{entry}{太阳翼}{4,6,17}
  \begin{phonetics}{太阳翼}{tai4yang2yi4}
    \definition{s.}{painel solar}
  \end{phonetics}
\end{entry}

\begin{entry}{太极拳}{4,7,10}
  \begin{phonetics}{太极拳}{tai4ji2quan2}
    \definition*{s.}{Tai Chi Chuan, Taiji, T'aichi ou T'aichichuan; forma tradicional de exercício físico ou relaxamento}
  \end{phonetics}
\end{entry}

\begin{entry}{太空}{4,8}
  \begin{phonetics}{太空}{tai4kong1}
    \definition{s.}{espaço sideral | espaço exterior}
  \end{phonetics}
\end{entry}

\begin{entry}{夫妻}{4,8}
  \begin{phonetics}{夫妻}{fu1qi1}
    \definition{s.}{casal | marido e eposa}
  \end{phonetics}
\end{entry}

\begin{entry}{孔}{4}[Radical 子]
  \begin{phonetics}{孔}{kong3}
    \definition*{s.}{sobrenome Kong}
    \definition{clas.}{para habitações em cavernas}
    \definition[个]{s.}{buraco}
  \end{phonetics}
\end{entry}

\begin{entry}{孔子}{4,3}
  \begin{phonetics}{孔子}{kong3zi3}
    \definition*{s.}{Confúcio (551-479 aC), pensador e filósofo social chinês}
  \seealsoref{孔夫子}{kong3fu1zi3}
  \end{phonetics}
\end{entry}

\begin{entry}{孔子学院}{4,3,8,9}
  \begin{phonetics}{孔子学院}{kong3zi3 xue2yuan4}
    \definition*{s.}{Instituto Confúcio, organização estabelecida internacionalmente pela República Popular da China, que promove a língua e a cultura chinesas}
  \end{phonetics}
\end{entry}

\begin{entry}{孔夫子}{4,4,3}
  \begin{phonetics}{孔夫子}{kong3fu1zi3}
    \definition*{s.}{Confúcio (551-479 aC), pensador e filósofo social chinês}
  \seealsoref{孔子}{kong3zi3}
  \end{phonetics}
\end{entry}

\begin{entry}{孔雀}{4,11}
  \begin{phonetics}{孔雀}{kong3que4}
    \definition{s.}{pavão}
  \end{phonetics}
\end{entry}

\begin{entry}{少}{4}[Radical 小]
  \begin{phonetics}{少}{shao3}
    \definition{adj.}{pouco, poucos}
    \definition{v.}{sentir falta | faltar | parar (de fazer algo)}
  \end{phonetics}
  \begin{phonetics}{少}{shao4}
    \definition{s.}{jovem}
  \end{phonetics}
\end{entry}

\begin{entry}{少年}{4,6}
  \begin{phonetics}{少年}{shao4 nian2}[][HSK 2]
    \definition[个]{s.}{adolescente | juventude precoce | menor | juventude | adolescente}
  \end{phonetics}
\end{entry}

\begin{entry}{少数}{4,13}
  \begin{phonetics}{少数}{shao3 shu4}[][HSK 2]
    \definition{s.}{pequeno número | poucos | minoria}
  \end{phonetics}
\end{entry}

\begin{entry}{尤其}{4,8}
  \begin{phonetics}{尤其}{you2qi2}
    \definition{adv.}{especialmente | particularmente}
  \end{phonetics}
\end{entry}

\begin{entry}{巴西}{4,6}
  \begin{phonetics}{巴西}{ba1xi1}
    \definition*{s.}{Brasil}
  \end{phonetics}
\end{entry}

\begin{entry}{巴西人}{4,6,2}
  \begin{phonetics}{巴西人}{ba1xi1ren2}
    \definition[个,位]{s.}{brasileiro | pessoa ou povo do Brasil}
    \example{他是巴西人。}[Ele é brasileiro.]
  \end{phonetics}
\end{entry}

\begin{entry}{巴西战舞}{4,6,9,14}
  \begin{phonetics}{巴西战舞}{ba1xi1zhan4wu3}
    \definition{s.}{capoeira}
  \end{phonetics}
\end{entry}

\begin{entry}{巴勒斯坦}{4,11,12,8}
  \begin{phonetics}{巴勒斯坦}{ba1le4si1tan3}
    \definition*{s.}{Palestina}
  \end{phonetics}
\end{entry}

\begin{entry}{幻觉}{4,9}
  \begin{phonetics}{幻觉}{huan4jue2}
    \definition{s.}{ilusão | alucinação}
  \end{phonetics}
\end{entry}

\begin{entry}{开}{4}[Radical 廾]
  \begin{phonetics}{开}{kai1}[][HSK 1]
    \definition{clas.}{quilate (ouro)}
    \definition{v.}{abrir | ligar | dirigir | iniciar (alguma coisa) | começar | ferver | escrever  (uma receita, cheque, fatura, etc.) | operar (um veículo) | abreviação de Kelvin 开尔文}
    \seeref{开尔文}{kai1'er3wen2}
  \end{phonetics}
\end{entry}

\begin{entry}{开口}{4,3}
  \begin{phonetics}{开口}{kai1kou3}
    \definition{v.}{abrir a boca de alguém | começar a falar}
  \end{phonetics}
\end{entry}

\begin{entry}{开心}{4,4}
  \begin{phonetics}{开心}{kai1xin1}[][HSK 2]
    \definition{v.}{sentir-se feliz | regozijar-se | divertir-se | tirar sarro de alguém}
  \end{phonetics}
\end{entry}

\begin{entry}{开车}{4,4}
  \begin{phonetics}{开车}{kai1 che1}[][HSK 1]
    \definition{v.+compl.}{conduzir | dirigir}
  \end{phonetics}
\end{entry}

\begin{entry}{开发区}{4,5,4}
  \begin{phonetics}{开发区}{kai1fa1qu1}
    \definition{s.}{zona de desenvolvimento}
  \end{phonetics}
\end{entry}

\begin{entry}{开头}{4,5}
  \begin{phonetics}{开头}{kai1tou2}
    \definition{s.}{início | começo}
    \definition{v.+compl.}{iniciar | começar | fazer um começo}
  \end{phonetics}
\end{entry}

\begin{entry}{开尔文}{4,5,4}
  \begin{phonetics}{开尔文}{kai1'er3wen2}
    \definition{s.}{Kelvin, temperatura absoluta | K, escala de temperatura}
  \end{phonetics}
\end{entry}

\begin{entry}{开会}{4,6}
  \begin{phonetics}{开会}{kai1 hui4}[][HSK 1]
    \definition{v.+compl.}{realizar uma reunião | ter uma reunião | participar de uma reunião (conferência)}
  \end{phonetics}
\end{entry}

\begin{entry}{开机}{4,6}
  \begin{phonetics}{开机}{kai1 ji1}[][HSK 2]
    \definition{v.}{começar a filmar um filme ou programa de TV | iniciar uma máquina}
  \end{phonetics}
\end{entry}

\begin{entry}{开启}{4,7}
  \begin{phonetics}{开启}{kai1qi3}
    \definition{v.}{abrir | iniciar | (computação) ativar}
  \end{phonetics}
\end{entry}

\begin{entry}{开花}{4,7}
  \begin{phonetics}{开花}{kai1hua1}
    \definition{v.}{florescer | (fig.) explodir, abrir-se | (fig.) explodir de alegria | (fig.) começar a existir de repente em todos os lugares}
  \end{phonetics}
\end{entry}

\begin{entry}{开夜车}{4,8,4}
  \begin{phonetics}{开夜车}{kai1ye4che1}
    \definition{expr.}{trabalho noturno | (literalmente) ``conduzir carro à noite''}
  \end{phonetics}
\end{entry}

\begin{entry}{开始}{4,8}
  \begin{phonetics}{开始}{kai1shi3}
    \definition{adv.}{inicial}
    \definition[个]{s.}{começo | início}
    \definition{v.}{começar | iniciar}
  \end{phonetics}
\end{entry}

\begin{entry}{开学}{4,8}
  \begin{phonetics}{开学}{kai1 xue2}[][HSK 2]
    \definition{v.}{iniciar as aulas | iniciar o semestre | começar as aulas}
  \end{phonetics}
\end{entry}

\begin{entry}{开玩笑}{4,8,10}
  \begin{phonetics}{开玩笑}{kai1 wan2xiao4}[][HSK 1]
    \definition{v.}{contar uma piada | brincar | fazer piada de | pregar uma peça | provocar}
  \end{phonetics}
\end{entry}

\begin{entry}{开锁}{4,12}
  \begin{phonetics}{开锁}{kai1suo3}
    \definition{v.}{desbloquear | destravar}
  \end{phonetics}
\end{entry}

\begin{entry}{引擎}{4,16}
  \begin{phonetics}{引擎}{yin3qing2}
    \definition[台]{s.}{motor | (empréstimo linguístico) \emph{engine}}
  \end{phonetics}
\end{entry}

\begin{entry}{心中}{4,4}
  \begin{phonetics}{心中}{xin1zhong1}[][HSK 2]
    \definition{adv.}{nos pensamentos | no coração}
    \definition{s.}{ponto central}
  \end{phonetics}
\end{entry}

\begin{entry}{心机}{4,6}
  \begin{phonetics}{心机}{xin1ji1}
    \definition{s.}{pensamento | esquema}
  \end{phonetics}
\end{entry}

\begin{entry}{心声}{4,7}
  \begin{phonetics}{心声}{xin1sheng1}
    \definition{s.}{desejo sincero | voz interior | aspiração}
  \end{phonetics}
\end{entry}

\begin{entry}{心里}{4,7}
  \begin{phonetics}{心里}{xin1 li3}[][HSK 2]
    \definition[把]{s.}{no coração | no coração de alguém | na mente}
  \end{phonetics}
\end{entry}

\begin{entry}{心疼}{4,10}
  \begin{phonetics}{心疼}{xin1teng2}
    \definition{adj.}{angustiado}
    \definition{v.}{sentir pena de alguém | arrepender-se | ressentir-se | ficar angustiado}
  \end{phonetics}
\end{entry}

\begin{entry}{心情}{4,11}
  \begin{phonetics}{心情}{xin1qing2}[][HSK 2]
    \definition{s.}{humor | sentimento | estado de espírito}
  \end{phonetics}
\end{entry}

\begin{entry}{手}{4}[Radical 手][Kangxi 64]
  \begin{phonetics}{手}{shou3}[][HSK 1]
    \definition{adj.}{conveniente}
    \definition{clas.}{de habilidade}
    \definition[双,只]{s.}{mão | pessoa envolvida em certos tipos de trabalho | pessoa qualificada para certos tipos de trabalho}
    \definition{v.}{segurar (formal)}
  \end{phonetics}
\end{entry}

\begin{entry}{手工}{4,3}
  \begin{phonetics}{手工}{shou3gong1}
    \definition{s.}{trabalho manual | artesanato}
  \end{phonetics}
\end{entry}

\begin{entry}{手工艺人}{4,3,4,2}
  \begin{phonetics}{手工艺人}{shou3gong1 yi4ren2}
    \definition{s.}{artesão}
  \end{phonetics}
\end{entry}

\begin{entry}{手边}{4,5}
  \begin{phonetics}{手边}{shou3bian1}
    \definition{adv.}{à mão | na mão}
  \end{phonetics}
\end{entry}

\begin{entry}{手机}{4,6}
  \begin{phonetics}{手机}{shou3ji1}[][HSK 1]
    \definition[部,支]{s.}{telefone celular ou móvel}
  \end{phonetics}
\end{entry}

\begin{entry}{手刹}{4,8}
  \begin{phonetics}{手刹}{shou3sha1}
    \definition{s.}{freio de mão}
  \end{phonetics}
\end{entry}

\begin{entry}{手表}{4,8}
  \begin{phonetics}{手表}{shou3biao3}[][HSK 2]
    \definition[块,只,个]{s.}{relógio de pulso}
  \end{phonetics}
\end{entry}

\begin{entry}{手指}{4,9}
  \begin{phonetics}{手指}{shou3zhi3}
    \definition[个,只]{s.}{dedo}
  \end{phonetics}
\end{entry}

\begin{entry}{手臂}{4,17}
  \begin{phonetics}{手臂}{shou3bi4}
    \definition{s.}{braço}
  \end{phonetics}
\end{entry}

\begin{entry}{支}{4}[Radical 支][Kangxi 65]
  \begin{phonetics}{支}{zhi1}
    \definition*{s.}{sobrenome Zhi}
    \definition{clas.}{para varetas como canetas e armas | para divisões do exército e para canções ou composições}
    \definition{v.}{sacar dinheiro | erguer | criar | suportar | sustentar}
  \end{phonetics}
\end{entry}

\begin{entry}{支支吾吾}{4,4,7,7}
  \begin{phonetics}{支支吾吾}{zhi1zhi1wu2wu2}
    \definition{v.}{falhar | murmurar | paralisar | gaguejar}
  \end{phonetics}
\end{entry}

\begin{entry}{支应}{4,7}
  \begin{phonetics}{支应}{zhi1ying4}
    \definition{v.}{lidar com | fornecer}
  \end{phonetics}
\end{entry}

\begin{entry}{支承}{4,8}
  \begin{phonetics}{支承}{zhi1cheng2}
    \definition{v.}{suportar o peso de (um edifício) | suportar}
  \end{phonetics}
\end{entry}

\begin{entry}{支持}{4,9}
  \begin{phonetics}{支持}{zhi1chi2}
    \definition[个]{s.}{apoio | suporte}
    \definition{v.}{apoiar | ser a favor de | suportar}
  \end{phonetics}
\end{entry}

\begin{entry}{支根}{4,10}
  \begin{phonetics}{支根}{zhi1gen1}
    \definition{s.}{raiz ramificada | raízes de apoio | radícula}
  \end{phonetics}
\end{entry}

\begin{entry}{支票}{4,11}
  \begin{phonetics}{支票}{zhi1piao4}
    \definition[本]{s.}{cheque (banco)}
  \end{phonetics}
\end{entry}

\begin{entry}{文化}{4,4}
  \begin{phonetics}{文化}{wen2hua4}
    \definition[个,种]{s.}{cultura | civilização}
  \end{phonetics}
\end{entry}

\begin{entry}{文化水平}{4,4,4,5}
  \begin{phonetics}{文化水平}{wen2hua4 shui3ping2}
    \definition{s.}{nível educacional}
  \end{phonetics}
\end{entry}

\begin{entry}{文化史}{4,4,5}
  \begin{phonetics}{文化史}{wen2hua4shi3}
    \definition*{s.}{História Cultural}
  \end{phonetics}
\end{entry}

\begin{entry}{文化层}{4,4,7}
  \begin{phonetics}{文化层}{wen2hua4ceng2}
    \definition{s.}{nível de cultura (em sítio arqueológico)}
  \end{phonetics}
\end{entry}

\begin{entry}{文化宫}{4,4,9}
  \begin{phonetics}{文化宫}{wen2hua4gong1}
    \definition{s.}{palácio cultural}
  \end{phonetics}
\end{entry}

\begin{entry}{文化热}{4,4,10}
  \begin{phonetics}{文化热}{wen2hua4re4}
    \definition{s.}{mania cultural | febre cultural}
  \end{phonetics}
\end{entry}

\begin{entry}{文化圈}{4,4,11}
  \begin{phonetics}{文化圈}{wen2hua4quan1}
    \definition{s.}{esfera de influência cultural}
  \end{phonetics}
\end{entry}

\begin{entry}{文化障碍}{4,4,13,13}
  \begin{phonetics}{文化障碍}{wen2hua4zhang4'ai4}
    \definition{s.}{barreira cultural}
  \end{phonetics}
\end{entry}

\begin{entry}{文学系}{4,8,7}
  \begin{phonetics}{文学系}{wen2xue2 xi4}
    \definition*{s.}{Faculdade de Letras}
  \end{phonetics}
\end{entry}

\begin{entry}{文明}{4,8}
  \begin{phonetics}{文明}{wen2ming2}
    \definition{adj.}{civilizado}
    \definition[个]{s.}{civilização | cultura}
  \end{phonetics}
\end{entry}

\begin{entry}{斤}{4}[Radical 斤]
  \begin{phonetics}{斤}{jin1}[][HSK 2]
    \definition{clas.}{peso igual a 500 g}
  \end{phonetics}
\end{entry}

\begin{entry}{方向}{4,6}
  \begin{phonetics}{方向}{fang1xiang4}[][HSK 2]
    \definition[个]{s.}{direção | orientação | alvo | meta | objetivo}
  \end{phonetics}
\end{entry}

\begin{entry}{方言}{4,7}
  \begin{phonetics}{方言}{fang1yan2}
    \definition*{s.}{o primeiro dicionário de dialeto chinês, editado por Yang Xiong 扬雄 no século I, contendo mais de 9.000 caracteres}
    \definition{s.}{dialeto}
  \seealsoref{扬雄}{yang2xiong2}
  \end{phonetics}
\end{entry}

\begin{entry}{方法}{4,8}
  \begin{phonetics}{方法}{fang1fa3}[][HSK 2]
    \definition[个]{s.}{método | meio}
  \end{phonetics}
\end{entry}

\begin{entry}{方便}{4,9}
  \begin{phonetics}{方便}{fang1bian4}[][HSK 2]
    \definition{adj.}{conveniente | adequado}
    \definition{v.}{facilitar, facilitar as coisas | ter dinheiro de sobra | (eufemismo) aliviar-se}
  \end{phonetics}
\end{entry}

\begin{entry}{方便面}{4,9,9}
  \begin{phonetics}{方便面}{fang1 bian4 mian4}[][HSK 2]
    \definition{s.}{macarrão instantâneo}
  \end{phonetics}
\end{entry}

\begin{entry}{方面}{4,9}
  \begin{phonetics}{方面}{fang1mian4}[][HSK 2]
    \definition[个]{s.}{lado | campo | aspecto}
  \end{phonetics}
\end{entry}

\begin{entry}{方案}{4,10}
  \begin{phonetics}{方案}{fang1'an4}
    \definition[个,套]{s.}{plano | programa (para uma ação, etc.) | proposta | proposta de projeto de lei}
  \end{phonetics}
\end{entry}

\begin{entry}{无}{4}[Radical 无][Kangxi 71]
  \begin{phonetics}{无}{wu2}
    \definition{adv.}{não ter algo | não há\dots}
  \end{phonetics}
\end{entry}

\begin{entry}{无人}{4,2}
  \begin{phonetics}{无人}{wu2ren2}
    \definition{adj.}{não tripulado | desabitado}
  \end{phonetics}
\end{entry}

\begin{entry}{无人机}{4,2,6}
  \begin{phonetics}{无人机}{wu2ren2ji1}
    \definition{s.}{\emph{drone} | veículo aéreo não tripulado}
  \end{phonetics}
\end{entry}

\begin{entry}{无论……也……}{4,6,3}
  \begin{phonetics}{无论……也……}{wu2lun4 ye3}
    \definition{conj.}{não apenas\dots, (o que, quem, como, etc.), \dots}
  \end{phonetics}
\end{entry}

\begin{entry}{无视}{4,8}
  \begin{phonetics}{无视}{wu2shi4}
    \definition{v.}{ignorar | desconsiderar}
  \end{phonetics}
\end{entry}

\begin{entry}{无故}{4,9}
  \begin{phonetics}{无故}{wu2gu4}
    \definition{adv.}{sem causa ou razão | sem motivo}
  \end{phonetics}
\end{entry}

\begin{entry}{无骨}{4,9}
  \begin{phonetics}{无骨}{wu2 gu3}
    \definition{adj.}{desossado}
  \end{phonetics}
\end{entry}

\begin{entry}{无敌}{4,10}
  \begin{phonetics}{无敌}{wu2di2}
    \definition{adj.}{invencível | inigualável}
  \end{phonetics}
\end{entry}

\begin{entry}{无氧}{4,10}
  \begin{phonetics}{无氧}{wu2yang3}
    \definition{adj.}{anaeróbico}
  \end{phonetics}
\end{entry}

\begin{entry}{日}{4}[Radical 日][Kangxi 72]
  \begin{phonetics}{日}{ri4}[][HSK 1]
    \definition*{s.}{Japão, abreviação de~日本}
    \definition{clas.}{dia (mais usado em escrita) | data, dia do mês}
    \seeref{日本}{ri4ben3}
  \end{phonetics}
\end{entry}

\begin{entry}{日子}{4,3}
  \begin{phonetics}{日子}{ri4zi5}[][HSK 2]
    \definition{s.}{dia | uma data (calendário) | dias de vida de alguém}
  \end{phonetics}
\end{entry}

\begin{entry}{日出}{4,5}
  \begin{phonetics}{日出}{ri4chu1}
    \definition{s.}{nascer do sol}
  \seealsoref{夕阳}{xi1yang2}
  \end{phonetics}
\end{entry}

\begin{entry}{日本}{4,5}
  \begin{phonetics}{日本}{ri4ben3}
    \definition*{s.}{Japão}
  \end{phonetics}
\end{entry}

\begin{entry}{日本人}{4,5,2}
  \begin{phonetics}{日本人}{ri4ben3ren2}
    \definition{s.}{japonês | pessoa ou povo do Japão}
  \end{phonetics}
\end{entry}

\begin{entry}{日光灯}{4,6,6}
  \begin{phonetics}{日光灯}{ri4guang1deng1}
    \definition{s.}{lâmpada fluorescente}
  \end{phonetics}
\end{entry}

\begin{entry}{日报}{4,7}
  \begin{phonetics}{日报}{ri4 bao4}[][HSK 2]
    \definition[张]{s.}{diário | jornal diários}
  \end{phonetics}
\end{entry}

\begin{entry}{日常}{4,11}
  \begin{phonetics}{日常}{ri4chang2}
    \definition{adv.}{diariamente | dia-a-dia | todo dia}
  \end{phonetics}
\end{entry}

\begin{entry}{日期}{4,12}
  \begin{phonetics}{日期}{ri4qi1}[][HSK 1]
    \definition{s.}{data}
  \end{phonetics}
\end{entry}

\begin{entry}{月}{4}[Radical 月][Kangxi 74]
  \begin{phonetics}{月}{yue4}[][HSK 1]
    \definition[个,轮]{s.}{mês}
  \end{phonetics}
\end{entry}

\begin{entry}{月月}{4,4}
  \begin{phonetics}{月月}{yue4yue4}
    \definition{adv.}{todo mês}
  \end{phonetics}
\end{entry}

\begin{entry}{月份}{4,6}
  \begin{phonetics}{月份}{yue4 fen4}[][HSK 2]
    \definition{s.}{mês}
  \end{phonetics}
\end{entry}

\begin{entry}{月径}{4,8}
  \begin{phonetics}{月径}{yue4jing4}
    \definition{s.}{diâmetro da lua | diâmetro da órbita da lua | caminho iluminado pela lua}
  \end{phonetics}
\end{entry}

\begin{entry}{月亮}{4,9}
  \begin{phonetics}{月亮}{yue4liang5}[][HSK 2]
    \definition{s.}{lua}
  \end{phonetics}
\end{entry}

\begin{entry}{月相}{4,9}
  \begin{phonetics}{月相}{yue4xiang4}
    \definition{s.}{fases da lua, a saber: lua nova 朔, lua crescente 上弦, lua cheia 望 e lua minguante 下弦}
  \end{phonetics}
\end{entry}

\begin{entry}{月饼}{4,9}
  \begin{phonetics}{月饼}{yue4bing3}
    \definition[张]{s.}{bolo da lua}
  \end{phonetics}
\end{entry}

\begin{entry}{月球}{4,11}
  \begin{phonetics}{月球}{yue4qiu2}
    \definition{s.}{a lua}
  \end{phonetics}
\end{entry}

\begin{entry}{月壤}{4,20}
  \begin{phonetics}{月壤}{yue4rang3}
    \definition{s.}{solo lunar}
  \end{phonetics}
\end{entry}

\begin{entry}{木头}{4,5}
  \begin{phonetics}{木头}{mu4tou5}
    \definition{adj.}{estúpido | cabeça-dura}
    \definition[块,根]{s.}{tronco (de madeira)}
  \end{phonetics}
\end{entry}

\begin{entry}{木偶}{4,11}
  \begin{phonetics}{木偶}{mu4'ou3}
    \definition{s.}{fantoche, marionete}
  \end{phonetics}
\end{entry}

\begin{entry}{歹徒}{4,10}
  \begin{phonetics}{歹徒}{dai3tu2}
    \definition{s.}{malfeitor | gangster | bandido}
  \end{phonetics}
\end{entry}

\begin{entry}{比}{4}[Radical 匕][Kangxi 81]
  \begin{phonetics}{比}{bi3}[][HSK 1]
    \definition*{s.}{Bélgica, abreviação de 比利时}
    \definition{part.}{partícula usada para comparação (superioridade)}
    \definition{prep.}{que | do que | (seguido por um substantivo e adjetivo) mais \{adj.\} do que \{s.\}}
    \definition{s.}{razão (taxa)}
    \definition{v.}{comparar | contrastar | gesticular (com as mãos)}
    \seeref{比利时}{bi3li4shi2}
  \end{phonetics}
\end{entry}

\begin{entry}{比亚迪}{4,6,8}
  \begin{phonetics}{比亚迪}{bi3ya4di2}
    \definition*{s.}{Montadora BYD}
  \end{phonetics}
\end{entry}

\begin{entry}{比如}{4,6}
  \begin{phonetics}{比如}{bi3ru2}[][HSK 2]
    \definition{conj.}{por exemplo | como}
  \end{phonetics}
\end{entry}

\begin{entry}{比如说}{4,6,9}
  \begin{phonetics}{比如说}{bi3 ru2 shuo1}[][HSK 2]
    \definition{adv.}{por exemplo}
  \end{phonetics}
\end{entry}

\begin{entry}{比利时}{4,7,7}
  \begin{phonetics}{比利时}{bi3li4shi2}
    \definition*{s.}{Bélgica}
  \end{phonetics}
\end{entry}

\begin{entry}{比拼}{4,9}
  \begin{phonetics}{比拼}{bi3pin1}
    \definition{s.}{concurso}
    \definition{v.}{competir ferozmente}
  \end{phonetics}
\end{entry}

\begin{entry}{比较}{4,10}
  \begin{phonetics}{比较}{bi3jiao4}
    \definition{adv.}{comparativamente | relativamente}
    \definition{s.}{comparação}
    \definition{v.}{comparar}
  \end{phonetics}
\end{entry}

\begin{entry}{比萨饼}{4,11,9}
  \begin{phonetics}{比萨饼}{bi3sa4bing3}
    \definition[张]{s.}{pizza}
  \end{phonetics}
\end{entry}

\begin{entry}{比赛}{4,14}
  \begin{phonetics}{比赛}{bi3sai4}
    \definition[场,次]{s.}{competição | concurso}
    \definition{v.}{competir}
  \end{phonetics}
\end{entry}

\begin{entry}{毛}{4}[Radical 毛][Kangxi 82]
  \begin{phonetics}{毛}{mao2}[][HSK 1]
    \definition*{s.}{sobrenome Mao}
    \definition{clas.}{1 mao corresponde a 10 centavos}
  \end{phonetics}
\end{entry}

\begin{entry}{气}{4}
  \begin{phonetics}{气}{qi4}[][HSK 2]
    \definition[口]{s.}{gás | ar | respiração | clima | cheiro | odor | espírito | moral | ares | maneira | estilo | insulto | intimidação | energia vital | energia da vida}
    \definition{v.}{ficar bravo | ficar enfurecido | irritar | enfurecer}
  \end{phonetics}
\end{entry}

\begin{entry}{气质}{4,8}
  \begin{phonetics}{气质}{qi4zhi4}
    \definition{s.}{traços de personalidade, temperamento, disposição | aura, ar, sentimento, \emph{vibe} | refinamento, sofisticação, classe}
  \end{phonetics}
\end{entry}

\begin{entry}{气球}{4,11}
  \begin{phonetics}{气球}{qi4qiu2}
    \definition{s.}{balão}
  \end{phonetics}
\end{entry}

\begin{entry}{气温}{4,12}
  \begin{phonetics}{气温}{qi4 wen1}[][HSK 2]
    \definition[个]{s.}{temperatura do ar}
  \end{phonetics}
\end{entry}

\begin{entry}{水}{4}[Radical 水][Kangxi 85]
  \begin{phonetics}{水}{shui3}[][HSK 1]
    \definition*{s.}{sobrenome Shui}
    \definition{clas.}{para número de lavagens}
    \definition{s.}{água | líquido | encargos ou receitas adicionais}
  \end{phonetics}
\end{entry}

\begin{entry}{水平}{4,5}
  \begin{phonetics}{水平}{shui3ping2}[][HSK 2]
    \definition{s.}{nível (de realização, etc.) | padrão | nível horizontal}
  \end{phonetics}
\end{entry}

\begin{entry}{水平以下}{4,5,4,3}
  \begin{phonetics}{水平以下}{shui3ping2 yi3xia4}
    \definition{s.}{sub-nível}
  \end{phonetics}
\end{entry}

\begin{entry}{水平尺}{4,5,4}
  \begin{phonetics}{水平尺}{shui3ping2chi3}
    \definition{s.}{nível espiritual}
  \end{phonetics}
\end{entry}

\begin{entry}{水平仪}{4,5,5}
  \begin{phonetics}{水平仪}{shui3ping2yi2}
    \definition{s.}{nível (dispositivo para determinar horizontal) | nível espiritual | nível de topógrafo}
  \end{phonetics}
\end{entry}

\begin{entry}{水平视差}{4,5,8,9}
  \begin{phonetics}{水平视差}{shui3ping2 shi4cha1}
    \definition{s.}{paralaxe horizontal}
  \end{phonetics}
\end{entry}

\begin{entry}{水平度}{4,5,9}
  \begin{phonetics}{水平度}{shui3ping2 du4}
    \definition{s.}{nivelamento}
  \end{phonetics}
\end{entry}

\begin{entry}{水平轴}{4,5,9}
  \begin{phonetics}{水平轴}{shui3ping2zhou2}
    \definition{s.}{eixo horizontal}
  \end{phonetics}
\end{entry}

\begin{entry}{水平面}{4,5,9}
  \begin{phonetics}{水平面}{shui3ping2mian4}
    \definition{s.}{plano horizontal | nível-da-água | superfície horizontal}
  \end{phonetics}
\end{entry}

\begin{entry}{水边}{4,5}
  \begin{phonetics}{水边}{shui3bian1}
    \definition{s.}{beira d'água | beira-mar | costa (de mar, lago ou rio)}
  \end{phonetics}
\end{entry}

\begin{entry}{水污染}{4,6,9}
  \begin{phonetics}{水污染}{shui3wu1ran3}
    \definition{s.}{poluição da água}
  \end{phonetics}
\end{entry}

\begin{entry}{水灵}{4,7}
  \begin{phonetics}{水灵}{shui3ling2}
    \definition{adj.}{cheio de vida (sobre uma pessoa, etc.) | úmido e brilhante (sobre os olhos) | fresco (sobre frutas, etc.) | brilhante | aparência saudável}
  \end{phonetics}
\end{entry}

\begin{entry}{水果}{4,8}
  \begin{phonetics}{水果}{shui3guo3}[][HSK 1]
    \definition[个]{s.}{fruta}
  \end{phonetics}
\end{entry}

\begin{entry}{水波}{4,8}
  \begin{phonetics}{水波}{shui3bo1}
    \definition{s.}{ondulação (na água) | onda}
  \end{phonetics}
\end{entry}

\begin{entry}{水饺}{4,9}
  \begin{phonetics}{水饺}{shui3jiao3}
    \definition{s.}{\emph{dumplings} | pastéis chineses cozidos}
  \end{phonetics}
\end{entry}

\begin{entry}{水瓶}{4,10}
  \begin{phonetics}{水瓶}{shui3 ping2}
    \definition{s.}{garrada de água}
  \end{phonetics}
\end{entry}

\begin{entry}{水培}{4,11}
  \begin{phonetics}{水培}{shui3pei2}
    \definition{v.}{cultivar plantas hidroponicamente}
  \end{phonetics}
\end{entry}

\begin{entry}{水豚}{4,11}
  \begin{phonetics}{水豚}{shui3tun2}
    \definition{s.}{capivara}
  \end{phonetics}
\end{entry}

\begin{entry}{水路}{4,13}
  \begin{phonetics}{水路}{shui3lu4}
    \definition{s.}{hidrovia}
  \end{phonetics}
\end{entry}

\begin{entry}{水槽}{4,15}
  \begin{phonetics}{水槽}{shui3cao2}
    \definition{s.}{pia (de cozinha)}
  \end{phonetics}
\end{entry}

\begin{entry}{火}{4}[Radical 火][Kangxi 86]
  \begin{phonetics}{火}{huo3}
    \definition*{s.}{sobrenome Huo}
    \definition{adj.}{urgente | ardente ou flamejante | quente (popular)}
    \definition{clas.}{para unidades militares (antigo)}
    \definition{s.}{fogo | munição | calor interno (medicina chinesa)}
  \end{phonetics}
\end{entry}

\begin{entry}{火车}{4,4}
  \begin{phonetics}{火车}{huo3che1}[][HSK 1]
    \definition[列,节,班,趟]{s.}{trem | comboio}
  \end{phonetics}
\end{entry}

\begin{entry}{火车司机}{4,4,5,6}
  \begin{phonetics}{火车司机}{huo3che1 si1ji1}
    \definition{s.}{maquinista de trem}
  \end{phonetics}
\end{entry}

\begin{entry}{火柴}{4,10}
  \begin{phonetics}{火柴}{huo3chai2}
    \definition[根,盒]{s.}{fósforo (palito de fósforo)}
  \end{phonetics}
\end{entry}

\begin{entry}{火海}{4,10}
  \begin{phonetics}{火海}{huo3hai3}
    \definition{s.}{um mar de chamas}
  \end{phonetics}
\end{entry}

\begin{entry}{父母亲}{4,5,9}
  \begin{phonetics}{父母亲}{fu4mu3qin1}
    \definition{s.}{pais}
  \end{phonetics}
\end{entry}

\begin{entry}{父亲}{4,9}
  \begin{phonetics}{父亲}{fu4qin1}
    \definition[个]{s.}{pai}
  \end{phonetics}
\end{entry}

\begin{entry}{片}{4}[Radical 片][Kangxi 91]
  \begin{phonetics}{片}{pian4}[][HSK 2]
    \definition{adj.}{parcial | incompleto | que só tem um lado}
    \definition{clas.}{para CDs, filmes, DVDs, etc. | para fatias, comprimidos, extensão de terra, área de água | usado com numeral~一:~para  cenário, cena, sentimento, atmosfera, som etc.}
    \definition{s.}{uma fatia | floco | filme | pedaço fino}
    \definition{v.}{fatiar | esculpir fino}
  \end{phonetics}
\end{entry}

\begin{entry}{牙}{4}[Radical 牙][Kangxi 92]
  \begin{phonetics}{牙}{ya2}
    \definition[颗]{s.}{dente | marfim}
  \end{phonetics}
\end{entry}

\begin{entry}{牙行}{4,6}
  \begin{phonetics}{牙行}{ya2hang2}
    \definition{s.}{corretor | \emph{broker}}
  \end{phonetics}
\end{entry}

\begin{entry}{牙医}{4,7}
  \begin{phonetics}{牙医}{ya2yi1}
    \definition{s.}{dentista}
  \end{phonetics}
\end{entry}

\begin{entry}{牙刷}{4,8}
  \begin{phonetics}{牙刷}{ya2shua1}
    \definition[把]{s.}{escova de dentes}
  \end{phonetics}
\end{entry}

\begin{entry}{牙线}{4,8}
  \begin{phonetics}{牙线}{ya2xian4}
    \definition[条]{s.}{fio dental}
  \end{phonetics}
\end{entry}

\begin{entry}{牙齿}{4,8}
  \begin{phonetics}{牙齿}{ya2chi3}
    \definition{adv.}{dental}
    \definition[颗]{s.}{dente}
  \end{phonetics}
\end{entry}

\begin{entry}{牙膏}{4,14}
  \begin{phonetics}{牙膏}{ya2gao1}
    \definition[管]{s.}{pasta de dente}
  \end{phonetics}
\end{entry}

\begin{entry}{牛}{4}[Radical 牛][Kangxi 93]
  \begin{phonetics}{牛}{niu2}
    \definition*{s.}{sobrenome Niu}
    \definition[条,头]{s.}{boi | touro | vaca | (gíria) incrível}
  \end{phonetics}
\end{entry}

\begin{entry}{牛人}{4,2}
  \begin{phonetics}{牛人}{niu2ren2}
    \definition{s.}{(coloquial) o cara | verdadeiro especialista | \emph{badass}}
  \end{phonetics}
\end{entry}

\begin{entry}{牛仔裤}{4,5,12}
  \begin{phonetics}{牛仔裤}{niu2zai3ku4}
    \definition[条]{s.}{calça de ganga, jeans}
  \end{phonetics}
\end{entry}

\begin{entry}{牛奶}{4,5}
  \begin{phonetics}{牛奶}{niu2nai3}[][HSK 1]
    \definition[瓶,杯]{s.}{leite de vaca}
  \end{phonetics}
\end{entry}

\begin{entry}{牛肉}{4,6}
  \begin{phonetics}{牛肉}{niu2rou4}
    \definition{s.}{carne de vaca | bife}
  \end{phonetics}
\end{entry}

\begin{entry}{牛郎织女}{4,8,8,3}
  \begin{phonetics}{牛郎织女}{niu2lang2zhi1nv3}
    \definition*{s.}{Vaqueiro e Tecelã (personagens de contos folclóricos) | amantes separados | Altair e Vega (estrelas)}
  \end{phonetics}
\end{entry}

\begin{entry}{牛顿}{4,10}
  \begin{phonetics}{牛顿}{niu2dun4}
    \definition*{s.}{Newton (nome) | newton (N, unidade de força do SI)}
  \end{phonetics}
\end{entry}

\begin{entry}{犬}{4}[Radical 犬][Kangxi 94]
  \begin{phonetics}{犬}{quan3}
    \definition{s.}{cachorro}
  \end{phonetics}
\end{entry}

\begin{entry}{王}{4}[Radical 玉]
  \begin{phonetics}{王}{wang2}
    \definition*{s.}{sobrenome Wang}
    \definition{adj.}{grande | ótimo}
    \definition{s.}{rei ou monarca | melhor ou mais forte do seu tipo}
  \end{phonetics}
  \begin{phonetics}{王}{wang4}
    \definition{v.}{(literário) (um monarca) reinar (um reino)}
  \end{phonetics}
\end{entry}

\begin{entry}{王五}{4,4}
  \begin{phonetics}{王五}{wang2wu3}
    \definition{s.}{Wang Wu | Zé Ninguém | nome para uma pessoa não especificada, 3 de 3}
  \seealsoref{李四}{li3si4}
  \seealsoref{张三}{zhang1san1}
  \end{phonetics}
\end{entry}

\begin{entry}{王朝}{4,12}
  \begin{phonetics}{王朝}{wang2chao2}
    \definition{s.}{dinastia}
  \end{phonetics}
\end{entry}

\begin{entry}{瓦}{4}[Radical 瓦][Kangxi 98]
  \begin{phonetics}{瓦}{wa3}
    \definition{s.}{telha | abreviação de 瓦特}
    \seeref{瓦特}{wa3te4}
  \end{phonetics}
\end{entry}

\begin{entry}{瓦努阿图}{4,7,7,8}
  \begin{phonetics}{瓦努阿图}{wa3nu3'a1tu2}
    \definition*{s.}{Vanuatu, país do sudoeste do Oceano Pacífico}
  \end{phonetics}
\end{entry}

\begin{entry}{瓦特}{4,10}
  \begin{phonetics}{瓦特}{wa3te4}
    \definition{s.}{(empréstimo linguístico) watt | medida de potência}
  \end{phonetics}
\end{entry}

\begin{entry}{艺人}{4,2}
  \begin{phonetics}{艺人}{yi4ren2}
    \definition{s.}{artista | ator}
  \end{phonetics}
\end{entry}

\begin{entry}{见}{4}[Radical 見]
  \begin{phonetics}{见}{jian4}[][HSK 1]
    \definition{s.}{opinião, visão}
    \definition{v.}{ver | entrevistar | encontrar alguém | parecer (ser alguma coisa)}
  \end{phonetics}
  \begin{phonetics}{见}{xian4}[][HSK 0]
    \definition{v.}{aparecer | também escrito como 现}
    \seeref{现}{xian4}
  \end{phonetics}
\end{entry}

\begin{entry}{见过}{4,6}
  \begin{phonetics}{见过}{jian4 guo4}[][HSK 2]
    \definition{s.}{visto (ver)}
  \end{phonetics}
\end{entry}

\begin{entry}{见到}{4,8}
  \begin{phonetics}{见到}{jian4 dao4}[][HSK 2]
    \definition{v.}{ver | esbarrar em | encontrar-se com}
  \end{phonetics}
\end{entry}

\begin{entry}{见面}{4,9}
  \begin{phonetics}{见面}{jian4 mian4}[][HSK 1]
    \definition{v.+compl.}{encontrar-se com alguém | ver alguém face-a-face}
  \end{phonetics}
\end{entry}

\begin{entry}{计划}{4,6}
  \begin{phonetics}{计划}{ji4hua4}[][HSK 2]
    \definition[个,项]{s.}{plano | projeto | programa}
    \definition{v.}{planejar | mapear}
  \end{phonetics}
\end{entry}

\begin{entry}{计算机}{4,14,6}
  \begin{phonetics}{计算机}{ji4 suan4 ji1}[][HSK 2]
    \definition[部,台]{s.}{computador | calculadora}
  \end{phonetics}
\end{entry}

\begin{entry}{认为}{4,4}
  \begin{phonetics}{认为}{ren4wei2}[][HSK 2]
    \definition{v.}{pensar | considerar | segurar | julgar}
  \end{phonetics}
\end{entry}

\begin{entry}{认识}{4,7}
  \begin{phonetics}{认识}{ren4shi5}[][HSK 1]
    \definition{s.}{conhecimento | saber | entendimento}
    \definition{v.}{estar familiarizado com | conhecer alguém | saber | reconhecer}
  \end{phonetics}
\end{entry}

\begin{entry}{认真}{4,10}
  \begin{phonetics}{认真}{ren4zhen1}[][HSK 1]
    \definition{adj.}{sério | consciencioso}
    \definition{adv.}{seriamente}
    \definition{v.}{levar a sério}
  \end{phonetics}
\end{entry}

\begin{entry}{车}{4}[Radical 車][Kangxi 159]
  \begin{phonetics}{车}{che1}[][HSK 1]
    \definition*{s.}{sobrenome Che}
    \definition[辆]{s.}{carro | veículo | viatura}
  \end{phonetics}
  \begin{phonetics}{车}{ju1}[][HSK 0]
    \definition{s.}{(arcaico) carruagem de guerra | torre (no xadrez)}
  \end{phonetics}
\end{entry}

\begin{entry}{车上}{4,3}
  \begin{phonetics}{车上}{che1 shang5}[][HSK 1]
    \definition{adv.}{no carro | dentro do veículo}
  \end{phonetics}
\end{entry}

\begin{entry}{车子}{4,3}
  \begin{phonetics}{车子}{che1zi5}
    \definition{s.}{qualquer veículo (carro, bicicleta, caminhão, etc)}
  \end{phonetics}
\end{entry}

\begin{entry}{车水马龙}{4,4,3,5}
  \begin{phonetics}{车水马龙}{che1shui3-ma3long2}
    \definition{expr.}{tráfego engarrafado | engarrafamento | (literalmente) ``fluxo interminável de cavalos e carruagens''}
  \end{phonetics}
\end{entry}

\begin{entry}{车主}{4,5}
  \begin{phonetics}{车主}{che1zhu3}
    \definition{s.}{proprietário do carro}
  \end{phonetics}
\end{entry}

\begin{entry}{车次}{4,6}
  \begin{phonetics}{车次}{che1ci4}
    \definition{s.}{número do trem}
  \end{phonetics}
\end{entry}

\begin{entry}{车库}{4,7}
  \begin{phonetics}{车库}{che1ku4}
    \definition{s.}{garagem}
  \end{phonetics}
\end{entry}

\begin{entry}{车站}{4,10}
  \begin{phonetics}{车站}{che1zhan4}[][HSK 1]
    \definition[处,个]{s.}{estação | ponto de ônibus}
  \end{phonetics}
\end{entry}

\begin{entry}{车票}{4,11}
  \begin{phonetics}{车票}{che1piao4}[][HSK 1]
    \definition{s.}{bilhete (de ônibus, trem, metrô)}
  \end{phonetics}
\end{entry}

\begin{entry}{车辆}{4,11}
  \begin{phonetics}{车辆}{che1 liang4}[][HSK 2]
    \definition{s.}{veículo | carro}
  \end{phonetics}
\end{entry}

\begin{entry}{车牌}{4,12}
  \begin{phonetics}{车牌}{che1pai2}
    \definition{s.}{matrícula | placa de carro}
  \end{phonetics}
\end{entry}

\begin{entry}{长}{4}[Radical 長]
  \begin{phonetics}{长}{chang2}[][HSK 2]
    \definition{adj.}{comprido | longo}
  \end{phonetics}
  \begin{phonetics}{长}{zhang3}[][HSK 2]
    \definition{s.}{chefe | ancião}
    \definition{v.}{crescer | desenvolver | aumentar | melhorar}
  \end{phonetics}
\end{entry}

\begin{entry}{长大}{4,3}
  \begin{phonetics}{长大}{zhang3 da4}[][HSK 2]
    \definition{v.}{crescer | ser criado}
  \end{phonetics}
\end{entry}

\begin{entry}{长城}{4,9}
  \begin{phonetics}{长城}{chang2cheng2}
    \definition*{s.}{Grande Muralha}
  \end{phonetics}
\end{entry}

\begin{entry}{长颈鹿}{4,11,11}
  \begin{phonetics}{长颈鹿}{chang2jing3lu4}
    \definition[只]{s.}{girafa}
  \end{phonetics}
\end{entry}

\begin{entry}{队}{4}[Radical 阜]
  \begin{phonetics}{队}{dui4}[][HSK 2]
    \definition[个]{s.}{esquadrão | equipe | grupo}
  \end{phonetics}
\end{entry}

\begin{entry}{队友}{4,4}
  \begin{phonetics}{队友}{dui4you3}
    \definition{s.}{companheiro de equipe}
  \end{phonetics}
\end{entry}

\begin{entry}{队长}{4,4}
  \begin{phonetics}{队长}{dui4 zhang3}[][HSK 2]
    \definition{s.}{capitão (de equipe) | líder da equipe}
  \end{phonetics}
\end{entry}

\begin{entry}{风}{4}[Radical 風][Kangxi 182]
  \begin{phonetics}{风}{feng1}[][HSK 1]
    \definition[阵,丝]{s.}{vento}
  \end{phonetics}
\end{entry}

\begin{entry}{风扇}{4,10}
  \begin{phonetics}{风扇}{feng1shan4}
    \definition{s.}{ventilador elétrico}
  \end{phonetics}
\end{entry}

\begin{entry}{风景}{4,12}
  \begin{phonetics}{风景}{feng1jing3}
    \definition{s.}{cenário | paisagem}
  \end{phonetics}
\end{entry}

\begin{entry}{风筝}{4,12}
  \begin{phonetics}{风筝}{feng1zheng5}
    \definition{s.}{pipa | papagaio | pandorga}
  \end{phonetics}
\end{entry}

%%%%% EOF %%%%%


 %%%
%%% 5画
%%%

\section*{5画}\addcontentsline{toc}{section}{5画}

\begin{entry}{㐌}{5}[Radical 乙]
  \begin{phonetics}{㐌}{ta1}
    \variantof{它}
  \end{phonetics}
\end{entry}

\begin{entry}{世代}{5,5}
  \begin{phonetics}{世代}{shi4dai4}
    \definition{adv.}{por muitas gerações, eras}
    \definition{s.}{geração | era}
  \end{phonetics}
\end{entry}

\begin{entry}{世界}{5,9}
  \begin{phonetics}{世界}{shi4jie4}
    \definition[个]{s.}{mundo}
  \end{phonetics}
\end{entry}

\begin{entry}{世界杯}{5,9,8}
  \begin{phonetics}{世界杯}{shi4jie4bei1}
    \definition*{s.}{Copa do Mundo}
  \end{phonetics}
\end{entry}

\begin{entry}{世锦赛}{5,13,14}
  \begin{phonetics}{世锦赛}{shi4jin3sai4}
    \definition*{s.}{Campeonato Mundial}
  \end{phonetics}
\end{entry}

\begin{entry}{丘陵}{5,10}
  \begin{phonetics}{丘陵}{qiu1ling2}
    \definition{s.}{colinas}
  \end{phonetics}
\end{entry}

\begin{entry}{东}{5}[Radical ⼀]
  \begin{phonetics}{东}{dong1}
    \definition*{s.}{sobrenome Dong}
    \definition{s.}{leste}
  \end{phonetics}
\end{entry}

\begin{entry}{东方}{5,4}
  \begin{phonetics}{东方}{dong1fang1}
    \definition*{s.}{sobrenome Dongfang}
    \definition{s.}{leste | oriente}
  \end{phonetics}
\end{entry}

\begin{entry}{东方学院}{5,4,8,9}
  \begin{phonetics}{东方学院}{dong1fang1 xue2yuan4}
    \definition*{s.}{Instituto Oriental}
  \end{phonetics}
\end{entry}

\begin{entry}{东北}{5,5}
  \begin{phonetics}{东北}{dong1bei3}
    \definition*{s.}{Nordeste da China | Manchúria}
    \definition{s.}{nordeste}
  \end{phonetics}
\end{entry}

\begin{entry}{东半球}{5,5,11}
  \begin{phonetics}{东半球}{dong1ban4qiu2}
    \definition*{s.}{Hemisfério Oriental}
  \end{phonetics}
\end{entry}

\begin{entry}{东边}{5,5}
  \begin{phonetics}{东边}{dong1bian5}
    \definition{s.}{este | leste | lado leste | oriente}
  \end{phonetics}
\end{entry}

\begin{entry}{东西}{5,6}
  \begin{phonetics}{东西}{dong1xi1}
    \definition{s.}{leste e oeste}
  \end{phonetics}
  \begin{phonetics}{东西}{dong1xi5}
    \definition[个,件]{s.}{coisa | material | pessoa}
  \end{phonetics}
\end{entry}

\begin{entry}{东面}{5,9}
  \begin{phonetics}{东面}{dong1mian4}
    \definition{s.}{lado leste (de algo)}
  \end{phonetics}
\end{entry}

\begin{entry}{东部}{5,10}
  \begin{phonetics}{东部}{dong1bu4}
    \definition{s.}{leste | oriente}
  \end{phonetics}
\end{entry}

\begin{entry}{丝}{5}[Radical 一]
  \begin{phonetics}{丝}{si1}
    \definition{adj.}{filiforme | delgado como um fio | que se assemelha a um fio}
    \definition{clas.}{um traço (de fumaça, etc.) | um pouquinho, etc.}
    \definition{s.}{seda | (cozinha) pedaços ou tiras de julienne, tiras cortadas finas}
  \end{phonetics}
\end{entry}

\begin{entry}{主义}{5,3}
  \begin{phonetics}{主义}{zhu3yi4}
    \definition{s.}{ideologia}
    \definition{suf.}{"ismo"}
  \end{phonetics}
\end{entry}

\begin{entry}{主席}{5,10}
  \begin{phonetics}{主席}{zhu3xi2}
    \definition*[个,位]{s.}{Presidente (da China) | Primeiro-Ministro}
  \end{phonetics}
\end{entry}

\begin{entry}{主席台}{5,10,5}
  \begin{phonetics}{主席台}{zhu3xi2tai2}
    \definition[个]{s.}{plataforma | tribuna}
  \end{phonetics}
\end{entry}

\begin{entry}{主席团}{5,10,6}
  \begin{phonetics}{主席团}{zhu3xi2tuan2}
    \definition{s.}{presídio}
  \end{phonetics}
\end{entry}

\begin{entry}{乐观}{5,6}
  \begin{phonetics}{乐观}{le4guan1}
    \definition{adj.}{otimista | esperançoso}
  \end{phonetics}
\end{entry}

\begin{entry}{乐园}{5,7}
  \begin{phonetics}{乐园}{le4yuan2}
    \definition{s.}{paraíso}
  \end{phonetics}
\end{entry}

\begin{entry}{乐高}{5,10}
  \begin{phonetics}{乐高}{le4gao1}
    \definition*{s.}{Lego (brinquedo)}
  \end{phonetics}
\end{entry}

\begin{entry}{他}{5}[Radical 人]
  \begin{phonetics}{他}{ta1}
    \definition{pron.}{ele | se, o, lhe | si, consigo, ele}
    \seeref{怹}{tan1}
  \end{phonetics}
\end{entry}

\begin{entry}{他们}{5,5}
  \begin{phonetics}{他们}{ta1men5}
    \definition{pron.}{eles | se, os, lhes | si, consigo, eles}
  \end{phonetics}
\end{entry}

\begin{entry}{他们的}{5,5,8}
  \begin{phonetics}{他们的}{ta1men5 de5}
    \definition{pron.}{deles}
  \end{phonetics}
\end{entry}

\begin{entry}{他妈的}{5,6,8}
  \begin{phonetics}{他妈的}{ta1ma1de5}
    \definition{interj.}{Dane-se! | Foda-se!}
  \end{phonetics}
\end{entry}

\begin{entry}{他的}{5,8}
  \begin{phonetics}{他的}{ta1 de5}
    \definition{pron.}{dele}
  \end{phonetics}
\end{entry}

\begin{entry}{付}{5}[Radical 人]
  \begin{phonetics}{付}{fu4}
    \definition*{s.}{sobrenome Fu}
    \definition{clas.}{para pares ou conjuntos de coisas}
    \definition{v.}{pagar}
  \end{phonetics}
\end{entry}

\begin{entry}{付款}{5,12}
  \begin{phonetics}{付款}{fu4kuan3}
    \definition{s.}{pagamento}
    \definition{v.+compl.}{pagar uma quantia em dinheiro}
  \end{phonetics}
\end{entry}

\begin{entry}{仙}{5}[Radical 人]
  \begin{phonetics}{仙}{xian1}
    \definition{s.}{imortal}
  \end{phonetics}
\end{entry}

\begin{entry}{代价}{5,6}
  \begin{phonetics}{代价}{dai4jia4}
    \definition{s.}{preço | custo}
  \end{phonetics}
\end{entry}

\begin{entry}{代言}{5,7}
  \begin{phonetics}{代言}{dai4yan2}
    \definition{v.}{ser um porta-voz | ser um embaixador (para uma marca) | endossar}
  \end{phonetics}
\end{entry}

\begin{entry}{代表团}{5,8,6}
  \begin{phonetics}{代表团}{dai4biao3tuan2}
    \definition[个]{s.}{delegação}
  \end{phonetics}
\end{entry}

\begin{entry}{代称}{5,10}
  \begin{phonetics}{代称}{dai4cheng1}
    \definition{s.}{nome alternativo | antonomásia}
    \definition{v.}{referir-se a algo ou alguém por outro nome}
  \end{phonetics}
\end{entry}

\begin{entry}{令人}{5,2}
  \begin{phonetics}{令人}{ling4ren2}
    \definition{v.}{causar alguém (a fazer alguma coisa) | fazer alguém ficar zangado, encantado, etc.}
  \end{phonetics}
\end{entry}

\begin{entry}{仪式}{5,6}
  \begin{phonetics}{仪式}{yi2shi4}
    \definition{s.}{cerimônia}
  \end{phonetics}
\end{entry}

\begin{entry}{们}{5}[Radical 人]
  \begin{phonetics}{们}{men5}
    \definition{part.}{sufixo para plural de pronomes e substantivos referentes a indivíduos}
  \end{phonetics}
\end{entry}

\begin{entry}{兄弟}{5,7}
  \begin{phonetics}{兄弟}{xiong1di4}
    \definition{adj.}{fraternal | \emph{brotherly}}
    \definition{pron.}{eu, me (termo de uso humilde por homens em discurso público)}
    \definition[个]{s.}{irmãos | irmão mais novo | \emph{brothers}}
  \end{phonetics}
\end{entry}

\begin{entry}{兰花}{5,7}
  \begin{phonetics}{兰花}{lan2hua1}
    \definition{s.}{orquídea}
  \end{phonetics}
\end{entry}

\begin{entry}{写}{5}[Radical 冖]
  \begin{phonetics}{写}{xie3}
    \definition{v.}{escrever}
  \end{phonetics}
\end{entry}

\begin{entry}{写字}{5,6}
  \begin{phonetics}{写字}{xie3zi4}
    \definition{v.}{escrever (à mão) | praticar caligrafia}
  \end{phonetics}
\end{entry}

\begin{entry}{写字匠}{5,6,6}
  \begin{phonetics}{写字匠}{xie3zi4 jiang4}
    \definition{s.}{calígrafo}
  \end{phonetics}
\end{entry}

\begin{entry}{写作}{5,7}
  \begin{phonetics}{写作}{xie3zuo4}
    \definition{s.}{escrita | redação | composição}
    \definition{v.}{escrever}
  \end{phonetics}
\end{entry}

\begin{entry}{写真}{5,10}
  \begin{phonetics}{写真}{xie3zhen1}
    \definition{s.}{retrato}
    \definition{v.}{descrever algo com precisão}
  \end{phonetics}
\end{entry}

\begin{entry}{写意}{5,13}
  \begin{phonetics}{写意}{xie3yi4}
    \definition{s.}{estilo de pintura chinesa à mão livre, caracterizado por traços ousados em vez de detalhes precisos}
    \definition{v.}{sugerir (em vez de descrever em detalhes)}
  \end{phonetics}
  \begin{phonetics}{写意}{xie4yi4}
    \definition{adj.}{confortável | agradável | relaxado}
  \end{phonetics}
\end{entry}

\begin{entry}{写照}{5,13}
  \begin{phonetics}{写照}{xie3zhao4}
    \definition{s.}{retrato}
  \end{phonetics}
\end{entry}

\begin{entry}{冬天}{5,4}
  \begin{phonetics}{冬天}{dong1tian1}
    \definition{s.}{inverno}
  \end{phonetics}
\end{entry}

\begin{entry}{冬瓜}{5,5}
  \begin{phonetics}{冬瓜}{dong1gua1}
    \definition{s.}{melão de inverno}
  \end{phonetics}
\end{entry}

\begin{entry}{出}{5}[Radical ⼐]
  \begin{phonetics}{出}{chu1}
    \definition{clas.}{para dramas, peças, óperas, etc.}
    \definition{v.}{sair | ir para fora | vir para fora}
  \end{phonetics}
\end{entry}

\begin{entry}{出口}{5,3}
  \begin{phonetics}{出口}{chu1kou3}
    \definition[个]{s.}{exportação}
    \definition{v.+compl.}{exportar}
  \end{phonetics}
\end{entry}

\begin{entry}{出击}{5,5}
  \begin{phonetics}{出击}{chu1ji1}
    \definition{v.}{atacar}
  \end{phonetics}
\end{entry}

\begin{entry}{出去}{5,5}
  \begin{phonetics}{出去}{chu1qu4}
    \definition{v.}{sair | ir para fora (a partir da minha localização)}
  \end{phonetics}
\end{entry}

\begin{entry}{出发}{5,5}
  \begin{phonetics}{出发}{chu1fa1}
    \definition{v.}{partir | começar (uma jornada)}
  \end{phonetics}
\end{entry}

\begin{entry}{出汗}{5,6}
  \begin{phonetics}{出汗}{chu1han4}
    \definition{v.}{transpirar | suar}
  \end{phonetics}
\end{entry}

\begin{entry}{出行}{5,6}
  \begin{phonetics}{出行}{chu1xing2}
    \definition{v.}{sair para algum lugar (viagem relativamente curta) | partir em uma viagem (viagem mais longa)}
  \end{phonetics}
\end{entry}

\begin{entry}{出来}{5,7}
  \begin{phonetics}{出来}{chu1lai2}
    \definition{v.}{sair | vir para fora (para a minha localização)}
  \end{phonetics}
\end{entry}

\begin{entry}{出版}{5,8}
  \begin{phonetics}{出版}{chu1ban3}
    \definition{v.}{publicar | editar}
  \end{phonetics}
\end{entry}

\begin{entry}{出版社}{5,8,7}
  \begin{phonetics}{出版社}{chu1ban3she4}
    \definition{s.}{editora}
  \end{phonetics}
\end{entry}

\begin{entry}{出差}{5,9}
  \begin{phonetics}{出差}{chu1chai1}
    \definition{v.+compl.}{fazer uma viagem oficial ou de negócios}
  \end{phonetics}
\end{entry}

\begin{entry}{出租}{5,10}
  \begin{phonetics}{出租}{chu1zu1}
    \definition{v.}{alugar | arrendar}
  \end{phonetics}
\end{entry}

\begin{entry}{出租车}{5,10,4}
  \begin{phonetics}{出租车}{chu1zu1che1}
    \definition{s.}{táxi}
  \seealsoref{出租汽车}{chu1zu1qi4che1}
  \end{phonetics}
\end{entry}

\begin{entry}{出租司机}{5,10,5,6}
  \begin{phonetics}{出租司机}{chu1zu1si1ji1}
    \definition{s.}{motorista de táxi}
  \end{phonetics}
\end{entry}

\begin{entry}{出租汽车}{5,10,7,4}
  \begin{phonetics}{出租汽车}{chu1zu1qi4che1}
    \definition[辆]{s.}{táxi}
  \seealsoref{出租车}{chu1zu1che1}
  \end{phonetics}
\end{entry}

\begin{entry}{出站}{5,10}
  \begin{phonetics}{出站}{chu1 zhan4}
    \definition{s.}{saída da estação}
  \end{phonetics}
\end{entry}

\begin{entry}{功夫}{5,4}
  \begin{phonetics}{功夫}{gong1fu5}
    \definition*{s.}{Gongfu (Kung Fu), arte marcial}
    \definition{s.}{esforço | habilidade}
  \end{phonetics}
\end{entry}

\begin{entry}{功臣}{5,6}
  \begin{phonetics}{功臣}{gong1chen2}
    \definition{s.}{oficial meritório | pessoa que presta serviço excepcional, herói | (fig.) alguém que desempenha um papel vital}
  \end{phonetics}
\end{entry}

\begin{entry}{加}{5}[Radical 力]
  \begin{phonetics}{加}{jia1}
    \definition*{s.}{Canadá, abreviação de~加拿大 | sobrenome Jia}
    \seeref{加拿大}{jia1na2da4}
  \end{phonetics}
\end{entry}

\begin{entry}{加入}{5,2}
  \begin{phonetics}{加入}{jia1ru4}
    \definition{v.}{tornar-se um membro | juntar-se | participar de | adicionar em}
  \end{phonetics}
\end{entry}

\begin{entry}{加工}{5,3}
  \begin{phonetics}{加工}{jia1gong1}
    \definition{s.}{processo | trabalho (de uma máquina)}
    \definition{v.}{processar}
  \end{phonetics}
\end{entry}

\begin{entry}{加油}{5,8}
  \begin{phonetics}{加油}{jia1you2}
    \definition{v.+compl.}{lubrificar | encher o tanque de combustível | fazer um esforço maior | fazer um esforço extra}
  \end{phonetics}
\end{entry}

\begin{entry}{加拿大}{5,10,3}
  \begin{phonetics}{加拿大}{jia1na2da4}
    \definition{s.}{Canadá}
  \end{phonetics}
\end{entry}

\begin{entry}{加拿大人}{5,10,3,2}
  \begin{phonetics}{加拿大人}{jia1na2da4ren2}
    \definition{s.}{canadense | pessoa ou povo do Canadá}
  \end{phonetics}
\end{entry}

\begin{entry}{加速}{5,10}
  \begin{phonetics}{加速}{jia1su4}
    \definition{v.}{acelerar | agilizar}
  \end{phonetics}
\end{entry}

\begin{entry}{加速度}{5,10,9}
  \begin{phonetics}{加速度}{jia1su4du4}
    \definition{s.}{aceleração}
  \end{phonetics}
\end{entry}

\begin{entry}{务实}{5,8}
  \begin{phonetics}{务实}{wu4shi2}
    \definition{adj.}{pragmático}
    \definition{v.}{lidar com assuntos concretos}
  \end{phonetics}
\end{entry}

\begin{entry}{包}{5}[Radical 勹]
  \begin{phonetics}{包}{bao1}
    \definition*{s.}{sobrenome Bao}
    \definition{clas.}{pacotes, sacos, sacolas, embrulhos}
    \definition[个,只]{s.}{bolsa | pacote | recipiente | embrulho}
    \definition{v.}{contratar | cobrir | segurar ou abraçar | incluir | assumir o comando | embrulhar}
  \end{phonetics}
\end{entry}

\begin{entry}{包子}{5,3}
  \begin{phonetics}{包子}{bao1zi5}
    \definition[个]{s.}{pão recheado cozido no vapor}
  \end{phonetics}
\end{entry}

\begin{entry}{包干}{5,3}
  \begin{phonetics}{包干}{bao1gan1}
    \definition{s.}{tarefa alocada}
    \definition{v.}{ter a responsabilidade total sobre um trabalho}
  \end{phonetics}
\end{entry}

\begin{entry}{包办}{5,4}
  \begin{phonetics}{包办}{bao1ban4}
    \definition{v.}{comandar todo o show | comprometer-se a fazer tudo sozinho}
  \end{phonetics}
\end{entry}

\begin{entry}{包括}{5,9}
  \begin{phonetics}{包括}{bao1kuo4}
    \definition{v.}{compreender | consistir em | incluir | incorporar | envolver}
  \end{phonetics}
\end{entry}

\begin{entry}{包容}{5,10}
  \begin{phonetics}{包容}{bao1rong2}
    \definition{adj.}{inclusivo}
    \definition{v.}{perdoar | mostrar tolerância | conter | segurar}
  \end{phonetics}
\end{entry}

\begin{entry}{包租}{5,10}
  \begin{phonetics}{包租}{bao1zu1}
    \definition{s.}{aluguel fixo para terras agrícolas}
    \definition{v.}{fretar | alugar | alugar um terreno ou uma casa para subarrendar}
  \end{phonetics}
\end{entry}

\begin{entry}{匆匆}{5,5}
  \begin{phonetics}{匆匆}{cong1cong1}
    \definition{adv.}{apressadamente}
  \end{phonetics}
\end{entry}

\begin{entry}{北}{5}[Radical 匕]
  \begin{phonetics}{北}{bei3}
    \definition{s.}{norte}
    \definition{v.}{(clássico) ser derrotado}
  \end{phonetics}
\end{entry}

\begin{entry}{北大西洋公约组织}{5,3,6,9,4,6,8,8}
  \begin{phonetics}{北大西洋公约组织}{bei3 da4xi1 yang2 gong1 yue1 zu3zhi1}
    \definition*{s.}{Organização do Tratado do Atlântico Norte, OTAN}
  \end{phonetics}
\end{entry}

\begin{entry}{北方}{5,4}
  \begin{phonetics}{北方}{bei3fang1}
    \definition{s.}{norte | a parte norte de um país}
  \end{phonetics}
\end{entry}

\begin{entry}{北边}{5,5}
  \begin{phonetics}{北边}{bei3bian1}
    \definition{adv.}{lado norte | ao norte de}
  \end{phonetics}
\end{entry}

\begin{entry}{北约}{5,6}
  \begin{phonetics}{北约}{bei3yue1}
    \definition*{s.}{OTAN (Organização do Tratado do Atlântico Norte), abreviação de 北大西洋公约组织}
    \seeref{北大西洋公约组织}{bei3 da4xi1 yang2 gong1 yue1 zu3zhi1}
  \end{phonetics}
\end{entry}

\begin{entry}{北极}{5,7}
  \begin{phonetics}{北极}{bei3ji2}
    \definition*{s.}{Ártico | Pólo Norte}
    \definition{s.}{pólo norte magnético}
  \end{phonetics}
\end{entry}

\begin{entry}{北京}{5,8}
  \begin{phonetics}{北京}{bei3jing1}
    \definition*{s.}{Beijing (Pequim), Capital da República Popular da China | Beijing (Pequim), governo da RPC}
  \end{phonetics}
\end{entry}

\begin{entry}{北面}{5,9}
  \begin{phonetics}{北面}{bei3mian4}
    \definition{s.}{lado norte}
  \end{phonetics}
\end{entry}

\begin{entry}{半}{5}[Radical 十]
  \begin{phonetics}{半}{ban4}
    \definition{adj.}{incompleto}
    \definition{num.}{(depois de um número) ``e meio''}
    \definition{pref.}{semi}
    \definition{s.}{metade}
  \end{phonetics}
\end{entry}

\begin{entry}{半音}{5,9}
  \begin{phonetics}{半音}{ban4yin1}
    \definition{s.}{semitom}
  \end{phonetics}
\end{entry}

\begin{entry}{半球}{5,11}
  \begin{phonetics}{半球}{ban4qiu2}
    \definition{s.}{hemisfério}
  \end{phonetics}
\end{entry}

\begin{entry}{卡片}{5,4}
  \begin{phonetics}{卡片}{ka3pian4}
    \definition{s.}{cartão}
  \end{phonetics}
\end{entry}

\begin{entry}{卡片游戏}{5,4,12,6}
  \begin{phonetics}{卡片游戏}{ka3pian4 you2xi4}
    \definition{s.}{carta de baralho}
  \end{phonetics}
\end{entry}

\begin{entry}{卡车司机}{5,4,5,6}
  \begin{phonetics}{卡车司机}{ka3che1 si1ji1}
    \definition{s.}{motorista de caminhão}
  \end{phonetics}
\end{entry}

\begin{entry}{卡通}{5,10}
  \begin{phonetics}{卡通}{ka3tong1}
    \definition{s.}{(empréstimo linguístico) \emph{cartoon}}
  \end{phonetics}
\end{entry}

\begin{entry}{卢旺达}{5,8,6}
  \begin{phonetics}{卢旺达}{lu2wang4da2}
    \definition*{s.}{Ruanda}
  \end{phonetics}
\end{entry}

\begin{entry}{厉害}{5,10}
  \begin{phonetics}{厉害}{li4hai5}
    \definition{adj.}{severo | rigoroso | exigente | radical | violento | feroz}
  \end{phonetics}
\end{entry}

\begin{entry}{厺}{5}
  \begin{phonetics}{厺}{qu4}
    \variantof{去}
  \end{phonetics}
\end{entry}

\begin{entry}{去}{5}[Radical 厶]
  \begin{phonetics}{去}{qu4}
    \definition{v.}{ir | (eufenismo) morrer}
  \end{phonetics}
\end{entry}

\begin{entry}{去年}{5,6}
  \begin{phonetics}{去年}{qu4nian2}
    \definition{s.}{ano passado}
  \end{phonetics}
\end{entry}

\begin{entry}{去死}{5,6}
  \begin{phonetics}{去死}{qu4si3}
    \definition{interj.}{Caia morto! | Vá para o Inferno!}
  \end{phonetics}
\end{entry}

\begin{entry}{发}{5}[Radical ⼜]
  \begin{phonetics}{发}{fa1}
    \definition{clas.}{para tiros (rodadas)}
    \definition{v.}{enviar | mandar}
  \end{phonetics}
  \begin{phonetics}{发}{fa4}
    \definition{s.}{cabelo}
  \end{phonetics}
\end{entry}

\begin{entry}{发生}{5,5}
  \begin{phonetics}{发生}{fa1sheng1}
    \definition{v.}{acontecer | ocorrer}
  \end{phonetics}
\end{entry}

\begin{entry}{发动机}{5,6,6}
  \begin{phonetics}{发动机}{fa1dong4ji1}
    \definition[台]{s.}{motor}
  \end{phonetics}
\end{entry}

\begin{entry}{发抖}{5,7}
  \begin{phonetics}{发抖}{fa1dou3}
    \definition{v.}{tremer | sacudir | estremecer}
  \end{phonetics}
\end{entry}

\begin{entry}{发财}{5,7}
  \begin{phonetics}{发财}{fa1cai2}
    \definition{v.+compl.}{ficar rico | fazer fortuna}
  \end{phonetics}
\end{entry}

\begin{entry}{发明者}{5,8,8}
  \begin{phonetics}{发明者}{fa1ming2zhe3}
    \definition{s.}{inventor}
  \end{phonetics}
\end{entry}

\begin{entry}{发现}{5,8}
  \begin{phonetics}{发现}{fa1xian4}
    \definition{s.}{descoberta}
    \definition{v.}{perceber, tornar-se ciente de | descobrir, encontrar, detectar}
  \end{phonetics}
\end{entry}

\begin{entry}{发现者}{5,8,8}
  \begin{phonetics}{发现者}{fa1xian4 zhe3}
    \definition{s.}{descobridor}
  \end{phonetics}
\end{entry}

\begin{entry}{发表}{5,8}
  \begin{phonetics}{发表}{fa1biao3}
    \definition{v.}{emitir | publicar}
  \end{phonetics}
\end{entry}

\begin{entry}{发型}{5,9}
  \begin{phonetics}{发型}{fa4xing2}
    \definition{s.}{penteado}
  \end{phonetics}
\end{entry}

\begin{entry}{发音}{5,9}
  \begin{phonetics}{发音}{fa1yin1}
    \definition{s.}{pronúncia}
    \definition{v.}{pronunciar}
  \end{phonetics}
\end{entry}

\begin{entry}{发展}{5,10}
  \begin{phonetics}{发展}{fa1zhan3}
    \definition{s.}{desenvolvimento}
    \definition{v.}{desenvolver}
  \end{phonetics}
\end{entry}

\begin{entry}{发烧}{5,10}
  \begin{phonetics}{发烧}{fa1shao1}
    \definition{v.}{ter febre}
  \end{phonetics}
\end{entry}

\begin{entry}{发票}{5,11}
  \begin{phonetics}{发票}{fa1piao4}
    \definition{s.}{fatura | recibo | conta}
  \end{phonetics}
\end{entry}

\begin{entry}{发愁}{5,13}
  \begin{phonetics}{发愁}{fa1chou2}
    \definition{v.+compl.}{preocupar-se | ficar ansioso | ficar triste}
  \end{phonetics}
\end{entry}

\begin{entry}{发簪}{5,18}
  \begin{phonetics}{发簪}{fa4zan1}
    \definition{s.}{grampo de cabelo}
  \end{phonetics}
\end{entry}

\begin{entry}{古}{5}[Radical ⼝]
  \begin{phonetics}{古}{gu3}
    \definition*{s.}{sobrenome Gu}
    \definition{adj.}{anciente | antigo | velho}
    \definition{pref.}{``paleo''}
  \end{phonetics}
\end{entry}

\begin{entry}{古人}{5,2}
  \begin{phonetics}{古人}{gu3ren2}
    \definition{s.}{pessoas dos tempos antigos | os antigos | espécies humanas extintas, como \emph{Homo erectus} ou \emph{Homo neanderthalensis} | (literário) pessoa falecida}
  \end{phonetics}
\end{entry}

\begin{entry}{古老}{5,6}
  \begin{phonetics}{古老}{gu3lao3}
    \definition{adj.}{ancestral | antigo | velho}
  \end{phonetics}
\end{entry}

\begin{entry}{古城}{5,9}
  \begin{phonetics}{古城}{gu3cheng2}
    \definition{s.}{cidade antiga}
  \end{phonetics}
\end{entry}

\begin{entry}{古铜色}{5,11,6}
  \begin{phonetics}{古铜色}{gu3tong2se4}
    \definition{s.}{bronze (cor)}
  \end{phonetics}
\end{entry}

\begin{entry}{句}{5}[Radical 口]
  \begin{phonetics}{句}{gou4}
    \variantof{勾}
  \end{phonetics}
  \begin{phonetics}{句}{ju4}
    \definition{clas.}{para orações, frases ou linhas de versos}
    \definition{s.}{sentença | cláusula | frase}
  \end{phonetics}
\end{entry}

\begin{entry}{句子}{5,3}
  \begin{phonetics}{句子}{ju4zi5}
    \definition[个]{s.}{sentença | frase | oração}
  \end{phonetics}
\end{entry}

\begin{entry}{另外}{5,5}
  \begin{phonetics}{另外}{ling4wai4}
    \definition{adv./pron.}{além disso}
  \end{phonetics}
\end{entry}

\begin{entry}{只}{5}[Radical 口]
  \begin{phonetics}{只}{zhi1}
    \definition{clas.}{para pássaros, gatos, cãezinhos, etc.}
  \end{phonetics}
  \begin{phonetics}{只}{zhi3}
    \definition{adv.}{apenas | só}
  \end{phonetics}
\end{entry}

\begin{entry}{只好}{5,6}
  \begin{phonetics}{只好}{zhi3hao3}
    \definition{adv.}{ser forçado a | ter que | sem nenhuma opção melhor | não ter outro remédio senão}
  \end{phonetics}
\end{entry}

\begin{entry}{只有……才……}{5,6,3}
  \begin{phonetics}{只有……才……}{zhi3you3 cai2}
    \definition{conj.}{só se\dots então\dots}
  \end{phonetics}
\end{entry}

\begin{entry}{只身}{5,7}
  \begin{phonetics}{只身}{zhi1shen1}
    \definition{adv.}{sozinho | por si só}
  \end{phonetics}
\end{entry}

\begin{entry}{只怕}{5,8}
  \begin{phonetics}{只怕}{zhi3pa4}
    \definition{adv.}{receio que\dots | talvez | muito provavelmente}
  \end{phonetics}
\end{entry}

\begin{entry}{只要}{5,9}
  \begin{phonetics}{只要}{zhi3yao4}
    \definition{conj.}{se apenas | contanto que}
  \end{phonetics}
\end{entry}

\begin{entry}{只要……就……}{5,9,12}
  \begin{phonetics}{只要……就……}{zhi3yao4 jiu4}
    \definition{conj.}{contanto que/desde que/se somente\dots, então\dots}
  \end{phonetics}
\end{entry}

\begin{entry}{只消}{5,10}
  \begin{phonetics}{只消}{zhi3xiao1}
    \definition{conj.}{desde que}
  \end{phonetics}
\end{entry}

\begin{entry}{只读}{5,10}
  \begin{phonetics}{只读}{zhi3du2}
    \definition{s.}{somente leitura (computação) | \emph{read-only}}
  \end{phonetics}
\end{entry}

\begin{entry}{只顾}{5,10}
  \begin{phonetics}{只顾}{zhi3gu4}
    \definition{adv.}{exclusivamente preocupado (com uma coisa)}
    \definition{v.}{cuidar de apenas um aspecto}
  \end{phonetics}
\end{entry}

\begin{entry}{只得}{5,11}
  \begin{phonetics}{只得}{zhi3de5}
    \definition{v.}{ser obrigado a | não ter outra alternativa senão}
  \end{phonetics}
\end{entry}

\begin{entry}{叫}{5}[Radical 口]
  \begin{phonetics}{叫}{jiao4}
    \definition{v.}{ser chamado | chamar-se | pedir | por (indica agente na voz passiva) | chamar | gritar | ordenar}
  \end{phonetics}
\end{entry}

\begin{entry}{叮嘱}{5,15}
  \begin{phonetics}{叮嘱}{ding1zhu3}
    \definition{v.}{exortar | avisar | insistir de novo e de novo}
  \end{phonetics}
\end{entry}

\begin{entry}{可}{5}[Radical 口]
  \begin{phonetics}{可}{ke3}
    \definition{adv.}{muito | realmente}
  \end{phonetics}
\end{entry}

\begin{entry}{可口可乐}{5,3,5,5}
  \begin{phonetics}{可口可乐}{ke3kou3ke3le4}
    \definition*{s.}{(empréstimo linguístico) Coca-Cola}
  \end{phonetics}
\end{entry}

\begin{entry}{可以}{5,4}
  \begin{phonetics}{可以}{ke3yi3}
    \definition{v.}{ser capaz de | poder}
  \end{phonetics}
\end{entry}

\begin{entry}{可卡因}{5,5,6}
  \begin{phonetics}{可卡因}{ke3ka3yin1}
    \definition{s.}{(empréstimo linguístico) cocaína}
  \end{phonetics}
\end{entry}

\begin{entry}{可怕}{5,8}
  \begin{phonetics}{可怕}{ke3pa4}
    \definition{adj.}{horrível | terrível | formidável | assustador | hediondo}
    \definition{adv.}{terrivelmente}
  \end{phonetics}
\end{entry}

\begin{entry}{可是}{5,9}
  \begin{phonetics}{可是}{ke3shi4}
    \definition{adv.}{(usado para dar ênfase) de fato}
    \definition{conj.}{porém | contudo | mas}
  \end{phonetics}
\end{entry}

\begin{entry}{可爱}{5,10}
  \begin{phonetics}{可爱}{ke3'ai4}
    \definition{adj.}{adorável | querido | fofo}
  \end{phonetics}
\end{entry}

\begin{entry}{可能}{5,10}
  \begin{phonetics}{可能}{ke3neng2}
    \definition{adj.}{possível | provável}
    \definition{adv.}{possivelmente | provavelmente}
    \definition[个]{s.}{possibilidade | probabilidade}
  \end{phonetics}
\end{entry}

\begin{entry}{可惜}{5,11}
  \begin{phonetics}{可惜}{ke3xi1}
    \definition{adj.}{é uma pena | que pena}
    \definition{adv.}{infelizmente | que pena | é uma pena}
  \end{phonetics}
\end{entry}

\begin{entry}{可编程}{5,12,12}
  \begin{phonetics}{可编程}{ke3bian1cheng2}
    \definition{adj.}{programável}
  \end{phonetics}
\end{entry}

\begin{entry*}{可擦写可编程只读存储器}{5,17,5,5,12,12,5,10,6,12,16}
  \begin{phonetics}{可擦写可编程只读存储器}{ke3ca1xie3ke3bian1cheng2zhi1du2cun2chu3qi4}
    \definition{s.}{EPROM (\emph{erasable programmable read-only memory})}
  \end{phonetics}
\end{entry*}

\begin{entry}{台}{5}[Radical 口]
  \begin{phonetics}{台}{tai2}
    \definition*{s.}{sobrenome Tai}
    \definition{clas.}{para aparelhos e máquinas}
    \definition{s.}{estação de transmissão | contador | \emph{help desk} | suporte técnico | plataforma | terraço | tufão}
  \end{phonetics}
\end{entry}

\begin{entry}{台下}{5,3}
  \begin{phonetics}{台下}{tai2xia4}
    \definition{s.}{platéia | fora do palco}
  \end{phonetics}
\end{entry}

\begin{entry}{台风}{5,4}
  \begin{phonetics}{台风}{tai2feng1}
    \definition{s.}{tufão}
  \end{phonetics}
\end{entry}

\begin{entry}{右}{5}[Radical 口]
  \begin{phonetics}{右}{you4}
    \definition{s.}{(política) a Direita}
    \definition{s.}{direita}
  \end{phonetics}
\end{entry}

\begin{entry}{右手}{5,4}
  \begin{phonetics}{右手}{you4shou3}
    \definition{s.}{mão direita | lado direito}
  \end{phonetics}
\end{entry}

\begin{entry}{右边}{5,5}
  \begin{phonetics}{右边}{you4bian5}
    \definition{adv.}{à direita | ao lado direito}
  \end{phonetics}
\end{entry}

\begin{entry}{右侧}{5,8}
  \begin{phonetics}{右侧}{you4ce4}
    \definition{s.}{lateral direita | lado direito}
  \end{phonetics}
\end{entry}

\begin{entry}{右转}{5,8}
  \begin{phonetics}{右转}{you4zhuan3}
    \definition{v.}{virar à direita}
  \end{phonetics}
\end{entry}

\begin{entry}{右面}{5,9}
  \begin{phonetics}{右面}{you4mian4}
    \definition{s.}{lado direito}
  \end{phonetics}
\end{entry}

\begin{entry}{右倾}{5,10}
  \begin{phonetics}{右倾}{you4qing1}
    \definition{adj.}{conservador | reacionário}
  \end{phonetics}
\end{entry}

\begin{entry}{右袒}{5,10}
  \begin{phonetics}{右袒}{you4tan3}
    \definition{v.}{ser tendencioso | ser parcial | favorecer um lado | tomar partido}
  \end{phonetics}
\end{entry}

\begin{entry}{号}{5}[Radical 口]
  \begin{phonetics}{号}{hao2}
    \definition[个]{s.}{rugido | choro}
  \end{phonetics}
  \begin{phonetics}{号}{hao4}
    \definition{clas.}{para indicar o número de pessoas}
    \definition{num.}{dia do mês | usado para indicar o número de pessoas}
    \definition[个]{s.}{número ordinal | dia de um mês | marca | sinal | estabelecimento comercial | tamanho | buzina (instrumento de sopro) | toque de corneta | nome suposto}
    \definition{suf.}{sufixo de navio}
    \definition{v.}{tomar um pulso}
  \end{phonetics}
\end{entry}

\begin{entry}{号角}{5,7}
  \begin{phonetics}{号角}{hao4jiao3}
    \definition{s.}{corneta | trombeta}
  \end{phonetics}
\end{entry}

\begin{entry}{号码}{5,8}
  \begin{phonetics}{号码}{hao4ma3}
    \definition[堆,个]{s.}{número}
  \end{phonetics}
\end{entry}

\begin{entry}{司机}{5,6}
  \begin{phonetics}{司机}{si1ji1}
    \definition{s.}{condutor | motorista | chofer}
  \end{phonetics}
\end{entry}

\begin{entry}{囘}{5}
  \begin{phonetics}{囘}{hui2}
    \variantof{回}
  \end{phonetics}
\end{entry}

\begin{entry}{四}{5}[Radical 囗]
  \begin{phonetics}{四}{si4}
    \definition{num.}{quatro; 4}
  \end{phonetics}
\end{entry}

\begin{entry}{四川}{5,3}
  \begin{phonetics}{四川}{si4chuan1}
    \definition*{s.}{Sichuan}
  \end{phonetics}
\end{entry}

\begin{entry}{四季分明}{5,8,4,8}
  \begin{phonetics}{四季分明}{si4ji4-fen1ming2}
    \definition{expr.}{as quatro estações são muito distintas}
  \end{phonetics}
\end{entry}

\begin{entry}{四季如春}{5,8,6,9}
  \begin{phonetics}{四季如春}{si4ji4-ru2chun1}
    \definition{expr.}{é primavera todo o ano | clima favorável durante todo o ano | quatro estações como a primavera}
  \end{phonetics}
\end{entry}

\begin{entry}{圣地}{5,6}
  \begin{phonetics}{圣地}{sheng4di4}
    \definition{s.}{terra santa (de uma religião) | lugar sagrado | santuário | cidade santa (como Jerusalém, Meca, etc.) | centro de interesse histórico}
  \end{phonetics}
\end{entry}

\begin{entry}{圣诞节}{5,8,5}
  \begin{phonetics}{圣诞节}{sheng4dan4jie2}
    \definition*{s.}{Natal}
  \end{phonetics}
\end{entry}

\begin{entry}{处}{5}[Radical ⼡]
  \begin{phonetics}{处}{chu3}
    \definition{v.}{residir | viver | habitar | estar dentro | estar situado em | ficar | se dar bem com | estar em uma posição de | lidar com | disciplinar | punir}
  \end{phonetics}
  \begin{phonetics}{处}{chu4}
    \definition{clas.}{para locais ou itens de danos: lugar, local}
    \definition{s.}{local | localização | lugar | ponto | escritório | departamento}
  \end{phonetics}
\end{entry}

\begin{entry}{处处}{5,5}
  \begin{phonetics}{处处}{chu4chu4}
    \definition{adv.}{em todos os lugares | em todos os aspectos}
  \end{phonetics}
\end{entry}

\begin{entry}{处罚}{5,9}
  \begin{phonetics}{处罚}{chu3fa2}
    \definition{v.}{penalizar | punir}
  \end{phonetics}
\end{entry}

\begin{entry}{外}{5}[Radical 夕]
  \begin{phonetics}{外}{wai4}
    \definition{s.}{fora | por fora | exterior | estrangeiro}
  \end{phonetics}
\end{entry}

\begin{entry}{外公}{5,4}
  \begin{phonetics}{外公}{wai4gong1}
    \definition{s.}{avô materno}
  \end{phonetics}
\end{entry}

\begin{entry}{外水}{5,4}
  \begin{phonetics}{外水}{wai4shui3}
    \definition{s.}{renda extra}
  \end{phonetics}
\end{entry}

\begin{entry}{外号}{5,5}
  \begin{phonetics}{外号}{wai4hao4}
    \definition{s.}{apelido}
  \end{phonetics}
\end{entry}

\begin{entry}{外边}{5,5}
  \begin{phonetics}{外边}{wai4bian5}
    \definition{adv.}{fora do país | superfície externa | fora | lugar diferente de sua casa}
  \end{phonetics}
\end{entry}

\begin{entry}{外交}{5,6}
  \begin{phonetics}{外交}{wai4jiao1}
    \definition{adj.}{diplomático}
    \definition[个]{s.}{diplomacia | relações exteriores}
  \end{phonetics}
\end{entry}

\begin{entry}{外协}{5,6}
  \begin{phonetics}{外协}{wai4xie2}
    \definition{s.}{terceirização | pessoas que julgam os outros pela aparência}
  \seealsoref{外貌协会}{wai4mao4xie2hui4}
  \end{phonetics}
\end{entry}

\begin{entry}{外孙}{5,6}
  \begin{phonetics}{外孙}{wai4sun1}
    \definition{s.}{filho da filha}
  \end{phonetics}
\end{entry}

\begin{entry}{外孙女}{5,6,3}
  \begin{phonetics}{外孙女}{wai4sun1nv3}
    \definition{s.}{filha da filha}
  \end{phonetics}
\end{entry}

\begin{entry}{外衣}{5,6}
  \begin{phonetics}{外衣}{wai4yi1}
    \definition{s.}{aparência | roupa de cima}
  \end{phonetics}
\end{entry}

\begin{entry}{外围}{5,7}
  \begin{phonetics}{外围}{wai4wei2}
    \definition{adv.}{arredores}
  \end{phonetics}
\end{entry}

\begin{entry}{外事}{5,8}
  \begin{phonetics}{外事}{wai4shi4}
    \definition{s.}{assuntos ou relações exteriores}
  \end{phonetics}
\end{entry}

\begin{entry}{外国}{5,8}
  \begin{phonetics}{外国}{wai4guo2}
    \definition[个]{s.}{país estrangeiro}
  \end{phonetics}
\end{entry}

\begin{entry}{外国人}{5,8,2}
  \begin{phonetics}{外国人}{wai4guo2ren2}
    \definition{s.}{estrangeiro | pessoa de fora do país}
  \end{phonetics}
\end{entry}

\begin{entry}{外语}{5,9}
  \begin{phonetics}{外语}{wai4yu3}
    \definition[门]{s.}{língua estrangeira}
  \end{phonetics}
\end{entry}

\begin{entry}{外贸}{5,9}
  \begin{phonetics}{外贸}{wai4mao4}
    \definition{s.}{comércio exterior}
  \end{phonetics}
\end{entry}

\begin{entry}{外面}{5,9}
  \begin{phonetics}{外面}{wai4mian4}
    \definition{adv.}{fora | por fora | exterior | superfície}
  \end{phonetics}
\end{entry}

\begin{entry}{外海}{5,10}
  \begin{phonetics}{外海}{wai4hai3}
    \definition{s.}{mar aberto}
  \end{phonetics}
\end{entry}

\begin{entry}{外积}{5,10}
  \begin{phonetics}{外积}{wai4ji1}
    \definition{s.}{produto exterior | (matemática) o produto vetorial de dois vetores}
  \end{phonetics}
\end{entry}

\begin{entry}{外婆}{5,11}
  \begin{phonetics}{外婆}{wai4po2}
    \definition{s.}{avó materna}
  \end{phonetics}
\end{entry}

\begin{entry}{外插}{5,12}
  \begin{phonetics}{外插}{wai4cha1}
    \definition{v.}{extrapolar | (computação) conectar (um dispositivo periférico, etc.)}
  \end{phonetics}
\end{entry}

\begin{entry}{外貌协会}{5,14,6,6}
  \begin{phonetics}{外貌协会}{wai4mao4xie2hui4}
    \definition{s.}{``o clube da boa aparência'': pessoas que dão grande importância à aparência de uma pessoa}
  \seealsoref{外协}{wai4xie2}
  \end{phonetics}
\end{entry}

\begin{entry}{失去}{5,5}
  \begin{phonetics}{失去}{shi1qu4}
    \definition{v.}{perder}
  \end{phonetics}
\end{entry}

\begin{entry}{失眠}{5,10}
  \begin{phonetics}{失眠}{shi1mian2}
    \definition{s.}{insônia}
    \definition{v.}{ter insônia}
  \end{phonetics}
\end{entry}

\begin{entry}{失望}{5,11}
  \begin{phonetics}{失望}{shi1wang4}
    \definition{adj.}{desapontado}
    \definition{v.}{perder a esperança | desesperar}
  \end{phonetics}
\end{entry}

\begin{entry}{失落}{5,12}
  \begin{phonetics}{失落}{shi1luo4}
    \definition{s.}{frustração | decepção | perda}
    \definition{v.}{perder (algo) | cair (algo) | sentir uma sensação de perda}
  \end{phonetics}
\end{entry}

\begin{entry}{失意}{5,13}
  \begin{phonetics}{失意}{shi1yi4}
    \definition{adj.}{desapontado | frustrado}
  \end{phonetics}
\end{entry}

\begin{entry}{头}{5}[Radical 大]
  \begin{phonetics}{头}{tou2}
    \definition{clas.}{para suínos ou gado}
    \definition[个]{s.}{cabeça}
  \end{phonetics}
  \begin{phonetics}{头}{tou5}
    \definition{suf.}{sufixo para nomes}
  \end{phonetics}
\end{entry}

\begin{entry}{头发}{5,5}
  \begin{phonetics}{头发}{tou2fa5}
    \definition{s.}{cabelo}
  \end{phonetics}
\end{entry}

\begin{entry}{头号}{5,5}
  \begin{phonetics}{头号}{tou2hao4}
    \definition{adj.}{primeira classe | número um | \emph{top rank}}
  \end{phonetics}
\end{entry}

\begin{entry}{头头}{5,5}
  \begin{phonetics}{头头}{tou2tou2}
    \definition{s.}{chefe | o cabeça}
  \end{phonetics}
\end{entry}

\begin{entry}{头脑风暴}{5,10,4,15}
  \begin{phonetics}{头脑风暴}{tou2nao3feng1bao4}
    \definition{s.}{\emph{brainstorm}}
  \end{phonetics}
\end{entry}

\begin{entry}{头像}{5,13}
  \begin{phonetics}{头像}{tou2xiang4}
    \definition{s.}{retrato | busto | avatar | imagem de perfil (computação)}
  \end{phonetics}
\end{entry}

\begin{entry}{奶奶}{5,5}
  \begin{phonetics}{奶奶}{nai3nai5}
    \definition[位]{s.}{avó (paterna) | (respeitoso) dona da casa}
  \end{phonetics}
\end{entry}

\begin{entry}{宁}{5}[Radical 宀]
  \begin{phonetics}{宁}{ning2}
    \definition*{s.}{sobrenome Ning}
    \definition{adj.}{calmo, pacífico, sereno | saudável}
  \end{phonetics}
  \begin{phonetics}{宁}{ning4}
    \definition{conj.}{mais\dots do que\dots, melhor\dots do que\dots}
  \end{phonetics}
\end{entry}

\begin{entry}{宁可}{5,5}
  \begin{phonetics}{宁可}{ning4ke3}
    \definition{conj.}{mais\dots do que\dots | melhor\dots do que\dots}
  \end{phonetics}
\end{entry}

\begin{entry}{宁可……也不……}{5,5,3,4}
  \begin{phonetics}{宁可……也不……}{ning4ke3 ye3bu4}
    \definition{conj.}{em vez de\dots}
  \end{phonetics}
\end{entry}

\begin{entry}{宁可……也要……}{5,5,3,9}
  \begin{phonetics}{宁可……也要……}{ning4ke3 ye3yao4}
    \definition{conj.}{mesmo que tenhamos que\dots nós iremos\dots}
  \end{phonetics}
\end{entry}

\begin{entry}{宁肯}{5,8}
  \begin{phonetics}{宁肯}{ning4ken3}
    \definition{conj.}{mais\dots do que\dots, melhor\dots do que\dots}
  \end{phonetics}
\end{entry}

\begin{entry}{宁愿}{5,14}
  \begin{phonetics}{宁愿}{ning4yuan4}
    \definition{conj.}{mais\dots do que\dots, melhor\dots do que\dots}
  \end{phonetics}
\end{entry}

\begin{entry}{它}{5}[Radical 宀]
  \begin{phonetics}{它}{ta1}
    \definition{pron.}{ele (para objetos inanimados) | se, o, lhe | si, consigo, eles}
  \end{phonetics}
\end{entry}

\begin{entry}{它们}{5,5}
  \begin{phonetics}{它们}{ta1men5}
    \definition{pron.}{eles (para objetos inanimados) | se, os, lhes | si, consigo, eles}
  \end{phonetics}
\end{entry}

\begin{entry}{对}{5}[Radical ⼨]
  \begin{phonetics}{对}{dui4}
    \definition{adj.}{correto | sim}
    \definition{clas.}{para casais}
    \definition{prep.}{com | para | para com}
  \end{phonetics}
\end{entry}

\begin{entry}{对不起}{5,4,10}
  \begin{phonetics}{对不起}{dui4bu5qi3}
    \definition{interj.}{Desculpe! | Desculpe-me! | Perdoe-me! | Desculpe? (por favor, repita)}
    \definition{v.}{desculpar | pedir desculpas | perdoar}
  \end{phonetics}
\end{entry}

\begin{entry}{对手}{5,4}
  \begin{phonetics}{对手}{dui4shou3}
    \definition{s.}{oponente | rival | concorrente | adversário}
  \end{phonetics}
\end{entry}

\begin{entry}{对……有兴趣}{5,6,6,15}
  \begin{phonetics}{对……有兴趣}{dui4 you3xing4qu4}
    \definition{expr.}{estar interessado em\dots | ter interesse em\dots | interessar-se por\dots}
    \seeref{对……感兴趣}{dui4 gan3xing4qu4}
  \end{phonetics}
\end{entry}

\begin{entry}{对话}{5,8}
  \begin{phonetics}{对话}{dui4hua4}
    \definition[个]{s.}{diálogo | conversa}
    \definition{v.}{dialogar | conversar}
  \end{phonetics}
\end{entry}

\begin{entry}{对……说}{5,9}
  \begin{phonetics}{对……说}{dui4 shuo5}
    \definition{v.}{dizer a alguém}
  \end{phonetics}
\end{entry}

\begin{entry}{对面}{5,9}
  \begin{phonetics}{对面}{dui4mian4}
    \definition{s.}{lado oposto}
  \end{phonetics}
\end{entry}

\begin{entry}{对得起}{5,11,10}
  \begin{phonetics}{对得起}{dui4de5qi3}
    \definition{v.}{não decepcionar alguém | tratar alguém de maneira justa | ser digno de}
  \end{phonetics}
\end{entry}

\begin{entry}{对……感兴趣}{5,13,6,15}
  \begin{phonetics}{对……感兴趣}{dui4 gan3xing4qu4}
    \definition{expr.}{estar interessado em\dots | ter interesse em\dots | interessar-se por\dots}
    \seeref{对……有兴趣}{dui4 you3xing4qu4}
  \end{phonetics}
\end{entry}

\begin{entry}{对……熟悉}{5,15,11}
  \begin{phonetics}{对……熟悉}{dui4 shu2xi1}
    \definition{expr.}{estar familiarizado com\dots}
  \end{phonetics}
\end{entry}

\begin{entry}{左}{5}[Radical 工]
  \begin{phonetics}{左}{zuo3}
    \definition*{s.}{sobrenome Zuo}
    \definition{p.l.}{esquerda}
  \end{phonetics}
\end{entry}

\begin{entry}{左右}{5,5}
  \begin{phonetics}{左右}{zuo3you4}
    \definition{adv.}{cerca de | aproximadamente}
  \end{phonetics}
\end{entry}

\begin{entry}{左边}{5,5}
  \begin{phonetics}{左边}{zuo3bian5}
    \definition{s.}{esquerda | lado esquerdo}
  \end{phonetics}
\end{entry}

\begin{entry}{左派}{5,9}
  \begin{phonetics}{左派}{zuo3pai4}
    \definition{s.}{(política) esquerda | esquerdista}
  \end{phonetics}
\end{entry}

\begin{entry}{左面}{5,9}
  \begin{phonetics}{左面}{zuo3mian4}
    \definition{s.}{esquerda | lado esquerdo}
  \end{phonetics}
\end{entry}

\begin{entry}{左倾}{5,10}
  \begin{phonetics}{左倾}{zuo3qing1}
    \definition{s.}{esquerdista | progressivo}
  \end{phonetics}
\end{entry}

\begin{entry}{左袒}{5,10}
  \begin{phonetics}{左袒}{zuo3tan3}
    \definition{v.}{ser tendencioso | ser parcial para | favorecer um lado | tomar partido com}
  \end{phonetics}
\end{entry}

\begin{entry}{左舷}{5,11}
  \begin{phonetics}{左舷}{zuo3xian2}
    \definition{s.}{porto (lado de um navio)}
  \end{phonetics}
\end{entry}

\begin{entry}{左翼}{5,17}
  \begin{phonetics}{左翼}{zuo3yi4}
    \definition{s.}{esquerda (política)}
  \end{phonetics}
\end{entry}

\begin{entry}{巧合}{5,6}
  \begin{phonetics}{巧合}{qiao3he2}
    \definition{s.}{coincidência}
    \definition{v.}{coincidir}
  \end{phonetics}
\end{entry}

\begin{entry}{巧克力}{5,7,2}
  \begin{phonetics}{巧克力}{qiao3ke4li4}
    \definition[块]{s.}{(empréstimo linguístico) chocolate}
  \end{phonetics}
\end{entry}

\begin{entry}{市中心}{5,4,4}
  \begin{phonetics}{市中心}{shi4zhong1xin1}
    \definition{s.}{centro da cidade}
  \end{phonetics}
\end{entry}

\begin{entry}{市区}{5,4}
  \begin{phonetics}{市区}{shi4qu1}
    \definition{s.}{centro da cidade | distrito urbano}
  \end{phonetics}
\end{entry}

\begin{entry}{市场}{5,6}
  \begin{phonetics}{市场}{shi4chang3}
    \definition{s.}{mercado (também no abstrato)}
  \end{phonetics}
\end{entry}

\begin{entry}{布}{5}[Radical 巾]
  \begin{phonetics}{布}{bu4}
    \definition{s.}{pano | tecido}
    \definition{v.}{declarar | anunciar | espalhar | fazer conhecer}
  \end{phonetics}
\end{entry}

\begin{entry}{布谷鸟}{5,7,5}
  \begin{phonetics}{布谷鸟}{bu4gu3niao3}
    \definition{s.}{cuco (pássaro)}
  \seealsoref{杜鹃}{du4juan1}
  \seealsoref{杜鹃鸟}{du4juan1niao3}
  \seealsoref{杜宇}{du4yu3}
  \end{phonetics}
\end{entry}

\begin{entry}{布署}{5,13}
  \begin{phonetics}{布署}{bu4shu3}
    \variantof{部署}
  \end{phonetics}
\end{entry}

\begin{entry}{帅}{5}[Radical 巾]
  \begin{phonetics}{帅}{shuai4}
    \definition*{s.}{sobrenome Shuai}
    \definition{adj.}{elegante | agradável à vista | gracioso | inteligente}
    \definition{interj.}{Legal!}
    \definition{s.}{comandante em chefe}
  \end{phonetics}
\end{entry}

\begin{entry}{平}{5}[Radical 干]
  \begin{phonetics}{平}{ping2}
    \definition*{s.}{sobrenome Ping}
    \definition{adj.}{calmo | pacífico}
    \definition{s.}{plano | nível}
    \definition{v.}{fazer a mesma pontuação | marcar uma pontuação}
  \end{phonetics}
\end{entry}

\begin{entry}{平台}{5,5}
  \begin{phonetics}{平台}{ping2tai2}
    \definition{s.}{plataforma | terraço | edifício de telhado plano}
  \end{phonetics}
\end{entry}

\begin{entry}{平地}{5,6}
  \begin{phonetics}{平地}{ping2di4}
    \definition{v.}{nivelar a terra | aplanar}
  \end{phonetics}
\end{entry}

\begin{entry}{平时}{5,7}
  \begin{phonetics}{平时}{ping2shi2}
    \definition{adv.}{normalmente | em tempos normais | em tempos de paz}
  \end{phonetics}
\end{entry}

\begin{entry}{幼儿园}{5,2,7}
  \begin{phonetics}{幼儿园}{you4'er2yuan2}
    \definition{s.}{jardim de infância | berçário}
  \end{phonetics}
\end{entry}

\begin{entry}{归}{5}[Radical ⼹]
  \begin{phonetics}{归}{gui1}
    \definition*{s.}{sobrenome Gui}
    \definition{s.}{divisão no ábaco com divisor de um dígito}
    \definition{v.}{retornar | voltar a | retribuir a | (uma responsabilidade) a ser resolvido por | pertencer | reunir-se | (usado entre dois verbos idênticos) apesar}
  \end{phonetics}
\end{entry}

\begin{entry}{必定}{5,8}
  \begin{phonetics}{必定}{bi4ding4}
    \definition{adv.}{sem falta | certamente | com certeza | definitivamente | inevitavelmente | com determinação}
    \definition{v.}{estar vinculado a | ter certeza de}
  \end{phonetics}
\end{entry}

\begin{entry}{必然}{5,12}
  \begin{phonetics}{必然}{bi4ran2}
    \definition{adv.}{sem falta | certamente | definitivamente | inevitavelmente}
  \end{phonetics}
\end{entry}

\begin{entry}{扑克}{5,7}
  \begin{phonetics}{扑克}{pu1ke4}
    \definition{s.}{(empréstimo linguístico) (jogo) \emph{poker}  | baralho}
  \end{phonetics}
\end{entry}

\begin{entry}{扒犁}{5,11}
  \begin{phonetics}{扒犁}{pa2li2}
    \definition{s.}{trenó}
    \seeref{爬犁}{pa2li2}
  \end{phonetics}
\end{entry}

\begin{entry}{打}{5}[Radical 手]
  \begin{phonetics}{打}{da2}
    \definition{s.}{(empréstimo linguístico) dúzia}
  \end{phonetics}
  \begin{phonetics}{打}{da3}
    \definition{adv.}{desde}
    \definition{v.}{jogar (um jogo) | bater | atacar | acertar | quebrar | digitar | misturar | construir | lutar | pegar | fazer | amarrar | atirar | calcular}
  \end{phonetics}
\end{entry}

\begin{entry}{打工}{5,3}
  \begin{phonetics}{打工}{da3gong1}
    \definition{v.}{(para alunos) ter um emprego fora do horário de aula ou durante as férias | trabalhar em um emprego temporá rio ou casual}
  \end{phonetics}
\end{entry}

\begin{entry}{打工人}{5,3,2}
  \begin{phonetics}{打工人}{da3gong1ren2}
    \definition{s.}{trabalhador}
  \end{phonetics}
\end{entry}

\begin{entry}{打电话}{5,5,8}
  \begin{phonetics}{打电话}{da3dian4hua4}
    \definition{v.}{telefonar | fazer uma chamada telefônica | dar um telefonema}
  \seealsoref{给……打电话}{gei3 da3dian4hua4}
  \end{phonetics}
\end{entry}

\begin{entry}{打压}{5,6}
  \begin{phonetics}{打压}{da3ya1}
    \definition{v.}{reprimir | derrotar}
  \end{phonetics}
\end{entry}

\begin{entry}{打屁股}{5,7,8}
  \begin{phonetics}{打屁股}{da3pi4gu5}
    \definition{v.}{dar um tapa no bumbum de alguém}
  \end{phonetics}
\end{entry}

\begin{entry}{打扮}{5,7}
  \begin{phonetics}{打扮}{da3ban5}
    \definition{v.}{arranjar-se | enfeitar-se}
  \end{phonetics}
\end{entry}

\begin{entry}{打扰}{5,7}
  \begin{phonetics}{打扰}{da3rao3}
    \definition{v.}{perturbar | incomodar}
  \end{phonetics}
\end{entry}

\begin{entry}{打针}{5,7}
  \begin{phonetics}{打针}{da3zhen1}
    \definition{v.+compl.}{dar injeção | levar injeção}
  \end{phonetics}
\end{entry}

\begin{entry}{打的}{5,8}
  \begin{phonetics}{打的}{da3di1}
    \definition{v.+compl.}{(coloquial) pegar um táxi | ir de táxi}
  \end{phonetics}
\end{entry}

\begin{entry}{打架}{5,9}
  \begin{phonetics}{打架}{da3jia4}
    \definition{v.+compl.}{lutar | brigar | participar de lutas, brigas}
  \end{phonetics}
\end{entry}

\begin{entry}{打结}{5,9}
  \begin{phonetics}{打结}{da3jie2}
    \definition{v.}{dar um nó | amarrar}
  \end{phonetics}
\end{entry}

\begin{entry}{打骂}{5,9}
  \begin{phonetics}{打骂}{da3ma4}
    \definition{v.}{bater e repreender}
  \end{phonetics}
\end{entry}

\begin{entry}{打猎}{5,11}
  \begin{phonetics}{打猎}{da3lie4}
    \definition{v.}{ir caçar}
  \end{phonetics}
\end{entry}

\begin{entry}{打球}{5,11}
  \begin{phonetics}{打球}{da3qiu2}
    \definition{v.}{jogar bola (com as mãos) | jogar (basquetebol, handbol, etc.)}
  \end{phonetics}
\end{entry}

\begin{entry}{打搅}{5,12}
  \begin{phonetics}{打搅}{da3jiao3}
    \definition{v.}{perturbar | incomodar}
  \end{phonetics}
\end{entry}

\begin{entry}{打算}{5,14}
  \begin{phonetics}{打算}{da3suan4}
    \definition[个]{s.}{plano | intenção}
    \definition{v.}{pensar | planejar | pretender}
  \end{phonetics}
\end{entry}

\begin{entry}{打瞌睡}{5,15,13}
  \begin{phonetics}{打瞌睡}{da3ke1shui4}
    \definition{v.}{cochilar}
  \end{phonetics}
\end{entry}

\begin{entry}{打磨}{5,16}
  \begin{phonetics}{打磨}{da3mo2}
    \definition{v.}{polir | fazer brilhar}
  \end{phonetics}
\end{entry}

\begin{entry}{扔}{5}[Radical 手]
  \begin{phonetics}{扔}{reng1}
    \definition{v.}{lançar | atirar}
  \end{phonetics}
\end{entry}

\begin{entry}{扔下}{5,3}
  \begin{phonetics}{扔下}{reng1xia4}
    \definition{v.}{lançar (uma bomba) | derrubar}
  \end{phonetics}
\end{entry}

\begin{entry}{扔弃}{5,7}
  \begin{phonetics}{扔弃}{reng1qi4}
    \definition{v.}{abandonar | descartar | jogar fora}
  \end{phonetics}
\end{entry}

\begin{entry}{扔掉}{5,11}
  \begin{phonetics}{扔掉}{reng1diao4}
    \definition{v.}{jogar fora}
  \end{phonetics}
\end{entry}

\begin{entry}{斥骂}{5,9}
  \begin{phonetics}{斥骂}{chi4ma4}
    \definition{v.}{repreender}
  \end{phonetics}
\end{entry}

\begin{entry}{旧}{5}[Radical 日]
  \begin{phonetics}{旧}{jiu4}
    \definition{adj.}{velho | antigo | desgastado (com a idade)}
  \end{phonetics}
\end{entry}

\begin{entry}{未}{5}[Radical 木]
  \begin{phonetics}{未}{wei4}
    \definition{adv.}{não ter | ainda não}
  \end{phonetics}
\end{entry}

\begin{entry}{未必}{5,5}
  \begin{phonetics}{未必}{wei4bi4}
    \definition{adv.}{não pode | não necessariamente}
  \end{phonetics}
\end{entry}

\begin{entry}{本}{5}[Radical 木]
  \begin{phonetics}{本}{ben3}
    \definition{adj.}{o atual | original | inerente}
    \definition{adv.}{originalmente}
    \definition{clas.}{para livros, dicionários, periódicos, arquivos, etc.}
    \definition{s.}{raiz | caule | origem | fonte}
  \end{phonetics}
\end{entry}

\begin{entry}{本子}{5,3}
  \begin{phonetics}{本子}{ben3zi5}
    \definition[本]{s.}{caderno}
  \end{phonetics}
\end{entry}

\begin{entry}{本来}{5,7}
  \begin{phonetics}{本来}{ben3lai2}
    \definition{adv.}{originalmente | apropriadamente | legalmente}
  \end{phonetics}
\end{entry}

\begin{entry}{正}{5}[Radical 止]
  \begin{phonetics}{正}{zheng1}
    \definition{s.}{primeiro mês do ano lunar}
  \end{phonetics}
  \begin{phonetics}{正}{zheng4}
    \definition{adj.}{reto | vertical | adequado | principal | (matemática) positivo}
    \definition{adv.}{agora mesmo | no processo de}
    \definition{v.}{corrigir | retificar}
  \end{phonetics}
\end{entry}

\begin{entry}{正正}{5,5}
  \begin{phonetics}{正正}{zheng4zheng4}
    \definition{adv.}{na hora certa | ordenadamente}
  \end{phonetics}
\end{entry}

\begin{entry}{正在}{5,6}
  \begin{phonetics}{正在}{zheng4zai4}
    \definition{adv.}{no processo de | atualmente | em andamento}
    \definition{v.}{estar a~+~v.inf. | estar~+~v.ger.}
  \end{phonetics}
\end{entry}

\begin{entry}{正宗}{5,8}
  \begin{phonetics}{正宗}{zheng4zong1}
    \definition{adj.}{autêntico | genuíno | \emph{old school} | (fig.) tradicional}
  \end{phonetics}
\end{entry}

\begin{entry}{正常}{5,11}
  \begin{phonetics}{正常}{zheng4chang2}
    \definition{adj.}{regular | normal | ordinário}
  \end{phonetics}
\end{entry}

\begin{entry}{母亲}{5,9}
  \begin{phonetics}{母亲}{mu3qin1}
    \definition[个]{s.}{mãe}
  \end{phonetics}
\end{entry}

\begin{entry}{母语}{5,9}
  \begin{phonetics}{母语}{mu3yu3}
    \definition{s.}{língua materna | língua nativa}
  \end{phonetics}
\end{entry}

\begin{entry}{民主}{5,5}
  \begin{phonetics}{民主}{min2zhu3}
    \definition{adj.}{democrático}
    \definition{s.}{democracia}
  \end{phonetics}
\end{entry}

\begin{entry}{民众}{5,6}
  \begin{phonetics}{民众}{min2zhong4}
    \definition{s.}{a população | as massas | as pessoas comuns}
  \end{phonetics}
\end{entry}

\begin{entry}{永不}{5,4}
  \begin{phonetics}{永不}{yong3bu4}
    \definition{adv.}{nunca}
  \end{phonetics}
\end{entry}

\begin{entry}{永远}{5,7}
  \begin{phonetics}{永远}{yong3yuan3}
    \definition{adv.}{para sempre, sempre | permanentemente}
  \end{phonetics}
\end{entry}

\begin{entry}{汉}{5}[Radical 水]
  \begin{phonetics}{汉}{han4}
    \definition{s.}{grupo étnico Han | chinês (língua) | dinastia Han (206 a.C.-220d.C.) | homem}
  \end{phonetics}
\end{entry}

\begin{entry}{汉字}{5,6}
  \begin{phonetics}{汉字}{han4zi4}
    \definition[个]{s.}{caracter chinês}
  \end{phonetics}
\end{entry}

\begin{entry}{汉服}{5,8}
  \begin{phonetics}{汉服}{han4fu2}
    \definition{s.}{vestido chinês tradicional Han}
  \end{phonetics}
\end{entry}

\begin{entry}{汉语}{5,9}
  \begin{phonetics}{汉语}{han4yu3}
    \definition[门]{s.}{língua chinesa, mandarim}
  \end{phonetics}
\end{entry}

\begin{entry}{汉堡王}{5,12,4}
  \begin{phonetics}{汉堡王}{han4bao3wang2}
    \definition*{s.}{Burguer King (restaurante \emph{fast-food})}
  \end{phonetics}
\end{entry}

\begin{entry}{汉堡包}{5,12,5}
  \begin{phonetics}{汉堡包}{han4bao3bao1}
    \definition[个]{s.}{hambúrguer}
  \end{phonetics}
\end{entry}

\begin{entry}{汉葡词典}{5,12,7,8}
  \begin{phonetics}{汉葡词典}{han4-pu2 ci2dian3}
    \definition[部,本]{s.}{dicionário chinês-português}
  \seealsoref{葡汉词典}{pu2-han4 ci2dian3}
  \end{phonetics}
\end{entry}

\begin{entry}{灭火}{5,4}
  \begin{phonetics}{灭火}{mie4huo3}
    \definition{s.}{combate a incêndios}
    \definition{v.}{extinguir um incêndio}
  \end{phonetics}
\end{entry}

\begin{entry}{犯法}{5,8}
  \begin{phonetics}{犯法}{fan4fa3}
    \definition{v.}{violar (quebrar) a lei}
  \end{phonetics}
\end{entry}

\begin{entry}{犯罪}{5,13}
  \begin{phonetics}{犯罪}{fan4zui4}
    \definition{v.+compl.}{cometer  um crime (uma ofensa)}
  \end{phonetics}
\end{entry}

\begin{entry}{玄学}{5,8}
  \begin{phonetics}{玄学}{xuan2xue2}
    \definition{s.}{Escola Philosófica Wei e Jin amalgamando os ideais daoísta e confucionistas | tradução da metafísica (形而上学)}
    \seeref{形而上学}{xing2'er2shang4xue2}
  \end{phonetics}
\end{entry}

\begin{entry}{玉}{5}[Radical 玉][Kangxi 96]
  \begin{phonetics}{玉}{yu4}
    \definition[块]{s.}{jade}
  \end{phonetics}
\end{entry}

\begin{entry}{玉米}{5,6}
  \begin{phonetics}{玉米}{yu4mi3}
    \definition[粒]{s.}{milho}
  \end{phonetics}
\end{entry}

\begin{entry}{玉米片}{5,6,4}
  \begin{phonetics}{玉米片}{yu4mi3pian4}
    \definition{s.}{flocos de milho | chips de tortilha}
  \end{phonetics}
\end{entry}

\begin{entry}{玉米花}{5,6,7}
  \begin{phonetics}{玉米花}{yu4mi3hua1}
    \definition{s.}{pipoca}
  \end{phonetics}
\end{entry}

\begin{entry}{玉米面}{5,6,9}
  \begin{phonetics}{玉米面}{yu4mi3mian4}
    \definition{s.}{fubá | farinha de milho}
  \end{phonetics}
\end{entry}

\begin{entry}{玉米饼}{5,6,9}
  \begin{phonetics}{玉米饼}{yu4mi3bing3}
    \definition{s.}{tortilha mexicana | bolo de milho}
  \end{phonetics}
\end{entry}

\begin{entry}{玉米笋}{5,6,10}
  \begin{phonetics}{玉米笋}{yu4mi3sun3}
    \definition{s.}{broto de milho}
  \end{phonetics}
\end{entry}

\begin{entry}{玉米粉}{5,6,10}
  \begin{phonetics}{玉米粉}{yu4mi3fen3}
    \definition{s.}{amido de milho | farinha de milho}
  \end{phonetics}
\end{entry}

\begin{entry}{玉米糁}{5,6,14}
  \begin{phonetics}{玉米糁}{yu4mi3san3}
    \definition{s.}{grãos de milho}
  \end{phonetics}
\end{entry}

\begin{entry}{玉米糕}{5,6,16}
  \begin{phonetics}{玉米糕}{yu4mi3gao1}
    \definition{s.}{bolo de milho | polenta}
  \end{phonetics}
\end{entry}

\begin{entry}{甘心}{5,4}
  \begin{phonetics}{甘心}{gan1xin1}
    \definition{v.}{estar disposto a | resignar-se a}
  \end{phonetics}
\end{entry}

\begin{entry}{甘薯}{5,16}
  \begin{phonetics}{甘薯}{gan1shu3}
    \definition{s.}{batata doce}
  \end{phonetics}
\end{entry}

\begin{entry}{生}{5}[Radical 生][Kangxi 100]
  \begin{phonetics}{生}{sheng1}
    \definition{adj.}{vida | estudante | cru | não cozido}
    \definition{v.}{nascer | dar a luz | crescer}
  \end{phonetics}
\end{entry}

\begin{entry}{生日}{5,4}
  \begin{phonetics}{生日}{sheng1ri4}
    \definition[个]{s.}{aniversário}
  \end{phonetics}
\end{entry}

\begin{entry}{生气}{5,4}
  \begin{phonetics}{生气}{sheng1qi4}
    \definition{s.}{vitalidade | vigor}
    \definition{v.+compl.}{irritar-se | zangar-se | ofender-se | ficar com raiva}
  \end{phonetics}
\end{entry}

\begin{entry}{生长}{5,4}
  \begin{phonetics}{生长}{sheng1zhang3}
    \definition{v.}{crescer | amadurecer | ser criado}
  \end{phonetics}
\end{entry}

\begin{entry}{生态}{5,8}
  \begin{phonetics}{生态}{sheng1tai4}
    \definition{adj.}{ecológico}
    \definition{s.}{ecologia}
  \end{phonetics}
\end{entry}

\begin{entry}{生物}{5,8}
  \begin{phonetics}{生物}{sheng1wu4}
    \definition{adj.}{biológico}
    \definition{s.}{biologia (disciplina) | organismo | ser vivo}
  \end{phonetics}
\end{entry}

\begin{entry}{生的}{5,8}
  \begin{phonetics}{生的}{sheng1de5}
    \definition{conj.}{para evitar isso | para que\dots não\dots}
  \end{phonetics}
\end{entry}

\begin{entry}{生鱼片}{5,8,4}
  \begin{phonetics}{生鱼片}{sheng1yu2pian4}
    \definition{s.}{fatias de peixe cru, \emph{sashimi}}
  \end{phonetics}
\end{entry}

\begin{entry}{生活}{5,9}
  \begin{phonetics}{生活}{sheng1huo2}
    \definition[道]{s.}{vida | atividade | meios de subsistência}
    \definition{v.}{viver}
  \end{phonetics}
\end{entry}

\begin{entry}{生活垃圾}{5,9,8,6}
  \begin{phonetics}{生活垃圾}{sheng1huo2la1ji1}
    \definition{s.}{lixo doméstico}
  \end{phonetics}
\end{entry}

\begin{entry}{生活型}{5,9,9}
  \begin{phonetics}{生活型}{sheng1huo2 xing2}
    \definition{s.}{forma de vida}
  \end{phonetics}
\end{entry}

\begin{entry}{生理}{5,11}
  \begin{phonetics}{生理}{sheng1li3}
    \definition{adj.}{fisiológico}
    \definition{s.}{fisiologia}
  \end{phonetics}
\end{entry}

\begin{entry}{生菜}{5,11}
  \begin{phonetics}{生菜}{sheng1cai4}
    \definition{s.}{alface}
  \end{phonetics}
\end{entry}

\begin{entry}{生意}{5,13}
  \begin{phonetics}{生意}{sheng1yi4}
    \definition{s.}{força vital | vitalidade}
  \end{phonetics}
  \begin{phonetics}{生意}{sheng1yi5}
    \definition{s.}{negócio}
  \end{phonetics}
\end{entry}

\begin{entry}{用}{5}[Radical 用][Kangxi 101]
  \begin{phonetics}{用}{yong4}
    \definition{v.}{usar}
  \end{phonetics}
\end{entry}

\begin{entry}{用心}{5,4}
  \begin{phonetics}{用心}{yong4xin1}
    \definition{s.}{motivo | intenção}
    \definition{v.+compl.}{ser diligente ou atencioso}
  \end{phonetics}
\end{entry}

\begin{entry}{用处}{5,5}
  \begin{phonetics}{用处}{yong4chu5}
    \definition[个]{s.}{usabilidade | utilidade}
  \end{phonetics}
\end{entry}

\begin{entry}{用料}{5,10}
  \begin{phonetics}{用料}{yong4liao4}
    \definition{s.}{ingredientes | materiais}
  \end{phonetics}
\end{entry}

\begin{entry}{田}{5}[Radical 田][Kangxi 102]
  \begin{phonetics}{田}{tian2}
    \definition*{s.}{sobrenome Tian}
    \definition[片]{s.}{fazenda | campo}
  \end{phonetics}
\end{entry}

\begin{entry}{田园}{5,7}
  \begin{phonetics}{田园}{tian2yuan2}
    \definition{adj.}{bucólico}
    \definition{s.}{campo | interior | rural}
  \end{phonetics}
\end{entry}

\begin{entry}{甲骨文}{5,9,4}
  \begin{phonetics}{甲骨文}{jia3gu3wen2}
    \definition{s.}{escrituras de oráculos | inscrições em ossos de oráculos (forma original de escritura chinesa)}
  \end{phonetics}
\end{entry}

\begin{entry}{电子}{5,3}
  \begin{phonetics}{电子}{dian4zi3}
    \definition{s.}{eletrônico | elétron}
  \end{phonetics}
\end{entry}

\begin{entry}{电子名片}{5,3,6,4}
  \begin{phonetics}{电子名片}{dian4zi3 ming2pian4}
    \definition{s.}{cartão de visita eletrônico}
  \end{phonetics}
\end{entry}

\begin{entry}{电子邮件}{5,3,7,6}
  \begin{phonetics}{电子邮件}{dian4zi3you2jian4}
    \definition[封,份]{s.}{correio eletrônico, \emph{e-mail}}
  \seealsoref{电邮}{dian4you2}
  \end{phonetics}
\end{entry}

\begin{entry}{电车司机}{5,4,5,6}
  \begin{phonetics}{电车司机}{dian4che1 si1ji1}
    \definition{s.}{motorista de bonde}
  \end{phonetics}
\end{entry}

\begin{entry}{电冰箱}{5,6,15}
  \begin{phonetics}{电冰箱}{dian4bing1xiang1}
    \definition[台]{s.}{frigorífico | refrigerador}
  \end{phonetics}
\end{entry}

\begin{entry}{电动}{5,6}
  \begin{phonetics}{电动}{dian4dong4}
    \definition{adj.}{movido a eletricidade | elétrico}
  \end{phonetics}
\end{entry}

\begin{entry}{电动车}{5,6,4}
  \begin{phonetics}{电动车}{dian4dong4che1}
    \definition{s.}{veículo elétrico (\emph{scooter}, bicicleta, carro, etc.)}
  \end{phonetics}
\end{entry}

\begin{entry}{电灯泡}{5,6,8}
  \begin{phonetics}{电灯泡}{dian4deng1pao4}
    \definition{s.}{lâmpada elétrica | (gíria) terceiro convidado indesejado}
  \end{phonetics}
\end{entry}

\begin{entry}{电邮}{5,7}
  \begin{phonetics}{电邮}{dian4you2}
    \definition{s.}{correio eletrônico, \emph{e-mail} | abreviação de~电子邮件}
  \seealsoref{电子邮件}{dian4zi3you2jian4}
  \end{phonetics}
\end{entry}

\begin{entry}{电视}{5,8}
  \begin{phonetics}{电视}{dian4shi4}
    \definition[台,个]{s.}{televisão | TV | televisor}
  \end{phonetics}
\end{entry}

\begin{entry}{电视机}{5,8,6}
  \begin{phonetics}{电视机}{dian4shi4ji1}
    \definition[台]{s.}{aparelho de televisão | televisor}
  \end{phonetics}
\end{entry}

\begin{entry}{电话}{5,8}
  \begin{phonetics}{电话}{dian4hua4}
    \definition[部]{s.}{telefone}
    \definition[通]{s.}{chamada telefônica}
  \end{phonetics}
\end{entry}

\begin{entry}{电脑}{5,10}
  \begin{phonetics}{电脑}{dian4nao3}
    \definition[台]{s.}{computador}
  \end{phonetics}
\end{entry}

\begin{entry}{电脑语言}{5,10,9,7}
  \begin{phonetics}{电脑语言}{dian4nao3yu3yan2}
    \definition{s.}{linguagem de programação | linguagem de computador}
  \end{phonetics}
\end{entry}

\begin{entry}{电梯}{5,11}
  \begin{phonetics}{电梯}{dian4ti1}
    \definition[台,部]{s.}{elevador | ascensor}
  \end{phonetics}
\end{entry}

\begin{entry}{电梯司机}{5,11,5,6}
  \begin{phonetics}{电梯司机}{dian4ti1 si1ji1}
    \definition{s.}{ascensorista}
  \end{phonetics}
\end{entry}

\begin{entry}{电影}{5,15}
  \begin{phonetics}{电影}{dian4ying3}
    \definition[部,片,幕,场]{s.}{filme}
  \end{phonetics}
\end{entry}

\begin{entry}{电影艺术}{5,15,4,5}
  \begin{phonetics}{电影艺术}{dian4ying3 yi4shu4}
    \definition{s.}{arte cinematográfica}
  \end{phonetics}
\end{entry}

\begin{entry}{电影术}{5,15,5}
  \begin{phonetics}{电影术}{dian4ying3 shu4}
    \definition{s.}{cinematografia}
  \end{phonetics}
\end{entry}

\begin{entry}{电影节}{5,15,5}
  \begin{phonetics}{电影节}{dian4ying3jie2}
    \definition{s.}{festival de cinema}
  \end{phonetics}
\end{entry}

\begin{entry}{电影奖}{5,15,9}
  \begin{phonetics}{电影奖}{dian4ying3jiang3}
    \definition{s.}{premiações de cinema}
  \end{phonetics}
\end{entry}

\begin{entry}{电影界}{5,15,9}
  \begin{phonetics}{电影界}{dian4ying3jie4}
    \definition{s.}{indústria cinematográfica}
  \end{phonetics}
\end{entry}

\begin{entry}{电影院}{5,15,9}
  \begin{phonetics}{电影院}{dian4ying3yuan4}
    \definition[次,家,座]{s.}{sala de cinema}
  \end{phonetics}
\end{entry}

\begin{entry}{电影音乐}{5,15,9,5}
  \begin{phonetics}{电影音乐}{dian4ying3 yin1yue4}
    \definition{s.}{música cinematográfica}
  \end{phonetics}
\end{entry}

\begin{entry}{电影票}{5,15,11}
  \begin{phonetics}{电影票}{dian4ying3piao4}
    \definition{s.}{ingresso de filme}
  \end{phonetics}
\end{entry}

\begin{entry}{电器}{5,16}
  \begin{phonetics}{电器}{dian4qi4}
    \definition{s.}{aparelho elétrico}
  \end{phonetics}
\end{entry}

\begin{entry}{白}{5}[Radical 白][Kangxi 106]
  \begin{phonetics}{白}{bai2}
    \definition*{s.}{sobrenome Bai}
    \definition{adj.}{branco | claro | puro | límpido | simples | em branco | grátis}
    \definition{adv.}{em vão | sem propósito | por nada}
    \definition{s.}{parte falada na ópera | diálogo | dialeto}
  \end{phonetics}
\end{entry}

\begin{entry}{白天}{5,4}
  \begin{phonetics}{白天}{bai2tian1}
    \definition{adv.}{dia | de dia}
    \definition[个]{s.}{dia}
  \end{phonetics}
\end{entry}

\begin{entry}{白色}{5,6}
  \begin{phonetics}{白色}{bai2se4}
    \definition{s.}{cor branca}
  \end{phonetics}
\end{entry}

\begin{entry}{白苋}{5,7}
  \begin{phonetics}{白苋}{bai2xian4}
    \definition{s.}{amaranto branco | brotos e folhas tenras de espinafre chinês usados como alimento}
  \end{phonetics}
\end{entry}

\begin{entry}{白拣}{5,8}
  \begin{phonetics}{白拣}{bai2jian3}
    \definition{s.}{uma escolha barata}
    \definition{v.}{escolher algo que não custa nada}
  \end{phonetics}
\end{entry}

\begin{entry}{白菜}{5,11}
  \begin{phonetics}{白菜}{bai2cai4}
    \definition[棵,个]{s.}{acelga | repolho chinês}
  \end{phonetics}
\end{entry}

\begin{entry}{白萝卜}{5,11,2}
  \begin{phonetics}{白萝卜}{bai2luo2bo5}
    \definition{s.}{rabanete branco}
  \end{phonetics}
\end{entry}

\begin{entry}{白蛋白}{5,11,5}
  \begin{phonetics}{白蛋白}{bai2dan4bai2}
    \definition{s.}{albumina}
  \end{phonetics}
\end{entry}

\begin{entry}{白鹄}{5,12}
  \begin{phonetics}{白鹄}{bai2hu2}
    \definition{s.}{cisne branco}
  \end{phonetics}
\end{entry}

\begin{entry}{白痴}{5,13}
  \begin{phonetics}{白痴}{bai2chi1}
    \definition{adj./s.}{estúpido | imbecil}
  \end{phonetics}
\end{entry}

\begin{entry}{皮}{5}[Radical 皮][Kangxi 107]
  \begin{phonetics}{皮}{pi2}
    \definition*{s.}{sobrenome Pi}
    \definition{adj.}{safado}
    \definition{pref.}{``pico'' (um trilhonésimo)}
    \definition[张]{s.}{couro | pele | pelagem}
  \end{phonetics}
\end{entry}

\begin{entry}{皮下}{5,3}
  \begin{phonetics}{皮下}{pi2xia4}
    \definition{adj.}{(injeção) subcutâneo | sob a pele}
  \end{phonetics}
\end{entry}

\begin{entry}{皮卡}{5,5}
  \begin{phonetics}{皮卡}{pi2ka3}
    \definition{s.}{(empréstimo linguístico) \emph{pick-up} | caminhonete}
  \end{phonetics}
\end{entry}

\begin{entry}{皮卡丘}{5,5,5}
  \begin{phonetics}{皮卡丘}{pi2ka3qiu1}
    \definition*{s.}{\emph{Pikachu} (Pokémon, 口袋妖怪)}
  \seealsoref{口袋妖怪}{kou3dai4 yao1guai4}
  \end{phonetics}
\end{entry}

\begin{entry}{皮肤}{5,8}
  \begin{phonetics}{皮肤}{pi2fu1}
    \definition[层,块]{s.}{pele}
  \end{phonetics}
\end{entry}

\begin{entry}{礼节}{5,5}
  \begin{phonetics}{礼节}{li3jie2}
    \definition{s.}{protocolo | cerimônia | etiqueta}
  \end{phonetics}
\end{entry}

\begin{entry}{礼让}{5,5}
  \begin{phonetics}{礼让}{li3rang4}
    \definition{s.}{cortesia}
    \definition{v.}{mostrar consideração por (outros) | ceder a (outro veículo, etc.)}
  \end{phonetics}
\end{entry}

\begin{entry}{礼物}{5,8}
  \begin{phonetics}{礼物}{li3wu4}
    \definition[件,个,份]{s.}{prenda | lembrança | presente}
  \end{phonetics}
\end{entry}

\begin{entry}{立刻}{5,8}
  \begin{phonetics}{立刻}{li4ke4}
    \definition{adv.}{imediatamente}
  \end{phonetics}
\end{entry}

\begin{entry}{立法}{5,8}
  \begin{phonetics}{立法}{li4fa3}
    \definition{s.}{legislação}
    \definition{v.}{promulgar leis | legislar}
  \end{phonetics}
\end{entry}

\begin{entry}{纠葛}{5,12}
  \begin{phonetics}{纠葛}{jiu1ge2}
    \definition{s.}{emaranhado | disputa}
  \end{phonetics}
\end{entry}

\begin{entry}{节日}{5,4}
  \begin{phonetics}{节日}{jie2ri4}
    \definition[个]{s.}{festival | feriado}
  \end{phonetics}
\end{entry}

\begin{entry}{节奏}{5,9}
  \begin{phonetics}{节奏}{jie2zou4}
    \definition{s.}{ritmo | cadência | batida}
  \end{phonetics}
\end{entry}

\begin{entry}{讨生活}{5,5,9}
  \begin{phonetics}{讨生活}{tao3sheng1huo2}
    \definition{v.}{ganhar a vida}
  \end{phonetics}
\end{entry}

\begin{entry}{让}{5}[Radical 言]
  \begin{phonetics}{让}{rang4}
    \definition{v.}{deixar alguém fazer alguma coisa |fazer alguém (sentir-se triste, etc.) | permitir | conceder}
  \end{phonetics}
\end{entry}

\begin{entry}{让步}{5,7}
  \begin{phonetics}{让步}{rang4bu4}
    \definition{v.+compl.}{fazer uma concessão | entregar | desistir | comprometer}
  \end{phonetics}
\end{entry}

\begin{entry}{记住}{5,7}
  \begin{phonetics}{记住}{ji4-zhu4}
    \definition{v.}{decorar | memorizar | ter em mente}
  \end{phonetics}
\end{entry}

\begin{entry}{记性}{5,8}
  \begin{phonetics}{记性}{ji4xing5}
    \definition{s.}{memória (habilidade em reter informações)}
  \end{phonetics}
\end{entry}

\begin{entry}{记得}{5,11}
  \begin{phonetics}{记得}{ji4de5}
    \definition{v.}{lembrar | lembrar-se}
  \end{phonetics}
\end{entry}

\begin{entry}{边}{5}[Radical 辵]
  \begin{phonetics}{边}{bian1}
    \definition{adv.}{simultaneamente}
    \definition[个]{s.}{fronteira | limite | borda | margem | lado}
  \end{phonetics}
  \begin{phonetics}{边}{bian5}
    \definition{suf.}{sufixo de uma palavra de localidade}
  \end{phonetics}
\end{entry}

\begin{entry}{边关}{5,6}
  \begin{phonetics}{边关}{bian1guan1}
    \definition{s.}{posto de fronteira | posição defensiva estratégica na fronteira}
  \end{phonetics}
\end{entry}

\begin{entry}{边防}{5,6}
  \begin{phonetics}{边防}{bian1fang2}
    \definition{s.}{defesa da fronteira}
  \end{phonetics}
\end{entry}

\begin{entry}{闪存盘}{5,6,11}
  \begin{phonetics}{闪存盘}{shan3cun2pan2}
    \definition{s.}{unidade de memória \emph{USB} | cartão de memória}
  \seealsoref{优盘}{you1pan2}
  \end{phonetics}
\end{entry}

\begin{entry}{鸟儿}{5,2}
  \begin{phonetics}{鸟儿}{niao3r5}
    \definition[只]{s.}{pássaro | ave}
  \end{phonetics}
\end{entry}

\begin{entry}{龙}{5}[Radical 龍]
  \begin{phonetics}{龙}{long2}
    \definition*{s.}{sobrenome Long}
    \definition{adj.}{imperial}
    \definition[条]{s.}{dragão chinês | (fig.) imperador | dragão | (forma ligada) dinossauro}
  \end{phonetics}
\end{entry}

\begin{entry}{龙山}{5,3}
  \begin{phonetics}{龙山}{long2shan1}
    \definition*{s.}{Longshan}
  \end{phonetics}
\end{entry}

\begin{entry}{龙虾}{5,9}
  \begin{phonetics}{龙虾}{long2xia1}
    \definition{s.}{lagosta}
  \end{phonetics}
\end{entry}

%%%%% EOF %%%%%


 %%%
%%% 6画
%%%
\section*{6画}\addcontentsline{toc}{section}{6画}\addcontentsline{loh}{figure}{\#\#\#\# 6画}

%%%%%%%%%% 丢 %%%%%%%%%%
\subsection*{丢}\addcontentsline{loh}{figure}{丢}

\begin{Entry}{丢}{6}{⼛}
  \begin{Phonetics}{丢}{diu1}[][HSK 5]
    \definition{v.}{perder; extraviar; estar ausente | lançar; atirar | colocar (deixar) de lado | deixar (para trás)}
  \end{Phonetics}
\end{Entry}

\begin{Entry}{丢人}{6,2}{⼛,⼈}
  \begin{Phonetics}{丢人}{diu1/ren2}[][HSK 7-9]
    \definition{v.+compl.}{ser desonrado; ser vergonhoso}
  \end{Phonetics}
\end{Entry}

\begin{Entry}{丢下}{6,3}{⼛,⼀}
  \begin{Phonetics}{丢下}{diu1xia4}
    \definition{v.}{abandonar}
  \end{Phonetics}
\end{Entry}

\begin{Entry}{丢开}{6,4}{⼛,⼶}
  \begin{Phonetics}{丢开}{diu1kai1}
    \definition{v.}{jogar fora ou deixar de lado | esquecer por um tempo}
  \end{Phonetics}
\end{Entry}

\begin{Entry}{丢失}{6,5}{⼛,⼤}
  \begin{Phonetics}{丢失}{diu1shi1}[][HSK 7-9]
    \definition{v.}{perder}
  \end{Phonetics}
\end{Entry}

\begin{Entry}{丢弃}{6,7}{⼛,⼶}
  \begin{Phonetics}{丢弃}{diu1qi4}[][HSK 7-9]
    \definition{v.}{abandonar; descartar; livrar-se de; jogar fora}
  \end{Phonetics}
\end{Entry}

\begin{Entry}{丢官}{6,8}{⼛,⼧}
  \begin{Phonetics}{丢官}{diu1guan1}
    \definition{v.}{(um funcionário) perder o emprego; ser demitido}
  \end{Phonetics}
\end{Entry}

\begin{Entry}{丢掉}{6,11}{⼛,⼿}
  \begin{Phonetics}{丢掉}{diu1diao4}[][HSK 7-9]
    \definition{v.}{perder | descartar; abandonar; jogar fora; lançar fora}
  \end{Phonetics}
\end{Entry}

\begin{Entry}{丢脸}{6,11}{⼛,⾁}
  \begin{Phonetics}{丢脸}{diu1/lian3}[][HSK 7-9]
    \definition{v.+compl.}{perder a dignidade; ser desonroso; estar envergonhado}
  \end{Phonetics}
\end{Entry}

%%%%%%%%%% 乒 %%%%%%%%%%
\subsection*{乒}\addcontentsline{loh}{figure}{乒}

\begin{Entry}{乒}{6}{⼃}
  \begin{Phonetics}{乒}{ping1}
    \definition{interj.}{Onomatopéia: estalo; estouro; estrondo | Onomatopéia: ``ping''}
    \definition{s.}{tênis de mesa; pingue-pongue; abreviação de 乒乓球 | bola de tênis de mesa; bola de pingue-pongue; abreviação de 乒乓球}
  \seealsoref{乒乓球}{ping1pang1qiu2}
  \end{Phonetics}
\end{Entry}

\begin{Entry}{乒乓球}{6,6,11}{⼃,⼃,⽟}
  \begin{Phonetics}{乒乓球}{ping1pang1qiu2}[][HSK 7-9]
    \definition[个,只]{s.}{pingue-pongue; tênis de mesa | bola de tênis de mesa; bola de pingue-pongue}
  \end{Phonetics}
\end{Entry}

%%%%%%%%%% 乓 %%%%%%%%%%
\subsection*{乓}\addcontentsline{loh}{figure}{乓}

\begin{Entry}{乓}{6}{⼃}
  \begin{Phonetics}{乓}{pang1}
    \definition{interj.}{Onomatopéia: barulho repentino feito por tiros, uma porta batendo, coisas quebrando, etc.; estrondo; estouro; batida; colisão}
  \end{Phonetics}
\end{Entry}

%%%%%%%%%% 乔 %%%%%%%%%%
\subsection*{乔}\addcontentsline{loh}{figure}{乔}

\begin{Entry}{乔}{6}{⼃}
  \begin{Phonetics}{乔}{qiao2}
    \definition*{s.}{Sobrenome: Qiao}
    \definition{adj.}{alto, imponente; orgulhoso, imponente}
  \end{Phonetics}
\end{Entry}

\begin{Entry}{乔装}{6,12}{⼃,⾐}
  \begin{Phonetics}{乔装}{qiao2zhuang1}[][HSK 7-9]
    \definition{v.}{disfarçar; vestir-se}
  \synonymref{着装}{zhuo2zhuang1}
  \end{Phonetics}
\end{Entry}

%%%%%%%%%% 买 %%%%%%%%%%
\subsection*{买}\addcontentsline{loh}{figure}{买}

\begin{Entry}{买}{6}{⼄}
  \begin{Phonetics}{买}{mai3}[][HSK 1]
    \definition*{s.}{Sobrenome: Mai}
    \definition{v.}{comprar; adquirir | comprar; subornar; usar dinheiro ou outros meios para angariar apoio| pedir; obter; trocar dinheiro por coisas}
  \synonymref{购}{gou4}
  \antonymref{卖}{mai4}
  \end{Phonetics}
\end{Entry}

\begin{Entry}{买不起}{6,4,10}{⼄,⼀,⾛}
  \begin{Phonetics}{买不起}{mai3 bu5 qi3}[][HSK 7-9]
    \definition{v.}{não ter condições de comprar | não ter condições de pagar}
  \end{Phonetics}
\end{Entry}

\begin{Entry}{买东西}{6,5,6}{⼄,⼀,⾑}
  \begin{Phonetics}{买东西}{mai3 dong1xi5}
    \definition{v.}{fazer compras; comprar bens ou serviços}
  \end{Phonetics}
\end{Entry}

\begin{Entry}{买卖}{6,8}{⼄,⼗}
  \begin{Phonetics}{买卖}{mai3mai4}
    \definition[笔,桩,宗,家]{s.}{negócio; compra e venda; transação comercial | loja (privada); armazém}
  \synonymref{交易}{jiao1yi4}
  \synonymref{贸易}{mao4yi4}
  \synonymref{生意}{sheng1yi5}
  \synonymref{营业}{ying2ye4}
  \end{Phonetics}
  \begin{Phonetics}{买卖}{mai3mai5}[][HSK 5]
    \definition[笔,桩,宗,家]{s.}{negócio; compra e venda; transação | Privado: loja; armazém}
  \synonymref{交易}{jiao1yi4}
  \synonymref{贸易}{mao4yi4}
  \synonymref{生意}{sheng1yi5}
  \synonymref{营业}{ying2ye4}
  \end{Phonetics}
\end{Entry}

%%%%%%%%%% 争 %%%%%%%%%%
\subsection*{争}\addcontentsline{loh}{figure}{争}

\begin{Entry}{争}{6}{⼑}
  \begin{Phonetics}{争}{zheng1}[][HSK 3]
    \definition*{s.}{Sobrenome: Zheng}
    \definition{adv.}{como; por que}
    \definition{v.}{competir; disputar; lutar; esforçar-se para obter ou alcançar | discutir; argumentar; contestar; debater | faltar; estar em falta}
  \end{Phonetics}
\end{Entry}

\begin{Entry}{争风吃醋}{6,4,6,15}{⼑,⾵,⼝,⾣}
  \begin{Phonetics}{争风吃醋}{zheng1feng1chi1cu4}
    \definition{v.}{rivalizar com alguém pelo carinho de um homem ou mulher | estar com ciúmes de um rival em um caso de amor}
  \end{Phonetics}
\end{Entry}

\begin{Entry}{争议}{6,5}{⼑,⾔}
  \begin{Phonetics}{争议}{zheng1yi4}[][HSK 5]
    \definition{s.}{disputa; controvérsia; situações e questões em que há divergências de opinião}
    \definition{v.}{debater; discutir}
  \end{Phonetics}
\end{Entry}

\begin{Entry}{争先}{6,6}{⼑,⼉}
  \begin{Phonetics}{争先}{zheng1xian1}
    \definition{v.}{competir para ser o primeiro | contestar o primeiro lugar}
  \end{Phonetics}
\end{Entry}

\begin{Entry}{争夺}{6,6}{⼑,⼤}
  \begin{Phonetics}{争夺}{zheng1duo2}[][HSK 6]
    \definition{v.}{disputar; competir}
  \end{Phonetics}
\end{Entry}

\begin{Entry}{争论}{6,6}{⼑,⾔}
  \begin{Phonetics}{争论}{zheng1lun4}[][HSK 4]
    \definition[番,场,次]{s.}{debate; discussão; argumentação; disputa}
    \definition{v.}{discutir; disputar; debater; argumentar; contestar}
  \end{Phonetics}
\end{Entry}

\begin{Entry}{争取}{6,8}{⼑,⼜}
  \begin{Phonetics}{争取}{zheng1qu3}[][HSK 3]
    \definition{v.}{lutar por; conquistar; vencer; se esforçar para conseguir}
  \end{Phonetics}
\end{Entry}

\begin{Entry}{争霸}{6,21}{⼑,⾬}
  \begin{Phonetics}{争霸}{zheng1ba4}
    \definition{s.}{hegemonia | uma luta pelo poder}
    \definition{v.}{disputar a hegemonia}
  \end{Phonetics}
\end{Entry}

%%%%%%%%%% 亚 %%%%%%%%%%
\subsection*{亚}\addcontentsline{loh}{figure}{亚}

\begin{Entry}{亚}{6}{⼆}
  \begin{Phonetics}{亚}{ya4}
    \definition*{s.}{Ásia, abreviação de 亚洲 | Sobrenome: Ya}
    \definition{adj.}{inferior | abaixo do padrão | (química) de menor valência atômica}
    \definition{pref.}{sub-}
  \seealsoref{亚洲}{ya4zhou1}
  \end{Phonetics}
\end{Entry}

\begin{Entry}{亚军}{6,6}{⼆,⼍}
  \begin{Phonetics}{亚军}{ya4jun1}[][HSK 5]
    \definition[个]{s.}{segundo lugar; vice-campeão; medalhista de prata}
  \end{Phonetics}
\end{Entry}

\begin{Entry}{亚运会}{6,7,6}{⼆,⾡,⼈}
  \begin{Phonetics}{亚运会}{ya4yun4hui4}[][HSK 4]
    \definition*{s.}{Jogos Asiáticos}
  \end{Phonetics}
\end{Entry}

\begin{Entry}{亚细亚洲}{6,8,6,9}{⼆,⽷,⼆,⽔}
  \begin{Phonetics}{亚细亚洲}{ya4xi4ya4zhou1}
    \definition*{s.}{Ásia}
  \end{Phonetics}
\end{Entry}

\begin{Entry}{亚洲}{6,9}{⼆,⽔}
  \begin{Phonetics}{亚洲}{ya4zhou1}
    \definition*{s.}{Ásia, abreviação de 亚细亚洲}
  \seealsoref{亚细亚洲}{ya4xi4ya4zhou1}
  \end{Phonetics}
\end{Entry}

\begin{Entry}{亚洲人}{6,9,2}{⼆,⽔,⼈}
  \begin{Phonetics}{亚洲人}{ya4zhou1ren2}
    \definition{s.}{asiático | pessoa ou povo da Ásia}
  \end{Phonetics}
\end{Entry}

\begin{Entry}{亚热带}{6,10,9}{⼆,⽕,⼱}
  \begin{Phonetics}{亚热带}{ya4re4dai4}
    \definition{s.}{zona ou clima subtropical; subtropical; semitropical}
  \end{Phonetics}
\end{Entry}

%%%%%%%%%% 交 %%%%%%%%%%
\subsection*{交}\addcontentsline{loh}{figure}{交}

\begin{Entry}{交}{6}{⼇}
  \begin{Phonetics}{交}{jiao1}[][HSK 2]
    \definition*{s.}{Sobrenome: Jiao}
    \definition{adv.}{mutuamente; recíprocamente; um ao outro | juntos; simultaneamente}
    \definition{s.}{amigo; conhecido; amizade; relacionamento | transação comercial; negócio; barganha | queda}
    \definition{v.}{entregar | (de lugares ou períodos de tempo) cruzar; encontrar; unir | chegar (a uma determinada hora ou estação); estabelecer-se; vir | cruzar; intersectar | associar-se a | ter relações sexuais | acasalar; reproduzir-se | transferir as coisas para as partes interessadas | unir (lugares ou períodos de tempo)}
  \antonymref{接}{jie1}
  \antonymref{收}{shou1}
  \end{Phonetics}
\end{Entry}

\begin{Entry}{交叉}{6,3}{⼇,⼜}
  \begin{Phonetics}{交叉}{jiao1cha1}[][HSK 7-9]
    \definition{v.}{cruzar; entrecruzar; interseccionar | coincidir | revezar-se}
  \end{Phonetics}
\end{Entry}

\begin{Entry}{交叉口}{6,3,3}{⼇,⼜,⼝}
  \begin{Phonetics}{交叉口}{jiao1cha1kou3}
    \definition{s.}{intersecção (rodovia)}
  \end{Phonetics}
\end{Entry}

\begin{Entry}{交叉点}{6,3,9}{⼇,⼜,⽕}
  \begin{Phonetics}{交叉点}{jiao1cha1dian3}
    \definition{s.}{interseção; cruzamento; encruzilhada; ponto de interseção; junção}
  \end{Phonetics}
\end{Entry}

\begin{Entry}{交付}{6,5}{⼇,⼈}
  \begin{Phonetics}{交付}{jiao1fu4}[][HSK 7-9]
    \definition{v.}{entregar; passar a responsabilidade; transferir a responsabilidade; pagar a}
  \synonymref{托付}{tuo1fu4}
  \end{Phonetics}
\end{Entry}

\begin{Entry}{交代}{6,5}{⼇,⼈}
  \begin{Phonetics}{交代}{jiao1dai4}[][HSK 5]
    \definition{v.}{contar; entregar | ordenar; insistir; contar aos outros sobre suas intenções, instruções | contar; admitir}
  \synonymref{打发}{da3fa5}
  \synonymref{叮嘱}{ding1zhu3}
  \synonymref{吩咐}{fen1fu4}
  \synonymref{交接}{jiao1jie1}
  \synonymref{派遣}{pai4qian3}
  \synonymref{嘱托}{zhu3tuo1}
  \synonymref{嘱咐}{zhu3fu5}
  \antonymref{抗拒}{kang4ju4}
  \end{Phonetics}
\end{Entry}

\begin{Entry}{交头接耳}{6,5,11,6}{⼇,⼤,⼿,⽿}
  \begin{Phonetics}{交头接耳}{jiao1tou2-jie1'er3}[][HSK 7-9]
    \definition{expr.}{falar ao ouvido um do outro; sussurrar um para o outro; trocar sussurros confidenciais; sussurrar um ao ouvido do outro; cochichar um com o outro}
  \end{Phonetics}
\end{Entry}

\begin{Entry}{交纳}{6,7}{⼇,⽷}
  \begin{Phonetics}{交纳}{jiao1na4}[][HSK 7-9]
    \definition{v.}{pagar (ao estado ou a uma organização); entregar uma quantia predeterminada de dinheiro ou bens a um governo ou órgão público}
  \synonymref{缴纳}{jiao3na4}
  \end{Phonetics}
\end{Entry}

\begin{Entry}{交运}{6,7}{⼇,⾡}
  \begin{Phonetics}{交运}{jiao1yun4}
    \definition{v.}{despachar (bagagem em um aeroporto, etc.) | entregar para transporte}
  \end{Phonetics}
\end{Entry}

\begin{Entry}{交际}{6,7}{⼇,⾩}
  \begin{Phonetics}{交际}{jiao1ji4}[][HSK 4]
    \definition{s.}{contato; comunicação; relações sociais; contato interpessoal, socialização}
  \synonymref{酬酢}{chou2zuo4}
  \synonymref{社交}{she4jiao1}
  \synonymref{外交}{wai4jiao1}
  \end{Phonetics}
\end{Entry}

\begin{Entry}{交往}{6,8}{⼇,⼻}
  \begin{Phonetics}{交往}{jiao1wang3}[][HSK 3]
    \definition{v.}{estar em contato com; associar-se a; interagir}
  \synonymref{交易}{jiao1yi4}
  \synonymref{来往}{lai2wang3}
  \synonymref{往来}{wang3lai2}
  \synonymref{相处}{xiang1chu3}
  \antonymref{断交}{duan4/jiao1}
  \end{Phonetics}
\end{Entry}

\begin{Entry}{交易}{6,8}{⼇,⽇}
  \begin{Phonetics}{交易}{jiao1yi4}[][HSK 3]
    \definition[笔,桩,个,场]{s.}{negócio; comércio; transação comercial; transação; atividades de compra e venda de mercadorias}
    \definition{v.}{negociar; comprar e vender mercadorias}
  \synonymref{成交}{cheng2/jiao1}
  \synonymref{交往}{jiao1wang3}
  \synonymref{来往}{lai2wang3}
  \synonymref{买卖}{mai3mai4}
  \synonymref{买卖}{mai3mai5}
  \synonymref{贸易}{mao4yi4}
  \synonymref{生意}{sheng1yi5}
  \synonymref{业务}{ye4wu4}
  \synonymref{营业}{ying2ye4}
  \antonymref{赠送}{zeng4song4}
  \end{Phonetics}
\end{Entry}

\begin{Entry}{交朋友}{6,8,4}{⼇,⽉,⼜}
  \begin{Phonetics}{交朋友}{jiao1 peng2you3}[][HSK 2]
    \definition{v.}{fazer amizade com alguém; fazer amigos}
  \synonymref{打交道}{da3 jiao1dao5}
  \end{Phonetics}
\end{Entry}

\begin{Entry}{交杯酒}{6,8,10}{⼇,⽊,⾣}
  \begin{Phonetics}{交杯酒}{jiao1bei1jiu3}
    \definition{s.}{copo de vinho nupcial}
  \end{Phonetics}
\end{Entry}

\begin{Entry}{交响}{6,9}{⼇,⼝}
  \begin{Phonetics}{交响}{jiao1xiang3}
    \definition{s.}{sinfonia}
  \synonymref{共鸣}{gong4ming2}
  \end{Phonetics}
\end{Entry}

\begin{Entry}{交响乐}{6,9,5}{⼇,⼝,⼃}
  \begin{Phonetics}{交响乐}{jiao1xiang3yue4}[][HSK 7-9]
    \definition{s.}{sinfonia; música sinfônica; as peças musicais de grande escala executadas por uma orquestra normalmente consistem em quatro movimentos e são capazes de expressar pensamentos e sentimentos diversos e complexos}
  \end{Phonetics}
\end{Entry}

\begin{Entry}{交界}{6,9}{⼇,⽥}
  \begin{Phonetics}{交界}{jiao1jie4}[][HSK 7-9]
    \definition{v.}{ter uma fronteira comum; ter um limite comum; fazer fronteira com}
  \end{Phonetics}
\end{Entry}

\begin{Entry}{交给}{6,9}{⼇,⽷}
  \begin{Phonetics}{交给}{jiao1gei3}[][HSK 2]
    \definition{v.}{entregar para | dar para}
  \end{Phonetics}
\end{Entry}

\begin{Entry}{交费}{6,9}{⼇,⾙}
  \begin{Phonetics}{交费}{jiao1fei4}[][HSK 3]
    \definition{v.}{pagar taxas ou impostos; pagar uma taxa ou imposto}
  \end{Phonetics}
\end{Entry}

\begin{Entry}{交换}{6,10}{⼇,⼿}
  \begin{Phonetics}{交换}{jiao1huan4}[][HSK 4]
    \definition{v.}{trocar; permutar; comutar; intercambiar}
  \synonymref{撤换}{che4huan4}
  \synonymref{兑换}{dui4huan4}
  \synonymref{交流}{jiao1liu2}
  \synonymref{替换}{ti4huan4}
  \antonymref{换取}{huan4qu3}
  \end{Phonetics}
\end{Entry}

\begin{Entry}{交流}{6,10}{⼇,⽔}
  \begin{Phonetics}{交流}{jiao1liu2}[][HSK 3]
    \definition{v.}{trocar; interagir; comunicar-se; compartilhar o que cada um tem com o outro}
  \synonymref{对话}{dui4hua4}
  \synonymref{沟通}{gou1tong1}
  \synonymref{互动}{hu4dong4}
  \synonymref{换取}{huan4qu3}
  \synonymref{交换}{jiao1huan4}
  \antonymref{封闭}{feng1bi4}
  \end{Phonetics}
\end{Entry}

\begin{Entry}{交涉}{6,10}{⼇,⽔}
  \begin{Phonetics}{交涉}{jiao1she4}[][HSK 7-9]
    \definition{v.}{negociar; entrar em contato com; fazer representações; discutir soluções para questões relacionadas com a outra parte}
  \synonymref{对话}{dui4hua4}
  \synonymref{谈判}{tan2pan4}
  \synonymref{协商}{xie2shang1}
  \end{Phonetics}
\end{Entry}

\begin{Entry}{交班}{6,10}{⼇,⽟}
  \begin{Phonetics}{交班}{jiao1ban1}
    \definition{v.}{passar para o próximo turno de trabalho}
  \synonymref{接班}{jie1/ban1}
  \end{Phonetics}
\end{Entry}

\begin{Entry}{交谈}{6,10}{⼇,⾔}
  \begin{Phonetics}{交谈}{jiao1tan2}[][HSK 7-9]
    \definition{v.}{conversar; bater papo; falar um com o outro}
  \end{Phonetics}
\end{Entry}

\begin{Entry}{交通}{6,10}{⼇,⾡}
  \begin{Phonetics}{交通}{jiao1tong1}[][HSK 2]
    \definition{s.}{tráfego | ligação; conexão | transporte; termo genérico para todos os tipos de transporte, como ferroviário e rodoviário}
    \definition{v.}{conspirar; fazer amizades; conchavar | estar conectado; estar ligado; estar vinculado | associar-se a; conspirar com}
  \synonymref{通行}{tong1xing2}
  \synonymref{运输}{yun4shu1}
  \antonymref{堵塞}{du3se4}
  \end{Phonetics}
\end{Entry}

\begin{Entry}{交通警察}{6,10,19,14}{⼇,⾡,⾔,⼧}
  \begin{Phonetics}{交通警察}{jiao1tong1 jing3cha2}
    \definition{s.}{policial de trânsito}
  \seealsoref{交警}{jiao1jing3}
  \end{Phonetics}
\end{Entry}

\begin{Entry}{交情}{6,11}{⼇,⼼}
  \begin{Phonetics}{交情}{jiao1qing5}[][HSK 7-9]
    \definition{s.}{amizade; relação amigável}
  \synonymref{情谊}{qing2yi4}
  \synonymref{友谊}{you3yi4}
  \end{Phonetics}
\end{Entry}

\begin{Entry}{交接}{6,11}{⼇,⼿}
  \begin{Phonetics}{交接}{jiao1jie1}[][HSK 7-9]
    \definition{v.}{juntar-se; conectar-se; conectar | entregar e assumir o controle; transferir | associar-se a; fazer amizade com; fazer amigos}
  \synonymref{交代}{jiao1dai4}
  \end{Phonetics}
\end{Entry}

\begin{Entry}{交替}{6,12}{⼇,⽈}
  \begin{Phonetics}{交替}{jiao1ti4}[][HSK 7-9]
    \definition{v.}{dar lugar a; substituir; suplantar; substituir coisas antigas por coisas novas | alternar; revezar-se}
  \synonymref{轮流}{lun2liu2}
  \antonymref{维持}{wei2chi2}
  \end{Phonetics}
\end{Entry}

\begin{Entry}{交锋}{6,12}{⼇,⾦}
  \begin{Phonetics}{交锋}{jiao1/feng1}[][HSK 7-9]
    \definition{v.+compl.}{entrar em conflito; cruzar espadas; confrontar; participar de uma batalha ou disputa; ter um confronto (com alguém)}
  \synonymref{打仗}{da3/zhang4}
  \synonymref{接触}{jie1chu4}
  \synonymref{战争}{zhan4zheng1}
  \antonymref{逃避}{tao2bi4}
  \end{Phonetics}
\end{Entry}

\begin{Entry}{交集}{6,12}{⼇,⾫}
  \begin{Phonetics}{交集}{jiao1ji2}[][HSK 7-9]
    \definition{s.}{sobreposição; conexão; terreno comum; pontos em comum; intersecção; convergência}
    \definition{v.}{(diferentes sentimentos) estar misturado; ocorrer simultaneamente}
  \synonymref{暴躁}{bao4zao4}
  \synonymref{烦躁}{fan2zao4}
  \synonymref{慌张}{huang1zhang1}
  \synonymref{焦虑}{jiao1lv4}
  \synonymref{焦躁}{jiao1zao4}
  \synonymref{恐慌}{kong3huang1}
  \synonymref{着急}{zhao2/ji2}
  \antonymref{分散}{fen1san4}
  \end{Phonetics}
\end{Entry}

\begin{Entry}{交叠}{6,13}{⼇,⼜}
  \begin{Phonetics}{交叠}{jiao1die2}
    \definition{s.}{sobreposição}
  \synonymref{重叠}{chong2die2}
  \synonymref{交集}{jiao1ji2}
  \end{Phonetics}
\end{Entry}

\begin{Entry}{交媾}{6,13}{⼇,⼥}
  \begin{Phonetics}{交媾}{jiao1gou4}
    \definition{v.}{copular | ter relações sexuais}
  \end{Phonetics}
\end{Entry}

\begin{Entry}{交警}{6,19}{⼇,⾔}
  \begin{Phonetics}{交警}{jiao1jing3}[][HSK 3]
    \definition{s.}{policial de trânsito, abreviação de 交通警察}
  \seealsoref{交通警察}{jiao1tong1 jing3cha2}
  \end{Phonetics}
\end{Entry}

%%%%%%%%%% 亦 %%%%%%%%%%
\subsection*{亦}\addcontentsline{loh}{figure}{亦}

\begin{Entry}{亦}{6}{⼇}
  \begin{Phonetics}{亦}{yi4}
    \definition*{s.}{Sobrenome: Yi}
    \definition{adv.}{também; também (que significa o mesmo)}
  \end{Phonetics}
\end{Entry}

%%%%%%%%%% 产 %%%%%%%%%%
\subsection*{产}\addcontentsline{loh}{figure}{产}

\begin{Entry}{产}{6}{⼇}
  \begin{Phonetics}{产}{chan3}[][HSK 7-9]
    \definition*{s.}{Sobrenome: Chan}
    \definition{s.}{produto | propriedade; espólio | (abreviação) indústria}
    \definition{v.}{dar à luz; ser entregue a | produzir; render | separar um ser humano ou animal de sua mãe}
  \end{Phonetics}
\end{Entry}

\begin{Entry}{产业}{6,5}{⼇,⼀}
  \begin{Phonetics}{产业}{chan3ye4}[][HSK 5]
    \definition{s.}{patrimônio; propriedade; bens pessoais, como terrenos, casas, fábricas, etc. | indústria; refere"-se especificamente à produção industrial moderna | setor; indústria; indústrias e setores da economia nacional}
  \synonymref{工业}{gong1ye4}
  \synonymref{行业}{hang2ye4}
  \end{Phonetics}
\end{Entry}

\begin{Entry}{产生}{6,5}{⼇,⽣}
  \begin{Phonetics}{产生}{chan3sheng1}[][HSK 3]
    \definition{v.}{produzir; evoluir; emergir; provocar; vir a ser; dar origem a; criar coisas novas e novos fenômenos a partir do que já existe}
  \synonymref{出现}{chu1xian4}
  \synonymref{发生}{fa1sheng1}
  \synonymref{发作}{fa1zuo4}
  \synonymref{形成}{xing2cheng2}
  \antonymref{消灭}{xiao1mie4}
  \end{Phonetics}
\end{Entry}

\begin{Entry}{产后}{6,6}{⼇,⼝}
  \begin{Phonetics}{产后}{chan3hou4}
    \definition{s.}{pós-parto}
  \end{Phonetics}
\end{Entry}

\begin{Entry}{产地}{6,6}{⼇,⼟}
  \begin{Phonetics}{产地}{chan3di4}[][HSK 7-9]
    \definition{s.}{local de produção (ou origem); área de produção; o local onde o item é produzido}
  \end{Phonetics}
\end{Entry}

\begin{Entry}{产物}{6,8}{⼇,⽜}
  \begin{Phonetics}{产物}{chan3wu4}[][HSK 7-9]
    \definition{s.}{resultado; produto; coisas que ocorrem sob certas condições}
  \end{Phonetics}
\end{Entry}

\begin{Entry}{产品}{6,9}{⼇,⼝}
  \begin{Phonetics}{产品}{chan3pin3}[][HSK 4]
    \definition[个,件,种,批,项,类]{s.}{produto; item produzido}
  \synonymref{产物}{chan3wu4}
  \synonymref{货物}{huo4wu4}
  \end{Phonetics}
\end{Entry}

\begin{Entry}{产值}{6,10}{⼇,⼈}
  \begin{Phonetics}{产值}{chan3zhi2}[][HSK 7-9]
    \definition{s.}{valor de saída; o valor monetário de todos os produtos ou de um produto específico em um período de tempo}
  \end{Phonetics}
\end{Entry}

\begin{Entry}{产量}{6,12}{⼇,⾥}
  \begin{Phonetics}{产量}{chan3liang4}[][HSK 6]
    \definition{v.}{rendimento; produção; a quantidade de produção; a quantidade total de produtos produzidos em um determinado período de tempo}
  \end{Phonetics}
\end{Entry}

%%%%%%%%%% 仰 %%%%%%%%%%
\subsection*{仰}\addcontentsline{loh}{figure}{仰}

\begin{Entry}{仰}{6}{⼈}
  \begin{Phonetics}{仰}{yang3}[][HSK 6]
    \definition*{s.}{Sobrenome: Yang}
    \definition{v.}{levantar | virar para cima | admirar; respeitar | confiar em; depender de}
  \antonymref{俯}{fu3}
  \end{Phonetics}
\end{Entry}

%%%%%%%%%% 件 %%%%%%%%%%
\subsection*{件}\addcontentsline{loh}{figure}{件}

\begin{Entry}{件}{6}{⼈}
  \begin{Phonetics}{件}{jian4}[][HSK 2]
    \definition*{s.}{Sobrenome: Jian}
    \definition{clas.}{item; peça; artigo; usado para coisas individuais}
    \definition{s.}{refere"-se a coisas que podem ser contadas uma a uma | papel; carta; documento; correspondência}
  \end{Phonetics}
\end{Entry}

%%%%%%%%%% 价 %%%%%%%%%%
\subsection*{价}\addcontentsline{loh}{figure}{价}

\begin{Entry}{价}{6}{⼈}
  \begin{Phonetics}{价}{jia4}[][HSK 5]
    \definition{s.}{preço | valor; (figurativo) valores (éticos, culturais etc.) | Química: valência}
  \end{Phonetics}
\end{Entry}

\begin{Entry}{价位}{6,7}{⼈,⼈}
  \begin{Phonetics}{价位}{jia4wei4}[][HSK 7-9]
    \definition{s.}{preço; nível de preço}
  \end{Phonetics}
\end{Entry}

\begin{Entry}{价值}{6,10}{⼈,⼈}
  \begin{Phonetics}{价值}{jia4zhi2}[][HSK 3]
    \definition{s.}{valor; o trabalho social necessário condensado nos produtos | valor; importância; efeitos positivos}
  \synonymref{代价}{dai4jia4}
  \synonymref{价钱}{jia4 qian2}
  \synonymref{价格}{jia4ge2}
  \end{Phonetics}
\end{Entry}

\begin{Entry}{价值观}{6,10,6}{⼈,⼈,⾒}
  \begin{Phonetics}{价值观}{jia4zhi2guan1}[][HSK 7-9]
    \definition{s.}{valores; a visão geral sobre economia, política, moralidade, dinheiro, etc; as pessoas têm valores diferentes devido aos seus diferentes níveis sociais}
  \end{Phonetics}
\end{Entry}

\begin{Entry}{价格}{6,10}{⼈,⽊}
  \begin{Phonetics}{价格}{jia4ge2}[][HSK 3]
    \definition[个,种]{s.}{preço; tarifa; o valor monetário da mercadoria}
  \synonymref{代价}{dai4jia4}
  \synonymref{价值}{jia4zhi2}
  \antonymref{价钱}{jia4 qian2}
  \end{Phonetics}
\end{Entry}

\begin{Entry}{价钱}{6,10}{⼈,⾦}
  \begin{Phonetics}{价钱}{jia4 qian2}[][HSK 3]
    \definition[个,种,笔]{s.}{preço}
  \synonymref{代价}{dai4jia4}
  \synonymref{价格}{jia4ge2}
  \synonymref{价值}{jia4zhi2}
  \end{Phonetics}
\end{Entry}

%%%%%%%%%% 任 %%%%%%%%%%
\subsection*{任}\addcontentsline{loh}{figure}{任}

\begin{Entry}{任}{6}{⼈}
  \begin{Phonetics}{任}{ren2}
    \definition*{s.}{Condado de Ren em Hebei (河北) | usado em nomes de lugares, por exemplo, Renxian (任县) e Renqi (任丘) ficam na província de Hebei (河北) | Sobrenome: Ren}
  \end{Phonetics}
  \begin{Phonetics}{任}{ren4}[][HSK 3]
    \definition{clas.}{usado para o número de mandatos cumpridos em um cargo oficial}
    \definition{conj.}{não importa (como, o que, etc.); orações de conexão, ou usadas antes de pronomes interrogativos, para expressar incondicionalidade, equivalente a 不管 ou 无论}
    \definition{s.}{escritório; posto oficial; cargo | dever; fardo; responsabilidade}
    \definition{v.}{nomear; designar alguém para um cargo | assumir um emprego; assumir um posto; assumir uma posição | deixar; permitir; dar rédea solta a | suportar; empreender | ceder; permitir sem restrições; deixar (alguém) fazer o que quiser}
  \seealsoref{不管}{bu4guan3}
  \seealsoref{无论}{wu2lun4}
  \antonymref{免}{mian3}
  \end{Phonetics}
\end{Entry}

\begin{Entry}{任人宰割}{6,2,10,12}{⼈,⼈,⼧,⼑}
  \begin{Phonetics}{任人宰割}{ren4ren2-zai3ge1}[][HSK 7-9]
    \definition{expr.}{(não pode deixar de) permitir que lhe pisoteiem; ser explorado; ser pisoteado}
  \end{Phonetics}
\end{Entry}

\begin{Entry}{任务}{6,5}{⼈,⼒}
  \begin{Phonetics}{任务}{ren4wu5}[][HSK 3]
    \definition[项,个,种,些]{s.}{tarefa; dever; missão; designação; trabalho designado; responsabilidades designadas}
  \synonymref{工作}{gong1zuo4}
  \synonymref{使命}{shi3ming4}
  \synonymref{团队}{tuan2dui4}
  \synonymref{义务}{yi4wu4}
  \synonymref{职业}{zhi2ye4}
  \synonymref{职责}{zhi2ze2}
  \end{Phonetics}
\end{Entry}

\begin{Entry}{任何}{6,7}{⼈,⼈}
  \begin{Phonetics}{任何}{ren4he2}[][HSK 3]
    \definition{pron.}{qualquer; qualquer que seja; o que for; não importa o que}
  \synonymref{所以}{suo3yi3}
  \synonymref{无论}{wu2lun4}
  \synonymref{一切}{yi2qie4}
  \antonymref{唯一}{wei2yi1}
  \end{Phonetics}
\end{Entry}

\begin{Entry}{任凭}{6,8}{⼈,⼏}
  \begin{Phonetics}{任凭}{ren4 ping2}
    \definition{conj.}{não importa (como, o quê, etc.) | mesmo que; embora}
    \definition{v.}{permitir; deixar (algo como: fazer o que lhe agrada); conforme a conveniência de alguém}
  \synonymref{听凭}{ting1ping2}
  \antonymref{束缚}{shu4fu4}
  \end{Phonetics}
\end{Entry}

\begin{Entry}{任命}{6,8}{⼈,⼝}
  \begin{Phonetics}{任命}{ren4ming4}[][HSK 7-9]
    \definition{v.}{nomear; designar; incumbir; comissionar}
  \synonymref{任职}{ren4/zhi2}
  \antonymref{免职}{mian3/zhi2}
  \antonymref{免除}{mian3chu2}
  \end{Phonetics}
\end{Entry}

\begin{Entry}{任职}{6,11}{⼈,⽿}
  \begin{Phonetics}{任职}{ren4/zhi2}[][HSK 7-9]
    \definition{v.+compl.}{ocupar um cargo; estar em um cargo de chefia}
  \synonymref{服务}{fu2wu4}
  \synonymref{任命}{ren4ming4}
  \antonymref{离职}{li2/zhi2}
  \antonymref{免职}{mian3/zhi2}
  \end{Phonetics}
\end{Entry}

\begin{Entry}{任期}{6,12}{⼈,⽉}
  \begin{Phonetics}{任期}{ren4qi1}[][HSK 7-9]
    \definition[个,届]{s.}{mandato; duração do mandato; mandato legal}
  \end{Phonetics}
\end{Entry}

\begin{Entry}{任意}{6,13}{⼈,⼼}
  \begin{Phonetics}{任意}{ren4yi4}[][HSK 7-9]
    \definition{adj.}{sem reservas; sem quaisquer condições}
    \definition{adv.}{arbitrariamente; sem restrições, sem limitações, faça o que quiser}
  \synonymref{大肆}{da4si4}
  \synonymref{放肆}{fang4si4}
  \synonymref{随便}{sui2/bian4}
  \synonymref{随意}{sui2/yi4}
  \antonymref{拘束}{ju1shu4}
  \end{Phonetics}
\end{Entry}

%%%%%%%%%% 份 %%%%%%%%%%
\subsection*{份}\addcontentsline{loh}{figure}{份}

\begin{Entry}{份}{6}{⼈}
  \begin{Phonetics}{份}{fen4}[][HSK 2]
    \definition{clas.}{usado para emparelhar itens em grupos | usado para jornais, documentos, etc. | usado para partes de um todo | usado para aparência, estado, etc.}
    \definition{s.}{porção; parte | a unidade de divisão; usado após 省, 县, 年, 月,  indica a unidade de divisão | grau; extensão de algo}
  \seealsoref{年}{nian2}
  \seealsoref{省}{sheng3}
  \seealsoref{县}{xian4}
  \seealsoref{月}{yue4}
  \end{Phonetics}
\end{Entry}

\begin{Entry}{份量}{6,12}{⼈,⾥}
  \begin{Phonetics}{份量}{fen4liang5}
    \variantof{分量}
  \end{Phonetics}
\end{Entry}

\begin{Entry}{份额}{6,15}{⼈,⾴}
  \begin{Phonetics}{份额}{fen4'e2}[][HSK 7-9]
    \definition{s.}{quota; quinhão; porção; a proporção ou percentagem do todo}
  \end{Phonetics}
\end{Entry}

%%%%%%%%%% 仿 %%%%%%%%%%
\subsection*{仿}\addcontentsline{loh}{figure}{仿}

\begin{Entry}{仿}{6}{⼈}
  \begin{Phonetics}{仿}{fang3}[][HSK 7-9]
    \definition{adv.}{semelhante; como}
    \definition{s.}{caracteres escritos segundo um modelo de caligrafia | cartas modeladas a partir de uma cópia; palavras escritas de acordo com o modelo}
    \definition{v.}{imitar; copiar | assemelhar-se; ser como}
  \end{Phonetics}
\end{Entry}

\begin{Entry}{仿佛}{6,7}{⼈,⼈}
  \begin{Phonetics}{仿佛}{fang3fu2}[][HSK 6]
    \definition{adv.}{parece que; como se}
    \definition{v.}{ser como; parecer}
  \synonymref{好象}{hao3xiang4}
  \synonymref{好像}{hao3xiang4}
  \synonymref{似乎}{si4hu1}
  \end{Phonetics}
\end{Entry}

\begin{Entry}{仿制}{6,8}{⼈,⼑}
  \begin{Phonetics}{仿制}{fang3zhi4}[][HSK 7-9]
    \definition{v.}{copiar; imitar; ser modelado em}
  \synonymref{模仿}{mo2fang3}
  \synonymref{照样}{zhao4yang4}
  \antonymref{创造}{chuang4zao4}
  \end{Phonetics}
\end{Entry}

%%%%%%%%%% 企 %%%%%%%%%%
\subsection*{企}\addcontentsline{loh}{figure}{企}

\begin{Entry}{企}{6}{⼈}
  \begin{Phonetics}{企}{qi3}
    \definition{v.}{ficar na ponta dos pés | esperar ansiosamente por algo; ansiar por | planejar um projeto}
  \end{Phonetics}
\end{Entry}

\begin{Entry}{企业}{6,5}{⼈,⼀}
  \begin{Phonetics}{企业}{qi3ye4}[][HSK 4]
    \definition[家,个]{s.}{empresa; estabelecimento; empreendimento; negócio; setores envolvidos em atividades econômicas como produção, transporte, comércio, etc., como fábricas, minas, ferrovias, empresas comerciais, etc.}
  \synonymref{产业}{chan3ye4}
  \synonymref{公司}{gong1si1}
  \end{Phonetics}
\end{Entry}

\begin{Entry}{企图}{6,8}{⼈,⼞}
  \begin{Phonetics}{企图}{qi3tu2}[][HSK 6]
    \definition[种]{s.}{plano; tentativa; intenção (principalmente negativa)}
    \definition{v.}{procurar; tentar; pretender}
  \synonymref{打算}{da3suan5}
  \synonymref{计划}{ji4hua4}
  \synonymref{计算}{ji4suan4}
  \synonymref{渴望}{ke3wang4}
  \synonymref{目的}{mu4di4}
  \synonymref{试图}{shi4tu2}
  \synonymref{妄想}{wang4xiang3}
  \synonymref{准备}{zhun3bei4}
  \end{Phonetics}
\end{Entry}

%%%%%%%%%% 伊 %%%%%%%%%%
\subsection*{伊}\addcontentsline{loh}{figure}{伊}

\begin{Entry}{伊}{6}{⼈}
  \begin{Phonetics}{伊}{yi1}
    \definition*{s.}{Iraque, abreviação de 伊拉克 | Irã,abreviação de  伊朗 | Sobrenome: Yi}
    \definition{part.}{(chinês clássico) partícula introdutória sem significado específico}
    \definition{pron.}{Literário: pronome de terceira pessoa do singular (ele ou ela) | pronome de segunda pessoa do singular (você) | disso (precedendo um substantivo)}
  \seealsoref{伊拉克}{yi1la1ke4}
  \seealsoref{伊朗}{yi1lang3}
  \end{Phonetics}
\end{Entry}

\begin{Entry}{伊马姆}{6,3,8}{⼈,⾺,⼥}
  \begin{Phonetics}{伊马姆}{yi1ma3mu3}
    \definition*{s.}{Iman}
  \seealsoref{伊玛目}{yi1ma3mu4}
  \seealsoref{伊曼}{yi1man4}
  \seealsoref{伊斯兰}{yi1si1lan2}
  \end{Phonetics}
\end{Entry}

\begin{Entry}{伊玛目}{6,7,5}{⼈,⽟,⽬}
  \begin{Phonetics}{伊玛目}{yi1ma3mu4}
    \definition*{s.}{Empréstimo linguístico: Iman (Islã)}
  \seealsoref{伊马姆}{yi1ma3mu3}
  \seealsoref{伊曼}{yi1man4}
  \seealsoref{伊斯兰}{yi1si1lan2}
  \end{Phonetics}
\end{Entry}

\begin{Entry}{伊拉克}{6,8,7}{⼈,⼿,⼗}
  \begin{Phonetics}{伊拉克}{yi1la1ke4}
    \definition*{s.}{Iraque}
  \end{Phonetics}
\end{Entry}

\begin{Entry}{伊朗}{6,10}{⼈,⽉}
  \begin{Phonetics}{伊朗}{yi1lang3}
    \definition*{s.}{Irã}
  \end{Phonetics}
\end{Entry}

\begin{Entry}{伊曼}{6,11}{⼈,⽈}
  \begin{Phonetics}{伊曼}{yi1man4}
    \definition*{s.}{Iman (nome próprio)}
  \seealsoref{伊马姆}{yi1ma3mu3}
  \seealsoref{伊玛目}{yi1ma3mu4}
  \seealsoref{伊斯兰}{yi1si1lan2}
  \end{Phonetics}
\end{Entry}

\begin{Entry}{伊斯兰}{6,12,5}{⼈,⽄,⼋}
  \begin{Phonetics}{伊斯兰}{yi1si1lan2}
    \definition*{s.}{Islã}
  \seealsoref{伊马姆}{yi1ma3mu3}
  \seealsoref{伊玛目}{yi1ma3mu4}
  \seealsoref{伊曼}{yi1man4}
  \end{Phonetics}
\end{Entry}

%%%%%%%%%% 休 %%%%%%%%%%
\subsection*{休}\addcontentsline{loh}{figure}{休}

\begin{Entry}{休}{6}{⼈}
  \begin{Phonetics}{休}{xiu1}
    \definition{adj.}{feliz; alegre; festivo}
    \definition{adv.}{não; indica proibição ou dissuasão, equivalente a 别 ou 不要}
    \definition{s.}{fortuna e infortúnio; bom e mau}
    \definition{v.}{parar; cessar | descansar | abandonar a esposa e mandá-la para casa; antigamente, o marido mandava a esposa de volta para a casa dos pais e rompia o relacionamento conjugal}
  \seealsoref{别}{bie2}
  \seealsoref{不要}{bu2yao4}
  \synonymref{歇}{xie1}
  \end{Phonetics}
\end{Entry}

\begin{Entry}{休兵}{6,7}{⼈,⼋}
  \begin{Phonetics}{休兵}{xiu1bing1}
    \definition{s.}{armistício; cessar fogo}
    \definition{v.}{cessar fogo}
  \end{Phonetics}
\end{Entry}

\begin{Entry}{休闲}{6,7}{⼈,⾨}
  \begin{Phonetics}{休闲}{xiu1xian2}[][HSK 5]
    \definition{s.}{ócio; lazer; tempo livre}
    \definition{v.}{desfrutar do lazer; sair de férias; aproveitar o tempo livre; parar de trabalhar ou estudar, estar em um estado de lazer e descontração | ficar ocioso}
  \end{Phonetics}
\end{Entry}

\begin{Entry}{休息}{6,10}{⼈,⼼}
  \begin{Phonetics}{休息}{xiu1xi5}[][HSK 1]
    \definition{s.}{descanço}
    \definition{v.}{descansar; descansar um pouco; fazer uma pausa; interromper o trabalho, os estudos ou as atividades para recuperar as energias | dormir}
  \synonymref{休憩}{xiu1qi4}
  \antonymref{工作}{gong1zuo4}
  \antonymref{劳动}{lao2dong5}
  \end{Phonetics}
\end{Entry}

\begin{Entry}{休息室}{6,10,9}{⼈,⼼,⼧}
  \begin{Phonetics}{休息室}{xiu1xi1shi4}
    \definition{s.}{saguão | salão}
  \end{Phonetics}
\end{Entry}

\begin{Entry}{休假}{6,11}{⼈,⼈}
  \begin{Phonetics}{休假}{xiu1/jia4}[][HSK 2]
    \definition{v.+compl.}{ter um feriado; tirar férias; sair de férias}
  \synonymref{放假}{fang4/jia4}
  \synonymref{休憩}{xiu1qi4}
  \synonymref{休息}{xiu1xi5}
  \antonymref{工作}{gong1zuo4}
  \antonymref{上班}{shang4/ban1}
  \end{Phonetics}
\end{Entry}

\begin{Entry}{休憩}{6,16}{⼈,⼼}
  \begin{Phonetics}{休憩}{xiu1qi4}
    \definition{v.}{relaxar | descansar | dar um tempo}
  \synonymref{停息}{ting2xi1}
  \synonymref{停歇}{ting2xie1}
  \synonymref{休整}{xiu1zheng3}
  \synonymref{休息}{xiu1xi5}
  \synonymref{暂停}{zan4ting2}
  \antonymref{忙碌}{mang2lu4}
  \end{Phonetics}
\end{Entry}

\begin{Entry}{休整}{6,16}{⼈,⽁}
  \begin{Phonetics}{休整}{xiu1zheng3}
    \definition{v.}{(militar) descansar e reorganizar}
  \synonymref{休憩}{xiu1qi4}
  \synonymref{休息}{xiu1xi5}
  \synonymref{沿着}{yan2zhe5}
  \end{Phonetics}
\end{Entry}

%%%%%%%%%% 众 %%%%%%%%%%
\subsection*{众}\addcontentsline{loh}{figure}{众}

\begin{Entry}{众}{6}{⼈}
  \begin{Phonetics}{众}{zhong4}
    \definition*{s.}{Câmara dos Deputados, abreviação de 众议院}
    \definition{adj.}{numerosos}
    \definition{s.}{multidão; as massas}
  \seealsoref{众议院}{zhong4yi4yuan4}
  \synonymref{多}{duo1}
  \synonymref{公}{gong1}
  \antonymref{寡}{gua3}
  \end{Phonetics}
\end{Entry}

\begin{Entry}{众议院}{6,5,9}{⼈,⾔,⾩}
  \begin{Phonetics}{众议院}{zhong4yi4yuan4}
    \definition*{s.}{Casa baixa da Assembléia Bicameral | Câmara dos Deputados}
  \end{Phonetics}
\end{Entry}

\begin{Entry}{众多}{6,6}{⼈,⼣}
  \begin{Phonetics}{众多}{zhong4duo1}[][HSK 5]
    \definition{adj.}{muitos; numerosos; multitudinários}
  \synonymref{稠密}{chou2mi4}
  \antonymref{单独}{dan1du2}
  \antonymref{单一}{dan1yi1}
  \antonymref{孤独}{gu1du2}
  \antonymref{唯独}{wei2du2}
  \end{Phonetics}
\end{Entry}

%%%%%%%%%% 优 %%%%%%%%%%
\subsection*{优}\addcontentsline{loh}{figure}{优}

\begin{Entry}{优}{6}{⼈}
  \begin{Phonetics}{优}{you1}
    \definition*{s.}{Sobrenome: You}
    \definition{adj.}{excelente; bom; excepcional | amplo; abundante}
    \definition{s.}{Arcaico: ator ou atriz}
    \definition{v.}{dar tratamento preferencial}
  \synonymref{好}{hao3}
  \synonymref{良}{liang2}
  \antonymref{差}{cha4}
  \antonymref{劣}{lie4}
  \end{Phonetics}
\end{Entry}

\begin{Entry}{优于}{6,3}{⼈,⼆}
  \begin{Phonetics}{优于}{you1yu2}
    \definition{v.}{superar}
  \end{Phonetics}
\end{Entry}

\begin{Entry}{优先}{6,6}{⼈,⼉}
  \begin{Phonetics}{优先}{you1xian1}[][HSK 5]
    \definition{adj.}{anterior; sênior; subjacente}
    \definition{v.}{ter prioridade; ter precedência; colocar-se à frente de outras pessoas ou assuntos}
  \antonymref{稍后}{shao1hou4}
  \end{Phonetics}
\end{Entry}

\begin{Entry}{优伶}{6,7}{⼈,⼈}
  \begin{Phonetics}{优伶}{you1ling2}
    \definition{s.}{Obsoleto: ator; atriz; artista performático}
  \synonymref{演员}{yan3yuan2}
  \synonymref{艺人}{yi4ren2}
  \end{Phonetics}
\end{Entry}

\begin{Entry}{优秀}{6,7}{⼈,⽲}
  \begin{Phonetics}{优秀}{you1xiu4}[][HSK 4]
    \definition{adj.}{esplêndido; excelente; extraordinário; excepcional; notável; descreve moral, qualidades, realizações, aprendizado, etc. muito bons.}
  \seealsoref{优美}{you1mei3}
  \synonymref{出色}{chu1se4}
  \synonymref{杰出}{jie2chu1}
  \synonymref{良好}{liang2hao3}
  \synonymref{突出}{tu1/chu1}
  \synonymref{先进}{xian1jin4}
  \synonymref{一流}{yi4liu2}
  \synonymref{优质}{you1zhi4}
  \antonymref{恶劣}{e4lie4}
  \end{Phonetics}
\end{Entry}

\begin{Entry}{优良}{6,7}{⼈,⾉}
  \begin{Phonetics}{优良}{you1liang2}[][HSK 4]
    \definition{adj.}{ótimo; bom; excelente; (variedade, qualidade, desempenho, estilo, etc.) muito bom}
  \synonymref{崇高}{chong2gao1}
  \synonymref{杰出}{jie2chu1}
  \synonymref{良好}{liang2hao3}
  \synonymref{优秀}{you1xiu4}
  \antonymref{恶劣}{e4lie4}
  \antonymref{劣质}{lie4zhi4}
  \end{Phonetics}
\end{Entry}

\begin{Entry}{优势}{6,8}{⼈,⼒}
  \begin{Phonetics}{优势}{you1shi4}[][HSK 3]
    \definition[种,个]{s.}{vantagem; superioridade; preponderância; posição dominante; uma situação favorável que permite superar o adversário}
  \antonymref{劣势}{lie4shi4}
  \antonymref{弱点}{ruo4dian3}
  \end{Phonetics}
\end{Entry}

\begin{Entry}{优质}{6,8}{⼈,⾙}
  \begin{Phonetics}{优质}{you1zhi4}[][HSK 6]
    \definition{adj.}{excelente qualidade; alta qualidade; qualidade superior; alto grau}
  \synonymref{优秀}{you1xiu4}
  \antonymref{变质}{bian4/zhi4}
  \antonymref{劣质}{lie4zhi4}
  \end{Phonetics}
\end{Entry}

\begin{Entry}{优厚}{6,9}{⼈,⼚}
  \begin{Phonetics}{优厚}{you1hou4}
    \definition{adj.}{generoso}
  \synonymref{丰厚}{feng1hou4}
  \antonymref{苛刻}{ke1ke4}
  \end{Phonetics}
\end{Entry}

\begin{Entry}{优点}{6,9}{⼈,⽕}
  \begin{Phonetics}{优点}{you1dian3}[][HSK 3]
    \definition[个,项,种,些]{s.}{mérito; virtude; ponto forte; vantagem (em oposição a 缺点)}
  \seealsoref{缺点}{que1dian3}
  \synonymref{长处}{chang2chu4}
  \synonymref{好处}{hao3chu5}
  \synonymref{便宜}{pian2yi5}
  \synonymref{所长}{suo3zhang3}
  \synonymref{甜头}{tian2tou5}
  \antonymref{弊端}{bi4duan1}
  \antonymref{短处}{duan3chu5}
  \antonymref{毛病}{mao2bing5}
  \antonymref{缺点}{que1dian3}
  \antonymref{缺陷}{que1xian4}
  \antonymref{弱点}{ruo4dian3}
  \end{Phonetics}
\end{Entry}

\begin{Entry}{优美}{6,9}{⼈,⽺}
  \begin{Phonetics}{优美}{you1mei3}[][HSK 4]
    \definition{adj.}{fino; elegante; gracioso; bonito}
  \synonymref{精美}{jing1mei3}
  \synonymref{美好}{mei3hao3}
  \synonymref{美丽}{mei3li4}
  \synonymref{美妙}{mei3miao4}
  \antonymref{丑陋}{chou3lou4}
  \end{Phonetics}
\end{Entry}

\begin{Entry}{优选}{6,9}{⼈,⾡}
  \begin{Phonetics}{优选}{you1xuan3}
    \definition{adj.}{preferido; selecionado a dedo; escolhido a dedo}
    \definition{v.}{otimizar; escolher a melhor opção}
  \end{Phonetics}
\end{Entry}

\begin{Entry}{优格}{6,10}{⼈,⽊}
  \begin{Phonetics}{优格}{you1ge2}
    \definition{s.}{Empréstimo linguístico: iogurte}
  \end{Phonetics}
\end{Entry}

\begin{Entry}{优盘}{6,11}{⼈,⽫}
  \begin{Phonetics}{优盘}{you1pan2}
    \definition[个]{s.}{unidade de memória USB}
  \seealsoref{闪存盘}{shan3cun2pan2}
  \end{Phonetics}
\end{Entry}

\begin{Entry}{优惠}{6,12}{⼈,⼼}
  \begin{Phonetics}{优惠}{you1hui4}[][HSK 5]
    \definition{adj.}{especial; pechincha; reduzido; com desconto | favorável; preferencial; melhores condições ou tratamento do que o normal, permitindo que as pessoas obtenham mais benefícios}
  \synonymref{打折}{da3/zhe2}
  \synonymref{实惠}{shi2hui4}
  \synonymref{特价}{te4jia4}
  \end{Phonetics}
\end{Entry}

\begin{Entry}{优等}{6,12}{⼈,⽵}
  \begin{Phonetics}{优等}{you1deng3}
    \definition{adj.}{excelente | de primeira linha | alta classe | da mais alta ordem, superior}
  \antonymref{劣质}{lie4zhi4}
  \end{Phonetics}
\end{Entry}

\begin{Entry}{优裕}{6,12}{⼈,⾐}
  \begin{Phonetics}{优裕}{you1yu4}
    \definition{adj.}{abundante | bastante}
    \definition{s.}{abundância}
  \synonymref{充足}{chong1zu2}
  \synonymref{富足}{fu4zu2}
  \antonymref{拮据}{jie2ju1}
  \antonymref{窘迫}{jiong3po4}
  \antonymref{贫穷}{pin2qiong2}
  \end{Phonetics}
\end{Entry}

%%%%%%%%%% 伙 %%%%%%%%%%
\subsection*{伙}\addcontentsline{loh}{figure}{伙}

\begin{Entry}{伙}{6}{⼈}
  \begin{Phonetics}{伙}{huo3}[][HSK 4]
    \definition{clas.}{grupo; multidão; banda}
    \definition{s.}{iguaria; alimentação; refeições | parceiro; companheiro | coletivo de colegas}
    \definition{v.}{combinar; unir}
  \end{Phonetics}
\end{Entry}

\begin{Entry}{伙伴}{6,7}{⼈,⼈}
  \begin{Phonetics}{伙伴}{huo3ban4}[][HSK 4]
    \definition[个,位,群]{s.}{parceiro; companheiro; antigo sistema militar de dez pessoas para uma fogueira, o chefe da fogueira, uma pessoa encarregada de cozinhar, com a fogueira é chamado de parceiro da fogueira, agora se refere à participação comum em uma determinada organização ou engajada em certas atividades}
  \synonymref{伴侣}{ban4lv3}
  \synonymref{搭档}{da1dang4}
  \synonymref{朋友}{peng2you5}
  \synonymref{同伴}{tong2ban4}
  \synonymref{同伙}{tong2huo3}
  \antonymref{敌人}{di2ren2}
  \antonymref{对手}{dui4shou3}
  \end{Phonetics}
\end{Entry}

\begin{Entry}{伙食}{6,9}{⼈,⾷}
  \begin{Phonetics}{伙食}{huo3shi2}[][HSK 7-9]
    \definition{s.}{angu; rancho; comida; refeições; refere"-se às refeições no refeitório coletivo da unidade}
  \synonymref{膳食}{shan4shi2}
  \end{Phonetics}
\end{Entry}

%%%%%%%%%% 会 %%%%%%%%%%
\subsection*{会}\addcontentsline{loh}{figure}{会}

\begin{Entry}{会}{6}{⼈}
  \begin{Phonetics}{会}{hui4}[][HSK 1,2]
    \definition{adv.}{um momento}
    \definition{clas.}{momento; um curto período de tempo}
    \definition{s.}{reunião; festa; conferência; reunião com um objetivo específico | reunião; reunião no trabalho | feira do templo; festival religioso | associação; sociedade; sindicato; certas organizações públicas | oportunidade; ocasião; momento oportuno | cidade principal; capital; cidade central}
    \definition{suf.}{união; grupo; associação}
    \definition{v.}{ser provável que; ter certeza de; indica que é possível realizar (é possível responder à pergunta separadamente) |  poder; ser capaz de; significa saber como fazer ou ter a capacidade de fazer (geralmente se refere a coisas que precisam ser aprendidas) | saber; compreender; entender | encontrar; ver | reunir-se; reunir; agregar; juntar | destacar-se em; ser bom em; ser hábil em; indica proficiência | pagar (ou custear) contas}
  \synonymref{会议}{hui4yi4}
  \synonymref{能}{neng2}
  \end{Phonetics}
  \begin{Phonetics}{会}{kuai4}
    \definition[个,场,次]{s.}{contabilidade}
    \definition{v.}{computar; calcular; equilibrar uma conta}
  \synonymref{计算}{ji4suan4}
  \end{Phonetics}
\end{Entry}

\begin{Entry}{会见}{6,4}{⼈,⾒}
  \begin{Phonetics}{会见}{hui4jian4}[][HSK 6]
    \definition{v.}{entrevistar; encontrar-se com (especialmente um visitante estrangeiro)}
  \synonymref{拜访}{bai4fang3}
  \synonymref{访问}{fang3wen4}
  \synonymref{会面}{hui4/mian4}
  \synonymref{会晤}{hui4wu4}
  \synonymref{见面}{jian4/mian4}
  \synonymref{接见}{jie1jian4}
  \end{Phonetics}
\end{Entry}

\begin{Entry}{会计}{6,4}{⼈,⾔}
  \begin{Phonetics}{会计}{kuai4ji5}[][HSK 4]
    \definition[个,位,名]{s.}{contabilidade | contador; contabilista; guarda-livros; pessoal que trabalha como contador}
  \end{Phonetics}
\end{Entry}

\begin{Entry}{会长}{6,4}{⼈,⾧}
  \begin{Phonetics}{会长}{hui4zhang3}[][HSK 6]
    \definition[位,名,个,些]{s.}{presidente de uma associação ou sociedade | presidente de um clube, comitê etc.}
  \end{Phonetics}
\end{Entry}

\begin{Entry}{会头}{6,5}{⼈,⼤}
  \begin{Phonetics}{会头}{hui4tou2}
    \definition{s.}{chefe da reunião}
  \end{Phonetics}
\end{Entry}

\begin{Entry}{会议}{6,5}{⼈,⾔}
  \begin{Phonetics}{会议}{hui4yi4}[][HSK 3]
    \definition[次,届,个,场]{s.}{reunião; conferência; reunião organizada pela organização relevante para ouvir opiniões, discutir questões e distribuir tarefas | conselho; congresso; um órgão ou organização permanente que discute e trata frequentemente assuntos importantes}
  \synonymref{集会}{ji2hui4}
  \synonymref{聚会}{ju4hui4}
  \synonymref{开会}{kai1/hui4}
  \synonymref{理解}{li3jie3}
  \synonymref{领会}{ling3hui4}
  \end{Phonetics}
\end{Entry}

\begin{Entry}{会场}{6,6}{⼈,⼟}
  \begin{Phonetics}{会场}{hui4chang3}[][HSK 7-9]
    \definition[个]{s.}{local de encontro; lugar onde as pessoas se reúnem}
  \end{Phonetics}
\end{Entry}

\begin{Entry}{会员}{6,7}{⼈,⼝}
  \begin{Phonetics}{会员}{hui4yuan2}[][HSK 3]
    \definition[位,名,个,些]{s.}{membro; associado; membros de certos grupos ou organizações}
  \synonymref{队员}{dui4yuan2}
  \synonymref{客户}{ke4hu4}
  \synonymref{团员}{tuan2yuan2}
  \end{Phonetics}
\end{Entry}

\begin{Entry}{会诊}{6,7}{⼈,⾔}
  \begin{Phonetics}{会诊}{hui4/zhen3}[][HSK 7-9]
    \definition{s.}{consulta de médicos; consulta (de grupo)}[医生举行会诊,决定是否需要动手术。===Os médicos realizam uma consulta para decidir se a cirurgia é necessária.]
    \definition{v.+compl.}{(médicos) realizar uma consulta médica; consultar}
  \synonymref{诊断}{zhen3duan4}
  \end{Phonetics}
\end{Entry}

\begin{Entry}{会面}{6,9}{⼈,⾯}
  \begin{Phonetics}{会面}{hui4/mian4}[][HSK 7-9]
    \definition{v.+compl.}{reunir-se; encontrar}
  \synonymref{会见}{hui4jian4}
  \synonymref{会晤}{hui4wu4}
  \synonymref{见面}{jian4/mian4}
  \synonymref{聚集}{ju4ji2}
  \antonymref{分手}{fen1/shou3}
  \antonymref{分离}{fen1li2}
  \end{Phonetics}
\end{Entry}

\begin{Entry}{会首}{6,9}{⼈,⾸}
  \begin{Phonetics}{会首}{hui4shou3}
    \definition{s.}{Obsoleto: chefe ou patrocinador de uma sociedade ou organização não oficial; líder}
  \seealsoref{会头}{hui4tou2}
  \end{Phonetics}
\end{Entry}

\begin{Entry}{会谈}{6,10}{⼈,⾔}
  \begin{Phonetics}{会谈}{hui4tan2}[][HSK 5]
    \definition{v.}{manter conversações; comumente usado em assuntos internacionais ou atividades diplomáticas}
  \synonymref{谈判}{tan2pan4}
  \end{Phonetics}
\end{Entry}

\begin{Entry}{会晤}{6,11}{⼈,⽇}
  \begin{Phonetics}{会晤}{hui4wu4}[][HSK 7-9]
    \definition{v.}{reunir-se (com líderes estaduais ou figuras sociais para discutir assuntos importantes)}
  \synonymref{会面}{hui4/mian4}
  \synonymref{会见}{hui4jian4}
  \synonymref{见面}{jian4/mian4}
  \synonymref{接见}{jie1jian4}
  \end{Phonetics}
\end{Entry}

\begin{Entry}{会意}{6,13}{⼈,⼼}
  \begin{Phonetics}{会意}{hui4yi4}[][HSK 7-9]
    \definition{s.}{compreensão; conhecimento | compostos associativos, uma das seis categorias de caracteres chineses, que são formados pela combinação de dois ou mais elementos, cada um com um significado próprio, para criar um novo significado, por exemplo, 信, um caractere composto de 人 (homem) e 言 (palavra), significando uma mensagem ou algo em que se pode acreditar ou confiar}
    \definition{v.}{entender; saber}
  \synonymref{领会}{ling3hui4}
  \synonymref{领略}{ling3lve4}
  \synonymref{体会}{ti3hui4}
  \antonymref{茫然}{mang2ran2}
  \end{Phonetics}
\end{Entry}

%%%%%%%%%% 伞 %%%%%%%%%%
\subsection*{伞}\addcontentsline{loh}{figure}{伞}

\begin{Entry}{伞}{6}{⼈}
  \begin{Phonetics}{伞}{san3}[][HSK 4]
    \definition*{s.}{Sobrenome: San}
    \definition[把]{s.}{guarda-chuva; proteção contra chuva ou sol | algo que tem o formato de um guarda-chuva}
  \end{Phonetics}
\end{Entry}

%%%%%%%%%% 伟 %%%%%%%%%%
\subsection*{伟}\addcontentsline{loh}{figure}{伟}

\begin{Entry}{伟}{6}{⼈}
  \begin{Phonetics}{伟}{wei3}
    \definition*{s.}{Sobrenome: Wei}
    \definition{adj.}{grande; ótimo; poderoso | Literário: grande}
  \end{Phonetics}
\end{Entry}

\begin{Entry}{伟大}{6,3}{⼈,⼤}
  \begin{Phonetics}{伟大}{wei3da4}[][HSK 3]
    \definition{adj.}{ótimo; excelente; extrovertido; descreve uma pessoa com moral e qualidades excelentes, habilidades e realizações excepcionais, que inspira grande respeito | ótimo; magnífico; descreve algo de grande importância, com impacto significativo, acima do normal, algo notável}
  \synonymref{崇高}{chong2gao1}
  \synonymref{广大}{guang3da4}
  \synonymref{宏大}{hong2da4}
  \synonymref{宏伟}{hong2wei3}
  \synonymref{巨大}{ju4da4}
  \synonymref{庞大}{pang2da4}
  \synonymref{雄伟}{xiong2wei3}
  \antonymref{渺小}{miao3xiao3}
  \antonymref{平凡}{ping2fan2}
  \end{Phonetics}
\end{Entry}

%%%%%%%%%% 传 %%%%%%%%%%
\subsection*{传}\addcontentsline{loh}{figure}{传}

\begin{Entry}{传}{6}{⼈}
  \begin{Phonetics}{传}{chuan2}[][HSK 3]
    \definition{v.}{passar; passar adiante | passar adiante; legar; passar de\dots para\dots; passar da geração anterior para a seguinte | transmitir (conhecimento, habilidade, etc.); comunicar; ensinar | espalhar; propagar | transmitir; conduzir; transferir | transmitir; expressar | convocar; dar ordem para chamar alguém | infectar; ser contagioso | enviar documentos por e-mail ou fax}
  \end{Phonetics}
  \begin{Phonetics}{传}{zhuan4}
    \definition{s.}{comentários sobre clássicos; obras que explicam as escrituras| biografia | romances sobre eventos históricos; obras que narram histórias históricas}
  \end{Phonetics}
\end{Entry}

\begin{Entry}{传人}{6,2}{⼈,⼈}
  \begin{Phonetics}{传人}{chuan2ren2}[][HSK 7-9]
    \definition{s.}{Literátio: alguém que pode herdar uma determinada disciplina acadêmica e fazê-la se espalhar}
    \definition{v.}{transmitir uma habilidade especial, etc.; ensinar}
  \synonymref{弟子}{di4zi3}
  \synonymref{后代}{hou4dai4}
  \synonymref{后人}{hou4ren2}
  \synonymref{徒弟}{tu2di5}
  \end{Phonetics}
\end{Entry}

\begin{Entry}{传出}{6,5}{⼈,⼐}
  \begin{Phonetics}{传出}{chuan2chu1}[][HSK 6]
    \definition{adj.}{eferente (nervo)}
    \definition{v.}{disseminar | transmitir para fora}
  \end{Phonetics}
\end{Entry}

\begin{Entry}{传达}{6,6}{⼈,⾡}
  \begin{Phonetics}{传达}{chuan2da2}[][HSK 5]
    \definition{s.}{recepção e registro de chamadas em um estabelecimento público | zelador | recepcionista}
    \definition{v.}{passar adiante (informações, etc.); transmitir; retransmitir; comunicar; expressar}
  \synonymref{传播}{chuan2bo1}
  \synonymref{传递}{chuan2di4}
  \synonymref{通报}{tong1bao4}
  \synonymref{转告}{zhuan3gao4}
  \antonymref{反馈}{fan3kui4}
  \end{Phonetics}
\end{Entry}

\begin{Entry}{传来}{6,7}{⼈,⽊}
  \begin{Phonetics}{传来}{chuan2lai2}[][HSK 3]
    \definition{v.}{(um som) passar; transmitir de algum lugar para o local onde o locutor se encontra | (notícias) chegar; transmitir mensagens ou informações}
  \end{Phonetics}
\end{Entry}

\begin{Entry}{传言}{6,7}{⼈,⾔}
  \begin{Phonetics}{传言}{chuan2yan2}[][HSK 6]
    \definition[个,种,些]{s.}{boato; rumor}
    \definition{v.}{passar uma mensagem | Obsoleto: fazer um discurso | falar; fazer uma declaração}
  \end{Phonetics}
\end{Entry}

\begin{Entry}{传奇}{6,8}{⼈,⼤}
  \begin{Phonetics}{传奇}{chuan2qi2}[][HSK 7-9]
    \definition[个,项,段]{s.}{contos das dinastias Tang e Song (618-1279); contos de maravilhas | dramas poéticos das dinastias Ming e Qing (1368-1911); dramas em verso | lenda; romance; histórias lendárias}
  \synonymref{传说}{chuan2shuo1}
  \synonymref{神话}{shen2hua4}
  \antonymref{事实}{shi4shi2}
  \end{Phonetics}
\end{Entry}

\begin{Entry}{传承}{6,8}{⼈,⼿}
  \begin{Phonetics}{传承}{chuan2cheng2}[][HSK 7-9]
    \definition{v.}{herdar; transmitir (para as gerações futuras); passar adiante (desde os tempos antigos)}
  \synonymref{秉承}{bing3cheng2}
  \synonymref{流传}{liu2chuan2}
  \synonymref{延续}{yan2xu4}
  \end{Phonetics}
\end{Entry}

\begin{Entry}{传染}{6,9}{⼈,⽊}
  \begin{Phonetics}{传染}{chuan2ran3}[][HSK 7-9]
    \definition{v.}{infectar; contagiar; comunicar; ser contagioso; patógenos que entram em outros organismos a partir de organismos doentes}
  \synonymref{感染}{gan3ran3}
  \synonymref{污染}{wu1ran3}
  \antonymref{抵抗}{di3kang4}
  \end{Phonetics}
\end{Entry}

\begin{Entry}{传染病}{6,9,10}{⼈,⽊,⽧}
  \begin{Phonetics}{传染病}{chuan2ran3bing4}[][HSK 7-9]
    \definition{s.}{doença contagiosa; doença infecciosa; pestilência}
  \end{Phonetics}
\end{Entry}

\begin{Entry}{传给}{6,9}{⼈,⽷}
  \begin{Phonetics}{传给}{chuan2gei3}
    \definition{v.}{passar para | transferir para | entregar a}
  \end{Phonetics}
\end{Entry}

\begin{Entry}{传统}{6,9}{⼈,⽷}
  \begin{Phonetics}{传统}{chuan2tong3}[][HSK 4]
    \definition{adj.}{tradicional; histórico; transmitido de geração em geração | antiquado, conservador e fora de sintonia com os tempos}
    \definition[个,种,项]{s.}{tradição; costume; fatores sociais, como costumes, moral, ideias, estilos, artes, instituições etc., que são transmitidos de uma geração para outra e que são característicos da sociedade}
  \synonymref{古板}{gu3ban3}
  \synonymref{古代}{gu3dai4}
  \antonymref{潮流}{chao2liu2}
  \antonymref{时尚}{shi2shang4}
  \antonymref{现代}{xian4dai4}
  \end{Phonetics}
\end{Entry}

\begin{Entry}{传说}{6,9}{⼈,⾔}
  \begin{Phonetics}{传说}{chuan2shuo1}[][HSK 3]
    \definition[个,种,段]{s.}{lenda; conto popular; folclore; coisas lendárias; especificamente, lendas populares}
    \definition{v.}{dizer que; ser dito; passar de boca em boca; transmitir oralmente, segundo a tradição}
  \synonymref{传奇}{chuan2qi2}
  \synonymref{传闻}{chuan2wen2}
  \synonymref{据说}{ju4shuo1}
  \synonymref{听说}{ting1shuo1}
  \end{Phonetics}
\end{Entry}

\begin{Entry}{传闻}{6,9}{⼈,⾨}
  \begin{Phonetics}{传闻}{chuan2wen2}[][HSK 7-9]
    \definition{s.}{boato; rumor; conversa}
    \definition{v.}{diz"-se; eles disseram; notícias não verificadas espalhadas por pessoas}
  \end{Phonetics}
\end{Entry}

\begin{Entry}{传真}{6,10}{⼈,⼗}
  \begin{Phonetics}{传真}{chuan2zhen1}[][HSK 5]
    \definition[部,台,份,个]{s.}{\emph{FAX}, facsímile; texto, diagramas, fotografias, etc., transmitidos por aparelho de fax}
    \definition{v.}{enviar um fax | retratar; reproduzir}[她的画传真了古代建筑。===Suas pinturas são reproduções fiéis da arquitetura antiga.]
  \end{Phonetics}
\end{Entry}

\begin{Entry}{传递}{6,10}{⼈,⾡}
  \begin{Phonetics}{传递}{chuan2di4}[][HSK 5]
    \definition{v.}{transmitir; entregar; transferir; passar adiante}
  \synonymref{传达}{chuan2da2}
  \synonymref{传输}{chuan2shu1}
  \synonymref{发送}{fa1song5}
  \synonymref{通报}{tong1bao4}
  \synonymref{转递}{zhuan3di4}
  \end{Phonetics}
\end{Entry}

\begin{Entry}{传授}{6,11}{⼈,⼿}
  \begin{Phonetics}{传授}{chuan2shou4}[][HSK 7-9]
    \definition{v.}{transmitir; ensinar; passar adiante (conhecimento, habilidade, etc.); ensinar conhecimento e habilidades aos outros}
  \synonymref{教授}{jiao4shou4}
  \synonymref{教学}{jiao4xue2}
  \antonymref{获取}{huo4qu3}
  \end{Phonetics}
\end{Entry}

\begin{Entry}{传媒}{6,12}{⼈,⼥}
  \begin{Phonetics}{传媒}{chuan2mei2}[][HSK 6]
    \definition{s.}{meios de comunicação de massa; mídia; jornais, rádio, televisão, \emph{Internet} e outras ferramentas de notícias | meio; veículo; vetor; o meio ou via de transmissão da doença}
  \synonymref{媒体}{mei2ti3}
  \end{Phonetics}
\end{Entry}

\begin{Entry}{传输}{6,13}{⼈,⾞}
  \begin{Phonetics}{传输}{chuan2shu1}[][HSK 6]
    \definition{v.}{transmitir, transportar (energia, informação, etc.)}
  \synonymref{传播}{chuan2bo1}
  \synonymref{传递}{chuan2di4}
  \synonymref{输送}{shu1song4}
  \synonymref{转递}{zhuan3di4}
  \antonymref{下载}{xia4zai3}
  \end{Phonetics}
\end{Entry}

\begin{Entry}{传播}{6,15}{⼈,⼿}
  \begin{Phonetics}{传播}{chuan2bo1}[][HSK 3]
    \definition{v.}{espalhar; difundir; propagar; disseminar}
  \synonymref{传达}{chuan2da2}
  \synonymref{传输}{chuan2shu1}
  \synonymref{鼓吹}{gu3chui1}
  \synonymref{流传}{liu2chuan2}
  \synonymref{流转}{liu2zhuan3}
  \synonymref{散布}{san4bu4}
  \synonymref{宣扬}{xuan1yang2}
  \synonymref{遗传}{yi2chuan2}
  \antonymref{宣传}{xuan1chuan2}
  \end{Phonetics}
\end{Entry}

%%%%%%%%%% 伤 %%%%%%%%%%
\subsection*{伤}\addcontentsline{loh}{figure}{伤}

\begin{Entry}{伤}{6}{⼈}
  \begin{Phonetics}{伤}{shang1}[][HSK 3]
    \definition*{s.}{Sobrenome: Shang}
    \definition[处]{s.}{ferida; ferimento}
    \definition{v.}{ferir; machucar | ter os sentimentos feridos | estar angustiado | enjoar de algo; desenvolver aversão a algo | ser prejudicial a; entravar}
  \end{Phonetics}
\end{Entry}

\begin{Entry}{伤亡}{6,3}{⼈,⼇}
  \begin{Phonetics}{伤亡}{shang1wang2}[][HSK 6]
    \definition{s.}{ferimentos e mortes; feridos e mortos; pessoas feridas e mortas; baixas}
    \definition{v.}{ser ferido e morto}
  \synonymref{伤害}{shang1hai4}
  \synonymref{死亡}{si3wang2}
  \synonymref{英勇}{ying1yong3}
  \end{Phonetics}
\end{Entry}

\begin{Entry}{伤口}{6,3}{⼈,⼝}
  \begin{Phonetics}{伤口}{shang1kou3}[][HSK 6]
    \definition[处]{s.}{corte; ferida; onde a pele, os músculos, etc. são feridos, rompidos ou onde são realizadas aberturas cirúrgicas}
  \end{Phonetics}
\end{Entry}

\begin{Entry}{伤心}{6,4}{⼈,⼼}
  \begin{Phonetics}{伤心}{shang1/xin1}[][HSK 3]
    \definition{v.+compl.}{estar triste; lamentar; estar com o coração partido; sentir-se triste por causa de infortúnio ou decepção}
  \antonymref{快乐}{kuai4le4}
  \end{Phonetics}
\end{Entry}

\begin{Entry}{伤员}{6,7}{⼈,⼝}
  \begin{Phonetics}{伤员}{shang1yuan2}[][HSK 6]
    \definition[名,位,个]{s.}{Exército: pessoal ferido; os feridos}
  \synonymref{病人}{bing4ren2}
  \end{Phonetics}
\end{Entry}

\begin{Entry}{伤势}{6,8}{⼈,⼒}
  \begin{Phonetics}{伤势}{shang1shi4}[][HSK 7-9]
    \definition{s.}{condição de uma lesão (ou ferida) | o estado de uma lesão (ou ferida)}
  \synonymref{伤害}{shang1hai4}
  \synonymref{伤口}{shang1kou3}
  \end{Phonetics}
\end{Entry}

\begin{Entry}{伤残}{6,9}{⼈,⽍}
  \begin{Phonetics}{伤残}{shang1can2}[][HSK 7-9]
    \definition{adj.}{ferido e incapacitado; deficiente; mutilado; desfigurado; aleijado | (um produto, etc.) defeituoso; falho; danificado}
  \synonymref{残疾}{can2ji5}
  \antonymref{完好}{wan2hao3}
  \end{Phonetics}
\end{Entry}

\begin{Entry}{伤害}{6,10}{⼈,⼧}
  \begin{Phonetics}{伤害}{shang1hai4}[][HSK 4]
    \definition[种]{v.}{ferir; prejudicar; machucar; magoar; causar danos físicos ou mentais}
  \synonymref{摧毁}{cui1hui3}
  \synonymref{虐待}{nve4dai4}
  \synonymref{破坏}{po4huai4}
  \synonymref{受伤}{shou4/shang1}
  \synonymref{损害}{sun3hai4}
  \synonymref{损伤}{sun3shang1}
  \synonymref{危害}{wei1hai4}
  \synonymref{危险}{wei1xian3}
  \antonymref{爱护}{ai4hu4}
  \antonymref{保护}{bao3hu4}
  \antonymref{关怀}{guan1huai2}
  \antonymref{救援}{jiu4yuan2}
  \antonymref{治病}{zhi4bing4}
  \antonymref{治疗}{zhi4liao2}
  \end{Phonetics}
\end{Entry}

\begin{Entry}{伤脑筋}{6,10,12}{⼈,⾁,⽵}
  \begin{Phonetics}{伤脑筋}{shang1 nao3jin1}[][HSK 7-9]
    \definition{adj.}{complicado; problemático; incômodo; que causa dor de cabeça em alguém; descreve uma situação como difícil e que exige muita reflexão}
  \end{Phonetics}
\end{Entry}

\begin{Entry}{伤痕}{6,11}{⼈,⽧}
  \begin{Phonetics}{伤痕}{shang1hen2}[][HSK 7-9]
    \definition[道,处]{s.}{cicatriz; ferida; também se refere a uma marca deixada após um objeto ter sido danificado | cicatriz; ferida; metáfora para trauma psicológico}
  \end{Phonetics}
\end{Entry}

\begin{Entry}{伤感}{6,13}{⼈,⼼}
  \begin{Phonetics}{伤感}{shang1gan3}[][HSK 7-9]
    \definition{adj.}{doente de coração; sentimental; piegas; triste}
  \end{Phonetics}
\end{Entry}

%%%%%%%%%% 伦 %%%%%%%%%%
\subsection*{伦}\addcontentsline{loh}{figure}{伦}

\begin{Entry}{伦}{6}{⼈}
  \begin{Phonetics}{伦}{lun2}
    \definition*{s.}{Sobrenome: Lun}
    \definition{s.}{relações humanas (especialmente como concebidas pela ética feudal) | lógica; ordem | par; correspondência; (mesma) classe | ética; relações humanas | sequência lógica; ordem | o mesmo tipo; semelhante; igual}
  \end{Phonetics}
\end{Entry}

\begin{Entry}{伦理}{6,11}{⼈,⽟}
  \begin{Phonetics}{伦理}{lun2li3}[][HSK 7-9]
    \definition{s.}{ética; princípios morais; refere"-se a vários princípios morais para relacionamentos interpessoais}
  \synonymref{道德}{dao4de2}
  \end{Phonetics}
\end{Entry}

\begin{Entry}{伦敦}{6,12}{⼈,⽁}
  \begin{Phonetics}{伦敦}{lun2dun1}
    \definition*{s.}{Londres}
  \end{Phonetics}
\end{Entry}

%%%%%%%%%% 伪 %%%%%%%%%%
\subsection*{伪}\addcontentsline{loh}{figure}{伪}

\begin{Entry}{伪}{6}{⼈}
  \begin{Phonetics}{伪}{wei3}
    \definition{adj.}{falso; falsificado | fantoche; colaboracionista; ilegal | forjado; falso}
    \definition{pref.}{pseudo-; quasi-; quase-}
  \synonymref{假}{jia3}
  \antonymref{真}{zhen1}
  \end{Phonetics}
\end{Entry}

\begin{Entry}{伪造}{6,10}{⼈,⾡}
  \begin{Phonetics}{伪造}{wei3zao4}[][HSK 7-9]
    \definition{v.}{forjar; falsificar; fabricar; fingir}
  \end{Phonetics}
\end{Entry}

\begin{Entry}{伪装}{6,12}{⼈,⾐}
  \begin{Phonetics}{伪装}{wei3zhuang1}[][HSK 7-9]
    \definition[种,个]{s.}{máscara; disfarce; camuflagem; aparência fingida; algo usado como disfarce}
    \definition{v.}{ser falso; fingir; camuflar; usar certos meios secretos em assuntos militares para enganar e confundir o inimigo}
  \synonymref{假装}{jia3zhuang1}
  \antonymref{真诚}{zhen1cheng2}
  \end{Phonetics}
\end{Entry}

%%%%%%%%%% 似 %%%%%%%%%%
\subsection*{似}\addcontentsline{loh}{figure}{似}

\begin{Entry}{似}{6}{⼈}
  \begin{Phonetics}{似}{shi4}
    \definition{v.}{ver; parecer}
  \synonymref{类}{lei4}
  \synonymref{如}{ru2}
  \synonymref{像}{xiang4}
  \end{Phonetics}
  \begin{Phonetics}{似}{si4}
    \definition*{s.}{Sobrenome: Si}
    \definition{adv.}{parece; como se}
    \definition{v.}{ser semelhante; parecer-se com | parecer; aparecer | exceder}
  \synonymref{类}{lei4}
  \synonymref{如}{ru2}
  \synonymref{像}{xiang4}
  \end{Phonetics}
\end{Entry}

\begin{Entry}{似乎}{6,5}{⼈,⼃}
  \begin{Phonetics}{似乎}{si4hu1}[][HSK 4]
    \definition{adv.}{como se; aparentemente; se parece como}
  \synonymref{仿佛}{fang3fu2}
  \synonymref{好似}{hao3si4}
  \synonymref{好像}{hao3xiang4}
  \synonymref{如同}{ru2tong2}
  \synonymref{相似}{xiang1si4}
  \antonymref{肯定}{ken3ding4}
  \antonymref{确定}{que4ding4}
  \end{Phonetics}
\end{Entry}

\begin{Entry}{似的}{6,8}{⼈,⽩}
  \begin{Phonetics}{似的}{shi4de5}[][HSK 4]
    \definition{part.}{como; como\dots como; como se (embora); usada após uma palavra ou frase para indicar uma semelhança com algo ou uma situação | usada para indicar alto grau}
  \end{Phonetics}
\end{Entry}

\begin{Entry}{似是而非}{6,9,6,8}{⼈,⽇,⽽,⾮}
  \begin{Phonetics}{似是而非}{si4shi4-er2fei1}[][HSK 7-9]
    \definition{expr.}{``Aparentemente verdade, mas na realidade falso.''; especioso; plausível; parecer certo, mas na verdade errado; aparentemente verdadeiro, mas na realidade errado; parecer o que realmente não é; ser como água e vinho; semelhante à realidade, mas não ser como ela é}
  \antonymref{天经地义}{tian1jing1-di4yi4}
  \end{Phonetics}
\end{Entry}

\begin{Entry}{似曾相识}{6,12,9,7}{⼈,⽈,⽬,⾔}
  \begin{Phonetics}{似曾相识}{si4ceng2-xiang1shi2}[][HSK 7-9]
    \definition{s.}{parecem já ter se encontrado antes; aparentemente já se conhecia; \emph{déjà vu} (a experiência de ver exatamente a mesma situação uma segunda vez); aparentemente familiar}
  \end{Phonetics}
\end{Entry}

%%%%%%%%%% 充 %%%%%%%%%%
\subsection*{充}\addcontentsline{loh}{figure}{充}

\begin{Entry}{充}{6}{⼉}
  \begin{Phonetics}{充}{chong1}[][HSK 7-9]
    \definition*{s.}{Sobrenome: Chong}
    \definition{adj.}{suficiente; completo; amplo; cheio}
    \definition{v.}{encher; carregar; atulhar | servir como; agir como | fingir ser; posar como; passar algo como}
  \end{Phonetics}
\end{Entry}

\begin{Entry}{充分}{6,4}{⼉,⼑}
  \begin{Phonetics}{充分}{chong1fen4}[][HSK 4]
    \definition{adj.}{cheio; amplo; abundante; suficiente; adequado}
    \definition{adv.}{totalmente; até o fim}
  \end{Phonetics}
\end{Entry}

\begin{Entry}{充电}{6,5}{⼉,⽥}
  \begin{Phonetics}{充电}{chong1 dian4}[][HSK 4]
    \definition{v.}{carregar (uma bateria); conectar uma fonte de alimentação CC aos terminais da bateria para recarregar a bateria | relaxar; passar o tempo livre; ``recarregar as baterias''; estudar para adquirir mais conhecimento; reabastecer (ou ampliar) o conhecimento; metaforicamente falando, para reabastecer a força física e a energia por meio do descanso e da recreação; também metaforicamente falando, para reabastecer novos conhecimentos e desenvolver novas habilidades por meio do reaprendizado}
  \end{Phonetics}
\end{Entry}

\begin{Entry}{充电器}{6,5,16}{⼉,⽥,⼝}
  \begin{Phonetics}{充电器}{chong1dian4qi4}[][HSK 4]
    \definition[个,台]{s.}{carregador de bateria; dispositivo para alimentar uma bateria com energia, forçando uma corrente através dela}
  \end{Phonetics}
\end{Entry}

\begin{Entry}{充当}{6,6}{⼉,⼹}
  \begin{Phonetics}{充当}{chong1dang1}[][HSK 7-9]
    \definition{v.}{agir como; servir como; desempenhar o papel de; assumir o comando de}
  \end{Phonetics}
\end{Entry}

\begin{Entry}{充沛}{6,7}{⼉,⽔}
  \begin{Phonetics}{充沛}{chong1pei4}[][HSK 7-9]
    \definition{adj.}{abundante; cheio de}
  \end{Phonetics}
\end{Entry}

\begin{Entry}{充足}{6,7}{⼉,⾜}
  \begin{Phonetics}{充足}{chong1zu2}[][HSK 5]
    \definition{adj.}{bastante; adequado; suficiente; mais do que suficiente para atender às necessidades (usado principalmente para coisas mais específicas)}
  \end{Phonetics}
\end{Entry}

\begin{Entry}{充实}{6,8}{⼉,⼧}
  \begin{Phonetics}{充实}{chong1shi2}[][HSK 7-9]
    \definition{adj.}{rico; cheio; substancial; gratificante}
    \definition{v.}{enriquecer; aumentar; substanciar (um argumento)}
  \end{Phonetics}
\end{Entry}

\begin{Entry}{充满}{6,13}{⼉,⽔}
  \begin{Phonetics}{充满}{chong1man3}[][HSK 3]
    \definition{v.}{preencher; encher; cobrir completamente | estar cheio de; estar repleto de; estar transbordando de; estar impregnado de}
  \end{Phonetics}
\end{Entry}

%%%%%%%%%% 兆 %%%%%%%%%%
\subsection*{兆}\addcontentsline{loh}{figure}{兆}

\begin{Entry}{兆}{6}{⼉}
  \begin{Phonetics}{兆}{zhao4}
    \definition*{s.}{Sobrenome: Zhao}
    \definition{num.}{million; 1.000.000; 100.0000 | trilhão; um milhão de milhões; na antiguidade, referia"-se a um trilhão}
    \definition{pref.}{mega-}[这台电脑内存有十六兆。===Este computador possui 16 megabytes de memória.]
    \definition{s.}{sinal; presságio; augúrio}
    \definition{v.}{pressagiar; predizer}
  \end{Phonetics}
\end{Entry}

%%%%%%%%%% 先 %%%%%%%%%%
\subsection*{先}\addcontentsline{loh}{figure}{先}

\begin{Entry}{先}{6}{⼉}
  \begin{Phonetics}{先}{xian1}[][HSK 1]
    \definition*{s.}{Sobrenome: Xian}
    \definition{adv.}{primeiro; antes; mais cedo; com antecedência | no momento; por enquanto; em um curto espaço de tempo; temporariamente}
    \definition{s.}{início; começo; em ordem cronológica ou de precedência | ancestral; geração mais velha; antepassado | tardio; falecido; morto (honrar os mortos)}
  \end{Phonetics}
\end{Entry}

\begin{Entry}{先不先}{6,4,6}{⼉,⼀,⼉}
  \begin{Phonetics}{先不先}{xian1bu4xian1}
    \definition{adv.}{(dialeto) antes de tudo | em primeiro lugar}
  \end{Phonetics}
\end{Entry}

\begin{Entry}{先天}{6,4}{⼉,⼤}
  \begin{Phonetics}{先天}{xian1tian1}
    \definition{adj.}{congênito | inato | natural}
    \definition{s.}{período embrionário}
  \end{Phonetics}
\end{Entry}

\begin{Entry}{先生}{6,5}{⼉,⽣}
  \begin{Phonetics}{先生}{xian1sheng5}[][HSK 1]
    \definition[个,位]{s.}{professor; títulos honoríficos para professores, médicos, etc. | marido; antigamente, referia"-se ao marido de outra pessoa ou ao próprio marido (ambos com pronomes pessoais como determinantes) | médico; títulos usados para se referir aos médicos no passado | refere"-se a pessoas cuja profissão envolve contar histórias, adivinhação, etc.; antigamente, era chamado de contador | senhor; \emph{sir}; título dado aos intelectuais}
  \end{Phonetics}
\end{Entry}

\begin{Entry}{先后}{6,6}{⼉,⼝}
  \begin{Phonetics}{先后}{xian1hou4}[][HSK 5]
    \definition{adv.}{sucessivamente; um após o outro}
    \definition{s.}{prioridade; ordem; cedo ou tarde; primeiro e último}
  \end{Phonetics}
\end{Entry}

\begin{Entry}{先有}{6,6}{⼉,⽉}
  \begin{Phonetics}{先有}{xian1you3}
    \definition{adj.}{preexistente | anterior}
  \end{Phonetics}
\end{Entry}

\begin{Entry}{先进}{6,7}{⼉,⾡}
  \begin{Phonetics}{先进}{xian1jin4}[][HSK 3]
    \definition{adj.}{avançado; progressos rápidos e nível elevado, podendo servir de exemplo a seguir}
    \definition{s.}{indivíduos ou grupos avançados}
  \end{Phonetics}
\end{Entry}

\begin{Entry}{先到先得}{6,8,6,11}{⼉,⼑,⼉,⼻}
  \begin{Phonetics}{先到先得}{xian1dao4xian1de2}
    \definition{expr.}{primeiro a chegar | primeiro a ser servido}
  \end{Phonetics}
\end{Entry}

\begin{Entry}{先前}{6,9}{⼉,⼑}
  \begin{Phonetics}{先前}{xian1qian2}[][HSK 5]
    \definition[出]{s.}{antes; anteriormente; refere"-se ao passado ou a um certo tempo anterior}
  \end{Phonetics}
\end{Entry}

\begin{Entry}{先烈}{6,10}{⼉,⽕}
  \begin{Phonetics}{先烈}{xian1lie4}
    \definition{s.}{mártir}
  \end{Phonetics}
\end{Entry}

\begin{Entry}{先验}{6,10}{⼉,⾺}
  \begin{Phonetics}{先验}{xian1yan4}
    \definition{adj.}{(filosofia) a priori}
  \end{Phonetics}
\end{Entry}

\begin{Entry}{先期}{6,12}{⼉,⽉}
  \begin{Phonetics}{先期}{xian1qi1}
    \definition{adv.}{antecipadamente}
    \definition{s.}{prematuro | \emph{front-end}}
  \end{Phonetics}
\end{Entry}

\begin{Entry}{先锋}{6,12}{⼉,⾦}
  \begin{Phonetics}{先锋}{xian1feng1}[][HSK 6]
    \definition{s.}{pioneiro; vanguarda; a vanguarda de uma batalha ou marcha; geralmente se refere a uma pessoa ou grupo que desempenha um papel de vanguarda}
  \end{Phonetics}
\end{Entry}

%%%%%%%%%% 光 %%%%%%%%%%
\subsection*{光}\addcontentsline{loh}{figure}{光}

\begin{Entry}{光}{6}{⼉}
  \begin{Phonetics}{光}{guang1}[][HSK 3]
    \definition*{s.}{Sobrenome: Guang}
    \definition{adj.}{suave; liso; brilhante | esgotado; sem nada sobrando | brilhante}
    \definition{adv.}{somente; sozinho; meramente}
    \definition{s.}{luz; raio | cenário; paisagem | honra; glória; brilho | claridade | favor; graça | momento | corpo celeste; referindo"-se especificamente a corpos celestes, como o sol, a lua e as estrelas}
    \definition{v.}{glorificar; recuperar; reconquistar | estar nu; expor}
  \end{Phonetics}
\end{Entry}

\begin{Entry}{光污染}{6,6,9}{⼉,⽔,⽊}
  \begin{Phonetics}{光污染}{guang1 wu1ran3}
    \definition{s.}{poluição luminosa}
  \end{Phonetics}
\end{Entry}

\begin{Entry}{光芒}{6,6}{⼉,⾋}
  \begin{Phonetics}{光芒}{guang1mang2}[][HSK 7-9]
    \definition[道]{s.}{brilho; radiância; raios brilhantes; raios de luz; luz forte irradiando em todas as direções}
  \end{Phonetics}
\end{Entry}

\begin{Entry}{光明}{6,8}{⼉,⽇}
  \begin{Phonetics}{光明}{guang1ming2}[][HSK 3]
    \definition{adj.}{brilhante; luminoso | sincero; ingênuo; metáfora da justiça e da esperança | justo; honesto; franco}
    \definition{s.}{luz}
  \end{Phonetics}
\end{Entry}

\begin{Entry}{光明磊落}{6,8,15,12}{⼉,⽇,⽯,⾋}
  \begin{Phonetics}{光明磊落}{guang1ming2-lei3luo4}[][HSK 7-9]
    \definition{expr.}{aberto e sincero; direto e honesto; descreve ser altruísta e de mente aberta; aberto e transparente}
  \end{Phonetics}
\end{Entry}

\begin{Entry}{光泽}{6,8}{⼉,⽔}
  \begin{Phonetics}{光泽}{guang1ze2}[][HSK 7-9]
    \definition{s.}{brilho; lustro; fulgor; luz brilhante refletida de uma superfície; cor e brilho}
  \end{Phonetics}
\end{Entry}

\begin{Entry}{光环}{6,8}{⼉,⽟}
  \begin{Phonetics}{光环}{guang1huan2}[][HSK 7-9]
    \definition[道]{s.}{um anel de luz; matéria brilhante ao redor de alguns planetas | halo; auréola; o halo anular na cabeça de uma divindade | um halo colorido que às vezes aparece ao redor do sol ou da lua | glória; distinção; esplendor; metáfora para fama e honra}
  \end{Phonetics}
\end{Entry}

\begin{Entry}{光线}{6,8}{⼉,⽷}
  \begin{Phonetics}{光线}{guang1xian4}[][HSK 5]
    \definition[条,道]{s.}{luz; feixe luminoso; raio de luz}
  \end{Phonetics}
\end{Entry}

\begin{Entry}{光临}{6,9}{⼉,⼁}
  \begin{Phonetics}{光临}{guang1lin2}[][HSK 4]
    \definition{v.}{honrar com sua presença, uma palavra de honra, usada para dizer que um convidado chegou}
  \end{Phonetics}
\end{Entry}

\begin{Entry}{光荣}{6,9}{⼉,⾋}
  \begin{Phonetics}{光荣}{guang1rong2}[][HSK 5]
    \definition{adj.}{honroso; honrado; glorioso; por fazer algo que é benéfico para o país ou para a coletividade e que é considerado por todos como digno de respeito ou elogio}
    \definition{s.}{honra; glória; crédito; sentimento de honra decorrente do fato de ser respeitado ou elogiado por fazer algo importante ou grandioso}
  \end{Phonetics}
\end{Entry}

\begin{Entry}{光顾}{6,10}{⼉,⾴}
  \begin{Phonetics}{光顾}{guang1gu4}[][HSK 7-9]
    \definition{v.}{patrocinar; honrar com; uma palavra que demonstra respeito a alguém, referindo"-se à chegada de um convidado; restaurantes e lojas costumam usá"-la para dar as boas"-vindas aos clientes; também é usada de forma metafórica e irônica}
  \end{Phonetics}
\end{Entry}

\begin{Entry}{光彩}{6,11}{⼉,⼺}
  \begin{Phonetics}{光彩}{guang1cai3}[][HSK 7-9]
    \definition{adj.}{glorioso; honroso; decente}
    \definition{s.}{brilho; esplendor; radiância}
  \end{Phonetics}
\end{Entry}

\begin{Entry}{光盘}{6,11}{⼉,⽫}
  \begin{Phonetics}{光盘}{guang1pan2}[][HSK 4]
    \definition[张,套,片]{s.}{CD; disco compacto; um disco circular feito de plástico rígido composto que usa um laser para registrar e ler informações}
  \end{Phonetics}
\end{Entry}

\begin{Entry}{光滑}{6,12}{⼉,⽔}
  \begin{Phonetics}{光滑}{guang1hua2}[][HSK 7-9]
    \definition{adj.}{liso; suave; brilhante}
  \end{Phonetics}
\end{Entry}

\begin{Entry}{光缆}{6,12}{⼉,⽷}
  \begin{Phonetics}{光缆}{guang1lan3}[][HSK 7-9]
    \definition[根,条]{s.}{cabo óptico; cabo de fibra óptica}
  \end{Phonetics}
\end{Entry}

\begin{Entry}{光辉}{6,12}{⼉,⾞}
  \begin{Phonetics}{光辉}{guang1hui1}[][HSK 6]
    \definition{adj.}{brilhante; magnífico; glorioso}
    \definition{s.}{esplendor; brilho; glória | chama; brilho; halo; labareda; fulguração; lustre}
  \end{Phonetics}
\end{Entry}

\begin{Entry}{光槃}{6,14}{⼉,⽊}
  \begin{Phonetics}{光槃}{guang1pan2}
    \variantof{光盘}
  \end{Phonetics}
\end{Entry}

\begin{Entry}{光碟}{6,14}{⼉,⽯}
  \begin{Phonetics}{光碟}{guang1die2}[][HSK 7-9]
    \definition[个,片,张]{s.}{disco compacto (CD); videodisco; CD; CD-ROM; disco ótico}
  \end{Phonetics}
\end{Entry}

%%%%%%%%%% 全 %%%%%%%%%%
\subsection*{全}\addcontentsline{loh}{figure}{全}

\begin{Entry}{全}{6}{⼊}
  \begin{Phonetics}{全}{quan2}[][HSK 2]
    \definition*{s.}{Sobrenome: Quan}
    \definition{adj.}{completo; total; inteiro}
    \definition{adv.}{inteiramente; totalmente; completamente; significa 100\%; equivalente a 完全 ou 全然}
    \definition{v.}{manter intacto; tornar perfeito ou completo; completar}
  \seealsoref{全然}{quan2ran2}
  \seealsoref{完全}{wan2quan2}
  \end{Phonetics}
\end{Entry}

\begin{Entry}{全力}{6,2}{⼊,⼒}
  \begin{Phonetics}{全力}{quan2li4}[][HSK 6]
    \definition{s.}{exercendo todos os seus esforços; energia ou força total; toda força ou energia}
  \end{Phonetics}
\end{Entry}

\begin{Entry}{全力以赴}{6,2,4,9}{⼊,⼒,⼈,⾛}
  \begin{Phonetics}{全力以赴}{quan2li4yi3fu4}[][HSK 7-9]
    \definition{expr.}{``Dê tudo de si.''; fazer a todo custo; dar o máximo de si; prosseguir; dedicar todas as suas forças a algo}
  \end{Phonetics}
\end{Entry}

\begin{Entry}{全心全意}{6,4,6,13}{⼊,⼼,⼊,⼼}
  \begin{Phonetics}{全心全意}{quan2xin1-quan2yi4}[][HSK 7-9]
    \definition{expr.}{``De todo o coração.''; dedicar-se de corpo e alma a; de corpo e alma; com todo o coração}
  \end{Phonetics}
\end{Entry}

\begin{Entry}{全文}{6,4}{⼊,⽂}
  \begin{Phonetics}{全文}{quan2wen2}[][HSK 7-9]
    \definition{s.}{texto completo}
  \end{Phonetics}
\end{Entry}

\begin{Entry}{全方位}{6,4,7}{⼊,⽅,⼈}
  \begin{Phonetics}{全方位}{quan2fang1wei4}[][HSK 7-9]
    \definition{adj.}{versátil | em todo o redor | completo | abrangente | holístico | omnidirecional}
  \end{Phonetics}
\end{Entry}

\begin{Entry}{全长}{6,4}{⼊,⾧}
  \begin{Phonetics}{全长}{quan2chang2}[][HSK 7-9]
    \definition{s.}{comprimento total | extensão; alcance}
  \end{Phonetics}
\end{Entry}

\begin{Entry}{全世界}{6,5,9}{⼊,⼀,⽥}
  \begin{Phonetics}{全世界}{quan2shi4jie4}[][HSK 5]
    \definition[种]{s.}{mundo inteiro; mundo todo | em todo o mundo}
  \end{Phonetics}
\end{Entry}

\begin{Entry}{全场}{6,6}{⼊,⼟}
  \begin{Phonetics}{全场}{quan2chang3}[][HSK 3]
    \definition{s.}{toda a audiência; todos os presentes; todo o público}
  \end{Phonetics}
\end{Entry}

\begin{Entry}{全年}{6,6}{⼊,⼲}
  \begin{Phonetics}{全年}{quan2nian2}[][HSK 2]
    \definition{s.}{ano inteiro | anual; todo ano}
  \end{Phonetics}
\end{Entry}

\begin{Entry}{全体}{6,7}{⼊,⼈}
  \begin{Phonetics}{全体}{quan2ti3}[][HSK 2]
    \definition{s.}{(frequentemente referido a pessoas) todos; número total; todos | por todo o corpo | todos; inteiro; a soma de todas as partes; a soma de todos os indivíduos (geralmente se refere a pessoas)}
  \end{Phonetics}
\end{Entry}

\begin{Entry}{全局}{6,7}{⼊,⼫}
  \begin{Phonetics}{全局}{quan2ju2}[][HSK 7-9]
    \definition{s.}{situação geral; situação como um todo}
  \end{Phonetics}
\end{Entry}

\begin{Entry}{全身}{6,7}{⼊,⾝}
  \begin{Phonetics}{全身}{quan2shen1}[][HSK 2]
    \definition{s.}{corpo inteiro; por todo o corpo; todo o corpo}
  \end{Phonetics}
\end{Entry}

\begin{Entry}{全国}{6,8}{⼊,⼞}
  \begin{Phonetics}{全国}{quan2guo2}[][HSK 2]
    \definition{s.}{toda a nação (ou país); em todo o país; em todo o território nacional | toda a nação; todo o país}
  \end{Phonetics}
\end{Entry}

\begin{Entry}{全面}{6,9}{⼊,⾯}
  \begin{Phonetics}{全面}{quan2mian4}[][HSK 3]
    \definition{adj.}{geral; completo; abrangente; onipotente}
    \definition{s.}{todos os aspectos; cada aspecto}
  \seealsoref{片面}{pian4mian4}
  \end{Phonetics}
\end{Entry}

\begin{Entry}{全家}{6,10}{⼊,⼧}
  \begin{Phonetics}{全家}{quan2jia1}[][HSK 2]
    \definition{s.}{toda a família; a família inteira}
  \end{Phonetics}
\end{Entry}

\begin{Entry}{全称特命全权大使}{6,10,10,8,6,6,3,8}{⼊,⽲,⽜,⼝,⼊,⽊,⼤,⼈}
  \begin{Phonetics}{全称特命全权大使}{quan2cheng1 te4ming4 quan2quan2 da4shi3}
    \definition*{s.}{Embaixador Extraordinário e Plenipotenciário}
  \end{Phonetics}
\end{Entry}

\begin{Entry}{全能}{6,10}{⼊,⾁}
  \begin{Phonetics}{全能}{quan2neng2}[][HSK 7-9]
    \definition{adj.}{todo-poderoso; onipotente | Esporte: versátil | pluripotente}
  \end{Phonetics}
\end{Entry}

\begin{Entry}{全部}{6,10}{⼊,⾢}
  \begin{Phonetics}{全部}{quan2bu4}[][HSK 2]
    \definition{adv.}{tudo; total; inteiro; completo; aplica-se a todos, sem exceção}
    \definition{s.}{totalidade; total; integridade; a soma de todas as partes; o todo}
  \end{Phonetics}
\end{Entry}

\begin{Entry}{全都}{6,10}{⼊,⾢}
  \begin{Phonetics}{全都}{quan2dou1}[][HSK 5]
    \definition{adv.}{tudo; todos; sem exceção}
  \end{Phonetics}
\end{Entry}

\begin{Entry}{全都不}{6,10,4}{⼊,⾢,⼀}
  \begin{Phonetics}{全都不}{quan2dou1 bu4}
    \definition{adj.}{nada; nenhum; nenhum deles; nada disso}
  \end{Phonetics}
\end{Entry}

\begin{Entry}{全球}{6,11}{⼊,⽟}
  \begin{Phonetics}{全球}{quan2qiu2}[][HSK 3]
    \definition[门]{s.}{o mundo inteiro; a Terra inteira}
  \end{Phonetics}
\end{Entry}

\begin{Entry}{全职}{6,11}{⼊,⽿}
  \begin{Phonetics}{全职}{quan2zhi2}
    \definition{s.}{período integral | tempo inteiro | (trabalho) \emph{full-time}}
  \end{Phonetics}
\end{Entry}

\begin{Entry}{全然}{6,12}{⼊,⽕}
  \begin{Phonetics}{全然}{quan2ran2}
    \definition{adv.}{completamente; inteiramente}
  \end{Phonetics}
\end{Entry}

\begin{Entry}{全程}{6,12}{⼊,⽲}
  \begin{Phonetics}{全程}{quan2cheng2}[][HSK 7-9]
    \definition{s.}{toda a jornada; todo o percurso}
  \end{Phonetics}
\end{Entry}

\begin{Entry}{全新}{6,13}{⼊,⽄}
  \begin{Phonetics}{全新}{quan2xin1}[][HSK 6]
    \definition{adj.}{totalmente novo; inteiramente/completamente novo; refere"-se a algo completamente novo, especialmente algo que não foi usado}
  \end{Phonetics}
\end{Entry}

%%%%%%%%%% 共 %%%%%%%%%%
\subsection*{共}\addcontentsline{loh}{figure}{共}

\begin{Entry}{共}{6}{⼋}
  \begin{Phonetics}{共}{gong4}[][HSK 4]
    \definition*{s.}{Partido Comunista, abreviação de 共产党 | Sobrenome: Gong}
    \definition{adj.}{conjunto; mútuo; geral; comum; o mesmo para todos}
    \definition{adv.}{juntos; juntamente; conjuntamente | em sua totalidade; em todos}
    \definition{v.}{compartilhar com; empreender ou realizar em conjunto}
  \seealsoref{共产党}{gong4chan3dang3}
  \end{Phonetics}
\end{Entry}

\begin{Entry}{共计}{6,4}{⼋,⾔}
  \begin{Phonetics}{共计}{gong4ji4}[][HSK 5]
    \definition{s.}{total; total geral; agregado; montante}
    \definition{v.}{contar até; somar até; totalizar}
  \end{Phonetics}
\end{Entry}

\begin{Entry}{共产}{6,6}{⼋,⼇}
  \begin{Phonetics}{共产}{gong4chan3}
    \definition{adj.}{comunista}
    \definition{s.}{comunismo}
  \end{Phonetics}
\end{Entry}

\begin{Entry}{共产主义}{6,6,5,3}{⼋,⼇,⼂,⼂}
  \begin{Phonetics}{共产主义}{gong4chan3 zhu3yi4}
    \definition*{s.}{Comunismo}
  \end{Phonetics}
\end{Entry}

\begin{Entry}{共产党}{6,6,10}{⼋,⼇,⼉}
  \begin{Phonetics}{共产党}{gong4chan3dang3}
    \definition*{s.}{Partido Comunista}
  \end{Phonetics}
\end{Entry}

\begin{Entry}{共同}{6,6}{⼋,⼝}
  \begin{Phonetics}{共同}{gong4tong2}[][HSK 3]
    \definition{adj.}{comum; compartilhado; colaborativo; todos têm}
    \definition{adv.}{juntos; conjuntamente; todos juntos (fazemos)}
  \end{Phonetics}
\end{Entry}

\begin{Entry}{共同体}{6,6,7}{⼋,⼝,⼈}
  \begin{Phonetics}{共同体}{gong4tong2ti3}[][HSK 7-9]
    \definition[个]{s.}{comunidade}[欧洲经济共同体===Comunidade Econômica Europeia]
  \end{Phonetics}
\end{Entry}

\begin{Entry}{共有}{6,6}{⼋,⽉}
  \begin{Phonetics}{共有}{gong4you3}[][HSK 3]
    \definition{v.}{compartilhar; possuir (por todos); possuir ou desfrutar em conjunto}
  \end{Phonetics}
\end{Entry}

\begin{Entry}{共时}{6,7}{⼋,⽇}
  \begin{Phonetics}{共时}{gong4shi2}
    \definition{adj.}{sincrônico; simultâneo}
  \antonymref{历时}{li4shi2}
  \end{Phonetics}
\end{Entry}

\begin{Entry}{共识}{6,7}{⼋,⾔}
  \begin{Phonetics}{共识}{gong4shi2}[][HSK 7-9]
    \definition{s.}{consenso; entendimento comum}
  \end{Phonetics}
\end{Entry}

\begin{Entry}{共享}{6,8}{⼋,⼇}
  \begin{Phonetics}{共享}{gong4xiang3}[][HSK 5]
    \definition{v.}{compartilhar; desfrutar juntos; aproveitar as coisas boas juntos}
  \end{Phonetics}
\end{Entry}

\begin{Entry}{共性}{6,8}{⼋,⼼}
  \begin{Phonetics}{共性}{gong4xing4}[][HSK 7-9]
    \definition{s.}{caráter geral (comum); natureza comum; generalidade; semelhança; universalidade}
  \end{Phonetics}
\end{Entry}

\begin{Entry}{共鸣}{6,8}{⼋,⿃}
  \begin{Phonetics}{共鸣}{gong4ming2}[][HSK 7-9]
    \definition{s.}{ressonância; fenômeno que ocorre quando um objeto ressoa, por exemplo, quando dois diapasões com a mesma frequência são colocados próximos um do outro, quando um vibra e emite um som, o outro também emite um som | resposta simpática; uma metáfora para ter as mesmas emoções que outra pessoa}
  \end{Phonetics}
\end{Entry}

%%%%%%%%%% 兲 %%%%%%%%%%
\subsection*{兲}\addcontentsline{loh}{figure}{兲}

\begin{Entry}{兲}{6}{⼋}
  \begin{Phonetics}{兲}{tian1}
    \variantof{天}
  \end{Phonetics}
\end{Entry}

%%%%%%%%%% 关 %%%%%%%%%%
\subsection*{关}\addcontentsline{loh}{figure}{关}

\begin{Entry}{关}{6}{⼋}
  \begin{Phonetics}{关}{guan1}[][HSK 1,4]
    \definition*{s.}{Sobrenome: Guan}
    \definition{s.}{passagem; ponto de controle | alfândega; escritórios de cobrança de impostos para exportação e importação de mercadorias | ponto de inflexão ou barreira; ponto de virada ou dificuldade | momento crítico; mecanismo}
    \definition{v.}{fechar; encerrar; amarrar algo | fechar; trancar | encerrar; sair do mercado; falir | conceder ou sacar o pagamento de um salário | desligar | envolver; preocupar-se; conectar-se}
  \end{Phonetics}
\end{Entry}

\begin{Entry}{关上}{6,3}{⼋,⼀}
  \begin{Phonetics}{关上}{guan1shang4}[][HSK 1]
    \definition{v.}{fechar (uma porta); fechar um objeto | desligar (luz, equipamento elétrico etc.); parar ou encerrar (uma atividade, situação, etc.)}
  \end{Phonetics}
\end{Entry}

\begin{Entry}{关于}{6,3}{⼋,⼆}
  \begin{Phonetics}{关于}{guan1yu2}[][HSK 4]
    \definition{prep.}{sobre; relativo a; pertencente a; uma questão de; com relação a}
  \end{Phonetics}
\end{Entry}

\begin{Entry}{关心}{6,4}{⼋,⼼}
  \begin{Phonetics}{关心}{guan1xin1}[][HSK 2]
    \definition{v.}{cuidar; preocupar-se com; manifestar interesse por; demonstrar solicitude por; (colocar uma pessoa ou coisa) sempre no coração; valorizar e cuidar}
  \end{Phonetics}
\end{Entry}

\begin{Entry}{关头}{6,5}{⼋,⼤}
  \begin{Phonetics}{关头}{guan1tou2}[][HSK 7-9]
    \definition{s.}{conjuntura; momento; um momento decisivo ou ponto de virada}
  \end{Phonetics}
\end{Entry}

\begin{Entry}{关节}{6,5}{⼋,⾋}
  \begin{Phonetics}{关节}{guan1jie2}[][HSK 7-9]
    \definition{s.}{articulação; as partes onde os ossos se conectam e que possibilitam o movimento | suborno; relacionamentos que podem ajudar as pessoas a obter benefícios por meios impróprios | elo (ou ponto) chave (ou crucial)}
  \end{Phonetics}
\end{Entry}

\begin{Entry}{关机}{6,6}{⼋,⽊}
  \begin{Phonetics}{关机}{guan1 ji1}[][HSK 2]
    \definition{v.}{encerrar; terminar; refere"-se especificamente à conclusão das filmagens de um filme ou série de TV | desligar; desligar a fonte de alimentação; parar o funcionamento da máquina}
  \end{Phonetics}
\end{Entry}

\begin{Entry}{关闭}{6,6}{⼋,⾨}
  \begin{Phonetics}{关闭}{guan1bi4}[][HSK 4]
    \definition{v.}{fechar | (empresa) falir}
  \end{Phonetics}
\end{Entry}

\begin{Entry}{关张}{6,7}{⼋,⼸}
  \begin{Phonetics}{关张}{guan1zhang1}
    \definition{v.}{Dialeto: (uma loja) fechar as portas; falir}
  \end{Phonetics}
\end{Entry}

\begin{Entry}{关怀}{6,7}{⼋,⼼}
  \begin{Phonetics}{关怀}{guan1huai2}[][HSK 5]
    \definition{v.}{mostrar cuidado amoroso por; mostrar solicitude por; cuidar, amar, apoiar ou ajudar os fracos ou grupos em dificuldade | geralmente usado para superiores para subordinados, anciãos para juniores ou organizações para indivíduos}
  \end{Phonetics}
\end{Entry}

\begin{Entry}{关系}{6,7}{⼋,⽷}
  \begin{Phonetics}{关系}{guan1xi5}[][HSK 3]
    \definition[个,种]{s.}{relações; conexões; relacionamento; a interligação entre pessoas ou coisas | consequência; impacto; significado a influência ou importância de algo; algo digno de nota (geralmente usado com 没有, 有). | causa; razão (geralmente usado com 由于 ou 因为); refere"-se genericamente a causas, condições, etc. | credenciais que mostram filiação a uma organização; documento que comprova a existência de algum tipo de relação organizacional}
    \definition{v.}{preocupar; afetar; ter influência sobre; ter a ver com}
  \seealsoref{没有}{mei2you5}
  \seealsoref{因为}{yin1wei5}
  \seealsoref{由于}{you2yu2}
  \seealsoref{有}{you3}
  \end{Phonetics}
\end{Entry}

\begin{Entry}{关注}{6,8}{⼋,⽔}
  \begin{Phonetics}{关注}{guan1zhu4}[][HSK 3]
    \definition{v.}{prestar atenção em; seguir algo de perto; seguir (nas redes sociais)}
  \end{Phonetics}
\end{Entry}

\begin{Entry}{关爱}{6,10}{⼋,⽖}
  \begin{Phonetics}{关爱}{guan1'ai4}[][HSK 6]
    \definition{v.}{cuidar; cuidar e amar}
  \end{Phonetics}
\end{Entry}

\begin{Entry}{关掉}{6,11}{⼋,⼿}
  \begin{Phonetics}{关掉}{guan1diao4}[][HSK 7-9]
    \definition{v.}{desligar}
  \end{Phonetics}
\end{Entry}

\begin{Entry}{关税}{6,12}{⼋,⽲}
  \begin{Phonetics}{关税}{guan1shui4}[][HSK 7-9]
    \definition{s.}{tarifa; taxa aduaneira; impostos cobrados pelo estado sobre mercadorias importadas e exportadas}
  \end{Phonetics}
\end{Entry}

\begin{Entry}{关联}{6,12}{⼋,⽿}
  \begin{Phonetics}{关联}{guan1lian2}[][HSK 6]
    \definition{s.}{conexão; inter-relação; a conexão entre as coisas}
    \definition{v.}{estar relacionado; estar conectado; as coisas estão envolvidas e influenciam umas às outras}
  \end{Phonetics}
\end{Entry}

\begin{Entry}{关照}{6,13}{⼋,⽕}
  \begin{Phonetics}{关照}{guan1zhao4}[][HSK 7-9]
    \definition{v.}{cuidar de; ficar de olho em; preocupar-se e cuidar de alguém e tomar a iniciativa de ajudar quando perceber que essa pessoa está com problemas | contar; notificar de boca em boca; notificação verbal para que as pessoas saibam ou se lembrem de algo}
  \end{Phonetics}
\end{Entry}

\begin{Entry}{关键}{6,13}{⼋,⾦}
  \begin{Phonetics}{关键}{guan1jian4}[][HSK 5]
    \definition{adj.}{crucial; decisivo; importante; que pode determinar o curso e o resultado dos eventos}
    \definition[个,点,些]{s.}{chave; ponto crucial; aspectos ou condições mais importantes que determinam o desenvolvimento e o resultado de algo}
  \end{Phonetics}
\end{Entry}

%%%%%%%%%% 兴 %%%%%%%%%%
\subsection*{兴}\addcontentsline{loh}{figure}{兴}

\begin{Entry}{兴}{6}{⼋}
  \begin{Phonetics}{兴}{xing1}
    \definition*{s.}{Sobrenome: Xing}
    \definition{adj.}{próspero; florescente}
    \definition{adv.}{Dialeto: talvez}
    \definition{v.}{ascender; prosperar; prevalecer; tornar-se popular | promover; encorajar; fazer prevalecer | começar; iniciar; lançar; mobilizar | erguer-se; levantar-se | (usualmente no negativo) permitir; deixar}
  \end{Phonetics}
  \begin{Phonetics}{兴}{xing4}
    \definition{s.}{sentimento ou desejo de fazer algo | interesse em algo | excitação}
  \end{Phonetics}
\end{Entry}

\begin{Entry}{兴奋}{6,8}{⼋,⼤}
  \begin{Phonetics}{兴奋}{xing1fen4}[][HSK 4]
    \definition{adj.}{animado; excitante; empolgante;}
    \definition{s.}{excitação; empolgação}
    \definition{v.}{excitar; intoxicar}
  \end{Phonetics}
\end{Entry}

\begin{Entry}{兴旺}{6,8}{⼋,⽇}
  \begin{Phonetics}{兴旺}{xing1wang4}[][HSK 6]
    \definition{adj.}{próspero; propício; favorável; auspicioso}
  \end{Phonetics}
\end{Entry}

\begin{Entry}{兴趣}{6,15}{⼋,⾛}
  \begin{Phonetics}{兴趣}{xing4qu4}[][HSK 4]
    \definition[个,种,点,股,份]{s.}{interesse (desejo de conhecer sobre alguma coisa ou coisa no qual está interessado) | \emph{hobby}}
  \end{Phonetics}
\end{Entry}

%%%%%%%%%% 再 %%%%%%%%%%
\subsection*{再}\addcontentsline{loh}{figure}{再}

\begin{Entry}{再}{6}{⼌}
  \begin{Phonetics}{再}{zai4}[][HSK 1]
    \definition{adv.}{mais uma vez; além disso; ainda mais; indica a repetição ou continuação de uma mesma ação ou comportamento; refere"-se principalmente a ações ou comportamentos não realizados ou contínuos | usado antes do adjetivo, indica intensificação, equivalente a 更 ou 更加 | (para uma ação adiada, precedida por uma expressão de tempo ou condição) então; somente então; depois de algo; indica que a ação ocorrerá após a conclusão de outra ação | além disso; indica um complemento, equivalente a 另外 ou 又 | próxima vez; indica que a ação ocorrerá após um determinado período de tempo | novamente; de novo}
  \seealsoref{更}{geng4}
  \seealsoref{更加}{geng4jia1}
  \seealsoref{另外}{ling4wai4}
  \seealsoref{又}{you4}
  \end{Phonetics}
\end{Entry}

\begin{Entry}{再三}{6,3}{⼌,⼀}
  \begin{Phonetics}{再三}{zai4san1}[][HSK 4]
    \definition{adv.}{repetidamente; repetidas vezes; de novo e de novo}
  \end{Phonetics}
\end{Entry}

\begin{Entry}{再也}{6,3}{⼌,⼄}
  \begin{Phonetics}{再也}{zai4ye3}[][HSK 5]
    \definition{adv.}{não mais; nunca mais; uma determinada situação ou ação nunca mais ocorrerá}
  \end{Phonetics}
\end{Entry}

\begin{Entry}{再不}{6,4}{⼌,⼀}
  \begin{Phonetics}{再不}{zai4bu4}
    \definition{adv.}{nunca mais}
  \end{Phonetics}
\end{Entry}

\begin{Entry}{再见}{6,4}{⼌,⾒}
  \begin{Phonetics}{再见}{zai4jian4}[][HSK 1]
    \definition{v.}{adeus; tchau; até logo; até mais; até mais tarde}
  \end{Phonetics}
\end{Entry}

\begin{Entry}{再发}{6,5}{⼌,⼜}
  \begin{Phonetics}{再发}{zai4 fa1}
    \definition{v.}{reenviar; reeditar}
  \end{Phonetics}
\end{Entry}

\begin{Entry}{再生}{6,5}{⼌,⽣}
  \begin{Phonetics}{再生}{zai4sheng1}[][HSK 6]
    \definition{v.}{reviver; ressuscitar; ressuscitar dos mortos | reproduzir; regenerar | reprocessar; reciclar; regenerar; processar um determinado produto residual para restaurar seu desempenho original e transformá-lo em um novo produto}
  \end{Phonetics}
\end{Entry}

\begin{Entry}{再次}{6,6}{⼌,⽋}
  \begin{Phonetics}{再次}{zai4ci4}[][HSK 5]
    \definition{adv.}{mais uma vez; uma segunda vez; outra vez}
  \end{Phonetics}
\end{Entry}

\begin{Entry}{再审}{6,8}{⼌,⼧}
  \begin{Phonetics}{再审}{zai4shen3}
    \definition{s.}{novo julgamento | revisão}
    \definition{v.}{ouvir um caso novamente}
  \end{Phonetics}
\end{Entry}

\begin{Entry}{再者}{6,8}{⼌,⽼}
  \begin{Phonetics}{再者}{zai4zhe3}
    \definition{conj.}{além do mais | além disso}
  \end{Phonetics}
\end{Entry}

\begin{Entry}{再育}{6,8}{⼌,⾁}
  \begin{Phonetics}{再育}{zai4yu4}
    \definition{v.}{aumentar | multiplicar | proliferar}
  \end{Phonetics}
\end{Entry}

\begin{Entry}{再临}{6,9}{⼌,⼁}
  \begin{Phonetics}{再临}{zai4lin2}
    \definition{v.}{vir de novo; voltar}
  \end{Phonetics}
\end{Entry}

\begin{Entry}{再度}{6,9}{⼌,⼴}
  \begin{Phonetics}{再度}{zai4du4}
    \definition{adv.}{outra vez | mais uma vez}
  \end{Phonetics}
\end{Entry}

\begin{Entry}{再说}{6,9}{⼌,⾔}
  \begin{Phonetics}{再说}{zai4shuo1}[][HSK 6]
    \definition{conj.}{além disso; o que é mais; indicando uma razão adicional, equivalente a 况且}
    \definition{v.}{adiar para mais tarde; deixar para processamento ou consideração posterior}
  \seealsoref{况且}{kuang4qie3}
  \end{Phonetics}
\end{Entry}

\begin{Entry}{再读}{6,10}{⼌,⾔}
  \begin{Phonetics}{再读}{zai4du2}
    \definition{v.}{ler novamente | rever (uma lição, etc.)}
  \end{Phonetics}
\end{Entry}

%%%%%%%%%% 军 %%%%%%%%%%
\subsection*{军}\addcontentsline{loh}{figure}{军}

\begin{Entry}{军}{6}{⼍}
  \begin{Phonetics}{军}{jun1}
    \definition*{s.}{Sobrenome: Jun}
    \definition{s.}{forças armadas; exército; tropas | exército; contingente; muitas pessoas participando de uma atividade | exército; unidades militares}
  \end{Phonetics}
\end{Entry}

\begin{Entry}{军人}{6,2}{⼍,⼈}
  \begin{Phonetics}{军人}{jun1ren2}[][HSK 5]
    \definition[名,位,个]{s.}{soldado; militar; pessoal militar; pessoas com status militar; pessoas servindo nas forças armadas}
  \end{Phonetics}
\end{Entry}

\begin{Entry}{军队}{6,4}{⼍,⾩}
  \begin{Phonetics}{军队}{jun1dui4}[][HSK 6]
    \definition[支,个]{s.}{forças armadas; exército; tropas}
  \end{Phonetics}
\end{Entry}

\begin{Entry}{军事}{6,8}{⼍,⼅}
  \begin{Phonetics}{军事}{jun1shi4}[][HSK 6]
    \definition{s.}{militar; assuntos militares; assuntos relativos aos militares e à guerra}
  \end{Phonetics}
\end{Entry}

\begin{Entry}{军官}{6,8}{⼍,⼧}
  \begin{Phonetics}{军官}{jun1guan1}[][HSK 7-9]
    \definition{s.}{oficial; militares com patente igual ou superior a tenente, também se refere a oficiais com patente igual ou superior a comandante de pelotão nas forças armadas}
  \end{Phonetics}
\end{Entry}

\begin{Entry}{军舰}{6,10}{⼍,⾈}
  \begin{Phonetics}{军舰}{jun1jian4}[][HSK 6]
    \definition[艘,只]{s.}{navio de guerra; embarcação naval | \emph{warcraft}; um termo geral para embarcações militares equipadas com armas e equipamentos que podem executar missões de combate, incluindo principalmente navios de guerra, cruzadores, contratorpedeiros, porta-aviões, submarinos, torpedeiros, etc.}
  \end{Phonetics}
\end{Entry}

\begin{Entry}{军装}{6,12}{⼍,⾐}
  \begin{Phonetics}{军装}{jun1zhuang1}
    \definition{s.}{uniforme militar}
  \end{Phonetics}
\end{Entry}

%%%%%%%%%% 农 %%%%%%%%%%
\subsection*{农}\addcontentsline{loh}{figure}{农}

\begin{Entry}{农}{6}{⼍}
  \begin{Phonetics}{农}{nong2}
    \definition*{s.}{Sobrenome: Nong}
    \definition{s.}{agricultura; criação de animais | camponês; fazendeiro}
  \end{Phonetics}
\end{Entry}

\begin{Entry}{农历}{6,4}{⼍,⼚}
  \begin{Phonetics}{农历}{nong2li4}[][HSK 7-9]
    \definition{s.}{o calendário lunar; o calendário tradicional chinês}
  \end{Phonetics}
\end{Entry}

\begin{Entry}{农业}{6,5}{⼍,⼀}
  \begin{Phonetics}{农业}{nong2ye4}[][HSK 3]
    \definition{s.}{agricultura}
  \end{Phonetics}
\end{Entry}

\begin{Entry}{农民}{6,5}{⼍,⽒}
  \begin{Phonetics}{农民}{nong2min2}[][HSK 3]
    \definition[个,位,名,些]{s.}{fazendeiro; camponês; campesinato; trabalhadores que participam da produção agrícola há muito tempo}
  \end{Phonetics}
\end{Entry}

\begin{Entry}{农民工}{6,5,3}{⼍,⽒,⼯}
  \begin{Phonetics}{农民工}{nong2min2gong1}[][HSK 7-9]
    \definition{s.}{trabalhadores migrantes; trabalhadores com registro de domicílio agrícola que exercem atividades não agrícolas em áreas urbanas}
  \end{Phonetics}
\end{Entry}

\begin{Entry}{农产品}{6,6,9}{⼍,⼇,⼝}
  \begin{Phonetics}{农产品}{nong2chan3pin3}[][HSK 5]
    \definition[批]{s.}{produtos agrícolas}
  \end{Phonetics}
\end{Entry}

\begin{Entry}{农场}{6,6}{⼍,⼟}
  \begin{Phonetics}{农场}{nong2chang3}[][HSK 7-9]
    \definition[座,家]{s.}{fazenda; empresas que utilizam máquinas e se dedicam à produção agrícola em larga escala}
  \end{Phonetics}
\end{Entry}

\begin{Entry}{农作物}{6,7,8}{⼍,⼈,⽜}
  \begin{Phonetics}{农作物}{nong2zuo4wu4}[][HSK 7-9]
    \definition{s.}{culturas agrícolas; plantas agrícolas; termo genérico para diversas culturas agrícolas, como grãos, oleaginosas, hortaliças, algodão e tabaco}
  \end{Phonetics}
\end{Entry}

\begin{Entry}{农村}{6,7}{⼍,⽊}
  \begin{Phonetics}{农村}{nong2cun1}[][HSK 3]
    \definition{s.}{aldeia; campo; área rural; locais onde vivem os trabalhadores principalmente dedicados à produção agrícola}
  \end{Phonetics}
\end{Entry}

%%%%%%%%%% 冰 %%%%%%%%%%
\subsection*{冰}\addcontentsline{loh}{figure}{冰}

\begin{Entry}{冰}{6}{⼎}
  \begin{Phonetics}{冰}{bing1}[][HSK 4]
    \definition[块,层,些]{s.}{gelo; água em estado sólido |  algo parecido com gelo | (gíria) metanfetamina}
    \definition{v.}{colocar gelo; colocar gelo ao redor; colocar no gelo; resfriar objetos com gelo ou água fria | sentir frio}
  \end{Phonetics}
\end{Entry}

\begin{Entry}{冰山}{6,3}{⼎,⼭}
  \begin{Phonetics}{冰山}{bing1shan1}[][HSK 7-9]
    \definition[座]{s.}{montanha gelada; montanha coberta de gelo | \emph{iceberg}; enormes blocos de gelo flutuando no mar | Figurativo: indivíduo ou grupo em que não se pode confiar por muito tempo; uma metáfora para um poder em que não se pode confiar por muito tempo}
  \end{Phonetics}
\end{Entry}

\begin{Entry}{冰天雪地}{6,4,11,6}{⼎,⼤,⾬,⼟}
  \begin{Phonetics}{冰天雪地}{bing1tian1-xue3di4}
    \definition{expr.}{um mundo de gelo e neve}
  \end{Phonetics}
\end{Entry}

\begin{Entry}{冰冷}{6,7}{⼎,⼎}
  \begin{Phonetics}{冰冷}{bing1leng3}
    \definition{adj.}{gelado; tão frio quanto gelo; muito frio | hostil; indiferente}
  \end{Phonetics}
\end{Entry}

\begin{Entry}{冰球}{6,11}{⼎,⽟}
  \begin{Phonetics}{冰球}{bing1qiu2}
    \definition[个]{s.}{hóquei no gelo | disco; a ``bola'' usada no hóquei no gelo}
  \end{Phonetics}
\end{Entry}

\begin{Entry}{冰雪}{6,11}{⼎,⾬}
  \begin{Phonetics}{冰雪}{bing1xue3}[][HSK 4]
    \definition{adj.}{puro como gelo e neve; descreve uma pessoa como pura}
    \definition[片,场]{s.}{gelo e neve}
  \end{Phonetics}
\end{Entry}

\begin{Entry}{冰棍}{6,12}{⼎,⽊}
  \begin{Phonetics}{冰棍}{bing1gun4}
    \definition[根,种,支]{s.}{picolé}
  \end{Phonetics}
\end{Entry}

\begin{Entry}{冰棍儿}{6,12,2}{⼎,⽊,⼉}
  \begin{Phonetics}{冰棍儿}{bing1gun4r5}[][HSK 7-9]
    \definition[根,种,支]{s.}{picolé; pirulito congelado}
  \end{Phonetics}
\end{Entry}

\begin{Entry}{冰箱}{6,15}{⼎,⾋}
  \begin{Phonetics}{冰箱}{bing1xiang1}[][HSK 4]
    \definition[台,个]{s.}{geladeira; freezer; refrigerador; aparelhos para congelar alimentos ou medicamentos com gelo para mantê-los frios}
  \end{Phonetics}
\end{Entry}

\begin{Entry}{冰激凌}{6,16,10}{⼎,⽔,⼎}
  \begin{Phonetics}{冰激凌}{bing1ji1ling2}
    \definition[份,盒,种,个]{s.}{sorvete}
  \end{Phonetics}
\end{Entry}

\begin{Entry}{冰糕}{6,16}{⼎,⽶}
  \begin{Phonetics}{冰糕}{bing1gao1}
    \definition{s.}{sorvete | picolé}
  \end{Phonetics}
\end{Entry}

%%%%%%%%%% 冲 %%%%%%%%%%
\subsection*{冲}\addcontentsline{loh}{figure}{冲}

\begin{Entry}{冲}{6}{⼎}
  \begin{Phonetics}{冲}{chong1}[][HSK 4]
    \definition{s.}{via pública; local importante; via de passagem; via local importante | um trecho de planície em uma área montanhosa | (astronomia) oposição; os planetas externos orbitam até ficarem alinhados com a Terra e o Sol, e a Terra está no meio}
    \definition{v.}{atacar; apressar; correr; passar rapidamente; passar por um obstáculo | colidir; chocar; bater | despejar água fervente sobre | enxaguar; dar descarga; lavar | revelar (filme) | neutralizar a má sorte}
  \end{Phonetics}
  \begin{Phonetics}{冲}{chong4}[][HSK 6]
    \definition{adj.}{poderoso; com vigor; com muita força; vigoroso | forte; odor forte e pungente (olfato)}
    \definition{prep.}{de frente; em direção a | na força de; com base em; em virtude de}
    \definition{v.}{estampar (máquina de estamparia)}
  \end{Phonetics}
\end{Entry}

\begin{Entry}{冲击}{6,5}{⼎,⼐}
  \begin{Phonetics}{冲击}{chong1ji1}[][HSK 6]
    \definition{v.}{chicotear; bater | correr; voar; atacar; assaltar; atacar bravamente em direção a um alvo predeterminado | chocar; metáfora para interferência ou golpe sério}
  \end{Phonetics}
\end{Entry}

\begin{Entry}{冲动}{6,6}{⼎,⼒}
  \begin{Phonetics}{冲动}{chong1dong4}[][HSK 5]
    \definition{adj.}{impulsivo; impetuoso}
    \definition{s.}{impulso; impetuosidade; impulso de movimento; fenômeno psicológico no qual as emoções são particularmente fortes e o controle racional é fraco}
    \definition{v.}{ficar animado; ser impetuoso; agir por impulso}
  \end{Phonetics}
\end{Entry}

\begin{Entry}{冲刺}{6,8}{⼎,⼑}
  \begin{Phonetics}{冲刺}{chong1ci4}[][HSK 7-9]
    \definition{v.}{arrancar; correr; disparar | metáfora para fazer o maior esforço ao se aproximar de uma meta ou estar prestes a ter sucesso}
  \end{Phonetics}
\end{Entry}

\begin{Entry}{冲洗}{6,9}{⼎,⽔}
  \begin{Phonetics}{冲洗}{chong1xi3}[][HSK 7-9]
    \definition{v.}{lavar; enxaguar | revelar filme; revelar e fixar o material fotossensível exposto}
  \end{Phonetics}
\end{Entry}

\begin{Entry}{冲突}{6,9}{⼎,⽳}
  \begin{Phonetics}{冲突}{chong1tu1}[][HSK 5]
    \definition{v.}{chocar"-se; entrar em conflito; conflitar | contradizer; duas coisas opostas que interferem uma na outra}
  \end{Phonetics}
\end{Entry}

\begin{Entry}{冲浪}{6,10}{⼎,⽔}
  \begin{Phonetics}{冲浪}{chong1lang4}[][HSK 7-9]
    \definition{v.}{surfar; um esporte aquático em que os atletas surfam em pranchas especialmente construídas e deslizam ao longo das ondas | metáfora para navegar na \emph{Internet}}
  \end{Phonetics}
\end{Entry}

\begin{Entry}{冲锋}{6,12}{⼎,⾦}
  \begin{Phonetics}{冲锋}{chong1feng1}
    \definition{v.}{cobrar | tomar de assalto}
  \end{Phonetics}
\end{Entry}

\begin{Entry}{冲撞}{6,15}{⼎,⼿}
  \begin{Phonetics}{冲撞}{chong1zhuang4}[][HSK 7-9]
    \definition{v.}{colidir; bater; sofrer impacto; voar; errar contra | ofender; ofender}
  \end{Phonetics}
\end{Entry}

%%%%%%%%%% 决 %%%%%%%%%%
\subsection*{决}\addcontentsline{loh}{figure}{决}

\begin{Entry}{决}{6}{⼎}
  \begin{Phonetics}{决}{jue2}
    \definition{v.}{decidir; determinar | executar uma pessoa | (de um dique, etc.) romper; desabar}
  \end{Phonetics}
\end{Entry}

\begin{Entry}{决不}{6,4}{⼎,⼀}
  \begin{Phonetics}{决不}{jue2bu4}[][HSK 5]
    \definition{adv.}{em hipótese alguma; nunca | definitivamente não; certamente não; sob nenhuma circunstância; de forma alguma}
  \end{Phonetics}
\end{Entry}

\begin{Entry}{决心}{6,4}{⼎,⼼}
  \begin{Phonetics}{决心}{jue2xin1}[][HSK 3]
    \definition{s.}{resolução; determinação; determinação inabalável}
    \definition{v.}{secidir-se; decidir fazer algo e não vacilar nem mudar de ideia}
  \end{Phonetics}
\end{Entry}

\begin{Entry}{决议}{6,5}{⼎,⾔}
  \begin{Phonetics}{决议}{jue2yi4}[][HSK 7-9]
    \definition[项]{s.}{resolução | resultado}
  \end{Phonetics}
\end{Entry}

\begin{Entry}{决定}{6,8}{⼎,⼧}
  \begin{Phonetics}{决定}{jue2ding4}[][HSK 3]
    \definition{adj.}{decisivo; as leis objetivas levam as coisas a se desenvolverem e mudarem em determinada direção}
    \definition[项,个]{s.}{decisão; resolução; assuntos decididos}
    \definition{v.}{decidir; determinar; algo se torna a base ou o pré-requisito para outra coisa; desempenha um papel dominante | decidir; resolver; tomar uma decisão; propor uma forma de agir}
  \end{Phonetics}
\end{Entry}

\begin{Entry}{决策}{6,12}{⼎,⽵}
  \begin{Phonetics}{决策}{jue2ce4}[][HSK 6]
    \definition{s.}{decisão política; decisão de importância estratégica; estratégia ou método de decisão}
    \definition{v.}{formular políticas; tomar uma decisão estratégica; decidir sobre uma estratégia ou abordagem}
  \end{Phonetics}
\end{Entry}

\begin{Entry}{决赛}{6,14}{⼎,⾙}
  \begin{Phonetics}{决赛}{jue2sai4}[][HSK 3]
    \definition[场]{s.}{finais (de uma competição); em competições esportivas, a última partida ou rodada disputada para determinar a classificação}
  \end{Phonetics}
\end{Entry}

%%%%%%%%%% 划 %%%%%%%%%%
\subsection*{划}\addcontentsline{loh}{figure}{划}

\begin{Entry}{划}{6}{⼑}
  \begin{Phonetics}{划}{hua2}[][HSK 4]
    \definition{adj.}{rentável; vale (o esforço); compensa (fazer alguma coisa)}
    \definition{v.}{remar | ser vantajoso para alguém; ser uma pechincha | arranhar; cortar a superfície de; cortar em outra coisa com um objeto pontiagudo | arranhar; golpear;  esfregar uma coisa ou varrer sobre outra}
  \end{Phonetics}
  \begin{Phonetics}{划}{hua4}[][HSK 4]
    \definition*{s.}{Sobrenome: Hua}
    \definition{s.}{traço de um caracter chinês}
    \definition{v.}{delimitar; diferenciar; delinear | transferir; ceder | planejar; programar | desenhar; marcar; delinear; fazer linhas ou escrever como marcadores com uma caneta ou objeto semelhante a uma caneta}
  \end{Phonetics}
\end{Entry}

\begin{Entry}{划分}{6,4}{⼑,⼑}
  \begin{Phonetics}{划分}{hua4fen1}[][HSK 5]
    \definition{v.}{dividir; particionar; reparticionar | diferenciar; encontrar aspectos diferentes}
  \end{Phonetics}
\end{Entry}

\begin{Entry}{划时代}{6,7,5}{⼑,⽇,⼈}
  \begin{Phonetics}{划时代}{hua4shi2dai4}[][HSK 7-9]
    \definition{adj.}{marcando uma nova época; marcando época}[具有划时代的意义。===É de importância histórica.]
  \end{Phonetics}
\end{Entry}

\begin{Entry}{划拳}{6,10}{⼑,⼿}
  \begin{Phonetics}{划拳}{hua2quan2}
    \definition{pron.}{jogo de adivinhação de dedos; ao beber, duas pessoas levantam os dedos e dizem um número, quem disser o número que corresponde ao total de dedos ganha, o perdedor bebe}
    \definition{v.}{jogar o jogo de adivinhação de dedos (jogado em um jantar por duas pessoas)}
  \end{Phonetics}
\end{Entry}

\begin{Entry}{划船}{6,11}{⼑,⾈}
  \begin{Phonetics}{划船}{hua2 chuan2}[][HSK 3]
    \definition[次,回]{s.}{remo (ato de remar); passeios de barco; a atividade ou esporte de ``remar um barco com remos''}
    \definition{v.}{remar um barco; a ação ou comportamento de mover um barco na água usando remos}
  \end{Phonetics}
\end{Entry}

\begin{Entry}{划艇}{6,12}{⼑,⾈}
  \begin{Phonetics}{划艇}{hua2ting3}
    \definition{s.}{barco a remo}
  \end{Phonetics}
\end{Entry}

\begin{Entry}{划算}{6,14}{⼑,⽵}
  \begin{Phonetics}{划算}{hua2suan4}[][HSK 7-9]
    \definition{adj.}{lucrativo; que vale a pena}
    \definition{v.}{calcular; pesar; planejar e esquematizar}
  \end{Phonetics}
\end{Entry}

%%%%%%%%%% 列 %%%%%%%%%%
\subsection*{列}\addcontentsline{loh}{figure}{列}

\begin{Entry}{列}{6}{⼑}
  \begin{Phonetics}{列}{lie4}[][HSK 4]
    \definition*{s.}{Sobrenome: Lie}
    \definition{clas.}{usado para coisas em linhas e colunas}
    \definition{pron.}{cada um e todos; cada; muito}
    \definition{s.}{linha; arquivo; classificação | classificação; escopo | ranque | tipo}
    \definition{v.}{organizar; alinhar; colocar em ordem | listar; inserir em uma lista; classificar | formar uma linha}
  \antonymref{行}{hang2}
  \end{Phonetics}
\end{Entry}

\begin{Entry}{列入}{6,2}{⼑,⼊}
  \begin{Phonetics}{列入}{lie4ru4}[][HSK 4]
    \definition{v.}{listar; entrar em uma lista; ser incluído em | incluir em uma lista; juntar-se; registrar-se}
  \end{Phonetics}
\end{Entry}

\begin{Entry}{列为}{6,4}{⼑,⼂}
  \begin{Phonetics}{列为}{lie4wei2}[][HSK 4]
    \definition{v.}{ser classificado como; ser listado como}
  \end{Phonetics}
\end{Entry}

\begin{Entry}{列车}{6,4}{⼑,⾞}
  \begin{Phonetics}{列车}{lie4che1}[][HSK 4]
    \definition[列,班,趟,辆,节]{s.}{trem; trem em uma composição contínua, puxado por uma locomotiva e equipado com uma tripulação e marcações prescritas; geralmente um trem de passageiros}
  \end{Phonetics}
\end{Entry}

\begin{Entry}{列举}{6,9}{⼑,⼂}
  \begin{Phonetics}{列举}{lie4ju3}[][HSK 7-9]
    \definition{v.}{listar; enumerar; listar um por um ou item por item}
  \end{Phonetics}
\end{Entry}

%%%%%%%%%% 刘 %%%%%%%%%%
\subsection*{刘}\addcontentsline{loh}{figure}{刘}

\begin{Entry}{刘}{6}{⼑}
  \begin{Phonetics}{刘}{liu2}
    \definition*{s.}{Sobrenome: Liu}
    \definition{s.}{Clássico: um tipo de machado de batalha}
    \definition{v.}{matar; massacrar}
  \end{Phonetics}
\end{Entry}

%%%%%%%%%% 刚 %%%%%%%%%%
\subsection*{刚}\addcontentsline{loh}{figure}{刚}

\begin{Entry}{刚}{6}{⼑}
  \begin{Phonetics}{刚}{gang1}[][HSK 2]
    \definition*{s.}{Sobrenome: Gang}
    \definition{adj.}{duro; firme; rígido; forte; (personalidade, atitude) forte; (vontade) determinada}
    \definition{adv.}{apenas; exatamente; justamente | apenas; apenas por pouco; significa atingir um certo nível com dificuldade | apenas; há pouco tempo; indica que a ação ou situação ocorreu há pouco tempo | assim que; somente neste momento; aconteceu que; use a palavra 就 para indicar que duas coisas estão intimamente relacionadas}
  \seealsoref{就}{jiu4}
  \end{Phonetics}
\end{Entry}

\begin{Entry}{刚才}{6,3}{⼑,⼿}
  \begin{Phonetics}{刚才}{gang1cai2}[][HSK 2]
    \definition{s.}{agora mesmo; há pouco; há pouco tempo; referindo"-se ao período recente que acabou de passar}
  \end{Phonetics}
\end{Entry}

\begin{Entry}{刚刚}{6,6}{⼑,⼑}
  \begin{Phonetics}{刚刚}{gang1gang1}[][HSK 2]
    \definition{adv.}{apenas; somente; exatamente; refere"-se a algo que é adequado em termos de grau, quantidade, tempo, etc., nem mais nem menos, nem cedo nem tarde, atingindo um estado satisfatório ou que atende exatamente às necessidades | agora mesmo; há pouco; há um momento atrás; referindo"-se a um período de tempo muito curto no passado}
  \end{Phonetics}
\end{Entry}

\begin{Entry}{刚好}{6,6}{⼑,⼥}
  \begin{Phonetics}{刚好}{gang1hao3}[][HSK 6]
    \definition{adj.}{apropriado; na medida certa}
    \definition{adv.}{apenas; acontece que; por acaso}
  \end{Phonetics}
\end{Entry}

\begin{Entry}{刚毅}{6,15}{⼑,⽎}
  \begin{Phonetics}{刚毅}{gang1yi4}[][HSK 7-9]
    \definition{adj.}{resoluto e firme | resoluto | robusto | firme}
  \end{Phonetics}
\end{Entry}

%%%%%%%%%% 创 %%%%%%%%%%
\subsection*{创}\addcontentsline{loh}{figure}{创}

\begin{Entry}{创}{6}{⼑}
  \begin{Phonetics}{创}{chuang1}
    \definition{s.}{ferimento; trauma}
  \end{Phonetics}
  \begin{Phonetics}{创}{chuang4}[][HSK 7-9]
    \definition{v.}{começar (fazer algo); alcançar (algo pela primeira vez); estabelecer; fazer pela primeira vez | estabelecer; fundar; criar; perceber algo novo, como um começo | ferir; machucar}
  \end{Phonetics}
\end{Entry}

\begin{Entry}{创办}{6,4}{⼑,⼒}
  \begin{Phonetics}{创办}{chuang4ban4}[][HSK 6]
    \definition{v.}{estabelecer; montar; fundar}
  \end{Phonetics}
\end{Entry}

\begin{Entry}{创业}{6,5}{⼑,⼀}
  \begin{Phonetics}{创业}{chuang4ye4}[][HSK 3]
    \definition{s.}{empreendedorismo}
    \definition{v.}{começar um empreendimento; iniciar/fundar um negócio, uma empresa;}
  \end{Phonetics}
\end{Entry}

\begin{Entry}{创立}{6,5}{⼑,⽴}
  \begin{Phonetics}{创立}{chuang4li4}[][HSK 5]
    \definition{v.}{fundar; originar; estabelecer}
  \end{Phonetics}
\end{Entry}

\begin{Entry}{创伤}{6,6}{⼑,⼈}
  \begin{Phonetics}{创伤}{chuang1shang1}[][HSK 7-9]
    \definition{s.}{ferida; parte do corpo lesionada, geralmente se refere a trauma | trauma; metáfora para dano emocional ou dano material}
  \end{Phonetics}
\end{Entry}

\begin{Entry}{创作}{6,7}{⼑,⼈}
  \begin{Phonetics}{创作}{chuang4zuo4}[][HSK 3]
    \definition[个]{s.}{criação; trabalho criativo; obras literárias e artísticas}
    \definition{v.}{escrever; criar; produzir; compor; criar obras artísticas}
  \end{Phonetics}
\end{Entry}

\begin{Entry}{创始人}{6,8,2}{⼑,⼥,⼈}
  \begin{Phonetics}{创始人}{chuang4shi3ren2}[][HSK 7-9]
    \definition{s.}{fundador; originador; iniciador | criador}
  \end{Phonetics}
\end{Entry}

\begin{Entry}{创建}{6,8}{⼑,⼵}
  \begin{Phonetics}{创建}{chuang4jian4}[][HSK 6]
    \definition{v.}{fundar; estabelecer; montar}
  \end{Phonetics}
\end{Entry}

\begin{Entry}{创造}{6,10}{⼑,⾡}
  \begin{Phonetics}{创造}{chuang4zao4}[][HSK 3]
    \definition{s.}{criação; inovação; primeiro a concluir ou a alcançar resultados}
    \definition{v.}{criar; produzir; trazer à tona; fazer ou estabelecer pela primeira vez; referir"-se de maneira geral a fazer ou estabelecer}
  \end{Phonetics}
\end{Entry}

\begin{Entry}{创意}{6,13}{⼑,⼼}
  \begin{Phonetics}{创意}{chuang4yi4}[][HSK 6]
    \definition[个]{s.}{criatividade; originalidade; novidade; uma ideia original, conceito, etc.}
    \definition{v.}{inovar; criar um novo conceito, ideia, etc. | propor designs criativos, ideias, etc.}
  \end{Phonetics}
\end{Entry}

\begin{Entry}{创新}{6,13}{⼑,⽄}
  \begin{Phonetics}{创新}{chuang4xin1}[][HSK 3]
    \definition[个,种,次]{s.}{inovação; algo novo ou diferente, uma ideia}
    \definition{v.}{trazer novas ideias; inovar; abrir novos caminhos; criar ou fazer algo novo, diferente do que era antes}
  \end{Phonetics}
\end{Entry}

%%%%%%%%%% 劣 %%%%%%%%%%
\subsection*{劣}\addcontentsline{loh}{figure}{劣}

\begin{Entry}{劣}{6}{⼒}
  \begin{Phonetics}{劣}{lie4}
    \definition{adj.}{ruim; inferior; desagradável; de baixa qualidade | Literário: menor; fraco}
  \antonymref{优}{you1}
  \end{Phonetics}
\end{Entry}

\begin{Entry}{劣势}{6,8}{⼒,⼒}
  \begin{Phonetics}{劣势}{lie4shi4}[][HSK 7-9]
    \definition[种]{s.}{força ou posição inferior; uma situação ou condição relativamente desfavorável}
  \end{Phonetics}
\end{Entry}

\begin{Entry}{劣质}{6,8}{⼒,⾙}
  \begin{Phonetics}{劣质}{lie4zhi4}[][HSK 7-9]
    \definition{adj.}{de má qualidade; de padrão inferior; de qualidade inferior}
  \end{Phonetics}
\end{Entry}

%%%%%%%%%% 动 %%%%%%%%%%
\subsection*{动}\addcontentsline{loh}{figure}{动}

\begin{Entry}{动}{6}{⼒}
  \begin{Phonetics}{动}{dong4}[][HSK 1]
    \definition{adj.}{não estacionário; móvel; variável; mutável}
    \definition{adv.}{facilmente; frequentemente}
    \definition{s.}{ação; movimento}
    \definition{v.}{mover; mexer; (pessoas ou coisas) mudar a posição ou o estado original | agir; começar a agir; entrar em ação | alterar; mudar; alterar a posição ou o estado original | usar; utilizar; tornar ativo | despertar; tocar (o coração de alguém); provocar mudanças emocionais, reações | [geralmente na forma negativa] comer ou beber | emocionar; deixar emocionado}
  \end{Phonetics}
\end{Entry}

\begin{Entry}{动人}{6,2}{⼒,⼈}
  \begin{Phonetics}{动人}{dong4ren2}[][HSK 3]
    \definition{adj.}{comovente; emocionante; tocante}
  \end{Phonetics}
\end{Entry}

\begin{Entry}{动力}{6,2}{⼒,⼒}
  \begin{Phonetics}{动力}{dong4li4}[][HSK 3]
    \definition[种,个]{s.}{poder; a força que faz com que as máquinas funcionem, por exemplo, energia elétrica, eólica, hidráulica, etc. | ímpeto; força motriz (ou propulsora); refere"-se, de maneira geral, à força que impulsiona o desenvolvimento das coisas}
  \end{Phonetics}
\end{Entry}

\begin{Entry}{动工}{6,3}{⼒,⼯}
  \begin{Phonetics}{动工}{dong4/gong1}[][HSK 7-9]
    \definition{v.+compl.}{começar a construção; começar a construir | construir; estar em construção | iniciar (um projeto de construção)}
  \end{Phonetics}
\end{Entry}

\begin{Entry}{动不动}{6,4,6}{⼒,⼀,⼒}
  \begin{Phonetics}{动不动}{dong4bu5dong4}[][HSK 7-9]
    \definition{adv.}{facilmente; frequentemente; a cada passo; indica que uma determinada ação ou situação (geralmente algo que você não quer que aconteça) provavelmente ocorrerá, geralmente usado com 就}
  \seealsoref{就}{jiu4}
  \end{Phonetics}
\end{Entry}

\begin{Entry}{动手}{6,4}{⼒,⼿}
  \begin{Phonetics}{动手}{dong4/shou3}[][HSK 5]
    \definition{v.+compl.}{iniciar o trabalho; começar a trabalhar | tocar; manusear; manipular | bater; levantar a mão (para bater); espancar}
  \end{Phonetics}
\end{Entry}

\begin{Entry}{动用}{6,5}{⼒,⽤}
  \begin{Phonetics}{动用}{dong4yong4}[][HSK 7-9]
    \definition{v.}{empregar; recorrer a; colocar em uso; usar (pessoal, dinheiro, etc. que se destinam a uso específico ou que não se destinam a ser usados casualmente)}
  \end{Phonetics}
\end{Entry}

\begin{Entry}{动向}{6,6}{⼒,⼝}
  \begin{Phonetics}{动向}{dong4xiang4}[][HSK 7-9]
    \definition{s.}{tendência; direção de movimento; atividades ou direções de desenvolvimento}
  \end{Phonetics}
\end{Entry}

\begin{Entry}{动机}{6,6}{⼒,⽊}
  \begin{Phonetics}{动机}{dong4ji1}[][HSK 5]
    \definition[个]{s.}{motivo; razão; intenção; ideias que motivam as pessoas a se envolverem em determinados comportamentos}
  \end{Phonetics}
\end{Entry}

\begin{Entry}{动作}{6,7}{⼒,⼈}
  \begin{Phonetics}{动作}{dong4zuo4}[][HSK 1]
    \definition[个]{s.}{movimento; ação; atividade de todo o corpo ou parte do corpo}
    \definition{v.}{agir; começar a se mover; entrar em ação}
  \end{Phonetics}
\end{Entry}

\begin{Entry}{动听}{6,7}{⼒,⼝}
  \begin{Phonetics}{动听}{dong4ting1}[][HSK 7-9]
    \definition{adj.}{interessante de ouvir; agradável de ouvir}
  \end{Phonetics}
\end{Entry}

\begin{Entry}{动员}{6,7}{⼒,⼝}
  \begin{Phonetics}{动员}{dong4yuan2}[][HSK 5]
    \definition{v.}{despertar; mobilizar; iniciar (para fazer algo ou participar de uma atividade) | mobilizar toda a nação; transferir dos setores militar, político e econômico para uma situação de guerra}
  \end{Phonetics}
\end{Entry}

\begin{Entry}{动身}{6,7}{⼒,⾝}
  \begin{Phonetics}{动身}{dong4/shen1}[][HSK 7-9]
    \definition{v.+compl.}{partir em uma jornada; partir (para um lugar distante); fazer uma viagem; começar uma viagem; partir; partir para outro lugar; começar uma jornada}
  \end{Phonetics}
\end{Entry}

\begin{Entry}{动态}{6,8}{⼒,⼼}
  \begin{Phonetics}{动态}{dong4tai4}[][HSK 5]
    \definition{s.}{tendências; desenvolvimentos; tendência geral dos assuntos; causa provável de ação; curso dos acontecimentos | expressão; comportamento ativo | estado dinâmico; condição dinâmica; de ou em relação a um estado de movimento}
  \end{Phonetics}
\end{Entry}

\begin{Entry}{动物}{6,8}{⼒,⽜}
  \begin{Phonetics}{动物}{dong4wu4}[][HSK 2]
    \definition[个,只,群,种]{s.}{animal; uma grande classe de seres vivos, que se alimentam principalmente de matéria orgânica, possuem sistema nervoso, são sensíveis e capazes de se mover; refere"-se a todos os tipos de coisas concretas ou abstratas}
  \end{Phonetics}
\end{Entry}

\begin{Entry}{动物园}{6,8,7}{⼒,⽜,⼞}
  \begin{Phonetics}{动物园}{dong4wu4yuan2}[][HSK 2]
    \definition[个,座,家]{s.}{jardim zoológico; zoo; parque que cria muitos tipos de animais (especialmente animais com valor científico ou raros na região) para exibição ao público}
  \end{Phonetics}
\end{Entry}

\begin{Entry}{动画}{6,8}{⼒,⽥}
  \begin{Phonetics}{动画}{dong4hua4}[][HSK 6]
    \definition[部]{s.}{desenho animado; animação; a imagem em movimento formada pela fotografia contínua das imagens desenhadas}
  \end{Phonetics}
\end{Entry}

\begin{Entry}{动画片}{6,8,4}{⼒,⽥,⽚}
  \begin{Phonetics}{动画片}{dong4hua4pian4}[][HSK 4]
    \definition[部,集,个]{s.}{desenho animado; animações; filme de animação}
  \end{Phonetics}
\end{Entry}

\begin{Entry}{动脉}{6,9}{⼒,⾁}
  \begin{Phonetics}{动脉}{dong4mai4}[][HSK 7-9]
    \definition[条]{s.}{Anatomia: artéria | Figurativo: estrada principal, linha ferroviária, rio, etc.}
  \end{Phonetics}
\end{Entry}

\begin{Entry}{动荡}{6,9}{⼒,⾋}
  \begin{Phonetics}{动荡}{dong4dang4}[][HSK 7-9]
    \definition{adj.}{(situação política, vida, etc.) caótico; instável; turbulento; metaforicamente falando, uma situação ou condição instável}
    \definition{v.}{ser turbulento; ser instável}
  \end{Phonetics}
\end{Entry}

\begin{Entry}{动弹}{6,11}{⼒,⼸}
  \begin{Phonetics}{动弹}{dong4tan5}[][HSK 7-9]
    \definition{v.}{mexer; mover (pessoas, animais ou coisas que se movem)}
  \end{Phonetics}
\end{Entry}

\begin{Entry}{动感}{6,13}{⼒,⼼}
  \begin{Phonetics}{动感}{dong4gan3}[][HSK 7-9]
    \definition{adj.}{realista | vívido}
    \definition{s.}{senso de movimento (geralmente em uma obra de arte estática)}
  \end{Phonetics}
\end{Entry}

\begin{Entry}{动摇}{6,13}{⼒,⼿}
  \begin{Phonetics}{动摇}{dong4yao2}[][HSK 4]
    \definition{adj.}{instável}
    \definition{v.}{ondular; pairar; agitar; balançar; sacudir | hesitar; vacilar; esmorecer; abalar}
  \end{Phonetics}
\end{Entry}

\begin{Entry}{动漫}{6,14}{⼒,⽔}
  \begin{Phonetics}{动漫}{dong4man4}
    \definition{s.}{desenhos animados; quadrinhos; anime; mangás; refere"-se a filmes de animação e histórias em quadrinhos}
  \seealsoref{漫画}{man4hua4}
  \end{Phonetics}
\end{Entry}

\begin{Entry}{动静}{6,14}{⼒,⾭}
  \begin{Phonetics}{动静}{dong4jing5}[][HSK 7-9]
    \definition{s.}{o som de algo se mexendo; ações ou sons da fala | atividade; movimento; (indagar ou explorar) a situação}
  \end{Phonetics}
\end{Entry}

%%%%%%%%%% 匈 %%%%%%%%%%
\subsection*{匈}\addcontentsline{loh}{figure}{匈}

\begin{Entry}{匈}{6}{⼓}
  \begin{Phonetics}{匈}{xiong1}
    \definition*{s.}{Hungria, abreviação de 匈牙利}
    \definition{s.}{peito; seio; tórax}
  \seealsoref{匈牙利}{xiong1ya2li4}
  \end{Phonetics}
\end{Entry}

\begin{Entry}{匈牙利}{6,4,7}{⼓,⽛,⼑}
  \begin{Phonetics}{匈牙利}{xiong1ya2li4}
    \definition*{s.}{Hungria}
  \end{Phonetics}
\end{Entry}

\begin{Entry}{匈奴}{6,5}{⼓,⼥}
  \begin{Phonetics}{匈奴}{xiong1nu2}
    \definition*{s.}{Xiongnu, um povo da estepe oriental que criou um império que floresceu na época das dinastias Qin e Han}
  \end{Phonetics}
\end{Entry}

%%%%%%%%%% 匠 %%%%%%%%%%
\subsection*{匠}\addcontentsline{loh}{figure}{匠}

\begin{Entry}{匠}{6}{⼕}
  \begin{Phonetics}{匠}{jiang4}
    \definition*{s.}{Sobrenome: Jiang}
    \definition{s.}{artesão | pessoa de realizações notáveis em um campo específico; mestre}
  \end{Phonetics}
\end{Entry}

%%%%%%%%%% 华 %%%%%%%%%%
\subsection*{华}\addcontentsline{loh}{figure}{华}

\begin{Entry}{华}{6}{⼗}
  \begin{Phonetics}{华}{hua2}
    \definition*{s.}{China; refere"-se à China (anteriormente conhecida como Huaxia, 华夏, mais tarde chamada de Zhonghua, 中华, ou simplesmente Hua, 华)}
    \definition{adj.}{esplêndido; magnífico | próspero; florescente | chamativo; extravagante; vaidoso | grisalho}
    \definition{s.}{corona; um halo colorido ao redor do sol ou da lua causado pela difração da luz através das nuvens | creme; melhor parte; a melhor parte das coisas | chinês; refere"-se à nacionalidade Han (língua e escrita) | vezes; anos; refere"-se a (bons) momentos | elixir; essência líquida; substâncias formadas pela sedimentação de minerais na água de nascente | Seu, palavra honorífica, usada para se referir a coisas relacionadas à outra pessoa}
  \seealsoref{华夏}{hua2xia4}
  \seealsoref{中华}{zhong1hua2}
  \end{Phonetics}
  \begin{Phonetics}{华}{hua4}
    \definition*{s.}{Huashan Mountain (na província de Shaanxi) | Sobrenome: Hua}
  \end{Phonetics}
\end{Entry}

\begin{Entry}{华人}{6,2}{⼗,⼈}
  \begin{Phonetics}{华人}{hua2ren2}[][HSK 3]
    \definition[名,位,个]{s.}{Chinês; chinês étnico | chineses no exterior; refere"-se a cidadãos estrangeiros de ascendência chinesa que obtiveram a nacionalidade do país em que residem}
  \end{Phonetics}
\end{Entry}

\begin{Entry}{华山}{6,3}{⼗,⼭}
  \begin{Phonetics}{华山}{hua4shan1}
    \definition{s.}{Monte Hua em Shaanxi, montanha ocidental das Cinco Montanhas Sagradas (五岳)}
  \seealsoref{五岳}{wu3yue4}
  \end{Phonetics}
\end{Entry}

\begin{Entry}{华氏}{6,4}{⼗,⽒}
  \begin{Phonetics}{华氏}{hua2shi4}
    \definition{s.}{graus Fahrenheit (°F)}
  \end{Phonetics}
\end{Entry}

\begin{Entry}{华丽}{6,7}{⼗,⼀}
  \begin{Phonetics}{华丽}{hua2li4}[][HSK 7-9]
    \definition{adj.}{magnífico; resplandecente; deslumbrante; lindo e radiante}
  \end{Phonetics}
\end{Entry}

\begin{Entry}{华侨}{6,8}{⼗,⼈}
  \begin{Phonetics}{华侨}{hua2qiao2}[][HSK 7-9]
    \definition[个,位,名]{s.}{chineses que vivem no exterior}
  \end{Phonetics}
\end{Entry}

\begin{Entry}{华语}{6,9}{⼗,⾔}
  \begin{Phonetics}{华语}{hua2yu3}[][HSK 5]
    \definition*{s.}{Chinês (idioma)}
  \end{Phonetics}
\end{Entry}

\begin{Entry}{华夏}{6,10}{⼗,⼢}
  \begin{Phonetics}{华夏}{hua2xia4}
    \definition*{s.}{Huaxia, nome antigo da China | Catai}
  \end{Phonetics}
\end{Entry}

\begin{Entry}{华盛顿}{6,11,10}{⼗,⽫,⾴}
  \begin{Phonetics}{华盛顿}{hua2sheng4dun4}
    \definition*{s.}{Washington}
  \end{Phonetics}
\end{Entry}

\begin{Entry}{华裔}{6,13}{⼗,⾐}
  \begin{Phonetics}{华裔}{hua2yi4}[][HSK 7-9]
    \definition[位,名,个]{s.}{etnia chinesa; crianças nascidas de chineses no exterior no país de residência e que adquiriram a nacionalidade do país de residência}
  \end{Phonetics}
\end{Entry}

%%%%%%%%%% 协 %%%%%%%%%%
\subsection*{协}\addcontentsline{loh}{figure}{协}

\begin{Entry}{协}{6}{⼗}
  \begin{Phonetics}{协}{xie2}
    \definition*{s.}{Sobrenome: Xie}
    \definition{adv.}{conjuntamente; coordenadamente; juntos}
    \definition{s.}{harmonioso}
    \definition{v.}{auxiliar; assistir; ajudar}
  \end{Phonetics}
\end{Entry}

\begin{Entry}{协议}{6,5}{⼗,⾔}
  \begin{Phonetics}{协议}{xie2yi4}[][HSK 5]
    \definition[份,项]{s.}{acordo; tratado; decisão conjunta alcançada através de negociação e consulta}
    \definition{v.}{concordar em}
  \end{Phonetics}
\end{Entry}

\begin{Entry}{协议书}{6,5,4}{⼗,⾔,⼄}
  \begin{Phonetics}{协议书}{xie2yi4shu1}[][HSK 5]
    \definition{s.}{contrato | protocolo}
  \end{Phonetics}
\end{Entry}

\begin{Entry}{协会}{6,6}{⼗,⼈}
  \begin{Phonetics}{协会}{xie2hui4}[][HSK 6]
    \definition[个]{s.}{sociedade; instituto; associação; uma organização de massa formada para promover uma causa comum}
  \end{Phonetics}
\end{Entry}

\begin{Entry}{协助}{6,7}{⼗,⼒}
  \begin{Phonetics}{协助}{xie2zhu4}[][HSK 6]
    \definition{v.}{ajudar; auxiliar; dar assistência; fornecer ajuda}
  \end{Phonetics}
\end{Entry}

\begin{Entry}{协调}{6,10}{⼗,⾔}
  \begin{Phonetics}{协调}{xie2tiao2}[][HSK 6]
    \definition{adj.}{coordenado; harmonioso; em sintonia}
    \definition{v.}{coordenar; concertar; integrar; harmonizar; fazer a harmonia apropriada}
  \end{Phonetics}
\end{Entry}

\begin{Entry}{协商}{6,11}{⼗,⼝}
  \begin{Phonetics}{协商}{xie2shang1}[][HSK 6]
    \definition{v.}{discutir; consultar; negociar; várias partes discutiram e decidiram em conjunto para chegar à mesma visão}
  \end{Phonetics}
\end{Entry}

%%%%%%%%%% 危 %%%%%%%%%%
\subsection*{危}\addcontentsline{loh}{figure}{危}

\begin{Entry}{危}{6}{⼙}
  \begin{Phonetics}{危}{wei1}
    \definition*{s.}{Wei, a décima segunda das vinte e oito constelações em que a esfera celeste foi dividida, consistindo de três estrelas em forma de triângulo obtuso, uma em Aquário e duas em Pégaso | Wei, uma das mansões lunares | Sobrenome: Wei}
    \definition{adj.}{arriscado; inseguro; perigoso | estar gravemente doente; estar morrendo | alto; íngreme}
    \definition{s.}{perigo | cumeeira (de um telhado)}
    \definition{v.}{pôr em perigo; colocar em perigo; comprometer}
  \antonymref{安}{an1}
  \end{Phonetics}
\end{Entry}

\begin{Entry}{危及}{6,3}{⼙,⼃}
  \begin{Phonetics}{危及}{wei1ji2}[][HSK 7-9]
    \definition{v.}{prejudicar; colocar em perigo; comprometer; ameaçar}
  \end{Phonetics}
\end{Entry}

\begin{Entry}{危机}{6,6}{⼙,⽊}
  \begin{Phonetics}{危机}{wei1ji1}[][HSK 6]
    \definition[个,次]{s.}{crise}
  \end{Phonetics}
\end{Entry}

\begin{Entry}{危急}{6,9}{⼙,⼼}
  \begin{Phonetics}{危急}{wei1ji2}[][HSK 7-9]
    \definition{adj.}{crítico; em perigo iminente; em uma situação desesperadora; perigoso e urgente}
  \synonymref{风险}{feng1xian3}
  \synonymref{紧急}{jin3ji2}
  \synonymref{紧迫}{jin3po4}
  \synonymref{紧张}{jin3zhang1}
  \synonymref{迫切}{po4qie4}
  \synonymref{危害}{wei1hai4}
  \synonymref{危机}{wei1ji1}
  \synonymref{危险}{wei1xian3}
  \synonymref{严重}{yan2zhong4}
  \antonymref{安全}{an1quan2}
  \antonymref{安稳}{an1wen3}
  \end{Phonetics}
\end{Entry}

\begin{Entry}{危险}{6,9}{⼙,⾩}
  \begin{Phonetics}{危险}{wei1xian3}[][HSK 3]
    \definition{adj.}{arriscado; perigoso}
  \end{Phonetics}
\end{Entry}

\begin{Entry}{危害}{6,10}{⼙,⼧}
  \begin{Phonetics}{危害}{wei1hai4}[][HSK 3]
    \definition[个,种]{s.}{prejuízo; perigo; dano}
    \definition{v.}{destruir; prejudicar; pôr em perigo; pôr em risco}
  \end{Phonetics}
\end{Entry}

\begin{Entry}{危难}{6,10}{⼙,⾫}
  \begin{Phonetics}{危难}{wei1nan4}
    \definition{s.}{calamidade}
  \end{Phonetics}
\end{Entry}

%%%%%%%%%% 压 %%%%%%%%%%
\subsection*{压}\addcontentsline{loh}{figure}{压}

\begin{Entry}{压}{6}{⼚}
  \begin{Phonetics}{压}{ya1}[][HSK 3]
    \definition{v.}{pressionar; empurrar para baixo; segurar; pesar | acalmar emoções agitadas ou situações ruins; tranquilizar | intimidar; reprimir; exercer pressão sobre; usar poder, posição ou padrões morais para coagir ou restringir as pessoas, impedindo-as de se expressar, decidir ou se desenvolver livremente | aproximar-se; estar chegando perto | arquivar; deixar de lado | pressionar; metáfora para uma grande carga emocional e psicológica | superar; ultrapassar; voz, capacidade e presença mais fortes do que os outros | apostar em um determinado resultado ao jogar | pressionar; força na superfície de contato do objeto}
  \end{Phonetics}
  \begin{Phonetics}{压}{ya4}
    \definition{adv.}{fundamentalmente; nunca (usado principalmente em frases negativas)}
  \seealsoref{压根儿}{ya4gen1r5}
  \end{Phonetics}
\end{Entry}

\begin{Entry}{压力}{6,2}{⼚,⼒}
  \begin{Phonetics}{压力}{ya1li4}[][HSK 3]
    \definition[份,个]{s.}{pressão; força atuando perpendicularmente à superfície de um objeto | pressão; força esmagadora; metáfora para a força que coage e intimida as pessoas (principalmente nos aspectos espirituais e psicológicos) | tensão; fardo; os encargos econômicos, psicológicos e espirituais impostos pelo mundo exterior}
  \end{Phonetics}
\end{Entry}

\begin{Entry}{压岁钱}{6,6,10}{⼚,⼭,⾦}
  \begin{Phonetics}{压岁钱}{ya1sui4qian2}
    \definition{s.}{dinheiro da sorte | dinheiro dado às crianças como presente no Ano Novo Chinês}
  \end{Phonetics}
\end{Entry}

\begin{Entry}{压迫}{6,8}{⼚,⾡}
  \begin{Phonetics}{压迫}{ya1po4}[][HSK 6]
    \definition{v.}{oprimir; reprimir; confiar no poder para suprimir e forçar | contrair; uma força externa comprime uma parte de um organismo}
  \end{Phonetics}
\end{Entry}

\begin{Entry}{压根儿}{6,10,2}{⼚,⽊,⼉}
  \begin{Phonetics}{压根儿}{ya4gen1r5}
    \definition{adv.}{(geralmente no negativo) nunca; fundamentalmente}
  \end{Phonetics}
\end{Entry}

\begin{Entry}{压碎}{6,13}{⼚,⽯}
  \begin{Phonetics}{压碎}{ya1sui4}
    \definition{v.}{esmagar em pedaços}
  \end{Phonetics}
\end{Entry}

\begin{Entry}{压韵}{6,13}{⼚,⾳}
  \begin{Phonetics}{压韵}{ya1yun4}
    \variantof{押韵}
  \end{Phonetics}
\end{Entry}

%%%%%%%%%% 吃 %%%%%%%%%%
\subsection*{吃}\addcontentsline{loh}{figure}{吃}

\begin{Entry}{吃}{6}{⼝}
  \begin{Phonetics}{吃}{chi1}[][HSK 1]
    \definition{s.}{alimentos; necessidades básicas}
    \definition{v.}{comer; pegar; fazer; colocar alimentos na boca, mastigar e engolir (incluindo sugar e beber) | viver; depender de algo para viver | aniquilar; eliminar (usado principalmente em jogos de guerra e jogos de tabuleiro) | esgotar; exaurir; ser um fardo; ser um esforço | absorver | sofrer; incorrer | entender; compreender | entrar um objeto em outro | expressar aceitação psicológica | fazer suas refeições; comer}
  \end{Phonetics}
\end{Entry}

\begin{Entry}{吃力}{6,2}{⼝,⼒}
  \begin{Phonetics}{吃力}{chi1li4}[][HSK 5]
    \definition{adj.}{suado; extenuante; trabalhoso; laborioso | cansado; fatigado}
  \end{Phonetics}
\end{Entry}

\begin{Entry}{吃亏}{6,3}{⼝,⼆}
  \begin{Phonetics}{吃亏}{chi1/kui1}[][HSK 7-9]
    \definition{adv.}{em desvantagem; em situação desfavorável}
    \definition{v.+compl.}{sofrer perdas; sofrer aflição; levar a pior; levar uma surra}
  \end{Phonetics}
\end{Entry}

\begin{Entry}{吃不上}{6,4,3}{⼝,⼀,⼀}
  \begin{Phonetics}{吃不上}{chi1bu5shang4}[][HSK 7-9]
    \definition{v.}{incapaz de comer alguma coisa | pular uma refeição; perder a chance de comer | não conseguir comer alguma coisa; não ter comida para comer}
  \end{Phonetics}
\end{Entry}

\begin{Entry}{吃饭}{6,7}{⼝,⾷}
  \begin{Phonetics}{吃饭}{chi1/fan4}[][HSK 1]
    \definition{v.+compl.}{comer; ter (comer) uma refeição | manter"-se vivo;  ganhar a vida; refere"-se à vida ou à sobrevivência em geral}
  \end{Phonetics}
\end{Entry}

\begin{Entry}{吃苦}{6,8}{⼝,⾋}
  \begin{Phonetics}{吃苦}{chi1/ku3}[][HSK 7-9]
    \definition{v.+compl.}{suportar dificuldades; sofrer}[他在工作中吃了很多苦。===Ele sofreu muito em seu trabalho.]
  \end{Phonetics}
\end{Entry}

\begin{Entry}{吃屎}{6,9}{⼝,⼫}
  \begin{Phonetics}{吃屎}{chi1 shi3}
    \definition{expr.}{Coma merda!}
  \end{Phonetics}
\end{Entry}

\begin{Entry}{吃惊}{6,11}{⼝,⼼}
  \begin{Phonetics}{吃惊}{chi1/jing1}[][HSK 4]
    \definition{v.+compl.}{ficar assustado; ficar chocado; ficar espantado; pegar de surpresa; ficar assustado inesperadamente}
  \synonymref{诧异}{cha4yi4}
  \synonymref{惊诧}{jing1cha4}
  \synonymref{惊奇}{jing1qi2}
  \synonymref{惊讶}{jing1ya4}
  \synonymref{惊异}{jing1yi4}
  \synonymref{受惊}{shou4/jing1}
  \antonymref{冷静}{leng3jing4}
  \end{Phonetics}
\end{Entry}

\begin{Entry}{吃喝玩乐}{6,12,8,5}{⼝,⼝,⽟,⼃}
  \begin{Phonetics}{吃喝玩乐}{chi1-he1-wan2-le4}[][HSK 7-9]
    \definition{expr.}{comer, beber e se divertir, passar o tempo com prazer | abandonar"-se a uma vida de prazer}
  \end{Phonetics}
\end{Entry}

%%%%%%%%%% 各 %%%%%%%%%%
\subsection*{各}\addcontentsline{loh}{figure}{各}

\begin{Entry}{各}{6}{⼝}
  \begin{Phonetics}{各}{ge4}[][HSK 3]
    \definition{adv.}{de várias maneiras; de diversas formas; respectivamente; indica que algo é feito separadamente ou que possui uma determinada característica separadamente}
    \definition{pron.}{todo; todos; cada; refere"-se a todos os indivíduos dentro de um determinado intervalo, equivalente a 每个}
  \seealsoref{每个}{mei3ge4}
  \end{Phonetics}
\end{Entry}

\begin{Entry}{各个}{6,3}{⼝,⼈}
  \begin{Phonetics}{各个}{ge4ge4}[][HSK 4]
    \definition{adv./pron.}{cada | um a um; um após o outro}
  \end{Phonetics}
\end{Entry}

\begin{Entry}{各地}{6,6}{⼝,⼟}
  \begin{Phonetics}{各地}{ge4di4}[][HSK 3]
    \definition{s.}{em todos os lugares; em vários locais}
  \end{Phonetics}
\end{Entry}

\begin{Entry}{各式各样}{6,6,6,10}{⼝,⼷,⼝,⽊}
  \begin{Phonetics}{各式各样}{ge4shi4-ge4yang4}[][HSK 7-9]
    \definition{expr.}{todo tipo de\dots; todos os tipos de\dots; todos os tipos de; de várias maneiras; de todas as descrições; uma variedade de; uma variedade de variedades com cores diferentes}
  \end{Phonetics}
\end{Entry}

\begin{Entry}{各自}{6,6}{⼝,⾃}
  \begin{Phonetics}{各自}{ge4zi4}[][HSK 3]
    \definition{pron.}{por si mesmo; por conta própria; cada um por si | cada um; indica cada uma das partes envolvidas}
  \end{Phonetics}
\end{Entry}

\begin{Entry}{各位}{6,7}{⼝,⼈}
  \begin{Phonetics}{各位}{ge4wei4}[][HSK 3]
    \definition{pron.}{todos; toda a gente; todo mundo | cada um}
  \end{Phonetics}
\end{Entry}

\begin{Entry}{各奔前程}{6,8,9,12}{⼝,⼤,⼑,⽲}
  \begin{Phonetics}{各奔前程}{ge4ben4qian2cheng2}[][HSK 7-9]
    \definition{expr.}{``Cada um segue seu próprio caminho.''; cada pessoa tem sua própria vida para viver; cada um deles desenvolve sua própria carreira ambiciosa; cada um segue seu próprio curso}
  \end{Phonetics}
\end{Entry}

\begin{Entry}{各种}{6,9}{⼝,⽲}
  \begin{Phonetics}{各种}{ge4zhong3}[][HSK 3]
    \definition{adv.}{todos os tipos; vários tipos}
  \end{Phonetics}
\end{Entry}

%%%%%%%%%% 合 %%%%%%%%%%
\subsection*{合}\addcontentsline{loh}{figure}{合}

\begin{Entry}{合}{6}{⼝}
  \begin{Phonetics}{合}{he2}[][HSK 3]
    \definition{adj.}{todo; completo; inteiro}
    \definition{clas.}{usado para rodadas | 100ml | medida para grãos secos igual a um décimo de 升, ou um centésimo de 斗}
    \definition{s.}{casamento; união matrimonial | (astronomia) conjunção | nota da escala em Gongchepu (工尺谱), correspondente ao 5 na notação musical numerada}
    \definition{v.}{fechar | juntar; combinar | adequar-se; concordar; conformar-se a | ser igual a; somar | ser adequado}
  \seealsoref{斗}{dou4}
  \seealsoref{工尺谱}{gong1 che3 pu3}
  \seealsoref{升}{sheng1}
  \antonymref{分}{fen1}
  \end{Phonetics}
\end{Entry}

\begin{Entry}{合计}{6,4}{⼝,⾔}
  \begin{Phonetics}{合计}{he2ji4}[][HSK 7-9]
    \definition{v.}{pensar sobre; descobrir | consultar | somar; totalizar}
  \end{Phonetics}
  \begin{Phonetics}{合计}{he2ji5}
    \definition{v.}{totalizar; somar; estimar | discutir; negociar; deliberar}
  \end{Phonetics}
\end{Entry}

\begin{Entry}{合乎}{6,5}{⼝,⼃}
  \begin{Phonetics}{合乎}{he2hu1}[][HSK 7-9]
    \definition{v.}{conformar-se com (ou a); corresponder a; concordar com; coincidir com; ser consistente com}
  \end{Phonetics}
\end{Entry}

\begin{Entry}{合伙}{6,6}{⼝,⼈}
  \begin{Phonetics}{合伙}{he2huo3}[][HSK 7-9]
    \definition{v.}{formar uma parceria; formar uma parceria relativamente fixa (para se envolver em atividades comerciais ou fazer coisas ruins)}
  \end{Phonetics}
\end{Entry}

\begin{Entry}{合同}{6,6}{⼝,⼝}
  \begin{Phonetics}{合同}{he2tong5}[][HSK 4]
    \definition[个,份]{s.}{contrato; acordo; uma disposição para observância mútua por duas ou mais partes na condução de um assunto com o objetivo de determinar seus respectivos direitos e obrigações.}
  \end{Phonetics}
\end{Entry}

\begin{Entry}{合并}{6,6}{⼝,⼲}
  \begin{Phonetics}{合并}{he2bing4}[][HSK 5]
    \definition{v.}{fundir; amalgamar; combinar várias coisas em uma coisa só | (doença) ser complicada por outra doença; uma doença levar a outra, ataques simultâneos (de várias doenças)}
  \end{Phonetics}
\end{Entry}

\begin{Entry}{合成}{6,6}{⼝,⼽}
  \begin{Phonetics}{合成}{he2cheng2}[][HSK 5]
    \definition{s.}{compor; integrar; combinar; misturar | Química: sintetizar, reação química para transformar uma substância com uma composição simples em uma substância com uma composição complexa}
  \end{Phonetics}
\end{Entry}

\begin{Entry}{合约}{6,6}{⼝,⽷}
  \begin{Phonetics}{合约}{he2yue1}[][HSK 6]
    \definition[份]{s.}{contrato; geralmente se refere a contratos com cláusulas mais simples}
  \end{Phonetics}
\end{Entry}

\begin{Entry}{合作}{6,7}{⼝,⼈}
  \begin{Phonetics}{合作}{he2zuo4}[][HSK 3]
    \definition{v.}{cooperar; colaborar; trabalhar em conjunto; trabalhar em conjunto para realizar algo ou concluir uma tarefa}
  \end{Phonetics}
\end{Entry}

\begin{Entry}{合作社}{6,7,7}{⼝,⼈,⽰}
  \begin{Phonetics}{合作社}{he2zuo4she4}[][HSK 7-9]
    \definition{s.}{cooperativa | cooperativa de trabalhadores ou produtores agrícolas, etc.}
  \end{Phonetics}
\end{Entry}

\begin{Entry}{合法}{6,8}{⼝,⽔}
  \begin{Phonetics}{合法}{he2fa3}[][HSK 3]
    \definition{adj.}{legal; legítimo; lícito;  justo; válido; em conformidade com as disposições legais}
  \end{Phonetics}
\end{Entry}

\begin{Entry}{合宪性}{6,9,8}{⼝,⼧,⼼}
  \begin{Phonetics}{合宪性}{he2xian4xing4}
    \definition{s.}{constitucionalismo}
  \end{Phonetics}
\end{Entry}

\begin{Entry}{合适}{6,9}{⼝,⾡}
  \begin{Phonetics}{合适}{he2shi4}[][HSK 2]
    \definition{adj.}{correto; adequado; apropriado; conveniente; em conformidade com a realidade ou com os requisitos objetivos}
  \end{Phonetics}
\end{Entry}

\begin{Entry}{合格}{6,10}{⼝,⽊}
  \begin{Phonetics}{合格}{he2ge2}[][HSK 3]
    \definition{adj.}{qualificado; dentro dos padrões; em conformidade com os requisitos ou normas}
  \end{Phonetics}
\end{Entry}

\begin{Entry}{合资}{6,10}{⼝,⾙}
  \begin{Phonetics}{合资}{he2zi1}[][HSK 7-9]
    \definition{s.}{consórcio; \emph{joint-venture} com capitais mistos; investimento conjunto por duas ou mais partes (diferente de 独资)}
    \definition{v.}{investir conjuntamente em}
  \seealsoref{独资}{du2zi1}
  \end{Phonetics}
\end{Entry}

\begin{Entry}{合唱}{6,11}{⼝,⼝}
  \begin{Phonetics}{合唱}{he2chang4}[][HSK 7-9]
    \definition{v.}{cantar em coro; cantar ou apresentar-se junto}[他们一起合唱一台戏。===Eles cantam uma ópera juntos.]
  \end{Phonetics}
\end{Entry}

\begin{Entry}{合情合理}{6,11,6,11}{⼝,⼼,⼝,⽟}
  \begin{Phonetics}{合情合理}{he2qing2-he2li3}[][HSK 7-9]
    \definition{expr.}{razoável; razoável e lógico; justo e racional; justificável e sensato; justo e razoável; justo e sensato}
  \end{Phonetics}
\end{Entry}

\begin{Entry}{合理}{6,11}{⼝,⽟}
  \begin{Phonetics}{合理}{he2li3}[][HSK 3]
    \definition{adj.}{racional; razoável; equitativo; razoável ou lógico}
  \end{Phonetics}
\end{Entry}

\begin{Entry}{合影}{6,15}{⼝,⼺}
  \begin{Phonetics}{合影}{he2/ying3}[][HSK 7-9]
    \definition[张,个]{s.}{foto de grupo; imagem de grupo}
    \definition{v.+compl.}{tirar uma foto em grupo; tirar uma foto}
  \end{Phonetics}
\end{Entry}

%%%%%%%%%% 吉 %%%%%%%%%%
\subsection*{吉}\addcontentsline{loh}{figure}{吉}

\begin{Entry}{吉}{6}{⼝}
  \begin{Phonetics}{吉}{ji2}
    \definition*{s.}{Província de Jilin, abreviação de 吉林 | Sobrenome: Ji}
    \definition{adj.}{sortudo; propício; auspicioso}
  \seealsoref{吉林}{ji2lin2}
  \antonymref{凶}{xiong1}
  \end{Phonetics}
\end{Entry}

\begin{Entry}{吉他}{6,5}{⼝,⼈}
  \begin{Phonetics}{吉他}{ji2ta1}[][HSK 7-9]
    \definition[把]{s.}{Empréstimo linguístico: violão}
  \end{Phonetics}
\end{Entry}

\begin{Entry}{吉利}{6,7}{⼝,⼑}
  \begin{Phonetics}{吉利}{ji2li4}[][HSK 6]
    \definition{adj.}{sortudo; auspicioso; propício}
  \end{Phonetics}
\end{Entry}

\begin{Entry}{吉林}{6,8}{⼝,⽊}
  \begin{Phonetics}{吉林}{ji2lin2}
    \definition*{s.}{Província de Jilin}
  \end{Phonetics}
\end{Entry}

\begin{Entry}{吉祥}{6,10}{⼝,⽰}
  \begin{Phonetics}{吉祥}{ji2xiang2}[][HSK 6]
    \definition{adj.}{sortudo; auspicioso; propício}
    \definition[个,种]{s.}{sorte; auspiciosidade; propiciação; um sinal ou símbolo de boa sorte ou fortuna}
  \end{Phonetics}
\end{Entry}

\begin{Entry}{吉祥物}{6,10,8}{⼝,⽰,⽜}
  \begin{Phonetics}{吉祥物}{ji2xiang2wu4}[][HSK 7-9]
    \definition{s.}{mascote}
  \end{Phonetics}
\end{Entry}

\begin{Entry}{吉普}{6,12}{⼝,⽇}
  \begin{Phonetics}{吉普}{ji2pu3}[][HSK 7-9]
    \definition*{s.}{Jeep (marca de carro)}
    \definition[辆]{s.}{Empéstimo linguístico: jipe}[他开着吉普车去沙漠旅行。===Ele fez uma viagem pelo deserto em seu jipe.]
  \seealsoref{吉普车}{ji2pu3che1}
  \end{Phonetics}
\end{Entry}

\begin{Entry}{吉普车}{6,12,4}{⼝,⽇,⾞}
  \begin{Phonetics}{吉普车}{ji2pu3che1}
    \definition[辆]{s.}{Empréstimo linguístico: jipe (veículo militar)}
  \seealsoref{吉普}{ji2pu3}
  \end{Phonetics}
\end{Entry}

%%%%%%%%%% 吊 %%%%%%%%%%
\subsection*{吊}\addcontentsline{loh}{figure}{吊}

\begin{Entry}{吊}{6}{⼝}
  \begin{Phonetics}{吊}{diao4}[][HSK 6]
    \definition{clas.}{uma sequência de 1.000 em dinheiro; antigamente, uma unidade monetária geralmente era composta por mil pequenas moedas de cobre}
    \definition{s.}{guindaste}
    \definition{v.}{pendurar; suspender | levantar ou abaixar com uma corda, etc. | colocar um forro de pele; adicionar revestimentos ou forros aos barris de couro para fazer roupas | revogar; retirar; recuperar documentos emitidos | lamentar; prestar homenagem os mortos ou oferecer condolências às famílias ou grupos que sofreram uma perda}
  \end{Phonetics}
\end{Entry}

\begin{Entry}{吊销}{6,12}{⼝,⾦}
  \begin{Phonetics}{吊销}{diao4xiao1}[][HSK 7-9]
    \definition{v.}{revogar; retirar; desativar; cancelar; reclamar e cancelar (certificados emitidos)}
  \end{Phonetics}
\end{Entry}

%%%%%%%%%% 同 %%%%%%%%%%
\subsection*{同}\addcontentsline{loh}{figure}{同}

\begin{Entry}{同}{6}{⼝}
  \begin{Phonetics}{同}{tong2}[][HSK 6]
    \definition{adj.}{como; igual; parecido; similar; o mesmo; sem diferença}
    \definition{adv.}{juntos; em comum; indica que diferentes atores realizam uma determinada ação juntos ou estão na mesma situação, o que equivale a 一同 ou 一起}
    \definition{v.}{ser o mesmo que}
  \seealsoref{一起}{yi4qi3}
  \seealsoref{一同}{yi4tong2}
  \end{Phonetics}
  \begin{Phonetics}{同}{tong4}
    \definition[条,处]{s.}{beco; rua estreita}
  \seealsoref{胡同}{hu2tong5}
  \end{Phonetics}
\end{Entry}

\begin{Entry}{同一}{6,1}{⼝,⼀}
  \begin{Phonetics}{同一}{tong2yi1}[][HSK 6]
    \definition{adj.}{mesmo; idêntico}
    \definition[讲]{s.}{identidade; unidade}
  \end{Phonetics}
\end{Entry}

\begin{Entry}{同人}{6,2}{⼝,⼈}
  \begin{Phonetics}{同人}{tong2ren2}[][HSK 7-9]
    \definition{s.}{colega de trabalho; pessoas do mesmo local de trabalho ou profissão | colega | entusiastas da cultura pop que criam fanfics etc.}
  \end{Phonetics}
\end{Entry}

\begin{Entry}{同伙}{6,6}{⼝,⼈}
  \begin{Phonetics}{同伙}{tong2huo3}[][HSK 7-9]
    \definition[个]{s.}{parceiro; cúmplice; aliado}
    \definition{v.}{trabalhar em parceria; conspirar (para praticar o mal) | conspirar com; participar conjuntamente em uma organização ou envolver-se em determinada atividade | associar"-se}
  \synonymref{伴侣}{ban4lv3}
  \synonymref{伙伴}{huo3ban4}
  \synonymref{朋友}{peng2you5}
  \synonymref{同伴}{tong2ban4}
  \synonymref{同盟}{tong2meng2}
  \antonymref{对手}{dui4shou3}
  \end{Phonetics}
\end{Entry}

\begin{Entry}{同年}{6,6}{⼝,⼲}
  \begin{Phonetics}{同年}{tong2nian2}[][HSK 7-9]
    \definition{adj.}{Dialeto: da mesma idade; no mesmo ano}
    \definition{s.}{mesmo ano | Obsoleto: candidatos que passaram nos exames imperiais no mesmo ano}
  \end{Phonetics}
\end{Entry}

\begin{Entry}{同舟共济}{6,6,6,9}{⼝,⾈,⼋,⽔}
  \begin{Phonetics}{同舟共济}{tong2zhou1-gong4ji4}[][HSK 7-9]
    \definition{expr.}{atravessar um rio no mesmo barco; remar juntos em tempos difíceis; unir-se em momentos de adversidade; pessoas no mesmo barco se ajudam mutuamente; unem forças para superar dificuldades}
  \antonymref{各奔前程}{ge4ben4qian2cheng2}
  \end{Phonetics}
\end{Entry}

\begin{Entry}{同行}{6,6}{⼝,⾏}
  \begin{Phonetics}{同行}{tong2hang2}[][HSK 6]
    \definition{s.}{do mesmo ofício ou ocupação; pessoas no mesmo setor}
    \definition{v.}{ser do mesmo ofício ou ocupação; trabalhar no mesmo setor}
  \end{Phonetics}
\end{Entry}

\begin{Entry}{同伴}{6,7}{⼝,⼈}
  \begin{Phonetics}{同伴}{tong2ban4}[][HSK 7-9]
    \definition[名,个]{s.}{companheiro; uma pessoa com quem você trabalha, mora ou realiza alguma atividade específica}
  \synonymref{伴侣}{ban4lv3}
  \synonymref{差错}{cha1cuo4}
  \synonymref{错误}{cuo4wu4}
  \synonymref{搭档}{da1dang4}
  \synonymref{过错}{guo4cuo4}
  \synonymref{伙伴}{huo3ban4}
  \synonymref{朋友}{peng2you5}
  \synonymref{同伙}{tong2huo3}
  \end{Phonetics}
\end{Entry}

\begin{Entry}{同志}{6,7}{⼝,⼼}
  \begin{Phonetics}{同志}{tong2zhi4}[][HSK 7-9]
    \definition[个,位,名,些]{s.}{camarada; pessoas que compartilham objetivos comuns; especificamente, membros do mesmo partido político | título habitual usado em ocasiões formais; a forma como as pessoas se tratam atualmente é geralmente usada em contextos formais e é menos comum na linguagem falada}
  \antonymref{敌人}{di2ren2}
  \end{Phonetics}
\end{Entry}

\begin{Entry}{同时}{6,7}{⼝,⽇}
  \begin{Phonetics}{同时}{tong2shi2}[][HSK 2]
    \definition{conj.}{além disso; além do mais; ainda mais; indica uma relação de equivalência, geralmente com um significado mais profundo}
    \definition{s.}{enquanto isso; ao mesmo tempo}
  \end{Phonetics}
\end{Entry}

\begin{Entry}{同步}{6,7}{⼝,⽌}
  \begin{Phonetics}{同步}{tong2bu4}[][HSK 7-9]
    \definition{s.}{sincronizar; sincronizar com; coordenar em tempo de progresso; geralmente se refere ao ritmo de ação coordenado e consistente de coisas interconectadas}
    \definition{s.}{sincronismo; sincronização; em ciência e tecnologia, refere-se a duas ou mais grandezas que se alteram ao longo do tempo, mantendo uma determinada relação relativa durante o processo de mudança}
  \synonymref{协调}{xie2tiao2}
  \synonymref{一起}{yi4qi3}
  \end{Phonetics}
\end{Entry}

\begin{Entry}{同事}{6,8}{⼝,⼅}
  \begin{Phonetics}{同事}{tong2shi4}[][HSK 2]
    \definition[个,位,名]{s.}{companheiro; colega; colega de trabalho; pessoas que trabalham juntas}
    \definition{v.}{trabalhar no mesmo lugar; trabalhar juntos; trabalhar na mesma unidade}
  \end{Phonetics}
\end{Entry}

\begin{Entry}{同学}{6,8}{⼝,⼦}
  \begin{Phonetics}{同学}{tong2xue2}[][HSK 1]
    \definition[位,个,些]{s.}{colega de escola; colega de turma; colega de estudos; pessoas que estudam na mesma escola}
  \end{Phonetics}
\end{Entry}

\begin{Entry}{同性恋}{6,8,10}{⼝,⼼,⼼}
  \begin{Phonetics}{同性恋}{tong2xing4lian4}
    \definition{s.}{homossexualidade | pessoa gay | amor gay}
  \end{Phonetics}
\end{Entry}

\begin{Entry}{同屋}{6,9}{⼝,⼫}
  \begin{Phonetics}{同屋}{tong2wu1}
    \definition[个]{s.}{companheiro de quarto | colega de quarto}
  \end{Phonetics}
\end{Entry}

\begin{Entry}{同砚}{6,9}{⼝,⽯}
  \begin{Phonetics}{同砚}{tong2yan4}
    \definition[位,个]{s.}{colega de classe | colega estudante}
  \end{Phonetics}
\end{Entry}

\begin{Entry}{同类}{6,9}{⼝,⽶}
  \begin{Phonetics}{同类}{tong2lei4}[][HSK 7-9]
    \definition{adj.}{similar; semelhante; do mesmo tipo; pessoas ou coisas do mesmo tipo}
  \synonymref{同样}{tong2yang4}
  \end{Phonetics}
\end{Entry}

\begin{Entry}{同胞}{6,9}{⼝,⾁}
  \begin{Phonetics}{同胞}{tong2bao1}[][HSK 6]
    \definition{s.}{nascidos dos mesmos pais | compatriota; conterrâneo; pessoas do mesmo país ou etnia}
  \end{Phonetics}
\end{Entry}

\begin{Entry}{同样}{6,10}{⼝,⽊}
  \begin{Phonetics}{同样}{tong2yang4}[][HSK 2]
    \definition{adj.}{igual; semelhante; similar; idêntico; sem diferença}
  \end{Phonetics}
\end{Entry}

\begin{Entry}{同流合污}{6,10,6,6}{⼝,⽔,⼝,⽔}
  \begin{Phonetics}{同流合污}{tong2liu2he2wu1}
    \definition{expr.}{chafurdar na lama com alguém | seguir o mau exemplo dos outros}
  \end{Phonetics}
\end{Entry}

\begin{Entry}{同情}{6,11}{⼝,⼼}
  \begin{Phonetics}{同情}{tong2qing2}[][HSK 4]
    \definition{s.}{simpatia}
    \definition{v.}{simpatizar com; solidarizar-se; compadecer-se; ter empatia emocional pelo que os outros estão passando}
  \end{Phonetics}
\end{Entry}

\begin{Entry}{同期}{6,12}{⼝,⽉}
  \begin{Phonetics}{同期}{tong2qi1}[][HSK 6]
    \definition{s.}{o período correspondente; o mesmo período; no mesmo tempo}
  \end{Phonetics}
\end{Entry}

\begin{Entry}{同等}{6,12}{⼝,⽵}
  \begin{Phonetics}{同等}{tong2deng3}[][HSK 7-9]
    \definition{adj.}{igual; que pertence à mesma classe social ou posição social; pessoas da mesma posição ou \emph{status}}
  \synonymref{平等}{ping2deng3}
  \synonymref{一律}{yi2lv4}
  \end{Phonetics}
\end{Entry}

\begin{Entry}{同意}{6,13}{⼝,⼼}
  \begin{Phonetics}{同意}{tong2yi4}[][HSK 3]
    \definition{v.}{concordar; consentir; aprovar; concordar com; dizer sim}
  \end{Phonetics}
\end{Entry}

\begin{Entry}{同感}{6,13}{⼝,⼼}
  \begin{Phonetics}{同感}{tong2gan3}[][HSK 7-9]
    \definition{s.}{consenso; simpatia; os mesmos pensamentos ou sentimentos}
  \end{Phonetics}
\end{Entry}

\begin{Entry}{同盟}{6,13}{⼝,⽫}
  \begin{Phonetics}{同盟}{tong2meng2}[][HSK 7-9]
    \definition[个]{s.}{aliança; liga}
  \synonymref{合作}{he2zuo4}
  \synonymref{联盟}{lian2meng2}
  \synonymref{同伙}{tong2huo3}
  \synonymref{战友}{zhan4you3}
  \end{Phonetics}
\end{Entry}

%%%%%%%%%% 名 %%%%%%%%%%
\subsection*{名}\addcontentsline{loh}{figure}{名}

\begin{Entry}{名}{6}{⼝}
  \begin{Phonetics}{名}{ming2}[][HSK 2]
    \definition*{s.}{Sobrenome: Ming}
    \definition{adj.}{notável; famoso; conhecido; renomado}
    \definition{clas.}{usado para pessoas | usado para classificação por ordem}
    \definition{s.}{nome; denominação | desculpa; pretexto | fama; reputação}
    \definition{v.}{nome próprio (é) | expressar; descrever | possuir; tomar; ter}
  \end{Phonetics}
\end{Entry}

\begin{Entry}{名人}{6,2}{⼝,⼈}
  \begin{Phonetics}{名人}{ming2ren2}[][HSK 4]
    \definition[位,个]{s.}{celebridade; pessoa famosa}
  \end{Phonetics}
\end{Entry}

\begin{Entry}{名义}{6,3}{⼝,⼂}
  \begin{Phonetics}{名义}{ming2yi4}[][HSK 6]
    \definition{s.}{nominal; em nome (geralmente seguido por 上); um nome ou título usado como base para fazer algo}[有人盗用我名义申请贷款。===Alguém solicitou um empréstimo em meu nome. | 他们只是名义上的夫妻。===Eles são marido e mulher apenas no nome.]
  \seealsoref{上}{shang4}
  \end{Phonetics}
\end{Entry}

\begin{Entry}{名气}{6,4}{⼝,⽓}
  \begin{Phonetics}{名气}{ming2qi5}[][HSK 7-9]
    \definition{s.}{nome; fama; reputação}
  \end{Phonetics}
\end{Entry}

\begin{Entry}{名片}{6,4}{⼝,⽚}
  \begin{Phonetics}{名片}{ming2pian4}[][HSK 4]
    \definition[张,盒,叠]{s.}{cartão de visita; um pedaço de papel retangular com o nome, o cargo, o endereço etc. impressos}
  \end{Phonetics}
\end{Entry}

\begin{Entry}{名字}{6,6}{⼝,⼦}
  \begin{Phonetics}{名字}{ming2zi5}[][HSK 1]
    \definition[个]{s.}{nome; nome próprio | nome (de uma coisa)}
  \end{Phonetics}
\end{Entry}

\begin{Entry}{名利}{6,7}{⼝,⼑}
  \begin{Phonetics}{名利}{ming2li4}[][HSK 7-9]
    \definition{s.}{fama e ganho; fama e riqueza | fama e dinheiro; refere"-se ao status e aos interesses de um indivíduo}
  \end{Phonetics}
\end{Entry}

\begin{Entry}{名声}{6,7}{⼝,⼠}
  \begin{Phonetics}{名声}{ming2sheng1}[][HSK 7-9]
    \definition{s.}{reputação; renome; prestígio; comentários amplamente divulgados na sociedade}
  \end{Phonetics}
\end{Entry}

\begin{Entry}{名言}{6,7}{⼝,⾔}
  \begin{Phonetics}{名言}{ming2yan2}[][HSK 7-9]
    \definition{s.}{ditados; comentário famoso; frases ou expressões famosas}
  \end{Phonetics}
\end{Entry}

\begin{Entry}{名单}{6,8}{⼝,⼗}
  \begin{Phonetics}{名单}{ming2dan1}[][HSK 2]
    \definition[个,份]{s.}{lista com nomes de pessoas ou nomes de organizações}
  \end{Phonetics}
\end{Entry}

\begin{Entry}{名胜}{6,9}{⼝,⾁}
  \begin{Phonetics}{名胜}{ming2sheng4}[][HSK 6]
    \definition[处,个]{s.}{pontos turísticos; atrações famosas; lugares famosos com locais históricos ou belas paisagens}
  \end{Phonetics}
\end{Entry}

\begin{Entry}{名贵}{6,9}{⼝,⾙}
  \begin{Phonetics}{名贵}{ming2gui4}[][HSK 7-9]
    \definition{adj.}{famoso e precioso; raro}
  \end{Phonetics}
\end{Entry}

\begin{Entry}{名流}{6,10}{⼝,⽔}
  \begin{Phonetics}{名流}{ming2liu2}
    \definition{s.}{personalidades ilustres; celebridades; figuras proeminentes (referindo-se principalmente a pessoas do meio acadêmico e político)}
  \synonymref{名人}{ming2ren2}
  \synonymref{绅士}{shen1shi4}
  \end{Phonetics}
\end{Entry}

\begin{Entry}{名称}{6,10}{⼝,⽲}
  \begin{Phonetics}{名称}{ming2cheng1}[][HSK 2]
    \definition[个,种]{s.}{nomes, apelidos e formas de se referir a pessoas ou coisas}
  \end{Phonetics}
\end{Entry}

\begin{Entry}{名副其实}{6,11,8,8}{⼝,⼑,⼋,⼧}
  \begin{Phonetics}{名副其实}{ming2fu4qi2shi2}[][HSK 7-9]
    \definition{expr.}{``Faz jus ao seu nome.''; ser digno do nome; ser algo na realidade, bem como no nome; ser digno da própria reputação; em nome e de fato; no verdadeiro sentido do termo; a reputação de alguém é justificada; o nome corresponde à realidade; merecer verdadeiramente o seu nome; fiel ao próprio nome}
  \end{Phonetics}
\end{Entry}

\begin{Entry}{名著}{6,11}{⼝,⽬}
  \begin{Phonetics}{名著}{ming2zhu4}[][HSK 7-9]
    \definition{s.}{clássico; livro famoso; obra famosa; obra-prima; obras valiosas e famosas}
  \end{Phonetics}
\end{Entry}

\begin{Entry}{名牌儿}{6,12,2}{⼝,⽚,⼉}
  \begin{Phonetics}{名牌儿}{ming2pai2r5}[][HSK 4]
    \definition{s.}{marca famosa}
  \end{Phonetics}
\end{Entry}

\begin{Entry}{名誉}{6,13}{⼝,⾔}
  \begin{Phonetics}{名誉}{ming2yu4}[][HSK 6]
    \definition{adj.}{honorário; nominal (geralmente se refere ao nome de um presente, com um sentido de respeito)}[他是学校的名誉教授。===Ele é professor honorário da escola.]
    \definition{s.}{fama; reputação; honra}[名誉才是最神圣的。===Reputação é a coisa mais sagrada. | 我用自己的名誉发誓。===Juro pela minha honra.]
  \end{Phonetics}
\end{Entry}

\begin{Entry}{名额}{6,15}{⼝,⾴}
  \begin{Phonetics}{名额}{ming2'e2}[][HSK 6]
    \definition[个]{s.}{cota de pessoas; número de pessoas designadas ou permitidas; número necessário de pessoal}
  \end{Phonetics}
\end{Entry}

%%%%%%%%%% 后 %%%%%%%%%%
\subsection*{后}\addcontentsline{loh}{figure}{后}

\begin{Entry}{后}{6}{⼝}
  \begin{Phonetics}{后}{hou4}[][HSK 1]
    \definition*{s.}{Sobrenome: Hou}
    \definition{s.}{atrás; traseiro; a direção oposta àquela para a qual a pessoa está voltada; a direção oposta àquela para a qual a parte de trás de uma casa está voltada | depois; mais tarde no tempo; futuro | último | posteridade; descendência | rainha; imperatriz | governante; soberano; monarca antigo}
  \antonymref{前}{qian2}
  \antonymref{先}{xian1}
  \end{Phonetics}
\end{Entry}

\begin{Entry}{后人}{6,2}{⼝,⼈}
  \begin{Phonetics}{后人}{hou4ren2}[][HSK 7-9]
    \definition{s.}{gerações posteriores; gerações futuras | posteridade; descendentes; futuridade}
  \end{Phonetics}
\end{Entry}

\begin{Entry}{后天}{6,4}{⼝,⼤}
  \begin{Phonetics}{后天}{hou4tian1}[][HSK 1]
    \definition{s.}{depois de amanhã; período em que uma pessoa ou animal vive e cresce sozinho após deixar o útero materno}
  \antonymref{先天}{xian1tian1}
  \end{Phonetics}
\end{Entry}

\begin{Entry}{后方}{6,4}{⼝,⽅}
  \begin{Phonetics}{后方}{hou4 fang1}
    \definition{s.}{traseira; retaguarda | na parte de trás; na parte traseira}
  \antonymref{前方}{qian2fang1}
  \antonymref{前线}{qian2xian4}
  \end{Phonetics}
\end{Entry}

\begin{Entry}{后代}{6,5}{⼝,⼈}
  \begin{Phonetics}{后代}{hou4dai4}[][HSK 7-9]
    \definition{s.}{períodos posteriores (na história); eras posteriores; a era após uma certa era | gerações posteriores; posteridade; descendentes; gerações futuras}
  \end{Phonetics}
\end{Entry}

\begin{Entry}{后台}{6,5}{⼝,⼝}
  \begin{Phonetics}{后台}{hou4tai2}[][HSK 7-9]
    \definition{s.}{bastidores; plano de fundo | apoiador de bastidores; apoiador dos bastidores; uma metáfora para uma pessoa ou grupo que manipula ou apoia algo nos bastidores}
  \end{Phonetics}
\end{Entry}

\begin{Entry}{后头}{6,5}{⼝,⼤}
  \begin{Phonetics}{后头}{hou4tou5}[][HSK 4]
    \definition{adv.}{posteriormente; atrás; mais tarde}
    \definition{s.}{a parte de trás; a parte traseira}
  \end{Phonetics}
\end{Entry}

\begin{Entry}{后边}{6,5}{⼝,⾡}
  \begin{Phonetics}{后边}{hou4bian5}[][HSK 1]
    \definition{adv.}{costas; traseira; atrás}
  \end{Phonetics}
\end{Entry}

\begin{Entry}{后年}{6,6}{⼝,⼲}
  \begin{Phonetics}{后年}{hou4nian2}[][HSK 3]
    \definition{s.}{daqui a dois anos; no ano seguinte ao próximo ano}
  \end{Phonetics}
\end{Entry}

\begin{Entry}{后来}{6,7}{⼝,⽊}
  \begin{Phonetics}{后来}{hou4lai2}[][HSK 2]
    \definition{adv.}{mais tarde; depois; refere"-se a um período posterior a um determinado momento no passado}
  \end{Phonetics}
\end{Entry}

\begin{Entry}{后备}{6,8}{⼝,⼡}
  \begin{Phonetics}{后备}{hou4bei4}[][HSK 7-9]
    \definition{s.}{reserva; preparado para reabastecimento (pessoal, suprimentos, etc.)}
  \end{Phonetics}
\end{Entry}

\begin{Entry}{后备箱}{6,8,15}{⼝,⼡,⾋}
  \begin{Phonetics}{后备箱}{hou4bei4xiang1}[][HSK 7-9]
    \definition{s.}{porta-malas (de um carro)}
  \end{Phonetics}
\end{Entry}

\begin{Entry}{后果}{6,8}{⼝,⽊}
  \begin{Phonetics}{后果}{hou4guo3}[][HSK 3]
    \definition{s.}{consequência; resultado (geralmente negativo)}
  \end{Phonetics}
\end{Entry}

\begin{Entry}{后者}{6,8}{⼝,⽼}
  \begin{Phonetics}{后者}{hou4zhe3}[][HSK 7-9]
    \definition{pron.}{o último; a última de duas ou mais pessoas ou coisas mencionadas ou autoevidentes}
    \definition{s.}{o último}
  \antonymref{前者}{qian2zhe3}
  \end{Phonetics}
\end{Entry}

\begin{Entry}{后盾}{6,9}{⼝,⽬}
  \begin{Phonetics}{后盾}{hou4dun4}[][HSK 7-9]
    \definition{s.}{apoio; força de apoio | suporte; assistência; suporte; apoiador}
  \end{Phonetics}
\end{Entry}

\begin{Entry}{后退}{6,9}{⼝,⾡}
  \begin{Phonetics}{后退}{hou4tui4}[][HSK 7-9]
    \definition{v.}{recuar; retrocerder; retornar (para um lugar posterior ou para um estágio anterior de desenvolvimento)}
  \end{Phonetics}
\end{Entry}

\begin{Entry}{后面}{6,9}{⼝,⾯}
  \begin{Phonetics}{后面}{hou4mian5}[][HSK 3]
    \definition{adv.}{parte de trás; retaguarda; atrás; a parte posterior do espaço ou localização | mais tarde; depois; no futuro; a parte posterior de um artigo ou discurso em relação ao que está sendo narrado no momento}
  \end{Phonetics}
\end{Entry}

\begin{Entry}{后悔}{6,10}{⼝,⼼}
  \begin{Phonetics}{后悔}{hou4hui3}[][HSK 5]
    \definition{v.}{lamentar; ter remorso; arrepender-se}
  \end{Phonetics}
\end{Entry}

\begin{Entry}{后顾之忧}{6,10,3,7}{⼝,⾴,⼂,⼼}
  \begin{Phonetics}{后顾之忧}{hou4gu4zhi1you1}[][HSK 7-9]
    \definition{expr.}{preocupações com o que ficou para trás; preocupações com problemas futuros; preocupações persistentes; preocupações não resolvidas; preocupações ou problemas potenciais; ``Ansiedade que exige olhar para trás.''; refere"-se a preocupações com o lar, a família ou o futuro que surgem ao seguir em frente ou sair}
  \end{Phonetics}
\end{Entry}

\begin{Entry}{后续}{6,11}{⼝,⽷}
  \begin{Phonetics}{后续}{hou4xu4}[][HSK 7-9]
    \definition{adj.}{subsequente; de acompanhamento; decorrente; seguimento}
    \definition{v.}{casar novamente após a morte da esposa}
  \end{Phonetics}
\end{Entry}

\begin{Entry}{后期}{6,12}{⼝,⽉}
  \begin{Phonetics}{后期}{hou4qi1}[][HSK 7-9]
    \definition{s.}{estágio posterior; período posterior; a última fase de um período}
  \end{Phonetics}
\end{Entry}

\begin{Entry}{后遗症}{6,12,10}{⼝,⾡,⽧}
  \begin{Phonetics}{后遗症}{hou4yi2zheng4}[][HSK 7-9]
    \definition{s.}{sequelas; sintomas como defeitos ou disfunções orgânicas que permanecem após a recuperação de certas doenças | ressaca; efeito colateral; consequência; efeito residual}
  \end{Phonetics}
\end{Entry}

\begin{Entry}{后勤}{6,13}{⼝,⼒}
  \begin{Phonetics}{后勤}{hou4qin2}[][HSK 7-9]
    \definition{s.}{logística; serviços de retaguarda; todo o trabalho de fornecimento de áreas distantes da linha de frente para as áreas de linha de frente; trabalho administrativo em agências governamentais, empresas, etc., incluindo finanças, reparos, etc.}
  \end{Phonetics}
\end{Entry}

\begin{Entry}{后裔}{6,13}{⼝,⾐}
  \begin{Phonetics}{后裔}{hou4yi4}[][HSK 7-9]
    \definition{s.}{descendente (de uma pessoa morta); prole | descendente; posteridade; progênie}
  \end{Phonetics}
\end{Entry}

%%%%%%%%%% 吐 %%%%%%%%%%
\subsection*{吐}\addcontentsline{loh}{figure}{吐}

\begin{Entry}{吐}{6}{⼝}
  \begin{Phonetics}{吐}{tu3}[][HSK 5]
    \definition{v.}{cuspir; sair pela boca | surgir ou aparecer pela boca ou por uma fenda | dizer; contar; falar abertamente}
  \end{Phonetics}
  \begin{Phonetics}{吐}{tu4}[][HSK 5]
    \definition{v.}{vomitar; sair pela boca | vomitar; expelir; metáfora para ser forçado a devolver bens usurpados}
  \end{Phonetics}
\end{Entry}

%%%%%%%%%% 向 %%%%%%%%%%
\subsection*{向}\addcontentsline{loh}{figure}{向}

\begin{Entry}{向}{6}{⼝}
  \begin{Phonetics}{向}{xiang4}[][HSK 2]
    \definition*{s.}{Sobrenome: Xiang}
    \definition{adv.}{sempre; o tempo todo}
    \definition{prep.}{em direção a; para}
    \definition{s.}{direção | a janela voltada para o norte}
    \definition{v.}{encarar; virar-se para | estar do lado de; ser parcial com; tomar o partido de alguém}
  \end{Phonetics}
\end{Entry}

\begin{Entry}{向上}{6,3}{⼝,⼀}
  \begin{Phonetics}{向上}{xiang4shang4}[][HSK 5]
    \definition{adv.}{o superior; acima}
    \definition{v.}{mover-se; subir; ir para um lugar mais alto; ir para um lugar mais alto em relação a um determinado ponto; ir para um desenvolvimento mais alto que o atual | avançar; continuar se aperfeiçoar; subir na vida; desenvolver-se em direção ao progresso}
  \end{Phonetics}
\end{Entry}

\begin{Entry}{向导}{6,6}{⼝,⼨}
  \begin{Phonetics}{向导}{xiang4dao3}[][HSK 5]
    \definition[位]{s.}{guia; a pessoa que lidera todos e lhes indica a direção ao caminhar}
    \definition{v.}{agir como um guia; mostrar a alguém o caminho; levar alguém a algum lugar}
  \end{Phonetics}
\end{Entry}

\begin{Entry}{向汪}{6,7}{⼝,⽔}
  \begin{Phonetics}{向汪}{xiang4wang1}
    \definition{v.}{esperar que}
  \end{Phonetics}
\end{Entry}

\begin{Entry}{向往}{6,8}{⼝,⼻}
  \begin{Phonetics}{向往}{xiang4wang3}
    \definition{v.}{ansiar por | esperar ansiosamente por}
  \end{Phonetics}
\end{Entry}

\begin{Entry}{向前}{6,9}{⼝,⼑}
  \begin{Phonetics}{向前}{xiang4qian2}[][HSK 5]
    \definition{adv.}{para frente; adiante}
    \definition{v.}{avançar; ir em direção à frente; mover-se para frente; avançar um pouco mais}
  \end{Phonetics}
\end{Entry}

%%%%%%%%%% 吓 %%%%%%%%%%
\subsection*{吓}\addcontentsline{loh}{figure}{吓}

\begin{Entry}{吓}{6}{⼝}
  \begin{Phonetics}{吓}{xia4}[][HSK 5]
    \definition{interj.}{demonstrar espanto; expressar insatisfação}
    \definition{v.}{ameaçar; intimidar; usar ameaças ou meios coercitivos para intimidar ou assustar}
  \end{Phonetics}
\end{Entry}

\begin{Entry}{吓人}{6,2}{⼝,⼈}
  \begin{Phonetics}{吓人}{xia4/ren2}
    \definition{adj.}{apavorante | assustador}
    \definition{v.+compl.}{assustar-se | tomar um susto}
  \end{Phonetics}
\end{Entry}

%%%%%%%%%% 吗 %%%%%%%%%%
\subsection*{吗}\addcontentsline{loh}{figure}{吗}

\begin{Entry}{吗}{6}{⼝}
  \begin{Phonetics}{吗}{ma2}
    \definition{adv.}{(coloquial) que?}
  \end{Phonetics}
  \begin{Phonetics}{吗}{ma3}
    \definition{s.}{usada em 吗啡, morfina}
  \seealsoref{吗啡}{ma3fei1}
  \end{Phonetics}
  \begin{Phonetics}{吗}{ma5}[][HSK 1]
    \definition{part.}{usado no final de uma pergunta | como uma pausa em uma frase antes de introduzir o ponto principal | usado no final de uma pergunta retórica}
  \end{Phonetics}
\end{Entry}

\begin{Entry}{吗啡}{6,11}{⼝,⼝}
  \begin{Phonetics}{吗啡}{ma3fei1}
    \definition{s.}{morfina (empréstimo linguístico)}
  \end{Phonetics}
\end{Entry}

%%%%%%%%%% 吸 %%%%%%%%%%
\subsection*{吸}\addcontentsline{loh}{figure}{吸}

\begin{Entry}{吸}{6}{⼝}
  \begin{Phonetics}{吸}{xi1}[][HSK 4]
    \definition{v.}{inalar; inspirar; aspirar | sugar (líquidos) | absorver; sugar | atrair; atrair para si mesmo | aspirar; introdução de líquidos, gases, etc. no corpo}
  \antonymref{呼}{hu1}
  \end{Phonetics}
\end{Entry}

\begin{Entry}{吸引}{6,4}{⼝,⼸}
  \begin{Phonetics}{吸引}{xi1yin3}[][HSK 4]
    \definition{v.}{atrair; apelar para; chamar a atenção de outros objetos, forças ou pessoas para si mesmo}
  \end{Phonetics}
\end{Entry}

\begin{Entry}{吸收}{6,6}{⼝,⽁}
  \begin{Phonetics}{吸收}{xi1shou1}[][HSK 4]
    \definition{v.}{imbuir; absorver; assimilar; sugar;  chupar; (animais, plantas, etc.) extrair material de fora dos tecidos para o interior dos tecidos | absorver; chupar;  sugar alguma substância de fora para dentro | recrutar; alistar; inscrever-se; matricular-se; admitir; (organizações ou coletivos) aceitar novos membros | absorver; aproveitar e usar a experiência, o conhecimento, o dinheiro e outras coisas valiosas de outras pessoas | absorver; diminuir, atenuar ou eliminar determinados efeitos ou fenômenos}
  \end{Phonetics}
\end{Entry}

\begin{Entry}{吸毒}{6,9}{⼝,⽏}
  \begin{Phonetics}{吸毒}{xi1 du2}[][HSK 6]
    \definition{s.}{droga}
    \definition{v.}{usar drogas viciantes; ser viciado em um narcótico; consumir drogas}
  \end{Phonetics}
\end{Entry}

\begin{Entry}{吸烟}{6,10}{⼝,⽕}
  \begin{Phonetics}{吸烟}{xi1/yan1}[][HSK 4]
    \definition{v.+compl.}{fumar}
  \end{Phonetics}
\end{Entry}

\begin{Entry}{吸铁石}{6,10,5}{⼝,⾦,⽯}
  \begin{Phonetics}{吸铁石}{xi1tie3shi2}
    \definition{s.}{imã | magneto}
  \seealsoref{磁铁}{ci2tie3}
  \end{Phonetics}
\end{Entry}

\begin{Entry}{吸管}{6,14}{⼝,⽵}
  \begin{Phonetics}{吸管}{xi1guan3}[][HSK 4]
    \definition[根,个,支]{s.}{tubo de sucção; sugador; canudo (para beber); refere"-se ao tubo fino usado para sugar bebidas | conta"-gotas; pipeta; cateter para bombeamento de líquidos usando pressão de ar}
  \end{Phonetics}
\end{Entry}

%%%%%%%%%% 回 %%%%%%%%%%
\subsection*{回}\addcontentsline{loh}{figure}{回}

\begin{Entry}{回}{6}{⼞}
  \begin{Phonetics}{回}{hui2}[][HSK 1,2]
    \definition*{s.}{Sobrenome: Hui}
    \definition*{s.}{Etnia Hui (mulçumanos chineses)}
    \definition{clas.}{usado para coisas, ações, número de vezes |  um trecho de um conto; um capítulo de um romance em capítulos | seção ou capítulo (de um livro clássico)}
    \definition{v.}{circular; enrolar | retornar; voltar; voltar ao lugar de origem | dar meia-volta | responder; contestar | relatar; reportar; responder}
  \end{Phonetics}
\end{Entry}

\begin{Entry}{回升}{6,4}{⼞,⼗}
  \begin{Phonetics}{回升}{hui2sheng1}[][HSK 7-9]
    \definition{v.}{levantar"-se novamente (após uma queda); levantar-se | recuperar; cair e depois subir novamente}
  \antonymref{回落}{hui2luo4}
  \end{Phonetics}
\end{Entry}

\begin{Entry}{回忆}{6,4}{⼞,⼼}
  \begin{Phonetics}{回忆}{hui2yi4}[][HSK 5]
    \definition[个,段]{s.}{memória; lembrança de eventos ou experiências passadas}
    \definition{v.}{lembrar; recordar}
  \end{Phonetics}
\end{Entry}

\begin{Entry}{回忆录}{6,4,8}{⼞,⼼,⼹}
  \begin{Phonetics}{回忆录}{hui2yi4lu4}[][HSK 7-9]
    \definition{s.}{memórias; reminiscências; lembranças; um gênero de escrita que relata experiências pessoais ou eventos históricos familiares}
  \end{Phonetics}
\end{Entry}

\begin{Entry}{回去}{6,5}{⼞,⼛}
  \begin{Phonetics}{回去}{hui2 qu5}[][HSK 1]
    \definition{v.}{retornar; voltar; estar de volta; (a partir da minha localização)}
  \end{Phonetics}
\end{Entry}

\begin{Entry}{回头}{6,5}{⼞,⼤}
  \begin{Phonetics}{回头}{hui2tou2}[][HSK 5]
    \definition{adv.}{mais tarde; depois de um tempo}
    \definition{conj.}{ou então; usado no início da segunda metade de uma frase para indicar o que acontecerá se você não fizer o que fez na primeira metade da frase}
    \definition{v.}{dar a meia-volta; virar a cabeça; virar a cabeça para trás | retornar; voltar | arrepender-se; corrigir seu caminho; reconhecer e corrigir erros}
  \end{Phonetics}
\end{Entry}

\begin{Entry}{回归}{6,5}{⼞,⼹}
  \begin{Phonetics}{回归}{hui2gui1}[][HSK 7-9]
    \definition{v.}{retornar; regredir; retornar para (local original, organização, etc.)}
  \end{Phonetics}
\end{Entry}

\begin{Entry}{回扣}{6,6}{⼞,⼿}
  \begin{Phonetics}{回扣}{hui2kou4}[][HSK 7-9]
    \definition[笔,的]{s.}{propina; desconto; comissão sobre vendas (para agente)}
  \end{Phonetics}
\end{Entry}

\begin{Entry}{回收}{6,6}{⼞,⽁}
  \begin{Phonetics}{回收}{hui2shou1}[][HSK 5]
    \definition{v.}{reciclar; reciclar itens (geralmente resíduos ou produtos antigos) para reutilização | recuperar; recolher; recuperar o que foi emitido ou demitido}
  \end{Phonetics}
\end{Entry}

\begin{Entry}{回应}{6,7}{⼞,⼴}
  \begin{Phonetics}{回应}{hui2ying4}[][HSK 6]
    \definition{v.}{responder}
  \end{Phonetics}
\end{Entry}

\begin{Entry}{回报}{6,7}{⼞,⼿}
  \begin{Phonetics}{回报}{hui2bao4}[][HSK 5]
    \definition{s.}{recompensa; pagamento; benefícios recebidos como resultado de assistência, esforço ou afeto | retornos; benefícios recebidos por meio de investimentos}
    \definition{v.}{pagar de volta; beneficiar pessoas ou organizações que os ajudaram ou cuidaram deles de alguma forma}
  \end{Phonetics}
\end{Entry}

\begin{Entry}{回来}{6,7}{⼞,⽊}
  \begin{Phonetics}{回来}{hui2 lai5}[][HSK 1]
    \definition{v.}{voltar; regressar (para a minha localização) | retornar; usado após um verbo, significa ``vir ao lugar original''}
  \end{Phonetics}
\end{Entry}

\begin{Entry}{回到}{6,8}{⼞,⼑}
  \begin{Phonetics}{回到}{hui2dao4}[][HSK 1]
    \definition{v.}{retornar para; voltar e chegar (ao lugar onde estava originalmente); (após uma mudança nas circunstâncias) retornar ao estado original}
  \end{Phonetics}
\end{Entry}

\begin{Entry}{回味}{6,8}{⼞,⼝}
  \begin{Phonetics}{回味}{hui2wei4}[][HSK 7-9]
    \definition{s.}{sabor residual; recordar o sabor agradável de\dots; o gosto residual que fica na boca depois de comer}
    \definition{v.}{recordar e ponderar sobre; reviver coisas que você vivenciou ou com as quais entrou em contato}
  \end{Phonetics}
\end{Entry}

\begin{Entry}{回国}{6,8}{⼞,⼞}
  \begin{Phonetics}{回国}{hui2 guo2}[][HSK 2]
    \definition{v.}{regressar ao seu país (terra natal); referindo"-se a voltar do exterior}
  \end{Phonetics}
\end{Entry}

\begin{Entry}{回信}{6,9}{⼞,⼈}
  \begin{Phonetics}{回信}{hui2/xin4}[][HSK 5]
    \definition[封]{s.}{uma carta em resposta; uma mensagem verbal em resposta}
    \definition{v.+compl.}{escrever em resposta; escrever de volta; responder uma carta; responder verbalmente uma mensagem}
  \end{Phonetics}
\end{Entry}

\begin{Entry}{回复}{6,9}{⼞,⼢}
  \begin{Phonetics}{回复}{hui2fu4}[][HSK 4]
    \definition{v.}{responder (a uma carta) | retornar ao estado normal; restaurar algo ao seu estado original}
  \end{Phonetics}
\end{Entry}

\begin{Entry}{回首}{6,9}{⼞,⾸}
  \begin{Phonetics}{回首}{hui2shou3}[][HSK 7-9]
    \definition{v.}{virar a cabeça; virar-se (em volta) | olhar para trás; lembrar-se; recordar}
  \end{Phonetics}
\end{Entry}

\begin{Entry}{回家}{6,10}{⼞,⼧}
  \begin{Phonetics}{回家}{hui2 jia1}[][HSK 1]
    \definition{v.}{ir (voltar) para casa; estar em casa; voltar para casa}
  \end{Phonetics}
\end{Entry}

\begin{Entry}{回顾}{6,10}{⼞,⾴}
  \begin{Phonetics}{回顾}{hui2gu4}[][HSK 5]
    \definition{v.}{olhar para trás | revisar; fazer uma retrospectiva; olhar para trás, pensar no passado}
  \end{Phonetics}
\end{Entry}

\begin{Entry}{回旋}{6,11}{⼞,⽅}
  \begin{Phonetics}{回旋}{hui2xuan2}
    \definition{v.}{circular | rodar | dar a volta}
  \end{Phonetics}
\end{Entry}

\begin{Entry}{回答}{6,12}{⼞,⽵}
  \begin{Phonetics}{回答}{hui2da2}[][HSK 1]
    \definition[个]{s.}{resposta}
    \definition{v.}{responder; explicar a questão; expressar opinião sobre a solicitação}
  \end{Phonetics}
\end{Entry}

\begin{Entry}{回落}{6,12}{⼞,⾋}
  \begin{Phonetics}{回落}{hui2luo4}[][HSK 7-9]
    \definition{v.}{(níveis de água, preços, etc.) cair após uma subida; diminuir | recuar | retornar ao nível baixo após uma subida (no nível da água, preço etc.)}
  \antonymref{回升}{hui2sheng1}
  \end{Phonetics}
\end{Entry}

\begin{Entry}{回馈}{6,12}{⼞,⾶}
  \begin{Phonetics}{回馈}{hui2kui4}[][HSK 7-9]
    \definition{v.}{retribuir; recompensar | dar \emph{feedback}; fornecer \emph{feedback}; dar retorno; dar parecer}
  \end{Phonetics}
\end{Entry}

\begin{Entry}{回想}{6,13}{⼞,⼼}
  \begin{Phonetics}{回想}{hui2xiang3}[][HSK 7-9]
    \definition{v.}{recordar; pensar de novo; pensar (no passado)}
  \end{Phonetics}
\end{Entry}

\begin{Entry}{回避}{6,16}{⼞,⾌}
  \begin{Phonetics}{回避}{hui2bi4}[][HSK 5]
    \definition{v.}{fugir (de um problema); em direito, refere"-se especificamente à não participação nos procedimentos de um caso de um oficial de justiça, etc., que tenha interesse no caso ou nas partes do caso | esquivar"-se; evadir"-se; evitar (encontrar alguém)}
  \end{Phonetics}
\end{Entry}

%%%%%%%%%% 因 %%%%%%%%%%
\subsection*{因}\addcontentsline{loh}{figure}{因}

\begin{Entry}{因}{6}{⼞}
  \begin{Phonetics}{因}{yin1}[][HSK 6]
    \definition*{s.}{Sobrenome: Yin}
    \definition{conj.}{porque; orações de conexão, indicando relações de causa e efeito}
    \definition{prep.}{com base em; à luz de; de acordo com; a introdução da ação comportamental equivale a 按照 ou 根据}
    \definition{s.}{causa; motivo; condições em que algo ocorre ou causa um determinado resultado}
    \definition{v.}{seguir; continuar; fazer como sempre fez | estar em conformidade com; estar de acordo com; depender; contar com}
  \seealsoref{按照}{an4zhao4}
  \seealsoref{根据}{gen1ju4}
  \antonymref{果}{guo3}
  \end{Phonetics}
\end{Entry}

\begin{Entry}{因为}{6,4}{⼞,⼂}
  \begin{Phonetics}{因为}{yin1wei5}[][HSK 2]
    \definition{conj.}{porque; indica o motivo e a frase seguinte indica o resultado}
    \definition{prep.}{por causa de; por conta de; indica razão ou justificativa}
  \end{Phonetics}
\end{Entry}

\begin{Entry}{因为……所以……}{6,4,8,4}{⼞,⼂,⼾,⼈}
  \begin{Phonetics}{因为……所以……}{yin1wei5 suo3yi3}
    \definition{conj.}{porque\dots portanto\dots}
  \end{Phonetics}
\end{Entry}

\begin{Entry}{因此}{6,6}{⼞,⽌}
  \begin{Phonetics}{因此}{yin1ci3}[][HSK 3]
    \definition{conj.}{assim; portanto; consequentemente}
  \end{Phonetics}
\end{Entry}

\begin{Entry}{因此就}{6,6,12}{⼞,⽌,⼪}
  \begin{Phonetics}{因此就}{yin1ci3 jiu4}
    \definition{conj.}{portanto}
  \end{Phonetics}
\end{Entry}

\begin{Entry}{因而}{6,6}{⼞,⽽}
  \begin{Phonetics}{因而}{yin1'er2}[][HSK 5]
    \definition{conj.}{assim; como resultado; com o resultado que; conecta frases, indicando relação de causa e efeito}
  \end{Phonetics}
\end{Entry}

\begin{Entry}{因素}{6,10}{⼞,⽷}
  \begin{Phonetics}{因素}{yin1su4}[][HSK 6]
    \definition[个,种]{s.}{fator; elemento; os componentes que constituem a essência das coisas | fator; as razões ou condições que determinam o sucesso ou o fracasso de algo}
  \end{Phonetics}
\end{Entry}

%%%%%%%%%% 团 %%%%%%%%%%
\subsection*{团}\addcontentsline{loh}{figure}{团}

\begin{Entry}{团}{6}{⼞}
  \begin{Phonetics}{团}{tuan2}[][HSK 3]
    \definition*{s.}{Liga da Juventude Comunista da China; Liga}
    \definition{adj.}{redondo; circular}
    \definition{clas.}{usado para algo em forma de bola}
    \definition[个]{s.}{bolinho de massa; comida em forma de bola feita de arroz ou farinha | algo em forma de bola | grupo; corpo; sociedade; organização; um grupo envolvido em um determinado trabalho ou atividade | regimento; unidade organizacional militar, geralmente abaixo do nível de divisão e acima do nível de batalhão}
    \definition{v.}{enrolar algo para formar uma bola; rolar | reunir; unir; conglomerar}
  \end{Phonetics}
\end{Entry}

\begin{Entry}{团长}{6,4}{⼞,⾧}
  \begin{Phonetics}{团长}{tuan2zhang3}[][HSK 5]
    \definition[位,名]{s.}{comandante do regimento | chefe (ou presidente) de uma delegação, trupe, etc. | líder de uma delegação}
  \end{Phonetics}
\end{Entry}

\begin{Entry}{团队}{6,4}{⼞,⾩}
  \begin{Phonetics}{团队}{tuan2dui4}[][HSK 6]
    \definition[个,支,种]{s.}{equipe; time; grupo; um grupo de alguma natureza}
  \end{Phonetics}
\end{Entry}

\begin{Entry}{团伙}{6,6}{⼞,⼈}
  \begin{Phonetics}{团伙}{tuan2huo3}[][HSK 7-9]
    \definition[个]{s.}{gangue; bando; panelinha | gangue (criminosa) | cúmplice; comparsa; membro de gangue}
  \end{Phonetics}
\end{Entry}

\begin{Entry}{团体}{6,7}{⼞,⼈}
  \begin{Phonetics}{团体}{tuan2ti3}[][HSK 3]
    \definition[种,个]{s.}{equipe; grupo; organização; um grupo de pessoas com objetivos e interesses comuns}
  \end{Phonetics}
\end{Entry}

\begin{Entry}{团员}{6,7}{⼞,⼝}
  \begin{Phonetics}{团员}{tuan2yuan2}[][HSK 7-9]
    \definition[名,位,个]{s.}{membro; membro de um grupo específico | membro da Liga; membro da Liga da Juventude Comunista da China; refere"-se especificamente aos membros da Liga da Juventude Comunista da China}
  \synonymref{会员}{hui4yuan2}
  \synonymref{团聚}{tuan2ju4}
  \end{Phonetics}
\end{Entry}

\begin{Entry}{团结}{6,9}{⼞,⽷}
  \begin{Phonetics}{团结}{tuan2jie2}[][HSK 3]
    \definition{adj.}{unido; amigável; harmonioso; relação harmoniosa e coexistência harmoniosa}
    \definition{v.}{unir; reunir}
  \end{Phonetics}
\end{Entry}

\begin{Entry}{团圆}{6,10}{⼞,⼞}
  \begin{Phonetics}{团圆}{tuan2yuan2}[][HSK 7-9]
    \definition{adj.}{redondo; indica que a forma é circular}
    \definition{v.}{reunir; reencontrar membros da família após um período de separação}
  \synonymref{团聚}{tuan2ju4}
  \antonymref{分离}{fen1li2}
  \end{Phonetics}
\end{Entry}

\begin{Entry}{团聚}{6,14}{⼞,⽿}
  \begin{Phonetics}{团聚}{tuan2ju4}[][HSK 7-9]
    \definition{v.}{reunir; reencontrar familiares ou amigos muito próximos após a separação | unir; reunir; congregar}
  \synonymref{重逢}{chong2feng2}
  \synonymref{欢聚}{huan1ju4}
  \synonymref{聚会}{ju4hui4}
  \synonymref{团圆}{tuan2yuan2}
  \synonymref{团员}{tuan2yuan2}
  \antonymref{分开}{fen1/kai1}
  \antonymref{分别}{fen1bie2}
  \antonymref{分离}{fen1li2}
  \antonymref{分散}{fen1san4}
  \end{Phonetics}
\end{Entry}

%%%%%%%%%% 在 %%%%%%%%%%
\subsection*{在}\addcontentsline{loh}{figure}{在}

\begin{Entry}{在}{6}{⼟}
  \begin{Phonetics}{在}{zai4}[][HSK 1]
    \definition{adv.}{em processo de; em curso de}
    \definition{prep.}{em; no (um lugar ou momento); indica tempo, local, âmbito, etc.}
    \definition{v.}{existir; estar vivo | estar em; estar no; estar em (um lugar); indica a localização de pessoas ou coisas | permanecer; ficar | depender de; residir em; repousar com | ingressar ou pertencer a uma organização; ser membro de uma organização}
  \end{Phonetics}
\end{Entry}

\begin{Entry}{在下}{6,3}{⼟,⼀}
  \begin{Phonetics}{在下}{zai4xia4}
    \definition{pron.}{eu mesmo (humildemente)}
  \end{Phonetics}
\end{Entry}

\begin{Entry}{在于}{6,3}{⼟,⼆}
  \begin{Phonetics}{在于}{zai4yu2}[][HSK 4]
    \definition{v.}{ser responsável por; caber a;  ser da competência de;  apontar a essência das coisas, ou do que elas se tratam | depender de; ser determinado por;  ser devido a (um determinado atributo)/(de um assunto a ser determinado)}
  \end{Phonetics}
\end{Entry}

\begin{Entry}{在内}{6,4}{⼟,⼌}
  \begin{Phonetics}{在内}{zai4nei4}[][HSK 5]
    \definition{adj.}{incluido}
    \definition{adv.}{dentro; internamente; entre eles}
    \definition{v.}{ser incluído}
  \end{Phonetics}
\end{Entry}

\begin{Entry}{在乎}{6,5}{⼟,⼃}
  \begin{Phonetics}{在乎}{zai4hu5}[][HSK 4]
    \definition{v.}{preocupar-se; preocupar-se com; levar a sério | ser responsável por; caber ao; ser da competência de}
  \end{Phonetics}
\end{Entry}

\begin{Entry}{在地}{6,6}{⼟,⼟}
  \begin{Phonetics}{在地}{zai4di4}
    \definition{s.}{local}
  \end{Phonetics}
\end{Entry}

\begin{Entry}{在场}{6,6}{⼟,⼟}
  \begin{Phonetics}{在场}{zai4chang3}[][HSK 5]
    \definition{v.}{estar presente; estar no local; estar em cena; estar presente onde as coisas estão acontecendo}
  \end{Phonetics}
\end{Entry}

\begin{Entry}{在此}{6,6}{⼟,⽌}
  \begin{Phonetics}{在此}{zai4 ci3}
    \definition{s.}{aqui}
  \end{Phonetics}
\end{Entry}

\begin{Entry}{在行}{6,6}{⼟,⾏}
  \begin{Phonetics}{在行}{zai4hang2}
    \definition{v.}{ser adepto de algo | ser um especialista em um comércio ou profissão}
  \end{Phonetics}
\end{Entry}

\begin{Entry}{在线}{6,8}{⼟,⽷}
  \begin{Phonetics}{在线}{zai4xian4}
    \definition{s.}{\emph{online}}
  \end{Phonetics}
\end{Entry}

\begin{Entry}{在家}{6,10}{⼟,⼧}
  \begin{Phonetics}{在家}{zai4jia1}[][HSK 1]
    \definition{v.}{estar em; estar em casa; estar no local de trabalho ou alojamento; sem sair de casa | continuar sendo um leigo; permanecer leigo; para monges, freiras, taoístas e outros que 出家, as pessoas comuns são consideradas leigas}
  \seealsoref{出家}{chu1 jia1}
  \end{Phonetics}
\end{Entry}

\begin{Entry}{在教}{6,11}{⼟,⽁}
  \begin{Phonetics}{在教}{zai4 jiao4}
    \definition{v.}{ser um crente (em uma religião)}
  \end{Phonetics}
\end{Entry}

\begin{Entry}{在意}{6,13}{⼟,⼼}
  \begin{Phonetics}{在意}{zai4/yi4}
    \definition{v.+compl.}{preocupar-se | importar-se | levar a sério}
  \end{Phonetics}
\end{Entry}

%%%%%%%%%% 地 %%%%%%%%%%
\subsection*{地}\addcontentsline{loh}{figure}{地}

\begin{Entry}{地}{6}{⼟}
  \begin{Phonetics}{地}{de5}[][HSK 1]
    \definition{part.}{(estrutural) utilizada antes de um verbo ou adjetivo, ligando-o ao adjunto adverbial modificador precedente}
  \end{Phonetics}
  \begin{Phonetics}{地}{di4}
    \definition*{s.}{A Terra | Sobrenome: Di}
    \definition[块,片]{s.}{terra; solo | campos | chão; piso | posição; situação | contexto; base | distância percorrida (medida em 里 ou paradas 站) | indicando estado de espírito | território | lugar; local | parte do espaço | distância}
  \end{Phonetics}
\end{Entry}

\begin{Entry}{地上}{6,3}{⼟,⼀}
  \begin{Phonetics}{地上}{di4shang5}[][HSK 1]
    \definition{adv.}{no chão; no solo; em terra}
  \end{Phonetics}
\end{Entry}

\begin{Entry}{地下}{6,3}{⼟,⼀}
  \begin{Phonetics}{地下}{di4xia5}[][HSK 4]
    \definition{s.}{subterrâneo | secreta (atividade) | recursos ocultos}
  \end{Phonetics}
\end{Entry}

\begin{Entry}{地下水}{6,3,4}{⼟,⼀,⽔}
  \begin{Phonetics}{地下水}{di4xia4shui3}[][HSK 7-9]
    \definition{s.}{água subterrânea, principalmente água da chuva e outras águas superficiais que se infiltram no solo e se acumulam nas fendas do solo ou nas formações rochosas}
  \end{Phonetics}
\end{Entry}

\begin{Entry}{地下室}{6,3,9}{⼟,⼀,⼧}
  \begin{Phonetics}{地下室}{di4xia4shi4}[][HSK 6]
    \definition{s.}{subterrâneo; porão; adega | abóbadas; cripta}
  \end{Phonetics}
\end{Entry}

\begin{Entry}{地区}{6,4}{⼟,⼖}
  \begin{Phonetics}{地区}{di4qu1}[][HSK 3]
    \definition[个,片]{s.}{área; distrito; região; um lugar maior | prefeitura; unidade administrativa | latitudes; localidade; lado | em determinadas circunstâncias, algumas regiões administrativas locais da China, como Hong Kong e Macau, participam individualmente em algumas atividades internacionais}
    \definition{suf.}{como sufixo do nome da cidade, significa prefeitura ou condado}
  \end{Phonetics}
\end{Entry}

\begin{Entry}{地方}{6,4}{⼟,⽅}
  \begin{Phonetics}{地方}{di4fang1}[][HSK 4]
    \definition[个]{s.}{distrito; localidade; o número total de unidades administrativas em todos os níveis abaixo do centro | governo local e população; refere"-se a outros setores que não o militar}
  \antonymref{中央}{zhong1yang1}
  \end{Phonetics}
  \begin{Phonetics}{地方}{di4fang5}[][HSK 1]
    \definition[个,处,块]{s.}{lugar; cômodo; área; refere"-se a um espaço específico | parte}
  \end{Phonetics}
\end{Entry}

\begin{Entry}{地名}{6,6}{⼟,⼝}
  \begin{Phonetics}{地名}{di4ming2}[][HSK 6]
    \definition{s.}{nome de um lugar | nome de lugar | topônimo}
  \end{Phonetics}
\end{Entry}

\begin{Entry}{地位}{6,7}{⼟,⼈}
  \begin{Phonetics}{地位}{di4wei4}[][HSK 4]
    \definition[个]{s.}{lugar; status; posição; posição da pessoa ou do grupo nas relações sociais | lugar; posição (ocupada por uma pessoa ou coisa); espaço ocupado por uma pessoa ou coisa}
  \end{Phonetics}
\end{Entry}

\begin{Entry}{地址}{6,7}{⼟,⼟}
  \begin{Phonetics}{地址}{di4zhi3}[][HSK 4]
    \definition[个,条]{s.}{endereço; local de residência ou correspondência}
  \end{Phonetics}
\end{Entry}

\begin{Entry}{地形}{6,7}{⼟,⼺}
  \begin{Phonetics}{地形}{di4xing2}[][HSK 5]
    \definition{s.}{topografia; forma do terreno; relevo; disposição do terreno; característica do relevo; característica da superfície; terreno}
  \end{Phonetics}
\end{Entry}

\begin{Entry}{地步}{6,7}{⼟,⽌}
  \begin{Phonetics}{地步}{di4bu4}[][HSK 7-9]
    \definition[个,种]{s.}{condição; situação; estado; geralmente ruim | grau; extensão; o grau de realização | margem de manobra; espaço}
  \end{Phonetics}
\end{Entry}

\begin{Entry}{地图}{6,8}{⼟,⼞}
  \begin{Phonetics}{地图}{di4tu2}[][HSK 1]
    \definition[张,本]{s.}{mapa; mapa que mostra a distribuição de coisas e fenômenos na superfície da Terra, com símbolos e textos, e às vezes também com cores}
  \end{Phonetics}
\end{Entry}

\begin{Entry}{地板}{6,8}{⼟,⽊}
  \begin{Phonetics}{地板}{di4ban3}[][HSK 6]
    \definition[块]{s.}{piso de madeira; tábuas de madeira especiais para pavimentação do piso | piso; piso interno pavimentado com tábuas de madeira; geralmente se refere ao piso de um edifício}
  \end{Phonetics}
\end{Entry}

\begin{Entry}{地质}{6,8}{⼟,⾙}
  \begin{Phonetics}{地质}{di4zhi4}[][HSK 7-9]
    \definition[种]{s.}{geologia; composição e estrutura da crosta terrestre}
  \end{Phonetics}
\end{Entry}

\begin{Entry}{地带}{6,9}{⼟,⼱}
  \begin{Phonetics}{地带}{di4dai4}[][HSK 5]
    \definition[个,条]{s.}{distrito; região; zona; área de uma determinada natureza ou extensão}
  \end{Phonetics}
\end{Entry}

\begin{Entry}{地段}{6,9}{⼟,⽎}
  \begin{Phonetics}{地段}{di4duan4}[][HSK 7-9]
    \definition[个]{s.}{setor (ou seção) de uma cidade, etc.; área | um setor de uma área; uma seção de uma área; alcance; extensão; lote}
  \end{Phonetics}
\end{Entry}

\begin{Entry}{地点}{6,9}{⼟,⽕}
  \begin{Phonetics}{地点}{di4dian3}[][HSK 1]
    \definition[个]{s.}{lugar; local; região; localização}
  \end{Phonetics}
\end{Entry}

\begin{Entry}{地狱}{6,9}{⼟,⽝}
  \begin{Phonetics}{地狱}{di4yu4}[][HSK 7-9]
    \definition[个,层,重,处]{s.}{inferno; submundo; algumas religiões se referem ao lugar onde a alma sofre após a morte | inferno na terra; lugar de tormento como o inferno; uma metáfora para um ambiente de vida sombrio e miserável}
  \end{Phonetics}
\end{Entry}

\begin{Entry}{地砖}{6,9}{⼟,⽯}
  \begin{Phonetics}{地砖}{di4zhuan1}
    \definition{s.}{ladrilho de piso}
  \end{Phonetics}
\end{Entry}

\begin{Entry}{地面}{6,9}{⼟,⾯}
  \begin{Phonetics}{地面}{di4mian4}[][HSK 4]
    \definition{s.}{a superfície da Terra | térreo; piso; camada de material colocada no chão dentro e ao redor dos edifícios | localidade; chão | região; território; principalmente áreas administrativas}
  \end{Phonetics}
\end{Entry}

\begin{Entry}{地核}{6,10}{⼟,⽊}
  \begin{Phonetics}{地核}{di4he2}
    \definition{s.}{(geologia) núcleo da Terra}
  \end{Phonetics}
\end{Entry}

\begin{Entry}{地铁}{6,10}{⼟,⾦}
  \begin{Phonetics}{地铁}{di4tie3}[][HSK 2]
    \definition[条,班,列,趟]{s.}{metrô; trem subterrâneo; também se refere ao vagão do metrô}
  \end{Phonetics}
\end{Entry}

\begin{Entry}{地铁站}{6,10,10}{⼟,⾦,⽴}
  \begin{Phonetics}{地铁站}{di4tie3zhan4}[][HSK 2]
    \definition[个,座]{s.}{estação de metrô}
  \end{Phonetics}
\end{Entry}

\begin{Entry}{地域}{6,11}{⼟,⼟}
  \begin{Phonetics}{地域}{di4yu4}[][HSK 7-9]
    \definition{s.}{região; distrito}
  \end{Phonetics}
\end{Entry}

\begin{Entry}{地球}{6,11}{⼟,⽟}
  \begin{Phonetics}{地球}{di4qiu2}[][HSK 2]
    \definition[个]{s.}{o planeta Terra}
  \end{Phonetics}
\end{Entry}

\begin{Entry}{地理}{6,11}{⼟,⽟}
  \begin{Phonetics}{地理}{di4li3}[][HSK 7-9]
    \definition{s.}{características geográficas de um lugar; a situação geral do mundo ou das montanhas e rios de uma região; ambiente natural, como clima e produtos; transporte; assentamentos e outros fatores socioeconômicos | geografia}
  \end{Phonetics}
\end{Entry}

\begin{Entry}{地毯}{6,12}{⼟,⽑}
  \begin{Phonetics}{地毯}{di4tan3}[][HSK 7-9]
    \definition[块,张]{s.}{tapete; carpete}
  \end{Phonetics}
\end{Entry}

\begin{Entry}{地道}{6,12}{⼟,⾡}
  \begin{Phonetics}{地道}{di4dao4}[][HSK 7-9]
    \definition[条,个]{s.}{metrô; túnel; passagem subterrânea; galeria}
  \end{Phonetics}
  \begin{Phonetics}{地道}{di4dao5}[][HSK 7-9]
    \definition{adj.}{autêntico; puro; típico; alta qualidade, atendendo a certos padrões; como o produto real | excelente; honesto; de acordo com os padrões; de alta qualidade | (da moral e das qualidades de uma pessoa) muito bom; nobre, frequentemente usado em frases negativas}
  \end{Phonetics}
\end{Entry}

\begin{Entry}{地震}{6,15}{⼟,⾬}
  \begin{Phonetics}{地震}{di4zhen4}[][HSK 5]
    \definition[场,次,级]{s.}{sismo; terremoto; tremor de terra; vibrações na crosta terrestre}
    \definition{v.}{sacudir com vibrações sísmicas}
  \end{Phonetics}
\end{Entry}

%%%%%%%%%% 场 %%%%%%%%%%
\subsection*{场}\addcontentsline{loh}{figure}{场}

\begin{Entry}{场}{6}{⼟}
  \begin{Phonetics}{场}{chang2}
    \definition{clas.}{usado para descrever o desenrolar dos acontecimentos}
    \definition{s.}{eira; espaço aberto e plano; um terreno plano, geralmente usado para secar grãos e moer cereais | mercado; feira rural}
  \end{Phonetics}
  \begin{Phonetics}{场}{chang3}[][HSK 2]
    \definition*{s.}{Sobrenome: Chang}
    \definition{clas.}{usado para atividades culturais, recreativas e esportivas | usado para pequenos trechos de uma peça}
    \definition{s.}{um local amplo utilizado para um fim específico | palco; campo | cena | Física: campo (por exemplo, campo magnético) | local para atividades recreativas, esportivas ou outras | um lugar onde as pessoas se reúnem | fazenda; quinta | abertura; encerramento; refere"-se ao processo completo de uma apresentação ou competição | local; ponto; o local onde ocorreu o incidente}
  \end{Phonetics}
\end{Entry}

\begin{Entry}{场合}{6,6}{⼟,⼝}
  \begin{Phonetics}{场合}{chang3he2}[][HSK 3]
    \definition[个,些,种,类]{s.}{ocasião; situação; um certo tempo, lugar ou situação}
  \end{Phonetics}
\end{Entry}

\begin{Entry}{场地}{6,6}{⼟,⼟}
  \begin{Phonetics}{场地}{chang3di4}[][HSK 6]
    \definition[片,块,个]{s.}{área; pátio; espaço; lugar; quadra; campo; um lugar onde construções ou atividades são realizadas}
  \end{Phonetics}
\end{Entry}

\begin{Entry}{场所}{6,8}{⼟,⼾}
  \begin{Phonetics}{场所}{chang3suo3}[][HSK 3]
    \definition{s.}{lugar; sítio; arena; local da atividade}
  \end{Phonetics}
\end{Entry}

\begin{Entry}{场面}{6,9}{⼟,⾯}
  \begin{Phonetics}{场面}{chang3mian4}[][HSK 5]
    \definition[个,种,番]{s.}{espetáculo; cena (em teatro, ficção, etc.); uma cena em uma produção teatral, cinematográfica ou televisiva que consiste em um cenário, música e personagens | cena; ocasião; literatura narrativa que consiste em situações da vida em que os personagens se relacionam entre si em determinadas ocasiões | orquestra ou instrumentos de acompanhamento (em ópera); refere"-se às pessoas e aos instrumentos musicais que acompanham a apresentação de uma ópera, divididos em dois tipos: música de sopro e cordas é uma cena cultural, e gongos e tambores são uma cena marcial | situação; referência geral a uma situação em um determinado contexto | frente; fachada; aparência; espetáculo superficial}
  \end{Phonetics}
\end{Entry}

\begin{Entry}{场馆}{6,11}{⼟,⾷}
  \begin{Phonetics}{场馆}{chang3guan3}[][HSK 6]
    \definition{s.}{ginásios e estádios | arena | local esportivo}
  \end{Phonetics}
\end{Entry}

\begin{Entry}{场景}{6,12}{⼟,⽇}
  \begin{Phonetics}{场景}{chang3jing3}[][HSK 6]
    \definition[个,幕,种]{s.}{espetáculo; cena (em drama, ficção, etc.); refere"-se a cenas de drama, cinema, televisão e obras literárias | cena; visão; circunstâncias; cenas e situações}
  \end{Phonetics}
\end{Entry}

%%%%%%%%%% 壮 %%%%%%%%%%
\subsection*{壮}\addcontentsline{loh}{figure}{壮}

\begin{Entry}{壮}{6}{⼠}
  \begin{Phonetics}{壮}{zhuang4}
    \definition*{s.}{Grupo étnico Zhuang (ou Chuang) | Sobrenome: Zhuang}
    \definition{adj.}{forte; robusto | magnífico; grandioso; majestoso}
    \definition{v.}{fortalecer; tornar melhor | expandir}
  \end{Phonetics}
\end{Entry}

\begin{Entry}{壮观}{6,6}{⼠,⾒}
  \begin{Phonetics}{壮观}{zhuang4guan1}[][HSK 6]
    \definition{adj.}{grandioso; magnífico; espetacular}
    \definition{s.}{grande vista; vista magnífica; espetáculo esplêndido}
  \end{Phonetics}
\end{Entry}

\begin{Entry}{壮族}{6,11}{⼠,⽅}
  \begin{Phonetics}{壮族}{zhuang4 zu2}
    \definition*{s.}{Grupo étnico Zhuang (ou Chuang) de Guangxi}
  \seealsoref{广东}{guang3dong1}
  \seealsoref{广西}{guang3xi1}
  \seealsoref{云南}{yun2nan2}
  \end{Phonetics}
\end{Entry}

%%%%%%%%%% 多 %%%%%%%%%%
\subsection*{多}\addcontentsline{loh}{figure}{多}

\begin{Entry}{多}{6}{⼣}
  \begin{Phonetics}{多}{duo1}[][HSK 1,2]
    \definition*{s.}{Sobrenome: Duo}
    \definition{adj.}{grande quantidade | excessivo; desnecessário | excessivo; em demasia; indica um grande grau de diferença | mais do que o número correto ou necessário; em excesso}
    \definition{adv.}{acima de um valor especificado; e mais | em que medida; usado em frases interrogativas para indagar sobre grau ou quantidade, equivalente a 多么 | uma extensão não especificada; usado em frases exclamativas para expressar um alto grau, equivalente a 多么 | quase; significa que a maior parte do intervalo é assim | mais;  sobre; ímpar; usado depois de um quantificador para indicar uma fração}
    \definition{num.}{(após um número) ímpar}
    \definition{pref.}{multi- | poli-}
    \definition{v.}{ter (uma quantidade específica) a mais ou a mais | ter algo em abundância  | (em perguntas) até que ponto | (em exclamações) até que ponto | ter mais}
  \seealsoref{多么}{duo1me5}
  \antonymref{寡}{gua3}
  \antonymref{少}{shao3}
  \end{Phonetics}
\end{Entry}

\begin{Entry}{多久}{6,3}{⼣,⼃}
  \begin{Phonetics}{多久}{duo1jiu3}[][HSK 2]
    \definition{pron.}{quanto tempo?; quanto tempo; perguntar quanto tempo leva}
  \end{Phonetics}
\end{Entry}

\begin{Entry}{多么}{6,3}{⼣,⼃}
  \begin{Phonetics}{多么}{duo1me5}[][HSK 2]
    \definition{adv.}{(em exclamações) como; o quê; em que medida; usado em frases exclamativas, indica um grau muito alto | em grau indeterminado; usado em frases declarativas, indica um grau mais profundo | como (usado em uma frase interrogativa para perguntar sobre grau ou número)}
  \end{Phonetics}
\end{Entry}

\begin{Entry}{多亏}{6,3}{⼣,⼆}
  \begin{Phonetics}{多亏}{duo1kui1}[][HSK 7-9]
    \definition{adv.}{felizmente; graças a; devido a; significa que alguém evitou algo desagradável ou ganhou algo bom devido à ajuda de outros ou alguns fatores favoráveis, e implica gratidão ou alívio}
  \end{Phonetics}
\end{Entry}

\begin{Entry}{多大}{6,3}{⼣,⼤}
  \begin{Phonetics}{多大}{duo1da4}
    \definition{adj.}{quantos anos? | que idade? | quão grande?}
  \end{Phonetics}
\end{Entry}

\begin{Entry}{多云}{6,4}{⼣,⼆}
  \begin{Phonetics}{多云}{duo1yun2}[][HSK 2]
    \definition{adj.}{céu nublado; em meteorologia, refere"-se a condições atmosféricas em que a cobertura de nuvens médias e baixas ocupa entre 40\% e 70\% da área do céu, ou a cobertura de nuvens altas ocupa entre 60\% e 100\% da área do céu}
  \end{Phonetics}
\end{Entry}

\begin{Entry}{多元}{6,4}{⼣,⼉}
  \begin{Phonetics}{多元}{duo1yuan2}[][HSK 7-9]
    \definition{adj.}{diverso; pluralista; multivariado}
  \end{Phonetics}
\end{Entry}

\begin{Entry}{多少}{6,4}{⼣,⼩}
  \begin{Phonetics}{多少}{duo1shao3}
    \definition{adv.}{um pouco; mais ou menos; até certo ponto}
    \definition{s.}{número; quantidade; volume}
  \end{Phonetics}
  \begin{Phonetics}{多少}{duo1shao5}[][HSK 1]
    \definition{adv.}{quantos?; quanto?; usado em perguntas para perguntar sobre quantidade | expressar uma quantidade ou número não especificado; quantidade indefinida}
  \end{Phonetics}
\end{Entry}

\begin{Entry}{多心}{6,4}{⼣,⼼}
  \begin{Phonetics}{多心}{duo1xin1}[][HSK 7-9]
    \definition{adj.}{hipersensível; paranóico; sensível | suspeito}
  \end{Phonetics}
\end{Entry}

\begin{Entry}{多方面}{6,4,9}{⼣,⽅,⾯}
  \begin{Phonetics}{多方面}{duo1fang1mian4}[][HSK 6]
    \definition{adj.}{de muitas maneiras; todos os aspectos}
    \definition{s.}{multifacetado; multiaspecto}
  \end{Phonetics}
\end{Entry}

\begin{Entry}{多功能}{6,5,10}{⼣,⼒,⾁}
  \begin{Phonetics}{多功能}{duo1gong1neng2}[][HSK 7-9]
    \definition{adj.}{multifuncional; multiuso (ou para todos os fins)}
  \end{Phonetics}
\end{Entry}

\begin{Entry}{多半}{6,5}{⼣,⼗}
  \begin{Phonetics}{多半}{duo1ban4}[][HSK 6]
    \definition{adv.}{geralmente; mais frequentemente do que não}
    \definition{num.}{a maioria; a maior parte; mais da metade}
  \end{Phonetics}
\end{Entry}

\begin{Entry}{多边}{6,5}{⼣,⾡}
  \begin{Phonetics}{多边}{duo1bian1}[][HSK 7-9]
    \definition{adj.}{multilateral (três ou mais partes)}
  \end{Phonetics}
\end{Entry}

\begin{Entry}{多年}{6,6}{⼣,⼲}
  \begin{Phonetics}{多年}{duo1nian2}[][HSK 4]
    \definition{adv.}{por muitos anos; durante muitos anos}
  \end{Phonetics}
\end{Entry}

\begin{Entry}{多年来}{6,6,7}{⼣,⼲,⽊}
  \begin{Phonetics}{多年来}{duo1nian2lai2}[][HSK 7-9]
    \definition{s.}{nos últimos anos}
  \end{Phonetics}
\end{Entry}

\begin{Entry}{多次}{6,6}{⼣,⽋}
  \begin{Phonetics}{多次}{duo1ci4}[][HSK 4]
    \definition{adv.}{muitas vezes; de vez em quando; repetidamente; em muitas ocasiões}
  \end{Phonetics}
\end{Entry}

\begin{Entry}{多余}{6,7}{⼣,⼈}
  \begin{Phonetics}{多余}{duo1yu2}[][HSK 7-9]
    \definition{adj.}{extra; excedente; excessivo; supérfluo; desnecessário}
    \definition{v.}{exceder; superar; transbordar}
  \end{Phonetics}
\end{Entry}

\begin{Entry}{多劳多得}{6,7,6,11}{⼣,⼒,⼣,⼻}
  \begin{Phonetics}{多劳多得}{duo1lao2-duo1de2}[][HSK 7-9]
    \definition{expr.}{trabalhe mais e ganhe mais; o princípio socialista de distribuição é que quanto mais você trabalha, mais você se beneficia, e se você não trabalha, você não ganha comida}
  \end{Phonetics}
\end{Entry}

\begin{Entry}{多咱}{6,9}{⼣,⼝}
  \begin{Phonetics}{多咱}{duo1 zan5}
    \definition{adv.}{que horas?; quando?}
  \end{Phonetics}
\end{Entry}

\begin{Entry}{多种}{6,9}{⼣,⽲}
  \begin{Phonetics}{多种}{duo1zhong3}[][HSK 4]
    \definition{adj.}{diverso; vários tipos de; múltiplo; diversificado}
    \definition{pref.}{multi-}
  \end{Phonetics}
\end{Entry}

\begin{Entry}{多重}{6,9}{⼣,⾥}
  \begin{Phonetics}{多重}{duo1chong2}
    \definition{pref.}{multi (facetado, cultural, étnico, etc.)}
  \end{Phonetics}
\end{Entry}

\begin{Entry}{多样}{6,10}{⼣,⽊}
  \begin{Phonetics}{多样}{duo1yang4}[][HSK 4]
    \definition{adj.}{diversos; variados; diversificado}
    \definition{s.}{diversidade}
  \end{Phonetics}
\end{Entry}

\begin{Entry}{多媒体}{6,12,7}{⼣,⼥,⼈}
  \begin{Phonetics}{多媒体}{duo1mei2ti3}[][HSK 6]
    \definition{s.}{multimídia; uma combinação de múltiplas mídias}
  \end{Phonetics}
\end{Entry}

\begin{Entry}{多数}{6,13}{⼣,⽁}
  \begin{Phonetics}{多数}{duo1shu4}[][HSK 2]
    \definition{adj.}{maioria; a maioria; plural}
    \definition{pref.}{pluri-}
  \end{Phonetics}
\end{Entry}

%%%%%%%%%% 夸 %%%%%%%%%%
\subsection*{夸}\addcontentsline{loh}{figure}{夸}

\begin{Entry}{夸}{6}{⼤}
  \begin{Phonetics}{夸}{kua1}[][HSK 7-9]
    \definition{v.}{exagerar; superestimar; vangloriar-se | elogiar; destacar as qualidades positivas de uma pessoa ou coisa}
  \end{Phonetics}
\end{Entry}

\begin{Entry}{夸大}{6,3}{⼤,⼤}
  \begin{Phonetics}{夸大}{kua1da4}[][HSK 7-9]
    \definition{v.}{exagerar; superestimar; magnificar; engrandecer}
  \end{Phonetics}
\end{Entry}

\begin{Entry}{夸夸其谈}{6,6,8,10}{⼤,⼤,⼋,⾔}
  \begin{Phonetics}{夸夸其谈}{kua1kua1-qi2tan2}[][HSK 7-9]
    \definition{expr.}{entregar-se à verborragia; conversa empolgante e jactanciosa sem muito significado; cheio de ar quente; um longo discurso cheio de pompa; tagarelice; falar demais; uma grande conversa; discursar; conversa pomposa; conversa ou escrita pomposa, mas sem sentido; discurso pomposo; exagerar; desabafar; exagerar (ou usar verborragia); tagarelar; falar pelos cotovelos}
  \end{Phonetics}
\end{Entry}

\begin{Entry}{夸张}{6,7}{⼤,⼸}
  \begin{Phonetics}{夸张}{kua1zhang1}[][HSK 7-9]
    \definition{adj.}{exagerado; superestimado; pomposo}
    \definition{s.}{hipérbole; uma figura de linguagem que utiliza palavras exageradas para descrever coisas}
  \end{Phonetics}
\end{Entry}

\begin{Entry}{夸奖}{6,9}{⼤,⼤}
  \begin{Phonetics}{夸奖}{kua1jiang3}[][HSK 7-9]
    \definition{v.}{louvar; elogiar; cumprimentar; demonstrar apreço e incentivar alguém por suas qualidades ou boas ações}
  \end{Phonetics}
\end{Entry}

\begin{Entry}{夸耀}{6,20}{⼤,⽻}
  \begin{Phonetics}{夸耀}{kua1yao4}[][HSK 7-9]
    \definition{v.}{ostentar; gabar"-se de; vangloriar"-se de; exibir (as próprias habilidades, conquistas, status e poder)}
  \end{Phonetics}
\end{Entry}

%%%%%%%%%% 夹 %%%%%%%%%%
\subsection*{夹}\addcontentsline{loh}{figure}{夹}

\begin{Entry}{夹}{6}{⼤}
  \begin{Phonetics}{夹}{ga1}
    \definition{s.}{axila; sovaco; atualmente, costuma-se escrever 胳肢窝}
  \seealsoref{胳肢窝}{ga1 zhi1 wo1}
  \end{Phonetics}
  \begin{Phonetics}{夹}{jia1}[][HSK 5]
    \definition{s.}{clipe, grampo, pasta, etc.}
    \definition{v.}{colocar no meio; pressionar de ambos os lados; aplicar força ou ação ao mesmo objeto de ambos os lados ao mesmo tempo | misturar; mesclar; intercalar}
  \end{Phonetics}
  \begin{Phonetics}{夹}{jia2}
    \definition{adj.}{forrado; com camada dupla; duas camadas (roupas, colchas, etc.) | pinçado; voz deliberadamente engraçada}
  \end{Phonetics}
\end{Entry}

\begin{Entry}{夹子}{6,3}{⼤,⼦}
  \begin{Phonetics}{夹子}{jia1 zi5}
    \definition[个,堆,盒]{s.}{pasta; carteira; algo para guardar dinheiro, papel, etc. | clipe; grampo; pasta; pinça; ferramentas para prender coisas}
  \end{Phonetics}
\end{Entry}

\begin{Entry}{夹杂}{6,6}{⼤,⽊}
  \begin{Phonetics}{夹杂}{jia1 za2}
    \definition{v.}{ser misturado com; estar carregado de; adicionar (algo mais)}
  \end{Phonetics}
\end{Entry}

\begin{Entry}{夹肢窝}{6,8,12}{⼤,⾁,⽳}
  \begin{Phonetics}{夹肢窝}{jia1 zhi1 wo1}
    \definition{s.}{axila; sovaco; também escrito como 胳肢窝}
  \seealsoref{胳肢窝}{ga1 zhi1 wo1}
  \end{Phonetics}
\end{Entry}

%%%%%%%%%% 夺 %%%%%%%%%%
\subsection*{夺}\addcontentsline{loh}{figure}{夺}

\begin{Entry}{夺}{6}{⼤}
  \begin{Phonetics}{夺}{duo2}[][HSK 6]
    \definition{v.}{tomar à força; apreender; arrancar; roubar | forçar a passagem; empurrar para abrir | lutar por; competir por; esforçar-se por; obter primeiro | privar; perder | perder; tirar | decidir; tomar uma decisão | omitir (palavra em um texto)}
  \end{Phonetics}
\end{Entry}

\begin{Entry}{夺取}{6,8}{⼤,⼜}
  \begin{Phonetics}{夺取}{duo2qu3}[][HSK 6]
    \definition{v.}{capturar; apreender; arrancar; tomar à força | esforçar-se para; alcançar}
  \end{Phonetics}
\end{Entry}

\begin{Entry}{夺冠}{6,9}{⼤,⼍}
  \begin{Phonetics}{夺冠}{duo2/guan4}[][HSK 7-9]
    \definition{v.+compl.}{chegar em primeiro lugar; ficar em primeiro lugar; ganhar um campeonato}
  \end{Phonetics}
\end{Entry}

\begin{Entry}{夺魁}{6,13}{⼤,⿁}
  \begin{Phonetics}{夺魁}{duo2/kui2}[][HSK 7-9]
    \definition{v.+compl.}{Literário: ganhar o primeiro prêmio; ganhar o campeonato; conquistar; vencer}
  \end{Phonetics}
\end{Entry}

%%%%%%%%%% 奸 %%%%%%%%%%
\subsection*{奸}\addcontentsline{loh}{figure}{奸}

\begin{Entry}{奸}{6}{⼥}
  \begin{Phonetics}{奸}{jian1}
    \definition{adj.}{perverso; maligno; traiçoeiro; malicioso}
    \definition{s.}{traidor; espião | pessoa perversa; pessoa traiçoeira | relações sexuais ilícitas; comportamento sexual impróprio}
    \definition{v.}{ter relações sexuais ilícitas}
  \end{Phonetics}
\end{Entry}

\begin{Entry}{奸夫}{6,4}{⼥,⼤}
  \begin{Phonetics}{奸夫}{jian1fu1}
    \definition{s.}{homem adúltero}
  \end{Phonetics}
\end{Entry}

\begin{Entry}{奸诈}{6,7}{⼥,⾔}
  \begin{Phonetics}{奸诈}{jian1zha4}[][HSK 7-9]
    \definition{adj.}{fraudulento; astuto; enganoso; traiçoeiro; hipócrita e enganador, não confiável}
  \end{Phonetics}
\end{Entry}

%%%%%%%%%% 她 %%%%%%%%%%
\subsection*{她}\addcontentsline{loh}{figure}{她}

\begin{Entry}{她}{6}{⼥}
  \begin{Phonetics}{她}{ta1}[][HSK 1]
    \definition{pron.}{ela | ela; referir-se a coisas que se ama ou aprecia, como a pátria, a bandeira nacional, etc.}
  \end{Phonetics}
\end{Entry}

\begin{Entry}{她们}{6,5}{⼥,⼈}
  \begin{Phonetics}{她们}{ta1men5}[][HSK 1]
    \definition{pron.}{elas; referindo"-se a várias mulheres: em textos escritos, use 她们 quando todas as pessoas forem mulheres e 他们 quando houver homens e mulheres}
  \seealsoref{他们}{ta1men5}
  \end{Phonetics}
\end{Entry}

\begin{Entry}{她们的}{6,5,8}{⼥,⼈,⽩}
  \begin{Phonetics}{她们的}{ta1men5 de5}
    \definition{pron.}{delas}
  \end{Phonetics}
\end{Entry}

\begin{Entry}{她的}{6,8}{⼥,⽩}
  \begin{Phonetics}{她的}{ta1 de5}
    \definition{pron.}{dela}
  \end{Phonetics}
\end{Entry}

%%%%%%%%%% 好 %%%%%%%%%%
\subsection*{好}\addcontentsline{loh}{figure}{好}

\begin{Entry}{好}{6}{⼥}
  \begin{Phonetics}{好}{hao3}[][HSK 1,2,4]
    \definition{adj.}{bom; ótimo; agradável; vantajoso; satisfatório | amigável; gentil; amistoso; amável | saudável; bem | pronto; concluído; usado após um verbo para indicar conclusão ou perfeição | fácil (de fazer); conveniente; responsável (por)}
    \definition{adv.}{muito; bastante; tão; usado na frente de uma palavra de quantidade ou uma palavra de tempo para indicar muito ou por muito tempo | em que medida; como; usado antes de adjetivos e verbos para indicar profundidade e com exclamação}
    \definition{interj.}{``O.K.!''; ``Tudo bem!''; aprovação, acordo ou encerramento | (no início de uma frase ou oração) expressa concordância (ou desaprovação, surpresa, etc.)}
    \definition{prep.}{de modo a; para que}
    \definition{s.}{referindo"-se a palavras de elogio ou aplauso | saudações; cumprimentos}
    \definition{suf.}{sufixo que indica conclusão ou prontidão | depois de um pronome significa ``olá''}
    \definition{v.}{deve; precisa; tem que; deveria | apaixonar"-se}
  \end{Phonetics}
  \begin{Phonetics}{好}{hao4}[][HSK 4]
    \definition*{s.}{Sobrenome: Hao}
    \definition{adv.}{algo que acontece com frequência, que é fácil de acontecer}
    \definition{v.}{gostar; amar; ter afeição por}
  \end{Phonetics}
\end{Entry}

\begin{Entry}{好人}{6,2}{⼥,⼈}
  \begin{Phonetics}{好人}{hao3ren2}[][HSK 2]
    \definition[个,位,名]{s.}{pessoa boa (ou excelente) | pessoa saudável | pessoa gentil que tenta se dar bem com todos (muitas vezes em detrimento dos princípios)}
  \antonymref{坏人}{huai4ren2}
  \end{Phonetics}
\end{Entry}

\begin{Entry}{好久}{6,3}{⼥,⼃}
  \begin{Phonetics}{好久}{hao3jiu3}[][HSK 2]
    \definition{adv.}{por muito tempo | por eras (no passado)}
  \end{Phonetics}
\end{Entry}

\begin{Entry}{好(不)容易}{6,4,10,8}{⼥,⼀,⼧,⽇}
  \begin{Phonetics}{好(不)容易}{hao3 bu4 rong2 yi4}
    \definition{adv.}{com grande dificuldade; muito difícil}
    \definition{v.}{ter dificuldade (em fazer algo)}
  \end{Phonetics}
\end{Entry}

\begin{Entry}{好友}{6,4}{⼥,⼜}
  \begin{Phonetics}{好友}{hao3you3}[][HSK 4]
    \definition[位,名,个,些]{s.}{bom amigo; amigo próximo}
  \end{Phonetics}
\end{Entry}

\begin{Entry}{好心}{6,4}{⼥,⼼}
  \begin{Phonetics}{好心}{hao3xin1}[][HSK 7-9]
    \definition{adj./s.}{bondade; boas intenções}
  \end{Phonetics}
\end{Entry}

\begin{Entry}{好心人}{6,4,2}{⼥,⼼,⼈}
  \begin{Phonetics}{好心人}{hao3xin1ren2}[][HSK 7-9]
    \definition{s.}{boa alma; pessoa de bom coração | pessoa gentil}
  \end{Phonetics}
\end{Entry}

\begin{Entry}{好歹}{6,4}{⼥,⽍}
  \begin{Phonetics}{好歹}{hao3dai3}[][HSK 7-9]
    \definition{adv.}{de qualquer forma; em qualquer caso | de alguma forma; não importa de que maneira; não importa o que}
    \definition{s.}{bom e mau; o que é bom e o que é mau | acidente; desastre; refere"-se a situações de risco de vida}
  \end{Phonetics}
\end{Entry}

\begin{Entry}{好比}{6,4}{⼥,⽐}
  \begin{Phonetics}{好比}{hao3bi3}[][HSK 7-9]
    \definition{v.}{pode ser comparado a; ser exatamente como}
  \end{Phonetics}
\end{Entry}

\begin{Entry}{好处}{6,5}{⼥,⼡}
  \begin{Phonetics}{好处}{hao3chu5}[][HSK 2]
    \definition[个]{s.}{bom; benefício; vantagem; fatores favoráveis a pessoas ou coisas | ganho; lucro; algo que não se deveria receber, dado por outra pessoa ou obtido através de uma oportunidade; geralmente tem conotação pejorativa}
  \end{Phonetics}
\end{Entry}

\begin{Entry}{好汉}{6,5}{⼥,⽔}
  \begin{Phonetics}{好汉}{hao3han4}
    \definition[条]{s.}{herói | pessoa forte e corajosa}
  \end{Phonetics}
\end{Entry}

\begin{Entry}{好生}{6,5}{⼥,⽣}
  \begin{Phonetics}{好生}{hao3sheng1}
    \definition{adv.}{bastante; extremamente | cuidadosamente; apropriadamente}
  \end{Phonetics}
\end{Entry}

\begin{Entry}{好用}{6,5}{⼥,⽤}
  \begin{Phonetics}{好用}{hao3yong4}
    \definition{adj.}{fácil de usar | adequado ao uso}
  \end{Phonetics}
\end{Entry}

\begin{Entry}{好似}{6,6}{⼥,⼈}
  \begin{Phonetics}{好似}{hao3si4}[][HSK 6]
    \definition{v.}{parecer; ser como}
  \end{Phonetics}
\end{Entry}

\begin{Entry}{好吃}{6,6}{⼥,⼝}
  \begin{Phonetics}{好吃}{hao3chi1}[][HSK 1]
    \definition{adj.}{bom; saboroso; delicioso; descreve o sabor agradável de algo, que as pessoas gostam de comer}
  \end{Phonetics}
  \begin{Phonetics}{好吃}{hao4chi1}
    \definition{v.}{ser guloso; gostar de comer boa comida}
  \end{Phonetics}
\end{Entry}

\begin{Entry}{好在}{6,6}{⼥,⼟}
  \begin{Phonetics}{好在}{hao3zai4}[][HSK 7-9]
    \definition{adv.}{felizmente; afortunadamente; indica que existem fatores favoráveis em condições difíceis ou desfavoráveis}
  \end{Phonetics}
\end{Entry}

\begin{Entry}{好多}{6,6}{⼥,⼣}
  \begin{Phonetics}{好多}{hao3duo1}[][HSK 2]
    \definition{adj.}{muitos; uma boa quantidade; uma grande quantidade; uma quantidade enorme}
    \definition{pron.}{quantos?; quanto?; frequentemente usado para perguntar sobre quantidade}
  \end{Phonetics}
\end{Entry}

\begin{Entry}{好好}{6,6}{⼥,⼥}
  \begin{Phonetics}{好好}{hao3hao3}[][HSK 3]
    \definition{adj.}{realmente bom/bem; em perfeitas condições; quando tudo está bem}
    \definition{adv.}{diretamente; seriamente; cuidadosamente; com todo o empenho; ao máximo}
  \end{Phonetics}
\end{Entry}

\begin{Entry}{好听}{6,7}{⼥,⼝}
  \begin{Phonetics}{好听}{hao3ting1}[][HSK 1]
    \definition{adj.}{agradável de ouvir (de som ou voz) | bom; palatável; satisfatório (de palavras)  | decente; honrado (de ações, etc.); descreve uma coisa que parece prestigiosa | interessante; descreve palavras, histórias e outras coisas interessantes}
  \end{Phonetics}
\end{Entry}

\begin{Entry}{好坏}{6,7}{⼥,⼟}
  \begin{Phonetics}{好坏}{hao3huai4}[][HSK 7-9]
    \definition{s.}{bom e mau; o que é bom e o que é mau | bom ou ruim | qualidade |padrão}
  \end{Phonetics}
\end{Entry}

\begin{Entry}{好评}{6,7}{⼥,⾔}
  \begin{Phonetics}{好评}{hao3ping2}[][HSK 7-9]
    \definition{s.}{comentário favorável; opinião elevada; boas críticas; altas críticas}
  \end{Phonetics}
\end{Entry}

\begin{Entry}{好运}{6,7}{⼥,⾡}
  \begin{Phonetics}{好运}{hao3yun4}[][HSK 5]
    \definition{s.}{boa sorte, fortuna ou oportunidade}
  \end{Phonetics}
\end{Entry}

\begin{Entry}{好事}{6,8}{⼥,⼅}
  \begin{Phonetics}{好事}{hao3shi4}[][HSK 2]
    \definition[个,件]{s.}{boa ação; gentileza | (antigo) obra de caridade | acontecimento feliz; evento festivo}
  \end{Phonetics}
  \begin{Phonetics}{好事}{hao4shi4}
    \definition[个,件]{s.}{intrometido; gostar de se meter na vida dos outros}
  \end{Phonetics}
\end{Entry}

\begin{Entry}{好奇}{6,8}{⼥,⼤}
  \begin{Phonetics}{好奇}{hao4qi2}[][HSK 3]
    \definition{adj.}{curioso; curiosidade e interesse por coisas não conhecidas}
    \definition{s.}{curiosidade}
    \definition{v.}{ser ou estar curioso}
  \end{Phonetics}
\end{Entry}

\begin{Entry}{好奇心}{6,8,4}{⼥,⼤,⼼}
  \begin{Phonetics}{好奇心}{hao4qi2xin1}[][HSK 7-9]
    \definition{s.}{curiosidade; uma emoção que expressa atenção especial a algo}
  \end{Phonetics}
\end{Entry}

\begin{Entry}{好学}{6,8}{⼥,⼦}
  \begin{Phonetics}{好学}{hao3xue2}
    \definition{adj.}{fácil de aprender}
  \end{Phonetics}
  \begin{Phonetics}{好学}{hao4xue2}[][HSK 6]
    \definition[个]{s.}{apaixonado para aprender; estudioso; erudito}
  \end{Phonetics}
\end{Entry}

\begin{Entry}{好玩儿}{6,8,2}{⼥,⽟,⼉}
  \begin{Phonetics}{好玩儿}{hao3wan2r5}[][HSK 1]
    \definition{adj.}{divertido; interessante; capaz de despertar interesse}
  \end{Phonetics}
\end{Entry}

\begin{Entry}{好转}{6,8}{⼥,⾞}
  \begin{Phonetics}{好转}{hao3zhuan3}[][HSK 6]
    \definition{v.}{melhorar; dar uma guinada para melhor; tomar um rumo favorável}
  \end{Phonetics}
\end{Entry}

\begin{Entry}{好客}{6,9}{⼥,⼧}
  \begin{Phonetics}{好客}{hao4ke4}[][HSK 7-9]
    \definition{adj.}{hospitaleiro; refere"-se a estar disposto a receber convidados e ser afetuoso com eles}
  \end{Phonetics}
\end{Entry}

\begin{Entry}{好看}{6,9}{⼥,⽬}
  \begin{Phonetics}{好看}{hao3kan4}[][HSK 1]
    \definition{adj.}{de boa aparência; agradável; bonito | interessante; descreve o enredo ou conteúdo de filmes, romances, performances, etc., como sendo cativante, agradável ou apreciável}
  \end{Phonetics}
\end{Entry}

\begin{Entry}{好说}{6,9}{⼥,⾔}
  \begin{Phonetics}{好说}{hao3shuo1}[][HSK 7-9]
    \definition{adj.}{palavras elogiosas; usadas quando alguém agradece ou elogia você; usadas para expressar que você não é digno do elogio | sem problemas; expressa concordância ou vontade de negociar}
  \end{Phonetics}
\end{Entry}

\begin{Entry}{好家伙}{6,10,6}{⼥,⼧,⼈}
  \begin{Phonetics}{好家伙}{hao3jia1huo5}[][HSK 7-9]
    \definition[个]{interj.}{``Bom Deus!''; ``Céus!''; ``Bom Senhor!''; expressa surpresa ou admiração}
  \end{Phonetics}
\end{Entry}

\begin{Entry}{好容易}{6,10,8}{⼥,⼧,⽇}
  \begin{Phonetics}{好容易}{hao3rong2yi4}[][HSK 6]
    \definition{adv.}{com grande dificuldade; ter muita dificuldade (em fazer algo)}
  \end{Phonetics}
\end{Entry}

\begin{Entry}{好笑}{6,10}{⼥,⽵}
  \begin{Phonetics}{好笑}{hao3xiao4}[][HSK 7-9]
    \definition{adj.}{engraçado; divertido; ridículo}
  \end{Phonetics}
\end{Entry}

\begin{Entry}{好象}{6,11}{⼥,⾗}
  \begin{Phonetics}{好象}{hao3xiang4}
    \variantof{好像}
  \end{Phonetics}
\end{Entry}

\begin{Entry}{好像}{6,13}{⼥,⼈}
  \begin{Phonetics}{好像}{hao3xiang4}[][HSK 2]
    \definition{adv.}{como se; um pouco parecido; como se fosse}
    \definition{v.}{parecer; ser como; parecer-se com}
  \end{Phonetics}
\end{Entry}

\begin{Entry}{好意}{6,13}{⼥,⼼}
  \begin{Phonetics}{好意}{hao3yi4}[][HSK 7-9]
    \definition{s.}{boas intenções; gentileza}
  \end{Phonetics}
\end{Entry}

\begin{Entry}{好感}{6,13}{⼥,⼼}
  \begin{Phonetics}{好感}{hao3gan3}[][HSK 7-9]
    \definition{s.}{boa opinião; impressão favorável; sentimentos de satisfação ou simpatia por pessoas ou coisas}
  \end{Phonetics}
\end{Entry}

%%%%%%%%%% 如 %%%%%%%%%%
\subsection*{如}\addcontentsline{loh}{figure}{如}

\begin{Entry}{如}{6}{⼥}
  \begin{Phonetics}{如}{ru2}[][HSK 6]
    \definition{adv.}{por exemplo; tal como; como}
    \definition{conj.}{se; no caso (de); no caso de; como se; como}
    \definition{prep.}{em conformidade com; de acordo com}
    \definition{v.}{estar em conformidade (ou de acordo) com | (geralmente no negativo) pode ser comparado com; ser comparável a; ser tão bom quanto | superar; exceder | (literário) ir para}
  \end{Phonetics}
\end{Entry}

\begin{Entry}{如一}{6,1}{⼥,⼀}
  \begin{Phonetics}{如一}{ru2yi1}[][HSK 6]
    \definition{adj.}{consistente; coerente}
  \end{Phonetics}
\end{Entry}

\begin{Entry}{如下}{6,3}{⼥,⼀}
  \begin{Phonetics}{如下}{ru2xia4}[][HSK 5]
    \definition{adv.}{como descrito ou listado abaixo; conforme segue; conforme abaixo}
  \end{Phonetics}
\end{Entry}

\begin{Entry}{如今}{6,4}{⼥,⼈}
  \begin{Phonetics}{如今}{ru2jin1}[][HSK 4]
    \definition{s.}{agora; hoje em dia; atualmente; no presente}
  \end{Phonetics}
\end{Entry}

\begin{Entry}{如同}{6,6}{⼥,⼝}
  \begin{Phonetics}{如同}{ru2tong2}[][HSK 5]
    \definition{v.}{parecer que; usado principalmente em metáforas}
  \end{Phonetics}
\end{Entry}

\begin{Entry}{如此}{6,6}{⼥,⽌}
  \begin{Phonetics}{如此}{ru2ci3}[][HSK 5]
    \definition{adv.}{assim; tal; dessa forma; dessa maneira; refere"-se a uma situação mencionada anteriormente, equivalente a 这样}
  \seealsoref{这样}{zhe4yang4}
  \end{Phonetics}
\end{Entry}

\begin{Entry}{如何}{6,7}{⼥,⼈}
  \begin{Phonetics}{如何}{ru2he2}[][HSK 3]
    \definition{pron.}{como?; o que?; usado para perguntar como resolver um problema | como?; o que?; usado para perguntar sobre a situação ou obter a opinião de outras pessoas}
  \end{Phonetics}
\end{Entry}

\begin{Entry}{如实}{6,8}{⼥,⼧}
  \begin{Phonetics}{如实}{ru2shi2}[][HSK 7-9]
    \definition{adv.}{factualmente; veridicamente; de acordo com a situação real}
  \end{Phonetics}
\end{Entry}

\begin{Entry}{如果}{6,8}{⼥,⽊}
  \begin{Phonetics}{如果}{ru2guo3}[][HSK 2]
    \definition{conj.}{se; no caso de; na eventualidade de; supondo que; para expressar suposições, pode-se usar 要是 na linguagem falada.}
  \seealsoref{要是}{yao4shi5}
  \end{Phonetics}
\end{Entry}

\begin{Entry}{如果说}{6,8,9}{⼥,⽊,⾔}
  \begin{Phonetics}{如果说}{ru2guo3 shuo1}[][HSK 7-9]
    \definition{conj.}{se}[如果说今天没空,就明天见。===Se você estiver ocupado hoje, nos vemos amanhã.]
  \end{Phonetics}
\end{Entry}

\begin{Entry}{如画}{6,8}{⼥,⽥}
  \begin{Phonetics}{如画}{ru2hua4}
    \definition{adj.}{pitoresco}
  \end{Phonetics}
\end{Entry}

\begin{Entry}{如意}{6,13}{⼥,⼼}
  \begin{Phonetics}{如意}{ru2/yi4}[][HSK 7-9]
    \definition{adj.}{satisfeito; contente; descreve algo como a realização do desejo do coração}
    \definition{v.+compl.}{estar satisfeito; atender aos desejos de alguém; combinar com o meu gosto}
  \end{Phonetics}
\end{Entry}

\begin{Entry}{如愿以偿}{6,14,4,11}{⼥,⽕,⼈,⼈}
  \begin{Phonetics}{如愿以偿}{ru2yuan4yi3chang2}[][HSK 7-9]
    \definition{expr.}{``Como eu desejava.''; ter um desejo realizado; alcançar (ou obter) o que se deseja; o desejo foi atendido conforme o esperado, ou seja, a aspiração foi realizada}
    \definition{s.}{favorabilidade}
  \end{Phonetics}
\end{Entry}

\begin{Entry}{如醉如痴}{6,15,6,13}{⼥,⾣,⼥,⽧}
  \begin{Phonetics}{如醉如痴}{ru2zui4-ru2chi1}[][HSK 7-9]
    \definition{expr.}{embriagado; como se estivesse embriagado e atordoado; intoxicado por alguma coisa; louco por alguma coisa; obcecado por}
  \end{Phonetics}
\end{Entry}

%%%%%%%%%% 妄 %%%%%%%%%%
\subsection*{妄}\addcontentsline{loh}{figure}{妄}

\begin{Entry}{妄}{6}{⼥}
  \begin{Phonetics}{妄}{wang4}
    \definition{adj.}{absurdo; absurdo | ultrajante; ridículo e irracional | precipitado; irresponsável; presunçoso; irracional; fora da rotina; aleatório}
  \end{Phonetics}
\end{Entry}

\begin{Entry}{妄想}{6,13}{⼥,⼼}
  \begin{Phonetics}{妄想}{wang4xiang3}[][HSK 7-9]
    \definition{s.}{ilusão; vã esperança; idéias falsas e irrealizáveis}
    \definition{v.}{fazer uma tentativa vã de; esperar em vão fazer algo; planos que não podem ser realizados}
  \synonymref{梦想}{meng4xiang3}
  \end{Phonetics}
\end{Entry}

%%%%%%%%%% 妆 %%%%%%%%%%
\subsection*{妆}\addcontentsline{loh}{figure}{妆}

\begin{Entry}{妆}{6}{⼥}
  \begin{Phonetics}{妆}{zhuang1}
    \definition{s.}{adornos femininos | enxoval; dote | adornos pessoais femininos; maquiagem e figurino de palco; costumava se referir às maquiagens em mulheres, mas agora se refere às maquiagens em atores}
    \definition{v.}{aplicar maquiagem; maquiar | arrumar-se; maquiar-se}
  \end{Phonetics}
\end{Entry}

\begin{Entry}{妆扮}{6,7}{⼥,⼿}
  \begin{Phonetics}{妆扮}{zhuang1ban4}
    \variantof{装扮}
  \end{Phonetics}
\end{Entry}

%%%%%%%%%% 妇 %%%%%%%%%%
\subsection*{妇}\addcontentsline{loh}{figure}{妇}

\begin{Entry}{妇}{6}{⼥}
  \begin{Phonetics}{妇}{fu4}
    \definition{s.}{mulher | mulher casada | esposa}
  \end{Phonetics}
\end{Entry}

\begin{Entry}{妇女}{6,3}{⼥,⼥}
  \begin{Phonetics}{妇女}{fu4nv3}[][HSK 6]
    \definition[个,位,群,名,帮]{s.}{mulher; mulheres; um termo geral para mulheres adultas}
  \end{Phonetics}
\end{Entry}

%%%%%%%%%% 妈 %%%%%%%%%%
\subsection*{妈}\addcontentsline{loh}{figure}{妈}

\begin{Entry}{妈}{6}{⼥}
  \begin{Phonetics}{妈}{ma1}
    \definition[个,位]{s.}{mãe; mamãe | uma forma de tratamento para uma mulher casada uma geração mais velha | (antigo) uma forma de tratamento para uma empregada doméstica de meia-idade ou velha}
  \seealsoref{妈妈}{ma1ma5}
  \end{Phonetics}
\end{Entry}

\begin{Entry}{妈妈}{6,6}{⼥,⼥}
  \begin{Phonetics}{妈妈}{ma1ma5}[][HSK 1]
    \definition[个,位]{s.}{mamãe; mãe | uma forma de chamar uma mulher de meia-idade; títulos de respeito para mulheres mais velhas}
  \end{Phonetics}
\end{Entry}

%%%%%%%%%% 字 %%%%%%%%%%
\subsection*{字}\addcontentsline{loh}{figure}{字}

\begin{Entry}{字}{6}{⼦}
  \begin{Phonetics}{字}{zi4}[][HSK 1]
    \definition[个]{s.}{palavra; caractere; texto | pronúncia (de uma palavra ou caractere); som do caractere | tipo de impressão; estilo de caligrafia; forma de um caractere escrito ou impresso; refere"-se às diferentes formas dos caracteres chineses; também se refere às diferentes escolas de caligrafia | escritas; obras de caligrafia | recibo; compromisso por escrito; documento | nome de estilo masculino adotado aos vinte anos de idade | sobrenome | um número indicado num contador elétrico, contador de água, etc.; registrar dos números dos medidores de consumo de água e eletricidade}
    \definition{v.}{Arcaico: ficar noiva}
  \end{Phonetics}
\end{Entry}

\begin{Entry}{字母}{6,5}{⼦,⽏}
  \begin{Phonetics}{字母}{zi4mu3}[][HSK 4]
    \definition[个,种]{s.}{letra; letras de um alfabeto | Fonologia: caractere que representa uma consoante inicial}
  \end{Phonetics}
\end{Entry}

\begin{Entry}{字字珠玉}{6,6,10,5}{⼦,⼦,⽟,⽟}
  \begin{Phonetics}{字字珠玉}{zi4zi4zhu1yu4}
    \definition{expr.}{cada palavra é uma jóia}
    \definition{s.}{escrita magnífica}
  \end{Phonetics}
\end{Entry}

\begin{Entry}{字典}{6,8}{⼦,⼋}
  \begin{Phonetics}{字典}{zi4dian3}[][HSK 2]
    \definition[本,册,部]{s.}{dicionário de caracteres chineses (contendo verbetes de caracteres únicos, em contraste com 词典 que contém verbetes para palavras com um ou mais caracteres)}
  \seealsoref{词典}{ci2dian3}
  \end{Phonetics}
\end{Entry}

\begin{Entry}{字眼}{6,11}{⼦,⽬}
  \begin{Phonetics}{字眼}{zi4yan3}
    \definition[个]{s.}{palavras | redação}
  \end{Phonetics}
\end{Entry}

\begin{Entry}{字脚}{6,11}{⼦,⾁}
  \begin{Phonetics}{字脚}{zi4jiao3}
    \definition[典]{s.}{gancho no final da pincelada | serifa}
  \end{Phonetics}
\end{Entry}

%%%%%%%%%% 存 %%%%%%%%%%
\subsection*{存}\addcontentsline{loh}{figure}{存}

\begin{Entry}{存}{6}{⼦}
  \begin{Phonetics}{存}{cun2}[][HSK 3]
    \definition{v.}{existir; viver; sobreviver | armazenar; manter | acumular; coletar | depositar | sair com; verificar | reservar; reter | permanecer em equilíbrio; estar em estoque | estimar; abrigar}
  \end{Phonetics}
\end{Entry}

\begin{Entry}{存心}{6,4}{⼦,⼼}
  \begin{Phonetics}{存心}{cun2xin1}[][HSK 7-9]
    \definition{adv.}{intencionalmente; deliberadamente; de propósito}
  \end{Phonetics}
\end{Entry}

\begin{Entry}{存在}{6,6}{⼦,⼟}
  \begin{Phonetics}{存在}{cun2zai4}[][HSK 3]
    \definition{s.}{existência; ser; ente; o mundo objetivo, que não depende da consciência humana para mudar, ou seja, a matéria}
    \definition{v.}{existir; ser; as coisas ocupam continuamente o tempo e o espaço; na verdade, ainda não desapareceram}
  \end{Phonetics}
\end{Entry}

\begin{Entry}{存折}{6,7}{⼦,⼿}
  \begin{Phonetics}{存折}{cun2zhe2}[][HSK 7-9]
    \definition{s.}{caderneta bancária; caderneta de poupança; livro de depósitos; livro de poupança bancária; um \emph{voucher} emitido por uma instituição financeira a um depositante como um certificado}
  \end{Phonetics}
\end{Entry}

\begin{Entry}{存放}{6,8}{⼦,⽅}
  \begin{Phonetics}{存放}{cun2fang4}[][HSK 7-9]
    \definition{v.}{armazenar; guardar; deixar com}
  \end{Phonetics}
\end{Entry}

\begin{Entry}{存款}{6,12}{⼦,⽋}
  \begin{Phonetics}{存款}{cun2kuan3}[][HSK 5]
    \definition[些,笔]{s.}{depósito; poupança bancária}
    \definition{v.}{depositar dinheiro; colocar dinheiro no banco}
  \end{Phonetics}
\end{Entry}

%%%%%%%%%% 孙 %%%%%%%%%%
\subsection*{孙}\addcontentsline{loh}{figure}{孙}

\begin{Entry}{孙}{6}{⼦}
  \begin{Phonetics}{孙}{sun1}
    \definition*{s.}{Sobrenome: Sun}
    \definition{s.}{neto; neta | gerações abaixo da do neto | parentes pertencentes à geração do neto | segundo crescimento das plantas}
  \end{Phonetics}
\end{Entry}

\begin{Entry}{孙女}{6,3}{⼦,⼥}
  \begin{Phonetics}{孙女}{sun1nv5}[][HSK 4]
    \definition[个]{s.}{filha do filho; neta}
  \end{Phonetics}
\end{Entry}

\begin{Entry}{孙子}{6,3}{⼦,⼦}
  \begin{Phonetics}{孙子}{sun1zi3}
    \definition*{s.}{Sun Tzu, também conhecido por Sun Wu (孙武), general, estrategista e filósofo autor do ``Arte da Guerra''《孙子兵法》}
  \seealsoref{孙武}{sun1wu3}
  \seealsoref{孙子兵法}{sun1zi3 bing1fa3}
  \end{Phonetics}
  \begin{Phonetics}{孙子}{sun1zi5}[][HSK 4]
    \definition[个]{s.}{filho do filho; neto}
  \end{Phonetics}
\end{Entry}

\begin{Entry}{孙子兵法}{6,3,7,8}{⼦,⼦,⼋,⽔}
  \begin{Phonetics}{孙子兵法}{sun1zi3 bing1fa3}
    \definition*{s.}{``Arte da Guerra'', o antigo clássico chinês sobre estratégia militar, escrito por Sun Tzu (孫子)}
  \seealsoref{孙武}{sun1wu3}
  \seealsoref{孙子}{sun1zi3}
  \end{Phonetics}
\end{Entry}

\begin{Entry}{孙武}{6,8}{⼦,⽌}
  \begin{Phonetics}{孙武}{sun1wu3}
    \definition*{s.}{Sun Wu, também conhecido por Sun Tzu (孙子) general, estrategista e filósofo autor do ``Arte da Guerra''《孙子兵法》}
  \seealsoref{孙子}{sun1zi3}
  \seealsoref{孙子兵法}{sun1zi3 bing1fa3}
  \end{Phonetics}
\end{Entry}

%%%%%%%%%% 宇 %%%%%%%%%%
\subsection*{宇}\addcontentsline{loh}{figure}{宇}

\begin{Entry}{宇}{6}{⼧}
  \begin{Phonetics}{宇}{yu3}
    \definition*{s.}{Sobrenome: Yu}
    \definition[座,栋]{s.}{beirais; calha; casa | espaço; universo; mundo | postura; porte}
  \end{Phonetics}
\end{Entry}

\begin{Entry}{宇宙}{6,8}{⼧,⼧}
  \begin{Phonetics}{宇宙}{yu3zhou4}
    \definition{s.}{universo | cosmos}
  \end{Phonetics}
\end{Entry}

\begin{Entry}{宇航员}{6,10,7}{⼧,⾈,⼝}
  \begin{Phonetics}{宇航员}{yu3hang2yuan2}[][HSK 6]
    \definition[位,名,个,些]{s.}{astronauta; cosmonauta;}
  \end{Phonetics}
\end{Entry}

%%%%%%%%%% 守 %%%%%%%%%%
\subsection*{守}\addcontentsline{loh}{figure}{守}

\begin{Entry}{守}{6}{⼧}
  \begin{Phonetics}{守}{shou3}[][HSK 4]
    \definition*{s.}{Sobrenome: Shou}
    \definition{adv.}{próximo; perto de; perto de algum lugar em posição, perto de algum lugar}
    \definition{v.}{guardar; defender; estar presente para cuidar; não ir embora | manter vigilância; defender do ataque do oponente em uma luta ou confronto | observar; cumprir; respeitar; fazer as coisas como elas devem ser feitas | manter, observar a integridade; honrar a palavra de alguém; manter a palavra de alguém}
  \end{Phonetics}
\end{Entry}

\begin{Entry}{守门员}{6,3,7}{⼧,⾨,⼝}
  \begin{Phonetics}{守门员}{shou3men2yuan2}
    \definition{s.}{goleiro}
  \end{Phonetics}
\end{Entry}

\begin{Entry}{守护}{6,7}{⼧,⼿}
  \begin{Phonetics}{守护}{shou3hu4}[][HSK 7-9]
    \definition{v.}{guardar; defender; aparar; vigiar}
  \synonymref{保护}{bao3hu4}
  \synonymref{保卫}{bao3wei4}
  \synonymref{防守}{fang2shou3}
  \antonymref{破坏}{po4huai4}
  \end{Phonetics}
\end{Entry}

\begin{Entry}{守候}{6,10}{⼧,⼈}
  \begin{Phonetics}{守候}{shou3hou4}[][HSK 7-9]
    \definition{v.}{esperar; esperar por | continuar observando; cuidar de}
  \synonymref{等待}{deng3dai4}
  \synonymref{等候}{deng3hou4}
  \synonymref{期待}{qi1dai4}
  \antonymref{放弃}{fang4qi4}
  \end{Phonetics}
\end{Entry}

\begin{Entry}{守株待兔}{6,10,9,8}{⼧,⽊,⼻,⼉}
  \begin{Phonetics}{守株待兔}{shou3zhu1-dai4tu4}[][HSK 7-9]
    \definition{expr.}{``Esperando que um coelho bata num toco de árvore.''; guardar um toco de árvore; esperar ociosamente por oportunidades; esperar por coelhos; conta a lenda que, durante o período dos Reinos Combatentes, um agricultor do Estado de Song viu um coelho morrer após se chocar contra um toco de árvore, ele então largou suas ferramentas agrícolas e esperou ali, na esperança de pegar outro coelho que tivesse morrido da mesma forma (de ``Han Feizi'', ``Cinco Pragas''); essa história é usada para descrever alguém que, em vez de se esforçar, nutre uma mentalidade de apostador e espera por um ganho inesperado}
  \synonymref{刻舟求剑}{ke4zhou1-qiu2jian4}
  \end{Phonetics}
\end{Entry}

%%%%%%%%%% 安 %%%%%%%%%%
\subsection*{安}\addcontentsline{loh}{figure}{安}

\begin{Entry}{安}{6}{⼧}
  \begin{Phonetics}{安}{an1}[][HSK 4]
    \definition*{s.}{Sobrenome: An}
    \definition{adj.}{pacífico; quieto; tranquilo; calmo | seguro; protegido | com boa saúde | em paz; bem}
    \definition{adv.}{pacificamente; silenciosamente | com segurança; em segurança | em perguntas retóricas: ``como?''}
    \definition{pron.}{usado como pronome interrogativo, como em 哪里,怎么; 谁,何,如何}
    \definition{s.}{segurança; proteção; paz | ampère; abreviação de ampère, 安培}
    \definition{v.}{tranquilizar (a mente de alguém); acalmar | contentar"-se; ficar satisfeito | colocar em uma posição adequada; encontrar um lugar para | instalar; consertar; encaixar; configurar | trazer (uma acusação contra alguém); dar (a alguém um apelido); reivindicar (crédito por algo) | abrigar (uma intenção) | acalmar; estabilizar | sentir"-se satisfeito e à vontade}
  \seealsoref{安培}{an1pei2}
  \seealsoref{何}{he2}
  \seealsoref{哪里}{na3li5}
  \seealsoref{如何}{ru2he2}
  \seealsoref{谁}{shei2}
  \seealsoref{怎么}{zen3me5}
  \antonymref{危}{wei1}
  \end{Phonetics}
\end{Entry}

\begin{Entry}{安心}{6,4}{⼧,⼼}
  \begin{Phonetics}{安心}{an1xin1}[][HSK 7-9]
    \definition{adj.}{aliviado; à vontade; tranquilo; seguro}
    \definition{v.}{abrigar (más) intenções; acalentar certas intenções; ter (pensamentos ruins) em mente}
  \end{Phonetics}
\end{Entry}

\begin{Entry}{安宁}{6,5}{⼧,⼧}
  \begin{Phonetics}{安宁}{an1ning2}[][HSK 7-9]
    \definition{adj.}{pacífico; tranquilo; descreve um estado de ordem normal sem fatores externos que causem desordem ou inquietação | calmo; composto; livre de preocupações, ansiedades ou inquietações}
  \end{Phonetics}
\end{Entry}

\begin{Entry}{安全}{6,6}{⼧,⼊}
  \begin{Phonetics}{安全}{an1quan2}[][HSK 2]
    \definition{adj.}{seguro; protegido; sem perigo; sem ameaças; sem acidentes}
    \definition{s.}{segurança; proteção; refere"-se a um estado ou conceito, geralmente indicando ausência de ameaças ou perigo}
  \end{Phonetics}
\end{Entry}

\begin{Entry}{安抚}{6,7}{⼧,⼿}
  \begin{Phonetics}{安抚}{an1fu3}[][HSK 7-9]
    \definition{v.}{pacificar; consolar; apaziguar; tranquilizar e acalmar}
  \end{Phonetics}
\end{Entry}

\begin{Entry}{安定}{6,8}{⼧,⼧}
  \begin{Phonetics}{安定}{an1ding4}[][HSK 7-9]
    \definition{adj.}{estável; tranquilo; estabelecido; pacífico e em ordem; estável e normal, sem flutuações}
    \definition{s.}{tranquilizante; medicina ocidental comumente usada, com efeitos sedativos e anticonvulsivantes}
    \definition{v.}{acalmar; estabilizar; manter}
  \end{Phonetics}
\end{Entry}

\begin{Entry}{安神}{6,9}{⼧,⽰}
  \begin{Phonetics}{安神}{an1/shen2}
    \definition{v.+compl.}{acalmar os nervos | aliviar a inquietação pela tranquilização da mente e do corpo}
  \end{Phonetics}
\end{Entry}

\begin{Entry}{安家}{6,10}{⼧,⼧}
  \begin{Phonetics}{安家}{an1/jia1}
    \definition{v.+compl.}{montar uma casa | estabelecer"-se}
  \end{Phonetics}
\end{Entry}

\begin{Entry}{安眠药}{6,10,9}{⼧,⽬,⾋}
  \begin{Phonetics}{安眠药}{an1mian2yao4}[][HSK 7-9]
    \definition[片,颗,粒,瓶]{s.}{comprimido para dormir; soporífero; pílula para dormir; medicamentos que podem suprimir o córtex cerebral e induzir o sono}
  \end{Phonetics}
\end{Entry}

\begin{Entry}{安培}{6,11}{⼧,⼟}
  \begin{Phonetics}{安培}{an1pei2}
    \definition{clas.}{A; empréstimo linguístico: ampere; física: unidade de corrente elétrica}
  \end{Phonetics}
\end{Entry}

\begin{Entry}{安排}{6,11}{⼧,⼿}
  \begin{Phonetics}{安排}{an1pai2}[][HSK 3]
    \definition{s.}{plano; programação; organização; tabela do plano de atividades ou horários}
    \definition{v.}{organizar (assuntos) de acordo com a sequência ou regras; tratar as coisas de acordo com uma determinada ordem ou regras | atribuir tarefas a alguém; colocar as pessoas nos cargos de trabalho determinados, conforme planejado}
  \end{Phonetics}
\end{Entry}

\begin{Entry}{安检}{6,11}{⼧,⽊}
  \begin{Phonetics}{安检}{an1jian3}[][HSK 6]
    \definition{s.}{verificação de segurança}
    \definition{v.}{realizar verificação de segurança}
  \end{Phonetics}
\end{Entry}

\begin{Entry}{安逸}{6,11}{⼧,⾡}
  \begin{Phonetics}{安逸}{an1yi4}[][HSK 7-9]
    \definition{adj.}{fácil; fácil e confortável; relaxado e confortável}
    \definition{s.}{conforto e facilidade; conforto}
  \end{Phonetics}
\end{Entry}

\begin{Entry}{安装}{6,12}{⼧,⾐}
  \begin{Phonetics}{安装}{an1zhuang1}[][HSK 3]
    \definition{v.}{instalar; consertar; configurar; fixar máquinas ou equipamentos (geralmente conjuntos) em um determinado local, de acordo com métodos e especificações específicos}
  \end{Phonetics}
\end{Entry}

\begin{Entry}{安置}{6,13}{⼧,⽹}
  \begin{Phonetics}{安置}{an1zhi4}[][HSK 4]
    \definition{v.}{providenciar; encontrar um lugar para; ajudar a estabelecer"-se; colocar pessoas ou coisas em uma determinada posição ou organizá"-las adequadamente}
  \end{Phonetics}
\end{Entry}

\begin{Entry}{安稳}{6,14}{⼧,⽲}
  \begin{Phonetics}{安稳}{an1wen3}[][HSK 7-9]
    \definition{adj.}{seguro; suave e estável; estável | composto; calmo e equilibrado; (comportamento) estável; calmo}
  \end{Phonetics}
\end{Entry}

\begin{Entry}{安静}{6,14}{⼧,⾭}
  \begin{Phonetics}{安静}{an1jing4}[][HSK 2]
    \definition{adj.}{silencioso; tranquilo; sem som; sem barulho e sem algazarra}
  \end{Phonetics}
\end{Entry}

\begin{Entry}{安慰}{6,15}{⼧,⼼}
  \begin{Phonetics}{安慰}{an1wei4}[][HSK 5]
    \definition{adj.}{confortar; tranquilizar; consolar; apaziguar;}
    \definition[句,通,番,声,个]{s.}{conforto; consolo; comportamento que alivia a dor de alguém e o acalma com palavras ou gestos}
    \definition{v.}{confortar; consolar; acalmar e confortar; deixar a mente tranquila}
  \end{Phonetics}
\end{Entry}

%%%%%%%%%% 寺 %%%%%%%%%%
\subsection*{寺}\addcontentsline{loh}{figure}{寺}

\begin{Entry}{寺}{6}{⼨}
  \begin{Phonetics}{寺}{si4}[][HSK 6]
    \definition*{s.}{Sobrenome: Si}
    \definition[座]{s.}{templo | (Islã) mesquita | Obsoleto: ministério; agência governamental na China antiga}
  \end{Phonetics}
\end{Entry}

\begin{Entry}{寺庙}{6,8}{⼨,⼴}
  \begin{Phonetics}{寺庙}{si4miao4}[][HSK 7-9]
    \definition[座]{s.}{mosteiro; casa de Deus; templo; templos budistas}
  \end{Phonetics}
\end{Entry}

%%%%%%%%%% 寻 %%%%%%%%%%
\subsection*{寻}\addcontentsline{loh}{figure}{寻}

\begin{Entry}{寻}{6}{⼨}
  \begin{Phonetics}{寻}{xun2}
    \definition*{s.}{Sobrenome: Xun}
    \definition{clas.}{uma unidade antiga de comprimento, igual a 8 尺}
    \definition{v.}{procurar; pesquisar; buscar}
  \seealsoref{尺}{chi3}
  \end{Phonetics}
\end{Entry}

\begin{Entry}{寻找}{6,7}{⼨,⼿}
  \begin{Phonetics}{寻找}{xun2zhao3}[][HSK 4]
    \definition{v.}{buscar; procurar; pesquisar; encontrar, que pode ser usado tanto para coisas concretas quanto para coisas abstratas}
  \end{Phonetics}
\end{Entry}

\begin{Entry}{寻求}{6,7}{⼨,⽔}
  \begin{Phonetics}{寻求}{xun2qiu2}[][HSK 5]
    \definition{v.}{procurar; perseguir; explorar; ir em busca de}
  \end{Phonetics}
\end{Entry}

%%%%%%%%%% 导 %%%%%%%%%%
\subsection*{导}\addcontentsline{loh}{figure}{导}

\begin{Entry}{导}{6}{⼨}
  \begin{Phonetics}{导}{dao3}
    \definition[个,位,名,些]{s.}{guia turístico | diretor}
    \definition{v.}{liderar; guiar | conduzir; transmitir | ensinar; instruir; dar orientação a}
  \end{Phonetics}
\end{Entry}

\begin{Entry}{导火索}{6,4,10}{⼨,⽕,⽷}
  \begin{Phonetics}{导火索}{dao3huo3suo3}[][HSK 7-9]
    \definition{s.}{fusível (para explosivo)}
  \end{Phonetics}
\end{Entry}

\begin{Entry}{导向}{6,6}{⼨,⼝}
  \begin{Phonetics}{导向}{dao3xiang4}[][HSK 7-9]
    \definition{s.}{orientação; direção}
    \definition{v.}{guiar; orientar; dirigir}
  \end{Phonetics}
\end{Entry}

\begin{Entry}{导师}{6,6}{⼨,⼱}
  \begin{Phonetics}{导师}{dao3shi1}[][HSK 7-9]
    \definition[位,个]{s.}{tutor; professor; orientador; supervisor; uma pessoa que orienta outras pessoas em seus estudos, educação continuada ou redação de trabalhos em faculdades, universidades ou instituições de pesquisa | mentor; guia de uma grande causa; uma pessoa que fornece orientação em grandes empreendimentos e movimentos}
  \end{Phonetics}
\end{Entry}

\begin{Entry}{导致}{6,10}{⼨,⾄}
  \begin{Phonetics}{导致}{dao3zhi4}[][HSK 4]
    \definition{v.}{causar; levar a; dar origem a (um resultado ruim)}
  \end{Phonetics}
\end{Entry}

\begin{Entry}{导航}{6,10}{⼨,⾈}
  \begin{Phonetics}{导航}{dao3hang2}[][HSK 7-9]
    \definition{s.}{navegação; tecnologia que guia aviões, navios ou carros por rotas seguras}
    \definition{v.}{navegar; guiar um avião, navio ou carro por uma rota segura}
  \end{Phonetics}
\end{Entry}

\begin{Entry}{导弹}{6,11}{⼨,⼸}
  \begin{Phonetics}{导弹}{dao3dan4}[][HSK 7-9]
    \definition[枚,颗,个]{s.}{míssil (guiado)}
  \seealsoref{飞弹}{fei1dan4}
  \end{Phonetics}
\end{Entry}

\begin{Entry}{导游}{6,12}{⼨,⽔}
  \begin{Phonetics}{导游}{dao3you2}[][HSK 4]
    \definition[个,位,名]{s.}{guia turístico; pessoas que trabalham como guias turísticos}
    \definition{v.}{guiar; conduzir um passeio turístico}
  \end{Phonetics}
\end{Entry}

\begin{Entry}{导演}{6,14}{⼨,⽔}
  \begin{Phonetics}{导演}{dao3yan3}[][HSK 3]
    \definition[位,名,个]{s.}{diretor; pessoa que exerce a função de diretor}
    \definition{v.}{dirigir (um filme, peça, etc.); ensaio de peças teatrais ou filmagem de filmes e séries de TV; organização e orientação do trabalho de produção}
  \end{Phonetics}
\end{Entry}

%%%%%%%%%% 尖 %%%%%%%%%%
\subsection*{尖}\addcontentsline{loh}{figure}{尖}

\begin{Entry}{尖}{6}{⼩}
  \begin{Phonetics}{尖}{jian1}[][HSK 6]
    \definition{adj.}{pontiagudo; afilado; agudo | agudo; estridente; penetrante | mesquinho; pão-duro | mordaz; cáustico}
    \definition{s.}{ponto; ponta; topo | o melhor do seu tipo; a melhor escolha; a nata da safra; uma pessoa ou coisa notável}
    \definition{v.}{tornar (a voz, etc.) aguda; estridente}
  \end{Phonetics}
\end{Entry}

\begin{Entry}{尖锐}{6,12}{⼩,⾦}
  \begin{Phonetics}{尖锐}{jian1rui4}[][HSK 7-9]
    \definition{adj.}{pontiagudo; a ponta do objeto é fina e pequena, podendo facilmente entrar ou passar por outros objetos | incisivo; penetrante; conhecer e observar as coisas com muita precisão e profundidade | estridente; penetrante; o som é alto e fino, fazendo com que as pessoas se sintam desconfortáveis | afiado; intenso; muito intenso (discussão, luta, etc.)}
  \end{Phonetics}
\end{Entry}

\begin{Entry}{尖端}{6,14}{⼩,⽴}
  \begin{Phonetics}{尖端}{jian1duan1}[][HSK 7-9]
    \definition{adj.}{mais avançado; sofisticado; o mais alto nível de desenvolvimento (ciência e tecnologia, etc.)}
    \definition{s.}{ponta pontiaguda; pico; a ponta fina e pontiaguda de algo | topo; ápice; pico; uma metáfora para o mais alto nível de ciência e tecnologia}
  \end{Phonetics}
\end{Entry}

%%%%%%%%%% 尧 %%%%%%%%%%
\subsection*{尧}\addcontentsline{loh}{figure}{尧}

\begin{Entry}{尧}{6}{⼪}
  \begin{Phonetics}{尧}{yao2}
    \definition*{s.}{Yao, um monarca lendário da China antiga | Sobrenome: Yao}
  \end{Phonetics}
\end{Entry}

%%%%%%%%%% 尽 %%%%%%%%%%
\subsection*{尽}\addcontentsline{loh}{figure}{尽}

\begin{Entry}{尽}{6}{⼫}
  \begin{Phonetics}{尽}{jin3}[][HSK 7-9]
    \definition{adv.}{na maior extensão possível | na extremidade mais distante de | usado antes de palavras que indicam direção, o mesmo que 最 | de agora em diante}
    \definition{prep.}{dentro dos limites de}
    \definition{v.}{dar prioridade a; deixar que certas pessoas ou coisas tenham precedência}
  \seealsoref{最}{zui4}
  \end{Phonetics}
  \begin{Phonetics}{尽}{jin4}[][HSK 6]
    \definition*{s.}{Sobrenome: Jin}
    \definition{adj.}{exausto; acabado | ao máximo; ao limite | tudo; exaustivo}
    \definition{v.}{esgotar | tentar o seu melhor; fazer o melhor uso possível | morrer; falecer | terminar | chegar ao fim ao máximo; alcançar extremos}
  \end{Phonetics}
\end{Entry}

\begin{Entry}{尽力}{6,2}{⼫,⼒}
  \begin{Phonetics}{尽力}{jin4/li4}[][HSK 4]
    \definition{v.+compl.}{esforçar-se ao máximo; esforçar-se ao máximo; usar toda a sua força; fazer algo com seu melhor esforço}
  \end{Phonetics}
\end{Entry}

\begin{Entry}{尽可能}{6,5,10}{⼫,⼝,⾁}
  \begin{Phonetics}{尽可能}{jin3ke3neng2}[][HSK 5]
    \definition{adv.}{na medida do possível; com o melhor de sua capacidade; tentar fazer algo, atingir um determinado nível ou extensão}
  \end{Phonetics}
\end{Entry}

\begin{Entry}{尽头}{6,5}{⼫,⼤}
  \begin{Phonetics}{尽头}{jin4tou2}[][HSK 7-9]
    \definition[台]{s.}{fim}[小路尽头是一片树林。===No fim do caminho havia um bosque de árvores.]
  \end{Phonetics}
\end{Entry}

\begin{Entry}{尽早}{6,6}{⼫,⽇}
  \begin{Phonetics}{尽早}{jin3zao3}[][HSK 7-9]
    \definition{adv.}{o mais cedo possível; assim que possível; indica que deve ser feito o mais cedo possível}
  \end{Phonetics}
\end{Entry}

\begin{Entry}{尽快}{6,7}{⼫,⼼}
  \begin{Phonetics}{尽快}{jin3kuai4}[][HSK 4]
    \definition{adv.}{com toda a velocidade; o mais rápido possível; o mais breve possível}
  \end{Phonetics}
\end{Entry}

\begin{Entry}{尽情}{6,11}{⼫,⼼}
  \begin{Phonetics}{尽情}{jin4qing2}[][HSK 7-9]
    \definition{v.}{expressar os próprios sentimentos de forma plena e livre; significa agir de acordo com os próprios sentimentos, na medida do possível}
  \end{Phonetics}
\end{Entry}

\begin{Entry}{尽量}{6,12}{⼫,⾥}
  \begin{Phonetics}{尽量}{jin3liang4}[][HSK 3]
    \definition{adv.}{tanto quanto possível; da melhor maneira possível}
  \end{Phonetics}
\end{Entry}

\begin{Entry}{尽管}{6,14}{⼫,⽵}
  \begin{Phonetics}{尽管}{jin3guan3}[][HSK 5]
    \definition{adv.}{justo; livremente; faça o que quiser, não se preocupe, não há restrições de movimento ou comportamento}
    \definition{conj.}{no entanto; embora; apesar de ; normalmente usado no início de uma frase anterior para introduzir um fato, seguido de 但是, etc. para introduzir um resultado que o fato não deveria ter; às vezes, também pode ser usado no início de uma frase posterior.}
  \seealsoref{但是}{dan4shi4}
  \end{Phonetics}
\end{Entry}

%%%%%%%%%% 岁 %%%%%%%%%%
\subsection*{岁}\addcontentsline{loh}{figure}{岁}

\begin{Entry}{岁}{6}{⼭}
  \begin{Phonetics}{岁}{sui4}[][HSK 1]
    \definition{clas.}{usado para anos (de idade)}
    \definition{s.}{ano (literário) | colheita do ano (literário) | idade | tempo (literário) | ano (de idade) | ano (para as colheitas)}
  \end{Phonetics}
\end{Entry}

\begin{Entry}{岁月}{6,4}{⼭,⽉}
  \begin{Phonetics}{岁月}{sui4yue4}[][HSK 5]
    \definition[段,番]{s.}{anos; ano e mês; refere"-se a tempo em geral}
  \end{Phonetics}
\end{Entry}

\begin{Entry}{岁数}{6,13}{⼭,⽁}
  \begin{Phonetics}{岁数}{sui4shu4}[][HSK 6]
    \definition{s.}{idade; anos; a idade de uma pessoa}
  \end{Phonetics}
\end{Entry}

%%%%%%%%%% 岂 %%%%%%%%%%
\subsection*{岂}\addcontentsline{loh}{figure}{岂}

\begin{Entry}{岂}{6}{⼭}
  \begin{Phonetics}{岂}{qi3}
    \definition*{s.}{Sobrenome: Qi}
    \definition{adv.}{Litarário: expressa uma pergunta retórica, equivalente a 哪里, 怎么 e 难道}
  \seealsoref{哪里}{na3li5}
  \seealsoref{难道}{nan2dao4}
  \seealsoref{怎么}{zen3me5}
  \end{Phonetics}
\end{Entry}

\begin{Entry}{岂有此理}{6,6,6,11}{⼭,⽉,⽌,⽟}
  \begin{Phonetics}{岂有此理}{qi3you3ci3li3}[][HSK 7-9]
    \definition{expr.}{Isso é um absurdo!; Que exorbitante!; Absurdo!; Como isso pode ser assim?; Ridículo!}
    \definition{interj.}{``Isso é um absurdo!''; ``Que exorbitante!''; ``Absurdo!''; ``Como isso pode ser assim?''; ``Ridículo!''}
  \end{Phonetics}
\end{Entry}

%%%%%%%%%% 巡 %%%%%%%%%%
\subsection*{巡}\addcontentsline{loh}{figure}{巡}

\begin{Entry}{巡}{6}{⾡}
  \begin{Phonetics}{巡}{xun2}
    \definition{clas.}{rodada de bebidas | usado para servir vinho a todos}
    \definition{v.}{patrulhar; fazer rondas; fazer uma excursão de inspeção}
  \end{Phonetics}
\end{Entry}

\begin{Entry}{巡逻}{6,11}{⾡,⾡}
  \begin{Phonetics}{巡逻}{xun2luo2}
    \definition{s.}{patrulha}
    \definition{v.}{patrulhar (polícia, exército ou marinha)}
  \end{Phonetics}
\end{Entry}

%%%%%%%%%% 巩 %%%%%%%%%%
\subsection*{巩}\addcontentsline{loh}{figure}{巩}

\begin{Entry}{巩}{6}{⼯}
  \begin{Phonetics}{巩}{gong3}
    \definition*{s.}{Sobrenome: Gong}
    \definition{s.}{seguro | sólido}
    \definition{v.}{consolidar}
  \end{Phonetics}
\end{Entry}

\begin{Entry}{巩固}{6,8}{⼯,⼞}
  \begin{Phonetics}{巩固}{gong3gu4}[][HSK 6]
    \definition{adj.}{sólido; estável; consolidado; não facilmente abalado (usado principalmente para coisas abstratas)}
    \definition{v.}{consolidar}
  \end{Phonetics}
\end{Entry}

%%%%%%%%%% 帆 %%%%%%%%%%
\subsection*{帆}\addcontentsline{loh}{figure}{帆}

\begin{Entry}{帆}{6}{⼱}
  \begin{Phonetics}{帆}{fan1}[][HSK 7-9]
    \definition{s.}{vela (de barco) | Literário: barco à vela; veleiro}
  \end{Phonetics}
\end{Entry}

\begin{Entry}{帆船}{6,11}{⼱,⾈}
  \begin{Phonetics}{帆船}{fan1chuan2}[][HSK 7-9]
    \definition[艘,条]{s.}{veleiro; barco a vela; um navio que usa velas para se impulsionar com a ajuda do vento}
  \end{Phonetics}
\end{Entry}

%%%%%%%%%% 师 %%%%%%%%%%
\subsection*{师}\addcontentsline{loh}{figure}{师}

\begin{Entry}{师}{6}{⼱}
  \begin{Phonetics}{师}{shi1}
    \definition*{s.}{Sobrenome: Shi}
    \definition[位,名,个]{s.}{professor; tutor; mestre | exemplo; modelo a seguir | título honorífico para um monge budista; (termo de respeito para um monge ou freira) mestre; mãe | do seu mestre ou professor | divisão; tropas; exército}
    \definition{suf.}{pessoa qualificada em determinada profissão}
    \definition{v.}{Literário: imitar; aprender}
  \end{Phonetics}
\end{Entry}

\begin{Entry}{师父}{6,4}{⼱,⽗}
  \begin{Phonetics}{师父}{shi1fu5}[][HSK 6]
    \definition[个,位,名,些]{s.}{mestre; mestre trabalhador; um título respeitoso dado por um aprendiz ao seu mestre | um título respeitoso para monges, freiras e sacerdotes taoístas}
  \end{Phonetics}
\end{Entry}

\begin{Entry}{师长}{6,4}{⼱,⾧}
  \begin{Phonetics}{师长}{shi1zhang3}[][HSK 7-9]
    \definition{s.}{Cortês: professor | Militar: comandante de divisão}
  \synonymref{教授}{jiao4shou4}
  \synonymref{老师}{lao3shi1}
  \synonymref{先生}{xian1sheng5}
  \antonymref{学生}{xue2sheng5}
  \end{Phonetics}
\end{Entry}

\begin{Entry}{师生}{6,5}{⼱,⽣}
  \begin{Phonetics}{师生}{shi1sheng1}[][HSK 6]
    \definition{s.}{mestre e discípulo; professores e alunos; um nome combinado para professores e alunos}
  \end{Phonetics}
\end{Entry}

\begin{Entry}{师范}{6,9}{⼱,⾋}
  \begin{Phonetics}{师范}{shi1fan4}[][HSK 7-9]
    \definition{s.}{pedagogia; formação de professores; escola normal; escola especializada na formação de professores, abreviação de 师范学校, 师范学院 ou 师范大学 | modelo; uma pessoa de virtude exemplar}
  \seealsoref{师范大学}{shi1fan4 da4xue2}
  \seealsoref{师范学校}{shi1fan4 xue2xiao4}
  \seealsoref{师范学院}{shi1fan4 xue2yuan4}
  \end{Phonetics}
\end{Entry}

\begin{Entry}{师范大学}{6,9,3,8}{⼱,⾋,⼤,⼦}
  \begin{Phonetics}{师范大学}{shi1fan4 da4xue2}
    \definition{s.}{universidade normal; universidade (normal) de professores; universidade de pedagogia}
  \end{Phonetics}
\end{Entry}

\begin{Entry}{师范学院}{6,9,8,9}{⼱,⾋,⼦,⾩}
  \begin{Phonetics}{师范学院}{shi1fan4 xue2yuan4}
    \definition{s.}{faculdade de formação de professores; faculdade de pedagogia}
  \end{Phonetics}
\end{Entry}

\begin{Entry}{师范学校}{6,9,8,10}{⼱,⾋,⼦,⽊}
  \begin{Phonetics}{师范学校}{shi1fan4 xue2xiao4}
    \definition{s.}{escola normal; escolas especializadas na formação de professores}
  \end{Phonetics}
\end{Entry}

\begin{Entry}{师资}{6,10}{⼱,⾙}
  \begin{Phonetics}{师资}{shi1zi1}[][HSK 7-9]
    \definition{s.}{pessoas qualificadas para ensinar; indivíduos talentosos e qualificados para serem professores; professores | corpo docente; qualificações dos professores; recursos para o corpo docente}
  \end{Phonetics}
\end{Entry}

\begin{Entry}{师傅}{6,12}{⼱,⼈}
  \begin{Phonetics}{师傅}{shi1fu5}[][HSK 5]
    \definition[个,位,名]{s.}{mestre; um trabalhador qualificado; título honorífico para pessoas habilidosas | mestre; professor (em certos ofícios); pessoas que ensinam técnicas em áreas como engenharia, comércio e teatro}
  \end{Phonetics}
\end{Entry}

%%%%%%%%%% 年 %%%%%%%%%%
\subsection*{年}\addcontentsline{loh}{figure}{年}

\begin{Entry}{年}{6}{⼲}
  \begin{Phonetics}{年}{nian2}[][HSK 1]
    \definition*{s.}{Sobrenome: Nian}
    \definition{clas.}{ano; usado para calcular o número de anos}
    \definition{s.}{ano | idade | um período (época) da história | colheita anual | Ano Novo | artigos para o dia de Ano Novo | um período da vida de uma pessoa; fases da vida humana divididas por idade}
  \end{Phonetics}
\end{Entry}

\begin{Entry}{年代}{6,5}{⼲,⼈}
  \begin{Phonetics}{年代}{nian2dai4}[][HSK 3]
    \definition[个]{s.}{idade; anos; tempo; um período de tempo com características distintas na história | uma década de um século; período de dez anos}
  \end{Phonetics}
\end{Entry}

\begin{Entry}{年级}{6,6}{⼲,⽷}
  \begin{Phonetics}{年级}{nian2ji2}[][HSK 2]
    \definition[个]{s.}{série; ano; níveis divididos de acordo com o tempo de estudo dos alunos na escola}
  \end{Phonetics}
\end{Entry}

\begin{Entry}{年纪}{6,6}{⼲,⽷}
  \begin{Phonetics}{年纪}{nian2ji4}[][HSK 3]
    \definition[把,个]{s.}{idade (de uma pessoa)}
  \end{Phonetics}
\end{Entry}

\begin{Entry}{年迈}{6,6}{⼲,⾡}
  \begin{Phonetics}{年迈}{nian2mai4}[][HSK 7-9]
    \definition{adj.}{velho; idoso}
  \end{Phonetics}
\end{Entry}

\begin{Entry}{年初}{6,7}{⼲,⾐}
  \begin{Phonetics}{年初}{nian2chu1}[][HSK 3]
    \definition{s.}{o começo do ano; os primeiros dias do ano}
  \end{Phonetics}
\end{Entry}

\begin{Entry}{年夜饭}{6,8,7}{⼲,⼣,⾷}
  \begin{Phonetics}{年夜饭}{nian2ye4fan4}[][HSK 7-9]
    \definition[顿,桌]{s.}{jantar de família na véspera de Ano Novo; um jantar especial realizado na véspera de Ano Novo (na noite de 31 de dezembro)}
  \end{Phonetics}
\end{Entry}

\begin{Entry}{年底}{6,8}{⼲,⼴}
  \begin{Phonetics}{年底}{nian2di3}[][HSK 3]
    \definition[个]{s.}{fim de ano; o fim do ano; geralmente os últimos dias de dezembro ou o fim do ano}
  \end{Phonetics}
\end{Entry}

\begin{Entry}{年画}{6,8}{⼲,⽥}
  \begin{Phonetics}{年画}{nian2hua4}[][HSK 7-9]
    \definition{s.}{fotos de Ano Novo (ou Festival da Primavera); durante o Ano Novo Lunar, são afixadas imagens que retratam alegria e prosperidade}
  \seealsoref{年画儿}{nian2hua4r5}
  \end{Phonetics}
\end{Entry}

\begin{Entry}{年画儿}{6,8,2}{⼲,⽥,⼉}
  \begin{Phonetics}{年画儿}{nian2hua4r5}
    \definition{s.}{foto de Ano Novo (Festival da Primavera)}
  \end{Phonetics}
\end{Entry}

\begin{Entry}{年终}{6,8}{⼲,⽷}
  \begin{Phonetics}{年终}{nian2zhong1}[][HSK 7-9]
    \definition{s.}{fim de ano; fim do ano}
  \end{Phonetics}
\end{Entry}

\begin{Entry}{年货}{6,8}{⼲,⾙}
  \begin{Phonetics}{年货}{nian2huo4}
    \definition{s.}{mercadorias vendidas no Ano Novo Chinês}
  \end{Phonetics}
\end{Entry}

\begin{Entry}{年限}{6,8}{⼲,⾩}
  \begin{Phonetics}{年限}{nian2xian4}[][HSK 7-9]
    \definition{s.}{limite de idade; número fixo de anos; o número de anos especificado ou usado como padrão geral}
  \end{Phonetics}
\end{Entry}

\begin{Entry}{年前}{6,9}{⼲,⼑}
  \begin{Phonetics}{年前}{nian2qian2}[][HSK 5]
    \definition{s.}{(pouco) antes da virada do ano | antes do final do ano | antes do ano novo}
  \end{Phonetics}
\end{Entry}

\begin{Entry}{年度}{6,9}{⼲,⼴}
  \begin{Phonetics}{年度}{nian2du4}[][HSK 5]
    \definition{s.}{ano; de acordo com a natureza e as necessidades de um negócio, há um prazo de doze meses com data de início e término definidas}
  \end{Phonetics}
\end{Entry}

\begin{Entry}{年轻}{6,9}{⼲,⾞}
  \begin{Phonetics}{年轻}{nian2qing1}[][HSK 2]
    \definition{adj.}{jovem; não muito velho (geralmente se refere a pessoas entre 10 e 20 anos)}
  \end{Phonetics}
\end{Entry}

\begin{Entry}{年龄}{6,13}{⼲,⿒}
  \begin{Phonetics}{年龄}{nian2ling2}[][HSK 5]
    \definition[个,段]{s.}{idade; animais, plantas e outros seres vivos vivem e crescem no mundo durante um determinado número de anos}
  \end{Phonetics}
\end{Entry}

\begin{Entry}{年薪}{6,16}{⼲,⾋}
  \begin{Phonetics}{年薪}{nian2xin1}[][HSK 7-9]
    \definition{s.}{salário anual; remuneração anual}[他的年薪已经达到了五十万元。===Seu salário anual chegou a 500.000 yuans.]
  \end{Phonetics}
\end{Entry}

%%%%%%%%%% 并 %%%%%%%%%%
\subsection*{并}\addcontentsline{loh}{figure}{并}

\begin{Entry}{并}{6}{⼲}
  \begin{Phonetics}{并}{bing4}[][HSK 3,4]
    \definition{adv.}{lado a lado; igualmente; simultaneamente | (usado para reforçar uma negação) na verdade; definitivamente | mesmo assim | (usado para reforçar uma negação) na verdade; de forma alguma | todos; indica o conjunto completo, equivalente a 全部}
    \definition{conj.}{e; além disso}
    \definition{v.}{combinar; fundir; incorporar | ficar (ou colocar) lado a lado | estar paralelo a | anexar; juntar}
  \seealsoref{全部}{quan2bu4}
  \end{Phonetics}
\end{Entry}

\begin{Entry}{并且}{6,5}{⼲,⼀}
  \begin{Phonetics}{并且}{bing4qie3}[][HSK 3]
    \definition{conj.}{e; bem como; usado entre verbos, adjetivos ou frases paralelas para indicar que várias ações são realizadas ao mesmo tempo ou que propriedades existem ao mesmo tempo | além disso; além do mais; ademais; usado na segunda metade de uma frase complexa para expressar um significado adicional}
  \end{Phonetics}
\end{Entry}

\begin{Entry}{并列}{6,6}{⼲,⼑}
  \begin{Phonetics}{并列}{bing4lie4}[][HSK 7-9]
    \definition{v.}{ficar lado a lado; ser justaposto; ter a mesma importância; organizar lado a lado}
  \end{Phonetics}
\end{Entry}

\begin{Entry}{并行}{6,6}{⼲,⾏}
  \begin{Phonetics}{并行}{bing4xing2}[][HSK 7-9]
    \definition{adj.}{simultâneo | Computação: paralelo | lado a lado (de dois processos, desenvolvimentos, pensamentos etc.)}
    \definition{v.}{caminhar lado a lado; correr lado a lado | fazer duas coisas ao mesmo tempo | prosseguir em paralelo}
  \end{Phonetics}
\end{Entry}

\begin{Entry}{并购}{6,8}{⼲,⾙}
  \begin{Phonetics}{并购}{bing4gou4}[][HSK 7-9]
    \definition{s.}{aquisição; fusão e aquisição}
    \definition{v.}{fundir; adquirir | assumir}
  \end{Phonetics}
\end{Entry}

\begin{Entry}{并非}{6,8}{⼲,⾮}
  \begin{Phonetics}{并非}{bing4fei1}[][HSK 7-9]
    \definition{adv.}{realmente não é; na verdade}
  \end{Phonetics}
\end{Entry}

\begin{Entry}{并排}{6,11}{⼲,⼿}
  \begin{Phonetics}{并排}{bing4pai2}
    \definition{adv.}{lado a lado}
  \end{Phonetics}
\end{Entry}

%%%%%%%%%% 庆 %%%%%%%%%%
\subsection*{庆}\addcontentsline{loh}{figure}{庆}

\begin{Entry}{庆}{6}{⼴}
  \begin{Phonetics}{庆}{qing4}
    \definition*{s.}{Sobrenome: Qing}
    \definition{s.}{celebração | ocasião para celebração; um aniversário que vale a pena comemorar}
    \definition{v.}{celebrar; felicitar; comemorar}
  \end{Phonetics}
\end{Entry}

\begin{Entry}{庆典}{6,8}{⼴,⼋}
  \begin{Phonetics}{庆典}{qing4dian3}[][HSK 7-9]
    \definition{s.}{cerimônia; celebração; uma cerimônia de celebração muito grandiosa}
  \end{Phonetics}
\end{Entry}

\begin{Entry}{庆幸}{6,8}{⼴,⼲}
  \begin{Phonetics}{庆幸}{qing4xing4}[][HSK 7-9]
    \definition{v.}{alegrar-se; ficar contente; ficar feliz por uma situação inesperadamente boa}
  \end{Phonetics}
\end{Entry}

\begin{Entry}{庆祝}{6,9}{⼴,⽰}
  \begin{Phonetics}{庆祝}{qing4zhu4}[][HSK 3]
    \definition{v.}{celebrar; comemorar; festejar; realizar atividades para comemorar ou celebrar festivais comuns e eventos felizes}
  \end{Phonetics}
\end{Entry}

\begin{Entry}{庆贺}{6,9}{⼴,⾙}
  \begin{Phonetics}{庆贺}{qing4he4}[][HSK 7-9]
    \definition{v.}{parabenizar; celebrar; celebrar uma ocasião alegre compartilhada ou parabenizar alguém que está recebendo boas notícias}
  \end{Phonetics}
\end{Entry}

%%%%%%%%%% 延 %%%%%%%%%%
\subsection*{延}\addcontentsline{loh}{figure}{延}

\begin{Entry}{延}{6}{⼵}
  \begin{Phonetics}{延}{yan2}
    \definition*{s.}{Sobrenome: Yan}
    \definition{v.}{prolongar; estender; alongar | adiar; atrasar | envolver (um professor, conselheiro, etc.); enviar para; convidar}
  \end{Phonetics}
\end{Entry}

\begin{Entry}{延长}{6,4}{⼵,⾧}
  \begin{Phonetics}{延长}{yan2chang2}[][HSK 4]
    \definition{v.}{estender; prolongar; alongar; aumentar o tempo, a distância ou a duração de algo específico}
  \end{Phonetics}
\end{Entry}

\begin{Entry}{延伸}{6,7}{⼵,⼈}
  \begin{Phonetics}{延伸}{yan2shen1}[][HSK 5]
    \definition{v.}{estender; esticar; alongar; estender-se}
  \end{Phonetics}
\end{Entry}

\begin{Entry}{延续}{6,11}{⼵,⽷}
  \begin{Phonetics}{延续}{yan2xu4}[][HSK 4]
    \definition{v.}{durar; continuar; prosseguir; continuar como antes; prolongar}
  \end{Phonetics}
\end{Entry}

\begin{Entry}{延期}{6,12}{⼵,⽉}
  \begin{Phonetics}{延期}{yan2/qi1}[][HSK 4]
    \definition{v.+compl.}{atrasar; adiar; postergar}
  \end{Phonetics}
\end{Entry}

%%%%%%%%%% 异 %%%%%%%%%%
\subsection*{异}\addcontentsline{loh}{figure}{异}

\begin{Entry}{异}{6}{⼶}
  \begin{Phonetics}{异}{yi4}
    \definition{adj.}{diferente | estranho; incomum; extraordinário; especial | outro}
    \definition{v.}{surpreender | separar; divorciar-se}
  \end{Phonetics}
\end{Entry}

\begin{Entry}{异常}{6,11}{⼶,⼱}
  \begin{Phonetics}{异常}{yi4chang2}[][HSK 6]
    \definition{adj.}{incomum; anormal; descreve uma situação diferente do normal}
    \definition{adv.}{extremamente; particularmente; excepcionalmente; descreve uma situação que atingiu um nível extremamente alto}
  \end{Phonetics}
\end{Entry}

%%%%%%%%%% 式 %%%%%%%%%%
\subsection*{式}\addcontentsline{loh}{figure}{式}

\begin{Entry}{式}{6}{⼷}
  \begin{Phonetics}{式}{shi4}[][HSK 5]
    \definition*{s.}{Sobrenome: Shi}
    \definition{s.}{tipo; estilo | forma; padrão | ritual; cerimônia | fórmula; conjunto de símbolos que expressam uma lei natural na ciência natural | humor; modo; categoria gramatical que expressa a atitude subjetiva do falante em relação ao que está sendo dito, como narrativa, imperativa e condicional}
  \end{Phonetics}
\end{Entry}

%%%%%%%%%% 当 %%%%%%%%%%
\subsection*{当}\addcontentsline{loh}{figure}{当}

\begin{Entry}{当}{6}{⼹}
  \begin{Phonetics}{当}{dang1}[][HSK 2]
    \definition*{s.}{Sobrenome: Dang}
    \definition{adj.}{igual; adequado; compatível}
    \definition{prep.}{na presença de alguém; na cara de alguém | exatamente em (um momento ou lugar); em algum momento, em algum lugar | na frente de alguém}
    \definition{s.}{Onomatopéia: barulho metálico, som de um gongo ou sino}
    \definition{s.}{topo; cume | uma lacuna no espaço ou no tempo; refere"-se a um espaço ou intervalo de tempo}
    \definition{v.}{dever; ter que; dever ser | trabalhar como; servir como; ser; assumir; desempenhar a função de | suportar; aceitar; merecer | dirigir; gerenciar; estar no comando; ser responsável por; presidir | conter; bloquear; segurar; reter; resistir}
  \end{Phonetics}
  \begin{Phonetics}{当}{dang4}[][HSK 6]
    \definition{adj.}{adequado; correto; apropriado | igual; o mesmo}
    \definition{pron.}{naquele mesmo (dia, etc.); refere"-se ao momento em que algo aconteceu}
    \definition{s.}{algo penhorado; penhor; garantia; objetos físicos penhorados em casas de penhores}
    \definition{v.}{corresponder; ser igual a; combinar | tratar como; considerar como; tomar como | pensar que; achar que | penhorar; empréstimo com garantia real em uma loja de penhores}
  \end{Phonetics}
\end{Entry}

\begin{Entry}{当下}{6,3}{⼹,⼀}
  \begin{Phonetics}{当下}{dang1xia4}[][HSK 7-9]
    \definition{adv.}{instantaneamente; imediatamente; de uma vez}
    \definition{s.}{o tempo presente}
  \end{Phonetics}
\end{Entry}

\begin{Entry}{当之无愧}{6,3,4,12}{⼹,⼂,⽆,⼼}
  \begin{Phonetics}{当之无愧}{dang1zhi1wu2kui4}[][HSK 7-9]
    \definition{expr.}{merecer plenamente (um título, uma honra, etc.); merecer a recompensa; ser merecedor | ser digno de; ser digno do nome}
  \end{Phonetics}
\end{Entry}

\begin{Entry}{当中}{6,4}{⼹,⼁}
  \begin{Phonetics}{当中}{dang1zhong1}[][HSK 3]
    \definition{prep.}{no meio; no centro | entre; dentro}
  \end{Phonetics}
\end{Entry}

\begin{Entry}{当今}{6,4}{⼹,⼈}
  \begin{Phonetics}{当今}{dang1jin1}[][HSK 7-9]
    \definition{s.}{o presente; hoje | Arcaico: imperador no trono; imperador reinante | agora; no presente; hoje em dia}
  \end{Phonetics}
\end{Entry}

\begin{Entry}{当天}{6,4}{⼹,⼤}
  \begin{Phonetics}{当天}{dang1tian1}[][HSK 6]
    \definition{s.}{no mesmo dia; naquele mesmo dia; refere"-se ao dia em que algo aconteceu no passado}
  \end{Phonetics}
\end{Entry}

\begin{Entry}{当心}{6,4}{⼹,⼼}
  \begin{Phonetics}{当心}{dang1xin1}[][HSK 7-9]
    \definition{s.}{centro; o centro do peito}
    \definition{v.}{ter cuidado com; ter cuidado}
  \end{Phonetics}
\end{Entry}

\begin{Entry}{当日}{6,4}{⼹,⽇}
  \begin{Phonetics}{当日}{dang1ri4}[][HSK 7-9]
    \definition[点]{s.}{nessa ocasião; naquela época; no mesmo dia; naquele mesmo dia}
  \end{Phonetics}
  \begin{Phonetics}{当日}{dang4ri4}
    \definition[点]{s.}{mesmo dia; naquele mesmo dia; refere"-se ao mesmo dia em que algo aconteceu; (neste) dia}
  \end{Phonetics}
\end{Entry}

\begin{Entry}{当代}{6,5}{⼹,⼈}
  \begin{Phonetics}{当代}{dang1dai4}[][HSK 5]
    \definition{s.}{a era atual; a era contemporânea}
  \end{Phonetics}
\end{Entry}

\begin{Entry}{当务之急}{6,5,3,9}{⼹,⼒,⼂,⼼}
  \begin{Phonetics}{当务之急}{dang1wu4zhi1ji2}[][HSK 7-9]
    \definition{expr.}{assunto mais urgente do momento; uma tarefa de alta prioridade; assunto urgente | assunto de vital importância; preocupações | trabalho de alta prioridade}
  \end{Phonetics}
\end{Entry}

\begin{Entry}{当众}{6,6}{⼹,⼈}
  \begin{Phonetics}{当众}{dang1zhong4}[][HSK 7-9]
    \definition{adv.}{abertamente; publicamente; em público; diante do público; na presença de todos; na frente de todos; de frente para a multidão}
  \end{Phonetics}
\end{Entry}

\begin{Entry}{当地}{6,6}{⼹,⼟}
  \begin{Phonetics}{当地}{dang1di4}[][HSK 3]
    \definition{s.}{local; o lugar onde as pessoas e as coisas estão ou onde as coisas acontecem}
  \end{Phonetics}
\end{Entry}

\begin{Entry}{当场}{6,6}{⼹,⼟}
  \begin{Phonetics}{当场}{dang1chang3}[][HSK 5]
    \definition{adv.}{na hora; de imediato; na mesma hora}
  \end{Phonetics}
\end{Entry}

\begin{Entry}{当年}{6,6}{⼹,⼲}
  \begin{Phonetics}{当年}{dang1nian2}[][HSK 5]
    \definition{s.}{aqueles anos (ou dias) | naqueles anos (ou dias) | durante esse tempo}
    \definition{v.}{estar no auge da vida}
  \end{Phonetics}
  \begin{Phonetics}{当年}{dang4nian2}
    \definition{s.}{no mesmo ano; naquele mesmo ano}
  \end{Phonetics}
\end{Entry}

\begin{Entry}{当成}{6,6}{⼹,⼽}
  \begin{Phonetics}{当成}{dang4cheng2}[][HSK 6]
    \definition{v.}{considerar como; tratar como; tomar por}
  \end{Phonetics}
\end{Entry}

\begin{Entry}{当作}{6,7}{⼹,⼈}
  \begin{Phonetics}{当作}{dang4zuo4}[][HSK 6]
    \definition{v.}{tratar como; considerar como}
  \end{Phonetics}
\end{Entry}

\begin{Entry}{当初}{6,7}{⼹,⾐}
  \begin{Phonetics}{当初}{dang1chu1}[][HSK 3]
    \definition{s.}{no começo; originalmente; no início; em primeiro lugar; refere"-se a algo que aconteceu no passado, seja em geral ou especificamente}
  \end{Phonetics}
\end{Entry}

\begin{Entry}{当即}{6,7}{⼹,⼙}
  \begin{Phonetics}{当即}{dang1ji2}[][HSK 7-9]
    \definition{adv.}{imediatamente}
  \end{Phonetics}
\end{Entry}

\begin{Entry}{当时}{6,7}{⼹,⽇}
  \begin{Phonetics}{当时}{dang1shi2}[][HSK 2]
    \definition{s.}{naquela época; aquela ocasião; aquela vez; refere"-se a algo que aconteceu no passado}
    \definition{v.}{ser o momento adequado; acontecer no momento certo}
  \end{Phonetics}
  \begin{Phonetics}{当时}{dang4shi2}
    \definition{adv.}{(depois de fazer algo ou algo acontecer) imediatamente; de imediato; agora mesmo}
  \end{Phonetics}
\end{Entry}

\begin{Entry}{当事人}{6,8,2}{⼹,⼅,⼈}
  \begin{Phonetics}{当事人}{dang1shi4ren2}[][HSK 7-9]
    \definition{s.}{litigante; parte (em um processo judicial); refere"-se especificamente a pessoas que têm uma relação direta com os fatos do caso, como a vítima, o promotor particular, o réu, etc. em um processo criminal | partes interessadas; pessoa (ou parte) envolvida; alguém que tem uma relação direta com algo}
  \end{Phonetics}
\end{Entry}

\begin{Entry}{当前}{6,9}{⼹,⼑}
  \begin{Phonetics}{当前}{dang1qian2}[][HSK 5]
    \definition{s.}{presente; atual}
    \definition{v.}{estar diante de alguém; estar frente a frente com alguém; na frente de, geralmente refere"-se a uma situação perigosa}
  \end{Phonetics}
\end{Entry}

\begin{Entry}{当选}{6,9}{⼹,⾡}
  \begin{Phonetics}{当选}{dang1xuan3}[][HSK 5]
    \definition{v.}{ser eleito}
  \end{Phonetics}
\end{Entry}

\begin{Entry}{当面}{6,9}{⼹,⾯}
  \begin{Phonetics}{当面}{dang1mian4}[][HSK 7-9]
    \definition{adv.}{na cara de alguém; na presença de alguém; cara a cara}
  \end{Phonetics}
\end{Entry}

\begin{Entry}{当真}{6,10}{⼹,⼗}
  \begin{Phonetics}{当真}{dang4zhen1}[][HSK 7-9]
    \definition{adj.}{verdadeiro; real; confiável}
    \definition{adv.}{realmente; verdadeiramente}
    \definition{v.}{levar a sério; acreditar}
  \end{Phonetics}
\end{Entry}

\begin{Entry}{当晚}{6,11}{⼹,⽇}
  \begin{Phonetics}{当晚}{dang1wan3}
    \definition{s.}{naquela noite; esta noite; a mesma noite}
  \end{Phonetics}
  \begin{Phonetics}{当晚}{dang4wan3}[][HSK 7-9]
    \definition{s.}{na mesma noite; esta noite}
  \end{Phonetics}
\end{Entry}

\begin{Entry}{当着}{6,11}{⼹,⽬}
  \begin{Phonetics}{当着}{dang1zhe5}[][HSK 7-9]
    \definition{prep.}{na frente de | na presença de}
  \end{Phonetics}
\end{Entry}

\begin{Entry}{当然}{6,12}{⼹,⽕}
  \begin{Phonetics}{当然}{dang1ran2}[][HSK 3]
    \definition{adj.}{natural; verdadeiro; espontâneo}
    \definition{adv.}{sem dúvida; certamente; claro}
  \end{Phonetics}
\end{Entry}

%%%%%%%%%% 忙 %%%%%%%%%%
\subsection*{忙}\addcontentsline{loh}{figure}{忙}

\begin{Entry}{忙}{6}{⼼}
  \begin{Phonetics}{忙}{mang2}[][HSK 1]
    \definition*{s.}{Sobrenome: Mang}
    \definition{adj.}{ocupado; movimentado; totalmente ocupado; muitas coisas para fazer, sem tempo livre | imperativo; ansioso; urgente}
    \definition{v.}{apressar-se; agitar-se; fazer algo com urgência e constantemente | trabalhar; fazer}
  \antonymref{闲}{xian2}
  \end{Phonetics}
\end{Entry}

\begin{Entry}{忙乱}{6,7}{⼼,⼄}
  \begin{Phonetics}{忙乱}{mang2luan4}[][HSK 7-9]
    \definition{adj.}{apressado e desordenado; às pressas e em meio à confusão}
    \definition{v.}{estar com pressa e desorganizado; realizar uma tarefa de forma apressada e desordenada; agir de forma apressada e desordenada}
  \end{Phonetics}
\end{Entry}

\begin{Entry}{忙活}{6,9}{⼼,⽔}
  \begin{Phonetics}{忙活}{mang2huo2}
    \definition{s.}{tarefa urgente}
    \definition{v.}{estar ocupado com o trabalho; estar envolvido com tarefas; estar com pressa para terminar as coisas}
  \end{Phonetics}
  \begin{Phonetics}{忙活}{mang2huo5}[][HSK 7-9]
    \definition{adj.}{Coloquial: ocupado; movimentado}
    \definition{v.}{estar ocupado; movimentar"-se freneticamente}
  \end{Phonetics}
\end{Entry}

\begin{Entry}{忙得}{6,11}{⼼,⼻}
  \begin{Phonetics}{忙得}{mang2de2}
    \definition{adj.}{muito ocupado}
  \end{Phonetics}
\end{Entry}

\begin{Entry}{忙碌}{6,13}{⼼,⽯}
  \begin{Phonetics}{忙碌}{mang2lu4}[][HSK 7-9]
    \definition{adj.}{ocupado; atarefado; agitado; movimentado; ocupado com várias coisas, sem tempo livre}
  \end{Phonetics}
\end{Entry}

%%%%%%%%%% 戏 %%%%%%%%%%
\subsection*{戏}\addcontentsline{loh}{figure}{戏}

\begin{Entry}{戏}{6}{⼽}
  \begin{Phonetics}{戏}{xi4}[][HSK 5]
    \definition*{s.}{Sobrenome: Xi}
    \definition[场,部,出,台]{s.}{drama; peça; espetáculo; \emph{show}}
    \definition{v.}{brincar; praticar esportes; jogar | zombar; brincar; provocar}
  \end{Phonetics}
\end{Entry}

\begin{Entry}{戏曲}{6,6}{⼽,⽈}
  \begin{Phonetics}{戏曲}{xi4qu3}[][HSK 6]
    \definition{s.}{drama; ópera chinesa; ópera tradicional; forma teatral tradicional | partes cantadas em 传奇 e zaju 杂剧}
  \seealsoref{传奇}{chuan2qi2}
  \seealsoref{杂剧}{za2ju4}
  \end{Phonetics}
\end{Entry}

\begin{Entry}{戏弄}{6,7}{⼽,⼶}
  \begin{Phonetics}{戏弄}{xi4nong4}
    \definition{v.}{zombar de | pregar peças | provocar}
  \end{Phonetics}
\end{Entry}

\begin{Entry}{戏法}{6,8}{⼽,⽔}
  \begin{Phonetics}{戏法}{xi4fa3}
    \definition{s.}{truque de mágica | prestidigitação}
  \end{Phonetics}
\end{Entry}

\begin{Entry}{戏耍}{6,9}{⼽,⽽}
  \begin{Phonetics}{戏耍}{xi4shua3}
    \definition{v.}{divertir-me | brincar com | provocar}
  \end{Phonetics}
\end{Entry}

\begin{Entry}{戏院}{6,9}{⼽,⾩}
  \begin{Phonetics}{戏院}{xi4yuan4}
    \definition{s.}{teatro}
  \end{Phonetics}
\end{Entry}

\begin{Entry}{戏剧}{6,10}{⼽,⼑}
  \begin{Phonetics}{戏剧}{xi4ju4}[][HSK 5]
    \definition[出,部]{s.}{drama; peça; teatro | roteiro; peça; cenário}
  \end{Phonetics}
\end{Entry}

\begin{Entry}{戏剧化地}{6,10,4,6}{⼽,⼑,⼔,⼟}
  \begin{Phonetics}{戏剧化地}{xi4ju4hua4di4}
    \definition{adv.}{dramaticamente | teatralmente}
  \end{Phonetics}
\end{Entry}

\begin{Entry}{戏剧性}{6,10,8}{⼽,⼑,⼼}
  \begin{Phonetics}{戏剧性}{xi4ju4xing4}
    \definition{adj.}{dramático}
  \end{Phonetics}
\end{Entry}

\begin{Entry}{戏剧家}{6,10,10}{⼽,⼑,⼧}
  \begin{Phonetics}{戏剧家}{xi4ju4jia1}
    \definition{s.}{dramaturgo}
  \end{Phonetics}
\end{Entry}

\begin{Entry}{戏剧效果}{6,10,10,8}{⼽,⼑,⽁,⽊}
  \begin{Phonetics}{戏剧效果}{xi4ju4xiao4guo3}
    \definition{s.}{efeito dramático}
  \end{Phonetics}
\end{Entry}

\begin{Entry}{戏剧般}{6,10,10}{⼽,⼑,⾈}
  \begin{Phonetics}{戏剧般}{xi4ju4ban1}
    \definition{adj.}{melodramático}
  \end{Phonetics}
\end{Entry}

\begin{Entry}{戏剧编剧}{6,10,12,10}{⼽,⼑,⽷,⼑}
  \begin{Phonetics}{戏剧编剧}{xi4ju4bian1ju4}
    \definition{s.}{dramaturgo}
  \end{Phonetics}
\end{Entry}

\begin{Entry}{戏剧演出}{6,10,14,5}{⼽,⼑,⽔,⼐}
  \begin{Phonetics}{戏剧演出}{xi4ju4yan3chu1}
    \definition{s.}{performance dramática}
  \end{Phonetics}
\end{Entry}

\begin{Entry}{戏谑}{6,11}{⼽,⾔}
  \begin{Phonetics}{戏谑}{xi4xue4}
    \definition{v.}{brincar | fazer piadas | ridicularizar}
  \end{Phonetics}
\end{Entry}

%%%%%%%%%% 成 %%%%%%%%%%
\subsection*{成}\addcontentsline{loh}{figure}{成}

\begin{Entry}{成}{6}{⼽}
  \begin{Phonetics}{成}{cheng2}[][HSK 2,6]
    \definition*{s.}{Sobrenome: Cheng}
    \definition{adj.}{capaz; competente | totalmente crescido; totalmente desenvolvido; maduro | estabelecido; Já definido; pronto para uso | em números ou quantidades consideráveis; inteiro; suficiente: enfatiza a quantidade ou a duração}
    \definition{clas.}{um décimo}
    \definition{interj.}{``O.K.!''; ``Tudo bem!''}
    \definition{s.}{resultado; conquista}
    \definition{v.}{ter sucesso; conseguir; ser bem"-sucedido | tornar"-se; transformar"-se | ajudar a completar; realizar}
  \end{Phonetics}
\end{Entry}

\begin{Entry}{成人}{6,2}{⼽,⼈}
  \begin{Phonetics}{成人}{cheng2ren2}[][HSK 4]
    \definition[个,名,位]{s.}{adulto; crescido; pessoa adulta}
    \definition{v.}{crescer; tornar"-se adulto}
  \end{Phonetics}
\end{Entry}

\begin{Entry}{成千上万}{6,3,3,3}{⼽,⼗,⼀,⼀}
  \begin{Phonetics}{成千上万}{cheng2qian1-shang4wan4}[][HSK 7-9]
    \definition{expr.}{aos milhares e dezenas de milhares; números incontáveis; descreve um grande número, também escrito como 成千成万 ou 成千累万 | milhares e dezenas de milhares; milhares e milhares}
  \seealsoref{成千成万}{cheng2qian1-cheng2wan4}
  \seealsoref{成千累万}{cheng2qian1-lei3wan4}
  \end{Phonetics}
\end{Entry}

\begin{Entry}{成千成万}{6,3,6,3}{⼽,⼗,⼽,⼀}
  \begin{Phonetics}{成千成万}{cheng2qian1-cheng2wan4}
    \definition{expr.}{milhares e dezenas de milhares; milhares e milhares | inumeráveis | Literário: aos milhares e dezenas de milhares; números incontáveis}
  \seealsoref{成千累万}{cheng2qian1-lei3wan4}
  \seealsoref{成千上万}{cheng2qian1-shang4wan4}
  \end{Phonetics}
\end{Entry}

\begin{Entry}{成千累万}{6,3,11,3}{⼽,⼗,⽷,⼀}
  \begin{Phonetics}{成千累万}{cheng2qian1-lei3wan4}
    \definition{expr.}{milhares e dezenas de milhares | milhares e milhares; inumeráveis | Literário: aos milhares e dezenas de milhares; números incontáveis}
  \seealsoref{成千成万}{cheng2qian1-cheng2wan4}
  \seealsoref{成千上万}{cheng2qian1-shang4wan4}
  \end{Phonetics}
\end{Entry}

\begin{Entry}{成才}{6,3}{⼽,⼿}
  \begin{Phonetics}{成才}{cheng2cai2}[][HSK 7-9]
    \definition{v.}{tornar"-se uma pessoa útil | tornar"-se uma pessoa digna de respeito | fazer algo de si mesmo}
  \end{Phonetics}
\end{Entry}

\begin{Entry}{成为}{6,4}{⼽,⼂}
  \begin{Phonetics}{成为}{cheng2wei2}[][HSK 2]
    \definition{v.}{tornar"-se; transformar"-se; revelar"-se; passar de uma situação, identidade ou estado para outro}
  \end{Phonetics}
\end{Entry}

\begin{Entry}{成分}{6,4}{⼽,⼑}
  \begin{Phonetics}{成分}{cheng2fen5}[][HSK 6]
    \definition[个,些,种]{s.}{composição; ingrediente; elemento; parte componente; as várias substâncias ou fatores que compõem as coisas | a condição de classe de alguém; a profissão ou a condição econômica de alguém; refere"-se à classe à qual uma família pertence; à principal experiência ou ocupação anterior de uma pessoa}
  \end{Phonetics}
\end{Entry}

\begin{Entry}{成天}{6,4}{⼽,⼤}
  \begin{Phonetics}{成天}{cheng2tian1}[][HSK 7-9]
    \definition{adv.}{o dia todo; o tempo todo}
  \end{Phonetics}
\end{Entry}

\begin{Entry}{成长}{6,4}{⼽,⾧}
  \begin{Phonetics}{成长}{cheng2zhang3}[][HSK 3]
    \definition{v.}{crescer; amadurecer; tornar"-se adulto; o desenvolvimento de seres humanos, animais ou plantas desde a infância até a maturidade}
  \end{Phonetics}
\end{Entry}

\begin{Entry}{成功}{6,5}{⼽,⼒}
  \begin{Phonetics}{成功}{cheng2gong1}[][HSK 3]
    \definition{adj.}{bem-sucedido; frutífero}
    \definition[个,次]{s.}{sucesso}
    \definition{v.}{ter sucesso; obter os resultados esperados}
  \end{Phonetics}
\end{Entry}

\begin{Entry}{成功率}{6,5,11}{⼽,⼒,⽞}
  \begin{Phonetics}{成功率}{cheng2gong1lv4}
    \definition{s.}{taxa de sucesso}
  \end{Phonetics}
\end{Entry}

\begin{Entry}{成本}{6,5}{⼽,⽊}
  \begin{Phonetics}{成本}{cheng2ben3}[][HSK 5]
    \definition{s.}{custo principal; custo; custo capitalizado; custo final; primeiro custo; custo próprio; custo de produção de um produto; inclui o custo dos materiais de produção consumidos durante o processo produtivo e a remuneração paga aos trabalhadores}
  \end{Phonetics}
\end{Entry}

\begin{Entry}{成立}{6,5}{⼽,⽴}
  \begin{Phonetics}{成立}{cheng2li4}[][HSK 3]
    \definition{v.}{fundar; estabelecer; criar; (organizações, instituições, etc.) começar a existir e a funcionar | ser válido; ser sustentável; fazer sentido; (teorias, pontos de vista, razões, etc.) fundamentados e válidos}
  \end{Phonetics}
\end{Entry}

\begin{Entry}{成交}{6,6}{⼽,⼇}
  \begin{Phonetics}{成交}{cheng2/jiao1}[][HSK 5]
    \definition{v.+compl.}{fechar um acordo; fazer uma barganha; concluir uma transação}
  \end{Phonetics}
\end{Entry}

\begin{Entry}{成吉思汗}{6,6,9,6}{⼽,⼝,⼼,⽔}
  \begin{Phonetics}{成吉思汗}{cheng2ji2si1han2}
    \definition*{s.}{Genghis Khan (1162-1227), fundador e governante do Império Mongol}
  \end{Phonetics}
\end{Entry}

\begin{Entry}{成年}{6,6}{⼽,⼲}
  \begin{Phonetics}{成年}{cheng2nian2}[][HSK 7-9]
    \definition{adv.}{o ano todo; durante todo o ano}
    \definition{v.}{atingir a maioridade (ser humano, animal, madeira); refere"-se à idade em que uma pessoa atinge a maturidade, ou ao período em que animais superiores ou árvores atingem a maturidade}
  \end{Phonetics}
\end{Entry}

\begin{Entry}{成色}{6,6}{⼽,⾊}
  \begin{Phonetics}{成色}{cheng2se4}
    \definition{v.}{sair"-se bem | ser bem sucedido}
  \end{Phonetics}
\end{Entry}

\begin{Entry}{成问题}{6,6,15}{⼽,⾨,⾴}
  \begin{Phonetics}{成问题}{cheng2wen4ti2}[][HSK 7-9]
    \definition{v.}{Coloquial: ser um problema; estar aberto a questionamentos (ou dúvidas, objeções)}
  \end{Phonetics}
\end{Entry}

\begin{Entry}{成员}{6,7}{⼽,⼝}
  \begin{Phonetics}{成员}{cheng2yuan2}[][HSK 3]
    \definition[个,些,名,位]{s.}{membro; membros de um grupo ou família}
  \end{Phonetics}
\end{Entry}

\begin{Entry}{成批}{6,7}{⼽,⼿}
  \begin{Phonetics}{成批}{cheng2pi1}
    \definition{s.}{em lotes | a granel}
  \end{Phonetics}
\end{Entry}

\begin{Entry}{成果}{6,8}{⼽,⽊}
  \begin{Phonetics}{成果}{cheng2guo3}[][HSK 3]
    \definition[个]{s.}{realização; resultado; conquista; recompensas no trabalho ou na carreira}
  \end{Phonetics}
\end{Entry}

\begin{Entry}{成品}{6,9}{⼽,⼝}
  \begin{Phonetics}{成品}{cheng2pin3}[][HSK 6]
    \definition[批]{s.}{produto final; produto acabado; produto processado e pronto para ser fornecido}
  \end{Phonetics}
\end{Entry}

\begin{Entry}{成型}{6,9}{⼽,⼟}
  \begin{Phonetics}{成型}{cheng2xing2}[][HSK 7-9]
    \definition{v.}{(peças ou produtos) estar em forma acabada; assumir a forma necessária}
  \end{Phonetics}
\end{Entry}

\begin{Entry}{成活}{6,9}{⼽,⽔}
  \begin{Phonetics}{成活}{cheng2huo2}
    \definition{v.}{sobreviver}
  \end{Phonetics}
\end{Entry}

\begin{Entry}{成语}{6,9}{⼽,⾔}
  \begin{Phonetics}{成语}{cheng2yu3}[][HSK 5]
    \definition[条,则,句,个]{s.}{expressão idiomática; frase de conjunto (frases de quatro caracteres em chinês, geralmente com alusões literárias)}
  \end{Phonetics}
\end{Entry}

\begin{Entry}{成家}{6,10}{⼽,⼧}
  \begin{Phonetics}{成家}{cheng2/jia1}[][HSK 7-9]
    \definition{v.+compl.}{(um homem) casar; (um homem) estabelecer"-se e casar"-se | tornar"-se um especialista (ou \emph{expert}); tornar"-se um especialista reconhecido}
  \end{Phonetics}
\end{Entry}

\begin{Entry}{成效}{6,10}{⼽,⽁}
  \begin{Phonetics}{成效}{cheng2xiao4}[][HSK 5]
    \definition{s.}{efeito; resultado}
  \end{Phonetics}
\end{Entry}

\begin{Entry}{成都}{6,10}{⼽,⾢}
  \begin{Phonetics}{成都}{cheng2du1}
    \definition*{s.}{Chengdu}
  \end{Phonetics}
\end{Entry}

\begin{Entry}{成婚}{6,11}{⼽,⼥}
  \begin{Phonetics}{成婚}{cheng2hun1}
    \definition{v.}{casar"-se}
  \end{Phonetics}
\end{Entry}

\begin{Entry}{成绩}{6,11}{⼽,⽷}
  \begin{Phonetics}{成绩}{cheng2ji4}[][HSK 2]
    \definition[项,个]{s.}{realização; sucesso; resultado (de trabalho ou estudo); refere"-se à pontuação obtida em exames e competições; classificação, também se refere aos resultados alcançados no trabalho}
  \end{Phonetics}
\end{Entry}

\begin{Entry}{成就}{6,12}{⼽,⼪}
  \begin{Phonetics}{成就}{cheng2jiu4}[][HSK 3]
    \definition[个,项]{s.}{realização; conquista; sucesso; realizações profissionais}
    \definition{v.}{realizar; alcançar; completar; concluir (carreira)}
  \end{Phonetics}
\end{Entry}

\begin{Entry}{成群结队}{6,13,9,4}{⼽,⽺,⽷,⾩}
  \begin{Phonetics}{成群结队}{cheng2qun2-jie2dui4}[][HSK 7-9]
    \definition{expr.}{em multidões; como uma grande multidão; horda; formar um grupo, constituir uma trupe; em grande número}
  \end{Phonetics}
\end{Entry}

\begin{Entry}{成群结对}{6,13,9,5}{⼽,⽺,⽷,⼨}
  \begin{Phonetics}{成群结对}{cheng2qun2-jie2dui4}
    \definition{expr.}{em grupos}
  \end{Phonetics}
\end{Entry}

\begin{Entry}{成熟}{6,15}{⼽,⽕}
  \begin{Phonetics}{成熟}{cheng2shu2}[][HSK 3]
    \definition{adj.}{maduro; amadurecido; totalmente desenvolvido; descreve que as oportunidades, condições, etc. estão perfeitas e que não haverá nenhum problema}
    \definition{v.}{amadurecer; atingir a maturidade; estar totalmente desenvolvido; frutas e outros frutos totalmente maduros, referindo"-se ao desenvolvimento completo de organismos vivos}
  \end{Phonetics}
\end{Entry}

\begin{Entry}{成器}{6,16}{⼽,⼝}
  \begin{Phonetics}{成器}{cheng2qi4}
    \definition{v.}{tornar"-se uma pessoa digna de respeito | fazer algo de si mesmo}
  \end{Phonetics}
\end{Entry}

%%%%%%%%%% 托 %%%%%%%%%%
\subsection*{托}\addcontentsline{loh}{figure}{托}

\begin{Entry}{托}{6}{⼿}
  \begin{Phonetics}{托}{tuo1}[][HSK 6]
    \definition{clas.}{torr, uma unidade de pressão, 1 torr é igual à pressão de 1 mmHg, ou 133,322 Pa}
    \definition{s.}{algo servindo como suporte | fantoche; cúmplice; pessoas que ajudam golpistas a enganar outras pessoas}
    \definition{v.}{segurar na palma; apoiar com a mão ou palma; suportar (um objeto) com um objeto ou com a palma da mão | destacar; servir como contraste | pedir; confiar | implorar; dar como pretexto | dever a; confiar em}
  \end{Phonetics}
\end{Entry}

\begin{Entry}{托付}{6,5}{⼿,⼈}
  \begin{Phonetics}{托付}{tuo1fu4}[][HSK 7-9]
    \definition{v.}{confiar; entregar algo aos cuidados de alguém; pedir a alguém que cuide de você ou que faça algo por você}
  \synonymref{拜托}{bai4tuo1}
  \synonymref{吩咐}{fen1fu4}
  \synonymref{寄托}{ji4tuo1}
  \synonymref{交付}{jiao1fu4}
  \synonymref{委托}{wei3tuo1}
  \synonymref{嘱托}{zhu3tuo1}
  \end{Phonetics}
\end{Entry}

%%%%%%%%%% 扛 %%%%%%%%%%
\subsection*{扛}\addcontentsline{loh}{figure}{扛}

\begin{Entry}{扛}{6}{⼿}
  \begin{Phonetics}{扛}{gang1}
    \definition{v.}{levantar com as duas mãos | carregar alguma coisa juntos (duas ou mais pessoas)}
  \end{Phonetics}
  \begin{Phonetics}{扛}{kang2}[][HSK 7-9]
    \definition{v.}{carregar objetos nos ombros |  suportar; aguentar | lidar; assumir}
  \end{Phonetics}
\end{Entry}

%%%%%%%%%% 扣 %%%%%%%%%%
\subsection*{扣}\addcontentsline{loh}{figure}{扣}

\begin{Entry}{扣}{6}{⼿}
  \begin{Phonetics}{扣}{kou4}[][HSK 6]
    \definition*{s.}{Sobrenome: Kou}
    \definition{clas.}{giro; volta; uma volta de uma rosca}
    \definition[个,颗,粒]{s.}{nó | fivela; botão | círculo de rosca (em um parafuso)}
    \definition{v.}{fivela; abotoar; amarrar ou prender com um laço ou anel | colocar uma xícara, tigela etc. de cabeça para baixo; cobrir com uma xícara, tigela etc. invertida; colocar a boca do recipiente para baixo | deter; prender; levar sob custódia | cravar; esmagar (a bola); arremessar ou bater (em uma bola) com força de cima para baixo | atracar; deduzir; descontar; subtrair uma parte do valor original | puxar; pressionar | impor; marcar sem fundamento; acusar injustamente; impor ou atribuir (um crime ou má fama) a alguém}
  \end{Phonetics}
\end{Entry}

\begin{Entry}{扣人心弦}{6,2,4,8}{⼿,⼈,⼼,⼸}
  \begin{Phonetics}{扣人心弦}{kou4ren2xin1xian2}[][HSK 7-9]
    \definition{expr.}{emocionante; cativar alguém; conquistar o coração de; comovente; tocar os sentimentos de alguém; tocar o coração de alguém; tocar profundamente; muito tocante; descrever poesia, prosa, performances, etc., como tendo uma qualidade contagiante que desperta emoções; eletrizante; de tirar o fôlego}
  \end{Phonetics}
\end{Entry}

\begin{Entry}{扣押}{6,8}{⼿,⼿}
  \begin{Phonetics}{扣押}{kou4ya1}[][HSK 7-9]
    \definition{s.}{detenção}
    \definition{v.}{deter; apreender; manter sob custódia | apreender; confiscar; embargar | sequestrar (ou tomar, reter) alguém como refém}
  \end{Phonetics}
\end{Entry}

\begin{Entry}{扣除}{6,9}{⼿,⾩}
  \begin{Phonetics}{扣除}{kou4chu2}[][HSK 7-9]
    \definition{v.}{deduzir; tirar; subtrair do total}
  \end{Phonetics}
\end{Entry}

\begin{Entry}{扣留}{6,10}{⼿,⽥}
  \begin{Phonetics}{扣留}{kou4liu2}[][HSK 7-9]
    \definition{s.}{apreensão; detenção}
    \definition{v.}{deter; manter sob custódia; pôr em prisão domiciliar; prender}
  \end{Phonetics}
\end{Entry}

%%%%%%%%%% 执 %%%%%%%%%%
\subsection*{执}\addcontentsline{loh}{figure}{执}

\begin{Entry}{执}{6}{⼿}
  \begin{Phonetics}{执}{zhi2}
    \definition*{s.}{Sobrenome: Zhi}
    \definition[期]{s.}{reconhecimento por escrito | (literário) amigo íntimo (ou do peito)}
    \definition{v.}{segurar; agarrar; pegar; capturar | assumir o comando de; dirigir; gerenciar; controlar; administrar; exercer | manter (os próprios pontos de vista, etc.); persistir; persistir em; manter-se em; insistir em | realizar; executar; implementar}
  \end{Phonetics}
\end{Entry}

\begin{Entry}{执行}{6,6}{⼿,⾏}
  \begin{Phonetics}{执行}{zhi2xing2}[][HSK 5]
    \definition{v.}{executar; implementar; realizar}
  \end{Phonetics}
\end{Entry}

\begin{Entry}{执着}{6,11}{⼿,⽬}
  \begin{Phonetics}{执着}{zhi2zhuo2}
    \definition{s.}{(budismo) apego}
    \definition{v.}{estar fortemente apegado a | ser dedicado | apegar-se a}
  \end{Phonetics}
\end{Entry}

%%%%%%%%%% 扩 %%%%%%%%%%
\subsection*{扩}\addcontentsline{loh}{figure}{扩}

\begin{Entry}{扩}{6}{⼿}
  \begin{Phonetics}{扩}{kuo4}[][HSK 7-9]
    \definition{v.}{expandir; ampliar; estender; alargar}
  \end{Phonetics}
\end{Entry}

\begin{Entry}{扩大}{6,3}{⼿,⼤}
  \begin{Phonetics}{扩大}{kuo4da4}[][HSK 4]
    \definition{v.}{ampliar; expandir; estender; alargar}
  \end{Phonetics}
\end{Entry}

\begin{Entry}{扩张}{6,7}{⼿,⼸}
  \begin{Phonetics}{扩张}{kuo4zhang1}[][HSK 7-9]
    \definition{v.}{expandir; aumentar; estender; espalhar; engrandecer | dilatar (dilatação vascular)}
  \end{Phonetics}
\end{Entry}

\begin{Entry}{扩建}{6,8}{⼿,⼵}
  \begin{Phonetics}{扩建}{kuo4jian4}[][HSK 7-9]
    \definition{v.}{expandir; ampliar; ampliar a escala original do edifício ou a escala da área}
  \end{Phonetics}
\end{Entry}

\begin{Entry}{扩展}{6,10}{⼿,⼫}
  \begin{Phonetics}{扩展}{kuo4zhan3}[][HSK 4]
    \definition{v.}{esticar; expandir; estender; espalhar}
  \end{Phonetics}
\end{Entry}

\begin{Entry}{扩散}{6,12}{⼿,⽁}
  \begin{Phonetics}{扩散}{kuo4san4}[][HSK 7-9]
    \definition{v.}{espalhar; difundir; dispersar; proliferar}
  \end{Phonetics}
\end{Entry}

%%%%%%%%%% 扫 %%%%%%%%%%
\subsection*{扫}\addcontentsline{loh}{figure}{扫}

\begin{Entry}{扫}{6}{⼿}
  \begin{Phonetics}{扫}{sao3}[][HSK 4]
    \definition{v.}{varrer; limpar | passar rapidamente ao longo ou sobre; varrer | juntar tudo | Computação: scanear}
  \end{Phonetics}
  \begin{Phonetics}{扫}{sao4}
    \definition{s.}{elemento formadore de palavra}
  \seealsoref{扫帚}{sao4zhou5}
  \end{Phonetics}
\end{Entry}

\begin{Entry}{扫兴}{6,6}{⼿,⼋}
  \begin{Phonetics}{扫兴}{sao3/xing4}[][HSK 7-9]
    \definition{v.+compl.}{sentir-se desapontado; ter o ânimo abalado; quando você está se sentindo feliz, algo desagradável pode abalar seu ânimo}
  \end{Phonetics}
\end{Entry}

\begin{Entry}{扫帚}{6,8}{⼿,⼱}
  \begin{Phonetics}{扫帚}{sao4zhou5}
    \definition[把,个]{s.}{vassoura; ferramenta de varredura feita de varas de bambu, etc., maior que uma vassora}
  \end{Phonetics}
\end{Entry}

\begin{Entry}{扫除}{6,9}{⼿,⾩}
  \begin{Phonetics}{扫除}{sao3chu2}[][HSK 7-9]
    \definition{s.}{limpeza; arrumação}
    \definition{v.}{limpar; arrumar | limpar; remover; eliminar}
  \end{Phonetics}
\end{Entry}

\begin{Entry}{扫描}{6,11}{⼿,⼿}
  \begin{Phonetics}{扫描}{sao3miao2}[][HSK 7-9]
    \definition{v.}{digitalizar | dar uma olhada rápida; percorrer (o olhar, etc.) | utilizar software especializado para inspecionar e pesquisar (dados, vírus, etc. em computadores)}
  \end{Phonetics}
\end{Entry}

\begin{Entry}{扫墓}{6,13}{⼿,⼟}
  \begin{Phonetics}{扫墓}{sao3/mu4}[][HSK 7-9]
    \definition{v.+compl.}{limpar sepulturas e prestar homenagens aos mortos; também se refere à realização de atividades comemorativas nos túmulos dos mártires}
  \end{Phonetics}
\end{Entry}

%%%%%%%%%% 扬 %%%%%%%%%%
\subsection*{扬}\addcontentsline{loh}{figure}{扬}

\begin{Entry}{扬}{6}{⼿}
  \begin{Phonetics}{扬}{yang2}
    \definition*{s.}{Yangzhou, abreviação de 扬州 | Sobrenome: Yang}
    \definition{v.}{levantar | separar e espalhar; peneirar | espalhar; fazer conhecido}
  \seealsoref{扬州}{yang2zhou1}
  \end{Phonetics}
\end{Entry}

\begin{Entry}{扬州}{6,6}{⼿,⼮}
  \begin{Phonetics}{扬州}{yang2zhou1}
    \definition*{s.}{Yangzhou, uma cidade na província de Jiangsu}
  \end{Phonetics}
\end{Entry}

\begin{Entry}{扬雄}{6,12}{⼿,⾫}
  \begin{Phonetics}{扬雄}{yang2xiong2}
    \definition*{s.}{Yang Xiong (53 AC-18 DC), estudioso, poeta e lexicógrafo, autor do primeiro dicionário de dialeto chinês 方言}
  \seealsoref{方言}{fang1yan2}
  \end{Phonetics}
\end{Entry}

%%%%%%%%%% 收 %%%%%%%%%%
\subsection*{收}\addcontentsline{loh}{figure}{收}

\begin{Entry}{收}{6}{⽁}
  \begin{Phonetics}{收}{shou1}[][HSK 2]
    \definition{expr.}{aos cuidados de (usado na linha de endereço após o nome)}
    \definition{v.}{recolocar; juntar; reunir e juntar coisas espalhadas ou dispersas | recolher; cobrar | ganhar; obter (benefícios econômicos) | colher; recolher; colher ou cortar frutas, legumes, cereais maduros, etc. | aceitar; receber; acolher | controlar; restringir; restringir, controlar os sentimentos ou ações, para voltar ao estado normal | finalizar; parar; concluir; encerrar | prender; deter; colocar sob custódia}
  \end{Phonetics}
\end{Entry}

\begin{Entry}{收入}{6,2}{⽁,⼊}
  \begin{Phonetics}{收入}{shou1ru4}[][HSK 2]
    \definition[笔,个]{s.}{renda; salário; dinheiro recebido}
    \definition{v.}{receber dinheiro | coletar; receber}
  \end{Phonetics}
\end{Entry}

\begin{Entry}{收支}{6,4}{⽁,⽀}
  \begin{Phonetics}{收支}{shou1zhi1}[][HSK 7-9]
    \definition{s.}{receitas e despesas; rendimentos e despesas}
  \synonymref{出入}{chu1ru4}
  \synonymref{进出}{jin4chu1}
  \end{Phonetics}
\end{Entry}

\begin{Entry}{收买}{6,6}{⽁,⼄}
  \begin{Phonetics}{收买}{shou1mai3}[][HSK 7-9]
    \definition{v.}{comprar; adquirir | comprar; subornar; conquistar corações e mentes}
  \synonymref{拉拢}{la1long3}
  \synonymref{收购}{shou1gou4}
  \antonymref{出卖}{chu1mai4}
  \antonymref{出售}{chu1shou4}
  \end{Phonetics}
\end{Entry}

\begin{Entry}{收回}{6,6}{⽁,⼞}
  \begin{Phonetics}{收回}{shou1 hui2}[][HSK 4]
    \definition{v.}{retomar; recuperar; relembrar; recordar; receber de volta o que foi enviado ou emprestado, ou o dinheiro que foi emprestado ou usado | sacar; retirar; recolher; rescindir; cancelar (uma opinião, ordem, etc.)}
  \end{Phonetics}
\end{Entry}

\begin{Entry}{收听}{6,7}{⽁,⼝}
  \begin{Phonetics}{收听}{shou1ting1}[][HSK 3]
    \definition{v.}{ouvir (rádio)}
  \end{Phonetics}
\end{Entry}

\begin{Entry}{收到}{6,8}{⽁,⼑}
  \begin{Phonetics}{收到}{shou1 dao4}[][HSK 2]
    \definition{v.}{conseguir; obter; receber; alcançar}
  \end{Phonetics}
\end{Entry}

\begin{Entry}{收取}{6,8}{⽁,⼜}
  \begin{Phonetics}{收取}{shou1qu3}[][HSK 6]
    \definition{v.}{obter; coletar; receber; aceitar o dinheiro pago pela outra parte}
  \end{Phonetics}
\end{Entry}

\begin{Entry}{收视率}{6,8,11}{⽁,⾒,⽞}
  \begin{Phonetics}{收视率}{shou1shi4lv4}[][HSK 7-9]
    \definition{s.}{classificações (de um programa de TV); a audiência televisiva refere-se à porcentagem de pessoas (ou domicílios) que assistem a um determinado canal de televisão (ou programa de televisão) durante um período específico, em relação ao número total de telespectadores (ou domicílios)}
  \end{Phonetics}
\end{Entry}

\begin{Entry}{收购}{6,8}{⽁,⾙}
  \begin{Phonetics}{收购}{shou1gou4}[][HSK 5]
    \definition{v.}{comprar; adquirir; comprar muito em vários lugares | adquirir uma empresa; obter o controle efetivo de uma empresa por meio de dinheiro, transações de ações, etc.}
  \end{Phonetics}
\end{Entry}

\begin{Entry}{收养}{6,9}{⽁,⼋}
  \begin{Phonetics}{收养}{shou1yang3}[][HSK 6]
    \definition{v.}{acolher e criar; adotar; acolher os filhos dos outros e criá-los como se fossem da sua própria família}
  \end{Phonetics}
\end{Entry}

\begin{Entry}{收复}{6,9}{⽁,⼢}
  \begin{Phonetics}{收复}{shou1fu4}[][HSK 7-9]
    \definition{v.}{recuperar; recapturar | retomar; recuperar | recuperar o próprio território}
  \synonymref{复原}{fu4/yuan2}
  \synonymref{复兴}{fu4xing1}
  \synonymref{恢复}{hui1fu4}
  \end{Phonetics}
\end{Entry}

\begin{Entry}{收拾}{6,9}{⽁,⼿}
  \begin{Phonetics}{收拾}{shou1shi5}[][HSK 5]
    \definition{v.}{arrumar; empacotar; limpar; organizar, policiar, restaurar a normalidade em situações adversas | consertar; reparar; restaurar algo que está danificado ao seu estado ou função original |  punir; punir alguém, geralmente com medidas mais severas | matar}
  \end{Phonetics}
\end{Entry}

\begin{Entry}{收看}{6,9}{⽁,⽬}
  \begin{Phonetics}{收看}{shou1kan4}[][HSK 3]
    \definition{v.}{assistir (a um programa de TV)}
  \end{Phonetics}
\end{Entry}

\begin{Entry}{收费}{6,9}{⽁,⾙}
  \begin{Phonetics}{收费}{shou1 fei4}[][HSK 3]
    \definition{v.}{cobrar; cobrar taxas}
  \end{Phonetics}
\end{Entry}

\begin{Entry}{收音机}{6,9,6}{⽁,⾳,⽊}
  \begin{Phonetics}{收音机}{shou1yin1ji1}[][HSK 3]
    \definition[部,台]{s.}{rádio; sem fio; um termo geral para receptores de rádio}
  \end{Phonetics}
\end{Entry}

\begin{Entry}{收留}{6,10}{⽁,⽥}
  \begin{Phonetics}{收留}{shou1liu2}[][HSK 7-9]
    \definition{v.}{acolher alguém; ter alguém sob seus cuidados; oferecer abrigo}
  \synonymref{放走}{fang4zou3}
  \synonymref{收养}{shou1yang3}
  \antonymref{赶走}{gan3zou3}
  \antonymref{抛弃}{pao1qi4}
  \antonymref{驱逐}{qu1zhu2}
  \end{Phonetics}
\end{Entry}

\begin{Entry}{收益}{6,10}{⽁,⽫}
  \begin{Phonetics}{收益}{shou1yi4}[][HSK 4]
    \definition{s.}{lucro; renda; benefício; ganhos; vantagens ou benefícios obtidos}
  \end{Phonetics}
\end{Entry}

\begin{Entry}{收获}{6,10}{⽁,⾋}
  \begin{Phonetics}{收获}{shou1huo4}[][HSK 4]
    \definition[次,番,份]{s.}{resultados; ganhos; metaforicamente falando, conhecimento, experiência, etc. obtidos em estudo ou trabalho; os resultados obtidos por meio de trabalho árduo | colheita; colheita de safras}
    \definition{v.}{colher; juntar as colheitas}
  \end{Phonetics}
\end{Entry}

\begin{Entry}{收据}{6,11}{⽁,⼿}
  \begin{Phonetics}{收据}{shou1ju4}[][HSK 7-9]
    \definition[张]{s.}{recibo; quitação; uma declaração escrita entregue à outra parte como prova após o recebimento de dinheiro ou bens}
  \end{Phonetics}
\end{Entry}

\begin{Entry}{收敛}{6,11}{⽁,⽁}
  \begin{Phonetics}{收敛}{shou1lian3}[][HSK 7-9]
    \definition{v.}{enfraquecer; desaparecer; diminuir ou desaparecer (sorriso, luz, etc.) | conter"-se; restringir e controlar (fala e comportamento sem restrições) | adstringir; provocar a contração do corpo ou reduzir a secreção glandular}
  \synonymref{约束}{yue1shu4}
  \antonymref{放肆}{fang4si4}
  \antonymref{放纵}{fang4zong4}
  \antonymref{展开}{zhan3kai1}
  \end{Phonetics}
\end{Entry}

\begin{Entry}{收集}{6,12}{⽁,⾫}
  \begin{Phonetics}{收集}{shou1ji2}[][HSK 5]
    \definition{v.}{coletar; reunir; recolher}
  \end{Phonetics}
\end{Entry}

\begin{Entry}{收缩}{6,14}{⽁,⽷}
  \begin{Phonetics}{收缩}{shou1suo1}[][HSK 7-9]
    \definition{v.}{contrair; encolher; (objeto) mudar de grande para pequeno ou de comprido para curto | recuar; concentrar as forças; apertar; mudar de dispersão para concentração}
  \synonymref{关上}{guan1shang4}
  \synonymref{减弱}{jian3ruo4}
  \synonymref{减少}{jian3shao3}
  \synonymref{紧缩}{jin3suo1}
  \synonymref{缩短}{suo1/duan3}
  \synonymref{缩小}{suo1/xiao3}
  \synonymref{萎缩}{wei3suo1}
  \synonymref{中断}{zhong1duan4}
  \antonymref{打开}{da3 kai1}
  \antonymref{发展}{fa1zhan3}
  \antonymref{开展}{kai1zhan3}
  \antonymref{扩大}{kuo4da4}
  \antonymref{扩散}{kuo4san4}
  \antonymref{扩张}{kuo4zhang1}
  \antonymref{膨胀}{peng2zhang4}
  \antonymref{拓宽}{tuo4kuan1}
  \antonymref{展开}{zhan3kai1}
  \end{Phonetics}
\end{Entry}

\begin{Entry}{收藏}{6,17}{⽁,⾋}
  \begin{Phonetics}{收藏}{shou1cang2}[][HSK 6]
    \definition{v.}{coletar; armazenar; consagrar}
  \end{Phonetics}
\end{Entry}

%%%%%%%%%% 早 %%%%%%%%%%
\subsection*{早}\addcontentsline{loh}{figure}{早}

\begin{Entry}{早}{6}{⽇}
  \begin{Phonetics}{早}{zao3}[][HSK 1]
    \definition{adj.}{precoce; antes do previsto ou planejado; antes do tempo; antes de um determinado momento}
    \definition{adv.}{há muito tempo; desde cedo; por muito tempo; há muito tempo atrás}
    \definition{interj.}{``Bom dia!''; ``Saudações!''; usadas para cumprimentar uns aos outros ao se encontrarem pela manhã}
    \definition[个]{s.}{manhã}
  \end{Phonetics}
\end{Entry}

\begin{Entry}{早上}{6,3}{⽇,⼀}
  \begin{Phonetics}{早上}{zao3shang5}[][HSK 1]
    \definition[个]{s.}{de manhã cedo; madrugada; o período antes e depois do nascer do sol; geralmente, desde o amanhecer até às 8h ou 9h da manhã; às vezes também se refere ao período entre o amanhecer e o meio-dia}
  \end{Phonetics}
\end{Entry}

\begin{Entry}{早亡}{6,3}{⽇,⼇}
  \begin{Phonetics}{早亡}{zao3wang2}
    \definition[个]{s.}{morte prematura}
    \definition{v.}{morrer prematuramente}
  \end{Phonetics}
\end{Entry}

\begin{Entry}{早已}{6,3}{⽇,⼰}
  \begin{Phonetics}{早已}{zao3yi3}[][HSK 3]
    \definition{adv.}{há muito tempo; por muito tempo | (dialeto) no passado}
  \end{Phonetics}
\end{Entry}

\begin{Entry}{早车}{6,4}{⽇,⾞}
  \begin{Phonetics}{早车}{zao3che1}
    \definition{s.}{trem matutino | ônibus matutino}
  \end{Phonetics}
\end{Entry}

\begin{Entry}{早安}{6,6}{⽇,⼧}
  \begin{Phonetics}{早安}{zao3'an1}
    \definition{interj.}{``Bom dia!''}
  \end{Phonetics}
\end{Entry}

\begin{Entry}{早早儿}{6,6,2}{⽇,⽇,⼉}
  \begin{Phonetics}{早早儿}{zao3zao3r5}
    \definition{adv.}{o mais cedo possível | o mais breve possível}
  \end{Phonetics}
\end{Entry}

\begin{Entry}{早饭}{6,7}{⽇,⾷}
  \begin{Phonetics}{早饭}{zao3fan4}[][HSK 1]
    \definition[份,顿]{s.}{o café da manhã}
  \end{Phonetics}
\end{Entry}

\begin{Entry}{早知}{6,8}{⽇,⽮}
  \begin{Phonetics}{早知}{zao3zhi1}
    \definition{v.}{prever | se alguém soubesse antes, \dots}
  \end{Phonetics}
\end{Entry}

\begin{Entry}{早前}{6,9}{⽇,⼑}
  \begin{Phonetics}{早前}{zao3qian2}
    \definition{adv.}{previamente}
  \end{Phonetics}
\end{Entry}

\begin{Entry}{早晚}{6,11}{⽇,⽇}
  \begin{Phonetics}{早晚}{zao3wan3}[][HSK 6]
    \definition{adv./s.}{manhã e noite | mais cedo ou mais tarde; cedo ou tarde | algum tempo no futuro; algum dia; em algum momento no futuro}
  \end{Phonetics}
\end{Entry}

\begin{Entry}{早晨}{6,11}{⽇,⽇}
  \begin{Phonetics}{早晨}{zao3chen5}[][HSK 2]
    \definition[个,段,番]{s.}{manhã cedo; manhãzinha; o período do amanhecer às oito ou nove horas; às vezes, o período da meia-noite ao meio-dia}
  \end{Phonetics}
\end{Entry}

\begin{Entry}{早就}{6,12}{⽇,⼪}
  \begin{Phonetics}{早就}{zao3jiu4}[][HSK 2]
    \definition{adv.}{já; há muito tempo; há muito tempo atrás}
  \end{Phonetics}
\end{Entry}

\begin{Entry}{早期}{6,12}{⽇,⽉}
  \begin{Phonetics}{早期}{zao3qi1}[][HSK 5]
    \definition{s.}{prófase; estágio inicial; fase inicial; a fase inicial de uma determinada época, processo ou vida de uma pessoa}
  \end{Phonetics}
\end{Entry}

\begin{Entry}{早餐}{6,16}{⽇,⾷}
  \begin{Phonetics}{早餐}{zao3can1}[][HSK 2]
    \definition[份,桌,顿]{s.}{café da manhã; desejum}
  \end{Phonetics}
\end{Entry}

%%%%%%%%%% 曲 %%%%%%%%%%
\subsection*{曲}\addcontentsline{loh}{figure}{曲}

\begin{Entry}{曲}{6}{⽈}
  \begin{Phonetics}{曲}{qu1}
    \definition*{s.}{Sobrenome: Qu}
    \definition{adj.}{dobrado; curvo; sinuoso | errado; injustificável}
    \definition{s.}{curva (de um rio, etc.) | fermento; levedura}
    \definition{v.}{dobrar; torcer}
  \antonymref{直}{zhi2}
  \end{Phonetics}
  \begin{Phonetics}{曲}{qu3}[][HSK 7-9]
    \definition{s.}{canção; melodia; partitura}
  \end{Phonetics}
\end{Entry}

\begin{Entry}{曲折}{6,7}{⽈,⼿}
  \begin{Phonetics}{曲折}{qu1zhe2}[][HSK 7-9]
    \definition{adj.}{sinuoso; tortuoso; não reto | complicado; intrincado; a situação e o enredo são complexos}
    \definition{s.}{complicações; enredo complexo e frustrante}
  \end{Phonetics}
\end{Entry}

\begin{Entry}{曲线}{6,8}{⽈,⽷}
  \begin{Phonetics}{曲线}{qu1xian4}[][HSK 7-9]
    \definition[条]{s.}{curva; em geometria, refere"-se à trajetória de um ponto que se move sob certas condições em um plano ou no espaço | corpo curvado; linhas onduladas; também se refere às linhas do corpo humano}
  \end{Phonetics}
\end{Entry}

\begin{Entry}{曲棍球}{6,12,11}{⽈,⽊,⽟}
  \begin{Phonetics}{曲棍球}{qu1gun4qiu2}
    \definition{s.}{hóquei em campo; hóquei | bola de hóquei}
  \end{Phonetics}
\end{Entry}

%%%%%%%%%% 有 %%%%%%%%%%
\subsection*{有}\addcontentsline{loh}{figure}{有}

\begin{Entry}{有}{6}{⽉}
  \begin{Phonetics}{有}{you3}[][HSK 1]
    \definition*{s.}{Sobrenome: You}
    \definition{pref.}{usado antes do nome de certas dinastias ou etnias}
    \definition{v.}{ter; possuir; indica posse ou propriedade | existe; há; indica que certas coisas existem em certos lugares | fazer uma estimativa ou uma comparação; expressar estimativa ou comparação | indicar ação; indica que algo aconteceu ou ocorreu | usado antes de substantivos abstratos, indica quantidade ou grandeza | em termos gerais, semelhante a 某; refere"-se de maneira geral a algo semelhante | usado antes de pessoa, hora e lugar, indica a existência parcial | usado antes de certos verbos para formar uma expressão idiomática, indicando cortesia, polidez}
  \seealsoref{某}{mou3}
  \end{Phonetics}
\end{Entry}

\begin{Entry}{有(一)些}{6,1,8}{⽉,⼀,⼆}
  \begin{Phonetics}{有(一)些}{you3 (yi4) xie1}
    \definition{adv.}{em vez disso; em vez de; de certa forma}
    \definition{pron.}{de certa forma}
  \seealsoref{有些}{you3xie1}
  \end{Phonetics}
\end{Entry}

\begin{Entry}{有(一)点儿}{6,1,9,2}{⽉,⼀,⽕,⼉}
  \begin{Phonetics}{有(一)点儿}{you3 yi4 dian3r5}
    \definition{adv.}{um pouco (有点儿 + {s.} ou {v. mental})}
  \seealsoref{有点儿}{you3dian3r5}
  \end{Phonetics}
\end{Entry}

\begin{Entry}{有人}{6,2}{⽉,⼈}
  \begin{Phonetics}{有人}{you3ren2}[][HSK 2]
    \definition{adj.}{ocupado (como no banheiro)}
    \definition{pron.}{qualquer um; alguém}
    \definition[所]{s.}{pessoas}
    \definition{v.}{ter alguém ali}
  \end{Phonetics}
\end{Entry}

\begin{Entry}{有力}{6,2}{⽉,⼒}
  \begin{Phonetics}{有力}{you3li4}[][HSK 5]
    \definition{adj.}{forte; vigoroso; poderoso; energético}
  \end{Phonetics}
\end{Entry}

\begin{Entry}{有用}{6,5}{⽉,⽤}
  \begin{Phonetics}{有用}{you3yong4}[][HSK 1]
    \definition{adj.}{útil; prático; funcional}
  \end{Phonetics}
\end{Entry}

\begin{Entry}{有关}{6,6}{⽉,⼋}
  \begin{Phonetics}{有关}{you3guan1}[][HSK 6]
    \definition{prep.}{no caminho de; sobre}
    \definition{v.}{preocupar-se com; relacionar-se com; ter algo a ver com; existir algum tipo de relacionamento}
  \end{Phonetics}
\end{Entry}

\begin{Entry}{有名}{6,6}{⽉,⼝}
  \begin{Phonetics}{有名}{you3 ming2}[][HSK 1]
    \definition{adj.}{conhecido; famoso; célebre; nome conhecido por todos}
  \end{Phonetics}
\end{Entry}

\begin{Entry}{有名无实}{6,6,4,8}{⽉,⼝,⽆,⼧}
  \begin{Phonetics}{有名无实}{you3ming2wu2shi2}
    \definition{v.}{(literal) tem um nome, mas não tem realidade | existe apenas no nome}
  \end{Phonetics}
\end{Entry}

\begin{Entry}{有利}{6,7}{⽉,⼑}
  \begin{Phonetics}{有利}{you3li4}[][HSK 3]
    \definition{adj.}{benéfico; favorável; vantajoso}
  \end{Phonetics}
\end{Entry}

\begin{Entry}{有利于}{6,7,3}{⽉,⼑,⼆}
  \begin{Phonetics}{有利于}{you3li4yu2}[][HSK 5]
    \definition{prep.}{disponível; é benéfico para alguém ou alguma coisa e pode ajudar e promovê-lo}
  \end{Phonetics}
\end{Entry}

\begin{Entry}{有劲儿}{6,7,2}{⽉,⼒,⼉}
  \begin{Phonetics}{有劲儿}{you3jin4er5}[][HSK 4]
    \definition{adj.}{interessante; divertido; estimulante | energético}
    \definition{v.}{ter força}
  \end{Phonetics}
\end{Entry}

\begin{Entry}{有劳}{6,7}{⽉,⼒}
  \begin{Phonetics}{有劳}{you3lao2}
    \definition{v.}{posso incomodá-lo; desculpe incomodá-lo | (educado) obrigado pelo seu trabalho (usado ao pedir um favor ou após ter recebido um)}
  \end{Phonetics}
\end{Entry}

\begin{Entry}{有时}{6,7}{⽉,⽇}
  \begin{Phonetics}{有时}{you3shi2}
    \definition{expr.}{às vezes; ocasionalmente; de vez em quando}
  \seealsoref{有的时候}{you3de5shi2hou4}
  \seealsoref{有时候}{you3shi2hou5}
  \end{Phonetics}
\end{Entry}

\begin{Entry}{有时……有时……}{6,7,6,7}{⽉,⽇,⽉,⽇}
  \begin{Phonetics}{有时……有时……}{you3shi2 you3shi2}
    \definition{adv.}{às vezes\dots às vezes\dots}
  \end{Phonetics}
\end{Entry}

\begin{Entry}{有时候}{6,7,10}{⽉,⽇,⼈}
  \begin{Phonetics}{有时候}{you3shi2hou5}[][HSK 1]
    \definition{adv.}{às vezes; indica um momento incerto, mas não frequente}
  \seealsoref{有的时候}{you3de5shi2hou4}
  \seealsoref{有时}{you3shi2}
  \end{Phonetics}
\end{Entry}

\begin{Entry}{有没有}{6,7,6}{⽉,⽔,⽉}
  \begin{Phonetics}{有没有}{you3mei2you3}[][HSK 6]
    \definition{adv.}{Você tem\dots?; Você já\dots? ; Existe algum\dots?}
  \end{Phonetics}
\end{Entry}

\begin{Entry}{有事}{6,8}{⽉,⼅}
  \begin{Phonetics}{有事}{you3shi4}[][HSK 6]
    \definition{v.}{estar ocupado; estar envolvido | ter algo acontecendo; sofrer um acidente; se meter em encrenca | (com 心里) ter algo em mente; estar ansioso; preocupar-se | ter um emprego; estar empregado}[看他这几天愁眉苦脸的, 心里一定有事。===Vendo como ele parece triste ultimamente, deve haver algo em sua mente.]
  \seealsoref{心里}{xin1li3}
  \end{Phonetics}
\end{Entry}

\begin{Entry}{有些}{6,8}{⽉,⼆}
  \begin{Phonetics}{有些}{you3xie1}[][HSK 1]
    \definition{adv.}{um pouco; bastante; ligeiramente}
    \definition{pron.}{uma parte; alguns}
    \definition{v.}{usado para indicar que há alguns, mas não muitos;}
  \seealsoref{有(一)些}{you3 (yi4) xie1}
  \end{Phonetics}
\end{Entry}

\begin{Entry}{有的}{6,8}{⽉,⽩}
  \begin{Phonetics}{有的}{you3de5}[][HSK 1]
    \definition{pron.}{algum, alguns}
  \end{Phonetics}
\end{Entry}

\begin{Entry}{有的时候}{6,8,7,10}{⽉,⽩,⽇,⼈}
  \begin{Phonetics}{有的时候}{you3de5shi2hou4}
    \definition{adv.}{às vezes; ocasionalmente}
  \seealsoref{有时}{you3shi2}
  \seealsoref{有时候}{you3shi2hou5}
  \end{Phonetics}
\end{Entry}

\begin{Entry}{有的是}{6,8,9}{⽉,⽩,⽇}
  \begin{Phonetics}{有的是}{you3de5shi4}[][HSK 3]
    \definition{expr.}{ter em abundância; não faltar; enfatizar que há muitos}
  \end{Phonetics}
\end{Entry}

\begin{Entry}{有空儿}{6,8,2}{⽉,⽳,⼉}
  \begin{Phonetics}{有空儿}{you3kong4r5}[][HSK 2]
    \definition{v.}{estar livre; ter tempo livre}
  \end{Phonetics}
\end{Entry}

\begin{Entry}{有限}{6,8}{⽉,⾩}
  \begin{Phonetics}{有限}{you3xian4}[][HSK 4]
    \definition{adj.}{finito; limitado; restrito | indica baixo grau; indica pouco número; número baixo; nível baixo}
  \end{Phonetics}
\end{Entry}

\begin{Entry}{有限公司}{6,8,4,5}{⽉,⾩,⼋,⼝}
  \begin{Phonetics}{有限公司}{you3xian4gong1si1}
    \definition{s.}{companhia limitada | corporação}
  \end{Phonetics}
\end{Entry}

\begin{Entry}{有毒}{6,9}{⽉,⽏}
  \begin{Phonetics}{有毒}{you3du2}[][HSK 5]
    \definition{adj.}{venenoso; tóxico; nocivo; geralmente é usada para descrever as propriedades nocivas à saúde de produtos químicos, plantas ou animais.}
  \end{Phonetics}
\end{Entry}

\begin{Entry}{有点儿}{6,9,2}{⽉,⽕,⼉}
  \begin{Phonetics}{有点儿}{you3dian3r5}[][HSK 2]
    \definition{adv.}{um pouco; indica um grau inferior, equivalente a 稍微 (usado principalmente para coisas que são insatisfatórias)}
    \definition{v.}{há um pouco; tem (ou ser de) algum; existem alguns}
  \seealsoref{稍微}{shao1wei1}
  \seealsoref{有(一)点儿}{you3 yi4 dian3r5}
  \end{Phonetics}
\end{Entry}

\begin{Entry}{有害}{6,10}{⽉,⼧}
  \begin{Phonetics}{有害}{you3hai4}[][HSK 5]
    \definition{adj.}{prejudicial; nocivo; danoso; que pode causar danos ou prejuízos a algo}
  \end{Phonetics}
\end{Entry}

\begin{Entry}{有效}{6,10}{⽉,⽁}
  \begin{Phonetics}{有效}{you3xiao4}[][HSK 3]
    \definition{adj.}{válido; eficiente; eficaz; capaz de alcançar os objetivos esperados}
  \end{Phonetics}
\end{Entry}

\begin{Entry}{有着}{6,11}{⽉,⽬}
  \begin{Phonetics}{有着}{you3zhe5}[][HSK 5]
    \definition{v.}{ter; possuir; haver; existir}
  \end{Phonetics}
\end{Entry}

\begin{Entry}{有道理}{6,12,11}{⽉,⾡,⽟}
  \begin{Phonetics}{有道理}{you3dao4li5}
    \definition{v.}{fazer sentido; ser bem fundamentado; haver verdade em}
  \end{Phonetics}
\end{Entry}

\begin{Entry}{有意思}{6,13,9}{⽉,⼼,⼼}
  \begin{Phonetics}{有意思}{you3yi4si5}[][HSK 2]
    \definition{adj.}{significativo; significativo e intrigante | interessante; agradável}
    \definition{v.}{ter interesse por; ser atraído sexualmente}
  \end{Phonetics}
\end{Entry}

\begin{Entry}{有趣}{6,15}{⽉,⾛}
  \begin{Phonetics}{有趣}{you3qu4}[][HSK 4]
    \definition{adj.}{interessante; fascinante; divertido; excitante; estimulante}
  \end{Phonetics}
\end{Entry}

%%%%%%%%%% 朴 %%%%%%%%%%
\subsection*{朴}\addcontentsline{loh}{figure}{朴}

\begin{Entry}{朴}{6}{⽊}
  \begin{Phonetics}{朴}{piao2}
    \definition*{s.}{Sobrenome coreano: Park, Pak ou Bak | Sobrenome: Piao}
    \definition{adj.}{simples; despretensioso}
  \end{Phonetics}
  \begin{Phonetics}{朴}{po1}
    \definition{s.}{uma arma tradicional de haste com uma lâmina longa e estreita, usada com ambas as mãos}
  \end{Phonetics}
  \begin{Phonetics}{朴}{po4}
    \definition{s.}{\emph{hackberry} chinês; celtis}
  \end{Phonetics}
  \begin{Phonetics}{朴}{pu3}
    \definition{adj.}{simples e direto; simples e honesto}
  \end{Phonetics}
\end{Entry}

\begin{Entry}{朴实}{6,8}{⽊,⼧}
  \begin{Phonetics}{朴实}{pu3shi2}[][HSK 7-9]
    \definition{adj.}{simples; descomplicado; despretensioso | sincero e honesto; ingênuo | realista; com os pés no chão}
  \end{Phonetics}
\end{Entry}

\begin{Entry}{朴素}{6,10}{⽊,⽷}
  \begin{Phonetics}{朴素}{pu3su4}[][HSK 7-9]
    \definition{adj.}{(cor, estilo, linguagem, etc.) simples; sem graça; despretensioso; sem adornos; refere"-se à linguagem e às emoções que são sinceras e não exageradas | (estilo de vida) frugal; econômico; simples e modesto; não extravagante | ingênuo; subdesenvolvido; embrionário}
  \end{Phonetics}
\end{Entry}

%%%%%%%%%% 朵 %%%%%%%%%%
\subsection*{朵}\addcontentsline{loh}{figure}{朵}

\begin{Entry}{朵}{6}{⽊}
  \begin{Phonetics}{朵}{duo3}[][HSK 5]
    \definition*{s.}{Sobrenome: Duo}
    \definition{clas.}{usado para flores, nuvens ou coisas que se assemelham a flores e nuvens}
  \end{Phonetics}
\end{Entry}

%%%%%%%%%% 机 %%%%%%%%%%
\subsection*{机}\addcontentsline{loh}{figure}{机}

\begin{Entry}{机}{6}{⽊}
  \begin{Phonetics}{机}{ji1}
    \definition*{s.}{Sobrenome: Ji}
    \definition{adj.}{flexível; perspicaz; destreza; agilidade}
    \definition[台]{s.}{máquina; motor | avião; aeronave; aeroplano; refere"-se especificamente a aeronaves | ponto crucial; os fatores"-chave para a ocorrência e mudança das coisas | chance; ocasião; oportunidade; um momento crítico ou oportuno para o desenvolvimento e mudança das coisas | organismo; funções vitais dos organismos | besta; mecanismo de disparo de flechas de madeira em uma besta antiga | assuntos importantes; assuntos extremamente importantes e confidenciais | ideia; intenção}
  \end{Phonetics}
\end{Entry}

\begin{Entry}{机甲}{6,5}{⽊,⽥}
  \begin{Phonetics}{机甲}{ji1jia3}
    \definition{s.}{\emph{mecha} (robôs operados por humanos em mangá japonês)}
  \end{Phonetics}
\end{Entry}

\begin{Entry}{机会}{6,6}{⽊,⼈}
  \begin{Phonetics}{机会}{ji1hui5}[][HSK 2]
    \definition[个,次,种,些]{s.}{chance; oportunidade; momento favorável raro}
  \end{Phonetics}
\end{Entry}

\begin{Entry}{机关}{6,6}{⽊,⼋}
  \begin{Phonetics}{机关}{ji1guan1}[][HSK 6]
    \definition{adj.}{operado por máquina | controlado mecanicamente}
    \definition[个]{s.}{engrenagem; mecanismo; Antigo: refere"-se a certos dispositivos controlados mecanicamente; também se refere às peças de frenagem de dispositivos mecânicos | escritório; órgão; corpo; instituição | esquema; maquinação; estratagema; um plano cuidadoso e inteligente}
  \end{Phonetics}
\end{Entry}

\begin{Entry}{机动}{6,6}{⽊,⼒}
  \begin{Phonetics}{机动}{ji1dong4}[][HSK 7-9]
    \definition{adj.}{motorizado; movido a energia | flexível; manobrável; conveniente; móvel | em reserva; para uso emergencial}
  \end{Phonetics}
\end{Entry}

\begin{Entry}{机动车}{6,6,4}{⽊,⼒,⾞}
  \begin{Phonetics}{机动车}{ji1dong4che1}[][HSK 6]
    \definition{s.}{veículo motorizado | veículo automotor; automóvel de passageiros: veículo comercial concebido e tecnicamente adequado para o transporte de passageiros e respetiva bagagem, incluindo o banco do condutor}
  \antonymref{人力车}{ren2li4che1}
  \end{Phonetics}
\end{Entry}

\begin{Entry}{机场}{6,6}{⽊,⼟}
  \begin{Phonetics}{机场}{ji1chang3}[][HSK 1]
    \definition[个,家,处,座]{s.}{aeródromo; campo de aviação; aeroporto; campo de voo}
  \end{Phonetics}
\end{Entry}

\begin{Entry}{机灵}{6,7}{⽊,⽕}
  \begin{Phonetics}{机灵}{ji1ling5}[][HSK 7-9]
    \definition{adj.}{inteligente; esperto; astuto; espirituoso}
  \end{Phonetics}
\end{Entry}

\begin{Entry}{机制}{6,8}{⽊,⼑}
  \begin{Phonetics}{机制}{ji1zhi4}[][HSK 5]
    \definition{s.}{mecanismo; processado por máquina; feito por máquina}
  \end{Phonetics}
\end{Entry}

\begin{Entry}{机构}{6,8}{⽊,⽊}
  \begin{Phonetics}{机构}{ji1gou4}[][HSK 4]
    \definition[所]{s.}{órgão; organização; instituição; instalações; aparelhamento; configuração | mecanismo; funcionamento interno de uma máquina ou unidade | estrutura interna de uma organização}
  \end{Phonetics}
\end{Entry}

\begin{Entry}{机舱}{6,10}{⽊,⾈}
  \begin{Phonetics}{机舱}{ji1cang1}[][HSK 7-9]
    \definition{s.}{sala de máquinas (de um navio) | compartimento de passageiros (de uma aeronave); cabine | espaço de máquinas; cabine de aeronave}
  \end{Phonetics}
\end{Entry}

\begin{Entry}{机密}{6,11}{⽊,⼧}
  \begin{Phonetics}{机密}{ji1mi4}[][HSK 7-9]
    \definition{adj.}{secreto; classificado; privado; confidencial}
    \definition{s.}{segredo; assuntos confidenciais}
  \end{Phonetics}
\end{Entry}

\begin{Entry}{机械}{6,11}{⽊,⽊}
  \begin{Phonetics}{机械}{ji1xie4}[][HSK 6]
    \definition{adj.}{rígido; mecânico; inflexível; uma metáfora para uma abordagem rígida e imutável}
    \definition[台,部,个]{s.}{máquina; maquinário; mecanismo; vários dispositivos compostos por princípios mecânicos}
  \end{Phonetics}
\end{Entry}

\begin{Entry}{机票}{6,11}{⽊,⽰}
  \begin{Phonetics}{机票}{ji1piao4}[][HSK 1]
    \definition[张]{s.}{passagem aérea; passagem de avião}
  \seealsoref{飞机票}{fei1ji1piao4}
  \end{Phonetics}
\end{Entry}

\begin{Entry}{机智}{6,12}{⽊,⽇}
  \begin{Phonetics}{机智}{ji1zhi4}[][HSK 7-9]
    \definition{adj.}{engenhoso; perspicaz; inteligente e adaptável}
  \end{Phonetics}
\end{Entry}

\begin{Entry}{机遇}{6,12}{⽊,⾡}
  \begin{Phonetics}{机遇}{ji1yu4}[][HSK 4]
    \definition[个]{s.}{chance; oportunidade; circunstâncias favoráveis}
  \end{Phonetics}
\end{Entry}

\begin{Entry}{机器}{6,16}{⽊,⼝}
  \begin{Phonetics}{机器}{ji1qi4}[][HSK 3]
    \definition[台,部,个]{s.}{máquina; maquinário; motor; dispositivos e máquinas que são montados a partir de peças, podem funcionar, transformar energia ou produzir trabalho útil podem ser usados como ferramentas de produção, reduzindo a intensidade do trabalho humano e aumentando a produtividade | aparato; sistema político e econômico}
  \end{Phonetics}
\end{Entry}

\begin{Entry}{机器人}{6,16,2}{⽊,⼝,⼈}
  \begin{Phonetics}{机器人}{ji1qi4ren2}[][HSK 5]
    \definition[个,些]{s.}{androide; golem | pessoa mecânica | robô}
  \end{Phonetics}
\end{Entry}

%%%%%%%%%% 杀 %%%%%%%%%%
\subsection*{杀}\addcontentsline{loh}{figure}{杀}

\begin{Entry}{杀}{6}{⽊}
  \begin{Phonetics}{杀}{sha1}[][HSK 5]
    \definition{adv.}{em extremo; excessivamente; usado após um verbo, indica grau intenso}
    \definition{v.}{matar; abater; esquartejar | lutar; entrar em batalha | enfraquecer; reduzir; diminuir | decolar; neutralizar}
  \end{Phonetics}
\end{Entry}

\begin{Entry}{杀手}{6,4}{⽊,⼿}
  \begin{Phonetics}{杀手}{sha1shou3}[][HSK 7-9]
    \definition{s.}{assassino; homicida | pessoa magistral (de certo tipo) | Esporte: jogador formidável | assassino de aluguel}
  \end{Phonetics}
\end{Entry}

\begin{Entry}{杀气}{6,4}{⽊,⽓}
  \begin{Phonetics}{杀气}{sha1qi4}
    \definition{s.}{espírito assassino | aura de morte}
    \definition{v.}{desabafar a raiva de alguém}
  \end{Phonetics}
\end{Entry}

\begin{Entry}{杀毒}{6,9}{⽊,⽏}
  \begin{Phonetics}{杀毒}{sha1 du2}[][HSK 5]
    \definition{s.}{Computação: antivírus}
    \definition{v.}{esterilizar; desinfetar | Computação: eliminar um vírus}
  \end{Phonetics}
\end{Entry}

\begin{Entry}{杀害}{6,10}{⽊,⼧}
  \begin{Phonetics}{杀害}{sha1hai4}[][HSK 7-9]
    \definition{v.}{assassinar; massacrar; trucidar; chacinar; matar (uma pessoa) por motivos ilegítimos}
  \end{Phonetics}
\end{Entry}

%%%%%%%%%% 杂 %%%%%%%%%%
\subsection*{杂}\addcontentsline{loh}{figure}{杂}

\begin{Entry}{杂}{6}{⽊}
  \begin{Phonetics}{杂}{za2}[][HSK 6]
    \definition{adj.}{diversos; misto; misturados | extra; irregular | variado}
    \definition{v.}{misturar}
  \end{Phonetics}
\end{Entry}

\begin{Entry}{杂志}{6,7}{⽊,⼼}
  \begin{Phonetics}{杂志}{za2zhi4}[][HSK 3]
    \definition[本,期,份,种]{s.}{jornal; revista; publicação}
  \end{Phonetics}
\end{Entry}

\begin{Entry}{杂志社}{6,7,7}{⽊,⼼,⽰}
  \begin{Phonetics}{杂志社}{za2zhi4 she4}
    \definition{s.}{editora de revista; a organização responsável pela edição, publicação e distribuição de revistas}
  \end{Phonetics}
\end{Entry}

\begin{Entry}{杂技}{6,7}{⽊,⼿}
  \begin{Phonetics}{杂技}{za2ji4}
    \definition[场,个]{s.}{acrobacia; um termo geral para várias performances (como habilidades com carros, ventriloquia, equilíbrio de tigelas, andar na corda bamba, dança do leão, mágica, etc.)}
  \end{Phonetics}
\end{Entry}

\begin{Entry}{杂剧}{6,10}{⽊,⼑}
  \begin{Phonetics}{杂剧}{za2ju4}
    \definition{s.}{(na Dinastia Song) peça de variedades que consiste em um prelúdio, a peça principal em uma ou duas cenas e um epílogo musical (nenhuma das peças de variedades Song existe até hoje) |  (na Dinastia Yuan) drama poético que consiste em quatro atos ou sequências de canções (折), ocasionalmente incluindo uma ``cunha'' (楔子) na forma de um prólogo (colocado antes do primeiro ato) ou um interlúdio (colocado entre os atos), todas as partes cantadas nos quatro atos atribuídas ao protagonista, seja homem ou mulher | uma forma de comédia musical da dinastia Yuan; uma forma de performance caracterizada pelo humor e pela brincadeira na Dinastia Song, desenvolveu-se como uma forma de ópera na Dinastia Yuan, cada obra consiste em quatro atos, às vezes com um prólogo no início ou entre os atos, cada ato é composto por um conjunto de melodias nórdicas na mesma melodia e rima palaciana, além de versos de convidados}
  \seealsoref{楔子}{xie1zi5}
  \seealsoref{折}{zhe2}
  \end{Phonetics}
\end{Entry}

%%%%%%%%%% 权 %%%%%%%%%%
\subsection*{权}\addcontentsline{loh}{figure}{权}

\begin{Entry}{权}{6}{⽊}
  \begin{Phonetics}{权}{quan2}[][HSK 6]
    \definition*{s.}{Sobrenome: Quan}
    \definition{adv.}{provisoriamente; por enquanto}
    \definition{s.}{Lliterário: contrapeso; peso deslizante de uma balança romana | poder; autoridade | direito | posição vantajosa | conveniência}
    \definition{v.}{pesar; medir o peso}
  \end{Phonetics}
\end{Entry}

\begin{Entry}{权力}{6,2}{⽊,⼒}
  \begin{Phonetics}{权力}{quan2li4}[][HSK 6]
    \definition[种]{s.}{poder; autoridade; o poder de liderança no âmbito da responsabilidade | poder; coerção política; o poder coercitivo do status social e político}
  \end{Phonetics}
\end{Entry}

\begin{Entry}{权利}{6,7}{⽊,⼑}
  \begin{Phonetics}{权利}{quan2li4}[][HSK 4]
    \definition[项,种,个,条,份]{s.}{direito; interesse; os poderes e benefícios exercidos por um cidadão ou pessoa jurídica de acordo com a lei}
  \antonymref{义务}{yi4wu4}
  \end{Phonetics}
\end{Entry}

\begin{Entry}{权威}{6,9}{⽊,⼥}
  \begin{Phonetics}{权威}{quan2wei1}[][HSK 7-9]
    \definition{adj.}{autoritativo; tem o poder e o prestígio para convencer as pessoas}
    \definition{s.}{autoridade; poder de decisão; o poder e o prestígio para inspirar a fé | autoridade; pessoa de autoridade; a pessoa ou coisa mais influente e relevante em um determinado âmbito ou área}
  \end{Phonetics}
\end{Entry}

\begin{Entry}{权益}{6,10}{⽊,⽫}
  \begin{Phonetics}{权益}{quan2yi4}[][HSK 7-9]
    \definition{s.}{direitos; interesses; direito legal; direitos e interesses; os direitos invioláveis que devem ser desfrutados}
  \end{Phonetics}
\end{Entry}

\begin{Entry}{权衡}{6,16}{⽊,⾏}
  \begin{Phonetics}{权衡}{quan2heng2}[][HSK 7-9]
    \definition{s.}{peso; peso de pesagem e balança de pesagem}
    \definition{v.}{pesar; equilibrar; calcular; refere"-se a pesar, comparar e considerar}
  \end{Phonetics}
\end{Entry}

%%%%%%%%%% 次 %%%%%%%%%%
\subsection*{次}\addcontentsline{loh}{figure}{次}

\begin{Entry}{次}{6}{⽋}
  \begin{Phonetics}{次}{ci4}[][HSK 1,4]
    \definition*{s.}{Sobrenome: Ci}
    \definition{adj.}{de segunda categoria; de qualidade inferior}
    \definition{clas.}{usado para coisas ou ações que podem ser repetidas}
    \definition{num.}{segundo; próximo}
    \definition{pref.}{(química) hipo-, radical ácido ou composto contendo dois átomos de oxigênio a menos}
    \definition{s.}{ordem; sequência; classificação | local de parada em uma viagem; escala}
  \end{Phonetics}
\end{Entry}

\begin{Entry}{次日}{6,4}{⽋,⽇}
  \begin{Phonetics}{次日}{ci4ri4}[][HSK 7-9]
    \definition{s.}{dia seguinte; amanhã}
  \end{Phonetics}
\end{Entry}

\begin{Entry}{次数}{6,13}{⽋,⽁}
  \begin{Phonetics}{次数}{ci4shu4}[][HSK 6]
    \definition{s.}{frequência; número de vezes; o número de vezes que uma ação ou evento é repetido}
  \end{Phonetics}
\end{Entry}

%%%%%%%%%% 欢 %%%%%%%%%%
\subsection*{欢}\addcontentsline{loh}{figure}{欢}

\begin{Entry}{欢}{6}{⽋}
  \begin{Phonetics}{欢}{huan1}
    \definition*{s.}{Sobrenome: Huan}
    \definition{adj.}{alegre; feliz; jubilante | vigoroso; energético; em pleno andamento; com grande impulso}
    \definition{s.}{amante; querida; um apelido usado por mulheres nos tempos antigos para se referir aos seus amantes; agora, geralmente se refere a alguém de quem você gosta ou com quem tem um relacionamento romântico}
  \end{Phonetics}
\end{Entry}

\begin{Entry}{欢乐}{6,5}{⽋,⼃}
  \begin{Phonetics}{欢乐}{huan1le4}[][HSK 3]
    \definition{adj.}{feliz; alegre; felicidade (geralmente coletiva)}
  \end{Phonetics}
\end{Entry}

\begin{Entry}{欢声笑语}{6,7,10,9}{⽋,⼠,⽵,⾔}
  \begin{Phonetics}{欢声笑语}{huan1sheng1-xiao4yu3}[][HSK 7-9]
    \definition{expr.}{risos felizes e vozes alegres}
  \end{Phonetics}
\end{Entry}

\begin{Entry}{欢快}{6,7}{⽋,⼼}
  \begin{Phonetics}{欢快}{huan1kuai4}[][HSK 7-9]
    \definition{adj.}{alegre; animado; alegre e despreocupado; feliz e alegre}
  \end{Phonetics}
\end{Entry}

\begin{Entry}{欢迎}{6,7}{⽋,⾡}
  \begin{Phonetics}{欢迎}{huan1ying2}[][HSK 2]
    \definition{adj.}{bem-vindo}
    \definition{v.}{dar as boas-vindas; cumprimentar; receber com alegria | dar as boas-vindas; receber favoravelmente (bem)}
  \end{Phonetics}
\end{Entry}

\begin{Entry}{欢呼}{6,8}{⽋,⼝}
  \begin{Phonetics}{欢呼}{huan1hu1}[][HSK 7-9]
    \definition{v.}{saudar; aplaudir; aclamar; dar vivas}
  \end{Phonetics}
\end{Entry}

\begin{Entry}{欢聚}{6,14}{⽋,⽿}
  \begin{Phonetics}{欢聚}{huan1ju4}[][HSK 7-9]
    \definition{s.}{celebração | festa}
    \definition{v.}{desfrutar de uma reunião feliz; reunir-se alegremente | celebrar | reunir-se socialmente}
  \end{Phonetics}
\end{Entry}

%%%%%%%%%% 此 %%%%%%%%%%
\subsection*{此}\addcontentsline{loh}{figure}{此}

\begin{Entry}{此}{6}{⽌}
  \begin{Phonetics}{此}{ci3}[][HSK 4]
    \definition*{s.}{Sobrenome: Ci}
    \definition{pron.}{esse; essa; isso; este; esta; isto; indica ou se refere a uma pessoa ou coisa que está mais próxima, equivalente a 这 ou 这个 | aqui e agora; refere"-se a um tempo ou lugar recente, equivalente a 这会儿 ou 这里}
  \seealsoref{这}{zhe4}
  \seealsoref{这会儿}{zhe4 hui4r5}
  \seealsoref{这里}{zhe4li3}
  \seealsoref{这个}{zhe4ge5}
  \antonymref{彼}{bi3}
  \end{Phonetics}
\end{Entry}

\begin{Entry}{此处}{6,5}{⽌,⼡}
  \begin{Phonetics}{此处}{ci3chu4}[][HSK 6]
    \definition{pron.}{este lugar; aqui (literário)}
  \end{Phonetics}
\end{Entry}

\begin{Entry}{此外}{6,5}{⽌,⼣}
  \begin{Phonetics}{此外}{ci3wai4}[][HSK 4]
    \definition{conj.}{além disso; em adição; além das coisas ou situações mencionadas acima}
  \end{Phonetics}
\end{Entry}

\begin{Entry}{此后}{6,6}{⽌,⼝}
  \begin{Phonetics}{此后}{ci3hou4}[][HSK 5]
    \definition{s.}{daqui em diante; doravante; depois disso; após isso; de agora em diante}
  \end{Phonetics}
\end{Entry}

\begin{Entry}{此次}{6,6}{⽌,⽋}
  \begin{Phonetics}{此次}{ci3ci4}[][HSK 6]
    \definition{adv.}{desta vez; refere"-se a um ponto específico no tempo ou período de tempo}
  \end{Phonetics}
\end{Entry}

\begin{Entry}{此时}{6,7}{⽌,⽇}
  \begin{Phonetics}{此时}{ci3shi2}[][HSK 5]
    \definition{s.}{agora; no presente; agora mesmo; neste momento; por enquanto}
  \end{Phonetics}
\end{Entry}

\begin{Entry}{此事}{6,8}{⽌,⼅}
  \begin{Phonetics}{此事}{ci3shi4}[][HSK 6]
    \definition{s.}{matéria; assunto}
  \end{Phonetics}
\end{Entry}

\begin{Entry}{此刻}{6,8}{⽌,⼑}
  \begin{Phonetics}{此刻}{ci3ke4}[][HSK 5]
    \definition{s.}{agora; no momento; exatamente agora; neste momento}
  \end{Phonetics}
\end{Entry}

\begin{Entry}{此前}{6,9}{⽌,⼑}
  \begin{Phonetics}{此前}{ci3qian2}[][HSK 6]
    \definition{adv.}{literário: antes; anteriormente | literário: antes disso}
  \end{Phonetics}
\end{Entry}

\begin{Entry}{此致}{6,10}{⽌,⾄}
  \begin{Phonetics}{此致}{ci3zhi4}[][HSK 6]
    \definition{expr.}{Atenciosamente; Sinceramente; Com os melhores votos; usada no final de uma carta ou correspondência oficial}
  \end{Phonetics}
\end{Entry}

\begin{Entry}{此起彼伏}{6,10,8,6}{⽌,⾛,⼻,⼈}
  \begin{Phonetics}{此起彼伏}{ci3qi3-bi3fu2}[][HSK 7-9]
    \definition{expr.}{``Quando um cai, outro se levanta.''  ou se levanta um após o outro | ``Assim que um desaparece, o próximo surge.'' | ocorrendo repetidamente (de aplausos, incêndios, acenos, protestos, conflitos, revoltas etc.) | repetindo continuamente | aqui em cima, lá embaixo; subir e descer em sucessão}
  \end{Phonetics}
\end{Entry}

%%%%%%%%%% 死 %%%%%%%%%%
\subsection*{死}\addcontentsline{loh}{figure}{死}

\begin{Entry}{死}{6}{⽍}
  \begin{Phonetics}{死}{si3}[][HSK 3]
    \definition{adj.}{até a morte | implacável; mortal | fixo; rígido; inflexível | intransitável; fechado | (expressando raiva, reclamação, etc., às vezes jocosamente) maldito}
    \definition{adv.}{(frequentemente no negativo) teimosamente; inflexivelmente}
    \definition{v.}{morrer; estar morto}
  \antonymref{活}{huo2}
  \antonymref{生}{sheng1}
  \end{Phonetics}
\end{Entry}

\begin{Entry}{死亡}{6,3}{⽍,⼇}
  \begin{Phonetics}{死亡}{si3wang2}[][HSK 6]
    \definition{s.}{morte; condenação; dar o último suspiro; refere"-se ao estado de vida desaparecendo}
    \definition{v.}{morrer; estar morto; perder a vida (em oposição à 生存)}
  \seealsoref{生存}{sheng1cun2}
  \end{Phonetics}
\end{Entry}

\begin{Entry}{死心}{6,4}{⽍,⼼}
  \begin{Phonetics}{死心}{si3xin1}[][HSK 7-9]
    \definition{v.}{abandonar a ideia para sempre; não alimentar mais ilusões sobre o assunto; desistir da ideia e parar de ter esperança}
  \antonymref{留恋}{liu2lian4}
  \antonymref{迷恋}{mi2lian4}
  \antonymref{牵挂}{qian1gua4}
  \end{Phonetics}
\end{Entry}

\begin{Entry}{死心塌地}{6,4,13,6}{⽍,⼼,⼟,⼟}
  \begin{Phonetics}{死心塌地}{si3xin1-ta1di4}[][HSK 7-9]
    \definition{expr.}{estar obstinado em; estar irremediavelmente decidido a; comprometer-se desesperadamente e irremediavelmente com um caminho maligno; desistir de qualquer ideia de uma alternativa; descreve uma decisão que está firmemente tomada e não será alterada}
  \end{Phonetics}
\end{Entry}

%%%%%%%%%% 毕 %%%%%%%%%%
\subsection*{毕}\addcontentsline{loh}{figure}{毕}

\begin{Entry}{毕}{6}{⽐}
  \begin{Phonetics}{毕}{bi4}
    \definition*{s.}{Bi, uma das mansões lunares; a décima nona das vinte e oito constelações em que a esfera celeste foi dividida, consistindo de oito estrelas, seis em Híades e duas em Touro | Sobrenome: Bi}
    \definition{adv.}{tudo; completamente; totalmente}
    \definition{v.}{terminar; realizar; concluir  | completar; terminar}
  \end{Phonetics}
\end{Entry}

\begin{Entry}{毕业}{6,5}{⽐,⼀}
  \begin{Phonetics}{毕业}{bi4/ye4}[][HSK 4]
    \definition{v.+compl.}{formar"-se}
  \end{Phonetics}
\end{Entry}

\begin{Entry}{毕业生}{6,5,5}{⽐,⼀,⽣}
  \begin{Phonetics}{毕业生}{bi4ye4sheng1}[][HSK 4]
    \definition[个,名,位,些]{s.}{diplomado; graduado; bacharel; pessoa que recebeu um diploma, grau ou certificado}
  \end{Phonetics}
\end{Entry}

\begin{Entry}{毕竟}{6,11}{⽐,⾳}
  \begin{Phonetics}{毕竟}{bi4jing4}[][HSK 5]
    \definition{adv.}{afinal de contas; quando tudo estiver dito e feito; em última análise; indica um resultado que não pode ser alterado, enfatizando que se trata de uma causa ou fato que precisa ser enfocado para referência | significa 到底, 究竟, 终究, indicando a conclusão final alcançada}
  \seealsoref{到底}{dao4di3}
  \seealsoref{究竟}{jiu1jing4}
  \seealsoref{终究}{zhong1jiu1}
  \end{Phonetics}
\end{Entry}

%%%%%%%%%% 汗 %%%%%%%%%%
\subsection*{汗}\addcontentsline{loh}{figure}{汗}

\begin{Entry}{汗}{6}{⽔}
  \begin{Phonetics}{汗}{han2}
    \definition*{s.}{Abreviação de Khan}[他是成吉思汗。===Ele é Genghis Khan.]
  \end{Phonetics}
  \begin{Phonetics}{汗}{han4}[][HSK 5]
    \definition{s.}{suor; transpiração; perspiração}
  \end{Phonetics}
\end{Entry}

\begin{Entry}{汗水}{6,4}{⽔,⽔}
  \begin{Phonetics}{汗水}{han4shui3}[][HSK 7-9]
    \definition{s.}{transpiração; suor (em grandes quantidades)}
  \end{Phonetics}
\end{Entry}

\begin{Entry}{汗液}{6,11}{⽔,⽔}
  \begin{Phonetics}{汗液}{han4ye4}
    \definition{s.}{suor}
  \end{Phonetics}
\end{Entry}

\begin{Entry}{汗腺}{6,13}{⽔,⾁}
  \begin{Phonetics}{汗腺}{han4xian4}
    \definition{s.}{glândula sudorípara}
  \end{Phonetics}
\end{Entry}

%%%%%%%%%% 江 %%%%%%%%%%
\subsection*{江}\addcontentsline{loh}{figure}{江}

\begin{Entry}{江}{6}{⽔}
  \begin{Phonetics}{江}{jiang1}[][HSK 4]
    \definition*{s.}{Rio Changjiang | Sobrenome: Jiang}
    \definition[条,道]{s.}{rio grande}
  \end{Phonetics}
\end{Entry}

\begin{Entry}{江水}{6,4}{⽔,⽔}
  \begin{Phonetics}{江水}{jiang1shui3}
    \definition{s.}{água do rio}
  \end{Phonetics}
\end{Entry}

\begin{Entry}{江西}{6,6}{⽔,⾑}
  \begin{Phonetics}{江西}{jiang1xi1}
    \definition*{s.}{Jiangxi}
  \end{Phonetics}
\end{Entry}

\begin{Entry}{江苏}{6,7}{⽔,⾋}
  \begin{Phonetics}{江苏}{jiang1su1}
    \definition*{s.}{Província de Jiangsu}
  \end{Phonetics}
\end{Entry}

\begin{Entry}{江南水乡}{6,9,4,3}{⽔,⼗,⽔,⼄}
  \begin{Phonetics}{江南水乡}{jiang1nan2shui3xiang1}
    \definition*{s.}{Vila Aquática de Jiangnan | Cidades Aquáticas}
  \end{Phonetics}
\end{Entry}

%%%%%%%%%% 池 %%%%%%%%%%
\subsection*{池}\addcontentsline{loh}{figure}{池}

\begin{Entry}{池}{6}{⽔}
  \begin{Phonetics}{池}{chi2}
    \definition*{s.}{Sobrenome: Chi}
    \definition[个,片]{s.}{piscina; lagoa | qualquer espaço fechado com laterais elevadas | baias (em um teatro); a parte frontal do salão principal do teatro | fosso}
  \end{Phonetics}
\end{Entry}

\begin{Entry}{池子}{6,3}{⽔,⼦}
  \begin{Phonetics}{池子}{chi2zi5}[][HSK 5]
    \definition{s.}{lago; lagoa; viveiro | piscina; piscina do balneário | (antigo) arquibancada (primeiras fileiras em um teatro) | pista de dança de um salão de baile}
  \end{Phonetics}
\end{Entry}

\begin{Entry}{池塘}{6,13}{⽔,⼟}
  \begin{Phonetics}{池塘}{chi2tang2}[][HSK 7-9]
    \definition[个]{s.}{lagoa; açude; grande poço de armazenamento de água | piscina comum (em um balneário)}
  \end{Phonetics}
\end{Entry}

%%%%%%%%%% 污 %%%%%%%%%%
\subsection*{污}\addcontentsline{loh}{figure}{污}

\begin{Entry}{污}{6}{⽔}
  \begin{Phonetics}{污}{wu1}
    \definition{adj.}{sujo; imundo; imundo | corrupto}
    \definition{s.}{sujeira; imundície | esgoto; água suja; coisas sujas}
    \definition{v.}{contaminar; sujar | manchar}
  \end{Phonetics}
\end{Entry}

\begin{Entry}{污水}{6,4}{⽔,⽔}
  \begin{Phonetics}{污水}{wu1shui3}[][HSK 5]
    \definition[桶,滩]{s.}{água suja (ou poluída, residual); esgoto; lodo | efluente; drenagem; água suja; água poluída; água residual}
  \end{Phonetics}
\end{Entry}

\begin{Entry}{污染}{6,9}{⽔,⽊}
  \begin{Phonetics}{污染}{wu1ran3}[][HSK 5]
    \definition{v.}{poluir; contaminar com substâncias nocivas e prejudiciais; refere"-se especificamente à destruição do ambiente natural causada por substâncias nocivas, tais como gases, líquidos e resíduos emitidos por indústrias, minas, veículos, etc. | contaminar; metáfora de que pensamentos prejudiciais causam efeitos negativos nas pessoas}
  \end{Phonetics}
\end{Entry}

\begin{Entry}{污染区}{6,9,4}{⽔,⽊,⼖}
  \begin{Phonetics}{污染区}{wu1ran3qu1}
    \definition{s.}{área contaminada}
  \end{Phonetics}
\end{Entry}

\begin{Entry}{污染物}{6,9,8}{⽔,⽊,⽜}
  \begin{Phonetics}{污染物}{wu1ran3wu4}
    \definition{s.}{poluente}
  \seealsoref{污染物质}{wu1ran3 wu4zhi4}
  \end{Phonetics}
\end{Entry}

\begin{Entry}{污染物质}{6,9,8,8}{⽔,⽊,⽜,⾙}
  \begin{Phonetics}{污染物质}{wu1ran3 wu4zhi4}
    \definition{s.}{poluente}
  \seealsoref{污染物}{wu1ran3wu4}
  \end{Phonetics}
\end{Entry}

%%%%%%%%%% 汤 %%%%%%%%%%
\subsection*{汤}\addcontentsline{loh}{figure}{汤}

\begin{Entry}{汤}{6}{⽔}
  \begin{Phonetics}{汤}{shang1}
    \definition{s.}{correnteza forte}
  \end{Phonetics}
  \begin{Phonetics}{汤}{tang1}[][HSK 3]
    \definition*{s.}{Sobrenome: Tang}
    \definition[勺,碗,杯,锅]{s.}{água quente; água fervente | fontes termais | água utilizada para ferver algo| sopa; caldo | uma preparação líquida de ervas medicinais; decocção}
  \end{Phonetics}
\end{Entry}

\begin{Entry}{汤圆}{6,10}{⽔,⼞}
  \begin{Phonetics}{汤圆}{tang1yuan2}[][HSK 7-9]
    \definition{s.}{bolinho doce; (geralmente bolinhos recheados) feitos de farinha de arroz glutinoso servidos em sopa}
  \synonymref{元宵}{yuan2xiao1}
  \end{Phonetics}
\end{Entry}

%%%%%%%%%% 灯 %%%%%%%%%%
\subsection*{灯}\addcontentsline{loh}{figure}{灯}

\begin{Entry}{灯}{6}{⽕}
  \begin{Phonetics}{灯}{deng1}[][HSK 2]
    \definition*{s.}{Sobrenome: Deng}
    \definition[盏,个]{s.}{lâmpada; luz; lanterna; dispositivo luminoso, usado principalmente para iluminação | queimador; um aparelho que brilha e aquece como uma lâmpada e pode ser usado para aquecer | tubo; válvula; o nome popular dado aos tubos eletrônicos com formato semelhante a lâmpadas encontrados em aparelhos antigos, como rádios}
  \end{Phonetics}
\end{Entry}

\begin{Entry}{灯丝}{6,5}{⽕,⼀}
  \begin{Phonetics}{灯丝}{deng1si1}
    \definition{s.}{filamento (de uma lâmpada)}
  \end{Phonetics}
\end{Entry}

\begin{Entry}{灯号}{6,5}{⽕,⼝}
  \begin{Phonetics}{灯号}{deng1hao4}
    \definition{s.}{sinal luminoso | luz indicadora}
  \end{Phonetics}
\end{Entry}

\begin{Entry}{灯光}{6,6}{⽕,⼉}
  \begin{Phonetics}{灯光}{deng1guang1}[][HSK 4]
    \definition[束,盏,点,打]{s.}{iluminação; luminosidade da lâmpada | luminação (palco); equipamento de iluminação para palco ou estúdio}
  \end{Phonetics}
\end{Entry}

\begin{Entry}{灯泡}{6,8}{⽕,⽔}
  \begin{Phonetics}{灯泡}{deng1pao4}[][HSK 7-9]
    \definition[只,个]{s.}{lâmpada (bulbo) | (gíria) terceiro indesejado estragando encontro de casal; é frequentemente usado para descrever a si mesmo ou a outros se sentindo estranhos ou indesejados em situações sociais}
  \seealsoref{电灯泡}{dian4deng1pao4}
  \end{Phonetics}
\end{Entry}

\begin{Entry}{灯标}{6,9}{⽕,⽊}
  \begin{Phonetics}{灯标}{deng1biao1}
    \definition{s.}{farol | luz de farol}
  \end{Phonetics}
\end{Entry}

\begin{Entry}{灯笼}{6,11}{⽕,⽵}
  \begin{Phonetics}{灯笼}{deng1long5}[][HSK 7-9]
    \definition[个,盏,只]{s.}{lanterna; luminárias suspensas ou portáteis, geralmente feitas de finas tiras de bambu ou arame de ferro como estrutura, cobertas com areia ou papel, com velas dentro; atualmente, lâmpadas elétricas são usadas principalmente como fontes de luz e como decoração}
  \end{Phonetics}
\end{Entry}

%%%%%%%%%% 灰 %%%%%%%%%%
\subsection*{灰}\addcontentsline{loh}{figure}{灰}

\begin{Entry}{灰}{6}{⽕}
  \begin{Phonetics}{灰}{hui1}[][HSK 7-9]
    \definition{adj.}{cinza (cor) | desanimado; desencorajado; deprimido}
    \definition[把,堆]{s.}{cinzas; pó que sobra após a queima de um objeto | pó; poeira; substância em pó | cal; argamassa (de cal)}
  \end{Phonetics}
\end{Entry}

\begin{Entry}{灰心}{6,4}{⽕,⼼}
  \begin{Phonetics}{灰心}{hui1/xin1}[][HSK 7-9]
    \definition{v.+compl.}{desanimar; ficar desanimado; ficar desapontado; (devido a dificuldades, fracassos) ficar deprimido}
  \end{Phonetics}
\end{Entry}

\begin{Entry}{灰尘}{6,6}{⽕,⼩}
  \begin{Phonetics}{灰尘}{hui1chen2}[][HSK 7-9]
    \definition[层,堆]{s.}{cinza; poeira; sujeira; pó}
  \end{Phonetics}
\end{Entry}

\begin{Entry}{灰色}{6,6}{⽕,⾊}
  \begin{Phonetics}{灰色}{hui1se4}[][HSK 5]
    \definition{adj.}{obscuro; ambíguo | sombrio; pessimista}
    \definition[种]{s.}{cor cinza; acinzentado}
  \end{Phonetics}
\end{Entry}

%%%%%%%%%% 爷 %%%%%%%%%%
\subsection*{爷}\addcontentsline{loh}{figure}{爷}

\begin{Entry}{爷}{6}{⽗}
  \begin{Phonetics}{爷}{ye2}
    \definition[个,位,名,些]{s.}{(dialeto) pai | (dialeto) avô | (uma forma respeitosa de se dirigir a um homem idoso) tio | (uma forma de se dirigir a um oficial ou homem rico) senhor; mestre; lorde; o antigo nome para burocratas, pessoas ricas, etc. | deus; forma de tratamento de um adorador para um deus}
  \end{Phonetics}
\end{Entry}

\begin{Entry}{爷爷}{6,6}{⽗,⽗}
  \begin{Phonetics}{爷爷}{ye2ye5}[][HSK 1]
    \definition[个,位]{s.}{avô (paterno)}
  \end{Phonetics}
\end{Entry}

%%%%%%%%%% 牝 %%%%%%%%%%
\subsection*{牝}\addcontentsline{loh}{figure}{牝}

\begin{Entry}{牝}{6}{⽜}
  \begin{Phonetics}{牝}{pin4}
    \definition{adj.}{(de certas aves e animais) fêmea}
    \definition{s.}{fêmea (de algumas aves e animais)}
  \antonymref{牡}{mu3}
  \end{Phonetics}
\end{Entry}

%%%%%%%%%% 百 %%%%%%%%%%
\subsection*{百}\addcontentsline{loh}{figure}{百}

\begin{Entry}{百}{6}{⽩}
  \begin{Phonetics}{百}{bai3}[][HSK 1]
    \definition{adj.}{todos; todos os tipos de; multifacetados; numerosos}
    \definition{adv.}{muito; sempre}
    \definition{num.}{cem; 100}
  \end{Phonetics}
\end{Entry}

\begin{Entry}{百分}{6,4}{⽩,⼑}
  \begin{Phonetics}{百分}{bai3fen1}
    \definition{s.}{por cento | nota máxima; pontuação máxima; 100 pontos (em um sistema de classificação de cem pontos) | um jogo específico; um jogo de pôquer}
  \end{Phonetics}
\end{Entry}

\begin{Entry}{百分比}{6,4,4}{⽩,⼑,⽐}
  \begin{Phonetics}{百分比}{bai3fen1bi3}[][HSK 7-9]
    \definition{s.}{porcentagem}[按百分比计算。===Calculado como uma porcentagem.]
  \end{Phonetics}
\end{Entry}

\begin{Entry}{百分点}{6,4,9}{⽩,⼑,⽕}
  \begin{Phonetics}{百分点}{bai3fen1dian3}[][HSK 6]
    \definition[个]{s.}{ponto percentual; em estatística, um por cento é chamado de ponto percentual}
  \end{Phonetics}
\end{Entry}

\begin{Entry}{百合}{6,6}{⽩,⼝}
  \begin{Phonetics}{百合}{bai3he2}[][HSK 7-9]
    \definition[朵,束,株,枝,个,片]{s.}{lírio | bulbo de lírio}
  \end{Phonetics}
\end{Entry}

\begin{Entry}{百货}{6,8}{⽩,⾙}
  \begin{Phonetics}{百货}{bai3huo4}[][HSK 4]
    \definition{s.}{mercadorias em geral; loja de departamentos; um termo geral para bens que incluem principalmente roupas, utensílios e necessidades diárias gerais}
  \end{Phonetics}
\end{Entry}

\begin{Entry}{百科全书}{6,9,6,4}{⽩,⽲,⼊,⼄}
  \begin{Phonetics}{百科全书}{bai3ke1 quan2shu1}[][HSK 7-9]
    \definition[本,部,套,集]{s.}{enciclopédia; tesauro; \emph{thesaurus}}
  \end{Phonetics}
\end{Entry}

\begin{Entry}{百般}{6,10}{⽩,⾈}
  \begin{Phonetics}{百般}{bai3ban1}
    \definition{adv.}{de todas as maneiras possíveis | por todos os meios}
  \end{Phonetics}
\end{Entry}

%%%%%%%%%% 竹 %%%%%%%%%%
\subsection*{竹}\addcontentsline{loh}{figure}{竹}

\begin{Entry}{竹}{6}{⽵}[Kangxi 118]
  \begin{Phonetics}{竹}{zhu2}
    \definition[根]{s.}{bambu | instrumento de sopro | tira de bambu}
  \end{Phonetics}
\end{Entry}

\begin{Entry}{竹子}{6,3}{⽵,⼦}
  \begin{Phonetics}{竹子}{zhu2zi5}[][HSK 5]
    \definition[根,棵,丛,支]{s.}{bambu; nome genérico para os tipos de bambu}
  \end{Phonetics}
\end{Entry}

\begin{Entry}{竹马}{6,3}{⽵,⾺}
  \begin{Phonetics}{竹马}{zhu2ma3}
    \definition{s.}{cavalo de bambu | vara de bambu usada como cavalo de brinquedo}
  \end{Phonetics}
\end{Entry}

\begin{Entry}{竹排}{6,11}{⽵,⼿}
  \begin{Phonetics}{竹排}{zhu2pai2}
    \definition{s.}{jangada de bambu}
  \end{Phonetics}
\end{Entry}

\begin{Entry}{竹编}{6,12}{⽵,⽷}
  \begin{Phonetics}{竹编}{zhu2bian1}
    \definition{s.}{vime | tecelagem de bambu}
  \end{Phonetics}
\end{Entry}

%%%%%%%%%% 米 %%%%%%%%%%
\subsection*{米}\addcontentsline{loh}{figure}{米}

\begin{Entry}{米}{6}{⽶}[Kangxi 119]
  \begin{Phonetics}{米}{mi3}[][HSK 2,3]
    \definition*{s.}{Sobrenome: Mi}
    \definition{clas.}{m, metro; unidade principal de comprimento do sistema métrico}
    \definition[粒,斤]{s.}{arroz | sementes descascadas; refere"-se a sementes comestíveis descascadas ou sem casca | qualquer coisa que se assemelhe a um grão de arroz}
  \end{Phonetics}
\end{Entry}

\begin{Entry}{米饭}{6,7}{⽶,⾷}
  \begin{Phonetics}{米饭}{mi3fan4}[][HSK 1]
    \definition{s.}{arroz (cozido)}
  \end{Phonetics}
\end{Entry}

%%%%%%%%%% 红 %%%%%%%%%%
\subsection*{红}\addcontentsline{loh}{figure}{红}

\begin{Entry}{红}{6}{⽷}
  \begin{Phonetics}{红}{hong2}[][HSK 2]
    \definition*{s.}{Sobrenome: Hong}
    \definition{adj.}{vermelho | popular; bem-sucedido; símbolo de sucesso ou valorização | vermelho; revolucionário; símbolo da revolução | festivo; símbolo de alegria}
    \definition{s.}{tecido vermelho, bandeirinhas, etc. usados em ocasiões festivas | bônus; dividendo}
  \end{Phonetics}
\end{Entry}

\begin{Entry}{红心}{6,4}{⽷,⼼}
  \begin{Phonetics}{红心}{hong2xin1}
    \definition{s.}{coração vermelho, um coração leal à causa da revolução proletária | alvo | coração ♥ (em jogos de cartas) | red, heart-shaped symbol}
  \seealsoref{方片}{fang1 pian4}
  \seealsoref{黑桃}{hei1tao2}
  \seealsoref{梅花}{mei2hua1}
  \end{Phonetics}
\end{Entry}

\begin{Entry}{红火}{6,4}{⽷,⽕}
  \begin{Phonetics}{红火}{hong2huo5}[][HSK 7-9]
    \definition{adj.}{florescente; próspero; descreve prosperidade, prosperidade e agitação | florescente; próspero (meio de vida, carreira)}
  \end{Phonetics}
\end{Entry}

\begin{Entry}{红包}{6,5}{⽷,⼓}
  \begin{Phonetics}{红包}{hong2bao1}[][HSK 4]
    \definition[个]{s.}{saco de papel vermelho ou envelope contendo dinheiro como presente, gorjeta ou bônus | suborno; propina}
  \end{Phonetics}
\end{Entry}

\begin{Entry}{红扑扑}{6,5,5}{⽷,⼿,⼿}
  \begin{Phonetics}{红扑扑}{hong2pu1pu1}[][HSK 7-9]
    \definition{adj.}{corado | vermelho | rosado}
  \end{Phonetics}
\end{Entry}

\begin{Entry}{红灯}{6,6}{⽷,⽕}
  \begin{Phonetics}{红灯}{hong2deng1}[][HSK 7-9]
    \definition[盏]{s.}{semáforo vermelho | Figurativo: barreira; proibição | luz vermelha}
  \end{Phonetics}
\end{Entry}

\begin{Entry}{红色}{6,6}{⽷,⾊}
  \begin{Phonetics}{红色}{hong2se4}[][HSK 2]
    \definition{adj.}{vermelho; revolucionário; símbolo da revolução ou da consciência política elevada}
    \definition{s.}{cor vermelha}
  \end{Phonetics}
\end{Entry}

\begin{Entry}{红宝石}{6,8,5}{⽷,⼧,⽯}
  \begin{Phonetics}{红宝石}{hong2bao3shi2}
    \definition{s.}{rubi}
  \end{Phonetics}
\end{Entry}

\begin{Entry}{红线}{6,8}{⽷,⽷}
  \begin{Phonetics}{红线}{hong2xian4}
    \definition{s.}{linha vermelha}
  \end{Phonetics}
\end{Entry}

\begin{Entry}{红茶}{6,9}{⽷,⾋}
  \begin{Phonetics}{红茶}{hong2cha2}[][HSK 3]
    \definition[杯,壶,斤,种]{s.}{chá preto; chá acabado produzido através de fermentação completa}
  \end{Phonetics}
\end{Entry}

\begin{Entry}{红润}{6,10}{⽷,⽔}
  \begin{Phonetics}{红润}{hong2run4}[][HSK 7-9]
    \definition[张]{adj.}{avermelhado; rosado | suave, macio e rosado (pele, bochechas, etc.)}
  \end{Phonetics}
\end{Entry}

\begin{Entry}{红烧}{6,10}{⽷,⽕}
  \begin{Phonetics}{红烧}{hong2shao1}
    \definition{s.}{guisado em molho de soja (prato)}
  \end{Phonetics}
\end{Entry}

\begin{Entry}{红酒}{6,10}{⽷,⾣}
  \begin{Phonetics}{红酒}{hong2jiu3}[][HSK 3]
    \definition[瓶,杯,壶,斤,箱]{s.}{vinho tinto}
  \end{Phonetics}
\end{Entry}

\begin{Entry}{红眼}{6,11}{⽷,⽬}
  \begin{Phonetics}{红眼}{hong2yan3}[][HSK 7-9]
    \definition{s.}{conjuntivite | Figurativo: inveja; ciúme}
    \definition{v.}{ficar furioso | Dialeto: ter inveja; ter ciúmes de}
  \end{Phonetics}
\end{Entry}

\begin{Entry}{红绿灯}{6,11,6}{⽷,⽷,⽕}
  \begin{Phonetics}{红绿灯}{hong2lv4deng1}
    \definition[个]{s.}{semáforo; sinal de trânsito; os semáforos que orientam os veículos estão localizados principalmente em cruzamentos urbanos; o vermelho significa pare e o verde significa siga}
  \end{Phonetics}
\end{Entry}

\begin{Entry}{红薯}{6,16}{⽷,⾋}
  \begin{Phonetics}{红薯}{hong2shu3}[][HSK 7-9]
    \definition[个]{s.}{batata doce}
  \end{Phonetics}
\end{Entry}

%%%%%%%%%% 约 %%%%%%%%%%
\subsection*{约}\addcontentsline{loh}{figure}{约}

\begin{Entry}{约}{6}{⽷}
  \begin{Phonetics}{约}{yao1}
    \definition{adj.}{econômico; frugal | simples; breve | indistinto}
    \definition{adv.}{cerca de; ao redor; aproximadamente}
    \definition{s.}{pacto; acordo; nomeação; coisa prometida}
    \definition{v.}{marcar uma consulta; organizar | perguntar ou convidar com antecedência | restringir; conter | reduzir (fração aproximada)}
  \end{Phonetics}
  \begin{Phonetics}{约}{yue1}[][HSK 3]
    \definition*{s.}{Sobrenome: Yue}
    \definition{adj.}{econômico; frugal | simples; breve; resumido | indistinto; confuso}
    \definition{adv.}{cerca de; ao redor; aproximadamente}
    \definition{s.}{pacto; acordo; nomeação; o que foi combinado}
    \definition{v.}{combinar; propor ou discutir antecipadamente (o que deve ser respeitado por todos) | convidar com antecedência | restringir; conter | reduzir (fração aproximada)}
  \end{Phonetics}
\end{Entry}

\begin{Entry}{约会}{6,6}{⽷,⼈}
  \begin{Phonetics}{约会}{yue1hui5}[][HSK 4]
    \definition[个,次]{s.}{data; compromisso; engajamento; reunião pré-agendada}
    \definition{v.}{marcar uma reunião; marcar um encontro}
  \end{Phonetics}
\end{Entry}

\begin{Entry}{约束}{6,7}{⽷,⽊}
  \begin{Phonetics}{约束}{yue1shu4}[][HSK 5]
    \definition{adj.}{amarrado}
    \definition{s.}{restrição; constrangimento; engajamento}
    \definition{v.}{amarrar; prender; reprimir; restringir; manter dentro de si}
  \end{Phonetics}
\end{Entry}

\begin{Entry}{约定}{6,8}{⽷,⼧}
  \begin{Phonetics}{约定}{yue1ding4}[][HSK 6]
    \definition{s.}{acordo; compromisso}
    \definition{v.}{determinar; chegar a um acordo; concordar com}
  \end{Phonetics}
\end{Entry}

%%%%%%%%%% 级 %%%%%%%%%%
\subsection*{级}\addcontentsline{loh}{figure}{级}

\begin{Entry}{级}{6}{⽷}
  \begin{Phonetics}{级}{ji2}[][HSK 2]
    \definition{clas.}{usado para degraus, escadas, pisos de torres, etc.}
    \definition[个,种]{s.}{nível; classificação; grau; classe | série; turma; qualquer uma das divisões anuais de um curso escolar | degrau}
  \end{Phonetics}
\end{Entry}

\begin{Entry}{级别}{6,7}{⽷,⼑}
  \begin{Phonetics}{级别}{ji2bie2}[][HSK 7-9]
    \definition[个]{s.}{classificação; nível; escala; a ordem da hierarquia devido às diferenças de identidade, \emph{status} e características}
  \end{Phonetics}
\end{Entry}

%%%%%%%%%% 纪 %%%%%%%%%%
\subsection*{纪}\addcontentsline{loh}{figure}{纪}

\begin{Entry}{纪}{6}{⽷}
  \begin{Phonetics}{纪}{ji3}
    \definition*{s.}{Sobrenome: Ji}
    \definition{s.}{disciplina | um período de doze anos (na China antiga); um período de anos | (geologia) subdivisão de uma era geológica; período}
    \definition{v.}{colocar por escrito; registrar; mesmo significado de 记, usado principalmente em 记录, 纪年, 纪元, 纪传, etc. | classificar (fios de seda)}
  \seealsoref{记}{ji4}
  \seealsoref{纪传}{ji4 zhuan4}
  \seealsoref{记录}{ji4lu4}
  \seealsoref{纪年}{ji4nian2}
  \seealsoref{纪元}{ji4yuan2}
  \end{Phonetics}
  \begin{Phonetics}{纪}{ji4}
    \definition*{s.}{Sobrenome: Ji}
    \definition{s.}{disciplina | idade; época | Geologia: período | um período de doze anos (na China antiga); um período de anos | Geologia: subdivisão de uma era geológica}
    \definition{v.}{colocar por escrito; registrar | registrar, mesmo significado de 记, usado principalmente em 记录, 纪年, 纪元, 纪传, etc. | classificar (fios de seda)}
  \seealsoref{记}{ji4}
  \seealsoref{纪传}{ji4 zhuan4}
  \seealsoref{记录}{ji4lu4}
  \seealsoref{纪年}{ji4nian2}
  \seealsoref{纪元}{ji4yuan2}
  \end{Phonetics}
\end{Entry}

\begin{Entry}{纪元}{6,4}{⽷,⼉}
  \begin{Phonetics}{纪元}{ji4yuan2}
    \definition{s.}{o início de uma era (por exemplo, o reinado de um imperador) | época; era}
  \end{Phonetics}
\end{Entry}

\begin{Entry}{纪传}{6,6}{⽷,⼈}
  \begin{Phonetics}{纪传}{ji4 zhuan4}
    \definition{s.}{crônica; biografia}
  \end{Phonetics}
\end{Entry}

\begin{Entry}{纪传体}{6,6,7}{⽷,⼈,⼈}
  \begin{Phonetics}{纪传体}{ji4 zhuan4 ti3}
    \definition{s.}{história apresentada em uma série de biografias | gênero histórico baseado em biografia}
  \end{Phonetics}
\end{Entry}

\begin{Entry}{纪年}{6,6}{⽷,⼲}
  \begin{Phonetics}{纪年}{ji4nian2}
    \definition{s.}{cronologia; uma maneira de numerar os anos | registro cronológico de eventos; anais; um dos gêneros de livros históricos é organizar fatos históricos em ordem cronológica}
  \end{Phonetics}
\end{Entry}

\begin{Entry}{纪实}{6,8}{⽷,⼧}
  \begin{Phonetics}{纪实}{ji4shi2}[][HSK 7-9]
    \definition{s.}{registro de eventos reais; relatório no local; cobertura ao vivo de eventos ou incidentes}
  \end{Phonetics}
\end{Entry}

\begin{Entry}{纪录}{6,8}{⽷,⼹}
  \begin{Phonetics}{纪录}{ji4lu4}[][HSK 3]
    \definition[项,个]{s.}{recorde (esportes); o número mais alto ou mais baixo registrado em um determinado período de tempo}
  \end{Phonetics}
\end{Entry}

\begin{Entry}{纪录片}{6,8,4}{⽷,⼹,⽚}
  \begin{Phonetics}{纪录片}{ji4lu4pian4}[][HSK 7-9]
    \definition[个,部]{s.}{documentário; filme documentário}
  \end{Phonetics}
\end{Entry}

\begin{Entry}{纪念}{6,8}{⽷,⼼}
  \begin{Phonetics}{纪念}{ji4nian4}[][HSK 3]
    \definition[个,次]{s.}{lembrança; recordação; usado para representar uma lembrança (objeto)}
    \definition{v.}{comemorar; expressar saudade por pessoas ou coisas através de objetos ou ações}
  \end{Phonetics}
\end{Entry}

\begin{Entry}{纪念日}{6,8,4}{⽷,⼼,⽇}
  \begin{Phonetics}{纪念日}{ji4nian4ri4}[][HSK 7-9]
    \definition{s.}{dia de comemoração; um dia memorável}
  \end{Phonetics}
\end{Entry}

\begin{Entry}{纪念馆}{6,8,11}{⽷,⼼,⾷}
  \begin{Phonetics}{纪念馆}{ji4nian4guan3}[][HSK 7-9]
    \definition{s.}{salão memorial; museu em memória de alguém | museu em memória de\dots}
  \end{Phonetics}
\end{Entry}

\begin{Entry}{纪念碑}{6,8,13}{⽷,⼼,⽯}
  \begin{Phonetics}{纪念碑}{ji4nian4bei1}[][HSK 7-9]
    \definition{s.}{monumento; memorial | cenotáfio; placa memorial}
  \end{Phonetics}
\end{Entry}

\begin{Entry}{纪律}{6,9}{⽷,⼻}
  \begin{Phonetics}{纪律}{ji4lv4}[][HSK 4]
    \definition{s.}{disciplina; código de conduta que cada membro da vida coletiva deve observar}
  \end{Phonetics}
\end{Entry}

%%%%%%%%%% 网 %%%%%%%%%%
\subsection*{网}\addcontentsline{loh}{figure}{网}

\begin{Entry}{网}{6}{⽹}[Kangxi 122]
  \begin{Phonetics}{网}{wang3}[][HSK 2]
    \definition[张]{s.}{rede; um dispositivo feito de corda ou barbante para capturar peixes e pássaros | algo que parece uma rede | rede; uma rede de organizações; um sistema}
    \definition{v.}{pegar com uma rede | cobrir como com uma rede}
  \end{Phonetics}
\end{Entry}

\begin{Entry}{网上}{6,3}{⽹,⼀}
  \begin{Phonetics}{网上}{wang3shang4}[][HSK 1]
    \definition{s.}{\emph{online}; refere"-se a acessar a \emph{Internet} através de um computador ou celular para pesquisar e consultar informações na rede}
  \end{Phonetics}
\end{Entry}

\begin{Entry}{网上银行}{6,3,11,6}{⽹,⼀,⾦,⾏}
  \begin{Phonetics}{网上银行}{wang3shang4yin2hang2}
    \definition[个]{s.}{banco \emph{online} | acesso a operações bancárias via \emph{Internet}}
  \seealsoref{网银}{wang3yin2}
  \end{Phonetics}
\end{Entry}

\begin{Entry}{网友}{6,4}{⽹,⼜}
  \begin{Phonetics}{网友}{wang3you3}[][HSK 1]
    \definition{s.}{internauta; usuário da \emph{Internet}; amigos que se conhecem pela Internet; também usado como forma de tratamento entre internautas}
  \end{Phonetics}
\end{Entry}

\begin{Entry}{网民}{6,5}{⽹,⽒}
  \begin{Phonetics}{网民}{wang3min2}[][HSK 7-9]
    \definition[位,名]{s.}{internauta; geralmente se refere a um usuário de rede de computadores}
  \end{Phonetics}
\end{Entry}

\begin{Entry}{网页}{6,6}{⽹,⾴}
  \begin{Phonetics}{网页}{wang3ye4}[][HSK 6]
    \definition[个]{s.}{site; página da web; \emph{website}; \emph{web page}}
  \end{Phonetics}
\end{Entry}

\begin{Entry}{网吧}{6,7}{⽹,⼝}
  \begin{Phonetics}{网吧}{wang3ba1}[][HSK 6]
    \definition[家,间]{s.}{cybercafé; \emph{Internet} café; refere"-se a um local comercial aberto ao público que utiliza redes de computadores para fornecer serviços de navegação, consulta e outras informações}
  \end{Phonetics}
\end{Entry}

\begin{Entry}{网址}{6,7}{⽹,⼟}
  \begin{Phonetics}{网址}{wang3zhi3}[][HSK 4]
    \definition[个]{s.}{\emph{website}; endereço da \emph{web}; endereço de um \emph{site} na \emph{Internet}, que os usuários podem acessar, consultar e obter recursos de informações nesse \emph{site} clicando nele}
  \end{Phonetics}
\end{Entry}

\begin{Entry}{网际网络}{6,7,6,9}{⽹,⾩,⽹,⽷}
  \begin{Phonetics}{网际网络}{wang3 ji4 wang3 luo4}
    \definition*{s.}{Internet}
  \seealsoref{互联网}{hu4lian2wang3}
  \seealsoref{网际网路}{wang3 ji4 wang3 lu4}
  \seealsoref{网路}{wang3 lu4}
  \end{Phonetics}
\end{Entry}

\begin{Entry}{网际网路}{6,7,6,13}{⽹,⾩,⽹,⾜}
  \begin{Phonetics}{网际网路}{wang3 ji4 wang3 lu4}
    \definition*{s.}{Internet}
  \seealsoref{互联网}{hu4lian2wang3}
  \seealsoref{网际网络}{wang3 ji4 wang3 luo4}
  \seealsoref{网路}{wang3 lu4}
  \end{Phonetics}
\end{Entry}

\begin{Entry}{网点}{6,9}{⽹,⽕}
  \begin{Phonetics}{网点}{wang3dian3}[][HSK 7-9]
    \definition{s.}{pontos de venda; rede (de estabelecimento comercial)}
  \end{Phonetics}
\end{Entry}

\begin{Entry}{网络}{6,9}{⽹,⽷}
  \begin{Phonetics}{网络}{wang3luo4}[][HSK 4]
    \definition{s.}{rede; um sistema que consiste em ramificações interconectadas; em um sistema elétrico, um circuito ou parte de um circuito que consiste em vários elementos que permitem a transmissão de sinais elétricos de acordo com determinados requisitos | rede; rede de computadores}
  \end{Phonetics}
\end{Entry}

\begin{Entry}{网站}{6,10}{⽹,⽴}
  \begin{Phonetics}{网站}{wang3zhan4}[][HSK 2]
    \definition[个,家]{s.}{\emph{web}; \emph{website}; um site virtual na Internet para uma organização ou indivíduo, geralmente consistindo em uma página inicial e muitas páginas da web}
  \end{Phonetics}
\end{Entry}

\begin{Entry}{网罟}{6,10}{⽹,⽹}
  \begin{Phonetics}{网罟}{wang3gu3}
    \definition{s.}{(fig.) a rede da justiça | rede usada para capturar peixes (ou outros animais, como pássaros)}
  \end{Phonetics}
\end{Entry}

\begin{Entry}{网球}{6,11}{⽹,⽟}
  \begin{Phonetics}{网球}{wang3qiu2}[][HSK 2]
    \definition[个,颗,些]{s.}{tênis (esporte) | bola de tênis}
  \end{Phonetics}
\end{Entry}

\begin{Entry}{网银}{6,11}{⽹,⾦}
  \begin{Phonetics}{网银}{wang3yin2}
    \definition{s.}{banco \emph{online} | acesso a operações bancárias via \emph{Internet}}
  \seealsoref{网上银行}{wang3shang4yin2hang2}
  \end{Phonetics}
\end{Entry}

\begin{Entry}{网路}{6,13}{⽹,⾜}
  \begin{Phonetics}{网路}{wang3 lu4}
    \definition*{s.}{Internet}
  \seealsoref{互联网}{hu4lian2wang3}
  \seealsoref{网际网路}{wang3 ji4 wang3 lu4}
  \seealsoref{网际网络}{wang3 ji4 wang3 luo4}
  \end{Phonetics}
\end{Entry}

%%%%%%%%%% 羊 %%%%%%%%%%
\subsection*{羊}\addcontentsline{loh}{figure}{羊}

\begin{Entry}{羊}{6}{⽺}[Kangxi 123]
  \begin{Phonetics}{羊}{yang2}[][HSK 3]
    \definition*{s.}{Sobrenome: Yang}
    \definition[只,头,群]{s.}{carneiro; ovelha; bode; cabra; antílope}
  \end{Phonetics}
\end{Entry}

%%%%%%%%%% 羽 %%%%%%%%%%
\subsection*{羽}\addcontentsline{loh}{figure}{羽}

\begin{Entry}{羽}{6}{⽻}[Kangxi 124]
  \begin{Phonetics}{羽}{yu3}
    \definition*{s.}{Sobrenome: Yu}
    \definition{s.}{pena; pluma | asas (de pássaros ou insetos) | uma nota da antiga escala chinesa de cinco tons, correspondente a 6 na notação musical numerada}
  \end{Phonetics}
\end{Entry}

\begin{Entry}{羽毛}{6,4}{⽻,⽑}
  \begin{Phonetics}{羽毛}{yu3mao2}
    \definition{s.}{pena | plumagem | pluma}
  \end{Phonetics}
\end{Entry}

\begin{Entry}{羽毛笔}{6,4,10}{⽻,⽑,⽵}
  \begin{Phonetics}{羽毛笔}{yu3mao2bi3}
    \definition{s.}{caneta de pena}
  \end{Phonetics}
\end{Entry}

\begin{Entry}{羽毛球}{6,4,11}{⽻,⽑,⽟}
  \begin{Phonetics}{羽毛球}{yu3mao2qiu2}[][HSK 5]
    \definition[只,个]{s.}{\emph{badminton}; esporte com bola, as regras e equipamentos são bastante semelhantes ao tênis | peteca}
  \end{Phonetics}
\end{Entry}

\begin{Entry}{羽林}{6,8}{⽻,⽊}
  \begin{Phonetics}{羽林}{yu3lin2}
    \definition{s.}{escolta armada}
  \end{Phonetics}
\end{Entry}

\begin{Entry}{羽冠}{6,9}{⽻,⼍}
  \begin{Phonetics}{羽冠}{yu3guan1}
    \definition{s.}{crista emplumada (de pássaro)}
  \end{Phonetics}
\end{Entry}

\begin{Entry}{羽绒服}{6,9,8}{⽻,⽷,⽉}
  \begin{Phonetics}{羽绒服}{yu3rong2fu2}[][HSK 5]
    \definition[件,个]{s.}{jaqueta de plumas; peça de vestuário com enchimento de plumas; casaco cujo interior é preenchido com penas de pato ou ganso}
  \end{Phonetics}
\end{Entry}

\begin{Entry}{羽流}{6,10}{⽻,⽔}
  \begin{Phonetics}{羽流}{yu3liu2}
    \definition{s.}{pluma}
  \end{Phonetics}
\end{Entry}

%%%%%%%%%% 老 %%%%%%%%%%
\subsection*{老}\addcontentsline{loh}{figure}{老}

\begin{Entry}{老}{6}{⽼}[Kangxi 125]
  \begin{Phonetics}{老}{lao3}[][HSK 1,2]
    \definition*{s.}{Sobrenome: Lao}
    \definition{adj.}{velho; envelhecido; idade avançada | antigo; de longa data; existe há muito tempo | antigo; desatualizado; obsoleto; ultrapassado  | antigo; tradicional; original | coberto de vegetação; os vegetais cresceram além do período ideal para serem consumidos | resistente; endurecido; alimentos muito cozidos | (sobre cores) escuro; profundo | último nascido; o mais novo | veterano; experiente; sofisticado}
    \definition{adv.}{longo; por muito tempo | sempre (fazendo algo) | muito}
    \definition{pref.}{usado para designar pessoas, ordem de classificação, certos nomes de animais e plantas}
    \definition{s.}{idosos; pessoas mais velhas | ancião; sênior; um título respeitoso para pessoas mais velhas}
    \definition{v.}{envelhecer | morrer; referindo"-se à morte de um idoso}
  \end{Phonetics}
\end{Entry}

\begin{Entry}{老人}{6,2}{⽼,⼈}
  \begin{Phonetics}{老人}{lao3ren2}[][HSK 1]
    \definition[位]{s.}{homem ou mulher de idade avançada; o idoso; o velho}
  \end{Phonetics}
\end{Entry}

\begin{Entry}{老人家}{6,2,10}{⽼,⼈,⼧}
  \begin{Phonetics}{老人家}{lao3ren5jia5}[][HSK 7-9]
    \definition[位,名,个]{s.}{avô; avó; pessoa idosa venerável; um título respeitoso para os idosos | maneira de chamar o pai ou a mãe idosos na frente dos outros; referir-se aos próprios pais ou aos pais, professores, etc. de outras pessoas}
  \end{Phonetics}
\end{Entry}

\begin{Entry}{老乡}{6,3}{⽼,⼄}
  \begin{Phonetics}{老乡}{lao3xiang1}[][HSK 6]
    \definition[个,位]{s.}{conterrâneo; conterrâneo | uma maneira de chamar um fazendeiro cujo nome você não conhece}
  \end{Phonetics}
\end{Entry}

\begin{Entry}{老大}{6,3}{⽼,⼤}
  \begin{Phonetics}{老大}{lao3da4}[][HSK 7-9]
    \definition{adj.}{velho; mais velho}
    \definition{adv.}{muito; extremamente (encontrado com mais frequência no chinês vernáculo antigo)}
    \definition{s.}{filho mais velho; o mais velho entre os irmãos | capitão de um barco; mestre de uma embarcação à vela; o barqueiro principal do barco de madeira, e também um termo genérico para barqueiros | líder; chefe; o título que algumas gangues ou grupos do crime organizado usam para seus líderes}
  \end{Phonetics}
\end{Entry}

\begin{Entry}{老公}{6,4}{⽼,⼋}
  \begin{Phonetics}{老公}{lao3gong5}[][HSK 4]
    \definition[个,位,名]{s.}{marido; esposo}
  \end{Phonetics}
\end{Entry}

\begin{Entry}{老化}{6,4}{⽼,⼔}
  \begin{Phonetics}{老化}{lao3hua4}[][HSK 7-9]
    \definition{s.}{Química: envelhecer | (pessoas) envelhecer; ficar velho | (conhecimento, etc.) tornam-se obsoletos}
    \definition{s.}{envelhecimento; período de amadurecimento; pré-queima; maturação}
  \end{Phonetics}
\end{Entry}

\begin{Entry}{老太太}{6,4,4}{⽼,⼤,⼤}
  \begin{Phonetics}{老太太}{lao3tai4tai5}[][HSK 3]
    \definition[位,名,个]{s.}{velha senhora; (em tratamento direto) Venerável Senhora; uma maneira respeitosa de chamar uma senhora idosa; título honorífico para mulheres idosas | (forma de tratamento) sua velha mãe; minha velha mãe, avó ou sogra; referindo"-se à própria mãe, à mãe do outro ou à mãe de outra pessoa, à sogra ou à sogra política}
  \end{Phonetics}
\end{Entry}

\begin{Entry}{老头儿}{6,5,2}{⽼,⼤,⼉}
  \begin{Phonetics}{老头儿}{lao3tou2r5}[][HSK 3]
    \definition{s.}{(coloquial) (com um tom de intimidade) velho; velho amigo}
  \seealsoref{老头子}{lao3 tou2zi5}
  \end{Phonetics}
\end{Entry}

\begin{Entry}{老头子}{6,5,3}{⽼,⼤,⼦}
  \begin{Phonetics}{老头子}{lao3 tou2zi5}
    \definition{s.}{velho antiquado (ou velho rabugento) | (referindo"-se ao marido idoso) meu velho | chefe de uma sociedade secreta | Coloquial: velho; velho rabugento}
  \seealsoref{老头儿}{lao3tou2r5}
  \end{Phonetics}
\end{Entry}

\begin{Entry}{老汉}{6,5}{⽼,⽔}
  \begin{Phonetics}{老汉}{lao3han4}[][HSK 7-9]
    \definition{pron.}{(usado por um homem idoso) eu, eu mesmo}
    \definition[个,位]{s.}{homem velho; homens idosos; um velho | um velho como eu; utilizado para autorreferência}
  \end{Phonetics}
\end{Entry}

\begin{Entry}{老字号}{6,6,5}{⽼,⼦,⼝}
  \begin{Phonetics}{老字号}{lao3zi4hao5}[][HSK 7-9]
    \definition[家]{s.}{nome antigo no ramo comercial; empresa (ou loja) de longa data; empreendimento (ou loja) antigo e famoso | loja, empresa ou marca de mercadorias com reputação consolidada; lojas tradicionais}
  \end{Phonetics}
\end{Entry}

\begin{Entry}{老师}{6,6}{⽼,⼱}
  \begin{Phonetics}{老师}{lao3shi1}[][HSK 1]
    \definition[个,位]{s.}{professor; título honorífico para professores; refere"-se, de maneira geral, a pessoas que transmitem cultura e tecnologia ou que são dignas de admiração em termos de ideias, moralidade e conhecimentos profissionais}
  \end{Phonetics}
\end{Entry}

\begin{Entry}{老年}{6,6}{⽼,⼲}
  \begin{Phonetics}{老年}{lao3nian2}[][HSK 2]
    \definition[个]{s.}{idoso; velhice; idade acima de 60 ou 70 anos}
  \end{Phonetics}
\end{Entry}

\begin{Entry}{老百姓}{6,6,8}{⽼,⽩,⼥}
  \begin{Phonetics}{老百姓}{lao3bai3xing4}[][HSK 3]
    \definition[些]{s.}{povo; civis; pessoas comuns; residentes (em contraste com militares e funcionários públicos)}
  \end{Phonetics}
\end{Entry}

\begin{Entry}{老伴}{6,7}{⽼,⼈}
  \begin{Phonetics}{老伴}{lao3ban4}
    \definition[个]{s.}{marido ou esposa (de um casal de idosos)}
  \seealsoref{老伴儿}{lao3ban4r5}
  \end{Phonetics}
\end{Entry}

\begin{Entry}{老伴儿}{6,7,2}{⽼,⼈,⼉}
  \begin{Phonetics}{老伴儿}{lao3ban4r5}[][HSK 7-9]
    \definition{s.}{marido ou esposa}
  \seealsoref{老伴}{lao3ban4}
  \end{Phonetics}
\end{Entry}

\begin{Entry}{老兵}{6,7}{⽼,⼋}
  \begin{Phonetics}{老兵}{lao3bing1}
    \definition{s.}{velho soldado | veterano de guerra | veterano (alguém que tem muita experiência em algum domínio)}
  \end{Phonetics}
\end{Entry}

\begin{Entry}{老远}{6,7}{⽼,⾡}
  \begin{Phonetics}{老远}{lao3 yuan3}[][HSK 7-9]
    \definition{adj.}{muito longe; longínquo; de grande distância}
  \end{Phonetics}
\end{Entry}

\begin{Entry}{老实}{6,8}{⽼,⼧}
  \begin{Phonetics}{老实}{lao3shi5}[][HSK 4]
    \definition{adj.}{franco; sincero; honesto | bom; bem-comportado | ingênuo; simplório; meio bobo; facilmente enganado; eufemismo para pouco inteligente}
  \end{Phonetics}
\end{Entry}

\begin{Entry}{老实说}{6,8,9}{⽼,⼧,⾔}
  \begin{Phonetics}{老实说}{lao3shi5shuo1}[][HSK 7-9]
    \definition{expr.}{``Honestamente.''; ser honesto; ser franco; falando francamente; sinceramente, muitas vezes funciona como uma interjeição em uma frase para indicar ênfase}
  \end{Phonetics}
\end{Entry}

\begin{Entry}{老朋友}{6,8,4}{⽼,⽉,⼜}
  \begin{Phonetics}{老朋友}{lao3peng2you3}[][HSK 2]
    \definition[个,位,名]{s.}{velho amigo; refere"-se a amigos que conhecemos há muito tempo e com quem temos uma relação íntima}
  \end{Phonetics}
\end{Entry}

\begin{Entry}{老板}{6,8}{⽼,⽊}
  \begin{Phonetics}{老板}{lao3ban3}[][HSK 3]
    \definition[个,位]{s.}{chefe; dono; líder; refere"-se ao gerente de uma empresa comercial ou industrial | antigo título honorífico dado a atores famosos de ópera ou atores que também eram diretores de companhias de ópera}
  \end{Phonetics}
\end{Entry}

\begin{Entry}{老虎}{6,8}{⽼,⾌}
  \begin{Phonetics}{老虎}{lao3hu3}
    \definition[只]{s.}{tigre}
  \seealsoref{虎}{hu3}
  \end{Phonetics}
\end{Entry}

\begin{Entry}{老是}{6,9}{⽼,⽇}
  \begin{Phonetics}{老是}{lao3shi4}[][HSK 2]
    \definition{adv.}{sempre; indica que a ação continua ou que o estado permanece inalterado, equivalente a 一直}
  \seealsoref{一直}{yi4zhi2}
  \end{Phonetics}
\end{Entry}

\begin{Entry}{老家}{6,10}{⽼,⼧}
  \begin{Phonetics}{老家}{lao3jia1}[][HSK 4]
    \definition{s.}{cidade natal; local de origem | lugar nativo; refere"-se às gerações anteriores da família ou ao local onde a pessoa nasceu ou viveu}
  \end{Phonetics}
\end{Entry}

\begin{Entry}{老婆}{6,11}{⽼,⼥}
  \begin{Phonetics}{老婆}{lao3po5}[][HSK 4]
    \definition[个,位,名]{s.}{esposa}
  \end{Phonetics}
\end{Entry}

%%%%%%%%%% 考 %%%%%%%%%%
\subsection*{考}\addcontentsline{loh}{figure}{考}

\begin{Entry}{考}{6}{⽼}
  \begin{Phonetics}{考}{kao3}[][HSK 1]
    \definition*{s.}{Sobrenome: Kao}
    \definition{adj.}{antigo; velho; com idade avançada}
    \definition{s.}{o pai falecido de alguém}
    \definition{v.}{examinar; dar (fazer) um exame, teste ou questionário | verificar; inspecionar | estudar; verificar; investigar | perguntar; testar; fazer perguntas para que o outro responda, a fim de testar suas habilidades em determinada área}
  \end{Phonetics}
\end{Entry}

\begin{Entry}{考生}{6,5}{⽼,⽣}
  \begin{Phonetics}{考生}{kao3sheng1}[][HSK 2]
    \definition{s.}{candidato a exame; alunos inscritos para o exame de admissão}
  \end{Phonetics}
\end{Entry}

\begin{Entry}{考场}{6,6}{⽼,⼟}
  \begin{Phonetics}{考场}{kao3chang3}[][HSK 6]
    \definition{s.}{sala de exames}
  \end{Phonetics}
\end{Entry}

\begin{Entry}{考试}{6,8}{⽼,⾔}
  \begin{Phonetics}{考试}{kao3/shi4}[][HSK 1]
    \definition[次]{s.}{teste; exame; prova; atividades realizadas para verificar conhecimentos ou habilidades}
    \definition{v.+compl.}{testar; avaliar; avaliar conhecimentos e habilidades por meio de perguntas escritas ou orais.}
  \end{Phonetics}
\end{Entry}

\begin{Entry}{考核}{6,10}{⽼,⽊}
  \begin{Phonetics}{考核}{kao3he2}[][HSK 5]
    \definition{v.}{examinar; checar; avaliar; avaliar (a proficiência de alguém)}
  \end{Phonetics}
\end{Entry}

\begin{Entry}{考虑}{6,10}{⽼,⾌}
  \begin{Phonetics}{考虑}{kao3lv4}[][HSK 4]
    \definition{v.}{considerar; refletir sobre; levar em conta}
  \end{Phonetics}
\end{Entry}

\begin{Entry}{考验}{6,10}{⽼,⾺}
  \begin{Phonetics}{考验}{kao3yan4}[][HSK 3]
    \definition[场,个,种]{s.}{teste; julgamento; atividade realizada para verificar se as habilidades, ideias, moral e qualidades de uma pessoa atendem aos requisitos}
    \definition{v.}{testar; testar as capacidades, ideias, moral e qualidades de uma pessoa através de situações, ações ou ambientes difíceis, para verificar se elas atendem aos requisitos}
  \end{Phonetics}
\end{Entry}

\begin{Entry}{考量}{6,12}{⽼,⾥}
  \begin{Phonetics}{考量}{kao3liang2}[][HSK 7-9]
    \definition{v.}{considerar; examinar e medir}
  \end{Phonetics}
\end{Entry}

\begin{Entry}{考察}{6,14}{⽼,⼧}
  \begin{Phonetics}{考察}{kao3cha2}[][HSK 4]
    \definition{v.}{inspecionar; investigar; observar e estudar}
  \end{Phonetics}
\end{Entry}

\begin{Entry}{考题}{6,15}{⽼,⾴}
  \begin{Phonetics}{考题}{kao3ti2}[][HSK 6]
    \definition{s.}{questões de exame; prova de exame; tópicos de exame}
  \end{Phonetics}
\end{Entry}

%%%%%%%%%% 而 %%%%%%%%%%
\subsection*{而}\addcontentsline{loh}{figure}{而}

\begin{Entry}{而}{6}{⽽}[Kangxi 126]
  \begin{Phonetics}{而}{er2}[][HSK 4]
    \definition{conj.}{e (coordenação) | e ainda (restrição) | conexão de componentes com continuidade semântica | conexão de componentes afirmativos e negativos que se complementam | conexão de componentes com significados opostos para indicar um contraste | conexão de componentes de causa e efeito no raciocínio | significa ``chegar'' ou ``alcançar'' | conexão de componentes que indicam tempo ou modo ao verbo | inserido entre o sujeito e o predicado, significa 如果}
  \seealsoref{如果}{ru2guo3}
  \end{Phonetics}
\end{Entry}

\begin{Entry}{而已}{6,3}{⽽,⼰}
  \begin{Phonetics}{而已}{er2yi3}[][HSK 7-9]
    \definition{part.}{isso é tudo; nada mais; usado no final de uma frase declarativa, geralmente é precedido por 不过 ou 只 para expressar que é exatamente assim (罢了)}
  \seealsoref{罢了}{ba4le5}
  \seealsoref{不过}{bu2guo4}
  \seealsoref{只}{zhi3}
  \end{Phonetics}
\end{Entry}

\begin{Entry}{而且}{6,5}{⽽,⼀}
  \begin{Phonetics}{而且}{er2qie3}[][HSK 2]
    \definition{conj.}{e também; indica igualdade | e isso; não só\dots mas (também); indica um passo adiante}
  \end{Phonetics}
\end{Entry}

\begin{Entry}{而况}{6,7}{⽽,⼎}
  \begin{Phonetics}{而况}{er2kuang4}
    \definition{conj.}{além disso | além do mais}
  \end{Phonetics}
\end{Entry}

\begin{Entry}{而是}{6,9}{⽽,⽇}
  \begin{Phonetics}{而是}{er2shi4}[][HSK 4]
    \definition{conj.}{mas; em vez disso; geralmente usada em conjunto com 不是 para formar o correlativo 不是……而是, indicando uma relação paralela}
  \seealsoref{不是……而是}{bu4shi4 er2shi4}
  \end{Phonetics}
\end{Entry}

%%%%%%%%%% 耳 %%%%%%%%%%
\subsection*{耳}\addcontentsline{loh}{figure}{耳}

\begin{Entry}{耳}{6}{⽿}[Kangxi 128]
  \begin{Phonetics}{耳}{er3}
    \definition*{s.}{Sobrenome: Er}
    \definition{part.}{(clássico) somente; apenas}
    \definition{s.}{orelha | coisa parecida com uma orelha | em ambos os lados; lado | orelha de um utensílio}
  \end{Phonetics}
\end{Entry}

\begin{Entry}{耳目一新}{6,5,1,13}{⽿,⽬,⼀,⽄}
  \begin{Phonetics}{耳目一新}{er3mu4-yi4xin1}[][HSK 7-9]
    \definition{expr.}{encontrar tudo fresco e novo; encontrar-se em um mundo inteiramente novo; apresentar uma nova aparência (de um lugar); uma mudança agradável de atmosfera; ``Tudo o que ouço e vejo mudou e parece novo.''}
  \end{Phonetics}
\end{Entry}

\begin{Entry}{耳光}{6,6}{⽿,⼉}
  \begin{Phonetics}{耳光}{er3guang1}[][HSK 7-9]
    \definition[个,记]{s.}{uma bofetada na orelha; um tapa na cara; (bater) no rosto em frente à orelha; a ação de bater no rosto}
  \end{Phonetics}
\end{Entry}

\begin{Entry}{耳朵}{6,6}{⽿,⽊}
  \begin{Phonetics}{耳朵}{er3duo5}[][HSK 5]
    \definition[双,只,个,对]{s.}{orelha; ouvido; órgão da audição e do equilíbrio}
  \end{Phonetics}
\end{Entry}

\begin{Entry}{耳机}{6,6}{⽿,⽊}
  \begin{Phonetics}{耳机}{er3ji1}[][HSK 4]
    \definition[副,个,对]{s.}{fone de ouvido; receptor (de telefone); dispositivos que permitem que uma pessoa ouça sons sozinha, como ouvir música, histórias, chamadas telefônicas etc., usados na cabeça ou inseridos nos ouvidos}
  \end{Phonetics}
\end{Entry}

\begin{Entry}{耳闻目睹}{6,9,5,13}{⽿,⾨,⽬,⽬}
  \begin{Phonetics}{耳闻目睹}{er3wen2-mu4du3}[][HSK 7-9]
    \definition{expr.}{testemunhar pessoalmente; ver e ouvir pessoalmente; o que se vê e se ouve}
  \end{Phonetics}
\end{Entry}

\begin{Entry}{耳熟能详}{6,15,10,8}{⽿,⽕,⾁,⾔}
  \begin{Phonetics}{耳熟能详}{er3shu2-neng2xiang2}[][HSK 7-9]
    \definition{expr.}{o que é ouvido com frequência pode ser repetido em detalhes; já ouvi isso muitas vezes e estou familiarizado o suficiente para falar sobre isso em detalhes}
  \end{Phonetics}
\end{Entry}

%%%%%%%%%% 肉 %%%%%%%%%%
\subsection*{肉}\addcontentsline{loh}{figure}{肉}

\begin{Entry}{肉}{6}{⾁}[Kangxi 130]
  \begin{Phonetics}{肉}{rou4}[][HSK 1]
    \definition{adj.}{não crocante; mole | lento (em movimento); preguiçoso | carnal; erótico}
    \definition[块]{s.}{carne (especialmente carne de porco) | carne | polpa (da fruta)}
  \end{Phonetics}
\end{Entry}

\begin{Entry}{肉桂}{6,10}{⾁,⽊}
  \begin{Phonetics}{肉桂}{rou4gui4}
    \definition{s.}{canela (árvore) | casca seca desta árvore; canela (uma especiaria aromática) | canela chinesa; cássia}
  \seealsoref{官桂}{guan1gui4}
  \end{Phonetics}
\end{Entry}

%%%%%%%%%% 肌 %%%%%%%%%%
\subsection*{肌}\addcontentsline{loh}{figure}{肌}

\begin{Entry}{肌}{6}{⾁}
  \begin{Phonetics}{肌}{ji1}
    \definition[块,片]{s.}{músculo; carne | pele;}
  \end{Phonetics}
\end{Entry}

\begin{Entry}{肌肉}{6,6}{⾁,⾁}
  \begin{Phonetics}{肌肉}{ji1rou4}[][HSK 5]
    \definition[身,块]{s.}{músculo; um dos tecidos básicos dos músculos humanos e de alguns animais, composto principalmente de células musculares fibrosas, pode se contrair, é o movimento do corpo e o corpo de digestão, respiração, circulação, excreção e outros processos fisiológicos da fonte de energia; pode ser dividido em três tipos: músculo liso, músculo esquelético e músculo cardíaco}
  \end{Phonetics}
\end{Entry}

\begin{Entry}{肌肤}{6,8}{⾁,⾁}
  \begin{Phonetics}{肌肤}{ji1fu1}[][HSK 7-9]
    \definition{s.}{músculo e pele (humano)}
  \end{Phonetics}
\end{Entry}

%%%%%%%%%% 自 %%%%%%%%%%
\subsection*{自}\addcontentsline{loh}{figure}{自}

\begin{Entry}{自}{6}{⾃}[Kangxi 132]
  \begin{Phonetics}{自}{zi4}[][HSK 4]
    \definition*{s.}{Sobrenome: Zi}
    \definition{adv.}{certamente; com certeza; é claro; naturalmente}
    \definition{prep.}{de; desde; a partir de; apresenta o ponto de partida, a fonte ou o horário de início do comportamento da ação, equivalente a 从 e 由}
    \definition{pron.}{si mesmo; próprio | próprio; indica que a ação é iniciada por e direcionada a si mesmo | por si mesmo; indica que a ação é autoiniciada e não é causada por uma força externa}
    \definition{v.}{iniciar}
  \seealsoref{从}{cong2}
  \seealsoref{由}{you2}
  \end{Phonetics}
\end{Entry}

\begin{Entry}{自个儿}{6,3,2}{⾃,⼈,⼉}
  \begin{Phonetics}{自个儿}{zi4ge3r5}
    \definition{pron.}{(dialeto) a si mesmo, por si mesmo}
  \end{Phonetics}
\end{Entry}

\begin{Entry}{自己}{6,3}{⾃,⼰}
  \begin{Phonetics}{自己}{zi4ji3}[][HSK 2]
    \definition{pron.}{a si próprio; a si mesmo; refere"-se ao substantivo ou pronome precedente (enfatiza principalmente que não é devido a forças externas)}
  \end{Phonetics}
\end{Entry}

\begin{Entry}{自己动手}{6,3,6,4}{⾃,⼰,⼒,⼿}
  \begin{Phonetics}{自己动手}{zi4ji3dong4shou3}
    \definition{v.}{fazer (algo) sozinho | ajudar-se a}
  \end{Phonetics}
\end{Entry}

\begin{Entry}{自从}{6,4}{⾃,⼈}
  \begin{Phonetics}{自从}{zi4cong2}[][HSK 3]
    \definition{prep.}{de; desde; a partir de; referir-se a um momento ou evento específico no passado}
  \end{Phonetics}
\end{Entry}

\begin{Entry}{自主}{6,5}{⾃,⼂}
  \begin{Phonetics}{自主}{zi4zhu3}[][HSK 3]
    \definition{v.}{agir por conta própria; decidir por si mesmo; manter a iniciativa em suas próprias mãos; tomar suas próprias decisões}
  \end{Phonetics}
\end{Entry}

\begin{Entry}{自由}{6,5}{⾃,⽥}
  \begin{Phonetics}{自由}{zi4you2}[][HSK 2]
    \definition{adj.}{livre; irrestrito}
    \definition[个]{s.}{liberdade; o direito de agir de acordo com a própria vontade dentro do âmbito da lei | liberdade; filosoficamente, liberdade é definida como o processo de as pessoas reconhecerem as leis que governam o desenvolvimento das coisas e aplicá-las conscientemente na prática}
  \end{Phonetics}
\end{Entry}

\begin{Entry}{自由泳}{6,5,8}{⾃,⽥,⽔}
  \begin{Phonetics}{自由泳}{zi4you2yong3}
    \definition{s.}{natação de estilo livre}
  \end{Phonetics}
\end{Entry}

\begin{Entry}{自动}{6,6}{⾃,⼒}
  \begin{Phonetics}{自动}{zi4dong4}[][HSK 3]
    \definition{adj.}{automático; auto"-atuante; uso de dispositivos mecânicos, elétricos, etc, para funcionar automaticamente, sem necessidade de controle humano}
    \definition{adv.}{voluntariamente; por vontade própria; por iniciativa própria | automaticamente; espontaneamente; refere"-se a movimentos, mudanças, etc., que não são causados pela ação humana, mas sim pelo próprio objeto}
  \end{Phonetics}
\end{Entry}

\begin{Entry}{自动化}{6,6,4}{⾃,⼒,⼔}
  \begin{Phonetics}{自动化}{zi4dong4hua4}
    \definition{s.}{automação}
  \end{Phonetics}
\end{Entry}

\begin{Entry}{自在}{6,6}{⾃,⼟}
  \begin{Phonetics}{自在}{zi4zai4}[][HSK 6]
    \definition{adj.}{livre; irrestrito}
  \end{Phonetics}
\end{Entry}

\begin{Entry}{自杀}{6,6}{⾃,⽊}
  \begin{Phonetics}{自杀}{zi4sha1}[][HSK 5]
    \definition{s.}{suicídio; auto-assassinato; auto-sacrifício}
    \definition{v.}{cometer suicídio; tentar suicídio; matar-se}
  \end{Phonetics}
\end{Entry}

\begin{Entry}{自行车}{6,6,4}{⾃,⾏,⾞}
  \begin{Phonetics}{自行车}{zi4xing2che1}[][HSK 2]
    \definition[辆]{s.}{bicicleta; um veículo de duas rodas que é impulsionado para a frente com os pedais}
  \end{Phonetics}
\end{Entry}

\begin{Entry}{自行车架}{6,6,4,9}{⾃,⾏,⾞,⽊}
  \begin{Phonetics}{自行车架}{zi4xing2che1jia4}
    \definition{s.}{suporte para bicicleta | bicicletário}
  \end{Phonetics}
\end{Entry}

\begin{Entry}{自行车馆}{6,6,4,11}{⾃,⾏,⾞,⾷}
  \begin{Phonetics}{自行车馆}{zi4xing2che1guan3}
    \definition{s.}{estádio de ciclismo | velódromo}
  \end{Phonetics}
\end{Entry}

\begin{Entry}{自行车赛}{6,6,4,14}{⾃,⾏,⾞,⾙}
  \begin{Phonetics}{自行车赛}{zi4xing2che1sai4}
    \definition{s.}{corrida de bicicleta}
  \end{Phonetics}
\end{Entry}

\begin{Entry}{自我}{6,7}{⾃,⼽}
  \begin{Phonetics}{自我}{zi4wo3}[][HSK 6]
    \definition{pref.}{auto-}
    \definition{pron.}{a si mesmo; eu próprio; geralmente usado antes de verbos dissílabos para indicar que a ação é realizada por alguém e dirigida a si mesmo | indicar o próprio caráter diferente dos outros; refere"-se às próprias características, personalidade, hobbies, etc.}
  \end{Phonetics}
\end{Entry}

\begin{Entry}{自我介绍}{6,7,4,8}{⾃,⼽,⼈,⽷}
  \begin{Phonetics}{自我介绍}{zi4wo3jie4shao4}
    \definition{s.}{defesa pessoal | auto-defesa}
  \end{Phonetics}
\end{Entry}

\begin{Entry}{自我安慰}{6,7,6,15}{⾃,⼽,⼧,⼼}
  \begin{Phonetics}{自我安慰}{zi4wo3'an1wei4}
    \definition{v.}{confortar-se | consolar-se | tranquilizar-se}
  \end{Phonetics}
\end{Entry}

\begin{Entry}{自我防卫}{6,7,6,3}{⾃,⼽,⾩,⼙}
  \begin{Phonetics}{自我防卫}{zi4wo3fang2wei4}
    \definition{s.}{defesa pessoal | auto-defesa}
  \end{Phonetics}
\end{Entry}

\begin{Entry}{自我吹嘘}{6,7,7,14}{⾃,⼽,⼝,⼝}
  \begin{Phonetics}{自我吹嘘}{zi4wo3 chui1xu1}
    \definition{expr.}{gabar-se}
    \definition{s.}{auto-ostentação; autoglorificação;}
  \end{Phonetics}
\end{Entry}

\begin{Entry}{自我批评}{6,7,7,7}{⾃,⼽,⼿,⾔}
  \begin{Phonetics}{自我批评}{zi4wo3 pi1ping2}
    \definition{s.}{autocrítica}
  \end{Phonetics}
\end{Entry}

\begin{Entry}{自我实现}{6,7,8,8}{⾃,⼽,⼧,⾒}
  \begin{Phonetics}{自我实现}{zi4wo3shi2xian4}
    \definition{s.}{(psicologia) auto-realização}
  \end{Phonetics}
\end{Entry}

\begin{Entry}{自我的人}{6,7,8,2}{⾃,⼽,⽩,⼈}
  \begin{Phonetics}{自我的人}{zi4wo3de5ren2}
    \definition{s.}{(minha, sua) própria pessoa | (afirmar) a própria personalidade}
  \end{Phonetics}
\end{Entry}

\begin{Entry}{自我保存}{6,7,9,6}{⾃,⼽,⼈,⼦}
  \begin{Phonetics}{自我保存}{zi4wo3 bao3cun2}
    \definition{v.}{autopreservação}
  \end{Phonetics}
\end{Entry}

\begin{Entry}{自我陶醉}{6,7,10,15}{⾃,⼽,⾩,⾣}
  \begin{Phonetics}{自我陶醉}{zi4wo3tao2zui4}
    \definition{s.}{narcisista | auto-imbuído | satisfeito consigo mesmo}
  \end{Phonetics}
\end{Entry}

\begin{Entry}{自我催眠}{6,7,13,10}{⾃,⼽,⼈,⽬}
  \begin{Phonetics}{自我催眠}{zi4wo3cui1mian2}
    \definition{v.}{consolar-me | tranquilizar-me}
  \end{Phonetics}
\end{Entry}

\begin{Entry}{自我意识}{6,7,13,7}{⾃,⼽,⼼,⾔}
  \begin{Phonetics}{自我意识}{zi4wo3 yi4shi2}
    \definition{s.}{autoconsciência; auto-consciente}
  \end{Phonetics}
\end{Entry}

\begin{Entry}{自我解嘲}{6,7,13,15}{⾃,⼽,⾓,⼝}
  \begin{Phonetics}{自我解嘲}{zi4wo3 jie3chao2}
    \definition{s.}{autodepreciação; referir"-se às próprias fraquezas ou falhas com humor autodepreciativo}
    \definition{v.}{encontrar desculpas para; consolar"-se}
  \end{Phonetics}
\end{Entry}

\begin{Entry}{自来水}{6,7,4}{⾃,⽊,⽔}
  \begin{Phonetics}{自来水}{zi4lai2shui3}[][HSK 6]
    \definition{s.}{água da torneira; água corrente; água purificada e desinfetada fornecida por sistema hidráulico através de tubulações | equipamentos para transporte de água natural tratada}
  \end{Phonetics}
\end{Entry}

\begin{Entry}{自言自语}{6,7,6,9}{⾃,⾔,⾃,⾔}
  \begin{Phonetics}{自言自语}{zi4yan2-zi4yu3}[][HSK 6]
    \definition{expr.}{falando sozinho; falar consigo mesmo; pensar em voz alta; solilóquio}
  \end{Phonetics}
\end{Entry}

\begin{Entry}{自身}{6,7}{⾃,⾝}
  \begin{Phonetics}{自身}{zi4shen1}[][HSK 3]
    \definition{pron.}{eu mesmo (enfatizando que não é outra pessoa ou outra coisa)}
  \end{Phonetics}
\end{Entry}

\begin{Entry}{自学}{6,8}{⾃,⼦}
  \begin{Phonetics}{自学}{zi4xue2}[][HSK 6]
    \definition{s.}{auto-estudo; autodidata; autoaprendizagem}
    \definition{v.}{estudar por conta própria; estudar de forma independente; ensinar a si mesmo}
  \end{Phonetics}
\end{Entry}

\begin{Entry}{自责}{6,8}{⾃,⾙}
  \begin{Phonetics}{自责}{zi4ze2}
    \definition{v.}{culpar"-se}
  \end{Phonetics}
\end{Entry}

\begin{Entry}{自信}{6,9}{⾃,⼈}
  \begin{Phonetics}{自信}{zi4xin4}[][HSK 4]
    \definition{adj.}{confiante; descreve a crença em suas próprias habilidades, decisões, etc., tendo confiança em si mesmo}
    \definition[份,种]{s.}{autoconfiança; confiança em si mesmo}
    \definition{v.}{acreditar em si mesmo}
  \end{Phonetics}
\end{Entry}

\begin{Entry}{自觉}{6,9}{⾃,⾒}
  \begin{Phonetics}{自觉}{zi4jue2}[][HSK 3]
    \definition{adj.}{autoconsciente; de livre e espontânea vontade; controlar o próprio comportamento e agir por iniciativa própria}
    \definition{v.}{estar ciente de}
  \end{Phonetics}
\end{Entry}

\begin{Entry}{自救}{6,11}{⾃,⽁}
  \begin{Phonetics}{自救}{zi4jiu4}
    \definition{s.}{autoajuda}
    \definition{v.}{salvar-se; prover-se e ajudar-se}
  \end{Phonetics}
\end{Entry}

\begin{Entry}{自然}{6,12}{⾃,⽕}
  \begin{Phonetics}{自然}{zi4ran5}[][HSK 3]
    \definition{adj.}{natural; no curso normal dos eventos; formado ou desenvolvido sem intervenção humana; algo que se desenvolve livremente}
    \definition{adv.}{naturalmente; definitivamente; certamente, isso significa que, de acordo com a lógica, deve ser assim}
    \definition{conj.}{usado para ligar duas frases, com a segunda introduzindo informações adicionais ou adversativas; indica explicação complementar ou uma mudança de significado}
    \definition{s.}{natureza; mundo natural; tudo o que não foi criado pelo ser humano}
  \end{Phonetics}
\end{Entry}

\begin{Entry}{自愿}{6,14}{⾃,⽕}
  \begin{Phonetics}{自愿}{zi4yuan4}[][HSK 5]
    \definition{adv.}{voluntariamente; por iniciativa própria; por vontade própria}
    \definition{s.}{voluntário}
  \end{Phonetics}
\end{Entry}

\begin{Entry}{自豪}{6,14}{⾃,⾗}
  \begin{Phonetics}{自豪}{zi4hao2}[][HSK 5]
    \definition{adj.}{orgulhar-se de; ter orgulho de; sentir-se honrado por possuir qualidades excelentes ou ter alcançado grandes conquistas, seja por si mesmo ou por um grupo ou indivíduo relacionado a si}
  \end{Phonetics}
\end{Entry}

\begin{Entry}{自燃}{6,16}{⾃,⽕}
  \begin{Phonetics}{自燃}{zi4ran2}
    \definition{s.}{combustão espontânea}
  \end{Phonetics}
\end{Entry}

%%%%%%%%%% 至 %%%%%%%%%%
\subsection*{至}\addcontentsline{loh}{figure}{至}

\begin{Entry}{至}{6}{⾄}[Kangxi 133]
  \begin{Phonetics}{至}{zhi4}[][HSK 5]
    \definition{adv.}{a maior parte; extremamente; indica o grau mais alto, equivalente a 极 ou 最}
    \definition{prep.}{para; até; chegar a um determinado ponto}
    \definition{s.}{extremo, máximo}
    \definition{v.}{chegar; alcançar}
  \seealsoref{极}{ji2}
  \seealsoref{最}{zui4}
  \end{Phonetics}
\end{Entry}

\begin{Entry}{至于}{6,3}{⾄,⼆}
  \begin{Phonetics}{至于}{zhi4yu2}[][HSK 6]
    \definition{adv.}{quanto a; na medida em que; indica atingir um certo nível, frequentemente usado em frases negativas e perguntas retóricas}
    \definition{conj.}{a respeito de; quanto a; geralmente é usado entre duas frases ou no início da frase seguinte para introduzir ou mudar um novo tópico}[至于他的计划,我不太了解。===Quanto aos seus planos, não sei muito sobre eles.]
  \end{Phonetics}
\end{Entry}

\begin{Entry}{至今}{6,4}{⾄,⼈}
  \begin{Phonetics}{至今}{zhi4jin1}[][HSK 3]
    \definition{adv.}{até agora; até o momento; até hoje}
  \end{Phonetics}
\end{Entry}

\begin{Entry}{至少}{6,4}{⾄,⼩}
  \begin{Phonetics}{至少}{zhi4shao3}[][HSK 3]
    \definition{adv.}{pelo menos; indica o limite mínimo}
  \end{Phonetics}
\end{Entry}

%%%%%%%%%% 舌 %%%%%%%%%%
\subsection*{舌}\addcontentsline{loh}{figure}{舌}

\begin{Entry}{舌}{6}{⾆}[Kangxi 135]
  \begin{Phonetics}{舌}{she2}
    \definition*{s.}{Sobrenome: She}
    \definition[片,条]{s.}{língua (de um ser humano ou animal); glossa | algo em forma de língua | língua de sino; badalo}
  \end{Phonetics}
\end{Entry}

\begin{Entry}{舌头}{6,5}{⾆,⼤}
  \begin{Phonetics}{舌头}{she2tou5}[][HSK 6]
    \definition[个]{s.}{língua; órgão que auxilia no paladar, na mastigação e na pronúncia | espião}
  \end{Phonetics}
\end{Entry}

%%%%%%%%%% 色 %%%%%%%%%%
\subsection*{色}\addcontentsline{loh}{figure}{色}

\begin{Entry}{色}{6}{⾊}[Kangxi 139]
  \begin{Phonetics}{色}{se4}[][HSK 4]
    \definition*{s.}{Sobrenome: Se}
    \definition[种]{s.}{cor | aparência; semblante; expressão | tipo; gênero; descrição | cena; cenário;  paisagem | qualidade (de metais preciosos, mercadorias, etc.) | aparência feminina; beleza feminina | erotismo; apetite sexual; luxúria; desejo sexual}
  \end{Phonetics}
  \begin{Phonetics}{色}{shai3}
    \definition[4]{s.}{cor; (~儿) tem o mesmo significado que 色, usado em algumas palavras faladas}
  \end{Phonetics}
\end{Entry}

\begin{Entry}{色狼}{6,10}{⾊,⽝}
  \begin{Phonetics}{色狼}{se4lang2}
    \definition*{s.}{Sátiro}
    \definition{adj.}{lascivo; lobo; pervertido; isso se refere a uma pessoa que persegue mulheres com ganância e as agride sexualmente de forma brutal}
  \end{Phonetics}
\end{Entry}

\begin{Entry}{色彩}{6,11}{⾊,⼺}
  \begin{Phonetics}{色彩}{se4cai3}[][HSK 4]
    \definition[种,丝]{s.}{cor; matiz; tonalidade | cor; sabor; característica; metáfora para um determinado estado de espírito ou tendência de pensamento}
  \end{Phonetics}
\end{Entry}

%%%%%%%%%% 芋 %%%%%%%%%%
\subsection*{芋}\addcontentsline{loh}{figure}{芋}

\begin{Entry}{芋}{6}{⾋}
  \begin{Phonetics}{芋}{yu4}
    \definition*{s.}{Sobrenome: Yu}
    \definition{s.}{taro; erva perene | tubérculos; geralmente se refere a batatas, etc.}
  \end{Phonetics}
\end{Entry}

\begin{Entry}{芋头}{6,5}{⾋,⼤}
  \begin{Phonetics}{芋头}{yu4tou5}
    \definition{s.}{taro, similar ao inhame e batata doce}
  \end{Phonetics}
\end{Entry}

\begin{Entry}{芋头色}{6,5,6}{⾋,⼤,⾊}
  \begin{Phonetics}{芋头色}{yu4tou5se4}
    \definition{s.}{cor lilás}
  \end{Phonetics}
\end{Entry}

%%%%%%%%%% 芝 %%%%%%%%%%
\subsection*{芝}\addcontentsline{loh}{figure}{芝}

\begin{Entry}{芝}{6}{⾋}
  \begin{Phonetics}{芝}{zhi1}
    \definition*{s.}{Sobrenome: Zhi}
    \definition{s.}{Arcaico: fungo mágico, ganoderma brilhante | Arcaico: raiz de angélica dahuriana}
  \end{Phonetics}
\end{Entry}

\begin{Entry}{芝麻}{6,11}{⾋,⿇}
  \begin{Phonetics}{芝麻}{zhi1ma5}
    \definition{s.}{semente de gergelim}
  \end{Phonetics}
\end{Entry}

%%%%%%%%%% 虫 %%%%%%%%%%
\subsection*{虫}\addcontentsline{loh}{figure}{虫}

\begin{Entry}{虫}{6}{⾍}[Kangxi 142]
  \begin{Phonetics}{虫}{chong2}
    \definition[只,条]{s.}{inseto; verme | (pejorativo) pessoas que se comportam de forma desprezível | fã; viciado | forma inferior de vida animal, incluindo insetos, larvas de insetos, vermes e criaturas semelhantes | pessoa com uma característica indesejável específica}
  \end{Phonetics}
\end{Entry}

\begin{Entry}{虫子}{6,3}{⾍,⼦}
  \begin{Phonetics}{虫子}{chong2zi5}[][HSK 4]
    \definition[条,只,种]{s.}{percevejo; besouro; inseto; verme; criaturas semelhantes a insetos}
  \end{Phonetics}
\end{Entry}

%%%%%%%%%% 血 %%%%%%%%%%
\subsection*{血}\addcontentsline{loh}{figure}{血}

\begin{Entry}{血}{6}{⾎}[Kangxi 143]
  \begin{Phonetics}{血}{xie3}
  \end{Phonetics}
  \begin{Phonetics}{血}{xue4}[][HSK 3]
    \definition[滴,袋,口,毫升]{s.}{sangue | parente consanguíneo; com laços de parentesco | pessoa ativa e animada; metáfora para uma personalidade ou espírito forte e sincero | medicina tradicional chinesa refere"-se à menstruação}
  \end{Phonetics}
\end{Entry}

\begin{Entry}{血汗}{6,6}{⾎,⽔}
  \begin{Phonetics}{血汗}{xue4han4}
    \definition{s.}{(fig.) suor e labuta, trabalho duro}
  \end{Phonetics}
\end{Entry}

\begin{Entry}{血液}{6,11}{⾎,⽔}
  \begin{Phonetics}{血液}{xue4ye4}[][HSK 6]
    \definition[毫升]{s.}{sangue | linha de vida; sangue vital; uma metáfora para o importante componente ou força que mantém a vitalidade coletiva}
  \end{Phonetics}
\end{Entry}

\begin{Entry}{血管}{6,14}{⾎,⽵}
  \begin{Phonetics}{血管}{xue4guan3}[][HSK 6]
    \definition[根,条,种]{s.}{vaso; vaso sanguíneo; os canais tubulares pelos quais o sangue circula são divididos em três tipos: artérias, veias e capilares}
  \end{Phonetics}
\end{Entry}

%%%%%%%%%% 行 %%%%%%%%%%
\subsection*{行}\addcontentsline{loh}{figure}{行}

\begin{Entry}{行}{6}{⾏}[Kangxi 144]
  \begin{Phonetics}{行}{hang2}[][HSK 3]
    \definition{adj.}{temporário; improvisado | capaz; competente}
    \definition{adv.}{logo; em breve}
    \definition{clas.}{linha; fileira; coisas usadas para formar filas, linhas}
    \definition{s.}{comportamento; conduta | linha; fileira | empresa comercial; certas instituições comerciais | comércio; profissão; ramo de atividade | especialista; conhecedor; refere"-se ao conhecimento e experiência em um determinado setor}
    \definition{v.}{ir; caminhar; viajar | estar atualizado; circular | fazer; executar; realizar | (antes de um verbo dissílabo, indicando a realização de alguma ação) | ficar bem; vai dar certo | (remédio) fazer efeito | classificar (entre irmãos e irmãs por ordem de idade)}
  \end{Phonetics}
  \begin{Phonetics}{行}{heng2}
    \definition{s.}{usado em 道行}
  \seealsoref{道行}{dao4 heng2}
  \end{Phonetics}
  \begin{Phonetics}{行}{xing2}[][HSK 1]
    \definition*{s.}{Sobrenome: Xing}
    \definition{adj.}{de viajar; relacionado a viagens | temporário; improvisado; provisório | capaz; competente}
    \definition{adv.}{em breve}
    \definition{s.}{comportamento; conduta | caligrafia cursiva (na caligrafia chinesa); escrita cursiva}
    \definition{v.}{ir | fazer uma viagem | estar em voga; prevalecer; circular | fazer; executar; realizar; envolver-se em | estar tudo bem; O.K. | indica a realização de uma determinada atividade (usado principalmente antes de verbos dissilábicos) | (em medicina) fazer efeito}
  \end{Phonetics}
\end{Entry}

\begin{Entry}{行人}{6,2}{⾏,⼈}
  \begin{Phonetics}{行人}{xing2ren2}[][HSK 2]
    \definition[个]{s.}{pedestre; transeunte; viajante à pé; pessoas caminhando na estrada}
  \end{Phonetics}
\end{Entry}

\begin{Entry}{行为}{6,4}{⾏,⼂}
  \begin{Phonetics}{行为}{xing2wei2}[][HSK 2]
    \definition[个,种,类]{s.}{ação; comportamento; conduta; atividades que são controladas por pensamentos e manifestadas externamente}
  \end{Phonetics}
\end{Entry}

\begin{Entry}{行凶}{6,4}{⾏,⼐}
  \begin{Phonetics}{行凶}{xing2/xiong1}
    \definition{v.+compl.}{cometer agressão física ou assassinato | fazer algo violento}
  \end{Phonetics}
\end{Entry}

\begin{Entry}{行业}{6,5}{⾏,⼀}
  \begin{Phonetics}{行业}{hang2ye4}[][HSK 4]
    \definition[种,个]{s.}{comércio; indústria; setor; profissão; categorias em negócios e indústria referem-se a ocupações em geral}
  \end{Phonetics}
\end{Entry}

\begin{Entry}{行礼}{6,5}{⾏,⽰}
  \begin{Phonetics}{行礼}{xing2li3}
    \definition{v.}{saudar | fazer saudação}
  \end{Phonetics}
\end{Entry}

\begin{Entry}{行列}{6,6}{⾏,⼑}
  \begin{Phonetics}{行列}{hang2lie4}[][HSK 7-9]
    \definition{s.}{fileiras}
  \end{Phonetics}
\end{Entry}

\begin{Entry}{行动}{6,6}{⾏,⼒}
  \begin{Phonetics}{行动}{xing2dong4}[][HSK 2]
    \definition[次,场,项]{s.}{ação; operação; comportamento;}
    \definition{v.}{circular; mover-se; andar | agir; tomar medidas; atividades para atingir um determinado propósito}
  \end{Phonetics}
\end{Entry}

\begin{Entry}{行李}{6,7}{⾏,⽊}
  \begin{Phonetics}{行李}{xing2li5}[][HSK 3]
    \definition[点,个]{s.}{bagagem, malas, cestas de vime, etc. que você leva quando sai de casa}
  \end{Phonetics}
\end{Entry}

\begin{Entry}{行进}{6,7}{⾏,⾡}
  \begin{Phonetics}{行进}{xing2jin4}
    \definition{s.}{avançar | movimentar-se para frente}
  \end{Phonetics}
\end{Entry}

\begin{Entry}{行驶}{6,8}{⾏,⾺}
  \begin{Phonetics}{行驶}{xing2shi3}[][HSK 5]
    \definition{v.}{ir; navegar; viajar (utilizando um veículo, navio, etc.)}
  \end{Phonetics}
\end{Entry}

\begin{Entry}{行星}{6,9}{⾏,⽇}
  \begin{Phonetics}{行星}{xing2xing1}
    \definition[颗]{s.}{planeta}
  \seealsoref{惑星}{huo4xing1}
  \end{Phonetics}
\end{Entry}

\begin{Entry}{行家}{6,10}{⾏,⼧}
  \begin{Phonetics}{行家}{hang2jia5}[][HSK 7-9]
    \definition[位,名,个,些]{s.}{especialista; \emph{expert}; conhecedor; \emph{connoisseur}}
  \end{Phonetics}
\end{Entry}

\begin{Entry}{行情}{6,11}{⾏,⼼}
  \begin{Phonetics}{行情}{hang2qing2}[][HSK 7-9]
    \definition{s.}{preço; cotações de mercado; o preço geral dos bens no mercado também se refere à situação geral das taxas de juros, taxas de câmbio, preços de títulos, etc. no mercado financeiro}
  \end{Phonetics}
\end{Entry}

\begin{Entry}{行程}{6,12}{⾏,⽲}
  \begin{Phonetics}{行程}{xing2cheng2}[][HSK 6]
    \definition{s.}{rota ou distância de viagem; distância; jornada | curso; progresso; processo | curso; deslocamento; viagem}[活塞行程有点不对劲。===Há algo errado com o curso do pistão.]
  \end{Phonetics}
\end{Entry}

%%%%%%%%%% 衣 %%%%%%%%%%
\subsection*{衣}\addcontentsline{loh}{figure}{衣}

\begin{Entry}{衣}{6}{⾐}
  \begin{Phonetics}{衣}{yi1}
    \definition[件]{s.}{roupa}
  \end{Phonetics}
  \begin{Phonetics}{衣}{yi4}
    \definition{v.}{vestir-se; vestir alguém}
  \end{Phonetics}
\end{Entry}

\begin{Entry}{衣甲}{6,5}{⾐,⽥}
  \begin{Phonetics}{衣甲}{yi1jia3}
    \definition{s.}{armadura}
  \end{Phonetics}
\end{Entry}

\begin{Entry}{衣服}{6,8}{⾐,⽉}
  \begin{Phonetics}{衣服}{yi1fu5}[][HSK 1]
    \definition[套,件]{s.}{roupas; vestuário; algo que se veste para cobrir o corpo e se proteger do frio}
  \end{Phonetics}
\end{Entry}

\begin{Entry}{衣柜}{6,8}{⾐,⽊}
  \begin{Phonetics}{衣柜}{yi1gui4}
    \definition[个]{s.}{armário | guarda-roupa}
  \end{Phonetics}
\end{Entry}

\begin{Entry}{衣架}{6,9}{⾐,⽊}
  \begin{Phonetics}{衣架}{yi1jia4}[][HSK 3]
    \definition[个,副,组]{s.}{cabideiro; móvel para pendurar roupas | estatura; figura; refere"-se ao tipo físico de uma pessoa; estrutura corporal}
  \end{Phonetics}
\end{Entry}

%%%%%%%%%% 西 %%%%%%%%%%
\subsection*{西}\addcontentsline{loh}{figure}{西}

\begin{Entry}{西}{6}{⾑}[Kangxi 146]
  \begin{Phonetics}{西}{xi1}[][HSK 1]
    \definition*{s.}{Espanha, abreviatura de 西班牙 | Paraíso Ocidental | Sobrenome: Xi}
    \definition{s.}{oeste; uma das quatro direções básicas, o lado onde o sol se põe | ocidental; refere"-se ao Ocidente (principalmente aos países europeus e americanos) | aqui e ali; significa 到处 ou 零散, 没有次序}
  \seealsoref{到处}{dao4chu4}
  \seealsoref{零散}{ling2san3}
  \seealsoref{没有次序}{mei2you3 ci4xu4}
  \seealsoref{西班牙}{xi1ban1ya2}
  \antonymref{东}{dong1}
  \end{Phonetics}
\end{Entry}

\begin{Entry}{西文}{6,4}{⾑,⽂}
  \begin{Phonetics}{西文}{xi1wen2}
    \definition{s.}{espanhol | língua espanhola}
  \seealsoref{西班牙文}{xi1ban1ya2wen2}
  \end{Phonetics}
\end{Entry}

\begin{Entry}{西方}{6,4}{⾑,⽅}
  \begin{Phonetics}{西方}{xi1fang1}[][HSK 2]
    \definition{s.}{oeste | o Ocidente; o Oeste; países europeus e americanos | Paraíso Ocidental, termo budista}
  \end{Phonetics}
\end{Entry}

\begin{Entry}{西兰花}{6,5,7}{⾑,⼋,⾋}
  \begin{Phonetics}{西兰花}{xi1lan2hua1}
    \definition{s.}{brócolis}
  \end{Phonetics}
\end{Entry}

\begin{Entry}{西北}{6,5}{⾑,⼔}
  \begin{Phonetics}{西北}{xi1bei3}[][HSK 2]
    \definition{s.}{noroeste | noroeste da China; o Noroeste}
  \end{Phonetics}
\end{Entry}

\begin{Entry}{西半球}{6,5,11}{⾑,⼗,⽟}
  \begin{Phonetics}{西半球}{xi1ban4qiu2}
    \definition{s.}{hemisfério oeste}
  \end{Phonetics}
\end{Entry}

\begin{Entry}{西瓜}{6,5}{⾑,⽠}
  \begin{Phonetics}{西瓜}{xi1gua5}[][HSK 4]
    \definition[个,颗,粒]{s.}{melancia; fruto que é uma baga de formato grande, globular ou oval, com muita polpa aguada e doce}
  \end{Phonetics}
\end{Entry}

\begin{Entry}{西边}{6,5}{⾑,⾡}
  \begin{Phonetics}{西边}{xi1bian5}[][HSK 1]
    \definition{s.}{lado oeste; (oeste) Uma das quatro direções principais; uma das direções cardeais}
  \antonymref{东方}{dong1fang1}
  \end{Phonetics}
\end{Entry}

\begin{Entry}{西安}{6,6}{⾑,⼧}
  \begin{Phonetics}{西安}{xi1'an1}
    \definition*{s.}{Xi'an, Capital da Província de Shaanxi}
  \end{Phonetics}
\end{Entry}

\begin{Entry}{西红柿}{6,6,9}{⾑,⽷,⽊}
  \begin{Phonetics}{西红柿}{xi1hong2shi4}[][HSK 5]
    \definition[种,只,株]{s.}{tomate}
  \end{Phonetics}
\end{Entry}

\begin{Entry}{西西}{6,6}{⾑,⾑}
  \begin{Phonetics}{西西}{xi1xi1}
    \definition{num.}{centímetro cúbico}
  \end{Phonetics}
\end{Entry}

\begin{Entry}{西医}{6,7}{⾑,⼖}
  \begin{Phonetics}{西医}{xi1yi1}[][HSK 2]
    \definition[名,位]{s.}{medicina ocidental; medicina introduzida na China a partir da Europa e da América | um médico treinado em medicina ocidental}
  \end{Phonetics}
\end{Entry}

\begin{Entry}{西南}{6,9}{⾑,⼗}
  \begin{Phonetics}{西南}{xi1nan2}[][HSK 2]
    \definition{s.}{sudoeste | o Sudoeste; Sudoeste da China}
  \end{Phonetics}
\end{Entry}

\begin{Entry}{西药}{6,9}{⾑,⾋}
  \begin{Phonetics}{西药}{xi1 yao4}
    \definition[片,粒]{s.}{medicina ocidental; refere"-se aos medicamentos usados na medicina ocidental, geralmente feitos por métodos sintéticos ou extraídos de produtos naturais, como comprimidos anti"-inflamatórios, aspirina, tintura de iodo, penicilina, etc.}
  \end{Phonetics}
\end{Entry}

\begin{Entry}{西语}{6,9}{⾑,⾔}
  \begin{Phonetics}{西语}{xi1yu3}
    \definition{s.}{línguas ocidentais | espanhol | língua espanhola}
  \seealsoref{西班牙语}{xi1ban1ya2yu3}
  \end{Phonetics}
\end{Entry}

\begin{Entry}{西面}{6,9}{⾑,⾯}
  \begin{Phonetics}{西面}{xi1mian4}
    \definition{s.}{oeste | lado oeste}
  \end{Phonetics}
\end{Entry}

\begin{Entry}{西班牙}{6,10,4}{⾑,⽟,⽛}
  \begin{Phonetics}{西班牙}{xi1ban1ya2}
    \definition*{s.}{Espanha}
  \end{Phonetics}
\end{Entry}

\begin{Entry}{西班牙文}{6,10,4,4}{⾑,⽟,⽛,⽂}
  \begin{Phonetics}{西班牙文}{xi1ban1ya2wen2}
    \definition{s.}{espanhol, língua espanhola}
  \seealsoref{西文}{xi1wen2}
  \end{Phonetics}
\end{Entry}

\begin{Entry}{西班牙语}{6,10,4,9}{⾑,⽟,⽛,⾔}
  \begin{Phonetics}{西班牙语}{xi1ban1ya2yu3}[][HSK 6]
    \definition[句]{s.}{espanhol | língua espanhola}
  \seealsoref{西语}{xi1yu3}
  \end{Phonetics}
\end{Entry}

\begin{Entry}{西部}{6,10}{⾑,⾢}
  \begin{Phonetics}{西部}{xi1bu4}[][HSK 3]
    \definition{s.}{(EUA) filme de faroeste; filme de \emph{cowboys}; um faroeste | filme da região ocidental (China) | parte ocidental; região oeste da China}
  \end{Phonetics}
\end{Entry}

\begin{Entry}{西装}{6,12}{⾑,⾐}
  \begin{Phonetics}{西装}{xi1zhuang1}[][HSK 5]
    \definition[件,套,个]{s.}{terno; roupas de estilo ocidental; roupas ocidentais, divididas em masculinas e femininas}
  \end{Phonetics}
\end{Entry}

\begin{Entry}{西蓝花}{6,13,7}{⾑,⾋,⾋}
  \begin{Phonetics}{西蓝花}{xi1lan2hua1}
    \variantof{西兰花}
  \end{Phonetics}
\end{Entry}

\begin{Entry}{西餐}{6,16}{⾑,⾷}
  \begin{Phonetics}{西餐}{xi1can1}[][HSK 2]
    \definition[份,顿,桌]{s.}{comida ocidental; comida de estilo ocidental, comida com garfo e faca (diferente da 中餐)}
  \seealsoref{中餐}{zhong1can1}
  \end{Phonetics}
\end{Entry}

\begin{Entry}{西藏}{6,17}{⾑,⾋}
  \begin{Phonetics}{西藏}{xi1zang4}
    \definition*{s.}{Xizang; Região Autônoma do Tibete, 西藏自治区}
  \seealsoref{西藏自治区}{xi1zang4 zi4zhi4qu1}
  \end{Phonetics}
\end{Entry}

\begin{Entry}{西藏自治区}{6,17,6,8,4}{⾑,⾋,⾃,⽔,⼖}
  \begin{Phonetics}{西藏自治区}{xi1zang4 zi4zhi4qu1}
    \definition*{s.}{Região Autônoma do Tibete}
  \end{Phonetics}
\end{Entry}

%%%%%%%%%% 观 %%%%%%%%%%
\subsection*{观}\addcontentsline{loh}{figure}{观}

\begin{Entry}{观}{6}{⾒}
  \begin{Phonetics}{观}{guan1}
    \definition*{s.}{Templo taoísta; ``Koon''}
    \definition{s.}{visão; vista | perspectiva; visão; conceito | aparência; perspectiva | alcance de visão | noção; ideia; conhecimento ou visão das coisas | ponto de vista; postura; uma visão de uma coisa}
    \definition{v.}{olhar para; assistir; observar | contemplar}
  \end{Phonetics}
  \begin{Phonetics}{观}{guan4}
    \definition*{s.}{Sobrenome: Guan}
    \definition{s.}{mosteiro taoísta | torre de vigia do portão do palácio | plataforma}
  \end{Phonetics}
\end{Entry}

\begin{Entry}{观众}{6,6}{⾒,⼈}
  \begin{Phonetics}{观众}{guan1zhong4}[][HSK 3]
    \definition[位,名,批,个]{s.}{espectador; público; audiência; pessoas que assistem a espetáculos ou competições}
  \end{Phonetics}
\end{Entry}

\begin{Entry}{观光}{6,6}{⾒,⼉}
  \begin{Phonetics}{观光}{guan1guang1}[][HSK 6]
    \definition{v.}{visitar; passear; fazer turismo; fazer um passeio em um país ou lugar estrangeiro}
  \end{Phonetics}
\end{Entry}

\begin{Entry}{观念}{6,8}{⾒,⼼}
  \begin{Phonetics}{观念}{guan1nian4}[][HSK 3]
    \definition[种,个]{s.}{ideia; conceito; consciência ideológica}
  \end{Phonetics}
\end{Entry}

\begin{Entry}{观测}{6,9}{⾒,⽔}
  \begin{Phonetics}{观测}{guan1ce4}[][HSK 7-9]
    \definition{v.}{pesquisar; observar e medir; observar e medir (astronomia, geografia, clima, direção, etc.) | observar; assistir e analisar; observar e medir (situação)}
  \end{Phonetics}
\end{Entry}

\begin{Entry}{观点}{6,9}{⾒,⽕}
  \begin{Phonetics}{观点}{guan1dian3}[][HSK 2]
    \definition[个,种]{s.}{ponto de vista; perspectiva; a visão ou atitude que se tem sobre algo a partir de uma determinada posição ou perspectiva | ponto de vista; perspectiva; a posição ou perspectiva adotada ao analisar uma questão}
  \end{Phonetics}
\end{Entry}

\begin{Entry}{观看}{6,9}{⾒,⽬}
  \begin{Phonetics}{观看}{guan1kan4}[][HSK 3]
    \definition{v.}{assistir; ver propositadamente; observar}
  \end{Phonetics}
\end{Entry}

\begin{Entry}{观望}{6,11}{⾒,⽉}
  \begin{Phonetics}{观望}{guan1wang4}[][HSK 7-9]
    \definition{v.}{esperar para ver; observar (de lado) | olhar ao redor}
  \end{Phonetics}
\end{Entry}

\begin{Entry}{观赏}{6,12}{⾒,⾙}
  \begin{Phonetics}{观赏}{guan1shang3}[][HSK 7-9]
    \definition{v.}{ver e admirar; apreciar a vista de; assistir e aproveitar}
  \end{Phonetics}
\end{Entry}

\begin{Entry}{观感}{6,13}{⾒,⼼}
  \begin{Phonetics}{观感}{guan1gan3}[][HSK 7-9]
    \definition{s.}{impressões; observações | impressões de alguém}
  \end{Phonetics}
\end{Entry}

\begin{Entry}{观察}{6,14}{⾒,⼧}
  \begin{Phonetics}{观察}{guan1cha2}[][HSK 3]
    \definition{v.}{assistir; pesquisar; observar; examinar cuidadosamente coisas ou fenômenos}
  \end{Phonetics}
\end{Entry}

\begin{Entry}{观摩}{6,15}{⾒,⼿}
  \begin{Phonetics}{观摩}{guan1mo2}[][HSK 7-9]
    \definition{v.}{inspecionar e aprender com o trabalho uns dos outros; visualizar e emular; observar, refere"-se principalmente a observar as conquistas uns dos outros, trocar experiências e aprender uns com os outros}
  \end{Phonetics}
\end{Entry}

%%%%%%%%%% 讲 %%%%%%%%%%
\subsection*{讲}\addcontentsline{loh}{figure}{讲}

\begin{Entry}{讲}{6}{⾔}
  \begin{Phonetics}{讲}{jiang3}[][HSK 2]
    \definition[种]{s.}{palestra; discurso}
    \definition{v.}{contar; falar | explicar; transmitir oralmente; esclarecer | negociar; barganhar | ser exigente com; valorizar; dar importância}
  \end{Phonetics}
\end{Entry}

\begin{Entry}{讲究}{6,7}{⾔,⽳}
  \begin{Phonetics}{讲究}{jiang3jiu5}[][HSK 4]
    \definition{adj.}{requintado; elegante; de bom gosto; exigente com a vida e com outros aspectos, buscando alto nível, qualidade e detalhes}
    \definition{s.}{estudo cuidadoso; algo que merece atenção; elementos e aspectos que merecem atenção especial}
    \definition{v.}{dar ênfase a; ser específico sobre; prestar atenção a}
  \end{Phonetics}
\end{Entry}

\begin{Entry}{讲学}{6,8}{⾔,⼦}
  \begin{Phonetics}{讲学}{jiang3/xue2}[][HSK 7-9]
    \definition{v.+compl.}{ministrar palestras; discorrer sobre um assunto acadêmico}
  \end{Phonetics}
\end{Entry}

\begin{Entry}{讲话}{6,8}{⾔,⾔}
  \begin{Phonetics}{讲话}{jiang3hua4}[][HSK 2]
    \definition[个]{s.}{discurso; palestra | guia; introdução}
    \definition{v.}{falar; conversar; dirigir-se a alguém | criticar}
  \end{Phonetics}
\end{Entry}

\begin{Entry}{讲述}{6,8}{⾔,⾡}
  \begin{Phonetics}{讲述}{jiang3shu4}[][HSK 7-9]
    \definition{v.}{narrar; relatar; contar sobre; dar um relato de}
  \end{Phonetics}
\end{Entry}

\begin{Entry}{讲座}{6,10}{⾔,⼴}
  \begin{Phonetics}{讲座}{jiang3zuo4}[][HSK 4]
    \definition[场,次]{s.}{palestra; um curso de palestras; a forma de instrução usada para ensinar um determinado assunto ou tópico, geralmente por meio de palestras ao vivo, seriados de rádio ou televisão ou seriados de jornal.}
  \end{Phonetics}
\end{Entry}

\begin{Entry}{讲课}{6,10}{⾔,⾔}
  \begin{Phonetics}{讲课}{jiang3 ke4}[][HSK 6]
    \definition{v.}{ensinar; dar palestras; proferir uma palestra | dar uma lição (palestra)}
  \end{Phonetics}
\end{Entry}

\begin{Entry}{讲解}{6,13}{⾔,⾓}
  \begin{Phonetics}{讲解}{jiang3jie3}[][HSK 7-9]
    \definition{v.}{explicar; interpretar; expor}
  \end{Phonetics}
\end{Entry}

%%%%%%%%%% 许 %%%%%%%%%%
\subsection*{许}\addcontentsline{loh}{figure}{许}

\begin{Entry}{许}{6}{⾔}
  \begin{Phonetics}{许}{xu3}
    \definition*{s.}{Xu, um estado da Dinastia Zhou | Sobrenome: Xu}
    \definition{adv.}{um pouco;  talvez; expressa especulação ou estimativa, equivalente a 或者 ou 可能}
    \definition{part.}{cerca de; aproximadamente; usado depois de certos numerais, frases de quantidade ou 些 ou 少 para indicar um número próximo a um certo número}
    \definition{pron.}{muitos; um monte de}
    \definition{v.}{elogiar; aprovar | prometer; prometer dar antecipadamente; dedicar | permitir; concordar; aprovar | (uma menina) estar prometida a; refere"-se especificamente ao noivado}
  \seealsoref{或者}{huo4zhe3}
  \seealsoref{可能}{ke3neng2}
  \seealsoref{少}{shao3}
  \seealsoref{些}{xie1}
  \end{Phonetics}
\end{Entry}

\begin{Entry}{许可}{6,5}{⾔,⼝}
  \begin{Phonetics}{许可}{xu3ke3}[][HSK 5]
    \definition{v.}{permitir; autorizar}
  \end{Phonetics}
\end{Entry}

\begin{Entry}{许多}{6,6}{⾔,⼣}
  \begin{Phonetics}{许多}{xu3duo1}[][HSK 2]
    \definition{num.}{muitos; muito; numerosos; uma grande quantidade de}
  \end{Phonetics}
\end{Entry}

%%%%%%%%%% 讹 %%%%%%%%%%
\subsection*{讹}\addcontentsline{loh}{figure}{讹}

\begin{Entry}{讹}{6}{⾔}
  \begin{Phonetics}{讹}{e2}
    \definition{adj.}{errôneo; equivocado}
    \definition{v.}{extorquir sob falsos pretextos; chantagear; enganar}
  \end{Phonetics}
\end{Entry}

\begin{Entry}{讹诈}{6,7}{⾔,⾔}
  \begin{Phonetics}{讹诈}{e2zha4}[][HSK 7-9]
    \definition{v.}{extorquir sob falsos pretextos; chantagear; intimidar}
  \end{Phonetics}
\end{Entry}

%%%%%%%%%% 论 %%%%%%%%%%
\subsection*{论}\addcontentsline{loh}{figure}{论}

\begin{Entry}{论}{6}{⾔}
  \begin{Phonetics}{论}{lun2}
    \definition*{s.}{Os Analectos de Confúcio, registro dos ditos e feitos de Confúcio e seus discípulos}
  \end{Phonetics}
  \begin{Phonetics}{论}{lun4}
    \definition*{s.}{Sobrenome: Lun}
    \definition{prep.}{por (uma certa unidade de medida) | de acordo com (um certo sistema ou princípio)}
    \definition{s.}{visão; opinião; declaração | (frequentemente em títulos) dissertação; ensaio; tratado | teoria; doutrina | ideia; palavras ou artigos que analisam e explicam coisas}
    \definition{v.}{discutir; falar sobre; discursar sobre; comentar | mencionar; considerar; falar de | decidir sobre; determinar | decidir sobre a natureza da culpa; punir | argumentar; analisar e explicar coisas | considerar; ponderar; medir; avaliar}
  \end{Phonetics}
\end{Entry}

\begin{Entry}{论文}{6,4}{⾔,⽂}
  \begin{Phonetics}{论文}{lun4wen2}[][HSK 4]
    \definition[篇]{s.}{tese; redação; artigo; artigo que discute ou examina uma questão}
  \end{Phonetics}
\end{Entry}

\begin{Entry}{论坛}{6,7}{⾔,⼟}
  \begin{Phonetics}{论坛}{lun4tan2}[][HSK 7-9]
    \definition[个,届]{s.}{fórum; tribuna; espaços para debate público, como jornais e fóruns}
  \end{Phonetics}
\end{Entry}

\begin{Entry}{论证}{6,7}{⾔,⾔}
  \begin{Phonetics}{论证}{lun4zheng4}[][HSK 7-9]
    \definition[次,番]{s.}{prova; demonstração; raciocínio lógico; refere"-se ao processo de pensamento para determinar a veracidade de um julgamento com base em verdades conhecidas | evidências; argumentação; fundamentos da argumentação; a base do argumento}
    \definition{v.}{expor e provar; discutir e comprovar}
  \end{Phonetics}
\end{Entry}

\begin{Entry}{论述}{6,8}{⾔,⾡}
  \begin{Phonetics}{论述}{lun4shu4}[][HSK 7-9]
    \definition{v.}{discutir; expor; elaborar; relacionar e analisar}
  \end{Phonetics}
\end{Entry}

%%%%%%%%%% 讽 %%%%%%%%%%
\subsection*{讽}\addcontentsline{loh}{figure}{讽}

\begin{Entry}{讽}{6}{⾔}
  \begin{Phonetics}{讽}{feng3}
    \definition{v.}{satirizar; zombar | Literário: cantar; entoar}
  \end{Phonetics}
\end{Entry}

\begin{Entry}{讽刺}{6,8}{⾔,⼑}
  \begin{Phonetics}{讽刺}{feng3ci4}[][HSK 7-9]
    \definition{adj.}{irônico; satírico; sarcástico}
    \definition{v.}{satirizar; ridicularizar; usar metáforas, exageros, ironia e outras expressões para expor, criticar ou ridicularizar}
  \end{Phonetics}
\end{Entry}

%%%%%%%%%% 设 %%%%%%%%%%
\subsection*{设}\addcontentsline{loh}{figure}{设}

\begin{Entry}{设}{6}{⾔}
  \begin{Phonetics}{设}{she4}[][HSK 7-9]
    \definition*{s.}{Sobrenome: She}
    \definition{conj.}{se; no caso | Matemática: dado; suponha; se}
    \definition{v.}{configurar; estabelecer; encontrar; colocar em prática}
  \end{Phonetics}
\end{Entry}

\begin{Entry}{设计}{6,4}{⾔,⾔}
  \begin{Phonetics}{设计}{she4ji4}[][HSK 3]
    \definition[份]{s.}{plano; esquema; refere"-se a um plano de design ou a um projeto para um plano, etc.}
    \definition{v.}{planejar; projetar; formular métodos, desenhos, etc. com antecedência, de acordo com determinados requisitos de finalidade, antes de iniciar oficialmente um trabalho | arquitetar; idear; tramar; fazer um plano}
  \end{Phonetics}
\end{Entry}

\begin{Entry}{设计师}{6,4,6}{⾔,⾔,⼱}
  \begin{Phonetics}{设计师}{she4ji4shi1}[][HSK 6]
    \definition[个,位,名,些]{s.}{planejador de projeto; designer | arquiteto}
  \end{Phonetics}
\end{Entry}

\begin{Entry}{设立}{6,5}{⾔,⽴}
  \begin{Phonetics}{设立}{she4li4}[][HSK 3]
    \definition{v.}{fundar; estabelecer; começar}
  \end{Phonetics}
\end{Entry}

\begin{Entry}{设备}{6,8}{⾔,⼡}
  \begin{Phonetics}{设备}{she4bei4}[][HSK 3]
    \definition[台,套]{s.}{instalação; equipamento; montagem; um conjunto de edifícios ou equipamentos necessários para executar uma determinada tarefa ou suprir uma determinada necessidade}
  \end{Phonetics}
\end{Entry}

\begin{Entry}{设定}{6,8}{⾔,⼧}
  \begin{Phonetics}{设定}{she4ding4}[][HSK 7-9]
    \definition{v.}{definir; configurar; instalar}
  \end{Phonetics}
\end{Entry}

\begin{Entry}{设法}{6,8}{⾔,⽔}
  \begin{Phonetics}{设法}{she4fa3}[][HSK 7-9]
    \definition{v.}{encontrar um jeito de; conseguir; fazer esforços para; significa tentar encontrar uma maneira (de fazer algo)}
  \end{Phonetics}
\end{Entry}

\begin{Entry}{设施}{6,9}{⾔,⽅}
  \begin{Phonetics}{设施}{she4shi1}[][HSK 4]
    \definition{s.}{facilidade; instalação; instituições, sistemas, organizações, edifícios, etc., estabelecidos para realizar um trabalho ou atender a uma necessidade}
  \end{Phonetics}
\end{Entry}

\begin{Entry}{设想}{6,13}{⾔,⼼}
  \begin{Phonetics}{设想}{she4xiang3}[][HSK 5]
    \definition[个,种]{s.}{plano provisório (ou ideia); (item, tipo) refere"-se a algo hipotético ou imaginário}
    \definition{v.}{imaginar; prever; conceber; supor | ter consideração por}
  \end{Phonetics}
\end{Entry}

\begin{Entry}{设置}{6,13}{⾔,⽹}
  \begin{Phonetics}{设置}{she4zhi4}[][HSK 4]
    \definition{v.}{estabelecer; colocar em prática; estabelecer ou criar instituições, empregos, profissões ou códigos, etc. | encaixar; ajustar; instalar; configurar; colocar}
  \end{Phonetics}
\end{Entry}

%%%%%%%%%% 访 %%%%%%%%%%
\subsection*{访}\addcontentsline{loh}{figure}{访}

\begin{Entry}{访}{6}{⾔}
  \begin{Phonetics}{访}{fang3}
    \definition{v.}{visitar; fazer uma visita; ligar para | procurar por meio de investigação ou busca; tentar obter; obter uma entrevista | entrevistar | investigar; procurar por meio de investigação (pesquisar)}
  \end{Phonetics}
\end{Entry}

\begin{Entry}{访问}{6,6}{⾔,⾨}
  \begin{Phonetics}{访问}{fang3wen4}[][HSK 3]
    \definition{v.}{visitar; ligar; entrevistar; visitar e conversar com um objetivo específico | visitar um \emph{site}}
  \end{Phonetics}
\end{Entry}

\begin{Entry}{访谈}{6,10}{⾔,⾔}
  \begin{Phonetics}{访谈}{fang3tan2}[][HSK 7-9]
    \definition{v.}{entrevistar; conversar; visitar e conversar}
  \end{Phonetics}
\end{Entry}

%%%%%%%%%% 诀 %%%%%%%%%%
\subsection*{诀}\addcontentsline{loh}{figure}{诀}

\begin{Entry}{诀}{6}{⾔}
  \begin{Phonetics}{诀}{jue2}
    \definition[条,个]{s.}{rimas; mnemônicos | jeito; truques do ofício; boas maneiras de resolver problemas}
    \definition{v.}{dar adeus; partir}
  \end{Phonetics}
\end{Entry}

\begin{Entry}{诀别}{6,7}{⾔,⼑}
  \begin{Phonetics}{诀别}{jue2bie2}[][HSK 7-9]
    \definition{v.}{se despedir | separar-se (geralmente com pouca esperança de um novo encontro)}
  \end{Phonetics}
\end{Entry}

\begin{Entry}{诀窍}{6,10}{⾔,⽳}
  \begin{Phonetics}{诀窍}{jue2qiao4}[][HSK 7-9]
    \definition{s.}{talento; habilidade;  segredo do sucesso; truques do ofício}
  \seealsoref{诀窍儿}{jue2qiao4r5}
  \end{Phonetics}
\end{Entry}

\begin{Entry}{诀窍儿}{6,10,2}{⾔,⽳,⼉}
  \begin{Phonetics}{诀窍儿}{jue2qiao4r5}
    \definition{s.}{segredo do sucesso; chave para o sucesso; truques do ofício; jeito para a coisa}
  \seealsoref{诀窍}{jue2qiao4}
  \end{Phonetics}
\end{Entry}

%%%%%%%%%% 负 %%%%%%%%%%
\subsection*{负}\addcontentsline{loh}{figure}{负}

\begin{Entry}{负}{6}{⾙}
  \begin{Phonetics}{负}{fu4}[][HSK 6]
    \definition{adj.}{negativo; menor que zero | negativo; referindo"-se ao que recebe elétrons}
    \definition{v.}{carregar; transportar nas costas ou nos ombros | suportar; assumir; encarar | confiar em; contar com; depender | sofrer | aproveitar; desfrutar | ter dívidas | trair; violar | perder; ser derrotado}
  \antonymref{正}{zheng4}
  \end{Phonetics}
\end{Entry}

\begin{Entry}{负有}{6,6}{⾙,⽉}
  \begin{Phonetics}{负有}{fu4you3}[][HSK 7-9]
    \definition{v.}{ser responsável por}
  \end{Phonetics}
\end{Entry}

\begin{Entry}{负担}{6,8}{⾙,⼿}
  \begin{Phonetics}{负担}{fu4dan1}[][HSK 4]
    \definition{s.}{carga; fardo; frete; ônus; pressão ou responsabilidade, despesas, etc.}
    \definition{v.}{carregar; carregar (um fardo); assumir (responsabilidade, trabalho, despesas, etc.)}
  \end{Phonetics}
\end{Entry}

\begin{Entry}{负责}{6,8}{⾙,⾙}
  \begin{Phonetics}{负责}{fu4ze2}[][HSK 3]
    \definition{adj.}{consciencioso; ser sério e responsável}
    \definition{v.}{ser responsável por; estar encarregado de; assumir responsabilidades}
  \end{Phonetics}
\end{Entry}

\begin{Entry}{负责人}{6,8,2}{⾙,⾙,⼈}
  \begin{Phonetics}{负责人}{fu4ze2ren2}[][HSK 5]
    \definition[位]{s.}{pessoa responsável; pessoa encarregada; pessoas com responsabilidades de liderança}
  \end{Phonetics}
\end{Entry}

\begin{Entry}{负面}{6,9}{⾙,⾯}
  \begin{Phonetics}{负面}{fu4mian4}[][HSK 7-9]
    \definition{adj.}{ruim; negativo; prejudicial; desvantajoso}
    \definition{s.}{lado reverso; o negativo; refere"-se aos aspectos ou partes ruins, negativas, prejudiciais ou desfavoráveis}
  \end{Phonetics}
\end{Entry}

%%%%%%%%%% 赱 %%%%%%%%%%
\subsection*{赱}\addcontentsline{loh}{figure}{赱}

\begin{Entry}{赱}{6}{⼟}
  \begin{Phonetics}{赱}{zou3}
    \variantof{走}
  \end{Phonetics}
\end{Entry}

%%%%%%%%%% 轨 %%%%%%%%%%
\subsection*{轨}\addcontentsline{loh}{figure}{轨}

\begin{Entry}{轨}{6}{⾞}
  \begin{Phonetics}{轨}{gui3}
    \definition{s.}{trilho; pista | curso; caminho | ordem; regulamento; regra | rotina; metaforicamente falando, métodos, regras, ordem, etc.}
    \definition{v.}{seguir | Literário: cumprir; aderir a}
  \end{Phonetics}
\end{Entry}

\begin{Entry}{轨迹}{6,9}{⾞,⾡}
  \begin{Phonetics}{轨迹}{gui3ji4}[][HSK 7-9]
    \definition{s.}{trilha; caminho; trajetória | órbita; caminho | trilha; pegada; uma metáfora para experiências de vida ou o caminho de desenvolvimento das coisas}
  \end{Phonetics}
\end{Entry}

\begin{Entry}{轨道}{6,12}{⾞,⾡}
  \begin{Phonetics}{轨道}{gui3dao4}[][HSK 6]
    \definition[条]{s.}{trilha; uma rota pavimentada com trilhos de aço para trens, bondes, etc. | órbita; trajetória; corpos celestes e objetos têm trajetórias de movimento regulares | caminho; curso; maneira adequada de fazer as coisas; curso adequado; uma metáfora para o desenvolvimento normal das coisas ou as normas e procedimentos que as pessoas devem seguir}
  \end{Phonetics}
\end{Entry}

%%%%%%%%%% 达 %%%%%%%%%%
\subsection*{达}\addcontentsline{loh}{figure}{达}

\begin{Entry}{达}{6}{⾡}
  \begin{Phonetics}{达}{da2}
    \definition*{s.}{Sobrenome: Da}
    \definition{adj.}{eminente; distinto; refere"-se a um funcionário distinto; \emph{status} elevado | otimista; de mente aberta}
    \definition{v.}{prolongar | alcançar; atingir; equivaler a | entender completamente; compreender (assuntos) | expressar; comunicar}
  \end{Phonetics}
\end{Entry}

\begin{Entry}{达成}{6,6}{⾡,⼽}
  \begin{Phonetics}{达成}{da2/cheng2}[][HSK 5]
    \definition{v.+compl.}{concluir; chegar (a um acordo); conseguir; obter (principalmente como resultado de uma negociação)}
  \end{Phonetics}
\end{Entry}

\begin{Entry}{达到}{6,8}{⾡,⼑}
  \begin{Phonetics}{达到}{da2/dao4}[][HSK 3]
    \definition{v.}{alcançar; atender o padrão; atingir (refere"-se principalmente a coisas abstratas ou graus); chegar a um determinado ponto ou grau}
  \end{Phonetics}
\end{Entry}

\begin{Entry}{达标}{6,9}{⾡,⽊}
  \begin{Phonetics}{达标}{da2biao1}[][HSK 7-9]
    \definition{v.}{atingir um padrão definido (até o padrão) | atingir um padrão definido (qualifica)}
  \end{Phonetics}
\end{Entry}

%%%%%%%%%% 迁 %%%%%%%%%%
\subsection*{迁}\addcontentsline{loh}{figure}{迁}

\begin{Entry}{迁}{6}{⾡}
  \begin{Phonetics}{迁}{qian1}[][HSK 7-9]
    \definition{v.}{mover algo para algum lugar; migrar | mudar}
  \end{Phonetics}
\end{Entry}

\begin{Entry}{迁移}{6,11}{⾡,⽲}
  \begin{Phonetics}{迁移}{qian1yi2}[][HSK 7-9]
    \definition{v.}{mover; migrar; mudar-se de sua localização original para outro lugar}
  \end{Phonetics}
\end{Entry}

\begin{Entry}{迁就}{6,12}{⾡,⼪}
  \begin{Phonetics}{迁就}{qian1jiu4}[][HSK 7-9]
    \definition{v.}{ceder a; acomodar-se a; atender aos interesses dos outros}
  \end{Phonetics}
\end{Entry}

%%%%%%%%%% 迄 %%%%%%%%%%
\subsection*{迄}\addcontentsline{loh}{figure}{迄}

\begin{Entry}{迄}{6}{⾡}
  \begin{Phonetics}{迄}{qi4}
    \definition{adv.}{até agora; ao longo de todo o processo; sempre; constantemente; usado antes de 未 ou 无}
    \definition{prep.}{até; para}
  \seealsoref{未}{wei4}
  \seealsoref{无}{wu2}
  \end{Phonetics}
\end{Entry}

\begin{Entry}{迄今}{6,4}{⾡,⼈}
  \begin{Phonetics}{迄今}{qi4jin1}[][HSK 7-9]
    \definition{adv.}{até agora; até o momento; até hoje}
  \end{Phonetics}
\end{Entry}

\begin{Entry}{迄今为止}{6,4,4,4}{⾡,⼈,⼂,⽌}
  \begin{Phonetics}{迄今为止}{qi4jin1-wei2zhi3}[][HSK 7-9]
    \definition{adv.}{até hoje; até agora; até o momento}
  \end{Phonetics}
\end{Entry}

%%%%%%%%%% 迅 %%%%%%%%%%
\subsection*{迅}\addcontentsline{loh}{figure}{迅}

\begin{Entry}{迅}{6}{⾡}
  \begin{Phonetics}{迅}{xun4}
    \definition{adj.}{rápido; veloz}
    \definition{adv.}{rapidamente; velozmente}
  \end{Phonetics}
\end{Entry}

\begin{Entry}{迅速}{6,10}{⾡,⾡}
  \begin{Phonetics}{迅速}{xun4su4}[][HSK 4]
    \definition{adv.}{rapidamente; velozmente; prontamente}
  \end{Phonetics}
\end{Entry}

%%%%%%%%%% 过 %%%%%%%%%%
\subsection*{过}\addcontentsline{loh}{figure}{过}

\begin{Entry}{过}{6}{⾡}
  \begin{Phonetics}{过}{guo1}
    \definition*{s.}{Sobrenome: Guo}
  \end{Phonetics}
  \begin{Phonetics}{过}{guo4}[][HSK 1,2]
    \definition{adv.}{excessivamente; em excesso}
    \definition{clas.}{tempo; número de vezes usado para a ação}
    \definition{s.}{falha; erro; demérito; equívoco; negligência}
    \definition{v.}{cruzar; passar; mudar-se de um lugar para outro; passar por | exceder; ir além; ultrapassar; usado após um adjetivo, significa ``mais do que'' | gastar (tempo); passar (tempo); exceder (um determinado limite ou limite) | celebrar; comemorar | mudar; transferir; transferir de um lado para o outro | passar por um processo; passar por; submeter a (algum tipo de tratamento) | visitar; fazer uma visita | falecer; morrer | infectar; ser contagioso; espalhar | exceder; ir além; usado após o verbo com o sufixo 得, significa ``superar'' ou ``passar'' | viver | revisar; examinar; usar os olhos para ver ou a mente para lembrar}
  \seealsoref{得}{de5}
  \antonymref{功}{gong1}
  \end{Phonetics}
  \begin{Phonetics}{过}{guo5}
    \definition{part.}{usado depois de um verbo para indicar conclusão | usado depois de um verbo para indicar que uma ação ou mudança ocorreu | usado depois de um adjetivo para indicar que algo já teve uma certa qualidade ou estado e para compará-lo com o presente}
  \end{Phonetics}
\end{Entry}

\begin{Entry}{过于}{6,3}{⾡,⼆}
  \begin{Phonetics}{过于}{guo4yu2}[][HSK 5]
    \definition{adv.}{demais; indevidamente; excessivamente; advérbios de grau ou quantidade excessiva}
  \end{Phonetics}
\end{Entry}

\begin{Entry}{过不去}{6,4,5}{⾡,⼀,⼛}
  \begin{Phonetics}{过不去}{guo4bu5qu4}[][HSK 7-9]
    \definition{v.}{não poder passar; ser incapaz de passar; ser ou estar bloqueado; ser intransitável | Coloquial: ser duro com; dificultar; envergonhar; colocar para fora | sentir pena; sentir-se mal | encontrar falhas em}
  \end{Phonetics}
\end{Entry}

\begin{Entry}{过不惯}{6,4,11}{⾡,⼀,⼼}
  \begin{Phonetics}{过不惯}{guo4 bu5 guan4}
    \definition{v.}{não se acostumar; não se habituar}
  \seealsoref{过惯}{guo4guan4}
  \end{Phonetics}
\end{Entry}

\begin{Entry}{过分}{6,4}{⾡,⼑}
  \begin{Phonetics}{过分}{guo4fen4}[][HSK 4]
    \definition{adj.}{excessivo; muito longe; demais; falar ou agir além dos limites ou graus adequados}
    \definition{adv.}{excessivamente; indevidamente; muito mesmo}
  \end{Phonetics}
\end{Entry}

\begin{Entry}{过日子}{6,4,3}{⾡,⽇,⼦}
  \begin{Phonetics}{过日子}{guo4 ri4zi5}[][HSK 7-9]
    \definition{v.}{viver; conviver; passar/viver a própria vida}
  \end{Phonetics}
\end{Entry}

\begin{Entry}{过半}{6,5}{⾡,⼗}
  \begin{Phonetics}{过半}{guo4ban4}[][HSK 7-9]
    \definition{s.}{maioria; mais da metade; mais de cinquenta por cento}
  \end{Phonetics}
\end{Entry}

\begin{Entry}{过去}{6,5}{⾡,⼛}
  \begin{Phonetics}{过去}{guo4 qu5}[][HSK 2,3]
    \definition{adv.}{(no) passado}
    \definition{s.}{o passado; refere"-se a um período anterior; também se refere a coisas anteriores}
    \definition{v.}{atravessar; passar; sair do local onde o interlocutor se encontra e deslocar"-se para outro local | acabar; passar; ficar para trás; indica que já passou por uma determinada fase | passar; indica que um determinado período ou situação já não existe mais | falecer | ir lá | passar por}
  \end{Phonetics}
\end{Entry}

\begin{Entry}{过失}{6,5}{⾡,⼤}
  \begin{Phonetics}{过失}{guo4shi1}[][HSK 7-9]
    \definition{s.}{falha; deslize; erro; erros cometidos por negligência | negligência; crime por negligência}
  \end{Phonetics}
\end{Entry}

\begin{Entry}{过头}{6,5}{⾡,⼤}
  \begin{Phonetics}{过头}{guo4/tou2}[][HSK 7-9]
    \definition{adv.}{excessivamente; acima da cabeça; por cima; ao alto}
    \definition{v.+compl.}{exagerar; ir além do limite; exceder o limite; ser excessivo}
  \end{Phonetics}
\end{Entry}

\begin{Entry}{过节}{6,5}{⾡,⾋}
  \begin{Phonetics}{过节}{guo4/jie2}[][HSK 7-9]
    \definition{v.+compl.}{celebrar um festival; passar as férias; comemorar durante as férias}[今年我们一起过节吧!===Vamos comemorar as festas juntos este ano!]
  \end{Phonetics}
\end{Entry}

\begin{Entry}{过关}{6,6}{⾡,⼋}
  \begin{Phonetics}{过关}{guo4/guan1}[][HSK 7-9]
    \definition{v.+compl.}{passar (um teste); alcançar (um padrão); cruzar uma barreira; superar (uma provação) ; passar por um posto de controle, frequentemente usado como metáfora}
  \end{Phonetics}
\end{Entry}

\begin{Entry}{过后}{6,6}{⾡,⼝}
  \begin{Phonetics}{过后}{guo4hou4}[][HSK 6]
    \definition[期]{s.}{depois; mais tarde}
  \end{Phonetics}
\end{Entry}

\begin{Entry}{过年}{6,6}{⾡,⼲}
  \begin{Phonetics}{过年}{guo4/nian2}[][HSK 2]
    \definition{v.+compl.}{comemorar o Ano Novo; comemorar o Festival da Primavera; passar o Ano Novo; passar o Festival da Primavera; realizar atividades comemorativas durante o Ano Novo ou o Festival da Primavera}
  \end{Phonetics}
\end{Entry}

\begin{Entry}{过早}{6,6}{⾡,⽇}
  \begin{Phonetics}{过早}{guo4 zao3}[][HSK 7-9]
    \definition{adj.}{prematuro; inoportuno | Dialeto: café da manha}
    \definition{adv.}{muito cedo; prematuramente | Dialeto: tomar o café da manhã}
  \end{Phonetics}
\end{Entry}

\begin{Entry}{过时}{6,7}{⾡,⽇}
  \begin{Phonetics}{过时}{guo4 shi2}[][HSK 6]
    \definition{adj.}{fora de moda; obsoleto; antiquado; desatualizado; o que era popular no passado não é mais popular}
    \definition{v.}{passar do tempo marcado; estar após o tempo estipulado}
  \end{Phonetics}
\end{Entry}

\begin{Entry}{过来}{6,7}{⾡,⽊}
  \begin{Phonetics}{过来}{guo4/lai2}[][HSK 2]
    \definition{v.+compl.}{vir até aqui | ser capaz de cuidar de | lidar com | administrar}
  \end{Phonetics}
\end{Entry}

\begin{Entry}{过往}{6,8}{⾡,⼻}
  \begin{Phonetics}{过往}{guo4wang3}[][HSK 7-9]
    \definition{v.}{ir e vir | ter relações amigáveis com; associar"-se a; lidar com}
  \end{Phonetics}
\end{Entry}

\begin{Entry}{过奖}{6,9}{⾡,⼤}
  \begin{Phonetics}{过奖}{guo4jiang3}[][HSK 7-9]
    \definition{v.}{elogiar demais; bajular; dar elogios imerecidos}
  \end{Phonetics}
\end{Entry}

\begin{Entry}{过度}{6,9}{⾡,⼴}
  \begin{Phonetics}{过度}{guo4du4}[][HSK 5]
    \definition{adj.}{excessivo; acima do limite; além do limite; além do que é apropriado}
  \end{Phonetics}
\end{Entry}

\begin{Entry}{过屠门而大嚼}{6,11,3,6,3,20}{⾡,⼫,⾨,⽽,⼤,⼝}
  \begin{Phonetics}{过屠门而大嚼}{guo4 tu2men2 er2 da4 jiao2}
    \definition{expr.}{``Passar pelo portão do açougueiro e comer com apetite.''; essa metáfora descreve alguém que admira algo, mas não pode tê-lo, e usa métodos irreais para se consolar; alimente-se de ilusões; tem significado negativo}
  \end{Phonetics}
\end{Entry}

\begin{Entry}{过惯}{6,11}{⾡,⼼}
  \begin{Phonetics}{过惯}{guo4guan4}
    \definition{v.}{estar acostumado (a um certo estilo de vida, etc.)}
  \seealsoref{过不惯}{guo4 bu5 guan4}
  \end{Phonetics}
\end{Entry}

\begin{Entry}{过敏}{6,11}{⾡,⽁}
  \begin{Phonetics}{过敏}{guo4min3}[][HSK 5]
    \definition{adj.}{sensível; excessivamente sensível; resposta acima do normal; ceticismo excessivo}
    \definition{v.}{ser alérgico a}
  \end{Phonetics}
\end{Entry}

\begin{Entry}{过剩}{6,12}{⾡,⼑}
  \begin{Phonetics}{过剩}{guo4sheng4}[][HSK 7-9]
    \definition{v.}{exceder; a quantidade excede em muito o limite necessário | saturar; oferecer em excesso; a oferta excede a demanda do mercado ou o poder de compra}
  \end{Phonetics}
\end{Entry}

\begin{Entry}{过期}{6,12}{⾡,⽉}
  \begin{Phonetics}{过期}{guo4/qi1}[][HSK 7-9]
    \definition{v.+compl.}{expirar; estar vencido; exceder o limite de tempo; exceder o período prescrito ou acordado}
  \end{Phonetics}
\end{Entry}

\begin{Entry}{过渡}{6,12}{⾡,⽔}
  \begin{Phonetics}{过渡}{guo4du4}[][HSK 6]
    \definition{v.}{fazer a transição; estar em transição; estar em fase de transição; mudar de um estágio para outro | atravessar; cruzar}
  \end{Phonetics}
\end{Entry}

\begin{Entry}{过硬}{6,12}{⾡,⽯}
  \begin{Phonetics}{过硬}{guo4/ying4}[][HSK 7-9]
    \definition{adj.}{perfeito; soberbo; à altura; verdadeiramente proficiente}
    \definition{v.+compl.}{ter domínio perfeito de algo; estar à altura; resistir a testes ou exames rigorosos}
  \end{Phonetics}
\end{Entry}

\begin{Entry}{过程}{6,12}{⾡,⽲}
  \begin{Phonetics}{过程}{guo4cheng2}[][HSK 3]
    \definition[个,段]{s.}{curso dos eventos; processo; o processo pelo qual as coisas acontecem ou se desenvolvem.}
  \end{Phonetics}
\end{Entry}

\begin{Entry}{过道}{6,12}{⾡,⾡}
  \begin{Phonetics}{过道}{guo4dao4}[][HSK 7-9]
    \definition{s.}{corredor; caminho; passarela; passagem; o corredor da porta para cada cômodo da nova casa | passagem; uma passarela que conecta os pátios de uma casa antiga, especialmente o cômodo ou metade do cômodo onde o portão está localizado}
  \end{Phonetics}
\end{Entry}

\begin{Entry}{过意不去}{6,13,4,5}{⾡,⼼,⼀,⼛}
  \begin{Phonetics}{过意不去}{guo4yi4bu2qu4}[][HSK 7-9]
    \definition{expr.}{sentir-se arrependido ou culpado; sentir-se mal ou envergonhado; sentir-se envergonhado ou arrependido; metáfora para aceitar um favor de alguém, mas não retribuí-lo, ou sentir pena de algo e não ser culpado, o que faz com que alguém se sinta arrependido e desconfortável}
  \end{Phonetics}
\end{Entry}

\begin{Entry}{过滤}{6,13}{⾡,⽔}
  \begin{Phonetics}{过滤}{guo4lv4}[][HSK 7-9]
    \definition{v.}{filtrar; separar sólidos ou componentes nocivos de gases ou líquidos por meio de materiais porosos, como papel de filtro e pano de filtro}[所有饮用水必须经过过滤。===Toda água potável deve ser filtrada.]
  \end{Phonetics}
\end{Entry}

\begin{Entry}{过路人}{6,13,2}{⾡,⾜,⼈}
  \begin{Phonetics}{过路人}{guo4lu4 ren2}
    \definition{s.}{transeunte}
  \end{Phonetics}
\end{Entry}

\begin{Entry}{过错}{6,13}{⾡,⾦}
  \begin{Phonetics}{过错}{guo4cuo4}[][HSK 7-9]
    \definition{s.}{falha; erro; engano | ações ilícitas; no direito civil, refere"-se a atos ilegais que prejudicam outras pessoas intencionalmente ou negligentemente}
  \end{Phonetics}
\end{Entry}

\begin{Entry}{过境}{6,14}{⾡,⼟}
  \begin{Phonetics}{过境}{guo4/jing4}[][HSK 7-9]
    \definition{v.+compl.}{estar em trânsito; passar pelo território de um país}
  \end{Phonetics}
\end{Entry}

\begin{Entry}{过瘾}{6,16}{⾡,⽧}
  \begin{Phonetics}{过瘾}{guo4/yin3}[][HSK 7-9]
    \definition{adj.}{gratificante; imensamente agradável; satisfatório; realizador}
    \definition{v.+compl.}{satisfazer um desejo; divertir-se ao máximo; fazer algo à vontade}
  \end{Phonetics}
\end{Entry}

%%%%%%%%%% 迈 %%%%%%%%%%
\subsection*{迈}\addcontentsline{loh}{figure}{迈}

\begin{Entry}{迈}{6}{⾡}
  \begin{Phonetics}{迈}{mai4}[][HSK 7-9]
    \definition{adj.}{velho; idoso}
    \definition{clas.}{milha}
    \definition{v.}{dar um passo; passar; avançar; levantar o pé e caminhar para a frente; dar um passo largo}
  \end{Phonetics}
\end{Entry}

\begin{Entry}{迈进}{6,7}{⾡,⾡}
  \begin{Phonetics}{迈进}{mai4jin4}[][HSK 7-9]
    \definition{v.}{avançar; seguir em frente; prosseguir com passos largos}
  \end{Phonetics}
\end{Entry}

%%%%%%%%%% 那 %%%%%%%%%%
\subsection*{那}\addcontentsline{loh}{figure}{那}

\begin{Entry}{那}{6}{⾢}
  \begin{Phonetics}{那}{na1}
    \definition*{s.}{Sobrenome: Na}
  \end{Phonetics}
  \begin{Phonetics}{那}{na3}
    \definition{adv.}{expressa negação em perguntas retóricas}
    \definition{pron.}{qual? | qualquer que seja; qualquer que; para expressar incerteza em uma declaração | variante de 哪}
  \seealsoref{哪}{na3}
  \end{Phonetics}
  \begin{Phonetics}{那}{na4}[][HSK 1,2]
    \definition{conj.}{então; nessa situação; nesse caso; o mesmo que 那么}
    \definition{pron.}{aquele; aquilo; indica pessoas ou coisas distantes | aquele; aquilo; expressa muitas coisas, sem se referir especificamente a uma pessoa ou coisa, e é frequentemente usado em conjunto com 这}
  \seealsoref{那么}{na4me5}
  \seealsoref{这}{zhe4}
  \end{Phonetics}
  \begin{Phonetics}{那}{ne4}
    \definition{conj.}{então; nesse caso; o mesmo que 那么}
    \definition{pron.}{aquele; aquilo; pronúncia coloquial de 那 (\dpy{na4})}
  \seealsoref{那么}{na4me5}
  \end{Phonetics}
  \begin{Phonetics}{那}{nei4}
    \definition{conj.}{então; o mesmo que 那么}
    \definition{pron.}{aquele; aquilo; A pronúncia coloquial de 那 (\dpy{na4})}
  \seealsoref{那么}{na4me5}
  \end{Phonetics}
  \begin{Phonetics}{那}{nuo2}
    \definition*{s.}{Sobrenome: Nuo}
  \end{Phonetics}
\end{Entry}

\begin{Entry}{那儿}{6,2}{⾢,⼉}
  \begin{Phonetics}{那儿}{na4r5}[][HSK 1]
    \definition{pron.}{lá; ali; naquele lugar | então; naquela época (usado após 打, 从 e 由)}
  \seealsoref{从}{cong2}
  \seealsoref{打}{da3}
  \seealsoref{由}{you2}
  \end{Phonetics}
\end{Entry}

\begin{Entry}{那个}{6,3}{⾢,⼈}
  \begin{Phonetics}{那个}{na4ge5}
    \definition{pron.}{aquele | usado antes de verbos e adjetivos para indicar exagero | para substituir o discurso direto inconveniente}
  \end{Phonetics}
\end{Entry}

\begin{Entry}{那么}{6,3}{⾢,⼃}
  \begin{Phonetics}{那么}{na4me5}[][HSK 2]
    \definition{conj.}{então; nesse caso; afirmar o resultado esperado ou fazer um julgamento}
    \definition{pron.}{assim; dessa maneira; indica a natureza, o estado, a forma, o grau, etc. | assim; sobre; colocado antes do numeral, indica uma estimativa}
  \end{Phonetics}
\end{Entry}

\begin{Entry}{那边}{6,5}{⾢,⾡}
  \begin{Phonetics}{那边}{na4bian5}[][HSK 1]
    \definition{pron.}{ali; acolá; aquele lado}
  \end{Phonetics}
\end{Entry}

\begin{Entry}{那会儿}{6,6,2}{⾢,⼈,⼉}
  \begin{Phonetics}{那会儿}{na4hui4r5}[][HSK 2]
    \definition{pron.}{então; naquela época; refere"-se ao passado ou ao futuro}
  \end{Phonetics}
\end{Entry}

\begin{Entry}{那时}{6,7}{⾢,⽇}
  \begin{Phonetics}{那时}{na4 shi2}
    \definition{pron.}{então; naquela época; naqueles dias; geralmente se refere a um período de tempo distante do presente}
  \seealsoref{那时候}{na4 shi2hou5}
  \end{Phonetics}
\end{Entry}

\begin{Entry}{那时候}{6,7,10}{⾢,⽇,⼈}
  \begin{Phonetics}{那时候}{na4 shi2hou5}[][HSK 2]
    \definition{adv.}{naquela hora; em algum momento no passado}
  \seealsoref{那时}{na4 shi2}
  \end{Phonetics}
\end{Entry}

\begin{Entry}{那里}{6,7}{⾢,⾥}
  \begin{Phonetics}{那里}{na4li3}[][HSK 1]
    \definition{pron./s.}{lá; ali; aquele lugar; indica um lugar distante}
  \end{Phonetics}
\end{Entry}

\begin{Entry}{那些}{6,8}{⾢,⼆}
  \begin{Phonetics}{那些}{na4xie1}[][HSK 1]
    \definition{pron.}{aqueles; indica duas ou mais pessoas ou coisas}
  \end{Phonetics}
\end{Entry}

\begin{Entry}{那咱}{6,9}{⾢,⼝}
  \begin{Phonetics}{那咱}{na4 zan5}
    \definition{s.}{(informal) naquela época; então | (antigo) naquela época}
  \end{Phonetics}
\end{Entry}

\begin{Entry}{那样}{6,10}{⾢,⽊}
  \begin{Phonetics}{那样}{na4yang4}[][HSK 2]
    \definition{pron.}{assim; tal; desse tipo; desse gênero; dessa natureza; desse tipo; indica a natureza, o estado, a maneira, o grau ou refere"-se a uma ação ou situação específica}
  \end{Phonetics}
\end{Entry}

\begin{Entry}{那麽}{6,14}{⾢,⿇}
  \begin{Phonetics}{那麽}{na4me5}
    \variantof{那么}
  \end{Phonetics}
\end{Entry}

%%%%%%%%%% 闭 %%%%%%%%%%
\subsection*{闭}\addcontentsline{loh}{figure}{闭}

\begin{Entry}{闭}{6}{⾨}
  \begin{Phonetics}{闭}{bi4}[][HSK 6]
    \definition*{s.}{Sobrenome: Bi}
    \definition{v.}{fechar; encerrar | bloquear; obstruir; parar}
  \end{Phonetics}
\end{Entry}

\begin{Entry}{闭幕}{6,13}{⾨,⼱}
  \begin{Phonetics}{闭幕}{bi4/mu4}[][HSK 5]
    \definition{v.+compl.}{fechar; concluir; (conferência, exposição, etc.) terminar | cair a cortina; abaixar a cortina; terminar a apresentação e a cortina se fechar em frente ao palco}
  \end{Phonetics}
\end{Entry}

\begin{Entry}{闭幕式}{6,13,6}{⾨,⼱,⼷}
  \begin{Phonetics}{闭幕式}{bi4mu4shi4}[][HSK 5]
    \definition{s.}{cerimônia de encerramento; cerimônia formal realizada no final de uma conferência ou exposição}
  \end{Phonetics}
\end{Entry}

\begin{Entry}{闭嘴}{6,16}{⾨,⼝}
  \begin{Phonetics}{闭嘴}{bi4zui3}
    \definition{expr.}{``Cale"-se!''; ``Pare de falar!''}
  \end{Phonetics}
\end{Entry}

%%%%%%%%%% 问 %%%%%%%%%%
\subsection*{问}\addcontentsline{loh}{figure}{问}

\begin{Entry}{问}{6}{⾨}
  \begin{Phonetics}{问}{wen4}[][HSK 1]
    \definition*{s.}{Sobrenome: Wen}
    \definition{prep.}{de; introduzir o objeto da ação, equivalente a 向 e 跟}
    \definition{v.}{perguntar; indagar; fazer com que as pessoas respondam ou esclareçam coisas que não sabem ou não têm certeza | perguntar (ou indagar) sobre | examinar; interrogar | intervir; responsabilizar; investigar | cuidar; preocupar-se; gerenciar; interferir}
  \seealsoref{跟}{gen1}
  \seealsoref{向}{xiang4}
  \end{Phonetics}
\end{Entry}

\begin{Entry}{问市}{6,5}{⾨,⼱}
  \begin{Phonetics}{问市}{wen4shi4}
    \definition{v.}{chegar ao mercado | bater o mercado | atingir o mercado}
  \end{Phonetics}
\end{Entry}

\begin{Entry}{问安}{6,6}{⾨,⼧}
  \begin{Phonetics}{问安}{wen4'an1}
    \definition{s.}{saudações}
    \definition{v.}{dar cumprimentos a | prestar homenagem}
  \end{Phonetics}
\end{Entry}

\begin{Entry}{问卷}{6,8}{⾨,⼙}
  \begin{Phonetics}{问卷}{wen4juan4}
    \definition[份]{s.}{questionário}
  \end{Phonetics}
\end{Entry}

\begin{Entry}{问候}{6,10}{⾨,⼈}
  \begin{Phonetics}{问候}{wen4hou4}[][HSK 4]
    \definition{v.}{prestar homenagem; enviar uma saudação;  dar os respeitos (cumprimentos) a alguém}
  \end{Phonetics}
\end{Entry}

\begin{Entry}{问鼎}{6,12}{⾨,⿍}
  \begin{Phonetics}{问鼎}{wen4ding3}
    \definition{v.}{visar (o primeiro lugar, etc.) | aspirar ao trono}
  \end{Phonetics}
\end{Entry}

\begin{Entry}{问路}{6,13}{⾨,⾜}
  \begin{Phonetics}{问路}{wen4 lu4}[][HSK 2]
    \definition{v.}{perguntar o caminho; pedir direções}
  \end{Phonetics}
\end{Entry}

\begin{Entry}{问题}{6,15}{⾨,⾴}
  \begin{Phonetics}{问题}{wen4ti2}[][HSK 2]
    \definition{adj.}{desqualificado; indesejável; anormal, não atende aos requisitos}
    \definition[个,种,类,串]{s.}{pergunta; problema; perguntas a serem respondidas | problema; questão; contradições que precisam ser estudadas e resolvidas | problema; acidente; incidente | chave; ponto crucial; pontos importantes}
  \end{Phonetics}
\end{Entry}

%%%%%%%%%% 闯 %%%%%%%%%%
\subsection*{闯}\addcontentsline{loh}{figure}{闯}

\begin{Entry}{闯}{6}{⾨}
  \begin{Phonetics}{闯}{chuang3}[][HSK 5]
    \definition*{s.}{Sobrenome: Chuang}
    \definition{v.}{apressar"-se; correr | moderar a si mesmo (lutando contra dificuldades e perigos); aventurar"-se no mundo | incorrer; causar (um desastre, etc.)}
  \end{Phonetics}
\end{Entry}

%%%%%%%%%% 防 %%%%%%%%%%
\subsection*{防}\addcontentsline{loh}{figure}{防}

\begin{Entry}{防}{6}{⾩}
  \begin{Phonetics}{防}{fang2}[][HSK 3]
    \definition*{s.}{Sobrenome: Fang}
    \definition{s.}{defesa | dique; aterro | barragem; represa; estrutura para conter a água}
    \definition{v.}{proteger contra; prevenir contra; tomar precauções contra | defender-se contra}
  \end{Phonetics}
\end{Entry}

\begin{Entry}{防卫}{6,3}{⾩,⼙}
  \begin{Phonetics}{防卫}{fang2wei4}[][HSK 7-9]
    \definition{v.}{defender}
  \end{Phonetics}
\end{Entry}

\begin{Entry}{防止}{6,4}{⾩,⽌}
  \begin{Phonetics}{防止}{fang2zhi3}[][HSK 3]
    \definition{v.}{evitar; prevenir; prevenir; proteger contra; preparar-se com antecedência para evitar que coisas ruins aconteçam}
  \end{Phonetics}
\end{Entry}

\begin{Entry}{防火墙}{6,4,14}{⾩,⽕,⼟}
  \begin{Phonetics}{防火墙}{fang2huo3qiang2}[][HSK 7-9]
    \definition[个,堵]{s.}{\emph{firewall} (Internet)}
  \end{Phonetics}
\end{Entry}

\begin{Entry}{防守}{6,6}{⾩,⼧}
  \begin{Phonetics}{防守}{fang2shou3}[][HSK 6]
    \definition{v.}{defender; guardar}
  \end{Phonetics}
\end{Entry}

\begin{Entry}{防汛}{6,6}{⾩,⽔}
  \begin{Phonetics}{防汛}{fang2xun4}[][HSK 7-9]
    \definition{s.}{prevenção ou controle de inundações | controle de enchentes}
  \end{Phonetics}
\end{Entry}

\begin{Entry}{防护}{6,7}{⾩,⼿}
  \begin{Phonetics}{防护}{fang2hu4}[][HSK 7-9]
    \definition{s.}{abrigo; proteção; meios ou medidas para proteger pessoas, coisas, o meio ambiente, etc. de adoecer de receber danos ou destruição}
    \definition{v.}{abrigar; proteger; proteger pessoas, coisas e o meio ambiente de doenças, danos ou destruição}
  \end{Phonetics}
\end{Entry}

\begin{Entry}{防治}{6,8}{⾩,⽔}
  \begin{Phonetics}{防治}{fang2zhi4}[][HSK 5]
    \definition{s.}{tratamento preventivo; prevenção e cura; profilaxia e tratamento}
  \end{Phonetics}
\end{Entry}

\begin{Entry}{防疫}{6,9}{⾩,⽧}
  \begin{Phonetics}{防疫}{fang2yi4}[][HSK 7-9]
    \definition{s.}{prevenção de epidemias | prevenção de doenças | proteção contra epidemias}
  \end{Phonetics}
\end{Entry}

\begin{Entry}{防范}{6,9}{⾩,⾋}
  \begin{Phonetics}{防范}{fang2fan4}[][HSK 6]
    \definition{v.}{vigiar; estar em guarda; ficar de olho}
  \end{Phonetics}
\end{Entry}

\begin{Entry}{防晒}{6,10}{⾩,⽇}
  \begin{Phonetics}{防晒}{fang2shai4}
    \definition{s.}{protetor solar}
  \end{Phonetics}
\end{Entry}

\begin{Entry}{防盗}{6,11}{⾩,⽫}
  \begin{Phonetics}{防盗}{fang2dao4}[][HSK 7-9]
    \definition{v.}{proteger-se contra roubos; tomar precauções contra ladrões; impedir que bandidos roubem}
  \end{Phonetics}
\end{Entry}

\begin{Entry}{防盗门}{6,11,3}{⾩,⽫,⾨}
  \begin{Phonetics}{防盗门}{fang2dao4men2}[][HSK 7-9]
    \definition{s.}{porta de segurança; equipado com trava antirroubo, ela resiste à abertura anormal sob certas condições e por um período determinado | porta à prova de roubo}
  \end{Phonetics}
\end{Entry}

\begin{Entry}{防御}{6,12}{⾩,⼻}
  \begin{Phonetics}{防御}{fang2yu4}[][HSK 7-9]
    \definition{v.}{guardar; defender; resistir ao ataque do inimigo}
  \end{Phonetics}
\end{Entry}

%%%%%%%%%% 阳 %%%%%%%%%%
\subsection*{阳}\addcontentsline{loh}{figure}{阳}

\begin{Entry}{阳}{6}{⾩}
  \begin{Phonetics}{阳}{yang2}
    \definition*{s.}{Sobrenome: Yang}
    \definition{adj.}{em relevo | aberto; evidente; revelado | pertencente a este mundo; preocupado com os seres vivos; superstição se refere a coisas que pertencem aos vivos e ao mundo | positivo; carregado positivamente}
    \definition{s.}{o Sol; luz solar | ao sul de uma colina ou ao norte de um rio | Yang (o princípio masculino ou positivo da natureza); na filosofia chinesa antiga, refere"-se a um dos dois opostos que permeiam todas as coisas no universo (o outro lado é Yin) | masculino; refere"-se aos órgãos genitais masculinos ou à função reprodutiva | varanda; refere"-se ao lugar onde o sol brilha}
  \seealsoref{阴阳}{yin1yang2}
  \antonymref{阴}{yin1}
  \end{Phonetics}
\end{Entry}

\begin{Entry}{阳台}{6,5}{⾩,⼝}
  \begin{Phonetics}{阳台}{yang2tai2}[][HSK 4]
    \definition[个,块,处]{s.}{varanda; terraço; sacada; pequeno terraço do edifício com grades para se refrescar, tomar sol ou olhar o horizonte}
  \end{Phonetics}
\end{Entry}

\begin{Entry}{阳光}{6,6}{⾩,⼉}
  \begin{Phonetics}{阳光}{yang2guang1}[][HSK 3]
    \definition{adj.}{alegre; otimista; personalidade positiva e alegre; cheio de vitalidade juvenil | aberto; transparente; público; conduzido sob supervisão pública}
    \definition[缕,束,道]{s.}{luz do sol; raio de sol}
  \end{Phonetics}
\end{Entry}

%%%%%%%%%% 阴 %%%%%%%%%%
\subsection*{阴}\addcontentsline{loh}{figure}{阴}

\begin{Entry}{阴}{6}{⾩}
  \begin{Phonetics}{阴}{yin1}[][HSK 2]
    \definition*{s.}{Yin, o princípio negativo de Yin e Yang | A Lua; refere"-se a Taiyin | Sobrenome: Yin}
    \definition{adj.}{nublado; opaco; sombrio | escondido; secreto; não exposto | sinistro | do mundo inferior; dos fantasmas | Física: negativo; cátodo | nublado; mais de 80\% do céu estão cobertos por nuvens | em talhe"-doce; rebaixado | (matéria) carregada negativamente}
    \definition[片]{s.}{sombra; lugar sombrio | partes íntimas (especialmente da mulher) | ao norte de uma colina ou ao sul de um rio | verso | entalhe}
  \seealsoref{阳}{yang2}
  \seealsoref{阴阳}{yin1yang2}
  \end{Phonetics}
\end{Entry}

\begin{Entry}{阴天}{6,4}{⾩,⼤}
  \begin{Phonetics}{阴天}{yin1tian1}[][HSK 2]
    \definition[个]{s.}{nublado; céu nublado; dia nublado; uma condição climática em que 80\% do céu está coberto por nuvens e apenas um pouco de sol pode ser visto}
  \end{Phonetics}
\end{Entry}

\begin{Entry}{阴阳}{6,6}{⾩,⾩}
  \begin{Phonetics}{阴阳}{yin1yang2}
    \definition*{s.}{Yin e Yang}
  \seealsoref{阳}{yang2}
  \seealsoref{阴}{yin1}
  \end{Phonetics}
\end{Entry}

\begin{Entry}{阴谋}{6,11}{⾩,⾔}
  \begin{Phonetics}{阴谋}{yin1mou2}[][HSK 6]
    \definition[个,场,起]{s.}{trama; conspiração; um esquema para fazer o mal em segredo}
    \definition{v.}{tramar; conspirar secretamente (fazer algo ruim)}
  \end{Phonetics}
\end{Entry}

\begin{Entry}{阴影}{6,15}{⾩,⼺}
  \begin{Phonetics}{阴影}{yin1ying3}[][HSK 6]
    \definition{s.}{sombra; sombra escura | uma analogia de elementos negativos em negócios, relacionamentos, estado mental, etc.}
  \end{Phonetics}
\end{Entry}

%%%%%%%%%% 阵 %%%%%%%%%%
\subsection*{阵}\addcontentsline{loh}{figure}{阵}

\begin{Entry}{阵}{6}{⾩}
  \begin{Phonetics}{阵}{zhen4}[][HSK 4]
    \definition{clas.}{um parágrafo que mostra o processo de um evento ou ação}
    \definition{s.}{matriz de batalha (formação); termo tático antigo para as fileiras ou formações de uma equipe de combate | \emph{front}; frente de batalha; posição | um período de tempo}
  \end{Phonetics}
\end{Entry}

\begin{Entry}{阵地}{6,6}{⾩,⼟}
  \begin{Phonetics}{阵地}{zhen4di4}
    \definition{s.}{posição (militar) | frente de batalha | \emph{front}}
  \end{Phonetics}
\end{Entry}

%%%%%%%%%% 阶 %%%%%%%%%%
\subsection*{阶}\addcontentsline{loh}{figure}{阶}

\begin{Entry}{阶}{6}{⾩}
  \begin{Phonetics}{阶}{jie1}
    \definition{s.}{degrau; escada; escadaria | classificação | escala | ordem | estágio}
  \end{Phonetics}
\end{Entry}

\begin{Entry}{阶级}{6,6}{⾩,⽷}
  \begin{Phonetics}{阶级}{jie1ji2}[][HSK 7-9]
    \definition[个,种]{s.}{classe (social); grupos sociais divididos de acordo com o \emph{status} socioeconômico das pessoas | degraus; escadas; passos | classificação; número de passos}
  \end{Phonetics}
\end{Entry}

\begin{Entry}{阶层}{6,7}{⾩,⼫}
  \begin{Phonetics}{阶层}{jie1ceng2}[][HSK 7-9]
    \definition{s.}{posição; seção; estrato (social); isso se refere à estratificação dentro da mesma classe com base em diferentes status socioeconômicos, como a divisão da classe camponesa em camponeses pobres, camponeses médios, etc.}
  \end{Phonetics}
\end{Entry}

\begin{Entry}{阶段}{6,9}{⾩,⽎}
  \begin{Phonetics}{阶段}{jie1duan4}[][HSK 4]
    \definition[个,段]{s.}{estágio; fase; período; bancada; gradação}
  \end{Phonetics}
\end{Entry}

\begin{Entry}{阶梯}{6,11}{⾩,⽊}
  \begin{Phonetics}{阶梯}{jie1ti1}[][HSK 7-9]
    \definition{s.}{lance de escadas; escada; degraus e escadas são metáforas para meios ou caminhos de ascensão social; equipamentos que funcionam de maneira semelhante a escadas}
  \end{Phonetics}
\end{Entry}

%%%%%%%%%% 页 %%%%%%%%%%
\subsection*{页}\addcontentsline{loh}{figure}{页}

\begin{Entry}{页}{6}{⾴}[Kangxi 181]
  \begin{Phonetics}{页}{ye4}[][HSK 1]
    \definition{clas.}{página; folha de papel; lâmina; antigamente, referia"-se a uma folha de um livro encadernado; atualmente, refere"-se a uma das faces de um livro impresso em ambos os lados}
    \definition{s.}{página; folha de papel; folhas soltas de um livro}
  \end{Phonetics}
\end{Entry}

%%%%%%%%%% 驮 %%%%%%%%%%
\subsection*{驮}\addcontentsline{loh}{figure}{驮}

\begin{Entry}{驮}{6}{⾺}
  \begin{Phonetics}{驮}{duo4}
    \definition{s.}{uma carga transportada por um animal de carga; mochila; mercadorias transportadas por animais}
  \end{Phonetics}
  \begin{Phonetics}{驮}{tuo2}[][HSK 7-9]
    \definition{v.}{carregar nas costas; apoiar objetos com as costas; suportar nas costas}
  \end{Phonetics}
\end{Entry}

%%%%%%%%%% 驰 %%%%%%%%%%
\subsection*{驰}\addcontentsline{loh}{figure}{驰}

\begin{Entry}{驰}{6}{⾺}
  \begin{Phonetics}{驰}{chi2}
    \definition*{s.}{Sobrenome: Chi}
    \definition{v.}{(veículos, carruagens, cavalos, etc.) acelerar; galopar; fazer correr muito rápido | espalhar | (pensamentos) voltar"-se ansiosamente para; vire"-se ansiosamente para}
  \end{Phonetics}
\end{Entry}

\begin{Entry}{驰名}{6,6}{⾺,⼝}
  \begin{Phonetics}{驰名}{chi2ming2}[][HSK 7-9]
    \definition{s.}{famoso; bem conhecido; renomado}
    \definition{v.}{tornar"-se famoso (celebrado): conhecido em toda parte}
  \end{Phonetics}
\end{Entry}

%%%%%%%%%% 齐 %%%%%%%%%%
\subsection*{齐}\addcontentsline{loh}{figure}{齐}

\begin{Entry}{齐}{6}{⿑}[Kangxi 210]
  \begin{Phonetics}{齐}{qi2}[][HSK 3]
    \definition*{s.}{Qi, um estado da Dinastia Zhou | Dinastia Qi do Sul (479-502), uma das Dinastias do Sul | Dinastia Qi do Norte (550-577), uma das Dinastias do Norte | Sobrenome: Qi}
    \definition{adj.}{arrumado; uniforme; regular; comprimento, tamanho, etc. são praticamente iguais; uniformes | semelhante; similar; da mesma forma; de acordo| tudo pronto; todos presentes; completo; perfeito}
    \definition{adv.}{juntos; simultaneamente; ao mesmo tempo}
    \definition{v.}{estar no mesmo nível que; alcançar o mesmo nível | estar nivelado em um ponto ou ao longo de uma linha; tornar consistente; harmonizar}
  \synonymref{全}{quan2}
  \antonymref{乱}{luan4}
  \end{Phonetics}
\end{Entry}

\begin{Entry}{齐心协力}{6,4,6,2}{⿑,⼼,⼗,⼒}
  \begin{Phonetics}{齐心协力}{qi2xin1-xie2li4}[][HSK 7-9]
    \definition{expr.}{trabalhar em conjunto; fazer esforços concertados; reunir; trabalhar como um; trabalhar com um propósito comum; fazer esforços conjuntos}
  \end{Phonetics}
\end{Entry}

\begin{Entry}{齐全}{6,6}{⿑,⼊}
  \begin{Phonetics}{齐全}{qi2quan2}[][HSK 5]
    \definition{adj.}{completo; tudo pronto}
  \synonymref{具备}{ju4bei4}
  \synonymref{完备}{wan2bei4}
  \synonymref{完好}{wan2hao3}
  \synonymref{完满}{wan2man3}
  \synonymref{完全}{wan2quan2}
  \antonymref{残缺}{can2que1}
  \antonymref{短缺}{duan3que1}
  \antonymref{欠缺}{qian4que1}
  \end{Phonetics}
\end{Entry}

\begin{Entry}{齐国}{6,8}{⿑,⼞}
  \begin{Phonetics}{齐国}{qi2 guo2}
    \definition*{s.}{Estado Qi de Zhou Ocidental e os Estados Combatentes (1122-265 a.C.), centrado em Shandong}
  \end{Phonetics}
\end{Entry}

%%%%% EOF %%%%%


 %%%
%%% 7画
%%%

\section*{7画}\addcontentsline{toc}{section}{7画}

\begin{entry}{两}{7}{⼀}
  \begin{phonetics}{两}{liang3}[][HSK 1,2]
    \definition*{s.}{sobrenome Liang}
    \definition{clas.}{liang, uma unidade de peso (=50 gramas)}
    \definition{num.}{dois (sempre usado antes de classificadores) | poucos; alguns; indica um número indeterminado}
    \definition{s.}{ambos (lados); qualquer (lado)}
  \end{phonetics}
\end{entry}

\begin{entry}{两边}{7,5}{⼀、⾡}
  \begin{phonetics}{两边}{liang3 bian1}[][HSK 4]
    \definition{s.}{ambos os lados; ambas as direções; ambos os lugares | ambas as partes; ambos os lados}
  \end{phonetics}
\end{entry}

\begin{entry}{两岸}{7,8}{⼀、⼭}
  \begin{phonetics}{两岸}{liang3 an4}[][HSK 5]
    \definition{s.}{ambos os lados; ambas as margens; ambas as costas; entre os dois lados do estreito; bilateral}
  \end{phonetics}
\end{entry}

\begin{entry}{两码事}{7,8,8}{⼀、⽯、⼅}
  \begin{phonetics}{两码事}{liang3ma3shi4}
    \definition{expr.}{duas coisas completamente diferentes}
  \end{phonetics}
\end{entry}

\begin{entry}{严}{7}{⼀}
  \begin{phonetics}{严}{yan2}[][HSK 4]
    \definition*{s.}{sobrenome Yan}
    \definition{adj.}{rígido; rigoroso; estrito; severo}
    \definition{s.}{pai; refere-se ao pai}
  \end{phonetics}
\end{entry}

\begin{entry}{严厉}{7,5}{⼀、⼚}
  \begin{phonetics}{严厉}{yan2li4}[][HSK 5]
    \definition{adj.}{severo; rigoroso}
  \end{phonetics}
\end{entry}

\begin{entry}{严肃}{7,8}{⼀、⾀}
  \begin{phonetics}{严肃}{yan2su4}[][HSK 5]
    \definition{adj.}{sério; solene; sincero; (expressão, atmosfera, etc.) faz as pessoas se sentirem admiradas e desconfortáveis | sóbrio; grave; sério; sincero}
    \definition{v.}{aplicar rigorosamente; fazer algo sério}
  \end{phonetics}
\end{entry}

\begin{entry}{严重}{7,9}{⼀、⾥}
  \begin{phonetics}{严重}{yan2zhong4}[][HSK 4]
    \definition{adj.}{sério; grave; crítico; severo}
  \end{phonetics}
\end{entry}

\begin{entry}{严重打伤}{7,9,5,6}{⼀、⾥、⼿、⼈}
  \begin{phonetics}{严重打伤}{yan2zhong4 da3 shang1}
    \definition{s.}{gravemente ferido}
  \end{phonetics}
\end{entry}

\begin{entry}{严重伤害}{7,9,6,10}{⼀、⾥、⼈、⼧}
  \begin{phonetics}{严重伤害}{yan2zhong4 shang1hai4}
    \definition{s.}{ferimento grave}
  \end{phonetics}
\end{entry}

\begin{entry}{严重关切}{7,9,6,4}{⼀、⾥、⼋、⼑}
  \begin{phonetics}{严重关切}{yan2zhong4guan1qie4}
    \definition{s.}{preocupação séria}
  \end{phonetics}
\end{entry}

\begin{entry}{严重危害}{7,9,6,10}{⼀、⾥、⼙、⼧}
  \begin{phonetics}{严重危害}{yan2zhong4wei1hai4}
    \definition{s.}{danos graves}
  \end{phonetics}
\end{entry}

\begin{entry}{严重后果}{7,9,6,8}{⼀、⾥、⼝、⽊}
  \begin{phonetics}{严重后果}{yan2zhong4hou4guo3}
    \definition{s.}{consequências sérias | repercursões graves}
  \end{phonetics}
\end{entry}

\begin{entry}{严重地}{7,9,6}{⼀、⾥、⼟}
  \begin{phonetics}{严重地}{yan2zhong4 di4}
    \definition{adv.}{seriamente | gravemente}
  \end{phonetics}
\end{entry}

\begin{entry}{严重问题}{7,9,6,15}{⼀、⾥、⾨、⾴}
  \begin{phonetics}{严重问题}{yan2zhong4wen4ti2}
    \definition{s.}{problema sério}
  \end{phonetics}
\end{entry}

\begin{entry}{严重性}{7,9,8}{⼀、⾥、⼼}
  \begin{phonetics}{严重性}{yan2zhong4xing4}
    \definition{s.}{seriedade | gravidade}
  \end{phonetics}
\end{entry}

\begin{entry}{严重破坏}{7,9,10,7}{⼀、⾥、⽯、⼟}
  \begin{phonetics}{严重破坏}{yan2zhong4 po4huai4}
    \definition{s.}{destruição grave}
  \end{phonetics}
\end{entry}

\begin{entry}{严格}{7,10}{⼀、⽊}
  \begin{phonetics}{严格}{yan2ge2}[][HSK 4]
    \definition{adj.}{rígido; estrito; rigoroso; muito consciente e meticuloso na implementação de sistemas e no domínio de padrões}
    \definition{v.}{tornar (sistemas, provisões, etc.) rigorosos;}
  \end{phonetics}
\end{entry}

\begin{entry}{乱}{7}{⼄}
  \begin{phonetics}{乱}{luan4}[][HSK 3]
    \definition{adj.}{bagunçado; confuso; desordenado | turbulento; perturbado (estado de espírito) | arbitrário; aleatório}
    \definition{adv.}{em confusão ou desordem; em um estado de espírito confuso}
    \definition{s.}{caos; tumulto; agitação; turbilhão | comportamento sexual promíscuo; promiscuidade}
    \definition{v.}{confundir; embaralhar; misturar}
  \end{phonetics}
\end{entry}

\begin{entry}{亩}{7}{⼇}
  \begin{phonetics}{亩}{mu3}
    \definition{clas.}{usado para campos | unidade de área igual a um décimo quinto de um hectare}
  \end{phonetics}
\end{entry}

\begin{entry}{估计}{7,4}{⼈、⾔}
  \begin{phonetics}{估计}{gu1ji4}[][HSK 5]
    \definition{v.}{fazer contas; estimar; calcular; julgar a natureza, quantidade, mudança, etc. de uma coisa em uma determinada situação | parecer; parecer como se; aparentar; fazer inferências aproximadas sobre a natureza, a quantidade e a mudança das coisas com base em determinadas circunstâncias}
  \end{phonetics}
\end{entry}

\begin{entry}{伲}{7}{⼈}
  \begin{phonetics}{伲}{ni4}
    \definition{pron.}{(dialeto) eu | meu | nosso | nós}
  \seealsoref{你}{ni3}
  \end{phonetics}
\end{entry}

\begin{entry}{伴侣}{7,8}{⼈、⼈}
  \begin{phonetics}{伴侣}{ban4lv3}
    \definition{s.}{companheiro | parceiro}
  \end{phonetics}
\end{entry}

\begin{entry}{伸}{7}{⼈}
  \begin{phonetics}{伸}{shen1}[][HSK 5]
    \definition{v.}{alongar; esticar; estender}
  \end{phonetics}
\end{entry}

\begin{entry}{但}{7}{⼈}
  \begin{phonetics}{但}{dan4}[][HSK 2]
    \definition*{s.}{sobrenome Dan}
    \definition{adv.}{apenas; meramente; indica uma restrição ao âmbito da ação, equivalente a 只 ou 仅}
    \definition{conj.}{mas; ainda assim; mesmo assim; no entanto; contudo; usado na última oração, conecta duas orações, expressando uma relação de transição, equivalente a 可是 ou 不过}
  \seealsoref{不过}{bu2guo4}
  \seealsoref{仅}{jin3}
  \seealsoref{可是}{ke3shi4}
  \seealsoref{只}{zhi3}
  \end{phonetics}
\end{entry}

\begin{entry}{但是}{7,9}{⼈、⽇}
  \begin{phonetics}{但是}{dan4 shi4}[][HSK 2]
    \definition{conj.}{mas; contudo; no entanto; mesmo assim; usado na segunda parte da frase para indicar uma mudança, geralmente acompanhada de expressões como 虽然 ou 尽管}
  \seealsoref{尽管}{jin3guan3}
  \seealsoref{虽然}{sui1 ran2}
  \end{phonetics}
\end{entry}

\begin{entry}{位}{7}{⼈}
  \begin{phonetics}{位}{wei4}[][HSK 2]
    \definition*{s.}{sobrenome Wei}
    \definition{clas.}{usado para pessoas (com cortesia, respeito) | usado para bits binários}[十六位 (16 bits)]
    \definition{s.}{lugar; localização; o lugar onde ou onde alguém está localizado | posto; \emph{status}; posição; a posição de uma pessoa em uma determinada área da vida social | trono; refere-se especificamente ao status do imperador | lugar; dígito; a posição de cada dígito em um número}
  \end{phonetics}
\end{entry}

\begin{entry}{位于}{7,3}{⼈、⼆}
  \begin{phonetics}{位于}{wei4yu2}[][HSK 4]
    \definition{v.}{estar localizado; estar situado}
  \end{phonetics}
\end{entry}

\begin{entry}{位子}{7,3}{⼈、⼦}
  \begin{phonetics}{位子}{wei4zi5}
    \definition{s.}{lugar | assento}
  \end{phonetics}
\end{entry}

\begin{entry}{位居}{7,8}{⼈、⼫}
  \begin{phonetics}{位居}{wei4ju1}
    \definition{v.}{estar localizado em}
  \end{phonetics}
\end{entry}

\begin{entry}{位置}{7,13}{⼈、⽹}
  \begin{phonetics}{位置}{wei4zhi4}[][HSK 4]
    \definition[通,个]{s.}{assento; lugar; localização | lugar; posição; \emph{status} | posição (por exemplo: cargo no escritório)}
  \end{phonetics}
\end{entry}

\begin{entry}{低}{7}{⼈}
  \begin{phonetics}{低}{di1}[][HSK 2]
    \definition*{s.}{sobrenome Di}
    \definition{adj.}{baixo; distância pequena de baixo para cima; próximo ao solo | abaixo da média; abaixo do padrão geral | inferior (em grau); de nível inferior}
    \definition{v.}{deixar cair; pendurar; abaixar (a cabeça)}
  \end{phonetics}
\end{entry}

\begin{entry}{低于}{7,3}{⼈、⼆}
  \begin{phonetics}{低于}{di1 yu2}[][HSK 5]
    \definition{v.}{ser inferior a; algo ou fenômeno é, de alguma forma, inferior ou pior do que outra coisa}
  \end{phonetics}
\end{entry}

\begin{entry}{低潮}{7,15}{⼈、⽔}
  \begin{phonetics}{低潮}{di1chao2}
    \definition{s.}{maré baixa/vazante; o nível mais baixo da maré durante um ciclo de maré (distinto da 高潮) | vazante baixa; o ponto mais baixo; uma metáfora para o baixo estágio de desenvolvimento das coisas}
  \seealsoref{高潮}{gao1chao2}
  \end{phonetics}
\end{entry}

\begin{entry}{住}{7}{⼈}
  \begin{phonetics}{住}{zhu4}[][HSK 1]
    \definition{adv.}{firmemente; indica estabilidade ou firmeza}
    \definition{v.}{viver; residir; morar; ficar | parar; cessar | (após um verbo) com firmeza; até parar | hospedar; acomodar | parar; interromper | ser competente; ser qualificado; estar à altura; usado com 得 ou 不, indica que a força é suficiente (ou insuficiente)}
  \seealsoref{不}{bu4}
  \seealsoref{得}{de5}
  \end{phonetics}
\end{entry}

\begin{entry}{住处}{7,5}{⼈、⼡}
  \begin{phonetics}{住处}{zhu4chu4}
    \definition{s.}{morada | habitação | residência}
  \end{phonetics}
\end{entry}

\begin{entry}{住宅}{7,6}{⼈、⼧}
  \begin{phonetics}{住宅}{zhu4zhai2}
    \definition{s.}{residência}
  \end{phonetics}
\end{entry}

\begin{entry}{住房}{7,8}{⼈、⼾}
  \begin{phonetics}{住房}{zhu4fang2}[][HSK 2]
    \definition{s.}{habitação}
  \end{phonetics}
\end{entry}

\begin{entry}{住所}{7,8}{⼈、⼾}
  \begin{phonetics}{住所}{zhu4suo3}
    \definition[处]{s.}{morada | habitação | residência}
  \end{phonetics}
\end{entry}

\begin{entry}{住院}{7,9}{⼈、⾩}
  \begin{phonetics}{住院}{zhu4 yuan4}[][HSK 2]
    \definition{v.}{estar hospitalizado | estar no hospital}
  \end{phonetics}
\end{entry}

\begin{entry}{住嘴}{7,16}{⼈、⼝}
  \begin{phonetics}{住嘴}{zhu4zui3}
    \definition{interj.}{Cale-se!}
    \definition{v.}{calar | calar-se}
  \end{phonetics}
\end{entry}

\begin{entry}{体力}{7,2}{⼈、⼒}
  \begin{phonetics}{体力}{ti3 li4}[][HSK 5]
    \definition{s.}{força física; vigor físico (ou corporal); a força do corpo humano para sustentar suas próprias atividades}
  \end{phonetics}
\end{entry}

\begin{entry}{体内}{7,4}{⼈、⼌}
  \begin{phonetics}{体内}{ti3nei4}
    \definition{adj.}{dentro do corpo | \emph{in vivo} (versus \emph{in vitro} | interno a}
  \end{phonetics}
\end{entry}

\begin{entry}{体会}{7,6}{⼈、⼈}
  \begin{phonetics}{体会}{ti3hui4}[][HSK 3]
    \definition{s.}{conhecimento; compreensão; experiência pessoal}
    \definition{v.}{perceber; saber (ou aprender) com a experiência}
  \end{phonetics}
\end{entry}

\begin{entry}{体现}{7,8}{⼈、⾒}
  \begin{phonetics}{体现}{ti3xian4}[][HSK 3]
    \definition{v.}{refletir; incorporar; encarnar}
  \end{phonetics}
\end{entry}

\begin{entry}{体育}{7,8}{⼈、⾁}
  \begin{phonetics}{体育}{ti3yu4}[][HSK 2]
    \definition{s.}{cultura física; treinamento físico; educação cuja principal tarefa é desenvolver a capacidade física e fortalecer a constituição física, alcançada através da participação em várias atividades esportivas | esportes; atividades esportivas; refere-se a esportes}
  \end{phonetics}
\end{entry}

\begin{entry}{体育场}{7,8,6}{⼈、⾁、⼟}
  \begin{phonetics}{体育场}{ti3 yu4 chang3}[][HSK 2]
    \definition[个,座]{s.}{estádio; campo esportivo; espaço ao ar livre para a prática de exercícios físicos ou competições esportivas}
  \end{phonetics}
\end{entry}

\begin{entry}{体育馆}{7,8,11}{⼈、⾁、⾷}
  \begin{phonetics}{体育馆}{ti3 yu4 guan3}[][HSK 2]
    \definition[个]{s.}{ginásio | estádio}
  \end{phonetics}
\end{entry}

\begin{entry}{体重}{7,9}{⼈、⾥}
  \begin{phonetics}{体重}{ti3 zhong4}[][HSK 4]
    \definition{s.}{peso corporal}
  \end{phonetics}
\end{entry}

\begin{entry}{体积}{7,10}{⼈、⽲}
  \begin{phonetics}{体积}{ti3ji1}[][HSK 5]
    \definition[个]{s.}{volume; quantidade; o tamanho do espaço ocupado pelo objeto}
  \end{phonetics}
\end{entry}

\begin{entry}{体验}{7,10}{⼈、⾺}
  \begin{phonetics}{体验}{ti3yan4}[][HSK 3]
    \definition[种]{s.}{experiência}
    \definition{v.}{aprender através da prática; aprender através da experiência pessoal}
  \end{phonetics}
\end{entry}

\begin{entry}{体检}{7,11}{⼈、⽊}
  \begin{phonetics}{体检}{ti3 jian3}[][HSK 4]
    \definition{s.}{exame clínico}
    \definition{v.}{fazer um exame médico}
  \end{phonetics}
\end{entry}

\begin{entry}{体操}{7,16}{⼈、⼿}
  \begin{phonetics}{体操}{ti3 cao1}[][HSK 4]
    \definition{s.}{ginástica; esportes, exercícios ou performances de vários movimentos, sem armas ou com o auxílio de determinados equipamentos}
  \end{phonetics}
\end{entry}

\begin{entry}{何}{7}{⼈}
  \begin{phonetics}{何}{he2}
    \definition*{s.}{sobrenome He}
    \definition{adv.}{expressa exclamação, equivalente a 多么}
    \definition{pron.}{O que?; Onde?; Por que? | expressa uma pergunta retórica, equivalente a 岂, 怎}
  \seealsoref{多么}{duo1me5}
  \seealsoref{岂}{qi3}
  \seealsoref{怎}{zen3}
  \end{phonetics}
\end{entry}

\begin{entry}{何不}{7,4}{⼈、⼀}
  \begin{phonetics}{何不}{he2bu4}
    \definition{adv.}{por que não?; use o tom interrogativo para expressar "deveria" ou "pode"}
  \end{phonetics}
\end{entry}

\begin{entry}{何况}{7,7}{⼈、⼎}
  \begin{phonetics}{何况}{he2kuang4}
    \definition{conj.}{além disso | muito menos}
  \end{phonetics}
\end{entry}

\begin{entry}{佛}{7}{⼈}
  \begin{phonetics}{佛}{fo2}
    \definition*{s.}{Buda, abreviação de 佛陀 | Budismo}
  \seealsoref{佛陀}{fo2tuo2}
  \end{phonetics}
  \begin{phonetics}{佛}{fu2}
    \definition{adv.}{aparentemente}
    \definition{s.}{ornamento da cabeça (feminino)}
  \end{phonetics}
\end{entry}

\begin{entry}{佛陀}{7,7}{⼈、⾩}
  \begin{phonetics}{佛陀}{fo2tuo2}
    \definition{s.}{Buda (uma pessoa que atingiu a Budeidade, ou especificamente Siddhartha Gautama)}
  \end{phonetics}
\end{entry}

\begin{entry}{作}{7}{⼈}
  \begin{phonetics}{作}{zuo1}
    \definition{adj.}{(gíria) incômodo}
    \definition{s.}{trabalhador | oficina | (pessoa) de alta manutenção}
  \end{phonetics}
  \begin{phonetics}{作}{zuo4}
    \definition{s.}{escritos ou obras}
    \definition{v.}{fazer | crescer | escrever ou compor | fingir | considerar como | sentir}
  \end{phonetics}
\end{entry}

\begin{entry}{作为}{7,4}{⼈、⼂}
  \begin{phonetics}{作为}{zuo4wei2}[][HSK 4]
    \definition{prep.}{como; na capacidade de; no caráter de; no papel de; em termos de uma certa identidade de uma pessoa ou de uma certa natureza de uma coisa}
    \definition{s.}{ato; ação; conduta; feito; comportamento | conquista; realização; especificamente, uma boa ação}
    \definition{v.}{considerar como; tomar por; olhar como; tratar como | realizar; fazer conquistas; deixar uma marca}
  \end{phonetics}
\end{entry}

\begin{entry}{作文}{7,4}{⼈、⽂}
  \begin{phonetics}{作文}{zuo4wen2}[][HSK 2]
    \definition[篇]{s.}{ensaio |  composição | redação}
    \definition{v.+compl.}{(de alunos) para escrever uma redação}
  \end{phonetics}
\end{entry}

\begin{entry}{作业}{7,5}{⼈、⼀}
  \begin{phonetics}{作业}{zuo4ye4}[][HSK 2]
    \definition[份,个]{s.}{tarefa escolar | trabalho | tarefa | operação}
  \end{phonetics}
\end{entry}

\begin{entry}{作出}{7,5}{⼈、⼐}
  \begin{phonetics}{作出}{zuo4 chu1}[][HSK 4]
    \definition{v.}{mostrar; tomar (decisões, conclusões, etc. por meio de consideração ou discussão); formar (uma conclusão, decisão, etc.) por meio de consideração ou discussão}
  \end{phonetics}
\end{entry}

\begin{entry}{作用}{7,5}{⼈、⽤}
  \begin{phonetics}{作用}{zuo4yong4}[][HSK 2]
    \definition{s.}{efeito | ação | função}
    \definition{v.}{afetar | agir em}
  \end{phonetics}
\end{entry}

\begin{entry}{作者}{7,8}{⼈、⽼}
  \begin{phonetics}{作者}{zuo4zhe3}[][HSK 3]
    \definition[位,名,个]{s.}{autor; escritor; uma pessoa que escreve artigos ou cria obras de arte}
  \end{phonetics}
\end{entry}

\begin{entry}{作品}{7,9}{⼈、⼝}
  \begin{phonetics}{作品}{zuo4pin3}[][HSK 3]
    \definition[个,部,篇,幅]{s.}{obra de arte; obras concluídas de literatura e arte}
  \end{phonetics}
\end{entry}

\begin{entry}{作家}{7,10}{⼈、⼧}
  \begin{phonetics}{作家}{zuo4jia1}[][HSK 2]
    \definition[位,个]{s.}{autor | escritor}
  \end{phonetics}
\end{entry}

\begin{entry}{你}{7}{⼈}
  \begin{phonetics}{你}{ni3}[][HSK 1]
    \definition{pron.}{você (segunda pessoa do singular); refere-se à pessoa com quem se está conversando | (referindo-se a qualquer pessoa) você; um; qualquer um | com 我 ou 你 em estruturas paralelas para indicar várias ou muitas pessoas se comportando da mesma maneira}
  \seealsoref{您}{nin2}
  \seealsoref{我}{wo3}
  \end{phonetics}
\end{entry}

\begin{entry}{你们}{7,5}{⼈、⼈}
  \begin{phonetics}{你们}{ni3men5}[][HSK 1]
    \definition{pron.}{você (segunda pessoa do plural); refere-se a mais de uma pessoa ou a várias pessoas, incluindo a outra parte}
  \end{phonetics}
\end{entry}

\begin{entry}{你们的}{7,5,8}{⼈、⼈、⽩}
  \begin{phonetics}{你们的}{ni3men5 de5}
    \definition{pron.}{vossos}
  \end{phonetics}
\end{entry}

\begin{entry}{你好}{7,6}{⼈、⼥}
  \begin{phonetics}{你好}{ni3hao3}
    \definition{interj.}{Olá! | Oi!}
  \end{phonetics}
\end{entry}

\begin{entry}{你的}{7,8}{⼈、⽩}
  \begin{phonetics}{你的}{ni3 de5}
    \definition{pron.}{seu}
  \end{phonetics}
\end{entry}

\begin{entry}{克}{7}{⼗}
  \begin{phonetics}{克}{ke4}[][HSK 2]
    \definition*{s.}{sobrenome Ke}
    \definition{clas.}{grama (g) | unidade tibetana de volume ou medida seca (com capacidade para cerca de 25 jin,  斤, de cevada) | unidade tibetana de área de terra equivalente a cerca de 1 mu, 亩}
    \definition{v.}{poder; ser capaz de | tolerar; conter; restringir; suprimir| subjugar; capturar; conquistar (uma cidade, etc.) | digerir (alimentos) | reduzir; diminuir | definir um limite de tempo}
  \seealsoref{斤}{jin1}
  \seealsoref{亩}{mu3}
  \end{phonetics}
\end{entry}

\begin{entry}{克服}{7,8}{⼗、⽉}
  \begin{phonetics}{克服}{ke4fu2}[][HSK 3]
    \definition{v.}{sobrepujar; superar; conquistar | suportar (dificuldades, inconveniências, etc.)}
  \end{phonetics}
\end{entry}

\begin{entry}{免费}{7,9}{⼉、⾙}
  \begin{phonetics}{免费}{mian3fei4}[][HSK 4]
    \definition{s.}{gratuito; sem custo}
    \definition{v.+compl.}{isentar de taxas; tonar grátis}
  \end{phonetics}
\end{entry}

\begin{entry}{免得}{7,11}{⼉、⼻}
  \begin{phonetics}{免得}{mian3de5}
    \definition{conj.}{de modo a não | para evitar | para que não}
  \end{phonetics}
\end{entry}

\begin{entry}{免税}{7,12}{⼉、⽲}
  \begin{phonetics}{免税}{mian3shui4}
    \definition{adj.}{isento de impostos (tributação)}
    \definition{s.}{livre de impostos | isenção de impostos}
    \definition{v.+compl.}{isentar impostos}
  \end{phonetics}
\end{entry}

\begin{entry}{兵}{7}{⼋}
  \begin{phonetics}{兵}{bing1}[][HSK 4]
    \definition[名]{s.}{armas; armamentos | soldado; pessoal militar | exército; tropas | soldado raso; membro mais jovem do exército | assuntos militares (estratégia) | peão, uma das peças do xadrez chinês}
  \end{phonetics}
\end{entry}

\begin{entry}{兵器}{7,16}{⼋、⼝}
  \begin{phonetics}{兵器}{bing1qi4}
    \definition{s.}{armas | armamento}
  \end{phonetics}
\end{entry}

\begin{entry}{况且}{7,5}{⼎、⼀}
  \begin{phonetics}{况且}{kuang4qie3}
    \definition{conj.}{além disso | além do mais}
  \end{phonetics}
\end{entry}

\begin{entry}{冷}{7}{⼎}
  \begin{phonetics}{冷}{leng3}[][HSK 1]
    \definition*{s.}{sobrenome Leng}
    \definition{adj.}{frio; baixa temperatura; sensação de frio | gelado; frio por natureza; sem entusiasmo; sem gentileza | desolado; pouco frequentado; quieto; sem agitação | negligenciado; indesejável; ignorado por todos | raro; estranho; incomum | feito em segredo; filmado de forma escondida; lançado secretamente}
    \definition{v.}{esfriar; resfriar | esfriar; congelar; tornar-se indiferente, apático | ignorar}
  \end{phonetics}
\end{entry}

\begin{entry}{冷门}{7,3}{⼎、⾨}
  \begin{phonetics}{冷门}{leng3men2}
    \definition{s.}{uma profissão, ofício ou ramo de aprendizagem que recebe pouca atenção | um vencedor inesperado; azarão}
  \end{phonetics}
\end{entry}

\begin{entry}{冷静}{7,14}{⼎、⾭}
  \begin{phonetics}{冷静}{leng3jing4}[][HSK 4]
    \definition{adj.}{calmo; descreve uma pessoa que consegue ficar atenta em uma situação importante ou de emergência e não toma decisões aleatórias por causa de seus sentimentos no momento | (lugar) tranquilo; quieto; deserto}
  \end{phonetics}
\end{entry}

\begin{entry}{冻}{7}{⼎}
  \begin{phonetics}{冻}{dong4}[][HSK 5]
    \definition*{s.}{sobrenome Dong}
    \definition{s.}{geleia; gelatina;}
    \definition{v.}{congelar; ser congelado | ficar com frio ou sentir frio}
  \end{phonetics}
\end{entry}

\begin{entry}{初}{7}{⾐}
  \begin{phonetics}{初}{chu1}[][HSK 3]
    \definition*{s.}{sobrenome Chu}
    \definition{adj.}{primeiro (em ordem) | elementar; rudimentar | original}
    \definition{adv.}{pela primeira vez}
    \definition{pref.}{anexado aos numerais de um a dez para indicar ordem (primeiro ao décimo)}
    \definition{s.}{no início de; na primeira parte de | o estágio júnior (pleno; sênior)}
  \end{phonetics}
\end{entry}

\begin{entry}{初中}{7,4}{⾐、⼁}
  \begin{phonetics}{初中}{chu1 zhong1}[][HSK 3]
    \definition[所,个]{s.}{ensino médio; ensino fundamental}
  \end{phonetics}
\end{entry}

\begin{entry}{初心}{7,4}{⾐、⼼}
  \begin{phonetics}{初心}{chu1xin1}
    \definition{s.}{intenção original de alguém, aspiração, etc. | (budismo) ``mente do iniciante'' (ter a mente aberta quando estudando um assunto como um iniciante no assunto teria)}
  \end{phonetics}
\end{entry}

\begin{entry}{初级}{7,6}{⾐、⽷}
  \begin{phonetics}{初级}{chu1ji2}[][HSK 3]
    \definition{adj.}{elementar; primário; júnior; inicial}
  \end{phonetics}
\end{entry}

\begin{entry}{初步}{7,7}{⾐、⽌}
  \begin{phonetics}{初步}{chu1bu4}[][HSK 3]
    \definition{adj.}{inicial; preliminar}
  \end{phonetics}
\end{entry}

\begin{entry}{初期}{7,12}{⾐、⽉}
  \begin{phonetics}{初期}{chu1 qi1}[][HSK 5]
    \definition{s.}{primórdio; estágio inicial; primeiros dias; estágio preliminar; período inicial}
  \end{phonetics}
\end{entry}

\begin{entry}{判断}{7,11}{⼑、⽄}
  \begin{phonetics}{判断}{pan4duan4}[][HSK 3]
    \definition[个]{s.}{julgamento}
    \definition{v.}{julgar; decidir}
  \end{phonetics}
\end{entry}

\begin{entry}{利}{7}{⼑}
  \begin{phonetics}{利}{li4}
    \definition*{s.}{sobrenome Li}
    \definition{adj.}{afiado; cortante | favorável; conveniente; sem dificuldades; sem ou com poucas dificuldades}
    \definition{s.}{benefício; vantagem | lucro; ganhos; juros}
    \definition{v.}{beneficiar; tornar vantajoso}
  \end{phonetics}
\end{entry}

\begin{entry}{利用}{7,5}{⼑、⽤}
  \begin{phonetics}{利用}{li4yong4}[][HSK 3]
    \definition{v.}{usar; utilizar; fazer uso de | explorar; tirar vantagem de}
  \end{phonetics}
\end{entry}

\begin{entry}{利息}{7,10}{⼑、⼼}
  \begin{phonetics}{利息}{li4xi1}[][HSK 4]
    \definition{s.}{acréscimo; juros; dinheiro recebido além do valor principal como resultado de depósitos ou empréstimos (diferenciado de 本金)}
  \seealsoref{本金}{ben3 jin1}
  \end{phonetics}
\end{entry}

\begin{entry}{利润}{7,10}{⼑、⽔}
  \begin{phonetics}{利润}{li4run4}[][HSK 5]
    \definition[笔]{s.}{lucro; o dinheiro ganho com atividades comerciais e industriais}
  \end{phonetics}
\end{entry}

\begin{entry}{利益}{7,10}{⼑、⽫}
  \begin{phonetics}{利益}{li4yi4}[][HSK 4]
    \definition[个,种]{s.}{ganho; lucro; juros; benefício}
  \end{phonetics}
\end{entry}

\begin{entry}{别}{7}{⼑}
  \begin{phonetics}{别}{bie2}[][HSK 1,4]
    \definition*{s.}{sobrenome Bie}
    \definition{adv.}{não; nada de (pedir a alguém para não fazer); é melhor não | talvez, usado em conjunto com a palavra 是 para indicar especulação.}
    \definition{pron.}{outro; algum outro}
    \definition{s.}{distinção; diferença | classificação}
    \definition{v.}{deixar; partir; separar | diferenciar; distinguir; encontrar aspectos diferentes | fixar objetos com pinos | girar; virar | aderir; colar; preder}
  \seealsoref{是}{shi4}
  \end{phonetics}
  \begin{phonetics}{别}{bie4}
    \definition{v.}{fazer com que alguém mude seus hábitos, opiniões, etc.}
  \end{phonetics}
\end{entry}

\begin{entry}{别人}{7,2}{⼑、⼈}
  \begin{phonetics}{别人}{bie2 ren2}[][HSK 1]
    \definition{pron.}{outros; outras pessoas}
    \definition{s.}{outros; pessoas; outras pessoas; refere-se a alguém diferente de si mesmo}
  \end{phonetics}
\end{entry}

\begin{entry}{别的}{7,8}{⼑、⽩}
  \begin{phonetics}{别的}{bie2 de5}[][HSK 1]
    \definition{pron.}{outro; o resto}
  \end{phonetics}
\end{entry}

\begin{entry}{别说}{7,9}{⼑、⾔}
  \begin{phonetics}{别说}{bie2shuo1}
    \definition{v.}{não falar de | não mencionar}
  \end{phonetics}
\end{entry}

\begin{entry}{助手}{7,4}{⼒、⼿}
  \begin{phonetics}{助手}{zhu4shou3}[][HSK 5]
    \definition[个]{s.}{ajudante; auxiliar; assistente; alguém que ajuda os outros com seu trabalho}
  \end{phonetics}
\end{entry}

\begin{entry}{助兴}{7,6}{⼒、⼋}
  \begin{phonetics}{助兴}{zhu4xing4}
    \definition{v.+compl.}{animar as coisas | juntar-se à diversão}
  \end{phonetics}
\end{entry}

\begin{entry}{助理}{7,11}{⼒、⽟}
  \begin{phonetics}{助理}{zhu4li3}[][HSK 5]
    \definition[个,名,位]{s.}{deputado; assistente; auxiliar do diretor responsável (geralmente usado em cargos) | ajudante; assistente; pessoa que auxilia o responsável a fazer as coisas}
  \end{phonetics}
\end{entry}

\begin{entry}{努力}{7,2}{⼒、⼒}
  \begin{phonetics}{努力}{nu3li4}[][HSK 2]
    \definition{adj.}{extenuante; árduo | diligente; trabalhador; quem faz as coisas com o máximo de capacidade ou esforço possível}
    \definition{s.}{esforço; tentativa; fazer o melhor possível}
    \definition{v.}{fazer grandes esforços; esforçar-se; empenhar-se | esforçar-se; usar toda a força possível}
  \end{phonetics}
\end{entry}

\begin{entry}{劳工同事}{7,3,6,8}{⼒、⼯、⼝、⼅}
  \begin{phonetics}{劳工同事}{lao2gong1 tong2shi4}
    \definition{s.}{colaborador | colega de trabalho}
  \end{phonetics}
\end{entry}

\begin{entry}{劳动}{7,6}{⼒、⼒}
  \begin{phonetics}{劳动}{lao2dong4}[][HSK 5]
    \definition[次]{s.}{trabalho; mão de obra; atividades intelectuais ou físicas que podem criar valor | trabalho físico; trabalho manual; referindo-se especificamente ao trabalho físico}
    \definition{v.}{realizar trabalho físico}
  \end{phonetics}
\end{entry}

\begin{entry}{医}{7}{⼖}
  \begin{phonetics}{医}{yi1}
    \definition{s.}{médico | medicina}
    \definition{v.}{curar | tratar}
  \end{phonetics}
\end{entry}

\begin{entry}{医生}{7,5}{⼖、⽣}
  \begin{phonetics}{医生}{yi1sheng1}[][HSK 1]
    \definition[位,个,名]{s.}{médico; clínico; pessoa que possui conhecimentos médicos e cuja profissão é tratar doenças}
  \end{phonetics}
\end{entry}

\begin{entry}{医疗}{7,7}{⼖、⽧}
  \begin{phonetics}{医疗}{yi1 liao2}[][HSK 4]
    \definition{s.}{tratamento médico; tratamento de doenças}
  \end{phonetics}
\end{entry}

\begin{entry}{医学}{7,8}{⼖、⼦}
  \begin{phonetics}{医学}{yi1 xue2}[][HSK 4]
    \definition{s.}{medicina; iatrologia; ciência médica; ciência da prevenção e do tratamento de doenças e da proteção e promoção da saúde humana}
  \end{phonetics}
\end{entry}

\begin{entry}{医院}{7,9}{⼖、⾩}
  \begin{phonetics}{医院}{yi1yuan4}[][HSK 1]
    \definition[家,所,个]{s.}{hospital; instituições que tratam e cuidam de pacientes, e também realizam exames de saúde, prevenção de doenças, etc.}
  \end{phonetics}
\end{entry}

\begin{entry}{即}{7}{⼙}
  \begin{phonetics}{即}{ji2}
    \definition{conj.}{e | até | mesmo se/embora}
  \end{phonetics}
\end{entry}

\begin{entry}{即使}{7,8}{⼙、⼈}
  \begin{phonetics}{即使}{ji2shi3}[][HSK 5]
    \definition{conj.}{mesmo; mesmo que; mesmo se; apesar de; expressando uma concessão hipotética}
  \end{phonetics}
\end{entry}

\begin{entry}{即或}{7,8}{⼙、⼽}
  \begin{phonetics}{即或}{ji2huo4}
    \definition{conj.}{mesmo se/embora}
  \end{phonetics}
\end{entry}

\begin{entry}{即若}{7,8}{⼙、⾋}
  \begin{phonetics}{即若}{ji2ruo4}
    \definition{conj.}{mesmo se/embora}
  \end{phonetics}
\end{entry}

\begin{entry}{即便}{7,9}{⼙、⼈}
  \begin{phonetics}{即便}{ji2bian4}
    \definition{conj.}{mesmo se/embora}
  \end{phonetics}
\end{entry}

\begin{entry}{即将}{7,9}{⼙、⼨}
  \begin{phonetics}{即将}{ji2jiang1}[][HSK 4]
    \definition{adv.}{em breve; estar prestes a; estar a ponto de}
  \end{phonetics}
\end{entry}

\begin{entry}{即是}{7,9}{⼙、⽇}
  \begin{phonetics}{即是}{ji2shi4}
    \definition{conj.}{aquilo é}
  \end{phonetics}
\end{entry}

\begin{entry}{却}{7}{⼙}
  \begin{phonetics}{却}{que4}[][HSK 4]
    \definition{adv.}{mas; contudo; no entanto; enquanto; indica um ponto de virada}
    \definition{v.}{recuar; retroceder | afastar; repelir; desencorajar | declinar; recusar; rejeitar}
    \definition{v.aux.}{usado depois de certos verbos para indicar a conclusão de uma ação}
  \end{phonetics}
\end{entry}

\begin{entry}{却是}{7,9}{⼙、⽇}
  \begin{phonetics}{却是}{que4shi4}
    \definition{conj.}{no entanto | realmente | o fato é\dots | mas isso é\dots}
  \end{phonetics}
\end{entry}

\begin{entry}{县}{7}{⼛}
  \begin{phonetics}{县}{xian4}[][HSK 4]
    \definition[个]{s.}{condado; unidade de divisão administrativa}
  \end{phonetics}
\end{entry}

\begin{entry}{君主立宪制}{7,5,5,9,8}{⼝、⼂、⽴、⼧、⼑}
  \begin{phonetics}{君主立宪制}{jun1zhu3li4xian4zhi4}
    \definition{s.}{monarquia constitucional}
  \end{phonetics}
\end{entry}

\begin{entry}{吟诗}{7,8}{⼝、⾔}
  \begin{phonetics}{吟诗}{yin2shi1}
    \definition{v.}{recitar poesia}
  \end{phonetics}
\end{entry}

\begin{entry}{否认}{7,4}{⼝、⾔}
  \begin{phonetics}{否认}{fou3ren4}[][HSK 3]
    \definition{v.}{negar; repudiar}
  \end{phonetics}
\end{entry}

\begin{entry}{否则}{7,6}{⼝、⼑}
  \begin{phonetics}{否则}{fou3ze2}[][HSK 4]
    \definition{conj.}{senão; se não; ou então; se não for isso}
  \end{phonetics}
\end{entry}

\begin{entry}{否定}{7,8}{⼝、⼧}
  \begin{phonetics}{否定}{fou3ding4}[][HSK 3]
    \definition{adj.}{negativo}
    \definition{s.}{negativo (resposta); negação}
    \definition{v.}{rejeitar; negar}
  \end{phonetics}
\end{entry}

\begin{entry}{吧}{7}{⼝}
  \begin{phonetics}{吧}{ba1}
    \definition{s.}{som de estalo, som crepitante}
    \definition{v.}{puxar o cachimbo; fumar | abreviação de ``bar''}
  \end{phonetics}
  \begin{phonetics}{吧}{ba5}[][HSK 1]
    \definition{part.}{indica discussão, sugestão, solicitação ou comando no final de uma frase | indica concordância ou aprovação no final de uma frase | indica uma pergunta ou especulação no final de uma frase | indica incerteza no final de uma frase | em uma frase, indica uma pausa, carrega um tom hipotético, frequentemente apresenta um contraste e implica um dilema}
  \end{phonetics}
\end{entry}

\begin{entry}{吨}{7}{⼝}
  \begin{phonetics}{吨}{dun1}[][HSK 5]
    \definition{clas.}{tonelada}
  \end{phonetics}
\end{entry}

\begin{entry}{含}{7}{⼝}
  \begin{phonetics}{含}{han2}[][HSK 4]
    \definition{v.}{manter na boca (sem engolir ou cuspir) | conter; incluir | cuidar; acalentar; abrigar}
  \end{phonetics}
\end{entry}

\begin{entry}{含义}{7,3}{⼝、⼂}
  \begin{phonetics}{含义}{han2yi4}[][HSK 4]
    \definition[个,种,层]{s.}{sentido; mensagem; significado; implicação}
  \end{phonetics}
\end{entry}

\begin{entry}{含有}{7,6}{⼝、⽉}
  \begin{phonetics}{含有}{han2 you3}[][HSK 4]
    \definition{v.}{conter; ter; incluir}
  \end{phonetics}
\end{entry}

\begin{entry}{含金量}{7,8,12}{⼝、⾦、⾥}
  \begin{phonetics}{含金量}{han2jin1liang4}
    \definition{adj.}{conteúdo de ouro | (fig.) valioso}
  \end{phonetics}
\end{entry}

\begin{entry}{含量}{7,12}{⼝、⾥}
  \begin{phonetics}{含量}{han2 liang4}[][HSK 4]
    \definition{s.}{conteúdo; a quantidade de um componente contido em uma substância}
  \end{phonetics}
\end{entry}

\begin{entry}{听}{7}{⼝}
  \begin{phonetics}{听}{ting1}[][HSK 1]
    \definition{clas.}{latas; usado para bebidas e alimentos para levar consigo}
    \definition{s.}{lata; embalagem metálica; recipiente cilíndrico utilizado para armazenar bebidas, alimentos, etc.}
    \definition{v.}{ouvir; escutar | obedecer; dar ouvidos; aceitar | supervisionar; administrar; tratar (assuntos políticos); julgar (casos) | permitir; deixar ser; deixar fazer}
  \end{phonetics}
  \begin{phonetics}{听}{yin3}
    \definition[个]{s.}{lata; embalagem metálica}
  \end{phonetics}
\end{entry}

\begin{entry}{听力}{7,2}{⼝、⼒}
  \begin{phonetics}{听力}{ting1li4}[][HSK 3]
    \definition{s.}{audição; capacidade auditiva | compreensão auditiva (na aprendizagem de línguas)}
  \end{phonetics}
\end{entry}

\begin{entry}{听力理解}{7,2,11,13}{⼝、⼒、⽟、⾓}
  \begin{phonetics}{听力理解}{ting1li4li3jie3}
    \definition{s.}{compreensão auditiva}
  \end{phonetics}
\end{entry}

\begin{entry}{听小骨}{7,3,9}{⼝、⼩、⾻}
  \begin{phonetics}{听小骨}{ting1xiao3gu3}
    \definition{s.}{ossículos (do ouvido médio)}
  \seealsoref{听骨}{ting1gu3}
  \end{phonetics}
\end{entry}

\begin{entry}{听见}{7,4}{⼝、⾒}
  \begin{phonetics}{听见}{ting1 jian4}[][HSK 1]
    \definition{v.}{ouvir; conseguir ouvir}
  \end{phonetics}
\end{entry}

\begin{entry}{听写}{7,5}{⼝、⼍}
  \begin{phonetics}{听写}{ting1 xie3}[][HSK 1]
    \definition{s.}{ditado}
    \definition{v.}{ouvir e escrever}
  \end{phonetics}
\end{entry}

\begin{entry}{听众}{7,6}{⼝、⼈}
  \begin{phonetics}{听众}{ting1 zhong4}[][HSK 3]
    \definition{s.}{audiência; ouvintes}
  \end{phonetics}
\end{entry}

\begin{entry}{听会}{7,6}{⼝、⼈}
  \begin{phonetics}{听会}{ting1hui4}
    \definition{v.}{participar de uma reunião (e ouvir o que é discutido)}
  \end{phonetics}
\end{entry}

\begin{entry}{听戏}{7,6}{⼝、⼽}
  \begin{phonetics}{听戏}{ting1xi4}
    \definition{v.}{assistir a uma ópera | ver uma ópera}
  \end{phonetics}
\end{entry}

\begin{entry}{听讲}{7,6}{⼝、⾔}
  \begin{phonetics}{听讲}{ting1 jiang3}[][HSK 2]
    \definition{v.+compl.}{assistir a uma palestra; ouvir palestras ou discursos}
  \end{phonetics}
\end{entry}

\begin{entry}{听来}{7,7}{⼝、⽊}
  \begin{phonetics}{听来}{ting1lai2}
    \definition{v.}{ouvir de algum lugar | soar (antigo, estrangeiro, excitante, certo, etc.) | soar como se (ou seja, dar uma impressão ao ouvinte)}
  \end{phonetics}
\end{entry}

\begin{entry}{听凭}{7,8}{⼝、⼏}
  \begin{phonetics}{听凭}{ting1ping2}
    \definition{v.}{permitir (alguém a fazer o que desejar)}
  \end{phonetics}
\end{entry}

\begin{entry}{听到}{7,8}{⼝、⼑}
  \begin{phonetics}{听到}{ting1 dao4}[][HSK 1]
    \definition{v.}{ouvir, escutar; ouvir atentamente, escutar atentamente}
  \end{phonetics}
\end{entry}

\begin{entry}{听命}{7,8}{⼝、⼝}
  \begin{phonetics}{听命}{ting1ming4}
    \definition{v.}{obedecer ordens | receber ordens}
  \end{phonetics}
\end{entry}

\begin{entry}{听说}{7,9}{⼝、⾔}
  \begin{phonetics}{听说}{ting1 shuo1}[][HSK 2]
    \definition{v.}{ser informado; ouvir falar de; ouvir dizer | ouvir e falar}
  \end{phonetics}
\end{entry}

\begin{entry}{听骨}{7,9}{⼝、⾻}
  \begin{phonetics}{听骨}{ting1gu3}
    \definition{s.}{ossículos (do ouvido médio)}
  \seealsoref{听小骨}{ting1xiao3gu3}
  \end{phonetics}
\end{entry}

\begin{entry}{听断}{7,11}{⼝、⽄}
  \begin{phonetics}{听断}{ting1duan4}
    \definition{v.}{ouvir e decidir | julgar (ou seja, ouvir e julgar em um tribunal)}
  \end{phonetics}
\end{entry}

\begin{entry}{听随}{7,11}{⼝、⾩}
  \begin{phonetics}{听随}{ting1sui2}
    \definition{v.}{permitir | obedecer}
  \end{phonetics}
\end{entry}

\begin{entry}{启发}{7,5}{⼝、⼜}
  \begin{phonetics}{启发}{qi3fa1}[][HSK 5]
    \definition{s.}{iluminação; esclarecimento; fenômenos e princípios que levam as pessoas a refletir e a abrir suas mentes}
    \definition{v.}{despertar; inspirar; esclarecer; orientar, fazer com que compreendam}
  \end{phonetics}
\end{entry}

\begin{entry}{启动}{7,6}{⼝、⼒}
  \begin{phonetics}{启动}{qi3 dong4}[][HSK 5]
    \definition{v.}{ligar (uma máquina); acionar; ligar máquinas, equipamentos elétricos, etc., para começar a trabalhar | entrar em vigor; começar a vigorar e a ser implementados planos, projetos, documentos jurídicos, etc.}
  \end{phonetics}
\end{entry}

\begin{entry}{启事}{7,8}{⼝、⼅}
  \begin{phonetics}{启事}{qi3shi4}[][HSK 5]
    \definition{s.}{aviso; anúncio; texto publicado em jornais ou afixado em paredes com o objetivo de divulgar publicamente algo}
  \end{phonetics}
\end{entry}

\begin{entry}{吵}{7}{⼝}
  \begin{phonetics}{吵}{chao3}[][HSK 3]
    \definition{adj.}{barulhento; ruidoso}
    \definition{v.}{perturbar fazendo barulho; fazer barulho | discutir; brigar; disputar}
  \end{phonetics}
\end{entry}

\begin{entry}{吵架}{7,9}{⼝、⽊}
  \begin{phonetics}{吵架}{chao3jia4}[][HSK 3]
    \definition{v.+compl.}{brigar; discutir; ter uma briga}
  \end{phonetics}
\end{entry}

\begin{entry}{吹}{7}{⼝}
  \begin{phonetics}{吹}{chui1}[][HSK 2]
    \definition{v.}{soprar; baforar | tocar (instrumentos de sopro) | (do vento) soprar | gabar-se; vangloriar-se | elogiar; louvar aos céus; adular | (relacionamento) romper; separar-se; (assunto) fracassar}
  \end{phonetics}
\end{entry}

\begin{entry}{吹牛}{7,4}{⼝、⽜}
  \begin{phonetics}{吹牛}{chui1niu2}
    \definition{v.+compl.}{ogulhar-se | gabar-se | destacar-se}
  \end{phonetics}
\end{entry}

\begin{entry}{吾}{7}{⼝}
  \begin{phonetics}{吾}{wu2}
    \definition*{s.}{sobrenome Wu}
    \definition{pron.}{eu | (antigo) meu}
  \end{phonetics}
\end{entry}

\begin{entry}{呀}{7}{⼝}
  \begin{phonetics}{呀}{ya5}[][HSK 4]
    \definition{part.}{usado no lugar de 啊 quando a palavra anterior termina com o som a, e, i, o ou ü}
  \seealsoref{啊}{a5}
  \end{phonetics}
\end{entry}

\begin{entry}{呆}{7}{⼝}
  \begin{phonetics}{呆}{dai1}[][HSK 5]
    \definition*{s.}{sobrenome Dai}
    \definition{adj.}{maçante; de raciocínio lento | em branco; de madeira; rígido; inflexível}
    \definition{v.}{ficar; permanecer;}
  \end{phonetics}
\end{entry}

\begin{entry}{告别}{7,7}{⼝、⼑}
  \begin{phonetics}{告别}{gao4bie2}[][HSK 3]
    \definition{v.+compl.}{dizer adeus a | deixar; partir de | prestar as últimas homenagens ao falecido}
  \end{phonetics}
\end{entry}

\begin{entry}{告诉}{7,7}{⼝、⾔}
  \begin{phonetics}{告诉}{gao4su4}
    \definition{v.}{dizer; informar (dar a conhecer); dizer aos outros, para que todos saibam}
  \end{phonetics}
  \begin{phonetics}{告诉}{gao4su5}[][HSK 1]
    \definition{v.}{dizer; informar (dar a conhecer)}
  \end{phonetics}
\end{entry}

\begin{entry}{告急}{7,9}{⼝、⼼}
  \begin{phonetics}{告急}{gao4ji2}
    \definition{v.}{estar em estado de emergência | relatar uma emergência | solicitar assistência de emergência}
  \end{phonetics}
\end{entry}

\begin{entry}{员}{7}{⼝}
  \begin{phonetics}{员}{yuan2}[][HSK 3]
    \definition{clas.}{para comandantes militares}
    \definition{s.}{uma pessoa envolvida em algum campo de atividade; refere-se a pessoas que trabalham ou estudam | membro; refere-se aos membros de um grupo ou organização}
  \end{phonetics}
\end{entry}

\begin{entry}{员工}{7,3}{⼝、⼯}
  \begin{phonetics}{员工}{yuan2gong1}[][HSK 3]
    \definition[位,名,个]{s.}{funcionário; atendente; balconista; empregado; trabalhador; pessoal}
  \end{phonetics}
\end{entry}

\begin{entry}{园林}{7,8}{⼞、⽊}
  \begin{phonetics}{园林}{yuan2lin2}[][HSK 5]
    \definition{s.}{parque; jardim; área paisagística com plantas e árvores para as pessoas apreciarem e descansarem.}
  \end{phonetics}
\end{entry}

\begin{entry}{囯}{7}{⼞}
  \begin{phonetics}{囯}{guo2}
    \variantof{国}
  \end{phonetics}
\end{entry}

\begin{entry}{困}{7}{⼞}
  \begin{phonetics}{困}{kun4}[][HSK 3]
    \definition{adj.}{cansado | sonolento}
    \definition{v.}{estar encalhado; estar em grande pressão | cercar; prender; sitiar; cercar; rodear}
  \end{phonetics}
\end{entry}

\begin{entry}{困扰}{7,7}{⼞、⼿}
  \begin{phonetics}{困扰}{kun4 rao3}[][HSK 5]
    \definition{v.}{perturbar; deixar perplexo; perseguir}
  \end{phonetics}
\end{entry}

\begin{entry}{困难}{7,10}{⼞、⾫}
  \begin{phonetics}{困难}{kun4nan5}[][HSK 3]
    \definition{adj.}{dificuldades financeiras; circunstâncias difíceis | complicado; nodoso; difícil; duro;}
    \definition[种]{s.}{dificuldade; situação difícil}
  \end{phonetics}
\end{entry}

\begin{entry}{围}{7}{⼞}
  \begin{phonetics}{围}{wei2}[][HSK 3]
    \definition*{s.}{sobrenome Wei}
    \definition{clas.}{o comprimento dos dois polegares e indicadores ou o comprimento de ambos os braços quando unidos}
    \definition{s.}{em volta de tudo; ao redor}
    \definition{v.}{cercar; rodear; circundar; encurralar | enrolar; envolver}
  \end{phonetics}
\end{entry}

\begin{entry}{围巾}{7,3}{⼞、⼱}
  \begin{phonetics}{围巾}{wei2jin1}[][HSK 4]
    \definition[条]{s.}{lenço; cachecol; echarpe; gravata; tiras longas de malha ou tecido usadas ao redor do pescoço para aquecimento, proteção do colarinho ou decoração}
  \end{phonetics}
\end{entry}

\begin{entry}{围绕}{7,9}{⼞、⽷}
  \begin{phonetics}{围绕}{wei2rao4}[][HSK 5]
    \definition{v.}{girar; circundar; dar voltas; girar em torno de algo; cercar | concentrar-se em; centrar-se em; centrar-se em uma questão ou evento (para realizar atividades)}
  \end{phonetics}
\end{entry}

\begin{entry}{坏}{7}{⼟}
  \begin{phonetics}{坏}{huai4}[][HSK 1]
    \definition{adj.}{ruim; prejudicial; insatisfatório; péssimo | mal; extremamente; indica um grau profundo, geralmente usado após verbos ou adjetivos que expressam estado psicológico | podre; estragado; impróprio; prejudicial ao uso}
    \definition[种]{s.}{ideia maligna; truque sujo; péssima ideia}
    \definition{v.}{estragar; destruir; corromper}
  \end{phonetics}
\end{entry}

\begin{entry}{坏人}{7,2}{⼟、⼈}
  \begin{phonetics}{坏人}{huai4 ren2}[][HSK 2]
    \definition[个,种]{s.}{malfeitor; canalha; pessoa má; pessoa de má qualidade; pessoa que faz coisas ruins}
  \end{phonetics}
\end{entry}

\begin{entry}{坏处}{7,5}{⼟、⼡}
  \begin{phonetics}{坏处}{huai4 chu4}[][HSK 2]
    \definition[个]{s.}{dano; prejuízo; desvantagem; fatores prejudiciais a pessoas ou coisas}
  \end{phonetics}
\end{entry}

\begin{entry}{坏蛋}{7,11}{⼟、⾍}
  \begin{phonetics}{坏蛋}{huai4dan4}
    \definition{s.}{bastardo | canalha | pessoa má}
  \end{phonetics}
\end{entry}

\begin{entry}{坐}{7}{⼟}
  \begin{phonetics}{坐}{zuo4}[][HSK 1]
    \definition*{s.}{sobrenome Zuo}
    \definition{adv.}{sem motivo algum; sem causa ou razão; sem motivo aparente}
    \definition{prep.}{porque; pelo fato de que; pela razão de que; pelo motivo de que}
    \definition{s.}{assento; lugar; posição}
    \definition{v.}{sentar; sentar-se; ocupar um lugar; colocar os glúteos sobre um objeto para apoiar o peso corporal | pegar; viajar de; pegar carona | ter as costas voltadas para | colocar (uma panela, chaleira, etc.) no fogo | recuo; coice (de rifles, armas, etc.)  | produzir frutos; formar sementes | ser punido; ser acusado de crime | contrair (ou ter) uma doença; sofrer de uma doença | (um edifício) afundar; ceder}
  \end{phonetics}
\end{entry}

\begin{entry}{坐下}{7,3}{⼟、⼀}
  \begin{phonetics}{坐下}{zuo4 xia5}[][HSK 1]
    \definition{v.}{sentar-se; tomar um assento; passar da posição em pé para a posição sentada}
  \end{phonetics}
\end{entry}

\begin{entry}{坐车}{7,4}{⼟、⾞}
  \begin{phonetics}{坐车}{zuo4che1}
    \definition{v.}{andar de carro, ônibus, trem, etc.}
  \end{phonetics}
\end{entry}

\begin{entry}{坐好}{7,6}{⼟、⼥}
  \begin{phonetics}{坐好}{zuo4hao3}
    \definition{v.}{sentar-se corretamente | sentar direito}
  \end{phonetics}
\end{entry}

\begin{entry}{坐享}{7,8}{⼟、⼇}
  \begin{phonetics}{坐享}{zuo4xiang3}
    \definition{v.}{curtir algo sem levantar um dedo}
  \end{phonetics}
\end{entry}

\begin{entry}{坐垫}{7,9}{⼟、⼟}
  \begin{phonetics}{坐垫}{zuo4dian4}
    \definition[块]{s.}{assento (motocicleta) | almofada}
  \end{phonetics}
\end{entry}

\begin{entry}{坐标}{7,9}{⼟、⽊}
  \begin{phonetics}{坐标}{zuo4biao1}
    \definition{s.}{coordenada (geometria)}
  \end{phonetics}
\end{entry}

\begin{entry}{坑}{7}{⼟}
  \begin{phonetics}{坑}{keng1}
    \definition{s.}{poço | depressão | túnel | buraco no chão}
    \definition{v.}{enganar | trapacear}
  \end{phonetics}
\end{entry}

\begin{entry}{坑人}{7,2}{⼟、⼈}
  \begin{phonetics}{坑人}{keng1ren2}
    \definition{v.+compl.}{trapacear alguém}
  \end{phonetics}
\end{entry}

\begin{entry}{块}{7}{⼟}
  \begin{phonetics}{块}{kuai4}[][HSK 1]
    \definition{clas.}{usado para coisas em pedaços | usado para coisas em pedaços ou em algumas formas de folhas | usado para moedas de prata ou notas de papel equivalentes a 圆}
    \definition{s.}{pedaço; pedaço (de terra); peça; algo que forma um pedaço ou massa}
  \seealsoref{圆}{yuan2}
  \end{phonetics}
\end{entry}

\begin{entry}{坚决}{7,6}{⼟、⼎}
  \begin{phonetics}{坚决}{jian1jue2}[][HSK 3]
    \definition{adj.}{firme; resoluto}
  \end{phonetics}
\end{entry}

\begin{entry}{坚守}{7,6}{⼟、⼧}
  \begin{phonetics}{坚守}{jian1shou3}
    \definition{v.}{agarrar-se}
  \end{phonetics}
\end{entry}

\begin{entry}{坚固}{7,8}{⼟、⼞}
  \begin{phonetics}{坚固}{jian1gu4}[][HSK 4]
    \definition{adj.}{firme; sólido; robusto; forte; durável; firmemente unidos e inquebráveis}
  \end{phonetics}
\end{entry}

\begin{entry}{坚定}{7,8}{⼟、⼧}
  \begin{phonetics}{坚定}{jian1ding4}[][HSK 5]
    \definition{adj.}{firme; inabalável; inamovível; (posição, opinião, vontade, etc.) firme e estável, inabalável}
    \definition{v.}{fortalecer}
  \end{phonetics}
\end{entry}

\begin{entry}{坚持}{7,9}{⼟、⼿}
  \begin{phonetics}{坚持}{jian1chi2}[][HSK 3]
    \definition{v.}{persistir e; perseverar em; sustentar; insistir em; manter-se fiel a; aderir a}
  \end{phonetics}
\end{entry}

\begin{entry}{坚强}{7,12}{⼟、⼸}
  \begin{phonetics}{坚强}{jian1qiang2}[][HSK 3]
    \definition{adj.}{forte; firme; convicto}
    \definition{v.}{fortalecer; tornar forte}
  \end{phonetics}
\end{entry}

\begin{entry}{坠}{7}{⼟}
  \begin{phonetics}{坠}{zhui4}
    \definition{v.}{cair | pesar | fazer vergar com o peso}
  \end{phonetics}
\end{entry}

\begin{entry}{坠落}{7,12}{⼟、⾋}
  \begin{phonetics}{坠落}{zhui4luo4}
    \definition{v.}{cair}
  \end{phonetics}
\end{entry}

\begin{entry}{声}{7}{⼠}
  \begin{phonetics}{声}{sheng1}[][HSK 5]
    \definition{clas.}{indica o número de vezes que um som é emitido}
    \definition{s.}{som; voz | reputação | consoante inicial (de uma sílaba chinesa) | tom; tom de voz | informação; notícia}
    \definition{v.}{declarar; anunciar; emitir um som}
  \end{phonetics}
\end{entry}

\begin{entry}{声明}{7,8}{⼠、⽇}
  \begin{phonetics}{声明}{sheng1ming2}[][HSK 3]
    \definition[项,份]{s.}{declaração}
    \definition{v.}{declarar}
    \definition{v.}{declarar; anunciar}
  \end{phonetics}
\end{entry}

\begin{entry}{声音}{7,9}{⼠、⾳}
  \begin{phonetics}{声音}{sheng1yin1}[][HSK 2]
    \definition[个,种]{s.}{som; voz; a percepção auditiva das ondas sonoras}
  \end{phonetics}
\end{entry}

\begin{entry}{壳}{7}{⼠}
  \begin{phonetics}{壳}{ke2}
    \definition{s.}{casca (de ovo, noz, caranguejo, etc.) | caixa | invólucro | alojamento (de uma máquina ou dispositivo)}
  \end{phonetics}
\end{entry}

\begin{entry}{妖}{7}{⼥}
  \begin{phonetics}{妖}{yao1}
    \definition{adj.}{enfeitiçante | encantador}
    \definition{s.}{\emph{goblin} | bruxa | diabo | monstro | fantasma | demônio}
  \end{phonetics}
\end{entry}

\begin{entry}{妙招}{7,8}{⼥、⼿}
  \begin{phonetics}{妙招}{miao4zhao1}
    \definition{adj.}{escorregadio}
    \definition{s.}{movimento inteligente | maneira inteligente de fazer algo}
  \end{phonetics}
\end{entry}

\begin{entry}{宋}{7}{⼧}
  \begin{phonetics}{宋}{song4}
    \definition*{s.}{sobrenome Song}
    \definition{s.}{Dinastia Song (960-1279) | Song das dinastias do sul (420-479)}
  \end{phonetics}
\end{entry}

\begin{entry}{完}{7}{⼧}
  \begin{phonetics}{完}{wan2}[][HSK 2]
    \definition*{s.}{sobrenome Wan}
    \definition{adj.}{inteiro; intacto; completo}
    \definition{v.}{acabar; terminar; completar | pagar | estar terminado; estar pronto para | esgotar; ser usado}
  \end{phonetics}
\end{entry}

\begin{entry}{完了}{7,2}{⼧、⼅}
  \begin{phonetics}{完了}{wan2 le5}[][HSK 5]
    \definition{v.}{acabar; terminar; concluir; chegar ao fim}
  \end{phonetics}
\end{entry}

\begin{entry}{完人}{7,2}{⼧、⼈}
  \begin{phonetics}{完人}{wan2ren2}
    \definition{s.}{pessoa perfeita}
  \end{phonetics}
\end{entry}

\begin{entry}{完全}{7,6}{⼧、⼊}
  \begin{phonetics}{完全}{wan2quan2}[][HSK 2]
    \definition{adj.}{inteiro; completo; não falta nada, está tudo completo}
    \definition{adv.}{completamente; representa tudo}
  \end{phonetics}
\end{entry}

\begin{entry}{完成}{7,6}{⼧、⼽}
  \begin{phonetics}{完成}{wan2cheng2}[][HSK 2]
    \definition{v.}{realizar; completar; terminar; cumprir; levar ao sucesso}
  \end{phonetics}
\end{entry}

\begin{entry}{完毕}{7,6}{⼧、⽐}
  \begin{phonetics}{完毕}{wan2bi4}
    \definition{v.}{completar | terminar | acabar}
  \end{phonetics}
\end{entry}

\begin{entry}{完完全全}{7,7,6,6}{⼧、⼧、⼊、⼊}
  \begin{phonetics}{完完全全}{wan2wan2quan2quan2}
    \definition{adv.}{completamente}
  \end{phonetics}
\end{entry}

\begin{entry}{完备}{7,8}{⼧、⼡}
  \begin{phonetics}{完备}{wan2bei4}
    \definition{adj.}{completo | impecável | perfeito}
    \definition{v.}{não deixar nada a desejar}
  \end{phonetics}
\end{entry}

\begin{entry}{完美}{7,9}{⼧、⽺}
  \begin{phonetics}{完美}{wan2mei3}[][HSK 3]
    \definition{adj.}{perfeito; impecável; consumado}
    \definition{adv.}{perfeitamente}
    \definition{s.}{perfeição}
  \end{phonetics}
\end{entry}

\begin{entry}{完善}{7,12}{⼧、⼝}
  \begin{phonetics}{完善}{wan2shan4}[][HSK 3]
    \definition{adj.}{perfeito; consumado}
    \definition{v.}{refinar; melhorar; tornar perfeito}
  \end{phonetics}
\end{entry}

\begin{entry}{完税}{7,12}{⼧、⽲}
  \begin{phonetics}{完税}{wan2shui4}
    \definition{v.}{pagar imposto}
  \end{phonetics}
\end{entry}

\begin{entry}{完满}{7,13}{⼧、⽔}
  \begin{phonetics}{完满}{wan2man3}
    \definition{adj.}{satisfatório | bem-sucedido}
  \end{phonetics}
\end{entry}

\begin{entry}{完整}{7,16}{⼧、⽁}
  \begin{phonetics}{完整}{wan2zheng3}[][HSK 3]
    \definition{adj.}{intacto; inteiro; completo; integrado}
  \end{phonetics}
\end{entry}

\begin{entry}{寿司}{7,5}{⼨、⼝}
  \begin{phonetics}{寿司}{shou4 si1}[][HSK 5]
    \definition[份]{s.}{\emph{sushi}; iguaria tradicional japonesa}
  \end{phonetics}
\end{entry}

\begin{entry}{尾巴}{7,4}{⼫、⼰}
  \begin{phonetics}{尾巴}{wei3ba5}[][HSK 4]
    \definition{s.}{cauda; projeções na extremidade do corpo de certos animais | parte semelhante a uma cauda; refere-se, em geral, ao final de algo | apêndice; anexo; adepto servil; pessoa que segue ou concorda com outra pessoa | (figura de linguagem) alguém que faz sombra a outro | fim; remanescente; parte restante (ou inacabada)}
  \end{phonetics}
\end{entry}

\begin{entry}{尿}{7}{⼫}
  \begin{phonetics}{尿}{niao4}
    \definition[泡]{s.}{urina}
    \definition{v.}{urinar}
  \end{phonetics}
  \begin{phonetics}{尿}{sui1}
    \definition{s.}{(coloquial) urina}
  \end{phonetics}
\end{entry}

\begin{entry}{局}{7}{⼫}
  \begin{phonetics}{局}{ju2}[][HSK 4]
    \definition{s.}{tabuleiro de xadrez | jogo; turno; \emph{set} | situação; estado das coisas | tolerância; grandeza ou pequenez da mente; grau de tolerância de uma pessoa em relação às outras | reunião de pessoas em festas | ardil; artidício; estratagema; armadilha | parte; porção; parcela | nome de determinadas lojas}
  \end{phonetics}
\end{entry}

\begin{entry}{局长}{7,4}{⼫、⾧}
  \begin{phonetics}{局长}{ju2 zhang3}[][HSK 5]
    \definition[位,个]{s.}{comissário; diretor; principais chefes de gabinete do governo}
  \end{phonetics}
\end{entry}

\begin{entry}{局面}{7,9}{⼫、⾯}
  \begin{phonetics}{局面}{ju2mian4}[][HSK 5]
    \definition[种]{s.}{aspecto; fase; situação; o estado das coisas em um período de tempo, em sua maior parte abstraído | escopo; escala}
  \end{phonetics}
\end{entry}

\begin{entry}{屁股}{7,8}{⼫、⾁}
  \begin{phonetics}{屁股}{pi4gu5}
    \definition{s.}{nádega | quadris}
  \end{phonetics}
\end{entry}

\begin{entry}{屁话}{7,8}{⼫、⾔}
  \begin{phonetics}{屁话}{pi4hua4}
    \definition{s.}{absurdo | tolice | besteira}
  \end{phonetics}
\end{entry}

\begin{entry}{层}{7}{⼫}
  \begin{phonetics}{层}{ceng2}[][HSK 2]
    \definition{clas.}{usado para coisas que se sobrepõem e se acumulam, como andares, camadas e estratos | usado para coisas que podem ser divididas em itens e etapas | usado para coisas que podem ser removidas ou apagadas da superfície de um objeto}
    \definition{s.}{camada; nível; coisas que se sobrepõem | nível; classificação; camada}
    \definition{v.}{sobrepor; empilhar camada sobre camada}
  \end{phonetics}
\end{entry}

\begin{entry}{层次}{7,6}{⼫、⽋}
  \begin{phonetics}{层次}{ceng2ci4}[][HSK 5]
    \definition{s.}{disposição ordenada do conteúdo (de um discurso ou texto) | nível ou estrutura administrativa; distinções entre a mesma coisa devido a diferenças de tamanho, altura, etc. | nível; níveis de afiliação}
  \end{phonetics}
\end{entry}

\begin{entry}{层层}{7,7}{⼫、⼫}
  \begin{phonetics}{层层}{ceng2ceng2}
    \definition{s.}{camada sobre camada}
  \end{phonetics}
\end{entry}

\begin{entry}{希望}{7,11}{⼱、⽉}
  \begin{phonetics}{希望}{xi1wang4}[][HSK 3]
    \definition[个]{s.}{esperança; desejo; expectativa | aquilo em que a esperança é depositada}
    \definition{v.}{ter esperança; desejar; esperar}
  \end{phonetics}
\end{entry}

\begin{entry}{床}{7}{⼴}
  \begin{phonetics}{床}{chuang2}[][HSK 1]
    \definition{clas.}{para colchas, roupas de cama, etc.}
    \definition[张]{s.}{cama; sofá; móveis para dormir | algo com o formato de uma cama}
  \end{phonetics}
\end{entry}

\begin{entry}{库}{7}{⼴}
  \begin{phonetics}{库}{ku4}[][HSK 5]
    \definition{s.}{depósito; tesouraria; armazém; almoxarifado; edifícios e equipamentos para armazenamento de mercadorias | banco de dados}
  \end{phonetics}
\end{entry}

\begin{entry}{应}{7}{⼴}
  \begin{phonetics}{应}{ying1}[][HSK 4,5]
    \definition{v.}{ecoar; responder; responder a; responder às chamadas, saudações, perguntas, etc. de outras pessoas | conceder; cumprir | adequar; adaptar; responder a | lidar com; enfrentar; abordar | tornar-se realidade; ser cumprido}
  \end{phonetics}
\end{entry}

\begin{entry}{应对}{7,5}{⼴、⼨}
  \begin{phonetics}{应对}{ying4dui4}
    \definition{v.}{responder | manusear | lidar}
  \end{phonetics}
\end{entry}

\begin{entry}{应用}{7,5}{⼴、⽤}
  \begin{phonetics}{应用}{ying4yong4}[][HSK 3]
    \definition{adj.}{aplicado (na vida ou na produção); usado diretamente na vida ou na produção}
    \definition{s.}{aplicativo}
    \definition{v.}{usar; aplicar}
  \end{phonetics}
\end{entry}

\begin{entry}{应用程序}{7,5,12,7}{⼴、⽤、⽲、⼴}
  \begin{phonetics}{应用程序}{ying4yong4cheng2xu4}
    \definition{s.}{aplicativo | programa de computador}
  \end{phonetics}
\end{entry}

\begin{entry}{应用程序接口}{7,5,12,7,11,3}{⼴、⽤、⽲、⼴、⼿、⼝}
  \begin{phonetics}{应用程序接口}{ying4yong4cheng2xu4jie1kou3}
    \definition{s.}{API (\emph{application programming interface})}
  \seealsoref{应用程序编程接口}{ying4yong4cheng2xu4bian1cheng2jie1kou3}
  \end{phonetics}
\end{entry}

\begin{entry*}{应用程序编程接口}{7,5,12,7,12,12,11,3}{⼴、⽤、⽲、⼴、⽷、⽲、⼿、⼝}
  \begin{phonetics}{应用程序编程接口}{ying4yong4cheng2xu4bian1cheng2jie1kou3}
    \definition{s.}{API (\emph{application programming interface})}
  \seealsoref{应用程序接口}{ying4yong4cheng2xu4jie1kou3}
  \end{phonetics}
\end{entry*}

\begin{entry}{应当}{7,6}{⼴、⼹}
  \begin{phonetics}{应当}{ying1 dang1}[][HSK 3]
    \definition{v.}{dever}
  \end{phonetics}
\end{entry}

\begin{entry}{应该}{7,8}{⼴、⾔}
  \begin{phonetics}{应该}{ying1gai1}[][HSK 2]
    \definition{v.}{dever | ter de}
  \end{phonetics}
\end{entry}

\begin{entry}{弄}{7}{⼶}
  \begin{phonetics}{弄}{long4}
    \definition{s.}{rua estreita; beco; viela; travessa}
  \end{phonetics}
  \begin{phonetics}{弄}{nong4}[][HSK 2]
    \definition{v.}{fazer, realizar; tratar; organizar | obter; buscar; tentar conseguir; encontrar uma maneira de conseguir | brincar com; enganar | pregar uma peça; brincar; manipular | mexer com; perturbar}
  \end{phonetics}
\end{entry}

\begin{entry}{弟}{7}{⼸}
  \begin{phonetics}{弟}{di4}[][HSK 1]
    \definition*{s.}{sobrenome Di}
    \definition[个]{s.}{irmão mais novo | (entre amigos homens) eu | geralmente se refere a colegas do sexo masculino mais jovens na família ou entre parentes | forma humilde que os amigos usam para se referir uns aos outros, usada principalmente em correspondência}
  \end{phonetics}
\end{entry}

\begin{entry}{弟弟}{7,7}{⼸、⼸}
  \begin{phonetics}{弟弟}{di4 di5}[][HSK 1]
    \definition[个,位]{s.}{irmão mais novo | primo}
  \end{phonetics}
\end{entry}

\begin{entry}{弟妹}{7,8}{⼸、⼥}
  \begin{phonetics}{弟妹}{di4mei4}
    \definition{s.}{esposa do irmão mais novo}
  \end{phonetics}
\end{entry}

\begin{entry}{张}{7}{⼸}
  \begin{phonetics}{张}{zhang1}[][HSK 3]
    \definition*{s.}{sobrenome Zhang}
    \definition*{s.}{Zhang, uma das mansões lunares}
    \definition{adj.}{nervoso; tenso}
    \definition{clas.}{para papel, couro, etc. | para camas, mesas, etc. | para a boca e o rosto | para arcos}
    \definition{s.}{folha de papel}
    \definition{v.}{consertar (uma corda de arco); encordoar (um instrumento musical ou um arco) | abrir; espalhar; esticar | expor; exibir |expandir; estender | ampliar; exagerar | olhar | dar rédea solta a; satisfazer | iniciar um negócio; abrir uma loja | colocar em bom uso; dar liberdade para | pegar com uma rede; montar armadilhas para capturar pássaros e animais}
  \end{phonetics}
\end{entry}

\begin{entry}{张三}{7,3}{⼸、⼀}
  \begin{phonetics}{张三}{zhang1san1}
    \definition*{s.}{Zhang San | Zé Ninguém | nome para uma pessoa não especificada, 1 de 3}
  \seealsoref{李四}{li3si4}
  \seealsoref{王五}{wang2wu3}
  \end{phonetics}
\end{entry}

\begin{entry}{张狂}{7,7}{⼸、⽝}
  \begin{phonetics}{张狂}{zhang1kuang2}
    \definition{adj.}{impetuoso | frenético | insolente}
  \end{phonetics}
\end{entry}

\begin{entry}{形式}{7,6}{⼺、⼷}
  \begin{phonetics}{形式}{xing2shi4}[][HSK 3]
    \definition[种,个]{s.}{forma; formato; modalidade | aparência, estrutura ou estado de algo}
  \end{phonetics}
\end{entry}

\begin{entry}{形成}{7,6}{⼺、⼽}
  \begin{phonetics}{形成}{xing2cheng2}[][HSK 3]
    \definition{v.}{moldar; formar; tomar forma | tornar-se algo ou algo através do desenvolvimento e da mudança}
  \end{phonetics}
\end{entry}

\begin{entry}{形而上学}{7,6,3,8}{⼺、⽽、⼀、⼦}
  \begin{phonetics}{形而上学}{xing2'er2shang4xue2}
    \definition{s.}{metafísica}
  \end{phonetics}
\end{entry}

\begin{entry}{形状}{7,7}{⼺、⽝}
  \begin{phonetics}{形状}{xing2zhuang4}[][HSK 3]
    \definition[个]{s.}{forma; aparência | a aparência de um objeto ou figura formada pela combinação de superfícies ou linhas externas}
  \end{phonetics}
\end{entry}

\begin{entry}{形势}{7,8}{⼺、⼒}
  \begin{phonetics}{形势}{xing2shi4}[][HSK 4]
    \definition[个]{s.}{terreno; características topográficas; situação geográfica, principalmente de uma perspectiva militar | situação; circunstâncias; a situação geral, a tendência de como as coisas estão se desenvolvendo e mudando | geralmente não é usado em situações pessoais}
  \end{phonetics}
\end{entry}

\begin{entry}{形态}{7,8}{⼺、⼼}
  \begin{phonetics}{形态}{xing2tai4}[][HSK 5]
    \definition{s.}{forma; forma como as coisas se apresentam | forma; padrão; postura | morfologia; forma; (gramática) refere-se às formas internas de mudança das palavras, incluindo a formação de palavras e as mudanças morfológicas}
  \end{phonetics}
\end{entry}

\begin{entry}{形容}{7,10}{⼺、⼧}
  \begin{phonetics}{形容}{xing2rong2}[][HSK 4]
    \definition{s.}{aparência; semblante}
    \definition{v.}{descrever}
  \end{phonetics}
\end{entry}

\begin{entry}{形象}{7,11}{⼺、⾗}
  \begin{phonetics}{形象}{xing2xiang4}[][HSK 3]
    \definition{adj.}{vívido}
    \definition[个]{s.}{imagem; forma; figura | uma forma ou gesto específico que pode despertar os pensamentos ou emoções das pessoas | imagem literária; imagem artística | pessoas ou coisas com características diferentes criadas na literatura, no cinema e em outras artes}
  \end{phonetics}
\end{entry}

\begin{entry}{彻底}{7,8}{⼻、⼴}
  \begin{phonetics}{彻底}{che4di3}[][HSK 4]
    \definition{adj.}{minucioso; completo; exaustivo; profundo e completo; nada é deixado de fora}
  \end{phonetics}
\end{entry}

\begin{entry}{忍}{7}{⼼}
  \begin{phonetics}{忍}{ren3}[][HSK 5]
    \definition{v.}{suportar; aguentar; tolerar; aturar | ter coragem para; ser insensível o suficiente para; ser capaz de endurecer o coração e fazer coisas que não se devem fazer por uma questão de razão}
  \end{phonetics}
\end{entry}

\begin{entry}{忍不住}{7,4,7}{⼼、⼀、⼈}
  \begin{phonetics}{忍不住}{ren3bu5zhu4}[][HSK 5]
    \definition{v.}{incapaz de suportar; não conseguir evitar fazer algo; não conseguir se controlar}
  \end{phonetics}
\end{entry}

\begin{entry}{忍受}{7,8}{⼼、⼜}
  \begin{phonetics}{忍受}{ren3shou4}[][HSK 5]
    \definition{v.}{suportar; sofrer; aguentar; tolerar; suportar com dificuldade o sofrimento, as dificuldades e as adversidades da vida}
  \end{phonetics}
\end{entry}

\begin{entry}{忍耐}{7,9}{⼼、⽽}
  \begin{phonetics}{忍耐}{ren3nai4}
    \definition{s.}{paciência | resistência}
    \definition{v.}{suportar | resistir | exercer paciência}
  \end{phonetics}
\end{entry}

\begin{entry}{志愿}{7,14}{⼼、⽕}
  \begin{phonetics}{志愿}{zhi4 yuan4}[][HSK 3]
    \definition{s.}{desejo; ideal; aspiração; meta que se espera alcançar}
    \definition{v.}{ser voluntário; tomar a iniciativa e esteja disposto a fazer um trabalho que não gere renda ou que tenha renda muito baixa, mas que possa ajudar outras pessoas}
  \end{phonetics}
\end{entry}

\begin{entry}{志愿者}{7,14,8}{⼼、⽕、⽼}
  \begin{phonetics}{志愿者}{zhi4yuan4zhe3}[][HSK 3]
    \definition{s.}{voluntário; pessoa que se voluntaria para servir em atividades de assistência social, eventos de grande porte, conferências, etc.}
  \end{phonetics}
\end{entry}

\begin{entry}{忘}{7}{⼼}
  \begin{phonetics}{忘}{wang4}[][HSK 1]
    \definition{v.}{esquecer | ignorar; negligenciar}
  \end{phonetics}
\end{entry}

\begin{entry}{忘本}{7,5}{⼼、⽊}
  \begin{phonetics}{忘本}{wang4ben3}
    \definition{v.}{esquecer as próprias raízes}
  \end{phonetics}
\end{entry}

\begin{entry}{忘记}{7,5}{⼼、⾔}
  \begin{phonetics}{忘记}{wang4ji4}[][HSK 1]
    \definition{v.}{esquecer | ignorar; negligenciar | sair da memória de alguém; não ser lembrado | descartar da mente; ignorar}
  \end{phonetics}
\end{entry}

\begin{entry}{忘却}{7,7}{⼼、⼙}
  \begin{phonetics}{忘却}{wang4que4}
    \definition{v.}{esquecer}
  \end{phonetics}
\end{entry}

\begin{entry}{忘怀}{7,7}{⼼、⼼}
  \begin{phonetics}{忘怀}{wang4huai2}
    \definition{v.}{esquecer}
  \end{phonetics}
\end{entry}

\begin{entry}{忘恩}{7,10}{⼼、⼼}
  \begin{phonetics}{忘恩}{wang4'en1}
    \definition{v.}{ser ingrato}
  \end{phonetics}
\end{entry}

\begin{entry}{忘掉}{7,11}{⼼、⼿}
  \begin{phonetics}{忘掉}{wang4diao4}
    \definition{v.}{esquecer}
  \end{phonetics}
\end{entry}

\begin{entry}{忘餐}{7,16}{⼼、⾷}
  \begin{phonetics}{忘餐}{wang4can1}
    \definition{v.}{esquecer as refeições}
  \end{phonetics}
\end{entry}

\begin{entry}{忧郁}{7,8}{⼼、⾢}
  \begin{phonetics}{忧郁}{you1yu4}
    \definition{adj.}{deprimido | melancólico | desanimado}
    \definition{s.}{depressão | melancolia}
  \end{phonetics}
\end{entry}

\begin{entry}{快}{7}{⼼}
  \begin{phonetics}{快}{kuai4}[][HSK 1]
    \definition*{s.}{sobrenome Kuai}
    \definition{adj.}{rápido; veloz (oposto a 慢) | apressado | perspicaz; ágil; inteligente; de ​​mente rápida | (de uma faca, espada, etc.) afiado (oposto a 钝) | direto; franco; sem rodeios | satisfeito; feliz; gratificado | rápido; veloz; alta velocidade; tempo de execução curto | satisfeito; feliz; contente | engenhoso; ágil | afiado; facas, tesouras, machados e outros objetos afiados | sincero}
    \definition{adv.}{em breve; antes de muito tempo; estar prestes a | rapidamente}
    \definition{s.}{policial; polícia | (antigo) oficial encarregado de efetuar prisões}
  \seealsoref{钝}{dun4}
  \seealsoref{慢}{man4}
  \end{phonetics}
\end{entry}

\begin{entry}{快乐}{7,5}{⼼、⼃}
  \begin{phonetics}{快乐}{kuai4le4}[][HSK 2]
    \definition{adj.}{feliz; alegre; animado; prazeiroso}
    \definition{s.}{felicidade | alegria}
  \end{phonetics}
\end{entry}

\begin{entry}{快活}{7,9}{⼼、⽔}
  \begin{phonetics}{快活}{kuai4huo5}[][HSK 5]
    \definition{adj.}{feliz; alegre; contente; animado}
  \end{phonetics}
\end{entry}

\begin{entry}{快点儿}{7,9,2}{⼼、⽕、⼉}
  \begin{phonetics}{快点儿}{kuai4 dian3r5}[][HSK 2]
    \definition{v.}{apressar-se}
  \end{phonetics}
\end{entry}

\begin{entry}{快要}{7,9}{⼼、⾑}
  \begin{phonetics}{快要}{kuai4 yao4}[][HSK 2]
    \definition{adv.}{estar prestes a; estar indo para; estar à beira de; em breve; em pouco tempo; indica que a situação está prestes a ocorrer}
  \end{phonetics}
\end{entry}

\begin{entry}{快递}{7,10}{⼼、⾡}
  \begin{phonetics}{快递}{kuai4 di4}[][HSK 4]
    \definition[个]{s.}{correio rápido; entrega expressa; entrega rápida}
    \definition{v.}{entregar (serviço de entrega rápida por transportadoras especializadas)}
  \end{phonetics}
\end{entry}

\begin{entry}{快速}{7,10}{⼼、⾡}
  \begin{phonetics}{快速}{kuai4 su4}[][HSK 3]
    \definition{adj.}{rápido; veloz; de alta velocidade}
  \end{phonetics}
\end{entry}

\begin{entry}{快餐}{7,16}{⼼、⾷}
  \begin{phonetics}{快餐}{kuai4 can1}[][HSK 2]
    \definition[份,顿]{s.}{pedido (comida) rápido; \emph{fast food}; refere-se a refeições simples preparadas com antecedência e que podem ser servidas rapidamente}
  \end{phonetics}
\end{entry}

\begin{entry}{怀旧}{7,5}{⼼、⽇}
  \begin{phonetics}{怀旧}{huai2jiu4}
    \definition{s.}{nostalgia}
    \definition{v.}{sentir-se nostálgico}
  \end{phonetics}
\end{entry}

\begin{entry}{怀念}{7,8}{⼼、⼼}
  \begin{phonetics}{怀念}{huai2nian4}[][HSK 4]
    \definition{v.}{pensar em; valorizar a memória de}
  \end{phonetics}
\end{entry}

\begin{entry}{怀疑}{7,14}{⼼、⽦}
  \begin{phonetics}{怀疑}{huai2yi2}[][HSK 4]
    \definition{v.}{duvidar; suspeitar | supor}
  \end{phonetics}
\end{entry}

\begin{entry}{我}{7}{⼽}
  \begin{phonetics}{我}{wo3}[][HSK 1]
    \definition{pron.}{eu; mim | um; qualquer um; usado para contrastar 他 e 我; refere-se a muitas pessoas em geral}
  \seealsoref{他}{ta1}
  \end{phonetics}
\end{entry}

\begin{entry}{我们}{7,5}{⼽、⼈}
  \begin{phonetics}{我们}{wo3men5}[][HSK 1]
    \definition{pron.}{nós; nos}
  \end{phonetics}
\end{entry}

\begin{entry}{我们的}{7,5,8}{⼽、⼈、⽩}
  \begin{phonetics}{我们的}{wo3men5 de5}
    \definition{pron.}{nosso, nossos}
  \end{phonetics}
\end{entry}

\begin{entry}{我去}{7,5}{⼽、⼛}
  \begin{phonetics}{我去}{wo3qu4}
    \definition{interj.}{(gíria) O que\dots!! | Oh meu Deus! | Isso é insano!}
  \end{phonetics}
\end{entry}

\begin{entry}{我的}{7,8}{⼽、⽩}
  \begin{phonetics}{我的}{wo3 de5}
    \definition{pron.}{meu, meus}
  \end{phonetics}
\end{entry}

\begin{entry}{戒}{7}{⼽}
  \begin{phonetics}{戒}{jie4}[][HSK 5]
    \definition[个]{s.}{advertência; exortação | disciplina monástica budista; preceitos budistas | anel (dedo)}
    \definition{v.}{proteger-se contra; estar preparado; estar atento | advertir; exortar; admoestar | abandonar; parar; desistir; desistir (de um hábito ruim)}
  \end{phonetics}
\end{entry}

\begin{entry}{扮演}{7,14}{⼿、⽔}
  \begin{phonetics}{扮演}{ban4yan3}[][HSK 5]
    \definition{v.}{desempenhar o papel de; ter um papel (em uma peça, etc.); atuar}
  \end{phonetics}
\end{entry}

\begin{entry}{扶}{7}{⼿}
  \begin{phonetics}{扶}{fu2}[][HSK 5]
    \definition*{s.}{sobrenome Fu}
    \definition{v.}{segurar; apoiar com a mão; segurar algo com o apoio das mãos para que ninguém, objeto ou pessoa caia | dar apoio a; ajudar uma pessoa deitada ou caída a se levantar com as mãos; endireitar um objeto caído com as mãos | ajudar; tirar de baixo}
  \end{phonetics}
\end{entry}

\begin{entry}{扶梯}{7,11}{⼿、⽊}
  \begin{phonetics}{扶梯}{fu2ti1}
    \definition{s.}{escada rolante}
  \end{phonetics}
\end{entry}

\begin{entry}{批}{7}{⼿}
  \begin{phonetics}{批}{pi1}[][HSK 4]
    \definition{adj.}{(compra ou venda) atacado; a granel; em grandes quantidades}
    \definition{clas.}{para mercadorias a granel, grande número de pessoas}
    \definition{s.}{fibras de algodão, linho, etc., prontas para serem estiradas e torcidas | anotação; comentário}
    \definition{v.}{escrever comentários ou críticas sobre documentos subordinados, textos de outras pessoas, tarefas etc. | refutar; criticar | dar um tapa}
  \end{phonetics}
\end{entry}

\begin{entry}{批评}{7,7}{⼿、⾔}
  \begin{phonetics}{批评}{pi1ping2}[][HSK 3]
    \definition{s.}{crítica}
    \definition{v.}{criticar; comentar sobre}
  \end{phonetics}
\end{entry}

\begin{entry}{批准}{7,10}{⼿、⼎}
  \begin{phonetics}{批准}{pi1zhun3}[][HSK 3]
    \definition{v.}{aprovar}
  \end{phonetics}
\end{entry}

\begin{entry}{找}{7}{⼿}
  \begin{phonetics}{找}{zhao3}[][HSK 1]
    \definition{v.}{procurar; tentar encontrar; buscar | querer ver; visitar; abordar; solicitar | dar troco | descobrir; esforçar-se para ver ou obter a pessoa ou coisa desejada | examinar; investigar; completar as partes que faltam | causar intencionalmente (um resultado indesejável, negativo)}
  \end{phonetics}
\end{entry}

\begin{entry}{找见}{7,4}{⼿、⾒}
  \begin{phonetics}{找见}{zhao3jian4}
    \definition{v.}{encontrar (algo que está procurando)}
  \end{phonetics}
\end{entry}

\begin{entry}{找出}{7,5}{⼿、⼐}
  \begin{phonetics}{找出}{zhao3 chu1}[][HSK 2]
    \definition{v.}{encontrar | procurar}
  \end{phonetics}
\end{entry}

\begin{entry}{找回}{7,6}{⼿、⼞}
  \begin{phonetics}{找回}{zhao3hui2}
    \definition{v.}{recuperar algo}
  \end{phonetics}
\end{entry}

\begin{entry}{找寻}{7,6}{⼿、⼨}
  \begin{phonetics}{找寻}{zhao3xun2}
    \definition{v.}{encontrar falhas | procurar | buscar}
  \end{phonetics}
\end{entry}

\begin{entry}{找事}{7,8}{⼿、⼅}
  \begin{phonetics}{找事}{zhao3shi4}
    \definition{v.}{procurar emprego | começar uma briga}
  \end{phonetics}
\end{entry}

\begin{entry}{找到}{7,8}{⼿、⼑}
  \begin{phonetics}{找到}{zhao3 dao4}[][HSK 1]
    \definition{v.}{encontrar; procurar; achar; encontar através de pesquisa, exploração, etc.;  ver ou encontrar coisas ou padrões que os antepassados não viram}
  \end{phonetics}
\end{entry}

\begin{entry}{找钱}{7,10}{⼿、⾦}
  \begin{phonetics}{找钱}{zhao3qian2}
    \definition{v.}{dar troco}
  \end{phonetics}
\end{entry}

\begin{entry}{找着}{7,11}{⼿、⽬}
  \begin{phonetics}{找着}{zhao3zhao2}
    \definition{v.}{encontrar}
  \end{phonetics}
\end{entry}

\begin{entry}{找遍}{7,12}{⼿、⾡}
  \begin{phonetics}{找遍}{zhao3bian4}
    \definition{v.}{pentear | pesquisar em todos os lugares}
  \end{phonetics}
\end{entry}

\begin{entry}{找零}{7,13}{⼿、⾬}
  \begin{phonetics}{找零}{zhao3ling2}
    \definition{v.}{trocar dinheiro | dar troco}
  \end{phonetics}
\end{entry}

\begin{entry}{找辙}{7,16}{⼿、⾞}
  \begin{phonetics}{找辙}{zhao3zhe2}
    \definition{v.}{procurar um pretexto}
  \end{phonetics}
\end{entry}

\begin{entry}{技巧}{7,5}{⼿、⼯}
  \begin{phonetics}{技巧}{ji4qiao3}[][HSK 4]
    \definition{s.}{habilidade; técnica; habilidades engenhosas expressas em artes, artesanato, esportes, etc.}
  \end{phonetics}
\end{entry}

\begin{entry}{技术}{7,5}{⼿、⽊}
  \begin{phonetics}{技术}{ji4shu4}[][HSK 3]
    \definition[种,门,项]{s.}{habilidade; técnica; tecnologia}
  \end{phonetics}
\end{entry}

\begin{entry}{技俩}{7,9}{⼿、⼈}
  \begin{phonetics}{技俩}{ji4liang3}
    \definition{s.}{truque | estratagema | ardil | esquema | estratégia | tática}
  \end{phonetics}
\end{entry}

\begin{entry}{技能}{7,10}{⼿、⾁}
  \begin{phonetics}{技能}{ji4 neng2}[][HSK 5]
    \definition[种,项]{s.}{habilidade técnica; domínio de uma habilidade ou técnica; capacidade de adquirir e aplicar conhecimento}
  \end{phonetics}
\end{entry}

\begin{entry}{抄}{7}{⼿}
  \begin{phonetics}{抄}{chao1}[][HSK 4]
    \definition*{s.}{sobrenome Chao}
    \definition{v.}{copiar; transcrever | plagiar | revistar e confiscar; fazer uma batida | pegar um atalho | dobrar (os braços) | agarrar; pegar}
  \end{phonetics}
\end{entry}

\begin{entry}{抄写}{7,5}{⼿、⼍}
  \begin{phonetics}{抄写}{chao1 xie3}[][HSK 4]
    \definition{v.}{copiar; transcrever}
  \end{phonetics}
\end{entry}

\begin{entry}{把}{7}{⼿}
  \begin{phonetics}{把}{ba3}[][HSK 3]
    \definition{clas.}{para objetos com alça | para objetos pequenos:~punhado}
    \definition{part.}{partícula tornando o substantivo seguinte um objeto direto}
    \definition{v.}{conter | alcançar | segurar}
  \end{phonetics}
  \begin{phonetics}{把}{ba4}
    \definition{v.}{lidar}
  \end{phonetics}
\end{entry}

\begin{entry}{把风}{7,4}{⼿、⾵}
  \begin{phonetics}{把风}{ba3feng1}
    \definition{v.}{estar atento | vigiar (durante uma atividade clandestina)}
  \end{phonetics}
\end{entry}

\begin{entry}{把关}{7,6}{⼿、⼋}
  \begin{phonetics}{把关}{ba3guan1}
    \definition{v.}{verificar estritamente | examinar cuidadosamente para ver se algo é feito de acordo com um padrão fixo | fazer a verificação final | guardar uma passagem, fronteira}
  \end{phonetics}
\end{entry}

\begin{entry}{把守}{7,6}{⼿、⼧}
  \begin{phonetics}{把守}{ba3shou3}
    \definition{v.}{vigiar | guardar}
  \end{phonetics}
\end{entry}

\begin{entry}{把式}{7,6}{⼿、⼷}
  \begin{phonetics}{把式}{ba3shi4}
    \definition{s.}{pessoa qualificada em um comércio}
  \end{phonetics}
\end{entry}

\begin{entry}{把戏}{7,6}{⼿、⼽}
  \begin{phonetics}{把戏}{ba3xi4}
    \definition{s.}{acrobacia | malabarismo | truque barato}
  \end{phonetics}
\end{entry}

\begin{entry}{把玩}{7,8}{⼿、⽟}
  \begin{phonetics}{把玩}{ba3wan2}
    \definition{v.}{brincar com | mexer com}
  \end{phonetics}
\end{entry}

\begin{entry}{把持}{7,9}{⼿、⼿}
  \begin{phonetics}{把持}{ba3chi2}
    \definition{v.}{controlar | dominar | monopolizar}
  \end{phonetics}
\end{entry}

\begin{entry}{把柄}{7,9}{⼿、⽊}
  \begin{phonetics}{把柄}{ba3bing3}
    \definition{s.}{(figurativo) informações que podem ser usadas contra alguém}
  \end{phonetics}
\end{entry}

\begin{entry}{把脉}{7,9}{⼿、⾁}
  \begin{phonetics}{把脉}{ba3mai4}
    \definition{v.}{sentir ou tomar o pulso de alguém}
  \end{phonetics}
\end{entry}

\begin{entry}{把握}{7,12}{⼿、⼿}
  \begin{phonetics}{把握}{ba3wo4}[][HSK 3]
    \definition{s.}{seguro | garantia | certeza}
    \definition{v.}{agarrar | segurar | aproveitar}
  \end{phonetics}
\end{entry}

\begin{entry}{把稳}{7,14}{⼿、⽲}
  \begin{phonetics}{把稳}{ba3wen3}
    \definition{adj.}{confiável}
  \end{phonetics}
\end{entry}

\begin{entry}{抓}{7}{⼿}
  \begin{phonetics}{抓}{zhua1}[][HSK 3]
    \definition{v.}{agarrar | arranhar | capturar | compreender; conhecer a chave ou a chave das coisas ou problemas | focar em algo; fortalecer o poder de fazer (algo) ou administrar (algum aspecto) | atrair a atenção de alguém}
  \end{phonetics}
\end{entry}

\begin{entry}{抓住}{7,7}{⼿、⼈}
  \begin{phonetics}{抓住}{zhua1 zhu4}[][HSK 3]
    \definition{v.}{apanhar; prender; capturar (uma pessoa ou animal) e ter sucesso | segurar; agarrar; segurar algo e deixá-lo imóvel}
  \end{phonetics}
\end{entry}

\begin{entry}{抓紧}{7,10}{⼿、⽷}
  \begin{phonetics}{抓紧}{zhua1jin3}[][HSK 4]
    \definition{v.}{agarrar com firmeza; segurar firme e não soltar | prestar muita atenção a}
  \end{phonetics}
\end{entry}

\begin{entry}{投}{7}{⼿}
  \begin{phonetics}{投}{tou2}[][HSK 4]
    \definition*{s.}{sobrenome Tou}
    \definition{pron.}{para; indica tempo, equivalente a 到, 临 | para; em direção a; indica orientação, direção, equivalente a 朝 ou 向}
    \definition{s.}{um jogo durante uma festa em que o vencedor era decidido pelo número de flechas lançadas em um pote distante | jogo de dados}
    \definition{v.}{lançar; arremessar; atirar | deixar cair; colocar em; lançar | mergulhar em; lançar-se em; pular dentro | lançar; projetar; sombrear | entregar; postar; enviar | ir até; ir para; buscar; juntar-se | sentir-se atraído por; adaptar-se a; concordar com; atender a}
  \seealsoref{朝}{chao2}
  \seealsoref{到}{dao4}
  \seealsoref{临}{lin2}
  \seealsoref{向}{xiang4}
  \end{phonetics}
\end{entry}

\begin{entry}{投入}{7,2}{⼿、⼊}
  \begin{phonetics}{投入}{tou2ru4}[][HSK 4]
    \definition{adj.}{sisudo; dedicado; devotado; absorto}
    \definition{s.}{investimento; insumo; refere-se à aplicação de recursos}
    \definition{v.}{lançar em; colocar em; jogar em; por em | entrar em uma situação; participar de | aplicar; investir; colocar fundos em}
  \end{phonetics}
\end{entry}

\begin{entry}{投诉}{7,7}{⼿、⾔}
  \begin{phonetics}{投诉}{tou2su4}[][HSK 4]
    \definition{v.}{reclamar; queixar-se; reclamar às autoridades ou pessoas envolvidas}
  \end{phonetics}
\end{entry}

\begin{entry}{投资}{7,10}{⼿、⾙}
  \begin{phonetics}{投资}{tou2zi1}[][HSK 4]
    \definition[次]{s.}{investimento}
    \definition{v.}{investir; aplicar dinheiro; investir dinheiro em negócios}
  \end{phonetics}
\end{entry}

\begin{entry}{投资人}{7,10,2}{⼿、⾙、⼈}
  \begin{phonetics}{投资人}{tou2zi1ren2}
    \definition{s.}{investidor}
  \seealsoref{投资家}{tou2zi1jia1}
  \seealsoref{投资者}{tou2zi1zhe3}
  \end{phonetics}
\end{entry}

\begin{entry}{投资风险}{7,10,4,9}{⼿、⾙、⾵、⾩}
  \begin{phonetics}{投资风险}{tou2zi1feng1xian3}
    \definition{s.}{risco de investimento}
  \end{phonetics}
\end{entry}

\begin{entry}{投资回报率}{7,10,6,7,11}{⼿、⾙、⼞、⼿、⽞}
  \begin{phonetics}{投资回报率}{tou2zi1hui2bao4lv4}
    \definition{s.}{retorno sobre o investimento (ROI)}
  \end{phonetics}
\end{entry}

\begin{entry}{投资者}{7,10,8}{⼿、⾙、⽼}
  \begin{phonetics}{投资者}{tou2zi1zhe3}
    \definition{s.}{investidor}
  \seealsoref{投资家}{tou2zi1jia1}
  \seealsoref{投资人}{tou2zi1ren2}
  \end{phonetics}
\end{entry}

\begin{entry}{投资家}{7,10,10}{⼿、⾙、⼧}
  \begin{phonetics}{投资家}{tou2zi1jia1}
    \definition{s.}{investidor}
  \seealsoref{投资人}{tou2zi1ren2}
  \seealsoref{投资者}{tou2zi1zhe3}
  \end{phonetics}
\end{entry}

\begin{entry}{投递}{7,10}{⼿、⾡}
  \begin{phonetics}{投递}{tou2di4}
    \definition{v.}{despachar | enviar}
  \end{phonetics}
\end{entry}

\begin{entry}{投票}{7,11}{⼿、⽰}
  \begin{phonetics}{投票}{tou2piao4}
    \definition{v.+compl.}{votar | depositar um voto}
  \end{phonetics}
\end{entry}

\begin{entry}{折}{7}{⼿}
  \begin{phonetics}{折}{she2}
    \definition{v.}{estalar; quebrar | perder dinheiro em um negócio}
  \end{phonetics}
  \begin{phonetics}{折}{zhe1}
    \definition{v.}{rolar; virar | despejar algo de um recipiente em outro; ficar despejando algo entre dois recipientes}
  \end{phonetics}
  \begin{phonetics}{折}{zhe2}[][HSK 4]
    \definition*{s.}{sobrenome Zhe}
    \definition{clas.}{uma passagem em um roteiro de ópera miscelânea de Yuan, aproximadamente equivalente a uma cena ou ato em uma ópera moderna}
    \definition[张,个,些]{s.}{fratura; quebra | abatimento; desconto | traços dos caracteres chineses que têm o formato de "𠃍" e "乚", etc. | pasta; livreto; \emph{folder}}
    \definition{v.}{estalar; quebrar; fazer quebrar | perder; sofrer a perda de | voltar para trás; mudar de direção; retornar |ser convencido; estar cheio de admiração | equivaler a; converter em | dobrar}
  \end{phonetics}
\end{entry}

\begin{entry}{折转}{7,8}{⼿、⾞}
  \begin{phonetics}{折转}{zhe2zhuan3}
    \definition{s.}{reflexo (ângulo)}
    \definition{v.}{voltar atrás}
  \end{phonetics}
\end{entry}

\begin{entry}{抢}{7}{⼿}
  \begin{phonetics}{抢}{qiang1}
    \definition{prep.}{contra; direção relativa inversa}
    \definition{v.}{bater; tocar}
  \end{phonetics}
  \begin{phonetics}{抢}{qiang3}[][HSK 5]
    \definition{v.}{roubar; saquear | agarrar; apanhar; arrebatar | disputar; lutar por; ser o primeiro; competir para ser o primeiro | correr; apressar-se; fazer uma incursão | raspar; arranhar; raspar ou esfregar uma camada da superfície de um objeto}
  \end{phonetics}
\end{entry}

\begin{entry}{抢掠}{7,11}{⼿、⼿}
  \begin{phonetics}{抢掠}{qiang3lve4}
    \definition{s.}{saque | pilhagem}
    \definition{v.}{saquear | pilhar}
  \end{phonetics}
\end{entry}

\begin{entry}{抢救}{7,11}{⼿、⽁}
  \begin{phonetics}{抢救}{qiang3jiu4}[][HSK 5]
    \definition{v.}{salvar; resgatar; prestar de socorro ou assistência rápidos em situações de emergência | salvar; tomar medidas rápidas para evitar ou minimizar perdas iminentes.}
  \end{phonetics}
\end{entry}

\begin{entry}{护士}{7,3}{⼿、⼠}
  \begin{phonetics}{护士}{hu4shi5}[][HSK 4]
    \definition[名,位]{s.}{enfermeiro; pessoas especializadas em enfermagem em hospitais ou instituições epidemiológicas}
  \end{phonetics}
\end{entry}

\begin{entry}{护照}{7,13}{⼿、⽕}
  \begin{phonetics}{护照}{hu4zhao4}[][HSK 2]
    \definition[本,个]{s.}{passaporte; documento emitido pela autoridade competente do país para comprovar a nacionalidade e a identidade dos cidadãos que viajam para o exterior}
  \end{phonetics}
\end{entry}

\begin{entry}{报}{7}{⼿}
  \begin{phonetics}{报}{bao4}[][HSK 3]
    \definition[份,张]{s.}{jornal | recompensa | relatório | vingança}
    \definition{v.}{anunciar | informar}
  \end{phonetics}
\end{entry}

\begin{entry}{报名}{7,6}{⼿、⼝}
  \begin{phonetics}{报名}{bao4ming2}[][HSK 2]
    \definition{v.+compl.}{inscrever-se; alistar-se; registrar seu nome; cadastrar-se; matricular-se; informar seu nome à pessoa responsável, órgão, grupo etc., indicando que você deseja participar de alguma atividade ou organização}
  \end{phonetics}
\end{entry}

\begin{entry}{报告}{7,7}{⼿、⼝}
  \begin{phonetics}{报告}{bao4gao4}[][HSK 3]
    \definition[份,篇,分,个,通]{s.}{relatório | discurso | palestra | aconselhamento}
    \definition{v.}{relatar | dar a conhecer | informar}
  \end{phonetics}
\end{entry}

\begin{entry}{报纸}{7,7}{⼿、⽷}
  \begin{phonetics}{报纸}{bao4zhi3}[][HSK 2]
    \definition[分,期,张]{s.}{jornal; publicações periódicas cujo conteúdo principal é notícias, geralmente referem-se a jornais diários | papel jornal; um tipo de papel usado para imprimir jornais ou publicações em geral}
  \end{phonetics}
\end{entry}

\begin{entry}{报到}{7,8}{⼿、⼑}
  \begin{phonetics}{报到}{bao4dao4}[][HSK 3]
    \definition{v.+compl.}{apresentar-se para o serviço | fazer check-in | registrar-se | assinar}
  \end{phonetics}
\end{entry}

\begin{entry}{报答}{7,12}{⼿、⽵}
  \begin{phonetics}{报答}{bao4da2}[][HSK 5]
    \definition{v.}{reembolsar; devolver; retribuir; pagar de volta; mostrar seu apreço de forma tangível}
  \end{phonetics}
\end{entry}

\begin{entry}{报道}{7,12}{⼿、⾡}
  \begin{phonetics}{报道}{bao4dao4}[][HSK 3]
    \definition[个,篇,分]{s.}{história | reportagem}
    \definition{v.}{cobrir | relatar (notícias)}
  \end{phonetics}
\end{entry}

\begin{entry}{报酬}{7,13}{⼿、⾣}
  \begin{phonetics}{报酬}{bao4chou5}
    \definition{s.}{recompensa | remuneração}
  \end{phonetics}
\end{entry}

\begin{entry}{报警}{7,19}{⼿、⾔}
  \begin{phonetics}{报警}{bao4jing3}[][HSK 5]
    \definition{v.}{relatar (um incidente) à polícia; relatar uma situação crítica ou sinalizar uma emergência às autoridades competentes}
  \end{phonetics}
\end{entry}

\begin{entry}{拒绝}{7,9}{⼿、⽷}
  \begin{phonetics}{拒绝}{ju4jue2}[][HSK 5]
    \definition{v.}{recusar; rejeitar; declinar; não aceitar (pedidos, sugestões ou presentes)}
  \end{phonetics}
\end{entry}

\begin{entry}{改}{7}{⽁}
  \begin{phonetics}{改}{gai3}[][HSK 2]
    \definition*{s.}{sobrenome Gai}
    \definition{v.}{mudar; converter; transformar; alterar; substituir | alterar; revisar; aperfeiçoar; modificar | corrigir; retificar; remediar; consertar}
  \end{phonetics}
\end{entry}

\begin{entry}{改正}{7,5}{⽁、⽌}
  \begin{phonetics}{改正}{gai3 zheng4}[][HSK 4]
    \definition{v.}{corrigir; emendar; mudar o errado para o correto}
  \end{phonetics}
\end{entry}

\begin{entry}{改良}{7,7}{⽁、⾉}
  \begin{phonetics}{改良}{gai3liang2}
    \definition{v.}{melhorar (algo) | reformar (um sistema)}
  \end{phonetics}
\end{entry}

\begin{entry}{改进}{7,7}{⽁、⾡}
  \begin{phonetics}{改进}{gai3jin4}[][HSK 3]
    \definition[个]{s.}{melhoria}
    \definition{v.}{aprimorar; aperfeiçoar; melhorar; tornar melhor
modificar}
  \end{phonetics}
\end{entry}

\begin{entry}{改变}{7,8}{⽁、⼜}
  \begin{phonetics}{改变}{gai3bian4}[][HSK 2]
    \definition{v.}{mudar; alterar; transformar; converter; moldar; modificar | causar mudanças; alterar}
  \end{phonetics}
\end{entry}

\begin{entry}{改革}{7,9}{⽁、⾰}
  \begin{phonetics}{改革}{gai3ge2}[][HSK 5]
    \definition[项,次,种]{s.}{reforma; reformação; iniciativas para aprimorar a inovação}
    \definition{v.}{reformar; transformar as antigas partes irracionais das coisas em novas que possam ser adaptadas à situação objetiva}
  \end{phonetics}
\end{entry}

\begin{entry}{改造}{7,10}{⽁、⾡}
  \begin{phonetics}{改造}{gai3 zao4}[][HSK 3]
    \definition{v.}{transformar; renovar | remodelar}
  \end{phonetics}
\end{entry}

\begin{entry}{改善}{7,12}{⽁、⼝}
  \begin{phonetics}{改善}{gai3shan4}[][HSK 4]
    \definition{v.}{melhorar; amenizar; mudar a situação original para torná-la melhor}
  \end{phonetics}
\end{entry}

\begin{entry}{改善关系}{7,12,6,7}{⽁、⼝、⼋、⽷}
  \begin{phonetics}{改善关系}{gai3shan4guan1xi5}
    \definition{v.}{melhorar a relação}
  \end{phonetics}
\end{entry}

\begin{entry}{改善通讯}{7,12,10,5}{⽁、⼝、⾡、⾔}
  \begin{phonetics}{改善通讯}{gai3shan4tong1xun4}
    \definition{v.}{melhorar a comunicação}
  \end{phonetics}
\end{entry}

\begin{entry}{时}{7}{⽇}
  \begin{phonetics}{时}{shi2}[][HSK 3]
    \definition*{s.}{sobrenome Shi}
    \definition{adj.}{atual; presente | a tempo; feito a tempo}
    \definition{adv.}{de vez em quando; ocasionalmente; de ​​tempos em tempos | às vezes\dots às vezes\dots}
    \definition{clas.}{hora; horas}
    \definition{s.}{dias; tempos; longo período de tempo | tempo; tempo fixo | hora; hora do dia | temporada | chance; oportunidade | atualidade; presente | tempo verbal}
  \end{phonetics}
\end{entry}

\begin{entry}{时代}{7,5}{⽇、⼈}
  \begin{phonetics}{时代}{shi2dai4}[][HSK 3]
    \definition[个]{s.}{idade; era; tempos; época | um período na vida de alguém}
  \end{phonetics}
\end{entry}

\begin{entry}{时光}{7,6}{⽇、⼉}
  \begin{phonetics}{时光}{shi2guang1}[][HSK 5]
    \definition[台]{s.}{tempo; passagem do tempo | dias; horas; anos; épocas; períodos}
  \end{phonetics}
\end{entry}

\begin{entry}{时机}{7,6}{⽇、⽊}
  \begin{phonetics}{时机}{shi2ji1}[][HSK 5]
    \definition{s.}{oportunidade; momento oportuno}
  \end{phonetics}
\end{entry}

\begin{entry}{时时}{7,7}{⽇、⽇}
  \begin{phonetics}{时时}{shi2shi2}
    \definition{adv.}{muitas vezes | constantemente}
  \end{phonetics}
\end{entry}

\begin{entry}{时间}{7,7}{⽇、⾨}
  \begin{phonetics}{时间}{shi2jian1}[][HSK 1]
    \definition[段]{s.}{tempo; refere-se à forma de existência do movimento da matéria, um sistema contínuo composto pelo passado, presente e futuro | tempo; período (duração); um período de tempo com início e fim | tempo (um ponto); em algum momento do tempo}
  \end{phonetics}
\end{entry}

\begin{entry}{时事}{7,8}{⽇、⼅}
  \begin{phonetics}{时事}{shi2shi4}[][HSK 5]
    \definition{s.}{acontecimentos atuais; assuntos atuais; eventos atuais | tendências atuais | como as coisas estão indo | a situação atual}
  \end{phonetics}
\end{entry}

\begin{entry}{时刻}{7,8}{⽇、⼑}
  \begin{phonetics}{时刻}{shi2ke4}[][HSK 3]
    \definition{adv.}{constantemente; sempre}
    \definition[个,段]{s.}{tempo; hora; momento; conjuntura}
  \end{phonetics}
\end{entry}

\begin{entry}{时差}{7,9}{⽇、⼯}
  \begin{phonetics}{时差}{shi2cha1}
    \definition{s.}{diferença de tempo | \emph{jet lag}}
  \end{phonetics}
\end{entry}

\begin{entry}{时候}{7,10}{⽇、⼈}
  \begin{phonetics}{时候}{shi2hou5}[][HSK 1]
    \definition[个]{s.}{(um ponto no) tempo; momento; um determinado momento no tempo | (a duração do) tempo; um período de tempo com início e fim}
  \end{phonetics}
\end{entry}

\begin{entry}{时常}{7,11}{⽇、⼱}
  \begin{phonetics}{时常}{shi2chang2}[][HSK 5]
    \definition{adv.}{frequentemente; com frequência}
  \end{phonetics}
\end{entry}

\begin{entry}{旷野}{7,11}{⽇、⾥}
  \begin{phonetics}{旷野}{kuang4ye3}
    \definition{s.}{região selvagem}
  \end{phonetics}
\end{entry}

\begin{entry}{更}{7}{⽈}
  \begin{phonetics}{更}{geng1}
    \definition*{s.}{sobrenome Geng}
    \definition{clas.}{um dos cinco períodos de duas horas em que a noite era anteriormente dividida; vigília; antigamente, a noite era dividida em cinco turnos, cada um com aproximadamente duas horas de duração}
    \definition{v.}{alterar; substituir | experimentar}
  \end{phonetics}
  \begin{phonetics}{更}{geng4}[][HSK 2]
    \definition{adv.}{mais; ainda mais | além disso; além do mais; ainda mais}
  \end{phonetics}
\end{entry}

\begin{entry}{更加}{7,5}{⽈、⼒}
  \begin{phonetics}{更加}{geng4 jia1}[][HSK 3]
    \definition{adv.}{mais; ainda mais; em maior grau}
  \end{phonetics}
\end{entry}

\begin{entry}{更换}{7,10}{⽈、⼿}
  \begin{phonetics}{更换}{geng1 huan4}[][HSK 5]
    \definition{v.}{alterar; mudar; substituir; comutar}
  \end{phonetics}
\end{entry}

\begin{entry}{更新}{7,13}{⽈、⽄}
  \begin{phonetics}{更新}{geng1xin1}[][HSK 5]
    \definition{v.}{renovar; atualizar; substituir; remover o antigo e substituir pelo novo}
  \end{phonetics}
\end{entry}

\begin{entry}{李}{7}{⽊}
  \begin{phonetics}{李}{li3}
    \definition*{s.}{sobrenome Li}
    \definition{s.}{ameixa}
  \end{phonetics}
\end{entry}

\begin{entry}{李子}{7,3}{⽊、⼦}
  \begin{phonetics}{李子}{li3zi5}
    \definition[个]{s.}{ameixa}
  \end{phonetics}
\end{entry}

\begin{entry}{李四}{7,5}{⽊、⼞}
  \begin{phonetics}{李四}{li3si4}
    \definition*{s.}{Li Si | Zé Ninguém | nome para uma pessoa não especificada, 2 de 3}
  \seealsoref{王五}{wang2wu3}
  \seealsoref{张三}{zhang1san1}
  \end{phonetics}
\end{entry}

\begin{entry}{材料}{7,10}{⽊、⽃}
  \begin{phonetics}{材料}{cai2liao4}[][HSK 4]
    \definition[份,个,种]{s.}{material; algo para fazer um produto acabado | material (figura de linguagem) | dados; material para estudo, pesquisa, etc.; conteúdo de uma obra}
  \end{phonetics}
\end{entry}

\begin{entry}{村}{7}{⽊}
  \begin{phonetics}{村}{cun1}[][HSK 3]
    \definition{adj.}{rústico; grosseiro}
    \definition{s.}{aldeia; vila}
  \end{phonetics}
\end{entry}

\begin{entry}{杜宇}{7,6}{⽊、⼧}
  \begin{phonetics}{杜宇}{du4yu3}
    \definition{s.}{cuco (pássaro)}
  \seealsoref{布谷鸟}{bu4gu3niao3}
  \seealsoref{杜鹃}{du4juan1}
  \seealsoref{杜鹃鸟}{du4juan1niao3}
  \end{phonetics}
\end{entry}

\begin{entry}{杜鹃}{7,12}{⽊、⿃}
  \begin{phonetics}{杜鹃}{du4juan1}
    \definition{s.}{cuco (pássaro)}
  \seealsoref{布谷鸟}{bu4gu3niao3}
  \seealsoref{杜鹃鸟}{du4juan1niao3}
  \seealsoref{杜宇}{du4yu3}
  \end{phonetics}
\end{entry}

\begin{entry}{杜鹃鸟}{7,12,5}{⽊、⿃、⿃}
  \begin{phonetics}{杜鹃鸟}{du4juan1niao3}
    \definition{s.}{cuco (pássaro)}
  \seealsoref{布谷鸟}{bu4gu3niao3}
  \seealsoref{杜鹃}{du4juan1}
  \seealsoref{杜宇}{du4yu3}
  \end{phonetics}
\end{entry}

\begin{entry}{束}{7}{⽊}
  \begin{phonetics}{束}{shu4}[][HSK 3]
    \definition*{s.}{sobrenome Shu}
    \definition{clas.}{para cachos, molhos, feixes, feixes de luz, etc.}
    \definition{s.}{monte; pacote; maço; feixe; cacho}
    \definition{v.}{atar; amarrar; vincular | controlar; restringir}
  \end{phonetics}
\end{entry}

\begin{entry}{束腰}{7,13}{⽊、⾁}
  \begin{phonetics}{束腰}{shu4yao1}
    \definition{s.}{cinto | cinta | cinturão}
  \end{phonetics}
\end{entry}

\begin{entry}{杠}{7}{⽊}
  \begin{phonetics}{杠}{gang1}
    \definition{s.}{mastro de bandeira | poste | passarela}
  \end{phonetics}
  \begin{phonetics}{杠}{gang4}
    \definition{s.}{vara grossa | barra | linha grossa | padrão, critério | hífen, traço}
    \definition{v.}{marcar com uma linha grossa | afiar (faca, navalha, etc.)}
  \end{phonetics}
\end{entry}

\begin{entry}{条}{7}{⽊}
  \begin{phonetics}{条}{tiao2}[][HSK 2]
    \definition*{s.}{sobrenome Tiao}
    \definition{clas.}{usado para objetos longos e finos; usado para sintetizar certas coisas longas e retangulares em quantidades fixas | usado para itemização | aplicado ao corpo humano}
    \definition{s.}{galho; galhos finos e longos | tira; faixa | item; artigo | ordem; método | nota; anotação em papel}
  \end{phonetics}
\end{entry}

\begin{entry}{条目}{7,5}{⽊、⽬}
  \begin{phonetics}{条目}{tiao2mu4}
    \definition{s.}{cláusulas e subcláusulas (em documento formal) | verbete (em um dicionário, enciclopédia, etc.)}
  \end{phonetics}
\end{entry}

\begin{entry}{条件}{7,6}{⽊、⼈}
  \begin{phonetics}{条件}{tiao2jian4}[][HSK 2]
    \definition[个,项,些]{s.}{condição; termo; fator; fatores que restringem a ocorrência, existência ou desenvolvimento das coisas | requisito; pré-requisito; qualificação; requisitos ou padrões estabelecidos para determinadas coisas | situação; estado; condição}
  \end{phonetics}
\end{entry}

\begin{entry}{条例}{7,8}{⽊、⼈}
  \begin{phonetics}{条例}{tiao2li4}
    \definition{s.}{código de conduta | ordenanças | regulamentos | regras | estatutos}
  \end{phonetics}
\end{entry}

\begin{entry}{条贯}{7,8}{⽊、⾙}
  \begin{phonetics}{条贯}{tiao2guan4}
    \definition{s.}{ordem | procedimentos | sequência | sistema}
  \end{phonetics}
\end{entry}

\begin{entry}{条幅}{7,12}{⽊、⼱}
  \begin{phonetics}{条幅}{tiao2fu2}
    \definition{s.}{faixa | banner | pergaminho de parede (para pintura ou caligrafia)}
  \end{phonetics}
\end{entry}

\begin{entry}{来}{7}{⽊}
  \begin{phonetics}{来}{lai2}[][HSK 1]
    \definition*{s.}{sobrenome Lai}
    \definition{part.}{usado após uma palavra numérica ou de quantidade; indica uma quantidade aproximada | usado depois de numerais como 一, 二, 三; para listar razões ou fatos, etc.}
    \definition{s.}{usado após uma expressão de tempo para indicar uma duração que vai do passado ao presente}
    \definition{v.}{vir; chegar; de outro lugar para o lugar onde o interlocutor se encontra | aparecer; acontecer; vir; (problemas, coisas, etc.) ocorrerem; surgirem | substitui um verbo com significado específico, indicando a realização de uma ação específica | estar indo para; usado antes de outro verbo, indica que algo será feito | vir para fazer algo; usado após outro verbo, indica que se vai fazer algo | usado para indicar um propósito; expressar o objetivo, fazer algo usando o método, a atitude ou a direção anteriores | usado com 得 ou 不 para indicar possibilidade, capacidade ou hábito}
  \seealsoref{不}{bu4}
  \seealsoref{得}{de5}
  \end{phonetics}
\end{entry}

\begin{entry}{来不及}{7,4,3}{⽊、⼀、⼃}
  \begin{phonetics}{来不及}{lai2bu5ji2}[][HSK 4]
    \definition{v.}{ser tarde demais; não ter tempo; não ter tempo suficiente (para fazer algo); não ser possível participar ou se atualizar devido a restrições de tempo}
  \end{phonetics}
\end{entry}

\begin{entry}{来自}{7,6}{⽊、⾃}
  \begin{phonetics}{来自}{lai2zi4}[][HSK 2]
    \definition{v.}{vir de (um local) | \emph{From:} (cabeçalho de \emph{e -mail})}
  \end{phonetics}
\end{entry}

\begin{entry}{来到}{7,8}{⽊、⼑}
  \begin{phonetics}{来到}{lai2 dao4}[][HSK 1]
    \definition{v.}{chegar; vir}
  \end{phonetics}
\end{entry}

\begin{entry}{来信}{7,9}{⽊、⼈}
  \begin{phonetics}{来信}{lai2 xin4}[][HSK 5]
    \definition{s.}{sua carta; carta recebida; carta ao interlocutor}
    \definition{v.}{enviar uma carta para aqui; enviar uma carta para o remetente}
  \end{phonetics}
\end{entry}

\begin{entry}{来得及}{7,11,3}{⽊、⼻、⼃}
  \begin{phonetics}{来得及}{lai2de5ji2}[][HSK 4]
    \definition{v.}{ainda ter tempo; ser capaz de fazê-lo; ser capaz de fazer algo a tempo; ainda ter tempo de chegar lá ou de se atualizar}
  \end{phonetics}
\end{entry}

\begin{entry}{来源}{7,13}{⽊、⽔}
  \begin{phonetics}{来源}{lai2yuan2}[][HSK 4]
    \definition{s.}{origem; causa; fonte; tabula rasa (ou seja, o lugar de onde as coisas vêm)}
    \definition{v.}{originar-se; surgir; ter origem; (algo) originar (seguido de 于)}
  \seealsoref{于}{yu2}
  \end{phonetics}
\end{entry}

\begin{entry}{极}{7}{⽊}
  \begin{phonetics}{极}{ji2}[][HSK 4]
    \definition*{s.}{sobrenome Ji}
    \definition{adj.}{máximo; extremo; final; supremo}
    \definition{adv.}{extremamente; excessivamente}
    \definition{s.}{o ponto máximo, mais alto; extremo; ápice; ponto culminante |
pólo; as extremidades norte e sul da Terra; as extremidades de um ímã; a extremidade de uma fonte de alimentação ou de um aparelho elétrico onde a corrente entra ou sai do aparelho}
    \definition{v.}{chegar ao fim de; levar a extremos}
  \end{phonetics}
\end{entry}

\begin{entry}{……极了}{7,2}{⽊、⼅}
  \begin{phonetics}{……极了}{ji2le5}[][HSK 3]
    \definition{expr.}{extremamente}
  \end{phonetics}
\end{entry}

\begin{entry}{极其}{7,8}{⽊、⼋}
  \begin{phonetics}{极其}{ji2qi2}[][HSK 4]
    \definition{adv.}{mais; extremamente; excessivamente}
  \end{phonetics}
\end{entry}

\begin{entry}{步}{7}{⽌}
  \begin{phonetics}{步}{bu4}[][HSK 3]
    \definition*{s.}{sobrenome Bu}
    \definition{clas.}{uma unidade antiga para medida de comprimento, equivalente a cinco chi}
    \definition{s.}{ritmo | passo | estágio | passo | condição | situação | estado}
    \definition{v.}{ir a pé | andar | pisar | contar passos}
  \end{phonetics}
\end{entry}

\begin{entry}{步行}{7,6}{⽌、⾏}
  \begin{phonetics}{步行}{bu4 xing2}[][HSK 4]
    \definition{v.}{caminhar; ir a pé; andar a pé (diferente de andar de carro, a cavalo, etc.)}
  \end{phonetics}
\end{entry}

\begin{entry}{每}{7}{⽏}
  \begin{phonetics}{每}{mei3}[][HSK 3]
    \definition*{s.}{sobrenome Mei}
    \definition{adv.}{frequentemente; todo}
    \definition{pron.}{cada; cada um; cada qual;  todo}
  \end{phonetics}
\end{entry}

\begin{entry}{每个人}{7,3,2}{⽏、⼈、⼈}
  \begin{phonetics}{每个人}{mei3ge5ren2}
    \definition{pron.}{todo mundo | todos}
  \end{phonetics}
\end{entry}

\begin{entry}{每天}{7,4}{⽏、⼤}
  \begin{phonetics}{每天}{mei3tian1}
    \definition{adv.}{todo dia | cada dia}
  \end{phonetics}
\end{entry}

\begin{entry}{每次}{7,6}{⽏、⽋}
  \begin{phonetics}{每次}{mei3ci4}
    \definition{adv.}{toda vez | cada vez}
  \end{phonetics}
\end{entry}

\begin{entry}{求}{7}{⽔}
  \begin{phonetics}{求}{qiu2}[][HSK 2]
    \definition*{s.}{sobrenome Qiu}
    \definition{v.}{implorar; solicitar; suplicar; rogar | lutar por; buscar; investigar | tentar; procurar; tentar obter | demandar}
  \end{phonetics}
\end{entry}

\begin{entry}{汹涌}{7,10}{⽔、⽔}
  \begin{phonetics}{汹涌}{xiong1yong3}
    \definition{adj.}{turbulento}
    \definition{v.}{aumentar ou emergir violentamente (oceano, rio, lago, etc.)}
  \end{phonetics}
\end{entry}

\begin{entry}{汽水}{7,4}{⽔、⽔}
  \begin{phonetics}{汽水}{qi4 shui3}[][HSK 4]
    \definition[罐,瓶]{s.}{refrigerante; refrigerante gaseificado; bebida refrescante, feita com a pressão de dióxido de carbono para dissolver na água e adicionar açúcar, suco de frutas, especiarias etc.}
  \end{phonetics}
\end{entry}

\begin{entry}{汽车}{7,4}{⽔、⾞}
  \begin{phonetics}{汽车}{qi4 che1}[][HSK 1]
    \definition[辆,种,款]{s.}{automóvel; carro; veículo motorizado; veículo movido a motor de combustão interna, que circula principalmente em rodovias ou ruas, geralmente com quatro ou mais pneus de borracha, usado para transportar pessoas ou mercadorias}
  \end{phonetics}
\end{entry}

\begin{entry}{汽油}{7,8}{⽔、⽔}
  \begin{phonetics}{汽油}{qi4you2}[][HSK 4]
    \definition{s.}{gasolina; mistura líquida de hidrocarbonetos com volatilidade e combustibilidade, que é usada como combustível a partir do fracionamento ou craqueamento do petróleo}
  \end{phonetics}
\end{entry}

\begin{entry}{沉}{7}{⽔}
  \begin{phonetics}{沉}{chen2}[][HSK 4]
    \definition{adj.}{profundo | pesado | pesado (sentir-se pesado)}
    \definition{v.}{afundar; submergir; imergir | manter baixo; abaixar | descansar; parar}
  \end{phonetics}
\end{entry}

\begin{entry}{沉重}{7,9}{⽔、⾥}
  \begin{phonetics}{沉重}{chen2zhong4}[][HSK 4]
    \definition{adj.}{(pressão, fardo, etc.) muito pesado; profundo | sério; pesado; humor pouco animador; fardo pesado de pensamentos}
  \end{phonetics}
\end{entry}

\begin{entry}{沉默}{7,16}{⽔、⿊}
  \begin{phonetics}{沉默}{chen2mo4}[][HSK 4]
    \definition{adj.}{silencioso; reticente; taciturno; não comunicativo}
    \definition{v.}{silenciar; não falar por causa de alguma coisa}
  \end{phonetics}
\end{entry}

\begin{entry}{沙}{7}{⽔}
  \begin{phonetics}{沙}{sha1}
    \definition*{s.}{sobrenome Sha}
    \definition[粒]{s.}{areia | cascalho | grânulo | pó}
  \end{phonetics}
\end{entry}

\begin{entry}{沙子}{7,3}{⽔、⼦}
  \begin{phonetics}{沙子}{sha1 zi5}[][HSK 3]
    \definition[粒,把]{s.}{areia; grão | \emph{pellets}; grãos pequenos}
  \end{phonetics}
\end{entry}

\begin{entry}{沙发}{7,5}{⽔、⼜}
  \begin{phonetics}{沙发}{sha1fa1}[][HSK 3]
    \definition[套,组,个,张]{s.}{sofá; divã}
  \end{phonetics}
\end{entry}

\begin{entry}{沙鱼}{7,8}{⽔、⿂}
  \begin{phonetics}{沙鱼}{sha1yu2}
    \variantof{鲨鱼}
  \end{phonetics}
\end{entry}

\begin{entry}{沙特}{7,10}{⽔、⽜}
  \begin{phonetics}{沙特}{sha1te4}
    \definition*{s.}{Saudita | abreviação de 沙特阿拉伯}
  \seealsoref{沙特阿拉伯}{sha1te4 a1la1bo2}
  \end{phonetics}
\end{entry}

\begin{entry}{沙特阿拉伯}{7,10,7,8,7}{⽔、⽜、⾩、⼿、⼈}
  \begin{phonetics}{沙特阿拉伯}{sha1te4 a1la1bo2}
    \definition*{s.}{Arábia Saudita}
  \end{phonetics}
\end{entry}

\begin{entry}{沙漠}{7,13}{⽔、⽔}
  \begin{phonetics}{沙漠}{sha1mo4}[][HSK 5]
    \definition[个]{s.}{deserto; superfície totalmente coberta por areia, sem água corrente, clima seco e vegetação escassa}
  \end{phonetics}
\end{entry}

\begin{entry}{沟}{7}{⽔}
  \begin{phonetics}{沟}{gou1}[][HSK 5]
    \definition[条]{s.}{canal; vala; sarjeta; trincheira; cursos d'água ou fortificações escavados | ranhura; sulco raso; uma depressão que se assemelha a uma vala | ravina; barranco; cursos d'água}
  \end{phonetics}
\end{entry}

\begin{entry}{沟通}{7,10}{⽔、⾡}
  \begin{phonetics}{沟通}{gou1tong1}[][HSK 5]
    \definition{v.}{comunicar; comunicar-se para entender as ideias, opiniões, etc. | conectar; ligar; estabelecer um paralelo entre os dois}
  \end{phonetics}
\end{entry}

\begin{entry}{没}{7}{⽔}
  \begin{phonetics}{没}{mei2}[][HSK 1]
    \definition{adv.}{não; nunca; negar que uma ação ou situação tenha ocorrido, com o significado de 不曾}
    \definition{pref.}{não (prefixo negativo para verbos, traduzido para outras línguas com verbos no pretérito)}
    \definition{v.}{não possuir; não ter | não existe; não há | ninguém; usado antes de 谁, 什么, 哪个, significa 全都不 | não ser tão bom quanto; ser inferior a; não chega a; não é tão bom quanto | menor que; insuficiente}
  \seealsoref{不曾}{bu4 ceng2}
  \seealsoref{哪个}{na3ge5}
  \seealsoref{全都不}{quan2dou1 bu4}
  \seealsoref{谁}{shei2}
  \seealsoref{什么}{shen2me5}
  \end{phonetics}
  \begin{phonetics}{没}{mo4}
    \definition{adj.}{último; final}
    \definition{v.}{afundar na água; submergir | transbordar; subir além; exceder ou ultrapassar | esconder-se; desaparecer; sumir; ocultar-se | confiscar; expropriar | morrer}
    \variantof{没}
  \end{phonetics}
\end{entry}

\begin{entry}{没了}{7,2}{⽔、⼅}
  \begin{phonetics}{没了}{mei2le5}
    \definition{v.}{estar morto | deixar de existir}
  \end{phonetics}
\end{entry}

\begin{entry}{没什么}{7,4,3}{⽔、⼈、⼃}
  \begin{phonetics}{没什么}{mei2 shen2 me5}[][HSK 1]
    \definition{expr.}{não é nada; está tudo bem; não importa}
  \end{phonetics}
\end{entry}

\begin{entry}{没用}{7,5}{⽔、⽤}
  \begin{phonetics}{没用}{mei2 yong4}[][HSK 3]
    \definition{adj.}{inútil; imprestável; sem valor; sem préstimo; vão; que não serve para nada}
  \end{phonetics}
\end{entry}

\begin{entry}{没关系}{7,6,7}{⽔、⼋、⽷}
  \begin{phonetics}{没关系}{mei2guan1xi5}[][HSK 1]
    \definition{v.}{está tudo bem; não é nada; não importa; não se preocupe}
  \seealsoref{没有关系}{mei2you3guan1xi5}
  \end{phonetics}
\end{entry}

\begin{entry}{没有}{7,6}{⽔、⽉}
  \begin{phonetics}{没有}{mei2 you3}[][HSK 1]
    \definition{adv.}{ainda não; (usado com o pretérito) não; ação ou estado negativo ocorreu}
    \definition{v.}{não há; não tem; não existe}
  \end{phonetics}
\end{entry}

\begin{entry}{没有关系}{7,6,6,7}{⽔、⽉、⼋、⽷}
  \begin{phonetics}{没有关系}{mei2you3guan1xi5}
    \definition{v.}{não ter problema | não ter importância | não fazer mal}
  \seealsoref{没关系}{mei2guan1xi5}
  \end{phonetics}
\end{entry}

\begin{entry}{没有次序}{7,6,6,7}{⽔、⽉、⽋、⼴}
  \begin{phonetics}{没有次序}{mei2you3 ci4xu4}
    \definition{adj.}{sem ordem; nenhuma ordem}
  \end{phonetics}
\end{entry}

\begin{entry}{没有意思}{7,6,13,9}{⽔、⽉、⼼、⼼}
  \begin{phonetics}{没有意思}{mei2you3yi4si5}
    \definition{adj.}{tedioso | chato | sem interesse}
  \end{phonetics}
\end{entry}

\begin{entry}{没事儿}{7,8,2}{⽔、⼅、⼉}
  \begin{phonetics}{没事儿}{mei2 shi4r5}[][HSK 1]
    \definition{expr.}{fora de perigo; nada sério | não importa; não é nada; está tudo bem; não importa | está tudo bem; sem problemas; não se preocupe com isso; não é grande coisa; não há nada errado}
    \definition{v.}{não ter nada para fazer; ser livre; estar perdido | estar desempregado; estar sem trabalho | não ter responsabilidade}
  \end{phonetics}
\end{entry}

\begin{entry}{没法儿}{7,8,2}{⽔、⽔、⼉}
  \begin{phonetics}{没法儿}{mei2 fa3r5}[][HSK 4]
    \definition{adv.}{não pode; sem chance}
  \end{phonetics}
\end{entry}

\begin{entry}{没想到}{7,13,8}{⽔、⼼、⼑}
  \begin{phonetics}{没想到}{mei2 xiang3 dao4}[][HSK 4]
    \definition{expr.}{não esperava; inesperado}
  \end{phonetics}
\end{entry}

\begin{entry}{没错}{7,13}{⽔、⾦}
  \begin{phonetics}{没错}{mei2 cuo4}[][HSK 4]
    \definition{adv.}{está certo; é isso mesmo; não há como errar}
  \end{phonetics}
\end{entry}

\begin{entry}{灵感}{7,13}{⽕、⼼}
  \begin{phonetics}{灵感}{ling2gan3}
    \definition{s.}{inspiração | explosão de criatividade em empreendimento científico ou artístico}
  \end{phonetics}
\end{entry}

\begin{entry}{灵魂}{7,13}{⽕、⿁}
  \begin{phonetics}{灵魂}{ling2hun2}
    \definition{s.}{alma | espírito}
  \end{phonetics}
\end{entry}

\begin{entry}{灶台}{7,5}{⽕、⼝}
  \begin{phonetics}{灶台}{zao4tai2}
    \definition{s.}{fogão}
  \end{phonetics}
\end{entry}

\begin{entry}{灾}{7}{⽕}
  \begin{phonetics}{灾}{zai1}[][HSK 5]
    \definition[个,场]{s.}{calamidade; desastre | infortúnio pessoal; adversidade | azar}
  \end{phonetics}
\end{entry}

\begin{entry}{灾区}{7,4}{⽕、⼖}
  \begin{phonetics}{灾区}{zai1 qu1}[][HSK 5]
    \definition{s.}{área de desastre; área afetada por catástrofes}
  \end{phonetics}
\end{entry}

\begin{entry}{灾害}{7,10}{⽕、⼧}
  \begin{phonetics}{灾害}{zai1hai4}[][HSK 5]
    \definition[个]{s.}{desastre; calamidade; danos causados pela seca, inundações, pragas, granizo, guerras, etc.}
  \end{phonetics}
\end{entry}

\begin{entry}{灾难}{7,10}{⽕、⾫}
  \begin{phonetics}{灾难}{zai1nan4}[][HSK 5]
    \definition[场,次]{s.}{desastre; sofrimento; calamidade; catástrofe; danos e sofrimentos causados por desastres naturais ou guerras}
  \end{phonetics}
\end{entry}

\begin{entry}{状况}{7,7}{⽝、⼎}
  \begin{phonetics}{状况}{zhuang4kuang4}[][HSK 3]
    \definition[个,种]{s.}{estado; \emph{status}; condição; estado de coisas}
  \end{phonetics}
\end{entry}

\begin{entry}{状态}{7,8}{⽝、⼼}
  \begin{phonetics}{状态}{zhuang4tai4}[][HSK 3]
    \definition[种,个]{s.}{\emph{status}; estado; condição; estado de coisas; a forma em que uma pessoa ou coisa aparece}
  \end{phonetics}
\end{entry}

\begin{entry}{犹豫}{7,15}{⽝、⾗}
  \begin{phonetics}{犹豫}{you2yu4}[][HSK 5]
    \definition{adj.}{hesitante; indeciso, incapaz de decidir ou agir}
    \definition{v.}{hesitar; ser indeciso}
  \end{phonetics}
\end{entry}

\begin{entry}{狂}{7}{⽝}
  \begin{phonetics}{狂}{kuang2}[][HSK 5]
    \definition*{s.}{sobrenome Kuang}
    \definition{adj.}{louco; maluco | violento; selvagem | selvagem; delirante; furioso; desenfreado; desinibido; sem restrições | arrogante; autoritário}
  \end{phonetics}
\end{entry}

\begin{entry}{狂欢节}{7,6,5}{⽝、⽋、⾋}
  \begin{phonetics}{狂欢节}{kuang2huan1jie2}
    \definition*{s.}{Carnaval}
  \end{phonetics}
\end{entry}

\begin{entry}{男}{7}{⽥}
  \begin{phonetics}{男}{nan2}[][HSK 1]
    \definition{adj.}{homem; macho; masculino (em oposição a 女)}
    \definition[个,位]{s.}{filho; menino | homem | barão (o mais baixo de cinco ordens de nobreza)}
  \seealsoref{女}{nv3}
  \end{phonetics}
\end{entry}

\begin{entry}{男人}{7,2}{⽥、⼈}
  \begin{phonetics}{男人}{nan2 ren2}[][HSK 1]
    \definition[个]{s.}{homem adulto; macho; cavalheiro | marido}
  \end{phonetics}
\end{entry}

\begin{entry}{男士}{7,3}{⽥、⼠}
  \begin{phonetics}{男士}{nan2 shi4}[][HSK 4]
    \definition{s.}{cavalheiro; \emph{gentleman}}
  \end{phonetics}
\end{entry}

\begin{entry}{男女}{7,3}{⽥、⼥}
  \begin{phonetics}{男女}{nan2 nv3}[][HSK 4]
    \definition{s.}{homens e mulheres; masculino e feminino}
  \end{phonetics}
\end{entry}

\begin{entry}{男子}{7,3}{⽥、⼦}
  \begin{phonetics}{男子}{nan2zi3}[][HSK 3]
    \definition[名]{s.}{homem; macho}
  \end{phonetics}
\end{entry}

\begin{entry}{男生}{7,5}{⽥、⽣}
  \begin{phonetics}{男生}{nan2 sheng1}[][HSK 1]
    \definition[个]{s.}{menino; estudante; estudante do sexo masculino; aluno do sexo masculino}
  \end{phonetics}
\end{entry}

\begin{entry}{男性}{7,8}{⽥、⼼}
  \begin{phonetics}{男性}{nan2 xing4}[][HSK 5]
    \definition{s.}{masculino; homem; masculinidade}
  \end{phonetics}
\end{entry}

\begin{entry}{男朋友}{7,8,4}{⽥、⽉、⼜}
  \begin{phonetics}{男朋友}{nan2 peng2 you5}[][HSK 1]
    \definition{s.}{namorado}
  \end{phonetics}
\end{entry}

\begin{entry}{男孩儿}{7,9,2}{⽥、⼦、⼉}
  \begin{phonetics}{男孩儿}{nan2hai2r5}[][HSK 1]
    \definition{s.}{menino; rapaz}
  \end{phonetics}
\end{entry}

\begin{entry}{疗养}{7,9}{⽧、⼋}
  \begin{phonetics}{疗养}{liao2 yang3}[][HSK 4]
    \definition{v.}{recuperar; convalescer; tratar pessoas com doenças crônicas ou debilitantes em instituições médicas especializadas com foco na recuperação}
  \end{phonetics}
\end{entry}

\begin{entry}{社}{7}{⽰}
  \begin{phonetics}{社}{she4}[][HSK 5]
    \definition[个]{s.}{agência; sociedade; órgão organizado; organização; comunidade | comuna popular | o deus da terra, sacrifícios a ele ou altares para tais sacrifícios; na antiguidade, o deus da terra, o local onde ele era venerado, o dia da veneração e o ritual eram chamados de 社 | agência de notícias |  imprensa}
  \end{phonetics}
\end{entry}

\begin{entry}{社区}{7,4}{⽰、⼖}
  \begin{phonetics}{社区}{she4qu1}[][HSK 5]
    \definition{s.}{bairro; comunidade residencial; bairros da cidade, divididos de acordo com a localização geográfica | distrito; comunidade (para pessoas da mesma classe social, etc.) ; lugar onde pessoas com características comuns, como classe social, vivem juntas}
  \end{phonetics}
\end{entry}

\begin{entry}{社会}{7,6}{⽰、⼈}
  \begin{phonetics}{社会}{she4hui4}[][HSK 3]
    \definition[个,种]{s.}{sociedade | comunidade}
  \end{phonetics}
\end{entry}

\begin{entry}{私人}{7,2}{⽲、⼈}
  \begin{phonetics}{私人}{si1ren2}[][HSK 5]
    \definition{adj.}{privado; pertencente a um indivíduo ou exercido a título individual; não público | interpessoal}
    \definition[个]{s.}{algo privado; pessoas que se aproximam de você por motivos pessoais ou interesses próprios}
  \end{phonetics}
\end{entry}

\begin{entry}{私人诊所}{7,2,7,8}{⽲、⼈、⾔、⼾}
  \begin{phonetics}{私人诊所}{si1ren2 zhen3suo3}
    \definition[些]{s.}{clínica privada}
  \end{phonetics}
\end{entry}

\begin{entry}{私人信件}{7,2,9,6}{⽲、⼈、⼈、⼈}
  \begin{phonetics}{私人信件}{si1ren2 xin4jian4}
    \definition{s.}{carta pessoal}
  \end{phonetics}
\end{entry}

\begin{entry}{私人钥匙}{7,2,9,11}{⽲、⼈、⾦、⼔}
  \begin{phonetics}{私人钥匙}{si1ren2yao4shi5}
    \definition{s.}{(criptografia) chave privada}
  \end{phonetics}
\end{entry}

\begin{entry}{私生活}{7,5,9}{⽲、⽣、⽔}
  \begin{phonetics}{私生活}{si1sheng1huo2}
    \definition{s.}{vida privada}
  \end{phonetics}
\end{entry}

\begin{entry}{私自}{7,6}{⽲、⾃}
  \begin{phonetics}{私自}{si1zi4}
    \definition{adj.}{privado | pessoal}
    \definition{adv.}{secretamente | sem aprovação explícita}
  \end{phonetics}
\end{entry}

\begin{entry}{究竟}{7,11}{⽳、⾳}
  \begin{phonetics}{究竟}{jiu1jing4}[][HSK 4]
    \definition{adv.}{de fato; exatamente; usado em frases interrogativas para buscar | afinal de contas, no final; ênfase em fatos ou motivos}
    \definition{s.}{resultado; desfecho; a causa, o efeito ou a história completa do que aconteceu}
  \end{phonetics}
\end{entry}

\begin{entry}{穷}{7}{⽳}
  \begin{phonetics}{穷}{qiong2}[][HSK 4]
    \definition{adj.}{remoto; isolado; de difícil acesso | pobre; atingido pela pobreza | situação difícil, sem saída}
    \definition{adv.}{completamente | extremamente}
    \definition{v.}{exaurir; esgotar; consmir | ir até o fim; perseguir completamente perseguido; sondar profundamente | gastar}
  \end{phonetics}
\end{entry}

\begin{entry}{穷人}{7,2}{⽳、⼈}
  \begin{phonetics}{穷人}{qiong2 ren2}[][HSK 4]
    \definition{s.}{os pobres; pessoas pobres}
  \end{phonetics}
\end{entry}

\begin{entry}{系}{7}{⽷}
  \begin{phonetics}{系}{ji4}
    \definition{v.}{amarrar; prender; abotoar}
  \end{phonetics}
  \begin{phonetics}{系}{xi4}[][HSK 3,4]
    \definition*{s.}{sobrenome Xi}
    \definition{s.}{faculdade (da universidade) | departamento}
    \definition{v.}{sistema; série | departamento; faculdade}
    \definition{v.}{relacionar-se com; suportar; depender de | sentir-se ansioso; estar preocupado | amarrar; prender | ser}
  \end{phonetics}
\end{entry}

\begin{entry}{系囚}{7,5}{⽷、⼞}
  \begin{phonetics}{系囚}{xi4qiu2}
    \definition{s.}{prisioneiro}
  \end{phonetics}
\end{entry}

\begin{entry}{系列}{7,6}{⽷、⼑}
  \begin{phonetics}{系列}{xi4lie4}[][HSK 4]
    \definition{s.}{série; conjunto; conjunto de coisas relacionadas (matemática)}
  \end{phonetics}
\end{entry}

\begin{entry}{系统}{7,9}{⽷、⽷}
  \begin{phonetics}{系统}{xi4tong3}[][HSK 4]
    \definition{adj.}{sistemático; organizado}
    \definition[个]{s.}{sistema; relação de tipos semelhantes (ou seja, grupo de coisas semelhantes)}
  \end{phonetics}
\end{entry}

\begin{entry}{纯}{7}{⽷}
  \begin{phonetics}{纯}{chun2}[][HSK 4]
    \definition{adj.}{puro; não misturado; livre de impurezas | simples; puro e simples | habilidoso; proficiente; bem versado}
  \end{phonetics}
\end{entry}

\begin{entry}{纯净水}{7,8,4}{⽷、⼎、⽔}
  \begin{phonetics}{纯净水}{chun2 jing4 shui3}[][HSK 4]
    \definition{s.}{água purificada}
  \end{phonetics}
\end{entry}

\begin{entry}{纯真}{7,10}{⽷、⼗}
  \begin{phonetics}{纯真}{chun2zhen1}
    \definition{adj.}{inocente e não afetado | puro e não adulterado}
    \definition{s.}{inocência}
  \end{phonetics}
\end{entry}

\begin{entry}{纷纷}{7,7}{⽷、⽷}
  \begin{phonetics}{纷纷}{fen1fen1}[][HSK 4]
    \definition{adj.}{numeroso e confuso; muitos e desordenados}
    \definition{adv.}{um após o outro; em sucessão; em rápida sucessão}
  \end{phonetics}
\end{entry}

\begin{entry}{纸}{7}{⽷}
  \begin{phonetics}{纸}{zhi3}[][HSK 2]
    \definition{clas.}{para documentos, cartas, etc.}
    \definition[张,沓]{s.}{papel}
  \end{phonetics}
\end{entry}

\begin{entry}{纸巾}{7,3}{⽷、⼱}
  \begin{phonetics}{纸巾}{zhi3jin1}
    \definition[张,包]{s.}{lenço | guardanapo | papel toalha}
  \end{phonetics}
\end{entry}

\begin{entry}{纸币}{7,4}{⽷、⼱}
  \begin{phonetics}{纸币}{zhi3bi4}
    \definition[张]{s.}{nota (dinheiro) | cédula}
  \end{phonetics}
\end{entry}

\begin{entry}{纸尿裤}{7,7,12}{⽷、⼫、⾐}
  \begin{phonetics}{纸尿裤}{zhi3niao4ku4}
    \definition{s.}{fralda descartável}
  \end{phonetics}
\end{entry}

\begin{entry}{纸张}{7,7}{⽷、⼸}
  \begin{phonetics}{纸张}{zhi3zhang1}
    \definition{s.}{papel}
  \end{phonetics}
\end{entry}

\begin{entry}{纸烟}{7,10}{⽷、⽕}
  \begin{phonetics}{纸烟}{zhi3yan1}
    \definition{s.}{cigarro}
  \end{phonetics}
\end{entry}

\begin{entry}{纹路}{7,13}{⽷、⾜}
  \begin{phonetics}{纹路}{wen2lu4}
    \definition{s.}{padrão de linhas | rugas | veias | veias (em mármore ou impressão digital) | grãos (em madeira, etc.)}
  \end{phonetics}
\end{entry}

\begin{entry}{肚}{7}{⾁}
  \begin{phonetics}{肚}{du3}
    \definition{s.}{tripas | entranhas}
  \end{phonetics}
  \begin{phonetics}{肚}{du4}
    \definition{s.}{barriga}
  \end{phonetics}
\end{entry}

\begin{entry}{肚子}{7,3}{⾁、⼦}
  \begin{phonetics}{肚子}{du4zi5}[][HSK 4]
    \definition[个,只]{s.}{abdômen; barriguinha; ventre; barriga}
  \end{phonetics}
\end{entry}

\begin{entry}{肠}{7}{⾁}
  \begin{phonetics}{肠}{chang2}[][HSK 5]
    \definition{s.}{intestinos | salsicha; linguiça | coração; sentimentos; emoções}
  \end{phonetics}
\end{entry}

\begin{entry}{良心}{7,4}{⾉、⼼}
  \begin{phonetics}{良心}{liang2xin1}
    \definition{s.}{consciência}
  \end{phonetics}
\end{entry}

\begin{entry}{良田}{7,5}{⾉、⽥}
  \begin{phonetics}{良田}{liang2tian2}
    \definition{s.}{terra agrícola boa | terra fértil}
  \end{phonetics}
\end{entry}

\begin{entry}{良好}{7,6}{⾉、⼥}
  \begin{phonetics}{良好}{liang2hao3}[][HSK 4]
    \definition{adj.}{bom; ótimo; bem}
  \end{phonetics}
\end{entry}

\begin{entry}{芥}{7}{⾋}
  \begin{phonetics}{芥}{gai4}
    \definition{s.}{usado em 芥蓝 \dpy{gai4lan2}}
  \seealsoref{芥蓝}{gai4lan2}
  \end{phonetics}
  \begin{phonetics}{芥}{jie4}
    \definition{s.}{mostarda}
  \end{phonetics}
\end{entry}

\begin{entry}{芥兰}{7,5}{⾋、⼋}
  \begin{phonetics}{芥兰}{gai4lan2}
    \variantof{芥蓝}
  \end{phonetics}
  \begin{phonetics}{芥兰}{jie4lan2}
    \definition{s.}{couve}
  \end{phonetics}
\end{entry}

\begin{entry}{芥蓝}{7,13}{⾋、⾋}
  \begin{phonetics}{芥蓝}{gai4lan2}
    \definition{s.}{brócolis chinês | couve chinesa | mostarda}
  \seealsoref{格兰菜}{ge2lan2cai4}
  \end{phonetics}
\end{entry}

\begin{entry}{芦笋}{7,10}{⾋、⽵}
  \begin{phonetics}{芦笋}{lu2sun3}
    \definition{s.}{aspargos}
  \end{phonetics}
\end{entry}

\begin{entry}{芯片}{7,4}{⾋、⽚}
  \begin{phonetics}{芯片}{xin1pian4}
    \definition{s.}{chip de computador | microchip}
  \end{phonetics}
\end{entry}

\begin{entry}{花}{7}{⾋}
  \begin{phonetics}{花}{hua1}[][HSK 1,2,4]
    \definition*{s.}{sobrenome Hua}
    \definition{adj.}{multicolorido; colorido | embaçado; obscuro; deslumbrado e confuso | extravagante; florido; vistoso}
    \definition[朵,支,束,把,盆,簇]{s.}{flor; órgãos de reprodução sexual de plantas com sementes | flor; planta ornamental |  qualquer coisa que se assemelhe a uma flor | fogos de artifício | padrão; design; design decorativo | flor; metáfora para a essência de uma causa | prostituta; cortesã; referindo-se a prostitutas ou a assuntos relacionados a prostitutas | algodão | varíola | ferimento; ferida; lesões traumáticas sofridas em combate}
    \definition{v.}{gastar; despender; consumir}
  \end{phonetics}
\end{entry}

\begin{entry}{花儿}{7,2}{⾋、⼉}
  \begin{phonetics}{花儿}{hua1r5}
    \definition[朵,支,束,把,盆,簇]{s.}{flor}
  \end{phonetics}
\end{entry}

\begin{entry}{花生}{7,5}{⾋、⽣}
  \begin{phonetics}{花生}{hua1sheng1}
    \definition[粒]{s.}{amendoim}
  \end{phonetics}
\end{entry}

\begin{entry}{花园}{7,7}{⾋、⼞}
  \begin{phonetics}{花园}{hua1 yuan2}[][HSK 2]
    \definition[个,座]{s.}{jardim; um local onde se plantam flores e árvores para passear e descansar}
  \end{phonetics}
\end{entry}

\begin{entry}{花店}{7,8}{⾋、⼴}
  \begin{phonetics}{花店}{hua1dian4}
    \definition{s.}{floricultura}
  \end{phonetics}
\end{entry}

\begin{entry}{花茶}{7,9}{⾋、⾋}
  \begin{phonetics}{花茶}{hua1cha2}
    \definition[杯,壶]{s.}{chá perfumado}
  \end{phonetics}
\end{entry}

\begin{entry}{花样游泳}{7,10,12,8}{⾋、⽊、⽔、⽔}
  \begin{phonetics}{花样游泳}{hua1yang4you2yong3}
    \definition{s.}{nado sincronizado}
  \end{phonetics}
\end{entry}

\begin{entry}{花椰菜}{7,12,11}{⾋、⽊、⾋}
  \begin{phonetics}{花椰菜}{hua1ye1cai4}
    \definition{s.}{couve-flor}
  \end{phonetics}
\end{entry}

\begin{entry}{芹菜}{7,11}{⾋、⾋}
  \begin{phonetics}{芹菜}{qin2cai4}
    \definition{s.}{salsão}
  \end{phonetics}
\end{entry}

\begin{entry}{苏格兰}{7,10,5}{⾋、⽊、⼋}
  \begin{phonetics}{苏格兰}{su1ge2lan2}
    \definition*{s.}{Escócia}
  \end{phonetics}
\end{entry}

\begin{entry}{补}{7}{⾐}
  \begin{phonetics}{补}{bu3}[][HSK 3]
    \definition*{s.}{sobrenome Bu}
    \definition{s.}{benefício | ajuda | uso}
    \definition{v.}{consertar | remendar | preencher | adicionar suplemento | suprir | compensar |nutrir}
  \end{phonetics}
\end{entry}

\begin{entry}{补充}{7,6}{⾐、⼉}
  \begin{phonetics}{补充}{bu3chong1}[][HSK 3]
    \definition{adj.}{adicional | suplementar}
    \definition[个]{s.}{aditivo | suplemento}
    \definition{v.}{reabastecer | suplementar | complementar}
  \end{phonetics}
\end{entry}

\begin{entry}{补贴}{7,9}{⾐、⾙}
  \begin{phonetics}{补贴}{bu3tie1}[][HSK 5]
    \definition[笔,项,种,份]{s.}{subsídio; ajuda de custo; custos de indenização ou assistência concedida a empresas ou indivíduos pelo estado ou governo}
    \definition{v.}{subsidiar; compensar a falta de dinheiro ou coisas; refere-se principalmente à compensação financeira ou ajuda dada pelo estado ou governo a empresas ou indivíduos}
  \end{phonetics}
\end{entry}

\begin{entry}{补偿}{7,11}{⾐、⼈}
  \begin{phonetics}{补偿}{bu3chang2}[][HSK 5]
    \definition{v.}{compensar (perda, consumo); compensar (deficiências, diferenças)}
  \end{phonetics}
\end{entry}

\begin{entry}{角}{7}{⾓}
  \begin{phonetics}{角}{jiao3}[][HSK 2]
    \definition*{s.}{Jiao, uma das mansões lunares}
    \definition{clas.}{uma unidade monetária fracionária na China (=1/10 de um yuan ou 10 fen)}
    \definition[个,只,对]{s.}{chifre; o objeto duro que cresce na cabeça de bovinos, ovinos, veados, etc. | buzina; corneta; instrumentos musicais tocados no exército antigo | algo com a forma de um chifre | cabo; promontório; península | esquina; canto; a junção entre duas arestas de um objeto | ângulo}
  \end{phonetics}
  \begin{phonetics}{角}{jue2}
    \definition*{s.}{sobrenome Jue}
    \definition{s.}{papel (teatro)}
    \definition{v.}{competir}
  \end{phonetics}
\end{entry}

\begin{entry}{角色}{7,6}{⾓、⾊}
  \begin{phonetics}{角色}{jue2se4}[][HSK 4]
    \definition{s.}{papel; personagem em uma peça; personagem representado por um ator | papel; função; parte}
  \end{phonetics}
\end{entry}

\begin{entry}{角度}{7,9}{⾓、⼴}
  \begin{phonetics}{角度}{jiao3du4}[][HSK 2]
    \definition[个,种]{s.}{perspectiva; ponto de vista; o ponto de partida para ver as coisas | ângulo; o tamanho do ângulo; normalmente expresso em graus ou radianos}
  \end{phonetics}
\end{entry}

\begin{entry}{言论}{7,6}{⾔、⾔}
  \begin{phonetics}{言论}{yan2lun4}
    \definition{s.}{expressão de opinião |  visualizações | comentários | argumentos}
  \end{phonetics}
\end{entry}

\begin{entry}{言语}{7,9}{⾔、⾔}
  \begin{phonetics}{言语}{yan2 yu3}[][HSK 5]
    \definition{s.}{verbal; fala; linguagem falada; conversa; palavras}
  \end{phonetics}
\end{entry}

\begin{entry}{证}{7}{⾔}
  \begin{phonetics}{证}{zheng4}[][HSK 3]
    \definition{s.}{evidência; prova; testemunho; testemunha | certificado; cartão | doença; enfermidade}
    \definition{v.}{provar; verificar; demonstrar}
  \end{phonetics}
\end{entry}

\begin{entry}{证书}{7,4}{⾔、⼄}
  \begin{phonetics}{证书}{zheng4shu1}[][HSK 5]
    \definition[张,份,些]{s.}{certificado; documentos emitidos por instituições, grupos, etc., que comprovem experiência, nível, honras, poderes, etc.}
  \end{phonetics}
\end{entry}

\begin{entry}{证件}{7,6}{⾔、⼈}
  \begin{phonetics}{证件}{zheng4jian4}[][HSK 3]
    \definition[个,本,张]{s.}{documentos; credenciais; certificado}
  \end{phonetics}
\end{entry}

\begin{entry}{证实}{7,8}{⾔、⼧}
  \begin{phonetics}{证实}{zheng4shi2}[][HSK 5]
    \definition{v.}{verificar; afirmar; confirmar; corroborar; demonstrar; autenticar; provar que é verdadeiro}
  \end{phonetics}
\end{entry}

\begin{entry}{证明}{7,8}{⾔、⽇}
  \begin{phonetics}{证明}{zheng4ming2}[][HSK 3]
    \definition[个,份]{s.}{certificado; testemunho; identificação; certificado ou carta de certificação; documentos que comprovem identidade, experiência, etc., como carteira de estudante, carteira de trabalho, certificado de graduação, etc.}
    \definition{v.}{provar; testemunhar; sustentar; usar materiais confiáveis ​​para mostrar ou determinar a autenticidade de uma pessoa ou coisa}
  \end{phonetics}
\end{entry}

\begin{entry}{证据}{7,11}{⾔、⼿}
  \begin{phonetics}{证据}{zheng4ju4}[][HSK 3]
    \definition{s.}{prova; evidência; testemunho; fatos ou materiais relevantes que podem provar a autenticidade de algo}
  \end{phonetics}
\end{entry}

\begin{entry}{评价}{7,6}{⾔、⼈}
  \begin{phonetics}{评价}{ping2jia4}[][HSK 3]
    \definition[个,项,条,份]{s.}{avaliação; apreciação}
    \definition{v.}{estimar; avaliar}
  \end{phonetics}
\end{entry}

\begin{entry}{评论}{7,6}{⾔、⾔}
  \begin{phonetics}{评论}{ping2lun4}[][HSK 5]
    \definition[篇]{s.}{revisão; comentário; artigos ou comentários críticos}
    \definition{v.}{discutir; comentar sobre algo ou alguém}
  \end{phonetics}
\end{entry}

\begin{entry}{评估}{7,7}{⾔、⼈}
  \begin{phonetics}{评估}{ping2gu1}[][HSK 5]
    \definition{v.}{estimar; avaliar; apreciar; avaliar e estimar (coisas abstratas)}
  \end{phonetics}
\end{entry}

\begin{entry}{诅咒}{7,8}{⾔、⼝}
  \begin{phonetics}{诅咒}{zu3zhou4}
    \definition{v.}{amaldiçoar}
  \end{phonetics}
\end{entry}

\begin{entry}{诊断}{7,11}{⾔、⽄}
  \begin{phonetics}{诊断}{zhen3duan4}[][HSK 5]
    \definition{s.}{diagnóstico; diacrisis}
    \definition{v.}{diagnosticar; após examinar os sintomas do paciente, determinar a doença e seu desenvolvimento}
  \end{phonetics}
\end{entry}

\begin{entry}{词}{7}{⾔}
  \begin{phonetics}{词}{ci2}[][HSK 2]
    \definition[个,组,句,段,首]{s.}{palavra; termo; antigamente, referia-se a palavras vazias; atualmente, refere-se a palavras com forma fonética fixa e significado específico na língua; a menor unidade que pode ser usada de forma independente | discurso; declaração; linguagem; texto | ci (um tipo de poesia clássica chinesa, originária da dinastia Tang e plenamente desenvolvida na dinastia Song); gênero poético escrito de acordo com uma estrutura fixa, com versos de comprimentos variados | palavras; redação; refere-se genericamente ao teatro; a parte da letra cantada em harmonia com a melodia em canções e certas artes vocais}
  \end{phonetics}
\end{entry}

\begin{entry}{词汇}{7,5}{⾔、⽔}
  \begin{phonetics}{词汇}{ci2hui4}[][HSK 4]
    \definition[个,组,批,串,堆]{s.}{vocabulário; termo geral para palavras usadas em um idioma}
  \end{phonetics}
\end{entry}

\begin{entry}{词典}{7,8}{⾔、⼋}
  \begin{phonetics}{词典}{ci2dian3}[][HSK 2]
    \definition[本,部]{s.}{dicionário, livro de referência que reúne palavras e explicações para consulta}
  \seealsoref{字典}{zi4 dian3}
  \end{phonetics}
\end{entry}

\begin{entry}{词语}{7,9}{⾔、⾔}
  \begin{phonetics}{词语}{ci2yu3}[][HSK 2]
    \definition[个,租]{s.}{termo; palavra; expressão; conjunto de palavras e frases}
  \end{phonetics}
\end{entry}

\begin{entry}{谷}{7}{⾕}[Kangxi 150]
  \begin{phonetics}{谷}{gu3}
    \definition{adj.}{bom; gentil;}
    \definition{s.}{vale; ravina; desfiladeiro; garganta; faixa estreita de terra com uma saída no meio de duas colinas ou dois platôs | arroz não descascado | salário de funcionário (na época feudal) |calha; cocho; canal | fossa sob o cerebelo (anatomia); valécula | dificuldade; dilema}
    \definition{v.}{criar (filhos) | crescer}
  \end{phonetics}
\end{entry}

\begin{entry}{豆角}{7,7}{⾖、⾓}
  \begin{phonetics}{豆角}{dou4jiao3}
    \definition{s.}{feijão verde}
  \end{phonetics}
\end{entry}

\begin{entry}{豆制品}{7,8,9}{⾖、⼑、⼝}
  \begin{phonetics}{豆制品}{dou4 zhi4 pin3}[][HSK 5]
    \definition{s.}{produtos de soja}
  \end{phonetics}
\end{entry}

\begin{entry}{豆荚}{7,9}{⾖、⾋}
  \begin{phonetics}{豆荚}{dou4jia2}
    \definition{s.}{vagem (de legumes)}
  \end{phonetics}
\end{entry}

\begin{entry}{豆腐}{7,14}{⾖、⾁}
  \begin{phonetics}{豆腐}{dou4fu5}[][HSK 4]
    \definition[块,盒,斤,盘,锅]{s.}{\emph{tofu}}
  \end{phonetics}
\end{entry}

\begin{entry}{财产}{7,6}{⾙、⼇}
  \begin{phonetics}{财产}{cai2chan3}[][HSK 4]
    \definition{s.}{ativos; propriedade; pertences; refere-se à posse de riqueza material, como dinheiro, bens, casas, terras, etc.}
  \end{phonetics}
\end{entry}

\begin{entry}{财富}{7,12}{⾙、⼧}
  \begin{phonetics}{财富}{cai2fu4}[][HSK 4]
    \definition{s.}{riqueza; fortuna}
  \end{phonetics}
\end{entry}

\begin{entry}{赤}{7}{⾚}[Kangxi 155]
  \begin{phonetics}{赤}{chi4}
    \definition*{s.}{sobrenome Chi}
    \definition{adj.}{vermelho; de cor vermelha | leal; sincero; de coração único | nu; sem roupa}
  \end{phonetics}
\end{entry}

\begin{entry}{走}{7}{⾛}[Kangxi 156]
  \begin{phonetics}{走}{zou3}[][HSK 1]
    \definition{v.}{andar; caminhar | correr | mover; movimentar; deslocar | sair; partir; ir embora | visitar; fazer uma visita; (entre amigos e familiares) troca de visitas | passar por; atravessar; ultrapassar | vazar; revelar; divulgar | afastar-se do original; alterar ou perder a forma, o sabor, a cor, etc. originais}
  \end{phonetics}
\end{entry}

\begin{entry}{走开}{7,4}{⾛、⼶}
  \begin{phonetics}{走开}{zou3 kai1}[][HSK 2]
    \definition{v.}{ir embora | fugir | ir para outro lugar}
  \end{phonetics}
\end{entry}

\begin{entry}{走去}{7,5}{⾛、⼛}
  \begin{phonetics}{走去}{zou3qu4}
    \definition{v.}{caminhar até (para)}
  \end{phonetics}
\end{entry}

\begin{entry}{走过}{7,6}{⾛、⾡}
  \begin{phonetics}{走过}{zou3 guo4}[][HSK 2]
    \definition{v.}{passar}
  \end{phonetics}
\end{entry}

\begin{entry}{走秀}{7,7}{⾛、⽲}
  \begin{phonetics}{走秀}{zou3xiu4}
    \definition{s.}{desfile de moda}
    \definition{v.}{andar na passarela (em um desfile de moda)}
  \end{phonetics}
\end{entry}

\begin{entry}{走进}{7,7}{⾛、⾡}
  \begin{phonetics}{走进}{zou3 jin4}[][HSK 2]
    \definition{v.}{entrar}
  \end{phonetics}
\end{entry}

\begin{entry}{走势}{7,8}{⾛、⼒}
  \begin{phonetics}{走势}{zou3shi4}
    \definition{s.}{caminho | tendência}
  \end{phonetics}
\end{entry}

\begin{entry}{走卒}{7,8}{⾛、⼗}
  \begin{phonetics}{走卒}{zou3zu2}
    \definition{s.}{lacaio (masculino) | peão (isto é, soldado de infantaria) | servo}
  \end{phonetics}
\end{entry}

\begin{entry}{走鬼}{7,9}{⾛、⿁}
  \begin{phonetics}{走鬼}{zou3gui3}
    \definition{s.}{vendedor ambulante sem licença}
  \end{phonetics}
\end{entry}

\begin{entry}{走索}{7,10}{⾛、⽷}
  \begin{phonetics}{走索}{zou3suo3}
    \definition{v.}{andar na corda bamba}
  \seealsoref{走绳}{zou3sheng2}
  \end{phonetics}
\end{entry}

\begin{entry}{走绳}{7,11}{⾛、⽷}
  \begin{phonetics}{走绳}{zou3sheng2}
    \definition{v.}{andar na corda bamba}
  \seealsoref{走索}{zou3suo3}
  \end{phonetics}
\end{entry}

\begin{entry}{走路}{7,13}{⾛、⾜}
  \begin{phonetics}{走路}{zou3 lu4}[][HSK 1]
    \definition{v.}{caminhar; ir a pé; andar em pé sobre a terra | sair; ir embora; partir}
  \end{phonetics}
\end{entry}

\begin{entry}{足}{7}{⾜}[Kangxi 157]
  \begin{phonetics}{足}{ju4}
    \definition{adj.}{excessivo}
  \end{phonetics}
  \begin{phonetics}{足}{zu2}
    \definition{adj.}{amplo}
    \definition{s.}{pé}
    \definition{v.}{ser suficiente}
  \end{phonetics}
\end{entry}

\begin{entry}{足月}{7,4}{⾜、⽉}
  \begin{phonetics}{足月}{zu2yue4}
    \definition{s.}{gestação completa}
  \end{phonetics}
\end{entry}

\begin{entry}{足足}{7,7}{⾜、⾜}
  \begin{phonetics}{足足}{zu2zu2}
    \definition{adv.}{tanto quanto | extremamente | completamente | não menos que}
  \end{phonetics}
\end{entry}

\begin{entry}{足够}{7,11}{⾜、⼣}
  \begin{phonetics}{足够}{zu2 gou4}[][HSK 3]
    \definition{adj.}{bastante; amplo; suficiente; na medida em que deve ser ou pode atender às necessidades}
    \definition{v.}{satisfazer; ser suficiente; estar a contento}
  \end{phonetics}
\end{entry}

\begin{entry}{足球}{7,11}{⾜、⽟}
  \begin{phonetics}{足球}{zu2qiu2}[][HSK 3]
    \definition[个,只,颗,袋]{s.}{futebol | bola de futebol}
  \end{phonetics}
\end{entry}

\begin{entry}{足球队}{7,11,4}{⾜、⽟、⾩}
  \begin{phonetics}{足球队}{zu2qiu2dui4}
    \definition{s.}{time de futebol}
  \end{phonetics}
\end{entry}

\begin{entry}{足球协会}{7,11,6,6}{⾜、⽟、⼗、⼈}
  \begin{phonetics}{足球协会}{zu2qiu2xie2hui4}
    \definition*{s.}{Associação de Futebol}
  \end{phonetics}
\end{entry}

\begin{entry}{足球场}{7,11,6}{⾜、⽟、⼟}
  \begin{phonetics}{足球场}{zu2qiu2chang3}
    \definition{s.}{campo de futebol}
  \end{phonetics}
\end{entry}

\begin{entry}{足球迷}{7,11,9}{⾜、⽟、⾡}
  \begin{phonetics}{足球迷}{zu2qiu2mi2}
    \definition{s.}{fã de futebol}
  \end{phonetics}
\end{entry}

\begin{entry}{足球赛}{7,11,14}{⾜、⽟、⾙}
  \begin{phonetics}{足球赛}{zu2qiu2sai4}
    \definition{s.}{competição de futebol | partida de futebol}
  \end{phonetics}
\end{entry}

\begin{entry}{身上}{7,3}{⾝、⼀}
  \begin{phonetics}{身上}{shen1 shang5}[][HSK 1]
    \definition{s.}{no corpo de alguém | em um;  com um}
  \end{phonetics}
\end{entry}

\begin{entry}{身亡}{7,3}{⾝、⼇}
  \begin{phonetics}{身亡}{shen1wang2}
    \definition{v.}{morrer}
  \end{phonetics}
\end{entry}

\begin{entry}{身边}{7,5}{⾝、⾡}
  \begin{phonetics}{身边}{shen1 bian1}[][HSK 2]
    \definition{adv.}{ao redor; ao lado de alguém; perto do corpo | carregar consigo (transportar); à mão}
  \end{phonetics}
\end{entry}

\begin{entry}{身份}{7,6}{⾝、⼈}
  \begin{phonetics}{身份}{shen1fen4}[][HSK 4]
    \definition[种]{s.}{status; capacidade; identidade; refere-se à origem, ao status e às qualificações de uma pessoa | dignidade; posição honrada; referência especial ao status respeitável}
  \end{phonetics}
\end{entry}

\begin{entry}{身份证}{7,6,7}{⾝、⼈、⾔}
  \begin{phonetics}{身份证}{shen1 fen4 zheng4}[][HSK 3]
    \definition[张]{s.}{ID; bilhete de identidade; carteira de identidade}
  \end{phonetics}
\end{entry}

\begin{entry}{身体}{7,7}{⾝、⼈}
  \begin{phonetics}{身体}{shen1ti3}[][HSK 1]
    \definition[具,个]{s.}{corpo | saúde; saúde das pessoas}
  \end{phonetics}
\end{entry}

\begin{entry}{身体乳}{7,7,8}{⾝、⼈、⼄}
  \begin{phonetics}{身体乳}{shen1ti3 ru3}
    \definition{s.}{loção corporal}
  \end{phonetics}
\end{entry}

\begin{entry}{身体能力}{7,7,10,2}{⾝、⼈、⾁、⼒}
  \begin{phonetics}{身体能力}{shen1ti3 neng2li4}
    \definition{s.}{habilidade física}
  \end{phonetics}
\end{entry}

\begin{entry}{身材}{7,7}{⾝、⽊}
  \begin{phonetics}{身材}{shen1cai2}[][HSK 4]
    \definition[副,种,个,具]{s.}{figura; estatura; altura e peso corporal}
  \end{phonetics}
\end{entry}

\begin{entry}{身高}{7,10}{⾝、⾼}
  \begin{phonetics}{身高}{shen1 gao1}[][HSK 4]
    \definition[个,种,段]{s.}{estatura; altura (de uma pessoa);}
  \end{phonetics}
\end{entry}

\begin{entry}{辛苦}{7,8}{⾟、⾋}
  \begin{phonetics}{辛苦}{xin1ku3}[][HSK 5]
    \definition{adj.}{difícil; trabalhoso; árduo; descreve muito trabalho, alta intensidade e pouco descanso}
    \definition{s.}{dificuldades}
    \definition{v.}{trabalhar duro; passar por grandes dificuldades; passar por dificuldades}
  \end{phonetics}
\end{entry}

\begin{entry}{迎接}{7,11}{⾡、⼿}
  \begin{phonetics}{迎接}{ying2jie1}[][HSK 3]
    \definition{v.}{conhecer; cumprimentar; dar as boas-vindas}
  \end{phonetics}
\end{entry}

\begin{entry}{运}{7}{⾡}
  \begin{phonetics}{运}{yun4}[][HSK 5]
    \definition*{s.}{sobrenome Yun}
    \definition{s.}{sorte; destino; fortuna}
    \definition{v.}{mover; deslocar | transportar; levar | usar; empunhar; utilizar}
  \end{phonetics}
\end{entry}

\begin{entry}{运气}{7,4}{⾡、⽓}
  \begin{phonetics}{运气}{yun4qi5}[][HSK 4]
    \definition{adj.}{sortudo; afortunado}
    \definition{s.}{sorte; fortuna}
  \end{phonetics}
\end{entry}

\begin{entry}{运用}{7,5}{⾡、⽤}
  \begin{phonetics}{运用}{yun4yong4}[][HSK 4]
    \definition{v.}{usar; utilizar; manejar; aplicar; explorar as coisas de acordo com suas características}
  \end{phonetics}
\end{entry}

\begin{entry}{运动}{7,6}{⾡、⼒}
  \begin{phonetics}{运动}{yun4dong4}[][HSK 2]
    \definition[场]{s.}{esporte | desporto}
    \definition{v.}{exercitar | mover-se}
  \end{phonetics}
\end{entry}

\begin{entry}{运动会}{7,6,6}{⾡、⼒、⼈}
  \begin{phonetics}{运动会}{yun4 dong4 hui4}[][HSK 4]
    \definition[个]{s.}{jogos; encontro esportivo; dia de esportes; reunião atlética}
  \end{phonetics}
\end{entry}

\begin{entry}{运动场}{7,6,6}{⾡、⼒、⼟}
  \begin{phonetics}{运动场}{yun4dong4chang3}
    \definition{s.}{campo desportivo | campo de jogos}
  \end{phonetics}
\end{entry}

\begin{entry}{运动员}{7,6,7}{⾡、⼒、⼝}
  \begin{phonetics}{运动员}{yun4 dong4 yuan2}[][HSK 4]
    \definition[名,个]{s.}{jogador; atleta; esportista; pessoas que participam de competições esportivas}
  \end{phonetics}
\end{entry}

\begin{entry}{运动学}{7,6,8}{⾡、⼒、⼦}
  \begin{phonetics}{运动学}{yun4dong4xue2}
    \definition{s.}{cinemática}
  \end{phonetics}
\end{entry}

\begin{entry}{运动服}{7,6,8}{⾡、⼒、⽉}
  \begin{phonetics}{运动服}{yun4dong4fu2}
    \definition{s.}{roupa para prática de esporte}
  \end{phonetics}
\end{entry}

\begin{entry}{运动衫}{7,6,8}{⾡、⼒、⾐}
  \begin{phonetics}{运动衫}{yun4dong4shan1}
    \definition[件]{s.}{moletom | camisa esportiva}
  \end{phonetics}
\end{entry}

\begin{entry}{运动家}{7,6,10}{⾡、⼒、⼧}
  \begin{phonetics}{运动家}{yun4dong4jia1}
    \definition{s.}{ativista | atleta | esportista}
  \end{phonetics}
\end{entry}

\begin{entry}{运动病}{7,6,10}{⾡、⼒、⽧}
  \begin{phonetics}{运动病}{yun4dong4bing4}
    \definition{s.}{enjôo (movimento, carro, etc.)}
  \end{phonetics}
\end{entry}

\begin{entry}{运动鞋}{7,6,15}{⾡、⼒、⾰}
  \begin{phonetics}{运动鞋}{yun4dong4xie2}
    \definition{s.}{tênis | sapatos esportivos}
  \end{phonetics}
\end{entry}

\begin{entry}{运行}{7,6}{⾡、⾏}
  \begin{phonetics}{运行}{yun4xing2}[][HSK 5]
    \definition{v.}{(corpos celestes, etc.) mover-se ao longo do curso | (figurativo) funcionar, estar em operação | (serviço de trem, etc.) operar | (computador) executar um programa}
  \end{phonetics}
\end{entry}

\begin{entry}{运河}{7,8}{⾡、⽔}
  \begin{phonetics}{运河}{yun4he2}
    \definition{s.}{canal (em um rio)}
  \end{phonetics}
\end{entry}

\begin{entry}{运输}{7,13}{⾡、⾞}
  \begin{phonetics}{运输}{yun4shu1}[][HSK 3]
    \definition{v.}{enviar; transportar; usar um carro, navio, avião, etc. para transportar pessoas ou coisas de um lugar para outro}
  \end{phonetics}
\end{entry}

\begin{entry}{近}{7}{⾡}
  \begin{phonetics}{近}{jin4}[][HSK 2]
    \definition{adj.}{próximo; perto; distância espacial ou temporal curta (oposto de 远) | íntimo; intimamente relacionado; relação estreita | fácil de entender}
  \seealsoref{远}{yuan3}
  \end{phonetics}
\end{entry}

\begin{entry}{近代}{7,5}{⾡、⼈}
  \begin{phonetics}{近代}{jin4dai4}[][HSK 4]
    \definition{s.}{tempos modernos; era passada relativamente próxima à era moderna, geralmente referida na história chinesa como 1840 a 1919 | na história mundial, geralmente se refere à era capitalista}
  \end{phonetics}
\end{entry}

\begin{entry}{近来}{7,7}{⾡、⽊}
  \begin{phonetics}{近来}{jin4lai2}[][HSK 5]
    \definition{adv.}{ultimamente; recentemente; de ​​tarde; refere-se a um período de tempo entre o passado imediato e o presente}
  \end{phonetics}
\end{entry}

\begin{entry}{近期}{7,12}{⾡、⽉}
  \begin{phonetics}{近期}{jin4 qi1}[][HSK 3]
    \definition{adv.}{num futuro próximo; brevemente}
  \end{phonetics}
\end{entry}

\begin{entry}{返回}{7,6}{⾡、⼞}
  \begin{phonetics}{返回}{fan3 hui2}[][HSK 5]
    \definition{v.}{retornar; ir (voltar); reverter; recorrer; retroceder; voltar para (o lugar original)}
  \end{phonetics}
\end{entry}

\begin{entry}{还}{7}{⾡}
  \begin{phonetics}{还}{hai2}[][HSK 1]
    \definition{adv.}{ainda; indica que a ação ou estado permanece inalterado, equivalente a 仍然 | também; além disso; em adição; indica que há um aumento ou suplemento além do escopo já indicado | ainda mais; usado com 比 para indicar que as características e o grau das coisas comparadas aumentaram, o que é equivalente a 更加 razoavelmente; medianamente; usado antes de um adjetivo, indica que algo atinge apenas o nível mínimo exigido | mesmo; usado na primeira parte da frase como complemento, e na segunda parte como conclusão, equivalente a 尚且 | que expressa realização ou descoberta; expressa surpresa por algo que não se esperava, mas que acabou acontecendo | tão cedo quanto; por um curto período de tempo; indica que já era assim há muito tempo | para dar ênfase; para reforçar o tom}
  \seealsoref{比}{bi3}
  \seealsoref{更加}{geng4 jia1}
  \seealsoref{仍然}{reng2ran2}
  \seealsoref{尚且}{shang4qie3}
  \end{phonetics}
  \begin{phonetics}{还}{huan2}[][HSK 1]
    \definition*{s.}{sobrenome Huan}
    \definition{v.}{voltar; retornar; voltar ao lugar original ou restaurar o estado original | retribuir; devolver; reembolsar; devolver o dinheiro ou os bens emprestados ao seu proprietário | dar ou fazer algo em troca; retribuir as ações dos outros}
  \end{phonetics}
\end{entry}

\begin{entry}{还有}{7,6}{⾡、⽉}
  \begin{phonetics}{还有}{hai2 you3}[][HSK 1]
    \definition{adv.}{também; ainda; além disso; então novamente; enfatizar as partes complementares, excedentes ou não mencionadas além do que já é conhecido}
  \end{phonetics}
\end{entry}

\begin{entry}{还是}{7,9}{⾡、⽇}
  \begin{phonetics}{还是}{hai2shi5}[][HSK 1]
    \definition{adv.}{ainda; ainda assim; não é a continuação de um determinado estado, fenômeno ou ação; o resultado é o mesmo de antes, sem mudanças  |que expressa uma preferência por uma alternativa; expressa comparação ou escolha feita após consideração cuidadosa, frequentemente usado para fazer sugestões | que expressa realização ou descoberta; indica que o resultado final foi inesperado}
    \definition{conj.}{ou (somente para frases interrogativas); indica várias opções, geralmente usado em perguntas | tudo; se; não importa; independentemente de; significa que, independentemente das mudanças que ocorram, o resultado permanecerá o mesmo}
  \end{phonetics}
\end{entry}

\begin{entry}{这}{7}{⾡}
  \begin{phonetics}{这}{zhe4}[][HSK 1]
    \definition{pron.}{este, isto; substitui pessoas ou coisas que estão mais próximas | agora; em vez de 这时候, tem o efeito de reforçar a ênfase}
  \seealsoref{这时候}{zhe4 shi2 hou5}
  \end{phonetics}
  \begin{phonetics}{这}{zhei4}
    \definition{pron.}{(coloquial) este}
  \end{phonetics}
\end{entry}

\begin{entry}{这儿}{7,2}{⾡、⼉}
  \begin{phonetics}{这儿}{zhe4r5}[][HSK 1]
    \definition{pron.}{aqui | agora; neste momento (utilizado apenas após 打, 从, 由)}
  \seealsoref{从}{cong2}
  \seealsoref{打}{da3}
  \seealsoref{由}{you2}
  \end{phonetics}
\end{entry}

\begin{entry}{这个}{7,3}{⾡、⼈}
  \begin{phonetics}{这个}{zhe4ge5}
    \definition{pron.}{isto; este | isso; em vez das coisas mencionadas anteriormente | assim; tal; usado antes de verbos e adjetivos, indica um grau muito profundo, com um sentido exagerado | usado junto com 那个 para indicar pessoas ou objetos indefinidos}
  \seealsoref{那个}{na4ge5}
  \end{phonetics}
\end{entry}

\begin{entry}{这么}{7,3}{⾡、⼃}
  \begin{phonetics}{这么}{zhe4 me5}[][HSK 2]
    \definition{adv.}{como este | desta maneira}
  \end{phonetics}
\end{entry}

\begin{entry}{这边}{7,5}{⾡、⾡}
  \begin{phonetics}{这边}{zhe4 bian1}[][HSK 1]
    \definition{pron.}{aqui; deste lado; refere-se a um lugar próximo}
  \end{phonetics}
\end{entry}

\begin{entry}{这时}{7,7}{⾡、⽇}
  \begin{phonetics}{这时}{zhe4 shi2}[][HSK 2]
    \definition{adv.}{neste momento}
  \end{phonetics}
\end{entry}

\begin{entry}{这时候}{7,7,10}{⾡、⽇、⼈}
  \begin{phonetics}{这时候}{zhe4 shi2 hou5}[][HSK 2]
    \definition{adv.}{neste momento}
  \end{phonetics}
\end{entry}

\begin{entry}{这里}{7,7}{⾡、⾥}
  \begin{phonetics}{这里}{zhe4 li3}[][HSK 1]
    \definition{pron.}{aqui; pronomes demonstrativo, indicando locais próximos}
  \end{phonetics}
\end{entry}

\begin{entry}{这些}{7,8}{⾡、⼆}
  \begin{phonetics}{这些}{zhe4 xie1}[][HSK 1]
    \definition{pron.}{estes; pronome demonstrativo, que indicam duas ou mais pessoas ou coisas que estão próximas}
  \end{phonetics}
\end{entry}

\begin{entry}{这样}{7,10}{⾡、⽊}
  \begin{phonetics}{这样}{zhe4 yang4}[][HSK 2]
    \definition{adv.}{assim | dessa maneira | deste modo}
  \end{phonetics}
\end{entry}

\begin{entry}{这麽}{7,14}{⾡、⿇}
  \begin{phonetics}{这麽}{zhe4 me5}
    \variantof{这么}
  \end{phonetics}
\end{entry}

\begin{entry}{进}{7}{⾡}
  \begin{phonetics}{进}{jin4}[][HSK 1]
    \definition*{s.}{sobrenome Jin}
    \definition{clas.}{para seções em um edifício ou complexo residencial; qualquer uma das várias fileiras de casas em um complexo residencial de estilo antigo}
    \definition{s.}{(matemática) base de um sistema numérico}
    \definition{v.}{avançar; ir adiante; seguir em frente; (oposto a 退) | entrar; entrar em; entrar ou sair; (oposto a 出) | receber | comer; tomar; beber | submeter; apresentar | marcar um gol}
    \definition{v.aux.}{usado após um verbo, significa ``para dentro''}
  \seealsoref{出}{chu1}
  \seealsoref{退}{tui4}
  \end{phonetics}
\end{entry}

\begin{entry}{进一步}{7,1,7}{⾡、⼀、⽌}
  \begin{phonetics}{进一步}{jin4 yi2 bu4}[][HSK 3]
    \definition{adv.}{mais; dar um passo adiante; avançar um passo}
  \end{phonetics}
\end{entry}

\begin{entry}{进入}{7,2}{⾡、⼊}
  \begin{phonetics}{进入}{jin4 ru4}[][HSK 2]
    \definition{v.}{entrar; entrar em}
  \end{phonetics}
\end{entry}

\begin{entry}{进口}{7,3}{⾡、⼝}
  \begin{phonetics}{进口}{jin4kou3}[][HSK 4]
    \definition{adj.}{importado}
    \definition{s.}{importação; entrada de um edifício ou local, também chamada de 入口}
    \definition{v.+compl.}{importar; comprar ou transportar mercadorias de outro país ou região | entrar no porto; navegar em direção a um porto}
  \seealsoref{入口}{ru4kou3}
  \end{phonetics}
\end{entry}

\begin{entry}{进化}{7,4}{⾡、⼔}
  \begin{phonetics}{进化}{jin4hua4}[][HSK 5]
    \definition[个]{s.}{evolução; os organismos se desenvolvem e evoluem do simples para o complexo e de níveis baixos para altos}
    \definition{v.}{evoluir; um termo geral usado para descrever uma mudança gradual para melhor}
  \end{phonetics}
\end{entry}

\begin{entry}{进出口}{7,5,3}{⾡、⼐、⼝}
  \begin{phonetics}{进出口}{jin4chu1kou3}
    \definition{s.}{importação e exportação}
    \definition{v.}{importar e exportar}
  \end{phonetics}
\end{entry}

\begin{entry}{进去}{7,5}{⾡、⼛}
  \begin{phonetics}{进去}{jin4 qu4}[][HSK 1]
    \definition{v.}{entrar (a partir da minha localização)}
    \definition{v.aux.}{usado depois de um verbo, significa ``ir para dentro''; para um determinado intervalo ou período de tempo}
  \end{phonetics}
\end{entry}

\begin{entry}{进行}{7,6}{⾡、⾏}
  \begin{phonetics}{进行}{jin4xing2}[][HSK 2]
    \definition{v.}{continuar; estar em andamento; estar em progresso | fazer; conduzir; realizar; executar | marchar; avançar; prosseguir; estar em marcha}
  \end{phonetics}
\end{entry}

\begin{entry}{进行编程}{7,6,12,12}{⾡、⾏、⽷、⽲}
  \begin{phonetics}{进行编程}{jin4xing2bian1cheng2}
    \definition{s.}{programa de computador executável}
  \end{phonetics}
\end{entry}

\begin{entry}{进来}{7,7}{⾡、⽊}
  \begin{phonetics}{进来}{jin4 lai2}[][HSK 1]
    \definition{v.}{entrar (para a minha localização)}
  \end{phonetics}
\end{entry}

\begin{entry}{进步}{7,7}{⾡、⽌}
  \begin{phonetics}{进步}{jin4bu4}[][HSK 3]
    \definition{adj.}{progressivo}
    \definition[个]{s.}{avanço; progresso; melhora}
    \definition{v.}{avançar; progredir; melhorar}
  \end{phonetics}
\end{entry}

\begin{entry}{进展}{7,10}{⾡、⼫}
  \begin{phonetics}{进展}{jin4zhan3}[][HSK 3]
    \definition{v.}{fazer progresso; progredir}
  \end{phonetics}
\end{entry}

\begin{entry}{远}{7}{⾡}
  \begin{phonetics}{远}{yuan3}[][HSK 1]
    \definition*{s.}{sobrenome Yuan}
    \definition{adj.}{distante (no tempo ou no espaço); longe; remoto; Longa distância espacial ou temporal (em oposição a 近) | (relações de parentesco) distante | com grande diferença}
    \definition{v.}{manter-se afastado de; não se aproximar}
  \seealsoref{近}{jin4}
  \end{phonetics}
\end{entry}

\begin{entry}{远天}{7,4}{⾡、⼤}
  \begin{phonetics}{远天}{yuan3tian1}
    \definition{s.}{paraíso | o céu distante}
  \end{phonetics}
\end{entry}

\begin{entry}{远方}{7,4}{⾡、⽅}
  \begin{phonetics}{远方}{yuan3fang1}
    \definition{s.}{longe | um local distante}
  \end{phonetics}
\end{entry}

\begin{entry}{远处}{7,5}{⾡、⼡}
  \begin{phonetics}{远处}{yuan3 chu4}[][HSK 5]
    \definition{s.}{distância; lugar distante}
  \end{phonetics}
\end{entry}

\begin{entry}{远远}{7,7}{⾡、⾡}
  \begin{phonetics}{远远}{yuan3yuan3}
    \definition{adv.}{de longe}
  \end{phonetics}
\end{entry}

\begin{entry}{远征}{7,8}{⾡、⼻}
  \begin{phonetics}{远征}{yuan3zheng1}
    \definition{s.}{uma expedição militar | marcha para regiões remotas}
  \end{phonetics}
\end{entry}

\begin{entry}{违反}{7,4}{⾡、⼜}
  \begin{phonetics}{违反}{wei2fan3}[][HSK 5]
    \definition{v.}{violar; transgredir; contrariar; não estar em conformidade (com as regras, regulamentos, etc.)}
  \end{phonetics}
\end{entry}

\begin{entry}{违法}{7,8}{⾡、⽔}
  \begin{phonetics}{违法}{wei2 fa3}[][HSK 5]
    \definition{v.}{ser ilegal; infringir a lei; violar a lei ou os regulamentos}
  \end{phonetics}
\end{entry}

\begin{entry}{违规}{7,8}{⾡、⾒}
  \begin{phonetics}{违规}{wei2 gui1}[][HSK 5]
    \definition{v.}{violar (regras); infringir as regras e regulamentos}
  \end{phonetics}
\end{entry}

\begin{entry}{违宪}{7,9}{⾡、⼧}
  \begin{phonetics}{违宪}{wei2xian4}
    \definition{adj.}{inconstitucional}
  \end{phonetics}
\end{entry}

\begin{entry}{连}{7}{⾡}
  \begin{phonetics}{连}{lian2}[][HSK 3]
    \definition*{s.}{sobrenome Lian}
    \definition{adv.}{em sucessão; um após o outro; repetidamente | até}
    \definition{prep.}{incluindo}
    \definition{s.}{companhia | conjunção}
    \definition{v.}{ligar; juntar; conectar | envolver (em problemas); implicar | costurar; coser}
  \end{phonetics}
\end{entry}

\begin{entry}{连忙}{7,6}{⾡、⼼}
  \begin{phonetics}{连忙}{lian2mang2}[][HSK 3]
    \definition{adv.}{prontamente; imediatamente; apressadamente}
  \end{phonetics}
\end{entry}

\begin{entry}{连接}{7,11}{⾡、⼿}
  \begin{phonetics}{连接}{lian2 jie1}[][HSK 5]
    \definition[条]{s.}{conexão}
    \definition{v.}{ligar; unir; relacionar, conectar; anexar}
  \end{phonetics}
\end{entry}

\begin{entry}{连续}{7,11}{⾡、⽷}
  \begin{phonetics}{连续}{lian2xu4}[][HSK 3]
    \definition{adv.}{continuamente; sucessivamente; em uma fileira}
  \end{phonetics}
\end{entry}

\begin{entry}{连续剧}{7,11,10}{⾡、⽷、⼑}
  \begin{phonetics}{连续剧}{lian2 xu4 ju4}[][HSK 3]
    \definition{s.}{série; novela}
  \end{phonetics}
\end{entry}

\begin{entry}{连锁反应}{7,12,4,7}{⾡、⾦、⼜、⼴}
  \begin{phonetics}{连锁反应}{lian2suo3fan3ying4}
    \definition{s.}{reação em cadeia}
  \end{phonetics}
\end{entry}

\begin{entry}{迟}{7}{⾡}
  \begin{phonetics}{迟}{chi2}[][HSK 5]
    \definition*{s.}{sobrenome Chi}
    \definition{adj.}{lento; tardio; demorado | atrasado | lento; obtuso}
  \end{phonetics}
\end{entry}

\begin{entry}{迟到}{7,8}{⾡、⼑}
  \begin{phonetics}{迟到}{chi2dao4}[][HSK 4]
    \definition{v.}{chegar atrasado; atrasar-se}
  \end{phonetics}
\end{entry}

\begin{entry}{邮包}{7,5}{⾢、⼓}
  \begin{phonetics}{邮包}{you2bao1}
    \definition{s.}{encomenda postal}
  \end{phonetics}
\end{entry}

\begin{entry}{邮市}{7,5}{⾢、⼱}
  \begin{phonetics}{邮市}{you2shi4}
    \definition{s.}{mercado postal}
  \end{phonetics}
\end{entry}

\begin{entry}{邮电}{7,5}{⾢、⽥}
  \begin{phonetics}{邮电}{you2dian4}
    \definition*{s.}{Correios e Telecomunicações}
  \end{phonetics}
\end{entry}

\begin{entry}{邮件}{7,6}{⾢、⼈}
  \begin{phonetics}{邮件}{you2 jian4}[][HSK 3]
    \definition[封,个]{s.}{correspondência; correio; assunto postal; um termo geral para cartas, encomendas, etc. recebidas, transportadas e entregues pelos correios | \emph{e-mail}; mensagens enviadas e recebidas por meio eletrônico}
  \end{phonetics}
\end{entry}

\begin{entry}{邮局}{7,7}{⾢、⼫}
  \begin{phonetics}{邮局}{you2ju2}[][HSK 4]
    \definition[家]{s.}{correio; agência dos correios; organizações que lidam com serviços postais}
  \end{phonetics}
\end{entry}

\begin{entry}{邮费}{7,9}{⾢、⾙}
  \begin{phonetics}{邮费}{you2fei4}
    \definition{s.}{postagem}
    \definition{v.}{postar}
  \end{phonetics}
\end{entry}

\begin{entry}{邮迷}{7,9}{⾢、⾡}
  \begin{phonetics}{邮迷}{you2mi2}
    \definition{s.}{filatelista | colecionador de selos}
  \end{phonetics}
\end{entry}

\begin{entry}{邮资}{7,10}{⾢、⾙}
  \begin{phonetics}{邮资}{you2zi1}
    \definition{s.}{postagem}
  \end{phonetics}
\end{entry}

\begin{entry}{邮递}{7,10}{⾢、⾡}
  \begin{phonetics}{邮递}{you2di4}
    \definition{v.}{enviar por correio}
  \end{phonetics}
\end{entry}

\begin{entry}{邮票}{7,11}{⾢、⽰}
  \begin{phonetics}{邮票}{you2 piao4}[][HSK 3]
    \definition[枚,张,套,版]{s.}{selo; selo postal; um \emph{voucher} vendido pelos correios e afixado na correspondência para indicar que a postagem foi paga}
  \end{phonetics}
\end{entry}

\begin{entry}{邮箱}{7,15}{⾢、⾋}
  \begin{phonetics}{邮箱}{you2 xiang1}[][HSK 3]
    \definition{s.}{caixa de correio | \emph{mailbox}; refere-se ao endereço de \emph{e-mail}}
  \end{phonetics}
\end{entry}

\begin{entry}{邻居}{7,8}{⾢、⼫}
  \begin{phonetics}{邻居}{lin2ju1}[][HSK 5]
    \definition[个,位,家]{s.}{vizinho; pessoas ou famílias que moram muito perto}
  \end{phonetics}
\end{entry}

\begin{entry}{里}{7}{⾥}[Kangxi 166]
  \begin{phonetics}{里}{li3}[][HSK 1]
    \definition*{s.}{sobrenome Li}
    \definition{clas.}{li, uma unidade chinesa de comprimento (= 1/2 quilômetro)}
    \definition{s.}{forro; revestimento; interior; parte de trás do tecido | interno; dentro; no interior | vizinhança; vizinhos | cidade natal; local de origem}
  \end{phonetics}
\end{entry}

\begin{entry}{里头}{7,5}{⾥、⼤}
  \begin{phonetics}{里头}{li3 tou5}[][HSK 2]
    \definition{s.}{dentro}
  \end{phonetics}
\end{entry}

\begin{entry}{里边}{7,5}{⾥、⾡}
  \begin{phonetics}{里边}{li3 bian5}[][HSK 1]
    \definition{s.}{em; dentro; no interior}
  \end{phonetics}
\end{entry}

\begin{entry}{里面}{7,9}{⾥、⾯}
  \begin{phonetics}{里面}{li3 mian4}[][HSK 3]
    \definition{s.}{dentro; interior}
  \end{phonetics}
\end{entry}

\begin{entry}{里斯本}{7,12,5}{⾥、⽄、⽊}
  \begin{phonetics}{里斯本}{li3si1ben3}
    \definition*{s.}{Lisboa}
  \end{phonetics}
\end{entry}

\begin{entry}{里斯本大学}{7,12,5,3,8}{⾥、⽄、⽊、⼤、⼦}
  \begin{phonetics}{里斯本大学}{li3si1ben3 da4xue2}
    \definition*{s.}{Universidade de Lisboa}
  \end{phonetics}
\end{entry}

\begin{entry}{针}{7}{⾦}
  \begin{phonetics}{针}{zhen1}[][HSK 4]
    \definition*{s.}{sobrenome Zhen}
    \definition[根]{s.}{agulha; ferramentas para costura de roupas | objetos semelhantes a agulhas; algo longo e fino como uma agulha | injeção | ponto de costura | pontos de acupuntura na medicina chinesa}
  \end{phonetics}
\end{entry}

\begin{entry}{针对}{7,5}{⾦、⼨}
  \begin{phonetics}{针对}{zhen1dui4}[][HSK 4]
    \definition{prep.}{em conexão com; de acordo com; à luz de; introdução de objetos de comportamento com uma finalidade clara}
    \definition{v.}{contrariar; apontar para; ter como objetivo; ser direcionado contra; fazer algo especificamente sobre um problema ou uma pessoa}
  \end{phonetics}
\end{entry}

\begin{entry}{闲}{7}{⾨}
  \begin{phonetics}{闲}{xian2}[][HSK 5]
    \definition{adj.}{ocioso; não ocupado; desocupado; sem coisas para fazer; sem atividades; tempo livre | desocupado; (casa, objeto, etc.) não em uso; ocioso | não oficial; não sério; não relacionado ao negócio}
    \definition{s.}{lazer; tempo livre}
  \end{phonetics}
\end{entry}

\begin{entry}{间}{7}{⾨}
  \begin{phonetics}{间}{jian1}[][HSK 1]
    \definition{clas.}{a menor unidade de uma casa; a menor unidade habitacional; cômodo}
    \definition{s.}{espaço entre duas partes  | (em um) tempo ou espaço definido | sala; quarto | uma seção de uma sala ou o espaço lateral entre dois pares de pilares | com um tempo ou espaço definido}
  \end{phonetics}
  \begin{phonetics}{间}{jian4}
    \definition{s.}{espaço entre as duas partes; abertura; lacuna}
    \definition{v.}{separar | semear a discórdia | desbastar (mudas); podar; remover ou arrancar as mudas em excesso}
  \end{phonetics}
\end{entry}

\begin{entry}{间或}{7,8}{⾨、⼽}
  \begin{phonetics}{间或}{jian4huo4}
    \definition{adv.}{às vezes | ocasionalmente | de vez em quando}
  \end{phonetics}
\end{entry}

\begin{entry}{间接}{7,11}{⾨、⼿}
  \begin{phonetics}{间接}{jian4jie1}[][HSK 5]
    \definition{adj.}{indireto; de segunda mão; em oposição a 直接}
  \seealsoref{直接}{zhi2jie1}
  \end{phonetics}
\end{entry}

\begin{entry}{闷热}{7,10}{⾨、⽕}
  \begin{phonetics}{闷热}{men1re4}
    \definition{adj.}{abafado | quente e abafado | sufocantemente quente | quente e sensual}
  \end{phonetics}
\end{entry}

\begin{entry}{阻止}{7,4}{⾩、⽌}
  \begin{phonetics}{阻止}{zu3zhi3}[][HSK 4]
    \definition{v.}{parar; reter; conter; interromper; impedir o avanço; impedir o movimento; obstruir}
  \end{phonetics}
\end{entry}

\begin{entry}{阻击}{7,5}{⾩、⼐}
  \begin{phonetics}{阻击}{zu3ji1}
    \definition{v.}{verificar | parar}
  \end{phonetics}
\end{entry}

\begin{entry}{阻碍}{7,13}{⾩、⽯}
  \begin{phonetics}{阻碍}{zu3'ai4}[][HSK 5]
    \definition{s.}{obstáculo; impedimento; barreira}
    \definition{v.}{bloquear; impedir; obstruir; impedir o bom andamento ou desenvolvimento}
  \end{phonetics}
\end{entry}

\begin{entry}{阿}{7}{⾩}
  \begin{phonetics}{阿}{a1}
    \definition{pref.}{em dialetos do sul para formar termos carinhosos, antes de nomes de animais de estimação, sobrenomes monossilábicos ou números que denotam ordem de antiguidade em uma; anexado a 大, 二, 三,\dots\ para indicar classificação (e, às vezes, intimidade) | antes dos termos de parentesco; na frente de um sobrenome, de um nome próprio ou de um determinado título, com uma conotação de intimidade | em alguns contextos, pode soar infantil ou muito informal (por exemplo, chamar um colega de trabalho por ``阿 + Nome'' sem intimidade)}[阿妈 (mamãe) | 阿明 (forma carinhosa de chamar alguém chamado Ming)]
  \end{phonetics}
  \begin{phonetics}{阿}{e1}
    \definition*{s.}{sobrenome E}
    \definition*{s.}{Dong'e (um condado na província de Shandong)}
    \definition{s.}{grande monte (ou colina) | um lugar sinuoso (montanha, água, etc.)}
    \definition{v.}{bajular; satisfazer}
  \end{phonetics}
\end{entry}

\begin{entry}{阿姨}{7,9}{⾩、⼥}
  \begin{phonetics}{阿姨}{a1yi2}[][HSK 4]
    \definition[个,位]{s.}{tia; uma forma de tratamento para uma mulher da geração dos pais; dirigir-se a uma mulher que tem aproximadamente a mesma idade da sua mãe, geralmente não é parente | babá em uma família; professora em um jardim de infância | tia; irmã da mãe (mais comum no sul da China)}[阿姨,生日快乐!(Tia, feliz aniversário!) | 阿姨,这个苹果多少钱一斤?(Tia/Senhora, quanto custa o quilo dessas maçãs?) | 阿姨,我想喝水。(Tia/Babá, eu quero beber água.)]
  \end{phonetics}
\end{entry}

\begin{entry}{阿哥}{7,10}{⾩、⼝}
  \begin{phonetics}{阿哥}{a1ge1}
    \definition{s.}{irmão mais velho (afetivo)}[阿哥,帮我拿一下书包!(Irmão, ajude-me com minha mochila escolar!)]
  \end{phonetics}
\end{entry}

\begin{entry}{附件}{7,6}{⾩、⼈}
  \begin{phonetics}{附件}{fu4jian4}[][HSK 5]
    \definition*{s.}{\emph{Adnexa Uteri} ; refere-se à genitália interna feminina que não seja o útero, as trompas de falópio e os ovários}
    \definition{s.}{apêndice; documentos que acompanham o documento principal | acessório; anexo; peças ou sobressalentes que não sejam peças principais de máquinas e equipamentos | anexo; documentos ou itens relevantes emitidos com o documento}
  \end{phonetics}
\end{entry}

\begin{entry}{附近}{7,7}{⾩、⾡}
  \begin{phonetics}{附近}{fu4jin4}[][HSK 4]
    \definition{adj.}{perto; vizinho}
    \definition{s.}{vizinhança; bairro}
  \end{phonetics}
\end{entry}

\begin{entry}{陆地}{7,6}{⾩、⼟}
  \begin{phonetics}{陆地}{lu4di4}[][HSK 4]
    \definition[块,片]{s.}{terra; terra seca (em oposição ao mar); superfície da Terra, excluindo os oceanos (e, às vezes, rios e lagos)}
  \end{phonetics}
\end{entry}

\begin{entry}{陆续}{7,11}{⾩、⽷}
  \begin{phonetics}{陆续}{lu4xu4}[][HSK 4]
    \definition{adv.}{sucessivamente; um após o outro; intermitentemente}
  \end{phonetics}
\end{entry}

\begin{entry}{陆路}{7,13}{⾩、⾜}
  \begin{phonetics}{陆路}{lu4lu4}
    \definition{s.}{rota terrestre}
  \end{phonetics}
\end{entry}

\begin{entry}{饭}{7}{⾷}
  \begin{phonetics}{饭}{fan4}[][HSK 1]
    \definition{s.}{(empréstimo linguístico) fã, devoto}
    \definition[顿,碗]{s.}{cereais cozidos; grãos cozidos | refeição; alimentos consumidos diariamente em horários regulares | trabalho; meio de subsistência; meio de vida}
  \end{phonetics}
\end{entry}

\begin{entry}{饭店}{7,8}{⾷、⼴}
  \begin{phonetics}{饭店}{fan4dian4}[][HSK 1]
    \definition[家,个]{s.}{restaurante | hotel; hotel grande e bem equipado}
  \end{phonetics}
\end{entry}

\begin{entry}{饭馆}{7,11}{⾷、⾷}
  \begin{phonetics}{饭馆}{fan4 guan3}[][HSK 2]
    \definition[家,个]{s.}{restaurante; lanchonete}
  \end{phonetics}
\end{entry}

\begin{entry}{饮食}{7,9}{⾷、⾷}
  \begin{phonetics}{饮食}{yin3shi2}[][HSK 5]
    \definition{s.}{dieta; alimentos e bebidas}
    \definition{v.}{comer; beber}
  \end{phonetics}
\end{entry}

\begin{entry}{饮料}{7,10}{⾷、⽃}
  \begin{phonetics}{饮料}{yin3liao4}[][HSK 5]
    \definition[杯,瓶,种]{s.}{bebida; drinque; líquidos processados e fabricados para consumo, como vinho, chá, refrigerantes, suco de laranja, etc.}
  \end{phonetics}
\end{entry}

\begin{entry}{驱}{7}{⾺}
  \begin{phonetics}{驱}{qu1}
    \definition{v.}{expulsar | repelir}
  \end{phonetics}
\end{entry}

\begin{entry}{驴}{7}{⾺}
  \begin{phonetics}{驴}{lv2}
    \definition[头]{s.}{burro | asno | jumento | jegue}
  \end{phonetics}
\end{entry}

\begin{entry}{鸡}{7}{⿃}
  \begin{phonetics}{鸡}{ji1}[][HSK 2]
    \definition*{s.}{sobrenome Ji}
    \definition[只]{s.}{galo, galinha, frango | palavra ofensiva para uma mulher que ganha dinheiro fazendo sexo com um homem}
  \end{phonetics}
\end{entry}

\begin{entry}{鸡蛋}{7,11}{⿃、⾍}
  \begin{phonetics}{鸡蛋}{ji1dan4}[][HSK 1]
    \definition[个,枚,筐,箱,打]{s.}{ovo de galinha}
  \end{phonetics}
\end{entry}

\begin{entry}{麦当劳}{7,6,7}{⿆、⼹、⼒}
  \begin{phonetics}{麦当劳}{mai4dang1lao2}
    \definition*{s.}{McDonald's (empresa de \emph{fast-food})}
  \seealsoref{麦当劳叔叔}{mai4dang1lao2 shu1shu5}
  \end{phonetics}
\end{entry}

\begin{entry}{麦当劳叔叔}{7,6,7,8,8}{⿆、⼹、⼒、⼜、⼜}
  \begin{phonetics}{麦当劳叔叔}{mai4dang1lao2 shu1shu5}
    \definition*{s.}{Ronald McDonald}
  \seealsoref{麦当劳}{mai4dang1lao2}
  \end{phonetics}
\end{entry}

\begin{entry}{麦淇淋}{7,11,11}{⿆、⽔、⽔}
  \begin{phonetics}{麦淇淋}{mai4qi2lin2}
    \definition{s.}{(empréstimo linguístico) margarina}
  \end{phonetics}
\end{entry}

\begin{entry}{龟速}{7,10}{⿔、⾡}
  \begin{phonetics}{龟速}{gui1su4}
    \definition{adv.}{tão lento quanto uma tartaruga}
  \end{phonetics}
\end{entry}

%%%%% EOF %%%%%


 %%%
%%% 8画
%%%
\section*{8画}\addcontentsline{toc}{section}{8画}\addcontentsline{loh}{figure}{\#\#\#\# 8画}

%%%%%%%%%% 丧 %%%%%%%%%%
\subsection*{丧}\addcontentsline{loh}{figure}{丧}

\begin{Entry}{丧}{8}{⼗}
  \begin{Phonetics}{丧}{sang1}
    \definition{adj.}{decepcionado; deprimido; desanimado}
    \definition{v.}{perder | desanimar; frustrar}
  \end{Phonetics}
  \begin{Phonetics}{丧}{sang4}
    \definition{adj.}{decepcionado | desanimado}
    \definition{v.}{estar enlutado (do cônjuge etc.) | morrer}
  \end{Phonetics}
\end{Entry}

\begin{Entry}{丧失}{8,5}{⼗,⼤}
  \begin{Phonetics}{丧失}{sang4shi1}[][HSK 6]
    \definition{v.}{perder (algo que se tem)}
  \end{Phonetics}
\end{Entry}

\begin{Entry}{丧生}{8,5}{⼗,⽣}
  \begin{Phonetics}{丧生}{sang4/sheng1}[][HSK 7-9]
    \definition{v.+compl.}{morrer; encontrar a morte; perder a vida; ser morto}
  \seealsoref{丧身}{sang4shen1}
  \end{Phonetics}
\end{Entry}

\begin{Entry}{丧身}{8,7}{⼗,⾝}
  \begin{Phonetics}{丧身}{sang4shen1}
    \definition{v.}{morrer; perder a vida}
  \seealsoref{丧生}{sang4/sheng1}
  \end{Phonetics}
\end{Entry}

\begin{Entry}{丧钟}{8,9}{⼗,⾦}
  \begin{Phonetics}{丧钟}{sang1zhong1}
    \definition{s.}{sentença de morte}
  \end{Phonetics}
\end{Entry}

%%%%%%%%%% 乖 %%%%%%%%%%
\subsection*{乖}\addcontentsline{loh}{figure}{乖}

\begin{Entry}{乖}{8}{⼃}
  \begin{Phonetics}{乖}{guai1}[][HSK 7-9]
    \definition{adj.}{(uma criança) bem comportado; bom; obediente | inteligente; astuto; esperto | (caráter, comportamento, etc.) estranho; anormal; irracional}
    \definition{v.}{perverter; ser contrário à razão; ir contra | (caráter, comportamento, etc.) ser anormal; ser estranho}
  \end{Phonetics}
\end{Entry}

\begin{Entry}{乖巧}{8,5}{⼃,⼯}
  \begin{Phonetics}{乖巧}{guai1qiao3}[][HSK 7-9]
    \definition{adj.}{fofo; adorável; agradável; descreve crianças, pequenos animais, etc. como sendo obedientes, fofos e simpáticos | inteligente; engenhoso; descreve uma pessoa que sempre fala ou faz coisas de acordo com os desejos de outras pessoas e é querida por elas}
  \synonymref{懂事}{dong3shi4}
  \synonymref{乖乖}{guai1guai1}
  \synonymref{可爱}{ke3'ai4}
  \synonymref{灵活}{ling2huo2}
  \synonymref{灵敏}{ling2min3}
  \synonymref{能干}{neng2gan4}
  \synonymref{听话}{ting1/hua4}
  \antonymref{顽皮}{wan2pi2}
  \end{Phonetics}
\end{Entry}

\begin{Entry}{乖张}{8,7}{⼃,⼸}
  \begin{Phonetics}{乖张}{guai1zhang1}
    \definition{adj.}{excêntrico e irracional; perverso; recalcitrante | não suave; sem sucesso | irritadiço | irrazoável}
  \synonymref{荒诞}{huang1dan4}
  \synonymref{荒谬}{huang1miu4}
  \synonymref{荒唐}{huang1tang2}
  \antonymref{随和}{sui2he5}
  \antonymref{温和}{wen1he2}
  \antonymref{温柔}{wen1rou2}
  \antonymref{温顺}{wen1shun4}
  \end{Phonetics}
\end{Entry}

\begin{Entry}{乖乖}{8,8}{⼃,⼃}
  \begin{Phonetics}{乖乖}{guai1guai1}
    \definition{adj.}{bem"-comportado; obediente}
    \definition{s.}{bebezinho; pequenino; querido; docinho (usado apenas para crianças)}
  \synonymref{乖巧}{guai1qiao3}
  \synonymref{调皮}{tiao2pi2}
  \synonymref{听话}{ting1/hua4}
  \antonymref{叛逆}{pan4ni4}
  \end{Phonetics}
  \begin{Phonetics}{乖乖}{guai1guai5}
    \definition{expr.}{Uau!; Nossa!; Meu Deus!; Oh meu Deus!}
  \end{Phonetics}
\end{Entry}

%%%%%%%%%% 乳 %%%%%%%%%%
\subsection*{乳}\addcontentsline{loh}{figure}{乳}

\begin{Entry}{乳}{8}{⼄}
  \begin{Phonetics}{乳}{ru3}
    \definition{adj.}{recém-nascido (animal); lactente}
    \definition{s.}{mama; peito | leite (em geral) | qualquer líquido semelhante ao leite}
    \definition{v.}{dar à luz}
  \end{Phonetics}
\end{Entry}

\begin{Entry}{乳制品}{8,8,9}{⼄,⼑,⼝}
  \begin{Phonetics}{乳制品}{ru3zhi4pin3}[][HSK 6]
    \definition{s.}{produtos lácteos}
  \end{Phonetics}
\end{Entry}

\begin{Entry}{乳房}{8,8}{⼄,⼾}
  \begin{Phonetics}{乳房}{ru3fang2}
    \definition{s.}{seio | mama | úbere}
  \end{Phonetics}
\end{Entry}

%%%%%%%%%% 事 %%%%%%%%%%
\subsection*{事}\addcontentsline{loh}{figure}{事}

\begin{Entry}{事}{8}{⼅}
  \begin{Phonetics}{事}{shi4}[][HSK 1]
    \definition[件,桩,回]{s.}{assunto; questão; coisa; negócio | problema; acidente | emprego; trabalho | responsabilidade; envolvimento | caso, coisa; o que aconteceu}
    \definition{v.}{servir; atender | estar envolvido em; dedicar"-se a}
  \end{Phonetics}
\end{Entry}

\begin{Entry}{事儿}{8,2}{⼅,⼉}
  \begin{Phonetics}{事儿}{shi4r5}
    \definition[件,桩]{s.}{o emprego | negócio | afazeres | assunto que precisa ser resolvido | matéria}
  \end{Phonetics}
\end{Entry}

\begin{Entry}{事业}{8,5}{⼅,⼀}
  \begin{Phonetics}{事业}{shi4ye4}[][HSK 3]
    \definition[个]{s.}{causa; carreira; empreendimento; atividades regulares realizadas por pessoas com um determinado objetivo, escala e sistema que têm impacto no desenvolvimento social | instituição; instalações; unidade de trabalho apoiada financeiramente pelo governo; refere"-se especificamente a empresas que não têm rendimentos de produção, são financiadas pelo Estado e não realizam contabilidade económica}
  \synonymref{工作}{gong1zuo4}
  \synonymref{事迹}{shi4ji4}
  \synonymref{职业}{zhi2ye4}
  \end{Phonetics}
\end{Entry}

\begin{Entry}{事务}{8,5}{⼅,⼒}
  \begin{Phonetics}{事务}{shi4wu4}[][HSK 7-9]
    \definition{s.}{trabalho; rotina; assuntos; refere"-se a um negócio específico | assuntos gerais | casos; tarefas específicas; trabalho diário}
  \synonymref{工作}{gong1zuo4}
  \synonymref{事件}{shi4jian4}
  \synonymref{事宜}{shi4yi2}
  \synonymref{事情}{shi4qing5}
  \end{Phonetics}
\end{Entry}

\begin{Entry}{事务所}{8,5,8}{⼅,⼒,⼾}
  \begin{Phonetics}{事务所}{shi4wu4suo3}[][HSK 7-9]
    \definition{s.}{escritório; firma; empresa}
  \end{Phonetics}
\end{Entry}

\begin{Entry}{事件}{8,6}{⼅,⼈}
  \begin{Phonetics}{事件}{shi4jian4}[][HSK 3]
    \definition[个,件,次]{s.}{evento; incidente; grandes eventos na história ou na sociedade}
  \synonymref{事故}{shi4gu4}
  \synonymref{事务}{shi4wu4}
  \synonymref{事项}{shi4xiang4}
  \synonymref{事宜}{shi4yi2}
  \synonymref{事情}{shi4qing5}
  \end{Phonetics}
\end{Entry}

\begin{Entry}{事先}{8,6}{⼅,⼉}
  \begin{Phonetics}{事先}{shi4xian1}[][HSK 4]
    \definition{adv.}{antes; de antemão; com antecedência; antecipadamente}
  \synonymref{早就}{zao3jiu4}
  \antonymref{当时}{dang1shi2}
  \antonymref{当时}{dang4shi2}
  \antonymref{事后}{shi4hou4}
  \end{Phonetics}
\end{Entry}

\begin{Entry}{事后}{8,6}{⼅,⼝}
  \begin{Phonetics}{事后}{shi4hou4}[][HSK 6]
    \definition{s.}{depois; depois do evento; após o incidente ocorrer ou o problema ser resolvido}
  \synonymref{过后}{guo4hou4}
  \synonymref{之后}{zhi1hou4}
  \antonymref{当时}{dang1shi2}
  \antonymref{当时}{dang4shi2}
  \antonymref{事先}{shi4xian1}
  \end{Phonetics}
\end{Entry}

\begin{Entry}{事宜}{8,8}{⼅,⼧}
  \begin{Phonetics}{事宜}{shi4yi2}[][HSK 7-9]
    \definition[个,项]{s.}{acordos; assuntos em questão; assuntos ou tarefas que precisam ser organizados ou tratados (frequentemente usado em documentos oficiais, leis, etc.)}
  \synonymref{符合}{fu2he2}
  \synonymref{合适}{he2shi4}
  \synonymref{恰当}{qia4dang4}
  \synonymref{适合}{shi4he2}
  \synonymref{事件}{shi4jian4}
  \synonymref{事务}{shi4wu4}
  \synonymref{适应}{shi4ying4}
  \synonymref{事情}{shi4qing5}
  \synonymref{妥当}{tuo3dang4}
  \synonymref{妥善}{tuo3shan4}
  \end{Phonetics}
\end{Entry}

\begin{Entry}{事实}{8,8}{⼅,⼧}
  \begin{Phonetics}{事实}{shi4shi2}[][HSK 3]
    \definition[个,件]{s.}{mito; lenda; uma narrativa sobre alguém ou algo que foi transmitida oralmente}
    \definition{v.}{dizer; contar; ser dito; contar a história}
  \synonymref{毕竟}{bi4jing4}
  \synonymref{到底}{dao4di3}
  \synonymref{结果}{jie2/guo3}
  \synonymref{究竟}{jiu1jing4}
  \synonymref{客观}{ke4guan1}
  \synonymref{真相}{zhen1xiang4}
  \antonymref{传奇}{chuan2qi2}
  \antonymref{据说}{ju4shuo1}
  \antonymref{梦想}{meng4xiang3}
  \antonymref{神话}{shen2hua4}
  \antonymref{虚伪}{xu1wei3}
  \end{Phonetics}
\end{Entry}

\begin{Entry}{事实上}{8,8,3}{⼅,⼧,⼀}
  \begin{Phonetics}{事实上}{shi4shi2shang5}[][HSK 3]
    \definition{adv.}{realmente; de fato; na realidade; na verdade; de fato}
  \synonymref{实际上}{shi2ji4shang5}
  \end{Phonetics}
\end{Entry}

\begin{Entry}{事态}{8,8}{⼅,⼼}
  \begin{Phonetics}{事态}{shi4tai4}[][HSK 7-9]
    \definition{s.}{estado de coisas; situação; circunstâncias (na maioria das vezes ruins)}
  \synonymref{大局}{da4ju2}
  \synonymref{局面}{ju2mian4}
  \synonymref{形势}{xing2shi4}
  \end{Phonetics}
\end{Entry}

\begin{Entry}{事物}{8,8}{⼅,⽜}
  \begin{Phonetics}{事物}{shi4wu4}[][HSK 4]
    \definition[种,类,个]{s.}{coisa; objeto; todos os objetos e fenômenos que existem objetivamente}
  \end{Phonetics}
\end{Entry}

\begin{Entry}{事故}{8,9}{⼅,⽁}
  \begin{Phonetics}{事故}{shi4gu4}[][HSK 3]
    \definition[起,桩,次,场]{s.}{acidente; perdas ou desastres repentinos, muitas vezes relacionados ao transporte, produção, trabalho e segurança pessoal}
  \synonymref{故障}{gu4zhang4}
  \synonymref{意外}{yi4wai4}
  \end{Phonetics}
\end{Entry}

\begin{Entry}{事迹}{8,9}{⼅,⾡}
  \begin{Phonetics}{事迹}{shi4ji4}[][HSK 7-9]
    \definition{s.}{feito; conquista; coisas importantes que um indivíduo ou grupo fez no passado}
  \synonymref{古迹}{gu3ji4}
  \synonymref{奇迹}{qi2ji4}
  \synonymref{事业}{shi4ye4}
  \synonymref{遗迹}{yi2ji4}
  \end{Phonetics}
\end{Entry}

\begin{Entry}{事项}{8,9}{⼅,⾴}
  \begin{Phonetics}{事项}{shi4xiang4}[][HSK 7-9]
    \definition{s.}{item; matéria; projeto}
  \synonymref{事故}{shi4gu4}
  \synonymref{事件}{shi4jian4}
  \synonymref{事情}{shi4qing5}
  \end{Phonetics}
\end{Entry}

\begin{Entry}{事情}{8,11}{⼅,⼼}
  \begin{Phonetics}{事情}{shi4qing5}[][HSK 2]
    \definition[件,个,些,种]{s.}{assunto; questão; coisa; negócio | erro; acidente; infortúnio | (coloquial) emprego; trabalho}
  \synonymref{工作}{gong1zuo4}
  \synonymref{活儿}{huo2r5}
  \synonymref{事故}{shi4gu4}
  \synonymref{事件}{shi4jian4}
  \synonymref{事务}{shi4wu4}
  \synonymref{事项}{shi4xiang4}
  \synonymref{事宜}{shi4yi2}
  \end{Phonetics}
\end{Entry}

%%%%%%%%%% 些 %%%%%%%%%%
\subsection*{些}\addcontentsline{loh}{figure}{些}

\begin{Entry}{些}{8}{⼆}
  \begin{Phonetics}{些}{xie1}[][HSK 4]
    \definition{adv.}{um pouco; um pouco mais; usado após um adjetivo ou parte de um verbo para indicar uma pequena quantidade, equivalente a 一点儿}
    \definition{clas.}{alguns; um pouco; denota uma quantidade indefinida}
  \seealsoref{一点儿}{yi4dian3r5}
  \end{Phonetics}
\end{Entry}

\begin{Entry}{些许}{8,6}{⼆,⾔}
  \begin{Phonetics}{些许}{xie1xu3}
    \definition{num.}{um pouco}
  \synonymref{适量}{shi4liang4}
  \antonymref{长足}{chang2zu2}
  \end{Phonetics}
\end{Entry}

%%%%%%%%%% 享 %%%%%%%%%%
\subsection*{享}\addcontentsline{loh}{figure}{享}

\begin{Entry}{享}{8}{⼇}
  \begin{Phonetics}{享}{xiang3}
    \definition{v.}{aproveitar}
  \end{Phonetics}
\end{Entry}

\begin{Entry}{享受}{8,8}{⼇,⼜}
  \begin{Phonetics}{享受}{xiang3shou4}[][HSK 5]
    \definition{v.}{aproveitar; desfrutar; estar satisfeito material ou espiritualmente}
  \antonymref{吃苦}{chi1/ku3}
  \antonymref{负担}{fu4dan1}
  \end{Phonetics}
\end{Entry}

%%%%%%%%%% 京 %%%%%%%%%%
\subsection*{京}\addcontentsline{loh}{figure}{京}

\begin{Entry}{京}{8}{⼇}
  \begin{Phonetics}{京}{jing1}
    \definition*{s.}{Pequim (Beijing), abreviação de 北京 | Sobrenome: Jing}
    \definition{num.}{dez milhões (um numeral antigo); 10.000.000; 1000.0000}
    \definition{s.}{capital de um país}
  \seealsoref{北京}{bei3jing1}
  \end{Phonetics}
\end{Entry}

\begin{Entry}{京二胡}{8,2,9}{⼇,⼆,⾁}
  \begin{Phonetics}{京二胡}{jing1'er4hu2}
    \definition{s.}{um tipo de violino chinês semelhante ao 二胡 de duas cordas, usado principalmente para acompanhamento do canto da ópera de Pequim | também chamado de 京胡 | jing'erhu, um violino de duas cordas, intermediário em tamanho e tom entre o 京胡 e o 二胡, usado para acompanhar a ópera chinesa}
  \seealsoref{二胡}{er4hu2}
  \seealsoref{京胡}{jing1hu2}
  \end{Phonetics}
\end{Entry}

\begin{Entry}{京胡}{8,9}{⼇,⾁}
  \begin{Phonetics}{京胡}{jing1hu2}
    \definition{s.}{jinghu, um instrumento de arco de duas cordas com registro agudo; violino da ópera de Pequim | também chamado de 京二胡 | jinghu, um 二胡 (violino de duas cordas) menor e mais agudo, usado para acompanhar a ópera chinesa}
  \seealsoref{二胡}{er4hu2}
  \seealsoref{胡琴}{hu2qin2}
  \seealsoref{京二胡}{jing1'er4hu2}
  \end{Phonetics}
\end{Entry}

\begin{Entry}{京剧}{8,10}{⼇,⼑}
  \begin{Phonetics}{京剧}{jing1ju4}[][HSK 3]
    \definition*[场,段]{s.}{Ópera de Pequim}
  \synonymref{戏剧}{xi4ju4}
  \end{Phonetics}
\end{Entry}

%%%%%%%%%% 佩 %%%%%%%%%%
\subsection*{佩}\addcontentsline{loh}{figure}{佩}

\begin{Entry}{佩}{8}{⼈}
  \begin{Phonetics}{佩}{pei4}
    \definition{s.}{um ornamento usado como pingente amarrados em cintos nos tempos antigos}
    \definition{v.}{vestir (na cintura, etc.) | (arcaico) admirar | (arcaico) usar, especialmente uma pistola ou espada, na cintura}
  \end{Phonetics}
\end{Entry}

\begin{Entry}{佩服}{8,8}{⼈,⽉}
  \begin{Phonetics}{佩服}{pei4fu2}[][HSK 7-9]
    \definition{v.}{admirar; respeitar; dar os parabéns a alguém; ter uma alta opinião de alguém; considerar respeitáveis e adoráveis}
  \end{Phonetics}
\end{Entry}

%%%%%%%%%% 佳 %%%%%%%%%%
\subsection*{佳}\addcontentsline{loh}{figure}{佳}

\begin{Entry}{佳}{8}{⼈}
  \begin{Phonetics}{佳}{jia1}
    \definition{adj.}{bom; ótimo; bonito; excelente | o melhor}
  \end{Phonetics}
\end{Entry}

\begin{Entry}{佳节}{8,5}{⼈,⾋}
  \begin{Phonetics}{佳节}{jia1jie2}[][HSK 7-9]
    \definition{s.}{festival; época feliz de festival}[春节是中国人最重要的佳节。===O Festival da Primavera é o festival mais importante para o povo chinês.]
  \end{Phonetics}
\end{Entry}

%%%%%%%%%% 使 %%%%%%%%%%
\subsection*{使}\addcontentsline{loh}{figure}{使}

\begin{Entry}{使}{8}{⼈}
  \begin{Phonetics}{使}{shi3}[][HSK 3]
    \definition{conj.}{se; supondo; usado como a primeira cláusula de uma frase complexa; indica uma relação hipotética; equivalente a 假如}
    \definition{s.}{enviado; mensageiro; pessoas em uma missão}
    \definition{v.}{enviar; despachar; dizer a alguém para fazer algo | usar; empregar; aplicar | deixar; chamar; habilitar}
  \seealsoref{假如}{jia3ru2}
  \end{Phonetics}
\end{Entry}

\begin{Entry}{使用}{8,5}{⼈,⽤}
  \begin{Phonetics}{使用}{shi3yong4}[][HSK 2]
    \definition{v.}{usar; empregar; aplicar; fazer com que pessoas, equipamentos, fundos, etc. sirvam a um determinado propósito}
  \end{Phonetics}
\end{Entry}

\begin{Entry}{使劲}{8,7}{⼈,⼒}
  \begin{Phonetics}{使劲}{shi3/jin4}[][HSK 4]
    \definition{v.+compl.}{colocar energia; exercer toda a sua força | esforçar"-se para ajudar; colocar energia para ajudar}
  \end{Phonetics}
\end{Entry}

\begin{Entry}{使命}{8,8}{⼈,⼝}
  \begin{Phonetics}{使命}{shi3ming4}[][HSK 7-9]
    \definition{s.}{dever; missão; propósito (uma grande responsabilidade ou dever que uma pessoa assume); uma metáfora para a grande responsabilidade que se carrega | missão (uma tarefa ou dever atribuído a alguém que deve ser cumprido); as ordens recebidas pelo instigador}
  \synonymref{任务}{ren4wu5}
  \end{Phonetics}
\end{Entry}

\begin{Entry}{使者}{8,8}{⼈,⽼}
  \begin{Phonetics}{使者}{shi3zhe3}[][HSK 7-9]
    \definition[名,位]{s.}{enviado; mensageiro; emissário; uma pessoa que está cumprindo uma missão (atualmente, geralmente se referindo a pessoal diplomático)}
  \synonymref{大使}{da4shi3}
  \end{Phonetics}
\end{Entry}

\begin{Entry}{使唤}{8,10}{⼈,⼝}
  \begin{Phonetics}{使唤}{shi3huan5}[][HSK 7-9]
    \definition{v.}{ordenar sobre; pedir a alguém para fazer coisas por você | Coloquial: usar (ferramentas, animais, etc.) ; manusear}
  \synonymref{吩咐}{fen1fu4}
  \antonymref{听从}{ting1cong2}
  \end{Phonetics}
\end{Entry}

\begin{Entry}{使得}{8,11}{⼈,⼻}
  \begin{Phonetics}{使得}{shi3de5}[][HSK 5]
    \definition{v.}{ser utilizável; poder ser usado | ser viável; ser exequível; ser possível;  poder fazer | fazer; tornar; causar um determinado resultado (intenção, plano, coisa)}
  \end{Phonetics}
\end{Entry}

%%%%%%%%%% 侃 %%%%%%%%%%
\subsection*{侃}\addcontentsline{loh}{figure}{侃}

\begin{Entry}{侃}{8}{⼈}
  \begin{Phonetics}{侃}{kan3}
    \definition{adj.}{íntegro e honesto; reto e franco; direto | amável; agradável | animado; alegre}
    \definition{v.}{Coloquial: bater papo ociosamente; conversar à toa; fofocar | gabar"-se | conversar fluentemente}
  \end{Phonetics}
\end{Entry}

\begin{Entry}{侃大山}{8,3,3}{⼈,⼤,⼭}
  \begin{Phonetics}{侃大山}{kan3 da4shan1}[][HSK 7-9]
    \definition{v.}{fofocar; dedurar; bater papo; bater um papo; conversar fiado}
  \end{Phonetics}
\end{Entry}

%%%%%%%%%% 例 %%%%%%%%%%
\subsection*{例}\addcontentsline{loh}{figure}{例}

\begin{Entry}{例}{8}{⼈}
  \begin{Phonetics}{例}{li4}
    \definition{adj.}{regular; rotineiro}
    \definition{s.}{exemplo; instância | precedente | caso; instância | regras; estatutos; regulamentos}
    \definition{v.}{analogizar}
  \end{Phonetics}
\end{Entry}

\begin{Entry}{例子}{8,3}{⼈,⼦}
  \begin{Phonetics}{例子}{li4zi5}[][HSK 2]
    \definition[个]{s.}{exemplo; algo usado para ajudar a explicar ou provar uma determinada situação ou afirmação}
  \end{Phonetics}
\end{Entry}

\begin{Entry}{例外}{8,5}{⼈,⼣}
  \begin{Phonetics}{例外}{li4wai4}[][HSK 5]
    \definition[个,种]{s.}{exceção; situações que não se enquadram nas regras gerais ou nas leis comuns}
    \definition{v.}{ser excepcional; ser uma exceção}
  \end{Phonetics}
\end{Entry}

\begin{Entry}{例如}{8,6}{⼈,⼥}
  \begin{Phonetics}{例如}{li4ru2}[][HSK 2]
    \definition{conj.}{por exemplo; tal como; como por exemplo; colocado antes do exemplo, indica que o exemplo vem a seguir}
  \end{Phonetics}
\end{Entry}

%%%%%%%%%% 侍 %%%%%%%%%%
\subsection*{侍}\addcontentsline{loh}{figure}{侍}

\begin{Entry}{侍}{8}{⼈}
  \begin{Phonetics}{侍}{shi4}
    \definition{v.}{atender a; servir a; acompanhar e servir}
  \end{Phonetics}
\end{Entry}

\begin{Entry}{侍候}{8,10}{⼈,⼈}
  \begin{Phonetics}{侍候}{shi4hou4}[][HSK 7-9]
    \definition{v.}{atender; cuidar de; prestar assistência | atender a; estar presente em}
  \synonymref{伺候}{ci4hou5}
  \end{Phonetics}
\end{Entry}

%%%%%%%%%% 供 %%%%%%%%%%
\subsection*{供}\addcontentsline{loh}{figure}{供}

\begin{Entry}{供}{8}{⼈}
  \begin{Phonetics}{供}{gong1}[][HSK 7-9]
    \definition*{s.}{Sobrenome: Gong}
    \definition{v.}{fornecer; alimentar |  fornecer algo (para uso ou conveniência de); fornecer algumas condições de exploração à outra parte}
  \end{Phonetics}
  \begin{Phonetics}{供}{gong4}
    \definition{s.}{oferendas | confissão}
    \definition{v.}{depositar (oferendas) | confessar}
  \end{Phonetics}
\end{Entry}

\begin{Entry}{供不应求}{8,4,7,7}{⼈,⼀,⼴,⽔}
  \begin{Phonetics}{供不应求}{gong1bu2ying4qiu2}[][HSK 7-9]
    \definition{expr.}{``A oferta fica aquém da demanda.'' ou ``A demanda excede a oferta.''}
  \end{Phonetics}
\end{Entry}

\begin{Entry}{供应}{8,7}{⼈,⼴}
  \begin{Phonetics}{供应}{gong1ying4}[][HSK 4]
    \definition{v.}{fornecer; prover de}
  \end{Phonetics}
\end{Entry}

\begin{Entry}{供求}{8,7}{⼈,⽔}
  \begin{Phonetics}{供求}{gong1qiu2}[][HSK 7-9]
    \definition{s.}{Economia: oferta e procura (principalmente de commodities)}
  \end{Phonetics}
\end{Entry}

\begin{Entry}{供奉}{8,8}{⼈,⼤}
  \begin{Phonetics}{供奉}{gong4feng4}[][HSK 7-9]
    \definition{s.}{artista servindo ao imperador; uma pessoa que serve ao imperador com alguma habilidade; especialmente um ator que é convocado ao palácio para atuar}
    \definition{v.}{consagrar; consagrar e adorar; colocar incenso e velas em frente aos retratos ou tábuas de deuses, Budas ou ancestrais; colocar oferendas; mostrar respeito | prestar homenagem à corte imperial}
  \end{Phonetics}
\end{Entry}

\begin{Entry}{供给}{8,9}{⼈,⽷}
  \begin{Phonetics}{供给}{gong1ji3}[][HSK 6]
    \definition{s.}{fornecer; prover; fornecer produção e necessidades de vida, dinheiro, etc. para aqueles que precisam}
  \end{Phonetics}
\end{Entry}

\begin{Entry}{供暖}{8,13}{⼈,⽇}
  \begin{Phonetics}{供暖}{gong1nuan3}[][HSK 7-9]
    \definition{s.}{fornecimento de aquecimento}
    \definition{v.}{fornecer aquecimento}
  \end{Phonetics}
\end{Entry}

%%%%%%%%%% 依 %%%%%%%%%%
\subsection*{依}\addcontentsline{loh}{figure}{依}

\begin{Entry}{依}{8}{⼈}
  \begin{Phonetics}{依}{yi1}
    \definition*{s.}{Sobrenome: Yi}
    \definition{prep.}{de acordo com; à luz de; julgando por}
    \definition{v.}{depender de; ser dependente de; confiar em | cumprir; ouvir; ceder a | inclinar"-se; descansar sobre (ou contra)}
  \end{Phonetics}
\end{Entry}

\begin{Entry}{依旧}{8,5}{⼈,⽇}
  \begin{Phonetics}{依旧}{yi1jiu4}[][HSK 5]
    \definition{adv.}{ainda; como antes; como sempre}
  \end{Phonetics}
\end{Entry}

\begin{Entry}{依次}{8,6}{⼈,⽋}
  \begin{Phonetics}{依次}{yi1ci4}[][HSK 6]
    \definition{adv.}{sucessivamente; na ordem correta; em ordem}
  \end{Phonetics}
\end{Entry}

\begin{Entry}{依法}{8,8}{⼈,⽔}
  \begin{Phonetics}{依法}{yi1fa3}[][HSK 5]
    \definition{adv.}{e acordo com regras (ou métodos) fixas | de acordo com a lei; por força da lei; em conformidade com as disposições legais}
  \end{Phonetics}
\end{Entry}

\begin{Entry}{依偎}{8,11}{⼈,⼈}
  \begin{Phonetics}{依偎}{yi1wei1}
    \definition{v.}{aninhar"-se | aconchegar"-se}
  \end{Phonetics}
\end{Entry}

\begin{Entry}{依据}{8,11}{⼈,⼿}
  \begin{Phonetics}{依据}{yi1ju4}[][HSK 5]
    \definition{prep.}{julgando por; de acordo com; à luz de; com base em; de acordo com; introduzir algo que possa servir como premissa ou base}
    \definition[个]{s.}{base; evidência; fundamento; base para tomar uma decisão ou realizar uma ação}
    \definition{v.}{basear"-se em; confiar em; depdender de; usar algo como premissa ou base}
  \end{Phonetics}
\end{Entry}

\begin{Entry}{依然}{8,12}{⼈,⽕}
  \begin{Phonetics}{依然}{yi1ran2}[][HSK 4]
    \definition{adv.}{ainda; como antes}
    \definition{v.}{estar quieto; estar como antes; estar como o original, sem alterações}
  \end{Phonetics}
\end{Entry}

\begin{Entry}{依照}{8,13}{⼈,⽕}
  \begin{Phonetics}{依照}{yi1zhao4}[][HSK 5]
    \definition{prep.}{de acordo com; à luz de; introduzir certos padrões para os eventos, o que equivale a 按照}
    \definition{v.}{seguir (com base em algo)}
  \seealsoref{按照}{an4zhao4}
  \end{Phonetics}
\end{Entry}

\begin{Entry}{依赖}{8,13}{⼈,⾙}
  \begin{Phonetics}{依赖}{yi1lai4}[][HSK 6]
    \definition{v.}{confiar em; ser dependente de; ser completamente dependente e inseparável | depender de; ser mutuamente dependentes e inseparáveis}
  \end{Phonetics}
\end{Entry}

\begin{Entry}{依靠}{8,15}{⼈,⾮}
  \begin{Phonetics}{依靠}{yi1kao4}[][HSK 4]
    \definition{s.}{apoio; suporte; algo em que se apoiar; alguém ou algo em quem você pode confiar}
    \definition{v.}{depender de; confiar em (alguém ou alguma coisa para atingir um determinado objetivo)}
  \end{Phonetics}
\end{Entry}

%%%%%%%%%% 侧 %%%%%%%%%%
\subsection*{侧}\addcontentsline{loh}{figure}{侧}

\begin{Entry}{侧}{8}{⼈}
  \begin{Phonetics}{侧}{ce4}[][HSK 6]
    \definition*{s.}{Sobrenome: Ce}
    \definition{s.}{lado | inclinação}
    \definition{v.}{inclinar; inclinar"-se}
  \end{Phonetics}
  \begin{Phonetics}{侧}{zhai1}
    \definition{adj.}{inclinado; torto}
  \end{Phonetics}
\end{Entry}

\begin{Entry}{侧重}{8,9}{⼈,⾥}
  \begin{Phonetics}{侧重}{ce4zhong4}[][HSK 7-9]
    \definition{v.}{enfatizar; focar em (um certo aspecto)}
  \end{Phonetics}
\end{Entry}

\begin{Entry}{侧面}{8,9}{⼈,⾯}
  \begin{Phonetics}{侧面}{ce4mian4}[][HSK 7-9]
    \definition[个]{s.}{lado; flanco | vias indiretas; canais informais; algum aspecto; outro aspecto}
  \end{Phonetics}
\end{Entry}

%%%%%%%%%% 兔 %%%%%%%%%%
\subsection*{兔}\addcontentsline{loh}{figure}{兔}

\begin{Entry}{兔}{8}{⼉}
  \begin{Phonetics}{兔}{tu4}[][HSK 5]
    \definition[只]{s.}{lebre; coelho}
  \end{Phonetics}
\end{Entry}

\begin{Entry}{兔子}{8,3}{⼉,⼦}
  \begin{Phonetics}{兔子}{tu4zi5}
    \definition[只]{s.}{coelho | lebre}
  \end{Phonetics}
\end{Entry}

%%%%%%%%%% 其 %%%%%%%%%%
\subsection*{其}\addcontentsline{loh}{figure}{其}

\begin{Entry}{其}{8}{⼋}
  \begin{Phonetics}{其}{qi2}[][HSK 5]
    \definition*{s.}{Sobrenome: Qi}
    \definition{adv.}{fazer uma suposição ou uma réplica | expressar comando, ordem}
    \definition{pron.}{dele (dela, deles, delas) | ele, ela, isso, eles; elas | isso; tal | isso (referindo"-se a nenhuma pessoa ou coisa específica)}
    \definition{suf.}{sufixo de palavra, anexado ao advérbio}
  \end{Phonetics}
\end{Entry}

\begin{Entry}{其中}{8,4}{⼋,⼁}
  \begin{Phonetics}{其中}{qi2zhong1}[][HSK 2]
    \definition{pron.}{dentro; entre (os quais, eles, etc.); em (o qual, ele, etc.); nas pessoas ou coisas mencionadas anteriormente}
  \end{Phonetics}
\end{Entry}

\begin{Entry}{其他}{8,5}{⼋,⼈}
  \begin{Phonetics}{其他}{qi2ta1}[][HSK 2]
    \definition{pron.}{outra pessoa/outra coisa | outras coisas; outras pessoas; em substituição de outras pessoas ou coisas}
  \end{Phonetics}
\end{Entry}

\begin{Entry}{其后}{8,6}{⼋,⼝}
  \begin{Phonetics}{其后}{qi2hou4}[][HSK 7-9]
    \definition{adv.}{mais tarde; depois; posteriormente | depois disso | próximo}
  \end{Phonetics}
\end{Entry}

\begin{Entry}{其次}{8,6}{⼋,⽋}
  \begin{Phonetics}{其次}{qi2ci4}[][HSK 3]
    \definition{adj.}{secundário}
    \definition{conj.}{próximo; então; em segundo lugar; mais tarde na ordem}
  \end{Phonetics}
\end{Entry}

\begin{Entry}{其余}{8,7}{⼋,⼈}
  \begin{Phonetics}{其余}{qi2yu2}[][HSK 4]
    \definition{pron.}{o resto; os outros; o restante}
  \end{Phonetics}
\end{Entry}

\begin{Entry}{其间}{8,7}{⼋,⾨}
  \begin{Phonetics}{其间}{qi2jian1}[][HSK 7-9]
    \definition{s.}{nele; deles; entre eles; no meio | durante este (ou aquele) período; dentro de um determinado período de tempo}
  \end{Phonetics}
\end{Entry}

\begin{Entry}{其实}{8,8}{⼋,⼧}
  \begin{Phonetics}{其实}{qi2shi2}[][HSK 3]
    \definition{adv.}{na verdade; na realidade; a primeira parte é a situação aparente, e 其实 é usado para introduzir a situação real}
  \end{Phonetics}
\end{Entry}

%%%%%%%%%% 具 %%%%%%%%%%
\subsection*{具}\addcontentsline{loh}{figure}{具}

\begin{Entry}{具}{8}{⼋}
  \begin{Phonetics}{具}{ju4}
    \definition*{s.}{Sobrenome: Ju}
    \definition{clas.}{(literário) usado para caixões, cadáveres e certos objetos}
    \definition{s.}{utensílio; ferramenta; implemento | capacidade; habilidade}
    \definition{v.}{possuir; ter | fornecer; prover | declarar; enumerar}
  \end{Phonetics}
\end{Entry}

\begin{Entry}{具有}{8,6}{⼋,⽉}
  \begin{Phonetics}{具有}{ju4you3}[][HSK 3]
    \definition{v.}{ter; possuir; ser provido de}
  \end{Phonetics}
\end{Entry}

\begin{Entry}{具体}{8,7}{⼋,⼈}
  \begin{Phonetics}{具体}{ju4ti3}[][HSK 3]
    \definition{adj.}{específico; particular | concreto; específico; mais detalhado; muito detalhado; muito claro | concreto; real; não é abstrato, tem uma forma definida; pode ser visto ou sentido}
    \definition{v.}{incorporar; objetivar; combinar teorias, princípios, padrões, etc. com pessoas ou coisas específicas}
  \end{Phonetics}
\end{Entry}

\begin{Entry}{具备}{8,8}{⼋,⼡}
  \begin{Phonetics}{具备}{ju4bei4}[][HSK 4]
    \definition{v.}{ter; possuir; ser provido de}
  \end{Phonetics}
\end{Entry}

%%%%%%%%%% 典 %%%%%%%%%%
\subsection*{典}\addcontentsline{loh}{figure}{典}

\begin{Entry}{典}{8}{⼋}
  \begin{Phonetics}{典}{dian3}
    \definition{s.}{lei; cânone; padrão; sistema; regulamentos | trabalho padrão de bolsa de estudos; livros que podem servir como padrões ou especificações | alusão; citação literária | cerimônia; uma grande cerimônia (nos tempos antigos, a etiqueta era um dos sistemas importantes do estado) | modelo; normas; regras}
    \definition{v.}{estar no comando de | hipotecar; usar imóveis ou casas como garantia ao pedir dinheiro emprestado}
  \end{Phonetics}
\end{Entry}

\begin{Entry}{典礼}{8,5}{⼋,⽰}
  \begin{Phonetics}{典礼}{dian3li3}[][HSK 5]
    \definition[个,次,场]{s.}{cerimônia; celebração; comemoração}
  \end{Phonetics}
\end{Entry}

\begin{Entry}{典型}{8,9}{⼋,⼟}
  \begin{Phonetics}{典型}{dian3xing2}[][HSK 4]
    \definition{adj.}{típico; representativo}
    \definition[个,种]{s.}{modelo; caso típico; indivíduo ou evento representativo | personagens típicos; personalidades modelo (em obras literárias); personagens na literatura e na arte que refletem a natureza de uma determinada sociedade e têm uma personalidade distinta}
  \end{Phonetics}
\end{Entry}

\begin{Entry}{典范}{8,9}{⼋,⾋}
  \begin{Phonetics}{典范}{dian3fan4}[][HSK 7-9]
    \definition{s.}{modelo; exemplo; paradigma; uma pessoa ou coisa que pode ser usada como padrão para aprendizagem ou emulação}
  \end{Phonetics}
\end{Entry}

%%%%%%%%%% 净 %%%%%%%%%%
\subsection*{净}\addcontentsline{loh}{figure}{净}

\begin{Entry}{净}{8}{⼎}
  \begin{Phonetics}{净}{jing4}[][HSK 6]
    \definition{adj.}{limpo | (depois de um verbo) terminado; sem nada sobrando | líquido | vazio; oco; nu}
    \definition{adv.}{todo; o tempo todo | somente; meramente; nada além de | inteiramente; indica puro e nada mais}
    \definition{s.}{o ``rosto pintado'', comumente conhecido como Hualian (花脸), um tipo de personagem da ópera de Pequim, etc.}
    \definition{v.}{tornar limpo | limpar; lavar; esfregar para limpar}
  \seealsoref{花脸}{hua1lian3}
  \end{Phonetics}
\end{Entry}

\begin{Entry}{净化}{8,4}{⼎,⼔}
  \begin{Phonetics}{净化}{jing4hua4}[][HSK 7-9]
    \definition{v.}{purificar; remover impurezas para purificar o objeto}
  \end{Phonetics}
\end{Entry}

%%%%%%%%%% 凭 %%%%%%%%%%
\subsection*{凭}\addcontentsline{loh}{figure}{凭}

\begin{Entry}{凭}{8}{⼏}
  \begin{Phonetics}{凭}{ping2}[][HSK 5]
    \definition{conj.}{não importa (o que, como, etc.); conecta frases complexas condicionais para expressar incondicionalidade, equivalente a 任凭 ou 不论}
    \definition{prep.}{introduzir a ação ou o comportamento com base em algo; quando a frase nominal após 凭 é longa, pode"-se adicionar 着 após 凭}
    \definition[张]{s.}{prova; evidência}
    \definition{v.}{apoiar"-se; encostar"-se | confiar em; depender de | basear"-se em; tomar como base}
  \seealsoref{不论}{bu2lun4}
  \seealsoref{任凭}{ren4 ping2}
  \seealsoref{着}{zhe5}
  \end{Phonetics}
\end{Entry}

\begin{Entry}{凭证}{8,7}{⼏,⾔}
  \begin{Phonetics}{凭证}{ping2zheng4}[][HSK 7-9]
    \definition{s.}{prova; certificado; comprovante; evidência}
  \end{Phonetics}
\end{Entry}

\begin{Entry}{凭借}{8,10}{⼏,⼈}
  \begin{Phonetics}{凭借}{ping2jie4}[][HSK 7-9]
    \definition{v.}{confiar em; depender de}
  \end{Phonetics}
\end{Entry}

\begin{Entry}{凭着}{8,11}{⼏,⽬}
  \begin{Phonetics}{凭着}{ping2zhe5}[][HSK 7-9]
    \definition{prep.}{em virtude de; se baseia em}[她凭着多年的经验做事。===Ela se baseia em seus anos de experiência para realizar as coisas.]
  \end{Phonetics}
\end{Entry}

%%%%%%%%%% 凯 %%%%%%%%%%
\subsection*{凯}\addcontentsline{loh}{figure}{凯}

\begin{Entry}{凯}{8}{⼏}
  \begin{Phonetics}{凯}{kai3}
    \definition*{s.}{Sobrenome: Kai}
    \definition{adj.}{vitorioso; triunfante}
    \definition{s.}{canção da vitória; canção triunfal | vitória; triunfo}
  \end{Phonetics}
\end{Entry}

\begin{Entry}{凯歌}{8,14}{⼏,⽋}
  \begin{Phonetics}{凯歌}{kai3ge1}[][HSK 7-9]
    \definition{s.}{canção de triunfo (ou vitória); hino; canções cantadas após uma vitória}
  \end{Phonetics}
\end{Entry}

%%%%%%%%%% 函 %%%%%%%%%%
\subsection*{函}\addcontentsline{loh}{figure}{函}

\begin{Entry}{函}{8}{⼐}
  \begin{Phonetics}{函}{han2}
    \definition*{s.}{Sobrenome: Han}
    \definition[封]{s.}{caixa; envelope; capa | carta}
  \end{Phonetics}
\end{Entry}

\begin{Entry}{函授}{8,11}{⼐,⼿}
  \begin{Phonetics}{函授}{han2shou4}[][HSK 7-9]
    \definition{v.}{ensinar por correspondência; utilizar principalmente tutoria por correspondência para ministrar cursos}
  \end{Phonetics}
\end{Entry}

\begin{Entry}{函数}{8,13}{⼐,⽁}
  \begin{Phonetics}{函数}{han2shu4}
    \definition[个]{s.}{Matemática: função; em um determinado processo, duas variáveis $x$ e $y$ têm um certo valor de $y$ correspondente a cada valor de $x$ dentro de um determinado intervalo, $y$ é uma função de $x$; essa relação geralmente é expressa como $y = f(x)$}
  \end{Phonetics}
\end{Entry}

%%%%%%%%%% 刮 %%%%%%%%%%
\subsection*{刮}\addcontentsline{loh}{figure}{刮}

\begin{Entry}{刮}{8}{⼑}
  \begin{Phonetics}{刮}{gua1}[][HSK 6]
    \definition{v.}{barbear; raspar; depilar | untar com (pasta, etc.)  | extorquir; pilhar; adquirir avidamente (propriedade) por vários meios | (do vento) soprar}
  \end{Phonetics}
\end{Entry}

\begin{Entry}{刮风}{8,4}{⼑,⾵}
  \begin{Phonetics}{刮风}{gua1/feng1}[][HSK 7-9]
    \definition{v.+compl.}{ventar; fazer vento; soprar (vento)}
  \end{Phonetics}
\end{Entry}

%%%%%%%%%% 到 %%%%%%%%%%
\subsection*{到}\addcontentsline{loh}{figure}{到}

\begin{Entry}{到}{8}{⼑}
  \begin{Phonetics}{到}{dao4}[][HSK 1]
    \definition*{s.}{Sobrenome: Dao}
    \definition{adj.}{atencioso}
    \definition{prep.}{a; até; para; indica o tempo em que a ação ou comportamento foi alcançado}
    \definition{v.}{ir para; partir para | chegar; alcançar; chegar a | como complemento de um verbo para mostrar o resultado de uma ação}
  \end{Phonetics}
\end{Entry}

\begin{Entry}{到处}{8,5}{⼑,⼡}
  \begin{Phonetics}{到处}{dao4chu4}[][HSK 2]
    \definition{adv.}{em todos os lugares; em todos os locais; por toda parte}
  \end{Phonetics}
\end{Entry}

\begin{Entry}{到头来}{8,5,7}{⼑,⼤,⽊}
  \begin{Phonetics}{到头来}{dao4tou2lai2}[][HSK 7-9]
    \definition{adv.}{finalmente; no fim; no final; resultado (usado principalmente em aspectos ruins)}
  \end{Phonetics}
\end{Entry}

\begin{Entry}{到达}{8,6}{⼑,⾡}
  \begin{Phonetics}{到达}{dao4da2}[][HSK 3]
    \definition{v.}{chegar (a um determinado local, a uma determinada fase); alcançar}
  \end{Phonetics}
\end{Entry}

\begin{Entry}{到位}{8,7}{⼑,⼈}
  \begin{Phonetics}{到位}{dao4/wei4}[][HSK 7-9]
    \definition{adj.}{bom; muito preciso; até um nível adequado/padrão/satisfatório}
    \definition{v.+compl.}{estar no lugar/posição; chegar ao local designado; alcançar o local especificado; atender aos requisitos especificados}
  \end{Phonetics}
\end{Entry}

\begin{Entry}{到来}{8,7}{⼑,⽊}
  \begin{Phonetics}{到来}{dao4lai2}[][HSK 5]
    \definition{v.}{chegar; chegar aqui de outro lugar}
  \end{Phonetics}
\end{Entry}

\begin{Entry}{到底}{8,8}{⼑,⼴}
  \begin{Phonetics}{到底}{dao4di3}[][HSK 3]
    \definition{adv.}{na terra (usado em frases interrogativas para expressar a determinação de alguém em encontrar uma resposta definitiva) | afinal | finalmente; por fim; no fim; indica uma situação que finalmente se concretizou após várias mudanças ou reviravoltas}
  \end{Phonetics}
\end{Entry}

\begin{Entry}{到期}{8,12}{⼑,⽉}
  \begin{Phonetics}{到期}{dao4/qi1}[][HSK 6]
    \definition{v.+compl.}{expirar; amadurecer; tornar"-se devido; tornar"-se devido}
  \end{Phonetics}
\end{Entry}

%%%%%%%%%% 制 %%%%%%%%%%
\subsection*{制}\addcontentsline{loh}{figure}{制}

\begin{Entry}{制}{8}{⼑}
  \begin{Phonetics}{制}{zhi4}
    \definition[套,项]{s.}{sistema | regras; regulamentos}
    \definition{v.}{formular; elaborar | fazer; fabricar | restringir; limitar; controlar; disciplinar}
  \end{Phonetics}
\end{Entry}

\begin{Entry}{制订}{8,4}{⼑,⾔}
  \begin{Phonetics}{制订}{zhi4ding4}[][HSK 4]
    \definition{v.}{esboçar; formular; elaborar; mapear}
  \end{Phonetics}
\end{Entry}

\begin{Entry}{制成}{8,6}{⼑,⼽}
  \begin{Phonetics}{制成}{zhi4cheng2}[][HSK 5]
    \definition{v.}{fabricar; ser feito de; produzir}
  \end{Phonetics}
\end{Entry}

\begin{Entry}{制约}{8,6}{⼑,⽷}
  \begin{Phonetics}{制约}{zhi4yue1}[][HSK 5]
    \definition{v.}{limitar; verificar; restringir; a existência e a mudança de uma coisa determinam a existência e a mudança de outra coisa}
  \end{Phonetics}
\end{Entry}

\begin{Entry}{制作}{8,7}{⼑,⼈}
  \begin{Phonetics}{制作}{zhi4zuo4}[][HSK 3]
    \definition{v.}{fazer; produzir; itens feitos com matérias-primas, geralmente pequenos e feitos à mão | fazer; produzir; criar gráficos, anúncios, filmes, jogos, etc., utilizando texto, imagens, sons, imagens, etc.}
  \end{Phonetics}
\end{Entry}

\begin{Entry}{制定}{8,8}{⼑,⼧}
  \begin{Phonetics}{制定}{zhi4ding4}[][HSK 3]
    \definition{v.}{rascunhar; formular; elaborar; estabelecer (leis, regulamentos, planos, etc.)}
  \end{Phonetics}
\end{Entry}

\begin{Entry}{制度}{8,9}{⼑,⼴}
  \begin{Phonetics}{制度}{zhi4du4}[][HSK 3]
    \definition[项,条,套,种]{s.}{regulamentação; regulamento; procedimentos operacionais ou diretrizes de conduta que todos devem seguir | sistema; o sistema político, econômico e cultural formado sob determinadas condições históricas}
  \end{Phonetics}
\end{Entry}

\begin{Entry}{制造}{8,10}{⼑,⾡}
  \begin{Phonetics}{制造}{zhi4zao4}[][HSK 3]
    \definition{v.}{fazer; produzir; manufaturar; transformar matérias-primas em produtos acabados | criar; agitar; criar artificialmente uma situação ou atmosfera desfavorável}
  \end{Phonetics}
\end{Entry}

\begin{Entry}{制裁}{8,12}{⼑,⾐}
  \begin{Phonetics}{制裁}{zhi4cai2}
    \definition{s.}{punição | sanção (inclusive econômica)}
    \definition{v.}{punir}
  \end{Phonetics}
\end{Entry}

%%%%%%%%%% 刷 %%%%%%%%%%
\subsection*{刷}\addcontentsline{loh}{figure}{刷}

\begin{Entry}{刷}{8}{⼑}
  \begin{Phonetics}{刷}{shua1}[][HSK 4]
    \definition{s.}{escova; pincel | (onomatopéia) farfalhar; descreve o som de uma passagem rápida}
    \definition{v.}{escovar; esfregar; remover com uma escova | borrar; colar; aplicar com um pincel | eliminar; remover; limpar | rolar; navegar; visualizar grandes quantidades de informações muito rapidamente em um curto período de tempo online ou em dispositivos móveis | deslizar (passar o cartão magnético)}
  \end{Phonetics}
  \begin{Phonetics}{刷}{shua4}
    \definition{adj.}{pálido ou branco-azulado}
    \definition{adv.}{bastante; completamente; extremamente; descreve movimentos ágeis}
  \end{Phonetics}
\end{Entry}

\begin{Entry}{刷子}{8,3}{⼑,⼦}
  \begin{Phonetics}{刷子}{shua1zi5}[][HSK 4]
    \definition[把,个]{s.}{escova; escovão; utensílio feito de lã, fio de plástico, fio de metal, etc., para remover sujeira ou aplicar óleo de unção, etc., geralmente longo ou oval, alguns com alças}
  \end{Phonetics}
\end{Entry}

\begin{Entry}{刷牙}{8,4}{⼑,⽛}
  \begin{Phonetics}{刷牙}{shua1/ya2}[][HSK 4]
    \definition{v.+compl.}{escovar os dentes}
  \end{Phonetics}
\end{Entry}

\begin{Entry}{刷新}{8,13}{⼑,⽄}
  \begin{Phonetics}{刷新}{shua1xin1}[][HSK 7-9]
    \definition{v.}{renovar; reformar; atualizar; criar novo | atualizar (um site, uma página da \emph{web}) | quebrar (um recorde); essa metáfora descreve a substituição de conquistas existentes por novas e melhores}
  \synonymref{改革}{gai3ge2}
  \synonymref{改进}{gai3jin4}
  \synonymref{改良}{gai3liang2}
  \synonymref{改善}{gai3shan4}
  \synonymref{改正}{gai3zheng4}
  \synonymref{革新}{ge2xin1}
  \end{Phonetics}
\end{Entry}

%%%%%%%%%% 券 %%%%%%%%%%
\subsection*{券}\addcontentsline{loh}{figure}{券}

\begin{Entry}{券}{8}{⼑}
  \begin{Phonetics}{券}{quan4}[][HSK 6]
    \definition[张]{s.}{certificado; bilhete; ingresso; uma conta ou pedaço de papel que serve como recibo}
  \end{Phonetics}
\end{Entry}

%%%%%%%%%% 刹 %%%%%%%%%%
\subsection*{刹}\addcontentsline{loh}{figure}{刹}

\begin{Entry}{刹}{8}{⼑}
  \begin{Phonetics}{刹}{cha4}
    \definition*{s.}{abreviação de 刹多罗 (Kshatara), sânscrito ``ksetra''}
    \definition{s.}{mosteiro, templo ou santuário budista}
  \seealsoref{刹多罗}{sha1duo1luo2}
  \end{Phonetics}
  \begin{Phonetics}{刹}{sha1}
    \definition{v.}{acionar o(s) freio(s); frear; brecar}
  \end{Phonetics}
\end{Entry}

\begin{Entry}{刹车}{8,4}{⼑,⾞}
  \begin{Phonetics}{刹车}{sha1/che1}[][HSK 7-9]
    \definition{s.}{freios; o mecanismo que impede o veículo de se mover}
    \definition{v.+compl.}{frear; pisar nos freios; utilizar os freios ou outros mecanismos para parar o movimento do veículo ou interromper o funcionamento da máquina | desligar uma máquina; parar uma máquina cortando a energia | desligue ou desconectar a fonte de alimentação para interromper o funcionamento da máquina | interromper (um projeto, etc.); uma metáfora para interromper algo imediatamente}
  \end{Phonetics}
\end{Entry}

\begin{Entry}{刹多罗}{8,6,8}{⼑,⼣,⽹}
  \begin{Phonetics}{刹多罗}{sha1duo1luo2}
    \definition*{s.}{Kshatara, sânscrito ``ksetra''}
  \end{Phonetics}
\end{Entry}

%%%%%%%%%% 刺 %%%%%%%%%%
\subsection*{刺}\addcontentsline{loh}{figure}{刺}

\begin{Entry}{刺}{8}{⼑}
  \begin{Phonetics}{刺}{ci1}
    \definition{s.}{(onomatopéia) som de rasgo, fricção, etc.}
  \end{Phonetics}
  \begin{Phonetics}{刺}{ci4}[][HSK 4]
    \definition*{s.}{Sobrenome: Ci}
    \definition{s.}{espinho; farpa; algo afiado como uma agulha | cartão de visita | saliências; projeções pequenas e pontiagudas na superfície de um objeto ou na pele de uma pessoa}
    \definition{v.}{esfaquear; perfurar | irritar; estimular | assassinar | espionar; detectar | criticar}
  \end{Phonetics}
\end{Entry}

\begin{Entry}{刺耳}{8,6}{⼑,⽿}
  \begin{Phonetics}{刺耳}{ci4'er3}[][HSK 7-9]
    \definition{adj.}{irritante (desagradável) para o ouvido; estridente; penetrante; áspero}
  \end{Phonetics}
\end{Entry}

\begin{Entry}{刺骨}{8,9}{⼑,⾻}
  \begin{Phonetics}{刺骨}{ci4gu3}[][HSK 7-9]
    \definition{adj.}{perfurante (até os ossos); cortante}
  \end{Phonetics}
\end{Entry}

\begin{Entry}{刺绣}{8,10}{⼑,⽷}
  \begin{Phonetics}{刺绣}{ci4xiu4}[][HSK 7-9]
    \definition{s.}{bordado; um artesanato popular tradicional que usa fios de seda coloridos para bordar padrões ou imagens em tecidos; produtos bordados}
    \definition{v.}{bordar; bordar padrões ou imagens em tecido usando fios de seda coloridos}
  \end{Phonetics}
\end{Entry}

\begin{Entry}{刺猬}{8,12}{⼑,⽝}
  \begin{Phonetics}{刺猬}{ci4wei5}
    \definition{s.}{porco-espinho | ouriço}
  \end{Phonetics}
\end{Entry}

\begin{Entry}{刺激}{8,16}{⼑,⽔}
  \begin{Phonetics}{刺激}{ci4ji1}[][HSK 4]
    \definition{adj.}{animado; entusiasmado; sensação de empolgação e nervosismo}
    \definition[个]{s.}{estímulo; estimulação; fortes efeitos físicos ou psicológicos}
    \definition{v.}{irritar; provocar; estimular | incentivar; estimular; incitar; (por algum meio) para mudar as coisas para melhor, para fazer coisas positivas}
  \end{Phonetics}
\end{Entry}

%%%%%%%%%% 刻 %%%%%%%%%%
\subsection*{刻}\addcontentsline{loh}{figure}{刻}

\begin{Entry}{刻}{8}{⼑}
  \begin{Phonetics}{刻}{ke4}[][HSK 2,5]
    \definition{adj.}{cruel; severo; rude; indelicado | no mais alto grau}
    \definition{clas.}{um quarto (de uma hora, 15min)}
    \definition[件]{s.}{quarto (de hora); momento}
    \definition{v.}{esculpir; inscrever; gravar; talhar com uma faca (padrões, texto, etc.) | definir um limite de tempo | imprimir (CD)}
  \end{Phonetics}
\end{Entry}

\begin{Entry}{刻舟求剑}{8,6,7,9}{⼑,⾈,⽔,⼑}
  \begin{Phonetics}{刻舟求剑}{ke4zhou1-qiu2jian4}[][HSK 7-9]
    \definition{expr.}{``Marcando o barco para encontrar a espada.''; um entalhe na lateral de um barco para localizar uma espada que caiu ao mar; Figurativo: uma ação que se torna inútil devido a circunstâncias alteradas; tomar medidas sem levar em conta mudanças nas circunstâncias; ``Um homem do estado de Chu deixou cair sua espada no rio enquanto o atravessava. Ele marcou o local onde a espada havia caído na lateral do barco. Quando o barco parou, ele entrou na água a partir do ponto marcado para procurar sua espada, mas, naturalmente, não a encontrou.'' de Lüshi Chunqiu (吕氏春秋), Observando o Presente (察今)}
  \end{Phonetics}
\end{Entry}

\begin{Entry}{刻画}{8,8}{⼑,⽥}
  \begin{Phonetics}{刻画}{ke4hua4}
    \definition{v.}{retratar | tirar um retrato}
  \end{Phonetics}
\end{Entry}

\begin{Entry}{刻苦}{8,8}{⼑,⾋}
  \begin{Phonetics}{刻苦}{ke4ku3}[][HSK 7-9]
    \definition{adj.}{assíduo; trabalhador; meticuloso; diligente e trabalhador, capaz de se dedicar ao trabalho árduo | simples e econômico}
  \end{Phonetics}
\end{Entry}

\begin{Entry}{刻钟}{8,9}{⼑,⾦}
  \begin{Phonetics}{刻钟}{ke4 zhong1}
    \definition{s.}{um quarto de hora}
  \end{Phonetics}
\end{Entry}

\begin{Entry}{刻意}{8,13}{⼑,⼼}
  \begin{Phonetics}{刻意}{ke4yi4}[][HSK 7-9]
    \definition{adv.}{diligentemente; assiduamente; fazendo tudo o que se pode; estar completamente absorto; dedicar"-se inteiramente a algo; isso enfatiza ações tomadas para atrair a atenção dos outros}
  \end{Phonetics}
\end{Entry}

%%%%%%%%%% 剂 %%%%%%%%%%
\subsection*{剂}\addcontentsline{loh}{figure}{剂}

\begin{Entry}{剂}{8}{⼑}
  \begin{Phonetics}{剂}{ji4}[][HSK 7-9]
    \definition*{s.}{Sobrenome: Ji}
    \definition{clas.}{dose}[一剂中药===uma dose de medicina chinesa]
    \definition{s.}{preparação (farmacêutica ou outra química) | pequeno pedaço de massa | agente; certas substâncias químicas}
    \definition{v.}{ajustar; regular}
  \end{Phonetics}
\end{Entry}

%%%%%%%%%% 势 %%%%%%%%%%
\subsection*{势}\addcontentsline{loh}{figure}{势}

\begin{Entry}{势}{8}{⼒}
  \begin{Phonetics}{势}{shi4}
    \definition{s.}{poder; força; influência | momentum; tendência | aparência externa de um objeto natural; fenômenos ou situações naturais | situação; estado de coisas; circunstâncias | sinal; gesto | genitais masculinos}
  \end{Phonetics}
\end{Entry}

\begin{Entry}{势力}{8,2}{⼒,⼒}
  \begin{Phonetics}{势力}{shi4li5}[][HSK 5]
    \definition[股]{s.}{força; poder; influência; forças políticas, econômicas, militares, etc.}
  \end{Phonetics}
\end{Entry}

\begin{Entry}{势不可当}{8,4,5,6}{⼒,⼀,⼝,⼹}
  \begin{Phonetics}{势不可当}{shi4bu4ke3dang1}[][HSK 7-9]
    \definition{expr.}{impossível de resistir (expressão idiomática); uma força irresistível | irresistível; implacável; imparável}
  \end{Phonetics}
\end{Entry}

\begin{Entry}{势头}{8,5}{⼒,⼤}
  \begin{Phonetics}{势头}{shi4tou5}[][HSK 7-9]
    \definition{s.}{ímpeto; impulso; momento; o estado das coisas; a situação}
  \end{Phonetics}
\end{Entry}

\begin{Entry}{势必}{8,5}{⼒,⼼}
  \begin{Phonetics}{势必}{shi4bi4}[][HSK 7-9]
    \definition{adv.}{inevitavelmente; certamente irá; estará obrigado a; isso indica que, com base na situação, infere"-se que uma determinada situação ocorrerá inevitavelmente}
  \synonymref{必然}{bi4ran2}
  \antonymref{也许}{ye3xu3}
  \end{Phonetics}
\end{Entry}

%%%%%%%%%% 卑 %%%%%%%%%%
\subsection*{卑}\addcontentsline{loh}{figure}{卑}

\begin{Entry}{卑}{8}{⼗}
  \begin{Phonetics}{卑}{bei1}
    \definition{adj.}{Literário: baixo | inferior; médio | Literário: modesto; humilde}
  \end{Phonetics}
\end{Entry}

\begin{Entry}{卑鄙}{8,13}{⼗,⾢}
  \begin{Phonetics}{卑鄙}{bei1bi3}[][HSK 7-9]
    \definition{adj.}{ruim; vulgar; vil; desprezível}
  \end{Phonetics}
\end{Entry}

%%%%%%%%%% 卒 %%%%%%%%%%
\subsection*{卒}\addcontentsline{loh}{figure}{卒}

\begin{Entry}{卒}{8}{⼗}
  \begin{Phonetics}{卒}{cu4}
    \variantof{猝}
  \end{Phonetics}
  \begin{Phonetics}{卒}{zu2}
    \definition{adv.}{finalmente; enfim}
    \definition{s.}{Obsoleto: soldado; recruta | Obsoleto: servo | peão (uma das peças do xadrez chinês)}
    \definition{v.}{Literário: terminar; finalizar | morrer}
  \end{Phonetics}
\end{Entry}

%%%%%%%%%% 单 %%%%%%%%%%
\subsection*{单}\addcontentsline{loh}{figure}{单}

\begin{Entry}{单}{8}{⼗}
  \begin{Phonetics}{单}{chan2}
    \definition{s.}{usado em 单于 \dpy{chan2yu2}}
  \seealsoref{单于}{chan2yu2}
  \end{Phonetics}
  \begin{Phonetics}{单}{dan1}[][HSK 4]
    \definition*{s.}{Sobrenome: Dan}
    \definition{adj.}{sozinho; único | ímpar; número ímpar | simples; poucos projetos e tipos; estrutura e ideias simples | fino; fraco; frágil}
    \definition{adv.}{isoladamente; sozinho; indica que uma ação ou coisa está dentro de um escopo limitado e não é combinada com outras; equivale a 只 ou 仅}
    \definition[个]{s.}{lençol; um único pedaço grande de pano usado para cobrir | conta; lista; pedaços de papel para anotações detalhadas (geralmente folhas soltas)}
  \seealsoref{仅}{jin3}
  \seealsoref{只}{zhi3}
  \antonymref{双}{shuang1}
  \end{Phonetics}
  \begin{Phonetics}{单}{shan4}
    \definition*{s.}{Sobrenome: Shan}
    \definition{s.}{material de tecido de largura simples (dupla) | número singular (plural)}
  \end{Phonetics}
\end{Entry}

\begin{Entry}{单一}{8,1}{⼗,⼀}
  \begin{Phonetics}{单一}{dan1yi1}[][HSK 5]
    \definition{adj.}{único; unitário; exclusivo}
  \end{Phonetics}
\end{Entry}

\begin{Entry}{单于}{8,3}{⼗,⼆}
  \begin{Phonetics}{单于}{chan2yu2}
    \definition{s.}{rei de Xiongnu (匈奴)}
  \seealsoref{匈奴}{xiong1nu2}
  \end{Phonetics}
\end{Entry}

\begin{Entry}{单元}{8,4}{⼗,⼉}
  \begin{Phonetics}{单元}{dan1yuan2}[][HSK 3]
    \definition[个,组,套]{s.}{unidade (de algo); um conjunto completo, com parágrafos e sistemas próprios, que forma uma unidade independente}
  \end{Phonetics}
\end{Entry}

\begin{Entry}{单方面}{8,4,9}{⼗,⽅,⾯}
  \begin{Phonetics}{单方面}{dan1fang1mian4}[][HSK 7-9]
    \definition{adj.}{unilateral}
    \definition{adv.}{unilateralmente}
  \end{Phonetics}
\end{Entry}

\begin{Entry}{单打}{8,5}{⼗,⼿}
  \begin{Phonetics}{单打}{dan1da3}[][HSK 6]
    \definition[场,局,次]{s.}{Esporte: simples; competição um contra um}
  \end{Phonetics}
\end{Entry}

\begin{Entry}{单边}{8,5}{⼗,⾡}
  \begin{Phonetics}{单边}{dan1bian1}[][HSK 7-9]
    \definition{adj.}{unilateral}
  \end{Phonetics}
\end{Entry}

\begin{Entry}{单向}{8,6}{⼗,⼝}
  \begin{Phonetics}{单向}{dan1xiang4}
    \definition{adj.}{de mão única; unidirecional (oposto de 双向)}
  \seealsoref{双向}{shuang1xiang4}
  \end{Phonetics}
\end{Entry}

\begin{Entry}{单位}{8,7}{⼗,⼈}
  \begin{Phonetics}{单位}{dan1wei4}[][HSK 2]
    \definition[个,家]{s.}{unidade (como padrão de medida) | unidade (como uma organização, departamento, divisão, seção, etc.) | unidade (grupo de pessoas como um todo) | unidade de trabalho (local de trabalho, especialmente na República Popular da China antes da reforma econômica)}
  \end{Phonetics}
\end{Entry}

\begin{Entry}{单纯}{8,7}{⼗,⽷}
  \begin{Phonetics}{单纯}{dan1chun2}[][HSK 4]
    \definition{adj.}{puro; simples; descomplicado}
    \definition{adv.}{sozinho; puramente; meramente}
  \end{Phonetics}
\end{Entry}

\begin{Entry}{单身}{8,7}{⼗,⾝}
  \begin{Phonetics}{单身}{dan1shen1}[][HSK 7-9]
    \definition{s.}{solteiro}
  \end{Phonetics}
\end{Entry}

\begin{Entry}{单单}{8,8}{⼗,⼗}
  \begin{Phonetics}{单单}{dan1dan1}
    \definition{adv.}{somente; sozinho; exceto; indica a identificação de um indivíduo dentro de um grupo geral de pessoas ou coisas}[别人都去了,单单她没去。===Todos os outros foram, somente ela não foi.]
  \end{Phonetics}
\end{Entry}

\begin{Entry}{单质}{8,8}{⼗,⾙}
  \begin{Phonetics}{单质}{dan1zhi4}
    \definition{s.}{substância simples (consistindo puramente de um elemento, como diamante, grafite, etc.)}
  \end{Phonetics}
\end{Entry}

\begin{Entry}{单独}{8,9}{⼗,⽝}
  \begin{Phonetics}{单独}{dan1du2}[][HSK 4]
    \definition{adv.}{solo; sozinho; por si mesmo; por conta própria}
  \end{Phonetics}
\end{Entry}

\begin{Entry}{单调}{8,10}{⼗,⾔}
  \begin{Phonetics}{单调}{dan1diao4}[][HSK 4]
    \definition{adj.}{maçante; monótono}
  \end{Phonetics}
\end{Entry}

\begin{Entry}{单脚滑行车}{8,11,12,6,4}{⼗,⾁,⽔,⾏,⾞}
  \begin{Phonetics}{单脚滑行车}{dan1jiao3hua2xing2che1}
    \definition{s.}{\emph{scooter}}
  \end{Phonetics}
\end{Entry}

\begin{Entry}{单薄}{8,16}{⼗,⾋}
  \begin{Phonetics}{单薄}{dan1bo2}[][HSK 7-9]
    \definition{adj.}{fino; pouco | frágil; magro e fraco | fino; frágil; insubstancial}
  \end{Phonetics}
\end{Entry}

%%%%%%%%%% 卖 %%%%%%%%%%
\subsection*{卖}\addcontentsline{loh}{figure}{卖}

\begin{Entry}{卖}{8}{⼗}
  \begin{Phonetics}{卖}{mai4}[][HSK 2]
    \definition*{s.}{Sobrenome: Mai}
    \definition{clas.}{um prato (nos tempos antigos); antigamente, os restaurantes chamavam cada prato vendido de 一卖 (uma porção)}
    \definition{v.}{vender | trair (o próprio país ou amigos); alcançar objetivos pessoais à custa dos interesses do país, da nação e dos outros | não poupar esforços; esforçar"-se ao máximo; tentar fazer o máximo possível | mostrar"-se intencionalmente; exibir"-se | vender o próprio trabalho; trabalhar em troca de dinheiro}
  \antonymref{买}{mai3}
  \end{Phonetics}
\end{Entry}

\begin{Entry}{卖弄}{8,7}{⼗,⼶}
  \begin{Phonetics}{卖弄}{mai4nong5}[][HSK 7-9]
    \definition{v.}{exibir"-se; desfilar; exibir ou ostentar intencionalmente (as próprias habilidades)}
  \end{Phonetics}
\end{Entry}

%%%%%%%%%% 卧 %%%%%%%%%%
\subsection*{卧}\addcontentsline{loh}{figure}{卧}

\begin{Entry}{卧}{8}{⾂}
  \begin{Phonetics}{卧}{wo4}[][HSK 7-9]
    \definition{adj.}{para dormir}
    \definition{s.}{vagão-leito (ou carruagem); leito | beliche | quarto | Dialeto: pochê (ovos)}
    \definition{v.}{deitar | Dialeto: deitar um bebê | (animais ou pássaros) agachar"-se; sentar"-se; empoleirar"-se | Figurativo: viver em reclusão}
  \antonymref{坐}{zuo4}
  \end{Phonetics}
\end{Entry}

\begin{Entry}{卧车}{8,4}{⾂,⾞}
  \begin{Phonetics}{卧车}{wo4che1}
    \definition{s.}{um carro-leito | vagão-leito}
  \end{Phonetics}
\end{Entry}

\begin{Entry}{卧式}{8,6}{⾂,⼷}
  \begin{Phonetics}{卧式}{wo4shi4}
    \definition{adj.}{horizontal}
  \end{Phonetics}
\end{Entry}

\begin{Entry}{卧床}{8,7}{⾂,⼴}
  \begin{Phonetics}{卧床}{wo4chuang2}
    \definition{adj.}{acamado}
    \definition{s.}{cama}
    \definition{v.}{deitar na cama}
  \end{Phonetics}
\end{Entry}

\begin{Entry}{卧室}{8,9}{⾂,⼧}
  \begin{Phonetics}{卧室}{wo4shi4}[][HSK 5]
    \definition[间,个]{s.}{quarto de dormir; quarto de uma casa usado para dormir}
  \end{Phonetics}
\end{Entry}

\begin{Entry}{卧倒}{8,10}{⾂,⼈}
  \begin{Phonetics}{卧倒}{wo4dao3}
    \definition{v.}{cair no chão | deitar-se}
  \end{Phonetics}
\end{Entry}

\begin{Entry}{卧病}{8,10}{⾂,⽧}
  \begin{Phonetics}{卧病}{wo4bing4}
    \definition{s.}{acamado | doente na cama}
  \end{Phonetics}
\end{Entry}

\begin{Entry}{卧舱}{8,10}{⾂,⾈}
  \begin{Phonetics}{卧舱}{wo4cang1}
    \definition{s.}{cabine de dormir em um barco ou trem}
  \end{Phonetics}
\end{Entry}

\begin{Entry}{卧推}{8,11}{⾂,⼿}
  \begin{Phonetics}{卧推}{wo4tui1}
    \definition{s.}{supino}
  \end{Phonetics}
\end{Entry}

\begin{Entry}{卧铺}{8,12}{⾂,⾦}
  \begin{Phonetics}{卧铺}{wo4pu4}[][HSK 6]
    \definition[个,排]{s.}{beliche para dormir; um beliche em um trem ou ônibus de longa distância}
  \end{Phonetics}
\end{Entry}

\begin{Entry}{卧榻}{8,14}{⾂,⽊}
  \begin{Phonetics}{卧榻}{wo4ta4}
    \definition{s.}{um sofá | uma cama estreita}
  \end{Phonetics}
\end{Entry}

%%%%%%%%%% 卷 %%%%%%%%%%
\subsection*{卷}\addcontentsline{loh}{figure}{卷}

\begin{Entry}{卷}{8}{⼙}
  \begin{Phonetics}{卷}{juan3}[][HSK 4]
    \definition{clas.}{usado para pequenas coisas enroladas (maço de papel dinheiro, carretel de filme, etc.) | usado para rolos, carretéis, bobinas, etc.}
    \definition[张]{s.}{rolo; carretel; bobina}
    \definition{v.}{enrolar; dobrar algo em um cilindro ou semicírculo | varrer; carregar; levar junto | envolver"-se; participar}
  \end{Phonetics}
  \begin{Phonetics}{卷}{juan4}[][HSK 4]
    \definition{clas.}{usado para capítulos, seções ou volumes; fascículos}
    \definition[大,小]{s.}{livro; livros e pinturas que são enrolados para coleção; geralmente se refere a pinturas e caligrafia | papel de exame | arquivo; dossiê}
  \end{Phonetics}
\end{Entry}

\begin{Entry}{卷入}{8,2}{⼙,⼊}
  \begin{Phonetics}{卷入}{juan3ru4}[][HSK 7-9]
    \definition{v.}{ser envolvido; estar envolvido; ser atraído para; estar envolvido em}
  \end{Phonetics}
\end{Entry}

\begin{Entry}{卷子}{8,3}{⼙,⼦}
  \begin{Phonetics}{卷子}{juan3zi5}
    \definition{s.}{rolinho primavera; um tipo de prato de massa, feito amassando em folhas finas, cobrindo um lado com óleo e sal, enrolando"-a e cozinhando"-a no vapor}
  \end{Phonetics}
  \begin{Phonetics}{卷子}{juan4zi5}[][HSK 7-9]
    \definition{s.}{prova; prova de exame; um caderno fino ou uma única folha de papel para anotar as respostas durante as provas}
  \end{Phonetics}
\end{Entry}

%%%%%%%%%% 厕 %%%%%%%%%%
\subsection*{厕}\addcontentsline{loh}{figure}{厕}

\begin{Entry}{厕}{8}{⼚}
  \begin{Phonetics}{厕}{ce4}
    \definition[个,间]{s.}{latrina; fossa sanitária; (componente formador de palavras)}
  \seealsoref{茅厕}{mao2ce4}
  \end{Phonetics}
  \begin{Phonetics}{厕}{si5}
    \definition{s.}{componente formador de palavras | latrina; fossa sanitária}
  \seealsoref{茅厕}{mao2ce4}
  \end{Phonetics}
\end{Entry}

\begin{Entry}{厕纸}{8,7}{⼚,⽷}
  \begin{Phonetics}{厕纸}{ce4zhi3}
    \definition{s.}{papel higiênico}
  \end{Phonetics}
\end{Entry}

\begin{Entry}{厕所}{8,8}{⼚,⼾}
  \begin{Phonetics}{厕所}{ce4suo3}[][HSK 6]
    \definition[个,间]{s.}{banheiro; lavatório; sanitário; latrina; um lugar para as pessoas urinarem e defecarem}
  \end{Phonetics}
\end{Entry}

%%%%%%%%%% 参 %%%%%%%%%%
\subsection*{参}\addcontentsline{loh}{figure}{参}

\begin{Entry}{参}{8}{⼛}
  \begin{Phonetics}{参}{can1}
    \definition{v.}{juntar"-se; entrar; tomar parte em; participar | referir; consultar; comparar com outros materiais | ligar para prestar homenagem a; fazer uma visita |  (significado antigo) acusar um funcionário perante o imperador; relatar ou expor ao imperador | explorar e compreender (verdade, significado, etc.)}
  \end{Phonetics}
\end{Entry}

\begin{Entry}{参与}{8,3}{⼛,⼀}
  \begin{Phonetics}{参与}{can1yu4}[][HSK 4]
    \definition{v.}{participar de; tomar parte em; ter uma mão em; envolver"-se em; participar (no planejamento, discussão e condução dos assuntos)}
  \end{Phonetics}
\end{Entry}

\begin{Entry}{参见}{8,4}{⼛,⾒}
  \begin{Phonetics}{参见}{can1jian4}[][HSK 7-9]
    \definition{v.}{(em referências) ver; ver também | consultar; visualizar | ler algo para referência | prestar homenagem a (um superior, etc.)}
  \end{Phonetics}
\end{Entry}

\begin{Entry}{参加}{8,5}{⼛,⼒}
  \begin{Phonetics}{参加}{can1jia1}[][HSK 2]
    \definition{v.}{aderir (a organizações); participar; participar (de atividades); participar de alguma organização ou atividade | dar (conselho, sugestão, etc.)}
  \end{Phonetics}
\end{Entry}

\begin{Entry}{参军}{8,6}{⼛,⼍}
  \begin{Phonetics}{参军}{can1/jun1}[][HSK 7-9]
    \definition{s.}{oficial do estado"-maior militar; título oficial antigo}
    \definition{v.+compl.}{juntar"-se ao exército; alistar"-se}
  \end{Phonetics}
\end{Entry}

\begin{Entry}{参考}{8,6}{⼛,⽼}
  \begin{Phonetics}{参考}{can1kao3}[][HSK 4]
    \definition{v.}{consultar; referir"-se a; acessar informações relevantes para estudo ou pesquisa | consultar; referir"-se a; lidar com coisas, observar, ler, aprender e usar materiais relevantes}
  \end{Phonetics}
\end{Entry}

\begin{Entry}{参观}{8,6}{⼛,⾒}
  \begin{Phonetics}{参观}{can1guan1}[][HSK 2]
    \definition{v.}{visitar; dar uma olhada; observação no local (resultados do trabalho, carreira, instalações, locais históricos e pontos turísticos, etc.)}
  \end{Phonetics}
\end{Entry}

\begin{Entry}{参展}{8,10}{⼛,⼫}
  \begin{Phonetics}{参展}{can1 zhan3}[][HSK 6]
    \definition{v.}{expor ou participar de uma feira comercial, etc.}
  \end{Phonetics}
\end{Entry}

\begin{Entry}{参谋}{8,11}{⼛,⾔}
  \begin{Phonetics}{参谋}{can1mou2}[][HSK 7-9]
    \definition{s.}{oficial de estado"-maior; pessoal militar envolvido em planejamento militar e outros assuntos | conselheiro; consultor}
    \definition{v.}{aconselhar; dar conselhos}
  \end{Phonetics}
\end{Entry}

\begin{Entry}{参照}{8,13}{⼛,⽕}
  \begin{Phonetics}{参照}{can1zhao4}[][HSK 7-9]
    \definition{v.}{consultar; referir"-se a; referir"-se e imitar (métodos, experiências, etc.)}
  \end{Phonetics}
\end{Entry}

\begin{Entry}{参赛}{8,14}{⼛,⾙}
  \begin{Phonetics}{参赛}{can1 sai4}[][HSK 6]
    \definition{v.}{participar de uma partida (ou competição); competir}
  \end{Phonetics}
\end{Entry}

%%%%%%%%%% 叔 %%%%%%%%%%
\subsection*{叔}\addcontentsline{loh}{figure}{叔}

\begin{Entry}{叔}{8}{⼜}
  \begin{Phonetics}{叔}{shu1}
    \definition*{s.}{Sobrenome: Shu}
    \definition{s.}{irmão mais novo do pai; tio (por parte de pai)| irmão mais novo do marido | terceiro entre irmãos | tio | uma forma de tratamento para um homem um pouco mais jovem que o pai; tio | terceiro tio (de quatro irmãos) | primo mais novo da mãe}
  \end{Phonetics}
\end{Entry}

\begin{Entry}{叔叔}{8,8}{⼜,⼜}
  \begin{Phonetics}{叔叔}{shu1shu5}[][HSK 4]
    \definition[个,位,名]{s.}{tio; irmão mais novo do pai | tio, dirigindo"-se a um homem da mesma geração que o pai e mais jovem em idade}
  \end{Phonetics}
\end{Entry}

%%%%%%%%%% 取 %%%%%%%%%%
\subsection*{取}\addcontentsline{loh}{figure}{取}

\begin{Entry}{取}{8}{⼜}
  \begin{Phonetics}{取}{qu3}[][HSK 2]
    \definition{v.}{pegar; obter; buscar; pegar de um lugar; pegar nas mãos | visar; procurar; obter; provocar | adotar; assumir; escolher; selecionar}
  \end{Phonetics}
\end{Entry}

\begin{Entry}{取水}{8,4}{⼜,⽔}
  \begin{Phonetics}{取水}{qu3shui3}
    \definition{v.}{obter água (de um poço, etc.)}
  \end{Phonetics}
\end{Entry}

\begin{Entry}{取代}{8,5}{⼜,⼈}
  \begin{Phonetics}{取代}{qu3dai4}[][HSK 7-9]
    \definition{v.}{deslocar; substituir; suplantar; substituir por; assumir o controle; tomar o lugar de}
  \end{Phonetics}
\end{Entry}

\begin{Entry}{取决于}{8,6,3}{⼜,⼎,⼆}
  \begin{Phonetics}{取决于}{qu3jue2 yu2}[][HSK 7-9]
    \definition{v.}{depender de; ser determinado por (algo)}
  \end{Phonetics}
\end{Entry}

\begin{Entry}{取而代之}{8,6,5,3}{⼜,⽽,⼈,⼂}
  \begin{Phonetics}{取而代之}{qu3'er2dai4zhi1}[][HSK 7-9]
    \definition{expr.}{substituir alguém; suplantar alguém; tomar o lugar de alguém ou de algo; assumir o controle}
  \end{Phonetics}
\end{Entry}

\begin{Entry}{取现}{8,8}{⼜,⾒}
  \begin{Phonetics}{取现}{qu3xian4}
    \definition{v.}{sacar dinheiro}
  \end{Phonetics}
\end{Entry}

\begin{Entry}{取经}{8,8}{⼜,⽷}
  \begin{Phonetics}{取经}{qu3/jing1}[][HSK 7-9]
    \definition{v.+compl.}{fazer uma peregrinação em busca de escrituras budistas | buscar experiência; aprender com a experiência de outra pessoa}
  \end{Phonetics}
\end{Entry}

\begin{Entry}{取胜}{8,9}{⼜,⾁}
  \begin{Phonetics}{取胜}{qu3sheng4}[][HSK 7-9]
    \definition{v.}{obter a vitória; alcançar o sucesso; alcançar a vitória}
  \end{Phonetics}
\end{Entry}

\begin{Entry}{取悦}{8,10}{⼜,⼼}
  \begin{Phonetics}{取悦}{qu3yue4}
    \definition{v.}{tentar agradar}
  \end{Phonetics}
\end{Entry}

\begin{Entry}{取消}{8,10}{⼜,⽔}
  \begin{Phonetics}{取消}{qu3xiao1}[][HSK 3]
    \definition{v.}{cancelar; suspender; anular; abolir; revogar; rescindir; tornar o sistema original, regulamentos, qualificações, direitos, etc. inválidos}
  \end{Phonetics}
\end{Entry}

\begin{Entry}{取笑}{8,10}{⼜,⽵}
  \begin{Phonetics}{取笑}{qu3xiao4}[][HSK 7-9]
    \definition{v.}{ridicularizar; zombar de; fazer alarde de; buscar diversão}
  \end{Phonetics}
\end{Entry}

\begin{Entry}{取得}{8,11}{⼜,⼻}
  \begin{Phonetics}{取得}{qu3de2}[][HSK 2]
    \definition{v.}{ganhar; adquirir; obter; ser o primeiro a conseguir}
  \end{Phonetics}
\end{Entry}

\begin{Entry}{取款}{8,12}{⼜,⽋}
  \begin{Phonetics}{取款}{qu3/kuan3}[][HSK 6]
    \definition{v.+compl.}{sacar dinheiro (de um banco); retirar o dinheiro que você depositou (geralmente se refere a retirar dinheiro do banco)}
  \end{Phonetics}
\end{Entry}

\begin{Entry}{取款机}{8,12,6}{⼜,⽋,⽊}
  \begin{Phonetics}{取款机}{qu3kuan3ji1}[][HSK 6]
    \definition{s.}{ATM; caixa eletrônico; um caixa eletrônico é uma máquina que pode concluir automaticamente operações bancárias, como saques e consultas de saldo}
  \end{Phonetics}
\end{Entry}

\begin{Entry}{取缔}{8,12}{⼜,⽷}
  \begin{Phonetics}{取缔}{qu3di4}[][HSK 7-9]
    \definition{v.}{proibir; criminalizar; suprimir; cancelar, encerrar ou proibir explicitamente}
  \end{Phonetics}
\end{Entry}

\begin{Entry}{取暖}{8,13}{⼜,⽇}
  \begin{Phonetics}{取暖}{qu3nuan3}[][HSK 7-9]
    \definition{v.}{aquecer"-se; utilizar a energia térmica para aquecer o corpo}
  \end{Phonetics}
\end{Entry}

%%%%%%%%%% 受 %%%%%%%%%%
\subsection*{受}\addcontentsline{loh}{figure}{受}

\begin{Entry}{受}{8}{⼜}
  \begin{Phonetics}{受}{shou4}[][HSK 3]
    \definition{v.}{receber; aceitar | sofrer; ser submetido a | aguentar; suportar; tolerar | ser agradável}
  \end{Phonetics}
\end{Entry}

\begin{Entry}{受不了}{8,4,2}{⼜,⼀,⼅}
  \begin{Phonetics}{受不了}{shou4bu5liao3}[][HSK 4]
    \definition{v.}{ser insuportável; não poder suportar algo; não suportar algo}
  \end{Phonetics}
\end{Entry}

\begin{Entry}{受伤}{8,6}{⼜,⼈}
  \begin{Phonetics}{受伤}{shou4/shang1}[][HSK 3]
    \definition{v.+compl.}{ser ferido; sofrer uma lesão}
  \end{Phonetics}
\end{Entry}

\begin{Entry}{受过}{8,6}{⼜,⾡}
  \begin{Phonetics}{受过}{shou4/guo4}[][HSK 7-9]
    \definition{v.+compl.}{assumir a culpa (por outra pessoa) | sofrer}
  \end{Phonetics}
\end{Entry}

\begin{Entry}{受灾}{8,7}{⼜,⽕}
  \begin{Phonetics}{受灾}{shou4 zai1}[][HSK 5]
    \definition{v.}{ser atingido por um desastre natural (ou calamidade) | ser atingido por uma adversidade natural}
  \end{Phonetics}
\end{Entry}

\begin{Entry}{受到}{8,8}{⼜,⼑}
  \begin{Phonetics}{受到}{shou4dao4}[][HSK 2]
    \definition{v.}{receber; receber itens, mensagens, instruções, etc. fornecidos por outras pessoas}
  \end{Phonetics}
\end{Entry}

\begin{Entry}{受苦}{8,8}{⼜,⾋}
  \begin{Phonetics}{受苦}{shou4/ku3}[][HSK 7-9]
    \definition{v.+compl.}{sofrer (dificuldades); passar por momentos difíceis}
  \synonymref{吃苦}{chi1/ku3}
  \synonymref{刻苦}{ke4ku3}
  \antonymref{舒服}{shu1fu5}
  \end{Phonetics}
\end{Entry}

\begin{Entry}{受限}{8,8}{⼜,⾩}
  \begin{Phonetics}{受限}{shou4xian4}
    \definition{v.}{ser limitado | ser restrito | ser constrangido}
  \end{Phonetics}
\end{Entry}

\begin{Entry}{受害}{8,10}{⼜,⼧}
  \begin{Phonetics}{受害}{shou4/hai4}[][HSK 7-9]
    \definition{v.+compl.}{sofrer lesão; ser vítima de queda; ser afetado por | ser afetado; ser afligido}
  \synonymref{受伤}{shou4/shang1}
  \antonymref{受益}{shou4yi4}
  \end{Phonetics}
\end{Entry}

\begin{Entry}{受害人}{8,10,2}{⼜,⼧,⼈}
  \begin{Phonetics}{受害人}{shou4hai4ren2}[][HSK 7-9]
    \definition{s.}{vítima | sofredor}
  \end{Phonetics}
\end{Entry}

\begin{Entry}{受益}{8,10}{⼜,⽫}
  \begin{Phonetics}{受益}{shou4yi4}[][HSK 7-9]
    \definition{v.}{lucrar com; beneficiar"-se de; ser beneficiado; receber benefícios; obter vantagens}
  \synonymref{利益}{li4yi4}
  \synonymref{收益}{shou1yi4}
  \antonymref{受害}{shou4/hai4}
  \end{Phonetics}
\end{Entry}

\begin{Entry}{受贿}{8,10}{⼜,⾙}
  \begin{Phonetics}{受贿}{shou4/hui4}[][HSK 7-9]
    \definition{v.+compl.}{aceitar (ou receber) subornos}
  \end{Phonetics}
\end{Entry}

\begin{Entry}{受得了}{8,11,2}{⼜,⼻,⼅}
  \begin{Phonetics}{受得了}{shou4de5liao3}
    \definition{v.}{suportar | aguentar}
  \end{Phonetics}
\end{Entry}

\begin{Entry}{受惊}{8,11}{⼜,⼼}
  \begin{Phonetics}{受惊}{shou4/jing1}[][HSK 7-9]
    \definition{adj.}{assustado; sobressaltado; medo causado por estímulos ou ameaças repentinas}
    \definition{v.+compl.}{ficar assustado; levar um susto}
  \synonymref{吃惊}{chi1/jing1}
  \antonymref{坦然}{tan3ran2}
  \end{Phonetics}
\end{Entry}

\begin{Entry}{受理}{8,11}{⼜,⽟}
  \begin{Phonetics}{受理}{shou4li3}[][HSK 7-9]
    \definition{v.}{aceitar e tratar um caso (autoridades judiciais) | aceitar e lidar com; aceitar e processar}
  \synonymref{办理}{ban4li3}
  \synonymref{承办}{cheng2ban4}
  \synonymref{处理}{chu3li3}
  \synonymref{处置}{chu3zhi4}
  \antonymref{驳回}{bo2hui2}
  \end{Phonetics}
\end{Entry}

\begin{Entry}{受骗}{8,12}{⼜,⾺}
  \begin{Phonetics}{受骗}{shou4/pian4}[][HSK 7-9]
    \definition{v.+compl.}{ser enganado; ser iludido; ser ludibriado}
  \synonymref{上当}{shang4/dang4}
  \end{Phonetics}
\end{Entry}

%%%%%%%%%% 变 %%%%%%%%%%
\subsection*{变}\addcontentsline{loh}{figure}{变}

\begin{Entry}{变}{8}{⼜}
  \begin{Phonetics}{变}{bian4}[][HSK 2]
    \definition{adj.}{alterado; mutável; que pode mudar; que está mudando ou já mudou}
    \definition{s.}{uma reviravolta inesperada nos acontecimentos; mudanças significativas repentinas}
    \definition{v.}{mudar; tornar"-se diferente; fazer mudanças | tornar"-se; transformar"-se; natureza, estado ou situação diferentes dos originais | alterar; mudar; transformar}
  \end{Phonetics}
\end{Entry}

\begin{Entry}{变为}{8,4}{⼜,⼂}
  \begin{Phonetics}{变为}{bian4wei2}[][HSK 3]
    \definition{v.}{transformar"-se em; tornar"-se | mudar para}
  \end{Phonetics}
\end{Entry}

\begin{Entry}{变化}{8,4}{⼜,⼔}
  \begin{Phonetics}{变化}{bian4hua4}[][HSK 3]
    \definition[个]{s.}{mudança; variação; a nova situação após uma mudança em pessoas ou coisas}
    \definition{v.}{mudar;  variar}
  \end{Phonetics}
\end{Entry}

\begin{Entry}{变幻莫测}{8,4,10,9}{⼜,⼳,⾋,⽔}
  \begin{Phonetics}{变幻莫测}{bian4huan4-mo4ce4}[][HSK 7-9]
    \definition{expr.}{mutável; imprevisível | errático | mudar imprevisivelmente | traiçoeiro}
  \end{Phonetics}
\end{Entry}

\begin{Entry}{变心}{8,4}{⼜,⼼}
  \begin{Phonetics}{变心}{bian4/xin1}
    \definition{v.+compl.}{deixar de ser fiel}
  \end{Phonetics}
\end{Entry}

\begin{Entry}{变节}{8,5}{⼜,⾋}
  \begin{Phonetics}{变节}{bian4jie2}
    \definition{s.}{traição | deserção | vira"-casaca}
    \definition{v.}{retratar"-se e submeter"-se; renunciar e render"-se | mudar de lado politicamente}
  \end{Phonetics}
\end{Entry}

\begin{Entry}{变动}{8,6}{⼜,⼒}
  \begin{Phonetics}{变动}{bian4dong4}[][HSK 5]
    \definition{s.}{mudança; alteração; oscilação; modificação; variação}
    \definition{v.}{mudar; alterar; oscilar; modificar; variar}
  \end{Phonetics}
\end{Entry}

\begin{Entry}{变异}{8,6}{⼜,⼶}
  \begin{Phonetics}{变异}{bian4yi4}[][HSK 7-9]
    \definition{s.}{variação; mutação; muta; diferenças nas características morfológicas e fisiológicas entre gerações da mesma espécie ou entre indivíduos da mesma geração}
    \definition{v.}{variar; mudar}
  \end{Phonetics}
\end{Entry}

\begin{Entry}{变成}{8,6}{⼜,⼽}
  \begin{Phonetics}{变成}{bian4 cheng2}[][HSK 2]
    \definition{v.}{crescer; tornar"-se; fazer; desenvolver"-se; revelar"-se; resultar; acontecer; passar a ser; passar para; acumular"-se; converter"-se; transformar"-se; transformar"-se em; mudar"-se em; transformação da situação ou condição anterior para a situação ou condição atual}
  \end{Phonetics}
\end{Entry}

\begin{Entry}{变迁}{8,6}{⼜,⾡}
  \begin{Phonetics}{变迁}{bian4qian1}[][HSK 7-9]
    \definition{s.}{mudanças; transição; vicissitudes; mudança em tendências ou condições; mudança de situação ou estágio}
  \end{Phonetics}
\end{Entry}

\begin{Entry}{变形}{8,7}{⼜,⼺}
  \begin{Phonetics}{变形}{bian4/xing2}[][HSK 6]
    \definition{v.+compl.}{deformar; ficar fora de forma | transformar; transformar"-se em outras formas}
  \end{Phonetics}
\end{Entry}

\begin{Entry}{变更}{8,7}{⼜,⽈}
  \begin{Phonetics}{变更}{bian4geng1}[][HSK 6]
    \definition{v.}{alterar; mudar; modificar}
  \end{Phonetics}
\end{Entry}

\begin{Entry}{变性}{8,8}{⼜,⼼}
  \begin{Phonetics}{变性}{bian4xing4}
    \definition{s.}{desnaturação | transexual}
    \definition{v.}{desnaturar | mudar de sexo}
  \end{Phonetics}
\end{Entry}

\begin{Entry}{变质}{8,8}{⼜,⾙}
  \begin{Phonetics}{变质}{bian4/zhi4}[][HSK 7-9]
    \definition{v.+compl.}{deteriorar"-se; estragar"-se | tornar"-se moralmente degenerado}
  \end{Phonetics}
\end{Entry}

\begin{Entry}{变革}{8,9}{⼜,⾰}
  \begin{Phonetics}{变革}{bian4ge2}[][HSK 7-9]
    \definition{s.}{mudança; transformação; a natureza das coisas foi reformada}
    \definition{v.}{transformar; mudar (de sistemas sociais, políticas, etc.); mudar o antigo e inovar; mudar a essência das coisas (referindo"-se principalmente aos sistemas sociais)}
  \end{Phonetics}
\end{Entry}

\begin{Entry}{变换}{8,10}{⼜,⼿}
  \begin{Phonetics}{变换}{bian4huan4}[][HSK 6]
    \definition{v.}{variar; alternar; mudar a forma ou o conteúdo de algo de uma coisa para outra}
  \end{Phonetics}
\end{Entry}

\begin{Entry}{变装}{8,12}{⼜,⾐}
  \begin{Phonetics}{变装}{bian4zhuang1}
    \definition{v.}{trocar de roupa | vestir"-se | vestir uma fantasia | disfarçar"-se ou fantasiar"-se de personagem real ou ficcional, \emph{cosplay} | travestir"-se}
  \end{Phonetics}
\end{Entry}

\begin{Entry}{变数}{8,13}{⼜,⽁}
  \begin{Phonetics}{变数}{bian4shu4}
    \definition{s.}{(matemática) variável | fatores variáveis}
  \end{Phonetics}
\end{Entry}

%%%%%%%%%% 呢 %%%%%%%%%%
\subsection*{呢}\addcontentsline{loh}{figure}{呢}

\begin{Entry}{呢}{8}{⼝}
  \begin{Phonetics}{呢}{ne5}[][HSK 1]
    \definition{part.}{usada no final de frases interrogativas (especificamente perguntas, perguntas de escolha e perguntas retóricas) para indicar um tom interrogativo | usada no final de uma frase declarativa, indica que uma ação ou situação está em andamento | usada em frases para indicar uma pausa (muitas vezes em pares) | usada no final de uma frase declarativa para confirmar um fato e convencer o interlocutor (com um tom de indicação e exagero)}
  \end{Phonetics}
  \begin{Phonetics}{呢}{ni2}
    \definition{s.}{(tecido feito de) lã; tecido de lã (para roupas pesadas); tecido de lã pesada; revestimento ou roupa de lã}
  \end{Phonetics}
\end{Entry}

%%%%%%%%%% 周 %%%%%%%%%%
\subsection*{周}\addcontentsline{loh}{figure}{周}

\begin{Entry}{周}{8}{⼝}
  \begin{Phonetics}{周}{zhou1}[][HSK 2]
    \definition*{s.}{Dinastia Zhou (1046--256 BC) | Dinastia Zhou do Norte (557--581), uma das Dinastias do Norte | Dinastia Zhou Posterior (951--960), uma das Cinco Dinastias | Sobrenome: Zhou}
    \definition{adj.}{universal; inteiro; por toda parte | atencioso; pensativo; completo; minucioso}
    \definition{adv.}{semanalmente}
    \definition{clas.}{usado para rodadas, voltas}
    \definition{s.}{periferia; arredores; círculo | semana | ciclo}
    \definition{v.}{fazer um circuito; mover"-se em um curso circular | ajudar alguém}
  \end{Phonetics}
\end{Entry}

\begin{Entry}{周末}{8,5}{⼝,⽊}
  \begin{Phonetics}{周末}{zhou1mo4}[][HSK 2]
    \definition[个]{s.}{final-de-semana}
  \end{Phonetics}
\end{Entry}

\begin{Entry}{周年}{8,6}{⼝,⼲}
  \begin{Phonetics}{周年}{zhou1nian2}[][HSK 2]
    \definition{s.}{aniversário}
  \end{Phonetics}
\end{Entry}

\begin{Entry}{周围}{8,7}{⼝,⼞}
  \begin{Phonetics}{周围}{zhou1wei2}[][HSK 3]
    \definition{s.}{ao redor; redondeza; vizinhança; a parte ao redor do centro}
  \end{Phonetics}
\end{Entry}

\begin{Entry}{周期}{8,12}{⼝,⽉}
  \begin{Phonetics}{周期}{zhou1qi1}[][HSK 5]
    \definition[个]{s.}{período; ciclo; no processo de mudança e movimento das coisas, certas características se repetem várias vezes, com um intervalo de tempo entre cada repetição | período; ciclo; refere"-se a um processo em que certas características se repetem várias vezes, e o tempo decorrido entre duas ocorrências consecutivas | classificação dos elementos na tabela periódica}
  \end{Phonetics}
\end{Entry}

%%%%%%%%%% 味 %%%%%%%%%%
\subsection*{味}\addcontentsline{loh}{figure}{味}

\begin{Entry}{味}{8}{⼝}
  \begin{Phonetics}{味}{wei4}
    \definition{clas.}{usado para ingredientes (de uma receita de medicina chinesa)}
    \definition{s.}{gosto; sabor | cheiro; odor | interesse; deleite | acepipe; \emph{delicacy} | significância; significado}
    \definition{v.}{distinguir (provar) o sabor de; saborear}
  \end{Phonetics}
\end{Entry}

\begin{Entry}{味儿}{8,2}{⼝,⼉}
  \begin{Phonetics}{味儿}{wei4r5}[][HSK 4]
    \definition{s.}{gosto; sabor; propriedade de uma substância que dá à língua uma determinada sensação de sabor | cheiro; odor; propriedade de uma substância que dá ao nariz um determinado sentido de cheiro | interesse; significado; deleite}
  \end{Phonetics}
\end{Entry}

\begin{Entry}{味道}{8,12}{⼝,⾡}
  \begin{Phonetics}{味道}{wei4dao5}[][HSK 2]
    \definition[个,股,种]{s.}{gosto; sabor | sensação; gosto; experiência | interesse; deleite | cheiro; odor}
  \end{Phonetics}
\end{Entry}

\begin{Entry}{味精}{8,14}{⼝,⽶}
  \begin{Phonetics}{味精}{wei4jing1}[][HSK 7-9]
    \definition[勺,袋,克]{s.}{glutamato monossódico (MSG); pó gourmet; temperos, em pó branco ou cristais granulados, adicionados a sopas e pratos para realçar o sabor umami}
  \end{Phonetics}
\end{Entry}

%%%%%%%%%% 呵 %%%%%%%%%%
\subsection*{呵}\addcontentsline{loh}{figure}{呵}

\begin{Entry}{呵}{8}{⼝}
  \begin{Phonetics}{呵}{a1}
    \variantof{啊}
  \end{Phonetics}
  \begin{Phonetics}{呵}{he1}
    \definition{interj.}{``Meu Deus!'' | ``Ah!''; ``Oh!''}
    \definition{v.}{expirar (com a boca aberta) | repreender}
  \end{Phonetics}
\end{Entry}

\begin{Entry}{呵护}{8,7}{⼝,⼿}
  \begin{Phonetics}{呵护}{he1hu4}[][HSK 7-9]
    \definition{v.}{proteger; cuidar bem de}
  \end{Phonetics}
\end{Entry}

%%%%%%%%%% 呶 %%%%%%%%%%
\subsection*{呶}\addcontentsline{loh}{figure}{呶}

\begin{Entry}{呶}{8}{⼝}
  \begin{Phonetics}{呶}{nao2}
    \definition{interj.}{Onomatopéia: ruído alto e contínuo}
    \definition{v.}{Literário: gritar; clamar; falar ruidosamente}
  \seealsoref{努}{nu3}
  \end{Phonetics}
\end{Entry}

%%%%%%%%%% 呼 %%%%%%%%%%
\subsection*{呼}\addcontentsline{loh}{figure}{呼}

\begin{Entry}{呼}{8}{⼝}
  \begin{Phonetics}{呼}{hu1}
    \definition*{s.}{Sobrenome: Hu}
    \definition{s.}{Onomatopéia: descreve o som do vento}
    \definition{v.}{expirar | gritar; clamar | chamar; ligar; ligar para alguém}
  \end{Phonetics}
\end{Entry}

\begin{Entry}{呼风唤雨}{8,4,10,8}{⼝,⾵,⼝,⾬}
  \begin{Phonetics}{呼风唤雨}{hu1feng1-huan4yu3}[][HSK 7-9]
    \definition{expr.}{``Fazer vento e chover.''; refere"-se originalmente ao poder mágico de imortais e taoístas; atualmente, é usado como metáfora para a capacidade de controlar a natureza e, às vezes, como metáfora para a realização de atividades inflamatórias; invocar vento e chuva, exercer poderes mágicos; causar problemas}
  \end{Phonetics}
\end{Entry}

\begin{Entry}{呼吁}{8,6}{⼝,⼝}
  \begin{Phonetics}{呼吁}{hu1yu4}[][HSK 7-9]
    \definition{v.}{apelar; chamar; apelar a um indivíduo ou sociedade, solicitar assistência ou hospedar um apelo a um indivíduo ou sociedade, na esperança de ganhar simpatia e apoio}
  \end{Phonetics}
\end{Entry}

\begin{Entry}{呼吸}{8,6}{⼝,⼝}
  \begin{Phonetics}{呼吸}{hu1xi1}[][HSK 4]
    \definition{s.}{um suspiro; metáfora para um período de tempo muito curto}
    \definition{v.}{respirar}
  \end{Phonetics}
\end{Entry}

\begin{Entry}{呼声}{8,7}{⼝,⼠}
  \begin{Phonetics}{呼声}{hu1sheng1}[][HSK 7-9]
    \definition[片]{s.}{choro; voz}[良心的呼声。===A voz da consciência.]
  \end{Phonetics}
\end{Entry}

\begin{Entry}{呼应}{8,7}{⼝,⼴}
  \begin{Phonetics}{呼应}{hu1ying4}[][HSK 7-9]
    \definition{v.}{ecoar; trabalhar em conjunto (com alguém); entrar em contato ou cuidar um do outro um dia de cada vez}
  \end{Phonetics}
\end{Entry}

\begin{Entry}{呼唤}{8,10}{⼝,⼝}
  \begin{Phonetics}{呼唤}{hu1huan4}[][HSK 7-9]
    \definition{v.}{chamar; gritar para}
  \end{Phonetics}
\end{Entry}

\begin{Entry}{呼啦啦}{8,11,11}{⼝,⼝,⼝}
  \begin{Phonetics}{呼啦啦}{hu1 la1 la1}
    \definition{s.}{Onomatopéia: som de bater asas}
  \end{Phonetics}
\end{Entry}

\begin{Entry}{呼啸}{8,11}{⼝,⼝}
  \begin{Phonetics}{呼啸}{hu1xiao4}
    \definition{v.}{assobiar}
  \end{Phonetics}
\end{Entry}

\begin{Entry}{呼救}{8,11}{⼝,⽁}
  \begin{Phonetics}{呼救}{hu1jiu4}[][HSK 7-9]
    \definition{v.}{pedir ajuda; enviar sinais de SOS}
  \end{Phonetics}
\end{Entry}

%%%%%%%%%% 命 %%%%%%%%%%
\subsection*{命}\addcontentsline{loh}{figure}{命}

\begin{Entry}{命}{8}{⼝}
  \begin{Phonetics}{命}{ming4}[][HSK 6-9]
    \definition[条]{s.}{vida | sorte; destino; fado | ordem; comando; instrução | atribuição de um nome, título etc.}
    \definition{v.}{ordenar; nomear | atribuir (um nome etc.)}
  \end{Phonetics}
\end{Entry}

\begin{Entry}{命令}{8,5}{⼝,⼈}
  \begin{Phonetics}{命令}{ming4ling4}[][HSK 5]
    \definition[条,项,道,个]{s.}{ordem; comando; instruções emitidas pelos superiores aos subordinados}
    \definition{v.}{ordenar; comandar}
  \end{Phonetics}
\end{Entry}

\begin{Entry}{命名}{8,6}{⼝,⼝}
  \begin{Phonetics}{命名}{ming4/ming2}[][HSK 7-9]
    \definition{v.+compl.}{dar nome a alguém ou a alguma coisa; atribuir um nome a; dar nome; conferir um nome; geralmente não usado para fins pessoais}
  \end{Phonetics}
\end{Entry}

\begin{Entry}{命运}{8,7}{⼝,⾡}
  \begin{Phonetics}{命运}{ming4yun4}[][HSK 3]
    \definition[个]{s.}{tendência de desenvolvimento; tendência de futuro; metáfora para a direção e tendência do desenvolvimento e das mudanças | destino; sina; sorte; refere"-se à vida e à morte, à riqueza e à pobreza e a todas as experiências da vida}
  \end{Phonetics}
\end{Entry}

\begin{Entry}{命题}{8,15}{⼝,⾴}
  \begin{Phonetics}{命题}{ming4/ti2}[][HSK 7-9]
    \definition[个]{s.}{proposição; afirmação; tese; em lógica, refere"-se à forma linguística usada para expressar juízos}['地球是圆的' 是一个命题。===A afirmação de que ``a Terra é redonda'' é uma proposição.]
    \definition{v.}{atribuir um tema; formular uma pergunta}
  \end{Phonetics}
\end{Entry}

%%%%%%%%%% 咀 %%%%%%%%%%
\subsection*{咀}\addcontentsline{loh}{figure}{咀}

\begin{Entry}{咀}{8}{⼝}
  \begin{Phonetics}{咀}{ju3}
    \definition{v.}{mastigar; mascar}
  \end{Phonetics}
  \begin{Phonetics}{咀}{zui3}
    \definition{s.}{boca; lábios; a palavra 嘴 é coloquialmente escrita como 咀 | boca (de um objeto)}
  \seealsoref{嘴}{zui3}
  \end{Phonetics}
\end{Entry}

\begin{Entry}{咀嚼}{8,20}{⼝,⼝}
  \begin{Phonetics}{咀嚼}{ju3jue2}
    \definition{v.}{mastigar; triturar alimentos com os dentes | refletir sobre; essa metáfora descreve o ato de ponderar e compreender algo repetidamente}
  \end{Phonetics}
\end{Entry}

%%%%%%%%%% 和 %%%%%%%%%%
\subsection*{和}\addcontentsline{loh}{figure}{和}

\begin{Entry}{和}{8}{⼝}
  \begin{Phonetics}{和}{he2}[][HSK 1]
    \definition*{s.}{Sobrenome: He}
    \definition{adj.}{gentil; suave; amável | harmonioso; em boas condições}
    \definition{conj.}{e (somente para palavras); unidos com}
    \definition{prep.}{relacionado com | para; com; indica correlação; comparação, etc.}
    \definition{s.}{soma; soma total | japonês; refere"-se ao Japão}
    \definition{v.}{disputar; reconciliar; acabar com a guerra ou a disputa | empatar; (próxima edição ou torneio) sem vencedor}
  \end{Phonetics}
  \begin{Phonetics}{和}{he4}
    \definition{v.}{compor um poema em resposta (ao poema de alguém) usando a mesma sequência de rimas | juntar-se à cantoria; cantar junto com outros em harmonia}
  \end{Phonetics}
  \begin{Phonetics}{和}{hu2}
    \definition{v.}{completar um conjunto de Mahjong, 麻将, ou cartas de baralho}
  \seealsoref{麻将}{ma2jiang4}
  \end{Phonetics}
  \begin{Phonetics}{和}{huo2}
    \definition{v.}{combinar uma substância em pó (farinha, gesso, etc.) com água; adicionar líquido ao pó e mexer ou amassar até ficar pegajoso ou espesso}
  \end{Phonetics}
  \begin{Phonetics}{和}{huo4}
    \definition{clas.}{usado para enxágues de roupas | usado para fervuras de ervas medicinais}
    \definition{v.}{misturar (ingredientes); misturar pós ou grãos; misturar com água para obter uma consistência mais líquida}
  \end{Phonetics}
\end{Entry}

\begin{Entry}{和气}{8,4}{⼝,⽓}
  \begin{Phonetics}{和气}{he2qi5}[][HSK 7-9]
    \definition{adj.}{gentil; bondoso; educado | amigável; amável; harmonioso}
    \definition{s.}{amizade; relações harmoniosas; atmosfera harmoniosa; sentimentos harmoniosos}
  \end{Phonetics}
\end{Entry}

\begin{Entry}{和平}{8,5}{⼝,⼲}
  \begin{Phonetics}{和平}{he2ping2}[][HSK 3]
    \definition{adj.}{pacífico; moderado; não violento | pacífico; tranquilo; sereno}
    \definition{s.}{paz ;ausência de guerra}
  \end{Phonetics}
\end{Entry}

\begin{Entry}{和平共处}{8,5,6,5}{⼝,⼲,⼋,⼡}
  \begin{Phonetics}{和平共处}{he2ping2 gong4chu3}[][HSK 7-9]
    \definition{expr.}{coexistência pacífica de nações, sociedades etc.; refere"-se a países com diferentes sistemas sociais que resolvem disputas pacificamente e desenvolvem laços econômicos e culturais com base na igualdade e no benefício mútuo}
  \end{Phonetics}
\end{Entry}

\begin{Entry}{和尚}{8,8}{⼝,⼩}
  \begin{Phonetics}{和尚}{he2shang5}[][HSK 7-9]
    \definition[个,名,位]{s.}{monge budista; refere"-se aos monges budistas do sexo masculino que praticam o budismo}
  \end{Phonetics}
\end{Entry}

\begin{Entry}{和谐}{8,11}{⼝,⾔}
  \begin{Phonetics}{和谐}{he2xie2}[][HSK 6]
    \definition{adj.}{harmonioso; sem conflitos | em perfeita harmonia; ajuste adequado e simétrico}
    \definition{v.}{(eufemismo) censurar}
  \end{Phonetics}
\end{Entry}

\begin{Entry}{和睦}{8,13}{⼝,⽬}
  \begin{Phonetics}{和睦}{he2mu4}[][HSK 7-9]
    \definition{adj.}{harmonioso; amigável; amistoso}
    \definition{s.}{harmonia; concórdia}
  \end{Phonetics}
\end{Entry}

\begin{Entry}{和解}{8,13}{⼝,⾓}
  \begin{Phonetics}{和解}{he2jie3}[][HSK 7-9]
    \definition{v.}{reconciliar}
  \end{Phonetics}
\end{Entry}

\begin{Entry}{和蔼}{8,14}{⼝,⾋}
  \begin{Phonetics}{和蔼}{he2'ai3}[][HSK 7-9]
    \definition{adj.}{gentil; afável; amável}
  \end{Phonetics}
\end{Entry}

%%%%%%%%%% 咒 %%%%%%%%%%
\subsection*{咒}\addcontentsline{loh}{figure}{咒}

\begin{Entry}{咒}{8}{⼝}
  \begin{Phonetics}{咒}{zhou4}
    \definition[个,句]{s.}{encantamento; feitiço}
    \definition{v.}{amaldiçoar; condenar | maltratar; dizer que você espera que as pessoas não tenham sucesso}
  \end{Phonetics}
\end{Entry}

\begin{Entry}{咒骂}{8,9}{⼝,⾺}
  \begin{Phonetics}{咒骂}{zhou4ma4}
    \definition{v.}{xingar | amaldiçoar | execrar}
  \end{Phonetics}
\end{Entry}

%%%%%%%%%% 咖 %%%%%%%%%%
\subsection*{咖}\addcontentsline{loh}{figure}{咖}

\begin{Entry}{咖}{8}{⼝}
  \begin{Phonetics}{咖}{ka1}
    \definition[杯]{s.}{classe | café | graduação}
  \end{Phonetics}
\end{Entry}

\begin{Entry}{咖啡}{8,11}{⼝,⼝}
  \begin{Phonetics}{咖啡}{ka1fei1}[][HSK 3]
    \definition[杯,瓶,罐,壶,包,袋,盒]{s.}{(empréstimo linguístico) café}
  \end{Phonetics}
\end{Entry}

\begin{Entry}{咖啡色}{8,11,6}{⼝,⼝,⾊}
  \begin{Phonetics}{咖啡色}{ka1fei1 se4}
    \definition{s.}{cor café}
  \end{Phonetics}
\end{Entry}

\begin{Entry}{咖啡馆}{8,11,11}{⼝,⼝,⾷}
  \begin{Phonetics}{咖啡馆}{ka1fei1guan3}
    \definition[家]{s.}{cafeteria}
  \end{Phonetics}
\end{Entry}

%%%%%%%%%% 哎 %%%%%%%%%%
\subsection*{哎}\addcontentsline{loh}{figure}{哎}

\begin{Entry}{哎}{8}{⼝}
  \begin{Phonetics}{哎}{ai1}[][HSK 7-9]
    \definition{interj.}{``Por que?''; ``Ei!''; ``Ai!''; expressa surpresa ou insatisfação | ``Ei!''; ``Cuidado!''}
  \end{Phonetics}
\end{Entry}

\begin{Entry}{哎呀}{8,7}{⼝,⼝}
  \begin{Phonetics}{哎呀}{ai1ya1}[][HSK 7-9]
    \definition{interj.}{expressa surpresa ou espanto | expressa reclamação, impaciência, etc.}
  \end{Phonetics}
\end{Entry}

%%%%%%%%%% 固 %%%%%%%%%%
\subsection*{固}\addcontentsline{loh}{figure}{固}

\begin{Entry}{固}{8}{⼞}
  \begin{Phonetics}{固}{gu4}
    \definition*{s.}{Sobrenome: Gu}
    \definition{adj.}{sólido; firme; forte | duro; sólido | mal informado; superficial; ignorante}
    \definition{adv.}{firmemente; resolutamente | originalmente; em primeiro lugar | certamente; reconhecidamente; seguramente}
    \definition{conj.}{usado da mesma forma que 固然}
    \definition{v.}{solidificar; consolidar; fortalecer | defender; proteger}
  \seealsoref{固然}{gu4ran2}
  \end{Phonetics}
\end{Entry}

\begin{Entry}{固执}{8,6}{⼞,⼿}
  \begin{Phonetics}{固执}{gu4zhi5}[][HSK 7-9]
    \definition{adj.}{obstinado; teimoso; mantém suas próprias opiniões e não quer mudá-las, mesmo que estejam erradas}
  \end{Phonetics}
\end{Entry}

\begin{Entry}{固定}{8,8}{⼞,⼧}
  \begin{Phonetics}{固定}{gu4ding4}[][HSK 4]
    \definition{adj.}{fixo; regular; inalterado ou imóvel}
    \definition{v.}{consertar; tornar fixo, não mover novamente; colocar as coisas em ordem, não mudá-las novamente}
  \end{Phonetics}
\end{Entry}

\begin{Entry}{固然}{8,12}{⼞,⽕}
  \begin{Phonetics}{固然}{gu4ran2}[][HSK 7-9]
    \definition{conj.}{usado para introduzir uma cláusula adversativa admitindo primeiro um certo fato; quando usado na primeira metade de uma frase, a segunda metade geralmente tem 可是 ou 但是 para ecoá-lo, indicando que o fato A é reconhecido, mas o fato B não se torna inválido por causa do fato A | admitir um fato sem negar outro; indica o reconhecimento de um fato, levando a uma transição no texto seguinte; indica o reconhecimento do fato A e não nega o fato B}
  \seealsoref{但是}{dan4shi4}
  \seealsoref{可是}{ke3shi4}
  \end{Phonetics}
\end{Entry}

%%%%%%%%%% 国 %%%%%%%%%%
\subsection*{国}\addcontentsline{loh}{figure}{国}

\begin{Entry}{国}{8}{⼞}
  \begin{Phonetics}{国}{guo2}[][HSK 1]
    \definition*{s.}{Sobrenome: Guo}
    \definition{adj.}{nacional; do estado; representante do país | o melhor de um país}
    \definition[个]{s.}{estado; nação; país}
  \end{Phonetics}
\end{Entry}

\begin{Entry}{国人}{8,2}{⼞,⼈}
  \begin{Phonetics}{国人}{guo2ren2}
    \definition{s.}{compatriota}
  \end{Phonetics}
\end{Entry}

\begin{Entry}{国土}{8,3}{⼞,⼟}
  \begin{Phonetics}{国土}{guo2tu3}[][HSK 7-9]
    \definition{s.}{terra; território; território nacional}
  \end{Phonetics}
\end{Entry}

\begin{Entry}{国内}{8,4}{⼞,⼌}
  \begin{Phonetics}{国内}{guo2nei4}[][HSK 3]
    \definition{s.}{interno (a um país); doméstico; lar; dentro de um determinado país}
  \end{Phonetics}
\end{Entry}

\begin{Entry}{国王}{8,4}{⼞,⽟}
  \begin{Phonetics}{国王}{guo2wang2}[][HSK 6]
    \definition[位,名,个,些]{s.}{rei; soberanos; o governante supremo de algumas monarquias antigas; nos tempos modernos, refere"-se ao chefe de estado de algumas monarquias}
  \end{Phonetics}
\end{Entry}

\begin{Entry}{国外}{8,5}{⼞,⼣}
  \begin{Phonetics}{国外}{guo2wai4}[][HSK 1]
    \definition{adj.}{externo; no exterior; fora do país; outros lugares fora do país; geralmente chamados de exterior;  exterior não é o mesmo que estrangeiro}
  \end{Phonetics}
\end{Entry}

\begin{Entry}{国民}{8,5}{⼞,⽒}
  \begin{Phonetics}{国民}{guo2min2}[][HSK 5]
    \definition{adj.}{nacional}
    \definition[个]{s.}{membro de uma nação; povo de uma nação}
  \end{Phonetics}
\end{Entry}

\begin{Entry}{国产}{8,6}{⼞,⼇}
  \begin{Phonetics}{国产}{guo2chan3}[][HSK 6]
    \definition{adj.}{doméstico; feito na China; produzido internamente, especificamente na China}
  \end{Phonetics}
\end{Entry}

\begin{Entry}{国会}{8,6}{⼞,⼈}
  \begin{Phonetics}{国会}{guo2hui4}[][HSK 6]
    \definition{s.}{parlamento; congresso}
  \end{Phonetics}
\end{Entry}

\begin{Entry}{国庆}{8,6}{⼞,⼴}
  \begin{Phonetics}{国庆}{guo2qing4}[][HSK 3]
    \definition*{s.}{Dia Nacional, o dia em que um país comemora sua independência ou fundação}
  \end{Phonetics}
\end{Entry}

\begin{Entry}{国庆节}{8,6,5}{⼞,⼴,⾋}
  \begin{Phonetics}{国庆节}{guo2qing4jie2}
    \definition*{s.}{Dia Nacional (1~de~outubro)}
  \end{Phonetics}
\end{Entry}

\begin{Entry}{国有}{8,6}{⼞,⽉}
  \begin{Phonetics}{国有}{guo2you3}[][HSK 7-9]
    \definition{v.}{pertencer ao estado; ser nacionalizado}
  \end{Phonetics}
\end{Entry}

\begin{Entry}{国防}{8,6}{⼞,⾩}
  \begin{Phonetics}{国防}{guo2fang2}[][HSK 7-9]
    \definition{s.}{defesa nacional; as instalações humanas, materiais e militares que um país possui para defender sua soberania territorial e impedir invasões estrangeiras}
  \end{Phonetics}
\end{Entry}

\begin{Entry}{国际}{8,7}{⼞,⾩}
  \begin{Phonetics}{国际}{guo2ji4}[][HSK 2]
    \definition{adj.}{internacional; entre países; entre nações}
    \definition{s.}{internacional; o mundo; entre nações; entre países de todo o mundo}
  \end{Phonetics}
\end{Entry}

\begin{Entry}{国际儿童节}{8,7,2,12,5}{⼞,⾩,⼉,⽴,⾋}
  \begin{Phonetics}{国际儿童节}{guo2ji4 er2tong2jie2}
    \definition*{s.}{Dia Internacional das Crianças (1 de junho)}
  \end{Phonetics}
\end{Entry}

\begin{Entry}{国际妇女节}{8,7,6,3,5}{⼞,⾩,⼥,⼥,⾋}
  \begin{Phonetics}{国际妇女节}{guo2ji4 fu4nv3jie2}
    \definition*{s.}{Dia Internacional das Mulheres (8 de março)}
  \end{Phonetics}
\end{Entry}

\begin{Entry}{国际劳动节}{8,7,7,6,5}{⼞,⾩,⼒,⼒,⾋}
  \begin{Phonetics}{国际劳动节}{guo2ji4 lao2dong4 jie2}
    \definition*{s.}{Dia Internacional dos Trabalhadores (1 de maio)}
  \end{Phonetics}
\end{Entry}

\begin{Entry}{国学}{8,8}{⼞,⼦}
  \begin{Phonetics}{国学}{guo2xue2}[][HSK 7-9]
    \definition*{s.}{Arcaico: O Colégio Imperial}
    \definition{s.}{estudos da cultura clássica chinesa (história, filosofia, literatura, língua, etc.) | cultura nacional chinesa | estudos da antiga civilização chinesa}
  \end{Phonetics}
\end{Entry}

\begin{Entry}{国宝}{8,8}{⼞,⼧}
  \begin{Phonetics}{国宝}{guo2bao3}[][HSK 7-9]
    \definition[件]{s.}{tesouro nacional}
  \end{Phonetics}
\end{Entry}

\begin{Entry}{国画}{8,8}{⼞,⽥}
  \begin{Phonetics}{国画}{guo2hua4}[][HSK 7-9]
    \definition[幅,张,卷]{s.}{pintura tradicional chinesa | arte chinesa | pintura nacional}
  \end{Phonetics}
\end{Entry}

\begin{Entry}{国语}{8,9}{⼞,⾔}
  \begin{Phonetics}{国语}{guo2yu3}
    \definition*{s.}{Língua Chinesa (Mandarim), enfatizando sua natureza nacional}
  \end{Phonetics}
\end{Entry}

\begin{Entry}{国家}{8,10}{⼞,⼧}
  \begin{Phonetics}{国家}{guo2jia1}[][HSK 1]
    \definition[个]{s.}{país; estado; nação; um lugar reconhecido internacionalmente e com soberania independente, incluindo as pessoas e as instituições administrativas desse lugar}
  \end{Phonetics}
\end{Entry}

\begin{Entry}{国宾馆}{8,10,11}{⼞,⼧,⾷}
  \begin{Phonetics}{国宾馆}{guo2bin1guan3}
    \definition{s.}{pousada estadual}
  \end{Phonetics}
\end{Entry}

\begin{Entry}{国情}{8,11}{⼞,⼼}
  \begin{Phonetics}{国情}{guo2qing2}[][HSK 7-9]
    \definition{s.}{condição (ou estado) do país; condições nacionais; as condições e características básicas da natureza social, política, economia, cultura etc. de um país também se referem especificamente às condições e características básicas de um país em um determinado período de tempo}
  \end{Phonetics}
\end{Entry}

\begin{Entry}{国旗}{8,14}{⼞,⽅}
  \begin{Phonetics}{国旗}{guo2qi2}[][HSK 6]
    \definition[面]{s.}{bandeira (de um país)}
  \end{Phonetics}
\end{Entry}

\begin{Entry}{国歌}{8,14}{⼞,⽋}
  \begin{Phonetics}{国歌}{guo2ge1}[][HSK 6]
    \definition[首,支]{s.}{hino nacional; o hino nacional da China, oficialmente designado pelo estado como a música que representa o país, é ``Marcha dos Voluntários''}
  \end{Phonetics}
\end{Entry}

\begin{Entry}{国徽}{8,17}{⼞,⼻}
  \begin{Phonetics}{国徽}{guo2hui1}[][HSK 7-9]
    \definition{s.}{emblema nacional; o emblema nacional da China, oficialmente designado pelo estado para representar o país, apresenta a Praça da Paz Celestial sob o céu brilhante de cinco estrelas, cercada por espigas de grãos e engrenagens}
  \end{Phonetics}
\end{Entry}

\begin{Entry}{国籍}{8,20}{⼞,⽵}
  \begin{Phonetics}{国籍}{guo2ji2}[][HSK 5]
    \definition[个]{s.}{nacionalidade; cidadania; refere"-se à identidade de um indivíduo como pertencente a um Estado | identidade nacional (de um avião, navio, etc.)}
  \end{Phonetics}
\end{Entry}

%%%%%%%%%% 图 %%%%%%%%%%
\subsection*{图}\addcontentsline{loh}{figure}{图}

\begin{Entry}{图}{8}{⼞}
  \begin{Phonetics}{图}{tu2}[][HSK 3]
    \definition*{s.}{Sobrenome: Tu}
    \definition[张]{s.}{mapa; gráfico; imagem; desenho | plano; esquema; tentativa}
    \definition{v.}{procurar; perseguir; esperar obter| desenhar; retratar; pintar | imaginar; planejar; pensar; maquinar}
  \end{Phonetics}
\end{Entry}

\begin{Entry}{图书}{8,4}{⼞,⼄}
  \begin{Phonetics}{图书}{tu2shu1}[][HSK 6]
    \definition{s.}{livros; um termo geral para publicações como livros e álbuns de imagens}[这些图书都可以借阅。===Esses livros estão disponíveis para empréstimo.]
  \end{Phonetics}
\end{Entry}

\begin{Entry}{图书馆}{8,4,11}{⼞,⼄,⾷}
  \begin{Phonetics}{图书馆}{tu2shu1guan3}[][HSK 1]
    \definition[个,家]{s.}{biblioteca; instituição que coleta, organiza e armazena livros e materiais para leitura e consulta}
  \end{Phonetics}
\end{Entry}

\begin{Entry}{图片}{8,4}{⼞,⽚}
  \begin{Phonetics}{图片}{tu2pian4}[][HSK 2]
    \definition[张,幅]{s.}{imagem; fotografia; um termo geral para imagens, fotografias, decalques, etc. usados para ilustrar algo}
  \end{Phonetics}
\end{Entry}

\begin{Entry}{图形}{8,7}{⼞,⼺}
  \begin{Phonetics}{图形}{tu2xing2}[][HSK 7-9]
    \definition[种,个]{s.}{figura; gráfico | sinal; carta; desenho; diagrama}
  \synonymref{图案}{tu2'an4}
  \end{Phonetics}
\end{Entry}

\begin{Entry}{图纸}{8,7}{⼞,⽷}
  \begin{Phonetics}{图纸}{tu2zhi3}[][HSK 7-9]
    \definition[张]{s.}{desenho; planta; folha de desenho; \emph{blueprint}; um documento técnico que utiliza desenhos e texto para descrever a estrutura, forma, dimensões e outros requisitos de estruturas de engenharia, máquinas, equipamentos, etc.}
  \end{Phonetics}
\end{Entry}

\begin{Entry}{图画}{8,8}{⼞,⽥}
  \begin{Phonetics}{图画}{tu2hua4}[][HSK 3]
    \definition[幅,张,套]{s.}{desenho; imagem; pintura}
  \end{Phonetics}
\end{Entry}

\begin{Entry}{图表}{8,8}{⼞,⾐}
  \begin{Phonetics}{图表}{tu2biao3}[][HSK 7-9]
    \definition[张,个]{s.}{carta; diagrama; gráfico | figura; gráfico; diagrama; pictograma; pictografia; cronograma; folha; tabela; termo geral para diagramas e tabelas que representam diversas situações e indicam vários números, como diagramas esquemáticos e tabelas estatísticas}
  \end{Phonetics}
\end{Entry}

\begin{Entry}{图标}{8,9}{⼞,⽊}
  \begin{Phonetics}{图标}{tu2biao1}
    \definition{s.}{ícone (informática)}[这个图标代表设置。===Este ícone representa as configurações.]
  \end{Phonetics}
\end{Entry}

\begin{Entry}{图案}{8,10}{⼞,⽊}
  \begin{Phonetics}{图案}{tu2'an4}[][HSK 4]
    \definition[种,个]{s.}{padrão; desenho; padrões e gráficos usados para decoração de edifícios, tecidos, artes e artesanato, etc.}
  \end{Phonetics}
\end{Entry}

\begin{Entry}{图像}{8,13}{⼞,⼈}
  \begin{Phonetics}{图像}{tu2xiang4}[][HSK 7-9]
    \definition{s.}{imagem; figura; imagens desenhadas, fotografadas ou impressas}
  \end{Phonetics}
\end{Entry}

%%%%%%%%%% 坡 %%%%%%%%%%
\subsection*{坡}\addcontentsline{loh}{figure}{坡}

\begin{Entry}{坡}{8}{⼟}
  \begin{Phonetics}{坡}{po1}[][HSK 6]
    \definition{adj.}{inclinado}
    \definition{s.}{declive | encosta}
  \end{Phonetics}
\end{Entry}

%%%%%%%%%% 坦 %%%%%%%%%%
\subsection*{坦}\addcontentsline{loh}{figure}{坦}

\begin{Entry}{坦}{8}{⼟}
  \begin{Phonetics}{坦}{tan3}
    \definition*{s.}{Sobrenome: Tan}
    \definition{adj.}{nivelado; suave; plano | calmo; composto | aberto; sincero; franco}
  \end{Phonetics}
\end{Entry}

\begin{Entry}{坦白}{8,5}{⼟,⽩}
  \begin{Phonetics}{坦白}{tan3bai2}[][HSK 7-9]
    \definition{adj.}{honesto; franco; sincero; puros de coração}
    \definition{v.}{ser franco; confessar; dizer a verdade sobre erros ou crimes}
  \synonymref{坦率}{tan3shuai4}
  \antonymref{抗拒}{kang4ju4}
  \antonymref{通知}{tong1zhi1}
  \end{Phonetics}
\end{Entry}

\begin{Entry}{坦克}{8,7}{⼟,⼗}
  \begin{Phonetics}{坦克}{tan3ke4}[][HSK 7-9]
    \definition[辆]{s.}{Empréstimo linguístico: tanque (veículo militar); veículos de combate blindados sobre esteiras, equipados com canhões, metralhadoras e torres giratórias}
  \end{Phonetics}
\end{Entry}

\begin{Entry}{坦诚}{8,8}{⼟,⾔}
  \begin{Phonetics}{坦诚}{tan3cheng2}[][HSK 7-9]
    \definition{adj.}{franco e sincero; franco e aberto}
  \synonymref{诚恳}{cheng2ken3}
  \synonymref{坦率}{tan3shuai4}
  \synonymref{真诚}{zhen1cheng2}
  \antonymref{撒谎}{sa1/huang3}
  \end{Phonetics}
\end{Entry}

\begin{Entry}{坦率}{8,11}{⼟,⽞}
  \begin{Phonetics}{坦率}{tan3shuai4}[][HSK 7-9]
    \definition{adj.}{sincero; franco; direto}
  \synonymref{爽快}{shuang3kuai5}
  \synonymref{坦白}{tan3bai2}
  \synonymref{坦诚}{tan3cheng2}
  \antonymref{通知}{tong1zhi1}
  \antonymref{委婉}{wei3wan3}
  \end{Phonetics}
\end{Entry}

\begin{Entry}{坦然}{8,12}{⼟,⽕}
  \begin{Phonetics}{坦然}{tan3ran2}[][HSK 7-9]
    \definition{adj.}{calmo; sereno; imperturbável; sem receios; descreve um estado de espírito como calmo e sem preocupações}
  \synonymref{安心}{an1xin1}
  \synonymref{平静}{ping2jing4}
  \antonymref{狼狈}{lang2bei4}
  \antonymref{受惊}{shou4/jing1}
  \end{Phonetics}
\end{Entry}

%%%%%%%%%% 垂 %%%%%%%%%%
\subsection*{垂}\addcontentsline{loh}{figure}{垂}

\begin{Entry}{垂}{8}{⼠}
  \begin{Phonetics}{垂}{chui2}[][HSK 7-9]
    \definition{adv.}{perto de; uma palavra respeitosa usada para se referir às ações de outros (geralmente anciãos ou superiores) em relação a si mesmo.}
    \definition{v.}{cair; deixar cair; pendurar (objeto de cabeça para baixo)  | Literário: (mais velhos ou superiores) condescender; um termo respeitoso usado para se referir às ações de outros (geralmente anciãos ou superiores) em relação a si mesmo | transmitir; legar à posteridade; passar para as gerações posteriores}
  \end{Phonetics}
\end{Entry}

\begin{Entry}{垂头丧气}{8,5,8,4}{⼠,⼤,⼗,⽓}
  \begin{Phonetics}{垂头丧气}{chui2tou2-sang4qi4}[][HSK 7-9]
    \definition{v.}{estar desanimado; estar abatido; abaixar a cabeça com um olhar abatido}
  \end{Phonetics}
\end{Entry}

%%%%%%%%%% 垃 %%%%%%%%%%
\subsection*{垃}\addcontentsline{loh}{figure}{垃}

\begin{Entry}{垃}{8}{⼟}
  \begin{Phonetics}{垃}{la1}
    \definition[堆]{s.}{lixo}
  \end{Phonetics}
\end{Entry}

\begin{Entry}{垃圾}{8,6}{⼟,⼟}
  \begin{Phonetics}{垃圾}{la1ji1}[][HSK 4]
    \definition{adj.}{lixo; inútil, ruim ou prejudicial}
    \definition[袋,桶,堆,车,片]{s.}{entulho; lixo; refugo; rejeito; resíduo; coisa inútil que é jogada fora; metáfora para alguém ou algo que perdeu seu valor ou serve a um propósito ruim}
  \end{Phonetics}
\end{Entry}

\begin{Entry}{垃圾工}{8,6,3}{⼟,⼟,⼯}
  \begin{Phonetics}{垃圾工}{la1ji1gong1}
    \definition{s.}{lixeiro | gari}
  \end{Phonetics}
\end{Entry}

\begin{Entry}{垃圾车}{8,6,4}{⼟,⼟,⾞}
  \begin{Phonetics}{垃圾车}{la1ji1che1}
    \definition{s.}{caminhão de lixo}
  \end{Phonetics}
\end{Entry}

\begin{Entry}{垃圾电邮}{8,6,5,7}{⼟,⼟,⽥,⾢}
  \begin{Phonetics}{垃圾电邮}{la1ji1 dian4you2}
    \definition{s.}{\emph{e-mail} de \emph{spam}}
  \seealsoref{垃圾邮件}{la1ji1 you2jian4}
  \end{Phonetics}
\end{Entry}

\begin{Entry}{垃圾邮件}{8,6,7,6}{⼟,⼟,⾢,⼈}
  \begin{Phonetics}{垃圾邮件}{la1ji1 you2jian4}
    \definition{s.}{\emph{spam}, \emph{e-mail} não solicitado}
  \seealsoref{垃圾电邮}{la1ji1 dian4you2}
  \end{Phonetics}
\end{Entry}

\begin{Entry}{垃圾食品}{8,6,9,9}{⼟,⼟,⾷,⼝}
  \begin{Phonetics}{垃圾食品}{la1ji1shi2pin3}
    \definition{s.}{\emph{junk food}}
  \end{Phonetics}
\end{Entry}

\begin{Entry}{垃圾堆}{8,6,11}{⼟,⼟,⼟}
  \begin{Phonetics}{垃圾堆}{la1ji1dui1}
    \definition{s.}{depósito de lixo}
  \end{Phonetics}
\end{Entry}

\begin{Entry}{垃圾筒}{8,6,12}{⼟,⼟,⽵}
  \begin{Phonetics}{垃圾筒}{la1ji1tong3}
    \definition{s.}{cesto de lixo}
  \end{Phonetics}
\end{Entry}

\begin{Entry}{垃圾箱}{8,6,15}{⼟,⼟,⾋}
  \begin{Phonetics}{垃圾箱}{la1ji1xiang1}
    \definition{s.}{cesto de lixo}
  \end{Phonetics}
\end{Entry}

%%%%%%%%%% 垄 %%%%%%%%%%
\subsection*{垄}\addcontentsline{loh}{figure}{垄}

\begin{Entry}{垄}{8}{⼟}
  \begin{Phonetics}{垄}{long3}
    \definition{s.}{lomba; crista (em um campo); cumes de terra construídos em terras cultivadas | caminho elevado entre campos | algo semelhante a uma crista, lomba}
  \end{Phonetics}
\end{Entry}

\begin{Entry}{垄断}{8,11}{⼟,⽄}
  \begin{Phonetics}{垄断}{long3duan4}[][HSK 7-9]
    \definition{v.}{monopolizar; ter o monopólio de}
  \end{Phonetics}
\end{Entry}

%%%%%%%%%% 备 %%%%%%%%%%
\subsection*{备}\addcontentsline{loh}{figure}{备}

\begin{Entry}{备}{8}{⼡}
  \begin{Phonetics}{备}{bei4}
    \definition*{s.}{Sobrenome: Bei}
    \definition{adv.}{totalmente; de todas as maneiras possíveis | todos; tudo}
    \definition{s.}{equipamento}
    \definition{v.}{estar equipar com; ter; possuir | preparar; ficar pronto | providenciar (ou preparar) contra; tomar precauções contra}
  \end{Phonetics}
\end{Entry}

\begin{Entry}{备用}{8,5}{⼡,⽤}
  \begin{Phonetics}{备用}{bei4yong4}[][HSK 7-9]
    \definition{v.}{reservar; guardar algo para uso futuro}
  \end{Phonetics}
\end{Entry}

\begin{Entry}{备份}{8,6}{⼡,⼈}
  \begin{Phonetics}{备份}{bei4fen4}
    \definition{s.}{cópia de segurança | \emph{backup}}
  \end{Phonetics}
\end{Entry}

\begin{Entry}{备受}{8,8}{⼡,⼜}
  \begin{Phonetics}{备受}{bei4shou4}[][HSK 7-9]
    \definition{v.}{experimentar plenamente (o bem ou o mal)}
  \end{Phonetics}
\end{Entry}

\begin{Entry}{备胎}{8,9}{⼡,⾁}
  \begin{Phonetics}{备胎}{bei4tai1}
    \definition{s.}{pneu sobressalente | Gíria: substituto}
  \end{Phonetics}
\end{Entry}

\begin{Entry}{备课}{8,10}{⼡,⾔}
  \begin{Phonetics}{备课}{bei4/ke4}[][HSK 7-9]
    \definition{v.+compl.}{(professor) preparar aulas}
  \end{Phonetics}
\end{Entry}

%%%%%%%%%% 夜 %%%%%%%%%%
\subsection*{夜}\addcontentsline{loh}{figure}{夜}

\begin{Entry}{夜}{8}{⼣}
  \begin{Phonetics}{夜}{ye4}[][HSK 2]
    \definition{s.}{noite; tarde; noturno; o período do anoitecer ao amanhecer; em meteorologia, refere"-se especificamente ao período das 20h do dia atual às 8h do dia seguinte}
  \antonymref{日}{ri4}
  \antonymref{昼}{zhou4}
  \end{Phonetics}
\end{Entry}

\begin{Entry}{夜生活}{8,5,9}{⼣,⽣,⽔}
  \begin{Phonetics}{夜生活}{ye4sheng1huo2}
    \definition{s.}{vida noturna}
  \end{Phonetics}
\end{Entry}

\begin{Entry}{夜鸟}{8,5}{⼣,⿃}
  \begin{Phonetics}{夜鸟}{ye4niao3}
    \definition{s.}{ave noturna}
  \end{Phonetics}
\end{Entry}

\begin{Entry}{夜场}{8,6}{⼣,⼟}
  \begin{Phonetics}{夜场}{ye4chang3}
    \definition{s.}{show noturno (em um teatro, etc.) | local de entretenimento noturno (bar, boate, discoteca, etc.)}
  \end{Phonetics}
\end{Entry}

\begin{Entry}{夜里}{8,7}{⼣,⾥}
  \begin{Phonetics}{夜里}{ye4li5}[][HSK 2]
    \definition{s.}{noturno; à noite; o período do anoitecer ao amanhecer}
  \end{Phonetics}
\end{Entry}

\begin{Entry}{夜间}{8,7}{⼣,⾨}
  \begin{Phonetics}{夜间}{ye4jian5}[][HSK 5]
    \definition{s.}{noite; à noite; noturno; durante a noite}
  \end{Phonetics}
\end{Entry}

\begin{Entry}{夜夜}{8,8}{⼣,⼣}
  \begin{Phonetics}{夜夜}{ye4ye4}
    \definition{adv.}{toda noite}
  \end{Phonetics}
\end{Entry}

\begin{Entry}{夜店}{8,8}{⼣,⼴}
  \begin{Phonetics}{夜店}{ye4dian4}
    \definition{s.}{boate | \emph{nightclub}}
  \end{Phonetics}
\end{Entry}

\begin{Entry}{夜晚}{8,11}{⼣,⽇}
  \begin{Phonetics}{夜晚}{ye4wan3}
    \definition[个]{s.}{noite}
  \end{Phonetics}
\end{Entry}

\begin{Entry}{夜深人静}{8,11,2,14}{⼣,⽔,⼈,⾭}
  \begin{Phonetics}{夜深人静}{ye4shen1ren2jing4}
    \definition{expr.}{``Na calada da noite.''; ``No silêncio da noite.''}
  \end{Phonetics}
\end{Entry}

\begin{Entry}{夜幕}{8,13}{⼣,⼱}
  \begin{Phonetics}{夜幕}{ye4mu4}
    \definition{s.}{escuridão crescente; cortina da noite; o véu da noite}
  \end{Phonetics}
\end{Entry}

%%%%%%%%%% 奇 %%%%%%%%%%
\subsection*{奇}\addcontentsline{loh}{figure}{奇}

\begin{Entry}{奇}{8}{⼤}
  \begin{Phonetics}{奇}{qi2}
    \definition{adj.}{ímpar (número); singular; solteiro; não em pares (ao contrário de 偶)}
    \definition{s.}{lotes ímpares; quantidade fracionária (acima daquela mencionada em um número redondo)}
  \seealsoref{偶}{ou3}
  \end{Phonetics}
\end{Entry}

\begin{Entry}{奇妙}{8,7}{⼤,⼥}
  \begin{Phonetics}{奇妙}{qi2miao4}[][HSK 6]
    \definition{adj.}{maravilhoso; milagroso; intrigante; muito inteligente e engenhoso (usado principalmente para descrever coisas interessantes e novas)}
  \end{Phonetics}
\end{Entry}

\begin{Entry}{奇花异草}{8,7,6,9}{⼤,⾋,⼶,⾋}
  \begin{Phonetics}{奇花异草}{qi2hua1-yi4cao3}[][HSK 7-9]
    \definition{expr.}{``Flores exóticas e ervas raras.''; vista espetacular do mundo botânico; muito raramente visto, incomum}
  \end{Phonetics}
\end{Entry}

\begin{Entry}{奇怪}{8,8}{⼤,⼼}
  \begin{Phonetics}{奇怪}{qi2guai4}[][HSK 3]
    \definition{adj.}{estranho; diferente do habitual; raramente visto, até um pouco irracional | estranho; esquisito; a descrição é diferente do imaginado e é difícil de entender}
    \definition{v.}{ficar perplexo; maravilhar"-se; sentir"-se surpreso; sentir"-se estranho; sentir"-se incompreensível}
  \end{Phonetics}
\end{Entry}

\begin{Entry}{奇迹}{8,9}{⼤,⾡}
  \begin{Phonetics}{奇迹}{qi2ji4}[][HSK 7-9]
    \definition[个,种]{s.}{milagre; maravilha; coisas extraordinárias inimagináveis}
  \end{Phonetics}
\end{Entry}

\begin{Entry}{奇特}{8,10}{⼤,⽜}
  \begin{Phonetics}{奇特}{qi2te4}[][HSK 7-9]
    \definition{adj.}{estranho; peculiar; singular; incomum; extraordinário}
  \end{Phonetics}
\end{Entry}

%%%%%%%%%% 奉 %%%%%%%%%%
\subsection*{奉}\addcontentsline{loh}{figure}{奉}

\begin{Entry}{奉}{8}{⼤}
  \begin{Phonetics}{奉}{feng4}
    \definition*{s.}{Sobrenome: Feng}
    \definition{v.}{Literário: dedicar ou presentear com respeito | receber (pedidos, instruções, etc.) | Literário: estimar; reverenciar | Litrário: acreditar em  | esperar; atender; servir}
  \end{Phonetics}
\end{Entry}

\begin{Entry}{奉献}{8,13}{⼤,⽝}
  \begin{Phonetics}{奉献}{feng4xian4}[][HSK 6]
    \definition{v.}{dedicar; oferecer como tributo; apresentar com todo respeito; entregar respeitosamente}
  \end{Phonetics}
\end{Entry}

%%%%%%%%%% 奋 %%%%%%%%%%
\subsection*{奋}\addcontentsline{loh}{figure}{奋}

\begin{Entry}{奋}{8}{⼤}
  \begin{Phonetics}{奋}{fen4}
    \definition{adv.}{energicamente; com força e espírito}
    \definition{v.}{esforçar"-se; agir vigorosamente; preparar"-se | levantar | aplicar energia; resolver; animar"-se | acenar; sacudir; levantar}
  \end{Phonetics}
\end{Entry}

\begin{Entry}{奋力}{8,2}{⼤,⼒}
  \begin{Phonetics}{奋力}{fen4li4}[][HSK 7-9]
    \definition{v.}{fazer tudo o que puder; não poupar esforços}
  \end{Phonetics}
\end{Entry}

\begin{Entry}{奋斗}{8,4}{⼤,⽃}
  \begin{Phonetics}{奋斗}{fen4dou4}[][HSK 4]
    \definition{v.}{lutar; esforçar"-se; batalhar; trabalhar duro para atingir um determinado objetivo}
  \end{Phonetics}
\end{Entry}

\begin{Entry}{奋勇}{8,9}{⼤,⼒}
  \begin{Phonetics}{奋勇}{fen4yong3}[][HSK 7-9]
    \definition{v.}{reunir toda a coragem e energia; criar coragem}
  \end{Phonetics}
\end{Entry}

\begin{Entry}{奋战}{8,9}{⼤,⼽}
  \begin{Phonetics}{奋战}{fen4zhan4}
    \definition{v.}{lutar bravamente | trabalhar duro}
  \end{Phonetics}
\end{Entry}

%%%%%%%%%% 奔 %%%%%%%%%%
\subsection*{奔}\addcontentsline{loh}{figure}{奔}

\begin{Entry}{奔}{8}{⼤}
  \begin{Phonetics}{奔}{ben1}
    \definition{v.}{correr rápido; correr com pressa | apressar | fugir; escapar | galopar | fugir; termo antigo para uma mulher que foge com um homem}
  \end{Phonetics}
  \begin{Phonetics}{奔}{ben4}[][HSK 7-9]
    \definition{prep.}{em direção a}
    \definition{v.}{ir direto em direção a; seguir em direção a; ir direto para o seu destino | aproximar"-se; estar prestes a | estar ocupado correndo por aí; correr por algo}
  \end{Phonetics}
\end{Entry}

\begin{Entry}{奔驰}{8,6}{⼤,⾺}
  \begin{Phonetics}{奔驰}{ben1chi2}
    \definition*{s.}{Benz de Mercedes"-Benz}
    \definition{v.}{acelerar; galopar; (carro, cavalo, etc.) mover"-se ou correr rapidamente}
  \seealsoref{梅赛德斯-奔驰}{mei2sai4de2si1-ben1chi2}
  \end{Phonetics}
\end{Entry}

\begin{Entry}{奔波}{8,8}{⼤,⽔}
  \begin{Phonetics}{奔波}{ben1bo1}[][HSK 7-9]
    \definition{v.}{correr; estar ocupado correndo; correr para frente e para trás, com dificuldade e ocupado}
  \end{Phonetics}
\end{Entry}

\begin{Entry}{奔赴}{8,9}{⼤,⾛}
  \begin{Phonetics}{奔赴}{ben1fu4}[][HSK 7-9]
    \definition{v.}{correr para; apressar"-se para; correr em direção a (um certo destino)}
  \end{Phonetics}
\end{Entry}

\begin{Entry}{奔跑}{8,12}{⼤,⾜}
  \begin{Phonetics}{奔跑}{ben1pao3}[][HSK 6]
    \definition{v.}{correr; correr muito rápido, com uma gama de aplicações mais ampla do que 奔驰, usado principalmente na linguagem falada}
  \seealsoref{奔驰}{ben1chi2}
  \end{Phonetics}
\end{Entry}

%%%%%%%%%% 妹 %%%%%%%%%%
\subsection*{妹}\addcontentsline{loh}{figure}{妹}

\begin{Entry}{妹}{8}{⼥}
  \begin{Phonetics}{妹}{mei4}
    \definition*{s.}{Sobrenome: Mei}
    \definition[个]{s.}{irmã mais nova | parente do sexo feminino da mesma geração | jovem garota; jovem mulher ou menina}
  \seealsoref{妹妹}{mei4mei5}
  \end{Phonetics}
\end{Entry}

\begin{Entry}{妹夫}{8,4}{⼥,⼤}
  \begin{Phonetics}{妹夫}{mei4fu5}
    \definition{s.}{marido da irmã mais nova}
  \end{Phonetics}
\end{Entry}

\begin{Entry}{妹妹}{8,8}{⼥,⼥}
  \begin{Phonetics}{妹妹}{mei4mei5}[][HSK 1]
    \definition[个]{s.}{irmã mais nova}
  \end{Phonetics}
\end{Entry}

%%%%%%%%%% 妻 %%%%%%%%%%
\subsection*{妻}\addcontentsline{loh}{figure}{妻}

\begin{Entry}{妻}{8}{⼥}
  \begin{Phonetics}{妻}{qi1}
    \definition{s.}{esposa}
  \end{Phonetics}
  \begin{Phonetics}{妻}{qi4}
    \definition{v.}{casar uma mulher com (alguém)}
  \end{Phonetics}
\end{Entry}

\begin{Entry}{妻子}{8,3}{⼥,⼦}
  \begin{Phonetics}{妻子}{qi1zi3}
    \definition[个]{s.}{esposa e filhos; (chinês antigo) refere"-se a esposas, filhos e filhas}
  \end{Phonetics}
  \begin{Phonetics}{妻子}{qi1zi5}[][HSK 4]
    \definition[个]{s.}{esposa (não é usado como um termo carinhoso)}
  \end{Phonetics}
\end{Entry}

%%%%%%%%%% 始 %%%%%%%%%%
\subsection*{始}\addcontentsline{loh}{figure}{始}

\begin{Entry}{始}{8}{⼥}
  \begin{Phonetics}{始}{shi3}
    \definition*{s.}{Sobrenome: Shi}
    \definition{adv.}{somente então; não\dots até}
    \definition{s.}{começo; início}
    \definition{v.}{começar; iniciar}
  \end{Phonetics}
\end{Entry}

\begin{Entry}{始终}{8,8}{⼥,⽷}
  \begin{Phonetics}{始终}{shi3zhong1}[][HSK 3]
    \definition{adv.}{sempre; o tempo todo; durante todo; do começo ao fim; indica continuidade do início ao fim}
    \definition{s.}{todo o processo do começo ao fim}
  \end{Phonetics}
\end{Entry}

%%%%%%%%%% 姐 %%%%%%%%%%
\subsection*{姐}\addcontentsline{loh}{figure}{姐}

\begin{Entry}{姐}{8}{⼥}
  \begin{Phonetics}{姐}{jie3}
    \definition[个,位]{s.}{irmã mais velha; irmã | termo genérico para mulheres jovens | mulheres da mesma geração que são mais velhas do que você (geralmente não inclui aquelas que podem ser chamadas de cunhadas) | um título respeitoso para mulheres jovens profissionais em determinados cargos}
  \seealsoref{姐姐}{jie3jie5}
  \end{Phonetics}
\end{Entry}

\begin{Entry}{姐夫}{8,4}{⼥,⼤}
  \begin{Phonetics}{姐夫}{jie3fu5}
    \definition{s.}{marido da irmã mais velha}
  \end{Phonetics}
\end{Entry}

\begin{Entry}{姐妹}{8,8}{⼥,⼥}
  \begin{Phonetics}{姐妹}{jie3mei4}[][HSK 4]
    \definition[个]{s.}{irmãs}
  \end{Phonetics}
\end{Entry}

\begin{Entry}{姐姐}{8,8}{⼥,⼥}
  \begin{Phonetics}{姐姐}{jie3jie5}[][HSK 1]
    \definition[个]{s.}{irmã mais velha}
  \end{Phonetics}
\end{Entry}

%%%%%%%%%% 姑 %%%%%%%%%%
\subsection*{姑}\addcontentsline{loh}{figure}{姑}

\begin{Entry}{姑}{8}{⼥}
  \begin{Phonetics}{姑}{gu1}
    \definition{adv.}{provisoriamente; por enquanto}
    \definition[个,位,名,些]{s.}{irmã do pai; tia | irmã do marido; cunhada | mãe do marido; sogra | freira; mulher que exerce uma ocupação religiosa | a irmã do pai de alguém | mulheres jovens (no campo)}
  \end{Phonetics}
\end{Entry}

\begin{Entry}{姑且}{8,5}{⼥,⼀}
  \begin{Phonetics}{姑且}{gu1qie3}
    \definition{adv.}{provisoriamente; por enquanto; temporariamente; indica temporário}
  \end{Phonetics}
\end{Entry}

\begin{Entry}{姑姑}{8,8}{⼥,⼥}
  \begin{Phonetics}{姑姑}{gu1gu5}[][HSK 6]
    \definition[个,位,名]{s.}{tia; tia paterna}
  \end{Phonetics}
\end{Entry}

\begin{Entry}{姑娘}{8,10}{⼥,⼥}
  \begin{Phonetics}{姑娘}{gu1niang5}[][HSK 3]
    \definition[位,名,个,些]{s.}{menina; jovem senhora; mulher solteira | filha}
  \end{Phonetics}
\end{Entry}

%%%%%%%%%% 姓 %%%%%%%%%%
\subsection*{姓}\addcontentsline{loh}{figure}{姓}

\begin{Entry}{姓}{8}{⼥}
  \begin{Phonetics}{姓}{xing4}[][HSK 2]
    \definition[个]{s.}{sobrenome; nome de família; um caractere que representa um sistema familiar, os chineses colocam o sobrenome em primeiro lugar e o nome em segundo}
    \definition{v.}{ter como sobrenome; tratar um ou mais caracteres como sobrenome}
  \end{Phonetics}
\end{Entry}

\begin{Entry}{姓氏}{8,4}{⼥,⽒}
  \begin{Phonetics}{姓氏}{xing4shi4}
    \definition{s.}{sobrenome}
  \end{Phonetics}
\end{Entry}

\begin{Entry}{姓名}{8,6}{⼥,⼝}
  \begin{Phonetics}{姓名}{xing4ming2}[][HSK 2]
    \definition{s.}{nome; nome completo; sobrenome e nome próprio}
  \end{Phonetics}
\end{Entry}

%%%%%%%%%% 委 %%%%%%%%%%
\subsection*{委}\addcontentsline{loh}{figure}{委}

\begin{Entry}{委}{8}{⼥}
  \begin{Phonetics}{委}{wei1}
    \definition{adj./adv.}{o mesmo que 逶 em 逶迤 sinuoso, curvo}
  \seealsoref{逶}{wei1}
  \seealsoref{逶迤}{wei1yi2}
  \end{Phonetics}
  \begin{Phonetics}{委}{wei3}
    \definition*{s.}{Sobrenome: Wei}
    \definition{adj.}{indireto; desviado | apático; abatido | sinuoso; tortuoso | desanimado; apático; sem inspiração}
    \definition{adv.}{realmente; certamente; na verdade}
    \definition{s.}{membro do comitê | comitê; comissão; conselho}
    \definition{v.}{confiar; nomear |  jogar fora; deixar de lado | culpar os outros | confiar | descartar; abandonar | mudar; empurrar | acumular}
  \end{Phonetics}
\end{Entry}

\begin{Entry}{委内瑞拉}{8,4,13,8}{⼥,⼌,⽟,⼿}
  \begin{Phonetics}{委内瑞拉}{wei3nei4rui4la1}
    \definition*{s.}{Venezuela}
  \end{Phonetics}
\end{Entry}

\begin{Entry}{委托}{8,6}{⼥,⼿}
  \begin{Phonetics}{委托}{wei3tuo1}[][HSK 5]
    \definition{v.}{confiar; confiar uma tarefa a outra pessoa ou instituição (para que seja realizada)}
  \end{Phonetics}
\end{Entry}

\begin{Entry}{委员}{8,7}{⼥,⼝}
  \begin{Phonetics}{委员}{wei3yuan2}[][HSK 7-9]
    \definition[个,位,名]{s.}{membro; membro do comitê; membro de uma comissão; membros de organizações de liderança coletiva, tais como instituições, grupos ou escolas; membros de organizações especializadas criadas para realizar determinadas tarefas | enviado com comissão especial; uma pessoa com responsabilidade atribuída a uma tarefa específica}
  \end{Phonetics}
\end{Entry}

\begin{Entry}{委员会}{8,7,6}{⼥,⼝,⼈}
  \begin{Phonetics}{委员会}{wei3yuan2hui4}[][HSK 7-9]
    \definition[个]{s.}{conselho; comitê; comissão; organizações de liderança coletiva em partidos políticos, grupos, agências e escolas | comitê; nome do departamento ou agência governamental | comitê; organizações especializadas estabelecidas por agências, grupos, escolas, etc. para completar certas tarefas}
  \synonymref{部门}{bu4men2}
  \synonymref{机构}{ji1gou4}
  \end{Phonetics}
\end{Entry}

\begin{Entry}{委屈}{8,8}{⼥,⼫}
  \begin{Phonetics}{委屈}{wei3qu5}[][HSK 7-9]
    \definition{adj.}{injustiçado; ofendido}
    \definition[点,次]{s.}{reclamação; tratamento injusto, injustiça ou maus tratos, enfatizando uma experiência ou estado negativo, triste e injusto}
    \definition{v.}{sentir"-se injustiçado; cuidar de uma queixa; sofrer com a injustiça; fazer alguém se sentir injustiçado; tratar alguém imerecidamente}
  \synonymref{无辜}{wu2gu1}
  \end{Phonetics}
\end{Entry}

\begin{Entry}{委婉}{8,11}{⼥,⼥}
  \begin{Phonetics}{委婉}{wei3wan3}[][HSK 7-9]
    \definition{adj.}{diplomático; indireto (de palavras); rítmico (de voz); a linguagem usada para descrevê"-lo não é muito direta ou o som é alto ou baixo, muito bonito}
  \synonymref{含蓄}{han2xu4}
  \antonymref{坦白}{tan3bai2}
  \end{Phonetics}
\end{Entry}

%%%%%%%%%% 季 %%%%%%%%%%
\subsection*{季}\addcontentsline{loh}{figure}{季}

\begin{Entry}{季}{8}{⼦}
  \begin{Phonetics}{季}{ji4}[][HSK 4]
    \definition*{s.}{Sobrenome: Ji}
    \definition{s.}{estação; o ano é dividido em quatro estações, primavera, verão, outono e inverno, e uma estação dura três meses | temporada | o fim de uma era | o último mês de uma temporada | o quarto ou mais novo entre irmãos; último na ordem de precedência}
  \end{Phonetics}
\end{Entry}

\begin{Entry}{季节}{8,5}{⼦,⾋}
  \begin{Phonetics}{季节}{ji4jie2}[][HSK 4]
    \definition[个]{s.}{estação (clima); um período característico do ano}
  \end{Phonetics}
\end{Entry}

\begin{Entry}{季度}{8,9}{⼦,⼴}
  \begin{Phonetics}{季度}{ji4du4}[][HSK 4]
    \definition[个]{s.}{trimestre; período de tempo trimestral}
  \end{Phonetics}
\end{Entry}

%%%%%%%%%% 孤 %%%%%%%%%%
\subsection*{孤}\addcontentsline{loh}{figure}{孤}

\begin{Entry}{孤}{8}{⼦}
  \begin{Phonetics}{孤}{gu1}
    \definition*{s.}{Sobrenome: Gu}
    \definition{adj.}{sozinho; solitário; isolado}
    \definition{pron.}{eu; meu humilde eu (usado por príncipes feudais); título autoproclamado dos príncipes feudais}
    \definition[个,名,位]{s.}{órfão}
  \end{Phonetics}
\end{Entry}

\begin{Entry}{孤儿}{8,2}{⼦,⼉}
  \begin{Phonetics}{孤儿}{gu1'er2}[][HSK 6]
    \definition[个,名,位]{s.}{órfão; criança sem pais; crianças que perderam os pais}
  \end{Phonetics}
\end{Entry}

\begin{Entry}{孤立}{8,5}{⼦,⽴}
  \begin{Phonetics}{孤立}{gu1li4}[][HSK 7-9]
    \definition{adj.}{isolado; condenado ao ostracismo; descreve a falta de ajuda e simpatia}
    \definition{v.}{isolar; ostracizar; privar uma pessoa de ajuda, apoio e confiança}
  \end{Phonetics}
\end{Entry}

\begin{Entry}{孤单}{8,8}{⼦,⼗}
  \begin{Phonetics}{孤单}{gu1dan1}[][HSK 7-9]
    \definition{adj.}{sozinho; solitário | fraco; inadequado; descreve um pequeno número de pessoas e poder fraco}
  \end{Phonetics}
\end{Entry}

\begin{Entry}{孤陋寡闻}{8,8,14,9}{⼦,⾩,⼧,⾨}
  \begin{Phonetics}{孤陋寡闻}{gu1lou4-gua3wen2}[][HSK 7-9]
    \definition{expr.}{ignorante e mal informado | ignorante e inexperiente | mal informado e tacanho}
  \end{Phonetics}
\end{Entry}

\begin{Entry}{孤独}{8,9}{⼦,⽝}
  \begin{Phonetics}{孤独}{gu1du2}[][HSK 6]
    \definition{adj.}{sozinho; solitário}
  \end{Phonetics}
\end{Entry}

\begin{Entry}{孤零零}{8,13,13}{⼦,⾬,⾬}
  \begin{Phonetics}{孤零零}{gu1ling2ling2}[][HSK 7-9]
    \definition{adj.}{solitário; sozinho; completamente sozinho; sem apoio ou companhia}
  \end{Phonetics}
\end{Entry}

%%%%%%%%%% 学 %%%%%%%%%%
\subsection*{学}\addcontentsline{loh}{figure}{学}

\begin{Entry}{学}{8}{⼦}
  \begin{Phonetics}{学}{xue2}[][HSK 1]
    \definition[所]{s.}{aprendizagem; conhecimento; sabedoria; erudição | objeto de estudo; ramo do conhecimento | escola; faculdade | teoria; doutrina}
    \definition{v.}{estudar; aprender | imitar; copiar}
  \end{Phonetics}
\end{Entry}

\begin{Entry}{学习}{8,3}{⼦,⼄}
  \begin{Phonetics}{学习}{xue2xi2}[][HSK 1]
    \definition{s.}{estudo}
    \definition{v.}{estudar; aprender; adquirir conhecimentos ou habilidades através da leitura, da audição, da pesquisa e da prática}
  \end{Phonetics}
\end{Entry}

\begin{Entry}{学分}{8,4}{⼦,⼑}
  \begin{Phonetics}{学分}{xue2fen1}[][HSK 4]
    \definition{s.}{créditos de um curso; uma unidade de medida do peso e do tempo do curso no ensino superior; cada curso vale um crédito para uma aula por semana durante um semestre; alunos devem concluir o número necessário de créditos para se formar}
  \end{Phonetics}
\end{Entry}

\begin{Entry}{学术}{8,5}{⼦,⽊}
  \begin{Phonetics}{学术}{xue2shu4}[][HSK 4]
    \definition[种]{s.}{aprendizagem; aprendizado; ciências; aprendizado sistemático e especializado}
  \end{Phonetics}
\end{Entry}

\begin{Entry}{学生}{8,5}{⼦,⽣}
  \begin{Phonetics}{学生}{xue2sheng5}[][HSK 1]
    \definition{s.}{aluno; estudante; pupilo}
  \end{Phonetics}
\end{Entry}

\begin{Entry}{学生证}{8,5,7}{⼦,⽣,⾔}
  \begin{Phonetics}{学生证}{xue2sheng5zheng4}
    \definition{s.}{cartão de identidade de estudante}
  \end{Phonetics}
\end{Entry}

\begin{Entry}{学会}{8,6}{⼦,⼈}
  \begin{Phonetics}{学会}{xue2 hui4}[][HSK 6]
    \definition[个]{s.}{sociedade; instituto; sociedade científica; um grupo acadêmico composto por pessoas que estudam um determinado assunto, como a Sociedade de Física, a Sociedade de Biologia, etc.}
    \definition{v.}{aprender; dominar; aprender e aplicar}
  \end{Phonetics}
\end{Entry}

\begin{Entry}{学好}{8,6}{⼦,⼥}
  \begin{Phonetics}{学好}{xue2hao3}
    \definition{v.}{seguir bons exemplos | aprender bem}
  \end{Phonetics}
\end{Entry}

\begin{Entry}{学年}{8,6}{⼦,⼲}
  \begin{Phonetics}{学年}{xue2nian2}[][HSK 4]
    \definition{s.}{ano letivo; ano acadêmico}
  \end{Phonetics}
\end{Entry}

\begin{Entry}{学问}{8,6}{⼦,⾨}
  \begin{Phonetics}{学问}{xue2wen5}[][HSK 4]
    \definition[门,种,个,项]{s.}{aprendizado, conhecimento, erudição; a compreensão correta do mundo objetivo que alguém tem | conhecimento; aprendizado sistemático; conhecimento sistemático sobre algo ou uma ciência que pode ser aprendido em um livro ou em uma experiência prática}
  \end{Phonetics}
\end{Entry}

\begin{Entry}{学位}{8,7}{⼦,⼈}
  \begin{Phonetics}{学位}{xue2wei4}[][HSK 5]
    \definition[个]{s.}{grau; grau acadêmico; título concedido com base no nível acadêmico profissional, como doutorado, mestrado, etc.}
  \end{Phonetics}
\end{Entry}

\begin{Entry}{学员}{8,7}{⼦,⼝}
  \begin{Phonetics}{学员}{xue2yuan2}[][HSK 6]
    \definition[位,名,批,个]{s.}{estudante; estagiário; geralmente se refere a pessoas que estudam em escolas ou cursos de treinamento diferentes de faculdades, escolas de ensino médio e escolas primárias}
  \end{Phonetics}
\end{Entry}

\begin{Entry}{学时}{8,7}{⼦,⽇}
  \begin{Phonetics}{学时}{xue2shi2}[][HSK 4]
    \definition{s.}{hora-aula; hora de aula; período}
  \end{Phonetics}
\end{Entry}

\begin{Entry}{学者}{8,8}{⼦,⽼}
  \begin{Phonetics}{学者}{xue2zhe3}[][HSK 5]
    \definition[位]{s.}{erudito; homem culto; pessoas que fazem pesquisas acadêmicas geralmente se referem àquelas que alcançaram certo sucesso acadêmico}
  \end{Phonetics}
\end{Entry}

\begin{Entry}{学科}{8,9}{⼦,⽲}
  \begin{Phonetics}{学科}{xue2ke1}[][HSK 5]
    \definition[门,级]{s.}{ramo do aprendizado; disciplina | disciplina escolar; curso de estudo | cursos teóricos oferecidos em treinamento militar ou físico | disciplina acadêmica | curso | assunto; tema}
  \antonymref{术科}{shu4ke1}
  \end{Phonetics}
\end{Entry}

\begin{Entry}{学费}{8,9}{⼦,⾙}
  \begin{Phonetics}{学费}{xue2fei4}[][HSK 3]
    \definition[笔]{s.}{mensalidade (taxa); prêmio; taxas que os alunos devem pagar para estudar na escola, conforme estabelecido pela escola | preço pelo que se aprendeu ao custo do próprio bolso; a metáfora do preço a pagar para obter uma determinada experiência | custo; preço; todas as despesas necessárias durante o período de estudos do aluno}
  \end{Phonetics}
\end{Entry}

\begin{Entry}{学院}{8,9}{⼦,⾩}
  \begin{Phonetics}{学院}{xue2yuan4}[][HSK 1]
    \definition[个,所]{s.}{academia; instituto; um tipo de instituição de ensino superior que se concentra em uma determinada área de especialização, como faculdades de engenharia, faculdades de música, faculdades de educação, etc.}
  \end{Phonetics}
\end{Entry}

\begin{Entry}{学校}{8,10}{⼦,⽊}
  \begin{Phonetics}{学校}{xue2xiao4}[][HSK 1]
    \definition[所,个]{s.}{escola; instituição de ensino}
  \end{Phonetics}
\end{Entry}

\begin{Entry}{学期}{8,12}{⼦,⽉}
  \begin{Phonetics}{学期}{xue2qi1}[][HSK 2]
    \definition[个,段]{s.}{semestre; período escolar; um ano acadêmico é dividido em dois semestres, um semestre do início do outono até as férias de inverno e um semestre do início da primavera até as férias de verão}
  \end{Phonetics}
\end{Entry}

%%%%%%%%%% 宗 %%%%%%%%%%
\subsection*{宗}\addcontentsline{loh}{figure}{宗}

\begin{Entry}{宗}{8}{⼧}
  \begin{Phonetics}{宗}{zong1}
    \definition*{s.}{Sobrenome: Zong}
    \definition{adj.}{do mesmo clã; da mesma família}
    \definition{clas.}{usado para matérias, cargas, etc.}
    \definition{s.}{ancestral; antepassado | clã; família | seita; facção; escola | objetivo principal; propósito | modelo; grande mestre | Obsoleto: unidade administrativa no Tibete, equivalente a um condado | templo ancestral}
    \definition{v.}{(no trabalho acadêmico ou artístico) tomar como modelo; modelar"-se em}
  \end{Phonetics}
\end{Entry}

\begin{Entry}{宗教}{8,11}{⼧,⽁}
  \begin{Phonetics}{宗教}{zong1jiao4}[][HSK 6]
    \definition[种]{s.}{religião; uma ideologia social é um reflexo ilusório do mundo objetivo, exigindo que as pessoas acreditem em Deus, no Xintoísmo, em espíritos, no carma, etc., e que depositem suas esperanças no chamado céu ou vida após a morte}
  \end{Phonetics}
\end{Entry}

%%%%%%%%%% 官 %%%%%%%%%%
\subsection*{官}\addcontentsline{loh}{figure}{官}

\begin{Entry}{官}{8}{⼧}
  \begin{Phonetics}{官}{guan1}[][HSK 4]
    \definition*{s.}{Sobrenome: Guan}
    \definition{adj.}{propriedade do governo; pertencente ao governo ou ao público | público}
    \definition[个,位,名,些]{s.}{funcionário do governo; oficial; servidor público; titular de cargo; funcionário público nomeado acima de um determinado nível | órgão (parte do tecido do corpo)}
  \end{Phonetics}
\end{Entry}

\begin{Entry}{官方}{8,4}{⼧,⽅}
  \begin{Phonetics}{官方}{guan1fang1}[][HSK 4]
    \definition{s.}{autoridade; (do ou pelo) governo | oficial (de uma organização ou instituição)}
  \end{Phonetics}
\end{Entry}

\begin{Entry}{官司}{8,5}{⼧,⼝}
  \begin{Phonetics}{官司}{guan1si5}[][HSK 6]
    \definition[场,个]{s.}{ação judicial}
  \end{Phonetics}
\end{Entry}

\begin{Entry}{官吏}{8,6}{⼧,⼝}
  \begin{Phonetics}{官吏}{guan1li4}[][HSK 7-9]
    \definition{s.}{funcionários do governo | burocrata | oficial}
  \end{Phonetics}
\end{Entry}

\begin{Entry}{官兵}{8,7}{⼧,⼋}
  \begin{Phonetics}{官兵}{guan1bing1}[][HSK 7-9]
    \definition{s.}{oficiais e soldados | Obsoleto: tropas governamentais}
  \end{Phonetics}
\end{Entry}

\begin{Entry}{官员}{8,7}{⼧,⼝}
  \begin{Phonetics}{官员}{guan1yuan2}[][HSK 7-9]
    \definition[名,位]{s.}{oficial; funcionários do governo de um determinado nível}
  \end{Phonetics}
\end{Entry}

\begin{Entry}{官桂}{8,10}{⼧,⽊}
  \begin{Phonetics}{官桂}{guan1gui4}
    \definition{s.}{canela; também escrito como 肉桂}
  \seealsoref{肉桂}{rou4gui4}
  \end{Phonetics}
\end{Entry}

\begin{Entry}{官僚}{8,14}{⼧,⼈}
  \begin{Phonetics}{官僚}{guan1liao2}[][HSK 7-9]
    \definition{s.}{burocrata | burocracia | oficial}
  \end{Phonetics}
\end{Entry}

\begin{Entry}{官僚主义}{8,14,5,3}{⼧,⼈,⼂,⼂}
  \begin{Phonetics}{官僚主义}{guan1liao2 zhu3yi4}[][HSK 7-9]
    \definition{s.}{burocracia; burocratismo}
  \end{Phonetics}
\end{Entry}

%%%%%%%%%% 定 %%%%%%%%%%
\subsection*{定}\addcontentsline{loh}{figure}{定}

\begin{Entry}{定}{8}{⼧}
  \begin{Phonetics}{定}{ding4}[][HSK 4]
    \definition{adj.}{calmo; estável}
    \definition{adv.}{certamente; com certeza; definitivamente; espressa certeza ou necessidade}
    \definition{v.}{decidir; fixar; definir; determinar; ter certeza | acalmar; estabilizar; tornar estável | assinar (um jornal, etc.); reservar (assentos, ingressos, etc.); encomendar (mercadorias, etc.)}
  \end{Phonetics}
\end{Entry}

\begin{Entry}{定义}{8,3}{⼧,⼂}
  \begin{Phonetics}{定义}{ding4yi4}[][HSK 7-9]
    \definition[个,种]{s.}{definição; delimitação; uma descrição precisa e concisa das características essenciais de uma coisa ou da conotação e extensão de um conceito}
    \definition{v.}{definir}
  \end{Phonetics}
\end{Entry}

\begin{Entry}{定为}{8,4}{⼧,⼂}
  \begin{Phonetics}{定为}{ding4wei2}[][HSK 7-9]
    \definition{v.}{prescrever como; estar marcado para}
  \end{Phonetics}
\end{Entry}

\begin{Entry}{定心丸}{8,4,3}{⼧,⼼,⼂}
  \begin{Phonetics}{定心丸}{ding4xin1wan2}[][HSK 7-9]
    \definition{s.}{algo que tranquiliza a mente; alívio | algo capaz de tranquilizar a mente de alguém; algo que acalma os nervos; tranquiliza a mente (paz); palavras ou ações que podem acalmar pensamentos e emoções}
  \end{Phonetics}
\end{Entry}

\begin{Entry}{定价}{8,6}{⼧,⼈}
  \begin{Phonetics}{定价}{ding4jia4}[][HSK 6]
    \definition{s.}{fixação de preços; preço especificado}
    \definition{v.}{fixar um preço | fazer um preço; definir um preço}
  \end{Phonetics}
\end{Entry}

\begin{Entry}{定向}{8,6}{⼧,⼝}
  \begin{Phonetics}{定向}{ding4xiang4}[][HSK 7-9]
    \definition{adj.}{direcional; orientado}
    \definition{s.}{orientação; sentido de orientação; orientação predeterminada}
    \definition{v.}{orientar}
  \end{Phonetics}
\end{Entry}

\begin{Entry}{定论}{8,6}{⼧,⾔}
  \begin{Phonetics}{定论}{ding4lun4}[][HSK 7-9]
    \definition{s.}{conclusão final | veredito final}
    \definition{v.}{concluir; chegar a uma conclusão (ou julgamento)}
  \end{Phonetics}
\end{Entry}

\begin{Entry}{定位}{8,7}{⼧,⼈}
  \begin{Phonetics}{定位}{ding4wei4}[][HSK 6]
    \definition{s.}{posição; localização; posição medida ou definida}
    \definition{v.}{localizar; posicionar; orientar; avaliar algo; usar instrumentos para determinar a localização de objetos; definir o \emph{status} das coisas}
  \end{Phonetics}
\end{Entry}

\begin{Entry}{定时}{8,7}{⼧,⽇}
  \begin{Phonetics}{定时}{ding4shi2}[][HSK 6]
    \definition{s.}{em um horário fixo; em intervalos regulares}
    \definition{v.}{cronometrar; fixar um tempo (para fazer algo)}
  \end{Phonetics}
\end{Entry}

\begin{Entry}{定居}{8,8}{⼧,⼫}
  \begin{Phonetics}{定居}{ding4/ju1}[][HSK 7-9]
    \definition{v.+compl.}{estabelecer"-se; fixar residência; viver permanentemente em um determinado lugar}
  \end{Phonetics}
\end{Entry}

\begin{Entry}{定金}{8,8}{⼧,⾦}
  \begin{Phonetics}{定金}{ding4jin1}[][HSK 7-9]
    \definition{s.}{sinal; depósito; o mesmo que 订金}
  \seealsoref{订金}{ding4jin1}
  \end{Phonetics}
\end{Entry}

\begin{Entry}{定做}{8,11}{⼧,⼈}
  \begin{Phonetics}{定做}{ding4zuo4}[][HSK 7-9]
    \definition{v.}{ter algo feito sob encomenda (medida); feito sob medida; personalizar}
  \end{Phonetics}
\end{Entry}

\begin{Entry}{定期}{8,12}{⼧,⽉}
  \begin{Phonetics}{定期}{ding4qi1}[][HSK 3]
    \definition{adj.}{regular; periódico; em intervalos regulares; com prazo determinado; por tempo limitado}
    \definition{v.}{fixar (definir) uma data; determinar a data; confirmar a data}
  \end{Phonetics}
\end{Entry}

%%%%%%%%%% 宝 %%%%%%%%%%
\subsection*{宝}\addcontentsline{loh}{figure}{宝}

\begin{Entry}{宝}{8}{⼧}
  \begin{Phonetics}{宝}{bao3}[][HSK 4]
    \definition*{s.}{Sobrenome: Bao}
    \definition{adj.}{antigo; precioso; estimado}
    \definition{pron.}{estimado; um termo educado usado para se referir à família, loja, etc. de alguém}
    \definition[个,件]{s.}{tesouro; objeto estimado; coisa preciosa | dinheiro; moeda; moeda antiga com furo quadrado no centro; moeda de prata}
  \end{Phonetics}
\end{Entry}

\begin{Entry}{宝贝}{8,4}{⼧,⾙}
  \begin{Phonetics}{宝贝}{bao3bei4}[][HSK 4]
    \definition{adj.}{excêntrico; estranho; imprestável; um termo depreciativo para uma pessoa incompetente ou ridícula}
    \definition[个,件]{s.}{tesouro; objeto estimado; coisa preciosa | querida; \emph{darling}; \emph{baby}; apelido para crianças}
  \end{Phonetics}
\end{Entry}

\begin{Entry}{宝石}{8,5}{⼧,⽯}
  \begin{Phonetics}{宝石}{bao3shi2}[][HSK 4]
    \definition[颗,枚,块,粒]{s.}{gema; jóia; pedra preciosa; mineral precioso que tem um brilho lindo e uma dureza de mais de sete graus, não é afetado pela atmosfera ou por produtos químicos e pode ser usado como decoração, suporte de instrumentos ou abrasivos}
  \end{Phonetics}
\end{Entry}

\begin{Entry}{宝库}{8,7}{⼧,⼴}
  \begin{Phonetics}{宝库}{bao3ku4}[][HSK 7-9]
    \definition[座,个]{s.}{tesouro; casa de tesouro; um lugar onde coisas preciosas são armazenadas (frequentemente usado metaforicamente)}[图书馆是知识的宝库。===A biblioteca é um tesouro de conhecimento.]
  \end{Phonetics}
\end{Entry}

\begin{Entry}{宝宝}{8,8}{⼧,⼧}
  \begin{Phonetics}{宝宝}{bao3bao5}[][HSK 4]
    \definition[个,位]{s.}{querida; \emph{darling}; \emph{baby}; apelido para crianças}
  \end{Phonetics}
\end{Entry}

\begin{Entry}{宝贵}{8,9}{⼧,⾙}
  \begin{Phonetics}{宝贵}{bao3gui4}[][HSK 4]
    \definition{adj.}{precioso; extremamente valioso, muito raro, pode ser usado para descrever coisas específicas, também pode ser usado para descrever coisas abstratas | valioso; como um tesouro}
  \end{Phonetics}
\end{Entry}

\begin{Entry}{宝藏}{8,17}{⼧,⾋}
  \begin{Phonetics}{宝藏}{bao3zang4}[][HSK 7-9]
    \definition[座,个]{s.}{depósitos preciosos (minerais); tesouros ou riquezas armazenadas, principalmente minerais}
  \end{Phonetics}
\end{Entry}

%%%%%%%%%% 实 %%%%%%%%%%
\subsection*{实}\addcontentsline{loh}{figure}{实}

\begin{Entry}{实}{8}{⼧}
  \begin{Phonetics}{实}{shi2}
    \definition{adj.}{sólido; cheio por dentro; sem espaços vazios | verdadeiro; real; atual; sincero | forte; eficaz; concreto; real}
    \definition{adv.}{verdadeiramente; realmente; de fato; originalmente}
    \definition{s.}{fato; realidade | semente; fruto}
    \definition{v.}{preencher}
  \antonymref{虚}{xu1}
  \end{Phonetics}
\end{Entry}

\begin{Entry}{实力}{8,2}{⼧,⼒}
  \begin{Phonetics}{实力}{shi2li4}[][HSK 3]
    \definition{s.}{força real; geralmente se refere à força militar e econômica de um país, grupo ou indivíduo, e também se refere à capacidade de um indivíduo ou grupo em uma competição}
  \end{Phonetics}
\end{Entry}

\begin{Entry}{实习}{8,3}{⼧,⼄}
  \begin{Phonetics}{实习}{shi2xi2}[][HSK 2]
    \definition{s.}{estagiário; prática; estágio}
    \definition{v.}{aplicar e testar os conhecimentos teóricos aprendidos no trabalho prático, a fim de exercitar a capacidade profissional}
  \end{Phonetics}
\end{Entry}

\begin{Entry}{实用}{8,5}{⼧,⽤}
  \begin{Phonetics}{实用}{shi2yong4}[][HSK 4]
    \definition{adj.}{prático; pragmático; funcional; atende aos requisitos reais da aplicação}
    \definition{v.}{colocar em uso prático}
  \end{Phonetics}
\end{Entry}

\begin{Entry}{实在}{8,6}{⼧,⼟}
  \begin{Phonetics}{实在}{shi2zai5}[][HSK 2]
    \definition{adj.}{honesto; sincero | verdadeiro; honesto; realista; não é falso, não é enganador}
    \definition{adv.}{verdadeiramente; de fato; na verdade; usado para reforçar o tom afirmativo, enfatizando que a situação é realmente assim}
  \end{Phonetics}
\end{Entry}

\begin{Entry}{实地}{8,6}{⼧,⼟}
  \begin{Phonetics}{实地}{shi2di4}[][HSK 7-9]
    \definition{adv.}{de fato; realmente; realmente muito sério}
    \definition{s.}{campo; no local}
  \end{Phonetics}
\end{Entry}

\begin{Entry}{实行}{8,6}{⼧,⾏}
  \begin{Phonetics}{实行}{shi2xing2}[][HSK 3]
    \definition{v.}{praticar; implementar; executar; colocar em prática; realizar (programa, política, plano, etc.) por meio de ação}
  \end{Phonetics}
\end{Entry}

\begin{Entry}{实体}{8,7}{⼧,⼈}
  \begin{Phonetics}{实体}{shi2ti3}[][HSK 7-9]
    \definition[种,个]{s.}{substância; um conceito da filosofia pré-marxista, que sustenta que a substância é o fundamento e a origem imutáveis de todas as coisas; o ``espírito'' dos idealistas e a ``matéria'' dos materialistas metafísicos são ambos tais substâncias | entidade; geralmente se refere a coisas objetivas que existem independentemente e possuem certos atributos}
  \end{Phonetics}
\end{Entry}

\begin{Entry}{实况}{8,7}{⼧,⼎}
  \begin{Phonetics}{实况}{shi2kuang4}[][HSK 7-9]
    \definition{s.}{o que está realmente acontecendo; evento atual; ao vivo | ao vivo (ex.: transmissão ou gravação) | cena | a situação real}
  \synonymref{实际}{shi2ji4}
  \end{Phonetics}
\end{Entry}

\begin{Entry}{实际}{8,7}{⼧,⾩}
  \begin{Phonetics}{实际}{shi2ji4}[][HSK 2]
    \definition{adj.}{real; efetivo; concreto; prático | factual; prático; realista; de acordo com os fatos}
    \definition{s.}{realidade; prática; coisas e situações que existem objetivamente}
  \end{Phonetics}
\end{Entry}

\begin{Entry}{实际上}{8,7,3}{⼧,⾩,⼀}
  \begin{Phonetics}{实际上}{shi2ji4shang5}[][HSK 3]
    \definition{adv.}{de fato; na verdade}
  \end{Phonetics}
\end{Entry}

\begin{Entry}{实事求是}{8,8,7,9}{⼧,⼅,⽔,⽇}
  \begin{Phonetics}{实事求是}{shi2shi4-qiu2shi4}[][HSK 7-9]
    \definition{expr.}{``Buscando a verdade nos fatos.''; ser prático e realista; devemos partir da situação real, sem exagerar nem minimizar, e abordar e lidar com os problemas de forma correta}
  \synonymref{恰如其分}{qia4ru2-qi2fen4}
  \antonymref{弄虚作假}{nong4xu1-zuo4jia3}
  \antonymref{有名无实}{you3ming2wu2shi2}
  \end{Phonetics}
\end{Entry}

\begin{Entry}{实物}{8,8}{⼧,⽜}
  \begin{Phonetics}{实物}{shi2wu4}[][HSK 7-9]
    \definition[件]{s.}{entidade; matéria; coisas reais e concretas | objeto material; objeto físico; aplicações práticas}
  \end{Phonetics}
\end{Entry}

\begin{Entry}{实现}{8,8}{⼧,⾒}
  \begin{Phonetics}{实现}{shi2xian4}[][HSK 2]
    \definition{v.}{alcançar; atingir; realizar; concretizar; tornar (ideais, planos, etc.) realidade}
  \end{Phonetics}
\end{Entry}

\begin{Entry}{实话}{8,8}{⼧,⾔}
  \begin{Phonetics}{实话}{shi2hua4}[][HSK 7-9]
    \definition[句]{s.}{verdade}
  \antonymref{谎话}{huang3hua4}
  \antonymref{谎言}{huang3yan2}
  \end{Phonetics}
\end{Entry}

\begin{Entry}{实话实说}{8,8,8,9}{⼧,⾔,⼧,⾔}
  \begin{Phonetics}{实话实说}{shi2hua4-shi2shuo1}[][HSK 7-9]
    \definition{expr.}{``Para dizer a verdade.''; falar francamente; falar sem rodeios; dizer as coisas como elas são; dizer a verdade}
  \end{Phonetics}
\end{Entry}

\begin{Entry}{实质}{8,8}{⼧,⾙}
  \begin{Phonetics}{实质}{shi2zhi4}[][HSK 7-9]
    \definition{s.}{essência; substância; natureza}
  \synonymref{本色}{ben3se4}
  \synonymref{本质}{ben3zhi4}
  \synonymref{内容}{nei4rong2}
  \synonymref{内心}{nei4xin1}
  \synonymref{实际}{shi2ji4}
  \antonymref{表面}{biao3mian4}
  \antonymref{面子}{mian4zi5}
  \antonymref{名义}{ming2yi4}
  \antonymref{现象}{xian4xiang4}
  \antonymref{形式}{xing2shi4}
  \end{Phonetics}
\end{Entry}

\begin{Entry}{实施}{8,9}{⼧,⽅}
  \begin{Phonetics}{实施}{shi2shi1}[][HSK 4]
    \definition{v.}{colocar em vigor; implementar (leis, políticas, etc.); executar; trazer (colocar) algo em vigor; fazer cumprir; colocar algo em (prática)}
  \end{Phonetics}
\end{Entry}

\begin{Entry}{实验}{8,10}{⼧,⾺}
  \begin{Phonetics}{实验}{shi2yan4}[][HSK 3]
    \definition[个,次]{s.}{teste; experimento; trabalho de laboratório}
    \definition{v.}{testar; experimentar; realizar uma operação ou se envolver em uma atividade para testar uma teoria ou hipótese científica}
  \end{Phonetics}
\end{Entry}

\begin{Entry}{实验室}{8,10,9}{⼧,⾺,⼧}
  \begin{Phonetics}{实验室}{shi2yan4shi4}[][HSK 3]
    \definition[个,间]{s.}{laboratório; salas especiais para experimentos científicos}
  \end{Phonetics}
\end{Entry}

\begin{Entry}{实惠}{8,12}{⼧,⼼}
  \begin{Phonetics}{实惠}{shi2hui4}[][HSK 5]
    \definition{adj.}{sólido; substancial; benefícios práticos}
    \definition{s.}{benefício material; benefícios tangíveis; benefícios reais}
  \end{Phonetics}
\end{Entry}

\begin{Entry}{实践}{8,12}{⼧,⾜}
  \begin{Phonetics}{实践}{shi2jian4}[][HSK 6]
    \definition{s.}{prática; filosoficamente, refere"-se às ações conscientes das pessoas para transformar a natureza e a sociedade; as atividades de produção são as atividades práticas mais básicas e também incluem atividades políticas, experimentos científicos, educação cultural, etc.}
    \definition{v.}{praticar; realizar; implementar planos e intenções em ações específicas}
  \end{Phonetics}
\end{Entry}

%%%%%%%%%% 宠 %%%%%%%%%%
\subsection*{宠}\addcontentsline{loh}{figure}{宠}

\begin{Entry}{宠}{8}{⼧}
  \begin{Phonetics}{宠}{chong3}[][HSK 7-9]
    \definition*{s.}{Sobrenome: Chong}
    \definition{v.}{mimar; estragar; conceder favor a | regalar; encontrar favor com alguém; estar nas boas graças de alguém}
  \end{Phonetics}
\end{Entry}

\begin{Entry}{宠物}{8,8}{⼧,⽜}
  \begin{Phonetics}{宠物}{chong3wu4}[][HSK 6]
    \definition[只]{s.}{animal de estimação; refere"-se a pequenos animais criados na família}
  \end{Phonetics}
\end{Entry}

\begin{Entry}{宠爱}{8,10}{⼧,⽖}
  \begin{Phonetics}{宠爱}{chong3'ai4}[][HSK 7-9]
    \definition{v.}{mimar; fazer de alguém um animal de estimação}[爷爷总是宠爱他的孙子。===O avô sempre mima o neto.]
  \end{Phonetics}
\end{Entry}

%%%%%%%%%% 审 %%%%%%%%%%
\subsection*{审}\addcontentsline{loh}{figure}{审}

\begin{Entry}{审}{8}{⼧}
  \begin{Phonetics}{审}{shen3}[][HSK 7-9]
    \definition*{s.}{Sobrenome: Shen}
    \definition{adj.}{cuidadoso; detalhado; completo}
    \definition{adv.}{Literário: realmente; de fato; como esperado}
    \definition{v.}{examinar; analizar | julgar; interrogar | Literário: saber}
  \end{Phonetics}
\end{Entry}

\begin{Entry}{审判}{8,7}{⼧,⼑}
  \begin{Phonetics}{审判}{shen3pan4}[][HSK 7-9]
    \definition{v.}{julgar; levar a julgamento; realizar um julgamento}
  \end{Phonetics}
\end{Entry}

\begin{Entry}{审批}{8,7}{⼧,⼿}
  \begin{Phonetics}{审批}{shen3pi1}[][HSK 7-9]
    \definition{v.}{examinar e aprovar; examinar e dar instruções; revisar e aprovar (planos escritos, relatórios, etc., submetidos por subordinados a superiores)}
  \end{Phonetics}
\end{Entry}

\begin{Entry}{审定}{8,8}{⼧,⼧}
  \begin{Phonetics}{审定}{shen3ding4}[][HSK 7-9]
    \definition{v.}{examinar e aprovar; examinar e finalizar | autorizar; verificar e decidir}
  \synonymref{鉴定}{jian4ding4}
  \end{Phonetics}
\end{Entry}

\begin{Entry}{审视}{8,8}{⼧,⾒}
  \begin{Phonetics}{审视}{shen3shi4}[][HSK 7-9]
    \definition{v.}{examinar; observar; olhar atentamente para algo/alguém; examinar com atenção}
  \synonymref{注视}{zhu4shi4}
  \end{Phonetics}
\end{Entry}

\begin{Entry}{审查}{8,9}{⼧,⽊}
  \begin{Phonetics}{审查}{shen3cha2}[][HSK 6]
    \definition{v.}{examinar; investigar; verificar se algo está correto e apropriado (geralmente referindo"-se a planos, propostas, escritos, qualificações pessoais, etc.); ler e avaliar (provas ou trabalhos de exame)}
  \end{Phonetics}
\end{Entry}

\begin{Entry}{审美}{8,9}{⼧,⽺}
  \begin{Phonetics}{审美}{shen3mei3}[][HSK 7-9]
    \definition{v.}{apreciar a beleza; apreciar, discernir e avaliar a beleza das coisas ou obras de arte}
  \end{Phonetics}
\end{Entry}

\begin{Entry}{审核}{8,10}{⼧,⽊}
  \begin{Phonetics}{审核}{shen3he2}[][HSK 7-9]
    \definition{v.}{verificar; examinar e confirmar; revisar e aprovar (geralmente referindo"-se a materiais escritos ou digitais)}
  \synonymref{考核}{kao3he2}
  \antonymref{放任}{fang4ren4}
  \end{Phonetics}
\end{Entry}

%%%%%%%%%% 尚 %%%%%%%%%%
\subsection*{尚}\addcontentsline{loh}{figure}{尚}

\begin{Entry}{尚}{8}{⼩}
  \begin{Phonetics}{尚}{shang4}[][HSK 7-9]
    \definition*{s.}{Sobrenome: Shang}
    \definition{adv.}{ainda}
    \definition{s.}{costume predominante; refere"-se à tendência predominante na sociedade; coisas que geralmente são admiradas pelas pessoas}
    \definition{v.}{valorizar; estimar; dar grande importância a; respeitar}
  \end{Phonetics}
\end{Entry}

\begin{Entry}{尚且}{8,5}{⼩,⼀}
  \begin{Phonetics}{尚且}{shang4 qie3}
    \definition{conj.}{nem\dots; muito menos\dots; é usado antes do verbo da primeira oração de uma frase complexa para apresentar alguns exemplos óbvios para comparação, a segunda oração frequentemente usa 何况 ou 更 para ecoar e tirar conclusões inevitáveis sobre exemplos semelhantes com diferentes graus de gravidade}
  \seealsoref{更}{geng4}
  \seealsoref{何况}{he2kuang4}
  \end{Phonetics}
\end{Entry}

\begin{Entry}{尚且……何况……}{8,5,7,7}{⼩,⼀,⼈,⼎}
  \begin{Phonetics}{尚且……何况……}{shang4qie3 he2kuang4}
    \definition{conj.}{ainda que\dots, \dots; além do mais\dots e muito menos\dots}
  \end{Phonetics}
\end{Entry}

\begin{Entry}{尚未}{8,5}{⼩,⽊}
  \begin{Phonetics}{尚未}{shang4wei4}[][HSK 7-9]
    \definition{adv.}{ainda não}[问题尚未解决。===O problema ainda não foi resolvido.]
  \end{Phonetics}
\end{Entry}

%%%%%%%%%% 居 %%%%%%%%%%
\subsection*{居}\addcontentsline{loh}{figure}{居}

\begin{Entry}{居}{8}{⼫}
  \begin{Phonetics}{居}{ju1}
    \definition*{s.}{Sobrenome: Ju}
    \definition{s.}{residência; casa | restaurante (em nomes de restaurantes)}
    \definition{v.}{residir; morar; viver | ocupar uma determinada posição; ocupar (um lugar); estar (em uma determinada posição) | reivindicar; afirmar | armazenar; guardar | ficar parado; estar parado}
  \end{Phonetics}
\end{Entry}

\begin{Entry}{居民}{8,5}{⼫,⽒}
  \begin{Phonetics}{居民}{ju1min2}[][HSK 4]
    \definition[个,户,位]{s.}{residente; habitante; pessoas que estão fixas em um único lugar}
  \end{Phonetics}
\end{Entry}

\begin{Entry}{居民楼}{8,5,13}{⼫,⽒,⽊}
  \begin{Phonetics}{居民楼}{ju1min2lou2}[][HSK 7-9]
    \definition{s.}{edifício residencial}
  \end{Phonetics}
\end{Entry}

\begin{Entry}{居住}{8,7}{⼫,⼈}
  \begin{Phonetics}{居住}{ju1zhu4}[][HSK 4]
    \definition{v.}{viver; residir; morar; habitar}
  \end{Phonetics}
\end{Entry}

\begin{Entry}{居高临下}{8,10,9,3}{⼫,⾼,⼁,⼀}
  \begin{Phonetics}{居高临下}{ju1gao1-lin2xia4}[][HSK 7-9]
    \definition{expr.}{``Olhando para baixo.''; ocupar uma posição (ou altura) dominante; olhar de cima para baixo; viver no alto e olhar para baixo; ocupar o terreno elevado; ignorar; elevar"-se acima | Figurativo: arrogância baseada na posição social de alguém}
  \end{Phonetics}
\end{Entry}

\begin{Entry}{居然}{8,12}{⼫,⽕}
  \begin{Phonetics}{居然}{ju1ran2}[][HSK 5]
    \definition{adv.}{inesperadamente; para surpresa de alguém; além da expectativa (expressão idiomática)}
    \definition{v.}{ir tão longe a ponto de; ter a impudência de; ter o descaramento de}
  \end{Phonetics}
\end{Entry}

%%%%%%%%%% 屈 %%%%%%%%%%
\subsection*{屈}\addcontentsline{loh}{figure}{屈}

\begin{Entry}{屈}{8}{⼫}
  \begin{Phonetics}{屈}{qu1}
    \definition*{s.}{Sobrenome: Qu}
    \definition[个]{s.}{injustiça; tratamento injusto | erro; queixa; injustiça}
    \definition{v.}{dobrar; curvar; encurvar | subjugar; submeter | tratar mal; tratar injustamente (ou deslealmente) | estar errado}
  \end{Phonetics}
\end{Entry}

\begin{Entry}{屈服}{8,8}{⼫,⽉}
  \begin{Phonetics}{屈服}{qu1fu2}[][HSK 7-9]
    \definition{v.}{subjugar; submeter"-se; ceder; dobrar"-se; ceder e recuar diante da pressão externa, desistir da luta}
  \end{Phonetics}
\end{Entry}

\begin{Entry}{屈原}{8,10}{⼫,⼚}
  \begin{Phonetics}{屈原}{qu1yuan2}
    \definition*{s.}{Qu Yuan, poeta, é uma figura histórica famosa na cultura chinesa que viveu durante o Período dos Reinos Combatentes (340--278 a.C.).}
  \end{Phonetics}
\end{Entry}

%%%%%%%%%% 届 %%%%%%%%%%
\subsection*{届}\addcontentsline{loh}{figure}{届}

\begin{Entry}{届}{8}{⼫}
  \begin{Phonetics}{届}{jie4}[][HSK 5]
    \definition{clas.}{sessões (de uma conferência); anos (de graduação); quantificador, ligeiramente equivalente a 次, usado para reuniões regulares ou turmas de formandos, etc.}
    \definition{v.}{vencer o prazo}
  \seealsoref{次}{ci4}
  \end{Phonetics}
\end{Entry}

\begin{Entry}{届时}{8,7}{⼫,⽇}
  \begin{Phonetics}{届时}{jie4shi2}[][HSK 7-9]
    \definition{adv.}{na ocasião; quando chegar a hora; no momento determinado; no horário combinado}
  \end{Phonetics}
\end{Entry}

%%%%%%%%%% 岭 %%%%%%%%%%
\subsection*{岭}\addcontentsline{loh}{figure}{岭}

\begin{Entry}{岭}{8}{⼭}
  \begin{Phonetics}{岭}{ling3}
    \definition{s.}{cordilheira}
  \end{Phonetics}
\end{Entry}

%%%%%%%%%% 岸 %%%%%%%%%%
\subsection*{岸}\addcontentsline{loh}{figure}{岸}

\begin{Entry}{岸}{8}{⼭}
  \begin{Phonetics}{岸}{an4}[][HSK 5]
    \definition{adj.}{arrogante; orgulhoso; grandioso (de maneira sombria ou condescendente)}
    \definition[条,道,段,面]{s.}{margem; costa; litoral; terreno à beira da água}
  \end{Phonetics}
\end{Entry}

\begin{Entry}{岸上}{8,3}{⼭,⼀}
  \begin{Phonetics}{岸上}{an4shang4}[][HSK 5]
    \definition{s.}{em terra; costa; margem | na margem do rio; na beira do rio}
  \end{Phonetics}
\end{Entry}

%%%%%%%%%% 帖 %%%%%%%%%%
\subsection*{帖}\addcontentsline{loh}{figure}{帖}

\begin{Entry}{帖}{8}{⼱}
  \begin{Phonetics}{帖}{tie1}
    \definition{adj.}{apropriado; adequado; seguro}
    \definition{v.}{obedecer; cumprir; seguir}
  \end{Phonetics}
  \begin{Phonetics}{帖}{tie3}
    \definition{clas.}{prescrição (uma combinação de vários ingredientes medicinais); fórmula (usada para se referir a vários ingredientes em uma decocção); Dialeto: para fitoterapia}
    \definition{s.}{convite | nota; cartão | \emph{Internet}: \emph{post}; postagem; publicação; tópico}
  \seealsoref{帖儿}{tie3r5}
  \seealsoref{帖子}{tie3zi5}
  \end{Phonetics}
  \begin{Phonetics}{帖}{tie4}
    \definition{s.}{um livro contendo modelos de caligrafia ou pintura para os alunos copiarem; exemplos para copiar}
  \end{Phonetics}
\end{Entry}

\begin{Entry}{帖儿}{8,2}{⼱,⼉}
  \begin{Phonetics}{帖儿}{tie3r5}
    \definition{s.}{\emph{Internet}: \emph{post}; postagem; publicação}
  \end{Phonetics}
\end{Entry}

\begin{Entry}{帖子}{8,3}{⼱,⼦}
  \begin{Phonetics}{帖子}{tie3zi5}[][HSK 7-9]
    \definition[个,张]{s.}{convite; notificação de convite para convidado | cartão; nota; pequenos pedaços de papel com escrita neles | postagens; tópicos; isso se refere a textos, imagens, etc., publicados na \emph{Internet} sobre um tópico específico}
  \end{Phonetics}
\end{Entry}

%%%%%%%%%% 帘 %%%%%%%%%%
\subsection*{帘}\addcontentsline{loh}{figure}{帘}

\begin{Entry}{帘}{8}{⼱}
  \begin{Phonetics}{帘}{lian2}
    \definition[块,个]{s.}{bandeira em mastro sobre adega; bandeira como placa de loja | cortina; tela de bambu ou tecido; objetos para cobrir portas e janelas}
  \end{Phonetics}
\end{Entry}

\begin{Entry}{帘子}{8,3}{⼱,⼦}
  \begin{Phonetics}{帘子}{lian2zi5}[][HSK 7-9]
    \definition{s.}{cortina; objetos feitos de tecido, bambu, junco, etc., usados para cobertura}
  \end{Phonetics}
\end{Entry}

%%%%%%%%%% 幷 %%%%%%%%%%
\subsection*{幷}\addcontentsline{loh}{figure}{幷}

\begin{Entry}{幷}{8}{⼲}
  \begin{Phonetics}{幷}{bing4}
    \variantof{并}
  \end{Phonetics}
\end{Entry}

%%%%%%%%%% 幸 %%%%%%%%%%
\subsection*{幸}\addcontentsline{loh}{figure}{幸}

\begin{Entry}{幸}{8}{⼲}
  \begin{Phonetics}{幸}{xing4}
    \definition*{s.}{Sobrenome: Xing}
    \definition{adj.}{feliz}
    \definition{adv.}{afortunadamente; felizmente}
    \definition{s.}{felicidade}
    \definition{v.}{alegrar"-se; sentir"-se feliz e contente | favorecer; patrocinar | vir; chegar; antigamente, referia"-se à chegada de um monarca a um determinado lugar}
  \end{Phonetics}
\end{Entry}

\begin{Entry}{幸亏}{8,3}{⼲,⼆}
  \begin{Phonetics}{幸亏}{xing4kui1}
    \definition{adv.}{felizmente}
  \end{Phonetics}
\end{Entry}

\begin{Entry}{幸运}{8,7}{⼲,⾡}
  \begin{Phonetics}{幸运}{xing4yun4}[][HSK 3]
    \definition{adj.}{sortudo; feliz; afortunado}
    \definition[个,点,丝]{s.}{boa sorte; boa fortuna}
  \end{Phonetics}
\end{Entry}

\begin{Entry}{幸运儿}{8,7,2}{⼲,⾡,⼉}
  \begin{Phonetics}{幸运儿}{xing4yun4'er2}
    \definition{s.}{pessoa de sorte}
  \end{Phonetics}
\end{Entry}

\begin{Entry}{幸运抽奖}{8,7,8,9}{⼲,⾡,⼿,⼤}
  \begin{Phonetics}{幸运抽奖}{xing4yun4chou1jiang3}
    \definition{s.}{loteria | sorteio}
  \end{Phonetics}
\end{Entry}

\begin{Entry}{幸福}{8,13}{⼲,⽰}
  \begin{Phonetics}{幸福}{xing4fu2}[][HSK 3]
    \definition{adj.}{feliz; a vida, a família e outras circunstâncias deixam as pessoas satisfeitas e felizes}
    \definition{s.}{felicidade; bem estar; sensação ou experiência satisfatória e feliz, etc.}
  \end{Phonetics}
\end{Entry}

%%%%%%%%%% 底 %%%%%%%%%%
\subsection*{底}\addcontentsline{loh}{figure}{底}

\begin{Entry}{底}{8}{⼴}
  \begin{Phonetics}{底}{de5}
    \definition{part.}{usada após uma palavra ou frase que é usada como determinante para indicar subordinação à palavra central}
  \end{Phonetics}
  \begin{Phonetics}{底}{di3}[][HSK 4]
    \definition*{s.}{Sobrenome: Di}
    \definition{pron.}{o que? |  isto; isso; aqui | assim; tal}
    \definition{s.}{base; fundo; parte inferior de um objeto | detalhes; o cerne da questão; base, fonte ou contexto de uma coisa | rascunho; cópia mantida como registro; rascunho que pode ser usado como base | final de um ano ou mês | chão; fundo; fundação | a última parte de algo}
  \end{Phonetics}
\end{Entry}

\begin{Entry}{底下}{8,3}{⼴,⼀}
  \begin{Phonetics}{底下}{di3xia5}[][HSK 3]
    \definition{adv.}{em baixo; abaixo; sob | próximo; mais tarde; depois; daqui para a frente}
  \end{Phonetics}
\end{Entry}

\begin{Entry}{底子}{8,3}{⼴,⼦}
  \begin{Phonetics}{底子}{di3zi5}[][HSK 7-9]
    \definition{s.}{fundo; base; a parte mais baixa de um objeto | solo; base; fundo; fundação | rascunho ou esboço; um rascunho para servir de base | cópia mantida como registro; cópia de arquivo | remanescente | detalhes; prós e contras | Literário: configuração (o padrão base)}
  \end{Phonetics}
\end{Entry}

\begin{Entry}{底气}{8,4}{⼴,⽓}
  \begin{Phonetics}{底气}{di3qi4}
    \definition{s.}{capacidade pulmonar | ousadia | confiança | autoconfiança | vigor}
  \end{Phonetics}
\end{Entry}

\begin{Entry}{底层}{8,7}{⼴,⼫}
  \begin{Phonetics}{底层}{di3ceng2}[][HSK 7-9]
    \definition[个]{s.}{andar térreo | fundo; o degrau mais baixo; classe social mais baixa | porão | subcamada; camada de base; subcapa; substrato}
  \end{Phonetics}
\end{Entry}

\begin{Entry}{底线}{8,8}{⼴,⽷}
  \begin{Phonetics}{底线}{di3xian4}[][HSK 7-9]
    \definition{s.}{linha de base (em esportes); limites em ambas as extremidades de campos esportivos como futebol, basquete, vôlei e badminton | um mínimo; o limite mais baixo; um limite mínimo; a menor quantidade possível; refere"-se às condições mínimas | um fantoche; um informante; um agente infiltrado; uma pessoa que se esconde dentro do inimigo para reunir informações ou conduzir outras atividades; um \emph{insider}}
  \end{Phonetics}
\end{Entry}

\begin{Entry}{底蕴}{8,15}{⼴,⾋}
  \begin{Phonetics}{底蕴}{di3yun4}[][HSK 7-9]
    \definition{s.}{detalhes; informações privilegiadas; história interna}
  \end{Phonetics}
\end{Entry}

%%%%%%%%%% 店 %%%%%%%%%%
\subsection*{店}\addcontentsline{loh}{figure}{店}

\begin{Entry}{店}{8}{⼴}
  \begin{Phonetics}{店}{dian4}[][HSK 2]
    \definition[家,间,个]{s.}{loja; armazém; loja de venda de mercadorias | pousada; pequena pousada com instalações simples | usado para nomes de lugares}
  \end{Phonetics}
\end{Entry}

\begin{Entry}{店主}{8,5}{⼴,⼂}
  \begin{Phonetics}{店主}{dian4zhu3}
    \definition{s.}{lojista | dono de loja}
  \end{Phonetics}
\end{Entry}

\begin{Entry}{店员}{8,7}{⼴,⼝}
  \begin{Phonetics}{店员}{dian4yuan2}
    \definition{s.}{assistente de loja | balconista | vendedor}
  \end{Phonetics}
\end{Entry}

%%%%%%%%%% 庙 %%%%%%%%%%
\subsection*{庙}\addcontentsline{loh}{figure}{庙}

\begin{Entry}{庙}{8}{⼴}
  \begin{Phonetics}{庙}{miao4}[][HSK 7-9]
    \definition[座,个,间]{s.}{templo; santuário | feira do templo | Literário: corte imperial; corte real | Literário: imperador falecido | casa de incenso; locais onde, no passado, foram consagradas tábuas ancestrais, divindades ou figuras históricas}
  \end{Phonetics}
\end{Entry}

\begin{Entry}{庙会}{8,6}{⼴,⼈}
  \begin{Phonetics}{庙会}{miao4hui4}[][HSK 7-9]
    \definition{s.}{feira; feira do templo; festival feira do templo; mercados montados em templos ou próximos a eles; geralmente realizados em festivais ou dias específicos}
  \end{Phonetics}
\end{Entry}

%%%%%%%%%% 庞 %%%%%%%%%%
\subsection*{庞}\addcontentsline{loh}{figure}{庞}

\begin{Entry}{庞}{8}{⼴}
  \begin{Phonetics}{庞}{pang2}
    \definition*{s.}{Sobrenome: Pang}
    \definition{adj.}{enorme | inúmeros e desordenados; numerosos e desorganizados}
    \definition{s.}{molde do rosto de alguém | rosto; placa frontal}
  \end{Phonetics}
\end{Entry}

\begin{Entry}{庞大}{8,3}{⼴,⼤}
  \begin{Phonetics}{庞大}{pang2da4}[][HSK 7-9]
    \definition{adj.}{enorme; colossal; gigantesco; imenso; (em termos de forma, estrutura, quantidade, etc.) é muito grande; excessivamente grande}
  \end{Phonetics}
\end{Entry}

%%%%%%%%%% 废 %%%%%%%%%%
\subsection*{废}\addcontentsline{loh}{figure}{废}

\begin{Entry}{废}{8}{⼴}
  \begin{Phonetics}{废}{fei4}[][HSK 7-9]
    \definition{adj.}{desperdíçado; inútil; fora de uso; inválido; tendo perdido sua função original | Literário: incapacitado; mutilado; aleijado; desabilitado}
    \definition{v.}{desistir; abandonar; abolir; revogar  | Coloquial: punir; bater em alguém | descartar; abandonar}
  \end{Phonetics}
\end{Entry}

\begin{Entry}{废物}{8,8}{⼴,⽜}
  \begin{Phonetics}{废物}{fei4wu4}[][HSK 7-9]
    \definition{s.}{lixo; material residual; coisas que perderam seu valor de uso original}
  \end{Phonetics}
  \begin{Phonetics}{废物}{fei4wu5}
    \definition{s.}{pessoa inútil; imprestável (insulto); uma metáfora para uma pessoa inútil (palavrão)}
  \end{Phonetics}
\end{Entry}

\begin{Entry}{废话}{8,8}{⼴,⾔}
  \begin{Phonetics}{废话}{fei4hua4}[][HSK 7-9]
    \definition{s.}{lixo; absurdo; palavras supérfluas; palavras redundantes e inúteis}
    \definition{v.}{falar bobagens; conversa fiada}
  \end{Phonetics}
\end{Entry}

\begin{Entry}{废品}{8,9}{⼴,⼝}
  \begin{Phonetics}{废品}{fei4pin3}[][HSK 7-9]
    \definition[件,吨,批,堆]{s.}{produto residual; rejeito; descarte; produtos não qualificados; produto descartado; sucata; refugo; material rejeitado}
  \end{Phonetics}
\end{Entry}

\begin{Entry}{废除}{8,9}{⼴,⾩}
  \begin{Phonetics}{废除}{fei4chu2}[][HSK 7-9]
    \definition{v.}{revogar; anular; cancelar; abolir (uma lei, sistema, tratado, etc.)}
  \end{Phonetics}
\end{Entry}

\begin{Entry}{废寝忘食}{8,13,7,9}{⼴,⼧,⼼,⾷}
  \begin{Phonetics}{废寝忘食}{fei4qin3-wang4shi2}[][HSK 7-9]
    \definition{expr.}{esquecer de comer e dormir; estar totalmente absorvido em}
  \end{Phonetics}
\end{Entry}

\begin{Entry}{废墟}{8,14}{⼴,⼟}
  \begin{Phonetics}{废墟}{fei4xu1}[][HSK 7-9]
    \definition[片,堆,个]{s.}{ruínas; terreno baldio; um lugar como uma cidade ou vila que ficou deserta e desolada após ser destruída ou sofrer um desastre natural}
  \end{Phonetics}
\end{Entry}

%%%%%%%%%% 建 %%%%%%%%%%
\subsection*{建}\addcontentsline{loh}{figure}{建}

\begin{Entry}{建}{8}{⼵}
  \begin{Phonetics}{建}{jian4}[][HSK 3]
    \definition*{s.}{Província de Fujian | Rio Jian Jiang (na província de Fujian) | Sobrenome: Jian}
    \definition{v.}{construir; construir; erigir | estabelecer; configurar; fundar | propor; defender; apresentar (suas próprias opiniões)}
  \end{Phonetics}
\end{Entry}

\begin{Entry}{建立}{8,5}{⼵,⽴}
  \begin{Phonetics}{建立}{jian4li4}[][HSK 3]
    \definition{v.}{estabelecer; construir; começar a construir | vir a ser; começar a surgir; começar a se formar}
  \end{Phonetics}
\end{Entry}

\begin{Entry}{建立者}{8,5,8}{⼵,⽴,⽼}
  \begin{Phonetics}{建立者}{jian4li4zhe3}
    \definition{s.}{fundador; construtor}
  \end{Phonetics}
\end{Entry}

\begin{Entry}{建议}{8,5}{⼵,⾔}
  \begin{Phonetics}{建议}{jian4yi4}[][HSK 3]
    \definition[个,点,条]{s.}{proposta; sugestão; recomendação; para que alguém ou alguma coisa evolua para melhor, para o coletivo; pontos de vista e opiniões apresentados pelos líderes, etc.}
    \definition{v.}{propor; sugerir; recomendar; em relação a determinada pessoa ou situação, apresentar seus pontos de vista e opiniões ao coletivo, aos líderes ou a indivíduos, para que as coisas evoluam para melhor}
  \end{Phonetics}
\end{Entry}

\begin{Entry}{建交}{8,6}{⼵,⼇}
  \begin{Phonetics}{建交}{jian4/jiao1}[][HSK 7-9]
    \definition{v.+compl.}{estabelecer relações diplomáticas}
  \end{Phonetics}
\end{Entry}

\begin{Entry}{建成}{8,6}{⼵,⼽}
  \begin{Phonetics}{建成}{jian4cheng2}[][HSK 3]
    \definition{v.}{terminar a construção}
  \end{Phonetics}
\end{Entry}

\begin{Entry}{建设}{8,6}{⼵,⾔}
  \begin{Phonetics}{建设}{jian4she4}[][HSK 3]
    \definition{s.}{reconstrução; desenvolvimento; trabalhos relacionados com a construção}
    \definition{v.}{construir; edificar; (Estado ou coletividade) criar novos empreendimentos ou aumento de novas instalações}
  \end{Phonetics}
\end{Entry}

\begin{Entry}{建设性}{8,6,8}{⼵,⾔,⼼}
  \begin{Phonetics}{建设性}{jian4she4xing4}
    \definition{adj.}{construtivo}
    \definition{s.}{construtividade}
  \end{Phonetics}
\end{Entry}

\begin{Entry}{建设者}{8,6,8}{⼵,⾔,⽼}
  \begin{Phonetics}{建设者}{jian4she4zhe3}
    \definition{s.}{construtor}
  \end{Phonetics}
\end{Entry}

\begin{Entry}{建树}{8,9}{⼵,⽊}
  \begin{Phonetics}{建树}{jian4shu4}[][HSK 7-9]
    \definition{s.}{realização; conquista}
    \definition{v.}{dar uma contribuição; contribuir | Literário: alcançar uma conquista}
  \end{Phonetics}
\end{Entry}

\begin{Entry}{建造}{8,10}{⼵,⾡}
  \begin{Phonetics}{建造}{jian4zao4}[][HSK 5]
    \definition{v.}{construir; edificar}
  \end{Phonetics}
\end{Entry}

\begin{Entry}{建筑}{8,12}{⼵,⽵}
  \begin{Phonetics}{建筑}{jian4zhu4}[][HSK 5]
    \definition[座,幢,排]{s.}{construção; estrutura; edifício; prédio}
    \definition{v.}{construir; erguer; edificar; construir casas, estradas, pontes, etc.}
  \end{Phonetics}
\end{Entry}

\begin{Entry}{建筑师}{8,12,6}{⼵,⽵,⼱}
  \begin{Phonetics}{建筑师}{jian4zhu4shi1}[][HSK 7-9]
    \definition{s.}{arquiteto; profissionais de engenharia e técnicos atuantes na indústria da construção}
  \end{Phonetics}
\end{Entry}

\begin{Entry}{建筑物}{8,12,8}{⼵,⽵,⽜}
  \begin{Phonetics}{建筑物}{jian4zhu4wu4}[][HSK 7-9]
    \definition[栋]{s.}{edifício; estrutura; projetos de engenharia civil realizados pelo homem, como casas e pontes}
  \end{Phonetics}
\end{Entry}

%%%%%%%%%% 廻 %%%%%%%%%%
\subsection*{廻}\addcontentsline{loh}{figure}{廻}

\begin{Entry}{廻}{8}{⼵}
  \begin{Phonetics}{廻}{hui2}
    \variantof{回}
  \end{Phonetics}
\end{Entry}

%%%%%%%%%% 弥 %%%%%%%%%%
\subsection*{弥}\addcontentsline{loh}{figure}{弥}

\begin{Entry}{弥}{8}{⼸}
  \begin{Phonetics}{弥}{mi2}
    \definition*{s.}{Sobrenome: Mi}
    \definition{adj.}{cheio; inteiro}
    \definition{adv.}{Literário: mais; ainda mais}
    \definition{v.}{transbordar; encher | cobrir; encher}
  \end{Phonetics}
\end{Entry}

\begin{Entry}{弥补}{8,7}{⼸,⾐}
  \begin{Phonetics}{弥补}{mi2bu3}[][HSK 7-9]
    \definition{v.}{remediar; compensar; reparar; restituir; consertar; preencher; inventar}
  \end{Phonetics}
\end{Entry}

\begin{Entry}{弥漫}{8,14}{⼸,⽔}
  \begin{Phonetics}{弥漫}{mi2man4}[][HSK 7-9]
    \definition{v.}{encher; transbordar; preencher o ar; espalhar"-se por toda parte}
  \end{Phonetics}
\end{Entry}

%%%%%%%%%% 录 %%%%%%%%%%
\subsection*{录}\addcontentsline{loh}{figure}{录}

\begin{Entry}{录}{8}{⼹}
  \begin{Phonetics}{录}{lu4}[][HSK 3]
    \definition{s.}{registro; cadastro; coleção; seleções}
    \definition{v.}{copiar; gravar; escrever; copiar; registrar | contratar; selecionar; empregar; adotar ou nomear | gravar em fita magnética}
  \end{Phonetics}
\end{Entry}

\begin{Entry}{录制}{8,8}{⼹,⼑}
  \begin{Phonetics}{录制}{lu4zhi4}[][HSK 7-9]
    \definition{v.}{gravar som ou imagem usando um gravador de fita ou de vídeo; processar e criar uma obra de arte}
  \end{Phonetics}
\end{Entry}

\begin{Entry}{录取}{8,8}{⼹,⼜}
  \begin{Phonetics}{录取}{lu4qu3}[][HSK 4]
    \definition{v.}{aceitar; admitir; recrutar; entrar; matricular (os aprovados no exame)}
  \end{Phonetics}
\end{Entry}

\begin{Entry}{录音}{8,9}{⼹,⾳}
  \begin{Phonetics}{录音}{lu4/yin1}[][HSK 3]
    \definition[段,个]{s.}{gravação de som; som gravado com equipamento especializado}
    \definition{v.+compl.}{gravar; converter o som em sinal elétrico e, em seguida, gravá-lo por meios mecânicos, ópticos ou eletromagnéticos}
  \end{Phonetics}
\end{Entry}

\begin{Entry}{录音机}{8,9,6}{⼹,⾳,⽊}
  \begin{Phonetics}{录音机}{lu4yin1ji1}[][HSK 6]
    \definition[台]{s.}{gravador de som; máquina de gravação (de fita)}
  \end{Phonetics}
\end{Entry}

\begin{Entry}{录像}{8,13}{⼹,⼈}
  \begin{Phonetics}{录像}{lu4/xiang4}[][HSK 6]
    \definition[段,个,些,盘]{s.}{vídeo; gravação; fita de vídeo; imagens gravadas com celulares, câmeras, etc.}
    \definition{v.+compl.}{gravar bídeo; gravar em fita de vídeo | usar celulares, câmeras e outros dispositivos para salvar registros de vídeo}
  \end{Phonetics}
\end{Entry}

\begin{Entry}{录像机}{8,13,6}{⼹,⼈,⽊}
  \begin{Phonetics}{录像机}{lu4xiang4ji1}
    \definition[台]{s.}{gravador de vídeo | VCR}
  \end{Phonetics}
\end{Entry}

\begin{Entry}{录像带}{8,13,9}{⼹,⼈,⼱}
  \begin{Phonetics}{录像带}{lu4xiang4dai4}
    \definition[盘]{s.}{video-cassete}
  \end{Phonetics}
\end{Entry}

%%%%%%%%%% 彼 %%%%%%%%%%
\subsection*{彼}\addcontentsline{loh}{figure}{彼}

\begin{Entry}{彼}{8}{⼻}
  \begin{Phonetics}{彼}{bi3}
    \definition{s.}{aquele; aquilo; outro | a outra parte}
  \antonymref{此}{ci3}
  \end{Phonetics}
\end{Entry}

\begin{Entry}{彼此}{8,6}{⼻,⽌}
  \begin{Phonetics}{彼此}{bi3ci3}[][HSK 5]
    \definition{pron.}{um ao outro; uns com os outros; este e aquele têm algum tipo de relacionamento; ambas as partes}
  \end{Phonetics}
\end{Entry}

%%%%%%%%%% 往 %%%%%%%%%%
\subsection*{往}\addcontentsline{loh}{figure}{往}

\begin{Entry}{往}{8}{⼻}
  \begin{Phonetics}{往}{wang3}[][HSK 2]
    \definition{adj.}{passado; anterior}
    \definition{prep.}{para; em direção a; na direção de}
    \definition{v.}{ir}
  \end{Phonetics}
\end{Entry}

\begin{Entry}{往日}{8,4}{⼻,⽇}
  \begin{Phonetics}{往日}{wang3ri4}[][HSK 7-9]
    \definition{adv.}{(nos) dias anteriores; (nos) últimos dias; (em) dias passados}
    \definition{s.}{o passado}
  \synonymref{昔日}{xi1ri4}
  \end{Phonetics}
\end{Entry}

\begin{Entry}{往生}{8,5}{⼻,⽣}
  \begin{Phonetics}{往生}{wang3sheng1}
    \definition{v.}{renascer | morrer | (Budismo) viver no paraíso}
  \end{Phonetics}
\end{Entry}

\begin{Entry}{往后}{8,6}{⼻,⼝}
  \begin{Phonetics}{往后}{wang3hou4}[][HSK 6]
    \definition{s.}{de agora em diante; mais tarde; no futuro | na parte traseira; na parte de trás | para trás; depois; à ré}
  \end{Phonetics}
\end{Entry}

\begin{Entry}{往年}{8,6}{⼻,⼲}
  \begin{Phonetics}{往年}{wang3nian2}[][HSK 6]
    \definition{s.}{(em) anos anteriores}
  \end{Phonetics}
\end{Entry}

\begin{Entry}{往来}{8,7}{⼻,⽊}
  \begin{Phonetics}{往来}{wang3lai2}[][HSK 6]
    \definition{s.}{contatos comerciais; relações comerciais; relações diplomáticas | negociações; visitas mútuas; comunicação}
    \definition{v.}{ir e vir | contatar; ter relações}
  \end{Phonetics}
\end{Entry}

\begin{Entry}{往返}{8,7}{⼻,⾡}
  \begin{Phonetics}{往返}{wang3fan3}[][HSK 7-9]
    \definition{v.}{transportar; ir e voltar; viajar de e para}
  \synonymref{来回}{lai2hui2}
  \synonymref{往复}{wang3fu4}
  \synonymref{往来}{wang3lai2}
  \end{Phonetics}
\end{Entry}

\begin{Entry}{往事}{8,8}{⼻,⼅}
  \begin{Phonetics}{往事}{wang3shi4}[][HSK 7-9]
    \definition[件,段]{s.}{eventos passados; o passado; coisas passadas}
  \end{Phonetics}
\end{Entry}

\begin{Entry}{往例}{8,8}{⼻,⼈}
  \begin{Phonetics}{往例}{wang3li4}
    \definition{s.}{prática (habitual) do passado | precedente}
  \end{Phonetics}
\end{Entry}

\begin{Entry}{往往}{8,8}{⼻,⼻}
  \begin{Phonetics}{往往}{wang3wang3}[][HSK 3]
    \definition{adv.}{frequentemente; muitas vezes; mais frequentemente do que não; indica que uma situação existe ou ocorre com frequência}
  \end{Phonetics}
\end{Entry}

\begin{Entry}{往昔}{8,8}{⼻,⽇}
  \begin{Phonetics}{往昔}{wang3xi1}
    \definition{s.}{o passado}
  \end{Phonetics}
\end{Entry}

\begin{Entry}{往复}{8,9}{⼻,⼢}
  \begin{Phonetics}{往复}{wang3fu4}
    \definition{s.}{para trás e para frente (por exemplo, da ação do pistão ou da bomba)}
    \definition{v.}{ir e voltar | fazer uma viagem de volta}
  \end{Phonetics}
\end{Entry}

\begin{Entry}{往迹}{8,9}{⼻,⾡}
  \begin{Phonetics}{往迹}{wang3ji4}
    \definition{s.}{Literário: eventos passados; coisa do passado; tempos antigos}
  \end{Phonetics}
\end{Entry}

\begin{Entry}{往常}{8,11}{⼻,⼱}
  \begin{Phonetics}{往常}{wang3chang2}[][HSK 7-9]
    \definition{s.}{habitual; normal; como sempre}
  \synonymref{平常}{ping2chang2}
  \synonymref{平时}{ping2shi2}
  \synonymref{以前}{yi3qian2}
  \antonymref{如今}{ru2jin1}
  \end{Phonetics}
\end{Entry}

\begin{Entry}{往程}{8,12}{⼻,⽲}
  \begin{Phonetics}{往程}{wang3cheng2}
    \definition{s.}{saída (de uma viagem de ônibus ou trem, etc.)}
  \end{Phonetics}
\end{Entry}

%%%%%%%%%% 征 %%%%%%%%%%
\subsection*{征}\addcontentsline{loh}{figure}{征}

\begin{Entry}{征}{8}{⼻}
  \begin{Phonetics}{征}{zheng1}
    \definition{s.}{prova; evidência | sinal; símbolo; presságio; sinais de manifestação; fenômeno}
    \definition{v.}{viajar; fazer uma jornada; pegar o caminho mais longo | iniciar uma campanha; fazer uma expedição punitiva | convocar; selecionar; recrutar | cobrar; impor; coletar | solicitar; pedir; procurar}
  \end{Phonetics}
\end{Entry}

\begin{Entry}{征求}{8,7}{⼻,⽔}
  \begin{Phonetics}{征求}{zheng1qiu2}[][HSK 4]
    \definition{v.}{procurar; buscar; solicitar; pedir abertamente opiniões, pontos de vista, etc.}
  \end{Phonetics}
\end{Entry}

\begin{Entry}{征服}{8,8}{⼻,⽉}
  \begin{Phonetics}{征服}{zheng1fu2}[][HSK 4]
    \definition{v.}{conquistar; cativar; usar a força para fazer a outra parte se submeter | subjugar; dominar; convencer as pessoas com poder infeccioso}
  \end{Phonetics}
\end{Entry}

%%%%%%%%%% 忠 %%%%%%%%%%
\subsection*{忠}\addcontentsline{loh}{figure}{忠}

\begin{Entry}{忠}{8}{⼼}
  \begin{Phonetics}{忠}{zhong1}
    \definition{adj.}{leal; fiel; devotado | honesto}
  \end{Phonetics}
\end{Entry}

\begin{Entry}{忠心}{8,4}{⼼,⼼}
  \begin{Phonetics}{忠心}{zhong1xin1}[][HSK 6]
    \definition{s.}{lealdade; devoção; fidelidade}
  \end{Phonetics}
\end{Entry}

%%%%%%%%%% 念 %%%%%%%%%%
\subsection*{念}\addcontentsline{loh}{figure}{念}

\begin{Entry}{念}{8}{⼼}
  \begin{Phonetics}{念}{nian4}[][HSK 3]
    \definition*{s.}{Sobrenome: Nian}
    \definition{num.}{vinte; 20; capitalização do número 廿}
    \definition{s.}{ideia; pensamento; pensamentos ou intenções internas}
    \definition{v.}{ler em voz alta | estudar; frequentar a escola | considerar; levar em conta | sentir falta; pensar em; pensar sobre; pensar frequentemente sobre}
  \seealsoref{廿}{nian4}
  \end{Phonetics}
\end{Entry}

\begin{Entry}{念书}{8,4}{⼼,⼄}
  \begin{Phonetics}{念书}{nian4/shu1}[][HSK 7-9]
    \definition{v.+compl.}{estudar; ir à escola; frequentar a escola | ler livros}
  \end{Phonetics}
\end{Entry}

\begin{Entry}{念头}{8,5}{⼼,⼤}
  \begin{Phonetics}{念头}{nian4tou5}[][HSK 7-9]
    \definition[个,种]{s.}{ideia; pensamento; intenção; plano}
  \end{Phonetics}
\end{Entry}

\begin{Entry}{念念不忘}{8,8,4,7}{⼼,⼼,⼀,⼼}
  \begin{Phonetics}{念念不忘}{nian4nian4-bu2wang4}[][HSK 7-9]
    \definition{expr.}{inesquecível; ter sempre em mente (expressão idiomática); nunca se esqueça}
  \end{Phonetics}
\end{Entry}

%%%%%%%%%% 忽 %%%%%%%%%%
\subsection*{忽}\addcontentsline{loh}{figure}{忽}

\begin{Entry}{忽}{8}{⼼}
  \begin{Phonetics}{忽}{hu1}
    \definition*{s.}{Sobrenome: Hu}
    \definition{adv.}{agora\dots, agora\dots | de repente; subitamente}[天气忽冷忽热。===O clima está frio em um minuto e quente no outro.]
    \definition{v.}{negligenciar; ignorar; não prestar atenção; não levar a sério}
  \end{Phonetics}
\end{Entry}

\begin{Entry}{忽视}{8,8}{⼼,⾒}
  \begin{Phonetics}{忽视}{hu1shi4}[][HSK 4]
    \definition{v.}{ignorar; negligenciar; menosprezar; desprezar; dar de ombros}
  \end{Phonetics}
\end{Entry}

\begin{Entry}{忽高忽低}{8,10,8,7}{⼼,⾼,⼼,⼈}
  \begin{Phonetics}{忽高忽低}{hu1gao1-hu1di1}[][HSK 7-9]
    \definition{expr.}{``Altos e baixos.''; ora alto, ora baixo}
  \end{Phonetics}
\end{Entry}

\begin{Entry}{忽悠}{8,11}{⼼,⼼}
  \begin{Phonetics}{忽悠}{hu1you5}[][HSK 7-9]
    \definition{v.}{balançar; cintilar; sacudir | enganar; enganar alguém}
  \end{Phonetics}
\end{Entry}

\begin{Entry}{忽略}{8,11}{⼼,⽥}
  \begin{Phonetics}{忽略}{hu1lve4}[][HSK 6]
    \definition{v.}{negligenciar; ignorar; não perceber}
  \end{Phonetics}
\end{Entry}

\begin{Entry}{忽然}{8,12}{⼼,⽕}
  \begin{Phonetics}{忽然}{hu1ran2}[][HSK 2]
    \definition{adv.}{repentinamente; de repente; sem aviso prévio; significa que algo aconteceu de forma rápida e inesperada}
  \end{Phonetics}
\end{Entry}

%%%%%%%%%% 态 %%%%%%%%%%
\subsection*{态}\addcontentsline{loh}{figure}{态}

\begin{Entry}{态}{8}{⼼}
  \begin{Phonetics}{态}{tai4}
    \definition{s.}{forma; aparência; condição | (física) estado | (linguística) voz}[气态===estado gasoso | 被动态===voz passiva]
  \end{Phonetics}
\end{Entry}

\begin{Entry}{态度}{8,9}{⼼,⼴}
  \begin{Phonetics}{态度}{tai4du5}[][HSK 2]
    \definition[种,个]{s.}{maneira; comportamento; atitude; comportamento e expressão facial das pessoas | atitude; abordagem; opinião sobre o assunto e medidas tomadas}
  \end{Phonetics}
\end{Entry}

%%%%%%%%%% 怕 %%%%%%%%%%
\subsection*{怕}\addcontentsline{loh}{figure}{怕}

\begin{Entry}{怕}{8}{⼼}
  \begin{Phonetics}{怕}{pa4}[][HSK 2]
    \definition{adv.}{(expressando suposição, julgamento, estimativa, etc.) talvez; suponho; receio (que)}
    \definition{adv.}{por medo; talvez; suponho}
    \definition{v.}{temer; ter medo; recear; sentir medo, ficar nervoso | estar preocupado com; estar preocupado por (ou sobre); ter medo de que algo possa acontecer | ser afetado por; não conseguir suportar; não aguentar mais}
  \end{Phonetics}
\end{Entry}

%%%%%%%%%% 怛 %%%%%%%%%%
\subsection*{怛}\addcontentsline{loh}{figure}{怛}

\begin{Entry}{怛}{8}{⼼}
  \begin{Phonetics}{怛}{da2}
    \definition{adj.}{Arcaico: aflito; angustiado | Arcaico: aterrorizado | Arcaico: alarmado; entristecido | Arcaico: chocado}
  \end{Phonetics}
\end{Entry}

%%%%%%%%%% 怜 %%%%%%%%%%
\subsection*{怜}\addcontentsline{loh}{figure}{怜}

\begin{Entry}{怜}{8}{⼼}
  \begin{Phonetics}{怜}{lian2}
    \definition{v.}{simpatizar com; ter pena; sentir compaixão}
  \end{Phonetics}
\end{Entry}

\begin{Entry}{怜惜}{8,11}{⼼,⼼}
  \begin{Phonetics}{怜惜}{lian2xi1}[][HSK 7-9]
    \definition{v.}{ter pena de; sentir compaixão por | sentir ternura e proteção em relação a}
  \end{Phonetics}
\end{Entry}

%%%%%%%%%% 性 %%%%%%%%%%
\subsection*{性}\addcontentsline{loh}{figure}{性}

\begin{Entry}{性}{8}{⼼}
  \begin{Phonetics}{性}{xing4}[][HSK 3]
    \definition[个]{s.}{natureza; caráter; personalidade | propriedade; qualidade; natureza e características das coisas | sexo; gênero | sexualidade; relacionado com a reprodução e a sexualidade | caráter; temperamento}
    \definition{suf.}{indica uma determinada propriedade ou característica de algo; segue um substantivo, verbo ou adjetivo, formando um substantivo abstrato ou um adjetivo que expressa uma propriedade}
  \end{Phonetics}
\end{Entry}

\begin{Entry}{性生活}{8,5,9}{⼼,⽣,⽔}
  \begin{Phonetics}{性生活}{xing4sheng1huo2}
    \definition{s.}{vida sexual}
  \end{Phonetics}
\end{Entry}

\begin{Entry}{性别}{8,7}{⼼,⼑}
  \begin{Phonetics}{性别}{xing4bie2}[][HSK 3]
    \definition[种]{s.}{sexo; gênero}
  \end{Phonetics}
\end{Entry}

\begin{Entry}{性质}{8,8}{⼼,⾙}
  \begin{Phonetics}{性质}{xing4zhi4}[][HSK 4]
    \definition[个,种,类]{s.}{natureza; qualidade; caráter; propriedade; propriedade fundamental que distingue uma coisa de outra}
  \end{Phonetics}
\end{Entry}

\begin{Entry}{性侵}{8,9}{⼼,⼈}
  \begin{Phonetics}{性侵}{xing4qin1}
    \definition{s.}{agressão sexual}
    \definition{v.}{agredir sexualmente}
  \end{Phonetics}
\end{Entry}

\begin{Entry}{性格}{8,10}{⼼,⽊}
  \begin{Phonetics}{性格}{xing4ge2}[][HSK 3]
    \definition[种,个]{s.}{caráter; temperamento; as características psicológicas manifestadas na atitude e no comportamento em relação às pessoas e às coisas}
  \end{Phonetics}
\end{Entry}

\begin{Entry}{性能}{8,10}{⼼,⾁}
  \begin{Phonetics}{性能}{xing4neng2}[][HSK 5]
    \definition{s.}{natureza; propriedade; desempenho; função (de uma máquina, etc.); grau de conformidade dos produtos mecânicos ou outros produtos industriais com os requisitos de projeto}
  \end{Phonetics}
\end{Entry}

%%%%%%%%%% 怪 %%%%%%%%%%
\subsection*{怪}\addcontentsline{loh}{figure}{怪}

\begin{Entry}{怪}{8}{⼼}
  \begin{Phonetics}{怪}{guai4}[][HSK 4,5]
    \definition*{s.}{Sobrenome: Guai}
    \definition{adj.}{estranho; esquisito; desconcertante | peculiar; excêntrico; pitoresco; monstruoso; anormal; incomum}
    \definition{adv.}{bastante; muito}
    \definition{s.}{monstro; demônio | diabo; ser maligno}
    \definition{v.}{culpar | achar algo estranho; maravilhar"-se com; ficar surpreso | repreender; culpar; reclamar}
  \end{Phonetics}
\end{Entry}

\begin{Entry}{怪不得}{8,4,11}{⼼,⼀,⼻}
  \begin{Phonetics}{怪不得}{guai4bu5de5}[][HSK 7-9]
    \definition{adv.}{não é de admirar; então é por isso; isso explica por que; isso significa que você entende o motivo e não acha mais uma situação estranha}
    \definition{v.}{não culpar; não acusar; não poder culpar, não se ofender}[你做错了,怪不得别人。===Você cometeu um erro, então não culpe os outros.]
  \end{Phonetics}
\end{Entry}

\begin{Entry}{怪异}{8,6}{⼼,⼶}
  \begin{Phonetics}{怪异}{guai4yi4}[][HSK 7-9]
    \definition{adj.}{monstruoso; estranho; incomum}
    \definition{s.}{fenômeno estranho; presságio; prodígio | monstruosidade}
  \end{Phonetics}
\end{Entry}

\begin{Entry}{怪物}{8,8}{⼼,⽜}
  \begin{Phonetics}{怪物}{guai4wu5}[][HSK 7-9]
    \definition{s.}{monstro; aberração; coisas imaginárias que parecem estranhas, mas têm habilidades especiais | pessoa excêntrica; pássaro estranho; uma pessoa com temperamento excêntrico}
  \end{Phonetics}
\end{Entry}

\begin{Entry}{怪兽}{8,11}{⼼,⼋}
  \begin{Phonetics}{怪兽}{guai4shou4}
    \definition{s.}{animal raro | animal mítico | monstro}
  \end{Phonetics}
\end{Entry}

\begin{Entry}{怪癖}{8,18}{⼼,⽧}
  \begin{Phonetics}{怪癖}{guai4pi3}
    \definition{adj.}{peculiar}
    \definition{s.}{excentricidade | peculiaridade | hobby estranho}
  \end{Phonetics}
\end{Entry}

%%%%%%%%%% 或 %%%%%%%%%%
\subsection*{或}\addcontentsline{loh}{figure}{或}

\begin{Entry}{或}{8}{⼽}
  \begin{Phonetics}{或}{huo4}[][HSK 2]
    \definition{adv.}{talvez; possivelmente; provavelmente | (geralmente na forma negativa) um pouco; ligeiramente}
    \definition{conj.}{ou (indicando escolha); ou\dots ou\dots}
    \definition{pron.}{alguém; algumas pessoas; refere"-se a alguém ou algo, equivalente a 有人 ou 有的}
  \seealsoref{有的}{you3de5}
  \seealsoref{有人}{you3ren2}
  \end{Phonetics}
\end{Entry}

\begin{Entry}{或多或少}{8,6,8,4}{⼽,⼣,⼽,⼩}
  \begin{Phonetics}{或多或少}{huo4duo1-huo4shao3}[][HSK 7-9]
    \definition{expr.}{mais ou menos}
  \end{Phonetics}
\end{Entry}

\begin{Entry}{或许}{8,6}{⼽,⾔}
  \begin{Phonetics}{或许}{huo4xu3}[][HSK 4]
    \definition{adv.}{talvez; possivelmente; receio; não tenho certeza}
  \end{Phonetics}
\end{Entry}

\begin{Entry}{或者}{8,8}{⼽,⽼}
  \begin{Phonetics}{或者}{huo4zhe3}[][HSK 2]
    \definition{adv.}{talvez; possivelmente}
    \definition{conj.}{ou (usado em expressões afirmativas); ou\dots ou\dots; usado em frases narrativas para indicar uma relação de escolha | ou (usado para indicar equação); indica relação de equivalência, indicando que os objetos anterior e posterior são iguais}
  \end{Phonetics}
\end{Entry}

\begin{Entry}{或是}{8,9}{⼽,⽇}
  \begin{Phonetics}{或是}{huo4shi4}[][HSK 5]
    \definition{adv.}{um ou outro; o outro}
    \definition{conj.}{ou; às vezes, é apenas uma de duas coisas}
  \end{Phonetics}
\end{Entry}

%%%%%%%%%% 房 %%%%%%%%%%
\subsection*{房}\addcontentsline{loh}{figure}{房}

\begin{Entry}{房}{8}{⼾}
  \begin{Phonetics}{房}{fang2}
    \definition*{s.}{Fang, a quarta das vinte e oito constelações nas quais a esfera celeste foi dividida, consistindo de quatro estrelas quase em linha reta em Escorpião | Sobrenome: Fang}
    \definition[幢,个,间]{s.}{casa; edifício | sala; quarto; câmara | estrutura semelhante a uma casa | um ramo de uma família extensa | loja; estoque | local de trabalho do artesão; oficina; moinho}
  \end{Phonetics}
\end{Entry}

\begin{Entry}{房子}{8,3}{⼾,⼦}
  \begin{Phonetics}{房子}{fang2zi5}[][HSK 1]
    \definition[栋,幢,座,套,间]{s.}{casa; edifício; prédio}
  \end{Phonetics}
\end{Entry}

\begin{Entry}{房东}{8,5}{⼾,⼀}
  \begin{Phonetics}{房东}{fang2dong1}[][HSK 3]
    \definition[个,位,名]{s.}{dono;  proprietário; senhorio; pessoas que alugam ou emprestam imóveis (para os 房客 )}
  \seealsoref{房客}{fang2ke4}
  \end{Phonetics}
\end{Entry}

\begin{Entry}{房主}{8,5}{⼾,⼂}
  \begin{Phonetics}{房主}{fang2zhu3}
    \definition{s.}{proprietário | dono de um imóvel}
  \end{Phonetics}
\end{Entry}

\begin{Entry}{房价}{8,6}{⼾,⼈}
  \begin{Phonetics}{房价}{fang2jia4}[][HSK 6]
    \definition{s.}{custo de moradia; tarifa de quarto | preço da casa}
  \end{Phonetics}
\end{Entry}

\begin{Entry}{房地产}{8,6,6}{⼾,⼟,⼇}
  \begin{Phonetics}{房地产}{fang2di4chan3}[][HSK 7-9]
    \definition{s.}{imóveis; um termo geral para imóveis e terrenos}
  \end{Phonetics}
\end{Entry}

\begin{Entry}{房间}{8,7}{⼾,⾨}
  \begin{Phonetics}{房间}{fang2jian1}[][HSK 1]
    \definition[个,间,套]{s.}{sala; câmara; escritório; apartamento; divisões internas da casa}
  \end{Phonetics}
\end{Entry}

\begin{Entry}{房客}{8,9}{⼾,⼧}
  \begin{Phonetics}{房客}{fang2ke4}
    \definition{s.}{inquilino (de um quarto ou casa); hóspede | inquilino; hóspede; pessoas que alugam ou emprestam imóveis para moradia}
  \antonymref{房东}{fang2dong1}
  \end{Phonetics}
\end{Entry}

\begin{Entry}{房屋}{8,9}{⼾,⼫}
  \begin{Phonetics}{房屋}{fang2wu5}[][HSK 3]
    \definition[间,所,套]{s.}{casas; habitação; edifícios}
  \end{Phonetics}
\end{Entry}

\begin{Entry}{房租}{8,10}{⼾,⽲}
  \begin{Phonetics}{房租}{fang2zu1}[][HSK 3]
    \definition[笔]{s.}{aluguel}
  \end{Phonetics}
\end{Entry}

%%%%%%%%%% 所 %%%%%%%%%%
\subsection*{所}\addcontentsline{loh}{figure}{所}

\begin{Entry}{所}{8}{⼾}
  \begin{Phonetics}{所}{suo3}[][HSK 3,6]
    \definition*{s.}{Sobrenome: Suo}
    \definition{clas.}{usado para casas, etc.}
    \definition{part.}{usado com 为 ou com 被 para indicar voz passiva | usado antes do verbo para formar um substantivo ou para qualificar um substantivo | usado antes do verbo na estrutura sujeito"-predicado usada como complemento, indica que o termo central é o objeto}
    \definition{s.}{lugar | usado como nome de órgãos governamentais ou outros locais de trabalho}
  \seealsoref{被}{bei4}
  \seealsoref{为}{wei4}
  \end{Phonetics}
\end{Entry}

\begin{Entry}{所以}{8,4}{⼾,⼈}
  \begin{Phonetics}{所以}{suo3yi3}[][HSK 2]
    \definition{conj.}{assim; portanto; como resultado; conecta frases, expressa resultados e costuma corresponder a expressões como 因为 e 由于}
    \definition[个]{s.}{motivo real; causa real; comportamento adequado}
  \seealsoref{因为}{yin1wei5}
  \seealsoref{由于}{you2yu2}
  \end{Phonetics}
\end{Entry}

\begin{Entry}{所长}{8,4}{⼾,⾧}
  \begin{Phonetics}{所长}{suo3chang2}
    \definition{s.}{aquilo em que alguém é bom; o ponto forte de alguém; o forte de alguém}
  \end{Phonetics}
  \begin{Phonetics}{所长}{suo3zhang3}[][HSK 3]
    \definition{s.}{chefe de um instituto, etc. | superintendente}
  \end{Phonetics}
\end{Entry}

\begin{Entry}{所在}{8,6}{⼾,⼟}
  \begin{Phonetics}{所在}{suo3zai4}[][HSK 5]
    \definition[个]{s.}{lugar; local; localização | o lugar onde alguém ou algo está}
  \end{Phonetics}
\end{Entry}

\begin{Entry}{所有}{8,6}{⼾,⽉}
  \begin{Phonetics}{所有}{suo3you3}[][HSK 2]
    \definition{adj.}{todo | tudo}
    \definition{adj.}{tudo}
    \definition{s.}{bens; posses;}
    \definition{v.}{possuir; ter}
  \end{Phonetics}
\end{Entry}

\begin{Entry}{所作所为}{8,7,8,4}{⼾,⼈,⼾,⼂}
  \begin{Phonetics}{所作所为}{suo3zuo4-suo3wei2}[][HSK 7-9]
    \definition{expr.}{as ações (de alguém); todos os atos de alguém; o que alguém faz; o que alguém faz e como se comporta; o comportamento ou conduta de alguém}
  \end{Phonetics}
\end{Entry}

\begin{Entry}{所谓}{8,11}{⼾,⾔}
  \begin{Phonetics}{所谓}{suo3wei4}[][HSK 7-9]
    \definition{adj.}{o que é chamado; o que é conhecido como; o que significa | o chamado; o que (algumas pessoas) disseram (implicando uma falta de reconhecimento)}
  \synonymref{竟然}{jing4ran2}
  \synonymref{显然}{xian3ran2}
  \synonymref{因为}{yin1wei5}
  \synonymref{终究}{zhong1jiu1}
  \end{Phonetics}
\end{Entry}

\begin{Entry}{所属}{8,12}{⼾,⼫}
  \begin{Phonetics}{所属}{suo3shu3}[][HSK 7-9]
    \definition{s.}{os subordinados, aqueles que estão sob o controle ou comando de alguém; aquilo a que alguém pertence ou com o que está afiliado}
  \end{Phonetics}
\end{Entry}

%%%%%%%%%% 承 %%%%%%%%%%
\subsection*{承}\addcontentsline{loh}{figure}{承}

\begin{Entry}{承}{8}{⼿}
  \begin{Phonetics}{承}{cheng2}
    \definition*{s.}{Sobrenome: Cheng}
    \definition{v.}{suportar; segurar; carregar; sustentar | empreender; contratar (para fazer um trabalho) | estar em dívida (com alguém por uma gentileza); receber um favor | continuar; prosseguir | receber de cima (instruções, mandato)}
  \end{Phonetics}
\end{Entry}

\begin{Entry}{承办}{8,4}{⼿,⼒}
  \begin{Phonetics}{承办}{cheng2ban4}[][HSK 5]
    \definition{v.}{ocupar"-se de; encarregar"-se de; (pessoas, organizações, instituições) aceitar (atividades, reuniões, negócios, etc.)}
  \end{Phonetics}
\end{Entry}

\begin{Entry}{承认}{8,4}{⼿,⾔}
  \begin{Phonetics}{承认}{cheng2ren4}[][HSK 4]
    \definition{s.}{reconhecimento (diplomático, artístico, etc.)}
    \definition{v.}{admitir; reconhecer | dar reconhecimento diplomático; reconhecer}
  \end{Phonetics}
\end{Entry}

\begin{Entry}{承包}{8,5}{⼿,⼓}
  \begin{Phonetics}{承包}{cheng2bao1}[][HSK 7-9]
    \definition{v.}{contratar (com; para); aceitar projetos ou pedidos em massa, etc. e ser responsável por concluí-los}[他承包了这个工程。===Ele foi contratado para esse projeto.]
  \end{Phonetics}
\end{Entry}

\begin{Entry}{承受}{8,8}{⼿,⼜}
  \begin{Phonetics}{承受}{cheng2shou4}[][HSK 4]
    \definition{v.}{suportar; resistir; realizar (tarefas, dificuldades, pressões, etc.); submeter"-se a (testes, etc.) | herdar}
  \end{Phonetics}
\end{Entry}

\begin{Entry}{承担}{8,8}{⼿,⼿}
  \begin{Phonetics}{承担}{cheng2dan1}[][HSK 4]
    \definition{v.}{suportar; empreender; assumir; tomar conta de algo}
  \end{Phonetics}
\end{Entry}

\begin{Entry}{承诺}{8,10}{⼿,⾔}
  \begin{Phonetics}{承诺}{cheng2nuo4}[][HSK 6]
    \definition[个,句,份]{s.}{juramento; promessa; compromisso}
    \definition{v.}{prometer fazer algo; prometer empreender; comprometer"-se a fazer algo}
  \end{Phonetics}
\end{Entry}

\begin{Entry}{承载}{8,10}{⼿,⾞}
  \begin{Phonetics}{承载}{cheng2zai4}[][HSK 7-9]
    \definition{v.}{suportar o peso; segurar o objeto e suportar seu peso}[桥梁承载着巨大的重量。===A ponte suporta uma carga pesada.]
  \end{Phonetics}
\end{Entry}

%%%%%%%%%% 抨 %%%%%%%%%%
\subsection*{抨}\addcontentsline{loh}{figure}{抨}

\begin{Entry}{抨}{8}{⼿}
  \begin{Phonetics}{抨}{peng1}
    \definition{s.}{Literário: \emph{impeachment}; censura}
    \definition{v.}{atacar; criticar; destituir; censurar}
  \end{Phonetics}
\end{Entry}

\begin{Entry}{抨击}{8,5}{⼿,⼐}
  \begin{Phonetics}{抨击}{peng1ji1}[][HSK 7-9]
    \definition{v.}{atacar (falando ou escrevendo); bombardear (com palavras); criticar}[他把电视采访作为一个机会,向反对党进行猛烈抨击。===Ele aproveitou a entrevista na televisão para lançar um ataque feroz contra o partido da oposição.]
  \end{Phonetics}
\end{Entry}

%%%%%%%%%% 披 %%%%%%%%%%
\subsection*{披}\addcontentsline{loh}{figure}{披}

\begin{Entry}{披}{8}{⼿}
  \begin{Phonetics}{披}{pi1}[][HSK 5]
    \definition{v.}{colocar sobre os ombros; enrolar em volta; cobrir ou colocar sobre os ombros | abrir; desenrolar; espalhar | abrir"-se; rachar}
  \end{Phonetics}
\end{Entry}

\begin{Entry}{披露}{8,21}{⼿,⾬}
  \begin{Phonetics}{披露}{pi1lu4}[][HSK 7-9]
    \definition{v.}{publicar; anunciar; tornar público | revelar; mostrar; divulgar}
  \end{Phonetics}
\end{Entry}

%%%%%%%%%% 抬 %%%%%%%%%%
\subsection*{抬}\addcontentsline{loh}{figure}{抬}

\begin{Entry}{抬}{8}{⼿}
  \begin{Phonetics}{抬}{tai2}[][HSK 5]
    \definition{clas.}{usado para objetos que precisam ser carregados por pessoas quando transportados (por exemplo, móveis)}
    \definition{v.}{levantar; elevar; puxar para cima | (por duas ou mais pessoas) carregar; transportar; duas ou mais pessoas carregando algo com as mãos ou nos ombros | discutir, debater (geralmente sem sentido ou sem importância)}
  \end{Phonetics}
\end{Entry}

\begin{Entry}{抬头}{8,5}{⼿,⼤}
  \begin{Phonetics}{抬头}{tai2 tou2}[][HSK 5]
    \definition{s.}{(em recibos, contas, etc.) nome do comprador ou beneficiário, o local no documento onde o nome do beneficiário ou destinatário é escrito}
    \definition{v.}{levantar a cabeça}
  \end{Phonetics}
\end{Entry}

\begin{Entry}{抬杠}{8,7}{⼿,⽊}
  \begin{Phonetics}{抬杠}{tai2/gang4}
    \definition{v.+compl.}{brigar; discutir; discutir por discutir; discutir sobre o certo e o errado (geralmente sem princípios) | Arcaico: carregar um caixão em barras resistentes}
  \end{Phonetics}
\end{Entry}

%%%%%%%%%% 抱 %%%%%%%%%%
\subsection*{抱}\addcontentsline{loh}{figure}{抱}

\begin{Entry}{抱}{8}{⼿}
  \begin{Phonetics}{抱}{bao4}[][HSK 4]
    \definition*{s.}{Sobrenome: Bao}
    \definition{clas.}{braçada; medida dos dois braços}
    \definition{v.}{carregar no peito; segurar com ambos os braços; abraçar | ter o primeiro filho ou neto | adotar um bebê ou criança | ficar juntos, unidos | encaixar ou servir perfeitamente (roupas e sapatos do tamanho certo) | estimar; nutrir; abrigar; ter em mente | continuar; sobrecarregar com | chocar ovos}
  \end{Phonetics}
\end{Entry}

\begin{Entry}{抱负}{8,6}{⼿,⾙}
  \begin{Phonetics}{抱负}{bao4fu4}[][HSK 7-9]
    \definition{s.}{aspiração; ambição; objetivo elevado; grandes intenções e determinação, frequentemente usados na linguagem escrita}
  \end{Phonetics}
\end{Entry}

\begin{Entry}{抱怨}{8,9}{⼿,⼼}
  \begin{Phonetics}{抱怨}{bao4yuan5}[][HSK 5]
    \definition{v.}{reclamar ou expressar descontentamento ou insatisfação; falar com os outros sobre pessoas ou coisas com as quais você não está satisfeito}
  \end{Phonetics}
\end{Entry}

\begin{Entry}{抱歉}{8,14}{⼿,⽋}
  \begin{Phonetics}{抱歉}{bao4qian4}[][HSK 6]
    \definition{adj.}{pesaroso; arrependido; sentir pena de alguém porque você causou perda, inconveniência ou não atendeu às suas necessidades}
  \end{Phonetics}
\end{Entry}

%%%%%%%%%% 抵 %%%%%%%%%%
\subsection*{抵}\addcontentsline{loh}{figure}{抵}

\begin{Entry}{抵}{8}{⼿}
  \begin{Phonetics}{抵}{di3}
    \definition{v.}{apoiar; sustentar | resistir; suportar | compensar; fazer o bem | hipotecar; dar como garantia; garantir | equilibrar; cancelar; compensar | ser igual a; corresponder | alcançar; chegar a | colidir; dar cabeçada (por animais com chifres)}
  \end{Phonetics}
\end{Entry}

\begin{Entry}{抵达}{8,6}{⼿,⾡}
  \begin{Phonetics}{抵达}{di3da2}[][HSK 6]
    \definition{v.}{chegar; alcançar}
  \end{Phonetics}
\end{Entry}

\begin{Entry}{抵抗}{8,7}{⼿,⼿}
  \begin{Phonetics}{抵抗}{di3kang4}[][HSK 6]
    \definition{s.}{resistência}
    \definition{v.}{resistir; usar ação para resistir ou parar o ataque da outra parte}
  \end{Phonetics}
\end{Entry}

\begin{Entry}{抵制}{8,8}{⼿,⼑}
  \begin{Phonetics}{抵制}{di3zhi4}[][HSK 7-9]
    \definition{v.}{resistir; boicotar; bloquear, prevenir e impedir que forças externas invadam ou causem danos}
  \end{Phonetics}
\end{Entry}

\begin{Entry}{抵押}{8,8}{⼿,⼿}
  \begin{Phonetics}{抵押}{di3ya1}[][HSK 7-9]
    \definition{s.}{hipoteca; segurança; garantia}
    \definition{v.}{hipotecar; manter em penhor; penhorar}
  \end{Phonetics}
\end{Entry}

\begin{Entry}{抵挡}{8,9}{⼿,⼿}
  \begin{Phonetics}{抵挡}{di3dang3}[][HSK 7-9]
    \definition{v.}{resistir; suportar; bloquear}
  \end{Phonetics}
\end{Entry}

\begin{Entry}{抵消}{8,10}{⼿,⽔}
  \begin{Phonetics}{抵消}{di3xiao1}[][HSK 7-9]
    \definition{v.}{compensar; neutralizar; anular; cancelar}
  \antonymref{加强}{jia1qiang2}
  \end{Phonetics}
\end{Entry}

\begin{Entry}{抵御}{8,12}{⼿,⼻}
  \begin{Phonetics}{抵御}{di3yu4}[][HSK 7-9]
    \definition{v.}{resistir; suportar; afastar}[我们要抵御外敌的侵略。===Devemos resistir à invasão estrangeira.]
  \end{Phonetics}
\end{Entry}

\begin{Entry}{抵销}{8,12}{⼿,⾦}
  \begin{Phonetics}{抵销}{di3xiao1}
    \definition{v.}{compensar}[三笔债务可以抵销。===As três dívidas podem ser compensadas.]
  \end{Phonetics}
\end{Entry}

\begin{Entry}{抵触}{8,13}{⼿,⾓}
  \begin{Phonetics}{抵触}{di3chu4}[][HSK 7-9]
    \definition{adj.}{conflitante; contraditório}
    \definition{s.}{conflito}
    \definition{v.}{entrar em conflito; contradizer}
  \end{Phonetics}
\end{Entry}

%%%%%%%%%% 抹 %%%%%%%%%%
\subsection*{抹}\addcontentsline{loh}{figure}{抹}

\begin{Entry}{抹}{8}{⼿}
  \begin{Phonetics}{抹}{ma1}
    \definition{v.}{esfregar; limpar | deslizar algo para fora; tirar}
  \end{Phonetics}
  \begin{Phonetics}{抹}{mo3}[][HSK 7-9]
    \definition{v.}{colocar; aplicar; untar; engessar | limpar | anular; apagar | (para nuvem, etc.) irradiar; raiar; riscar; traçar | riscar; cancelar; marcar; remover; excluir}
  \end{Phonetics}
  \begin{Phonetics}{抹}{mo4}
    \definition{v.}{rebocar; engessar; alisar a massa ou o gesso com uma espátula | virar; contornar; dar uma volta de perto}
  \end{Phonetics}
\end{Entry}

\begin{Entry}{抹泪}{8,8}{⼿,⽔}
  \begin{Phonetics}{抹泪}{mo3lei4}
    \definition{v.}{limpar as lágrimas | Figurativo: derramar lágrimas}
  \end{Phonetics}
\end{Entry}

%%%%%%%%%% 押 %%%%%%%%%%
\subsection*{押}\addcontentsline{loh}{figure}{押}

\begin{Entry}{押}{8}{⼿}
  \begin{Phonetics}{押}{ya1}
    \definition*{s.}{Sobrenome: Ya}
    \definition{s.}{assinatura; marca em vez de assinatura; nome assinado ou símbolo desenhado}
    \definition{v.}{dar como garantia; hipotecar; penhorar | deter; levar sob custódia | escoltar | assinar (um documento, contrato, etc.); colocar sua assinatura (ou marcar no lugar da assinatura)}
  \end{Phonetics}
\end{Entry}

\begin{Entry}{押后}{8,6}{⼿,⼝}
  \begin{Phonetics}{押后}{ya1hou4}
    \definition{v.}{encerrar | adiar}
  \end{Phonetics}
\end{Entry}

\begin{Entry}{押运}{8,7}{⼿,⾡}
  \begin{Phonetics}{押运}{ya1yun4}
    \definition{v.}{escoltar sob guarda | escoltar (bens ou fundos)}
  \end{Phonetics}
\end{Entry}

\begin{Entry}{押注}{8,8}{⼿,⽔}
  \begin{Phonetics}{押注}{ya1zhu4}
    \definition{v.}{apostar}
  \end{Phonetics}
\end{Entry}

\begin{Entry}{押金}{8,8}{⼿,⾦}
  \begin{Phonetics}{押金}{ya1jin1}[][HSK 5]
    \definition[笔,份,些]{s.}{caução; sinal; depósito; dinheiro como garantia}
  \end{Phonetics}
\end{Entry}

\begin{Entry}{押送}{8,9}{⼿,⾡}
  \begin{Phonetics}{押送}{ya1song4}
    \definition{v.}{enviar sob escolta | transportar um detido}
  \end{Phonetics}
\end{Entry}

\begin{Entry}{押租}{8,10}{⼿,⽲}
  \begin{Phonetics}{押租}{ya1zu1}
    \definition{s.}{depósito de aluguel}
  \end{Phonetics}
\end{Entry}

\begin{Entry}{押韵}{8,13}{⼿,⾳}
  \begin{Phonetics}{押韵}{ya1yun4}
    \definition{v.}{rimar}
  \end{Phonetics}
\end{Entry}

%%%%%%%%%% 抽 %%%%%%%%%%
\subsection*{抽}\addcontentsline{loh}{figure}{抽}

\begin{Entry}{抽}{8}{⼿}
  \begin{Phonetics}{抽}{chou1}[][HSK 4]
    \definition{v.}{retirar; tirar (do meio); retirar, puxar ou arrancar algo que está preso ou emaranhado em outra coisa | tirar, retirar (uma parte de um todo) | (certas plantas) começar a crescer, produzir | bombear | encolher; contrair | chicotear; açoitar; surrar | dirigir; conduzir | encontrar tempo; libertar"-se; sair de alguma coisa}
  \end{Phonetics}
\end{Entry}

\begin{Entry}{抽屉}{8,8}{⼿,⼫}
  \begin{Phonetics}{抽屉}{chou1ti4}[][HSK 7-9]
    \definition[个,层,组]{s.}{gaveta}
  \end{Phonetics}
\end{Entry}

\begin{Entry}{抽奖}{8,9}{⼿,⼤}
  \begin{Phonetics}{抽奖}{chou1 jiang3}[][HSK 4]
    \definition{s.}{loteria; sorteio de loteria}
  \end{Phonetics}
\end{Entry}

\begin{Entry}{抽烟}{8,10}{⼿,⽕}
  \begin{Phonetics}{抽烟}{chou1/yan1}[][HSK 4]
    \definition{v.+compl.}{fumar (um cigarro ou um cachimbo)}
  \end{Phonetics}
\end{Entry}

\begin{Entry}{抽象}{8,11}{⼿,⾗}
  \begin{Phonetics}{抽象}{chou1xiang4}[][HSK 7-9]
    \definition{adj.}{abstrato}[抽象的艺术需要想象力。===A arte abstrata requer imaginação.]
    \definition{v.}{abstrair}[这个理论很难抽象。===Essa teoria é difícil de abstrair.]
  \end{Phonetics}
\end{Entry}

\begin{Entry}{抽签}{8,13}{⼿,⽵}
  \begin{Phonetics}{抽签}{chou1/qian1}[][HSK 7-9]
    \definition{v.+compl.}{tirar/lançar sorte; realizar/fazer um sorteio}[他们抽签决定胜者。===Eles fizeram um sorteio para decidir o vencedor.]
  \end{Phonetics}
\end{Entry}

%%%%%%%%%% 担 %%%%%%%%%%
\subsection*{担}\addcontentsline{loh}{figure}{担}

\begin{Entry}{担}{8}{⼿}
  \begin{Phonetics}{担}{dan1}[][HSK 7-9]
    \definition{v.}{carregar em uma vara de ombro e baldes; carregar nos ombros | assumir; empreender; não ter medo de correr riscos}
  \end{Phonetics}
  \begin{Phonetics}{担}{dan4}[][HSK 7-9]
    \definition{clas.}{dan, uma unidade de peso (=50 quilogramas) ; 100 jin = 1 dan | usado em coisas usadas para transportar cargas}
    \definition{s.}{carga; fardo; cargas de mercadorias transportadas em uma vara de ombro por um mascate itinerante}
  \end{Phonetics}
\end{Entry}

\begin{Entry}{担子}{8,3}{⼿,⼦}
  \begin{Phonetics}{担子}{dan4zi5}[][HSK 7-9]
    \definition[副,个]{s.}{vara de transporte (ou de ombro) e as cargas sob ela; canga; carga; fardo | tarefa}
  \end{Phonetics}
\end{Entry}

\begin{Entry}{担心}{8,4}{⼿,⼼}
  \begin{Phonetics}{担心}{dan1/xin1}[][HSK 4]
    \definition{v.+compl.}{preocupar"-se; ficar ansioso; sentir"-se desconfortável com algo}
  \end{Phonetics}
\end{Entry}

\begin{Entry}{担任}{8,6}{⼿,⼈}
  \begin{Phonetics}{担任}{dan1ren4}[][HSK 4]
    \definition{v.}{servir como; assumir o cargo de; ocupar o posto de; ocupar um determinado cargo ou emprego}
  \end{Phonetics}
\end{Entry}

\begin{Entry}{担当}{8,6}{⼿,⼹}
  \begin{Phonetics}{担当}{dan1dang1}[][HSK 7-9]
    \definition{v.}{aceitar e assumir responsabilidade; empreender (responsabilidade, trabalho, despesas)}
  \end{Phonetics}
\end{Entry}

\begin{Entry}{担负}{8,6}{⼿,⾙}
  \begin{Phonetics}{担负}{dan1fu4}[][HSK 7-9]
    \definition{v.}{suportar; carregar; assumir; ser encarregado de}
  \end{Phonetics}
\end{Entry}

\begin{Entry}{担忧}{8,7}{⼿,⼼}
  \begin{Phonetics}{担忧}{dan1you1}[][HSK 6]
    \definition[项,条,套,种]{v.}{preocupar"-se; estar ansioso}
  \end{Phonetics}
\end{Entry}

\begin{Entry}{担保}{8,9}{⼿,⼈}
  \begin{Phonetics}{担保}{dan1bao3}[][HSK 4]
    \definition{v.}{garantir; atestar; expressar responsabilidade e garantir que não haverá problemas ou que eles serão resolvidos}
  \end{Phonetics}
\end{Entry}

%%%%%%%%%% 拆 %%%%%%%%%%
\subsection*{拆}\addcontentsline{loh}{figure}{拆}

\begin{Entry}{拆}{8}{⼿}
  \begin{Phonetics}{拆}{chai1}[][HSK 5]
    \definition{v.}{rasgar; desmontar; separar o que está unido | derrubar; desmantelar; demolir; refere"-se especificamente à demolição de edifícios}
  \end{Phonetics}
\end{Entry}

\begin{Entry}{拆迁}{8,6}{⼿,⾡}
  \begin{Phonetics}{拆迁}{chai1qian1}[][HSK 6]
    \definition{v.}{demolir uma casa velha e realocar seus ocupantes em outro lugar; devido às necessidades de construção, unidades ou casas residenciais são demolidas e realocadas em outros lugares}
  \end{Phonetics}
\end{Entry}

\begin{Entry}{拆除}{8,9}{⼿,⾩}
  \begin{Phonetics}{拆除}{chai1chu2}[][HSK 5]
    \definition{v.}{desmantelar; demolir; derrubar; remover (um edifício, etc.)}
  \end{Phonetics}
\end{Entry}

%%%%%%%%%% 拉 %%%%%%%%%%
\subsection*{拉}\addcontentsline{loh}{figure}{拉}

\begin{Entry}{拉}{8}{⼿}
  \begin{Phonetics}{拉}{la1}[][HSK 2]
    \definition{s.}{abreviação de América Latina, 拉丁美洲}
    \definition{v.}{puxar; arrastar; rebocar | transportar por veículo; rebocar | arrastar (ou puxar) para fora | mover (tropas para um lugar) | dar uma mãozinha; ajudar | arrastar para dentro; implicar; envolver | criar (criança) | atrair; conquistar; solicitar; angariar votos | bater"-papo | organizar; preparar | ter dívidas; estar endividado | pressionar; recrutar à força | (no tênis, tênis de mesa, etc.) levantar (a bola) | tocar (certos instrumentos musicais); puxar uma parte do instrumento para que ele emita som | prolongar; espaçar | envolver"-se em | Coloquial: esvaziar os intestinos | levantar, uma das técnicas do tênis de mesa | destruir; esmagar; quebrar}
  \seealsoref{拉丁美洲}{la1ding1 mei3zhou1}
  \end{Phonetics}
  \begin{Phonetics}{拉}{la4}
    \definition{s.}{usado em 拉拉蛄}
  \seealsoref{拉拉蛄}{la4la4gu3}
  \end{Phonetics}
\end{Entry}

\begin{Entry}{拉丁美洲}{8,2,9,9}{⼿,⼀,⽺,⽔}
  \begin{Phonetics}{拉丁美洲}{la1ding1 mei3zhou1}
    \definition*{s.}{América Latina, nome coletivo dos países da América Central e do Sul, devido ao fato de a maioria de seus habitantes ser descendente de povos latinos e de a língua falada ser do grupo latino}
  \end{Phonetics}
\end{Entry}

\begin{Entry}{拉开}{8,4}{⼿,⼶}
  \begin{Phonetics}{拉开}{la1/kai5}[][HSK 4]
    \definition{v.+compl.}{puxar para abrir; recuar | ampliar; espaçar; distanciar; afastar; separar}
  \end{Phonetics}
\end{Entry}

\begin{Entry}{拉布布}{8,5,5}{⼿,⼱,⼱}
  \begin{Phonetics}{拉布布}{la1bu4bu4}
    \definition*{s.}{Labubu}
  \end{Phonetics}
\end{Entry}

\begin{Entry}{拉动}{8,6}{⼿,⼒}
  \begin{Phonetics}{拉动}{la1dong4}[][HSK 7-9]
    \definition{v.}{estimular; impulsionar; promover; estimular ou promover o desenvolvimento de um setor específico (geralmente relacionado à economia)}
  \end{Phonetics}
\end{Entry}

\begin{Entry}{拉拉队}{8,8,4}{⼿,⼿,⾩}
  \begin{Phonetics}{拉拉队}{la1la1dui4}
    \definition{s.}{torcida organizada; torcedores | também escrito como 啦啦队 | equipe de líderes de torcida}
  \seealsoref{啦啦队}{la1la1dui4}
  \end{Phonetics}
\end{Entry}

\begin{Entry}{拉拉蛄}{8,8,11}{⼿,⼿,⾍}
  \begin{Phonetics}{拉拉蛄}{la4la4gu3}
    \variantof{蝲蝲蛄}
  \end{Phonetics}
\end{Entry}

\begin{Entry}{拉拢}{8,8}{⼿,⼿}
  \begin{Phonetics}{拉拢}{la1long3}[][HSK 7-9]
    \definition{v.}{atrair alguém para o seu lado; conquistar; envolver | aconchegar"-se; envolver"-se; utilizar"-se de métodos para atrair outros para o próprio lado para benefício próprio}
  \antonymref{排挤}{pai2ji3}
  \end{Phonetics}
\end{Entry}

\begin{Entry}{拉萨}{8,11}{⼿,⾋}
  \begin{Phonetics}{拉萨}{la1sa4}
    \definition*{s.}{Lhasa, capital da Região Autônoma do Tibete, 西藏自治区}
  \seealsoref{西藏自治区}{xi1zang4 zi4zhi4qu1}
  \end{Phonetics}
\end{Entry}

\begin{Entry}{拉锁}{8,12}{⼿,⾦}
  \begin{Phonetics}{拉锁}{la1suo3}[][HSK 7-9]
    \definition{s.}{zíper; um acessório de metal ou plástico em forma de corrente que pode ser separado e travado, usado para ser costurado em roupas, bolsos ou bolsas}
  \end{Phonetics}
\end{Entry}

%%%%%%%%%% 拊 %%%%%%%%%%
\subsection*{拊}\addcontentsline{loh}{figure}{拊}

\begin{Entry}{拊}{8}{⼿}
  \begin{Phonetics}{拊}{fu3}
    \definition{v.}{Literário: bater palmas; esbofetear; golpear}
  \end{Phonetics}
\end{Entry}

%%%%%%%%%% 拌 %%%%%%%%%%
\subsection*{拌}\addcontentsline{loh}{figure}{拌}

\begin{Entry}{拌}{8}{⼿}
  \begin{Phonetics}{拌}{ban4}[][HSK 7-9]
    \definition{v.}{misturar | mexer e misturar | discutir; brigar; ter uma discussão}
  \end{Phonetics}
\end{Entry}

%%%%%%%%%% 拍 %%%%%%%%%%
\subsection*{拍}\addcontentsline{loh}{figure}{拍}

\begin{Entry}{拍}{8}{⼿}
  \begin{Phonetics}{拍}{pai1}[][HSK 3]
    \definition[个,副,对]{s.}{bastão; raquete | batida; tempo; (música) uma unidade para medir a duração de uma nota musical}
    \definition{v.}{tirar (uma foto); usar uma câmera para capturar imagens de pessoas e objetos em filme | dar um tapinha; bater suavemente com as mãos ou ferramentas | bater asas | bater (ondas do mar) | enviar (um telegrama, etc.) | bajular}
  \end{Phonetics}
\end{Entry}

\begin{Entry}{拍马}{8,3}{⼿,⾺}
  \begin{Phonetics}{拍马}{pai1ma3}
    \definition{v.}{instigar um cavalo dando tapinhas em seu traseiro | lisonjear | bajular}
  \seealsoref{拍马屁}{pai1ma3pi4}
  \end{Phonetics}
\end{Entry}

\begin{Entry}{拍马屁}{8,3,7}{⼿,⾺,⼫}
  \begin{Phonetics}{拍马屁}{pai1ma3pi4}
    \definition{s.}{puxa-saco | bajulador}
    \definition{v.}{puxar o saco | bajular}
  \seealsoref{拍马}{pai1ma3}
  \end{Phonetics}
\end{Entry}

\begin{Entry}{拍戏}{8,6}{⼿,⼽}
  \begin{Phonetics}{拍戏}{pai1/xi4}[][HSK 7-9]
    \definition{v.+compl.}{fazer um filme ou peça de televisão; filmar uma cena | filmar}
  \end{Phonetics}
\end{Entry}

\begin{Entry}{拍卖}{8,8}{⼿,⼗}
  \begin{Phonetics}{拍卖}{pai1mai4}[][HSK 7-9]
    \definition{s.}{leilão; uma forma pública de venda de produtos onde todos oferecem publicamente o seu preço, e o item é vendido para quem oferecer o preço mais alto}
    \definition{v.}{leiloar; realizar atividades de leilão | vender mercadorias a preços reduzidos; baixar o preço para vender as mercadorias rapidamente}
  \end{Phonetics}
\end{Entry}

\begin{Entry}{拍板}{8,8}{⼿,⽊}
  \begin{Phonetics}{拍板}{pai1/ban3}[][HSK 7-9]
    \definition{s.}{aplausos}
    \definition{v.+compl.}{marcar o tempo com palmas | bater o martelo | ter a palavra final; dar o veredicto final; tomar a decisão final}
  \end{Phonetics}
\end{Entry}

\begin{Entry}{拍摄}{8,13}{⼿,⼿}
  \begin{Phonetics}{拍摄}{pai1she4}[][HSK 5]
    \definition{s.}{fotografar; tirar (uma foto); usar uma câmera fotográfica para capturar imagens de pessoas e objetos}
  \end{Phonetics}
\end{Entry}

\begin{Entry}{拍照}{8,13}{⼿,⽕}
  \begin{Phonetics}{拍照}{pai1/zhao4}[][HSK 4]
    \definition{v.+compl.}{fotografar; tirar uma foto}
  \end{Phonetics}
\end{Entry}

%%%%%%%%%% 拎 %%%%%%%%%%
\subsection*{拎}\addcontentsline{loh}{figure}{拎}

\begin{Entry}{拎}{8}{⼿}
  \begin{Phonetics}{拎}{lin1}[][HSK 7-9]
    \definition{v.}{levantar; carregar}
  \end{Phonetics}
\end{Entry}

%%%%%%%%%% 拐 %%%%%%%%%%
\subsection*{拐}\addcontentsline{loh}{figure}{拐}

\begin{Entry}{拐}{8}{⼿}
  \begin{Phonetics}{拐}{guai3}[][HSK 6]
    \definition[支,根,副]{s.}{muleta; bengala; uma bengala com uma barra horizontal na parte superior, usada por pessoas com doenças ou deficiências nos membros inferiores para ajudá-las a caminhar |  sete; forma falada do numeral 七 | esquina; curva; canto}
    \definition{v.}{virar; girar; mudar de direção enquanto se move | enganar | mudar; transformar | mancar}
  \seealsoref{七}{qi1}
  \end{Phonetics}
\end{Entry}

\begin{Entry}{拐杖}{8,7}{⼿,⽊}
  \begin{Phonetics}{拐杖}{guai3zhang4}[][HSK 7-9]
    \definition[个,根,支,副]{s.}{muleta; bengala}
  \end{Phonetics}
\end{Entry}

\begin{Entry}{拐弯}{8,9}{⼿,⼸}
  \begin{Phonetics}{拐弯}{guai3/wan1}[][HSK 7-9]
    \definition[个]{s.}{esquina; curva; canto}
    \definition{v.}{virar; virar uma esquina; indica mudança de direção da viagem | dar meia"-volta; seguir um novo curso; indica mudança de ideias, linguagem, etc.}
  \end{Phonetics}
\end{Entry}

%%%%%%%%%% 拓 %%%%%%%%%%
\subsection*{拓}\addcontentsline{loh}{figure}{拓}

\begin{Entry}{拓}{8}{⼿}
  \begin{Phonetics}{拓}{ta4}
    \definition{v.}{fazer decalques de formas, textos e gráficos em inscrições em pedra, artefatos de bronze, etc.}
  \end{Phonetics}
  \begin{Phonetics}{拓}{tuo4}
    \definition*{s.}{Sobrenome: Tuo}
    \definition{v.}{abrir; desenvolver | expandir}
  \end{Phonetics}
\end{Entry}

\begin{Entry}{拓宽}{8,10}{⼿,⼧}
  \begin{Phonetics}{拓宽}{tuo4kuan1}[][HSK 7-9]
    \definition{v.}{ampliar; expandir}
  \synonymref{开拓}{kai1tuo4}
  \synonymref{拓展}{tuo4zhan3}
  \antonymref{收缩}{shou1suo1}
  \end{Phonetics}
\end{Entry}

\begin{Entry}{拓展}{8,10}{⼿,⼫}
  \begin{Phonetics}{拓展}{tuo4zhan3}[][HSK 7-9]
    \definition{v.}{expandir; desenvolver; disseminar; ampliar o escopo}
  \synonymref{开阔}{kai1kuo4}
  \synonymref{开拓}{kai1tuo4}
  \synonymref{扩展}{kuo4zhan3}
  \synonymref{拓宽}{tuo4kuan1}
  \synonymref{延伸}{yan2shen1}
  \antonymref{缩小}{suo1/xiao3}
  \end{Phonetics}
\end{Entry}

%%%%%%%%%% 拔 %%%%%%%%%%
\subsection*{拔}\addcontentsline{loh}{figure}{拔}

\begin{Entry}{拔}{8}{⼿}
  \begin{Phonetics}{拔}{ba2}[][HSK 5]
    \definition{v.aux.}{puxar para cima; puxar para fora; arrastar para fora | extrair; sugar | escolher; selecionar | superar; destacar"-se entre | apreender; capturar | esfriar na água; mergulhar algo em água fria para que esfrie}
  \end{Phonetics}
\end{Entry}

\begin{Entry}{拔尖}{8,6}{⼿,⼩}
  \begin{Phonetics}{拔尖}{ba2/jian1}
    \definition{adj.}{topo de linha | fora do comum | o melhor}
    \definition{v.+compl.}{empurrar"-se para a frente | sentir que é superior aos outros}
  \end{Phonetics}
\end{Entry}

%%%%%%%%%% 拖 %%%%%%%%%%
\subsection*{拖}\addcontentsline{loh}{figure}{拖}

\begin{Entry}{拖}{8}{⼿}
  \begin{Phonetics}{拖}{tuo1}[][HSK 6]
    \definition{v.}{puxar; arrastar; transportar; puxar um objeto para movê-lo contra o solo ou outra superfície | esfregar; limpar o chão com uma ferramenta especial para esfregar | atrasar; prolongar; procrastinar; arrastar; coisas que deveriam ser feitas nunca são iniciadas ou concluídas; uma certa nota é prolongada por um longo tempo | atrasar; conter; segurar; restringir}
  \end{Phonetics}
\end{Entry}

\begin{Entry}{拖欠}{8,4}{⼿,⽋}
  \begin{Phonetics}{拖欠}{tuo1qian4}[][HSK 7-9]
    \definition{v.}{estar em atraso; estar com pagamentos atrasados; atrasar a devolução ou reter o pagamento}
  \synonymref{拖延}{tuo1yan2}
  \antonymref{偿还}{chang2huan2}
  \end{Phonetics}
\end{Entry}

\begin{Entry}{拖延}{8,6}{⼿,⼵}
  \begin{Phonetics}{拖延}{tuo1yan2}[][HSK 7-9]
    \definition{v.}{adiar; postergar; procrastinar; pendurar; prolongar o tempo, não processar rapidamente}
  \synonymref{迟到}{chi2/dao4}
  \synonymref{耽搁}{dan1ge5}
  \synonymref{耽误}{dan1wu5}
  \synonymref{缓慢}{huan3man4}
  \synonymref{拖欠}{tuo1qian4}
  \end{Phonetics}
\end{Entry}

\begin{Entry}{拖拉机}{8,8,6}{⼿,⼿,⽊}
  \begin{Phonetics}{拖拉机}{tuo1la1ji1}
    \definition[台]{s.}{trator}
  \end{Phonetics}
\end{Entry}

\begin{Entry}{拖累}{8,11}{⼿,⽷}
  \begin{Phonetics}{拖累}{tuo1lei3}[][HSK 7-9]
    \definition{v.}{onerar; ser um fardo para; sobrecarregar outras pessoas ou coisas, impedindo seu desenvolvimento harmonioso}
  \antonymref{轻松}{qing1song1}
  \end{Phonetics}
\end{Entry}

\begin{Entry}{拖鞋}{8,15}{⼿,⾰}
  \begin{Phonetics}{拖鞋}{tuo1xie2}[][HSK 6]
    \definition[双,只]{s.}{chinelos; sandálias; babouche; sapatos sem cabedal geralmente são usados em ambientes fechados}
  \end{Phonetics}
\end{Entry}

%%%%%%%%%% 拘 %%%%%%%%%%
\subsection*{拘}\addcontentsline{loh}{figure}{拘}

\begin{Entry}{拘}{8}{⼿}
  \begin{Phonetics}{拘}{ju1}
    \definition{adj.}{inflexível; sem flexibilidade}
    \definition{v.}{prender; deter | restringir; limitar; constranger | aderir rigidamente; ser inflexível |limitar}
  \end{Phonetics}
\end{Entry}

\begin{Entry}{拘束}{8,7}{⼿,⽊}
  \begin{Phonetics}{拘束}{ju1shu4}[][HSK 7-9]
    \definition{adj.}{desajeitado; desconfortável; constrangido; reservado; não natural}
    \definition{v.}{restringir; limitar; restringir excessivamente as palavras e ações de outras pessoas}
  \end{Phonetics}
\end{Entry}

\begin{Entry}{拘留}{8,10}{⼿,⽥}
  \begin{Phonetics}{拘留}{ju1liu2}[][HSK 7-9]
    \definition{v.}{deter; manter sob custódia; colocar em prisão provisória; restringir a liberdade pessoal}
  \end{Phonetics}
\end{Entry}

%%%%%%%%%% 招 %%%%%%%%%%
\subsection*{招}\addcontentsline{loh}{figure}{招}

\begin{Entry}{招}{8}{⼿}
  \begin{Phonetics}{招}{zhao1}[][HSK 6]
    \definition*{s.}{Sobrenome: Zhao}
    \definition{s.}{\emph{banner}; Faixas e outros itens costumavam ser pendurados nas entradas de hotéis, restaurantes ou lojas para atrair clientes | movimento; estratagema; artifício; meios ou táticas | movimentos de artes marciais}
    \definition{v.}{acenar; gestuar para alguém ver | alistar; inscrever; recrutar | incorrer; cortejar; atrair; provocar (um certo resultado ou reação) | provocar; tocar ou provocar a outra pessoa com palavras ou ações | confessar (culpa); assumir (culpa) | infectar; ser contagioso}
  \end{Phonetics}
\end{Entry}

\begin{Entry}{招手}{8,4}{⼿,⼿}
  \begin{Phonetics}{招手}{zhao1/shou3}[][HSK 5]
    \definition{v.+compl.}{acenar; chamar a atenção; levantar a mão e acenar com a palma, para indicar que a outra pessoa se aproxime ou para cumprimentá-la}
  \end{Phonetics}
\end{Entry}

\begin{Entry}{招生}{8,5}{⼿,⽣}
  \begin{Phonetics}{招生}{zhao1/sheng1}[][HSK 5]
    \definition{v.+compl.}{conseguir alunos; matricular novos alunos; recrutar novos alunos}
  \end{Phonetics}
\end{Entry}

\begin{Entry}{招呼}{8,8}{⼿,⼝}
  \begin{Phonetics}{招呼}{zhao1hu5}[][HSK 4]
    \definition{v.}{chamar; chamar a atenção com palavras ou gestos | cumprimentar; saudar; cumprimentar ou despedir"-se das pessoas com palavras ou gestos | pedir a alguém para fazer algo; fazer solicitações, pedir ajuda ou fazer coisas | receber e dar boas"-vindas aos convidados}
  \end{Phonetics}
\end{Entry}

\begin{Entry}{招数}{8,13}{⼿,⽁}
  \begin{Phonetics}{招数}{zhao1shu4}
    \definition{s.}{estratégia | movimento (no xadrez, no palco, nas artes marciais) | esquema | truque}
  \end{Phonetics}
\end{Entry}

\begin{Entry}{招聘}{8,13}{⼿,⽿}
  \begin{Phonetics}{招聘}{zhao1pin4}[][HSK 6]
    \definition{v.}{contratar; procurar; recrutar; convidar candidatos para um emprego}
  \end{Phonetics}
\end{Entry}

%%%%%%%%%% 拣 %%%%%%%%%%
\subsection*{拣}\addcontentsline{loh}{figure}{拣}

\begin{Entry}{拣}{8}{⼿}
  \begin{Phonetics}{拣}{jian3}[][HSK 7-9]
    \definition{v.}{escolher; selecionar | pegar; coletar; reunir | o mesmo que 捡}
  \seealsoref{捡}{jian3}
  \end{Phonetics}
\end{Entry}

%%%%%%%%%% 拥 %%%%%%%%%%
\subsection*{拥}\addcontentsline{loh}{figure}{拥}

\begin{Entry}{拥}{8}{⼿}
  \begin{Phonetics}{拥}{yong1}
    \definition{v.}{segurar nos braços; abraçar | reunir em volta; envolver em volta | aglomerar"-se; enxamear | para apoiar | (literário) ter; possuir}
  \end{Phonetics}
\end{Entry}

\begin{Entry}{拥有}{8,6}{⼿,⽉}
  \begin{Phonetics}{拥有}{yong1you3}[][HSK 5]
    \definition{v.}{possuir; deter; ter (grande quantidade de terras, população, bens, etc.)}
  \end{Phonetics}
\end{Entry}

\begin{Entry}{拥抱}{8,8}{⼿,⼿}
  \begin{Phonetics}{拥抱}{yong1bao4}[][HSK 5]
    \definition[个,次]{s.}{abraço}
    \definition{v.}{abraçar; segurar em seus braços; abraçar para demonstrar afeto}
  \end{Phonetics}
\end{Entry}

%%%%%%%%%% 拦 %%%%%%%%%%
\subsection*{拦}\addcontentsline{loh}{figure}{拦}

\begin{Entry}{拦}{8}{⼿}
  \begin{Phonetics}{拦}{lan2}[][HSK 7-9]
    \definition{v.}{barrar; bloquear a passagem; dificultar; obstruir; impedir}
  \end{Phonetics}
\end{Entry}

%%%%%%%%%% 拧 %%%%%%%%%%
\subsection*{拧}\addcontentsline{loh}{figure}{拧}

\begin{Entry}{拧}{8}{⼿}
  \begin{Phonetics}{拧}{ning2}[][HSK 7-9]
    \definition{v.}{torcer | beliscar; torcer a pele com os dedos e virá-la com força}
  \end{Phonetics}
  \begin{Phonetics}{拧}{ning3}[][HSK 7-9]
    \definition{adj.}{errado; equivocado; de cabeça para baixo; oposto}
    \definition{v.}{torcer; parafusar | divergir; discordar; estar em desacordo}
  \end{Phonetics}
  \begin{Phonetics}{拧}{ning4}
    \definition{adj.}{teimoso}
  \end{Phonetics}
\end{Entry}

\begin{Entry}{拧开}{8,4}{⼿,⼶}
  \begin{Phonetics}{拧开}{ning3kai1}
    \definition{v.}{ligar ou desligar (girando um botão) | girar (a maçaneta de uma porta) | abrir (uma torneira) | desenroscar (uma tampa) | desaparafusar | arrancar à força}
  \end{Phonetics}
\end{Entry}

%%%%%%%%%% 拨 %%%%%%%%%%
\subsection*{拨}\addcontentsline{loh}{figure}{拨}

\begin{Entry}{拨}{8}{⼿}
  \begin{Phonetics}{拨}{bo1}[][HSK 7-9]
    \definition{clas.}{usado para agrupar pessoas; grupo; lote}
    \definition{v.}{mover (mexer) com a mão, o pé, o bastão, etc.; usar as mãos, os pés ou os bastões para mover objetos | atribuir; alocar; reservar | virar"-se; inverter a marcha | dedilhar (uma corda de violão) com os dedos ou com um instrumento | chamar (alguém)}
  \end{Phonetics}
\end{Entry}

\begin{Entry}{拨及}{8,3}{⼿,⼃}
  \begin{Phonetics}{拨及}{bo1ji2}
    \definition{v.}{espalhar para; envolver; afetar}
  \end{Phonetics}
\end{Entry}

\begin{Entry}{拨打}{8,5}{⼿,⼿}
  \begin{Phonetics}{拨打}{bo1da3}[][HSK 6]
    \definition{v.}{ligar; discar; de acordo com o número da chamada, discar o número no telefone ou pressionar as teclas numéricas para fazer uma chamada}
  \end{Phonetics}
\end{Entry}

\begin{Entry}{拨转}{8,8}{⼿,⾞}
  \begin{Phonetics}{拨转}{bo1zhuan3}
    \definition{v.}{transferir (fundos, etc.) | virar | dar a volta}
  \end{Phonetics}
\end{Entry}

\begin{Entry}{拨通}{8,10}{⼿,⾡}
  \begin{Phonetics}{拨通}{bo1/tong1}[][HSK 7-9]
    \definition{v.+compl.}{discar (os números de um telefone, etc.)}
  \end{Phonetics}
\end{Entry}

\begin{Entry}{拨款}{8,12}{⼿,⽋}
  \begin{Phonetics}{拨款}{bo1kuan3}[][HSK 7-9]
    \definition[项,笔]{s.}{dinheiro apropriado; apropriação; subsídio financeiro do estado; alocação de fundos; financiamento alocado}
    \definition{v.}{apropriar"-se de dinheiro; alocar fundos}
  \end{Phonetics}
\end{Entry}

%%%%%%%%%% 放 %%%%%%%%%%
\subsection*{放}\addcontentsline{loh}{figure}{放}

\begin{Entry}{放}{8}{⽅}
  \begin{Phonetics}{放}{fang4}[][HSK 1]
    \definition{v.}{deixar ir; libertar; soltar | ceder; deixar"-se levar | levar para se alimentar; pastar | soltar; liberar (ou expelir) | exibir (um filme, etc.); reproduzir (um disco, etc.) | acender; inflamar | emprestar (dinheiro) com juros | tornar maior ou mais longo; soltar; abaixar | moderar (a atitude ou o comportamento de alguém) | (de flores) florescer; abrir | colocar; posicionar; deitar | fazer com que algo (ou alguém) caia no chão | deixar de lado; guardar (para uso futuro); conservar | (seguido por 着\dots 不\dots) permitir que algo permaneça (por fazer, por pegar, por usar, etc.) | adicionar; colocar | colocar em pastagem; soltar para caçar | deixar de lado; suspender; interromper | remover; aliviar; livrar"-se; proteger; libertar | deixar"-se levar; sem restrições; libertino | mandar embora; tirar o prisioneiro da prisão e deportá"-lo para uma região remota | distribuir; emitir; lançar | atear fogo | expandir; ampliar; prolongar | reajustar"-se até certo ponto; controlar suas ações, adotar uma determinada atitude, atingir um certo equilíbrio | derrubar}
  \end{Phonetics}
\end{Entry}

\begin{Entry}{放下}{8,3}{⽅,⼀}
  \begin{Phonetics}{放下}{fang4xia4}[][HSK 2]
    \definition{v.}{deitar-se; colocar no chão| deixar ir; soltar; desistir; largar | colocar; acomodar; depositar}
  \end{Phonetics}
\end{Entry}

\begin{Entry}{放大}{8,3}{⽅,⼤}
  \begin{Phonetics}{放大}{fang4/da4}[][HSK 5]
    \definition{v.+compl.}{amplificar; magnificar; aumentar; ampliar; aumentar o tamanho de imagens, textos, sons, etc.}
  \end{Phonetics}
\end{Entry}

\begin{Entry}{放飞}{8,3}{⽅,⾶}
  \begin{Phonetics}{放飞}{fang4fei1}
    \definition{s.}{deixar voar}
  \end{Phonetics}
\end{Entry}

\begin{Entry}{放心}{8,4}{⽅,⼼}
  \begin{Phonetics}{放心}{fang4/xin1}[][HSK 2]
    \definition{adj.}{despreocupado}
    \definition{v.+compl.}{confiar; ter confiança em alguém; sentir"-se aliviado; ficar tranquilo; ficar com a consciência tranquila}
  \end{Phonetics}
\end{Entry}

\begin{Entry}{放水}{8,4}{⽅,⽔}
  \begin{Phonetics}{放水}{fang4/shui3}[][HSK 7-9]
    \definition{v.+compl.}{ligar a água; deixar a água fluir, geralmente significa abrir a fonte de água ou fornecer uma determinada vazão de água | (reservatório, etc.) retirar água; drenar água de reservatórios, lagoas, etc. para irrigação ou outros fins | (em uma competição, etc.) facilitar as coisas para alguém; perder um jogo intencionalmente; deixar deliberadamente o adversário vencer facilmente durante uma partida}
  \end{Phonetics}
\end{Entry}

\begin{Entry}{放出}{8,5}{⽅,⼐}
  \begin{Phonetics}{放出}{fang4chu1}
    \definition{v.}{liberar | libertar}
  \end{Phonetics}
\end{Entry}

\begin{Entry}{放电}{8,5}{⽅,⽥}
  \begin{Phonetics}{放电}{fang4dian4}
    \definition{s.}{descarga elétrica}
  \end{Phonetics}
\end{Entry}

\begin{Entry}{放任}{8,6}{⽅,⼈}
  \begin{Phonetics}{放任}{fang4ren4}
    \definition{v.}{ignorar | saciar-se | deixar sozinho}
  \end{Phonetics}
\end{Entry}

\begin{Entry}{放过}{8,6}{⽅,⾡}
  \begin{Phonetics}{放过}{fang4guo4}[][HSK 7-9]
    \definition{v.}{deixar escapar; perder}
  \end{Phonetics}
\end{Entry}

\begin{Entry}{放弃}{8,7}{⽅,⼶}
  \begin{Phonetics}{放弃}{fang4qi4}[][HSK 5]
    \definition{v.}{desistir, abandonar; descartar (direitos originais, reivindicações, opiniões, etc.)}
  \end{Phonetics}
\end{Entry}

\begin{Entry}{放弃权利}{8,7,6,7}{⽅,⼶,⽊,⼑}
  \begin{Phonetics}{放弃权利}{fang4qi4 quan2li4}
    \definition{s.}{renúncia}
  \end{Phonetics}
\end{Entry}

\begin{Entry}{放弃者}{8,7,8}{⽅,⼶,⽼}
  \begin{Phonetics}{放弃者}{fang4qi4zhe3}
    \definition{s.}{desistente}
  \end{Phonetics}
\end{Entry}

\begin{Entry}{放纵}{8,7}{⽅,⽷}
  \begin{Phonetics}{放纵}{fang4zong4}[][HSK 7-9]
    \definition{adj.}{grosseiro; inculto; autoindulgente; indisciplinado}
    \definition{v.}{satisfazer; ser conivente com; bajular; deixar alguém fazer o que quer}
  \end{Phonetics}
\end{Entry}

\begin{Entry}{放走}{8,7}{⽅,⾛}
  \begin{Phonetics}{放走}{fang4zou3}
    \definition{v.}{permitir (uma pessoa ou um animal) ir | liberar | libertar}
  \end{Phonetics}
\end{Entry}

\begin{Entry}{放到}{8,8}{⽅,⼑}
  \begin{Phonetics}{放到}{fang4 dao4}[][HSK 3]
    \definition{v.}{colocar em; meter}
  \end{Phonetics}
\end{Entry}

\begin{Entry}{放学}{8,8}{⽅,⼦}
  \begin{Phonetics}{放学}{fang4/xue2}[][HSK 1]
    \definition{v.+compl.}{encerrar; sair da escola; as aulas terminaram; a escola acabou (por hoje); voltar para casa depois de um dia ou meio dia de aula}
  \end{Phonetics}
\end{Entry}

\begin{Entry}{放松}{8,8}{⽅,⽊}
  \begin{Phonetics}{放松}{fang4song1}[][HSK 4]
    \definition{v.}{relaxar; afrouxar; soltar; desprender}
  \end{Phonetics}
\end{Entry}

\begin{Entry}{放养}{8,9}{⽅,⼋}
  \begin{Phonetics}{放养}{fang4yang3}
    \definition{v.}{criar (gado, peixes, culturas, etc.) | crescer | criar}
  \end{Phonetics}
\end{Entry}

\begin{Entry}{放映}{8,9}{⽅,⽇}
  \begin{Phonetics}{放映}{fang4ying4}[][HSK 7-9]
    \definition{v.}{mostrar (um filme); exibir; projetar; usar um dispositivo de luz forte para iluminar a imagem de uma foto ou filme em uma tela ou parede}
  \end{Phonetics}
\end{Entry}

\begin{Entry}{放假}{8,11}{⽅,⼈}
  \begin{Phonetics}{放假}{fang4/jia4}[][HSK 1]
    \definition{v.}{tirar férias (ou feriado); ter um dia de folga}
    \definition{v.+compl.}{tirar férias (ou feriado); começar as férias; ter um dia de folga; estar de férias (feriado)}
  \end{Phonetics}
\end{Entry}

\begin{Entry}{放置}{8,13}{⽅,⽹}
  \begin{Phonetics}{放置}{fang4zhi4}[][HSK 7-9]
    \definition{v.}{colocar; deitar; deixar de lado}
  \end{Phonetics}
\end{Entry}

\begin{Entry}{放肆}{8,13}{⽅,⾀}
  \begin{Phonetics}{放肆}{fang4si4}[][HSK 7-9]
    \definition{adj.}{desenfreado; devasso; atrevido; descontrolado; descreve agir de forma imprudente e sem escrúpulos}
  \end{Phonetics}
\end{Entry}

\begin{Entry}{放鞭炮}{8,18,9}{⽅,⾰,⽕}
  \begin{Phonetics}{放鞭炮}{fang4bian1pao4}
    \definition{s.}{um conjunto de bombinhas ou traques}
  \end{Phonetics}
\end{Entry}

%%%%%%%%%% 斧 %%%%%%%%%%
\subsection*{斧}\addcontentsline{loh}{figure}{斧}

\begin{Entry}{斧}{8}{⽄}
  \begin{Phonetics}{斧}{fu3}
    \definition[把,只]{s.}{machado; machadinha | machado de batalha (um tipo de arma usada na China antiga)}
  \end{Phonetics}
\end{Entry}

\begin{Entry}{斧子}{8,3}{⽄,⼦}
  \begin{Phonetics}{斧子}{fu3zi5}[][HSK 7-9]
    \definition[把,个]{s.}{machado; machadinha}
  \end{Phonetics}
\end{Entry}

%%%%%%%%%% 斩 %%%%%%%%%%
\subsection*{斩}\addcontentsline{loh}{figure}{斩}

\begin{Entry}{斩}{8}{⽄}
  \begin{Phonetics}{斩}{zhan3}
    \definition*{s.}{Sobrenome: Zhan}
    \definition{v.}{matar; cortar; picar | (dialeto) tosquiar; chantagear | decapitar}
  \end{Phonetics}
\end{Entry}

\begin{Entry}{斩获}{8,10}{⽄,⾋}
  \begin{Phonetics}{斩获}{zhan3huo4}
    \definition{v.}{matar ou capturar (em batalha) | (figurativo) (esportes) marcar (um gol), ganhar (uma medalha) | (figurativo) colher recompensas, obter ganhos}
  \end{Phonetics}
\end{Entry}

%%%%%%%%%% 旺 %%%%%%%%%%
\subsection*{旺}\addcontentsline{loh}{figure}{旺}

\begin{Entry}{旺}{8}{⽇}
  \begin{Phonetics}{旺}{wang4}[][HSK 7-9]
    \definition{adj.}{próspero; florescente; vigoroso | abundante; numeroso}
  \end{Phonetics}
\end{Entry}

\begin{Entry}{旺季}{8,8}{⽇,⼦}
  \begin{Phonetics}{旺季}{wang4ji4}[][HSK 7-9]
    \definition{s.}{alta temporada; período de pico; temporada movimentada; a estação em que um determinado produto é produzido em grandes quantidades ou quando os negócios estão crescendo}
  \seealsoref{淡季}{dan4ji4}
  \antonymref{淡季}{dan4ji4}
  \end{Phonetics}
\end{Entry}

\begin{Entry}{旺盛}{8,11}{⽇,⽫}
  \begin{Phonetics}{旺盛}{wang4sheng4}[][HSK 7-9]
    \definition{adj.}{vigoroso; exuberante; de forte vitalidade; de bom humor}
  \synonymref{繁华}{fan2hua2}
  \synonymref{繁荣}{fan2rong2}
  \synonymref{红火}{hong2huo5}
  \synonymref{焕发}{huan4fa1}
  \synonymref{茂盛}{mao4sheng4}
  \synonymref{蓬勃}{peng2bo2}
  \synonymref{兴旺}{xing1wang4}
  \antonymref{衰退}{shuai1tui4}
  \end{Phonetics}
\end{Entry}

%%%%%%%%%% 昂 %%%%%%%%%%
\subsection*{昂}\addcontentsline{loh}{figure}{昂}

\begin{Entry}{昂}{8}{⽇}
  \begin{Phonetics}{昂}{ang2}
    \definition{adj.}{alto; subindo}
    \definition{v.}{manter (a cabeça) erguida | elevar; levantar; olhar para cima}
  \end{Phonetics}
\end{Entry}

\begin{Entry}{昂贵}{8,9}{⽇,⾙}
  \begin{Phonetics}{昂贵}{ang2gui4}[][HSK 7-9]
    \definition{adj.}{caro; dispendioso; algo é muito caro, o preço é particularmente alto; metaforicamente, o custo de fazer algo é particularmente alto}
  \end{Phonetics}
\end{Entry}

%%%%%%%%%% 昆 %%%%%%%%%%
\subsection*{昆}\addcontentsline{loh}{figure}{昆}

\begin{Entry}{昆}{8}{⽇}
  \begin{Phonetics}{昆}{kun1}
    \definition*{s.}{Sobrenome: Kun}
    \definition{s.}{irmão mais velho | descendentes; filhos}
  \end{Phonetics}
\end{Entry}

\begin{Entry}{昆虫}{8,6}{⽇,⾍}
  \begin{Phonetics}{昆虫}{kun1chong2}[][HSK 7-9]
    \definition[只,种,个,群,堆]{s.}{inseto; uma classe de artrópodes}
  \end{Phonetics}
\end{Entry}

%%%%%%%%%% 昌 %%%%%%%%%%
\subsection*{昌}\addcontentsline{loh}{figure}{昌}

\begin{Entry}{昌}{8}{⽇}
  \begin{Phonetics}{昌}{chang1}
    \definition*{s.}{Sobrenome: Chang}
    \definition{adj.}{próspero; florescente | adequado; bom}
  \end{Phonetics}
\end{Entry}

\begin{Entry}{昌盛}{8,11}{⽇,⽫}
  \begin{Phonetics}{昌盛}{chang1sheng4}[][HSK 6]
    \definition{adj.}{(país, nação, etc.) próspero; florescente}
  \end{Phonetics}
\end{Entry}

%%%%%%%%%% 明 %%%%%%%%%%
\subsection*{明}\addcontentsline{loh}{figure}{明}

\begin{Entry}{明}{8}{⽇}
  \begin{Phonetics}{明}{ming2}
    \definition*{s.}{Dinastia Ming (1368--1644) | Sobrenome: Ming}
    \definition{adj.}{claro; brilhante; brilhante | claro; distinto; de fácil entendimento | aberto; evidente; explícito; exposto | de olhos aguçados; boa visão; visão nítida | honesto}
    \definition{adv.}{claramente; definitivamente; aparentemente; de fato}
    \definition{s.}{imediatamente a seguir no tempo; ao lado deste ano e hoje; visão}
    \definition{v.}{mostrar; revelar; tornar conhecido; deixar claro | entender; compreender}
  \end{Phonetics}
\end{Entry}

\begin{Entry}{明天}{8,4}{⽇,⼤}
  \begin{Phonetics}{明天}{ming2tian1}[][HSK 1]
    \definition{s.}{amanhã | futuro próximo}
  \end{Phonetics}
\end{Entry}

\begin{Entry}{明日}{8,4}{⽇,⽇}
  \begin{Phonetics}{明日}{ming2ri4}[][HSK 6]
    \definition{s.}{amanhã}
  \seealsoref{明天}{ming2tian1}
  \end{Phonetics}
\end{Entry}

\begin{Entry}{明白}{8,5}{⽇,⽩}
  \begin{Phonetics}{明白}{ming2bai5}[][HSK 1]
    \definition{adj.}{claro; óbvio; evidente; inequívoco | sensato; razoável | aberto; franco; inequívoco; explícito}
    \definition{v.}{entender; compreender; saber}
  \end{Phonetics}
\end{Entry}

\begin{Entry}{明年}{8,6}{⽇,⼲}
  \begin{Phonetics}{明年}{ming2nian2}[][HSK 1]
    \definition{s.}{próximo ano}
  \end{Phonetics}
\end{Entry}

\begin{Entry}{明明}{8,8}{⽇,⽇}
  \begin{Phonetics}{明明}{ming2ming2}[][HSK 5]
    \definition{adv.}{obviamente; claramente; sem dúvida; indica que o fenômeno ou princípio é evidente}
  \end{Phonetics}
\end{Entry}

\begin{Entry}{明码}{8,8}{⽇,⽯}
  \begin{Phonetics}{明码}{ming2ma3}
    \definition{s.}{código simples, em claro | preço claramente marcado}
  \antonymref{密码}{mi4ma3}
  \end{Phonetics}
\end{Entry}

\begin{Entry}{明亮}{8,9}{⽇,⼇}
  \begin{Phonetics}{明亮}{ming2liang4}[][HSK 5]
    \definition{adj.}{claro; bem iluminado | brilhante; resplandecente | claro; simples; compreensível}
  \end{Phonetics}
\end{Entry}

\begin{Entry}{明星}{8,9}{⽇,⽇}
  \begin{Phonetics}{明星}{ming2xing1}[][HSK 2]
    \definition[个,位,颗,名]{s.}{estrela; ator, atleta, cantor famosos, etc. | talento de ponta; profissional de destaque; também é usado como metáfora para pessoas ou grupos que se destacam pelo seu bom desempenho ou excelência | estrela brilhante; estrela resplandecente; referindo"-se a estrelas muito brilhantes}
  \end{Phonetics}
\end{Entry}

\begin{Entry}{明显}{8,9}{⽇,⽇}
  \begin{Phonetics}{明显}{ming2xian3}[][HSK 3]
    \definition{adj.}{claro; óbvio; distinto; claramente visível}
  \end{Phonetics}
\end{Entry}

\begin{Entry}{明朗}{8,10}{⽇,⽉}
  \begin{Phonetics}{明朗}{ming2lang3}[][HSK 7-9]
    \definition{adj.}{brilhante e claro; bem iluminado (geralmente se referindo a ambientes externos) | claro; óbvio; inequívoco | franco; direto; alegre e otimista; íntegro e honesto; (pensamentos, mentalidade, caráter, etc.) otimista, alegre e não melancólico ou deprimido}
  \end{Phonetics}
\end{Entry}

\begin{Entry}{明珠}{8,10}{⽇,⽟}
  \begin{Phonetics}{明珠}{ming2zhu1}
    \definition{s.}{pérola | jóia (de grande valor)}
  \end{Phonetics}
\end{Entry}

\begin{Entry}{明媚}{8,12}{⽇,⼥}
  \begin{Phonetics}{明媚}{ming2mei4}[][HSK 7-9]
    \definition{adj.}{brilhante e belo; radiante e encantador}
  \end{Phonetics}
\end{Entry}

\begin{Entry}{明智}{8,12}{⽇,⽇}
  \begin{Phonetics}{明智}{ming2zhi4}[][HSK 7-9]
    \definition{adj.}{sábio; sensato; sagaz; previdente; ponderado}
  \end{Phonetics}
\end{Entry}

\begin{Entry}{明确}{8,12}{⽇,⽯}
  \begin{Phonetics}{明确}{ming2que4}[][HSK 3]
    \definition{adj.}{claro; definido; específico}
    \definition{v.}{deixar claro; tornar definitivo; tornar um ponto de vista, uma tarefa, etc. claro, compreensível e definitivo}
  \end{Phonetics}
\end{Entry}

%%%%%%%%%% 昏 %%%%%%%%%%
\subsection*{昏}\addcontentsline{loh}{figure}{昏}

\begin{Entry}{昏}{8}{⽇}
  \begin{Phonetics}{昏}{hun1}[][HSK 6]
    \definition*{s.}{Sobrenome: Hun}
    \definition{adj.}{escuro; fraco; embaçado | confuso; embaraçado; inconsciente}
    \definition{s.}{crepúsculo; tarde}
    \definition{v.}{perder a consciência; desmaiar}
  \end{Phonetics}
\end{Entry}

\begin{Entry}{昏迷}{8,9}{⽇,⾡}
  \begin{Phonetics}{昏迷}{hun1mi2}[][HSK 7-9]
    \definition{s.}{coma; um estado em que uma pessoa perde a sensibilidade e o conhecimento}
    \definition{v.}{entrar em coma}
  \end{Phonetics}
\end{Entry}

%%%%%%%%%% 易 %%%%%%%%%%
\subsection*{易}\addcontentsline{loh}{figure}{易}

\begin{Entry}{易}{8}{⽇}
  \begin{Phonetics}{易}{yi4}
    \definition*{s.}{Sobrenome: Yi}
    \definition{adj.}{fácil | amigável; pacífico}
    \definition{v.}{modificar; transformar | trocar | subestimar; desprezar}
  \end{Phonetics}
\end{Entry}

%%%%%%%%%% 昔 %%%%%%%%%%
\subsection*{昔}\addcontentsline{loh}{figure}{昔}

\begin{Entry}{昔}{8}{⽇}
  \begin{Phonetics}{昔}{xi1}
    \definition{s.}{tempos antigos; o passado; era uma vez}
  \end{Phonetics}
\end{Entry}

\begin{Entry}{昔日}{8,4}{⽇,⽇}
  \begin{Phonetics}{昔日}{xi1ri4}[][HSK 7-9]
    \definition{s.}{o passado; no passado; tempos passados; (em) dias (ou tempos) antigos; dias antes}
  \synonymref{从前}{cong2qian2}
  \synonymref{往日}{wang3ri4}
  \end{Phonetics}
\end{Entry}

%%%%%%%%%% 朋 %%%%%%%%%%
\subsection*{朋}\addcontentsline{loh}{figure}{朋}

\begin{Entry}{朋}{8}{⽉}
  \begin{Phonetics}{朋}{peng2}
    \definition*{s.}{Sobrenome: Peng}
    \definition{s.}{amigo}
    \definition{v.}{Literário: rivalizar; igualar; comparar | Literário: reunir"-se em grupo; juntar"-se em grupo}
  \end{Phonetics}
\end{Entry}

\begin{Entry}{朋友}{8,4}{⽉,⼜}
  \begin{Phonetics}{朋友}{peng2you5}[][HSK 1]
    \definition[个,位,帮,群]{s.}{amigo; pessoas que têm um bom relacionamento, uma boa relação, se entendem e se ajudam mutuamente | namorado; namorada}
  \end{Phonetics}
\end{Entry}

%%%%%%%%%% 服 %%%%%%%%%%
\subsection*{服}\addcontentsline{loh}{figure}{服}

\begin{Entry}{服}{8}{⽉}
  \begin{Phonetics}{服}{fu2}[][HSK 6]
    \definition*{s.}{Sobrenome: Fu}
    \definition{s.}{roupas | vestuário de luto; refere"-se a roupas de luto}
    \definition{v.}{vestir (roupas) | tomar (remédio) | envolver"-se em; servir | obedecer; ser convencido | convencer; persuadir | adaptar"-se; acostumar"-se a}
  \end{Phonetics}
  \begin{Phonetics}{服}{fu4}
    \definition{clas.}{usado para remédio: dose; usado na medicina tradicional chinesa}
  \end{Phonetics}
\end{Entry}

\begin{Entry}{服从}{8,4}{⽉,⼈}
  \begin{Phonetics}{服从}{fu2cong2}[][HSK 5]
    \definition{v.}{obedecer; submeter"-se a; estar subordinado a}
  \end{Phonetics}
\end{Entry}

\begin{Entry}{服务}{8,5}{⽉,⼒}
  \begin{Phonetics}{服务}{fu2wu4}[][HSK 2]
    \definition{v.}{prestar serviço a; estar a serviço de; servir; trabalhar para o benefício coletivo (ou de outras pessoas) ou para uma causa específica | trabalhar; servir}
  \end{Phonetics}
\end{Entry}

\begin{Entry}{服务员}{8,5,7}{⽉,⼒,⼝}
  \begin{Phonetics}{服务员}{fu2wu4yuan2}
    \definition{s.}{atendente | garçom | garçonete | pessoal de atendimento ao cliente}
  \end{Phonetics}
\end{Entry}

\begin{Entry}{服务器}{8,5,16}{⽉,⼒,⼝}
  \begin{Phonetics}{服务器}{fu2wu4qi4}[][HSK 7-9]
    \definition[个,台]{s.}{Computção: servidor; um dispositivo dedicado que fornece serviços aos usuários em uma rede eletrônica de computadores}
  \end{Phonetics}
\end{Entry}

\begin{Entry}{服用}{8,5}{⽉,⽤}
  \begin{Phonetics}{服用}{fu2yong4}[][HSK 7-9]
    \definition{v.}{tomar (remédio)}[他已开始服用这种药。===Ele começou a tomar o remédio.]
  \end{Phonetics}
\end{Entry}

\begin{Entry}{服饰}{8,8}{⽉,⾷}
  \begin{Phonetics}{服饰}{fu2shi4}[][HSK 7-9]
    \definition[套]{s.}{roupa; vestido; traje; vestimenta e adorno pessoal}
  \end{Phonetics}
\end{Entry}

\begin{Entry}{服装}{8,12}{⽉,⾐}
  \begin{Phonetics}{服装}{fu2zhuang1}[][HSK 3]
    \definition[套,件,身]{s.}{roupas; vestuário; trajes; termo genérico para roupas, sapatos e chapéus, geralmente referido especificamente a roupas}
  \end{Phonetics}
\end{Entry}

%%%%%%%%%% 杯 %%%%%%%%%%
\subsection*{杯}\addcontentsline{loh}{figure}{杯}

\begin{Entry}{杯}{8}{⽊}
  \begin{Phonetics}{杯}{bei1}[][HSK 1]
    \definition{clas.}{para certos recipientes de líquidos: copo, xícara, etc.}
    \definition[只,个]{s.}{copo; caneca; xícara | taça; troféu; prêmio em forma de taça}
  \end{Phonetics}
\end{Entry}

\begin{Entry}{杯子}{8,3}{⽊,⼦}
  \begin{Phonetics}{杯子}{bei1zi5}[][HSK 1]
    \definition[个,只,种]{s.}{xícara; copo; recipiente para bebidas ou outros líquidos, geralmente cilíndrico ou com a parte inferior ligeiramente mais estreita, com capacidade geralmente pequena}
  \end{Phonetics}
\end{Entry}

\begin{Entry}{杯具}{8,8}{⽊,⼋}
  \begin{Phonetics}{杯具}{bei1ju4}
    \definition{s.}{parachoque | fiasco | (gíria) tragédia}
  \end{Phonetics}
\end{Entry}

%%%%%%%%%% 杰 %%%%%%%%%%
\subsection*{杰}\addcontentsline{loh}{figure}{杰}

\begin{Entry}{杰}{8}{⽊}
  \begin{Phonetics}{杰}{jie2}
    \definition{adj.}{notável; proeminente; fora do comum}
    \definition[位,名,个,些]{s.}{pessoa excepcional; herói; uma pessoa com talentos excepcionais}
  \end{Phonetics}
\end{Entry}

\begin{Entry}{杰出}{8,5}{⽊,⼐}
  \begin{Phonetics}{杰出}{jie2chu1}[][HSK 6]
    \definition{adj.}{notável; proeminente; (talento, realização) excepcional}
  \end{Phonetics}
\end{Entry}

%%%%%%%%%% 松 %%%%%%%%%%
\subsection*{松}\addcontentsline{loh}{figure}{松}

\begin{Entry}{松}{8}{⽊}
  \begin{Phonetics}{松}{song1}[][HSK 4]
    \definition*{s.}{Sobrenome: Song}
    \definition{adj.}{solto; frouxo; folgado | leve e crocante; macio | relaxado; confortável}
    \definition[棵]{s.}{pinheiro | fio de carne seca; carne moída seca; alimentos macios ou quebradiços}
    \definition{v.}{afrouxar; relaxar; abrandar | desamarrar; desatar; liberar}
  \end{Phonetics}
\end{Entry}

\begin{Entry}{松木}{8,4}{⽊,⽊}
  \begin{Phonetics}{松木}{song1mu4}
    \definition{s.}{pinheiro}
  \end{Phonetics}
\end{Entry}

\begin{Entry}{松弛}{8,6}{⽊,⼸}
  \begin{Phonetics}{松弛}{song1chi2}[][HSK 7-9]
    \definition{adj.}{frouxo; solto; relaxado; não tenso | frouxo; (regras, regulamentos, etc.) não são rigorosamente aplicados}
  \synonymref{缓和}{huan3he2}
  \synonymref{宽容}{kuan1rong2}
  \synonymref{宽松}{kuan1song1}
  \synonymref{马虎}{ma3hu5}
  \synonymref{轻松}{qing1song1}
  \synonymref{疏忽}{shu1hu5}
  \synonymref{随便}{sui2/bian4}
  \antonymref{坚实}{jian1shi2}
  \antonymref{紧急}{jin3ji2}
  \antonymref{紧张}{jin3zhang1}
  \antonymref{牢牢}{lao2lao2}
  \antonymref{严格}{yan2ge2}
  \end{Phonetics}
\end{Entry}

\begin{Entry}{松树}{8,9}{⽊,⽊}
  \begin{Phonetics}{松树}{song1shu4}[][HSK 4]
    \definition[棵]{s.}{pinheiro; conífera comum, geralmente com folhas longas e pontiagudas e cones lenhosos}
  \end{Phonetics}
\end{Entry}

\begin{Entry}{松绑}{8,9}{⽊,⽷}
  \begin{Phonetics}{松绑}{song1/bang3}[][HSK 7-9]
    \definition{v.+compl.}{desatar; desamarrar uma pessoa | Figurativo: relaxar as restrições; flexibilizar as restrições; desatar; libertar}
  \end{Phonetics}
\end{Entry}

%%%%%%%%%% 板 %%%%%%%%%%
\subsection*{板}\addcontentsline{loh}{figure}{板}

\begin{Entry}{板}{8}{⽊}
  \begin{Phonetics}{板}{ban3}[][HSK 3]
    \definition{adj.}{rígido; não natural; inflexível}
    \definition[块,个]{s.}{tábua; placa; prato; objeto rígido em forma de placa | veneziana; persiana; refere"-se especificamente aos painéis de portas de lojas | badalos (instrumento musical que marca o ritmo) | uma batida acentuada (ritmo) na música e na ópera tradicional | chefe}
    \definition{v.}{parecer sério | corrigir maus hábitos ou defeitos | ser rígido como uma tábua}
  \end{Phonetics}
\end{Entry}

\begin{Entry}{板块}{8,7}{⽊,⼟}
  \begin{Phonetics}{板块}{ban3kuai4}[][HSK 7-9]
    \definition[个]{s.}{placa tectônica; segmentos móveis da crosta terrestre | seção; uma metáfora para uma combinação de partes que têm algo em comum ou conectado}
  \end{Phonetics}
\end{Entry}

%%%%%%%%%% 构 %%%%%%%%%%
\subsection*{构}\addcontentsline{loh}{figure}{构}

\begin{Entry}{构}{8}{⽊}
  \begin{Phonetics}{构}{gou4}
    \definition{s.}{composição literária}
    \definition{v.}{construir; formar; compor | fabricar; inventar | construir; erguer uma casa}
    \variantof{够}
  \end{Phonetics}
\end{Entry}

\begin{Entry}{构成}{8,6}{⽊,⼽}
  \begin{Phonetics}{构成}{gou4/cheng2}[][HSK 4]
    \definition{s.}{parte; componente; composição; estrutura}
    \definition{v.+compl.}{formar; compor; constituir; compor; encaixar muitas partes para formar um todo | consistir; causar; formar (principalmente em termos jurídicos)}
  \end{Phonetics}
\end{Entry}

\begin{Entry}{构建}{8,8}{⽊,⼵}
  \begin{Phonetics}{构建}{gou4jian4}[][HSK 6]
    \definition{v.}{estabelecer (usado principalmente para coisas abstratas); montar; instalar}
  \end{Phonetics}
\end{Entry}

\begin{Entry}{构思}{8,9}{⽊,⼼}
  \begin{Phonetics}{构思}{gou4si1}[][HSK 7-9]
    \definition{s.}{concepção (ideia); o resultado da concepção}
    \definition{v.}{elaborar o enredo de uma obra literária ou a composição de uma pintura; pensar bem antes de escrever artigos ou criar obras literárias}
  \end{Phonetics}
\end{Entry}

\begin{Entry}{构造}{8,10}{⽊,⾡}
  \begin{Phonetics}{构造}{gou4zao4}[][HSK 4]
    \definition[种]{s.}{estrutura; construção; disposição, organização e inter"-relação dos componentes}
    \definition{v.}{formar; construir}
  \end{Phonetics}
\end{Entry}

\begin{Entry}{构想}{8,13}{⽊,⼼}
  \begin{Phonetics}{构想}{gou4xiang3}[][HSK 7-9]
    \definition{s.}{ideia; concepção; ideias formadas}
    \definition[种,个]{v.}{pensar (em um plano, projeto, etc.); conceber; usar a mente ao escrever ou criar arte}
  \end{Phonetics}
\end{Entry}

%%%%%%%%%% 枕 %%%%%%%%%%
\subsection*{枕}\addcontentsline{loh}{figure}{枕}

\begin{Entry}{枕}{8}{⽊}
  \begin{Phonetics}{枕}{zhen3}
    \definition*{s.}{Sobrenome: Zhen}
    \definition[个]{s.}{travesseiro; almofada | Mecânica: bloco}
    \definition{v.}{descansar a cabeça no travesseiro, almofada}
  \end{Phonetics}
\end{Entry}

%%%%%%%%%% 枚 %%%%%%%%%%
\subsection*{枚}\addcontentsline{loh}{figure}{枚}

\begin{Entry}{枚}{8}{⽊}
  \begin{Phonetics}{枚}{mei2}[][HSK 7-9]
    \definition*{s.}{Sobrenome: Mei}
    \definition{clas.}{utilizado para objetos pequenos | utilizado para moedas, anéis, distintivos, pérolas, medalhas esportivas, foguetes, satélites etc.}
    \definition{s.}{Arcaico: tronco de árvore (significado original) | Arcaico: chicote | Arcaico: pino de madeira, usado como mordaça para soldados em marcha}
  \end{Phonetics}
\end{Entry}

%%%%%%%%%% 果 %%%%%%%%%%
\subsection*{果}\addcontentsline{loh}{figure}{果}

\begin{Entry}{果}{8}{⽊}
  \begin{Phonetics}{果}{guo3}
    \definition*{s.}{Sobrenome: Guo}
    \definition{adj.}{resoluto; determinado; sem exitação}
    \definition{adv.}{realmente; como esperado; com certeza; isso significa que as coisas são consistentes com as expectativas, equivalente a 果然}
    \definition{conj.}{se realmente; se de fato}
    \definition[个,些,种]{s.}{fruta; fruto da planta | resultado; consequência; o resultado final de um assunto}
  \seealsoref{果然}{guo3ran2}
  \antonymref{因}{yin1}
  \end{Phonetics}
\end{Entry}

\begin{Entry}{果子}{8,3}{⽊,⼦}
  \begin{Phonetics}{果子}{guo3zi5}
    \definition{s.}{fruta}
  \end{Phonetics}
\end{Entry}

\begin{Entry}{果汁}{8,5}{⽊,⽔}
  \begin{Phonetics}{果汁}{guo3zhi1}[][HSK 3]
    \definition[杯,瓶,种]{s.}{suco; suco de frutas frescas; também se refere a bebidas feitas com suco de frutas frescas}
  \end{Phonetics}
\end{Entry}

\begin{Entry}{果园}{8,7}{⽊,⼞}
  \begin{Phonetics}{果园}{guo3yuan2}[][HSK 7-9]
    \definition[个,座]{s.}{pomar; um jardim onde são plantadas árvores frutíferas}
  \end{Phonetics}
\end{Entry}

\begin{Entry}{果实}{8,8}{⽊,⼧}
  \begin{Phonetics}{果实}{guo3shi2}[][HSK 4]
    \definition[种]{s.}{fruta; o órgão que se desenvolve a partir do ovário ou com outras partes da flor após a fertilização da flor | ganhos; frutos;  uma metáfora para conquista ou recompensa por trabalho árduo}
  \end{Phonetics}
\end{Entry}

\begin{Entry}{果树}{8,9}{⽊,⽊}
  \begin{Phonetics}{果树}{guo3shu4}[][HSK 6]
    \definition[棵,个,片]{s.}{árvore frutífera; árvores cujos frutos são principalmente comestíveis, como pessegueiros e macieiras}
  \end{Phonetics}
\end{Entry}

\begin{Entry}{果真}{8,10}{⽊,⼗}
  \begin{Phonetics}{果真}{guo3zhen1}[][HSK 7-9]
    \definition{adv.}{realmente; como esperado; com certeza}
    \definition{conj.}{se de fato; se realmente; se for o caso}[果真如此, 我就放心了。===Se for esse o caso, então ficarei aliviado.]
  \end{Phonetics}
\end{Entry}

\begin{Entry}{果断}{8,11}{⽊,⽄}
  \begin{Phonetics}{果断}{guo3duan4}[][HSK 7-9]
    \definition{adj.}{resoluto; decisivo; agir decisivamente sem hesitação}
  \end{Phonetics}
\end{Entry}

\begin{Entry}{果然}{8,12}{⽊,⽕}
  \begin{Phonetics}{果然}{guo3ran2}[][HSK 3]
    \definition{adv.}{realmente; como esperado; com certeza; indica que os fatos correspondem ao que foi dito ou esperado}
    \definition{conj.}{se realmente; se de fato; suponha que os fatos correspondam ao que foi dito ou esperado}
  \end{Phonetics}
\end{Entry}

\begin{Entry}{果酱}{8,13}{⽊,⾣}
  \begin{Phonetics}{果酱}{guo3jiang4}[][HSK 6]
    \definition{s.}{geléia | compota ou doce (de frutas); fruta em conserva}
  \end{Phonetics}
\end{Entry}

%%%%%%%%%% 枝 %%%%%%%%%%
\subsection*{枝}\addcontentsline{loh}{figure}{枝}

\begin{Entry}{枝}{8}{⽊}
  \begin{Phonetics}{枝}{zhi1}[][HSK 6]
    \definition*{s.}{Sobrenome: Zhi}
    \definition{clas.}{usado para flores com galhos, ramos | usado para objetos em forma de haste}
    \definition{s.}{ramo; galho}
  \end{Phonetics}
\end{Entry}

%%%%%%%%%% 枢 %%%%%%%%%%
\subsection*{枢}\addcontentsline{loh}{figure}{枢}

\begin{Entry}{枢}{8}{⽊}
  \begin{Phonetics}{枢}{shu1}
    \definition[户]{s.}{dobradiça de porta | pivô; eixo; centro; parte importante ou central}
  \end{Phonetics}
\end{Entry}

\begin{Entry}{枢纽}{8,7}{⽊,⽷}
  \begin{Phonetics}{枢纽}{shu1niu3}[][HSK 7-9]
    \definition{s.}{eixo; pivô; posição chave; a chave para tudo, o elo central na interconexão das coisas}
  \synonymref{关键}{guan1jian4}
  \synonymref{关节}{guan1jie2}
  \synonymref{环节}{huan2jie2}
  \synonymref{纽带}{niu3dai4}
  \end{Phonetics}
\end{Entry}

%%%%%%%%%% 枪 %%%%%%%%%%
\subsection*{枪}\addcontentsline{loh}{figure}{枪}

\begin{Entry}{枪}{8}{⽊}
  \begin{Phonetics}{枪}{qiang1}[][HSK 5]
    \definition*{s.}{Sobrenome: Qiang}
    \definition[把,杆,支,挺]{s.}{lança | arma; rifle; arma de fogo | uma coisa em forma de arma | enxada; ferramenta para cavar a terra}
    \definition{v.}{escrever artigos ou responder perguntas para outras pessoas}
  \end{Phonetics}
\end{Entry}

\begin{Entry}{枪毙}{8,10}{⽊,⽐}
  \begin{Phonetics}{枪毙}{qiang1bi4}[][HSK 7-9]
    \definition{v.}{executar por disparo; matar | Humor figurativo: rejeitar; vetar; recusar (uma proposta, manuscrito, etc.) | matar a tiros}
  \end{Phonetics}
\end{Entry}

%%%%%%%%%% 枫 %%%%%%%%%%
\subsection*{枫}\addcontentsline{loh}{figure}{枫}

\begin{Entry}{枫}{8}{⽊}
  \begin{Phonetics}{枫}{feng1}
    \definition[棵]{s.}{goma doce chinesa | bordo; \emph{maple}}
  \end{Phonetics}
\end{Entry}

\begin{Entry}{枫叶}{8,5}{⽊,⼝}
  \begin{Phonetics}{枫叶}{feng1ye4}
    \definition{s.}{folha de bordo (maple, tipo de árvore)}
  \end{Phonetics}
\end{Entry}

%%%%%%%%%% 柜 %%%%%%%%%%
\subsection*{柜}\addcontentsline{loh}{figure}{柜}

\begin{Entry}{柜}{8}{⽊}
  \begin{Phonetics}{柜}{gui4}
    \definition{s.}{baú; armário; gabinete | loja; balcão}
  \end{Phonetics}
  \begin{Phonetics}{柜}{ju3}
    \definition{s.}{faia; salgueiro}
  \end{Phonetics}
\end{Entry}

\begin{Entry}{柜子}{8,3}{⽊,⼦}
  \begin{Phonetics}{柜子}{gui4zi5}[][HSK 5]
    \definition[个]{s.}{gabinete; armário; dispositivo para guardar roupas, documentos, livros, etc.}
  \end{Phonetics}
\end{Entry}

\begin{Entry}{柜台}{8,5}{⽊,⼝}
  \begin{Phonetics}{柜台}{gui4tai2}[][HSK 7-9]
    \definition[个,排,组]{s.}{bar; balcão; uma longa área semelhante a uma mesa em uma loja ou banco usada para vender mercadorias ou conduzir negócios}
  \end{Phonetics}
\end{Entry}

%%%%%%%%%% 欣 %%%%%%%%%%
\subsection*{欣}\addcontentsline{loh}{figure}{欣}

\begin{Entry}{欣}{8}{⽋}
  \begin{Phonetics}{欣}{xin1}
    \definition*{s.}{Sobrenome: Xin}
    \definition{adj.}{alegre; feliz; contente}
  \end{Phonetics}
\end{Entry}

\begin{Entry}{欣赏}{8,12}{⽋,⾙}
  \begin{Phonetics}{欣赏}{xin1shang3}[][HSK 5]
    \definition{v.}{apreciar; admirar; valorizar; apreciar as coisas boas e descubrir o prazer que elas proporcionam | apreciar; gostar; considerar bom}
  \end{Phonetics}
\end{Entry}

%%%%%%%%%% 欧 %%%%%%%%%%
\subsection*{欧}\addcontentsline{loh}{figure}{欧}

\begin{Entry}{欧}{8}{⽋}
  \begin{Phonetics}{欧}{ou1}
    \definition*{s.}{Europa, abreviação de 欧洲 | Sobrenome: Ou}
  \seealsoref{欧洲}{ou1zhou1}
  \end{Phonetics}
\end{Entry}

\begin{Entry}{欧阳询}{8,6,8}{⽋,⾩,⾔}
  \begin{Phonetics}{欧阳询}{ou1yang2 xun2}
    \definition*{s.}{Ouyang Xun (557--641), um dos quatro grandes calígrafos do início da dinastia Tang (唐初四大家)}
  \seealsoref{唐初四大家}{tang2 chu1 si4 da4jia1}
  \end{Phonetics}
\end{Entry}

\begin{Entry}{欧洲}{8,9}{⽋,⽔}
  \begin{Phonetics}{欧洲}{ou1zhou1}
    \definition*{s.}{Europa}
  \end{Phonetics}
\end{Entry}

\begin{Entry}{欧洲人}{8,9,2}{⽋,⽔,⼈}
  \begin{Phonetics}{欧洲人}{ou1zhou1ren2}
    \definition{s.}{europeu | pessoa ou povo da Europa}
  \end{Phonetics}
\end{Entry}

\begin{Entry}{欧洲共同体}{8,9,6,6,7}{⽋,⽔,⼋,⼝,⼈}
  \begin{Phonetics}{欧洲共同体}{ou1zhou1 gong4tong2ti3}
    \definition*{s.}{Comunidade Europeia}
  \end{Phonetics}
\end{Entry}

\begin{Entry}{欧盟}{8,13}{⽋,⽫}
  \begin{Phonetics}{欧盟}{ou1meng2}
    \definition*{s.}{União Europeia (EU)}
  \end{Phonetics}
\end{Entry}

%%%%%%%%%% 武 %%%%%%%%%%
\subsection*{武}\addcontentsline{loh}{figure}{武}

\begin{Entry}{武}{8}{⽌}
  \begin{Phonetics}{武}{wu3}
    \definition*{s.}{Sobrenome: Wu}
    \definition{adj.}{valente; corajoso}
    \definition{s.}{militar; atividades e comportamentos relacionados a habilidades militares e de combate | arte marcial | passo; meio passo; pegadas}
  \antonymref{文}{wen2}
  \end{Phonetics}
\end{Entry}

\begin{Entry}{武力}{8,2}{⽌,⼒}
  \begin{Phonetics}{武力}{wu3li4}[][HSK 7-9]
    \definition{s.}{força | força militar; poderio armado; poderio bélico; força das armas | força armada (poder)}[霸道政策依靠武力。===Políticas autoritárias dependem da força.]
  \synonymref{暴力}{bao4li4}
  \end{Phonetics}
\end{Entry}

\begin{Entry}{武士}{8,3}{⽌,⼠}
  \begin{Phonetics}{武士}{wu3shi4}
    \definition{s.}{samurai | guerreiro}
  \end{Phonetics}
\end{Entry}

\begin{Entry}{武大戏}{8,3,6}{⽌,⼤,⼽}
  \begin{Phonetics}{武大戏}{wu3 da4xi4}
    \definition*{s.}{Drama de Luta Acrobática | Drama Wu}
  \end{Phonetics}
\end{Entry}

\begin{Entry}{武艺}{8,4}{⽌,⾋}
  \begin{Phonetics}{武艺}{wu3yi4}
    \definition{s.}{arte marcial | habilidade militar}
  \end{Phonetics}
\end{Entry}

\begin{Entry}{武术}{8,5}{⽌,⽊}
  \begin{Phonetics}{武术}{wu3shu4}[][HSK 3]
    \definition[种,套,门]{s.}{arte marcial; autodefesa; \emph{wushu}; um esporte tradicional chinês que utiliza técnicas com os punhos, pernas, pés ou armas como facas e espadas}
  \end{Phonetics}
\end{Entry}

\begin{Entry}{武官}{8,8}{⽌,⼧}
  \begin{Phonetics}{武官}{wu3guan1}
    \definition{s.}{oficial militar | adido militar}
  \end{Phonetics}
\end{Entry}

\begin{Entry}{武断}{8,11}{⽌,⽄}
  \begin{Phonetics}{武断}{wu3duan4}
    \definition{adj.}{arbitrário | dogmático | subjetivo}
  \end{Phonetics}
\end{Entry}

\begin{Entry}{武装}{8,12}{⽌,⾐}
  \begin{Phonetics}{武装}{wu3zhuang1}[][HSK 7-9]
    \definition{s.}{armas; equipamento militar; uniforme de combate | forças armadas}
    \definition{v.}{armar; equipar com armas; fornecer armas}
  \end{Phonetics}
\end{Entry}

\begin{Entry}{武器}{8,16}{⽌,⼝}
  \begin{Phonetics}{武器}{wu3qi4}[][HSK 3]
    \definition[批,个,件,种]{s.}{arma; equipamentos e dispositivos utilizados diretamente para matar inimigos ou destruir suas instalações defensivas e ofensivas | armas; armamento; metáfora usada como ferramenta de luta}
  \end{Phonetics}
\end{Entry}

%%%%%%%%%% 歧 %%%%%%%%%%
\subsection*{歧}\addcontentsline{loh}{figure}{歧}

\begin{Entry}{歧}{8}{⽌}
  \begin{Phonetics}{歧}{qi2}
    \definition{adj.}{divergente; diferente; ambíguo; inconsistente}
    \definition{s.}{bifurcação; ramificação; bifurcação da estrada; uma estrada que se ramifica de uma estrada principal}
  \end{Phonetics}
\end{Entry}

\begin{Entry}{歧视}{8,8}{⽌,⾒}
  \begin{Phonetics}{歧视}{qi2shi4}[][HSK 7-9]
    \definition[种,些,点]{s.}{discriminação}
    \definition{v.}{discriminar contra; discriminar; tratar alguém ou um grupo de forma desigual, com uma atitude injusta ou desproporcional}
  \end{Phonetics}
\end{Entry}

%%%%%%%%%% 殴 %%%%%%%%%%
\subsection*{殴}\addcontentsline{loh}{figure}{殴}

\begin{Entry}{殴}{8}{⽎}
  \begin{Phonetics}{殴}{ou1}
    \definition{v.}{espancar; bater; acertar}
  \end{Phonetics}
\end{Entry}

\begin{Entry}{殴打}{8,5}{⽎,⼿}
  \begin{Phonetics}{殴打}{ou1da3}[][HSK 7-9]
    \definition{v.}{espancar; bater; bater em alguém}
  \end{Phonetics}
\end{Entry}

%%%%%%%%%% 氛 %%%%%%%%%%
\subsection*{氛}\addcontentsline{loh}{figure}{氛}

\begin{Entry}{氛}{8}{⽓}
  \begin{Phonetics}{氛}{fen1}
    \definition{s.}{atmosfera; gás}
  \end{Phonetics}
\end{Entry}

\begin{Entry}{氛围}{8,7}{⽓,⼞}
  \begin{Phonetics}{氛围}{fen1wei2}[][HSK 7-9]
    \definition[种,片,股]{s.}{atmosfera; a atmosfera e o humor ao redor}
  \end{Phonetics}
\end{Entry}

%%%%%%%%%% 沮 %%%%%%%%%%
\subsection*{沮}\addcontentsline{loh}{figure}{沮}

\begin{Entry}{沮}{8}{⽔}
  \begin{Phonetics}{沮}{ju3}
    \definition{v.}{Literário: parar; prevenir; evitar | ficar melancólico; ficar triste}
  \end{Phonetics}
  \begin{Phonetics}{沮}{ju4}
    \definition{s.}{lama e folhas em decomposição}
  \end{Phonetics}
\end{Entry}

\begin{Entry}{沮丧}{8,8}{⽔,⼗}
  \begin{Phonetics}{沮丧}{ju3sang4}[][HSK 7-9]
    \definition{adj.}{desanimado; abatido; decepcionado}
  \end{Phonetics}
\end{Entry}

%%%%%%%%%% 河 %%%%%%%%%%
\subsection*{河}\addcontentsline{loh}{figure}{河}

\begin{Entry}{河}{8}{⽔}
  \begin{Phonetics}{河}{he2}[][HSK 2]
    \definition*{s.}{Astronomia: o sistema da Via Láctea | O Rio Amarelo; O Rio Huanghe | Sobrenome: He}
    \definition[条,道]{s.}{rio; refere"-se a grandes cursos de água}
  \end{Phonetics}
\end{Entry}

\begin{Entry}{河南}{8,9}{⽔,⼗}
  \begin{Phonetics}{河南}{he2nan2}
    \definition*{s.}{Província de Henan}
  \end{Phonetics}
\end{Entry}

\begin{Entry}{河流}{8,10}{⽔,⽔}
  \begin{Phonetics}{河流}{he2liu2}[][HSK 7-9]
    \definition[条]{s.}{rio; córrego; um termo geral para grandes fluxos naturais de água (como rios, etc.) na superfície da Terra}
  \end{Phonetics}
\end{Entry}

\begin{Entry}{河畔}{8,10}{⽔,⽥}
  \begin{Phonetics}{河畔}{he2pan4}[][HSK 7-9]
    \definition{s.}{planície fluvial | beira"-rio}
  \end{Phonetics}
\end{Entry}

\begin{Entry}{河蚌}{8,10}{⽔,⾍}
  \begin{Phonetics}{河蚌}{he2bang4}
    \definition{s.}{mexilhões | bivalves cultivados em rios e lagos}
  \end{Phonetics}
\end{Entry}

%%%%%%%%%% 沸 %%%%%%%%%%
\subsection*{沸}\addcontentsline{loh}{figure}{沸}

\begin{Entry}{沸}{8}{⽔}
  \begin{Phonetics}{沸}{fei4}
    \definition{adj.}{fervente; borbulhante; em ebulição}
    \definition{v.}{ferver | borbulhar}
  \end{Phonetics}
\end{Entry}

\begin{Entry}{沸沸扬扬}{8,8,6,6}{⽔,⽔,⼿,⼿}
  \begin{Phonetics}{沸沸扬扬}{fei4fei4yang2yang2}[][HSK 7-9]
    \definition{expr.}{levantar uma confusão de críticas sobre; borbulhando de barulho; discutir animadamente; dar origem a muita discussão; em um rebuliço; ``Tão barulhento quanto água fervente.'', frequentemente usado para descrever muita discussão}
  \end{Phonetics}
\end{Entry}

\begin{Entry}{沸腾}{8,13}{⽔,⾁}
  \begin{Phonetics}{沸腾}{fei4teng2}[][HSK 7-9]
    \definition{v.}{ferver; vaporizar | fervilhar de excitação; uma metáfora para alto astral ou vozes barulhentas}
  \end{Phonetics}
\end{Entry}

%%%%%%%%%% 油 %%%%%%%%%%
\subsection*{油}\addcontentsline{loh}{figure}{油}

\begin{Entry}{油}{8}{⽔}
  \begin{Phonetics}{油}{you2}[][HSK 2]
    \definition*{s.}{Sobrenome: You}
    \definition{adj.}{oleoso; gorduroso}
    \definition[瓶,滴,层]{s.}{óleo; gordura; graxa; petróleo}
    \definition{v.}{aplicar óleo de tungue, verniz ou tinta | estar manchado ou sujo com óleo ou graxa | aplicar óleo de tungue ou tinta}
  \end{Phonetics}
\end{Entry}

%%%%%%%%%% 治 %%%%%%%%%%
\subsection*{治}\addcontentsline{loh}{figure}{治}

\begin{Entry}{治}{8}{⽔}
  \begin{Phonetics}{治}{zhi4}[][HSK 4]
    \definition*{s.}{Sobrenome: Zhi}
    \definition{adj.}{calmo e pacífico}
    \definition{s.}{sede de um antigo governo local}
    \definition{v.}{reger; administrar; governar; gerenciar; gerir | tratar (uma doença); curar; sarar | eliminar; controlar pragas | controlar (um rio); restaurar um curso d'água por meio de dragagem | punir; castigar | estudar; pesquisar; explorar}
  \end{Phonetics}
\end{Entry}

\begin{Entry}{治安}{8,6}{⽔,⼧}
  \begin{Phonetics}{治安}{zhi4'an1}[][HSK 5]
    \definition{s.}{ordem pública; segurança pública; ordem social estável}
  \end{Phonetics}
\end{Entry}

\begin{Entry}{治疗}{8,7}{⽔,⽧}
  \begin{Phonetics}{治疗}{zhi4liao2}[][HSK 4]
    \definition{s.}{diagnóstico; tratamento}
    \definition{v.}{tratar; curar; remediar; eliminar doenças por meio de medicamentos, cirurgia, etc.}
  \end{Phonetics}
\end{Entry}

\begin{Entry}{治病}{8,10}{⽔,⽧}
  \begin{Phonetics}{治病}{zhi4bing4}[][HSK 6]
    \definition{v.}{tratar uma doença; eliminar doenças por meio de medicamentos, cirurgias, etc.}
  \end{Phonetics}
\end{Entry}

\begin{Entry}{治理}{8,11}{⽔,⽟}
  \begin{Phonetics}{治理}{zhi4li3}[][HSK 5]
    \definition{s.}{governo | governança}
    \definition{v.}{dirigir; gerenciar; governar; administrar | tratar; aproveitar; colocar sob controle; colocar em ordem}
  \end{Phonetics}
\end{Entry}

\begin{Entry}{治愈}{8,13}{⽔,⼼}
  \begin{Phonetics}{治愈}{zhi4yu4}
    \definition{v.}{curar | restaurar a saúde}
  \end{Phonetics}
\end{Entry}

%%%%%%%%%% 沽 %%%%%%%%%%
\subsection*{沽}\addcontentsline{loh}{figure}{沽}

\begin{Entry}{沽}{8}{⽔}
  \begin{Phonetics}{沽}{gu1}
    \definition*{s.}{Município de Tianjin; outro nome para Tianjin}
    \definition{v.}{comprar | vender}
  \end{Phonetics}
\end{Entry}

\begin{Entry}{沽名钓誉}{8,6,8,13}{⽔,⼝,⾦,⾔}
  \begin{Phonetics}{沽名钓誉}{gu1ming2-diao4yu4}[][HSK 7-9]
    \definition{expr.}{``Buscando fama e reputação.''; pescar fama e elogios; tentar alcançar a fama}
  \end{Phonetics}
\end{Entry}

%%%%%%%%%% 沿 %%%%%%%%%%
\subsection*{沿}\addcontentsline{loh}{figure}{沿}

\begin{Entry}{沿}{8}{⽔}
  \begin{Phonetics}{沿}{yan2}[][HSK 6]
    \definition{prep.}{ao longo}
    \definition{s.}{beira; borda; acabamento}
    \definition{v.}{seguir (uma tradição, padrão, etc.) | enfeitar (com fita, faixa, etc.)}
  \end{Phonetics}
\end{Entry}

\begin{Entry}{沿海}{8,10}{⽔,⽔}
  \begin{Phonetics}{沿海}{yan2hai3}[][HSK 6]
    \definition{s.}{costa; litoral; área ou região ao longo da costa}
  \end{Phonetics}
\end{Entry}

\begin{Entry}{沿着}{8,11}{⽔,⽬}
  \begin{Phonetics}{沿着}{yan2zhe5}[][HSK 6]
    \definition{prep.}{ao longo (de uma determinada rota)}
  \end{Phonetics}
\end{Entry}

%%%%%%%%%% 泄 %%%%%%%%%%
\subsection*{泄}\addcontentsline{loh}{figure}{泄}

\begin{Entry}{泄}{8}{⽔}
  \begin{Phonetics}{泄}{xie4}
    \definition*{s.}{Sobrenome: Xie}
    \definition{v.}{deixar sair (um fluido ou gás); descarregar; liberar | revelar (um segredo); vazar (notícias, segredos, etc.) | dar vazão a; desabafar}
  \end{Phonetics}
\end{Entry}

\begin{Entry}{泄气}{8,4}{⽔,⽓}
  \begin{Phonetics}{泄气}{xie4/qi4}
    \definition{adj.}{decepcionante | frustrante | patético}
    \definition{v.+compl.}{perder o coração | sentir"-se desencorajado | ficar desanimado}
  \end{Phonetics}
\end{Entry}

\begin{Entry}{泄底}{8,8}{⽔,⼴}
  \begin{Phonetics}{泄底}{xie4di3}
    \definition{v.}{revelar ou expor o que está no fundo de algo | divulgar a história interna; vazar segredos}
  \end{Phonetics}
\end{Entry}

\begin{Entry}{泄洪}{8,9}{⽔,⽔}
  \begin{Phonetics}{泄洪}{xie4hong2}
    \definition{v.}{liberar água da enchente (descarga de inundação)}
  \end{Phonetics}
\end{Entry}

\begin{Entry}{泄愤}{8,12}{⽔,⼼}
  \begin{Phonetics}{泄愤}{xie4fen4}
    \definition{v.}{dar vazão à raiva}
  \end{Phonetics}
\end{Entry}

\begin{Entry}{泄露}{8,21}{⽔,⾬}
  \begin{Phonetics}{泄露}{xie4lou4}
    \definition{v.}{vazar; deixar escapar; divulgar; revelar (um segredo, etc.) | vazar; escapar; descarregar (um fluido ou gás)}
  \end{Phonetics}
\end{Entry}

%%%%%%%%%% 法 %%%%%%%%%%
\subsection*{法}\addcontentsline{loh}{figure}{法}

\begin{Entry}{法}{8}{⽔}
  \begin{Phonetics}{法}{fa3}[][HSK 4]
    \definition*{s.}{Doutrina budista; o dharma | França, abreviação de 法国 | Sobrenome: Fa}
    \definition{adj.}{(usado após advérbios negativos) legal; cumpridor da lei}
    \definition{clas.}{F; Farad, medida de capacitância}
    \definition{s.}{lei; termo geral para regras de comportamento estabelecidas ou endossadas pelo Estado | maneira; método; modo; meios | padrão; modelo | artes mágicas; feitiço}
    \definition{v.}{seguir; imitar; aprender (os pontos fortes dos outros)}
  \seealsoref{法国}{fa3guo2}
  \end{Phonetics}
\end{Entry}

\begin{Entry}{法文}{8,4}{⽔,⽂}
  \begin{Phonetics}{法文}{fa3wen2}
    \definition[份]{s.}{françês, língua francesa}
  \end{Phonetics}
\end{Entry}

\begin{Entry}{法网}{8,6}{⽔,⽹}
  \begin{Phonetics}{法网}{fa3wang3}
    \definition*{s.}{Torneio de Roland Garros (French Open), torneio de tênis}
  \end{Phonetics}
\end{Entry}

\begin{Entry}{法制}{8,8}{⽔,⼑}
  \begin{Phonetics}{法制}{fa3zhi4}[][HSK 5]
    \definition{s.}{legalidade; instituições jurídicas; sistema jurídico}
  \end{Phonetics}
\end{Entry}

\begin{Entry}{法国}{8,8}{⽔,⼞}
  \begin{Phonetics}{法国}{fa3guo2}
    \definition*{s.}{França}
  \end{Phonetics}
\end{Entry}

\begin{Entry}{法国人}{8,8,2}{⽔,⼞,⼈}
  \begin{Phonetics}{法国人}{fa3guo2ren2}
    \definition{s.}{francês | pessoa ou povo da França}
  \end{Phonetics}
\end{Entry}

\begin{Entry}{法官}{8,8}{⽔,⼧}
  \begin{Phonetics}{法官}{fa3guan1}[][HSK 4]
    \definition[位,名,个,些]{s.}{juiz; justiça; termo genérico para um membro do judiciário em um tribunal de justiça}
  \end{Phonetics}
\end{Entry}

\begin{Entry}{法规}{8,8}{⽔,⾒}
  \begin{Phonetics}{法规}{fa3gui1}[][HSK 5]
    \definition[部,项,条,套,个]{s.}{lei e regulamento; estatuto; termo geral para leis, decretos, regulamentos, regras, estatutos, etc.}
  \end{Phonetics}
\end{Entry}

\begin{Entry}{法庭}{8,9}{⽔,⼴}
  \begin{Phonetics}{法庭}{fa3ting2}[][HSK 6]
    \definition{s.}{corte; tribunal | tribunal; um órgão estatal que exerce o poder judicial de forma independente}
  \end{Phonetics}
\end{Entry}

\begin{Entry}{法律}{8,9}{⽔,⼻}
  \begin{Phonetics}{法律}{fa3lv4}[][HSK 4]
    \definition[项,条,套,个]{s.}{lei; estatuto; regras de conduta formuladas pelo legislativo e cuja aplicação é garantida pelo poder estatal}
  \end{Phonetics}
\end{Entry}

\begin{Entry}{法语}{8,9}{⽔,⾔}
  \begin{Phonetics}{法语}{fa3yu3}[][HSK 6]
    \definition[种,门,句,段]{s.}{françês, língua francesa}
  \end{Phonetics}
\end{Entry}

\begin{Entry}{法院}{8,9}{⽔,⾩}
  \begin{Phonetics}{法院}{fa3yuan4}[][HSK 4]
    \definition[所,座]{s.}{tribunal; corte; órgãos estatais que exercem poder judicial independente}
  \end{Phonetics}
\end{Entry}

%%%%%%%%%% 泡 %%%%%%%%%%
\subsection*{泡}\addcontentsline{loh}{figure}{泡}

\begin{Entry}{泡}{8}{⽔}
  \begin{Phonetics}{泡}{pao1}
    \definition{adj.}{esponjoso; oco e macio; não duro}
    \definition{clas.}{usado para fezes e urina}
    \definition[串,个]{s.}{algo fofo e macio | pequeno lago}
  \end{Phonetics}
  \begin{Phonetics}{泡}{pao4}[][HSK 6]
    \definition[串,个]{s.}{bolha | algo em forma de bolha}
    \definition{v.}{mergulhar; encharcar | despejar água fervente em (chá, sopa instantânea, etc.) | enrolar; demorar-se; ficar por aí | (coloquial) (de um homem) brincar no campo; brincar com uma mulher | perder tempo; matar o tempo deliberadamente}
  \end{Phonetics}
\end{Entry}

\begin{Entry}{泡沫}{8,8}{⽔,⽔}
  \begin{Phonetics}{泡沫}{pao4mo4}[][HSK 7-9]
    \definition{s.}{espuma; pequenas bolhas se aglomeraram na superfície do líquido | ilusão; bolha econômica; essa metáfora descreve a prosperidade superficial e o florescimento de algo que, na realidade, é vazio e irreal}
  \end{Phonetics}
\end{Entry}

%%%%%%%%%% 波 %%%%%%%%%%
\subsection*{波}\addcontentsline{loh}{figure}{波}

\begin{Entry}{波}{8}{⽔}
  \begin{Phonetics}{波}{bo1}
    \definition*{s.}{Polônia, abreviação de 波兰 | Sobrenome: Bo}
    \definition{s.}{ondas, a superfície irregular da água em rios, lagos e oceanos | onda, o processo de propagação da vibração | mudanças inesperadas; uma reviravolta inesperada nos acontecimentos; metáfora para mudanças inesperadas nas coisas | olhos; metáfora do olhar errante}
  \seealsoref{波兰}{bo1lan2}
  \end{Phonetics}
\end{Entry}

\begin{Entry}{波及}{8,3}{⽔,⼃}
  \begin{Phonetics}{波及}{bo1ji2}[][HSK 7-9]
    \definition{v.}{espalhar; espalhar"-se por; envolver; afetar}
  \synonymref{涉及}{she4ji2}
  \end{Phonetics}
\end{Entry}

\begin{Entry}{波兰}{8,5}{⽔,⼋}
  \begin{Phonetics}{波兰}{bo1lan2}
    \definition*{s.}{Polônia}
  \end{Phonetics}
\end{Entry}

\begin{Entry}{波动}{8,6}{⽔,⼒}
  \begin{Phonetics}{波动}{bo1dong4}[][HSK 6]
    \definition{s.}{ondulação; flutuação; movimento de onda}
    \definition{v.}{ondular; flutuar}
  \end{Phonetics}
\end{Entry}

\begin{Entry}{波折}{8,7}{⽔,⼿}
  \begin{Phonetics}{波折}{bo1zhe2}[][HSK 7-9]
    \definition{s.}{reviravoltas; contratempo; as reviravoltas que ocorrem durante o curso das coisas, o que significa que você sofre dificuldades ou contratempos}
  \end{Phonetics}
\end{Entry}

\begin{Entry}{波音}{8,9}{⽔,⾳}
  \begin{Phonetics}{波音}{bo1yin1}
    \definition*{s.}{Boeing (empresa aeroespacial)}
    \definition{s.}{mordente (música)}
  \end{Phonetics}
\end{Entry}

\begin{Entry}{波浪}{8,10}{⽔,⽔}
  \begin{Phonetics}{波浪}{bo1lang4}[][HSK 6]
    \definition{s.}{onda; a superfície irregular da água nos rios, lagos e oceanos, geralmente se refere a águas menores e mais bonitas, frequentemente usada na linguagem falada}
  \end{Phonetics}
\end{Entry}

\begin{Entry}{波涛}{8,10}{⽔,⽔}
  \begin{Phonetics}{波涛}{bo1tao1}[][HSK 7-9]
    \definition{s.}{grandes (enormes) ondas; ondas de maré; ondas grandes costumam se referir a paisagens espetaculares ou emocionantes; são usadas tanto na linguagem falada quanto na escrita.}
  \end{Phonetics}
\end{Entry}

\begin{Entry}{波澜}{8,15}{⽔,⽔}
  \begin{Phonetics}{波澜}{bo1lan2}[][HSK 7-9]
    \definition[个,场,阵]{s.}{ondas grandes}
  \end{Phonetics}
\end{Entry}

%%%%%%%%%% 泥 %%%%%%%%%%
\subsection*{泥}\addcontentsline{loh}{figure}{泥}

\begin{Entry}{泥}{8}{⽔}
  \begin{Phonetics}{泥}{ni2}[][HSK 6]
    \definition*{s.}{Sobrenome: Ni}
    \definition{s.}{lama; atoleiro | pasta ou polpa; amassado | qualquer matéria pastosa; purê de vegetais ou frutas}
  \end{Phonetics}
  \begin{Phonetics}{泥}{ni4}
    \definition{adj.}{fanático; teimoso; obstinado; cabeçudo}
    \definition{v.}{cobrir ou rebocar com gesso, massa de vidraceiro, etc.}
  \end{Phonetics}
\end{Entry}

\begin{Entry}{泥土}{8,3}{⽔,⼟}
  \begin{Phonetics}{泥土}{ni2tu3}[][HSK 7-9]
    \definition[块,堆]{s.}{lama; solo; argila; terra}
  \end{Phonetics}
\end{Entry}

\begin{Entry}{泥潭}{8,15}{⽔,⽔}
  \begin{Phonetics}{泥潭}{ni2tan2}[][HSK 7-9]
    \definition{s.}{pântano; charco; atoleiro | poça de lama}
  \end{Phonetics}
\end{Entry}

%%%%%%%%%% 注 %%%%%%%%%%
\subsection*{注}\addcontentsline{loh}{figure}{注}

\begin{Entry}{注}{8}{⽔}
  \begin{Phonetics}{注}{zhu4}
    \definition{s.}{apostas (em jogos de azar) | notas (em um texto)}
    \definition{v.}{derramar; encher | concentrar-se em; fixar-se em; focar em  | anotar; explicar com notas | registrar; gravar | irrigar | dar exegese ou explicação}
  \end{Phonetics}
\end{Entry}

\begin{Entry}{注册}{8,5}{⽔,⼌}
  \begin{Phonetics}{注册}{zhu4/ce4}[][HSK 5]
    \definition{v.+compl.}{inscrever"-se; matricular"-se; registrar"-se; registrar"-se junto à autoridade ou escola competente para obter status legal; refere"-se especificamente ao usuário de uma determinada rede de computadores que insere o nome de usuário, senha, etc. na rede para obter permissão para usar a rede}
  \end{Phonetics}
\end{Entry}

\begin{Entry}{注册人}{8,5,2}{⽔,⼌,⼈}
  \begin{Phonetics}{注册人}{zhu4ce4ren2}
    \definition{s.}{registrante}
  \end{Phonetics}
\end{Entry}

\begin{Entry}{注册表}{8,5,8}{⽔,⼌,⾐}
  \begin{Phonetics}{注册表}{zhu4ce4biao3}
    \definition[份,个,张]{s.}{registro do Windows}
  \end{Phonetics}
\end{Entry}

\begin{Entry}{注册商标}{8,5,11,9}{⽔,⼌,⼝,⽊}
  \begin{Phonetics}{注册商标}{zhu4ce4 shang1biao1}
    \definition{s.}{marca registrada}
  \end{Phonetics}
\end{Entry}

\begin{Entry}{注视}{8,8}{⽔,⾒}
  \begin{Phonetics}{注视}{zhu4shi4}[][HSK 5]
    \definition{v.}{olhar atentamente para; observar atentamente}
  \end{Phonetics}
\end{Entry}

\begin{Entry}{注重}{8,9}{⽔,⾥}
  \begin{Phonetics}{注重}{zhu4zhong4}[][HSK 5]
    \definition{v.}{enfatizar; dar ênfase a; dar ênfase a; prestar atenção a; dar importância a}
  \end{Phonetics}
\end{Entry}

\begin{Entry}{注射}{8,10}{⽔,⼨}
  \begin{Phonetics}{注射}{zhu4she4}[][HSK 5]
    \definition{v.}{injetar; usar uma seringa para administrar medicamento líquido em um organismo}
  \end{Phonetics}
\end{Entry}

\begin{Entry}{注意}{8,13}{⽔,⼼}
  \begin{Phonetics}{注意}{zhu4/yi4}[][HSK 3]
    \definition{v.aux.}{prestar atenção; notar; ficar de olho; concentrar os pensamentos em um aspecto específico}
  \end{Phonetics}
\end{Entry}

\begin{Entry}{注意力}{8,13,2}{⽔,⼼,⼒}
  \begin{Phonetics}{注意力}{zhu4yi4li4}
    \definition{s.}{atenção}
  \end{Phonetics}
\end{Entry}

\begin{Entry}{注意力缺失症}{8,13,2,10,5,10}{⽔,⼼,⼒,⽸,⼤,⽧}
  \begin{Phonetics}{注意力缺失症}{zhu4yi4 li4 que1shi1 zheng4}
    \definition{s.}{transtorno de déficit de atenção}[他被诊断出注意力缺失症。===Ele foi diagnosticado com transtorno de déficit de atenção.]
  \end{Phonetics}
\end{Entry}

\begin{Entry}{注意地}{8,13,6}{⽔,⼼,⼟}
  \begin{Phonetics}{注意地}{zhu4yi4di4}
    \definition{s.}{área de cuidado, de observação}
  \end{Phonetics}
\end{Entry}

%%%%%%%%%% 泪 %%%%%%%%%%
\subsection*{泪}\addcontentsline{loh}{figure}{泪}

\begin{Entry}{泪}{8}{⽔}
  \begin{Phonetics}{泪}{lei4}[][HSK 4]
    \definition[滴,行]{s.}{lágrima | algo semelhante a uma lágrima}
  \end{Phonetics}
\end{Entry}

\begin{Entry}{泪水}{8,4}{⽔,⽔}
  \begin{Phonetics}{泪水}{lei4shui3}[][HSK 4]
    \definition[滴,行]{s.}{lágrima}
  \end{Phonetics}
\end{Entry}

%%%%%%%%%% 泳 %%%%%%%%%%
\subsection*{泳}\addcontentsline{loh}{figure}{泳}

\begin{Entry}{泳}{8}{⽔}
  \begin{Phonetics}{泳}{yong3}
    \definition{v.}{nadar}
  \end{Phonetics}
\end{Entry}

\begin{Entry}{泳池}{8,6}{⽔,⽔}
  \begin{Phonetics}{泳池}{yong3chi2}
    \definition{s.}{piscina}
  \seealsoref{游泳池}{you2yong3chi2}
  \seealsoref{游泳馆}{you2yong3guan3}
  \end{Phonetics}
\end{Entry}

\begin{Entry}{泳衣}{8,6}{⽔,⾐}
  \begin{Phonetics}{泳衣}{yong3yi1}
    \definition{s.}{roupa de banho | maiô}
  \seealsoref{游泳衣}{you2yong3yi1}
  \end{Phonetics}
\end{Entry}

%%%%%%%%%% 泼 %%%%%%%%%%
\subsection*{泼}\addcontentsline{loh}{figure}{泼}

\begin{Entry}{泼}{8}{⽔}
  \begin{Phonetics}{泼}{po1}[][HSK 5]
    \definition{adj.}{rude e irracional; mal"-humorado | Dialeto: ousado e vigoroso; ousado e resoluto}
    \definition{v.}{espalhar; salpicar; derramar; derramar ou espalhar o líquido com força para fora}
  \end{Phonetics}
\end{Entry}

\begin{Entry}{泼冷水}{8,7,4}{⽔,⼎,⽔}
  \begin{Phonetics}{泼冷水}{po1 leng3shui3}[][HSK 7-9]
    \definition{v.}{desencorajar; jogar um balde de água fria em; arrefecer o entusiasmo (ou o ânimo) de alguém; isso se refere metaforicamente a suprimir ou limitar o entusiasmo de alguém, ou a fazê-lo cair em si}
  \end{Phonetics}
\end{Entry}

%%%%%%%%%% 浅 %%%%%%%%%%
\subsection*{浅}\addcontentsline{loh}{figure}{浅}

\begin{Entry}{浅}{8}{⽔}
  \begin{Phonetics}{浅}{jian1}
    \definition{adj.}{murmurando, fluindo suavemente, gorgolejando suavemente}
    \definition{s.}{Onomatopéia: som de água em movimento}
  \end{Phonetics}
  \begin{Phonetics}{浅}{qian3}[][HSK 4]
    \definition{adj.}{raso; superficial | fácil; simples; redação, conteúdo, etc. simples e fáceis de entender | superficial; não é profundo em aprendizado, percepção e sabedoria | não próximo; não íntimo; sentimentos não profundos | (cor) claro; pálido;  cor pouco intensa; leve |experiência breve; duração de tempo breve | baixo grau; peso leve; nível baixo}
  \antonymref{深}{shen1}
  \end{Phonetics}
\end{Entry}

%%%%%%%%%% 炉 %%%%%%%%%%
\subsection*{炉}\addcontentsline{loh}{figure}{炉}

\begin{Entry}{炉}{8}{⽕}
  \begin{Phonetics}{炉}{lu2}
    \definition{s.}{fogão; forno; fornalha | Metalurgia: fornalha}
  \end{Phonetics}
\end{Entry}

\begin{Entry}{炉子}{8,3}{⽕,⼦}
  \begin{Phonetics}{炉子}{lu2zi5}[][HSK 7-9]
    \definition[个,只]{s.}{fogão; forno; fornalha; aparelhos ou equipamentos usados para cozinhar, ferver água, aquecer, fundir, etc.}
  \end{Phonetics}
\end{Entry}

\begin{Entry}{炉灶}{8,7}{⽕,⽕}
  \begin{Phonetics}{炉灶}{lu2zao4}[][HSK 7-9]
    \definition[个]{s.}{fogão; lareira; fogão para cozinhar; termo coletivo para fogão e lareira}
  \end{Phonetics}
\end{Entry}

%%%%%%%%%% 炎 %%%%%%%%%%
\subsection*{炎}\addcontentsline{loh}{figure}{炎}

\begin{Entry}{炎}{8}{⽕}
  \begin{Phonetics}{炎}{yan2}
    \definition{adj.}{escaldante; ardente}
    \definition{s.}{inflamação | poder; influência}
  \end{Phonetics}
\end{Entry}

\begin{Entry}{炎热}{8,10}{⽕,⽕}
  \begin{Phonetics}{炎热}{yan2re4}
    \definition{adj.}{extremamente quente | escaldante (clima)}
  \end{Phonetics}
\end{Entry}

%%%%%%%%%% 炒 %%%%%%%%%%
\subsection*{炒}\addcontentsline{loh}{figure}{炒}

\begin{Entry}{炒}{8}{⽕}
  \begin{Phonetics}{炒}{chao3}[][HSK 6]
    \definition{v.}{saltear; refogar; aquecer os alimentos em uma panela e mexer repetidamente para cozinhá-los ou secá-los | especular (na bolsa de valores, etc.) | exagerar; dar publicidade exagerada; a fim de ampliar a influência, por meio de publicidade repetida e exagerada na mídia | demitir; despedir}
  \end{Phonetics}
\end{Entry}

\begin{Entry}{炒作}{8,7}{⽕,⼈}
  \begin{Phonetics}{炒作}{chao3zuo4}[][HSK 6]
    \definition{v.}{promover (na mídia); exagerar artificialmente e promover ou desvalorizar de forma inadequada | especular; comprar e vender frequentemente no mercado de negociação para obter lucros}
  \end{Phonetics}
\end{Entry}

\begin{Entry}{炒股}{8,8}{⽕,⾁}
  \begin{Phonetics}{炒股}{chao3/gu3}[][HSK 6]
    \definition{v.+compl.}{especular em ações; comprar e vender ações; jogar no mercado}
  \end{Phonetics}
\end{Entry}

%%%%%%%%%% 炖 %%%%%%%%%%
\subsection*{炖}\addcontentsline{loh}{figure}{炖}

\begin{Entry}{炖}{8}{⽕}
  \begin{Phonetics}{炖}{dun4}[][HSK 7-9]
    \definition{v.}{cozinhar | aquecer algo colocando o recipiente em água quente}
  \end{Phonetics}
\end{Entry}

%%%%%%%%%% 爬 %%%%%%%%%%
\subsection*{爬}\addcontentsline{loh}{figure}{爬}

\begin{Entry}{爬}{8}{⽖}
  \begin{Phonetics}{爬}{pa2}[][HSK 2]
    \definition{v.}{rastejar; arrastar"-se; engatinhar | escalar; trepar; subir com dificuldade | sentar"-se; levantar"-se; levantar"-se da posição deitada ou sentada}
  \end{Phonetics}
\end{Entry}

\begin{Entry}{爬上}{8,3}{⽖,⼀}
  \begin{Phonetics}{爬上}{pa2shang4}
    \definition{v.}{escalar}
  \end{Phonetics}
\end{Entry}

\begin{Entry}{爬山}{8,3}{⽖,⼭}
  \begin{Phonetics}{爬山}{pa2/shan1}[][HSK 2]
    \definition{v.+compl.}{escalar uma montanha}
  \end{Phonetics}
\end{Entry}

\begin{Entry}{爬升}{8,4}{⽖,⼗}
  \begin{Phonetics}{爬升}{pa2sheng1}
    \definition{v.}{ascender | ganhar promoção | subir (números de vendas, etc.) | aumentar}
  \end{Phonetics}
\end{Entry}

\begin{Entry}{爬行}{8,6}{⽖,⾏}
  \begin{Phonetics}{爬行}{pa2xing2}
    \definition{v.}{rastejar | arrastar | engatinhar}
  \end{Phonetics}
\end{Entry}

\begin{Entry}{爬杆}{8,7}{⽖,⽊}
  \begin{Phonetics}{爬杆}{pa2gan1}
    \definition{s.}{escalada em poste}
    \definition{v.}{escalar um poste}
  \end{Phonetics}
\end{Entry}

\begin{Entry}{爬竿}{8,9}{⽖,⽵}
  \begin{Phonetics}{爬竿}{pa2gan1}
    \definition{s.}{poste de escalada | escalada em poste (como ginástica ou ato de circo)}
  \end{Phonetics}
\end{Entry}

\begin{Entry}{爬梳}{8,11}{⽖,⽊}
  \begin{Phonetics}{爬梳}{pa2shu1}
    \definition{v.}{vasculhar (documentos históricos, etc.) | desvendar}
  \end{Phonetics}
\end{Entry}

\begin{Entry}{爬犁}{8,11}{⽖,⽜}
  \begin{Phonetics}{爬犁}{pa2li2}
    \definition{s.}{trenó}
  \seealsoref{扒犁}{pa2li2}
  \end{Phonetics}
\end{Entry}

\begin{Entry}{爬墙}{8,14}{⽖,⼟}
  \begin{Phonetics}{爬墙}{pa2qiang2}
    \definition{v.}{escalar uma parede}
  \end{Phonetics}
\end{Entry}

%%%%%%%%%% 爸 %%%%%%%%%%
\subsection*{爸}\addcontentsline{loh}{figure}{爸}

\begin{Entry}{爸}{8}{⽗}
  \begin{Phonetics}{爸}{ba4}
    \definition[个,位]{s.}{(informal) pai}
  \seealsoref{爸爸}{ba4ba5}
  \end{Phonetics}
\end{Entry}

\begin{Entry}{爸妈}{8,6}{⽗,⼥}
  \begin{Phonetics}{爸妈}{ba4ma1}
    \definition{s.}{pai e mãe}
  \end{Phonetics}
\end{Entry}

\begin{Entry}{爸爸}{8,8}{⽗,⽗}
  \begin{Phonetics}{爸爸}{ba4ba5}[][HSK 1]
    \definition[个,位,名,群]{s.}{(informal) pai; papai; papa}
  \seealsoref{爸}{ba4}
  \end{Phonetics}
\end{Entry}

%%%%%%%%%% 版 %%%%%%%%%%
\subsection*{版}\addcontentsline{loh}{figure}{版}

\begin{Entry}{版}{8}{⽚}
  \begin{Phonetics}{版}{ban3}[][HSK 5]
    \definition{clas.}{usado como uma palavra de medida para materiais impressos (por exemplo, livros, jornais, edições)}
    \definition{s.}{chapa, placa ou bloco de impressão | edição (livros impressos) | página (de um jornal) | moldes ou fromas de construção}
  \end{Phonetics}
\end{Entry}

%%%%%%%%%% 牦 %%%%%%%%%%
\subsection*{牦}\addcontentsline{loh}{figure}{牦}

\begin{Entry}{牦}{8}{⽜}
  \begin{Phonetics}{牦}{mao2}
    \definition[头]{s.}{iaque; boi da Tartária}
  \end{Phonetics}
\end{Entry}

\begin{Entry}{牦牛}{8,4}{⽜,⽜}
  \begin{Phonetics}{牦牛}{mao2niu2}
    \definition{s.}{iaque}
  \end{Phonetics}
\end{Entry}

%%%%%%%%%% 牧 %%%%%%%%%%
\subsection*{牧}\addcontentsline{loh}{figure}{牧}

\begin{Entry}{牧}{8}{⽜}
  \begin{Phonetics}{牧}{mu4}
    \definition*{s.}{Sobrenome: Mu}
    \definition{v.}{cuidar (de ovelhas, gado, etc.); pastorear}
  \end{Phonetics}
\end{Entry}

\begin{Entry}{牧民}{8,5}{⽜,⽒}
  \begin{Phonetics}{牧民}{mu4min2}[][HSK 7-9]
    \definition[个]{s.}{pastor; pessoas em áreas pastoris que ganham a vida criando gado}
  \end{Phonetics}
\end{Entry}

\begin{Entry}{牧场}{8,6}{⽜,⼟}
  \begin{Phonetics}{牧场}{mu4chang3}[][HSK 7-9]
    \definition[个,片]{s.}{pastagem; campo de pastoreio; área de pastagem; pastos}
  \end{Phonetics}
\end{Entry}

%%%%%%%%%% 物 %%%%%%%%%%
\subsection*{物}\addcontentsline{loh}{figure}{物}

\begin{Entry}{物}{8}{⽜}
  \begin{Phonetics}{物}{wu4}
    \definition{s.}{coisa; matéria; objeto | mundo exterior distinto de si mesmo; outras pessoas; refere"-se a outras pessoas além de si mesmo ou ao ambiente em relação a si mesmo | essência; conteúdo; substância | criatura; criação}
  \end{Phonetics}
\end{Entry}

\begin{Entry}{物业}{8,5}{⽜,⼀}
  \begin{Phonetics}{物业}{wu4ye4}[][HSK 5]
    \definition[处]{s.}{propriedade; gestão de propriedades; gestão patrimonial; administração de imóveis | empresa de administração de imóveis; empresa de gestão imobiliária; empresa de administração de bens imóveis}
  \end{Phonetics}
\end{Entry}

\begin{Entry}{物价}{8,6}{⽜,⼈}
  \begin{Phonetics}{物价}{wu4jia4}[][HSK 5]
    \definition[个]{s.}{preços das commodities; preços das matérias-primas; preço das mercadorias}
  \end{Phonetics}
\end{Entry}

\begin{Entry}{物体}{8,7}{⽜,⼈}
  \begin{Phonetics}{物体}{wu4ti3}[][HSK 7-9]
    \definition[个]{s.}{corpo; substância; objeto; coisa}
  \synonymref{东西}{dong1xi5}
  \end{Phonetics}
\end{Entry}

\begin{Entry}{物证}{8,7}{⽜,⾔}
  \begin{Phonetics}{物证}{wu4zheng4}[][HSK 7-9]
    \definition[件]{s.}{provas materiais | evidências físicas | prova}
  \end{Phonetics}
\end{Entry}

\begin{Entry}{物质}{8,8}{⽜,⾙}
  \begin{Phonetics}{物质}{wu4zhi4}[][HSK 5]
    \definition[种,类,个]{s.}{matéria; substância; algo que existe além do espírito, que pode ser visto, tocado, cheirado ou detectado por instrumentos científicos | material; meios de subsistência; coisas que permitem às pessoas viver ou viver melhor, como comida, roupas, casas, dinheiro, etc.}
  \end{Phonetics}
\end{Entry}

\begin{Entry}{物品}{8,9}{⽜,⼝}
  \begin{Phonetics}{物品}{wu4pin3}[][HSK 6]
    \definition[件,个]{s.}{artigos; itens; bens}
  \end{Phonetics}
\end{Entry}

\begin{Entry}{物流}{8,10}{⽜,⽔}
  \begin{Phonetics}{物流}{wu4liu2}[][HSK 7-9]
    \definition{s.}{logística; a movimentação de mercadorias de um lugar para outro, incluindo embalagem, armazenamento e transporte}
  \synonymref{货运}{huo4yun4}
  \end{Phonetics}
\end{Entry}

\begin{Entry}{物资}{8,10}{⽜,⾙}
  \begin{Phonetics}{物资}{wu4zi1}[][HSK 7-9]
    \definition[份]{s.}{suprimentos; bens e materiais; recursos materiais necessários para a produção e a vida diária}
  \synonymref{物质}{wu4zhi4}
  \synonymref{资源}{zi1yuan2}
  \end{Phonetics}
\end{Entry}

\begin{Entry}{物理}{8,11}{⽜,⽟}
  \begin{Phonetics}{物理}{wu4li3}
    \definition{s.}{física (disciplina)}
  \end{Phonetics}
\end{Entry}

%%%%%%%%%% 狒 %%%%%%%%%%
\subsection*{狒}\addcontentsline{loh}{figure}{狒}

\begin{Entry}{狒}{8}{⽝}
  \begin{Phonetics}{狒}{fei4}
    \definition{s.}{babuíno (uma espécie de macaco)}
  \end{Phonetics}
\end{Entry}

\begin{Entry}{狒狒}{8,8}{⽝,⽝}
  \begin{Phonetics}{狒狒}{fei4fei4}
    \definition{s.}{babuíno}
  \end{Phonetics}
\end{Entry}

%%%%%%%%%% 狗 %%%%%%%%%%
\subsection*{狗}\addcontentsline{loh}{figure}{狗}

\begin{Entry}{狗}{8}{⽝}
  \begin{Phonetics}{狗}{gou3}[][HSK 2]
    \definition[条,只,群]{s.}{cão; cachorro | palavrão usado para se referir a pessoas más ou seus capangas}
  \end{Phonetics}
\end{Entry}

%%%%%%%%%% 玩 %%%%%%%%%%
\subsection*{玩}\addcontentsline{loh}{figure}{玩}

\begin{Entry}{玩}{8}{⽟}
  \begin{Phonetics}{玩}{wan2}
    \definition*{s.}{Sobrenome: Wan}
    \definition{s.}{objeto de apreciação; coisas para assistir}
    \definition{v.}{(~儿) divertir"-se; entreter"-se; fazer atividades que te deixem feliz | jogar; praticar algum tipo de atividade cultural, de entretenimento ou esportiva | recorrer a; usar métodos e meios impróprios para atingir o objetivo | provocar; subestimar; tratar com uma atitude frívola; desprezar | desfrutar; apreciar; observar | (~儿) envolver"-se em; tomar parte em; perseguir ou expressar deliberadamente um certo sentimento | ponderar; pensar cuidadosamente; apreciar}
  \end{Phonetics}
\end{Entry}

\begin{Entry}{玩儿}{8,2}{⽟,⼉}
  \begin{Phonetics}{玩儿}{wan2r5}[][HSK 1]
    \definition{v.}{divertir"-se; (entretenimento) relaxar ou experimentar alguma atividade}
  \end{Phonetics}
\end{Entry}

\begin{Entry}{玩艺}{8,4}{⽟,⾋}
  \begin{Phonetics}{玩艺}{wan2yi4}
    \variantof{玩意}
  \end{Phonetics}
\end{Entry}

\begin{Entry}{玩伴}{8,7}{⽟,⼈}
  \begin{Phonetics}{玩伴}{wan2ban4}
    \definition{s.}{parceiro de brincadeira}
  \end{Phonetics}
\end{Entry}

\begin{Entry}{玩具}{8,8}{⽟,⼋}
  \begin{Phonetics}{玩具}{wan2ju4}[][HSK 3]
    \definition[个,件,套]{s.}{brinquedo; coisas para brincar}
  \end{Phonetics}
\end{Entry}

\begin{Entry}{玩具厂}{8,8,2}{⽟,⼋,⼚}
  \begin{Phonetics}{玩具厂}{wan2ju4chang3}
    \definition{s.}{fábrica de brinquedos}
  \end{Phonetics}
\end{Entry}

\begin{Entry}{玩具车}{8,8,4}{⽟,⼋,⾞}
  \begin{Phonetics}{玩具车}{wan2ju4 che1}
    \definition{s.}{carrinho de brinquedo}
  \end{Phonetics}
\end{Entry}

\begin{Entry}{玩味}{8,8}{⽟,⼝}
  \begin{Phonetics}{玩味}{wan2wei4}
    \definition{v.}{ponderar sutilezas | ruminar (pensamentos)}
  \end{Phonetics}
\end{Entry}

\begin{Entry}{玩者}{8,8}{⽟,⽼}
  \begin{Phonetics}{玩者}{wan2zhe3}
    \definition{s.}{jogador}
  \end{Phonetics}
\end{Entry}

\begin{Entry}{玩耍}{8,9}{⽟,⽽}
  \begin{Phonetics}{玩耍}{wan2shua3}[][HSK 7-9]
    \definition{v.}{brincar; divertir"-se; entreter"-se; envolver"-se em atividades que lhe tragam alegria; jogar jogos}
  \synonymref{游玩}{you2wan2}
  \synonymref{游戏}{you2xi4}
  \antonymref{休息}{xiu1xi5}
  \end{Phonetics}
\end{Entry}

\begin{Entry}{玩家}{8,10}{⽟,⼧}
  \begin{Phonetics}{玩家}{wan2jia1}
    \definition{s.}{entusiasta (áudio, modelos de aviões, etc.) | jogador (de um jogo)}
  \end{Phonetics}
\end{Entry}

\begin{Entry}{玩偶}{8,11}{⽟,⼈}
  \begin{Phonetics}{玩偶}{wan2'ou3}
    \definition{s.}{estatueta de brinquedo | boneco de ação | bicho de pelúcia | boneca}
  \end{Phonetics}
\end{Entry}

\begin{Entry}{玩遍}{8,12}{⽟,⾡}
  \begin{Phonetics}{玩遍}{wan2bian4}
    \definition{v.}{passear (todo o país, toda a cidade, etc.) | visitar (um grande número de lugares)}
  \end{Phonetics}
\end{Entry}

\begin{Entry}{玩意}{8,13}{⽟,⼼}
  \begin{Phonetics}{玩意}{wan2yi4}
    \definition{s.}{ato | brinquedo | coisa | truque (em uma performance, show de palco, acrobacias, etc.)}
  \end{Phonetics}
\end{Entry}

\begin{Entry}{玩意儿}{8,13,2}{⽟,⼼,⼉}
  \begin{Phonetics}{玩意儿}{wan2yi4r5}[][HSK 7-9]
    \definition[个]{s.}{brinquedo; objeto de diversão | mágica, acrobacias, diálogos cruzados, canto de baladas, etc.; refere"-se às artes folclóricas, acrobacias, artes marciais, etc. | coisa; objeto | usado de forma desdenhosa; termos pejorativos para pessoas ou coisas}[他就一个没用的玩意儿。===Ele não passa de um lixo inútil.]
  \end{Phonetics}
\end{Entry}

%%%%%%%%%% 玫 %%%%%%%%%%
\subsection*{玫}\addcontentsline{loh}{figure}{玫}

\begin{Entry}{玫}{8}{⽟}
  \begin{Phonetics}{玫}{mei2}
    \definition[朵]{s.}{rosa | Arcaico: um tipo de jade bonito}
  \end{Phonetics}
\end{Entry}

\begin{Entry}{玫瑰}{8,13}{⽟,⽟}
  \begin{Phonetics}{玫瑰}{mei2gui5}[][HSK 7-9]
    \definition[束,朵,棵,株]{s.}{rosa; um arbusto decíduo, seus ramos são espinhosos, suas folhas são ovais e suas flores, que podem ser vermelho"-púrpura, brancas e de outras cores, são perfumadas e ornamentais}
  \end{Phonetics}
\end{Entry}

%%%%%%%%%% 环 %%%%%%%%%%
\subsection*{环}\addcontentsline{loh}{figure}{环}

\begin{Entry}{环}{8}{⽟}
  \begin{Phonetics}{环}{huan2}[][HSK 3]
    \definition*{s.}{Sobrenome: Huan}
    \definition{clas.}{usado para anéis}
    \definition[个,串]{s.}{anel; arco | elo; \emph{link}; passo; etapa | anel; objeto em forma de círculo | arredores}
    \definition{v.}{cercar; rodear; circular; circundar}
  \end{Phonetics}
\end{Entry}

\begin{Entry}{环卫}{8,3}{⽟,⼙}
  \begin{Phonetics}{环卫}{huan2wei4}
    \definition{s.}{limpeza pública; saneamento ambiental; saneamento geral; abreviação de 环境卫生 | Arcaico: guardas imperiais; guardas}
  \seealsoref{环境卫生}{huan2jing4wei4sheng1}
  \end{Phonetics}
\end{Entry}

\begin{Entry}{环节}{8,5}{⽟,⾋}
  \begin{Phonetics}{环节}{huan2jie2}[][HSK 5]
    \definition[个]{s.}{\emph{link}; ligação; vínculo; uma das muitas coisas que estão inter-relacionadas | segmento; estrutura anelar de alguns animais inferiores}
  \end{Phonetics}
\end{Entry}

\begin{Entry}{环保}{8,9}{⽟,⼈}
  \begin{Phonetics}{环保}{huan2bao3}[][HSK 3]
    \definition{adj.}{ecológico; benefício para o meio ambiente; não prejudica o meio ambiente}
    \definition{s.}{proteção ambiental}
  \end{Phonetics}
\end{Entry}

\begin{Entry}{环绕}{8,9}{⽟,⽷}
  \begin{Phonetics}{环绕}{huan2rao4}[][HSK 7-9]
    \definition{v.}{cercar; rodear; envolver}
  \end{Phonetics}
\end{Entry}

\begin{Entry}{环球}{8,11}{⽟,⽟}
  \begin{Phonetics}{环球}{huan2qiu2}[][HSK 7-9]
    \definition*{s.}{Terra}
    \definition{adj.}{global; mundial}
    \definition{adv.}{ao redor do mundo; circulando a Terra}
    \definition{s.}{mundo inteiro}
  \end{Phonetics}
\end{Entry}

\begin{Entry}{环境}{8,14}{⽟,⼟}
  \begin{Phonetics}{环境}{huan2jing4}[][HSK 3]
    \definition[个]{s.}{ambiente; os arredores | arredores; circunstâncias; condições políticas, econômicas, culturais, etc., dentro de um determinado âmbito}
  \end{Phonetics}
\end{Entry}

\begin{Entry}{环境卫生}{8,14,3,5}{⽟,⼟,⼙,⽣}
  \begin{Phonetics}{环境卫生}{huan2jing4wei4sheng1}
    \definition{s.}{saneamento ambiental; saneamento geral | saneamento}
  \seealsoref{环卫}{huan2wei4}
  \end{Phonetics}
\end{Entry}

%%%%%%%%%% 现 %%%%%%%%%%
\subsection*{现}\addcontentsline{loh}{figure}{现}

\begin{Entry}{现}{8}{⾒}
  \begin{Phonetics}{现}{xian4}
    \definition{adj.}{(dinheiro, etc.) em mãos}
    \definition{adv.}{recente; de improviso; naquela época; temporariamente}
    \definition{s.}{presente; atual; existente | dinheiro; dinheiro de pronto}
    \definition{v.}{mostrar; revelar; aparecer; tornar"-se visível}
  \seealsoref{见}{xian4}
  \end{Phonetics}
\end{Entry}

\begin{Entry}{现代}{8,5}{⾒,⼈}
  \begin{Phonetics}{现代}{xian4dai4}[][HSK 3]
    \definition*{s.}{Hyundai, empresa sul"-coreana}
    \definition{adj.}{moderno; contemporâneo; com características, estilo e conceitos modernos, refletindo a vanguarda, a moda e a inovação da atualidade}
    \definition{s.}{tempos modernos; era contemporânea; atualmente, na divisão cronológica da história da China, refere"-se principalmente ao período desde o Movimento 4 de Maio até os dias atuais}
  \end{Phonetics}
\end{Entry}

\begin{Entry}{现在}{8,6}{⾒,⼟}
  \begin{Phonetics}{现在}{xian4zai4}[][HSK 1]
    \definition{adv.}{agora; no momento; atualmente; neste momento, quando se fala, às vezes inclui um período de tempo mais ou menos longo antes ou depois da fala (diferente de 过去 ou 将来)}
  \seealsoref{过去}{guo4 qu5}
  \seealsoref{将来}{jiang1lai2}
  \end{Phonetics}
\end{Entry}

\begin{Entry}{现场}{8,6}{⾒,⼟}
  \begin{Phonetics}{现场}{xian4chang3}[][HSK 3]
    \definition[个,处]{s.}{local onde ocorreu o acidente, incidente ou desastre | local; ponto; local onde se realizam diretamente atividades como produção, apresentações e competições}
  \end{Phonetics}
\end{Entry}

\begin{Entry}{现有}{8,6}{⾒,⽉}
  \begin{Phonetics}{现有}{xian4you3}[][HSK 5]
    \definition{adj.}{agora disponível; existente}
    \definition{v.}{estar disponível agora; existir | Literário: ter em mãos; ter em posse}
  \end{Phonetics}
\end{Entry}

\begin{Entry}{现抓}{8,7}{⾒,⼿}
  \begin{Phonetics}{现抓}{xian4zhua1}
    \definition{v.}{improvisar}
  \end{Phonetics}
\end{Entry}

\begin{Entry}{现状}{8,7}{⾒,⽝}
  \begin{Phonetics}{现状}{xian4zhuang4}[][HSK 5]
    \definition{s.}{situação atual}
  \end{Phonetics}
\end{Entry}

\begin{Entry}{现实}{8,8}{⾒,⼧}
  \begin{Phonetics}{现实}{xian4shi2}[][HSK 3]
    \definition{adj.}{real; efetivo; verdadeiro; de acordo com circunstâncias objetivas}
    \definition[个]{s.}{realidade; factualidade; coisas que existem objetivamente}
  \end{Phonetics}
\end{Entry}

\begin{Entry}{现货}{8,8}{⾒,⾙}
  \begin{Phonetics}{现货}{xian4huo4}
    \definition{s.}{produtos à vista}
  \end{Phonetics}
\end{Entry}

\begin{Entry}{现货的}{8,8,8}{⾒,⾙,⽩}
  \begin{Phonetics}{现货的}{xian4huo4 de5}
    \definition{s.}{produtos em estoque}
  \end{Phonetics}
\end{Entry}

\begin{Entry}{现金}{8,8}{⾒,⾦}
  \begin{Phonetics}{现金}{xian4jin1}[][HSK 3]
    \definition[笔]{s.}{dinheiro; dinheiro vivo; moeda que pode ser usada diretamente | reserva de dinheiro em um banco; o dinheiro guardado no cofre do banco}
  \end{Phonetics}
\end{Entry}

\begin{Entry}{现做}{8,11}{⾒,⼈}
  \begin{Phonetics}{现做}{xian4zuo4}
    \definition{adj.}{fresco}
    \definition{v.}{fazer (comida) no local}
  \end{Phonetics}
\end{Entry}

\begin{Entry}{现象}{8,11}{⾒,⾗}
  \begin{Phonetics}{现象}{xian4xiang4}[][HSK 3]
    \definition[个,种]{s.}{aparência (das coisas); fenômeno; a forma externa e as relações manifestadas pelas coisas em seu desenvolvimento e mudança}
  \end{Phonetics}
\end{Entry}

%%%%%%%%%% 画 %%%%%%%%%%
\subsection*{画}\addcontentsline{loh}{figure}{画}

\begin{Entry}{画}{8}{⽥}
  \begin{Phonetics}{画}{hua4}[][HSK 2]
    \definition*{s.}{Sobrenome: Hua}
    \definition{clas.}{traços (de um caractere chinês)}
    \definition[张,幅]{s.}{desenho; pintura; imagem; figura desenhada | traço horizontal (em caracteres chineses)}
    \definition{v.}{desenhar; pintar | desenhar; marcar; assinar}
  \seealsoref{划}{hua4}
  \end{Phonetics}
\end{Entry}

\begin{Entry}{画儿}{8,2}{⽥,⼉}
  \begin{Phonetics}{画儿}{hua4r5}[][HSK 2]
    \definition[幅,张]{s.}{imagem; desenho; pintura; obra de arte pintada}
  \end{Phonetics}
\end{Entry}

\begin{Entry}{画册}{8,5}{⽥,⼌}
  \begin{Phonetics}{画册}{hua4ce4}[][HSK 7-9]
    \definition[部,本]{s.}{álbum de imagens; álbum de pinturas; pinturas ou imagens encadernadas}
  \end{Phonetics}
\end{Entry}

\begin{Entry}{画龙点睛}{8,5,9,13}{⽥,⿓,⽕,⽬}
  \begin{Phonetics}{画龙点睛}{hua4long2-dian3jing1}[][HSK 7-9]
    \definition{expr.}{``Toque final.''; dar vida a um dragão pintado colocando as pupilas dos seus olhos, adicionar o toque que dá vida a uma obra de arte; adicionar o toque final; adicionar uma palavra apropriada para concluir o ponto; adicionar uma ou duas palavras para finalizar o ponto; um toque crucial que reforça um ponto que de outra forma seria difícil de explicar; dar os retoques finais no trabalho de alguém; completar uma imagem}
  \end{Phonetics}
\end{Entry}

\begin{Entry}{画地为牢}{8,6,4,7}{⽥,⼟,⼂,⼧}
  \begin{Phonetics}{画地为牢}{hua4di4wei2lao2}
    \definition{expr.}{desenhar um círculo no chão para servir como uma prisão; restringir as atividades de alguém a uma área ou esfera designada; limitar; restringir | (literário) ser confinado dentro de um círculo desenhado no chão | (figurativo) limitar-se a uma gama restrita de atividades}
  \end{Phonetics}
\end{Entry}

\begin{Entry}{画面}{8,9}{⽥,⾯}
  \begin{Phonetics}{画面}{hua4mian4}[][HSK 5]
    \definition[个,幅,帧]{s.}{quadro; aparência geral de uma imagem; imagem apresentada no quadro, na tela, etc.}
  \end{Phonetics}
\end{Entry}

\begin{Entry}{画家}{8,10}{⽥,⼧}
  \begin{Phonetics}{画家}{hua4jia1}[][HSK 2]
    \definition[个,位,名,些]{s.}{pintor; pessoa com talento para pintura}
  \end{Phonetics}
\end{Entry}

\begin{Entry}{画展}{8,10}{⽥,⼫}
  \begin{Phonetics}{画展}{hua4zhan3}[][HSK 7-9]
    \definition{s.}{exposição de pinturas; exposição de arte}
  \end{Phonetics}
\end{Entry}

\begin{Entry}{画蛇添足}{8,11,11,7}{⽥,⾍,⽔,⾜}
  \begin{Phonetics}{画蛇添足}{hua4she2-tian1zu2}[][HSK 7-9]
    \definition{expr.}{arruinar o efeito adicionando algo supérfluo; dourar; fazer coisas desnecessárias pode sair pela culatra e piorar as coisas}
  \end{Phonetics}
\end{Entry}

%%%%%%%%%% 畅 %%%%%%%%%%
\subsection*{畅}\addcontentsline{loh}{figure}{畅}

\begin{Entry}{畅}{8}{⽥}
  \begin{Phonetics}{畅}{chang4}
    \definition*{s.}{Sobrenome: Chang}
    \definition{adj.}{suave; desimpedido; sem obstáculos; desobstruído | livre; desinibido}
  \end{Phonetics}
\end{Entry}

\begin{Entry}{畅谈}{8,10}{⽥,⾔}
  \begin{Phonetics}{畅谈}{chang4tan2}[][HSK 7-9]
    \definition{v.}{falar livremente e com entusiasmo; falar livremente e com satisfação; falar com entusiasmo sobre}
  \end{Phonetics}
\end{Entry}

\begin{Entry}{畅通}{8,10}{⽥,⾡}
  \begin{Phonetics}{畅通}{chang4tong1}[][HSK 6]
    \definition{adj.}{desimpedido; sem bloqueios | (canal, rota) desobstruído; não bloqueado}
  \synonymref{贯通}{guan4tong1}
  \synonymref{流畅}{liu2chang4}
  \synonymref{流利}{liu2li4}
  \synonymref{流通}{liu2tong1}
  \antonymref{堵车}{du3/che1}
  \antonymref{堵塞}{du3se4}
  \antonymref{瓶颈}{ping2jing3}
  \antonymref{阻碍}{zu3'ai4}
  \end{Phonetics}
\end{Entry}

\begin{Entry}{畅销}{8,12}{⽥,⾦}
  \begin{Phonetics}{畅销}{chang4xiao1}[][HSK 7-9]
    \definition{adj.}{mais vendido; \emph{best"-seller}}
    \definition{v.}{vender bem; ter grande procura; ter um mercado pronto; ser um \emph{best"-seller}}
  \end{Phonetics}
\end{Entry}

%%%%%%%%%% 的 %%%%%%%%%%
\subsection*{的}\addcontentsline{loh}{figure}{的}

\begin{Entry}{的}{8}{⽩}
  \begin{Phonetics}{的}{de5}[][HSK 1]
    \definition{part.}{usado para indicar posse | formar uma frase nominal ou expressão nominal | substituir a pessoa ou coisa mencionada anteriormente | no final de uma frase declarativa, para dar ênfase; usado após o verbo predicativo, enfatiza o agente da ação, o tempo, o local, etc. | usado no final de uma frase declarativa, expressa afirmação, ênfase, certeza, etc. | indica que alguém obteve uma determinada posição ou status | usado com 是 para indicar predicado ou ênfase; indica que alguém é o objeto da ação | e assim por diante; e assim por diante; e similares; usado após palavras paralelas, significa 等等, 之类 | indica uma ação (o pronome é o objeto da ação); combinado com o verbo anterior, expressa uma ação, e o pronome é o objeto dessa ação}
  \seealsoref{等等}{deng3 deng3}
  \seealsoref{是}{shi4}
  \seealsoref{之类}{zhi1lei4}
  \end{Phonetics}
  \begin{Phonetics}{的}{di1}
    \definition{s.}{abreviação de 的士: um táxi}
  \seealsoref{的士}{di1shi4}
  \end{Phonetics}
  \begin{Phonetics}{的}{di2}
    \definition{adv.}{verdadeiramente; exatamente; realmente}
  \end{Phonetics}
  \begin{Phonetics}{的}{di4}
    \definition{adj.}{alvo; centro do alvo}
  \end{Phonetics}
\end{Entry}

\begin{Entry}{的士}{8,3}{⽩,⼠}
  \begin{Phonetics}{的士}{di1shi4}
    \definition{s.}{(empréstimo linguístico) táxi}
  \end{Phonetics}
\end{Entry}

\begin{Entry}{的时候}{8,7,10}{⽩,⽇,⼈}
  \begin{Phonetics}{的时候}{de5 shi2hou4}
    \definition{part.}{naquele momento; quando; em; descreve o momento específico em que um evento ocorreu}
  \end{Phonetics}
\end{Entry}

\begin{Entry}{的话}{8,8}{⽩,⾔}
  \begin{Phonetics}{的话}{de5hua4}[][HSK 2]
    \definition{part.}{se; caso; suponha que; partícula usada após uma frase hipotética para introduzir o texto seguinte}
  \end{Phonetics}
\end{Entry}

\begin{Entry}{的确}{8,12}{⽩,⽯}
  \begin{Phonetics}{的确}{di2que4}[][HSK 4]
    \definition{adv.}{realmente; de fato, ao expressar certeza sobre a situação}
  \end{Phonetics}
\end{Entry}

%%%%%%%%%% 盲 %%%%%%%%%%
\subsection*{盲}\addcontentsline{loh}{figure}{盲}

\begin{Entry}{盲}{8}{⽬}
  \begin{Phonetics}{盲}{mang2}
    \definition{adj.}{cego | incapaz de distinguir coisas | totalmente incompetente}
    \definition{adv.}{cegamente}
  \end{Phonetics}
\end{Entry}

\begin{Entry}{盲人}{8,2}{⽬,⼈}
  \begin{Phonetics}{盲人}{mang2ren2}[][HSK 6]
    \definition[个,位,名]{s.}{cego; pessoa cega; pessoas com deficiência visual}
  \end{Phonetics}
\end{Entry}

\begin{Entry}{盲目}{8,5}{⽬,⽬}
  \begin{Phonetics}{盲目}{mang2mu4}[][HSK 7-9]
    \definition{adj.}{cego; sem rumo; essa metáfora descreve a falta de compreensão clara; consideração incompleta ou descuidada; objetivos pouco claros}
  \end{Phonetics}
\end{Entry}

%%%%%%%%%% 直 %%%%%%%%%%
\subsection*{直}\addcontentsline{loh}{figure}{直}

\begin{Entry}{直}{8}{⽬}
  \begin{Phonetics}{直}{zhi2}[][HSK 3]
    \definition*{s.}{Sobrenome: Zhi}
    \definition{adj.}{reto | vertical; perpendicular | justo; íntegro; imparcial | franco; sincero; direto ao ponto | rígido; entorpecido | direto; em linha reta; rígido | ereto; perpendicular ao solo}
    \definition{adv.}{diretamente; sempre; reto | continuamente; constantemente | apenas; simplesmente | de fato}
    \definition[条]{s.}{traço vertical (em caracteres chineses, 竖)}
    \definition{v.}{endireitar; tornar reto | alongar}
  \seealsoref{竖}{shu4}
  \antonymref{横}{heng2}
  \antonymref{曲}{qu1}
  \antonymref{弯}{wan1}
  \end{Phonetics}
\end{Entry}

\begin{Entry}{直升机}{8,4,6}{⽬,⼗,⽊}
  \begin{Phonetics}{直升机}{zhi2sheng1ji1}[][HSK 6]
    \definition[架,台,个]{s.}{helicóptero; uma aeronave que pode decolar e pousar verticalmente, com uma hélice montada na parte superior da fuselagem que gira horizontalmente, permitindo que ela permaneça no ar e decole e pouse em uma pequena área}
  \end{Phonetics}
\end{Entry}

\begin{Entry}{直译}{8,7}{⽬,⾔}
  \begin{Phonetics}{直译}{zhi2yi4}
    \definition{s.}{tradução literal}
  \seealsoref{意译}{yi4yi4}
  \end{Phonetics}
\end{Entry}

\begin{Entry}{直译器}{8,7,16}{⽬,⾔,⼝}
  \begin{Phonetics}{直译器}{zhi2yi4qi4}
    \definition{s.}{(computação) interpretador}
  \end{Phonetics}
\end{Entry}

\begin{Entry}{直到}{8,8}{⽬,⼑}
  \begin{Phonetics}{直到}{zhi2dao4}[][HSK 3]
    \definition{adv.}{até (geralmente se refere ao tempo); até que}
  \end{Phonetics}
\end{Entry}

\begin{Entry}{直线}{8,8}{⽬,⽷}
  \begin{Phonetics}{直线}{zhi2xian4}[][HSK 5]
    \definition{adj.}{direto; retilíneo | íngreme; acentuada (subida ou descida)}
    \definition[条]{s.}{linha reta}
  \end{Phonetics}
\end{Entry}

\begin{Entry}{直接}{8,11}{⽬,⼿}
  \begin{Phonetics}{直接}{zhi2jie1}[][HSK 2]
    \definition{adj.}{direto | imediato}
  \antonymref{间接}{jian4jie1}
  \end{Phonetics}
\end{Entry}

\begin{Entry}{直播}{8,15}{⽬,⼿}
  \begin{Phonetics}{直播}{zhi2bo1}[][HSK 3]
    \definition{v.}{transmissão ao vivo; transmitir diretamente, sem gravar}
  \end{Phonetics}
\end{Entry}

%%%%%%%%%% 知 %%%%%%%%%%
\subsection*{知}\addcontentsline{loh}{figure}{知}

\begin{Entry}{知}{8}{⽮}
  \begin{Phonetics}{知}{zhi1}
    \definition{s.}{conhecimento | amigo íntimo; refere"-se a um confidente}
    \definition{v.}{saber; entender; perceber; estar ciente de | contar; informar; notificar; tornar conhecido | administrar; estar encarregado de; supervisionar}
  \end{Phonetics}
\end{Entry}

\begin{Entry}{知了}{8,2}{⽮,⼅}
  \begin{Phonetics}{知了}{zhi1liao3}
    \definition[通]{s.}{cigarra}
  \end{Phonetics}
\end{Entry}

\begin{Entry}{知名}{8,6}{⽮,⼝}
  \begin{Phonetics}{知名}{zhi1ming2}[][HSK 6]
    \definition{adj.}{notável; famoso; celebrado; bem conhecido}
  \end{Phonetics}
\end{Entry}

\begin{Entry}{知识}{8,7}{⽮,⾔}
  \begin{Phonetics}{知识}{zhi1shi5}[][HSK 1]
    \definition[个,门,种]{s.}{conhecimento; conjunto de conhecimentos e experiências adquiridos pelas pessoas na prática de transformar o mundo | intelectual; refere"-se à cultura acadêmica}
  \end{Phonetics}
\end{Entry}

\begin{Entry}{知道}{8,12}{⽮,⾡}
  \begin{Phonetics}{知道}{zhi1dao5}[][HSK 1]
    \definition{v.}{saber; perceber; estar ciente de; ter conhecimento dos fatos ou da razão; ser sensato}
  \end{Phonetics}
\end{Entry}

\begin{Entry}{知道了}{8,12,2}{⽮,⾡,⼅}
  \begin{Phonetics}{知道了}{zhi1dao4le5}
    \definition{interj.}{``Entendi!''; ``O.K.!''}
  \end{Phonetics}
\end{Entry}

%%%%%%%%%% 矿 %%%%%%%%%%
\subsection*{矿}\addcontentsline{loh}{figure}{矿}

\begin{Entry}{矿}{8}{⽯}
  \begin{Phonetics}{矿}{kuang4}[][HSK 6]
    \definition[个,座]{s.}{depósito de minério | minério | mina}
  \end{Phonetics}
\end{Entry}

\begin{Entry}{矿泉水}{8,9,4}{⽯,⽔,⽔}
  \begin{Phonetics}{矿泉水}{kuang4quan2shui3}[][HSK 4]
    \definition[瓶,杯,口]{s.}{água mineral de nascente}
  \end{Phonetics}
\end{Entry}

\begin{Entry}{矿藏}{8,17}{⽯,⾋}
  \begin{Phonetics}{矿藏}{kuang4cang2}[][HSK 7-9]
    \definition{s.}{depósito mineral; recurso mineral; termo genérico para todos os tipos de minerais enterrados no subsolo}
  \end{Phonetics}
\end{Entry}

%%%%%%%%%% 码 %%%%%%%%%%
\subsection*{码}\addcontentsline{loh}{figure}{码}

\begin{Entry}{码}{8}{⽯}
  \begin{Phonetics}{码}{ma3}[][HSK 7-9]
    \definition{clas.}{refere"-se a um assunto específico ou a uma categoria de assuntos; refere"-se a uma coisa ou a uma classe de coisas | jarda; unidade de comprimento britânica e americana}
    \definition{s.}{um sinal ou objeto que indica número; código; símbolos ou ferramentas que indicam números, como códigos de barras ou \emph{QR code}}
    \definition{v.}{empilhar; acumular}
  \end{Phonetics}
\end{Entry}

\begin{Entry}{码头}{8,5}{⽯,⼤}
  \begin{Phonetics}{码头}{ma3tou5}[][HSK 5]
    \definition[个,座]{s.}{doca; cais; píer; molhe; edifícios à beira"-mar ou à beira do rio destinados exclusivamente à atracação de embarcações, embarque e desembarque de passageiros e carga e descarga de mercadorias | cidade portuária; centro comercial e de transportes; refere"-se a uma cidade comercial com transporte terrestre e marítimo bem desenvolvido.}
  \end{Phonetics}
\end{Entry}

%%%%%%%%%% 祅 %%%%%%%%%%
\subsection*{祅}\addcontentsline{loh}{figure}{祅}

\begin{Entry}{祅}{8}{⽰}
  \begin{Phonetics}{祅}{yao1}
    \definition{s.}{espírito maligno | \emph{goblin} | bruxaria}
    \variantof{妖}
  \end{Phonetics}
\end{Entry}

%%%%%%%%%% 祈 %%%%%%%%%%
\subsection*{祈}\addcontentsline{loh}{figure}{祈}

\begin{Entry}{祈}{8}{⽰}
  \begin{Phonetics}{祈}{qi2}
    \definition{v.}{orar; rezar}
  \end{Phonetics}
\end{Entry}

\begin{Entry}{祈祷}{8,11}{⽰,⽰}
  \begin{Phonetics}{祈祷}{qi2dao3}[][HSK 7-9]
    \definition{v.}{orar; realizar um ritual religioso; expressar silenciosamente seus desejos a Deus}
  \end{Phonetics}
\end{Entry}

%%%%%%%%%% 秉 %%%%%%%%%%
\subsection*{秉}\addcontentsline{loh}{figure}{秉}

\begin{Entry}{秉}{8}{⽲}
  \begin{Phonetics}{秉}{bing3}
    \definition*{s.}{Sobrenome: Bing}
    \definition{clas.}{unidade antiga de volume; 16 hu}
    \definition{v.}{Literário: segurar; agarrar | Literário: controlar; presidir; assumir o comando de}
  \seealsoref{斛}{hu2}
  \end{Phonetics}
\end{Entry}

\begin{Entry}{秉承}{8,8}{⽲,⼿}
  \begin{Phonetics}{秉承}{bing3cheng2}[][HSK 7-9]
    \definition{v.}{receber (ordens); receber (comandos); aceitar e seguir (uma ordem ou instrução)}
  \end{Phonetics}
\end{Entry}

%%%%%%%%%% 空 %%%%%%%%%%
\subsection*{空}\addcontentsline{loh}{figure}{空}

\begin{Entry}{空}{8}{⽳}
  \begin{Phonetics}{空}{kong1}[][HSK 3]
    \definition*{s.}{Sobrenome: Kong}
    \definition{adj.}{vazio; oco; nulo; não inclui nada; não contém nada ou não tem conteúdo; irrealista}
    \definition{adv.}{por nada; em vão; sem efeito}
    \definition{s.}{céu; ar | vazio; vazio do mundo dos sentidos}
  \end{Phonetics}
  \begin{Phonetics}{空}{kong4}[][HSK 4]
    \definition*{s.}{Sobrenome: Kong}
    \definition{adj.}{vazio; oco; nulo; que não contém nada; que não tem nada ou nenhum conteúdo; impraticável}
    \definition{adv.}{para nada; em vão; sem efeito}
    \definition{s.}{céu; ar | vazio; ausência do mundo dos sentidos}
  \end{Phonetics}
\end{Entry}

\begin{Entry}{空儿}{8,2}{⽳,⼉}
  \begin{Phonetics}{空儿}{kong4r5}[][HSK 3]
    \definition[个]{s.}{tempo livre; sem horário específico | sala; espaço (não utilizado); área ainda não utilizada}
    \definition{v.}{ter tempo livre}
  \end{Phonetics}
\end{Entry}

\begin{Entry}{空中}{8,4}{⽳,⼁}
  \begin{Phonetics}{空中}{kong1zhong1}[][HSK 5]
    \definition{adj.}{aéreo; aerotransportado; refere"-se à transmissão de sinais de rádio}
    \definition{s.}{no céu; no ar}
  \end{Phonetics}
\end{Entry}

\begin{Entry}{空中小姐}{8,4,3,8}{⽳,⼁,⼩,⼥}
  \begin{Phonetics}{空中小姐}{kong1zhong1xiao3jie3}
    \definition{s.}{aeromoça}
  \end{Phonetics}
\end{Entry}

\begin{Entry}{空心菜}{8,4,11}{⽳,⼼,⾋}
  \begin{Phonetics}{空心菜}{kong1xin1cai4}
    \definition{s.}{espinafre aquático | \emph{ong choy} | repolho do pântano | convolvulus aquático | glória-da-manhã aquática}
  \seealsoref{蕹菜}{weng4cai4}
  \end{Phonetics}
\end{Entry}

\begin{Entry}{空气}{8,4}{⽳,⽓}
  \begin{Phonetics}{空气}{kong1qi4}[][HSK 2]
    \definition[缕,股,份,阵]{s.}{ar; gases que compõe a atmosfera terrestre | atmosfera}
  \end{Phonetics}
\end{Entry}

\begin{Entry}{空白}{8,5}{⽳,⽩}
  \begin{Phonetics}{空白}{kong4bai2}[][HSK 7-9]
    \definition[块,片,个]{s.}{espaço; margem; espaço em branco; (na diagramação da página, páginas do livro, ilustrações, etc.) partes vazias, não preenchidas ou não utilizadas}
  \end{Phonetics}
\end{Entry}

\begin{Entry}{空军}{8,6}{⽳,⼍}
  \begin{Phonetics}{空军}{kong1jun1}[][HSK 6]
    \definition[名,位,个,支]{s.}{força aérea; um exército que luta no ar, geralmente composto por várias unidades de aviação e unidades terrestres da força aérea}
  \end{Phonetics}
\end{Entry}

\begin{Entry}{空地}{8,6}{⽳,⼟}
  \begin{Phonetics}{空地}{kong1di4}
    \definition{s.}{abertura; espaço vazio; área; gramado; terreno baldio; espaço aberto}
  \end{Phonetics}
  \begin{Phonetics}{空地}{kong4di4}[][HSK 7-9]
    \definition{s.}{abertura; espaço vazio; área; gramado; terreno baldio; espaço aberto}
  \end{Phonetics}
\end{Entry}

\begin{Entry}{空间}{8,7}{⽳,⾨}
  \begin{Phonetics}{空间}{kong1jian1}[][HSK 4]
    \definition[个]{s.}{espaço; recinto; cômodo; espaço em branco; interespaço}
  \end{Phonetics}
\end{Entry}

\begin{Entry}{空间站}{8,7,10}{⽳,⾨,⽴}
  \begin{Phonetics}{空间站}{kong1jian1zhan4}
    \definition{s.}{estação espacial}
  \end{Phonetics}
\end{Entry}

\begin{Entry}{空姐}{8,8}{⽳,⼥}
  \begin{Phonetics}{空姐}{kong1jie3}
    \definition[名,位,个]{s.}{aeromoça; comissária de bordo; abreviação de 空中小姐}
  \seealsoref{空中小姐}{kong1zhong1xiao3jie3}
  \end{Phonetics}
\end{Entry}

\begin{Entry}{空前}{8,9}{⽳,⼑}
  \begin{Phonetics}{空前}{kong1qian2}[][HSK 7-9]
    \definition{adj.}{sem precedentes; nunca antes}
  \end{Phonetics}
\end{Entry}

\begin{Entry}{空荡荡}{8,9,9}{⽳,⾋,⾋}
  \begin{Phonetics}{空荡荡}{kong1dang4dang4}[][HSK 7-9]
    \definition{adj.}{vazio; deserto; descreve uma casa, terreno, etc., como estando muito vazio | vazio; desolado; descreve um estado de vazio espiritual e falta de plenitude}
  \end{Phonetics}
\end{Entry}

\begin{Entry}{空调}{8,10}{⽳,⾔}
  \begin{Phonetics}{空调}{kong1tiao2}[][HSK 3]
    \definition[台,个]{s.}{ar-condicionado;  condicionador de ar}
  \end{Phonetics}
\end{Entry}

\begin{Entry}{空难}{8,10}{⽳,⾫}
  \begin{Phonetics}{空难}{kong1nan4}[][HSK 7-9]
    \definition{s.}{desastre aéreo; acidente aéreo; incidente aéreo; acidente de aviação}
  \end{Phonetics}
\end{Entry}

\begin{Entry}{空虚}{8,11}{⽳,⾌}
  \begin{Phonetics}{空虚}{kong1xu1}[][HSK 7-9]
    \definition{adj.}{vazio; oco; não contém nada de substancial; não é substancial}
  \end{Phonetics}
\end{Entry}

\begin{Entry}{空隙}{8,12}{⽳,⾩}
  \begin{Phonetics}{空隙}{kong4xi4}[][HSK 7-9]
    \definition{s.}{lacuna; vazio; espaço; folga; o espaço vazio no meio | intervalo; interstício; tempo livre não utilizado | chance; ocasião; oportunidade; lacunas; oportunidades a explorar}
  \end{Phonetics}
\end{Entry}

\begin{Entry}{空想}{8,13}{⽳,⼼}
  \begin{Phonetics}{空想}{kong1xiang3}[][HSK 7-9]
    \definition{s.}{pensamento irrealista; fantasia; devaneio | fantasia; sonho vão; esperança vã}
    \definition{v.}{entregar"-se à fantasia; sonhar acordado}
  \end{Phonetics}
\end{Entry}

%%%%%%%%%% 线 %%%%%%%%%%
\subsection*{线}\addcontentsline{loh}{figure}{线}

\begin{Entry}{线}{8}{⽷}
  \begin{Phonetics}{线}{xian4}[][HSK 3]
    \definition{clas.}{usado para coisas abstratas, o número é limitado a ``一'' (1)}
    \definition[根,个]{s.}{fio; corda; arame; objetos finos e longos feitos de seda, algodão, metal, etc. | linha; figura formada pelo movimento arbitrário de um ponto| feito de fio de algodão | algo em forma de linha, fio, etc. | rota de transporte; linha | linha de demarcação; limite; zona de fronteira; zona de transição | beira; borda | linha ideológica e política | pista; fio}
  \end{Phonetics}
\end{Entry}

\begin{Entry}{线香}{8,9}{⽷,⾹}
  \begin{Phonetics}{线香}{xian4xiang1}
    \definition{s.}{bastão ou vareta de incenso}
  \end{Phonetics}
\end{Entry}

\begin{Entry}{线索}{8,10}{⽷,⽷}
  \begin{Phonetics}{线索}{xian4suo3}[][HSK 5]
    \definition[条,个]{s.}{pista; fio; metáfora para o desenvolvimento das coisas ou a maneira de explorar um problema | fio; linha; refere"-se ao contexto de desenvolvimento do enredo em obras literárias}
  \end{Phonetics}
\end{Entry}

\begin{Entry}{线路}{8,13}{⽷,⾜}
  \begin{Phonetics}{线路}{xian4lu4}[][HSK 6]
    \definition[条]{s.}{linha; rota; as rotas que os veículos de transporte percorrem, etc., que as pessoas podem usar para chegar aos seus destinos | Eletricidade: linha; circuito; a rota da corrente elétrica}
  \end{Phonetics}
\end{Entry}

%%%%%%%%%% 练 %%%%%%%%%%
\subsection*{练}\addcontentsline{loh}{figure}{练}

\begin{Entry}{练}{8}{⽷}
  \begin{Phonetics}{练}{lian4}[][HSK 2]
    \definition*{s.}{Sobrenome: Lian}
    \definition{adj.}{habilidoso; experiente; bem treinado}
    \definition{s.}{seda branca}
    \definition{v.}{tratar, amaciar e branquear a seda por meio de fervura; cozinhar seda crua ou tecidos de seda crua | treinar; praticar; exercitar}
  \end{Phonetics}
\end{Entry}

\begin{Entry}{练习}{8,3}{⽷,⼄}
  \begin{Phonetics}{练习}{lian4xi2}[][HSK 2]
    \definition[项,次]{s.}{exercício (em livros); tarefas ou exercícios organizados para consolidar os resultados da aprendizagem}
    \definition{v.}{praticar; exercitar; repitir várias vezes até ficar bem treinado}
  \end{Phonetics}
\end{Entry}

%%%%%%%%%% 组 %%%%%%%%%%
\subsection*{组}\addcontentsline{loh}{figure}{组}

\begin{Entry}{组}{8}{⽷}
  \begin{Phonetics}{组}{zu3}[][HSK 2]
    \definition{clas.}{usado para conjuntos, séries, suítes, baterias}
    \definition[个]{s.}{grupo; uma unidade composta por um pequeno número de pessoas}
    \definition{v.}{formar; organizar; combinar pessoas ou coisas dispersas em um todo ou sistema}
  \end{Phonetics}
\end{Entry}

\begin{Entry}{组长}{8,4}{⽷,⾧}
  \begin{Phonetics}{组长}{zu3zhang3}[][HSK 2]
    \definition[名,位,个]{s.}{líder de grupo; um supervisor de grupo}
  \end{Phonetics}
\end{Entry}

\begin{Entry}{组合}{8,6}{⽷,⼝}
  \begin{Phonetics}{组合}{zu3he2}[][HSK 3]
    \definition{s.}{associação; combinação; o todo organizado | combinação; retirar n elementos diferentes de m elementos e agrupá-los, independentemente da ordem, em que cada grupo contenha pelo menos um elemento diferente, o resultado obtido é chamado de combinação de n elementos de m}
    \definition{v.}{compor; constituir; formar}
  \end{Phonetics}
\end{Entry}

\begin{Entry}{组成}{8,6}{⽷,⼽}
  \begin{Phonetics}{组成}{zu3/cheng2}[][HSK 2]
    \definition{v.+compl.}{formar; compor; inventar}
  \end{Phonetics}
\end{Entry}

\begin{Entry}{组织}{8,8}{⽷,⽷}
  \begin{Phonetics}{组织}{zu3zhi1}[][HSK 5]
    \definition{s.}{organização; um coletivo ou grupo estabelecido de acordo com determinados objetivos e princípios | sistema organizado; vários fatores interligados de determinada maneira, formando um sistema | tecer; a combinação de linhas horizontais e verticais nos têxteis | tecido; os seres humanos, os animais, as plantas e outros seres vivos são compostos por uma combinação de células com formas e funções semelhantes, que formam os tecidos; os tecidos são as unidades que compõem os diversos órgãos}
  \end{Phonetics}
\end{Entry}

%%%%%%%%%% 绅 %%%%%%%%%%
\subsection*{绅}\addcontentsline{loh}{figure}{绅}

\begin{Entry}{绅}{8}{⽷}
  \begin{Phonetics}{绅}{shen1}
    \definition[个]{s.}{Arcaico: cinto (usado por funcionários e homens de letras) | nobreza}
  \end{Phonetics}
\end{Entry}

\begin{Entry}{绅士}{8,3}{⽷,⼠}
  \begin{Phonetics}{绅士}{shen1shi4}[][HSK 7-9]
    \definition{adj.}{cavalheiro; bem"-educado}
    \definition[位,名]{s.}{cavalheiro; antigamente, isso se referia a indivíduos influentes e bem"-sucedidos em uma área local, geralmente proprietários de terras ou funcionários aposentados}
  \synonymref{名流}{ming2liu2}
  \synonymref{伸手}{shen1/shou3}
  \antonymref{恶霸}{e4ba4}
  \antonymref{混蛋}{hun4dan4}
  \end{Phonetics}
\end{Entry}

%%%%%%%%%% 细 %%%%%%%%%%
\subsection*{细}\addcontentsline{loh}{figure}{细}

\begin{Entry}{细}{8}{⽷}
  \begin{Phonetics}{细}{xi4}[][HSK 4]
    \definition{adj.}{fino; delgado; esguio; esbelto | fino; em partículas pequenas; grãos pequenos | fino e macio;  um sussuro | fino; requintado; delicado | cuidadoso; detalhado; meticuloso | ínfimo; minúsculo; insignificante; diminuto | jovem; pequeno}
  \antonymref{粗}{cu1}
  \end{Phonetics}
\end{Entry}

\begin{Entry}{细心}{8,4}{⽷,⼼}
  \begin{Phonetics}{细心}{xi4xin1}[][HSK 7-9]
    \definition{adj.}{cuidadoso; atento; meticuloso}
  \seealsoref{认真}{ren4zhen1}
  \seealsoref{细致}{xi4zhi4}
  \synonymref{谨慎}{jin3shen4}
  \synonymref{精心}{jing1xin1}
  \synonymref{留心}{liu2/xin1}
  \synonymref{留意}{liu2/yi4}
  \synonymref{小心}{xiao3xin5}
  \synonymref{用心}{yong4 xin1}
  \synonymref{注意}{zhu4/yi4}
  \synonymref{仔细}{zi3xi4}
  \antonymref{粗心}{cu1xin1}
  \antonymref{大意}{da4yi5}
  \antonymref{含糊}{han2hu5}
  \antonymref{鲁莽}{lu3mang3}
  \antonymref{马虎}{ma3hu5}
  \end{Phonetics}
\end{Entry}

\begin{Entry}{细节}{8,5}{⽷,⾋}
  \begin{Phonetics}{细节}{xi4jie2}[][HSK 4]
    \definition[处]{s.}{detalhe; particularidade; aspectos secundários ou partes sutis de um enredo ou episódios secundários usados em uma obra literária para expressar o caráter de uma pessoa ou as características essenciais de uma coisa}
  \end{Phonetics}
\end{Entry}

\begin{Entry}{细胞}{8,9}{⽷,⾁}
  \begin{Phonetics}{细胞}{xi4bao1}[][HSK 6]
    \definition[个]{s.}{célula; unidade estrutural e funcional básica de um organismo, com uma variedade de formas, composta principalmente pelo núcleo, citoplasma e membrana celular; as plantas também possuem paredes celulares fora da membrana celular}
  \end{Phonetics}
\end{Entry}

\begin{Entry}{细致}{8,10}{⽷,⾄}
  \begin{Phonetics}{细致}{xi4zhi4}[][HSK 4]
    \definition{adj.}{meticuloso; cuidadoso; minucioso | intrincado; delicado}
  \end{Phonetics}
\end{Entry}

\begin{Entry}{细菌}{8,11}{⽷,⾋}
  \begin{Phonetics}{细菌}{xi4jun1}[][HSK 6]
    \definition[个]{s.}{germe; bactéria; um organismo muito pequeno, invisível aos olhos humanos}
  \end{Phonetics}
\end{Entry}

\begin{Entry}{细菌战}{8,11,9}{⽷,⾋,⼽}
  \begin{Phonetics}{细菌战}{xi4jun1zhan4}
    \definition{s.}{guerra biológica}
  \end{Phonetics}
\end{Entry}

\begin{Entry}{细微}{8,13}{⽷,⼻}
  \begin{Phonetics}{细微}{xi4wei1}[][HSK 7-9]
    \definition{adj.}{leve; minúsculo; sutil; minuto}
  \synonymref{渺小}{miao3xiao3}
  \synonymref{轻微}{qing1wei1}
  \synonymref{微小}{wei1xiao3}
  \synonymref{细节}{xi4jie2}
  \antonymref{巨大}{ju4da4}
  \antonymref{庞大}{pang2da4}
  \antonymref{显著}{xian3zhu4}
  \antonymref{重大}{zhong4da4}
  \end{Phonetics}
\end{Entry}

\begin{Entry}{细腻}{8,13}{⽷,⾁}
  \begin{Phonetics}{细腻}{xi4ni4}[][HSK 7-9]
    \definition{adj.}{exótico; delicado; fino e suave; suave e delicado | pormenorizado; soberbo; requintado; (descrição, desempenho, etc.) detalhado e meticuloso}
  \synonymref{精细}{jing1xi4}
  \synonymref{精致}{jing1zhi4}
  \synonymref{细致}{xi4zhi4}
  \antonymref{粗糙}{cu1cao1}
  \antonymref{粗鲁}{cu1lu3}
  \end{Phonetics}
\end{Entry}

%%%%%%%%%% 织 %%%%%%%%%%
\subsection*{织}\addcontentsline{loh}{figure}{织}

\begin{Entry}{织}{8}{⽷}
  \begin{Phonetics}{织}{zhi1}[][HSK 6]
    \definition{v.}{tecer; fazer fios ou linhas cruzarem para fazer seda, tecido, lã, etc. | tricotar; usar agulhas para fazer fios ou linhas entrelaçados para confeccionar suéteres, meias, rendas, redes, etc. | sobrepor-se; entrelaçar-se; cruzar; entrelaçar}
  \end{Phonetics}
\end{Entry}

%%%%%%%%%% 终 %%%%%%%%%%
\subsection*{终}\addcontentsline{loh}{figure}{终}

\begin{Entry}{终}{8}{⽷}
  \begin{Phonetics}{终}{zhong1}
    \definition*{s.}{Sobrenome: Zhong}
    \definition{adj.}{tudo; todo; inteiro; o tempo todo do começo ao fim}
    \definition{adv.}{afinal; no final; eventualmente; finalmente}
    \definition{s.}{fim; término | tempo todo; período inteiro; o tempo todo | final | morte; refere"-se à morte}
  \end{Phonetics}
\end{Entry}

\begin{Entry}{终于}{8,3}{⽷,⼆}
  \begin{Phonetics}{终于}{zhong1yu2}[][HSK 3]
    \definition{adv.}{finalmente; por fim; eventualmente; no final; indica uma situação que surge após várias mudanças ou espera}
  \end{Phonetics}
\end{Entry}

\begin{Entry}{终止}{8,4}{⽷,⽌}
  \begin{Phonetics}{终止}{zhong1zhi3}[][HSK 5]
    \definition{v.}{parar; terminar | anular; encerrar; expirar; revogar}
  \end{Phonetics}
\end{Entry}

\begin{Entry}{终究}{8,7}{⽷,⽳}
  \begin{Phonetics}{终究}{zhong1jiu1}
    \definition{adv.}{afinal de contas; enfatiza que, não importa o que aconteça, a natureza das pessoas e das coisas não mudará e que as características básicas devem ser reconhecidas (tem o efeito de fortalecer o tom) |  no final; indica que um determinado resultado ocorrerá ou não, frequentemente usado em especulações, julgamentos etc. | afinal de contas; indica que, apesar do grande esforço ou da grande esperança, o resultado objetivo ainda é insatisfatório, geralmente com o significado de pesar ou pena | afinal de contas; indica que um resultado desejado finalmente aparece}
  \end{Phonetics}
\end{Entry}

\begin{Entry}{终身}{8,7}{⽷,⾝}
  \begin{Phonetics}{终身}{zhong1shen1}[][HSK 5]
    \definition{s.}{vida inteira; por toda a vida; por toda a vida}
  \end{Phonetics}
\end{Entry}

\begin{Entry}{终点}{8,9}{⽷,⽕}
  \begin{Phonetics}{终点}{zhong1dian3}[][HSK 5]
    \definition[个]{s.}{destino; ponto terminal; ponto de chegada; lugar onde uma jornada termina | final; refere"-se especificamente ao local onde a corrida é interrompida}
  \end{Phonetics}
\end{Entry}

%%%%%%%%%% 绍 %%%%%%%%%%
\subsection*{绍}\addcontentsline{loh}{figure}{绍}

\begin{Entry}{绍}{8}{⽷}
  \begin{Phonetics}{绍}{shao4}
    \definition*{s.}{Shaoxing, abreviação de 绍兴 | Sobrenome: Shao}
    \definition{v.}{continuar; herdar}
  \seealsoref{绍兴}{shao4xing1}
  \end{Phonetics}
\end{Entry}

\begin{Entry}{绍兴}{8,6}{⽷,⼋}
  \begin{Phonetics}{绍兴}{shao4xing1}
    \definition*{s.}{Shaoxing, anteriormente conhecida como Kuaiji, é uma cidade de nível de prefeitura na província de Zhejiang, na China; é uma grande cidade localizada na parte centro"-norte da província de Zhejiang}
  \end{Phonetics}
\end{Entry}

%%%%%%%%%% 经 %%%%%%%%%%
\subsection*{经}\addcontentsline{loh}{figure}{经}

\begin{Entry}{经}{8}{⽷}
  \begin{Phonetics}{经}{jing1}[][HSK 7-9]
    \definition*{s.}{Sobrenome: Jing}
    \definition{adj.}{constante; regular}
    \definition{prep.}{como resultado de; depois; através de}
    \definition{s.}{urdidura, os fios longitudinais de um tecido | Medicina chinesa: canais principais e colaterais | Geografia: longitude | escritura; sutra; cânone; clássico | menstruação}
    \definition{v.}{Literário: gerenciar; lidar com; envolver"-se em | enforcar"-se | suportar; ficar de pé; aguentar; resistir | passar por; sofrer; experimentar}
  \antonymref{纬}{wei3}
  \end{Phonetics}
  \begin{Phonetics}{经}{jing4}
    \definition{s.}{fio de urdidura na tecelagem}
  \end{Phonetics}
\end{Entry}

\begin{Entry}{经久不息}{8,3,4,10}{⽷,⼃,⼀,⼼}
  \begin{Phonetics}{经久不息}{jing1jiu3-bu4xi1}[][HSK 7-9]
    \definition{expr.}{prolongado; duradouro}
  \end{Phonetics}
\end{Entry}

\begin{Entry}{经历}{8,4}{⽷,⼚}
  \begin{Phonetics}{经历}{jing1li4}[][HSK 3]
    \definition[个,次,段,种]{s.}{experiência; coisas que você viu, fez ou sofreu pessoalmente}
    \definition{v.}{passar por; atravessar; ter visto, feito ou sofrido pessoalmente}
  \end{Phonetics}
\end{Entry}

\begin{Entry}{经过}{8,6}{⽷,⾡}
  \begin{Phonetics}{经过}{jing1guo4}[][HSK 2]
    \definition{prep.}{depois; através; como resultado de; passar por uma atividade ou evento que traz novas mudanças para pessoas ou coisas}
    \definition[个,段,番]{s.}{processo; curso; experiência}
    \definition{v.}{passar; atravessar; passar por; através de (local, tempo, ação, etc.)}
  \end{Phonetics}
\end{Entry}

\begin{Entry}{经典}{8,8}{⽷,⼋}
  \begin{Phonetics}{经典}{jing1dian3}[][HSK 4]
    \definition{adj.}{clássico; (escritos ou obras, etc.) que são típicos, autorizados}
    \definition{s.}{clássicos; escritos tradicionais e valiosos; os livros mais importantes e fundamentais da religião | escrituras; escritos de doutrinas religiosas}
  \end{Phonetics}
\end{Entry}

\begin{Entry}{经受}{8,8}{⽷,⼜}
  \begin{Phonetics}{经受}{jing1shou4}[][HSK 7-9]
    \definition{v.}{experimentar; suportar; resistir; aguentar; passar por}
  \end{Phonetics}
\end{Entry}

\begin{Entry}{经线}{8,8}{⽷,⽷}
  \begin{Phonetics}{经线}{jing1xian4}
    \definition{s.}{urdidura | Geografia: meridiano | linha de longitude}
  \end{Phonetics}
\end{Entry}

\begin{Entry}{经度}{8,9}{⽷,⼴}
  \begin{Phonetics}{经度}{jing1du4}[][HSK 7-9]
    \definition{s.}{longitude}
  \end{Phonetics}
\end{Entry}

\begin{Entry}{经济}{8,9}{⽷,⽔}
  \begin{Phonetics}{经济}{jing1ji4}[][HSK 3]
    \definition{adj.}{econômico; parcimonioso; descreve algo que custa pouco e rende muito; preço acessível}
    \definition{s.}{economia; a soma das relações de produção social em um determinado período histórico|econômico; de valor industrial ou econômico; refere"-se à economia nacional; também se refere a um determinado setor da economia nacional | economia; refere"-se às atividades econômicas, incluindo produção, circulação, distribuição e consumo, bem como atividades ou processos financeiros, de seguros, etc. | renda; situação financeira; refere"-se à situação financeira de uma pessoa}
    \definition{v.}{governar o país e beneficiar o povo}
  \end{Phonetics}
\end{Entry}

\begin{Entry}{经贸}{8,9}{⽷,⾙}
  \begin{Phonetics}{经贸}{jing1mao4}[][HSK 7-9]
    \definition{s.}{economia e comércio; o termo coletivo para economia e comércio}
  \end{Phonetics}
\end{Entry}

\begin{Entry}{经费}{8,9}{⽷,⾙}
  \begin{Phonetics}{经费}{jing1fei4}[][HSK 5]
    \definition[笔]{s.}{fundos; desembolso; gastos | despesas; gastos}
  \end{Phonetics}
\end{Entry}

\begin{Entry}{经验}{8,10}{⽷,⾺}
  \begin{Phonetics}{经验}{jing1yan4}[][HSK 3]
    \definition[个,次,种]{s.}{experiência; conhecimento ou habilidades adquiridos através da prática}
    \definition{v.}{experimentar; passar por; ter visto, feito ou sofrido pessoalmente}
  \end{Phonetics}
\end{Entry}

\begin{Entry}{经商}{8,11}{⽷,⼝}
  \begin{Phonetics}{经商}{jing1/shang1}[][HSK 7-9]
    \definition{v.+compl.}{dedicar"-se ao comércio; estar em atividade comercial; envolver"-se em atividades comerciais}
  \end{Phonetics}
\end{Entry}

\begin{Entry}{经常}{8,11}{⽷,⼱}
  \begin{Phonetics}{经常}{jing1chang2}[][HSK 2]
    \definition{adj.}{habitual; cotidiano; diário; do dia a dia}
    \definition{adv.}{frequentemente; regularmente; constantemente; com frequência; indica que a ação ocorre repetidamente}
  \end{Phonetics}
\end{Entry}

\begin{Entry}{经理}{8,11}{⽷,⽟}
  \begin{Phonetics}{经理}{jing1li3}[][HSK 2]
    \definition[个,位,名]{s.}{gerente; diretor; pessoas responsáveis pela gestão e administração de empresas ou corporações}
  \end{Phonetics}
\end{Entry}

\begin{Entry}{经营}{8,11}{⽷,⾋}
  \begin{Phonetics}{经营}{jing1ying2}[][HSK 3]
    \definition{v.}{executar; gerenciar; operar; envolver"-se em; planejar e gerenciar (empresas, etc.) | gerenciar; refere"-se a planos e organizações em geral}
  \end{Phonetics}
\end{Entry}

%%%%%%%%%% 罔 %%%%%%%%%%
\subsection*{罔}\addcontentsline{loh}{figure}{罔}

\begin{Entry}{罔}{8}{⼌}
  \begin{Phonetics}{罔}{wang3}
    \definition{v.}{enganar}
  \end{Phonetics}
\end{Entry}

%%%%%%%%%% 罗 %%%%%%%%%%
\subsection*{罗}\addcontentsline{loh}{figure}{罗}

\begin{Entry}{罗}{8}{⽹}
  \begin{Phonetics}{罗}{luo2}[][HSK 7-9]
    \definition*{s.}{Sobrenome: Luo}
    \definition{clas.}{uma grosa; uma bruta; doze dúzias; 144 unidades}
    \definition{s.}{uma rede para capturar pássaros; redes de proteção contra pássaros | peneira; coador; tela | uma espécie de gaze de seda; tecidos de seda com textura solta}
    \definition{v.}{capturar pássaros com uma rede; prender pássaros em redes | espalhar; exibir; mostrar | coletar; reunir; recrutar; invocar | peneirar}
  \end{Phonetics}
\end{Entry}

%%%%%%%%%% 者 %%%%%%%%%%
\subsection*{者}\addcontentsline{loh}{figure}{者}

\begin{Entry}{者}{8}{⽼}
  \begin{Phonetics}{者}{zhe3}[][HSK 3]
    \definition*{s.}{Sobrenome: Zhe}
    \definition{part.}{significa 是 e é usado após palavras, frases e orações para indicar uma pausa}
    \definition{pron.}{usado para se referir à pessoa, coisa ou assunto que realiza uma ação ou possui um determinado atributo | pessoas; caras (Usado para se referir a alguém envolvido em uma determinada profissão, que acredita em uma determinada ideologia ou que tem uma forte tendência para algo) | usado após certos números ou palavras direcionais para se referir a coisas mencionadas anteriormente | significado semelhante a 这 (mais comum na linguagem coloquial antiga)}
  \seealsoref{是}{shi4}
  \seealsoref{这}{zhe4}
  \end{Phonetics}
\end{Entry}

%%%%%%%%%% 肏 %%%%%%%%%%
\subsection*{肏}\addcontentsline{loh}{figure}{肏}

\begin{Entry}{肏}{8}{⼊}
  \begin{Phonetics}{肏}{cao4}
    \definition{v.}{(vulgar) foder; palavras sujas usadas para insultar pessoas; refere"-se à relação sexual masculina}
  \end{Phonetics}
\end{Entry}

%%%%%%%%%% 股 %%%%%%%%%%
\subsection*{股}\addcontentsline{loh}{figure}{股}

\begin{Entry}{股}{8}{⾁}
  \begin{Phonetics}{股}{gu3}[][HSK 6]
    \definition*{s.}{Sobrenome: Gu}
    \definition{clas.}{usado para coisas em tiras, longas e estreitas | usado para gás, odor, força, etc. | Pejorativo: usado para um grupo de pessoas}
    \definition{s.}{coxa; ancas | seção (de um escritório, empresa, etc.); unidades organizacionais em agências governamentais, empresas e grupos | fio; camada | uma das várias partes iguais de propriedade | ação; \emph{stock}; ação do capital social; uma parte igual de fundos ou propriedade | a perna mais longa de um triângulo retângulo}
  \end{Phonetics}
\end{Entry}

\begin{Entry}{股东}{8,5}{⾁,⼀}
  \begin{Phonetics}{股东}{gu3dong1}[][HSK 6]
    \definition[个,位,名,家]{s.}{acionista de uma sociedade anônima com direito a participar e votar nas assembleias gerais; refere"-se também a investidores em outras empresas industriais e comerciais administradas por sociedades}
  \end{Phonetics}
\end{Entry}

\begin{Entry}{股市}{8,5}{⾁,⼱}
  \begin{Phonetics}{股市}{gu3shi4}[][HSK 7-9]
    \definition{s.}{mercado de ações; mercado de compra e venda de ações | cotações na bolsa de valores}
  \end{Phonetics}
\end{Entry}

\begin{Entry}{股民}{8,5}{⾁,⽒}
  \begin{Phonetics}{股民}{gu3min2}[][HSK 7-9]
    \definition{s.}{pessoa que compra e vende ações; acionista | corretor de ações | investidor em ações}
  \end{Phonetics}
\end{Entry}

\begin{Entry}{股份}{8,6}{⾁,⼈}
  \begin{Phonetics}{股份}{gu3fen4}[][HSK 7-9]
    \definition{s.}{ação; unidade de distribuição de capital de uma sociedade anônima ou de uma empresa cooperativa, com uma parcela igual do capital total}
  \end{Phonetics}
\end{Entry}

\begin{Entry}{股票}{8,11}{⾁,⽰}
  \begin{Phonetics}{股票}{gu3piao4}[][HSK 6]
    \definition[只,股]{s.}{ação; quotas; certificado de ações; título de capital; capital social; títulos utilizados para representar ações}
  \end{Phonetics}
\end{Entry}

%%%%%%%%%% 肥 %%%%%%%%%%
\subsection*{肥}\addcontentsline{loh}{figure}{肥}

\begin{Entry}{肥}{8}{⾁}
  \begin{Phonetics}{肥}{fei2}[][HSK 4]
    \definition{adj.}{gordo; gorduroso; contém muita gordura, , geralmente não usado para descrever pessoas | fértil; rico | solto; largo; folgado; (roupas, etc.) largas | lucrativo; rendendo bons lucros}
    \definition{s.}{fertilizante; esterco}
    \definition{v.}{fertilizar; tornar fértil ou obeso | enriquecer com renda ilegal, ilícita}
  \antonymref{瘦}{shou4}
  \end{Phonetics}
\end{Entry}

\begin{Entry}{肥沃}{8,7}{⾁,⽔}
  \begin{Phonetics}{肥沃}{fei2wo4}[][HSK 7-9]
    \definition{adj.}{fértil; rico (de solo); (terra) contém mais nutrientes e água adequados para o crescimento das plantas}
  \end{Phonetics}
\end{Entry}

\begin{Entry}{肥皂}{8,7}{⾁,⽩}
  \begin{Phonetics}{肥皂}{fei2zao4}[][HSK 7-9]
    \definition[块,条]{s.}{sabão; produtos químicos usados para limpeza}
  \end{Phonetics}
\end{Entry}

\begin{Entry}{肥胖}{8,9}{⾁,⾁}
  \begin{Phonetics}{肥胖}{fei2pang4}[][HSK 7-9]
    \definition{adj.}{gordo; obeso; corpulento; excesso de gordura corporal}
  \end{Phonetics}
\end{Entry}

\begin{Entry}{肥料}{8,10}{⾁,⽃}
  \begin{Phonetics}{肥料}{fei2liao4}[][HSK 7-9]
    \definition[种,袋,把]{s.}{esterco; fertilizante}
  \end{Phonetics}
\end{Entry}

%%%%%%%%%% 肩 %%%%%%%%%%
\subsection*{肩}\addcontentsline{loh}{figure}{肩}

\begin{Entry}{肩}{8}{⾁}
  \begin{Phonetics}{肩}{jian1}[][HSK 5]
    \definition*{s.}{Sobrenome: Jian}
    \definition{s.}{ombro; torso}
    \definition{v.}{assumir; empreender; carregar; suportar; suportar um fardo}
  \end{Phonetics}
\end{Entry}

\begin{Entry}{肩负}{8,6}{⾁,⾙}
  \begin{Phonetics}{肩负}{jian1fu4}[][HSK 7-9]
    \definition{v.}{assumir; empreender; carregar; suportar; ser confiado a}
  \end{Phonetics}
\end{Entry}

\begin{Entry}{肩膀}{8,14}{⾁,⾁}
  \begin{Phonetics}{肩膀}{jian1bang3}[][HSK 7-9]
    \definition[个,副]{s.}{ombro}
  \end{Phonetics}
\end{Entry}

%%%%%%%%%% 肯 %%%%%%%%%%
\subsection*{肯}\addcontentsline{loh}{figure}{肯}

\begin{Entry}{肯}{8}{⾁}
  \begin{Phonetics}{肯}{ken3}[][HSK 6]
    \definition{s.}{carne presa ao osso}
    \definition{v.}{concordar; consentir}
    \definition{v.aux.}{estar disposto a; estar pronto para; para expressar vontade subjetiva; vontade de aceitar}
  \end{Phonetics}
\end{Entry}

\begin{Entry}{肯定}{8,8}{⾁,⼧}
  \begin{Phonetics}{肯定}{ken3ding4}[][HSK 5]
    \definition{adj.}{certo; definitivo; positivo; afirmativo | positivo; afirmativo; aceitável}
    \definition{adv.}{certamente; definitivamente; sem dúvida; sem dúvida alguma}
    \definition{v.}{afirmar; aprovar; confirmar; considerar positivo; reconhecer a existência de algo ou sua autenticidade ou racionalidade}
  \antonymref{否定}{fou3ding4}
  \end{Phonetics}
\end{Entry}

%%%%%%%%%% 肺 %%%%%%%%%%
\subsection*{肺}\addcontentsline{loh}{figure}{肺}

\begin{Entry}{肺}{8}{⾁}
  \begin{Phonetics}{肺}{fei4}[][HSK 6]
    \definition[叶]{s.}{pulmão | pulmões; órgãos respiratórios de humanos e animais superiores}
  \end{Phonetics}
\end{Entry}

%%%%%%%%%% 肾 %%%%%%%%%%
\subsection*{肾}\addcontentsline{loh}{figure}{肾}

\begin{Entry}{肾}{8}{⾁}
  \begin{Phonetics}{肾}{shen4}[][HSK 7-9]
    \definition[个,只]{s.}{rim}
  \end{Phonetics}
\end{Entry}

%%%%%%%%%% 肿 %%%%%%%%%%
\subsection*{肿}\addcontentsline{loh}{figure}{肿}

\begin{Entry}{肿}{8}{⾁}
  \begin{Phonetics}{肿}{zhong3}[][HSK 6]
    \definition{s.}{inchaço; protuberância}
    \definition{v.}{inchar; estar inchado}
  \end{Phonetics}
\end{Entry}

%%%%%%%%%% 舍 %%%%%%%%%%
\subsection*{舍}\addcontentsline{loh}{figure}{舍}

\begin{Entry}{舍}{8}{⾆}
  \begin{Phonetics}{舍}{she3}
    \definition{v.}{abandonar; desistir; descartar; jogar fora | dar esmola; dispensar caridade}
  \end{Phonetics}
  \begin{Phonetics}{舍}{she4}
    \definition*{s.}{Sobrenome: She}
    \definition{clas.}{uma unidade antiga de distância igual a 30 li, 里}
    \definition{pron.}{meu, uma palavra humilde usada para se referir aos parentes mais jovens ou de geração inferior}
    \definition{s.}{cabana; casa | minha casa; minha humilde morada | chiqueiro; galpão; curral de gado}
  \seealsoref{里}{li3}
  \end{Phonetics}
\end{Entry}

\begin{Entry}{舍不得}{8,4,11}{⾆,⼀,⼻}
  \begin{Phonetics}{舍不得}{she3bu5de5}[][HSK 5]
    \definition{v.}{não se pode abandonar ou deixar, não se quer usar ou descartar; detestar separar-me ou usar}
  \end{Phonetics}
\end{Entry}

\begin{Entry}{舍得}{8,11}{⾆,⼻}
  \begin{Phonetics}{舍得}{she3 de5}[][HSK 5]
    \definition{v.}{não guardar rancor; estar disposto a abrir mão de algo; estar disposto a gastar dinheiro, tempo, etc.; estar disposto a abrir mão de pessoas, oportunidades, coisas, etc. que são importantes para você}
  \end{Phonetics}
\end{Entry}

%%%%%%%%%% 艰 %%%%%%%%%%
\subsection*{艰}\addcontentsline{loh}{figure}{艰}

\begin{Entry}{艰}{8}{⾉}
  \begin{Phonetics}{艰}{jian1}
    \definition{adj.}{difícil; duro}
  \end{Phonetics}
\end{Entry}

\begin{Entry}{艰巨}{8,4}{⾉,⼯}
  \begin{Phonetics}{艰巨}{jian1ju4}[][HSK 7-9]
    \definition{adj.}{árduo; oneroso; difícil; formidável}
  \end{Phonetics}
\end{Entry}

\begin{Entry}{艰辛}{8,7}{⾉,⾟}
  \begin{Phonetics}{艰辛}{jian1xin1}[][HSK 7-9]
    \definition{adj.}{duro; árduo; difícil}
  \end{Phonetics}
\end{Entry}

\begin{Entry}{艰苦}{8,8}{⾉,⾋}
  \begin{Phonetics}{艰苦}{jian1ku3}[][HSK 5]
    \definition{adj.}{duro; resistente; árduo; difícil; condições de trabalho ou de vida ruins que tornam as pessoas miseráveis}
  \end{Phonetics}
\end{Entry}

\begin{Entry}{艰苦奋斗}{8,8,8,4}{⾉,⾋,⼤,⽃}
  \begin{Phonetics}{艰苦奋斗}{jian1ku3-fen4dou4}[][HSK 7-9]
    \definition{expr.}{luta árdua; trabalho duro persistente}
    \definition{v.}{trabalhar duro e perseverantemente; lutar arduamente em meio às dificuldades; trabalhar diligentemente desafiando (apesar) das dificuldades; travar uma luta árdua}
  \end{Phonetics}
\end{Entry}

\begin{Entry}{艰险}{8,9}{⾉,⾩}
  \begin{Phonetics}{艰险}{jian1xian3}[][HSK 7-9]
    \definition{adj.}{perigoso}
    \definition{s.}{dificuldades e perigos}
  \end{Phonetics}
\end{Entry}

\begin{Entry}{艰难}{8,10}{⾉,⾫}
  \begin{Phonetics}{艰难}{jian1nan2}[][HSK 5]
    \definition{adj.}{duro; árduo; difícil}
  \end{Phonetics}
\end{Entry}

%%%%%%%%%% 苗 %%%%%%%%%%
\subsection*{苗}\addcontentsline{loh}{figure}{苗}

\begin{Entry}{苗}{8}{⾋}
  \begin{Phonetics}{苗}{miao2}[][HSK 7-9]
    \definition*{s.}{Miao, grupo étnico, abreviação de 苗族 | Sobrenome: Miao}
    \definition[棵,株,尾,头,些]{s.}{planta jovem; muda; broto; plantas recém"-germinadas | descendente; filho | filhotes de alguns animais; animais domésticos recém"-nascidos | vacina | algo que se assemelha a uma planta jovem}
  \seealsoref{苗族}{miao2zu2}
  \end{Phonetics}
\end{Entry}

\begin{Entry}{苗头}{8,5}{⾋,⼤}
  \begin{Phonetics}{苗头}{miao2tou5}[][HSK 7-9]
    \definition{s.}{sintoma de uma tendência; indício de um novo desenvolvimento; uma tendência ou situação de desenvolvimento ligeiramente emergente}
  \end{Phonetics}
\end{Entry}

\begin{Entry}{苗条}{8,7}{⾋,⽊}
  \begin{Phonetics}{苗条}{miao2tiao5}[][HSK 7-9]
    \definition{adj.}{(figura feminina) magra; esbelta; esbelta e graciosa}
  \end{Phonetics}
\end{Entry}

\begin{Entry}{苗族}{8,11}{⾋,⽅}
  \begin{Phonetics}{苗族}{miao2zu2}
    \definition*{s.}{Grupo étnico Hmong ou Miao do sudoeste da China; uma das minorias étnicas da China, distribuída por Guizhou 贵州, Hunan 湖南, Yunnan 云南, Guangxi 广西, Sichuan 四川, Guangdong 广东 e Hubei 湖北}
  \seealsoref{广东}{guang3dong1}
  \seealsoref{广西}{guang3xi1}
  \seealsoref{贵州}{gui4zhou1}
  \seealsoref{湖北}{hu2bei3}
  \seealsoref{湖南}{hu2nan2}
  \seealsoref{四川}{si4chuan1}
  \seealsoref{云南}{yun2nan2}
  \end{Phonetics}
\end{Entry}

%%%%%%%%%% 苛 %%%%%%%%%%
\subsection*{苛}\addcontentsline{loh}{figure}{苛}

\begin{Entry}{苛}{8}{⾋}
  \begin{Phonetics}{苛}{ke1}
    \definition{adj.}{duro; severo; exigente; opressivo | excessivamente elaborado; exorbitante; diverso; variado}
  \end{Phonetics}
\end{Entry}

\begin{Entry}{苛刻}{8,8}{⾋,⼑}
  \begin{Phonetics}{苛刻}{ke1ke4}[][HSK 7-9]
    \definition{adj.}{severo; rigoroso; os requisitos são muito rigorosos ou as condições são muito elevadas}
  \end{Phonetics}
\end{Entry}

%%%%%%%%%% 若 %%%%%%%%%%
\subsection*{若}\addcontentsline{loh}{figure}{若}

\begin{Entry}{若}{8}{⾋}
  \begin{Phonetics}{若}{ruo4}[][HSK 6]
    \definition*{s.}{Sobrenome: Ruo}
    \definition{adv.}{como se; como se fosse; usado antes do verbo para indicar que o que foi dito é mais ou menos assim, equivalente a 好像}
    \definition{conj.}{se; usado na primeira parte de uma frase composta, expressa uma relação hipotética, equivalente a 如果}
    \definition{pron.}{você; referir"-se ao interlocutor como 你 ou 你的}
    \definition{v.}{parecer}
  \seealsoref{好像}{hao3xiang4}
  \seealsoref{你}{ni3}
  \seealsoref{你的}{ni3 de5}
  \seealsoref{如果}{ru2guo3}
  \end{Phonetics}
\end{Entry}

\begin{Entry}{若干}{8,3}{⾋,⼲}
  \begin{Phonetics}{若干}{ruo4gan1}[][HSK 7-9]
    \definition{pron.}{alguns; vários; um certo número ou quantidade; significam 某些, 有些 ou 一些 | quantos?; quanto custa?}
  \seealsoref{某些}{mou3 xie1}
  \seealsoref{一些}{yi4xie1}
  \seealsoref{有些}{you3xie1}
  \end{Phonetics}
\end{Entry}

%%%%%%%%%% 苦 %%%%%%%%%%
\subsection*{苦}\addcontentsline{loh}{figure}{苦}

\begin{Entry}{苦}{8}{⾋}
  \begin{Phonetics}{苦}{ku3}[][HSK 4]
    \definition{adj.}{amargo; descreve um sabor parecido com o de melão amargo ou raiz de coptis | difícil; doloroso; sofrido}
    \definition{adv.}{meticulosamente; diligentemente; pacientemente}
    \definition{v.}{causar sofrimento a alguém; dificultar a vida de alguém; causar dor; tornar desconfortável | sofrer de; ser incomodado por; sentir"-se angustiado com uma situação | estar desgastado; cortar demais; descrever a superação de um certo nível em algum aspecto}
  \antonymref{甘}{gan1}
  \antonymref{甜}{tian2}
  \end{Phonetics}
\end{Entry}

\begin{Entry}{苦力}{8,2}{⾋,⼒}
  \begin{Phonetics}{苦力}{ku3li4}[][HSK 7-9]
    \definition{s.}{trabalhador hindu ou chinês; carregador | Empréstimo linguístico: \emph{coolie}, trabalhador chinês não qualificado nos tempos coloniais | trabalho amargo | trabalho árduo}
  \end{Phonetics}
\end{Entry}

\begin{Entry}{苦心}{8,4}{⾋,⼼}
  \begin{Phonetics}{苦心}{ku3xin1}[][HSK 7-9]
    \definition{adv.}{com esmero; assiduamente; meticulosamente; com esforço ou dedicação; isso demonstra que você dedicou muito esforço e energia}
    \definition{s.}{dores; dificuldade tomada; o esforço e a energia investidos em trabalhar arduamente por algo}
  \end{Phonetics}
\end{Entry}

\begin{Entry}{苦瓜}{8,5}{⾋,⽠}
  \begin{Phonetics}{苦瓜}{ku3gua1}
    \definition{s.}{melão amargo (cabaça amarga, pêra bálsamo, maçã bálsamo, pepino amargo)}
  \end{Phonetics}
\end{Entry}

\begin{Entry}{苦练}{8,8}{⾋,⽷}
  \begin{Phonetics}{苦练}{ku3 lian4}[][HSK 7-9]
    \definition{v.}{treinar diligentemente; praticar bastante}
  \end{Phonetics}
\end{Entry}

\begin{Entry}{苦恼}{8,9}{⾋,⼼}
  \begin{Phonetics}{苦恼}{ku3nao3}[][HSK 7-9]
    \definition{adj.}{aflito; preocupado; atormentado}
    \definition{v.}{atormentar; causar dor e sofrimento}
  \end{Phonetics}
\end{Entry}

\begin{Entry}{苦笑}{8,10}{⾋,⽵}
  \begin{Phonetics}{苦笑}{ku3xiao4}[][HSK 7-9]
    \definition{s.}{sorriso forçado; sorriso irônico; sorriso amargo}
    \definition{v.}{forçar um sorriso; esboçar um sorriso irônico; dar uma risada amarga; produzir um sorriso forçado; forçar um sorriso quando você não está de bom humor}
  \end{Phonetics}
\end{Entry}

\begin{Entry}{苦难}{8,10}{⾋,⾫}
  \begin{Phonetics}{苦难}{ku3nan4}[][HSK 7-9]
    \definition{adj.}{sofrimento; dificuldade}
    \definition{s.}{sofrimento; miséria; angústia; tribulação; dor e desastre}
  \end{Phonetics}
\end{Entry}

%%%%%%%%%% 英 %%%%%%%%%%
\subsection*{英}\addcontentsline{loh}{figure}{英}

\begin{Entry}{英}{8}{⾋}
  \begin{Phonetics}{英}{ying1}
    \definition*{s.}{Reino Unido, abreviação de 英国 | Sobrenome: Ying}
    \definition{s.}{flor | herói; pessoa excepcional | uma pessoa de talento ou sabedoria extraordinários}
  \seealsoref{英国}{ying1guo2}
  \end{Phonetics}
\end{Entry}

\begin{Entry}{英文}{8,4}{⾋,⽂}
  \begin{Phonetics}{英文}{ying1wen2}[][HSK 2]
    \definition{s.}{inglês, língua inglesa; a forma escrita do inglês}
  \end{Phonetics}
\end{Entry}

\begin{Entry}{英国}{8,8}{⾋,⼞}
  \begin{Phonetics}{英国}{ying1guo2}
    \definition*{s.}{Reino Unido; Grã-Bretanha; Inglaterra}
  \end{Phonetics}
\end{Entry}

\begin{Entry}{英国人}{8,8,2}{⾋,⼞,⼈}
  \begin{Phonetics}{英国人}{ying1guo2ren2}
    \definition{s.}{inglês | pessoa ou povo do Reino Unido}
  \end{Phonetics}
\end{Entry}

\begin{Entry}{英明}{8,8}{⾋,⽇}
  \begin{Phonetics}{英明}{ying1ming2}
    \definition{adj.}{sábio; brilhante; excelente e sábio}
  \end{Phonetics}
\end{Entry}

\begin{Entry}{英勇}{8,9}{⾋,⼒}
  \begin{Phonetics}{英勇}{ying1yong3}[][HSK 4]
    \definition{adj.}{heroico; valente; bravo; corajoso; extraordinariamente corajoso}
  \end{Phonetics}
\end{Entry}

\begin{Entry}{英语}{8,9}{⾋,⾔}
  \begin{Phonetics}{英语}{ying1yu3}[][HSK 2]
    \definition{s.}{inglês, língua inglesa}
  \end{Phonetics}
\end{Entry}

\begin{Entry}{英雄}{8,12}{⾋,⾫}
  \begin{Phonetics}{英雄}{ying1xiong2}[][HSK 6]
    \definition{adj.}{heróico}
    \definition[名,个,位]{s.}{herói; uma pessoa cujas habilidades e coragem superam as das pessoas comuns | herói; aqueles que não têm medo das dificuldades, dos perigos ou da morte e que lutam bravamente pelos interesses do povo, mesmo ao custo das suas próprias vidas}
  \end{Phonetics}
\end{Entry}

%%%%%%%%%% 苹 %%%%%%%%%%
\subsection*{苹}\addcontentsline{loh}{figure}{苹}

\begin{Entry}{苹}{8}{⾋}
  \begin{Phonetics}{苹}{ping2}
    \definition[个]{s.}{uma espécie de artemísia | maçã | lentilha"-d'água}
  \end{Phonetics}
\end{Entry}

\begin{Entry}{苹果}{8,8}{⾋,⽊}
  \begin{Phonetics}{苹果}{ping2guo3}[][HSK 3]
    \definition[个,斤,筐,箱,棵,种]{s.}{maçã}
  \end{Phonetics}
\end{Entry}

%%%%%%%%%% 茂 %%%%%%%%%%
\subsection*{茂}\addcontentsline{loh}{figure}{茂}

\begin{Entry}{茂}{8}{⾋}
  \begin{Phonetics}{茂}{mao4}
    \definition*{s.}{Sobrenome: Mao}
    \definition{adj.}{luxuriante; exuberante; profuso | rico e esplêndido; rico e requintado}
    \definition[种]{s.}{ciclopentadieno; composto orgânico, fórmula molecular $C_5H_6$, líquido incolor, usado na fabricação de pesticidas, plásticos, etc.}
  \end{Phonetics}
\end{Entry}

\begin{Entry}{茂密}{8,11}{⾋,⼧}
  \begin{Phonetics}{茂密}{mao4mi4}[][HSK 7-9]
    \definition{adj.}{(grama ou árvores) denso; espesso; exuberante e denso}
  \end{Phonetics}
\end{Entry}

\begin{Entry}{茂盛}{8,11}{⾋,⽫}
  \begin{Phonetics}{茂盛}{mao4sheng4}[][HSK 7-9]
    \definition{adj.}{exuberante; viçoso; abundante; florescente | exuberante; florescente; próspero}
  \end{Phonetics}
\end{Entry}

%%%%%%%%%% 茄 %%%%%%%%%%
\subsection*{茄}\addcontentsline{loh}{figure}{茄}

\begin{Entry}{茄}{8}{⾋}
  \begin{Phonetics}{茄}{jia1}
    \definition{s.}{caracter fonético usado em empréstimos linguísticos para o som ``jia'', embora 夹 seja mais comum}
  \seealsoref{夹}{jia1}
  \end{Phonetics}
  \begin{Phonetics}{茄}{qie2}
    \definition[只]{s.}{berinjela}
  \end{Phonetics}
\end{Entry}

\begin{Entry}{茄子}{8,3}{⾋,⼦}
  \begin{Phonetics}{茄子}{qie2zi5}[][HSK 6]
    \definition{interj.}{Onomatopéia: ``Xis!'' fonético (ao ser fotografado), equivale ao ``Diga xis!''}
    \definition[个,根]{s.}{berinjela (fruto e planta)}
  \end{Phonetics}
\end{Entry}

%%%%%%%%%% 茅 %%%%%%%%%%
\subsection*{茅}\addcontentsline{loh}{figure}{茅}

\begin{Entry}{茅}{8}{⾋}
  \begin{Phonetics}{茅}{mao2}
    \definition*{s.}{Sobrenome: Mao}
    \definition[座]{s.}{capim-cogon | planta semelhante ao capim-cogon (como palha)}
  \end{Phonetics}
\end{Entry}

\begin{Entry}{茅台}{8,5}{⾋,⼝}
  \begin{Phonetics}{茅台}{mao2tai2}
    \definition*{s.}{Moutai (bebida alcoólica)}
  \seealsoref{茅台酒}{mao2tai2 jiu3}
  \end{Phonetics}
\end{Entry}

\begin{Entry}{茅台酒}{8,5,10}{⾋,⼝,⾣}
  \begin{Phonetics}{茅台酒}{mao2tai2 jiu3}
    \definition*[瓶,斤,箱,口,杯]{s.}{Maotai; Moutai (espírito forte) | Maotai (um licor chinês); Mao Tai}
  \end{Phonetics}
\end{Entry}

\begin{Entry}{茅厕}{8,8}{⾋,⼚}
  \begin{Phonetics}{茅厕}{mao2ce4}
    \definition{s.}{(dialeto) latrina}
  \end{Phonetics}
\end{Entry}

%%%%%%%%%% 茎 %%%%%%%%%%
\subsection*{茎}\addcontentsline{loh}{figure}{茎}

\begin{Entry}{茎}{8}{⾋}
  \begin{Phonetics}{茎}{jing1}[][HSK 7-9]
    \definition{clas.}{Literário: utilizado como classificador para indicar um objeto em forma de tira}[几茎小草。===Algumas folhas de grama.]
    \definition[根]{s.}{caule (de uma planta); talo; tronco | algo como um caule ou haste}
  \end{Phonetics}
\end{Entry}

%%%%%%%%%% 虎 %%%%%%%%%%
\subsection*{虎}\addcontentsline{loh}{figure}{虎}

\begin{Entry}{虎}{8}{⾌}
  \begin{Phonetics}{虎}{hu3}[][HSK 5]
    \definition*{s.}{Sobrenome: Hu}
    \definition{adj.}{corajoso; bravo; valente; vigoroso}
    \definition[只]{s.}{tigre}
    \definition{v.}{blefar; o mesmo que 唬 | parecer feroz; mostrar a aparência feroz de alguém}
  \seealsoref{唬}{hu3}
  \seealsoref{老虎}{lao3hu3}
  \end{Phonetics}
\end{Entry}

\begin{Entry}{虎口}{8,3}{⾌,⼝}
  \begin{Phonetics}{虎口}{hu3kou3}
    \definition{s.}{lugar perigoso | cova do tigre}
  \end{Phonetics}
\end{Entry}

\begin{Entry}{虎虎}{8,8}{⾌,⾌}
  \begin{Phonetics}{虎虎}{hu3hu3}
    \definition{adj.}{formidável | forte | vigoroso}
  \end{Phonetics}
\end{Entry}

\begin{Entry}{虎鼬}{8,18}{⾌,⿏}
  \begin{Phonetics}{虎鼬}{hu3you4}
    \definition{s.}{doninha}
  \end{Phonetics}
\end{Entry}

%%%%%%%%%% 表 %%%%%%%%%%
\subsection*{表}\addcontentsline{loh}{figure}{表}

\begin{Entry}{表}{8}{⾐}
  \begin{Phonetics}{表}{biao3}[][HSK 2]
    \definition*{s.}{Sobrenome: Biao}
    \definition{s.}{exterior; superfície; externo | a relação entre os filhos ou netos de um irmão e uma irmã ou de irmãs | modelo; exemplo; padrão | memorial a um imperador; um tipo de petição da antiguidade, frequentemente usado para expressar intenções; mais tarde, também usado para expressar opiniões sobre eventos importantes | formulário; lista; gráfico; tabela | medidor; instrumento para medir uma determinada quantidade | relógio; um dispositivo para medir o tempo, menor que um relógio, que geralmente pode ser carregado no bolso | medidor de luz solar; antiga vara de madeira para medir o tempo através da sombra do sol | coluna usada antigamente para marcação}
    \definition{v.}{mostrar; expressar; expressar ideias, pensamentos, sentimentos, etc. | administrar medicamentos para aliviar o resfriado; na medicina tradicional chinesa refere"-se ao uso de medicamentos para dissipar o frio e o vento que afetam o corpo}
  \end{Phonetics}
\end{Entry}

\begin{Entry}{表白}{8,5}{⾐,⽩}
  \begin{Phonetics}{表白}{biao3bai2}[][HSK 7-9]
    \definition{v.}{justificar; explicar"-se; expressar ou declarar claramente; explicar (as próprias intenções) aos outros}
  \end{Phonetics}
\end{Entry}

\begin{Entry}{表示}{8,5}{⾐,⽰}
  \begin{Phonetics}{表示}{biao3shi4}[][HSK 2]
    \definition{s.}{expressão; indicação}
    \definition{v.}{mostrar; expressar; indicar | significar | expressar pensamentos e sentimentos através de palavras, ações ou expressões faciais}
  \end{Phonetics}
\end{Entry}

\begin{Entry}{表决}{8,6}{⾐,⼎}
  \begin{Phonetics}{表决}{biao3jue2}[][HSK 7-9]
    \definition{v.}{votar; colocar em votação; decidir por votação}
  \end{Phonetics}
\end{Entry}

\begin{Entry}{表扬}{8,6}{⾐,⼿}
  \begin{Phonetics}{表扬}{biao3yang2}[][HSK 4]
    \definition{v.}{elogiar; louvar; elogiar publicamente as pessoas boas e as boas ações}
  \end{Phonetics}
\end{Entry}

\begin{Entry}{表扬信}{8,6,9}{⾐,⼿,⼈}
  \begin{Phonetics}{表扬信}{biao3yang2 xin4}
    \definition{s.}{carta de elogio; depoimento}
  \end{Phonetics}
\end{Entry}

\begin{Entry}{表达}{8,6}{⾐,⾡}
  \begin{Phonetics}{表达}{biao3da2}[][HSK 3]
    \definition{v.}{entregar; expressar; mostrar; manifestar; transmitir; comunicar; refere"-se ao processo de transmitir pensamentos, sentimentos ou opiniões pessoais a outras pessoas por meio de linguagem, texto, ações, etc.}
  \end{Phonetics}
\end{Entry}

\begin{Entry}{表态}{8,8}{⾐,⼼}
  \begin{Phonetics}{表态}{biao3/tai4}[][HSK 7-9]
    \definition{v.+compl.}{tornar conhecida a sua posição; declarar onde se posiciona; comprometer"-se; expressar claramente a atitude de alguém em relação a algo}
  \end{Phonetics}
\end{Entry}

\begin{Entry}{表明}{8,8}{⾐,⽇}
  \begin{Phonetics}{表明}{biao3ming2}[][HSK 3]
    \definition{v.}{indicar; demonstrar; expressar; marcar; expressar claramente; expressar de forma clara}
  \end{Phonetics}
\end{Entry}

\begin{Entry}{表现}{8,8}{⾐,⾒}
  \begin{Phonetics}{表现}{biao3xian4}[][HSK 3]
    \definition[个,种,份]{s.}{desempenho; expressão; manifestação; comportamento; as ideias, o estilo, as qualidades, o nível ou as capacidades demonstrados em ação}
    \definition{v.}{mostrar; expressar; exibir; manifestar; descrever; demonstrar algum tipo de pensamento, espírito, qualidade, sentimento ou habilidade, etc. | exibir"-se; demonstrar de forma inadequada e intencional alguma habilidade, ponto forte ou vantagem}
  \end{Phonetics}
\end{Entry}

\begin{Entry}{表述}{8,8}{⾐,⾡}
  \begin{Phonetics}{表述}{biao3shu4}[][HSK 7-9]
    \definition{s.}{formulação; expressão}
    \definition{v.}{declarar; explicar; descrever em palavras ou texto}
  \end{Phonetics}
\end{Entry}

\begin{Entry}{表面}{8,9}{⾐,⾯}
  \begin{Phonetics}{表面}{biao3mian4}[][HSK 3]
    \definition{s.}{superfície; face; exterior; aparência | aparência; superficialidade | mostrador (placa); mostrador do relógio | aparência; a aparência externa das coisas ou a parte não essencial delas}
  \end{Phonetics}
\end{Entry}

\begin{Entry}{表面上}{8,9,3}{⾐,⾯,⼀}
  \begin{Phonetics}{表面上}{biao3mian4shang4}[][HSK 6]
    \definition{adj.}{superficial; ostensivo; aparente}
  \end{Phonetics}
\end{Entry}

\begin{Entry}{表格}{8,10}{⾐,⽊}
  \begin{Phonetics}{表格}{biao3ge2}[][HSK 3]
    \definition[张,份,个]{s.}{tabela; formulário}
  \end{Phonetics}
\end{Entry}

\begin{Entry}{表情}{8,11}{⾐,⼼}
  \begin{Phonetics}{表情}{biao3qing2}[][HSK 4]
    \definition[个,种,幅]{s.}{expressão; expressão facial; expressão de pensamentos e sentimentos internos por meio de mudanças faciais ou de gestos}
    \definition{v.}{expressar pensamentos e sentimentos internos por meio de mudanças faciais ou de gestos}
  \end{Phonetics}
\end{Entry}

\begin{Entry}{表率}{8,11}{⾐,⽞}
  \begin{Phonetics}{表率}{biao3shuai4}[][HSK 7-9]
    \definition{s.}{modelo; exemplo}
  \end{Phonetics}
\end{Entry}

\begin{Entry}{表彰}{8,14}{⾐,⼺}
  \begin{Phonetics}{表彰}{biao3zhang1}[][HSK 7-9]
    \definition{v.}{citar; honrar; elogiar}
  \end{Phonetics}
\end{Entry}

\begin{Entry}{表演}{8,14}{⾐,⽔}
  \begin{Phonetics}{表演}{biao3yan3}[][HSK 3]
    \definition[场]{s.}{performance; exposição; refere"-se às atividades expressas pelos atores por meio da linguagem, voz, expressões faciais, instrumentos musicais ou movimentos}
    \definition{v.}{atuar; representar; interpretar | demonstrar; fazer demonstrações | fingir; agir de forma afetada; metáfora para fingir deliberadamente uma determinada atitude para enganar alguém}
  \end{Phonetics}
\end{Entry}

\begin{Entry}{表演艺术}{8,14,4,5}{⾐,⽔,⾋,⽊}
  \begin{Phonetics}{表演艺术}{biao3yan3 yi4shu4}
    \definition{s.}{arte performática}
  \end{Phonetics}
\end{Entry}

\begin{Entry}{表演者}{8,14,8}{⾐,⽔,⽼}
  \begin{Phonetics}{表演者}{biao3yan3 zhe3}
    \definition{s.}{artista; intérprete}
  \end{Phonetics}
\end{Entry}

\begin{Entry}{表演特技}{8,14,10,7}{⾐,⽔,⽜,⼿}
  \begin{Phonetics}{表演特技}{biao3yan3 te4ji4}
    \definition{s.}{acrobacia | pirueta | façanha}
  \end{Phonetics}
\end{Entry}

\begin{Entry}{表演游戏}{8,14,12,6}{⾐,⽔,⽔,⼽}
  \begin{Phonetics}{表演游戏}{biao3yan3 you2xi4}
    \definition{s.}{exibição dramática}
  \end{Phonetics}
\end{Entry}

\begin{Entry}{表演赛}{8,14,14}{⾐,⽔,⾙}
  \begin{Phonetics}{表演赛}{biao3yan3sai4}
    \definition{s.}{partida de exibição; jogo de exibição; uma competição realizada para celebração, comemoração, demonstração, publicidade, etc.}
  \end{Phonetics}
\end{Entry}

%%%%%%%%%% 衬 %%%%%%%%%%
\subsection*{衬}\addcontentsline{loh}{figure}{衬}

\begin{Entry}{衬}{8}{⾐}
  \begin{Phonetics}{衬}{chen4}
    \definition[件,个]{s.}{forro}
    \definition{v.}{forrar; colocar algo embaixo | fornecer um pano de fundo para; destacar; servir como contraste para}
  \end{Phonetics}
\end{Entry}

\begin{Entry}{衬托}{8,6}{⾐,⼿}
  \begin{Phonetics}{衬托}{chen4tuo1}[][HSK 7-9]
    \definition{v.}{destacar; acentuar}[红色衬托了她的笑容。===O vermelho acentuava seu sorriso.]
  \end{Phonetics}
\end{Entry}

\begin{Entry}{衬衣}{8,6}{⾐,⾐}
  \begin{Phonetics}{衬衣}{chen4yi1}[][HSK 3]
    \definition[件,个]{s.}{camisa; também se refere a uma peça de roupa usada por baixo do casaco}
  \end{Phonetics}
\end{Entry}

\begin{Entry}{衬衫}{8,8}{⾐,⾐}
  \begin{Phonetics}{衬衫}{chen4shan1}[][HSK 3]
    \definition[件,个]{s.}{camisa; blusa; camisa ocidental usada por baixo}
  \end{Phonetics}
\end{Entry}

%%%%%%%%%% 规 %%%%%%%%%%
\subsection*{规}\addcontentsline{loh}{figure}{规}

\begin{Entry}{规}{8}{⾒}
  \begin{Phonetics}{规}{gui1}
    \definition*{s.}{Sobrenome: Gui}
    \definition[个,种]{s.}{bússola | regulamentação; regra | (mecânica) medidor | compasso; ferramenta para desenhar círculos}
    \definition{v.}{admoestar; aconselhar; advertir | planejar; fazer planos}
  \end{Phonetics}
\end{Entry}

\begin{Entry}{规划}{8,6}{⾒,⼑}
  \begin{Phonetics}{规划}{gui1hua4}[][HSK 5]
    \definition[个,项]{s.}{plano; projeto; planejamento; programa; programação; esquematização; plano de desenvolvimento de longo prazo mais abrangente}
    \definition{v.}{planejar; programar}
  \end{Phonetics}
\end{Entry}

\begin{Entry}{规则}{8,6}{⾒,⼑}
  \begin{Phonetics}{规则}{gui1ze2}[][HSK 4]
    \definition{adj.}{ordenado; regular; descreve a forma, estrutura, arranjo, etc., que se conformam a uma determinada maneira organizada}
    \definition{s.}{regra; regulamento; sistema ou código de conduta prescrito para observância comum | lei; norma}
  \end{Phonetics}
\end{Entry}

\begin{Entry}{规定}{8,8}{⾒,⼧}
  \begin{Phonetics}{规定}{gui1ding4}[][HSK 3]
    \definition[个,条,项,款]{s.}{regra; regulamento; estipulação; tomar decisões sobre a forma, o método, a quantidade ou a qualidade de algo}
    \definition{v.}{estipular; prover; prescrever; estabelecer requisitos ou restrições em termos de métodos, qualidade, quantidade, tempo, etc.}
  \end{Phonetics}
\end{Entry}

\begin{Entry}{规律}{8,9}{⾒,⼻}
  \begin{Phonetics}{规律}{gui1lv4}[][HSK 4]
    \definition{adj.}{estável; regular; coisas, comportamentos, fenômenos, etc. que ocorrem em um determinado momento}
    \definition{s.}{lei; padrão regular; conexão essencial e recorrente entre as coisas}
  \end{Phonetics}
\end{Entry}

\begin{Entry}{规矩}{8,9}{⾒,⽮}
  \begin{Phonetics}{规矩}{gui1ju5}[][HSK 7-9]
    \definition{adj.}{adequado; bem comportado; bem disciplinado; honesto e correto; de acordo com os padrões ou o senso comum}
    \definition[条,个,项]{s.}{regra; costume; prática estabelecida; certos padrões, regras ou costumes}
  \end{Phonetics}
\end{Entry}

\begin{Entry}{规范}{8,9}{⾒,⾋}
  \begin{Phonetics}{规范}{gui1fan4}[][HSK 3]
    \definition{adj.}{regular; normal; padrão; que atende às especificações; em conformidade com as normas}
    \definition{s.}{norma; padrão; diretriz}
    \definition{v.}{regular; padronizar; tornar conforme as normas}
  \end{Phonetics}
\end{Entry}

\begin{Entry}{规格}{8,10}{⾒,⽊}
  \begin{Phonetics}{规格}{gui1ge2}[][HSK 7-9]
    \definition[种]{s.}{normas; padrões; especificações; padrões de qualidade do produto, como determinados tamanho, peso, precisão, desempenho, etc. | formato; padrão; requisito; geralmente se refere a requisitos ou condições especificados}
  \end{Phonetics}
\end{Entry}

\begin{Entry}{规模}{8,14}{⾒,⽊}
  \begin{Phonetics}{规模}{gui1mo2}[][HSK 4]
    \definition[个,种]{s.}{escala; escopo; dimensões; padrão, forma ou escopo (de um empreendimento, instituição, projeto, movimento, etc.)}
  \end{Phonetics}
\end{Entry}

%%%%%%%%%% 视 %%%%%%%%%%
\subsection*{视}\addcontentsline{loh}{figure}{视}

\begin{Entry}{视}{8}{⾒}
  \begin{Phonetics}{视}{shi4}
    \definition{v.}{olhar para | considerar; olhar para | inspecionar; observar}
  \end{Phonetics}
\end{Entry}

\begin{Entry}{视力}{8,2}{⾒,⼒}
  \begin{Phonetics}{视力}{shi4li4}[][HSK 7-9]
    \definition{s.}{visão; vista; a capacidade do olho de distinguir a forma de um objeto a uma certa distância}
  \synonymref{见识}{jian4shi5}
  \synonymref{眼光}{yan3guang1}
  \synonymref{眼睛}{yan3jing5}
  \end{Phonetics}
\end{Entry}

\begin{Entry}{视为}{8,4}{⾒,⼂}
  \begin{Phonetics}{视为}{shi4wei2}[][HSK 5]
    \definition{v.}{considerar; ver como; considerar como; considerar ser; achar que é}
  \end{Phonetics}
\end{Entry}

\begin{Entry}{视角}{8,7}{⾒,⾓}
  \begin{Phonetics}{视角}{shi4jiao3}[][HSK 7-9]
    \definition{s.}{ângulo de visão; ângulo visual; em física, o ângulo de visão é o ângulo entre dois raios de luz que atingem o olho vindos de extremidades opostas de um objeto | abordagem; ponto de vista; perspectiva; ângulo a partir do qual se analisa um problema; um ângulo para observar e examinar problemas | perspectiva}
  \end{Phonetics}
\end{Entry}

\begin{Entry}{视线}{8,8}{⾒,⽷}
  \begin{Phonetics}{视线}{shi4xian4}[][HSK 7-9]
    \definition[道]{s.}{vista; linha de visão; campo de visão (em topografia); a linha reta imaginária entre o olho e o objeto, ou a área do campo de visão ao olhar para um objeto | atenção; direção e objetivo metafóricos}
  \synonymref{视野}{shi4ye3}
  \end{Phonetics}
\end{Entry}

\begin{Entry}{视觉}{8,9}{⾒,⾒}
  \begin{Phonetics}{视觉}{shi4jue2}[][HSK 7-9]
    \definition{s.}{sentido da visão; a sensação produzida pelos raios de luz, sejam eles provenientes diretamente de uma fonte de luz ou refletidos por um objeto, ao atuarem sobre a retina}
  \antonymref{触觉}{chu4jue2}
  \end{Phonetics}
\end{Entry}

\begin{Entry}{视野}{8,11}{⾒,⾥}
  \begin{Phonetics}{视野}{shi4ye3}[][HSK 7-9]
    \definition{s.}{campo visual; a extensão do espaço vista pelos olhos | campo de visão; domínio metafórico do pensamento ou do conhecimento}
  \synonymref{视线}{shi4xian4}
  \end{Phonetics}
\end{Entry}

\begin{Entry}{视频}{8,13}{⾒,⾴}
  \begin{Phonetics}{视频}{shi4pin2}[][HSK 5]
    \definition[个,段,条]{s.}{vídeo; videoclipe}
  \end{Phonetics}
\end{Entry}

\begin{Entry}{视察}{8,14}{⾒,⼧}
  \begin{Phonetics}{视察}{shi4cha2}[][HSK 7-9]
    \definition{s.}{visitação; revisão; visita de inspeção; o pessoal superior inspeciona o trabalho das organizações subordinadas}
    \definition{v.}{inspecionar | observar; assistir}
  \synonymref{参观}{can1guan1}
  \synonymref{调查}{diao4cha2}
  \synonymref{调研}{diao4yan2}
  \synonymref{观测}{guan1ce4}
  \synonymref{观察}{guan1cha2}
  \synonymref{考察}{kao3cha2}
  \synonymref{考核}{kao3he2}
  \end{Phonetics}
\end{Entry}

%%%%%%%%%% 试 %%%%%%%%%%
\subsection*{试}\addcontentsline{loh}{figure}{试}

\begin{Entry}{试}{8}{⾔}
  \begin{Phonetics}{试}{shi4}[][HSK 1]
    \definition{s.}{teste; exame; avaliação de conhecimentos ou habilidades através de métodos específicos}
    \definition{v.}{tentar; investigar resultados ou verificar a natureza, não se envolver formalmente (em determinada atividade)}
  \end{Phonetics}
\end{Entry}

\begin{Entry}{试用}{8,5}{⾔,⽤}
  \begin{Phonetics}{试用}{shi4yong4}[][HSK 7-9]
    \definition{v.}{fazer um teste; estar em período probatório; antes de usar formalmente, experimentar primeiro para ver se atende aos requisitos}
  \synonymref{尝试}{chang2shi4}
  \end{Phonetics}
\end{Entry}

\begin{Entry}{试用期}{8,5,12}{⾔,⽤,⽉}
  \begin{Phonetics}{试用期}{shi4yong4qi1}[][HSK 7-9]
    \definition{s.}{período probatório; período de experiência; a Lei do Contrato de Trabalho estabeleceu disposições específicas para abordar os problemas de abuso dos períodos de experiência e de períodos de experiência excessivamente longos}
  \end{Phonetics}
\end{Entry}

\begin{Entry}{试行}{8,6}{⾔,⾏}
  \begin{Phonetics}{试行}{shi4xing2}[][HSK 7-9]
    \definition{v.}{experimentar; testar | colocar em uso experimental; realizar uma implementação experimental}
  \end{Phonetics}
\end{Entry}

\begin{Entry}{试卷}{8,8}{⾔,⼙}
  \begin{Phonetics}{试卷}{shi4juan4}[][HSK 4]
    \definition[分,张]{s.}{folha de teste; folha de exame; papel usado para escrever as respostas nos exames}
  \end{Phonetics}
\end{Entry}

\begin{Entry}{试图}{8,8}{⾔,⼞}
  \begin{Phonetics}{试图}{shi4tu2}[][HSK 5]
    \definition{v.}{tentar; pretender, fazer o possível para realizar algo}
  \end{Phonetics}
\end{Entry}

\begin{Entry}{试点}{8,9}{⾔,⽕}
  \begin{Phonetics}{试点}{shi4dian3}[][HSK 6]
    \definition[个]{s.}{local onde um experimento é conduzido; unidade experimental; local de teste; um lugar para pequenos experimentos}
    \definition{v.}{experimentar; fazer experimentos; realizar testes em pontos selecionados; lançar um projeto piloto}
  \end{Phonetics}
\end{Entry}

\begin{Entry}{试验}{8,10}{⾔,⾺}
  \begin{Phonetics}{试验}{shi4yan4}[][HSK 3]
    \definition{v.}{testar; fazer um teste; fazer um experimento; para examinar o efeito ou desempenho de algo, primeiro experimente em um laboratório ou em uma escala menor}
  \end{Phonetics}
\end{Entry}

\begin{Entry}{试探}{8,11}{⾔,⼿}
  \begin{Phonetics}{试探}{shi4tan4}
    \definition{v.}{investigar; explorar; descobrir (uma questão); tentar explorar (um problema ou uma situação)}
  \synonymref{尝试}{chang2shi4}
  \synonymref{摸索}{mo1suo3}
  \synonymref{试验}{shi4yan4}
  \synonymref{探索}{tan4suo3}
  \end{Phonetics}
  \begin{Phonetics}{试探}{shi4tan5}[][HSK 7-9]
    \definition{v.}{sondar; testar (a reação de alguém); utilizar palavras ou ações vagas para obter uma resposta da outra parte, compreendendo assim o seu significado}
  \synonymref{尝试}{chang2shi4}
  \synonymref{摸索}{mo1suo3}
  \synonymref{试验}{shi4yan4}
  \synonymref{探索}{tan4suo3}
  \end{Phonetics}
\end{Entry}

\begin{Entry}{试题}{8,15}{⾔,⾴}
  \begin{Phonetics}{试题}{shi4ti2}[][HSK 3]
    \definition[道]{s.}{questões de um exame}
  \end{Phonetics}
\end{Entry}

%%%%%%%%%% 诗 %%%%%%%%%%
\subsection*{诗}\addcontentsline{loh}{figure}{诗}

\begin{Entry}{诗}{8}{⾔}
  \begin{Phonetics}{诗}{shi1}[][HSK 4]
    \definition[首,句,行]{s.}{poesia; verso; poema; um gênero literário que reflete a vida e expressa emoções por meio de uma linguagem rítmica e rimada}
  \seealsoref{诗经}{shi1jing1}
  \end{Phonetics}
\end{Entry}

\begin{Entry}{诗人}{8,2}{⾔,⼈}
  \begin{Phonetics}{诗人}{shi1ren2}[][HSK 4]
    \definition[个,位,名,些]{s.}{poeta; escritor de poesia}
  \end{Phonetics}
\end{Entry}

\begin{Entry}{诗句}{8,5}{⾔,⼝}
  \begin{Phonetics}{诗句}{shi1ju4}
    \definition[行]{s.}{verso | versículo}
  \end{Phonetics}
\end{Entry}

\begin{Entry}{诗词}{8,7}{⾔,⾔}
  \begin{Phonetics}{诗词}{shi1ci2}
    \definition{s.}{verso}
  \end{Phonetics}
\end{Entry}

\begin{Entry}{诗经}{8,8}{⾔,⽷}
  \begin{Phonetics}{诗经}{shi1jing1}
    \definition*{s.}{Shijing, o Livro das Canções, antiga coleção de poemas chineses e um dos Cinco Clássicos do Confucionismo}
  \end{Phonetics}
\end{Entry}

\begin{Entry}{诗意}{8,13}{⾔,⼼}
  \begin{Phonetics}{诗意}{shi1yi4}
    \definition{adj.}{poético}
    \definition{s.}{poesia}
  \end{Phonetics}
\end{Entry}

\begin{Entry}{诗歌}{8,14}{⾔,⽋}
  \begin{Phonetics}{诗歌}{shi1ge1}[][HSK 5]
    \definition[本,首,段]{s.}{poesia; poemas e canções; refere"-se a todos os tipos de poesia}
  \end{Phonetics}
\end{Entry}

%%%%%%%%%% 诚 %%%%%%%%%%
\subsection*{诚}\addcontentsline{loh}{figure}{诚}

\begin{Entry}{诚}{8}{⾔}
  \begin{Phonetics}{诚}{cheng2}
    \definition{adj.}{sincero; honesto; verdadeiro}
    \definition{adv.}{na verdade; realmente; de fato}
    \definition{s.}{sinceridade; genuinidade; seriedade}
  \end{Phonetics}
\end{Entry}

\begin{Entry}{诚心诚意}{8,4,8,13}{⾔,⼼,⾔,⼼}
  \begin{Phonetics}{诚心诚意}{cheng2xin1-cheng2yi4}[][HSK 7-9]
    \definition{expr.}{sincero e sério; com toda a sinceridade; é uma expressão idiomática chinesa que vem da Biografia de Ma Yuan no Livro da Dinastia Han Posterior | genuíno; sincero}
  \end{Phonetics}
\end{Entry}

\begin{Entry}{诚实}{8,8}{⾔,⼧}
  \begin{Phonetics}{诚实}{cheng2shi5}[][HSK 4]
    \definition{adj.}{honesto; sincero e honesto, não hipócrita}
  \end{Phonetics}
\end{Entry}

\begin{Entry}{诚实地}{8,8,6}{⾔,⼧,⼟}
  \begin{Phonetics}{诚实地}{cheng2shi2 di4}
    \definition{adv.}{honestamente; veridicamente}
  \end{Phonetics}
\end{Entry}

\begin{Entry}{诚信}{8,9}{⾔,⼈}
  \begin{Phonetics}{诚信}{cheng2xin4}[][HSK 4]
    \definition{adj.}{honesto e confiável}
    \definition[种]{s.}{fé; honestidade; padrão e princípio de comportamento: não contar mentiras, prometer aos outros o que eles podem fazer e ter a confiança dos outros}
  \end{Phonetics}
\end{Entry}

\begin{Entry}{诚恳}{8,10}{⾔,⼼}
  \begin{Phonetics}{诚恳}{cheng2ken3}[][HSK 7-9]
    \definition{adj.}{sincero; sério; a atitude é muito real e pé no chão}
  \end{Phonetics}
\end{Entry}

\begin{Entry}{诚挚}{8,10}{⾔,⼿}
  \begin{Phonetics}{诚挚}{cheng2zhi4}[][HSK 7-9]
    \definition{adj.}{sincero; cordial; honesto}
  \end{Phonetics}
\end{Entry}

\begin{Entry}{诚意}{8,13}{⾔,⼼}
  \begin{Phonetics}{诚意}{cheng2yi4}[][HSK 7-9]
    \definition{s.}{boa fé; sinceridade; intenções sinceras}
  \end{Phonetics}
\end{Entry}

%%%%%%%%%% 话 %%%%%%%%%%
\subsection*{话}\addcontentsline{loh}{figure}{话}

\begin{Entry}{话}{8}{⾔}
  \begin{Phonetics}{话}{hua4}[][HSK 1]
    \definition[句,段,番,种]{s.}{palavra; conversa; a voz que expressa os pensamentos quando falada, ou os caracteres que registram essa voz}
    \definition{v.}{falar sobre; falar a respeito}
  \end{Phonetics}
\end{Entry}

\begin{Entry}{话语}{8,9}{⾔,⾔}
  \begin{Phonetics}{话语}{hua4yu3}[][HSK 7-9]
    \definition[句]{s.}{palavras; fala; enunciado; discurso; palavras ditas}
  \end{Phonetics}
\end{Entry}

\begin{Entry}{话费}{8,9}{⾔,⾙}
  \begin{Phonetics}{话费}{hua4fei4}[][HSK 7-9]
    \definition{s.}{uma conta ou taxa telefônica; tarifas de uso do telefone; às vezes se refere simplesmente ao custo de uso do telefone}
  \end{Phonetics}
\end{Entry}

\begin{Entry}{话剧}{8,10}{⾔,⼑}
  \begin{Phonetics}{话剧}{hua4ju4}[][HSK 3]
    \definition[场,幕,部,出,台]{s.}{drama moderno; peça de teatro; peça teatral representada através de diálogos e ações}
  \end{Phonetics}
\end{Entry}

\begin{Entry}{话筒}{8,12}{⾔,⽵}
  \begin{Phonetics}{话筒}{hua4tong3}[][HSK 7-9]
    \definition[个,只]{s.}{microfone; um termo geral para microfones | megafone; um tubo em forma de cone usado para falar alto com muitas pessoas próximas também é chamado de megafone | receptor (telefone); bocal}
  \end{Phonetics}
\end{Entry}

\begin{Entry}{话题}{8,15}{⾔,⾴}
  \begin{Phonetics}{话题}{hua4ti2}[][HSK 3]
    \definition[个,种,项]{s.}{assunto de uma palestra; tópico de uma conversa; o foco da conversa}
  \end{Phonetics}
\end{Entry}

%%%%%%%%%% 诞 %%%%%%%%%%
\subsection*{诞}\addcontentsline{loh}{figure}{诞}

\begin{Entry}{诞}{8}{⾔}
  \begin{Phonetics}{诞}{dan4}
    \definition{adj.}{absurdo; fantástico; irreal; irracional}
    \definition{adv.}{absurdamente; fantasticamente}
    \definition{s.}{aniversário de nascimento | nascimento}
    \definition{v.}{nascer | dar à luz}
  \end{Phonetics}
\end{Entry}

\begin{Entry}{诞生}{8,5}{⾔,⽣}
  \begin{Phonetics}{诞生}{dan4sheng1}[][HSK 6]
    \definition{v.}{nascer; vir a existir; uma pessoa nasce; também significa que algo novo surgiu e tem um impacto positivo na sociedade}
  \end{Phonetics}
\end{Entry}

\begin{Entry}{诞辰}{8,7}{⾔,⾠}
  \begin{Phonetics}{诞辰}{dan4chen2}[][HSK 7-9]
    \definition[周年]{s.}{aniversário (usado principalmente para pessoas respeitadas)}[9月28日是孔子诞辰日。===28 de setembro é o aniversário de Confúcio.]
  \end{Phonetics}
\end{Entry}

%%%%%%%%%% 诟 %%%%%%%%%%
\subsection*{诟}\addcontentsline{loh}{figure}{诟}

\begin{Entry}{诟}{8}{⾔}
  \begin{Phonetics}{诟}{gou4}
    \definition*{s.}{Sobrenome: Gou}
    \definition{s.}{vergonha; humilhação}
    \definition{v.}{insultar; xingar; falar de forma abusiva}
  \end{Phonetics}
\end{Entry}

\begin{Entry}{诟骂}{8,9}{⾔,⾺}
  \begin{Phonetics}{诟骂}{gou4ma4}
    \definition{v.}{abusar verbalmente | insultar | criticar}
  \end{Phonetics}
\end{Entry}

%%%%%%%%%% 询 %%%%%%%%%%
\subsection*{询}\addcontentsline{loh}{figure}{询}

\begin{Entry}{询}{8}{⾔}
  \begin{Phonetics}{询}{xun2}
    \definition{v.}{perguntar; indagar; reunir informações | consultar; buscar conselho}
  \end{Phonetics}
\end{Entry}

\begin{Entry}{询问}{8,6}{⾔,⾨}
  \begin{Phonetics}{询问}{xun2wen4}[][HSK 5]
    \definition{v.}{indagar; perguntar sobre; pedir conselho}
  \end{Phonetics}
\end{Entry}

%%%%%%%%%% 该 %%%%%%%%%%
\subsection*{该}\addcontentsline{loh}{figure}{该}

\begin{Entry}{该}{8}{⾔}
  \begin{Phonetics}{该}{gai1}[][HSK 2,7-9]
    \definition{adj.}{completo; integral; abrangente; inclusivo; o mesmo que 赅}
    \definition{pron.}{isto; aquilo; o referido; o acima mencionado; indica a pessoa ou coisa mencionada acima, equivalente a 此, 这个, etc.}
    \definition{v.}{deveria ser; deveria ser assim | caber a alguém; ser a vez (ou dever) de alguém fazer algo | merecer; servir a alguém de direito; indica que algo deve ser feito | dever | deve; provavelmente irá; muito provavelmente; pode ser razoavelmente ou naturalmente esperado que; expressa uma conclusão lógica ou provável com base na razão ou na experiência}
    \definition{v.aux.}{usado em frases exclamativas, tem a função de reforçar o tom}
  \seealsoref{此}{ci3}
  \seealsoref{赅}{gai1}
  \seealsoref{这个}{zhe4ge5}
  \end{Phonetics}
\end{Entry}

%%%%%%%%%% 详 %%%%%%%%%%
\subsection*{详}\addcontentsline{loh}{figure}{详}

\begin{Entry}{详}{8}{⾔}
  \begin{Phonetics}{详}{xiang2}
    \definition{adj.}{conhecido; reconhecido; saber claramente | detalhado; minucioso; pormenorizado}
    \definition{s.}{detalhes; particularidades}
    \definition{v.}{contar; explicar; elaborar | saber claramente}
  \antonymref{略}{lve4}
  \end{Phonetics}
\end{Entry}

\begin{Entry}{详细}{8,8}{⾔,⽷}
  \begin{Phonetics}{详细}{xiang2xi4}[][HSK 5]
    \definition{adj.}{explícito; detalhado; minucioso; circunstancial; meticuloso}
  \end{Phonetics}
\end{Entry}

%%%%%%%%%% 诧 %%%%%%%%%%
\subsection*{诧}\addcontentsline{loh}{figure}{诧}

\begin{Entry}{诧}{8}{⾔}
  \begin{Phonetics}{诧}{cha4}
    \definition{v.}{ficar surpreso}
  \end{Phonetics}
\end{Entry}

\begin{Entry}{诧异}{8,6}{⾔,⼶}
  \begin{Phonetics}{诧异}{cha4yi4}[][HSK 7-9]
    \definition{v.}{ficar surpreso; ficar espantado}
  \end{Phonetics}
\end{Entry}

%%%%%%%%%% 责 %%%%%%%%%%
\subsection*{责}\addcontentsline{loh}{figure}{责}

\begin{Entry}{责}{8}{⾙}
  \begin{Phonetics}{责}{ze2}
    \definition{s.}{dever; responsabilidade}
    \definition{v.}{exigir; requerer; exigir que algo seja feito ou que atenda a certos padrões | questionar atentamente; chamar alguém para prestar contas; interrogar| reprovar; culpar | punir}
  \end{Phonetics}
\end{Entry}

\begin{Entry}{责任}{8,6}{⾙,⼈}
  \begin{Phonetics}{责任}{ze2ren4}[][HSK 3]
    \definition[个,种,份]{s.}{dever; responsabilidade; de acordo com a profissão, cargo, identidade, etc., as coisas que você deve fazer ou as tarefas que deve assumir | culpa; responsabilidade por uma falha ou erro; não ter feito o que era sua obrigação e, portanto, ser responsável pela falha}
  \end{Phonetics}
\end{Entry}

\begin{Entry}{责怪}{8,8}{⾙,⼼}
  \begin{Phonetics}{责怪}{ze2guai4}
    \definition{v.}{repreender | censurar}
  \end{Phonetics}
\end{Entry}

%%%%%%%%%% 败 %%%%%%%%%%
\subsection*{败}\addcontentsline{loh}{figure}{败}

\begin{Entry}{败}{8}{⾒}
  \begin{Phonetics}{败}{bai4}[][HSK 4]
    \definition{adj.}{ruim; deteriorado; murcho; dilapidado; decadente}
    \definition{v.}{ser derrotado; perder | derrotar; bater | falha | estragar; arruinar | decair; murchar | quebrar; neutralizar; dissipar}
  \antonymref{成}{cheng2}
  \antonymref{胜}{sheng4}
  \end{Phonetics}
\end{Entry}

%%%%%%%%%% 账 %%%%%%%%%%
\subsection*{账}\addcontentsline{loh}{figure}{账}

\begin{Entry}{账}{8}{⾙}
  \begin{Phonetics}{账}{zhang4}[][HSK 6]
    \definition[笔,本]{s.}{conta | livro de contas | dívida; conta | crédito (de dívidas)}
  \end{Phonetics}
\end{Entry}

\begin{Entry}{账户}{8,4}{⾙,⼾}
  \begin{Phonetics}{账户}{zhang4hu4}[][HSK 6]
    \definition[个]{s.}{conta; refere"-se à classificação de vários usos de fundos, fontes e processos de rotatividade no livro contábil}
  \end{Phonetics}
\end{Entry}

%%%%%%%%%% 货 %%%%%%%%%%
\subsection*{货}\addcontentsline{loh}{figure}{货}

\begin{Entry}{货}{8}{⾙}
  \begin{Phonetics}{货}{huo4}[][HSK 4]
    \definition{s.}{dinheiro; moeda | bens; mercadorias; \emph{commodity} | refere"-se a uma pessoa com um certo mau caráter (usado como um insulto) | riqueza; fortuna; um termo geral para dinheiro, joias, tecidos, etc.}
    \definition{v.}{vender}
  \end{Phonetics}
\end{Entry}

\begin{Entry}{货币}{8,4}{⾙,⼱}
  \begin{Phonetics}{货币}{huo4bi4}[][HSK 7-9]
    \definition[种]{s.}{dinheiro; moeda}
  \end{Phonetics}
\end{Entry}

\begin{Entry}{货车}{8,4}{⾙,⾞}
  \begin{Phonetics}{货车}{huo4che1}[][HSK 7-9]
    \definition[辆]{s.}{trem de mercadorias; trem de carga | vagão de carga; caminhão de carga | caminhão; caminhões e veículos de entrega}
  \end{Phonetics}
\end{Entry}

\begin{Entry}{货机}{8,6}{⾙,⽊}
  \begin{Phonetics}{货机}{huo4ji1}
    \definition{s.}{aeronave de carga (ou avião); cargueiro aéreo; aeronaves utilizadas principalmente para o transporte de carga}
  \antonymref{客机}{ke4ji1}
  \end{Phonetics}
\end{Entry}

\begin{Entry}{货运}{8,7}{⾙,⾡}
  \begin{Phonetics}{货运}{huo4yun4}[][HSK 7-9]
    \definition{s.}{transporte de carga | carga; frete | mercadorias transportadas}
  \antonymref{客运}{ke4yun4}
  \end{Phonetics}
\end{Entry}

\begin{Entry}{货物}{8,8}{⾙,⽜}
  \begin{Phonetics}{货物}{huo4wu4}[][HSK 7-9]
    \definition[件,批,些,吨]{s.}{bens; mercadoria; \emph{commodity}; geralmente se refere a itens para venda}
  \end{Phonetics}
\end{Entry}

%%%%%%%%%% 质 %%%%%%%%%%
\subsection*{质}\addcontentsline{loh}{figure}{质}

\begin{Entry}{质}{8}{⾙}
  \begin{Phonetics}{质}{zhi4}
    \definition*{s.}{Sobrenome: Zhi}
    \definition{adj.}{simples; claro; sem adornos}
    \definition{s.}{natureza; caráter; essência; substância | qualidade | matéria; substância | segurança; penhor; garantia}
    \definition{v.}{penhorar | hipotecar | questionar; chamar à responsabilidade; acusar}
  \end{Phonetics}
\end{Entry}

\begin{Entry}{质量}{8,12}{⾙,⾥}
  \begin{Phonetics}{质量}{zhi4liang4}[][HSK 4]
    \definition{s.}{qualidade; o quão bom ou ruim é o produto ou o trabalho | Física: massa}
  \end{Phonetics}
\end{Entry}

%%%%%%%%%% 贩 %%%%%%%%%%
\subsection*{贩}\addcontentsline{loh}{figure}{贩}

\begin{Entry}{贩}{8}{⾙}
  \begin{Phonetics}{贩}{fan4}
    \definition[个]{s.}{comerciante; mascate; negociante; vendedor ambulante}
    \definition{v.}{(comerciantes) comprar para revender}
  \end{Phonetics}
\end{Entry}

\begin{Entry}{贩卖}{8,8}{⾙,⼗}
  \begin{Phonetics}{贩卖}{fan4mai4}[][HSK 7-9]
    \definition{v.}{vender; traficar}
  \end{Phonetics}
\end{Entry}

%%%%%%%%%% 贪 %%%%%%%%%%
\subsection*{贪}\addcontentsline{loh}{figure}{贪}

\begin{Entry}{贪}{8}{⾙}
  \begin{Phonetics}{贪}{tan1}[][HSK 7-9]
    \definition{v.}{apropriar"-se indevidamente; desviar fundos; praticar corrupção; ser corrupto; originalmente, referia"-se ao amor ao dinheiro; posteriormente, passou a ser associado à corrupção | ter um desejo insaciável por; ter um desejo voraz por | cobiçar; ansiar por; ser ganancioso por}
  \end{Phonetics}
\end{Entry}

\begin{Entry}{贪污}{8,6}{⾙,⽔}
  \begin{Phonetics}{贪污}{tan1wu1}[][HSK 7-9]
    \definition{v.}{desviar fundos; obter ilegalmente terras, dinheiro ou propriedade estatal, coletiva ou unitária, abusando do poder ou da posição}
  \synonymref{腐败}{fu3bai4}
  \antonymref{廉洁}{lian2jie2}
  \end{Phonetics}
\end{Entry}

\begin{Entry}{贪玩儿}{8,8,2}{⾙,⽟,⼉}
  \begin{Phonetics}{贪玩儿}{tan1wan2r5}[][HSK 7-9]
    \definition{s.}{brincalhão}
    \definition{v.}{ser brincalhão}
  \end{Phonetics}
\end{Entry}

\begin{Entry}{贪婪}{8,11}{⾙,⼥}
  \begin{Phonetics}{贪婪}{tan1lan2}[][HSK 7-9]
    \definition{adj.}{ganancioso; avarento; descreve uma pessoa ou animal como alguém que nunca está satisfeito | ganancioso; ávido; descreve alguém que nunca está satisfeito com coisas boas como conhecimento e ar puro}
    \definition{s.}{ganância}
  \synonymref{海量}{hai3liang4}
  \antonymref{满足}{man3zu2}
  \end{Phonetics}
\end{Entry}

%%%%%%%%%% 贫 %%%%%%%%%%
\subsection*{贫}\addcontentsline{loh}{figure}{贫}

\begin{Entry}{贫}{8}{⾙}
  \begin{Phonetics}{贫}{pin2}
    \definition{adj.}{pobre; empobrecido | inadequado; deficiente; insuficiente | tagarela; loquaz; falante; chato e irritante}
  \end{Phonetics}
\end{Entry}

\begin{Entry}{贫民窟}{8,5,13}{⾙,⽒,⽳}
  \begin{Phonetics}{贫民窟}{pin2min2ku1}
    \definition{s.}{favela}
  \end{Phonetics}
\end{Entry}

\begin{Entry}{贫困}{8,7}{⾙,⼞}
  \begin{Phonetics}{贫困}{pin2kun4}[][HSK 6]
    \definition{adj.}{pobre; indigente; necessitado; empobrecido; assolado pela pobreza; em circunstâncias difíceis}
  \end{Phonetics}
\end{Entry}

\begin{Entry}{贫穷}{8,7}{⾙,⽳}
  \begin{Phonetics}{贫穷}{pin2qiong2}[][HSK 7-9]
    \definition{adj.}{pobre; empobrecido; falta de meios de produção e de meios de subsistência}
  \end{Phonetics}
\end{Entry}

\begin{Entry}{贫富}{8,12}{⾙,⼧}
  \begin{Phonetics}{贫富}{pin2 fu4}[][HSK 7-9]
    \definition{s.}{pobreza e riqueza; pobres e ricos}
  \end{Phonetics}
\end{Entry}

%%%%%%%%%% 贬 %%%%%%%%%%
\subsection*{贬}\addcontentsline{loh}{figure}{贬}

\begin{Entry}{贬}{8}{⾙}
  \begin{Phonetics}{贬}{bian3}
    \definition{adj.}{depreciativo; derrogativo; rebaixante}
    \definition{v.}{(nos tempos antigos) rebaixar de posição; (nos tempos modernos) diminuir de valor | reduzir valor; desvalorizar | censurar; menosprezar; depreciar; dar uma avaliação ruim | degradar; rebaixar; relegar}
  \end{Phonetics}
\end{Entry}

\begin{Entry}{贬值}{8,10}{⾙,⼈}
  \begin{Phonetics}{贬值}{bian3zhi2}[][HSK 7-9]
    \definition{v.}{depreciar; tornar"-se desvalorizado; refere"-se à diminuição do poder de compra do dinheiro | depreciar; geralmente se refere à diminuição do valor de algo | desvalorizar; reduzir o teor de ouro da moeda de um país ou reduzir a taxa de câmbio da moeda de um país em relação às moedas estrangeiras}
  \end{Phonetics}
\end{Entry}

%%%%%%%%%% 购 %%%%%%%%%%
\subsection*{购}\addcontentsline{loh}{figure}{购}

\begin{Entry}{购}{8}{⾙}
  \begin{Phonetics}{购}{gou4}[][HSK 7-9]
    \definition{v.}{comprar}
  \end{Phonetics}
\end{Entry}

\begin{Entry}{购买}{8,6}{⾙,⼄}
  \begin{Phonetics}{购买}{gou4mai3}[][HSK 4]
    \definition{v.}{comprar; adquirir; usar dinheiro para obter itens}
  \end{Phonetics}
\end{Entry}

\begin{Entry}{购物}{8,8}{⾙,⽜}
  \begin{Phonetics}{购物}{gou4wu4}[][HSK 4]
    \definition{s.}{compras; itens comprados; \emph{shopping}}
    \definition{v.}{ir às compras; fazer compras}
  \end{Phonetics}
\end{Entry}

%%%%%%%%%% 贯 %%%%%%%%%%
\subsection*{贯}\addcontentsline{loh}{figure}{贯}

\begin{Entry}{贯}{8}{⾙}
  \begin{Phonetics}{贯}{guan4}
    \definition*{s.}{Sobrenome: Guan}
    \definition{clas.}{uma sequência de 1.000 em dinheiro; antigamente, o dinheiro era amarrado com cordas, e cada mil moedas era uma corda.}
    \definition{s.}{lugar nativo; local de nascimento; lugar do lar ancestral; lugar onde gerações viveram | Literário: exemplo; instância; caso; precedente | Arcaico: guan (uma corda de 1.000 moedas de cobre); corda para amarrar dinheiro nos tempos antigos}
    \definition{v.}{passar através de; perfurar; enfiar; penetrar | estar ligados entre si; seguir em linha contínua; estar conectado | Literário: comparecer}
  \end{Phonetics}
\end{Entry}

\begin{Entry}{贯彻}{8,7}{⾙,⼻}
  \begin{Phonetics}{贯彻}{guan4che4}[][HSK 7-9]
    \definition{v.}{executar; implementar; pôr em prática; realizar ou incorporar completamente (diretrizes, políticas, espírito, etc.)}
  \end{Phonetics}
\end{Entry}

\begin{Entry}{贯穿}{8,9}{⾙,⽳}
  \begin{Phonetics}{贯穿}{guan4chuan1}[][HSK 7-9]
    \definition{v.}{cruzar; conectar; penetrar; correr através; passar através | permear; estar cheio de}
  \end{Phonetics}
\end{Entry}

\begin{Entry}{贯通}{8,10}{⾙,⾡}
  \begin{Phonetics}{贯通}{guan4tong1}[][HSK 7-9]
    \definition{v.}{ter um conhecimento profundo de; ser bem versado (em); (acadêmico, ideológico, etc.) ter compreensão completa | ligar; encadear}
  \end{Phonetics}
\end{Entry}

%%%%%%%%%% 转 %%%%%%%%%%
\subsection*{转}\addcontentsline{loh}{figure}{转}

\begin{Entry}{转}{8}{⾞}
  \begin{Phonetics}{转}{zhuai3}
  \end{Phonetics}
  \begin{Phonetics}{转}{zhuan3}[][HSK 3]
    \definition{v.}{mudar; deslocar; transferir; virar; mudar de direção, posição, situação, circunstâncias, etc. | transmitir; transferir; passar adiante}
  \end{Phonetics}
  \begin{Phonetics}{转}{zhuan4}[][HSK 6]
    \definition{clas.}{usado para rotações (por minuto, por segundo, etc.): RPM}
    \definition{v.}{girar; rodar; revolver; movimento em torno de um centro | passear; dar uma volta}
  \end{Phonetics}
\end{Entry}

\begin{Entry}{转化}{8,4}{⾞,⼔}
  \begin{Phonetics}{转化}{zhuan3hua4}[][HSK 5]
    \definition{v.}{mudar; transformar | inverter; converter}
  \end{Phonetics}
\end{Entry}

\begin{Entry}{转让}{8,5}{⾞,⾔}
  \begin{Phonetics}{转让}{zhuan3rang4}[][HSK 5]
    \definition{v.}{ceder; fazer a entrega; transferir a posse de; ceder seus bens ou direitos a outra pessoa}
  \end{Phonetics}
\end{Entry}

\begin{Entry}{转产}{8,6}{⾞,⼇}
  \begin{Phonetics}{转产}{zhuan3chan3}
    \definition{v.}{mudar a produção | mudar para novos produtos}
  \end{Phonetics}
\end{Entry}

\begin{Entry}{转动}{8,6}{⾞,⼒}
  \begin{Phonetics}{转动}{zhuan3dong4}[][HSK 4]
    \definition{v.}{girar; rodar; dar voltas; torcer | dar a volta em algo}
  \end{Phonetics}
  \begin{Phonetics}{转动}{zhuan4dong4}[][HSK 6]
    \definition{s.}{tambor; roda}
    \definition{v.}{girar; correr; rolar; revolver; rotacionar; torcer}
  \end{Phonetics}
\end{Entry}

\begin{Entry}{转向}{8,6}{⾞,⼝}
  \begin{Phonetics}{转向}{zhuan3/xiang4}[][HSK 5]
    \definition{v.+compl.}{desviar; desviar"-se; mudar a direção do avanço | mudar a posição política de alguém | mudar de direção; virar"-se para (a outra parte)}
  \end{Phonetics}
  \begin{Phonetics}{转向}{zhuan4/xiang4}
    \definition{v.+compl.}{perder"-se; perder o rumo; não consiguir distinguir a direção; estar perdido}
  \end{Phonetics}
\end{Entry}

\begin{Entry}{转告}{8,7}{⾞,⼝}
  \begin{Phonetics}{转告}{zhuan3gao4}[][HSK 4]
    \definition{v.}{passar adiante; comunicar; transmitir; ser instruído a dizer a outra parte o que uma pessoa diz, o que está acontecendo, etc.}
  \end{Phonetics}
\end{Entry}

\begin{Entry}{转身}{8,7}{⾞,⾝}
  \begin{Phonetics}{转身}{zhuan3 shen1}[][HSK 4]
    \definition{adv.}{em um instante; em um piscar de olhos}
    \definition{v.}{dar a volta; dar meia"-volta; dar a volta por cima | virar; girar; refere"-se a uma mudança de direção, localização, natureza, etc.}
  \end{Phonetics}
\end{Entry}

\begin{Entry}{转变}{8,8}{⾞,⼜}
  \begin{Phonetics}{转变}{zhuan3bian4}[][HSK 3]
    \definition{v.}{mudar; converter; transformar}
  \end{Phonetics}
\end{Entry}

\begin{Entry}{转念}{8,8}{⾞,⼼}
  \begin{Phonetics}{转念}{zhuan3nian4}
    \definition{v.}{ter dúvidas sobre algo | pensar melhor}
  \end{Phonetics}
\end{Entry}

\begin{Entry}{转账}{8,8}{⾞,⾙}
  \begin{Phonetics}{转账}{zhuan3/zhang4}
    \definition{v.+compl.}{transferir entre contas | trazer à frente | incluir uma soma de dinheiro do balanço anterior no seguinte}
  \end{Phonetics}
\end{Entry}

\begin{Entry}{转弯}{8,9}{⾞,⼸}
  \begin{Phonetics}{转弯}{zhuan3/wan1}[][HSK 4]
    \definition{s.}{esquina; curva}[小心急转弯。===Tenha cuidado em curvas fechadas.]
    \definition{v.+compl.}{rodar; desviar; virar uma esquina; fazer uma curva; fazer uma curva de 180º}
  \end{Phonetics}
\end{Entry}

\begin{Entry}{转换}{8,10}{⾞,⼿}
  \begin{Phonetics}{转换}{zhuan3huan4}[][HSK 5]
    \definition{v.}{mudar; trocar; converter; transformar; alterar}
  \end{Phonetics}
\end{Entry}

\begin{Entry}{转递}{8,10}{⾞,⾡}
  \begin{Phonetics}{转递}{zhuan3di4}
    \definition{v.}{passar | retransmitir}
  \end{Phonetics}
\end{Entry}

\begin{Entry}{转悠}{8,11}{⾞,⼼}
  \begin{Phonetics}{转悠}{zhuan4you5}
    \definition{v.}{aparecer repetidamente | rolar | passear por aí}
  \end{Phonetics}
\end{Entry}

\begin{Entry}{转移}{8,11}{⾞,⽲}
  \begin{Phonetics}{转移}{zhuan3yi2}[][HSK 4]
    \definition{v.}{deslocar; desviar; transferir; redirecionar; reposicionar; reorientar | mudar; transformar}
  \end{Phonetics}
\end{Entry}

\begin{Entry}{转游}{8,12}{⾞,⽔}
  \begin{Phonetics}{转游}{zhuan4you5}
  \seealsoref{转悠}{zhuan4you5}
  \end{Phonetics}
\end{Entry}

%%%%%%%%%% 轮 %%%%%%%%%%
\subsection*{轮}\addcontentsline{loh}{figure}{轮}

\begin{Entry}{轮}{8}{⾞}
  \begin{Phonetics}{轮}{lun2}[][HSK 4]
    \definition{clas.}{usado para sol vermelho, lua brilhante, etc. | usado para rodadas | doze anos de idade (os doze ramos terrestres são usados para lembrar o gênero humano e cada doze anos de idade é um ciclo)}
    \definition{s.}{roda | anel; disco; objeto semelhante a uma roda | navio a vapor; barco a vapor}
    \definition{v.}{revezar; substituir um ao outro em sequência (para fazer algo)}
  \end{Phonetics}
\end{Entry}

\begin{Entry}{轮子}{8,3}{⾞,⼦}
  \begin{Phonetics}{轮子}{lun2zi5}[][HSK 4]
    \definition[个,只]{s.}{roda; peças circulares de veículos ou máquinas com capacidade de rotação}
  \end{Phonetics}
\end{Entry}

\begin{Entry}{轮回}{8,6}{⾞,⼞}
  \begin{Phonetics}{轮回}{lun2hui2}
    \definition[个]{s.}{reencarnação (Budismo) | ciclo}
    \definition{v.}{reencarnar}
  \end{Phonetics}
\end{Entry}

\begin{Entry}{轮胎}{8,9}{⾞,⾁}
  \begin{Phonetics}{轮胎}{lun2tai1}[][HSK 7-9]
    \definition{s.}{pneu; as rodas de carros, bicicletas, etc., possuem aros de borracha espessos na parte externa}
  \end{Phonetics}
\end{Entry}

\begin{Entry}{轮换}{8,10}{⾞,⼿}
  \begin{Phonetics}{轮换}{lun2huan4}[][HSK 7-9]
    \definition{v.}{rotacionar; revezar"-se | alternar; girar}
  \end{Phonetics}
\end{Entry}

\begin{Entry}{轮流}{8,10}{⾞,⽔}
  \begin{Phonetics}{轮流}{lun2liu2}[][HSK 7-9]
    \definition{v.}{alternar turnos; fazer algo em turnos; um após o outro, em ordem}
  \end{Phonetics}
\end{Entry}

\begin{Entry}{轮船}{8,11}{⾞,⾈}
  \begin{Phonetics}{轮船}{lun2chuan2}[][HSK 4]
    \definition[艘,班]{s.}{vapor; navio a vapor; barco a vapor}
  \end{Phonetics}
\end{Entry}

\begin{Entry}{轮椅}{8,12}{⾞,⽊}
  \begin{Phonetics}{轮椅}{lun2yi3}[][HSK 4]
    \definition{s.}{cadeira de rodas; dispositivo de assento especialmente projetado com rodas para pessoas com dificuldade de locomoção, que pode ser acionado por um disco de roda ou manivela operados manualmente}
  \end{Phonetics}
\end{Entry}

\begin{Entry}{轮廓}{8,13}{⾞,⼴}
  \begin{Phonetics}{轮廓}{lun2kuo4}[][HSK 7-9]
    \definition[个,点]{s.}{esboço; contorno; rascunho; as linhas que formam a borda externa de uma figura, objeto ou corpo humano | visão geral; situação geral; visão geral da situação}
  \end{Phonetics}
\end{Entry}

%%%%%%%%%% 软 %%%%%%%%%%
\subsection*{软}\addcontentsline{loh}{figure}{软}

\begin{Entry}{软}{8}{⾞}
  \begin{Phonetics}{软}{ruan3}[][HSK 5]
    \definition*{s.}{Sobrenome: Ruan}
    \definition{adj.}{macio; flexível; maleável; maleável | suave; brando; delicado | fraco; débil | de baixa qualidade, capacidade, etc. | facilmente movido (ou influenciado) | de maneira suave (ou gentil) | indulgente; tolerante | maleável; flexível | fácil de se emocionar ou abalar}
  \antonymref{硬}{ying4}
  \end{Phonetics}
\end{Entry}

\begin{Entry}{软件}{8,6}{⾞,⼈}
  \begin{Phonetics}{软件}{ruan3jian4}[][HSK 5]
    \definition[款,个]{s.}{\emph{software}; programas de computador, procedimentos, regras e quaisquer arquivos, documentos e dados relacionados à operação do sistema de computador}
  \end{Phonetics}
\end{Entry}

\begin{Entry}{软实力}{8,8,2}{⾞,⼧,⼒}
  \begin{Phonetics}{软实力}{ruan3shi2li4}[][HSK 7-9]
    \definition{s.}{\emph{soft power} (nas relações internacionais)}
  \end{Phonetics}
\end{Entry}

\begin{Entry}{软弱}{8,10}{⾞,⼸}
  \begin{Phonetics}{软弱}{ruan3ruo4}[][HSK 7-9]
    \definition{adj.}{fraco; descreve o eu interior, a personalidade, etc. de alguém como fraco ou sem força | fraco; débil; flácido; descreve a falta de força física}
  \end{Phonetics}
\end{Entry}

%%%%%%%%%% 轰 %%%%%%%%%%
\subsection*{轰}\addcontentsline{loh}{figure}{轰}

\begin{Entry}{轰}{8}{⾞}
  \begin{Phonetics}{轰}{hong1}[][HSK 7-9]
    \definition{interj.}{Onomatopéia: ``Bum!''; ``Bang!''; refere"-se aos ruídos altos feitos por trovões, fogo de artilharia, etc.}
    \definition{v.}{retumbar; trovejar; bombardear; explodir | espantar; expulsar}
  \end{Phonetics}
\end{Entry}

\begin{Entry}{轰动}{8,6}{⾞,⼒}
  \begin{Phonetics}{轰动}{hong1dong4}[][HSK 7-9]
    \definition{v.}{causar (criar) uma sensação; fazer um rebuliço; criar um rebuliço}
  \end{Phonetics}
\end{Entry}

\begin{Entry}{轰鸣}{8,8}{⾞,⿃}
  \begin{Phonetics}{轰鸣}{hong1ming2}
    \definition{s.}{trovão; rugido}
    \definition{v.}{rosnar; rugir; trovejar}
  \end{Phonetics}
\end{Entry}

\begin{Entry}{轰炸}{8,9}{⾞,⽕}
  \begin{Phonetics}{轰炸}{hong1zha4}[][HSK 7-9]
    \definition{v.}{bombardear; lançar bombas de aeronaves sobre vários alvos no solo ou na água}
  \end{Phonetics}
\end{Entry}

\begin{Entry}{轰炸机}{8,9,6}{⾞,⽕,⽊}
  \begin{Phonetics}{轰炸机}{hong1zha4ji1}
    \definition{s.}{bombardeiro (aeronave)}
  \end{Phonetics}
\end{Entry}

%%%%%%%%%% 迫 %%%%%%%%%%
\subsection*{迫}\addcontentsline{loh}{figure}{迫}

\begin{Entry}{迫}{8}{⾡}
  \begin{Phonetics}{迫}{po4}
    \definition{adj.}{urgente; premente}
    \definition{s.}{morteiro; artilharia}
    \definition{v.}{compelir; forçar; pressionar | aproximar"-se; ir em direção a (ou perto de)}
  \end{Phonetics}
\end{Entry}

\begin{Entry}{迫不及待}{8,4,3,9}{⾡,⼀,⼃,⼻}
  \begin{Phonetics}{迫不及待}{po4bu4ji2dai4}[][HSK 7-9]
    \definition{expr.}{``Mal posso esperar!''; incapaz de se conter; ansioso para fazer algo sem demora; urgente demais para esperar mais}
  \end{Phonetics}
\end{Entry}

\begin{Entry}{迫切}{8,4}{⾡,⼑}
  \begin{Phonetics}{迫切}{po4qie4}[][HSK 4]
    \definition{adj.}{urgente; premente; muito ansiosamente, a ponto de ser difícil esperar}
  \end{Phonetics}
\end{Entry}

\begin{Entry}{迫使}{8,8}{⾡,⼈}
  \begin{Phonetics}{迫使}{po4shi3}[][HSK 7-9]
    \definition{v.}{forçar; compelir; obrigar; impor; coagir (por meio de poder político ou econômico)}
  \end{Phonetics}
\end{Entry}

\begin{Entry}{迫害}{8,10}{⾡,⼧}
  \begin{Phonetics}{迫害}{po4hai4}[][HSK 7-9]
    \definition{v.}{perseguir; oprimir cruelmente (frequentemente referindo"-se à opressão política)}
  \end{Phonetics}
\end{Entry}

%%%%%%%%%% 迭 %%%%%%%%%%
\subsection*{迭}\addcontentsline{loh}{figure}{迭}

\begin{Entry}{迭}{8}{⾡}
  \begin{Phonetics}{迭}{die2}
    \definition*{s.}{Sobrenome: Die}
    \definition{adv.}{repetidamente; de novo e de novo | a tempo para}
    \definition{v.}{alternar; mudar; revezar"-se; substituir}
  \end{Phonetics}
\end{Entry}

\begin{Entry}{迭起}{8,10}{⾡,⾛}
  \begin{Phonetics}{迭起}{die2qi3}[][HSK 7-9]
    \definition{v.}{ocorrer repetidamente; acontecer com frequência | surgir repetidamente}
  \end{Phonetics}
\end{Entry}

%%%%%%%%%% 郁 %%%%%%%%%%
\subsection*{郁}\addcontentsline{loh}{figure}{郁}

\begin{Entry}{郁}{8}{⾢}
  \begin{Phonetics}{郁}{yu4}
    \definition*{s.}{Sobrenome: Yu}
    \definition{adj.}{fortemente perfumado | luxuriante; exuberante | sombrio; deprimido}
  \end{Phonetics}
\end{Entry}

\begin{Entry}{郁郁葱葱}{8,8,12,12}{⾢,⾢,⾋,⾋}
  \begin{Phonetics}{郁郁葱葱}{yu4yu4cong1cong1}
    \definition{adj.}{exuberante e verde}
    \definition{expr.}{verdejante e exuberante; uma profusão selvagem de vegetação; luxuriantemente verde; ela cresce mais verde e mais fresca}
  \end{Phonetics}
\end{Entry}

%%%%%%%%%% 郊 %%%%%%%%%%
\subsection*{郊}\addcontentsline{loh}{figure}{郊}

\begin{Entry}{郊}{8}{⾢}
  \begin{Phonetics}{郊}{jiao1}
    \definition*{s.}{Sobrenome: Jiao}
    \definition{s.}{subúrbios; periferias; áreas ao redor da cidade}
  \end{Phonetics}
\end{Entry}

\begin{Entry}{郊区}{8,4}{⾢,⼖}
  \begin{Phonetics}{郊区}{jiao1qu1}[][HSK 5]
    \definition[个,片,块]{s.}{subúrbios; arredores; periferia; área ao redor da cidade que está administrativamente sob a jurisdição da cidade}
  \end{Phonetics}
\end{Entry}

\begin{Entry}{郊外}{8,5}{⾢,⼣}
  \begin{Phonetics}{郊外}{jiao1wai4}[][HSK 7-9]
    \definition{s.}{subúrbio; periferia; a zona rural ao redor de uma cidade; a área fora da cidade (referindo"-se a uma cidade específica)}
  \end{Phonetics}
\end{Entry}

\begin{Entry}{郊游}{8,12}{⾢,⽔}
  \begin{Phonetics}{郊游}{jiao1you2}[][HSK 7-9]
    \definition{s.}{passeio; excursão}
  \end{Phonetics}
\end{Entry}

%%%%%%%%%% 采 %%%%%%%%%%
\subsection*{采}\addcontentsline{loh}{figure}{采}

\begin{Entry}{采}{8}{⾤}
  \begin{Phonetics}{采}{cai3}[][HSK 7-9]
    \definition*{s.}{Sobrenome: Cai}
    \definition{s.}{espírito; tez; cor e expressão facial | cores}
    \definition{v.}{escolher; arrancar; reunir; colher (flores, folhas, frutas) | minerar; extrair | reunir; coletar | adotar; pegar; selecionar}
  \end{Phonetics}
  \begin{Phonetics}{采}{cai4}
    \definition{s.}{atribuição a um nobre feudal; a terra (incluindo os escravos que cultivavam a terra) concedida pelos antigos príncipes aos nobres; também chamada de feudo}
  \end{Phonetics}
\end{Entry}

\begin{Entry}{采用}{8,5}{⾤,⽤}
  \begin{Phonetics}{采用}{cai3yong4}[][HSK 3]
    \definition{v.}{selecionar e usar; adotar; considerar adequado e utilizar}
  \end{Phonetics}
\end{Entry}

\begin{Entry}{采访}{8,6}{⾤,⾔}
  \begin{Phonetics}{采访}{cai3fang3}[][HSK 4]
    \definition{s.}{cobertura; entrevista; coleta de notícias; entrevistas, pesquisas, gravações de áudio e vídeo, etc., com o objetivo de coletar os materiais necessários}
    \definition{v.}{cobrir; entrevistar; reunir novas informações}
  \end{Phonetics}
\end{Entry}

\begin{Entry}{采纳}{8,7}{⾤,⽷}
  \begin{Phonetics}{采纳}{cai3na4}[][HSK 6]
    \definition{v.}{aceitar; adotar; tomar (opiniões, sugestões, solicitações, etc.)}
  \end{Phonetics}
\end{Entry}

\begin{Entry}{采取}{8,8}{⾤,⼜}
  \begin{Phonetics}{采取}{cai3qu3}[][HSK 3]
    \definition{v.}{adotar; escolha da implementação (diretrizes, políticas, métodos, ações, etc.) | reunir; coletar; tomar; assumir}
  \end{Phonetics}
\end{Entry}

\begin{Entry}{采矿}{8,8}{⾤,⽯}
  \begin{Phonetics}{采矿}{cai3/kuang4}[][HSK 7-9]
    \definition{s.}{mina}
    \definition{v.+compl.}{minerar; extrair minerais}
  \end{Phonetics}
\end{Entry}

\begin{Entry}{采购}{8,8}{⾤,⾙}
  \begin{Phonetics}{采购}{cai3gou4}[][HSK 5]
    \definition[名]{s.}{comprador; responsável pelas compras}
    \definition{v.}{adquirir; comprar; fazer compras para uma organização; fazer compras para uma empresa}
  \end{Phonetics}
\end{Entry}

\begin{Entry}{采集}{8,12}{⾤,⾫}
  \begin{Phonetics}{采集}{cai3ji2}[][HSK 7-9]
    \definition{v.}{reunir; coletar}
  \end{Phonetics}
\end{Entry}

%%%%%%%%%% 金 %%%%%%%%%%
\subsection*{金}\addcontentsline{loh}{figure}{金}

\begin{Entry}{金}{8}{⾦}[Kangxi 167]
  \begin{Phonetics}{金}{jin1}[][HSK 3]
    \definition*{s.}{Dinastia Jin (1115--1234) | Sobrenome: Jin}
    \definition{adj.}{dourado | altamente respeitado; precioso. metáfora de nobreza}
    \definition[锭,块]{s.}{ouro | metal | dinheiro | instrumento antigo de percussão de metal}
  \end{Phonetics}
\end{Entry}

\begin{Entry}{金子}{8,3}{⾦,⼦}
  \begin{Phonetics}{金子}{jin1zi5}[][HSK 7-9]
    \definition{s.}{ouro; elemento metálico, símbolo Au (aurum) amarelo-avermelhado, macio, dúctil, quimicamente estável é um metal precioso, usado para fabricar dinheiro, ornamentos etc.}
  \end{Phonetics}
\end{Entry}

\begin{Entry}{金刚石}{8,6,5}{⾦,⼑,⽯}
  \begin{Phonetics}{金刚石}{jin1gang1shi2}
    \definition{s.}{diamante, também chamado de 钻石}[金刚石比什么金属都硬。===O diamante é mais duro que qualquer metal.]
  \seealsoref{钻石}{zuan4shi2}
  \end{Phonetics}
\end{Entry}

\begin{Entry}{金字塔}{8,6,12}{⾦,⼦,⼟}
  \begin{Phonetics}{金字塔}{jin1zi4ta3}[][HSK 7-9]
    \definition[座]{s.}{pirâmide (edifício ou estrutura); as pirâmides egípcias, um tipo de estrutura utilizada por alguns povos antigos, são pirâmides de pedra com três ou mais lados, que, à distância, lembram o caractere chinês 金 (ouro); elas serviam de túmulo para antigos imperadores}
  \end{Phonetics}
\end{Entry}

\begin{Entry}{金色}{8,6}{⾦,⾊}
  \begin{Phonetics}{金色}{jin1 se4}
    \definition{s.}{cor ouro; dourado}
  \end{Phonetics}
\end{Entry}

\begin{Entry}{金钱}{8,10}{⾦,⾦}
  \begin{Phonetics}{金钱}{jin1qian2}[][HSK 6]
    \definition[沓,笔,堆]{s.}{dinheiro; moeda}
  \end{Phonetics}
\end{Entry}

\begin{Entry}{金属}{8,12}{⾦,⼫}
  \begin{Phonetics}{金属}{jin1shu3}[][HSK 7-9]
    \definition[种,块,片]{s.}{metal; um tipo de substância com superfície relativamente lisa e brilhante, porém opaca, capaz de conduzir eletricidade e calor}
  \end{Phonetics}
\end{Entry}

\begin{Entry}{金牌}{8,12}{⾦,⽚}
  \begin{Phonetics}{金牌}{jin1pai2}[][HSK 3]
    \definition[枚]{s.}{medalha de ouro; refere"-se à medalha conquistada pelo campeão em uma competição esportiva | ficha de ouro; placa de ouro usada como símbolo}
  \end{Phonetics}
\end{Entry}

\begin{Entry}{金额}{8,15}{⾦,⾴}
  \begin{Phonetics}{金额}{jin1'e2}[][HSK 6]
    \definition[份,笔]{s.}{quantidade de dinheiro; soma de dinheiro}
  \end{Phonetics}
\end{Entry}

\begin{Entry}{金融}{8,16}{⾦,⿀}
  \begin{Phonetics}{金融}{jin1rong2}[][HSK 6]
    \definition{s.}{finanças; serviços bancários; refere"-se a atividades econômicas como a emissão, circulação e retirada de moeda, a concessão e retirada de empréstimos, o depósito e retirada de depósitos e transações de câmbio}
  \end{Phonetics}
\end{Entry}

%%%%%%%%%% 钓 %%%%%%%%%%
\subsection*{钓}\addcontentsline{loh}{figure}{钓}

\begin{Entry}{钓}{8}{⾦}
  \begin{Phonetics}{钓}{diao4}
    \definition{v.}{pescar com anzol e linha | buscar (fama e ganho pessoal) | fisgar; defraudar por meios desleais}
  \end{Phonetics}
\end{Entry}

\begin{Entry}{钓鱼}{8,8}{⾦,⿂}
  \begin{Phonetics}{钓鱼}{diao4yu2}[][HSK 7-9]
    \definition{v.}{pescar; ir pescar; atividade de captura de peixes com equipamentos de pesca na beira da água, que é uma forma de lazer e entretenimento | Figurativo: aprisionar; Internet: 钓鱼 significa que alguém publica deliberadamente algo que pode causar controvérsia, raiva ou outras emoções fortes, a fim de atrair pessoas para responder e discutir}
  \end{Phonetics}
\end{Entry}

%%%%%%%%%% 闸 %%%%%%%%%%
\subsection*{闸}\addcontentsline{loh}{figure}{闸}

\begin{Entry}{闸}{8}{⾨}
  \begin{Phonetics}{闸}{zha2}
    \definition[个,道]{s.}{comporta; comporta | freio | (coloquial) interruptor}
    \definition{v.}{represar um córrego, rio, etc. | represar a água; parar a água}
  \end{Phonetics}
\end{Entry}

\begin{Entry}{闸门}{8,3}{⾨,⾨}
  \begin{Phonetics}{闸门}{zha2men2}
    \definition{s.}{eclusa | comporta}
  \end{Phonetics}
\end{Entry}

%%%%%%%%%% 闹 %%%%%%%%%%
\subsection*{闹}\addcontentsline{loh}{figure}{闹}

\begin{Entry}{闹}{8}{⾾}
  \begin{Phonetics}{闹}{nao4}[][HSK 4]
    \definition{adj.}{barulhento}
    \definition{v.}{fazer barulho; provocar problemas | dar vazão (à sua raiva, ressentimento, etc.) | sofrer de; ser incomodado por; ocorrer (um desastre ou coisa ruim) | fazer;  entrar em ação | agitar; perturbar | brincar; fazer bagunça}
  \end{Phonetics}
\end{Entry}

\begin{Entry}{闹事}{8,8}{⾾,⼅}
  \begin{Phonetics}{闹事}{nao4/shi4}[][HSK 7-9]
    \definition{v.+compl.}{criar perturbação; causar problemas}
  \end{Phonetics}
\end{Entry}

\begin{Entry}{闹钟}{8,9}{⾾,⾦}
  \begin{Phonetics}{闹钟}{nao4zhong1}[][HSK 4]
    \definition[个,台,只,款]{s.}{despertador; relógios capazes de tocar alarmes em horários predeterminados}
  \end{Phonetics}
\end{Entry}

\begin{Entry}{闹着玩儿}{8,11,8,2}{⾾,⽬,⽟,⼉}
  \begin{Phonetics}{闹着玩儿}{nao4zhe5wan2r5}[][HSK 7-9]
    \definition{expr.}{``Estou brincando.''; piada; brincar; fazer algo por diversão; estar brincando}
  \end{Phonetics}
\end{Entry}

%%%%%%%%%% 陌 %%%%%%%%%%
\subsection*{陌}\addcontentsline{loh}{figure}{陌}

\begin{Entry}{陌}{8}{⾩}
  \begin{Phonetics}{陌}{mo4}
    \definition[个]{s.}{Literário: caminho entre campos (indo de leste a oeste); trilhas entre campos que correm de leste a oeste; geralmente se refere a estradas nos campos | Obsoleto: estrada}
  \end{Phonetics}
\end{Entry}

\begin{Entry}{陌生}{8,5}{⾩,⽣}
  \begin{Phonetics}{陌生}{mo4sheng1}[][HSK 7-9]
    \definition{adj.}{estranho; desconhecido; inexperiente; indica algo desconhecido ou não familiar e é frequentemente usado como um atributo ou predicado}
  \end{Phonetics}
\end{Entry}

%%%%%%%%%% 降 %%%%%%%%%%
\subsection*{降}\addcontentsline{loh}{figure}{降}

\begin{Entry}{降}{8}{⾩}
  \begin{Phonetics}{降}{jiang4}[][HSK 4]
    \definition*{s.}{Sobrenome: Jiang}
    \definition{v.}{cair; descer; quedar"-se | diminuir; reduzir; cair; abaixar | nascer}
  \antonymref{升}{sheng1}
  \end{Phonetics}
\end{Entry}

\begin{Entry}{降价}{8,6}{⾩,⼈}
  \begin{Phonetics}{降价}{jiang4 jia4}[][HSK 4]
    \definition{v.}{ficar mais barato; cortar o preço; reduzir o preço}
  \end{Phonetics}
\end{Entry}

\begin{Entry}{降低}{8,7}{⾩,⼈}
  \begin{Phonetics}{降低}{jiang4di1}[][HSK 4]
    \definition{v.}{reduzir; cortar; diminuir; rebaixar; cair; abaixar}
  \end{Phonetics}
\end{Entry}

\begin{Entry}{降临}{8,9}{⾩,⼁}
  \begin{Phonetics}{降临}{jiang4lin2}[][HSK 7-9]
    \definition{v.}{acontecer; chegar; vir}[春天降临,万物复苏。===A primavera chega e tudo renasce.]
  \end{Phonetics}
\end{Entry}

\begin{Entry}{降温}{8,12}{⾩,⽔}
  \begin{Phonetics}{降温}{jiang4 wen1}[][HSK 4]
    \definition{v.}{baixar a temperatura (como em uma oficina);  recusar | cair a temperatura | esfriar; resfriar; metáfora para um declínio no entusiasmo ou uma diminuição no ímpeto de algo}
  \end{Phonetics}
\end{Entry}

\begin{Entry}{降落}{8,12}{⾩,⾋}
  \begin{Phonetics}{降落}{jiang4luo4}[][HSK 4]
    \definition{v.}{aterrissar; descer; descer do céu}
  \end{Phonetics}
\end{Entry}

%%%%%%%%%% 限 %%%%%%%%%%
\subsection*{限}\addcontentsline{loh}{figure}{限}

\begin{Entry}{限}{8}{⾩}
  \begin{Phonetics}{限}{xian4}
    \definition{s.}{limite | limiar}
    \definition{v.}{definir um limite; limitar; restringir}
  \end{Phonetics}
\end{Entry}

\begin{Entry}{限制}{8,8}{⾩,⼑}
  \begin{Phonetics}{限制}{xian4zhi4}[][HSK 4]
    \definition[些]{s.}{limite; restrição; confinamento}
    \definition{v.}{limitar; adstringir; restringir; confinar; fechar em (sobre)}
  \end{Phonetics}
\end{Entry}

%%%%%%%%%% 隶 %%%%%%%%%%
\subsection*{隶}\addcontentsline{loh}{figure}{隶}

\begin{Entry}{隶}{8}{⾪}[Kangxi 171]
  \begin{Phonetics}{隶}{li4}
    \definition*{s.}{Sobrenome: Li}
    \definition{s.}{escravo; pessoa em servidão; pessoas escravizadas | Arcaico: corredor de cargo governamental na China feudal | um dos estilos antigos da caligrafia chinesa}
    \definition{v.}{estar subordinado a; estar afiliado a (ou com)}
  \end{Phonetics}
\end{Entry}

%%%%%%%%%% 雨 %%%%%%%%%%
\subsection*{雨}\addcontentsline{loh}{figure}{雨}

\begin{Entry}{雨}{8}{⾬}[Kangxi 173]
  \begin{Phonetics}{雨}{yu3}[][HSK 1]
    \definition*{s.}{Sobrenome: Yu}
    \definition[场,阵,滴]{s.}{chuva; água que cai das nuvens para o solo}
  \end{Phonetics}
  \begin{Phonetics}{雨}{yu4}
    \definition{v.}{cair (chuva, neve, etc.) | precipitar | chover | molhar}
  \end{Phonetics}
\end{Entry}

\begin{Entry}{雨水}{8,4}{⾬,⽔}
  \begin{Phonetics}{雨水}{yu3shui3}[][HSK 5]
    \definition{s.}{água da chuva; precipitação; chuva; água proveniente da chuva}
  \end{Phonetics}
\end{Entry}

\begin{Entry}{雨伞}{8,6}{⾬,⼈}
  \begin{Phonetics}{雨伞}{yu3san3}
    \definition[把]{s.}{guarda"-chuva}
  \end{Phonetics}
\end{Entry}

\begin{Entry}{雨衣}{8,6}{⾬,⾐}
  \begin{Phonetics}{雨衣}{yu3yi1}[][HSK 6]
    \definition[件,个]{s.}{capa de chuva; jaqueta impermeável; roupas impermeáveis}
  \end{Phonetics}
\end{Entry}

\begin{Entry}{雨蚀}{8,9}{⾬,⾷}
  \begin{Phonetics}{雨蚀}{yu3shi2}
    \definition{s.}{erosão da chuva}
  \end{Phonetics}
\end{Entry}

\begin{Entry}{雨靴}{8,13}{⾬,⾰}
  \begin{Phonetics}{雨靴}{yu3xue1}
    \definition[双]{s.}{botas de chuva}
  \end{Phonetics}
\end{Entry}

%%%%%%%%%% 青 %%%%%%%%%%
\subsection*{青}\addcontentsline{loh}{figure}{青}

\begin{Entry}{青}{8}{⾭}[Kangxi 174]
  \begin{Phonetics}{青}{qing1}[][HSK 5]
    \definition*{s.}{Província de Qinghai, abreviação de 青海 | Sobrenome: Qing}
    \definition{adj.}{azul ou verde | preto | jovens (pessoas)}
    \definition{s.}{grama verde | colheitas jovens (não maduras) | tiras de bambu verde}
  \seealsoref{青海}{qing1hai3}
  \end{Phonetics}
\end{Entry}

\begin{Entry}{青天}{8,4}{⾭,⼤}
  \begin{Phonetics}{青天}{qing1tian1}
    \definition{s.}{céu claro, limpo ou azul}
  \end{Phonetics}
\end{Entry}

\begin{Entry}{青少年}{8,4,6}{⾭,⼩,⼲}
  \begin{Phonetics}{青少年}{qing1shao4nian2}[][HSK 2]
    \definition[位,名,个,些]{s.}{adolescentes}
  \end{Phonetics}
\end{Entry}

\begin{Entry}{青玉米}{8,5,6}{⾭,⽟,⽶}
  \begin{Phonetics}{青玉米}{qing1yu4mi3}
    \definition{s.}{milho verde}
  \end{Phonetics}
\end{Entry}

\begin{Entry}{青年}{8,6}{⾭,⼲}
  \begin{Phonetics}{青年}{qing1nian2}[][HSK 2]
    \definition[个,位,名,些]{s.}{juventude; jovem; refere"-se ao período entre os 15 e os 30 anos de idade}
  \end{Phonetics}
\end{Entry}

\begin{Entry}{青年节}{8,6,5}{⾭,⼲,⾋}
  \begin{Phonetics}{青年节}{qing1nian2jie2}
    \definition*{s.}{Dia da Juventude (4 de maio)}
  \end{Phonetics}
\end{Entry}

\begin{Entry}{青春}{8,9}{⾭,⽇}
  \begin{Phonetics}{青春}{qing1chun1}[][HSK 4]
    \definition[个]{s.}{juventude; jovialidade}
  \end{Phonetics}
\end{Entry}

\begin{Entry}{青春期}{8,9,12}{⾭,⽇,⽉}
  \begin{Phonetics}{青春期}{qing1chun1qi1}[][HSK 7-9]
    \definition{s.}{puberdade; adolescência; refere"-se ao período em que os órgãos sexuais masculinos e femininos se desenvolvem rapidamente até a maturidade completa, tipicamente entre os 14 e 16 anos para os meninos e entre os 13 e 14 anos para as meninas}
  \end{Phonetics}
\end{Entry}

\begin{Entry}{青海}{8,10}{⾭,⽔}
  \begin{Phonetics}{青海}{qing1hai3}
    \definition*{s.}{Província de Qinghai}
  \end{Phonetics}
\end{Entry}

\begin{Entry}{青菜}{8,11}{⾭,⾋}
  \begin{Phonetics}{青菜}{qing1cai4}
    \definition{s.}{verduras}
  \end{Phonetics}
\end{Entry}

\begin{Entry}{青铜}{8,11}{⾭,⾦}
  \begin{Phonetics}{青铜}{qing1tong2}
    \definition{s.}{bronze (liga de cobre, 銅, e estanho, 锡)}
  \end{Phonetics}
\end{Entry}

\begin{Entry}{青椒}{8,12}{⾭,⽊}
  \begin{Phonetics}{青椒}{qing1jiao1}
    \definition{s.}{pimenta verde}
  \end{Phonetics}
\end{Entry}

\begin{Entry}{青蛙}{8,12}{⾭,⾍}
  \begin{Phonetics}{青蛙}{qing1wa1}[][HSK 7-9]
    \definition[只]{s.}{sapo}
  \end{Phonetics}
\end{Entry}

%%%%%%%%%% 非 %%%%%%%%%%
\subsection*{非}\addcontentsline{loh}{figure}{非}

\begin{Entry}{非}{8}{⾮}[Kangxi 175]
  \begin{Phonetics}{非}{fei1}[][HSK 4]
    \definition*{s.}{África, abreviação de 非洲 | Sobrenome: Fei}
    \definition{adv.}{Em resposta a 不, indica necessidade (deve)}
    \definition{pref.}{indicando negatividade ou exclusão}
    \definition{s.}{engano; erro}
    \definition{v.}{opor"-se a; culpar; censurar | não estar em conformidade com; ser contrário a | não ser | ter que; simplesmente precisar (fazer algo)}
  \seealsoref{不}{bu4}
  \seealsoref{非洲}{fei1zhou1}
  \end{Phonetics}
\end{Entry}

\begin{Entry}{非凡}{8,3}{⾮,⼏}
  \begin{Phonetics}{非凡}{fei1fan2}[][HSK 7-9]
    \definition{adj.}{excepcional; extraordinário; incomum; mais do que o normal}
  \end{Phonetics}
\end{Entry}

\begin{Entry}{非法}{8,8}{⾮,⽔}
  \begin{Phonetics}{非法}{fei1fa3}[][HSK 7-9]
    \definition{adj.}{ilegal; ilícito; fora da lei}
  \end{Phonetics}
\end{Entry}

\begin{Entry}{非金属}{8,8,12}{⾮,⾦,⼫}
  \begin{Phonetics}{非金属}{fei1jin4shu3}
    \definition{s.}{Química: não metal; metalóide; com exceção do bromo, os elementos que geralmente não têm brilho metálico nem ductilidade e não conduzem facilmente eletricidade ou calor são gases ou sólidos à temperatura ambiente, como oxigênio, enxofre, nitrogênio, fósforo, etc.}
  \end{Phonetics}
\end{Entry}

\begin{Entry}{非洲}{8,9}{⾮,⽔}
  \begin{Phonetics}{非洲}{fei1zhou1}
    \definition*{s.}{África}
  \end{Phonetics}
\end{Entry}

\begin{Entry}{非洲人}{8,9,2}{⾮,⽔,⼈}
  \begin{Phonetics}{非洲人}{fei1zhou1ren2}
    \definition{s.}{africano | pessoa ou povo da África}
  \end{Phonetics}
\end{Entry}

\begin{Entry}{非常}{8,11}{⾮,⼱}
  \begin{Phonetics}{非常}{fei1chang2}[][HSK 1]
    \definition{adj.}{extraordinário; incomum; especial}
    \definition{adv.}{muito; extremamente; altamente}
  \end{Phonetics}
\end{Entry}

\begin{Entry}{非得}{8,11}{⾮,⼻}
  \begin{Phonetics}{非得}{fei1dei3}[][HSK 7-9]
    \definition{adv.}{(geralmente usado comcomitantemente com 不 ou 才) tem que; deve}[你非得服从命令不可。===Você deve obedecer às ordens.]
  \seealsoref{不}{bu4}
  \seealsoref{才}{cai2}
  \end{Phonetics}
\end{Entry}

%%%%%%%%%% 靣 %%%%%%%%%%
\subsection*{靣}\addcontentsline{loh}{figure}{靣}

\begin{Entry}{靣}{8}{⼀}[Kangxi 176]
  \begin{Phonetics}{靣}{mian4}
    \variantof{面}
  \end{Phonetics}
\end{Entry}

%%%%%%%%%% 顶 %%%%%%%%%%
\subsection*{顶}\addcontentsline{loh}{figure}{顶}

\begin{Entry}{顶}{8}{⾴}
  \begin{Phonetics}{顶}{ding3}[][HSK 4]
    \definition{adv.}{muito (linguagem falada); a maioria; extremamente; expressa o grau mais alto, equivalente a 最 e 极}
    \definition{clas.}{usado para coisas que têm um topo}
    \definition{prep.}{até}
    \definition{s.}{coroa da cabeça; parte mais alta do corpo ou objeto | topo; limite superior; ponto mais alto}
    \definition{v.}{carregar na cabeça; carregar em sua cabeça | empurrar (ou apoiar) para cima; empurrar por baixo (ou por trás) | dar cabeçadas; dar uma coronhada | sustentar; apoiar; suportar | resistir; ir contra; enfrentar | rebater; retorquir; responder de volta | cooperar; enfrentar; apoiar; dar suporte | igualar; ser equivalente a | substituir; tomar o lugar de | assumir o controle; transferir ou adquirir o direito de administrar um negócio ou alugar uma casa ou terreno}
  \seealsoref{极}{ji2}
  \seealsoref{最}{zui4}
  \end{Phonetics}
\end{Entry}

\begin{Entry}{顶多}{8,6}{⾴,⼣}
  \begin{Phonetics}{顶多}{ding3duo1}[][HSK 7-9]
    \definition{adv.}{na melhor das hipóteses; no máximo, na opinião do orador, o número real não será maior que o maior número estimado}
  \end{Phonetics}
\end{Entry}

\begin{Entry}{顶尖}{8,6}{⾴,⼩}
  \begin{Phonetics}{顶尖}{ding3jian1}[][HSK 7-9]
    \definition{adj.}{melhor; de primeira classe; de mais alto nível}
    \definition{s.}{centro; ápice | topo pontiagudo; ponta; pico; a parte mais alta e pontiaguda}
  \end{Phonetics}
\end{Entry}

\begin{Entry}{顶级}{8,6}{⾴,⽷}
  \begin{Phonetics}{顶级}{ding3ji2}[][HSK 7-9]
    \definition{adj.}{de primeira classe; de alta qualidade; de ponta}
  \end{Phonetics}
\end{Entry}

%%%%%%%%%% 饱 %%%%%%%%%%
\subsection*{饱}\addcontentsline{loh}{figure}{饱}

\begin{Entry}{饱}{8}{⾷}
  \begin{Phonetics}{饱}{bao3}[][HSK 2]
    \definition{adj.}{cheio; comer até ficar satisfeito | cheio; rechonchudo}
    \definition{adv.}{totalmente; completamente; plenamente}
    \definition{v.}{satisfazer}
  \end{Phonetics}
\end{Entry}

\begin{Entry}{饱和}{8,8}{⾷,⼝}
  \begin{Phonetics}{饱和}{bao3he2}[][HSK 7-9]
    \definition{v.}{estar saturado; a uma certa temperatura ou pressão, a quantidade de soluto contida na solução atinge seu limite máximo e não consegue mais se dissolver | estar saturado; metaforicamente, a quantidade de algo atinge um máximo dentro de um certo intervalo}
  \end{Phonetics}
\end{Entry}

\begin{Entry}{饱满}{8,13}{⾷,⽔}
  \begin{Phonetics}{饱满}{bao3man3}[][HSK 7-9]
    \definition{adj.}{cheio; rechonchudo; bem empilhado; preenchido | robusto; abundante; pleno; vigoroso}
  \end{Phonetics}
\end{Entry}

%%%%%%%%%% 饲 %%%%%%%%%%
\subsection*{饲}\addcontentsline{loh}{figure}{饲}

\begin{Entry}{饲}{8}{⾷}
  \begin{Phonetics}{饲}{si4}
    \definition{s.}{forragem; alimento; ração}
    \definition{v.}{criar; manter; reproduzir}
  \end{Phonetics}
\end{Entry}

\begin{Entry}{饲养}{8,9}{⾷,⼋}
  \begin{Phonetics}{饲养}{si4yang3}[][HSK 7-9]
    \definition{v.}{criar; alimentar aves, gado e outros animais}
  \synonymref{放养}{fang4yang3}
  \synonymref{喂养}{wei4yang3}
  \end{Phonetics}
\end{Entry}

\begin{Entry}{饲料}{8,10}{⾷,⽃}
  \begin{Phonetics}{饲料}{si4liao4}[][HSK 7-9]
    \definition[袋]{s.}{alimento; forragem; alimento para gado ou aves}
  \end{Phonetics}
\end{Entry}

%%%%%%%%%% 驻 %%%%%%%%%%
\subsection*{驻}\addcontentsline{loh}{figure}{驻}

\begin{Entry}{驻}{8}{⾺}
  \begin{Phonetics}{驻}{zhu4}[][HSK 6]
    \definition{v.}{parar; ficar | estar estacionado; acampar; (tropas ou pessoal) viver no local onde desempenham suas funções; (organização) estar localizada em um determinado lugar}
  \end{Phonetics}
\end{Entry}

\begin{Entry}{驻军}{8,6}{⾺,⼍}
  \begin{Phonetics}{驻军}{zhu4jun1}
    \definition{s.}{guarnição}
    \definition{v.}{guarcener ou prover uma tropa}
  \end{Phonetics}
\end{Entry}

%%%%%%%%%% 驾 %%%%%%%%%%
\subsection*{驾}\addcontentsline{loh}{figure}{驾}

\begin{Entry}{驾}{8}{⾺}
  \begin{Phonetics}{驾}{jia4}[][HSK 7-9]
    \definition*{s.}{Sobrenome: Jia}
    \definition{pron.}{Cortês: você; você mesmo}
    \definition[点]{s.}{carruagem do imperador; refere"-se especificamente ao carro do imperador, referindo"-se ao imperador | referindo"-se a um veículo, usado como um termo respeitoso para uma pessoa}
    \definition{v.}{atrelar; puxar (uma carroça, etc.) | dirigir (um veículo); pilotar (um avião); velejar (um barco) | montar; cavalgar}
  \end{Phonetics}
\end{Entry}

\begin{Entry}{驾车}{8,4}{⾺,⾞}
  \begin{Phonetics}{驾车}{jia4 che1}[][HSK 7-9]
    \definition{v.}{dirigir um veículo}
  \end{Phonetics}
\end{Entry}

\begin{Entry}{驾驭}{8,5}{⾺,⾺}
  \begin{Phonetics}{驾驭}{jia4yu4}[][HSK 7-9]
    \definition{v.}{dirigir; conduzir animais ou veículos para a frente | controlar; fazer algo agir de acordo com a vontade de alguém}
  \end{Phonetics}
\end{Entry}

\begin{Entry}{驾驶}{8,8}{⾺,⾺}
  \begin{Phonetics}{驾驶}{jia4shi3}[][HSK 5]
    \definition{v.}{dirigir; pilotar; conduzir; guiar; operar (um carro, navio, avião, trator, etc.) para fazê-lo mover}
  \end{Phonetics}
\end{Entry}

\begin{Entry}{驾照}{8,13}{⾺,⽕}
  \begin{Phonetics}{驾照}{jia4zhao4}[][HSK 5]
    \definition[本,张]{s.}{carteira de motorista}
  \end{Phonetics}
\end{Entry}

%%%%%%%%%% 鱼 %%%%%%%%%%
\subsection*{鱼}\addcontentsline{loh}{figure}{鱼}

\begin{Entry}{鱼}{8}{⿂}[Kangxi 195]
  \begin{Phonetics}{鱼}{yu2}[][HSK 2]
    \definition*{s.}{Sobrenome: Yu}
    \definition[条,种,尾]{s.}{peixe; um vertebrado que vive na água; geralmente possui um corpo achatado lateralmente, fusiforme e com muitas escamas; nada com as nadadeiras e respira com as brânquias; sua temperatura corporal varia de acordo com a temperatura externa; existem muitas espécies, a maioria das quais comestíveis | carne de peixe; peixe (como alimento)}
  \end{Phonetics}
\end{Entry}

\begin{Entry}{鱼片}{8,4}{⿂,⽚}
  \begin{Phonetics}{鱼片}{yu2pian4}
    \definition{s.}{fatia de peixe | filé de peixe}
  \end{Phonetics}
\end{Entry}

\begin{Entry}{鱼汛}{8,6}{⿂,⽔}
  \begin{Phonetics}{鱼汛}{yu2xun4}
    \variantof{渔汛}
  \end{Phonetics}
\end{Entry}

\begin{Entry}{鱼网}{8,6}{⿂,⽹}
  \begin{Phonetics}{鱼网}{yu2wang3}
    \variantof{渔网}
  \end{Phonetics}
\end{Entry}

\begin{Entry}{鱼具}{8,8}{⿂,⼋}
  \begin{Phonetics}{鱼具}{yu2ju4}
    \variantof{渔具}
  \end{Phonetics}
\end{Entry}

\begin{Entry}{鱼香}{8,9}{⿂,⾹}
  \begin{Phonetics}{鱼香}{yu2xiang1}
    \definition{s.}{um tempero da culinária chinesa que normalmente contém alho, cebolinha, gengibre, açúcar, sal, pimenta, etc.; embora 鱼香 signifique literalmente ``fragrância de peixe'', não contém frutos do mar}
  \end{Phonetics}
\end{Entry}

\begin{Entry}{鱼香肉丝}{8,9,6,5}{⿂,⾹,⾁,⼀}
  \begin{Phonetics}{鱼香肉丝}{yu2xiang1rou4si1}
    \definition{s.}{Prato: tiras de carne de porco salteadas com molho picante}
  \seealsoref{鱼香}{yu2xiang1}
  \end{Phonetics}
\end{Entry}

\begin{Entry}{鱼船}{8,11}{⿂,⾈}
  \begin{Phonetics}{鱼船}{yu2chuan2}
    \definition{s.}{barco de pesca}
  \seealsoref{渔船}{yu2chuan2}
  \end{Phonetics}
\end{Entry}

%%%%%%%%%% 鸣 %%%%%%%%%%
\subsection*{鸣}\addcontentsline{loh}{figure}{鸣}

\begin{Entry}{鸣}{8}{⿃}
  \begin{Phonetics}{鸣}{ming2}
    \definition{v.}{chorar (pássaros, animais e insetos) | fazer um som | dar voz (gratidão, queixas, etc.)}
  \end{Phonetics}
\end{Entry}

%%%%%%%%%% 齿 %%%%%%%%%%
\subsection*{齿}\addcontentsline{loh}{figure}{齿}

\begin{Entry}{齿}{8}{⿒}[Kangxi 211]
  \begin{Phonetics}{齿}{chi3}
    \definition[颗]{s.}{dente | uma parte de qualquer coisa semelhante a um dente; parte dentada de um objeto | idade (de uma pessoa); faixa etária}
    \definition{v.}{mencionar; falar de}
  \end{Phonetics}
\end{Entry}

\begin{Entry}{齿儿}{8,2}{⿒,⼉}
  \begin{Phonetics}{齿儿}{chi3r5}
    \definition{s.}{dentes}
  \end{Phonetics}
\end{Entry}

%%%%% EOF %%%%%


 %%%
%%% 9画
%%%
\section*{9画}\addcontentsline{toc}{section}{9画}\addcontentsline{loh}{figure}{\#\#\#\# 9画}

%%%%%%%%%% 临 %%%%%%%%%%
\subsection*{临}\addcontentsline{loh}{figure}{临}

\begin{Entry}{临}{9}{⼁}
  \begin{Phonetics}{临}{lin2}[][HSK 7-9]
    \definition*{s.}{Sobrenome: Lin}
    \definition{adv.}{pouco antes; prestes a; no ponto de; indica que uma ação está prestes a ocorrer}
    \definition{v.}{encarar; enfrentar; aproximar-se | chegar; estar presente | copiar (um modelo de caligrafia ou pintura); traçar sobre as palavras ou figuras | olhar de cima para baixo | ir de cima para baixo}
  \end{Phonetics}
\end{Entry}

\begin{Entry}{临床}{9,7}{⼁,⼴}
  \begin{Phonetics}{临床}{lin2chuang2}[][HSK 7-9]
    \definition{v.}{praticar medicina clínica; atender diretamente os pacientes}
  \end{Phonetics}
\end{Entry}

\begin{Entry}{临时}{9,7}{⼁,⽇}
  \begin{Phonetics}{临时}{lin2shi2}[][HSK 4]
    \definition{adj.}{temporário; provisório; por um breve período}
    \definition{adv.}{no momento em que algo acontece (quando as coisas dão errado)}
  \end{Phonetics}
\end{Entry}

\begin{Entry}{临近}{9,7}{⼁,⾡}
  \begin{Phonetics}{临近}{lin2jin4}[][HSK 7-9]
    \definition{v.}{aproximar-se; estar perto de}
  \end{Phonetics}
\end{Entry}

\begin{Entry}{临街}{9,12}{⼁,⾏}
  \begin{Phonetics}{临街}{lin2jie1}[][HSK 7-9]
    \definition{s.}{de frente para a rua; encostado na rua}
  \end{Phonetics}
\end{Entry}

%%%%%%%%%% 举 %%%%%%%%%%
\subsection*{举}\addcontentsline{loh}{figure}{举}

\begin{Entry}{举}{9}{⼂}
  \begin{Phonetics}{举}{ju3}[][HSK 2]
    \definition*{s.}{Sobrenome: Ju}
    \definition{adj.}{inteiro; completo}
    \definition{s.}{ato; ação; movimento; comportamento | (nas dinastias Ming e Qing) candidato aprovado nos exames imperiais a nível provincial}
    \definition{v.}{levantar; erguer; sustentar | começar; iniciar; surgir | eleger; escolher; recomendar; selecionar | citar; enumerar; propor; revelar}
  \end{Phonetics}
\end{Entry}

\begin{Entry}{举一反三}{9,1,4,3}{⼂,⼀,⼜,⼀}
  \begin{Phonetics}{举一反三}{ju3yi1-fan3san1}[][HSK 7-9]
    \definition{expr.}{aprender por analogia; inferir outras coisas a partir de um fato; aprender muitas coisas por analogia a partir de uma única coisa}
  \end{Phonetics}
\end{Entry}

\begin{Entry}{举办}{9,4}{⼂,⼒}
  \begin{Phonetics}{举办}{ju3ban4}[][HSK 3]
    \definition{v.}{conduzir; organizar; realizar}
  \end{Phonetics}
\end{Entry}

\begin{Entry}{举手}{9,4}{⼂,⼿}
  \begin{Phonetics}{举手}{ju3 shou3}[][HSK 2]
    \definition{v.}{levantar a mão ou as mãos; levantar a mão para sinalizar ou responder a uma pergunta}
  \end{Phonetics}
\end{Entry}

\begin{Entry}{举止}{9,4}{⼂,⽌}
  \begin{Phonetics}{举止}{ju3zhi3}[][HSK 7-9]
    \definition{s.}{maneira; comportamento; porte; postura; refere"-se à postura e ao comportamento}
  \end{Phonetics}
\end{Entry}

\begin{Entry}{举世无双}{9,5,4,4}{⼂,⼀,⽆,⼜}
  \begin{Phonetics}{举世无双}{ju3shi4-wu2shuang1}[][HSK 7-9]
    \definition{expr.}{``Inigualável no mundo.''; inigualável; incomparável; sem igual; único; número um do mundo}
  \end{Phonetics}
\end{Entry}

\begin{Entry}{举世闻名}{9,5,9,6}{⼂,⼀,⾨,⼝}
  \begin{Phonetics}{举世闻名}{ju3shi4-wen2ming2}[][HSK 7-9]
    \definition{expr.}{``De renome mundial.''; mundialmente famoso}
  \end{Phonetics}
\end{Entry}

\begin{Entry}{举世瞩目}{9,5,17,5}{⼂,⼀,⽬,⽬}
  \begin{Phonetics}{举世瞩目}{ju3shi4-zhu3mu4}[][HSK 7-9]
    \definition{expr.}{``De renome mundial.''; atrair a atenção mundial; tornar-se o centro das atenções mundiais; o mundo inteiro está assistindo}
  \end{Phonetics}
\end{Entry}

\begin{Entry}{举动}{9,6}{⼂,⼒}
  \begin{Phonetics}{举动}{ju3dong4}[][HSK 5]
    \definition{s.}{ato; atividade; movimento; ação}
  \end{Phonetics}
\end{Entry}

\begin{Entry}{举行}{9,6}{⼂,⾏}
  \begin{Phonetics}{举行}{ju3xing2}[][HSK 2]
    \definition{v.}{realizar (uma reunião, cerimônia, etc.); realizar (atividades formais ou solenes)}
  \end{Phonetics}
\end{Entry}

\begin{Entry}{举报}{9,7}{⼂,⼿}
  \begin{Phonetics}{举报}{ju3bao4}[][HSK 7-9]
    \definition{v.}{relatar; denunciar}[我决定举报不法行为。===Decidi denunciar as atividades ilegais.]
  \end{Phonetics}
\end{Entry}

\begin{Entry}{举例}{9,8}{⼂,⼈}
  \begin{Phonetics}{举例}{ju3/li4}[][HSK 7-9]
    \definition{v.+compl.}{dar um exemplo; citar um caso}
  \end{Phonetics}
\end{Entry}

\begin{Entry}{举重}{9,9}{⼂,⾥}
  \begin{Phonetics}{举重}{ju3zhong4}[][HSK 7-9]
    \definition{s.}{levantamento de peso}
    \definition{v.}{levantar pesos}
  \end{Phonetics}
\end{Entry}

\begin{Entry}{举措}{9,11}{⼂,⼿}
  \begin{Phonetics}{举措}{ju3cuo4}[][HSK 7-9]
    \definition{v.}{mover; agir; medir}
  \end{Phonetics}
\end{Entry}

%%%%%%%%%% 亭 %%%%%%%%%%
\subsection*{亭}\addcontentsline{loh}{figure}{亭}

\begin{Entry}{亭}{9}{⼇}
  \begin{Phonetics}{亭}{ting2}
    \definition{s.}{pavilhão | cabine | quiosque}
  \end{Phonetics}
\end{Entry}

%%%%%%%%%% 亮 %%%%%%%%%%
\subsection*{亮}\addcontentsline{loh}{figure}{亮}

\begin{Entry}{亮}{9}{⼇}
  \begin{Phonetics}{亮}{liang4}[][HSK 2]
    \definition*{s.}{Sobrenome: Lian}
    \definition{adj.}{brilhante; claro | alto e claro; retumbante | esclarecido; aberto e claro}
    \definition{s.}{luz}
    \definition{v.}{iluminar; clarear; brilhar | elevar a voz; ressoar; tornar o som mais alto | revelar; mostrar; aparecer; exibir}
  \end{Phonetics}
\end{Entry}

\begin{Entry}{亮丽}{9,7}{⼇,⼀}
  \begin{Phonetics}{亮丽}{liang2li2}[][HSK 7-9]
    \definition{adj.}{brilhante e bonito | belo; gracioso}
  \seealsoref{靓丽}{liang4li4}
  \end{Phonetics}
\end{Entry}

\begin{Entry}{亮点}{9,9}{⼇,⽕}
  \begin{Phonetics}{亮点}{liang4dian3}[][HSK 7-9]
    \definition{s.}{destaque; ponto alto; mérito; algo que vale a pena mencionar; pontos que emitem luz | mérito; ponto final; um mérito excepcional; uma metáfora para uma pessoa ou coisa bela, louvável e que chama a atenção}
  \end{Phonetics}
\end{Entry}

\begin{Entry}{亮相}{9,9}{⼇,⽬}
  \begin{Phonetics}{亮相}{liang4/xiang4}[][HSK 7-9]
    \definition{v.+compl.}{fazer pose (ópera chinesa); nas apresentações tradicionais de ópera chinesa, certos personagens às vezes assumem uma breve pose estática ao entrar ou sair do palco, ou ao final de uma dança, para transmitir seu estado mental | expressar as próprias opiniões; declarar a própria posição; metaforicamente, significa expressar abertamente as próprias opiniões | fazer sua estreia; fazer uma aparição; metaforicamente, significa fazer uma aparição ou apresentação pública}
  \end{Phonetics}
\end{Entry}

%%%%%%%%%% 亲 %%%%%%%%%%
\subsection*{亲}\addcontentsline{loh}{figure}{亲}

\begin{Entry}{亲}{9}{⼇}
  \begin{Phonetics}{亲}{qin1}[][HSK 3]
    \definition{adj.}{parente próximo; relacionado por sangue; de parentesco consanguíneo; parente consanguíneo mais próximo | querido; próximo; íntimo; relações próximas entre pessoas; sentimentos profundos | em si mesmo; pessoalmente}
    \definition[位]{s.}{pais; refere"-se aos pais; também se refere apenas ao pai ou à mãe | parente; refere"-se a pessoas que são relacionadas por sangue ou casamento| casal; casamento; refere"-se ao casamento ou relacionamento conjugal | noiva; refere"-se especificamente à noiva}
    \definition{v.}{beijar | (países, partidos, etc.) a favor de; apoiar; estar perto de}
  \antonymref{疏}{shu1}
  \end{Phonetics}
  \begin{Phonetics}{亲}{qing4}
    \definition{s.}{parentes por afinidade; parentes por casamento}
  \end{Phonetics}
\end{Entry}

\begin{Entry}{亲人}{9,2}{⼇,⼈}
  \begin{Phonetics}{亲人}{qin1ren2}[][HSK 3]
    \definition[个,位]{s.}{um membro da família; os pais, o cônjuge, os filhos, etc.; refere"-se a parentes ou cônjuges | queridos; entes queridos; aqueles queridos para alguém; uma metáfora para pessoas que têm um relacionamento próximo e sentimentos profundos}
  \end{Phonetics}
\end{Entry}

\begin{Entry}{亲切}{9,4}{⼇,⼑}
  \begin{Phonetics}{亲切}{qin1qie4}[][HSK 3]
    \definition{adj.}{gentil; cordial; cheio de sinceridade e cuidado, fazendo com que as pessoas se sintam acolhidas e acessíveis | próximo; íntimo; por familiaridade e afeição}
  \end{Phonetics}
\end{Entry}

\begin{Entry}{亲友}{9,4}{⼇,⼜}
  \begin{Phonetics}{亲友}{qin1you3}[][HSK 7-9]
    \definition[位,个]{s.}{amigos e parentes; parentes próximos}
  \end{Phonetics}
\end{Entry}

\begin{Entry}{亲手}{9,4}{⼇,⼿}
  \begin{Phonetics}{亲手}{qin1shou3}[][HSK 7-9]
    \definition{adv.}{si mesmo; pessoalmente; com as próprias mãos}
  \end{Phonetics}
\end{Entry}

\begin{Entry}{亲生}{9,5}{⼇,⽣}
  \begin{Phonetics}{亲生}{qin1sheng1}[][HSK 7-9]
    \definition{adj.}{próprio; biológico (filhos, pais); aquelas que dão à luz a si mesmas ou têm filhos próprios}
    \definition{v.}{ser filho biológico de alguém (ou seja, não adotado)}
  \end{Phonetics}
\end{Entry}

\begin{Entry}{亲自}{9,6}{⼇,⾃}
  \begin{Phonetics}{亲自}{qin1zi4}[][HSK 3]
    \definition{adv.}{pessoalmente; em pessoa; si mesmo; fazer algo diretamente por si mesmo}
  \end{Phonetics}
\end{Entry}

\begin{Entry}{亲身}{9,7}{⼇,⾝}
  \begin{Phonetics}{亲身}{qin1shen1}[][HSK 7-9]
    \definition{adj.}{pessoal; em primeira mão}
    \definition{adv.}{pessoalmente}
  \end{Phonetics}
\end{Entry}

\begin{Entry}{亲近}{9,7}{⼇,⾡}
  \begin{Phonetics}{亲近}{qin1jin4}[][HSK 7-9]
    \definition{adj.}{íntimo e próximo}
    \definition{v.}{ser próximo de; ter intimidade com}
  \end{Phonetics}
\end{Entry}

\begin{Entry}{亲和力}{9,8,2}{⼇,⼝,⼒}
  \begin{Phonetics}{亲和力}{qin1he2li4}[][HSK 7-9]
    \definition{s.}{afinidade; as forças que interagem quando duas ou mais substâncias se combinam para formar um composto | amabilidade; sociabilidade; forte atração; metaforicamente falando, refere"-se a uma força que faz as pessoas se sentirem confortáveis, amigáveis e dispostas a se aproximar}
  \end{Phonetics}
\end{Entry}

\begin{Entry}{亲朋好友}{9,8,6,4}{⼇,⽉,⼥,⼜}
  \begin{Phonetics}{亲朋好友}{qin1peng2-hao3you3}[][HSK 7-9]
    \definition{s.}{``Amigos e familiares.''; amigos e família; parentes e amigos}
  \end{Phonetics}
\end{Entry}

\begin{Entry}{亲热}{9,10}{⼇,⽕}
  \begin{Phonetics}{亲热}{qin1re4}[][HSK 7-9]
    \definition{adj.}{afetuoso; íntimo; caloroso; íntimo e acolhedor}
    \definition{v.}{comportar-se afetuosamente; demonstrar intimidade e entusiasmo}
  \end{Phonetics}
\end{Entry}

\begin{Entry}{亲爱}{9,10}{⼇,⽖}
  \begin{Phonetics}{亲爱}{qin1'ai4}[][HSK 4]
    \definition{adj.}{querido; amado; termo carinhoso que expressa intimidade e afeto}
  \end{Phonetics}
\end{Entry}

\begin{Entry}{亲密}{9,11}{⼇,⼧}
  \begin{Phonetics}{亲密}{qin1mi4}[][HSK 4]
    \definition{adj.}{próximo; íntimo; relacionamento afetuoso e próximo}
  \end{Phonetics}
\end{Entry}

\begin{Entry}{亲情}{9,11}{⼇,⼼}
  \begin{Phonetics}{亲情}{qin1qing2}[][HSK 7-9]
    \definition{s.}{afeto; laços familiares; vínculo emocional entre membros da família; o afeto entre membros da família}
  \end{Phonetics}
\end{Entry}

\begin{Entry}{亲戚}{9,11}{⼇,⼽}
  \begin{Phonetics}{亲戚}{qin1qi5}[][HSK 7-9]
    \definition[门,个,位]{s.}{parentes; pessoas com laços matrimoniais ou consanguíneos em sua própria família}
  \end{Phonetics}
\end{Entry}

\begin{Entry}{亲眼}{9,11}{⼇,⽬}
  \begin{Phonetics}{亲眼}{qin1yan3}[][HSK 6]
    \definition{adv.}{pessoalmente; com os próprios olhos}
  \end{Phonetics}
\end{Entry}

\begin{Entry}{亲属}{9,12}{⼇,⼫}
  \begin{Phonetics}{亲属}{qin1shu3}[][HSK 6]
    \definition{s.}{parentes; cognatos}
  \end{Phonetics}
\end{Entry}

%%%%%%%%%% 侵 %%%%%%%%%%
\subsection*{侵}\addcontentsline{loh}{figure}{侵}

\begin{Entry}{侵}{9}{⼈}
  \begin{Phonetics}{侵}{qin1}
    \definition*{s.}{Sobrenome: Qin}
    \definition{prep.}{aproximando-se; aproximar}
    \definition{v.}{invadir; intrometer"-se em; infringir | aproximar"-se (amanhecer)}
  \end{Phonetics}
\end{Entry}

\begin{Entry}{侵入}{9,2}{⼈,⼊}
  \begin{Phonetics}{侵入}{qin1ru4}
    \definition{v.}{invadir; intrometer-se em; fazer incursões em; (o inimigo) entra no território; (coisas estranhas ou nocivas) entram no interior}
  \end{Phonetics}
\end{Entry}

\begin{Entry}{侵占}{9,5}{⼈,⼘}
  \begin{Phonetics}{侵占}{qin1zhan4}[][HSK 7-9]
    \definition{v.}{ocupar à força; tomar posse ilegal da propriedade alheia | invadir e ocupar o território de outro país; ocupar o território de outro país por meio de agressão}
  \end{Phonetics}
\end{Entry}

\begin{Entry}{侵犯}{9,5}{⼈,⽝}
  \begin{Phonetics}{侵犯}{qin1fan4}[][HSK 6]
    \definition{v.}{violar; invadir; infringir; interferência ilegal com terceiros e violação de seus direitos | violar; fazer incursões; invadir o território de outro país}
  \end{Phonetics}
\end{Entry}

\begin{Entry}{侵权}{9,6}{⼈,⽊}
  \begin{Phonetics}{侵权}{qin1quan2}[][HSK 7-9]
    \definition{s.}{infração}
    \definition{v.}{infringir os direitos de}
  \end{Phonetics}
\end{Entry}

\begin{Entry}{侵害}{9,10}{⼈,⼧}
  \begin{Phonetics}{侵害}{qin1hai4}[][HSK 7-9]
    \definition{v.}{prejudicar; violar; infringir}
  \end{Phonetics}
\end{Entry}

\begin{Entry}{侵略}{9,11}{⼈,⽥}
  \begin{Phonetics}{侵略}{qin1lve4}[][HSK 7-9]
    \definition{v.}{invadir; agredir; violar o território e a soberania de outro país por meio de invasão armada, interferência política ou infiltração econômica e cultural, prejudicando assim os interesses desse outro país}
  \end{Phonetics}
\end{Entry}

%%%%%%%%%% 便 %%%%%%%%%%
\subsection*{便}\addcontentsline{loh}{figure}{便}

\begin{Entry}{便}{9}{⼈}
  \begin{Phonetics}{便}{bian4}[][HSK 6]
    \definition{adj.}{prático; conveniente | simples; comum; informal}
    \definition{adv.}{então; apenas no caso de; mesmo significado e uso de 就}
    \definition{conj.}{mesmo que; expressa uma concessão hipotética}
    \definition{s.}{facilidade; conveniência; o momento certo; a oportunidade | fezes ou urina}
    \definition{v.}{aliviar"-se; excretar fezes e urina}
  \seealsoref{就}{jiu4}
  \end{Phonetics}
  \begin{Phonetics}{便}{pian2}
    \definition*{s.}{Sobrenome: Pian}
    \definition{adj.}{silencioso e confortável}
  \end{Phonetics}
\end{Entry}

\begin{Entry}{便于}{9,3}{⼈,⼆}
  \begin{Phonetics}{便于}{bian4yu2}[][HSK 5]
    \definition{v.}{ser fácil para; ser conveniente para (algo ou fazer algo)}
  \end{Phonetics}
\end{Entry}

\begin{Entry}{便利}{9,7}{⼈,⼑}
  \begin{Phonetics}{便利}{bian4li4}[][HSK 5]
    \definition{adj.}{fácil; conveniente}
    \definition{s.}{facilidade; conveniência; coisas ou condições convenientes}
    \definition{v.}{facilitar; fornecer ajuda para que os outros se sintam confortáveis}
  \end{Phonetics}
\end{Entry}

\begin{Entry}{便利店}{9,7,8}{⼈,⼑,⼴}
  \begin{Phonetics}{便利店}{bian4li4dian4}[][HSK 7-9]
    \definition[个,家]{s.}{loja de conveniência; pequenas lojas de varejo localizadas em áreas residenciais para a conveniência dos moradores}
  \end{Phonetics}
\end{Entry}

\begin{Entry}{便条}{9,7}{⼈,⽊}
  \begin{Phonetics}{便条}{bian4tiao2}[][HSK 5]
    \definition[张,个]{s.}{nota ou mensagem informal; geralmente uma mensagem ou notificação}
  \end{Phonetics}
\end{Entry}

\begin{Entry}{便饭}{9,7}{⼈,⾷}
  \begin{Phonetics}{便饭}{bian4fan4}[][HSK 7-9]
    \definition[顿]{s.}{refeição comum; refeição simples}
    \definition{v.}{fazer uma refeição leve}
  \end{Phonetics}
\end{Entry}

\begin{Entry}{便函}{9,8}{⼈,⼐}
  \begin{Phonetics}{便函}{bian4han2}
    \definition{s.}{carta informal enviada por uma organização}
  \antonymref{公函}{gong1han2}
  \end{Phonetics}
\end{Entry}

\begin{Entry}{便宜}{9,8}{⼈,⼧}
  \begin{Phonetics}{便宜}{bian4yi2}
    \definition{adj.}{prático; conveniente; adequado}
  \end{Phonetics}
  \begin{Phonetics}{便宜}{pian2yi5}[][HSK 2]
    \definition{adj.}{barato; acessível}
    \definition[个,份,件]{s.}{vantagem em algum aspecto | ganho; lucro; vantagem; benefício indevido}
    \definition{v.}{deixar alguém escapar impune; obter algum benefício}
  \end{Phonetics}
\end{Entry}

\begin{Entry}{便是}{9,9}{⼈,⽇}
  \begin{Phonetics}{便是}{bian4shi4}[][HSK 6]
    \definition{adv.}{exatamente; precisamente; para expressar afirmação ou ênfase}
    \definition{conj.}{mesmo; mesmo que; usado para introduzir um caso extremo hipotético, enfatizando que o mesmo resultado ocorreria em circunstâncias tão extremas, sem mencionar circunstâncias normais; você também pode usar 即便是}
    \definition{part.}{usada no final de uma frase para expressar afirmação}
  \seealsoref{即便是}{ji2bian4shi4}
  \end{Phonetics}
\end{Entry}

\begin{Entry}{便捷}{9,11}{⼈,⼿}
  \begin{Phonetics}{便捷}{bian4jie2}[][HSK 7-9]
    \definition{adj.}{fácil; conveniente; direto e simples; direto e conveniente | ligeiro; ágil}
  \end{Phonetics}
\end{Entry}

\begin{Entry}{便道}{9,12}{⼈,⾡}
  \begin{Phonetics}{便道}{bian4dao4}[][HSK 7-9]
    \definition[条]{s.}{atalho | pavimento; calçada | estrada improvisada; estrada temporária}
  \end{Phonetics}
\end{Entry}

%%%%%%%%%% 促 %%%%%%%%%%
\subsection*{促}\addcontentsline{loh}{figure}{促}

\begin{Entry}{促}{9}{⼈}
  \begin{Phonetics}{促}{cu4}
    \definition{adj.}{curto; apressado; urgente}
    \definition{v.}{urgir; promover | estar perto de; estar perto}
  \end{Phonetics}
\end{Entry}

\begin{Entry}{促成}{9,6}{⼈,⼽}
  \begin{Phonetics}{促成}{cu4cheng2}[][HSK 7-9]
    \definition{v.}{facilitar; ajudar a concretizar}
  \end{Phonetics}
\end{Entry}

\begin{Entry}{促进}{9,7}{⼈,⾡}
  \begin{Phonetics}{促进}{cu4jin4}[][HSK 4]
    \definition{v.}{impulsionar; promover; avançar; incentivar o desenvolvimento}
  \end{Phonetics}
\end{Entry}

\begin{Entry}{促使}{9,8}{⼈,⼈}
  \begin{Phonetics}{促使}{cu4shi3}[][HSK 4]
    \definition{v.}{incitar; estimular; impelir; causar; provocar uma mudança em alguém ou em algo}
  \end{Phonetics}
\end{Entry}

\begin{Entry}{促销}{9,12}{⼈,⾦}
  \begin{Phonetics}{促销}{cu4xiao1}[][HSK 4]
    \definition{v.}{promover vendas}
  \end{Phonetics}
\end{Entry}

%%%%%%%%%% 俄 %%%%%%%%%%
\subsection*{俄}\addcontentsline{loh}{figure}{俄}

\begin{Entry}{俄}{9}{⼈}
  \begin{Phonetics}{俄}{e2}
    \definition*{s.}{Rússia, abreviação de 俄罗斯}
    \definition{adv.}{muito em breve; em breve; de repente}
  \seealsoref{俄罗斯}{e2luo2si1}
  \end{Phonetics}
\end{Entry}

\begin{Entry}{俄罗斯}{9,8,12}{⼈,⽹,⽄}
  \begin{Phonetics}{俄罗斯}{e2luo2si1}
    \definition*{s.}{Rússia}
  \end{Phonetics}
\end{Entry}

\begin{Entry}{俄罗斯人}{9,8,12,2}{⼈,⽹,⽄,⼈}
  \begin{Phonetics}{俄罗斯人}{e2luo2si1ren2}
    \definition{s.}{russo | pessoa ou povo da Rússia}
  \end{Phonetics}
\end{Entry}

\begin{Entry}{俄语}{9,9}{⼈,⾔}
  \begin{Phonetics}{俄语}{e2yu3}[][HSK 7-9]
    \definition{s.}{russo; língua russa}
  \end{Phonetics}
\end{Entry}

%%%%%%%%%% 俊 %%%%%%%%%%
\subsection*{俊}\addcontentsline{loh}{figure}{俊}

\begin{Entry}{俊}{9}{⼈}
  \begin{Phonetics}{俊}{jun4}[][HSK 7-9]
    \definition*{s.}{Sobrenome: Jun}
    \definition{adj.}{bonito; lindo; encantador; atraente; delicado e bonito | talentoso; inteligente; brilhante; excepcional; excepcionalmente inteligente}
    \definition{s.}{uma pessoa de talento excepcional; pessoas com inteligência excepcional}
  \end{Phonetics}
\end{Entry}

\begin{Entry}{俊俏}{9,9}{⼈,⼈}
  \begin{Phonetics}{俊俏}{jun4qiao4}[][HSK 7-9]
    \definition{adj.}{Coloquial: bonito e encantador}
  \end{Phonetics}
\end{Entry}

%%%%%%%%%% 俘 %%%%%%%%%%
\subsection*{俘}\addcontentsline{loh}{figure}{俘}

\begin{Entry}{俘}{9}{⼈}
  \begin{Phonetics}{俘}{fu2}
    \definition{s.}{prisioneiro de guerra; cativo}
    \definition{v.}{capturar; fazer prisioneiro | fazer prisioneiro de guerra}
  \end{Phonetics}
\end{Entry}

\begin{Entry}{俘虏}{9,8}{⼈,⾌}
  \begin{Phonetics}{俘虏}{fu2lu3}[][HSK 7-9]
    \definition[个,名,批,群]{s.}{cativo; prisioneiro de guerra; inimigos capturados durante a batalha}
    \definition{v.}{capturar (um inimigo) durante o combate; fazer prisioneiro; capturar}
  \end{Phonetics}
\end{Entry}

\begin{Entry}{俘获}{9,10}{⼈,⾋}
  \begin{Phonetics}{俘获}{fu2huo4}[][HSK 7-9]
    \definition{s.}{Física: captura; aprisionamento}[中子俘获===Captura de nêutrons]
    \definition{v.}{capturar; apreender}[军队成功俘获敌方指挥官。===O exército conseguiu capturar o comandante inimigo.]
  \end{Phonetics}
\end{Entry}

%%%%%%%%%% 保 %%%%%%%%%%
\subsection*{保}\addcontentsline{loh}{figure}{保}

\begin{Entry}{保}{9}{⼈}
  \begin{Phonetics}{保}{bao3}[][HSK 3]
    \definition*{s.}{Sobrenome: Bao}
    \definition{s.}{fiador; babá ou responsável pela guarda de crianças | oficial responsável; sistema administrativo; unidade administrativa do antigo registro civil}
    \definition{v.}{defender; proteger | manter; preservar; conservar em boas condições | assegurar; garantir | ser fiador de alguém}
  \end{Phonetics}
\end{Entry}

\begin{Entry}{保卫}{9,3}{⼈,⼙}
  \begin{Phonetics}{保卫}{bao3wei4}[][HSK 5]
    \definition{v.}{defender; proteger; salvaguardar; proteger"-se de ser violado}
  \end{Phonetics}
\end{Entry}

\begin{Entry}{保存}{9,6}{⼈,⼦}
  \begin{Phonetics}{保存}{bao3cun2}[][HSK 3]
    \definition{v.}{salvar; preservar; conservar; manter a existência com ênfase em que as coisas, as propriedades, os significados, os estilos, etc. não sofram perdas ou mudanças | (computação) salvar (um arquivo, etc.)}
  \end{Phonetics}
\end{Entry}

\begin{Entry}{保守}{9,6}{⼈,⼧}
  \begin{Phonetics}{保守}{bao3shou3}[][HSK 4]
    \definition{adj.}{retrógrado; conservador; pensamentos e conceitos que são retrógrados e não conseguem acompanhar o desenvolvimento da situação}
    \definition{v.}{manter; guardar; evitar perder}
  \end{Phonetics}
\end{Entry}

\begin{Entry}{保安}{9,6}{⼈,⼧}
  \begin{Phonetics}{保安}{bao3'an1}[][HSK 3]
    \definition[个,位,名]{s.}{guarda de segurança; segurança}
    \definition{v.}{proteger; manter em segurança; defender a segurança social | garantir a segurança; proteger a segurança dos trabalhadores e prevenir acidentes durante o processo de produção}
  \end{Phonetics}
\end{Entry}

\begin{Entry}{保佑}{9,7}{⼈,⼈}
  \begin{Phonetics}{保佑}{bao3you4}[][HSK 7-9]
    \definition{s.}{benção}[他希望神明保佑。===Ele espera a bênção de Deus.]
    \definition{v.}{abençoar e proteger; superstição, refere"-se à proteção e ajuda dos deuses}[神保佑我今天顺利。===Deus me abençoe hoje.]
  \end{Phonetics}
\end{Entry}

\begin{Entry}{保护}{9,7}{⼈,⼿}
  \begin{Phonetics}{保护}{bao3hu4}[][HSK 3]
    \definition{v.}{proteger, guardar, cuidar; salvaguardar; cuidar ao máximo, para que não seja danificado, referindo"-se principalmente a coisas concretas}
  \end{Phonetics}
\end{Entry}

\begin{Entry}{保护区}{9,7,4}{⼈,⼿,⼖}
  \begin{Phonetics}{保护区}{bao3hu4qu1}
    \definition[个,片]{s.}{zona de proteção | área de preservação; reserva natural}
  \end{Phonetics}
\end{Entry}

\begin{Entry}{保护主义}{9,7,5,3}{⼈,⼿,⼂,⼂}
  \begin{Phonetics}{保护主义}{bao3hu4zhu3yi4}
    \definition{s.}{protecionismo}
  \end{Phonetics}
\end{Entry}

\begin{Entry}{保护色}{9,7,6}{⼈,⼿,⾊}
  \begin{Phonetics}{保护色}{bao3hu4se4}
    \definition{s.}{camuflagem | coloração protetora}
  \end{Phonetics}
\end{Entry}

\begin{Entry}{保护剂}{9,7,8}{⼈,⼿,⼑}
  \begin{Phonetics}{保护剂}{bao3hu4ji4}
    \definition{s.}{agente protetor; protetor}
  \end{Phonetics}
\end{Entry}

\begin{Entry}{保护国}{9,7,8}{⼈,⼿,⼞}
  \begin{Phonetics}{保护国}{bao3hu4guo2}
    \definition{s.}{protetorado}
  \end{Phonetics}
\end{Entry}

\begin{Entry}{保护性}{9,7,8}{⼈,⼿,⼼}
  \begin{Phonetics}{保护性}{bao3hu4xing4}
    \definition{s.}{proteção; protetor}
  \end{Phonetics}
\end{Entry}

\begin{Entry}{保护物}{9,7,8}{⼈,⼿,⽜}
  \begin{Phonetics}{保护物}{bao3hu4 wu4}
    \definition{s.}{protetor}
  \end{Phonetics}
\end{Entry}

\begin{Entry}{保护者}{9,7,8}{⼈,⼿,⽼}
  \begin{Phonetics}{保护者}{bao3hu4zhe3}
    \definition{s.}{protetor | segurador}
  \end{Phonetics}
\end{Entry}

\begin{Entry}{保护神}{9,7,9}{⼈,⼿,⽰}
  \begin{Phonetics}{保护神}{bao3hu4shen2}
    \definition{s.}{anjo da guarda | santo patrono}
  \end{Phonetics}
\end{Entry}

\begin{Entry}{保证}{9,7}{⼈,⾔}
  \begin{Phonetics}{保证}{bao3zheng4}[][HSK 3]
    \definition[种,份]{s.}{compromisso; garantia; caução; aval; condições ou coisas que garantem a realização de algo}
    \definition{v.}{prometer; garantir; assegurar; certamente concluir algo; garantir que determinados padrões e requisitos sejam alcançados}
  \end{Phonetics}
\end{Entry}

\begin{Entry}{保姆}{9,8}{⼈,⼥}
  \begin{Phonetics}{保姆}{bao3mu3}[][HSK 7-9]
    \definition[位,名,个]{s.}{babá; mulheres empregadas como cuidadoras de crianças ou empregadas domésticas}
  \end{Phonetics}
\end{Entry}

\begin{Entry}{保质期}{9,8,12}{⼈,⾙,⽉}
  \begin{Phonetics}{保质期}{bao3zhi4qi1}[][HSK 7-9]
    \definition{s.}{data de validade; data de uso (em alimentos); período durante o qual os alimentos pré-embalados mantêm a sua qualidade nas condições de armazenamento especificadas no rótulo}
  \end{Phonetics}
\end{Entry}

\begin{Entry}{保亭}{9,9}{⼈,⼇}
  \begin{Phonetics}{保亭}{bao3ting2}
    \definition*{s.}{Condado autônomo de Li e Miao, Hainan}
  \end{Phonetics}
\end{Entry}

\begin{Entry}{保修}{9,9}{⼈,⼈}
  \begin{Phonetics}{保修}{bao3xiu1}[][HSK 7-9]
    \definition{v.}{garantir que a loja ou fábrica forneça serviço de reparo gratuito dentro de um determinado período de tempo; refere"-se ao reparo gratuito de mercadorias pela unidade de vendas ou pelo fabricante, de acordo com os regulamentos, após a venda das mercadorias | manter; manter em bom estado}
  \end{Phonetics}
\end{Entry}

\begin{Entry}{保养}{9,9}{⼈,⼋}
  \begin{Phonetics}{保养}{bao3yang3}[][HSK 5]
    \definition{v.}{preservar; cuidar bem (ou conservar) da saúde |  fazer manutenção; conservar; manter; manter em bom estado de conservação}
  \end{Phonetics}
\end{Entry}

\begin{Entry}{保持}{9,9}{⼈,⼿}
  \begin{Phonetics}{保持}{bao3chi2}[][HSK 3]
    \definition{v.}{manter; conservar; reter; preservar; manter um determinado estado, para que não desapareça ou não se altere}
  \end{Phonetics}
\end{Entry}

\begin{Entry}{保重}{9,9}{⼈,⾥}
  \begin{Phonetics}{保重}{bao3zhong4}[][HSK 7-9]
    \definition{v.}{cuidar de si mesmo}
  \end{Phonetics}
\end{Entry}

\begin{Entry}{保险}{9,9}{⼈,⾩}
  \begin{Phonetics}{保险}{bao3xian3}[][HSK 3]
    \definition{adj.}{seguro; pode ficar tranquilo}
    \definition[个,份,种]{s.}{seguro; um tipo de seguro comercial que garante que o segurado receba uma indenização em caso de prejuízo}
    \definition{v.}{ter certeza; estar obrigado a; garantir que algo aconteça (o que as pessoas desejam)}
  \end{Phonetics}
\end{Entry}

\begin{Entry}{保健}{9,10}{⼈,⼈}
  \begin{Phonetics}{保健}{bao3jian4}[][HSK 6]
    \definition{s.}{cuidados de saúde; proteção da saúde}
    \definition{v.}{cuidar da sua saúde; proteger sua saúde}
  \end{Phonetics}
\end{Entry}

\begin{Entry}{保留}{9,10}{⼈,⽥}
  \begin{Phonetics}{保留}{bao3liu2}[][HSK 3]
    \definition{v.}{manter; continuar a ter; manter o estado original inalterado | conter; reter; deixar ficar; não tirar | reservar; colocar os direitos, opiniões, etc. de lado, não exercê-los ou expressá-los por enquanto}
  \end{Phonetics}
\end{Entry}

\begin{Entry}{保密}{9,11}{⼈,⼧}
  \begin{Phonetics}{保密}{bao3/mi4}[][HSK 4]
    \definition{v.+compl.}{manter segredo; manter algo em segredo; manter a confidencialidade}
  \end{Phonetics}
\end{Entry}

\begin{Entry}{保暖}{9,13}{⼈,⽇}
  \begin{Phonetics}{保暖}{bao3/nuan3}[][HSK 7-9]
    \definition{v.+compl.}{manter"-se aquecido; ficar aquecido}[保暖能防感冒。===Manter"-se aquecido pode prevenir resfriados.]
  \end{Phonetics}
\end{Entry}

\begin{Entry}{保障}{9,13}{⼈,⾩}
  \begin{Phonetics}{保障}{bao3zhang4}[][HSK 7-9]
    \definition{s.}{segurança; garantia; coisas que podem fornecer proteção}
    \definition{v.}{proteger; salvaguardar; proteger contra violação e destruição | assegurar; garantir; prometer; garantir que não ocorrerá acidentes}
  \end{Phonetics}
\end{Entry}

\begin{Entry}{保管}{9,14}{⼈,⽵}
  \begin{Phonetics}{保管}{bao3guan3}[][HSK 7-9]
    \definition{adv.}{certamente; expressa confiança}
    \definition{s.}{gerente; almoxarife; lojista; pessoas que realizam trabalho de custódia}
    \definition{v.}{cuidar de; ser responsável por}
  \end{Phonetics}
\end{Entry}

\begin{Entry}{保鲜}{9,14}{⼈,⿂}
  \begin{Phonetics}{保鲜}{bao3xian1}[][HSK 7-9]
    \definition{v.}{manter fresco; preservar o frescor}
  \end{Phonetics}
\end{Entry}

%%%%%%%%%% 信 %%%%%%%%%%
\subsection*{信}\addcontentsline{loh}{figure}{信}

\begin{Entry}{信}{9}{⼈}
  \begin{Phonetics}{信}{xin4}[][HSK 2,3]
    \definition*{s.}{Sobrenome: Xin}
    \definition{adj.}{verdade}
    \definition[封,个,张]{s.}{carta; correio | mensagem; notícia; informação | sinal; evidência | confiança; fé; crédito | detonador (de bombas, etc.) | arsênico}
    \definition{v.}{acreditar; fazer um balanço; dar crédito | deixar à vontade; deixar à mercê; deixar ao acaso | professar fé em; acreditar em}
  \end{Phonetics}
\end{Entry}

\begin{Entry}{信心}{9,4}{⼈,⼼}
  \begin{Phonetics}{信心}{xin4xin1}[][HSK 2]
    \definition[个]{s.}{confiança; fé (em alguém ou algo); a crença de que os desejos se tornarão realidade}
  \end{Phonetics}
\end{Entry}

\begin{Entry}{信号}{9,5}{⼈,⼝}
  \begin{Phonetics}{信号}{xin4hao4}[][HSK 2]
    \definition[个,道]{s.}{sinal; luz, ondas de rádio, som, movimento, etc. usados para transmitir mensagens ou comandos | ponte de sinalização; marcação para chamar a atenção, ajudar na identificação e na memória}
  \end{Phonetics}
\end{Entry}

\begin{Entry}{信号灯}{9,5,6}{⼈,⼝,⽕}
  \begin{Phonetics}{信号灯}{xin4hao4deng1}
    \definition[盏]{s.}{lâmpada de sinalização; semáforo | luz de sinalização}
  \end{Phonetics}
\end{Entry}

\begin{Entry}{信用}{9,5}{⼈,⽤}
  \begin{Phonetics}{信用}{xin4yong4}[][HSK 6]
    \definition{adj.}{crédito; não é necessária nenhuma garantia material e o dinheiro pode ser reembolsado no prazo}
    \definition[些]{s.}{crédito; confiabilidade; a confiança que você ganha ao fazer o que prometeu | crédito; uma relação de empréstimo-pagamento ou situação em que o empréstimo é condicionado ao pagamento; uma situação em que um banco empresta dinheiro temporariamente a um cliente e este posteriormente devolve o dinheiro ao banco}
  \end{Phonetics}
\end{Entry}

\begin{Entry}{信用卡}{9,5,5}{⼈,⽤,⼘}
  \begin{Phonetics}{信用卡}{xin4yong4ka3}[][HSK 2]
    \definition[张]{s.}{cartão de crédito; moeda eletrônica emitida por um banco ou outra instituição especializada para consumidores; os titulares do cartão podem usá-lo para sacar dinheiro ou fazer compras de acordo com os regulamentos}
  \end{Phonetics}
\end{Entry}

\begin{Entry}{信仰}{9,6}{⼈,⼈}
  \begin{Phonetics}{信仰}{xin4yang3}[][HSK 6]
    \definition[种]{s.}{crença; religião; refere"-se à ideia de acreditar, adorar e tomar algo como padrão e guia para palavras e ações}
    \definition{v.}{acreditar; crer em; acreditar e adorar uma determinada religião ou doutrina e tomá"-la como guia para palavras e ações}
  \end{Phonetics}
\end{Entry}

\begin{Entry}{信任}{9,6}{⼈,⼈}
  \begin{Phonetics}{信任}{xin4ren4}[][HSK 3]
    \definition{s.}{confiança; um estado mental positivo e conexão emocional}
    \definition{v.}{confiar; ter confiança em; acreditar e ousar confiar}
  \end{Phonetics}
\end{Entry}

\begin{Entry}{信访}{9,6}{⼈,⾔}
  \begin{Phonetics}{信访}{xin4fang3}
    \definition{s.}{carta de reclamação | carta de petição}
  \seealsoref{上访}{shang4fang3}
  \end{Phonetics}
\end{Entry}

\begin{Entry}{信念}{9,8}{⼈,⼼}
  \begin{Phonetics}{信念}{xin4nian4}[][HSK 5]
    \definition[个,种]{s.}{fé; crença; convicção; concepções consideradas corretas e acreditadas com convicção}
  \end{Phonetics}
\end{Entry}

\begin{Entry}{信经}{9,8}{⼈,⽷}
  \begin{Phonetics}{信经}{xin4jing1}
    \definition[个]{s.}{crença | credo (seção da missa católica)}
  \end{Phonetics}
\end{Entry}

\begin{Entry}{信封}{9,9}{⼈,⼨}
  \begin{Phonetics}{信封}{xin4feng1}[][HSK 3]
    \definition[个,封]{s.}{envelope para cartas}
  \end{Phonetics}
\end{Entry}

\begin{Entry}{信息}{9,10}{⼈,⼼}
  \begin{Phonetics}{信息}{xin4xi1}[][HSK 2]
    \definition[个,条,段,些]{s.}{notícias; informações; as últimas notícias sobre alguém ou alguma coisa | mensagem; informação; na teoria da informação, uma mensagem transmitida usando símbolos, cujo conteúdo é desconhecido pelo receptor}
  \end{Phonetics}
\end{Entry}

\begin{Entry}{信箱}{9,15}{⼈,⾋}
  \begin{Phonetics}{信箱}{xin4xiang1}[][HSK 5]
    \definition{s.}{caixa de correio; caixa postal instalada pelos correios para que as pessoas possam depositar cartas | caixa postal; caixas com números, localizadas nos correios, que podem ser alugadas para receber correspondência; chamadas de caixas postais exclusivas}
  \end{Phonetics}
\end{Entry}

%%%%%%%%%% 俩 %%%%%%%%%%
\subsection*{俩}\addcontentsline{loh}{figure}{俩}

\begin{Entry}{俩}{9}{⼈}
  \begin{Phonetics}{俩}{lia3}[][HSK 4]
    \definition{num.}{ambos; dois; contração de 两个 | alguns; vários; refere"-se a um pequeno número}
  \end{Phonetics}
\end{Entry}

\begin{Entry}{俩钱}{9,10}{⼈,⾦}
  \begin{Phonetics}{俩钱}{lia3qian2}
    \definition{s.}{uma pequena quantia de dinheiro}
  \end{Phonetics}
\end{Entry}

%%%%%%%%%% 俭 %%%%%%%%%%
\subsection*{俭}\addcontentsline{loh}{figure}{俭}

\begin{Entry}{俭}{9}{⼈}
  \begin{Phonetics}{俭}{jian3}
    \definition*{s.}{Sobrenome: Jian}
    \definition{adj.}{econômico; frugal | querendo; faltando; curto}
  \end{Phonetics}
\end{Entry}

\begin{Entry}{俭省}{9,9}{⼈,⽬}
  \begin{Phonetics}{俭省}{jian3sheng3}
    \definition{adj.}{econômico}
  \end{Phonetics}
\end{Entry}

%%%%%%%%%% 修 %%%%%%%%%%
\subsection*{修}\addcontentsline{loh}{figure}{修}

\begin{Entry}{修}{9}{⼈}
  \begin{Phonetics}{修}{xiu1}[][HSK 3]
    \definition*{s.}{Sobrenome: Xiu}
    \definition{adj.}{comprido; alto e esbelto}
    \definition{s.}{revisionismo}
    \definition{v.}{embelezar; decorar | consertar; reparar; reformar | escrever; redigir; compilar | estudar; cultivar; aprender e praticar para aperfeiçoar ou melhorar (o caráter e o conhecimento) | construir; edificar | cortar ou aparar, para deixar bonito e arrumado | dedicar-se à prática da religião}
  \end{Phonetics}
\end{Entry}

\begin{Entry}{修车}{9,4}{⼈,⾞}
  \begin{Phonetics}{修车}{xiu1che1}[][HSK 6]
    \definition{v.}{consertar uma bicicleta (carro etc.)}[我打算明天去修车。===Pretendo consertar meu carro amanhã.]
  \end{Phonetics}
\end{Entry}

\begin{Entry}{修改}{9,7}{⼈,⽁}
  \begin{Phonetics}{修改}{xiu1gai3}[][HSK 3]
    \definition{v.}{revisar; retocar; corrigir erros e falhas em artigos, planos, etc.}
  \end{Phonetics}
\end{Entry}

\begin{Entry}{修建}{9,8}{⼈,⼵}
  \begin{Phonetics}{修建}{xiu1jian4}[][HSK 5]
    \definition{v.}{construir; erguer; animar; edificar; construir com tijolos, telhas, madeira, cimento, areia, etc.}
  \end{Phonetics}
\end{Entry}

\begin{Entry}{修规}{9,8}{⼈,⾒}
  \begin{Phonetics}{修规}{xiu1gui1}
    \definition{s.}{plano de construção}
  \end{Phonetics}
\end{Entry}

\begin{Entry}{修养}{9,9}{⼈,⼋}
  \begin{Phonetics}{修养}{xiu1yang3}[][HSK 5]
    \definition[种]{s.}{treinamento; domínio; realização; refere"-se a um determinado nível em termos de teoria, conhecimento, arte, pensamento, etc. | auto"-cultivo; refere"-se à atitude e ao comportamento cultivados ao longo do tempo, em conformidade com as exigências sociais}
  \end{Phonetics}
\end{Entry}

\begin{Entry}{修复}{9,9}{⼈,⼢}
  \begin{Phonetics}{修复}{xiu1fu4}[][HSK 5]
    \definition{v.}{reparar; restaurar; renovar | reparar; melhorar e restaurar (o relacionamento)}
  \end{Phonetics}
\end{Entry}

\begin{Entry}{修理}{9,11}{⼈,⽟}
  \begin{Phonetics}{修理}{xiu1li3}[][HSK 4]
    \definition{v.}{consertar; reparar; restaurar algo danificado à sua forma ou função original | aparar; podar; cortar com tesouras e outras ferramentas para deixar árvores, flores, cabelos, etc. arrumados | culpar; punir; criticar ou punir uma pessoa para mostrar que ela está errada}
  \end{Phonetics}
\end{Entry}

%%%%%%%%%% 养 %%%%%%%%%%
\subsection*{养}\addcontentsline{loh}{figure}{养}

\begin{Entry}{养}{9}{⼋}
  \begin{Phonetics}{养}{yang3}[][HSK 2]
    \definition*{s.}{Sobrenome: Yang}
    \definition{adj.}{adotivo; órfão; adotado; não biológico}
    \definition{s.}{qualidade; (caráter moral, desempenho acadêmico, etc.) boas qualidades}
    \definition{v.}{apoiar; prover; fornecer dinheiro e materiais necessários para viver | aumentar; manter; crescer; alimentar os animais e cuidar de suas vidas para que possam crescer | dar à luz | formar; adquirir; cultivar | descansar; curar; convalescer; recuperar a saúde | manter; manter em bom estado | deixar (o cabelo) crescer | ajudar; apoiar | cultivar (plantações ou flores)}
  \end{Phonetics}
\end{Entry}

\begin{Entry}{养分}{9,4}{⼋,⼑}
  \begin{Phonetics}{养分}{yang3fen4}
    \definition{s.}{nutriente}
  \end{Phonetics}
\end{Entry}

\begin{Entry}{养成}{9,6}{⼋,⼽}
  \begin{Phonetics}{养成}{yang3cheng2}[][HSK 4]
    \definition{v.}{cultivar; desenvolver; cultivar para formar; nutrir para crescer}
  \end{Phonetics}
\end{Entry}

\begin{Entry}{养老}{9,6}{⼋,⽼}
  \begin{Phonetics}{养老}{yang3 lao3}[][HSK 6]
    \definition{v.}{prover assistência aos idosos (geralmente os pais) | viver a vida na aposentadoria; refere"-se ao idoso que descansa em casa}
  \end{Phonetics}
\end{Entry}

\begin{Entry}{养料}{9,10}{⼋,⽃}
  \begin{Phonetics}{养料}{yang3liao4}
    \definition{s.}{nutrição}
  \end{Phonetics}
\end{Entry}

%%%%%%%%%% 冒 %%%%%%%%%%
\subsection*{冒}\addcontentsline{loh}{figure}{冒}

\begin{Entry}{冒}{9}{⽇}
  \begin{Phonetics}{冒}{mao4}[][HSK 5]
    \definition*{s.}{Sobrenome: Mao}
    \definition{adv.}{com ousadia; precipitadamente | fingidamente; falsamente; fraudulentamente}
    \definition{v.}{emitir; liberar; enviar (para cima) | arriscar; ser corajoso}
  \end{Phonetics}
\end{Entry}

\begin{Entry}{冒犯}{9,5}{⽇,⽝}
  \begin{Phonetics}{冒犯}{mao4fan4}[][HSK 7-9]
    \definition{v.}{ofender; afrontar}
  \end{Phonetics}
\end{Entry}

\begin{Entry}{冒充}{9,6}{⽇,⼉}
  \begin{Phonetics}{冒充}{mao4chong1}[][HSK 7-9]
    \definition{v.}{fingir ser; fazer passar alguém/algo por; confundir o falso com o verdadeiro; confundir o mau com o bom}
  \end{Phonetics}
\end{Entry}

\begin{Entry}{冒昧}{9,9}{⽇,⽇}
  \begin{Phonetics}{冒昧}{mao4mei4}[][HSK 7-9]
    \definition{v.}{ter a ousadia de fazer algo; tomar a liberdade de; descreve as palavras e ações de alguém como frívolas, desconsiderando seu status, posição ou a ocasião (frequentemente usado como uma expressão de humildade)}
  \end{Phonetics}
\end{Entry}

\begin{Entry}{冒险}{9,9}{⽇,⾩}
  \begin{Phonetics}{冒险}{mao4/xian3}[][HSK 7-9]
    \definition{v.+compl.}{arriscar; aventurar-se; correr um risco; assumir um risco}
  \end{Phonetics}
\end{Entry}

%%%%%%%%%% 冠 %%%%%%%%%%
\subsection*{冠}\addcontentsline{loh}{figure}{冠}

\begin{Entry}{冠}{9}{⼍}
  \begin{Phonetics}{冠}{guan1}
    \definition{s.}{chapéu | corona; coroa; copa | crista}
  \end{Phonetics}
  \begin{Phonetics}{冠}{guan4}
    \definition*{s.}{Sobrenome: Guan}
    \definition{s.}{primeiro lugar; o melhor; classificado em primeiro lugar}
    \definition{v.}{colocar um chapéu (boné) | preceder com (por); coroar com; adicionar um nome ou texto na frente}
  \end{Phonetics}
\end{Entry}

\begin{Entry}{冠军}{9,6}{⼍,⼍}
  \begin{Phonetics}{冠军}{guan4jun1}[][HSK 5]
    \definition[位,名,项,个]{s.}{campeão; medalhista de ouro; primeiro lugar em esportes e outras competições}
  \end{Phonetics}
\end{Entry}

%%%%%%%%%% 前 %%%%%%%%%%
\subsection*{前}\addcontentsline{loh}{figure}{前}

\begin{Entry}{前}{9}{⼑}
  \begin{Phonetics}{前}{qian2}[][HSK 1]
    \definition*{s.}{Sobrenome: Qian}
    \definition{s.}{frente | futuro; perspectiva | atrás; antes; mais cedo do que uma coisa ou um momento | à frente; para a frente; na parte frontal (referindo"-se ao espaço) | precedente; antes que algo aconteça | antigo; antigamente | topo; primeiro; primeiro na ordem | frente; campo de batalha | A.C. (Antes de Cristo)}[前293年===293 a.C.]
    \definition{v.}{seguir em frente; ir em frente}
  \seealsoref{公元}{gong1yuan2}
  \antonymref{后}{hou4}
  \end{Phonetics}
\end{Entry}

\begin{Entry}{前夕}{9,3}{⼑,⼣}
  \begin{Phonetics}{前夕}{qian2xi1}[][HSK 7-9]
    \definition{s.}{véspera; na noite anterior | a véspera; geralmente se refere ao período de tempo imediatamente anterior à ocorrência de um evento ou ao momento em que um evento está prestes a ocorrer}
  \end{Phonetics}
\end{Entry}

\begin{Entry}{前不久}{9,4,3}{⼑,⼀,⼃}
  \begin{Phonetics}{前不久}{qian2bu4jiu3}[][HSK 7-9]
    \definition{adv.}{não faz muito tempo | não muito tempo antes}
  \end{Phonetics}
\end{Entry}

\begin{Entry}{前天}{9,4}{⼑,⼤}
  \begin{Phonetics}{前天}{qian2tian1}[][HSK 1]
    \definition{adv.}{anteontem; dia anterior a ontem}
  \end{Phonetics}
\end{Entry}

\begin{Entry}{前方}{9,4}{⼑,⽅}
  \begin{Phonetics}{前方}{qian2fang1}[][HSK 6]
    \definition{s.}{frente; o espaço à frente; a direção voltada para a frente; a frente | linha de frente; frente de batalha; áreas onde os exércitos de ambos os lados estão se aproximando ou lutando}
  \antonymref{后方}{hou4 fang1}
  \end{Phonetics}
\end{Entry}

\begin{Entry}{前无古人}{9,4,5,2}{⼑,⽆,⼝,⼈}
  \begin{Phonetics}{前无古人}{qian2wu2gu3ren2}[][HSK 7-9]
    \definition[位,名,个,些]{expr.}{``Sem precedentes.''; refere"-se a algo que nunca foi possuído ou alcançado antes; algo sem precedentes; sem paralelo na história}
  \end{Phonetics}
\end{Entry}

\begin{Entry}{前台}{9,5}{⼑,⼝}
  \begin{Phonetics}{前台}{qian2tai2}[][HSK 7-9]
    \definition[个]{s.}{proscênio; trabalho diverso para uma apresentação; refere"-se a diversas tarefas administrativas relacionadas ao desempenho | palco; frente do palco; a parte da frente do palco, voltada para a plateia, é onde os atores se apresentam; geralmente, ela é separada da área dos bastidores por cortinas ou outras barreiras | (em um hotel) balcão de recepção; balcões de atendimento em restaurantes, casas noturnas, hotéis, etc., responsáveis pela recepção, cadastro e pagamento | lugar público; referindo"-se metaforicamente a uma ocasião pública (frequentemente usado em sentido pejorativo)}
  \end{Phonetics}
\end{Entry}

\begin{Entry}{前头}{9,5}{⼑,⼤}
  \begin{Phonetics}{前头}{qian2tou5}[][HSK 4]
    \definition{s.}{à frente; na frente; adiante}
  \end{Phonetics}
\end{Entry}

\begin{Entry}{前边}{9,5}{⼑,⾡}
  \begin{Phonetics}{前边}{qian2bian5}[][HSK 1]
    \definition{adv.}{à frente; na frente}
  \end{Phonetics}
\end{Entry}

\begin{Entry}{前仰后合}{9,6,6,6}{⼑,⼈,⼝,⼝}
  \begin{Phonetics}{前仰后合}{qian2yang3-hou4he2}[][HSK 7-9]
    \definition{expr.}{``Inclinando-se para a frente e para trás.''; balançar (com risos); balançar para frente e para trás}
  \end{Phonetics}
\end{Entry}

\begin{Entry}{前任}{9,6}{⼑,⼈}
  \begin{Phonetics}{前任}{qian2ren4}[][HSK 7-9]
    \definition[个,位,名]{s.}{predecessor; a pessoa que ocupava este cargo anteriormente}
  \end{Phonetics}
\end{Entry}

\begin{Entry}{前后}{9,6}{⼑,⼝}
  \begin{Phonetics}{前后}{qian2hou4}[][HSK 3]
    \definition{s.}{em volta; sobre; um período de tempo ligeiramente anterior ou posterior a um horário específico| do início ao fim; refere"-se ao período de tempo do início ao fim de algo | frente e verso; na frente e atrás de algo}
  \end{Phonetics}
\end{Entry}

\begin{Entry}{前年}{9,6}{⼑,⼲}
  \begin{Phonetics}{前年}{qian2nian2}[][HSK 2]
    \definition{adv.}{há dois anos; dois anos atrás}
  \end{Phonetics}
\end{Entry}

\begin{Entry}{前来}{9,7}{⼑,⽊}
  \begin{Phonetics}{前来}{qian2lai2}[][HSK 6]
    \definition{v.}{vir; em direção à localização e direção do falante}
  \end{Phonetics}
\end{Entry}

\begin{Entry}{前进}{9,7}{⼑,⾡}
  \begin{Phonetics}{前进}{qian2jin4}[][HSK 3]
    \definition{v.}{marchar; avançar; para ir em frente; seguir em frente; geralmente se refere ao desenvolvimento futuro}
  \end{Phonetics}
\end{Entry}

\begin{Entry}{前往}{9,8}{⼑,⼻}
  \begin{Phonetics}{前往}{qian2wang3}[][HSK 3]
    \definition{v.}{ir para; prosseguir para; partir para; ir em frente}
  \end{Phonetics}
\end{Entry}

\begin{Entry}{前所未有}{9,8,5,6}{⼑,⼾,⽊,⽉}
  \begin{Phonetics}{前所未有}{qian2suo3wei4you3}[][HSK 7-9]
    \definition{expr.}{``Sem precedentes.''; nunca existiu antes; até então desconhecido; nunca antes na história}
  \end{Phonetics}
\end{Entry}

\begin{Entry}{前沿}{9,8}{⼑,⽔}
  \begin{Phonetics}{前沿}{qian2yan2}[][HSK 7-9]
    \definition{s.}{Militar: posição avançada | Figurativo: fronteira na pesquisa científica | fronteira (da ciência, tecnologia etc.) | posto avançado}
  \end{Phonetics}
\end{Entry}

\begin{Entry}{前线}{9,8}{⼑,⽷}
  \begin{Phonetics}{前线}{qian2xian4}[][HSK 7-9]
    \definition{s.}{linha de frente; frente | frente de batalha; a área onde os dois exércitos se aproximam durante uma batalha}
  \antonymref{后方}{hou4 fang1}
  \end{Phonetics}
\end{Entry}

\begin{Entry}{前者}{9,8}{⼑,⽼}
  \begin{Phonetics}{前者}{qian2zhe3}[][HSK 7-9]
    \definition{adj.}{antigo; anterior; primeiro; precedente; refere"-se à primeira das duas coisas ou pessoas listadas acima}
  \antonymref{后者}{hou4zhe3}
  \end{Phonetics}
\end{Entry}

\begin{Entry}{前赴后继}{9,9,6,10}{⼑,⾛,⼝,⽷}
  \begin{Phonetics}{前赴后继}{qian2fu4-hou4ji4}[][HSK 7-9]
    \definition{expr.}{``Um após o outro.''; avançar destemidamente em onda após onda; os que estão na frente sobem, e os que estão atrás os seguem, o que demonstra um espírito de progresso entusiasmado e contínuo}
  \end{Phonetics}
\end{Entry}

\begin{Entry}{前面}{9,9}{⼑,⾯}
  \begin{Phonetics}{前面}{qian2mian5}[][HSK 3]
    \definition{s.}{frente; a parte frontal do espaço ou posição | parte anterior; acima; a parte que vem primeiro na ordem; a parte de um artigo ou discurso que precede a narração atual}
  \end{Phonetics}
\end{Entry}

\begin{Entry}{前途}{9,10}{⼑,⾡}
  \begin{Phonetics}{前途}{qian2tu2}[][HSK 4]
    \definition[片,段,种]{s.}{futuro; perspectiva; prospecto; originalmente, refere"-se à jornada à frente, mas, metaforicamente, refere"-se ao futuro}
  \end{Phonetics}
\end{Entry}

\begin{Entry}{前提}{9,12}{⼑,⼿}
  \begin{Phonetics}{前提}{qian2ti2}[][HSK 5]
    \definition[个,项]{s.}{premissa; pressuposto | pré-requisito; pressuposição; condições prévias para que algo aconteça ou se desenvolva}
  \end{Phonetics}
\end{Entry}

\begin{Entry}{前景}{9,12}{⼑,⽇}
  \begin{Phonetics}{前景}{qian2jing3}[][HSK 5]
    \definition{s.}{primeiro plano (de uma vista, imagem, foto, etc.); as imagens que parecem mais próximas do espectador em pinturas, palcos e telas | vista; perspectiva; prospecto; ponto de vista; situações que podem ocorrer no trabalho, na carreira, etc.}
  \end{Phonetics}
\end{Entry}

\begin{Entry}{前期}{9,12}{⼑,⽉}
  \begin{Phonetics}{前期}{qian2qi1}[][HSK 7-9]
    \definition{s.}{estágio inicial; primeiros dias; prófase; a etapa anterior de um determinado período}
  \end{Phonetics}
\end{Entry}

\begin{Entry}{前辈}{9,12}{⼑,⾞}
  \begin{Phonetics}{前辈}{qian2bei4}[][HSK 7-9]
    \definition{s.}{idoso; sênior; geração mais velha; refere"-se a pessoas mais velhas ou com mais experiência na mesma indústria ou área de atuação | predecessor; antepassados; ancestrais; a geração anterior ou gerações anteriores, também se referindo aos ancestrais}
  \end{Phonetics}
\end{Entry}

%%%%%%%%%% 剑 %%%%%%%%%%
\subsection*{剑}\addcontentsline{loh}{figure}{剑}

\begin{Entry}{剑}{9}{⼑}
  \begin{Phonetics}{剑}{jian4}[][HSK 6]
    \definition[把,口]{s.}{espada; sabre; florete}
  \end{Phonetics}
\end{Entry}

\begin{Entry}{剑客}{9,9}{⼑,⼧}
  \begin{Phonetics}{剑客}{jian4ke4}
    \definition{s.}{espada | esgrimista, espadachim}
  \end{Phonetics}
\end{Entry}

%%%%%%%%%% 勇 %%%%%%%%%%
\subsection*{勇}\addcontentsline{loh}{figure}{勇}

\begin{Entry}{勇}{9}{⼒}
  \begin{Phonetics}{勇}{yong3}
    \definition*{s.}{Sobrenome: Yong}
    \definition{adj.}{bravo; valente; corajoso}
    \definition{s.}{recrutas temporários em tempos de guerra na Dinastia Qing}
  \end{Phonetics}
\end{Entry}

\begin{Entry}{勇士}{9,3}{⼒,⼠}
  \begin{Phonetics}{勇士}{yong3shi4}
    \definition{s.}{um guerreiro | uma pessoa corajosa}
  \end{Phonetics}
\end{Entry}

\begin{Entry}{勇气}{9,4}{⼒,⽓}
  \begin{Phonetics}{勇气}{yong3qi4}[][HSK 4]
    \definition[种,股]{s.}{coragem; arrojo; nervos; coragem para agir sem medo}
  \end{Phonetics}
\end{Entry}

\begin{Entry}{勇敢}{9,11}{⼒,⽁}
  \begin{Phonetics}{勇敢}{yong3gan3}[][HSK 4]
    \definition{adj.}{bravo; valente; galante; corajoso}
  \end{Phonetics}
\end{Entry}

%%%%%%%%%% 勉 %%%%%%%%%%
\subsection*{勉}\addcontentsline{loh}{figure}{勉}

\begin{Entry}{勉}{9}{⼒}
  \begin{Phonetics}{勉}{mian3}
    \definition{s.}{Sobrenome: Mian}
    \definition{v.}{esforçar-se; esforçar-se para; lutar | encorajar; instar; exortar | forçar-se a fazer algo; esforçar-se para fazer o que está além de suas capacidades; fazer algo contra a vontade; esforçar-se para trabalhar arduamente; empenhar-se em realizar trabalhos árduos; dar o seu melhor, mesmo que não tenha forças suficientes}
  \end{Phonetics}
\end{Entry}

\begin{Entry}{勉强}{9,12}{⼒,⼸}
  \begin{Phonetics}{勉强}{mian3qiang3}[][HSK 7-9]
    \definition{adj.}{relutante; indisposto; a contragosto; não muito disposto | inadequado; pouco convincente; forçado; inverossímil | mal o suficiente; mal dá para\dots}[这些钱勉强够用一周的。===Esse dinheiro mal dá para uma semana.]
    \definition{v.}{empurrar; forçar alguém a fazer algo; obrigar alguém a fazer algo que não quer fazer}
  \end{Phonetics}
\end{Entry}

%%%%%%%%%% 南 %%%%%%%%%%
\subsection*{南}\addcontentsline{loh}{figure}{南}

\begin{Entry}{南}{9}{⼗}
  \begin{Phonetics}{南}{nan2}[][HSK 1]
    \definition*{s.}{Sobrenome: Nan}
    \definition{s.}{sul; uma das quatro direções básicas, o lado direito quando se está de frente para o sol pela manhã | especificamente no sul da China}
  \antonymref{北}{bei3}
  \end{Phonetics}
\end{Entry}

\begin{Entry}{南方}{9,4}{⼗,⽅}
  \begin{Phonetics}{南方}{nan2fang1}[][HSK 2]
    \definition{s.}{sul; indica a direção sul | o sul; a região sul}
  \end{Phonetics}
\end{Entry}

\begin{Entry}{南北}{9,5}{⼗,⼔}
  \begin{Phonetics}{南北}{nan2bei3}[][HSK 5]
    \definition{s.}{(território) norte e sul | (distância) de norte a sul}
  \end{Phonetics}
\end{Entry}

\begin{Entry}{南瓜}{9,5}{⼗,⽠}
  \begin{Phonetics}{南瓜}{nan2gua1}[][HSK 7-9]
    \definition[个,斤,块]{s.}{abóbora}
  \end{Phonetics}
\end{Entry}

\begin{Entry}{南边}{9,5}{⼗,⾡}
  \begin{Phonetics}{南边}{nan2bian5}[][HSK 1]
    \definition{s.}{sul; lado sul}
  \end{Phonetics}
\end{Entry}

\begin{Entry}{南极}{9,7}{⼗,⽊}
  \begin{Phonetics}{南极}{nan2ji2}[][HSK 5]
    \definition*{s.}{Polo Sul; Polo Antártico | Polo sul magnético}
    \definition{s.}{polo sul magnético}
  \end{Phonetics}
\end{Entry}

\begin{Entry}{南京}{9,8}{⼗,⼇}
  \begin{Phonetics}{南京}{nan2jing1}
    \definition*{s.}{Nanquim, capital da província de Jiangsu, 江苏}
  \seealsoref{江苏}{jiang1su1}
  \end{Phonetics}
\end{Entry}

\begin{Entry}{南面}{9,9}{⼗,⾯}
  \begin{Phonetics}{南面}{nan2mian4}
    \definition{s.}{sul | lado sul}
  \end{Phonetics}
\end{Entry}

\begin{Entry}{南部}{9,10}{⼗,⾢}
  \begin{Phonetics}{南部}{nan2bu4}[][HSK 3]
    \definition{s.}{parte sul; sul | a parte sul}
  \end{Phonetics}
\end{Entry}

%%%%%%%%%% 厘 %%%%%%%%%%
\subsection*{厘}\addcontentsline{loh}{figure}{厘}

\begin{Entry}{厘}{9}{⼚}
  \begin{Phonetics}{厘}{li2}
    \definition*{s.}{Sobrenome: Li}
    \definition{clas.}{li, uma unidade tradicional de comprimento, igual a 0,001 chi (市尺), e equivalente a 0,333 milímetro ou 0,013 polegada | li, uma unidade tradicional de peso, igual a 0,0001 jin (市斤), e equivalente a 5 centigramas ou 0,771 grãos | li, uma unidade tradicional de área, igual a 0,01 mu (市亩), e equivalente a 0,667 metro quadrado ou 0,797 jarda quadrada | li, unidade monetária chinesa, igual a 0,1 fen ou 0,001 yuan | li, unidade de taxa de juros, igual a 0,1\% de juros mensais ou 1\% de juros anuais | quantidade muito pequena; fração; o mínimo}
    \definition{v.}{regular; retificar | administrar}
  \seealsoref{市尺}{shi4 chi3}
  \seealsoref{市斤}{shi4jin1}
  \seealsoref{市亩}{shi4mu3}
  \end{Phonetics}
\end{Entry}

\begin{Entry}{厘米}{9,6}{⼚,⽶}
  \begin{Phonetics}{厘米}{li2mi3}[][HSK 4]
    \definition{clas.}{centímetro; unidade de comprimento, símbolo cm, 1 metro é igual a 100 centímetros}
  \end{Phonetics}
\end{Entry}

%%%%%%%%%% 厚 %%%%%%%%%%
\subsection*{厚}\addcontentsline{loh}{figure}{厚}

\begin{Entry}{厚}{9}{⼚}
  \begin{Phonetics}{厚}{hou4}[][HSK 4]
    \definition*{s.}{Sobrenome: Hou}
    \definition{adj.}{espesso; grosso | profundo | gentil; magnânimo | grande; generoso | rico ou forte em sabor}
    \definition[米,厘米]{s.}{espessura | profundidade}
    \definition{v.}{favorecer; enfatizar}
  \antonymref{薄}{bao2}
  \end{Phonetics}
\end{Entry}

\begin{Entry}{厚度}{9,9}{⼚,⼴}
  \begin{Phonetics}{厚度}{hou4du4}[][HSK 7-9]
    \definition{s.}{espessura; a distância entre a parte superior e inferior de um objeto plano}
  \end{Phonetics}
\end{Entry}

\begin{Entry}{厚道}{9,12}{⼚,⾡}
  \begin{Phonetics}{厚道}{hou4dao5}[][HSK 7-9]
    \definition{adj.}{honesto e gentil; sincero e generoso}
  \end{Phonetics}
\end{Entry}

%%%%%%%%%% 叛 %%%%%%%%%%
\subsection*{叛}\addcontentsline{loh}{figure}{叛}

\begin{Entry}{叛}{9}{⼜}
  \begin{Phonetics}{叛}{pan4}
    \definition{adj.}{rebelde}
    \definition{s.}{rebelião}
    \definition{v.}{trair | rebelar-se | revoltar-se}
  \end{Phonetics}
\end{Entry}

\begin{Entry}{叛逆}{9,9}{⼜,⾡}
  \begin{Phonetics}{叛逆}{pan4ni4}[][HSK 7-9]
    \definition{s.}{rebelde; pessoas que traem}
    \definition{v.}{rebelar-se/revoltar-se contra; trair}
  \end{Phonetics}
\end{Entry}

%%%%%%%%%% 咧 %%%%%%%%%%
\subsection*{咧}\addcontentsline{loh}{figure}{咧}

\begin{Entry}{咧}{9}{⼝}
  \begin{Phonetics}{咧}{lie1}
    \definition{adv./interj./part./v.}{elemento formador de palavras}
  \end{Phonetics}
  \begin{Phonetics}{咧}{lie3}
    \definition{part.}{usado da mesma forma que 了, 啦 e 哩}
    \definition{v.}{sorrir}
  \seealsoref{啦}{la5}
  \seealsoref{了}{le5}
  \seealsoref{哩}{li5}
  \end{Phonetics}
  \begin{Phonetics}{咧}{lie5}
    \definition{part.}{usado da mesma forma que 了, 啦 e 哩}
  \seealsoref{啦}{la5}
  \seealsoref{了}{le5}
  \seealsoref{哩}{li5}
  \end{Phonetics}
\end{Entry}

\begin{Entry}{咧嘴}{9,16}{⼝,⼝}
  \begin{Phonetics}{咧嘴}{lie3/zui3}[][HSK 7-9]
    \definition{v.+compl.}{retrair os cantos da boca; sorrir}
  \end{Phonetics}
\end{Entry}

%%%%%%%%%% 咨 %%%%%%%%%%
\subsection*{咨}\addcontentsline{loh}{figure}{咨}

\begin{Entry}{咨}{9}{⼝}
  \begin{Phonetics}{咨}{zi1}
    \definition[行]{s.}{comunicação oficial; relatório entregue pelo chefe de um governo sobre assuntos de Estado}
    \definition{v.}{consultar; discutir com}
  \end{Phonetics}
\end{Entry}

\begin{Entry}{咨询}{9,8}{⼝,⾔}
  \begin{Phonetics}{咨询}{zi1xun2}[][HSK 6]
    \definition{v.}{consultar; aconselhar-se com; buscar conselho de; pedir conselhos}
  \end{Phonetics}
\end{Entry}

%%%%%%%%%% 咬 %%%%%%%%%%
\subsection*{咬}\addcontentsline{loh}{figure}{咬}

\begin{Entry}{咬}{9}{⼝}
  \begin{Phonetics}{咬}{yao3}[][HSK 5]
    \definition{v.}{morder; estalar; pressionar os dentes superiores e inferiores com força | latir | agarrar; morder | incriminar outra pessoa (geralmente inocente) quando culpada ou interrogada | pronunciar; articular; pronunciar corretamente | corroer (metais); irritar (a pele) | ser minucioso (com relação ao uso de palavras) | aproximar"-se de; pressionar em direção a; avançar sobre}
  \end{Phonetics}
\end{Entry}

%%%%%%%%%% 咱 %%%%%%%%%%
\subsection*{咱}\addcontentsline{loh}{figure}{咱}

\begin{Entry}{咱}{9}{⼝}
  \begin{Phonetics}{咱}{za2}
  \end{Phonetics}
  \begin{Phonetics}{咱}{zan2}[][HSK 2]
    \definition{pron.}{nós; nos (incluindo tanto o falante quanto a pessoa ou pessoas às quais se dirige) | eu; mim}
  \end{Phonetics}
  \begin{Phonetics}{咱}{zan5}
    \definition{adv.}{quando; agora; então; naquele momento; usado em 这咱, 那咱, 多咱, uma combinação das duas palavras 早晚}
  \seealsoref{多咱}{duo1 zan5}
  \seealsoref{那咱}{na4 zan5}
  \seealsoref{早晚}{zao3wan3}
  \seealsoref{这咱}{zhe4 zan5}
  \end{Phonetics}
\end{Entry}

\begin{Entry}{咱们}{9,5}{⼝,⼈}
  \begin{Phonetics}{咱们}{zan2men5}[][HSK 2]
    \definition{pron.}{dirige"-se tanto ao falante (eu, nós) quanto ao ouvinte (você, vocês) | eu; mim; refere"-se ao próprio orador, eu}
  \end{Phonetics}
\end{Entry}

\begin{Entry}{咱俩}{9,9}{⼝,⼈}
  \begin{Phonetics}{咱俩}{zan2lia3}
    \definition{pron.}{nós dois}
  \end{Phonetics}
\end{Entry}

\begin{Entry}{咱家}{9,10}{⼝,⼧}
  \begin{Phonetics}{咱家}{za2jia1}
    \definition{pron.}{eu (frequentemente usado na literatura vernácula antiga) | me | mim | comigo}
  \end{Phonetics}
\end{Entry}

%%%%%%%%%% 咳 %%%%%%%%%%
\subsection*{咳}\addcontentsline{loh}{figure}{咳}

\begin{Entry}{咳}{9}{⼝}
  \begin{Phonetics}{咳}{hai1}
    \definition{interj.}{expressa tristeza, arrependimento ou espanto}
  \end{Phonetics}
  \begin{Phonetics}{咳}{ke2}[][HSK 5]
    \definition{v.}{tossir}
  \end{Phonetics}
\end{Entry}

\begin{Entry}{咳嗽}{9,14}{⼝,⼝}
  \begin{Phonetics}{咳嗽}{ke2sou5}[][HSK 7-9]
    \definition[次,声,阵,回]{s.}{tosse}
    \definition{v.}{ter tosse; tossir}
  \end{Phonetics}
\end{Entry}

%%%%%%%%%% 咸 %%%%%%%%%%
\subsection*{咸}\addcontentsline{loh}{figure}{咸}

\begin{Entry}{咸}{9}{⼝}
  \begin{Phonetics}{咸}{xian2}[][HSK 4]
    \definition*{s.}{Sobrenome: Xian}
    \definition{adj.}{salgado; em conserva; sabor salgado}
    \definition{adv.}{todos; indica a totalidade de um intervalo, equivalente a 全 e 都}
  \seealsoref{都}{dou1}
  \seealsoref{全}{quan2}
  \end{Phonetics}
\end{Entry}

\begin{Entry}{咸水}{9,4}{⼝,⽔}
  \begin{Phonetics}{咸水}{xian2shui3}
    \definition{s.}{salmora | água salgada}
  \end{Phonetics}
\end{Entry}

\begin{Entry}{咸肉}{9,6}{⼝,⾁}
  \begin{Phonetics}{咸肉}{xian2rou4}
    \definition{s.}{\emph{bacon} | carne curada com sal}
  \end{Phonetics}
\end{Entry}

\begin{Entry}{咸鱼}{9,8}{⼝,⿂}
  \begin{Phonetics}{咸鱼}{xian2yu2}
    \definition{s.}{peixe salgado}
  \end{Phonetics}
\end{Entry}

\begin{Entry}{咸涩}{9,10}{⼝,⽔}
  \begin{Phonetics}{咸涩}{xian2se4}
    \definition{s.}{ácido | salgado e amargo}
  \end{Phonetics}
\end{Entry}

\begin{Entry}{咸盐}{9,10}{⼝,⽫}
  \begin{Phonetics}{咸盐}{xian2yan2}
    \definition{s.}{(coloquial) sal | sal de mesa}
  \end{Phonetics}
\end{Entry}

\begin{Entry}{咸淡}{9,11}{⼝,⽔}
  \begin{Phonetics}{咸淡}{xian2dan4}
    \definition{s.}{água salobra | grau de salinidade | salgado e sem sal (sabores)}
  \end{Phonetics}
\end{Entry}

\begin{Entry}{咸菜}{9,11}{⼝,⾋}
  \begin{Phonetics}{咸菜}{xian2cai4}
    \definition{s.}{legumes salgados | \emph{pickles}}
  \end{Phonetics}
\end{Entry}

%%%%%%%%%% 哀 %%%%%%%%%%
\subsection*{哀}\addcontentsline{loh}{figure}{哀}

\begin{Entry}{哀}{9}{⼝}
  \begin{Phonetics}{哀}{ai1}
    \definition*{s.}{Sobrenome: Ai}
    \definition{adj.}{triste; pesaroso}
    \definition{adv.}{tristemente; lamentavelmente}
    \definition{s.}{luto | tristeza; pesar | pena; misericórdia}
    \definition{v.}{lamentar; lamentar"-se por | Literário: estar triste}
  \synonymref{悲}{bei1}
  \antonymref{乐}{le4}
  \end{Phonetics}
\end{Entry}

\begin{Entry}{哀求}{9,7}{⼝,⽔}
  \begin{Phonetics}{哀求}{ai1qiu2}[][HSK 7-9]
    \definition{v.}{suplicar; implorar | suplicar; implorar piedosamente}
  \synonymref{恳求}{ken3qiu2}
  \synonymref{乞求}{qi3qiu2}
  \synonymref{请求}{qing3qiu2}
  \synonymref{要求}{yao1qiu2}
  \antonymref{逼迫}{bi1po4}
  \antonymref{逼狭}{bi1xia2}
  \antonymref{命令}{ming4ling4}
  \end{Phonetics}
\end{Entry}

%%%%%%%%%% 品 %%%%%%%%%%
\subsection*{品}\addcontentsline{loh}{figure}{品}

\begin{Entry}{品}{9}{⼝}
  \begin{Phonetics}{品}{pin3}[][HSK 5]
    \definition*{s.}{Sobrenome: Pin}
    \definition{s.}{artigo; produto | grau; classe; classificação; nível | caráter; qualidade | classificação; os graus dos funcionários públicos antigos, num total de nove graus}
    \definition{v.}{provar; saborear; degustar algo com discernimento | soprar; tocar (instrumentos de sopro) | avaliar; distinguir o bom do ruim}
  \end{Phonetics}
\end{Entry}

\begin{Entry}{品行}{9,6}{⼝,⾏}
  \begin{Phonetics}{品行}{pin3xing2}[][HSK 7-9]
    \definition[位]{s.}{conduta moral; comportamento | caráter; conduta}
  \end{Phonetics}
\end{Entry}

\begin{Entry}{品位}{9,7}{⼝,⼈}
  \begin{Phonetics}{品位}{pin3wei4}[][HSK 7-9]
    \definition{s.}{qualidade; gosto; refere"-se à qualidade ou ao valor de uma pessoa ou coisa | grau; qualidade; a quantidade de um elemento desejado ou de seu composto contida em um minério (geralmente expressa em porcentagem) | classificação; na antiguidade, referia"-se à posição e ao cargo oficial}
  \end{Phonetics}
\end{Entry}

\begin{Entry}{品质}{9,8}{⼝,⾙}
  \begin{Phonetics}{品质}{pin3zhi4}[][HSK 4]
    \definition[个,种]{s.}{qualidade; caráter; natureza do pensamento, da compreensão, do caráter, etc., conforme expresso no comportamento, no estilo, etc. | qualidade (de produtos, mercadorias, etc.)}
  \end{Phonetics}
\end{Entry}

\begin{Entry}{品尝}{9,9}{⼝,⼩}
  \begin{Phonetics}{品尝}{pin3chang2}[][HSK 7-9]
    \definition{v.}{provar; degustar; saborear; saborear a comida devagar e com atenção}
  \end{Phonetics}
\end{Entry}

\begin{Entry}{品种}{9,9}{⼝,⽲}
  \begin{Phonetics}{品种}{pin3zhong3}[][HSK 5]
    \definition[个,些]{s.}{raça; linhagem; variedade; refere"-se a um grupo de organismos com características genéticas comuns, formados por meio da seleção e cultivo artificiais de culturas, gado, aves, etc. | variedade; sortimento; referência geral ao tipo de item}
  \end{Phonetics}
\end{Entry}

\begin{Entry}{品牌}{9,12}{⼝,⽚}
  \begin{Phonetics}{品牌}{pin3pai2}[][HSK 6]
    \definition[个,种]{s.}{marca registrada; nome de marca}
  \end{Phonetics}
\end{Entry}

\begin{Entry}{品德}{9,15}{⼝,⼻}
  \begin{Phonetics}{品德}{pin3de2}[][HSK 7-9]
    \definition[种]{s.}{caráter moral; qualidade; qualidade e moralidade}
  \end{Phonetics}
\end{Entry}

%%%%%%%%%% 哄 %%%%%%%%%%
\subsection*{哄}\addcontentsline{loh}{figure}{哄}

\begin{Entry}{哄}{9}{⼝}
  \begin{Phonetics}{哄}{hong1}[][HSK 7-9]
    \definition{interj.}{Onomatopéia: gargalhadas ou alvoroço}
    \definition{s.}{rugido; clamor; comoção}
  \end{Phonetics}
  \begin{Phonetics}{哄}{hong3}[][HSK 7-9]
    \definition{v.}{brincar; enganar; tapear | persuadir; agradar os outros com palavras ou ações, especialmente observando ou cuidando de crianças}
  \end{Phonetics}
  \begin{Phonetics}{哄}{hong4}[][HSK 7-9]
    \definition{s.}{comoção; tumulto}
  \end{Phonetics}
\end{Entry}

\begin{Entry}{哄堂大笑}{9,11,3,10}{⼝,⼟,⼤,⽵}
  \begin{Phonetics}{哄堂大笑}{hong1tang2-da4xiao4}[][HSK 7-9]
    \definition{expr.}{fazer a sala inteira rugir (em alvoroço); (causar) uma explosão geral de risos; ``Todos caíram na gargalhada.''; uma explosão de risos na plateia; ``A plateia caiu na gargalhada.''; ``As pessoas de toda a casa caíram na gargalhada.''; ``A sala inteira riu (balançando).''}
  \end{Phonetics}
\end{Entry}

%%%%%%%%%% 哆 %%%%%%%%%%
\subsection*{哆}\addcontentsline{loh}{figure}{哆}

\begin{Entry}{哆}{9}{⼝}
  \begin{Phonetics}{哆}{duo1}
    \definition{part.}{usado em 哆嗦}
  \seealsoref{哆嗦}{duo1suo5}
  \end{Phonetics}
\end{Entry}

\begin{Entry}{哆嗦}{9,13}{⼝,⼝}
  \begin{Phonetics}{哆嗦}{duo1suo5}[][HSK 7-9]
    \definition{v.}{tremer; estremecer (tremores corporais involuntários devido a estímulos externos)}
  \end{Phonetics}
\end{Entry}

%%%%%%%%%% 哇 %%%%%%%%%%
\subsection*{哇}\addcontentsline{loh}{figure}{哇}

\begin{Entry}{哇}{9}{⼝}
  \begin{Phonetics}{哇}{wa1}
    \definition{interj.}{Onomatopéia: som de choro ou vômito | ``Uau!''; expressa surpresa}
  \end{Phonetics}
  \begin{Phonetics}{哇}{wa5}[][HSK 6]
    \definition{part.}{a mudança do som de 啊 devido à influência do som final da palavra anterior, ``u'' ou ``ao''}
  \seealsoref{啊}{a5}
  \end{Phonetics}
\end{Entry}

\begin{Entry}{哇塞}{9,13}{⼝,⼟}
  \begin{Phonetics}{哇塞}{wa1sai1}
    \definition{interj.}{``Uau!'; exclamação de espanto, admiração, etc.}
  \end{Phonetics}
\end{Entry}

\begin{Entry}{哇噻}{9,16}{⼝,⼝}
  \begin{Phonetics}{哇噻}{wa1sai1}
    \variantof{哇塞}
  \end{Phonetics}
\end{Entry}

%%%%%%%%%% 哈 %%%%%%%%%%
\subsection*{哈}\addcontentsline{loh}{figure}{哈}

\begin{Entry}{哈}{9}{⼝}
  \begin{Phonetics}{哈}{ha1}
    \definition{interj.}{Onomatopéia: ``Ha''; descreve o riso, usado principalmente em duplicata | indica orgulho ou satisfação, frequentemente usado de forma duplicada}
    \definition{v.}{soprar; expirar (com a boca aberta) | dobrar}
  \seealsoref{哈哈}{ha1ha5}
  \end{Phonetics}
  \begin{Phonetics}{哈}{ha3}
    \definition*{s.}{Sobrenome: Ha}
    \definition{v.}{repreender}
  \end{Phonetics}
\end{Entry}

\begin{Entry}{哈马斯}{9,3,12}{⼝,⾺,⽄}
  \begin{Phonetics}{哈马斯}{ha1ma3si1}
    \definition*{s.}{Hamas (Grupo Palestino)}
  \end{Phonetics}
\end{Entry}

\begin{Entry}{哈哈}{9,9}{⼝,⼝}
  \begin{Phonetics}{哈哈}{ha1ha5}[][HSK 3]
    \definition{expr.}{Onomatopéia: ``Ha Ha''; o som de uma gargalhada}
  \end{Phonetics}
\end{Entry}

%%%%%%%%%% 响 %%%%%%%%%%
\subsection*{响}\addcontentsline{loh}{figure}{响}

\begin{Entry}{响}{9}{⼝}
  \begin{Phonetics}{响}{xiang3}[][HSK 2]
    \definition{adj.}{barulhento; ressonante}
    \definition[声,阵]{s.}{som; ruído; barulho | eco}
    \definition{v.}{tocar; soar; ressoar; fazer um som | soar; fazer algo emitir um som}
  \end{Phonetics}
\end{Entry}

\begin{Entry}{响声}{9,7}{⼝,⼠}
  \begin{Phonetics}{响声}{xiang3sheng1}[][HSK 6]
    \definition{s.}{som; ruído}
  \end{Phonetics}
\end{Entry}

\begin{Entry}{响亮}{9,9}{⼝,⼇}
  \begin{Phonetics}{响亮}{xiang3liang4}
    \definition{adj.}{vibrante; ressonante; sonoro; ressonante; alto e claro}
  \end{Phonetics}
\end{Entry}

%%%%%%%%%% 哗 %%%%%%%%%%
\subsection*{哗}\addcontentsline{loh}{figure}{哗}

\begin{Entry}{哗}{9}{⼝}
  \begin{Phonetics}{哗}{hua1}
    \definition{s.}{(onomatopéia) sons de impacto, batida, fluxo de água, etc.}
  \end{Phonetics}
  \begin{Phonetics}{哗}{hua2}
    \definition{v.}{ser barulhento; fazer alvoroço}
  \end{Phonetics}
\end{Entry}

\begin{Entry}{哗变}{9,8}{⼝,⼜}
  \begin{Phonetics}{哗变}{hua2bian4}[][HSK 7-9]
    \definition{s.}{(um exército) motim | rebelião}
  \end{Phonetics}
\end{Entry}

\begin{Entry}{哗啦啦}{9,11,11}{⼝,⼝,⼝}
  \begin{Phonetics}{哗啦啦}{hua1la1 la5}
    \definition{s.}{(onomatopéia) som de colisão, batida}
  \end{Phonetics}
\end{Entry}

\begin{Entry}{哗然}{9,12}{⼝,⽕}
  \begin{Phonetics}{哗然}{hua2ran2}[][HSK 7-9]
    \definition{adj.}{Literário: barulhento; em alvoroço; em comoção}[举座哗然。===Todo o público ficou em alvoroço.]
  \end{Phonetics}
\end{Entry}

%%%%%%%%%% 哪 %%%%%%%%%%
\subsection*{哪}\addcontentsline{loh}{figure}{哪}

\begin{Entry}{哪}{9}{⼝}
  \begin{Phonetics}{哪}{na3}[][HSK 1]
    \definition{adv.}{para expressar uma pergunta retórica, indicando que é impossível}
    \definition{pron.}{qual?; o que?; expressa a necessidade de determinar um entre várias pessoas ou coisas | qualquer; ser usado em um sentido geral | qual?; o que?; (usado sozinho, o mesmo que 什么, frequentemente usado de forma intercambiável com 什么) | qualquer; qualquer que seja; refere"-se a qualquer um, geralmente seguido por 都 ou 也, ou usando dois 哪 antes e depois | qual (indica algo incerto)}
  \seealsoref{都}{dou1}
  \seealsoref{什么}{shen2me5}
  \seealsoref{也}{ye3}
  \end{Phonetics}
  \begin{Phonetics}{哪}{na5}[][HSK 4]
    \definition{part.}{usado depois de uma palavra com a terminação -n, é equivalente a 啊}
  \seealsoref{啊}{a5}
  \end{Phonetics}
  \begin{Phonetics}{哪}{nei3}
    \definition{part.}{qual? (interrogativo, seguido de classificador ou numeral-classificador)}
  \end{Phonetics}
\end{Entry}

\begin{Entry}{哪儿}{9,2}{⼝,⼉}
  \begin{Phonetics}{哪儿}{na3r5}[][HSK 1]
    \definition{adv.}{usado para perguntas retóricas, indicando negação}
    \definition{pron.}{onde? | onde quer que seja; em qualquer lugar | usado como uma resposta educada a um elogio}
  \end{Phonetics}
\end{Entry}

\begin{Entry}{哪个}{9,3}{⼝,⼈}
  \begin{Phonetics}{哪个}{na3ge5}
    \definition{pron.}{qual deles (pergunta sobre o objeto) | quem (perguntar a alguém ou indicar qualquer pessoa)}
  \end{Phonetics}
\end{Entry}

\begin{Entry}{哪里}{9,7}{⼝,⾥}
  \begin{Phonetics}{哪里}{na3li5}[][HSK 1]
    \definition{adv.}{usado em perguntas retóricas para expressar um significado negativo}
    \definition{pron.}{onde?; em que lugar? | onde quer que seja; em qualquer lugar | usado como uma resposta educada a um elogio}
  \end{Phonetics}
\end{Entry}

\begin{Entry}{哪些}{9,8}{⼝,⼆}
  \begin{Phonetics}{哪些}{na3xie1}[][HSK 1]
    \definition{pron.}{quais?}
  \end{Phonetics}
\end{Entry}

\begin{Entry}{哪国人}{9,8,2}{⼝,⼞,⼈}
  \begin{Phonetics}{哪国人}{na3 guo2ren2}
    \definition{expr.}{de qual país?}
  \end{Phonetics}
\end{Entry}

\begin{Entry}{哪怕}{9,8}{⼝,⼼}
  \begin{Phonetics}{哪怕}{na3pa4}[][HSK 4]
    \definition{conj.}{mesmo; mesmo se; mesmo que; não importa o quão}
  \end{Phonetics}
\end{Entry}

\begin{Entry}{哪知道}{9,8,12}{⼝,⽮,⾡}
  \begin{Phonetics}{哪知道}{na3 zhi1dao4}[][HSK 7-9]
    \definition{expr.}{Quem sabe?}[我哪知道 你挑起来的。===Como eu ia saber que você tinha começado?]
  \end{Phonetics}
\end{Entry}

%%%%%%%%%% 型 %%%%%%%%%%
\subsection*{型}\addcontentsline{loh}{figure}{型}

\begin{Entry}{型}{9}{⼟}
  \begin{Phonetics}{型}{xing2}[][HSK 4]
    \definition{s.}{molde; modelo | modelo; tipo; padrão}
  \end{Phonetics}
\end{Entry}

\begin{Entry}{型号}{9,5}{⼟,⼝}
  \begin{Phonetics}{型号}{xing2hao4}[][HSK 4]
    \definition[个,种]{s.}{modelo; tipo; refere"-se ao desempenho, às especificações e ao tamanho de aeronaves, máquinas, implementos agrícolas, etc.}
  \end{Phonetics}
\end{Entry}

%%%%%%%%%% 垫 %%%%%%%%%%
\subsection*{垫}\addcontentsline{loh}{figure}{垫}

\begin{Entry}{垫}{9}{⼟}
  \begin{Phonetics}{垫}{dian4}[][HSK 7-9]
    \definition[个]{s.}{almofada}
    \definition{v.}{colocar algo sob; elevar ou nivelar; encher; preencher | pagar por alguém e esperar ser reembolsado mais tarde | colocar algo sob algo para elevá-lo ou nivelá-lo; usar algo para apoiar, espalhar ou forrar algo para torná-lo mais alto, mais grosso ou mais plano | preencher uma vaga; preencher uma lacuna}
  \end{Phonetics}
\end{Entry}

\begin{Entry}{垫子}{9,3}{⼟,⼦}
  \begin{Phonetics}{垫子}{dian4zi5}[][HSK 7-9]
    \definition{s.}{colchão; esteira; almofada; algo para colocar em uma cama, cadeira, banquinho ou em outro lugar}
  \end{Phonetics}
\end{Entry}

\begin{Entry}{垫底}{9,8}{⼟,⼴}
  \begin{Phonetics}{垫底}{dian4/di3}[][HSK 7-9]
    \definition{v.+compl.}{rebasear; forrar; cobrir}
  \end{Phonetics}
\end{Entry}

%%%%%%%%%% 垮 %%%%%%%%%%
\subsection*{垮}\addcontentsline{loh}{figure}{垮}

\begin{Entry}{垮}{9}{⼟}
  \begin{Phonetics}{垮}{kua3}[][HSK 7-9]
    \definition{v.}{colapsar; cair (quebrar); desmoronar | desmoronar (declínio mental e físico)}
  \end{Phonetics}
\end{Entry}

%%%%%%%%%% 城 %%%%%%%%%%
\subsection*{城}\addcontentsline{loh}{figure}{城}

\begin{Entry}{城}{9}{⼟}
  \begin{Phonetics}{城}{cheng2}[][HSK 3]
    \definition*{s.}{Sobrenome: Cheng}
    \definition[座,道,个]{s.}{muralha da cidade; muralha | cidade | centro de um determinado tipo (por exemplo, negócios, entretenimento, etc.)}
  \end{Phonetics}
\end{Entry}

\begin{Entry}{城乡}{9,3}{⼟,⼄}
  \begin{Phonetics}{城乡}{cheng2xiang1}[][HSK 6]
    \definition{s.}{cidade e campo; áreas urbanas e rurais; a cidade e o campo | cidade e campo; urbano e rural}
  \end{Phonetics}
\end{Entry}

\begin{Entry}{城区}{9,4}{⼟,⼖}
  \begin{Phonetics}{城区}{cheng2qu1}[][HSK 6]
    \definition{s.}{cidade propriamente dita | área metropolitana; área urbana; áreas urbanas e suburbanas}
  \antonymref{郊区}{jiao1qu1}
  \end{Phonetics}
\end{Entry}

\begin{Entry}{城市}{9,5}{⼟,⼱}
  \begin{Phonetics}{城市}{cheng2shi4}[][HSK 3]
    \definition[个,座]{s.}{cidade; regiões com alta densidade populacional, comércio e indústria desenvolvidos e cuja população é predominantemente não agrícola são geralmente centros políticos, econômicos e culturais das regiões vizinhas}
  \end{Phonetics}
\end{Entry}

\begin{Entry}{城里}{9,7}{⼟,⾥}
  \begin{Phonetics}{城里}{cheng2li3}[][HSK 5]
    \definition{s.}{na cidade; dentro da cidade; originalmente referia"-se à área dentro das muralhas da cidade, agora refere"-se principalmente à área urbana}
  \end{Phonetics}
\end{Entry}

\begin{Entry}{城度}{9,9}{⼟,⼴}
  \begin{Phonetics}{城度}{cheng2du4}
    \definition{s.}{na cidade; dentro da cidade; originalmente se referia à área dentro das muralhas da cidade, agora se refere principalmente à área urbana}
  \end{Phonetics}
\end{Entry}

\begin{Entry}{城堡}{9,12}{⼟,⼟}
  \begin{Phonetics}{城堡}{cheng2bao3}
    \definition[座,个]{s.}{forte; castelo; cidadela; uma pequena cidade com muralhas que facilitam a defesa}
  \end{Phonetics}
\end{Entry}

\begin{Entry}{城墙}{9,14}{⼟,⼟}
  \begin{Phonetics}{城墙}{cheng2qiang2}[][HSK 7-9]
    \definition{s.}{muralha (da cidade); as altas e grossas muralhas de proteção que cercam a cidade antiga}
  \end{Phonetics}
\end{Entry}

\begin{Entry}{城镇}{9,15}{⼟,⾦}
  \begin{Phonetics}{城镇}{cheng2zhen4}[][HSK 6]
    \definition[个]{s.}{cidade; cidades e vilas}
  \end{Phonetics}
\end{Entry}

%%%%%%%%%% 复 %%%%%%%%%%
\subsection*{复}\addcontentsline{loh}{figure}{复}

\begin{Entry}{复}{9}{⼢}
  \begin{Phonetics}{复}{fu4}
    \definition*{s.}{Sobrenome: Fu}
    \definition{adj.}{composto; complexo; nem um único; dois ou mais}
    \definition{adv.}{de novo; novamente; indica o reaparecimento de uma situação, equivalente a 再}
    \definition{s.}{jaqueta; roupas forradas}
    \definition{v.}{virar; virar-se | responder; retornar | recuperar; retornar a; restaurar | vingar | duplicar; repetir}
  \seealsoref{再}{zai4}
  \end{Phonetics}
\end{Entry}

\begin{Entry}{复习}{9,3}{⼢,⼄}
  \begin{Phonetics}{复习}{fu4xi2}[][HSK 2]
    \definition{s.}{revisão}
    \definition{v.}{revisar; corrigir (lições, etc.); repetir o que já aprendeu para consolidar o conhecimento}
  \end{Phonetics}
\end{Entry}

\begin{Entry}{复元}{9,4}{⼢,⼉}
  \begin{Phonetics}{复元}{fu4/yuan2}
    \variantof{复原}
  \end{Phonetics}
\end{Entry}

\begin{Entry}{复印}{9,5}{⼢,⼙}
  \begin{Phonetics}{复印}{fu4yin4}[][HSK 3]
    \definition{v.}{fotografar; fotocopiar; duplicar; sem passar pelo processo de impressão, obter uma cópia diretamente do original (geralmente referindo"-se à cópia feita com uma copiadora)}
  \end{Phonetics}
\end{Entry}

\begin{Entry}{复发}{9,5}{⼢,⼜}
  \begin{Phonetics}{复发}{fu4fa1}[][HSK 7-9]
    \definition{v.}{ter uma recaída; recorrer | reaparecer; recrudescer | recorrer (de uma doença) | recair (em um antigo estado ruim)}
  \end{Phonetics}
\end{Entry}

\begin{Entry}{复兴}{9,6}{⼢,⼋}
  \begin{Phonetics}{复兴}{fu4xing1}[][HSK 7-9]
    \definition{v.}{reviver; rejuvenescer | reviver; desenvolver-se e tornar-se mais forte}
  \end{Phonetics}
\end{Entry}

\begin{Entry}{复合}{9,6}{⼢,⼝}
  \begin{Phonetics}{复合}{fu4he2}[][HSK 7-9]
    \definition{v.}{compor; tornar complexo; combinar; juntar | recombinar; juntar; metáfora para estarem juntos novamente depois de estarem separados}
  \end{Phonetics}
\end{Entry}

\begin{Entry}{复杂}{9,6}{⼢,⽊}
  \begin{Phonetics}{复杂}{fu4za2}[][HSK 3]
    \definition{adj.}{complexo; complicado}
  \antonymref{单纯}{dan1chun2}
  \antonymref{简单}{jian3dan1}
  \end{Phonetics}
\end{Entry}

\begin{Entry}{复苏}{9,7}{⼢,⾋}
  \begin{Phonetics}{复苏}{fu4su1}[][HSK 6]
    \definition{s.}{recuperação}
    \definition{v.}{reviver; recuperar; ressuscitar; voltar à vida}
  \end{Phonetics}
\end{Entry}

\begin{Entry}{复制}{9,8}{⼢,⼑}
  \begin{Phonetics}{复制}{fu4zhi4}[][HSK 4]
    \definition{v.}{copiar; duplicar; reproduzir; fazer uma cópia de; fazer uma cópia do original ou reproduzí"-lo, reimprimí"-lo ou copiá"-lo em sua forma original (geralmente referindo"-se a relíquias culturais ou obras de arte)}
  \end{Phonetics}
\end{Entry}

\begin{Entry}{复刻}{9,8}{⼢,⼑}
  \begin{Phonetics}{复刻}{fu4ke4}
    \definition{v.}{reimprimir (um trabalho que esteve fora do catálogo) | reeditar (um disco de vinil, um CD, etc.) | replicar | recriar | (empréstimo linguístico) (computação) \emph{fork}}
  \end{Phonetics}
\end{Entry}

\begin{Entry}{复查}{9,9}{⼢,⽊}
  \begin{Phonetics}{复查}{fu4cha2}[][HSK 7-9]
    \definition{v.}{verificar novamente; reexaminar; revisar}
  \end{Phonetics}
\end{Entry}

\begin{Entry}{复活}{9,9}{⼢,⽔}
  \begin{Phonetics}{复活}{fu4huo2}[][HSK 7-9]
    \definition{s.}{ressurreição (cristianismo)}
    \definition{v.}{reviver; voltar à vida; morrer e voltar à vida, frequentemente usado como metáfora}
  \end{Phonetics}
\end{Entry}

\begin{Entry}{复活节}{9,9,5}{⼢,⽔,⾋}
  \begin{Phonetics}{复活节}{fu4huo2jie2}
    \definition*{s.}{Páscoa; festival cristão que comemora a ressurreição de Jesus ocorre no primeiro domingo após a primeira lua cheia após o equinócio da primavera}
  \end{Phonetics}
\end{Entry}

\begin{Entry}{复原}{9,10}{⼢,⼚}
  \begin{Phonetics}{复原}{fu4/yuan2}[][HSK 7-9]
    \definition{s.}{reconversão; recuperação; redefinição; reabilitação; restauração; recura; analepsia; analepse}
    \definition{v.+compl.}{recuperar-se de uma doença; ter a saúde restaurada | restaurar; reabilitar}
  \end{Phonetics}
\end{Entry}

%%%%%%%%%% 奏 %%%%%%%%%%
\subsection*{奏}\addcontentsline{loh}{figure}{奏}

\begin{Entry}{奏}{9}{⼤}
  \begin{Phonetics}{奏}{zou4}[][HSK 6]
    \definition{v.}{tocar (música); executar (em um instrumento musical)  | alcançar; produzir; alcançar ou estabelecer (desempenho ou realização) | (antigo) apresentar um memorial a um imperador; fazer uma petição}
  \end{Phonetics}
\end{Entry}

\begin{Entry}{奏效}{9,10}{⼤,⽁}
  \begin{Phonetics}{奏效}{zou4xiao4}
    \definition{v.}{mostrar resultados | ser eficaz}
  \end{Phonetics}
\end{Entry}

%%%%%%%%%% 契 %%%%%%%%%%
\subsection*{契}\addcontentsline{loh}{figure}{契}

\begin{Entry}{契}{9}{⼤}
  \begin{Phonetics}{契}{qi4}
    \definition{s.}{contrato; escritura | Arcaico: personagens esculpidos}
    \definition{v.}{Literário: gravar; esculpir | concordar; dar-se bem}
  \end{Phonetics}
  \begin{Phonetics}{契}{xie4}
    \definition*{s.}{Xie (ancestral da dinastia Yin, considerado ministro do Imperador Shun)}
  \end{Phonetics}
\end{Entry}

\begin{Entry}{契机}{9,6}{⼤,⽊}
  \begin{Phonetics}{契机}{qi4ji1}[][HSK 7-9]
    \definition{s.}{oportunidade; ponto de virada; a chave para mudar as coisas numa direção favorável}
  \end{Phonetics}
\end{Entry}

\begin{Entry}{契约}{9,6}{⼤,⽷}
  \begin{Phonetics}{契约}{qi4yue1}[][HSK 7-9]
    \definition{s.}{escritura; carta; contrato; documentos comprovativos de vendas, hipotecas, arrendamentos, etc.}
  \end{Phonetics}
\end{Entry}

%%%%%%%%%% 奖 %%%%%%%%%%
\subsection*{奖}\addcontentsline{loh}{figure}{奖}

\begin{Entry}{奖}{9}{⼤}
  \begin{Phonetics}{奖}{jiang3}[][HSK 4]
    \definition[个,次]{s.}{prêmio; recompensa | elogio; loa}
    \definition{v.}{elogiar; recompensar; recomendar; incentivar}
  \end{Phonetics}
\end{Entry}

\begin{Entry}{奖励}{9,7}{⼤,⼒}
  \begin{Phonetics}{奖励}{jiang3li4}[][HSK 5]
    \definition{s.}{prêmio; recompensa; dinheiro ou honras dadas em troca de elogios ou incentivos}
    \definition{v.}{recompensar; incentivar; encorajar}
  \end{Phonetics}
\end{Entry}

\begin{Entry}{奖学金}{9,8,8}{⼤,⼦,⾦}
  \begin{Phonetics}{奖学金}{jiang3xue2jin1}[][HSK 4]
    \definition[笔]{s.}{bolsa de estudos; exposição; prêmios concedidos por escolas, organizações ou indivíduos a alunos com bom desempenho acadêmico}
  \end{Phonetics}
\end{Entry}

\begin{Entry}{奖杯}{9,8}{⼤,⽊}
  \begin{Phonetics}{奖杯}{jiang3bei1}[][HSK 7-9]
    \definition[个,座]{s.}{taça (como prêmio); copa; troféu; os prêmios em formato de taça, concedidos aos vencedores em competições esportivas, geralmente são feitos de ouro ou prata}
  \end{Phonetics}
\end{Entry}

\begin{Entry}{奖金}{9,8}{⼤,⾦}
  \begin{Phonetics}{奖金}{jiang3jin1}[][HSK 4]
    \definition[个,笔]{s.}{bônus; recompensa; prêmio; prêmio em dinheiro; dinheiro de recompensa, dinheiro dado às pessoas para incentivá-las ou elogiá-las por terem se saído bem em alguma coisa}
  \end{Phonetics}
\end{Entry}

\begin{Entry}{奖品}{9,9}{⼤,⼝}
  \begin{Phonetics}{奖品}{jiang3pin3}[][HSK 7-9]
    \definition[个,些,份]{s.}{prêmio; itens para recompensa}
  \end{Phonetics}
\end{Entry}

\begin{Entry}{奖项}{9,9}{⼤,⾴}
  \begin{Phonetics}{奖项}{jiang3xiang4}[][HSK 7-9]
    \definition[项]{s.}{prêmio; projetos premiados}
  \end{Phonetics}
\end{Entry}

\begin{Entry}{奖牌}{9,12}{⼤,⽚}
  \begin{Phonetics}{奖牌}{jiang3pai2}[][HSK 7-9]
    \definition{s.}{medalha (concedida como prêmio); as medalhas são divididas em ouro, prata e bronze; os vencedores em competições esportivas recebem prêmios de acordo com esses três níveis}
  \end{Phonetics}
\end{Entry}

%%%%%%%%%% 姜 %%%%%%%%%%
\subsection*{姜}\addcontentsline{loh}{figure}{姜}

\begin{Entry}{姜}{9}{⼥}
  \begin{Phonetics}{姜}{jiang1}[][HSK 7-9]
    \definition*{s.}{Sobrenome: Jiang}
    \definition[磅,斤,两]{s.}{gengibre; rizoma de gengibre}
  \end{Phonetics}
\end{Entry}

%%%%%%%%%% 姥 %%%%%%%%%%
\subsection*{姥}\addcontentsline{loh}{figure}{姥}

\begin{Entry}{姥}{9}{⼥}
  \begin{Phonetics}{姥}{lao3}
    \definition{s.}{avó; avó materna}
  \end{Phonetics}
  \begin{Phonetics}{姥}{mu3}
    \definition*{s.}{Sobrenome: Mu}
    \definition{s.}{mulher idosa; Literário: velha senhora}
  \end{Phonetics}
\end{Entry}

\begin{Entry}{姥爷}{9,6}{⼥,⽗}
  \begin{Phonetics}{姥爷}{lao3ye5}[][HSK 7-9]
    \definition{s.}{avô; avô materno}
  \seealsoref{爷爷}{ye2ye5}
  \end{Phonetics}
\end{Entry}

\begin{Entry}{姥姥}{9,9}{⼥,⼥}
  \begin{Phonetics}{姥姥}{lao3lao5}[][HSK 7-9]
    \definition[个,位,名,些]{s.}{avó; mãe da mãe; avó materna | parteira}
  \end{Phonetics}
\end{Entry}

%%%%%%%%%% 姨 %%%%%%%%%%
\subsection*{姨}\addcontentsline{loh}{figure}{姨}

\begin{Entry}{姨}{9}{⼥}
  \begin{Phonetics}{姨}{yi2}
    \definition[个,位,名,些]{s.}{irmã da mãe; tia materna; tia | irmã da esposa; cunhada}
  \end{Phonetics}
\end{Entry}

\begin{Entry}{姨父}{9,4}{⼥,⽗}
  \begin{Phonetics}{姨父}{yi2fu5}
    \definition{s.}{marido da irmã da mãe ou da tia materna; tio}
  \antonymref{阿姨}{a1yi2}
  \end{Phonetics}
\end{Entry}

\begin{Entry}{姨妈}{9,6}{⼥,⼥}
  \begin{Phonetics}{姨妈}{yi2ma1}
    \definition[个,位,名,些]{s.}{tia; tia materna; irmã da mãe; tia (referindo"-se a uma mulher casada)}
  \synonymref{阿姨}{a1yi2}
  \synonymref{姨娘}{yi2niang2}
  \end{Phonetics}
\end{Entry}

\begin{Entry}{姨娘}{9,10}{⼥,⼥}
  \begin{Phonetics}{姨娘}{yi2niang2}
    \definition{s.}{Obsoleto: forma de tratamento para a concubina do pai | Dialeto: (casada) tia materna; tia}
  \synonymref{阿姨}{a1yi2}
  \synonymref{大妈}{da4ma1}
  \synonymref{姨妈}{yi2ma1}
  \end{Phonetics}
\end{Entry}

%%%%%%%%%% 威 %%%%%%%%%%
\subsection*{威}\addcontentsline{loh}{figure}{威}

\begin{Entry}{威}{9}{⼥}
  \begin{Phonetics}{威}{wei1}
    \definition*{s.}{Sobrenome: Wei}
    \definition{adj.}{forte; poderoso}
    \definition{s.}{força impressionante; poder; força}
    \definition{v.}{ameaçar pela força; intimidar com força}
  \end{Phonetics}
\end{Entry}

\begin{Entry}{威胁}{9,8}{⼥,⾁}
  \begin{Phonetics}{威胁}{wei1xie2}[][HSK 6]
    \definition{v.}{pôr em perigo; ameaçar; intimidar}
  \end{Phonetics}
\end{Entry}

%%%%%%%%%% 娃 %%%%%%%%%%
\subsection*{娃}\addcontentsline{loh}{figure}{娃}

\begin{Entry}{娃}{9}{⼥}
  \begin{Phonetics}{娃}{wa2}
    \definition[个,名,位,只]{s.}{bebê; criança | filho ou filha; criança | Dialeto: animal recém-nascido | Literário: menina; jovem mulher | Literário: menina bonita}
  \end{Phonetics}
\end{Entry}

\begin{Entry}{娃娃}{9,9}{⼥,⼥}
  \begin{Phonetics}{娃娃}{wa2wa5}[][HSK 6]
    \definition[个,名,位]{s.}{bebê; criança; criança pequena | boneca; brinquedos em forma de crianças}
  \end{Phonetics}
\end{Entry}

%%%%%%%%%% 娇 %%%%%%%%%%
\subsection*{娇}\addcontentsline{loh}{figure}{娇}

\begin{Entry}{娇}{9}{⼥}
  \begin{Phonetics}{娇}{jiao1}
    \definition{adj.}{terno; adorável; encantador | frágil; delicado | melindroso; exigente}
    \definition{v.}{mimar; estragar}
  \end{Phonetics}
\end{Entry}

\begin{Entry}{娇气}{9,4}{⼥,⽓}
  \begin{Phonetics}{娇气}{jiao1qi4}[][HSK 7-9]
    \definition{adj.}{exigente; melindroso; com personalidade frágil, incapaz de suportar dificuldades ou injustiças | terno; frágil; delicado; (os itens) são facilmente danificados; (as plantas) não são fáceis de cultivar}
    \definition{s.}{delicadeza; fragilidade; personalidade e estilo frágeis}
  \end{Phonetics}
\end{Entry}

\begin{Entry}{娇惯}{9,11}{⼥,⼼}
  \begin{Phonetics}{娇惯}{jiao1guan4}[][HSK 7-9]
    \definition{v.}{mimar; paparicar; estragar}
  \end{Phonetics}
\end{Entry}

%%%%%%%%%% 孩 %%%%%%%%%%
\subsection*{孩}\addcontentsline{loh}{figure}{孩}

\begin{Entry}{孩}{9}{⼦}
  \begin{Phonetics}{孩}{hai2}
    \definition[个]{s.}{criança}
  \end{Phonetics}
\end{Entry}

\begin{Entry}{孩子}{9,3}{⼦,⼦}
  \begin{Phonetics}{孩子}{hai2zi5}[][HSK 1]
    \definition[个]{s.}{criança; crianças; pessoas com idade entre alguns anos ou na adolescência, geralmente com menos de 14 anos | crianças; filho ou filha}
  \end{Phonetics}
\end{Entry}

%%%%%%%%%% 孪 %%%%%%%%%%
\subsection*{孪}\addcontentsline{loh}{figure}{孪}

\begin{Entry}{孪}{9}{⼦}
  \begin{Phonetics}{孪}{luan2}
    \definition[对]{s.}{gêmeos}
  \end{Phonetics}
\end{Entry}

\begin{Entry}{孪生}{9,5}{⼦,⽣}
  \begin{Phonetics}{孪生}{luan2sheng1}[][HSK 7-9]
    \definition{adj.}{gêmeo | geminado}
    \definition{s.}{geminação | cristal geminado; hemitropismo}
  \end{Phonetics}
\end{Entry}

%%%%%%%%%% 客 %%%%%%%%%%
\subsection*{客}\addcontentsline{loh}{figure}{客}

\begin{Entry}{客}{9}{⼧}
  \begin{Phonetics}{客}{ke4}
    \definition*{s.}{Sobrenome: Ke}
    \definition{adj.}{objetivo; independente da consciência humana | estrangeiro; não desta região, unidade ou indústria}
    \definition{clas.}{porção (de comida, bebida, etc.); em algumas áreas, é usado para vender alimentos e bebidas em porções}
    \definition[个,位,名,些]{s.}{convidado; visitante; aquele que é convidado; aquele que vem visitar | viajante; passageiro | comerciante viajante; refere"-se especificamente a comerciantes que transportam mercadorias de um lugar para o outro | cliente; patrono; consumidor | uma pessoa envolvida em alguma atividade específica; pessoas que viajam fazendo algum tipo de atividade}
    \definition{v.}{ser um estranho; estabelecer-se (ou viver) em um lugar estranho; estar longe de casa ou morar no exterior}
  \antonymref{主}{zhu3}
  \end{Phonetics}
\end{Entry}

\begin{Entry}{客人}{9,2}{⼧,⼈}
  \begin{Phonetics}{客人}{ke4ren5}[][HSK 2]
    \definition[位,个,桌,拨,批]{s.}{visitante; convidado | cliente; passageiro; hóspede; viajante}
  \end{Phonetics}
\end{Entry}

\begin{Entry}{客厅}{9,4}{⼧,⼚}
  \begin{Phonetics}{客厅}{ke4ting1}[][HSK 5]
    \definition[间,个]{s.}{sala de estar; sala de visitas; sala para receber convidados}
  \end{Phonetics}
\end{Entry}

\begin{Entry}{客户}{9,4}{⼧,⼾}
  \begin{Phonetics}{客户}{ke4hu4}[][HSK 5]
    \definition[位,个,家,批]{s.}{cliente; consumidor}
  \end{Phonetics}
\end{Entry}

\begin{Entry}{客气}{9,4}{⼧,⽓}
  \begin{Phonetics}{客气}{ke4qi5}[][HSK 5]
    \definition{adj.}{educado; modesto; cortês}
    \definition{v.}{ser educado; ser cortês; fazer comentários educados ou agir educadamente}
  \end{Phonetics}
\end{Entry}

\begin{Entry}{客车}{9,4}{⼧,⾞}
  \begin{Phonetics}{客车}{ke4che1}[][HSK 6]
    \definition[辆,列,次,趟]{s.}{ônibus; veículo de passageiros; veículos que transportam passageiros em ferrovias e estradas}
  \end{Phonetics}
\end{Entry}

\begin{Entry}{客机}{9,6}{⼧,⽊}
  \begin{Phonetics}{客机}{ke4ji1}[][HSK 7-9]
    \definition[架]{s.}{avião de passageiros; avião comercial}
  \antonymref{货机}{huo4ji1}
  \end{Phonetics}
\end{Entry}

\begin{Entry}{客观}{9,6}{⼧,⾒}
  \begin{Phonetics}{客观}{ke4guan1}[][HSK 3]
    \definition{adj.}{objetivo; justo e razoável; imparcial; com base na situação real, sem preconceitos pessoais}
    \definition{s.}{objetivo; existe fora da consciência, sem depender da consciência subjetiva}
  \end{Phonetics}
\end{Entry}

\begin{Entry}{客运}{9,7}{⼧,⾡}
  \begin{Phonetics}{客运}{ke4yun4}[][HSK 7-9]
    \definition{s.}{transporte de passageiros; (setor de transporte) o negócio de transporte de passageiros}
  \end{Phonetics}
\end{Entry}

\begin{Entry}{客房}{9,8}{⼧,⼾}
  \begin{Phonetics}{客房}{ke4fang2}[][HSK 7-9]
    \definition{s.}{quarto de hóspedes; quartos para viajantes ou hóspedes}
  \end{Phonetics}
\end{Entry}

\begin{Entry}{客流}{9,10}{⼧,⽔}
  \begin{Phonetics}{客流}{ke4liu2}[][HSK 7-9]
    \definition{s.}{fluxo de passageiros; o setor de transportes refere"-se ao fluxo de passageiros em uma determinada direção dentro de um determinado período de tempo | fluxo de clientes (frequentando uma loja, etc.)}
  \end{Phonetics}
\end{Entry}

%%%%%%%%%% 宣 %%%%%%%%%%
\subsection*{宣}\addcontentsline{loh}{figure}{宣}

\begin{Entry}{宣}{9}{⼧}
  \begin{Phonetics}{宣}{xuan1}
    \definition*{s.}{Sobrenome: Xuan}
    \definition{v.}{declarar; proclamar; anunciar; falar publicamente | drenar (líquidos)}
  \end{Phonetics}
\end{Entry}

\begin{Entry}{宣布}{9,5}{⼧,⼱}
  \begin{Phonetics}{宣布}{xuan1bu4}[][HSK 3]
    \definition{v.}{declarar; proclamar; pronunciar; anunciar; informar oficialmente a todos sobre as últimas decisões e situações}
  \end{Phonetics}
\end{Entry}

\begin{Entry}{宣传}{9,6}{⼧,⼈}
  \begin{Phonetics}{宣传}{xuan1chuan2}[][HSK 3]
    \definition[个]{v.}{propagar; divulgar; fazer propaganda; explicar e esclarecer às pessoas, para que elas acreditem e sigam as ações}
  \end{Phonetics}
\end{Entry}

\begin{Entry}{宣扬}{9,6}{⼧,⼿}
  \begin{Phonetics}{宣扬}{xuan1yang2}
    \definition{v.}{divulgar | anunciar | espalhar por toda parte}
  \end{Phonetics}
\end{Entry}

%%%%%%%%%% 室 %%%%%%%%%%
\subsection*{室}\addcontentsline{loh}{figure}{室}

\begin{Entry}{室}{9}{⼧}
  \begin{Phonetics}{室}{shi4}[][HSK 3]
    \definition*{s.}{Shi, a décima terceira das vinte e oito constelações da esfera celeste, composta por duas estrelas em linha reta na constelação de Pégaso | Sobrenome: Shi}
    \definition{s.}{sala; quarto; casa | departamento; sala como unidade administrativa ou de trabalho; órgãos públicos, fábricas, escolas e outras unidades de trabalho internas | esposa; familiares ou esposa | família; clã | cavidade; órgão com forma semelhante a uma câmara}
  \end{Phonetics}
\end{Entry}

%%%%%%%%%% 宪 %%%%%%%%%%
\subsection*{宪}\addcontentsline{loh}{figure}{宪}

\begin{Entry}{宪}{9}{⼧}
  \begin{Phonetics}{宪}{xian4}
    \definition*{s.}{Sobrenome: Xian}
    \definition{s.}{estatuto; decreto | constituição}
  \end{Phonetics}
\end{Entry}

\begin{Entry}{宪制}{9,8}{⼧,⼑}
  \begin{Phonetics}{宪制}{xian4zhi4}
    \definition{adj.}{constitucional}
    \definition{s.}{sistema de governo constitucional}
  \end{Phonetics}
\end{Entry}

\begin{Entry}{宪法法院}{9,8,8,9}{⼧,⽔,⽔,⾩}
  \begin{Phonetics}{宪法法院}{xian4fa3fa3yuan4}
    \definition{s.}{tribunal constitucional}
  \end{Phonetics}
\end{Entry}

\begin{Entry}{宪政}{9,9}{⼧,⽁}
  \begin{Phonetics}{宪政}{xian4zheng4}
    \definition{s.}{governo constitucional}
  \end{Phonetics}
\end{Entry}

%%%%%%%%%% 宫 %%%%%%%%%%
\subsection*{宫}\addcontentsline{loh}{figure}{宫}

\begin{Entry}{宫}{9}{⼧}
  \begin{Phonetics}{宫}{gong1}[][HSK 6]
    \definition*{s.}{Sobrenome: Gong}
    \definition[座]{s.}{palácio imperial; palácio; casas onde o imperador, a imperatriz, o príncipe, etc. vivem | morada de seres sobrenaturais; palácio; paraíso; casas onde vivem os deuses na mitologia | templo (usado em um nome de templo) | local para atividades culturais e recreativas; um edifício para atividades culturais e recreativas; casas para fins culturais e de entretenimento | útero | uma nota da antiga escala chinesa de cinco tons, correspondente a 1 na notação musical numerada}
  \end{Phonetics}
\end{Entry}

\begin{Entry}{宫廷}{9,6}{⼧,⼵}
  \begin{Phonetics}{宫廷}{gong1ting2}[][HSK 7-9]
    \definition{s.}{palácio imperial; residência do imperador; palácio | corte real ou imperial; o monarca e seus funcionários; corte}
  \end{Phonetics}
\end{Entry}

\begin{Entry}{宫殿}{9,13}{⼧,⽎}
  \begin{Phonetics}{宫殿}{gong1dian4}[][HSK 7-9]
    \definition[座]{s.}{palácio; geralmente se refere às casas magníficas onde os imperadores vivem}
  \end{Phonetics}
\end{Entry}

%%%%%%%%%% 封 %%%%%%%%%%
\subsection*{封}\addcontentsline{loh}{figure}{封}

\begin{Entry}{封}{9}{⼨}
  \begin{Phonetics}{封}{feng1}[][HSK 2,5]
    \definition*{s.}{Sobrenome: Feng}
    \definition{clas.}{usado para objetos selados, especialmente cartas}
    \definition{s.}{feudalismo | embalagem; envelope | pacote}
    \definition{v.}{conferir (um título, território, etc.) a | selar | acender uma fogueira | fechar}
  \end{Phonetics}
\end{Entry}

\begin{Entry}{封口}{9,3}{⼨,⼝}
  \begin{Phonetics}{封口}{feng1kou3}
    \definition{v.}{selar | fechar | curar (uma ferida) | manter os lábios selados}
  \end{Phonetics}
\end{Entry}

\begin{Entry}{封印}{9,5}{⼨,⼙}
  \begin{Phonetics}{封印}{feng1yin4}
    \definition{s.}{selo (em envelopes)}
  \end{Phonetics}
\end{Entry}

\begin{Entry}{封闭}{9,6}{⼨,⾨}
  \begin{Phonetics}{封闭}{feng1bi4}[][HSK 4]
    \definition{adj.}{fechado; aqueles que não têm contato com o mundo exterior; aqueles que são muito conservadores (em seu pensamento) e não se comunicam com os outros}
    \definition{v.}{selar; fechar; lacrar; vedar; de modo a impedir a passagem, o uso ou a abertura}
  \end{Phonetics}
\end{Entry}

\begin{Entry}{封冻}{9,7}{⼨,⼎}
  \begin{Phonetics}{封冻}{feng1dong4}
    \definition{v.}{congelar (água ou terra)}
  \end{Phonetics}
\end{Entry}

\begin{Entry}{封底}{9,8}{⼨,⼴}
  \begin{Phonetics}{封底}{feng1di3}
    \definition{s.}{contracapa de um livro}
  \end{Phonetics}
\end{Entry}

\begin{Entry}{封建}{9,8}{⼨,⼵}
  \begin{Phonetics}{封建}{feng1jian4}[][HSK 7-9]
    \definition{adj.}{feudal}
    \definition{s.}{feudalismo; sistema de feudo; o sistema político de feudos e estabelecimento de estados vassalos; esse sistema foi implementado pela primeira vez durante a Dinastia Zhou Ocidental | ideologia feudal; pensamento feudal}
  \end{Phonetics}
\end{Entry}

\begin{Entry}{封顶}{9,8}{⼨,⾴}
  \begin{Phonetics}{封顶}{feng1ding3}[][HSK 7-9]
    \definition{v.}{parar de crescer (de broto ou galho de planta) | Figurativo: impor um teto (sobre preços, salários, bônus, etc.) | cobrir (um edifício, etc.); cobrir o telhado (finalizar um projeto de construção); colocar um telhado (em um edifício); completar; encerrar | Figurativo: atingir o ponto mais alto (de crescimento, lucro, taxas de juros)}
  \end{Phonetics}
\end{Entry}

\begin{Entry}{封面}{9,9}{⼨,⾯}
  \begin{Phonetics}{封面}{feng1mian4}[][HSK 7-9]
    \definition{s.}{capa (de uma publicação) ; capa frontal; sobrecapa}
  \end{Phonetics}
\end{Entry}

\begin{Entry}{封斋}{9,10}{⼨,⽂}
  \begin{Phonetics}{封斋}{feng1zhai1}
    \definition*{s.}{Ramadã (Islã)}
  \end{Phonetics}
\end{Entry}

\begin{Entry}{封盖}{9,11}{⼨,⽫}
  \begin{Phonetics}{封盖}{feng1gai4}
    \definition{s.}{arremesso bloqueado (basquete) | boné | capa | selo}
    \definition{v.}{(no basquete) bloquear (um arremesso) | cobrir}
  \end{Phonetics}
\end{Entry}

\begin{Entry}{封锁}{9,12}{⼨,⾦}
  \begin{Phonetics}{封锁}{feng1suo3}[][HSK 7-9]
    \definition{v.}{bloquear; selar; tomar medidas militares, etc.) impedir a passagem}
  \end{Phonetics}
\end{Entry}

%%%%%%%%%% 将 %%%%%%%%%%
\subsection*{将}\addcontentsline{loh}{figure}{将}

\begin{Entry}{将}{9}{⼨}
  \begin{Phonetics}{将}{jiang1}[][HSK 5]
    \definition*{s.}{Sobrenome: Jiang}
    \definition{adv.}{estar indo para; parcialmente\dots parcialmente\dots}
    \definition{part.}{expressar uma direção, como 进来, 出去; usado no meio de verbos e complementos que indicam tendência, como 进来, 出去, etc.}
    \definition{prep.}{com; por meio de; por | usado da mesma forma que 把}
    \definition{v.}{fazer algo; lidar com (um assunto) | dar um cheque-mate | cuidar (da saúde) | incitar alguém a agir; desafiar; estimular | segurar; pegar | colocar; tirar | levar; trazer | dar suporte; dar apoio}
  \seealsoref{把}{ba3}
  \seealsoref{出去}{chu1 qu5}
  \seealsoref{进来}{jin4 lai5}
  \end{Phonetics}
  \begin{Phonetics}{将}{jiang4}
    \definition{s.}{general; nome do posto; abaixo de marechal de campo; acima de coronel}
    \definition{v.}{comandar; liderar}
  \end{Phonetics}
  \begin{Phonetics}{将}{qiang1}
    \definition{v.}{pedir; apelar para}
  \end{Phonetics}
\end{Entry}

\begin{Entry}{将军}{9,6}{⼨,⼍}
  \begin{Phonetics}{将军}{jiang1/jun1}[][HSK 6]
    \definition[位,名]{s.}{general; geralmente se refere a generais seniores}
    \definition{v.+compl.}{dar xeque-mate; atacar o general ou rei do oponente no xadrez; colocar alguém em grandes apuros; metáfora para dar a alguém um problema difícil ou dificultar a tarefa para essa pessoa}
  \end{Phonetics}
\end{Entry}

\begin{Entry}{将来}{9,7}{⼨,⽊}
  \begin{Phonetics}{将来}{jiang1lai2}[][HSK 3]
    \definition[个]{s.}{no futuro (geralmente se refere a um período mais longo)}
  \end{Phonetics}
\end{Entry}

\begin{Entry}{将近}{9,7}{⼨,⾡}
  \begin{Phonetics}{将近}{jiang1jin4}[][HSK 3]
    \definition{adv.}{quase}
  \end{Phonetics}
\end{Entry}

\begin{Entry}{将要}{9,9}{⼨,⾑}
  \begin{Phonetics}{将要}{jiang1yao4}[][HSK 5]
    \definition{adv.}{irá; deverá; estará prestes a; irá a; indica que um ato ou situação ocorre logo em seguida}
  \end{Phonetics}
\end{Entry}

%%%%%%%%%% 尝 %%%%%%%%%%
\subsection*{尝}\addcontentsline{loh}{figure}{尝}

\begin{Entry}{尝}{9}{⼩}
  \begin{Phonetics}{尝}{chang2}[][HSK 5]
    \definition{adv.}{alguma vez; uma vez}
    \definition{v.}{provar; experimentar o sabor de | provar; experimentar; conhecer | tentar; testar}
  \end{Phonetics}
\end{Entry}

\begin{Entry}{尝试}{9,8}{⼩,⾔}
  \begin{Phonetics}{尝试}{chang2shi4}[][HSK 5]
    \definition{v.}{tentar; provar; experimentar}
  \end{Phonetics}
\end{Entry}

%%%%%%%%%% 屋 %%%%%%%%%%
\subsection*{屋}\addcontentsline{loh}{figure}{屋}

\begin{Entry}{屋}{9}{⼫}
  \begin{Phonetics}{屋}{wu1}[][HSK 5]
    \definition[间,座]{s.}{casa | quarto}
  \end{Phonetics}
\end{Entry}

\begin{Entry}{屋子}{9,3}{⼫,⼦}
  \begin{Phonetics}{屋子}{wu1zi5}[][HSK 3]
    \definition[间,座,栋]{s.}{quarto; sala}
  \end{Phonetics}
\end{Entry}

%%%%%%%%%% 屌 %%%%%%%%%%
\subsection*{屌}\addcontentsline{loh}{figure}{屌}

\begin{Entry}{屌}{9}{⼫}
  \begin{Phonetics}{屌}{diao3}
    \definition{adj.}{(gíria) legal ou extraordinário}
    \definition{s.}{órgão genital masculino; pênis}
    \definition{v.}{(cantonês) foder}
  \end{Phonetics}
\end{Entry}

\begin{Entry}{屌丝}{9,5}{⼫,⼀}
  \begin{Phonetics}{屌丝}{diao3si1}
    \definition{adj.}{panaca | zé-ninguém | (gíria de \emph{Internet}) \emph{looser}}
  \end{Phonetics}
\end{Entry}

%%%%%%%%%% 屎 %%%%%%%%%%
\subsection*{屎}\addcontentsline{loh}{figure}{屎}

\begin{Entry}{屎}{9}{⼫}
  \begin{Phonetics}{屎}{shi3}
    \definition{s.}{fezes | excrementos | (forma ligada) secreção (do ouvido, olho, etc.)}
  \end{Phonetics}
\end{Entry}

%%%%%%%%%% 屏 %%%%%%%%%%
\subsection*{屏}\addcontentsline{loh}{figure}{屏}

\begin{Entry}{屏}{9}{⼫}
  \begin{Phonetics}{屏}{bing1}
    \definition{s.}{antigamente, referia"-se à pequena parede de tela em frente ao portão de um antigo palácio; no chinês moderno, também é usado como uma palavra humilde para expressar o significado de 惶恐}
  \seealsoref{惶恐}{huang2kong3}
  \end{Phonetics}
  \begin{Phonetics}{屏}{bing3}
    \definition*{s.}{Sobrenome: Bing}
    \definition{v.}{prender (a respiração); conter a respiração | rejeitar; livrar"-se de; remover; pôr (colocar) de lado; abandonar; descartar}
  \end{Phonetics}
  \begin{Phonetics}{屏}{ping2}
    \definition{s.}{tela | um conjunto de pergaminhos; tiras de tela}
    \definition{v.}{proteger alguém ou algo; resguardar}
  \end{Phonetics}
\end{Entry}

\begin{Entry}{屏幕}{9,13}{⼫,⼱}
  \begin{Phonetics}{屏幕}{ping2mu4}[][HSK 6]
    \definition[个,块]{s.}{tela; a parte dos computadores, televisores, celulares, etc. que exibe texto, imagens, etc.}
  \end{Phonetics}
\end{Entry}

%%%%%%%%%% 差 %%%%%%%%%%
\subsection*{差}\addcontentsline{loh}{figure}{差}

\begin{Entry}{差}{9}{⼯}
  \begin{Phonetics}{差}{cha1}
    \definition{adj.}{diferente; diferente ou inconsistente com um determinado padrão}
    \definition{adv.}{ligeiramente; comparativamente; um pouco}
    \definition{s.}{diferença; resto após a subtração de dois números | erro; engano}
  \end{Phonetics}
  \begin{Phonetics}{差}{cha4}[][HSK 1]
    \definition{adj.}{não está de acordo com o padrão; pobre; ruim; inferior | errado; incorreto | mesmo significado de 差 \dpy{cha1}}
    \definition{v.}{faltar}
  \end{Phonetics}
  \begin{Phonetics}{差}{chai1}
    \definition{s.}{tarefa; trabalho; ser enviado para fazer algo; deveres oficiais; posição | corvéia; mensageiro ou oficial de justiça em um yamen feudal; Antigo: refere"-se a pessoas que são enviadas para fazer coisas}
    \definition{v.}{enviar uma mensagem; despachar; fnviar (para fazer algo)}
  \end{Phonetics}
\end{Entry}

\begin{Entry}{差(一)点儿}{9,1,9,2}{⼯,⼀,⽕,⼉}
  \begin{Phonetics}{差(一)点儿}{cha1yi4dian3r5}
    \definition{adj.}{não é bom o suficiente; ligeiramente inferior a; não está à altura da marca;  (qualidade, tecnologia, desempenho, etc.) ligeiramente inferior}
    \definition{adv.}{quase; à beira de; praticamente; aproximadamente; significa que algo está perto de ser alcançado, mas não foi alcançado, ou algo foi alcançado, mas mal foi alcançado}
  \end{Phonetics}
\end{Entry}

\begin{Entry}{差不多}{9,4,6}{⼯,⼀,⼣}
  \begin{Phonetics}{差不多}{cha4bu5duo1}[][HSK 2]
    \definition{adj.}{semelhante; aproximadamente igual | não muito longe; quase certo (suficiente); basicamente, próximo dos padrões e requisitos; normal | prestes a (terminar; acabar); descreve que (algo) está quase acabando; (uma tarefa) está quase concluída}
    \definition{adv.}{quase; perto; indica proximidade}
  \end{Phonetics}
\end{Entry}

\begin{Entry}{差异}{9,6}{⼯,⼶}
  \begin{Phonetics}{差异}{cha1yi4}[][HSK 6]
    \definition{s.}{diferença; divergência; discrepância}
  \end{Phonetics}
\end{Entry}

\begin{Entry}{差别}{9,7}{⼯,⼑}
  \begin{Phonetics}{差别}{cha1bie2}[][HSK 5]
    \definition{s.}{diferença; disparidade; dissimilaridade; distinção; não semelhança; diferenças na forma ou no conteúdo}
  \end{Phonetics}
\end{Entry}

\begin{Entry}{差点儿}{9,9,2}{⼯,⽕,⼉}
  \begin{Phonetics}{差点儿}{cha4dian3r5}[][HSK 5]
    \definition{adv.}{por pouco | por um triz | quase}
  \end{Phonetics}
\end{Entry}

\begin{Entry}{差距}{9,11}{⼯,⾜}
  \begin{Phonetics}{差距}{cha1ju4}[][HSK 5]
    \definition[个,些,段]{s.}{lacuna; disparidade; discrepância; diferença; grau de diferença entre as coisas, especialmente em termos de distância de algum padrão.}
  \end{Phonetics}
\end{Entry}

\begin{Entry}{差错}{9,13}{⼯,⾦}
  \begin{Phonetics}{差错}{cha1cuo4}[][HSK 7-9]
    \definition[出]{s.}{deslize; erro; engano | acidente; contratempo; mudanças inesperadas}
  \end{Phonetics}
\end{Entry}

\begin{Entry}{差额}{9,15}{⼯,⾴}
  \begin{Phonetics}{差额}{cha1'e2}[][HSK 7-9]
    \definition{s.}{equilíbrio; diferencial; margem; a diferença entre um valor usado como padrão ou comparação}
  \end{Phonetics}
\end{Entry}

%%%%%%%%%% 帝 %%%%%%%%%%
\subsection*{帝}\addcontentsline{loh}{figure}{帝}

\begin{Entry}{帝}{9}{⼱}
  \begin{Phonetics}{帝}{di4}
    \definition*{s.}{Ser Supremo; Deus}
    \definition[位,名,个]{s.}{imperador | (abreviação) imperialismo}
  \end{Phonetics}
\end{Entry}

\begin{Entry}{帝国}{9,8}{⼱,⼞}
  \begin{Phonetics}{帝国}{di4guo2}[][HSK 7-9]
    \definition[个]{s.}{império}[罗马帝国。===Império romano.]
  \end{Phonetics}
\end{Entry}

\begin{Entry}{帝国主义}{9,8,5,3}{⼱,⼞,⼂,⼂}
  \begin{Phonetics}{帝国主义}{di4guo2 zhu3yi4}[][HSK 7-9]
    \definition{s.}{imperialismo}[帝国主义是垄断的、寄生的、垂死的资本主义。===Imperialismo é capitalismo monopolista, parasitário e moribundo.]
  \end{Phonetics}
\end{Entry}

%%%%%%%%%% 带 %%%%%%%%%%
\subsection*{带}\addcontentsline{loh}{figure}{带}

\begin{Entry}{带}{9}{⼱}
  \begin{Phonetics}{带}{dai4}[][HSK 2]
    \definition*{s.}{Sobrenome: Dai}
    \definition[根]{s.}{cinto; faixa; banda; fita; fita adesiva; algo parecido com uma fita | pneu | zona; área; faixa; cinturão; região; uma determinada área geográfica com determinadas características | leucorreia; corrimento branco; corrimento vaginal}
    \definition{v.}{levar; trazer; transportar | liderar; dirigir; conduzir; assumir | cuidar de crianças; criar filhos; educar | fazer uma coisa e, ao mesmo tempo, fazer outra coisa |suportar; conter | ter algo anexado, simultâneo | trazer consigo | carregar consigo | demonstrar; parecer | incluir; acrescentar}
  \end{Phonetics}
\end{Entry}

\begin{Entry}{带队}{9,4}{⼱,⾩}
  \begin{Phonetics}{带队}{dai4dui4}[][HSK 7-9]
    \definition{v.}{liderar um grupo (para fazer algo)}
  \end{Phonetics}
\end{Entry}

\begin{Entry}{带头}{9,5}{⼱,⼤}
  \begin{Phonetics}{带头}{dai4/tou2}[][HSK 7-9]
    \definition{v.+compl.}{assumir a liderança; ser o primeiro; tomar a iniciativa; dar o exemplo}
  \end{Phonetics}
\end{Entry}

\begin{Entry}{带头人}{9,5,2}{⼱,⼤,⼈}
  \begin{Phonetics}{带头人}{dai4tou2 ren2}[][HSK 7-9]
    \definition{s.}{líder; pioneiro}
  \end{Phonetics}
\end{Entry}

\begin{Entry}{带动}{9,6}{⼱,⼒}
  \begin{Phonetics}{带动}{dai4 dong4}[][HSK 3]
    \definition{v.}{dirigir; ativar; fazer algo funcionar; acionar | liderar; trazer; estimular; motivar; atrair; liderar o avanço; dar o exemplo e fazer com que os outros sigam o exemplo}
  \end{Phonetics}
\end{Entry}

\begin{Entry}{带有}{9,6}{⼱,⽉}
  \begin{Phonetics}{带有}{dai4you3}[][HSK 5]
    \definition{v.}{ter; envolver; carregar; implicar}
  \end{Phonetics}
\end{Entry}

\begin{Entry}{带来}{9,7}{⼱,⽊}
  \begin{Phonetics}{带来}{dai4 lai2}[][HSK 2]
    \definition{v.}{provocar; produzir; causar}
  \end{Phonetics}
\end{Entry}

\begin{Entry}{带领}{9,11}{⼱,⾴}
  \begin{Phonetics}{带领}{dai4ling3}[][HSK 3]
    \definition{v.}{guiar, na frente, liderando | liderar e comandar}
  \end{Phonetics}
\end{Entry}

\begin{Entry}{带路}{9,13}{⼱,⾜}
  \begin{Phonetics}{带路}{dai4/lu4}[][HSK 7-9]
    \definition{v.+compl.}{mostrar o caminho; agir como um guia; guiar}
  \end{Phonetics}
\end{Entry}

%%%%%%%%%% 帮 %%%%%%%%%%
\subsection*{帮}\addcontentsline{loh}{figure}{帮}

\begin{Entry}{帮}{9}{⼱}
  \begin{Phonetics}{帮}{bang1}[][HSK 1]
    \definition*{s.}{Sobrenome: Bang}
    \definition{clas.}{um grupo de; um bando de; uma gangue de; um grupo de pessoas}
    \definition{s.}{lateral; superior; partes ao lado ou ao redor do objeto | folha externa; parte mais grossa das folhas externas dos vegetais | gangue; banda; grupo; conglomerado | trabalho; refere"-se ao envolvimento em trabalho assalariado}
    \definition{v.}{ajudar; assistir; auxiliar}
  \end{Phonetics}
\end{Entry}

\begin{Entry}{帮手}{9,4}{⼱,⼿}
  \begin{Phonetics}{帮手}{bang1shou5}[][HSK 7-9]
    \definition{s.}{assistente; ajudante}
  \synonymref{帮忙}{bang1/mang2}
  \synonymref{帮助}{bang1zhu4}
  \synonymref{协助}{xie2zhu4}
  \synonymref{助理}{zhu4li3}
  \synonymref{助手}{zhu4shou3}
  \end{Phonetics}
\end{Entry}

\begin{Entry}{帮忙}{9,6}{⼱,⼼}
  \begin{Phonetics}{帮忙}{bang1/mang2}[][HSK 1]
    \definition{v.+compl.}{ajudar; dar uma mão; dar uma mãozinha; fazer um favor; fazer uma boa ação; ajudar os outros a fazer algo, referindo"-se, de maneira geral, a oferecer ajuda quando alguém está com dificuldades}
  \end{Phonetics}
\end{Entry}

\begin{Entry}{帮佣}{9,7}{⼱,⼈}
  \begin{Phonetics}{帮佣}{bang1yong1}
    \definition{s.}{trabalhador doméstico; empregada doméstica; servo; servente}
    \definition{v.}{trabalhar ou ser contratado como trabalhador doméstico, servo, etc.}
  \end{Phonetics}
\end{Entry}

\begin{Entry}{帮助}{9,7}{⼱,⼒}
  \begin{Phonetics}{帮助}{bang1zhu4}[][HSK 2]
    \definition[个,次,回,份,种]{s.}{ajuda; auxílio; socorro; função de promoção ou auxílio}
    \definition{v.}{ajudar; assistir; apoiar; quando alguém está passando por dificuldades, oferecer apoio financeiro ou material, ou ainda apoio moral, dar conselhos, pensar em soluções, fazer coisas por essa pessoa, etc.}
  \end{Phonetics}
\end{Entry}

\begin{Entry}{帮教}{9,11}{⼱,⽁}
  \begin{Phonetics}{帮教}{bang1jiao4}
    \definition{v.}{orientar}
  \end{Phonetics}
\end{Entry}

%%%%%%%%%% 幽 %%%%%%%%%%
\subsection*{幽}\addcontentsline{loh}{figure}{幽}

\begin{Entry}{幽}{9}{⼳}
  \begin{Phonetics}{幽}{you1}
    \definition*{s.}{Sobrenome: You}
    \definition{adj.}{profundo e remoto; isolado; escuro | secreto; escondido; oculto; não público | quieto; tranquilo; sereno | do mundo inferior}
    \definition{s.}{mundo inferior}
  \end{Phonetics}
\end{Entry}

\begin{Entry}{幽默}{9,16}{⼳,⿊}
  \begin{Phonetics}{幽默}{you1mo4}[][HSK 5]
    \definition{adj.}{humorístico; interessante ou engraçado, mas com um significado profundo}
    \definition{s.}{humor; lado engraçado; graça; características, temperamento, palavras ou comportamentos interessantes, engraçados ou significativos}
  \end{Phonetics}
\end{Entry}

%%%%%%%%%% 度 %%%%%%%%%%
\subsection*{度}\addcontentsline{loh}{figure}{度}

\begin{Entry}{度}{9}{⼴}
  \begin{Phonetics}{度}{du4}[][HSK 2]
    \definition*{s.}{Sobrenome: Du}
    \definition{clas.}{grau; unidade de medida para ângulos, temperatura, etc. | quilowatt"-hora (kWh) | usado para indicar a quantidade de álcool presente no vinho | usado para arcos e ângulos | usado para indicar o grau de curvatura da lente dos óculos ou o grau de miopia | tempo; número de vezes | usado para longitude e latitude, localização geográfica}
    \definition{s.}{medida linear; padrões e instrumentos para medir comprimentos | grau de intensidade; refere"-se especificamente ao grau alcançado por uma determinada propriedade de uma coisa | limite; extensão; grau; quota | regras; código de conduta; diretrizes | tolerância; magnanimidade; refere"-se especificamente ao grau de tolerância | maneira; temperamento; disposição; a personalidade ou aparência de uma pessoa | indicador de grau, nível alcançado por algo | tempo ou espaço limitado; um determinado período de tempo ou espaço}
    \definition{v.}{passar; atravessar; passar por cima | (em termos de tempo) passar; passar por | (de monges ou monjas budistas, ou sacerdotes taoístas) pregar; converter; proselitar}
  \end{Phonetics}
  \begin{Phonetics}{度}{duo2}
    \definition{v.}{supor; estimar; especular}
  \end{Phonetics}
\end{Entry}

\begin{Entry}{度过}{9,6}{⼴,⾡}
  \begin{Phonetics}{度过}{du4guo4}[][HSK 4]
    \definition{s.}{passar o tempo; fazer o tempo desaparecer no trabalho, na vida, no lazer e no descanso}
  \end{Phonetics}
\end{Entry}

\begin{Entry}{度知名度}{9,8,6,9}{⼴,⽮,⼝,⼴}
  \begin{Phonetics}{度知名度}{du4 zhi1ming2du4}
    \definition{s.}{popularidade}
  \end{Phonetics}
\end{Entry}

\begin{Entry}{度假}{9,11}{⼴,⼈}
  \begin{Phonetics}{度假}{du4jia4}[][HSK 7-9]
    \definition{v.}{sair de férias; passar as férias}
  \end{Phonetics}
\end{Entry}

%%%%%%%%%% 弯 %%%%%%%%%%
\subsection*{弯}\addcontentsline{loh}{figure}{弯}

\begin{Entry}{弯}{9}{⼸}
  \begin{Phonetics}{弯}{wan1}[][HSK 4]
    \definition{adj.}{curvo; tortuoso; torto | para algo curvo, como a lua, etc. | dobrado; flexível}
    \definition[个,道]{s.}{curva; dobra; volta}
    \definition{v.}{dobrar; flexionar; curvar | Literário: desenhar}
  \end{Phonetics}
\end{Entry}

\begin{Entry}{弯曲}{9,6}{⼸,⽈}
  \begin{Phonetics}{弯曲}{wan1qu1}[][HSK 6]
    \definition{s.}{torto; curvo; sinuoso; tortuoso; não reto}
    \definition{v.}{dobrar; curvar; flexionar}
  \end{Phonetics}
\end{Entry}

%%%%%%%%%% 待 %%%%%%%%%%
\subsection*{待}\addcontentsline{loh}{figure}{待}

\begin{Entry}{待}{9}{⼻}
  \begin{Phonetics}{待}{dai1}[][HSK 5]
    \definition{v.}{ficar; permanecer | ir além (de um período de tempo)}
  \end{Phonetics}
  \begin{Phonetics}{待}{dai4}[][HSK 7-9]
    \definition*{s.}{Sobrenome: Dai}
    \definition{v.}{tratar; lidar com | entreter; receber (convidados) | aguardar; esperar por | precisar; necessitar | desejar; pretender; querer}
  \end{Phonetics}
\end{Entry}

\begin{Entry}{待会儿}{9,6,2}{⼻,⼈,⼉}
  \begin{Phonetics}{待会儿}{dai1hui4r5}[][HSK 6]
    \definition{adv.}{em um momento; depois de um tempo | mais tarde; depois}
  \end{Phonetics}
\end{Entry}

\begin{Entry}{待遇}{9,12}{⼻,⾡}
  \begin{Phonetics}{待遇}{dai4yu4}[][HSK 4]
    \definition[种,项,份]{s.}{tratamento; refere"-se a direitos, status social, etc. | salário; ordenado; remuneração}
  \end{Phonetics}
\end{Entry}

%%%%%%%%%% 很 %%%%%%%%%%
\subsection*{很}\addcontentsline{loh}{figure}{很}

\begin{Entry}{很}{9}{⼻}
  \begin{Phonetics}{很}{hen3}[][HSK 1]
    \definition{adv.}{muito; bastante; terrivelmente; indica um grau bastante elevado; definitivo; o mais alto}
  \end{Phonetics}
\end{Entry}

\begin{Entry}{很难说}{9,10,9}{⼻,⾫,⾔}
  \begin{Phonetics}{很难说}{hen3 nan2shuo1}[][HSK 6]
    \definition{adj.}{difícil dizer}
  \end{Phonetics}
\end{Entry}

%%%%%%%%%% 律 %%%%%%%%%%
\subsection*{律}\addcontentsline{loh}{figure}{律}

\begin{Entry}{律}{9}{⼻}
  \begin{Phonetics}{律}{lv4}
    \definition*{s.}{Sobrenome: Lü}
    \definition{s.}{lei; regra; estatuto; regulamento}
    \definition{v.}{restringir; disciplinar; manter sob controle}
  \end{Phonetics}
\end{Entry}

\begin{Entry}{律师}{9,6}{⼻,⼱}
  \begin{Phonetics}{律师}{lv4shi1}[][HSK 4]
    \definition[名,个,位]{s.}{advogado; procurador; profissionais encarregados pelas partes ou nomeados pelo tribunal para auxiliar as partes no litígio, para comparecer ao tribunal para defesa e para tratar de assuntos jurídicos relacionados, de acordo com a lei}
  \end{Phonetics}
\end{Entry}

%%%%%%%%%% 怎 %%%%%%%%%%
\subsection*{怎}\addcontentsline{loh}{figure}{怎}

\begin{Entry}{怎}{9}{⼼}
  \begin{Phonetics}{怎}{zen3}
    \definition{adv.}{como}
  \end{Phonetics}
\end{Entry}

\begin{Entry}{怎么}{9,3}{⼼,⼃}
  \begin{Phonetics}{怎么}{zen3me5}[][HSK 1]
    \definition{pron.}{como?; o quê?; perguntas sobre natureza, situação, método, motivo, etc. | de qualquer maneira; não importa como; de uma certa maneira; referência geral à natureza, condição ou modo | que? (usado sozinho no início de uma frase para expressar surpresa) | usado após 不 e 没, indica um grau baixo e é uma forma mais educada de se expressar | usado em perguntas retóricas}
  \seealsoref{不}{bu4}
  \seealsoref{没}{mei2}
  \end{Phonetics}
\end{Entry}

\begin{Entry}{怎么了}{9,3,2}{⼼,⼃,⼅}
  \begin{Phonetics}{怎么了}{zen3me5le5}
    \definition{expr.}{O que aconteceu? | O que está acontecendo? | E aí?}
  \end{Phonetics}
\end{Entry}

\begin{Entry}{怎么办}{9,3,4}{⼼,⼃,⼒}
  \begin{Phonetics}{怎么办}{zen3me5ban4}[][HSK 2]
    \definition{adv.}{o que fazer?; o que deve ser feito?}
  \end{Phonetics}
\end{Entry}

\begin{Entry}{怎么回事}{9,3,6,8}{⼼,⼃,⼞,⼅}
  \begin{Phonetics}{怎么回事}{zen3me5hui2shi4}
    \definition{expr.}{O que aconteceu? | O que se passou?}
  \end{Phonetics}
\end{Entry}

\begin{Entry}{怎么样}{9,3,10}{⼼,⼃,⽊}
  \begin{Phonetics}{怎么样}{zen3me5yang4}[][HSK 2]
    \definition{adv.}{como?; o que?; como é?; como estão as coisas?; o que você acha?; pergunte sobre o método, natureza, situação, opinião, etc. | substitui uma ação ou situação não dita (usado apenas na forma negativa, mais eufemístico do que uma declaração direta); indaga sobre a natureza, condição, método, razão, etc.}
  \end{Phonetics}
\end{Entry}

\begin{Entry}{怎么得了}{9,3,11,2}{⼼,⼃,⼻,⼅}
  \begin{Phonetics}{怎么得了}{zen3me5de2liao3}
    \definition{expr.}{Como isso pode ser? | Que bagunça horrível! | O que deve ser feito?}
  \end{Phonetics}
\end{Entry}

\begin{Entry}{怎么搞的}{9,3,13,8}{⼼,⼃,⼿,⽩}
  \begin{Phonetics}{怎么搞的}{zen3me5gao3de5}
    \definition{expr.}{Como isso aconteceu? | O que deu errado? | E aí? | O que está errado?}
  \end{Phonetics}
\end{Entry}

\begin{Entry}{怎样}{9,10}{⼼,⽊}
  \begin{Phonetics}{怎样}{zen3yang4}[][HSK 2]
    \definition{pron.}{como?; o que?; indagar sobre a natureza, condição ou método, etc. | como?; indica uma referência virtual | de uma certa maneira; de qualquer maneira; não importa como; indica qualquer | como?; usado como predicado, objeto ou complemento para indagar sobre uma situação}
  \end{Phonetics}
\end{Entry}

%%%%%%%%%% 怒 %%%%%%%%%%
\subsection*{怒}\addcontentsline{loh}{figure}{怒}

\begin{Entry}{怒}{9}{⼼}
  \begin{Phonetics}{怒}{nu4}
    \definition{adj.}{zangado; furioso | feroz; forte; descreve um forte impulso}
    \definition{adv.}{com força; vigorosamente; dinamicamente | com raiva}
    \definition{s.}{raiva; fúria}
    \definition{v.}{enfurecer-se; ficar com raiva}
  \end{Phonetics}
\end{Entry}

\begin{Entry}{怒放}{9,8}{⼼,⽅}
  \begin{Phonetics}{怒放}{nu4fang4}
    \definition{v.}{florescer em plena floração}
  \end{Phonetics}
\end{Entry}

\begin{Entry}{怒骂}{9,9}{⼼,⾺}
  \begin{Phonetics}{怒骂}{nu4ma4}
    \definition{v.}{praguejar de raiva}
  \end{Phonetics}
\end{Entry}

%%%%%%%%%% 思 %%%%%%%%%%
\subsection*{思}\addcontentsline{loh}{figure}{思}

\begin{Entry}{思}{9}{⼼}
  \begin{Phonetics}{思}{si1}
    \definition*{s.}{Sobrenome: Si}
    \definition{s.}{pensamento; ideias | pensamentos; emoções; humor}
    \definition{v.}{pensar; considerar; deliberar | pensar em; ansiar por}
  \end{Phonetics}
\end{Entry}

\begin{Entry}{思考}{9,6}{⼼,⽼}
  \begin{Phonetics}{思考}{si1kao3}[][HSK 4]
    \definition{v.}{pensar; ponderar; considerar; deliberar; envolver-se em atividades de pensamento, como análise, síntese, julgamento, raciocínio e generalização}
  \end{Phonetics}
\end{Entry}

\begin{Entry}{思维}{9,11}{⼼,⽷}
  \begin{Phonetics}{思维}{si1wei2}[][HSK 5]
    \definition[种]{s.}{pensamento; reflexão; organizar e transformar os materiais obtidos através do conhecimento sensorial para formar conceitos, julgamentos e raciocínios}
    \definition{v.}{pensar}
  \end{Phonetics}
\end{Entry}

\begin{Entry}{思想}{9,13}{⼼,⼼}
  \begin{Phonetics}{思想}{si1xiang3}[][HSK 3]
    \definition[个,种]{s.}{reflexão; pensamento; ideologia; a existência objetiva é refletida na consciência das pessoas por meio de atividades de pensamento, que pertencem à cognição racional | ideia; pensamento}
  \end{Phonetics}
\end{Entry}

%%%%%%%%%% 怠 %%%%%%%%%%
\subsection*{怠}\addcontentsline{loh}{figure}{怠}

\begin{Entry}{怠}{9}{⼼}
  \begin{Phonetics}{怠}{dai4}
    \definition{adj.}{ocioso; relaxado; negligente | preguiçoso; indolente}
    \definition{v.}{ficar ocioso; negligenciar; folgar}
  \end{Phonetics}
\end{Entry}

\begin{Entry}{怠工}{9,3}{⼼,⼯}
  \begin{Phonetics}{怠工}{dai4/gong1}[][HSK 7-9]
    \definition{v.+compl.}{ir devagar (como uma forma de greve); ir devagar | afrouxar no trabalho}
  \end{Phonetics}
\end{Entry}

\begin{Entry}{怠慢}{9,14}{⼼,⼼}
  \begin{Phonetics}{怠慢}{dai4man4}[][HSK 7-9]
    \definition{v.}{ignorar; tratar com indiferença | deixar de dar a devida atenção para alguém}
  \end{Phonetics}
\end{Entry}

%%%%%%%%%% 急 %%%%%%%%%%
\subsection*{急}\addcontentsline{loh}{figure}{急}

\begin{Entry}{急}{9}{⼼}
  \begin{Phonetics}{急}{ji2}[][HSK 2]
    \definition{adj.}{impaciente; ansioso | irritado; aborrecido; incomodado | rápido e intenso; veloz | urgente; premente}
    \definition{s.}{urgência; emergência; assunto urgente e grave}
    \definition{v.}{preocupar; deixar ansioso | estar ansioso para ajudar; tratar os problemas dos outros como se fossem urgentes e ajudar a resolvê-los imediatamente}
  \antonymref{缓}{huan3}
  \end{Phonetics}
\end{Entry}

\begin{Entry}{急于}{9,3}{⼼,⼆}
  \begin{Phonetics}{急于}{ji2yu2}[][HSK 7-9]
    \definition{v.}{estar (ser) ansioso; estar (ser) impaciente; estar ansioso para}
  \end{Phonetics}
\end{Entry}

\begin{Entry}{急忙}{9,6}{⼼,⼼}
  \begin{Phonetics}{急忙}{ji2mang2}[][HSK 4]
    \definition{adv.}{apressadamente; com pressa}
  \end{Phonetics}
\end{Entry}

\begin{Entry}{急诊}{9,7}{⼼,⾔}
  \begin{Phonetics}{急诊}{ji2zhen3}[][HSK 7-9]
    \definition{s.}{pronto-socorro; emergência; tratamento de emergência; uma clínica ambulatorial especial em um hospital para pessoas com doenças agudas}
  \end{Phonetics}
\end{Entry}

\begin{Entry}{急性}{9,8}{⼼,⼼}
  \begin{Phonetics}{急性}{ji2xing4}[][HSK 7-9]
    \definition{adj.}{aguda}
    \definition{s.}{pessoa impetuosa; cabeça quente}
  \seealsoref{急性儿}{ji2xing4r5}
  \antonymref{慢性}{man4xing4}
  \end{Phonetics}
\end{Entry}

\begin{Entry}{急性儿}{9,8,2}{⼼,⼼,⼉}
  \begin{Phonetics}{急性儿}{ji2xing4r5}
    \definition{adj.}{impetuoso; temperamental; de temperamento explosivo; de disposição impaciente}
  \end{Phonetics}
\end{Entry}

\begin{Entry}{急转弯}{9,8,9}{⼼,⾞,⼸}
  \begin{Phonetics}{急转弯}{ji2zhuan3wan1}[][HSK 7-9]
    \definition{s.}{curva fechada; curva acetuada; cotovelo}
    \definition{v.}{Coloquial: (uma atitude, política, etc.) fazer uma mudança repentina | fazer uma curva repentina | fazer uma mudança radical}
  \seealsoref{急转弯儿}{ji2zhuan3wan1r5}
  \end{Phonetics}
\end{Entry}

\begin{Entry}{急转弯儿}{9,8,9,2}{⼼,⾞,⼸,⼉}
  \begin{Phonetics}{急转弯儿}{ji2zhuan3wan1r5}
    \definition{s.}{curva fechada}
  \end{Phonetics}
\end{Entry}

\begin{Entry}{急迫}{9,8}{⼼,⾡}
  \begin{Phonetics}{急迫}{ji2po4}[][HSK 7-9]
    \definition{adj.}{urgente; premente; imperativo}
  \end{Phonetics}
\end{Entry}

\begin{Entry}{急剧}{9,10}{⼼,⼑}
  \begin{Phonetics}{急剧}{ji2ju4}[][HSK 7-9]
    \definition{adj.}{rápido; agudo; repentino}
    \definition{adv.}{Formal: rapidamente (geralmente mudando em uma direção ruim ou potencialmente levando a resultados ruins)}
  \end{Phonetics}
\end{Entry}

\begin{Entry}{急救}{9,11}{⼼,⽁}
  \begin{Phonetics}{急救}{ji2jiu4}[][HSK 6]
    \definition{s.}{primeiros socorros; tratamento médico de emergência (para pessoas gravemente doentes ou gravemente feridas)}
    \definition{v.}{prestar primeiros socorros; dar tratamento de emergência}
  \end{Phonetics}
\end{Entry}

\begin{Entry}{急需}{9,14}{⼼,⾬}
  \begin{Phonetics}{急需}{ji2xu1}[][HSK 7-9]
    \definition{v.}{estar em extrema necessidade de}
  \end{Phonetics}
\end{Entry}

%%%%%%%%%% 怨 %%%%%%%%%%
\subsection*{怨}\addcontentsline{loh}{figure}{怨}

\begin{Entry}{怨}{9}{⼼}
  \begin{Phonetics}{怨}{yuan4}[][HSK 5]
    \definition{s.}{ressentimento; inimizade; rancor}
    \definition{v.}{culpar; reclamar}
  \end{Phonetics}
\end{Entry}

%%%%%%%%%% 怹 %%%%%%%%%%
\subsection*{怹}\addcontentsline{loh}{figure}{怹}

\begin{Entry}{怹}{9}{⼼}
  \begin{Phonetics}{怹}{tan1}
    \definition{pron.}{ele, ela (cortês)}
  \synonymref{他}{ta1}
  \end{Phonetics}
\end{Entry}

%%%%%%%%%% 总 %%%%%%%%%%
\subsection*{总}\addcontentsline{loh}{figure}{总}

\begin{Entry}{总}{9}{⼼}
  \begin{Phonetics}{总}{zong3}[][HSK 3]
    \definition{adj.}{total; geral; global | responsável (liderança)}
    \definition{adv.}{sempre; invariavelmente | de qualquer forma; afinal; eventualmente; mais cedo ou mais tarde; no fim das contas | certamente; provavelmente; com certeza; expressa estimativa; suposição; equivalente a 大概}
    \definition{v.}{reunir; resumir; juntar; compilar}
  \seealsoref{大概}{da4gai4}
  \end{Phonetics}
\end{Entry}

\begin{Entry}{总之}{9,3}{⼼,⼂}
  \begin{Phonetics}{总之}{zong3zhi1}[][HSK 4]
    \definition{conj.}{em uma palavra; em suma; em resumo; indica que a declaração seguinte é uma declaração geral}
  \end{Phonetics}
\end{Entry}

\begin{Entry}{总长}{9,4}{⼼,⾧}
  \begin{Phonetics}{总长}{zong3chang2}
    \definition{s.}{comprimento total}
  \end{Phonetics}
\end{Entry}

\begin{Entry}{总务}{9,5}{⼼,⼒}
  \begin{Phonetics}{总务}{zong3wu4}
    \definition{s.}{divisão de assuntos gerais | assuntos gerais | pessoa responsável geral}
  \end{Phonetics}
\end{Entry}

\begin{Entry}{总台}{9,5}{⼼,⼝}
  \begin{Phonetics}{总台}{zong3tai2}
    \definition{s.}{recepção | balcão de recepção}
  \end{Phonetics}
\end{Entry}

\begin{Entry}{总价}{9,6}{⼼,⼈}
  \begin{Phonetics}{总价}{zong3jia4}
    \definition{s.}{preço total}
  \end{Phonetics}
\end{Entry}

\begin{Entry}{总共}{9,6}{⼼,⼋}
  \begin{Phonetics}{总共}{zong3gong4}[][HSK 4]
    \definition{adv.}{em tudo; em todos; no total; completamente; totalmente; em conjunto}
  \end{Phonetics}
\end{Entry}

\begin{Entry}{总体}{9,7}{⼼,⼈}
  \begin{Phonetics}{总体}{zong3ti3}[][HSK 5]
    \definition{s.}{total; geral; conjunto; totalidade; massa; população; o todo formado pela união de vários indivíduos; a totalidade das coisas}
  \end{Phonetics}
\end{Entry}

\begin{Entry}{总线}{9,8}{⼼,⽷}
  \begin{Phonetics}{总线}{zong3xian4}
    \definition{s.}{barramento (computador) | \emph{computer bus}}
  \end{Phonetics}
\end{Entry}

\begin{Entry}{总经理}{9,8,11}{⼼,⽷,⽟}
  \begin{Phonetics}{总经理}{zong3jing1li3}[][HSK 6]
    \definition[位,名,个,些]{s.}{CEO; gerente geral; o mais alto executivo de uma empresa ou organização similar, que geralmente tem o poder de decidir políticas administrativas e de gestão}
  \end{Phonetics}
\end{Entry}

\begin{Entry}{总是}{9,9}{⼼,⽇}
  \begin{Phonetics}{总是}{zong3shi4}[][HSK 3]
    \definition{adv.}{sempre; indica como tem sido durante um determinado período de tempo; um determinado estado permanece inalterado | afinal; significa que, independentemente do que acontecer, haverá ou será um resultado}
  \end{Phonetics}
\end{Entry}

\begin{Entry}{总结}{9,9}{⼼,⽷}
  \begin{Phonetics}{总结}{zong3jie2}[][HSK 3]
    \definition[个,篇]{s.}{resumo; síntese; conclusão resumida}
    \definition{v.}{resumir; sumariar; sintetizar; analisar e estudar as experiências para chegar a conclusões}
  \end{Phonetics}
\end{Entry}

\begin{Entry}{总统}{9,9}{⼼,⽷}
  \begin{Phonetics}{总统}{zong3tong3}[][HSK 4]
    \definition*[个,位,名,家]{s.}{Presidente (de um país); Título dos líderes de determinadas repúblicas}
  \end{Phonetics}
\end{Entry}

\begin{Entry}{总值}{9,10}{⼼,⼈}
  \begin{Phonetics}{总值}{zong3zhi2}
    \definition{s.}{valor total}
  \end{Phonetics}
\end{Entry}

\begin{Entry}{总监}{9,10}{⼼,⽫}
  \begin{Phonetics}{总监}{zong3jian1}[][HSK 6]
    \definition[名,位]{s.}{inspetor geral; inspetor-chefe}
  \end{Phonetics}
\end{Entry}

\begin{Entry}{总站}{9,10}{⼼,⽴}
  \begin{Phonetics}{总站}{zong3zhan4}
    \definition{s.}{terminal}
  \end{Phonetics}
\end{Entry}

\begin{Entry}{总部}{9,10}{⼼,⾢}
  \begin{Phonetics}{总部}{zong3bu4}[][HSK 6]
    \definition{s.}{sede geral; escritório central}
  \end{Phonetics}
\end{Entry}

\begin{Entry}{总得}{9,11}{⼼,⼻}
  \begin{Phonetics}{总得}{zong3dei3}
    \definition{adv.}{prestes a}
    \definition{v.}{dever | precisar}
  \end{Phonetics}
\end{Entry}

\begin{Entry}{总理}{9,11}{⼼,⽟}
  \begin{Phonetics}{总理}{zong3li3}[][HSK 4]
    \definition*[个,位,名]{s.}{Primeiro-Ministro do Conselho de Estado; Título do líder do Conselho de Estado da China | Título do chefe de governo em determinados países | Primeiro-Ministro; Título de líderes de determinados partidos políticos | Título dos chefes de determinadas instituições e empresas nos velhos tempos}
    \definition{v.}{assumir a responsabilidade total}
  \end{Phonetics}
\end{Entry}

\begin{Entry}{总裁}{9,12}{⼼,⾐}
  \begin{Phonetics}{总裁}{zong3cai2}[][HSK 5]
    \definition[位,名,个]{s.}{presidente (de uma empresa); nomes de certos líderes de partidos políticos ou grandes empresas}
  \end{Phonetics}
\end{Entry}

\begin{Entry}{总量}{9,12}{⼼,⾥}
  \begin{Phonetics}{总量}{zong3liang4}[][HSK 6]
    \definition{s.}{capacidade total; quantidade bruta | valor total | total}
  \end{Phonetics}
\end{Entry}

\begin{Entry}{总数}{9,13}{⼼,⽁}
  \begin{Phonetics}{总数}{zong3shu4}[][HSK 5]
    \definition{s.}{soma; total; totalidade; inventário; número total; soma total}
  \end{Phonetics}
\end{Entry}

\begin{Entry}{总督}{9,13}{⼼,⽬}
  \begin{Phonetics}{总督}{zong3du1}
    \definition*{s.}{Governador-Geral | Governador | Vice-Rei}
  \end{Phonetics}
\end{Entry}

\begin{Entry}{总算}{9,14}{⼼,⽵}
  \begin{Phonetics}{总算}{zong3suan4}[][HSK 5]
    \definition{adv.}{finalmente; por fim; indica que, após um longo período de tempo, um desejo finalmente se tornou realidade | suficiente; considerando tudo; no geral; considerando todos os aspectos; significa que, em geral, está tudo bem}
  \end{Phonetics}
\end{Entry}

%%%%%%%%%% 恍 %%%%%%%%%%
\subsection*{恍}\addcontentsline{loh}{figure}{恍}

\begin{Entry}{恍}{9}{⼼}
  \begin{Phonetics}{恍}{huang3}
    \definition{adv.}{(junto com 如, 若, etc.) parecer; como se | de repente}
  \seealsoref{如}{ru2}
  \seealsoref{若}{ruo4}
  \end{Phonetics}
\end{Entry}

\begin{Entry}{恍然大悟}{9,12,3,10}{⼼,⽕,⼤,⼼}
  \begin{Phonetics}{恍然大悟}{huang3ran2-da4wu4}[][HSK 7-9]
    \definition{expr.}{de repente ver a luz; de repente perceber o que aconteceu; perceber de repente}
  \end{Phonetics}
\end{Entry}

%%%%%%%%%% 恒 %%%%%%%%%%
\subsection*{恒}\addcontentsline{loh}{figure}{恒}

\begin{Entry}{恒}{9}{⼼}
  \begin{Phonetics}{恒}{heng2}
    \definition*{s.}{Sobrenome: Heng}
    \definition{adj.}{permanente; duradouro | usual; comum; constante | usual; frequente; constante}
    \definition{s.}{perseverança; constância}
  \end{Phonetics}
\end{Entry}

\begin{Entry}{恒山}{9,3}{⼼,⼭}
  \begin{Phonetics}{恒山}{heng2shan1}
    \definition*{s.}{Monte Heng em Shanxi, montanha norte das Cinco Montanhas Sagradas (五岳) | Distrito de Hengshan na cidade de Jixi (鸡西), Heilongjiang}
  \seealsoref{鸡西}{ji1xi1}
  \seealsoref{五岳}{wu3yue4}
  \end{Phonetics}
\end{Entry}

\begin{Entry}{恒星系}{9,9,7}{⼼,⽇,⽷}
  \begin{Phonetics}{恒星系}{heng2xing1xi4}
    \definition{s.}{sistema estelar | galáxia}
  \end{Phonetics}
\end{Entry}

%%%%%%%%%% 恢 %%%%%%%%%%
\subsection*{恢}\addcontentsline{loh}{figure}{恢}

\begin{Entry}{恢}{9}{⼼}
  \begin{Phonetics}{恢}{hui1}
    \definition{adj.}{extenso; vasto | grande; ótimo}
    \definition{v.}{recuperar; restaurar; restabelecer}
  \end{Phonetics}
\end{Entry}

\begin{Entry}{恢复}{9,9}{⼼,⼢}
  \begin{Phonetics}{恢复}{hui1fu4}[][HSK 5]
    \definition{v.}{retomar; renovar; restaurar; voltar a | reviver; recuperar; reencontrar | restaurar; restabelecer; reabilitar; regenerar; ressurgir; restabelecer alguém em; recuperar o que foi perdido}
  \end{Phonetics}
\end{Entry}

%%%%%%%%%% 恤 %%%%%%%%%%
\subsection*{恤}\addcontentsline{loh}{figure}{恤}

\begin{Entry}{恤}{9}{⼼}
  \begin{Phonetics}{恤}{xu4}
    \definition{v.}{ter pena; simpatizar | dar alívio; compensar}
  \end{Phonetics}
\end{Entry}

%%%%%%%%%% 恨 %%%%%%%%%%
\subsection*{恨}\addcontentsline{loh}{figure}{恨}

\begin{Entry}{恨}{9}{⼼}
  \begin{Phonetics}{恨}{hen4}[][HSK 5]
    \definition{s.}{ódio; resentimento}
    \definition{v.}{odiar; ressentir-se}
  \end{Phonetics}
\end{Entry}

\begin{Entry}{恨不得}{9,4,11}{⼼,⼀,⼻}
  \begin{Phonetics}{恨不得}{hen4bu5de5}[][HSK 7-9]
    \definition{v.}{estar muito ansioso para; querer poder (fazer algo); mal poder esperar para; expressa um desejo ansioso de realizar algo, geralmente usado para coisas que não podem realmente ser feitas}
  \end{Phonetics}
\end{Entry}

%%%%%%%%%% 恰 %%%%%%%%%%
\subsection*{恰}\addcontentsline{loh}{figure}{恰}

\begin{Entry}{恰}{9}{⼼}
  \begin{Phonetics}{恰}{qia4}
    \definition{adv.}{exatamente | apenas}
  \end{Phonetics}
\end{Entry}

\begin{Entry}{恰巧}{9,5}{⼼,⼯}
  \begin{Phonetics}{恰巧}{qia4qiao3}[][HSK 7-9]
    \definition{adv.}{por acaso; felizmente ou infelizmente; perfeitamente; por coincidência}
  \end{Phonetics}
\end{Entry}

\begin{Entry}{恰好}{9,6}{⼼,⼥}
  \begin{Phonetics}{恰好}{qia4hao3}[][HSK 6]
    \definition{adv.}{na medida certa; como a sorte quis}
  \end{Phonetics}
\end{Entry}

\begin{Entry}{恰如其分}{9,6,8,4}{⼼,⼥,⼋,⼑}
  \begin{Phonetics}{恰如其分}{qia4ru2-qi2fen4}[][HSK 7-9]
    \definition{expr.}{``Na medida certa.''; apropriado; agir ou falar de maneira diplomática}
  \end{Phonetics}
\end{Entry}

\begin{Entry}{恰当}{9,6}{⼼,⼹}
  \begin{Phonetics}{恰当}{qia4dang4}[][HSK 6]
    \definition{adj.}{adequado; apropriado; conveniente; apropriado; a linguagem ou abordagem é muito apropriada}
  \end{Phonetics}
\end{Entry}

\begin{Entry}{恰到好处}{9,8,6,5}{⼼,⼑,⼥,⼡}
  \begin{Phonetics}{恰到好处}{qia4dao4-hao3chu4}[][HSK 7-9]
    \definition{expr.}{``Na medida certa.''; perfeito (para o propósito ou ocasião); significa que as palavras e ações de alguém atingiram o ponto mais apropriado}
  \end{Phonetics}
\end{Entry}

\begin{Entry}{恰恰}{9,9}{⼼,⼼}
  \begin{Phonetics}{恰恰}{qia4qia4}[][HSK 6]
    \definition{adv.}{justamente; exatamente; precisamente; bem na hora}
  \end{Phonetics}
\end{Entry}

\begin{Entry}{恰恰相反}{9,9,9,4}{⼼,⼼,⽬,⼜}
  \begin{Phonetics}{恰恰相反}{qia4qia4 xiang1fan3}[][HSK 7-9]
    \definition{expr.}{pelo contrário}
  \end{Phonetics}
\end{Entry}

%%%%%%%%%% 恼 %%%%%%%%%%
\subsection*{恼}\addcontentsline{loh}{figure}{恼}

\begin{Entry}{恼}{9}{⼼}
  \begin{Phonetics}{恼}{nao3}
    \definition{adj.}{infeliz; preocupado; aflito; angustiado}
    \definition{v.}{perturbar; irritar; incomodar | ficar com raiva; ficar irritado}
  \end{Phonetics}
\end{Entry}

\begin{Entry}{恼羞成怒}{9,10,6,9}{⼼,⽺,⼽,⼼}
  \begin{Phonetics}{恼羞成怒}{nao3xiu1-cheng2nu4}[][HSK 7-9]
    \definition{expr.}{ficar furioso de vergonha; ser envergonhado a ponto de ficar com raiva; sentir vergonha a ponto de ficar com raiva; irritação; ficar furioso por causa da humilhação}
  \end{Phonetics}
\end{Entry}

%%%%%%%%%% 战 %%%%%%%%%%
\subsection*{战}\addcontentsline{loh}{figure}{战}

\begin{Entry}{战}{9}{⼽}
  \begin{Phonetics}{战}{zhan4}
    \definition{s.}{luta | guerra | batalha}
    \definition{v.}{lutar}
  \end{Phonetics}
\end{Entry}

\begin{Entry}{战士}{9,3}{⼽,⼠}
  \begin{Phonetics}{战士}{zhan4shi4}[][HSK 4]
    \definition[个,些,名,位]{s.}{soldado; membros mais jovens do exército | campeão; guerreiro; lutador; geralmente, uma pessoa que se engaja em alguma causa justa ou participa de alguma luta justa}
  \end{Phonetics}
\end{Entry}

\begin{Entry}{战友}{9,4}{⼽,⼜}
  \begin{Phonetics}{战友}{zhan4you3}[][HSK 6]
    \definition{s.}{camarada de armas; companheiro de guerra | companheiro de batalha; pessoas lutando juntas}
  \end{Phonetics}
\end{Entry}

\begin{Entry}{战斗}{9,4}{⼽,⽃}
  \begin{Phonetics}{战斗}{zhan4dou4}[][HSK 4]
    \definition[场,次]{s.}{luta; batalha; combate; ação; conflito armado entre as partes oponentes}
    \definition{v.}{lutar; combater | lutar; metáfora para trabalho duro ou labor}
  \end{Phonetics}
\end{Entry}

\begin{Entry}{战术}{9,5}{⼽,⽊}
  \begin{Phonetics}{战术}{zhan4shu4}[][HSK 6]
    \definition[种,套]{s.}{táticas para resolver problemas; geralmente se refere ao método de resolução de problemas locais | táticas (militares); princípios e métodos de condução de combate}
  \end{Phonetics}
\end{Entry}

\begin{Entry}{战争}{9,6}{⼽,⼑}
  \begin{Phonetics}{战争}{zhan4zheng1}[][HSK 4]
    \definition[场,次]{s.}{guerra; conflito; luta armada entre povos, entre nações, entre classes ou entre grupos políticos}
  \end{Phonetics}
\end{Entry}

\begin{Entry}{战场}{9,6}{⼽,⼟}
  \begin{Phonetics}{战场}{zhan4chang3}[][HSK 6]
    \definition[个,片,处]{s.}{campo de batalha; frente de batalha}
  \end{Phonetics}
\end{Entry}

\begin{Entry}{战胜}{9,9}{⼽,⾁}
  \begin{Phonetics}{战胜}{zhan4sheng4}[][HSK 4]
    \definition{v.}{derrotar; vencer; superar; triunfar sobre; metáfora para superar dificuldades e alcançar o sucesso}
  \end{Phonetics}
\end{Entry}

\begin{Entry}{战略}{9,11}{⼽,⽥}
  \begin{Phonetics}{战略}{zhan4lve4}[][HSK 6]
    \definition[个,条]{s.}{estratégia; uma estratégia que orienta todo o processo de guerra (diferente de 战术) | estratégia; refere"-se a uma diretriz geral}
  \seealsoref{战术}{zhan4shu4}
  \end{Phonetics}
\end{Entry}

%%%%%%%%%% 扁 %%%%%%%%%%
\subsection*{扁}\addcontentsline{loh}{figure}{扁}

\begin{Entry}{扁}{9}{⼾}
  \begin{Phonetics}{扁}{bian3}[][HSK 6]
    \definition{adj.}{plano}
    \definition{v.}{Coloquial: bater em alguém}
  \end{Phonetics}
  \begin{Phonetics}{扁}{pian1}
    \definition{adj.}{pequeno | fora do caminho; remoto}
  \end{Phonetics}
\end{Entry}

\begin{Entry}{扁舟}{9,6}{⼾,⾈}
  \begin{Phonetics}{扁舟}{pian1 zhou1}
    \definition[叶,艘]{s.}{pequeno barco; esquife}
  \end{Phonetics}
\end{Entry}

%%%%%%%%%% 拜 %%%%%%%%%%
\subsection*{拜}\addcontentsline{loh}{figure}{拜}

\begin{Entry}{拜}{9}{⼿}
  \begin{Phonetics}{拜}{bai4}
    \definition*{s.}{Sobrenome: Bai}
    \definition{adv.}{respeitosamente (usado na comunicação interpessoal)}
    \definition{v.}{fazer uma visita de cortesia | adorar; prestar homenagem | fazer uma chamada cerimonial | ligar; fazer uma visita | intitular alguém com cerimônia; conceder uma posição oficial ou um determinado título com certa etiqueta | estabelecer ou jurar formalmente relacionamentos}
  \end{Phonetics}
\end{Entry}

\begin{Entry}{拜见}{9,4}{⼿,⾒}
  \begin{Phonetics}{拜见}{bai4jian4}[][HSK 7-9]
    \definition{v.}{fazer uma visita formal; ligar para prestar homenagens | encontrar"-se com alguém superior ou senior}
  \end{Phonetics}
\end{Entry}

\begin{Entry}{拜会}{9,6}{⼿,⼈}
  \begin{Phonetics}{拜会}{bai4hui4}[][HSK 7-9]
    \definition{v.}{fazer uma visita oficial; fazer uma visita de cortesia; visitar; visitar e conhecer (agora usado principalmente para visitas diplomáticas oficiais)}
  \end{Phonetics}
\end{Entry}

\begin{Entry}{拜年}{9,6}{⼿,⼲}
  \begin{Phonetics}{拜年}{bai4/nian2}[][HSK 7-9]
    \definition{v.+compl.}{fazer uma visita de Ano Novo; desejar a alguém um Feliz Ano Novo; fazer uma visita cerimonial no Ano Novo}
  \end{Phonetics}
\end{Entry}

\begin{Entry}{拜托}{9,6}{⼿,⼿}
  \begin{Phonetics}{拜托}{bai4tuo1}[][HSK 7-9]
    \definition{v.}{pedir a alguém para fazer algo; pedir para outra pessoa fazer coisas para você}
  \end{Phonetics}
\end{Entry}

\begin{Entry}{拜访}{9,6}{⼿,⾔}
  \begin{Phonetics}{拜访}{bai4fang3}[][HSK 5]
    \definition{v.}{visitar; fazer uma visita (respeitosamente)}
  \end{Phonetics}
\end{Entry}

%%%%%%%%%% 括 %%%%%%%%%%
\subsection*{括}\addcontentsline{loh}{figure}{括}

\begin{Entry}{括}{9}{⼿}
  \begin{Phonetics}{括}{kuo4}
    \definition{v.}{unir (músculos, etc.); contrair | incluir | adicionar colchetes a | amarrar; empacotar}
  \end{Phonetics}
\end{Entry}

\begin{Entry}{括号}{9,5}{⼿,⼝}
  \begin{Phonetics}{括号}{kuo4hao4}[][HSK 4]
    \definition{s.}{chaves, colchetes e parênteses (em fórmulas aritméticas ou algébricas, os símbolos que indicam a combinação e a ordem de vários números ou termos) | colchetes e parênteses usados como um tipo de sinal de pontuação para mostrar a parte explicativa de uma passagem em um texto}
  \end{Phonetics}
\end{Entry}

\begin{Entry}{括弧}{9,8}{⼿,⼸}
  \begin{Phonetics}{括弧}{kuo4hu2}[][HSK 7-9]
    \definition{s.}{parênteses; também podem se referir a indicadores}
  \end{Phonetics}
\end{Entry}

%%%%%%%%%% 拮 %%%%%%%%%%
\subsection*{拮}\addcontentsline{loh}{figure}{拮}

\begin{Entry}{拮}{9}{⼿}
  \begin{Phonetics}{拮}{jie2}
    \definition{adj.}{trabalhoso | sem dinheiro | antagônico | trabalhando duro | pressionado}
  \end{Phonetics}
\end{Entry}

\begin{Entry}{拮据}{9,11}{⼿,⼿}
  \begin{Phonetics}{拮据}{jie2ju1}
    \definition{adj.}{em circunstâncias difíceis; sem dinheiro; em dificuldades}
  \end{Phonetics}
\end{Entry}

%%%%%%%%%% 拱 %%%%%%%%%%
\subsection*{拱}\addcontentsline{loh}{figure}{拱}

\begin{Entry}{拱}{9}{⼿}
  \begin{Phonetics}{拱}{gong3}[][HSK 7-9]
    \definition*{s.}{Sobrenome: Gong}
    \definition{s.}{Arquitetura: arco}[游客们在拱门前留影。===Turistas tiram fotos em frente ao arco.]
    \definition{v.}{colocar uma mão na outra em frente ao peito (em saudação) | cercar | arquear-se | empurrar sem usar as mãos; bater em um objeto com seu corpo | (porcos, etc.) cavar a terra com o focinho; (minhocas, etc.) contorcer-se na terra | brotar através da terra}
  \end{Phonetics}
\end{Entry}

%%%%%%%%%% 拴 %%%%%%%%%%
\subsection*{拴}\addcontentsline{loh}{figure}{拴}

\begin{Entry}{拴}{9}{⼿}
  \begin{Phonetics}{拴}{shuan1}[][HSK 7-9]
    \definition{v.}{amarrar; prender; enrolar uma corda ou objeto similar em volta do objeto e dar um nó | estar preso a algo; estar atolado; restringir e privar as pessoas de sua liberdade}
  \end{Phonetics}
\end{Entry}

%%%%%%%%%% 拷 %%%%%%%%%%
\subsection*{拷}\addcontentsline{loh}{figure}{拷}

\begin{Entry}{拷}{9}{⼿}
  \begin{Phonetics}{拷}{kao3}
    \definition{v.}{açoitar; bater; torturar | copiar | Dialeto, Empréstimo linguístico: \emph{call}, ligar | vencer | interrogar sob tortura}
  \end{Phonetics}
\end{Entry}

%%%%%%%%%% 拼 %%%%%%%%%%
\subsection*{拼}\addcontentsline{loh}{figure}{拼}

\begin{Entry}{拼}{9}{⼿}
  \begin{Phonetics}{拼}{pin1}[][HSK 5]
    \definition{v.}{montar; juntar as peças | dar tudo de si no trabalho; estar disposto a arriscar a vida (em lutas, no trabalho, etc.); fazer tudo o que for preciso; arriscar tudo}
  \end{Phonetics}
\end{Entry}

\begin{Entry}{拼命}{9,8}{⼿,⼝}
  \begin{Phonetics}{拼命}{pin1/ming4}[][HSK 7-9]
    \definition{adv.}{desesperadamente; arriscar a vida; com todas as forças; por favor, faça algo com o máximo empenho e energia}
    \definition{v.+compl.}{arriscar a própria vida; esforçar-se ao máximo; brigar com alguém ou fazer coisas sem considerar a segurança}
  \end{Phonetics}
\end{Entry}

\begin{Entry}{拼音}{9,9}{⼿,⾳}
  \begin{Phonetics}{拼音}{pin1yin1}
    \definition{s.}{escrita fonética | pinyin (romanização chinesa)}
  \end{Phonetics}
\end{Entry}

\begin{Entry}{拼搏}{9,13}{⼿,⼿}
  \begin{Phonetics}{拼搏}{pin1bo2}[][HSK 7-9]
    \definition{v.}{dar tudo de si; encarar (um desafio) de frente; utilizar todas as suas forças para obter algo ou atingir um objetivo}
  \end{Phonetics}
\end{Entry}

%%%%%%%%%% 拾 %%%%%%%%%%
\subsection*{拾}\addcontentsline{loh}{figure}{拾}

\begin{Entry}{拾}{9}{⼿}
  \begin{Phonetics}{拾}{she4}
    \definition{v.}{subir em passos leves}
  \end{Phonetics}
  \begin{Phonetics}{拾}{shi2}[][HSK 5]
    \definition{num.}{dez (usado no lugar do numeral 十 em cheques, notas bancárias, etc., para evitar erros ou alterações)}
    \definition{v.}{pegar (do chão); recolher}
  \synonymref{捡}{jian3}
  \antonymref{丢}{diu1}
  \end{Phonetics}
\end{Entry}

%%%%%%%%%% 持 %%%%%%%%%%
\subsection*{持}\addcontentsline{loh}{figure}{持}

\begin{Entry}{持}{9}{⼿}
  \begin{Phonetics}{持}{chi2}[][HSK 7-9]
    \definition{v.}{segurar; agarrar | opor; confrontar | apoiar; manter | gerenciar; supervisionar | sequestrar; agarrar (controlar; forçar)}
  \end{Phonetics}
\end{Entry}

\begin{Entry}{持久}{9,3}{⼿,⼃}
  \begin{Phonetics}{持久}{chi2jiu3}[][HSK 7-9]
    \definition{adj.}{duradouro; prolongado; persistente; permanente}
  \end{Phonetics}
\end{Entry}

\begin{Entry}{持之以恒}{9,3,4,9}{⼿,⼂,⼈,⼼}
  \begin{Phonetics}{持之以恒}{chi2zhi1-yi3heng2}[][HSK 7-9]
    \definition{expr.}{perseguir incessantemente | de forma persistente | perseverar em (fazer algo)}
  \end{Phonetics}
\end{Entry}

\begin{Entry}{持有}{9,6}{⼿,⽉}
  \begin{Phonetics}{持有}{chi2you3}[][HSK 6]
    \definition{v.}{segurar; possuir | segurar; ter; abrigar; ter em mente (ideias, opiniões, etc.)}
  \end{Phonetics}
\end{Entry}

\begin{Entry}{持续}{9,11}{⼿,⽷}
  \begin{Phonetics}{持续}{chi2xu4}[][HSK 3]
    \definition{v.}{durar; continuar; sustentar; manter a situação ou as condições como estão, sem alterações}
  \end{Phonetics}
\end{Entry}

%%%%%%%%%% 挂 %%%%%%%%%%
\subsection*{挂}\addcontentsline{loh}{figure}{挂}

\begin{Entry}{挂}{9}{⼿}
  \begin{Phonetics}{挂}{gua4}[][HSK 3]
    \definition{clas.}{usado principalmente para coisas que vêm em conjuntos ou séries}
    \definition{v.}{pendurar; colocar; suspender; usando cordas, ganchos, pregos e outros itens para prender objetos em um ou mais pontos específicos | interromper chamada (telefônica) | colocar alguém em contato com; ligar; telefonar; refere"-se a ligar o telefone, bem como a fazer uma chamada | falhar; fracassar | colocar em registro; registrar | pegar carona; ser pego | preocupar"-se com | ser revestido com; ser coberto com | estar pendente; deixar algo sem solução}
  \end{Phonetics}
\end{Entry}

\begin{Entry}{挂号}{9,5}{⼿,⼝}
  \begin{Phonetics}{挂号}{gua4/hao4}[][HSK 7-9]
    \definition{v.+compl.}{registrar-se (em um hospital, etc.) | enviar através de carta registrada}
  \end{Phonetics}
\end{Entry}

\begin{Entry}{挂号信}{9,5,9}{⼿,⼝,⼈}
  \begin{Phonetics}{挂号信}{gua4hao4xin4}
    \definition{s.}{carta registrada}
  \end{Phonetics}
\end{Entry}

\begin{Entry}{挂失}{9,5}{⼿,⼤}
  \begin{Phonetics}{挂失}{gua4/shi1}[][HSK 7-9]
    \definition{v.+compl.}{relatar a perda de algo; se perder uma nota ou certificado, você deve registrá-lo junto à autoridade emissora ou declará-lo inválido}
  \end{Phonetics}
\end{Entry}

\begin{Entry}{挂念}{9,8}{⼿,⼼}
  \begin{Phonetics}{挂念}{gua4nian4}[][HSK 7-9]
    \definition{v.}{sentir falta; preocupar-se com alguém que está ausente}
  \end{Phonetics}
\end{Entry}

\begin{Entry}{挂钩}{9,9}{⼿,⾦}
  \begin{Phonetics}{挂钩}{gua4gou1}[][HSK 7-9]
    \definition[个,种]{s.}{(vagões ferroviários) acoplamento; manilha; engate | gancho}
    \definition{v.}{acoplar (dois vagões ferroviários); articular | conectar-se com; estabelecer contato com; entrar em contato com; vincular-se a}
  \end{Phonetics}
\end{Entry}

%%%%%%%%%% 指 %%%%%%%%%%
\subsection*{指}\addcontentsline{loh}{figure}{指}

\begin{Entry}{指}{9}{⼿}
  \begin{Phonetics}{指}{zhi3}[][HSK 3]
    \definition*{s.}{Sobrenome: Zhi}
    \definition{clas.}{dígito; largura do dedo; a largura de um dedo é chamada de 一指, que é usado para medir profundidade, largura, etc.}
    \definition{s.}{dedo}
    \definition{v.}{apontar para; indicar; usar o dedo ou a ponta de um objeto para apontar | (pelo) eriçar;  (cabelo) ficar em pé | indicar; mostrar; apontar; demonstrar | referir-se a; dirigir-se a | confiar em; contar com; depender de | criticar; repreender}
  \end{Phonetics}
\end{Entry}

\begin{Entry}{指出}{9,5}{⼿,⼐}
  \begin{Phonetics}{指出}{zhi3chu1}[][HSK 3]
    \definition{v.}{apontar; indicar}
  \end{Phonetics}
\end{Entry}

\begin{Entry}{指头}{9,5}{⼿,⼤}
  \begin{Phonetics}{指头}{zhi3tou5}[][HSK 6]
    \definition[个,根,只]{s.}{dedo da mão ou do pé}
  \end{Phonetics}
\end{Entry}

\begin{Entry}{指甲}{9,5}{⼿,⽥}
  \begin{Phonetics}{指甲}{zhi3jia5}[][HSK 5]
    \definition[个,种]{s.}{unha; unha de agulha; unha de dedo; camada córnea na ponta dos dedos}
  \end{Phonetics}
\end{Entry}

\begin{Entry}{指示}{9,5}{⼿,⽰}
  \begin{Phonetics}{指示}{zhi3shi4}[][HSK 5]
    \definition[点,条,项,个]{s.}{diretriz; instruções; para orientar o trabalho, os superiores emitem opiniões verbais ou escritas aos subordinados}
    \definition{v.}{indicar; apontar; apontar para alguém | instruir; superiores emitem opiniões verbais ou escritas para orientar o trabalho dos subordinados}
  \end{Phonetics}
\end{Entry}

\begin{Entry}{指导}{9,6}{⼿,⼨}
  \begin{Phonetics}{指导}{zhi3dao3}[][HSK 3]
    \definition[位]{s.}{guia; diretor; pessoa que dá orientações}
    \definition{v.}{orientar; dirigir; instruir}
  \end{Phonetics}
\end{Entry}

\begin{Entry}{指定}{9,8}{⼿,⼧}
  \begin{Phonetics}{指定}{zhi3ding4}[][HSK 6]
    \definition{adv.}{certamente; com certeza; reforça o tom de palpite e estimativa}
    \definition{v.}{nomear; atribuir; determinar a pessoa, evento, lugar, conteúdo, etc. que faz algo}
  \end{Phonetics}
\end{Entry}

\begin{Entry}{指责}{9,8}{⼿,⾙}
  \begin{Phonetics}{指责}{zhi3ze2}[][HSK 5]
    \definition{v.}{censurar; criticar; encontrar falhas; repreender}
  \end{Phonetics}
\end{Entry}

\begin{Entry}{指南针}{9,9,7}{⼿,⼗,⾦}
  \begin{Phonetics}{指南针}{zhi3nan2zhen1}
    \definition{s.}{bússola}
  \end{Phonetics}
\end{Entry}

\begin{Entry}{指挥}{9,9}{⼿,⼿}
  \begin{Phonetics}{指挥}{zhi3hui1}[][HSK 4]
    \definition[个,位,名]{s.}{diretor; comandante; despachante; operador | maestro; condutor; pessoa na frente de uma orquestra ou coro que dá instruções sobre como tocar ou cantar}
    \definition{v.}{dirigir; conduzir; comandar; direcionar; emitir ordens de agendamento}
  \end{Phonetics}
\end{Entry}

\begin{Entry}{指标}{9,9}{⼿,⽊}
  \begin{Phonetics}{指标}{zhi3biao1}[][HSK 5]
    \definition[个,种]{s.}{meta; cota; norma; índice; objetivos a serem alcançados | alvo; índice; refletir os requisitos de desenvolvimento em determinados aspectos através de números absolutos ou percentagens de aumento ou diminuição, inclui indicadores quantitativos e qualitativos}
  \end{Phonetics}
\end{Entry}

\begin{Entry}{指着}{9,11}{⼿,⽬}
  \begin{Phonetics}{指着}{zhi3zhe5}[][HSK 6]
    \definition{v.}{apontar}
  \end{Phonetics}
\end{Entry}

\begin{Entry}{指数}{9,13}{⼿,⽁}
  \begin{Phonetics}{指数}{zhi3shu4}[][HSK 6]
    \definition{s.}{Matemática: expoente; refere"-se ao número de vezes que um número é multiplicado por si mesmo, o que é registrado no canto superior direito do número | Estatística: índice, refere"-se à razão entre o valor de um fenômeno econômico em um determinado período e o valor de outro período usado como padrão de comparação, geralmente expresso como uma porcentagem, como o índice de preços ao consumidor}
  \end{Phonetics}
\end{Entry}

%%%%%%%%%% 按 %%%%%%%%%%
\subsection*{按}\addcontentsline{loh}{figure}{按}

\begin{Entry}{按}{9}{⼿}
  \begin{Phonetics}{按}{an4}[][HSK 3]
    \definition{prep.}{de acordo com; à luz de; com base em; em conformidade com}
    \definition{v.}{pressionar; empurrar para baixo; pressionar ou apertar com a mão ou os dedos | pôr de parte; deixar de lado; deixar para mais tarde | restringir; controlar; inibir; parar | verificar; consultar | comentar ou anotar (por um editor ou autor)}
  \end{Phonetics}
\end{Entry}

\begin{Entry}{按时}{9,7}{⼿,⽇}
  \begin{Phonetics}{按时}{an4shi2}[][HSK 4]
    \definition{adv.}{na hora; no horário; pontualmente; de acordo com o tempo estipulado}
  \end{Phonetics}
\end{Entry}

\begin{Entry}{按说}{9,9}{⼿,⾔}
  \begin{Phonetics}{按说}{an4shuo1}[][HSK 7-9]
    \definition{adv.}{no curso normal dos eventos; ordinariamente; normalmente | de acordo com o fato (senso comum); refere"-se a falar de acordo com fatos ou senso comum; como uma questão de razão; expressões semelhantes incluem 按理 e 按理说}
  \seealsoref{按理}{an4li3}
  \seealsoref{按理说}{an4li3 shuo1}
  \end{Phonetics}
\end{Entry}

\begin{Entry}{按理}{9,11}{⼿,⽟}
  \begin{Phonetics}{按理}{an4li3}
    \definition{adv.}{de acordo com o princípio ou a razão; no curso normal dos eventos; normalmente | de acordo com a razão; de acordo com a prática comum; por (bons) direitos}
  \end{Phonetics}
\end{Entry}

\begin{Entry}{按理说}{9,11,9}{⼿,⽟,⾔}
  \begin{Phonetics}{按理说}{an4li3 shuo1}[][HSK 7-9]
    \definition{adv.}{de acordo com o princípio ou a razão; no curso normal dos eventos; normalmente | é razoável dizer que\dots}
  \end{Phonetics}
\end{Entry}

\begin{Entry}{按照}{9,13}{⼿,⽕}
  \begin{Phonetics}{按照}{an4zhao4}[][HSK 3]
    \definition{prep.}{de acordo com; em conformidade com; à luz de; com base em; apresentar os fundamentos ou critérios de julgamento para fazer algo}
  \end{Phonetics}
\end{Entry}

\begin{Entry}{按键}{9,13}{⼿,⾦}
  \begin{Phonetics}{按键}{an4jian4}[][HSK 7-9]
    \definition[个]{s.}{tecla; botão; teclas pressionadas manualmente}[键盘上的按键非常灵敏。===As teclas do teclado são muito responsivas.]
  \end{Phonetics}
\end{Entry}

\begin{Entry}{按摩}{9,15}{⼿,⼿}
  \begin{Phonetics}{按摩}{an4mo2}[][HSK 5]
    \definition{s.}{massagem; empurrar, pressionar, beliscar e amassar o corpo de uma pessoa com as mãos para promover a circulação sanguínea, aumentar a resistência da pele e regular a função dos nervos}
  \end{Phonetics}
\end{Entry}

%%%%%%%%%% 挎 %%%%%%%%%%
\subsection*{挎}\addcontentsline{loh}{figure}{挎}

\begin{Entry}{挎}{9}{⼿}
  \begin{Phonetics}{挎}{kua4}[][HSK 7-9]
    \definition{v.}{carregar no braço | carregar algo sobre o ombro, ao redor do pescoço ou ao lado do corpo | pendurar coisas no ombro, pescoço ou cintura}
  \end{Phonetics}
\end{Entry}

%%%%%%%%%% 挑 %%%%%%%%%%
\subsection*{挑}\addcontentsline{loh}{figure}{挑}

\begin{Entry}{挑}{9}{⼿}
  \begin{Phonetics}{挑}{tiao1}[][HSK 4]
    \definition{clas.}{usado para coisas que são escolhidas ou selecionadas | usado para coisas que podem ser usadas como palhetas}
    \definition{s.}{vara comprida com algo pendurado nas pontas; haste de transporte}
    \definition{v.}{escolher; selecionar | fazer picuinhas; ser hipercrítico; ser meticuloso; ser excessivamente rigoroso nos detalhes | carregar com uma haste de transporte; carregar no ombro; pendurar coisas nas pontas de varas longas e carregá-las em seus ombros}
  \end{Phonetics}
  \begin{Phonetics}{挑}{tiao3}[][HSK 4]
    \definition{s.}{um dos traços dos caracteres chineses; inclinado para cima da esquerda para a direita}
    \definition{v.}{levantar; elevar; erguer | levantar ou apoiar com uma extremidade de uma vara ou objeto semelhante; segurar ou apoiar com a ponta de uma vara etc. | causar conflitos deliberadamente; provocar deliberadamente um conflito | (método de bordado) usar uma agulha para levantar os fios de urdidura ou trama, com a agulha e a linha passando por baixo para formar padrões e desenhos}
  \end{Phonetics}
\end{Entry}

\begin{Entry}{挑战}{9,9}{⼿,⼽}
  \begin{Phonetics}{挑战}{tiao3/zhan4}[][HSK 4]
    \definition{v.+compl.}{desafiar; deixar um oponente deliberadamente irritado e sair para lutar ou lutar consigo mesmo; estimular um oponente a lutar consigo mesmo}
  \end{Phonetics}
\end{Entry}

\begin{Entry}{挑选}{9,9}{⼿,⾡}
  \begin{Phonetics}{挑选}{tiao1xuan3}[][HSK 4]
    \definition{v.}{escolher; optar; selecionar; escolher a pessoa ou coisa certa para o trabalho}
  \end{Phonetics}
\end{Entry}

\begin{Entry}{挑衅}{9,11}{⼿,⾎}
  \begin{Phonetics}{挑衅}{tiao3xin4}
    \definition{s.}{provocação}
    \definition{v.}{provocar; causar problemas; tentar causar conflito ou guerra}
  \end{Phonetics}
\end{Entry}

%%%%%%%%%% 挖 %%%%%%%%%%
\subsection*{挖}\addcontentsline{loh}{figure}{挖}

\begin{Entry}{挖}{9}{⼿}
  \begin{Phonetics}{挖}{wa1}[][HSK 6]
    \definition{v.}{cavar; escavar; arrancar | explorar; sondar | (dialeto) arranhar | escavar a superfície de um objeto com ferramentas ou mãos}
  \end{Phonetics}
\end{Entry}

\begin{Entry}{挖掘机}{9,11,6}{⼿,⼿,⽊}
  \begin{Phonetics}{挖掘机}{wa1jue2ji1}
    \definition{s.}{escavadeira | escavador | escavadora | pá mecânica}
  \end{Phonetics}
\end{Entry}

%%%%%%%%%% 挠 %%%%%%%%%%
\subsection*{挠}\addcontentsline{loh}{figure}{挠}

\begin{Entry}{挠}{9}{⼿}
  \begin{Phonetics}{挠}{nao2}[][HSK 7-9]
    \definition{v.}{coçar; (usar os dedos) para segurar delicadamente | dificultar; obstruir; impedir que os outros façam as coisas sem problemas | recuar; ceder; dar a mão; dobrar, metaforicamente significando ceder}
  \end{Phonetics}
\end{Entry}

%%%%%%%%%% 挡 %%%%%%%%%%
\subsection*{挡}\addcontentsline{loh}{figure}{挡}

\begin{Entry}{挡}{9}{⼿}
  \begin{Phonetics}{挡}{dang3}[][HSK 5]
    \definition{s.}{persiana; veneziana; paralama; coisas para cobrir ou bloquear | caixa de câmbio (automóvel)}
    \definition{v.}{bloquear; resistir; manter afastado; afastar | cobrir; bloquear; atrapalhar}
  \end{Phonetics}
  \begin{Phonetics}{挡}{dang4}
    \definition{v.}{organizar}
  \end{Phonetics}
\end{Entry}

\begin{Entry}{挡风玻璃}{9,4,9,14}{⼿,⾵,⽟,⽟}
  \begin{Phonetics}{挡风玻璃}{dang3feng1bo1li5}
    \definition{s.}{parabrisa}
  \end{Phonetics}
\end{Entry}

%%%%%%%%%% 挣 %%%%%%%%%%
\subsection*{挣}\addcontentsline{loh}{figure}{挣}

\begin{Entry}{挣}{9}{⼿}
  \begin{Phonetics}{挣}{zheng1}
    \definition{v.}{tentar; fazer o possível para apoiar ou perseverar | lutar; lutar e apoiar}
  \end{Phonetics}
  \begin{Phonetics}{挣}{zheng4}[][HSK 5]
    \definition{v.}{empurrar e puxar; tentar se livrar; lutar para se libertar; esforçar-se para se libertar das amarras | ganhar; fazer; trabalhar para; trocar trabalho por}
  \end{Phonetics}
\end{Entry}

\begin{Entry}{挣扎}{9,4}{⼿,⼿}
  \begin{Phonetics}{挣扎}{zheng1zha2}
    \definition{v.}{lutar}
  \end{Phonetics}
\end{Entry}

\begin{Entry}{挣钱}{9,10}{⼿,⾦}
  \begin{Phonetics}{挣钱}{zheng4/qian2}[][HSK 5]
    \definition{v.+compl.}{ganhar dinheiro; fazer dinheiro; lucrar; trabalhar para ganhar dinheiro}
  \end{Phonetics}
\end{Entry}

\begin{Entry}{挣得}{9,11}{⼿,⼻}
  \begin{Phonetics}{挣得}{zheng4de2}
    \definition{v.}{ganhar renda ou dinheiro}
  \end{Phonetics}
\end{Entry}

%%%%%%%%%% 挤 %%%%%%%%%%
\subsection*{挤}\addcontentsline{loh}{figure}{挤}

\begin{Entry}{挤}{9}{⼿}
  \begin{Phonetics}{挤}{ji3}[][HSK 5]
    \definition{adj.}{lotado; congestionado; descreve um grande número de pessoas ou coisas e muito pouco espaço}
    \definition{v.}{empacotar; amontoar; aglomerar | sacudir; empurrar contra; empurrar alguém ou algo para longe com seu corpo com toda a força que puder| pressionar; apertar; expulsar por pressão}
  \end{Phonetics}
\end{Entry}

\begin{Entry}{挤压}{9,6}{⼿,⼚}
  \begin{Phonetics}{挤压}{ji3ya1}[][HSK 7-9]
    \definition{v.}{pressionar; espremer; esmagar; aperta | Metalurgia: extrudar}
  \end{Phonetics}
\end{Entry}

%%%%%%%%%% 挥 %%%%%%%%%%
\subsection*{挥}\addcontentsline{loh}{figure}{挥}

\begin{Entry}{挥}{9}{⼿}
  \begin{Phonetics}{挥}{hui1}[][HSK 7-9]
    \definition{v.}{acenar; empunhar; socar | limpar lágrimas, suor, etc. com as mãos | comandar (um exército) | espalhar; dispersar | afastar-se; livrar-se de}
  \end{Phonetics}
\end{Entry}

\begin{Entry}{挥汗如雨}{9,6,6,8}{⼿,⽔,⼥,⾬}
  \begin{Phonetics}{挥汗如雨}{hui1han4ru2yu3}
    \definition{s.}{suor derramado}
    \definition{v.}{pingar com suor}
  \end{Phonetics}
\end{Entry}

%%%%%%%%%% 挪 %%%%%%%%%%
\subsection*{挪}\addcontentsline{loh}{figure}{挪}

\begin{Entry}{挪}{9}{⼿}
  \begin{Phonetics}{挪}{nuo2}[][HSK 7-9]
    \definition{v.}{mover; deslocar; transportar}
  \end{Phonetics}
\end{Entry}

%%%%%%%%%% 挺 %%%%%%%%%%
\subsection*{挺}\addcontentsline{loh}{figure}{挺}

\begin{Entry}{挺}{9}{⼿}
  \begin{Phonetics}{挺}{ting3}[][HSK 2,4]
    \definition{adj.}{rígido; ereto; vertical; reto | notável; destacado; distinto}
    \definition{adv.}{muito; bastante}
    \definition{clas.}{usado para metralhadoras}
    \definition{v.}{sobressair; endireitar-se; protrudir (protuberância ou saliência) | suportar; aguentar; resistir; perseverar}
  \end{Phonetics}
\end{Entry}

\begin{Entry}{挺尸}{9,3}{⼿,⼫}
  \begin{Phonetics}{挺尸}{ting3shi1}
    \definition{v.}{(coloquial) dormir | (literalmente) ficar deitado duro como um cadáver}
  \end{Phonetics}
\end{Entry}

\begin{Entry}{挺立}{9,5}{⼿,⽴}
  \begin{Phonetics}{挺立}{ting3li4}
    \definition{v.}{ficar ereto | ficar de pé}
  \end{Phonetics}
\end{Entry}

\begin{Entry}{挺好}{9,6}{⼿,⼥}
  \begin{Phonetics}{挺好}{ting3hao3}[][HSK 2]
    \definition{adj.}{nada mal; surpreendentemente bom}
  \end{Phonetics}
\end{Entry}

\begin{Entry}{挺过}{9,6}{⼿,⾡}
  \begin{Phonetics}{挺过}{ting3guo4}
    \definition{s.}{sobreviver}
  \end{Phonetics}
\end{Entry}

\begin{Entry}{挺住}{9,7}{⼿,⼈}
  \begin{Phonetics}{挺住}{ting3zhu4}
    \definition{v.}{permanecer firme | manter-se firme (diante da adversidade ou da dor)}
  \end{Phonetics}
\end{Entry}

\begin{Entry}{挺杆}{9,7}{⼿,⽊}
  \begin{Phonetics}{挺杆}{ting3gan3}
    \definition{s.}{tucho (peça de máquina)}
  \end{Phonetics}
\end{Entry}

\begin{Entry}{挺身}{9,7}{⼿,⾝}
  \begin{Phonetics}{挺身}{ting3shen1}
    \definition{v.}{endireitar as costas}
  \end{Phonetics}
\end{Entry}

\begin{Entry}{挺进}{9,7}{⼿,⾡}
  \begin{Phonetics}{挺进}{ting3jin4}
    \definition{s.}{progresso | avanço}
    \definition{v.}{progredir | avançar}
  \end{Phonetics}
\end{Entry}

\begin{Entry}{挺拔}{9,8}{⼿,⼿}
  \begin{Phonetics}{挺拔}{ting3ba2}
    \definition{adj.}{alto e reto}
  \end{Phonetics}
\end{Entry}

\begin{Entry}{挺腰}{9,13}{⼿,⾁}
  \begin{Phonetics}{挺腰}{ting3yao1}
    \definition{v.}{arquear as costas | endireitar as costas}
  \end{Phonetics}
\end{Entry}

%%%%%%%%%% 政 %%%%%%%%%%
\subsection*{政}\addcontentsline{loh}{figure}{政}

\begin{Entry}{政}{9}{⽁}
  \begin{Phonetics}{政}{zheng4}
    \definition*{s.}{Sobrenome: Zheng}
    \definition{s.}{política; assuntos políticos | certos aspectos administrativos do governo | assuntos de uma família ou de uma organização; refere"-se a assuntos familiares ou de grupo}
  \end{Phonetics}
\end{Entry}

\begin{Entry}{政权}{9,6}{⽁,⽊}
  \begin{Phonetics}{政权}{zheng4quan2}[][HSK 6]
    \definition{s.}{poder político ou estatal; regime}
  \end{Phonetics}
\end{Entry}

\begin{Entry}{政纲}{9,7}{⽁,⽷}
  \begin{Phonetics}{政纲}{zheng4gang1}
    \definition{s.}{programa ou plataforma política}
  \end{Phonetics}
\end{Entry}

\begin{Entry}{政府}{9,8}{⽁,⼴}
  \begin{Phonetics}{政府}{zheng4fu3}[][HSK 4]
    \definition{s.}{governo;  órgãos executivos do poder do Estado, ou seja, órgãos administrativos do Estado, como o Conselho de Estado (Governo Popular Central) e os governos populares locais em todos os níveis na China}
  \end{Phonetics}
\end{Entry}

\begin{Entry}{政治}{9,8}{⽁,⽔}
  \begin{Phonetics}{政治}{zheng4zhi4}[][HSK 4]
    \definition{s.}{política; assuntos políticos; questões políticas; as atividades de governos, partidos políticos, grupos sociais e indivíduos em assuntos internos e relações internacionais}
  \end{Phonetics}
\end{Entry}

\begin{Entry}{政治局}{9,8,7}{⽁,⽔,⼫}
  \begin{Phonetics}{政治局}{zheng4zhi4ju2}
    \definition{s.}{o principal comitê de políticas de um partido comunista}
  \end{Phonetics}
\end{Entry}

\begin{Entry}{政党}{9,10}{⽁,⼉}
  \begin{Phonetics}{政党}{zheng4dang3}[][HSK 6]
    \definition[个,些]{s.}{partido político; uma organização política que representa um determinado estágio, classe ou grupo e luta para concretizar seus interesses}
  \end{Phonetics}
\end{Entry}

\begin{Entry}{政策}{9,12}{⽁,⽵}
  \begin{Phonetics}{政策}{zheng4ce4}[][HSK 6]
    \definition[项,条,个]{s.}{política; um código de conduta formulado por um país ou partido político para alcançar sua política em um determinado período histórico}
  \end{Phonetics}
\end{Entry}

%%%%%%%%%% 故 %%%%%%%%%%
\subsection*{故}\addcontentsline{loh}{figure}{故}

\begin{Entry}{故}{9}{⽁}
  \begin{Phonetics}{故}{gu4}[][HSK 7-9]
    \definition*{s.}{Sobrenome: Gu}
    \definition{adj.}{velho; antigo; original}
    \definition{adv.}{propositalmente; intencionalmente; deliberadamente}
    \definition{conj.}{assim; portanto; consequentemente; pelo contrário}
    \definition{s.}{evento; incidente; acontecimento; acidente | causa; razão | amigo; conhecido | o velho; refere"-se a coisas antigas e passadas}
    \definition{v.}{morrer}
  \end{Phonetics}
\end{Entry}

\begin{Entry}{故乡}{9,3}{⽁,⼄}
  \begin{Phonetics}{故乡}{gu4xiang1}[][HSK 3]
    \definition[个]{s.}{cidade natal; terra natal; local de nascimento ou onde viveu por muito tempo}
  \end{Phonetics}
\end{Entry}

\begin{Entry}{故事}{9,8}{⽁,⼅}
  \begin{Phonetics}{故事}{gu4shi5}[][HSK 2]
    \definition[个,段,篇,则]{s.}{história; conto; coisas reais ou fictícias usadas como objeto de narrativa, com coerência, atraentes e capazes de emocionar as pessoas | enredo; trama; enredo que consegue mostrar a personalidade dos personagens e refletir a ideia central da obra literária}
  \end{Phonetics}
\end{Entry}

\begin{Entry}{故宫}{9,9}{⽁,⼧}
  \begin{Phonetics}{故宫}{gu4gong1}
    \definition*{s.}{O Palácio Imperial; O Museu do Palácio (em Pequim); A Cidade Proibida}
  \end{Phonetics}
\end{Entry}

\begin{Entry}{故意}{9,13}{⽁,⼼}
  \begin{Phonetics}{故意}{gu4yi4}[][HSK 2]
    \definition{adv.}{deliberadamente; intencionalmente; não é por descuido, mas sim conscientemente (geralmente coisas que não se devem fazer ou que não são necessárias)}
    \definition{s.}{intenção; um tipo de mentalidade, uma pessoa sabe claramente que seus atos podem causar danos a outras pessoas ou trazer consequências negativas para a sociedade, mas mesmo assim não faz nada para impedir isso}
  \end{Phonetics}
\end{Entry}

\begin{Entry}{故障}{9,13}{⽁,⾩}
  \begin{Phonetics}{故障}{gu4zhang4}[][HSK 6]
    \definition[出]{s.}{problema; falha; parada; mau funcionamento; avaria; situações em que máquinas, instrumentos, etc. não podem funcionar normalmente devido a problemas}
  \end{Phonetics}
\end{Entry}

%%%%%%%%%% 施 %%%%%%%%%%
\subsection*{施}\addcontentsline{loh}{figure}{施}

\begin{Entry}{施}{9}{⽅}
  \begin{Phonetics}{施}{shi1}
    \definition*{s.}{Sobrenome: Shi}
    \definition{v.}{pôr em prática; executar; realizar | outorgar; conceder; distribuir | exercer; impor | usar; aplicar}
  \end{Phonetics}
\end{Entry}

\begin{Entry}{施工}{9,3}{⽅,⼯}
  \begin{Phonetics}{施工}{shi1/gong1}[][HSK 7-9]
    \definition{v.+compl.}{construir; realizar obras (ou grandes reparos); construir casas, pontes, estradas, projetos de conservação de água, etc., de acordo com as especificações e requisitos do projeto}
  \synonymref{动工}{dong4/gong1}
  \synonymref{开工}{kai1/gong1}
  \antonymref{竣工}{jun4gong1}
  \end{Phonetics}
\end{Entry}

\begin{Entry}{施加}{9,5}{⽅,⼒}
  \begin{Phonetics}{施加}{shi1jia1}[][HSK 7-9]
    \definition{v.}{exercer (pressão, influência, etc.); aplicar sobre}
  \antonymref{承受}{cheng2shou4}
  \antonymref{遭受}{zao1shou4}
  \end{Phonetics}
\end{Entry}

\begin{Entry}{施压}{9,6}{⽅,⼚}
  \begin{Phonetics}{施压}{shi1ya1}[][HSK 7-9]
    \definition{v.}{aplicar pressão; pressionar |exercer pressão sobre}
  \end{Phonetics}
\end{Entry}

\begin{Entry}{施行}{9,6}{⽅,⾏}
  \begin{Phonetics}{施行}{shi1xing2}[][HSK 7-9]
    \definition{v.}{aplicar; implementar; implementar de uma determinada maneira ou forma | executar; entrar em vigor; leis, regulamentos e normas entrarão em vigor após a sua promulgação}
  \synonymref{实施}{shi2shi1}
  \synonymref{执行}{zhi2xing2}
  \antonymref{废除}{fei4chu2}
  \end{Phonetics}
\end{Entry}

%%%%%%%%%% 既 %%%%%%%%%%
\subsection*{既}\addcontentsline{loh}{figure}{既}

\begin{Entry}{既}{9}{⽆}
  \begin{Phonetics}{既}{ji4}[][HSK 4]
    \definition*{s.}{Sobrenome: Ji}
    \definition{adv.}{já}
    \definition{conj.}{desde; como; agora que | assim como; e também; ambos\dots e\dots; usado em conjunto com advérbios como 且, 又, 也 para indicar uma combinação de ambas as situações}
  \seealsoref{且}{qie3}
  \seealsoref{也}{ye3}
  \seealsoref{又}{you4}
  \end{Phonetics}
\end{Entry}

\begin{Entry}{既又}{9,2}{⽆,⼜}
  \begin{Phonetics}{既又}{ji4you4}
    \definition{conj.}{desde | como | agora isso | os dois e | assim como}
  \end{Phonetics}
\end{Entry}

\begin{Entry}{既不……又不……}{9,4,2,4}{⽆,⼀,⼜,⼀}
  \begin{Phonetics}{既不……又不……}{ji4bu4 you4bu4}
    \definition{conj.}{nem\dots nem\dots}
  \end{Phonetics}
\end{Entry}

\begin{Entry}{既然}{9,12}{⽆,⽕}
  \begin{Phonetics}{既然}{ji4ran2}[][HSK 4]
    \definition{conj.}{como; desde; agora que; usado na primeira metade de uma frase, muitas vezes repetido na segunda metade pelos advérbios 就, 也, 还 para indicar que a premissa é primeiro declarada e depois inferida}
  \seealsoref{还}{hai2}
  \seealsoref{就}{jiu4}
  \seealsoref{也}{ye3}
  \end{Phonetics}
\end{Entry}

%%%%%%%%%% 星 %%%%%%%%%%
\subsection*{星}\addcontentsline{loh}{figure}{星}

\begin{Entry}{星}{9}{⽇}
  \begin{Phonetics}{星}{xing1}
    \definition*{s.}{Xing, a vigésima quinta das vinte e oito constelações em que a esfera celeste era dividida na antiga astronomia chinesa, consistindo em sete estrelas em Hydra}
    \definition[颗]{s.}{estrela | (astronomia) corpo celeste | partícula | pequenas marcas no braço de uma balança romana indicando jin e suas frações | artista famoso (estrela de cinema, estrela de jogos de bola, etc.) | satélite (artificial) | pequena quantidade}
  \end{Phonetics}
\end{Entry}

\begin{Entry}{星火}{9,4}{⽇,⽕}
  \begin{Phonetics}{星火}{xing1huo3}
    \definition{s.}{trilha de meteoro (usada principalmente em expressões como 急如星火) | faísca}
  \end{Phonetics}
\end{Entry}

\begin{Entry}{星辰}{9,7}{⽇,⾠}
  \begin{Phonetics}{星辰}{xing1chen2}
    \definition{s.}{estrelas}
  \end{Phonetics}
\end{Entry}

\begin{Entry}{星表}{9,8}{⽇,⾐}
  \begin{Phonetics}{星表}{xing1biao3}
    \definition{s.}{catálogo de estrelas}
  \end{Phonetics}
\end{Entry}

\begin{Entry}{星星}{9,9}{⽇,⽇}
  \begin{Phonetics}{星星}{xing1xing5}[][HSK 2]
    \definition[颗,群,片]{s.}{estrela; em astronomia, refere"-se aos corpos celestes luminosos no universo, como as estrelas que brilham no céu noturno | estrela; uma metáfora para alguém ou algo que se destaca em um determinado campo e atrai atenção | objetos em forma de estrela}
  \end{Phonetics}
\end{Entry}

\begin{Entry}{星座}{9,10}{⽇,⼴}
  \begin{Phonetics}{星座}{xing1zuo4}
    \definition[张]{s.}{Astronomia: constelação, cada uma das várias divisões (regiões) do céu noturno}
  \end{Phonetics}
\end{Entry}

\begin{Entry}{星期}{9,12}{⽇,⽉}
  \begin{Phonetics}{星期}{xing1qi1}[][HSK 1]
    \definition[个]{s.}{semana | dias da semana; usado em conjunto com 日, 一, 二, 三, 四, 五, 六, 天, indica um determinado dia da semana | abreviação de domingo}
  \seealsoref{星期二}{xing1qi1er4}
  \seealsoref{星期六}{xing1qi1liu4}
  \seealsoref{星期日}{xing1qi1ri4}
  \seealsoref{星期三}{xing1qi1san1}
  \seealsoref{星期四}{xing1qi1si4}
  \seealsoref{星期天}{xing1qi1tian1}
  \seealsoref{星期五}{xing1qi1wu3}
  \seealsoref{星期一}{xing1qi1yi1}
  \end{Phonetics}
\end{Entry}

\begin{Entry}{星期一}{9,12,1}{⽇,⽉,⼀}
  \begin{Phonetics}{星期一}{xing1qi1yi1}
    \definition{s.}{segunda-feira}
  \end{Phonetics}
\end{Entry}

\begin{Entry}{星期二}{9,12,2}{⽇,⽉,⼆}
  \begin{Phonetics}{星期二}{xing1qi1er4}
    \definition{s.}{terça-feira}
  \end{Phonetics}
\end{Entry}

\begin{Entry}{星期三}{9,12,3}{⽇,⽉,⼀}
  \begin{Phonetics}{星期三}{xing1qi1san1}
    \definition{s.}{quarta-feira}
  \end{Phonetics}
\end{Entry}

\begin{Entry}{星期五}{9,12,4}{⽇,⽉,⼆}
  \begin{Phonetics}{星期五}{xing1qi1wu3}
    \definition{s.}{sexta-feira}
  \end{Phonetics}
\end{Entry}

\begin{Entry}{星期六}{9,12,4}{⽇,⽉,⼋}
  \begin{Phonetics}{星期六}{xing1qi1liu4}
    \definition{s.}{sábado}
  \end{Phonetics}
\end{Entry}

\begin{Entry}{星期天}{9,12,4}{⽇,⽉,⼤}
  \begin{Phonetics}{星期天}{xing1qi1tian1}[][HSK 1]
    \definition{s.}{domingo}
  \seealsoref{星期日}{xing1qi1ri4}
  \end{Phonetics}
\end{Entry}

\begin{Entry}{星期日}{9,12,4}{⽇,⽉,⽇}
  \begin{Phonetics}{星期日}{xing1qi1ri4}[][HSK 1]
    \definition{s.}{domingo}
  \seealsoref{星期天}{xing1qi1tian1}
  \end{Phonetics}
\end{Entry}

\begin{Entry}{星期四}{9,12,5}{⽇,⽉,⼞}
  \begin{Phonetics}{星期四}{xing1qi1si4}
    \definition{s.}{quinta-feira}
  \end{Phonetics}
\end{Entry}

%%%%%%%%%% 春 %%%%%%%%%%
\subsection*{春}\addcontentsline{loh}{figure}{春}

\begin{Entry}{春}{9}{⽇}
  \begin{Phonetics}{春}{chun1}
    \definition*{s.}{Sobrenome: Chun}
    \definition{s.}{primavera | amor; luxúria | vida; vitalidade}
  \end{Phonetics}
\end{Entry}

\begin{Entry}{春天}{9,4}{⽇,⼤}
  \begin{Phonetics}{春天}{chun1tian1}[][HSK 2]
    \definition[个,段,季,番]{s.}{primavera; época da primavera | primavera; renascimento; uma atmosfera cheia de energia e esperança}
  \end{Phonetics}
\end{Entry}

\begin{Entry}{春节}{9,5}{⽇,⾋}
  \begin{Phonetics}{春节}{chun1 jie2}[][HSK 2]
    \definition*[个]{s.}{Festival da Primavera (Ano Novo Chinês); o primeiro dia do primeiro mês do calendário lunar, também se refere aos dias seguintes ao primeiro dia do primeiro mês}
  \end{Phonetics}
\end{Entry}

\begin{Entry}{春季}{9,8}{⽇,⼦}
  \begin{Phonetics}{春季}{chun1ji4}[][HSK 4]
    \definition{s.}{primavera; primeiro trimestre do ano, que na China se refere ao período de três meses entre o início da primavera e o início do verão, e também se refere aos três meses do calendário lunar, a saber, o primeiro, o segundo e o terceiro meses}
  \end{Phonetics}
\end{Entry}

%%%%%%%%%% 昨 %%%%%%%%%%
\subsection*{昨}\addcontentsline{loh}{figure}{昨}

\begin{Entry}{昨}{9}{⽇}
  \begin{Phonetics}{昨}{zuo2}
    \definition{s.}{ontem | o passado}
  \end{Phonetics}
\end{Entry}

\begin{Entry}{昨天}{9,4}{⽇,⼤}
  \begin{Phonetics}{昨天}{zuo2tian1}[][HSK 1]
    \definition{s.}{ontem}
  \end{Phonetics}
\end{Entry}

\begin{Entry}{昨日}{9,4}{⽇,⽇}
  \begin{Phonetics}{昨日}{zuo2ri4}
    \definition{adv.}{ontem}
  \end{Phonetics}
\end{Entry}

\begin{Entry}{昨夜}{9,8}{⽇,⼣}
  \begin{Phonetics}{昨夜}{zuo2ye4}
    \definition{adv.}{noite passada}
  \end{Phonetics}
\end{Entry}

\begin{Entry}{昨晚}{9,11}{⽇,⽇}
  \begin{Phonetics}{昨晚}{zuo2wan3}
    \definition{adv.}{noite passada | ontem à noite}
  \end{Phonetics}
\end{Entry}

%%%%%%%%%% 是 %%%%%%%%%%
\subsection*{是}\addcontentsline{loh}{figure}{是}

\begin{Entry}{是}{9}{⽇}
  \begin{Phonetics}{是}{shi4}[][HSK 1]
    \definition*{s.}{Sobrenome: Shi}
    \definition{adj.}{correto; certo | verdadeiro}
    \definition{adv.}{(expressar afirmação firme) de fato; realmente}
    \definition{pron.}{isso; isto |  todos; qualquer um; usado antes de substantivos, tem o significado de 凡是}
    \definition{s.}{assuntos (importantes); grandes planos}
    \definition{v.}{usado como ``ser'' antes de substantivos ou pronomes para identificar, descrever ou ampliar o sujeito; indica que duas coisas são iguais, ou que a segunda explica a primeira | usado entre duas palavras idênticas; relacionar duas palavras semelhantes |  (usado antes de substantivos) ser exatamente; ser corretamente; usado antes de substantivos, tem o significado de 适合 | elogiar; justificar | expressar afirmação ou concordância (frequentemente usado sozinho) | usado para escolher perguntas, perguntas sim/não ou perguntas retóricas | (usado no início de uma frase) enfatizar uma determinada parte de uma frase | usado em perguntas sim-não}
  \seealsoref{凡是}{fan2shi4}
  \seealsoref{适合}{shi4he2}
  \end{Phonetics}
\end{Entry}

\begin{Entry}{是不是}{9,4,9}{⽇,⼀,⽇}
  \begin{Phonetics}{是不是}{shi4bu5shi4}[][HSK 1]
    \definition{expr.}{sim ou não; é ou não é; se ou não; questões levantadas sobre a confirmação e a negação dos fatos}
  \end{Phonetics}
\end{Entry}

\begin{Entry}{是否}{9,7}{⽇,⼝}
  \begin{Phonetics}{是否}{shi4fou3}[][HSK 4]
    \definition{adv.}{se; se ou não; sim ou não}
  \end{Phonetics}
\end{Entry}

\begin{Entry}{是的}{9,8}{⽇,⽩}
  \begin{Phonetics}{是的}{shi4de5}
    \definition{adv.}{sim | está certo}
  \end{Phonetics}
\end{Entry}

\begin{Entry}{是非}{9,8}{⽇,⾮}
  \begin{Phonetics}{是非}{shi4fei1}[][HSK 7-9]
    \definition{s.}{1. Certo e errado
Correto e incorreto
2. briga; disputa}
  \synonymref{长短}{chang2duan3}
  \synonymref{吵嘴}{chao3/zui3}
  \synonymref{黑白}{hei1bai2}
  \synonymref{利害}{li4hai4}
  \synonymref{辱骂}{ru3ma4}
  \end{Phonetics}
\end{Entry}

%%%%%%%%%% 昼 %%%%%%%%%%
\subsection*{昼}\addcontentsline{loh}{figure}{昼}

\begin{Entry}{昼}{9}{⽇}
  \begin{Phonetics}{昼}{zhou4}
    \definition*{s.}{Sobrenome: Zhou}
    \definition{s.}{diurno; luz do dia; dia | dia; o período do amanhecer ao anoitecer; diurno}
  \antonymref{夜}{ye4}
  \end{Phonetics}
\end{Entry}

%%%%%%%%%% 显 %%%%%%%%%%
\subsection*{显}\addcontentsline{loh}{figure}{显}

\begin{Entry}{显}{9}{⽇}
  \begin{Phonetics}{显}{xian3}[][HSK 5]
    \definition*{s.}{Sobrenome: Xian}
    \definition{adj.}{aparente; óbvio; perceptível | ilustre e influente | evidente; óbvio}
    \definition{v.}{mostrar; exibir; manifestar | aparecer; mostrar; revelar}
  \end{Phonetics}
\end{Entry}

\begin{Entry}{显出}{9,5}{⽇,⼐}
  \begin{Phonetics}{显出}{xian3 chu1}[][HSK 6]
    \definition{v.}{mostrar; revelar | dar provas; expressar; exibir}
  \end{Phonetics}
\end{Entry}

\begin{Entry}{显示}{9,5}{⽇,⽰}
  \begin{Phonetics}{显示}{xian3shi4}[][HSK 3]
    \definition{v.}{mostrar; manifestar-se claramente | exibir; ostentar}
  \end{Phonetics}
\end{Entry}

\begin{Entry}{显得}{9,11}{⽇,⼻}
  \begin{Phonetics}{显得}{xian3de5}[][HSK 3]
    \definition{v.}{parecer; aparecer; manifestar (alguma situação)}
  \end{Phonetics}
\end{Entry}

\begin{Entry}{显著}{9,11}{⽇,⽬}
  \begin{Phonetics}{显著}{xian3zhu4}[][HSK 4]
    \definition{adj.}{notável; significativo; notável; extraordinário; muito óbvio; muito claramente demonstrado; muito facilmente visto ou sentido}
  \end{Phonetics}
\end{Entry}

\begin{Entry}{显然}{9,12}{⽇,⽕}
  \begin{Phonetics}{显然}{xian3ran2}[][HSK 3]
    \definition{adj.}{claro; evidente; óbvio; fatos, verdades e outras coisas que são fáceis de descobrir, perceber ou sentir claramente}
  \end{Phonetics}
\end{Entry}

%%%%%%%%%% 枯 %%%%%%%%%%
\subsection*{枯}\addcontentsline{loh}{figure}{枯}

\begin{Entry}{枯}{9}{⽊}
  \begin{Phonetics}{枯}{ku1}
    \definition{adj.}{murcho | (de um poço, rio, etc.) seco | chato; desinteressante | magro e abatido; emaciado}
    \definition[片]{s.}{borra; resíduo}
  \end{Phonetics}
\end{Entry}

\begin{Entry}{枯木}{9,4}{⽊,⽊}
  \begin{Phonetics}{枯木}{ku1mu4}
    \definition{s.}{árvore morta | madeira morta}
  \end{Phonetics}
\end{Entry}

\begin{Entry}{枯燥}{9,17}{⽊,⽕}
  \begin{Phonetics}{枯燥}{ku1zao4}[][HSK 7-9]
    \definition{adj.}{sem graça; monótono e sem vida; monótono; desinteressante}
  \end{Phonetics}
\end{Entry}

%%%%%%%%%% 架 %%%%%%%%%%
\subsection*{架}\addcontentsline{loh}{figure}{架}

\begin{Entry}{架}{9}{⽊}
  \begin{Phonetics}{架}{jia4}[][HSK 3]
    \definition{clas.}{usado para coisas com pilares ou componentes mecânicos | quadrado (usado para montanhas)}
    \definition{s.}{estrutura; organização do corpo humano ou das coisas | prateleira; estante; suporte; componentes que sustentam objetos ou utensílios para colocar objetos, etc.}
    \definition{v.}{colocar para cima; erigir | brigar; discutir | resistir; repelir; afastar | sequestrar; levar alguém à força}
  \end{Phonetics}
\end{Entry}

\begin{Entry}{架子}{9,3}{⽊,⼦}
  \begin{Phonetics}{架子}{jia4zi5}[][HSK 7-9]
    \definition[个,种,套]{s.}{estrutura; suporte; um objeto feito de madeira, metal ou outros materiais que pode ser usado para armazenar ou pendurar coisas | esboço; estrutura; a organização e estrutura das coisas | ares; arrogância; maneiras altivas; pensar que você é melhor que os outros e fingir ser de uma certa maneira | postura; posição; pose}
  \end{Phonetics}
\end{Entry}

\begin{Entry}{架式}{9,6}{⽊,⼷}
  \begin{Phonetics}{架式}{jia4shi5}
    \variantof{架势}
  \end{Phonetics}
\end{Entry}

\begin{Entry}{架势}{9,8}{⽊,⼒}
  \begin{Phonetics}{架势}{jia4shi5}[][HSK 7-9]
    \definition{s.}{postura; atitude; posição (sobre um assunto, etc.)}
  \end{Phonetics}
\end{Entry}

%%%%%%%%%% 柁 %%%%%%%%%%
\subsection*{柁}\addcontentsline{loh}{figure}{柁}

\begin{Entry}{柁}{9}{⽊}
  \begin{Phonetics}{柁}{tuo2}
    \definition{s.}{leme; leme | viga; uma grande viga horizontal em uma treliça de telhado de madeira}
  \seealsoref{舵}{duo4}
  \end{Phonetics}
\end{Entry}

%%%%%%%%%% 柏 %%%%%%%%%%
\subsection*{柏}\addcontentsline{loh}{figure}{柏}

\begin{Entry}{柏}{9}{⽊}
  \begin{Phonetics}{柏}{bai3}
  \seealsoref{柏树}{bai3shu4}
  \end{Phonetics}
  \begin{Phonetics}{柏}{bo2}
    \definition{s.}{cipreste | usado para transcrever nomes}[柏林,德国城市名。===Berlim, uma cidade alemã.]
  \end{Phonetics}
  \begin{Phonetics}{柏}{bo4}
    \definition{s.}{cedro; cipreste amarelo}
  \end{Phonetics}
\end{Entry}

\begin{Entry}{柏林}{9,8}{⽊,⽊}
  \begin{Phonetics}{柏林}{bo2lin2}
    \definition*{s.}{Berlim, capital da Alemanha}
  \end{Phonetics}
\end{Entry}

\begin{Entry}{柏树}{9,9}{⽊,⽊}
  \begin{Phonetics}{柏树}{bai3shu4}[][HSK 7-9]
    \definition[棵,株]{s.}{cipreste}
  \end{Phonetics}
\end{Entry}

%%%%%%%%%% 某 %%%%%%%%%%
\subsection*{某}\addcontentsline{loh}{figure}{某}

\begin{Entry}{某}{9}{⽊}
  \begin{Phonetics}{某}{mou3}[][HSK 3]
    \definition{pron.}{alguém ou algo indefinido; refere"-se a pessoas ou coisas incertas | referindo"-se a si mesmo; em vez do seu próprio nome | alguns; certos; refere"-se a uma pessoa ou coisa específica cujo nome não se sabe ou não se pode revelar | tal e tal; substituir o nome de outra pessoa (geralmente com um tom rude)}
  \end{Phonetics}
\end{Entry}

\begin{Entry}{某些}{9,8}{⽊,⼆}
  \begin{Phonetics}{某些}{mou3 xie1}
    \definition{pron.}{certos; alguns; uns poucos; refere"-se a pessoas ou coisas que são conhecidas, mas das quais não se fala}
  \end{Phonetics}
\end{Entry}

%%%%%%%%%% 染 %%%%%%%%%%
\subsection*{染}\addcontentsline{loh}{figure}{染}

\begin{Entry}{染}{9}{⽊}
  \begin{Phonetics}{染}{ran3}[][HSK 5]
    \definition*{s.}{Sobrenome: Ran}
    \definition{s.}{soja fermentada e temperada em forma de pasta}
    \definition{v.}{tingir; pintar | pegar (uma doença); cair em (um mau hábito, etc.) | sujar; contaminar | pegar (contrair) (uma doença) | adquirir (um mau hábito, etc.); contaminar}
  \end{Phonetics}
\end{Entry}

%%%%%%%%%% 柔 %%%%%%%%%%
\subsection*{柔}\addcontentsline{loh}{figure}{柔}

\begin{Entry}{柔}{9}{⽊}
  \begin{Phonetics}{柔}{rou2}
    \definition*{s.}{Sobrenome: Rou}
    \definition{adj.}{macio; flexível; maleável | gentil; flexível; brando}
    \definition{v.}{tornar macio; amolecer | apaziguar}
  \end{Phonetics}
\end{Entry}

\begin{Entry}{柔和}{9,8}{⽊,⼝}
  \begin{Phonetics}{柔和}{rou2he2}[][HSK 7-9]
    \definition{adj.}{suave; delicado; ameno; macio}
  \end{Phonetics}
\end{Entry}

\begin{Entry}{柔软}{9,8}{⽊,⾞}
  \begin{Phonetics}{柔软}{rou2ruan3}[][HSK 7-9]
    \definition{adj.}{macio; flexível; maleável}
  \end{Phonetics}
\end{Entry}

%%%%%%%%%% 柠 %%%%%%%%%%
\subsection*{柠}\addcontentsline{loh}{figure}{柠}

\begin{Entry}{柠}{9}{⽊}
  \begin{Phonetics}{柠}{ning2}
    \definition{s.}{limão}
  \end{Phonetics}
\end{Entry}

\begin{Entry}{柠檬}{9,17}{⽊,⽊}
  \begin{Phonetics}{柠檬}{ning2meng2}
    \definition[个,片,只]{s.}{limão}
  \end{Phonetics}
\end{Entry}

%%%%%%%%%% 查 %%%%%%%%%%
\subsection*{查}\addcontentsline{loh}{figure}{查}

\begin{Entry}{查}{9}{⽊}
  \begin{Phonetics}{查}{cha2}[][HSK 2]
    \definition{v.}{examinar; verificar cuidadosamente | examinar; investigar; entender bem a situação | procurar; consultar; revisar (documentos bibliográficos)}
  \end{Phonetics}
  \begin{Phonetics}{查}{zha1}
    \definition*{s.}{Sobrenome: Zha}
    \definition{s.}{espinheiro-chinês}
  \end{Phonetics}
\end{Entry}

\begin{Entry}{查出}{9,5}{⽊,⼐}
  \begin{Phonetics}{查出}{cha2chu1}[][HSK 6]
    \definition{v.}{rastrear; desentocar}
  \end{Phonetics}
\end{Entry}

\begin{Entry}{查处}{9,5}{⽊,⼡}
  \begin{Phonetics}{查处}{cha2chu3}[][HSK 7-9]
    \definition{v.}{investigar e lidar (com um caso criminal)}
  \end{Phonetics}
\end{Entry}

\begin{Entry}{查找}{9,7}{⽊,⼿}
  \begin{Phonetics}{查找}{cha2zhao3}[][HSK 7-9]
    \definition{v.}{procurar; pesquisar; tentar encontrar as informações que você precisa}
  \end{Phonetics}
\end{Entry}

\begin{Entry}{查明}{9,8}{⽊,⽇}
  \begin{Phonetics}{查明}{cha2ming2}[][HSK 7-9]
    \definition{v.}{provar por meio de investigação; descobrir; apurar}
  \end{Phonetics}
\end{Entry}

\begin{Entry}{查询}{9,8}{⽊,⾔}
  \begin{Phonetics}{查询}{cha2xun2}[][HSK 5]
    \definition{v.}{indagar; inquirir; perguntar sobre}
  \end{Phonetics}
\end{Entry}

\begin{Entry}{查看}{9,9}{⽊,⽬}
  \begin{Phonetics}{查看}{cha2kan4}[][HSK 6]
    \definition{v.}{verificar; examinar; checar; investigar; verificar e observar a existência das coisas}
  \end{Phonetics}
\end{Entry}

%%%%%%%%%% 柬 %%%%%%%%%%
\subsection*{柬}\addcontentsline{loh}{figure}{柬}

\begin{Entry}{柬}{9}{⽊}
  \begin{Phonetics}{柬}{jian3}
    \definition*{s.}{Sobrenome: Jian}
    \definition[张,封]{s.}{cartão; nota; carta; um termo geral para cartas, cartões de visita, postagens, etc.}
  \end{Phonetics}
\end{Entry}

\begin{Entry}{柬埔寨}{9,10,14}{⽊,⼟,⼧}
  \begin{Phonetics}{柬埔寨}{jian3pu3zhai4}
    \definition*{s.}{Camboja}
  \end{Phonetics}
\end{Entry}

%%%%%%%%%% 柱 %%%%%%%%%%
\subsection*{柱}\addcontentsline{loh}{figure}{柱}

\begin{Entry}{柱}{9}{⽊}
  \begin{Phonetics}{柱}{zhu4}
    \definition*{s.}{Sobrenome: Zhu}
    \definition[根]{s.}{poste; pilar; coluna | algo em forma de coluna | Matemática: cilindro}
  \end{Phonetics}
\end{Entry}

\begin{Entry}{柱子}{9,3}{⽊,⼦}
  \begin{Phonetics}{柱子}{zhu4zi5}[][HSK 6]
    \definition{s.}{poste; pilar; coluna; estrutura de suporte vertical de um edifício, feita de madeira, pedra, aço, concreto armado, etc.}
  \end{Phonetics}
\end{Entry}

%%%%%%%%%% 柳 %%%%%%%%%%
\subsection*{柳}\addcontentsline{loh}{figure}{柳}

\begin{Entry}{柳}{9}{⽊}
  \begin{Phonetics}{柳}{liu3}
    \definition*{s.}{Liu, a vigésima quarta das vinte e oito constelações, consistindo de oito estrelas em Hydra | Liu, uma das mansões lunares | Sobrenome: Liu}
    \definition[棵]{s.}{salgueiro}
  \end{Phonetics}
\end{Entry}

\begin{Entry}{柳树}{9,9}{⽊,⽊}
  \begin{Phonetics}{柳树}{liu3shu4}[][HSK 7-9]
    \definition[棵,株]{s.}{salgueiro; vime}
  \end{Phonetics}
\end{Entry}

\begin{Entry}{柳橙汁}{9,16,5}{⽊,⽊,⽔}
  \begin{Phonetics}{柳橙汁}{liu3cheng2zhi1}
    \definition[瓶,杯,罐,盒]{s.}{suco de laranja}
  \seealsoref{橙汁}{cheng2zhi1}
  \seealsoref{橘子汁}{ju2zi5zhi1}
  \end{Phonetics}
\end{Entry}

%%%%%%%%%% 柿 %%%%%%%%%%
\subsection*{柿}\addcontentsline{loh}{figure}{柿}

\begin{Entry}{柿}{9}{⽊}
  \begin{Phonetics}{柿}{shi4}
    \definition{s.}{caqui | árvore de caqui}
  \end{Phonetics}
\end{Entry}

\begin{Entry}{柿子}{9,3}{⽊,⼦}
  \begin{Phonetics}{柿子}{shi4zi5}[][HSK 7-9]
    \definition[个]{s.}{caqui | caquizeiro}
  \end{Phonetics}
\end{Entry}

%%%%%%%%%% 标 %%%%%%%%%%
\subsection*{标}\addcontentsline{loh}{figure}{标}

\begin{Entry}{标}{9}{⽊}
  \begin{Phonetics}{标}{biao1}[][HSK 7-9]
    \definition{clas.}{usado para equipes (o numeral é limitado a um, 一, o que é comum no chinês moderno)}
    \definition[个]{s.}{copa da árvore (significado original) | marca; sinal | padrão; cota | sinal externo; sintoma | prêmio; troféu | oferta; licitação comercial pública | a ponta de uma árvore | aparência externa; ramos ou superfícies | partes aéreas das plantas | rótulo; etiqueta; identificação; sinal | regimento na Dinastia Qing; uma das organizações militares no final da Dinastia Qing}
    \definition{v.}{colocar uma marca, etiqueta ou rótulo em; rotular | agrupar; formar equipe | marcar; expressar com palavras ou outras coisas}
  \end{Phonetics}
\end{Entry}

\begin{Entry}{标本}{9,5}{⽊,⽊}
  \begin{Phonetics}{标本}{biao1ben3}[][HSK 7-9]
    \definition{s.}{espécime; amostra | Medicina chinesa: causa raiz e sintomas de uma doença}
  \end{Phonetics}
\end{Entry}

\begin{Entry}{标示}{9,5}{⽊,⽰}
  \begin{Phonetics}{标示}{biao1shi4}[][HSK 7-9]
    \definition{v.}{marcar; indicar | indicar; exibir como texto ou gráfico}
  \end{Phonetics}
\end{Entry}

\begin{Entry}{标志}{9,7}{⽊,⼼}
  \begin{Phonetics}{标志}{biao1zhi4}[][HSK 4]
    \definition[个,种]{s.}{sinal; marca; logotipo; símbolo; emblema; marcações que caracterizam um objeto para facilitar a identificação}
    \definition{v.}{marcar; indicar; simbolizar; identificar}
  \end{Phonetics}
\end{Entry}

\begin{Entry}{标语}{9,9}{⽊,⾔}
  \begin{Phonetics}{标语}{biao1yu3}[][HSK 7-9]
    \definition[幅,张,条,个]{s.}{\emph{slogan}; cartaz; \emph{slogans} curtos de propaganda afixados ou pendurados em locais públicos}
  \end{Phonetics}
\end{Entry}

\begin{Entry}{标准}{9,10}{⽊,⼎}
  \begin{Phonetics}{标准}{biao1zhun3}[][HSK 3]
    \definition{adj.}{padrão (que serve como ou está em conformidade com um padrão); em conformidade com os documentos e princípios regulamentares}
    \definition[个,条,项,种]{s.}{padrão; critério; critérios de avaliação das coisas}
  \end{Phonetics}
\end{Entry}

\begin{Entry}{标致}{9,10}{⽊,⾄}
  \begin{Phonetics}{标致}{biao1zhi4}
    \definition*{s.}{Peugeot, montadora de automóveis}
    \definition{adj.}{bela aparência e postura (principalmente para mulheres)}
  \end{Phonetics}
  \begin{Phonetics}{标致}{biao1zhi5}[][HSK 7-9]
    \definition{adj.}{bonita (mulher)}
  \end{Phonetics}
\end{Entry}

\begin{Entry}{标签}{9,13}{⽊,⽵}
  \begin{Phonetics}{标签}{biao1qian1}[][HSK 7-9]
    \definition[个,张,枚,套]{s.}{rótulo; etiqueta; um pedaço de papel anexado ou amarrado a um item para indicar o nome do produto, finalidade, preço, etc.}
  \end{Phonetics}
\end{Entry}

\begin{Entry}{标榜}{9,14}{⽊,⽊}
  \begin{Phonetics}{标榜}{biao1bang3}[][HSK 7-9]
    \definition{v.}{ostentar; anunciar; desfilar | elogiar; elogiar excessivamente | dar publicidade favorável a; fazer uma exibição de; gabar"-se | impulsionar; elogiar excessivamente; gabar"-se de}
  \end{Phonetics}
\end{Entry}

\begin{Entry}{标题}{9,15}{⽊,⾴}
  \begin{Phonetics}{标题}{biao1ti2}[][HSK 3]
    \definition[个,条,篇]{s.}{título; manchete; cabeçalho; resumo conciso do conteúdo da obra}
  \end{Phonetics}
\end{Entry}

%%%%%%%%%% 栋 %%%%%%%%%%
\subsection*{栋}\addcontentsline{loh}{figure}{栋}

\begin{Entry}{栋}{9}{⽊}
  \begin{Phonetics}{栋}{dong4}[][HSK 7-9]
    \definition*{s.}{Sobrenome: Dong}
    \definition{clas.}{edifício; prédio}
    \definition{s.}{Literário: cumeeira; viga principal}
  \end{Phonetics}
\end{Entry}

\begin{Entry}{栋梁}{9,11}{⽊,⽊}
  \begin{Phonetics}{栋梁}{dong4liang2}[][HSK 7-9]
    \definition{s.}{cumeeira e vigas; esteio (de organização); viga de cumeeira; cumeeira; placa de cumeeira; tábua de cumeeira | pessoa capaz de suportar grandes responsabilidades | pilar (do estado)}
  \end{Phonetics}
\end{Entry}

%%%%%%%%%% 栏 %%%%%%%%%%
\subsection*{栏}\addcontentsline{loh}{figure}{栏}

\begin{Entry}{栏}{9}{⽊}
  \begin{Phonetics}{栏}{lan2}[][HSK 7-9]
    \definition{s.}{cerca; corrimão; balaustrada | curral; galpão; celeiro; chiqueiro | coluna (de uma página ou tabela, ou de um jornal) | quadro (de avisos); prancha; tabuleiro | Esporte: obstáculo}
  \end{Phonetics}
\end{Entry}

\begin{Entry}{栏目}{9,5}{⽊,⽬}
  \begin{Phonetics}{栏目}{lan2mu4}[][HSK 6]
    \definition[个,档]{s.}{coluna; programa; seções nomeadas de jornais, revistas, etc. divididas de acordo com a natureza de seu conteúdo}
  \end{Phonetics}
\end{Entry}

\begin{Entry}{栏杆}{9,7}{⽊,⽊}
  \begin{Phonetics}{栏杆}{lan2gan1}[][HSK 7-9]
    \definition[道,排]{s.}{corrimão; balaustrada; instalações que servem como barreiras e proteção ao longo das laterais de estradas, pontes, varandas, arquibancadas, etc.}
  \end{Phonetics}
\end{Entry}

%%%%%%%%%% 树 %%%%%%%%%%
\subsection*{树}\addcontentsline{loh}{figure}{树}

\begin{Entry}{树}{9}{⽊}
  \begin{Phonetics}{树}{shu4}[][HSK 1]
    \definition*{s.}{Sobrenome: Shu}
    \definition[棵,株]{s.}{árvore; nome comum das plantas lenhosas}
    \definition{v.}{plantar; cultivar | configurar; manter; estabelecer}
  \end{Phonetics}
\end{Entry}

\begin{Entry}{树木}{9,4}{⽊,⽊}
  \begin{Phonetics}{树木}{shu4mu4}[][HSK 7-9]
    \definition[棵,株]{s.}{árvore}
  \synonymref{树林}{shu4lin2}
  \synonymref{植物}{zhi2wu4}
  \end{Phonetics}
\end{Entry}

\begin{Entry}{树叶}{9,5}{⽊,⼝}
  \begin{Phonetics}{树叶}{shu4ye4}[][HSK 4]
    \definition[片,枚,堆]{s.}{folha; folhagem}
  \end{Phonetics}
\end{Entry}

\begin{Entry}{树立}{9,5}{⽊,⽴}
  \begin{Phonetics}{树立}{shu4li4}[][HSK 7-9]
    \definition{v.}{estabelecer; configurar; possibilitar a formação e o estabelecimento de algo bom e com impacto positivo, um conceito abstrato}
  \synonymref{成立}{cheng2li4}
  \synonymref{创办}{chuang4ban4}
  \synonymref{创建}{chuang4jian4}
  \synonymref{创立}{chuang4li4}
  \synonymref{建立}{jian4li4}
  \synonymref{建设}{jian4she4}
  \synonymref{确立}{que4li4}
  \synonymref{设立}{she4li4}
  \synonymref{设置}{she4zhi4}
  \antonymref{倒闭}{dao3bi4}
  \end{Phonetics}
\end{Entry}

\begin{Entry}{树阴儿}{9,6,2}{⽊,⾩,⼉}
  \begin{Phonetics}{树阴儿}{shu4yin1r5}
    \definition{s.}{sombra da árvore}
  \end{Phonetics}
\end{Entry}

\begin{Entry}{树林}{9,8}{⽊,⽊}
  \begin{Phonetics}{树林}{shu4lin2}[][HSK 4]
    \definition[片,座]{s.}{bosque; muitas árvores que crescem em fragmentos, menores que as florestas}
  \end{Phonetics}
\end{Entry}

\begin{Entry}{树枝}{9,8}{⽊,⽊}
  \begin{Phonetics}{树枝}{shu4zhi1}[][HSK 7-9]
    \definition[根]{s.}{galho; ramo}
  \synonymref{树梢}{shu4shao1}
  \end{Phonetics}
\end{Entry}

\begin{Entry}{树荫}{9,9}{⽊,⾋}
  \begin{Phonetics}{树荫}{shu4yin4}[][HSK 7-9]
    \definition[片]{s.}{sombra (de uma árvore)}
  \seealsoref{树阴儿}{shu4yin1r5}
  \seealsoref{树荫儿}{shu4yin1r5}
  \end{Phonetics}
\end{Entry}

\begin{Entry}{树荫儿}{9,9,2}{⽊,⾋,⼉}
  \begin{Phonetics}{树荫儿}{shu4yin1r5}
    \definition{s.}{sombra}
  \end{Phonetics}
\end{Entry}

\begin{Entry}{树莓}{9,10}{⽊,⾋}
  \begin{Phonetics}{树莓}{shu4mei2}
    \definition{s.}{framboesa}
  \end{Phonetics}
\end{Entry}

\begin{Entry}{树梢}{9,11}{⽊,⽊}
  \begin{Phonetics}{树梢}{shu4shao1}[][HSK 7-9]
    \definition[根]{s.}{ponta de uma árvore; copa da árvore}
  \seealsoref{树梢儿}{shu4shao1r5}
  \end{Phonetics}
\end{Entry}

\begin{Entry}{树梢儿}{9,11,2}{⽊,⽊,⼉}
  \begin{Phonetics}{树梢儿}{shu4shao1r5}
    \definition[根]{s.}{ponta de uma árvore; copa da árvore}
  \end{Phonetics}
\end{Entry}

%%%%%%%%%% 歪 %%%%%%%%%%
\subsection*{歪}\addcontentsline{loh}{figure}{歪}

\begin{Entry}{歪}{9}{⽌}
  \begin{Phonetics}{歪}{wai1}
    \definition{adj.}{torto | tortuoso | nocivo}
  \end{Phonetics}
\end{Entry}

\begin{Entry}{歪果仁}{9,8,4}{⽌,⽊,⼈}
  \begin{Phonetics}{歪果仁}{wai1 guo3 ren2}
    \definition{s.}{gíria na \emph{Internet} para estrangeiro (外国人)}
  \seealsoref{外国人}{wai4 guo2 ren2}
  \end{Phonetics}
\end{Entry}

%%%%%%%%%% 残 %%%%%%%%%%
\subsection*{残}\addcontentsline{loh}{figure}{残}

\begin{Entry}{残}{9}{⽍}
  \begin{Phonetics}{残}{can2}[][HSK 7-9]
    \definition{adj.}{incompleto; fragmentário; deficiente | remanescente; restante | cruel; feroz | opressivo; selvagem; bárbaro}
    \definition{v.}{ferir; danificar | estragar; prejudicar; destruir}
  \end{Phonetics}
\end{Entry}

\begin{Entry}{残忍}{9,7}{⽍,⼼}
  \begin{Phonetics}{残忍}{can2ren3}[][HSK 7-9]
    \definition{adj.}{cruel; implacável; impiedoso}
  \end{Phonetics}
\end{Entry}

\begin{Entry}{残留}{9,10}{⽍,⽥}
  \begin{Phonetics}{残留}{can2liu2}[][HSK 7-9]
    \definition{adj.}{residual; restante}
    \definition{s.}{vestígio; resto}
    \definition{v.}{sobrar}
  \end{Phonetics}
\end{Entry}

\begin{Entry}{残疾}{9,10}{⽍,⽧}
  \begin{Phonetics}{残疾}{can2ji5}[][HSK 6]
    \definition{s.}{deformidade; deficiência; deficiência física; defeitos de membros, órgãos ou funções fisiológicas}
  \end{Phonetics}
\end{Entry}

\begin{Entry}{残疾人}{9,10,2}{⽍,⽧,⼈}
  \begin{Phonetics}{残疾人}{can2ji2ren2}[][HSK 6]
    \definition[位,名]{s.}{pessoa com deficiência (ou incapacitada); o incapacitado (ou deficiente); pessoas com deficiência visual, auditiva, de linguagem, intelectual, física ou mental são os principais alvos da medicina de reabilitação}
  \end{Phonetics}
\end{Entry}

\begin{Entry}{残缺}{9,10}{⽍,⽸}
  \begin{Phonetics}{残缺}{can2que1}[][HSK 7-9]
    \definition{adj.}{incompleto; fragmentário; com partes faltando}
  \end{Phonetics}
\end{Entry}

\begin{Entry}{残酷}{9,14}{⽍,⾣}
  \begin{Phonetics}{残酷}{can2ku4}[][HSK 6]
    \definition{adj.}{cruel; brutal; implacável}
  \end{Phonetics}
\end{Entry}

%%%%%%%%%% 段 %%%%%%%%%%
\subsection*{段}\addcontentsline{loh}{figure}{段}

\begin{Entry}{段}{9}{⽎}
  \begin{Phonetics}{段}{duan4}[][HSK 2]
    \definition*{s.}{Sobrenome: Duan}
    \definition{clas.}{parte; seção; segmento; usado para dividir objetos em várias partes | passagem; parágrafo; parte de algo que tem características de continuidade | seção; período; usado para uma certa distância no tempo ou no espaço}
    \definition{s.}{nível; dan (no judô, weiqi, etc.) | seção (como nível administrativo em uma mina ou fábrica) | parte; etapa; estágio}
    \definition{v.}{cortar; separar}
  \end{Phonetics}
\end{Entry}

\begin{Entry}{段落}{9,12}{⽎,⾋}
  \begin{Phonetics}{段落}{duan4luo4}[][HSK 7-9]
    \definition[个]{s.}{seção; parágrafo; (artigo, evento) dividido em seções de acordo com o conteúdo | fase; estágio; o estágio relativamente independente de trabalho, coisas, etc.}
  \end{Phonetics}
\end{Entry}

%%%%%%%%%% 毒 %%%%%%%%%%
\subsection*{毒}\addcontentsline{loh}{figure}{毒}

\begin{Entry}{毒}{9}{⽏}
  \begin{Phonetics}{毒}{du2}[][HSK 5]
    \definition*{s.}{Sobrenome: Du}
    \definition{adj.}{veneno; toxina; propriedade ou substância prejudicial aos organismos vivos | droga; narcóticos | vírus; vírus de computador | influência venenosa}
    \definition{adj.}{venenoso; tóxico; envenenado | malicioso; cruel; feroz}
    \definition{v.}{matar com veneno; envenenar | envenenar (a mente de alguém)}
  \end{Phonetics}
\end{Entry}

\begin{Entry}{毒杀}{9,6}{⽏,⽊}
  \begin{Phonetics}{毒杀}{du2sha1}
    \definition{v.}{matar por envenenamento}
  \end{Phonetics}
\end{Entry}

\begin{Entry}{毒物}{9,8}{⽏,⽜}
  \begin{Phonetics}{毒物}{du2wu4}
    \definition{s.}{substância venenosa | toxina}
  \end{Phonetics}
\end{Entry}

\begin{Entry}{毒品}{9,9}{⽏,⼝}
  \begin{Phonetics}{毒品}{du2pin3}[][HSK 6]
    \definition[种,点]{s.}{drogas; veneno; narcóticos; refere"-se ao ópio, morfina, heroína, etc. usados como vício}
  \end{Phonetics}
\end{Entry}

\begin{Entry}{毒害}{9,10}{⽏,⼧}
  \begin{Phonetics}{毒害}{du2hai4}
    \definition{s.}{envenenamento}
    \definition{v.}{envenenar (prejudicar com uma substância tóxica) | envenenar (as mentes das pessoas)}
  \end{Phonetics}
\end{Entry}

\begin{Entry}{毒蛇}{9,11}{⽏,⾍}
  \begin{Phonetics}{毒蛇}{du2she2}
    \definition{s.}{víbora | cobra venenosa}
  \end{Phonetics}
\end{Entry}

%%%%%%%%%% 泉 %%%%%%%%%%
\subsection*{泉}\addcontentsline{loh}{figure}{泉}

\begin{Entry}{泉}{9}{⽔}
  \begin{Phonetics}{泉}{quan2}[][HSK 5]
    \definition*{s.}{Sobrenome: Quan}
    \definition[股,眼,汪]{s.}{fonte (de água mineral) | a nascente de um rio | termo antigo para moeda}
  \end{Phonetics}
\end{Entry}

%%%%%%%%%% 洁 %%%%%%%%%%
\subsection*{洁}\addcontentsline{loh}{figure}{洁}

\begin{Entry}{洁}{9}{⽔}
  \begin{Phonetics}{洁}{jie2}
    \definition{adj.}{limpo; arrumado | honesto; íntegro}
    \definition{v.}{limpar; purificar; tornar limpo | tornar inocente}
  \end{Phonetics}
\end{Entry}

\begin{Entry}{洁净}{9,8}{⽔,⼎}
  \begin{Phonetics}{洁净}{jie2jing4}[][HSK 7-9]
    \definition{adj.}{limpo; impecável}
  \end{Phonetics}
\end{Entry}

%%%%%%%%%% 洋 %%%%%%%%%%
\subsection*{洋}\addcontentsline{loh}{figure}{洋}

\begin{Entry}{洋}{9}{⽔}
  \begin{Phonetics}{洋}{yang2}[][HSK 6]
    \definition*{s.}{Sobrenome: Yang}
    \definition{adj.}{vasto; rico; transbordante | estrangeiro (especialmente ocidental) | moderno}
    \definition[个,片]{s.}{oceano | moeda de prata}
  \seealsoref{土}{tu3}
  \antonymref{土}{tu3}
  \end{Phonetics}
\end{Entry}

\begin{Entry}{洋葱}{9,12}{⽔,⾋}
  \begin{Phonetics}{洋葱}{yang2cong1}
    \definition{s.}{cebola}
  \end{Phonetics}
\end{Entry}

%%%%%%%%%% 洒 %%%%%%%%%%
\subsection*{洒}\addcontentsline{loh}{figure}{洒}

\begin{Entry}{洒}{9}{⽔}
  \begin{Phonetics}{洒}{sa3}[][HSK 5]
    \definition{adj.}{natural e sem restrições; confortável (sem restrições)}
    \definition{v.}{derramar; espalhar; borrifar; salpicar; fazer com que (água ou outra coisa) caia de forma dispersa | derramar; cair de forma dispersa}
  \end{Phonetics}
\end{Entry}

\begin{Entry}{洒水}{9,4}{⽔,⽔}
  \begin{Phonetics}{洒水}{sa3shui3}
    \definition{v.}{borrifar}
  \end{Phonetics}
\end{Entry}

%%%%%%%%%% 洗 %%%%%%%%%%
\subsection*{洗}\addcontentsline{loh}{figure}{洗}

\begin{Entry}{洗}{9}{⽔}
  \begin{Phonetics}{洗}{xi3}[][HSK 1]
    \definition[个]{s.}{pequeno recipiente contendo água para enxaguar os pincéis de escrever | batismo}
    \definition{v.}{lavar; tomar banho; remover a sujeira do objeto com água ou outro solvente | batizar | eliminar; corrigir; reparar | saquear; matar e pilhar; matar ou roubar tudo, como se tivesse sido lavado | revelar filmes; imprimir fotos | apagar; limpar (uma gravação, etc.) | embaralhar (cartas, etc.)}
  \end{Phonetics}
\end{Entry}

\begin{Entry}{洗手}{9,4}{⽔,⼿}
  \begin{Phonetics}{洗手}{xi3shou3}
    \definition{v.}{ir ao banheiro | lavar as mãos}
  \end{Phonetics}
\end{Entry}

\begin{Entry}{洗手不干}{9,4,4,3}{⽔,⼿,⼀,⼲}
  \begin{Phonetics}{洗手不干}{xi3shou3bu2gan4}
    \definition{v.}{parar totalmente de fazer algo}
  \end{Phonetics}
\end{Entry}

\begin{Entry}{洗手池}{9,4,6}{⽔,⼿,⽔}
  \begin{Phonetics}{洗手池}{xi3shou3chi2}
    \definition{s.}{pia de banheiro | lavatório}
  \seealsoref{洗手盆}{xi3shou3pen2}
  \end{Phonetics}
\end{Entry}

\begin{Entry}{洗手间}{9,4,7}{⽔,⼿,⾨}
  \begin{Phonetics}{洗手间}{xi3shou3jian1}[][HSK 1]
    \definition[个]{s.}{banheiro; lavatório; lavabo}
  \end{Phonetics}
\end{Entry}

\begin{Entry}{洗手乳}{9,4,8}{⽔,⼿,⼄}
  \begin{Phonetics}{洗手乳}{xi3shou3ru3}
    \definition{s.}{sabonete líquido para lavar as mãos}
  \seealsoref{洗手液}{xi3shou3ye4}
  \end{Phonetics}
\end{Entry}

\begin{Entry}{洗手盆}{9,4,9}{⽔,⼿,⽫}
  \begin{Phonetics}{洗手盆}{xi3shou3pen2}
    \definition{s.}{pia de banheiro | lavatório}
  \seealsoref{洗手池}{xi3shou3chi2}
  \end{Phonetics}
\end{Entry}

\begin{Entry}{洗手液}{9,4,11}{⽔,⼿,⽔}
  \begin{Phonetics}{洗手液}{xi3shou3ye4}
    \definition{s.}{sabonete líquido para lavar as mãos}
  \seealsoref{洗手乳}{xi3shou3ru3}
  \end{Phonetics}
\end{Entry}

\begin{Entry}{洗礼}{9,5}{⽔,⽰}
  \begin{Phonetics}{洗礼}{xi3li3}
    \definition{s.}{batismo}
    \definition{v.}{batizar}
  \end{Phonetics}
\end{Entry}

\begin{Entry}{洗衣机}{9,6,6}{⽔,⾐,⽊}
  \begin{Phonetics}{洗衣机}{xi3yi1ji1}[][HSK 2]
    \definition[台]{s.}{máquina de lavar roupa; eletrodomésticos para lavagem automática ou semiautomática de roupas}
  \end{Phonetics}
\end{Entry}

\begin{Entry}{洗衣粉}{9,6,10}{⽔,⾐,⽶}
  \begin{Phonetics}{洗衣粉}{xi3yi1fen3}[][HSK 6]
    \definition[袋,包,勺]{s.}{sabão em pó; detergente para roupa (em pó); detergente em pó sintetizado quimicamente, específico para uso em lavanderia}
  \end{Phonetics}
\end{Entry}

\begin{Entry}{洗劫}{9,7}{⽔,⼒}
  \begin{Phonetics}{洗劫}{xi3jie2}
    \definition{v.}{saquear; pilhar; roubar}
  \end{Phonetics}
\end{Entry}

\begin{Entry}{洗净}{9,8}{⽔,⼎}
  \begin{Phonetics}{洗净}{xi3jing4}
    \definition{v.}{lavar (limpeza)}
  \end{Phonetics}
\end{Entry}

\begin{Entry}{洗胃}{9,9}{⽔,⾁}
  \begin{Phonetics}{洗胃}{xi3wei4}
    \definition{s.}{(medicina) lavagem gástrica}
    \definition{v.}{ter o estômago lavado}
  \end{Phonetics}
\end{Entry}

\begin{Entry}{洗涤}{9,10}{⽔,⽔}
  \begin{Phonetics}{洗涤}{xi3di2}
    \definition{s.}{enxágue | lava}
    \definition{v.}{enxaguar | lavar}
  \end{Phonetics}
\end{Entry}

\begin{Entry}{洗涤间}{9,10,7}{⽔,⽔,⾨}
  \begin{Phonetics}{洗涤间}{xi3di2jian1}
    \definition{s.}{lavanderia}
  \end{Phonetics}
\end{Entry}

\begin{Entry}{洗脱}{9,11}{⽔,⾁}
  \begin{Phonetics}{洗脱}{xi3tuo1}
    \definition{v.}{limpar | purgar | lavar}
  \end{Phonetics}
\end{Entry}

\begin{Entry}{洗碗}{9,13}{⽔,⽯}
  \begin{Phonetics}{洗碗}{xi3wan3}
    \definition{v.}{lavar pratos}
  \end{Phonetics}
\end{Entry}

\begin{Entry}{洗澡}{9,16}{⽔,⽔}
  \begin{Phonetics}{洗澡}{xi3/zao3}[][HSK 2]
    \definition{v.+compl.}{tomar banho; tomar banho de chuveiro; lavar-se}
  \end{Phonetics}
\end{Entry}

\begin{Entry}{洗澡间}{9,16,7}{⽔,⽔,⾨}
  \begin{Phonetics}{洗澡间}{xi3zao3jian1}
    \definition[间]{s.}{banheiro}
  \end{Phonetics}
\end{Entry}

%%%%%%%%%% 洞 %%%%%%%%%%
\subsection*{洞}\addcontentsline{loh}{figure}{洞}

\begin{Entry}{洞}{9}{⽔}
  \begin{Phonetics}{洞}{dong4}[][HSK 5]
    \definition{adj.}{profundo; minucioso; claro; completo; abrangente}
    \definition{s.}{buraco; cavidade; orifício; furo; parte penetrante ou profundamente recuada de um objeto; uma caverna}
  \end{Phonetics}
\end{Entry}

\begin{Entry}{洞穴}{9,5}{⽔,⽳}
  \begin{Phonetics}{洞穴}{dong4xue2}
    \definition{s.}{caverna}
  \end{Phonetics}
\end{Entry}

%%%%%%%%%% 津 %%%%%%%%%%
\subsection*{津}\addcontentsline{loh}{figure}{津}

\begin{Entry}{津}{9}{⽔}
  \begin{Phonetics}{津}{jin1}
    \definition*{s.}{abreviação de Tianjin, 天津}
    \definition{adj.}{úmido; molhado; hidratado}
    \definition{s.}{suor | travessia de balsa; vau; balsa | metáfora para cargos importantes | saliva}
  \seealsoref{天津}{tian1jin1}
  \end{Phonetics}
\end{Entry}

\begin{Entry}{津津有味}{9,9,6,8}{⽔,⽔,⽉,⼝}
  \begin{Phonetics}{津津有味}{jin1jin1-you3wei4}[][HSK 7-9]
    \definition{expr.}{com grande prazer; o sabor é delicioso; com entusiasmo; com grande satisfação}
  \end{Phonetics}
\end{Entry}

\begin{Entry}{津贴}{9,9}{⽔,⾙}
  \begin{Phonetics}{津贴}{jin1tie1}[][HSK 7-9]
    \definition{s.}{subsídio; auxílio financeiro; abono; pensão; além dos salários, os subsídios também se referem aos auxílios de custo de vida para funcionários do sistema de fornecimento}
    \definition{v.}{subsidiar; conceder auxílio financeiro; subsidiar pessoas com dinheiro ou bens}
  \end{Phonetics}
\end{Entry}

%%%%%%%%%% 洪 %%%%%%%%%%
\subsection*{洪}\addcontentsline{loh}{figure}{洪}

\begin{Entry}{洪}{9}{⽔}
  \begin{Phonetics}{洪}{hong2}
    \definition*{s.}{Sobrenome: Hong}
    \definition{adj.}{alto; vasto | grande; grandioso}
    \definition[场]{s.}{enchente; inundação}
  \end{Phonetics}
\end{Entry}

\begin{Entry}{洪水}{9,4}{⽔,⽔}
  \begin{Phonetics}{洪水}{hong2shui3}[][HSK 6]
    \definition[场]{s.}{dilúvio; inundação; enchente; um aumento repentino em um rio causado por chuva forte ou derretimento de neve}
  \end{Phonetics}
\end{Entry}

\begin{Entry}{洪亮}{9,9}{⽔,⼇}
  \begin{Phonetics}{洪亮}{hong2liang4}[][HSK 7-9]
    \definition{adj.}{alto e claro; ressonante; sonoro}
  \end{Phonetics}
\end{Entry}

%%%%%%%%%% 洲 %%%%%%%%%%
\subsection*{洲}\addcontentsline{loh}{figure}{洲}

\begin{Entry}{洲}{9}{⽔}
  \begin{Phonetics}{洲}{zhou1}
    \definition{s.}{continente | ilha em um rio}
  \end{Phonetics}
\end{Entry}

%%%%%%%%%% 活 %%%%%%%%%%
\subsection*{活}\addcontentsline{loh}{figure}{活}

\begin{Entry}{活}{9}{⽔}
  \begin{Phonetics}{活}{huo2}[][HSK 3]
    \definition{adj.}{vivo; vivendo; indica que (alguma ação) foi realizada enquanto a pessoa ainda estava viva | vívido; animado; ativo | móvel; em movimento; ativo}
    \definition{adv.}{exatamente; simplesmente; expressa um grau elevado, equivalente a 真正 ou 简直}
    \definition{s.}{emprego; meios de subsistência; trabalho (geralmente refere"-se a trabalho físico) | produto; algo fabricado}
    \definition{v.}{viver; ter vida; sobreviver | salvar (a vida de uma pessoa); fazer sobreviver; manter a vida}
  \seealsoref{简直}{jian3zhi2}
  \seealsoref{真正}{zhen1zheng4}
  \antonymref{死}{si3}
  \end{Phonetics}
\end{Entry}

\begin{Entry}{活儿}{9,2}{⽔,⼉}
  \begin{Phonetics}{活儿}{huo2r5}[][HSK 7-9]
    \definition[点]{s.}{emprego; trabalho; geralmente trabalho físico | produto; produtos acabados: artesanato, tecnologia}
  \end{Phonetics}
\end{Entry}

\begin{Entry}{活力}{9,2}{⽔,⼒}
  \begin{Phonetics}{活力}{huo2li4}[][HSK 5]
    \definition{s.}{vigor; vitalidade; energia; muito forte, geralmente usado para descrever pessoas, cidades, empresas, economias, etc.}
  \end{Phonetics}
\end{Entry}

\begin{Entry}{活动}{9,6}{⽔,⼒}
  \begin{Phonetics}{活动}{huo2dong5}[][HSK 2]
    \definition{adj.}{móvel; flexível para alterações ou mudanças}
    \definition[些,个,种,类,次]{s.}{atividade; ação tomada com o objetivo de alcançar um determinado objetivo}
    \definition{v.}{fazer exercício; movimentar-se | usar influência pessoal; usar meios irregulares | mover-se}
  \end{Phonetics}
\end{Entry}

\begin{Entry}{活泼}{9,8}{⽔,⽔}
  \begin{Phonetics}{活泼}{huo2po5}[][HSK 5]
    \definition{adj.}{vívido; ativo; animado; brilhante; vivaz; cheio de vida | Química: reativo; significa que a substância é ativa e reage facilmente com outras substâncias}
  \end{Phonetics}
\end{Entry}

\begin{Entry}{活该}{9,8}{⽔,⾔}
  \begin{Phonetics}{活该}{huo2gai1}[][HSK 7-9]
    \definition{v.aux.}{merecer (uma consequência negativa) | deveria (incluindo o significado de destino); ser decretado pelo destino}
  \end{Phonetics}
\end{Entry}

\begin{Entry}{活着}{9,11}{⽔,⽬}
  \begin{Phonetics}{活着}{huo2zhe5}
    \definition{adj.}{vivo}
  \end{Phonetics}
\end{Entry}

\begin{Entry}{活跃}{9,11}{⽔,⾜}
  \begin{Phonetics}{活跃}{huo2yue4}[][HSK 6]
    \definition{adj.}{ativo; dinâmico; pensamentos, ações ou atividades positivas; ocorrências frequentes | rápido; ativo; dinâmico}
    \definition{v.}{animar; tornar ativo | ser ativo}
  \end{Phonetics}
\end{Entry}

\begin{Entry}{活期}{9,12}{⽔,⽉}
  \begin{Phonetics}{活期}{huo2qi1}[][HSK 7-9]
    \definition{adj.}{atual; corrente; presente}
  \end{Phonetics}
\end{Entry}

\begin{Entry}{活路}{9,13}{⽔,⾜}
  \begin{Phonetics}{活路}{huo2lu4}
    \definition{s.}{maneira de sobreviver | meio de subsistência}
  \end{Phonetics}
  \begin{Phonetics}{活路}{huo2lu5}
    \definition{s.}{labor | trabalho físico}
  \end{Phonetics}
\end{Entry}

%%%%%%%%%% 洼 %%%%%%%%%%
\subsection*{洼}\addcontentsline{loh}{figure}{洼}

\begin{Entry}{洼}{9}{⽔}
  \begin{Phonetics}{洼}{wa1}
    \definition{adj.}{oco; baixo}
    \definition{s.}{área baixa; depressão; oco}
  \end{Phonetics}
\end{Entry}

%%%%%%%%%% 洽 %%%%%%%%%%
\subsection*{洽}\addcontentsline{loh}{figure}{洽}

\begin{Entry}{洽}{9}{⽔}
  \begin{Phonetics}{洽}{qia4}
    \definition{adj.}{em harmonia; em acordo | extenso; amplo}
    \definition{v.}{consultar; combinar com}
  \end{Phonetics}
\end{Entry}

\begin{Entry}{洽谈}{9,10}{⽔,⾔}
  \begin{Phonetics}{洽谈}{qia4tan2}[][HSK 7-9]
    \definition{v.}{negociar; negociação e consulta geralmente se referem às conversas ou discussões realizadas em atividades comerciais relacionadas a negócios, transações de mercadorias e compra e venda}
  \end{Phonetics}
\end{Entry}

%%%%%%%%%% 派 %%%%%%%%%%
\subsection*{派}\addcontentsline{loh}{figure}{派}

\begin{Entry}{派}{9}{⽔}
  \begin{Phonetics}{派}{pai4}[][HSK 3]
    \definition{adj.}{elegante; bonito; imponente}
    \definition{clas.}{usado para grupos, escolas de pensamento ou arte, etc. | usado para um discursos, situações, cenas, etc.}
    \definition[个,块,种]{s.}{panelinha; facção; pessoas com ideias, visões e estilos semelhantes | torta; um alimento recheado comumente consumido pelos ocidentais, geralmente doce | maneira e ar; estilo ou comportamento | afluente; braço de rio}
    \definition{v.}{enviar; despachar; arranjar ou ordenar que uma pessoa faça algo; providenciar transporte | alocar; repartir; distribuir}
  \end{Phonetics}
\end{Entry}

\begin{Entry}{派出}{9,5}{⽔,⼐}
  \begin{Phonetics}{派出}{pai4chu1}[][HSK 6]
    \definition{v.}{despachar; expedi | enviar}
  \end{Phonetics}
\end{Entry}

\begin{Entry}{派别}{9,7}{⽔,⼑}
  \begin{Phonetics}{派别}{pai4bie2}[][HSK 7-9]
    \definition{s.}{grupo; seita; escola; facção | categorias; panelinha}
  \end{Phonetics}
\end{Entry}

\begin{Entry}{派遣}{9,13}{⽔,⾡}
  \begin{Phonetics}{派遣}{pai4qian3}[][HSK 7-9]
    \definition{v.}{despachar; enviar alguém em missão (governo, organização, etc.)}
  \end{Phonetics}
\end{Entry}

%%%%%%%%%% 浇 %%%%%%%%%%
\subsection*{浇}\addcontentsline{loh}{figure}{浇}

\begin{Entry}{浇}{9}{⽔}
  \begin{Phonetics}{浇}{jiao1}[][HSK 7-9]
    \definition{adj.}{decadente; precipitado e pérfido}
    \definition{v.}{derramar líquido sobre; borrifar água sobre | aguar; regar; irrigar | injetar fluido no molde}
  \end{Phonetics}
\end{Entry}

%%%%%%%%%% 浊 %%%%%%%%%%
\subsection*{浊}\addcontentsline{loh}{figure}{浊}

\begin{Entry}{浊}{9}{⽔}
  \begin{Phonetics}{浊}{zhuo2}
    \definition*{s.}{Sobrenome: Zhuo}
    \definition{adj.}{turvo; lamacento; imundo | profundo e espesso | caótico; confuso; corrompido}
  \antonymref{清}{qing1}
  \end{Phonetics}
\end{Entry}

%%%%%%%%%% 测 %%%%%%%%%%
\subsection*{测}\addcontentsline{loh}{figure}{测}

\begin{Entry}{测}{9}{⽔}
  \begin{Phonetics}{测}{ce4}[][HSK 4]
    \definition{v.}{pesquisar; sondar; medir | conjecturar; advinhar}
  \end{Phonetics}
\end{Entry}

\begin{Entry}{测定}{9,8}{⽔,⼧}
  \begin{Phonetics}{测定}{ce4ding4}[][HSK 6]
    \definition{v.}{verificar por medição (ou levantamento); determinar; medir; avaliar}
  \end{Phonetics}
\end{Entry}

\begin{Entry}{测试}{9,8}{⽔,⾔}
  \begin{Phonetics}{测试}{ce4shi4}[][HSK 4]
    \definition[次,场]{s.}{exame; teste; medição do conhecimento humano, das habilidades ou do funcionamento de máquinas, ferramentas ou instrumentos}
    \definition{v.}{examinar | testar, medição do desempenho e da precisão de máquinas, instrumentos, aparelhos, etc.}
  \end{Phonetics}
\end{Entry}

\begin{Entry}{测验}{9,10}{⽔,⾺}
  \begin{Phonetics}{测验}{ce4yan4}[][HSK 7-9]
    \definition[次,个]{s.}{teste; execução de teste; testes com instrumentos ou outros métodos}
    \definition{v.}{testar; verificar o desempenho acadêmico, etc.}
  \end{Phonetics}
\end{Entry}

\begin{Entry}{测量}{9,12}{⽔,⾥}
  \begin{Phonetics}{测量}{ce4liang2}[][HSK 4]
    \definition{v.}{aferir; pesquisar; medir; determinar valores relevantes para espaço, tempo, temperatura, velocidade, função, etc.}
  \end{Phonetics}
\end{Entry}

\begin{Entry}{测算}{9,14}{⽔,⽵}
  \begin{Phonetics}{测算}{ce4suan4}[][HSK 7-9]
    \definition{v.}{medir e calcular | adivinhar e estimar; especular}
  \end{Phonetics}
\end{Entry}

%%%%%%%%%% 浏 %%%%%%%%%%
\subsection*{浏}\addcontentsline{loh}{figure}{浏}

\begin{Entry}{浏}{9}{⽔}
  \begin{Phonetics}{浏}{liu2}
    \definition{adj.}{Literário: (água) clara; límpida | Literário: (som) claro e ressonante | Literário: (vento) veloz; rápido; repentino}
  \end{Phonetics}
\end{Entry}

\begin{Entry}{浏览}{9,9}{⽔,⾒}
  \begin{Phonetics}{浏览}{liu2lan3}[][HSK 7-9]
    \definition{v.}{examinar; navegar; dar uma olhada rápida; folhear; passar os olhos por algo}
  \end{Phonetics}
\end{Entry}

\begin{Entry}{浏览器}{9,9,16}{⽔,⾒,⼝}
  \begin{Phonetics}{浏览器}{liu2lan3qi4}[][HSK 7-9]
    \definition[个]{s.}{Internet: navegador; \emph{browser}; uma ferramenta de \emph{software} que permite aos usuários acessar facilmente informações sobre diversos sites na \emph{Internet}}
  \end{Phonetics}
\end{Entry}

%%%%%%%%%% 浑 %%%%%%%%%%
\subsection*{浑}\addcontentsline{loh}{figure}{浑}

\begin{Entry}{浑}{9}{⽔}
  \begin{Phonetics}{浑}{hun2}
    \definition*{s.}{Sobrenome: Hun}
    \definition{adj.}{lamacento; turvo | tolo; estúpido | simples e natural; sem sofisticação | inteiro; por toda parte}
    \variantof{混}
  \end{Phonetics}
\end{Entry}

\begin{Entry}{浑身}{9,7}{⽔,⾝}
  \begin{Phonetics}{浑身}{hun2shen1}[][HSK 7-9]
    \definition{s.}{por todo o corpo; da cabeça aos pés; corpo inteiro}
  \end{Phonetics}
\end{Entry}

%%%%%%%%%% 浓 %%%%%%%%%%
\subsection*{浓}\addcontentsline{loh}{figure}{浓}

\begin{Entry}{浓}{9}{⽔}
  \begin{Phonetics}{浓}{nong2}[][HSK 4]
    \definition{adj.}{denso; espesso; concentrado; um líquido ou gás que contém mais de um determinado ingrediente | grande; forte; profundo (de grau ou extensão) | profundo; (algumas cores) escuro}
  \end{Phonetics}
\end{Entry}

\begin{Entry}{浓郁}{9,8}{⽔,⾢}
  \begin{Phonetics}{浓郁}{nong2yu4}[][HSK 7-9]
    \definition{adj.}{(perfume, fragrância, aroma, etc.) forte | denso; espesso; exuberante | (interesse, cor, etc.) forte; rico}
  \end{Phonetics}
\end{Entry}

\begin{Entry}{浓厚}{9,9}{⽔,⼚}
  \begin{Phonetics}{浓厚}{nong2hou4}[][HSK 7-9]
    \definition{adj.}{muito forte (de cor, interesse, intenção, atmosfera, etc.); descreve algo como sendo de grande interesse ou significativo em termos de cor, visibilidade, atmosfera, etc. | espesso; denso; descreve fumaça, neblina ou nuvens como abundantes e densas}
  \end{Phonetics}
\end{Entry}

\begin{Entry}{浓重}{9,9}{⽔,⾥}
  \begin{Phonetics}{浓重}{nong2zhong4}[][HSK 7-9]
    \definition{adj.}{denso; espesso; forte; (fumaça, cheiro, cor, etc.) muito denso e pesado}
  \end{Phonetics}
\end{Entry}

\begin{Entry}{浓缩}{9,14}{⽔,⽷}
  \begin{Phonetics}{浓缩}{nong2suo1}[][HSK 7-9]
    \definition{v.}{concentrar; condensar; a concentração de uma solução é aumentada pela evaporação do solvente através de métodos como o aquecimento | enriquecer; geralmente se refere à redução das partes desnecessárias de algo, de modo que o conteúdo das partes necessárias aumente relativamente}
  \end{Phonetics}
\end{Entry}

%%%%%%%%%% 炮 %%%%%%%%%%
\subsection*{炮}\addcontentsline{loh}{figure}{炮}

\begin{Entry}{炮}{9}{⽕}
  \begin{Phonetics}{炮}{bao1}
    \definition{v.}{processar; o método de preparação da medicina chinesa é colocar as ervas cruas em uma panela de ferro em alta temperatura e fritá-las até que fiquem marrons e estourem | secar alimentos pelo calor; refogar}
  \end{Phonetics}
  \begin{Phonetics}{炮}{pao2}
    \definition{v.}{Medicina tradicional chinesa: preparar a medicina chinesa assando-a em uma panela de ferro quente até dourar e estalar}
  \end{Phonetics}
  \begin{Phonetics}{炮}{pao4}[][HSK 6]
    \definition{s.}{arma grande; canhão; peça de artilharia | fogo de artifício | buraco de explosão cheio de dinamite | canhão, uma das peças do xadrez chinês}
  \end{Phonetics}
\end{Entry}

\begin{Entry}{炮车}{9,4}{⽕,⾞}
  \begin{Phonetics}{炮车}{pao4che1}
    \definition{s.}{veículo de artilharia; tanque de guerra}
  \end{Phonetics}
\end{Entry}

%%%%%%%%%% 炸 %%%%%%%%%%
\subsection*{炸}\addcontentsline{loh}{figure}{炸}

\begin{Entry}{炸}{9}{⽕}
  \begin{Phonetics}{炸}{zha2}
    \definition{v.}{explodir; estourar; romper | dinamitar; bombardear; explodir; detonar com explosivos | encolerizar-se; explodir em fúria | correr; fugir em pânico}
  \end{Phonetics}
  \begin{Phonetics}{炸}{zha4}[][HSK 6]
    \definition{v.}{fritar em gordura ou óleo | escaldar (como forma de cozinhar)}
  \end{Phonetics}
\end{Entry}

\begin{Entry}{炸药}{9,9}{⽕,⾋}
  \begin{Phonetics}{炸药}{zha4yao4}[][HSK 6]
    \definition[包,种]{s.}{explosivo; cargas explosivas; dinamite; substâncias que explodem quando aquecidas ou impactadas, produzindo grandes quantidades de energia e gases de alta temperatura, como dinamite e pólvora negra}
  \end{Phonetics}
\end{Entry}

\begin{Entry}{炸弹}{9,11}{⽕,⼸}
  \begin{Phonetics}{炸弹}{zha4dan4}[][HSK 6]
    \definition{s.}{bomba; uma arma com invólucro de ferro e explosivos dentro que explodem quando um fusível é acionado, geralmente lançada de um avião}
  \end{Phonetics}
\end{Entry}

%%%%%%%%%% 点 %%%%%%%%%%
\subsection*{点}\addcontentsline{loh}{figure}{点}

\begin{Entry}{点}{9}{⽕}
  \begin{Phonetics}{点}{dian3}[][HSK 1]
    \definition{clas.}{hora cheia | ponto, uma unidade de medida para tipos; antigamente, a contagem do tempo durante a noite era feita por turnos, sendo cada turno dividido em cinco pontos | quantidade ínfima; um pouco; um pouquinho; alguma coisa; indica uma pequena quantidade | usado para itens}
    \definition{s.}{gota (de líquido); (ponto) pequena gota de líquido | mancha; ponto; salpico; (um pouco) Um pequeno vestígio | (ponto) traço de um caractere chinês, cuja forma é ``、''  | Matemática: ponto; refere"-se a uma figura geométrica que não tem comprimento, largura ou altura, mas apenas uma posição | gongo, instrumento musical de metal | ponto decimal; refere"-se ao ponto decimal, símbolo matemático que representa os números decimais | lugar específico | lanche leve; petisco | lugar; grau; sinalização de um determinado local ou grau | hora marcada; hora regulamentar | aspecto; característica; partes ou aspectos específicos de algo | ritmo; batida}
    \definition{v.}{andar na ponta dos pés | dar uma dica, sugestão | tocar levemente com o dedo, pincel ou vara; tocar muito brevemente; passar rapidamente | acenar; baixar ligeiramente a cabeça e levantar rapidamente | gotejar; fazer cair líquido | semear em buracos; plantar com um plantador | verificar um por um | colocar um ponto; usar caneta e outras ferramentas para adicionar ideias | sugerir; indicar; dar uma dica | decorar; realçar | selecionar; escolher; especificar o que é exigido | acender; queimar; inflamar | (pedido) comer uma pequena quantidade de comida para saciar a fome}
  \end{Phonetics}
\end{Entry}

\begin{Entry}{点子}{9,3}{⽕,⼦}
  \begin{Phonetics}{点子}{dian3zi5}[][HSK 7-9]
    \definition[滴,个]{s.}{gota (de líquido); pequenas gotas | mancha; ponto; pinta | batidas de instrumentos de percussão | ponto-chave; aspecto vital | ideia; indicador; método}
  \end{Phonetics}
\end{Entry}

\begin{Entry}{点心}{9,4}{⽕,⼼}
  \begin{Phonetics}{点心}{dian3xin1}
    \definition[块,份,袋,包,盒]{s.}{lanche; refeição leve}
  \end{Phonetics}
  \begin{Phonetics}{点心}{dian3xin5}[][HSK 7-9]
    \definition[块,份,袋,包,盒]{s.}{doces; docinhos; sobremesa; lanches leves; pequenos alimentos além de arroz e vegetais, como bolos, biscoitos, etc.}
  \end{Phonetics}
\end{Entry}

\begin{Entry}{点火}{9,4}{⽕,⽕}
  \begin{Phonetics}{点火}{dian3/huo3}[][HSK 7-9]
    \definition{s.}{ignição}
    \definition{v.+compl.}{acender; acender o fogo; acender uma luz; disparar; iniciar a combustão; acender as chamas | inflamar | causar problemas}
  \end{Phonetics}
\end{Entry}

\begin{Entry}{点击率}{9,5,11}{⽕,⼐,⽞}
  \begin{Phonetics}{点击率}{dian3ji1lv4}[][HSK 7-9]
    \definition{s.}{Internet: taxa de cliques (CTR, \emph{click-through rate})}
  \end{Phonetics}
\end{Entry}

\begin{Entry}{点头}{9,5}{⽕,⼤}
  \begin{Phonetics}{点头}{dian3tou2}[][HSK 2]
    \definition{v.}{acenar com a cabeça; balançar a cabeça; mover ligeiramente a cabeça para baixo; indicar permissão, aprovação, compreensão ou saudação}
  \end{Phonetics}
\end{Entry}

\begin{Entry}{点名}{9,6}{⽕,⼝}
  \begin{Phonetics}{点名}{dian3 ming2}[][HSK 4]
    \definition{v.}{fazer a lista de chamada; manter o controle da presença de alguém; chamar nomes para controle de presença | mencionar alguém pelo nome}
  \end{Phonetics}
\end{Entry}

\begin{Entry}{点评}{9,7}{⽕,⾔}
  \begin{Phonetics}{点评}{dian3ping2}[][HSK 7-9]
    \definition{s.}{comentário; crítica | um comentário ponto por ponto}
    \definition{v.}{comentar; fazer comentários}
  \end{Phonetics}
\end{Entry}

\begin{Entry}{点缀}{9,11}{⽕,⽷}
  \begin{Phonetics}{点缀}{dian3zhui4}[][HSK 7-9]
    \definition{v.}{adornar; embelezar; ornamentar; realçar, decorar, fazer com que pareça melhor | usar algo apenas para mostrar; decorar a fachada; adaptar-se à ocasião}
  \end{Phonetics}
\end{Entry}

\begin{Entry}{点燃}{9,16}{⽕,⽕}
  \begin{Phonetics}{点燃}{dian3ran2}[][HSK 5]
    \definition{v.}{acender; inflamar; acender uma fogueira, para iluminar}
  \end{Phonetics}
\end{Entry}

%%%%%%%%%% 炼 %%%%%%%%%%
\subsection*{炼}\addcontentsline{loh}{figure}{炼}

\begin{Entry}{炼}{9}{⽕}
  \begin{Phonetics}{炼}{lian4}[][HSK 7-9]
    \definition*{s.}{Sobrenome: Lian}
    \definition{v.}{fundir; refinar; purificar ou endurecer uma substância por meio de aquecimento ou outros métodos | temperar (um metal) com fogo | ponderar as próprias palavras; buscar a expressão adequada; aprimorar; analisar e refinar cuidadosamente as palavras e frases para torná-las concisas e belas}
  \end{Phonetics}
\end{Entry}

%%%%%%%%%% 烂 %%%%%%%%%%
\subsection*{烂}\addcontentsline{loh}{figure}{烂}

\begin{Entry}{烂}{9}{⽕}
  \begin{Phonetics}{烂}{lan4}[][HSK 5]
    \definition{adj.}{macio; pastoso; amassado | podre; deteriorado | quebrado; esfarrapado; gasto | desorganizado; indigno}
    \definition{adv.}{totalmente; extremamente; completamente; expressa um grau muito profundo}
    \definition{v.}{apodrecer; infeccionar; decompor-se}
  \end{Phonetics}
\end{Entry}

%%%%%%%%%% 牲 %%%%%%%%%%
\subsection*{牲}\addcontentsline{loh}{figure}{牲}

\begin{Entry}{牲}{9}{⽜}
  \begin{Phonetics}{牲}{sheng1}
    \definition[头]{s.}{gado (para sacrifício) | sacrifício de animais | animal doméstico}
  \end{Phonetics}
\end{Entry}

\begin{Entry}{牲畜}{9,10}{⽜,⽥}
  \begin{Phonetics}{牲畜}{sheng1chu4}[][HSK 7-9]
    \definition[种,群]{s.}{gado; animais domésticos}
  \end{Phonetics}
\end{Entry}

%%%%%%%%%% 牵 %%%%%%%%%%
\subsection*{牵}\addcontentsline{loh}{figure}{牵}

\begin{Entry}{牵}{9}{⽜}
  \begin{Phonetics}{牵}{qian1}[][HSK 6]
    \definition{v.}{conduzir (segurando a mão, o cabresto, etc.); puxar | envolver-se | sentir falta; preocupar-se com | controlar; restringir; ser retido; ser constrangido}
  \end{Phonetics}
\end{Entry}

\begin{Entry}{牵头}{9,5}{⽜,⼤}
  \begin{Phonetics}{牵头}{qian1/tou2}[][HSK 7-9]
    \definition{v.+compl.}{intermediar (por exemplo: casamenteiro) | coordenar (uma operação combinada) | conduzir (um animal pela cabeça) | mediar | assumir a liderança}
  \end{Phonetics}
\end{Entry}

\begin{Entry}{牵扯}{9,7}{⽜,⼿}
  \begin{Phonetics}{牵扯}{qian1che3}[][HSK 7-9]
    \definition{v.}{envolver; arrastar para}
  \end{Phonetics}
\end{Entry}

\begin{Entry}{牵制}{9,8}{⽜,⼑}
  \begin{Phonetics}{牵制}{qian1zhi4}[][HSK 7-9]
    \definition{v.}{conter; imobilizar; amarrar; restringir ou impedir a livre circulação (frequentemente usado em contextos militares)}
  \end{Phonetics}
\end{Entry}

\begin{Entry}{牵挂}{9,9}{⽜,⼿}
  \begin{Phonetics}{牵挂}{qian1gua4}[][HSK 7-9]
    \definition{v.}{preocupar-se; estar preocupado; perder}
  \end{Phonetics}
\end{Entry}

\begin{Entry}{牵涉}{9,10}{⽜,⽔}
  \begin{Phonetics}{牵涉}{qian1she4}[][HSK 7-9]
    \definition{v.}{preocupar-se com; envolver; arrastar para; uma coisa está relacionada a outras coisas ou pessoas}
  \end{Phonetics}
\end{Entry}

%%%%%%%%%% 狠 %%%%%%%%%%
\subsection*{狠}\addcontentsline{loh}{figure}{狠}

\begin{Entry}{狠}{9}{⽝}
  \begin{Phonetics}{狠}{hen3}[][HSK 6]
    \definition{adj.}{impiedoso; implacável; feroz | firme; resoluto; severo; determinado}
    \definition{adv.}{muito; bastante; bastante | também, frequentemente usado antes de um adjetivo sem intensificar seu significado, ou seja, como um elemento sintático sem sentido}
    \definition{v.}{endurecer (o coração); suprimir (os próprios sentimentos)}
    \variantof{很}
  \seealsoref{很}{hen3}
  \end{Phonetics}
\end{Entry}

%%%%%%%%%% 狡 %%%%%%%%%%
\subsection*{狡}\addcontentsline{loh}{figure}{狡}

\begin{Entry}{狡}{9}{⽝}
  \begin{Phonetics}{狡}{jiao3}
    \definition{adj.}{astuto; esperto; ardiloso}
  \end{Phonetics}
\end{Entry}

\begin{Entry}{狡猾}{9,12}{⽝,⽜}
  \begin{Phonetics}{狡猾}{jiao3hua2}[][HSK 7-9]
    \definition{adj.}{astuto; ardiloso; manhoso; esperto; escorregadio; astuto e indigno de confiança}
  \end{Phonetics}
\end{Entry}

%%%%%%%%%% 独 %%%%%%%%%%
\subsection*{独}\addcontentsline{loh}{figure}{独}

\begin{Entry}{独}{9}{⽝}
  \begin{Phonetics}{独}{du2}[][HSK 7-9]
    \definition*{s.}{Sobrenome: Du}
    \definition{adj.}{só; solteiro | (coloquial) distante | único; só}
    \definition{adv.}{unicamente; somente | sozinho; por si mesmo; em solidão}
    \definition{s.}{idosos sem descendência; os sem filhos}
  \end{Phonetics}
\end{Entry}

\begin{Entry}{独一无二}{9,1,4,2}{⽝,⼀,⽆,⼆}
  \begin{Phonetics}{独一无二}{du2yi1-wu2'er4}[][HSK 7-9]
    \definition{expr.}{único; incomparável; inigualável | o único; em uma classe própria (si mesmo); incomparável; o único de seu tipo; o original sem cópias; sem igual; não há semelhança; não há comparação}
  \end{Phonetics}
\end{Entry}

\begin{Entry}{独立}{9,5}{⽝,⽴}
  \begin{Phonetics}{独立}{du2li4}[][HSK 4]
    \definition{adj.}{independente; por conta própria | separado; respectivo; descreve algo que é separado e não está em contato com outra coisa}
    \definition{v.}{ficar sozinho | alcançar a independência; tornar-se um país independente; liberdade de um Estado, regime ou organização contra interferência, controle e dominação por forças externas}
  \end{Phonetics}
\end{Entry}

\begin{Entry}{独立自主}{9,5,6,5}{⽝,⽴,⾃,⼂}
  \begin{Phonetics}{独立自主}{du2li4-zi4zhu3}[][HSK 7-9]
    \definition{expr.}{manter a independência e manter a iniciativa; agir de forma independente e manter a iniciativa; manter-se independente; ser independente}
  \end{Phonetics}
\end{Entry}

\begin{Entry}{独自}{9,6}{⽝,⾃}
  \begin{Phonetics}{独自}{du2zi4}[][HSK 4]
    \definition{adv.}{sozinho; por si mesmo; por conta própria}
  \end{Phonetics}
\end{Entry}

\begin{Entry}{独身}{9,7}{⽝,⾝}
  \begin{Phonetics}{独身}{du2shen1}[][HSK 7-9]
    \definition{adj.}{separado da família | solteiro}
    \definition{s.}{celibato; solteirice}
  \end{Phonetics}
\end{Entry}

\begin{Entry}{独家}{9,10}{⽝,⼧}
  \begin{Phonetics}{独家}{du2jia1}[][HSK 7-9]
    \definition{adj.}{único; exclusivo}
  \end{Phonetics}
\end{Entry}

\begin{Entry}{独特}{9,10}{⽝,⽜}
  \begin{Phonetics}{独特}{du2te4}[][HSK 4]
    \definition{adj.}{único; distinto; original; especial}
  \end{Phonetics}
\end{Entry}

\begin{Entry}{独资}{9,10}{⽝,⾙}
  \begin{Phonetics}{独资}{du2zi1}
    \definition[家]{s.}{investimento exclusivo | empresa unipessoal; propriedade integral (geralmente de uma empresa estrangeira)}
  \end{Phonetics}
\end{Entry}

\begin{Entry}{独唱}{9,11}{⽝,⼝}
  \begin{Phonetics}{独唱}{du2chang4}[][HSK 7-9]
    \definition{s.}{(em canto) solo}
  \end{Phonetics}
\end{Entry}

%%%%%%%%%% 狭 %%%%%%%%%%
\subsection*{狭}\addcontentsline{loh}{figure}{狭}

\begin{Entry}{狭}{9}{⽝}
  \begin{Phonetics}{狭}{xia2}
    \definition{adj.}{estreito}
  \antonymref{广}{guang3}
  \end{Phonetics}
\end{Entry}

%%%%%%%%%% 狮 %%%%%%%%%%
\subsection*{狮}\addcontentsline{loh}{figure}{狮}

\begin{Entry}{狮}{9}{⽝}
  \begin{Phonetics}{狮}{shi1}
    \definition[只,头]{s.}{leão}
  \end{Phonetics}
\end{Entry}

\begin{Entry}{狮子}{9,3}{⽝,⼦}
  \begin{Phonetics}{狮子}{shi1zi5}[][HSK 7-9]
    \definition[头,只,群]{s.}{leão}
  \end{Phonetics}
\end{Entry}

%%%%%%%%%% 玻 %%%%%%%%%%
\subsection*{玻}\addcontentsline{loh}{figure}{玻}

\begin{Entry}{玻}{9}{⽟}
  \begin{Phonetics}{玻}{bo1}
    \definition{s.}{vidro}
  \end{Phonetics}
\end{Entry}

\begin{Entry}{玻璃}{9,14}{⽟,⽟}
  \begin{Phonetics}{玻璃}{bo1li5}[][HSK 5]
    \definition[张,块]{s.}{vidro; corpo duro, quebradiço e transparente, geralmente feito de areia, calcário, carbonato de sódio, etc. | \emph{nylon}; plástico; refere"-se a determinados plásticos que se assemelham ao vidro}
  \end{Phonetics}
\end{Entry}

%%%%%%%%%% 珍 %%%%%%%%%%
\subsection*{珍}\addcontentsline{loh}{figure}{珍}

\begin{Entry}{珍}{9}{⽟}
  \begin{Phonetics}{珍}{zhen1}
    \definition{adj.}{precioso; valioso; raro | inestimável}
    \definition{s.}{tesouro | objetos de valor}
    \definition{v.}{valorizar muito; estimar}
  \end{Phonetics}
\end{Entry}

\begin{Entry}{珍贵}{9,9}{⽟,⾙}
  \begin{Phonetics}{珍贵}{zhen1gui4}[][HSK 5]
    \definition{adj.}{raro; valioso; precioso; de grande valor; profundo significado}
  \end{Phonetics}
\end{Entry}

\begin{Entry}{珍珠}{9,10}{⽟,⽟}
  \begin{Phonetics}{珍珠}{zhen1zhu1}[][HSK 5]
    \definition[颗,串]{s.}{pérola; grânulos redondos produzidos nas conchas de certos animais aquáticos, de cor branca, rosa, etc., bonitos e brilhantes, frequentemente usados como adornos}
  \end{Phonetics}
\end{Entry}

\begin{Entry}{珍惜}{9,11}{⽟,⼼}
  \begin{Phonetics}{珍惜}{zhen1xi1}[][HSK 5]
    \definition{v.}{valorizar; estimar; valorizar e evitar o desperdício}
  \end{Phonetics}
\end{Entry}

%%%%%%%%%% 甚 %%%%%%%%%%
\subsection*{甚}\addcontentsline{loh}{figure}{甚}

\begin{Entry}{甚}{9}{⽢}
  \begin{Phonetics}{甚}{shen4}
    \definition{adv.}{muito; extremamente}
    \definition{pron.}{o que}
    \definition{v.}{exceder; superar}
  \seealsoref{什么}{shen2me5}
  \end{Phonetics}
\end{Entry}

\begin{Entry}{甚而}{9,6}{⽢,⽽}
  \begin{Phonetics}{甚而}{shen4'er2}
    \definition{conj.}{(ir) tão longe quanto | tanto que}
  \end{Phonetics}
\end{Entry}

\begin{Entry}{甚至}{9,6}{⽢,⾄}
  \begin{Phonetics}{甚至}{shen4zhi4}[][HSK 4]
    \definition{conj.}{e até mesmo; nem mesmo; para apresentar uma situação típica e especial, para enfatizar a profundidade e a seriedade de uma situação}
  \end{Phonetics}
\end{Entry}

\begin{Entry}{甚至于}{9,6,3}{⽢,⾄,⼆}
  \begin{Phonetics}{甚至于}{shen4zhi4yu2}[][HSK 7-9]
    \definition{adv.}{até (na medida em que)}
    \definition{conj.}{(ir) tão longe a ponto de; tanto que; ainda mais}
  \end{Phonetics}
\end{Entry}

\begin{Entry}{甚或}{9,8}{⽢,⼽}
  \begin{Phonetics}{甚或}{shen4huo4}
    \definition{conj.}{(ir) tão longe quanto | tanto que}
  \end{Phonetics}
\end{Entry}

%%%%%%%%%% 甭 %%%%%%%%%%
\subsection*{甭}\addcontentsline{loh}{figure}{甭}

\begin{Entry}{甭}{9}{⽤}
  \begin{Phonetics}{甭}{beng2}
    \definition{adv.}{não; não precisa; não tem que; contração de 不用}
  \seealsoref{不用}{bu2yong4}
  \end{Phonetics}
\end{Entry}

%%%%%%%%%% 界 %%%%%%%%%%
\subsection*{界}\addcontentsline{loh}{figure}{界}

\begin{Entry}{界}{9}{⽥}
  \begin{Phonetics}{界}{jie4}[][HSK 6]
    \definition{s.}{fronteira; limite | escopo; extensão | círculos | divisão primária; reino | era geológica | (matemática) limite | mundo; faixa dividida por ocupação, emprego ou gênero, etc. | grupo}
  \end{Phonetics}
\end{Entry}

\begin{Entry}{界定}{9,8}{⽥,⼧}
  \begin{Phonetics}{界定}{jie4ding4}[][HSK 7-9]
    \definition{v.}{definir; delimitar; especificar os limites; definir escopo | definir; dar uma definição}
  \end{Phonetics}
\end{Entry}

\begin{Entry}{界线}{9,8}{⽥,⽷}
  \begin{Phonetics}{界线}{jie4xian4}[][HSK 7-9]
    \definition{s.}{fronteira; limite; a fronteira entre coisas diferentes; a linha que divide duas regiões}
  \end{Phonetics}
\end{Entry}

\begin{Entry}{界限}{9,8}{⽥,⾩}
  \begin{Phonetics}{界限}{jie4xian4}[][HSK 7-9]
    \definition{s.}{linha; limites; fronteiras; demarcação (ou divisão); a fronteira entre coisas diferentes | limite; fim}
  \end{Phonetics}
\end{Entry}

\begin{Entry}{界碑}{9,13}{⽥,⽯}
  \begin{Phonetics}{界碑}{jie4bei1}
    \definition{s.}{marco de fronteira}
  \end{Phonetics}
\end{Entry}

%%%%%%%%%% 疯 %%%%%%%%%%
\subsection*{疯}\addcontentsline{loh}{figure}{疯}

\begin{Entry}{疯}{9}{⽧}
  \begin{Phonetics}{疯}{feng1}[][HSK 5]
    \definition{adj.}{louco; insano | tolo; leviano | (planta, safra de grãos, etc.) esguia; refere"-se ao crescimento vigoroso das plantações, mas sem frutos | com todas as forças; fazer o máximo possível}
    \definition{v.}{jogar sem restrições}
  \end{Phonetics}
\end{Entry}

\begin{Entry}{疯子}{9,3}{⽧,⼦}
  \begin{Phonetics}{疯子}{feng1zi5}[][HSK 7-9]
    \definition[个,些,种]{s.}{maníaco; lunático; louco; pessoas com doenças mentais graves}[别把我当成疯子!===Não me trate como um louco!]
  \end{Phonetics}
\end{Entry}

\begin{Entry}{疯狂}{9,7}{⽧,⽝}
  \begin{Phonetics}{疯狂}{feng1kuang2}[][HSK 5]
    \definition{adj.}{louco; insano; frenético; desenfreado}
  \end{Phonetics}
\end{Entry}

%%%%%%%%%% 皆 %%%%%%%%%%
\subsection*{皆}\addcontentsline{loh}{figure}{皆}

\begin{Entry}{皆}{9}{⽩}
  \begin{Phonetics}{皆}{jie1}[][HSK 7-9]
    \definition{adv.}{todos; em todos os casos; cada um e todos}
  \end{Phonetics}
\end{Entry}

%%%%%%%%%% 皇 %%%%%%%%%%
\subsection*{皇}\addcontentsline{loh}{figure}{皇}

\begin{Entry}{皇}{9}{⽩}
  \begin{Phonetics}{皇}{huang2}
    \definition*{s.}{Sobrenome: Huang}
    \definition{adj.}{grandioso; magnífico}
    \definition{s.}{imperador, o governante supremo de uma dinastia feudal após a Dinastia Qin; soberano}
  \end{Phonetics}
\end{Entry}

\begin{Entry}{皇上}{9,3}{⽩,⼀}
  \begin{Phonetics}{皇上}{huang2shang5}[][HSK 7-9]
    \definition*{s.}{Sua Majestade; Vossa Majestade | Sua Majestade Imperial | Sua Majestade o Imperador}
    \definition{s.}{imperador; trono; soberano reinante}
  \end{Phonetics}
\end{Entry}

\begin{Entry}{皇后}{9,6}{⽩,⼝}
  \begin{Phonetics}{皇后}{huang2hou4}[][HSK 7-9]
    \definition[个,位,任]{s.}{rainha; imperatriz; a esposa do imperador}
  \end{Phonetics}
\end{Entry}

\begin{Entry}{皇室}{9,9}{⽩,⼧}
  \begin{Phonetics}{皇室}{huang2shi4}[][HSK 7-9]
    \definition{s.}{família imperial (ou casa) | governo imperial; corte real | casa imperial | membro da família real}
  \end{Phonetics}
\end{Entry}

\begin{Entry}{皇宫}{9,9}{⽩,⼧}
  \begin{Phonetics}{皇宫}{huang2gong1}[][HSK 7-9]
    \definition{s.}{palácio (imperial) | palácio imperial; onde o imperador morava}
  \end{Phonetics}
\end{Entry}

\begin{Entry}{皇帝}{9,9}{⽩,⼱}
  \begin{Phonetics}{皇帝}{huang2di4}[][HSK 6]
    \definition[个,位,任]{s.}{imperador; o título do mais alto governante feudal na China começou com o título de Imperador Qin Shi Huang}
  \end{Phonetics}
\end{Entry}

%%%%%%%%%% 盆 %%%%%%%%%%
\subsection*{盆}\addcontentsline{loh}{figure}{盆}

\begin{Entry}{盆}{9}{⽫}
  \begin{Phonetics}{盆}{pen2}[][HSK 5]
    \definition*{s.}{Sobrenome: Pen}
    \definition{s.}{bacia; banheira; panela; utensílios para guardar ou lavar coisas}
  \end{Phonetics}
\end{Entry}

\begin{Entry}{盆友}{9,4}{⽫,⼜}
  \begin{Phonetics}{盆友}{pen2you3}
    \definition{s.}{Gíria da \emph{Internet}: amigo (trocadilho com 朋友)}
  \seealsoref{朋友}{peng2you5}
  \end{Phonetics}
\end{Entry}

%%%%%%%%%% 相 %%%%%%%%%%
\subsection*{相}\addcontentsline{loh}{figure}{相}

\begin{Entry}{相}{9}{⽬}
  \begin{Phonetics}{相}{xiang1}
    \definition*{s.}{Sobrenome: Xiang}
    \definition{adv.}{uns aos outros; mutuamente | (para uma ação realizada por uma pessoa em relação a outra) | indica a ação de uma parte em relação à outra parte}
    \definition{s.}{qualidade; substância}
    \definition{v.}{ver por si mesmo (se algo ou algo é do seu agrado)}
  \end{Phonetics}
  \begin{Phonetics}{相}{xiang4}
    \definition*{s.}{Sobrenome: Xiang}
    \definition{s.}{aparência | postura; porte; postura sentada, em pé, etc. | Física: fase; refere"-se a uma parte homogênea de uma substância com a mesma composição e as mesmas propriedades físicas e químicas | fotografia | primeiro"-ministro (na China antiga) | ministro; títulos oficiais de certos países | fácies marinha (carvão) | elefante, uma das peças do xadrez chinês | recepcionista (pessoa que ajuda o anfitrião a receber o hóspede); antigamente, referia"-se a alguém que ajudava o anfitrião a receber convidados}
    \definition{v.}{olhar e avaliar; observe a aparência das coisas; julgar sua qualidade | assistir; ajudar; auxiliar}
  \end{Phonetics}
\end{Entry}

\begin{Entry}{相互}{9,4}{⽬,⼆}
  \begin{Phonetics}{相互}{xiang1hu4}[][HSK 3]
    \definition{adj.}{mútuo; recíproco; entre duas pessoas ou coisas}
    \definition{adv.}{mutuamente; um ao outro; tratamento recíproco}
  \end{Phonetics}
\end{Entry}

\begin{Entry}{相反}{9,4}{⽬,⼜}
  \begin{Phonetics}{相反}{xiang1fan3}[][HSK 4]
    \definition{adj.}{oposto; contrário; dois aspectos das coisas são contraditórios e mutuamente exclusivos}
    \definition{conj.}{pelo contrário; usado no início ou no meio de uma frase para indicar uma contradição de significado com o que foi dito anteriormente.}
  \end{Phonetics}
\end{Entry}

\begin{Entry}{相比}{9,4}{⽬,⽐}
  \begin{Phonetics}{相比}{xiang1bi3}[][HSK 3]
    \definition{v.}{combinar; comparar com | comparar mutuamente, usar uma coisa como padrão, perceber as características de outra coisa ou obter uma opinião}
  \end{Phonetics}
\end{Entry}

\begin{Entry}{相片}{9,4}{⽬,⽚}
  \begin{Phonetics}{相片}{xiang4pian4}[][HSK 4]
    \definition[张]{s.}{foto; fotografia; uma imagem de uma pessoa ou objeto feita pela exposição de papel fotográfico a um negativo fotográfico e, em seguida, revelando e fixando a imagem.}
  \end{Phonetics}
\end{Entry}

\begin{Entry}{相处}{9,5}{⽬,⼡}
  \begin{Phonetics}{相处}{xiang1chu3}[][HSK 4]
    \definition{v.}{dar-se bem; viver juntos; dar-se bem (uns com os outros); viver uns com os outros; entrar em contato uns com os outros, tratar uns aos outros}
  \end{Phonetics}
\end{Entry}

\begin{Entry}{相似}{9,6}{⽬,⼈}
  \begin{Phonetics}{相似}{xiang1si4}[][HSK 3]
    \definition{v.}{assemelhar-se; ser semelhante; ser parecido}
  \end{Phonetics}
\end{Entry}

\begin{Entry}{相关}{9,6}{⽬,⼋}
  \begin{Phonetics}{相关}{xiang1guan1}[][HSK 3]
    \definition{v.}{estar mutuamente relacionado; estar intimamente relacionado; estar inter-relacionado}
  \end{Phonetics}
\end{Entry}

\begin{Entry}{相同}{9,6}{⽬,⼝}
  \begin{Phonetics}{相同}{xiang1tong2}[][HSK 2]
    \definition{adj.}{semelhante; similar; igual; idêntico; o mesmo; consistentes entre si, sem diferença}
  \end{Phonetics}
\end{Entry}

\begin{Entry}{相当}{9,6}{⽬,⼹}
  \begin{Phonetics}{相当}{xiang1dang1}[][HSK 3]
    \definition{adj.}{adequado; apropriado}
    \definition{adv.}{bastante; razoavelmente; consideravelmente; indica um grau relativamente alto e profundo}
    \definition{v.}{combinar; equilibrar; corresponder a; ser aproximadamente igual a; ser proporcional a}
  \end{Phonetics}
\end{Entry}

\begin{Entry}{相机}{9,6}{⽬,⽊}
  \begin{Phonetics}{相机}{xiang4ji1}[][HSK 2]
    \definition[台,部,架,个]{s.}{câmera; máquina fotográfica}
    \definition{v.}{ficar atento a uma oportunidade; procurar oportunidades}
  \end{Phonetics}
\end{Entry}

\begin{Entry}{相声}{9,7}{⽬,⼠}
  \begin{Phonetics}{相声}{xiang4sheng5}[][HSK 5]
    \definition[个,段]{s.}{conversa cruzada; diálogo cômico; forma de performance humorística, em que os atores usam piadas, canções e imitações para satirizar e elogiar}
  \end{Phonetics}
\end{Entry}

\begin{Entry}{相应}{9,7}{⽬,⼴}
  \begin{Phonetics}{相应}{xiang1ying5}[][HSK 5]
    \definition{adj.}{Dialeto: barato}
    \definition{v.}{corresponder}
  \end{Phonetics}
\end{Entry}

\begin{Entry}{相宜}{9,8}{⽬,⼧}
  \begin{Phonetics}{相宜}{xiang1yi2}
    \definition{adj.}{adequado | apropriado}
    \definition{v.}{ser adequado ou apropriado}
  \end{Phonetics}
\end{Entry}

\begin{Entry}{相亲}{9,9}{⽬,⼇}
  \begin{Phonetics}{相亲}{xiang1qin1}
    \definition{s.}{encontro às cegas | entrevista arranjada para avaliar a proposta de um parceiro de casamento | apegar-se profundamente um ao outro}
  \end{Phonetics}
\end{Entry}

\begin{Entry}{相信}{9,9}{⽬,⼈}
  \begin{Phonetics}{相信}{xiang1xin4}[][HSK 2]
    \definition{v.}{acreditar em; estar convencido de; ter fé em; acreditar que algo é certo ou verdadeiro sem dúvida}
  \end{Phonetics}
\end{Entry}

\begin{Entry}{相思病}{9,9,10}{⽬,⼼,⽧}
  \begin{Phonetics}{相思病}{xiang1si1bing4}
    \definition{s.}{saudade de amor}
  \end{Phonetics}
\end{Entry}

\begin{Entry}{相等}{9,12}{⽬,⽵}
  \begin{Phonetics}{相等}{xiang1deng3}[][HSK 5]
    \definition{v.}{ser igual a; possuir a mesma quantidade, peso, tamanho e grau}
  \end{Phonetics}
\end{Entry}

\begin{Entry}{相遇}{9,12}{⽬,⾡}
  \begin{Phonetics}{相遇}{xiang1yu4}
    \definition{v.}{encontrar (reunião, encontro, etc.)}
  \end{Phonetics}
\end{Entry}

\begin{Entry}{相聚}{9,14}{⽬,⽿}
  \begin{Phonetics}{相聚}{xiang1ju4}
    \definition{v.}{reunir-se | montar}
  \end{Phonetics}
\end{Entry}

%%%%%%%%%% 盼 %%%%%%%%%%
\subsection*{盼}\addcontentsline{loh}{figure}{盼}

\begin{Entry}{盼}{9}{⽬}
  \begin{Phonetics}{盼}{pan4}[][HSK 7-9]
    \definition*{s.}{Sobrenome: Pan}
    \definition{adj.}{(olhos) com preto e branco fortemente contrastados; olhos claros}
    \definition{v.}{olhar | esperar por; ansiar por | sentir falta de ; continuar pensando sobre}
  \end{Phonetics}
\end{Entry}

\begin{Entry}{盼望}{9,11}{⽬,⽉}
  \begin{Phonetics}{盼望}{pan4wang4}[][HSK 6]
    \definition{v.}{esperar por; ansiar por; esperar que algo aconteça em breve}
  \end{Phonetics}
\end{Entry}

%%%%%%%%%% 省 %%%%%%%%%%
\subsection*{省}\addcontentsline{loh}{figure}{省}

\begin{Entry}{省}{9}{⽬}
  \begin{Phonetics}{省}{sheng3}[][HSK 2]
    \definition*{s.}{Sobrenome: Sheng}
    \definition{s.}{província; unidade administrativa, subordinada diretamente ao governo central | capital provincial; refere"-se à capital da província, localização da administração provincial | abreviação (de palavras)}
    \definition{v.}{economizar; poupar; reduzir o consumo | omitir; deixar de fora}
  \antonymref{费}{fei4}
  \end{Phonetics}
  \begin{Phonetics}{省}{xing3}
    \definition{v.}{examinar"-se criticamente; verificar (os próprios pensamentos, palavras e ações) | visitar (especialmente os pais ou pessoas mais velhas) | estar ciente; tornar"-se consciente; compreender; tomar consciência | examinar minuciosamente; inspecionar; escrutinar}
  \end{Phonetics}
\end{Entry}

\begin{Entry}{省力}{9,2}{⽬,⼒}
  \begin{Phonetics}{省力}{sheng3li4}
    \definition{v.}{economizar esforço ou trabalho}
  \end{Phonetics}
\end{Entry}

\begin{Entry}{省心}{9,4}{⽬,⼼}
  \begin{Phonetics}{省心}{sheng3xin1}
    \definition{adj.}{despreocupado}
    \definition{v.}{ser poupado de preocupações | despreocupar"-se}
  \end{Phonetics}
\end{Entry}

\begin{Entry}{省长}{9,4}{⽬,⾧}
  \begin{Phonetics}{省长}{sheng3zhang3}
    \definition[位,任]{s.}{governador; governador de uma província}
  \end{Phonetics}
\end{Entry}

\begin{Entry}{省会}{9,6}{⽬,⼈}
  \begin{Phonetics}{省会}{sheng3hui4}
    \definition{s.}{capital da província}
  \synonymref{省城}{sheng3cheng2}
  \end{Phonetics}
\end{Entry}

\begin{Entry}{省却}{9,7}{⽬,⼙}
  \begin{Phonetics}{省却}{sheng3que4}
    \definition{v.}{livrar-se (para economizar espaço) | salvar}
  \end{Phonetics}
\end{Entry}

\begin{Entry}{省事}{9,8}{⽬,⼅}
  \begin{Phonetics}{省事}{sheng3/shi4}[][HSK 7-9]
    \definition{adj.}{conveniente; prático; sem problemas}
    \definition{v.+compl.}{evitar problemas; simplificar as coisas}
  \end{Phonetics}
\end{Entry}

\begin{Entry}{省俭}{9,9}{⽬,⼈}
  \begin{Phonetics}{省俭}{sheng3jian3}
    \definition{s.}{econômico | frugal}
    \definition{v.}{economizar}
  \end{Phonetics}
\end{Entry}

\begin{Entry}{省城}{9,9}{⽬,⼟}
  \begin{Phonetics}{省城}{sheng3cheng2}
    \definition{s.}{capital da província}
  \synonymref{省会}{sheng3hui4}
  \synonymref{首府}{shou3fu3}
  \end{Phonetics}
\end{Entry}

\begin{Entry}{省悟}{9,10}{⽬,⼼}
  \begin{Phonetics}{省悟}{xing3wu4}
    \definition{v.}{voltar a si | constatar | ver a verdade | acordar para a realidade}
  \end{Phonetics}
\end{Entry}

\begin{Entry}{省钱}{9,10}{⽬,⾦}
  \begin{Phonetics}{省钱}{sheng3qian2}[][HSK 6]
    \definition{adj.}{barato; não caro}
    \definition{v.}{economizar dinheiro}
  \end{Phonetics}
\end{Entry}

\begin{Entry}{省略}{9,11}{⽬,⽥}
  \begin{Phonetics}{省略}{sheng3lve4}[][HSK 7-9]
    \definition{v.}{omitir; excluir; apagar; eliminar linguagem e procedimentos desnecessários}
  \synonymref{减少}{jian3shao3}
  \synonymref{删除}{shan1chu2}
  \end{Phonetics}
\end{Entry}

%%%%%%%%%% 眉 %%%%%%%%%%
\subsection*{眉}\addcontentsline{loh}{figure}{眉}

\begin{Entry}{眉}{9}{⽬}
  \begin{Phonetics}{眉}{mei2}
    \definition*{s.}{Sobrenome: Mei}
    \definition[个]{s.}{sobrancelha | a margem superior de uma página; o espaço em branco na parte superior da página}
  \end{Phonetics}
\end{Entry}

\begin{Entry}{眉开眼笑}{9,4,11,10}{⽬,⼶,⽬,⽵}
  \begin{Phonetics}{眉开眼笑}{mei2kai1-yan3xiao4}[][HSK 7-9]
    \definition{expr.}{parecer alegre; um semblante radiante; estar sempre sorrindo; irradiar alegria; estar cheio de felicidade; rosto transbordando de sorrisos; sentir-se feliz e sorrir; sorrir de orelha a orelha; os olhos brilhando de alegria; o rosto se iluminando de sorrisos; os olhos brilhantes dançando de alegria; sorrir alegremente (de orelha a orelha); muito feliz}
  \end{Phonetics}
\end{Entry}

\begin{Entry}{眉毛}{9,4}{⽬,⽑}
  \begin{Phonetics}{眉毛}{mei2mao5}[][HSK 7-9]
    \definition[根]{s.}{sobrancelha; pelos que crescem ao longo da borda superior da órbita ocular humana}
  \end{Phonetics}
\end{Entry}

\begin{Entry}{眉头}{9,5}{⽬,⼤}
  \begin{Phonetics}{眉头}{mei2tou2}
    \definition{s.}{testa; área próxima às sobrancelhas}
  \end{Phonetics}
\end{Entry}

%%%%%%%%%% 看 %%%%%%%%%%
\subsection*{看}\addcontentsline{loh}{figure}{看}

\begin{Entry}{看}{9}{⽬}
  \begin{Phonetics}{看}{kan1}[][HSK 6]
    \definition{v.}{cuidar de; tomar conta de; cuidar de; proteger | manter sob vigilância}
  \end{Phonetics}
  \begin{Phonetics}{看}{kan4}[][HSK 1]
    \definition{interj.}{``Cuidado!'' (para um perigo)}
    \definition{part.}{tentar, usado depois de outros verbos}
    \definition{v.}{ver; olhar para; observar; fazer contato visual com pessoas ou objetos | pensar; considerar; observar; julgar; observar e analisar | visitar; ver; fazer uma visita | olhar para; considerar; tratar | tratar (um paciente ou uma doença) | cuidar | ficar atento; ficar de olho | depender de; ser dependente de | ler}
  \end{Phonetics}
\end{Entry}

\begin{Entry}{看上去}{9,3,5}{⽬,⼀,⼛}
  \begin{Phonetics}{看上去}{kan4shang4qu5}[][HSK 3]
    \definition{adv.}{parece que}
  \end{Phonetics}
\end{Entry}

\begin{Entry}{看不起}{9,4,10}{⽬,⼀,⾛}
  \begin{Phonetics}{看不起}{kan4bu5qi3}[][HSK 4]
    \definition{v.}{desprezar; desdenhar; menosprezar; ter desprezo; olhar de cima para baixo}
  \end{Phonetics}
\end{Entry}

\begin{Entry}{看中}{9,4}{⽬,⼁}
  \begin{Phonetics}{看中}{kan4/zhong4}[][HSK 7-9]
    \definition{v.+compl.}{optar por; gostar de; sentir-se satisfeito com}
  \end{Phonetics}
\end{Entry}

\begin{Entry}{看见}{9,4}{⽬,⾒}
  \begin{Phonetics}{看见}{kan4 jian5}[][HSK 1]
    \definition{v.}{ver; avistar; ao olhar, descobrir alguém ou algo}
  \end{Phonetics}
\end{Entry}

\begin{Entry}{看出}{9,5}{⽬,⼐}
  \begin{Phonetics}{看出}{kan4 chu1}[][HSK 5]
    \definition{v.}{decifrar; ver; sondar; encontrar; discernir; perceber | descobrir; estar ciente de}
  \end{Phonetics}
\end{Entry}

\begin{Entry}{看台}{9,5}{⽬,⼝}
  \begin{Phonetics}{看台}{kan4tai2}[][HSK 7-9]
    \definition{s.}{arquibancada | arquibancada para espectadores | terraço | plataforma de visualização}
  \end{Phonetics}
\end{Entry}

\begin{Entry}{看似}{9,6}{⽬,⼈}
  \begin{Phonetics}{看似}{kan4si4}[][HSK 7-9]
    \definition{v.}{parecer; dar a impressão de ser}
  \end{Phonetics}
\end{Entry}

\begin{Entry}{看好}{9,6}{⽬,⼥}
  \begin{Phonetics}{看好}{kan4hao3}[][HSK 6]
    \definition{v.}{elogiar; apreciar; encorajar; acreditar que pessoas ou coisas terão uma boa tendência | estar prestes a surgir uma boa tendência}
  \end{Phonetics}
\end{Entry}

\begin{Entry}{看成}{9,6}{⽬,⼽}
  \begin{Phonetics}{看成}{kan4cheng2}[][HSK 5]
    \definition{v.}{ser capaz de ver ou assistir | tomar como; olhar como; considerar como | tratar como; considerar como; pensar como; ter como}
  \end{Phonetics}
\end{Entry}

\begin{Entry}{看作}{9,7}{⽬,⼈}
  \begin{Phonetics}{看作}{kan4zuo4}[][HSK 6]
    \definition{v.}{considerar como; olhar como}
  \end{Phonetics}
\end{Entry}

\begin{Entry}{看护}{9,7}{⽬,⼿}
  \begin{Phonetics}{看护}{kan1hu4}[][HSK 7-9]
    \definition{s.}{Obsoleto: enfermeira hospitalar}
    \definition{v.}{cuidar; zelar por; tratar de}
  \end{Phonetics}
\end{Entry}

\begin{Entry}{看来}{9,7}{⽬,⽊}
  \begin{Phonetics}{看来}{kan4lai5}[][HSK 4]
    \definition{adv.}{parecer; parecer como se (ou embora); refere"-se a um julgamento aproximado; expressa um julgamento por observação}
    \definition{v.}{ser considerado; na visão de alguém; na opinião de alguém; expressar a ideia aproximada que o locutor tem da situação}
  \end{Phonetics}
\end{Entry}

\begin{Entry}{看到}{9,8}{⽬,⼑}
  \begin{Phonetics}{看到}{kan4 dao4}[][HSK 1]
    \definition{v.}{ver; avistar}
  \end{Phonetics}
\end{Entry}

\begin{Entry}{看法}{9,8}{⽬,⽔}
  \begin{Phonetics}{看法}{kan4fa5}[][HSK 2]
    \definition[个,种,点]{s.}{opinião; perspectiva; (ponto de) vista; uma maneira de ver uma coisa | opinião desfavorável (ou crítica) sobre alguém}
  \end{Phonetics}
\end{Entry}

\begin{Entry}{看待}{9,9}{⽬,⼻}
  \begin{Phonetics}{看待}{kan4dai4}[][HSK 5]
    \definition{v.}{tratar; considerar; olhar com atenção; ter uma certa atitude ou visão em relação a alguém ou alguma coisa}
  \end{Phonetics}
\end{Entry}

\begin{Entry}{看重}{9,9}{⽬,⾥}
  \begin{Phonetics}{看重}{kan4zhong4}[][HSK 7-9]
    \definition{v.}{ter em alta consideração; considerar importante; atribuir importância a}
  \end{Phonetics}
\end{Entry}

\begin{Entry}{看样子}{9,10,3}{⽬,⽊,⼦}
  \begin{Phonetics}{看样子}{kan4 yang4zi5}[][HSK 7-9]
    \definition{v.}{parecer; aparentar; dar a impressão de ser; estimar com base na situação (frequentemente usado como elemento inserido em uma frase)}
  \end{Phonetics}
\end{Entry}

\begin{Entry}{看热闹}{9,10,8}{⽬,⽕,⾾}
  \begin{Phonetics}{看热闹}{kan4 re4nao5}[][HSK 7-9]
    \definition{v.}{observar a emoção (ou a diversão) | regozijar"-se com (ou sobre); observar de braços cruzados; ficar de fora | observar a cena movimentada; ser um espectador; observar; ver (assistir) a diversão}
  \end{Phonetics}
\end{Entry}

\begin{Entry}{看病}{9,10}{⽬,⽧}
  \begin{Phonetics}{看病}{kan4/bing4}[][HSK 1]
    \definition{v.+compl.}{(de um médico) ver um paciente | (de um paciente) ver (consultar) um médico}
  \end{Phonetics}
\end{Entry}

\begin{Entry}{看起来}{9,10,7}{⽬,⾛,⽊}
  \begin{Phonetics}{看起来}{kan4qi3lai5}[][HSK 3]
    \definition{v.}{parecer; aparentar; dar a impressão de (ou como se)}
  \end{Phonetics}
\end{Entry}

\begin{Entry}{看得见}{9,11,4}{⽬,⼻,⾒}
  \begin{Phonetics}{看得见}{kan4de5jian4}[][HSK 6]
    \definition{adj.}{perceptível; visível; tangível}
  \end{Phonetics}
\end{Entry}

\begin{Entry}{看得出}{9,11,5}{⽬,⼻,⼐}
  \begin{Phonetics}{看得出}{kan4de5chu1}[][HSK 7-9]
    \definition{expr.}{ser evidente; poder ver; poder ser visto}
  \end{Phonetics}
\end{Entry}

\begin{Entry}{看得起}{9,11,10}{⽬,⼻,⾛}
  \begin{Phonetics}{看得起}{kan4de5qi3}[][HSK 6]
    \definition{v.}{ter uma boa opinião sobre; pensar muito (ou muito) sobre}
  \end{Phonetics}
\end{Entry}

\begin{Entry}{看望}{9,11}{⽬,⽉}
  \begin{Phonetics}{看望}{kan4wang5}[][HSK 4]
    \definition{v.}{ver; visitar; ligar; dar uma olhada; ir até os pais, idosos, professores ou amigos para cumprimentá-los}
  \end{Phonetics}
\end{Entry}

\begin{Entry}{看淡}{9,11}{⽬,⽔}
  \begin{Phonetics}{看淡}{kan4dan4}
    \definition{v.}{considerar sem importância | ser indiferente a (fama, riqueza, etc.) | (de uma economia ou mercado) enfraquecer, ficar mais lento, diminuir a velocidade}
  \end{Phonetics}
\end{Entry}

\begin{Entry}{看管}{9,14}{⽬,⽵}
  \begin{Phonetics}{看管}{kan1guan3}[][HSK 6]
    \definition{v.}{cuidar; atender | guardar; vigiar; ficar de olho em | assumir o comando; estar no comando}
  \end{Phonetics}
\end{Entry}

%%%%%%%%%% 矜 %%%%%%%%%%
\subsection*{矜}\addcontentsline{loh}{figure}{矜}

\begin{Entry}{矜}{9}{⽭}
  \begin{Phonetics}{矜}{jin1}
    \definition{adj.}{presunçoso; vaidoso | contido; reservado; determinado}
    \definition{v.}{ter pena; simpatizar com; compadecer-se}
  \end{Phonetics}
\end{Entry}

%%%%%%%%%% 砂 %%%%%%%%%%
\subsection*{砂}\addcontentsline{loh}{figure}{砂}

\begin{Entry}{砂}{9}{⽯}
  \begin{Phonetics}{砂}{sha1}
    \variantof{沙}
  \end{Phonetics}
\end{Entry}

\begin{Entry}{砂糖}{9,16}{⽯,⽶}
  \begin{Phonetics}{砂糖}{sha1tang2}[][HSK 7-9]
    \definition{s.}{açúcar granulado}
  \end{Phonetics}
\end{Entry}

%%%%%%%%%% 砍 %%%%%%%%%%
\subsection*{砍}\addcontentsline{loh}{figure}{砍}

\begin{Entry}{砍}{9}{⽯}
  \begin{Phonetics}{砍}{kan3}[][HSK 7-9]
    \definition{v.}{cortar; picar; talhar; desbastar; derrubar; podar | reduzir; diminuir; remover | Dialeto: atirar algo em | Coloquial: conversar à toa; fofocar}
  \end{Phonetics}
\end{Entry}

\begin{Entry}{砍刀}{9,2}{⽯,⼑}
  \begin{Phonetics}{砍刀}{kan3dao1}
    \definition{s.}{facão | machete}
  \end{Phonetics}
\end{Entry}

\begin{Entry}{砍头}{9,5}{⽯,⼤}
  \begin{Phonetics}{砍头}{kan3tou2}
    \definition{v.}{decapitar}
  \end{Phonetics}
\end{Entry}

\begin{Entry}{砍价}{9,6}{⽯,⼈}
  \begin{Phonetics}{砍价}{kan3jia4}
    \definition{v.}{barganhar | cortar ou derrubar um preço}
  \end{Phonetics}
\end{Entry}

\begin{Entry}{砍伤}{9,6}{⽯,⼈}
  \begin{Phonetics}{砍伤}{kan3shang1}
    \definition{v.}{ferir com lâmina ou machado}
  \end{Phonetics}
\end{Entry}

\begin{Entry}{砍杀}{9,6}{⽯,⽊}
  \begin{Phonetics}{砍杀}{kan3sha1}
    \definition{v.}{atacar com arma branca}
  \end{Phonetics}
\end{Entry}

\begin{Entry}{砍死}{9,6}{⽯,⽍}
  \begin{Phonetics}{砍死}{kan3si3}
    \definition{v.}{matar com um machado}
  \end{Phonetics}
\end{Entry}

\begin{Entry}{砍树}{9,9}{⽯,⽊}
  \begin{Phonetics}{砍树}{kan3shu4}
    \definition{v.}{derrubar árvores}
  \end{Phonetics}
\end{Entry}

\begin{Entry}{砍掉}{9,11}{⽯,⼿}
  \begin{Phonetics}{砍掉}{kan3diao4}
    \definition{v.}{amputar}
  \end{Phonetics}
\end{Entry}

\begin{Entry}{砍断}{9,11}{⽯,⽄}
  \begin{Phonetics}{砍断}{kan3duan4}
    \definition{v.}{cortar}
  \end{Phonetics}
\end{Entry}

%%%%%%%%%% 研 %%%%%%%%%%
\subsection*{研}\addcontentsline{loh}{figure}{研}

\begin{Entry}{研}{9}{⽯}
  \begin{Phonetics}{研}{yan2}
    \definition{s.}{(abreviação)  pesquisador adjunto, 副研}
    \definition{v.}{moer; esmerilhar; triturar; pulverizar | estudar; pesquisar}
  \seealsoref{副研}{fu4yan2}
  \end{Phonetics}
\end{Entry}

\begin{Entry}{研发}{9,5}{⽯,⼜}
  \begin{Phonetics}{研发}{yan2fa1}[][HSK 6]
    \definition{s.}{pesquisa e desenvolvimento; P\&D}
    \definition{v.}{pesquisar e/ou desenvolver}
  \end{Phonetics}
\end{Entry}

\begin{Entry}{研究}{9,7}{⽯,⽳}
  \begin{Phonetics}{研究}{yan2jiu1}[][HSK 4]
    \definition{v.}{estudar; pesquisar | discutir; considerar}
  \end{Phonetics}
\end{Entry}

\begin{Entry}{研究生}{9,7,5}{⽯,⽳,⽣}
  \begin{Phonetics}{研究生}{yan2jiu1sheng1}[][HSK 4]
    \definition[位,名,个,些]{s.}{pós-graduado; estudante de pós-graduação}
  \end{Phonetics}
\end{Entry}

\begin{Entry}{研究所}{9,7,8}{⽯,⽳,⼾}
  \begin{Phonetics}{研究所}{yan2jiu1suo3}[][HSK 5]
    \definition[家,个]{s.}{instituto de pesquisa; instituição de pesquisa científica envolvida em pesquisas em um determinado campo}
  \end{Phonetics}
\end{Entry}

\begin{Entry}{研制}{9,8}{⽯,⼑}
  \begin{Phonetics}{研制}{yan2zhi4}[][HSK 4]
    \definition{v.}{desenvolver; fabricar; produzir | triturar; (medicina chinesa) moer}
  \end{Phonetics}
\end{Entry}

%%%%%%%%%% 砖 %%%%%%%%%%
\subsection*{砖}\addcontentsline{loh}{figure}{砖}

\begin{Entry}{砖}{9}{⽯}
  \begin{Phonetics}{砖}{zhuan1}
    \definition[块]{s.}{tijolo}
  \end{Phonetics}
\end{Entry}

%%%%%%%%%% 祖 %%%%%%%%%%
\subsection*{祖}\addcontentsline{loh}{figure}{祖}

\begin{Entry}{祖}{9}{⽰}
  \begin{Phonetics}{祖}{zu3}
    \definition*{s.}{Sobrenome: Zu}
    \definition{s.}{avô; geração anterior dos pais | ancestral; antepassado | fundador (de um negócio, facção, seita religiosa, etc.); originador; fundador; mestre fundador}
  \end{Phonetics}
\end{Entry}

\begin{Entry}{祖父}{9,4}{⽰,⽗}
  \begin{Phonetics}{祖父}{zu3fu4}[][HSK 6]
    \definition{s.}{avô (paterno)}
  \end{Phonetics}
\end{Entry}

\begin{Entry}{祖母}{9,5}{⽰,⽏}
  \begin{Phonetics}{祖母}{zu3mu3}[][HSK 6]
    \definition{s.}{avó (paterna)}
  \end{Phonetics}
\end{Entry}

\begin{Entry}{祖国}{9,8}{⽰,⼞}
  \begin{Phonetics}{祖国}{zu3guo2}[][HSK 6]
    \definition{s.}{país; pátria; próprio país}
  \end{Phonetics}
\end{Entry}

%%%%%%%%%% 祝 %%%%%%%%%%
\subsection*{祝}\addcontentsline{loh}{figure}{祝}

\begin{Entry}{祝}{9}{⽰}
  \begin{Phonetics}{祝}{zhu4}[][HSK 3]
    \definition*{s.}{Sobrenome: Zhu}
    \definition{v.}{expressar bons votos; desejar; abençoar | rezar aos deuses ou espíritos para obter bênçãos}
  \end{Phonetics}
\end{Entry}

\begin{Entry}{祝好}{9,6}{⽰,⼥}
  \begin{Phonetics}{祝好}{zhu4hao3}
    \definition{expr.}{desejo-lhe tudo de melhor! (ao encerrar uma correspondência)}
  \end{Phonetics}
\end{Entry}

\begin{Entry}{祝寿}{9,7}{⽰,⼨}
  \begin{Phonetics}{祝寿}{zhu4shou4}
    \definition{v.}{dar parabéns pelo aniversário (a uma pessoa idosa)}
  \end{Phonetics}
\end{Entry}

\begin{Entry}{祝贺}{9,9}{⽰,⾙}
  \begin{Phonetics}{祝贺}{zhu4he4}[][HSK 5]
    \definition[个]{s.}{congratulações; felicitações}
    \definition{v.}{congratular; felicitar; parabenizar}
  \end{Phonetics}
\end{Entry}

\begin{Entry}{祝酒}{9,10}{⽰,⾣}
  \begin{Phonetics}{祝酒}{zhu4jiu3}
    \definition{v.}{parabenizar e fazer um brinde | brindar}
  \end{Phonetics}
\end{Entry}

\begin{Entry}{祝颂}{9,10}{⽰,⾴}
  \begin{Phonetics}{祝颂}{zhu4song4}
    \definition{v.}{expressar bons desejos}
  \end{Phonetics}
\end{Entry}

\begin{Entry}{祝祷}{9,11}{⽰,⽰}
  \begin{Phonetics}{祝祷}{zhu4dao3}
    \definition{v.}{rezar | orar}
  \end{Phonetics}
\end{Entry}

\begin{Entry}{祝谢}{9,12}{⽰,⾔}
  \begin{Phonetics}{祝谢}{zhu4xie4}
    \definition{v.}{agradecer | dar parabéns}
  \end{Phonetics}
\end{Entry}

\begin{Entry}{祝福}{9,13}{⽰,⽰}
  \begin{Phonetics}{祝福}{zhu4fu2}[][HSK 4]
    \definition[个]{s.}{bênção; benzedura; benzimento; originalmente, referia"-se à oração para obter as bênçãos de Deus, mas, mais tarde, refere"-se ao desejo de paz e felicidade às pessoas}
    \definition{v.}{desejar boa sorte a alguém}
  \end{Phonetics}
\end{Entry}

\begin{Entry}{祝愿}{9,14}{⽰,⽕}
  \begin{Phonetics}{祝愿}{zhu4yuan4}[][HSK 6]
    \definition{v.}{desejar; expressar bons desejos}
  \end{Phonetics}
\end{Entry}

%%%%%%%%%% 神 %%%%%%%%%%
\subsection*{神}\addcontentsline{loh}{figure}{神}

\begin{Entry}{神}{9}{⽰}
  \begin{Phonetics}{神}{shen2}[][HSK 5]
    \definition*{s.}{Deus | Sobrenome: Shen}
    \definition{adj.}{inteligente; esperto | mágico; sobrenatural}
    \definition[个,位,尊,名]{s.}{divindade; deidade | espírito; mente; refere"-se ao espírito, energia ou atenção de uma pessoa | olhar; expressão; expressões que refletem o estado interior}
  \end{Phonetics}
\end{Entry}

\begin{Entry}{神气}{9,4}{⽰,⽓}
  \begin{Phonetics}{神气}{shen2qi4}[][HSK 7-9]
    \definition{adj.}{enérgico; vigoroso; cheio de energia | arrogante; que se acha; presunçoso; exibe um ar de superioridade ou arrogância}
    \definition{s.}{ar; modo; expressão; olhar}
  \synonymref{脸色}{lian3se4}
  \synonymref{心情}{xin1qing2}
  \synonymref{样子}{yang4zi5}
  \end{Phonetics}
\end{Entry}

\begin{Entry}{神仙}{9,5}{⽰,⼈}
  \begin{Phonetics}{神仙}{shen2xian1}[][HSK 7-9]
    \definition{s.}{pessoa livre das preocupações mundanas; monge iluminado; uma metáfora para uma pessoa despreocupada, sem restrições e sem preocupações | (na ficção moderna) fada, elfo, duende etc.; figuras míticas possuem habilidades sobre-humanas, transcendendo o reino mortal e alcançando a imortalidade; imortal | profeta; vidente; uma metáfora para alguém que consegue prever ou compreender as coisas}
  \end{Phonetics}
\end{Entry}

\begin{Entry}{神圣}{9,5}{⽰,⼟}
  \begin{Phonetics}{神圣}{shen2sheng4}[][HSK 7-9]
    \definition{adj.}{santo; sagrado; extremamente sublime e solene; inviolável}
  \end{Phonetics}
\end{Entry}

\begin{Entry}{神奇}{9,8}{⽰,⼤}
  \begin{Phonetics}{神奇}{shen2qi2}[][HSK 5]
    \definition{adj.}{mágico; peculiar; místico; milagroso; faz as pessoas se sentirem muito revigoradas; é completamente inesperado e geralmente traz boas influências}
    \definition{adj.}{mágico; peculiar; místico; milagroso; algo que parece muito novo; algo que ninguém imaginaria, mas que geralmente traz bons resultados}
  \end{Phonetics}
\end{Entry}

\begin{Entry}{神态}{9,8}{⽰,⼼}
  \begin{Phonetics}{神态}{shen2tai4}[][HSK 7-9]
    \definition{s.}{semblante; maneira; porte; expressão; expressão e atitude}
  \synonymref{表情}{biao3qing2}
  \synonymref{脸色}{lian3se4}
  \synonymref{模样}{mu2yang4}
  \synonymref{容貌}{rong2mao4}
  \synonymref{神情}{shen2qing2}
  \synonymref{心情}{xin1qing2}
  \synonymref{形状}{xing2zhuang4}
  \synonymref{样子}{yang4zi5}
  \end{Phonetics}
\end{Entry}

\begin{Entry}{神明}{9,8}{⽰,⽇}
  \begin{Phonetics}{神明}{shen2ming2}
    \definition{s.}{divindades | deuses}
  \end{Phonetics}
\end{Entry}

\begin{Entry}{神经}{9,8}{⽰,⽷}
  \begin{Phonetics}{神经}{shen2jing1}[][HSK 5]
    \definition{adj.}{excêntrico; estranho; peculiar; descreve anormalidade neurológica}
    \definition[根,条]{s.}{nervo; um tipo de tecido presente no corpo humano ou animal que conecta o cérebro aos órgãos, transmitindo as sensações ao cérebro e as informações do cérebro aos órgãos}
  \end{Phonetics}
\end{Entry}

\begin{Entry}{神经病学}{9,8,10,8}{⽰,⽷,⽧,⼦}
  \begin{Phonetics}{神经病学}{shen2jing1bing4 xue2}
    \definition{s.}{neurologia}
  \end{Phonetics}
\end{Entry}

\begin{Entry}{神经病的}{9,8,10,8}{⽰,⽷,⽧,⽩}
  \begin{Phonetics}{神经病的}{shen2jing1bing4 de5}
    \definition{adj.}{neuropático; neurótico}
  \end{Phonetics}
\end{Entry}

\begin{Entry}{神话}{9,8}{⽰,⾔}
  \begin{Phonetics}{神话}{shen2hua4}[][HSK 4]
    \definition[段,篇]{s.}{mito; mitologia; conto de fadas; refere"-se a deuses e deusas lendários e histórias de heróis antigos deificados | lorota; refere"-se a alegações ridículas e infundadas}
  \end{Phonetics}
\end{Entry}

\begin{Entry}{神秘}{9,10}{⽰,⽲}
  \begin{Phonetics}{神秘}{shen2mi4}[][HSK 4]
    \definition{adj.}{místico; misterioso}
  \end{Phonetics}
\end{Entry}

\begin{Entry}{神兽}{9,11}{⽰,⼋}
  \begin{Phonetics}{神兽}{shen2shou4}
    \definition{s.}{animal mitológico | fera}
  \end{Phonetics}
\end{Entry}

\begin{Entry}{神情}{9,11}{⽰,⼼}
  \begin{Phonetics}{神情}{shen2qing2}[][HSK 5]
    \definition{s.}{aparência; expressão; atividades internas reveladas no rosto das pessoas}
  \end{Phonetics}
\end{Entry}

\begin{Entry}{神器}{9,16}{⽰,⼝}
  \begin{Phonetics}{神器}{shen2qi4}
    \definition{s.}{objeto mágico | objeto simbólico do poder imperial | arma fina | ferramenta muito útil}
  \end{Phonetics}
\end{Entry}

%%%%%%%%%% 秋 %%%%%%%%%%
\subsection*{秋}\addcontentsline{loh}{figure}{秋}

\begin{Entry}{秋}{9}{⽲}
  \begin{Phonetics}{秋}{qiu1}
    \definition*{s.}{Sobrenome: Qiu}
    \definition{s.}{outono | época da colheita; a estação em que as colheitas amadurecem; colheitas maduras no outono | ano; refere"-se a um ano | um período de tempo (geralmente conturbado)}
  \end{Phonetics}
\end{Entry}

\begin{Entry}{秋天}{9,4}{⽲,⼤}
  \begin{Phonetics}{秋天}{qiu1tian1}[][HSK 2]
    \definition[个,段,季,番]{s.}{outono}
  \end{Phonetics}
\end{Entry}

\begin{Entry}{秋季}{9,8}{⽲,⼦}
  \begin{Phonetics}{秋季}{qiu1ji4}[][HSK 4]
    \definition[个]{s.}{outono; terceiro trimestre do ano, segundo o costume chinês, refere"-se ao período de três meses entre o outono e o inverno, também se refere aos sétimo, oitavo e nono meses do calendário lunar}
  \end{Phonetics}
\end{Entry}

%%%%%%%%%% 种 %%%%%%%%%%
\subsection*{种}\addcontentsline{loh}{figure}{种}

\begin{Entry}{种}{9}{⽲}
  \begin{Phonetics}{种}{zhong3}[][HSK 3]
    \definition{clas.}{indica tipo, usado para pessoas e qualquer coisa}
    \definition{s.}{espécie | etnia | semente; estirpe; linhagem; material para reprodução biológica em cadeia | coragem; determinação; garra; força de caráter; refere"-se à coragem ou determinação}
  \end{Phonetics}
  \begin{Phonetics}{种}{zhong4}[][HSK 4]
    \definition{v.}{semear; cultivar; plantar}
  \end{Phonetics}
\end{Entry}

\begin{Entry}{种子}{9,3}{⽲,⼦}
  \begin{Phonetics}{种子}{zhong3zi5}[][HSK 3]
    \definition[颗,粒,个,号]{s.}{semente; um órgão exclusivo de certas plantas, geralmente composto de três partes: tegumento, embrião e endosperma, as sementes podem germinar e se tornar novas plantas sob certas condições | jogador classificado; durante a competição, nas eliminatórias, os jogadores mais fortes de cada equipe são escalados}
  \end{Phonetics}
\end{Entry}

\begin{Entry}{种地}{9,6}{⽲,⼟}
  \begin{Phonetics}{种地}{zhong4di4}
    \definition{v.}{cultivar | trabalhar a terra}
  \end{Phonetics}
\end{Entry}

\begin{Entry}{种种}{9,9}{⽲,⽲}
  \begin{Phonetics}{种种}{zhong3zhong3}[][HSK 6]
    \definition{adj.}{todos os tipos de; uma variedade de}
  \end{Phonetics}
\end{Entry}

\begin{Entry}{种类}{9,9}{⽲,⽶}
  \begin{Phonetics}{种类}{zhong3lei4}[][HSK 4]
    \definition[个]{s.}{espécie; classe; tipo; variedade; categoria; classificação de alguma coisa de acordo com sua natureza e características}
  \end{Phonetics}
\end{Entry}

\begin{Entry}{种族灭绝}{9,11,5,9}{⽲,⽅,⽕,⽷}
  \begin{Phonetics}{种族灭绝}{zhong3zu2mie4jue2}
    \definition{s.}{genocídio | extinção étnica}
  \end{Phonetics}
\end{Entry}

\begin{Entry}{种麻}{9,11}{⽲,⿇}
  \begin{Phonetics}{种麻}{zhong3ma2}
    \definition{s.}{planta de cânhamo (feminina)}
  \end{Phonetics}
\end{Entry}

\begin{Entry}{种植}{9,12}{⽲,⽊}
  \begin{Phonetics}{种植}{zhong4zhi2}[][HSK 4]
    \definition{v.}{plantar; crescer; cultivar; enterrar as sementes de uma planta no solo; plantar as mudas de uma planta no solo}
  \end{Phonetics}
\end{Entry}

\begin{Entry}{种薯}{9,16}{⽲,⾋}
  \begin{Phonetics}{种薯}{zhong3shu3}
    \definition{s.}{tubérculo semente}
  \end{Phonetics}
\end{Entry}

%%%%%%%%%% 科 %%%%%%%%%%
\subsection*{科}\addcontentsline{loh}{figure}{科}

\begin{Entry}{科}{9}{⽲}
  \begin{Phonetics}{科}{ke1}[][HSK 2]
    \definition*{s.}{Sobrenome: Ke}
    \definition{s.}{um ramo de estudo acadêmico ou profissional |uma divisão ou subdivisão de uma unidade administrativa | família | instruções de palco no drama chinês clássico; nos roteiros de peças clássicas, termos usados para indicar as ações dos personagens | nível; classificação; categoria | sessão de exames; refere"-se às disciplinas, notas e anos das provas para a seleção de candidatos a cargos públicos militares e civis na antiguidade | tecnológico | assunto | lei; regulamento; decreto | penalidade; pena; punição | treinamento profissional ou formal; curso profissionalizante}
    \definition{v.}{proferir uma sentença (penal)}
  \end{Phonetics}
\end{Entry}

\begin{Entry}{科幻}{9,4}{⽲,⼳}
  \begin{Phonetics}{科幻}{ke1huan4}[][HSK 7-9]
    \definition[个]{s.}{ficção científica}
  \end{Phonetics}
\end{Entry}

\begin{Entry}{科目}{9,5}{⽲,⽬}
  \begin{Phonetics}{科目}{ke1mu4}[][HSK 7-9]
    \definition[个]{s.}{curso; disciplina (em um currículo); categorias como disciplinas acadêmicas, classificadas de acordo com suas diferentes naturezas | cabeçalhos em um livro de contas; registros contábeis}
  \end{Phonetics}
\end{Entry}

\begin{Entry}{科技}{9,7}{⽲,⼿}
  \begin{Phonetics}{科技}{ke1ji4}[][HSK 3]
    \definition{s.}{ciência e tecnologia}
  \end{Phonetics}
\end{Entry}

\begin{Entry}{科学}{9,8}{⽲,⼦}
  \begin{Phonetics}{科学}{ke1xue2}[][HSK 2]
    \definition{adj.}{científico; em conformidade com as leis da ciência}
    \definition[门,个,种]{s.}{ciência; um conjunto de conhecimentos que reflete as leis objetivas da natureza, da sociedade, do pensamento, etc.}
  \end{Phonetics}
\end{Entry}

\begin{Entry}{科学家}{9,8,10}{⽲,⼦,⼧}
  \begin{Phonetics}{科学家}{ke1xue2jia1}
    \definition[位,名,个]{s.}{cientista; pessoas com realizações significativas no campo da pesquisa científica}
  \end{Phonetics}
\end{Entry}

\begin{Entry}{科研}{9,9}{⽲,⽯}
  \begin{Phonetics}{科研}{ke1yan2}[][HSK 6]
    \definition{s.}{pesquisa científica}
    \definition{v.}{envolver-se em pesquisa científica}
  \end{Phonetics}
\end{Entry}

\begin{Entry}{科普}{9,12}{⽲,⽇}
  \begin{Phonetics}{科普}{ke1pu3}[][HSK 7-9]
    \definition{v.}{popularizar a ciência}
  \end{Phonetics}
\end{Entry}

%%%%%%%%%% 秒 %%%%%%%%%%
\subsection*{秒}\addcontentsline{loh}{figure}{秒}

\begin{Entry}{秒}{9}{⽲}
  \begin{Phonetics}{秒}{miao3}[][HSK 5]
    \definition{adv.}{instantaneamente}
    \definition{s.}{segundo (unidade de tempo) | segundo (unidade de medida angular)}
  \end{Phonetics}
\end{Entry}

%%%%%%%%%% 穿 %%%%%%%%%%
\subsection*{穿}\addcontentsline{loh}{figure}{穿}

\begin{Entry}{穿}{9}{⽳}
  \begin{Phonetics}{穿}{chuan1}[][HSK 1]
    \definition{adj.}{direto; através; usado após certos verbos, indica um estado de revelação completa}
    \definition{s.}{vestuário; roupas; refere"-se a roupas, sapatos, meias, etc.}
    \definition{v.}{usar; vestir; estar vestido; ter\dots vestido;  vestir roupas, sapatos, meias, etc. | perfurar através de; penetrar; formar orifícios por meio de cinzéis, brocas ou pontas afiadas | enfiar; amarrar; usar cordas e fios para ligar coisas | passar por; atravessar; passar por; através de (buracos, fendas, espaços vazios, etc.)}
  \end{Phonetics}
\end{Entry}

\begin{Entry}{穿上}{9,3}{⽳,⼀}
  \begin{Phonetics}{穿上}{chuan1shang5}[][HSK 4]
    \definition{v.}{vestir (roupas, etc.); colocar roupas}
  \end{Phonetics}
\end{Entry}

\begin{Entry}{穿小鞋}{9,3,15}{⽳,⼩,⾰}
  \begin{Phonetics}{穿小鞋}{chuan1 xiao3xie2}[][HSK 7-9]
    \definition{v.}{dar a alguém sapatos apertados para usar; dificultar as coisas para alguém abusando do seu poder; metáfora para dificultar secretamente as coisas para alguém ou impor restrições ou obstáculos a essa pessoa}
  \end{Phonetics}
\end{Entry}

\begin{Entry}{穿过}{9,6}{⽳,⾡}
  \begin{Phonetics}{穿过}{chuan1guo4}[][HSK 7-9]
    \definition{v.}{atravessar; penetrar; passar; passar por cima}
  \end{Phonetics}
\end{Entry}

\begin{Entry}{穿着}{9,11}{⽳,⽬}
  \begin{Phonetics}{穿着}{chuan1zhuo2}[][HSK 7-9]
    \definition[次]{s.}{vestido; vestuário; o que alguém veste; o efeito geral das roupas e decorações que as pessoas usam}
  \end{Phonetics}
\end{Entry}

\begin{Entry}{穿越}{9,12}{⽳,⾛}
  \begin{Phonetics}{穿越}{chuan1yue4}[][HSK 7-9]
    \definition{v.}{passar através de; cortar através de; atravessar; passar por cima}
  \end{Phonetics}
\end{Entry}

%%%%%%%%%% 突 %%%%%%%%%%
\subsection*{突}\addcontentsline{loh}{figure}{突}

\begin{Entry}{突}{9}{⽳}
  \begin{Phonetics}{突}{tu1}
    \definition{adv.}{de repente; abruptamente; inesperadamente}
    \definition{s.}{chaminé}
    \definition{v.}{avançar rapidamente; atacar | projetar; destacar-se | romper | projetar-se; inchar; fazer bojo}
  \end{Phonetics}
\end{Entry}

\begin{Entry}{突出}{9,5}{⽳,⼐}
  \begin{Phonetics}{突出}{tu1/chu1}[][HSK 3]
    \definition{adj.}{proeminente; excelente; mais que a média}
    \definition{v.+compl.}{romper | enfatizar; destacar; dar destaque a | sobressair; projetar-se; destacar-se}
  \end{Phonetics}
\end{Entry}

\begin{Entry}{突破}{9,10}{⽳,⽯}
  \begin{Phonetics}{突破}{tu1/po4}[][HSK 5]
    \definition{v.+compl.}{romper; fazer uma descoberta revolucionária; concentrar esforços em um único ponto de ataque, reunir o sucesso | quebrar (limite); superar (dificuldade); superar dificuldades; ultrapassar os números ou limites anteriores, superar recordes anteriores, etc.; romper com as limitações e restrições anteriores}
  \end{Phonetics}
\end{Entry}

\begin{Entry}{突然}{9,12}{⽳,⽕}
  \begin{Phonetics}{突然}{tu1ran2}[][HSK 3]
    \definition{adj.}{repentino; abrupto; inesperado}
    \definition{adv.}{de repente; abruptamente; inesperadamente; subitamente}
  \end{Phonetics}
\end{Entry}

%%%%%%%%%% 窃 %%%%%%%%%%
\subsection*{窃}\addcontentsline{loh}{figure}{窃}

\begin{Entry}{窃}{9}{⽳}
  \begin{Phonetics}{窃}{qie4}
    \definition{adv.}{secretamente; sorrateiramente; furtivamente | usado antes de verbos para demonstrar modéstia, frequentemente significando ``Eu humildemente penso\dots'' ou ``Eu acredito em particular\dots''}[臣窃谓此策虽妙,实难实行。===Acredito humildemente que, embora essa estratégia seja engenhosa, na realidade é difícil de implementar.]
    \definition{pron.}{Literário: (referindo"-se às próprias opiniões) meu; minha}
    \definition{v.}{roubar; furtar | apoderar"-se ou ocupar ilegitimamente; tomar posse sem direito}
  \end{Phonetics}
\end{Entry}

\begin{Entry}{窃取}{9,8}{⽳,⼜}
  \begin{Phonetics}{窃取}{qie4qu3}[][HSK 7-9]
    \definition{v.}{usurpar; apoderar-se; roubar (frequentemente usado metaforicamente)}
  \end{Phonetics}
\end{Entry}

%%%%%%%%%% 竖 %%%%%%%%%%
\subsection*{竖}\addcontentsline{loh}{figure}{竖}

\begin{Entry}{竖}{9}{⽴}
  \begin{Phonetics}{竖}{shu4}[][HSK 7-9]
    \definition*{s.}{Sobrenome: Shu}
    \definition{adj.}{vertical; ereto; perpendicular ao solo}
    \definition{s.}{traço vertical (em caracteres chineses) | empregados domésticos; jovens criados}
    \definition{v.}{colocar em pé; erguer; ficar de pé; colocar o objeto perpendicular ao solo}
  \antonymref{横}{heng2}
  \end{Phonetics}
\end{Entry}

\begin{Entry}{竖向}{9,6}{⽴,⼝}
  \begin{Phonetics}{竖向}{shu4xiang4}
    \definition{adj.}{vertical}
  \end{Phonetics}
\end{Entry}

%%%%%%%%%% 类 %%%%%%%%%%
\subsection*{类}\addcontentsline{loh}{figure}{类}

\begin{Entry}{类}{9}{⽶}
  \begin{Phonetics}{类}{lei4}[][HSK 3]
    \definition*{s.}{Sobrenome: Lei}
    \definition{clas.}{tipo; espécie; categoria usada para pessoas ou coisas}
    \definition{s.}{classe; categoria; tipo; variedade; a combinação de muitas coisas semelhantes ou iguais}
    \definition{v.}{assemelhar-se a; ser semelhante a}
  \end{Phonetics}
\end{Entry}

\begin{Entry}{类似}{9,6}{⽶,⼈}
  \begin{Phonetics}{类似}{lei4si4}[][HSK 3]
    \definition{adj.}{semelhante; análogo}
  \end{Phonetics}
\end{Entry}

\begin{Entry}{类别}{9,7}{⽶,⼑}
  \begin{Phonetics}{类别}{lei4bie2}[][HSK 7-9]
    \definition[个,种]{s.}{categoria; classificação; diferentes tipos de coisas, ideias, etc.}
  \end{Phonetics}
\end{Entry}

\begin{Entry}{类型}{9,9}{⽶,⼟}
  \begin{Phonetics}{类型}{lei4xing2}[][HSK 4]
    \definition[种,个]{s.}{tipo; espécie; categoria; tipos formados por coisas com características comuns}
  \end{Phonetics}
\end{Entry}

%%%%%%%%%% 绑 %%%%%%%%%%
\subsection*{绑}\addcontentsline{loh}{figure}{绑}

\begin{Entry}{绑}{9}{⽷}
  \begin{Phonetics}{绑}{bang3}[][HSK 7-9]
    \definition{v.}{amarrar; atar | enrolar ou amarrar com corda}
  \end{Phonetics}
\end{Entry}

\begin{Entry}{绑架}{9,9}{⽷,⽊}
  \begin{Phonetics}{绑架}{bang3jia4}[][HSK 7-9]
    \definition{v.}{sequestrar; abduzir | amarrar; atar}
  \end{Phonetics}
\end{Entry}

%%%%%%%%%% 结 %%%%%%%%%%
\subsection*{结}\addcontentsline{loh}{figure}{结}

\begin{Entry}{结}{9}{⽷}
  \begin{Phonetics}{结}{jie1}[][HSK 7-9]
    \definition{v.}{dar (frutos); formar (sementes); produzir frutos ou sementes (uma planta)}
  \end{Phonetics}
  \begin{Phonetics}{结}{jie2}[][HSK 4]
    \definition*{s.}{Sobrenome: Jie}
    \definition{s.}{nó | declaração juramentada; garantia por escrito; documento que, antigamente, significava um reconhecimento de encerramento ou uma garantia de responsabilidade}
    \definition{v.}{amarrar; tricotar; dar nó; tecer | formar; forjar; cimentar; solidificar | resolver; concluir | combinar; formar um relacionamento}
  \end{Phonetics}
\end{Entry}

\begin{Entry}{结冰}{9,6}{⽷,⼎}
  \begin{Phonetics}{结冰}{jie2bing1}[][HSK 7-9]
    \definition{v.}{congelar; cobrir com gelo}
  \end{Phonetics}
\end{Entry}

\begin{Entry}{结合}{9,6}{⽷,⼝}
  \begin{Phonetics}{结合}{jie2he2}[][HSK 3]
    \definition{v.}{ligar; unir; combinar; integrar; formar uma relação estreita entre pessoas ou coisas | casar-se; unir-se em matrimônio; referir-se especificamente a casais}
  \end{Phonetics}
\end{Entry}

\begin{Entry}{结论}{9,6}{⽷,⾔}
  \begin{Phonetics}{结论}{jie2lun4}[][HSK 4]
    \definition[个]{s.}{conclusão; palavra final sobre uma pessoa ou coisa após investigação e pesquisa | veredito; julgamento deduzido de premissas também é chamado de conclusão}
  \end{Phonetics}
\end{Entry}

\begin{Entry}{结尾}{9,7}{⽷,⼫}
  \begin{Phonetics}{结尾}{jie2wei3}[][HSK 7-9]
    \definition{s.}{final; fase de encerramento; fim de fase ou parte}
    \definition{v.}{encerrar; concluir a etapa final}
  \end{Phonetics}
\end{Entry}

\begin{Entry}{结局}{9,7}{⽷,⼫}
  \begin{Phonetics}{结局}{jie2ju2}[][HSK 7-9]
    \definition[个]{s.}{final; desfecho; resultado final; resultado final; conclusão}
  \end{Phonetics}
\end{Entry}

\begin{Entry}{结束}{9,7}{⽷,⽊}
  \begin{Phonetics}{结束}{jie2shu4}[][HSK 3]
    \definition{v.}{finalizar; fechar; terminar; concluir; encerrar; desenvolver ou avançar até a fase final, sem continuidade}
  \end{Phonetics}
\end{Entry}

\begin{Entry}{结束工作}{9,7,3,7}{⽷,⽊,⼯,⼈}
  \begin{Phonetics}{结束工作}{jie2shu4gong1zuo4}
    \definition{s.}{trabalho final}
    \definition{v.}{terminar o trabalho}
  \end{Phonetics}
\end{Entry}

\begin{Entry}{结束区}{9,7,4}{⽷,⽊,⼖}
  \begin{Phonetics}{结束区}{jie2shu4 qu1}
    \definition{s.}{zona final}
  \end{Phonetics}
\end{Entry}

\begin{Entry}{结束文本}{9,7,4,5}{⽷,⽊,⽂,⽊}
  \begin{Phonetics}{结束文本}{jie2shu4 wen2ben3}
    \definition{s.}{texto final}
  \end{Phonetics}
\end{Entry}

\begin{Entry}{结束剂}{9,7,8}{⽷,⽊,⼑}
  \begin{Phonetics}{结束剂}{jie2shu4 ji4}
    \definition{s.}{finalizador}
  \end{Phonetics}
\end{Entry}

\begin{Entry}{结束语}{9,7,9}{⽷,⽊,⾔}
  \begin{Phonetics}{结束语}{jie2shu4yu3}
    \definition{s.}{conclusões finais | considerações finais}
  \end{Phonetics}
\end{Entry}

\begin{Entry}{结束辩论}{9,7,16,6}{⽷,⽊,⾟,⾔}
  \begin{Phonetics}{结束辩论}{jie2shu4 bian4 lun4}
    \definition{s.}{debate de encerramento}
  \end{Phonetics}
\end{Entry}

\begin{Entry}{结社自由}{9,7,6,5}{⽷,⽰,⾃,⽥}
  \begin{Phonetics}{结社自由}{jie2she4zi4you2}
    \definition{s.}{(constitucional) liberdade de associação}
  \end{Phonetics}
\end{Entry}

\begin{Entry}{结识}{9,7}{⽷,⾔}
  \begin{Phonetics}{结识}{jie2shi2}[][HSK 7-9]
    \definition{v.}{conhecer alguém; fazer amizade com; familiarizar-se com alguém}
  \end{Phonetics}
\end{Entry}

\begin{Entry}{结实}{9,8}{⽷,⼧}
  \begin{Phonetics}{结实}{jie1shi5}[][HSK 3]
    \definition{adj.}{sólido; resistente; durável | forte; resistente; robusto}
  \end{Phonetics}
\end{Entry}

\begin{Entry}{结构}{9,8}{⽷,⽊}
  \begin{Phonetics}{结构}{jie2gou4}[][HSK 4]
    \definition[个]{s.}{estrutura; composição; construção; formação; constituição; tecido; forma; sistematização; mecânica; organização | arquitetura; estrutura; construção; construção de partes de edifícios com suporte de carga ou com carga externa | Geologia: textura}[这些矿物质具有致密结构。===Esses minerais têm uma estrutura densa.]
  \end{Phonetics}
\end{Entry}

\begin{Entry}{结果}{9,8}{⽷,⽊}
  \begin{Phonetics}{结果}{jie1/guo3}[][HSK 7-9]
    \definition{v.+compl.}{frutificar; dar frutos}
  \end{Phonetics}
  \begin{Phonetics}{结果}{jie2/guo3}[][HSK 2]
    \definition{conj.}{como resultado; no final}
    \definition{v.}{despachar | matar}
    \definition{v.+compl.}{resultado; conclusão; consequência}
  \end{Phonetics}
\end{Entry}

\begin{Entry}{结婚}{9,11}{⽷,⼥}
  \begin{Phonetics}{结婚}{jie2/hun1}[][HSK 3]
    \definition{v.+compl.}{casar; casar"-se; casar"-se bem}
  \end{Phonetics}
\end{Entry}

\begin{Entry}{结婚礼服}{9,11,5,8}{⽷,⼥,⽰,⽉}
  \begin{Phonetics}{结婚礼服}{jie2hun1 li3 fu2}
    \definition{s.}{vestido de casamento; vestido de noiva}
  \end{Phonetics}
\end{Entry}

\begin{Entry}{结晶}{9,12}{⽷,⽇}
  \begin{Phonetics}{结晶}{jie2jing1}[][HSK 7-9]
    \definition{s.}{cristal; substâncias cristalinas | Figurativo: os frutos (do trabalho ou da atividade); metáfora para conquistas preciosas}
    \definition{v.}{cristalizar; as substâncias podem formar cristais a partir de um estado líquido (solução ou fundido) ou gasoso}
  \end{Phonetics}
\end{Entry}

%%%%%%%%%% 绕 %%%%%%%%%%
\subsection*{绕}\addcontentsline{loh}{figure}{绕}

\begin{Entry}{绕}{9}{⽷}
  \begin{Phonetics}{绕}{rao4}[][HSK 5]
    \definition*{s.}{Sobrenome: Rao}
    \definition{v.}{enrolar; bobinar; rebobinar | mover-se em círculo; girar; revolver | fazer um desvio; contornar; dar a volta | confundir; desorientar}
  \end{Phonetics}
\end{Entry}

\begin{Entry}{绕行}{9,6}{⽷,⾏}
  \begin{Phonetics}{绕行}{rao4xing2}[][HSK 7-9]
    \definition{v.}{desviar; contornar | mover-se em círculo; circular}
  \end{Phonetics}
\end{Entry}

%%%%%%%%%% 绘 %%%%%%%%%%
\subsection*{绘}\addcontentsline{loh}{figure}{绘}

\begin{Entry}{绘}{9}{⽷}
  \begin{Phonetics}{绘}{hui4}
    \definition{v.}{pintar; desenhar}
  \end{Phonetics}
\end{Entry}

\begin{Entry}{绘声绘色}{9,7,9,6}{⽷,⼠,⽷,⾊}
  \begin{Phonetics}{绘声绘色}{hui4sheng1-hui4se4}[][HSK 7-9]
    \definition{expr.}{vívido e colorido; uma descrição animada; vívido; animado}
  \end{Phonetics}
\end{Entry}

\begin{Entry}{绘画}{9,8}{⽷,⽥}
  \begin{Phonetics}{绘画}{hui4hua4}[][HSK 6]
    \definition{s.}{desenho; pintura}
    \definition{v.}{desenhar; pintar}
  \end{Phonetics}
\end{Entry}

%%%%%%%%%% 给 %%%%%%%%%%
\subsection*{给}\addcontentsline{loh}{figure}{给}

\begin{Entry}{给}{9}{⽷}
  \begin{Phonetics}{给}{gei3}[][HSK 1]
    \definition{prep.}{por; expressa significado passivo; tem o mesmo significado que 被, 叫; pode ser seguido pelo agente da ação; o agente da ação também pode não aparecer na frase | para; a; seguido por quem se beneficia da ação; igual a 为 | em direção a; seguido pelo destinatário da ação; o mesmo que 向 | indica transmissão}
    \definition{v.}{dar; conceder; fazer com que a outra parte obtenha algo | passar; pagar; indicar que a outra pessoa faça algo | deixar; permitir que alguém faça algo; autorizar alguém a fazer algo}
    \definition{v.aux.}{usado antes de verbos predicativos que expressam passividade, disposição, etc., para reforçar o tom}
  \seealsoref{被}{bei4}
  \seealsoref{叫}{jiao4}
  \seealsoref{为}{wei4}
  \seealsoref{向}{xiang4}
  \end{Phonetics}
  \begin{Phonetics}{给}{ji3}
    \definition{adj.}{abundante; próspero; bem provido para}
    \definition{v.}{fornecer; prover}
  \end{Phonetics}
\end{Entry}

\begin{Entry}{给予}{9,4}{⽷,⼅}
  \begin{Phonetics}{给予}{ji3yu3}[][HSK 6]
    \definition{v.}{dar; conceder; dar em troca}
  \end{Phonetics}
\end{Entry}

\begin{Entry}{给……打电话}{9,5,5,8}{⽷,⼿,⽥,⾔}
  \begin{Phonetics}{给……打电话}{gei3 da3 dian4hua4}
    \definition{expr.}{dar um telefonema para alguém}
  \seealsoref{打电话}{da3 dian4hua4}
  \end{Phonetics}
\end{Entry}

\begin{Entry}{给……定向}{9,8,6}{⽷,⼧,⼝}
  \begin{Phonetics}{给……定向}{gei3 ding4xiang4}
    \definition{v.}{dar orientação para algo; orientar algo}
  \end{Phonetics}
\end{Entry}

%%%%%%%%%% 络 %%%%%%%%%%
\subsection*{络}\addcontentsline{loh}{figure}{络}

\begin{Entry}{络}{9}{⽷}
  \begin{Phonetics}{络}{lao4}
    \definition{s.}{rede; saco de malha}
  \end{Phonetics}
  \begin{Phonetics}{络}{luo4}
    \definition{s.}{algo semelhante a uma rede; malha | canais subsidiários no corpo humano por onde circulam energia vital, sangue e nutrientes; na Medicina Tradicional Chinesa, isso se refere aos ramos laterais ou ramificações menores dos canais de circulação do Qi (气) e do sangue no corpo humano}
    \definition{v.}{manter algo no lugar com uma rede; enredar; usar uma rede para prender | entrelaçar; enrolar; emaranhar}
  \seealsoref{气}{qi4}
  \end{Phonetics}
\end{Entry}

\begin{Entry}{络绎不绝}{9,8,4,9}{⽷,⽷,⼀,⽷}
  \begin{Phonetics}{络绎不绝}{luo4yi4-bu4jue2}[][HSK 7-9]
    \definition{expr.}{``Um após o outro.''; ir e vir em fluxos constantes; ir e vir em um fluxo contínuo (sem fim); vir em um fluxo contínuo; vir um após o outro; em uma linha contínua (ininterrupta); em um fluxo sem fim; prosseguir (vir) em um fluxo constante; descreve algo como contínuo ou ininterrupto; geralmente se refere a pedestres, carruagens ou barcos em um rio}
  \end{Phonetics}
\end{Entry}

%%%%%%%%%% 绝 %%%%%%%%%%
\subsection*{绝}\addcontentsline{loh}{figure}{绝}

\begin{Entry}{绝}{9}{⽷}
  \begin{Phonetics}{绝}{jue2}[][HSK 6]
    \definition{adj.}{exausto; esgotado; acabado | desesperado; sem esperança | único; soberbo; incomparável | não deixar margem de manobra; não fazer concessões; intransigente}
    \definition{adv.}{extremamente; mais | (antes de uma negativa) absolutamente; no mínimo; por qualquer meio; em qualquer conta}
    \definition{s.}{(literário) jueju, um poema de quatro linhas}
    \definition{v.}{cortar; romper | parar de respirar; morrer}
  \end{Phonetics}
\end{Entry}

\begin{Entry}{绝大多数}{9,3,6,13}{⽷,⼤,⼣,⽁}
  \begin{Phonetics}{绝大多数}{jue2da4duo1shu4}[][HSK 6]
    \definition{expr.}{maioria absoluta | uma maioria esmagadora}
  \end{Phonetics}
\end{Entry}

\begin{Entry}{绝不}{9,4}{⽷,⼀}
  \begin{Phonetics}{绝不}{jue2bu4}
    \definition{adv.}{definitivamente não | de forma alguma | sob nenhuma circunstância}
  \end{Phonetics}
\end{Entry}

\begin{Entry}{绝对}{9,5}{⽷,⼨}
  \begin{Phonetics}{绝对}{jue2dui4}[][HSK 3]
    \definition{adj.}{absoluto; sem condições; sem restrições | absoluto; extremo; incompleto; sem margem para negociação ou alteração}
    \definition{adv.}{absolutamente; completamente; com certeza}
  \end{Phonetics}
\end{Entry}

\begin{Entry}{绝技}{9,7}{⽷,⼿}
  \begin{Phonetics}{绝技}{jue2ji4}[][HSK 7-9]
    \definition{s.}{habilidade única; habilidade consumada | \emph{tour-de-force}; uma atuação ou conquista impressionante que foi realizada ou gerenciada com grande habilidade | façanha | feito supremo}
  \end{Phonetics}
\end{Entry}

\begin{Entry}{绝招}{9,8}{⽷,⼿}
  \begin{Phonetics}{绝招}{jue2zhao1}[][HSK 7-9]
    \definition{s.}{habilidade única; movimento delicado inesperado (como último recurso); habilidades únicas e magníficas; métodos engenhosos}
  \end{Phonetics}
\end{Entry}

\begin{Entry}{绝版}{9,8}{⽷,⽚}
  \begin{Phonetics}{绝版}{jue2ban3}
    \definition{adj.}{esgotado | fora de catálogo}
  \end{Phonetics}
\end{Entry}

\begin{Entry}{绝望}{9,11}{⽷,⽉}
  \begin{Phonetics}{绝望}{jue2/wang4}[][HSK 5]
    \definition{v.+compl.}{desesperar; desistir de toda esperança; perder toda esperança de}
  \end{Phonetics}
\end{Entry}

\begin{Entry}{绝缘}{9,12}{⽷,⽷}
  \begin{Phonetics}{绝缘}{jue2yuan2}[][HSK 7-9]
    \definition{v.}{isolar; utilizar materiais como borracha ou madeira para bloquear a eletricidade, impedindo sua passagem | ser separado de; ser isolado de; completamente isolado do mundo exterior ou de um objeto específico}
  \end{Phonetics}
\end{Entry}

%%%%%%%%%% 绞 %%%%%%%%%%
\subsection*{绞}\addcontentsline{loh}{figure}{绞}

\begin{Entry}{绞}{9}{⽷}
  \begin{Phonetics}{绞}{jiao3}[][HSK 7-9]
    \definition{clas.}{meada; novelo; utilizado para fios, lã, etc.}
    \definition{v.}{torcer; espremer; emaranhar | dar corda | picar; moer | pendurar pelo pescoço; estrangular | entrelaçar}
  \end{Phonetics}
\end{Entry}

%%%%%%%%%% 统 %%%%%%%%%%
\subsection*{统}\addcontentsline{loh}{figure}{统}

\begin{Entry}{统}{9}{⽷}
  \begin{Phonetics}{统}{tong3}
    \definition{adv.}{todos; juntos; de forma unificada | inteiramente; totalmente}
    \definition{s.}{interligado; inter-relacionado | sistema interconectado | qualquer parte em forma de tubo de uma peça de roupa, etc.; o mesmo que 筒}
    \definition{v.}{reunir em um; unir | unir; liderar; comandar}
  \seealsoref{筒}{tong3}
  \end{Phonetics}
\end{Entry}

\begin{Entry}{统一}{9,1}{⽷,⼀}
  \begin{Phonetics}{统一}{tong3yi1}[][HSK 4]
    \definition{adj.}{unificado; unitário; centralizado; consistente}
    \definition{v.}{unificar; unir; integrar; padronizar}
  \end{Phonetics}
\end{Entry}

\begin{Entry}{统计}{9,4}{⽷,⾔}
  \begin{Phonetics}{统计}{tong3ji4}[][HSK 4]
    \definition{v.}{compilar estatísticas; refere"-se à realização de trabalho estatístico, ou seja, coletar, reunir, analisar e extrapolar dados sobre um fenômeno | somar; adicionar; contar}
  \end{Phonetics}
\end{Entry}

%%%%%%%%%% 缸 %%%%%%%%%%
\subsection*{缸}\addcontentsline{loh}{figure}{缸}

\begin{Entry}{缸}{9}{⽸}
  \begin{Phonetics}{缸}{gang1}[][HSK 7-9]
    \definition[口,个]{s.}{jarra; pote de barro; recipientes feitos de barro, porcelana, vidro, etc. geralmente têm uma boca grande e um fundo pequeno | composto de areia, argila, etc. para fazer louça de barro | vaso em forma de jarro; objetos em forma de potes}
  \end{Phonetics}
\end{Entry}

%%%%%%%%%% 罚 %%%%%%%%%%
\subsection*{罚}\addcontentsline{loh}{figure}{罚}

\begin{Entry}{罚}{9}{⽹}
  \begin{Phonetics}{罚}{fa2}[][HSK 5]
    \definition{s.}{punição; penalidade}
    \definition{v.}{punir; penalizar; multar; confiscar}
  \end{Phonetics}
\end{Entry}

\begin{Entry}{罚款}{9,12}{⽹,⽋}
  \begin{Phonetics}{罚款}{fa2/kuan3}[][HSK 5]
    \definition[笔,次,宗]{s.}{multa; penalidade; refere"-se ao dinheiro pago por uma pessoa ou entidade de acordo com as disposições de um delito ou violação de contrato ou contrato}
    \definition{v.+compl.}{multar; penalizar; exigir, de acordo com os regulamentos, uma determinada quantia de dinheiro de uma pessoa ou entidade que tenha violado a lei ou descumprido um regulamento ou contrato}
  \end{Phonetics}
\end{Entry}

%%%%%%%%%% 美 %%%%%%%%%%
\subsection*{美}\addcontentsline{loh}{figure}{美}

\begin{Entry}{美}{9}{⽺}
  \begin{Phonetics}{美}{mei3}[][HSK 3]
    \definition*{s.}{América, abreviatura de 美洲 | Estados Unidos da América, abreviatura de 美国 | As Américas (美洲)}
    \definition{adj.}{belo; bonito | muito satisfatório; bom; agradável}
    \definition{s.}{beleza}
    \definition{v.}{embelezar; tornar mais bonito | estar satisfeito consigo mesmo; orgulhar-se; sentir-se presunçoso}
  \seealsoref{美国}{mei3guo2}
  \seealsoref{美洲}{mei3zhou1}
  \antonymref{丑}{chou3}
  \end{Phonetics}
\end{Entry}

\begin{Entry}{美人}{9,2}{⽺,⼈}
  \begin{Phonetics}{美人}{mei3ren2}[][HSK 7-9]
    \definition[个,位,名]{s.}{beleza; mulher bonita}
  \end{Phonetics}
\end{Entry}

\begin{Entry}{美女}{9,3}{⽺,⼥}
  \begin{Phonetics}{美女}{mei3nv3}[][HSK 4]
    \definition[个,位,名,些]{s.}{beldade; mulher bonita; uma jovem linda}
  \end{Phonetics}
\end{Entry}

\begin{Entry}{美中不足}{9,4,4,7}{⽺,⼁,⼀,⾜}
  \begin{Phonetics}{美中不足}{mei3zhong1-bu4zu2}[][HSK 7-9]
    \definition{expr.}{belo, porém incompleto (que carece de perfeição); uma imperfeição em algo perfeito; uma falha em algo aparentemente perfeito; um problema; a parte desagradável de algo agradável; algumas imperfeições em algo aparentemente perfeito; alguma pequena falta de perfeição; algo que não atinge a perfeição; há uma falha no ato; é bom, mas ainda tem falhas; um obstáculo; uma falha que prejudica a beleza; um defeito em algo aparentemente perfeito}
  \end{Phonetics}
\end{Entry}

\begin{Entry}{美元}{9,4}{⽺,⼉}
  \begin{Phonetics}{美元}{mei3yuan2}[][HSK 3]
    \definition*[元,笔,沓]{s.}{Dólar Americano; a moeda dos Estados Unidos}
  \end{Phonetics}
\end{Entry}

\begin{Entry}{美化}{9,4}{⽺,⼔}
  \begin{Phonetics}{美化}{mei3hua4}[][HSK 7-9]
    \definition{v.}{embelezar; enfeitar; adornar; estética através da decoração e do embelezamento | embelezar; retratar o mal como bem e o feio como belo}
  \end{Phonetics}
\end{Entry}

\begin{Entry}{美术}{9,5}{⽺,⽊}
  \begin{Phonetics}{美术}{mei3shu4}[][HSK 3]
    \definition[种]{s.}{arte; artes plásticas: arte que ocupa um determinado espaço, compõe imagens estéticas e permite que as pessoas apreciem visualmente, incluindo pintura, escultura, arquitetura, etc. | pintura; pintura tradicional chinesa}
  \end{Phonetics}
\end{Entry}

\begin{Entry}{美甲}{9,5}{⽺,⽥}
  \begin{Phonetics}{美甲}{mei3jia3}
    \definition{s.}{manicure e/ou pedicure}
  \end{Phonetics}
\end{Entry}

\begin{Entry}{美好}{9,6}{⽺,⼥}
  \begin{Phonetics}{美好}{mei3hao3}[][HSK 3]
    \definition{adj.}{bem; feliz; glorioso; descreve a vida, os desejos, etc. como sendo muito bons e satisfatórios}
  \end{Phonetics}
\end{Entry}

\begin{Entry}{美观}{9,6}{⽺,⾒}
  \begin{Phonetics}{美观}{mei3guan1}[][HSK 7-9]
    \definition{adj.}{artístico; belo; agradável à vista; que agrada aos olhos}
  \end{Phonetics}
\end{Entry}

\begin{Entry}{美丽}{9,7}{⽺,⼀}
  \begin{Phonetics}{美丽}{mei3li4}[][HSK 3]
    \definition{adj.}{bonito; lindo; capaz de proporcionar uma sensação de beleza}
  \end{Phonetics}
\end{Entry}

\begin{Entry}{美妙}{9,7}{⽺,⼥}
  \begin{Phonetics}{美妙}{mei3miao4}[][HSK 7-9]
    \definition{adj.}{esplêndido; belo; maravilhoso; descreve obras literárias, música, sentimentos, etc., como belos, únicos ou engenhosos}
  \end{Phonetics}
\end{Entry}

\begin{Entry}{美味}{9,8}{⽺,⼝}
  \begin{Phonetics}{美味}{mei3wei4}[][HSK 7-9]
    \definition[顿]{s.}{comida deliciosa; iguaria (\emph{delicacy}); sabor delicioso}
  \end{Phonetics}
\end{Entry}

\begin{Entry}{美国}{9,8}{⽺,⼞}
  \begin{Phonetics}{美国}{mei3guo2}
    \definition*{s.}{Estados Unidos da América}
  \end{Phonetics}
\end{Entry}

\begin{Entry}{美国人}{9,8,2}{⽺,⼞,⼈}
  \begin{Phonetics}{美国人}{mei3guo2ren2}
    \definition{s.}{americano | pessoa ou povo dos Estados Unidos da América}
  \end{Phonetics}
\end{Entry}

\begin{Entry}{美学}{9,8}{⽺,⼦}
  \begin{Phonetics}{美学}{mei3xue2}
    \definition{s.}{estética; a ciência que estuda as leis e os princípios gerais da beleza na natureza, na sociedade e na arte explora principalmente a natureza da beleza, a relação entre arte e realidade e as leis gerais da criação artística}
  \end{Phonetics}
\end{Entry}

\begin{Entry}{美金}{9,8}{⽺,⾦}
  \begin{Phonetics}{美金}{mei3jin1}[][HSK 4]
    \definition{s.}{USD; dólar americano: a moeda local dos Estados Unidos}
  \end{Phonetics}
\end{Entry}

\begin{Entry}{美洲}{9,9}{⽺,⽔}
  \begin{Phonetics}{美洲}{mei3zhou1}
    \definition*{s.}{América (incluindo Norte, Central e Sul)}
  \end{Phonetics}
\end{Entry}

\begin{Entry}{美洲人}{9,9,2}{⽺,⽔,⼈}
  \begin{Phonetics}{美洲人}{mei3zhou1ren2}
    \definition{s.}{americano | pessoa ou povo do continente Americano}
  \end{Phonetics}
\end{Entry}

\begin{Entry}{美食}{9,9}{⽺,⾷}
  \begin{Phonetics}{美食}{mei3shi2}[][HSK 3]
    \definition[种,道,桌]{s.}{iguaria; (gastronomia) comida saborosa}
  \end{Phonetics}
\end{Entry}

\begin{Entry}{美容}{9,10}{⽺,⼧}
  \begin{Phonetics}{美容}{mei3rong2}[][HSK 6]
    \definition{v.}{embelezar; melhorar a aparência de alguém; deixar seu rosto bonito retocando, cuidando, etc.}
  \end{Phonetics}
\end{Entry}

\begin{Entry}{美景}{9,12}{⽺,⽇}
  \begin{Phonetics}{美景}{mei3jing3}[][HSK 7-9]
    \definition{s.}{paisagem deslumbrante; paisagens belíssimas (como o mar, a terra ou o céu)}
  \end{Phonetics}
\end{Entry}

\begin{Entry}{美滋滋}{9,12,12}{⽺,⽔,⽔}
  \begin{Phonetics}{美滋滋}{mei3zi1zi1}[][HSK 7-9]
    \definition{interj.}{``Me sentindo ótimo!''; exultante; muito feliz; muito satisfeito consigo mesmo}
  \end{Phonetics}
\end{Entry}

\begin{Entry}{美满}{9,13}{⽺,⽔}
  \begin{Phonetics}{美满}{mei3man3}[][HSK 7-9]
    \definition{adj.}{feliz; perfeitamente satisfatório; lindo e perfeito}
  \end{Phonetics}
\end{Entry}

\begin{Entry}{美德}{9,15}{⽺,⼻}
  \begin{Phonetics}{美德}{mei3de2}[][HSK 7-9]
    \definition*{s.}{My Duc District (Hanoi)}
    \definition[种,些]{s.}{virtude; excelência moral; bom caráter}
  \end{Phonetics}
\end{Entry}

%%%%%%%%%% 耍 %%%%%%%%%%
\subsection*{耍}\addcontentsline{loh}{figure}{耍}

\begin{Entry}{耍}{9}{⽽}
  \begin{Phonetics}{耍}{shua3}[][HSK 7-9]
    \definition*{s.}{Sobrenome: Shua}
    \definition{v.}{brincar (com truques); se divertir | brincar com; agitar; provocar}
  \end{Phonetics}
\end{Entry}

\begin{Entry}{耍赖}{9,13}{⽽,⾙}
  \begin{Phonetics}{耍赖}{shua3lai4}[][HSK 7-9]
    \definition{v.}{agir sem vergonha; trapacear; agir desonestamente; comportar-se de forma teimosa; usar métodos inescrupulosos}
  \end{Phonetics}
\end{Entry}

%%%%%%%%%% 耐 %%%%%%%%%%
\subsection*{耐}\addcontentsline{loh}{figure}{耐}

\begin{Entry}{耐}{9}{⽽}
  \begin{Phonetics}{耐}{nai4}[][HSK 7-9]
    \definition{v.}{ser capaz de suportar (ou tolerar); poder resistir; poder suportar | suportar; aguentar; resistir}
  \end{Phonetics}
\end{Entry}

\begin{Entry}{耐人寻味}{9,2,6,8}{⽽,⼈,⼨,⼝}
  \begin{Phonetics}{耐人寻味}{nai4ren2xun2wei4}[][HSK 7-9]
    \definition{expr.}{intrigante; instigante; é algo profundo e que merece uma reflexão cuidadosa}
  \end{Phonetics}
\end{Entry}

\begin{Entry}{耐心}{9,4}{⽽,⼼}
  \begin{Phonetics}{耐心}{nai4xin1}[][HSK 5]
    \definition{adj.}{paciente}
    \definition[些]{s.}{paciência; uma pessoa que não se importa com problemas e é paciente}
  \end{Phonetics}
\end{Entry}

\begin{Entry}{耐性}{9,8}{⽽,⼼}
  \begin{Phonetics}{耐性}{nai4xing4}[][HSK 7-9]
    \definition{s.}{paciência; resistência | tolerância; uma personalidade paciente e sem pressa}
  \end{Phonetics}
\end{Entry}

%%%%%%%%%% 胃 %%%%%%%%%%
\subsection*{胃}\addcontentsline{loh}{figure}{胃}

\begin{Entry}{胃}{9}{⾁}
  \begin{Phonetics}{胃}{wei4}[][HSK 5]
    \definition*{s.}{Wei, uma das mansões lunares | Wei, uma das vinte e oito constelações}
    \definition{s.}{estômago; parte do aparelho digestivo}
  \end{Phonetics}
\end{Entry}

\begin{Entry}{胃口}{9,3}{⾁,⼝}
  \begin{Phonetics}{胃口}{wei4kou3}
    \definition{s.}{apetite}
  \end{Phonetics}
\end{Entry}

%%%%%%%%%% 胆 %%%%%%%%%%
\subsection*{胆}\addcontentsline{loh}{figure}{胆}

\begin{Entry}{胆}{9}{⾁}
  \begin{Phonetics}{胆}{dan3}[][HSK 5]
    \definition[个,颗]{s.}{vesícula biliar | coragem; bravura | um recipiente interno semelhante a uma bexiga; algo que se encaixa dentro de um objeto e pode conter água, ar, etc.}
  \end{Phonetics}
\end{Entry}

\begin{Entry}{胆子}{9,3}{⾁,⼦}
  \begin{Phonetics}{胆子}{dan3zi5}[][HSK 7-9]
    \definition{s.}{coragem}
  \end{Phonetics}
\end{Entry}

\begin{Entry}{胆小}{9,3}{⾁,⼩}
  \begin{Phonetics}{胆小}{dan3xiao3}[][HSK 5]
    \definition{adj.}{tímido; covarde}
  \end{Phonetics}
\end{Entry}

\begin{Entry}{胆小鬼}{9,3,9}{⾁,⼩,⿁}
  \begin{Phonetics}{胆小鬼}{dan3xiao3gui3}
    \definition{adj.}{covarde | medroso}
  \end{Phonetics}
\end{Entry}

\begin{Entry}{胆怯}{9,8}{⾁,⼼}
  \begin{Phonetics}{胆怯}{dan3qie4}[][HSK 7-9]
    \definition{v.+compl.}{tímido; covarde; descreve a aparência ou sentimento de alguém que tem muito medo de fazer algo}
  \end{Phonetics}
\end{Entry}

%%%%%%%%%% 背 %%%%%%%%%%
\subsection*{背}\addcontentsline{loh}{figure}{背}

\begin{Entry}{背}{9}{⾁}
  \begin{Phonetics}{背}{bei1}[][HSK 3]
    \definition{clas.}{carga; pacote; para transportar coisas nas costas}
    \definition{v.}{carregar nas costas | suportar; carregar}
  \end{Phonetics}
  \begin{Phonetics}{背}{bei4}[][HSK 2]
    \definition{adj.}{azarado | fora do caminho; um lugar muito distante do centro movimentado, onde poucas pessoas aparecem | deficiente auditivo}
    \definition{s.}{parte posterior do corpo; costas; coluna vertebral; parte do tronco entre os ombros e a região lombar | parte de trás de um objeto}
    \definition{v.}{afastar"-se; virar as costas | decorar; memorizar; recitar de memória | esconder algo de; fazer algo em segredo | sair, ir embora; partir; abandonar | quebrar; violar; agir de forma contrária a}
  \end{Phonetics}
\end{Entry}

\begin{Entry}{背心}{9,4}{⾁,⼼}
  \begin{Phonetics}{背心}{bei4xin1}[][HSK 6]
    \definition[件]{s.}{colete; vestimenta sem mangas; \emph{tops} sem gola e sem mangas}
  \end{Phonetics}
\end{Entry}

\begin{Entry}{背包}{9,5}{⾁,⼓}
  \begin{Phonetics}{背包}{bei1bao1}[][HSK 5]
    \definition[个,只,款]{s.}{mochila; mochila de ataque; mochila de infantaria; pacotes de roupas carregados nas costas quando marcham}
  \end{Phonetics}
\end{Entry}

\begin{Entry}{背后}{9,6}{⾁,⼝}
  \begin{Phonetics}{背后}{bei4hou4}[][HSK 3]
    \definition{s.}{parte posterior; parte de trás; traseira | pelas costas de alguém}
  \end{Phonetics}
\end{Entry}

\begin{Entry}{背叛}{9,9}{⾁,⼜}
  \begin{Phonetics}{背叛}{bei4pan4}[][HSK 7-9]
    \definition{s.}{traição; deslealdade; refere"-se ao ato ou evento de traição}
    \definition{v.}{trair; refere"-se à traição, rebelião e atos que violam a moralidade e traem a confiança}
  \end{Phonetics}
\end{Entry}

\begin{Entry}{背诵}{9,9}{⾁,⾔}
  \begin{Phonetics}{背诵}{bei4song4}[][HSK 7-9]
    \definition{v.}{recitar; repetir de memória; ler de memória o texto ou as frases que você leu}
  \end{Phonetics}
\end{Entry}

\begin{Entry}{背面}{9,9}{⾁,⾯}
  \begin{Phonetics}{背面}{bei4mian4}[][HSK 7-9]
    \definition{s.}{costas; lado reverso; lado avesso | o verso; o reverso; o avesso}
  \antonymref{正面}{zheng4mian4}
  \end{Phonetics}
\end{Entry}

\begin{Entry}{背着}{9,11}{⾁,⽬}
  \begin{Phonetics}{背着}{bei4zhe5}[][HSK 6]
    \definition{adv.}{pelas costas; atrás de alguém}
    \definition{v.}{carregar nas costas}
  \end{Phonetics}
\end{Entry}

\begin{Entry}{背景}{9,12}{⾁,⽇}
  \begin{Phonetics}{背景}{bei4jing3}[][HSK 4]
    \definition[种]{s.}{pano de fundo; fundo; cenário de teatro, filme ou drama de TV | fundo; cenário que permeia a imagem principal na tela | condições sociais; ambientes históricos (significativamente influentes para algo ou alguém) | poder que dá suporte a alguém}
  \end{Phonetics}
\end{Entry}

%%%%%%%%%% 胖 %%%%%%%%%%
\subsection*{胖}\addcontentsline{loh}{figure}{胖}

\begin{Entry}{胖}{9}{⾁}
  \begin{Phonetics}{胖}{pan2}
    \definition{adj.}{saudável}
  \end{Phonetics}
  \begin{Phonetics}{胖}{pang4}[][HSK 3]
    \definition{adj.}{gordo; robusto; rechonchudo; (corpo humano) com muita gordura ou carne}
  \antonymref{瘦}{shou4}
  \end{Phonetics}
\end{Entry}

\begin{Entry}{胖子}{9,3}{⾁,⼦}
  \begin{Phonetics}{胖子}{pang4zi5}[][HSK 4]
    \definition[个]{s.}{obeso; gordo; pessoa gorda}
  \end{Phonetics}
\end{Entry}

%%%%%%%%%% 胚 %%%%%%%%%%
\subsection*{胚}\addcontentsline{loh}{figure}{胚}

\begin{Entry}{胚}{9}{⾁}
  \begin{Phonetics}{胚}{pei1}
    \definition{s.}{embrião}
  \end{Phonetics}
\end{Entry}

\begin{Entry}{胚胎}{9,9}{⾁,⾁}
  \begin{Phonetics}{胚胎}{pei1tai1}[][HSK 7-9]
    \definition{s.}{embrião}[子宫里的胚胎。===Um embrião no útero.]
  \end{Phonetics}
\end{Entry}

%%%%%%%%%% 胜 %%%%%%%%%%
\subsection*{胜}\addcontentsline{loh}{figure}{胜}

\begin{Entry}{胜}{9}{⾁}
  \begin{Phonetics}{胜}{sheng4}[][HSK 3]
    \definition{adj.}{soberbo; maravilhoso; adorável}
    \definition[场]{s.}{vitória; sucesso | penteado de mulher; joias usadas pelas mulheres na antiguidade}
    \definition{v.}{vencer | derrotar | (frequentemente seguido por 于, etc.) superar; ser superior a; levar a melhor sobre | vencer; ter sucesso; derrotar o adversário | ultrapassar; ser superior ao outro | suportar; ser capaz de suportar ou aguentar}
  \seealsoref{于}{yu2}
  \antonymref{败}{bai4}
  \antonymref{负}{fu4}
  \end{Phonetics}
\end{Entry}

\begin{Entry}{胜出}{9,5}{⾁,⼐}
  \begin{Phonetics}{胜出}{sheng4chu1}[][HSK 7-9]
    \definition{v.}{(em um jogo, competição, etc.) superar; vencer; derrotar um oponente | ganhar}
  \antonymref{出局}{chu1/ju2}
  \end{Phonetics}
\end{Entry}

\begin{Entry}{胜任}{9,6}{⾁,⼈}
  \begin{Phonetics}{胜任}{sheng4ren4}[][HSK 7-9]
    \definition{v.}{ser competente; ser qualificado; ser igual a; estar à altura de; possuir capacidade suficiente para desempenhar (o trabalho, a função, etc.)}
  \synonymref{担当}{dan1dang1}
  \synonymref{担任}{dan1ren4}
  \end{Phonetics}
\end{Entry}

\begin{Entry}{胜负}{9,6}{⾁,⾙}
  \begin{Phonetics}{胜负}{sheng4-fu4}[][HSK 5]
    \definition{s.}{vitória ou derrota; sucesso ou fracasso}
  \end{Phonetics}
\end{Entry}

\begin{Entry}{胜利}{9,7}{⾁,⼑}
  \begin{Phonetics}{胜利}{sheng4li4}[][HSK 3]
    \definition{adv.}{com sucesso; triunfantemente; atingir o objetivo previsto}
    \definition{v.}{ganhar; vencer; triunfar; ter sucesso}
  \end{Phonetics}
\end{Entry}

\begin{Entry}{胜算}{9,14}{⾁,⽵}
  \begin{Phonetics}{胜算}{sheng4suan4}
    \definition{s.}{probabilidade de sucesso | estratégia que garante o sucesso}
    \definition{v.}{ter certeza do sucesso}
  \end{Phonetics}
\end{Entry}

%%%%%%%%%% 胡 %%%%%%%%%%
\subsection*{胡}\addcontentsline{loh}{figure}{胡}

\begin{Entry}{胡}{9}{⾁}
  \begin{Phonetics}{胡}{hu2}
    \definition*{s.}{Sobrenome: Hu}
    \definition{adj.}{introduzidos de nacionalidades do norte e do oeste ou do exterior | nos tempos antigos, o termo ``Oriente e Ocidente'' se referia às minorias étnicas do norte e do oeste, e também, de modo geral, às pessoas do exterior}
    \definition{adv.}{imprudentemente; desenfreadamente; escandalosamente; sem lei, ordem ou razão}
    \definition{pron.}{``Por que?''; palavras interrogativas: 为什么, 何故}
    \definition{s.}{nos tempos antigos, geralmente se referia às minorias étnicas do norte e do oeste | violino chinês | barba; bigode}
  \seealsoref{何故}{he2gu4}
  \seealsoref{为什么}{wei4shen2me5}
  \end{Phonetics}
\end{Entry}

\begin{Entry}{胡子}{9,3}{⾁,⼦}
  \begin{Phonetics}{胡子}{hu2zi5}[][HSK 5]
    \definition[团,根,个,撮]{s.}{barba; bigode | bandido; salteador}
  \end{Phonetics}
\end{Entry}

\begin{Entry}{胡同}{9,6}{⾁,⼝}
  \begin{Phonetics}{胡同}{hu2tong5}
    \definition[条,个]{s.}{beco; rua pequena}
  \end{Phonetics}
\end{Entry}

\begin{Entry}{胡同儿}{9,6,2}{⾁,⼝,⼉}
  \begin{Phonetics}{胡同儿}{hu2tong4r5}[][HSK 5]
    \definition{s.}{beco}
  \end{Phonetics}
\end{Entry}

\begin{Entry}{胡闹}{9,8}{⾁,⾾}
  \begin{Phonetics}{胡闹}{hu2nao4}[][HSK 7-9]
    \definition{v.}{correr solto; ser travesso; causar problemas; agir de forma irracional | agir intencionalmente; fazer uma cena; agir de forma imprudente; fazer coisas de forma imprudente}
  \end{Phonetics}
\end{Entry}

\begin{Entry}{胡思乱想}{9,9,7,13}{⾁,⼼,⼄,⼼}
  \begin{Phonetics}{胡思乱想}{hu2si1-luan4xiang3}[][HSK 7-9]
    \definition{expr.}{``Tem uma abelha em sua capota.''; deixar-se levar pela fantasia; dar lugar a fantasias tolas; deixar a imaginação correr solta; divagar}
  \end{Phonetics}
\end{Entry}

\begin{Entry}{胡说}{9,9}{⾁,⾔}
  \begin{Phonetics}{胡说}{hu2shuo1}[][HSK 7-9]
    \definition{v.}{falar bobagens}
  \end{Phonetics}
\end{Entry}

\begin{Entry}{胡萝卜}{9,11,2}{⾁,⾋,⼘}
  \begin{Phonetics}{胡萝卜}{hu2luo2bo5}
    \definition{s.}{cenoura}
  \end{Phonetics}
\end{Entry}

\begin{Entry}{胡琴}{9,12}{⾁,⽟}
  \begin{Phonetics}{胡琴}{hu2qin2}
    \definition{s.}{huqin, um termo geral para certos instrumentos de arco de duas cordas, como 二胡, 京胡, etc. | família de violinos chineses de duas cordas, com caixa de ressonância de madeira revestida de pele de cobra e arco de bambu com corda de crina de cavalo}
  \seealsoref{二胡}{er4hu2}
  \seealsoref{京胡}{jing1hu2}
  \end{Phonetics}
\end{Entry}

%%%%%%%%%% 脉 %%%%%%%%%%
\subsection*{脉}\addcontentsline{loh}{figure}{脉}

\begin{Entry}{脉}{9}{⾁}
  \begin{Phonetics}{脉}{mai4}
    \definition{s.}{artérias e veias | pulso | nervura (de uma folha, asa de inseto, etc.) | rede; sistema; malha | Coloquial: milhas}
  \end{Phonetics}
  \begin{Phonetics}{脉}{mo4}
    \definition{adv.}{afetuosamente; amorosamente; carinhosamente; expressar afeto silenciosamente através dos olhos ou ações}
  \end{Phonetics}
\end{Entry}

\begin{Entry}{脉络}{9,9}{⾁,⽷}
  \begin{Phonetics}{脉络}{mai4luo4}[][HSK 7-9]
    \definition{s.}{termo geral para artérias, veias e canais por onde circula a energia vital; a medicina tradicional chinesa se refere aos vasos sanguíneos e meridianos por todo o corpo | uma linha de raciocínio; uma sequência de ideias; metaforicamente, refere"-se à ordem ou estrutura das coisas ou da escrita | as nervuras (de uma folha, etc.); veios nas folhas das plantas ou em outras estruturas}
  \end{Phonetics}
\end{Entry}

\begin{Entry}{脉搏}{9,13}{⾁,⼿}
  \begin{Phonetics}{脉搏}{mai4bo2}[][HSK 7-9]
    \definition{s.}{pulso; o fenômeno das artérias pulsarem ritmicamente à medida que o sangue bombeado impacta as paredes arteriais durante a contração cardíaca | uma metáfora para o desenvolvimento, as mudanças ou as tendências da sociedade, da vida, etc.}
  \end{Phonetics}
\end{Entry}

%%%%%%%%%% 舁 %%%%%%%%%%
\subsection*{舁}\addcontentsline{loh}{figure}{舁}

\begin{Entry}{舁}{9}{⾅}
  \begin{Phonetics}{舁}{yu2}
    \definition{v.}{levantar; elevar | aumentar}
  \end{Phonetics}
\end{Entry}

%%%%%%%%%% 范 %%%%%%%%%%
\subsection*{范}\addcontentsline{loh}{figure}{范}

\begin{Entry}{范}{9}{⾋}
  \begin{Phonetics}{范}{fan4}
    \definition*{s.}{Sobrenome: Fan}
    \definition{s.}{padrão; molde; matriz | modelo; exemplo; modelo a seguir | limites; escopo | restrição; limite}
  \end{Phonetics}
\end{Entry}

\begin{Entry}{范成大}{9,6,3}{⾋,⼽,⼤}
  \begin{Phonetics}{范成大}{fan4 cheng2da4}
    \definition*{s.}{Fan Chengda (1126–1193), de nome de cortesia Zhineng 致能 e também Youyuan 幼元, autodenominou-se Cishan Jushi 此山 居士em seus primeiros anos e Shihu Jushi 石湖 居士em seus últimos anos; natural do Condado de Wu, Suzhou (atual Cidade de Suzhou, Província de Jiangsu), foi um oficial, poeta e escritor durante a Dinastia Song do Sul; seu nome póstumo foi Wenmu 文穆}
  \end{Phonetics}
\end{Entry}

\begin{Entry}{范围}{9,7}{⾋,⼞}
  \begin{Phonetics}{范围}{fan4wei2}[][HSK 3]
    \definition[个]{s.}{escopo; limite; alcance}
    \definition{v.}{estabelecer limites para; limitar o escopo de}
  \end{Phonetics}
\end{Entry}

\begin{Entry}{范畴}{9,12}{⾋,⽥}
  \begin{Phonetics}{范畴}{fan4chou2}[][HSK 7-9]
    \definition{s.}{tipo; domínio; escopo; alcance; categoria}
  \end{Phonetics}
\end{Entry}

%%%%%%%%%% 茫 %%%%%%%%%%
\subsection*{茫}\addcontentsline{loh}{figure}{茫}

\begin{Entry}{茫}{9}{⾋}
  \begin{Phonetics}{茫}{mang2}
    \definition{adj.}{ilimitado e indistinto | ignorante; no escuro | disseminado e pouco claro; descreve a água ou outras coisas como ilimitadas e incertas}
  \end{Phonetics}
\end{Entry}

\begin{Entry}{茫然}{9,12}{⾋,⽕}
  \begin{Phonetics}{茫然}{mang2ran2}[][HSK 7-9]
    \definition{adj.}{vazio; vago; ignorante; no escuro; descreve um estado de completa ignorância ou perplexidade | frustrado; decepcionado; descreve uma aparência atordoada ou distraída devido à decepção}
  \end{Phonetics}
\end{Entry}

%%%%%%%%%% 茶 %%%%%%%%%%
\subsection*{茶}\addcontentsline{loh}{figure}{茶}

\begin{Entry}{茶}{9}{⾋}
  \begin{Phonetics}{茶}{cha2}[][HSK 1]
    \definition{adj.}{moreno; fulvo; amarelo-acastanhado}
    \definition[杯,壶]{s.}{chá (a bebida); bebida feita com folhas de chá | chá (a planta) | certos tipos de bebidas ou alimentos líquidos | árvore de chá-de-óleo | camélia}
  \end{Phonetics}
\end{Entry}

\begin{Entry}{茶叶}{9,5}{⾋,⼝}
  \begin{Phonetics}{茶叶}{cha2ye4}[][HSK 4]
    \definition[包,袋,盒,斤,把,种]{s.}{chá; folhas de chá; as folhas jovens da planta do chá que são processadas para produzir bebidas}
  \end{Phonetics}
\end{Entry}

\begin{Entry}{茶馆儿}{9,11,2}{⾋,⾷,⼉}
  \begin{Phonetics}{茶馆儿}{cha2guan3r5}[][HSK 7-9]
    \definition[家,个,间]{s.}{casa de chá}
  \end{Phonetics}
\end{Entry}

\begin{Entry}{茶道}{9,12}{⾋,⾡}
  \begin{Phonetics}{茶道}{cha2dao4}[][HSK 7-9]
    \definition{s.}{cerimônia do chá ; cerimônia japonesa do chá; Sadō}
  \end{Phonetics}
\end{Entry}

%%%%%%%%%% 荆 %%%%%%%%%%
\subsection*{荆}\addcontentsline{loh}{figure}{荆}

\begin{Entry}{荆}{9}{⾋}
  \begin{Phonetics}{荆}{jing1}
    \definition*{s.}{Sobrenome: Jing}
    \definition{s.}{árvore-da-castidade; vitex | uma vara para açoitar; varas de punição antigas feitas de galhos de espinho | a própria esposa; uma forma humilde de se referir à esposa antigamente}
  \end{Phonetics}
\end{Entry}

\begin{Entry}{荆棘}{9,12}{⾋,⽊}
  \begin{Phonetics}{荆棘}{jing1ji2}[][HSK 7-9]
    \definition{s.}{silvas; cardos e espinhos; vegetação rasteira espinhosa; geralmente se refere a arbustos espinhosos que crescem nas montanhas e nos campos}
  \end{Phonetics}
\end{Entry}

%%%%%%%%%% 草 %%%%%%%%%%
\subsection*{草}\addcontentsline{loh}{figure}{草}

\begin{Entry}{草}{9}{⾋}
  \begin{Phonetics}{草}{cao3}[][HSK 2]
    \definition*{s.}{Sobrenome: Cao}
    \definition{adj.}{descuidado; rude | rascunho; inicial | femea; na linguagem coloquial, refere"-se a animais domésticos e aves fêmeas | precipitado; pouco cuidadoso | rascunho; não definitivo; preliminar; informal}
    \definition[种,棵,撮,株,根]{s.}{grama; gramado | palha | campo; zona rural; área selvagem | letra cursiva | letra cursiva (ou caligráfica) de um alfabeto fonético | rascunho | caligrafia cursiva; um tipo de escrita chinesa}
    \definition{v.}{esboçar; redigir}
  \end{Phonetics}
\end{Entry}

\begin{Entry}{草地}{9,6}{⾋,⼟}
  \begin{Phonetics}{草地}{cao3di4}[][HSK 2]
    \definition[片,块]{s.}{prado; gramado; campo; pastagem ou grande área de terra plantada com pastagem | gramado; relvado; local com grama alta ou gramado}
  \end{Phonetics}
\end{Entry}

\begin{Entry}{草纸}{9,7}{⾋,⽷}
  \begin{Phonetics}{草纸}{cao3zhi3}
    \definition{s.}{papel pardo | pergaminho | papel de palha áspero | papel higiênico}
  \end{Phonetics}
\end{Entry}

\begin{Entry}{草坪}{9,8}{⾋,⼟}
  \begin{Phonetics}{草坪}{cao3ping2}[][HSK 7-9]
    \definition{s.}{gramado; prado plano; agora se refere principalmente a um pasto plano inteiro cultivado artificialmente}
  \end{Phonetics}
\end{Entry}

\begin{Entry}{草原}{9,10}{⾋,⼚}
  \begin{Phonetics}{草原}{cao3yuan2}[][HSK 5]
    \definition[片,个]{s.}{estepe; pradaria; grandes áreas de terra coberta de vegetação em áreas semiáridas, intercaladas com árvores tolerantes à seca}
  \end{Phonetics}
\end{Entry}

\begin{Entry}{草案}{9,10}{⾋,⽊}
  \begin{Phonetics}{草案}{cao3'an4}[][HSK 7-9]
    \definition[个,项,份,部]{s.}{rascunho (de um plano, lei, etc.); leis e regulamentos que foram escritos, mas ainda não foram finalizados pelos departamentos relevantes ou ainda estão sendo testados}
  \end{Phonetics}
\end{Entry}

\begin{Entry}{草莓}{9,10}{⾋,⾋}
  \begin{Phonetics}{草莓}{cao3mei2}
    \definition[颗]{s.}{morango}
  \end{Phonetics}
\end{Entry}

%%%%%%%%%% 荒 %%%%%%%%%%
\subsection*{荒}\addcontentsline{loh}{figure}{荒}

\begin{Entry}{荒}{9}{⾋}
  \begin{Phonetics}{荒}{huang1}[][HSK 7-9]
    \definition*{s.}{Sobrenome: Huang}
    \definition{adj.}{(terra) não utilizada; não cultivada | desolado; estéril | irracional; delirante; fantástico; absurdo | incerto; duvidoso | dissoluto; autoindulgente | grosseiramente processado; bruto}
    \definition[片,块]{s.}{terra devastada; terra inculta; deserto | fome; quebra de safra | escassez | lixo; restos | terra selvagem (floresta)}
    \definition{v.}{(coloquial) negligenciar; estar fora de prática}
  \end{Phonetics}
\end{Entry}

\begin{Entry}{荒芜}{9,7}{⾋,⾋}
  \begin{Phonetics}{荒芜}{huang1wu2}
    \definition{adj.}{estéril}
  \end{Phonetics}
\end{Entry}

\begin{Entry}{荒诞}{9,8}{⾋,⾔}
  \begin{Phonetics}{荒诞}{huang1dan4}[][HSK 7-9]
    \definition{adj.}{fantástico; absurdo; incrível; inacreditável}
  \end{Phonetics}
\end{Entry}

\begin{Entry}{荒凉}{9,10}{⾋,⼎}
  \begin{Phonetics}{荒凉}{huang1liang2}[][HSK 7-9]
    \definition{adj.}{selvagem; sombrio e desolado; escassamente povoado; deserto}
  \end{Phonetics}
\end{Entry}

\begin{Entry}{荒唐}{9,10}{⾋,⼝}
  \begin{Phonetics}{荒唐}{huang1tang2}
    \definition{adj.}{absurdo; fantástico; grosseiramente exagerado; descreve pensamentos, palavras ou comportamentos anormais, fazendo as pessoas se sentirem estranhas ou ridículas | dissipado; dissoluto; descreve pessoas que não controlam seus desejos, não são limitadas por restrições e fazem as coisas casualmente}
  \end{Phonetics}
\end{Entry}

\begin{Entry}{荒谬}{9,13}{⾋,⾔}
  \begin{Phonetics}{荒谬}{huang1miu4}[][HSK 7-9]
    \definition{adj.}{absurdo; ridículo; extremamente errado; extremamente irracional}
  \end{Phonetics}
\end{Entry}

%%%%%%%%%% 荔 %%%%%%%%%%
\subsection*{荔}\addcontentsline{loh}{figure}{荔}

\begin{Entry}{荔}{9}{⾋}
  \begin{Phonetics}{荔}{li4}
    \definition[颗]{s.}{lichia | (arcaico) uma espécie de grama semelhante à taboa}
  \end{Phonetics}
\end{Entry}

\begin{Entry}{荔枝}{9,8}{⾋,⽊}
  \begin{Phonetics}{荔枝}{li4zhi1}
    \definition{s.}{lichia}
  \end{Phonetics}
\end{Entry}

%%%%%%%%%% 荡 %%%%%%%%%%
\subsection*{荡}\addcontentsline{loh}{figure}{荡}

\begin{Entry}{荡}{9}{⾋}
  \begin{Phonetics}{荡}{dang4}
    \definition*{s.}{Sobrenome: Dang}
    \definition{adj.}{indiferente às restrições morais; devasso, libertino, depravado, desregrado | vasto, amplo e nivelado}
    \definition{s.}{lago raso; pântano}
    \definition{v.}{oscilar; gingar; ondular | vadiar; vagabundear | enxaguar | limpar; varrer | vagar; vaguear; andar por aí; passear por aí}
  \end{Phonetics}
\end{Entry}

\begin{Entry}{荡漾}{9,14}{⾋,⽔}
  \begin{Phonetics}{荡漾}{dang4yang4}[][HSK 7-9]
    \definition{v.}{ondular; agitar; ser agitado}
  \end{Phonetics}
\end{Entry}

%%%%%%%%%% 荣 %%%%%%%%%%
\subsection*{荣}\addcontentsline{loh}{figure}{荣}

\begin{Entry}{荣}{9}{⾋}
  \begin{Phonetics}{荣}{rong2}
    \definition*{s.}{Sobrenome: Rong}
    \definition{adj.}{próspero; florescente | exuberante | glorioso}
    \definition{s.}{honra; glória | guarda-sol chinês | flor; flor de planta herbácea | beirais virados para cima}
    \definition{v.}{glorificar; luxuriar; crescer abundantemente; florescer | florescer | lançar}
  \antonymref{辱}{ru3}
  \end{Phonetics}
\end{Entry}

\begin{Entry}{荣幸}{9,8}{⾋,⼲}
  \begin{Phonetics}{荣幸}{rong2xing4}[][HSK 7-9]
    \definition{adj.}{ser honrado; honroso; glorioso e afortunado}
    \definition[种]{s.}{honra}
  \end{Phonetics}
\end{Entry}

\begin{Entry}{荣获}{9,10}{⾋,⾋}
  \begin{Phonetics}{荣获}{rong2huo4}[][HSK 7-9]
    \definition{v.}{ganhar; ser homenageado com; receber honras}
  \end{Phonetics}
\end{Entry}

\begin{Entry}{荣誉}{9,13}{⾋,⾔}
  \begin{Phonetics}{荣誉}{rong2yu4}[][HSK 7-9]
    \definition[种,份]{s.}{honra; glória; crédito; reputação honrosa; reputação gloriosa}
  \end{Phonetics}
\end{Entry}

%%%%%%%%%% 荤 %%%%%%%%%%
\subsection*{荤}\addcontentsline{loh}{figure}{荤}

\begin{Entry}{荤}{9}{⾋}[Kangxi 9]
  \begin{Phonetics}{荤}{hun1}
    \definition{adj.}{obsceno; lascivo; vulgar}
    \definition{s.}{carne ou peixe | Budismo: vegetais picantes proibidos aos vegetarianos budistas, como cebola, alho-poró, alho, etc. | alimentos não vegetarianos (carne, peixe etc.) | vegetais com cheiro forte (alho etc.)}
  \antonymref{素}{su4}
  \end{Phonetics}
\end{Entry}

%%%%%%%%%% 药 %%%%%%%%%%
\subsection*{药}\addcontentsline{loh}{figure}{药}

\begin{Entry}{药}{9}{⾋}
  \begin{Phonetics}{药}{yao4}[][HSK 2]
    \definition*{s.}{Sobrenome: Yao}
    \definition[片,粒,颗,瓶,服]{s.}{droga; loção; remédio; medicamento; substâncias que podem prevenir e tratar doenças, pragas ou melhorar funções corporais | certos produtos químicos com efeitos específicos}
    \definition{v.}{curar com remédios; tomar remédios para tratar doenças | matar com veneno; envenenar}
  \end{Phonetics}
\end{Entry}

\begin{Entry}{药丸}{9,3}{⾋,⼂}
  \begin{Phonetics}{药丸}{yao4wan2}
    \definition[粒]{s.}{pílula}
  \end{Phonetics}
\end{Entry}

\begin{Entry}{药水}{9,4}{⾋,⽔}
  \begin{Phonetics}{药水}{yao4shui3}[][HSK 2]
    \definition*{s.}{Yaksu na Coreia do Norte, perto da fronteira com Liaoning e a província de Jilin}
    \definition{s.}{medicamento líquido; líquido medicinal | loção | remédio engarrafado | medicamento em forma líquida}
  \end{Phonetics}
\end{Entry}

\begin{Entry}{药片}{9,4}{⾋,⽚}
  \begin{Phonetics}{药片}{yao4pian4}[][HSK 2]
    \definition[颗,片]{s.}{pílula; comprimido; preparações em comprimidos}
  \end{Phonetics}
\end{Entry}

\begin{Entry}{药补}{9,7}{⾋,⾐}
  \begin{Phonetics}{药补}{yao4bu3}
    \definition{s.}{suplemento dietético medicinal que ajuda a melhorar a saúde}
  \end{Phonetics}
\end{Entry}

\begin{Entry}{药典}{9,8}{⾋,⼋}
  \begin{Phonetics}{药典}{yao4dian3}
    \definition{s.}{farmacopéia}
  \end{Phonetics}
\end{Entry}

\begin{Entry}{药店}{9,8}{⾋,⼴}
  \begin{Phonetics}{药店}{yao4dian4}[][HSK 2]
    \definition[家]{s.}{farmácia; drogaria; lojas que vendem medicamentos}
  \end{Phonetics}
\end{Entry}

\begin{Entry}{药房}{9,8}{⾋,⼾}
  \begin{Phonetics}{药房}{yao4fang2}
    \definition{s.}{farmácia | drogaria}
  \end{Phonetics}
\end{Entry}

\begin{Entry}{药物}{9,8}{⾋,⽜}
  \begin{Phonetics}{药物}{yao4wu4}[][HSK 4]
    \definition[种]{s.}{droga; medicamento; remédio; substâncias que controlam doenças, pragas, etc.}
  \end{Phonetics}
\end{Entry}

\begin{Entry}{药品}{9,9}{⾋,⼝}
  \begin{Phonetics}{药品}{yao4pin3}[][HSK 6]
    \definition[个,些,种,类,批]{s.}{medicamentos e reagentes químicos; um termo geral para vários medicamentos e reagentes químicos}
  \end{Phonetics}
\end{Entry}

\begin{Entry}{药签}{9,13}{⾋,⽵}
  \begin{Phonetics}{药签}{yao4qian1}
    \definition{s.}{cotonete médico}
  \end{Phonetics}
\end{Entry}

\begin{Entry}{药膳}{9,16}{⾋,⾁}
  \begin{Phonetics}{药膳}{yao4shan4}
    \definition{s.}{alimentos medicamentosos; alimentos cozidos com ervas medicinais | cozinha medicinal}
  \end{Phonetics}
\end{Entry}

\begin{Entry}{药罐}{9,23}{⾋,⽸}
  \begin{Phonetics}{药罐}{yao4guan4}
    \definition{s.}{frasco de remédio; pote de remédio}
  \end{Phonetics}
\end{Entry}

%%%%%%%%%% 虐 %%%%%%%%%%
\subsection*{虐}\addcontentsline{loh}{figure}{虐}

\begin{Entry}{虐}{9}{⾌}
  \begin{Phonetics}{虐}{nve4}
    \definition{adj.}{cruel; tirânico; brutal e cruel}
  \end{Phonetics}
\end{Entry}

\begin{Entry}{虐待}{9,9}{⾌,⼻}
  \begin{Phonetics}{虐待}{nve4dai4}[][HSK 7-9]
    \definition{v.}{maltratar; tiranizar; tratar com métodos cruéis e impiedosos}
  \end{Phonetics}
\end{Entry}

%%%%%%%%%% 虽 %%%%%%%%%%
\subsection*{虽}\addcontentsline{loh}{figure}{虽}

\begin{Entry}{虽}{9}{⾍}
  \begin{Phonetics}{虽}{sui1}[][HSK 6]
    \definition{conj.}{no entanto; embora | mesmo se}
  \end{Phonetics}
\end{Entry}

\begin{Entry}{虽然}{9,12}{⾍,⽕}
  \begin{Phonetics}{虽然}{sui1ran2}[][HSK 2]
    \definition{conj.}{apesar de; embora (frequentemente usado correlativamente com 可是, 但是, etc); geralmente é usado no início de uma frase para indicar que o fato anterior foi reconhecido, mas não mudará o que acontecerá em seguida}
  \seealsoref{但是}{dan4shi4}
  \seealsoref{可是}{ke3shi4}
  \end{Phonetics}
\end{Entry}

%%%%%%%%%% 虾 %%%%%%%%%%
\subsection*{虾}\addcontentsline{loh}{figure}{虾}

\begin{Entry}{虾}{9}{⾍}
  \begin{Phonetics}{虾}{xia1}
    \definition{s.}{camarão}
  \end{Phonetics}
\end{Entry}

%%%%%%%%%% 蚂 %%%%%%%%%%
\subsection*{蚂}\addcontentsline{loh}{figure}{蚂}

\begin{Entry}{蚂}{9}{⾍}
  \begin{Phonetics}{蚂}{ma1}
    \definition{part.}{caracter formador de palavras}
    \definition[只]{s.}{libélula}
  \end{Phonetics}
  \begin{Phonetics}{蚂}{ma3}
    \definition{part.}{caracter formador de palavras}
  \end{Phonetics}
  \begin{Phonetics}{蚂}{ma4}
    \definition{part.}{caracter formador de palavras}
  \end{Phonetics}
\end{Entry}

\begin{Entry}{蚂蚁}{9,9}{⾍,⾍}
  \begin{Phonetics}{蚂蚁}{ma3yi3}
    \definition{s.}{formiga}
  \end{Phonetics}
\end{Entry}

%%%%%%%%%% 要 %%%%%%%%%%
\subsection*{要}\addcontentsline{loh}{figure}{要}

\begin{Entry}{要}{9}{⾑}
  \begin{Phonetics}{要}{yao1}
    \definition*{s.}{Sobrenome: Yao}
    \definition{v.}{exigir; pedir; requerer; solicitar; buscar; insistir com base em algo em que se apoia | forçar; coagir; ameaçar}
  \end{Phonetics}
  \begin{Phonetics}{要}{yao4}[][HSK 1,4]
    \definition{adj.}{importante; essencial}
    \definition{conj.}{suponha; no caso; se, indicando um relacionamento hipotético | ou; ou\dots ou\dots}
    \definition{s.}{ponto principal; manchete; conteúdo importante}
    \definition{v.}{querer; desejar; pensar | querer; pedir; deseja; querer obter; querer manter | recuperar algo; dizer a alguém para guardar algo para você ou devolver | pedir (ou querer) que alguém faça algo; pedir a alguém para fazer algo, quando usado para conseguir que alguém faça algo, tem um tom de comando e pode ser indelicado | precisar; tomar; pegar | deve; deveria; é necessário (imperativo, essencial) que\dots | estar indo para | querer; ter um desejo por; expressar determinação ou desejo de fazer algo | poder; dever;  indica uma estimativa, usada para comparação}
  \seealsoref{要是}{yao4shi5}
  \end{Phonetics}
\end{Entry}

\begin{Entry}{要么}{9,3}{⾑,⼃}
  \begin{Phonetics}{要么}{yao4me5}[][HSK 6]
    \definition{conj.}{ou; ou\dots ou\dots; indica uma escolha entre duas situações ou dois desejos}
  \seealsoref{要么……要么……}{yao4me5 yao4me5}
  \end{Phonetics}
\end{Entry}

\begin{Entry}{要么……要么……}{9,3,9,3}{⾑,⼃,⾑,⼃}
  \begin{Phonetics}{要么……要么……}{yao4me5 yao4me5}
    \definition{conj.}{ou\dots ou\dots}
  \seealsoref{要么}{yao4me5}
  \end{Phonetics}
\end{Entry}

\begin{Entry}{要义}{9,3}{⾑,⼂}
  \begin{Phonetics}{要义}{yao4yi4}
    \definition{s.}{resumo | o essencial}
  \end{Phonetics}
\end{Entry}

\begin{Entry}{要不}{9,4}{⾑,⼀}
  \begin{Phonetics}{要不}{yao4 bu4}
    \definition{conj.}{ou então; caso contrário; se você não fizer isso (haverá um resultado ruim) | usado para propor educadamente; usado para fazer uma sugestão educadamente | ou; se você não fizer isso, faça aquilo}
  \end{Phonetics}
\end{Entry}

\begin{Entry}{要不然}{9,4,12}{⾑,⼀,⽕}
  \begin{Phonetics}{要不然}{yao4bu5ran2}[][HSK 6]
    \definition{conj.}{caso contrário; ou então; se você não fizer isso (haverá um resultado ruim) | ou então; usado entre duas frases em um relacionamento de escolha; significa escolher uma entre as duas; equivalente a 要不}
  \seealsoref{要不}{yao4 bu4}
  \end{Phonetics}
\end{Entry}

\begin{Entry}{要好}{9,6}{⾑,⼥}
  \begin{Phonetics}{要好}{yao4hao3}[][HSK 6]
    \definition{adj.}{estar em bons termos; ser amigos próximos; relacionamento harmonioso | estar ansioso para melhorar a si mesmo; esforçar-se para progredir | ansioso para melhorar a si mesmo; esforçar-se para progredir}
  \end{Phonetics}
\end{Entry}

\begin{Entry}{要死}{9,6}{⾑,⽍}
  \begin{Phonetics}{要死}{yao4si3}
    \definition{adv.}{extremamente | muito}
  \end{Phonetics}
\end{Entry}

\begin{Entry}{要求}{9,7}{⾑,⽔}
  \begin{Phonetics}{要求}{yao1qiu2}[][HSK 2]
    \definition[个,点]{s.}{exigência; demanda; reivindicação; desejos ou condições específicas propostas}
    \definition{v.}{pedir; exigir; exigir; reivindicar; apresentar desejos ou condições específicas, esperando que sejam satisfeitos ou realizados}
  \end{Phonetics}
\end{Entry}

\begin{Entry}{要挟}{9,9}{⾑,⼿}
  \begin{Phonetics}{要挟}{yao1xie2}
    \definition{v.}{chantagear | ameaçar}
  \end{Phonetics}
\end{Entry}

\begin{Entry}{要是}{9,9}{⾑,⽇}
  \begin{Phonetics}{要是}{yao4shi5}[][HSK 3]
    \definition{conj.}{se; suponha; no caso de; conecta frases, expressa uma relação hipotética, equivalente a 如果, e pode ser usado em conjunto com 的话}
  \seealsoref{的话}{de5hua4}
  \seealsoref{如果}{ru2guo3}
  \end{Phonetics}
\end{Entry}

\begin{Entry}{要是……的话}{9,9,8,8}{⾑,⽇,⽩,⾔}
  \begin{Phonetics}{要是……的话}{yao4shi5 de5hua4}
    \definition{conj.}{se for assim\dots}
  \end{Phonetics}
\end{Entry}

\begin{Entry}{要点}{9,9}{⾑,⽕}
  \begin{Phonetics}{要点}{yao4dian3}
    \definition{s.}{pontos principais | essencial}
  \end{Phonetics}
\end{Entry}

\begin{Entry}{要素}{9,10}{⾑,⽷}
  \begin{Phonetics}{要素}{yao4su4}[][HSK 6]
    \definition[个]{s.}{fator essencial; elemento-chave; os elementos essenciais que compõem as coisas}
  \end{Phonetics}
\end{Entry}

\begin{Entry}{要谎}{9,11}{⾑,⾔}
  \begin{Phonetics}{要谎}{yao4huang3}
    \definition{v.}{pedir um preço enorme (como primeiro passo de negociação)}
  \end{Phonetics}
\end{Entry}

\begin{Entry}{要强}{9,12}{⾑,⼸}
  \begin{Phonetics}{要强}{yao4qiang2}
    \definition{adj.}{ansioso para se destacar | ansioso para progredir na vida | obstinado}
  \end{Phonetics}
\end{Entry}

%%%%%%%%%% 觉 %%%%%%%%%%
\subsection*{觉}\addcontentsline{loh}{figure}{觉}

\begin{Entry}{觉}{9}{⾒}
  \begin{Phonetics}{觉}{jiao4}[][HSK 6]
    \definition[个]{s.}{sono; o processo desde adormecer até acordar}
  \end{Phonetics}
  \begin{Phonetics}{觉}{jue2}
    \definition{s.}{sentimento; senso; percepção e discriminação de estímulos externos}
    \definition{v.}{sentir; perceber | acordar | tornar-se consciente; tornar-se desperto; despertar; entender}
  \end{Phonetics}
\end{Entry}

\begin{Entry}{觉悟}{9,10}{⾒,⼼}
  \begin{Phonetics}{觉悟}{jue2wu4}[][HSK 6]
    \definition{s.}{consciência; percepção; compreensão; nível de consciência}
    \definition{v.}{vir a compreender; tornar-se consciente de; tornar-se politicamente desperto; despertar}
  \end{Phonetics}
\end{Entry}

\begin{Entry}{觉得}{9,11}{⾒,⼻}
  \begin{Phonetics}{觉得}{jue2de5}[][HSK 1]
    \definition{v.}{sentir; estar ciente; pressentir; causar uma sensação | pensar; sentir; encontrar; considerar (tom menos assertivo)}
  \end{Phonetics}
\end{Entry}

\begin{Entry}{觉醒}{9,16}{⾒,⾣}
  \begin{Phonetics}{觉醒}{jue2xing3}[][HSK 7-9]
    \definition{v.}{despertar; acordar; perceber; passar da confusão à clareza, do erro à correção, após reconhecer os erros e problemas}
  \end{Phonetics}
\end{Entry}

%%%%%%%%%% 语 %%%%%%%%%%
\subsection*{语}\addcontentsline{loh}{figure}{语}

\begin{Entry}{语}{9}{⾔}
  \begin{Phonetics}{语}{yu3}
    \definition{s.}{língua; linguagem | dito; provérbio; refere"-se especialmente a coloquialismos, provérbios, expressões idiomáticas ou palavras de livros antigos | sinal; meio não linguístico de comunicar ideias ; ações ou sinais que substituem palavras para expressar significado | palavras; expressão; refere"-se a uma palavra, frase ou sentença}
    \definition{v.}{dizer; falar | (pássaros, insetos, etc.) gorjear; pipilar}
  \end{Phonetics}
  \begin{Phonetics}{语}{yu4}
    \definition{v.}{contar; informar}
  \end{Phonetics}
\end{Entry}

\begin{Entry}{语气}{9,4}{⾔,⽓}
  \begin{Phonetics}{语气}{yu3qi4}
    \definition[个]{s.}{maneira de falar | tom}
  \end{Phonetics}
\end{Entry}

\begin{Entry}{语言}{9,7}{⾔,⾔}
  \begin{Phonetics}{语言}{yu3yan2}[][HSK 2]
    \definition[种,门]{s.}{linguagem; é uma ferramenta exclusiva dos humanos para expressar ideias e comunicar pensamentos; é um fenômeno social especial e consiste em um sistema específico de pronúncia, vocabulário e gramática | linguagem falada}
  \end{Phonetics}
\end{Entry}

\begin{Entry}{语言实验室}{9,7,8,10,9}{⾔,⾔,⼧,⾺,⼧}
  \begin{Phonetics}{语言实验室}{yu3yan2shi2yan4shi4}
    \definition{s.}{laboratório de línguas}
  \end{Phonetics}
\end{Entry}

\begin{Entry}{语法}{9,8}{⾔,⽔}
  \begin{Phonetics}{语法}{yu3fa3}[][HSK 4]
    \definition[个]{s.}{gramática; maneira como o idioma é estruturado, incluindo a formação e as variações de palavras, a organização de frases e sentenças | estudo da gramática; estudo das regras de estrutura linguística}
  \end{Phonetics}
\end{Entry}

\begin{Entry}{语法术语}{9,8,5,9}{⾔,⽔,⽊,⾔}
  \begin{Phonetics}{语法术语}{yu3fa3 shu4yu3}
    \definition{s.}{termo gramatical}
  \end{Phonetics}
\end{Entry}

\begin{Entry}{语音}{9,9}{⾔,⾳}
  \begin{Phonetics}{语音}{yu3yin1}[][HSK 4]
    \definition{s.}{voz; pronúncia; sons da fala; som de alguém falando | pronúncia; som do idioma}
  \end{Phonetics}
\end{Entry}

\begin{Entry}{语调}{9,10}{⾔,⾔}
  \begin{Phonetics}{语调}{yu3diao4}
    \definition[个]{s.}{entonação}
  \end{Phonetics}
\end{Entry}

%%%%%%%%%% 误 %%%%%%%%%%
\subsection*{误}\addcontentsline{loh}{figure}{误}

\begin{Entry}{误}{9}{⾔}
  \begin{Phonetics}{误}{wu4}[][HSK 6]
    \definition{adj.}{errado; falso; impreciso | acidental}
    \definition{adv.}{por engano; por acidente; não intencional}
    \definition{s.}{engano; erro}
    \definition{v.}{perder | dificultar; impedir; prejudicar | confundir; entender mal; cometer um erro | causar desvantagem a. causar dano}
  \end{Phonetics}
\end{Entry}

\begin{Entry}{误会}{9,6}{⾔,⼈}
  \begin{Phonetics}{误会}{wu4hui4}[][HSK 4]
    \definition[场]{s.}{mal-entendido; desentendimentos ou conflitos decorrentes de mal-entendidos}
    \definition{v.}{entender mal; entender errado; interpretar mal; não entender; não entender corretamente o significado}
  \end{Phonetics}
\end{Entry}

\begin{Entry}{误点}{9,9}{⾔,⽕}
  \begin{Phonetics}{误点}{wu4/dian3}
    \definition{v.+compl.}{atrasar | chegar tarde}
  \end{Phonetics}
\end{Entry}

\begin{Entry}{误解}{9,13}{⾔,⾓}
  \begin{Phonetics}{误解}{wu4jie3}[][HSK 5]
    \definition[个,种]{s.}{equívoco; mal-entendido; desentendimento}
    \definition{v.}{interpretar mal; interpretar erroneamente; não compreender corretamente}
  \end{Phonetics}
\end{Entry}

%%%%%%%%%% 诱 %%%%%%%%%%
\subsection*{诱}\addcontentsline{loh}{figure}{诱}

\begin{Entry}{诱}{9}{⾔}
  \begin{Phonetics}{诱}{you4}
    \definition{v.}{guiar; liderar; dirigir | atrair; seduzir; aliciar | induzir; causar; resultar de; levar a}
  \end{Phonetics}
\end{Entry}

\begin{Entry}{诱人}{9,2}{⾔,⼈}
  \begin{Phonetics}{诱人}{you4ren2}
    \definition{adj.}{atraente | cativante}
  \end{Phonetics}
\end{Entry}

%%%%%%%%%% 说 %%%%%%%%%%
\subsection*{说}\addcontentsline{loh}{figure}{说}

\begin{Entry}{说}{9}{⾔}
  \begin{Phonetics}{说}{shui4}
    \definition{v.}{persuadir}
  \end{Phonetics}
  \begin{Phonetics}{说}{shuo1}[][HSK 1]
    \definition{s.}{uma teoria (normalmente o último caractere, como em 日心说, teoria heliocêntrica); ensinamentos; doutrina}
    \definition{v.}{falar; conversar; dizer | explicar | repreender | atuar como casamenteiro | referir-se a; indicar | criticar; aconselhar | fazer uma combinação; conciliar; mediar | discutir; falar sobre; conversar sobre | uma forma de expressão linguística da arte cênica}
  \seealsoref{日心说}{ri4 xin1 shuo1}
  \end{Phonetics}
\end{Entry}

\begin{Entry}{说不定}{9,4,8}{⾔,⼀,⼧}
  \begin{Phonetics}{说不定}{shuo1bu5ding4}[][HSK 4]
    \definition{adv.}{talvez; indica uma estimativa, possivelmente, provavelmente}
    \definition{v.}{não ter certeza; não estar certo; ser impreciso}
  \end{Phonetics}
\end{Entry}

\begin{Entry}{说好}{9,6}{⾔,⼥}
  \begin{Phonetics}{说好}{shuo1hao3}
    \definition{v.}{chegar a um acordo | concluir negociações}
  \end{Phonetics}
\end{Entry}

\begin{Entry}{说完}{9,7}{⾔,⼧}
  \begin{Phonetics}{说完}{shuo1-wan2}
    \definition{expr.}{acabar/terminar palavras}
  \end{Phonetics}
\end{Entry}

\begin{Entry}{说实话}{9,8,8}{⾔,⼧,⾔}
  \begin{Phonetics}{说实话}{shuo1/shi2hua4}[][HSK 6]
    \definition{v.+compl.}{falar a verdade; dizer a verdade sobre (os próprios erros ou crimes)}
  \end{Phonetics}
\end{Entry}

\begin{Entry}{说明}{9,8}{⾔,⽇}
  \begin{Phonetics}{说明}{shuo1ming2}[][HSK 2]
    \definition[本,个]{s.}{legenda; instrução; explicação}
    \definition{v.}{mostrar; explicar; ilustrar | indicar; mostrar; provar; demonstrar; usar materiais confiáveis para demonstrar ou determinar a autenticidade de pessoas ou coisas}
  \end{Phonetics}
\end{Entry}

\begin{Entry}{说明书}{9,8,4}{⾔,⽇,⼄}
  \begin{Phonetics}{说明书}{shuo1ming2shu1}[][HSK 6]
    \definition[本]{s.}{manual; livro de instruções; descrições textuais da finalidade, especificações, desempenho e uso de itens, bem como enredos de peças e filmes, etc.}
  \end{Phonetics}
\end{Entry}

\begin{Entry}{说服}{9,8}{⾔,⽉}
  \begin{Phonetics}{说服}{shuo1/fu2}[][HSK 4]
    \definition{v.+compl.}{persuadir; convencer; convencer a outra parte com palavras bem fundamentadas}
  \end{Phonetics}
\end{Entry}

\begin{Entry}{说法}{9,8}{⾔,⽔}
  \begin{Phonetics}{说法}{shuo1fa5}[][HSK 5]
    \definition[种,个]{s.}{formulação; maneira de dizer uma coisa; formas de expressar opiniões | versão; argumento; declaração; opinião | explicação; acordo; palavras justas; razões ou fundamentos legítimos}
  \end{Phonetics}
\end{Entry}

\begin{Entry}{说话}{9,8}{⾔,⾔}
  \begin{Phonetics}{说话}{shuo1 hua4}[][HSK 1]
    \definition{adv.}{imediatamente; em um minuto; refere"-se ao tempo que leva para falar, indicando um período muito curto}
    \definition{v.}{falar; conversar; dizer; expressar o significado através da linguagem | conversar (conversa fiada); bater papo | fofocar; conversar; criticar; censurar}
  \end{Phonetics}
\end{Entry}

\begin{Entry}{说理}{9,11}{⾔,⽟}
  \begin{Phonetics}{说理}{shuo1li3}
    \definition{v.}{racionalizar | discutir logicamente}
  \end{Phonetics}
\end{Entry}

\begin{Entry}{说谎}{9,11}{⾔,⾔}
  \begin{Phonetics}{说谎}{shuo1/huang3}
    \definition{v.+compl.}{mentir | contar uma mentira}
  \end{Phonetics}
\end{Entry}

%%%%%%%%%% 贱 %%%%%%%%%%
\subsection*{贱}\addcontentsline{loh}{figure}{贱}

\begin{Entry}{贱}{9}{⾙}
  \begin{Phonetics}{贱}{jian4}[][HSK 7-9]
    \definition*{s.}{Sobrenome: Jian}
    \definition{adj.}{baixo preço; barato | humilde | baixo; básico; desprezível | humilde; baixa posição social}
    \definition{pron.}{meu (autodepreciativo)}
  \antonymref{贵}{gui4}
  \end{Phonetics}
\end{Entry}

%%%%%%%%%% 贴 %%%%%%%%%%
\subsection*{贴}\addcontentsline{loh}{figure}{贴}

\begin{Entry}{贴}{9}{⾙}
  \begin{Phonetics}{贴}{tie1}[][HSK 4]
    \definition{adj.}{submisso; obediente | apropriado}
    \definition{clas.}{usado em gessos, emplastros}
    \definition{s.}{subsídio; subvenção}
    \definition{v.}{grudar; colar | aninhar-se a; aconchegar-se a; aconchegar-se em | subsidiar; ajudar financeiramente}
  \end{Phonetics}
\end{Entry}

%%%%%%%%%% 贵 %%%%%%%%%%
\subsection*{贵}\addcontentsline{loh}{figure}{贵}

\begin{Entry}{贵}{9}{⾙}
  \begin{Phonetics}{贵}{gui4}[][HSK 1]
    \definition*{s.}{Província de Guizhou, abreviação de 贵州 | Sobrenome: Gui}
    \definition{adj.}{caro; dispendioso | altamente valorizado; valioso | de alta patente; nobre | caro; preço ou valor elevado | digno de ser valorizado ou apreciado | nobre; honrado; posição social elevada}
    \definition{pron.}{Honrado: Seu}
  \seealsoref{贵州}{gui4zhou1}
  \antonymref{贱}{jian4}
  \end{Phonetics}
\end{Entry}

\begin{Entry}{贵州}{9,6}{⾙,⼮}
  \begin{Phonetics}{贵州}{gui4zhou1}
    \definition*{s.}{Província de Guizhou}
  \end{Phonetics}
\end{Entry}

\begin{Entry}{贵姓}{9,8}{⾙,⼥}
  \begin{Phonetics}{贵姓}{gui4xing4}
    \definition{expr.}{qual seu sobrenome?}
  \end{Phonetics}
\end{Entry}

\begin{Entry}{贵重}{9,9}{⾙,⾥}
  \begin{Phonetics}{贵重}{gui4zhong4}[][HSK 7-9]
    \definition{adj.}{valioso; precioso; alto valor; digno de atenção}
  \end{Phonetics}
\end{Entry}

\begin{Entry}{贵宾}{9,10}{⾙,⼧}
  \begin{Phonetics}{贵宾}{gui4bin1}[][HSK 7-9]
    \definition[位]{s.}{convidado de honra; convidado distinto; um convidado de alto escalão, importante e respeitado}
  \end{Phonetics}
\end{Entry}

\begin{Entry}{贵族}{9,11}{⾙,⽅}
  \begin{Phonetics}{贵族}{gui4zu2}[][HSK 7-9]
    \definition{s.}{nobre; nobreza; aristocracia; a classe alta da classe dominante na sociedade escravista ou feudal e na monarquia moderna goza de privilégios}
  \end{Phonetics}
\end{Entry}

%%%%%%%%%% 贷 %%%%%%%%%%
\subsection*{贷}\addcontentsline{loh}{figure}{贷}

\begin{Entry}{贷}{9}{⾙}
  \begin{Phonetics}{贷}{dai4}
    \definition[笔]{s.}{empréstimo; valor do empréstimo}
    \definition{v.}{pedir dinheiro emprestado ou emprestar dinheiro | fugir da responsabilidade | perdoar}
  \end{Phonetics}
\end{Entry}

\begin{Entry}{贷款}{9,12}{⾙,⽋}
  \begin{Phonetics}{贷款}{dai4kuan3}[][HSK 5]
    \definition[个,笔]{s.}{empréstimo; crédito}
    \definition{v.}{fornecer um empréstimo; conceder um empréstimo; conceder crédito a; emprestar dinheiro para quem precisa}
  \end{Phonetics}
\end{Entry}

%%%%%%%%%% 贸 %%%%%%%%%%
\subsection*{贸}\addcontentsline{loh}{figure}{贸}

\begin{Entry}{贸}{9}{⾙}
  \begin{Phonetics}{贸}{mao4}
    \definition*{s.}{Sobrenome: Mao}
    \definition{s.}{comércio; negociação}
  \end{Phonetics}
\end{Entry}

\begin{Entry}{贸易}{9,8}{⾙,⽇}
  \begin{Phonetics}{贸易}{mao4yi4}[][HSK 5]
    \definition[笔,宗,项,个]{s.}{comércio; troca; negócios; refere"-se a atividades comerciais, como a troca de mercadorias}
    \definition{v.}{fazer uma transação comercial}
  \end{Phonetics}
\end{Entry}

%%%%%%%%%% 费 %%%%%%%%%%
\subsection*{费}\addcontentsline{loh}{figure}{费}

\begin{Entry}{费}{9}{⾙}
  \begin{Phonetics}{费}{fei4}[][HSK 3]
    \definition*{s.}{Sobrenome: Fei}
    \definition{s.}{taxa; despesa; encargo}
    \definition{v.}{custar; gastar; despender | ser desperdiçador; consumir em excesso; gastar algo muito rapidamente; consumo excessivo}
  \antonymref{省}{sheng3}
  \end{Phonetics}
\end{Entry}

\begin{Entry}{费用}{9,5}{⾙,⽤}
  \begin{Phonetics}{费用}{fei4yong5}[][HSK 3]
    \definition[笔,个]{s.}{custo; despesa; desembolso}
  \end{Phonetics}
\end{Entry}

\begin{Entry}{费劲}{9,7}{⾙,⼒}
  \begin{Phonetics}{费劲}{fei4/jin4}[][HSK 7-9]
    \definition{adj.}{extenuante}
    \definition{v.+compl.}{ser extenuante; precisar ou usar grande esforço; gastar energia; fazer coisas ou falar com cuidado; trabalhar duro}
  \end{Phonetics}
\end{Entry}

%%%%%%%%%% 贺 %%%%%%%%%%
\subsection*{贺}\addcontentsline{loh}{figure}{贺}

\begin{Entry}{贺}{9}{⾙}
  \begin{Phonetics}{贺}{he4}
    \definition*{s.}{Sobrenome: He}
    \definition{v.}{parabenizar; congratular | celebrar; comemorar}
  \end{Phonetics}
\end{Entry}

\begin{Entry}{贺卡}{9,5}{⾙,⼘}
  \begin{Phonetics}{贺卡}{he4ka3}[][HSK 5]
    \definition[张]{s.}{cartão de felicitações; pedaço de papel para parabenizar amigos e parentes em seu casamento, aniversário ou festivais, geralmente impresso com palavras e desenhos de felicitações}
  \end{Phonetics}
\end{Entry}

\begin{Entry}{贺电}{9,5}{⾙,⽥}
  \begin{Phonetics}{贺电}{he4dian4}[][HSK 7-9]
    \definition[封]{s.}{mensagem de felicitações; telegrama de felicitações}
  \end{Phonetics}
\end{Entry}

\begin{Entry}{贺信}{9,9}{⾙,⼈}
  \begin{Phonetics}{贺信}{he4xin4}[][HSK 7-9]
    \definition{s.}{carta de felicitações; carta de congratulações}
  \end{Phonetics}
\end{Entry}

%%%%%%%%%% 赴 %%%%%%%%%%
\subsection*{赴}\addcontentsline{loh}{figure}{赴}

\begin{Entry}{赴}{9}{⾛}
  \begin{Phonetics}{赴}{fu4}[][HSK 7-9]
    \definition{v.}{ir para | comparecer}
  \end{Phonetics}
\end{Entry}

%%%%%%%%%% 趴 %%%%%%%%%%
\subsection*{趴}\addcontentsline{loh}{figure}{趴}

\begin{Entry}{趴}{9}{⾜}
  \begin{Phonetics}{趴}{pa1}[][HSK 7-9]
    \definition{v.}{deitar"-se de bruços; arriar; espreguiçar"-se | curvar"-se; apoiar"-se em; inclinar"-se para a frente apoiando"-se em um objeto}
  \end{Phonetics}
\end{Entry}

%%%%%%%%%% 轴 %%%%%%%%%%
\subsection*{轴}\addcontentsline{loh}{figure}{轴}

\begin{Entry}{轴}{9}{⾞}
  \begin{Phonetics}{轴}{zhou2}
    \definition{adj.}{(movimento) inflexível; rígido; desajeitado | direto; franco; decidido | em pergaminho}
    \definition{clas.}{usado para as linhas enroladas ao redor do eixo e as pinturas montadas no eixo}
    \definition{s.}{eixo | carretel; haste | rolo; pergaminho; objeto de enrolamento cilíndrico}
  \end{Phonetics}
  \begin{Phonetics}{轴}{zhou4}
    \definition{s.}{a parte final da performance; a última e central peça de uma peça dramática}
  \end{Phonetics}
\end{Entry}

\begin{Entry}{轴承}{9,8}{⾞,⼿}
  \begin{Phonetics}{轴承}{zhou2cheng2}
    \definition{s.}{(mecânico) rolamento}
  \end{Phonetics}
\end{Entry}

%%%%%%%%%% 轻 %%%%%%%%%%
\subsection*{轻}\addcontentsline{loh}{figure}{轻}

\begin{Entry}{轻}{9}{⾞}
  \begin{Phonetics}{轻}{qing1}[][HSK 2]
    \definition{adj.}{de pouco peso; leve | (de carga, equipamento, etc.) pequeno; simples | pequeno em número, grau, etc. | não sério; relaxante; leve | sem importância | suave; delicado | levianos, crédulos | leve; peso leve; densidade baixa | leve; descontraído; fácil | imprudente; descuidado | inconstante; frívolo}
    \definition{v.}{menosprezar; subestimar}
  \antonymref{重}{zhong4}
  \end{Phonetics}
\end{Entry}

\begin{Entry}{轻而易举}{9,6,8,9}{⾞,⽽,⽇,⼂}
  \begin{Phonetics}{轻而易举}{qing1'er2yi4ju3}[][HSK 7-9]
    \definition{expr.}{pode ser feito de forma descuidada; com facilidade; leve e fácil de levantar; descreve algo como fácil de fazer, que exige pouco esforço}
  \end{Phonetics}
\end{Entry}

\begin{Entry}{轻易}{9,8}{⾞,⽇}
  \begin{Phonetics}{轻易}{qing1yi5}[][HSK 4]
    \definition{adv.}{facilmente; prontamente | facilmente; precipitadamente; indica que uma ação é realizada casualmente, geralmente usado em frases negativas}
  \end{Phonetics}
\end{Entry}

\begin{Entry}{轻松}{9,8}{⾞,⽊}
  \begin{Phonetics}{轻松}{qing1song1}[][HSK 4]
    \definition{adj.}{leve; relaxado; livre de fardos; não nervoso; não cansado}
    \definition{v.}{sentir-se livre de fardos; não se sentir nervoso ou cansado}
  \end{Phonetics}
\end{Entry}

\begin{Entry}{轻型}{9,9}{⾞,⼟}
  \begin{Phonetics}{轻型}{qing1xing2}[][HSK 7-9]
    \definition{adj.}{leve (máquinas, aeronaves etc.); leve}
  \antonymref{重型}{zhong4xing2}
  \end{Phonetics}
\end{Entry}

\begin{Entry}{轻微}{9,13}{⾞,⼻}
  \begin{Phonetics}{轻微}{qing1wei1}[][HSK 7-9]
    \definition{adj.}{leve; insignificante; banal; trivial}
  \end{Phonetics}
\end{Entry}

\begin{Entry}{轻蔑}{9,14}{⾞,⾋}
  \begin{Phonetics}{轻蔑}{qing1mie4}[][HSK 7-9]
    \definition{v.}{desprezar; menosprezar; ignorar}
  \end{Phonetics}
\end{Entry}

%%%%%%%%%% 迷 %%%%%%%%%%
\subsection*{迷}\addcontentsline{loh}{figure}{迷}

\begin{Entry}{迷}{9}{⾡}
  \begin{Phonetics}{迷}{mi2}[][HSK 3]
    \definition[个]{s.}{fã; entusiasta; aficionado; pessoa que gosta excessivamente de algo}
    \definition{v.}{estar confuso; perder o rumo; se perder-se; perda da capacidade de discernimento e julgamento | ficar fascinado por; entregar-se a; ficar encantado com (por); ser louco por | confundir; desorientar; fascinar; encantar; tornar indistinto; deixar encantado e fascinado}
  \end{Phonetics}
\end{Entry}

\begin{Entry}{迷人}{9,2}{⾡,⼈}
  \begin{Phonetics}{迷人}{mi2ren2}[][HSK 5]
    \definition{adj.}{encantador; fascinante; sedutor; hipnotizante}
    \definition{v.}{confundir; intrigar; enganar}
  \end{Phonetics}
\end{Entry}

\begin{Entry}{迷失}{9,5}{⾡,⼤}
  \begin{Phonetics}{迷失}{mi2shi1}[][HSK 7-9]
    \definition{v.}{perder-se; ser incapaz de distinguir (direções, estradas, etc.)}
  \end{Phonetics}
\end{Entry}

\begin{Entry}{迷你}{9,7}{⾡,⼈}
  \begin{Phonetics}{迷你}{mi2ni3}
    \definition{adj.}{(empréstimo linguístico) mini, como em minissaia ou \emph{Mini Cooper}}
  \end{Phonetics}
\end{Entry}

\begin{Entry}{迷信}{9,9}{⾡,⼈}
  \begin{Phonetics}{迷信}{mi2xin4}[][HSK 5]
    \definition{s.}{superstição; crença supersticiosa | fé cega; adoração cega}
    \definition{v.}{ter fé cega em; ter um fetiche de}
  \end{Phonetics}
\end{Entry}

\begin{Entry}{迷宫}{9,9}{⾡,⼧}
  \begin{Phonetics}{迷宫}{mi2gong1}
    \definition{s.}{labirinto}
  \end{Phonetics}
\end{Entry}

\begin{Entry}{迷恋}{9,10}{⾡,⼼}
  \begin{Phonetics}{迷恋}{mi2lian4}[][HSK 7-9]
    \definition{v.}{ser obcecado por; estar apaixonado por; agarrar-se loucamente a; ter um carinho excessivo por algo e achar difícil abrir mão disso}
  \end{Phonetics}
\end{Entry}

\begin{Entry}{迷惑}{9,12}{⾡,⼼}
  \begin{Phonetics}{迷惑}{mi2huo4}[][HSK 7-9]
    \definition{adj.}{perplexo; confuso; desnorteado; pouco claro; incompreesível}
    \definition{v.}{intrigar; confundir; deixar perplexo; desconcertar}
  \end{Phonetics}
\end{Entry}

\begin{Entry}{迷惑不解}{9,12,4,13}{⾡,⼼,⼀,⾓}
  \begin{Phonetics}{迷惑不解}{mi2huo4-bu4jie3}[][HSK 7-9]
    \definition{expr.}{sentir-se perplexo; estar confuso; ficar intrigado}
  \end{Phonetics}
\end{Entry}

\begin{Entry}{迷路}{9,13}{⾡,⾜}
  \begin{Phonetics}{迷路}{mi2/lu4}[][HSK 7-9]
    \definition{v.+compl.}{desviar-se; perder-se; errar o caminho; perder a noção de direção; ir pelo caminho errado; não conseguir encontrar o caminho | perder-se; estar perdido na vida; essa metáfora descreve a perda da direção correta}
  \end{Phonetics}
\end{Entry}

%%%%%%%%%% 迹 %%%%%%%%%%
\subsection*{迹}\addcontentsline{loh}{figure}{迹}

\begin{Entry}{迹}{9}{⾡}
  \begin{Phonetics}{迹}{ji4}
    \definition[点,丝]{s.}{marca; traço; marca deixada para trás | restos; ruínas; vestígio; coisas deixadas por gerações anteriores (principalmente edifícios) | evento importante do passado; coisas feitas; feitos | aparência; ação; figura (escrita)
vestígios}
  \end{Phonetics}
\end{Entry}

\begin{Entry}{迹象}{9,11}{⾡,⾗}
  \begin{Phonetics}{迹象}{ji4xiang4}[][HSK 7-9]
    \definition[种]{s.}{sinal; símbolo; indicação; refere"-se a vestígios e fenômenos que podem ser usados para inferir o passado ou o futuro das coisas}
  \end{Phonetics}
\end{Entry}

%%%%%%%%%% 追 %%%%%%%%%%
\subsection*{追}\addcontentsline{loh}{figure}{追}

\begin{Entry}{追}{9}{⾡}
  \begin{Phonetics}{追}{zhui1}[][HSK 3]
    \definition{v.}{perseguir; correr atrás; seguir de perto | rastrear; investigar; chegar ao fundo de | procurar; ir atrás; esforçar-se para alcançar um determinado objetivo | recordar; relembrar | fazer depois do ocorrido; retrabalhar | cortejar (uma mulher)}
  \end{Phonetics}
\end{Entry}

\begin{Entry}{追求}{9,7}{⾡,⽔}
  \begin{Phonetics}{追求}{zhui1qiu2}[][HSK 4]
    \definition{s.}{perseguição (ações e metas positivas)}[她的追求是获得成功。===Sua meta é alcançar o sucesso.]
    \definition{v.}{procurar; aspirar; perseguir | cortejar; refere"-se especificamente ao namoro}
  \end{Phonetics}
\end{Entry}

\begin{Entry}{追究}{9,7}{⾡,⽳}
  \begin{Phonetics}{追究}{zhui1jiu1}[][HSK 6]
    \definition{v.}{descobrir; investigar}
  \end{Phonetics}
\end{Entry}

\begin{Entry}{追赶}{9,10}{⾡,⾛}
  \begin{Phonetics}{追赶}{zhui1gan3}
    \definition{v.}{perseguir | acelerar | alcançar | ultrapassar}
  \end{Phonetics}
\end{Entry}

%%%%%%%%%% 退 %%%%%%%%%%
\subsection*{退}\addcontentsline{loh}{figure}{退}

\begin{Entry}{退}{9}{⾡}
  \begin{Phonetics}{退}{tui4}[][HSK 3]
    \definition{v.}{recuar; mover"-se para trás | remover; retirar; fazer recuar; mover para trás | desistir; retirar"-se de | refluir; declinar; retroceder | aposentar"-se; deixar o emprego por atingir a idade estipulada ou por problemas de saúde | retornar; reembolsar; devolver | romper; cancelar o que foi decidido}
  \antonymref{进}{jin4}
  \end{Phonetics}
\end{Entry}

\begin{Entry}{退出}{9,5}{⾡,⼐}
  \begin{Phonetics}{退出}{tui4 chu1}[][HSK 3]
    \definition{v.}{desistir; retirar-se; separar-se; retirar-se de; abandonar o local ou outro lugar e parar de participar; abandonaar o grupo ou organização}
  \end{Phonetics}
\end{Entry}

\begin{Entry}{退休}{9,6}{⾡,⼈}
  \begin{Phonetics}{退休}{tui4/xiu1}[][HSK 3]
    \definition{v.+compl.}{aposentar-se; os trabalhadores que deixarem o emprego por velhice ou invalidez causada pelo trabalho receberão as despesas de subsistência conforme o cronograma}
  \end{Phonetics}
\end{Entry}

\begin{Entry}{退场}{9,6}{⾡,⼟}
  \begin{Phonetics}{退场}{tui4chang3}
    \definition{v.}{(atletas) retirar-se da arena (como após a cerimônia de abertura); sair da arena em marcha; atletas deixando o campo após a competição | (uma plateia) sair do teatro (como quando uma peça termina)}
  \end{Phonetics}
\end{Entry}

\begin{Entry}{退票}{9,11}{⾡,⽰}
  \begin{Phonetics}{退票}{tui4piao4}[][HSK 6]
    \definition{s.}{bilhete devolvido (ou não utilizado) | reembolso do bilhete}
    \definition{v.}{devolver um bilhete; obter um reembolso por um bilhete | devolver (um cheque)}
  \end{Phonetics}
\end{Entry}

%%%%%%%%%% 送 %%%%%%%%%%
\subsection*{送}\addcontentsline{loh}{figure}{送}

\begin{Entry}{送}{9}{⾡}
  \begin{Phonetics}{送}{song4}[][HSK 1]
    \definition*{s.}{Sobrenome: Song}
    \definition{v.}{transportar; entregar | dar; dar como presente; presentear | acompanhar; despedir-se de alguém (ao sair); acompanhar a pessoa que está partindo até o destino ou caminhar um trecho com ela | escoltar}
  \end{Phonetics}
\end{Entry}

\begin{Entry}{送礼}{9,5}{⾡,⽰}
  \begin{Phonetics}{送礼}{song4 li3}[][HSK 6]
    \definition{v.}{dar um presente a alguém; presentear alguém com um presente | enviar presentes (para obter favores) | dar um presente; enviar um presente}
  \end{Phonetics}
\end{Entry}

\begin{Entry}{送行}{9,6}{⾡,⾏}
  \begin{Phonetics}{送行}{song4 xing2}[][HSK 6]
    \definition{v.}{ver alguém partir; ir até o local onde o viajante iniciou sua jornada, despedir-se dele e observar ele partir | dar uma festa de despedida; realizar uma festa de despedida | despedir-se do falecido}
  \end{Phonetics}
\end{Entry}

\begin{Entry}{送到}{9,8}{⾡,⼑}
  \begin{Phonetics}{送到}{song4 dao4}[][HSK 2]
    \definition{v.}{enviar para (lugar)}
  \end{Phonetics}
\end{Entry}

\begin{Entry}{送给}{9,9}{⾡,⽷}
  \begin{Phonetics}{送给}{song4gei3}[][HSK 2]
    \definition{v.}{dar a (alguém ou organização); dar como algo gratuito; dar como presente}
  \end{Phonetics}
\end{Entry}

%%%%%%%%%% 适 %%%%%%%%%%
\subsection*{适}\addcontentsline{loh}{figure}{适}

\begin{Entry}{适}{9}{⾡}
  \begin{Phonetics}{适}{shi4}
    \definition*{s.}{Sobrenome: Shi}
    \definition{adj.}{confortável; bem | adequado; apropriado | certo; oportuno}
    \definition{v.}{ser apto; ser adequado; ser apropriado | ir; seguir; perseguir | (de uma mulher) casar}
  \end{Phonetics}
\end{Entry}

\begin{Entry}{适用}{9,5}{⾡,⽤}
  \begin{Phonetics}{适用}{shi4yong4}[][HSK 3]
    \definition{adj.}{adequado; aplicável}
  \end{Phonetics}
\end{Entry}

\begin{Entry}{适合}{9,6}{⾡,⼝}
  \begin{Phonetics}{适合}{shi4he2}[][HSK 3]
    \definition{v.}{servir; caber; se adequar; atender às necessidades de uma determinada situação ou pessoa}
  \end{Phonetics}
\end{Entry}

\begin{Entry}{适当}{9,6}{⾡,⼹}
  \begin{Phonetics}{适当}{shi4dang4}[][HSK 6]
    \definition{s.}{adequado; apropriado}
  \end{Phonetics}
\end{Entry}

\begin{Entry}{适应}{9,7}{⾡,⼴}
  \begin{Phonetics}{适应}{shi4ying4}[][HSK 3]
    \definition{v.}{ajustar-se; adequar-se; adaptar-se; fazer as alterações correspondentes para se adequar à medida que as condições mudam}
  \end{Phonetics}
\end{Entry}

\begin{Entry}{适时}{9,7}{⾡,⽇}
  \begin{Phonetics}{适时}{shi4shi2}[][HSK 7-9]
    \definition{adj.}{oportuno; em boa hora; no momento certo; momento apropriado; nem muito cedo, nem muito tarde}
  \end{Phonetics}
\end{Entry}

\begin{Entry}{适宜}{9,8}{⾡,⼧}
  \begin{Phonetics}{适宜}{shi4yi2}[][HSK 7-9]
    \definition{adj.}{adequado; apropriado; conveniente}
  \synonymref{得当}{de2dang4}
  \synonymref{符合}{fu2he2}
  \synonymref{合适}{he2shi4}
  \synonymref{恰当}{qia4dang4}
  \synonymref{适当}{shi4dang4}
  \synonymref{适合}{shi4he2}
  \synonymref{适应}{shi4ying4}
  \antonymref{不适}{bu2shi4}
  \antonymref{不宜}{bu4yi2}
  \end{Phonetics}
\end{Entry}

\begin{Entry}{适度}{9,9}{⾡,⼴}
  \begin{Phonetics}{适度}{shi4du4}[][HSK 7-9]
    \definition{adj.}{adequado; moderado; apropriado}
  \synonymref{截至}{jie2zhi4}
  \synonymref{戒指}{jie4zhi5}
  \antonymref{过度}{guo4du4}
  \antonymref{极度}{ji2du4}
  \end{Phonetics}
\end{Entry}

\begin{Entry}{适量}{9,12}{⾡,⾥}
  \begin{Phonetics}{适量}{shi4liang4}[][HSK 7-9]
    \definition{adj.}{moderado; quantidade adequada; apropriado; a quantidade é ideal, nem muita, nem pouca}
  \synonymref{适当}{shi4dang4}
  \synonymref{些许}{xie1xu3}
  \end{Phonetics}
\end{Entry}

%%%%%%%%%% 逃 %%%%%%%%%%
\subsection*{逃}\addcontentsline{loh}{figure}{逃}

\begin{Entry}{逃}{9}{⾡}
  \begin{Phonetics}{逃}{tao2}[][HSK 5]
    \definition{v.}{fugir; escapar; correr; dar no pé | evadir; esquivar-se; escapar}
  \end{Phonetics}
\end{Entry}

\begin{Entry}{逃走}{9,7}{⾡,⾛}
  \begin{Phonetics}{逃走}{tao2 zou3}[][HSK 5]
    \definition{v.}{escapar; afastar-se de pessoas, coisas ou lugares que não são bons para você ou que você não gosta}
  \end{Phonetics}
\end{Entry}

\begin{Entry}{逃跑}{9,12}{⾡,⾜}
  \begin{Phonetics}{逃跑}{tao2pao3}[][HSK 5]
    \definition{v.}{fugir; escapar; correr; partir para fugir de um ambiente ou de coisas que não lhe são favoráveis}
  \end{Phonetics}
\end{Entry}

\begin{Entry}{逃避}{9,16}{⾡,⾌}
  \begin{Phonetics}{逃避}{tao2bi4}
    \definition{v.}{esquivar-se; evadir-se; escapar; evitar coisas que você não quer ou não se atreve a tocar}
  \end{Phonetics}
\end{Entry}

%%%%%%%%%% 逆 %%%%%%%%%%
\subsection*{逆}\addcontentsline{loh}{figure}{逆}

\begin{Entry}{逆}{9}{⾡}
  \begin{Phonetics}{逆}{ni4}[][HSK 7-9]
    \definition{adj.}{contrário; contra; oposto; inverso | traidor; rebelde}
    \definition{adv.}{antecipadamente; com antecedência}
    \definition{s.}{traidor; rebelde}
    \definition{v.}{ir contra; opor-se; desobedecer; resistir; desafiar | Lliterário: saudar; cumprimentar}
  \antonymref{顺}{shun4}
  \end{Phonetics}
\end{Entry}

\begin{Entry}{逆境}{9,14}{⾡,⼟}
  \begin{Phonetics}{逆境}{ni4jing4}
    \definition[对]{s.}{adversidade; tribulação; circunstâncias adversas; circunstâncias desfavoráveis}
  \end{Phonetics}
\end{Entry}

%%%%%%%%%% 选 %%%%%%%%%%
\subsection*{选}\addcontentsline{loh}{figure}{选}

\begin{Entry}{选}{9}{⾡}
  \begin{Phonetics}{选}{xuan3}[][HSK 2]
    \definition{s.}{pessoa ou coisa selecionada | seleções; antologia; trabalhos selecionados e compilados}
    \definition{v.}{selecionar; escolher | eleger}
  \end{Phonetics}
\end{Entry}

\begin{Entry}{选手}{9,4}{⾡,⼿}
  \begin{Phonetics}{选手}{xuan3shou3}[][HSK 3]
    \definition[位,名,个,些]{s.}{jogador; (selecionado) competidor; atleta selecionado para uma competição esportiva; participantes selecionados entre um grande número de candidatos}
  \end{Phonetics}
\end{Entry}

\begin{Entry}{选拔}{9,8}{⾡,⼿}
  \begin{Phonetics}{选拔}{xuan3ba2}[][HSK 6]
    \definition{v.}{selecionar; escolher}
  \end{Phonetics}
\end{Entry}

\begin{Entry}{选择}{9,8}{⾡,⼿}
  \begin{Phonetics}{选择}{xuan3ze2}[][HSK 4]
    \definition[个,种,次]{s.}{escolha; opção; resultado da escolha; possibilidade de escolha}
    \definition{v.}{selecionar; escolher}
  \end{Phonetics}
\end{Entry}

\begin{Entry}{选举}{9,9}{⾡,⼂}
  \begin{Phonetics}{选举}{xuan3ju3}[][HSK 6]
    \definition[次,个]{s.}{eleição; as eleições são o processo pelo qual os cidadãos escolhem os seus representantes ou líderes através do voto}
    \definition{v.}{votar; eleger; eleger representantes ou responsáveis votando ou levantando as mãos}
  \end{Phonetics}
\end{Entry}

\begin{Entry}{选修}{9,9}{⾡,⼈}
  \begin{Phonetics}{选修}{xuan3xiu1}[][HSK 5]
    \definition{v.}{fazer como disciplina eletiva; escolher entre uma seleção de cursos disponíveis}
  \end{Phonetics}
\end{Entry}

%%%%%%%%%% 重 %%%%%%%%%%
\subsection*{重}\addcontentsline{loh}{figure}{重}

\begin{Entry}{重}{9}{⾥}
  \begin{Phonetics}{重}{chong2}[][HSK 3]
    \definition*{s.}{Sobrenome: Chong}
    \definition{adv.}{novamente; mais uma vez}
    \definition{clas.}{usado para camadas}
    \definition{v.}{repetir; duplicar}
  \end{Phonetics}
  \begin{Phonetics}{重}{zhong4}[][HSK 1]
    \definition{adj.}{pesado; densidade elevada | profundo; sério; grau profundo | importante; significativo | discreto; prudente | considerável em quantidade ou valor}
    \definition[斤,公,斤,吨]{s.}{peso}
    \definition{v.}{colocar (colocar, pôr) ênfase em; dar valor a; atribuir importância a}
  \end{Phonetics}
\end{Entry}

\begin{Entry}{重大}{9,3}{⾥,⼤}
  \begin{Phonetics}{重大}{zhong4da4}[][HSK 3]
    \definition{adj.}{excelente; importante; significativo; de grande importância}
  \end{Phonetics}
\end{Entry}

\begin{Entry}{重申}{9,5}{⾥,⽥}
  \begin{Phonetics}{重申}{chong2shen1}[][HSK 7-9]
    \definition{v.}{reafirmar; reiterar}[他重申了自己对古巴的承诺。===Ele reiterou seu compromisso com Cuba.]
  \end{Phonetics}
\end{Entry}

\begin{Entry}{重合}{9,6}{⾥,⼝}
  \begin{Phonetics}{重合}{chong2he2}[][HSK 7-9]
    \definition{s.}{coincidir; sobrepor"-se}[两个假期的时间重合了。===Os dois feriados se sobrepõem.]
  \end{Phonetics}
\end{Entry}

\begin{Entry}{重阳节}{9,6,5}{⾥,⾩,⾋}
  \begin{Phonetics}{重阳节}{chong2yang2jie2}
    \definition*{s.}{Festa do Duplo Nove, Festival Yang, dia de subir aos lugares mais altos para evitar calamidades e dia do culto aos antepassados (9º dia do nono mês lunar)}
  \end{Phonetics}
\end{Entry}

\begin{Entry}{重返}{9,7}{⾥,⾡}
  \begin{Phonetics}{重返}{chong2fan3}[][HSK 7-9]
    \definition{v.}{retornar; voltar; de volta ao lugar original}
  \end{Phonetics}
\end{Entry}

\begin{Entry}{重建}{9,8}{⾥,⼵}
  \begin{Phonetics}{重建}{chong2jian4}[][HSK 6]
    \definition{s.}{restabelecimento; reconstrução}
    \definition{v.}{reconstruir; reconstruir; restabelecer; reabilitar}
  \end{Phonetics}
\end{Entry}

\begin{Entry}{重现}{9,8}{⾥,⾒}
  \begin{Phonetics}{重现}{chong2xian4}[][HSK 7-9]
    \definition{v.}{reaparecer | reproduzir}
  \end{Phonetics}
\end{Entry}

\begin{Entry}{重组}{9,8}{⾥,⽷}
  \begin{Phonetics}{重组}{chong2zu3}[][HSK 6]
    \definition{v.}{reestruturar; reorganizar; remanejar}
  \end{Phonetics}
\end{Entry}

\begin{Entry}{重视}{9,8}{⾥,⾒}
  \begin{Phonetics}{重视}{zhong4shi4}[][HSK 2]
    \definition{v.}{valorizar; dar peso a; atribuir importância a; prestar atenção a; considerar a virtude ou o talento de uma pessoa ou o papel de algo como importante e levá-lo a sério}
  \end{Phonetics}
\end{Entry}

\begin{Entry}{重型}{9,9}{⾥,⼟}
  \begin{Phonetics}{重型}{zhong4xing2}
    \definition{adj.}{pesado (serviço); pesado}
  \antonymref{轻型}{qing1xing2}
  \end{Phonetics}
\end{Entry}

\begin{Entry}{重复}{9,9}{⾥,⼢}
  \begin{Phonetics}{重复}{chong2fu4}[][HSK 2]
    \definition{v.}{repetir; iterar; duplicar; reduplicar | fazer algo novamente; repetir as mesmas palavras, fazer as mesmas coisas}
  \end{Phonetics}
\end{Entry}

\begin{Entry}{重点}{9,9}{⾥,⽕}
  \begin{Phonetics}{重点}{chong2dian3}
    \definition[个]{adj./adv./s.}{nota principal; ponto-chave; ponto focal; ênfase}
  \end{Phonetics}
  \begin{Phonetics}{重点}{zhong4dian3}[][HSK 2]
    \definition[个]{s.}{nota principal; ponto-chave; ponto}
  \end{Phonetics}
\end{Entry}

\begin{Entry}{重要}{9,9}{⾥,⾑}
  \begin{Phonetics}{重要}{zhong4yao4}[][HSK 1]
    \definition{adj.}{importante; significativo; relevante; de grande importância, função e impacto}
  \synonymref{关键}{guan1jian4}
  \synonymref{首要}{shou3yao4}
  \synonymref{严重}{yan2zhong4}
  \synonymref{重大}{zhong4da4}
  \synonymref{主要}{zhu3yao4}
  \end{Phonetics}
\end{Entry}

\begin{Entry}{重重}{9,9}{⾥,⾥}
  \begin{Phonetics}{重重}{chong2chong2}
    \definition{adv.}{camada após camada | um após o outro}
  \end{Phonetics}
  \begin{Phonetics}{重重}{zhong4zhong4}
    \definition{adv.}{fortemente | severamente}
  \end{Phonetics}
\end{Entry}

\begin{Entry}{重逢}{9,10}{⾥,⾡}
  \begin{Phonetics}{重逢}{chong2feng2}
    \definition{v.}{reunir; encontrar"-se novamente; reencontrar"-se após uma longa separação; rever"-se}
  \end{Phonetics}
\end{Entry}

\begin{Entry}{重量}{9,12}{⾥,⾥}
  \begin{Phonetics}{重量}{zhong4liang4}[][HSK 4]
    \definition[个]{s.}{peso; a magnitude da força da gravidade em um objeto}
  \end{Phonetics}
\end{Entry}

\begin{Entry}{重叠}{9,13}{⾥,⼜}
  \begin{Phonetics}{重叠}{chong2die2}[][HSK 7-9]
    \definition{s.}{reduplicação; gramaticalmente, refere"-se ao uso da mesma palavra ou frase duas vezes}
    \definition{v.}{sobrepor; por um em cima do outro; colocar as mesmas coisas juntas camada por camada}
  \end{Phonetics}
\end{Entry}

\begin{Entry}{重新}{9,13}{⾥,⽄}
  \begin{Phonetics}{重新}{chong2xin1}[][HSK 2]
    \definition{adv.}{novamente; de novo; significa repetir uma ação ou comportamento já realizado | indica que se deve começar do início (mudança de método ou conteúdo)}
  \end{Phonetics}
\end{Entry}

\begin{Entry}{重播}{9,15}{⾥,⼿}
  \begin{Phonetics}{重播}{chong2bo1}[][HSK 7-9]
    \definition{v.}{retransmitir (um programa de rádio ou TV) | Agricultura: ressemear (o mesmo campo)}
  \end{Phonetics}
\end{Entry}

%%%%%%%%%% 钙 %%%%%%%%%%
\subsection*{钙}\addcontentsline{loh}{figure}{钙}

\begin{Entry}{钙}{9}{⾦}
  \begin{Phonetics}{钙}{gai4}[][HSK 7-9]
    \definition[克,毫克]{s.}{Ca, cálcio}
  \end{Phonetics}
\end{Entry}

%%%%%%%%%% 钝 %%%%%%%%%%
\subsection*{钝}\addcontentsline{loh}{figure}{钝}

\begin{Entry}{钝}{9}{⾦}
  \begin{Phonetics}{钝}{dun4}
    \definition{adj.}{sem corte; opaco | estúpido; sem noção | maçante}
  \antonymref{快}{kuai4}
  \antonymref{利}{li4}
  \antonymref{锐}{rui4}
  \end{Phonetics}
\end{Entry}

%%%%%%%%%% 钞 %%%%%%%%%%
\subsection*{钞}\addcontentsline{loh}{figure}{钞}

\begin{Entry}{钞}{9}{⾦}
  \begin{Phonetics}{钞}{chao1}
    \definition*{s.}{Sobrenome: Chao}
    \definition[张,把,叠,摞]{s.}{cédula; nota de banco; papel-moeda; nota bancária | escritos coletados; texto transcrito}
    \definition{v.}{copiar; transcrever; o mesmo que 抄}
  \seealsoref{抄}{chao1}
  \end{Phonetics}
\end{Entry}

\begin{Entry}{钞票}{9,11}{⾦,⽰}
  \begin{Phonetics}{钞票}{chao1piao4}[][HSK 7-9]
    \definition[张,扎]{s.}{cédula; nota de banco; papel-moeda}
  \end{Phonetics}
\end{Entry}

%%%%%%%%%% 钟 %%%%%%%%%%
\subsection*{钟}\addcontentsline{loh}{figure}{钟}

\begin{Entry}{钟}{9}{⾦}
  \begin{Phonetics}{钟}{zhong1}[][HSK 3]
    \definition*{s.}{Sobrenome: Zhong}
    \definition[顶,个,口]{s.}{sino; campainha; um instrumento de percussão antigo, oco, feito de cobre ou ferro | relógio; um aparelho para medir o tempo que não se leva consigo | tempo medido em horas e minutos; referindo"-se ao tempo ou momento| um recipiente antigo para guardar vinho, com barriga grande e gargalo pequeno | sino; refere"-se especificamente aos sinos pendurados em templos ou outros locais, cujo som é usado para marcar as horas, alertar ou convocar pessoas}
    \definition{v.}{focar; concentrar (as afeições de alguém, etc.)}
  \end{Phonetics}
\end{Entry}

\begin{Entry}{钟头}{9,5}{⾦,⼤}
  \begin{Phonetics}{钟头}{zhong1tou2}[][HSK 6]
    \definition[个]{s.}{hora; 60 minutos}[三四个钟头过去了。===Três ou quatro horas se passaram.]
  \end{Phonetics}
\end{Entry}

\begin{Entry}{钟室}{9,9}{⾦,⼧}
  \begin{Phonetics}{钟室}{zhong1shi4}
    \definition{s.}{campanário | sala do relógio}
  \end{Phonetics}
\end{Entry}

\begin{Entry}{钟罩}{9,13}{⾦,⽹}
  \begin{Phonetics}{钟罩}{zhong1zhao4}
    \definition{s.}{redoma | dossel de sino}
  \end{Phonetics}
\end{Entry}

%%%%%%%%%% 钢 %%%%%%%%%%
\subsection*{钢}\addcontentsline{loh}{figure}{钢}

\begin{Entry}{钢}{9}{⾦}
  \begin{Phonetics}{钢}{gang1}[][HSK 7-9]
    \definition[吨,块,根]{s.}{aço; liga de ferro e carbono}
  \end{Phonetics}
\end{Entry}

\begin{Entry}{钢丝}{9,5}{⾦,⼀}
  \begin{Phonetics}{钢丝}{gang1si1}
    \definition{s.}{cabo de aço | corda bamba}
  \end{Phonetics}
\end{Entry}

\begin{Entry}{钢笔}{9,10}{⾦,⽵}
  \begin{Phonetics}{钢笔}{gang1bi3}[][HSK 5]
    \definition[支,杆]{s.}{caneta-tinteiro; canetas com ponta metálica}
  \end{Phonetics}
\end{Entry}

\begin{Entry}{钢琴}{9,12}{⾦,⽟}
  \begin{Phonetics}{钢琴}{gang1qin2}[][HSK 5]
    \definition[架,台]{s.}{piano}
  \end{Phonetics}
\end{Entry}

%%%%%%%%%% 钥 %%%%%%%%%%
\subsection*{钥}\addcontentsline{loh}{figure}{钥}

\begin{Entry}{钥}{9}{⾦}
  \begin{Phonetics}{钥}{yao4}
    \definition{s.}{chave}
  \end{Phonetics}
\end{Entry}

\begin{Entry}{钥匙}{9,11}{⾦,⼔}
  \begin{Phonetics}{钥匙}{yao4shi5}
    \definition[把]{s.}{chave}
  \end{Phonetics}
\end{Entry}

\begin{Entry}{钥匙孔}{9,11,4}{⾦,⼔,⼦}
  \begin{Phonetics}{钥匙孔}{yao4shi5kong3}
    \definition{s.}{buraco da fechadura}
  \end{Phonetics}
\end{Entry}

\begin{Entry}{钥匙卡}{9,11,5}{⾦,⼔,⼘}
  \begin{Phonetics}{钥匙卡}{yao4shi5ka3}
    \definition{s.}{cartão de acesso | cartão-chave; cartão magnético}
  \end{Phonetics}
\end{Entry}

\begin{Entry}{钥匙洞孔}{9,11,9,4}{⾦,⼔,⽔,⼦}
  \begin{Phonetics}{钥匙洞孔}{yao4shi5dong4kong3}
    \definition{s.}{buraco da fechadura}
  \end{Phonetics}
\end{Entry}

\begin{Entry}{钥匙圈}{9,11,11}{⾦,⼔,⼞}
  \begin{Phonetics}{钥匙圈}{yao4shi5quan1}
    \definition{s.}{chaveiro}
  \end{Phonetics}
\end{Entry}

%%%%%%%%%% 钦 %%%%%%%%%%
\subsection*{钦}\addcontentsline{loh}{figure}{钦}

\begin{Entry}{钦}{9}{⾦}
  \begin{Phonetics}{钦}{qin1}
    \definition*{s.}{Sobrenome: Qin}
    \definition{adv.}{pelo próprio imperador}
    \definition{v.}{admirar; respeitar}
  \end{Phonetics}
\end{Entry}

\begin{Entry}{钦佩}{9,8}{⾦,⼈}
  \begin{Phonetics}{钦佩}{qin1pei4}[][HSK 7-9]
    \definition{v.}{admirar; respeitar; ter em alta consideração alguém; sentir respeito e afeto por alguém}
  \end{Phonetics}
\end{Entry}

%%%%%%%%%% 钩 %%%%%%%%%%
\subsection*{钩}\addcontentsline{loh}{figure}{钩}

\begin{Entry}{钩}{9}{⾦}
  \begin{Phonetics}{钩}{gou1}[][HSK 7-9]
    \definition*{s.}{Sobrenome: Gou}
    \definition[只,个]{s.}{gancho | traço de gancho em caracteres chineses | marca de verificação; visto; \emph{tick}; \emph{check mark} | marca em forma de gancho | uma espada em forma de gancho | forma falada do numeral 9 em certas ocasiões}
    \definition{v.}{prender com um gancho; enganchar | fazer crochê | costurar com pontos grandes | costurar com pontos longos}
  \end{Phonetics}
\end{Entry}

\begin{Entry}{钩子}{9,3}{⾦,⼦}
  \begin{Phonetics}{钩子}{gou1zi5}[][HSK 7-9]
    \definition[个]{s.}{gancho | coisa parecida com um gancho}
  \end{Phonetics}
\end{Entry}

%%%%%%%%%% 闺 %%%%%%%%%%
\subsection*{闺}\addcontentsline{loh}{figure}{闺}

\begin{Entry}{闺}{9}{⾨}
  \begin{Phonetics}{闺}{gui1}
    \definition{s.}{Arcaico: (em uma casa) porta pequena; porta com arco | quarto da senhora; \emph{boudoir} | Literário: um pequeno portão; parte superior redonda e porta pequena na parte inferior}
  \end{Phonetics}
\end{Entry}

\begin{Entry}{闺女}{9,3}{⾨,⼥}
  \begin{Phonetics}{闺女}{gui1nv5}[][HSK 7-9]
    \definition[个]{s.}{menina; donzela; mulher solteira | filha}
  \end{Phonetics}
\end{Entry}

%%%%%%%%%% 闻 %%%%%%%%%%
\subsection*{闻}\addcontentsline{loh}{figure}{闻}

\begin{Entry}{闻}{9}{⾨}
  \begin{Phonetics}{闻}{wen2}[][HSK 2]
    \definition*{s.}{Sobrenome: Wen}
    \definition{adj.}{bem conhecido; famoso}
    \definition{s.}{notícia; história | reputação | boato; rumor}
    \definition{v.}{cheirar | ouvir}
  \end{Phonetics}
\end{Entry}

%%%%%%%%%% 阀 %%%%%%%%%%
\subsection*{阀}\addcontentsline{loh}{figure}{阀}

\begin{Entry}{阀}{9}{⾨}
  \begin{Phonetics}{阀}{fa2}
    \definition[个]{s.}{casa estabelecida ou grupo de poder; uma pessoa ou família poderosa; refere"-se a uma pessoa ou família que tem uma influência dominante em uma determinada área | válvula (mecânica)}
  \end{Phonetics}
\end{Entry}

\begin{Entry}{阀门}{9,3}{⾨,⾨}
  \begin{Phonetics}{阀门}{fa2men2}[][HSK 7-9]
    \definition{s.}{válvula (mecânica); dispositivos para controlar o fluxo de água e ar em máquinas e tubulações}
  \end{Phonetics}
\end{Entry}

%%%%%%%%%% 阁 %%%%%%%%%%
\subsection*{阁}\addcontentsline{loh}{figure}{阁}

\begin{Entry}{阁}{9}{⾨}
  \begin{Phonetics}{阁}{ge2}
    \definition{s.}{pavilhão (geralmente de dois andares) | gabinete (de um governo) | Obsoleto: quarto da mulher; \emph{boudoir} | prateleira}
  \end{Phonetics}
\end{Entry}

\begin{Entry}{阁下}{9,3}{⾨,⼀}
  \begin{Phonetics}{阁下}{ge2xia4}
    \definition{pron.}{Sua Excelência | Sua Majestade | \emph{Sire}}
  \end{Phonetics}
\end{Entry}

%%%%%%%%%% 陡 %%%%%%%%%%
\subsection*{陡}\addcontentsline{loh}{figure}{陡}

\begin{Entry}{陡}{9}{⾩}
  \begin{Phonetics}{陡}{dou3}[][HSK 7-9]
    \definition{adj.}{íngreme; precipitado}
    \definition{adv.}{Literário: abruptamente; de repente; inesperadamente; significa que a ação ou situação acontece de forma rápida e inesperada, o que equivale a 突然}
  \seealsoref{突然}{tu1ran2}
  \end{Phonetics}
\end{Entry}

%%%%%%%%%% 院 %%%%%%%%%%
\subsection*{院}\addcontentsline{loh}{figure}{院}

\begin{Entry}{院}{9}{⾩}
  \begin{Phonetics}{院}{yuan4}[][HSK 2]
    \definition*{s.}{Sobrenome: Yuan}
    \definition[个]{s.}{pátio; quintal; complexo | designação para certos escritórios governamentais e locais públicos | faculdade; academia; instituto de ensino superior | hospital}
  \end{Phonetics}
\end{Entry}

\begin{Entry}{院子}{9,3}{⾩,⼦}
  \begin{Phonetics}{院子}{yuan4zi5}[][HSK 2]
    \definition[个,座,处]{s.}{quintal; pátio; o espaço aberto na frente ou atrás de uma casa cercado por muros ou cercas}
  \end{Phonetics}
\end{Entry}

\begin{Entry}{院长}{9,4}{⾩,⾧}
  \begin{Phonetics}{院长}{yuan4zhang3}[][HSK 2]
    \definition[个,位,名]{s.}{reitor; diretor; o mais alto funcionário de qualquer instituição ou escola pública ou privada}
  \end{Phonetics}
\end{Entry}

%%%%%%%%%% 除 %%%%%%%%%%
\subsection*{除}\addcontentsline{loh}{figure}{除}

\begin{Entry}{除}{9}{⾩}
  \begin{Phonetics}{除}{chu2}[][HSK 6]
    \definition*{s.}{Sobrenome: Chu}
    \definition{prep.}{exceto; não incluído | além do mais}
    \definition{s.}{degraus de uma casa; degraus de uma porta; escadaria}
    \definition{v.}{remover; livrar"-se de; eliminar; limpar | dividir; executar operação de divisão | nomear para o cargo}
  \end{Phonetics}
\end{Entry}

\begin{Entry}{除了}{9,2}{⾩,⼅}
  \begin{Phonetics}{除了}{chu2le5}[][HSK 3]
    \definition{prep.}{exceto; à parte; indica que o que foi dito não é levado em consideração | além disso; além de; usado em conjunto com 还, 也 e 只, indica que, além de algo, há ainda outra coisa | ou\dots ou\dots; usado em conjunto com 就是, significa ``ou assim ou assado''}
  \seealsoref{还}{hai2}
  \seealsoref{就是}{jiu4shi4}
  \seealsoref{也}{ye3}
  \seealsoref{只}{zhi3}
  \end{Phonetics}
\end{Entry}

\begin{Entry}{除夕}{9,3}{⾩,⼣}
  \begin{Phonetics}{除夕}{chu2xi1}[][HSK 5]
    \definition*{s.}{Véspera de Ano Novo Lunar; a noite do último dia do ano, também se refere ao último dia do ano}
  \end{Phonetics}
\end{Entry}

\begin{Entry}{除去}{9,5}{⾩,⼛}
  \begin{Phonetics}{除去}{chu2qu4}[][HSK 7-9]
    \definition{prep.}{além disso; além de; exceto por}
    \definition{v.}{livrar"-se de; eliminar; remover}
  \end{Phonetics}
\end{Entry}

\begin{Entry}{除外}{9,5}{⾩,⼣}
  \begin{Phonetics}{除外}{chu2wai4}[][HSK 7-9]
    \definition{prep.}{exceto; não contando; não incluindo | mas; exclusivo de; com exceção de; impedindo; além de; exceto; exceto por; a menos que; aquém de | além de; diferente de; de outra forma que não; em (para) o negócio; em cima de}
  \end{Phonetics}
\end{Entry}

\begin{Entry}{除此之外}{9,6,3,5}{⾩,⽌,⼂,⼣}
  \begin{Phonetics}{除此之外}{chu2ci3zhi1wai4}[][HSK 7-9]
    \definition{expr.}{além disso; além destes; além do mais}
  \end{Phonetics}
\end{Entry}

\begin{Entry}{除非}{9,8}{⾩,⾮}
  \begin{Phonetics}{除非}{chu2fei1}[][HSK 5]
    \definition{conj.}{a menos que; somente se; indica a única condição, equivalente a 只有, frequentemente combinada com 才, 否则, 不然, etc.}
  \seealsoref{不然}{bu4ran2}
  \seealsoref{才}{cai2}
  \seealsoref{否则}{fou3ze2}
  \seealsoref{只有}{zhi3you3}
  \end{Phonetics}
\end{Entry}

%%%%%%%%%% 险 %%%%%%%%%%
\subsection*{险}\addcontentsline{loh}{figure}{险}

\begin{Entry}{险}{9}{⾩}
  \begin{Phonetics}{险}{xian3}[][HSK 6]
    \definition{adj.}{perigoso; arriscado | sinistro; cruel; venenoso}
    \definition{adv.}{por um fio de cabelo; por centímetros; quase}
    \definition{s.}{lugar de difícil acesso; lugar perigoso e difícil de atravessar; passagem estreita; desfiladeiro | abreviação de seguro, 保险 | perigo; risco}
  \seealsoref{保险}{bao3xian3}
  \end{Phonetics}
\end{Entry}

%%%%%%%%%% 面 %%%%%%%%%%
\subsection*{面}\addcontentsline{loh}{figure}{面}

\begin{Entry}{面}{9}{⾯}[Kangxi 176]
  \begin{Phonetics}{面}{mian4}[][HSK 2]
    \definition*{s.}{Sobrenome: Mian}
    \definition{adj.}{macio e farinhento; descreve algo que é muito macio ao comer | superficial}
    \definition{adv.}{diretamente; pessoalmente; na frente de alguém; cara a cara}
    \definition{clas.}{usado para objetos planos | usado para indicar o número de vezes que as pessoas se encontram}
    \definition[斤,两,碗]{s.}{face; parte frontal da cabeça; rosto | topo; superfície | capa; exterior; a parte externa de um objeto ou a face frontal de um tecido | Matemática: superfície | cara; sentimento; emoção | geral; área total; abrangente; toda a região | lado; aspecto | escopo; escala; extensão; alcance; âmbito | farinha; farinha de trigo | pó; algo em pó | macarrão; \emph{noodle}}
    \definition{suf.}{sufixo para localização ou direção; anexado ao final de palavras que indicam localização, equivalente a 边}
    \definition{v.}{encarar algo | encontrar; revelar-se}
  \seealsoref{边}{bian1}
  \antonymref{里}{li3}
  \end{Phonetics}
\end{Entry}

\begin{Entry}{面子}{9,3}{⾯,⼦}
  \begin{Phonetics}{面子}{mian4zi5}[][HSK 5]
    \definition{s.}{face; exterior; parte externa; superfície do objeto | imagem; reputação; prestígio; decência; vaidade superficial | sentimentos; sensibilidades | pó}
  \end{Phonetics}
\end{Entry}

\begin{Entry}{面包}{9,5}{⾯,⼓}
  \begin{Phonetics}{面包}{mian4bao1}[][HSK 1]
    \definition[个,片,袋,块]{s.}{pão}[我买八个面包了。===Comprei oito pães. | 他在吃两片面包。===Ele está comendo duas fatias de pão. | 我在家里带了一袋面包。===Trouxe um saco de pão para casa. | 我拿了一块面包。===Peguei um pedaço de pão.]
  \end{Phonetics}
\end{Entry}

\begin{Entry}{面对}{9,5}{⾯,⼨}
  \begin{Phonetics}{面对}{mian4dui4}[][HSK 3]
    \definition{v.}{enfrentar; defrontar; olhar para (uma pessoa ou um objeto específico) | confrontar (problema); problemas, dificuldades e outras questões que precisam ser resolvidas e que merecem atenção}
  \end{Phonetics}
\end{Entry}

\begin{Entry}{面对面}{9,5,9}{⾯,⼨,⾯}
  \begin{Phonetics}{面对面}{mian4dui4mian4}[][HSK 6]
    \definition{adj./expr.}{frente a frente; cara a cara; vis"-à"-vis}
  \end{Phonetics}
\end{Entry}

\begin{Entry}{面对面吃面}{9,5,9,6,9}{⾯,⼨,⾯,⼝,⾯}
  \begin{Phonetics}{面对面吃面}{mian4dui4mian4 chi1 mian4}
    \definition{expr.}{Comer macarrão cara a cara; indica que o seu estado atual, ou algumas das posições em que você está, ou algumas das coisas que você fez são muito claras}
  \end{Phonetics}
\end{Entry}

\begin{Entry}{面目全非}{9,5,6,8}{⾯,⽬,⼊,⾮}
  \begin{Phonetics}{面目全非}{mian4mu4-quan2fei1}[][HSK 7-9]
    \definition{expr.}{perder a própria identidade; uma mudança completa; tudo parece errado ou diferente; ser alterado (distorcido) a ponto de ficar irreconhecível; não ser mais como era antes; ser muito diferente do original; os originais não existem mais; a aparência das coisas mudou drasticamente (frequentemente com uma conotação negativa); mudança além do reconhecimento (frequentemente com uma conotação pejorativa)}
  \end{Phonetics}
\end{Entry}

\begin{Entry}{面向}{9,6}{⾯,⼝}
  \begin{Phonetics}{面向}{mian4xiang4}[][HSK 6]
    \definition{v.}{virar o rosto para; virar na direção de; defrontar; voltado para algum lugar | estar orientado para as necessidades de; atender a; principalmente para um certo tipo de pessoas}
  \end{Phonetics}
\end{Entry}

\begin{Entry}{面团}{9,6}{⾯,⼞}
  \begin{Phonetics}{面团}{mian4tuan2}
    \definition{s.}{massa | pasta}
  \end{Phonetics}
\end{Entry}

\begin{Entry}{面红耳赤}{9,6,6,7}{⾯,⽷,⽿,⾚}
  \begin{Phonetics}{面红耳赤}{mian4hong2-er3chi4}[][HSK 7-9]
    \definition{expr.}{ficar corado; ficar ruborizado de raiva; descreve um rosto corado devido à impaciência ou timidez}
  \end{Phonetics}
\end{Entry}

\begin{Entry}{面条}{9,7}{⾯,⽊}
  \begin{Phonetics}{面条}{mian4tiao2}
    \definition{s.}{macarrão | espaguete}
  \end{Phonetics}
\end{Entry}

\begin{Entry}{面条儿}{9,7,2}{⾯,⽊,⼉}
  \begin{Phonetics}{面条儿}{mian4tiao2r5}[][HSK 1]
    \definition{s.}{macarrão; \emph{noodles}}
  \end{Phonetics}
\end{Entry}

\begin{Entry}{面试}{9,8}{⾯,⾔}
  \begin{Phonetics}{面试}{mian4shi4}[][HSK 4]
    \definition{v.}{entrevistar (é realizado na forma de perguntas e respostas orais presenciais)}
  \end{Phonetics}
\end{Entry}

\begin{Entry}{面临}{9,9}{⾯,⼁}
  \begin{Phonetics}{面临}{mian4lin2}[][HSK 4]
    \definition{v.}{ser confrontado com; encontrar (uma situação) na frente de}
  \end{Phonetics}
\end{Entry}

\begin{Entry}{面前}{9,9}{⾯,⼑}
  \begin{Phonetics}{面前}{mian4qian2}[][HSK 2]
    \definition{s.}{antes; na frente de; diante de}
  \end{Phonetics}
\end{Entry}

\begin{Entry}{面面俱到}{9,9,10,8}{⾯,⾯,⼈,⼑}
  \begin{Phonetics}{面面俱到}{mian4mian4-ju4dao4}[][HSK 7-9]
    \definition{expr.}{cuidar de tudo; resolver tudo; contemplar todos os aspectos e não deixa nada de fora; dar atenção a todos os aspectos de uma questão}
  \end{Phonetics}
\end{Entry}

\begin{Entry}{面积}{9,10}{⾯,⽲}
  \begin{Phonetics}{面积}{mian4ji1}[][HSK 3]
    \definition{s.}{área (de um andar, pedaço de terreno, etc.); área de uma superfície; o tamanho de uma superfície plana ou da superfície de um objeto}
  \end{Phonetics}
\end{Entry}

\begin{Entry}{面粉}{9,10}{⾯,⽶}
  \begin{Phonetics}{面粉}{mian4fen3}[][HSK 7-9]
    \definition[份]{s.}{farinha; farinha de trigo}
  \end{Phonetics}
\end{Entry}

\begin{Entry}{面部}{9,10}{⾯,⾢}
  \begin{Phonetics}{面部}{mian4bu4}[][HSK 7-9]
    \definition{s.}{rosto; face}
  \end{Phonetics}
\end{Entry}

\begin{Entry}{面貌}{9,14}{⾯,⾘}
  \begin{Phonetics}{面貌}{mian4mao4}[][HSK 5]
    \definition[种,个]{s.}{rosto; traços faciais; formato do rosto; aparência | aparência; aspecto; aparência (das coisas)}
  \end{Phonetics}
\end{Entry}

%%%%%%%%%% 革 %%%%%%%%%%
\subsection*{革}\addcontentsline{loh}{figure}{革}

\begin{Entry}{革}{9}{⾰}[Kangxi 177]
  \begin{Phonetics}{革}{ge2}
    \definition*{s.}{Sobrenome: Ge}
    \definition{s.}{couro; pele; peles de animais depiladas e processadas}
    \definition{v.}{mudar; transformar; reformar | demitir; remover do cargo; expulsar}
  \end{Phonetics}
\end{Entry}

\begin{Entry}{革命}{9,8}{⾰,⼝}
  \begin{Phonetics}{革命}{ge2ming4}[][HSK 7-9]
    \definition{adj.}{revolucionário}
    \definition[次,场]{s.}{revolução; a classe oprimida toma o poder pela violência, destrói o antigo sistema social decadente e estabelece um novo sistema social progressista; a revolução destrói as antigas relações de produção, estabelece novas relações de produção, libera as forças produtivas e promove o desenvolvimento social}
    \definition{v.}{participar da revolução; originalmente se referia à reforma do Mandato do Céu, ou seja, à mudança de dinastias; agora, refere"-se à classe oprimida usando a violência para tomar o poder, destruir o antigo sistema social, estabelecer um novo sistema social e promover o desenvolvimento social}
  \end{Phonetics}
\end{Entry}

\begin{Entry}{革新}{9,13}{⾰,⽄}
  \begin{Phonetics}{革新}{ge2xin1}[][HSK 6]
    \definition{v.}{inovar; renovar; livrar"-se do velho e criar o novo}
  \end{Phonetics}
\end{Entry}

%%%%%%%%%% 韭 %%%%%%%%%%
\subsection*{韭}\addcontentsline{loh}{figure}{韭}

\begin{Entry}{韭}{9}{⾲}[Kangxi 179]
  \begin{Phonetics}{韭}{jiu3}
    \definition{s.}{alho de flor perfumada; cebolinha chinesa}
  \end{Phonetics}
\end{Entry}

\begin{Entry}{韭菜}{9,11}{⾲,⾋}
  \begin{Phonetics}{韭菜}{jiu3cai4}
    \definition{s.}{cebolinha-de-alho; cebolinha chinesa | Coloquial: pessoas ingênuas ou facilmente exploráveis, especialmente pequenos investidores ou consumidores, comparadas à cebolinha, que pode ser colhida repetidamente para obter lucro}
  \end{Phonetics}
\end{Entry}

%%%%%%%%%% 音 %%%%%%%%%%
\subsection*{音}\addcontentsline{loh}{figure}{音}

\begin{Entry}{音}{9}{⾳}[Kangxi 180]
  \begin{Phonetics}{音}{yin1}
    \definition[个,种]{s.}{som; som musical | notícias; novidades; informação | tom; refere"-se especificamente a uma sílaba ou fonética | sílaba; refere"-se a sílabas (um caractere chinês é uma sílaba)}
    \definition{v.}{vocalizar}
  \end{Phonetics}
\end{Entry}

\begin{Entry}{音乐}{9,5}{⾳,⼃}
  \begin{Phonetics}{音乐}{yin1yue4}[][HSK 2]
    \definition[种,段,张,曲]{s.}{música; ramo da arte que cria imagens artísticas, expressa pensamentos e sentimentos e reflete a vida real por meio da melodia e do ritmo da música; geralmente é dividido em duas categorias: música vocal e música instrumental}
  \end{Phonetics}
\end{Entry}

\begin{Entry}{音乐厅}{9,5,4}{⾳,⼃,⼚}
  \begin{Phonetics}{音乐厅}{yin1yue4ting1}
    \definition{s.}{auditório | teatro | \emph{concert hall}}
  \end{Phonetics}
\end{Entry}

\begin{Entry}{音乐节}{9,5,5}{⾳,⼃,⾋}
  \begin{Phonetics}{音乐节}{yin1yue4jie2}
    \definition{s.}{festival de música}
  \end{Phonetics}
\end{Entry}

\begin{Entry}{音乐会}{9,5,6}{⾳,⼃,⼈}
  \begin{Phonetics}{音乐会}{yin1yue4hui4}[][HSK 2]
    \definition[场]{s.}{concerto; atividades de execução de obras musicais}
  \end{Phonetics}
\end{Entry}

\begin{Entry}{音乐光碟}{9,5,6,14}{⾳,⼃,⼉,⽯}
  \begin{Phonetics}{音乐光碟}{yin1yue4 guang1die2}
    \definition{s.}{CD de música}
  \end{Phonetics}
\end{Entry}

\begin{Entry}{音乐学}{9,5,8}{⾳,⼃,⼦}
  \begin{Phonetics}{音乐学}{yin1yue4xue2}
    \definition{s.}{musicologia}
  \end{Phonetics}
\end{Entry}

\begin{Entry}{音乐学院}{9,5,8,9}{⾳,⼃,⼦,⾩}
  \begin{Phonetics}{音乐学院}{yin1yue4xue2yuan4}
    \definition{s.}{conservatório | academia de música}
  \end{Phonetics}
\end{Entry}

\begin{Entry}{音乐院}{9,5,9}{⾳,⼃,⾩}
  \begin{Phonetics}{音乐院}{yin1yue4yuan4}
    \definition{s.}{conservatório | instituto de música}
  \end{Phonetics}
\end{Entry}

\begin{Entry}{音乐家}{9,5,10}{⾳,⼃,⼧}
  \begin{Phonetics}{音乐家}{yin1yue4jia1}
    \definition{s.}{músico}
  \end{Phonetics}
\end{Entry}

\begin{Entry}{音节}{9,5}{⾳,⾋}
  \begin{Phonetics}{音节}{yin1jie2}[][HSK 2]
    \definition{s.}{sílaba}
  \end{Phonetics}
\end{Entry}

\begin{Entry}{音量}{9,12}{⾳,⾥}
  \begin{Phonetics}{音量}{yin1liang4}[][HSK 6]
    \definition[把]{s.}{volume; volume do som; a força de um som}
  \end{Phonetics}
\end{Entry}

\begin{Entry}{音像}{9,13}{⾳,⼈}
  \begin{Phonetics}{音像}{yin1xiang4}[][HSK 6]
    \definition{s.}{audiovisual; produtos audiovisuais; o nome coletivo para gravações de áudio e vídeo}
  \end{Phonetics}
\end{Entry}

%%%%%%%%%% 项 %%%%%%%%%%
\subsection*{项}\addcontentsline{loh}{figure}{项}

\begin{Entry}{项}{9}{⾴}
  \begin{Phonetics}{项}{xiang4}[][HSK 4]
    \definition*{s.}{Sobrenome: Xiang}
    \definition{clas.}{usado para itens discriminados; taxonomia}
    \definition{s.}{nuca (do pescoço); a parte de trás do pescoço | soma (de dinheiro); fundos para fins especiais | termo; em álgebra, significa uma única fórmula que não é unida por um sinal de mais ou de menos | item}
  \end{Phonetics}
\end{Entry}

\begin{Entry}{项目}{9,5}{⾴,⽬}
  \begin{Phonetics}{项目}{xiang4mu4}[][HSK 4]
    \definition{s.}{evento; categorias em que as coisas são divididas | item; projeto; trabalhos de engenharia, acadêmicos, etc., de conteúdo específico}
  \end{Phonetics}
\end{Entry}

\begin{Entry}{项羽}{9,6}{⾴,⽻}
  \begin{Phonetics}{项羽}{xiang4 yu3}
    \definition*{s.}{Xiang Yu, o Conquistador (232-202 a.C.), senhor da guerra derrotado pelo primeiro imperador Han}
  \end{Phonetics}
\end{Entry}

%%%%%%%%%% 顺 %%%%%%%%%%
\subsection*{顺}\addcontentsline{loh}{figure}{顺}

\begin{Entry}{顺}{9}{⾴}
  \begin{Phonetics}{顺}{shun4}[][HSK 6]
    \definition{adj.}{(de escritos) legível; claro e bem escrito; organizado | favorável; harmonioso | favorável; bem-sucedido}
    \definition{prep.}{conforme a conveniência de alguém | ao longo; a introdução da rota, situação ou oportunidade que a ação segue pode ser seguida por 着 | com a corrente; na mesma direção |  com; na mesma direção que}
    \definition{v.}{organizar; colocar em ordem; tornar as coisas organizadas ou ordenadas | obedecer; ceder a; agir em submissão a | ser adequado; ser agradável}
  \seealsoref{着}{zhe5}
  \end{Phonetics}
\end{Entry}

\begin{Entry}{顺从}{9,4}{⾴,⼈}
  \begin{Phonetics}{顺从}{shun4cong2}[][HSK 7-9]
    \definition{v.}{obedecer; cumprir com; agir de acordo com os desejos dos outros}
  \synonymref{服从}{fu2cong2}
  \antonymref{抵制}{di3zhi4}
  \antonymref{对峙}{dui4zhi4}
  \antonymref{反抗}{fan3kang4}
  \antonymref{固执}{gu4zhi5}
  \antonymref{抗拒}{kang4ju4}
  \antonymref{克服}{ke4fu2}
  \antonymref{迁就}{qian1jiu4}
  \antonymref{挑衅}{tiao3xin4}
  \antonymref{挣扎}{zheng1zha2}
  \end{Phonetics}
\end{Entry}

\begin{Entry}{顺心}{9,4}{⾴,⼼}
  \begin{Phonetics}{顺心}{shun4/xin1}[][HSK 7-9]
    \definition{v.+compl.}{estar satisfeito; estar feliz}
  \synonymref{如意}{ru2/yi4}
  \synonymref{舒服}{shu1fu5}
  \synonymref{顺眼}{shun4yan3}
  \antonymref{别扭}{bie4niu5}
  \end{Phonetics}
\end{Entry}

\begin{Entry}{顺手}{9,4}{⾴,⼿}
  \begin{Phonetics}{顺手}{shun4shou3}[][HSK 7-9]
    \definition{adj.}{suave; sem dificuldade; tudo correu bem, sem obstáculos | prático; conveniente e fácil de usar}
    \definition{adv.}{suavemente; a propósito; aliás; juntamente com}
  \synonymref{顺利}{shun4li4}
  \synonymref{随手}{sui2shou3}
  \antonymref{棘手}{ji2shou3}
  \end{Phonetics}
\end{Entry}

\begin{Entry}{顺水}{9,4}{⾴,⽔}
  \begin{Phonetics}{顺水}{shun4shui3}
    \definition{v.}{ir com o fluxo}
  \end{Phonetics}
\end{Entry}

\begin{Entry}{顺延}{9,6}{⾴,⼵}
  \begin{Phonetics}{顺延}{shun4yan2}
    \definition{v.}{adiar | procrastinar}
  \end{Phonetics}
\end{Entry}

\begin{Entry}{顺当}{9,6}{⾴,⼹}
  \begin{Phonetics}{顺当}{shun4dang5}
    \definition{adv.}{suavemente}
  \end{Phonetics}
\end{Entry}

\begin{Entry}{顺耳}{9,6}{⾴,⽿}
  \begin{Phonetics}{顺耳}{shun4'er3}
    \definition{adj.}{agradável ao ouvido}
  \end{Phonetics}
\end{Entry}

\begin{Entry}{顺利}{9,7}{⾴,⼑}
  \begin{Phonetics}{顺利}{shun4li4}[][HSK 2]
    \definition{adj.}{sem problemas; com sucesso; sem dificuldades; sem contratempos; sem obstáculos; sem obstáculos ou dificuldades significativas no desempenho das tarefas}
  \end{Phonetics}
\end{Entry}

\begin{Entry}{顺序}{9,7}{⾴,⼴}
  \begin{Phonetics}{顺序}{shun4xu4}[][HSK 4]
    \definition{adv.}{por sua vez; na ordem correta; na devida ordem; na ordem adequada; na ordem apropriada}
    \definition[个]{s.}{ordem; sequência; sucessão; subsequência; sequência simples; ordem de prioridade}
  \end{Phonetics}
\end{Entry}

\begin{Entry}{顺应}{9,7}{⾴,⼴}
  \begin{Phonetics}{顺应}{shun4ying4}[][HSK 7-9]
    \definition{v.}{cumprir com; estar em conformidade com | ajustar; obedecer; adaptar}
  \synonymref{适合}{shi4he2}
  \synonymref{适应}{shi4ying4}
  \end{Phonetics}
\end{Entry}

\begin{Entry}{顺其自然}{9,8,6,12}{⾴,⼋,⾃,⽕}
  \begin{Phonetics}{顺其自然}{shun4qi2zi4ran2}[][HSK 7-9]
    \definition{expr.}{``Deixe a natureza seguir seu curso.''; de acordo com sua tendência natural; significa seguir o curso natural das coisas sem interferência ou coerção}
  \end{Phonetics}
\end{Entry}

\begin{Entry}{顺势}{9,8}{⾴,⼒}
  \begin{Phonetics}{顺势}{shun4shi4}[][HSK 7-9]
    \definition{adv.}{aproveitar uma oportunidade (proporcionada por uma jogada imprudente do oponente); deixe"-se levar; aproveite a oportunidade | convenientemente; de ​​passagem | conforme a conveniência de alguém; (fazer algo) sem se esforçar muito; aliás; por acaso}
  \synonymref{惯性}{guan4xing4}
  \end{Phonetics}
\end{Entry}

\begin{Entry}{顺畅}{9,8}{⾴,⽥}
  \begin{Phonetics}{顺畅}{shun4chang4}[][HSK 7-9]
    \definition{adj.}{suave; desimpedido; suave e desobstruído}
  \synonymref{畅通}{chang4tong1}
  \synonymref{流畅}{liu2chang4}
  \synonymref{流利}{liu2li4}
  \antonymref{堵塞}{du3se4}
  \end{Phonetics}
\end{Entry}

\begin{Entry}{顺便}{9,9}{⾴,⼈}
  \begin{Phonetics}{顺便}{shun4bian4}[][HSK 7-9]
    \definition{adv.}{convenientemente; de ​​passagem; incidentalmente; a propósito; (a propósito) enquanto faz outra coisa (faça outra coisa)}
  \synonymref{趁机}{chen4ji1}
  \synonymref{附带}{fu4dai4}
  \synonymref{随时}{sui2shi2}
  \antonymref{特别}{te4bie2}
  \antonymref{特地}{te4di4}
  \antonymref{特意}{te4yi4}
  \antonymref{专门}{zhuan1men2}
  \end{Phonetics}
\end{Entry}

\begin{Entry}{顺叙}{9,9}{⾴,⼜}
  \begin{Phonetics}{顺叙}{shun4xu4}
    \definition{s.}{narrativa cronológica}
  \end{Phonetics}
\end{Entry}

\begin{Entry}{顺差}{9,9}{⾴,⼯}
  \begin{Phonetics}{顺差}{shun4cha1}[][HSK 7-9]
    \definition{s.}{um excedente; um saldo favorável; a balança comercial no comércio internacional refere"-se à diferença entre o valor total das exportações e o valor total das importações}
  \end{Phonetics}
\end{Entry}

\begin{Entry}{顺理成章}{9,11,6,11}{⾴,⽟,⼽,⾳}
  \begin{Phonetics}{顺理成章}{shun4li3-cheng2zhang1}[][HSK 7-9]
    \definition{expr.}{desenvolve"-se naturalmente; faz todo o sentido; descreve a escrita ou a execução de tarefas de forma clara e organizada}
  \synonymref{理所当然}{li3suo3dang1ran2}
  \synonymref{理直气壮}{li3zhi2-qi4zhuang4}
  \end{Phonetics}
\end{Entry}

\begin{Entry}{顺眼}{9,11}{⾴,⽬}
  \begin{Phonetics}{顺眼}{shun4yan3}
    \definition{adj.}{agradável aos olhos}
  \end{Phonetics}
\end{Entry}

\begin{Entry}{顺着}{9,11}{⾴,⽬}
  \begin{Phonetics}{顺着}{shun4zhe5}[][HSK 7-9]
    \definition{v.}{deslocar-se ao longo de uma determinada rota ou direção | falar e agir de acordo com os desejos dos outros}
  \end{Phonetics}
\end{Entry}

\begin{Entry}{顺路}{9,13}{⾴,⾜}
  \begin{Phonetics}{顺路}{shun4lu4}[][HSK 7-9]
    \definition{adj.}{ser uma rota direta; significa que a estrada é lisa e desobstruída; também se diz que é conveniente caminhar por ela, ou ``estar a caminho''}
    \definition{adv.}{a caminho; (a caminho) seguindo a rota que fizemos (para outro lugar)}
  \antonymref{迷路}{mi2/lu4}
  \end{Phonetics}
\end{Entry}

\begin{Entry}{顺境}{9,14}{⾴,⼟}
  \begin{Phonetics}{顺境}{shun4jing4}
    \definition{s.}{circunstâncias fáceis (ou favoráveis)}
  \antonymref{逆境}{ni4jing4}
  \end{Phonetics}
\end{Entry}

\begin{Entry}{顺嘴}{9,16}{⾴,⼝}
  \begin{Phonetics}{顺嘴}{shun4zui3}
    \definition{v.}{deixar escapar (sem pensar) | ler suavemente (texto) | adequar-se  ao gosto (comida)}
  \end{Phonetics}
\end{Entry}

%%%%%%%%%% 飒 %%%%%%%%%%
\subsection*{飒}\addcontentsline{loh}{figure}{飒}

\begin{Entry}{飒}{9}{⾵}
  \begin{Phonetics}{飒}{sa4}
    \definition{adj.}{(mulheres) natural e desenfreada; elegante; valente}
    \definition{interj.}{Onomatopéia: farfalhar; sussurrar | Onomatopéia: som do vento}
    \definition{v.}{murchar}
  \end{Phonetics}
\end{Entry}

\begin{Entry}{飒飒}{9,9}{⾵,⾵}
  \begin{Phonetics}{飒飒}{sa4sa4}
    \definition{s.}{o farfalhar | sussurro | murmúrio (do vento nas árvores, o mar, etc.)}
  \end{Phonetics}
\end{Entry}

%%%%%%%%%% 食 %%%%%%%%%%
\subsection*{食}\addcontentsline{loh}{figure}{食}

\begin{Entry}{食}{9}{⾷}[Kangxi 184]
  \begin{Phonetics}{食}{shi2}
    \definition{adj.}{para cozinhar; comestível}
    \definition{s.}{refeição; comida; o que as pessoas e os animais comem | alimentação; alimento para animais; ração | eclipse solar; eclipse lunar}
    \definition{v.}{comer}
  \end{Phonetics}
  \begin{Phonetics}{食}{si4}
    \definition{v.}{alimentar; dar comida a}
  \end{Phonetics}
\end{Entry}

\begin{Entry}{食用}{9,5}{⾷,⽤}
  \begin{Phonetics}{食用}{shi2yong4}[][HSK 7-9]
    \definition{adj.}{comestível; que pode ser usado como alimento.}
    \definition{v.}{usar como alimento; ser comestível}
  \end{Phonetics}
\end{Entry}

\begin{Entry}{食物}{9,8}{⾷,⽜}
  \begin{Phonetics}{食物}{shi2wu4}[][HSK 2]
    \definition[种]{s.}{comida; alimentos; comestíveis}
  \end{Phonetics}
\end{Entry}

\begin{Entry}{食品}{9,9}{⾷,⼝}
  \begin{Phonetics}{食品}{shi2pin3}[][HSK 3]
    \definition[种]{s.}{comida; gêneros alimentícios; provisões; alimentos vendidos em lojas que passaram por algum processamento}
  \end{Phonetics}
\end{Entry}

\begin{Entry}{食堂}{9,11}{⾷,⼟}
  \begin{Phonetics}{食堂}{shi2tang2}[][HSK 4]
    \definition[个,间]{s.}{cantina; refeitório}
  \end{Phonetics}
\end{Entry}

\begin{Entry}{食宿}{9,11}{⾷,⼧}
  \begin{Phonetics}{食宿}{shi2su4}[][HSK 7-9]
    \definition{s.}{pensão e alojamento; alimentação e acomodação | alojamento e alimentação}
  \end{Phonetics}
\end{Entry}

\begin{Entry}{食欲}{9,11}{⾷,⽋}
  \begin{Phonetics}{食欲}{shi2yu4}[][HSK 6]
    \definition{adj.}{apetitoso}
    \definition{s.}{apetite; desejo humano de comer}
  \end{Phonetics}
\end{Entry}

%%%%%%%%%% 饶 %%%%%%%%%%
\subsection*{饶}\addcontentsline{loh}{figure}{饶}

\begin{Entry}{饶}{9}{⾷}
  \begin{Phonetics}{饶}{rao2}[][HSK 7-9]
    \definition{adj.}{rico; abundante; farto}
    \definition{conj.}{embora; apesar do fato de que; indica concessão, com significado semelhante a 虽然 ou 尽管}
    \definition{v.}{ter misericórdia de; absolver alguém; perdoar; absolver | dar algo a mais; permitir que alguém receba algo em troca | desculpar; perdoar; tolerar}
  \seealsoref{尽管}{jin3guan3}
  \seealsoref{虽然}{sui1ran2}
  \end{Phonetics}
\end{Entry}

\begin{Entry}{饶恕}{9,10}{⾷,⼼}
  \begin{Phonetics}{饶恕}{rao2shu4}[][HSK 7-9]
    \definition{v.}{desculpar; absolver; perdoar; não punir quando a punição é devida}
  \end{Phonetics}
\end{Entry}

%%%%%%%%%% 饺 %%%%%%%%%%
\subsection*{饺}\addcontentsline{loh}{figure}{饺}

\begin{Entry}{饺}{9}{⾷}
  \begin{Phonetics}{饺}{jiao3}
    \definition[盘,碗,顿,个]{s.}{bolinho de massa; \emph{dumpling}}
  \end{Phonetics}
\end{Entry}

\begin{Entry}{饺子}{9,3}{⾷,⼦}
  \begin{Phonetics}{饺子}{jiao3zi5}[][HSK 2]
    \definition[个,盘,碗,锅]{s.}{jiaozi; bolinho chinês; bolinho de massa}
  \end{Phonetics}
\end{Entry}

%%%%%%%%%% 饼 %%%%%%%%%%
\subsection*{饼}\addcontentsline{loh}{figure}{饼}

\begin{Entry}{饼}{9}{⾷}
  \begin{Phonetics}{饼}{bing3}[][HSK 5]
    \definition[张]{s.}{um bolo redondo e plano; massa assada ou cozida no vapor | algo que tem o formato de um bolo; semelhante a uma torta}
  \end{Phonetics}
\end{Entry}

\begin{Entry}{饼干}{9,3}{⾷,⼲}
  \begin{Phonetics}{饼干}{bing3gan1}[][HSK 5]
    \definition[块,片,包,盒,袋]{s.}{biscoito; bolacha; \emph{cookie}; alimentos, pedaços pequenos e finos cozidos em farinha com açúcar, ovos, leite, etc.}
  \end{Phonetics}
\end{Entry}

%%%%%%%%%% 首 %%%%%%%%%%
\subsection*{首}\addcontentsline{loh}{figure}{首}

\begin{Entry}{首}{9}{⾸}[Kangxi 185]
  \begin{Phonetics}{首}{shou3}[][HSK 4,6]
    \definition*{s.}{Sobrenome: Shou}
    \definition{adj.}{primeiro}
    \definition{adv.}{inicialmente; como o primeiro; em primeiro lugar}
    \definition{clas.}{usado para canções e poemas}
    \definition{s.}{cabeça | cabeça; chefe; líder | capital (cidade)}
    \definition{v.}{apresentar acusações contra alguém}
  \end{Phonetics}
\end{Entry}

\begin{Entry}{首先}{9,6}{⾸,⼉}
  \begin{Phonetics}{首先}{shou3xian1}[][HSK 3]
    \definition{adv.}{primeiramente; antes de todos os outros}
    \definition{conj.}{acima de tudo; primeiramente; em primeiro lugar}
  \end{Phonetics}
\end{Entry}

\begin{Entry}{首创}{9,6}{⾸,⼑}
  \begin{Phonetics}{首创}{shou3chuang4}[][HSK 7-9]
    \definition{v.}{originar; iniciar; ser pioneiro}
  \synonymref{创办}{chuang4ban4}
  \synonymref{开创}{kai1chuang4}
  \antonymref{模仿}{mo2fang3}
  \end{Phonetics}
\end{Entry}

\begin{Entry}{首次}{9,6}{⾸,⽋}
  \begin{Phonetics}{首次}{shou3ci4}[][HSK 6]
    \definition{s.}{o primeiro; pela primeira vez}
  \end{Phonetics}
\end{Entry}

\begin{Entry}{首批}{9,7}{⾸,⼿}
  \begin{Phonetics}{首批}{shou3pi1}[][HSK 7-9]
    \definition{adj./adv.}{primeiro lote}
  \end{Phonetics}
\end{Entry}

\begin{Entry}{首府}{9,8}{⾸,⼴}
  \begin{Phonetics}{首府}{shou3fu3}[][HSK 7-9]
    \definition{s.}{prefeitura principal (a prefeitura onde se localizava a capital da província); anteriormente se referia à localização da capital provincial | capital de uma região autônoma ou prefeitura; hoje em dia, geralmente se refere à sede do governo de uma região autônoma ou prefeitura autônoma | capital de uma dependência ou colônia
A sede do governo supremo de um estado dependente, colônia ou território sob tutela}
  \synonymref{省城}{sheng3cheng2}
  \end{Phonetics}
\end{Entry}

\begin{Entry}{首饰}{9,8}{⾸,⾷}
  \begin{Phonetics}{首饰}{shou3shi4}[][HSK 7-9]
    \definition[件,套,串]{s.}{joias; adorno de cabeça; originalmente referindo-se a ornamentos usados ​​na cabeça, agora geralmente se refere a brincos, colares, anéis, pulseiras, etc.}
  \end{Phonetics}
\end{Entry}

\begin{Entry}{首相}{9,9}{⾸,⽬}
  \begin{Phonetics}{首相}{shou3xiang4}[][HSK 6]
    \definition*[个,名,位]{s.}{Primeiro-Ministro (Japão, UK, etc.); o mais alto cargo oficial no gabinete de uma monarquia; o chefe do governo central de alguns países não monárquicos às vezes usa esse nome}
  \end{Phonetics}
\end{Entry}

\begin{Entry}{首要}{9,9}{⾸,⾑}
  \begin{Phonetics}{首要}{shou3yao4}[][HSK 7-9]
    \definition{adj.}{principal; primordial; de suma importância; prioridade máxima | líder}
  \synonymref{严重}{yan2zhong4}
  \synonymref{重要}{zhong4yao4}
  \synonymref{主体}{zhu3ti3}
  \synonymref{主要}{zhu3yao4}
  \end{Phonetics}
\end{Entry}

\begin{Entry}{首席}{9,10}{⾸,⼱}
  \begin{Phonetics}{首席}{shou3xi2}[][HSK 6]
    \definition{adj.}{chefe; a primeira; a posição mais alta}
    \definition{s.}{assento de honra; o assento mais honroso}
  \end{Phonetics}
\end{Entry}

\begin{Entry}{首席执行官}{9,10,6,6,8}{⾸,⼱,⼿,⾏,⼧}
  \begin{Phonetics}{首席执行官}{shou3xi2 zhi2xing2 guan1}
    \definition{s.}{\emph{chief executive officer}, CEO}
  \end{Phonetics}
\end{Entry}

\begin{Entry}{首脑}{9,10}{⾸,⾁}
  \begin{Phonetics}{首脑}{shou3nao3}[][HSK 6]
    \definition[位]{s.}{cabeça; líder; chefe}
  \end{Phonetics}
\end{Entry}

\begin{Entry}{首都}{9,10}{⾸,⾢}
  \begin{Phonetics}{首都}{shou3du1}[][HSK 3]
    \definition[个,座]{s.}{capital (cidade); a sede do mais alto poder político do país e o centro político do país}
  \end{Phonetics}
\end{Entry}

%%%%%%%%%% 香 %%%%%%%%%%
\subsection*{香}\addcontentsline{loh}{figure}{香}

\begin{Entry}{香}{9}{⾹}[Kangxi 186]
  \begin{Phonetics}{香}{xiang1}[][HSK 3]
    \definition*{s.}{Sobrenome: Xiang}
    \definition{adj.}{aromático; perfumado; fragrante; cheiroso | saboroso; saboroso; delicioso; apetitoso | com gosto; com bom apetite | (sono) profundo; dormir confortavelmente e tranquilamente | popular; valorizado; apreciado}
    \definition[根,炷]{s.}{especiaria; perfume; fragrância; aromatizante; substância com aroma intenso | incenso; bastão de incenso; tiras finas feitas de serragem e especiarias, queimadas em rituais para honrar os antepassados ou deuses e budas, e também usadas para afastar odores desagradáveis ou mosquitos| antigamente, referia"-se a coisas relacionadas com mulheres}
  \antonymref{臭}{chou4}
  \end{Phonetics}
\end{Entry}

\begin{Entry}{香气}{9,4}{⾹,⽓}
  \begin{Phonetics}{香气}{xiang1qi4}
    \definition{s.}{fragrância | aroma | incenso}
  \end{Phonetics}
\end{Entry}

\begin{Entry}{香皂}{9,7}{⾹,⽩}
  \begin{Phonetics}{香皂}{xiang1zao4}
    \definition{s.}{sabonete | sabonete perfumado}
  \end{Phonetics}
\end{Entry}

\begin{Entry}{香肠}{9,7}{⾹,⾁}
  \begin{Phonetics}{香肠}{xiang1chang2}[][HSK 5]
    \definition[根]{s.}{salsicha; linguiça; alimento feito com intestino de porco, recheado com carne picada e temperos}
  \end{Phonetics}
\end{Entry}

\begin{Entry}{香味}{9,8}{⾹,⼝}
  \begin{Phonetics}{香味}{xiang1wei4}
    \definition[股]{s.}{fragrância | cheiro doce}
  \end{Phonetics}
\end{Entry}

\begin{Entry}{香波}{9,8}{⾹,⽔}
  \begin{Phonetics}{香波}{xiang1bo1}
    \definition{s.}{xampu}
  \end{Phonetics}
\end{Entry}

\begin{Entry}{香炉}{9,8}{⾹,⽕}
  \begin{Phonetics}{香炉}{xiang1lu2}
    \definition{s.}{incensário (para queimar incenso) | queimador de incenso | insensório, turíbulo}
  \end{Phonetics}
\end{Entry}

\begin{Entry}{香烟}{9,10}{⾹,⽕}
  \begin{Phonetics}{香烟}{xiang1yan1}
    \definition[支,条]{s.}{cigarro | fumaça de incenso queimado}
  \end{Phonetics}
\end{Entry}

\begin{Entry}{香艳}{9,10}{⾹,⾊}
  \begin{Phonetics}{香艳}{xiang1yan4}
    \definition{adj.}{atraente | erótico | romântico}
  \end{Phonetics}
\end{Entry}

\begin{Entry}{香港}{9,12}{⾹,⽔}
  \begin{Phonetics}{香港}{xiang1gang3}
    \definition*{s.}{Hong Kong}
  \seealsoref{香港岛}{xiang1gang3 dao3}
  \end{Phonetics}
\end{Entry}

\begin{Entry}{香港岛}{9,12,7}{⾹,⽔,⼭}
  \begin{Phonetics}{香港岛}{xiang1gang3 dao3}
    \definition*{s.}{Ilha de Hong Kong}
  \seealsoref{香港}{xiang1gang3}
  \end{Phonetics}
\end{Entry}

\begin{Entry}{香槟酒}{9,14,10}{⾹,⽊,⾣}
  \begin{Phonetics}{香槟酒}{xiang1bin1jiu3}
    \definition[杯]{s.}{(empréstimo linguístico) \emph{champagne}}
  \end{Phonetics}
\end{Entry}

\begin{Entry}{香蕈}{9,15}{⾹,⾋}
  \begin{Phonetics}{香蕈}{xiang1xun4}
    \definition{s.}{\emph{shiitake}, cogumelo comestível}
  \end{Phonetics}
\end{Entry}

\begin{Entry}{香蕉}{9,15}{⾹,⾋}
  \begin{Phonetics}{香蕉}{xiang1jiao1}[][HSK 3]
    \definition[枝,根,个,把,串,束,弓]{s.}{banana}
  \end{Phonetics}
\end{Entry}

%%%%%%%%%% 骂 %%%%%%%%%%
\subsection*{骂}\addcontentsline{loh}{figure}{骂}

\begin{Entry}{骂}{9}{⾺}
  \begin{Phonetics}{骂}{ma4}[][HSK 5]
    \definition{v.}{abusar; xingar; insultar; insultar alguém com palavras grosseiras ou maliciosas | repreender; censurar; condenar}
  \end{Phonetics}
\end{Entry}

\begin{Entry}{骂名}{9,6}{⾺,⼝}
  \begin{Phonetics}{骂名}{ma4ming2}
    \definition{s.}{infâmia}
  \end{Phonetics}
\end{Entry}

\begin{Entry}{骂街}{9,12}{⾺,⾏}
  \begin{Phonetics}{骂街}{ma4jie1}
    \definition{v.}{gritar abusos na rua}
  \end{Phonetics}
\end{Entry}

%%%%%%%%%% 骄 %%%%%%%%%%
\subsection*{骄}\addcontentsline{loh}{figure}{骄}

\begin{Entry}{骄}{9}{⾺}
  \begin{Phonetics}{骄}{jiao1}
    \definition{adj.}{orgulhoso; arrogante; vaidoso | Literário: feroz; intenso; forte; violento}
  \end{Phonetics}
\end{Entry}

\begin{Entry}{骄傲}{9,12}{⾺,⼈}
  \begin{Phonetics}{骄傲}{jiao1'ao4}[][HSK 6]
    \definition{adj.}{arrogante; vaidoso; orgulhoso}
    \definition{s.}{orgulho; pessoas ou coisas das quais se orgulhar}
  \end{Phonetics}
\end{Entry}

%%%%%%%%%% 骆 %%%%%%%%%%
\subsection*{骆}\addcontentsline{loh}{figure}{骆}

\begin{Entry}{骆}{9}{⾺}
  \begin{Phonetics}{骆}{luo4}
    \definition*{s.}{Sobrenome: Luo}
    \definition[只]{s.}{Arcaico: um cavalo branco com crina preta, mencionado em antigos livros chineses}
  \end{Phonetics}
\end{Entry}

\begin{Entry}{骆驼}{9,8}{⾺,⾺}
  \begin{Phonetics}{骆驼}{luo4tuo5}
    \definition[头,只,匹]{s.}{camelo | coloquial: cabeça-dura, idiota}
  \end{Phonetics}
\end{Entry}

%%%%%%%%%% 骇 %%%%%%%%%%
\subsection*{骇}\addcontentsline{loh}{figure}{骇}

\begin{Entry}{骇}{9}{⾺}
  \begin{Phonetics}{骇}{hai4}
    \definition{adj.}{assustado; chocado}
    \definition{v.}{ficar surpreso; ficar chocado}
  \end{Phonetics}
\end{Entry}

\begin{Entry}{骇人听闻}{9,2,7,9}{⾺,⼈,⼝,⾨}
  \begin{Phonetics}{骇人听闻}{hai4ren2ting1wen2}[][HSK 7-9]
    \definition{expr.}{chocante; terrível; assustador para os ouvidos; espantoso; fabuloso; chocante (notícia); aterrorizante; horripilante}
  \end{Phonetics}
\end{Entry}

%%%%%%%%%% 骨 %%%%%%%%%%
\subsection*{骨}\addcontentsline{loh}{figure}{骨}

\begin{Entry}{骨}{9}{⾻}[Kangxi 188]
  \begin{Phonetics}{骨}{gu3}
    \definition*{s.}{Sobrenome: Gu}
    \definition[根,块]{s.}{osso | esqueleto; estrutura | caráter; espírito | cadáver; corpo}
  \end{Phonetics}
\end{Entry}

\begin{Entry}{骨干}{9,3}{⾻,⼲}
  \begin{Phonetics}{骨干}{gu3gan4}[][HSK 7-9]
    \definition[名,个,位]{s.}{diáfise; a parte central de um osso longo, conectada à epífise em ambas as extremidades, contém uma cavidade | espinha dorsal; esteio; metaforicamente falando, uma pessoa ou coisa que desempenha um papel importante}
  \end{Phonetics}
\end{Entry}

\begin{Entry}{骨气}{9,4}{⾻,⽓}
  \begin{Phonetics}{骨气}{gu3qi4}[][HSK 7-9]
    \definition[些,种]{s.}{espinha dorsal; integridade moral; força de caráter | vigor dos traços caligráficos; refere"-se ao impulso forte e vertical expresso pela caligrafia}
  \end{Phonetics}
\end{Entry}

\begin{Entry}{骨头}{9,5}{⾻,⼤}
  \begin{Phonetics}{骨头}{gu3tou5}[][HSK 4]
    \definition[根,块]{s.}{osso; tecidos mais duros no corpo de uma pessoa ou de alguns animais que sustentam o corpo ou protegem os órgãos do corpo | caráter de uma pessoa; refere"-se à qualidade do caráter de uma pessoa}
  \end{Phonetics}
\end{Entry}

\begin{Entry}{骨折}{9,7}{⾻,⼿}
  \begin{Phonetics}{骨折}{gu3zhe2}[][HSK 7-9]
    \definition{v.}{sofrer uma fratura; quebrar (um osso)}
  \end{Phonetics}
\end{Entry}

%%%%%%%%%% 鬼 %%%%%%%%%%
\subsection*{鬼}\addcontentsline{loh}{figure}{鬼}

\begin{Entry}{鬼}{9}{⿁}[Kangxi 194]
  \begin{Phonetics}{鬼}{gui3}[][HSK 5]
    \definition*{s.}{Gui, uma das mansões lunares | Gui, a vigésima terceira das vinte e oito constelações em que a esfera celeste foi dividida, consistindo de quatro estrelas em Câncer | Sobrenome: Gui}
    \definition{adj.}{evasivo; furtivo; sub"-reptício; ardiloso; enganoso, malicioso; obscuro | terrível; ruim; severo; vil | esperto; astuto; inteligente}
    \definition{s.}{espírito; fantasma; aparição; refere"-se à alma de uma pessoa após a morte | usado para formar um termo de abuso para caráter ignóbil; refere"-se a pessoas que têm maus hábitos ou cujo comportamento é repugnante | companheiro; pessoa que é considerada divertida}
  \end{Phonetics}
\end{Entry}

\begin{Entry}{鬼火}{9,4}{⿁,⽕}
  \begin{Phonetics}{鬼火}{gui3huo3}
    \definition{s.}{fogo-fátuo | boitatá | fogo corredor | fogo de santelmo}
  \end{Phonetics}
\end{Entry}

\begin{Entry}{鬼怪}{9,8}{⿁,⼼}
  \begin{Phonetics}{鬼怪}{gui3guai4}
    \definition{s.}{\emph{hobgoblin} | bicho-papão | fantasma}
  \end{Phonetics}
\end{Entry}

%%%%% EOF %%%%%


 %%%
%%% 10画
%%%
\section*{10画}\addcontentsline{toc}{section}{10画}

%%%%%%%%%% 乘 %%%%%%%%%%
\subsection*{乘}

\begin{Entry}{乘}{10}{⽲}
  \begin{Phonetics}{乘}{cheng2}[][HSK 5]
    \definition*{s.}{Sobrenome: Cheng}
    \definition{s.}{uma divisão principal das escolas budistas; uma seita ou doutrina do budismo}
    \definition{v.}{cavalgar; andar a cavalo; utilizar um veículo ou animal em vez de caminhar | aproveitar-se de; valer-se de; tirar vantagem de; tirar proveito de | multiplicar; realizar multiplicação | perseguir; caçar}
  \end{Phonetics}
  \begin{Phonetics}{乘}{sheng4}
    \definition{clas.}{usado para carruagens de guerra puxada por quatro cavalos}
    \definition{s.}{obras históricas; livros de história geral | antigamente, uma carruagem puxada por quatro cavalos}
  \end{Phonetics}
\end{Entry}

\begin{Entry}{乘人之危}{10,2,3,6}{⽲、⼈、⼂、⼙}
  \begin{Phonetics}{乘人之危}{cheng2ren2zhi1wei1}[][HSK 7-9]
    \definition{expr.}{tirar vantagem das dificuldades dos outros (posição precária; problema); capitalizar as dificuldades de alguém (desastres); fazer uso (utilizar) da situação precária em que alguém se encontra; tentar usar o dilema de alguém para\dots; aproveitar-se da angústia dos outros para prejudicá-los; atacar em um momento de crise}
  \end{Phonetics}
\end{Entry}

\begin{Entry}{乘车}{10,4}{⽲、⾞}
  \begin{Phonetics}{乘车}{cheng2 che1}[][HSK 5]
    \definition{v.}{montar; dirigir; conduzir; andar a cavalo, de moto, de bicicleta, etc.}
  \end{Phonetics}
\end{Entry}

\begin{Entry}{乘坐}{10,7}{⽲、⼟}
  \begin{Phonetics}{乘坐}{cheng2zuo4}[][HSK 5]
    \definition{v.}{pegar (um trem, ônibus, etc.); andar de (bicicleta, moto, etc.)}
  \end{Phonetics}
\end{Entry}

\begin{Entry}{乘客}{10,9}{⽲、⼧}
  \begin{Phonetics}{乘客}{cheng2 ke4}[][HSK 5]
    \definition[个,位,名]{s.}{passageiro; pessoas viajando de carro, navio ou avião}
  \end{Phonetics}
\end{Entry}

\begin{Entry}{乘客数}{10,9,13}{⽲、⼧、⽁}
  \begin{Phonetics}{乘客数}{cheng2ke4 shu4}
    \definition{s.}{número de passageiros}
  \end{Phonetics}
\end{Entry}

\begin{Entry}{乘积}{10,10}{⽲、⽲}
  \begin{Phonetics}{乘积}{cheng2ji1}
    \definition{s.}{(matemática) produto (resultado da multiplicação)}
  \end{Phonetics}
\end{Entry}

%%%%%%%%%% 俯 %%%%%%%%%%
\subsection*{俯}

\begin{Entry}{俯}{10}{⼈}
  \begin{Phonetics}{俯}{fu3}
    \definition{v.}{curvar (a cabeça), oposto a 仰 | inclinar-se | Obsoleto: (em documentos ou cartas oficiais) condescender com | curvar-se; fazer uma reverência}
  \seealsoref{仰}{yang3}
  \end{Phonetics}
\end{Entry}

\begin{Entry}{俯首}{10,9}{⼈、⾸}
  \begin{Phonetics}{俯首}{fu3shou3}[][HSK 7-9]
    \definition{v.}{abaixar a cabeça; curvar-se; inclinar-se}
  \end{Phonetics}
\end{Entry}

%%%%%%%%%% 俱 %%%%%%%%%%
\subsection*{俱}

\begin{Entry}{俱}{10}{⼈}
  \begin{Phonetics}{俱}{ju4}
    \definition{adv.}{(literário) tudo; completamente; inteiramente}
  \end{Phonetics}
\end{Entry}

\begin{Entry}{俱乐部}{10,5,10}{⼈、⼃、⾢}
  \begin{Phonetics}{俱乐部}{ju4le4bu4}[][HSK 5]
    \definition[个,家,间]{s.}{clube; grupos e locais para atividades sociais, políticas, literárias, recreativas e outras}
  \end{Phonetics}
\end{Entry}

%%%%%%%%%% 倂 %%%%%%%%%%
\subsection*{倂}

\begin{Entry}{倂}{10}{⼈}
  \begin{Phonetics}{倂}{bing4}
    \variantof{并}
  \end{Phonetics}
\end{Entry}

%%%%%%%%%% 倍 %%%%%%%%%%
\subsection*{倍}

\begin{Entry}{倍}{10}{⼈}
  \begin{Phonetics}{倍}{bei4}[][HSK 4]
    \definition{adv.}{ainda mais; especialmente | (antes de certos adjetivos) muito; particularmente; é pronunciado como um som erhua e é usado antes de certos adjetivos para expressar um alto grau de profundidade, equivalente a 非常 ou 特别}
    \definition{clas.}{vezes; usado após um numeral, significa que o valor anterior é multiplicado por este número}[增长了五倍。===Aumentou cinco vezes. | 二的三倍是六。===Três vezes dois é seis.]
  \seealsoref{非常}{fei1chang2}
  \seealsoref{特别}{te4bie2}
  \end{Phonetics}
\end{Entry}

%%%%%%%%%% 倒 %%%%%%%%%%
\subsection*{倒}

\begin{Entry}{倒}{10}{⼈}
  \begin{Phonetics}{倒}{dao3}[][HSK 2]
    \definition{v.}{cair; tombar | falhar; entrar em colapso | ficar rouco | mudar; trocar; transferir; converter | movimentar-se; manobrar | oferecer (casa, loja) para venda; vender mercadorias ou lojas a terceiros a um preço fixo | derrubar; derrubar com}
  \end{Phonetics}
  \begin{Phonetics}{倒}{dao4}[][HSK 2]
    \definition{adj.}{inverso; invertido; de cabeça para baixo}
    \definition{adv.}{mas; pelo contrário; expressa o contrário do esperado, equivalente a 反倒 | indicando que algo não é o que se pensa; indica que as coisas não são assim | usado para indicar uma transição ou concessão | transmitindo a sensação de ``urgência''; expressa pressa ou insistência, com um tom impaciente}
    \definition{v.}{ser inverso; estar invertido; estar de cabeça para baixo; inverter a posição original para cima e para baixo ou para a frente e para trás | recuar; virar de cabeça para baixo; fazer mover na direção oposta ou inverter | inclinar ou virar o recipiente para retirar o conteúdo; inclinar; derramar}
  \seealsoref{反倒}{fan3dao4}
  \end{Phonetics}
\end{Entry}

\begin{Entry}{倒下}{10,3}{⼈、⼀}
  \begin{Phonetics}{倒下}{dao3xia4}[][HSK 7-9]
    \definition{v.}{entrar em colapso | tombar}
  \end{Phonetics}
\end{Entry}

\begin{Entry}{倒计时}{10,4,7}{⼈、⾔、⽇}
  \begin{Phonetics}{倒计时}{dao4ji4shi2}[][HSK 7-9]
    \definition{s.}{contagem regressiva; contagem de tempo a partir de um determinado ponto no futuro até o presente, de mais para menos, até que o tempo chegue a zero; é frequentemente usado para expressar que um certo momento está se aproximando}
  \end{Phonetics}
\end{Entry}

\begin{Entry}{倒车}{10,4}{⼈、⾞}
  \begin{Phonetics}{倒车}{dao3che1}[][HSK 4]
    \definition{v.}{trocar de trem ou ônibus (no meio do caminho)}
  \end{Phonetics}
  \begin{Phonetics}{倒车}{dao4che1}[][HSK 4]
    \definition{v.}{dar marcha à ré (em um veículo)}
  \end{Phonetics}
\end{Entry}

\begin{Entry}{倒地}{10,6}{⼈、⼟}
  \begin{Phonetics}{倒地}{dao3di4}
    \definition{v.}{cair no chão}
  \end{Phonetics}
\end{Entry}

\begin{Entry}{倒血霉}{10,6,15}{⼈、⾎、⾬}
  \begin{Phonetics}{倒血霉}{dao3xue4mei2}
    \definition{v.}{ter muito azar (versão mais forte de 倒霉)}
  \seealsoref{倒霉}{dao3/mei2}
  \end{Phonetics}
\end{Entry}

\begin{Entry}{倒闭}{10,6}{⼈、⾨}
  \begin{Phonetics}{倒闭}{dao3bi4}[][HSK 4]
    \definition{v.}{fechar; ir à falência; entrar em liquidação; sair do negócio; (empresa, loja ou banco) deixar de operar devido ao baixo desempenho}
  \end{Phonetics}
\end{Entry}

\begin{Entry}{倒卖}{10,8}{⼈、⼗}
  \begin{Phonetics}{倒卖}{dao3mai4}[][HSK 7-9]
    \definition{v.}{revender com lucro}
  \end{Phonetics}
\end{Entry}

\begin{Entry}{倒是}{10,9}{⼈、⽇}
  \begin{Phonetics}{倒是}{dao4 shi4}[][HSK 5]
    \definition{adv.}{usado para indicar o oposto do que geralmente é verdade; ao contrário do senso comum; pelo contrário | usado para indicar o que é contrário aos fatos, com um toque de crítica; indica que as coisas não são assim (com um sentimento de culpa) | usado de algo inesperado; expressando surpresa | usado para indicar concessão | usado para indicar uma mudança de significado; indica um ponto de virada | usado para modificar ou suavizar uma declaração anterior; para suavizar o tom | usado para pressionar ou questionar alguém; para instar ou perguntar}
  \end{Phonetics}
\end{Entry}

\begin{Entry}{倒塌}{10,13}{⼈、⼟}
  \begin{Phonetics}{倒塌}{dao3ta1}[][HSK 7-9]
    \definition{v.}{colapsar; desabar}
  \end{Phonetics}
\end{Entry}

\begin{Entry}{倒数}{10,13}{⼈、⽁}
  \begin{Phonetics}{倒数}{dao4shu3}[][HSK 7-9]
    \definition{v.}{contar de trás para frente (contagem regressiva)}
  \end{Phonetics}
  \begin{Phonetics}{倒数}{dao4shu4}
    \definition{s.}{número inverso; Matemática: recíproco}
  \end{Phonetics}
\end{Entry}

\begin{Entry}{倒楣}{10,13}{⼈、⽊}
  \begin{Phonetics}{倒楣}{dao3mei2}
    \definition{adj.}{azarado; infeliz; tendo má sorte}
  \end{Phonetics}
\end{Entry}

\begin{Entry}{倒霉}{10,15}{⼈、⾬}
  \begin{Phonetics}{倒霉}{dao3/mei2}[][HSK 7-9]
    \definition{adj.}{azarado}
    \definition{s.}{azar; má sorte}
    \definition{v.+compl.}{cair em dias maus; cair em tempos difíceis; encontrar coisas desfavoráveis; ter má sorte}
  \seealsoref{倒血霉}{dao3xue4mei2}
  \end{Phonetics}
\end{Entry}

%%%%%%%%%% 倔 %%%%%%%%%%
\subsection*{倔}

\begin{Entry}{倔}{10}{⼈}
  \begin{Phonetics}{倔}{jue2}
    \definition{adj.}{rude; mal-humorado; abrupto; curto (uso limitado em 倔犟) | teimoso; direto e franco}
  \end{Phonetics}
  \begin{Phonetics}{倔}{jue4}[][HSK 7-9]
    \definition{adj.}{teimoso; direto; rude; grosseiro; de natureza direta, com uma atitude severa em relação aos outros}
  \end{Phonetics}
\end{Entry}

\begin{Entry}{倔强}{10,12}{⼈、⼸}
  \begin{Phonetics}{倔强}{jue2jiang4}[][HSK 7-9]
    \definition{adj.}{teimoso; rígido; inflexível; de personalidade forte e teimosa}
  \end{Phonetics}
\end{Entry}

%%%%%%%%%% 倘 %%%%%%%%%%
\subsection*{倘}

\begin{Entry}{倘}{10}{⼈}
  \begin{Phonetics}{倘}{chang2}
  \end{Phonetics}
  \begin{Phonetics}{倘}{tang3}
    \definition{conj.}{se; supondo; no caso}
  \end{Phonetics}
\end{Entry}

\begin{Entry}{倘使}{10,8}{⼈、⼈}
  \begin{Phonetics}{倘使}{tang3shi3}
    \definition{conj.}{se | supondo que | no caso}
  \end{Phonetics}
\end{Entry}

\begin{Entry}{倘或}{10,8}{⼈、⼽}
  \begin{Phonetics}{倘或}{tang3huo4}
    \definition{conj.}{se | supondo que | no caso}
  \end{Phonetics}
\end{Entry}

\begin{Entry}{倘若}{10,8}{⼈、⾋}
  \begin{Phonetics}{倘若}{tang3ruo4}
    \definition{conj.}{se | supondo que | no caso}
  \end{Phonetics}
\end{Entry}

%%%%%%%%%% 候 %%%%%%%%%%
\subsection*{候}

\begin{Entry}{候}{10}{⼈}
  \begin{Phonetics}{候}{hou4}
    \definition*{s.}{Sobrenome: Hou}
    \definition{s.}{tempo; estação | condição; estado | situação meteorológica | uma unidade tradicional de tempo no antigo calendário chinês; antigamente, cinco dias constituíam uma estação, o que ainda é usado na meteorologia hoje em dia}
    \definition{v.}{esperar; aguardar | perguntar depois | assistir; observar}
  \end{Phonetics}
\end{Entry}

\begin{Entry}{候选人}{10,9,2}{⼈、⾡、⼈}
  \begin{Phonetics}{候选人}{hou4xuan3ren2}[][HSK 7-9]
    \definition[个,名,位]{s.}{candidato}
  \end{Phonetics}
\end{Entry}

%%%%%%%%%% 借 %%%%%%%%%%
\subsection*{借}

\begin{Entry}{借}{10}{⼈}
  \begin{Phonetics}{借}{jie4}[][HSK 2]
    \definition{adv.}{por meio de}
    \definition{v.}{emprestar | pedir emprestado | usar como pretexto | aproveitar; tirar proveito (de uma oportunidade, etc.)}
  \end{Phonetics}
\end{Entry}

\begin{Entry}{借口}{10,3}{⼈、⼝}
  \begin{Phonetics}{借口}{jie4kou3}[][HSK 7-9]
    \definition[个,种]{s.}{desculpa; pretexto; razões falsas apresentadas para atingir um objetivo}
    \definition{v.}{usar como desculpa; usar sob o pretexto de; usar com a justificativa de; usar (algo) como motivo (que não seja um motivo real)}
  \end{Phonetics}
\end{Entry}

\begin{Entry}{借书证}{10,4,7}{⼈、⼄、⾔}
  \begin{Phonetics}{借书证}{jie4shu1zheng4}
    \definition{s.}{cartão da biblioteca; comprovante de solicitação}
  \seealsoref{借书证卡}{jie4shu1zheng4 ka3}
  \end{Phonetics}
\end{Entry}

\begin{Entry}{借书证卡}{10,4,7,5}{⼈、⼄、⾔、⼘}
  \begin{Phonetics}{借书证卡}{jie4shu1zheng4 ka3}
    \definition{s.}{cartão da biblioteca}
  \seealsoref{借书证}{jie4shu1zheng4}
  \end{Phonetics}
\end{Entry}

\begin{Entry}{借用}{10,5}{⼈、⽤}
  \begin{Phonetics}{借用}{jie4yong4}[][HSK 7-9]
    \definition{v.}{tomar emprestado; ter o empréstimo de | usar algo para outro propósito}
  \end{Phonetics}
\end{Entry}

\begin{Entry}{借助}{10,7}{⼈、⼒}
  \begin{Phonetics}{借助}{jie4zhu4}[][HSK 7-9]
    \definition{v.}{contar com a ajuda de; obter apoio de}
  \end{Phonetics}
\end{Entry}

\begin{Entry}{借条}{10,7}{⼈、⽊}
  \begin{Phonetics}{借条}{jie4tiao2}[][HSK 7-9]
    \definition{s.}{recibo de empréstimo; nota promissória}
  \seealsoref{借条儿}{jie4tiao2r5}
  \end{Phonetics}
\end{Entry}

\begin{Entry}{借条儿}{10,7,2}{⼈、⽊、⼉}
  \begin{Phonetics}{借条儿}{jie4tiao2r5}
    \definition{s.}{nota promissória}
  \end{Phonetics}
\end{Entry}

\begin{Entry}{借鉴}{10,13}{⼈、⾦}
  \begin{Phonetics}{借鉴}{jie4jian4}[][HSK 6]
    \definition{s.}{tirar lições de; aproveitar a experiência de; ganhar experiência e lições com o passado ou com as experiências de outras pessoas}
  \end{Phonetics}
\end{Entry}

%%%%%%%%%% 倡 %%%%%%%%%%
\subsection*{倡}

\begin{Entry}{倡}{10}{⼈}
  \begin{Phonetics}{倡}{chang4}
    \definition{v.}{iniciar; propor; defender | promover; assumir a liderança}
  \end{Phonetics}
\end{Entry}

\begin{Entry}{倡议}{10,5}{⼈、⾔}
  \begin{Phonetics}{倡议}{chang4yi4}[][HSK 7-9]
    \definition[项,条,次]{s.}{proposta; iniciativa; primeiras sugestões}
    \definition{v.}{propor; iniciar; defender}
  \end{Phonetics}
\end{Entry}

\begin{Entry}{倡导}{10,6}{⼈、⼨}
  \begin{Phonetics}{倡导}{chang4dao3}[][HSK 5]
    \definition{v.}{iniciar; propor; promover; defender; advogar}
  \end{Phonetics}
\end{Entry}

%%%%%%%%%% 债 %%%%%%%%%%
\subsection*{债}

\begin{Entry}{债}{10}{⼈}
  \begin{Phonetics}{债}{zhai4}[][HSK 6]
    \definition[笔]{s.}{dívida | empréstimo}
  \end{Phonetics}
\end{Entry}

%%%%%%%%%% 值 %%%%%%%%%%
\subsection*{值}

\begin{Entry}{值}{10}{⼈}
  \begin{Phonetics}{值}{zhi2}[][HSK 3]
    \definition{adj.}{significativo; valioso; digno de nota}
    \definition{prep.}{quando; introduz o momento em que algo acontece ou existe, equivalente a 当 ou 在}
    \definition{s.}{preço; valor | valor de um número, de uma variável}
    \definition{v.}{valer; custar; a mercadoria é adequada ao preço | ir de encontro; encontrar; cruzar | estar de serviço; ter sua vez em algo; assumir o cargo que lhe cabe | é a vez de (executar uma determinada função pública)}
  \seealsoref{当}{dang1}
  \seealsoref{在}{zai4}
  \end{Phonetics}
\end{Entry}

\begin{Entry}{值班}{10,10}{⼈、⽟}
  \begin{Phonetics}{值班}{zhi2ban1}[][HSK 5]
    \definition{v.}{estar em serviço ou plantão; trabalhar em um turno; (em rodízio) desempenhar funções durante um período de tempo determinado}
  \end{Phonetics}
\end{Entry}

\begin{Entry}{值得}{10,11}{⼈、⼻}
  \begin{Phonetics}{值得}{zhi2de5}[][HSK 3]
    \definition{adj.}{que tem valor; (fazer algo) é vantajoso, sem prejuízos}
    \definition{v.}{merecer; ter valor; significa que fazer isso terá bons resultados; que é valioso e significativo}
  \end{Phonetics}
\end{Entry}

%%%%%%%%%% 倾 %%%%%%%%%%
\subsection*{倾}

\begin{Entry}{倾}{10}{⼈}
  \begin{Phonetics}{倾}{qing1}
    \definition{s.}{desvio; tendência}
    \definition{v.}{inclinar; inclinar-se; dobrar-se | colapsar | virar e despejar; esvaziar | fazer tudo o que puder; usar todos os recursos | sobrecarregar; dominar; dominar | admirar | superar}
  \end{Phonetics}
\end{Entry}

\begin{Entry}{倾向}{10,6}{⼈、⼝}
  \begin{Phonetics}{倾向}{qing1xiang4}[][HSK 6]
    \definition{s.}{tendência; desvio; inclinação; direção do desenvolvimento}
    \definition{v.}{preferir; estar inclinado a; concordar com uma determinada opinião}
  \end{Phonetics}
\end{Entry}

\begin{Entry}{倾城}{10,9}{⼈、⼟}
  \begin{Phonetics}{倾城}{qing1cheng2}
    \definition{adj.}{sedutora (mulher)}
    \definition{adv.}{de todo o lugar | vindo de todos os lugares}
    \definition{v.}{arruinar e derrubar o estado}
  \end{Phonetics}
\end{Entry}

%%%%%%%%%% 健 %%%%%%%%%%
\subsection*{健}

\begin{Entry}{健}{10}{⼈}
  \begin{Phonetics}{健}{jian4}
    \definition{adj.}{forte; saudável; bem definido | ser forte em; ser bom em; apresentar um grau superior à média em determinado aspecto}
    \definition{v.}{fortalecer; endurecer; revigorar}
  \end{Phonetics}
\end{Entry}

\begin{Entry}{健全}{10,6}{⼈、⼊}
  \begin{Phonetics}{健全}{jian4quan2}[][HSK 5]
    \definition{adj.}{saudável; íntegro; capaz; apto; robusto e sem mácula | sólido; completo; perfeito}
    \definition{v.}{aperfeiçoar; melhorar; fortalecer; reforçar}
  \end{Phonetics}
\end{Entry}

\begin{Entry}{健壮}{10,6}{⼈、⼠}
  \begin{Phonetics}{健壮}{jian4zhuang4}[][HSK 7-9]
    \definition{adj.}{robusto; saudável e forte}
  \end{Phonetics}
\end{Entry}

\begin{Entry}{健身}{10,7}{⼈、⾝}
  \begin{Phonetics}{健身}{jian4/shen1}[][HSK 4]
    \definition{s.}{exercício físico | \emph{fitness}}
    \definition{v.+compl.}{exercitar-se; manter a forma; praticar um esporte, especialmente a ginástica, inclusive em aparelhos, para desenvolver força, flexibilidade, aumentar a resistência, melhorar a coordenação e o controle de todas as partes do corpo}
  \end{Phonetics}
\end{Entry}

\begin{Entry}{健美}{10,9}{⼈、⽺}
  \begin{Phonetics}{健美}{jian4mei3}[][HSK 7-9]
    \definition{adj.}{forte e bonito; vigoroso e gracioso; robusto e elegante}
    \definition[次]{s.}{fisiculturismo; exercícios que desenvolvem os músculos e o físico}
  \end{Phonetics}
\end{Entry}

\begin{Entry}{健康}{10,11}{⼈、⼴}
  \begin{Phonetics}{健康}{jian4kang1}[][HSK 2]
    \definition{adj.}{em forma; saudável; descreve que a pessoa está em ótimo estado físico ou mental, sem nenhum problema | sudável; tudo está normal, sem problemas | saudável; livre de doenças; bom para a saúde}
    \definition{s.}{saúde; físico; estado de saúde}
  \end{Phonetics}
\end{Entry}

%%%%%%%%%% 党 %%%%%%%%%%
\subsection*{党}

\begin{Entry}{党}{10}{⼉}
  \begin{Phonetics}{党}{dang3}[][HSK 6]
    \definition*{s.}{O Partido (Partido Comunista da China) | Sobrenome: Dang}
    \definition{s.}{partido político; partido | camarilha; facção; gangue | Obsoleto: parentes}
    \definition{v.}{ser parcial; tomar partido de}
  \end{Phonetics}
\end{Entry}

%%%%%%%%%% 兼 %%%%%%%%%%
\subsection*{兼}

\begin{Entry}{兼}{10}{⼋}
  \begin{Phonetics}{兼}{jian1}[][HSK 7-9]
    \definition*{s.}{Sobrenome: Jian}
    \definition{adj.}{duplo; dobrado; duplicado | simultâneo; concomitante}
    \definition{adv.}{simultaneamente; concomitivamente; envolve várias coisas ao mesmo tempo.}
    \definition{v.}{ocupar um cargo simultâneo | ter dois ou mais empregos simultaneamente; fazer várias coisas ao mesmo tempo ou possuir várias coisas | Literário: reunir; unir em um só; anexar}
  \end{Phonetics}
\end{Entry}

\begin{Entry}{兼任}{10,6}{⼋、⼈}
  \begin{Phonetics}{兼任}{jian1ren4}[][HSK 7-9]
    \definition{v.}{ocupar um cargo simultâneo; ter vários empregos ao mesmo tempo | realizar algo em tempo parcial; trabalhar em tempo parcial}
  \end{Phonetics}
\end{Entry}

\begin{Entry}{兼容}{10,10}{⼋、⼧}
  \begin{Phonetics}{兼容}{jian1rong2}[][HSK 7-9]
    \definition{v.}{abranger a todos; ser compatível; aceitar e acomodar simultaneamente coisas ou aspectos diferentes.}
  \end{Phonetics}
\end{Entry}

\begin{Entry}{兼顾}{10,10}{⼋、⾴}
  \begin{Phonetics}{兼顾}{jian1gu4}[][HSK 7-9]
    \definition{v.}{levar em consideração duas ou mais coisas; dar atenção a duas ou mais coisas}
  \end{Phonetics}
\end{Entry}

\begin{Entry}{兼职}{10,11}{⼋、⽿}
  \begin{Phonetics}{兼职}{jian1zhi2}[][HSK 7-9]
    \definition[份]{pron.}{vaga simultânea; emprego de meio período; cargos ocupados fora da função principal de emprego}
    \definition{v.}{ocupar dois ou mais cargos simultaneamente; exercer outras funções além do trabalho principal}
  \end{Phonetics}
\end{Entry}

%%%%%%%%%% 准 %%%%%%%%%%
\subsection*{准}

\begin{Entry}{准}{10}{⼎}
  \begin{Phonetics}{准}{zhun3}[][HSK 3]
    \definition{adj.}{exato; preciso; algo determinado a ser imutável | preciso; exato; correto | perto; parcialmente; quase; próximo de algo em termos de padrão}
    \definition{adv.}{definitivamente; certamente}
    \definition{pref.}{quasi-; para-}
    \definition{prep.}{de acordo com; baseado em}
    \definition{s.}{norma; padrão; critério | confiança certa; uma ideia definida, certeza, etc. (geralmente usada depois de 有 ou 没有)}
    \definition{v.}{autorizar; conceder; consentir; permitir}
  \seealsoref{没有}{mei2 you3}
  \seealsoref{有}{you3}
  \end{Phonetics}
\end{Entry}

\begin{Entry}{准时}{10,7}{⼎、⽇}
  \begin{Phonetics}{准时}{zhun3shi2}[][HSK 4]
    \definition{adj.}{pontual}
    \definition{adv.}{na hora certa; dentro do prazo; no horário especificado}
  \end{Phonetics}
\end{Entry}

\begin{Entry}{准备}{10,8}{⼎、⼡}
  \begin{Phonetics}{准备}{zhun3bei4}[][HSK 1]
    \definition{v.}{preparar; ficar pronto; planejar ou organizar com antecedência | pretender; planejar}
  \end{Phonetics}
\end{Entry}

\begin{Entry}{准确}{10,12}{⼎、⽯}
  \begin{Phonetics}{准确}{zhun3que4}[][HSK 2]
    \definition{adj.}{exato; preciso; acurado; os resultados da ação são completamente consistentes com os resultados reais ou esperados}
  \end{Phonetics}
\end{Entry}

%%%%%%%%%% 凉 %%%%%%%%%%
\subsection*{凉}

\begin{Entry}{凉}{10}{⼎}
  \begin{Phonetics}{凉}{liang2}[][HSK 2]
    \definition{adj.}{frio; gelado; ligeiramente fria (menos do que 冷) | sombrio; desolado; sem animação | desanimado; desapontado | usado para prevenir o calor e manter a temperatura amena; para proteção contra o calor}
    \definition{s.}{frio; refere-se a um ambiente fresco ou a uma brisa fresca}
  \seealsoref{冷}{leng3}
  \end{Phonetics}
  \begin{Phonetics}{凉}{liang4}
    \definition{v.}{deixar algo esfriar; deixar um objeto quente descansar por um tempo para que a temperatura diminua}
  \end{Phonetics}
\end{Entry}

\begin{Entry}{凉水}{10,4}{⼎、⽔}
  \begin{Phonetics}{凉水}{liang2 shui3}[][HSK 3]
    \definition{s.}{água fria; água não aquecida | água não fervida}
  \end{Phonetics}
\end{Entry}

\begin{Entry}{凉快}{10,7}{⼎、⼼}
  \begin{Phonetics}{凉快}{liang2kuai5}[][HSK 2]
    \definition{adj.}{fresco; refrescante}
    \definition{v.}{refrescar; refrescar-se; deixar o corpo fresco e revigorado}
  \end{Phonetics}
\end{Entry}

\begin{Entry}{凉爽}{10,11}{⼎、⽘}
  \begin{Phonetics}{凉爽}{liang2shuang3}[][HSK 7-9]
    \definition{adj.}{agradável e fresco; agradavelmente fresco; não está quente, é uma sensação agradável}
  \end{Phonetics}
\end{Entry}

\begin{Entry}{凉鞋}{10,15}{⼎、⾰}
  \begin{Phonetics}{凉鞋}{liang2 xie2}[][HSK 6]
    \definition[双,只]{s.}{sandália; alpargata; alpercata; alparca ; sapatos de verão com cabedal ventilado}
  \end{Phonetics}
\end{Entry}

%%%%%%%%%% 凌 %%%%%%%%%%
\subsection*{凌}

\begin{Entry}{凌}{10}{⼎}
  \begin{Phonetics}{凌}{ling2}
    \definition*{s.}{Sobrenome: Ling}
    \definition{s.}{gelo}
    \definition{v.}{insultar; invadir; violar; abusar | elevar-se alto; subir; ascender | abordar; aproximar-se}
  \end{Phonetics}
\end{Entry}

\begin{Entry}{凌晨}{10,11}{⼎、⽇}
  \begin{Phonetics}{凌晨}{ling2chen2}[][HSK 7-9]
    \definition[个]{s.}{antes do amanhecer; nas primeiras horas da manhã; com a aproximação do amanhecer}
  \end{Phonetics}
\end{Entry}

%%%%%%%%%% 剥 %%%%%%%%%%
\subsection*{剥}

\begin{Entry}{剥}{10}{⼑}
  \begin{Phonetics}{剥}{bao1}[][HSK 7-9]
    \definition{v.}{descascar; despelar; remover a casca ou pele externa}
  \end{Phonetics}
  \begin{Phonetics}{剥}{bo1}
    \definition{v.}{Dialeto: descascar; despelar; remover a casca ou pele externa | (pele, tinta, etc.) sair; descascar | privar; explorar}
  \end{Phonetics}
\end{Entry}

\begin{Entry}{剥夺}{10,6}{⼑、⼤}
  \begin{Phonetics}{剥夺}{bo1duo2}[][HSK 7-9]
    \definition{v.}{roubar algo de alguém; tirar algo de alguém à força | privar; privar por lei; cancelar de acordo com a lei}
  \end{Phonetics}
\end{Entry}

\begin{Entry}{剥削}{10,9}{⼑、⼑}
  \begin{Phonetics}{剥削}{bo1xue1}[][HSK 7-9]
    \definition{v.}{explorar; apropriar-se do trabalho ou dos frutos do trabalho de outrem sem remuneração}
  \end{Phonetics}
\end{Entry}

%%%%%%%%%% 剧 %%%%%%%%%%
\subsection*{剧}

\begin{Entry}{剧}{10}{⼑}
  \begin{Phonetics}{剧}{ju4}[][HSK 6]
    \definition*{s.}{Sobrenome: Ju}
    \definition{adj.}{agudo; grave; intenso; violento}
    \definition[部,个,种]{s.}{obra teatral; drama; peça; ópera}
  \end{Phonetics}
\end{Entry}

\begin{Entry}{剧本}{10,5}{⼑、⽊}
  \begin{Phonetics}{剧本}{ju4ben3}[][HSK 5]
    \definition[部,个]{s.}{cenário; roteiro (para drama, filme, etc.); gênero de obra literária que consiste em diálogos entre personagens (às vezes cantados) e indicações de palco}
  \end{Phonetics}
\end{Entry}

\begin{Entry}{剧目}{10,5}{⼑、⽬}
  \begin{Phonetics}{剧目}{ju4mu4}[][HSK 7-9]
    \definition{s.}{repertório; programa; lista de peças teatrais ou óperas}
  \end{Phonetics}
\end{Entry}

\begin{Entry}{剧团}{10,6}{⼑、⼞}
  \begin{Phonetics}{剧团}{ju4tuan2}[][HSK 7-9]
    \definition[家,个]{s.}{companhia teatral; grupo de ópera; trupe | grupo de teatro}
  \end{Phonetics}
\end{Entry}

\begin{Entry}{剧场}{10,6}{⼑、⼟}
  \begin{Phonetics}{剧场}{ju4 chang3}[][HSK 3]
    \definition[个,坐]{s.}{teatro; local para apresentações teatrais, musicais, etc.}
  \end{Phonetics}
\end{Entry}

\begin{Entry}{剧组}{10,8}{⼑、⽷}
  \begin{Phonetics}{剧组}{ju4zu3}[][HSK 7-9]
    \definition{s.}{equipe de produção; elenco e equipe técnica; um grupo composto por diretores, atores e membros da equipe com o objetivo de apresentar uma peça teatral ou filmar um filme ou série de televisão}
  \end{Phonetics}
\end{Entry}

\begin{Entry}{剧院}{10,9}{⼑、⾩}
  \begin{Phonetics}{剧院}{ju4yuan4}[][HSK 7-9]
    \definition[家,座]{s.}{teatro; casa de espetáculos | companhias teatrais maiores e de classe mais alta}
  \end{Phonetics}
\end{Entry}

\begin{Entry}{剧烈}{10,10}{⼑、⽕}
  \begin{Phonetics}{剧烈}{ju4lie4}[][HSK 7-9]
    \definition{adj.}{violento; agudo; severo; feroz; rápido e intenso}
  \end{Phonetics}
\end{Entry}

\begin{Entry}{剧情}{10,11}{⼑、⼼}
  \begin{Phonetics}{剧情}{ju4qing2}[][HSK 7-9]
    \definition[个,段]{s.}{a história; o enredo (de uma peça ou ópera)}
  \end{Phonetics}
\end{Entry}

%%%%%%%%%% 原 %%%%%%%%%%
\subsection*{原}

\begin{Entry}{原}{10}{⼚}
  \begin{Phonetics}{原}{yuan2}[][HSK 6]
    \definition*{s.}{Sobrenome: Yuan}
    \definition{adj.}{inicial; básico; primitivo | cru; bruto; não processado | virgem; primário; original; antigo; inalterado}
    \definition{adv.}{originalmente}
    \definition[项,条,片]{s.}{planície; país aberto; terreno plano e amplo | início; fonte; origem; aparência original | origem; a raiz ou o começo das coisas}
    \definition{v.}{desculpar; perdoar; tolerar; compreender | rastrear; sondar; investigar (a origem das coisas)}
  \end{Phonetics}
\end{Entry}

\begin{Entry}{原木}{10,4}{⼚、⽊}
  \begin{Phonetics}{原木}{yuan2mu4}
    \definition{s.}{registro | \emph{logs}}
  \end{Phonetics}
\end{Entry}

\begin{Entry}{原先}{10,6}{⼚、⼉}
  \begin{Phonetics}{原先}{yuan2xian1}[][HSK 5]
    \definition{adj.}{antigo; original}
    \definition{s.}{antigamente; no início; no passado; no começo}
  \end{Phonetics}
\end{Entry}

\begin{Entry}{原则}{10,6}{⼚、⼑}
  \begin{Phonetics}{原则}{yuan2ze2}[][HSK 4]
    \definition{adv.}{em geral; em princípio; refere-se a um aspecto geral; geralmente}
    \definition[个,条,项,点]{s.}{princípios; leis ou padrões pelos quais alguém fala ou age}
  \end{Phonetics}
\end{Entry}

\begin{Entry}{原因}{10,6}{⼚、⼞}
  \begin{Phonetics}{原因}{yuan2yin1}[][HSK 2]
    \definition[个,条,种,些]{s.}{causa; razão; motivo; as condições que fazem com que algo aconteça ou produzam um certo resultado}
  \end{Phonetics}
\end{Entry}

\begin{Entry}{原有}{10,6}{⼚、⽉}
  \begin{Phonetics}{原有}{yuan2 you3}[][HSK 5]
    \definition{v.}{já estar pronto, não é necessário fazer ou procurar nada; ser o original}
  \end{Phonetics}
\end{Entry}

\begin{Entry}{原色}{10,6}{⼚、⾊}
  \begin{Phonetics}{原色}{yuan2 se4}
    \definition{s.}{cor primária}
  \end{Phonetics}
\end{Entry}

\begin{Entry}{原告}{10,7}{⼚、⼝}
  \begin{Phonetics}{原告}{yuan2gao4}[][HSK 6]
    \definition{s.}{(em casos civis) autor; solicitante | (em casos criminais) promotor; acusador; reclamante (oposto a 被告)}
  \seealsoref{被告}{bei4gao4}
  \end{Phonetics}
\end{Entry}

\begin{Entry}{原来}{10,7}{⼚、⽊}
  \begin{Phonetics}{原来}{yuan2lai2}[][HSK 2]
    \definition{adj.}{original; anterior; em primeiro lugar; inicialmente; inalterado}
    \definition{adv.}{na verdade; de fato; como se vê; expressar compreensão repentina}
    \definition{s.}{a princípio; no passado; antigamente}
  \end{Phonetics}
\end{Entry}

\begin{Entry}{原始}{10,8}{⼚、⼥}
  \begin{Phonetics}{原始}{yuan2shi3}[][HSK 5]
    \definition{s.}{original; de primeira mão | primitivo; mais antigo; não desenvolvido; não civilizado}
  \end{Phonetics}
\end{Entry}

\begin{Entry}{原料}{10,10}{⼚、⽃}
  \begin{Phonetics}{原料}{yuan2liao4}[][HSK 4]
    \definition[种,个]{s.}{matéria-prima; refere-se a materiais que não foram processados e fabricados, como minérios para metalurgia e algodão para têxteis}
  \end{Phonetics}
\end{Entry}

\begin{Entry}{原谅}{10,10}{⼚、⾔}
  \begin{Phonetics}{原谅}{yuan2liang4}[][HSK 6]
    \definition{v.}{perdoar; perdoar a negligência, os erros ou as falhas das pessoas sem culpá-las ou puni-las}
  \end{Phonetics}
\end{Entry}

\begin{Entry}{原理}{10,11}{⼚、⽟}
  \begin{Phonetics}{原理}{yuan2li3}[][HSK 5]
    \definition[个,条]{s.}{princípio; axioma; teoria; teoria básica ou princípio científico de significado universal}
  \end{Phonetics}
\end{Entry}

%%%%%%%%%% 哥 %%%%%%%%%%
\subsection*{哥}

\begin{Entry}{哥}{10}{⼝}
  \begin{Phonetics}{哥}{ge1}[][HSK 1]
    \definition[个,位,名,些]{s.}{irmão mais velho | forma de se dirigir a um parente masculino mais velho de sua geração | irmão; termo amigável para se dirigir a conhecidos mais velhos do sexo masculino}
  \seealsoref{哥哥}{ge1 ge5}
  \end{Phonetics}
\end{Entry}

\begin{Entry}{哥们}{10,5}{⼝、⼈}
  \begin{Phonetics}{哥们}{ge1men5}
    \definition{expr.}{\emph{Brothers!}}
    \definition{s.}{(coloquial) cara | irmão (forma diminuta de tratamento entre homens)}
  \end{Phonetics}
\end{Entry}

\begin{Entry}{哥哥}{10,10}{⼝、⼝}
  \begin{Phonetics}{哥哥}{ge1 ge5}[][HSK 1]
    \definition[个,位]{s.}{irmão mais velho | primo}
  \end{Phonetics}
\end{Entry}

\begin{Entry}{哥斯拉}{10,12,8}{⼝、⽄、⼿}
  \begin{Phonetics}{哥斯拉}{ge1si1la1}
    \definition*{s.}{Godzilla}
  \seealsoref{酷斯拉}{ku4si1la1}
  \end{Phonetics}
\end{Entry}

%%%%%%%%%% 哦 %%%%%%%%%%
\subsection*{哦}

\begin{Entry}{哦}{10}{⼝}
  \begin{Phonetics}{哦}{e2}
    \definition{v.}{cantar suavemente (um poema)}
  \end{Phonetics}
  \begin{Phonetics}{哦}{o2}
    \definition{interj.}{Oh! (indicando dúvida ou surpresa)}
  \end{Phonetics}
  \begin{Phonetics}{哦}{o4}
    \definition{interj.}{Oh! (indicando que acabou de aprender algo)}
  \end{Phonetics}
  \begin{Phonetics}{哦}{o5}
    \definition{part.}{usada no final da frase para indicar que uma pessoa está afirmando um fato que a outra pessoa não sabe | usada no final da frase para transmitir informalidade, calor, simpatia ou intimidade}
  \end{Phonetics}
\end{Entry}

%%%%%%%%%% 哩 %%%%%%%%%%
\subsection*{哩}

\begin{Entry}{哩}{10}{⼝}
  \begin{Phonetics}{哩}{li3}
    \definition{clas.}{milha (unidade de comprimento igual a 1.609,344 m)}
  \end{Phonetics}
  \begin{Phonetics}{哩}{li5}
    \definition{part.}{(dialeto) final modal semelhante a 呢 ou 啦, usado em um tom definido, mas um tanto exagerado}
  \seealsoref{啦}{la5}
  \seealsoref{呢}{ne5}
  \end{Phonetics}
\end{Entry}

\begin{Entry}{哩哩啦啦}{10,10,11,11}{⼝、⼝、⼝、⼝}
  \begin{Phonetics}{哩哩啦啦}{li1 li1 la1 la1}
    \definition{adj.}{espalhado; disperso; disseminado; difuso; esporádico; aqui e ali}
  \end{Phonetics}
\end{Entry}

%%%%%%%%%% 哭 %%%%%%%%%%
\subsection*{哭}

\begin{Entry}{哭}{10}{⼝}
  \begin{Phonetics}{哭}{ku1}[][HSK 2]
    \definition{v.}{chorar; soluçar; lamentar-se; chorar de dor ou emoção}
  \end{Phonetics}
\end{Entry}

\begin{Entry}{哭泣}{10,8}{⼝、⽔}
  \begin{Phonetics}{哭泣}{ku1qi4}[][HSK 7-9]
    \definition{v.}{chorar; soluçar; chorar copiosamente}
  \end{Phonetics}
\end{Entry}

\begin{Entry}{哭笑不得}{10,10,4,11}{⼝、⽵、⼀、⼻}
  \begin{Phonetics}{哭笑不得}{ku1xiao4-bu4de2}[][HSK 7-9]
    \definition{expr.}{``Sem saber se ria ou chorava.''; a incapacidade de chorar ou rir descreve uma situação constrangedora em que a pessoa não sabe o que fazer}
  \end{Phonetics}
\end{Entry}

\begin{Entry}{哭墙}{10,14}{⼝、⼟}
  \begin{Phonetics}{哭墙}{ku1qiang2}
    \definition*{s.}{Muro das Lamentações (Jerusalém)}
  \end{Phonetics}
\end{Entry}

%%%%%%%%%% 哮 %%%%%%%%%%
\subsection*{哮}

\begin{Entry}{哮}{10}{⼝}
  \begin{Phonetics}{哮}{xiao4}
    \definition{s.}{respiração pesada; chiado}
    \definition{v.}{rugir; uivar}
  \end{Phonetics}
\end{Entry}

\begin{Entry}{哮喘}{10,12}{⼝、⼝}
  \begin{Phonetics}{哮喘}{xiao4chuan3}
    \definition{s.}{asma; sintomas de dispneia: os pacientes sentem que a respiração está muito difícil; pneumonia, insuficiência cardíaca, bronquite crônica e outras doenças causadas por espasmo da musculatura lisa respiratória frequentemente apresentam esse sintoma}
    \definition{v.}{sofrer de asma}
  \end{Phonetics}
\end{Entry}

%%%%%%%%%% 哲 %%%%%%%%%%
\subsection*{哲}

\begin{Entry}{哲}{10}{⼝}
  \begin{Phonetics}{哲}{zhe2}
    \definition{adj.}{sábio; sagaz}
    \definition[位,名,个]{s.}{pessoas sábias; sábio |sabedoria}
  \end{Phonetics}
\end{Entry}

\begin{Entry}{哲学}{10,8}{⼝、⼦}
  \begin{Phonetics}{哲学}{zhe2xue2}[][HSK 6]
    \definition{s.}{filosofia; é uma disciplina que explora questões fundamentais e conceitos básicos}
  \end{Phonetics}
\end{Entry}

\begin{Entry}{哲理}{10,11}{⼝、⽟}
  \begin{Phonetics}{哲理}{zhe2li3}
    \definition{s.}{filosofia | teoria filosófica}
  \end{Phonetics}
\end{Entry}

%%%%%%%%%% 哺 %%%%%%%%%%
\subsection*{哺}

\begin{Entry}{哺}{10}{⼝}
  \begin{Phonetics}{哺}{bu3}
    \definition{s.}{comida na boca; chorando por comida | alimentos de mastigação; mastigando comida}
    \definition{v.}{alimentar (um bebê); amamentar}
  \end{Phonetics}
\end{Entry}

\begin{Entry}{哺育}{10,8}{⼝、⾁}
  \begin{Phonetics}{哺育}{bu3yu4}[][HSK 7-9]
    \definition{v.}{alimentar | Figurativo: nutrir; fomentar | desenvolver}
  \end{Phonetics}
\end{Entry}

%%%%%%%%%% 哼 %%%%%%%%%%
\subsection*{哼}

\begin{Entry}{哼}{10}{⼝}
  \begin{Phonetics}{哼}{heng1}[][HSK 7-9]
    \definition{v.}{gemer; bufar | cantarolar}
  \end{Phonetics}
  \begin{Phonetics}{哼}{hng5}
    \definition{interj.}{Hmm; Humph; expressa insatisfação, desprezo, desdém ou indignação}
  \end{Phonetics}
\end{Entry}

%%%%%%%%%% 唇 %%%%%%%%%%
\subsection*{唇}

\begin{Entry}{唇}{10}{⼝}
  \begin{Phonetics}{唇}{chun2}
    \definition[片]{s.}{lábios}
  \end{Phonetics}
\end{Entry}

%%%%%%%%%% 唉 %%%%%%%%%%
\subsection*{唉}

\begin{Entry}{唉}{10}{⼝}
  \begin{Phonetics}{唉}{ai1}
    \definition{interj.}{Sim!; Certo!; Bem! | Ai de mim!; o som dos suspiros}
  \end{Phonetics}
  \begin{Phonetics}{唉}{ai4}[][HSK 7-9]
    \definition{interj.}{Oh!; Ah!; Bem!; interjeição que expressa tristeza ou arrependimento | Bem!; Argh!; usado para responder ou reconhecer}
  \end{Phonetics}
\end{Entry}

%%%%%%%%%% 唐 %%%%%%%%%%
\subsection*{唐}

\begin{Entry}{唐}{10}{⼝}
  \begin{Phonetics}{唐}{tang2}
    \definition*{s.}{Dinastia estabelecida pelo Imperador Yao, 尧, no período lendário da história chinesa | Dinastia Tang (618-907) | Dinastia Tang posterior (923-936), uma das cinco dinastias | Sobrenome: Tang}
    \definition{adj.}{exagerado; bombástico; orgulhoso | em vão; por nada}
  \seealsoref{尧}{yao2}
  \end{Phonetics}
\end{Entry}

\begin{Entry}{唐人街}{10,2,12}{⼝、⼈、⾏}
  \begin{Phonetics}{唐人街}{tang2ren2 jie1}
    \definition*[条,座]{s.}{Bairro Chinês; Chinatown; refere-se ao mercado de rua onde os chineses do exterior vivem e abrem muitas lojas com características chinesas}
  \seealsoref{中国城}{zhong1guo2cheng2}
  \end{Phonetics}
\end{Entry}

\begin{Entry}{唐初四大家}{10,7,5,3,10}{⼝、⾐、⼞、⼤、⼧}
  \begin{Phonetics}{唐初四大家}{tang2 chu1 si4 da4jia1}
    \definition*{s.}{Quatro grandes calígrafos do início da dinastia Tang; refere-se a Yu Shi'nan 虞世南, Ouyang Xun 欧阳询, Chu Suiliang 褚遂良 e Xue Ji 薛稷}
  \seealsoref{褚遂良}{chu3 sui4liang2}
  \seealsoref{欧阳询}{ou1yang2 xun2}
  \seealsoref{薛稷}{xue1 ji4}
  \seealsoref{虞世南}{yu2 shi4'nan2}
  \end{Phonetics}
\end{Entry}

%%%%%%%%%% 唠 %%%%%%%%%%
\subsection*{唠}

\begin{Entry}{唠}{10}{⼝}
  \begin{Phonetics}{唠}{lao2}
    \definition{v.}{conversar; falar sobre}
  \end{Phonetics}
  \begin{Phonetics}{唠}{lao4}
    \definition{v.}{Dialeto: conversar; falar sobre | fofocar}
  \end{Phonetics}
\end{Entry}

\begin{Entry}{唠叨}{10,5}{⼝、⼝}
  \begin{Phonetics}{唠叨}{lao2dao5}[][HSK 7-9]
    \definition{v.}{tagarelar; ser loquaz; falar indefinidamente; ser interminável}
  \end{Phonetics}
\end{Entry}

%%%%%%%%%% 唤 %%%%%%%%%%
\subsection*{唤}

\begin{Entry}{唤}{10}{⼝}
  \begin{Phonetics}{唤}{huan4}
    \definition{v.}{chamar; fazer um barulho alto para fazer a outra parte acordar, prestar atenção ou vir até você}
  \end{Phonetics}
\end{Entry}

\begin{Entry}{唤起}{10,10}{⼝、⾛}
  \begin{Phonetics}{唤起}{huan4qi3}[][HSK 7-9]
    \definition{v.}{despertar | chamar; evocar}
  \end{Phonetics}
\end{Entry}

%%%%%%%%%% 啊 %%%%%%%%%%
\subsection*{啊}

\begin{Entry}{啊}{10}{⼝}
  \begin{Phonetics}{啊}{a1}[][HSK 2]
    \definition{interj.}{Ah!; Oh!; expressar surpresa ou admiração}
  \end{Phonetics}
  \begin{Phonetics}{啊}{a2}[][HSK 2]
    \definition{interj.}{Eh?; Ei?; Que?; Por que?; expressar questionamento, dúvida ou solicitar opinião}
  \end{Phonetics}
  \begin{Phonetics}{啊}{a3}[][HSK 2]
    \definition{interj.}{Eh?; Meu!; E aí?; Que?; expressar surpresa e dúvida}
  \end{Phonetics}
  \begin{Phonetics}{啊}{a4}[][HSK 2]
    \definition{interj.}{Bem!; Sim!; expressa concordância, pronúncia mais curta | Oh!; Ah!; indica que compreendeu, com pronúncia mais longa | Oh!; expressa surpresa ou admiração, com pronúncia mais longa | Desgraça!; expressa tristeza ou pesar}
  \end{Phonetics}
  \begin{Phonetics}{啊}{a5}[][HSK 2,4]
    \definition{part.}{usado no final da frase para expressar admiração | usado no final da frase para expressar afirmação, justificativa, insistência, recomendação, etc. | usado no final da frase para indicar dúvida | usado para fazer uma pequena pausa na frase, chamando a atenção para o que vem a seguir | usado após os itens enumerados | usado após verbos repetitivos, indica um processo longo}
  \end{Phonetics}
\end{Entry}

\begin{Entry}{啊呀}{10,7}{⼝、⼝}
  \begin{Phonetics}{啊呀}{a1ya1}
    \definition{interj.}{Oh meu Deus! | interjeição de surpresa}
  \end{Phonetics}
\end{Entry}

\begin{Entry}{啊哟}{10,9}{⼝、⼝}
  \begin{Phonetics}{啊哟}{a1yo5}
    \definition{interj.}{Meu Deus! | Oh! | Ai! | interjeição de surpresa ou dor}
  \end{Phonetics}
\end{Entry}

%%%%%%%%%% 圆 %%%%%%%%%%
\subsection*{圆}

\begin{Entry}{圆}{10}{⼞}
  \begin{Phonetics}{圆}{yuan2}[][HSK 4]
    \definition*{s.}{Sobrenome: Yuan}
    \definition{adj.}{redondo; circular; esférico; arredondado | diplomático; satisfatório}
    \definition[个,轮]{s.}{círculo; circunferência | uma moeda de valor e peso fixos}
    \definition{v.}{tornar plausível; justificar; tornar completo; completar}
  \end{Phonetics}
\end{Entry}

\begin{Entry}{圆珠笔}{10,10,10}{⼞、⽟、⽵}
  \begin{Phonetics}{圆珠笔}{yuan2 zhu1 bi3}[][HSK 6]
    \definition[支,枝]{s.}{caneta esferográfica}
  \end{Phonetics}
\end{Entry}

\begin{Entry}{圆满}{10,13}{⼞、⽔}
  \begin{Phonetics}{圆满}{yuan2man3}[][HSK 4]
    \definition{adj.}{perfeito; satisfatório; sem defeitos}
  \end{Phonetics}
\end{Entry}

%%%%%%%%%% 埋 %%%%%%%%%%
\subsection*{埋}

\begin{Entry}{埋}{10}{⼟}
  \begin{Phonetics}{埋}{mai2}[][HSK 6]
    \definition{v.}{cobrir (com terra, neve, etc.); enterrar | esconder | enterrar (uma pessoa morta)}
  \end{Phonetics}
  \begin{Phonetics}{埋}{man2}
    \definition{part.}{caracter formador de palavras}
  \end{Phonetics}
\end{Entry}

\begin{Entry}{埋伏}{10,6}{⼟、⼈}
  \begin{Phonetics}{埋伏}{mai2fu2}
    \definition{s.}{emboscada}
    \definition{v.}{emboscar}
  \end{Phonetics}
\end{Entry}

%%%%%%%%%% 壶 %%%%%%%%%%
\subsection*{壶}

\begin{Entry}{壶}{10}{⼠}
  \begin{Phonetics}{壶}{hu2}[][HSK 6]
    \definition*{s.}{Sobrenome: Hu}
    \definition[个,把]{s.}{chaleira; panela | garrafa; frasco; recipiente para líquidos}
  \end{Phonetics}
\end{Entry}

%%%%%%%%%% 夏 %%%%%%%%%%
\subsection*{夏}

\begin{Entry}{夏}{10}{⼢}
  \begin{Phonetics}{夏}{xia4}
    \definition*{s.}{Dinastia Xia (2070-1600 a.C.) | China; refere-se à China | Sobrenome: Xia}
    \definition{s.}{verão}
  \end{Phonetics}
\end{Entry}

\begin{Entry}{夏天}{10,4}{⼢、⼤}
  \begin{Phonetics}{夏天}{xia4 tian1}[][HSK 2]
    \definition[个]{s.}{verão}
  \end{Phonetics}
\end{Entry}

\begin{Entry}{夏日}{10,4}{⼢、⽇}
  \begin{Phonetics}{夏日}{xia4ri4}
    \definition{s.}{horário de verão}
  \end{Phonetics}
\end{Entry}

\begin{Entry}{夏季}{10,8}{⼢、⼦}
  \begin{Phonetics}{夏季}{xia4 ji4}[][HSK 4]
    \definition[个]{s.}{verão; segundo trimestre do ano, habitualmente chamado na China de período de três meses, do início do verão ao início do outono, também chamado de ``quarto, quinto e sexto'' meses do calendário lunar}
  \end{Phonetics}
\end{Entry}

%%%%%%%%%% 套 %%%%%%%%%%
\subsection*{套}

\begin{Entry}{套}{10}{⼤}
  \begin{Phonetics}{套}{tao4}[][HSK 2]
    \definition{clas.}{usado para coisas agrupadas: conjuntos, coleções, séries, etc.}
    \definition{s.}{estojo; capa; bainha | local onde o rio ou a cordilheira faz uma curva (usado principalmente em nomes de lugares) | enchimento de algodão em roupas e edredons | arreios; corda para amarrar animais | nó; laço; um objeto circular feito com corda ou algo semelhante | cortersia; convenção; conversa fiada; métodos repetitivos | armadilha; truque; conspiração}
    \definition{v.}{sobrepor; interligar | deslizar sobre; cobrir por fora | atrelar; engatar; usar um cinto de segurança | copiar; imitar; seguir o modelo de | extrair; induzir a falar; persuadir alguém a revelar um segredo; induzir; provocar | tentar vencer; aproximar-se de; aproximar-se intencionalmente de outras pessoas para algum propósito | fazer a rosca de um parafuso; usar um macho de rosca ou uma chave de rosca para fazer roscas}
  \end{Phonetics}
\end{Entry}

\begin{Entry}{套问}{10,6}{⼤、⾨}
  \begin{Phonetics}{套问}{tao4wen4}
    \definition{s.}{retórica}
    \definition{v.}{descobrir por meio de questionamento indireto diplomático}
  \end{Phonetics}
\end{Entry}

\begin{Entry}{套餐}{10,16}{⼤、⾷}
  \begin{Phonetics}{套餐}{tao4 can1}[][HSK 4]
    \definition{s.}{combo; pacote de produtos; pacote de serviços; metaforicamente, bens ou projetos que são combinados e levados ao mercado | refeição preparada; pacotes de refeições completos}
  \end{Phonetics}
\end{Entry}

%%%%%%%%%% 娱 %%%%%%%%%%
\subsection*{娱}

\begin{Entry}{娱}{10}{⼥}
  \begin{Phonetics}{娱}{yu2}
    \definition{s.}{alegria; prazer; diversão; felicidade}
    \definition{v.}{dar prazer a; divertir; fazer feliz}
  \end{Phonetics}
\end{Entry}

\begin{Entry}{娱乐}{10,5}{⼥、⼃}
  \begin{Phonetics}{娱乐}{yu2le4}[][HSK 6]
    \definition[项]{s.}{entretenimento; diversão; recreação; passa-tempo; atividades recreativas, prazerosas e divertidas}
    \definition{v.}{recrear; divertir; distrair; entreter; passar o tempo}
  \end{Phonetics}
\end{Entry}

%%%%%%%%%% 孬 %%%%%%%%%%
\subsection*{孬}

\begin{Entry}{孬}{10}{⼥}
  \begin{Phonetics}{孬}{nao1}
    \definition{adj.}{ruim | covarde | Dialeto: não (é) bom (contração de 不 + 好)}
  \seealsoref{不}{bu4}
  \seealsoref{好}{hao3}
  \end{Phonetics}
\end{Entry}

%%%%%%%%%% 害 %%%%%%%%%%
\subsection*{害}

\begin{Entry}{害}{10}{⼧}
  \begin{Phonetics}{害}{hai4}[][HSK 5]
    \definition{adj.}{prejudicial; destrutivo; injurioso; nocivo}
    \definition{s.}{mal; maldade; dano; calamidade}
    \definition{v.}{prejudicar; fazer mal a; causar problemas a | matar; assassinar | sofrer de; contrair (uma doença) | sentir-se (envergonhado, com medo, etc.); despertar (um sentimento ou uma emoção)}
  \end{Phonetics}
\end{Entry}

\begin{Entry}{害虫}{10,6}{⼧、⾍}
  \begin{Phonetics}{害虫}{hai4chong2}[][HSK 7-9]
    \definition[种,只,个]{s.}{verme; bicho; inseto nocivo (ou destrutivo); praga (oposto a 益虫)}
  \seealsoref{益虫}{yi4chong2}
  \end{Phonetics}
\end{Entry}

\begin{Entry}{害怕}{10,8}{⼧、⼼}
  \begin{Phonetics}{害怕}{hai4pa4}[][HSK 3]
    \definition{v.}{estar assustado; ter medo; encontrar dificuldades, perigos, etc., e sentir-se inquieto ou nervoso}
  \end{Phonetics}
\end{Entry}

\begin{Entry}{害羞}{10,10}{⼧、⽺}
  \begin{Phonetics}{害羞}{hai4/xiu1}[][HSK 7-9]
    \definition{v.+compl.}{ser tímido; parecer tímido; tornar-se tímido}
  \end{Phonetics}
\end{Entry}

\begin{Entry}{害臊}{10,17}{⼧、⾁}
  \begin{Phonetics}{害臊}{hai4/sao4}[][HSK 7-9]
    \definition{v.+compl.}{sentir vergonha; ser tímido}
  \end{Phonetics}
\end{Entry}

%%%%%%%%%% 宴 %%%%%%%%%%
\subsection*{宴}

\begin{Entry}{宴}{10}{⼧}
  \begin{Phonetics}{宴}{yan4}
    \definition{adj.}{tranquilo e confortável}
    \definition[个,场]{s.}{festa; banquete | facilidade e conforto; felicidade; lazer}
    \definition{v.}{entreter em um banquete; festejar}
  \end{Phonetics}
\end{Entry}

\begin{Entry}{宴会}{10,6}{⼧、⼈}
  \begin{Phonetics}{宴会}{yan4hui4}[][HSK 6]
    \definition[个,场,次]{s.}{festa; banquete; jantar; uma reunião onde convidados e anfitriões bebem e comem juntos (referindo-se a uma ocasião mais solene)}
  \end{Phonetics}
\end{Entry}

%%%%%%%%%% 家 %%%%%%%%%%
\subsection*{家}

\begin{Entry}{家}{10}{⼧}
  \begin{Phonetics}{家}{jia1}[][HSK 1,2]
    \definition*{s.}{Sobrenome: Jia}
    \definition{adj.}{domado; domesticado; criado; alimentado | interno}
    \definition{clas.}{usado para famílias ou estabelecimentos comerciais; para uso doméstico; lojas; fábricas, etc.}
    \definition{pron.}{Educado: meu (irmã, tio, etc.)}
    \definition[个]{s.}{família; domicílio; clã | lar; casa; residência da família | pessoa ou família envolvida em um determinado comércio; pessoas que trabalham em determinada profissão ou que possuem determinada identidade | especialista em um determinado campo; pessoa que possui conhecimentos especializados ou se dedica a atividades específicas | escola de pensamento; rscola acadêmica | (em cartas de baralho, mah-jong etc.) festa; lado; refere-se a jogar xadrez ou cartas, em que uma das partes joga contra a outra | nacionalidade; referindo-se à etnia | membros da família; parentes; pessoas ou famílias com quem você tem algum tipo de relação | membro do mesmo clã; pessoas com o mesmo sobrenome}
    \definition{suf.}{sufixo substantivo para designar um especialista em alguma atividade, como um músico ou revolucionário, para designar uma profissão como em -eiro, -ista, por exemplo 科学家}
  \seealsoref{科学家}{ke1xue2jia1}
  \end{Phonetics}
\end{Entry}

\begin{Entry}{家人}{10,2}{⼧、⼈}
  \begin{Phonetics}{家人}{jia1 ren2}[][HSK 1]
    \definition{s.}{família (de alguém); membro da família; os membros de uma família}
  \end{Phonetics}
\end{Entry}

\begin{Entry}{家乡}{10,3}{⼧、⼄}
  \begin{Phonetics}{家乡}{jia1xiang1}[][HSK 3]
    \definition[片,座]{s.}{cidade natal; o lugar onde sua família vive há gerações}
  \end{Phonetics}
\end{Entry}

\begin{Entry}{家长}{10,4}{⼧、⾧}
  \begin{Phonetics}{家长}{jia1 zhang3}[][HSK 2]
    \definition[位,名,个]{s.}{pais; patriarca; tutor; guardião; refere-se aos pais ou outros responsáveis legais}
  \end{Phonetics}
\end{Entry}

\begin{Entry}{家务}{10,5}{⼧、⼒}
  \begin{Phonetics}{家务}{jia1wu4}[][HSK 4]
    \definition[堆,次,件]{s.}{trabalho doméstico; tarefas domésticas}
  \end{Phonetics}
\end{Entry}

\begin{Entry}{家用}{10,5}{⼧、⽤}
  \begin{Phonetics}{家用}{jia1yong4}[][HSK 7-9]
    \definition{adj.}{doméstico; para uso doméstico}
    \definition[本]{s.}{despesas familiares; dinheiro para manutenção da casa}
    \definition{v.}{ser usado em casa; para uso doméstico}
  \end{Phonetics}
\end{Entry}

\begin{Entry}{家用电器}{10,5,5,16}{⼧、⽤、⽥、⼝}
  \begin{Phonetics}{家用电器}{jia1yong4 dian4qi4}
    \definition{s.}{eletrodoméstico; refere-se a diversos aparelhos elétricos utilizados na vida doméstica e coletiva}
  \end{Phonetics}
\end{Entry}

\begin{Entry}{家电}{10,5}{⼧、⽥}
  \begin{Phonetics}{家电}{jia1 dian4}[][HSK 6]
    \definition[件,台]{s.}{eletrodomésticos, abreviação de 家用电器}
  \seealsoref{家用电器}{jia1yong4 dian4qi4}
  \end{Phonetics}
\end{Entry}

\begin{Entry}{家伙}{10,6}{⼧、⼈}
  \begin{Phonetics}{家伙}{jia1huo5}[][HSK 7-9]
    \definition[些,个,群,帮]{s.}{ferramenta; utensílio; arma; refere-se a ferramentas ou armas | cara; companheiro; refere-se a pessoas (com desprezo ou humor)  | gado; animal doméstico}
  \end{Phonetics}
\end{Entry}

\begin{Entry}{家园}{10,7}{⼧、⼞}
  \begin{Phonetics}{家园}{jia1 yuan2}[][HSK 6]
    \definition{s.}{casa; terra natal; um jardim em casa, geralmente referindo-se à cidade natal ou à família}
  \end{Phonetics}
\end{Entry}

\begin{Entry}{家里}{10,7}{⼧、⾥}
  \begin{Phonetics}{家里}{jia1 li3}[][HSK 1]
    \definition{s.}{(em) casa; (em sua) família | esposa}
  \end{Phonetics}
\end{Entry}

\begin{Entry}{家具}{10,8}{⼧、⼋}
  \begin{Phonetics}{家具}{jia1ju4}[][HSK 3]
    \definition[件,套,些,个]{s.}{móveis; mobiliário de casa; utensílios domésticos, incluem principalmente camas, mesas, cadeiras, armários, etc.}
  \end{Phonetics}
\end{Entry}

\begin{Entry}{家庭}{10,9}{⼧、⼴}
  \begin{Phonetics}{家庭}{jia1ting2}[][HSK 2]
    \definition[个,户]{s.}{família}
  \end{Phonetics}
\end{Entry}

\begin{Entry}{家政}{10,9}{⼧、⽁}
  \begin{Phonetics}{家政}{jia1zheng4}[][HSK 7-9]
    \definition{s.}{tarefas domésticas; trabalho de gestão doméstica}
  \end{Phonetics}
\end{Entry}

\begin{Entry}{家俱}{10,10}{⼧、⼈}
  \begin{Phonetics}{家俱}{jia1ju4}
    \definition{s.}{mobília}
  \end{Phonetics}
\end{Entry}

\begin{Entry}{家家户户}{10,10,4,4}{⼧、⼧、⼾、⼾}
  \begin{Phonetics}{家家户户}{jia1jia1hu4hu4}[][HSK 7-9]
    \definition{expr.}{cada família; cada lar}
  \end{Phonetics}
\end{Entry}

\begin{Entry}{家教}{10,11}{⼧、⽁}
  \begin{Phonetics}{家教}{jia1jiao4}[][HSK 7-9]
    \definition[个,名,位]{s.}{educação; educação familiar; disciplina doméstica; a educação de moral e etiqueta que os pais dão aos seus filhos |  tutor}
  \end{Phonetics}
\end{Entry}

\begin{Entry}{家族}{10,11}{⼧、⽅}
  \begin{Phonetics}{家族}{jia1zu2}[][HSK 7-9]
    \definition[个]{s.}{clã; família; uma organização social baseada em relações de sangue, incluindo várias gerações do mesmo sangue}
  \end{Phonetics}
\end{Entry}

\begin{Entry}{家喻户晓}{10,12,4,10}{⼧、⼝、⼾、⽇}
  \begin{Phonetics}{家喻户晓}{jia1yu4-hu4xiao3}[][HSK 7-9]
    \definition{expr.}{nome familiar | conhecido por todas as famílias; amplamente conhecido; conhecido por todos}[他是家喻户晓的演员。===Ele é um ator famoso.]
  \end{Phonetics}
\end{Entry}

\begin{Entry}{家属}{10,12}{⼧、⼫}
  \begin{Phonetics}{家属}{jia1shu3}[][HSK 3]
    \definition{s.}{membros da família; dependentes (familiares); os membros da família que não sejam o próprio chefe da família, ou seja, os membros da família que não sejam o próprio trabalhador}
  \end{Phonetics}
\end{Entry}

\begin{Entry}{家禽}{10,12}{⼧、⽱}
  \begin{Phonetics}{家禽}{jia1qin2}[][HSK 7-9]
    \definition{s.}{aves domésticas | ave; pássaro doméstico}
  \end{Phonetics}
\end{Entry}

\begin{Entry}{家境}{10,14}{⼧、⼟}
  \begin{Phonetics}{家境}{jia1jing4}[][HSK 7-9]
    \definition{s.}{situação financeira familiar; circunstâncias familiares}
  \end{Phonetics}
\end{Entry}

%%%%%%%%%% 容 %%%%%%%%%%
\subsection*{容}

\begin{Entry}{容}{10}{⼧}
  \begin{Phonetics}{容}{rong2}
    \definition*{s.}{Sobrenome: Rong}
    \definition{adv.}{talvez; provavelmente; possivelmente}
    \definition{s.}{expressão facial e tez | aparência; o estado ou condição das coisas}
    \definition{v.}{permitir; quando os outros querem fazer algo, deixe-os fazer | tolerar; ser capaz de aceitar pessoas ou coisas com as quais você não está satisfeito | conter (número de pessoas ou coisas que podem ser colocadas em um determinado espaço)}
  \end{Phonetics}
\end{Entry}

\begin{Entry}{容易}{10,8}{⼧、⽇}
  \begin{Phonetics}{容易}{rong2yi4}[][HSK 3]
    \definition{adj.}{fácil; simples; sem complicações | provável; passível; inclinado; indica uma alta probabilidade de algo acontecer}
  \end{Phonetics}
\end{Entry}

\begin{Entry}{容貌}{10,14}{⼧、⾘}
  \begin{Phonetics}{容貌}{rong2mao4}
    \definition{s.}{aparência | aspecto | características}
  \end{Phonetics}
\end{Entry}

%%%%%%%%%% 宽 %%%%%%%%%%
\subsection*{宽}

\begin{Entry}{宽}{10}{⼧}
  \begin{Phonetics}{宽}{kuan1}[][HSK 4]
    \definition*{s.}{Sobrenome: Kuan}
    \definition{adj.}{largo; amplo; espaçoso; grandes distâncias horizontais (em oposição a 窄) | leniente; generoso; indulgente | bem de vida; rico; confortável}
    \definition[米]{s.}{largura; amplitude}[桌子有一米宽。===A mesa tem um metro de largura.]
    \definition{v.}{relaxar; aliviar}
  \seealsoref{窄}{zhai3}
  \end{Phonetics}
\end{Entry}

\begin{Entry}{宽广}{10,3}{⼧、⼴}
  \begin{Phonetics}{宽广}{kuan1 guang3}[][HSK 4]
    \definition{adj.}{vasto; amplo; espaçoso; extenso}
  \end{Phonetics}
\end{Entry}

\begin{Entry}{宽泛}{10,7}{⼧、⽔}
  \begin{Phonetics}{宽泛}{kuan1fan4}[][HSK 7-9]
    \definition{adj.}{abrangente; (conteúdo, significado) abrange uma ampla gama de aspectos}
  \end{Phonetics}
\end{Entry}

\begin{Entry}{宽松}{10,8}{⼧、⽊}
  \begin{Phonetics}{宽松}{kuan1song1}[][HSK 7-9]
    \definition{adj.}{(roupas) folgado e confortável; espaçoso e sem aglomeração; relaxante e sem aperto | (estado mental, atmosfera, etc.) relaxado; aliviado; sem tensão; livre de preocupações; descontraído; não há tensão | Economia: abundante; que tem dinheiro suficiente para viver bem e sem problemas, mas sem ser excessivamente rico}
  \end{Phonetics}
\end{Entry}

\begin{Entry}{宽厚}{10,9}{⼧、⼚}
  \begin{Phonetics}{宽厚}{kuan1hou4}[][HSK 7-9]
    \definition{adj.}{largo e espesso; amplo e sólido | tolerante; gentil e generoso; tolerância e bondade | simples; sincero; (voz) profunda e ressonante}
  \end{Phonetics}
\end{Entry}

\begin{Entry}{宽度}{10,9}{⼧、⼴}
  \begin{Phonetics}{宽度}{kuan1 du4}[][HSK 5]
    \definition{s.}{largura; amplitude; duração; o grau de largura e estreiteza; a distância horizontal (no caso de um retângulo, a distância entre os dois lados mais longos)}
  \end{Phonetics}
\end{Entry}

\begin{Entry}{宽容}{10,10}{⼧、⼧}
  \begin{Phonetics}{宽容}{kuan1rong2}[][HSK 7-9]
    \definition{adj.}{tolerante; generoso e magnânimo, não mesquinho ou oportunista}
    \definition{v.}{tolerar; ter paciência com; ser tolerante com os outros, não guardar rancor nem insistir no assunto}
  \end{Phonetics}
\end{Entry}

\begin{Entry}{宽恕}{10,10}{⼧、⼼}
  \begin{Phonetics}{宽恕}{kuan1shu4}[][HSK 7-9]
    \definition{v.}{perdoar; desculpar; absolver}
  \end{Phonetics}
\end{Entry}

\begin{Entry}{宽敞}{10,12}{⼧、⽁}
  \begin{Phonetics}{宽敞}{kuan1chang5}[][HSK 7-9]
    \definition{adj.}{amplo; espaçoso; descreve um espaço ou área grande}
  \end{Phonetics}
\end{Entry}

\begin{Entry}{宽阔}{10,12}{⼧、⾨}
  \begin{Phonetics}{宽阔}{kuan1 kuo4}[][HSK 6]
    \definition{adj.}{amplo; largo; espaçoso | tolerante; mente aberta; descreve uma mente alegre e ampla}
  \end{Phonetics}
\end{Entry}

\begin{Entry}{宽影片}{10,15,4}{⼧、⼺、⽚}
  \begin{Phonetics}{宽影片}{kuan1ying3pian4}
    \definition{s.}{filme \emph{widescreen}}
  \end{Phonetics}
\end{Entry}

%%%%%%%%%% 宾 %%%%%%%%%%
\subsection*{宾}

\begin{Entry}{宾}{10}{⼧}
  \begin{Phonetics}{宾}{bin1}
    \definition*{s.}{Sobrenome: Bin}
    \definition[个,位,名,些]{s.}{convidado}
  \end{Phonetics}
\end{Entry}

\begin{Entry}{宾馆}{10,11}{⼧、⾷}
  \begin{Phonetics}{宾馆}{bin1guan3}[][HSK 5]
    \definition[家,个,座]{s.}{hotel; acomodações públicas para hóspedes}
  \end{Phonetics}
\end{Entry}

%%%%%%%%%% 射 %%%%%%%%%%
\subsection*{射}

\begin{Entry}{射}{10}{⼨}
  \begin{Phonetics}{射}{she4}[][HSK 5]
    \definition*{s.}{Sobrenome: She}
    \definition{v.}{atirar; disparar | descarregar em jato; jorrar | emitir (luz, calor, etc.) | irradiar | aludir a algo ou alguém; insinuar}
  \end{Phonetics}
\end{Entry}

\begin{Entry}{射击}{10,5}{⼨、⼐}
  \begin{Phonetics}{射击}{she4ji1}[][HSK 5]
    \definition{s.}{tiro; tiro ao alvo}
    \definition{v.}{disparar; atirar}
  \end{Phonetics}
\end{Entry}

%%%%%%%%%% 展 %%%%%%%%%%
\subsection*{展}

\begin{Entry}{展}{10}{⼫}
  \begin{Phonetics}{展}{zhan3}
    \definition*{s.}{Sobrenome: Zhan}
    \definition{s.}{exposição}
    \definition{v.}{abrir; espalhar; desdobrar | fazer bom uso; dar liberdade para | adiar; estender; prolongar | expandir | abrir; deixar ir | exibir; mostrar}
  \end{Phonetics}
\end{Entry}

\begin{Entry}{展开}{10,4}{⼫、⼶}
  \begin{Phonetics}{展开}{zhan3kai1}[][HSK 3]
    \definition{s.}{desenvolvimento; expansão; explosão; evolução}
    \definition{v.}{espalhar; desdobrar; abrir | lançar; desdobrar; desenvolver; realizar em grande escala | espalhar; desenrolar; amplificar; desenvolver; expandir; explodir; evoluir; alongar}
  \end{Phonetics}
\end{Entry}

\begin{Entry}{展示}{10,5}{⼫、⽰}
  \begin{Phonetics}{展示}{zhan3shi4}[][HSK 5]
    \definition{v.}{mostrar; revelar; pôr a nu; abrir diante de alguém; expor claramente; manifestar de forma evidente}
  \end{Phonetics}
\end{Entry}

\begin{Entry}{展现}{10,8}{⼫、⾒}
  \begin{Phonetics}{展现}{zhan3xian4}[][HSK 5]
    \definition{v.}{mostrar; surgir; manifestar}
  \end{Phonetics}
\end{Entry}

\begin{Entry}{展览}{10,9}{⼫、⾒}
  \begin{Phonetics}{展览}{zhan3lan3}[][HSK 5]
    \definition[个,次,场]{s.}{exposição; exibição; atividades expostas; itens expostos}
    \definition{v.}{mostrar; exibir; expor; expor algo para que as pessoas vejam}
  \end{Phonetics}
\end{Entry}

%%%%%%%%%% 峰 %%%%%%%%%%
\subsection*{峰}

\begin{Entry}{峰}{10}{⼭}
  \begin{Phonetics}{峰}{feng1}
    \definition{clas.}{usado para camelos}
    \definition{s.}{pico; cume; o pico proeminente de uma montanha | coisa parecida com um pico; coisas em forma de montanhas}
  \end{Phonetics}
\end{Entry}

\begin{Entry}{峰会}{10,6}{⼭、⼈}
  \begin{Phonetics}{峰会}{feng1 hui4}[][HSK 6]
    \definition{s.}{cúpula; reunião de cúpula}
  \end{Phonetics}
\end{Entry}

\begin{Entry}{峰回路转}{10,6,13,8}{⼭、⼞、⾜、⾞}
  \begin{Phonetics}{峰回路转}{feng1hui2-lu4zhuan3}[][HSK 7-9]
    \definition{expr.}{cume em meio a elevações circundantes e estradas sinuosas;  (estrada de montanha) torcendo e virando; a estrada da montanha serpenteia em torno de cada novo pico | boa (ou nova) reviravolta nos acontecimentos; uma oportunidade surgiu inesperadamente; as coisas tomaram um novo rumo}
  \end{Phonetics}
\end{Entry}

%%%%%%%%%% 席 %%%%%%%%%%
\subsection*{席}

\begin{Entry}{席}{10}{⼱}
  \begin{Phonetics}{席}{xi2}
    \definition*{s.}{Sobrenome: Xi}
    \definition[卷,张]{s.}{esteira | assento; lugar; caixa | assento (em uma assembleia legislativa) | festim; banquete; jantar}
  \end{Phonetics}
\end{Entry}

\begin{Entry}{席卷}{10,8}{⼱、⼙}
  \begin{Phonetics}{席卷}{xi2juan3}
    \definition{v.}{engolfar | varrer | levar tudo para fora}
  \end{Phonetics}
\end{Entry}

%%%%%%%%%% 座 %%%%%%%%%%
\subsection*{座}

\begin{Entry}{座}{10}{⼴}
  \begin{Phonetics}{座}{zuo4}[][HSK 2]
    \definition{clas.}{usado para montanhas, edifícios e objetos imóveis semelhantes}
    \definition{s.}{assento; lugar | suporte; pedestal; base | (astronomia) constalação | (antigo) forma de tratamento a altos funcionários |}
  \end{Phonetics}
\end{Entry}

\begin{Entry}{座子}{10,3}{⼴、⼦}
  \begin{Phonetics}{座子}{zuo4zi5}
    \definition{s.}{soquete | pedestal | sela}
  \end{Phonetics}
\end{Entry}

\begin{Entry}{座位}{10,7}{⼴、⼈}
  \begin{Phonetics}{座位}{zuo4wei4}[][HSK 2]
    \definition[个,排]{s.}{assento; lugar}
  \end{Phonetics}
\end{Entry}

\begin{Entry}{座标}{10,9}{⼴、⽊}
  \begin{Phonetics}{座标}{zuo4biao1}
    \variantof{坐标}
  \end{Phonetics}
\end{Entry}

\begin{Entry}{座谈会}{10,10,6}{⼴、⾔、⼈}
  \begin{Phonetics}{座谈会}{zuo4 tan2 hui4}[][HSK 6]
    \definition{s.}{fórum; simpósio; discussão informal | conferência | sessão de rap}
  \end{Phonetics}
\end{Entry}

%%%%%%%%%% 弱 %%%%%%%%%%
\subsection*{弱}

\begin{Entry}{弱}{10}{⼸}
  \begin{Phonetics}{弱}{ruo4}[][HSK 4]
    \definition{adj.}{fraco; debilitado | jovem | inferior; pior | colocado depois de uma fração ou decimal para indicar que é um pouco menor que esse número (oposto a 强)}
    \definition{v.}{perder (através da morte)}
  \seealsoref{强}{qiang2}
  \end{Phonetics}
\end{Entry}

%%%%%%%%%% 徐 %%%%%%%%%%
\subsection*{徐}

\begin{Entry}{徐}{10}{⼻}
  \begin{Phonetics}{徐}{xu2}
    \definition*{s.}{Sobrenome: Xu}
    \definition{adv.}{lentamente; suavemente}
  \end{Phonetics}
\end{Entry}

\begin{Entry}{徐徐}{10,10}{⼻、⼻}
  \begin{Phonetics}{徐徐}{xu2xu2}
    \definition{adv.}{lentamente; suavemente}
  \end{Phonetics}
\end{Entry}

%%%%%%%%%% 徒 %%%%%%%%%%
\subsection*{徒}

\begin{Entry}{徒}{10}{⼻}
  \begin{Phonetics}{徒}{tu2}
    \definition{adj.}{vazio; nu}
    \definition{adv.}{somente; meramente; apenas | a pé | em vão; sem sucesso; sem sucesso}
    \definition{s.}{aprendiz; aluno | seguidor; crente | (pejorativo) pessoas da mesma facção | (pejorativo) pessoa; companheiro | (prisão) pena; prisão; sentença | estudante}
    \definition{v.}{estar a pé | andar}
  \end{Phonetics}
\end{Entry}

\begin{Entry}{徒手}{10,4}{⼻、⼿}
  \begin{Phonetics}{徒手}{tu2shou3}
    \definition{adj.}{com as mãos vazias | desarmado | mão livre (desenho) | lutando mão-a-mão}
  \end{Phonetics}
\end{Entry}

\begin{Entry}{徒弟}{10,7}{⼻、⼸}
  \begin{Phonetics}{徒弟}{tu2di4}[][HSK 6]
    \definition[位,名,个]{s.}{discípulo; aprendiz; uma pessoa que aprende com um mestre; geralmente se refere a uma pessoa que aprende com um especialista}[他是我的徒弟。===Ele é meu aprendiz.]
  \end{Phonetics}
\end{Entry}

%%%%%%%%%% 恋 %%%%%%%%%%
\subsection*{恋}

\begin{Entry}{恋}{10}{⼼}
  \begin{Phonetics}{恋}{lian4}
    \definition*{s.}{Sobrenome: Lian}
    \definition{v.}{amor (romântico) | ansiar por; sentir-se apegado a | amar; apaixonar-se por | não querendo se separar de; sentir sua falta para sempre; não suportar ficar separado}
  \end{Phonetics}
\end{Entry}

\begin{Entry}{恋恋不舍}{10,10,4,8}{⼼、⼼、⼀、⾆}
  \begin{Phonetics}{恋恋不舍}{lian4lian4-bu4she4}[][HSK 7-9]
    \definition{expr.}{afastar-se de algo; relutar em se separar; sentir-se apegado a algo; descrevendo a relutância em partir}
  \end{Phonetics}
\end{Entry}

\begin{Entry}{恋爱}{10,10}{⼼、⽖}
  \begin{Phonetics}{恋爱}{lian4'ai4}[][HSK 5]
    \definition[个,场,段]{s.}{namoro; afeto; amor romântico; ações que demonstram o amor mútuo}
    \definition{v.}{amar; estar apaixonado}
  \end{Phonetics}
\end{Entry}

%%%%%%%%%% 恐 %%%%%%%%%%
\subsection*{恐}

\begin{Entry}{恐}{10}{⼼}
  \begin{Phonetics}{恐}{kong3}
    \definition{adv.}{talvez; provavelmente}
    \definition{v.}{temer; recear; ter medo de | ameaçar; aterrorizar; intimidar}
  \end{Phonetics}
\end{Entry}

\begin{Entry}{恐龙}{10,5}{⼼、⿓}
  \begin{Phonetics}{恐龙}{kong3long2}[][HSK 7-9]
    \definition[只,头]{s.}{dinossauro | garota feia (gíria da \emph{Internet}, ofensiva)}
  \end{Phonetics}
\end{Entry}

\begin{Entry}{恐吓}{10,6}{⼼、⼝}
  \begin{Phonetics}{恐吓}{kong3he4}[][HSK 7-9]
    \definition{v.}{ameaçar; assustar; intimidar; ameaçar alguém com palavras ou meios ameaçadores}
  \end{Phonetics}
\end{Entry}

\begin{Entry}{恐怕}{10,8}{⼼、⼼}
  \begin{Phonetics}{恐怕}{kong3pa4}[][HSK 3]
    \definition{adv.}{talvez; provavelmente; pode ser; expressa suposição; estimativa. | por medo de; expressar estimativa e preocupação}
    \definition{v.}{ter medo de; temer; recear}
  \end{Phonetics}
\end{Entry}

\begin{Entry}{恐怖}{10,8}{⼼、⼼}
  \begin{Phonetics}{恐怖}{kong3bu4}[][HSK 7-9]
    \definition[部]{adj.}{terrível; aterrador; horripilante; medo causado por ameaças à vida ou por presenciar violência ou derramamento de sangue | assustador; aterrorizante | terroristas; o comportamento ou os métodos utilizados são extremamente cruéis e perversos, causando choque e medo}
  \end{Phonetics}
\end{Entry}

\begin{Entry}{恐怖主义}{10,8,5,3}{⼼、⼼、⼂、⼂}
  \begin{Phonetics}{恐怖主义}{kong3bu4zhu3yi4}
    \definition{adj.}{terrorista}
    \definition{s.}{terrorismo}
  \end{Phonetics}
\end{Entry}

\begin{Entry}{恐惧}{10,11}{⼼、⼼}
  \begin{Phonetics}{恐惧}{kong3ju4}[][HSK 7-9]
    \definition{adj.}{assustado; com medo; muito assustado}
  \end{Phonetics}
\end{Entry}

\begin{Entry}{恐慌}{10,12}{⼼、⼼}
  \begin{Phonetics}{恐慌}{kong3huang1}[][HSK 7-9]
    \definition{adj.}{pânico; em pânico; pânico devido ao medo}
    \definition{s.}{pânico; medo}
  \end{Phonetics}
\end{Entry}

%%%%%%%%%% 恩 %%%%%%%%%%
\subsection*{恩}

\begin{Entry}{恩}{10}{⼼}
  \begin{Phonetics}{恩}{en1}
    \definition*{s.}{Sobrenome: En}
    \definition{s.}{bondade; favor; graça; gentileza}
  \end{Phonetics}
\end{Entry}

\begin{Entry}{恩人}{10,2}{⼼、⼈}
  \begin{Phonetics}{恩人}{en1 ren2}[][HSK 6]
    \definition{s.}{benfeitor; uma pessoa que ajudou significativamente alguém}
  \end{Phonetics}
\end{Entry}

\begin{Entry}{恩怨}{10,9}{⼼、⼼}
  \begin{Phonetics}{恩怨}{en1yuan4}[][HSK 7-9]
    \definition{s.}{sentimento de gratidão ou ressentimento (inimizade) | ressentimento; queixa; velhas contas}
  \end{Phonetics}
\end{Entry}

\begin{Entry}{恩情}{10,11}{⼼、⼼}
  \begin{Phonetics}{恩情}{en1qing2}[][HSK 7-9]
    \definition{s.}{amor; bondade; afeição profunda}
  \end{Phonetics}
\end{Entry}

\begin{Entry}{恩惠}{10,12}{⼼、⼼}
  \begin{Phonetics}{恩惠}{en1hui4}[][HSK 7-9]
    \definition[份]{s.}{favor; generosidade | bondade; graça; benefícios concedidos ou recebidos}
  \end{Phonetics}
\end{Entry}

\begin{Entry}{恩赐}{10,12}{⼼、⾙}
  \begin{Phonetics}{恩赐}{en1ci4}[][HSK 7-9]
    \definition{s.}{favor; caridade; esmola}
    \definition{v.}{conceder (favores, caridade, etc.); recompensar}
  \end{Phonetics}
\end{Entry}

%%%%%%%%%% 恭 %%%%%%%%%%
\subsection*{恭}

\begin{Entry}{恭}{10}{⼼}
  \begin{Phonetics}{恭}{gong1}
    \definition{adj.}{respeitoso; reverente | educado}
  \end{Phonetics}
\end{Entry}

\begin{Entry}{恭维}{10,11}{⼼、⽷}
  \begin{Phonetics}{恭维}{gong1wei2}[][HSK 7-9]
    \definition{v.}{bajular; elogiar}
  \end{Phonetics}
\end{Entry}

\begin{Entry}{恭喜}{10,12}{⼼、⼝}
  \begin{Phonetics}{恭喜}{gong1xi3}[][HSK 7-9]
    \definition{v.}{parabenizar; uma maneira educada de parabenizar alguém por seu feliz evento}
  \end{Phonetics}
\end{Entry}

%%%%%%%%%% 恳 %%%%%%%%%%
\subsection*{恳}

\begin{Entry}{恳}{10}{⼼}
  \begin{Phonetics}{恳}{ken3}
    \definition{adj.}{sério; sincero | cordial; honesto}
    \definition{v.}{pedir; suplicar; implorar; rogar}
  \end{Phonetics}
\end{Entry}

\begin{Entry}{恳求}{10,7}{⼼、⽔}
  \begin{Phonetics}{恳求}{ken3qiu2}[][HSK 7-9]
    \definition{v.}{implorar; suplicar; rogar; solicitar encarecidamente}
  \end{Phonetics}
\end{Entry}

%%%%%%%%%% 恶 %%%%%%%%%%
\subsection*{恶}

\begin{Entry}{恶}{10}{⼼}
  \begin{Phonetics}{恶}{e3}
    \definition{part.}{elementos formadores de palavras}
  \end{Phonetics}
  \begin{Phonetics}{恶}{e4}[][HSK 7-9]
    \definition{adj.}{feroz | ruim; maligno; perverso | vicioso | feio | grosseiro}
    \definition{s.}{mal; vício; crime (oposto a 善) | maldade; comportamento muito ruim; coisas criminosas}
  \seealsoref{善}{shan4}
  \end{Phonetics}
  \begin{Phonetics}{恶}{wu1}
    \definition{interj.}{Droga!; Ah não!; expressa surpresa}
    \definition{pron.}{como?; por que?; refere-se a um lugar ou coisa; expressa uma pergunta retórica; equivalente a 何 ou 怎么}
  \seealsoref{何}{he2}
  \seealsoref{怎么}{zen3me5}
  \end{Phonetics}
  \begin{Phonetics}{恶}{wu4}
    \definition{v.}{não gostar; odiar; detestar; repugnar}
  \end{Phonetics}
\end{Entry}

\begin{Entry}{恶化}{10,4}{⼼、⼔}
  \begin{Phonetics}{恶化}{e4hua4}[][HSK 7-9]
    \definition{v.}{piorar; deteriorar; exacerbar | piorar a situação}
  \end{Phonetics}
\end{Entry}

\begin{Entry}{恶心}{10,4}{⼼、⼼}
  \begin{Phonetics}{恶心}{e3xin5}[][HSK 4]
    \definition{adj.}{nauseante; repugnante}
    \definition{s.}{náusea; repugnância}
    \definition{v.}{repugnar; ser nauseante; sentir-se mal | envergonhar (deliberadamente)}
  \end{Phonetics}
  \begin{Phonetics}{恶心}{e4xin1}
    \definition{s.}{mau hábito; hábito vicioso; vício}
  \end{Phonetics}
\end{Entry}

\begin{Entry}{恶劣}{10,6}{⼼、⼒}
  \begin{Phonetics}{恶劣}{e4lie4}[][HSK 7-9]
    \definition{adj.}{mau; odioso; abominável; repugnante; desprezível; muito mau; muito ruim}
  \end{Phonetics}
\end{Entry}

\begin{Entry}{恶性}{10,8}{⼼、⼼}
  \begin{Phonetics}{恶性}{e4xing4}[][HSK 7-9]
    \definition{adj.}{maligno; pernicioso; vicioso (oposto a 良性) | produzindo o mal | rápido (declínio) | descontrolada (inflação) | vicioso (círculo) | perverso}
  \seealsoref{良性}{liang2xing4}
  \end{Phonetics}
\end{Entry}

\begin{Entry}{恶意}{10,13}{⼼、⼼}
  \begin{Phonetics}{恶意}{e4yi4}[][HSK 7-9]
    \definition[丝]{s.}{malícia; má vontade; más intenções}
  \end{Phonetics}
\end{Entry}

%%%%%%%%%% 悄 %%%%%%%%%%
\subsection*{悄}

\begin{Entry}{悄}{10}{⼼}
  \begin{Phonetics}{悄}{qiao1}
    \definition{adj.}{quieto; silencioso}
  \end{Phonetics}
  \begin{Phonetics}{悄}{qiao3}
    \definition{adj.}{quieto; silencioso | triste; preocupado; aflito}
  \end{Phonetics}
\end{Entry}

\begin{Entry}{悄悄}{10,10}{⼼、⼼}
  \begin{Phonetics}{悄悄}{qiao1qiao1}[][HSK 5]
    \definition{adv.}{silenciosamente; em silêncio; aos sussuros; sem som ou em voz baixa; com o mínimo de ruído possível}
  \end{Phonetics}
\end{Entry}

%%%%%%%%%% 悔 %%%%%%%%%%
\subsection*{悔}

\begin{Entry}{悔}{10}{⼼}
  \begin{Phonetics}{悔}{hui3}
    \definition{v.}{lamentar; arrepender-se}
  \end{Phonetics}
\end{Entry}

\begin{Entry}{悔恨}{10,9}{⼼、⼼}
  \begin{Phonetics}{悔恨}{hui3hen4}[][HSK 7-9]
    \definition{v.}{arrepender-se profundamente; estar amargamente arrependido}
  \end{Phonetics}
\end{Entry}

%%%%%%%%%% 扇 %%%%%%%%%%
\subsection*{扇}

\begin{Entry}{扇}{10}{⼾}
  \begin{Phonetics}{扇}{shan1}[][HSK 5]
    \definition{s.}{ventilar; agitar um leque para fazer o ar circular | dar um tapa; bater com a palma da mão | bater asas; esvoaçar | incitar; instigar; estimular; agitar}
  \end{Phonetics}
  \begin{Phonetics}{扇}{shan4}[][HSK 5]
    \definition{clas.}{usado para portas, janelas, etc.}
    \definition[把]{s.}{leque | folha; algo em forma de placa ou folha}
  \end{Phonetics}
\end{Entry}

\begin{Entry}{扇子}{10,3}{⼾、⼦}
  \begin{Phonetics}{扇子}{shan4zi5}[][HSK 5]
    \definition[把,个]{s.}{leque; abano; abanador; utensílios que produzem vento ao serem agitados}
  \end{Phonetics}
\end{Entry}

%%%%%%%%%% 拳 %%%%%%%%%%
\subsection*{拳}

\begin{Entry}{拳}{10}{⼿}
  \begin{Phonetics}{拳}{quan2}
    \definition*{s.}{Sobrenome: Quan}
    \definition[个,记,套]{s.}{punho | boxe; pugilismo}
    \definition{v.}{enrolar}
  \end{Phonetics}
\end{Entry}

\begin{Entry}{拳王}{10,4}{⼿、⽟}
  \begin{Phonetics}{拳王}{quan2wang2}
    \definition{s.}{pugilista | boxeador}
  \end{Phonetics}
\end{Entry}

\begin{Entry}{拳法}{10,8}{⼿、⽔}
  \begin{Phonetics}{拳法}{quan2fa3}
    \definition{s.}{boxe | luta}
  \end{Phonetics}
\end{Entry}

%%%%%%%%%% 拿 %%%%%%%%%%
\subsection*{拿}

\begin{Entry}{拿}{10}{⼿}
  \begin{Phonetics}{拿}{na2}[][HSK 1]
    \definition{part.}{usado da mesma forma que 把: para marcar o seguinte substantivo seguinte como objeto direto}
    \definition{prep.}{ferramentas, materiais, métodos, etc. utilizados para a introdução | os objetos que estão sendo manipulados para introdução}
    \definition{v.}{segurar; pegar; pegar ou mover objetos com as mãos ou de outra forma | apreender; capturar; prender; usar força bruta para capturar | ter certeza de; ser capaz de fazer; ter uma compreensão firme de | tornar as coisas difíceis para alguém; colocar alguém em uma situação difícil; obstruir; chantagear; coagir; causar dificuldades intencionalmente | fingir ou fazer (algum tipo de postura ou aparência) | ter certeza de; tomar uma decisão | obter; ganhar; receber}
  \end{Phonetics}
\end{Entry}

\begin{Entry}{拿出}{10,5}{⼿、⼐}
  \begin{Phonetics}{拿出}{na2 chu1}[][HSK 2]
    \definition{v.}{apresentar (evidências) | fornecer | apresentar (uma proposta) | oferecer; servir | retirar; tirar}
  \end{Phonetics}
\end{Entry}

\begin{Entry}{拿走}{10,7}{⼿、⾛}
  \begin{Phonetics}{拿走}{na2 zou3}[][HSK 6]
    \definition{v.}{tirar; remover}
  \end{Phonetics}
\end{Entry}

\begin{Entry}{拿到}{10,8}{⼿、⼑}
  \begin{Phonetics}{拿到}{na2 dao4}[][HSK 2]
    \definition{v.}{pegar; obter, conseguir}
  \end{Phonetics}
\end{Entry}

%%%%%%%%%% 挨 %%%%%%%%%%
\subsection*{挨}

\begin{Entry}{挨}{10}{⼿}
  \begin{Phonetics}{挨}{ai1}
    \definition{prep.}{por turnos; em sequência; indica sequencialmente}
    \definition{v.}{estar próximo de; estar (ou chegar) perto de; abordar}
  \end{Phonetics}
  \begin{Phonetics}{挨}{ai2}[][HSK 6]
    \definition{v.}{sofrer; suportar | arrastar-se; lutar para sobreviver (tempos difíceis); passar (tempo) com dificuldade | parar; atrasar; adiar; procrastinar}
  \end{Phonetics}
\end{Entry}

\begin{Entry}{挨打}{10,5}{⼿、⼿}
  \begin{Phonetics}{挨打}{ai2/da3}[][HSK 6]
    \definition{v.+compl.}{levar uma surra; ser atacado; ser espancado}
  \end{Phonetics}
\end{Entry}

\begin{Entry}{挨家挨户}{10,10,10,4}{⼿、⼧、⼿、⼾}
  \begin{Phonetics}{挨家挨户}{ai1jia1-ai1hu4}[][HSK 7-9]
    \definition{expr.}{ir de casa em casa, de porta em porta ; um após o outro}
  \end{Phonetics}
\end{Entry}

\begin{Entry}{挨着}{10,11}{⼿、⽬}
  \begin{Phonetics}{挨着}{ai1 zhe5}[][HSK 6]
    \definition{adv.}{ao lado de; perto de; imediatamente depois}
  \end{Phonetics}
\end{Entry}

%%%%%%%%%% 挫 %%%%%%%%%%
\subsection*{挫}

\begin{Entry}{挫}{10}{⼿}
  \begin{Phonetics}{挫}{cuo4}
    \definition{v.}{frustrar | diminuir; embotar; desinflar | pressionar para baixo; abaixar}
  \end{Phonetics}
\end{Entry}

\begin{Entry}{挫折}{10,7}{⼿、⼿}
  \begin{Phonetics}{挫折}{cuo4zhe2}[][HSK 7-9]
    \definition[个,次]{s.}{retrocesso; reversão; frustração | derrota, fracasso, insucesso}
    \definition{v.}{falhar; derrotar; fracassar}
  \end{Phonetics}
\end{Entry}

%%%%%%%%%% 振 %%%%%%%%%%
\subsection*{振}

\begin{Entry}{振}{10}{⼿}
  \begin{Phonetics}{振}{zhen4}
    \definition{v.}{sacudir; acenar; bater as asas; empunhar | vibrar | recompor-se; levantar-se; animar}
  \end{Phonetics}
\end{Entry}

\begin{Entry}{振动}{10,6}{⼿、⼒}
  \begin{Phonetics}{振动}{zhen4dong4}[][HSK 5]
    \definition{s.}{vibração}
    \definition{v.}{sacudir; balançar; tremer; roncar; tagarelar; vibrar; oscilar; a física se refere ao movimento contínuo de um objeto em torno de um determinado ponto no espaço, como o movimento de um pêndulo, um diapasão ou uma corda de violão}
  \end{Phonetics}
\end{Entry}

%%%%%%%%%% 捆 %%%%%%%%%%
\subsection*{捆}

\begin{Entry}{捆}{10}{⼿}
  \begin{Phonetics}{捆}{kun3}[][HSK 7-9]
    \definition{clas.}{feixe; maço; materiais usados ​​para amarrar}
    \definition{s.}{coisas que estão agrupadas}
    \definition{v.}{amarrar; prender; agrupar | amarrar; acorrentar; algemar | agrupar; enfardar}
  \seealsoref{捆儿}{kun3r5}
  \end{Phonetics}
\end{Entry}

\begin{Entry}{捆儿}{10,2}{⼿、⼉}
  \begin{Phonetics}{捆儿}{kun3r5}
    \definition{s.}{coisas que estão agrupadas}
  \seealsoref{捆}{kun3}
  \end{Phonetics}
\end{Entry}

%%%%%%%%%% 捉 %%%%%%%%%%
\subsection*{捉}

\begin{Entry}{捉}{10}{⼿}
  \begin{Phonetics}{捉}{zhuo1}[][HSK 6]
    \definition{v.}{agarrar; segurar; apreender | pegar; capturar; aprisionar}
  \end{Phonetics}
\end{Entry}

%%%%%%%%%% 捍 %%%%%%%%%%
\subsection*{捍}

\begin{Entry}{捍}{10}{⼿}
  \begin{Phonetics}{捍}{han4}
    \definition{v.}{defender; guardar | defender-se | afastar (um golpe) | resistir}
  \end{Phonetics}
\end{Entry}

\begin{Entry}{捍卫}{10,3}{⼿、⼙}
  \begin{Phonetics}{捍卫}{han4wei4}[][HSK 7-9]
    \definition{v.}{defender; guardar; proteger; defender-se pela força ou outros meios de ser violado ou prejudicado}
  \end{Phonetics}
\end{Entry}

%%%%%%%%%% 捐 %%%%%%%%%%
\subsection*{捐}

\begin{Entry}{捐}{10}{⼿}
  \begin{Phonetics}{捐}{juan1}[][HSK 6]
    \definition{s.}{imposto}
    \definition{v.}{renunciar; abandonar | contribuir; doar; assinar}
  \end{Phonetics}
\end{Entry}

\begin{Entry}{捐助}{10,7}{⼿、⼒}
  \begin{Phonetics}{捐助}{juan1 zhu4}[][HSK 6]
    \definition{v.}{oferecer (assistência financeira ou material); contribuir; doar}
  \end{Phonetics}
\end{Entry}

\begin{Entry}{捐款}{10,12}{⼿、⽋}
  \begin{Phonetics}{捐款}{juan1/kuan3}[][HSK 6]
    \definition[笔]{s.}{doação; contribuição (de dinheiro); valor doado}
    \definition{v.+compl.}{doar; contribuir com dinheiro}
  \end{Phonetics}
\end{Entry}

\begin{Entry}{捐献}{10,13}{⼿、⽝}
  \begin{Phonetics}{捐献}{juan1xian4}[][HSK 7-9]
    \definition{v.}{doar; apresentar; contribuir (para uma organização); doar bens ao (estado, a uma cooperativa, etc.)}
  \end{Phonetics}
\end{Entry}

\begin{Entry}{捐赠}{10,16}{⼿、⾙}
  \begin{Phonetics}{捐赠}{juan1 zeng4}[][HSK 6]
    \definition{v.}{apresentar; contribuir (como um presente); doar (itens para um país ou grupo)}
  \end{Phonetics}
\end{Entry}

%%%%%%%%%% 捕 %%%%%%%%%%
\subsection*{捕}

\begin{Entry}{捕}{10}{⼿}
  \begin{Phonetics}{捕}{bu3}[][HSK 6]
    \definition{v.}{pegar; apreender; prender}
  \end{Phonetics}
\end{Entry}

\begin{Entry}{捕捉}{10,10}{⼿、⼿}
  \begin{Phonetics}{捕捉}{bu3zhuo1}[][HSK 7-9]
    \definition{v.}{caçar; perseguir; pegar; capturar; apreender; pegar; fazer uma pessoa ou animal cair nas mãos; pode ser usado tanto para pessoas quanto para coisas; tem uma ampla gama de aplicações; usado tanto na linguagem falada quanto na escrita}
  \end{Phonetics}
\end{Entry}

%%%%%%%%%% 捞 %%%%%%%%%%
\subsection*{捞}

\begin{Entry}{捞}{10}{⼿}
  \begin{Phonetics}{捞}{lao1}[][HSK 7-9]
    \definition{v.}{arrastar para; pescar para; recolher; dragar (para fora); retirar algo da água ou de outros líquidos | ganhar; obter por meios ilícitos | sair andando com alguma coisa; puxar ou pegar casualmente}
  \end{Phonetics}
\end{Entry}

%%%%%%%%%% 损 %%%%%%%%%%
\subsection*{损}

\begin{Entry}{损}{10}{⼿}
  \begin{Phonetics}{损}{sun3}
    \definition{adj.}{sarcástico; cortante; de ​​língua afiada; maldoso; mau; cruel}
    \definition{v.}{diminuir; perder; reduzir | prejudicar; danificar; degradar; destruir; arruinar; destruir o estado original ou fazê-lo perder sua eficácia original | ser sarcástico; ser cáustico; ser cortante; ferir; insultar; usar palavras duras para zombar de alguém}
  \end{Phonetics}
\end{Entry}

\begin{Entry}{损失}{10,5}{⼿、⼤}
  \begin{Phonetics}{损失}{sun3shi1}[][HSK 5]
    \definition{s.}{perda; desperdício; algo que se consome ou se perde sem custo algum}
    \definition{v.}{perder; consumir ou perder}
  \end{Phonetics}
\end{Entry}

\begin{Entry}{损害}{10,10}{⼿、⼧}
  \begin{Phonetics}{损害}{sun3 hai4}[][HSK 5]
    \definition{v.}{prejudicar; danificar; ferir; causar danos; causar perdas}
  \end{Phonetics}
\end{Entry}

%%%%%%%%%% 捡 %%%%%%%%%%
\subsection*{捡}

\begin{Entry}{捡}{10}{⼿}
  \begin{Phonetics}{捡}{jian3}[][HSK 6]
    \definition{v.}{coletar; reunir; apanhar; pegar coisas do chão}
  \end{Phonetics}
\end{Entry}

%%%%%%%%%% 换 %%%%%%%%%%
\subsection*{换}

\begin{Entry}{换}{10}{⼿}
  \begin{Phonetics}{换}{huan4}[][HSK 2]
    \definition{v.}{negociar; trocar; permutar; dar algo a alguém e, ao mesmo tempo, obter algo dele em troca | mudar; transformar; substituir | trocar dinheiro (câmbio) | transfundir (sangue) | transplantar (um órgão)}
  \end{Phonetics}
\end{Entry}

\begin{Entry}{换成}{10,6}{⼿、⼽}
  \begin{Phonetics}{换成}{huan4cheng2}[][HSK 7-9]
    \definition{v.}{trocar (algo) por (outro); indica a substituição de um objeto, estado ou situação por outro}
  \end{Phonetics}
\end{Entry}

\begin{Entry}{换位}{10,7}{⼿、⼈}
  \begin{Phonetics}{换位}{huan4wei4}[][HSK 7-9]
    \definition{v.}{trocar posições; transpor | mudar de posição}
  \end{Phonetics}
\end{Entry}

\begin{Entry}{换言之}{10,7,3}{⼿、⾔、⼂}
  \begin{Phonetics}{换言之}{huan4yan2zhi1}[][HSK 7-9]
    \definition{adv.}{em outras palavras}
  \end{Phonetics}
\end{Entry}

\begin{Entry}{换取}{10,8}{⼿、⼜}
  \begin{Phonetics}{换取}{huan4qu3}[][HSK 7-9]
    \definition{v.}{trocar (ou escambo) algo por; obter em troca | trocar algo por; obter por troca}
  \end{Phonetics}
\end{Entry}

\begin{Entry}{换钱}{10,10}{⼿、⾦}
  \begin{Phonetics}{换钱}{huan4/qian2}
    \definition{v.+compl.}{trocar dinheiro (em pequenas valores ou em outra moeda) | trocar (mercadorias) por dinheiro | vender}
  \end{Phonetics}
\end{Entry}

%%%%%%%%%% 捣 %%%%%%%%%%
\subsection*{捣}

\begin{Entry}{捣}{10}{⼿}
  \begin{Phonetics}{捣}{dao3}
    \definition{v.}{bater com um pilão, etc.; bater; esmagar | assediar; perturbar | bater com um pedaço de pau}
  \end{Phonetics}
\end{Entry}

\begin{Entry}{捣乱}{10,7}{⼿、⼄}
  \begin{Phonetics}{捣乱}{dao3/luan4}[][HSK 7-9]
    \definition{v.+compl.}{causar problemas; criar uma perturbação; causar intencionalmente problemas para os outros; interromper | perturbar; interferir com; causar problemas intencionalmente}
  \end{Phonetics}
\end{Entry}

%%%%%%%%%% 效 %%%%%%%%%%
\subsection*{效}

\begin{Entry}{效}{10}{⽁}
  \begin{Phonetics}{效}{xiao4}
    \definition{s.}{efeito; função | eficiência; resultado}
    \definition{v.}{imitar; seguir o exemplo de | dedicar (a energia ou a vida de alguém) a; prestar (um serviço)}
  \end{Phonetics}
\end{Entry}

\begin{Entry}{效果}{10,8}{⽁、⽊}
  \begin{Phonetics}{效果}{xiao4guo3}[][HSK 3]
    \definition[种,个]{s.}{efeito; resultado | efeitos sonoros; vários sons ou fenômenos naturais criados para combinar com o enredo em dramas e filmes, como vento e chuva, tiros, fogo, neve, etc.}
  \end{Phonetics}
\end{Entry}

\begin{Entry}{效率}{10,11}{⽁、⽞}
  \begin{Phonetics}{效率}{xiao4lv4}[][HSK 4]
    \definition[种]{s.}{eficiência; produtividade; a quantidade de trabalho concluído por unidade de tempo}
  \end{Phonetics}
\end{Entry}

%%%%%%%%%% 敌 %%%%%%%%%%
\subsection*{敌}

\begin{Entry}{敌}{10}{⾆}
  \begin{Phonetics}{敌}{di2}
    \definition[个,名,位,种]{s.}{inimigo; adversário}
    \definition{v.}{opor-se; lutar; resistir; suportar | combinar; igualar}
  \end{Phonetics}
\end{Entry}

\begin{Entry}{敌人}{10,2}{⾆、⼈}
  \begin{Phonetics}{敌人}{di2ren2}[][HSK 4]
    \definition[群,伙,帮,个,队]{s.}{inimigo; pessoa hostil; parte hostil}
  \end{Phonetics}
\end{Entry}

%%%%%%%%%% 料 %%%%%%%%%%
\subsection*{料}

\begin{Entry}{料}{10}{⽃}
  \begin{Phonetics}{料}{liao4}[][HSK 6]
    \definition{clas.}{usado na medicina tradicional chinesa para preparar pílulas | unidade usada para calcular um pedaço de madeira, é a seção transversal em ambas as extremidades, que é de 1 pé (quadrado) com 7 pés de comprimento}
    \definition{s.}{material; coisa | (grão) alimento; forragem | artigos de vidro; vidros coloridos opacos | (para pílulas de medicina chinesa) prescrição}
    \definition{v.}{supor; esperar; antecipar | gerenciar; cuidar de | prever}
  \end{Phonetics}
\end{Entry}

\begin{Entry}{料到}{10,8}{⽃、⼑}
  \begin{Phonetics}{料到}{liao4dao4}[][HSK 7-9]
    \definition{v.}{prever; esperar; significa que as coisas estão se desenvolvendo conforme o esperado}
  \end{Phonetics}
\end{Entry}

\begin{Entry}{料理}{10,11}{⽃、⽟}
  \begin{Phonetics}{料理}{liao4li3}[][HSK 7-9]
    \definition[个,种]{s.}{prato; culinária; cozinha; refere-se a um certo estilo de culinária}
    \definition{v.}{organizar; gerir; cuidar de; dar atenção a; cuidar; lidar com isso | cozinhar; preparar alimentos}
  \end{Phonetics}
\end{Entry}

%%%%%%%%%% 旁 %%%%%%%%%%
\subsection*{旁}

\begin{Entry}{旁}{10}{⽅}
  \begin{Phonetics}{旁}{pang2}[][HSK 5]
    \definition{adj.}{outro | abundante; abrangente}
    \definition{s.}{lado | radical lateral de um caractere chinês}
  \end{Phonetics}
\end{Entry}

\begin{Entry}{旁边}{10,5}{⽅、⾡}
  \begin{Phonetics}{旁边}{pang2bian1}[][HSK 1]
    \definition{s.}{junto a; próximo de; ao lado}
  \end{Phonetics}
\end{Entry}

%%%%%%%%%% 旅 %%%%%%%%%%
\subsection*{旅}

\begin{Entry}{旅}{10}{⽅}
  \begin{Phonetics}{旅}{lv3}
    \definition{adv.}{juntos; conjuntamente}
    \definition[个]{s.}{brigada; unidade organizacional militar, abaixo do nível de divisão e acima do nível de regimento ou batalhão | força; tropas; geralmente se refere aos militares | viajante; passageiro; turista | viagem; jornada | pessoas}
    \definition{v.}{viajar; ficar longe de casa; ir para longe; morar longe de casa}
  \end{Phonetics}
\end{Entry}

\begin{Entry}{旅行}{10,6}{⽅、⾏}
  \begin{Phonetics}{旅行}{lv3xing2}[][HSK 2]
    \definition{v.}{viajar; passear; para tratar de assuntos ou passear, ir de um lugar para outro (geralmente se refere a distâncias longas)}
  \end{Phonetics}
\end{Entry}

\begin{Entry}{旅行社}{10,6,7}{⽅、⾏、⽰}
  \begin{Phonetics}{旅行社}{lv3 xing2 she4}[][HSK 3]
    \definition[家]{s.}{agência de viagens; agência especializada em serviços relacionados a viagens, que providencia hospedagem, transporte e outros serviços para viajantes}
  \end{Phonetics}
\end{Entry}

\begin{Entry}{旅店}{10,8}{⽅、⼴}
  \begin{Phonetics}{旅店}{lv3 dian5}[][HSK 6]
    \definition[家,个]{s.}{pousada; albergue; hotel}
  \end{Phonetics}
\end{Entry}

\begin{Entry}{旅客}{10,9}{⽅、⼧}
  \begin{Phonetics}{旅客}{lv3 ke4}[][HSK 2]
    \definition[名,位,个,些]{s.}{viajante; passageiro; as agências de transporte e turismo referem-se às pessoas que viajam}
  \end{Phonetics}
\end{Entry}

\begin{Entry}{旅途}{10,10}{⽅、⾡}
  \begin{Phonetics}{旅途}{lv3tu2}[][HSK 7-9]
    \definition{s.}{viagem; jornada; durante a viagem}
  \end{Phonetics}
\end{Entry}

\begin{Entry}{旅馆}{10,11}{⽅、⾷}
  \begin{Phonetics}{旅馆}{lv3 guan3}[][HSK 3]
    \definition[家,个,所]{s.}{pousada; hotel; local comercial destinado ao alojamento de viajantes}
  \end{Phonetics}
\end{Entry}

\begin{Entry}{旅游}{10,12}{⽅、⽔}
  \begin{Phonetics}{旅游}{lv3you2}[][HSK 2]
    \definition{v.}{viajar para outros lugares para passear e fazer turismo}
  \end{Phonetics}
\end{Entry}

\begin{Entry}{旅程}{10,12}{⽅、⽲}
  \begin{Phonetics}{旅程}{lv3cheng2}[][HSK 7-9]
    \definition[条,段]{s.}{rota; itinerário; viagem; distância percorrida; a jornada}
  \end{Phonetics}
\end{Entry}

%%%%%%%%%% 晃 %%%%%%%%%%
\subsection*{晃}

\begin{Entry}{晃}{10}{⽇}
  \begin{Phonetics}{晃}{huang3}[][HSK 7-9]
    \definition*{s.}{Sobrenome: Huang}
    \definition{adj.}{deslumbrante}
    \definition{v.}{passar rapidamente | deslumbrar; cegar}
  \end{Phonetics}
  \begin{Phonetics}{晃}{huang4}[][HSK 7-9]
    \definition{v.}{sacudir; balançar}
  \end{Phonetics}
\end{Entry}

\begin{Entry}{晃荡}{10,9}{⽇、⾋}
  \begin{Phonetics}{晃荡}{huang4dang5}[][HSK 7-9]
    \definition{v.}{balançar; sacudir | Coloquial: vagar; ficar ocioso; divagar | oscilar}
  \end{Phonetics}
\end{Entry}

%%%%%%%%%% 晋 %%%%%%%%%%
\subsection*{晋}

\begin{Entry}{晋}{10}{⽇}
  \begin{Phonetics}{晋}{jin4}
    \definition*{s.}{Estado da Dinastia Zhou (1046-256 a.C.), ocupando partes do que hoje são Shanxi, Shaanxi, Hebei e Henan |
Dinastia Jin Ocidental (265-316), Dinastia Jin Oriental (317-420) e Dinastia Jin Posterior (936-946) | Nome abreviado da província de Shanxi: 山西 | Sobrenome: Jin}
    \definition{v.}{avançar | promover}
  \seealsoref{山西}{shan1xi1}
  \end{Phonetics}
\end{Entry}

\begin{Entry}{晋升}{10,4}{⽇、⼗}
  \begin{Phonetics}{晋升}{jin4sheng1}[][HSK 7-9]
    \definition{v.}{elevar; promover (a um cargo superior)}
  \end{Phonetics}
\end{Entry}

%%%%%%%%%% 晒 %%%%%%%%%%
\subsection*{晒}

\begin{Entry}{晒}{10}{⽇}
  \begin{Phonetics}{晒}{shai4}[][HSK 4]
    \definition{v.}{(sol) brilhar sobre | aquecer-se; secar ao sol; colocar algo sob a luz do sol para secar | ignorar (alguém) | mostrar; divulgar o conteúdo de sua vida privada na Internet}
  \end{Phonetics}
\end{Entry}

\begin{Entry}{晒干}{10,3}{⽇、⼲}
  \begin{Phonetics}{晒干}{shai4gan1}
    \definition{v.}{secar ao sol}
  \end{Phonetics}
\end{Entry}

%%%%%%%%%% 晓 %%%%%%%%%%
\subsection*{晓}

\begin{Entry}{晓}{10}{⽇}
  \begin{Phonetics}{晓}{xiao3}
    \definition{s.}{amanhecer; alvorada}
    \definition{v.}{(um dia) amanhecer; romper | saber; deixar alguém saber; dizer}
  \end{Phonetics}
\end{Entry}

\begin{Entry}{晓得}{10,11}{⽇、⼻}
  \begin{Phonetics}{晓得}{xiao3 de2}[][HSK 6]
    \definition{v.}{saber; entender}[我不晓得他在哪里。===Não sei onde ele está.]
  \end{Phonetics}
\end{Entry}

%%%%%%%%%% 晕 %%%%%%%%%%
\subsection*{晕}

\begin{Entry}{晕}{10}{⽇}
  \begin{Phonetics}{晕}{yun1}[][HSK 6]
    \definition{adj.}{tonto; vertiginoso; confuso; sensação de que as coisas estão girando ao seu redor e, às vezes, sensação de que você vai cair}
    \definition{v.}{desmaiar; desfalecer}
  \end{Phonetics}
  \begin{Phonetics}{晕}{yun4}
    \definition{s.}{auréola; o círculo de luz formado pela refração da luz solar ou do luar através dos cristais de gelo nas nuvens | halo em torno de alguma cor ou luz; áreas desfocadas em torno de luz, sombra e cor}
    \definition{v.}{ficar tonto; desmaiar; desfalecer; sensação de tontura, como se os objetos ao seu redor estivessem girando e como se você estivesse prestes a cair}
  \end{Phonetics}
\end{Entry}

\begin{Entry}{晕车}{10,4}{⽇、⾞}
  \begin{Phonetics}{晕车}{yun4 che1}[][HSK 6]
    \definition{v.}{ter enjoo no carro; ter tontura e vômito ao andar de carro}
  \end{Phonetics}
\end{Entry}

%%%%%%%%%% 朗 %%%%%%%%%%
\subsection*{朗}

\begin{Entry}{朗}{10}{⽉}
  \begin{Phonetics}{朗}{lang3}
    \definition*{s.}{Sobrenome: Lang}
    \definition{adj.}{claro; brilhante | alto e claro (som)}
  \end{Phonetics}
\end{Entry}

\begin{Entry}{朗诵}{10,9}{⽉、⾔}
  \begin{Phonetics}{朗诵}{lang3song4}[][HSK 7-9]
    \definition{v.}{recitar; ler em voz alta com expressividade; ler poemas ou prosa em voz alta para expressar as emoções transmitidas pela obra}
  \end{Phonetics}
\end{Entry}

\begin{Entry}{朗读}{10,10}{⽉、⾔}
  \begin{Phonetics}{朗读}{lang3du2}[][HSK 5]
    \definition{v.}{ler em voz alta; recitar com voz clara e alta}
  \end{Phonetics}
\end{Entry}

%%%%%%%%%% 校 %%%%%%%%%%
\subsection*{校}

\begin{Entry}{校}{10}{⽊}
  \begin{Phonetics}{校}{jiao4}
    \definition{v.}{verificar | comparar | revisar}
  \end{Phonetics}
  \begin{Phonetics}{校}{xiao4}
    \definition[所]{s.}{oficial militar | escola}
  \end{Phonetics}
\end{Entry}

\begin{Entry}{校长}{10,4}{⽊、⾧}
  \begin{Phonetics}{校长}{xiao4zhang3}[][HSK 2]
    \definition[个,位,名]{s.}{diretor; presidente; reitor; o mais alto líder administrativo e empresarial de uma escola}
  \end{Phonetics}
\end{Entry}

\begin{Entry}{校园}{10,7}{⽊、⼞}
  \begin{Phonetics}{校园}{xiao4 yuan2}[][HSK 2]
    \definition[个]{s.}{campus; pátio da escola; refere-se a todos os terrenos e edifícios dentro da área escolar}
  \end{Phonetics}
\end{Entry}

\begin{Entry}{校服}{10,8}{⽊、⽉}
  \begin{Phonetics}{校服}{xiao4fu2}
    \definition{s.}{uniforme escolar}
  \end{Phonetics}
\end{Entry}

\begin{Entry}{校规}{10,8}{⽊、⾒}
  \begin{Phonetics}{校规}{xiao4gui1}
    \definition{s.}{regras e regulamentos escolares}
  \end{Phonetics}
\end{Entry}

\begin{Entry}{校监}{10,10}{⽊、⽫}
  \begin{Phonetics}{校监}{xiao4jian1}
    \definition{s.}{diretor | supervisor (de escola)}
  \end{Phonetics}
\end{Entry}

%%%%%%%%%% 样 %%%%%%%%%%
\subsection*{样}

\begin{Entry}{样}{10}{⽊}
  \begin{Phonetics}{样}{yang4}[][HSK 6]
    \definition{clas.}{usado para tipos de coisas}[这里有四样东西。===Há quatro coisas aqui.]
    \definition[个]{s.}{aparência; aspecto;  forma; aparência; a forma do objeto | modelo; amostra; padrão; coisas usadas como padrões | ar; maneira; aparência; a aparência ou expressão de uma pessoa | tendência; probabilidade; a situação ou tendência das coisas}
  \end{Phonetics}
\end{Entry}

\begin{Entry}{样儿}{10,2}{⽊、⼉}
  \begin{Phonetics}{样儿}{yang4r5}
    \definition{s.}{aparência | forma | modelo}
  \seealsoref{样子}{yang4zi5}
  \end{Phonetics}
\end{Entry}

\begin{Entry}{样子}{10,3}{⽊、⼦}
  \begin{Phonetics}{样子}{yang4zi5}[][HSK 2]
    \definition[个,种,副]{s.}{forma; aparência; estilo | ar; maneira; modalidade; estado | tendência; probabilidade; usado com 看 e 照 para expressar uma estimativa de uma tendência | modelo; amostra; padrão; uma pessoa ou coisa que pode ser usada como um padrão para as pessoas verificarem, seguirem ou aprenderem com ela}
  \seealsoref{看}{kan4}
  \seealsoref{样儿}{yang4r5}
  \seealsoref{照}{zhao4}
  \end{Phonetics}
\end{Entry}

\begin{Entry}{样品}{10,9}{⽊、⼝}
  \begin{Phonetics}{样品}{yang4pin3}
    \definition{s.}{amostra | espécime}
  \end{Phonetics}
\end{Entry}

\begin{Entry}{样样}{10,10}{⽊、⽊}
  \begin{Phonetics}{样样}{yang4yang4}
    \definition{adv.}{todos os tipos}
  \end{Phonetics}
\end{Entry}

\begin{Entry}{样章}{10,11}{⽊、⾳}
  \begin{Phonetics}{样章}{yang4zhang1}
    \definition{s.}{capítulo de amostra}
  \end{Phonetics}
\end{Entry}

%%%%%%%%%% 核 %%%%%%%%%%
\subsection*{核}

\begin{Entry}{核}{10}{⽊}
  \begin{Phonetics}{核}{he2}[][HSK 7-9]
    \definition{adj.}{Literário: verdadeiro; fiel}
    \definition{s.}{poço; pedra; caroço | núcleo | núcleo atômico}
    \definition{v.}{examinar; verificar}
  \end{Phonetics}
  \begin{Phonetics}{核}{hu2}
    \definition{s.}{semente; o mesmo que 核}
  \end{Phonetics}
\end{Entry}

\begin{Entry}{核心}{10,4}{⽊、⼼}
  \begin{Phonetics}{核心}{he2xin1}[][HSK 6]
    \definition[个]{s.}{núcleo; elite; coração; centro; parte principal (em termos de relacionamento entre as coisas)}
  \end{Phonetics}
\end{Entry}

\begin{Entry}{核对}{10,5}{⽊、⼨}
  \begin{Phonetics}{核对}{he2dui4}[][HSK 7-9]
    \definition{v.}{verificar; checar; verificar cuidadosamente (para ver se corresponde)}
  \end{Phonetics}
\end{Entry}

\begin{Entry}{核电站}{10,5,10}{⽊、⽥、⽴}
  \begin{Phonetics}{核电站}{he2dian4zhan4}[][HSK 7-9]
    \definition{s.}{usina nuclear; usina que utiliza energia nuclear para gerar eletricidade}
  \end{Phonetics}
\end{Entry}

\begin{Entry}{核实}{10,8}{⽊、⼧}
  \begin{Phonetics}{核实}{he2shi2}[][HSK 7-9]
    \definition{v.}{verificar; checar; verificar se é verdade}
  \end{Phonetics}
\end{Entry}

\begin{Entry}{核武器}{10,8,16}{⽊、⽌、⼝}
  \begin{Phonetics}{核武器}{he2wu3qi4}[][HSK 7-9]
    \definition[个]{s.}{arma nuclear}
  \end{Phonetics}
\end{Entry}

\begin{Entry}{核桃}{10,10}{⽊、⽊}
  \begin{Phonetics}{核桃}{he2tao5}[][HSK 7-9]
    \definition[颗,个,棵,顆]{s.}{noz | nogueira}
  \end{Phonetics}
\end{Entry}

\begin{Entry}{核能}{10,10}{⽊、⾁}
  \begin{Phonetics}{核能}{he2neng2}[][HSK 7-9]
    \definition{s.}{energia nuclear}
  \end{Phonetics}
\end{Entry}

%%%%%%%%%% 根 %%%%%%%%%%
\subsection*{根}

\begin{Entry}{根}{10}{⽊}
  \begin{Phonetics}{根}{gen1}[][HSK 4]
    \definition*{s.}{Sobrenome: Gen}
    \definition{adv.}{completamente; minuciosamente; radicalmente}
    \definition{clas.}{usado para objetos finos, alongados}
    \definition{s.}{raiz (de uma planta) | descendentes; posteridade; analogia com as gerações futuras | raiz (abreviação de raiz quadrada) | radical (química, refere-se a radicais carregados) | base; pé; raiz; parte inferior, base ou parte de um objeto que está presa a outra coisa | a parte de baixo das coisas; fonte; a origem  das coisas | base; fundamento}
  \end{Phonetics}
\end{Entry}

\begin{Entry}{根本}{10,5}{⽊、⽊}
  \begin{Phonetics}{根本}{gen1ben3}[][HSK 3]
    \definition{adj.}{básico; essencial; fundamental; importante; decisivo}
    \definition{adv.}{nunca; simplesmente; de forma alguma | radicalmente; completamente; nunca (mais usado em negativas)}
    \definition[个]{s.}{base; raiz; fundação; a origem, a base ou a parte mais importante das coisas}
  \end{Phonetics}
\end{Entry}

\begin{Entry}{根治}{10,8}{⽊、⽔}
  \begin{Phonetics}{根治}{gen1zhi4}[][HSK 7-9]
    \definition{v.}{efetuar uma cura radical; curar de uma vez por todas; colocar sob controle permanente; curar completamente (referindo-se à erradicação de pragas ou doenças)}
  \end{Phonetics}
\end{Entry}

\begin{Entry}{根基}{10,11}{⽊、⼟}
  \begin{Phonetics}{根基}{gen1ji1}[][HSK 7-9]
    \definition{s.}{base; fundação; alicerce; parte subterrânea de um edifício | recursos; propriedade acumulada ao longo de um longo período}
  \end{Phonetics}
\end{Entry}

\begin{Entry}{根据}{10,11}{⽊、⼿}
  \begin{Phonetics}{根据}{gen1ju4}[][HSK 4]
    \definition{prep.}{com base em; de acordo com; à luz de}
    \definition[个]{s.}{base; fundamentos; razão; fundo; alicerce}
    \definition{v.}{basear; usar algo como premissa para uma conclusão ou como base para uma ação verbal}
  \end{Phonetics}
\end{Entry}

\begin{Entry}{根深蒂固}{10,11,12,8}{⽊、⽔、⾋、⼞}
  \begin{Phonetics}{根深蒂固}{gen1shen1-di4gu4}[][HSK 7-9]
    \definition{expr.}{arraigado; inveterado; tornar-se profundamente enraizado em; profundamente enraizado; profundamente enraizado; profundamente enraizado e firmemente plantado -- bem fundado; ter uma base firme; ter raízes profundas e uma base firme; bem estabelecido; significa que a fundação é sólida e não se abala facilmente}
  \end{Phonetics}
\end{Entry}

\begin{Entry}{根源}{10,13}{⽊、⽔}
  \begin{Phonetics}{根源}{gen1yuan2}[][HSK 7-9]
    \definition{s.}{fonte; origem; raiz | raízes da grama; fonte; nascente; raiz; fundo}
    \definition{v.}{originar-se; provir de}
  \end{Phonetics}
\end{Entry}

%%%%%%%%%% 格 %%%%%%%%%%
\subsection*{格}

\begin{Entry}{格}{10}{⽊}
  \begin{Phonetics}{格}{ge1}
    \definition{s.}{Onomatopéia: estalo (som); riso zombeteiro}
  \end{Phonetics}
  \begin{Phonetics}{格}{ge2}[][HSK 7-9]
    \definition*{s.}{Sobrenome: Ge}
    \definition{s.}{quadrados formados por linhas cruzadas; quadriculado; grade | divisão (horizontal ou não); treliça | padrão; forma; formato; estilo | caso; as categorias morfológicas de substantivos, pronomes e adjetivos em algumas línguas}
    \definition{v.}{resistir; dificultar; obstruir; impedir | estudar cuidadosamente; investigar | lutar; bater}
  \end{Phonetics}
\end{Entry}

\begin{Entry}{格兰菜}{10,5,11}{⽊、⼋、⾋}
  \begin{Phonetics}{格兰菜}{ge2lan2cai4}
    \definition{s.}{brócolis chinês | couve chinesa | mostarda}
  \seealsoref{芥蓝}{gai4lan2}
  \end{Phonetics}
\end{Entry}

\begin{Entry}{格外}{10,5}{⽊、⼣}
  \begin{Phonetics}{格外}{ge2wai4}[][HSK 4]
    \definition{adv.}{especialmente; particularmente; ainda mais; indica mais do que a média | adicionalmente; indica adicional ou extra}
  \end{Phonetics}
\end{Entry}

\begin{Entry}{格式}{10,6}{⽊、⼷}
  \begin{Phonetics}{格式}{ge2shi5}[][HSK 7-9]
    \definition[种]{s.}{forma; estilo; \emph{layout}; padrão; formato; modo}
  \end{Phonetics}
\end{Entry}

\begin{Entry}{格局}{10,7}{⽊、⼫}
  \begin{Phonetics}{格局}{ge2ju2}[][HSK 7-9]
    \definition{s.}{padrão; configuração; estrutura; estilo; maneira; arranjo | a visão ou percepção de uma situação geral; a visão de uma pessoa, a altura e a profundidade da consideração do problema}
  \end{Phonetics}
\end{Entry}

\begin{Entry}{格格不入}{10,10,4,2}{⽊、⽊、⼀、⼊}
  \begin{Phonetics}{格格不入}{ge2ge2-bu2ru4}[][HSK 7-9]
    \definition{expr.}{incompatível com; fora de sintonia com; estranho; fora do seu elemento; como uma estaca quadrada em um buraco redondo; desarmônico}
  \end{Phonetics}
\end{Entry}

%%%%%%%%%% 栽 %%%%%%%%%%
\subsection*{栽}

\begin{Entry}{栽}{10}{⽊}
  \begin{Phonetics}{栽}{zai1}
    \definition{v.}{cultivar | plantar}
  \end{Phonetics}
\end{Entry}

\begin{Entry}{栽种}{10,9}{⽊、⽲}
  \begin{Phonetics}{栽种}{zai1zhong4}
    \definition{v.}{plantar}
  \end{Phonetics}
\end{Entry}

\begin{Entry}{栽倒}{10,10}{⽊、⼈}
  \begin{Phonetics}{栽倒}{zai1dao3}
    \definition{v.}{cair | sofrer uma queda}
  \end{Phonetics}
\end{Entry}

\begin{Entry}{栽赃}{10,10}{⽊、⾙}
  \begin{Phonetics}{栽赃}{zai1zang1}
    \definition{v.}{enquadrar alguém (plantar provas nele)}
  \end{Phonetics}
\end{Entry}

\begin{Entry}{栽培}{10,11}{⽊、⼟}
  \begin{Phonetics}{栽培}{zai1pei2}
    \definition{v.}{cultivar | educar | patrocinar | treinar}
  \end{Phonetics}
\end{Entry}

\begin{Entry}{栽培种}{10,11,9}{⽊、⼟、⽲}
  \begin{Phonetics}{栽培种}{zai1pei2 zhong3}
    \definition{s.}{espécies cultivadas}
  \end{Phonetics}
\end{Entry}

\begin{Entry}{栽植}{10,12}{⽊、⽊}
  \begin{Phonetics}{栽植}{zai1zhi2}
    \definition{v.}{plantar | transplantar}
  \end{Phonetics}
\end{Entry}

%%%%%%%%%% 桂 %%%%%%%%%%
\subsection*{桂}

\begin{Entry}{桂}{10}{⽊}
  \begin{Phonetics}{桂}{gui4}
    \definition*{s.}{outro nome para o rio Guijiang 桂江 (em Guangxi 广西) | outro nome para Guangxi 广西 (Região Autônoma de Zhuang) | Sobrenome: Gui}
    \definition[棵]{s.}{louro; loureiro | osmanthus de aroma doce | árvore de casca de cássia | canela; osmanthus}
  \seealsoref{广西}{guang3xi1}
  \seealsoref{桂江}{gui4jiang1}
  \end{Phonetics}
\end{Entry}

\begin{Entry}{桂江}{10,6}{⽊、⽔}
  \begin{Phonetics}{桂江}{gui4jiang1}
    \definition*{s.}{Rio Guijiang}
  \end{Phonetics}
\end{Entry}

\begin{Entry}{桂花}{10,7}{⽊、⾋}
  \begin{Phonetics}{桂花}{gui4hua1}[][HSK 7-9]
    \definition{s.}{jasmim do imperador; um arbusto perene ou pequena árvore, cujas flores também são chamadas de osmanthus, são muito perfumadas e podem ser usadas para extrair óleos aromáticos ou fazer especiarias. Variedades comuns incluem Jingui 金桂 (flores amarelo-alaranjadas), Dangui 丹桂 (flores vermelho-alaranjadas), Yingui 银桂 (flores branco-amareladas) e Sijigui 四季桂 (flores branco-amareladas).}
  \end{Phonetics}
\end{Entry}

%%%%%%%%%% 桃 %%%%%%%%%%
\subsection*{桃}

\begin{Entry}{桃}{10}{⽊}
  \begin{Phonetics}{桃}{tao2}[][HSK 5]
    \definition*{s.}{Sobrenome: Tao}
    \definition[个,箱,袋,斤,棵,种]{s.}{pêssego | em forma de pêssego | pessegueiro}
  \end{Phonetics}
\end{Entry}

\begin{Entry}{桃花}{10,7}{⽊、⾋}
  \begin{Phonetics}{桃花}{tao2 hua1}[][HSK 5]
    \definition[朵,枝,株]{s.}{Figurativo: caso amoroso | flor de pessegueiro}
  \end{Phonetics}
\end{Entry}

\begin{Entry}{桃树}{10,9}{⽊、⽊}
  \begin{Phonetics}{桃树}{tao2 shu4}[][HSK 5]
    \definition[棵,株]{s.}{pêssego (árvore) | pessegueiro; pêssegos}
  \end{Phonetics}
\end{Entry}

%%%%%%%%%% 框 %%%%%%%%%%
\subsection*{框}

\begin{Entry}{框}{10}{⽊}
  \begin{Phonetics}{框}{kuang4}[][HSK 7-9]
    \definition{s.}{moldura; estojo | caixa; bloco}
    \definition{v.}{Obsoleto: desenhar uma moldura ao redor; adicionar linhas ao redor do texto e das imagens | Obsoleto: restringir; confinar; conter; amarrar; colocar em uma camisa de força}
  \end{Phonetics}
\end{Entry}

\begin{Entry}{框架}{10,9}{⽊、⽊}
  \begin{Phonetics}{框架}{kuang4jia4}[][HSK 7-9]
    \definition[个,副,种,套]{s.}{moldura; estrutura; na construção civil, as estruturas são formadas por conexões como vigas e colunas | estrutura; estrutura básica de um sistema, texto, etc.; metáfora para a organização e estrutura das coisas}
  \end{Phonetics}
\end{Entry}

%%%%%%%%%% 案 %%%%%%%%%%
\subsection*{案}

\begin{Entry}{案}{10}{⽊}
  \begin{Phonetics}{案}{an4}
    \definition{s.}{mesa; escrivaninha; mesa longa | caso; caso de direito (legal) | registro; arquivo; arquivo de caso | um plano submetido para consideração; proposta; um documento que propõe planos, sugestões, métodos, etc.}
  \end{Phonetics}
\end{Entry}

\begin{Entry}{案件}{10,6}{⽊、⼈}
  \begin{Phonetics}{案件}{an4jian4}[][HSK 7-9]
    \definition[个,起,件,类]{s.}{caso; caso de direito; caso legal; contencioso e eventos ilegais}
  \end{Phonetics}
\end{Entry}

%%%%%%%%%% 桌 %%%%%%%%%%
\subsection*{桌}

\begin{Entry}{桌}{10}{⽊}
  \begin{Phonetics}{桌}{zhuo1}
    \definition{clas.}{usado para mesas de convidados em um banquete etc.}
    \definition{s.}{mesa}
  \end{Phonetics}
\end{Entry}

\begin{Entry}{桌子}{10,3}{⽊、⼦}
  \begin{Phonetics}{桌子}{zhuo1zi5}[][HSK 1]
    \definition[张,套]{s.}{mesa; escrivaninha; móveis, com uma superfície plana na parte superior e uma estrutura de suporte na parte inferior, para colocar objetos ou realizar atividades}
  \end{Phonetics}
\end{Entry}

\begin{Entry}{桌布}{10,5}{⽊、⼱}
  \begin{Phonetics}{桌布}{zhuo1bu4}
    \definition[条,块,张]{s.}{(computação) plano de fundo da área de trabalho | toalha de mesa | papel de parede}
  \end{Phonetics}
\end{Entry}

\begin{Entry}{桌机}{10,6}{⽊、⽊}
  \begin{Phonetics}{桌机}{zhuo1ji1}
    \definition{s.}{computador \emph{desktop}}
  \end{Phonetics}
\end{Entry}

\begin{Entry}{桌灯}{10,6}{⽊、⽕}
  \begin{Phonetics}{桌灯}{zhuo1deng1}
    \definition{s.}{luminária | lâmpada de mesa}
  \end{Phonetics}
\end{Entry}

\begin{Entry}{桌面}{10,9}{⽊、⾯}
  \begin{Phonetics}{桌面}{zhuo1mian4}
    \definition{s.}{área de trabalho | mesa}
  \end{Phonetics}
\end{Entry}

\begin{Entry}{桌球}{10,11}{⽊、⽟}
  \begin{Phonetics}{桌球}{zhuo1qiu2}
    \definition{s.}{bilhar | sinuca | mesa de ping-pong}
  \end{Phonetics}
\end{Entry}

\begin{Entry}{桌游}{10,12}{⽊、⽔}
  \begin{Phonetics}{桌游}{zhuo1you2}
    \definition{s.}{jogo de tabuleiro}
  \end{Phonetics}
\end{Entry}

%%%%%%%%%% 桑 %%%%%%%%%%
\subsection*{桑}

\begin{Entry}{桑}{10}{⽊}
  \begin{Phonetics}{桑}{sang1}
    \definition*{s.}{Sobrenome: Sang}
    \definition[棵]{s.}{amoreira}
  \end{Phonetics}
\end{Entry}

\begin{Entry}{桑巴舞}{10,4,14}{⽊、⼰、⾇}
  \begin{Phonetics}{桑巴舞}{sang1ba1wu3}
    \definition{s.}{samba}
  \end{Phonetics}
\end{Entry}

\begin{Entry}{桑树}{10,9}{⽊、⽊}
  \begin{Phonetics}{桑树}{sang1shu4}
    \definition{s.}{amoreira, suas folhas são utilizadas para alimentar bichos-da-seda}
  \end{Phonetics}
\end{Entry}

%%%%%%%%%% 档 %%%%%%%%%%
\subsection*{档}

\begin{Entry}{档}{10}{⽊}
  \begin{Phonetics}{档}{dang4}[][HSK 6]
    \definition{clas.}{festa; usado para eventos, shows}
    \definition{s.}{prateleiras (para arquivos); compartimentos para documentos | arquivos; arquivos | travessa (de uma mesa, etc.) | qualidade; nota}
  \end{Phonetics}
\end{Entry}

\begin{Entry}{档次}{10,6}{⽊、⽋}
  \begin{Phonetics}{档次}{dang4ci4}[][HSK 7-9]
    \definition{s.}{classe; grau; qualidade; nível; diferentes níveis divididos de acordo com certos padrões}
  \end{Phonetics}
\end{Entry}

\begin{Entry}{档案}{10,10}{⽊、⽊}
  \begin{Phonetics}{档案}{dang4'an4}[][HSK 6]
    \definition[份,个]{s.}{arquivos; registro; dossiê; arquivos e materiais armazenados de forma classificada para referência futura}
  \end{Phonetics}
\end{Entry}

%%%%%%%%%% 桥 %%%%%%%%%%
\subsection*{桥}

\begin{Entry}{桥}{10}{⽊}
  \begin{Phonetics}{桥}{qiao2}[][HSK 3]
    \definition*{s.}{Sobrenome: Qiao}
    \definition[座]{s.}{ponte; construção que atravessa a água conectando as duas margens}
  \end{Phonetics}
\end{Entry}

\begin{Entry}{桥梁}{10,11}{⽊、⽊}
  \begin{Phonetics}{桥梁}{qiao2liang2}[][HSK 6]
    \definition[座]{s.}{ponte; acesso; uma obra construída na superfície do rio, conectando as duas margens | ponte; metáfora para pessoas ou coisas que podem se comunicar}
  \end{Phonetics}
\end{Entry}

%%%%%%%%%% 桩 %%%%%%%%%%
\subsection*{桩}

\begin{Entry}{桩}{10}{⽊}
  \begin{Phonetics}{桩}{zhuang1}
    \definition{clas.}{para eventos, casos, transações, assuntos, etc.}
    \definition{s.}{toco | estaca | pilha}
  \end{Phonetics}
\end{Entry}

%%%%%%%%%% 欱 %%%%%%%%%%
\subsection*{欱}

\begin{Entry}{欱}{10}{⽋}
  \begin{Phonetics}{欱}{he1}
    \definition{v.}{beber | beber bebida alcoólica}
    \variantof{喝}
  \end{Phonetics}
\end{Entry}

%%%%%%%%%% 氧 %%%%%%%%%%
\subsection*{氧}

\begin{Entry}{氧}{10}{⽓}
  \begin{Phonetics}{氧}{yang3}
    \definition{s.}{oxigênio}
  \end{Phonetics}
\end{Entry}

\begin{Entry}{氧气}{10,4}{⽓、⽓}
  \begin{Phonetics}{氧气}{yang3qi4}[][HSK 6]
    \definition{s.}{oxigênio (O); gás oxigênio}
  \end{Phonetics}
\end{Entry}

%%%%%%%%%% 流 %%%%%%%%%%
\subsection*{流}

\begin{Entry}{流}{10}{⽔}
  \begin{Phonetics}{流}{liu2}[][HSK 2]
    \definition*{s.}{Sobrenome: Liu}
    \definition{adj.}{fluente; tão suave quanto a água corrente}
    \definition{clas.}{lúmen; abreviação de lumens, 流明}
    \definition[名,个]{s.}{corrente de água | corrente; algo que se assemelha a um fluxo de água | razão; taxa; classe; grau; ramificação; facção; hierarquia}
    \definition{v.}{(de líquido) fluir | vaguear; vagar; mover-se de um lugar para outro; movimentar-se sem direção fixa | espalhar; circular; transmitir; divulgar | degenerar; mudar para pior; tender (aspecto negativo) | banir; enviar para o exílio | correr (ou fluir) como líquido; refere-se à parte do rio após deixar sua nascente (em contraste com a 源)}
  \seealsoref{流明}{liu2ming2}
  \seealsoref{源}{yuan2}
  \end{Phonetics}
\end{Entry}

\begin{Entry}{流入}{10,2}{⽔、⼊}
  \begin{Phonetics}{流入}{liu2ru4}[][HSK 7-9]
    \definition{s.}{fluxo de entrada; afluxo}
    \definition{v.}{fluir (ou correr) em | cair em | deixar-se levar para dentro | fluir para dentro}
  \end{Phonetics}
\end{Entry}

\begin{Entry}{流水}{10,4}{⽔、⽔}
  \begin{Phonetics}{流水}{liu2shui3}[][HSK 7-9]
    \definition{s.}{água corrente (uma metáfora para coisas contínuas) | faturamento (nos negócios); isso se refere à receita de vendas de uma loja}
  \end{Phonetics}
\end{Entry}

\begin{Entry}{流失}{10,5}{⽔、⼤}
  \begin{Phonetics}{流失}{liu2shi1}[][HSK 7-9]
    \definition{v.}{escorrer; ser levado pela água; rochas, água e solo na natureza desaparecem por si mesmos ou são levados pela água e pelo vento | perder; esgotar; fluir para longe; perder algo útil e valioso | ir embora; perder funcionários; essa metáfora descreve pessoas que deixam sua região ou local de trabalho}
  \end{Phonetics}
\end{Entry}

\begin{Entry}{流传}{10,6}{⽔、⼈}
  \begin{Phonetics}{流传}{liu2chuan2}[][HSK 4]
    \definition[间]{v.}{espalhar; circular; passar adiante}
  \end{Phonetics}
\end{Entry}

\begin{Entry}{流动}{10,6}{⽔、⼒}
  \begin{Phonetics}{流动}{liu2 dong4}[][HSK 5]
    \definition{v.}{(água, ar, etc.) fluir; correr; circular | ir de um lugar para outro; estar em movimento; ser móvel (oposto a 固定)}
  \seealsoref{固定}{gu4ding4}
  \end{Phonetics}
\end{Entry}

\begin{Entry}{流向}{10,6}{⽔、⼝}
  \begin{Phonetics}{流向}{liu2xiang4}[][HSK 7-9]
    \definition{s.}{direção de uma corrente | direção do fluxo (de pessoal, mercadorias, etc.) | direção}
    \definition{v.}{fluir (ou correr, mover-se) em direção a}
  \end{Phonetics}
\end{Entry}

\begin{Entry}{流血}{10,6}{⽔、⾎}
  \begin{Phonetics}{流血}{liu2xie3}
    \definition{s.}{sangrar; estar sangrando; derramar sangue}
  \end{Phonetics}
  \begin{Phonetics}{流血}{liu2xue4}[][HSK 7-9]
    \definition{v.}{sangrar; derramar sangue; metaforicamente se refere a sacrifício ou ferimento}
  \end{Phonetics}
\end{Entry}

\begin{Entry}{流行}{10,6}{⽔、⾏}
  \begin{Phonetics}{流行}{liu2xing2}[][HSK 2]
    \definition{adj.}{popular; na moda; muito popular}
    \definition{v.}{ser popular; prevalecer; espalhar-se amplamente; divulgar amplamente}
  \end{Phonetics}
\end{Entry}

\begin{Entry}{流行性感冒}{10,6,8,13,9}{⽔、⾏、⼼、⼼、⽇}
  \begin{Phonetics}{流行性感冒}{liu2xing2 xing4 gan3mao4}
    \definition{s.}{gripe muito forte; influenza}
  \end{Phonetics}
\end{Entry}

\begin{Entry}{流利}{10,7}{⽔、⼑}
  \begin{Phonetics}{流利}{liu2li4}[][HSK 2]
    \definition{adj.}{fluente; suave; lúcido; falar e escrever com fluência e clareza | com fluência; sem dificuldades}
  \end{Phonetics}
\end{Entry}

\begin{Entry}{流明}{10,8}{⽔、⽇}
  \begin{Phonetics}{流明}{liu2ming2}
    \definition{s.}{(empréstimo linguístico) lúmen (unidade de fluxo luminoso)}
  \end{Phonetics}
\end{Entry}

\begin{Entry}{流氓}{10,8}{⽔、⽒}
  \begin{Phonetics}{流氓}{liu2mang2}[][HSK 7-9]
    \definition[群,帮,伙,个]{s.}{malandro; delinquente; arruaceiro; originalmente, o termo se referia a vagabundos desempregados; posteriormente, passou a se referir a pessoas que não se dedicam a um trabalho honesto e frequentemente cometem atos ilícitos | vandalismo; indecência; comportamento imoral; isso se refere a atos hediondos, como o assédio a mulheres}
  \end{Phonetics}
\end{Entry}

\begin{Entry}{流泪}{10,8}{⽔、⽔}
  \begin{Phonetics}{流泪}{liu2lei4}[][HSK 7-9]
    \definition{v.}{derramar lágrimas}
  \end{Phonetics}
\end{Entry}

\begin{Entry}{流畅}{10,8}{⽔、⽥}
  \begin{Phonetics}{流畅}{liu2chang4}[][HSK 7-9]
    \definition{adj.}{(escrita, fala, etc.) fluente; fácil e sem problema; suave}
  \end{Phonetics}
\end{Entry}

\begin{Entry}{流转}{10,8}{⽔、⾞}
  \begin{Phonetics}{流转}{liu2zhuan3}[][HSK 7-9]
    \definition{adj.}{(escritos, pessoas, etc.) suave; fluido; fluente; refere-se à poesia e à prosa que são fluentes e bem elaboradas}
    \definition{v.}{vagar; perambular; estar em movimento; flutuando e mudando de lugar; não fixo em um só lugar | (bens ou capital) circular; a movimentação de bens ou fundos durante o processo de circulação}
  \end{Phonetics}
\end{Entry}

\begin{Entry}{流星}{10,9}{⽔、⽇}
  \begin{Phonetics}{流星}{liu2xing1}
    \definition{s.}{meteoro | estrela cadente}
  \end{Phonetics}
\end{Entry}

\begin{Entry}{流浪}{10,10}{⽔、⽔}
  \begin{Phonetics}{流浪}{liu2lang4}[][HSK 7-9]
    \definition{v.}{vagar sem rumo; levar uma vida errante}
  \end{Phonetics}
\end{Entry}

\begin{Entry}{流通}{10,10}{⽔、⾡}
  \begin{Phonetics}{流通}{liu2tong1}[][HSK 5]
    \definition{v.}{(ar, dinheiro, mercadorias, etc.) fluir; circular}
  \end{Phonetics}
\end{Entry}

\begin{Entry}{流域}{10,11}{⽔、⼟}
  \begin{Phonetics}{流域}{liu2yu4}[][HSK 7-9]
    \definition[个]{s.}{vale fluvial (ou bacia hidrográfica); área de drenagem | bacia hidrográfica; bacia; área de captação; bacia de drenagem; área de alimentação; área de captação de água; vale; bacia fluvial}
  \end{Phonetics}
\end{Entry}

\begin{Entry}{流淌}{10,11}{⽔、⽔}
  \begin{Phonetics}{流淌}{liu2tang3}[][HSK 7-9]
    \definition{v.}{fluir; gotejar}
  \end{Phonetics}
\end{Entry}

\begin{Entry}{流程}{10,12}{⽔、⽲}
  \begin{Phonetics}{流程}{liu2cheng2}[][HSK 7-9]
    \definition{s.}{um caminho de fluxo; uma distância percorrida pela água; a distância do fluxo de água | fluxo de trabalho; um processo tecnológico; abreviação de fluxograma de processo, 工艺流程 | circuito; procedimentos para cada processo}
  \seealsoref{工艺流程}{gong1yi4 liu2cheng2}
  \end{Phonetics}
\end{Entry}

\begin{Entry}{流量}{10,12}{⽔、⾥}
  \begin{Phonetics}{流量}{liu2liang4}[][HSK 7-9]
    \definition[个]{s.}{(taxa de) fluxo; descarga; a quantidade de fluido que passa por uma seção transversal por unidade de tempo. geralmente é calculada em metros cúbicos por segundo | Internet: tráfego do site; tráfego de rede refere-se à quantidade de dados transmitidos por uma rede por unidade de tempo | fluxo/volume de tráfego; o número de pedestres, veículos, etc., que passam por um determinado local por unidade de tempo}
  \end{Phonetics}
\end{Entry}

\begin{Entry}{流感}{10,13}{⽔、⼼}
  \begin{Phonetics}{流感}{liu2 gan3}[][HSK 6]
    \definition{s.}{gripe; influenza; abreviação de 流行性感冒}
  \seealsoref{流行性感冒}{liu2xing2 xing4 gan3mao4}
  \end{Phonetics}
\end{Entry}

\begin{Entry}{流露}{10,21}{⽔、⾬}
  \begin{Phonetics}{流露}{liu2lu4}[][HSK 7-9]
    \definition{v.}{revelar; trair; mostrar involuntariamente (os próprios pensamentos ou sentimentos); (significado, emoção) revelar inconscientemente}
  \end{Phonetics}
\end{Entry}

%%%%%%%%%% 浙 %%%%%%%%%%
\subsection*{浙}

\begin{Entry}{浙}{10}{⽔}
  \begin{Phonetics}{浙}{zhe4}
    \definition{s.}{abreviação de província de Zhejiang,  浙江, no leste da China}
  \seealsoref{浙江}{zhe4jiang1}
  \end{Phonetics}
\end{Entry}

\begin{Entry}{浙江}{10,6}{⽔、⽔}
  \begin{Phonetics}{浙江}{zhe4jiang1}
    \definition*{s.}{Província de Zhejiang}
  \end{Phonetics}
\end{Entry}

%%%%%%%%%% 浩 %%%%%%%%%%
\subsection*{浩}

\begin{Entry}{浩}{10}{⽔}
  \begin{Phonetics}{浩}{hao4}
    \definition*{s.}{Sobrenome: Hao}
    \definition{adj.}{grande; vasto; grandioso; sem limites | um grande número; infinito}
  \end{Phonetics}
\end{Entry}

\begin{Entry}{浩劫}{10,7}{⽔、⼒}
  \begin{Phonetics}{浩劫}{hao4jie2}[][HSK 7-9]
    \definition[场,次]{s.}{grande calamidade; catástrofe | devastação; holocausto; flagelo}
  \end{Phonetics}
\end{Entry}

%%%%%%%%%% 浪 %%%%%%%%%%
\subsection*{浪}

\begin{Entry}{浪}{10}{⽔}
  \begin{Phonetics}{浪}{lang4}[][HSK 7-9]
    \definition*{s.}{Sobrenome: Lang}
    \definition{adj.}{desenfreado; perdulário}
    \definition{adv.}{livremente}
    \definition[朵,阵,波]{s.}{onda; vagalhão; rebentação | algo ondulatório | coisas ondulando como ondas}
    \definition{v.}{passear; divagar}
  \end{Phonetics}
\end{Entry}

\begin{Entry}{浪花}{10,7}{⽔、⾋}
  \begin{Phonetics}{浪花}{lang4hua1}
    \definition[朵]{s.}{\emph{spray} | \emph{spray} do oceano | (figurativo) acontecimentos de sua vida}
  \end{Phonetics}
\end{Entry}

\begin{Entry}{浪费}{10,9}{⽔、⾙}
  \begin{Phonetics}{浪费}{lang4fei4}[][HSK 3]
    \definition{adj.}{desperdiçado; extravagante; não econômico}
    \definition{v.}{desperdiçar; dissipar; esbanjar; ser extravagante; uso excessivo ou inadequado de bens, recursos humanos, tempo, etc.}
  \end{Phonetics}
\end{Entry}

\begin{Entry}{浪漫}{10,14}{⽔、⽔}
  \begin{Phonetics}{浪漫}{lang4man4}[][HSK 5]
    \definition{adj.}{romântico; poético | não convencional; boêmio; abandonado; libertino; devasso; comportar-se de maneira descuidada e descuidada (geralmente se referindo a relacionamentos entre pessoas) | irrealista; impraticável}
  \end{Phonetics}
\end{Entry}

%%%%%%%%%% 浮 %%%%%%%%%%
\subsection*{浮}

\begin{Entry}{浮}{10}{⽔}
  \begin{Phonetics}{浮}{fu2}[][HSK 6]
    \definition*{s.}{Sobrenome: Fu}
    \definition{adj.}{superficial; na superfície | móvel; removível | temporário; provisório | superficial e frívolo; volátil; impetuoso | oco; vazio; inflado | excessivo; excedente}
    \definition{v.}{flutuar (oposto a 沉) | (dialeto) nadar | flutuar; derivar; flutuar na superfície do líquido}
  \seealsoref{沉}{chen2}
  \end{Phonetics}
\end{Entry}

\begin{Entry}{浮力}{10,2}{⽔、⼒}
  \begin{Phonetics}{浮力}{fu2li4}[][HSK 7-9]
    \definition{s.}{flutuabilidade; a força de empuxo, a força ascendente exercida sobre um objeto em um fluido, é igual ao peso do fluido deslocado pelo objeto}
  \end{Phonetics}
\end{Entry}

\begin{Entry}{浮图}{10,8}{⽔、⼞}
  \begin{Phonetics}{浮图}{fu2tu2}
    \definition*{s.}{Termo alternativo para 佛陀}
    \variantof{浮屠}
  \seealsoref{佛陀}{fo2tuo2}
  \end{Phonetics}
\end{Entry}

\begin{Entry}{浮现}{10,8}{⽔、⾒}
  \begin{Phonetics}{浮现}{fu2xian4}[][HSK 7-9]
    \definition{v.}{(experiência passada) ressurgir; vir à mente | aparecer; apresentar; revelar}
  \end{Phonetics}
\end{Entry}

\begin{Entry}{浮屠}{10,11}{⽔、⼫}
  \begin{Phonetics}{浮屠}{fu2tu2}
    \definition*{s.}{Buda | Templo (Stupa) Budista (transliteração de Pali Thuo)}
  \end{Phonetics}
\end{Entry}

\begin{Entry}{浮躁}{10,20}{⽔、⾜}
  \begin{Phonetics}{浮躁}{fu2zao4}[][HSK 7-9]
    \definition{adj.}{inquieto; impetuoso; impaciente; frívolo e impaciente}
  \end{Phonetics}
\end{Entry}

%%%%%%%%%% 海 %%%%%%%%%%
\subsection*{海}

\begin{Entry}{海}{10}{⽔}
  \begin{Phonetics}{海}{hai3}[][HSK 2]
    \definition*{s.}{Sobrenome: Hai}
    \definition{adj.}{extragrande; de grande capacidade; descreve capacidade, tom de voz, etc.}
    \definition{adv.}{aleatoriamente; sem rumo; sem limites; sem restrições}
    \definition[片]{s.}{mar; grande lago; a parte do oceano próxima à costa, alguns grandes lagos também são chamados de mar | grande número de pessoas ou coisas reunidas; metáfora para muitas coisas semelhantes que formam um grande conjunto}
  \end{Phonetics}
\end{Entry}

\begin{Entry}{海内外}{10,4,5}{⽔、⼌、⼣}
  \begin{Phonetics}{海内外}{hai3 nei4wai4}[][HSK 7-9]
    \definition{s.}{em casa e no exterior | nacional e internacional}
  \end{Phonetics}
\end{Entry}

\begin{Entry}{海水}{10,4}{⽔、⽔}
  \begin{Phonetics}{海水}{hai3 shui3}[][HSK 4]
    \definition[把]{s.}{água do mar; salmoura}
  \end{Phonetics}
\end{Entry}

\begin{Entry}{海风}{10,4}{⽔、⾵}
  \begin{Phonetics}{海风}{hai3feng1}
    \definition{s.}{brisa do mar | vento que vem do mar}
  \end{Phonetics}
\end{Entry}

\begin{Entry}{海外}{10,5}{⽔、⼣}
  \begin{Phonetics}{海外}{hai3 wai4}[][HSK 6]
    \definition[次]{s.}{fora das fronteiras nacionais; no exterior}
  \end{Phonetics}
\end{Entry}

\begin{Entry}{海边}{10,5}{⽔、⾡}
  \begin{Phonetics}{海边}{hai3 bian1}[][HSK 2]
    \definition{s.}{praia; costa; litoral; orla marítima; a parte marginal do oceano e as grandes áreas de água salgada cercadas por terra firme, onde a terra e a água se encontram, formam a costa}
  \end{Phonetics}
\end{Entry}

\begin{Entry}{海关}{10,6}{⽔、⼋}
  \begin{Phonetics}{海关}{hai3guan1}[][HSK 3]
    \definition[个]{s.}{alfândega; órgão administrativo nacional, sua principal função é supervisionar e inspecionar os bens e meios de transporte que entram e saem do país, cobrar impostos alfandegários e reprimir o contrabando}
  \end{Phonetics}
\end{Entry}

\begin{Entry}{海军}{10,6}{⽔、⼍}
  \begin{Phonetics}{海军}{hai3 jun1}[][HSK 6]
    \definition[支,名,位,个]{s.}{marinha; o exército que luta no mar geralmente é composto por navios de superfície, submarinos, aviação naval, fuzileiros navais e outros ramos, além de diversas forças profissionais}
  \end{Phonetics}
\end{Entry}

\begin{Entry}{海报}{10,7}{⽔、⼿}
  \begin{Phonetics}{海报}{hai3 bao4}[][HSK 6]
    \definition[张,份,幅]{s.}{pôster; cartaz; cartazes anunciando apresentações culturais, exibições de filmes ou competições esportivas, etc.}
  \end{Phonetics}
\end{Entry}

\begin{Entry}{海运}{10,7}{⽔、⾡}
  \begin{Phonetics}{海运}{hai3yun4}[][HSK 7-9]
    \definition{s.}{transporte marítimo; transporte oceânico}
    \definition{v.}{transportar pelo mar}
  \end{Phonetics}
\end{Entry}

\begin{Entry}{海里}{10,7}{⽔、⾥}
  \begin{Phonetics}{海里}{hai3li3}
    \definition{s.}{milha náutica}
  \end{Phonetics}
\end{Entry}

\begin{Entry}{海岸}{10,8}{⽔、⼭}
  \begin{Phonetics}{海岸}{hai3'an4}[][HSK 7-9]
    \definition[段]{s.}{litoral; costa; praia}
  \end{Phonetics}
\end{Entry}

\begin{Entry}{海底}{10,8}{⽔、⼴}
  \begin{Phonetics}{海底}{hai3 di3}[][HSK 6]
    \definition{s.}{fundo do mar; fundo do oceano; solo oceânico}
  \end{Phonetics}
\end{Entry}

\begin{Entry}{海拔}{10,8}{⽔、⼿}
  \begin{Phonetics}{海拔}{hai3ba2}[][HSK 7-9]
    \definition{s.}{altitude; altura em relação ao nível médio do mar}
  \end{Phonetics}
\end{Entry}

\begin{Entry}{海峡}{10,9}{⽔、⼭}
  \begin{Phonetics}{海峡}{hai3xia2}[][HSK 7-9]
    \definition*{s.}{Estreito de Taiwan}
    \definition[个]{s.}{estreito; canal; um canal estreito que conecta dois oceanos entre duas massas de terra}
  \end{Phonetics}
\end{Entry}

\begin{Entry}{海洋}{10,9}{⽔、⽔}
  \begin{Phonetics}{海洋}{hai3yang2}[][HSK 6]
    \definition[片,个]{s.}{mar; oceano; um termo geral para os mares e oceanos que formam uma entidade contínua na superfície da Terra; também pode ser usado para descrever um grande número de coisas semelhantes}
  \end{Phonetics}
\end{Entry}

\begin{Entry}{海面}{10,9}{⽔、⾯}
  \begin{Phonetics}{海面}{hai3mian4}[][HSK 7-9]
    \definition{s.}{nível do mar | superfície do mar}
  \end{Phonetics}
\end{Entry}

\begin{Entry}{海鸥}{10,9}{⽔、⿃}
  \begin{Phonetics}{海鸥}{hai3'ou1}
    \definition{s.}{gaivota}
  \end{Phonetics}
\end{Entry}

\begin{Entry}{海浪}{10,10}{⽔、⽔}
  \begin{Phonetics}{海浪}{hai3 lang4}[][HSK 6]
    \definition{s.}{ondas do mar}
  \end{Phonetics}
\end{Entry}

\begin{Entry}{海啸}{10,11}{⽔、⼝}
  \begin{Phonetics}{海啸}{hai3xiao4}[][HSK 7-9]
    \definition{s.}{\emph{tsunami}; maremoto}
  \end{Phonetics}
\end{Entry}

\begin{Entry}{海域}{10,11}{⽔、⼟}
  \begin{Phonetics}{海域}{hai3yu4}[][HSK 7-9]
    \definition{s.}{área marítima; espaço marítimo; refere-se a uma determinada área do oceano (tanto acima quanto abaixo da água)}
  \end{Phonetics}
\end{Entry}

\begin{Entry}{海盗}{10,11}{⽔、⽫}
  \begin{Phonetics}{海盗}{hai3dao4}[][HSK 7-9]
    \definition{s.}{pirata; viajante do mar | bucaneiro; viking; pirata}
  \end{Phonetics}
\end{Entry}

\begin{Entry}{海绵}{10,11}{⽔、⽷}
  \begin{Phonetics}{海绵}{hai3mian2}[][HSK 7-9]
    \definition{s.}{espuma de borracha; espuma de plástico; esponja; um material poroso feito de borracha ou plástico que é elástico, como uma esponja | esponja marinha seca; poríferos; osso esponjoso; refere-se especificamente ao esqueleto queratinoso das esponjas}
  \end{Phonetics}
\end{Entry}

\begin{Entry}{海棠}{10,12}{⽔、⽊}
  \begin{Phonetics}{海棠}{hai3tang2}
    \definition{s.}{begônia}
  \end{Phonetics}
\end{Entry}

\begin{Entry}{海湾}{10,12}{⽔、⽔}
  \begin{Phonetics}{海湾}{hai3 wan1}[][HSK 6]
    \definition{s.}{baía; golfo | lago}
  \end{Phonetics}
\end{Entry}

\begin{Entry}{海量}{10,12}{⽔、⾥}
  \begin{Phonetics}{海量}{hai3liang4}[][HSK 7-9]
    \definition{s.}{magnanimidade | alta tolerância ao álcool | cargas de; uma grande quantidade de}
  \end{Phonetics}
\end{Entry}

\begin{Entry}{海滨}{10,13}{⽔、⽔}
  \begin{Phonetics}{海滨}{hai3bin1}[][HSK 7-9]
    \definition{s.}{praia; beira-mar; litoral; um lugar perto do mar}
  \end{Phonetics}
\end{Entry}

\begin{Entry}{海滩}{10,13}{⽔、⽔}
  \begin{Phonetics}{海滩}{hai3tan1}[][HSK 7-9]
    \definition[个,片]{s.}{praia; praia com declive suave em direção ao mar}
  \end{Phonetics}
\end{Entry}

\begin{Entry}{海鲜}{10,14}{⽔、⿂}
  \begin{Phonetics}{海鲜}{hai3xian1}[][HSK 4]
    \definition[种,份,桌,批,些]{s.}{frutos do mar; mariscos; peixes marinhos frescos, camarões, etc., para consumo |}
  \end{Phonetics}
\end{Entry}

\begin{Entry}{海藻}{10,19}{⽔、⾋}
  \begin{Phonetics}{海藻}{hai3zao3}[][HSK 7-9]
    \definition{s.}{alga marinha; planta marítima}
  \end{Phonetics}
\end{Entry}

%%%%%%%%%% 浸 %%%%%%%%%%
\subsection*{浸}

\begin{Entry}{浸}{10}{⽔}
  \begin{Phonetics}{浸}{jin4}
    \definition{adv.}{gradualmente; passo a passo; pouco a pouco | Literário: gradualmente; cada vez mais}
    \definition{v.}{deixar de molho; imergir; mergulhar | saturar}
  \end{Phonetics}
\end{Entry}

\begin{Entry}{浸泡}{10,8}{⽔、⽔}
  \begin{Phonetics}{浸泡}{jin4pao4}[][HSK 7-9]
    \definition{v.}{mergulhar; banhar; imergir; deixar de molho em líquido}
  \end{Phonetics}
\end{Entry}

%%%%%%%%%% 消 %%%%%%%%%%
\subsection*{消}

\begin{Entry}{消}{10}{⽔}
  \begin{Phonetics}{消}{xiao1}
    \definition{v.}{desaparecer | dissipar; remover; eliminar; fazer desaparecer | passar o tempo de forma descontraída (recreação) | precisar; tomar (necessidade, geralmente precedido por 不, 几, 何)}
  \seealsoref{不}{bu4}
  \seealsoref{何}{he2}
  \seealsoref{几}{ji3}
  \end{Phonetics}
\end{Entry}

\begin{Entry}{消化}{10,4}{⽔、⼔}
  \begin{Phonetics}{消化}{xiao1hua4}[][HSK 4]
    \definition{v.}{digerir (alimentos) | digerir (conhecimento); pensar e absorver; uma metáfora para a compreensão total de novos conhecimentos ou informações e a capacidade de transformá-los em algo que possa ser usado}
  \end{Phonetics}
\end{Entry}

\begin{Entry}{消失}{10,5}{⽔、⼤}
  \begin{Phonetics}{消失}{xiao1shi1}[][HSK 3]
    \definition{v.}{desaparecer; desvanecer; dissolver; dissipar; evaporar; sumir}
  \end{Phonetics}
\end{Entry}

\begin{Entry}{消灭}{10,5}{⽔、⽕}
  \begin{Phonetics}{消灭}{xiao1mie4}[][HSK 6]
    \definition{v.}{perecer; morrer; falecer; desaparecer | abolir; erradicar; eliminar; aniquilar; exterminar; acabar com; fazer com que não exista}
  \end{Phonetics}
\end{Entry}

\begin{Entry}{消防}{10,6}{⽔、⾩}
  \begin{Phonetics}{消防}{xiao1fang2}[][HSK 5]
    \definition{s.}{combate a incêncios; controle de incêndios}
  \end{Phonetics}
\end{Entry}

\begin{Entry}{消防员}{10,6,7}{⽔、⾩、⼝}
  \begin{Phonetics}{消防员}{xiao1fang2yuan2}
    \definition{s.}{bombeiro}
  \end{Phonetics}
\end{Entry}

\begin{Entry}{消极}{10,7}{⽔、⽊}
  \begin{Phonetics}{消极}{xiao1ji2}[][HSK 5]
    \definition{adj.}{negativo; oposto; adverso | passivo; inativo; sem ambição; sem iniciativa; desanimado; apático}
  \end{Phonetics}
\end{Entry}

\begin{Entry}{消毒}{10,9}{⽔、⽏}
  \begin{Phonetics}{消毒}{xiao1du2}[][HSK 5]
    \definition{v.}{desinfetar; esterilizar; matar os microrganismos causadores de doenças por meios físicos ou químicos}
  \end{Phonetics}
\end{Entry}

\begin{Entry}{消费}{10,9}{⽔、⾙}
  \begin{Phonetics}{消费}{xiao1fei4}[][HSK 3]
    \definition{v.}{gastar; consumir; consumir materiais para satisfazer as necessidades de produção ou de vida (geralmente refere-se ao consumo doméstico) | consumir (recursos naturais)}
  \end{Phonetics}
\end{Entry}

\begin{Entry}{消费者}{10,9,8}{⽔、⾙、⽼}
  \begin{Phonetics}{消费者}{xiao1 fei4 zhe3}[][HSK 5]
    \definition{s.}{consumidor; cliente; consumo; indivíduos membros da sociedade que compram e utilizam bens e serviços para consumo pessoal}
  \end{Phonetics}
\end{Entry}

\begin{Entry}{消除}{10,9}{⽔、⾩}
  \begin{Phonetics}{消除}{xiao1chu2}[][HSK 5]
    \definition{v.}{dissipar; eliminar; limpar; tornar algo inexistente; remover (algo desfavorável)}
  \end{Phonetics}
\end{Entry}

\begin{Entry}{消息}{10,10}{⽔、⼼}
  \begin{Phonetics}{消息}{xiao1xi5}[][HSK 3]
    \definition[个,条,篇,些]{s.}{notícias; informação; reportagem sobre pessoas ou situações | notícias; novidades;}
  \end{Phonetics}
\end{Entry}

\begin{Entry}{消耗}{10,10}{⽔、⽾}
  \begin{Phonetics}{消耗}{xiao1hao4}[][HSK 6]
    \definition{v.}{gastar; esgotar; consumir; usar; (espírito, força, coisas, etc.) diminuir gradualmente devido ao uso ou perda}
  \end{Phonetics}
\end{Entry}

%%%%%%%%%% 涉 %%%%%%%%%%
\subsection*{涉}

\begin{Entry}{涉}{10}{⽔}
  \begin{Phonetics}{涉}{she4}[][HSK 6]
    \definition*{s.}{Sobrenome: She}
    \definition{v.}{vadear; atravessar ou passar um rio ou um obstáculo | passar por; experimentar | envolver; implicar}
  \end{Phonetics}
\end{Entry}

\begin{Entry}{涉及}{10,3}{⽔、⼃}
  \begin{Phonetics}{涉及}{she4ji2}[][HSK 6]
    \definition{v.}{envolver; relacionar-se com; referir-se a; tocar em}
  \end{Phonetics}
\end{Entry}

%%%%%%%%%% 涝 %%%%%%%%%%
\subsection*{涝}

\begin{Entry}{涝}{10}{⽔}
  \begin{Phonetics}{涝}{lao4}[][HSK 7-9]
    \definition{adj.}{inundado; alagado}
    \definition{s.}{alagamento (de terras ou plantações); água acumulada nos campos devido às fortes chuvas}
    \definition{v.}{inundar; alagar}
  \end{Phonetics}
\end{Entry}

%%%%%%%%%% 涨 %%%%%%%%%%
\subsection*{涨}

\begin{Entry}{涨}{10}{⽔}
  \begin{Phonetics}{涨}{zhang3}[][HSK 5,6]
    \definition{v.}{subir; inchar; aumentar; elevar; melhorar}
  \end{Phonetics}
  \begin{Phonetics}{涨}{zhang4}
    \definition{v.}{inchar; ter o volume aumentado | ser inundado por uma torrente de sangue; ter uma dor de cabeça; ficar com o rosto vermelho de raiva | ser mais, maior, etc. do que o esperado}
  \end{Phonetics}
\end{Entry}

\begin{Entry}{涨价}{10,6}{⽔、⼈}
  \begin{Phonetics}{涨价}{zhang3/jia4}[][HSK 5]
    \definition{s.}{aumento de preços}
    \definition{v.+compl.}{(preços) subir; aumentar o preço}
  \end{Phonetics}
\end{Entry}

%%%%%%%%%% 烈 %%%%%%%%%%
\subsection*{烈}

\begin{Entry}{烈}{10}{⽕}
  \begin{Phonetics}{烈}{lie4}
    \definition*{s.}{Sobrenome: Lie}
    \definition{adj.}{forte; violento; intenso; feroz | justo; severo | firme; convicto; rigoroso}
    \definition{s.}{pessoa que morreu por uma causa justa | conquistas; façanhas | mártir sacrificando-se por uma causa justa}
  \end{Phonetics}
\end{Entry}

\begin{Entry}{烈士}{10,3}{⽕、⼠}
  \begin{Phonetics}{烈士}{lie4shi4}[][HSK 7-9]
    \definition[位,名]{s.}{mártir; pessoas que se sacrificaram pela causa da justiça | uma pessoa de grande empenho; na antiguidade, referia-se a uma pessoa que aspirava a alcançar grandes feitos}
  \end{Phonetics}
\end{Entry}

%%%%%%%%%% 烘 %%%%%%%%%%
\subsection*{烘}

\begin{Entry}{烘}{10}{⽕}
  \begin{Phonetics}{烘}{hong1}
    \definition{v.}{secar; assar; aquecer; usar fogo ou vapor para aquecer o corpo ou para cozinhar, aquecer ou secar algo | destacar}
  \end{Phonetics}
\end{Entry}

\begin{Entry}{烘干}{10,3}{⽕、⼲}
  \begin{Phonetics}{烘干}{hong1gan1}[][HSK 7-9]
    \definition{v.}{secar em fogo alto | secar ao lado ou sobre o fogo | assar; secar no forno}
  \end{Phonetics}
\end{Entry}

\begin{Entry}{烘托}{10,6}{⽕、⼿}
  \begin{Phonetics}{烘托}{hong1tuo1}[][HSK 7-9]
    \definition{v.}{adicionar sombreamento ao redor de um objeto para destacá-lo; um dos métodos de pintura chinesa, que utiliza tinta ou cores claras para pontilhar o contorno do objeto e torná-lo mais claro | destacar por contraste; colocar em nítido relevo; fazer com que se destaque}
  \end{Phonetics}
\end{Entry}

%%%%%%%%%% 烟 %%%%%%%%%%
\subsection*{烟}

\begin{Entry}{烟}{10}{⽕}
  \begin{Phonetics}{烟}{yan1}[][HSK 3]
    \definition[股,支,根,盒,包]{s.}{fumaça; gás produzido pela combustão de materiais, misturado com pequenas partículas não completamente queimadas | névoa; neblina | tabaco; planta de tabaco | fumo; cigarro; termo geral para cigarros, charutos, etc. | ópio | fuligem; fumaça de carvão}
    \definition{v.}{ficar irritado com a fumaça (os olhos lacrimejam ou não conseguem abrir)}
  \end{Phonetics}
\end{Entry}

\begin{Entry}{烟火}{10,4}{⽕、⽕}
  \begin{Phonetics}{烟火}{yan1huo3}
    \definition{s.}{fogo de artifício}
  \end{Phonetics}
\end{Entry}

\begin{Entry}{烟叶}{10,5}{⽕、⼝}
  \begin{Phonetics}{烟叶}{yan1ye4}
    \definition{s.}{folha de tabaco}
  \end{Phonetics}
\end{Entry}

\begin{Entry}{烟头}{10,5}{⽕、⼤}
  \begin{Phonetics}{烟头}{yan1tou2}
    \definition[根]{s.}{bituca de cigarro}
  \end{Phonetics}
\end{Entry}

\begin{Entry}{烟囱}{10,7}{⽕、⼞}
  \begin{Phonetics}{烟囱}{yan1cong1}
    \definition{s.}{chaminé}
  \end{Phonetics}
\end{Entry}

\begin{Entry}{烟花}{10,7}{⽕、⾋}
  \begin{Phonetics}{烟花}{yan1 hua1}[][HSK 6]
    \definition[场,朵]{s.}{fogos de artifício; uma coisa que emite faíscas de várias cores quando exposta à observação | prostituta; antigamente, referia-se a algo relacionado à prostituição}
  \end{Phonetics}
\end{Entry}

\begin{Entry}{烟雨}{10,8}{⽕、⾬}
  \begin{Phonetics}{烟雨}{yan1yu3}
    \definition{s.}{chuvisco | garoa}
  \end{Phonetics}
\end{Entry}

\begin{Entry}{烟草}{10,9}{⽕、⾋}
  \begin{Phonetics}{烟草}{yan1cao3}
    \definition{s.}{tabaco}
  \end{Phonetics}
\end{Entry}

%%%%%%%%%% 烤 %%%%%%%%%%
\subsection*{烤}

\begin{Entry}{烤}{10}{⽕}
  \begin{Phonetics}{烤}{kao3}
    \definition{v.}{assar | grelhar}
  \end{Phonetics}
\end{Entry}

\begin{Entry}{烤肉}{10,6}{⽕、⾁}
  \begin{Phonetics}{烤肉}{kao3 rou4}[][HSK 5]
    \definition[块,串,片,盘]{s.}{churrasco (literalmente carne assada)}
  \end{Phonetics}
\end{Entry}

\begin{Entry}{烤鸭}{10,10}{⽕、⿃}
  \begin{Phonetics}{烤鸭}{kao3ya1}[][HSK 5]
    \definition[只,盘]{s.}{pato assado; pato recheado e assado em um forno especial após ser abatido}
  \end{Phonetics}
\end{Entry}

%%%%%%%%%% 烦 %%%%%%%%%%
\subsection*{烦}

\begin{Entry}{烦}{10}{⽕}
  \begin{Phonetics}{烦}{fan2}[][HSK 4]
    \definition{adj.}{redundante e confuso | supérfluo e confuso; muito bagunçado}
    \definition{v.}{aborrecer | irritar; incomodar; estar cansado de; ficar irritado | incomodar; solicitar}
  \end{Phonetics}
\end{Entry}

\begin{Entry}{烦闷}{10,7}{⽕、⾨}
  \begin{Phonetics}{烦闷}{fan2men4}[][HSK 7-9]
    \definition{adj.}{infeliz; deprimido; mal-humorado | desconfortável}
  \end{Phonetics}
\end{Entry}

\begin{Entry}{烦恼}{10,9}{⽕、⼼}
  \begin{Phonetics}{烦恼}{fan2nao3}[][HSK 7-9]
    \definition{adj.}{irritado; preocupado; incomodado}
    \definition[个,种,些]{s.}{aborrecimento; coisas que te incomodam}
  \end{Phonetics}
\end{Entry}

\begin{Entry}{烦躁}{10,20}{⽕、⾜}
  \begin{Phonetics}{烦躁}{fan2zao4}[][HSK 7-9]
    \definition{adj.}{inquieto; agitado; irritável}
  \end{Phonetics}
\end{Entry}

%%%%%%%%%% 烧 %%%%%%%%%%
\subsection*{烧}

\begin{Entry}{烧}{10}{⽕}
  \begin{Phonetics}{烧}{shao1}[][HSK 4]
    \definition[次]{s.}{febre; temperatura corporal mais alta do que o normal}
    \definition{v.}{queimar; pegar fogo | cozinhar; aquecer; assar | guisar depois de fritar ou fritar depois de guisar | assar; grelhar os ingredientes dos alimentos diretamente sobre o fogo | ter febre; estar com febre | danificar (matar ou murchar) as plantas pelo uso excessivo (ou inadequado) de fertilizantes | tornar-se arrogante ou presunçoso; metáfora de estar em uma boa posição e se deixar levar}
  \end{Phonetics}
\end{Entry}

\begin{Entry}{烧烤}{10,10}{⽕、⽕}
  \begin{Phonetics}{烧烤}{shao1kao3}
    \definition{s.}{churrasco}
    \definition{v.}{assar}
  \end{Phonetics}
\end{Entry}

%%%%%%%%%% 热 %%%%%%%%%%
\subsection*{热}

\begin{Entry}{热}{10}{⽕}
  \begin{Phonetics}{热}{re4}[][HSK 1]
    \definition{adj.}{quente; temperatura elevada | ardente; caloroso; profundamente afetuoso | ansioso; invejoso; descreve inveja e desejo de possuir algo | térmico; altamente radioativo | popular; muito procurado; muito apreciado por muitas pessoas}
    \definition{s.}{calor; energia liberada pelo movimento irregular das moléculas dentro de um objeto | febre; febre alta causada por doença | moda passageira; mania; febre}
    \definition{v.}{aquecer (geralmente se refere a alimentos)}
  \end{Phonetics}
\end{Entry}

\begin{Entry}{热门}{10,3}{⽕、⾨}
  \begin{Phonetics}{热门}{re4men2}[][HSK 5]
    \definition{adj.}{popular; durante um período de tempo, foi algo que interessava a todos}
    \definition{s.}{algo que desperta o interesse popular; metáfora para algo que está na moda e recebe a atenção de todos (em contraste com 冷门)}
  \seealsoref{冷门}{leng3men2}
  \end{Phonetics}
\end{Entry}

\begin{Entry}{热心}{10,4}{⽕、⼼}
  \begin{Phonetics}{热心}{re4xin1}[][HSK 4]
    \definition{adj.}{ardente; sincero; entusiasmado; afetuoso; apaixonado; interessado}
    \definition{v.}{ser entusiasmado com alguma coisa}
  \end{Phonetics}
\end{Entry}

\begin{Entry}{热水}{10,4}{⽕、⽔}
  \begin{Phonetics}{热水}{re4 shui3}[][HSK 6]
    \definition{s.}{água quente; água em temperatura mais alta}
  \end{Phonetics}
\end{Entry}

\begin{Entry}{热水器}{10,4,16}{⽕、⽔、⼝}
  \begin{Phonetics}{热水器}{re4 shui3 qi4}[][HSK 6]
    \definition[台]{s.}{aquecedor de água; aparelhos que aquecem água usando eletricidade, gás natural, gás liquefeito de petróleo ou energia solar}
  \end{Phonetics}
\end{Entry}

\begin{Entry}{热血沸腾}{10,6,8,13}{⽕、⾎、⽔、⾁}
  \begin{Phonetics}{热血沸腾}{re4xue4-fei4teng2}
    \definition{expr.}{estar animado; ter o sangue correndo}
  \end{Phonetics}
\end{Entry}

\begin{Entry}{热泪盈眶}{10,8,9,11}{⽕、⽔、⽫、⽬}
  \begin{Phonetics}{热泪盈眶}{re4lei4ying2kuang4}
    \definition{expr.}{olhos cheios de lágrimas de emoção | extremamente emocionado}
  \end{Phonetics}
\end{Entry}

\begin{Entry}{热线}{10,8}{⽕、⽷}
  \begin{Phonetics}{热线}{re4 xian4}[][HSK 6]
    \definition[条]{s.}{raio infravermelho | linha direta; \emph{hot line}; uma linha telefônica ou telegráfica direta; uma linha para um ponto de acesso | rota quente (ou movimentada, popular) | raio de calor}
  \end{Phonetics}
\end{Entry}

\begin{Entry}{热闹}{10,8}{⽕、⾾}
  \begin{Phonetics}{热闹}{re4nao5}[][HSK 4]
    \definition{adj.}{animado; agitado; movimentado com barulho e excitação; descreve uma cena animada com uma atmosfera calorosa}
    \definition{s.}{uma vista emocionante; uma cena de agitação e excitação; atmosfera acolhedora}
    \definition{v.}{animar; divertir-se}
  \end{Phonetics}
\end{Entry}

\begin{Entry}{热点}{10,9}{⽕、⽕}
  \begin{Phonetics}{热点}{re4 dian3}[][HSK 6]
    \definition{s.}{ponto de acesso; \emph{hotspot}}
  \end{Phonetics}
\end{Entry}

\begin{Entry}{热烈}{10,10}{⽕、⽕}
  \begin{Phonetics}{热烈}{re4lie4}[][HSK 3]
    \definition{adj.}{caloroso; fervoroso; ardente; entusiasmado; excitado}
  \end{Phonetics}
\end{Entry}

\begin{Entry}{热爱}{10,10}{⽕、⽖}
  \begin{Phonetics}{热爱}{re4'ai4}[][HSK 3]
    \definition{v.}{amar ardentemente; amar de coração; ter amor profundo por; amar apaixonadamente}
  \end{Phonetics}
\end{Entry}

\begin{Entry}{热情}{10,11}{⽕、⼼}
  \begin{Phonetics}{热情}{re4qing2}[][HSK 2]
    \definition{adj.}{caloroso; fervoroso; entusiasmado; cordial; descreve sentimentos calorosos por alguém}
    \definition{s.}{entusiasmo; ardor; devoção; calor humano; zelo; sentimentos calorosos}
  \end{Phonetics}
\end{Entry}

\begin{Entry}{热量}{10,12}{⽕、⾥}
  \begin{Phonetics}{热量}{re4 liang4}[][HSK 5]
    \definition{s.}{calor; quantidade de calor; calorias; em física, refere-se à energia transferida entre objetos com temperaturas diferentes, do objeto com temperatura mais alta para o objeto com temperatura mais baixa}
  \end{Phonetics}
\end{Entry}

%%%%%%%%%% 爱 %%%%%%%%%%
\subsection*{爱}

\begin{Entry}{爱}{10}{⽖}
  \begin{Phonetics}{爱}{ai4}[][HSK 1]
    \definition*{s.}{Sobrenome: Ai}
    \definition[个]{s.}{amor; afeição profunda; preocupação profunda; especialmente amor entre pessoas}[爱是理解和包容。===O amor é compreensão e tolerância.]
    \definition{v.}{amar; ter sentimentos profundos por pessoas ou coisas | gostar; gostar de; estar interessado em |  cuidar; valorizar; ter em alta estima; cuidar bem de | estar apto a; ter o hábito de}[他们深深爱着对方。===Eles se amam profundamente. | 我爱我的家人。===Eu amo minha família. | 我爱旅行。===Eu adoro viajar.]
  \end{Phonetics}
\end{Entry}

\begin{Entry}{爱人}{10,2}{⽖、⼈}
  \begin{Phonetics}{爱人}{ai4 ren5}[][HSK 2]
    \definition[个]{s.}{amante; \emph{dollbaby}; namorado(a) | marido ou esposa; mais usado em ocasiões formais}[这是我的爱人。===Este é o meu/minha esposo/companheiro. | 她是我一生的爱人。===Ela é o amor da minha vida. | 请携带爱人出席晚宴。===Por favor, traga seu cônjuge para o jantar.]
  \end{Phonetics}
\end{Entry}

\begin{Entry}{爱上}{10,3}{⽖、⼀}
  \begin{Phonetics}{爱上}{ai4shang4}
    \definition{v.}{perder o coração por; apaixonar-se por}[他在旅行时爱上了一位法国女孩。===Ele se apaixonou por uma garota francesa durante a viagem.  | 来到杭州后,我爱上了龙井茶。===Depois de chegar em Hangzhou, me apaixonei pelo chá Longjing. | 我从来没想过自己会爱上健身。===Eu nunca imaginei que iria me apaixonar por academia.]
  \end{Phonetics}
\end{Entry}

\begin{Entry}{爱不释手}{10,4,12,4}{⽖、⼀、⾤、⼿}
  \begin{Phonetics}{爱不释手}{ai4bu2shi4shou3}[][HSK 7-9]
    \definition{expr.}{``Não consigo parar de ler.''; ``Amo tanto que não consigo deixar passar.''; gostar (amar) algo tanto que não se pode suportar separar-se dele}
  \end{Phonetics}
\end{Entry}

\begin{Entry}{爱心}{10,4}{⽖、⼼}
  \begin{Phonetics}{爱心}{ai4xin1}[][HSK 3]
    \definition[片]{s.}{amor; carinho; compaixão; um sentimento de preocupação e carinho por outras pessoas ou animais}
  \end{Phonetics}
\end{Entry}

\begin{Entry}{爱好}{10,6}{⽖、⼥}
  \begin{Phonetics}{爱好}{ai4 hao4}[][HSK 1]
    \definition[个,种]{s.}{passatempo; interesse; \emph{hobby}; sentimentos de interesse especial ou afeição por algo | 爱好 é mais usado para atividades regulares (esportes, música), enquanto 喜欢 é para preferências gerais}[他的爱好是收集邮票。===Seu hobby era colecionar selos.  | 我的爱好是读书和旅行。===Meus hobbies são ler e viajar.]
    \definition{v.}{estar interessado em; ter prazer em; ter um forte interesse em algo; ter sentimentos profundos por alguém ou algo}
  \seealsoref{喜欢}{xi3huan5}
  \end{Phonetics}
\end{Entry}

\begin{Entry}{爱好者}{10,6,8}{⽖、⼥、⽼}
  \begin{Phonetics}{爱好者}{ai4 hao4 zhe3}
    \definition{s.}{hobbista; amador; entusiasta; fã; amante (de arte, esportes, etc.)}[他是一位摄影爱好者。===Ele é um entusiasta de fotografia. | 她是位潜水爱好者,经常去东南亚潜水。===Ela é uma mergulhadora amadora e frequentemente mergulha no Sudeste Asiático.  | 我们为书法爱好者创建了一个微信群。===Criamos um grupo no WeChat para amantes de caligrafia.]
  \end{Phonetics}
\end{Entry}

\begin{Entry}{爱抚}{10,7}{⽖、⼿}
  \begin{Phonetics}{爱抚}{ai4fu3}
    \definition{s.}{carinho; carícia}
    \definition{v.}{acariciar; afagar; cuidar (com ternura)}[他轻轻爱抚她的头发。===Ele afagou suavemente o cabelo dela. | 母亲爱抚婴儿的脸颊。===A mãe acaricia a bochecha do bebê. | 她爱抚着小猫的耳朵。===Ela acariciou as orelhas do gatinho.]
  \end{Phonetics}
\end{Entry}

\begin{Entry}{爱护}{10,7}{⽖、⼿}
  \begin{Phonetics}{爱护}{ai4hu4}[][HSK 4]
    \definition{v.}{acalentar; valorizar; salvaguardar; cuidar bem de}[全社会都应爱护老年人。===Toda a sociedade deve tratar os idosos com cuidado e respeito. | 请爱护公园里的小动物。===Por favor, tratem os animais do parque com cuidado.]
  \end{Phonetics}
\end{Entry}

\begin{Entry}{爱国}{10,8}{⽖、⼞}
  \begin{Phonetics}{爱国}{ai4 guo2}[][HSK 4]
    \definition{adj.}{patriótico; patriotismo}[爱国是每个公民的责任。===O patriotismo é o dever de todo cidadão. | 这部电影讲述了英雄的爱国故事。===Este filme conta a história patriótica de um herói.]
    \definition{v.}{ser patriota; amar o seu país}
  \end{Phonetics}
\end{Entry}

\begin{Entry}{爱面子}{10,9,3}{⽖、⾯、⼦}
  \begin{Phonetics}{爱面子}{ai4/mian4zi5}[][HSK 7-9]
    \definition{v.+compl.}{estar preocupado em salvar a face; ser vigilante em relação à reputação; ser sensível ao próprio orgulho; valorizar minha própria dignidade e ter medo que os outros me desprezem}[他爱面子,怕别人笑话他。===Ele se importa com sua reputação e tem medo que os outros riam dele.]
  \end{Phonetics}
\end{Entry}

\begin{Entry}{爱爱}{10,10}{⽖、⽖}
  \begin{Phonetics}{爱爱}{ai4'ai5}
    \definition{v.}{Coloquial: fazer amor ou relações íntimas | pode ser usado como um apelido entre casais, transmitindo ternura | pode soar vulgar se usado em contextos inadequados}[他们俩刚结婚,天天都想爱爱。===Eles acabaram de se casar e querem fazer amor todo dia. | 爱爱,你今天好漂亮!===Amor, você está linda hoje!]
  \end{Phonetics}
\end{Entry}

\begin{Entry}{爱情}{10,11}{⽖、⼼}
  \begin{Phonetics}{爱情}{ai4qing2}[][HSK 2]
    \definition{s.}{amor (entre pessoas); afeição}[爱情是盲目的。===O amor é cego. | 爱情如同玫瑰,美丽却带刺。===O amor é como uma rosa, bela mas com espinhos.  | 这首歌讲述了破碎的爱情故事。===Esta música conta uma história de amor fracassado.]
  \end{Phonetics}
\end{Entry}

\begin{Entry}{爱惜}{10,11}{⽖、⼼}
  \begin{Phonetics}{爱惜}{ai4xi1}[][HSK 7-9]
    \definition{v.}{valorizar; prezar; estimar; usar com moderação; não desperdiçar}
  \end{Phonetics}
\end{Entry}

\begin{Entry}{爱理不理}{10,11,4,11}{⽖、⽟、⼀、⽟}
  \begin{Phonetics}{爱理不理}{ai4li3-bu4li3}[][HSK 7-9]
    \definition{expr.}{frio e indiferente; distante}
  \end{Phonetics}
\end{Entry}

\begin{Entry}{爱戴}{10,17}{⽖、⼽}
  \begin{Phonetics}{爱戴}{ai4dai4}
    \definition{v.}{reverenciar; adorar; amar profundamente e respeitar (a líderes, celebridades, etc.)}
  \end{Phonetics}
\end{Entry}

%%%%%%%%%% 爹 %%%%%%%%%%
\subsection*{爹}

\begin{Entry}{爹}{10}{⽗}
  \begin{Phonetics}{爹}{die1}[][HSK 7-9]
    \definition[个]{s.}{Coloquial: pai; papai | velho pai; um título respeitoso para homens idosos em algumas áreas}
  \end{Phonetics}
\end{Entry}

%%%%%%%%%% 特 %%%%%%%%%%
\subsection*{特}

\begin{Entry}{特}{10}{⽜}
  \begin{Phonetics}{特}{te4}[][HSK 6]
    \definition{adj.}{especial; incomum; particular; excepcional; diferente do geral | especial; solteiro; solitário}
    \definition{adv.}{muito; extremamente | especialmente; para um propósito especial |mas; somente}
    \definition{clas.}{TEX; abreviação para unidades de medida como TEX; a unidade de medida TEX indica a espessura de um fio têxtil através do seu peso}
    \definition{s.}{espião; agente secreto}
  \end{Phonetics}
\end{Entry}

\begin{Entry}{特大}{10,3}{⽜、⼤}
  \begin{Phonetics}{特大}{te4 da4}[][HSK 6]
    \definition{adj.}{especialmente (excepcionalmente) grande; o mais}
  \end{Phonetics}
\end{Entry}

\begin{Entry}{特价}{10,6}{⽜、⼈}
  \begin{Phonetics}{特价}{te4 jia4}[][HSK 4]
    \definition{s.}{oferta especial; preço de barganha; preço especial reduzido}
  \end{Phonetics}
\end{Entry}

\begin{Entry}{特地}{10,6}{⽜、⼟}
  \begin{Phonetics}{特地}{te4 di4}[][HSK 6]
    \definition{adv.}{especialmente; propositalmente; para um propósito especial}
  \end{Phonetics}
\end{Entry}

\begin{Entry}{特有}{10,6}{⽜、⽉}
  \begin{Phonetics}{特有}{te4 you3}[][HSK 5]
    \definition{adj.}{específico; peculiar; característico; único; exclusivo; especial}
  \end{Phonetics}
\end{Entry}

\begin{Entry}{特色}{10,6}{⽜、⾊}
  \begin{Phonetics}{特色}{te4se4}[][HSK 3]
    \definition{s.}{característica; característica distintiva; a cor única, estilo, etc. de um objeto}
  \end{Phonetics}
\end{Entry}

\begin{Entry}{特别}{10,7}{⽜、⼑}
  \begin{Phonetics}{特别}{te4bie2}[][HSK 2]
    \definition{adj.}{especial; particular; fora do comum; diferente dos outros, com características próprias}
    \definition{adv.}{especialmente; particularmente | ainda mais; em particular; frequentemente usado com 是 | especialmente; deliberadamente; para um propósito específico}
  \seealsoref{是}{shi4}
  \end{Phonetics}
\end{Entry}

\begin{Entry}{特别快车}{10,7,7,4}{⽜、⼑、⼼、⾞}
  \begin{Phonetics}{特别快车}{te4bie2 kuai4che1}
    \definition{s.}{trem expresso; expresso; expresso especial; refere-se a trens de passageiros que param em menos estações e têm menor tempo de viagem do que trens expressos diretos}
  \end{Phonetics}
\end{Entry}

\begin{Entry}{特快}{10,7}{⽜、⼼}
  \begin{Phonetics}{特快}{te4 kuai4}[][HSK 6]
    \definition{adj.}{expresso (trem, entrega etc.)}
    \definition{s.}{trem expresso (opp. 普快); abreviação de 特别快车}
  \seealsoref{特别快车}{te4bie2 kuai4che1}
  \end{Phonetics}
\end{Entry}

\begin{Entry}{特技}{10,7}{⽜、⼿}
  \begin{Phonetics}{特技}{te4ji4}
    \definition{s.}{efeito especial | dublê}
  \end{Phonetics}
\end{Entry}

\begin{Entry}{特定}{10,8}{⽜、⼧}
  \begin{Phonetics}{特定}{te4ding4}[][HSK 5]
    \definition{adj.}{específico; especialmente designado | dado; especificado; específico (pessoa, hora, lugar, local, ambiente, etc.)}
  \end{Phonetics}
\end{Entry}

\begin{Entry}{特征}{10,8}{⽜、⼻}
  \begin{Phonetics}{特征}{te4zheng1}[][HSK 4]
    \definition[个,种]{s.}{característica; aparência ou o fenômeno característico de uma pessoa ou coisa que pode ser visto de fora}
  \end{Phonetics}
\end{Entry}

\begin{Entry}{特性}{10,8}{⽜、⼼}
  \begin{Phonetics}{特性}{te4 xing4}[][HSK 5]
    \definition[种,个]{s.}{propriedade específica (ou característica) | característica; sabores | propriedade}
  \end{Phonetics}
\end{Entry}

\begin{Entry}{特点}{10,9}{⽜、⽕}
  \begin{Phonetics}{特点}{te4dian3}[][HSK 2]
    \definition[个,大]{s.}{característica; peculiaridade; traço distintivo; a singularidade de uma pessoa ou coisa}
  \end{Phonetics}
\end{Entry}

\begin{Entry}{特殊}{10,10}{⽜、⽍}
  \begin{Phonetics}{特殊}{te4shu1}[][HSK 4]
    \definition{adj.}{especial; particular; peculiar; excepcional; incomum}
  \end{Phonetics}
\end{Entry}

\begin{Entry}{特意}{10,13}{⽜、⼼}
  \begin{Phonetics}{特意}{te4yi4}[][HSK 6]
    \definition{adv.}{especialmente; para um propósito especial}
  \end{Phonetics}
\end{Entry}

%%%%%%%%%% 牺 %%%%%%%%%%
\subsection*{牺}

\begin{Entry}{牺}{10}{⽜}
  \begin{Phonetics}{牺}{xi1}
    \definition{s.}{um animal de cor uniforme para sacrifício; sacrifício; gado com pelagem pura usado para sacrifício}
  \end{Phonetics}
\end{Entry}

\begin{Entry}{牺牲}{10,9}{⽜、⽜}
  \begin{Phonetics}{牺牲}{xi1sheng1}[][HSK 6]
    \definition[份]{s.}{sacrifício; um animal abatido para sacrifício; refere-se ao sacrifício da própria vida ou dos próprios interesses por um propósito justo, ou refere-se ao preço pago por um determinado propósito}
    \definition{v.}{sacrificar-se; morrer como mártir; dar a própria vida; sacrificar sua vida pela justiça | sacrificar; desistir; fazer algo às custas de; geralmente se refere a pagar um preço ou sofrer danos por alguém ou algo}
  \end{Phonetics}
\end{Entry}

%%%%%%%%%% 狼 %%%%%%%%%%
\subsection*{狼}

\begin{Entry}{狼}{10}{⽝}
  \begin{Phonetics}{狼}{lang2}[][HSK 7-9]
    \definition*{s.}{Sírius (estrela) | Sobrenome: Lang}
    \definition[只,匹,群,条]{s.}{lobo}
  \end{Phonetics}
\end{Entry}

\begin{Entry}{狼狈}{10,7}{⽝、⽝}
  \begin{Phonetics}{狼狈}{lang2bei4}[][HSK 7-9]
    \definition{adj.}{em uma posição difícil; em um canto apertado; descreve um estado de angústia ou constrangimento}
  \end{Phonetics}
\end{Entry}

%%%%%%%%%% 猃 %%%%%%%%%%
\subsection*{猃}

\begin{Entry}{猃}{10}{⽝}
  \begin{Phonetics}{猃}{xian3}
    \definition{s.}{(arcaico) um tipo de cão com focinho longo}
  \end{Phonetics}
\end{Entry}

\begin{Entry}{猃狁}{10,7}{⽝、⽝}
  \begin{Phonetics}{猃狁}{xian3yun3}
    \definition*{s.}{Termo da dinastia Zhou para uma tribo nômade do norte mais tarde chamou o Xiongnu (匈奴) nas dinastias Qin e Han}
  \seealsoref{匈奴}{xiong1nu2}
  \end{Phonetics}
\end{Entry}

%%%%%%%%%% 珠 %%%%%%%%%%
\subsection*{珠}

\begin{Entry}{珠}{10}{⽟}
  \begin{Phonetics}{珠}{zhu1}
    \definition[粒,颗]{s.}{pérola | conta (de colar, ábaco, etc.) | coisa parecida com uma bola (como um globo ocular)}
  \end{Phonetics}
\end{Entry}

\begin{Entry}{珠子}{10,3}{⽟、⼦}
  \begin{Phonetics}{珠子}{zhu1zi5}
    \definition[粒,颗]{s.}{pérola | contas}
  \end{Phonetics}
\end{Entry}

\begin{Entry}{珠宝}{10,8}{⽟、⼧}
  \begin{Phonetics}{珠宝}{zhu1 bao3}[][HSK 6]
    \definition[串]{s.}{joias; pérolas; um termo geral para pérolas, pedras preciosas e outros ornamentos}
  \end{Phonetics}
\end{Entry}

%%%%%%%%%% 班 %%%%%%%%%%
\subsection*{班}

\begin{Entry}{班}{10}{⽟}
  \begin{Phonetics}{班}{ban1}[][HSK 1]
    \definition*{s.}{Sobrenome: Ban}
    \definition{adj.}{regular; programado; executado regularmente; com horários fixos (meios de transporte)}
    \definition{clas.}{um grupo de; uma classe de; usado para pessoas | meios de transporte com horários fixos}
    \definition[个]{s.}{equipe; turma; organização estruturada | dever; turno; período de trabalho dentro de um dia | equipe; esquadrão; unidade básica das forças armadas | nome usado antigamente para designar uma companhia teatral}
    \definition{v.}{mover; implantar; implementar}
  \end{Phonetics}
\end{Entry}

\begin{Entry}{班长}{10,4}{⽟、⾧}
  \begin{Phonetics}{班长}{ban1 zhang3}[][HSK 2]
    \definition[个,位,名]{s.}{monitor de turma; líder de equipe; alunos responsáveis nas turmas da escola | líder de esquadrão; responsável por uma turma de soldados, geralmente com patente de sargento}
  \end{Phonetics}
\end{Entry}

\begin{Entry}{班级}{10,6}{⽟、⽷}
  \begin{Phonetics}{班级}{ban1 ji2}[][HSK 3]
    \definition[个]{s.}{classe; série (na escola); o nome geral para as séries e turmas da escola}
  \end{Phonetics}
\end{Entry}

%%%%%%%%%% 瓶 %%%%%%%%%%
\subsection*{瓶}

\begin{Entry}{瓶}{10}{⽡}
  \begin{Phonetics}{瓶}{ping2}[][HSK 2]
    \definition*{s.}{Sobrenome: Ping}
    \definition{clas.}{usado para coisas que são engarrafadas; quantidade contida em um frasco, vaso, garrafa}
    \definition[个]{s.}{jarra; vaso; frasco; garrafa}
  \end{Phonetics}
\end{Entry}

\begin{Entry}{瓶子}{10,3}{⽡、⼦}
  \begin{Phonetics}{瓶子}{ping2zi5}[][HSK 2]
    \definition[个,只,种]{s.}{garrafa; recipientes com gargalo feitos de cerâmica, vidro, plástico, etc., geralmente em forma cilíndrica}
  \end{Phonetics}
\end{Entry}

\begin{Entry}{瓶盖}{10,11}{⽡、⽫}
  \begin{Phonetics}{瓶盖}{ping2gai4}
    \definition{s.}{tampa de garrafa}
  \end{Phonetics}
\end{Entry}

\begin{Entry}{瓶装}{10,12}{⽡、⾐}
  \begin{Phonetics}{瓶装}{ping2zhuang1}
    \definition{adj.}{engarrafado}
  \end{Phonetics}
\end{Entry}

%%%%%%%%%% 瓷 %%%%%%%%%%
\subsection*{瓷}

\begin{Entry}{瓷}{10}{⽡}
  \begin{Phonetics}{瓷}{ci2}[][HSK 7-9]
    \definition{adj.}{Dialeto: (relação) próxima; íntima}
    \definition{s.}{artigos de porcelana}
  \end{Phonetics}
\end{Entry}

\begin{Entry}{瓷器}{10,16}{⽡、⼝}
  \begin{Phonetics}{瓷器}{ci2qi4}[][HSK 7-9]
    \definition[件,种]{s.}{porcelana; louça; utensílios feitos de argila de porcelana, feldspato, quartzo, etc.}
  \end{Phonetics}
\end{Entry}

%%%%%%%%%% 留 %%%%%%%%%%
\subsection*{留}

\begin{Entry}{留}{10}{⽥}
  \begin{Phonetics}{留}{liu2}[][HSK 2]
    \definition*{s.}{Sobrenome: Liu}
    \definition{v.}{ficar; permanecer; parar em um determinado local ou posição; não se afastar | estudar no exterior (geralmente seguido pelo nome de um país com uma sílaba) | pedir a alguém para ficar; manter alguém onde está | concentrar-se em; concentrar a atenção em algo | manter; guardar; reservar; não joger fora | acumular; deixar crescer | aceitar; receber | transmitir (legado); deixar para trás}
  \end{Phonetics}
\end{Entry}

\begin{Entry}{留下}{10,3}{⽥、⼀}
  \begin{Phonetics}{留下}{liu2 xia4}[][HSK 2]
    \definition{v.}{deixar; parar em algum lugar}
  \end{Phonetics}
\end{Entry}

\begin{Entry}{留心}{10,4}{⽥、⼼}
  \begin{Phonetics}{留心}{liu2/xin1}[][HSK 7-9]
    \definition{v.+compl.}{cuidar; ser atencioso | ficar atento; estar atento (a)}
  \end{Phonetics}
\end{Entry}

\begin{Entry}{留言}{10,7}{⽥、⾔}
  \begin{Phonetics}{留言}{liu2 yan2}[][HSK 6]
    \definition[条]{s.}{mensagem; recado}
    \definition{v.}{deixar uma mensagem; deixar seus comentários}
  \end{Phonetics}
\end{Entry}

\begin{Entry}{留学}{10,8}{⽥、⼦}
  \begin{Phonetics}{留学}{liu2xue2}[][HSK 3]
    \definition{v.}{estudar no exterior; permanecer no estrangeiro para estudar ou pesquisar}
  \end{Phonetics}
\end{Entry}

\begin{Entry}{留学生}{10,8,5}{⽥、⼦、⽣}
  \begin{Phonetics}{留学生}{liu2 xue2 sheng1}[][HSK 2]
    \definition[个,位,名,批,帮]{s.}{estudante estrangeiro; estudante que retornou; estudante que estuda no exterior}
  \end{Phonetics}
\end{Entry}

\begin{Entry}{留念}{10,8}{⽥、⼼}
  \begin{Phonetics}{留念}{liu2nian4}[][HSK 7-9]
    \definition{v.}{aceitar como lembrança; guardar como lembrança (frequentemente usado como presente de despedida)}
  \end{Phonetics}
\end{Entry}

\begin{Entry}{留神}{10,9}{⽥、⽰}
  \begin{Phonetics}{留神}{liu2/shen2}[][HSK 7-9]
    \definition{v.+compl.}{ser cuidadoso; tomar cuidado; ficar atento; manter os olhos bem abertos; ficar de olho na situação; ficar atento às condições climáticas; estar atento a}
  \end{Phonetics}
\end{Entry}

\begin{Entry}{留恋}{10,10}{⽥、⼼}
  \begin{Phonetics}{留恋}{liu2lian4}[][HSK 7-9]
    \definition{v.}{relutante em partir; odiar ter que ir | indisposto a desistir | recordar com carinho}
  \end{Phonetics}
\end{Entry}

\begin{Entry}{留意}{10,13}{⽥、⼼}
  \begin{Phonetics}{留意}{liu2/yi4}[][HSK 7-9]
    \definition{v.+compl.}{ter cuidado; ficar atento; manter os olhos abertos; prestar atenção}
  \end{Phonetics}
\end{Entry}

%%%%%%%%%% 畜 %%%%%%%%%%
\subsection*{畜}

\begin{Entry}{畜}{10}{⽥}
  \begin{Phonetics}{畜}{chu4}
    \definition*{s.}{Sobrenome: Chu}
    \definition{s.}{animal doméstico; gado; bestas, principalmente referindo-se ao gado}
  \end{Phonetics}
  \begin{Phonetics}{畜}{xu4}
    \definition{v.}{criar (animais domésticos)}
  \end{Phonetics}
\end{Entry}

%%%%%%%%%% 疼 %%%%%%%%%%
\subsection*{疼}

\begin{Entry}{疼}{10}{⽧}
  \begin{Phonetics}{疼}{teng2}[][HSK 2]
    \definition{adj.}{dolorido; doído; sensação de extremo desconforto causada por ferimentos, doenças, etc.}
    \definition{v.}{ferir; machucar | adorar; amar profundamente; gostar muito; cuidar}
  \end{Phonetics}
\end{Entry}

\begin{Entry}{疼痛}{10,12}{⽧、⽧}
  \begin{Phonetics}{疼痛}{teng2 tong4}[][HSK 6]
    \definition[阵,种]{s.}{dor; sofrimento; ferimento; descreve a sensação de dor causada por lesão ou doença}
  \end{Phonetics}
\end{Entry}

%%%%%%%%%% 疾 %%%%%%%%%%
\subsection*{疾}

\begin{Entry}{疾}{10}{⽧}
  \begin{Phonetics}{疾}{ji2}
    \definition*{s.}{Sobrenome: Ji}
    \definition{s.}{doença; enfermidade; moléstia; padecimento | sofrimento; dor; dificuldade; mazela}
  \end{Phonetics}
\end{Entry}

\begin{Entry}{疾病}{10,10}{⽧、⽧}
  \begin{Phonetics}{疾病}{ji2bing4}[][HSK 6]
    \definition[种]{s.}{doença; enfermidade; termo geral para doença}
  \end{Phonetics}
\end{Entry}

%%%%%%%%%% 病 %%%%%%%%%%
\subsection*{病}

\begin{Entry}{病}{10}{⽧}
  \begin{Phonetics}{病}{bing4}[][HSK 1]
    \definition[种]{s.}{doença; enfermidade | doença; males | falha; defeito; desvantagem; erro}
    \definition{v.}{adoecer; ficar doente | ferir; causar danos a | angustiar; desaprovar}
  \end{Phonetics}
\end{Entry}

\begin{Entry}{病人}{10,2}{⽧、⼈}
  \begin{Phonetics}{病人}{bing4 ren2}[][HSK 1]
    \definition[个,位]{s.}{doente; paciente; pessoas doentes; pessoas em tratamento}
  \end{Phonetics}
\end{Entry}

\begin{Entry}{病床}{10,7}{⽧、⼴}
  \begin{Phonetics}{病床}{bing4chuang2}[][HSK 7-9]
    \definition[号,张]{s.}{cama de hospital | leito de doente}
  \end{Phonetics}
\end{Entry}

\begin{Entry}{病房}{10,8}{⽧、⼾}
  \begin{Phonetics}{病房}{bing4 fang2}[][HSK 6]
    \definition[个,间]{s.}{enfermaria de um hospital; quartos onde ficam os pacientes em hospitais e onde vivem em casas de repouso}
  \end{Phonetics}
\end{Entry}

\begin{Entry}{病毒}{10,9}{⽧、⽏}
  \begin{Phonetics}{病毒}{bing4du2}[][HSK 5]
    \definition[种,株,类]{s.}{vírus; patógenos que são menores que os germes e visíveis somente com um microscópio eletrônico | Computação: vírus de computador}
  \end{Phonetics}
\end{Entry}

\begin{Entry}{病症}{10,10}{⽧、⽧}
  \begin{Phonetics}{病症}{bing4zheng4}[][HSK 7-9]
    \definition[种]{s.}{doença; enfermidade}
  \end{Phonetics}
\end{Entry}

\begin{Entry}{病情}{10,11}{⽧、⼼}
  \begin{Phonetics}{病情}{bing4 qing2}[][HSK 6]
    \definition{s.}{estado de uma doença; condição do paciente; mudanças na doença}
  \end{Phonetics}
\end{Entry}

%%%%%%%%%% 症 %%%%%%%%%%
\subsection*{症}

\begin{Entry}{症}{10}{⽧}
  \begin{Phonetics}{症}{zheng1}
    \definition{s.}{doença; enfermidade | (figurativo) ponto de atrito | tumor abdominal | obstrução intestinal}
  \end{Phonetics}
  \begin{Phonetics}{症}{zheng4}
    \definition{s.}{doença; enfermidade}
  \end{Phonetics}
\end{Entry}

\begin{Entry}{症状}{10,7}{⽧、⽝}
  \begin{Phonetics}{症状}{zheng4zhuang4}[][HSK 6]
    \definition[种,些]{s.}{sintoma; estado anormal de um organismo devido a uma doença, como tosse, febre, etc.}
  \end{Phonetics}
\end{Entry}

%%%%%%%%%% 益 %%%%%%%%%%
\subsection*{益}

\begin{Entry}{益}{10}{⽫}
  \begin{Phonetics}{益}{yi4}
    \definition*{s.}{Sobrenome: Yi}
    \definition{adj.}{benéfico}
    \definition{adv.}{Literário: mais; cada vez mais}[空气污染问题日益严重。===O problema da poluição do ar está se tornando cada vez mais sério.]
    \definition{s.}{benefício; lucro; vantagem}
    \definition{v.}{aumentar}
  \end{Phonetics}
\end{Entry}

\begin{Entry}{益虫}{10,6}{⽫、⾍}
  \begin{Phonetics}{益虫}{yi4chong2}
    \definition{s.}{inseto benéfico (oposto a 害虫)}
  \seealsoref{害虫}{hai4chong2}
  \end{Phonetics}
\end{Entry}

%%%%%%%%%% 盏 %%%%%%%%%%
\subsection*{盏}

\begin{Entry}{盏}{10}{⽫}
  \begin{Phonetics}{盏}{zhan3}
    \definition{clas.}{usado para lâmpadas, iluminação}[一盏煤油灯。===Uma lamparina de querosene.]
    \definition{s.}{copo pequeno}
  \end{Phonetics}
\end{Entry}

%%%%%%%%%% 盐 %%%%%%%%%%
\subsection*{盐}

\begin{Entry}{盐}{10}{⽫}
  \begin{Phonetics}{盐}{yan2}[][HSK 4]
    \definition[袋,勺,把,包,粒]{s.}{sal (de cozinha) | Química: sal (produto formado pela neutralização de um ácido por uma base)}
  \end{Phonetics}
\end{Entry}

%%%%%%%%%% 监 %%%%%%%%%%
\subsection*{监}

\begin{Entry}{监}{10}{⽫}
  \begin{Phonetics}{监}{jian1}
    \definition{s.}{prisão; cadeia}
    \definition{v.}{supervisionar; inspecionar; observar}
  \end{Phonetics}
\end{Entry}

\begin{Entry}{监护}{10,7}{⽫、⼿}
  \begin{Phonetics}{监护}{jian1hu4}[][HSK 7-9]
    \definition{s.}{tutela}
    \definition{v.}{Lei: desempenhar as funções de tutor; agir como guardião; tutelar | Medicina: cuidar e zelar; vigiar}
  \end{Phonetics}
\end{Entry}

\begin{Entry}{监视}{10,8}{⽫、⾒}
  \begin{Phonetics}{监视}{jian1shi4}[][HSK 7-9]
    \definition{v.}{manter vigilância; ficar de olho em; observar atentamente}
  \end{Phonetics}
\end{Entry}

\begin{Entry}{监测}{10,9}{⽫、⽔}
  \begin{Phonetics}{监测}{jian1 ce4}[][HSK 6]
    \definition{v.}{monitorar; supervisionar e testar}
  \end{Phonetics}
\end{Entry}

\begin{Entry}{监狱}{10,9}{⽫、⽝}
  \begin{Phonetics}{监狱}{jian1yu4}[][HSK 7-9]
    \definition[个,所,座]{s.}{cadeia; prisão; instituições estatais responsáveis ​​pela aplicação de penas criminais; locais onde os presos são mantidos}
  \end{Phonetics}
\end{Entry}

\begin{Entry}{监控}{10,11}{⽫、⼿}
  \begin{Phonetics}{监控}{jian1kong4}[][HSK 7-9]
    \definition{s.}{monitor}
    \definition{v.}{monitorar}
  \end{Phonetics}
\end{Entry}

\begin{Entry}{监督}{10,13}{⽫、⽬}
  \begin{Phonetics}{监督}{jian1du1}[][HSK 6]
    \definition[个,位,名]{s.}{monitoramento; supervisão; pessoas que supervisionam}
    \definition{v.}{controlar; supervisionar; superintender; monitorar e supervisionar de perto}
  \end{Phonetics}
\end{Entry}

\begin{Entry}{监察}{10,14}{⽫、⼧}
  \begin{Phonetics}{监察}{jian1cha2}[][HSK 7-9]
    \definition{v.}{supervisionar; controlar}
  \end{Phonetics}
\end{Entry}

\begin{Entry}{监管}{10,14}{⽫、⽵}
  \begin{Phonetics}{监管}{jian1guan3}[][HSK 7-9]
    \definition{v.}{monitorar; supervisionar}
  \end{Phonetics}
\end{Entry}

%%%%%%%%%% 眞 %%%%%%%%%%
\subsection*{眞}

\begin{Entry}{眞}{10}{⽬}
  \begin{Phonetics}{眞}{zhen1}
    \variantof{真}
  \end{Phonetics}
\end{Entry}

%%%%%%%%%% 真 %%%%%%%%%%
\subsection*{真}

\begin{Entry}{真}{10}{⼗}
  \begin{Phonetics}{真}{zhen1}[][HSK 1]
    \definition*{s.}{Sobrenome: Zhen}
    \definition{adj.}{verdadeiro; real; genuíno (oposto de 假, 伪) | claro; inequívoco | genuíno; conforme os fatos objetivos (em oposição a 假 e 伪) | sincero}
    \definition{adv.}{realmente; verdadeiramente; de fato}
    \definition{s.}{escrita regular | retrato; imagem; cópia exata de algo | instintos naturais (ou caráter, disposição); natureza; qualidade inerente; origem | estado original; refere-se à forma original das coisas}
  \seealsoref{假}{jia4}
  \seealsoref{伪}{wei3}
  \end{Phonetics}
\end{Entry}

\begin{Entry}{真切}{10,4}{⼗、⼑}
  \begin{Phonetics}{真切}{zhen1qie4}
    \definition{adj.}{claro | distinto | honesto | sincero | vívido}
  \end{Phonetics}
\end{Entry}

\begin{Entry}{真心}{10,4}{⼗、⼼}
  \begin{Phonetics}{真心}{zhen1xin1}
    \definition{adj.}{sincero}
    \definition[片]{s.}{sinceridade}
  \end{Phonetics}
\end{Entry}

\begin{Entry}{真牛}{10,4}{⼗、⽜}
  \begin{Phonetics}{真牛}{zhen1niu2}
    \definition{adj.}{(gíria) muito legal, incrível}
  \end{Phonetics}
\end{Entry}

\begin{Entry}{真正}{10,5}{⼗、⽌}
  \begin{Phonetics}{真正}{zhen1zheng4}[][HSK 2]
    \definition{adj.}{verdadeiro; real; genuíno}
    \definition{adv.}{realmente; de fato; expressa afirmação de uma ação ou situação, equivalente a 确实}
  \seealsoref{确实}{que4shi2}
  \end{Phonetics}
\end{Entry}

\begin{Entry}{真声}{10,7}{⼗、⼠}
  \begin{Phonetics}{真声}{zhen1sheng1}
    \definition{s.}{voz modal; voz natural; voz verdadeira (oposto a 假声)}
  \seealsoref{假声}{jia3sheng1}
  \end{Phonetics}
\end{Entry}

\begin{Entry}{真实}{10,8}{⼗、⼧}
  \begin{Phonetics}{真实}{zhen1shi2}[][HSK 3]
    \definition{adj.}{verdadeiro; real; autêntico; de acordo com fatos objetivos}
  \end{Phonetics}
\end{Entry}

\begin{Entry}{真的}{10,8}{⼗、⽩}
  \begin{Phonetics}{真的}{zhen1 de5}[][HSK 1]
    \definition{adv.}{realmente; salientar que a situação existe realmente | verdadeiramente; realmente; existente na realidade; consistente com os fatos objetivos}
  \end{Phonetics}
\end{Entry}

\begin{Entry}{真诚}{10,8}{⼗、⾔}
  \begin{Phonetics}{真诚}{zhen1 cheng2}[][HSK 5]
    \definition{adj.}{verdadeiro; honesto; sério; sincero; genuíno; descreve uma pessoa que fala e age com sinceridade, de coração, fazendo com que os outros acreditem nela}
  \end{Phonetics}
\end{Entry}

\begin{Entry}{真相}{10,9}{⼗、⽬}
  \begin{Phonetics}{真相}{zhen1xiang4}[][HSK 5]
    \definition[个]{s.}{face; verdade; verdade nua e crua; a situação real; o estado real das coisas; a verdadeira situação}
  \end{Phonetics}
\end{Entry}

\begin{Entry}{真珠}{10,10}{⼗、⽟}
  \begin{Phonetics}{真珠}{zhen1zhu1}
    \variantof{珍珠}
  \end{Phonetics}
\end{Entry}

\begin{Entry}{真真}{10,10}{⼗、⼗}
  \begin{Phonetics}{真真}{zhen1zhen1}
    \definition{adv.}{genuinamente | realmente | escrupulosamente}
  \end{Phonetics}
\end{Entry}

\begin{Entry}{真理}{10,11}{⼗、⽟}
  \begin{Phonetics}{真理}{zhen1li3}[][HSK 5]
    \definition[条,个]{s.}{verdade; o reflexo correto das coisas objetivas e suas leis no cérebro humano}
  \end{Phonetics}
\end{Entry}

\begin{Entry}{真释}{10,12}{⼗、⾤}
  \begin{Phonetics}{真释}{zhen1shi4}
    \definition{s.}{razão genuína | explicação verdadeira}
  \end{Phonetics}
\end{Entry}

%%%%%%%%%% 破 %%%%%%%%%%
\subsection*{破}

\begin{Entry}{破}{10}{⽯}
  \begin{Phonetics}{破}{po4}[][HSK 3]
    \definition{adj.}{quebrado; danificado; rasgado; desgastado | insignificante; péssimo; medíocre}
    \definition{v.}{quebrar; danificar | dividir; cortar; separar | trocar (dinheiro) | livrar-se de; destruir; romper com | derrotar; capturar (uma cidade, etc.) | gastar dinheiro | revelar a verdade sobre; expor | mudar; romper; quebrar (regras, hábitos, ideias, etc.)}
  \end{Phonetics}
\end{Entry}

\begin{Entry}{破产}{10,6}{⽯、⼇}
  \begin{Phonetics}{破产}{po4/chan3}[][HSK 4]
    \definition{v.+compl.}{falir; ir à falência; tornar-se insolvente; entrar em liquidação; perder todo o patrimônio | falhar; fracassar; não dar em nada; figura de linguagem (geralmente com uma conotação depreciativa)}
  \end{Phonetics}
\end{Entry}

\begin{Entry}{破坏}{10,7}{⽯、⼟}
  \begin{Phonetics}{破坏}{po4huai4}[][HSK 3]
    \definition{v.}{demolir; naufragar; soçobrar; destruir; obliterar | quebrar; violar (um acordo, regulamento, etc.); não cumprir (disposições legais, regras, acordos, princípios, etc.) | prejudicar; perturbar; sabotar; causar grande dano; causar danos às coisas | reverter; mudar (um sistema social, costume, etc.) completamente ou violentamente | destruir; decompor; danificar o tecido ou a estrutura de um objeto}
  \end{Phonetics}
\end{Entry}

\begin{Entry}{破坏性}{10,7,8}{⽯、⼟、⼼}
  \begin{Phonetics}{破坏性}{po4huai4xing4}
    \definition{adj.}{destrutivo}
    \definition{s.}{poder destrutivo}
  \end{Phonetics}
\end{Entry}

%%%%%%%%%% 砸 %%%%%%%%%%
\subsection*{砸}

\begin{Entry}{砸}{10}{⽯}
  \begin{Phonetics}{砸}{za2}
    \definition{v.}{esmagar | bater | falhar | estragar}
  \end{Phonetics}
\end{Entry}

%%%%%%%%%% 离 %%%%%%%%%%
\subsection*{离}

\begin{Entry}{离}{10}{⼇}
  \begin{Phonetics}{离}{li2}[][HSK 2]
    \definition*{s.}{Um dos Oito Diagramas | Sobrenome: Li}
    \definition{prep.}{(ser longe) de\dots até\dots}
    \definition{v.}{partir; separar-se; afastar-se; estar longe de | prescindir; dispensar; ser independente de | mudar de; desviar-se de | mudar de; desviar-se de; trair; ser incompatível}
  \end{Phonetics}
\end{Entry}

\begin{Entry}{离不开}{10,4,4}{⼇、⼀、⼶}
  \begin{Phonetics}{离不开}{li2 bu4 kai1}[][HSK 4]
    \definition{v.}{não pode prescindir; ser inseparável de; não ser capaz de se separar ou deixar uma pessoa, coisa ou circunstância}
  \end{Phonetics}
\end{Entry}

\begin{Entry}{离开}{10,4}{⼇、⼶}
  \begin{Phonetics}{离开}{li2kai1}[][HSK 2]
    \definition{v.}{deixar; partir; desviar-se; separar-se das pessoas, dos lugares e das coisas}
  \end{Phonetics}
\end{Entry}

\begin{Entry}{离奇}{10,8}{⼇、⼤}
  \begin{Phonetics}{离奇}{li2qi2}[][HSK 7-9]
    \definition{adj.}{estranho; esquisito; fantástico; bizarro}
  \end{Phonetics}
\end{Entry}

\begin{Entry}{离婚}{10,11}{⼇、⼥}
  \begin{Phonetics}{离婚}{li2/hun1}[][HSK 3]
    \definition{v.+compl.}{divórciar; romper um casamento; obter o divórcio}
  \end{Phonetics}
\end{Entry}

\begin{Entry}{离职}{10,11}{⼇、⽿}
  \begin{Phonetics}{离职}{li2/zhi2}[][HSK 7-9]
    \definition{v.+compl.}{deixar o emprego temporariamente | demitir-se; deixar o cargo; abandonar o emprego}
  \end{Phonetics}
\end{Entry}

\begin{Entry}{离谱儿}{10,14,2}{⼇、⾔、⼉}
  \begin{Phonetics}{离谱儿}{li2/pu3r5}[][HSK 7-9]
    \definition{v.+compl.}{ir além do que é apropriado; estar fora de lugar; exagerar; estar muito longe do que é normal; falar ou agir de uma forma que não esteja em conformidade com os padrões geralmente aceitos}
  \end{Phonetics}
\end{Entry}

%%%%%%%%%% 秘 %%%%%%%%%%
\subsection*{秘}

\begin{Entry}{秘}{10}{⽲}
  \begin{Phonetics}{秘}{bi4}
    \definition*{s.}{Abreviação de Peru, 秘鲁 | Sobrenome: Bi}
  \seealsoref{秘鲁}{bi4lu3}
  \end{Phonetics}
  \begin{Phonetics}{秘}{mi4}
    \definition{adj.}{secreto; misterioso | raro; raramente visto; estranho}
    \definition{adv.}{secretamente; privadamente}
    \definition{s.}{secretário}
    \definition{v.}{manter algo em segredo; esconder algo; guardar segredos | bloquear; obstruir; ter dificuldade para defecar}
  \end{Phonetics}
\end{Entry}

\begin{Entry}{秘书}{10,4}{⽲、⼄}
  \begin{Phonetics}{秘书}{mi4shu1}[][HSK 4]
    \definition[个,位,名]{s.}{o cargo de secretário; funções de secretariado | secretário; pessoas encarregadas da correspondência e que auxiliam o chefe do órgão ou departamento na condução diária de seu trabalho}
  \end{Phonetics}
\end{Entry}

\begin{Entry}{秘书长}{10,4,4}{⽲、⼄、⾧}
  \begin{Phonetics}{秘书长}{mi4 shu1 zhang3}[][HSK 6]
    \definition{s.}{secretário-geral}
  \end{Phonetics}
\end{Entry}

\begin{Entry}{秘密}{10,11}{⽲、⼧}
  \begin{Phonetics}{秘密}{mi4mi4}[][HSK 4]
    \definition{adj.}{secreto}
    \definition[个,条,些]{s.}{segredo; algo secreto; coisas que você não quer que as pessoas saibam}
  \end{Phonetics}
\end{Entry}

\begin{Entry}{秘鲁}{10,12}{⽲、⿂}
  \begin{Phonetics}{秘鲁}{bi4lu3}
    \definition*{s.}{Peru}
  \end{Phonetics}
\end{Entry}

%%%%%%%%%% 租 %%%%%%%%%%
\subsection*{租}

\begin{Entry}{租}{10}{⽲}
  \begin{Phonetics}{租}{zu1}[][HSK 2]
    \definition{s.}{aluguel | imposto sobre a terra; tributação; (antigo) refere-se ao imposto predial}
    \definition{v.}{contratar; alugar; fretar | alugar; arrendar}
  \end{Phonetics}
\end{Entry}

\begin{Entry}{租用}{10,5}{⽲、⽤}
  \begin{Phonetics}{租用}{zu1yong4}
    \definition{v.}{contratar | alugar | alugar (algo de alguém)}
  \end{Phonetics}
\end{Entry}

\begin{Entry}{租让}{10,5}{⽲、⾔}
  \begin{Phonetics}{租让}{zu1rang4}
    \definition{v.}{alugar | alugar (a propriedade de alguém para outra pessoa)}
  \end{Phonetics}
\end{Entry}

\begin{Entry}{租约}{10,6}{⽲、⽷}
  \begin{Phonetics}{租约}{zu1yue1}
    \definition{s.}{aluguel}
  \end{Phonetics}
\end{Entry}

\begin{Entry}{租房}{10,8}{⽲、⼾}
  \begin{Phonetics}{租房}{zu1fang2}
    \definition{v.}{alugar um apartamento}
  \end{Phonetics}
\end{Entry}

\begin{Entry}{租金}{10,8}{⽲、⾦}
  \begin{Phonetics}{租金}{zu1 jin1}[][HSK 6]
    \definition[笔]{s.}{aluguel; aluguer; o custo do aluguel de terras, casas ou itens; a renda do aluguel de terras, casas ou itens}
  \seealsoref{租钱}{zu1qian5}
  \end{Phonetics}
\end{Entry}

\begin{Entry}{租赁}{10,10}{⽲、⾙}
  \begin{Phonetics}{租赁}{zu1lin4}
    \definition{v.}{contratar | alugar}
  \end{Phonetics}
\end{Entry}

\begin{Entry}{租钱}{10,10}{⽲、⾦}
  \begin{Phonetics}{租钱}{zu1qian5}
    \definition{s.}{aluguel}
  \seealsoref{租金}{zu1 jin1}
  \end{Phonetics}
\end{Entry}

\begin{Entry}{租船}{10,11}{⽲、⾈}
  \begin{Phonetics}{租船}{zu1chuan2}
    \definition{v.}{fretar um navio | alugar um navio}
  \end{Phonetics}
\end{Entry}

%%%%%%%%%% 秤 %%%%%%%%%%
\subsection*{秤}

\begin{Entry}{秤}{10}{⽲}
  \begin{Phonetics}{秤}{cheng4}[][HSK 7-9]
    \definition[把,杆,台]{s.}{balança; balança romana; um instrumento para medir o peso de um objeto}
  \end{Phonetics}
\end{Entry}

%%%%%%%%%% 积 %%%%%%%%%%
\subsection*{积}

\begin{Entry}{积}{10}{⽲}
  \begin{Phonetics}{积}{ji1}[][HSK 7-9]
    \definition{adj.}{de longa data; pendente há muito tempo | antiquíssimo; acumulado ao longo de um longo período de tempo}
    \definition{s.}{(medicina chinesa) indigestão (em bebês e crianças) | (matemática)  abreviação de produto, 乘积}
    \definition{v.}{acumular; juntar; amontoar; reunir; coletar}
  \seealsoref{乘积}{cheng2ji1}
  \end{Phonetics}
\end{Entry}

\begin{Entry}{积木}{10,4}{⽲、⽊}
  \begin{Phonetics}{积木}{ji1mu4}
    \definition{s.}{blocos de montar (brinquedo)}
  \end{Phonetics}
\end{Entry}

\begin{Entry}{积极}{10,7}{⽲、⽊}
  \begin{Phonetics}{积极}{ji1ji2}[][HSK 3]
    \definition{adj.}{ativo; descreve uma atitude proativa e esforçada | positivo; que tem um efeito positivo e ajuda no desenvolvimento das coisas}
  \end{Phonetics}
\end{Entry}

\begin{Entry}{积淀}{10,11}{⽲、⽔}
  \begin{Phonetics}{积淀}{ji1dian4}[][HSK 7-9]
    \definition{s.}{acumulação | depósitos acumulados ao longo de longos períodos | Figurativo: experiência valiosa, sabedoria acumulada}
    \definition{v.}{acumular}
  \end{Phonetics}
\end{Entry}

\begin{Entry}{积累}{10,11}{⽲、⽷}
  \begin{Phonetics}{积累}{ji1lei3}[][HSK 4]
    \definition{s.}{acúmulo; acumulação}
    \definition{v.}{acumular}
  \end{Phonetics}
\end{Entry}

\begin{Entry}{积蓄}{10,13}{⽲、⾋}
  \begin{Phonetics}{积蓄}{ji1xu4}[][HSK 7-9]
    \definition[笔]{s.}{poupança; economias}
    \definition{v.}{acumular; poupar; economizar}
  \end{Phonetics}
\end{Entry}

%%%%%%%%%% 称 %%%%%%%%%%
\subsection*{称}

\begin{Entry}{称}{10}{⽲}
  \begin{Phonetics}{称}{chen4}
    \definition{adj.}{ajustado; encaixado; adequado}
    \definition{v.}{ajustar; adequar; combinar; estar em conformidade com; ser adequado para | ter; possuir}
  \end{Phonetics}
  \begin{Phonetics}{称}{cheng1}[][HSK 2,5]
    \definition*{s.}{Sobrenome: Cheng}
    \definition{s.}{nome}
    \definition{v.}{chamar; ser chamado | dizer; declarar | elogiar; louvar; expressar afirmação ou elogio a pessoas ou coisas por meio de palavras | pesar; medir o peso | elevar; levantar; erguer | aplaudir; concordar; expressar suas opiniões ou sentimentos por meio de palavras ou ações | declarar-se como; declarar que é; reivindicar ser alguém em virtude do próprio poder}
  \end{Phonetics}
\end{Entry}

\begin{Entry}{称为}{10,4}{⽲、⼂}
  \begin{Phonetics}{称为}{cheng1 wei2}[][HSK 3]
    \definition{v.}{ser chamado de; ser conhecido como; denominar}
  \end{Phonetics}
\end{Entry}

\begin{Entry}{称号}{10,5}{⽲、⼝}
  \begin{Phonetics}{称号}{cheng1hao4}[][HSK 5]
    \definition{s.}{título; nome; designação; nome dado a alguém, a uma organização ou a alguma coisa (geralmente usado de forma honrosa)}
  \end{Phonetics}
\end{Entry}

\begin{Entry}{称作}{10,7}{⽲、⼈}
  \begin{Phonetics}{称作}{cheng1zuo4}[][HSK 7-9]
    \definition{v.}{ser chamado | ser conhecido como}
  \end{Phonetics}
\end{Entry}

\begin{Entry}{称呼}{10,8}{⽲、⼝}
  \begin{Phonetics}{称呼}{cheng1hu5}[][HSK 7-9]
    \definition{v.}{chamar; dirigir-se (a alguém)}[我称呼他为老师。===Eu o chamo de professor.]
  \end{Phonetics}
\end{Entry}

\begin{Entry}{称赞}{10,16}{⽲、⾙}
  \begin{Phonetics}{称赞}{cheng1zan4}[][HSK 4]
    \definition[句,声,番,次]{s.}{elogio; aclamação; louvor; avaliação positiva de um desempenho ou conquista}
    \definition{v.}{elogiar; aclamar; louvar; usar palavras para expressar um carinho pelas virtudes de uma pessoa ou coisa}
  \end{Phonetics}
\end{Entry}

%%%%%%%%%% 窄 %%%%%%%%%%
\subsection*{窄}

\begin{Entry}{窄}{10}{⽳}
  \begin{Phonetics}{窄}{zhai3}
    \definition{adj.}{estreito; pequena distância horizontal | mesquinho; estreito; (mente) não alegre; (capacidade) pequena | difícil; mal; falta de; (vida) não bem de vida}
  \end{Phonetics}
\end{Entry}

%%%%%%%%%% 站 %%%%%%%%%%
\subsection*{站}

\begin{Entry}{站}{10}{⽴}
  \begin{Phonetics}{站}{zhan4}[][HSK 1,2]
    \definition*{s.}{Sobrenome: Zhan}
    \definition{s.}{parada; estação; ponto de parada | central; estação; instituição criada para um determinado tipo de atividade | filial de uma empresa ou organização; local de trabalho criado para realizar uma determinada tarefa | \emph{website}; na rede de computadores, refere-se a um \emph{site}}
    \definition{v.}{ficar em pé; estar em pé | parar; interromper; fazer uma pausa}
  \end{Phonetics}
\end{Entry}

\begin{Entry}{站长}{10,4}{⽴、⾧}
  \begin{Phonetics}{站长}{zhan4zhang3}
    \definition{s.}{pessoa responsável pela estação de trem | chefe da estação | \emph{webmaster} | gerente de centro de voluntariado}
  \end{Phonetics}
\end{Entry}

\begin{Entry}{站台}{10,5}{⽴、⼝}
  \begin{Phonetics}{站台}{zhan4 tai2}[][HSK 6]
    \definition{s.}{plataforma (em uma estação ferroviária)}
  \end{Phonetics}
\end{Entry}

\begin{Entry}{站住}{10,7}{⽴、⼈}
  \begin{Phonetics}{站住}{zhan4 zhu4}[][HSK 2]
    \definition{v.}{parar; deter; parar enquanto se move | ficar firme nos pés; manter os pés; permanecer firme | manter-se firme; consolidar a posição de alguém; estabelecer-se em uma determinada unidade ou lugar | sustentar a opinião}
  \end{Phonetics}
\end{Entry}

\begin{Entry}{站姿}{10,9}{⽴、⼥}
  \begin{Phonetics}{站姿}{zhan4zi1}
    \definition{s.}{postura}
  \end{Phonetics}
\end{Entry}

\begin{Entry}{站点}{10,9}{⽴、⽕}
  \begin{Phonetics}{站点}{zhan4dian3}
    \definition{s.}{\emph{website}}
  \end{Phonetics}
\end{Entry}

%%%%%%%%%% 竞 %%%%%%%%%%
\subsection*{竞}

\begin{Entry}{竞}{10}{⽴}
  \begin{Phonetics}{竞}{jing4}
    \definition{adj.}{forte; poderoso}
    \definition{v.}{competir; contender; disputar | contestar}
  \end{Phonetics}
\end{Entry}

\begin{Entry}{竞争}{10,6}{⽴、⼑}
  \begin{Phonetics}{竞争}{jing4zheng1}[][HSK 5]
    \definition{v.}{competir; disputar; lutar; entre duas ou mais partes; em prol de seus próprios interesses; lutar pela vitória por meio de uma disputa de sua própria força contra outra}
  \end{Phonetics}
\end{Entry}

\begin{Entry}{竞技}{10,7}{⽴、⼿}
  \begin{Phonetics}{竞技}{jing4ji4}[][HSK 7-9]
    \definition{s.}{atletismo; provas de atletismo; esportes; pista e campo}
    \definition{v.}{competir; desafiar; geralmente referindo-se a competições atléticas}
  \end{Phonetics}
\end{Entry}

\begin{Entry}{竞相}{10,9}{⽴、⽬}
  \begin{Phonetics}{竞相}{jing4xiang1}[][HSK 7-9]
    \definition{adv.}{ansiosamente}
    \definition{s.}{competição}
    \definition{v.}{competir; disputar}
  \end{Phonetics}
\end{Entry}

\begin{Entry}{竞选}{10,9}{⽴、⾡}
  \begin{Phonetics}{竞选}{jing4xuan3}[][HSK 7-9]
    \definition{s.}{eleição; campanha eleitoral}
    \definition{v.}{participar de uma disputa eleitoral; fazer campanha para (um cargo); candidatar-se a}
  \end{Phonetics}
\end{Entry}

\begin{Entry}{竞赛}{10,14}{⽴、⾙}
  \begin{Phonetics}{竞赛}{jing4sai4}[][HSK 5]
    \definition[个]{s.}{concurso; competição; partida; corrida}
    \definition{v.}{correr; competir; competir uns com os outros por superioridade; em esportes, produção e outras atividades, para comparar competência, habilidade etc., usado principalmente na linguagem falada}
  \end{Phonetics}
\end{Entry}

%%%%%%%%%% 笋 %%%%%%%%%%
\subsection*{笋}

\begin{Entry}{笋}{10}{⽵}
  \begin{Phonetics}{笋}{sun3}
    \definition{s.}{broto de bambu}
  \end{Phonetics}
\end{Entry}

%%%%%%%%%% 笑 %%%%%%%%%%
\subsection*{笑}

\begin{Entry}{笑}{10}{⽵}
  \begin{Phonetics}{笑}{xiao4}[][HSK 1]
    \definition{adj.}{ridículo; engraçado; risível; hilário}
    \definition{v.}{sorrir; rir; mostrar expressão de alegria; emitir sons de alegria | ridicularizar; rir de; zombar}
  \end{Phonetics}
\end{Entry}

\begin{Entry}{笑声}{10,7}{⽵、⼠}
  \begin{Phonetics}{笑声}{xiao4 sheng1}[][HSK 6]
    \definition{s.}{riso; risada}
  \end{Phonetics}
\end{Entry}

\begin{Entry}{笑话}{10,8}{⽵、⾔}
  \begin{Phonetics}{笑话}{xiao4hua5}[][HSK 2]
    \definition[个]{s.}{piada; brincadeira; uma conversa ou história que faz as pessoas rirem; algo que as pessoas usam como piada}
    \definition{v.}{ridicularizar; zombar; rir de;}
  \end{Phonetics}
\end{Entry}

\begin{Entry}{笑话儿}{10,8,2}{⽵、⾔、⼉}
  \begin{Phonetics}{笑话儿}{xiao4 hua4r5}[][HSK 2]
    \definition{s.}{piada; brincadeira; gracejo}
  \end{Phonetics}
\end{Entry}

\begin{Entry}{笑容}{10,10}{⽵、⼧}
  \begin{Phonetics}{笑容}{xiao4 rong2}[][HSK 6]
    \definition[丝,抹,个]{s.}{sorriso; expressão sorridente; o olhar no rosto de alguém ao sorrir}
  \end{Phonetics}
\end{Entry}

\begin{Entry}{笑脸}{10,11}{⽵、⾁}
  \begin{Phonetics}{笑脸}{xiao4 lian3}[][HSK 6]
    \definition{s.}{\emph{smiley}; rosto sorridente (emoji)}
  \end{Phonetics}
\end{Entry}

%%%%%%%%%% 笔 %%%%%%%%%%
\subsection*{笔}

\begin{Entry}{笔}{10}{⽵}
  \begin{Phonetics}{笔}{bi3}[][HSK 2]
    \definition{clas.}{usado para grandes quantias de dinheiro, compras, negócios, propriedades, etc. | usado em caligrafia e pintura, etc.}
    \definition[支,枝]{s.}{caneta; lápis; pincel para escrever; ferramentas para escrever ou desenhar | técnica de escrita; caligrafia ou desenho | traço}
    \definition{v.}{escrever à mão}
  \end{Phonetics}
\end{Entry}

\begin{Entry}{笔记}{10,5}{⽵、⾔}
  \begin{Phonetics}{笔记}{bi3 ji4}[][HSK 2]
    \definition[篇,本,个]{s.}{notas; anotações feitas durante aulas, palestras e leituras | ensaios; esboços}
    \definition{v.}{tomar nota (por escrito)}
  \end{Phonetics}
\end{Entry}

\begin{Entry}{笔记本}{10,5,5}{⽵、⾔、⽊}
  \begin{Phonetics}{笔记本}{bi3ji4ben3}[][HSK 2]
    \definition[个,本]{s.}{caderno para anotações | \emph{laptop}; refere-se a um computador portátil}
    \definition{s.}{\emph{laptop}}
  \end{Phonetics}
\end{Entry}

\begin{Entry}{笔试}{10,8}{⽵、⾔}
  \begin{Phonetics}{笔试}{bi3 shi4}[][HSK 6]
    \definition{s.}{exame escrito; um tipo de exame que exige respostas escritas; diferente de 口试}
  \seealsoref{口试}{kou3 shi4}
  \end{Phonetics}
\end{Entry}

%%%%%%%%%% 粉 %%%%%%%%%%
\subsection*{粉}

\begin{Entry}{粉}{10}{⽶}
  \begin{Phonetics}{粉}{fen3}[][HSK 7-9]
    \definition{adj.}{branco | rosa}
    \definition{s.}{pó | cosméticos em pó | farinha de trigo | macarrão ou outro alimento feito de feijão, arroz, batata, amido de batata-doce, etc. | macarrão de arroz}
    \definition{v.}{virar pó | Dialeto: caiar}
  \end{Phonetics}
\end{Entry}

\begin{Entry}{粉丝}{10,5}{⽶、⼀}
  \begin{Phonetics}{粉丝}{fen3si1}[][HSK 7-9]
    \definition{s.}{(empréstimo linguístico) fã | entusiasta de alguém ou alguma coisa}
    \definition[个,群,位,名,些,批]{s.}{aletria de amido de feijão ou batata; aletria chinesa; macarrão de celofane ou macarrão de vidro (transparente) | Empréstimo linguístico: fã; refere-se a uma pessoa que é obcecada ou adora uma celebridade}
  \end{Phonetics}
\end{Entry}

\begin{Entry}{粉色}{10,6}{⽶、⾊}
  \begin{Phonetics}{粉色}{fen3 se4}
    \definition{s.}{cor-de-rosa}
  \end{Phonetics}
\end{Entry}

\begin{Entry}{粉碎}{10,13}{⽶、⽯}
  \begin{Phonetics}{粉碎}{fen3sui4}[][HSK 7-9]
    \definition{adj.}{pulverizado; quebrado em pedaços; descreve algo que está muito quebrado, quebrado em partículas muito pequenas}
    \definition{v.}{esmagar; transformar as coisas em partículas muito pequenas | esmagar; quebrar; estilhaçar; fazer com que a outra parte falhe ou seja completamente destruída}
  \end{Phonetics}
\end{Entry}

%%%%%%%%%% 素 %%%%%%%%%%
\subsection*{素}

\begin{Entry}{素}{10}{⽷}
  \begin{Phonetics}{素}{su4}
    \definition{adj.}{branco; de cor natural | simples; natural; singelo; de cor simples | nativo; original | normal; usual; geral}
    \definition{adv.}{geralmente; sempre; habitualmente}
    \definition{s.}{vegetais, frutas e outros alimentos (em oposição à 荤) | matéria-prima; matéria-prima básico; tecidos de seda naturais e não processados | elemento; os componentes básicos de algo}
  \seealsoref{荤}{hun1}
  \end{Phonetics}
\end{Entry}

\begin{Entry}{素质}{10,8}{⽷、⾙}
  \begin{Phonetics}{素质}{su4zhi4}[][HSK 6]
    \definition[个,种]{s.}{qualidade; características; caráter; o nível físico, moral, mental, intelectual e cultural de uma pessoa}
  \end{Phonetics}
\end{Entry}

%%%%%%%%%% 索 %%%%%%%%%%
\subsection*{索}

\begin{Entry}{索}{10}{⽷}
  \begin{Phonetics}{索}{suo3}
    \definition*{s.}{Sobrenome: Suo}
    \definition{adj.}{completamente sozinho; sozinho | maçante; insípido; sem significado}
    \definition[根]{s.}{corda; cabo; cordão; corrente | uma corda grande}
    \definition{v.}{(literário) pesquisar | exigir; pedir}
  \end{Phonetics}
\end{Entry}

\begin{Entry}{索性}{10,8}{⽷、⼼}
  \begin{Phonetics}{索性}{suo3xing4}
    \definition{adv.}{poderia muito bem | simplesmente | apenas}
  \end{Phonetics}
\end{Entry}

%%%%%%%%%% 紧 %%%%%%%%%%
\subsection*{紧}

\begin{Entry}{紧}{10}{⽷}
  \begin{Phonetics}{紧}{jin3}[][HSK 3]
    \definition{adj.}{tenso; apertado; o estado em que um objeto se encontra após ser submetido a uma grande força de tração ou pressão.| seguro; firme | cerrado; apertado | urgente; premente; tenso | rigoroso; rígido; severo | difícil; sem dinheiro}
    \definition{v.}{apertar; tornar mais apertado}
  \end{Phonetics}
\end{Entry}

\begin{Entry}{紧张}{10,7}{⽷、⼸}
  \begin{Phonetics}{紧张}{jin3zhang1}[][HSK 3]
    \definition{adj.}{nervoso; tenso; mentalmente em estado de alerta, excitado e inquieto | apertado; em falta; o que está disponível não satisfaz os requisitos| tenso; intenso; intenso ou urgente, causando tensão mental}
  \end{Phonetics}
\end{Entry}

\begin{Entry}{紧迫}{10,8}{⽷、⾡}
  \begin{Phonetics}{紧迫}{jin3po4}[][HSK 7-9]
    \definition{adj.}{urgente; premente; iminente; sem margem para manobras}
  \end{Phonetics}
\end{Entry}

\begin{Entry}{紧急}{10,9}{⽷、⼼}
  \begin{Phonetics}{紧急}{jin3ji2}[][HSK 3]
    \definition{adj./adj.}{urgente; premente; crítico}
  \end{Phonetics}
\end{Entry}

\begin{Entry}{紧紧}{10,10}{⽷、⽷}
  \begin{Phonetics}{紧紧}{jin3 jin3}[][HSK 5]
    \definition{adv.}{firmemente; estreitamente; apertadamente; prestar muita atenção (em algo)}
  \end{Phonetics}
\end{Entry}

\begin{Entry}{紧缺}{10,10}{⽷、⽸}
  \begin{Phonetics}{紧缺}{jin3que1}[][HSK 7-9]
    \definition{adj.}{em falta; extremamente necessário | escasso}
  \end{Phonetics}
\end{Entry}

\begin{Entry}{紧凑}{10,11}{⽷、⼎}
  \begin{Phonetics}{紧凑}{jin3cou4}[][HSK 7-9]
    \definition{adj.}{compacto; conciso; bem estruturado; rígido; sucinto}
  \end{Phonetics}
\end{Entry}

\begin{Entry}{紧密}{10,11}{⽷、⼧}
  \begin{Phonetics}{紧密}{jin3 mi4}[][HSK 4]
    \definition{adj.}{próximos; inseparáveis | incessante; rápido e intenso}
  \end{Phonetics}
\end{Entry}

\begin{Entry}{紧接着}{10,11,11}{⽷、⼿、⽬}
  \begin{Phonetics}{紧接着}{jin3 jie1zhe5}[][HSK 7-9]
    \definition{expr.}{imediatamente depois; uma coisa aconteceu após a outra}
  \end{Phonetics}
\end{Entry}

\begin{Entry}{紧缩}{10,14}{⽷、⽷}
  \begin{Phonetics}{紧缩}{jin3suo1}[][HSK 7-9]
    \definition{v.}{reduzir; cortar; desmantelar; encolher}
  \end{Phonetics}
\end{Entry}

%%%%%%%%%% 绣 %%%%%%%%%%
\subsection*{绣}

\begin{Entry}{绣}{10}{⽷}
  \begin{Phonetics}{绣}{xiu4}
    \definition{s.}{bordado}
    \definition{v.}{bordar}
  \end{Phonetics}
\end{Entry}

%%%%%%%%%% 继 %%%%%%%%%%
\subsection*{继}

\begin{Entry}{继}{10}{⽷}
  \begin{Phonetics}{继}{ji4}[][HSK 7-9]
    \definition{adv.}{então; depois}
    \definition{s.}{filhos; prole}
    \definition{v.}{continuar; ter sucesso; seguir}
  \end{Phonetics}
\end{Entry}

\begin{Entry}{继父}{10,4}{⽷、⽗}
  \begin{Phonetics}{继父}{ji4fu4}[][HSK 7-9]
    \definition{s.}{padrasto}
  \end{Phonetics}
\end{Entry}

\begin{Entry}{继母}{10,5}{⽷、⽏}
  \begin{Phonetics}{继母}{ji4mu3}[][HSK 7-9]
    \definition{s.}{madrasta}
  \end{Phonetics}
\end{Entry}

\begin{Entry}{继而}{10,6}{⽷、⽽}
  \begin{Phonetics}{继而}{ji4'er2}[][HSK 7-9]
    \definition{adv.}{então; depois; mais tarde}
  \end{Phonetics}
\end{Entry}

\begin{Entry}{继承}{10,8}{⽷、⼿}
  \begin{Phonetics}{继承}{ji4cheng2}[][HSK 5]
    \definition{v.}{herdar (o patrimônio de uma pessoa falecida, etc.) de acordo com a lei | continuar; geralmente se refere à aceitação do estilo, da cultura, do conhecimento, etc., daqueles que nos precederam | continuar; os descendentes continuam o trabalho deixado por seus antecessores.}
  \end{Phonetics}
\end{Entry}

\begin{Entry}{继续}{10,11}{⽷、⽷}
  \begin{Phonetics}{继续}{ji4xu4}[][HSK 3]
    \definition{s.}{continuação}
    \definition{v.}{continuar; prosseguir | prosseguir; continuar; seguir em frente (com); (atividades, eventos, etc.) continuar após uma pausa ou um determinado período de tempo}
  \end{Phonetics}
\end{Entry}

%%%%%%%%%% 缺 %%%%%%%%%%
\subsection*{缺}

\begin{Entry}{缺}{10}{⽸}
  \begin{Phonetics}{缺}{que1}[][HSK 3]
    \definition{adj.}{incompleto; imperfeito}
    \definition[种]{s.}{vaga; abertura; falta}
    \definition{v.}{estar com falta de; faltar | estar ausente}
  \end{Phonetics}
\end{Entry}

\begin{Entry}{缺乏}{10,4}{⽸、⼃}
  \begin{Phonetics}{缺乏}{que1fa2}[][HSK 5]
    \definition{v.}{faltar; estar em falta de; não ter ou não ter totalmente (algo que deveria possuir ou é desejaria possuir)}
  \end{Phonetics}
\end{Entry}

\begin{Entry}{缺少}{10,4}{⽸、⼩}
  \begin{Phonetics}{缺少}{que1shao3}[][HSK 3]
    \definition{v.}{falta; estar com falta de; estar em falta de; geralmente se refere à falta de pessoas ou coisas}
  \end{Phonetics}
\end{Entry}

\begin{Entry}{缺点}{10,9}{⽸、⽕}
  \begin{Phonetics}{缺点}{que1dian3}[][HSK 3]
    \definition[个,些]{s.}{desvantagem; deficiência; inconveniência; ponto fraco; uma deficiência ou imperfeição (em oposição a 优点)}
  \seealsoref{优点}{you1dian3}
  \end{Phonetics}
\end{Entry}

\begin{Entry}{缺陷}{10,10}{⽸、⾩}
  \begin{Phonetics}{缺陷}{que1xian4}[][HSK 6]
    \definition[个,处,项]{pron.}{defeito; falha; inconveniência; mancha; um lugar onde uma pessoa ou coisa está incompleta ou tem falhas porque algo está faltando}
  \end{Phonetics}
\end{Entry}

\begin{Entry}{缺勤}{10,13}{⽸、⼒}
  \begin{Phonetics}{缺勤}{que1/qin2}
    \definition{v.+compl.}{ausentar-se do dever (trabalho)}
  \end{Phonetics}
\end{Entry}

%%%%%%%%%% 罢 %%%%%%%%%%
\subsection*{罢}

\begin{Entry}{罢}{10}{⽹}
  \begin{Phonetics}{罢}{ba4}
    \definition{v.}{parar; cessar | revogar; destituir; encerrar | terminar | abandonar uma ideia; esqueçer sobre algo; deixar estar (passar)}
  \end{Phonetics}
  \begin{Phonetics}{罢}{ba5}
    \definition{part.}{partícula final, a mesma que 吧}
  \seealsoref{吧}{ba5}
  \end{Phonetics}
\end{Entry}

\begin{Entry}{罢了}{10,2}{⽹、⼅}
  \begin{Phonetics}{罢了}{ba4 le5}[][HSK 6]
    \definition{part.}{usado no final de uma frase, significa 仅此而已, geralmente seguido de 无非, 不过, 只是}
  \seealsoref{不过}{bu2guo4}
  \seealsoref{仅此而已}{jin3ci3'er2yi3}
  \seealsoref{无非}{wu2fei1}
  \seealsoref{只是}{zhi3 shi4}
  \end{Phonetics}
  \begin{Phonetics}{罢了}{ba4 liao3}
    \definition{part.}{uma partícula modal indicando (não se preocupe, ok)}
  \end{Phonetics}
\end{Entry}

\begin{Entry}{罢工}{10,3}{⽹、⼯}
  \begin{Phonetics}{罢工}{ba4gong1}[][HSK 6]
    \definition{v.}{parar de trabalhar; entrar em greve; abandonar o emprego}
  \end{Phonetics}
\end{Entry}

\begin{Entry}{罢休}{10,6}{⽹、⼈}
  \begin{Phonetics}{罢休}{ba4xiu1}[][HSK 7-9]
    \definition{v.}{parar; desistir; deixar o assunto de lado; parar de fazer algo, enfatizando a determinação de parar de fazê-lo}
  \end{Phonetics}
\end{Entry}

\begin{Entry}{罢免}{10,7}{⽹、⼉}
  \begin{Phonetics}{罢免}{ba4mian3}[][HSK 7-9]
    \definition{v.}{destituir; remover do cargo; demitir alguém do seu posto | destituir do cargo uma pessoa eleita pelo eleitorado ou por um órgão representativo}
  \end{Phonetics}
\end{Entry}

%%%%%%%%%% 翅 %%%%%%%%%%
\subsection*{翅}

\begin{Entry}{翅}{10}{⽻}
  \begin{Phonetics}{翅}{chi4}
    \definition[只]{s.}{asa | barbatana de tubarão | coisa parecida com uma asa}
  \end{Phonetics}
\end{Entry}

\begin{Entry}{翅膀}{10,14}{⽻、⾁}
  \begin{Phonetics}{翅膀}{chi4bang3}[][HSK 7-9]
    \definition[只,个,对]{s.}{asa; os órgãos de voo de animais como pássaros e insetos geralmente aparecem em pares | barbatana; aba; lâmina; a parte de algo que tem o formato ou age como uma asa}
  \end{Phonetics}
\end{Entry}

%%%%%%%%%% 耕 %%%%%%%%%%
\subsection*{耕}

\begin{Entry}{耕}{10}{⽾}
  \begin{Phonetics}{耕}{geng1}
    \definition{v.}{arar; cultivar | trabalhar; fazer | ganhar a vida}
  \end{Phonetics}
\end{Entry}

\begin{Entry}{耕地}{10,6}{⽾、⼟}
  \begin{Phonetics}{耕地}{geng1/di4}[][HSK 7-9]
    \definition[块,公顷]{s.}{terra cultivada; terra para cultivo}
    \definition{v.+compl.}{lavrar; arar}
  \end{Phonetics}
\end{Entry}

%%%%%%%%%% 耗 %%%%%%%%%%
\subsection*{耗}

\begin{Entry}{耗}{10}{⽾}
  \begin{Phonetics}{耗}{hao4}[][HSK 7-9]
    \definition{s.}{más notícias}[听到噩耗,他碎心裂胆。===Ele ficou arrasado ao ouvir as más notícias.]
    \definition{v.}{consumir; custar | perder tempo; procrastinar}
  \end{Phonetics}
\end{Entry}

\begin{Entry}{耗时}{10,7}{⽾、⽇}
  \begin{Phonetics}{耗时}{hao4shi2}[][HSK 7-9]
    \definition{adj.}{demorado; levar um período de ($x$ quantidade de tempo)}
  \end{Phonetics}
\end{Entry}

\begin{Entry}{耗费}{10,9}{⽾、⾙}
  \begin{Phonetics}{耗费}{hao4fei4}[][HSK 7-9]
    \definition{v.}{gastar; consumir; esgotar}
  \end{Phonetics}
\end{Entry}

%%%%%%%%%% 耻 %%%%%%%%%%
\subsection*{耻}

\begin{Entry}{耻}{10}{⽿}
  \begin{Phonetics}{耻}{chi3}
    \definition{s.}{vergonha; desgraça; humilhação}
    \definition{v.}{estar envergonhado de; considerar vergonhoso}
  \end{Phonetics}
\end{Entry}

\begin{Entry}{耻笑}{10,10}{⽿、⽵}
  \begin{Phonetics}{耻笑}{chi3xiao4}[][HSK 7-9]
    \definition{v.}{ridicularizar alguém; zombar; zombar de; rir de}
  \end{Phonetics}
\end{Entry}

\begin{Entry}{耻辱}{10,10}{⽿、⾠}
  \begin{Phonetics}{耻辱}{chi3ru3}[][HSK 7-9]
    \definition{s.}{vergonha; desgraça; humilhação; danos à reputação; incidente vergonhoso}
  \end{Phonetics}
\end{Entry}

%%%%%%%%%% 耽 %%%%%%%%%%
\subsection*{耽}

\begin{Entry}{耽}{10}{⽿}
  \begin{Phonetics}{耽}{dan1}
    \definition*{s.}{Sobrenome: Dan}
    \definition{v.}{atrasar | (literário) abandonar-se a; entregar-se a}
  \end{Phonetics}
\end{Entry}

\begin{Entry}{耽心}{10,4}{⽿、⼼}
  \begin{Phonetics}{耽心}{dan1xin1}
    \variantof{担心}
  \end{Phonetics}
\end{Entry}

\begin{Entry}{耽误}{10,9}{⽿、⾔}
  \begin{Phonetics}{耽误}{dan1wu5}[][HSK 7-9]
    \definition{v.}{atrasar; segurar; perder algo devido a atraso ou oportunidade perdida; perder (oportunidade)}
  \end{Phonetics}
\end{Entry}

\begin{Entry}{耽搁}{10,12}{⽿、⼿}
  \begin{Phonetics}{耽搁}{dan1ge5}[][HSK 7-9]
    \definition{v.}{ficar; fazer uma parada | atrasar | perder (uma oportunidade, um prazo)}
  \end{Phonetics}
\end{Entry}

%%%%%%%%%% 耿 %%%%%%%%%%
\subsection*{耿}

\begin{Entry}{耿}{10}{⽿}
  \begin{Phonetics}{耿}{geng3}
    \definition*{s.}{Sobrenome: Geng}
    \definition{adj.}{Literário: brilhante | honesto e justo; correto; íntegro | dedicado; leal}
  \end{Phonetics}
\end{Entry}

\begin{Entry}{耿直}{10,8}{⽿、⽬}
  \begin{Phonetics}{耿直}{geng3zhi2}[][HSK 7-9]
    \definition{adj.}{íntregro; franco; correto; honesto e franco}
  \end{Phonetics}
\end{Entry}

%%%%%%%%%% 胳 %%%%%%%%%%
\subsection*{胳}

\begin{Entry}{胳}{10}{⾁}
  \begin{Phonetics}{胳}{ga1}
    \definition{s.}{usado em 胳肢窝}
  \seealsoref{胳肢窝}{ga1 zhi1 wo1}
  \end{Phonetics}
  \begin{Phonetics}{胳}{ge1}
    \definition{s.}{axila; sovaco}
  \end{Phonetics}
  \begin{Phonetics}{胳}{ge2}
    \definition{v.}{usado em 胳肢}
  \seealsoref{胳肢}{ge2zhi5}
  \end{Phonetics}
\end{Entry}

\begin{Entry}{胳肢}{10,8}{⾁、⾁}
  \begin{Phonetics}{胳肢}{ge2zhi5}
    \definition{v.}{(dialeto) fazer cócegas}
  \end{Phonetics}
\end{Entry}

\begin{Entry}{胳肢窝}{10,8,12}{⾁、⾁、⽳}
  \begin{Phonetics}{胳肢窝}{ga1 zhi1 wo1}
    \definition{s.}{axila; sovaco; também escrito 夹肢窝}
  \seealsoref{夹肢窝}{jia1 zhi1 wo1}
  \end{Phonetics}
\end{Entry}

\begin{Entry}{胳膊}{10,14}{⾁、⾁}
  \begin{Phonetics}{胳膊}{ge1bo5}[][HSK 7-9]
    \definition[条,双,只]{s.}{braço; a área abaixo do ombro e acima do pulso}
  \end{Phonetics}
\end{Entry}

%%%%%%%%%% 胶 %%%%%%%%%%
\subsection*{胶}

\begin{Entry}{胶}{10}{⾁}
  \begin{Phonetics}{胶}{jiao1}
    \definition*{s.}{Sobrenome: Jiao}
    \definition{adj.}{pegajoso; viscoso; grudento}
    \definition{s.}{cola; goma; adesivo | borracha | gel; colóide}
    \definition{v.}{colar com cola | colar; grudar}
  \end{Phonetics}
\end{Entry}

\begin{Entry}{胶水}{10,4}{⾁、⽔}
  \begin{Phonetics}{胶水}{jiao1shui3}[][HSK 5]
    \definition[瓶]{s.}{cola; mucilagem; cola líquida}
  \end{Phonetics}
\end{Entry}

\begin{Entry}{胶片}{10,4}{⾁、⽚}
  \begin{Phonetics}{胶片}{jiao1pian4}[][HSK 7-9]
    \definition[卷]{s.}{filme; cartucho; filme fotográfico}
  \end{Phonetics}
\end{Entry}

\begin{Entry}{胶卷}{10,8}{⾁、⼙}
  \begin{Phonetics}{胶卷}{jiao1juan3}
    \definition{s.}{filme | rolo de filme}
  \end{Phonetics}
\end{Entry}

\begin{Entry}{胶带}{10,9}{⾁、⼱}
  \begin{Phonetics}{胶带}{jiao1 dai4}[][HSK 5]
    \definition[卷,条,段]{s.}{fita de embalagem transparente; fita adesiva | fita magnética de plástico; fita de gravação | fita emborrachada; cinta de borracha}
  \end{Phonetics}
\end{Entry}

\begin{Entry}{胶囊}{10,22}{⾁、⾐}
  \begin{Phonetics}{胶囊}{jiao1nang2}[][HSK 7-9]
    \definition{s.}{Medicina: cápsula; refere-se a uma cápsula de gelatina usada para encapsular medicamentos em pó ou granulados, facilitando a ingestão}
  \end{Phonetics}
\end{Entry}

%%%%%%%%%% 胸 %%%%%%%%%%
\subsection*{胸}

\begin{Entry}{胸}{10}{⾁}
  \begin{Phonetics}{胸}{xiong1}
    \definition{s.}{peito | tórax}
  \end{Phonetics}
\end{Entry}

\begin{Entry}{胸部}{10,10}{⾁、⾢}
  \begin{Phonetics}{胸部}{xiong1 bu4}[][HSK 4]
    \definition{s.}{peito; tórax; seios}
  \end{Phonetics}
\end{Entry}

%%%%%%%%%% 能 %%%%%%%%%%
\subsection*{能}

\begin{Entry}{能}{10}{⾁}
  \begin{Phonetics}{能}{neng2}[][HSK 1]
    \definition*{s.}{Sobrenome: Neng}
    \definition{adv.}{talvez}
    \definition{s.}{habilidade; capacidade; competência | potência; energia; em física, refere-se à energia}
    \definition{v.}{poder fazer; ser capaz de | ser possível | entre 不 \dots 不 para expressar obrigação, certeza ou grande probabilidade | poder; ter permissão para | ser bom em fazer algo | permitir}
  \end{Phonetics}
\end{Entry}

\begin{Entry}{能力}{10,2}{⾁、⼒}
  \begin{Phonetics}{能力}{neng2li4}[][HSK 3]
    \definition[个,种]{s.}{habilidade; capacidade; aptidão; as condições subjetivas para ser competente para uma tarefa}
  \end{Phonetics}
\end{Entry}

\begin{Entry}{能上能下}{10,3,10,3}{⾁、⼀、⾁、⼀}
  \begin{Phonetics}{能上能下}{neng2shang4neng2xia4}
    \definition{s.}{pronto para aceitar qualquer trabalho, alto ou baixo}
  \end{Phonetics}
\end{Entry}

\begin{Entry}{能干}{10,3}{⾁、⼲}
  \begin{Phonetics}{能干}{neng2gan4}[][HSK 4]
    \definition{adj.}{apto; capaz; competente}
  \end{Phonetics}
\end{Entry}

\begin{Entry}{能不能}{10,4,10}{⾁、⼀、⾁}
  \begin{Phonetics}{能不能}{neng2 bu4 neng2}[][HSK 3]
    \definition{adv.}{pode ou não pode\dots?}
  \end{Phonetics}
\end{Entry}

\begin{Entry}{能否}{10,7}{⾁、⼝}
  \begin{Phonetics}{能否}{neng2 fou3}[][HSK 6]
    \definition{adv.}{é possível; se ou não; pode ou não pode; Você consegue?; expressa dúvida, frequentemente usado em perguntas de sim ou não}
  \end{Phonetics}
\end{Entry}

\begin{Entry}{能够}{10,11}{⾁、⼣}
  \begin{Phonetics}{能够}{neng2 gou4}[][HSK 2]
    \definition{v.}{poder; ser capaz de; indica que possui uma determinada capacidade ou que atingiu um determinado nível de eficiência | poder; ser capaz de; indica que algo é permitido sob certas condições ou por motivos razoáveis}
  \end{Phonetics}
\end{Entry}

\begin{Entry}{能量}{10,12}{⾁、⾥}
  \begin{Phonetics}{能量}{neng2liang4}[][HSK 5]
    \definition[种]{s.}{energia; quantidade de energia; Uma grandeza física que mede a capacidade da matéria de realizar trabalho | capacidade; competências; capacidade e papel que uma pessoa pode desempenhar}
  \end{Phonetics}
\end{Entry}

%%%%%%%%%% 脂 %%%%%%%%%%
\subsection*{脂}

\begin{Entry}{脂}{10}{⾁}
  \begin{Phonetics}{脂}{zhi1}
    \definition*{s.}{Sobrenome: Zhi}
    \definition{s.}{gordura; graxa; sebo | (cosméticos) rouge | (cosméticos) baton; protetor labial}
  \end{Phonetics}
\end{Entry}

\begin{Entry}{脂麻}{10,11}{⾁、⿇}
  \begin{Phonetics}{脂麻}{zhi1ma5}
    \variantof{芝麻}
  \end{Phonetics}
\end{Entry}

%%%%%%%%%% 脆 %%%%%%%%%%
\subsection*{脆}

\begin{Entry}{脆}{10}{⾁}
  \begin{Phonetics}{脆}{cui4}[][HSK 5]
    \definition{adj.}{frágil; quebradiço (oposto a 韧) | crocante | (voz) clara; nítida | puro}
  \seealsoref{韧}{ren4}
  \end{Phonetics}
\end{Entry}

\begin{Entry}{脆弱}{10,10}{⾁、⼸}
  \begin{Phonetics}{脆弱}{cui4ruo4}[][HSK 7-9]
    \definition{adj.}{frágil; débil; fraco; incapaz de suportar contratempos}
  \end{Phonetics}
\end{Entry}

%%%%%%%%%% 脊 %%%%%%%%%%
\subsection*{脊}

\begin{Entry}{脊}{10}{⾁}
  \begin{Phonetics}{脊}{ji2}
    \definition{s.}{coluna vertebral (de humanos e animais) | espinha; costas; cumeeira; a parte de um objeto em forma de espinha}
  \end{Phonetics}
  \begin{Phonetics}{脊}{ji3}
    \definition{s.}{espinha dorsal; coluna vertebral | crista; cumeeira; espinhaço | vértebra}
  \end{Phonetics}
\end{Entry}

\begin{Entry}{脊梁}{10,11}{⾁、⽊}
  \begin{Phonetics}{脊梁}{ji3liang2}[][HSK 7-9]
    \definition{s.}{espinha dorsal | coluna vertebral}
  \end{Phonetics}
\end{Entry}

%%%%%%%%%% 脏 %%%%%%%%%%
\subsection*{脏}

\begin{Entry}{脏}{10}{⾁}
  \begin{Phonetics}{脏}{zang1}[][HSK 2]
    \definition{adj.}{sujo; imundo | imundo; metáfora para vulgaridade e obscenidade}
    \definition{v.}{tornar algo sujo ou impuro}
  \end{Phonetics}
  \begin{Phonetics}{脏}{zang4}
    \definition[处]{s.}{vísceras; órgãos internos do corpo, geralmente o coração, o fígado, o baço, os pulmões e os rins; um termo geral para órgãos nas cavidades torácica e abdominal de humanos ou animais | (anatomia) órgão; a medicina tradicional chinesa chama o coração, o fígado, o baço, os pulmões e os rins de órgãos internos}
  \end{Phonetics}
\end{Entry}

\begin{Entry}{脏土}{10,3}{⾁、⼟}
  \begin{Phonetics}{脏土}{zang1tu3}
    \definition{s.}{solo sujo | lama | lixo}
  \end{Phonetics}
\end{Entry}

\begin{Entry}{脏字}{10,6}{⾁、⼦}
  \begin{Phonetics}{脏字}{zang1zi4}
    \definition{s.}{obscenidade}
  \end{Phonetics}
\end{Entry}

\begin{Entry}{脏病}{10,10}{⾁、⽧}
  \begin{Phonetics}{脏病}{zang1bing4}
    \definition{s.}{doença venérea}
  \end{Phonetics}
\end{Entry}

\begin{Entry}{脏脏}{10,10}{⾁、⾁}
  \begin{Phonetics}{脏脏}{zang1zang1}
    \definition{adj.}{sujo}
  \end{Phonetics}
\end{Entry}

\begin{Entry}{脏煤}{10,13}{⾁、⽕}
  \begin{Phonetics}{脏煤}{zang1mei2}
    \definition{s.}{carvão sujo | sujeira (de uma mina de carvão)}
  \end{Phonetics}
\end{Entry}

\begin{Entry}{脏器}{10,16}{⾁、⼝}
  \begin{Phonetics}{脏器}{zang4qi4}
    \definition{s.}{órgãos internos}
  \end{Phonetics}
\end{Entry}

\begin{Entry}{脏辫}{10,17}{⾁、⾟}
  \begin{Phonetics}{脏辫}{zang1bian4}
    \definition{s.}{\emph{dreadlocks}}
  \end{Phonetics}
\end{Entry}

%%%%%%%%%% 脑 %%%%%%%%%%
\subsection*{脑}

\begin{Entry}{脑}{10}{⾁}
  \begin{Phonetics}{脑}{nao3}
    \definition{s.}{(fisiologia) cérebro | tofu;  substância branca semelhante ao cérebro ou à medula espinhal cerebral | cabeça | a essência de um objeto}
  \end{Phonetics}
\end{Entry}

\begin{Entry}{脑子}{10,3}{⾁、⼦}
  \begin{Phonetics}{脑子}{nao3 zi5}[][HSK 5]
    \definition[个]{s.}{cérebro | mente; cabeça; cérebro; inteligência; poder mental; refere-se à capacidade de pensar, memorizar, raciocinar, etc.; inteligência}
  \end{Phonetics}
\end{Entry}

\begin{Entry}{脑瓜}{10,5}{⾁、⽠}
  \begin{Phonetics}{脑瓜}{nao3gua1}
    \definition{s.}{crânio | cérebro | cabeça | mente | mentalidade | ideia}
  \seealsoref{脑瓜子}{nao3gua1zi5}
  \end{Phonetics}
\end{Entry}

\begin{Entry}{脑瓜子}{10,5,3}{⾁、⽠、⼦}
  \begin{Phonetics}{脑瓜子}{nao3gua1zi5}
    \definition{s.}{Coloquial: crânio; cérebro; cabeça; mente; mentalidade; ideia}
  \seealsoref{脑瓜}{nao3gua1}
  \end{Phonetics}
\end{Entry}

\begin{Entry}{脑袋}{10,11}{⾁、⾐}
  \begin{Phonetics}{脑袋}{nao3dai5}[][HSK 4]
    \definition[颗,个]{s.}{cabeça; a parte mais alta do corpo humano ou a parte mais alta de um animal que contém órgãos como a boca, o nariz, os olhos etc. | mente; cérebro; capacidade de pensar, lembrar, etc.}
  \end{Phonetics}
\end{Entry}

%%%%%%%%%% 臭 %%%%%%%%%%
\subsection*{臭}

\begin{Entry}{臭}{10}{⾃}
  \begin{Phonetics}{臭}{chou4}[][HSK 5]
    \definition{adj.}{sujo; malcheiroso; fedorento; contrário de 香 | repugnante; nojento; repulsivo | ruim; pobre; péssimo}
    \definition{adv.}{severamente; firmemente}
    \definition{v.}{falhar em detonar (bala)}
  \seealsoref{香}{xiang1}
  \end{Phonetics}
  \begin{Phonetics}{臭}{xiu4}
    \definition{s.}{odor; cheiro}
    \definition{v.}{cheirar; farejar; o mesmo que 嗅}
  \seealsoref{嗅}{xiu4}
  \end{Phonetics}
\end{Entry}

\begin{Entry}{臭气}{10,4}{⾃、⽓}
  \begin{Phonetics}{臭气}{chou4qi4}
    \definition{s.}{fedor}
  \end{Phonetics}
\end{Entry}

%%%%%%%%%% 致 %%%%%%%%%%
\subsection*{致}

\begin{Entry}{致}{10}{⾄}
  \begin{Phonetics}{致}{zhi4}
    \definition{adj.}{fino; delicado; meticuloso; preciso}
    \definition{s.}{interesse}
    \definition{v.}{enviar; estender; entregar; dar; mostrar (cortesia, afeto, etc.) à outra parte | concentrar-se; trabalhar para; dedicar (os próprios esforços, etc.); focar em um aspecto | causar; incorrer; convidar; levar a | alcançar}
  \end{Phonetics}
\end{Entry}

\begin{Entry}{致敬}{10,12}{⾄、⽁}
  \begin{Phonetics}{致敬}{zhi4jing4}
    \definition{v.}{saudar; prestar homenagem a; demonstrar respeito (homenagem) a}
  \end{Phonetics}
\end{Entry}

%%%%%%%%%% 航 %%%%%%%%%%
\subsection*{航}

\begin{Entry}{航}{10}{⾈}
  \begin{Phonetics}{航}{hang2}
    \definition*{s.}{Sobrenome: Hang}
    \definition[趟]{s.}{barco; navio}
    \definition{v.}{navegar (por água ou ar) | velejar}
  \end{Phonetics}
\end{Entry}

\begin{Entry}{航天}{10,4}{⾈、⼤}
  \begin{Phonetics}{航天}{hang2tian1}[][HSK 7-9]
    \definition{s.}{voo espacial; astronáutica}
    \definition{v.}{voar ou viajar no espaço}
  \end{Phonetics}
\end{Entry}

\begin{Entry}{航天员}{10,4,7}{⾈、⼤、⼝}
  \begin{Phonetics}{航天员}{hang2tian1yuan2}[][HSK 7-9]
    \definition[名,位,个]{s.}{astronauta}
  \end{Phonetics}
\end{Entry}

\begin{Entry}{航行}{10,6}{⾈、⾏}
  \begin{Phonetics}{航行}{hang2xing2}[][HSK 7-9]
    \definition{v.}{velejar; voar; navegar pela água, pelo ar}
  \end{Phonetics}
\end{Entry}

\begin{Entry}{航运}{10,7}{⾈、⾡}
  \begin{Phonetics}{航运}{hang2yun4}[][HSK 7-9]
    \definition{s.}{transporte hidroviário; transporte marítimo}
  \end{Phonetics}
\end{Entry}

\begin{Entry}{航空}{10,8}{⾈、⽳}
  \begin{Phonetics}{航空}{hang2kong1}[][HSK 4]
    \definition{s.}{viagem; aviação; refere-se ao voo de uma aeronave no ar}
  \end{Phonetics}
\end{Entry}

\begin{Entry}{航海}{10,10}{⾈、⽔}
  \begin{Phonetics}{航海}{hang2hai3}[][HSK 7-9]
    \definition{v.}{velejar; navegar}
  \end{Phonetics}
\end{Entry}

\begin{Entry}{航班}{10,10}{⾈、⽟}
  \begin{Phonetics}{航班}{hang2ban1}[][HSK 4]
    \definition[个,次]{s.}{número do voo; voo programado; o horário de um navio ou avião de passageiros}
  \end{Phonetics}
\end{Entry}

%%%%%%%%%% 般 %%%%%%%%%%
\subsection*{般}

\begin{Entry}{般}{10}{⾈}
  \begin{Phonetics}{般}{ban1}
    \definition{clas.}{tipo; classe; gênero; amostra}
    \definition{part.}{(o mesmo) que; como; semelhante}
  \end{Phonetics}
  \begin{Phonetics}{般}{bo1}
    \definition{s.}{utilizado em 般若}
  \seealsoref{般若}{bo1re3}
  \end{Phonetics}
  \begin{Phonetics}{般}{pan2}
    \definition{adj.}{feliz; bem-aventurado}
  \end{Phonetics}
\end{Entry}

\begin{Entry}{般乐}{10,5}{⾈、⼃}
  \begin{Phonetics}{般乐}{pan2le4}
    \definition{v.}{jogar | divertir-se}
  \end{Phonetics}
\end{Entry}

\begin{Entry}{般若}{10,8}{⾈、⾋}
  \begin{Phonetics}{般若}{bo1re3}
    \definition*{s.}{Prajña (sânscrito), \emph{insight} sobre a verdadeira natureza da realidade}
    \definition{s.}{budismo: sabedoria}
  \end{Phonetics}
\end{Entry}

%%%%%%%%%% 舱 %%%%%%%%%%
\subsection*{舱}

\begin{Entry}{舱}{10}{⾈}
  \begin{Phonetics}{舱}{cang1}[][HSK 7-9]
    \definition{s.}{cabine (de um avião ou navio) | módulo (de uma nave espacial) | espaço em um navio ou aeronave para transportar pessoas, carga ou máquinas}
  \end{Phonetics}
\end{Entry}

%%%%%%%%%% 荷 %%%%%%%%%%
\subsection*{荷}

\begin{Entry}{荷}{10}{⾋}
  \begin{Phonetics}{荷}{he2}
    \definition*{s.}{Países Baixos; Holanda, abreviação de 荷兰 | Sobrenome: He}
    \definition{s.}{lótus}
  \seealsoref{荷兰}{he2lan2}
  \end{Phonetics}
  \begin{Phonetics}{荷}{he4}
    \definition{s.}{fardo; responsabilidade}
    \definition{v.}{carregar no ombro ou nas costas | aceitar um favor, frequentemente usado em cartas para expressar cortesia}
  \end{Phonetics}
\end{Entry}

\begin{Entry}{荷兰}{10,5}{⾋、⼋}
  \begin{Phonetics}{荷兰}{he2lan2}
    \definition*{s.}{Países Baixos; Holanda}
  \end{Phonetics}
\end{Entry}

\begin{Entry}{荷花}{10,7}{⾋、⾋}
  \begin{Phonetics}{荷花}{he2hua1}[][HSK 7-9]
    \definition[朵,枝,片]{s.}{lótus; flor de lótus}
  \end{Phonetics}
\end{Entry}

%%%%%%%%%% 莎 %%%%%%%%%%
\subsection*{莎}

\begin{Entry}{莎}{10}{⾋}
  \begin{Phonetics}{莎}{sha1}
    \definition{s.}{em nomes pessoais e de lugares | cigarra | fonético "sha" usado na transliteração}
  \end{Phonetics}
  \begin{Phonetics}{莎}{suo1}
  \end{Phonetics}
\end{Entry}

\begin{Entry}{莎莎舞}{10,10,14}{⾋、⾋、⾇}
  \begin{Phonetics}{莎莎舞}{sha1sha1wu3}
    \definition{s.}{salsa (dança)}
  \end{Phonetics}
\end{Entry}

%%%%%%%%%% 莫 %%%%%%%%%%
\subsection*{莫}

\begin{Entry}{莫}{10}{⾋}
  \begin{Phonetics}{莫}{mo4}
    \definition*{s.}{Sobrenome: Mo}
    \definition{adv.}{não, frequentemente usado em frases imperativas | não; não pode | pode ser que; não pode ser que; é possível que}
    \definition{pron.}{nenhum; nada; ninguém; significa 没有谁 ou 没有哪一种东西}
  \seealsoref{没有哪一种东西}{mei2you3 na3 yi4 zhong3 dong1xi1}
  \seealsoref{没有谁}{mei2you3 shei2}
  \end{Phonetics}
\end{Entry}

\begin{Entry}{莫名其妙}{10,6,8,7}{⾋、⼝、⼋、⼥}
  \begin{Phonetics}{莫名其妙}{mo4ming2qi2miao4}
    \definition{adj.}{desconcertante | bizzaro | inexplicável | perplexo}
  \end{Phonetics}
\end{Entry}

\begin{Entry}{莫非}{10,8}{⾋、⾮}
  \begin{Phonetics}{莫非}{mo4fei1}
    \definition{expr.}{Não é mesmo?; é frequentemente usado com 不成}
    \definition{v.}{pode ser que; é possível que}
  \seealsoref{不成}{bu4 cheng2}
  \end{Phonetics}
\end{Entry}

%%%%%%%%%% 莲 %%%%%%%%%%
\subsection*{莲}

\begin{Entry}{莲}{10}{⾋}
  \begin{Phonetics}{莲}{lian2}
    \definition*{s.}{Sobrenome: Lian}
    \definition[粒]{s.}{lótus}
  \end{Phonetics}
\end{Entry}

\begin{Entry}{莲子}{10,3}{⾋、⼦}
  \begin{Phonetics}{莲子}{lian2zi3}[][HSK 7-9]
    \definition[颗,个]{s.}{semente de lótus; a semente de lótus tem formato oval, com um miolo verde no centro e polpa branco-leitosa; pode ser consumida e usada na medicina}
  \end{Phonetics}
\end{Entry}

\begin{Entry}{莲花}{10,7}{⾋、⾋}
  \begin{Phonetics}{莲花}{lian2hua1}
    \definition{s.}{flor de lótus | lírio aquático}
  \end{Phonetics}
\end{Entry}

\begin{Entry}{莲藕}{10,18}{⾋、⾋}
  \begin{Phonetics}{莲藕}{lian2'ou3}
    \definition{s.}{raiz de Lotus}
  \end{Phonetics}
\end{Entry}

%%%%%%%%%% 获 %%%%%%%%%%
\subsection*{获}

\begin{Entry}{获}{10}{⾋}
  \begin{Phonetics}{获}{huo4}[][HSK 4]
    \definition*{s.}{Sobrenome: Huo}
    \definition{v.}{capturar; pegar | obter; ganhar; colher | colher; ceifar}
  \end{Phonetics}
\end{Entry}

\begin{Entry}{获取}{10,8}{⾋、⼜}
  \begin{Phonetics}{获取}{huo4 qu3}[][HSK 4]
    \definition{v.}{adquirir; obter; ganhar; colher}
  \end{Phonetics}
\end{Entry}

\begin{Entry}{获奖}{10,9}{⾋、⼤}
  \begin{Phonetics}{获奖}{huo4 jiang3}[][HSK 4]
    \definition{v.}{ganhar prêmio; ser recompensado; ganhar um prêmio; receber um prêmio}
  \end{Phonetics}
\end{Entry}

\begin{Entry}{获胜}{10,9}{⾋、⾁}
  \begin{Phonetics}{获胜}{huo4sheng4}[][HSK 7-9]
    \definition{v.}{vencer; ser vitorioso; triunfar; alcançar a vitória}
  \end{Phonetics}
\end{Entry}

\begin{Entry}{获得}{10,11}{⾋、⼻}
  \begin{Phonetics}{获得}{huo4de2}[][HSK 4]
    \definition{v.}{adquirir; ganhar; obter; alcançar}
  \end{Phonetics}
\end{Entry}

\begin{Entry}{获悉}{10,11}{⾋、⼼}
  \begin{Phonetics}{获悉}{huo4xi1}[][HSK 7-9]
    \definition{v.}{saber (de um evento); receber notícias; ser informado}
  \end{Phonetics}
\end{Entry}

%%%%%%%%%% 蚊 %%%%%%%%%%
\subsection*{蚊}

\begin{Entry}{蚊}{10}{⾍}
  \begin{Phonetics}{蚊}{wen2}
    \definition{s.}{mosquito; pernilongo}
  \end{Phonetics}
\end{Entry}

\begin{Entry}{蚊子}{10,3}{⾍、⼦}
  \begin{Phonetics}{蚊子}{wen2zi5}
    \definition{s.}{pernilongo}
  \end{Phonetics}
\end{Entry}

\begin{Entry}{蚊香}{10,9}{⾍、⾹}
  \begin{Phonetics}{蚊香}{wen2xiang1}
    \definition{s.}{incenso ou espiral repelente de mosquitos}
  \end{Phonetics}
\end{Entry}

%%%%%%%%%% 蚕 %%%%%%%%%%
\subsection*{蚕}

\begin{Entry}{蚕}{10}{⾍}
  \begin{Phonetics}{蚕}{can2}
    \definition[只,条]{s.}{bicho-da-seda; um inseto que pode fiar seda e fazer casulos}
  \end{Phonetics}
\end{Entry}

\begin{Entry}{蚕纸}{10,7}{⾍、⽷}
  \begin{Phonetics}{蚕纸}{can2zhi3}
    \definition{s.}{papel onde o bicho-da-seda põe seus ovos}
  \end{Phonetics}
\end{Entry}

%%%%%%%%%% 蚝 %%%%%%%%%%
\subsection*{蚝}

\begin{Entry}{蚝}{10}{⾍}
  \begin{Phonetics}{蚝}{hao2}
    \definition[只]{s.}{ostra}
  \end{Phonetics}
\end{Entry}

%%%%%%%%%% 袖 %%%%%%%%%%
\subsection*{袖}

\begin{Entry}{袖}{10}{⾐}
  \begin{Phonetics}{袖}{xiu4}
    \definition{s.}{manga (de camisa, de camiseta, etc.)}
  \end{Phonetics}
\end{Entry}

\begin{Entry}{袖珍}{10,9}{⾐、⽟}
  \begin{Phonetics}{袖珍}{xiu4 zhen1}[][HSK 6]
    \definition{adj.}{do tamanho do bolso; de bolso (livro, agenda, etc.)}
  \end{Phonetics}
\end{Entry}

%%%%%%%%%% 袜 %%%%%%%%%%
\subsection*{袜}

\begin{Entry}{袜}{10}{⾐}
  \begin{Phonetics}{袜}{wa4}
    \definition[只,双,打]{s.}{meias; meias-calças}
  \end{Phonetics}
\end{Entry}

\begin{Entry}{袜子}{10,3}{⾐、⼦}
  \begin{Phonetics}{袜子}{wa4zi5}[][HSK 4]
    \definition[双,只,对]{s.}{meias; peúgas; meias-calças}
  \end{Phonetics}
\end{Entry}

%%%%%%%%%% 被 %%%%%%%%%%
\subsection*{被}

\begin{Entry}{被}{10}{⾐}
  \begin{Phonetics}{被}{bei4}[][HSK 3]
    \definition{part.}{usada antes de verbos para formar frases verbais passivas}
    \definition{prep.}{usado em uma estrutura passiva para introduzir o executor da ação ou apenas a ação | usado em frases para expressar passividade, com o sujeito sendo o objeto}
    \definition{s.}{colcha}
    \definition{v.}{cobrir; espalhar | sofrer}
  \end{Phonetics}
\end{Entry}

\begin{Entry}{被子}{10,3}{⾐、⼦}
  \begin{Phonetics}{被子}{bei4zi5}[][HSK 3]
    \definition[条,床]{s.}{colcha; cobertor; algo com que você se cobre quando dorme, geralmente feito de pano ou seda, com forro de pano e preenchido com algodão ou fio de seda}
  \end{Phonetics}
\end{Entry}

\begin{Entry}{被动}{10,6}{⾐、⼒}
  \begin{Phonetics}{被动}{bei4dong4}[][HSK 5]
    \definition{adj.}{passivo;  agir com base em um impulso externo (oposto de 主动) | passivo; impossibilidade de prosseguir como pretendido devido a resistência ou interferência}
  \seealsoref{主动}{zhu3dong4}
  \end{Phonetics}
\end{Entry}

\begin{Entry}{被告}{10,7}{⾐、⼝}
  \begin{Phonetics}{被告}{bei4gao4}[][HSK 6]
    \definition{s.}{réu; indiciado; acusado (oposto a 原告)}
  \seealsoref{原告}{yuan2gao4}
  \end{Phonetics}
\end{Entry}

\begin{Entry}{被单}{10,8}{⾐、⼗}
  \begin{Phonetics}{被单}{bei4dan1}
    \definition[床]{s.}{lençol (de cama) | envelope para uma colcha acolchoada}
  \end{Phonetics}
\end{Entry}

\begin{Entry}{被迫}{10,8}{⾐、⾡}
  \begin{Phonetics}{被迫}{bei4 po4}[][HSK 4]
    \definition{v.}{ser forçado; ser coagido; ser compelido; ser constrangido; ser forçado a fazer algo por força externa}
  \end{Phonetics}
\end{Entry}

\begin{Entry}{被套}{10,10}{⾐、⼤}
  \begin{Phonetics}{被套}{bei4tao4}
    \definition{s.}{capa de \emph{edredon}}
    \definition{v.}{ter dinheiro preso (em ações, imóveis, etc.)}
  \end{Phonetics}
\end{Entry}

\begin{Entry}{被捕}{10,10}{⾐、⼿}
  \begin{Phonetics}{被捕}{bei4bu3}[][HSK 7-9]
    \definition{v.}{ser preso; estar sob prisão}
  \end{Phonetics}
\end{Entry}

\begin{Entry}{被窝}{10,12}{⾐、⽳}
  \begin{Phonetics}{被窝}{bei4wo1}
    \definition{s.}{colcha}
  \end{Phonetics}
\end{Entry}

%%%%%%%%%% 请 %%%%%%%%%%
\subsection*{请}

\begin{Entry}{请}{10}{⾔}
  \begin{Phonetics}{请}{qing3}[][HSK 1]
    \definition*{s.}{Sobrenome: Qing}
    \definition{v.}{solicitar; perguntar | convidar; envolver | por favor; uma expressão educada usada quando você quer que alguém faça algo | comprar coisas sagradas para sacrifício, como incenso, velas, cavalos de papel e santuários de Buda; superstição se refere à compra de estátuas de Buda, santuários, etc. | entreter}
  \end{Phonetics}
\end{Entry}

\begin{Entry}{请问}{10,6}{⾔、⾨}
  \begin{Phonetics}{请问}{qing3 wen4}[][HSK 1]
    \definition{expr.}{Com licença, posso perguntar\dots? (para perguntar por qualquer coisa); uma maneira educada de pedir para alguém responder a uma pergunta}
  \end{Phonetics}
\end{Entry}

\begin{Entry}{请坐}{10,7}{⾔、⼟}
  \begin{Phonetics}{请坐}{qing3 zuo4}[][HSK 1]
    \definition{v.}{por favor, sente-se; convidar outras pessoas para sentar ou descansar}
  \end{Phonetics}
\end{Entry}

\begin{Entry}{请求}{10,7}{⾔、⽔}
  \begin{Phonetics}{请求}{qing3qiu2}[][HSK 2]
    \definition[个,次]{s.}{pedido; petição; solicitação; refere-se à exigência apresentada}
    \definition{v.}{pedir; solicitar; requerer; peticionar; fazer uma solicitação e pedir que a outra parte concorde com ela}
  \end{Phonetics}
\end{Entry}

\begin{Entry}{请进}{10,7}{⾔、⾡}
  \begin{Phonetics}{请进}{qing3 jin4}[][HSK 1]
    \definition{v.}{por favor entre; convidar alguém para um espaço ou lugar}
  \end{Phonetics}
\end{Entry}

\begin{Entry}{请客}{10,9}{⾔、⼧}
  \begin{Phonetics}{请客}{qing3/ke4}[][HSK 2]
    \definition{v.+compl.}{receber convidados; hospedar convidados | oferecer; convidar; pagar a conta; arcar com os custos; convidar alguém para comer, tomar chá, etc.}
  \end{Phonetics}
\end{Entry}

\begin{Entry}{请假}{10,11}{⾔、⼈}
  \begin{Phonetics}{请假}{qing3/jia4}[][HSK 1]
    \definition{v.+compl.}{pedir licença para sair; solicitar permissão para não trabalhar ou estudar por um determinado período de tempo devido a doença ou outros motivos}
  \end{Phonetics}
\end{Entry}

\begin{Entry}{请假条}{10,11,7}{⾔、⼈、⽊}
  \begin{Phonetics}{请假条}{qing3jia4tiao2}
    \definition{s.}{pedido de licença de ausência (do trabalho ou da escola)}
  \end{Phonetics}
\end{Entry}

\begin{Entry}{请教}{10,11}{⾔、⽁}
  \begin{Phonetics}{请教}{qing3jiao4}[][HSK 3]
    \definition{v.}{consultar; pedir conselho}
  \end{Phonetics}
\end{Entry}

%%%%%%%%%% 诸 %%%%%%%%%%
\subsection*{诸}

\begin{Entry}{诸}{10}{⾔}
  \begin{Phonetics}{诸}{zhu1}
    \definition*{s.}{Sobrenome: Zhu}
    \definition{adj.}{todos; cada; vários}
    \definition{prep.}{em; para; de}
  \end{Phonetics}
\end{Entry}

\begin{Entry}{诸位}{10,7}{⾔、⼈}
  \begin{Phonetics}{诸位}{zhu1wei4}[][HSK 6]
    \definition{pron.}{senhores; todos; todos vocês; senhoras e senhores; um termo educado que se refere a várias pessoas}
  \end{Phonetics}
\end{Entry}

%%%%%%%%%% 诺 %%%%%%%%%%
\subsection*{诺}

\begin{Entry}{诺}{10}{⾔}
  \begin{Phonetics}{诺}{nuo4}
    \definition*{s.}{Sobrenome: Nuo}
    \definition{interj.}{Sim!}
    \definition{v.}{prometer}
  \end{Phonetics}
\end{Entry}

\begin{Entry}{诺贝尔奖}{10,4,5,9}{⾔、⾙、⼩、⼤}
  \begin{Phonetics}{诺贝尔奖}{nuo4bei4'er3 jiang3}
    \definition*{s.}{Prêmio Nobel}
  \end{Phonetics}
\end{Entry}

\begin{Entry}{诺奖}{10,9}{⾔、⼤}
  \begin{Phonetics}{诺奖}{nuo4jiang3}
    \definition*{s.}{Prêmio Nobel, abreviação de 诺贝尔奖}
  \seealsoref{诺贝尔奖}{nuo4bei4'er3 jiang3}
  \end{Phonetics}
\end{Entry}

%%%%%%%%%% 读 %%%%%%%%%%
\subsection*{读}

\begin{Entry}{读}{10}{⾔}
  \begin{Phonetics}{读}{dou4}
    \definition{s.}{vírgula; uma breve pausa na leitura}
  \end{Phonetics}
  \begin{Phonetics}{读}{du2}[][HSK 1]
    \definition*{s.}{Sobrenome: Du}
    \definition{v.}{ler em voz alta | ler; ler o texto e compreendera seu significado | frequentar a escola; refere-se a ir à escola ou estudar | (computação) ler dados}
  \end{Phonetics}
\end{Entry}

\begin{Entry}{读书}{10,4}{⾔、⼄}
  \begin{Phonetics}{读书}{du2/shu1}[][HSK 1]
    \definition{v.+compl.}{ler; estudar | frequentar a escola}
  \end{Phonetics}
\end{Entry}

\begin{Entry}{读者}{10,8}{⾔、⽼}
  \begin{Phonetics}{读者}{du2 zhe3}[][HSK 3]
    \definition[个,位,名,些,群]{s.}{leitor; (para obras, autores, revistas, etc.) Pessoas que compram ou leem livros, revistas, artigos, jornais, etc.}
  \end{Phonetics}
\end{Entry}

\begin{Entry}{读音}{10,9}{⾔、⾳}
  \begin{Phonetics}{读音}{du2 yin1}[][HSK 2]
    \definition[种]{s.}{pronúncia}
  \end{Phonetics}
\end{Entry}

%%%%%%%%%% 诽 %%%%%%%%%%
\subsection*{诽}

\begin{Entry}{诽}{10}{⾔}
  \begin{Phonetics}{诽}{fei3}
    \definition{v.}{calúnia}
  \end{Phonetics}
\end{Entry}

\begin{Entry}{诽谤}{10,12}{⾔、⾔}
  \begin{Phonetics}{诽谤}{fei3bang4}[][HSK 7-9]
    \definition{v.}{difamar; caluniar; falar mal; espalhar boatos e caluniar os outros; falar mal dos outros e prejudicar sua reputação}
  \end{Phonetics}
\end{Entry}

%%%%%%%%%% 课 %%%%%%%%%%
\subsection*{课}

\begin{Entry}{课}{10}{⾔}
  \begin{Phonetics}{课}{ke4}[][HSK 1]
    \definition{clas.}{aula; lição; unidade de tempo de ensino; parágrafo do material didático}
    \definition[门,节]{s.}{classe; aula; ensino por etapas planejado | disciplina; curso | imposto; antiga referência a impostos | seção; departamentos de escritório criados no antigo governo}
    \definition{v.}{cobrar; impor; taxar}
  \end{Phonetics}
\end{Entry}

\begin{Entry}{课文}{10,4}{⾔、⽂}
  \begin{Phonetics}{课文}{ke4 wen2}[][HSK 1]
    \definition[篇,段]{s.}{texto (de uma lição); texto principal do livro didático (diferente das notas de rodapé, exercícios, etc.)}
  \end{Phonetics}
\end{Entry}

\begin{Entry}{课本}{10,5}{⾔、⽊}
  \begin{Phonetics}{课本}{ke4 ben3}[][HSK 1]
    \definition[本]{s.}{livro didático; livro-texto}
  \end{Phonetics}
\end{Entry}

\begin{Entry}{课堂}{10,11}{⾔、⼟}
  \begin{Phonetics}{课堂}{ke4 tang2}[][HSK 2]
    \definition[间,节,个]{s.}{sala de aula; local onde se realizam as aulas; local onde se realizam as atividades de ensino}
  \end{Phonetics}
\end{Entry}

\begin{Entry}{课程}{10,12}{⾔、⽲}
  \begin{Phonetics}{课程}{ke4cheng2}[][HSK 3]
    \definition[个,堂,节,门]{s.}{curso; currículo; as disciplinas e o programa letivo da escola}
  \end{Phonetics}
\end{Entry}

\begin{Entry}{课题}{10,15}{⾔、⾴}
  \begin{Phonetics}{课题}{ke4ti2}[][HSK 5]
    \definition[组]{s.}{uma questão para estudo ou discussão; principais questões a serem pesquisadas ou discutidas, ou assuntos importantes que precisam ser resolvidos com urgência | tarefa; problema; questões a serem resolvidas}
  \end{Phonetics}
\end{Entry}

%%%%%%%%%% 谁 %%%%%%%%%%
\subsection*{谁}

\begin{Entry}{谁}{10}{⾔}
  \begin{Phonetics}{谁}{shei2}[][HSK 1]
    \definition{pron.}{quem? | (em pergunta retórica) quem?; usado em perguntas retóricas, para indicar que não há ninguém | refere-se a pessoas que não têm certeza, incluindo aquelas que não sabem | alguém; qualquer pessoa; indica qualquer pessoa ou qualquer um | repetido em uma frase para se referir a uma pessoa | (repetido em duas frases) quem quer que seja; fazer com que o sujeito e o objeto se refiram a duas pessoas diferentes}
  \end{Phonetics}
  \begin{Phonetics}{谁}{shui2}[][HSK 1]
  \end{Phonetics}
\end{Entry}

%%%%%%%%%% 调 %%%%%%%%%%
\subsection*{调}

\begin{Entry}{调}{10}{⾔}
  \begin{Phonetics}{调}{diao4}[][HSK 3]
    \definition{s.}{sotaque; pronúncia | nota (musical) | melodia; música | tom; refere-se ao tom da fala, ou seja, a elevação e descida do tom das palavras | estilo; ambiente; estilo metafórico, talento, etc. | argumento; discurso}
    \definition{v.}{deslocar; mover; transferir; mover (pessoas, objetos, etc.) de um lugar para outro | examinar; investigar}
  \end{Phonetics}
  \begin{Phonetics}{调}{tiao2}[][HSK 3]
    \definition{adj.}{harmonioso; boa coordenação}
    \definition{v.}{misturar; ajustar; fazer o ajuste uniforme e apropriado | provocar; importunar; fazer pouco de | incitar; instigar; provocar; semear discórdia | mediar; trazer harmonia}
  \end{Phonetics}
\end{Entry}

\begin{Entry}{调皮}{10,5}{⾔、⽪}
  \begin{Phonetics}{调皮}{tiao2pi2}[][HSK 4]
    \definition{adj.}{travesso; malicioso; malandro | indisciplinado; desordeiro; indomável; astuto | inteligente e desonesto}
  \end{Phonetics}
\end{Entry}

\begin{Entry}{调节}{10,5}{⾔、⾋}
  \begin{Phonetics}{调节}{tiao2jie2}[][HSK 5]
    \definition{v.}{regular; ajustar; ajustar e controlar de várias maneiras para atender aos requisitos}
  \end{Phonetics}
\end{Entry}

\begin{Entry}{调动}{10,6}{⾔、⼒}
  \begin{Phonetics}{调动}{diao4dong4}[][HSK 5]
    \definition{v.}{mudar; transferir; pessoal, trabalho | mobilizar; despertar; pôr em jogo; melhorar (motivação, entusiasmo, etc.) por meio de alguns meios | reunir; manobrar; mover (tropas); mobilizar forças militares}
  \end{Phonetics}
\end{Entry}

\begin{Entry}{调度}{10,9}{⾔、⼴}
  \begin{Phonetics}{调度}{diao4du4}[][HSK 7-9]
    \definition[位,个]{s.}{despachante; pessoal responsável pelo despacho do trabalho}
    \definition{v.}{organizar e despachar; arranjar e despachar}
  \end{Phonetics}
\end{Entry}

\begin{Entry}{调律}{10,9}{⾔、⼻}
  \begin{Phonetics}{调律}{tiao2lv4}
    \definition{v.}{afinar (por exemplo, um piano)}
  \end{Phonetics}
\end{Entry}

\begin{Entry}{调查}{10,9}{⾔、⽊}
  \begin{Phonetics}{调查}{diao4cha2}[][HSK 3]
    \definition[项,个,份]{s.}{pesquisa; investigação; informações obtidas após perguntar a outras pessoas ou investigar}
    \definition{v.}{investigar; indagar; inquerir; examinar; realizar uma investigação (geralmente no local) para entender a situação}
  \end{Phonetics}
\end{Entry}

\begin{Entry}{调研}{10,9}{⾔、⽯}
  \begin{Phonetics}{调研}{diao4 yan2}[][HSK 6]
    \definition{v.}{pesquisar e estudar; investigar e pesquisar; pesquisar}
  \end{Phonetics}
\end{Entry}

\begin{Entry}{调解}{10,13}{⾔、⾓}
  \begin{Phonetics}{调解}{tiao2jie3}[][HSK 5]
    \definition{v.}{mediar; fazer as pazes; resolver conflitos através da persuasão}
  \end{Phonetics}
\end{Entry}

\begin{Entry}{调整}{10,16}{⾔、⽁}
  \begin{Phonetics}{调整}{tiao2zheng3}[][HSK 3]
    \definition{v.}{ajustar; revisar; regularizar; fazer as alterações apropriadas no estado original para se adaptar à nova situação}
  \end{Phonetics}
\end{Entry}

%%%%%%%%%% 谅 %%%%%%%%%%
\subsection*{谅}

\begin{Entry}{谅}{10}{⾔}
  \begin{Phonetics}{谅}{liang4}
    \definition*{s.}{Sobrenome: Liang}
    \definition{v.}{perdoar; compreender | supor; presumir | desculpar | pensar; acreditar; supor}
  \end{Phonetics}
\end{Entry}

\begin{Entry}{谅解}{10,13}{⾔、⾓}
  \begin{Phonetics}{谅解}{liang4jie3}[][HSK 7-9]
    \definition{v.}{compreender; levar em consideração; compreender e perdoar os outros após entender a situação real}
  \end{Phonetics}
\end{Entry}

%%%%%%%%%% 谈 %%%%%%%%%%
\subsection*{谈}

\begin{Entry}{谈}{10}{⾔}
  \begin{Phonetics}{谈}{tan2}[][HSK 3]
    \definition*{s.}{Sobrenome: Tan}
    \definition{s.}{o que é dito ou falado; discurso}
    \definition{v.}{falar; bater papo; discutir}
  \end{Phonetics}
\end{Entry}

\begin{Entry}{谈判}{10,7}{⾔、⼑}
  \begin{Phonetics}{谈判}{tan2pan4}[][HSK 3]
    \definition{v.}{negociar; manter conversações; para resolver um grande problema, as partes relevantes trocaram opiniões entre si, na esperança de encontrar uma solução com a qual todos pudessem concordar}
  \end{Phonetics}
\end{Entry}

\begin{Entry}{谈话}{10,8}{⾔、⾔}
  \begin{Phonetics}{谈话}{tan2/hua4}[][HSK 3]
    \definition[次]{s.}{declaração; opiniões (principalmente políticas) expressas na forma de conversas}
    \definition{v.+compl.}{conversar; discutir | falar; refere-se especificamente ao uso da conversa para entender a situação, fazer trabalho ideológico, etc. (usado principalmente por superiores para subordinados)}
  \end{Phonetics}
\end{Entry}

\begin{Entry}{谈恋爱}{10,10,10}{⾔、⼼、⽖}
  \begin{Phonetics}{谈恋爱}{tan2lian4'ai4}
    \definition{v.}{namorar | apaixonar-se}
  \end{Phonetics}
\end{Entry}

%%%%%%%%%% 豹 %%%%%%%%%%
\subsection*{豹}

\begin{Entry}{豹}{10}{⾘}
  \begin{Phonetics}{豹}{bao4}[][HSK 7-9]
    \definition*{s.}{Sobrenome: Bao}
    \definition[只]{s.}{leopardo; pantera | espécies de gato da montanha}
  \end{Phonetics}
\end{Entry}

\begin{Entry}{豹子}{10,3}{⾘、⼦}
  \begin{Phonetics}{豹子}{bao4zi5}
    \definition[头]{s.}{leopardo}
  \end{Phonetics}
\end{Entry}

%%%%%%%%%% 贿 %%%%%%%%%%
\subsection*{贿}

\begin{Entry}{贿}{10}{⾙}
  \begin{Phonetics}{贿}{hui4}
    \definition[行]{s.}{bens; riqueza; objetos de valor; propriedade | suborno | Literário: wealth}
    \definition{v.}{subornar}
  \end{Phonetics}
\end{Entry}

\begin{Entry}{贿赂}{10,10}{⾙、⾙}
  \begin{Phonetics}{贿赂}{hui4lu4}[][HSK 7-9]
    \definition[笔]{s.}{suborno}
    \definition{v.}{subornar; subornar outros com dinheiro}
  \end{Phonetics}
\end{Entry}

%%%%%%%%%% 资 %%%%%%%%%%
\subsection*{资}

\begin{Entry}{资}{10}{⾙}
  \begin{Phonetics}{资}{zi1}
    \definition{s.}{recursos | capital | dinheiro | despesa}
    \definition{v.}{fornecer | suprir}
  \end{Phonetics}
\end{Entry}

\begin{Entry}{资本}{10,5}{⾙、⽊}
  \begin{Phonetics}{资本}{zi1ben3}[][HSK 5]
    \definition{s.}{capital; meios de produção ou moeda utilizados para fins lucrativos | o que é capitalizado; algo usado em benefício próprio; metáfora para obter benefícios}
  \end{Phonetics}
\end{Entry}

\begin{Entry}{资产}{10,6}{⾙、⼇}
  \begin{Phonetics}{资产}{zi1chan3}[][HSK 5]
    \definition{s.}{propriedade; bens; patrimônio | capital; fundo de capital; recursos financeiros da empresa | ativos; na contabilidade, refere-se à utilização de fundos}
  \end{Phonetics}
\end{Entry}

\begin{Entry}{资助}{10,7}{⾙、⼒}
  \begin{Phonetics}{资助}{zi1zhu4}[][HSK 5]
    \definition{s.}{subsídio}
    \definition{v.}{subsidiar; patrocinar; ajudar financeiramente; ajudar com recursos financeiros}
  \end{Phonetics}
\end{Entry}

\begin{Entry}{资金}{10,8}{⾙、⾦}
  \begin{Phonetics}{资金}{zi1jin1}[][HSK 3]
    \definition[笔]{s.}{fundo; capital; capital necessário para atividades comerciais, etc.}
  \end{Phonetics}
\end{Entry}

\begin{Entry}{资料}{10,10}{⾙、⽃}
  \begin{Phonetics}{资料}{zi1liao4}[][HSK 4]
    \definition[份,堆,本,个]{s.}{dados; material; material informativo para referência ou para ser considerado confiável | material de produção; meios de subsistência; requisitos de produção ou subsistência}
  \end{Phonetics}
\end{Entry}

\begin{Entry}{资格}{10,10}{⾙、⽊}
  \begin{Phonetics}{资格}{zi1ge2}[][HSK 3]
    \definition{s.}{qualificação; condições e identidades necessárias para exercer uma determinada atividade | senioridade; identidade formada pelo tempo dedicado a um determinado trabalho ou atividade}
  \end{Phonetics}
\end{Entry}

\begin{Entry}{资源}{10,13}{⾙、⽔}
  \begin{Phonetics}{资源}{zi1yuan2}[][HSK 4]
    \definition{s.}{recurso; fontes naturais de meios de produção ou subsistência}
  \end{Phonetics}
\end{Entry}

%%%%%%%%%% 赅 %%%%%%%%%%
\subsection*{赅}

\begin{Entry}{赅}{10}{⾙}
  \begin{Phonetics}{赅}{gai1}
    \definition*{s.}{Sobrenome: Gai}
    \definition{adj.}{completo; integral; abrangente; inclusivo}
  \end{Phonetics}
\end{Entry}

%%%%%%%%%% 赶 %%%%%%%%%%
\subsection*{赶}

\begin{Entry}{赶}{10}{⾛}
  \begin{Phonetics}{赶}{gan3}[][HSK 3]
    \definition*{s.}{Sobrenome: Gan}
    \definition{prep.}{por; até; até que; até quando; introduzir o momento em que algo aconteceu, indicando que se espera até um determinado momento}
    \definition{v.}{ultrapassar; alcançar | perseguir; correr atrás; tentar alcançar; dar uma corrida; acelerar ou intensificar  | dirigir; conduzir | expulsar; afugentar; afastar | encontrar; deparar-se com; esbarrar em; acontecer; encontrar-se em (uma situação); aproveitar-se de (uma oportunidade) | ir para; participar (atividades com horário marcado)}
  \end{Phonetics}
\end{Entry}

\begin{Entry}{赶上}{10,3}{⾛、⼀}
  \begin{Phonetics}{赶上}{gan3 shang4}[][HSK 6]
    \definition{v.}{alcançar; manter o ritmo com; acompanhar alguém ou o padrão do planejador | chegar a tempo para; ter tempo suficiente; não ser tarde demais | encontrar; topar com; cruzar com; encontrar-se com; acontecer de encontrar; encontrar algo, em um determinado momento ou oportunidade}
  \end{Phonetics}
\end{Entry}

\begin{Entry}{赶不上}{10,4,3}{⾛、⼀、⼀}
  \begin{Phonetics}{赶不上}{gan3 bu5 shang4}[][HSK 6]
    \definition{v.}{ficar para trás; ser incapaz de alcançar; não conseguir alcançar; não conseguir acompanhar | ser tarde demais (para fazer algo); (não) existir tempo suficiente (para fazer algo) |  deixar de ter; ser incapaz de encontrar ou ter a chance de encontrar; não encontrar; não encontrar (boa oportunidade) | não poder ser comparado a}
  \end{Phonetics}
\end{Entry}

\begin{Entry}{赶忙}{10,6}{⾛、⼼}
  \begin{Phonetics}{赶忙}{gan3 mang2}[][HSK 6]
    \definition{adv.}{imediatamente; com pressa; às pressas; rapidamente}
  \end{Phonetics}
\end{Entry}

\begin{Entry}{赶早}{10,6}{⾛、⽇}
  \begin{Phonetics}{赶早}{gan3zao3}
    \definition{adv.}{o mais breve possível | na primeira oportunidade | antes que seja tarde | quanto antes melhor}
  \end{Phonetics}
\end{Entry}

\begin{Entry}{赶快}{10,7}{⾛、⼼}
  \begin{Phonetics}{赶快}{gan3kuai4}[][HSK 3]
    \definition{adv.}{rapidamente; imediatamente; aproveite o momento e acelere o ritmo}
  \end{Phonetics}
\end{Entry}

\begin{Entry}{赶走}{10,7}{⾛、⾛}
  \begin{Phonetics}{赶走}{gan3zou3}
    \definition{v.}{expulsar | voltar atrás}
  \end{Phonetics}
\end{Entry}

\begin{Entry}{赶到}{10,8}{⾛、⼑}
  \begin{Phonetics}{赶到}{gan3 dao4}[][HSK 3]
    \definition{v.}{correr (para algum lugar); apressar-se}
  \end{Phonetics}
\end{Entry}

\begin{Entry}{赶往}{10,8}{⾛、⼻}
  \begin{Phonetics}{赶往}{gan3wang3}[][HSK 7-9]
    \definition{v.}{apressar-se para (algum lugar)}
  \end{Phonetics}
\end{Entry}

\begin{Entry}{赶赴}{10,9}{⾛、⾛}
  \begin{Phonetics}{赶赴}{gan3fu4}[][HSK 7-9]
    \definition{v.}{apressar-se para; correr para | apressar-se}
  \end{Phonetics}
\end{Entry}

\begin{Entry}{赶紧}{10,10}{⾛、⽷}
  \begin{Phonetics}{赶紧}{gan3jin3}[][HSK 3]
    \definition{adv.}{apressadamente; precipitadamente; às pressas; significa agir imediatamente, sem demora}
  \end{Phonetics}
\end{Entry}

\begin{Entry}{赶脚}{10,11}{⾛、⾁}
  \begin{Phonetics}{赶脚}{gan3jiao3}
    \definition{v.}{transportar mercadorias para ganhar a vida (especialmente de burro) | trabalhar como carroceiro ou porteiro}
  \end{Phonetics}
\end{Entry}

\begin{Entry}{赶跑}{10,12}{⾛、⾜}
  \begin{Phonetics}{赶跑}{gan3pao3}
    \definition{v.}{afastar | forçar a saída | repelir}
  \end{Phonetics}
\end{Entry}

\begin{Entry}{赶集}{10,12}{⾛、⾫}
  \begin{Phonetics}{赶集}{gan3ji2}
    \definition{v.}{ir a uma feira | ir ao mercado}
  \end{Phonetics}
\end{Entry}

\begin{Entry}{赶路}{10,13}{⾛、⾜}
  \begin{Phonetics}{赶路}{gan3lu4}
    \definition{v.}{apressar a jornada | apressar-se}
  \end{Phonetics}
\end{Entry}

%%%%%%%%%% 起 %%%%%%%%%%
\subsection*{起}

\begin{Entry}{起}{10}{⾛}
  \begin{Phonetics}{起}{qi3}[][HSK 1]
    \definition{clas.}{caso; instância | lote; grupo}
    \definition{prep.}{de; colocado antes de uma palavra de tempo ou lugar, indica um ponto de partida | por; colocado antes de uma palavra de lugar, indica um lugar por onde passou}
    \definition{v.}{levantar-se; ficar de pé| iniciar; lançar; deicar a posição original | subir; ascender | aparecer; levantar; crescer (bolhas, protuberâncias, brotoeja) | puxar para cima; puxar para fora; tirar o que está guardado ou incorporado | crescer; aumentar | esboçar; elaborar | construir; montar; estabelecer | receber (comprovante) | começar; iniciar; combina com 从 e 由; indica quando, onde e quem começou | buscar; pegar; usado após um verbo, indica movimento para cima | indicar se alguém tem força suficiente ou não; usado após um verbo, indica que a força é suficiente ou insuficiente | indicar que a ação envolve alguém ou algo; equivalente a 及 ou 到 | começar; iniciar; usado depois de um verbo, indica o início de uma ação | juntar; implodir; (informal) usado depois de um verbo, para unir coisas ou fechá-las}
  \seealsoref{从}{cong2}
  \seealsoref{到}{dao4}
  \seealsoref{及}{ji2}
  \seealsoref{由}{you2}
  \end{Phonetics}
\end{Entry}

\begin{Entry}{起飞}{10,3}{⾛、⾶}
  \begin{Phonetics}{起飞}{qi3fei1}[][HSK 2]
    \definition{v.}{decolar; levantar voo | crescer rapidamente; decolar; disparar; metáfora para o rápido desenvolvimento de negócios, economia, etc.}
  \end{Phonetics}
\end{Entry}

\begin{Entry}{起床}{10,7}{⾛、⼴}
  \begin{Phonetics}{起床}{qi3/chuang2}[][HSK 1]
    \definition{v.+compl.}{levantar-se; sair da cama; acordar e sair da cama (geralmente pela manhã); levantar-se da posição sentada, deitada ou deitada de bruços, ou sentar-se a partir da posição deitada}
  \end{Phonetics}
\end{Entry}

\begin{Entry}{起来}{10,7}{⾛、⽊}
  \begin{Phonetics}{起来}{qi3/lai2}[][HSK 1]
    \definition{v.+compl.}{levantar-se; passar de posições como deitado, sentado ou ajoelhado para ficar em pé | levantar-se; sair da cama | levantar-se; revoltar-se; rebelar-se; refere-se a ascensão, surgimento, levantamento, etc.}
  \end{Phonetics}
  \begin{Phonetics}{起来}{qi3lai5}
    \definition{v.aux.}{usado depois de um verbo para indicar movimento ascendente}
  \end{Phonetics}
  \begin{Phonetics}{起来}{qi5lai2}
    \definition{v.}{descrever resultados, retratar comportamentos, transmitir movimento}
  \end{Phonetics}
\end{Entry}

\begin{Entry}{起诉}{10,7}{⾛、⾔}
  \begin{Phonetics}{起诉}{qi3 su4}[][HSK 6]
    \definition{v.}{processar; entrar com uma ação judicial}
  \end{Phonetics}
\end{Entry}

\begin{Entry}{起到}{10,8}{⾛、⼑}
  \begin{Phonetics}{起到}{qi3 dao4}[][HSK 5]
    \definition{v.}{ter (um efeito motivador, etc.); desempenhar (um papel estabilizador, etc.)}
  \end{Phonetics}
\end{Entry}

\begin{Entry}{起码}{10,8}{⾛、⽯}
  \begin{Phonetics}{起码}{qi3ma3}[][HSK 5]
    \definition{adj.}{mínimo; elementar; rudimentar}
    \definition{adv.}{mínimamente; pelo menos;}
  \end{Phonetics}
\end{Entry}

\begin{Entry}{起点}{10,9}{⾛、⽕}
  \begin{Phonetics}{起点}{qi3 dian3}[][HSK 6]
    \definition[个]{s.}{ponto de partida (para o tempo ou local do início de algo); o lugar ou hora de início | ponto de partida (para o nível ou base de algo feito inicialmente); refere-se especificamente ao ponto de partida designado em um evento de pista}
  \end{Phonetics}
\end{Entry}

\begin{Entry}{起跳}{10,13}{⾛、⾜}
  \begin{Phonetics}{起跳}{qi3tiao4}
    \definition{v.}{(atletismo) decolar (no início de um salto) | (de preço, salário, etc.) começar (de um determinado nível)}
  \end{Phonetics}
\end{Entry}

%%%%%%%%%% 轿 %%%%%%%%%%
\subsection*{轿}

\begin{Entry}{轿}{10}{⾞}
  \begin{Phonetics}{轿}{jiao4}
    \definition{s.}{liteira; palanquim; cadeira de arruar}
  \end{Phonetics}
\end{Entry}

\begin{Entry}{轿车}{10,4}{⾞、⾞}
  \begin{Phonetics}{轿车}{jiao4che1}[][HSK 7-9]
    \definition[辆]{s.}{carruagem (puxada por cavalos); carruagens puxadas por animais com cortinas cobrindo os compartimentos de passageiros antigamente | ônibus; carro; sedã; um carro relativamente luxuoso e confortável, com teto e assentos para passageiros}
  \end{Phonetics}
\end{Entry}

%%%%%%%%%% 较 %%%%%%%%%%
\subsection*{较}

\begin{Entry}{较}{10}{⾞}
  \begin{Phonetics}{较}{jiao4}[][HSK 3]
    \definition{adj.}{claro; óbvio; evidente}
    \definition{adv.}{comparativamente; relativamente; razoavelmente; bastante; bastante}
    \definition{prep.}{usado para comparar características e graus; introduzir o objeto de comparação; equivalente a 比}
    \definition{v.}{comparar | disputar}
  \seealsoref{比}{bi3}
  \end{Phonetics}
\end{Entry}

\begin{Entry}{较劲}{10,7}{⾞、⼒}
  \begin{Phonetics}{较劲}{jiao4/jin4}[][HSK 7-9]
    \definition{v.+compl.}{exigir esforço extra | adequar a própria força a (competição de força; disputa de habilidade) | colocar-se contra alguém; ir um contra o outro}
  \end{Phonetics}
\end{Entry}

\begin{Entry}{较量}{10,12}{⾞、⾥}
  \begin{Phonetics}{较量}{jiao4liang4}[][HSK 7-9]
    \definition{v.}{realizar uma competição; medir a própria força com; determinar quem é superior ou inferior através da competição, da luta ou de outros meios | regatear; discutir; disputar; calcular}
  \end{Phonetics}
\end{Entry}

%%%%%%%%%% 辱 %%%%%%%%%%
\subsection*{辱}

\begin{Entry}{辱}{10}{⾠}
  \begin{Phonetics}{辱}{ru3}
    \definition*{s.}{Sobrenome: Ru}
    \definition{s.}{desgraça; desonra (oposto a 荣)}
    \definition{v.}{trazer desgraça (ou humilhação) para | trazer desgraça; ser uma desgraça para | estar em dívida (com alguém por uma gentileza) | humilhar; insultar}
  \seealsoref{荣}{rong2}
  \end{Phonetics}
\end{Entry}

\begin{Entry}{辱骂}{10,9}{⾠、⾺}
  \begin{Phonetics}{辱骂}{ru3ma4}
    \definition{v.}{insultar | abusar}
  \end{Phonetics}
\end{Entry}

%%%%%%%%%% 透 %%%%%%%%%%
\subsection*{透}

\begin{Entry}{透}{10}{⾡}
  \begin{Phonetics}{透}{tou4}[][HSK 4]
    \definition{adv.}{totalmente; completamente; minuciosamente | profundamente; extremamente}
    \definition{v.}{penetrar; passar através de; infiltrar-se através de | revelar; deixar transparecer; contar secretamente |mostrar; aparecer}
  \end{Phonetics}
\end{Entry}

\begin{Entry}{透支}{10,4}{⾡、⽀}
  \begin{Phonetics}{透支}{tou4zhi1}
    \definition{v.}{cheque especial (bancário) | saque a descoberto}
  \end{Phonetics}
\end{Entry}

\begin{Entry}{透气}{10,4}{⾡、⽓}
  \begin{Phonetics}{透气}{tou4qi4}
    \definition{v.}{respirar (sobre tecido, etc.) | fluir livremente (sobre ar) | respirar ar fresco | ventilar}
  \end{Phonetics}
\end{Entry}

\begin{Entry}{透水}{10,4}{⾡、⽔}
  \begin{Phonetics}{透水}{tou4shui3}
    \definition{adj.}{permeável}
    \definition{s.}{vazamento de água}
  \end{Phonetics}
\end{Entry}

\begin{Entry}{透过}{10,6}{⾡、⾡}
  \begin{Phonetics}{透过}{tou4guo4}
    \definition{v.}{passar através | penetrar}
  \end{Phonetics}
\end{Entry}

\begin{Entry}{透彻}{10,7}{⾡、⼻}
  \begin{Phonetics}{透彻}{tou4che4}
    \definition{adj.}{minucioso | incisivo | penetrante}
  \end{Phonetics}
\end{Entry}

\begin{Entry}{透明}{10,8}{⾡、⽇}
  \begin{Phonetics}{透明}{tou4ming2}[][HSK 4]
    \definition{adj.}{transparente; diáfano; capaz de transmitir luz | evidente; transparente; situação ou assunto que seja aberto e não oculto | transparente; diáfano; indica pureza, ausência de impurezas}
  \end{Phonetics}
\end{Entry}

\begin{Entry}{透顶}{10,8}{⾡、⾴}
  \begin{Phonetics}{透顶}{tou4ding3}
    \definition{adv.}{completamente}
  \end{Phonetics}
\end{Entry}

\begin{Entry}{透亮}{10,9}{⾡、⼇}
  \begin{Phonetics}{透亮}{tou4liang4}
    \definition{adj.}{brilhante | claro como cristal}
  \end{Phonetics}
\end{Entry}

\begin{Entry}{透辟}{10,13}{⾡、⾟}
  \begin{Phonetics}{透辟}{tou4pi4}
    \definition{adj.}{incisivo | penetrante}
  \end{Phonetics}
\end{Entry}

\begin{Entry}{透澈}{10,15}{⾡、⽔}
  \begin{Phonetics}{透澈}{tou4che4}
    \variantof{透彻}
  \end{Phonetics}
\end{Entry}

\begin{Entry}{透露}{10,21}{⾡、⾬}
  \begin{Phonetics}{透露}{tou4lu4}[][HSK 6]
    \definition{v.}{vazar; revelar; expor; divulgar; contar deliberadamente um segredo a alguém; revelar um certo significado}
  \end{Phonetics}
\end{Entry}

%%%%%%%%%% 逐 %%%%%%%%%%
\subsection*{逐}

\begin{Entry}{逐}{10}{⾡}
  \begin{Phonetics}{逐}{zhu2}
    \definition{prep.}{um por um; um a um}[逐月===mês a mês]
    \definition{v.}{ir atrás de; perseguir | expulsar; banir | correr atrás; alcançar}
  \end{Phonetics}
\end{Entry}

\begin{Entry}{逐步}{10,7}{⾡、⽌}
  \begin{Phonetics}{逐步}{zhu2bu4}[][HSK 4]
    \definition{adv.}{gradualmente; passo a passo; progressivamente}
  \end{Phonetics}
\end{Entry}

\begin{Entry}{逐渐}{10,11}{⾡、⽔}
  \begin{Phonetics}{逐渐}{zhu2jian4}[][HSK 4]
    \definition{adv.}{gradualmente; aos poucos; por etapas; indica mudanças lentas e ordenadas no grau, na quantidade, etc.}
  \end{Phonetics}
\end{Entry}

%%%%%%%%%% 递 %%%%%%%%%%
\subsection*{递}

\begin{Entry}{递}{10}{⾡}
  \begin{Phonetics}{递}{di4}[][HSK 5]
    \definition{adv.}{na ordem correta; sucessivamente}
    \definition{v.}{entregar; passar; dar; transmitir}
  \end{Phonetics}
\end{Entry}

\begin{Entry}{递交}{10,6}{⾡、⼇}
  \begin{Phonetics}{递交}{di4jiao1}[][HSK 7-9]
    \definition{v.}{apresentar; submeter; entregar; entregar pessoalmente}
  \end{Phonetics}
\end{Entry}

\begin{Entry}{递给}{10,9}{⾡、⽷}
  \begin{Phonetics}{递给}{di4 gei3}[][HSK 5]
    \definition{v.}{entregar algo a alguém; passar itens ou coisas para outras pessoas}
  \end{Phonetics}
\end{Entry}

%%%%%%%%%% 途 %%%%%%%%%%
\subsection*{途}

\begin{Entry}{途}{10}{⾡}
  \begin{Phonetics}{途}{tu2}
    \definition[条]{s.}{caminho; estrada; rota | jornada; caminho}
  \end{Phonetics}
\end{Entry}

\begin{Entry}{途中}{10,4}{⾡、⼁}
  \begin{Phonetics}{途中}{tu2 zhong1}[][HSK 4]
    \definition[家]{adv.}{no caminho; ao longo do caminho}
  \end{Phonetics}
\end{Entry}

\begin{Entry}{途径}{10,8}{⾡、⼻}
  \begin{Phonetics}{途径}{tu2jing4}[][HSK 6]
    \definition[种,条,个]{s.}{caminho; canal; metaforicamente falando, uma maneira ou método de resolver um problema ou fazer algo}
  \end{Phonetics}
\end{Entry}

%%%%%%%%%% 逗 %%%%%%%%%%
\subsection*{逗}

\begin{Entry}{逗}{10}{⾡}
  \begin{Phonetics}{逗}{dou4}[][HSK 7-9]
    \definition{adj.}{engraçado; divertido}
    \definition{s.}{ligeira pausa na leitura; antigamente, referia-se ao lugar em um artigo onde o significado de uma frase não era completado e uma pausa era necessária durante a leitura}
    \definition{v.}{provocar; brincar com | divertir; provocar (risos, etc.) | ficar; parar}
  \end{Phonetics}
\end{Entry}

%%%%%%%%%% 通 %%%%%%%%%%
\subsection*{通}

\begin{Entry}{通}{10}{⾡}
  \begin{Phonetics}{通}{tong1}[][HSK 2]
    \definition*{s.}{Sobrenome: Tong}
    \definition{adj.}{lógico; coerente | geral; comum | tudo; inteiro | aberto; através de | total}
    \definition{clas.}{(antigo) usado para cartas, telegramas, documentos oficiais, etc.}
    \definition{s.}{autoridade; especialista}
    \definition{suf.}{especialista}
    \definition{v.}{abrir; atravessar | abrir ou limpar cutucando ou espetando | levar a; ir a | conectar; comunicar | notificar; informar | compreender; saber | cutucar; dar uma pancada | transmitir; conectar; interagir | dominar; compreender; entender}
  \end{Phonetics}
  \begin{Phonetics}{通}{tong4}
    \definition{clas.}{usado para uma atividade, tomada em sua totalidade (discurso de abuso, período de reprodução de música, bebedeira, etc.)}
  \end{Phonetics}
\end{Entry}

\begin{Entry}{通用}{10,5}{⾡、⽤}
  \begin{Phonetics}{通用}{tong1yong4}[][HSK 5]
    \definition[家]{adj.}{de uso comum; universal; (em um determinado âmbito) de uso generalizado | intercambiável; alguns caracteres chineses com grafia diferente, mas pronúncia igual, podem ser usados indistintamente (alguns limitados a um determinado significado)}
  \end{Phonetics}
\end{Entry}

\begin{Entry}{通讯}{10,5}{⾡、⾔}
  \begin{Phonetics}{通讯}{tong1xun4}[][HSK 6]
    \definition[个,种]{s.}{relatório; comunicação; boletim informativo; correspondência; reportagem; despacho de notícias; artigos que relatam fatos objetivos ou números típicos de forma detalhada e vívida}
    \definition{v.}{usar equipamentos de telecomunicações para transmitir mensagens}
  \end{Phonetics}
\end{Entry}

\begin{Entry}{通红}{10,6}{⾡、⽷}
  \begin{Phonetics}{通红}{tong1 hong2}[][HSK 6]
    \definition{adj.}{muito vermelho; vermelho por completo}
  \end{Phonetics}
\end{Entry}

\begin{Entry}{通行}{10,6}{⾡、⾏}
  \begin{Phonetics}{通行}{tong1 xing2}[][HSK 6]
    \definition{adj.}{atual; geral}
    \definition{v.}{passar (ou ir) através; passar por; atravessar | prevalecer; predominar; ser corrente | (pedestres, veículos, etc.) passar na linha de trânsito}
  \end{Phonetics}
\end{Entry}

\begin{Entry}{通观}{10,6}{⾡、⾒}
  \begin{Phonetics}{通观}{tong1guan1}
    \definition{v.}{ter uma visão geral de algo}
  \end{Phonetics}
\end{Entry}

\begin{Entry}{通过}{10,6}{⾡、⾡}
  \begin{Phonetics}{通过}{tong1guo4}[][HSK 2]
    \definition{prep.}{por; através de; por meio de; por meio de; meios, métodos, etc. para introduzir ações}
    \definition{v.}{atravessar; passar por; transitar | aprovar; adotar | solicitar o consentimento ou aprovação de}
  \end{Phonetics}
\end{Entry}

\begin{Entry}{通报}{10,7}{⾡、⼿}
  \begin{Phonetics}{通报}{tong1 bao4}[][HSK 6]
    \definition[份]{s.}{circular | boletim; jornal; publicação | sumário; notificação para informações gerais}
    \definition{v.}{circular um aviso (aviso por escrito) | notificar; dar informações com; compartilhar informações com}
  \end{Phonetics}
\end{Entry}

\begin{Entry}{通识}{10,7}{⾡、⾔}
  \begin{Phonetics}{通识}{tong1shi2}
    \definition{s.}{conhecimento comum | erudição | conhecimento geral | amplamente conhecido}
  \end{Phonetics}
\end{Entry}

\begin{Entry}{通知}{10,8}{⾡、⽮}
  \begin{Phonetics}{通知}{tong1zhi1}[][HSK 2]
    \definition[份,个,张]{s.}{aviso; circular; notificação por escrito ou verbal}
    \definition{v.}{aconselhar; notificar; informar; dar aviso prévio}
  \end{Phonetics}
\end{Entry}

\begin{Entry}{通知书}{10,8,4}{⾡、⽮、⼄}
  \begin{Phonetics}{通知书}{tong1 zhi1 shu1}[][HSK 4]
    \definition[份]{s.}{aviso; observação; notificação}
  \end{Phonetics}
\end{Entry}

\begin{Entry}{通话}{10,8}{⾡、⾔}
  \begin{Phonetics}{通话}{tong1 hua4}[][HSK 6]
    \definition{v.}{comunicar por telefone | conversar; comunicar; falar em uma língua que ambos possam entender}
  \end{Phonetics}
\end{Entry}

\begin{Entry}{通信}{10,9}{⾡、⼈}
  \begin{Phonetics}{通信}{tong1/xin4}[][HSK 3]
    \definition{v.+compl.}{corresponder; comunicar por carta; comunicar situações e informações escrevendo cartas | transmitir (ou transportar) mensagem; passar (ou transmitir) informação; usar ondas de rádio e outros sinais para transmitir texto, imagens, etc.}
  \end{Phonetics}
\end{Entry}

\begin{Entry}{通常}{10,11}{⾡、⼱}
  \begin{Phonetics}{通常}{tong1chang2}[][HSK 3]
    \definition{adj.}{usual; normal; geral}
    \definition{adv.}{habitualmente; usualmente; geralmente; ordinariamente}
  \end{Phonetics}
\end{Entry}

\begin{Entry}{通道}{10,12}{⾡、⾡}
  \begin{Phonetics}{通道}{tong1 dao4}[][HSK 6]
    \definition[条,个]{s.}{acesso; corredor; passagem; caminhos que levam ao exterior de teatros, minas, etc. | passagem; via pública}
  \end{Phonetics}
\end{Entry}

\begin{Entry}{通牒}{10,13}{⾡、⽚}
  \begin{Phonetics}{通牒}{tong1die2}
    \definition{s.}{nota diplomática}
  \end{Phonetics}
\end{Entry}

%%%%%%%%%% 逛 %%%%%%%%%%
\subsection*{逛}

\begin{Entry}{逛}{10}{⾡}
  \begin{Phonetics}{逛}{guang4}[][HSK 4]
    \definition{v.}{perambular; passear; vaguear}
  \end{Phonetics}
\end{Entry}

%%%%%%%%%% 逞 %%%%%%%%%%
\subsection*{逞}

\begin{Entry}{逞}{10}{⾡}
  \begin{Phonetics}{逞}{cheng3}
    \definition*{s.}{Sobrenome: Cheng}
    \definition{v.}{exibir-se; ostentar; gabar-se | executar (um plano maligno); ter sucesso (em um esquema) | saciar; satisfazer; dar rédea solta a; deliciar-se}
  \end{Phonetics}
\end{Entry}

\begin{Entry}{逞能}{10,10}{⾡、⾁}
  \begin{Phonetics}{逞能}{cheng3/neng2}[][HSK 7-9]
    \definition{v.+compl.}{exibir a própria habilidade (ou capacidade); exibir a própria capacidade | mostrar sua habilidade ou capacidade}
  \end{Phonetics}
\end{Entry}

\begin{Entry}{逞强}{10,12}{⾡、⼸}
  \begin{Phonetics}{逞强}{cheng3/qiang2}[][HSK 7-9]
    \definition{v.+compl.}{exibir-se; ser orgulhoso; ser teimoso; ostentar a própria superioridade}
  \end{Phonetics}
\end{Entry}

%%%%%%%%%% 速 %%%%%%%%%%
\subsection*{速}

\begin{Entry}{速}{10}{⾡}
  \begin{Phonetics}{速}{su4}
    \definition{adj.}{rápido; veloz}
    \definition{s.}{velocidade}
    \definition{v.aux.}{convidar}
  \end{Phonetics}
\end{Entry}

\begin{Entry}{速度}{10,9}{⾡、⼴}
  \begin{Phonetics}{速度}{su4du4}[][HSK 3]
    \definition[个,种]{s.}{velocidade; taxa; ritmo; andamento; uma quantidade física que indica a velocidade e a direção do movimento de um objeto, ou seja, a distância que um objeto percorre em uma direção por unidade de tempo | velocidade; rapidez; geralmente se refere ao grau de velocidade}
  \end{Phonetics}
\end{Entry}

%%%%%%%%%% 造 %%%%%%%%%%
\subsection*{造}

\begin{Entry}{造}{10}{⾡}
  \begin{Phonetics}{造}{zao4}[][HSK 3]
    \definition*{s.}{Sobrenome: Zao}
    \definition{clas.}{para colheitas ou número de colheitas de safras}
    \definition{s.}{uma das duas partes em um acordo legal ou um processo judicial | (dialeto) colheita; safra | realizações; conquistas}
    \definition{v.}{fazer; construir; criar; produzir | forjar; inventar | correr solto; bagunçar as coisas | expor sem restrições |  treinar; educar | fabricar | alcançar; atingir}
  \end{Phonetics}
\end{Entry}

\begin{Entry}{造成}{10,6}{⾡、⼽}
  \begin{Phonetics}{造成}{zao4cheng2}[][HSK 3]
    \definition{v.}{criar; dar origem a; provocar; causar (geralmente se refere a resultados negativos)}
  \end{Phonetics}
\end{Entry}

\begin{Entry}{造型}{10,9}{⾡、⼟}
  \begin{Phonetics}{造型}{zao4xing2}[][HSK 4]
    \definition[个,种]{s.}{molde; modelo; formato; forma; moldagem}
    \definition{v.}{modelar; moldar}
  \end{Phonetics}
\end{Entry}

%%%%%%%%%% 逢 %%%%%%%%%%
\subsection*{逢}

\begin{Entry}{逢}{10}{⾡}
  \begin{Phonetics}{逢}{feng2}[][HSK 7-9]
    \definition*{s.}{Sobrenome: Feng}
    \definition{v.}{encontrar; vir até; encontrar-se por acaso}
  \end{Phonetics}
\end{Entry}

%%%%%%%%%% 部 %%%%%%%%%%
\subsection*{部}

\begin{Entry}{部}{10}{⾢}
  \begin{Phonetics}{部}{bu4}[][HSK 3]
    \definition*{s.}{Sobrenome: Bu}
    \definition{clas.}{usado para obras de literatura, livros, filmes, etc.}
    \definition[根]{s.}{parte; seção | unidade; ministério; departamento; conselho | sede; matriz; quartel general | tropas; forças | divisão; região}
    \definition{v.}{comandar; liderar}
  \end{Phonetics}
\end{Entry}

\begin{Entry}{部下}{10,3}{⾢、⼀}
  \begin{Phonetics}{部下}{bu4xia4}
    \definition{s.}{subordinado | tropas sob comando de alguém}
  \end{Phonetics}
\end{Entry}

\begin{Entry}{部门}{10,3}{⾢、⾨}
  \begin{Phonetics}{部门}{bu4men2}[][HSK 3]
    \definition[个]{s.}{departamento; ramo; classe; seção; partes ou unidades que compõem um todo}
  \end{Phonetics}
\end{Entry}

\begin{Entry}{部分}{10,4}{⾢、⼑}
  \begin{Phonetics}{部分}{bu4fen5}[][HSK 2]
    \definition[个,些,快,份]{s.}{parte; seção; porção; parte do todo; alguns indivíduos dentro do todo | ramo; parte separada de um sistema ou entidade}
  \end{Phonetics}
\end{Entry}

\begin{Entry}{部长}{10,4}{⾢、⾧}
  \begin{Phonetics}{部长}{bu4 zhang3}[][HSK 3]
    \definition[个,位,名]{s.}{ministro; chefe de departamento; um alto funcionário do estado encarregado pelo chefe de estado ou chefe executivo do governo da gestão das atividades governamentais de um departamento | chefe de seção; líder tribal}
  \end{Phonetics}
\end{Entry}

\begin{Entry}{部队}{10,4}{⾢、⾩}
  \begin{Phonetics}{部队}{bu4 dui4}[][HSK 6]
    \definition[支,个]{s.}{militar; exército; forças armadas | tropas; refere-se a uma parte do exército}
  \end{Phonetics}
\end{Entry}

\begin{Entry}{部件}{10,6}{⾢、⼈}
  \begin{Phonetics}{部件}{bu4jian4}[][HSK 7-9]
    \definition[个]{s.}{peças; partes; componentes; um componente de uma máquina, montado a partir de várias partes | partes; componentes (para caracteres chineses); uma unidade de caracteres chineses composta por traços, por exemplo, 氵, 礻, 口 são todos componentes de caracteres chineses}
  \end{Phonetics}
\end{Entry}

\begin{Entry}{部位}{10,7}{⾢、⼈}
  \begin{Phonetics}{部位}{bu4wei4}[][HSK 5]
    \definition{s.}{lugar; posição (usado principalmente para o corpo humano)}
  \end{Phonetics}
\end{Entry}

\begin{Entry}{部族}{10,11}{⾢、⽅}
  \begin{Phonetics}{部族}{bu4zu2}
    \definition{adj.}{tribal}
    \definition{s.}{tribo}
  \end{Phonetics}
\end{Entry}

\begin{Entry}{部属}{10,12}{⾢、⼫}
  \begin{Phonetics}{部属}{bu4shu3}
    \definition{s.}{afiliado a um ministério | subordinado | tropas sob comando de alguém}
  \end{Phonetics}
\end{Entry}

\begin{Entry}{部署}{10,13}{⾢、⽹}
  \begin{Phonetics}{部署}{bu4shu3}[][HSK 7-9]
    \definition{v.}{organizar; implantar; dispor; organizar ou dispor de maneira planejada (usado principalmente em grandes aspectos)}
  \end{Phonetics}
\end{Entry}

%%%%%%%%%% 都 %%%%%%%%%%
\subsection*{都}

\begin{Entry}{都}{10}{⾢}
  \begin{Phonetics}{都}{dou1}[][HSK 1]
    \definition{adv.}{todos; representa a soma total | apenas por causa de; usado em conjunto com a palavra 是, explica o motivo | mesmo; até; indicativo de ênfase | já; significa 已经}
  \seealsoref{是}{shi4}
  \seealsoref{已经}{yi3jing1}
  \end{Phonetics}
  \begin{Phonetics}{都}{du1}
    \definition*{s.}{Sobrenome: Du}
    \definition[座]{s.}{capital | cidade grande; metrópole}
  \end{Phonetics}
\end{Entry}

\begin{Entry}{都市}{10,5}{⾢、⼱}
  \begin{Phonetics}{都市}{du1 shi4}[][HSK 6]
    \definition[个]{s.}{cidade grande; grandes cidades}
  \end{Phonetics}
\end{Entry}

\begin{Entry}{都会}{10,6}{⾢、⼈}
  \begin{Phonetics}{都会}{du1hui4}[][HSK 7-9]
    \definition{s.}{cidade; metrópole}
  \end{Phonetics}
\end{Entry}

%%%%%%%%%% 配 %%%%%%%%%%
\subsection*{配}

\begin{Entry}{配}{10}{⾣}
  \begin{Phonetics}{配}{pei4}[][HSK 3]
    \definition{adj.}{adequado; bem combinado}
    \definition{s.}{cônjuge (geralmente referindo-se a uma esposa)}
    \definition{v.}{unir-se em matrimônio | (animais) acasalar; copular | compor; combinar; mesclar; amalgamar; misturar | distribuir de forma planejada; repartir | encontrar algo para encaixar ou substituir outra coisa; compensar as partes faltantes de acordo com certos padrões | combinar; harmonizar com; estar em harmonia com | exilar; banir; nos tempos antigos, referia-se ao exílio de criminosos}
    \definition{v.aux.}{adequar-se a; merecer; ser qualificado; ser digno de}
  \end{Phonetics}
\end{Entry}

\begin{Entry}{配合}{10,6}{⾣、⼝}
  \begin{Phonetics}{配合}{pei4he2}[][HSK 3]
    \definition{v.}{cooperar; coordenar; todas as partes trabalham juntas para concluir tarefas comuns}
  \end{Phonetics}
\end{Entry}

\begin{Entry}{配备}{10,8}{⾣、⼡}
  \begin{Phonetics}{配备}{pei4bei4}[][HSK 5]
    \definition{s.}{equipamento; material; conjunto completo de utensílios, etc.}
    \definition{v.}{fornecer; alocar; equipar; distribuir conforme necessário | posicionar; dispor (tropas, etc.)}
  \end{Phonetics}
\end{Entry}

\begin{Entry}{配套}{10,10}{⾣、⼤}
  \begin{Phonetics}{配套}{pei4/tao4}[][HSK 5]
    \definition{v.+compl.}{formar um conjunto ou sistema completo; combinar vários elementos relacionados em um conjunto completo}
  \end{Phonetics}
\end{Entry}

\begin{Entry}{配置}{10,13}{⾣、⽹}
  \begin{Phonetics}{配置}{pei4 zhi4}[][HSK 6]
    \definition{s.}{configuração; refere-se especificamente à seleção e combinação de software e hardware em várias partes de computadores, carros, etc.}
    \definition{v.}{implantar; alocar; dispor (tropas, etc.); equipar e configurar}
  \end{Phonetics}
\end{Entry}

%%%%%%%%%% 酒 %%%%%%%%%%
\subsection*{酒}

\begin{Entry}{酒}{10}{⾣}
  \begin{Phonetics}{酒}{jiu3}[][HSK 2]
    \definition*{s.}{Sobrenome: Jiu}
    \definition[口,杯,瓶,罐,桶,缸]{s.}{bebida alcoólica; vinho; licor; bebidas destiladas}
  \end{Phonetics}
\end{Entry}

\begin{Entry}{酒水}{10,4}{⾣、⽔}
  \begin{Phonetics}{酒水}{jiu3 shui3}[][HSK 6]
    \definition{s.}{bebidas; bebidas e álcool | Dialeto: festa; banquete}
  \end{Phonetics}
\end{Entry}

\begin{Entry}{酒吧}{10,7}{⾣、⼝}
  \begin{Phonetics}{酒吧}{jiu3ba1}[][HSK 4]
    \definition[家,间]{s.}{bar; \emph{pub}; um local onde são vendidas bebidas alcoólicas e onde as pessoas podem beber e conversar, referindo-se principalmente a um restaurante ou hotel de estilo ocidental especializado na venda de bebidas alcoólicas.}
  \end{Phonetics}
\end{Entry}

\begin{Entry}{酒店}{10,8}{⾣、⼴}
  \begin{Phonetics}{酒店}{jiu3 dian4}[][HSK 2]
    \definition[家,个]{s.}{hotel; Estabelecimento comercial que oferece hospedagem e alimentação aos hóspedes | restaurante}
  \end{Phonetics}
\end{Entry}

\begin{Entry}{酒鬼}{10,9}{⾣、⿁}
  \begin{Phonetics}{酒鬼}{jiu3gui3}[][HSK 5]
    \definition[个]{s.}{bebedor de vinho; beberrão; ébrio | alcoólatra}
  \end{Phonetics}
\end{Entry}

\begin{Entry}{酒馆}{10,11}{⾣、⾷}
  \begin{Phonetics}{酒馆}{jiu3guan3}
    \definition{s.}{bar | taverna | adega}
  \end{Phonetics}
\end{Entry}

\begin{Entry}{酒楼}{10,13}{⾣、⽊}
  \begin{Phonetics}{酒楼}{jiu3lou2}[][HSK 7-9]
    \definition[座,家]{s.}{restaurante (em nomes de restaurantes)}[广东酒楼===Restaurante Guangdong]
  \end{Phonetics}
\end{Entry}

\begin{Entry}{酒精}{10,14}{⾣、⽶}
  \begin{Phonetics}{酒精}{jiu3jing1}[][HSK 7-9]
    \definition{s.}{álcool; álcool etílico; etanol}
  \end{Phonetics}
\end{Entry}

%%%%%%%%%% 钱 %%%%%%%%%%
\subsection*{钱}

\begin{Entry}{钱}{10}{⾦}
  \begin{Phonetics}{钱}{qian2}[][HSK 1]
    \definition*{s.}{Sobrenome: Qian}
    \definition{clas.}{qian, uma unidade de peso (=5 gramas) | qian, uma unidade de peso (um décimo de um tael 两)}
    \definition[笔]{s.}{dinheiro; riqueza; bens | moeda de cobre; dinheiro | objeto em forma de moeda de cobre | fundo; montante | dinheiro guardado ou gasto para algum fim específico (geralmente se refere a quantias significativas de dinheiro que entram e saem de órgãos públicos, organizações, etc.)}
  \seealsoref{两}{liang3}
  \end{Phonetics}
\end{Entry}

\begin{Entry}{钱包}{10,5}{⾦、⼓}
  \begin{Phonetics}{钱包}{qian2 bao1}[][HSK 1]
    \definition[个]{s.}{carteira; bolsa; bolsa de dinheiro}
  \end{Phonetics}
\end{Entry}

%%%%%%%%%% 钻 %%%%%%%%%%
\subsection*{钻}

\begin{Entry}{钻}{10}{⾦}
  \begin{Phonetics}{钻}{zuan1}
    \definition{v.}{furar; perfurar; girar um objeto pontiagudo para perfurar outro objeto | perfurar; entrar; penetrar; passar por | aprofundar-se; estudar intensivamente; fazer um estudo penetrante de | buscar ganho pessoal; tramar; refere-se a esquemas}
  \end{Phonetics}
  \begin{Phonetics}{钻}{zuan4}[][HSK 6]
    \definition[把]{s.}{broca; pua; sonda; existem muitos tipos de ferramentas para perfuração, incluindo manivela, elétrica e pneumática | joia; diamante}
    \definition{v.}{furar; perfurar;  girar um objeto pontiagudo para perfurar outro objeto}
  \end{Phonetics}
\end{Entry}

\begin{Entry}{钻石}{10,5}{⾦、⽯}
  \begin{Phonetics}{钻石}{zuan4shi2}
    \definition[颗]{s.}{diamante}
  \end{Phonetics}
\end{Entry}

\begin{Entry}{钻戒}{10,7}{⾦、⼽}
  \begin{Phonetics}{钻戒}{zuan4jie4}
    \definition[枚]{s.}{anel de diamante}
  \end{Phonetics}
\end{Entry}

%%%%%%%%%% 钿 %%%%%%%%%%
\subsection*{钿}

\begin{Entry}{钿}{10}{⾦}
  \begin{Phonetics}{钿}{dian4}
    \definition{s.}{ornamento incrustado antigo em forma de flor | enfeite de cabelo feminino com flores douradas | incrustação de madrepérola; um padrão incrustado com conchas de caracóis em madeira e laca}
    \definition{v.}{incrustar com ouro, prata, etc.}
  \end{Phonetics}
  \begin{Phonetics}{钿}{tian2}
    \definition{s.}{(dialeto) moeda | dinheiro; moeda | uma quantia de dinheiro}
  \end{Phonetics}
\end{Entry}

%%%%%%%%%% 铁 %%%%%%%%%%
\subsection*{铁}

\begin{Entry}{铁}{10}{⾦}
  \begin{Phonetics}{铁}{tie3}[][HSK 3]
    \definition*{s.}{Sobrenome: Tie}
    \definition{adj.}{duro; forte; sólido como ferro; metáfora para natureza dura; vontade forte | violento | inabalável; inalterável; determinado; metáfora para violência ou crueldade}
    \definition{s.}{ferro (Fe) | arma; armamento; refere-se a facas, armas de fogo, etc.}
    \definition{v.}{resolver; determinar}
  \end{Phonetics}
\end{Entry}

\begin{Entry}{铁轨}{10,6}{⾦、⾞}
  \begin{Phonetics}{铁轨}{tie3gui3}
    \definition[根]{s.}{trilho | trilho ferroviário}
  \end{Phonetics}
\end{Entry}

\begin{Entry}{铁路}{10,13}{⾦、⾜}
  \begin{Phonetics}{铁路}{tie3 lu4}[][HSK 3]
    \definition[条,公里]{s.}{ferrovia; estrada de ferro; uma estrada com trilhos de aço dispostos no leito da estrada para a circulação de trens}
  \end{Phonetics}
\end{Entry}

%%%%%%%%%% 铃 %%%%%%%%%%
\subsection*{铃}

\begin{Entry}{铃}{10}{⾦}
  \begin{Phonetics}{铃}{ling2}[][HSK 5]
    \definition[串,个]{s.}{sino; instrumento musical feito de metal | objetos em forma de sino | cápsula; botão; broto}
  \end{Phonetics}
\end{Entry}

\begin{Entry}{铃声}{10,7}{⾦、⼠}
  \begin{Phonetics}{铃声}{ling2 sheng1}[][HSK 5]
    \definition{s.}{o tilintar de sinos; o som de um sino tocando}
  \end{Phonetics}
\end{Entry}

%%%%%%%%%% 铅 %%%%%%%%%%
\subsection*{铅}

\begin{Entry}{铅}{10}{⾦}
  \begin{Phonetics}{铅}{qian1}
    \definition[根,盒]{s.}{chumbo (Pb) | grafite (em um lápis); grafite preta |}
  \end{Phonetics}
\end{Entry}

\begin{Entry}{铅笔}{10,10}{⾦、⽵}
  \begin{Phonetics}{铅笔}{qian1bi3}[][HSK 6]
    \definition[支,盒,种,枝,杆]{s.}{lápis; canetas com pontas de grafite ou argila pigmentada}
  \end{Phonetics}
\end{Entry}

%%%%%%%%%% 阅 %%%%%%%%%%
\subsection*{阅}

\begin{Entry}{阅}{10}{⾨}
  \begin{Phonetics}{阅}{yue4}
    \definition{v.}{ler; repassar; examinar | revisar; inspecionar | experimentar; passar por}
  \end{Phonetics}
\end{Entry}

\begin{Entry}{阅兵式}{10,7,6}{⾨、⼋、⼷}
  \begin{Phonetics}{阅兵式}{yue4bing1shi4}
    \definition{s.}{parada militar; desfile militar}
  \end{Phonetics}
\end{Entry}

\begin{Entry}{阅览室}{10,9,9}{⾨、⾒、⼧}
  \begin{Phonetics}{阅览室}{yue4 lan3 shi4}[][HSK 5]
    \definition[个,间]{s.}{sala de leitura; a biblioteca dispõe de salas para leitura e pesquisa, equipadas com mesas e cadeiras adequadas, livros, jornais, revistas, etc.}
  \end{Phonetics}
\end{Entry}

\begin{Entry}{阅读}{10,10}{⾨、⾔}
  \begin{Phonetics}{阅读}{yue4du2}[][HSK 4]
    \definition{v.}{ler; examinar; olhar (livros, jornais, etc.) e entender seu conteúdo}
  \end{Phonetics}
\end{Entry}

\begin{Entry}{阅读广度}{10,10,3,9}{⾨、⾔、⼴、⼴}
  \begin{Phonetics}{阅读广度}{yue4du2guang3du4}
    \definition{s.}{intervalo de leitura}
  \end{Phonetics}
\end{Entry}

\begin{Entry}{阅读时间}{10,10,7,7}{⾨、⾔、⽇、⾨}
  \begin{Phonetics}{阅读时间}{yue4 du2 shi2 jian1}
    \definition{s.}{tempo de leitura}
  \end{Phonetics}
\end{Entry}

\begin{Entry}{阅读理解}{10,10,11,13}{⾨、⾔、⽟、⾓}
  \begin{Phonetics}{阅读理解}{yue4du2li3jie3}
    \definition{s.}{compreensão de leitura}
  \end{Phonetics}
\end{Entry}

\begin{Entry}{阅读装置}{10,10,12,13}{⾨、⾔、⾐、⽹}
  \begin{Phonetics}{阅读装置}{yue4du2zhuang1zhi4}
    \definition{s.}{dispositivo de leitura (por exemplo, para códigos de barras, etiquetas RFID, etc.)}
  \end{Phonetics}
\end{Entry}

\begin{Entry}{阅读障碍}{10,10,13,13}{⾨、⾔、⾩、⽯}
  \begin{Phonetics}{阅读障碍}{yue4du2zhang4ai4}
    \definition{s.}{dislexia}
  \end{Phonetics}
\end{Entry}

\begin{Entry}{阅读器}{10,10,16}{⾨、⾔、⼝}
  \begin{Phonetics}{阅读器}{yue4du2qi4}
    \definition{s.}{leitor (\emph{software})}
  \end{Phonetics}
\end{Entry}

%%%%%%%%%% 陪 %%%%%%%%%%
\subsection*{陪}

\begin{Entry}{陪}{10}{⾩}
  \begin{Phonetics}{陪}{pei2}[][HSK 5]
    \definition{v.}{servir; acompanhar; cuidar; fazer companhia a alguém | auxiliar; ajudar}
  \end{Phonetics}
\end{Entry}

\begin{Entry}{陪同}{10,6}{⾩、⼝}
  \begin{Phonetics}{陪同}{pei2 tong2}[][HSK 6]
    \definition{v.}{acompanhar; acompanhar alguém para fazer uma atividade ou trabalhar junto}
  \end{Phonetics}
\end{Entry}

%%%%%%%%%% 陵 %%%%%%%%%%
\subsection*{陵}

\begin{Entry}{陵}{10}{⾩}
  \begin{Phonetics}{陵}{ling2}
    \definition*{s.}{Sobrenome: Ling}
    \definition{s.}{colina; monte | túmulo imperial; mausoléu}
    \definition{v.}{(literário) intimidar; violar}
  \end{Phonetics}
\end{Entry}

\begin{Entry}{陵园}{10,7}{⾩、⼞}
  \begin{Phonetics}{陵园}{ling2yuan2}
    \definition{s.}{cemitério}
  \end{Phonetics}
\end{Entry}

%%%%%%%%%% 陷 %%%%%%%%%%
\subsection*{陷}

\begin{Entry}{陷}{10}{⾩}
  \begin{Phonetics}{陷}{xian4}
    \definition[个]{s.}{armadilha; cilada | defeito | deficiência; desvantagem}
    \definition{v.}{ficar preso (ou atolado); enredar | afundar; desabar | acusar falsamente; incriminar; armar | (de uma cidade, etc.) ser capturado; cair | ser enquadrado; ser capturado}
  \end{Phonetics}
\end{Entry}

\begin{Entry}{陷入}{10,2}{⾩、⼊}
  \begin{Phonetics}{陷入}{xian4ru4}[][HSK 6]
    \definition{v.}{afundar em; cair em; cair em uma situação desfavorável | estar perdido em; estar profundamente em; estar imerso em; metaforicamente, estar profundamente imerso em (uma situação ou pensamento) | estar atolado (lama fofa, areia, etc.)}
  \end{Phonetics}
\end{Entry}

%%%%%%%%%% 难 %%%%%%%%%%
\subsection*{难}

\begin{Entry}{难}{10}{⾫}
  \begin{Phonetics}{难}{nan2}[][HSK 1]
    \definition{adj.}{difícil; duro; problemático (oposto a 易) | dificilmente possível; inevitável | ruim; desagradável | problemático; improvável}
    \definition{s.}{dificuldade}
    \definition{v.}{colocar alguém em uma situação difícil}
  \seealsoref{易}{yi4}
  \end{Phonetics}
  \begin{Phonetics}{难}{nan4}
    \definition{s.}{catástrofe; calamidade; desastre; adversidade; grande infortúnio}
    \definition{v.}{acusar; culpar}
  \end{Phonetics}
\end{Entry}

\begin{Entry}{难以}{10,4}{⾫、⼈}
  \begin{Phonetics}{难以}{nan2 yi3}[][HSK 5]
    \definition{adj.}{difícil; complicado}
  \end{Phonetics}
\end{Entry}

\begin{Entry}{难过}{10,6}{⾫、⾡}
  \begin{Phonetics}{难过}{nan2guo4}[][HSK 2]
    \definition{adj.}{triste; ruim; psicologicamente desconfortável | difícil; árduo}
  \end{Phonetics}
\end{Entry}

\begin{Entry}{难免}{10,7}{⾫、⼉}
  \begin{Phonetics}{难免}{nan4mian3}[][HSK 4]
    \definition{adj.}{inevitável; difícil de evitar}
  \end{Phonetics}
\end{Entry}

\begin{Entry}{难听}{10,7}{⾫、⼝}
  \begin{Phonetics}{难听}{nan2 ting1}[][HSK 2]
    \definition{adj.}{desagradável de ouvir | ofensivo; grosseiro; vulgar e desagradável | escandaloso; indigno}
  \end{Phonetics}
\end{Entry}

\begin{Entry}{难忘}{10,7}{⾫、⼼}
  \begin{Phonetics}{难忘}{nan2 wang4}[][HSK 6]
    \definition{adj.}{memorável; inesquecível}
  \end{Phonetics}
\end{Entry}

\begin{Entry}{难受}{10,8}{⾫、⼜}
  \begin{Phonetics}{难受}{nan2shou4}[][HSK 2]
    \definition{adj.}{sentir dor; sentir-se mal; sentir-se desconfortável | sentir-se mal; sentir-se infeliz; de mau humor; triste}
  \end{Phonetics}
\end{Entry}

\begin{Entry}{难度}{10,9}{⾫、⼴}
  \begin{Phonetics}{难度}{nan2 du4}[][HSK 3]
    \definition{s.}{dificuldade; grau de dificuldade}
  \end{Phonetics}
\end{Entry}

\begin{Entry}{难看}{10,9}{⾫、⽬}
  \begin{Phonetics}{难看}{nan2 kan4}[][HSK 2]
    \definition{adj.}{feio; desagradável à vista | vergonhoso; embaraçoso; desonroso; sem glória; sem dignidade}
  \end{Phonetics}
\end{Entry}

\begin{Entry}{难得}{10,11}{⾫、⼻}
  \begin{Phonetics}{难得}{nan2de2}[][HSK 5]
    \definition{adj.}{raro; difícil de encontrar; difícil de obter ou realizar, indicando que é valioso}
    \definition{adv.}{raramente; com pouca frequência}
  \end{Phonetics}
\end{Entry}

\begin{Entry}{难道}{10,12}{⾫、⾡}
  \begin{Phonetics}{难道}{nan2dao4}[][HSK 3]
    \definition{adv.}{certamente não significa que\dots?; é possível que\dots?; não me diga\dots; poderia ser que\dots?; usado em frases interrogativas para reforçar o tom interrogativo; frequentemente usado com palavras como "吗" e "不成".}
  \seealsoref{不成}{bu4 cheng2}
  \seealsoref{吗}{ma5}
  \end{Phonetics}
\end{Entry}

\begin{Entry}{难题}{10,15}{⾫、⾴}
  \begin{Phonetics}{难题}{nan2 ti2}[][HSK 2]
    \definition[个,道]{s.}{desafio; problema difícil; questão difícil; questões difíceis de responder ou resolver}
  \end{Phonetics}
\end{Entry}

%%%%%%%%%% 顽 %%%%%%%%%%
\subsection*{顽}

\begin{Entry}{顽}{10}{⾴}
  \begin{Phonetics}{顽}{wan2}
    \definition*{s.}{Sobrenome: Wan}
    \definition{adj.}{estúpido; denso; insensível | teimoso; obstinado; não é facilmente persuadido ou subjugado | travesso; pernicioso | cabeça dura; estúpido e ignorante}
    \definition{v.}{brincar; divertir-se; divertir-se | empregar; recorrer a | envolver-se em; tomar parte em}
  \end{Phonetics}
\end{Entry}

\begin{Entry}{顽皮}{10,5}{⾴、⽪}
  \begin{Phonetics}{顽皮}{wan2 pi2}[][HSK 6]
    \definition{adj.}{atrevido; travesso; arteiro; levado; (crianças, adolescentes, etc.) adoram brincar e causar problemas e não dão ouvidos a conselhos}
  \end{Phonetics}
\end{Entry}

\begin{Entry}{顽强}{10,12}{⾴、⼸}
  \begin{Phonetics}{顽强}{wan2qiang2}[][HSK 6]
    \definition{adj.}{firme; tenaz; indomável; forte; resistente}
  \end{Phonetics}
\end{Entry}

%%%%%%%%%% 顾 %%%%%%%%%%
\subsection*{顾}

\begin{Entry}{顾}{10}{⾴}
  \begin{Phonetics}{顾}{gu4}[][HSK 6]
    \definition*{s.}{Sobrenome: Gu}
    \definition{adv.}{em vez disso; pelo contrário; indica o oposto, equivalente a 却 ou 反而}
    \definition{conj.}{mas; no entanto}
    \definition{v.}{olhar para trás; olhar para; virar-se e olhar para | cuidar de; atender a; levar em conta ou consideração | visitar; chamar | sentir pena de}
  \seealsoref{反而}{fan3'er2}
  \seealsoref{却}{que4}
  \end{Phonetics}
\end{Entry}

\begin{Entry}{顾及}{10,3}{⾴、⼃}
  \begin{Phonetics}{顾及}{gu4ji2}[][HSK 7-9]
    \definition{v.}{atender a; levar em conta; dar consideração a; cuidar de; notar}
  \end{Phonetics}
\end{Entry}

\begin{Entry}{顾不上}{10,4,3}{⾴、⼀、⼀}
  \begin{Phonetics}{顾不上}{gu4bu5shang4}[][HSK 7-9]
    \definition{v.}{não conseguir; não conseguir atender; incapaz de cuidar de (fazer algo)}
  \end{Phonetics}
\end{Entry}

\begin{Entry}{顾不得}{10,4,11}{⾴、⼀、⼻}
  \begin{Phonetics}{顾不得}{gu4bu5de5}[][HSK 7-9]
    \definition{v.}{incapaz de mudar algo | incapaz de lidar com}
  \end{Phonetics}
\end{Entry}

\begin{Entry}{顾全大局}{10,6,3,7}{⾴、⼊、⼤、⼫}
  \begin{Phonetics}{顾全大局}{gu4quan2-da4ju2}[][HSK 7-9]
    \definition{expr.}{``Considere a situação geral.''; levar em conta os interesses do todo; considerar a situação como um todo; levar em consideração o panorama geral; trabalhar para o benefício de todos}
  \end{Phonetics}
\end{Entry}

\begin{Entry}{顾问}{10,6}{⾴、⾨}
  \begin{Phonetics}{顾问}{gu4wen4}[][HSK 5]
    \definition[个,位,名]{s.}{conselheiro; consultor; assessor; pessoas com conhecimento especializado ou experiência contratadas para prestar consultoria a organizações ou indivíduos}
  \end{Phonetics}
\end{Entry}

\begin{Entry}{顾客}{10,9}{⾴、⼧}
  \begin{Phonetics}{顾客}{gu4ke4}[][HSK 2]
    \definition[个,位,名,些]{s.}{cliente; comprador; consumidor; paciente}
  \end{Phonetics}
\end{Entry}

\begin{Entry}{顾虑}{10,10}{⾴、⾌}
  \begin{Phonetics}{顾虑}{gu4lv4}[][HSK 7-9]
    \definition[丝,点]{s.}{preocupação; escrúpulo; receio; apreensão}
    \definition{v.}{estar apreensivo (sobre as consequências da própria ação)}
  \end{Phonetics}
\end{Entry}

%%%%%%%%%% 顿 %%%%%%%%%%
\subsection*{顿}

\begin{Entry}{顿}{10}{⾴}
  \begin{Phonetics}{顿}{dun4}[][HSK 3]
    \definition*{s.}{Sobrenome: Dun}
    \definition{adj.}{cansado; fatigado}
    \definition{adv.}{de repente; imediatamente; indica que o tempo é curto, equivalente a 立刻}
    \definition{clas.}{usado para refeições | usado para surras, repreensões, castigos físicos, etc.}
    \definition{s.}{um lugar para ficar; acomodação e alimentação}
    \definition{v.}{pausar; parar; fazer uma pausa | pausar na escrita para reforçar o início ou o fim de um traço; ao escrever com pincel, pressione o pincel com força e pare um pouco sobre o papel | tocar o chão (com a cabeça) | bater o pé); chutar o chão ou bater no chão com um objeto | resolver; arranjar | montar acampamento; ficar temporariamente; parar para se hospedar; acampar}
  \seealsoref{立刻}{li4ke4}
  \end{Phonetics}
\end{Entry}

\begin{Entry}{顿时}{10,7}{⾴、⽇}
  \begin{Phonetics}{顿时}{dun4shi2}[][HSK 7-9]
    \definition{adv.}{de repente; imediatamente; repentinamente; indica que uma ação ou comportamento ocorre sob certas circunstâncias ou imediatamente após algo; usado principalmente na escrita; usado apenas para descrever eventos passados}
  \end{Phonetics}
\end{Entry}

%%%%%%%%%% 颁 %%%%%%%%%%
\subsection*{颁}

\begin{Entry}{颁}{10}{⾴}
  \begin{Phonetics}{颁}{ban1}
    \definition{v.}{promulgar; emitir; enviar | conceder ou conferir}
  \end{Phonetics}
\end{Entry}

\begin{Entry}{颁发}{10,5}{⾴、⼜}
  \begin{Phonetics}{颁发}{ban1fa1}[][HSK 7-9]
    \definition{v.}{promulgar; emitir (comandos, instruções, regulamentos, etc.) a um superior | premiar (prêmio, medalha, certificado, etc.)}
  \end{Phonetics}
\end{Entry}

\begin{Entry}{颁布}{10,5}{⾴、⼱}
  \begin{Phonetics}{颁布}{ban1bu4}[][HSK 7-9]
    \definition{v.}{promulgar; emitir; publicar; anunciar (leis, regulamentos, etc.), com um escopo de uso mais restrito do que 公布}
  \seealsoref{公布}{gong1bu4}
  \end{Phonetics}
\end{Entry}

\begin{Entry}{颁奖}{10,9}{⾴、⼤}
  \begin{Phonetics}{颁奖}{ban1/jiang3}[][HSK 7-9]
    \definition{v.+compl.}{conceder prêmios, bônus, certificados, etc.; distribuir prêmios, bônus, certificados, etc.}
  \end{Phonetics}
\end{Entry}

%%%%%%%%%% 预 %%%%%%%%%%
\subsection*{预}

\begin{Entry}{预}{10}{⾴}
  \begin{Phonetics}{预}{yu4}
    \definition{adv.}{antecipadamente}
    \definition{v.}{avançar | preparar}
  \end{Phonetics}
\end{Entry}

\begin{Entry}{预习}{10,3}{⾴、⼄}
  \begin{Phonetics}{预习}{yu4xi2}[][HSK 3]
    \definition{v.}{pré-visualizar; preparar uma lição; estudar antecipadamente as matérias que serão abordadas nas aulas}
  \end{Phonetics}
\end{Entry}

\begin{Entry}{预见}{10,4}{⾴、⾒}
  \begin{Phonetics}{预见}{yu4jian4}
    \definition{s.}{previsão; intuição; vislumbre}
    \definition{v.}{prever}
  \end{Phonetics}
\end{Entry}

\begin{Entry}{预计}{10,4}{⾴、⾔}
  \begin{Phonetics}{预计}{yu4 ji4}[][HSK 3]
    \definition{v.}{estimar; calcular com antecedência}
  \end{Phonetics}
\end{Entry}

\begin{Entry}{预订}{10,4}{⾴、⾔}
  \begin{Phonetics}{预订}{yu4ding4}[][HSK 4]
    \definition{v.}{reservar; fazer uma reserva}
  \end{Phonetics}
\end{Entry}

\begin{Entry}{预付}{10,5}{⾴、⼈}
  \begin{Phonetics}{预付}{yu4fu4}
    \definition{s.}{pré-pago}
    \definition{v.}{pagar antecipadamente}
  \end{Phonetics}
\end{Entry}

\begin{Entry}{预约}{10,6}{⾴、⽷}
  \begin{Phonetics}{预约}{yu4 yue1}[][HSK 6]
    \definition[个]{s.}{reserva}
    \definition{v.}{reservar; agendar; marcar compromisso; marcar uma consulta}
  \end{Phonetics}
\end{Entry}

\begin{Entry}{预防}{10,6}{⾴、⾩}
  \begin{Phonetics}{预防}{yu4fang2}[][HSK 3]
    \definition{v.}{prevenir; proteger-se contra; tomar precauções contra; preparar-se com antecedência para evitar que algo ruim aconteça}
  \end{Phonetics}
\end{Entry}

\begin{Entry}{预判}{10,7}{⾴、⼑}
  \begin{Phonetics}{预判}{yu4pan4}
    \definition{v.}{prever | antecipar}
  \end{Phonetics}
\end{Entry}

\begin{Entry}{预报}{10,7}{⾴、⼿}
  \begin{Phonetics}{预报}{yu4bao4}[][HSK 3]
    \definition[个,项]{s.}{boletim meteorológico; previsões meteorológicas antecipadas}
    \definition{v.}{prever (o tempo); relatar antes que algo aconteça, usado principalmente em relação ao clima, astronomia, desastres naturais, etc.}
  \end{Phonetics}
\end{Entry}

\begin{Entry}{预备}{10,8}{⾴、⼡}
  \begin{Phonetics}{预备}{yu4 bei4}[][HSK 5]
    \definition{v.}{preparar-se; ficar pronto}
  \end{Phonetics}
\end{Entry}

\begin{Entry}{预定}{10,8}{⾴、⼧}
  \begin{Phonetics}{预定}{yu4ding4}
    \definition{v.}{agendar com antecedência}
  \end{Phonetics}
\end{Entry}

\begin{Entry}{预购}{10,8}{⾴、⾙}
  \begin{Phonetics}{预购}{yu4gou4}
    \definition{s.}{compra antecipada}
    \definition{v.}{comprar antecipadamente}
  \end{Phonetics}
\end{Entry}

\begin{Entry}{预测}{10,9}{⾴、⽔}
  \begin{Phonetics}{预测}{yu4 ce4}[][HSK 4]
    \definition{v.}{prever; prognosticar; predizer}
  \end{Phonetics}
\end{Entry}

\begin{Entry}{预祝}{10,9}{⾴、⽰}
  \begin{Phonetics}{预祝}{yu4zhu4}
    \definition{v.}{parabenizar de antemão | oferecer os melhores votos para}
  \end{Phonetics}
\end{Entry}

\begin{Entry}{预览}{10,9}{⾴、⾒}
  \begin{Phonetics}{预览}{yu4lan3}
    \definition{s.}{visualização}
    \definition{v.}{visualizar}
  \end{Phonetics}
\end{Entry}

\begin{Entry}{预留}{10,10}{⾴、⽥}
  \begin{Phonetics}{预留}{yu4liu2}
    \definition{v.}{separar | reservar}
  \end{Phonetics}
\end{Entry}

\begin{Entry}{预配}{10,10}{⾴、⾣}
  \begin{Phonetics}{预配}{yu4pei4}
    \definition{s.}{pré-alocado | pré-cabeado}
    \definition{v.}{pré-alocar | pré-cabear}
  \end{Phonetics}
\end{Entry}

\begin{Entry}{预谋}{10,11}{⾴、⾔}
  \begin{Phonetics}{预谋}{yu4mou2}
    \definition{adj.}{premeditado}
    \definition{v.}{planejar algo com antecedência (especialmente um crime)}
  \end{Phonetics}
\end{Entry}

\begin{Entry}{预提}{10,12}{⾴、⼿}
  \begin{Phonetics}{预提}{yu4ti2}
    \definition{s.}{retenção}
    \definition{v.}{reter (imposto)}
  \end{Phonetics}
\end{Entry}

\begin{Entry}{预期}{10,12}{⾴、⽉}
  \begin{Phonetics}{预期}{yu4qi1}[][HSK 5]
    \definition{v.}{esperar; antecipar; imaginar; antecipar com expectativa}
  \end{Phonetics}
\end{Entry}

\begin{Entry}{预感}{10,13}{⾴、⼼}
  \begin{Phonetics}{预感}{yu4gan3}
    \definition{s.}{premonição}
    \definition{v.}{ter uma premonição}
  \end{Phonetics}
\end{Entry}

\begin{Entry}{预警}{10,19}{⾴、⾔}
  \begin{Phonetics}{预警}{yu4jing3}
    \definition{s.}{aviso | aviso antecipado}
  \end{Phonetics}
\end{Entry}

%%%%%%%%%% 饿 %%%%%%%%%%
\subsection*{饿}

\begin{Entry}{饿}{10}{⾷}
  \begin{Phonetics}{饿}{e4}[][HSK 1]
    \definition{adj.}{faminto}
    \definition{v.}{passar fome; causar fome}
  \end{Phonetics}
\end{Entry}

%%%%%%%%%% 骏 %%%%%%%%%%
\subsection*{骏}

\begin{Entry}{骏}{10}{⾺}
  \begin{Phonetics}{骏}{jun4}
    \definition{s.}{belo cavalo; corcel; animal de montaria}
  \end{Phonetics}
\end{Entry}

\begin{Entry}{骏马}{10,3}{⾺、⾺}
  \begin{Phonetics}{骏马}{jun4ma3}[][HSK 7-9]
    \definition[匹,群]{s.}{belo cavalo; corcel; animal de montaria}
  \end{Phonetics}
\end{Entry}

%%%%%%%%%% 高 %%%%%%%%%%
\subsection*{高}

\begin{Entry}{高}{10}{⾼}[Kangxi 189]
  \begin{Phonetics}{高}{gao1}[][HSK 1]
    \definition*{s.}{Sobrenome: Gao}
    \definition{adj.}{alto; elevado; grande distância de baixo para cima; longe do chão | barulhento | sofisticado; caro; de preço elevado; acima do valor real ou do preço de mercado | acima da média; de alto nível ou grau; acima do padrão geral ou da média; de nível superior}
    \definition{s.}{altura; altitude}
  \end{Phonetics}
\end{Entry}

\begin{Entry}{高于}{10,3}{⾼、⼆}
  \begin{Phonetics}{高于}{gao1 yu2}[][HSK 5]
    \definition{v.}{ser mais alto do que; sobrepujar}
  \end{Phonetics}
\end{Entry}

\begin{Entry}{高大}{10,3}{⾼、⼤}
  \begin{Phonetics}{高大}{gao1 da4}[][HSK 5]
    \definition{adj.}{alto e grande; alto | elevado; sublime; nobre}
  \end{Phonetics}
\end{Entry}

\begin{Entry}{高山}{10,3}{⾼、⼭}
  \begin{Phonetics}{高山}{gao1shan1}[][HSK 7-9]
    \definition[座]{s.}{alta montanha; alpes}
  \end{Phonetics}
\end{Entry}

\begin{Entry}{高中}{10,4}{⾼、⼁}
  \begin{Phonetics}{高中}{gao1 zhong1}[][HSK 2]
    \definition[所,个]{s.}{ensino médio; escola secundária de ensino médio}
  \end{Phonetics}
\end{Entry}

\begin{Entry}{高手}{10,4}{⾼、⼿}
  \begin{Phonetics}{高手}{gao1 shou3}[][HSK 6]
    \definition[位,个,名,些,群]{s.}{ás; mestre; especialista; \emph{expert}; uma pessoa com habilidades excepcionais}
  \end{Phonetics}
\end{Entry}

\begin{Entry}{高尔夫}{10,5,4}{⾼、⼩、⼤}
  \begin{Phonetics}{高尔夫}{gao1'er3fu1}
    \definition{s.}{(empréstimo linguístico) \emph{golf}}
  \end{Phonetics}
\end{Entry}

\begin{Entry}{高尔夫球}{10,5,4,11}{⾼、⼩、⼤、⽟}
  \begin{Phonetics}{高尔夫球}{gao1'er3fu1qiu2}[][HSK 7-9]
    \definition[个,只,场,些]{s.}{golfe | bola de golfe}
  \end{Phonetics}
\end{Entry}

\begin{Entry}{高价}{10,6}{⾼、⼈}
  \begin{Phonetics}{高价}{gao1 jia4}[][HSK 4]
    \definition{s.}{preço alto; bilhete caro; custo elevado; dispendioso}
  \end{Phonetics}
\end{Entry}

\begin{Entry}{高兴}{10,6}{⾼、⼋}
  \begin{Phonetics}{高兴}{gao1xing4}[][HSK 1]
    \definition{adj.}{contente; feliz; exultante; alegre; satisfeito; animado}
    \definition{v.}{estar contente; estar feliz; estar animado; estar de bom humor; fazer algo com alegria; gostar}
  \end{Phonetics}
\end{Entry}

\begin{Entry}{高压}{10,6}{⾼、⼚}
  \begin{Phonetics}{高压}{gao1ya1}[][HSK 7-9]
    \definition{s.}{Física, Meteorologia: alta pressão (oposto a 低压) | Eletricidade: alta tensão; alta voltagem (oposto a 低压) | Política: mão de ferro; arrogância | Medicina: pressão sistólica; pressão máxima | perseguição cruel; opressão extrema}
  \seealsoref{低压}{di1ya1}
  \end{Phonetics}
\end{Entry}

\begin{Entry}{高级}{10,6}{⾼、⽷}
  \begin{Phonetics}{高级}{gao1ji2}[][HSK 2]
    \definition{adj.}{sênior; de alto escalão; de alto nível; elevado; excelente; superior; estágio avançado | e alta qualidade; de primeira qualidade; avançado}
  \end{Phonetics}
\end{Entry}

\begin{Entry}{高考}{10,6}{⾼、⽼}
  \begin{Phonetics}{高考}{gao1 kao3}[][HSK 6]
    \definition[次,回,场]{s.}{vestibular; exame de admissão em instituições de ensino superior}
  \end{Phonetics}
\end{Entry}

\begin{Entry}{高血压}{10,6,6}{⾼、⾎、⼚}
  \begin{Phonetics}{高血压}{gao1xue4ya1}[][HSK 7-9]
    \definition{adj.}{hipertenso}
    \definition[点儿]{s.}{hipertenção; pressão alta}
  \end{Phonetics}
\end{Entry}

\begin{Entry}{高低}{10,7}{⾼、⼈}
  \begin{Phonetics}{高低}{gao1di1}[][HSK 7-9]
    \definition{adv.}{apenas; simplesmente; em qualquer caso; de qualquer forma; em qualquer conta; indica que não importa o que | finalmente; no final, depois de tudo}
    \definition{s.}{inclinação; nível; altura | diferença de grau; superioridade ou inferioridade relativa | discrição; senso de propriedade | (falar ou fazer coisas) medida; profundidade, leveza e peso}
  \end{Phonetics}
\end{Entry}

\begin{Entry}{高层}{10,7}{⾼、⼫}
  \begin{Phonetics}{高层}{gao1 ceng2}[][HSK 6]
    \definition{adj.}{(de um edifício) arranha-céu | (de posição oficial) alto nível}
    \definition{s.}{nível superior; piso, camada, etc. | arranha-céus; um prédio de apartamentos alto}
  \end{Phonetics}
\end{Entry}

\begin{Entry}{高技术}{10,7,5}{⾼、⼿、⽊}
  \begin{Phonetics}{高技术}{gao1 ji4 shu4}
    \definition{s.}{alta tecnologia; \emph{hight tech}}
  \seealsoref{高科技}{gao1 ke1 ji4}
  \end{Phonetics}
\end{Entry}

\begin{Entry}{高尚}{10,8}{⾼、⼩}
  \begin{Phonetics}{高尚}{gao1shang4}[][HSK 4]
    \definition{adj.}{nobre; elevado; descreve um alto padrão moral e uma boa qualidade de pensamento | significativo e não de mau gosto}
  \end{Phonetics}
\end{Entry}

\begin{Entry}{高昂}{10,8}{⾼、⽇}
  \begin{Phonetics}{高昂}{gao1'ang2}[][HSK 7-9]
    \definition{adj.}{alto; eufórico; exaltado | caro; exorbitante}
    \definition{v.}{manter erguida (a cabeça, etc.)}
  \end{Phonetics}
\end{Entry}

\begin{Entry}{高明}{10,8}{⾼、⽇}
  \begin{Phonetics}{高明}{gao1ming2}[][HSK 7-9]
    \definition{adj.}{sábio; brilhante; (percepção, habilidades) excelente}
    \definition{s.}{pessoa sábia; pessoa habilidosa}
  \end{Phonetics}
\end{Entry}

\begin{Entry}{高空}{10,8}{⾼、⽳}
  \begin{Phonetics}{高空}{gao1kong1}[][HSK 7-9]
    \definition{s.}{alta altitude; ar superior (oposto a 低空)}
  \seealsoref{低空}{di1kong1}
  \end{Phonetics}
\end{Entry}

\begin{Entry}{高度}{10,9}{⾼、⼴}
  \begin{Phonetics}{高度}{gao1 du4}[][HSK 5]
    \definition{adj.}{alto; elevado; avançado; alto grau | alta concentração; intenso}
    \definition[个]{s.}{altura; altitude; elevação; distância de baixo para cima; o grau e o nível em que as coisas se desenvolveram}
  \end{Phonetics}
\end{Entry}

\begin{Entry}{高科技}{10,9,7}{⾼、⽲、⼿}
  \begin{Phonetics}{高科技}{gao1 ke1 ji4}[][HSK 6]
    \definition[种,类]{s.}{alta tecnologia; \emph{high tech}}
  \seealsoref{高技术}{gao1 ji4 shu4}
  \end{Phonetics}
\end{Entry}

\begin{Entry}{高贵}{10,9}{⾼、⾙}
  \begin{Phonetics}{高贵}{gao1gui4}[][HSK 7-9]
    \definition{adj.}{(caráter pessoal) nobre; honrado; moralmente elevado; magnânimo | grandeza; extremamente valioso | elitista; altamente privilegiado; refere-se àqueles com status elevado e vida superior}
  \end{Phonetics}
\end{Entry}

\begin{Entry}{高原}{10,10}{⾼、⼚}
  \begin{Phonetics}{高原}{gao1 yuan2}[][HSK 5]
    \definition[片]{s.}{platô; terras altas; planalto | planalto continental}
  \end{Phonetics}
\end{Entry}

\begin{Entry}{高峰}{10,10}{⾼、⼭}
  \begin{Phonetics}{高峰}{gao1feng1}[][HSK 6]
    \definition[个,座]{s.}{cume; pináculo; pico da montanha | pico (de atividade, qualidade ou realização); uma metáfora para o ponto mais alto no desenvolvimento das coisas | cúpula; principais líderes; uma metáfora para o mais alto nível de liderança}
  \end{Phonetics}
\end{Entry}

\begin{Entry}{高峰期}{10,10,12}{⾼、⼭、⽉}
  \begin{Phonetics}{高峰期}{gao1feng1qi1}[][HSK 7-9]
    \definition[个]{s.}{período de pico; (de tráfego) horas de pico; o período em que ocorre com mais frequência ou se desenvolve mais próspero}
  \end{Phonetics}
\end{Entry}

\begin{Entry}{高效}{10,10}{⾼、⽁}
  \begin{Phonetics}{高效}{gao1xiao4}[][HSK 7-9]
    \definition{adj.}{altamente eficiente | eficiente | altamente eficaz}
  \end{Phonetics}
\end{Entry}

\begin{Entry}{高档}{10,10}{⾼、⽊}
  \begin{Phonetics}{高档}{gao1dang4}[][HSK 6]
    \definition{adj.}{grau superior; alta qualidade; alo grau; qualidade superior; boa qualidade, preço alto (produto)}
  \end{Phonetics}
\end{Entry}

\begin{Entry}{高涨}{10,10}{⾼、⽔}
  \begin{Phonetics}{高涨}{gao1zhang3}[][HSK 7-9]
    \definition{v.}{ascender; subir alto; avançar; (preços, sentimento, etc.) subir rapidamente (em oposição a 低落)}
  \seealsoref{低落}{di1luo4}
  \end{Phonetics}
\end{Entry}

\begin{Entry}{高调}{10,10}{⾼、⾔}
  \begin{Phonetics}{高调}{gao1diao4}[][HSK 7-9]
    \definition{adj.}{alto perfil; retaliação; significa agir de forma muito ostensiva e chamativa, deixando claro para todos; também pode significar opor-se deliberadamente aos outros, chegando até a provocar uma briga}
    \definition{s.}{tom elevado; palavras de alto som; sons mais agudos do que o normal ao cantar ou falar}
  \end{Phonetics}
\end{Entry}

\begin{Entry}{高速}{10,10}{⾼、⾡}
  \begin{Phonetics}{高速}{gao1 su4}[][HSK 3]
    \definition{adj.}{alta velocidade; veloz; rápido}
    \definition[条]{s.}{autoestrada; via expressa; rodovia}
  \end{Phonetics}
\end{Entry}

\begin{Entry}{高速公路}{10,10,4,13}{⾼、⾡、⼋、⾜}
  \begin{Phonetics}{高速公路}{gao1su4gong1lu4}[][HSK 3]
    \definition[条]{s.}{via expressa; rodovia; autoestrada; as rodovias destinadas exclusivamente ao tráfego de veículos em alta velocidade são retas e, quando cruzam outras vias, utilizam cruzamentos em nível}
  \end{Phonetics}
\end{Entry}

\begin{Entry}{高铁}{10,10}{⾼、⾦}
  \begin{Phonetics}{高铁}{gao1 tie3}[][HSK 4]
    \definition{s.}{trem de alta velocidade; trem bala}
  \end{Phonetics}
\end{Entry}

\begin{Entry}{高傲}{10,12}{⾼、⼈}
  \begin{Phonetics}{高傲}{gao1'ao4}[][HSK 7-9]
    \definition{adj.}{arrogante; altivo | orgulhoso; respeitoso; nobre}
  \end{Phonetics}
\end{Entry}

\begin{Entry}{高温}{10,12}{⾼、⽔}
  \begin{Phonetics}{高温}{gao1 wen1}[][HSK 5]
    \definition{s.}{alta temperatura (oposto a 低温); temperatura elevada; hipertermia; megatemperatura; inferno}
  \seealsoref{低温}{di1 wen1}
  \end{Phonetics}
\end{Entry}

\begin{Entry}{高等}{10,12}{⾼、⽵}
  \begin{Phonetics}{高等}{gao1 deng3}[][HSK 6]
    \definition{adj.}{superior; avançado (oposto a 低等) | alto nível}
  \seealsoref{低等}{di1 deng3}
  \end{Phonetics}
\end{Entry}

\begin{Entry}{高超}{10,12}{⾼、⾛}
  \begin{Phonetics}{高超}{gao1chao1}[][HSK 7-9]
    \definition{adj.}{soberbo; excelente; descreve um nível muito alto, excedendo a maioria dos níveis}
  \end{Phonetics}
\end{Entry}

\begin{Entry}{高雅}{10,12}{⾼、⾫}
  \begin{Phonetics}{高雅}{gao1ya3}[][HSK 7-9]
    \definition{adj.}{delicado | elegante}
    \definition{s.}{delicadeza; decoro | elegância}
  \end{Phonetics}
\end{Entry}

\begin{Entry}{高新技术}{10,13,7,5}{⾼、⽄、⼿、⽊}
  \begin{Phonetics}{高新技术}{gao1xin1-ji4shu4}[][HSK 7-9]
    \definition[项,门,套]{s.}{nova e alta tecnologia; \emph{high‐tech}}
  \end{Phonetics}
\end{Entry}

\begin{Entry}{高楼}{10,13}{⾼、⽊}
  \begin{Phonetics}{高楼}{gao1lou2}
    \definition[座]{s.}{edifício alto | edifício de muitos andares | arranha-céu}
  \end{Phonetics}
\end{Entry}

\begin{Entry}{高跟鞋}{10,13,15}{⾼、⾜、⾰}
  \begin{Phonetics}{高跟鞋}{gao1 gen1 xie2}[][HSK 5]
    \definition[双]{s.}{salto alto; sapatos de salto alto; sapato feminino com salto mais alto e mais distante do chão}
  \end{Phonetics}
\end{Entry}

\begin{Entry}{高龄}{10,13}{⾼、⿒}
  \begin{Phonetics}{高龄}{gao1ling2}[][HSK 7-9]
    \definition{adj.}{mais velho que o normal | avançado em anos}
    \definition{s.}{idade avançada; idade venerável}
  \end{Phonetics}
\end{Entry}

\begin{Entry}{高潮}{10,15}{⾼、⽔}
  \begin{Phonetics}{高潮}{gao1chao2}[][HSK 4]
    \definition[个,场]{s.}{maré alta; o nível mais alto da maré em um ciclo de maré | pico; aumento; maré alta; uma metáfora para o estágio mais próspero de desenvolvimento das coisas (diferente de 低潮) | (ficção, drama e filmes) clímax}
  \seealsoref{低潮}{di1chao2}
  \end{Phonetics}
\end{Entry}

\begin{Entry}{高额}{10,15}{⾼、⾴}
  \begin{Phonetics}{高额}{gao1'e2}[][HSK 7-9]
    \definition{s.}{quantidade enorme; cota grande | grande quantidade}
  \end{Phonetics}
\end{Entry}

%%%%%%%%%% 髟 %%%%%%%%%%
\subsection*{髟}

\begin{Entry}{髟}{10}{⾽}[Kangxi 190]
  \begin{Phonetics}{髟}{biao1}
    \definition{adj.}{(de cabelo) solto, caído}
  \end{Phonetics}
\end{Entry}

%%%%%%%%%% 鬯 %%%%%%%%%%
\subsection*{鬯}

\begin{Entry}{鬯}{10}{⾿}[Kangxi 192]
  \begin{Phonetics}{鬯}{chang4}
    \definition{adj.}{suave; desimpedido | livre; desinibido}
    \definition{s.}{um tipo de vinho usado em sacrifícios antigos | (antigo) estojo ou bolsa para arco | o mesmo que 畅}
  \seealsoref{畅}{chang4}
  \end{Phonetics}
\end{Entry}

%%%%%%%%%% 鬲 %%%%%%%%%%
\subsection*{鬲}

\begin{Entry}{鬲}{10}{⿀}[Kangxi 193]
  \begin{Phonetics}{鬲}{ge2}
    \definition{s.}{um antigo utensílio de cozinha semelhante a um caldeirão; uma grande panela de barro | utilizado em nomes geográficos ou pessoais}
  \end{Phonetics}
  \begin{Phonetics}{鬲}{li4}
    \definition{s.}{recipiente de cerâmica antigo com três pernas usado para cozinhar, com marcas de cordão na parte externa e pernas ocas}
  \end{Phonetics}
\end{Entry}

%%%%%%%%%% 鸭 %%%%%%%%%%
\subsection*{鸭}

\begin{Entry}{鸭}{10}{⿃}
  \begin{Phonetics}{鸭}{ya1}
    \definition[只]{s.}{pato | (gíria) prostituto}
  \end{Phonetics}
\end{Entry}

\begin{Entry}{鸭子}{10,3}{⿃、⼦}
  \begin{Phonetics}{鸭子}{ya1 zi5}[][HSK 5]
    \definition[只,群]{s.}{pato | Gíria: prostituto}
  \end{Phonetics}
\end{Entry}

%%%%%%%%%% 鸵 %%%%%%%%%%
\subsection*{鸵}

\begin{Entry}{鸵}{10}{⿃}
  \begin{Phonetics}{鸵}{tuo2}
    \definition[只]{s.}{avestruz}
  \end{Phonetics}
\end{Entry}

\begin{Entry}{鸵鸟}{10,5}{⿃、⿃}
  \begin{Phonetics}{鸵鸟}{tuo2niao3}
    \definition{s.}{avestruz}
  \end{Phonetics}
\end{Entry}

%%%%% EOF %%%%%


 %%%
%%% 11画
%%%
\section*{11画}\addcontentsline{toc}{section}{11画}\addcontentsline{loh}{figure}{\#\#\#\# 11画}

%%%%%%%%%% 假 %%%%%%%%%%
\subsection*{假}\addcontentsline{loh}{figure}{假}

\begin{Entry}{假}{11}{⼈}
  \begin{Phonetics}{假}{jia3}[][HSK 2]
    \definition{adj.}{falso; artificial}
    \definition{conj.}{se; caso; no caso de; conecta frases, expressa relação hipotética, geralmente usada com 如, 若 e 使, equivalente a 如果}
    \definition[个,天]{s.}{falsificação; coisas falsas, irreais ou forjadas}
    \definition{v.}{emprestar | valer-se de; aproveitar; utilizar | supor; presumir; pressupor}
  \seealsoref{如}{ru2}
  \seealsoref{如果}{ru2guo3}
  \seealsoref{若}{ruo4}
  \seealsoref{使}{shi3}
  \end{Phonetics}
  \begin{Phonetics}{假}{jia4}
    \definition[个,天]{s.}{feriado; férias; período de suspensão temporária do trabalho ou dos estudos, legal ou aprovado | licença; afastamento temporário; período de licença temporária do trabalho ou dos estudos, após aprovação}
  \end{Phonetics}
\end{Entry}

\begin{Entry}{假日}{11,4}{⼈,⽇}
  \begin{Phonetics}{假日}{jia4ri4}[][HSK 6]
    \definition[节]{s.}{feriado; dia de folga}
  \end{Phonetics}
\end{Entry}

\begin{Entry}{假如}{11,6}{⼈,⼥}
  \begin{Phonetics}{假如}{jia3ru2}[][HSK 4]
    \definition{conj.}{se; supondo; no caso}
  \end{Phonetics}
\end{Entry}

\begin{Entry}{假设}{11,6}{⼈,⾔}
  \begin{Phonetics}{假设}{jia3she4}[][HSK 7-9]
    \definition[个,种,些]{s.}{hipótese; na pesquisa científica, refere"-se à explicação ou conclusão que precisa ser comprovada com base em determinados fenômenos.}
    \definition{v.}{conceder; supor; assumir; presumir | fabricar; inventar; não ser baseado em fatos reais}
  \end{Phonetics}
\end{Entry}

\begin{Entry}{假声}{11,7}{⼈,⼠}
  \begin{Phonetics}{假声}{jia3sheng1}
    \definition{s.}{falsete}
  \seealsoref{真声}{zhen1sheng1}
  \end{Phonetics}
\end{Entry}

\begin{Entry}{假证件}{11,7,6}{⼈,⾔,⼈}
  \begin{Phonetics}{假证件}{jia3zheng4jian4}
    \definition{s.}{documentos falsos}
  \end{Phonetics}
\end{Entry}

\begin{Entry}{假使}{11,8}{⼈,⼈}
  \begin{Phonetics}{假使}{jia3shi3}[][HSK 7-9]
    \definition{conj.}{se; no caso de; supondo que}
  \end{Phonetics}
\end{Entry}

\begin{Entry}{假定}{11,8}{⼈,⼧}
  \begin{Phonetics}{假定}{jia3ding4}[][HSK 7-9]
    \definition{adj.}{suposto; assim chamado}
    \definition[个,种]{s.}{hipótese; hipótese científica | suposição; postulação; presunção}
    \definition{v.}{supor; assumir; conceder; presumir}
  \end{Phonetics}
\end{Entry}

\begin{Entry}{假的}{11,8}{⼈,⽩}
  \begin{Phonetics}{假的}{jia3de5}
    \definition{adj.}{falso | substituto | simulado}
  \end{Phonetics}
\end{Entry}

\begin{Entry}{假冒}{11,9}{⼈,⽇}
  \begin{Phonetics}{假冒}{jia3mao4}[][HSK 7-9]
    \definition{v.}{passar"-se por; passar por falso; fingir ser genuíno}
  \end{Phonetics}
\end{Entry}

\begin{Entry}{假期}{11,12}{⼈,⽉}
  \begin{Phonetics}{假期}{jia4qi1}[][HSK 2]
    \definition[个,段,次,种]{s.}{férias; feriados; período de licença}
  \end{Phonetics}
\end{Entry}

\begin{Entry}{假装}{11,12}{⼈,⾐}
  \begin{Phonetics}{假装}{jia3zhuang1}[][HSK 7-9]
    \definition{v.}{fingir; assumir; simular; vestir; tentar fazer de conta; agir deliberadamente de uma forma diferente da situação real para fazer os outros acreditarem}
  \end{Phonetics}
\end{Entry}

%%%%%%%%%% 偏 %%%%%%%%%%
\subsection*{偏}\addcontentsline{loh}{figure}{偏}

\begin{Entry}{偏}{11}{⼈}
  \begin{Phonetics}{偏}{pian1}[][HSK 6]
    \definition{adj.}{parcial; preconceituoso; injusto; focando apenas em um lado | torto; inclinado | não dominante; auxiliar | remoto; periférico; longe do centro; incomum}
    \definition{adv.}{intencionalmente; insistentemente; persistentemente; indica ir intencionalmente contra o senso comum ou a solicitação de outra pessoa}
    \definition{expr.}{uma expressão educada para indicar que alguém já tomou chá ou comeu}
    \definition{v.}{divergir; não ser igual a; ser diferente de; exceder ou ficar aquém dos padrões normais | desviar-se; afastar-se; sair na direção certa}
  \antonymref{正}{zheng4}
  \end{Phonetics}
\end{Entry}

\begin{Entry}{偏方}{11,4}{⼈,⽅}
  \begin{Phonetics}{偏方}{pian1fang1}[][HSK 7-9]
    \definition{s.}{receita popular | remédio popular | remédio caseiro}
  \seealsoref{偏方儿}{pian1fang1r5}
  \end{Phonetics}
\end{Entry}

\begin{Entry}{偏方儿}{11,4,2}{⼈,⽅,⼉}
  \begin{Phonetics}{偏方儿}{pian1fang1r5}
    \definition{s.}{remédio popular}
  \end{Phonetics}
\end{Entry}

\begin{Entry}{偏见}{11,4}{⼈,⾒}
  \begin{Phonetics}{偏见}{pian1jian4}[][HSK 7-9]
    \definition[种]{s.}{preconceito; viés; visões limitadas a um determinado aspecto}
  \end{Phonetics}
\end{Entry}

\begin{Entry}{偏向}{11,6}{⼈,⼝}
  \begin{Phonetics}{偏向}{pian1xiang4}[][HSK 7-9]
    \definition{s.}{desvio; tendência errônea; tendências incorretas ou incompletas}
    \definition{v.}{proteger; ter predileção por; favorecer alguém em detrimento de outro; dar apoio ou proteção sem princípios a | preferir; inclinar"-se; favorecer}
  \end{Phonetics}
\end{Entry}

\begin{Entry}{偏远}{11,7}{⼈,⾡}
  \begin{Phonetics}{偏远}{pian1yuan3}[][HSK 7-9]
    \definition{adj.}{remoto; distante}
  \end{Phonetics}
\end{Entry}

\begin{Entry}{偏差}{11,9}{⼈,⼯}
  \begin{Phonetics}{偏差}{pian1cha1}[][HSK 7-9]
    \definition{s.}{(ângulo de) desvio; declinação; o ângulo em que um objeto em movimento se afasta de uma direção definida | desvio; erro; erros excessivos ou insuficientes no trabalho}
  \end{Phonetics}
\end{Entry}

\begin{Entry}{偏偏}{11,11}{⼈,⼈}
  \begin{Phonetics}{偏偏}{pian1pian1}[][HSK 7-9]
    \definition{adv.}{de forma deliberada; persistentemente; isso indica um ato deliberado contrário aos requisitos ou circunstâncias objetivas | infelizmente; isso indica que os fatos são exatamente o oposto do que se esperava ou desejava | somente; sozinho; indica um intervalo, semelhante a 单单}
  \seealsoref{单单}{dan1dan1}
  \end{Phonetics}
\end{Entry}

\begin{Entry}{偏僻}{11,15}{⼈,⼈}
  \begin{Phonetics}{偏僻}{pian1pi4}[][HSK 7-9]
    \definition{adj.}{remoto; fora de mão; fica longe da área movimentada e o transporte é inconveniente}
  \end{Phonetics}
\end{Entry}

%%%%%%%%%% 做 %%%%%%%%%%
\subsection*{做}\addcontentsline{loh}{figure}{做}

\begin{Entry}{做}{11}{⼈}
  \begin{Phonetics}{做}{zuo4}[][HSK 1]
    \definition{v.}{fabricar; produzir; criar | escrever; compor | fazer; trabalhar em; dedicar"-se a; exercer uma determinada profissão ou atividade | realizar uma festa em família; comemorar | ser; tornar"-se; agir como; atuar como | ser usado como | formar ou estabelecer um relacionamento; conectar"-se (em algum tipo de relação) | fingir (alguma coisa) | cozinhar; preparar}
  \end{Phonetics}
\end{Entry}

\begin{Entry}{做生活}{11,5,9}{⼈,⽣,⽔}
  \begin{Phonetics}{做生活}{zuo4sheng1huo2}
    \definition{v.}{fazer tabalhos manuais}
  \end{Phonetics}
\end{Entry}

\begin{Entry}{做戏}{11,6}{⼈,⼽}
  \begin{Phonetics}{做戏}{zuo4xi4}
    \definition{v.}{atuar em uma peça | fazer uma peça}
  \end{Phonetics}
\end{Entry}

\begin{Entry}{做作}{11,7}{⼈,⼈}
  \begin{Phonetics}{做作}{zuo4zuo5}
    \definition{adj.}{afetado | artificial}
  \end{Phonetics}
\end{Entry}

\begin{Entry}{做饭}{11,7}{⼈,⾷}
  \begin{Phonetics}{做饭}{zuo4 fan4}[][HSK 2]
    \definition{v.}{cozinhar; preparar uma refeição; cozinhar refeições e transformar alimentos crus em alimentos cozidos}
  \end{Phonetics}
\end{Entry}

\begin{Entry}{做到}{11,8}{⼈,⼑}
  \begin{Phonetics}{做到}{zuo4 dao4}[][HSK 2]
    \definition{v.}{alcançar; realizar; atingir um determinado objetivo; atingir um determinado padrão}
  \end{Phonetics}
\end{Entry}

\begin{Entry}{做法}{11,8}{⼈,⽔}
  \begin{Phonetics}{做法}{zuo4fa5}[][HSK 2]
    \definition[种,个]{s.}{método; maneira de fazer algo; métodos de lidar com coisas ou fazer coisas}
  \end{Phonetics}
\end{Entry}

\begin{Entry}{做客}{11,9}{⼈,⼧}
  \begin{Phonetics}{做客}{zuo4 ke4}[][HSK 3]
    \definition{v.}{visitar; ser um convidado; ser hóspede}
  \end{Phonetics}
\end{Entry}

\begin{Entry}{做活}{11,9}{⼈,⽔}
  \begin{Phonetics}{做活}{zuo4huo2}
    \definition{v.}{trabalhar para ganhar a vida (especialmente de mulher costureira)}
  \end{Phonetics}
\end{Entry}

\begin{Entry}{做梦}{11,11}{⼈,⼣}
  \begin{Phonetics}{做梦}{zuo4 meng4}[][HSK 4]
    \definition{s.}{sonho; ilusões e visões na consciência durante o sono}
    \definition{v.}{sonhar; ter um sonho | sonhar acordado, ter um sonho impossível; metáfora para fantasia irrealista}[别做梦了,她不会嫁给你的。===Pare de sonhar, ela não se casará com você.]
  \end{Phonetics}
\end{Entry}

\begin{Entry}{做眼}{11,11}{⼈,⽬}
  \begin{Phonetics}{做眼}{zuo4yan3}
    \definition{v.}{agir como um guia | trabalhar como espião}
  \end{Phonetics}
\end{Entry}

%%%%%%%%%% 停 %%%%%%%%%%
\subsection*{停}\addcontentsline{loh}{figure}{停}

\begin{Entry}{停}{11}{⼈}
  \begin{Phonetics}{停}{ting2}[][HSK 2]
    \definition{adj.}{pronto; resolvido; bem organizado}
    \definition{clas.}{usado para partes (de um total); porções}
    \definition{v.}{parar; interromper; cessar; fazer uma pausa | permanecer; ficar; fazer uma parada (para descansar) | estacionar; ancorar; atracar}
  \end{Phonetics}
\end{Entry}

\begin{Entry}{停下}{11,3}{⼈,⼀}
  \begin{Phonetics}{停下}{ting2xia4}[][HSK 4]
    \definition{v.}{encerrar; desligar; parar}
  \end{Phonetics}
\end{Entry}

\begin{Entry}{停工}{11,3}{⼈,⼯}
  \begin{Phonetics}{停工}{ting2gong1}
    \definition{v.}{parar de trabalhar | parar a produção}
  \end{Phonetics}
\end{Entry}

\begin{Entry}{停办}{11,4}{⼈,⼒}
  \begin{Phonetics}{停办}{ting2ban4}
    \definition{v.}{cancelar | sair do negócio | desligar | terminar}
  \end{Phonetics}
\end{Entry}

\begin{Entry}{停止}{11,4}{⼈,⽌}
  \begin{Phonetics}{停止}{ting2zhi3}[][HSK 3]
    \definition{v.}{parar; suspender; cessar; cancelar}
  \end{Phonetics}
\end{Entry}

\begin{Entry}{停火}{11,4}{⼈,⽕}
  \begin{Phonetics}{停火}{ting2/huo3}
    \definition{s.}{cessar-fogo}
    \definition{v.+compl.}{cessar fogo}
  \end{Phonetics}
\end{Entry}

\begin{Entry}{停车}{11,4}{⼈,⾞}
  \begin{Phonetics}{停车}{ting2 che1}[][HSK 2]
    \definition{v.}{(veículo) parar; frear | estacionar o veículo | parar; deixar de funcionar}
  \end{Phonetics}
\end{Entry}

\begin{Entry}{停车场}{11,4,6}{⼈,⾞,⼟}
  \begin{Phonetics}{停车场}{ting2che1chang3}[][HSK 2]
    \definition[个]{s.}{estacionamento; área de estacionamento; local para estacionamento de veículos}
  \end{Phonetics}
\end{Entry}

\begin{Entry}{停车位}{11,4,7}{⼈,⾞,⼈}
  \begin{Phonetics}{停车位}{ting2che1wei4}[][HSK 7-9]
    \definition[个]{s.}{vagas de estacionamento; um espaço onde carros ou outros veículos podem ser estacionados}
  \end{Phonetics}
\end{Entry}

\begin{Entry}{停业}{11,5}{⼈,⼀}
  \begin{Phonetics}{停业}{ting2/ye4}[][HSK 7-9]
    \definition{v.+compl.}{cessar as atividades comerciais; encerramento das atividades comerciais; finalizar os negócios; suspensão das atividades comerciais; fechar as portas}[修理内部,停业5天。===O estabelecimento ficará fechado por 5 dias para reparos internos.]
  \synonymref{倒闭}{dao3bi4}
  \synonymref{破产}{po4/chan3}
  \antonymref{开业}{kai1 ye4}
  \antonymref{开张}{kai1/zhang1}
  \antonymref{营业}{ying2ye4}
  \end{Phonetics}
\end{Entry}

\begin{Entry}{停用}{11,5}{⼈,⽤}
  \begin{Phonetics}{停用}{ting2yong4}
    \definition{v.}{desabilitar | descontinuar | parar de usar | suspender}
  \end{Phonetics}
\end{Entry}

\begin{Entry}{停电}{11,5}{⼈,⽥}
  \begin{Phonetics}{停电}{ting2dian4}[][HSK 7-9]
    \definition{v.}{cortar o fornecimento de energia; ter uma falha de energia}
  \antonymref{来电}{lai2dian4}
  \end{Phonetics}
\end{Entry}

\begin{Entry}{停当}{11,6}{⼈,⼹}
  \begin{Phonetics}{停当}{ting2dang5}
    \definition{adj.}{realizado | preparado | assentado}
  \end{Phonetics}
\end{Entry}

\begin{Entry}{停放}{11,8}{⼈,⽅}
  \begin{Phonetics}{停放}{ting2fang4}[][HSK 7-9]
    \definition{v.}{estacionar (um veículo) | colocar (um caixão) | deixar algo (em um lugar) | atracar (um barco, etc.)}
  \end{Phonetics}
\end{Entry}

\begin{Entry}{停泊}{11,8}{⼈,⽔}
  \begin{Phonetics}{停泊}{ting2bo2}[][HSK 7-9]
    \definition{v.}{(navios) atracar; fundear; ancorar}
  \end{Phonetics}
\end{Entry}

\begin{Entry}{停息}{11,10}{⼈,⼼}
  \begin{Phonetics}{停息}{ting2xi1}
    \definition{v.}{cessar | parar}
  \end{Phonetics}
\end{Entry}

\begin{Entry}{停留}{11,10}{⼈,⽥}
  \begin{Phonetics}{停留}{ting2liu2}[][HSK 5]
    \definition{v.}{permanecer; ficar por muito tempo; parar temporariamente em algum lugar, sem continuar avançando | permanecer; parar por um longo tempo; parar em um determinado estágio ou nível, sem evoluir}
  \end{Phonetics}
\end{Entry}

\begin{Entry}{停课}{11,10}{⼈,⾔}
  \begin{Phonetics}{停课}{ting2ke4}
    \definition{v.}{fechar (escola) | parar as aulas}
  \end{Phonetics}
\end{Entry}

\begin{Entry}{停顿}{11,10}{⼈,⾴}
  \begin{Phonetics}{停顿}{ting2dun4}[][HSK 7-9]
    \definition{v.}{parar; interromper; pausar; o assunto foi suspenso ou interrompido | dar pausas (na fala)}
  \synonymref{堵塞}{du3se4}
  \synonymref{平息}{ping2xi1}
  \synonymref{停留}{ting2liu2}
  \synonymref{停止}{ting2zhi3}
  \synonymref{休息}{xiu1xi5}
  \synonymref{中断}{zhong1duan4}
  \antonymref{畅通}{chang4tong1}
  \antonymref{持续}{chi2xu4}
  \end{Phonetics}
\end{Entry}

\begin{Entry}{停歇}{11,13}{⼈,⽋}
  \begin{Phonetics}{停歇}{ting2xie1}
    \definition{v.}{parar para descansar}
  \end{Phonetics}
\end{Entry}

%%%%%%%%%% 偶 %%%%%%%%%%
\subsection*{偶}\addcontentsline{loh}{figure}{偶}

\begin{Entry}{偶}{11}{⼈}
  \begin{Phonetics}{偶}{ou3}
    \definition{adv.}{por acaso; por acidente; de vez em quando; ocasionalmente | par; número par; pareado}
    \definition{s.}{imagem; ídolo; figuras feitas de madeira, barro, etc. | companheiro; cônjuge; parceiro; refere"-se a um casal ou a um dos casais}
  \antonymref{奇}{qi2}
  \end{Phonetics}
\end{Entry}

\begin{Entry}{偶尔}{11,5}{⼈,⼩}
  \begin{Phonetics}{偶尔}{ou3'er3}[][HSK 5]
    \definition{adj.}{ocasional}
    \definition{adv.}{ocasionalmente; de vez em quando; às vezes}
  \end{Phonetics}
\end{Entry}

\begin{Entry}{偶然}{11,12}{⼈,⽕}
  \begin{Phonetics}{偶然}{ou3ran2}[][HSK 5]
    \definition{adj.}{acidental; ocasional}
    \definition{adv.}{por acaso; acidentalmente; sem querer; inesperadamente | ocasionalmente; de vez em quando; às vezes}
  \end{Phonetics}
\end{Entry}

\begin{Entry}{偶像}{11,13}{⼈,⼈}
  \begin{Phonetics}{偶像}{ou3xiang4}[][HSK 5]
    \definition[位,个,名]{s.}{ídolo; pessoa amada pelas pessoas; refere"-se a uma pessoa que é apreciada por todos e que, em certos aspectos, é digna de admiração e respeito}
  \end{Phonetics}
\end{Entry}

%%%%%%%%%% 偷 %%%%%%%%%%
\subsection*{偷}\addcontentsline{loh}{figure}{偷}

\begin{Entry}{偷}{11}{⼈}
  \begin{Phonetics}{偷}{tou1}[][HSK 5]
    \definition{adv.}{furtivamente; secretamente; às escondidas}
    \definition{s.}{ladrão; furtador}
    \definition{v.}{roubar; furtar; levar sem pagar; roubar os bens alheios às escondidas | encontrar (tempo) | deixar"-se levar; viver apenas para o presente, sem se preocupar com o futuro}
  \end{Phonetics}
\end{Entry}

\begin{Entry}{偷安}{11,6}{⼈,⼧}
  \begin{Phonetics}{偷安}{tou1'an1}
    \definition{v.}{buscar facilidade temporária}
  \end{Phonetics}
\end{Entry}

\begin{Entry}{偷听}{11,7}{⼈,⼝}
  \begin{Phonetics}{偷听}{tou1ting1}
    \definition{v.}{bisbilhotar; monitorar (secretamente)}
  \end{Phonetics}
\end{Entry}

\begin{Entry}{偷看}{11,9}{⼈,⽬}
  \begin{Phonetics}{偷看}{tou1kan4}[][HSK 7-9]
    \definition{v.}{dar uma olhada rápida; espiar; dar uma olhadinha | espiar; espreitar}
  \end{Phonetics}
\end{Entry}

\begin{Entry}{偷窃}{11,9}{⼈,⽳}
  \begin{Phonetics}{偷窃}{tou1qie4}
    \definition{v.}{furtar | roubar}
  \end{Phonetics}
\end{Entry}

\begin{Entry}{偷偷}{11,11}{⼈,⼈}
  \begin{Phonetics}{偷偷}{tou1tou1}[][HSK 5]
    \definition{adv.}{secretamente; dissimuladamente; furtivamente; às escondidas; descreve uma ação que não é notada pelos outros; em segredo ou em privado, não revelada}
  \end{Phonetics}
\end{Entry}

\begin{Entry}{偷情}{11,11}{⼈,⼼}
  \begin{Phonetics}{偷情}{tou1qing2}
    \definition{v.}{manter um caso amoroso clandestino; anteriormente usado para se referir a ter um relacionamento romântico secreto, agora geralmente se refere a ter um relacionamento impróprio entre um homem e uma mulher}
  \end{Phonetics}
\end{Entry}

\begin{Entry}{偷袭}{11,11}{⼈,⾐}
  \begin{Phonetics}{偷袭}{tou1xi2}
    \definition{s.}{ataque surpresa}
    \definition{v.}{montar um ataque furtivo | invadir}
  \end{Phonetics}
\end{Entry}

\begin{Entry}{偷渡}{11,12}{⼈,⽔}
  \begin{Phonetics}{偷渡}{tou1du4}
    \definition{s.}{contrabando | imigração ilegal | clandestino (em um navio)}
    \definition{v.}{executar um bloqueio | roubar através da fronteira internacional}
  \end{Phonetics}
\end{Entry}

\begin{Entry}{偷税}{11,12}{⼈,⽲}
  \begin{Phonetics}{偷税}{tou1shui4}
    \definition{s.}{evasão fiscal}
  \end{Phonetics}
\end{Entry}

\begin{Entry}{偷窥}{11,13}{⼈,⽳}
  \begin{Phonetics}{偷窥}{tou1kui1}[][HSK 7-9]
    \definition{v.}{espiar; espionar; bisbilhotar; observar secretamente (especialmente para prazer sexual)}
  \end{Phonetics}
\end{Entry}

\begin{Entry}{偷懒}{11,16}{⼈,⼼}
  \begin{Phonetics}{偷懒}{tou1/lan3}[][HSK 7-9]
    \definition{v.+compl.}{ser preguiçoso; vadiar no trabalho; buscar conforto e conveniência, esquivar"-se das responsabilidades}
  \synonymref{怠慢}{dai4man4}
  \synonymref{懒惰}{lan3duo4}
  \synonymref{懒散}{lan3san3}
  \antonymref{练习}{lian4xi2}
  \end{Phonetics}
\end{Entry}

%%%%%%%%%% 偸 %%%%%%%%%%
\subsection*{偸}\addcontentsline{loh}{figure}{偸}

\begin{Entry}{偸}{11}{⼈}
  \begin{Phonetics}{偸}{tou1}
    \variantof{偷}
  \end{Phonetics}
\end{Entry}

%%%%%%%%%% 偿 %%%%%%%%%%
\subsection*{偿}\addcontentsline{loh}{figure}{偿}

\begin{Entry}{偿}{11}{⼈}
  \begin{Phonetics}{偿}{chang2}
    \definition{v.}{reembolsar; compensar | realizar; cumprir | cumprir; satisfazer}
  \end{Phonetics}
\end{Entry}

\begin{Entry}{偿还}{11,7}{⼈,⾡}
  \begin{Phonetics}{偿还}{chang2huan2}[][HSK 7-9]
    \definition{v.}{reembolsar; pagar de volta; pagar (uma dívida)}
  \end{Phonetics}
\end{Entry}

%%%%%%%%%% 兜 %%%%%%%%%%
\subsection*{兜}\addcontentsline{loh}{figure}{兜}

\begin{Entry}{兜}{11}{⼉}
  \begin{Phonetics}{兜}{dou1}[][HSK 7-9]
    \definition{s.}{bolsa; bolso; coisas tipo bolso}
    \definition{v.}{embrulhar em um pedaço de pano, etc.; fazer um formato de bolso para guardar coisas | mover"-se; dar uma volta; fazer um desvio | solicitar; sondar; recrutamentar | assumir a responsabilidade por algo; assumir o controle de}
  \end{Phonetics}
\end{Entry}

\begin{Entry}{兜儿}{11,2}{⼉,⼉}
  \begin{Phonetics}{兜儿}{dou1r5}[][HSK 7-9]
    \definition{s.}{bolso}
  \end{Phonetics}
\end{Entry}

\begin{Entry}{兜售}{11,11}{⼉,⼝}
  \begin{Phonetics}{兜售}{dou1shou4}[][HSK 7-9]
    \definition{v.}{vender; apregoar | vender; fazer uma venda de}
  \end{Phonetics}
\end{Entry}

%%%%%%%%%% 兽 %%%%%%%%%%
\subsection*{兽}\addcontentsline{loh}{figure}{兽}

\begin{Entry}{兽}{11}{⼋}
  \begin{Phonetics}{兽}{shou4}
    \definition{adj.}{bestial; brutal}
    \definition{s.}{besta; animal}
  \end{Phonetics}
\end{Entry}

\begin{Entry}{兽力车}{11,2,4}{⼋,⼒,⾞}
  \begin{Phonetics}{兽力车}{shou4li4che1}
    \definition{s.}{veículo puxado por animais | carruagem; carroça}
  \antonymref{人力车}{ren2li4che1}
  \end{Phonetics}
\end{Entry}

\begin{Entry}{兽行}{11,6}{⼋,⾏}
  \begin{Phonetics}{兽行}{shou4xing2}
    \definition{s.}{ato brutal; brutalidade | bestialidade}
  \end{Phonetics}
\end{Entry}

%%%%%%%%%% 减 %%%%%%%%%%
\subsection*{减}\addcontentsline{loh}{figure}{减}

\begin{Entry}{减}{11}{⼎}
  \begin{Phonetics}{减}{jian3}[][HSK 4]
    \definition*{s.}{Sobrenome: Jian}
    \definition{v.}{subtrair; remover uma parte da quantidade original | reduzir; diminuir; cortar}
  \end{Phonetics}
\end{Entry}

\begin{Entry}{减少}{11,4}{⼎,⼩}
  \begin{Phonetics}{减少}{jian3shao3}[][HSK 4]
    \definition{v.}{cair; reduzir; diminuir; subtrair uma parte}
  \end{Phonetics}
\end{Entry}

\begin{Entry}{减压}{11,6}{⼎,⼚}
  \begin{Phonetics}{减压}{jian3ya1}[][HSK 7-9]
    \definition{v.}{reduzir a pressão; descomprimir | reduzir o fardo | reduzir a pressão; despressurizar; descomprimir | relaxar}
  \end{Phonetics}
\end{Entry}

\begin{Entry}{减免}{11,7}{⼎,⼉}
  \begin{Phonetics}{减免}{jian3mian3}[][HSK 7-9]
    \definition{v.}{mitigar ou anular (uma punição) | reduzir ou isentar (impostos, etc.)}
  \end{Phonetics}
\end{Entry}

\begin{Entry}{减肥}{11,8}{⼎,⾁}
  \begin{Phonetics}{减肥}{jian3/fei2}[][HSK 4]
    \definition{v.+compl.}{perder peso; dieta, exercícios, medicamentos, massagem, cirurgia, etc., para reduzir o excesso de gordura corporal, de modo que o grau de obesidade seja reduzido}
  \end{Phonetics}
\end{Entry}

\begin{Entry}{减轻}{11,9}{⼎,⾞}
  \begin{Phonetics}{减轻}{jian3qing1}[][HSK 5]
    \definition{v.}{aliviar; remeter; clarear; facilitar; mitigar}
  \end{Phonetics}
\end{Entry}

\begin{Entry}{减弱}{11,10}{⼎,⼸}
  \begin{Phonetics}{减弱}{jian3ruo4}[][HSK 7-9]
    \definition{v.}{reduzir | enfraquecer}
  \end{Phonetics}
\end{Entry}

\begin{Entry}{减速}{11,10}{⼎,⾡}
  \begin{Phonetics}{减速}{jian3/su4}[][HSK 7-9]
    \definition{v.+compl.}{diminuir a velocidade; desacelerar; retardar | moderar; reduzir a velocidade; reduzir a marcha; atrasar; desacelerar; retardar}
  \antonymref{加速}{jia1su4}
  \end{Phonetics}
\end{Entry}

%%%%%%%%%% 凑 %%%%%%%%%%
\subsection*{凑}\addcontentsline{loh}{figure}{凑}

\begin{Entry}{凑}{11}{⼎}
  \begin{Phonetics}{凑}{cou4}[][HSK 7-9]
    \definition{v.}{reunir; coletar; ajuntar | acontecer por acaso; aproveitar; esbarrar em; alcançar; tirar vantagem de | aproximar; mover"-se para perto de}
  \end{Phonetics}
\end{Entry}

\begin{Entry}{凑巧}{11,5}{⼎,⼯}
  \begin{Phonetics}{凑巧}{cou4qiao3}[][HSK 7-9]
    \definition{adj.}{afortunado; sortudo; coincidente; significa que é o momento certo ou que algo que você quer ou não quer está acontecendo}
  \end{Phonetics}
\end{Entry}

\begin{Entry}{凑合}{11,6}{⼎,⼝}
  \begin{Phonetics}{凑合}{cou4he5}[][HSK 7-9]
    \definition{v.}{contentar"-se com algo; ser razoável; ser razoavelmente bom, mas não excelente; aceitar relutantemente coisas ou condições de um nível ou grau inferior | improvisar | reunir}
  \end{Phonetics}
\end{Entry}

%%%%%%%%%% 凰 %%%%%%%%%%
\subsection*{凰}\addcontentsline{loh}{figure}{凰}

\begin{Entry}{凰}{11}{⼏}
  \begin{Phonetics}{凰}{huang2}
    \definition[只]{s.}{Mitologia: fênix fêmea}
  \end{Phonetics}
\end{Entry}

%%%%%%%%%% 剪 %%%%%%%%%%
\subsection*{剪}\addcontentsline{loh}{figure}{剪}

\begin{Entry}{剪}{11}{⼑}
  \begin{Phonetics}{剪}{jian3}[][HSK 5]
    \definition[把]{s.}{tesouras; tesouras de poda; cortadores | pinças; tenazes}
    \definition{v.}{cortar; aparar; tosquiar; cortar (com uma tesoura) | exterminar; eliminar; acabar com}
  \end{Phonetics}
\end{Entry}

\begin{Entry}{剪刀}{11,2}{⼑,⼑}
  \begin{Phonetics}{剪刀}{jian3dao1}[][HSK 5]
    \definition[把,个]{s.}{tesoura; tesoura de jardim; instrumento de ferro para cortar tecido, papel, barbante, etc., com duas lâminas interligadas que podem ser abertas e fechadas}
  \end{Phonetics}
\end{Entry}

\begin{Entry}{剪子}{11,3}{⼑,⼦}
  \begin{Phonetics}{剪子}{jian3zi5}[][HSK 5]
    \definition[把]{s.}{tesouras; tesouras de podar; tosquiadeiras}
  \end{Phonetics}
\end{Entry}

%%%%%%%%%% 副 %%%%%%%%%%
\subsection*{副}\addcontentsline{loh}{figure}{副}

\begin{Entry}{副}{11}{⼑}
  \begin{Phonetics}{副}{fu4}[][HSK 6]
    \definition{adj.}{segundo em exercício; deputado; auxiliar | subsidiário; incidental; secundário}
    \definition{clas.}{usado para conjuntos completos de itens; usado para \emph{kits} | usado para expressões faciais | usado para som ou voz}
    \definition{pref.}{vice-}
    \definition{s.}{assistente; ajudante; auxiliar; posição auxiliar; pessoa que ocupa uma posição auxiliar}
    \definition{v.}{ajustar; corresponder a; conformar"-se a}
  \end{Phonetics}
\end{Entry}

\begin{Entry}{副作用}{11,7,5}{⼑,⼈,⽤}
  \begin{Phonetics}{副作用}{fu4zuo4yong4}[][HSK 7-9]
    \definition{s.}{efeito colateral; efeitos adversos além dos efeitos principais}
  \end{Phonetics}
\end{Entry}

\begin{Entry}{副研}{11,9}{⼑,⽯}
  \begin{Phonetics}{副研}{fu4yan2}
    \definition{s.}{pesquisador adjunto}
  \end{Phonetics}
\end{Entry}

%%%%%%%%%% 勒 %%%%%%%%%%
\subsection*{勒}\addcontentsline{loh}{figure}{勒}

\begin{Entry}{勒}{11}{⼒}
  \begin{Phonetics}{勒}{le4}
    \definition*{s.}{Sobrenome: Le}
    \definition{clas.}{Física: lux (abreviação de 勒克斯)}[这个房间的光照度是500勒克斯。===A sala possui uma iluminância de 500 lux.]
    \definition{s.}{Literário: freio; cabresto; arreio; uma rédea com mecanismo de mastigação}
    \definition{v.}{conter; recuar | forçar; coagir; obrigar | colocar o cabresto | Literário: dar ordens | Literátio: esculpir; gravar; inscrever}
  \seealsoref{勒克斯}{le4ke4si1}
  \end{Phonetics}
  \begin{Phonetics}{勒}{lei1}
    \definition{v.}{amarrar ou prender algo firmemente; apertar; esticar bem}
  \end{Phonetics}
\end{Entry}

\begin{Entry}{勒克斯}{11,7,12}{⼒,⼗,⽄}
  \begin{Phonetics}{勒克斯}{le4ke4si1}
    \definition{s.}{Empréstimo linguístico: lux (unidade de iluminância); lux é a iluminância de um lúmen de luz distribuído uniformemente sobre uma área de um metro quadrado}
  \end{Phonetics}
\end{Entry}

%%%%%%%%%% 勘 %%%%%%%%%%
\subsection*{勘}\addcontentsline{loh}{figure}{勘}

\begin{Entry}{勘}{11}{⼒}
  \begin{Phonetics}{勘}{kan1}
    \definition{v.}{ler e corrigir; conferir | investigar; realizar levantamento | ler e corrigir o texto de; revisar}
  \end{Phonetics}
\end{Entry}

\begin{Entry}{勘探}{11,11}{⼒,⼿}
  \begin{Phonetics}{勘探}{kan1tan4}[][HSK 7-9]
    \definition{v.}{examinar; prospectar; investigar a distribuição de depósitos minerais e determinar a localização, forma, tamanho, regularidade metalogenética, propriedades das rochas e estrutura geológica dos corpos de minério}
  \end{Phonetics}
\end{Entry}

%%%%%%%%%% 唬 %%%%%%%%%%
\subsection*{唬}\addcontentsline{loh}{figure}{唬}

\begin{Entry}{唬}{11}{⼝}
  \begin{Phonetics}{唬}{hu3}
    \definition{v.}{blefar, exagerar para assustar ou confundir}
  \end{Phonetics}
\end{Entry}

%%%%%%%%%% 售 %%%%%%%%%%
\subsection*{售}\addcontentsline{loh}{figure}{售}

\begin{Entry}{售}{11}{⼝}
  \begin{Phonetics}{售}{shou4}
    \definition{v.}{vender | fazer (o plano, truque, etc.) funcionar; continuar (as intrigas) | realizar (intrigas)}
  \end{Phonetics}
\end{Entry}

\begin{Entry}{售价}{11,6}{⼝,⼈}
  \begin{Phonetics}{售价}{shou4jia4}[][HSK 7-9]
    \definition{s.}{preço; preço de venda}
  \end{Phonetics}
\end{Entry}

\begin{Entry}{售货员}{11,8,7}{⼝,⾙,⼝}
  \begin{Phonetics}{售货员}{shou4huo4yuan2}[][HSK 4]
    \definition[名,位]{s.}{vendedor; balconista; assistente de loja; equipe que vende produtos em lojas}
  \end{Phonetics}
\end{Entry}

\begin{Entry}{售票}{11,11}{⼝,⽰}
  \begin{Phonetics}{售票}{shou4piao4}[][HSK 7-9]
    \definition{s.}{venda (vendedor) de ingressos}
    \definition{v.}{vender ingressos}
  \synonymref{售货员}{shou4huo4yuan2}
  \end{Phonetics}
\end{Entry}

%%%%%%%%%% 唯 %%%%%%%%%%
\subsection*{唯}\addcontentsline{loh}{figure}{唯}

\begin{Entry}{唯}{11}{⼝}
  \begin{Phonetics}{唯}{wei2}[][HSK 7-9]
    \definition{adv.}{somente; sozinho | ainda; somente; exceto que}
  \end{Phonetics}
  \begin{Phonetics}{唯}{wei3}
    \definition{interj.}{Onomatopéia: ``Sim!''; ``Yea!''; significa uma palavra que indica acordo}
  \end{Phonetics}
\end{Entry}

\begin{Entry}{唯一}{11,1}{⼝,⼀}
  \begin{Phonetics}{唯一}{wei2yi1}[][HSK 5]
    \definition{adj.}{único; exclusivo; singular; apenas um; nenhum outro}
  \end{Phonetics}
\end{Entry}

\begin{Entry}{唯独}{11,9}{⼝,⽝}
  \begin{Phonetics}{唯独}{wei2du2}[][HSK 7-9]
    \definition{adv.}{apenas; sozinho; pode ser usado antes de frases verbais, significando 只 ou 仅仅; também pode ser usado antes de uma frase sujeito"-predicado, geralmente no início de uma frase, significando 只有; às vezes pode ser colocado diretamente antes do substantivo ou pronome, mas precisa ser seguido por um verbo e uma cláusula sexual}
  \seealsoref{仅仅}{jin3jin3}
  \seealsoref{只}{zhi3}
  \synonymref{唯一}{wei2yi1}
  \synonymref{只有}{zhi3you3}
  \antonymref{众多}{zhong4duo1}
  \end{Phonetics}
\end{Entry}

%%%%%%%%%% 唱 %%%%%%%%%%
\subsection*{唱}\addcontentsline{loh}{figure}{唱}

\begin{Entry}{唱}{11}{⼝}
  \begin{Phonetics}{唱}{chang4}[][HSK 1]
    \definition*{s.}{Sobrenome: Chang}
    \definition{s.}{uma música ou uma parte cantada de uma ópera chinesa; canções; letras de óperas tradicionais}
    \definition{v.}{cantar; seguir o ritmo da música | chorar; chamar; gritar, falar ou recitar em voz alta}
  \end{Phonetics}
\end{Entry}

\begin{Entry}{唱片}{11,4}{⼝,⽚}
  \begin{Phonetics}{唱片}{chang4pian4}[][HSK 4]
    \definition[枚,张]{s.}{disco; disco feito de goma-laca, plástico, etc. com ranhuras em espiral na superfície para registrar alterações no som que podem reproduzir o som gravado em um fonógrafo}
  \end{Phonetics}
\end{Entry}

\begin{Entry}{唱歌}{11,14}{⼝,⽋}
  \begin{Phonetics}{唱歌}{chang4/ge1}[][HSK 1]
    \definition{v.+compl.}{cantar (uma música); emitir sons com entonação ritmada e melodiosa; emitir sons (musicais) com a boca; emitir sons de acordo com a melodia}
  \end{Phonetics}
\end{Entry}

%%%%%%%%%% 唾 %%%%%%%%%%
\subsection*{唾}\addcontentsline{loh}{figure}{唾}

\begin{Entry}{唾}{11}{⼝}
  \begin{Phonetics}{唾}{tuo4}
    \definition[口]{s.}{saliva; cuspe}
    \definition{v.}{cuspir (mostrar desprezo) | rejeitar}
  \end{Phonetics}
\end{Entry}

\begin{Entry}{唾骂}{11,9}{⼝,⾺}
  \begin{Phonetics}{唾骂}{tuo4ma4}
    \definition{v.}{insultar | amaldiçoar}
  \end{Phonetics}
\end{Entry}

\begin{Entry}{唾液}{11,11}{⼝,⽔}
  \begin{Phonetics}{唾液}{tuo4ye4}[][HSK 7-9]
    \definition[口]{s.}{saliva; cuspe}
  \synonymref{口水}{kou3shui3}
  \end{Phonetics}
\end{Entry}

%%%%%%%%%% 啃 %%%%%%%%%%
\subsection*{啃}\addcontentsline{loh}{figure}{啃}

\begin{Entry}{啃}{11}{⼝}
  \begin{Phonetics}{啃}{ken3}[][HSK 7-9]
    \definition{v.}{roer; mordiscar | Figurativo: estudar}
  \end{Phonetics}
\end{Entry}

%%%%%%%%%% 商 %%%%%%%%%%
\subsection*{商}\addcontentsline{loh}{figure}{商}

\begin{Entry}{商}{11}{⼝}
  \begin{Phonetics}{商}{shang1}
    \definition*{s.}{Dinastia Shang (1600--1046 a.C.) | Shang, nome da estrela da constelação do coração entre as vinte e oito constelações | Sobrenome: Shang}
    \definition{s.}{comércio; negócio; a atividade econômica de compra e venda de mercadorias | comerciante; negociante; comerciante; empresário; pessoas que compram e vendem mercadorias | (matemática) quociente;  o resultado de uma operação de divisão em aritmética | uma nota da antiga escala chinesa de cinco tons, correspondente a 2 na notação musical numerada}
    \definition{v.}{discutir; consultar; trocar ideias}
  \end{Phonetics}
\end{Entry}

\begin{Entry}{商人}{11,2}{⼝,⼈}
  \begin{Phonetics}{商人}{shang1ren2}[][HSK 2]
    \definition[位,名]{s.}{comerciante; mercador; empresário; homem de negócios; pessoas que trabalham com a distribuição de mercadorias}
  \end{Phonetics}
\end{Entry}

\begin{Entry}{商业}{11,5}{⼝,⼀}
  \begin{Phonetics}{商业}{shang1ye4}[][HSK 3]
    \definition[个,种]{s.}{barganha; negócio; comércio; atividade econômica que circula mercadorias por meio de compra e venda}
  \end{Phonetics}
\end{Entry}

\begin{Entry}{商务}{11,5}{⼝,⼒}
  \begin{Phonetics}{商务}{shang1wu4}[][HSK 4]
    \definition[种,类,项]{s.}{negócios; assuntos de negócios; assuntos comerciais}
  \end{Phonetics}
\end{Entry}

\begin{Entry}{商讨}{11,5}{⼝,⾔}
  \begin{Phonetics}{商讨}{shang1tao3}[][HSK 7-9]
    \definition{v.}{discutir; deliberar sobre; trocar ideias e discutir para resolver problemas maiores e mais complexos}
  \end{Phonetics}
\end{Entry}

\begin{Entry}{商场}{11,6}{⼝,⼟}
  \begin{Phonetics}{商场}{shang1chang3}[][HSK 1]
    \definition[家]{s.}{mercado; shopping center; loja de departamentos; loja de grande área com uma variedade completa de produtos | o mundo dos negócios; referindo"-se ao mundo dos negócios | mercado; mercado composto por várias lojas reunidas em um ou vários edifícios interligados}
  \end{Phonetics}
\end{Entry}

\begin{Entry}{商店}{11,8}{⼝,⼴}
  \begin{Phonetics}{商店}{shang1dian4}[][HSK 1]
    \definition[间,家,个]{s.}{loja; armazém; local de venda de mercadorias em recinto fechado}
  \end{Phonetics}
\end{Entry}

\begin{Entry}{商贩}{11,8}{⼝,⾙}
  \begin{Phonetics}{商贩}{shang1fan4}[][HSK 7-9]
    \definition{s.}{varejista; vendedor ambulante; comerciante; pequenos comerciantes que vendem ou comercializam produtos}
  \end{Phonetics}
\end{Entry}

\begin{Entry}{商品}{11,9}{⼝,⼝}
  \begin{Phonetics}{商品}{shang1pin3}[][HSK 3]
    \definition[种,个,件,批]{s.}{bens; mercadoria; \emph{merchande}; os produtos do trabalho produzidos para troca têm a dupla natureza de valor de uso e valor; as mercadorias incorporam diferentes relações de produção em diferentes sistemas sociais}
  \end{Phonetics}
\end{Entry}

\begin{Entry}{商城}{11,9}{⼝,⼟}
  \begin{Phonetics}{商城}{shang1cheng2}[][HSK 6]
    \definition{s.}{um mercado; um centro comercial; um \emph{shopping center}; refere"-se a um complexo comercial contíguo com um grande espaço de construção}
  \end{Phonetics}
\end{Entry}

\begin{Entry}{商标}{11,9}{⼝,⽊}
  \begin{Phonetics}{商标}{shang1biao1}[][HSK 5]
    \definition[个]{s.}{marca; marca registrada; \emph{trademark}; marca ou símbolo (desenho, padrão, texto, etc.) gravado ou impresso na superfície ou embalagem de um produto, para diferenciá-lo de outros produtos semelhantes}
  \end{Phonetics}
\end{Entry}

\begin{Entry}{商贸}{11,9}{⼝,⾙}
  \begin{Phonetics}{商贸}{shang1mao4}
    \definition{s.}{comércio}
  \end{Phonetics}
\end{Entry}

\begin{Entry}{商贾}{11,10}{⼝,⾙}
  \begin{Phonetics}{商贾}{shang1gu3}[][HSK 7-9]
    \definition{s.}{Literário: comerciante}
  \end{Phonetics}
\end{Entry}

\begin{Entry}{商量}{11,12}{⼝,⾥}
  \begin{Phonetics}{商量}{shang1liang5}[][HSK 2]
    \definition{v.}{consultar; discutir; conversar sobre; discutir e trocar opiniões}
  \end{Phonetics}
\end{Entry}

%%%%%%%%%% 啤 %%%%%%%%%%
\subsection*{啤}\addcontentsline{loh}{figure}{啤}

\begin{Entry}{啤}{11}{⼝}
  \begin{Phonetics}{啤}{pi2}
    \definition{s.}{cerveja}
  \end{Phonetics}
\end{Entry}

\begin{Entry}{啤酒}{11,10}{⼝,⾣}
  \begin{Phonetics}{啤酒}{pi2jiu3}[][HSK 3]
    \definition[杯,瓶,罐,桶,缸]{s.}{(empréstimo linguístico) cerveja; uma bebida de baixo teor alcoólico feita de malte de cevada e lúpulo, com espuma e aroma especial}
  \end{Phonetics}
\end{Entry}

\begin{Entry}{啤酒馆}{11,10,11}{⼝,⾣,⾷}
  \begin{Phonetics}{啤酒馆}{pi2jiu3guan3}
    \definition{s.}{cervejaria}
  \end{Phonetics}
\end{Entry}

%%%%%%%%%% 啥 %%%%%%%%%%
\subsection*{啥}\addcontentsline{loh}{figure}{啥}

\begin{Entry}{啥}{11}{⼝}
  \begin{Phonetics}{啥}{sha2}
    \definition{pron.}{Dialeto: O que?; equivalente a 什么}
  \end{Phonetics}
\end{Entry}

%%%%%%%%%% 啦 %%%%%%%%%%
\subsection*{啦}\addcontentsline{loh}{figure}{啦}

\begin{Entry}{啦}{11}{⼝}
  \begin{Phonetics}{啦}{la1}
    \definition{s.}{Onomatoméia: som de canto, aplausos etc.; usado para palavras como 呼啦啦, 哗啦啦, 哩哩啦啦, etc.}
  \seealsoref{呼啦啦}{hu1 la1 la1}
  \seealsoref{哗啦啦}{hua1la1 la5}
  \seealsoref{哩哩啦啦}{li1 li1 la1 la1}
  \end{Phonetics}
  \begin{Phonetics}{啦}{la5}[][HSK 6]
    \definition{part.}{uma palavra composta de 了 e 啊, que tem o significado de ambos}
  \seealsoref{啊}{a5}
  \seealsoref{了}{le5}
  \end{Phonetics}
\end{Entry}

\begin{Entry}{啦啦队}{11,11,4}{⼝,⼝,⾩}
  \begin{Phonetics}{啦啦队}{la1la1dui4}[][HSK 7-9]
    \definition{s.}{equipe de líderes de torcida; durante competições esportivas, um grupo de pessoas que torce pelos atletas é chamado de 拉拉队}
  \seealsoref{拉拉队}{la1la1dui4}
  \end{Phonetics}
\end{Entry}

%%%%%%%%%% 啵 %%%%%%%%%%
\subsection*{啵}\addcontentsline{loh}{figure}{啵}

\begin{Entry}{啵}{11}{⼝}
  \begin{Phonetics}{啵}{bo1}
    \definition{part.}{denotando pedido, comando, etc.; o uso é semelhante ao de 吧, que é mais comum no vernáculo antigo}
    \definition{v.aux.}{indicando uma sugestão, pedido ou comando suave | indicando consentimento ou aprovação | em uma pergunta tendenciosa que pede a confirmação de uma suposição | indicando alguma dúvida na mente do falante | marcando uma pausa após suposições como alternativas}
  \seealsoref{吧}{ba5}
  \end{Phonetics}
  \begin{Phonetics}{啵}{bo5}
    \definition{part.}{partícula gramaticalmente equivalente a 吧}
  \seealsoref{吧}{ba5}
  \end{Phonetics}
\end{Entry}

%%%%%%%%%% 圈 %%%%%%%%%%
\subsection*{圈}\addcontentsline{loh}{figure}{圈}

\begin{Entry}{圈}{11}{⼞}
  \begin{Phonetics}{圈}{juan1}
    \definition{v.}{prender aves e animais de criação | Coloquial: prender os criminosos; colocar na cadeia, prisão | confinar; encarcerar}
  \end{Phonetics}
  \begin{Phonetics}{圈}{juan4}[][HSK 7-9]
    \definition*{s.}{Sobrenome: Juan}
    \definition{s.}{curral; local onde o gado ou as aves são mantidos, geralmente cercado ou murado, alguns com galpões}
  \end{Phonetics}
  \begin{Phonetics}{圈}{quan1}[][HSK 4]
    \definition[个]{s.}{anel; círculo; refere"-se a algo em forma de anel | domínio; grupo; escopo; círculo(s)}
    \definition{v.}{cercar; rodear; circundar | marcar com um círculo}
  \end{Phonetics}
\end{Entry}

\begin{Entry}{圈子}{11,3}{⼞,⼦}
  \begin{Phonetics}{圈子}{quan1zi5}[][HSK 7-9]
    \definition[个]{s.}{anel; círculo; uma figura plana, redonda e oca; um objeto em forma de anel | círculo; panelinha; grupo; âmbito; refere"-se ao âmbito das atividades humanas ou à área de um grupo}
  \end{Phonetics}
\end{Entry}

\begin{Entry}{圈套}{11,10}{⼞,⼤}
  \begin{Phonetics}{圈套}{quan1tao4}[][HSK 7-9]
    \definition[个]{s.}{malha; armadilha; laço; esquemas para enganar pessoas}
  \end{Phonetics}
\end{Entry}

\begin{Entry}{圈粉}{11,10}{⼞,⽶}
  \begin{Phonetics}{圈粉}{quan1fen3}
    \definition{s.}{(neologismo, coloquial) ganhar alguém como fã, obter novos fãs}
  \end{Phonetics}
\end{Entry}

%%%%%%%%%% 埦 %%%%%%%%%%
\subsection*{埦}\addcontentsline{loh}{figure}{埦}

\begin{Entry}{埦}{11}{⼟}
  \begin{Phonetics}{埦}{wan3}
    \variantof{碗}
  \end{Phonetics}
\end{Entry}

%%%%%%%%%% 培 %%%%%%%%%%
\subsection*{培}\addcontentsline{loh}{figure}{培}

\begin{Entry}{培}{11}{⼟}
  \begin{Phonetics}{培}{pei2}
    \definition{v.}{aterrar com terra; aterrar | fomentar; treinar | cultivar; crescer e desenvolver-se propositalmente}
  \end{Phonetics}
\end{Entry}

\begin{Entry}{培训}{11,5}{⼟,⾔}
  \begin{Phonetics}{培训}{pei2xun4}[][HSK 4]
    \definition{v.}{treinar (trabalhadores técnicos, quadros profissionais, etc.)}
  \end{Phonetics}
\end{Entry}

\begin{Entry}{培训班}{11,5,10}{⼟,⾔,⽟}
  \begin{Phonetics}{培训班}{pei2xun4ban1}[][HSK 4]
    \definition{s.}{aula de treinamento; curso de treinamento}
  \end{Phonetics}
\end{Entry}

\begin{Entry}{培育}{11,8}{⼟,⾁}
  \begin{Phonetics}{培育}{pei2yu4}[][HSK 4]
    \definition{v.}{criar; fomentar; educar; procriar; nutrir; cultivar}
  \end{Phonetics}
\end{Entry}

\begin{Entry}{培养}{11,9}{⼟,⼋}
  \begin{Phonetics}{培养}{pei2yang3}[][HSK 4]
    \definition{v.}{cultivar (plantas, microorganismos) | promover; treinar ou desenvolver; educar e treinar para um determinado propósito durante um longo período de tempo; fazer crescer | progredir gradualmente; desenvolver ou cultivar gradualmente (hábito, qualidade, caráter, emoção, estilo, interesse, habilidade, etc.)}
  \end{Phonetics}
\end{Entry}

%%%%%%%%%% 基 %%%%%%%%%%
\subsection*{基}\addcontentsline{loh}{figure}{基}

\begin{Entry}{基}{11}{⼟}
  \begin{Phonetics}{基}{ji1}
    \definition{adj.}{chave; básico; primário; cardinal; fundamental}
    \definition{s.}{base; fundação | base; grupo; radical; (química) uma parte dos átomos contidos na molécula de um composto, quando considerada como uma unidade, é chamada de base}
  \end{Phonetics}
\end{Entry}

\begin{Entry}{基于}{11,3}{⼟,⼆}
  \begin{Phonetics}{基于}{ji1yu2}[][HSK 7-9]
    \definition{prep.}{devido a; por causa de; em vista de; apresentando a premissa ou base de uma ação}
  \end{Phonetics}
\end{Entry}

\begin{Entry}{基本}{11,5}{⼟,⽊}
  \begin{Phonetics}{基本}{ji1ben3}[][HSK 3]
    \definition{adj.}{básico; fundamental; elementar | principal}
    \definition{adv.}{basicamente; em geral; no geral; em termos gerais}
    \definition{s.}{fundação}
  \end{Phonetics}
\end{Entry}

\begin{Entry}{基本上}{11,5,3}{⼟,⽊,⼀}
  \begin{Phonetics}{基本上}{ji1ben3shang5}[][HSK 3]
    \definition{adv.}{basicamente; principalmente | em geral; de modo geral}
  \end{Phonetics}
\end{Entry}

\begin{Entry}{基本功}{11,5,5}{⼟,⽊,⼒}
  \begin{Phonetics}{基本功}{ji1ben3gong1}[][HSK 7-9]
    \definition{s.}{treinamento básico; habilidade básica; técnica essencial}
  \end{Phonetics}
\end{Entry}

\begin{Entry}{基本法}{11,5,8}{⼟,⽊,⽔}
  \begin{Phonetics}{基本法}{ji1ben3fa3}
    \definition{s.}{lei básica (constituição)}
  \end{Phonetics}
\end{Entry}

\begin{Entry}{基因}{11,6}{⼟,⼞}
  \begin{Phonetics}{基因}{ji1yin1}[][HSK 7-9]
    \definition[段,个]{s.}{gene; unidade básica de um organismo que carrega e transmite informações genéticas; está localizado nos cromossomos do núcleo da célula}
  \end{Phonetics}
\end{Entry}

\begin{Entry}{基地}{11,6}{⼟,⼟}
  \begin{Phonetics}{基地}{ji1di4}[][HSK 5]
    \definition{s.}{base; como base para alguns negócios | base; um local dedicado à realização de um negócio}
  \end{Phonetics}
\end{Entry}

\begin{Entry}{基层}{11,7}{⼟,⼫}
  \begin{Phonetics}{基层}{ji1ceng2}[][HSK 7-9]
    \definition{s.}{base; nível local; nível básico; o nível mais baixo de qualquer organização, tem a conexão mais direta com as massas}
  \end{Phonetics}
\end{Entry}

\begin{Entry}{基金}{11,8}{⼟,⾦}
  \begin{Phonetics}{基金}{ji1jin1}[][HSK 5]
    \definition[只,笔]{s.}{fundo; fundos reservados ou destinados ao estabelecimento ou desenvolvimento de uma empresa}
  \end{Phonetics}
\end{Entry}

\begin{Entry}{基准}{11,10}{⼟,⼎}
  \begin{Phonetics}{基准}{ji1zhun3}[][HSK 7-9]
    \definition{s.}{referência; base; padrão inicial para medição | um critério; um padrão; uma referência; refere"-se a padrões básicos}
  \end{Phonetics}
\end{Entry}

\begin{Entry}{基础}{11,10}{⼟,⽯}
  \begin{Phonetics}{基础}{ji1chu3}[][HSK 3]
    \definition[个,种,点,层]{s.}{base; fundamento; fundação; a essência ou o ponto de partida do desenvolvimento das coisas | básico; fundamental; refere"-se às condições mínimas | fundação do edifício; base do edifício}
  \end{Phonetics}
\end{Entry}

\begin{Entry}{基督教}{11,13,11}{⼟,⽬,⽁}
  \begin{Phonetics}{基督教}{ji1du1jiao4}[][HSK 6]
    \definition*{s.}{Cristianismo; A Religião Cristã | Cristão}
  \end{Phonetics}
\end{Entry}

%%%%%%%%%% 堂 %%%%%%%%%%
\subsection*{堂}\addcontentsline{loh}{figure}{堂}

\begin{Entry}{堂}{11}{⼟}
  \begin{Phonetics}{堂}{tang2}[][HSK 7-9]
    \definition*{s.}{Sobrenome: Tang}
    \definition{clas.}{classe; uma turma é dividida em seções; uma seção é chamada de classe | caso; um julgamento de cada vez é chamado de sessão judicial | conjunto; conjuntos de móveis | utilizado para cenas; murais, etc.}
    \definition[节,门]{s.}{sala; cômodos principais; hall como símbolo das casas principais no sistema familiar | tribunal; antigamente, era um local onde se realizavam cerimônias em repartições públicas; um local para audiências judiciais | placa da loja; o nome de uma loja; utilizado para a identidade visual da loja | mãe; pais; salão interno; metaforicamente referindo"-se à mãe | do mesmo clã; parentesco entre primos, etc., do mesmo avô paterno ou bisavô | salão; casas projetadas especificamente para uma determinada atividade}[我参观了三槐堂。===Visitei o Salão Sanhuaitang.]
  \end{Phonetics}
\end{Entry}

%%%%%%%%%% 堆 %%%%%%%%%%
\subsection*{堆}\addcontentsline{loh}{figure}{堆}

\begin{Entry}{堆}{11}{⼟}
  \begin{Phonetics}{堆}{dui1}[][HSK 5]
    \definition{clas.}{amontoado; pilha; multidão; usado para pilhas de coisas}
    \definition{s.}{amontoado; pilha; empilhamento | (em nomes de lugares)  colina; monte| multidão de pessoas ou coisas}
    \definition{v.}{empilhar; amontoar; acumular; juntar; reunir}
  \end{Phonetics}
\end{Entry}

\begin{Entry}{堆砌}{11,9}{⼟,⽯}
  \begin{Phonetics}{堆砌}{dui1qi4}[][HSK 7-9]
    \definition{v.}{Figurativo: preencher (escrever com frases elaboradas) | Literário: empilhar (tijolos) | embalar}
  \end{Phonetics}
\end{Entry}

%%%%%%%%%% 堕 %%%%%%%%%%
\subsection*{堕}\addcontentsline{loh}{figure}{堕}

\begin{Entry}{堕}{11}{⼟}
  \begin{Phonetics}{堕}{duo4}
    \definition{v.}{cair; afundar}
  \end{Phonetics}
\end{Entry}

\begin{Entry}{堕落}{11,12}{⼟,⾋}
  \begin{Phonetics}{堕落}{duo4luo4}[][HSK 7-9]
    \definition{adj.}{corrupto; decadente}
    \definition{v.}{cair; afundar; degenerar; deteriorar}
  \end{Phonetics}
\end{Entry}

%%%%%%%%%% 堵 %%%%%%%%%%
\subsection*{堵}\addcontentsline{loh}{figure}{堵}

\begin{Entry}{堵}{11}{⼟}
  \begin{Phonetics}{堵}{du3}[][HSK 4]
    \definition*{s.}{Sobrenome: Du}
    \definition{adj.}{asfixiado; abafado; sufocado; oprimido}
    \definition{clas.}{usado para paredes}
    \definition{s.}{parede}
    \definition{v.}{impedir; bloquear}
  \end{Phonetics}
\end{Entry}

\begin{Entry}{堵车}{11,4}{⼟,⾞}
  \begin{Phonetics}{堵车}{du3/che1}[][HSK 4]
    \definition{s.}{congestionamento; tráfego intenso; ficar congestionado (no tráfego); bloqueio de vias devido ao excesso de tráfego, etc.}
    \definition{v.+compl.}{congestionar (trânsito)}
  \end{Phonetics}
\end{Entry}

\begin{Entry}{堵塞}{11,13}{⼟,⼟}
  \begin{Phonetics}{堵塞}{du3se4}[][HSK 7-9]
    \definition{v.}{parar; bloquear; tornar obstruído}
  \end{Phonetics}
\end{Entry}

%%%%%%%%%% 够 %%%%%%%%%%
\subsection*{够}\addcontentsline{loh}{figure}{够}

\begin{Entry}{够}{11}{⼣}
  \begin{Phonetics}{够}{gou4}[][HSK 2]
    \definition{adj.}{suficiente; adequado; apropriado; atingir e ultrapassar um determinado limite, difícil de suportar}
    \definition{adv.}{suficientemente; o suficiente (para atingir um determinado nível); indica que atingiu um determinado padrão ou nível elevado}
    \definition{v.}{alcançar (algo, esticando-se); (usando membros, etc.) esticar-se para alcançar ou tocar em locais de difícil acesso | atingir (um padrão ou nível); satisfazer ou atingir a quantidade, os padrões, etc. necessários}
  \end{Phonetics}
\end{Entry}

\begin{Entry}{够不着}{11,4,11}{⼣,⼀,⽬}
  \begin{Phonetics}{够不着}{gou4bu5zhao2}
    \definition{v.}{ser incapaz de alcançar}
  \end{Phonetics}
\end{Entry}

\begin{Entry}{够本}{11,5}{⼣,⽊}
  \begin{Phonetics}{够本}{gou4ben3}
    \definition{v.}{empatar | fazer valer o dinheiro}
  \end{Phonetics}
\end{Entry}

\begin{Entry}{够呛}{11,7}{⼣,⼝}
  \begin{Phonetics}{够呛}{gou4qiang4}[][HSK 7-9]
    \definition{adj.}{terrível; insuportável; descreve uma situação extremamente grave e insuportável | improvável; bastante improvável; quase impossível; descreve como difícil de alcançar}
  \end{Phonetics}
\end{Entry}

\begin{Entry}{够味}{11,8}{⼣,⼝}
  \begin{Phonetics}{够味}{gou4wei4}
    \definition{adj.}{excelente | na medida}
  \end{Phonetics}
\end{Entry}

\begin{Entry}{够戗}{11,8}{⼣,⼽}
  \begin{Phonetics}{够戗}{gou4qiang4}
    \variantof{够呛}
  \end{Phonetics}
\end{Entry}

\begin{Entry}{够朋友}{11,8,4}{⼣,⽉,⼜}
  \begin{Phonetics}{够朋友}{gou4peng2you5}
    \definition{v.}{ser um amigo verdadeiro}
  \end{Phonetics}
\end{Entry}

\begin{Entry}{够格}{11,10}{⼣,⽊}
  \begin{Phonetics}{够格}{gou4ge2}
    \definition{adj.}{apto | qualificado | apresentável}
  \end{Phonetics}
\end{Entry}

\begin{Entry}{够得着}{11,11,11}{⼣,⼻,⽬}
  \begin{Phonetics}{够得着}{gou4de5zhao2}
    \definition{v.}{estar à altura | alcançar}
  \end{Phonetics}
\end{Entry}

%%%%%%%%%% 奢 %%%%%%%%%%
\subsection*{奢}\addcontentsline{loh}{figure}{奢}

\begin{Entry}{奢}{11}{⼤}
  \begin{Phonetics}{奢}{she1}
    \definition{adj.}{luxuoso; extravagante | excessivo; desmedido; extravagante}
  \end{Phonetics}
\end{Entry}

\begin{Entry}{奢侈}{11,8}{⼤,⼈}
  \begin{Phonetics}{奢侈}{she1chi3}[][HSK 7-9]
    \definition{adj.}{luxuoso; de luxo}
  \end{Phonetics}
\end{Entry}

\begin{Entry}{奢望}{11,11}{⼤,⽉}
  \begin{Phonetics}{奢望}{she1wang4}[][HSK 7-9]
    \definition{s.}{desejos desmedidos; esperanças extravagantes; esperança excessiva}
    \definition{v.}{esperar demais de alguém; nutrir expectativas excessivas}
  \end{Phonetics}
\end{Entry}

%%%%%%%%%% 娶 %%%%%%%%%%
\subsection*{娶}\addcontentsline{loh}{figure}{娶}

\begin{Entry}{娶}{11}{⼥}
  \begin{Phonetics}{娶}{qu3}[][HSK 7-9]
    \definition{v.}{casar (com uma mulher); tomar por esposa}
  \end{Phonetics}
\end{Entry}

%%%%%%%%%% 婚 %%%%%%%%%%
\subsection*{婚}\addcontentsline{loh}{figure}{婚}

\begin{Entry}{婚}{11}{⼥}
  \begin{Phonetics}{婚}{hun1}
    \definition{s.}{casamento}
    \definition{v.}{casar}
  \end{Phonetics}
\end{Entry}

\begin{Entry}{婚礼}{11,5}{⼥,⽰}
  \begin{Phonetics}{婚礼}{hun1li3}[][HSK 4]
    \definition[场]{s.}{casamento; núpcias; cerimônia de casamento}
  \end{Phonetics}
\end{Entry}

\begin{Entry}{婚纱}{11,7}{⼥,⽷}
  \begin{Phonetics}{婚纱}{hun1sha1}[][HSK 7-9]
    \definition[件,套,个]{s.}{vestido de noiva; um vestido especial usado pela noiva em seu casamento}
  \end{Phonetics}
\end{Entry}

\begin{Entry}{婚姻}{11,9}{⼥,⼥}
  \begin{Phonetics}{婚姻}{hun1yin1}[][HSK 7-9]
    \definition[桩,次,段]{s.}{casamento; matrimônio}
  \end{Phonetics}
\end{Entry}

%%%%%%%%%% 宿 %%%%%%%%%%
\subsection*{宿}\addcontentsline{loh}{figure}{宿}

\begin{Entry}{宿}{11}{⼧}
  \begin{Phonetics}{宿}{su4}
    \definition*{s.}{Sobrenome: Su}
    \definition{adj.}{de longa data; antigo; velho | veterano; velho; experiente}
    \definition{v.}{hospedar-se para passar a noite; passar a noite}
  \end{Phonetics}
  \begin{Phonetics}{宿}{xiu3}
    \definition{s.}{usado para calcular a noite}[谈了半宿。===Conversamos por metade da noite.]
  \end{Phonetics}
  \begin{Phonetics}{宿}{xiu4}
    \definition{s.}{(astronomia) um termo antigo para constelação}
  \end{Phonetics}
\end{Entry}

\begin{Entry}{宿舍}{11,8}{⼧,⾆}
  \begin{Phonetics}{宿舍}{su4she4}[][HSK 5]
    \definition[间,幢]{s.}{alojamento; dormitório; república; albergue; casas onde escolas, empresas, etc. acomodam seus alunos ou funcionários}
  \end{Phonetics}
\end{Entry}

%%%%%%%%%% 寂 %%%%%%%%%%
\subsection*{寂}\addcontentsline{loh}{figure}{寂}

\begin{Entry}{寂}{11}{⼧}
  \begin{Phonetics}{寂}{ji4}
    \definition{adj.}{quieto; parado; silencioso | solitário}
  \end{Phonetics}
\end{Entry}

\begin{Entry}{寂寞}{11,13}{⼧,⼧}
  \begin{Phonetics}{寂寞}{ji4mo4}[][HSK 7-9]
    \definition{adj.}{solitário; só; isolado; deserto | quieto; parado; silencioso}
  \end{Phonetics}
\end{Entry}

\begin{Entry}{寂寥}{11,14}{⼧,⼧}
  \begin{Phonetics}{寂寥}{ji4liao2}
    \definition{s.}{solidão | vasto e vazio | quieto e desolado (literário)}
  \end{Phonetics}
\end{Entry}

\begin{Entry}{寂静}{11,14}{⼧,⾭}
  \begin{Phonetics}{寂静}{ji4jing4}[][HSK 7-9]
    \definition{adj.}{quieto; parado; silencioso; sem som; muito silencioso}
  \end{Phonetics}
\end{Entry}

%%%%%%%%%% 寄 %%%%%%%%%%
\subsection*{寄}\addcontentsline{loh}{figure}{寄}

\begin{Entry}{寄}{11}{⼧}
  \begin{Phonetics}{寄}{ji4}[][HSK 4]
    \definition*{s.}{Sobrenome: Ji}
    \definition{adj.}{adotado; fomentado; promovido}
    \definition{v.}{enviar; postar; remeter | confiar; depositar; colocar | depender de; apegar-se a}
  \end{Phonetics}
\end{Entry}

\begin{Entry}{寄予}{11,4}{⼧,⼅}
  \begin{Phonetics}{寄予}{ji4yu3}
    \definition{v.}{expressar | colocar (esperança, importância, etc.) em | mostrar}
  \end{Phonetics}
\end{Entry}

\begin{Entry}{寄生}{11,5}{⼧,⽣}
  \begin{Phonetics}{寄生}{ji4sheng1}
    \definition{s.}{parasita | parasitismo}
    \definition{v.}{viver tirando vantagem dos outros | viver dentro ou sobre outro organismo como um parasita}
  \end{Phonetics}
\end{Entry}

\begin{Entry}{寄生生活}{11,5,5,9}{⼧,⽣,⽣,⽔}
  \begin{Phonetics}{寄生生活}{ji4sheng1sheng1huo2}
    \definition{s.}{parasitismo | vida parasitária}
  \end{Phonetics}
\end{Entry}

\begin{Entry}{寄存}{11,6}{⼧,⼦}
  \begin{Phonetics}{寄存}{ji4cun2}
    \definition{v.}{depositar | deixar algo com alguém | armazenar}
  \end{Phonetics}
\end{Entry}

\begin{Entry}{寄托}{11,6}{⼧,⼿}
  \begin{Phonetics}{寄托}{ji4tuo1}[][HSK 7-9]
    \definition{v.}{deixar com alguém; confiar aos cuidados de alguém; confiar | repousar; colocar (esperança, etc.) em; encontrar sustento em; depositar ideais, esperanças, sentimentos, etc. em (alguém ou alguma coisa)}
  \end{Phonetics}
\end{Entry}

\begin{Entry}{寄卖}{11,8}{⼧,⼗}
  \begin{Phonetics}{寄卖}{ji4mai4}
    \definition{v.}{consignar para venda}
  \end{Phonetics}
\end{Entry}

\begin{Entry}{寄居}{11,8}{⼧,⼫}
  \begin{Phonetics}{寄居}{ji4ju1}
    \definition{s.}{morar longe de casa}
  \end{Phonetics}
\end{Entry}

\begin{Entry}{寄放}{11,8}{⼧,⽅}
  \begin{Phonetics}{寄放}{ji4fang4}
    \definition{v.}{deixar algo com alguém}
  \end{Phonetics}
\end{Entry}

\begin{Entry}{寄养}{11,9}{⼧,⼋}
  \begin{Phonetics}{寄养}{ji4yang3}
    \definition{v.}{embarcar | promover | colocar sob os cuidados de alguém (uma criança, animal de estimação, etc.)}
  \end{Phonetics}
\end{Entry}

\begin{Entry}{寄送}{11,9}{⼧,⾡}
  \begin{Phonetics}{寄送}{ji4song4}
    \definition{v.}{enviar | transmitir}
  \end{Phonetics}
\end{Entry}

\begin{Entry}{寄递}{11,10}{⼧,⾡}
  \begin{Phonetics}{寄递}{ji4di4}
    \definition{s.}{entrega de correspondência}
  \end{Phonetics}
\end{Entry}

\begin{Entry}{寄售}{11,11}{⼧,⼝}
  \begin{Phonetics}{寄售}{ji4shou4}
    \definition{v.}{venda em consignação}
  \end{Phonetics}
\end{Entry}

\begin{Entry}{寄宿}{11,11}{⼧,⼧}
  \begin{Phonetics}{寄宿}{ji4su4}
    \definition{s.}{embarque}
    \definition{v.}{embarcar}
  \end{Phonetics}
\end{Entry}

\begin{Entry}{寄望}{11,11}{⼧,⽉}
  \begin{Phonetics}{寄望}{ji4wang4}
    \definition{v.}{depositar esperanças em}
  \end{Phonetics}
\end{Entry}

%%%%%%%%%% 密 %%%%%%%%%%
\subsection*{密}\addcontentsline{loh}{figure}{密}

\begin{Entry}{密}{11}{⼧}
  \begin{Phonetics}{密}{mi4}[][HSK 4]
    \definition*{s.}{Sobrenome: Mi}
    \definition{adj.}{fechado; denso; espesso | íntimo; próximo; afetuoso | delicado; fino; cuidadoso; meticuloso}
    \definition{adv.}{secretamente}
    \definition{s.}{segredo | densidade | senha; \emph{password}}
  \end{Phonetics}
\end{Entry}

\begin{Entry}{密不可分}{11,4,5,4}{⼧,⼀,⼝,⼑}
  \begin{Phonetics}{密不可分}{mi4bu4ke3fen1}[][HSK 7-9]
    \definition{expr.}{inextricavelmente ligados (expressão idiomática) | inseparáveis}
  \end{Phonetics}
\end{Entry}

\begin{Entry}{密切}{11,4}{⼧,⼑}
  \begin{Phonetics}{密切}{mi4qie4}[][HSK 4]
    \definition{adj.}{próximo; íntimo; relacionamento próximo}
    \definition{adv.}{cuidadosamente; atentamente; intimamente}
    \definition{v.}{tornar"-se próximo; tornar"-se íntimo; conectar"-se}
  \end{Phonetics}
\end{Entry}

\begin{Entry}{密码}{11,8}{⼧,⽯}
  \begin{Phonetics}{密码}{mi4ma3}[][HSK 4]
    \definition[个,种]{s.}{código; senha; um código secreto especialmente formulado usado entre as partes acordadas (diferente do 明码)}
  \seealsoref{明码}{ming2ma3}
  \end{Phonetics}
\end{Entry}

\begin{Entry}{密封}{11,9}{⼧,⼨}
  \begin{Phonetics}{密封}{mi4feng1}[][HSK 7-9]
    \definition{v.}{selar; vedar; selar hermeticamente; selar completamente; lacrar rigorosamente}
  \end{Phonetics}
\end{Entry}

\begin{Entry}{密度}{11,9}{⼧,⼴}
  \begin{Phonetics}{密度}{mi4du4}[][HSK 7-9]
    \definition[口]{s.}{densidade; espessura | Física: densidade; a gravidade específica é a razão entre a massa de um objeto e seu volume; anteriormente, era conhecida como densidade específica}
  \end{Phonetics}
\end{Entry}

\begin{Entry}{密集}{11,12}{⼧,⾫}
  \begin{Phonetics}{密集}{mi4ji2}[][HSK 7-9]
    \definition{adj.}{denso; intensivo; concentrado}
    \definition{v.}{concentrar"-se; aglomerar"-se}
  \end{Phonetics}
\end{Entry}

%%%%%%%%%% 屠 %%%%%%%%%%
\subsection*{屠}\addcontentsline{loh}{figure}{屠}

\begin{Entry}{屠}{11}{⼫}
  \begin{Phonetics}{屠}{tu2}
    \definition*{s.}{Sobrenome: Tu}
    \definition[个,位,名,些]{v.}{abater (animais para alimentação) | massacre; carnificina}
  \end{Phonetics}
\end{Entry}

\begin{Entry}{屠杀}{11,6}{⼫,⽊}
  \begin{Phonetics}{屠杀}{tu2sha1}[][HSK 7-9]
    \definition{s.}{massacre; carnificina; matança; assassinatos em massa}
  \end{Phonetics}
\end{Entry}

%%%%%%%%%% 崇 %%%%%%%%%%
\subsection*{崇}\addcontentsline{loh}{figure}{崇}

\begin{Entry}{崇}{11}{⼭}
  \begin{Phonetics}{崇}{chong2}
    \definition*{s.}{Sobrenome: Chong}
    \definition{adj.}{alto; elevado; sublime}
    \definition{v.}{adorar; reverenciar; venerar; estimar | respeitar}
  \end{Phonetics}
\end{Entry}

\begin{Entry}{崇尚}{11,8}{⼭,⼩}
  \begin{Phonetics}{崇尚}{chong2shang4}[][HSK 7-9]
    \definition{v.}{sustentar; defender; valorizar}[我们崇尚公平与正义。===Nós defendemos a justiça e a equidade.]
  \end{Phonetics}
\end{Entry}

\begin{Entry}{崇拜}{11,9}{⼭,⼿}
  \begin{Phonetics}{崇拜}{chong2bai4}[][HSK 6]
    \definition{v.}{adorar; idolatrar; venerar}
  \end{Phonetics}
\end{Entry}

\begin{Entry}{崇高}{11,10}{⼭,⾼}
  \begin{Phonetics}{崇高}{chong2gao1}[][HSK 7-9]
    \definition{adj.}{alto; elevado; sublime; nobre}
  \end{Phonetics}
\end{Entry}

%%%%%%%%%% 崖 %%%%%%%%%%
\subsection*{崖}\addcontentsline{loh}{figure}{崖}

\begin{Entry}{崖}{11}{⼭}
  \begin{Phonetics}{崖}{ya2}
    \definition{s.}{precipício | penhasco}
  \end{Phonetics}
\end{Entry}

%%%%%%%%%% 崛 %%%%%%%%%%
\subsection*{崛}\addcontentsline{loh}{figure}{崛}

\begin{Entry}{崛}{11}{⼭}
  \begin{Phonetics}{崛}{jue2}
    \definition{v.}{Literário: subir abruptamente; levantar"-se abruptamente; empinar}
  \end{Phonetics}
\end{Entry}

\begin{Entry}{崛起}{11,10}{⼭,⾛}
  \begin{Phonetics}{崛起}{jue2qi3}[][HSK 7-9]
    \definition{v.}{surgir abruptamente; subir repentinamente; aparecer de repente no horizonte | ganhar destaque; ascender}
  \end{Phonetics}
\end{Entry}

%%%%%%%%%% 崩 %%%%%%%%%%
\subsection*{崩}\addcontentsline{loh}{figure}{崩}

\begin{Entry}{崩}{11}{⼭}
  \begin{Phonetics}{崩}{beng1}
    \definition{v.}{colapsar |  estourar; quebrar | atingir por explosão | matar atirando; atirar; executar | (de um imperador) morrer | rachar; romper | atingir | executar atirando}
  \end{Phonetics}
\end{Entry}

\begin{Entry}{崩溃}{11,12}{⼭,⽔}
  \begin{Phonetics}{崩溃}{beng1kui4}[][HSK 7-9]
    \definition{v.}{colapsar; desmoronar; cair aos pedaços; as coisas estão destruídas; as emoções das pessoas estão fora de controle}
  \end{Phonetics}
\end{Entry}

%%%%%%%%%% 巢 %%%%%%%%%%
\subsection*{巢}\addcontentsline{loh}{figure}{巢}

\begin{Entry}{巢}{11}{⼮}
  \begin{Phonetics}{巢}{chao2}
    \definition*{s.}{Sobrenome: Chao}
    \definition[个]{s.}{ninho (de aves, insetos, etc.)}
  \end{Phonetics}
\end{Entry}

%%%%%%%%%% 常 %%%%%%%%%%
\subsection*{常}\addcontentsline{loh}{figure}{常}

\begin{Entry}{常}{11}{⼱}
  \begin{Phonetics}{常}{chang2}[][HSK 1]
    \definition*{s.}{Sobrenome: Chang}
    \definition{adj.}{normal; comum; ordinário; indica frequência, normalidade, universalidade | constante; invariável; imutável; permanente}
    \definition{adv.}{frequentemente; geralmente; com frequência;}
    \definition{s.}{normas; disciplina, ordem social e lei e ordem do Estado}
  \end{Phonetics}
\end{Entry}

\begin{Entry}{常人}{11,2}{⼱,⼈}
  \begin{Phonetics}{常人}{chang2ren2}[][HSK 7-9]
    \definition{s.}{pessoa comum; homem da rua}
  \end{Phonetics}
\end{Entry}

\begin{Entry}{常见}{11,4}{⼱,⾒}
  \begin{Phonetics}{常见}{chang2jian4}[][HSK 2]
    \definition{adj.}{comum; frequentemente visto}
  \end{Phonetics}
\end{Entry}

\begin{Entry}{常用}{11,5}{⼱,⽤}
  \begin{Phonetics}{常用}{chang2yong4}[][HSK 2]
    \definition{adj.}{em uso comum; frequentemente utilizado}
  \end{Phonetics}
\end{Entry}

\begin{Entry}{常年}{11,6}{⼱,⼲}
  \begin{Phonetics}{常年}{chang2nian2}[][HSK 6]
    \definition{adj.}{perene; anual}
    \definition{adv.}{ano após ano; ao longo do ano; durante todo o ano; longo prazo}
  \end{Phonetics}
\end{Entry}

\begin{Entry}{常问问题}{11,6,6,15}{⼱,⾨,⾨,⾴}
  \begin{Phonetics}{常问问题}{chang2wen4wen4ti2}
    \definition{s.}{FAQ; perguntas frequentes}
  \end{Phonetics}
\end{Entry}

\begin{Entry}{常识}{11,7}{⼱,⾔}
  \begin{Phonetics}{常识}{chang2shi2}[][HSK 4]
    \definition[门]{s.}{senso comum; conhecimento geral; conhecimento que uma pessoa comum deve ter}
  \end{Phonetics}
\end{Entry}

\begin{Entry}{常态}{11,8}{⼱,⼼}
  \begin{Phonetics}{常态}{chang2tai4}[][HSK 7-9]
    \definition{s.}{normalidade; \emph{habitus}; comportamento normal; condições normais; estado normal ou usual}
  \end{Phonetics}
\end{Entry}

\begin{Entry}{常规}{11,8}{⼱,⾒}
  \begin{Phonetics}{常规}{chang2gui1}[][HSK 6]
    \definition[个,种]{s.}{convenção; prática comum; rotina | (medicina) rotina | regra; sulco}
  \end{Phonetics}
\end{Entry}

\begin{Entry}{常常}{11,11}{⼱,⼱}
  \begin{Phonetics}{常常}{chang2chang2}[][HSK 1]
    \definition{adv.}{frequentemente; muitas vezes; geralmente; indica que a ação ocorreu várias vezes}
  \end{Phonetics}
\end{Entry}

\begin{Entry}{常理}{11,11}{⼱,⽟}
  \begin{Phonetics}{常理}{chang2li3}[][HSK 7-9]
    \definition{s.}{regra geral; o que é normal | senso comum; pensamento lógico | raciocínio convencional e moral}
  \end{Phonetics}
\end{Entry}

\begin{Entry}{常温}{11,12}{⼱,⽔}
  \begin{Phonetics}{常温}{chang2wen1}[][HSK 7-9]
    \definition{s.}{temperatura atmosférica normal; temperatura ordinária | homeotermia}
  \end{Phonetics}
\end{Entry}

%%%%%%%%%% 庵 %%%%%%%%%%
\subsection*{庵}\addcontentsline{loh}{figure}{庵}

\begin{Entry}{庵}{11}{⼴}
  \begin{Phonetics}{庵}{an1}
    \definition*{s.}{Sobrenome: An}
    \definition[个,座]{s.}{cabana | convento de freiras; templos budistas, principalmente onde vivem as freiras}
  \end{Phonetics}
\end{Entry}

%%%%%%%%%% 庶 %%%%%%%%%%
\subsection*{庶}\addcontentsline{loh}{figure}{庶}

\begin{Entry}{庶}{11}{⼴}
  \begin{Phonetics}{庶}{shu4}
    \definition*{s.}{Sobrenome: Shu}
    \definition{adj.}{multitudinário; numeroso}
    \definition{conj.}{para que; de modo a}
    \definition{s.}{da ou pela concubina (diferentemente da esposa legal); no sistema patriarcal, refere"-se ao ramo lateral da família}
  \end{Phonetics}
\end{Entry}

\begin{Entry}{庶民}{11,5}{⼴,⽒}
  \begin{Phonetics}{庶民}{shu4min2}
    \definition{s.}{(antigo) pessoas comuns | (antigo) plebeu; plebeus | (antigo) a multidão de pessoas comuns (na literatura erudita)}
  \end{Phonetics}
\end{Entry}

%%%%%%%%%% 康 %%%%%%%%%%
\subsection*{康}\addcontentsline{loh}{figure}{康}

\begin{Entry}{康}{11}{⼴}
  \begin{Phonetics}{康}{kang1}
    \definition*{s.}{Sobrenome: Kang}
    \definition{adj.}{saudável |  fácil; pacífico; abundante | amplo; largo | Dialeto: de baixa qualidade; inferior}
    \definition{s.}{bem"-estar; saúde | palha; farelo; casca}
    \definition{v.}{(normalmente de um rabanete) tornar"-se esponjoso}
  \end{Phonetics}
\end{Entry}

\begin{Entry}{康复}{11,9}{⼴,⼢}
  \begin{Phonetics}{康复}{kang1fu4}[][HSK 6]
    \definition{v.}{Saúde: estaurar; recuperar; reabilitar}
  \end{Phonetics}
\end{Entry}

%%%%%%%%%% 廊 %%%%%%%%%%
\subsection*{廊}\addcontentsline{loh}{figure}{廊}

\begin{Entry}{廊}{11}{⼴}
  \begin{Phonetics}{廊}{lang2}
    \definition[个]{s.}{varanda; corredor}
  \end{Phonetics}
\end{Entry}

\begin{Entry}{廊坊}{11,7}{⼴,⼟}
  \begin{Phonetics}{廊坊}{lang2fang2}
    \definition*{s.}{Cidade de Langfang em Hebei}
  \end{Phonetics}
\end{Entry}

%%%%%%%%%% 弹 %%%%%%%%%%
\subsection*{弹}\addcontentsline{loh}{figure}{弹}

\begin{Entry}{弹}{11}{⼸}
  \begin{Phonetics}{弹}{dan4}
    \definition{s.}{bola; pelota; pequenas bolas disparadas com um estilingue | bomba; bala; explosivos que podem ser lançados ou arremessados, com poder destrutivo e letal}
  \end{Phonetics}
  \begin{Phonetics}{弹}{tan2}[][HSK 5]
    \definition{v.}{enviar; atirar (como com uma catapulta, etc.); usar a elasticidade de um objeto para lançar outro objeto | afofar; preparar fibras; usar um dispositivo elástico para amolecer as fibras | virar; sacudir | dedilhar; tocar (um instrumento musical de cordas) | acusar; atacar; criticar; relatar | saltar; quicar}
  \end{Phonetics}
\end{Entry}

\begin{Entry}{弹性}{11,8}{⼸,⼼}
  \begin{Phonetics}{弹性}{tan2xing4}[][HSK 7-9]
    \definition{adj.}{elástico; flexível; essa metáfora descreve a natureza das coisas, que são ajustáveis e adaptáveis de acordo com as necessidades reais}
    \definition{s.}{elasticidade; resiliência; a propriedade de um objeto se deformar sob a ação de uma força externa e retornar à sua forma original após a remoção dessa força}
  \synonymref{韧性}{ren4xing4}
  \end{Phonetics}
\end{Entry}

%%%%%%%%%% 彩 %%%%%%%%%%
\subsection*{彩}\addcontentsline{loh}{figure}{彩}

\begin{Entry}{彩}{11}{⼺}
  \begin{Phonetics}{彩}{cai3}
    \definition{s.}{cor | aplausos; vivas | variedade; brilho; esplendor | prêmio; loteria | sangue de uma ferida | habilidades especiais empregadas em mágica ou ópera para alcançar um efeito desejado | seda colorida | cores variadas | graça na arte; graciosidade | prêmio de loteria; ganhos | efeitos especiais no teatro chinês (simbolizando sangue, fogo, etc.)}
  \end{Phonetics}
\end{Entry}

\begin{Entry}{彩电}{11,5}{⼺,⽥}
  \begin{Phonetics}{彩电}{cai3dian4}[][HSK 7-9]
    \definition[台,个]{s.}{TV à cores}
  \end{Phonetics}
\end{Entry}

\begin{Entry}{彩色}{11,6}{⼺,⾊}
  \begin{Phonetics}{彩色}{cai3se4}[][HSK 3]
    \definition[个,种]{s.}{multicolorido; cor; várias cores}
  \end{Phonetics}
\end{Entry}

\begin{Entry}{彩虹}{11,9}{⼺,⾍}
  \begin{Phonetics}{彩虹}{cai3hong2}[][HSK 7-9]
    \definition[道,条]{s.}{arco-íris}
  \end{Phonetics}
\end{Entry}

\begin{Entry}{彩票}{11,11}{⼺,⽰}
  \begin{Phonetics}{彩票}{cai3piao4}[][HSK 5]
    \definition[张,注]{s.}{bilhete de loteria; um título com números, vendido pelo valor de face; após o sorteio, o portador do bilhete premiado pode reivindicar o prêmio de acordo com o regulamento}
  \end{Phonetics}
\end{Entry}

\begin{Entry}{彩霞}{11,17}{⼺,⾬}
  \begin{Phonetics}{彩霞}{cai3xia2}[][HSK 7-9]
    \definition[片]{s.}{nuvens rosadas (ou cor"-de"-rosa) | nuvens tingidas com tons de pôr do sol; nuvens coloridas}
  \end{Phonetics}
\end{Entry}

%%%%%%%%%% 彪 %%%%%%%%%%
\subsection*{彪}\addcontentsline{loh}{figure}{彪}

\begin{Entry}{彪}{11}{⾌}
  \begin{Phonetics}{彪}{biao1}
    \definition*{s.}{Sobrenome: Biao}
    \definition{adj.}{semelhante a um tigre (metáfora para estatura alta)}
    \definition{s.}{tigre jovem}
  \end{Phonetics}
\end{Entry}

%%%%%%%%%% 彬 %%%%%%%%%%
\subsection*{彬}\addcontentsline{loh}{figure}{彬}

\begin{Entry}{彬}{11}{⼺}
  \begin{Phonetics}{彬}{bin1}
    \definition*{s.}{Sobrenome: Bin}
    \definition{adj.}{Literário: fino; elegante}
  \end{Phonetics}
\end{Entry}

\begin{Entry}{彬彬有礼}{11,11,6,5}{⼺,⼺,⽉,⽰}
  \begin{Phonetics}{彬彬有礼}{bin1bin1-you3li3}[][HSK 7-9]
    \definition{expr.}{refinado e cortês; urbano}
  \end{Phonetics}
\end{Entry}

%%%%%%%%%% 得 %%%%%%%%%%
\subsection*{得}\addcontentsline{loh}{figure}{得}

\begin{Entry}{得}{11}{⼻}
  \begin{Phonetics}{得}{de2}[][HSK 2]
    \definition{adj.}{adequado; apropriado | satisfeito; complacente; orgulhoso de si mesmo}
    \definition{interj.}{usado para encerrar uma conversa para indicar concordância ou proibição | usado quando a situação não é a esperada, para expressar impotência}
    \definition{v.}{obter; conseguir; ganhar |  (de um cálculo) igual; resultar em | estar pronto; estar acabado | pegar; apanhar; contrair uma doença}
    \definition{v.aux.}{usado antes de outros verbos para expressar permissão | usado antes de outros verbos para indicar que é possível (usado principalmente na forma negativa) | usado em conversas para indicar que não há necessidade de dizer mais nada}
  \antonymref{失}{shi1}
  \end{Phonetics}
  \begin{Phonetics}{得}{de5}[][HSK 2]
    \definition{part.}{depois de um verbo ou adjetivo para expressar possibilidade ou capacidade | entre um verbo e seu complemento para expressar possibilidade | ligando um verbo ou um adjetivo a um complemento que descreve a maneira ou o grau}
  \end{Phonetics}
  \begin{Phonetics}{得}{dei3}[][HSK 4]
    \definition{v.}{precisar; expressa uma necessidade lógica, factual ou subjetiva; deve; é necessário | ter de; ser obrigado a; indica uma necessidade de vontade ou de fato | certamente irá; expressa a inevitabilidade da especulação}
  \end{Phonetics}
\end{Entry}

\begin{Entry}{得了}{11,2}{⼻,⼅}
  \begin{Phonetics}{得了}{de2le5}[][HSK 5]
    \definition{expr.}{Tudo bem!; É o bastante!}
  \end{Phonetics}
  \begin{Phonetics}{得了}{de2liao3}
    \definition{adj.}{(enfaticamente, em perguntas retóricas) possível; indica que a situação é séria (usado principalmente em perguntas retóricas ou formas negativas)}
  \end{Phonetics}
\end{Entry}

\begin{Entry}{得力}{11,2}{⼻,⼒}
  \begin{Phonetics}{得力}{de2li4}[][HSK 7-9]
    \definition{adj.}{capaz; competente; capaz de fazer coisas | eficiente; poderoso}
    \definition{v.}{beneficiar"-se de; obter ajuda de; beneficiar}
  \end{Phonetics}
\end{Entry}

\begin{Entry}{得不偿失}{11,4,11,5}{⼻,⼀,⼈,⼤}
  \begin{Phonetics}{得不偿失}{de2bu4chang2shi1}[][HSK 7-9]
    \definition{expr.}{``A perda supera o ganho.''; ``Os ganhos não compensam as perdas.''; perder mais do que ganhar; ``O jogo não vale a pena.''; ``O que é ganho não compensa o que é perdido.''}
  \end{Phonetics}
\end{Entry}

\begin{Entry}{得以}{11,4}{⼻,⼈}
  \begin{Phonetics}{得以}{de2yi3}[][HSK 5]
    \definition{v.}{ser capaz de; para que\dots possa (ou possa)\dots}
  \end{Phonetics}
\end{Entry}

\begin{Entry}{得分}{11,4}{⼻,⼑}
  \begin{Phonetics}{得分}{de2 fen1}[][HSK 3]
    \definition{s.}{pontuação; classificação; nota; pontuação obtida em jogos ou competições}
    \definition{v.}{fazer pontos; pontuar}
  \end{Phonetics}
\end{Entry}

\begin{Entry}{得天独厚}{11,4,9,9}{⼻,⼤,⽝,⼚}
  \begin{Phonetics}{得天独厚}{de2tian1du2hou4}[][HSK 7-9]
    \definition{expr.}{ser ricamente dotado pela natureza; abundar em dádivas da natureza; desfrutar de vantagens excepcionais | abençoado pelo céu | desfrutar de vantagens excepcionais | favorecido pela natureza}
  \end{Phonetics}
\end{Entry}

\begin{Entry}{得手}{11,4}{⼻,⼿}
  \begin{Phonetics}{得手}{de2shou3}[][HSK 7-9]
    \definition{adj.}{Coloquial: prático; conveniente e fácil de usar}
    \definition{v.}{fazer algo suavemente; ter sucesso; atingir seu objetivo | ir suavemente; sair; fazer bem; fazer as coisas sem problemas}
  \end{Phonetics}
\end{Entry}

\begin{Entry}{得出}{11,5}{⼻,⼐}
  \begin{Phonetics}{得出}{de2 chu1}[][HSK 2]
    \definition{v.}{chegar (a uma conclusão); obter (a um resultado); deduzir ou calcular (conclusão ou resultado)}
  \end{Phonetics}
\end{Entry}

\begin{Entry}{得失}{11,5}{⼻,⼤}
  \begin{Phonetics}{得失}{de2shi1}[][HSK 7-9]
    \definition{s.}{ganho e perda; sucesso e fracasso | méritos e deméritos; vantagens e desvantagens; prós e contras}
  \end{Phonetics}
\end{Entry}

\begin{Entry}{得当}{11,6}{⼻,⼹}
  \begin{Phonetics}{得当}{de2dang4}[][HSK 7-9]
    \definition{adj.}{apropriado; próprio; adequado | apto}
  \end{Phonetics}
\end{Entry}

\begin{Entry}{得体}{11,7}{⼻,⼈}
  \begin{Phonetics}{得体}{de2ti3}[][HSK 7-9]
    \definition{adj.}{(fala, comportamento, etc.) apropriado; moderado}
  \end{Phonetics}
\end{Entry}

\begin{Entry}{得到}{11,8}{⼻,⼑}
  \begin{Phonetics}{得到}{de2 dao4}[][HSK 1]
    \definition{v.}{obter; conseguir; ganhar; receber; possuir algo; adquirir}
  \end{Phonetics}
\end{Entry}

\begin{Entry}{得知}{11,8}{⼻,⽮}
  \begin{Phonetics}{得知}{de2zhi1}[][HSK 7-9]
    \definition{v.}{saber; ser informado de; aprender}
  \end{Phonetics}
\end{Entry}

\begin{Entry}{得益于}{11,10,3}{⼻,⽫,⼆}
  \begin{Phonetics}{得益于}{de2yi4 yu2}[][HSK 7-9]
    \definition{s.}{correlação positiva; benefício}
  \end{Phonetics}
\end{Entry}

\begin{Entry}{得意}{11,13}{⼻,⼼}
  \begin{Phonetics}{得意}{de2yi4}[][HSK 4]
    \definition{adj.}{complacente; orgulhoso de si mesmo; satisfeito consigo mesmo}
  \end{Phonetics}
\end{Entry}

\begin{Entry}{得意扬扬}{11,13,6,6}{⼻,⼼,⼿,⼿}
  \begin{Phonetics}{得意扬扬}{de2yi4-yang2yang2}[][HSK 7-9]
    \definition{expr.}{orgulhoso e complacente | estar imensamente orgulhoso; parecer triunfante}
  \end{Phonetics}
\end{Entry}

\begin{Entry}{得罪}{11,13}{⼻,⽹}
  \begin{Phonetics}{得罪}{de2zui4}[][HSK 7-9]
    \definition{v.}{ofender; desagradar; causar desprazer ou ressentimento}
  \end{Phonetics}
\end{Entry}

%%%%%%%%%% 徘 %%%%%%%%%%
\subsection*{徘}\addcontentsline{loh}{figure}{徘}

\begin{Entry}{徘}{11}{⼻}
  \begin{Phonetics}{徘}{pai2}
    \definition{adj.}{irresoluto; indeciso}
    \definition{v.}{vagar}
  \end{Phonetics}
\end{Entry}

\begin{Entry}{徘徊}{11,9}{⼻,⼻}
  \begin{Phonetics}{徘徊}{pai2huai2}[][HSK 7-9]
    \definition{v.}{andar de um lado para o outro no mesmo lugar | Figurativo: vacilar; hesitar; uma metáfora para hesitação e indecisão | flutuar; essa metáfora descreve as coisas mudando para cima e para baixo dentro de uma determinada faixa}
  \end{Phonetics}
\end{Entry}

%%%%%%%%%% 悉 %%%%%%%%%%
\subsection*{悉}\addcontentsline{loh}{figure}{悉}

\begin{Entry}{悉}{11}{⼼}
  \begin{Phonetics}{悉}{xi1}
    \definition*{s.}{Sobrenome: Xi}
    \definition{adj.}{tudo; inteiro; total | detalhado}
    \definition{v.}{saber; aprender; ser informado de}
  \end{Phonetics}
\end{Entry}

\begin{Entry}{悉心}{11,4}{⼼,⼼}
  \begin{Phonetics}{悉心}{xi1xin1}
    \definition{adv.}{colocar o coração (e a alma) em algo | com muito cuidado}
  \end{Phonetics}
\end{Entry}

\begin{Entry}{悉尼}{11,5}{⼼,⼫}
  \begin{Phonetics}{悉尼}{xi1ni2}
    \definition*{s.}{Sidney}
  \end{Phonetics}
\end{Entry}

\begin{Entry}{悉数}{11,13}{⼼,⽁}
  \begin{Phonetics}{悉数}{xi1shu3}
    \definition{adv.}{enumerar em detalhes | explicar claramente}
  \end{Phonetics}
  \begin{Phonetics}{悉数}{xi1shu4}
    \definition{adv.}{todos | cada um | toda a soma}
  \end{Phonetics}
\end{Entry}

%%%%%%%%%% 患 %%%%%%%%%%
\subsection*{患}\addcontentsline{loh}{figure}{患}

\begin{Entry}{患}{11}{⼼}
  \begin{Phonetics}{患}{huan4}[][HSK 7-9]
    \definition*{s.}{Sobrenome: Huan}
    \definition{s.}{perigo; problema; desastre; flagelo | preocupação; ansiedade}
    \definition{v.}{contrair (doença); sofrer de}
  \end{Phonetics}
\end{Entry}

\begin{Entry}{患有}{11,6}{⼼,⽉}
  \begin{Phonetics}{患有}{huan4you3}[][HSK 7-9]
    \definition{v.}{sofrer de; refere"-se a alguém que sofre de uma doença ou condição específica}
  \end{Phonetics}
\end{Entry}

\begin{Entry}{患者}{11,8}{⼼,⽼}
  \begin{Phonetics}{患者}{huan4zhe3}[][HSK 6]
    \definition[个,位,名]{s.}{paciente; sofredor; pessoas com certas doenças}
  \end{Phonetics}
\end{Entry}

\begin{Entry}{患病}{11,10}{⼼,⽧}
  \begin{Phonetics}{患病}{huan4bing4}[][HSK 7-9]
    \definition{v.}{estar doente; ficar doente; adoecer; sofrer de uma doença}
  \end{Phonetics}
\end{Entry}

%%%%%%%%%% 您 %%%%%%%%%%
\subsection*{您}\addcontentsline{loh}{figure}{您}

\begin{Entry}{您}{11}{⼼}
  \begin{Phonetics}{您}{nin2}[][HSK 1]
    \definition{pron.}{você; a forma de tratamento respeitosa da segunda pessoa do singular 你}
  \seealsoref{你}{ni3}
  \end{Phonetics}
\end{Entry}

%%%%%%%%%% 悬 %%%%%%%%%%
\subsection*{悬}\addcontentsline{loh}{figure}{悬}

\begin{Entry}{悬}{11}{⼼}
  \begin{Phonetics}{悬}{xuan2}[][HSK 6]
    \definition{adj.}{pendente; não resolvido; sem nenhum resultado | distante; a distância é grande; a diferença é grande | Dialeto: perigoso}
    \definition{v.}{pendurar; suspender | levantar; elevar | sentir"-se ansioso; ser solícito | imaginar}
  \end{Phonetics}
\end{Entry}

\begin{Entry}{悬心吊胆}{11,4,6,9}{⼼,⼼,⼝,⾁}
  \begin{Phonetics}{悬心吊胆}{xuan2xin1-diao4dan3}
    \definition{expr.}{de tirar o fôlego; estar ansioso e com medo; estar à beira de um ataque de nervos; ter o coração na boca; estar em suspense}
  \synonymref{提心吊胆}{ti2xin1-diao4dan3}
  \end{Phonetics}
\end{Entry}

\begin{Entry}{悬挂}{11,9}{⼼,⼿}
  \begin{Phonetics}{悬挂}{xuan2gua4}
    \definition{v.}{pendurar; pender; suspender; prender um objeto em um ou mais pontos em algum lugar com a ajuda de uma corda, gancho, prego, etc.}
  \end{Phonetics}
\end{Entry}

\begin{Entry}{悬崖}{11,11}{⼼,⼭}
  \begin{Phonetics}{悬崖}{xuan2ya2}
    \definition{s.}{precipício | penhasco}
  \end{Phonetics}
\end{Entry}

%%%%%%%%%% 悼 %%%%%%%%%%
\subsection*{悼}\addcontentsline{loh}{figure}{悼}

\begin{Entry}{悼}{11}{⼼}
  \begin{Phonetics}{悼}{dao4}
    \definition*{s.}{Sobrenome: Dao}
    \definition{v.}{lamentar; expressar pesar}
  \end{Phonetics}
\end{Entry}

\begin{Entry}{悼念}{11,8}{⼼,⼼}
  \begin{Phonetics}{悼念}{dao4nian4}[][HSK 7-9]
    \definition{v.}{lamentar; lamentar"-se por | lamentar por; expressar pesar}
  \end{Phonetics}
\end{Entry}

%%%%%%%%%% 情 %%%%%%%%%%
\subsection*{情}\addcontentsline{loh}{figure}{情}

\begin{Entry}{情}{11}{⼼}
  \begin{Phonetics}{情}{qing2}[][HSK 7-9]
    \definition{s.}{sentimento; afeição | amor; paixão | paixão sexual; luxúria | favor; gentileza | situação; circunstâncias; condição | razão; sentido | sensibilidades; sentimentos}
  \end{Phonetics}
\end{Entry}

\begin{Entry}{情人}{11,2}{⼼,⼈}
  \begin{Phonetics}{情人}{qing2ren2}[][HSK 7-9]
    \definition[对,个,位]{s.}{amante; namorado(a) | concubina; amante}
  \end{Phonetics}
\end{Entry}

\begin{Entry}{情不自禁}{11,4,6,13}{⼼,⼀,⾃,⽰}
  \begin{Phonetics}{情不自禁}{qing2bu2zi4jin1}[][HSK 7-9]
    \definition{expr.}{``Não consigo ajudar.''; não conseguir se conter; não conseguir evitar (fazer algo); ser tomado por um impulso repentino de; dominado pela emoção; enfatizando o controle total sobre as próprias emoções}
  \end{Phonetics}
\end{Entry}

\begin{Entry}{情节}{11,5}{⼼,⾋}
  \begin{Phonetics}{情节}{qing2jie2}[][HSK 5]
    \definition[个,段]{s.}{enredo; trama; desenrolar específico dos acontecimentos | circunstância; detalhes do crime ou erro | enredo; roteiro; refere"-se especificamente ao processo de desenvolvimento e evolução dos conflitos e contradições em obras literárias narrativas}
  \end{Phonetics}
\end{Entry}

\begin{Entry}{情况}{11,7}{⼼,⼎}
  \begin{Phonetics}{情况}{qing2kuang4}[][HSK 3]
    \definition[种,个,些]{s.}{condição; situação; circunstâncias; estado das coisas | mudanças notáveis e impactantes}
  \end{Phonetics}
\end{Entry}

\begin{Entry}{情形}{11,7}{⼼,⼺}
  \begin{Phonetics}{情形}{qing2xing5}[][HSK 5]
    \definition[个,种]{s.}{situação; condição; circunstâncias; estado de coisas; a situação específica das coisas}
  \end{Phonetics}
\end{Entry}

\begin{Entry}{情怀}{11,7}{⼼,⼼}
  \begin{Phonetics}{情怀}{qing2huai2}[][HSK 7-9]
    \definition{s.}{sentimentos; um estado de espírito que contém uma determinada emoção}
  \end{Phonetics}
\end{Entry}

\begin{Entry}{情报}{11,7}{⼼,⼿}
  \begin{Phonetics}{情报}{qing2bao4}[][HSK 7-9]
    \definition[个,份]{s.}{inteligência; informação; notícias e reportagens sobre determinada situação são frequentemente classificadas como confidenciais}
  \end{Phonetics}
\end{Entry}

\begin{Entry}{情侣}{11,8}{⼼,⼈}
  \begin{Phonetics}{情侣}{qing2lv3}[][HSK 7-9]
    \definition[对,双,群]{s.}{amantes; namorados; um casal apaixonado ou um deles}
  \end{Phonetics}
\end{Entry}

\begin{Entry}{情结}{11,9}{⼼,⽷}
  \begin{Phonetics}{情结}{qing2jie2}[][HSK 7-9]
    \definition{s.}{complexidade; a turbulência emocional em meu coração; um certo sentimento que frequentemente persiste em minha mente}
  \end{Phonetics}
\end{Entry}

\begin{Entry}{情调}{11,10}{⼼,⾔}
  \begin{Phonetics}{情调}{qing2diao4}[][HSK 7-9]
    \definition[出]{s.}{sentimento; apelo emocional; tom afetivo; o estilo expresso através de pensamentos e sentimentos; a natureza das coisas que podem evocar diversas emoções diferentes nas pessoas}
  \end{Phonetics}
\end{Entry}

\begin{Entry}{情谊}{11,10}{⼼,⾔}
  \begin{Phonetics}{情谊}{qing2yi4}[][HSK 7-9]
    \definition{s.}{amizade; sentimentos amigáveis; emoções amistosas; os sentimentos de carinho e amor entre pessoas}
  \end{Phonetics}
\end{Entry}

\begin{Entry}{情绪}{11,11}{⼼,⽷}
  \begin{Phonetics}{情绪}{qing2xu4}[][HSK 6]
    \definition[种,片,股,丝]{s.}{mau humor; depressão; um sentimento ruim no coração, especialmente um estado mental desagradável quando se sente injusto | emoção; humor; moral; sentimento; o estado mental de uma pessoa ao longo de um período de tempo}
  \end{Phonetics}
\end{Entry}

\begin{Entry}{情景}{11,12}{⼼,⽇}
  \begin{Phonetics}{情景}{qing2jing3}[][HSK 4]
    \definition[个,幕,种]{s.}{cena; vista; circunstâncias}
  \end{Phonetics}
\end{Entry}

\begin{Entry}{情感}{11,13}{⼼,⼼}
  \begin{Phonetics}{情感}{qing2gan3}[][HSK 3]
    \definition[份]{s.}{emoção; sentimento | afeição; apego; reações psicológicas positivas ou negativas a estímulos externos, como gosto, raiva, tristeza, medo, amor, nojo, etc.}
  \end{Phonetics}
\end{Entry}

\begin{Entry}{情愿}{11,14}{⼼,⽕}
  \begin{Phonetics}{情愿}{qing2yuan4}[][HSK 7-9]
    \definition{v.}{estar disposto a; estar genuinamente disposto a fazer algo que outros não estão dispostos a fazer}
  \end{Phonetics}
\end{Entry}

%%%%%%%%%% 惊 %%%%%%%%%%
\subsection*{惊}\addcontentsline{loh}{figure}{惊}

\begin{Entry}{惊}{11}{⼼}
  \begin{Phonetics}{惊}{jing1}[][HSK 7-9]
    \definition{v.}{assustar; ficar assustado; ficar nervoso devido a estímulo repentino; ficar com medo | surpreender; chocar; alarmar}
  \end{Phonetics}
\end{Entry}

\begin{Entry}{惊人}{11,2}{⼼,⼈}
  \begin{Phonetics}{惊人}{jing1ren2}[][HSK 6]
    \definition{adj.}{surpreso; espantado; atônito; surpreendente}
  \end{Phonetics}
\end{Entry}

\begin{Entry}{惊天动地}{11,4,6,6}{⼼,⼤,⼒,⼟}
  \begin{Phonetics}{惊天动地}{jing1tian1-dong4di4}[][HSK 7-9]
    \definition{expr.}{que abala os céus e a terra; que faz a terra tremer; assustar os céus e mover a terra; sacudir os céus e assustar a terra; que abala o mundo; de proporções sísmicas; de impacto mundial}
  \end{Phonetics}
\end{Entry}

\begin{Entry}{惊心动魄}{11,4,6,14}{⼼,⼼,⼒,⿁}
  \begin{Phonetics}{惊心动魄}{jing1xin1-dong4po4}[][HSK 7-9]
    \definition{expr.}{comovente; profundamente impactante; ficar apavorado (horror); de tirar o fôlego; de arrepiar os cabelos; emocionante; fazer o coração de alguém disparar; abalar alguém profundamente; deixar alguém sem fôlego}
  \end{Phonetics}
\end{Entry}

\begin{Entry}{惊叹}{11,5}{⼼,⼝}
  \begin{Phonetics}{惊叹}{jing1tan4}[][HSK 7-9]
    \definition{v.}{maravilhar"-se com; admirar"-se com; exclamar (com admiração)}
  \end{Phonetics}
\end{Entry}

\begin{Entry}{惊异}{11,6}{⼼,⼶}
  \begin{Phonetics}{惊异}{jing1yi4}
    \definition{adj.}{surpreso; admirado; estupefato; atônito}
  \end{Phonetics}
\end{Entry}

\begin{Entry}{惊讶}{11,6}{⼼,⾔}
  \begin{Phonetics}{惊讶}{jing1ya4}[][HSK 7-9]
    \definition{adj.}{surpreso; admirado; estupefato; atônito; sentindo"-me surpreso e estranho}
  \end{Phonetics}
\end{Entry}

\begin{Entry}{惊呆}{11,7}{⼼,⼝}
  \begin{Phonetics}{惊呆}{jing1dai1}
    \definition{adj.}{atordoado; estupefato; chocado}
  \end{Phonetics}
\end{Entry}

\begin{Entry}{惊奇}{11,8}{⼼,⼤}
  \begin{Phonetics}{惊奇}{jing1qi2}[][HSK 7-9]
    \definition{v.}{maravilhar"-se; ficar surpreso; ficar boquiaberto}
  \end{Phonetics}
\end{Entry}

\begin{Entry}{惊诧}{11,8}{⼼,⾔}
  \begin{Phonetics}{惊诧}{jing1cha4}[][HSK 7-9]
    \definition{adj.}{surpreso; admirado; estupefato | muito surpreso}
  \end{Phonetics}
\end{Entry}

\begin{Entry}{惊险}{11,9}{⼼,⾩}
  \begin{Phonetics}{惊险}{jing1xian3}[][HSK 7-9]
    \definition{adj.}{emocionante; de tirar o fôlego; alarmantemente perigoso}
  \end{Phonetics}
\end{Entry}

\begin{Entry}{惊喜}{11,12}{⼼,⼝}
  \begin{Phonetics}{惊喜}{jing1xi3}[][HSK 6]
    \definition{s.}{boa surpresa; agradavelmente surpreso}
  \end{Phonetics}
\end{Entry}

\begin{Entry}{惊慌}{11,12}{⼼,⼼}
  \begin{Phonetics}{惊慌}{jing1huang1}[][HSK 7-9]
    \definition{adj.}{assustado; alarmado; amedrontado; atemorizado; em pânico}
  \end{Phonetics}
\end{Entry}

\begin{Entry}{惊慌失措}{11,12,5,11}{⼼,⼼,⼤,⼿}
  \begin{Phonetics}{惊慌失措}{jing1huang1-shi1cuo4}[][HSK 7-9]
    \definition{expr.}{apavorado; tomado pelo pânico; em pânico; perder a cabeça de medo; no Capítulo 11 da Parte 4 de ``O Oriente'', de Wei Wei: ``Guo Xiang e seus homens lançaram um ataque feroz, e o inimigo, em pânico, esqueceu"-se de resistir, preocupando"-se apenas em subir nos tanques.''; confusão aterrorizante; ficar apavorado; ficar em pânico}
  \end{Phonetics}
\end{Entry}

\begin{Entry}{惊醒}{11,16}{⼼,⾣}
  \begin{Phonetics}{惊醒}{jing1xing3}[][HSK 7-9]
    \definition{v.}{acordar sobressaltado; despertado pelo susto; despertado pelo choque}
  \end{Phonetics}
\end{Entry}

%%%%%%%%%% 惋 %%%%%%%%%%
\subsection*{惋}\addcontentsline{loh}{figure}{惋}

\begin{Entry}{惋}{11}{⼼}
  \begin{Phonetics}{惋}{wan3}
    \definition{s.}{Literário: suspiro}
    \definition{v.}{Literário: suspirar}
  \end{Phonetics}
\end{Entry}

\begin{Entry}{惋惜}{11,11}{⼼,⼼}
  \begin{Phonetics}{惋惜}{wan3xi1}[][HSK 7-9]
    \definition{v.}{lamentar; sentir que é uma pena; sentir pena de alguém; estar arrependido; expressar simpatia e pena pelos infortúnios das pessoas ou por mudanças insatisfatórias nas coisas}
  \synonymref{可惜}{ke3xi1}
  \synonymref{怜惜}{lian2xi1}
  \synonymref{无奈}{wu2nai4}
  \synonymref{遗憾}{yi2han4}
  \antonymref{庆幸}{qing4xing4}
  \end{Phonetics}
\end{Entry}

%%%%%%%%%% 惦 %%%%%%%%%%
\subsection*{惦}\addcontentsline{loh}{figure}{惦}

\begin{Entry}{惦}{11}{⼼}
  \begin{Phonetics}{惦}{dian4}
    \definition{v.}{lembrar com preocupação; estar preocupado com; continuar pensando sobre; ficar pensando em alguém ou em alguma coisa e se preocupar com eles; sentir falta deles}
  \end{Phonetics}
\end{Entry}

\begin{Entry}{惦记}{11,5}{⼼,⾔}
  \begin{Phonetics}{惦记}{dian4ji4}[][HSK 7-9]
    \definition{s.}{lembrar com preocupação; estar preocupado com; continuar pensando sobre; (sobre uma pessoa ou coisa) continuar pensando nisso e não deixar passar}
  \end{Phonetics}
\end{Entry}

%%%%%%%%%% 惨 %%%%%%%%%%
\subsection*{惨}\addcontentsline{loh}{figure}{惨}

\begin{Entry}{惨}{11}{⽕}
  \begin{Phonetics}{惨}{can3}[][HSK 6]
    \definition{adj.}{miserável; trágico | cruel; brutal; implacável | desastroso; terrível; esmagador | lamentável; desaventurado | em um grau sério; grau grave; dano grave | selvagem; desumano; vicioso; cruel}
  \end{Phonetics}
\end{Entry}

\begin{Entry}{惨白}{11,5}{⽕,⽩}
  \begin{Phonetics}{惨白}{can3bai2}[][HSK 7-9]
    \definition{adj.}{(cenário) escuro; fraco; sombrio; sombrio | (rosto) mortalmente pálido; medonho | terrivelmente (fantasmagórico) pálido; pálido; escuro}
  \end{Phonetics}
\end{Entry}

\begin{Entry}{惨重}{11,9}{⽕,⾥}
  \begin{Phonetics}{惨重}{can3zhong4}[][HSK 7-9]
    \definition{adj.}{pesado; doloroso; desastroso | calamitoso (perdas extremamente graves)}
  \end{Phonetics}
\end{Entry}

\begin{Entry}{惨痛}{11,12}{⽕,⽧}
  \begin{Phonetics}{惨痛}{can3tong4}[][HSK 7-9]
    \definition{adj.}{grave; terrivelmente doloroso; amargo; agonizante | profundamente triste; doloroso}
  \end{Phonetics}
\end{Entry}

%%%%%%%%%% 惭 %%%%%%%%%%
\subsection*{惭}\addcontentsline{loh}{figure}{惭}

\begin{Entry}{惭}{11}{⼼}
  \begin{Phonetics}{惭}{can2}
    \definition{adj.}{envergonhado}
    \definition{s.}{vergonha}
    \definition{v.}{sentir vergonha}
  \end{Phonetics}
\end{Entry}

\begin{Entry}{惭愧}{11,12}{⼼,⼼}
  \begin{Phonetics}{惭愧}{can2kui4}[][HSK 7-9]
    \definition{adj.}{envergonhado; sentir"-se inseguro por ter deficiências, fazer algo errado ou não cumprir responsabilidades}
  \end{Phonetics}
\end{Entry}

%%%%%%%%%% 惯 %%%%%%%%%%
\subsection*{惯}\addcontentsline{loh}{figure}{惯}

\begin{Entry}{惯}{11}{⼼}
  \begin{Phonetics}{惯}{guan4}[][HSK 7-9]
    \definition{adj.}{habitual; costumeiro; usual | incorrigível; endurecido}
    \definition{v.}{estar acostumado a; ter o hábito de | mimar; estragar}
  \end{Phonetics}
\end{Entry}

\begin{Entry}{惯例}{11,8}{⼼,⼈}
  \begin{Phonetics}{惯例}{guan4li4}[][HSK 7-9]
    \definition[个]{s.}{rotina; convenção; prática usual; prática habitual | precedente; embora não haja nenhuma disposição explícita na lei, há práticas que foram implementadas no passado e podem ser imitadas}
  \end{Phonetics}
\end{Entry}

\begin{Entry}{惯性}{11,8}{⼼,⼼}
  \begin{Phonetics}{惯性}{guan4xing4}[][HSK 7-9]
    \definition{s.}{Física: inércia; a força da inércia}
  \end{Phonetics}
\end{Entry}

%%%%%%%%%% 捧 %%%%%%%%%%
\subsection*{捧}\addcontentsline{loh}{figure}{捧}

\begin{Entry}{捧}{11}{⼿}
  \begin{Phonetics}{捧}{peng3}[][HSK 7-9]
    \definition{clas.}{utilizado para coisas que podem ser seguradas}
    \definition{v.}{segurar ou carregar com ambas as mãos; apoiar com ambas as mãos | impulsionar; elogiar; exaltar; lisonjear; vangloriar}
  \end{Phonetics}
\end{Entry}

\begin{Entry}{捧场}{11,6}{⼿,⼟}
  \begin{Phonetics}{捧场}{peng3/chang3}[][HSK 7-9]
    \definition{v.+compl.}{aplaudir; comparecer e apoiar (uma reunião, apresentação, etc.) | promover; elogiar; bajular | ser membro de uma claque | ser membro de um grupo de apoio; elogiar profusamente; prestar homenagem pública a alguém; promover alguém no programa; gabar"-se para os outros}
  \end{Phonetics}
\end{Entry}

%%%%%%%%%% 据 %%%%%%%%%%
\subsection*{据}\addcontentsline{loh}{figure}{据}

\begin{Entry}{据}{11}{⼿}
  \begin{Phonetics}{据}{ju1}
    \definition{part.}{elemento formador de palavras}
  \seealsoref{拮据}{jie2ju1}
  \end{Phonetics}
  \begin{Phonetics}{据}{ju4}[][HSK 6]
    \definition*{s.}{Sobrenome: Ju}
    \definition{prep.}{de acordo com; com base em}
    \definition{s.}{evidência; certificado; prova}
    \definition{v.}{ocupar; apreender | confiar em; depender de}
  \end{Phonetics}
\end{Entry}

\begin{Entry}{据此}{11,6}{⼿,⽌}
  \begin{Phonetics}{据此}{ju4ci3}[][HSK 7-9]
    \definition{v.}{se basear nesses fundamentos; ter em vista o exposto acima; fazer algo em conformidade; se basear nas circunstâncias ou razões já mencionadas}
  \end{Phonetics}
\end{Entry}

\begin{Entry}{据说}{11,9}{⼿,⾔}
  \begin{Phonetics}{据说}{ju4shuo1}[][HSK 3]
    \definition{v.}{é o que dizem; é o que se diz}
  \end{Phonetics}
\end{Entry}

\begin{Entry}{据悉}{11,11}{⼿,⼼}
  \begin{Phonetics}{据悉}{ju4xi1}[][HSK 7-9]
    \definition{adv.}{é relatado (que); de acordo com o que aprendi}
  \end{Phonetics}
\end{Entry}

%%%%%%%%%% 捶 %%%%%%%%%%
\subsection*{捶}\addcontentsline{loh}{figure}{捶}

\begin{Entry}{捶}{11}{⼿}
  \begin{Phonetics}{捶}{chui2}[][HSK 7-9]
    \definition{v.}{bater (com um pedaço de pau, martelo ou punho)}
  \end{Phonetics}
\end{Entry}

\begin{Entry}{捶子}{11,3}{⼿,⼦}
  \begin{Phonetics}{捶子}{chui2zi5}
    \definition[把]{s.}{martelo}
  \end{Phonetics}
\end{Entry}

%%%%%%%%%% 捷 %%%%%%%%%%
\subsection*{捷}\addcontentsline{loh}{figure}{捷}

\begin{Entry}{捷}{11}{⼿}
  \begin{Phonetics}{捷}{jie2}
    \definition*{s.}{Sobrenome: Jie}
    \definition{adj.}{rápido; ágil}
    \definition{s.}{vitória; triunfo; sucesso}
  \end{Phonetics}
\end{Entry}

\begin{Entry}{捷径}{11,8}{⼿,⼻}
  \begin{Phonetics}{捷径}{jie2jing4}
    \definition{s.}{atalho}
  \end{Phonetics}
\end{Entry}

%%%%%%%%%% 授 %%%%%%%%%%
\subsection*{授}\addcontentsline{loh}{figure}{授}

\begin{Entry}{授}{11}{⼿}
  \begin{Phonetics}{授}{shou4}
    \definition*{s.}{Sobrenome: Shou}
    \definition{v.}{premiar; conferir; conceder; dar; entregar | ensinar; instruir; transmitir; ensinar}
  \synonymref{教}{jiao1}
  \synonymref{教}{jiao4}
  \antonymref{受}{shou4}
  \end{Phonetics}
\end{Entry}

\begin{Entry}{授予}{11,4}{⼿,⼅}
  \begin{Phonetics}{授予}{shou4yu3}[][HSK 7-9]
    \definition{v.}{conferir; premiar; conceder; atribuir (medalhas, graus, títulos, etc.)}
  \synonymref{赋予}{fu4yu3}
  \synonymref{授权}{shou4quan2}
  \antonymref{剥夺}{bo1duo2}
  \end{Phonetics}
\end{Entry}

\begin{Entry}{授权}{11,6}{⼿,⽊}
  \begin{Phonetics}{授权}{shou4quan2}[][HSK 7-9]
    \definition{v.}{capacitar; autorizar; garantir; delegar poder a indivíduos ou organizações para execução}
  \synonymref{授予}{shou4yu3}
  \synonymref{委托}{wei3tuo1}
  \end{Phonetics}
\end{Entry}

%%%%%%%%%% 掉 %%%%%%%%%%
\subsection*{掉}\addcontentsline{loh}{figure}{掉}

\begin{Entry}{掉}{11}{⼿}
  \begin{Phonetics}{掉}{diao4}[][HSK 2]
    \definition{v.}{cair; soltar-se; desprender-se | ficar para trás | perder; desaparecer; omitir | diminuir; reduzir | balançar; abanar; oscilar | virar; voltar; retornar | alterar; trocar; intercambiar}
    \definition{v.aux.}{usado após certos verbos para indicar a conclusão de uma ação}
  \end{Phonetics}
\end{Entry}

\begin{Entry}{掉队}{11,4}{⼿,⾩}
  \begin{Phonetics}{掉队}{diao4/dui4}[][HSK 7-9]
    \definition{v.+compl.}{abandonar (ou sair); ficar para trás; cair fora}
  \end{Phonetics}
\end{Entry}

\begin{Entry}{掉包}{11,5}{⼿,⼓}
  \begin{Phonetics}{掉包}{diao4bao1}
    \definition{v.}{vender uma falsificação pelo artigo genuíno | roubar o item valioso de alguém e substituí-lo por um item de aparência semelhante, mas sem valor}
  \end{Phonetics}
\end{Entry}

\begin{Entry}{掉头}{11,5}{⼿,⼤}
  \begin{Phonetics}{掉头}{diao4/tou2}[][HSK 7-9]
    \definition{v.+compl.}{virar"-se; afastar"-se (das pessoas) | dar meia"-volta (carro, barco, etc.); (carro, barco, etc.) virar na direção oposta}
  \end{Phonetics}
\end{Entry}

\begin{Entry}{掉线}{11,8}{⼿,⽷}
  \begin{Phonetics}{掉线}{diao4xian4}
    \definition{v.}{desconectar-se (da \emph{Internet})}
  \end{Phonetics}
\end{Entry}

\begin{Entry}{掉转}{11,8}{⼿,⾞}
  \begin{Phonetics}{掉转}{diao4zhuan3}
    \definition{v.}{dar a volta}
  \end{Phonetics}
\end{Entry}

\begin{Entry}{掉落}{11,12}{⼿,⾋}
  \begin{Phonetics}{掉落}{diao4luo4}
    \definition{v.}{derrubar}
  \end{Phonetics}
\end{Entry}

\begin{Entry}{掉膘}{11,15}{⼿,⾁}
  \begin{Phonetics}{掉膘}{diao4biao1}
    \definition{v.}{perder peso (gado)}
  \end{Phonetics}
\end{Entry}

%%%%%%%%%% 掏 %%%%%%%%%%
\subsection*{掏}\addcontentsline{loh}{figure}{掏}

\begin{Entry}{掏}{11}{⼿}
  \begin{Phonetics}{掏}{tao1}[][HSK 6]
    \definition{v.}{extrair; retirar; pescar | cavar (um buraco, etc.); escavar; retirar | (coloquial) roubar do bolso de alguém | tirar}
  \end{Phonetics}
\end{Entry}

\begin{Entry}{掏钱}{11,10}{⼿,⾦}
  \begin{Phonetics}{掏钱}{tao1 qian2}[][HSK 7-9]
    \definition{v.}{pagar; pagar dinheiro; gastar dinheiro}
  \end{Phonetics}
\end{Entry}

%%%%%%%%%% 掐 %%%%%%%%%%
\subsection*{掐}\addcontentsline{loh}{figure}{掐}

\begin{Entry}{掐}{11}{⼿}
  \begin{Phonetics}{掐}{qia1}[][HSK 7-9]
    \definition{s.}{Dialeto: um punhado, maço, pitada, etc. de}
    \definition{v.}{beliscar; dar uma mordidinha | agarrar}
  \seealsoref{掐儿}{qia1r5}
  \end{Phonetics}
\end{Entry}

\begin{Entry}{掐儿}{11,2}{⼿,⼉}
  \begin{Phonetics}{掐儿}{qia1r5}
    \definition{s.}{Dialeto: um punhado, maço, pitada, etc. de}
  \end{Phonetics}
\end{Entry}

%%%%%%%%%% 排 %%%%%%%%%%
\subsection*{排}\addcontentsline{loh}{figure}{排}

\begin{Entry}{排}{11}{⼿}
  \begin{Phonetics}{排}{pai2}[][HSK 2,3]
    \definition{clas.}{usado para linhas, filas; coisas usadas para formar filas}
    \definition{s.}{linha; fileira; fileiras horizontais | pelotão; unidade militar, abaixo do nível de companhia, acima do nível de pelotão | jangada; balsa; um meio de transporte aquático feito de bambu e madeira unidos lado a lado; também se refere a bambu e madeira amarrados em fileiras para facilitar o transporte aquático | torta; bolo de carne; bolinho assado; comida cozida no vapor}
    \definition{v.}{organizar; alinhar; colocar em ordem; posicionar ou organizar em uma determinada ordem; ordenar | ensaiar | ejetar; excluir; dispensar; remover; eliminar | empurrar o obstáculo para fora do caminho}
  \end{Phonetics}
\end{Entry}

\begin{Entry}{排水}{11,4}{⼿,⽔}
  \begin{Phonetics}{排水}{pai2shui3}
    \definition{v.}{drenar}
  \end{Phonetics}
\end{Entry}

\begin{Entry}{排队}{11,4}{⼿,⾩}
  \begin{Phonetics}{排队}{pai2/dui4}[][HSK 2]
    \definition{v.+compl.}{formar uma fila; alinhar"-se; enfileirar"-se; organizar em sequência | listar; classificar}
  \end{Phonetics}
\end{Entry}

\begin{Entry}{排斥}{11,5}{⼿,⽄}
  \begin{Phonetics}{排斥}{pai2chi4}[][HSK 7-9]
    \definition{v.}{repelir; rejeitar; excluir; fazer com que (uma pessoa ou coisa) se afaste do seu próprio grupo}
  \end{Phonetics}
\end{Entry}

\begin{Entry}{排列}{11,6}{⼿,⼑}
  \begin{Phonetics}{排列}{pai2lie4}[][HSK 4]
    \definition{v.}{classificar; colocar; variar; organizar; pôr em ordem}
  \end{Phonetics}
\end{Entry}

\begin{Entry}{排名}{11,6}{⼿,⼝}
  \begin{Phonetics}{排名}{pai2 ming2}[][HSK 3]
    \definition{s.}{classificação; resultado; organizado de acordo com determinados critérios}
  \end{Phonetics}
\end{Entry}

\begin{Entry}{排行榜}{11,6,14}{⼿,⾏,⽊}
  \begin{Phonetics}{排行榜}{pai2hang2bang3}[][HSK 6]
    \definition{s.}{lista; classificação; lista de classificação; (de registros) os gráficos; uma lista em uma determinada ordem publicada com base em certos resultados estatísticos}
  \end{Phonetics}
\end{Entry}

\begin{Entry}{排放}{11,8}{⼿,⽅}
  \begin{Phonetics}{排放}{pai2fang4}[][HSK 7-9]
    \definition{v.}{colocar (as coisas) em ordem adequada | emitir; descarregar (gases de escape, águas residuais, etc.); deixar sair; drenar}
  \end{Phonetics}
\end{Entry}

\begin{Entry}{排练}{11,8}{⼿,⽷}
  \begin{Phonetics}{排练}{pai2lian4}[][HSK 7-9]
    \definition{v.}{ensaiar; ensaiar ou praticar uma determinada cerimônia ou apresentação}
  \end{Phonetics}
\end{Entry}

\begin{Entry}{排挤}{11,9}{⼿,⼿}
  \begin{Phonetics}{排挤}{pai2ji3}
    \definition{v.}{ostracizar; afastar; expulsar; espremer; excluir; marginalizar; usar o poder ou os meios para fazer com que aqueles que lhe são desfavoráveis percam seu status ou seus interesses}
  \end{Phonetics}
\end{Entry}

\begin{Entry}{排除}{11,9}{⼿,⾩}
  \begin{Phonetics}{排除}{pai2chu2}[][HSK 5]
    \definition{v.}{remover; superar; excluir; eliminar; livrar"-se de}
  \end{Phonetics}
\end{Entry}

\begin{Entry}{排球}{11,11}{⼿,⽟}
  \begin{Phonetics}{排球}{pai2qiu2}[][HSK 2]
    \definition[场,只,个]{s.}{voleibol; bola de voleibol}
  \end{Phonetics}
\end{Entry}

%%%%%%%%%% 掠 %%%%%%%%%%
\subsection*{掠}\addcontentsline{loh}{figure}{掠}

\begin{Entry}{掠}{11}{⼿}
  \begin{Phonetics}{掠}{lve3}
    \definition{v.}{agarrar; tomar | copiar casualmente; copiar}
  \end{Phonetics}
  \begin{Phonetics}{掠}{lve4}
    \definition{v.}{pilhar; saquear; roubar | passar por cima; roçar; raspar; deslizar sobre; passar rapidamente; limpar ou escovar suavemente | Literário: bater ou açoitar (com um bastão ou chicote)}
  \end{Phonetics}
\end{Entry}

\begin{Entry}{掠夺}{11,6}{⼿,⼤}
  \begin{Phonetics}{掠夺}{lve4duo2}[][HSK 7-9]
    \definition{v.}{roubar; pilhar; saquear; usar força ou violência para roubar coisas}
  \end{Phonetics}
\end{Entry}

%%%%%%%%%% 探 %%%%%%%%%%
\subsection*{探}\addcontentsline{loh}{figure}{探}

\begin{Entry}{探}{11}{⼿}
  \begin{Phonetics}{探}{tan4}[][HSK 7-9]
    \definition[个,位,名]{s.}{batedor; espião; detetive}
    \definition{v.}{tentar descobrir; explorar; soar | explorar; espionar | visitar; fazer uma visita em | se destacar | preocupar"-se com; envolver"-se em | ver; invocar}
  \end{Phonetics}
\end{Entry}

\begin{Entry}{探讨}{11,5}{⼿,⾔}
  \begin{Phonetics}{探讨}{tan4tao3}[][HSK 6]
    \definition{v.}{examinar; indagar; investigar; discutir}
  \end{Phonetics}
\end{Entry}

\begin{Entry}{探求}{11,7}{⼿,⽔}
  \begin{Phonetics}{探求}{tan4qiu2}[][HSK 7-9]
    \definition{v.}{procurar; perseguir; buscar; explorar e buscar}
  \synonymref{考虑}{kao3lv4}
  \synonymref{商量}{shang1liang5}
  \synonymref{搜索}{sou1suo3}
  \synonymref{探索}{tan4suo3}
  \synonymref{推测}{tui1ce4}
  \synonymref{寻求}{xun2qiu2}
  \synonymref{寻找}{xun2zhao3}
  \synonymref{研究}{yan2jiu1}
  \synonymref{追求}{zhui1qiu2}
  \end{Phonetics}
\end{Entry}

\begin{Entry}{探亲}{11,9}{⼿,⼇}
  \begin{Phonetics}{探亲}{tan4/qin1}[][HSK 7-9]
    \definition{v.+compl.}{ir para casa para visitar a família; visitar parentes em outra cidade (geralmente referindo"-se à visita aos pais ou cônjuge)}
  \end{Phonetics}
\end{Entry}

\begin{Entry}{探测}{11,9}{⼿,⽔}
  \begin{Phonetics}{探测}{tan4ce4}[][HSK 7-9]
    \definition{v.}{sondar; examinar; explorar; utilizar ferramentas para observar ou medir coisas que não podem ser observadas ou medidas diretamente}
  \end{Phonetics}
\end{Entry}

\begin{Entry}{探险}{11,9}{⼿,⾩}
  \begin{Phonetics}{探险}{tan4/xian3}[][HSK 7-9]
    \definition{v.+compl.}{explorar; fazer explorações; aventurar"-se no desconhecido; investigar lugares onde ninguém jamais esteve ou onde pouquíssimas pessoas estiveram (o mundo natural)}
  \synonymref{冒险}{mao4/xian3}
  \synonymref{探求}{tan4qiu2}
  \end{Phonetics}
\end{Entry}

\begin{Entry}{探索}{11,10}{⼿,⽷}
  \begin{Phonetics}{探索}{tan4suo3}[][HSK 6]
    \definition{v.}{sondar; explorar; procurar respostas de várias fontes para resolver dúvidas}
  \end{Phonetics}
\end{Entry}

\begin{Entry}{探望}{11,11}{⼿,⽉}
  \begin{Phonetics}{探望}{tan4wang4}[][HSK 7-9]
    \definition{v.}{olhar ao redor; observar (tentar descobrir o que está acontecendo) | ver; visitar; fazer uma visita a alguém (geralmente de longe)}
  \synonymref{拜访}{bai4fang3}
  \synonymref{访问}{fang3wen4}
  \synonymref{看望}{kan4wang5}
  \antonymref{回避}{hui2bi4}
  \end{Phonetics}
\end{Entry}

%%%%%%%%%% 接 %%%%%%%%%%
\subsection*{接}\addcontentsline{loh}{figure}{接}

\begin{Entry}{接}{11}{⼿}
  \begin{Phonetics}{接}{jie1}[][HSK 2]
    \definition*{s.}{Sobrenome: Jie}
    \definition{v.}{entrar em contato com; aproximar"-se de | conectar; unir; juntar | continuar; prosseguir | assumir o controle; assumir o trabalho de outra pessoa e continuar a fazê"-lo | pegar; agarrar; segurar ou sustentar com as mãos | receber; aceitar | encontrar; dar as boas"-vindas}
  \end{Phonetics}
\end{Entry}

\begin{Entry}{接二连三}{11,2,7,3}{⼿,⼆,⾡,⼀}
  \begin{Phonetics}{接二连三}{jie1'er4-lian2san1}[][HSK 7-9]
    \definition{expr.}{um após o outro; em rápida sucessão}
  \end{Phonetics}
\end{Entry}

\begin{Entry}{接力}{11,2}{⼿,⼒}
  \begin{Phonetics}{接力}{jie1li4}[][HSK 7-9]
    \definition{s.}{relé; trabalho por revezamento; revezamento}
  \end{Phonetics}
\end{Entry}

\begin{Entry}{接下来}{11,3,7}{⼿,⼀,⽊}
  \begin{Phonetics}{接下来}{jie1xia4lai2}[][HSK 2]
    \definition{expr.}{próximo; seguinte; indica uma sequência temporal subsequente}
  \end{Phonetics}
\end{Entry}

\begin{Entry}{接手}{11,4}{⼿,⼿}
  \begin{Phonetics}{接手}{jie1shou3}[][HSK 7-9]
    \definition{v.}{assumir (responsabilidades, etc.); assumir problemas; assumir o trabalho de outra pessoa.}
  \end{Phonetics}
\end{Entry}

\begin{Entry}{接见}{11,4}{⼿,⾒}
  \begin{Phonetics}{接见}{jie1jian4}[][HSK 7-9]
    \definition{v.}{receber alguém; conceder uma entrevista a; reunir"-se com as pessoas que vieram}
  \end{Phonetics}
\end{Entry}

\begin{Entry}{接(电话)}{11,5,8}{⼿,⽥,⾔}
  \begin{Phonetics}{接(电话)}{jie1(dian4hua4)}
    \definition{v.}{atender (o telefone) | receber (uma ligação telefônica)}
  \end{Phonetics}
\end{Entry}

\begin{Entry}{接收}{11,6}{⼿,⽁}
  \begin{Phonetics}{接收}{jie1shou1}[][HSK 6]
    \definition{v.}{aceitar; receber | assumir; expropriar; tomar posse (de uma instituição, propriedade, etc.) de acordo com a lei | admitir; aceitar; absorver}
  \end{Phonetics}
\end{Entry}

\begin{Entry}{接轨}{11,6}{⼿,⾞}
  \begin{Phonetics}{接轨}{jie1/gui3}[][HSK 7-9]
    \definition{s.}{junção; integração; ligação}
    \definition{v.+compl.}{ligar; juntar; conectar os trilhos | integrar; juntar"-se a; mudar para; entrar na onda; alinhar"-se; alinhar a; essa metáfora descreve como sistemas e métodos podem ser interconectados e consistentes}
  \end{Phonetics}
\end{Entry}

\begin{Entry}{接听}{11,7}{⼿,⼝}
  \begin{Phonetics}{接听}{jie1ting1}[][HSK 7-9]
    \definition{v.}{atender (o telefone)}
  \end{Phonetics}
\end{Entry}

\begin{Entry}{接纳}{11,7}{⼿,⽷}
  \begin{Phonetics}{接纳}{jie1na4}[][HSK 7-9]
    \definition{v.}{ser admitido (em uma organização); aceitar (como membro); incluir (indivíduos ou grupos que ingressam na organização) | adotar; aceitar; tomar}
  \end{Phonetics}
\end{Entry}

\begin{Entry}{接近}{11,7}{⼿,⾡}
  \begin{Phonetics}{接近}{jie1jin4}[][HSK 3]
    \definition{adj.}{perto; próximo; a diferença entre os dois é mínima}
    \definition{v.}{estar perto de; aproximar; aproximar"-se}
  \end{Phonetics}
\end{Entry}

\begin{Entry}{接连}{11,7}{⼿,⾡}
  \begin{Phonetics}{接连}{jie1lian2}[][HSK 5]
    \definition{adv.}{no final; em sucessão; em uma fileira; um após o outro; seguindo o anterior; continuando}
  \end{Phonetics}
\end{Entry}

\begin{Entry}{接到}{11,8}{⼿,⼑}
  \begin{Phonetics}{接到}{jie1dao4}[][HSK 2]
    \definition{v.}{receber (carta, etc.)}
  \end{Phonetics}
\end{Entry}

\begin{Entry}{接受}{11,8}{⼿,⼜}
  \begin{Phonetics}{接受}{jie1shou4}[][HSK 2]
    \definition{v.}{aceitar; não recusar (o que os outros oferecem) | concordar; não recusar (opiniões/sugestões/críticas/convites de outras pessoas, etc.)}
  \end{Phonetics}
\end{Entry}

\begin{Entry}{接待}{11,9}{⼿,⼻}
  \begin{Phonetics}{接待}{jie1dai4}[][HSK 3]
    \definition{v.}{receber (alguém); acolher; recepcionar; receber com cordialidade e generosidade}
  \end{Phonetics}
\end{Entry}

\begin{Entry}{接济}{11,9}{⼿,⽔}
  \begin{Phonetics}{接济}{jie1ji4}[][HSK 7-9]
    \definition{v.}{prestar assistência material a; dar ajuda financeira a; prestar auxílio material a}
  \end{Phonetics}
\end{Entry}

\begin{Entry}{接送}{11,9}{⼿,⾡}
  \begin{Phonetics}{接送}{jie1song4}[][HSK 7-9]
    \definition{v.}{buscar e levar}
  \end{Phonetics}
\end{Entry}

\begin{Entry}{接班}{11,10}{⼿,⽟}
  \begin{Phonetics}{接班}{jie1/ban1}[][HSK 7-9]
    \definition{v.}{assumir o turno de alguém; substituir alguém; assumir o lugar de; dar continuidade a; (sucessor) Assumir o trabalho do turno anterior | ter sucesso; dar continuidade a algo iniciado por seu antecessor}
  \seealsoref{接班儿}{jie1ban1r5}
  \end{Phonetics}
\end{Entry}

\begin{Entry}{接班人}{11,10,2}{⼿,⽟,⼈}
  \begin{Phonetics}{接班人}{jie1ban1ren2}[][HSK 7-9]
    \definition{s.}{sucessor; a pessoa que assume o trabalho do turno anterior é frequentemente usada metaforicamente}
  \end{Phonetics}
\end{Entry}

\begin{Entry}{接班儿}{11,10,2}{⼿,⽟,⼉}
  \begin{Phonetics}{接班儿}{jie1ban1r5}
    \definition{v.}{assumir o turno de alguém; substituir alguém | ter sucesso; dar continuidade a algo iniciado por seu antecessor}
  \seealsoref{接班}{jie1/ban1}
  \end{Phonetics}
\end{Entry}

\begin{Entry}{接通}{11,10}{⼿,⾡}
  \begin{Phonetics}{接通}{jie1tong1}[][HSK 7-9]
    \definition{v.}{transmitir; fazer ligação telefônica | conectar; completar a ligação; conseguir passar | fechar; encerrar; interromper; inserir; ativar; ligar; completar}
  \end{Phonetics}
\end{Entry}

\begin{Entry}{接着}{11,11}{⼿,⽬}
  \begin{Phonetics}{接着}{jie1zhe5}[][HSK 2]
    \definition{adv.}{por sua vez; um após o outro; sucessivamente; conectado (à frase anterior); imediatamente após (a ação anterior)}
    \definition{v.}{seguir; prosseguir; continuar; seguir em frente; ficar ao lado | pegar com as mãos; apanhar}
  \end{Phonetics}
\end{Entry}

\begin{Entry}{接替}{11,12}{⼿,⽈}
  \begin{Phonetics}{接替}{jie1ti4}[][HSK 7-9]
    \definition{v.}{assumir o controle; substituir | suceder; ocupar o lugar de}
  \end{Phonetics}
\end{Entry}

\begin{Entry}{接触}{11,13}{⼿,⾓}
  \begin{Phonetics}{接触}{jie1chu4}[][HSK 5]
    \definition{v.}{entrar em contato com | entrar em contato; tocar; interagir | engajar; o termo militar refere"-se a fogo cruzado}
  \end{Phonetics}
\end{Entry}

%%%%%%%%%% 控 %%%%%%%%%%
\subsection*{控}\addcontentsline{loh}{figure}{控}

\begin{Entry}{控}{11}{⼿}
  \begin{Phonetics}{控}{kong4}
    \definition{v.}{acusar; cobrar | controlar; dominar | manter (parte do corpo em uma determinada posição) sem apoio | virar (um recipiente) de cabeça para baixo para deixar o líquido escorrer}
  \end{Phonetics}
\end{Entry}

\begin{Entry}{控告}{11,7}{⼿,⼝}
  \begin{Phonetics}{控告}{kong4gao4}[][HSK 7-9]
    \definition{v.}{acusar; denunciar; incriminar; indiciar; processar alguém; apresentar uma queixa legal contra alguém}
  \end{Phonetics}
\end{Entry}

\begin{Entry}{控制}{11,8}{⼿,⼑}
  \begin{Phonetics}{控制}{kong4zhi4}[][HSK 5]
    \definition{v.}{controlar; restringir; dominar; fazer com que não ultrapasse um determinado limite | controlar; dominar; comandar; ocupar, fazer com que não se perca}
  \end{Phonetics}
\end{Entry}

%%%%%%%%%% 推 %%%%%%%%%%
\subsection*{推}\addcontentsline{loh}{figure}{推}

\begin{Entry}{推}{11}{⼿}
  \begin{Phonetics}{推}{tui1}[][HSK 2]
    \definition{v.}{empurrar; dar um encontrão | girar um moinho ou uma pedra de amolar; moer | cortar; aparar | impulsionar; promover; avançar | inferir; deduzir | afastar; fugir; deslocar | adiar | eleger; escolher | ter em alta estima; elogiar muito | declinar | selecionar | elogiar muito}
  \end{Phonetics}
\end{Entry}

\begin{Entry}{推广}{11,3}{⼿,⼴}
  \begin{Phonetics}{推广}{tui1guang3}[][HSK 3]
    \definition{v.}{espalhar; estender; promover; popularizar; expandir o escopo de uso ou função de algo}
  \end{Phonetics}
\end{Entry}

\begin{Entry}{推介}{11,4}{⼿,⼈}
  \begin{Phonetics}{推介}{tui1jie4}
    \definition{s.}{promoção}
    \definition{v.}{promover | introduzir e recomendar}
  \end{Phonetics}
\end{Entry}

\begin{Entry}{推开}{11,4}{⼿,⼶}
  \begin{Phonetics}{推开}{tui1kai1}[][HSK 3]
    \definition{v.}{declinar; rejeitar | empurrar para longe; aplicar força em uma determinada direção para mover uma pessoa ou objeto para longe de seu lugar original | empurrar para abrir (um portão, etc.); empurrar para fora para abrir algo que está fechado | estender; popularizar; promover para um alcance mais amplo e realizar em uma escala mais ampla}
  \end{Phonetics}
\end{Entry}

\begin{Entry}{推出}{11,5}{⼿,⼐}
  \begin{Phonetics}{推出}{tui1chu1}[][HSK 6]
    \definition{v.}{lançar; apresentar; fazer com que apareça diante do público | deduzir; tirar conclusões da análise}
  \end{Phonetics}
\end{Entry}

\begin{Entry}{推动}{11,6}{⼿,⼒}
  \begin{Phonetics}{推动}{tui1 dong4}[][HSK 3]
    \definition{v.}{promover; atuar; impulsionar; empurrar para a frente; dar ímpeto a; começar ou avançar algo (com alguma força); começar a trabalhar}
  \end{Phonetics}
\end{Entry}

\begin{Entry}{推行}{11,6}{⼿,⾏}
  \begin{Phonetics}{推行}{tui1xing2}[][HSK 5]
    \definition{v.}{realizar; prosseguir; praticar | implementar; praticar; implementação generalizada; divulgar (experiências, métodos, etc.)}
  \end{Phonetics}
\end{Entry}

\begin{Entry}{推进}{11,7}{⼿,⾡}
  \begin{Phonetics}{推进}{tui1jin4}[][HSK 3]
    \definition{v.}{avançar; empurrar; levar adiante; dar ímpeto a; promover o trabalho e fazê-lo avançar | empurrar; dirigir; avançar; seguir em frente; seguir em frente}
  \end{Phonetics}
\end{Entry}

\begin{Entry}{推迟}{11,7}{⼿,⾡}
  \begin{Phonetics}{推迟}{tui1chi2}[][HSK 4]
    \definition{v.}{adiar; postergar; tardar; deixar para mais tarde}
  \end{Phonetics}
\end{Entry}

\begin{Entry}{推卸}{11,9}{⼿,⼙}
  \begin{Phonetics}{推卸}{tui1xie4}[][HSK 7-9]
    \definition{v.}{esquivar"-se (da responsabilidade); recusar"-se a assumir (responsabilidades e obrigações, etc.)}
  \synonymref{推辞}{tui1ci2}
  \synonymref{退却}{tui4que4}
  \antonymref{承担}{cheng2dan1}
  \antonymref{承受}{cheng2shou4}
  \antonymref{担负}{dan1fu4}
  \end{Phonetics}
\end{Entry}

\begin{Entry}{推测}{11,9}{⼿,⽔}
  \begin{Phonetics}{推测}{tui1ce4}[][HSK 7-9]
    \definition{v.}{inferir; supor; conjecturar; especular; estimar ou imaginar o desconhecido com base no conhecido}
  \synonymref{猜测}{cai1ce4}
  \synonymref{猜想}{cai1xiang3}
  \synonymref{揣测}{chuai3ce4}
  \synonymref{揣摩}{chuai3mo2}
  \synonymref{估计}{gu1ji4}
  \synonymref{探求}{tan4qiu2}
  \synonymref{推断}{tui1duan4}
  \antonymref{断定}{duan4ding4}
  \end{Phonetics}
\end{Entry}

\begin{Entry}{推荐}{11,9}{⼿,⾋}
  \begin{Phonetics}{推荐}{tui1jian4}[][HSK 7-9]
    \definition[份]{s.}{recomendação}
    \definition{v.}{recomendar; apresentar pessoas ou coisas boas a pessoas ou organizações, na esperança de empregá-las ou aceitá-las}
  \seealsoref{介绍}{jie4shao4}
  \synonymref{推出}{tui1chu1}
  \synonymref{推选}{tui1xuan3}
  \end{Phonetics}
\end{Entry}

\begin{Entry}{推选}{11,9}{⼿,⾡}
  \begin{Phonetics}{推选}{tui1xuan3}[][HSK 7-9]
    \definition{v.}{eleger; escolher; nomear}
  \seealsoref{选}{xuan3}
  \synonymref{竞选}{jing4xuan3}
  \synonymref{推荐}{tui1jian4}
  \synonymref{选举}{xuan3ju3}
  \antonymref{指定}{zhi3ding4}
  \end{Phonetics}
\end{Entry}

\begin{Entry}{推断}{11,11}{⼿,⽄}
  \begin{Phonetics}{推断}{tui1duan4}[][HSK 7-9]
    \definition{v.}{inferir; deduzir; especular e concluir}
  \synonymref{猜测}{cai1ce4}
  \synonymref{猜想}{cai1xiang3}
  \synonymref{揣测}{chuai3ce4}
  \synonymref{估计}{gu1ji4}
  \synonymref{判断}{pan4duan4}
  \synonymref{推测}{tui1ce4}
  \synonymref{推理}{tui1li3}
  \antonymref{断定}{duan4ding4}
  \end{Phonetics}
\end{Entry}

\begin{Entry}{推理}{11,11}{⼿,⽟}
  \begin{Phonetics}{推理}{tui1li3}[][HSK 7-9]
    \definition{s.}{inferência; raciocínio; o processo de tirar novas conclusões com base em informações existentes}
  \synonymref{推断}{tui1duan4}
  \end{Phonetics}
\end{Entry}

\begin{Entry}{推移}{11,11}{⼿,⽲}
  \begin{Phonetics}{推移}{tui1yi2}[][HSK 7-9]
    \definition{v.}{passar; decorrer; movimento, mudança ou desenvolvimento}
  \synonymref{发展}{fa1zhan3}
  \synonymref{推动}{tui1 dong4}
  \synonymref{推进}{tui1jin4}
  \synonymref{移动}{yi2dong4}
  \end{Phonetics}
\end{Entry}

\begin{Entry}{推销}{11,12}{⼿,⾦}
  \begin{Phonetics}{推销}{tui1xiao1}[][HSK 4]
    \definition{v.}{vender; comercializar; promover vendas; promover a comercialização de mercadorias}
  \end{Phonetics}
\end{Entry}

\begin{Entry}{推辞}{11,13}{⼿,⾟}
  \begin{Phonetics}{推辞}{tui1ci2}[][HSK 7-9]
    \definition{s.}{indicar recusa (de compromissos, convites, presentes, etc.); recusar (pedidos, opiniões ou presentes)}
  \synonymref{不要}{bu2yao4}
  \synonymref{拒绝}{ju4jue2}
  \synonymref{推卸}{tui1xie4}
  \synonymref{退却}{tui4que4}
  \antonymref{承诺}{cheng2nuo4}
  \antonymref{答应}{da1ying5}
  \antonymref{接纳}{jie1na4}
  \antonymref{接收}{jie1shou1}
  \antonymref{接受}{jie1shou4}
  \antonymref{提出}{ti2 chu1}
  \end{Phonetics}
\end{Entry}

\begin{Entry}{推敲}{11,14}{⼿,⽁}
  \begin{Phonetics}{推敲}{tui1qiao1}[][HSK 7-9]
    \definition{v.}{pesar; deliberar; pensar repetidamente ao escrever ou realizar tarefas}
  \synonymref{揣摩}{chuai3mo2}
  \synonymref{考虑}{kao3lv4}
  \synonymref{商量}{shang1liang5}
  \synonymref{思考}{si1kao3}
  \synonymref{思索}{si1suo3}
  \synonymref{研究}{yan2jiu1}
  \end{Phonetics}
\end{Entry}

\begin{Entry}{推算}{11,14}{⼿,⽵}
  \begin{Phonetics}{推算}{tui1suan4}[][HSK 7-9]
    \definition{v.}{calcular; estimar; calcular os valores relevantes com base nos dados existentes}
  \seealsoref{推测}{tui1ce4}
  \synonymref{计算}{ji4suan4}
  \synonymref{算计}{suan4ji4}
  \synonymref{阴谋}{yin1mou2}
  \antonymref{断定}{duan4ding4}
  \end{Phonetics}
\end{Entry}

\begin{Entry}{推翻}{11,18}{⼿,⽻}
  \begin{Phonetics}{推翻}{tui1/fan1}[][HSK 7-9]
    \definition{v.+compl.}{tombar; virar; derrubar; derrubar o regime original ou mudar o sistema social | cancelar; reverter; repudiar; negar completamente as declarações, conclusões, decisões, etc., existentes}
  \synonymref{撤销}{che4xiao1}
  \synonymref{摧毁}{cui1hui3}
  \synonymref{打倒}{da3/dao3}
  \synonymref{颠覆}{dian1fu4}
  \antonymref{创建}{chuang4jian4}
  \antonymref{创立}{chuang4li4}
  \antonymref{建立}{jian4li4}
  \antonymref{证明}{zheng4ming2}
  \end{Phonetics}
\end{Entry}

%%%%%%%%%% 措 %%%%%%%%%%
\subsection*{措}\addcontentsline{loh}{figure}{措}

\begin{Entry}{措}{11}{⼿}
  \begin{Phonetics}{措}{cuo4}
    \definition{s.}{iniciativa; solução; medida}
    \definition{v.}{organizar; gerenciar; lidar | fazer planos; administrar; organizar}
  \end{Phonetics}
\end{Entry}

\begin{Entry}{措手不及}{11,4,4,3}{⼿,⼿,⼀,⼃}
  \begin{Phonetics}{措手不及}{cuo4shou3-bu4ji2}[][HSK 7-9]
    \definition{expr.}{ser pego de surpresa; ser pego de surpresa (despreparado); ser tarde demais para fazer algo a respeito; ficar surpreso demais para se defender; não conseguir fazer uma defesa adequada; não conseguir pensar a tempo em uma maneira de se defender; não ter tempo para colocar em prática; surpreender alguém; pegar alguém desprevenido | ser pego desprevenido; ser pego de surpresa}
  \end{Phonetics}
\end{Entry}

\begin{Entry}{措施}{11,9}{⼿,⽅}
  \begin{Phonetics}{措施}{cuo4shi1}[][HSK 4]
    \definition[项,个]{s.}{medida; etapa; passo; abordagem adotada para lidar com as coisas}
  \end{Phonetics}
\end{Entry}

%%%%%%%%%% 掺 %%%%%%%%%%
\subsection*{掺}\addcontentsline{loh}{figure}{掺}

\begin{Entry}{掺}{11}{⼿}
  \begin{Phonetics}{掺}{can4}
    \definition{s.}{um estilo antigo de tocar bateria; uma antiga canção de tambor}
  \end{Phonetics}
  \begin{Phonetics}{掺}{chan1}[][HSK 7-9]
    \definition{v.}{misturar; mesclar; adicionar}
  \end{Phonetics}
  \begin{Phonetics}{掺}{shan3}
    \definition{v.}{misturar; mesclar | conter; reter}
  \end{Phonetics}
\end{Entry}

%%%%%%%%%% 描 %%%%%%%%%%
\subsection*{描}\addcontentsline{loh}{figure}{描}

\begin{Entry}{描}{11}{⼿}
  \begin{Phonetics}{描}{miao2}
    \definition{v.}{traçar; copiar | retocar; retocar | traçar um desenho | retratar | esboçar}
  \end{Phonetics}
\end{Entry}

\begin{Entry}{描写}{11,5}{⼿,⼍}
  \begin{Phonetics}{描写}{miao2xie3}[][HSK 4]
    \definition{v.}{representar; retratar; descrever; usar a linguagem e as palavras para transmitir uma imagem concreta de uma pessoa, evento ou situação}
  \end{Phonetics}
\end{Entry}

\begin{Entry}{描述}{11,8}{⼿,⾡}
  \begin{Phonetics}{描述}{miao2shu4}[][HSK 4]
    \definition[段,种]{s.}{descrição; trecho que descreve um evento ou uma cena}
    \definition{v.}{descrever; representar}
  \end{Phonetics}
\end{Entry}

\begin{Entry}{描绘}{11,9}{⼿,⽷}
  \begin{Phonetics}{描绘}{miao2hui4}[][HSK 7-9]
    \definition{v.}{descrever; retratar; representar; desenhar}
  \end{Phonetics}
\end{Entry}

%%%%%%%%%% 敎 %%%%%%%%%%
\subsection*{敎}\addcontentsline{loh}{figure}{敎}

\begin{Entry}{敎}{11}{⽁}
  \begin{Phonetics}{敎}{jiao4}
    \variantof{教}
  \end{Phonetics}
\end{Entry}

%%%%%%%%%% 敏 %%%%%%%%%%
\subsection*{敏}\addcontentsline{loh}{figure}{敏}

\begin{Entry}{敏}{11}{⽁}
  \begin{Phonetics}{敏}{min3}
    \definition*{s.}{Sobrenome: Min}
    \definition{adj.}{rápido; ágil | perspicaz; inteligente; rápido | inteligente; esperto}
  \end{Phonetics}
\end{Entry}

\begin{Entry}{敏捷}{11,11}{⽁,⼿}
  \begin{Phonetics}{敏捷}{min3jie2}[][HSK 7-9]
    \definition{adj.}{ágil; rápido; descreve reações rápidas em ações, pensamentos, etc.}
  \end{Phonetics}
\end{Entry}

\begin{Entry}{敏锐}{11,12}{⽁,⾦}
  \begin{Phonetics}{敏锐}{min3rui4}[][HSK 7-9]
    \definition{adj.}{agudo; perspicaz; aguçado; (pensamento) rápido de raciocínio, (intuição) aguçado}
  \end{Phonetics}
\end{Entry}

\begin{Entry}{敏感}{11,13}{⽁,⼼}
  \begin{Phonetics}{敏感}{min3gan3}[][HSK 5]
    \definition{adj.}{sensível; descreve pessoas ou animais que rapidamente percebem mudanças ou estímulos externos | reativo; sensível; fácil de causar reações intensas}
  \end{Phonetics}
\end{Entry}

%%%%%%%%%% 救 %%%%%%%%%%
\subsection*{救}\addcontentsline{loh}{figure}{救}

\begin{Entry}{救}{11}{⽁}
  \begin{Phonetics}{救}{jiu4}[][HSK 3]
    \definition*{s.}{Sobrenome: Jiu}
    \definition{v.}{resgatar; salvar | salvar de; aliviar (angústia, etc.) | resgatar; livrar alguém de um desastre ou perigo | ajudar; aliviar; socorrer; livrar pessoas e coisas de desastres e perigos}
  \end{Phonetics}
\end{Entry}

\begin{Entry}{救出}{11,5}{⽁,⼐}
  \begin{Phonetics}{救出}{jiu4chu1}
    \definition{v.}{resgatar | tirar do perigo}
  \end{Phonetics}
\end{Entry}

\begin{Entry}{救助}{11,7}{⽁,⼒}
  \begin{Phonetics}{救助}{jiu4zhu4}[][HSK 6]
    \definition{v.}{ajudar alguém em perigo ou dificuldade; socorrer; resgatar e ajudar}
  \end{Phonetics}
\end{Entry}

\begin{Entry}{救护车}{11,7,4}{⽁,⼿,⾞}
  \begin{Phonetics}{救护车}{jiu4hu4che1}[][HSK 7-9]
    \definition[辆]{s.}{ambulância; os veículos que transportam os feridos estão equipados com instalações que permitem à equipe médica prestar primeiros socorros temporários, cuidados médicos e serviços de enfermagem aos feridos}
  \end{Phonetics}
\end{Entry}

\begin{Entry}{救灾}{11,7}{⽁,⽕}
  \begin{Phonetics}{救灾}{jiu4 zai1}[][HSK 5]
    \definition{v.}{ajudar as vítimas de desastres, aliviar o desastre; resgatar pessoas afetadas por desastres; recuperar danos causados por desastres}
  \end{Phonetics}
\end{Entry}

\begin{Entry}{救命}{11,8}{⽁,⼝}
  \begin{Phonetics}{救命}{jiu4/ming4}[][HSK 6]
    \definition{interj.}{``Socorro!''; ``Salve-me!''}
    \definition{v.+compl.}{ajudar; salvar a vida de alguém}
  \end{Phonetics}
\end{Entry}

\begin{Entry}{救治}{11,8}{⽁,⽔}
  \begin{Phonetics}{救治}{jiu4zhi4}[][HSK 7-9]
    \definition{v.}{retirar um paciente do perigo; tratar e curar}
  \end{Phonetics}
\end{Entry}

\begin{Entry}{救济}{11,9}{⽁,⽔}
  \begin{Phonetics}{救济}{jiu4ji4}[][HSK 7-9]
    \definition{v.}{fornecer ajuda com dinheiro ou bens; usar dinheiro e bens para ajudar vítimas de desastres ou outras pessoas que vivem em situação de pobreza}
  \end{Phonetics}
\end{Entry}

\begin{Entry}{救援}{11,12}{⽁,⼿}
  \begin{Phonetics}{救援}{jiu4yuan2}[][HSK 6]
    \definition{v.}{resgatar; socorrer; vir em auxílio de alguém (resgate)}
  \end{Phonetics}
\end{Entry}

%%%%%%%%%% 教 %%%%%%%%%%
\subsection*{教}\addcontentsline{loh}{figure}{教}

\begin{Entry}{教}{11}{⽁}
  \begin{Phonetics}{教}{jiao1}
    \definition*{s.}{Sobrenome: Jiao}
    \definition{prep.}{em uma frase passiva para introduzir o executor da ação}
    \definition{s.}{religião | professor; referência à educação ou aos professores}
    \definition{v.}{ensinar; instruir |  pedir; ordenar; dizer | permitir; possibilitar}
  \synonymref{授}{shou4}
  \antonymref{学}{xue2}
  \end{Phonetics}
  \begin{Phonetics}{教}{jiao4}[][HSK 1]
    \definition*{s.}{Sobrenome: Jiao}
    \definition{prep.}{em uma frase passiva para apresentar o autor da ação}
    \definition{s.}{religião | educação; professor}
    \definition{v.}{ensinar; instruir | perguntar; ordenar; contar | permitir; permitir}
  \synonymref{授}{shou4}
  \antonymref{学}{xue2}
  \end{Phonetics}
\end{Entry}

\begin{Entry}{教长}{11,4}{⽁,⾧}
  \begin{Phonetics}{教长}{jiao4zhang3}
    \definition{s.}{imã (Islã) | mulá}
  \end{Phonetics}
\end{Entry}

\begin{Entry}{教训}{11,5}{⽁,⾔}
  \begin{Phonetics}{教训}{jiao4xun5}[][HSK 4]
    \definition[个,次,番,顿]{s.}{moral; lição}
    \definition{v.}{repreender; educar; ensinar uma lição a alguém; dar uma bronca em alguém; dar um sermão em alguém (por ter cometido um erro, etc.)}
  \end{Phonetics}
\end{Entry}

\begin{Entry}{教会}{11,6}{⽁,⼈}
  \begin{Phonetics}{教会}{jiao1hui4}
    \definition{v.}{mostrar | ensinar}
  \end{Phonetics}
  \begin{Phonetics}{教会}{jiao4hui4}
    \definition{s.}{igreja cristã}
  \end{Phonetics}
\end{Entry}

\begin{Entry}{教导}{11,6}{⽁,⼨}
  \begin{Phonetics}{教导}{jiao4dao3}
    \definition{s.}{instrução | orientação | ensino}
    \definition{v.}{instruir | orientar | ensinar}
  \end{Phonetics}
\end{Entry}

\begin{Entry}{教师}{11,6}{⽁,⼱}
  \begin{Phonetics}{教师}{jiao4shi1}[][HSK 2]
    \definition[个,位,名]{s.}{professor; professor de escola}
  \end{Phonetics}
\end{Entry}

\begin{Entry}{教材}{11,7}{⽁,⽊}
  \begin{Phonetics}{教材}{jiao4cai2}[][HSK 3]
    \definition[本,套]{s.}{livro didático; materiais didáticos, incluindo livros didáticos, apostilas, materiais de referência, vídeos, imagens, etc.}
  \end{Phonetics}
\end{Entry}

\begin{Entry}{教条}{11,7}{⽁,⽊}
  \begin{Phonetics}{教条}{jiao4tiao2}[][HSK 7-9]
    \definition{adj.}{dogmático; opinativo}
    \definition{s.}{dogma; doutrina; credo | princípio; chavão}
  \end{Phonetics}
\end{Entry}

\begin{Entry}{教学}{11,8}{⽁,⼦}
  \begin{Phonetics}{教学}{jiao4xue2}[][HSK 2]
    \definition[个,门]{s.}{ensino; educação; o processo de transmissão de conhecimentos e habilidades}
  \end{Phonetics}
\end{Entry}

\begin{Entry}{教学楼}{11,8,13}{⽁,⼦,⽊}
  \begin{Phonetics}{教学楼}{jiao4xue2lou2}[][HSK 1]
    \definition{s.}{prédio da escola; bloco de ensino; edifícios utilizados para atividades educacionais, geralmente incluindo salas de aula, laboratórios, auditórios, etc.}
  \end{Phonetics}
\end{Entry}

\begin{Entry}{教官}{11,8}{⽁,⼧}
  \begin{Phonetics}{教官}{jiao4guan1}
    \definition{s.}{instrutor militar; um oficial que serviu como treinador no antigo exército ou escola}
  \end{Phonetics}
\end{Entry}

\begin{Entry}{教练}{11,8}{⽁,⽷}
  \begin{Phonetics}{教练}{jiao4lian4}[][HSK 3]
    \definition[个,位,名]{s.}{instrutor; treinador (esportes); pessoas que trabalham como treinadores}
    \definition{v.}{treinar; treinar outras pessoas para dominarem uma determinada técnica (como esportes, dirigir carros, pilotar aviões, etc.)}
  \end{Phonetics}
\end{Entry}

\begin{Entry}{教育}{11,8}{⽁,⾁}
  \begin{Phonetics}{教育}{jiao4yu4}[][HSK 2]
    \definition{s.}{educação; refere"-se a atividades sociais cujo objetivo direto é influenciar o desenvolvimento físico e mental das pessoas; refere"-se principalmente ao processo de formação dos alunos nas escolas}
    \definition{v.}{ensinar; educar; inspirar, fazer compreender a razão}
  \end{Phonetics}
\end{Entry}

\begin{Entry}{教育部}{11,8,10}{⽁,⾁,⾢}
  \begin{Phonetics}{教育部}{jiao4yu4bu4}[][HSK 6]
    \definition*{s.}{Ministério da Educação}
  \end{Phonetics}
\end{Entry}

\begin{Entry}{教养}{11,9}{⽁,⼋}
  \begin{Phonetics}{教养}{jiao4yang3}[][HSK 7-9]
    \definition{s.}{criação; educação; formação}
  \end{Phonetics}
\end{Entry}

\begin{Entry}{教室}{11,9}{⽁,⼧}
  \begin{Phonetics}{教室}{jiao4shi4}[][HSK 2]
    \definition[间]{s.}{sala de aula}
  \end{Phonetics}
\end{Entry}

\begin{Entry}{教科书}{11,9,4}{⽁,⽲,⼄}
  \begin{Phonetics}{教科书}{jiao4ke1shu1}[][HSK 7-9]
    \definition[本]{s.}{livro didático; livro do aluno | livro escolar; um livro escrito especialmente para alunos usarem em sala de aula e para revisão}
  \end{Phonetics}
\end{Entry}

\begin{Entry}{教堂}{11,11}{⽁,⼟}
  \begin{Phonetics}{教堂}{jiao4tang2}[][HSK 6]
    \definition[座,所,间]{s.}{igreja; capela; catedral; casa de deus; um lugar onde os cristãos realizam cerimônias religiosas}
  \end{Phonetics}
\end{Entry}

\begin{Entry}{教授}{11,11}{⽁,⼿}
  \begin{Phonetics}{教授}{jiao4shou4}[][HSK 4]
    \definition[个,位,名]{s.}{professor (universitário); o professor com a classificação mais alta em uma universidade}
    \definition{v.}{ensinar; instruir; dar aulas; dar palestras}
  \end{Phonetics}
\end{Entry}

%%%%%%%%%% 敢 %%%%%%%%%%
\subsection*{敢}\addcontentsline{loh}{figure}{敢}

\begin{Entry}{敢}{11}{⽁}
  \begin{Phonetics}{敢}{gan3}[][HSK 3]
    \definition{adj.}{ousado; corajoso; audacioso; valente}
    \definition{adv.}{talvez; provavelmente}
    \definition{v.}{ser ousado o suficiente; atrever-se | ter confiança em; ter certeza; estar certo | aventurar-se; ter coragem de fazer algo | ser ousado; arriscar-se}
  \end{Phonetics}
\end{Entry}

\begin{Entry}{敢于}{11,3}{⽁,⼆}
  \begin{Phonetics}{敢于}{gan3yu2}[][HSK 6]
    \definition{v.}{ousar; ser ousado em; ter determinação; ter coragem (para fazer ou se esforçar para fazer)}
  \end{Phonetics}
\end{Entry}

\begin{Entry}{敢情}{11,11}{⽁,⼼}
  \begin{Phonetics}{敢情}{gan3qing5}[][HSK 7-9]
    \definition{adv.}{por que; então; eu digo; indica a descoberta de algo que não foi descoberto anteriormente | claro; de fato; realmente; isso significa que a razão é óbvia e não há necessidade de duvidar dela}
  \end{Phonetics}
\end{Entry}

%%%%%%%%%% 斛 %%%%%%%%%%
\subsection*{斛}\addcontentsline{loh}{figure}{斛}

\begin{Entry}{斛}{11}{⽃}
  \begin{Phonetics}{斛}{hu2}
    \definition*{s.}{Sobrenome: Hu}
    \definition{s.}{Arcaico: uma medida seca usada antigamente, originalmente igual a 10 dou (斗), mais tarde 5 dou}
  \seealsoref{斗}{dou4}
  \end{Phonetics}
\end{Entry}

%%%%%%%%%% 斜 %%%%%%%%%%
\subsection*{斜}\addcontentsline{loh}{figure}{斜}

\begin{Entry}{斜}{11}{⽃}
  \begin{Phonetics}{斜}{xie2}[][HSK 5]
    \definition{adj.}{oblíquo; inclinado | enviesado; chanfrado; diagonal; torto; nem paralelo nem perpendicular a um plano ou linha}
    \definition{v.}{virar de lado; inclinar}
  \end{Phonetics}
\end{Entry}

\begin{Entry}{斜阳}{11,6}{⽃,⾩}
  \begin{Phonetics}{斜阳}{xie2yang2}
    \definition{s.}{sol poente}
  \end{Phonetics}
\end{Entry}

%%%%%%%%%% 断 %%%%%%%%%%
\subsection*{断}\addcontentsline{loh}{figure}{断}

\begin{Entry}{断}{11}{⽄}
  \begin{Phonetics}{断}{duan4}[][HSK 3]
    \definition*{s.}{Sobrenome: Duan}
    \definition{adv.}{(geralmente na forma negativa) absolutamente; decididamente}
    \definition{v.}{quebrar; partir; (objetos longos) dividir em segmentos não conectados | parar; interromper; romper; isolar; fazer com que não se sucedam mais | desistir; abster"-se de; parar de fumar, beber, etc. | julgar; decidir | interceptar}
  \end{Phonetics}
\end{Entry}

\begin{Entry}{断交}{11,6}{⽄,⼇}
  \begin{Phonetics}{断交}{duan4/jiao1}
    \definition{v.+compl.}{terminar uma amizade | romper relações diplomáticas}
  \end{Phonetics}
\end{Entry}

\begin{Entry}{断定}{11,8}{⽄,⼧}
  \begin{Phonetics}{断定}{duan4ding4}[][HSK 7-9]
    \definition{v.}{decidir; determinar; concluir; formar um julgamento; fazer um julgamento definitivo}
  \end{Phonetics}
\end{Entry}

\begin{Entry}{断断续续}{11,11,11,11}{⽄,⽄,⽷,⽷}
  \begin{Phonetics}{断断续续}{duan4duan4xu4xu4}[][HSK 7-9]
    \definition{adj.}{intermitente; esporádico; ocasional; aos trancos e barrancos}
  \end{Phonetics}
\end{Entry}

\begin{Entry}{断裂}{11,12}{⽄,⾐}
  \begin{Phonetics}{断裂}{duan4lie4}[][HSK 7-9]
    \definition{s.}{fraturar; quebrar; surtar | fender; rachar; romper}
  \end{Phonetics}
\end{Entry}

%%%%%%%%%% 旋 %%%%%%%%%%
\subsection*{旋}\addcontentsline{loh}{figure}{旋}

\begin{Entry}{旋}{11}{⽅}
  \begin{Phonetics}{旋}{xuan2}
    \definition*{s.}{Sobrenome: Xuan}
    \definition{adv.}{em breve; rapidamente}
    \definition{s.}{redemoinho; turbilhão; vórtice}
    \definition{v.}{girar; circular; rodar | retornar; voltar}
  \end{Phonetics}
\end{Entry}

\begin{Entry}{旋转}{11,8}{⽅,⾞}
  \begin{Phonetics}{旋转}{xuan2zhuan3}[][HSK 6]
    \definition{v.}{girar; rodar; revolver; rodopiar; o movimento circular de um objeto em torno de um ponto ou eixo}
  \end{Phonetics}
\end{Entry}

%%%%%%%%%% 族 %%%%%%%%%%
\subsection*{族}\addcontentsline{loh}{figure}{族}

\begin{Entry}{族}{11}{⽅}
  \begin{Phonetics}{族}{zu2}[][HSK 6]
    \definition{s.}{clã; família | uma pena de morte na China antiga, imposta ao infrator e a toda a sua família, ou mesmo às famílias de sua mãe e esposa; uma antiga forma de tortura | etnia; nacionalidade | uma grande categoria de coisas que compartilham algum atributo comum}
  \end{Phonetics}
\end{Entry}

%%%%%%%%%% 旣 %%%%%%%%%%
\subsection*{旣}\addcontentsline{loh}{figure}{旣}

\begin{Entry}{旣}{11}{⽆}
  \begin{Phonetics}{旣}{ji4}
    \variantof{既}
  \end{Phonetics}
\end{Entry}

%%%%%%%%%% 晚 %%%%%%%%%%
\subsection*{晚}\addcontentsline{loh}{figure}{晚}

\begin{Entry}{晚}{11}{⽇}
  \begin{Phonetics}{晚}{wan3}[][HSK 1]
    \definition*{s.}{Sobrenome: Wan}
    \definition{adj.}{tarde; tardio; passado o prazo acordado | júnior; mais jovem | mais tarde no tempo}
    \definition{s.}{noite; à noite; após o pôr do sol | últimos anos; última vida; um período posterior; refere"-se especificamente à velhice de uma pessoa | pôr do sol; ao pôr do sol}
  \end{Phonetics}
\end{Entry}

\begin{Entry}{晚上}{11,3}{⽇,⼀}
  \begin{Phonetics}{晚上}{wan3shang5}[][HSK 1]
    \definition[个]{s.}{noite; o período entre o pôr do sol e a madrugada}
  \end{Phonetics}
\end{Entry}

\begin{Entry}{晚会}{11,6}{⽇,⼈}
  \begin{Phonetics}{晚会}{wan3hui4}[][HSK 2]
    \definition[场,个,次]{s.}{festa noturna; entretenimento noturno}
  \end{Phonetics}
\end{Entry}

\begin{Entry}{晚安}{11,6}{⽇,⼧}
  \begin{Phonetics}{晚安}{wan3'an1}[][HSK 2]
    \definition{expr.}{Tenha uma boa noite; uma frase educada usada para se despedir ou cumprimentar as pessoas à noite}
  \end{Phonetics}
\end{Entry}

\begin{Entry}{晚年}{11,6}{⽇,⼲}
  \begin{Phonetics}{晚年}{wan3nian2}[][HSK 7-9]
    \definition{s.}{velhice; ocaso; os últimos anos (restantes) da vida; crepúsculo; os estágios finais da vida}
  \end{Phonetics}
\end{Entry}

\begin{Entry}{晚报}{11,7}{⽇,⼿}
  \begin{Phonetics}{晚报}{wan3bao4}[][HSK 2]
    \definition[份,张]{s.}{jornal vespertino; um jornal publicado todas as tardes}
  \end{Phonetics}
\end{Entry}

\begin{Entry}{晚近}{11,7}{⽇,⾡}
  \begin{Phonetics}{晚近}{wan3jin4}
    \definition{s.}{nos últimos anos; durante os últimos anos | tarde | mais recente no passado | recentemente}
  \end{Phonetics}
\end{Entry}

\begin{Entry}{晚间}{11,7}{⽇,⾨}
  \begin{Phonetics}{晚间}{wan3jian1}[][HSK 7-9]
    \definition{s.}{(à) noite; (ao) anoitecer}
  \synonymref{晚上}{wan3shang5}
  \antonymref{早晨}{zao3chen5}
  \end{Phonetics}
\end{Entry}

\begin{Entry}{晚饭}{11,7}{⽇,⾷}
  \begin{Phonetics}{晚饭}{wan3fan4}[][HSK 1]
    \definition[顿]{s.}{jantar}
  \end{Phonetics}
\end{Entry}

\begin{Entry}{晚育}{11,8}{⽇,⾁}
  \begin{Phonetics}{晚育}{wan3yu4}
    \definition{s.}{parto tardio}
    \definition{v.}{ter um filho mais tarde}
  \end{Phonetics}
\end{Entry}

\begin{Entry}{晚点}{11,9}{⽇,⽕}
  \begin{Phonetics}{晚点}{wan3 dian3}[][HSK 4]
    \definition{v.}{atrasar; adiar; (veículo, navio ou avião) partir, operar ou chegar mais tarde do que o horário especificado}
  \end{Phonetics}
\end{Entry}

\begin{Entry}{晚景}{11,12}{⽇,⽇}
  \begin{Phonetics}{晚景}{wan3jing3}
    \definition{s.}{circunstâncias dos anos de declínio de alguém | cena noturna}
  \end{Phonetics}
\end{Entry}

\begin{Entry}{晚期}{11,12}{⽇,⽉}
  \begin{Phonetics}{晚期}{wan3qi1}[][HSK 7-9]
    \definition{s.}{estágio tardio; (da doença) estágio terminal | período posterior | fase final | terminal}
  \antonymref{早期}{zao3qi1}
  \end{Phonetics}
\end{Entry}

\begin{Entry}{晚餐}{11,16}{⽇,⾷}
  \begin{Phonetics}{晚餐}{wan3can1}[][HSK 2]
    \definition[份,顿,次]{s.}{ceia; jantar}
  \end{Phonetics}
\end{Entry}

%%%%%%%%%% 望 %%%%%%%%%%
\subsection*{望}\addcontentsline{loh}{figure}{望}

\begin{Entry}{望}{11}{⽉}
  \begin{Phonetics}{望}{wang4}[][HSK 7-9]
    \definition*{s.}{Sobrenome: Wang}
    \definition{prep.}{para; em direção a; em ``olhando para frente (望前看)'', ``caminhando para o leste (望东走)'', etc.; 望 é frequentemente escrito como 往}
    \definition{s.}{prestígio; reputação; fama | lua cheia | o 15º dia de um mês lunar}
    \definition{v.}{olhar por cima; olhar para a distância; olhar para longe na distância | visitar; ligar para | ter esperança; esperar | odiar; ressentir"-se | pensar em atingir um determinado objetivo ou uma determinada situação em mente}
  \seealsoref{往}{wang3}
  \synonymref{看}{kan4}
  \synonymref{瞧}{qiao2}
  \synonymref{视}{shi4}
  \end{Phonetics}
\end{Entry}

\begin{Entry}{望见}{11,4}{⽉,⾒}
  \begin{Phonetics}{望见}{wang4jian4}[][HSK 6]
    \definition{v.}{espiar; ver; pôr os olhos em | detectar}
  \end{Phonetics}
\end{Entry}

\begin{Entry}{望远镜}{11,7,16}{⽉,⾡,⾦}
  \begin{Phonetics}{望远镜}{wang4yuan3jing4}[][HSK 7-9]
    \definition[架,个,付,副,部]{s.}{telescópio; o telescópio refrator mais simples consiste em dois conjuntos de lentes | binóculos; um instrumento óptico para observar objetos distantes}
  \end{Phonetics}
\end{Entry}

%%%%%%%%%% 桶 %%%%%%%%%%
\subsection*{桶}\addcontentsline{loh}{figure}{桶}

\begin{Entry}{桶}{11}{⽊}
  \begin{Phonetics}{桶}{tong3}[][HSK 7-9]
    \definition{clas.}{barril; uma unidade de capacidade}
    \definition[只,个]{s.}{barril; tina; tonel; balde; vaso sanitário}
  \end{Phonetics}
\end{Entry}

%%%%%%%%%% 梅 %%%%%%%%%%
\subsection*{梅}\addcontentsline{loh}{figure}{梅}

\begin{Entry}{梅}{11}{⽊}
  \begin{Phonetics}{梅}{mei2}
    \definition*{s.}{Sobrenome: Mei}
    \definition{s.}{ameixa | flor de ameixa | ameixeira | estação chuvosa}
  \end{Phonetics}
\end{Entry}

\begin{Entry}{梅花}{11,7}{⽊,⾋}
  \begin{Phonetics}{梅花}{mei2hua1}[][HSK 6]
    \definition[朵,枝,片,瓣,束,株]{s.}{paus ♣ (um naipe em jogos de cartas) | flor de ameixa | doçura"-de"-inverno; refere"-se especificamente à flor"-de"-inverno ; também se refere a algo que se parece com esta flor}
  \seealsoref{方片}{fang1 pian4}
  \seealsoref{黑桃}{hei1tao2}
  \seealsoref{红心}{hong2xin1}
  \end{Phonetics}
\end{Entry}

\begin{Entry}{梅赛德斯-奔驰}{11,14,15,12,8,6}{⽊,⾙,⼻,⽄,⼤,⾺}
  \begin{Phonetics}{梅赛德斯-奔驰}{mei2sai4de2si1-ben1chi2}
    \definition*{s.}{Mercedes-Benz}
  \end{Phonetics}
\end{Entry}

%%%%%%%%%% 梦 %%%%%%%%%%
\subsection*{梦}\addcontentsline{loh}{figure}{梦}

\begin{Entry}{梦}{11}{⼣}
  \begin{Phonetics}{梦}{meng4}[][HSK 4]
    \definition*{s.}{Sobrenome: Meng}
    \definition[个,场]{s.}{sonho; atividade de representação no cérebro durante o sono}
    \definition{v.}{sonhar; ter um sonho}
  \end{Phonetics}
\end{Entry}

\begin{Entry}{梦幻}{11,4}{⼣,⼳}
  \begin{Phonetics}{梦幻}{meng4huan4}[][HSK 7-9]
    \definition[种,片]{s.}{ilusão; sonho; devaneio; cenas e situações estranhas que aparecem nos sonhos}
  \end{Phonetics}
\end{Entry}

\begin{Entry}{梦见}{11,4}{⼣,⾒}
  \begin{Phonetics}{梦见}{meng4jian4}[][HSK 4]
    \definition{v.}{sonhar; sonhar com; ver em um sonho}
  \end{Phonetics}
\end{Entry}

\begin{Entry}{梦想}{11,13}{⼣,⼼}
  \begin{Phonetics}{梦想}{meng4xiang3}[][HSK 4]
    \definition[个,种,些,番]{s.}{sonho; esperança vã; sonho irreal; divagação; um desejo ou ideia que você espera particularmente realizar}
    \definition{v.}{sonhar; desejar sinceramente; ansiar}
  \end{Phonetics}
\end{Entry}

%%%%%%%%%% 梨 %%%%%%%%%%
\subsection*{梨}\addcontentsline{loh}{figure}{梨}

\begin{Entry}{梨}{11}{⽊}
  \begin{Phonetics}{梨}{li2}[][HSK 5]
    \definition*{s.}{Sobrenome: Li}
    \definition[个,只,斤,棵,种]{s.}{perira; árvore de pera | pera}
  \end{Phonetics}
\end{Entry}

%%%%%%%%%% 梯 %%%%%%%%%%
\subsection*{梯}\addcontentsline{loh}{figure}{梯}

\begin{Entry}{梯}{11}{⽊}
  \begin{Phonetics}{梯}{ti1}
    \definition*{s.}{Sobrenome: Ti}
    \definition{adj.}{em forma de escada; em socalcos}
    \definition[个]{s.}{escada; degrau; socalco (são plataformas niveladas, semelhantes a degraus, cortadas em encostas de morros para permitir o cultivo agrícola e evitar a erosão do solo)}
  \end{Phonetics}
\end{Entry}

\begin{Entry}{梯子}{11,3}{⽊,⼦}
  \begin{Phonetics}{梯子}{ti1zi5}[][HSK 7-9]
    \definition[个,把]{s.}{escada; escada de mão; ferramentas para facilitar o acesso das pessoas}
  \end{Phonetics}
\end{Entry}

\begin{Entry}{梯恩梯}{11,10,11}{⽊,⼼,⽊}
  \begin{Phonetics}{梯恩梯}{ti1'en1ti1}
    \definition{s.}{Empréstimo linguístico: TNT, trinitrotolueno}
  \end{Phonetics}
\end{Entry}

%%%%%%%%%% 梳 %%%%%%%%%%
\subsection*{梳}\addcontentsline{loh}{figure}{梳}

\begin{Entry}{梳}{11}{⽊}
  \begin{Phonetics}{梳}{shu1}[][HSK 7-9]
    \definition{s.}{pente}
    \definition{v.}{pentear (o cabelo, etc.)}
  \end{Phonetics}
\end{Entry}

\begin{Entry}{梳子}{11,3}{⽊,⼦}
  \begin{Phonetics}{梳子}{shu1zi5}[][HSK 7-9]
    \definition[把,个]{s.}{pente; ferramentas para cuidar do cabelo e da barba}
  \end{Phonetics}
\end{Entry}

\begin{Entry}{梳理}{11,11}{⽊,⽟}
  \begin{Phonetics}{梳理}{shu1li3}[][HSK 7-9]
    \definition{v.}{pentear; na fabricação têxtil, o uso de agulhas ou dentes em uma máquina para alinhar as fibras e remover fibras curtas e impurezas é um processo | pentear (o cabelo); desembaraçar (o cabelo); usar um pente para pentear (barba, cabelo, etc.) | organizar; resolver assuntos, problemas, etc.}
  \synonymref{整顿}{zheng3dun4}
  \synonymref{整理}{zheng3li3}
  \synonymref{整治}{zheng3zhi4}
  \synonymref{治理}{zhi4li3}
  \end{Phonetics}
\end{Entry}

%%%%%%%%%% 检 %%%%%%%%%%
\subsection*{检}\addcontentsline{loh}{figure}{检}

\begin{Entry}{检}{11}{⽊}
  \begin{Phonetics}{检}{jian3}
    \definition*{s.}{Sobrenome: Jian}
    \definition{v.}{verificar; inspecionar; examinar | conter-se; ter cuidado na conduta}
  \end{Phonetics}
\end{Entry}

\begin{Entry}{检讨}{11,5}{⽊,⾔}
  \begin{Phonetics}{检讨}{jian3tao3}[][HSK 7-9]
    \definition[份,个]{s.}{autocrítica}
    \definition{v.}{fazer uma autocrítica; identificar e reconhecer suas próprias falhas ou erros | examinar; inspecionar}
  \end{Phonetics}
\end{Entry}

\begin{Entry}{检查}{11,9}{⽊,⽊}
  \begin{Phonetics}{检查}{jian3cha2}[][HSK 2]
    \definition[份,个,次]{s.}{autocrítica; reconhecer e criticar os próprios erros verbais ou escritos}
    \definition{v.}{verificar; inspecionar; examinar; verificar cuidadosamente para descobrir o problema | criticar a si mesmo; identificar seus pontos fracos e erros, e criticar seu próprio comportamento}
  \end{Phonetics}
\end{Entry}

\begin{Entry}{检测}{11,9}{⽊,⽔}
  \begin{Phonetics}{检测}{jian3ce4}[][HSK 4]
    \definition{v.}{testar; detectar; verificar}
  \end{Phonetics}
\end{Entry}

\begin{Entry}{检验}{11,10}{⽊,⾺}
  \begin{Phonetics}{检验}{jian3yan4}[][HSK 5]
    \definition{v.}{testar; examinar; inspecionar}
  \end{Phonetics}
\end{Entry}

\begin{Entry}{检察}{11,14}{⽊,⼧}
  \begin{Phonetics}{检察}{jian3cha2}[][HSK 7-9]
    \definition{v.}{realizar trabalho de procuradoria; refere"-se especificamente às atividades de supervisão legal realizadas pelas autoridades de supervisão legal do Estado, em conformidade com a lei}
  \end{Phonetics}
\end{Entry}

%%%%%%%%%% 欲 %%%%%%%%%%
\subsection*{欲}\addcontentsline{loh}{figure}{欲}

\begin{Entry}{欲}{11}{⽋}
  \begin{Phonetics}{欲}{yu4}
    \definition{adj.}{desejo | apetite | paixão | luxúria | ganância}
    \definition{v.}{desejar}
  \end{Phonetics}
\end{Entry}

%%%%%%%%%% 毫 %%%%%%%%%%
\subsection*{毫}\addcontentsline{loh}{figure}{毫}

\begin{Entry}{毫}{11}{⽊}
  \begin{Phonetics}{毫}{hao2}
    \definition{adv.}{nem um pouco; absolutamente nenhum; completamente sem}
    \definition{clas.}{hao, uma unidade de comprimento igual a um milésimo de polegada ou 1/30 de milímetro | hao, uma unidade de peso igual a um milésimo de um centavo ou 0,005 grama | uma fração minúscula; uma parte muito pequena}
    \definition{pref.}{mili-, usado com a unidade de uma quantidade física para representar um milésimo dessa quantidade}
    \definition{s.}{cabelo longo e fino | pincel de escrita | uma das duas ou três alças de uma balança para pendurar na mão do usuário | cerda; uma corda de mão em uma balança ou equilíbrio | fio de cabelo}
  \end{Phonetics}
\end{Entry}

\begin{Entry}{毫不}{11,4}{⽊,⼀}
  \begin{Phonetics}{毫不}{hao2 bu4}[][HSK 7-9]
    \definition{adv.}{dificilmente; de jeito nenhum; nem um pouco}
  \end{Phonetics}
\end{Entry}

\begin{Entry}{毫不犹豫}{11,4,7,15}{⽊,⼀,⽝,⾗}
  \begin{Phonetics}{毫不犹豫}{hao2 bu4 you2yu4}[][HSK 7-9]
    \definition{expr.}{sem hesitação; sem a menor hesitação}
  \end{Phonetics}
\end{Entry}

\begin{Entry}{毫不费力}{11,4,9,2}{⽊,⼀,⾙,⼒}
  \begin{Phonetics}{毫不费力}{hao2bu2fei4li4}
    \definition{expr.}{sem esforço; não despender o menor esforço}
  \end{Phonetics}
\end{Entry}

\begin{Entry}{毫升}{11,4}{⽊,⼗}
  \begin{Phonetics}{毫升}{hao2sheng1}[][HSK 4]
    \definition{clas.}{mililitro; unidade de volume, milésimo de um litro (ml)}
  \end{Phonetics}
\end{Entry}

\begin{Entry}{毫无}{11,4}{⽊,⽆}
  \begin{Phonetics}{毫无}{hao2wu2}[][HSK 7-9]
    \definition{adv.}{não; nada; de jeito nenhum}
  \end{Phonetics}
\end{Entry}

\begin{Entry}{毫米}{11,6}{⽊,⽶}
  \begin{Phonetics}{毫米}{hao2mi3}[][HSK 4]
    \definition{clas.}{milímetro; unidade legal de medida de comprimento, 1 mm equivale a 0,1 cm}
  \end{Phonetics}
\end{Entry}

%%%%%%%%%% 涮 %%%%%%%%%%
\subsection*{涮}\addcontentsline{loh}{figure}{涮}

\begin{Entry}{涮}{11}{⽔}
  \begin{Phonetics}{涮}{shuan4}[][HSK 7-9]
    \definition{v.}{lavar; balançar a mão ou algo parecido na água | enxaguar; colcar água dentro do recipiente e agitar para enxaguar | mergulhar; escaldar as fatias de carne rapidamente em água fervente, depois retirar e mergulhar no molho antes de servir | enganar; ludibriar; perturbar}
  \end{Phonetics}
\end{Entry}

%%%%%%%%%% 液 %%%%%%%%%%
\subsection*{液}\addcontentsline{loh}{figure}{液}

\begin{Entry}{液}{11}{⽔}
  \begin{Phonetics}{液}{ye4}
    \definition{s.}{líquido; fluido; suco}
  \end{Phonetics}
\end{Entry}

\begin{Entry}{液体}{11,7}{⽔,⼈}
  \begin{Phonetics}{液体}{ye4ti3}
    \definition{adj./s.}{líquido}
  \end{Phonetics}
\end{Entry}

%%%%%%%%%% 涵 %%%%%%%%%%
\subsection*{涵}\addcontentsline{loh}{figure}{涵}

\begin{Entry}{涵}{11}{⽔}
  \begin{Phonetics}{涵}{han2}
    \definition{s.}{bueiro; galeria}
    \definition{v.}{conter; incorporar}
  \end{Phonetics}
\end{Entry}

\begin{Entry}{涵义}{11,3}{⽔,⼂}
  \begin{Phonetics}{涵义}{han2yi4}[][HSK 7-9]
    \definition[层,种]{s.}{significado; implicação | conotação | conteúdo}
  \end{Phonetics}
\end{Entry}

\begin{Entry}{涵盖}{11,11}{⽔,⽫}
  \begin{Phonetics}{涵盖}{han2gai4}[][HSK 7-9]
    \definition{v.}{cobrir; incluir; conter; conter completamente}
  \end{Phonetics}
\end{Entry}

%%%%%%%%%% 淀 %%%%%%%%%%
\subsection*{淀}\addcontentsline{loh}{figure}{淀}

\begin{Entry}{淀}{11}{⽔}
  \begin{Phonetics}{淀}{dian4}
    \definition{s.}{lago raso, frequentemente usado em nomes de lugares}
    \definition{v.}{formar sedimentos | sedimentar; precipitar}
  \end{Phonetics}
\end{Entry}

\begin{Entry}{淀粉}{11,10}{⽔,⽶}
  \begin{Phonetics}{淀粉}{dian4fen3}[][HSK 7-9]
    \definition[包,勺,克,的]{s.}{amido; amilo; os carboidratos são os principais componentes das sementes, raízes e tubérculos}
  \end{Phonetics}
\end{Entry}

%%%%%%%%%% 淋 %%%%%%%%%%
\subsection*{淋}\addcontentsline{loh}{figure}{淋}

\begin{Entry}{淋}{11}{⽔}
  \begin{Phonetics}{淋}{lin2}[][HSK 7-9]
    \definition{v.}{derramar; encharcar | pulverizar; borrifar}
  \end{Phonetics}
  \begin{Phonetics}{淋}{lin4}
    \definition{v.}{filtrar; coar}
  \end{Phonetics}
\end{Entry}

%%%%%%%%%% 淌 %%%%%%%%%%
\subsection*{淌}\addcontentsline{loh}{figure}{淌}

\begin{Entry}{淌}{11}{⽔}
  \begin{Phonetics}{淌}{tang3}[][HSK 7-9]
    \definition{v.}{gotejar; escorrer; pingar | fluir para baixo}
  \synonymref{流}{liu2}
  \end{Phonetics}
\end{Entry}

%%%%%%%%%% 淘 %%%%%%%%%%
\subsection*{淘}\addcontentsline{loh}{figure}{淘}

\begin{Entry}{淘}{11}{⽔}
  \begin{Phonetics}{淘}{tao2}[][HSK 7-9]
    \definition{adj.}{travesso}
    \definition{v.}{lavar em uma bacia ou cesto; enxaguar | limpar; dragar; retirar com concha; recolher; retirar esgoto, etc. | causar problemas; ser um fardo para a mente; perturbar o cérebro com algo}
  \seealsoref{掏}{tao1}
  \end{Phonetics}
\end{Entry}

\begin{Entry}{淘气}{11,4}{⽔,⽓}
  \begin{Phonetics}{淘气}{tao2/qi4}[][HSK 7-9]
    \definition*{v.+compl.}{Dialeto: ficar com raiva; perder a paciência}
    \definition{adj.}{travesso; malicioso; descreve uma criança como particularmente brincalhona e travessa}
  \synonymref{调皮}{tiao2pi2}
  \synonymref{顽皮}{wan2pi2}
  \antonymref{老实}{lao3shi5}
  \antonymref{听话}{ting1/hua4}
  \end{Phonetics}
\end{Entry}

\begin{Entry}{淘汰}{11,7}{⽔,⽔}
  \begin{Phonetics}{淘汰}{tao2tai4}[][HSK 7-9]
    \definition{v.}{eliminar por meio de seleção ou competição; ao selecionar, você pode eliminar os itens ruins ou inadequados}
  \synonymref{减少}{jian3shao3}
  \synonymref{剔除}{ti1chu2}
  \antonymref{达标}{da2biao1}
  \antonymref{合格}{he2ge2}
  \antonymref{挑选}{tiao1xuan3}
  \antonymref{通过}{tong1guo4}
  \antonymref{选择}{xuan3ze2}
  \antonymref{引进}{yin3jin4}
  \end{Phonetics}
\end{Entry}

%%%%%%%%%% 淡 %%%%%%%%%%
\subsection*{淡}\addcontentsline{loh}{figure}{淡}

\begin{Entry}{淡}{11}{⽔}
  \begin{Phonetics}{淡}{dan4}[][HSK 4]
    \definition*{s.}{Sobrenome: Dan}
    \definition{adj.}{sem gosto; fraco; não tem sabor forte; não é salgado | leve; fraco; pálido | indiferente; frio; sem entusiasmo | frouxo; sem brilho | sem sentido; trivial | fino; leve}
  \end{Phonetics}
\end{Entry}

\begin{Entry}{淡化}{11,4}{⽔,⼔}
  \begin{Phonetics}{淡化}{dan4hua4}[][HSK 7-9]
    \definition{v.}{dessalinizar; transformar água com alto teor de sal em água com baixo teor de sal | desaparecer; enfraquecer; tornar ou tornar"-se menos importante}
  \end{Phonetics}
\end{Entry}

\begin{Entry}{淡季}{11,8}{⽔,⼦}
  \begin{Phonetics}{淡季}{dan4ji4}[][HSK 7-9]
    \definition{s.}{baixa temporada; temporada fraca (ou monótona, fora de temporada) | uma estação em que a produção de um determinado produto é baixa; uma estação em que os negócios estão lentos (diferente da 旺季)}
  \seealsoref{旺季}{wang4ji4}
  \end{Phonetics}
\end{Entry}

%%%%%%%%%% 淤 %%%%%%%%%%
\subsection*{淤}\addcontentsline{loh}{figure}{淤}

\begin{Entry}{淤}{11}{⽔}
  \begin{Phonetics}{淤}{yu1}
    \definition{adj.}{assoreado}
    \definition{s.}{lodo}
    \definition[出]{s.}{(medicina chinesa) estase de sangue}
    \definition{v.}{ficar assoreado; ficar sufocado com lodo | derramar; transbordar}
  \end{Phonetics}
\end{Entry}

\begin{Entry}{淤泥}{11,8}{⽔,⽔}
  \begin{Phonetics}{淤泥}{yu1ni2}
    \definition[出]{s.}{gosma | limo | lodo}
  \end{Phonetics}
\end{Entry}

%%%%%%%%%% 深 %%%%%%%%%%
\subsection*{深}\addcontentsline{loh}{figure}{深}

\begin{Entry}{深}{11}{⽔}
  \begin{Phonetics}{深}{shen1}[][HSK 3]
    \definition*{s.}{Sobrenome: Shen}
    \definition{adj.}{profundo | difícil; intenso; profundo | completo; penetrante; intenso; profundo | próximo; íntimo; afeição profunda; relacionamento próximo | escuro; profundo | tardio}
    \definition{adv.}{muito; grandemente; profundamente}
    \definition{s.}{profundidade}
  \seealsoref{浅}{qian3}
  \end{Phonetics}
\end{Entry}

\begin{Entry}{深入}{11,2}{⽔,⼊}
  \begin{Phonetics}{深入}{shen1ru4}[][HSK 3]
    \definition{adj.}{profundo; completo}
    \definition{v.}{ir fundo em; penetrar em; penetrar o exterior; alcançar o interior ou o centro de algo}
  \end{Phonetics}
\end{Entry}

\begin{Entry}{深入人心}{11,2,2,4}{⽔,⼊,⼈,⼼}
  \begin{Phonetics}{深入人心}{shen1ru4-ren2xin1}[][HSK 7-9]
    \definition{expr.}{``Profundamente enraizado nos corações das pessoas.''; criar raízes nos corações das pessoas; com profundidade e uma abordagem fácil de entender; penetrar profundamente no coração das pessoas; ter um impacto real nas pessoas}
  \synonymref{家喻户晓}{jia1yu4-hu4xiao3}
  \end{Phonetics}
\end{Entry}

\begin{Entry}{深切}{11,4}{⽔,⼑}
  \begin{Phonetics}{深切}{shen1qie4}[][HSK 7-9]
    \definition{adj.}{sincero; profundo; descreve um afeto muito profundo ou um relacionamento muito próximo | aguçado; penetrante; minucioso; descreve um nível profundo e genuíno de compreensão ou sentimento}
  \synonymref{深刻}{shen1ke4}
  \synonymref{深入}{shen1ru4}
  \synonymref{深深}{shen1shen1}
  \synonymref{真切}{zhen1qie4}
  \end{Phonetics}
\end{Entry}

\begin{Entry}{深化}{11,4}{⽔,⼔}
  \begin{Phonetics}{深化}{shen1hua4}[][HSK 6]
    \definition{v.}{aprofundar; avançar; intensificar; tornar"-se mais profundo; tornar mais profundo}
  \end{Phonetics}
\end{Entry}

\begin{Entry}{深处}{11,5}{⽔,⼡}
  \begin{Phonetics}{深处}{shen1chu4}[][HSK 5]
    \definition{s.}{profundidades; recantos; recessos | profundezas}
  \end{Phonetics}
\end{Entry}

\begin{Entry}{深远}{11,7}{⽔,⾡}
  \begin{Phonetics}{深远}{shen1yuan3}[][HSK 7-9]
    \definition{adj.}{de longo alcance; profundo e duradouro; descreve algo como tendo um impacto profundo ou um significado que perdura por muito tempo}
  \synonymref{长久}{chang2jiu3}
  \synonymref{长远}{chang2yuan3}
  \synonymref{深厚}{shen1hou4}
  \synonymref{深刻}{shen1ke4}
  \synonymref{深切}{shen1qie4}
  \synonymref{深入}{shen1ru4}
  \synonymref{永远}{yong3yuan3}
  \end{Phonetics}
\end{Entry}

\begin{Entry}{深刻}{11,8}{⽔,⼑}
  \begin{Phonetics}{深刻}{shen1ke4}[][HSK 3]
    \definition{adj.}{profundo; instenso; chegar à essência de um assunto ou problema}
  \end{Phonetics}
\end{Entry}

\begin{Entry}{深受}{11,8}{⽔,⼜}
  \begin{Phonetics}{深受}{shen1shou4}[][HSK 7-9]
    \definition{adv.}{profundamente (amado; apreciado; recebido com carinho)}
  \end{Phonetics}
\end{Entry}

\begin{Entry}{深夜}{11,8}{⽔,⼣}
  \begin{Phonetics}{深夜}{shen1ye4}[][HSK 7-9]
    \definition[个]{s.}{tarde da noite; depois da meia-noite; as primeiras horas da manhã}
  \synonymref{半夜}{ban4ye4}
  \synonymref{午夜}{wu3ye4}
  \antonymref{傍晚}{bang4wan3}
  \antonymref{清晨}{qing1chen2}
  \end{Phonetics}
\end{Entry}

\begin{Entry}{深信}{11,9}{⽔,⼈}
  \begin{Phonetics}{深信}{shen1xin4}[][HSK 7-9]
    \definition{v.}{estar profundamente convencido; acreditar firmemente; ter muita fé}
  \synonymref{坚信}{jian1xin4}
  \synonymref{确信}{que4xin4}
  \antonymref{怀疑}{huai2yi2}
  \end{Phonetics}
\end{Entry}

\begin{Entry}{深厚}{11,9}{⽔,⼚}
  \begin{Phonetics}{深厚}{shen1hou4}[][HSK 4]
    \definition{adj.}{profundo; sentimentos fortes | sólido; profundamente enraizado; fundação sólida}
  \end{Phonetics}
\end{Entry}

\begin{Entry}{深度}{11,9}{⽔,⼴}
  \begin{Phonetics}{深度}{shen1du4}[][HSK 5]
    \definition{adj.}{(em grau ou extensão) profundo; sério; grave}
    \definition{s.}{profundidade; grau de profundidade; | profundidade; rigor; meticulosidade; grau de contato com a essência das coisas | estágio avançado (ou em deterioração) de desenvolvimento; grau de crescimento e desenvolvimento das coisas}
  \end{Phonetics}
\end{Entry}

\begin{Entry}{深思}{11,9}{⽔,⼼}
  \begin{Phonetics}{深思}{shen1si1}[][HSK 7-9]
    \definition{v.}{ponderar; meditar; estar imerso em pensamentos; refletir profundamente sobre}
  \synonymref{沉思}{chen2si1}
  \synonymref{反思}{fan3si1}
  \end{Phonetics}
\end{Entry}

\begin{Entry}{深情}{11,11}{⽔,⼼}
  \begin{Phonetics}{深情}{shen1qing2}[][HSK 7-9]
    \definition{adj.}{afetuoso; bondoso}
    \definition{s.}{amor profundo; sentimento profundo; profundo afeto}
  \synonymref{情谊}{qing2yi4}
  \antonymref{仇恨}{chou2hen4}
  \end{Phonetics}
\end{Entry}

\begin{Entry}{深深}{11,11}{⽔,⽔}
  \begin{Phonetics}{深深}{shen1shen1}[][HSK 6]
    \definition{adj.}{profundo; intenso}
    \definition{adv.}{profundamente; intensamente; descreve um grau profundo ou forte}
  \end{Phonetics}
\end{Entry}

\begin{Entry}{深奥}{11,12}{⽔,⼤}
  \begin{Phonetics}{深奥}{shen1'ao4}[][HSK 7-9]
    \definition{adj.}{profundo; abstruso; (os princípios e significados) são profundos e difíceis de compreender.}
  \synonymref{深厚}{shen1hou4}
  \end{Phonetics}
\end{Entry}

%%%%%%%%%% 混 %%%%%%%%%%
\subsection*{混}\addcontentsline{loh}{figure}{混}

\begin{Entry}{混}{11}{⽔}
  \begin{Phonetics}{混}{hun2}
    \definition{adj.}{nublado; o mesmo que 浑, turvo | confuso; embaraçado; irracional}
    \variantof{浑}
  \end{Phonetics}
  \begin{Phonetics}{混}{hun4}[][HSK 6]
    \definition{adj.}{confuso; imundo; turvo; lamacento; impuro}
    \definition{adv.}{de forma imprudente; irresponsável; irrefletidamente}
    \definition{v.}{misturar; confundir; misturar verdadeiro e falso | passar por; esgueirar-se | vagar à deriva; arrastar-se; sobreviver de maneira superficial; contentar-se com | se dar bem com alguém}
  \end{Phonetics}
\end{Entry}

\begin{Entry}{混合}{11,6}{⽔,⼝}
  \begin{Phonetics}{混合}{hun4he2}[][HSK 6]
    \definition{s.}{híbrido; composto; refere"-se a duas ou mais substâncias misturadas sem reação química, mas ainda mantendo suas respectivas propriedades (diferente de 化合)}
    \definition{v.}{misturar; mixar; misturar"-se}
  \seealsoref{化合}{hua4he2}
  \end{Phonetics}
\end{Entry}

\begin{Entry}{混乱}{11,7}{⽔,⼄}
  \begin{Phonetics}{混乱}{hun4luan4}[][HSK 6]
    \definition{adj.}{caótico; confuso; desordenado; desorganizado; fora de ordem}
    \definition[片]{s.}{caos; confusão}
  \end{Phonetics}
\end{Entry}

\begin{Entry}{混饭}{11,7}{⽔,⾷}
  \begin{Phonetics}{混饭}{hun4/fan4}
    \definition{v.+compl.}{trabalhar para viver | Coloquial: se envolver em um trabalho apenas para ganhar a vida (sem ter nenhum interesse real nele) | comer às custas de outra pessoa}
  \end{Phonetics}
\end{Entry}

\begin{Entry}{混浊}{11,9}{⽔,⽔}
  \begin{Phonetics}{混浊}{hun4zhuo2}[][HSK 7-9]
    \definition{adj.}{lamacento; turvo; nublado | impuro; não transparente; não claro; turvo}
    \definition{s.}{nubécula; opacidade na córnea do olho}
  \end{Phonetics}
\end{Entry}

\begin{Entry}{混淆}{11,11}{⽔,⽔}
  \begin{Phonetics}{混淆}{hun4xiao2}[][HSK 7-9]
    \definition{v.}{misturar; confundir; colocar duas coisas muito parecidas juntas sem conseguir diferenciá-las}
  \end{Phonetics}
\end{Entry}

\begin{Entry}{混蛋}{11,11}{⽔,⾍}
  \begin{Phonetics}{混蛋}{hun4dan4}
    \definition{s.}{miserável; bastardo; canalha; patife; refere"-se a uma pessoa irracional (um insulto)}
  \synonymref{犯浑}{fan4/hun2}
  \synonymref{王八蛋}{wang2 ba1 dan4}
  \antonymref{绅士}{shen1shi4}
  \end{Phonetics}
\end{Entry}

\begin{Entry}{混凝土}{11,16,3}{⽔,⼎,⼟}
  \begin{Phonetics}{混凝土}{hun4ning2tu3}[][HSK 7-9]
    \definition{s.}{concreto; material de construção feito pela mistura de cimento, areia, cascalho e água em uma determinada proporção; após o endurecimento, apresenta propriedades como resistência à pressão, resistência à água e resistência ao fogo}
  \end{Phonetics}
\end{Entry}

%%%%%%%%%% 添 %%%%%%%%%%
\subsection*{添}\addcontentsline{loh}{figure}{添}

\begin{Entry}{添}{11}{⽔}
  \begin{Phonetics}{添}{tian1}[][HSK 6]
    \definition{v.}{adicionar; aumentar | dar à luz}
  \end{Phonetics}
\end{Entry}

\begin{Entry}{添加}{11,5}{⽔,⼒}
  \begin{Phonetics}{添加}{tian1jia1}[][HSK 7-9]
    \definition{v.}{adicionar à; aumentar; adicionar}
  \synonymref{补充}{bu3chong1}
  \synonymref{加上}{jia1shang5}
  \synonymref{增加}{zeng1jia1}
  \antonymref{扣除}{kou4chu2}
  \antonymref{去除}{qu4chu2}
  \antonymref{删除}{shan1chu2}
  \end{Phonetics}
\end{Entry}

%%%%%%%%%% 清 %%%%%%%%%%
\subsection*{清}\addcontentsline{loh}{figure}{清}

\begin{Entry}{清}{11}{⽔}
  \begin{Phonetics}{清}{qing1}[][HSK 6]
    \definition*{s.}{Dinastia Qing (1644--1911) | Sobrenome: Qing}
    \definition{adj.}{claro; não misturado; (líquido ou gasoso) puro e sem mistura | silencioso; quieto | justo e honesto | distinto; claro; esclarecido | simples; puro, sem qualquer adulteração ou combinação | limpo; puro}
    \definition{v.}{limpar; tornar limpo | resolver; esclarecer; pagar; liquidar | contar; inspecionar}
  \antonymref{浊}{zhuo2}
  \end{Phonetics}
\end{Entry}

\begin{Entry}{清彻}{11,7}{⽔,⼻}
  \begin{Phonetics}{清彻}{qing1che4}
    \variantof{清澈}
  \end{Phonetics}
\end{Entry}

\begin{Entry}{清单}{11,8}{⽔,⼗}
  \begin{Phonetics}{清单}{qing1dan1}[][HSK 7-9]
    \definition[张]{s.}{lista detalhada; relato detalhado; um catálogo; um inventário; formulário de inscrição detalhado para projetos relevantes}
  \end{Phonetics}
\end{Entry}

\begin{Entry}{清明}{11,8}{⽔,⽇}
  \begin{Phonetics}{清明}{qing1ming2}[][HSK 7-9]
    \definition*{s.}{Festival de Qingming (nos dias 4, 5 e 6 de abril); é costume limpar os túmulos nos dias 4, 5 ou 6 de abril, de acordo com a tradição popular}
    \definition{adj.}{limpo e em pé; (política) é legal e ordeira | calmo; sereno; (mental) claro e calmo | claro; brilhante}
  \end{Phonetics}
\end{Entry}

\begin{Entry}{清明节}{11,8,5}{⽔,⽇,⾋}
  \begin{Phonetics}{清明节}{qing1ming2jie2}[][HSK 6]
    \definition*{s.}{Qingming ou Festival do Brilho Puro ou Dia da Varredura de Túmulos, Dia dos Finados (uma das 24~divisões do ano solar no calendário lunar chinês:~dia~4 ou 5~de~abril solar)}
  \end{Phonetics}
\end{Entry}

\begin{Entry}{清洁}{11,9}{⽔,⽔}
  \begin{Phonetics}{清洁}{qing1jie2}[][HSK 6]
    \definition{adj.}{limpo; sem poeira, gordura, etc.}
    \definition{v.}{limpar}
  \end{Phonetics}
\end{Entry}

\begin{Entry}{清洁工}{11,9,3}{⽔,⽔,⼯}
  \begin{Phonetics}{清洁工}{qing1jie2 gong1}[][HSK 6]
    \definition{s.}{coletor de lixo; trabalhador de saneamento; limpador de rua; trabalhadores envolvidos na limpeza do ambiente, remoção de lixo e fezes, etc.}
  \end{Phonetics}
\end{Entry}

\begin{Entry}{清洗}{11,9}{⽔,⽔}
  \begin{Phonetics}{清洗}{qing1xi3}[][HSK 6]
    \definition{v.}{enxaguar; lavar; limpar | purgar; limpar | eliminar}
  \end{Phonetics}
\end{Entry}

\begin{Entry}{清除}{11,9}{⽔,⾩}
  \begin{Phonetics}{清除}{qing1chu2}[][HSK 7-9]
    \definition{v.}{eliminar; remover; livrar"-se de; remover completamente}
  \end{Phonetics}
\end{Entry}

\begin{Entry}{清凉}{11,10}{⽔,⼎}
  \begin{Phonetics}{清凉}{qing1liang2}[][HSK 7-9]
    \definition{adj.}{fresco e refrescante; agradavelmente fresco; fresco e agradável; refrescante e gelado}
  \end{Phonetics}
\end{Entry}

\begin{Entry}{清真寺}{11,10,6}{⽔,⼗,⼨}
  \begin{Phonetics}{清真寺}{qing1zhen1si4}[][HSK 7-9]
    \definition[所,座,个]{s.}{mesquita; as mesquitas islâmicas também são chamadas de salas de oração}
  \end{Phonetics}
\end{Entry}

\begin{Entry}{清脆}{11,10}{⽔,⾁}
  \begin{Phonetics}{清脆}{qing1cui4}[][HSK 7-9]
    \definition{adj.}{claro e melodioso; nítido e agradável ao ouvido, não abafado | (comida) crocante e refrescante}
  \end{Phonetics}
\end{Entry}

\begin{Entry}{清唱}{11,11}{⽔,⼝}
  \begin{Phonetics}{清唱}{qing1chang4}
    \definition{v.}{cantar à capela}
  \end{Phonetics}
\end{Entry}

\begin{Entry}{清晨}{11,11}{⽔,⽇}
  \begin{Phonetics}{清晨}{qing1chen2}[][HSK 5]
    \definition{s.}{matinal; manhã cedo; geralmente se refere ao período do amanhecer até logo após o nascer do sol}
  \end{Phonetics}
\end{Entry}

\begin{Entry}{清淡}{11,11}{⽔,⽔}
  \begin{Phonetics}{清淡}{qing1dan4}[][HSK 7-9]
    \definition{adj.}{leve; fraco; suave; delicado; (cor, cheiro) leve e suave; não forte | leve; não gorduroso ou com sabor forte; (alimento) com baixo teor de gordura | suave; simples; descreve uma vida ou ritmo de vida como simples e descomplicado}
  \end{Phonetics}
\end{Entry}

\begin{Entry}{清爽}{11,11}{⽔,⽘}
  \begin{Phonetics}{清爽}{qing1shuang3}
    \definition{adj.}{refrescante | relaxado}
  \end{Phonetics}
\end{Entry}

\begin{Entry}{清理}{11,11}{⽔,⽟}
  \begin{Phonetics}{清理}{qing1li3}[][HSK 5]
    \definition{v.}{esclarecer; resolver; verificar; colocar em ordem; organizar tudo e jogar fora o que não for útil}
  \end{Phonetics}
\end{Entry}

\begin{Entry}{清晰}{11,12}{⽔,⽇}
  \begin{Phonetics}{清晰}{qing1xi1}[][HSK 7-9]
    \definition{adj.}{claro; distinto; consegue ver e ouvir com clareza, compreender com clareza; tem um processo de pensamento claro}
  \end{Phonetics}
\end{Entry}

\begin{Entry}{清新}{11,13}{⽔,⽄}
  \begin{Phonetics}{清新}{qing1xin1}[][HSK 7-9]
    \definition{adj.}{fresco; puro e fresco; fresco e limpo; fresco e refrescante | (estilo) romance; original; inovador e único}
  \end{Phonetics}
\end{Entry}

\begin{Entry}{清楚}{11,13}{⽔,⽊}
  \begin{Phonetics}{清楚}{qing1chu5}[][HSK 2]
    \definition{adj.}{claro; distinto; compreensível; organizado; fácil de identificar e entender | plenamente consciente de; claro sobre}
    \definition{v.}{ter clareza sobre; compreender; ação que expressa compreensão e conhecimento}
  \end{Phonetics}
\end{Entry}

\begin{Entry}{清静}{11,14}{⽔,⾭}
  \begin{Phonetics}{清静}{qing1jing4}[][HSK 7-9]
    \definition{adj.}{(ambiente) calmo; pacífico; tranquilo; isolado; silencioso}
  \end{Phonetics}
\end{Entry}

\begin{Entry}{清澈}{11,15}{⽔,⽔}
  \begin{Phonetics}{清澈}{qing1che4}
    \definition{adj.}{claro | límpido}
  \end{Phonetics}
\end{Entry}

\begin{Entry}{清醒}{11,16}{⽔,⾣}
  \begin{Phonetics}{清醒}{qing1xing3}[][HSK 4]
    \definition{adj.}{sóbrio; lúcido}
    \definition{v.}{recuperar a consciência; recuperar"-se de um coma}
  \end{Phonetics}
\end{Entry}

%%%%%%%%%% 渐 %%%%%%%%%%
\subsection*{渐}\addcontentsline{loh}{figure}{渐}

\begin{Entry}{渐}{11}{⽔}
  \begin{Phonetics}{渐}{jian1}
    \definition{v.}{encharcar; ficar saturado com | fluir para}
  \end{Phonetics}
  \begin{Phonetics}{渐}{jian4}
    \definition{adv.}{gradualmente; por graus}
  \end{Phonetics}
\end{Entry}

\begin{Entry}{渐渐}{11,11}{⽔,⽔}
  \begin{Phonetics}{渐渐}{jian4jian4}[][HSK 4]
    \definition{adv.}{gradualmente; pouco a pouco; passo a passo; indica um aumento ou diminuição gradual em grau ou quantidade}
  \end{Phonetics}
\end{Entry}

%%%%%%%%%% 渔 %%%%%%%%%%
\subsection*{渔}\addcontentsline{loh}{figure}{渔}

\begin{Entry}{渔}{11}{⽔}
  \begin{Phonetics}{渔}{yu2}
    \definition[条]{s.}{pescador}
    \definition{v.}{pescar}
  \end{Phonetics}
\end{Entry}

\begin{Entry}{渔夫}{11,4}{⽔,⼤}
  \begin{Phonetics}{渔夫}{yu2fu1}
    \definition{s.}{pescador}
  \end{Phonetics}
\end{Entry}

\begin{Entry}{渔民}{11,5}{⽔,⽒}
  \begin{Phonetics}{渔民}{yu2min2}
    \definition{s.}{pescadores | povo pescador}
  \end{Phonetics}
\end{Entry}

\begin{Entry}{渔场}{11,6}{⽔,⼟}
  \begin{Phonetics}{渔场}{yu2chang3}
    \definition{s.}{área de pesca}
  \end{Phonetics}
\end{Entry}

\begin{Entry}{渔汛}{11,6}{⽔,⽔}
  \begin{Phonetics}{渔汛}{yu2xun4}
    \definition{s.}{temporada de pesca}
  \end{Phonetics}
\end{Entry}

\begin{Entry}{渔网}{11,6}{⽔,⽹}
  \begin{Phonetics}{渔网}{yu2wang3}
    \definition{s.}{rede de pesca | tresmalho}
  \end{Phonetics}
\end{Entry}

\begin{Entry}{渔具}{11,8}{⽔,⼋}
  \begin{Phonetics}{渔具}{yu2ju4}
    \definition{s.}{equipamento de pesca}
  \end{Phonetics}
\end{Entry}

\begin{Entry}{渔轮}{11,8}{⽔,⾞}
  \begin{Phonetics}{渔轮}{yu2lun2}
    \definition{s.}{navio de pesca}
  \end{Phonetics}
\end{Entry}

\begin{Entry}{渔捞}{11,10}{⽔,⼿}
  \begin{Phonetics}{渔捞}{yu2lao1}
    \definition{s.}{pesca (como atividade comercial); pesca em massa; operações de pesca em grande escala}
  \end{Phonetics}
\end{Entry}

\begin{Entry}{渔猎}{11,11}{⽔,⽝}
  \begin{Phonetics}{渔猎}{yu2lie4}
    \definition{s.}{pesca e caça}
    \definition{v.}{saquear | pilhar}
  \end{Phonetics}
\end{Entry}

\begin{Entry}{渔笼}{11,11}{⽔,⽵}
  \begin{Phonetics}{渔笼}{yu2long2}
    \definition{s.}{gaiola de pesca | armadilha de pesca}
  \end{Phonetics}
\end{Entry}

\begin{Entry}{渔船}{11,11}{⽔,⾈}
  \begin{Phonetics}{渔船}{yu2chuan2}
    \definition[条]{s.}{barco de pesca}
  \seealsoref{鱼船}{yu2chuan2}
  \end{Phonetics}
\end{Entry}

\begin{Entry}{渔船队}{11,11,4}{⽔,⾈,⾩}
  \begin{Phonetics}{渔船队}{yu2chuan2 dui4}
    \definition{s.}{frota pesqueira}
  \end{Phonetics}
\end{Entry}

%%%%%%%%%% 渗 %%%%%%%%%%
\subsection*{渗}\addcontentsline{loh}{figure}{渗}

\begin{Entry}{渗}{11}{⽔}
  \begin{Phonetics}{渗}{shen4}[][HSK 7-9]
    \definition{v.}{infiltrar; permear; infiltrar"-se em}
  \end{Phonetics}
\end{Entry}

\begin{Entry}{渗透}{11,10}{⽔,⾡}
  \begin{Phonetics}{渗透}{shen4tou4}[][HSK 7-9]
    \definition{s.}{osmose; dois gases ou dois líquidos miscíveis são misturados ao passarem por uma membrana porosa}
    \definition{v.}{infiltrar"-se; permear | infiltrar; essa metáfora descreve como algo ou alguma força entra gradualmente em outras áreas}
  \synonymref{分泌}{fen1mi4}
  \end{Phonetics}
\end{Entry}

%%%%%%%%%% 渠 %%%%%%%%%%
\subsection*{渠}\addcontentsline{loh}{figure}{渠}

\begin{Entry}{渠}{11}{⽊}
  \begin{Phonetics}{渠}{qu2}
    \definition*{s.}{Sobrenome: Qu}
    \definition{adj.}{Literário: grande}
    \definition{pron.}{Dialeto: ele; ela}
    \definition[条]{s.}{canal; vala; fosso; trincheira | borda externa da roda | escudo}
  \end{Phonetics}
\end{Entry}

\begin{Entry}{渠道}{11,12}{⽊,⾡}
  \begin{Phonetics}{渠道}{qu2dao4}[][HSK 6]
    \definition[条,个,种]{s.}{vala de irrigação; os cursos de água escavados pelos trabalhadores para drenagem e irrigação | maneira; meio; caminho}
  \end{Phonetics}
\end{Entry}

%%%%%%%%%% 烹 %%%%%%%%%%
\subsection*{烹}\addcontentsline{loh}{figure}{烹}

\begin{Entry}{烹}{11}{⽕}
  \begin{Phonetics}{烹}{peng1}
    \definition{v.}{ferver; cozinhar | saltear; refogar; fritar}
  \end{Phonetics}
\end{Entry}

\begin{Entry}{烹调}{11,10}{⽕,⾔}
  \begin{Phonetics}{烹调}{peng1tiao2}[][HSK 7-9]
    \definition{v.}{cozinhar; cozinhar e temperar}
  \end{Phonetics}
\end{Entry}

%%%%%%%%%% 焊 %%%%%%%%%%
\subsection*{焊}\addcontentsline{loh}{figure}{焊}

\begin{Entry}{焊}{11}{⽕}
  \begin{Phonetics}{焊}{han4}[][HSK 7-9]
    \definition{v.}{soldar; usar metal fundido para reparar objetos de metal ou conectar peças de metal}
  \end{Phonetics}
\end{Entry}

%%%%%%%%%% 焕 %%%%%%%%%%
\subsection*{焕}\addcontentsline{loh}{figure}{焕}

\begin{Entry}{焕}{11}{⽕}
  \begin{Phonetics}{焕}{huan4}
    \definition{adj.}{Literário: brilhante; reluzente; radiante}
  \end{Phonetics}
\end{Entry}

\begin{Entry}{焕发}{11,5}{⽕,⼜}
  \begin{Phonetics}{焕发}{huan4fa1}[][HSK 7-9]
    \definition{v.}{brilhar; revigorar; irradiar}
  \end{Phonetics}
\end{Entry}

%%%%%%%%%% 爽 %%%%%%%%%%
\subsection*{爽}\addcontentsline{loh}{figure}{爽}

\begin{Entry}{爽}{11}{⽘}
  \begin{Phonetics}{爽}{shuang3}[][HSK 6]
    \definition{adj.}{claro; nítido; brilhante | franco; de coração aberto; direto | relaxado; confortável}
    \definition{v.}{desviar; afastar | tornar confortável; ficar confortável}
  \end{Phonetics}
\end{Entry}

\begin{Entry}{爽快}{11,7}{⽘,⼼}
  \begin{Phonetics}{爽快}{shuang3kuai5}[][HSK 7-9]
    \definition{adj.}{relaxado; confortável; devido a fatores ambientais e climáticos, sinto"-me relaxado e confortável tanto física quanto mentalmente | direto; franco; isso descreve alguém que consegue tomar decisões rapidamente, seja falando ou agindo, sem hesitação ou indecisão}
  \synonymref{干脆}{gan1cui4}
  \synonymref{坦率}{tan3shuai4}
  \antonymref{迟疑}{chi2yi2}
  \antonymref{犹豫}{you2yu4}
  \end{Phonetics}
\end{Entry}

%%%%%%%%%% 猎 %%%%%%%%%%
\subsection*{猎}\addcontentsline{loh}{figure}{猎}

\begin{Entry}{猎}{11}{⽝}
  \begin{Phonetics}{猎}{lie4}
    \definition[个]{s.}{traje de caça}
    \definition{v.}{caçar | procurar; perseguir}
  \end{Phonetics}
\end{Entry}

\begin{Entry}{猎人}{11,2}{⽝,⼈}
  \begin{Phonetics}{猎人}{lie4ren2}[][HSK 7-9]
    \definition[个,位,名]{s.}{caçador; pessoas que ganham a vida caçando}
  \end{Phonetics}
\end{Entry}

\begin{Entry}{猎犬}{11,4}{⽝,⽝}
  \begin{Phonetics}{猎犬}{lie4quan3}[][HSK 7-9]
    \definition{s.}{cão de caça}
  \end{Phonetics}
\end{Entry}

\begin{Entry}{猎物}{11,8}{⽝,⽜}
  \begin{Phonetics}{猎物}{lie4wu4}
    \definition{s.}{presa (vítima de um predador)}
  \end{Phonetics}
\end{Entry}

%%%%%%%%%% 猖 %%%%%%%%%%
\subsection*{猖}\addcontentsline{loh}{figure}{猖}

\begin{Entry}{猖}{11}{⽝}
  \begin{Phonetics}{猖}{chang1}
    \definition{adj.}{louco; indisciplinado; dissoluto; licencioso; precipitado; imprudente | Literário: feroz}
  \end{Phonetics}
\end{Entry}

\begin{Entry}{猖狂}{11,7}{⽝,⽝}
  \begin{Phonetics}{猖狂}{chang1kuang2}[][HSK 7-9]
    \definition{adj.}{selvagem; desenfreado; furioso; imprudente; arrogante e presunçoso}
  \end{Phonetics}
\end{Entry}

%%%%%%%%%% 猛 %%%%%%%%%%
\subsection*{猛}\addcontentsline{loh}{figure}{猛}

\begin{Entry}{猛}{11}{⽝}
  \begin{Phonetics}{猛}{meng3}[][HSK 6]
    \definition*{s.}{Sobrenome: Meng}
    \definition{adj.}{feroz; violento | enérgico; vigoroso | valente}
    \definition{adv.}{de repente; abruptamente | vigorosamente; com força repentina | (coloquial) ao contentamento do coração; de todo o coração | ferozmente; violentamente}
  \end{Phonetics}
\end{Entry}

\begin{Entry}{猛烈}{11,10}{⽝,⽕}
  \begin{Phonetics}{猛烈}{meng3lie4}[][HSK 7-9]
    \definition{adj.}{feroz; violento; vigoroso}
  \end{Phonetics}
\end{Entry}

\begin{Entry}{猛然}{11,12}{⽝,⽕}
  \begin{Phonetics}{猛然}{meng3ran2}[][HSK 7-9]
    \definition{adv.}{de repente; abruptamente; indica ação repentina e rápida}
  \end{Phonetics}
\end{Entry}

%%%%%%%%%% 猜 %%%%%%%%%%
\subsection*{猜}\addcontentsline{loh}{figure}{猜}

\begin{Entry}{猜}{11}{⽝}
  \begin{Phonetics}{猜}{cai1}[][HSK 5]
    \definition{v.}{adivinhar; conjecturar; especular | suspeitar; ser cauteloso com os outros; desconfiar dos outros}
  \end{Phonetics}
\end{Entry}

\begin{Entry}{猜忌}{11,7}{⽝,⼼}
  \begin{Phonetics}{猜忌}{cai1 ji4}
    \definition{v.}{ser desconfiado e invejoso | ser desconfiado e ciumento de}
  \end{Phonetics}
\end{Entry}

\begin{Entry}{猜测}{11,9}{⽝,⽔}
  \begin{Phonetics}{猜测}{cai1ce4}[][HSK 5]
    \definition[个,种]{s.}{advinhação; conjectura; suposição; especulação}
    \definition{v.}{adivinhar; conjecturar; especular; estimar a partir da imaginação}
  \end{Phonetics}
\end{Entry}

\begin{Entry}{猜谜}{11,11}{⽝,⾔}
  \begin{Phonetics}{猜谜}{cai1 mi2}[][HSK 7-9]
    \definition{v.}{adivinhar um enigma}
  \end{Phonetics}
\end{Entry}

\begin{Entry}{猜想}{11,13}{⽝,⼼}
  \begin{Phonetics}{猜想}{cai1xiang3}[][HSK 7-9]
    \definition{s.}{suposição; conjectura; palpite; especulação}
    \definition{v.}{supor; adivinhar; suspeitar}
  \end{Phonetics}
\end{Entry}

\begin{Entry}{猜疑}{11,14}{⽝,⽦}
  \begin{Phonetics}{猜疑}{cai1 yi2}
    \definition{v.}{abrigar suspeitas; ser desconfiado; ter receios; levantar suspeitas do nada}
  \end{Phonetics}
\end{Entry}

%%%%%%%%%% 猝 %%%%%%%%%%
\subsection*{猝}\addcontentsline{loh}{figure}{猝}

\begin{Entry}{猝}{11}{⽝}
  \begin{Phonetics}{猝}{cu4}
    \definition{adj.}{Literário: repentino; abrupto; inesperado}
    \definition{adv.}{Literário: de repente; inesperadamente}
  \end{Phonetics}
\end{Entry}

%%%%%%%%%% 猪 %%%%%%%%%%
\subsection*{猪}\addcontentsline{loh}{figure}{猪}

\begin{Entry}{猪}{11}{⽝}
  \begin{Phonetics}{猪}{zhu1}[][HSK 3]
    \definition[头,只,口]{s.}{porco; suíno}
  \end{Phonetics}
\end{Entry}

\begin{Entry}{猪头}{11,5}{⽝,⼤}
  \begin{Phonetics}{猪头}{zhu1tou2}
    \definition{s.}{tolo | idiota}
  \end{Phonetics}
\end{Entry}

\begin{Entry}{猪柳}{11,9}{⽝,⽊}
  \begin{Phonetics}{猪柳}{zhu1liu3}
    \definition{s.}{filé de porco}
  \end{Phonetics}
\end{Entry}

\begin{Entry}{猪笼}{11,11}{⽝,⽵}
  \begin{Phonetics}{猪笼}{zhu1long2}
    \definition{s.}{estrutura cilíndrica de bambu ou arame usada para restringir um porco durante o transporte}
  \end{Phonetics}
\end{Entry}

\begin{Entry}{猪窠}{11,13}{⽝,⽳}
  \begin{Phonetics}{猪窠}{zhu1ke1}
    \definition{s.}{chiqueiro}
  \end{Phonetics}
\end{Entry}

%%%%%%%%%% 猫 %%%%%%%%%%
\subsection*{猫}\addcontentsline{loh}{figure}{猫}

\begin{Entry}{猫}{11}{⽝}
  \begin{Phonetics}{猫}{mao1}[][HSK 2]
    \definition*[只,种,群,窝,个]{s.}{gato |  Empréstimo linguístico: MODEM}
    \definition{v.}{esconder"-se; entrar em esconderijo | inclinar"-se para a frente; curvar"-se}
  \end{Phonetics}
  \begin{Phonetics}{猫}{mao2}
    \definition{v.}{utilizado em 猫腰 \dpy{mao2yao1}}
  \seealsoref{猫腰}{mao2yao1}
  \end{Phonetics}
\end{Entry}

\begin{Entry}{猫腰}{11,13}{⽝,⾁}
  \begin{Phonetics}{猫腰}{mao2yao1}
    \definition{v.}{curvar-se}
  \end{Phonetics}
\end{Entry}

\begin{Entry}{猫熊}{11,14}{⽝,⽕}
  \begin{Phonetics}{猫熊}{mao1xiong2}
    \definition[把,只]{s.}{panda gigante}
  \seealsoref{熊猫}{xiong2mao1}
  \end{Phonetics}
\end{Entry}

%%%%%%%%%% 率 %%%%%%%%%%
\subsection*{率}\addcontentsline{loh}{figure}{率}

\begin{Entry}{率}{11}{⽞}
  \begin{Phonetics}{率}{lv4}[][HSK 7-9]
    \definition{s.}{taxa; razão; proporção; a relação proporcional entre duas grandezas relacionadas}
  \end{Phonetics}
  \begin{Phonetics}{率}{shuai4}[][HSK 7-9]
    \definition*{s.}{Sobrenome: Shuai}
    \definition{adj.}{precipitado; não cuidadoso; não cauteloso | franco; direto | elegante; bonito; o mesmo que 帅}
    \definition{adv.}{geralmente; expressa uma estimativa incerta, equivalente a 大约 e 大抵}
    \definition{s.}{modelo; exemplo}
    \definition{v.}{liderar; comandar | obedecer; seguir}
  \seealsoref{大抵}{da4di3}
  \seealsoref{大约}{da4yue1}
  \seealsoref{帅}{shuai4}
  \end{Phonetics}
\end{Entry}

\begin{Entry}{率先}{11,6}{⽞,⼉}
  \begin{Phonetics}{率先}{shuai4xian1}[][HSK 4]
    \definition{v.}{tomar a iniciativa de fazer algo; ser o primeiro a fazer algo; assumir a liderança}
  \end{Phonetics}
\end{Entry}

\begin{Entry}{率领}{11,11}{⽞,⾴}
  \begin{Phonetics}{率领}{shuai4ling3}[][HSK 5]
    \definition{v.}{liderar (equipe ou grupo); chefiar; comandar}
  \end{Phonetics}
\end{Entry}

%%%%%%%%%% 球 %%%%%%%%%%
\subsection*{球}\addcontentsline{loh}{figure}{球}

\begin{Entry}{球}{11}{⽟}
  \begin{Phonetics}{球}{qiu2}[][HSK 1]
    \definition[个,颗,筐]{s.}{esfera; globo; equipamento de jogo antigo, objeto tridimensional circular, feito de couro, recheado com penas, para ser chutado com os pés ou batido com um bastão | qualquer coisa com formato de bola; algo esférico ou quase esférico | bola; refere"-se a certos artigos esportivos (geralmente redondos e tridimensionais) | jogo; partida; referência a esportes com bola | o Globo; a Terra; referindo"-se especificamente à Terra}
  \end{Phonetics}
\end{Entry}

\begin{Entry}{球队}{11,4}{⽟,⾩}
  \begin{Phonetics}{球队}{qiu2dui4}[][HSK 2]
    \definition[个,支]{s.}{equipe (basquete, futebol, etc.); equipe de atletas formada para competições esportivas com bola, como times de basquete, futebol, etc.}
  \end{Phonetics}
\end{Entry}

\begin{Entry}{球场}{11,6}{⽟,⼟}
  \begin{Phonetics}{球场}{qiu2chang3}[][HSK 2]
    \definition[个,座]{s.}{quadra; campo; terreno para jogos com bola; campos para a prática de esportes com bola, como basquete, futebol, tênis e vôlei, cuja forma, tamanho e equipamentos variam de acordo com as exigências de cada esporte}
  \end{Phonetics}
\end{Entry}

\begin{Entry}{球衣}{11,6}{⽟,⾐}
  \begin{Phonetics}{球衣}{qiu2yi1}
    \definition{s.}{uniforme, roupa (de uma equipe específica); camisa; camisa polo}
  \end{Phonetics}
\end{Entry}

\begin{Entry}{球员}{11,7}{⽟,⼝}
  \begin{Phonetics}{球员}{qiu2yuan2}[][HSK 6]
    \definition[名,位,个]{s.}{Esporte: jogador | membro do clube esportivo}
  \end{Phonetics}
\end{Entry}

\begin{Entry}{球拍}{11,8}{⽟,⼿}
  \begin{Phonetics}{球拍}{qiu2pai1}[][HSK 6]
    \definition[支]{s.}{(tênis, badminton, etc.) raquete}
  \end{Phonetics}
\end{Entry}

\begin{Entry}{球星}{11,9}{⽟,⽇}
  \begin{Phonetics}{球星}{qiu2xing1}[][HSK 6]
    \definition[位,名]{s.}{estrela do esporte (esporte com bola)}
  \end{Phonetics}
\end{Entry}

\begin{Entry}{球迷}{11,9}{⽟,⾡}
  \begin{Phonetics}{球迷}{qiu2mi2}[][HSK 3]
    \definition[个,位,名,些]{s.}{fã (de esportes de bola); pessoas obcecadas por jogar ou assistir jogos de bola}
  \end{Phonetics}
\end{Entry}

\begin{Entry}{球鞋}{11,15}{⽟,⾰}
  \begin{Phonetics}{球鞋}{qiu2xie2}[][HSK 2]
    \definition[双,只,款]{s.}{tênis de ginástica; tênis de tênis; tênis esportivos}
  \end{Phonetics}
\end{Entry}

%%%%%%%%%% 理 %%%%%%%%%%
\subsection*{理}\addcontentsline{loh}{figure}{理}

\begin{Entry}{理}{11}{⽟}
  \begin{Phonetics}{理}{li3}[][HSK 6]
    \definition*{s.}{Sobrenome: Li}
    \definition{s.}{textura; grão (em madeira, pele, etc.) | ordem; sequência | razão; lógica; verdade | ciências naturais (especialmente física)}
    \definition{v.}{gerenciar; executar | colocar em ordem; arrumar | (geralmente no negativo) prestar atenção a; fazer um gesto ou falar com | tratar | colocar em ordem; limpar | tomar conhecimento de; prestar atenção a; expressar uma atitude; expressar uma opinião}
  \end{Phonetics}
\end{Entry}

\begin{Entry}{理发}{11,5}{⽟,⼜}
  \begin{Phonetics}{理发}{li3/fa4}[][HSK 3]
    \definition{v.+compl.}{cortar e aparar o cabelo; ter (dar) um corte de cabelo}
  \end{Phonetics}
\end{Entry}

\begin{Entry}{理由}{11,5}{⽟,⽥}
  \begin{Phonetics}{理由}{li3you2}[][HSK 3]
    \definition[个,条,种,堆]{s.}{razão; justificativa; fundamento; a razão pela qual as coisas são feitas desta ou daquela maneira}
  \end{Phonetics}
\end{Entry}

\begin{Entry}{理会}{11,6}{⽟,⼈}
  \begin{Phonetics}{理会}{li3hui4}[][HSK 7-9]
    \definition{v.}{entender; compreender | prestar atenção em; observar; frequentemente usada em negação | cuidar de; lidar com | argumentar; debater; debater o certo e o errado; negociar (expressão encontrada principalmente no chinês vernáculo antigo)}
  \end{Phonetics}
\end{Entry}

\begin{Entry}{理论}{11,6}{⽟,⾔}
  \begin{Phonetics}{理论}{li3lun4}[][HSK 3]
    \definition[套,个]{s.}{teoria; uma série de conclusões tiradas pelas pessoas sobre atividades naturais ou sociais}
    \definition{v.}{argumentar; raciocinar com alguém; discutir com outras pessoas sobre quem está certo ou errado}
  \end{Phonetics}
\end{Entry}

\begin{Entry}{理财}{11,7}{⽟,⾙}
  \begin{Phonetics}{理财}{li3 cai2}[][HSK 6]
    \definition{v.}{administrar questões financeiras; conduzir transações financeiras; administrar propriedade; ser responsável pelo trabalho financeiro}
  \end{Phonetics}
\end{Entry}

\begin{Entry}{理事}{11,8}{⽟,⼅}
  \begin{Phonetics}{理事}{li3shi4}[][HSK 7-9]
    \definition[位]{s.}{membro de um conselho executivo ou de um conselho de administração; diretor; gerente | membro de um conselho; uma pessoa que representa um grupo no exercício de sua autoridade e na condução de assuntos}
    \definition{v.}{lidar com assuntos; administrar negócios}
  \end{Phonetics}
\end{Entry}

\begin{Entry}{理念}{11,8}{⽟,⼼}
  \begin{Phonetics}{理念}{li3nian4}[][HSK 7-9]
    \definition[个,套]{s.}{filosofia; ideia; pensamento; as ideias ou pontos de vista básicos sobre pessoas ou coisas}
  \end{Phonetics}
\end{Entry}

\begin{Entry}{理性}{11,8}{⽟,⼼}
  \begin{Phonetics}{理性}{li3xing4}[][HSK 7-9]
    \definition{adj.}{racional; isso se refere a atividades de pensamento abstrato, como conceitos, julgamentos e raciocínio}
    \definition{s.}{razão; faculdade racional; a capacidade de controlar o comportamento de forma racional}
  \antonymref{感性}{gan3xing4}
  \end{Phonetics}
\end{Entry}

\begin{Entry}{理所当然}{11,8,6,12}{⽟,⼾,⼹,⽕}
  \begin{Phonetics}{理所当然}{li3suo3dang1ran2}[][HSK 7-9]
    \definition{expr.}{é claro; naturalmente; sem dúvida; naturalmente; logicamente falando, deveria ser assim}
  \end{Phonetics}
\end{Entry}

\begin{Entry}{理直气壮}{11,8,4,6}{⽟,⽬,⽓,⼠}
  \begin{Phonetics}{理直气壮}{li3zhi2-qi4zhuang4}[][HSK 7-9]
    \definition{s.}{justo e autoconfiante; ousado e confiante, com a justiça ao seu lado; uma razão forte faz com que alguém fale com confiança}
  \end{Phonetics}
\end{Entry}

\begin{Entry}{理科}{11,9}{⽟,⽲}
  \begin{Phonetics}{理科}{li3ke1}[][HSK 7-9]
    \definition{s.}{ciência; departamento de ciências em uma faculdade; no contexto do ensino, é um termo genérico para disciplinas como física, química, matemática e biologia}
  \end{Phonetics}
\end{Entry}

\begin{Entry}{理智}{11,12}{⽟,⽇}
  \begin{Phonetics}{理智}{li3zhi4}[][HSK 6]
    \definition{adj.}{racional; sensato; cabeça fria; sóbrio; calmo}
    \definition{s.}{sentido; razão; intelecto; a capacidade de distinguir o certo do errado, analisar e julgar e controlar as emoções e o comportamento de acordo}
  \end{Phonetics}
\end{Entry}

\begin{Entry}{理想}{11,13}{⽟,⼼}
  \begin{Phonetics}{理想}{li3xiang3}[][HSK 2]
    \definition{adj.}{ideal; perfeito | conforme o desejado; satisfatório}
    \definition{adv.}{idealmente}
    \definition[个,种]{s.}{ideal; sonho; aspiração}
  \end{Phonetics}
\end{Entry}

\begin{Entry}{理睬}{11,13}{⽟,⽬}
  \begin{Phonetics}{理睬}{li3cai3}[][HSK 7-9]
    \definition{v.}{prestar atenção em; demonstrar interesse em; expressar uma atitude ou opinião sobre as palavras e ações de outras pessoas (geralmente usado em um sentido negativo)}
  \end{Phonetics}
\end{Entry}

\begin{Entry}{理解}{11,13}{⽟,⾓}
  \begin{Phonetics}{理解}{li3jie3}[][HSK 3]
    \definition{v.}{entender; compreender; compreender o significado por trás de algo através da reflexão e do aprendizado | entender com empatia; achar que os outros não conseguem fazer determinada coisa e demonstrar compaixão, perdão e não crítica}
  \end{Phonetics}
\end{Entry}

%%%%%%%%%% 甜 %%%%%%%%%%
\subsection*{甜}\addcontentsline{loh}{figure}{甜}

\begin{Entry}{甜}{11}{⽢}
  \begin{Phonetics}{甜}{tian2}[][HSK 3]
    \definition{adj.}{doce; melado | agradável; confortável; fazer as pessoas se sentirem confortáveis e felizes | (sono) profundo | feliz; descreve o sentimento de felicidade}
  \end{Phonetics}
\end{Entry}

\begin{Entry}{甜心}{11,4}{⽢,⼼}
  \begin{Phonetics}{甜心}{tian2xin1}
    \definition{s.}{querido}
  \end{Phonetics}
\end{Entry}

\begin{Entry}{甜头}{11,5}{⽢,⼤}
  \begin{Phonetics}{甜头}{tian2tou5}[][HSK 7-9]
    \definition{s.}{sabor doce; sabor agradável; (doçura) Um sabor levemente adocicado, geralmente referindo"-se a um sabor delicioso | bom; benefício (como incentivo); (doçura) benefícios; vantagens (frequentemente referindo"-se a algo tentador)}
  \synonymref{便宜}{bian4yi2}
  \synonymref{长处}{chang2chu4}
  \synonymref{好处}{hao3chu5}
  \synonymref{利益}{li4yi4}
  \synonymref{优点}{you1dian3}
  \end{Phonetics}
\end{Entry}

\begin{Entry}{甜玉米}{11,5,6}{⽢,⽟,⽶}
  \begin{Phonetics}{甜玉米}{tian2 yu4mi3}
    \definition{s.}{milho doce}
  \end{Phonetics}
\end{Entry}

\begin{Entry}{甜言}{11,7}{⽢,⾔}
  \begin{Phonetics}{甜言}{tian2yan2}
    \definition{s.}{boa conversa | palavras amáveis}
  \end{Phonetics}
\end{Entry}

\begin{Entry}{甜品}{11,9}{⽢,⼝}
  \begin{Phonetics}{甜品}{tian2pin3}
    \definition{s.}{sobremesa}
  \end{Phonetics}
\end{Entry}

\begin{Entry}{甜美}{11,9}{⽢,⽺}
  \begin{Phonetics}{甜美}{tian2mei3}[][HSK 7-9]
    \definition{adj.}{doce; melado; adocicado | agradável; refrescante; descreve uma sensação de prazer, conforto ou beleza}
  \synonymref{甜蜜}{tian2mi4}
  \antonymref{苦恼}{ku3nao3}
  \end{Phonetics}
\end{Entry}

\begin{Entry}{甜食}{11,9}{⽢,⾷}
  \begin{Phonetics}{甜食}{tian2shi2}
    \definition{s.}{doces | sobremesa}
  \end{Phonetics}
\end{Entry}

\begin{Entry}{甜酒}{11,10}{⽢,⾣}
  \begin{Phonetics}{甜酒}{tian2jiu3}
    \definition{s.}{licor doce}
  \end{Phonetics}
\end{Entry}

\begin{Entry}{甜甜圈}{11,11,11}{⽢,⽢,⼞}
  \begin{Phonetics}{甜甜圈}{tian2tian2quan1}
    \definition{s.}{rosquinha | \emph{doughnut}}
  \end{Phonetics}
\end{Entry}

\begin{Entry}{甜菊}{11,11}{⽢,⾋}
  \begin{Phonetics}{甜菊}{tian2ju2}
    \definition{s.}{estévia, arbusto cujas folhas produzem um substituto para o açúcar}
  \end{Phonetics}
\end{Entry}

\begin{Entry}{甜筒}{11,12}{⽢,⽵}
  \begin{Phonetics}{甜筒}{tian2tong3}
    \definition{s.}{sorvete de casquinha}
  \end{Phonetics}
\end{Entry}

\begin{Entry}{甜稚}{11,13}{⽢,⽲}
  \begin{Phonetics}{甜稚}{tian2zhi4}
    \definition{s.}{doce e inocente}
  \end{Phonetics}
\end{Entry}

\begin{Entry}{甜蜜}{11,14}{⽢,⾍}
  \begin{Phonetics}{甜蜜}{tian2mi4}[][HSK 7-9]
    \definition{adj.}{doce; feliz; descrevendo a sensação de felicidade, alegria e conforto}
  \seealsoref{甜}{tian2}
  \antonymref{痛苦}{tong4ku3}
  \end{Phonetics}
\end{Entry}

\begin{Entry}{甜酸}{11,14}{⽢,⾣}
  \begin{Phonetics}{甜酸}{tian2suan1}
    \definition{adj.}{agridoce}
  \end{Phonetics}
\end{Entry}

%%%%%%%%%% 略 %%%%%%%%%%
\subsection*{略}\addcontentsline{loh}{figure}{略}

\begin{Entry}{略}{11}{⽥}
  \begin{Phonetics}{略}{lve4}[][HSK 7-9]
    \definition*{s.}{Sobrenome: Lüe}
    \definition{adj.}{breve; superficial | esquemático; simples}
    \definition{adv.}{Literário: ligeiramente; só um pouquinho; um tanto}
    \definition{s.}{resumo; breve relato; esboço | estratégia; plano; esquema | currículo}
    \definition{v.}{omitir; apagar; deixar de fora | capturar; apreender | simplificar; omitir}
  \antonymref{详}{xiang2}
  \end{Phonetics}
\end{Entry}

\begin{Entry}{略微}{11,13}{⽥,⼻}
  \begin{Phonetics}{略微}{lve4wei1}[][HSK 7-9]
    \definition{adv.}{ligeiramente; brevemente; o grau não é elevado; a quantidade não é grande}
  \end{Phonetics}
\end{Entry}

%%%%%%%%%% 痕 %%%%%%%%%%
\subsection*{痕}\addcontentsline{loh}{figure}{痕}

\begin{Entry}{痕}{11}{⽧}
  \begin{Phonetics}{痕}{hen2}
    \definition[个]{s.}{marca; traço}
  \end{Phonetics}
\end{Entry}

\begin{Entry}{痕迹}{11,9}{⽧,⾡}
  \begin{Phonetics}{痕迹}{hen2ji4}[][HSK 7-9]
    \definition{s.}{marca; traço; a marca deixada por um objeto que entra em contato com outro | traço; rastro; vestígio; coisas ou fenômenos deixados por algo que já existiu}
  \end{Phonetics}
\end{Entry}

%%%%%%%%%% 盒 %%%%%%%%%%
\subsection*{盒}\addcontentsline{loh}{figure}{盒}

\begin{Entry}{盒}{11}{⽫}
  \begin{Phonetics}{盒}{he2}[][HSK 5]
    \definition{clas.}{caixa (de pequena dimensão)}
    \definition[个]{s.}{caixa; estojo; recipiente; receptáculo}
  \end{Phonetics}
\end{Entry}

\begin{Entry}{盒子}{11,3}{⽫,⼦}
  \begin{Phonetics}{盒子}{he2zi5}[][HSK 5]
    \definition[个,只,堆]{s.}{caixa; recipiente que têm tampas na parte superior e podem conter coisas dentro, geralmente é pequeno e plano}
  \end{Phonetics}
\end{Entry}

\begin{Entry}{盒饭}{11,7}{⽫,⾷}
  \begin{Phonetics}{盒饭}{he2fan4}[][HSK 5]
    \definition[份]{s.}{refeição embalada; marmita; \emph{fast-food} vendida em caixas}
  \end{Phonetics}
\end{Entry}

%%%%%%%%%% 盖 %%%%%%%%%%
\subsection*{盖}\addcontentsline{loh}{figure}{盖}

\begin{Entry}{盖}{11}{⽫}
  \begin{Phonetics}{盖}{gai4}[][HSK 4]
    \definition*{s.}{Sobrenome: Gai}
    \definition{adj.}{excelente; soberbo; fantástico}
    \definition{adv.}{cerca de; ao redor; aproximadamente; expressa um julgamento especulativo sobre algo, ou uma explicação da causa, o que é equivalente a 大概 ou 原来}
    \definition{conj.}{para; porque; dando continuidade à frase anterior, afirmando a razão ou causa, com tom incerto}
    \definition{s.}{tampa; capa; cobertura; algo que cobre ou sela a parte superior de um objeto | carapaça; concha (de tartaruga, caranguejo, etc.); ossos em formato de crânio em certas partes do corpo humano; as conchas nas costas de certos animais | dossel; capota; toldo | nivelador (uma ferramenta agrícola usada para nivelar terras)}
    \definition{v.}{cobrir; proteger; colocar uma capa em; colocar uma tampa em um objeto | selar; afixar um selo em | superar; sobressair; sobrepujar; ultrapassar | construir; colocar para cima | esconder; ocultar; encobrir | nivelar o terreno com um nivelador (ferramenta agrícola)}
  \seealsoref{大概}{da4gai4}
  \seealsoref{原来}{yuan2lai2}
  \end{Phonetics}
  \begin{Phonetics}{盖}{ge3}
    \definition*{s.}{Sobrenome: Ge}
  \end{Phonetics}
\end{Entry}

\begin{Entry}{盖子}{11,3}{⽫,⼦}
  \begin{Phonetics}{盖子}{gai4zi5}[][HSK 7-9]
    \definition[个]{s.}{tampa; cobertura; capa; topo; algo que tem um efeito de proteção na parte superior de um objeto | casco (de tartaruga, etc.); conchas nas costas dos animais}
  \end{Phonetics}
\end{Entry}

%%%%%%%%%% 盗 %%%%%%%%%%
\subsection*{盗}\addcontentsline{loh}{figure}{盗}

\begin{Entry}{盗}{11}{⽫}
  \begin{Phonetics}{盗}{dao4}[][HSK 7-9]
    \definition[个,伙,帮,窝]{s.}{ladrão; assaltante}
    \definition{v.}{roubar; saquear | usurpar; buscar ganho pessoal ou ganho por meios impróprios}
  \end{Phonetics}
\end{Entry}

\begin{Entry}{盗版}{11,8}{⽫,⽚}
  \begin{Phonetics}{盗版}{dao4ban3}[][HSK 6]
    \definition{s.}{cópia ilegal; cópia pirata; refere"-se a livros, periódicos e produtos audiovisuais pirateados (diferentes dos 正版)}
    \definition{v.}{piratear; copiar ou vender ilegalmente; para obter lucros enormes, reimprimir ou copiar livros, periódicos ou produtos audiovisuais em grandes quantidades sem o consentimento do detentor dos direitos autorais}
  \seealsoref{正版}{zheng4ban3}
  \end{Phonetics}
\end{Entry}

\begin{Entry}{盗窃}{11,9}{⽫,⽳}
  \begin{Phonetics}{盗窃}{dao4qie4}[][HSK 7-9]
    \definition{v.}{roubar; furtar; obter ilegalmente por meios secretos}
  \end{Phonetics}
\end{Entry}

%%%%%%%%%% 盘 %%%%%%%%%%
\subsection*{盘}\addcontentsline{loh}{figure}{盘}

\begin{Entry}{盘}{11}{⽫}
  \begin{Phonetics}{盘}{pan2}[][HSK 4,7-9]
    \definition*{s.}{Sobrenome: Pan}
    \definition{clas.}{usado para pratos, pedras de moer, etc. | usado para jogos de xadrez e de bola | usado para as coisas que estão entrelaçadas, emaranhadas}
    \definition[套,只]{s.}{bandeja; tabuleiro | recipiente plano e raso, como uma bandeja, prato, travessa etc.  | preço atual; cotação de mercado; refere"-se ao preço básico pelo qual as commodities são negociadas}
    \definition{v.}{enrolar; torcer; enrolar (para cima); entrelaçar; cercar | construir (assentando tijolos, pedras, etc.) | checar; examinar; interrogar; verificar um por um ou repetidamente (quantidade, situação, etc.) | transferir a propriedade de; passar para outra pessoa | carregar; transportar}
  \end{Phonetics}
\end{Entry}

\begin{Entry}{盘子}{11,3}{⽫,⼦}
  \begin{Phonetics}{盘子}{pan2zi5}[][HSK 4]
    \definition[个,叠,套,只]{s.}{prato; utensílio de fundo raso para guardar objetos, maior do que um pires, geralmente de formato redondo | situação de mercado; taxa de mercado; transação comercial}
  \end{Phonetics}
\end{Entry}

\begin{Entry}{盘算}{11,14}{⽫,⽵}
  \begin{Phonetics}{盘算}{pan2suan5}[][HSK 7-9]
    \definition{v.}{calcular; determinar; planejar | considerar e ponderar; premeditar; deliberar}
  \end{Phonetics}
\end{Entry}

%%%%%%%%%% 盛 %%%%%%%%%%
\subsection*{盛}\addcontentsline{loh}{figure}{盛}

\begin{Entry}{盛}{11}{⽫}
  \begin{Phonetics}{盛}{cheng2}[][HSK 7-9]
    \definition{v.}{encher; encher com uma concha; colocar as coisas em recipientes; especialmente colocar alimentos em tigelas, pratos e outros recipientes | segurar; conter; acomodar}
  \end{Phonetics}
  \begin{Phonetics}{盛}{sheng4}
    \definition*{s.}{Sobrenome: Sheng}
    \definition{adj.}{florescente; próspero | vigoroso; enérgico | grandioso; magnífico | abundante; profundo | popular; comum; difundido; universal | amplo; generoso; abundante; suficiente | ótimo}
    \definition{adv.}{muito; profundamente}
  \end{Phonetics}
\end{Entry}

\begin{Entry}{盛大}{11,3}{⽫,⼤}
  \begin{Phonetics}{盛大}{sheng4da4}[][HSK 7-9]
    \definition{adj.}{grandioso; magnífico; em grande escala; solene (atividade em grupo)}
  \synonymref{广大}{guang3da4}
  \synonymref{广泛}{guang3fan4}
  \synonymref{广阔}{guang3kuo4}
  \synonymref{隆重}{long2zhong4}
  \synonymref{无边}{wu2bian1}
  \antonymref{简陋}{jian3lou4}
  \end{Phonetics}
\end{Entry}

\begin{Entry}{盛开}{11,4}{⽫,⼶}
  \begin{Phonetics}{盛开}{sheng4kai1}[][HSK 7-9]
    \definition{v.}{estar em plena floração; (flores) desabrochar em abundância}
  \synonymref{开放}{kai1fang4}
  \synonymref{怒放}{nu4fang4}
  \end{Phonetics}
\end{Entry}

\begin{Entry}{盛气凌人}{11,4,10,2}{⽫,⽓,⼎,⼈}
  \begin{Phonetics}{盛气凌人}{sheng4qi4-ling2ren2}[][HSK 7-9]
    \definition{expr.}{dominador; arrogante; autoritário e prepotente; valentão arrogante; dominante; autoritário}
  \synonymref{目中无人}{mu4zhong1-wu2ren2}
  \end{Phonetics}
\end{Entry}

\begin{Entry}{盛会}{11,6}{⽫,⼈}
  \begin{Phonetics}{盛会}{sheng4hui4}[][HSK 7-9]
    \definition{s.}{grande evento; reunião ilustre; encontro importante}
  \synonymref{佳节}{jia1jie2}
  \end{Phonetics}
\end{Entry}

\begin{Entry}{盛行}{11,6}{⽫,⾏}
  \begin{Phonetics}{盛行}{sheng4xing2}[][HSK 6]
    \definition{v.}{predominar; estar atual; estar na moda; ser amplamente popular}
  \end{Phonetics}
\end{Entry}

\begin{Entry}{盛宴}{11,10}{⽫,⼧}
  \begin{Phonetics}{盛宴}{sheng4yan4}
    \definition{s.}{celebração}
  \end{Phonetics}
\end{Entry}

%%%%%%%%%% 眯 %%%%%%%%%%
\subsection*{眯}\addcontentsline{loh}{figure}{眯}

\begin{Entry}{眯}{11}{⽬}
  \begin{Phonetics}{眯}{mi1}
    \definition{v.}{estreitar os olhos | esmagar | (dialeto) tirar uma soneca}
  \end{Phonetics}
  \begin{Phonetics}{眯}{mi2}
    \definition{v.}{cegar (como com poeira)}
  \end{Phonetics}
\end{Entry}

%%%%%%%%%% 眼 %%%%%%%%%%
\subsection*{眼}\addcontentsline{loh}{figure}{眼}

\begin{Entry}{眼}{11}{⽬}
  \begin{Phonetics}{眼}{yan3}[][HSK 2]
    \definition{clas.}{usado para grandes coisas ocas: poços, fogões, panelas, etc.}
    \definition[双,只]{s.}{olho; o órgão visual dos humanos ou animais | abertura; pequeno furo; pequeno buraco | ponto"-chave; refere"-se ao ponto"-chave das coisas | armadilha; um termo do jogo Go que se refere a um espaço vazio cercado pelas peças de um jogador, onde o outro jogador não pode colocar uma peça, a menos que haja circunstâncias especiais | uma batida sem acento na música tradicional chinesa}
  \end{Phonetics}
\end{Entry}

\begin{Entry}{眼光}{11,6}{⽬,⼉}
  \begin{Phonetics}{眼光}{yan3guang1}[][HSK 5]
    \definition{s.}{olho; visão | visão; percepção; previsão; capacidade de observar e identificar coisas | vista; ponto de vista}
  \end{Phonetics}
\end{Entry}

\begin{Entry}{眼花缭乱}{11,7,15,7}{⽬,⾋,⽷,⼄}
  \begin{Phonetics}{眼花缭乱}{yan3hua1liao2luan4}
    \definition{v.}{ficar deslumbrado | deslumbrar}
  \end{Phonetics}
\end{Entry}

\begin{Entry}{眼证}{11,7}{⽬,⾔}
  \begin{Phonetics}{眼证}{yan3zheng4}
    \definition{s.}{testemunha ocular}
  \end{Phonetics}
\end{Entry}

\begin{Entry}{眼里}{11,7}{⽬,⾥}
  \begin{Phonetics}{眼里}{yan3li3}[][HSK 4]
    \definition{s.}{aos olhos de alguém; na opinião (ou visão) de alguém}
  \end{Phonetics}
\end{Entry}

\begin{Entry}{眼泪}{11,8}{⽬,⽔}
  \begin{Phonetics}{眼泪}{yan3lei4}[][HSK 4]
    \definition[滴,行]{s.}{lágrimas; termo genérico para lágrimas; fluido incolor e transparente secretado pelas glândulas lacrimais no olho, que serve para proteger o olho}
  \end{Phonetics}
\end{Entry}

\begin{Entry}{眼前}{11,9}{⽬,⼑}
  \begin{Phonetics}{眼前}{yan3qian2}[][HSK 3]
    \definition{adv.}{agora; (no) momento}
    \definition{s.}{diante dos olhos; diante de | agora; (no) momento}
  \end{Phonetics}
\end{Entry}

\begin{Entry}{眼柄}{11,9}{⽬,⽊}
  \begin{Phonetics}{眼柄}{yan3bing3}
    \definition{s.}{pedúnculo ocular (de crustáceo, etc.)}
  \end{Phonetics}
\end{Entry}

\begin{Entry}{眼看}{11,9}{⽬,⽬}
  \begin{Phonetics}{眼看}{yan3kan4}[][HSK 6]
    \definition{adv.}{em breve; em um momento; imediatamente}
    \definition{v.}{observar impotentemente; olhar passivamente; observar (o que está acontecendo)}
  \end{Phonetics}
\end{Entry}

\begin{Entry}{眼袋}{11,11}{⽬,⾐}
  \begin{Phonetics}{眼袋}{yan3dai4}
    \definition{s.}{inchaço sob os olhos}
  \end{Phonetics}
\end{Entry}

\begin{Entry}{眼睛}{11,13}{⽬,⽬}
  \begin{Phonetics}{眼睛}{yan3jing5}[][HSK 2]
    \definition[双,只]{s.}{olho(s)}
  \end{Phonetics}
\end{Entry}

\begin{Entry}{眼镜}{11,16}{⽬,⾦}
  \begin{Phonetics}{眼镜}{yan3jing4}[][HSK 4]
    \definition[副]{s.}{óculos; óculos de grau; lentes usadas nos olhos para melhorar a visão ou proteger os olhos, feitas de vidro ou cristal incolor ou colorido}
  \end{Phonetics}
\end{Entry}

%%%%%%%%%% 着 %%%%%%%%%%
\subsection*{着}\addcontentsline{loh}{figure}{着}

\begin{Entry}{着}{11}{⽬}
  \begin{Phonetics}{着}{zhao1}
    \definition{interj.}{``Tudo bem!''; ``Tudo certo!''; ``O.K.!''}
    \definition{s.}{uma jogada no xadrez | truque; meio; artifício; manobra; estratégia}
    \definition{v.}{colocar dentro; guardar}
  \end{Phonetics}
  \begin{Phonetics}{着}{zhao2}[][HSK 4]
    \definition{v.}{tocar (contato físico) | sentir; ser afetado por | queimar; acender | adormecer; cair no sono | acertar em cheio; ter sucesso em; usado após o verbo, indica que o objetivo foi alcançado ou que houve um resultado}
  \end{Phonetics}
  \begin{Phonetics}{着}{zhe5}[][HSK 1]
    \definition{part.}{adicionar a um verbo ou adjetivo para indicar uma ação ou estado contínuo | em frases que começam com uma palavra que indica um lugar, acrescente ao verbo para indicar um estado resultante | em frases imperativas, usado após verbos ou adjetivos para dar ênfase | adicionado após certos verbos, transforma-se em preposição}
    \definition{s.}{um movimento no xadrez | movimento; estratégia; estratagema}
  \end{Phonetics}
  \begin{Phonetics}{着}{zhuo2}
    \definition{v.}{vestir (roupas); vestir"-se | tocar; entrar em contato com; aproximar"-se de; (contato físico) | enviar; despachar | expressão usada em documentos oficiais antigos, indicando um tom de ordem | aplicar; usar; adicionar; anexar}
  \end{Phonetics}
\end{Entry}

\begin{Entry}{着手}{11,4}{⽬,⼿}
  \begin{Phonetics}{着手}{zhuo2shou3}
    \definition{v.}{colocar a mão nisso | estabelecer | começar uma tarefa}
  \end{Phonetics}
\end{Entry}

\begin{Entry}{着火}{11,4}{⽬,⽕}
  \begin{Phonetics}{着火}{zhao2/huo3}[][HSK 4]
    \definition{v.+compl.}{pegar fogo; estar em chamas}
  \end{Phonetics}
\end{Entry}

\begin{Entry}{着地}{11,6}{⽬,⼟}
  \begin{Phonetics}{着地}{zhao2di4}
    \definition{v.}{pousar | tocar o chão}
  \end{Phonetics}
\end{Entry}

\begin{Entry}{着花}{11,7}{⽬,⾋}
  \begin{Phonetics}{着花}{zhao2hua1}
    \definition{v.}{florescer}
  \end{Phonetics}
  \begin{Phonetics}{着花}{zhuo2hua1}
    \definition{s.}{floração}
    \definition{v.}{florescer}
  \end{Phonetics}
\end{Entry}

\begin{Entry}{着急}{11,9}{⽬,⼼}
  \begin{Phonetics}{着急}{zhao2/ji2}[][HSK 4]
    \definition{adj.}{ansioso; preocupado}
    \definition{s.}{preocupação; ansiedade}
    \definition{v.+compl.}{preocupar"-se; sentir"-se ansioso | sentir urgência; estar com pressa}
  \end{Phonetics}
\end{Entry}

\begin{Entry}{着凉}{11,10}{⽬,⼎}
  \begin{Phonetics}{着凉}{zhao2liang2}
    \definition{v.}{pegar um resfriado}
  \end{Phonetics}
\end{Entry}

\begin{Entry}{着眼}{11,11}{⽬,⽬}
  \begin{Phonetics}{着眼}{zhuo2yan3}
    \definition{v.}{ter seus olhos em (um objetivo) | ter algo em mente | concentrar-se}
  \end{Phonetics}
\end{Entry}

\begin{Entry}{着装}{11,12}{⽬,⾐}
  \begin{Phonetics}{着装}{zhuo2zhuang1}
    \definition{s.}{roupa | vestimenta}
    \definition{v.}{vestir}
  \end{Phonetics}
\end{Entry}

\begin{Entry}{着想}{11,13}{⽬,⼼}
  \begin{Phonetics}{着想}{zhuo2xiang3}
    \definition{v.}{considerar (as necessidades de outras pessoas) | pensar (para os outros)}
  \end{Phonetics}
\end{Entry}

\begin{Entry}{着数}{11,13}{⽬,⽁}
  \begin{Phonetics}{着数}{zhao1shu4}
    \definition{s.}{estratégia | movimento (no xadrez, no palco, nas artes marciais) | esquema | truque}
  \end{Phonetics}
\end{Entry}

%%%%%%%%%% 矫 %%%%%%%%%%
\subsection*{矫}\addcontentsline{loh}{figure}{矫}

\begin{Entry}{矫}{11}{⽮}
  \begin{Phonetics}{矫}{jiao2}
    \definition{s.}{usado em 矫情}
  \seealsoref{矫情}{jiao2qing5}
  \end{Phonetics}
  \begin{Phonetics}{矫}{jiao3}
    \definition*{s.}{Sobrenome: Jiao}
    \definition{adj.}{forte; corajoso}
    \definition{v.}{retificar; corrigir; resolver | fingir; simular; dissimular}
  \end{Phonetics}
\end{Entry}

\begin{Entry}{矫正}{11,5}{⽮,⽌}
  \begin{Phonetics}{矫正}{jiao3zheng4}[][HSK 7-9]
    \definition{v.}{corrigir; retificar}
  \end{Phonetics}
\end{Entry}

\begin{Entry}{矫情}{11,11}{⽮,⼼}
  \begin{Phonetics}{矫情}{jiao2qing2}
    \definition{v.}{ser afetadamente não convencional; fingir ser incomum; ir deliberadamente contra o senso comum para demonstrar superioridade ou ser diferente dos outros}
  \end{Phonetics}
  \begin{Phonetics}{矫情}{jiao2qing5}
    \definition{adj.}{briguento; contencioso; irracional; isso se refere a apresentar argumentos descabidos e causar problemas.}
  \end{Phonetics}
\end{Entry}

%%%%%%%%%% 硕 %%%%%%%%%%
\subsection*{硕}\addcontentsline{loh}{figure}{硕}

\begin{Entry}{硕}{11}{⽯}
  \begin{Phonetics}{硕}{shuo4}
    \definition*{s.}{Sobrenome: Shuo}
    \definition{adj.}{grande; enorme}
    \definition{s.}{mestrado (MBA)}
  \end{Phonetics}
\end{Entry}

\begin{Entry}{硕士}{11,3}{⽯,⼠}
  \begin{Phonetics}{硕士}{shuo4shi4}[][HSK 5]
    \definition[个,位,名]{s.}{mestrado; um diploma concedido por uma universidade ou faculdade a um aluno após um ou dois anos de estudo adicional após o bacharelado}
  \end{Phonetics}
\end{Entry}

\begin{Entry}{硕果}{11,8}{⽯,⽊}
  \begin{Phonetics}{硕果}{shuo4guo3}[][HSK 7-9]
    \definition{s.}{resultados frutíferos; frutos grandes e maduros | ótimo trabalho | grande conquista | sucesso triunfante}
  \synonymref{成果}{cheng2guo3}
  \synonymref{果实}{guo3shi2}
  \synonymref{收获}{shou1huo4}
  \end{Phonetics}
\end{Entry}

%%%%%%%%%% 票 %%%%%%%%%%
\subsection*{票}\addcontentsline{loh}{figure}{票}

\begin{Entry}{票}{11}{⽰}
  \begin{Phonetics}{票}{piao4}[][HSK 1]
    \definition{clas.}{para grupos, lotes, transações comerciais}
    \definition[张]{s.}{bilhete; passagem; ingresso | cédula | nota bancária; conta | pessoa mantida em cativeiro por sequestradores para obter resgate; refém | apresentação amadora (de ópera de Pequim, etc.); peças teatrais amadoras}
    \definition{v.}{atuar como amador (na ópera de Pequim)}
  \end{Phonetics}
\end{Entry}

\begin{Entry}{票价}{11,6}{⽰,⼈}
  \begin{Phonetics}{票价}{piao4jia4}[][HSK 3]
    \definition[个]{s.}{o preço de um ingresso; taxa de admissão; taxa de entrada}
  \end{Phonetics}
\end{Entry}

\begin{Entry}{票房}{11,8}{⽰,⼾}
  \begin{Phonetics}{票房}{piao4fang2}[][HSK 7-9]
    \definition{s.}{Coloquial: bilheteria (em uma estação ferroviária, aeroporto, etc.); bilheteria (em um teatro, estádio, etc.) | valor de bilheteria; receita de bilheteria | Obsoleto: clube para artistas amadores de ópera de Pequim, etc.}
  \seealsoref{票房儿}{piao4fang2r5}
  \end{Phonetics}
\end{Entry}

\begin{Entry}{票房儿}{11,8,2}{⽰,⼾,⼉}
  \begin{Phonetics}{票房儿}{piao4fang2r5}
    \definition{s.}{bilheteria}
  \end{Phonetics}
\end{Entry}

%%%%%%%%%% 祭 %%%%%%%%%%
\subsection*{祭}\addcontentsline{loh}{figure}{祭}

\begin{Entry}{祭}{11}{⽰}
  \begin{Phonetics}{祭}{ji4}[][HSK 7-9]
    \definition{v.}{oferecer um sacrifício a | realizar uma cerimônia memorial para | empunhar; usar (arma mágica)}
  \end{Phonetics}
  \begin{Phonetics}{祭}{zhai4}
    \definition*{s.}{Sobrenome: Zhai}
  \end{Phonetics}
\end{Entry}

\begin{Entry}{祭祀}{11,7}{⽰,⽰}
  \begin{Phonetics}{祭祀}{ji4si4}[][HSK 7-9]
    \definition{v.}{oferecer sacrifícios aos deuses ou ancestrais; o antigo costume é preparar oferendas aos deuses, Budas ou ancestrais para mostrar respeito e buscar bênçãos; acreditar em (religião)}
  \end{Phonetics}
\end{Entry}

\begin{Entry}{祭奠}{11,12}{⽰,⼤}
  \begin{Phonetics}{祭奠}{ji4dian4}[][HSK 7-9]
    \definition{v.}{realizar uma cerimônia memorial para; lembrar e mostrar respeito por (os mortos) | realizar ou comparecer a um serviço memorial |  oferecer sacrifícios (aos ancestrais)}
  \end{Phonetics}
\end{Entry}

%%%%%%%%%% 祸 %%%%%%%%%%
\subsection*{祸}\addcontentsline{loh}{figure}{祸}

\begin{Entry}{祸}{11}{⽰}
  \begin{Phonetics}{祸}{huo4}
    \definition[场]{s.}{infortúnio; desastre; calamidade | desgraça; catástrofe}
    \definition{v.}{trazer desastre; arruinar | causar problemas}
  \antonymref{福}{fu2}
  \end{Phonetics}
\end{Entry}

\begin{Entry}{祸害}{11,10}{⽰,⼧}
  \begin{Phonetics}{祸害}{huo4hai5}[][HSK 7-9]
    \definition[个]{s.}{desastre; calamidade | maldição; flagelo}
    \definition{v.}{causar desastre; danificar; destruir; arruinar}
  \end{Phonetics}
\end{Entry}

%%%%%%%%%% 移 %%%%%%%%%%
\subsection*{移}\addcontentsline{loh}{figure}{移}

\begin{Entry}{移}{11}{⽲}
  \begin{Phonetics}{移}{yi2}[][HSK 4]
    \definition*{s.}{Sobrenome: Yi}
    \definition{v.}{mover; remover; deslocar; mudar | mudar; alterar}
  \end{Phonetics}
\end{Entry}

\begin{Entry}{移民}{11,5}{⽲,⽒}
  \begin{Phonetics}{移民}{yi2min2}[][HSK 4]
    \definition[个,批]{s.}{emigrante; migrantes; aqueles que se mudam para um país ou estado estrangeiro para se estabelecer}
    \definition{v.}{migrar; imigrar}
  \end{Phonetics}
\end{Entry}

\begin{Entry}{移动}{11,6}{⽲,⼒}
  \begin{Phonetics}{移动}{yi2dong4}[][HSK 4]
    \definition{v.}{deslocar; mover; mudar}
  \end{Phonetics}
\end{Entry}

%%%%%%%%%% 竟 %%%%%%%%%%
\subsection*{竟}\addcontentsline{loh}{figure}{竟}

\begin{Entry}{竟}{11}{⾳}
  \begin{Phonetics}{竟}{jing4}[][HSK 7-9]
    \definition{adj.}{todo; por toda parte; do começo ao fim}
    \definition{adv.}{no final; eventualmente | na verdade; inesperadamente; significa algo inesperado, equivalente a 居然}
    \definition{v.}{terminar; completar | investigar}
  \seealsoref{居然}{ju1ran2}
  \end{Phonetics}
\end{Entry}

\begin{Entry}{竟敢}{11,11}{⾳,⽁}
  \begin{Phonetics}{竟敢}{jing4gan3}[][HSK 7-9]
    \definition{v.}{ousar de fato; ter a audácia; ter a impertinência | ousar}
  \end{Phonetics}
\end{Entry}

\begin{Entry}{竟然}{11,12}{⾳,⽕}
  \begin{Phonetics}{竟然}{jing4ran2}[][HSK 4]
    \definition{adv.}{de fato; inesperadamente; para surpresa de alguém; chegar ao ponto de; indica que algo é um pouco inesperado}
  \end{Phonetics}
\end{Entry}

%%%%%%%%%% 章 %%%%%%%%%%
\subsection*{章}\addcontentsline{loh}{figure}{章}

\begin{Entry}{章}{11}{⾳}
  \begin{Phonetics}{章}{zhang1}[][HSK 6]
    \definition*{s.}{Sobrenome: Zhang}
    \definition[枚,个,方]{s.}{(para um livro, carta, etc.) capítulo; seção | ordem | regras; regulamentos; constituição | item; cláusula | (arcaico) memorial ao imperador; memorial ao trono | (arcaico) figura; padrão decorativo | selo; carimbo | distintivo; insígnia; medalha | verso; trecho do poema | escrita literária}
  \end{Phonetics}
\end{Entry}

\begin{Entry}{章鱼}{11,8}{⾳,⿂}
  \begin{Phonetics}{章鱼}{zhang1yu2}
    \definition{s.}{polvo | octópode}
  \end{Phonetics}
\end{Entry}

%%%%%%%%%% 笛 %%%%%%%%%%
\subsection*{笛}\addcontentsline{loh}{figure}{笛}

\begin{Entry}{笛}{11}{⽵}
  \begin{Phonetics}{笛}{di2}
    \definition[只]{s.}{flauta de bambu | sirene; apito; buzina}
  \end{Phonetics}
\end{Entry}

\begin{Entry}{笛子}{11,3}{⽵,⼦}
  \begin{Phonetics}{笛子}{di2zi5}[][HSK 7-9]
    \definition[管]{s.}{flauta; flauta de bambu; um instrumento de sopro transversal feito de bambu ou metal com seis furos de tom dispostos em uma fileira de acordo com o tom}
  \end{Phonetics}
\end{Entry}

%%%%%%%%%% 符 %%%%%%%%%%
\subsection*{符}\addcontentsline{loh}{figure}{符}

\begin{Entry}{符}{11}{⽵}
  \begin{Phonetics}{符}{fu2}
    \definition*{s.}{Sobrenome: Fu}
    \definition[个]{s.}{registro emitido por um governante para generais, enviados, etc., como credenciais na China antiga | símbolo; emblema | figuras mágicas desenhadas por sacerdotes taoístas para invocar ou expulsar espíritos e trazer boa ou má sorte | marca; sinal}
    \definition{v.}{(usado com 相 xiāng ou 不) coincidir com; concordar com | encaixar bem; combinar com; em conformidade com}
  \seealsoref{不}{bu4}
  \seealsoref{相}{xiang1}
  \end{Phonetics}
\end{Entry}

\begin{Entry}{符号}{11,5}{⽵,⼝}
  \begin{Phonetics}{符号}{fu2hao4}[][HSK 4]
    \definition[个]{s.}{marca; símbolo; sinais que marcam as coisas | insígnia; emblema; um símbolo usado no corpo para indicar posição, \emph{status}, etc.}
  \end{Phonetics}
\end{Entry}

\begin{Entry}{符合}{11,6}{⽵,⼝}
  \begin{Phonetics}{符合}{fu2he2}[][HSK 4]
    \definition{v.}{conformar"-se com, estar de acordo com, estar em conformidade com}
  \end{Phonetics}
\end{Entry}

%%%%%%%%%% 笨 %%%%%%%%%%
\subsection*{笨}\addcontentsline{loh}{figure}{笨}

\begin{Entry}{笨}{11}{⽵}
  \begin{Phonetics}{笨}{ben4}[][HSK 4]
    \definition{adj.}{estúpido; sem graça; tolo; de pouca habilidade; sem inteligência | desajeitado; tosco; inflexível | incômodo; pesado; desajeitado; difícil de manejar; trabalhoso}
  \end{Phonetics}
\end{Entry}

\begin{Entry}{笨重}{11,9}{⽵,⾥}
  \begin{Phonetics}{笨重}{ben4zhong4}[][HSK 7-9]
    \definition{adj.}{pesado; desajeitado; incômodo; grande e pesado, inconveniente de usar | pesado; difícil de manejar; pesado e trabalhoso}
  \end{Phonetics}
\end{Entry}

\begin{Entry}{笨蛋}{11,11}{⽵,⾍}
  \begin{Phonetics}{笨蛋}{ben4dan4}[][HSK 7-9]
    \definition[个]{s.}{tolo; idiota; (depreciativo) refere"-se a uma pessoa muito estúpida ou sem cérebro; geralmente usado para insultar pessoas}
  \end{Phonetics}
\end{Entry}

%%%%%%%%%% 第 %%%%%%%%%%
\subsection*{第}\addcontentsline{loh}{figure}{第}

\begin{Entry}{第}{11}{⽵}
  \begin{Phonetics}{第}{di4}[][HSK 1]
    \definition*{s.}{Sobrenome: Di}
    \definition{adv.}{mas, apenas, somente; Indica que a ação não está sujeita a restrições ou condições; equivalente a 只管}
    \definition{conj.}{mas; contudo; orações de conexão; indicando uma relação de transição; equivalente a 但是}
    \definition{pref.}{palavra auxiliar para números ordinais; usado antes de números inteiros, indica ordem}
    \definition{s.}{diferentes notas dos candidatos aprovados nos exames imperiais | a residência de um alto funcionário; grandes residências dos burocratas da era feudal}
  \seealsoref{但是}{dan4shi4}
  \seealsoref{只管}{zhi3guan3}
  \end{Phonetics}
\end{Entry}

\begin{Entry}{第一手}{11,1,4}{⽵,⼀,⼿}
  \begin{Phonetics}{第一手}{di4yi1shou3}[][HSK 7-9]
    \definition{s.}{em primeira mão; obtido por meio de prática e investigação pessoal; obtido diretamente}
  \end{Phonetics}
\end{Entry}

\begin{Entry}{第一线}{11,1,8}{⽵,⼀,⽷}
  \begin{Phonetics}{第一线}{di4yi1xian4}[][HSK 7-9]
    \definition{s.}{vanguarda; linha de frente; primeira linha | frente (linha), primeira linha; a linha de frente do campo de batalha também se refere ao local onde um determinado trabalho é realizado diretamente}
  \end{Phonetics}
\end{Entry}

%%%%%%%%%% 笼 %%%%%%%%%%
\subsection*{笼}\addcontentsline{loh}{figure}{笼}

\begin{Entry}{笼}{11}{⽵}
  \begin{Phonetics}{笼}{long2}
    \definition{s.}{armação fechada de bambu, arame, etc. | jaula | gaiola}
  \end{Phonetics}
  \begin{Phonetics}{笼}{long3}
    \definition{v.}{envolver | cobrir}
  \end{Phonetics}
\end{Entry}

\begin{Entry}{笼子}{11,3}{⽵,⼦}
  \begin{Phonetics}{笼子}{long2zi5}[][HSK 7-9]
    \definition[只,个]{s.}{gaiola; cesta; jaula; recipiente; utensílios feitos de tiras de bambu, tiras de madeira, galhos ou arame, usados para criar insetos e pássaros ou para armazenar coisas}
  \end{Phonetics}
  \begin{Phonetics}{笼子}{long3zi5}
    \definition[只,个]{s.}{caixa; baú; uma caixa relativamente grande}
  \end{Phonetics}
\end{Entry}

\begin{Entry}{笼统}{11,9}{⽵,⽷}
  \begin{Phonetics}{笼统}{long3tong3}[][HSK 7-9]
    \definition{adj.}{geral; abrangente; que falta análise específica; pouco claro; ambíguo}
  \end{Phonetics}
\end{Entry}

\begin{Entry}{笼罩}{11,13}{⽵,⽹}
  \begin{Phonetics}{笼罩}{long3zhao4}[][HSK 7-9]
    \definition{v.}{envolver; cobrir}
  \end{Phonetics}
\end{Entry}

%%%%%%%%%% 粒 %%%%%%%%%%
\subsection*{粒}\addcontentsline{loh}{figure}{粒}

\begin{Entry}{粒}{11}{⽶}
  \begin{Phonetics}{粒}{li4}[][HSK 7-9]
    \definition{clas.}{utilizado para materiais granulares, grânulos}
    \definition{s.}{partículas pequenas; grão; grânulo; \emph{pellet}}
  \seealsoref{粒儿}{li4r5}
  \end{Phonetics}
\end{Entry}

\begin{Entry}{粒儿}{11,2}{⽶,⼉}
  \begin{Phonetics}{粒儿}{li4r5}
    \definition{s.}{grão}
  \seealsoref{粒}{li4}
  \end{Phonetics}
\end{Entry}

%%%%%%%%%% 粗 %%%%%%%%%%
\subsection*{粗}\addcontentsline{loh}{figure}{粗}

\begin{Entry}{粗}{11}{⽶}
  \begin{Phonetics}{粗}{cu1}[][HSK 4]
    \definition{adj.}{largo (em diâmetro); grosso | grosseiro; rude; áspero | áspero; rouco | descuidado; negligente | rude; sem refinamento; vulgar}
    \definition{adv.}{grosseiramente; vagamente}
  \end{Phonetics}
\end{Entry}

\begin{Entry}{粗心}{11,4}{⽶,⼼}
  \begin{Phonetics}{粗心}{cu1xin1}[][HSK 4]
    \definition{adj.}{descuidado; irrefletido; (fazer as coisas) de forma desleixada, sem cuidado}
  \end{Phonetics}
\end{Entry}

\begin{Entry}{粗心大意}{11,4,3,13}{⽶,⼼,⼤,⼼}
  \begin{Phonetics}{粗心大意}{cu1xin1-da4yi4}[][HSK 7-9]
    \definition{expr.}{ser negligente; descuidado; inadvertido; desmiolado; descuidado e negligente; negligente; remisso; refere"-se a fazer as coisas de forma descuidada}
  \end{Phonetics}
\end{Entry}

\begin{Entry}{粗心地做}{11,4,6,11}{⽶,⼼,⼟,⼈}
  \begin{Phonetics}{粗心地做}{cu1xin1 di4 zuo4}
    \definition{adj.}{feito descuidadamente}
  \end{Phonetics}
\end{Entry}

\begin{Entry}{粗略}{11,11}{⽶,⽥}
  \begin{Phonetics}{粗略}{cu1lve4}[][HSK 7-9]
    \definition{adj.}{grosseiro; rudimentar; superficial}
  \end{Phonetics}
\end{Entry}

\begin{Entry}{粗鲁}{11,12}{⽶,⿂}
  \begin{Phonetics}{粗鲁}{cu1lu3}[][HSK 7-9]
    \definition{adj.}{rude; grosseiro; incivilizado}
  \end{Phonetics}
\end{Entry}

\begin{Entry}{粗暴}{11,15}{⽶,⽇}
  \begin{Phonetics}{粗暴}{cu1bao4}[][HSK 7-9]
    \definition{adj.}{rude; áspero; bruto; brutal; violento}
  \end{Phonetics}
\end{Entry}

\begin{Entry}{粗糙}{11,16}{⽶,⽶}
  \begin{Phonetics}{粗糙}{cu1cao1}[][HSK 7-9]
    \definition{adj.}{áspero; grosseiro; não é liso; não é redondo; não é fino | desleixado; descuidado; não meticuloso}
  \end{Phonetics}
\end{Entry}

%%%%%%%%%% 粘 %%%%%%%%%%
\subsection*{粘}\addcontentsline{loh}{figure}{粘}

\begin{Entry}{粘}{11}{⽶}
  \begin{Phonetics}{粘}{nian2}
    \variantof{黏}
  \end{Phonetics}
\end{Entry}

%%%%%%%%%% 累 %%%%%%%%%%
\subsection*{累}\addcontentsline{loh}{figure}{累}

\begin{Entry}{累}{11}{⽷}
  \begin{Phonetics}{累}{lei2}
    \definition*{s.}{Sobrenome: Lei}
    \definition{adj.}{incômodo; complicado}
    \definition{s.}{corda; cordão | touro na época de acasalamento}
    \definition{v.}{amarrar; prender; atar | copular}
  \end{Phonetics}
  \begin{Phonetics}{累}{lei3}
    \definition*{s.}{Sobrenome: Lei}
    \definition{adj.}{em andamento; repetido; contínuo}
    \definition{v.}{acumular; empilhar; colocar em cima de outro | envolver; implicar | construir empilhando tijolos, pedras, terra, etc.}
  \end{Phonetics}
  \begin{Phonetics}{累}{lei4}[][HSK 1]
    \definition{adj.}{cansado; exausto; fatigado}
    \definition{v.}{cansar; desgastar; fatigar; esgotar | labutar; trabalhar duro}
  \end{Phonetics}
\end{Entry}

\begin{Entry}{累计}{11,4}{⽷,⾔}
  \begin{Phonetics}{累计}{lei3ji4}[][HSK 7-9]
    \definition{v.}{somar; calcular; totalizar}
  \end{Phonetics}
\end{Entry}

\begin{Entry}{累积}{11,10}{⽷,⽲}
  \begin{Phonetics}{累积}{lei3ji1}[][HSK 7-9]
    \definition{v.}{acumular; agregar}
  \end{Phonetics}
\end{Entry}

%%%%%%%%%% 绯 %%%%%%%%%%
\subsection*{绯}\addcontentsline{loh}{figure}{绯}

\begin{Entry}{绯}{11}{⽷}
  \begin{Phonetics}{绯}{fei1}
    \definition{adj.}{escarlate; vermelho; vermelho escuro; vermelho profundo}
  \end{Phonetics}
\end{Entry}

\begin{Entry}{绯闻}{11,9}{⽷,⾨}
  \begin{Phonetics}{绯闻}{fei1wen2}[][HSK 7-9]
    \definition{s.}{boato/fofoca sobre escândalos sexuais | escândalo sexual; rumores sobre relacionamentos entre homens e mulheres}
  \end{Phonetics}
\end{Entry}

%%%%%%%%%% 绰 %%%%%%%%%%
\subsection*{绰}\addcontentsline{loh}{figure}{绰}

\begin{Entry}{绰}{11}{⽷}
  \begin{Phonetics}{绰}{chuo4}
    \definition{adj.}{amplo; espaçoso | (do porte de uma menina) graciosa; flexível}
  \end{Phonetics}
\end{Entry}

\begin{Entry}{绰号}{11,5}{⽷,⼝}
  \begin{Phonetics}{绰号}{chuo4hao4}[][HSK 7-9]
    \definition[个]{s.}{apelido; um nome informal dado a alguém com base em suas características, muitas vezes expressando afeição, antipatia ou brincadeira; também chamado de 外号}
  \seealsoref{外号}{wai4hao4}
  \end{Phonetics}
\end{Entry}

%%%%%%%%%% 绳 %%%%%%%%%%
\subsection*{绳}\addcontentsline{loh}{figure}{绳}

\begin{Entry}{绳}{11}{⽷}
  \begin{Phonetics}{绳}{sheng2}
    \definition*{s.}{Sobrenome: Sheng}
    \definition[根]{s.}{corda; cordão; barbante | a linha no marcador de tinta de carpinteiro}
    \definition{v.}{restringir; corrigir; sancionar | medir | continuar}
  \end{Phonetics}
\end{Entry}

\begin{Entry}{绳子}{11,3}{⽷,⼦}
  \begin{Phonetics}{绳子}{sheng2zi5}[][HSK 7-9]
    \definition[条,根]{s.}{corda; barbante; fio; um objeto longo e estreito feito torcendo dois ou mais fios de linha, palha ou cânhamo, frequentemente usado para amarrar ou puxar coisas}
  \end{Phonetics}
\end{Entry}

%%%%%%%%%% 维 %%%%%%%%%%
\subsection*{维}\addcontentsline{loh}{figure}{维}

\begin{Entry}{维}{11}{⽷}
  \begin{Phonetics}{维}{wei2}
    \definition*{s.}{Sobrenome: Wei}
    \definition{s.}{pensamento | dimensão; conceitos básicos de geometria e teoria do espaço}
    \definition{v.}{ligar; amarrar; manter unido; conectar | manter; manter; salvaguardar; preservar}
  \end{Phonetics}
\end{Entry}

\begin{Entry}{维生素}{11,5,10}{⽷,⽣,⽷}
  \begin{Phonetics}{维生素}{wei2sheng1su4}[][HSK 6]
    \definition[点]{s.}{vitamina}[西瓜中含丰富的维生素。===A melancia é rica em vitaminas.]
  \end{Phonetics}
\end{Entry}

\begin{Entry}{维吾尔}{11,7,5}{⽷,⼝,⼩}
  \begin{Phonetics}{维吾尔}{wei2wu2'er3}
    \definition*{s.}{Etnia Uigur de Xinjiang}
  \end{Phonetics}
\end{Entry}

\begin{Entry}{维护}{11,7}{⽷,⼿}
  \begin{Phonetics}{维护}{wei2hu4}[][HSK 4]
    \definition{v.}{defender; proteger; manter; preservar}
  \end{Phonetics}
\end{Entry}

\begin{Entry}{维修}{11,9}{⽷,⼈}
  \begin{Phonetics}{维修}{wei2xiu1}[][HSK 4]
    \definition{v.}{prestar serviços; manter; reparar; manter em (bom) estado de conservação}
  \end{Phonetics}
\end{Entry}

\begin{Entry}{维持}{11,9}{⽷,⼿}
  \begin{Phonetics}{维持}{wei2chi2}[][HSK 4]
    \definition{v.}{manter; conservar; guardar; manter vivo}
  \end{Phonetics}
\end{Entry}

%%%%%%%%%% 绷 %%%%%%%%%%
\subsection*{绷}\addcontentsline{loh}{figure}{绷}

\begin{Entry}{绷}{11}{⽷}
  \begin{Phonetics}{绷}{beng1}[][HSK 7-9]
    \definition{s.}{estrutura de cama amarrada com cordas, tiras de vime, etc.}
    \definition{v.}{esticar (ou puxar) com força | saltar; quicar | alinhavar; fixar | Dialeto: conseguir fazer algo com dificuldade | (roupas) apertar | costurar ou alfinetar com parcimônia |Dialeto: fraudar; roubar dinheiro}
  \end{Phonetics}
  \begin{Phonetics}{绷}{beng3}
    \definition{v.}{mostrar uma cara sombria, tensa; parecer descontente | conter o próprio temperamento}
  \end{Phonetics}
  \begin{Phonetics}{绷}{beng4}
    \definition{adv.}{muito; extremamente; altamente; usado antes de certos adjetivos para indicar um alto grau de severidade}
    \definition{v.}{rachar; dividir; rasgar; fissurar}
  \end{Phonetics}
\end{Entry}

\begin{Entry}{绷带}{11,9}{⽷,⼱}
  \begin{Phonetics}{绷带}{beng1dai4}[][HSK 7-9]
    \definition[条,卷]{s.}{curativo | Empréstimo linguístico: \emph{bandage}; a atadura de gaze usada para enfaixar feridas ou áreas afetadas}
  \end{Phonetics}
\end{Entry}

%%%%%%%%%% 综 %%%%%%%%%%
\subsection*{综}\addcontentsline{loh}{figure}{综}

\begin{Entry}{综}{11}{⽷}
  \begin{Phonetics}{综}{zeng4}
    \definition{s.}{liço; fuso; um dispositivo em um tear que separa os fios da urdidura em um padrão alternado para permitir a passagem da lançadeira}
  \end{Phonetics}
  \begin{Phonetics}{综}{zong1}
    \definition*{s.}{Sobrenome: Zong}
    \definition{v.}{reunir; resumir | combinar; reunir}
  \end{Phonetics}
\end{Entry}

\begin{Entry}{综合}{11,6}{⽷,⼝}
  \begin{Phonetics}{综合}{zong1he2}[][HSK 4]
    \definition{s.}{síntese}
    \definition{v.}{sintetizar; resumir as partes de uma coisa em um todo unificado após análise; reunir coisas de um tipo ou natureza diferente}
  \antonymref{分析}{fen1xi1}
  \end{Phonetics}
\end{Entry}

%%%%%%%%%% 绿 %%%%%%%%%%
\subsection*{绿}\addcontentsline{loh}{figure}{绿}

\begin{Entry}{绿}{11}{⽷}
  \begin{Phonetics}{绿}{lv4}[][HSK 2]
    \definition*{s.}{Sobrenome: Lü}
    \definition{adj.}{verde}
    \definition{v.}{tornar"-se verde; ficar verde}
  \end{Phonetics}
\end{Entry}

\begin{Entry}{绿化}{11,4}{⽷,⼔}
  \begin{Phonetics}{绿化}{lv4hua4}[][HSK 6]
    \definition{v.}{tornar verde plantando árvores, flores, etc.; arborizar; reflorestar; plantar árvores, flores e plantas para embelezar o ambiente ou prevenir a erosão do solo}
  \end{Phonetics}
\end{Entry}

\begin{Entry}{绿地}{11,6}{⽷,⼟}
  \begin{Phonetics}{绿地}{lv4di4}[][HSK 7-9]
    \definition{s.}{pequenas áreas isoladas de vegetação verde; terras reflorestadas | área verde (por exemplo: parque urbano ou jardim); espaços abertos em cidades e vilas que foram ajardinados ou arborizados}
  \end{Phonetics}
\end{Entry}

\begin{Entry}{绿灯}{11,6}{⽷,⽕}
  \begin{Phonetics}{绿灯}{lv4deng1}[][HSK 7-9]
    \definition[盏]{s.}{luz verde; (interseção) sinal verde indicando passagem permitida | permissão para prosseguir com algum projeto; autorização para prosseguir}
  \end{Phonetics}
\end{Entry}

\begin{Entry}{绿色}{11,6}{⽷,⾊}
  \begin{Phonetics}{绿色}{lv4se4}[][HSK 2]
    \definition{adj.}{verde; ecológico; sem poluição; em conformidade com os requisitos ambientais}
    \definition{s.}{cor verde}
  \end{Phonetics}
\end{Entry}

\begin{Entry}{绿豆}{11,7}{⽷,⾖}
  \begin{Phonetics}{绿豆}{lv4dou4}
    \definition{s.}{vagens}
  \end{Phonetics}
\end{Entry}

\begin{Entry}{绿豆芽}{11,7,7}{⽷,⾖,⾋}
  \begin{Phonetics}{绿豆芽}{lv4dou4 ya2}
    \definition{s.}{broto de feijão verde}
  \end{Phonetics}
\end{Entry}

\begin{Entry}{绿茶}{11,9}{⽷,⾋}
  \begin{Phonetics}{绿茶}{lv4cha2}[][HSK 3]
    \definition{s.}{chá verde; chá produzido apenas através dos processos de maturação, enrolamento (ou sem enrolamento) e secagem, sem passar por fermentação, com cor verde-claro}
  \end{Phonetics}
\end{Entry}

%%%%%%%%%% 聊 %%%%%%%%%%
\subsection*{聊}\addcontentsline{loh}{figure}{聊}

\begin{Entry}{聊}{11}{⽿}
  \begin{Phonetics}{聊}{liao2}[][HSK 6]
    \definition*{s.}{Sobrenome: Liao}
    \definition{adv.}{apenas; meramente; provisoriamente; por enquanto | um pouco; ligeiramente}
    \definition{v.}{tagarelar; conversar; bater papo | confiar (ou depender, recorrer) a}
  \end{Phonetics}
\end{Entry}

\begin{Entry}{聊天}{11,4}{⽿,⼤}
  \begin{Phonetics}{聊天}{liao2/tian1}
    \definition{v.+compl.}{papear | bater papo}
  \end{Phonetics}
\end{Entry}

\begin{Entry}{聊天儿}{11,4,2}{⽿,⼤,⼉}
  \begin{Phonetics}{聊天儿}{liao2/tian1r5}[][HSK 6]
    \definition{v.+compl.}{conversar; fofocar; bater papo; duas ou mais pessoas conversando sem um tópico ou propósito específico}
  \end{Phonetics}
\end{Entry}

%%%%%%%%%% 聋 %%%%%%%%%%
\subsection*{聋}\addcontentsline{loh}{figure}{聋}

\begin{Entry}{聋}{11}{⽿}
  \begin{Phonetics}{聋}{long2}[][HSK 7-9]
    \definition{adj.}{surdo; com deficiência auditiva}
  \end{Phonetics}
\end{Entry}

\begin{Entry}{聋人}{11,2}{⽿,⼈}
  \begin{Phonetics}{聋人}{long2ren2}[][HSK 7-9]
    \definition[个,位,名]{s.}{surdo (pessoa surda)}
  \end{Phonetics}
\end{Entry}

%%%%%%%%%% 职 %%%%%%%%%%
\subsection*{职}\addcontentsline{loh}{figure}{职}

\begin{Entry}{职}{11}{⽿}
  \begin{Phonetics}{职}{zhi2}
    \definition*{s.}{Sobrenome: Zhi}
    \definition{prep.}{para; devido a; por causa de}
    \definition{prep.}{Obsoleto: Eu (em relatórios oficiais aos superiores)}
    \definition{s.}{dever; trabalho | cargo; posto; função; responsabilidades; posição}
    \definition{v.}{gerenciar; dirigir | administrar}
  \end{Phonetics}
\end{Entry}

\begin{Entry}{职工}{11,3}{⽿,⼯}
  \begin{Phonetics}{职工}{zhi2gong1}[][HSK 3]
    \definition[个,位,名,些]{s.}{pessoal; trabalhadores e funcionários administrativos}
  \end{Phonetics}
\end{Entry}

\begin{Entry}{职业}{11,5}{⽿,⼀}
  \begin{Phonetics}{职业}{zhi2ye4}[][HSK 3]
    \definition{adj.}{profissional; não amador}
    \definition[种,份,个]{s.}{ocupação; profissão; vocação; o trabalho que um indivíduo realiza na sociedade como sua principal fonte de subsistência}
  \end{Phonetics}
\end{Entry}

\begin{Entry}{职务}{11,5}{⽿,⼒}
  \begin{Phonetics}{职务}{zhi2wu4}[][HSK 5]
    \definition{s.}{cargo; posto; deveres; função; funções que devem ser desempenhadas de acordo com as especificações do cargo}
  \end{Phonetics}
\end{Entry}

\begin{Entry}{职位}{11,7}{⽿,⼈}
  \begin{Phonetics}{职位}{zhi2wei4}[][HSK 5]
    \definition[个]{s.}{posto; posição; cargo que exerce determinadas funções em órgãos ou entidades}
  \end{Phonetics}
\end{Entry}

\begin{Entry}{职员}{11,7}{⽿,⼝}
  \begin{Phonetics}{职员}{zhi2yuan2}
    \definition[个,位]{s.}{empregado | trabalhador de escritório | membro da equipe}
  \end{Phonetics}
\end{Entry}

\begin{Entry}{职责}{11,8}{⽿,⾙}
  \begin{Phonetics}{职责}{zhi2ze2}[][HSK 6]
    \definition[种]{s.}{dever; obrigação; responsabilidade; coisas que você deve fazer por causa de sua profissão ou identidade}
  \end{Phonetics}
\end{Entry}

\begin{Entry}{职能}{11,10}{⽿,⾁}
  \begin{Phonetics}{职能}{zhi2neng2}[][HSK 5]
    \definition[种,项]{s.}{função; funções ou papéis que as organizações, instituições, etc. devem desempenhar}
  \end{Phonetics}
\end{Entry}

%%%%%%%%%% 脖 %%%%%%%%%%
\subsection*{脖}\addcontentsline{loh}{figure}{脖}

\begin{Entry}{脖}{11}{⾁}
  \begin{Phonetics}{脖}{bo2}
    \definition[个]{s.}{pescoço | em forma de pescoço | parte semelhante ao pescoço}
  \end{Phonetics}
\end{Entry}

\begin{Entry}{脖子}{11,3}{⾁,⼦}
  \begin{Phonetics}{脖子}{bo2zi5}[][HSK 7-9]
    \definition[条,个]{s.}{pescoço; a parte onde a cabeça e o tronco se conectam}
  \end{Phonetics}
\end{Entry}

%%%%%%%%%% 脚 %%%%%%%%%%
\subsection*{脚}\addcontentsline{loh}{figure}{脚}

\begin{Entry}{脚}{11}{⾁}
  \begin{Phonetics}{脚}{jiao3}[][HSK 2]
    \definition{clas.}{usado para chutes}
    \definition[只,双]{s.}{pé; a parte inferior das pernas de pessoas ou animais, que entra em contato com o solo | base; pé; a parte inferior do objeto | antigamente, referia"-se ao trabalho físico de transporte de cargas | resíduos; sobras}
  \end{Phonetics}
  \begin{Phonetics}{脚}{jue2}
    \variantof{角}
  \end{Phonetics}
\end{Entry}

\begin{Entry}{脚印}{11,5}{⾁,⼙}
  \begin{Phonetics}{脚印}{jiao3yin4}[][HSK 6]
    \definition{s.}{trilha; pegada; marca de pé; os rastros deixados pelos passos}
  \end{Phonetics}
\end{Entry}

\begin{Entry}{脚步}{11,7}{⾁,⽌}
  \begin{Phonetics}{脚步}{jiao3bu4}[][HSK 5]
    \definition{s.}{pé; passo; pisada; refere"-se ao movimento das pernas ao caminhar | ritmo; passo; distância entre os pés dianteiros e traseiros ao caminhar}
  \end{Phonetics}
\end{Entry}

%%%%%%%%%% 脱 %%%%%%%%%%
\subsection*{脱}\addcontentsline{loh}{figure}{脱}

\begin{Entry}{脱}{11}{⾁}
  \begin{Phonetics}{脱}{tuo1}[][HSK 4]
    \definition{conj.}{se; no caso}
    \definition{v.}{(cabelo, pele) soltar"-se; desprender"-se; cair | retirar peça de roupa do corpo | sair de; escapar de | perder (palavras) | livrar"-se de algo}
  \end{Phonetics}
\end{Entry}

\begin{Entry}{脱口而出}{11,3,6,5}{⾁,⼝,⽽,⼐}
  \begin{Phonetics}{脱口而出}{tuo1kou3'er2chu1}[][HSK 7-9]
    \definition{expr.}{deixar escapar; dizer algo sem querer; sem pensar, simplesmente deixando escapar}
  \synonymref{不假思索}{bu4jia3-si1suo3}
  \end{Phonetics}
\end{Entry}

\begin{Entry}{脱毛}{11,4}{⾁,⽑}
  \begin{Phonetics}{脱毛}{tuo1mao2}
    \definition{s.}{depilação}
    \definition{v.}{perder cabelo ou penas | depilar | fazer a barba}
  \end{Phonetics}
\end{Entry}

\begin{Entry}{脱节}{11,5}{⾁,⾋}
  \begin{Phonetics}{脱节}{tuo1/jie2}[][HSK 7-9]
    \definition{v.+compl.}{desfazer"-se; estar desalinhado; estar em desacordo com | desengatar | estar fora de articulação}
  \synonymref{摆脱}{bai3/tuo1}
  \synonymref{离开}{li2/kai1}
  \synonymref{脱离}{tuo1li2}
  \antonymref{连接}{lian2jie1}
  \antonymref{联系}{lian2xi4}
  \end{Phonetics}
\end{Entry}

\begin{Entry}{脱身}{11,7}{⾁,⾝}
  \begin{Phonetics}{脱身}{tuo1/shen1}[][HSK 7-9]
    \definition{v.+compl.}{escapar; libertar"-se; livrar"-se; sair de um determinado lugar; livrar"-se de algo}
  \end{Phonetics}
\end{Entry}

\begin{Entry}{脱险}{11,9}{⾁,⾩}
  \begin{Phonetics}{脱险}{tuo1xian3}
    \definition{v.}{sair do perigo}
  \end{Phonetics}
\end{Entry}

\begin{Entry}{脱离}{11,10}{⾁,⼇}
  \begin{Phonetics}{脱离}{tuo1li2}[][HSK 5]
    \definition{v.}{separar"-se; divorciar"-se; afastar"-se; sair (de um determinado ambiente ou situação); romper (uma determinada relação)}
  \end{Phonetics}
\end{Entry}

\begin{Entry}{脱落}{11,12}{⾁,⾋}
  \begin{Phonetics}{脱落}{tuo1luo4}[][HSK 7-9]
    \definition{v.}{cair; despencar; desprender; descascar | omitir (um caractere ao escrever); omitir texto}
  \end{Phonetics}
\end{Entry}

\begin{Entry}{脱颖而出}{11,13,6,5}{⾁,⾴,⽽,⼐}
  \begin{Phonetics}{脱颖而出}{tuo1ying3'er2chu1}[][HSK 7-9]
    \definition{expr.}{``A ponta inteira da sovela é visível através do saco de pano.''; isso significa, metaforicamente, que os talentos de uma pessoa talentosa acabarão sendo revelados; destacar"-se; sobressair"-se da multidão; sobressair"-se; emergir de forma proeminente; distinguir"-se}
  \end{Phonetics}
\end{Entry}

%%%%%%%%%% 脸 %%%%%%%%%%
\subsection*{脸}\addcontentsline{loh}{figure}{脸}

\begin{Entry}{脸}{11}{⾁}
  \begin{Phonetics}{脸}{lian3}[][HSK 2]
    \definition[张,个]{s.}{rosto (de pessoas ou animais); a parte frontal da cabeça, da testa ao queixo | parte frontal de algo | cara; autoestima; aparência | rosto; expressões faciais}
  \end{Phonetics}
\end{Entry}

\begin{Entry}{脸色}{11,6}{⾁,⾊}
  \begin{Phonetics}{脸色}{lian3se4}[][HSK 5]
    \definition{s.}{aparência; tez; cor da pele | aparência; expressão facial | (indicando a condição física de alguém) aparência; cor}
  \end{Phonetics}
\end{Entry}

\begin{Entry}{脸盆}{11,9}{⾁,⽫}
  \begin{Phonetics}{脸盆}{lian3pen2}[][HSK 5]
    \definition[个]{s.}{lavatório; bacia para lavar as mãos e o rosto}
  \end{Phonetics}
\end{Entry}

\begin{Entry}{脸颊}{11,12}{⾁,⾴}
  \begin{Phonetics}{脸颊}{lian3jia2}[][HSK 7-9]
    \definition{s.}{rosto; bochechas; ambos os lados do rosto}
  \end{Phonetics}
\end{Entry}

%%%%%%%%%% 舵 %%%%%%%%%%
\subsection*{舵}\addcontentsline{loh}{figure}{舵}

\begin{Entry}{舵}{11}{⾈}
  \begin{Phonetics}{舵}{duo4}
    \definition{s.}{leme; dispositivos para controlar a direção de navios, aeronaves, etc.}
  \seealsoref{柁}{tuo2}
  \end{Phonetics}
\end{Entry}

\begin{Entry}{舵手}{11,4}{⾈,⼿}
  \begin{Phonetics}{舵手}{duo4shou3}[][HSK 7-9]
    \definition{s.}{timoneiro}
  \end{Phonetics}
\end{Entry}

%%%%%%%%%% 船 %%%%%%%%%%
\subsection*{船}\addcontentsline{loh}{figure}{船}

\begin{Entry}{船}{11}{⾈}
  \begin{Phonetics}{船}{chuan2}[][HSK 2]
    \definition*{s.}{Sobrenome: Chuan}
    \definition[条,艘,叶,只]{s.}{barco; navio | embarcação; meio de transporte aquático, nome genérico para embarcações}
  \end{Phonetics}
\end{Entry}

\begin{Entry}{船长}{11,4}{⾈,⾧}
  \begin{Phonetics}{船长}{chuan2zhang3}[][HSK 6]
    \definition{s.}{capitão do navio; mestre; marinheiro; comandante; o oficial chefe a bordo}
  \end{Phonetics}
\end{Entry}

\begin{Entry}{船只}{11,5}{⾈,⼝}
  \begin{Phonetics}{船只}{chuan2zhi1}[][HSK 6]
    \definition[艘,条]{s.}{transporte marítimo; embarcação | navio; veleiro}
  \end{Phonetics}
\end{Entry}

\begin{Entry}{船员}{11,7}{⾈,⼝}
  \begin{Phonetics}{船员}{chuan2yuan2}[][HSK 6]
    \definition[名,位,个]{s.}{tripulação (do navio) | membro da tripulação (do navio); marinheiro; marujo; barqueiro; velejador}
  \end{Phonetics}
\end{Entry}

\begin{Entry}{船桨}{11,10}{⾈,⽊}
  \begin{Phonetics}{船桨}{chuan2jiang3}[][HSK 7-9]
    \definition{s.}{remo}
  \end{Phonetics}
\end{Entry}

\begin{Entry}{船舶}{11,11}{⾈,⾈}
  \begin{Phonetics}{船舶}{chuan2bo2}[][HSK 7-9]
    \definition[艘,条,只]{s.}{transporte marítimo; barcos e navios; refere"-se a vários navios}
  \end{Phonetics}
\end{Entry}

%%%%%%%%%% 菊 %%%%%%%%%%
\subsection*{菊}\addcontentsline{loh}{figure}{菊}

\begin{Entry}{菊}{11}{⾋}
  \begin{Phonetics}{菊}{ju2}
    \definition*{s.}{Sobrenome: Ju}
    \definition[朵]{s.}{crisântemo}
  \end{Phonetics}
\end{Entry}

\begin{Entry}{菊花}{11,7}{⾋,⾋}
  \begin{Phonetics}{菊花}{ju2hua1}[][HSK 7-9]
    \definition[朵,枝,把,束,棵,株]{s.}{crisântemo | Gíria: ânus; uma metáfora para o ânus (ou reto) humano}
  \end{Phonetics}
\end{Entry}

%%%%%%%%%% 菜 %%%%%%%%%%
\subsection*{菜}\addcontentsline{loh}{figure}{菜}

\begin{Entry}{菜}{11}{⾋}
  \begin{Phonetics}{菜}{cai4}[][HSK 1]
    \definition*{s.}{Sobrenome: Cai}
    \definition{adj.}{pouca habilidade; baixo nível; baixa capacidade}
    \definition[棵,个,道]{s.}{legumes; verduras; plantas que podem ser usadas como alimentos complementares | óleo de canola | prato; item ou prato do cardápio (seja de carne ou de vegetais)}
  \end{Phonetics}
\end{Entry}

\begin{Entry}{菜刀}{11,2}{⾋,⼑}
  \begin{Phonetics}{菜刀}{cai4dao1}
    \definition[把]{s.}{faca de vegetais | faca de cozinha | cutelo}
  \end{Phonetics}
\end{Entry}

\begin{Entry}{菜市场}{11,5,6}{⾋,⼱,⼟}
  \begin{Phonetics}{菜市场}{cai4shi4chang3}[][HSK 7-9]
    \definition[个,家]{s.}{mercado de alimentos; mercearia verde; mercado de produtos agrícolas; mercado de vegetais; um mercado em uma cidade ou município que vende vegetais, carne, ovos e outros alimentos não básicos}
  \end{Phonetics}
\end{Entry}

\begin{Entry}{菜单}{11,8}{⾋,⼗}
  \begin{Phonetics}{菜单}{cai4dan1}[][HSK 2]
    \definition[个,分,张]{s.}{menu; lista de pratos | menu (para computadores); lista utilizada para selecionar várias operações diferentes}
  \end{Phonetics}
\end{Entry}

%%%%%%%%%% 菠 %%%%%%%%%%
\subsection*{菠}\addcontentsline{loh}{figure}{菠}

\begin{Entry}{菠}{11}{⾋}
  \begin{Phonetics}{菠}{bo1}
    \definition{s.}{espinafre}
  \end{Phonetics}
\end{Entry}

\begin{Entry}{菠菜}{11,11}{⾋,⾋}
  \begin{Phonetics}{菠菜}{bo1cai4}
    \definition[棵]{s.}{espinafre}
  \end{Phonetics}
\end{Entry}

%%%%%%%%%% 菩 %%%%%%%%%%
\subsection*{菩}\addcontentsline{loh}{figure}{菩}

\begin{Entry}{菩}{11}{⾋}
  \begin{Phonetics}{菩}{pu2}
    \definition*[个]{s.}{Buda; Bodhisattva | Estátua de Bodhisattva}
  \end{Phonetics}
\end{Entry}

\begin{Entry}{菩萨}{11,11}{⾋,⾋}
  \begin{Phonetics}{菩萨}{pu2sa4}[][HSK 7-9]
    \definition*{s.}{Bodhisattva Guanyin; Buda; Deus}
    \definition[个]{s.}{divindade; deus | Figurativo: pessoa bondosa | ídolo budista}
  \end{Phonetics}
\end{Entry}

%%%%%%%%%% 菱 %%%%%%%%%%
\subsection*{菱}\addcontentsline{loh}{figure}{菱}

\begin{Entry}{菱}{11}{⾋}
  \begin{Phonetics}{菱}{ling2}
    \definition{s.}{maruca; caltrop aquático; castanha d'água}
  \end{Phonetics}
\end{Entry}

\begin{Entry}{菱角}{11,7}{⾋,⾓}
  \begin{Phonetics}{菱角}{ling2jiao5}
    \definition{s.}{castanha d'água}
  \end{Phonetics}
\end{Entry}

%%%%%%%%%% 萌 %%%%%%%%%%
\subsection*{萌}\addcontentsline{loh}{figure}{萌}

\begin{Entry}{萌}{11}{⾋}
  \begin{Phonetics}{萌}{meng2}
    \definition*{s.}{Sobrenome: Meng}
    \definition{s.}{broto; rebento | Arcaico: o povo comum}
    \definition{v.}{(plantas) brotar; surgir; brotar; germinar | começar; surgir; ocorrer; emergir}
  \end{Phonetics}
\end{Entry}

\begin{Entry}{萌发}{11,5}{⾋,⼜}
  \begin{Phonetics}{萌发}{meng2fa1}[][HSK 7-9]
    \definition{v.}{brotar; germinar; germinação de sementes ou esporos | emergir; vir à tona; isso se refere metaforicamente ao início de um determinado pensamento, ideia ou sentimento}
  \end{Phonetics}
\end{Entry}

\begin{Entry}{萌芽}{11,7}{⾋,⾋}
  \begin{Phonetics}{萌芽}{meng2ya2}[][HSK 7-9]
    \definition{s.}{semente; germe; rudimento; uma metáfora para algo novo e ainda não totalmente desenvolvido}
    \definition{v.}{brotar; germinar; desabrochar; o brotar de plantas é uma metáfora para algo que está apenas começando a acontecer}
  \end{Phonetics}
\end{Entry}

%%%%%%%%%% 萍 %%%%%%%%%%
\subsection*{萍}\addcontentsline{loh}{figure}{萍}

\begin{Entry}{萍}{11}{⾋}
  \begin{Phonetics}{萍}{ping2}
    \definition{s.}{lentilha"-d'água}
  \end{Phonetics}
\end{Entry}

\begin{Entry}{萍水相逢}{11,4,9,10}{⾋,⽔,⽬,⾡}
  \begin{Phonetics}{萍水相逢}{ping2shui3-xiang1feng2}[][HSK 7-9]
    \definition{expr.}{(estranhos) encontram"-se por acaso como tufos de lentilha"-d'água à deriva; um conhecido casual; realizar uma reunião informal e temporária}
  \end{Phonetics}
\end{Entry}

%%%%%%%%%% 萎 %%%%%%%%%%
\subsection*{萎}\addcontentsline{loh}{figure}{萎}

\begin{Entry}{萎}{11}{⾋}
  \begin{Phonetics}{萎}{wei1}
    \definition{v.}{murchar; cair}
  \synonymref{缩}{suo1}
  \synonymref{谢}{xie4}
  \end{Phonetics}
  \begin{Phonetics}{萎}{wei3}
    \definition{adj.}{em declínio; decadente | sem ânimo; abatido}
    \definition{v.}{murchar; definhar; tombar; deixar cair}
  \synonymref{缩}{suo1}
  \synonymref{谢}{xie4}
  \end{Phonetics}
\end{Entry}

\begin{Entry}{萎缩}{11,14}{⾋,⽷}
  \begin{Phonetics}{萎缩}{wei3suo1}[][HSK 7-9]
    \definition{v.}{murchar; (corpo, vegetação, etc.) secar | ceder; encolher; contrair}
  \synonymref{收缩}{shou1suo1}
  \antonymref{发达}{fa1da2}
  \antonymref{蔓延}{man4yan2}
  \antonymref{蓬勃}{peng2bo2}
  \antonymref{膨胀}{peng2zhang4}
  \end{Phonetics}
\end{Entry}

%%%%%%%%%% 萝 %%%%%%%%%%
\subsection*{萝}\addcontentsline{loh}{figure}{萝}

\begin{Entry}{萝}{11}{⾋}
  \begin{Phonetics}{萝}{luo2}
    \definition[个]{s.}{plantas trepadeiras; videira}
  \end{Phonetics}
\end{Entry}

\begin{Entry}{萝卜}{11,2}{⾋,⼘}
  \begin{Phonetics}{萝卜}{luo2bo5}[][HSK 7-9]
    \definition[个,根]{s.}{rabanete; nabo | planta de rabanete}
  \end{Phonetics}
\end{Entry}

%%%%%%%%%% 营 %%%%%%%%%%
\subsection*{营}\addcontentsline{loh}{figure}{营}

\begin{Entry}{营}{11}{⾋}
  \begin{Phonetics}{营}{ying2}
    \definition*{s.}{Sobrenome: Ying}
    \definition{s.}{acampamento; quartel; onde o exército está estacionado | batalhão; unidades militares}
    \definition{v.}{procurar | operar; executar; gerenciar}
  \end{Phonetics}
\end{Entry}

\begin{Entry}{营业}{11,5}{⾋,⼀}
  \begin{Phonetics}{营业}{ying2ye4}[][HSK 4]
    \definition{v.}{fazer negócios; estar aberto para negócios}
  \end{Phonetics}
\end{Entry}

\begin{Entry}{营养}{11,9}{⾋,⼋}
  \begin{Phonetics}{营养}{ying2yang3}[][HSK 3]
    \definition[种]{s.}{nutrição; alimentação; a função do organismo de absorver as substâncias necessárias do meio externo para manter atividades vitais, como crescimento e desenvolvimento | nutrição; alimentação; ato ou processo de fornecer nutrição}
  \end{Phonetics}
\end{Entry}

%%%%%%%%%% 著 %%%%%%%%%%
\subsection*{著}\addcontentsline{loh}{figure}{著}

\begin{Entry}{著}{11}{⽬}
  \begin{Phonetics}{著}{zhu4}
    \definition{adj.}{marcado; excelente; óbvio}
    \definition{s.}{livro; trabalho | nativo; pessoa/povo indígena; refere"-se a pessoas que se estabeleceram em um lugar por gerações}
    \definition{v.}{mostrar; provar; revelar | escrever}
  \end{Phonetics}
\end{Entry}

\begin{Entry}{著名}{11,6}{⽬,⼝}
  \begin{Phonetics}{著名}{zhu4ming2}[][HSK 4]
    \definition[位]{adj.}{famoso; bem conhecido; célebre}
  \end{Phonetics}
\end{Entry}

\begin{Entry}{著作}{11,7}{⽬,⼈}
  \begin{Phonetics}{著作}{zhu4zuo4}[][HSK 4]
    \definition[部,本,类]{s.}{obra; livro; escritos}
    \definition{v.}{escrever; usar palavras para expressar opiniões, conhecimentos, ideias, sentimentos, etc.}
  \end{Phonetics}
\end{Entry}

%%%%%%%%%% 虚 %%%%%%%%%%
\subsection*{虚}\addcontentsline{loh}{figure}{虚}

\begin{Entry}{虚}{11}{⾌}
  \begin{Phonetics}{虚}{xu1}
    \definition*{s.}{Xu, a décima primeira das vinte e oito constelações em que a esfera celeste foi dividida, consistindo de duas estrelas em linha reta, uma em Aquário e a outra em Equuleus | Xu, uma das mansões lunares | Sobrenome: Xu}
    \definition{adj.}{vazio; oco; desocupado | desconfiado; tímido | falso; nominal | humilde; modesto | fraco; com saúde debilitada | Física: virtual}
    \definition{adv.}{em vão}
    \definition{s.}{vazio; nulidade; anulação | resumo; teoria; princípios orientadores; ideologia política e outros aspectos}
    \definition{v.}{reservar espaço}
  \antonymref{实}{shi2}
  \end{Phonetics}
\end{Entry}

\begin{Entry}{虚心}{11,4}{⾌,⼼}
  \begin{Phonetics}{虚心}{xu1xin1}[][HSK 5]
    \definition{adj.}{modesto; humilde; de mente aberta; não ser presunçoso, ser capaz de aceitar as opiniões dos outros}
  \end{Phonetics}
\end{Entry}

\begin{Entry}{虚伪}{11,6}{⾌,⼈}
  \begin{Phonetics}{虚伪}{xu1wei3}
    \definition{adj.}{falso | hipócrita | artificial}
  \end{Phonetics}
\end{Entry}

%%%%%%%%%% 蛇 %%%%%%%%%%
\subsection*{蛇}\addcontentsline{loh}{figure}{蛇}

\begin{Entry}{蛇}{11}{⾍}
  \begin{Phonetics}{蛇}{she2}[][HSK 5]
    \definition[条]{s.}{cobra; serpente; répteis}
  \end{Phonetics}
\end{Entry}

%%%%%%%%%% 蛋 %%%%%%%%%%
\subsection*{蛋}\addcontentsline{loh}{figure}{蛋}

\begin{Entry}{蛋}{11}{⾍}
  \begin{Phonetics}{蛋}{dan4}[][HSK 2]
    \definition[个,只]{s.}{ovo; ovos produzidos por aves, tartarugas, cobras, etc. | algo em forma de ovo | tolo; idiota; metáfora para pessoas com determinadas características (com conotação pejorativa) | se perder; colocado após certos verbos, forma um verbo com conotação pejorativa | testículos; em algumas regiões, refere"-se aos testículos de certos animais ou pessoas}
  \end{Phonetics}
\end{Entry}

\begin{Entry}{蛋白质}{11,5,8}{⾍,⽩,⾙}
  \begin{Phonetics}{蛋白质}{dan4bai2zhi4}[][HSK 7-9]
    \definition{s.}{proteína}
  \end{Phonetics}
\end{Entry}

\begin{Entry}{蛋糕}{11,16}{⾍,⽶}
  \begin{Phonetics}{蛋糕}{dan4gao1}[][HSK 5]
    \definition[个,块,盒]{s.}{bolo; bolo fofo feito de ovos e farinha com açúcar e óleo}
  \end{Phonetics}
\end{Entry}

%%%%%%%%%% 袋 %%%%%%%%%%
\subsection*{袋}\addcontentsline{loh}{figure}{袋}

\begin{Entry}{袋}{11}{⾐}
  \begin{Phonetics}{袋}{dai4}[][HSK 4]
    \definition{clas.}{usado para coisas que podem ser colocadas nos bolsos | usado para cigarros, narguilé ou tabaco seco}
    \definition[口]{s.}{saco; sacola; bolso; bolsa}
  \end{Phonetics}
\end{Entry}

%%%%%%%%%% 袭 %%%%%%%%%%
\subsection*{袭}\addcontentsline{loh}{figure}{袭}

\begin{Entry}{袭}{11}{⾐}
  \begin{Phonetics}{袭}{xi2}
    \definition*{s.}{Sobrenome: Xi}
    \definition{clas.}{usado para conjuntos completos de roupas}
    \definition{v.}{fazer um ataque surpresa a; invadir | seguir o padrão de; continuar como antes; fazer o mesmo}
  \end{Phonetics}
\end{Entry}

\begin{Entry}{袭击}{11,5}{⾐,⼐}
  \begin{Phonetics}{袭击}{xi2ji1}[][HSK 7-9]
    \definition{v.}{invadir; atacar de surpresa; fazer um ataque surpresa contra; em termos militares, isso se refere ao seu ataque surpresa | atingir; golpear; geralmente se refere a um ataque repentino}
  \synonymref{报复}{bao4fu4}
  \synonymref{冲击}{chong1ji1}
  \synonymref{挫折}{cuo4zhe2}
  \synonymref{打击}{da3ji1}
  \synonymref{攻击}{gong1ji1}
  \synonymref{进攻}{jin4gong1}
  \synonymref{抨击}{peng1ji1}
  \synonymref{障碍}{zhang4'ai4}
  \antonymref{保卫}{bao3wei4}
  \end{Phonetics}
\end{Entry}

%%%%%%%%%% 谋 %%%%%%%%%%
\subsection*{谋}\addcontentsline{loh}{figure}{谋}

\begin{Entry}{谋}{11}{⾔}
  \begin{Phonetics}{谋}{mou2}
    \definition*{s.}{Sobrenome: Mou}
    \definition[个]{s.}{estratagema; plano; esquema | estratégia; ideia; esquema; plano}
    \definition{v.}{trabalhar para; buscar | consultar | planejar; traçar | conferir; discutir}
  \end{Phonetics}
\end{Entry}

\begin{Entry}{谋生}{11,5}{⾔,⽣}
  \begin{Phonetics}{谋生}{mou2sheng1}[][HSK 7-9]
    \definition{v.}{ganhar a vida; obter renda; tentar encontrar uma maneira de ganhar a vida}
  \end{Phonetics}
\end{Entry}

\begin{Entry}{谋求}{11,7}{⾔,⽔}
  \begin{Phonetics}{谋求}{mou2qiu2}[][HSK 7-9]
    \definition{v.}{buscar; esforçar"-se por; estar em questão de; tentar obter}
  \end{Phonetics}
\end{Entry}

\begin{Entry}{谋害}{11,10}{⾔,⼧}
  \begin{Phonetics}{谋害}{mou2hai4}[][HSK 7-9]
    \definition{v.}{planejar um assassinato; conspirar para matar; tramar para matar | planejar uma conspiração contra; conspirar contra alguém; tramar para incriminar}
  \end{Phonetics}
\end{Entry}

%%%%%%%%%% 谎 %%%%%%%%%%
\subsection*{谎}\addcontentsline{loh}{figure}{谎}

\begin{Entry}{谎}{11}{⾔}
  \begin{Phonetics}{谎}{huang3}
    \definition[句]{s.}{mentira; falsidade}
    \definition{v.}{contar uma mentira; mentir}
  \end{Phonetics}
\end{Entry}

\begin{Entry}{谎言}{11,7}{⾔,⾔}
  \begin{Phonetics}{谎言}{huang3yan2}[][HSK 7-9]
    \definition[个,派]{s.}{mentira; falsidade; é uma declaração falsa e inverídica, frequentemente usada para enganar os outros}
  \end{Phonetics}
\end{Entry}

\begin{Entry}{谎话}{11,8}{⾔,⾔}
  \begin{Phonetics}{谎话}{huang3hua4}[][HSK 7-9]
    \definition[个]{s.}{mentira; falsidade; palavras falsas e enganosas}
  \end{Phonetics}
\end{Entry}

%%%%%%%%%% 谐 %%%%%%%%%%
\subsection*{谐}\addcontentsline{loh}{figure}{谐}

\begin{Entry}{谐}{11}{⾔}
  \begin{Phonetics}{谐}{xie2}
    \definition{adj.}{harmonioso | humorístico}
  \end{Phonetics}
\end{Entry}

%%%%%%%%%% 谜 %%%%%%%%%%
\subsection*{谜}\addcontentsline{loh}{figure}{谜}

\begin{Entry}{谜}{11}{⾔}
  \begin{Phonetics}{谜}{mei4}
    \definition[个]{s.}{enigma}
  \seealsoref{谜儿}{mei4r5}
  \end{Phonetics}
  \begin{Phonetics}{谜}{mi2}[][HSK 7-9]
    \definition[个]{s.}{enigma; charada | enigma; mistério; quebra"-cabeça}
  \end{Phonetics}
\end{Entry}

\begin{Entry}{谜儿}{11,2}{⾔,⼉}
  \begin{Phonetics}{谜儿}{mei4r5}
    \definition{s.}{enigma; mistério}
  \end{Phonetics}
\end{Entry}

\begin{Entry}{谜团}{11,6}{⾔,⼞}
  \begin{Phonetics}{谜团}{mi2tuan2}[][HSK 7-9]
    \definition{s.}{dúvidas e suspeitas | assuntos elusivos | enigma | situação imprevisível}
  \end{Phonetics}
\end{Entry}

\begin{Entry}{谜底}{11,8}{⾔,⼴}
  \begin{Phonetics}{谜底}{mi2di3}[][HSK 7-9]
    \definition[个]{s.}{resposta a um enigma | verdade; metáfora para a verdade dos fatos}
  \end{Phonetics}
\end{Entry}

\begin{Entry}{谜语}{11,9}{⾔,⾔}
  \begin{Phonetics}{谜语}{mi2yu3}[][HSK 7-9]
    \definition[条,则]{s.}{enigma; charada; um enigma é uma mensagem críptica que alude a coisas ou palavras, deixando ao leitor a tarefa de adivinhar; consiste principalmente em duas partes: o próprio enigma e a resposta}
  \end{Phonetics}
\end{Entry}

%%%%%%%%%% 象 %%%%%%%%%%
\subsection*{象}\addcontentsline{loh}{figure}{象}

\begin{Entry}{象}{11}{⾗}
  \begin{Phonetics}{象}{xiang4}
    \definition*{s.}{Sobrenome: Xiang}
    \definition[头,群,个]{s.}{elefante | elefante, uma das peças do xadrez chinês | aparência; forma; imagem}
    \definition{v.}{imitar | latir}
  \end{Phonetics}
\end{Entry}

\begin{Entry}{象征}{11,8}{⾗,⼻}
  \begin{Phonetics}{象征}{xiang4zheng1}[][HSK 5]
    \definition[种]{s.}{símbolo; emblema; insígnia; \emph{token}; objeto concreto que simboliza um significado especial}
    \definition{v.}{simbolizar; significar; representar; expressar um significado especial através de algo concreto}
  \end{Phonetics}
\end{Entry}

\begin{Entry}{象棋}{11,12}{⾗,⽊}
  \begin{Phonetics}{象棋}{xiang4qi2}
    \definition[副]{s.}{xadrez chinês; um tipo de jogo de xadrez em que dois jogadores têm dezesseis peças cada: um general, dois soldados, dois elefantes, duas carruagens, dois cavalos, dois canhões e cinco soldados ; cada jogador joga de acordo com as regras e o vencedor é aquele que der o xeque no general do adversário}
  \end{Phonetics}
\end{Entry}

%%%%%%%%%% 距 %%%%%%%%%%
\subsection*{距}\addcontentsline{loh}{figure}{距}

\begin{Entry}{距}{11}{⾜}
  \begin{Phonetics}{距}{ju4}[][HSK 7-9]
    \definition{s.}{distância | espora (de um galo, etc.)}
    \definition{v.}{estar separado (longe) de; estar distante de}
  \end{Phonetics}
\end{Entry}

\begin{Entry}{距离}{11,10}{⾜,⼇}
  \begin{Phonetics}{距离}{ju4li2}[][HSK 4]
    \definition[个,段]{s.}{distância}
    \definition{v.}{estar distante de}
  \end{Phonetics}
\end{Entry}

%%%%%%%%%% 躭 %%%%%%%%%%
\subsection*{躭}\addcontentsline{loh}{figure}{躭}

\begin{Entry}{躭}{11}{⾝}
  \begin{Phonetics}{躭}{dan1}
    \definition{v.}{entregar"-se a; adiar}
  \end{Phonetics}
\end{Entry}

%%%%%%%%%% 辅 %%%%%%%%%%
\subsection*{辅}\addcontentsline{loh}{figure}{辅}

\begin{Entry}{辅}{11}{⾞}
  \begin{Phonetics}{辅}{fu3}
    \definition*{s.}{Sobrenome: Fu}
    \definition{adj.}{subsidiário}
    \definition{s.}{barras laterais do carrinho atuando como proteção da roda; duas barras retas de madeira são adicionadas na parte externa da roda para prender o cubo | maçã do rosto | assistente oficial; títulos oficiais antigos | (literário) território que circunda a capital}
    \definition{v.}{auxiliar; complementar; suplementar | ajudar}
  \end{Phonetics}
\end{Entry}

\begin{Entry}{辅导}{11,6}{⾞,⼨}
  \begin{Phonetics}{辅导}{fu3dao3}[][HSK 7-9]
    \definition{v.}{orientar no estudo ou treinamento; treinar; guiar; dar aulas particulares}
  \end{Phonetics}
\end{Entry}

\begin{Entry}{辅助}{11,7}{⾞,⼒}
  \begin{Phonetics}{辅助}{fu3zhu4}[][HSK 5]
    \definition{adj.}{auxiliar; suplementar; complementar}
    \definition{v.}{auxiliar; ajudar; colocar os outros em primeiro lugar e dar-lhes alguma ajuda externa}
  \end{Phonetics}
\end{Entry}

%%%%%%%%%% 辆 %%%%%%%%%%
\subsection*{辆}\addcontentsline{loh}{figure}{辆}

\begin{Entry}{辆}{11}{⾞}
  \begin{Phonetics}{辆}{liang4}[][HSK 2]
    \definition{clas.}{usado para automóveis, veículos, etc.}
  \end{Phonetics}
\end{Entry}

%%%%%%%%%% 逮 %%%%%%%%%%
\subsection*{逮}\addcontentsline{loh}{figure}{逮}

\begin{Entry}{逮}{11}{⾡}
  \begin{Phonetics}{逮}{dai3}[][HSK 7-9]
    \definition{v.}{capturar; pegar}[猫逮老鼠。===Gatos pegam ratos.]
  \end{Phonetics}
  \begin{Phonetics}{逮}{dai4}
    \definition*{s.}{Sobrenome: Dai}
    \definition{v.}{alcançar | prender, usado em 逮捕}
  \seealsoref{逮捕}{dai4bu3}
  \end{Phonetics}
\end{Entry}

\begin{Entry}{逮捕}{11,10}{⾡,⼿}
  \begin{Phonetics}{逮捕}{dai4bu3}[][HSK 7-9]
    \definition{v.}{prender; apreender; levar sob custódia}
  \end{Phonetics}
\end{Entry}

%%%%%%%%%% 逶 %%%%%%%%%%
\subsection*{逶}\addcontentsline{loh}{figure}{逶}

\begin{Entry}{逶}{11}{⾡}
  \begin{Phonetics}{逶}{wei1}
    \definition{adj.}{sinuoso; tortuoso}
  \end{Phonetics}
\end{Entry}

\begin{Entry}{逶迤}{11,8}{⾡,⾡}
  \begin{Phonetics}{逶迤}{wei1yi2}
    \definition{adj.}{sinuoso; tortuoso; descreve a aparência sinuosa e contínua de estradas, montanhas, rios, etc.}
  \end{Phonetics}
\end{Entry}

%%%%%%%%%% 逻 %%%%%%%%%%
\subsection*{逻}\addcontentsline{loh}{figure}{逻}

\begin{Entry}{逻}{11}{⾡}
  \begin{Phonetics}{逻}{luo2}
    \definition{s.}{patrulha | (literário) a beira de um riacho de montanha}
    \definition{v.}{patrulhar; fazer rondas}
  \end{Phonetics}
\end{Entry}

\begin{Entry}{逻辑}{11,13}{⾡,⾞}
  \begin{Phonetics}{逻辑}{luo2ji5}[][HSK 5]
    \definition[套,条,种]{s.}{lógica; lei objetiva; a objetividade das leis que regem o desenvolvimento das coisas | lógica; razão; regras para o pensamento | lógica como ciência do raciocínio, do pensamento; disciplina que estuda a lógica}
  \end{Phonetics}
\end{Entry}

%%%%%%%%%% 野 %%%%%%%%%%
\subsection*{野}\addcontentsline{loh}{figure}{野}

\begin{Entry}{野}{11}{⾥}
  \begin{Phonetics}{野}{ye3}[][HSK 6]
    \definition*{s.}{Sobrenome: Ye}
    \definition{adj.}{(de plantas ou animais) selvagem; incultivado; não domesticado; indomável (opp. 家) | rude; áspero | desenfreado; abandonado; indisciplinado | ilícito; sem licença}
    \definition{s.}{espaço aberto; o aberto | limite; fronteira | não está no poder; fora do cargo}
  \seealsoref{家}{jia1}
  \end{Phonetics}
\end{Entry}

\begin{Entry}{野生}{11,5}{⾥,⽣}
  \begin{Phonetics}{野生}{ye3sheng1}[][HSK 6]
    \definition{adj.}{selvagem; não cultivado; não domesticado}
  \end{Phonetics}
\end{Entry}

%%%%%%%%%% 铜 %%%%%%%%%%
\subsection*{铜}\addcontentsline{loh}{figure}{铜}

\begin{Entry}{铜}{11}{⾦}
  \begin{Phonetics}{铜}{tong2}[][HSK 7-9]
    \definition[块]{s.}{Cu; cobre}
  \end{Phonetics}
\end{Entry}

\begin{Entry}{铜牌}{11,12}{⾦,⽚}
  \begin{Phonetics}{铜牌}{tong2pai2}[][HSK 6]
    \definition[枚]{s.}{medalha de bronze; o bronze | placa de bronze com nome ou logotipo comercial, etc.}
  \end{Phonetics}
\end{Entry}

%%%%%%%%%% 铝 %%%%%%%%%%
\subsection*{铝}\addcontentsline{loh}{figure}{铝}

\begin{Entry}{铝}{11}{⾦}
  \begin{Phonetics}{铝}{lv3}[][HSK 7-9]
    \definition[锭]{s.}{Química: alumínio, Al; um elemento metálico, tem cor prateada, é leve, quimicamente reativo, altamente maleável e possui excelente condutividade elétrica e térmica}
  \end{Phonetics}
\end{Entry}

%%%%%%%%%% 铭 %%%%%%%%%%
\subsection*{铭}\addcontentsline{loh}{figure}{铭}

\begin{Entry}{铭}{11}{⾦}
  \begin{Phonetics}{铭}{ming2}
    \definition*{s.}{Sobrenome: Ming}
    \definition{s.}{inscrição; textos antigos fundidos ou gravados em objetos e estelas para registrar fatos, realizações ou para servir de advertência}
    \definition{v.}{gravar; lembrar; inscrever; inscrever textos comemorativos em objetos; uma metáfora para recordar profundamente}
  \end{Phonetics}
\end{Entry}

\begin{Entry}{铭记}{11,5}{⾦,⾔}
  \begin{Phonetics}{铭记}{ming2ji4}[][HSK 7-9]
    \definition{v.}{lembrar sempre; gravar na mente; guardar algo na memória com muito carinho}
  \end{Phonetics}
\end{Entry}

%%%%%%%%%% 铲 %%%%%%%%%%
\subsection*{铲}\addcontentsline{loh}{figure}{铲}

\begin{Entry}{铲}{11}{⾦}
  \begin{Phonetics}{铲}{chan3}[][HSK 7-9]
    \definition[个,把]{s.}{pá}
    \definition{v.}{trabalhar com uma pá (ou enxada) | levantar (mover) com uma pá}
  \end{Phonetics}
\end{Entry}

\begin{Entry}{铲子}{11,3}{⾦,⼦}
  \begin{Phonetics}{铲子}{chan3zi5}[][HSK 7-9]
    \definition[把]{s.}{pá; uma ferramenta com uma chapa de ferro grossa, quase quadrada, em uma extremidade e um cabo longo na outra}
  \end{Phonetics}
\end{Entry}

\begin{Entry}{铲车}{11,4}{⾦,⾞}
  \begin{Phonetics}{铲车}{chan3che1}
    \definition[台]{s.}{empilhadeira}
  \end{Phonetics}
\end{Entry}

%%%%%%%%%% 银 %%%%%%%%%%
\subsection*{银}\addcontentsline{loh}{figure}{银}

\begin{Entry}{银}{11}{⾦}
  \begin{Phonetics}{银}{yin2}[][HSK 3]
    \definition*{s.}{Sobrenome: Yin}
    \definition{adj.}{prateado; como a cor da prata}
    \definition[锭]{s.}{Ag, prata | refere"-se a moeda ou a coisas relacionadas com moeda}
  \end{Phonetics}
\end{Entry}

\begin{Entry}{银色}{11,6}{⾦,⾊}
  \begin{Phonetics}{银色}{yin2 se4}
    \definition{s.}{cor prata; prateado}
  \end{Phonetics}
\end{Entry}

\begin{Entry}{银行}{11,6}{⾦,⾏}
  \begin{Phonetics}{银行}{yin2hang2}[][HSK 2]
    \definition[个,家,所]{s.}{banco; instituições financeiras que operam depósitos, empréstimos, câmbio, poupança e outros negócios}
  \end{Phonetics}
\end{Entry}

\begin{Entry}{银行卡}{11,6,5}{⾦,⾏,⼘}
  \begin{Phonetics}{银行卡}{yin2hang2ka3}[][HSK 2]
    \definition{s.}{cartão bancário; cartão ATM}
  \end{Phonetics}
\end{Entry}

\begin{Entry}{银河}{11,8}{⾦,⽔}
  \begin{Phonetics}{银河}{yin2he2}
    \definition*{s.}{Via Láctea}
  \seealsoref{银河系}{yin2he2xi4}
  \end{Phonetics}
\end{Entry}

\begin{Entry}{银河系}{11,8,7}{⾦,⽔,⽷}
  \begin{Phonetics}{银河系}{yin2he2xi4}
    \definition*{s.}{Galáxia Via Láctea}
  \seealsoref{银河}{yin2he2}
  \end{Phonetics}
\end{Entry}

\begin{Entry}{银牌}{11,12}{⾦,⽚}
  \begin{Phonetics}{银牌}{yin2pai2}[][HSK 3]
    \definition[枚]{s.}{medalha de prata; um tipo de medalha, concedida ao segundo colocado}
  \end{Phonetics}
\end{Entry}

%%%%%%%%%% 阐 %%%%%%%%%%
\subsection*{阐}\addcontentsline{loh}{figure}{阐}

\begin{Entry}{阐}{11}{⾨}
  \begin{Phonetics}{阐}{chan3}
    \definition{v.}{explicar; expor; expressar; divulgar; esclarecer; elucidar}
  \end{Phonetics}
\end{Entry}

\begin{Entry}{阐述}{11,8}{⾨,⾡}
  \begin{Phonetics}{阐述}{chan3shu4}[][HSK 7-9]
    \definition{v.}{explicar; expor; elaborar; discutir}
  \end{Phonetics}
\end{Entry}

%%%%%%%%%% 隆 %%%%%%%%%%
\subsection*{隆}\addcontentsline{loh}{figure}{隆}

\begin{Entry}{隆}{11}{⾩}
  \begin{Phonetics}{隆}{long2}
    \definition*{s.}{Sobrenome: Long}
    \definition{adj.}{grandioso; magnífico | próspero; florescente | intenso; profundo; no auge}
    \definition{v.}{inchar; protuberar; elevar"-se}
  \end{Phonetics}
\end{Entry}

\begin{Entry}{隆重}{11,9}{⾩,⾥}
  \begin{Phonetics}{隆重}{long2zhong4}[][HSK 7-9]
    \definition{adj.}{grandioso; solene; cerimonioso}
  \end{Phonetics}
\end{Entry}

%%%%%%%%%% 随 %%%%%%%%%%
\subsection*{随}\addcontentsline{loh}{figure}{随}

\begin{Entry}{随}{11}{⾩}
  \begin{Phonetics}{随}{sui2}[][HSK 3]
    \definition*{s.}{Sobrenome: Sui}
    \definition{adv.}{fazer algo imediatamente assim que ocorre, sem demora ou hesitação; usado antes de dois verbos ou frases verbais para indicar que a última ação segue a anterior}
    \definition{prep.}{junto com (alguma outra ação) | apresentando as condições das quais a ação depende}
    \definition{v.}{seguir; vir (ou ir) junto com | concordar com; adaptar"-se a | deixar (alguém fazer o que quiser) | (dialeto) parecer"-se com; assemelhar"-se a | seguir ou agir de acordo com a condição ou circunstância da qual a ação depende}
  \end{Phonetics}
\end{Entry}

\begin{Entry}{随大流}{11,3,10}{⾩,⼤,⽔}
  \begin{Phonetics}{随大流}{sui2 da4liu2}[][HSK 5]
    \definition{v.}{deixar"-se levar (ou nadar) pela correnteza; seguir a tendência geral; fazer como os outros fazem | seguir a maré | seguir a multidão}
  \synonymref{随大溜}{sui2 da4liu4}
  \end{Phonetics}
\end{Entry}

\begin{Entry}{随大溜}{11,3,13}{⾩,⼤,⽔}
  \begin{Phonetics}{随大溜}{sui2 da4liu4}[][HSK 7-9]
    \definition{v.}{seguir a multidão; deixar"-se levar (ou nadar) pela correnteza; seguir a tendência geral; fazer como os outros fazem}
  \synonymref{随大流}{sui2 da4liu2}
  \end{Phonetics}
\end{Entry}

\begin{Entry}{随心所欲}{11,4,8,11}{⾩,⼼,⼾,⽋}
  \begin{Phonetics}{随心所欲}{sui2xin1suo3yu4}[][HSK 7-9]
    \definition{expr.}{faça do seu jeito; fazer o que se quer, segundo a própria vontade (mais tarde, passou a significar fazer tudo o que se deseja); seguir as próprias inclinações; fazer do seu jeito; agir como bem entender}
  \antonymref{力不从心}{li4bu4cong2xin1}
  \antonymref{力所能及}{li4suo3neng2ji2}
  \end{Phonetics}
\end{Entry}

\begin{Entry}{随手}{11,4}{⾩,⼿}
  \begin{Phonetics}{随手}{sui2shou3}[][HSK 4]
    \definition{adv.}{convenientemente; sem problemas adicionais; casualmente}
  \end{Phonetics}
\end{Entry}

\begin{Entry}{随处}{11,5}{⾩,⼡}
  \begin{Phonetics}{随处}{sui2chu4}
    \definition{adv.}{em qualquer lugar}
  \end{Phonetics}
\end{Entry}

\begin{Entry}{随处可见}{11,5,5,4}{⾩,⼡,⼝,⾒}
  \begin{Phonetics}{随处可见}{sui2chu4 ke3 jian4}[][HSK 7-9]
    \definition{expr.}{em todos os lugares; por toda parte; onipresente}
  \end{Phonetics}
\end{Entry}

\begin{Entry}{随后}{11,6}{⾩,⼝}
  \begin{Phonetics}{随后}{sui2hou4}[][HSK 5]
    \definition{adv.}{logo em seguida; logo depois; indica que segue imediatamente após a ação ou situação anterior (geralmente usado em conjunto com 就)}
  \seealsoref{就}{jiu4}
  \end{Phonetics}
\end{Entry}

\begin{Entry}{随地}{11,6}{⾩,⼟}
  \begin{Phonetics}{随地}{sui2di4}
    \definition{adv.}{qualquer lugar | todo lugar}
  \end{Phonetics}
\end{Entry}

\begin{Entry}{随机}{11,6}{⾩,⽊}
  \begin{Phonetics}{随机}{sui2ji1}[][HSK 7-9]
    \definition{adj.}{aleatório; indica fazer algo em conformidade com as mudanças no tempo e nas circunstâncias}
    \definition{adv.}{de acordo com a situação; sem quaisquer condições; fazer algo arbitrariamente}
  \synonymref{立即}{li4ji2}
  \synonymref{立刻}{li4ke4}
  \end{Phonetics}
\end{Entry}

\begin{Entry}{随机存取记忆体}{11,6,6,8,5,4,7}{⾩,⽊,⼦,⼜,⾔,⼼,⼈}
  \begin{Phonetics}{随机存取记忆体}{sui2ji1cun2qu3ji4yi4ti3}
    \definition{s.}{RAM (\emph{random access memory})}
  \seealsoref{内存}{nei4cun2}
  \seealsoref{随机存取存储器}{sui2ji1cun2qu3cun2chu3qi4}
  \end{Phonetics}
\end{Entry}

\begin{Entry}{随机存取存储器}{11,6,6,8,6,12,16}{⾩,⽊,⼦,⼜,⼦,⼈,⼝}
  \begin{Phonetics}{随机存取存储器}{sui2ji1cun2qu3cun2chu3qi4}
    \definition{s.}{RAM (\emph{random access memory})}
  \seealsoref{内存}{nei4cun2}
  \seealsoref{随机存取记忆体}{sui2ji1cun2qu3ji4yi4ti3}
  \end{Phonetics}
\end{Entry}

\begin{Entry}{随即}{11,7}{⾩,⼙}
  \begin{Phonetics}{随即}{sui2ji2}[][HSK 7-9]
    \definition{adv.}{imediatamente; atualmente; indica que algo acontece imediatamente após uma ação ou situação anterior}
  \synonymref{理科}{li3ke1}
  \synonymref{立即}{li4ji2}
  \synonymref{立刻}{li4ke4}
  \antonymref{永远}{yong3yuan3}
  \end{Phonetics}
\end{Entry}

\begin{Entry}{随时}{11,7}{⾩,⽇}
  \begin{Phonetics}{随时}{sui2shi2}[][HSK 2]
    \definition{adv.}{a qualquer momento; em todos os momentos}
  \end{Phonetics}
\end{Entry}

\begin{Entry}{随时随地}{11,7,11,6}{⾩,⽇,⾩,⼟}
  \begin{Phonetics}{随时随地}{sui2shi2-sui2di4}[][HSK 7-9]
    \definition{expr.}{``A qualquer hora, em qualquer lugar.''; sempre que possível; em todos os lugares; em todos os momentos e lugares; quando e onde}
  \end{Phonetics}
\end{Entry}

\begin{Entry}{随身}{11,7}{⾩,⾝}
  \begin{Phonetics}{随身}{sui2shen1}[][HSK 7-9]
    \definition{adj.}{significa ``carregar consigo'' ou ``estar ao lado'', indicando que algo é mantido junto ao corpo ou próximo a ele, levado para todos os lugares; é frequentemente usada para descrever pertences pessoais, bagagem ou equipamentos, como: bagagem de mão (随身行李), itens de mão (随身携带)}
  \seealsoref{随身行李}{sui2shen1 hang2li3}
  \seealsoref{随身携带}{sui2shen1 xie2dai4}
  \end{Phonetics}
\end{Entry}

\begin{Entry}{随身行李}{11,7,6,7}{⾩,⾝,⾏,⽊}
  \begin{Phonetics}{随身行李}{sui2shen1 hang2li3}[][HSK 7-9]
    \definition{s.}{bagagem de mão; itens para bagagem de mão}
  \end{Phonetics}
\end{Entry}

\begin{Entry}{随身携带}{11,7,13,9}{⾩,⾝,⼿,⼱}
  \begin{Phonetics}{随身携带}{sui2shen1 xie2dai4}
    \definition{s.}{itens de mão (leve"-o consigo)}
  \end{Phonetics}
\end{Entry}

\begin{Entry}{随和}{11,8}{⾩,⼝}
  \begin{Phonetics}{随和}{sui2he5}
    \definition{adj.}{afável; prestativo; tranquilo; fácil de conviver; gentil}
  \synonymref{和蔼}{he2'ai3}
  \synonymref{温和}{wen1he2}
  \antonymref{反对}{fan3dui4}
  \antonymref{固执}{gu4zhi5}
  \antonymref{乖张}{guai1zhang1}
  \antonymref{矫情}{jiao2qing5}
  \antonymref{严肃}{yan2su4}
  \end{Phonetics}
\end{Entry}

\begin{Entry}{随便}{11,9}{⾩,⼈}
  \begin{Phonetics}{随便}{sui2/bian4}[][HSK 2]
    \definition{adj.}{relaxado; descontraído; sem restrições; sem limitações | aleatório; casual; descuidado; indiferente; distraído, não pensa bem antes de falar ou agir | casual; informal; não dá importância aos detalhes}
    \definition{conj.}{qualquer; qualquer que seja; não importa}
    \definition{v.+compl.}{deixar alguém à vontade}
  \end{Phonetics}
\end{Entry}

\begin{Entry}{随着}{11,11}{⾩,⽬}
  \begin{Phonetics}{随着}{sui2zhe5}[][HSK 5]
    \definition{prep.}{junto com; na esteira de; em sintonia com; usado no início da frase ou antes do verbo, indica as condições necessárias para que uma ação, comportamento ou evento ocorra}
  \end{Phonetics}
\end{Entry}

\begin{Entry}{随意}{11,13}{⾩,⼼}
  \begin{Phonetics}{随意}{sui2/yi4}[][HSK 5]
    \definition{adj.}{aleatório; casual; à vontade; como se deseja}
  \end{Phonetics}
\end{Entry}

%%%%%%%%%% 隐 %%%%%%%%%%
\subsection*{隐}\addcontentsline{loh}{figure}{隐}

\begin{Entry}{隐}{11}{⾩}
  \begin{Phonetics}{隐}{yin3}
    \definition*{s.}{Sobrenome: Yin}
    \definition{adj.}{escondido; escondido profundamente | latente; adormecido; à espreita}
    \definition{pref.}{cripto-}
    \definition{s.}{segredo; assuntos ocultos}
    \definition{v.}{esconder; esconder da vista; ocultar}
  \end{Phonetics}
\end{Entry}

\begin{Entry}{隐私}{11,7}{⾩,⽲}
  \begin{Phonetics}{隐私}{yin3si1}[][HSK 6]
    \definition[点,些]{s.}{privacidade; segredos de alguém; assuntos pessoais que você não quer contar ou tornar públicos}
  \end{Phonetics}
\end{Entry}

\begin{Entry}{隐藏}{11,17}{⾩,⾋}
  \begin{Phonetics}{隐藏}{yin3cang2}[][HSK 6]
    \definition{v.}{esconder; ocultar}
  \end{Phonetics}
\end{Entry}

%%%%%%%%%% 雪 %%%%%%%%%%
\subsection*{雪}\addcontentsline{loh}{figure}{雪}

\begin{Entry}{雪}{11}{⾬}
  \begin{Phonetics}{雪}{xue3}[][HSK 2]
    \definition*{s.}{Sobrenome: Xue}
    \definition[场,层]{s.}{neve | algo parecido com neve}
    \definition{v.}{limpar; enxugar; remover}
  \end{Phonetics}
\end{Entry}

\begin{Entry}{雪人}{11,2}{⾬,⼈}
  \begin{Phonetics}{雪人}{xue3ren2}
    \definition{s.}{boneco de neve | \emph{Yeti}}
  \end{Phonetics}
\end{Entry}

\begin{Entry}{雪山}{11,3}{⾬,⼭}
  \begin{Phonetics}{雪山}{xue3shan1}
    \definition{s.}{montanha coberta de neve}
  \end{Phonetics}
\end{Entry}

\begin{Entry}{雪花}{11,7}{⾬,⾋}
  \begin{Phonetics}{雪花}{xue3hua1}
    \definition{s.}{floco de neve}
  \end{Phonetics}
\end{Entry}

\begin{Entry}{雪板}{11,8}{⾬,⽊}
  \begin{Phonetics}{雪板}{xue3ban3}
    \definition{s.}{prancha de \emph{snowboard}}
    \definition{v.}{praticar \textit{snowboard}}
  \end{Phonetics}
\end{Entry}

\begin{Entry}{雪葩}{11,12}{⾬,⾋}
  \begin{Phonetics}{雪葩}{xue3pa1}
    \definition{s.}{sorvete}
  \end{Phonetics}
\end{Entry}

\begin{Entry}{雪鞋}{11,15}{⾬,⾰}
  \begin{Phonetics}{雪鞋}{xue3xie2}
    \definition[双]{s.}{sapatos de neve}
  \end{Phonetics}
\end{Entry}

\begin{Entry}{雪糕}{11,16}{⾬,⽶}
  \begin{Phonetics}{雪糕}{xue3gao1}
    \definition{s.}{picolé}
  \end{Phonetics}
\end{Entry}

%%%%%%%%%% 领 %%%%%%%%%%
\subsection*{领}\addcontentsline{loh}{figure}{领}

\begin{Entry}{领}{11}{⾴}
  \begin{Phonetics}{领}{ling3}[][HSK 3]
    \definition{clas.}{usado para roupas, mantos, esteiras, tapetes, telas, etc.}
    \definition{s.}{pescoço; gargalo | gola; colarinho; faixa de pescoço | esboço; ponto principal; essência}
    \definition{v.}{conduzir; guiar; orientar | possuir; ser o possuidor de; ter jurisdição sobre | obter; conseguir; receber (o que foi distribuído) | aceitar; tomar |entender; compreender (o significado)}
  \end{Phonetics}
\end{Entry}

\begin{Entry}{领土}{11,3}{⾴,⼟}
  \begin{Phonetics}{领土}{ling3tu3}[][HSK 7-9]
    \definition[寸,片,块]{s.}{território | domínio}
  \end{Phonetics}
\end{Entry}

\begin{Entry}{领队}{11,4}{⾴,⾩}
  \begin{Phonetics}{领队}{ling3dui4}[][HSK 7-9]
    \definition{s.}{o líder de um grupo, equipe esportiva, etc.}
    \definition{v.}{liderar um grupo; liderar a equipe}
  \end{Phonetics}
\end{Entry}

\begin{Entry}{领会}{11,6}{⾴,⼈}
  \begin{Phonetics}{领会}{ling3hui4}[][HSK 7-9]
    \definition{v.}{compreender; entender; apreciar as coisas e obter compreensão}
  \end{Phonetics}
\end{Entry}

\begin{Entry}{领先}{11,6}{⾴,⼉}
  \begin{Phonetics}{领先}{ling3/xian1}[][HSK 3]
    \definition{v.+compl.}{liderar; assumir a liderança; estar na liderança; (velocidade, desempenho, etc.) superar pessoas ou coisas semelhantes, estar na vanguarda}
  \end{Phonetics}
\end{Entry}

\begin{Entry}{领军}{11,6}{⾴,⼍}
  \begin{Phonetics}{领军}{ling3jun1}[][HSK 7-9]
    \definition{v.}{comandar um exército; liderar tropas | Figurativo: liderar; desempenhar um papel de liderança}
  \end{Phonetics}
\end{Entry}

\begin{Entry}{领导}{11,6}{⾴,⼨}
  \begin{Phonetics}{领导}{ling3dao3}[][HSK 3]
    \definition[个,位,名,些]{s.}{líder; liderança; pessoa que ocupa uma posição de liderança}
    \definition{v.}{liderar; exercer liderança; (elogio) liderar, gerenciar outras pessoas;  trabalhar com outras pessoas ou avançar em direção a um objetivo}
  \end{Phonetics}
\end{Entry}

\begin{Entry}{领事}{11,8}{⾴,⼅}
  \begin{Phonetics}{领事}{ling3shi4}[][HSK 7-9]
    \definition[位]{s.}{cônsul}
  \end{Phonetics}
\end{Entry}

\begin{Entry}{领事馆}{11,8,11}{⾴,⼅,⾷}
  \begin{Phonetics}{领事馆}{ling3shi4guan3}[][HSK 7-9]
    \definition[个]{s.}{consulado; um escritório de representação consular de um governo em uma cidade ou região de outro país}
  \end{Phonetics}
\end{Entry}

\begin{Entry}{领取}{11,8}{⾴,⼜}
  \begin{Phonetics}{领取}{ling3qu3}[][HSK 6]
    \definition{v.}{sacar; receber; obter; receber o que lhe é enviado}
  \end{Phonetics}
\end{Entry}

\begin{Entry}{领养}{11,9}{⾴,⼋}
  \begin{Phonetics}{领养}{ling3yang3}[][HSK 7-9]
    \definition{s.}{adoção}
    \definition{v.}{adotar (uma criança) | assumir a responsabilidade por}
  \end{Phonetics}
\end{Entry}

\begin{Entry}{领带}{11,9}{⾴,⼱}
  \begin{Phonetics}{领带}{ling3dai4}[][HSK 5]
    \definition[条]{s.}{colar; gargantilha; gravata}
  \end{Phonetics}
\end{Entry}

\begin{Entry}{领悟}{11,10}{⾴,⼼}
  \begin{Phonetics}{领悟}{ling3wu4}[][HSK 7-9]
    \definition{v.}{compreender; entender}
  \end{Phonetics}
\end{Entry}

\begin{Entry}{领袖}{11,10}{⾴,⾐}
  \begin{Phonetics}{领袖}{ling3xiu4}[][HSK 6]
    \definition[个,位,名]{s.}{líder de estados, grupos políticos, organizações de massa, etc.}
  \end{Phonetics}
\end{Entry}

\begin{Entry}{领域}{11,11}{⾴,⼟}
  \begin{Phonetics}{领域}{ling3yu4}[][HSK 7-9]
    \definition[块,片,个]{s.}{território; domínio; uma região na qual um país exerce sua soberania | campo; esfera; domínio; reino; o âmbito do pensamento acadêmico ou das atividades sociais}
  \end{Phonetics}
\end{Entry}

\begin{Entry}{领情}{11,11}{⾴,⼼}
  \begin{Phonetics}{领情}{ling3/qing2}
    \definition{v.+compl.}{sentir"-se grato a alguém}
  \end{Phonetics}
\end{Entry}

\begin{Entry}{领略}{11,11}{⾴,⽥}
  \begin{Phonetics}{领略}{ling3lve4}[][HSK 7-9]
    \definition{v.}{perceber; apreciar; ter um gostinho de; entender as circunstâncias das coisas e então reconhecer seu significado ou discernir seu sabor}
  \end{Phonetics}
\end{Entry}

%%%%%%%%%% 颇 %%%%%%%%%%
\subsection*{颇}\addcontentsline{loh}{figure}{颇}

\begin{Entry}{颇}{11}{⽪}
  \begin{Phonetics}{颇}{po1}[][HSK 7-9]
    \definition*{s.}{Sobrenome: Po}
    \definition{adj.}{oblíquo; inclinado para um lado | Literário: tendencioso; incorreto}
    \definition{adv.}{muito; bastante; consideravelmente}
  \end{Phonetics}
\end{Entry}

%%%%%%%%%% 颈 %%%%%%%%%%
\subsection*{颈}\addcontentsline{loh}{figure}{颈}

\begin{Entry}{颈}{11}{⾴}
  \begin{Phonetics}{颈}{geng3}
    \definition{s.}{nuca}
  \end{Phonetics}
  \begin{Phonetics}{颈}{jing3}
    \definition{s.}{pescoço}
  \end{Phonetics}
\end{Entry}

\begin{Entry}{颈部}{11,10}{⾴,⾢}
  \begin{Phonetics}{颈部}{jing3bu4}[][HSK 7-9]
    \definition{s.}{pescoço}
  \end{Phonetics}
\end{Entry}

%%%%%%%%%% 骑 %%%%%%%%%%
\subsection*{骑}\addcontentsline{loh}{figure}{骑}

\begin{Entry}{骑}{11}{⾺}
  \begin{Phonetics}{骑}{qi2}[][HSK 2]
    \definition{s.}{cavalos ou outros animais para montaria | cavalaria; cavaleiro, também se refere genericamente a qualquer pessoa que monta a cavalo}
    \definition{v.}{montar (um animal ou bicicleta); sentar"-se na parte de trás de | montar; abranger ambos os lados}
  \end{Phonetics}
\end{Entry}

\begin{Entry}{骑车}{11,4}{⾺,⾞}
  \begin{Phonetics}{骑车}{qi2che1}[][HSK 2]
    \definition{v.}{andar de bicicleta; pedalar}
  \end{Phonetics}
\end{Entry}

%%%%%%%%%% 鸽 %%%%%%%%%%
\subsection*{鸽}\addcontentsline{loh}{figure}{鸽}

\begin{Entry}{鸽}{11}{⿃}
  \begin{Phonetics}{鸽}{ge1}
    \definition[只]{s.}{pombo}[和平鸽。===Pomba da Paz.]
  \end{Phonetics}
\end{Entry}

\begin{Entry}{鸽子}{11,3}{⿃,⼦}
  \begin{Phonetics}{鸽子}{ge1zi5}[][HSK 7-9]
    \definition[只,对,群]{s.}{pombo}
  \end{Phonetics}
\end{Entry}

%%%%%%%%%% 鹿 %%%%%%%%%%
\subsection*{鹿}\addcontentsline{loh}{figure}{鹿}

\begin{Entry}{鹿}{11}{⿅}[Kangxi 198]
  \begin{Phonetics}{鹿}{lu4}[][HSK 7-9]
    \definition*{s.}{Sobrenome: Lu}
    \definition[只,头,群]{s.}{cervo; veado; gazela}
  \end{Phonetics}
\end{Entry}

%%%%%%%%%% 麻 %%%%%%%%%%
\subsection*{麻}\addcontentsline{loh}{figure}{麻}

\begin{Entry}{麻}{11}{⿇}[Kangxi 200]
  \begin{Phonetics}{麻}{ma1}
    \definition{adj.}{sombrio; escuro; completamente escuro}
  \end{Phonetics}
  \begin{Phonetics}{麻}{ma2}[][HSK 7-9]
    \definition*{s.}{Sobrenome: Ma}
    \definition{adj.}{áspero; grosseiro | marcado; manchado | espinhas; manchas ásperas; cicatrizes deixadas após a varíola}
    \definition[棵,株]{s.}{nome geral para cânhamo, linho, etc. | fibra de cânhamo, linho, etc. para têxteis | sésamo; gergelim | marcas de varíola; um rosto com marcas de varíola}
    \definition{v.}{anestesiar | corromper (a mente de alguém); envenenar}
  \end{Phonetics}
\end{Entry}

\begin{Entry}{麻木}{11,4}{⿇,⽊}
  \begin{Phonetics}{麻木}{ma2mu4}[][HSK 7-9]
    \definition{adj.}{dormente; descreve uma sensação de dormência ou perda de sensibilidade em uma parte do corpo devido ao frio ou à inatividade prolongada | entorpecido; insensível; sem vida; apático; de raciocínio lento}
  \end{Phonetics}
\end{Entry}

\begin{Entry}{麻将}{11,9}{⿇,⼨}
  \begin{Phonetics}{麻将}{ma2jiang4}[][HSK 7-9]
    \definition*[副]{s.}{Mahjong}
  \end{Phonetics}
\end{Entry}

\begin{Entry}{麻烦}{11,10}{⿇,⽕}
  \begin{Phonetics}{麻烦}{ma2fan5}[][HSK 3]
    \definition{adj.}{incômodo; inconveniente; complicado; trabalhoso; burocrático | incômodo; inconveniente; (a situação) é confusa e complicada}
    \definition[个,些,点,堆]{s.}{problema; inconveniência; assuntos complicados e difíceis de resolver}
    \definition{v.}{incomodar; perturbar; incomodar alguém; irritar; aborrecer; causar incômodo ou sobrecarregar outras pessoas}
  \end{Phonetics}
\end{Entry}

\begin{Entry}{麻痹}{11,13}{⿇,⽧}
  \begin{Phonetics}{麻痹}{ma2bi4}[][HSK 7-9]
    \definition{adj.}{insensível; descuidado; negligente; essa metáfora descreve um estado de entorpecimento mental e perda de vigilância}
    \definition{v.}{ficar paralisado; ficar insensível; perder total ou parcialmente as funções sensoriais e motoras em uma parte do corpo | estar insensível; baixar a guarda; entorpecer a mente}
  \end{Phonetics}
\end{Entry}

\begin{Entry}{麻辣}{11,14}{⿇,⾟}
  \begin{Phonetics}{麻辣}{ma2la4}[][HSK 7-9]
    \definition{adj.}{picante; quente e dormente; descreve um sabor como dormente e picante, semelhante à pimenta"-de"-Sichuan ou à pimenta malagueta}
  \end{Phonetics}
\end{Entry}

\begin{Entry}{麻辣豆腐}{11,14,7,14}{⿇,⾟,⾖,⾁}
  \begin{Phonetics}{麻辣豆腐}{ma2la4 dou4fu5}
    \definition{s.}{tofú guisado em molho picante (prato)}
  \end{Phonetics}
\end{Entry}

\begin{Entry}{麻醉}{11,15}{⿇,⾣}
  \begin{Phonetics}{麻醉}{ma2zui4}[][HSK 7-9]
    \definition{v.}{narcotizar; anestesiar; utilizar drogas, acupuntura ou outros métodos para deixar temporariamente todo ou parte de um organismo inconsciente | corromper (a mente de alguém); desgastar (a força de vontade de alguém); essa metáfora descreve o uso de certos métodos para desgastar a vontade de uma pessoa, fazendo com que ela perca a capacidade de distinguir o certo do errado}
  \end{Phonetics}
\end{Entry}

%%%%%%%%%% 黄 %%%%%%%%%%
\subsection*{黄}\addcontentsline{loh}{figure}{黄}

\begin{Entry}{黄}{11}{⿈}[Kangxi 201]
  \begin{Phonetics}{黄}{huang2}[][HSK 2]
    \definition*{s.}{Rio Huanghe, abreviação de 黄河 | Refere"-se ao Imperador Amarelo, um imperador da mitologia chinesa antiga | Sobrenome: Huang ou Hwang}
    \definition{adj.}{amarelo | obsceno; indecente; pornográfico; símbolo de corrupção e decadência, referindo"-se especificamente à pornografia}
    \definition{s.}{gema; ovas de caranguejo; refere"-se a certas coisas de cor amarela}
    \definition{v.}{fracassar; dar errado}
  \seealsoref{黄河}{huang2he2}
  \end{Phonetics}
\end{Entry}

\begin{Entry}{黄瓜}{11,5}{⿈,⽠}
  \begin{Phonetics}{黄瓜}{huang2gua5}[][HSK 4]
    \definition[根,棵,株,条]{s.}{pepino}
  \end{Phonetics}
\end{Entry}

\begin{Entry}{黄色}{11,6}{⿈,⾊}
  \begin{Phonetics}{黄色}{huang2se4}[][HSK 2]
    \definition{adj.}{decadente; obsceno; erótico; pornográfico; símbolo de corrupção e decadência, referindo"-se especificamente à pornografia}
    \definition[种]{s.}{cor amarela}
  \end{Phonetics}
\end{Entry}

\begin{Entry}{黄昏}{11,8}{⿈,⽇}
  \begin{Phonetics}{黄昏}{huang2hun1}[][HSK 7-9]
    \definition[个]{s.}{crepúsculo; refere"-se ao período do pôr do sol ao anoitecer}
  \end{Phonetics}
\end{Entry}

\begin{Entry}{黄河}{11,8}{⿈,⽔}
  \begin{Phonetics}{黄河}{huang2he2}
    \definition*{s.}{Rio Amarelo | Rio Huang He}
  \end{Phonetics}
\end{Entry}

\begin{Entry}{黄油}{11,8}{⿈,⽔}
  \begin{Phonetics}{黄油}{huang2you2}
    \definition[盒]{s.}{manteiga}
  \end{Phonetics}
\end{Entry}

\begin{Entry}{黄金}{11,8}{⿈,⾦}
  \begin{Phonetics}{黄金}{huang2jin1}[][HSK 4]
    \definition{adj.}{de primeira qualidade; dourado;}
    \definition[块,克,两]{s.}{ouro; \emph{aurum}; um tipo de metal, de cor amarela, mais precioso, abreviação de 金, frequentemente falado como 金子}
  \seealsoref{金}{jin1}
  \seealsoref{金子}{jin1zi5}
  \end{Phonetics}
\end{Entry}

%%%%% EOF %%%%%


 %%%
%%% 12画
%%%
\section*{12画}\addcontentsline{toc}{section}{12画}\addcontentsline{loh}{figure}{\#\#\#\# 12画}

%%%%%%%%%% 傍 %%%%%%%%%%
\subsection*{傍}\addcontentsline{loh}{figure}{傍}

\begin{Entry}{傍}{12}{⼈}
  \begin{Phonetics}{傍}{bang4}
    \definition*{s.}{Sobrenome: Bang}
    \definition{v.}{estar perto de (à distância); aproximar"-se | estar perto de (no tempo) | depender de; confiar em}
  \end{Phonetics}
\end{Entry}

\begin{Entry}{傍晚}{12,11}{⼈,⽇}
  \begin{Phonetics}{傍晚}{bang4wan3}[][HSK 6]
    \definition[个]{s.}{ao entardecer; ao cair da noite; (tarde) refere"-se ao momento em que se aproxima o anoitecer, frequentemente usado na linguagem escrita}
  \end{Phonetics}
\end{Entry}

%%%%%%%%%% 傢 %%%%%%%%%%
\subsection*{傢}\addcontentsline{loh}{figure}{傢}

\begin{Entry}{傢}{12}{⼈}
  \begin{Phonetics}{傢}{jia1}
    \definition{s.}{usado em 家伙  e 家俱}
    \variantof{家}
  \seealsoref{傢伙}{jia1huo5}
  \seealsoref{家俱}{jia1ju4}
  \end{Phonetics}
\end{Entry}

\begin{Entry}{傢伙}{12,6}{⼈,⼈}
  \begin{Phonetics}{傢伙}{jia1huo5}
    \variantof{家伙}
  \end{Phonetics}
\end{Entry}

\begin{Entry}{傢俱}{12,10}{⼈,⼈}
  \begin{Phonetics}{傢俱}{jia1ju4}
    \variantof{家俱}
  \end{Phonetics}
\end{Entry}

%%%%%%%%%% 储 %%%%%%%%%%
\subsection*{储}\addcontentsline{loh}{figure}{储}

\begin{Entry}{储}{12}{⼈}
  \begin{Phonetics}{储}{chu3}
    \definition*{s.}{Sobrenome: Chu}
    \definition{s.}{herdeiro de um trono | herdeiro}
    \definition{v.}{armazenar | guardar; manter (ter) em reserva}
  \end{Phonetics}
\end{Entry}

\begin{Entry}{储存}{12,6}{⼈,⼦}
  \begin{Phonetics}{储存}{chu3cun2}[][HSK 6]
    \definition{v.}{armazenar; depositar; colocar em; economizar dinheiro ou coisas que você não precisará em um futuro próximo}
  \end{Phonetics}
\end{Entry}

\begin{Entry}{储备}{12,8}{⼈,⼡}
  \begin{Phonetics}{储备}{chu3bei4}[][HSK 7-9]
    \definition{v.}{guardar; armazenar para uso futuro}
  \end{Phonetics}
\end{Entry}

\begin{Entry}{储蓄}{12,13}{⼈,⾋}
  \begin{Phonetics}{储蓄}{chu3xu4}[][HSK 7-9]
    \definition[笔,份]{s.}{depósitos; poupanças; refere"-se a dinheiro ou coisas acumuladas}
    \definition{v.}{poupar; depositar; guardar dinheiro ou coisas que são guardadas ou não são usadas temporariamente, geralmente significa depositar dinheiro em um banco}
  \end{Phonetics}
\end{Entry}

%%%%%%%%%% 傲 %%%%%%%%%%
\subsection*{傲}\addcontentsline{loh}{figure}{傲}

\begin{Entry}{傲}{12}{⼈}
  \begin{Phonetics}{傲}{ao4}[][HSK 7-9]
    \definition{adj.}{orgulhoso; altivo | arrogante}
    \definition{v.}{recusar"-se a ceder; desafiar}
  \end{Phonetics}
\end{Entry}

\begin{Entry}{傲慢}{12,14}{⼈,⼼}
  \begin{Phonetics}{傲慢}{ao4man4}[][HSK 7-9]
    \definition{adj.}{altivo; arrogante; autoritário}
  \end{Phonetics}
\end{Entry}

%%%%%%%%%% 剩 %%%%%%%%%%
\subsection*{剩}\addcontentsline{loh}{figure}{剩}

\begin{Entry}{剩}{12}{⼑}
  \begin{Phonetics}{剩}{sheng4}[][HSK 5]
    \definition*{s.}{Sobrenome: Sheng}
    \definition{v.}{permanecer; ser deixado (para trás)}
  \end{Phonetics}
\end{Entry}

\begin{Entry}{剩下}{12,3}{⼑,⼀}
  \begin{Phonetics}{剩下}{sheng4/xia5}[][HSK 5]
    \definition{v.+compl.}{permanecer; ser deixado (para trás); consumir e utilizar, restando apenas os resíduos}
  \end{Phonetics}
\end{Entry}

\begin{Entry}{剩余}{12,7}{⼑,⼈}
  \begin{Phonetics}{剩余}{sheng4yu2}[][HSK 7-9]
    \definition{s.}{excedente; restante; o que sobra depois de subtrair uma porção de uma determinada quantidade}
    \definition{v.}{sobrar; restar; ser excedente}
  \antonymref{短缺}{duan3que1}
  \antonymref{缺少}{que1shao3}
  \end{Phonetics}
\end{Entry}

%%%%%%%%%% 割 %%%%%%%%%%
\subsection*{割}\addcontentsline{loh}{figure}{割}

\begin{Entry}{割}{12}{⼑}
  \begin{Phonetics}{割}{ge1}[][HSK 7-9]
    \definition{v.}{cortar; ceifar | dividir; cortar}
  \end{Phonetics}
\end{Entry}

%%%%%%%%%% 募 %%%%%%%%%%
\subsection*{募}\addcontentsline{loh}{figure}{募}

\begin{Entry}{募}{12}{⼒}
  \begin{Phonetics}{募}{mu4}
    \definition*{s.}{Sobrenome: Mu}
    \definition{v.}{arrecadar; coletar | alistar; recrutar}
  \end{Phonetics}
\end{Entry}

\begin{Entry}{募捐}{12,10}{⼒,⼿}
  \begin{Phonetics}{募捐}{mu4/juan1}[][HSK 7-9]
    \definition[场,次]{v.+compl.}{solicitar contribuições; arrecadar doações; passar o chapéu}
  \end{Phonetics}
\end{Entry}

%%%%%%%%%% 博 %%%%%%%%%%
\subsection*{博}\addcontentsline{loh}{figure}{博}

\begin{Entry}{博}{12}{⼗}
  \begin{Phonetics}{博}{bo2}
    \definition*{s.}{Sobrenome: Bo}
    \definition{adj.}{rico; abundante | erudito; bem informado | solto; grande | grande}
    \definition{s.}{doutor em filosofia; doutorado}
    \definition{v.}{ter um amplo conhecimento de; ser bem lido | ganhar; vencer | jogar}
  \end{Phonetics}
\end{Entry}

\begin{Entry}{博士}{12,3}{⼗,⼠}
  \begin{Phonetics}{博士}{bo2shi4}[][HSK 5]
    \definition[位,名,个,些]{s.}{doutorado; grau de doutor; nível mais alto de um diploma; também, uma pessoa que obteve esse diploma | doutor; antigo título honorífico para uma pessoa que é habilidosa em um determinado ofício ou especializada em uma determinada ocupação | doutor; autoridades que ensinavam as escrituras na China nos tempos antigos}
  \end{Phonetics}
\end{Entry}

\begin{Entry}{博文}{12,4}{⼗,⽂}
  \begin{Phonetics}{博文}{bo2wen2}
    \definition{s.}{artigo em um blog}
    \definition{v.}{escrever um artigo em um blog}
  \end{Phonetics}
\end{Entry}

\begin{Entry}{博主}{12,5}{⼗,⼂}
  \begin{Phonetics}{博主}{bo2zhu3}
    \definition{s.}{blogueiro}
  \end{Phonetics}
\end{Entry}

\begin{Entry}{博物馆}{12,8,11}{⼗,⽜,⾷}
  \begin{Phonetics}{博物馆}{bo2wu4guan3}[][HSK 5]
    \definition[座,个]{s.}{museu; locais para coleta, armazenamento, pesquisa, exibição e exposição de relíquias culturais ou espécimes relacionados à história, cultura, arte, ciências naturais, ciência e tecnologia, etc.}
  \end{Phonetics}
\end{Entry}

\begin{Entry}{博客}{12,9}{⼗,⼧}
  \begin{Phonetics}{博客}{bo2ke4}[][HSK 5]
    \definition[个]{s.}{\emph{blog}; página da Web ou site gerenciado por um indivíduo, geralmente composto por postagens organizadas da mais recente para a mais antiga | blogueiro; \emph{blogger}; pessoas que possuem ou escrevem \emph{blogs}}
  \end{Phonetics}
\end{Entry}

\begin{Entry}{博览会}{12,9,6}{⼗,⾒,⼈}
  \begin{Phonetics}{博览会}{bo2lan3hui4}[][HSK 5]
    \definition[次,届]{s.}{exposição; feira internacional; exposições de produtos em grande escala}
  \end{Phonetics}
\end{Entry}

%%%%%%%%%% 厨 %%%%%%%%%%
\subsection*{厨}\addcontentsline{loh}{figure}{厨}

\begin{Entry}{厨}{12}{⼚}
  \begin{Phonetics}{厨}{chu2}
    \definition[个]{s.}{cozinha}
  \end{Phonetics}
\end{Entry}

\begin{Entry}{厨师}{12,6}{⼚,⼱}
  \begin{Phonetics}{厨师}{chu2shi1}[][HSK 6]
    \definition[名,位,个]{s.}{chefe de cozinha; cozinheiro; alguém que é bom em cozinhar e faz disso uma profissão}
  \end{Phonetics}
\end{Entry}

\begin{Entry}{厨房}{12,8}{⼚,⼾}
  \begin{Phonetics}{厨房}{chu2fang2}[][HSK 5]
    \definition[间,个]{s.}{cozinha}
  \end{Phonetics}
\end{Entry}

%%%%%%%%%% 喂 %%%%%%%%%%
\subsection*{喂}\addcontentsline{loh}{figure}{喂}

\begin{Entry}{喂}{12}{⼝}
  \begin{Phonetics}{喂}{wei4}[][HSK 2,4]
    \definition{interj.}{``Ei!'', ``Olá!'', para chamar atenção | ``Alô?'' (quando respondendo uma chamada telefônica, pronuncia"-se como \dpy{wei2})}
    \definition{v.}{criar; alimentar (animais); dar comida a um animal | alimentar (pessoas); colocar alimentos, medicamentos, etc. na boca de alguém}
  \end{Phonetics}
\end{Entry}

\begin{Entry}{喂奶}{12,5}{⼝,⼥}
  \begin{Phonetics}{喂奶}{wei4nai3}
    \definition{v.}{amamentar}
  \end{Phonetics}
\end{Entry}

\begin{Entry}{喂母乳}{12,5,8}{⼝,⽏,⼄}
  \begin{Phonetics}{喂母乳}{wei4mu3ru3}
    \definition{s.}{amamentação}
  \end{Phonetics}
\end{Entry}

\begin{Entry}{喂养}{12,9}{⼝,⼋}
  \begin{Phonetics}{喂养}{wei4yang3}
    \definition{v.}{alimentar (uma criança, animal doméstico, etc.) | manter | criar (um animal)}
  \end{Phonetics}
\end{Entry}

\begin{Entry}{喂食}{12,9}{⼝,⾷}
  \begin{Phonetics}{喂食}{wei4shi2}
    \definition{v.}{alimentar}
  \end{Phonetics}
\end{Entry}

\begin{Entry}{喂哺}{12,10}{⼝,⼝}
  \begin{Phonetics}{喂哺}{wei4bu3}
    \definition{v.}{alimentar (um bebê)}
  \end{Phonetics}
\end{Entry}

\begin{Entry}{喂料}{12,10}{⼝,⽃}
  \begin{Phonetics}{喂料}{wei4liao4}
    \definition{v.}{alimentar (também no sentido figurativo)}
  \end{Phonetics}
\end{Entry}

%%%%%%%%%% 善 %%%%%%%%%%
\subsection*{善}\addcontentsline{loh}{figure}{善}

\begin{Entry}{善}{12}{⼝}
  \begin{Phonetics}{善}{shan4}[][HSK 7-9]
    \definition*{s.}{Sobrenome: Shan}
    \definition{adj.}{bom; bem | bom; satisfatório | gentil; amigável | familiar}
    \definition{adv.}{bom; bem}
    \definition{s.}{boa ação; ato benevolente; coisas boas}
    \definition{v.}{fazer sucesso; fazer bem; fazer acontecer | ser bom em; ser especialista (versado) em | ser apto a}
  \antonymref{恶}{e4}
  \end{Phonetics}
\end{Entry}

\begin{Entry}{善于}{12,3}{⼝,⼆}
  \begin{Phonetics}{善于}{shan4yu2}[][HSK 4]
    \definition{adv./v.}{ser bom em; ser hábil em}
  \end{Phonetics}
\end{Entry}

\begin{Entry}{善良}{12,7}{⼝,⾉}
  \begin{Phonetics}{善良}{shan4liang2}[][HSK 4]
    \definition{adj.}{de bom coração; bom e honesto; de bom coração e cheio de boa vontade}
  \end{Phonetics}
\end{Entry}

\begin{Entry}{善意}{12,13}{⼝,⼼}
  \begin{Phonetics}{善意}{shan4yi4}[][HSK 7-9]
    \definition[片]{s.}{boa vontade;  boa intenção;  benevolência; bondade}
  \end{Phonetics}
\end{Entry}

%%%%%%%%%% 喇 %%%%%%%%%%
\subsection*{喇}\addcontentsline{loh}{figure}{喇}

\begin{Entry}{喇}{12}{⼝}
  \begin{Phonetics}{喇}{la1}
    \definition{s.}{Onomatopéia: som do vento, da chuva etc.}
  \end{Phonetics}
  \begin{Phonetics}{喇}{la2}
    \definition{v.}{babar; ter saliva escorrendo}
  \end{Phonetics}
  \begin{Phonetics}{喇}{la3}
    \definition{s.}{Fonético: la}
  \end{Phonetics}
\end{Entry}

\begin{Entry}{喇叭}{12,5}{⼝,⼝}
  \begin{Phonetics}{喇叭}{la3ba1}[][HSK 7-9]
    \definition[只,个]{s.}{trompete; instrumento de sopro de metal. instrumento musical, estreito na parte superior e largo na parte inferior, com uma grande abertura redonda na extremidade inferior; o som é produzido soprando na parte superior | alto-falante; dispositivos que podem amplificar o som}
  \end{Phonetics}
\end{Entry}

%%%%%%%%%% 喉 %%%%%%%%%%
\subsection*{喉}\addcontentsline{loh}{figure}{喉}

\begin{Entry}{喉}{12}{⼝}
  \begin{Phonetics}{喉}{hou2}
    \definition{s.}{laringe; garganta; a parte do órgão respiratório de humanos e vertebrados terrestres, localizada entre a faringe e a traqueia, tem as funções de ventilação e pronúncia; a faringe e a laringe são geralmente misturadas e chamadas de garganta ou caixa vocal}
  \end{Phonetics}
\end{Entry}

\begin{Entry}{喉咙}{12,8}{⼝,⼝}
  \begin{Phonetics}{喉咙}{hou2long2}[][HSK 7-9]
    \definition{s.}{garganta; laringe}
  \end{Phonetics}
\end{Entry}

%%%%%%%%%% 喊 %%%%%%%%%%
\subsection*{喊}\addcontentsline{loh}{figure}{喊}

\begin{Entry}{喊}{12}{⼝}
  \begin{Phonetics}{喊}{han3}[][HSK 2]
    \definition{v.}{gritar; clamar; berrar | chamar (uma pessoa) | chamar; dirigir-se a}
  \end{Phonetics}
\end{Entry}

%%%%%%%%%% 喔 %%%%%%%%%%
\subsection*{喔}\addcontentsline{loh}{figure}{喔}

\begin{Entry}{喔}{12}{⼝}
  \begin{Phonetics}{喔}{o1}
    \definition{interj.}{``Oh!'', ``Entendi!'', usado para indicar realização, compreensão}
  \end{Phonetics}
\end{Entry}

%%%%%%%%%% 喘 %%%%%%%%%%
\subsection*{喘}\addcontentsline{loh}{figure}{喘}

\begin{Entry}{喘}{12}{⼝}
  \begin{Phonetics}{喘}{chuan3}[][HSK 7-9]
    \definition{s.}{Medicina: asma}
    \definition{v.}{respirar pesadamente; ofegar por ar; ofegar | sopro; respiração rápida}
  \end{Phonetics}
\end{Entry}

\begin{Entry}{喘息}{12,10}{⼝,⼼}
  \begin{Phonetics}{喘息}{chuan3xi1}[][HSK 7-9]
    \definition{s.}{sopro; respiração rápida |respirador; pausas curtas durante atividades intensas | síndrome caracterizada por dispneia | (ponto de acupuntura) chuanxi}
    \definition{v.}{ofegar; ofegar por ar | fazer uma pausa para respirar; fazer uma pausa}
  \end{Phonetics}
\end{Entry}

%%%%%%%%%% 喜 %%%%%%%%%%
\subsection*{喜}\addcontentsline{loh}{figure}{喜}

\begin{Entry}{喜}{12}{⼝}
  \begin{Phonetics}{喜}{xi3}
    \definition{adj.}{feliz; satisfeito; encantado}
    \definition[桩,件]{s.}{evento feliz (especialmente casamento); ocasião para celebração; algo para comemorar | gravidez | casamento ou coisas relacionadas a ele}
    \definition{v.}{gostar; fonte de; ter inclinação para | precisa; requer; combina melhor com; (um certo organismo) precisa ou é adequado para (um certo ambiente ou algo)}
  \end{Phonetics}
\end{Entry}

\begin{Entry}{喜欢}{12,6}{⼝,⽋}
  \begin{Phonetics}{喜欢}{xi3huan5}[][HSK 1]
    \definition{adj.}{feliz; encantado; exultante; cheio de alegria}
    \definition{v.}{gostar; amar; ter afeição por; estar interessado em; ter uma boa impressão ou interesse por alguém ou algo}
  \end{Phonetics}
\end{Entry}

\begin{Entry}{喜剧}{12,10}{⼝,⼑}
  \begin{Phonetics}{喜剧}{xi3ju4}[][HSK 5]
    \definition[部,出]{s.}{comédia | comédia; uma das principais categorias do teatro; usa o exagero para satirizar e ridicularizar o feio; fenômenos retrógrados; destaca as contradições inerentes a esses fenômenos e seu conflito com coisas saudáveis; costuma provocar risadas; o final geralmente é feliz}
  \antonymref{悲剧}{bei1ju4}
  \end{Phonetics}
\end{Entry}

\begin{Entry}{喜爱}{12,10}{⼝,⽖}
  \begin{Phonetics}{喜爱}{xi3'ai4}[][HSK 4]
    \definition{v.}{gostar; amar; ter afeição por; estar interessado em; ter uma queda ou sentir interesse por pessoas ou coisas}
  \end{Phonetics}
\end{Entry}

%%%%%%%%%% 喝 %%%%%%%%%%
\subsection*{喝}\addcontentsline{loh}{figure}{喝}

\begin{Entry}{喝}{12}{⼝}
  \begin{Phonetics}{喝}{he1}[][HSK 1]
    \definition{interj.}{``Meu Deus!''; ``Oh!''; ``Ah!''; ``Uau!''}
    \definition{s.}{bebida; especificamente, vinho}
    \definition{v.}{beber; engolir líquidos ou alimentos líquidos | beber bebida alcoólica; referência específica ao consumo de álcool}
  \end{Phonetics}
  \begin{Phonetics}{喝}{he4}
    \definition{v.}{gritar bem alto}
  \end{Phonetics}
\end{Entry}

\begin{Entry}{喝采}{12,8}{⼝,⾤}
  \begin{Phonetics}{喝采}{he4/cai3}
    \definition{v.+compl.}{aclamar; aplaudir}
  \end{Phonetics}
\end{Entry}

\begin{Entry}{喝彩}{12,11}{⼝,⼺}
  \begin{Phonetics}{喝彩}{he4cai3}
    \definition{s.}{aclamar | torcer}
  \end{Phonetics}
\end{Entry}

\begin{Entry}{喝醉}{12,15}{⼝,⾣}
  \begin{Phonetics}{喝醉}{he1zui4}
    \definition{v.}{ficar bêbado}
  \end{Phonetics}
\end{Entry}

%%%%%%%%%% 喷 %%%%%%%%%%
\subsection*{喷}\addcontentsline{loh}{figure}{喷}

\begin{Entry}{喷}{12}{⼝}
  \begin{Phonetics}{喷}{pen1}[][HSK 5]
    \definition{v.}{jorrar; esguichar; expelir sob pressão | borrifar; espalhar; pulverizar}
  \end{Phonetics}
  \begin{Phonetics}{喷}{pen4}
    \definition{s.}{na época; tempo no mercado; época em que frutas, peixes e camarões são comercializados em grande quantidade | colheita; número de vezes que floresceu e frutificou; número de vezes que foi colhido na maturação}
  \end{Phonetics}
\end{Entry}

\begin{Entry}{喷泉}{12,9}{⼝,⽔}
  \begin{Phonetics}{喷泉}{pen1quan2}[][HSK 7-9]
    \definition[个,处,注]{s.}{fonte; fonte que jorra água}
  \end{Phonetics}
\end{Entry}

%%%%%%%%%% 喻 %%%%%%%%%%
\subsection*{喻}\addcontentsline{loh}{figure}{喻}

\begin{Entry}{喻}{12}{⼝}
  \begin{Phonetics}{喻}{yu4}
    \definition{s.}{analogia | símile | metáfora | alegoria}
    \definition{v.}{descrever algo como}
  \end{Phonetics}
\end{Entry}

%%%%%%%%%% 堡 %%%%%%%%%%
\subsection*{堡}\addcontentsline{loh}{figure}{堡}

\begin{Entry}{堡}{12}{⼟}
  \begin{Phonetics}{堡}{bao3}
    \definition{s.}{forte; fortaleza | uma terraplenagem | castelo | posição de defesa | usado em nomes de lugares}
  \end{Phonetics}
  \begin{Phonetics}{堡}{bu3}
    \definition{s.}{forte; vila; cidade; assentamento fortificado (uma vila ou cidade cercada por muros de terra, frequentemente usada em nomes de lugares) | cidade; frequentemente usado em nomes de lugares}
  \end{Phonetics}
  \begin{Phonetics}{堡}{pu4}
    \definition{s.}{cidade ou rua (frequentemente usado em nomes de lugares)}
  \end{Phonetics}
\end{Entry}

\begin{Entry}{堡垒}{12,9}{⼟,⼟}
  \begin{Phonetics}{堡垒}{bao3lei3}[][HSK 7-9]
    \definition[处,座,个]{s.}{forte; fortaleza; casamata | bastião | fortificação}
  \end{Phonetics}
\end{Entry}

%%%%%%%%%% 堤 %%%%%%%%%%
\subsection*{堤}\addcontentsline{loh}{figure}{堤}

\begin{Entry}{堤}{12}{⼟}
  \begin{Phonetics}{堤}{di1}[][HSK 7-9]
    \definition[道,条]{s.}{dique; aterro}
  \end{Phonetics}
\end{Entry}

\begin{Entry}{堤坝}{12,7}{⼟,⼟}
  \begin{Phonetics}{堤坝}{di1ba4}[][HSK 7-9]
    \definition[座,道,个]{s.}{diques; barragem; represa}
  \end{Phonetics}
\end{Entry}

%%%%%%%%%% 堪 %%%%%%%%%%
\subsection*{堪}\addcontentsline{loh}{figure}{堪}

\begin{Entry}{堪}{12}{⼟}
  \begin{Phonetics}{堪}{kan1}
    \definition*{s.}{Sobrenome: Kan}
    \definition{v.}{pode; consegue | suportar; resistir; aguentar}
  \end{Phonetics}
\end{Entry}

\begin{Entry}{堪称}{12,10}{⼟,⽲}
  \begin{Phonetics}{堪称}{kan1cheng1}[][HSK 7-9]
    \definition{v.}{pode ser classificado como; pode ser chamado assim; merece ser chamado assim}
  \end{Phonetics}
\end{Entry}

%%%%%%%%%% 塔 %%%%%%%%%%
\subsection*{塔}\addcontentsline{loh}{figure}{塔}

\begin{Entry}{塔}{12}{⼟}
  \begin{Phonetics}{塔}{ta3}[][HSK 6]
    \definition*{s.}{Sobrenome: Ta}
    \definition[个,座]{s.}{pagode budista; pagode | torre | (química) coluna; torre}[蒸馏塔===torre de destilação]
  \end{Phonetics}
\end{Entry}

%%%%%%%%%% 奠 %%%%%%%%%%
\subsection*{奠}\addcontentsline{loh}{figure}{奠}

\begin{Entry}{奠}{12}{⼤}
  \begin{Phonetics}{奠}{dian4}
    \definition{v.}{estabelecer; construir; fundar; lançar a pedra fundamental | fazer oferendas aos espíritos dos mortos | consertar}
  \end{Phonetics}
\end{Entry}

\begin{Entry}{奠定}{12,8}{⼤,⼧}
  \begin{Phonetics}{奠定}{dian4ding4}[][HSK 7-9]
    \definition{v.}{estabelecer; estabelecer de forma estável; tornar estável; estabelecer a base}
  \end{Phonetics}
\end{Entry}

%%%%%%%%%% 奥 %%%%%%%%%%
\subsection*{奥}\addcontentsline{loh}{figure}{奥}

\begin{Entry}{奥}{12}{⼤}
  \begin{Phonetics}{奥}{ao4}
    \definition*{s.}{Oersted, a unidade eletromagnética de intensidade do campo magnético; abreviação de 奥斯特 | Sobrenome: Ao}
    \definition{adj.}{profundo e difícil de entender; abstruso | significado profundo, não é fácil de entender}
    \definition{s.}{canto secreto da casa; antigamente, referia"-se ao canto sudoeste de uma casa e também, de modo geral, à profundidade de uma casa}
  \seealsoref{奥斯特}{ao4 si1 te4}
  \end{Phonetics}
\end{Entry}

\begin{Entry}{奥运}{12,7}{⼤,⾡}
  \begin{Phonetics}{奥运}{ao4yun4}
    \definition*[届,次]{s.}{Jogos Olímpicos, Olimpíadas; abreviação de 奥林匹克运动会}
  \seealsoref{奥林匹克运动会}{ao4lin2pi3ke4 yun4dong4hui4}
  \end{Phonetics}
\end{Entry}

\begin{Entry}{奥运会}{12,7,6}{⼤,⾡,⼈}
  \begin{Phonetics}{奥运会}{ao4yun4hui4}[][HSK 7-9]
    \definition*[届,次]{s.}{Jogos Olímpicos, Olimpíadas; abreviação de 奥林匹克运动会}
  \seealsoref{奥林匹克运动会}{ao4lin2pi3ke4 yun4dong4hui4}
  \end{Phonetics}
\end{Entry}

\begin{Entry}{奥林匹克运动会}{12,8,4,7,7,6,6}{⼤,⽊,⼖,⼗,⾡,⼒,⼈}
  \begin{Phonetics}{奥林匹克运动会}{ao4lin2pi3ke4 yun4dong4hui4}
    \definition*{s.}{Jogos Olímpicos, Olimpíadas}
  \end{Phonetics}
\end{Entry}

\begin{Entry}{奥特曼}{12,10,11}{⼤,⽜,⽈}
  \begin{Phonetics}{奥特曼}{ao4te4man4}
    \definition*{s.}{Ultraman,  super-herói de ficção científica japonesa}
  \end{Phonetics}
\end{Entry}

\begin{Entry}{奥秘}{12,10}{⼤,⽲}
  \begin{Phonetics}{奥秘}{ao4mi4}[][HSK 7-9]
    \definition[个]{s.}{enigma; mistério profundo; fenômenos ou princípios profundos e misteriosos}
  \end{Phonetics}
\end{Entry}

\begin{Entry}{奥斯特}{12,12,10}{⼤,⽄,⽜}
  \begin{Phonetics}{奥斯特}{ao4 si1 te4}
    \definition{s.}{Oersted}
  \end{Phonetics}
\end{Entry}

%%%%%%%%%% 媒 %%%%%%%%%%
\subsection*{媒}\addcontentsline{loh}{figure}{媒}

\begin{Entry}{媒}{12}{⼥}
  \begin{Phonetics}{媒}{mei2}
    \definition{s.}{casamenteiro; intermediário | intermediário; médio}
    \definition{v.}{fazer uma combinação}
  \end{Phonetics}
\end{Entry}

\begin{Entry}{媒体}{12,7}{⼥,⼈}
  \begin{Phonetics}{媒体}{mei2ti3}[][HSK 3]
    \definition[家,个,种]{s.}{mídia; mídia de massa; vários meios de comunicação e transmissão de informações, como televisão, rádio, jornais, etc.}
  \end{Phonetics}
\end{Entry}

%%%%%%%%%% 嫂 %%%%%%%%%%
\subsection*{嫂}\addcontentsline{loh}{figure}{嫂}

\begin{Entry}{嫂}{12}{⼥}
  \begin{Phonetics}{嫂}{sao3}
    \definition[个,位,名,些]{s.}{esposa do irmão mais velho; cunhada | irmã (uma forma de tratamento para uma mulher casada, mais ou menos da mesma idade)}
  \end{Phonetics}
\end{Entry}

\begin{Entry}{嫂子}{12,3}{⼥,⼦}
  \begin{Phonetics}{嫂子}{sao3zi5}[][HSK 7-9]
    \definition[个,名,位]{s.}{esposa do irmão mais velho; cunhada}
  \end{Phonetics}
\end{Entry}

%%%%%%%%%% 富 %%%%%%%%%%
\subsection*{富}\addcontentsline{loh}{figure}{富}

\begin{Entry}{富}{12}{⼧}
  \begin{Phonetics}{富}{fu4}[][HSK 3]
    \definition*{s.}{Sobrenome: Fu}
    \definition{adj.}{rico; abastado; abundante; refere"-se a ter muito dinheiro | rico; abundante}
    \definition{v.}{tornar"-se rico; enriquecer}
  \antonymref{贫}{pin2}
  \end{Phonetics}
\end{Entry}

\begin{Entry}{富人}{12,2}{⼧,⼈}
  \begin{Phonetics}{富人}{fu4ren2}[][HSK 6]
    \definition{s.}{os ricos; os abastados}
  \end{Phonetics}
\end{Entry}

\begin{Entry}{富有}{12,6}{⼧,⽉}
  \begin{Phonetics}{富有}{fu4you3}[][HSK 6]
    \definition{adj.}{rico; abastado; possuir uma grande quantidade de propriedades | rico em espírito; metáfora para uma vida espiritual rica}
    \definition{v.}{ser rico ou abundante em; principalmente referindo"-se a coisas abstratas com significados positivos que são suficientes}
  \end{Phonetics}
\end{Entry}

\begin{Entry}{富含}{12,7}{⼧,⼝}
  \begin{Phonetics}{富含}{fu4han2}[][HSK 7-9]
    \definition{v.}{ser rico em}[橘子富含维生素。===As laranjas são ricas em vitaminas.]
  \end{Phonetics}
\end{Entry}

\begin{Entry}{富足}{12,7}{⼧,⾜}
  \begin{Phonetics}{富足}{fu4zu2}[][HSK 7-9]
    \definition{adj.}{rico; pleno; abundante}
  \end{Phonetics}
\end{Entry}

\begin{Entry}{富翁}{12,10}{⼧,⽺}
  \begin{Phonetics}{富翁}{fu4weng1}[][HSK 7-9]
    \definition[个,名,位]{s.}{homem rico; homem de riqueza; pessoas que possuem muitas propriedades}
  \end{Phonetics}
\end{Entry}

\begin{Entry}{富强}{12,12}{⼧,⼸}
  \begin{Phonetics}{富强}{fu4qiang2}[][HSK 7-9]
    \definition{adj.}{próspero e forte; próspero e poderoso; rico e poderoso; (país) rico e poderoso}
  \end{Phonetics}
\end{Entry}

\begin{Entry}{富裕}{12,12}{⼧,⾐}
  \begin{Phonetics}{富裕}{fu4yu4}[][HSK 7-9]
    \definition{adj.}{rico; próspero; abastado; em boas condições econômicas e com dinheiro suficiente}
  \end{Phonetics}
\end{Entry}

\begin{Entry}{富豪}{12,14}{⼧,⾗}
  \begin{Phonetics}{富豪}{fu4hao2}[][HSK 7-9]
    \definition{s.}{rico; pessoa com muito dinheiro e grande poder}
  \end{Phonetics}
\end{Entry}

%%%%%%%%%% 寒 %%%%%%%%%%
\subsection*{寒}\addcontentsline{loh}{figure}{寒}

\begin{Entry}{寒}{12}{⼧}
  \begin{Phonetics}{寒}{han2}
    \definition*{s.}{Sobrenome: Han}
    \definition{adj.}{frio | pobre; necessitado | (autodepreciativo) meu/minha humilde\dots | assustado; medroso | com medo; tremendo (de medo) | humilde}
    \definition{s.}{estação fria; inverno | (medicina chinesa) sintomas causados por fatores frios}
  \antonymref{暑}{shu3}
  \end{Phonetics}
\end{Entry}

\begin{Entry}{寒冷}{12,7}{⼧,⼎}
  \begin{Phonetics}{寒冷}{han2leng3}[][HSK 4]
    \definition[度,阵,股]{adj.}{frio; frígido; gélido; gelado}
  \end{Phonetics}
\end{Entry}

\begin{Entry}{寒假}{12,11}{⼧,⼈}
  \begin{Phonetics}{寒假}{han2jia4}[][HSK 4]
    \definition[个,段]{s.}{férias de inverno (feriados); férias escolares no meio do inverno, em janeiro e fevereiro (na China)}
  \end{Phonetics}
\end{Entry}

%%%%%%%%%% 寓 %%%%%%%%%%
\subsection*{寓}\addcontentsline{loh}{figure}{寓}

\begin{Entry}{寓}{12}{⼧}
  \begin{Phonetics}{寓}{yu4}
    \definition[座,间,栋]{s.}{residência; morada}
    \definition{v.}{(literário) residir; viver | implicar; conter}
  \end{Phonetics}
\end{Entry}

\begin{Entry}{寓意}{12,13}{⼧,⼼}
  \begin{Phonetics}{寓意}{yu4yi4}
    \definition{s.}{moral (de uma história),  lição a ser aprendida, implicação, mensagem, significado metafórico}
  \end{Phonetics}
\end{Entry}

%%%%%%%%%% 尊 %%%%%%%%%%
\subsection*{尊}\addcontentsline{loh}{figure}{尊}

\begin{Entry}{尊}{12}{⼨}
  \begin{Phonetics}{尊}{zun1}
    \definition*{s.}{Sobrenome: Zun}
    \definition{adj.}{sênior; de uma geração sênior; alto status ou antiguidade}
    \definition{clas.}{usado para estátuas, canhões, etc.}
    \definition{pron.}{seu; vossa; antigamente, referia"-se a pessoas ou coisas relacionadas entre si}
    \definition{s.}{um tipo de recipiente para vinho usado nos tempos antigos}
    \definition{v.}{respeitar; reverenciar; venerar; honrar}
  \end{Phonetics}
\end{Entry}

\begin{Entry}{尊重}{12,9}{⼨,⾥}
  \begin{Phonetics}{尊重}{zun1zhong4}[][HSK 5]
    \definition{adj.}{sério; adequado; correto; (linguagem, comportamento) não ser descuidado; não ser leviano}
    \definition{v.}{respeitar; valorizar; estimar; tratar com educação; valorizar | tratar com seriedade; levar a sério e tratar com seriedade}
  \end{Phonetics}
\end{Entry}

\begin{Entry}{尊敬}{12,12}{⼨,⽁}
  \begin{Phonetics}{尊敬}{zun1jing4}[][HSK 5]
    \definition{adj.}{respeitoso; respeitável}
    \definition{v.}{respeitar; honrar; estimar}
  \end{Phonetics}
\end{Entry}

%%%%%%%%%% 就 %%%%%%%%%%
\subsection*{就}\addcontentsline{loh}{figure}{就}

\begin{Entry}{就}{12}{⼪}
  \begin{Phonetics}{就}{jiu4}[][HSK 1]
    \definition{adv.}{de imediato; imediatamente; indica que algo ocorrerá em breve | tão cedo quanto; já; há muito tempo; indica que a ação ocorreu há muito tempo | assim que; logo depois; indica que os eventos se sucedem imediatamente | nesse caso; então; indica que, sob determinadas condições, ocorre naturalmente um determinado resultado | exatamente; precisamente; indica que é exatamente assim | apenas; meramente; somente | tantos quanto; enfatiza a quantidade | apenas; simplesmente; reforço da afirmação | colocado entre dois componentes idênticos, significa tolerância ou indiferença}
    \definition{prep.}{tirar proveito de alguém (algo); expressa condições, oportunidades, etc., equivalente a 趁 | quando se trata de alguém (algo); relativo a; com relação a; sobre; objeto ou escopo da introdução da ação |no local; introduz o local próximo ao qual a ação ocorreu}
    \definition{v.}{ser comido com; ir com; pratos, frutas, etc., acompanhados de alimentos básicos ou bebidas alcoólicas | aproximar-se; mover-se em direção a | ir para; assumir; empreender; envolver-se em; entrar em | realizar; fazer | tirar proveito de; acomodar-se a; adequar-se; encaixar-se | assumir; começar a entrar ou a exercer | seguir; acompanhar}
  \seealsoref{趁}{chen4}
  \end{Phonetics}
\end{Entry}

\begin{Entry}{就业}{12,5}{⼪,⼀}
  \begin{Phonetics}{就业}{jiu4/ye4}[][HSK 3]
    \definition{v.+compl.}{conseguir um emprego; obter emprego; assumir uma ocupação; começar a trabalhar}
  \end{Phonetics}
\end{Entry}

\begin{Entry}{就可以了}{12,5,4,2}{⼪,⼝,⼈,⼅}
  \begin{Phonetics}{就可以了}{jiu4 ke3yi3le5}
    \definition{expr.}{é isso; é o suficiente}
  \end{Phonetics}
\end{Entry}

\begin{Entry}{就任}{12,6}{⼪,⼈}
  \begin{Phonetics}{就任}{jiu4ren4}[][HSK 7-9]
    \definition{v.}{assumir o cargo; tomar posse | assumir o próprio cargo}
  \end{Phonetics}
\end{Entry}

\begin{Entry}{就地}{12,6}{⼪,⼟}
  \begin{Phonetics}{就地}{jiu4di4}[][HSK 7-9]
    \definition{adv.}{no local; no próprio local | localmente}
  \end{Phonetics}
\end{Entry}

\begin{Entry}{就医}{12,7}{⼪,⼖}
  \begin{Phonetics}{就医}{jiu4/yi1}[][HSK 7-9]
    \definition{v.+compl.}{consultar um médico; ir ao médico; buscar aconselhamento médico}
  \end{Phonetics}
\end{Entry}

\begin{Entry}{就坐}{12,7}{⼪,⼟}
  \begin{Phonetics}{就坐}{jiu4zuo4}
    \definition{v.}{sentar-se; estar sentado}
  \end{Phonetics}
\end{Entry}

\begin{Entry}{就诊}{12,7}{⼪,⾔}
  \begin{Phonetics}{就诊}{jiu4/zhen3}[][HSK 7-9]
    \definition{v.+compl.}{consultar um médico; procurar aconselhamento médico}
  \end{Phonetics}
\end{Entry}

\begin{Entry}{就近}{12,7}{⼪,⾡}
  \begin{Phonetics}{就近}{jiu4jin4}[][HSK 7-9]
    \definition{adv.}{(fazer ou obter algo) nas proximidades; na vizinhança; sem ter que ir longe; significa que está por perto}
  \end{Phonetics}
\end{Entry}

\begin{Entry}{就是}{12,9}{⼪,⽇}
  \begin{Phonetics}{就是}{jiu4shi4}[][HSK 3]
    \definition{adv.}{exatamente; precisamente; expressar concordância com a afirmação da outra pessoa ou confirmar que a afirmação da outra pessoa está correta | apenas; simplesmente; expressa afirmação, determinação ou ênfase, o significado específico deve ser determinado com base no contexto anterior ou posterior | usado para indicar escolha}
    \definition{conj.}{ainda que; mesmo que se reconheça que essa situação é verdadeira, a situação posterior não mudará}
    \definition{part.}{usado no final de uma frase para expressar afirmação}
  \end{Phonetics}
\end{Entry}

\begin{Entry}{就是说}{12,9,9}{⼪,⽇,⾔}
  \begin{Phonetics}{就是说}{jiu4shi4shuo1}[][HSK 6]
    \definition{expr.}{ou seja; isto é; em outras palavras; é frequentemente usado como uma interjeição em uma frase para indicar que as palavras seguintes são uma explicação ou esclarecimento das anteriores}
  \end{Phonetics}
\end{Entry}

\begin{Entry}{就要}{12,9}{⼪,⾑}
  \begin{Phonetics}{就要}{jiu4yao4}[][HSK 2]
    \definition{adv.}{estar prestes a; estar indo para; estar no ponto de}
  \end{Phonetics}
\end{Entry}

\begin{Entry}{就座}{12,10}{⼪,⼴}
  \begin{Phonetics}{就座}{jiu4/zuo4}[][HSK 7-9]
    \definition{v.+compl.}{sentar-se; estar sentado | ocupar o próprio lugar (assento)}
  \seealsoref{就坐}{jiu4zuo4}
  \end{Phonetics}
\end{Entry}

\begin{Entry}{就读}{12,10}{⼪,⾔}
  \begin{Phonetics}{就读}{jiu4du2}[][HSK 7-9]
    \definition{v.}{fazer um curso; frequentar a escola; ir à escola; ingressar na escola}
  \end{Phonetics}
\end{Entry}

\begin{Entry}{就职}{12,11}{⼪,⽿}
  \begin{Phonetics}{就职}{jiu4/zhi2}
    \definition{v.+compl.}{assumir o cargo; assumir oficialmente um cargo (geralmente referindo"-se a uma posição de maior hierarquia)}
  \end{Phonetics}
\end{Entry}

\begin{Entry}{就算}{12,14}{⼪,⽵}
  \begin{Phonetics}{就算}{jiu4suan4}[][HSK 6]
    \definition{conj.}{mesmo que; concedido que; expressam uma relação hipotética e concessiva, frequentemente usadas com 也, equivalente a 即使}
  \seealsoref{即使}{ji2shi3}
  \seealsoref{也}{ye3}
  \end{Phonetics}
\end{Entry}

\begin{Entry}{就餐}{12,16}{⼪,⾷}
  \begin{Phonetics}{就餐}{jiu4can1}[][HSK 7-9]
    \definition{v.}{comer; jantar; fazer uma refeição; ir comer}
  \end{Phonetics}
\end{Entry}

%%%%%%%%%% 属 %%%%%%%%%%
\subsection*{属}\addcontentsline{loh}{figure}{属}

\begin{Entry}{属}{12}{⼫}
  \begin{Phonetics}{属}{shu3}[][HSK 3]
    \definition{s.}{categoria | gênero | membros da família; dependentes; familiares; parentes}
    \definition{v.}{estar sob; subordinado a | pertencer a | nascer no ano de (um dos doze animais do zodíaco)}
  \end{Phonetics}
  \begin{Phonetics}{属}{zhu3}
    \definition{v.}{juntar; combinar | fixar (a mente) em; centrar (a atenção, etc.) em}
  \end{Phonetics}
\end{Entry}

\begin{Entry}{属于}{12,3}{⼫,⼆}
  \begin{Phonetics}{属于}{shu3yu2}[][HSK 3]
    \definition{v.}{pertencer a; fazer parte de; pertencer ou ser propriedade de uma determinada parte}
  \end{Phonetics}
\end{Entry}

\begin{Entry}{属性}{12,8}{⼫,⼼}
  \begin{Phonetics}{属性}{shu3xing4}[][HSK 7-9]
    \definition{s.}{um atributo; uma propriedade; as propriedades inerentes das coisas; tudo possui múltiplos atributos, que podem ser categorizados em atributos essenciais e atributos não essenciais}
  \synonymref{本质}{ben3zhi4}
  \synonymref{特性}{te4xing4}
  \synonymref{特质}{te4zhi4}
  \end{Phonetics}
\end{Entry}

%%%%%%%%%% 屡 %%%%%%%%%%
\subsection*{屡}\addcontentsline{loh}{figure}{屡}

\begin{Entry}{屡}{12}{⼫}
  \begin{Phonetics}{屡}{lv3}[][HSK 7-9]
    \definition{adv.}{repetidamente; diversas vezes; muitas vezes}
  \end{Phonetics}
\end{Entry}

\begin{Entry}{屡次}{12,6}{⼫,⽋}
  \begin{Phonetics}{屡次}{lv3ci4}[][HSK 7-9]
    \definition{adv.}{repetidamente; várias e várias vezes; vez após vez}
  \end{Phonetics}
\end{Entry}

%%%%%%%%%% 帽 %%%%%%%%%%
\subsection*{帽}\addcontentsline{loh}{figure}{帽}

\begin{Entry}{帽}{12}{⼱}
  \begin{Phonetics}{帽}{mao4}
    \definition[个,顶]{s.}{chapéu; boné | capa; uma coisa que cobre um objeto e tem a função ou formato de um chapéu | elmo; capacete}
  \end{Phonetics}
\end{Entry}

\begin{Entry}{帽子}{12,3}{⼱,⼦}
  \begin{Phonetics}{帽子}{mao4zi5}[][HSK 4]
    \definition[顶,个,种]{s.}{boné; chapéu; capacete | etiqueta; rótulo; marca}
  \end{Phonetics}
\end{Entry}

%%%%%%%%%% 幅 %%%%%%%%%%
\subsection*{幅}\addcontentsline{loh}{figure}{幅}

\begin{Entry}{幅}{12}{⼱}
  \begin{Phonetics}{幅}{fu2}[][HSK 5]
    \definition{clas.}{usado para tecidos, telas de lã, pinturas, etc.}
    \definition{s.}{largura do tecido, seda, tweed, etc. | tamanho; largura; geralmente se refere à largura}
  \end{Phonetics}
\end{Entry}

\begin{Entry}{幅度}{12,9}{⼱,⼴}
  \begin{Phonetics}{幅度}{fu2du4}[][HSK 5]
    \definition{s.}{alcance; escopo; extensão; largura; largura da propagação de um objeto que vibra ou balança, uma metáfora para a magnitude de uma mudança em algo}
  \end{Phonetics}
\end{Entry}

%%%%%%%%%% 强 %%%%%%%%%%
\subsection*{强}\addcontentsline{loh}{figure}{强}

\begin{Entry}{强}{12}{⼸}
  \begin{Phonetics}{强}{jiang4}
    \definition{adj.}{teimoso; inflexível}
  \end{Phonetics}
  \begin{Phonetics}{强}{qiang2}[][HSK 3]
    \definition*{s.}{Sobrenome: Qiang}
    \definition{adj.}{forte; poderoso | melhor; superior | mais; extra; adicional; um pouco mais que; usado após uma fração ou decimal para indicar que é um pouco maior que o número | resoluto; firme | violento | alto padrão}
    \definition{v.}{fortalecer; tornar forte; tornar poderoso}
  \antonymref{弱}{ruo4}
  \end{Phonetics}
  \begin{Phonetics}{强}{qiang3}
    \definition{v.}{fazer um esforço; esforçar-se}
  \end{Phonetics}
\end{Entry}

\begin{Entry}{强大}{12,3}{⼸,⼤}
  \begin{Phonetics}{强大}{qiang2da4}[][HSK 3]
    \definition{adj.}{forte; poderoso; potente; possante; descreve força forte e grande poder}
  \end{Phonetics}
\end{Entry}

\begin{Entry}{强化}{12,4}{⼸,⼔}
  \begin{Phonetics}{强化}{qiang2hua4}[][HSK 6]
    \definition{v.}{intensificar; fortalecer; consolidar; tornar mais forte, melhorar sua habilidade e nível}
  \end{Phonetics}
\end{Entry}

\begin{Entry}{强加}{12,5}{⼸,⼒}
  \begin{Phonetics}{强加}{qiang2jia1}[][HSK 7-9]
    \definition{v.}{forçar; impor; forçar alguém a aceitar uma determinada opinião ou prática}
  \end{Phonetics}
\end{Entry}

\begin{Entry}{强占}{12,5}{⼸,⼘}
  \begin{Phonetics}{强占}{qiang2zhan4}[][HSK 7-9]
    \definition{v.}{ocupar à força; apoderar-se | ocupar à força; confiscar; anexar}
  \end{Phonetics}
\end{Entry}

\begin{Entry}{强壮}{12,6}{⼸,⼠}
  \begin{Phonetics}{强壮}{qiang2zhuang4}[][HSK 6]
    \definition{s.}{(corpo) forte, poderoso, robusto, resistente}
    \definition{v.}{fortalecer; construir}
  \end{Phonetics}
\end{Entry}

\begin{Entry}{强行}{12,6}{⼸,⾏}
  \begin{Phonetics}{强行}{qiang2xing2}[][HSK 7-9]
    \definition{adv.}{à força; com força}
  \end{Phonetics}
\end{Entry}

\begin{Entry}{强劲}{12,7}{⼸,⼒}
  \begin{Phonetics}{强劲}{qiang2jing4}[][HSK 7-9]
    \definition{adj.}{poderoso}
  \end{Phonetics}
\end{Entry}

\begin{Entry}{强制}{12,8}{⼸,⼑}
  \begin{Phonetics}{强制}{qiang2zhi4}[][HSK 7-9]
    \definition{v.}{forçar; compelir; coagir; utilizar o poder legal, político e econômico para forçar}
  \end{Phonetics}
\end{Entry}

\begin{Entry}{强势}{12,8}{⼸,⼒}
  \begin{Phonetics}{强势}{qiang2shi4}[][HSK 6]
    \definition*{adj.}{forte; poderoso; dominante}
    \definition{s.}{momento; ímpeto; grande impulso; forte impulso | força; influência dominante; forças poderosas}
  \end{Phonetics}
\end{Entry}

\begin{Entry}{强迫}{12,8}{⼸,⾡}
  \begin{Phonetics}{强迫}{qiang3po4}[][HSK 5]
    \definition{v.}{impelir; forçar; impor; compelir; aplicar pessão para obedecer}
  \end{Phonetics}
\end{Entry}

\begin{Entry}{强度}{12,9}{⼸,⼴}
  \begin{Phonetics}{强度}{qiang2du4}[][HSK 5]
    \definition[个,种]{s.}{intensidade; força | magnitude; rigor; avidez}
  \end{Phonetics}
\end{Entry}

\begin{Entry}{强项}{12,9}{⼸,⾴}
  \begin{Phonetics}{强项}{qiang2xiang4}[][HSK 7-9]
    \definition{adj.}{Literário: resoluto e inflexível; reto e inabalável}
    \definition{s.}{ponto forte (de um atleta ou de uma equipe) | jogo, evento ou assunto no qual alguém é forte | principal ponto forte | especialidade}
  \end{Phonetics}
\end{Entry}

\begin{Entry}{强烈}{12,10}{⼸,⽕}
  \begin{Phonetics}{强烈}{qiang2lie4}[][HSK 3]
    \definition{adj.}{muito forte; intenso; poderoso | violento; impetuoso; nível muito alto; atitude muito firme, sem espaço para mudanças | afiado; marcante; mostrado em contraste; muito claro}
  \end{Phonetics}
\end{Entry}

\begin{Entry}{强调}{12,10}{⼸,⾔}
  \begin{Phonetics}{强调}{qiang2diao4}[][HSK 3]
    \definition{v.}{salientar; sublinhar; enfatizar; dar ênfase a; vincar}
  \end{Phonetics}
\end{Entry}

\begin{Entry}{强盗}{12,11}{⼸,⽫}
  \begin{Phonetics}{强盗}{qiang2dao4}[][HSK 6]
    \definition[个,群,伙,帮]{s.}{ladrão; bandido; uma pessoa que usa violência para confiscar a propriedade de outros; também se refere a uma pessoa ou força que se envolve em comportamento semelhante}
  \end{Phonetics}
\end{Entry}

\begin{Entry}{强硬}{12,12}{⼸,⽯}
  \begin{Phonetics}{强硬}{qiang2ying4}[][HSK 7-9]
    \definition{adj.}{forte; resistente; inflexível; poderoso; não disposto a recuar}
  \end{Phonetics}
\end{Entry}

%%%%%%%%%% 循 %%%%%%%%%%
\subsection*{循}\addcontentsline{loh}{figure}{循}

\begin{Entry}{循}{12}{⼻}
  \begin{Phonetics}{循}{xun2}
    \definition{v.}{seguir; cumprir; cumprir com}
  \end{Phonetics}
\end{Entry}

\begin{Entry}{循环}{12,8}{⼻,⽟}
  \begin{Phonetics}{循环}{xun2huan2}[][HSK 6]
    \definition{s.}{ciclo; circulação}
    \definition{v.}{circular; as coisas se movem ou mudam em um ciclo}
  \end{Phonetics}
\end{Entry}

%%%%%%%%%% 悲 %%%%%%%%%%
\subsection*{悲}\addcontentsline{loh}{figure}{悲}

\begin{Entry}{悲}{12}{⽕}
  \begin{Phonetics}{悲}{bei1}
    \definition{adj.}{triste; pesaroso; melancólico | compassivo; misericordioso}
  \end{Phonetics}
\end{Entry}

\begin{Entry}{悲伤}{12,6}{⽕,⼈}
  \begin{Phonetics}{悲伤}{bei1shang1}[][HSK 5]
    \definition{adj.}{triste; pesaroso}
  \end{Phonetics}
\end{Entry}

\begin{Entry}{悲欢离合}{12,6,10,6}{⽕,⽋,⼇,⼝}
  \begin{Phonetics}{悲欢离合}{bei1huan1-li2he2}[][HSK 7-9]
    \definition{expr.}{alegrias e tristezas | separações e reencontros | as vicissitudes da vida}
  \end{Phonetics}
\end{Entry}

\begin{Entry}{悲观}{12,6}{⽕,⾒}
  \begin{Phonetics}{悲观}{bei1guan1}[][HSK 7-9]
    \definition{adj.}{pessimista; negativismo, falta de confiança no futuro}
  \antonymref{乐观}{le4guan1}
  \end{Phonetics}
\end{Entry}

\begin{Entry}{悲哀}{12,9}{⽕,⼝}
  \begin{Phonetics}{悲哀}{bei1'ai1}[][HSK 7-9]
    \definition{adj.}{triste}
    \definition{s.}{tristeza; refere"-se a coisas tristes e dolorosas}
  \end{Phonetics}
\end{Entry}

\begin{Entry}{悲剧}{12,10}{⽕,⼑}
  \begin{Phonetics}{悲剧}{bei1ju4}[][HSK 5]
    \definition[出,部]{s.}{tragédia; drama trágico; uma das principais categorias de teatro, caracterizada basicamente pela representação do conflito irreconciliável entre o protagonista e a realidade e seu final trágico | tragédia; evento triste; metáfora para encontro infeliz}
  \end{Phonetics}
\end{Entry}

\begin{Entry}{悲惨}{12,11}{⽕,⽕}
  \begin{Phonetics}{悲惨}{bei1can3}[][HSK 6]
    \definition{adj.}{trágico; miserável; extremamente doloroso e triste}
  \end{Phonetics}
\end{Entry}

\begin{Entry}{悲痛}{12,12}{⽕,⽧}
  \begin{Phonetics}{悲痛}{bei1tong4}[][HSK 7-9]
    \definition{adj.}{triste; aflito}
    \definition{s.}{tristeza; sofrimento}
  \end{Phonetics}
\end{Entry}

%%%%%%%%%% 惑 %%%%%%%%%%
\subsection*{惑}\addcontentsline{loh}{figure}{惑}

\begin{Entry}{惑}{12}{⼼}
  \begin{Phonetics}{惑}{huo4}
    \definition{v.}{ficar confuso; ficar perplexo | iludir; enganar; confundir}
  \end{Phonetics}
\end{Entry}

\begin{Entry}{惑星}{12,9}{⼼,⽇}
  \begin{Phonetics}{惑星}{huo4xing1}
    \definition{s.}{planeta}
  \seealsoref{行星}{xing2xing1}
  \end{Phonetics}
\end{Entry}

%%%%%%%%%% 惩 %%%%%%%%%%
\subsection*{惩}\addcontentsline{loh}{figure}{惩}

\begin{Entry}{惩}{12}{⼼}
  \begin{Phonetics}{惩}{cheng2}
    \definition{v.}{receber ou dar aviso | punir; penalizar}
  \end{Phonetics}
\end{Entry}

\begin{Entry}{惩处}{12,5}{⼼,⼡}
  \begin{Phonetics}{惩处}{cheng2chu3}[][HSK 7-9]
    \definition{v.}{penalizar; punir | punir; administrar justiça}
  \end{Phonetics}
\end{Entry}

\begin{Entry}{惩罚}{12,9}{⼼,⽹}
  \begin{Phonetics}{惩罚}{cheng2fa2}[][HSK 7-9]
    \definition[次,种]{s.}{punição; o ato ou método de punição}
    \definition{v.}{punir (severamente); penalizar}
  \end{Phonetics}
\end{Entry}

%%%%%%%%%% 惴 %%%%%%%%%%
\subsection*{惴}\addcontentsline{loh}{figure}{惴}

\begin{Entry}{惴}{12}{⼼}
  \begin{Phonetics}{惴}{zhui4}
    \definition{adj.}{ansioso e medroso; Literário: é ao mesmo tempo preocupante e assustador}
  \end{Phonetics}
\end{Entry}

%%%%%%%%%% 惶 %%%%%%%%%%
\subsection*{惶}\addcontentsline{loh}{figure}{惶}

\begin{Entry}{惶}{12}{⼼}
  \begin{Phonetics}{惶}{huang2}
    \definition{adj.}{cheio de medo; assustado}
    \definition{s.}{medo; pânico}
    \definition{v.}{temer}
  \end{Phonetics}
\end{Entry}

\begin{Entry}{惶恐}{12,10}{⼼,⼼}
  \begin{Phonetics}{惶恐}{huang2kong3}
    \definition{adj.}{aterrorizado; em pânico; petrificado | inquieto; apreensivo}
  \end{Phonetics}
\end{Entry}

%%%%%%%%%% 惹 %%%%%%%%%%
\subsection*{惹}\addcontentsline{loh}{figure}{惹}

\begin{Entry}{惹}{12}{⼼}
  \begin{Phonetics}{惹}{re3}[][HSK 7-9]
    \definition{v.}{provocar; convidar ou pedir (algo indesejável); (características de uma pessoa ou coisa) evocar sentimentos de amor ou ódio | ofender; provocar; instigar; (palavras e ações) tocar na outra pessoa | atrair; causar (algo ruim)}
  \end{Phonetics}
\end{Entry}

%%%%%%%%%% 愉 %%%%%%%%%%
\subsection*{愉}\addcontentsline{loh}{figure}{愉}

\begin{Entry}{愉}{12}{⼼}
  \begin{Phonetics}{愉}{yu2}
    \definition{adj.}{satisfeito; feliz; alegre}
  \end{Phonetics}
\end{Entry}

\begin{Entry}{愉快}{12,7}{⼼,⼼}
  \begin{Phonetics}{愉快}{yu2kuai4}[][HSK 6]
    \definition{adj.}{feliz; alegre; de bom humor, muito feliz}
  \end{Phonetics}
\end{Entry}

%%%%%%%%%% 愣 %%%%%%%%%%
\subsection*{愣}\addcontentsline{loh}{figure}{愣}

\begin{Entry}{愣}{12}{⼼}
  \begin{Phonetics}{愣}{leng4}[][HSK 7-9]
    \definition{adj.}{precipitado; rude; imprudente; temerário; falar e agir sem considerar as consequências}
    \definition{adv.}{insistir em algo de forma veemente, sem provas ou fundamentos}
    \definition{v.}{ficar entorpecido; ficar em branco; ficar distraído; ficar atordoado; ficar desatento; ficar perdido em pensamentos}
  \end{Phonetics}
\end{Entry}

%%%%%%%%%% 愤 %%%%%%%%%%
\subsection*{愤}\addcontentsline{loh}{figure}{愤}

\begin{Entry}{愤}{12}{⼼}
  \begin{Phonetics}{愤}{fen4}
    \definition{s.}{raiva; indignação; ressentimento; exasperação}
    \definition{v.}{ressentir-se; ficar indignado; ficar com raiva}
  \end{Phonetics}
\end{Entry}

\begin{Entry}{愤世嫉俗}{12,5,13,9}{⼼,⼀,⼥,⼈}
  \begin{Phonetics}{愤世嫉俗}{fen4shi4ji2su2}
    \definition{v.}{ser cínico | ser amargurado}
  \end{Phonetics}
\end{Entry}

\begin{Entry}{愤怒}{12,9}{⼼,⼼}
  \begin{Phonetics}{愤怒}{fen4nu4}[][HSK 6]
    \definition{adj.}{zangado; enraivecido; iracundo; furioso; emocionalmente agitado por extrema insatisfação}
  \end{Phonetics}
\end{Entry}

%%%%%%%%%% 慌 %%%%%%%%%%
\subsection*{慌}\addcontentsline{loh}{figure}{慌}

\begin{Entry}{慌}{12}{⼼}
  \begin{Phonetics}{慌}{huang1}[][HSK 5]
    \definition{adj.}{agitado; perturbado; confuso; que inspira terror}
    \definition{v.}{estar em estado de pânico; ficar com medo; ficar nervoso | estar com pressa}
  \end{Phonetics}
\end{Entry}

\begin{Entry}{慌忙}{12,6}{⼼,⼼}
  \begin{Phonetics}{慌忙}{huang1mang2}[][HSK 5]
    \definition{adj.}{apressado; afobado; com muita pressa}
    \definition{adv.}{apressadamente}
  \end{Phonetics}
\end{Entry}

\begin{Entry}{慌乱}{12,7}{⼼,⼄}
  \begin{Phonetics}{慌乱}{huang1luan4}[][HSK 7-9]
    \definition{adj.}{agitado; alarmado e confuso; em pânico e ocupado}
  \end{Phonetics}
\end{Entry}

\begin{Entry}{慌张}{12,7}{⼼,⼸}
  \begin{Phonetics}{慌张}{huang1zhang1}[][HSK 7-9]
    \definition{adj.}{em pânico; agitado; perturbado; confuso}
  \end{Phonetics}
\end{Entry}

%%%%%%%%%% 掌 %%%%%%%%%%
\subsection*{掌}\addcontentsline{loh}{figure}{掌}

\begin{Entry}{掌}{12}{⼿}
  \begin{Phonetics}{掌}{zhang3}
    \definition{s.}{palma da mão | sola do pé | pata | ferradura}
    \definition{v.}{dar um tapa | segurar na mão | empunhar}
  \end{Phonetics}
\end{Entry}

\begin{Entry}{掌声}{12,7}{⼿,⼠}
  \begin{Phonetics}{掌声}{zhang3sheng1}[][HSK 6]
    \definition[阵]{s.}{aplausos; palmas; o som dos aplausos}
  \end{Phonetics}
\end{Entry}

\begin{Entry}{掌握}{12,12}{⼿,⼿}
  \begin{Phonetics}{掌握}{zhang3wo4}[][HSK 5]
    \definition{v.}{compreender; dominar; conhecer bem; compreender as coisas; ser capaz de dominar ou utilizar plenamente | segurar; controlar; ter em mãos; tomar nas mãos}
  \end{Phonetics}
\end{Entry}

%%%%%%%%%% 掰 %%%%%%%%%%
\subsection*{掰}\addcontentsline{loh}{figure}{掰}

\begin{Entry}{掰}{12}{⼿}
  \begin{Phonetics}{掰}{bai1}[][HSK 7-9]
    \definition{v.}{separar ou quebrar coisas com as mãos | Dialeto: romper (relacionamento); cortar | Dialeto: analisar; estudar; examinar}
  \end{Phonetics}
\end{Entry}

%%%%%%%%%% 掱 %%%%%%%%%%
\subsection*{掱}\addcontentsline{loh}{figure}{掱}

\begin{Entry}{掱}{12}{⼿}
  \begin{Phonetics}{掱}{shou3}
    \variantof{手}
  \end{Phonetics}
\end{Entry}

%%%%%%%%%% 揉 %%%%%%%%%%
\subsection*{揉}\addcontentsline{loh}{figure}{揉}

\begin{Entry}{揉}{12}{⼿}
  \begin{Phonetics}{揉}{rou2}[][HSK 7-9]
    \definition{v.}{esfregar; esfregar ou friccionar com as mãos | amassar; enrolar | Literário: dobrar; torcer}
  \end{Phonetics}
\end{Entry}

\begin{Entry}{揉碎}{12,13}{⼿,⽯}
  \begin{Phonetics}{揉碎}{rou2sui4}
    \definition{v.}{desfazer-se em pedaços | esmagar}
  \end{Phonetics}
\end{Entry}

%%%%%%%%%% 提 %%%%%%%%%%
\subsection*{提}\addcontentsline{loh}{figure}{提}

\begin{Entry}{提}{12}{⼿}
  \begin{Phonetics}{提}{ti2}[][HSK 2]
    \definition*{s.}{Sobrenome: Ti}
    \definition{s.}{concha; utensílio para servir óleo ou vinho | traço ascendente (em caracteres chineses)}
    \definition{v.}{carregar (na mão, com o braço para baixo) ; segurar com as mãos para baixo | elevar; levantar; promover | avançar; antecipar uma data; mudar para uma data anterior; adiar o prazo previsto | levantar; apresentar; indicar ou citar | extrair; retirar (tirar) | (prisioneiros) trazer; entregar | mencionar; referir-se a; abordar}
  \end{Phonetics}
\end{Entry}

\begin{Entry}{提及}{12,3}{⼿,⼃}
  \begin{Phonetics}{提及}{ti2ji2}
    \definition{v.}{mencionar | levantar (um assunto) | chamar a atenção de alguém}
  \end{Phonetics}
\end{Entry}

\begin{Entry}{提升}{12,4}{⼿,⼗}
  \begin{Phonetics}{提升}{ti2sheng1}[][HSK 6]
    \definition{v.}{promover; avançar; melhorar (posição, grau, qualidade, etc.) | içar; elevar; transportar (minerais, materiais, etc.) para um local mais alto usando um guincho, etc.}
  \end{Phonetics}
\end{Entry}

\begin{Entry}{提心吊胆}{12,4,6,9}{⼿,⼼,⼝,⾁}
  \begin{Phonetics}{提心吊胆}{ti2xin1-diao4dan3}[][HSK 7-9]
    \definition{expr.}{nervoso; com o coração na boca; em constante medo; extremamente preocupado; em estado de ansiedade; estar em suspense; descreve um estado de grande preocupação ou medo}
  \seealsoref{悬心吊胆}{xuan2xin1-diao4dan3}
  \end{Phonetics}
\end{Entry}

\begin{Entry}{提出}{12,5}{⼿,⼐}
  \begin{Phonetics}{提出}{ti2 chu1}[][HSK 2]
    \definition{v.}{levantar; propor; apresentar; expressar seus desejos, ideias, sugestões, etc. por meio de palavras ou textos}
  \end{Phonetics}
\end{Entry}

\begin{Entry}{提示}{12,5}{⼿,⽰}
  \begin{Phonetics}{提示}{ti2shi4}[][HSK 5]
    \definition[个]{s.}{dica; lembrete; pistas ou informações fornecidas para chamar a atenção, fazer com que a outra pessoa pense ou compreenda}
    \definition{v.}{solicitar; lembrar; indicar; alertar; levantar questões que o outro não tenha pensado ou não tenha imaginado, para chamar a atenção dele}
  \end{Phonetics}
\end{Entry}

\begin{Entry}{提议}{12,5}{⼿,⾔}
  \begin{Phonetics}{提议}{ti2yi4}[][HSK 7-9]
    \definition[项,个]{s.}{proposta; moção}
    \definition{v.}{propor; sugerir; ao discutir assuntos, propor ideias para que todos possam debatê-las}
  \synonymref{倡导}{chang4dao3}
  \synonymref{倡议}{chang4yi4}
  \synonymref{发起}{fa1qi3}
  \synonymref{建议}{jian4yi4}
  \synonymref{提出}{ti2 chu1}
  \synonymref{提倡}{ti2chang4}
  \end{Phonetics}
\end{Entry}

\begin{Entry}{提交}{12,6}{⼿,⼇}
  \begin{Phonetics}{提交}{ti2jiao1}[][HSK 6]
    \definition{v.}{referir-se a; submeter (um problema, etc.) a; enviar questões que precisam ser discutidas, decididas ou tratadas para agências ou reuniões relevantes}
  \end{Phonetics}
\end{Entry}

\begin{Entry}{提名}{12,6}{⼿,⼝}
  \begin{Phonetics}{提名}{ti2/ming2}[][HSK 7-9]
    \definition{v.+compl.}{indicar; nomear; propor pessoas ou coisas que provavelmente serão eleitas em uma seleção ou eleição}
  \end{Phonetics}
\end{Entry}

\begin{Entry}{提早}{12,6}{⼿,⽇}
  \begin{Phonetics}{提早}{ti2zao3}[][HSK 7-9]
    \definition{v.}{antecipar"-se ao horário previsto; chegar mais cedo do que o planejado ou esperado}
  \synonymref{赶早}{gan3zao3}
  \synonymref{尽快}{jin3kuai4}
  \synonymref{尽早}{jin3zao3}
  \synonymref{提前}{ti2qian2}
  \antonymref{顺延}{shun4yan2}
  \antonymref{推迟}{tui1chi2}
  \end{Phonetics}
\end{Entry}

\begin{Entry}{提问}{12,6}{⼿,⾨}
  \begin{Phonetics}{提问}{ti2wen4}[][HSK 3]
    \definition{v.}{\emph{quiz}; fazer uma pergunta; colocar questões para}
  \end{Phonetics}
\end{Entry}

\begin{Entry}{提防}{12,6}{⼿,⾩}
  \begin{Phonetics}{提防}{di1fang5}[][HSK 7-9]
    \definition{v.}{proteger-se contra; ter cuidado com; tomar precauções contra; tomar cuidado}
  \end{Phonetics}
\end{Entry}

\begin{Entry}{提供}{12,8}{⼿,⼈}
  \begin{Phonetics}{提供}{ti2gong1}[][HSK 4]
    \definition{v.}{oferecer; fornecer; suprir; prover; proporcionar}
  \end{Phonetics}
\end{Entry}

\begin{Entry}{提到}{12,8}{⼿,⼑}
  \begin{Phonetics}{提到}{ti2dao4}[][HSK 2]
    \definition{v.}{mencionar; referir-se a; levantar (assunto)}
  \end{Phonetics}
\end{Entry}

\begin{Entry}{提拔}{12,8}{⼿,⼿}
  \begin{Phonetics}{提拔}{ti2ba2}[][HSK 7-9]
    \definition{v.}{1. promover; favorecer
Selecionar pessoal para assumir cargos mais importantes.}
  \synonymref{教育}{jiao4yu4}
  \synonymref{培育}{pei2yu4}
  \synonymref{提升}{ti2sheng1}
  \synonymref{选拔}{xuan3ba2}
  \end{Phonetics}
\end{Entry}

\begin{Entry}{提前}{12,9}{⼿,⼑}
  \begin{Phonetics}{提前}{ti2qian2}[][HSK 3]
    \definition{adv.}{antecipadamente; faça uma coisa antes de fazer outra}
    \definition{v.}{avançar; adiantar; mudar para uma data anterior; trazer para frente}
  \end{Phonetics}
\end{Entry}

\begin{Entry}{提炼}{12,9}{⼿,⽕}
  \begin{Phonetics}{提炼}{ti2lian4}[][HSK 7-9]
    \definition{v.}{extrair e purificar; abstrair; refinar | extrair}
  \antonymref{融合}{rong2he2}
  \end{Phonetics}
\end{Entry}

\begin{Entry}{提倡}{12,10}{⼿,⼈}
  \begin{Phonetics}{提倡}{ti2chang4}[][HSK 5]
    \definition{v.}{promover; incentivar; recomendar; apresentar as vantagens de algo para incentivar as pessoas a usá-lo ou implementá-lo}
  \end{Phonetics}
\end{Entry}

\begin{Entry}{提起}{12,10}{⼿,⾛}
  \begin{Phonetics}{提起}{ti2qi3}[][HSK 5]
    \definition{v.}{mencionar; falar sobre; abordar | levantar; despertar; estimular; revigorar; alegrar/animar | iniciar; instituir; propor | levantar; pegar}
  \end{Phonetics}
\end{Entry}

\begin{Entry}{提速}{12,10}{⼿,⾡}
  \begin{Phonetics}{提速}{ti2/su4}[][HSK 7-9]
    \definition{v.+compl.}{acelerar; aumentar a velocidade | para aumentar a velocidade de cruzeiro especificada | ganhar velocidade}
  \synonymref{加速}{jia1su4}
  \end{Phonetics}
\end{Entry}

\begin{Entry}{提高}{12,10}{⼿,⾼}
  \begin{Phonetics}{提高}{ti2/gao1}[][HSK 2]
    \definition{v.+compl.}{elevar; aprimorar; aumentar; melhorar a posição, o nível, a quantidade, a qualidade e outros aspectos em relação ao estado original}
  \end{Phonetics}
\end{Entry}

\begin{Entry}{提醒}{12,16}{⼿,⾣}
  \begin{Phonetics}{提醒}{ti2/xing3}[][HSK 4]
    \definition{v.+compl.}{alertar; avisar; advertir; lembrar; apontar para ou chamar a atenção para}
  \end{Phonetics}
\end{Entry}

%%%%%%%%%% 插 %%%%%%%%%%
\subsection*{插}\addcontentsline{loh}{figure}{插}

\begin{Entry}{插}{12}{⼿}
  \begin{Phonetics}{插}{cha1}[][HSK 5]
    \definition{v.}{perfurar; inserir | interpor; inserir; colocar no meio}
  \end{Phonetics}
\end{Entry}

\begin{Entry}{插手}{12,4}{⼿,⼿}
  \begin{Phonetics}{插手}{cha1/shou3}[][HSK 7-9]
    \definition{v.+compl.}{participar; dar uma mão | meter a mão em; meter o nariz em; intrometer"-se | ter (tomar) uma mão em}
  \end{Phonetics}
\end{Entry}

\begin{Entry}{插图}{12,8}{⼿,⼞}
  \begin{Phonetics}{插图}{cha1tu2}[][HSK 7-9]
    \definition[张,幅]{s.}{ilustração (artística ou científica) | ilustração; figura; mapa; demonstração; inserção}
  \end{Phonetics}
\end{Entry}

\begin{Entry}{插话}{12,8}{⼿,⾔}
  \begin{Phonetics}{插话}{cha1/hua4}
    \definition{s.}{interrupção | digressão}
    \definition{v.+compl.}{interromper (a fala de alguém)}
  \end{Phonetics}
\end{Entry}

\begin{Entry}{插嘴}{12,16}{⼿,⼝}
  \begin{Phonetics}{插嘴}{cha1/zui3}[][HSK 7-9]
    \definition{v.+compl.}{interromper; intrometer"-se; participar da conversa (geralmente de forma inadequada)}
  \end{Phonetics}
\end{Entry}

%%%%%%%%%% 握 %%%%%%%%%%
\subsection*{握}\addcontentsline{loh}{figure}{握}

\begin{Entry}{握}{12}{⼿}
  \begin{Phonetics}{握}{wo4}[][HSK 5]
    \definition{v.}{segurar; agarrar | agarrar; segurar; empunhar; controlar | pegar pela mão}
  \end{Phonetics}
\end{Entry}

\begin{Entry}{握手}{12,4}{⼿,⼿}
  \begin{Phonetics}{握手}{wo4/shou3}[][HSK 3]
    \definition{v.+compl.}{apertar as mãos; dar um aperto de mão; estender a mão e apertar a mão do outro é uma forma de saudação ao se encontrar ou se despedir, e também é usado para expressar felicitações ou condolências}
  \end{Phonetics}
\end{Entry}

%%%%%%%%%% 揣 %%%%%%%%%%
\subsection*{揣}\addcontentsline{loh}{figure}{揣}

\begin{Entry}{揣}{12}{⼿}
  \begin{Phonetics}{揣}{chuai1}[][HSK 7-9]
    \definition{v.}{esconder (ou carregar) nas roupas | Dialeto: encher"-se de comida; comer demais; encher alguém de comida; alimentar em excesso}
  \end{Phonetics}
  \begin{Phonetics}{揣}{chuai3}
    \definition*{s.}{Sobrenome: Chuai}
    \definition{v.}{contar; calcular; medir | estimar; palpitar; conjecturar}
  \end{Phonetics}
\end{Entry}

\begin{Entry}{揣测}{12,9}{⼿,⽔}
  \begin{Phonetics}{揣测}{chuai3ce4}[][HSK 7-9]
    \definition{v.}{adivinhar; conjecturar | supor; calcular; especular}
  \end{Phonetics}
\end{Entry}

\begin{Entry}{揣摩}{12,15}{⼿,⼿}
  \begin{Phonetics}{揣摩}{chuai3mo2}[][HSK 7-9]
    \definition{v.}{tentar compreender; tentar descobrir; obter algo por meio de estudo cuidadoso; pesar e considerar}
  \end{Phonetics}
\end{Entry}

%%%%%%%%%% 揪 %%%%%%%%%%
\subsection*{揪}\addcontentsline{loh}{figure}{揪}

\begin{Entry}{揪}{12}{⼿}
  \begin{Phonetics}{揪}{jiu1}[][HSK 7-9]
    \definition{v.}{segurar com firmeza; agarrar e puxar}
  \end{Phonetics}
\end{Entry}

%%%%%%%%%% 揭 %%%%%%%%%%
\subsection*{揭}\addcontentsline{loh}{figure}{揭}

\begin{Entry}{揭}{12}{⼿}
  \begin{Phonetics}{揭}{jie1}[][HSK 6]
    \definition*{s.}{Sobrenome: Jie}
    \definition{v.}{rasgar; arrancar; tirar | descobrir; levantar (a tampa, etc.) | expor; mostrar; trazer à luz | (literário) levantar; içar}
  \end{Phonetics}
\end{Entry}

\begin{Entry}{揭发}{12,5}{⼿,⼜}
  \begin{Phonetics}{揭发}{jie1fa1}[][HSK 7-9]
    \definition{v.}{expor; desmascarar; trazer à luz; expor e denunciar (pessoas más e más ações)}
  \end{Phonetics}
\end{Entry}

\begin{Entry}{揭示}{12,5}{⼿,⽰}
  \begin{Phonetics}{揭示}{jie1shi4}[][HSK 7-9]
    \definition{v.}{anunciar; promulgar; exibir publicamente | revelar; desvendar; trazer à luz; apontar ou esclarecer a essência de coisas que não são facilmente visíveis}
  \end{Phonetics}
\end{Entry}

\begin{Entry}{揭晓}{12,10}{⼿,⽇}
  \begin{Phonetics}{揭晓}{jie1xiao3}[][HSK 7-9]
    \definition{v.}{revelar; anunciar; tornar conhecido; divulgar publicamente os resultados da investigação para que todos fiquem cientes}
  \end{Phonetics}
\end{Entry}

\begin{Entry}{揭露}{12,21}{⼿,⾬}
  \begin{Phonetics}{揭露}{jie1lu4}[][HSK 7-9]
    \definition{v.}{expor; desmascarar; descobrir; revelar o que estava oculto}
  \end{Phonetics}
\end{Entry}

%%%%%%%%%% 援 %%%%%%%%%%
\subsection*{援}\addcontentsline{loh}{figure}{援}

\begin{Entry}{援}{12}{⼿}
  \begin{Phonetics}{援}{yuan2}
    \definition*{s.}{Sobrenome: Yuan}
    \definition{v.}{puxar com a mão; segurar | citar; referenciar | ajudar; auxiliar; resgatar}
  \end{Phonetics}
\end{Entry}

\begin{Entry}{援助}{12,7}{⼿,⼒}
  \begin{Phonetics}{援助}{yuan2zhu4}[][HSK 6]
    \definition{s.}{ajuda; assistência; auxílio}
    \definition{v.}{ajudar; apoiar; auxiliar}
  \end{Phonetics}
\end{Entry}

%%%%%%%%%% 揽 %%%%%%%%%%
\subsection*{揽}\addcontentsline{loh}{figure}{揽}

\begin{Entry}{揽}{12}{⼿}
  \begin{Phonetics}{揽}{lan3}[][HSK 7-9]
    \definition{v.}{puxar (ou tomar) para os braços; puxar alguém para os braços; abraçar; envolver alguém em seus braços | prender com uma corda, etc.; reunir os materiais soltos com cordas ou meios semelhantes | assumir; tomar a iniciativa; fazer campanha | agarrar; monopolizar; agarre-se a; controlar}
  \end{Phonetics}
\end{Entry}

%%%%%%%%%% 搀 %%%%%%%%%%
\subsection*{搀}\addcontentsline{loh}{figure}{搀}

\begin{Entry}{搀}{12}{⼿}
  \begin{Phonetics}{搀}{chan1}[][HSK 7-9]
    \definition{v.}{apoiar alguém pelo braço; apoiar alguém com a mão; apoiar | misturar}
  \end{Phonetics}
\end{Entry}

%%%%%%%%%% 搁 %%%%%%%%%%
\subsection*{搁}\addcontentsline{loh}{figure}{搁}

\begin{Entry}{搁}{12}{⼿}
  \begin{Phonetics}{搁}{ge1}[][HSK 7-9]
    \definition{v.}{pôr; colocar | colocar à parte; deixar para trás; deixar para mais tarde| deixar de lado}
  \end{Phonetics}
  \begin{Phonetics}{搁}{ge2}
    \definition{v.}{suportar; resistir}
  \end{Phonetics}
\end{Entry}

\begin{Entry}{搁浅}{12,8}{⼿,⽔}
  \begin{Phonetics}{搁浅}{ge1/qian3}[][HSK 7-9]
    \definition{v.+compl.}{ficar encalhado (navio); encalhar | ser retido; chegar a um impasse; metaforicamente, algo está bloqueado e não pode prosseguir}
  \end{Phonetics}
\end{Entry}

\begin{Entry}{搁置}{12,13}{⼿,⽹}
  \begin{Phonetics}{搁置}{ge1zhi4}[][HSK 7-9]
    \definition{v.}{arquivar; deixar de lado; suspender; classificar; deitar; adiar; colocar na prateleira}
  \end{Phonetics}
\end{Entry}

%%%%%%%%%% 搂 %%%%%%%%%%
\subsection*{搂}\addcontentsline{loh}{figure}{搂}

\begin{Entry}{搂}{12}{⼿}
  \begin{Phonetics}{搂}{lou3}[][HSK 7-9]
    \definition{v.}{juntar; ajuntar; reunir | segurar; pegar; aconchegar; usar as mãos para juntar e levantar o objeto | fazer  (dinheiro); extorquir | puxar; virar | verificar; analisar}
  \end{Phonetics}
\end{Entry}

%%%%%%%%%% 搅 %%%%%%%%%%
\subsection*{搅}\addcontentsline{loh}{figure}{搅}

\begin{Entry}{搅}{12}{⼿}
  \begin{Phonetics}{搅}{jiao3}[][HSK 7-9]
    \definition{v.}{mexer; misturar | perturbar; incomodar; interromper}
  \end{Phonetics}
\end{Entry}

\begin{Entry}{搅拌}{12,8}{⼿,⼿}
  \begin{Phonetics}{搅拌}{jiao3ban4}[][HSK 7-9]
    \definition{v.}{misturar; mexer; agitar; usar uma colher, um palito ou um utensílio semelhante para girar a mistura e homogeneizá-la}
  \end{Phonetics}
\end{Entry}

%%%%%%%%%% 搓 %%%%%%%%%%
\subsection*{搓}\addcontentsline{loh}{figure}{搓}

\begin{Entry}{搓}{12}{⼿}
  \begin{Phonetics}{搓}{cuo1}[][HSK 7-9]
    \definition{s.}{torção}
    \definition{v.}{esfregar ou rolar entre as mãos ou dedos |  (no tênis, tênis de mesa, críquete, etc.) cortar | (roupa, etc.) torcer}
  \end{Phonetics}
\end{Entry}

%%%%%%%%%% 搜 %%%%%%%%%%
\subsection*{搜}\addcontentsline{loh}{figure}{搜}

\begin{Entry}{搜}{12}{⼿}
  \begin{Phonetics}{搜}{sou1}[][HSK 5]
    \definition{v.}{procurar | pesquisar | coletar; reunir | procurar ou revistar um lugar de forma completa e desordenada}
  \seealsoref{搜查}{sou1cha2}
  \end{Phonetics}
\end{Entry}

\begin{Entry}{搜寻}{12,6}{⼿,⼨}
  \begin{Phonetics}{搜寻}{sou1xun2}[][HSK 7-9]
    \definition{v.}{procurar; caçar; buscar}
  \synonymref{搜查}{sou1cha2}
  \synonymref{搜救}{sou1jiu4}
  \synonymref{搜索}{sou1suo3}
  \end{Phonetics}
\end{Entry}

\begin{Entry}{搜查}{12,9}{⼿,⽊}
  \begin{Phonetics}{搜查}{sou1cha2}[][HSK 7-9]
    \definition{v.}{procurar; vasculhar; revirar | buscar e inspecionar (criminosos ou contrabando); examinar minuciosamente para encontrar problemas}
  \seealsoref{搜}{sou1}
  \synonymref{搜索}{sou1suo3}
  \synonymref{搜寻}{sou1xun2}
  \antonymref{隐藏}{yin3cang2}
  \end{Phonetics}
\end{Entry}

\begin{Entry}{搜索}{12,10}{⼿,⽷}
  \begin{Phonetics}{搜索}{sou1suo3}[][HSK 5]
    \definition{v.}{procurar; caçar; explorar; pesquisar cuidadosamente; refere"-se especificamente à busca militar para identificar situações suspeitas em determinada região, área marítima ou aérea}
  \end{Phonetics}
\end{Entry}

\begin{Entry}{搜救}{12,11}{⼿,⽁}
  \begin{Phonetics}{搜救}{sou1jiu4}[][HSK 7-9]
    \definition{v.}{buscar e resgatar}
  \synonymref{救助}{jiu4zhu4}
  \synonymref{抢救}{qiang3jiu4}
  \synonymref{搜寻}{sou1xun2}
  \synonymref{援助}{yuan2zhu4}
  \end{Phonetics}
\end{Entry}

\begin{Entry}{搜集}{12,12}{⼿,⾫}
  \begin{Phonetics}{搜集}{sou1ji2}[][HSK 7-9]
    \definition{v.}{coletar; reunir; pesquisar coleção}
  \synonymref{采集}{cai3ji2}
  \synonymref{汇集}{hui4ji2}
  \synonymref{收集}{shou1ji2}
  \synonymref{网络}{wang3luo4}
  \synonymref{征求}{zheng1qiu2}
  \end{Phonetics}
\end{Entry}

%%%%%%%%%% 搭 %%%%%%%%%%
\subsection*{搭}\addcontentsline{loh}{figure}{搭}

\begin{Entry}{搭}{12}{⼿}
  \begin{Phonetics}{搭}{da1}[][HSK 6]
    \definition{v.}{colocar em prática; construir | ficar pendurado; colocar para cima | entrar em contato; juntar"-se | adicionar (mais pessoas, dinheiro, etc.) | levantar algo junto |
pegar (um navio, avião, etc.); viajar (ou ir) por}
    \variantof{褡}
  \end{Phonetics}
\end{Entry}

\begin{Entry}{搭讪}{12,5}{⼿,⾔}
  \begin{Phonetics}{搭讪}{da1shan4}
    \definition{v.}{bater em alguém | incitar uma conversa | começar a conversar para acabar com um silêncio constrangedor ou uma situação embaraçosa}
  \end{Phonetics}
\end{Entry}

\begin{Entry}{搭建}{12,8}{⼿,⼵}
  \begin{Phonetics}{搭建}{da1jian4}[][HSK 7-9]
    \definition{v.}{montar (um galpão, abrigo temporário, etc.) | criar (uma organização) | construir (especialmente com materiais simples) | juntar (um galpão temporário) | armar}
  \end{Phonetics}
\end{Entry}

\begin{Entry}{搭乘}{12,10}{⼿,⽲}
  \begin{Phonetics}{搭乘}{da1cheng2}[][HSK 7-9]
    \definition{v.}{viajar de (carro, barco, avião etc.)}
  \end{Phonetics}
\end{Entry}

\begin{Entry}{搭档}{12,10}{⼿,⽊}
  \begin{Phonetics}{搭档}{da1dang4}[][HSK 6]
    \definition[个,名,位]{s.}{parceiro; colega de trabalho}
    \definition{v.}{cooperar; trabalhar em conjunto; formar pares; colaborar; formar uma parceria}
  \end{Phonetics}
\end{Entry}

\begin{Entry}{搭配}{12,10}{⼿,⾣}
  \begin{Phonetics}{搭配}{da1pei4}[][HSK 6]
    \definition{v.}{emparelhar; organizar em pares ou grupos; organizar a distribuição de acordo com certos requisitos | encaixar; combinar}
  \end{Phonetics}
\end{Entry}

%%%%%%%%%% 敞 %%%%%%%%%%
\subsection*{敞}\addcontentsline{loh}{figure}{敞}

\begin{Entry}{敞}{12}{⽁}
  \begin{Phonetics}{敞}{chang3}
    \definition{adj.}{espaçoso; aberto; desobstruído | Dialeto: (casa, pátio, etc.) espaçoso}
    \definition{v.}{abrir; descobrir}
  \end{Phonetics}
\end{Entry}

\begin{Entry}{敞开}{12,4}{⽁,⼶}
  \begin{Phonetics}{敞开}{chang3kai1}[][HSK 7-9]
    \definition{adj.}{aberto; irrestrito}
    \definition{adv.}{livremente; sem reservas; ilimitadamente; irrestritamente; totalmente aberto; infinitamente; sem limite}
    \definition{v.}{abrir o máximo possível}
  \end{Phonetics}
\end{Entry}

%%%%%%%%%% 散 %%%%%%%%%%
\subsection*{散}\addcontentsline{loh}{figure}{散}

\begin{Entry}{散}{12}{⽁}
  \begin{Phonetics}{散}{san3}[][HSK 5]
    \definition{adj.}{disperso; fragmentado; não integrado}
    \definition{s.}{medicamento em forma de pó}
    \definition{v.}{divergir; espalhar-se; separar-se; soltar-se; não se manter unido;  desintegrar}
  \end{Phonetics}
  \begin{Phonetics}{散}{san4}[][HSK 4]
    \definition{v.}{quebrar; fragmentar; dispersar | dar; distribuir; disseminar; divulgar | dissipar; deixar sai  | terminar um acordo ou contrato; demitir}
  \end{Phonetics}
\end{Entry}

\begin{Entry}{散心}{12,4}{⽁,⼼}
  \begin{Phonetics}{散心}{san4/xin1}
    \definition{v.+compl.}{aliviar o tédio | desfrutar de uma diversão | estar despreocupado}
  \end{Phonetics}
\end{Entry}

\begin{Entry}{散文}{12,4}{⽁,⽂}
  \begin{Phonetics}{散文}{san3wen2}[][HSK 5]
    \definition[篇,种]{s.}{ensaio; prosa; gênero literário, na antiguidade, referia"-se a textos em prosa, em oposição à poesia e à prosa paralela; atualmente, refere"-se a obras literárias que não sejam poesia, teatro ou romance, incluindo ensaios, contos, crônicas, relatos de viagem, etc.}
  \end{Phonetics}
\end{Entry}

\begin{Entry}{散发}{12,5}{⽁,⼜}
  \begin{Phonetics}{散发}{san4fa4}[][HSK 7-9]
    \definition{v.}{emitir; difundir; enviar; divulgar | emitir; distribuir; dar}
  \end{Phonetics}
\end{Entry}

\begin{Entry}{散布}{12,5}{⽁,⼱}
  \begin{Phonetics}{散布}{san4bu4}[][HSK 7-9]
    \definition{v.}{espalhar; distribuir; disseminar | espalhar; propagar}
  \end{Phonetics}
\end{Entry}

\begin{Entry}{散步}{12,7}{⽁,⽌}
  \begin{Phonetics}{散步}{san4/bu4}[][HSK 3]
    \definition{v.+compl.}{dar uma volta; dar um passeio; dar uma caminhada}
  \end{Phonetics}
\end{Entry}

%%%%%%%%%% 敦 %%%%%%%%%%
\subsection*{敦}\addcontentsline{loh}{figure}{敦}

\begin{Entry}{敦}{12}{⽁}
  \begin{Phonetics}{敦}{dui4}
    \definition{s.}{um recipiente tradicional para armazenar arroz ou grãos; utensílios antigos para guardar painço}
  \end{Phonetics}
  \begin{Phonetics}{敦}{dun1}
    \definition*{s.}{Sobrenome: Dun}
    \definition{adj.}{honesto; sincero}
  \end{Phonetics}
\end{Entry}

\begin{Entry}{敦促}{12,9}{⽁,⼈}
  \begin{Phonetics}{敦促}{dun1cu4}[][HSK 7-9]
    \definition{v.}{instar; pressionar; urgir}
  \end{Phonetics}
\end{Entry}

\begin{Entry}{敦厚}{12,9}{⽁,⼚}
  \begin{Phonetics}{敦厚}{dun1hou4}[][HSK 7-9]
    \definition{adj.}{genuíno | honesto e sincero}
  \end{Phonetics}
\end{Entry}

%%%%%%%%%% 敬 %%%%%%%%%%
\subsection*{敬}\addcontentsline{loh}{figure}{敬}

\begin{Entry}{敬}{12}{⽁}
  \begin{Phonetics}{敬}{jing4}[][HSK 7-9]
    \definition*{s.}{Sobrenome: Jing}
    \definition{adj.}{respeitoso; reverente}
    \definition{adv.}{respeitosamente}
    \definition{v.}{respeitar; honrar; estimar | oferecer educadamente | envolver-se em; dedicar-se a}
  \end{Phonetics}
\end{Entry}

\begin{Entry}{敬业}{12,5}{⽁,⼀}
  \begin{Phonetics}{敬业}{jing4ye4}[][HSK 7-9]
    \definition{v.}{ser dedicado ao próprio trabalho; refere"-se a um espírito louvável, dedicado aos estudos ou ao trabalho}
  \end{Phonetics}
\end{Entry}

\begin{Entry}{敬礼}{12,5}{⽁,⽰}
  \begin{Phonetics}{敬礼}{jing4/li3}[][HSK 7-9]
    \definition{s.}{honoríficos, usados no final de uma carta}[结尾都是``此致敬礼''。===Todas elas terminam com ``Atenciosamente''.]
    \definition{v.+compl.}{saudar; prestar continência; demonstrar respeito através de gestos como olhar para alguém, levantar a mão, ficar em posição de sentido e fazer uma reverência}
  \end{Phonetics}
\end{Entry}

\begin{Entry}{敬而远之}{12,6,7,3}{⽁,⽽,⾡,⼂}
  \begin{Phonetics}{敬而远之}{jing4'er2yuan3zhi1}[][HSK 7-9]
    \definition{expr.}{``Respeite, mas mantenha distância.''; mantenha uma distância respeitosa de alguém; afaste-se de; demonstrar respeito à distância}
  \end{Phonetics}
\end{Entry}

\begin{Entry}{敬佩}{12,8}{⽁,⼈}
  \begin{Phonetics}{敬佩}{jing4pei4}[][HSK 7-9]
    \definition{v.}{estimar; admirar; respeitar e admirar}
  \end{Phonetics}
\end{Entry}

\begin{Entry}{敬重}{12,9}{⽁,⾥}
  \begin{Phonetics}{敬重}{jing4zhong4}[][HSK 7-9]
    \definition{v.}{reverenciar; honrar; estimar; respeitar profundamente}
  \end{Phonetics}
\end{Entry}

\begin{Entry}{敬爱}{12,10}{⽁,⽖}
  \begin{Phonetics}{敬爱}{jing4'ai4}[][HSK 7-9]
    \definition{v.}{amar; estimar; acarinhar; respeitar e amar}
  \end{Phonetics}
\end{Entry}

\begin{Entry}{敬请}{12,10}{⽁,⾔}
  \begin{Phonetics}{敬请}{jing4qing3}[][HSK 7-9]
    \definition{v.}{solicitar; convidar respeitosamente; termos educados usados para convidar ou pedir (que alguém faça algo)}
  \end{Phonetics}
\end{Entry}

\begin{Entry}{敬酒}{12,10}{⽁,⾣}
  \begin{Phonetics}{敬酒}{jing4/jiu3}[][HSK 7-9]
    \definition{v.+compl.}{brindar; propor um brinde; levantar seu copo respeitosamente para convidar a outra pessoa a beber}
  \end{Phonetics}
\end{Entry}

\begin{Entry}{敬意}{12,13}{⽁,⼼}
  \begin{Phonetics}{敬意}{jing4yi4}[][HSK 7-9]
    \definition{s.}{respeito; tributo; homenagem}
  \end{Phonetics}
\end{Entry}

%%%%%%%%%% 斑 %%%%%%%%%%
\subsection*{斑}\addcontentsline{loh}{figure}{斑}

\begin{Entry}{斑}{12}{⽂}
  \begin{Phonetics}{斑}{ban1}
    \definition{adj.}{manchado; listrado; de cor variegada}
    \definition[块,片,个]{s.}{mancha; pinta; salpico; listra | nódoa; estria; mácula; imperfeição}
  \end{Phonetics}
\end{Entry}

\begin{Entry}{斑点}{12,9}{⽂,⽕}
  \begin{Phonetics}{斑点}{ban1dian3}[][HSK 7-9]
    \definition[块,片,个]{s.}{salpico; cisco; ponto; pinta; sardas | mancha; marca}
  \end{Phonetics}
\end{Entry}

%%%%%%%%%% 斯 %%%%%%%%%%
\subsection*{斯}\addcontentsline{loh}{figure}{斯}

\begin{Entry}{斯}{12}{⽄}
  \begin{Phonetics}{斯}{si1}
    \definition*{s.}{Sobrenome: Si}
    \definition{adv.}{então; assim}
    \definition{pron.}{isto; aqui}
  \end{Phonetics}
\end{Entry}

\begin{Entry}{斯巴达}{12,4,6}{⽄,⼰,⾡}
  \begin{Phonetics}{斯巴达}{si1ba1da2}
    \definition*{s.}{Esparta}
  \end{Phonetics}
\end{Entry}

%%%%%%%%%% 普 %%%%%%%%%%
\subsection*{普}\addcontentsline{loh}{figure}{普}

\begin{Entry}{普}{12}{⽇}
  \begin{Phonetics}{普}{pu3}
    \definition*{s.}{Sobrenome: Pu}
    \definition{adj.}{geral; universal}
  \end{Phonetics}
\end{Entry}

\begin{Entry}{普及}{12,3}{⽇,⼃}
  \begin{Phonetics}{普及}{pu3ji2}[][HSK 3]
    \definition{adj.}{popular; universal; onipresente; amplamente compreendido, aceito ou utilizado}
    \definition[种]{v.}{popularizar; disseminar; espalhar entre as pessoas; promover amplamente o conhecimento, a educação, a tecnologia, etc. para popularizá-los}
  \end{Phonetics}
\end{Entry}

\begin{Entry}{普快}{12,7}{⽇,⼼}
  \begin{Phonetics}{普快}{pu3 kuai4}
    \definition*{s.}{Expresso Comum}
  \end{Phonetics}
\end{Entry}

\begin{Entry}{普通}{12,10}{⽇,⾡}
  \begin{Phonetics}{普通}{pu3tong1}[][HSK 2]
    \definition{adj.}{comum; normal; geral; médio; em geral, nada de especial, como a maioria das pessoas ou coisas}
  \end{Phonetics}
\end{Entry}

\begin{Entry}{普通人}{12,10,2}{⽇,⾡,⼈}
  \begin{Phonetics}{普通人}{pu3tong1 ren2}[][HSK 7-9]
    \definition{s.}{pessoa comum; cidadão comum}
  \end{Phonetics}
\end{Entry}

\begin{Entry}{普通话}{12,10,8}{⽇,⾡,⾔}
  \begin{Phonetics}{普通话}{pu3tong1hua4}[][HSK 2]
    \definition*{s.}{Mandarim (``linguagem comum'') | Putonghua (fala comum da língua chinesa) | Língua oficial da China}
  \end{Phonetics}
\end{Entry}

\begin{Entry}{普遍}{12,12}{⽇,⾡}
  \begin{Phonetics}{普遍}{pu3bian4}[][HSK 3]
    \definition{adj.}{geral; comum; universal; difundido; a existência é muito ampla; tem semelhança}
  \end{Phonetics}
\end{Entry}

%%%%%%%%%% 景 %%%%%%%%%%
\subsection*{景}\addcontentsline{loh}{figure}{景}

\begin{Entry}{景}{12}{⽇}
  \begin{Phonetics}{景}{jing3}[][HSK 6]
    \definition*{s.}{Sobrenome: Jing}
    \definition{adj.}{grandioso; elevado; grande}
    \definition{s.}{vista; cenário; cena | situação; condição | cenário (de uma peça ou filme) | cena (de uma peça)}
    \definition{v.}{admirar; reverenciar; respeitar}
  \end{Phonetics}
\end{Entry}

\begin{Entry}{景区}{12,4}{⽇,⼖}
  \begin{Phonetics}{景区}{jing3qu1}[][HSK 7-9]
    \definition{s.}{ponto turístico | área cênica}
  \end{Phonetics}
\end{Entry}

\begin{Entry}{景色}{12,6}{⽇,⾊}
  \begin{Phonetics}{景色}{jing3se4}[][HSK 3]
    \definition[片,幅,道,处]{s.}{vista; cena; cenário; paisagem}
  \end{Phonetics}
\end{Entry}

\begin{Entry}{景观}{12,6}{⽇,⾒}
  \begin{Phonetics}{景观}{jing3guan1}[][HSK 7-9]
    \definition{s.}{cenário; paisagem; paisagem formada naturalmente; também se refere a paisagem criada artificialmente}
  \end{Phonetics}
\end{Entry}

\begin{Entry}{景点}{12,9}{⽇,⽕}
  \begin{Phonetics}{景点}{jing3dian3}[][HSK 6]
    \definition[个,处]{s.}{local cênico; atração turística; um lugar onde se concentram as atrações turísticas, incluindo atrações naturais e culturais}
  \end{Phonetics}
\end{Entry}

\begin{Entry}{景象}{12,11}{⽇,⾗}
  \begin{Phonetics}{景象}{jing3xiang4}[][HSK 5]
    \definition[个,种]{s.}{cena; visão; vista; quadro}
  \end{Phonetics}
\end{Entry}

%%%%%%%%%% 晴 %%%%%%%%%%
\subsection*{晴}\addcontentsline{loh}{figure}{晴}

\begin{Entry}{晴}{12}{⽇}
  \begin{Phonetics}{晴}{qing2}[][HSK 2]
    \definition{adj.}{ensolarado; bom; claro; não há nuvens no céu ou há poucas nuvens}
  \end{Phonetics}
\end{Entry}

\begin{Entry}{晴天}{12,4}{⽇,⼤}
  \begin{Phonetics}{晴天}{qing2tian1}[][HSK 2]
    \definition[个]{s.}{dia ensolarado; tempo sem nuvens ou com poucas nuvens; em meteorologia, refere"-se a um tempo em que a cobertura de nuvens no céu é inferior a 10\%}
  \end{Phonetics}
\end{Entry}

\begin{Entry}{晴朗}{12,10}{⽇,⽉}
  \begin{Phonetics}{晴朗}{qing2lang3}[][HSK 5]
    \definition{adj.}{bom; claro; ensolarado; céu limpo e sem nuvens}
  \end{Phonetics}
\end{Entry}

%%%%%%%%%% 晶 %%%%%%%%%%
\subsection*{晶}\addcontentsline{loh}{figure}{晶}

\begin{Entry}{晶}{12}{⽇}
  \begin{Phonetics}{晶}{jing1}
    \definition{adj.}{brilhante; cristalino}
    \definition{s.}{cristal de rocha; cristal; quartzo}
  \end{Phonetics}
\end{Entry}

\begin{Entry}{晶莹}{12,10}{⽇,⾋}
  \begin{Phonetics}{晶莹}{jing1ying2}[][HSK 7-9]
    \definition{adj.}{brilhante e cristalino; cintilante e translúcido; brilhante e transparente}
  \end{Phonetics}
\end{Entry}

%%%%%%%%%% 智 %%%%%%%%%%
\subsection*{智}\addcontentsline{loh}{figure}{智}

\begin{Entry}{智}{12}{⽇}
  \begin{Phonetics}{智}{zhi4}
    \definition*{s.}{Sobrenome: Zhi}
    \definition{adj.}{engenhoso; sábio; inteligente; astuto}
    \definition{s.}{discernimento; engenhosidade; sagacidade | inteligência; conhecimento; sabedoria; percepção}
  \end{Phonetics}
\end{Entry}

\begin{Entry}{智力}{12,2}{⽇,⼒}
  \begin{Phonetics}{智力}{zhi4li4}[][HSK 4]
    \definition{s.}{inteligência; refere"-se à capacidade de uma pessoa de conhecer e entender coisas objetivas e aplicar o conhecimento e a experiência para resolver problemas, incluindo memória, observação, imaginação, pensamento e julgamento}
  \end{Phonetics}
\end{Entry}

\begin{Entry}{智能}{12,10}{⽇,⾁}
  \begin{Phonetics}{智能}{zhi4neng2}[][HSK 4]
    \definition{adj.}{inteligente (telefone, sistema, etc.); descreve máquinas, equipamentos, tecnologia, etc. que foram processados com alta tecnologia e têm a capacidade de falar, pensar, calcular, resolver problemas, etc., como um ser humano}
    \definition{s.}{intelecto; a capacidade de aprender, agir, pensar, inventar, criar, resolver problemas, etc.}
  \end{Phonetics}
\end{Entry}

\begin{Entry}{智商}{12,11}{⽇,⼝}
  \begin{Phonetics}{智商}{zhi4shang1}
    \definition{s.}{quociente de inteligência, QI}
  \end{Phonetics}
\end{Entry}

\begin{Entry}{智障}{12,13}{⽇,⾩}
  \begin{Phonetics}{智障}{zhi4zhang4}
    \definition{adj./s.}{retardado}
  \end{Phonetics}
\end{Entry}

\begin{Entry}{智慧}{12,15}{⽇,⼼}
  \begin{Phonetics}{智慧}{zhi4hui4}[][HSK 6]
    \definition[种]{s.}{sagacidade; sabedoria; inteligência; capacidade de analisar, julgar, inventar, criar e resolver problemas}
  \end{Phonetics}
\end{Entry}

%%%%%%%%%% 暂 %%%%%%%%%%
\subsection*{暂}\addcontentsline{loh}{figure}{暂}

\begin{Entry}{暂}{12}{⽇}
  \begin{Phonetics}{暂}{zan4}
    \definition{adj.}{de curta duração | curto; momentâneo; pouco tempo}
    \definition{adv.}{temporariamente; por enquanto}
  \antonymref{久}{jiu3}
  \end{Phonetics}
\end{Entry}

\begin{Entry}{暂时}{12,7}{⽇,⽇}
  \begin{Phonetics}{暂时}{zan4shi2}[][HSK 5]
    \definition{adj.}{transitório; temporário}
    \definition{adv.}{por enquanto; em pouco tempo}
  \end{Phonetics}
\end{Entry}

\begin{Entry}{暂停}{12,11}{⽇,⼈}
  \begin{Phonetics}{暂停}{zan4ting2}[][HSK 5]
    \definition{s.}{suspensão temporária; refere"-se especificamente à suspensão temporária de certas competições desportivas de acordo com as regras}
    \definition{v.}{pausar; suspender; esgotar o tempo}
  \end{Phonetics}
\end{Entry}

%%%%%%%%%% 暑 %%%%%%%%%%
\subsection*{暑}\addcontentsline{loh}{figure}{暑}

\begin{Entry}{暑}{12}{⽇}
  \begin{Phonetics}{暑}{shu3}
    \definition{adj.}{calor; clima quente; quente}
    \definition{s.}{verão}
  \antonymref{寒}{han2}
  \end{Phonetics}
\end{Entry}

\begin{Entry}{暑假}{12,11}{⽇,⼈}
  \begin{Phonetics}{暑假}{shu3jia4}[][HSK 4]
    \definition[个]{s.}{férias de verão; feriado de verão; férias escolares de verão, na China, durante o sétimo e o oitavo meses do calendário gregoriano}
  \end{Phonetics}
\end{Entry}

\begin{Entry}{暑期}{12,12}{⽇,⽉}
  \begin{Phonetics}{暑期}{shu3qi1}[][HSK 7-9]
    \definition[个]{s.}{época de férias de verão | durante as férias de verão}
  \synonymref{暑假}{shu3jia4}
  \antonymref{寒假}{han2jia4}
  \end{Phonetics}
\end{Entry}

%%%%%%%%%% 曾 %%%%%%%%%%
\subsection*{曾}\addcontentsline{loh}{figure}{曾}

\begin{Entry}{曾}{12}{⽈}
  \begin{Phonetics}{曾}{ceng2}[][HSK 4]
    \definition{adv.}{indica que uma ação já aconteceu ou um estado já existiu}
  \end{Phonetics}
  \begin{Phonetics}{曾}{zeng1}
    \definition*{s.}{Sobrenome: Zeng}
    \definition{s.}{relacionamento entre bisnetos e bisavós; (parentesco) duas gerações de diferença}
  \end{Phonetics}
\end{Entry}

\begin{Entry}{曾经}{12,8}{⽈,⽷}
  \begin{Phonetics}{曾经}{ceng2jing1}[][HSK 3]
    \definition{adv.}{uma vez; indica que houve algum comportamento ou situação}
  \end{Phonetics}
\end{Entry}

%%%%%%%%%% 替 %%%%%%%%%%
\subsection*{替}\addcontentsline{loh}{figure}{替}

\begin{Entry}{替}{12}{⽈}
  \begin{Phonetics}{替}{ti4}[][HSK 4]
    \definition{prep.}{para; em nome de}
    \definition{s.}{decadência; declínio; enfraquecimento}
    \definition{v.}{substituir; substituir por; tomar o lugar de}
  \end{Phonetics}
\end{Entry}

\begin{Entry}{替代}{12,5}{⽈,⼈}
  \begin{Phonetics}{替代}{ti4dai4}[][HSK 4]
    \definition{v.}{substituir; suplantar}
  \end{Phonetics}
\end{Entry}

\begin{Entry}{替身}{12,7}{⽈,⾝}
  \begin{Phonetics}{替身}{ti4shen1}[][HSK 7-9]
    \definition{s.}{bode expiatório | dublê de corpo | dublê}
    \definition{v.}{substituir; substituir alguém; ocupar o lugar de}
  \end{Phonetics}
\end{Entry}

\begin{Entry}{替换}{12,10}{⽈,⼿}
  \begin{Phonetics}{替换}{ti4huan4}[][HSK 7-9]
    \definition{v.}{substituir; substituir por; deslocar; tomar o lugar de; alternar}
  \synonymref{撤换}{che4huan4}
  \synonymref{更换}{geng1huan4}
  \synonymref{交换}{jiao1huan4}
  \synonymref{替代}{ti4dai4}
  \antonymref{代替}{dai4ti4}
  \end{Phonetics}
\end{Entry}

%%%%%%%%%% 最 %%%%%%%%%%
\subsection*{最}\addcontentsline{loh}{figure}{最}

\begin{Entry}{最}{12}{⽈}
  \begin{Phonetics}{最}{zui4}[][HSK 1]
    \definition{adv.}{(diante de um adjetivo ou verbo) o mais | (colocado antes de um substantivo de localidade ou de uma palavra que indica um lugar)  mais distante ou mais próximo de (um lugar) | mais; melhor; pior; primeiro; muito; menos; acima de tudo; indica que uma determinada característica excede todas as outras pessoas ou coisas do mesmo tipo}
    \definition{s.}{o máximo; o melhor (ou o mais alto, o maior, etc.)}
  \end{Phonetics}
\end{Entry}

\begin{Entry}{最少}{12,4}{⽈,⼩}
  \begin{Phonetics}{最少}{zui4shao3}
    \definition{adv.}{finalmente}
  \end{Phonetics}
\end{Entry}

\begin{Entry}{最优}{12,6}{⽈,⼈}
  \begin{Phonetics}{最优}{zui4you1}
    \definition{adj.}{ótimo}
  \end{Phonetics}
\end{Entry}

\begin{Entry}{最先}{12,6}{⽈,⼉}
  \begin{Phonetics}{最先}{zui4xian1}
    \definition{adv.}{o primeiro}
  \end{Phonetics}
\end{Entry}

\begin{Entry}{最后}{12,6}{⽈,⼝}
  \begin{Phonetics}{最后}{zui4hou4}[][HSK 1]
    \definition{s.}{último; final; definitivo; refere"-se ao tempo, local, etc. que vem depois de outros tempos, locais, etc. na ordem sequencial}
  \end{Phonetics}
\end{Entry}

\begin{Entry}{最多}{12,6}{⽈,⼣}
  \begin{Phonetics}{最多}{zui4duo1}
    \definition{adv.}{no máximo | máximo}
  \end{Phonetics}
\end{Entry}

\begin{Entry}{最好}{12,6}{⽈,⼥}
  \begin{Phonetics}{最好}{zui4hao3}[][HSK 1]
    \definition{adj.}{melhor; de primeira qualidade; excelente}
    \definition{adv.}{seria melhor; seria o ideal; indica a escolha mais adequada entre várias possibilidades}
  \end{Phonetics}
\end{Entry}

\begin{Entry}{最初}{12,7}{⽈,⾐}
  \begin{Phonetics}{最初}{zui4chu1}[][HSK 4]
    \definition{adj.}{primordial; inicial; primeiro}
    \definition{adv.}{inicialmente; originalmente}
    \definition{s.}{o período mais antigo; início; começo}
  \end{Phonetics}
\end{Entry}

\begin{Entry}{最近}{12,7}{⽈,⾡}
  \begin{Phonetics}{最近}{zui4jin4}[][HSK 2]
    \definition{adj.}{mais próximo}
    \definition{s.}{recentemente; ultimamente; de tarde; refere"-se aos dias antes ou logo depois de um discurso | em breve; no futuro próximo; o futuro próximo}
  \end{Phonetics}
\end{Entry}

\begin{Entry}{最远}{12,7}{⽈,⾡}
  \begin{Phonetics}{最远}{zui4yuan3}
    \definition{adv.}{mais distante | mais longe}
  \end{Phonetics}
\end{Entry}

\begin{Entry}{最佳}{12,8}{⽈,⼈}
  \begin{Phonetics}{最佳}{zui4jia1}[][HSK 6]
    \definition{adj.}{o melhor (atleta, filme etc); ótimo}
  \end{Phonetics}
\end{Entry}

\begin{Entry}{最终}{12,8}{⽈,⽷}
  \begin{Phonetics}{最终}{zui4zhong1}[][HSK 6]
    \definition{adv.}{finalmente; por fim}
    \definition{s.}{final; definitivo}
  \end{Phonetics}
\end{Entry}

\begin{Entry}{最高}{12,10}{⽈,⾼}
  \begin{Phonetics}{最高}{zui4gao1}
    \definition{adj.}{altíssimo | supremo | mais alto}
  \end{Phonetics}
\end{Entry}

\begin{Entry}{最善}{12,12}{⽈,⼝}
  \begin{Phonetics}{最善}{zui4shan4}
    \definition{adj.}{ótimo | o melhor}
  \end{Phonetics}
\end{Entry}

\begin{Entry}{最新}{12,13}{⽈,⽄}
  \begin{Phonetics}{最新}{zui4xin1}
    \definition{adv.}{mais recente | mais novo}
  \end{Phonetics}
\end{Entry}

%%%%%%%%%% 朝 %%%%%%%%%%
\subsection*{朝}\addcontentsline{loh}{figure}{朝}

\begin{Entry}{朝}{12}{⽉}
  \begin{Phonetics}{朝}{chao2}[][HSK 3]
    \definition*{s.}{Sobrenome: Chao}
    \definition{prep.}{para; em direção a; a direção ou o objeto da ação introduzida, equivalente a 向 ou 对}
    \definition[个]{s.}{corte real; governo; assembleia realizada por um soberano; também se refere à posição no poder | dinastia, todo o período de governo transmitido de geração em geração por um determinado sobrenome imperial | reinado (de um soberano); o período de reinado de um determinado monarca}
    \definition{v.}{fazer uma peregrinação para; ter uma audiência com (um rei, um imperador, etc.) | estar voltado para; estar em frente a}
  \seealsoref{对}{dui4}
  \seealsoref{向}{xiang4}
  \antonymref{野}{ye3}
  \end{Phonetics}
  \begin{Phonetics}{朝}{zhao1}
    \definition{s.}{manhã cedo; manhã | dia}
  \end{Phonetics}
\end{Entry}

\begin{Entry}{朝代}{12,5}{⽉,⼈}
  \begin{Phonetics}{朝代}{chao2dai4}[][HSK 7-9]
    \definition[个]{s.}{dinastia; refere"-se a um período ou era histórica}
  \end{Phonetics}
\end{Entry}

\begin{Entry}{朝廷}{12,6}{⽉,⼵}
  \begin{Phonetics}{朝廷}{chao2ting2}
    \definition{s.}{corte imperial | dinastia}
  \end{Phonetics}
\end{Entry}

\begin{Entry}{朝着}{12,11}{⽉,⽬}
  \begin{Phonetics}{朝着}{chao2zhe5}[][HSK 7-9]
    \definition{prep.}{voltado para; em direção a; pessoas ou coisas voltadas para uma direção}
  \end{Phonetics}
\end{Entry}

\begin{Entry}{朝鲜}{12,14}{⽉,⿂}
  \begin{Phonetics}{朝鲜}{chao2xian3}
    \definition*{s.}{Coréia do Norte}
  \end{Phonetics}
\end{Entry}

%%%%%%%%%% 期 %%%%%%%%%%
\subsection*{期}\addcontentsline{loh}{figure}{期}

\begin{Entry}{期}{12}{⽉}
  \begin{Phonetics}{期}{qi1}[][HSK 3]
    \definition{clas.}{questão; número; termo; coisas usadas para parcelamento}
    \definition{s.}{um período de tempo; fase; estágio | horário agendado; data agendada | tempo designado (programado)}
    \definition{v.}{marcar uma consulta | esperar; aguardar | esperar; ter esperança}
  \end{Phonetics}
\end{Entry}

\begin{Entry}{期中}{12,4}{⽉,⼁}
  \begin{Phonetics}{期中}{qi1zhong1}[][HSK 4]
    \definition{adj.}{provisório; interino; intermediário}
  \end{Phonetics}
\end{Entry}

\begin{Entry}{期末}{12,5}{⽉,⽊}
  \begin{Phonetics}{期末}{qi1mo4}[][HSK 4]
    \definition{s.}{terminal; final do prazo; fim do período}
  \end{Phonetics}
\end{Entry}

\begin{Entry}{期间}{12,7}{⽉,⾨}
  \begin{Phonetics}{期间}{qi1jian1}[][HSK 4]
    \definition{s.}{prazo; tempo; período}
  \end{Phonetics}
\end{Entry}

\begin{Entry}{期限}{12,8}{⽉,⾩}
  \begin{Phonetics}{期限}{qi1xian4}[][HSK 4]
    \definition{s.}{prazo; limite de tempo; tempo alocado; período de tempo limitado, também o limite final do limite de tempo; \emph{deadline}}
  \end{Phonetics}
\end{Entry}

\begin{Entry}{期待}{12,9}{⽉,⼻}
  \begin{Phonetics}{期待}{qi1dai4}[][HSK 4]
    \definition{v.}{aguardar; esperar; aguardar ansiosamente; ter em mente a realização de um determinado fim ou a ocorrência de uma determinada situação}
  \end{Phonetics}
\end{Entry}

\begin{Entry}{期盼}{12,9}{⽉,⽬}
  \begin{Phonetics}{期盼}{qi1pan4}[][HSK 7-9]
    \definition{v.}{esperar; aguardar}
  \end{Phonetics}
\end{Entry}

\begin{Entry}{期望}{12,11}{⽉,⽉}
  \begin{Phonetics}{期望}{qi1wang4}[][HSK 5]
    \definition{s.}{esperança; expectativa}
    \definition{v.}{esperar; ter esperança}
  \end{Phonetics}
\end{Entry}

%%%%%%%%%% 棉 %%%%%%%%%%
\subsection*{棉}\addcontentsline{loh}{figure}{棉}

\begin{Entry}{棉}{12}{⽊}
  \begin{Phonetics}{棉}{mian2}[][HSK 6]
    \definition{adj.}{almofadado com algodão; acolchoado}
    \definition[些,种,类]{s.}{termo genérico para algodão ou paina | algodão | material semelhante ao algodão | acolchoado ou estofado de algodão}
  \end{Phonetics}
\end{Entry}

\begin{Entry}{棉花}{12,7}{⽊,⾋}
  \begin{Phonetics}{棉花}{mian2hua1}[][HSK 7-9]
    \definition[团,斤,种]{s.}{algodão; as fibras dos caroços de algodão são utilizadas para fiar fios, enchimento de roupas e roupas de cama, etc. | algodão; o nome comum para o capim-algodão}
  \end{Phonetics}
\end{Entry}

%%%%%%%%%% 棋 %%%%%%%%%%
\subsection*{棋}\addcontentsline{loh}{figure}{棋}

\begin{Entry}{棋}{12}{⽊}
  \begin{Phonetics}{棋}{qi2}[][HSK 7-9]
    \definition[盘]{s.}{xadrez | jogo semelhante ao xadrez | uma partida de xadrez}
    \definition[个,颗]{s.}{peça de xadrez}
  \end{Phonetics}
\end{Entry}

\begin{Entry}{棋子}{12,3}{⽊,⼦}
  \begin{Phonetics}{棋子}{qi2zi5}[][HSK 7-9]
    \definition[个,颗]{s.}{peça (em um jogo de tabuleiro); peça de xadrez}
  \seealsoref{棋子儿}{qi2zi3r5}
  \end{Phonetics}
\end{Entry}

\begin{Entry}{棋子儿}{12,3,2}{⽊,⼦,⼉}
  \begin{Phonetics}{棋子儿}{qi2zi3r5}
    \definition{s.}{peça de xadrez}
  \end{Phonetics}
\end{Entry}

%%%%%%%%%% 棍 %%%%%%%%%%
\subsection*{棍}\addcontentsline{loh}{figure}{棍}

\begin{Entry}{棍}{12}{⽊}
  \begin{Phonetics}{棍}{gun4}[][HSK 7-9]
    \definition[根]{s.}{vara; bastão; porrete | canalha; patife; ladino; bandido}
  \end{Phonetics}
\end{Entry}

\begin{Entry}{棍子}{12,3}{⽊,⼦}
  \begin{Phonetics}{棍子}{gun4zi5}[][HSK 7-9]
    \definition[根]{s.}{vara; bastão; um objeto longo e redondo feito de madeira, bambu ou metal}
  \end{Phonetics}
\end{Entry}

%%%%%%%%%% 棒 %%%%%%%%%%
\subsection*{棒}\addcontentsline{loh}{figure}{棒}

\begin{Entry}{棒}{12}{⽊}
  \begin{Phonetics}{棒}{bang4}[][HSK 5]
    \definition{adj.}{bom; forte; excelente}
    \definition[根]{s.}{porrete; bastão; cajado; clava}
  \end{Phonetics}
\end{Entry}

\begin{Entry}{棒冰}{12,6}{⽊,⼎}
  \begin{Phonetics}{棒冰}{bang4bing1}
    \definition{s.}{picolé}
  \end{Phonetics}
\end{Entry}

\begin{Entry}{棒球}{12,11}{⽊,⽟}
  \begin{Phonetics}{棒球}{bang4qiu2}[][HSK 7-9]
    \definition[个,只]{s.}{beisebol}
  \end{Phonetics}
\end{Entry}

\begin{Entry}{棒棒糖}{12,12,16}{⽊,⽊,⽶}
  \begin{Phonetics}{棒棒糖}{bang4bang4tang2}
    \definition[根]{s.}{pirulito}
  \end{Phonetics}
\end{Entry}

%%%%%%%%%% 棕 %%%%%%%%%%
\subsection*{棕}\addcontentsline{loh}{figure}{棕}

\begin{Entry}{棕}{12}{⽊}
  \begin{Phonetics}{棕}{zong1}
    \definition{adj.}{marrom}
    \definition[个]{s.}{palmeira | fibra de palmeira; fibra de coco}
  \end{Phonetics}
\end{Entry}

\begin{Entry}{棕褐色}{12,14,6}{⽊,⾐,⾊}
  \begin{Phonetics}{棕褐色}{zong1he4 se4}
    \definition{s.}{cor sépia | bronzeado}
  \end{Phonetics}
\end{Entry}

%%%%%%%%%% 棘 %%%%%%%%%%
\subsection*{棘}\addcontentsline{loh}{figure}{棘}

\begin{Entry}{棘}{12}{⽊}
  \begin{Phonetics}{棘}{ji2}
    \definition*{s.}{Sobrenome: Ji}
    \definition{s.}{árvore de jujuba | arbustos espinhosos; silvas | espinho}
  \end{Phonetics}
\end{Entry}

\begin{Entry}{棘手}{12,4}{⽊,⼿}
  \begin{Phonetics}{棘手}{ji2shou3}[][HSK 7-9]
    \definition{adj.}{complicado; difícil; espinhoso; difícil de manusear}
  \end{Phonetics}
\end{Entry}

%%%%%%%%%% 森 %%%%%%%%%%
\subsection*{森}\addcontentsline{loh}{figure}{森}

\begin{Entry}{森}{12}{⽊}
  \begin{Phonetics}{森}{sen1}
    \definition{adj.}{cheio de árvores | multitudinário; em multidões | escuro; sombrio}
  \end{Phonetics}
\end{Entry}

\begin{Entry}{森林}{12,8}{⽊,⽊}
  \begin{Phonetics}{森林}{sen1lin2}[][HSK 4]
    \definition[片,座,处]{s.}{floresta; bosque; normalmente, refere"-se a uma grande área de árvores em crescimento; na silvicultura, refere"-se a um grande número de árvores que crescem em uma área razoavelmente grande de terra, juntamente com os animais e outras plantas}
  \end{Phonetics}
\end{Entry}

%%%%%%%%%% 棱 %%%%%%%%%%
\subsection*{棱}\addcontentsline{loh}{figure}{棱}

\begin{Entry}{棱}{12}{⽊}
  \begin{Phonetics}{棱}{leng1}
    \definition{adj.}{vermelho vivo; vermelho intenso}
  \end{Phonetics}
  \begin{Phonetics}{棱}{leng2}
    \definition[个,条]{s.}{aresta; borda; canto; a parte onde dois planos em direções diferentes se encontram em um objeto | ondulação; crista; ângulo elevado; as partes salientes do objeto}
  \end{Phonetics}
  \begin{Phonetics}{棱}{ling2}
    \definition*{s.}{Muling, 穆棱  (um nome de lugar na província de Heilongjiang, China)}
  \seealsoref{穆棱}{mu4ling2}
  \end{Phonetics}
\end{Entry}

\begin{Entry}{棱角}{12,7}{⽊,⾓}
  \begin{Phonetics}{棱角}{leng2jiao3}[][HSK 7-9]
    \definition{s.}{bordas e cantos | arestas (de observações, comentários, etc.)}
  \end{Phonetics}
\end{Entry}

%%%%%%%%%% 棵 %%%%%%%%%%
\subsection*{棵}\addcontentsline{loh}{figure}{棵}

\begin{Entry}{棵}{12}{⽊}
  \begin{Phonetics}{棵}{ke1}[][HSK 4]
    \definition{clas.}{usado para plantas, árvores}
  \end{Phonetics}
\end{Entry}

%%%%%%%%%% 棹 %%%%%%%%%%
\subsection*{棹}\addcontentsline{loh}{figure}{棹}

\begin{Entry}{棹}{12}{⽊}
  \begin{Phonetics}{棹}{zhuo1}
    \variantof{桌}
  \end{Phonetics}
\end{Entry}

%%%%%%%%%% 棺 %%%%%%%%%%
\subsection*{棺}\addcontentsline{loh}{figure}{棺}

\begin{Entry}{棺}{12}{⽊}
  \begin{Phonetics}{棺}{guan1}
    \definition[副]{s.}{caixão; esquife; ataúde}
  \end{Phonetics}
\end{Entry}

\begin{Entry}{棺材}{12,7}{⽊,⽊}
  \begin{Phonetics}{棺材}{guan1cai5}[][HSK 7-9]
    \definition[具,口]{s.}{caixão; esquife; ataúde; féretro; urna funerária usada para enterrar os mortos, geralmente feito de madeira}
  \end{Phonetics}
\end{Entry}

%%%%%%%%%% 椅 %%%%%%%%%%
\subsection*{椅}\addcontentsline{loh}{figure}{椅}

\begin{Entry}{椅}{12}{⽊}
  \begin{Phonetics}{椅}{yi3}
    \definition*{s.}{Sobrenome: Yi}
    \definition{s.}{cadeira}
  \end{Phonetics}
\end{Entry}

\begin{Entry}{椅子}{12,3}{⽊,⼦}
  \begin{Phonetics}{椅子}{yi3zi5}[][HSK 2]
    \definition[把,套,排]{s.}{cadeira; assentos com encosto, feitos principalmente de madeira, bambu, rattan, etc.; móveis com pernas, mas sem encosto para as pessoas se sentarem}
  \end{Phonetics}
\end{Entry}

%%%%%%%%%% 植 %%%%%%%%%%
\subsection*{植}\addcontentsline{loh}{figure}{植}

\begin{Entry}{植}{12}{⽊}
  \begin{Phonetics}{植}{zhi2}
    \definition*{s.}{Sobrenome: Zhi}
    \definition{s.}{flora; planta; vegetação}
    \definition{v.}{plantar; crescer; cultivar | configurar; estabelecer}
  \end{Phonetics}
\end{Entry}

\begin{Entry}{植物}{12,8}{⽊,⽜}
  \begin{Phonetics}{植物}{zhi2wu4}[][HSK 4]
    \definition[种,株,棵,盆]{s.}{planta; vegetação; flora}
  \end{Phonetics}
\end{Entry}

%%%%%%%%%% 椰 %%%%%%%%%%
\subsection*{椰}\addcontentsline{loh}{figure}{椰}

\begin{Entry}{椰}{12}{⽊}
  \begin{Phonetics}{椰}{ye1}
    \definition[只,棵]{s.}{coqueiro; coco}
  \end{Phonetics}
\end{Entry}

\begin{Entry}{椰汁}{12,5}{⽊,⽔}
  \begin{Phonetics}{椰汁}{ye1zhi1}
    \definition{s.}{água de coco}
  \end{Phonetics}
\end{Entry}

%%%%%%%%%% 欺 %%%%%%%%%%
\subsection*{欺}\addcontentsline{loh}{figure}{欺}

\begin{Entry}{欺}{12}{⽋}
  \begin{Phonetics}{欺}{qi1}
    \definition{v.}{enganar; trapacear | intimidar; tirar vantagem de alguém; tirar vantagem da fraqueza de (alguém, etc.)}
  \end{Phonetics}
\end{Entry}

\begin{Entry}{欺负}{12,6}{⽋,⾙}
  \begin{Phonetics}{欺负}{qi1fu5}[][HSK 6]
    \definition{v.}{violar, oprimir ou insultar com meios irracionais; \emph{bully}}
  \end{Phonetics}
\end{Entry}

\begin{Entry}{欺诈}{12,7}{⽋,⾔}
  \begin{Phonetics}{欺诈}{qi1zha4}[][HSK 7-9]
    \definition{v.}{trapacear; enganar; usar métodos astutos para enganar as pessoas para obter lucro}
  \end{Phonetics}
\end{Entry}

\begin{Entry}{欺骗}{12,12}{⽋,⾺}
  \begin{Phonetics}{欺骗}{qi1pian4}[][HSK 7-9]
    \definition{v.}{enganar; ludibriar; usar palavras ou ações falsas para encobrir a verdade e enganar as pessoas}
  \end{Phonetics}
\end{Entry}

%%%%%%%%%% 款 %%%%%%%%%%
\subsection*{款}\addcontentsline{loh}{figure}{款}

\begin{Entry}{款}{12}{⽋}
  \begin{Phonetics}{款}{kuan3}
    \definition{adj.}{sincero | lento; sem pressa | vazio; oco; irreal; é intercambiável com 窾}
    \definition{s.}{parágrafo; seção (de um artigo em um documento legal, etc.); os itens listados de acordo com as disposições de leis, regulamentos, tratados, etc. | dinheiro; fundo; uma quantia de dinheiro | o nome do remetente ou destinatário inscrito em uma pintura ou obra de caligrafia oferecida como presente; inscrições fundidas em recipientes de bronze, como sinos e tripés; títulos em pinturas e caligrafia | forma; estilo; especificações}
    \definition[笔,个]{v.}{entreter; receber com hospitalidade | Literário: bater; prostrar-se}
  \seealsoref{窾}{kuan3}
  \end{Phonetics}
\end{Entry}

\begin{Entry}{款式}{12,6}{⽋,⼷}
  \begin{Phonetics}{款式}{kuan3shi4}[][HSK 7-9]
    \definition[种,个,款,类]{s.}{modelo; estilo; design; moda; padrão; formato}
  \end{Phonetics}
\end{Entry}

\begin{Entry}{款项}{12,9}{⽋,⾴}
  \begin{Phonetics}{款项}{kuan3xiang4}[][HSK 7-9]
    \definition[宗]{s.}{soma de dinheiro; refere"-se a uma grande quantia de dinheiro com um propósito específico | cláusula (ordem, regra, tratado); artigos (em leis, regulamentos, tratados, etc.); itens dentro de artigos de leis, regulamentos, tratados, etc., são geralmente divididos em cláusulas, e cláusulas em subitens}
  \end{Phonetics}
\end{Entry}

%%%%%%%%%% 殖 %%%%%%%%%%
\subsection*{殖}\addcontentsline{loh}{figure}{殖}

\begin{Entry}{殖}{12}{⽍}
  \begin{Phonetics}{殖}{zhi2}
    \definition{v.}{crescer | reproduzir}
  \end{Phonetics}
\end{Entry}

%%%%%%%%%% 毯 %%%%%%%%%%
\subsection*{毯}\addcontentsline{loh}{figure}{毯}

\begin{Entry}{毯}{12}{⽑}
  \begin{Phonetics}{毯}{tan3}
    \definition[条]{s.}{cobertor; tapete; carpete}
  \end{Phonetics}
\end{Entry}

\begin{Entry}{毯子}{12,3}{⽑,⼦}
  \begin{Phonetics}{毯子}{tan3zi5}[][HSK 7-9]
    \definition[条,张,床,面]{s.}{cobertor; tecidos grossos e macios podem ser usados ​​como cortinas, capas ou decorações}
  \synonymref{被子}{bei4zi5}
  \synonymref{垫子}{dian4zi5}
  \end{Phonetics}
\end{Entry}

%%%%%%%%%% 渡 %%%%%%%%%%
\subsection*{渡}\addcontentsline{loh}{figure}{渡}

\begin{Entry}{渡}{12}{⽔}
  \begin{Phonetics}{渡}{du4}[][HSK 6]
    \definition{s.}{(usualmente em nomes de lugares) travessia de balsa}
    \definition{v.}{atravessar (um rio, o mar, etc.) | superar; sobreviver | transportar (pessoas, mercadorias, etc.) através}
  \end{Phonetics}
\end{Entry}

\begin{Entry}{渡过}{12,6}{⽔,⾡}
  \begin{Phonetics}{渡过}{du4guo4}[][HSK 7-9]
    \definition{v.}{viajar; atravessar; cruzar}[渡过最困难的时期。===Atravessar os momentos mais difíceis.]
  \end{Phonetics}
\end{Entry}

%%%%%%%%%% 温 %%%%%%%%%%
\subsection*{温}\addcontentsline{loh}{figure}{温}

\begin{Entry}{温}{12}{⽔}
  \begin{Phonetics}{温}{wen1}
    \definition{adj.}{morno; quente; suave}
    \definition{s.}{temperatura | doenças transmissíveis agudas; praga}
    \definition{v.}{aquecer; reaquecer; aquecer ligeiramente | revisar; repassar}
  \end{Phonetics}
\end{Entry}

\begin{Entry}{温和}{12,8}{⽔,⼝}
  \begin{Phonetics}{温和}{wen1he2}[][HSK 5]
    \definition{adj.}{gentil; suave; moderado}
  \end{Phonetics}
\end{Entry}

\begin{Entry}{温度}{12,9}{⽔,⼴}
  \begin{Phonetics}{温度}{wen1du4}[][HSK 2]
    \definition[度,级,档,个]{s.}{temperatura}
  \end{Phonetics}
\end{Entry}

\begin{Entry}{温度计}{12,9,4}{⽔,⼴,⾔}
  \begin{Phonetics}{温度计}{wen1du4ji4}
    \definition{s.}{termógrafo | termômetro}
  \end{Phonetics}
\end{Entry}

\begin{Entry}{温度表}{12,9,8}{⽔,⼴,⾐}
  \begin{Phonetics}{温度表}{wen1du4biao3}
    \definition{s.}{termômetro}
  \end{Phonetics}
\end{Entry}

\begin{Entry}{温度梯度}{12,9,11,9}{⽔,⼴,⽊,⼴}
  \begin{Phonetics}{温度梯度}{wen1du4ti1du4}
    \definition{s.}{gradiente de temperatura}
  \end{Phonetics}
\end{Entry}

\begin{Entry}{温柔}{12,9}{⽔,⽊}
  \begin{Phonetics}{温柔}{wen1rou2}
    \definition{adj.}{gentil e suave | terno | doce (comumente usado para descrever uma menina ou mulher)}
  \end{Phonetics}
\end{Entry}

\begin{Entry}{温顺}{12,9}{⽔,⾴}
  \begin{Phonetics}{温顺}{wen1shun4}
    \definition{adj.}{dócil; manso; gentil e obediente}
  \synonymref{和气}{he2qi5}
  \synonymref{暖和}{nuan3huo5}
  \synonymref{平和}{ping2he2}
  \synonymref{温和}{wen1he2}
  \synonymref{温暖}{wen1nuan3}
  \synonymref{温柔}{wen1rou2}
  \antonymref{暴躁}{bao4zao4}
  \antonymref{乖张}{guai1zhang1}
  \end{Phonetics}
\end{Entry}

\begin{Entry}{温暖}{12,13}{⽔,⽇}
  \begin{Phonetics}{温暖}{wen1nuan3}[][HSK 3]
    \definition{adj.}{caloroso; gentil; amigável | caloroso; quente}
    \definition{v.}{aquecer; fazer com que se sinta calor}
  \end{Phonetics}
\end{Entry}

%%%%%%%%%% 港 %%%%%%%%%%
\subsection*{港}\addcontentsline{loh}{figure}{港}

\begin{Entry}{港}{12}{⽔}
  \begin{Phonetics}{港}{gang3}[][HSK 7-9]
    \definition*{s.}{Hong Kong, abreviação de 香港 | Sobrenome: Gang}
    \definition{s.}{porto; ancoradouro}
  \seealsoref{香港}{xiang1gang3}
  \end{Phonetics}
\end{Entry}

\begin{Entry}{港口}{12,3}{⽔,⼝}
  \begin{Phonetics}{港口}{gang3kou3}[][HSK 6]
    \definition[个,座]{s.}{porto; locais com certas condições naturais e instalações portuárias para atracação de navios, embarque e desembarque de passageiros e coleta e distribuição de cargas}
  \end{Phonetics}
\end{Entry}

%%%%%%%%%% 渴 %%%%%%%%%%
\subsection*{渴}\addcontentsline{loh}{figure}{渴}

\begin{Entry}{渴}{12}{⽔}
  \begin{Phonetics}{渴}{ke3}[][HSK 1]
    \definition{adj.}{sedento}
    \definition{adv.}{ansiosamente; metáfora de urgência}
    \definition{v.}{desejar; ansiar por}
  \end{Phonetics}
\end{Entry}

\begin{Entry}{渴望}{12,11}{⽔,⽉}
  \begin{Phonetics}{渴望}{ke3wang4}[][HSK 5]
    \definition{v.}{aspirar; (ter sede, ansiar, desejar) por}
  \end{Phonetics}
\end{Entry}

%%%%%%%%%% 游 %%%%%%%%%%
\subsection*{游}\addcontentsline{loh}{figure}{游}

\begin{Entry}{游}{12}{⽔}
  \begin{Phonetics}{游}{you2}[][HSK 3]
    \definition*{s.}{Sobrenome: You}
    \definition{adj.}{itinerante; não fixo; que se move frequentemente}
    \definition{s.}{parte de um rio; uma seção do rio; trecho; bacia; curso}
    \definition{v.}{nadar | vagar por aí; caminhar; viajar; fazer turismo | associar com (comunicação) | vagar; passear; andar tranquilamente por todos os lugares}
  \end{Phonetics}
\end{Entry}

\begin{Entry}{游人}{12,2}{⽔,⼈}
  \begin{Phonetics}{游人}{you2ren2}[][HSK 6]
    \definition[个,名,位,批]{s.}{visitante (de um parque, etc.); turista}
  \end{Phonetics}
\end{Entry}

\begin{Entry}{游戏}{12,6}{⽔,⼽}
  \begin{Phonetics}{游戏}{you2xi4}[][HSK 3]
    \definition[场]{s.}{jogo; recreação; atividades recreativas, como esconde-esconde, adivinhar charadas, etc.; certas atividades esportivas não competitivas; jogos recreativos}
    \definition{v.}{jogar; fazer atividades divertidas e agradáveis, sozinho ou com outras pessoas}
  \end{Phonetics}
\end{Entry}

\begin{Entry}{游戏机}{12,6,6}{⽔,⼽,⽊}
  \begin{Phonetics}{游戏机}{you2xi4ji1}[][HSK 6]
    \definition[台]{s.}{jogador de videogame | console | videogame}
  \end{Phonetics}
\end{Entry}

\begin{Entry}{游行}{12,6}{⽔,⾏}
  \begin{Phonetics}{游行}{you2xing2}[][HSK 6]
    \definition{s.}{desfilar; marchar; manifestar-se; marchar em grupos nas ruas para celebrar, comemorar, manifestar-se, etc.}
  \end{Phonetics}
\end{Entry}

\begin{Entry}{游泳}{12,8}{⽔,⽔}
  \begin{Phonetics}{游泳}{you2/yong3}[][HSK 3]
    \definition[次]{s.}{natação; refere"-se ao esporte ou atividade de natação}
    \definition{v.+compl.}{nadar; pessoas ou animais nadando na água}
  \end{Phonetics}
\end{Entry}

\begin{Entry}{游泳池}{12,8,6}{⽔,⽔,⽔}
  \begin{Phonetics}{游泳池}{you2yong3chi2}[][HSK 5]
    \definition[场,个]{s.}{piscina; piscinas artificiais para natação, divididas em duas categorias: internas e externas}
  \seealsoref{泳池}{yong3chi2}
  \seealsoref{游泳馆}{you2yong3guan3}
  \end{Phonetics}
\end{Entry}

\begin{Entry}{游泳衣}{12,8,6}{⽔,⽔,⾐}
  \begin{Phonetics}{游泳衣}{you2yong3yi1}
    \definition{s.}{roupa de banho}
  \seealsoref{泳衣}{yong3yi1}
  \end{Phonetics}
\end{Entry}

\begin{Entry}{游泳馆}{12,8,11}{⽔,⽔,⾷}
  \begin{Phonetics}{游泳馆}{you2yong3guan3}
    \definition{s.}{natatório; piscina coberta; edifícios esportivos usados principalmente para esportes aquáticos, como natação, mergulho e polo aquático}
  \seealsoref{泳池}{yong3chi2}
  \seealsoref{游泳池}{you2yong3chi2}
  \end{Phonetics}
\end{Entry}

\begin{Entry}{游泳镜}{12,8,16}{⽔,⽔,⾦}
  \begin{Phonetics}{游泳镜}{you2yong3jing4}
    \definition{s.}{óculos de natação}
  \end{Phonetics}
\end{Entry}

\begin{Entry}{游玩}{12,8}{⽔,⽟}
  \begin{Phonetics}{游玩}{you2wan2}[][HSK 6]
    \definition{v.}{brincar; jogar; divertir-se | passear; vagar; fazer turismo}
  \end{Phonetics}
\end{Entry}

\begin{Entry}{游客}{12,9}{⽔,⼧}
  \begin{Phonetics}{游客}{you2ke4}[][HSK 2]
    \definition[个,位,名,群]{s.}{visitante; turista | (jogo online) jogador convidado}
  \end{Phonetics}
\end{Entry}

\begin{Entry}{游艇}{12,12}{⽔,⾈}
  \begin{Phonetics}{游艇}{you2ting3}
    \definition[只]{s.}{barcaça | iate}
  \end{Phonetics}
\end{Entry}

%%%%%%%%%% 渺 %%%%%%%%%%
\subsection*{渺}\addcontentsline{loh}{figure}{渺}

\begin{Entry}{渺}{12}{⽔}
  \begin{Phonetics}{渺}{miao3}
    \definition{adj.}{(uma vasta extensão de água) vasta; que se estende ao longe | distante e indistinto; vago | minúsculo; insignificante}
  \end{Phonetics}
\end{Entry}

\begin{Entry}{渺小}{12,3}{⽔,⼩}
  \begin{Phonetics}{渺小}{miao3xiao3}[][HSK 7-9]
    \definition{adj.}{minúsculo; reles; desprezível; insignificante}
  \end{Phonetics}
\end{Entry}

%%%%%%%%%% 湖 %%%%%%%%%%
\subsection*{湖}\addcontentsline{loh}{figure}{湖}

\begin{Entry}{湖}{12}{⽔}
  \begin{Phonetics}{湖}{hu2}[][HSK 2]
    \definition*{s.}{Huzhou, abreviação de 湖州 | Um nome que se refere às províncias de Hunan, 湖南,  e Hubei, 湖北}
    \definition[个,片]{s.}{lago}
  \seealsoref{湖北}{hu2bei3}
  \seealsoref{湖南}{hu2nan2}
  \seealsoref{湖州}{hu2zhou1}
  \end{Phonetics}
\end{Entry}

\begin{Entry}{湖北}{12,5}{⽔,⼔}
  \begin{Phonetics}{湖北}{hu2bei3}
    \definition*{s.}{Província de Hubei (Hupeh), na China central}
  \end{Phonetics}
\end{Entry}

\begin{Entry}{湖州}{12,6}{⽔,⼮}
  \begin{Phonetics}{湖州}{hu2zhou1}
    \definition*{s.}{Cidade de Huzhou, em Zhejiang}
  \end{Phonetics}
\end{Entry}

\begin{Entry}{湖泊}{12,8}{⽔,⽔}
  \begin{Phonetics}{湖泊}{hu2po1}[][HSK 7-9]
    \definition[个,片,些]{s.}{lago; nome geral para lagos}[湖泊中有丰富的鱼类。===O lago é abundante em peixes.]
  \end{Phonetics}
\end{Entry}

\begin{Entry}{湖南}{12,9}{⽔,⼗}
  \begin{Phonetics}{湖南}{hu2nan2}
    \definition*{s.}{Província de Hunan}
  \end{Phonetics}
\end{Entry}

%%%%%%%%%% 湿 %%%%%%%%%%
\subsection*{湿}\addcontentsline{loh}{figure}{湿}

\begin{Entry}{湿}{12}{⽔}
  \begin{Phonetics}{湿}{shi1}[][HSK 4]
    \definition{adj.}{molhado; úmido; algo com água ou com muita água dentro}
  \end{Phonetics}
\end{Entry}

\begin{Entry}{湿度}{12,9}{⽔,⼴}
  \begin{Phonetics}{湿度}{shi1du4}[][HSK 7-9]
    \definition{s.}{umidade; a quantidade de água contida em uma substância; especificamente, a quantidade de umidade contida no ar}
  \end{Phonetics}
\end{Entry}

\begin{Entry}{湿润}{12,10}{⽔,⽔}
  \begin{Phonetics}{湿润}{shi1run4}[][HSK 7-9]
    \definition{adj.}{úmido; molhado; (solo, ar, etc.) úmido e hidratado}
  \synonymref{潮湿}{chao2shi1}
  \antonymref{干燥}{gan1zao4}
  \end{Phonetics}
\end{Entry}

%%%%%%%%%% 溅 %%%%%%%%%%
\subsection*{溅}\addcontentsline{loh}{figure}{溅}

\begin{Entry}{溅}{12}{⽔}
  \begin{Phonetics}{溅}{jian4}[][HSK 7-9]
    \definition{s.}{respingo; o líquido foi ejetado em todas as direções devido ao impacto}
    \definition{v.}{respingar; salpicar}
  \end{Phonetics}
\end{Entry}

%%%%%%%%%% 滑 %%%%%%%%%%
\subsection*{滑}\addcontentsline{loh}{figure}{滑}

\begin{Entry}{滑}{12}{⽔}
  \begin{Phonetics}{滑}{hua2}[][HSK 5]
    \definition*{s.}{Sobrenome: Hua}
    \definition{adj.}{escorregadio; liso; objetos com superfícies lisas e baixo atrito | astuto; ardiloso; escorregadio}
    \definition{v.}{escorregar; deslizar | se atrapalhar; se safar de algo}
  \end{Phonetics}
\end{Entry}

\begin{Entry}{滑冰}{12,6}{⽔,⼎}
  \begin{Phonetics}{滑冰}{hua2bing1}[][HSK 7-9]
    \definition{s.}{patinação no gelo; um evento esportivo em que os atletas usam patins especiais para patinar no gelo, competindo em velocidade ou realizando manobras}
    \definition{v.}{patinar; patinar no gelo; deslizar no gelo}
  \end{Phonetics}
\end{Entry}

\begin{Entry}{滑梯}{12,11}{⽔,⽊}
  \begin{Phonetics}{滑梯}{hua2ti1}[][HSK 7-9]
    \definition{s.}{escorregador infantil}
  \end{Phonetics}
\end{Entry}

\begin{Entry}{滑雪}{12,11}{⽔,⾬}
  \begin{Phonetics}{滑雪}{hua2/xue3}[][HSK 7-9]
    \definition{v.+compl.}{esquiar; praticar esqui; usar pranchas especiais nos pés para deslizar na neve}
  \end{Phonetics}
\end{Entry}

\begin{Entry}{滑稽}{12,15}{⽔,⽲}
  \begin{Phonetics}{滑稽}{hua2ji1}[][HSK 7-9]
    \definition{adj.}{engraçado; divertido; cômico; (palavras, ações ou gestos) que fazem as pessoas rirem}
    \definition{s.}{conversa cômica; um tipo de arte popular, popular nas áreas de Xangai, Jiangsu e Zhejiang, semelhante ao \emph{crosstalk}}
  \end{Phonetics}
\end{Entry}

%%%%%%%%%% 焚 %%%%%%%%%%
\subsection*{焚}\addcontentsline{loh}{figure}{焚}

\begin{Entry}{焚}{12}{⽕}
  \begin{Phonetics}{焚}{fen2}
    \definition{v.}{queimar}
  \end{Phonetics}
\end{Entry}

\begin{Entry}{焚香}{12,9}{⽕,⾹}
  \begin{Phonetics}{焚香}{fen2xiang1}
    \definition{v.}{queimar incenso}
  \end{Phonetics}
\end{Entry}

\begin{Entry}{焚烧}{12,10}{⽕,⽕}
  \begin{Phonetics}{焚烧}{fen2shao1}[][HSK 7-9]
    \definition{v.}{queimar; incendiar; colocar fogo}
  \end{Phonetics}
\end{Entry}

%%%%%%%%%% 焦 %%%%%%%%%%
\subsection*{焦}\addcontentsline{loh}{figure}{焦}

\begin{Entry}{焦}{12}{⽕}
  \begin{Phonetics}{焦}{jiao1}[][HSK 7-9]
    \definition*{s.}{Sobrenome: Jiao}
    \definition{adj.}{queimado; chamuscado; carbonizado | preocupado; ansioso}
    \definition{clas.}{J; Joule, abreviação}
    \definition{pref.}{(química) piro-}
    \definition{s.}{Metalurgia: coque}
  \end{Phonetics}
\end{Entry}

\begin{Entry}{焦急}{12,9}{⽕,⼼}
  \begin{Phonetics}{焦急}{jiao1ji2}[][HSK 7-9]
    \definition{adj.}{ansioso; preocupado; com pressa}
  \end{Phonetics}
\end{Entry}

\begin{Entry}{焦点}{12,9}{⽕,⽕}
  \begin{Phonetics}{焦点}{jiao1dian3}[][HSK 6]
    \definition{s.}{foco; ponto focal; Matemática: refere"-se a um ponto que tem uma relação especial com uma elipse, hipérbole, parábola, etc. | foco; ponto focal; Óptica: refere"-se à intersecção de feixes de luz paralelos após serem refratados por uma lente ou refletidos por um espelho curvo | foco; questão central; metaforicamente, uma coisa ou princípio que chama a atenção para o foco}
  \end{Phonetics}
\end{Entry}

\begin{Entry}{焦虑}{12,10}{⽕,⾌}
  \begin{Phonetics}{焦虑}{jiao1lv4}[][HSK 7-9]
    \definition{adj.}{ansioso; preocupado; apreensivo}
  \end{Phonetics}
\end{Entry}

\begin{Entry}{焦距}{12,11}{⽕,⾜}
  \begin{Phonetics}{焦距}{jiao1ju4}[][HSK 7-9]
    \definition{s.}{Ótica: distância focal; comprimento focal}
  \end{Phonetics}
\end{Entry}

\begin{Entry}{焦躁}{12,20}{⽕,⾜}
  \begin{Phonetics}{焦躁}{jiao1zao4}[][HSK 7-9]
    \definition{adj.}{impaciente; inquieto de ansiedade; ansioso e irritadiço}
  \end{Phonetics}
\end{Entry}

%%%%%%%%%% 然 %%%%%%%%%%
\subsection*{然}\addcontentsline{loh}{figure}{然}

\begin{Entry}{然}{12}{⽕}
  \begin{Phonetics}{然}{ran2}
    \definition{conj.}{mas | no entanto}
  \end{Phonetics}
\end{Entry}

\begin{Entry}{然后}{12,6}{⽕,⼝}
  \begin{Phonetics}{然后}{ran2hou4}[][HSK 2]
    \definition{conj.}{então; depois disso; posteriormente; indica que algo segue após uma ação ou situação}
  \end{Phonetics}
\end{Entry}

\begin{Entry}{然而}{12,6}{⽕,⽽}
  \begin{Phonetics}{然而}{ran2'er2}[][HSK 4]
    \definition{conj.}{ainda; mas; contudo; todavia; usado no início de uma frase para indicar uma transição; para indicar uma transição, geralmente é precedido por uma conjunção como 虽然 para indicar concessão}
  \seealsoref{虽然}{sui1ran2}
  \end{Phonetics}
\end{Entry}

%%%%%%%%%% 煮 %%%%%%%%%%
\subsection*{煮}\addcontentsline{loh}{figure}{煮}

\begin{Entry}{煮}{12}{⽕}
  \begin{Phonetics}{煮}{zhu3}[][HSK 6]
    \definition*{s.}{Sobrenome: Zhu}
    \definition{v.}{ferver; cozinhar; aquecer alimentos ou outros itens em água}
  \end{Phonetics}
\end{Entry}

%%%%%%%%%% 牌 %%%%%%%%%%
\subsection*{牌}\addcontentsline{loh}{figure}{牌}

\begin{Entry}{牌}{12}{⽚}
  \begin{Phonetics}{牌}{pai2}[][HSK 4]
    \definition[块,副,张,个,种]{s.}{placa; tabuleta; quadro; placar | marca; marca registrada; marca comercial; \emph{trademark} | cartas, dominó, etc. | a tonalidade de uma música}
  \end{Phonetics}
\end{Entry}

\begin{Entry}{牌子}{12,3}{⽚,⼦}
  \begin{Phonetics}{牌子}{pai2zi5}[][HSK 3]
    \definition[个,种,块]{s.}{sinal; placa; placas feitas de madeira ou outros materiais, geralmente com texto nelas | marca; marca registrada; um nome especial dado por uma empresa ao seu próprio produto}
  \end{Phonetics}
\end{Entry}

\begin{Entry}{牌照}{12,13}{⽚,⽕}
  \begin{Phonetics}{牌照}{pai2zhao4}[][HSK 7-9]
    \definition{s.}{placa de matrícula; certificado de licenciamento; certificado de registro de veículo ou licença comercial emitida pelo departamento administrativo competente}
  \end{Phonetics}
\end{Entry}

%%%%%%%%%% 猩 %%%%%%%%%%
\subsection*{猩}\addcontentsline{loh}{figure}{猩}

\begin{Entry}{猩}{12}{⽝}
  \begin{Phonetics}{猩}{xing1}
    \definition[只]{s.}{orangotango}
  \end{Phonetics}
\end{Entry}

\begin{Entry}{猩猩}{12,12}{⽝,⽝}
  \begin{Phonetics}{猩猩}{xing1xing5}
    \definition{s.}{orangotango}
  \end{Phonetics}
\end{Entry}

%%%%%%%%%% 猴 %%%%%%%%%%
\subsection*{猴}\addcontentsline{loh}{figure}{猴}

\begin{Entry}{猴}{12}{⽝}
  \begin{Phonetics}{猴}{hou2}[][HSK 5]
    \definition{adj.}{esperto; inteligente; perspicaz | travesso (menino)}
    \definition[只,群]{s.}{macaco}
  \end{Phonetics}
\end{Entry}

\begin{Entry}{猴子}{12,3}{⽝,⼦}
  \begin{Phonetics}{猴子}{hou2zi5}
    \definition[只]{s.}{macaco}
  \end{Phonetics}
\end{Entry}

%%%%%%%%%% 琴 %%%%%%%%%%
\subsection*{琴}\addcontentsline{loh}{figure}{琴}

\begin{Entry}{琴}{12}{⽟}
  \begin{Phonetics}{琴}{qin2}[][HSK 5]
    \definition*{s.}{Sobrenome: Qin}
    \definition[架,台]{s.}{cítara; qin; guqin (um instrumento de cordas dedilhadas com sete cordas, em alguns aspectos semelhante à cítara)  | nome genérico para certos instrumentos musicais}
  \end{Phonetics}
\end{Entry}

\begin{Entry}{琴键}{12,13}{⽟,⾦}
  \begin{Phonetics}{琴键}{qin2jian4}
    \definition{s.}{tecla de piano}
  \end{Phonetics}
\end{Entry}

%%%%%%%%%% 甁 %%%%%%%%%%
\subsection*{甁}\addcontentsline{loh}{figure}{甁}

\begin{Entry}{甁}{12}{⽡}
  \begin{Phonetics}{甁}{ping2}
    \variantof{瓶}
  \end{Phonetics}
\end{Entry}

%%%%%%%%%% 番 %%%%%%%%%%
\subsection*{番}\addcontentsline{loh}{figure}{番}

\begin{Entry}{番}{12}{⽥}
  \begin{Phonetics}{番}{fan1}[][HSK 6]
    \definition{adj.}{estrangeiro; de tribos estrangeiras; estrangeiro ou alienígena}
    \definition{clas.}{usado para o número de vezes que uma ação é executada, equivalente a 回 ou 次 | usado para o tipo de coisas, equivalente a 种}
    \definition{s.}{estrangeiro; de tribos estrangeiras; Arcaico: refere"-se a países estrangeiros ou raças estrangeiras | tomate; batata"-doce | aborígenes; nativos; povos indígenas}
    \definition{v.}{revezar; rotacionar; substituir}
  \seealsoref{次}{ci4}
  \seealsoref{回}{hui2}
  \seealsoref{种}{zhong3}
  \end{Phonetics}
\end{Entry}

\begin{Entry}{番茄}{12,8}{⽥,⾋}
  \begin{Phonetics}{番茄}{fan1qie2}[][HSK 6]
    \definition[个,斤,磅,公斤]{s.}{tomate | tomateiro}
  \end{Phonetics}
\end{Entry}

%%%%%%%%%% 疏 %%%%%%%%%%
\subsection*{疏}\addcontentsline{loh}{figure}{疏}

\begin{Entry}{疏}{12}{⽦}
  \begin{Phonetics}{疏}{shu1}
    \definition*{s.}{Sobrenome: Shu}
    \definition{adj.}{fino; esparso; disperso | espalhado; disperso; difuso; a distância entre as coisas é grande; as lacunas entre as partes das coisas são grandes | distante; relacionamento distante; não próximo (de relações familiares ou sociais) | não familiarizado com; desconhecido | escasso; vazio}
    \definition{s.}{memorial; memorial ao trono; um texto em que um ministro na era feudal apresentava seus assuntos ao monarca em detalhes | comentário; anotações mais detalhadas de livros antigos do que 注}
    \definition{v.}{dragar (um rio, etc.) | negligenciar | dispersar; espalhar}
  \seealsoref{注}{zhu4}
  \antonymref{密}{mi4}
  \end{Phonetics}
\end{Entry}

\begin{Entry}{疏导}{12,6}{⽦,⼨}
  \begin{Phonetics}{疏导}{shu1dao3}[][HSK 7-9]
    \definition{v.}{drenar; a analogia de guiar a água para fluir em um rio também é usada para descrever o ato de fazer com que pessoas e veículos presos no trânsito comecem a se mover | aconselhar; melhorar estados psicológicos negativos através de orientação}
  \synonymref{沟通}{gou1tong1}
  \synonymref{疏通}{shu1tong1}
  \synonymref{引导}{yin3dao3}
  \antonymref{堵塞}{du3se4}
  \end{Phonetics}
\end{Entry}

\begin{Entry}{疏忽}{12,8}{⽦,⼼}
  \begin{Phonetics}{疏忽}{shu1hu5}[][HSK 7-9]
    \definition{v.}{cometer um deslize; negligenciar; ser descuidado; não perceber por descuido}
  \synonymref{粗心}{cu1xin1}
  \synonymref{大意}{da4yi5}
  \synonymref{怠慢}{dai4man4}
  \synonymref{忽略}{hu1lve4}
  \synonymref{忽视}{hu1shi4}
  \synonymref{马虎}{ma3hu5}
  \synonymref{松弛}{song1chi2}
  \synonymref{无视}{wu2shi4}
  \antonymref{谨慎}{jin3shen4}
  \antonymref{精细}{jing1xi4}
  \antonymref{精心}{jing1xin1}
  \antonymref{警惕}{jing3ti4}
  \antonymref{留意}{liu2/yi4}
  \antonymref{慎重}{shen4zhong4}
  \antonymref{注意}{zhu4/yi4}
  \antonymref{仔细}{zi3xi4}
  \end{Phonetics}
\end{Entry}

\begin{Entry}{疏通}{12,10}{⽦,⾡}
  \begin{Phonetics}{疏通}{shu1tong1}[][HSK 7-9]
    \definition{v.}{dragar | mediar entre duas partes}
  \synonymref{畅通}{chang4tong1}
  \synonymref{沟通}{gou1tong1}
  \synonymref{疏导}{shu1dao3}
  \synonymref{运动}{yun4dong5}
  \antonymref{堵塞}{du3se4}
  \end{Phonetics}
\end{Entry}

\begin{Entry}{疏散}{12,12}{⽦,⽁}
  \begin{Phonetics}{疏散}{shu1san4}[][HSK 7-9]
    \definition{adj.}{esparso; disperso; espalhado; escasso}
    \definition{v.}{desocupar; dispersar; evacuar; dispersar as pessoas ou coisas que estão reunidas}
  \synonymref{分散}{fen1san4}
  \antonymref{集结}{ji2jie2}
  \antonymref{集中}{ji2zhong1}
  \antonymref{紧紧}{jin3jin3}
  \antonymref{密集}{mi4ji2}
  \end{Phonetics}
\end{Entry}

%%%%%%%%%% 痛 %%%%%%%%%%
\subsection*{痛}\addcontentsline{loh}{figure}{痛}

\begin{Entry}{痛}{12}{⽧}
  \begin{Phonetics}{痛}{tong4}[][HSK 3,7-9]
    \definition{adv.}{extremamente; profundamente; amargamente}
    \definition{s.}{dor; sofrimento | tristeza; pesar}
    \definition{v.}{sentir dor; ter dor | sentir tristeza; sentir aflição}
  \end{Phonetics}
\end{Entry}

\begin{Entry}{痛心}{12,4}{⽧,⼼}
  \begin{Phonetics}{痛心}{tong4xin1}[][HSK 7-9]
    \definition{adj.}{de partir o coração; dolorido; angustiado; aflito; enlutado; extrema tristeza}
  \end{Phonetics}
\end{Entry}

\begin{Entry}{痛快}{12,7}{⽧,⼼}
  \begin{Phonetics}{痛快}{tong4kuai5}[][HSK 4]
    \definition{adj.}{encantado; alegre; muito feliz; confortável | franco; direto; simples e direto}
  \end{Phonetics}
\end{Entry}

\begin{Entry}{痛苦}{12,8}{⽧,⾋}
  \begin{Phonetics}{痛苦}{tong4ku3}[][HSK 3]
    \definition{adj.}{doloroso; angustiado; sentindo"-se muito desconfortável física ou mentalmente}
    \definition[降,种]{s.}{dor; agonia; sofrimento; refere"-se a um estado ou sentimento de extremo desconforto físico ou mental}
  \end{Phonetics}
\end{Entry}

\begin{Entry}{痛骂}{12,9}{⽧,⾺}
  \begin{Phonetics}{痛骂}{tong4ma4}
    \definition{v.}{repreender severamente}
  \end{Phonetics}
\end{Entry}

%%%%%%%%%% 痠 %%%%%%%%%%
\subsection*{痠}\addcontentsline{loh}{figure}{痠}

\begin{Entry}{痠}{12}{⽧}
  \begin{Phonetics}{痠}{suan1}
    \definition{v.}{doer | estar dolorido}
    \variantof{酸}
  \end{Phonetics}
\end{Entry}

%%%%%%%%%% 登 %%%%%%%%%%
\subsection*{登}\addcontentsline{loh}{figure}{登}

\begin{Entry}{登}{12}{⽨}
  \begin{Phonetics}{登}{deng1}[][HSK 4]
    \definition{v.}{subir; montar; escalar (uma altura) | publicar; registrar; inserir | recolher e levar para a eira | pisar em; pisar | calçar (calçados ou calças) | partir; começar uma jornada; embarcar em uma jornada}
  \end{Phonetics}
\end{Entry}

\begin{Entry}{登山}{12,3}{⽨,⼭}
  \begin{Phonetics}{登山}{deng1 shan1}[][HSK 4]
    \definition{s.}{escalar; fazer alpinismo; subir uma montanha}
  \end{Phonetics}
\end{Entry}

\begin{Entry}{登记}{12,5}{⽨,⾔}
  \begin{Phonetics}{登记}{deng1/ji4}[][HSK 4]
    \definition{v.+compl.}{registrar-se; fazer o \emph{check-in} | registrar; reportar; informar; relatar por escrito a um superior ou autoridade relevante (usado principalmente para documentos legais)}
  \end{Phonetics}
\end{Entry}

\begin{Entry}{登机}{12,6}{⽨,⽊}
  \begin{Phonetics}{登机}{deng1ji1}[][HSK 7-9]
    \definition{v.}{embarcar; embarcar em um avião}
  \end{Phonetics}
\end{Entry}

\begin{Entry}{登陆}{12,7}{⽨,⾩}
  \begin{Phonetics}{登陆}{deng1/lu4}[][HSK 7-9]
    \definition{v.+compl.}{desembarcar; chegar à costa | entrar em um mercado; Metáfora: mercadorias entram em um determinado mercado e começam a ser vendidas; comerciantes entram em um determinado mercado e começam a fazer negócios}
  \end{Phonetics}
\end{Entry}

\begin{Entry}{登录}{12,8}{⽨,⼹}
  \begin{Phonetics}{登录}{deng1lu4}[][HSK 4]
    \definition{v.}{fazer \emph{logon}; fazer \emph{login} | gravar; registrar; computadores eletrônicos e sua terminologia de rede, referindo"-se ao acesso ao sistema operacional ou ao site a ser visitado}
  \end{Phonetics}
\end{Entry}

%%%%%%%%%% 短 %%%%%%%%%%
\subsection*{短}\addcontentsline{loh}{figure}{短}

\begin{Entry}{短}{12}{⽮}
  \begin{Phonetics}{短}{duan3}[][HSK 2]
    \definition{adj.}{curto; comprimento pequeno de uma extremidade à outra | curto; breve; a distância entre o ponto inicial e o ponto final de um determinado período é pequena | raso; superficial}
    \definition{s.}{falha; defeito; ponto fraco; desvantagens | tonelada curta (EUA)}
    \definition{v.}{dever; carecer}
  \antonymref{长}{zhang3}
  \end{Phonetics}
\end{Entry}

\begin{Entry}{短少}{12,4}{⽮,⼩}
  \begin{Phonetics}{短少}{duan3shao3}
    \definition{v.}{estar aquém do valor total}
  \end{Phonetics}
\end{Entry}

\begin{Entry}{短片}{12,4}{⽮,⽚}
  \begin{Phonetics}{短片}{duan3pian4}[][HSK 6]
    \definition{s.}{curta-metragem; curtas-metragens documentais ou educativos exibidos individualmente ou em série}
  \end{Phonetics}
\end{Entry}

\begin{Entry}{短处}{12,5}{⽮,⼡}
  \begin{Phonetics}{短处}{duan3chu5}[][HSK 3]
    \definition[个]{s.}{deficiência; ponto fraco; defeito; fraqueza}
  \end{Phonetics}
\end{Entry}

\begin{Entry}{短视}{12,8}{⽮,⾒}
  \begin{Phonetics}{短视}{duan3shi4}
    \definition{adj.}{míope}
  \end{Phonetics}
\end{Entry}

\begin{Entry}{短促}{12,9}{⽮,⼈}
  \begin{Phonetics}{短促}{duan3cu4}
    \definition{adj.}{curto (tom de voz) | fugaz | ofegante (respiração) | curto no tempo}
  \end{Phonetics}
\end{Entry}

\begin{Entry}{短信}{12,9}{⽮,⼈}
  \begin{Phonetics}{短信}{duan3xin4}[][HSK 2]
    \definition[条,个,封]{s.}{mensagem de texto; refere"-se especificamente a mensagens de texto curtas, imagens, etc., enviadas ou recebidas por celular}
  \end{Phonetics}
\end{Entry}

\begin{Entry}{短缺}{12,10}{⽮,⽸}
  \begin{Phonetics}{短缺}{duan3que1}[][HSK 7-9]
    \definition{s.}{falta; déficit; escassez; insuficiência}
  \end{Phonetics}
\end{Entry}

\begin{Entry}{短暂}{12,12}{⽮,⽇}
  \begin{Phonetics}{短暂}{duan3zan4}[][HSK 7-9]
    \definition{adj.}{breve; transitório; momentâneo; de curta duração}
  \end{Phonetics}
\end{Entry}

\begin{Entry}{短期}{12,12}{⽮,⽉}
  \begin{Phonetics}{短期}{duan3qi1}[][HSK 3]
    \definition{adj.}{de curta duração; de prazo curto}
    \definition[个]{s.}{curto prazo}
  \end{Phonetics}
\end{Entry}

\begin{Entry}{短裤}{12,12}{⽮,⾐}
  \begin{Phonetics}{短裤}{duan3ku4}[][HSK 3]
    \definition[条]{s.}{calças curtas; calção; \emph{shorts}; calças com bainha acima do joelho}
  \end{Phonetics}
\end{Entry}

\begin{Entry}{短跑}{12,12}{⽮,⾜}
  \begin{Phonetics}{短跑}{duan3 pao3}
    \definition{s.}{corrida de curta distância; corrida rápida}
  \antonymref{长跑}{chang2pao3}
  \end{Phonetics}
\end{Entry}

%%%%%%%%%% 硬 %%%%%%%%%%
\subsection*{硬}\addcontentsline{loh}{figure}{硬}

\begin{Entry}{硬}{12}{⽯}
  \begin{Phonetics}{硬}{ying4}[][HSK 5]
    \definition{adj.}{duro; rígido; resistente;  objeto resistente e não se deforma facilmente quando submetido a forças externas | firme; forte; resistente; obstinado; (vontade, atitude, etc.) inabalável, forte e poderoso | capaz (pessoa); boa (qualidade) | rígido; severo; sem flexibilidade | duro; rígido; rigoroso; imutável}
    \definition{adv.}{conseguir fazer algo com dificuldade; indica fazer algo à força, independentemente das circunstâncias}
  \antonymref{软}{ruan3}
  \end{Phonetics}
\end{Entry}

\begin{Entry}{硬件}{12,6}{⽯,⼈}
  \begin{Phonetics}{硬件}{ying4jian4}[][HSK 5]
    \definition[种]{s.}{\emph{hardware}; nome genérico dado aos vários elementos, componentes e dispositivos que constituem um computador | máquina, materiais; equipamento; referência a máquinas, equipamentos, materiais físicos, etc., utilizados nos processos de produção, pesquisa científica, gestão, etc.}
  \end{Phonetics}
\end{Entry}

%%%%%%%%%% 确 %%%%%%%%%%
\subsection*{确}\addcontentsline{loh}{figure}{确}

\begin{Entry}{确}{12}{⽯}
  \begin{Phonetics}{确}{que4}
    \definition{adj.}{autenticado | sólido | firme | real | verdadeiro}
  \end{Phonetics}
\end{Entry}

\begin{Entry}{确切}{12,4}{⽯,⼑}
  \begin{Phonetics}{确切}{que4qie4}[][HSK 7-9]
    \definition{adj.}{verdadeiro; exato; definido; preciso; apropriado | verdadeiro; seguro; confiável; credível; digno de confiança}
  \end{Phonetics}
\end{Entry}

\begin{Entry}{确认}{12,4}{⽯,⾔}
  \begin{Phonetics}{确认}{que4ren4}[][HSK 4]
    \definition{v.}{afirmar; confirmar; reconhecer; confirmar explicitamente (fatos, princípios, etc.)}
  \end{Phonetics}
\end{Entry}

\begin{Entry}{确立}{12,5}{⽯,⽴}
  \begin{Phonetics}{确立}{que4li4}[][HSK 5]
    \definition{v.}{estabelecer; criar; construir; estabelecer ou consolidar firmemente}
  \end{Phonetics}
\end{Entry}

\begin{Entry}{确诊}{12,7}{⽯,⾔}
  \begin{Phonetics}{确诊}{que4zhen3}[][HSK 7-9]
    \definition{v.}{diagnosticar; fazer um diagnóstico definitivo; fazer um diagnóstico preciso (da doença)}
  \end{Phonetics}
\end{Entry}

\begin{Entry}{确定}{12,8}{⽯,⼧}
  \begin{Phonetics}{确定}{que4ding4}[][HSK 3]
    \definition{adj.}{definido; certo; claro}
    \definition{v.}{firmar; definir; determinar; tomar uma decisão clara e não mudar}
  \end{Phonetics}
\end{Entry}

\begin{Entry}{确实}{12,8}{⽯,⼧}
  \begin{Phonetics}{确实}{que4shi2}[][HSK 3]
    \definition{adj.}{verdadeiro; confiável; autêntico}
    \definition{adv.}{verdadeiramente; realmente; de fato; afirmar a autenticidade de fatos objetivos}
  \end{Phonetics}
\end{Entry}

\begin{Entry}{确保}{12,9}{⽯,⼈}
  \begin{Phonetics}{确保}{que4bao3}[][HSK 3]
    \definition{v.}{assegurar; garantir; manter ou garantir com certeza}
  \end{Phonetics}
\end{Entry}

\begin{Entry}{确信}{12,9}{⽯,⼈}
  \begin{Phonetics}{确信}{que4xin4}[][HSK 7-9]
    \definition{s.}{confirmação; informações autênticas e confiáveis}
    \definition{v.}{ter certeza; acreditar firmemente; estar convencido; acreditar plenamente, sem dúvida alguma}
  \end{Phonetics}
\end{Entry}

\begin{Entry}{确凿}{12,12}{⽯,⼐}
  \begin{Phonetics}{确凿}{que4zao2}[][HSK 7-9]
    \definition{adj.}{autêntico; irrefutável; inegável; absolutamente verdadeiro; inegavelmente verdadeiro}
  \end{Phonetics}
\end{Entry}

%%%%%%%%%% 禅 %%%%%%%%%%
\subsection*{禅}\addcontentsline{loh}{figure}{禅}

\begin{Entry}{禅}{12}{⽰}
  \begin{Phonetics}{禅}{chan2}
    \definition{s.}{Budismo: contemplação prolongada e intensa; meditação profunda | budista; refere"-se geralmente a coisas relacionadas ao budismo}
  \end{Phonetics}
  \begin{Phonetics}{禅}{shan4}
    \definition{v.}{abdicar e entregar a coroa a outra pessoa}
  \end{Phonetics}
\end{Entry}

\begin{Entry}{禅杖}{12,7}{⽰,⽊}
  \begin{Phonetics}{禅杖}{chan2zhang4}[][HSK 7-9]
    \definition[根,支]{s.}{cajado (bastão) do monge budista | uma bengala com cabeça acolchoada para bater na cabeça de quem adormece}
  \end{Phonetics}
\end{Entry}

%%%%%%%%%% 禽 %%%%%%%%%%
\subsection*{禽}\addcontentsline{loh}{figure}{禽}

\begin{Entry}{禽}{12}{⽱}
  \begin{Phonetics}{禽}{qin2}
    \definition*{s.}{Sobrenome: Qin}
    \definition[只]{s.}{aves; pássaros | termo genérico para aves e animais}
  \end{Phonetics}
\end{Entry}

%%%%%%%%%% 程 %%%%%%%%%%
\subsection*{程}\addcontentsline{loh}{figure}{程}

\begin{Entry}{程}{12}{⽲}
  \begin{Phonetics}{程}{cheng2}
    \definition{s.}{regra; regulamento; lei | ordem; procedimento | jornada; etapa de uma jornada; estrada; um trecho de estrada | distância percorrida ou movida por um objeto | programação | medição; termo geral para pesos e medidas}
  \end{Phonetics}
\end{Entry}

\begin{Entry}{程序}{12,7}{⽲,⼴}
  \begin{Phonetics}{程序}{cheng2xu4}[][HSK 4]
    \definition[个,套,种]{s.}{ordem; curso; sequência; procedimento; ordem em que algo é feito; também, um determinado número de etapas em um trabalho | programa; conjunto de instruções de computador projetado em sequência para permitir que um computador execute uma ou mais operações}
  \end{Phonetics}
\end{Entry}

\begin{Entry}{程序设计}{12,7,6,4}{⽲,⼴,⾔,⾔}
  \begin{Phonetics}{程序设计}{cheng2xu4she4ji4}
    \definition{s.}{programação de computadores}
  \end{Phonetics}
\end{Entry}

\begin{Entry}{程序库}{12,7,7}{⽲,⼴,⼴}
  \begin{Phonetics}{程序库}{cheng2xu4ku4}
    \definition{s.}{biblioteca de funções e procedimentos para programas de computador}
  \end{Phonetics}
\end{Entry}

\begin{Entry}{程度}{12,9}{⽲,⼴}
  \begin{Phonetics}{程度}{cheng2du4}[][HSK 3]
    \definition[种]{s.}{nível; grau (de cultura, educação, aprendizagem, etc.) | extensão; grau; a situação, o nível ou o estágio em que as coisas mudam}
  \end{Phonetics}
\end{Entry}

\begin{Entry}{程控}{12,11}{⽲,⼿}
  \begin{Phonetics}{程控}{cheng2kong4}
    \definition{s.}{programado | sob controle automático}
  \end{Phonetics}
\end{Entry}

%%%%%%%%%% 稍 %%%%%%%%%%
\subsection*{稍}\addcontentsline{loh}{figure}{稍}

\begin{Entry}{稍}{12}{⽲}
  \begin{Phonetics}{稍}{shao1}[][HSK 5]
    \definition{adv.}{ligeiramente; um pouco; um pouquinho}
  \seealsoref{稍稍}{shao1shao1}
  \end{Phonetics}
  \begin{Phonetics}{稍}{shao4}
    \definition{adv.}{à vontade}
  \end{Phonetics}
\end{Entry}

\begin{Entry}{稍后}{12,6}{⽲,⼝}
  \begin{Phonetics}{稍后}{shao1hou4}[][HSK 7-9]
    \definition{s.}{um pouco mais tarde (no tempo ou no espaço)}
  \end{Phonetics}
\end{Entry}

\begin{Entry}{稍候}{12,10}{⽲,⼈}
  \begin{Phonetics}{稍候}{shao1hou4}[][HSK 7-9]
    \definition{v.}{aguardar um momento}
  \end{Phonetics}
\end{Entry}

\begin{Entry}{稍稍}{12,12}{⽲,⽲}
  \begin{Phonetics}{稍稍}{shao1shao1}[][HSK 7-9]
    \definition{adv.}{um pouco; ligeiramente}
  \end{Phonetics}
\end{Entry}

\begin{Entry}{稍微}{12,13}{⽲,⼻}
  \begin{Phonetics}{稍微}{shao1wei1}[][HSK 5]
    \definition{adv.}{um pouco; um pouquinho; uma ninharia; indica que a quantidade é pequena ou o grau é superficial}
  \end{Phonetics}
\end{Entry}

%%%%%%%%%% 税 %%%%%%%%%%
\subsection*{税}\addcontentsline{loh}{figure}{税}

\begin{Entry}{税}{12}{⽲}
  \begin{Phonetics}{税}{shui4}[][HSK 6]
    \definition*{s.}{Sobrenome: Shui}
    \definition{s.}{imposto; taxa; tarifa}
  \end{Phonetics}
\end{Entry}

\begin{Entry}{税务}{12,5}{⽲,⼒}
  \begin{Phonetics}{税务}{shui4wu4}[][HSK 7-9]
    \definition{s.}{tributação; administração tributária | serviço de receita estadual | serviços de tributação}
  \end{Phonetics}
\end{Entry}

\begin{Entry}{税收}{12,6}{⽲,⽁}
  \begin{Phonetics}{税收}{shui4shou1}[][HSK 7-9]
    \definition{s.}{receita tributária; imposto; o Estado obtém receita através da cobrança de impostos de acordo com a lei}
  \end{Phonetics}
\end{Entry}

%%%%%%%%%% 窗 %%%%%%%%%%
\subsection*{窗}\addcontentsline{loh}{figure}{窗}

\begin{Entry}{窗}{12}{⽳}
  \begin{Phonetics}{窗}{chuang1}
    \definition[扇,个]{s.}{janela}
  \end{Phonetics}
\end{Entry}

\begin{Entry}{窗口}{12,3}{⽳,⼝}
  \begin{Phonetics}{窗口}{chuang1kou3}[][HSK 6]
    \definition[个,号]{s.}{janela; em frente à janela; perto da janela | janela; postigo; refere"-se a uma abertura especial em forma de janela | janela; meio; intermediário; peça de exibição; campo de testes; uma metáfora para um lugar com muitas interações com o mundo exterior e através do qual o entendimento mútuo é alcançado |  janela; uma metáfora para um lugar que pode refletir ou exibir a totalidade ou parte de algo |  caixa de diálogo; uma caixa de operação quadrada para aplicativos ou arquivos que aparece na tela do computador}
  \end{Phonetics}
\end{Entry}

\begin{Entry}{窗子}{12,3}{⽳,⼦}
  \begin{Phonetics}{窗子}{chuang1zi5}[][HSK 4]
    \definition[扇,个]{s.}{janela}
  \end{Phonetics}
\end{Entry}

\begin{Entry}{窗户}{12,4}{⽳,⼾}
  \begin{Phonetics}{窗户}{chuang1hu5}[][HSK 4]
    \definition[个,扇,面,排]{s.}{janela; dispositivo de ventilação e transmissão de luz nas paredes}
  \end{Phonetics}
\end{Entry}

\begin{Entry}{窗台}{12,5}{⽳,⼝}
  \begin{Phonetics}{窗台}{chuang1tai2}[][HSK 4]
    \definition{s.}{parapeito da janela; peitoril; parte plana de uma janela que segura a moldura}
  \end{Phonetics}
\end{Entry}

\begin{Entry}{窗帘}{12,8}{⽳,⼱}
  \begin{Phonetics}{窗帘}{chuang1lian2}[][HSK 5]
    \definition[个,套,片,对]{s.}{cortinas para janelas}
  \end{Phonetics}
\end{Entry}

%%%%%%%%%% 窘 %%%%%%%%%%
\subsection*{窘}\addcontentsline{loh}{figure}{窘}

\begin{Entry}{窘}{12}{⽳}
  \begin{Phonetics}{窘}{jiong3}
    \definition{adj.}{em situação financeira precária; sem dinheiro; pobre | desajeitado; constrangido; desconfortável; difícil}
    \definition{v.}{constranger; desconcertar; dificultar as coisas}
  \end{Phonetics}
\end{Entry}

\begin{Entry}{窘迫}{12,8}{⽳,⾡}
  \begin{Phonetics}{窘迫}{jiong3po4}[][HSK 7-9]
    \definition{adj.}{muito pobre; miserável; extremamente ruim | envergonhado; pressionado; em apuros; descreve uma situação que causa constrangimento}
  \end{Phonetics}
\end{Entry}

%%%%%%%%%% 窜 %%%%%%%%%%
\subsection*{窜}\addcontentsline{loh}{figure}{窜}

\begin{Entry}{窜}{12}{⽳}
  \begin{Phonetics}{窜}{cuan4}[][HSK 7-9]
    \definition{v.}{fugir; correr (usado para bandidos, tropas inimigas e animais) | Literário: exilar; expulsar | Obsoleto: mudar (a redação de um texto, manuscrito, etc.); alterar}
  \end{Phonetics}
\end{Entry}

%%%%%%%%%% 竣 %%%%%%%%%%
\subsection*{竣}\addcontentsline{loh}{figure}{竣}

\begin{Entry}{竣}{12}{⽴}
  \begin{Phonetics}{竣}{jun4}
    \definition{v.}{concluir; terminar; finalizar}
  \end{Phonetics}
\end{Entry}

\begin{Entry}{竣工}{12,3}{⽴,⼯}
  \begin{Phonetics}{竣工}{jun4gong1}[][HSK 7-9]
    \definition{v.}{ser concluído, finalizado (projetos)}
  \end{Phonetics}
\end{Entry}

%%%%%%%%%% 童 %%%%%%%%%%
\subsection*{童}\addcontentsline{loh}{figure}{童}

\begin{Entry}{童}{12}{⽴}
  \begin{Phonetics}{童}{tong2}
    \definition*{s.}{Sobrenome: Tong}
    \definition{adj.}{virgem; solteira | nu; careca | árido; estéril}
    \definition{s.}{criança | jovem servo; antigamente, referia"-se a um servo menor de idade}
  \end{Phonetics}
\end{Entry}

\begin{Entry}{童年}{12,6}{⽴,⼲}
  \begin{Phonetics}{童年}{tong2nian2}[][HSK 4]
    \definition[对]{s.}{infância; primeiros anos de vida}
  \end{Phonetics}
\end{Entry}

\begin{Entry}{童话}{12,8}{⽴,⾔}
  \begin{Phonetics}{童话}{tong2hua4}[][HSK 4]
    \definition[个,部]{s.}{conto de fadas; gênero de literatura infantil no qual as histórias adequadas para a diversão das crianças são escritas com muita imaginação, fantasia e exagero}
  \end{Phonetics}
\end{Entry}

%%%%%%%%%% 等 %%%%%%%%%%
\subsection*{等}\addcontentsline{loh}{figure}{等}

\begin{Entry}{等}{12}{⽵}
  \begin{Phonetics}{等}{deng3}[][HSK 1,2]
    \definition*{s.}{Sobrenome: Deng}
    \definition{adj.}{igual; na mesma medida ou quantidade}
    \definition{clas.}{usado para classe, grau, classificação | usado para tipo}
    \definition{part.}{e assim por diante; etc.; indica que a enumeração não está completa (pode ser usada repetidamente) | indica o fim de uma enumeração; após a enumeração, é usado para encerrar; geralmente é seguido pelo total dos itens anteriores}
    \definition{pron.}{usado após pronomes pessoais ou substantivos que se referem a pessoas; indica plural}
    \definition{s.}{classe; série; posição | equilíbrio; balança para pesar pequenas quantidades de objetos valiosos e ervas medicinais; atualmente, geralmente escrita como 戥}
    \definition{v.}{esperar; aguardar | esperar até}
  \end{Phonetics}
\end{Entry}

\begin{Entry}{等于}{12,3}{⽵,⼆}
  \begin{Phonetics}{等于}{deng3yu2}[][HSK 2]
    \definition{adv.}{igual a | equivalente a}
    \definition{v.}{equivaler a; ser equivalente a; ser quase igual a; não ter diferença}
  \end{Phonetics}
\end{Entry}

\begin{Entry}{等级}{12,6}{⽵,⽷}
  \begin{Phonetics}{等级}{deng3ji2}[][HSK 5]
    \definition[个]{s.}{grau; classificação; posição; distinções por qualidade, grau, status, etc. | estado social; estrato social; ordem e grau; grupos sociais desiguais em termos de status social e legal}
  \end{Phonetics}
\end{Entry}

\begin{Entry}{等到}{12,8}{⽵,⼑}
  \begin{Phonetics}{等到}{deng3dao4}[][HSK 2]
    \definition{prep.}{na hora; quando; expressão de condições temporais | esperar até; aguardar até}
  \end{Phonetics}
\end{Entry}

\begin{Entry}{等待}{12,9}{⽵,⼻}
  \begin{Phonetics}{等待}{deng3dai4}[][HSK 3]
    \definition{v.}{esperar; aguardar; não agir até que a pessoa, coisa ou situação desejada apareça}
  \end{Phonetics}
\end{Entry}

\begin{Entry}{等候}{12,10}{⽵,⼈}
  \begin{Phonetics}{等候}{deng3hou4}[][HSK 5]
    \definition{v.}{esperar; aguardar; expectar; usado principalmente para objetos específicos}
  \end{Phonetics}
\end{Entry}

\begin{Entry}{等等}{12,12}{⽵,⽵}
  \begin{Phonetics}{等等}{deng3 deng3}
    \definition{part.}{etc.; e assim por diante; usada depois de duas ou mais palavras paralelas para indicar que a lista não está completa}
  \end{Phonetics}
\end{Entry}

%%%%%%%%%% 筋 %%%%%%%%%%
\subsection*{筋}\addcontentsline{loh}{figure}{筋}

\begin{Entry}{筋}{12}{⽵}
  \begin{Phonetics}{筋}{jin1}[][HSK 7-9]
    \definition*{s.}{Sobrenome: Jin}
    \definition[根,条]{s.}{músculo; Coloquial: tendão; ligamento; Coloquial: veias salientes sob a pele; qualquer coisa que se assemelhe a um tendão ou veia}
  \seealsoref{筋儿}{jin1r5}
  \end{Phonetics}
\end{Entry}

\begin{Entry}{筋儿}{12,2}{⽵,⼉}
  \begin{Phonetics}{筋儿}{jin1r5}
    \definition{s.}{Coloquial: tendão; ligamento | qualquer coisa que se assemelhe a um tendão ou veia}
  \seealsoref{筋}{jin1}
  \end{Phonetics}
\end{Entry}

%%%%%%%%%% 筏 %%%%%%%%%%
\subsection*{筏}\addcontentsline{loh}{figure}{筏}

\begin{Entry}{筏}{12}{⽵}
  \begin{Phonetics}{筏}{fa2}
    \definition[条]{s.}{jangada (de troncos, bambus, etc.)}
  \end{Phonetics}
\end{Entry}

%%%%%%%%%% 筐 %%%%%%%%%%
\subsection*{筐}\addcontentsline{loh}{figure}{筐}

\begin{Entry}{筐}{12}{⽵}
  \begin{Phonetics}{筐}{kuang1}[][HSK 7-9]
    \definition[个,只]{s.}{cesto; cestaria; caixa; recipientes trançados com tiras de bambu, galhos de salgueiro e sarças}
  \end{Phonetics}
\end{Entry}

%%%%%%%%%% 筒 %%%%%%%%%%
\subsection*{筒}\addcontentsline{loh}{figure}{筒}

\begin{Entry}{筒}{12}{⽵}
  \begin{Phonetics}{筒}{tong3}[][HSK 7-9]
    \definition[个]{s.}{seção de bambu grosso; tubo grosso de bambu | objeto em forma de tubo largo | a parte em forma de tubo das roupas etc.}
    \definition{v.}{colocar dentro de (um objeto cilíndrico)}
  \end{Phonetics}
\end{Entry}

%%%%%%%%%% 答 %%%%%%%%%%
\subsection*{答}\addcontentsline{loh}{figure}{答}

\begin{Entry}{答}{12}{⽵}
  \begin{Phonetics}{答}{da1}
    \definition{v.}{concordar; responder | responder; prestar atenção}
  \end{Phonetics}
  \begin{Phonetics}{答}{da2}[][HSK 5]
    \definition{v.}{responder; dar resposta a; responder a | retribuir; devolver (uma visita, etc.); retribuir um favor feito a alguém por outro; fazer o bem}
  \end{Phonetics}
\end{Entry}

\begin{Entry}{答应}{12,7}{⽵,⼴}
  \begin{Phonetics}{答应}{da1ying5}[][HSK 2]
    \definition{v.}{responder; retribuir; reagir; retrucar | concordar; prometer; cumprir}
  \end{Phonetics}
\end{Entry}

\begin{Entry}{答复}{12,9}{⽵,⼢}
  \begin{Phonetics}{答复}{da2fu5}[][HSK 5]
    \definition[个]{s.}{resposta; respostas a perguntas ou solicitações}
    \definition{v.}{responder; dar uma resposta}
  \end{Phonetics}
\end{Entry}

\begin{Entry}{答案}{12,10}{⽵,⽊}
  \begin{Phonetics}{答案}{da2'an4}[][HSK 4]
    \definition[个,条,种,些]{s.}{chave; resposta; solução}
  \end{Phonetics}
\end{Entry}

\begin{Entry}{答辩}{12,16}{⽵,⾟}
  \begin{Phonetics}{答辩}{da2bian4}[][HSK 7-9]
    \definition{v.}{responder a perguntas, acusações, etc. de outras pessoas; defender as próprias opiniões ou ações}
  \end{Phonetics}
\end{Entry}

%%%%%%%%%% 策 %%%%%%%%%%
\subsection*{策}\addcontentsline{loh}{figure}{策}

\begin{Entry}{策}{12}{⽵}
  \begin{Phonetics}{策}{ce4}
    \definition*{s.}{Sobrenome: Ce}
    \definition[个,项,根]{s.}{plano; esquema | tiras de bambu ou madeira usadas para escrever na China antiga | questões sobre atualidades definidas para os exames imperiais | chicote de montaria antigo | um tipo de ensaio na China antiga; um estilo de escrita para exames antigos | estratégia; método}
    \definition{v.}{chicotear (um cavalo) com um chicote de montaria | incitar com um chicote de cavalo, espora}
  \end{Phonetics}
\end{Entry}

\begin{Entry}{策划}{12,6}{⽵,⼑}
  \begin{Phonetics}{策划}{ce4hua4}[][HSK 6]
    \definition{v.}{planejar; traçar; esquematizar; pensar repetidamente para elaborar um plano}
  \end{Phonetics}
\end{Entry}

\begin{Entry}{策略}{12,11}{⽵,⽥}
  \begin{Phonetics}{策略}{ce4lve4}[][HSK 6]
    \definition{adj.}{diplomático; (métodos) flexíveis sem sacrificar princípios}
    \definition[种,个,条,套]{s.}{tática; estratégia; política; para atingir determinadas tarefas estratégicas, o curso de ação e os métodos de luta são formulados de acordo com o desenvolvimento da situação}
  \end{Phonetics}
\end{Entry}

%%%%%%%%%% 筛 %%%%%%%%%%
\subsection*{筛}\addcontentsline{loh}{figure}{筛}

\begin{Entry}{筛}{12}{⽵}
  \begin{Phonetics}{筛}{shai1}[][HSK 7-9]
    \definition{s.}{peneira; coador; tela}
    \definition{v.}{peneirar; crivar; triar | aquecer o vinho sobre uma fogueira; aquecer uma panela de vinho em fogo baixo | servir vinho, chá, etc. | bater; golpear | Dialeto: bater (o gongo)}
  \end{Phonetics}
\end{Entry}

\begin{Entry}{筛选}{12,9}{⽵,⾡}
  \begin{Phonetics}{筛选}{shai1xuan3}[][HSK 7-9]
    \definition{v.}{selecionar; filtrar; eliminar por seleção; eliminar o que é ruim e selecionar o que é bom}
  \end{Phonetics}
\end{Entry}

%%%%%%%%%% 粤 %%%%%%%%%%
\subsection*{粤}\addcontentsline{loh}{figure}{粤}

\begin{Entry}{粤}{12}{⾔}
  \begin{Phonetics}{粤}{yue4}
    \definition*{s.}{Outro nome para a Província de Guangdong, 广东}
  \seealsoref{广东}{guang3dong1}
  \end{Phonetics}
\end{Entry}

\begin{Entry}{粤语}{12,9}{⾔,⾔}
  \begin{Phonetics}{粤语}{yue4yu3}
    \definition{s.}{cantonês | língua cantonesa}
  \end{Phonetics}
\end{Entry}

%%%%%%%%%% 粥 %%%%%%%%%%
\subsection*{粥}\addcontentsline{loh}{figure}{粥}

\begin{Entry}{粥}{12}{⽶}
  \begin{Phonetics}{粥}{yu4}
    \definition{v.}{dar a luz; ter filhos}
  \end{Phonetics}
  \begin{Phonetics}{粥}{zhou1}[][HSK 6]
    \definition[碗,锅,口]{s.}{mingau; mingau de aveia; alimentos semilíquidos feitos de grãos ou grãos misturados com outras coisas}
  \end{Phonetics}
\end{Entry}

%%%%%%%%%% 粪 %%%%%%%%%%
\subsection*{粪}\addcontentsline{loh}{figure}{粪}

\begin{Entry}{粪}{12}{⽶}
  \begin{Phonetics}{粪}{fen4}[][HSK 7-9]
    \definition[缸,桶]{s.}{excremento; fezes; esterco}
    \definition{v.}{Literário: aplicar esterco; fertilizar | Literário: limpar; remover; eliminar; acabar com}
  \end{Phonetics}
\end{Entry}

\begin{Entry}{粪便}{12,9}{⽶,⼈}
  \begin{Phonetics}{粪便}{fen4bian4}[][HSK 7-9]
    \definition{s.}{excremento; fezes; esterco; excrementos e urina}
  \end{Phonetics}
\end{Entry}

%%%%%%%%%% 紫 %%%%%%%%%%
\subsection*{紫}\addcontentsline{loh}{figure}{紫}

\begin{Entry}{紫}{12}{⽷}
  \begin{Phonetics}{紫}{zi3}[][HSK 5]
    \definition*{s.}{Sobrenome: Zi}
    \definition{adj.}{roxo; púrpura; violeta; cor resultante da combinação do vermelho e do azul}
  \end{Phonetics}
\end{Entry}

\begin{Entry}{紫色}{12,6}{⽷,⾊}
  \begin{Phonetics}{紫色}{zi3 se4}
    \definition{s.}{cor púrpura | cor violeta}
  \end{Phonetics}
\end{Entry}

%%%%%%%%%% 絫 %%%%%%%%%%
\subsection*{絫}\addcontentsline{loh}{figure}{絫}

\begin{Entry}{絫}{12}{⽷}
  \begin{Phonetics}{絫}{lei3}
    \variantof{累}
  \end{Phonetics}
\end{Entry}

%%%%%%%%%% 缅 %%%%%%%%%%
\subsection*{缅}\addcontentsline{loh}{figure}{缅}

\begin{Entry}{缅}{12}{⽷}
  \begin{Phonetics}{缅}{mian3}
    \definition*{s.}{Mianmar (antiga Birmânia), abreviação de 缅甸}
    \definition{adj.}{remoto; muito distante | detalhado}
    \definition{s.}{Literário: filamento fino}
    \definition{v.}{Dialeto: enrolar; virar | ponderar; refletir}
  \seealsoref{缅甸}{mian3dian4}
  \end{Phonetics}
\end{Entry}

\begin{Entry}{缅怀}{12,7}{⽷,⼼}
  \begin{Phonetics}{缅怀}{mian3huai2}[][HSK 7-9]
    \definition{v.}{recordar (atos passados); guardar com carinho a memória de}
  \end{Phonetics}
\end{Entry}

\begin{Entry}{缅甸}{12,7}{⽷,⽥}
  \begin{Phonetics}{缅甸}{mian3dian4}
    \definition*{s.}{Birmânia; Mianmar}
  \end{Phonetics}
\end{Entry}

%%%%%%%%%% 缆 %%%%%%%%%%
\subsection*{缆}\addcontentsline{loh}{figure}{缆}

\begin{Entry}{缆}{12}{⽷}
  \begin{Phonetics}{缆}{lan3}
    \definition{s.}{cabo de amarração; corda de amarração; cabo | corda grossa; cabo}
    \definition{v.}{amarrar (um barco) com uma corda ou cabo}
  \end{Phonetics}
\end{Entry}

\begin{Entry}{缆车}{12,4}{⽷,⾞}
  \begin{Phonetics}{缆车}{lan3che1}[][HSK 7-9]
    \definition{s.}{teleférico; um veículo utilizado para subir colinas, etc; a cabine é presa a um cabo, e um motor elétrico aciona o cabo, fazendo com que a cabine se mova | equipamentos para teleférico; guinchos de navio para armazenar ou recuperar cabos}
  \end{Phonetics}
\end{Entry}

%%%%%%%%%% 缓 %%%%%%%%%%
\subsection*{缓}\addcontentsline{loh}{figure}{缓}

\begin{Entry}{缓}{12}{⽶}
  \begin{Phonetics}{缓}{huan3}[][HSK 7-9]
    \definition{adj.}{lento; sem pressa | sem tensão; relaxado}
    \definition{v.}{atrasar; adiar; protelar | recuperar; reviver; voltar a si}
  \end{Phonetics}
\end{Entry}

\begin{Entry}{缓和}{12,8}{⽶,⼝}
  \begin{Phonetics}{缓和}{huan3he2}[][HSK 7-9]
    \definition{adj.}{relaxado; moderado; suave; pacífico e relaxante; não tenso ou intenso}
    \definition{v.}{relaxar; aliviar; atenuar; facilitar}
  \end{Phonetics}
\end{Entry}

\begin{Entry}{缓缓}{12,12}{⽶,⽶}
  \begin{Phonetics}{缓缓}{huan3huan3}[][HSK 7-9]
    \definition{adv.}{lentamente; vagarosamente; gradualmente}
  \end{Phonetics}
\end{Entry}

\begin{Entry}{缓解}{12,13}{⽶,⾓}
  \begin{Phonetics}{缓解}{huan3jie3}[][HSK 4]
    \definition{v.}{facilitar; aliviar; atenuar; amenizar; reduzir}
  \end{Phonetics}
\end{Entry}

\begin{Entry}{缓慢}{12,14}{⽶,⼼}
  \begin{Phonetics}{缓慢}{huan3man4}[][HSK 7-9]
    \definition{adj.}{lento; vagaroso}
    \definition{adv.}{lentamente; vagarosamente}
  \end{Phonetics}
\end{Entry}

%%%%%%%%%% 缕 %%%%%%%%%%
\subsection*{缕}\addcontentsline{loh}{figure}{缕}

\begin{Entry}{缕}{12}{⽷}
  \begin{Phonetics}{缕}{lv3}[][HSK 7-9]
    \definition{adj.}{detalhado; em detalhes | em detalhes minuciosos; com muitos detalher}
    \definition{clas.}{mecha; rufo; fio; usado para coisas finas}
    \definition{s.}{fio; bordado}
  \end{Phonetics}
\end{Entry}

%%%%%%%%%% 编 %%%%%%%%%%
\subsection*{编}\addcontentsline{loh}{figure}{编}

\begin{Entry}{编}{12}{⽷}
  \begin{Phonetics}{编}{bian1}[][HSK 4]
    \definition*{s.}{Sobrenome: Bian}
    \definition{s.}{livro; volume; parte de um livro | organização e pessoal; estabelecimento}
    \definition{v.}{tecer; trançar; entrançar | fazer uma lista; organizar em uma lista; organizar; agrupar | editar; compilar | compor; escrever | fabricar; inventar; fazer; preparar}
  \end{Phonetics}
\end{Entry}

\begin{Entry}{编写}{12,5}{⽷,⼍}
  \begin{Phonetics}{编写}{bian1xie3}[][HSK 7-9]
    \definition{v.}{compilar; organizar materiais existentes em um livro ou artigo | escrever; compor; criar | oletar informações e organizá-las ou criar algo}
  \end{Phonetics}
\end{Entry}

\begin{Entry}{编号}{12,5}{⽷,⼝}
  \begin{Phonetics}{编号}{bian1hao4}[][HSK 7-9]
    \definition{s.}{número de série; números dados em sequência}
    \definition{v.}{numerar; dar números às pessoas ou coisas em ordem}
  \end{Phonetics}
\end{Entry}

\begin{Entry}{编制}{12,8}{⽷,⼑}
  \begin{Phonetics}{编制}{bian1zhi4}[][HSK 6]
    \definition{s.}{estabelecimento; organização e pessoal; refere"-se à estrutura organizacional de uma unidade, cotas de pessoal, alocação de tarefas, etc.}
    \definition{v.}{tecer; trançar; entrelaçar tiras de vime, salgueiro, bambu, etc. para fazer objetos | resolver; realizar; elaborar; fazer de acordo com os dados (procedimentos, planos, etc.)}
  \end{Phonetics}
\end{Entry}

\begin{Entry}{编剧}{12,10}{⽷,⼑}
  \begin{Phonetics}{编剧}{bian1ju4}[][HSK 7-9]
    \definition[个,位,名]{s.}{dramaturgo | roteirista}
    \definition{v.}{escrever uma peça, um roteiro, etc.}
  \end{Phonetics}
\end{Entry}

\begin{Entry}{编造}{12,10}{⽷,⾡}
  \begin{Phonetics}{编造}{bian1zao4}[][HSK 7-9]
    \definition{v.}{compilar; compor; preparar | fabricar; inventar; criar; cozinhar | criar a partir da imaginação}
  \end{Phonetics}
\end{Entry}

\begin{Entry}{编排}{12,11}{⽷,⼿}
  \begin{Phonetics}{编排}{bian1pai2}[][HSK 7-9]
    \definition{v.}{dispor; organizar em uma determinada ordem | escrever uma peça e ensaiá-la}
  \end{Phonetics}
\end{Entry}

\begin{Entry}{编程}{12,12}{⽷,⽲}
  \begin{Phonetics}{编程}{bian1cheng2}
    \definition{v.}{programar computador}
  \end{Phonetics}
\end{Entry}

\begin{Entry}{编辑}{12,13}{⽷,⾞}
  \begin{Phonetics}{编辑}{bian1ji2}[][HSK 5]
    \definition[名,位,个]{s.}{editor; compilador; uma pessoa que organiza e processa dados ou trabalhos existentes}
    \definition{v.}{editar; compilar; organizar e processar dados ou trabalhos existentes}
  \end{Phonetics}
  \begin{Phonetics}{编辑}{bian1ji5}
    \definition{s.}{editor; compilador; pessoa que organiza e processa dados ou trabalhos existentes}
  \end{Phonetics}
\end{Entry}

%%%%%%%%%% 缘 %%%%%%%%%%
\subsection*{缘}\addcontentsline{loh}{figure}{缘}

\begin{Entry}{缘}{12}{⽷}
  \begin{Phonetics}{缘}{yuan2}
    \definition{s.}{causa | razão | karma | destino | predestinação}
  \end{Phonetics}
\end{Entry}

\begin{Entry}{缘分}{12,4}{⽷,⼑}
  \begin{Phonetics}{缘分}{yuan2fen4}
    \definition{s.}{destino ou acaso que une as pessoas | afinidade ou relacionamento predestinado | destino (Budismo)}
  \end{Phonetics}
\end{Entry}

\begin{Entry}{缘故}{12,9}{⽷,⽁}
  \begin{Phonetics}{缘故}{yuan2gu4}[][HSK 6]
    \definition{s.}{causa; razão}
  \end{Phonetics}
\end{Entry}

%%%%%%%%%% 羡 %%%%%%%%%%
\subsection*{羡}\addcontentsline{loh}{figure}{羡}

\begin{Entry}{羡}{12}{⽺}
  \begin{Phonetics}{羡}{xian4}
    \definition{v.}{admirar; invejar}
  \end{Phonetics}
\end{Entry}

\begin{Entry}{羡慕}{12,14}{⽺,⼼}
  \begin{Phonetics}{羡慕}{xian4mu4}
    \definition{v.}{invejar; admirar; ver os outros terem certos pontos fortes ou vantagens e desejar tê-los também}
  \end{Phonetics}
\end{Entry}

%%%%%%%%%% 翘 %%%%%%%%%%
\subsection*{翘}\addcontentsline{loh}{figure}{翘}

\begin{Entry}{翘}{12}{⽻}
  \begin{Phonetics}{翘}{qiao2}
    \definition{v.}{levantar (a cabeça) | empenar; tornar"-se deformado}
  \end{Phonetics}
  \begin{Phonetics}{翘}{qiao4}[][HSK 7-9]
    \definition{v.}{manter (segurar) erguido; dobrar (virar) para cima; enrolar-se}
  \end{Phonetics}
\end{Entry}

%%%%%%%%%% 联 %%%%%%%%%%
\subsection*{联}\addcontentsline{loh}{figure}{联}

\begin{Entry}{联}{12}{⽿}
  \begin{Phonetics}{联}{lian2}
    \definition{s.}{dísticos (antitéticos)}
    \definition{v.}{aliar-se a; unir-se; juntar-se a}
  \end{Phonetics}
\end{Entry}

\begin{Entry}{联手}{12,4}{⽿,⼿}
  \begin{Phonetics}{联手}{lian2shou3}[][HSK 6]
    \definition{v.}{dar as mãos; cooperar | Literário: dar as mãos | agir em conjunto}
  \end{Phonetics}
\end{Entry}

\begin{Entry}{联合}{12,6}{⽿,⼝}
  \begin{Phonetics}{联合}{lian2he2}[][HSK 3]
    \definition{adj.}{conjunto; unido; federal; combinado}
    \definition{s.}{aliado; união; aliança; conectar-se ou unir-se para agir em conjunto}
  \end{Phonetics}
\end{Entry}

\begin{Entry}{联合会}{12,6,6}{⽿,⼝,⼈}
  \begin{Phonetics}{联合会}{lian2he2hui4}
    \definition{s.}{federação}
  \end{Phonetics}
\end{Entry}

\begin{Entry}{联合国}{12,6,8}{⽿,⼝,⼞}
  \begin{Phonetics}{联合国}{lian2he2guo2}[][HSK 3]
    \definition*{s.}{Nações Unidas; Organização internacional fundada em 1945, após o fim da Segunda Guerra Mundial, com sede em Nova Iorque, Estados Unidos ; as suas principais instituições são a Assembleia Geral, o Conselho de Segurança, o Conselho Econômico e Social e o Secretariado; de acordo com a Carta das Nações Unidas, os seus principais objetivos são manter a paz e a segurança internacionais, desenvolver relações amigáveis entre os países e promover a cooperação internacional nas áreas econômica e cultural}
  \end{Phonetics}
\end{Entry}

\begin{Entry}{联欢}{12,6}{⽿,⽋}
  \begin{Phonetics}{联欢}{lian2huan1}[][HSK 7-9]
    \definition{v.}{ter um encontro; fazer uma reunião; (membros de um grupo ou de dois ou mais grupos) reunir-se para celebrar ou fortalecer a união}
  \end{Phonetics}
\end{Entry}

\begin{Entry}{联网}{12,6}{⽿,⽹}
  \begin{Phonetics}{联网}{lian2/wang3}[][HSK 7-9]
    \definition{pref.}{cyber-}
    \definition{v.+compl.}{conectar computadores (ou fios elétricos, linhas de comunicação, etc.) para formar uma rede; estar online; estar conectado}
  \end{Phonetics}
\end{Entry}

\begin{Entry}{联邦}{12,6}{⽿,⾢}
  \begin{Phonetics}{联邦}{lian2bang1}[][HSK 7-9]
    \definition{s.}{federação; união; comunidade}
  \end{Phonetics}
\end{Entry}

\begin{Entry}{联系}{12,7}{⽿,⽷}
  \begin{Phonetics}{联系}{lian2xi4}[][HSK 3]
    \definition[个,种,层]{s.}{relacionamento; relacionamento entre duas coisas}
    \definition{v.}{entrar em contato; contatar; comunicar-se com alguém por telefone, e-mail ou carta | agendar; entrar em contato com; estabelecer algum tipo de relação com a outra parte | relacionar; combinar; integrar}
  \end{Phonetics}
\end{Entry}

\begin{Entry}{联络}{12,9}{⽿,⽷}
  \begin{Phonetics}{联络}{lian2luo4}[][HSK 5]
    \definition{v.}{entrar em contato; comunicar-se; entrar em contato com}
  \end{Phonetics}
\end{Entry}

\begin{Entry}{联想}{12,13}{⽿,⼼}
  \begin{Phonetics}{联想}{lian2xiang3}[][HSK 5]
    \definition*{s.}{Lenovo (empresa)}
    \definition{v.}{associar-se a; estabelecer uma conexão mental; lembrar-se de algo; lembrar-se de outras pessoas ou coisas relacionadas devido a alguém ou algo; evocar outros conceitos relacionados devido a um determinado conceito}
  \end{Phonetics}
\end{Entry}

\begin{Entry}{联盟}{12,13}{⽿,⽫}
  \begin{Phonetics}{联盟}{lian2meng2}[][HSK 6]
    \definition{s.}{aliança; coalizão; liga; união}
  \end{Phonetics}
\end{Entry}

\begin{Entry}{联赛}{12,14}{⽿,⾙}
  \begin{Phonetics}{联赛}{lian2sai4}[][HSK 6]
    \definition{s.}{jogos da liga | liga (esportiva) | torneio da liga}
  \end{Phonetics}
\end{Entry}

%%%%%%%%%% 脾 %%%%%%%%%%
\subsection*{脾}\addcontentsline{loh}{figure}{脾}

\begin{Entry}{脾}{12}{⾁}
  \begin{Phonetics}{脾}{pi2}[][HSK 7-9]
    \definition{s.}{baço (órgão interno)}
  \end{Phonetics}
\end{Entry}

\begin{Entry}{脾气}{12,4}{⾁,⽓}
  \begin{Phonetics}{脾气}{pi2qi5}[][HSK 5]
    \definition[股]{s.}{temperamento; disposição; referindo"-se ao caráter de uma pessoa | mau humor; temperamento irascível}
  \end{Phonetics}
\end{Entry}

%%%%%%%%%% 腊 %%%%%%%%%%
\subsection*{腊}\addcontentsline{loh}{figure}{腊}

\begin{Entry}{腊}{12}{⾁}
  \begin{Phonetics}{腊}{la4}
    \definition{s.}{sacrifício antigo que ocorria a cada ano (lunar) logo após o solstício de inverno; na antiguidade, o culto conjunto de vários deuses no décimo segundo mês do calendário lunar era chamado de 腊, daí o décimo segundo mês do calendário lunar ser chamado de 腊月 | curado (carne, peixe, etc., geralmente feito no décimo segundo mês lunar); conservado por secagem ao ar ou defumação}
  \seealsoref{腊月}{la4yue4}
  \end{Phonetics}
  \begin{Phonetics}{腊}{xi1}
    \definition{s.}{carne seca}
  \end{Phonetics}
\end{Entry}

\begin{Entry}{腊月}{12,4}{⾁,⽉}
  \begin{Phonetics}{腊月}{la4yue4}[][HSK 7-9]
    \definition[个]{s.}{o décimo segundo mês lunar; dezembro do calendário lunar}
  \end{Phonetics}
\end{Entry}

%%%%%%%%%% 腔 %%%%%%%%%%
\subsection*{腔}\addcontentsline{loh}{figure}{腔}

\begin{Entry}{腔}{12}{⾁}
  \begin{Phonetics}{腔}{qiang1}[][HSK 7-9]
    \definition{clas.}{Arcaico: utilizado para carcaças de animais abatidos}
    \definition{s.}{cavidade; câmara (em corpos humanos ou animais) | afinação; tom; tom de voz | sotaque (na fala) | discurso | (geralmente no vernáculo antigo para cabra ou ovelha abatida) carcaça}
  \seealsoref{腔儿}{qiang1r5}
  \end{Phonetics}
\end{Entry}

\begin{Entry}{腔儿}{12,2}{⾁,⼉}
  \begin{Phonetics}{腔儿}{qiang1r5}
    \definition{s.}{afinação; tom | sotaque | fala}
  \end{Phonetics}
\end{Entry}

%%%%%%%%%% 舒 %%%%%%%%%%
\subsection*{舒}\addcontentsline{loh}{figure}{舒}

\begin{Entry}{舒}{12}{⾆}
  \begin{Phonetics}{舒}{shu1}
    \definition*{s.}{Sobrenome: Shu}
    \definition{adj.}{lento; vagaroso; sem pressa | confortável; relaxado e feliz}
    \definition{v.}{esticar; desdobrar | alongar; relaxar}
  \end{Phonetics}
\end{Entry}

\begin{Entry}{舒服}{12,8}{⾆,⽉}
  \begin{Phonetics}{舒服}{shu1fu5}[][HSK 2]
    \definition{adj.}{confortável; sentir-se relaxado e feliz, tanto física quanto mentalmente}
  \end{Phonetics}
\end{Entry}

\begin{Entry}{舒畅}{12,8}{⾆,⽥}
  \begin{Phonetics}{舒畅}{shu1chang4}[][HSK 7-9]
    \definition{adj.}{feliz; confortável; completamente livre de preocupações}
  \synonymref{安逸}{an1yi4}
  \synonymref{高兴}{gao1xing4}
  \synonymref{舒适}{shu1shi4}
  \synonymref{舒服}{shu1fu5}
  \synonymref{痛快}{tong4kuai5}
  \synonymref{写意}{xie4yi4}
  \antonymref{沉闷}{chen2men4}
  \antonymref{烦闷}{fan2men4}
  \antonymref{难过}{nan2guo4}
  \end{Phonetics}
\end{Entry}

\begin{Entry}{舒适}{12,9}{⾆,⾡}
  \begin{Phonetics}{舒适}{shu1shi4}[][HSK 4]
    \definition{adj.}{aconchegante; confortável; acolhedor; cômodo}
  \end{Phonetics}
\end{Entry}

%%%%%%%%%% 舜 %%%%%%%%%%
\subsection*{舜}\addcontentsline{loh}{figure}{舜}

\begin{Entry}{舜}{12}{⾇}
  \begin{Phonetics}{舜}{shun4}
    \definition*{s.}{Shun, o nome de um monarca lendário da China antiga | Sobrenome: Shun}
  \end{Phonetics}
\end{Entry}

%%%%%%%%%% 落 %%%%%%%%%%
\subsection*{落}\addcontentsline{loh}{figure}{落}

\begin{Entry}{落}{12}{⾋}
  \begin{Phonetics}{落}{la4}[][HSK 5]
    \definition{v.}{deixar de fora; estar ausente | deixar para trás; esquecer de trazer; deixar algo em algum lugar e esquecer de levar | ficar para trás (ou cair); não conseguir acompanhar}
  \end{Phonetics}
  \begin{Phonetics}{落}{lao4}
    \definition{v.}{cair; cair de uma altura elevada | se abaixar; descer; ir para baixo | permanecer; fazer uma parada; deixar para trás | obter; ter; receber}
  \end{Phonetics}
  \begin{Phonetics}{落}{luo4}[][HSK 4]
    \definition*{s.}{Sobrenome: Luo}
    \definition{s.}{paradeiro; lugar para ficar; local de descanso | assentamento; local de reunião | parte curta; área pequena; refere"-se a um pequeno lugar ou área}
    \definition{v.}{cair; cair de uma altura elevada | se abaixar; descer; ir para baixo | abaixar; deixar cair (ou descer); fazer descer | afundar; declinar; descer | ficar para trás; ficar para trás ou ficar de fora | permanecer; fazer uma parada; deixar para trás | cair sobre; repousar com | obter; ter; receber | anotar; escrever no papel | cair em; entrar em; ficar preso}
  \end{Phonetics}
\end{Entry}

\begin{Entry}{落下}{12,3}{⾋,⼀}
  \begin{Phonetics}{落下}{luo4xia4}[][HSK 7-9]
    \definition{v.}{soltar; um objeto cai de uma altura para um lugar baixo | cair; algo é movido de um lugar alto para um lugar baixo | atingir o solo (um projétil)}
  \end{Phonetics}
\end{Entry}

\begin{Entry}{落户}{12,4}{⾋,⼾}
  \begin{Phonetics}{落户}{luo4/hu4}[][HSK 7-9]
    \definition{v.+compl.}{estabelecer-se | registrar-se; solicitar uma autorização de residência}
  \end{Phonetics}
\end{Entry}

\begin{Entry}{落日}{12,4}{⾋,⽇}
  \begin{Phonetics}{落日}{luo4ri4}
    \definition{s.}{pôr do sol}
  \end{Phonetics}
\end{Entry}

\begin{Entry}{落后}{12,6}{⾋,⼝}
  \begin{Phonetics}{落后}{luo4/hou4}[][HSK 3]
    \definition{adj.}{atrasado; trabalho em atraso, nível de desenvolvimento ou grau de reconhecimento}
    \definition{v.+compl.}{ficar para trás; ficar atrasado; ficar para trás em relação aos outros durante o avanço ou o progresso do trabalho}
  \antonymref{进步}{jin4bu4}
  \end{Phonetics}
\end{Entry}

\begin{Entry}{落地}{12,6}{⾋,⼟}
  \begin{Phonetics}{落地}{luo4/di4}[][HSK 7-9]
    \definition{v.+compl.}{cair no chão; (objeto) cai de uma altura até o chão | pôr em prática; isso se refere metaforicamente ao fim de uma conversa ou ao início da implementação de um plano ou política | nascer, o nascimento de um bebê}
  \end{Phonetics}
\end{Entry}

\begin{Entry}{落汤鸡}{12,6,7}{⾋,⽔,⿃}
  \begin{Phonetics}{落汤鸡}{luo4tang1ji1}
    \definition{s.}{uma pessoa que parece encharcada e acamada | sofrimento profundo}
  \end{Phonetics}
\end{Entry}

\begin{Entry}{落花生}{12,7,5}{⾋,⾋,⽣}
  \begin{Phonetics}{落花生}{luo4 hua1 sheng1}
    \definition{s.}{amendoim | noz de macaco}
  \end{Phonetics}
\end{Entry}

\begin{Entry}{落实}{12,8}{⾋,⼧}
  \begin{Phonetics}{落实}{luo4shi2}[][HSK 5]
    \definition{adj.}{sentimento de tranquilidade; (humor) estável; seguro}
    \definition{v.}{implementar; ser praticável; tornar os planos, políticas, medidas, etc. específicos e compreensíveis, de modo a que possam ser realizados | implementar; colocar em prática; pôr em prática significa que os planos, políticas e medidas são específicos e claros, e podem ser realizados}
  \end{Phonetics}
\end{Entry}

\begin{Entry}{落差}{12,9}{⾋,⼯}
  \begin{Phonetics}{落差}{luo4cha1}[][HSK 7-9]
    \definition{s.}{queda de elevação; desnível; a diferença no nível da água causada por mudanças na elevação do leito do rio; por exemplo, se a elevação da superfície da água no local A for de 20 metros e no local B for de 18 metros, a diferença de elevação será de 2 metros | lacuna; disparidade; discrepância; comparação metafórica de lacunas ou diferenças}
  \end{Phonetics}
\end{Entry}

\begin{Entry}{落魄}{12,14}{⾋,⿁}
  \begin{Phonetics}{落魄}{luo4bo2}
    \definition{adj.}{pobre, indigente; miserável; empobrecidos e desamparados, sem meios de subsistência | de espírito livre; natural; relaxado; de mente aberta}
  \end{Phonetics}
  \begin{Phonetics}{落魄}{luo4po4}
    \definition{adj.}{desencorajado; abatido; desanimado; sem rumo | desinibido; de espírito livre; tranquilo; ousado e desinibido | em pânico; assustado; petrificado; aterrorizado}
  \end{Phonetics}
\end{Entry}

%%%%%%%%%% 葡 %%%%%%%%%%
\subsection*{葡}\addcontentsline{loh}{figure}{葡}

\begin{Entry}{葡}{12}{⾋}
  \begin{Phonetics}{葡}{pu2}
    \definition*{s.}{Portugal, abreviação de 葡萄牙}
  \seealsoref{葡萄牙}{pu2tao2ya2}
  \end{Phonetics}
\end{Entry}

\begin{Entry}{葡文}{12,4}{⾋,⽂}
  \begin{Phonetics}{葡文}{pu2wen2}
    \definition{s.}{português, língua portuguesa}
  \seealsoref{葡萄牙文}{pu2tao2ya2wen2}
  \end{Phonetics}
\end{Entry}

\begin{Entry}{葡汉词典}{12,5,7,8}{⾋,⽔,⾔,⼋}
  \begin{Phonetics}{葡汉词典}{pu2-han4 ci2dian3}
    \definition{s.}{dicionário português-chinês}
  \seealsoref{汉葡词典}{han4-pu2 ci2dian3}
  \end{Phonetics}
\end{Entry}

\begin{Entry}{葡语}{12,9}{⾋,⾔}
  \begin{Phonetics}{葡语}{pu2yu3}
    \definition{s.}{português, língua portuguesa}
  \seealsoref{葡萄牙语}{pu2tao2ya2yu3}
  \end{Phonetics}
\end{Entry}

\begin{Entry}{葡萄}{12,11}{⾋,⾋}
  \begin{Phonetics}{葡萄}{pu2tao5}[][HSK 5]
    \definition[串,颗,粒,棵,种]{s.}{parreira | uva}
  \end{Phonetics}
\end{Entry}

\begin{Entry}{葡萄牙}{12,11,4}{⾋,⾋,⽛}
  \begin{Phonetics}{葡萄牙}{pu2tao2ya2}
    \definition{s.}{Portugal}
  \end{Phonetics}
\end{Entry}

\begin{Entry}{葡萄牙文}{12,11,4,4}{⾋,⾋,⽛,⽂}
  \begin{Phonetics}{葡萄牙文}{pu2tao2ya2wen2}
    \definition{s.}{português, língua portuguesa}
  \seealsoref{葡文}{pu2wen2}
  \end{Phonetics}
\end{Entry}

\begin{Entry}{葡萄牙语}{12,11,4,9}{⾋,⾋,⽛,⾔}
  \begin{Phonetics}{葡萄牙语}{pu2tao2ya2yu3}
    \definition{s.}{português, língua portuguesa}
  \seealsoref{葡语}{pu2yu3}
  \end{Phonetics}
\end{Entry}

\begin{Entry}{葡萄酒}{12,11,10}{⾋,⾋,⾣}
  \begin{Phonetics}{葡萄酒}{pu2tao2jiu3}[][HSK 5]
    \definition[瓶,杯,口,桶]{s.}{vinho (de uva)}
  \end{Phonetics}
\end{Entry}

%%%%%%%%%% 董 %%%%%%%%%%
\subsection*{董}\addcontentsline{loh}{figure}{董}

\begin{Entry}{董}{12}{⾋}
  \begin{Phonetics}{董}{dong3}
    \definition*{s.}{Sobrenome: Dong}
    \definition{s.}{diretor; administrador}
    \definition{v.}{Literário: dirigir; supervisionar; supervisionar}
  \end{Phonetics}
\end{Entry}

\begin{Entry}{董事}{12,8}{⾋,⼅}
  \begin{Phonetics}{董事}{dong3shi4}[][HSK 7-9]
    \definition[个,位,名]{s.}{administrador; diretor}
  \end{Phonetics}
\end{Entry}

\begin{Entry}{董事长}{12,8,4}{⾋,⼅,⾧}
  \begin{Phonetics}{董事长}{dong3shi4zhang3}[][HSK 7-9]
    \definition[个,位,名]{s.}{presidente; presidente do conselho; \emph{chairman}; o principal responsável pelo conselho de administração}
  \end{Phonetics}
\end{Entry}

\begin{Entry}{董事会}{12,8,6}{⾋,⼅,⼈}
  \begin{Phonetics}{董事会}{dong3shi4hui4}[][HSK 7-9]
    \definition[届]{s.}{conselho de administração (numa empresa); conselho de curadores (numa instituição de ensino); órgão decisório de uma sociedade anônima, escola ou grupo}
  \end{Phonetics}
\end{Entry}

%%%%%%%%%% 葫 %%%%%%%%%%
\subsection*{葫}\addcontentsline{loh}{figure}{葫}

\begin{Entry}{葫}{12}{⾋}
  \begin{Phonetics}{葫}{hu2}
    \definition{s.}{cabaça}
  \end{Phonetics}
\end{Entry}

\begin{Entry}{葫芦}{12,7}{⾋,⾋}
  \begin{Phonetics}{葫芦}{hu2lu5}
    \definition{adj.}{confuso}
    \definition{s.}{cabaça | termo genérico para bloco e equipamento (ou partes dele)}
  \end{Phonetics}
\end{Entry}

%%%%%%%%%% 葬 %%%%%%%%%%
\subsection*{葬}\addcontentsline{loh}{figure}{葬}

\begin{Entry}{葬}{12}{⾋}
  \begin{Phonetics}{葬}{zang4}
    \definition{v.}{enterrar (os mortos) | sepultar}
  \end{Phonetics}
\end{Entry}

%%%%%%%%%% 葱 %%%%%%%%%%
\subsection*{葱}\addcontentsline{loh}{figure}{葱}

\begin{Entry}{葱}{12}{⾋}
  \begin{Phonetics}{葱}{cong1}[][HSK 7-9]
    \definition{adj.}{verde; turquesa}
    \definition[根,把,捆]{s.}{cebola; cebolinha}
  \end{Phonetics}
\end{Entry}

%%%%%%%%%% 葵 %%%%%%%%%%
\subsection*{葵}\addcontentsline{loh}{figure}{葵}

\begin{Entry}{葵}{12}{⾋}
  \begin{Phonetics}{葵}{kui2}
    \definition*{s.}{Sobrenome: Kui}
    \definition[朵]{s.}{certas ervas de flores grandes}
  \end{Phonetics}
\end{Entry}

\begin{Entry}{葵花}{12,7}{⾋,⾋}
  \begin{Phonetics}{葵花}{kui2hua1}
    \definition{s.}{girassol (flor)}
  \end{Phonetics}
\end{Entry}

%%%%%%%%%% 蛮 %%%%%%%%%%
\subsection*{蛮}\addcontentsline{loh}{figure}{蛮}

\begin{Entry}{蛮}{12}{⾍}
  \begin{Phonetics}{蛮}{man2}[][HSK 7-9]
    \definition{adj.}{grosseiro; rude; feroz; irracional; cruel | imprudente; implacável}
    \definition{adv.}{muito; bastante; razoavelmente}
    \definition{s.}{um nome antigo para os grupos étnicos do sul}
  \end{Phonetics}
\end{Entry}

%%%%%%%%%% 街 %%%%%%%%%%
\subsection*{街}\addcontentsline{loh}{figure}{街}

\begin{Entry}{街}{12}{⾏}
  \begin{Phonetics}{街}{jie1}[][HSK 2]
    \definition[条]{s.}{rua; avenida com prédios dos dois lados | mercado; feira rural}
  \end{Phonetics}
\end{Entry}

\begin{Entry}{街头}{12,5}{⾏,⼤}
  \begin{Phonetics}{街头}{jie1tou2}[][HSK 6]
    \definition{s.}{rua; esquina da rua}
  \end{Phonetics}
\end{Entry}

\begin{Entry}{街道}{12,12}{⾏,⾡}
  \begin{Phonetics}{街道}{jie1dao4}[][HSK 4]
    \definition[条]{s.}{caminho; rua; estrada; via pública com casas em ambos os lados, relativamente larga | escritório do subdistrito; tipo de organização responsável por gerenciar determinados aspectos da rua}
  \end{Phonetics}
\end{Entry}

\begin{Entry}{街舞}{12,14}{⾏,⾇}
  \begin{Phonetics}{街舞}{jie1wu3}
    \definition{s.}{dança de rua, \emph{street dance} (por exemplo, \emph{breakdance})}
  \end{Phonetics}
\end{Entry}

%%%%%%%%%% 裁 %%%%%%%%%%
\subsection*{裁}\addcontentsline{loh}{figure}{裁}

\begin{Entry}{裁}{12}{⾐}
  \begin{Phonetics}{裁}{cai2}[][HSK 7-9]
    \definition{clas.}{divisão de papel de impressão de tamanho padrão}
    \definition{s.}{planejamento | tipo de escrita | planejamento mental; arranjo e seleção, usados principalmente na literatura e na arte | sanção; restrição | estilo; forma | (impressão) tamanho do corte de papel}
    \definition{v.}{cortar (papel, tecido, etc.) em partes | reduzir; cortar; dispensar | julgar; decidir | verificar; sancionar | cortar; eliminar; remover coisas desnecessárias ou redundantes | discernir; medir; julgar}
  \end{Phonetics}
\end{Entry}

\begin{Entry}{裁决}{12,6}{⾐,⼎}
  \begin{Phonetics}{裁决}{cai2jue2}[][HSK 7-9]
    \definition[项]{s.}{sentença arbitral; decisão; adjudicação}
    \definition{v.}{fazer uma decisão; julgar; decidir; adjudicar veredicto}
  \end{Phonetics}
\end{Entry}

\begin{Entry}{裁判}{12,7}{⾐,⼑}
  \begin{Phonetics}{裁判}{cai2pan4}[][HSK 5]
    \definition[个,位,名]{s.}{árbitro; juiz; pessoa que desempenha funções de arbitragem em esportes e outras competições}
    \definition{v.}{arbitrar; atuar como árbitro; em esportes e outras atividades competitivas, julgar o desempenho dos atletas, vitórias e derrotas, classificações e problemas que ocorrem durante a competição de acordo com as regras da competição | julgar; refere"-se a um terceiro que faz um julgamento quando surge uma disputa entre duas partes}
  \end{Phonetics}
\end{Entry}

\begin{Entry}{裁定}{12,8}{⾐,⼧}
  \begin{Phonetics}{裁定}{cai2ding4}[][HSK 7-9]
    \definition{v.}{decidir (ou declarar) judicialmente; governar}
  \end{Phonetics}
\end{Entry}

%%%%%%%%%% 裂 %%%%%%%%%%
\subsection*{裂}\addcontentsline{loh}{figure}{裂}

\begin{Entry}{裂}{12}{⾐}
  \begin{Phonetics}{裂}{lie4}[][HSK 6]
    \definition{s.}{entalhe; incisão; entalhes grandes e profundos nas bordas das folhas ou corolas | brecha; lacuna; rachadura; refere"-se à rachadura ou divisão que aparece na superfície ou no interior de um objeto}
    \definition{v.}{dividir; rachar; rasgar | Figurativo: quebrar; esmagar; arruinar}
  \end{Phonetics}
\end{Entry}

\begin{Entry}{裂痕}{12,11}{⾐,⽧}
  \begin{Phonetics}{裂痕}{lie4hen2}[][HSK 7-9]
    \definition{s.}{fenda; rachadura; fissura; vestígios de quebra no objeto}
  \end{Phonetics}
\end{Entry}

\begin{Entry}{裂缝}{12,13}{⾐,⽷}
  \begin{Phonetics}{裂缝}{lie4feng4}[][HSK 7-9]
    \definition[道]{s.}{fenda; rachadura; fissura}
    \definition{v.}{rachar; fender; fissurar; dividir em uma fenda estreita}
  \end{Phonetics}
\end{Entry}

%%%%%%%%%% 装 %%%%%%%%%%
\subsection*{装}\addcontentsline{loh}{figure}{装}

\begin{Entry}{装}{12}{⾐}
  \begin{Phonetics}{装}{zhuang1}[][HSK 2]
    \definition*{s.}{Sobrenome: Zhuang}
    \definition{s.}{vestido; traje; vestimenta; roupa | maquiagem e figurino de palco; maquiagem de ator}
    \definition{v.}{enfeitar; adornar; vestir; decorar; vestir-se; vestir-se bem | fingir; fazer de conta | segurar; embalar; carregar; colocar as coisas em recipientes; colocar as coisas no transporte | encaixar; instalar; equipar; aparelhar; montar | embalar; encaixotar; embrulhar produtos ou colocá-los em caixas, garrafas, etc.}
  \end{Phonetics}
\end{Entry}

\begin{Entry}{装扮}{12,7}{⾐,⼿}
  \begin{Phonetics}{装扮}{zhuang1ban4}
    \definition{v.}{enfeitar | decorar | disfarçar-me | vestir-se}
  \end{Phonetics}
\end{Entry}

\begin{Entry}{装备}{12,8}{⾐,⼡}
  \begin{Phonetics}{装备}{zhuang1bei4}[][HSK 6]
    \definition[套]{s.}{equipamento; equipagem; traje}
    \definition{v.}{equipar}
  \end{Phonetics}
\end{Entry}

\begin{Entry}{装饰}{12,8}{⾐,⾷}
  \begin{Phonetics}{装饰}{zhuang1shi4}[][HSK 5]
    \definition[件,个]{s.}{decoração; acessórios decorativos}
    \definition{v.}{enfeitar; adornar; decorar; ornamentar; embelezar; destacar}
  \end{Phonetics}
\end{Entry}

\begin{Entry}{装修}{12,9}{⾐,⼈}
  \begin{Phonetics}{装修}{zhuang1xiu1}[][HSK 4]
    \definition{v.}{equipar; renovar; decorar (equipar uma sala ou prédio com equipamentos ou decorações); reboco, pintura e instalação de portas, janelas, encanamentos e outros equipamentos em projetos habitacionais}
  \end{Phonetics}
\end{Entry}

\begin{Entry}{装配}{12,10}{⾐,⾣}
  \begin{Phonetics}{装配}{zhuang1pei4}
    \definition{v.}{montar | encaixar}
  \end{Phonetics}
\end{Entry}

\begin{Entry}{装置}{12,13}{⾐,⽹}
  \begin{Phonetics}{装置}{zhuang1zhi4}[][HSK 4]
    \definition[个,台,种,些]{s.}{dispositivo; equipamento; máquinas, instrumentos ou outros equipamentos de construção mais complexa e com alguma função independente}
    \definition{v.}{instalar; ajustar; configurar; equipar; montar}
  \end{Phonetics}
\end{Entry}

%%%%%%%%%% 裙 %%%%%%%%%%
\subsection*{裙}\addcontentsline{loh}{figure}{裙}

\begin{Entry}{裙}{12}{⾐}
  \begin{Phonetics}{裙}{qun2}
    \definition[条]{s.}{saia | avental | algo como uma saia}
  \end{Phonetics}
\end{Entry}

\begin{Entry}{裙子}{12,3}{⾐,⼦}
  \begin{Phonetics}{裙子}{qun2zi5}[][HSK 3]
    \definition[条,件]{s.}{saia (peça de vestuário); uma vestimenta usada abaixo da cintura}
  \end{Phonetics}
\end{Entry}

%%%%%%%%%% 裤 %%%%%%%%%%
\subsection*{裤}\addcontentsline{loh}{figure}{裤}

\begin{Entry}{裤}{12}{⾐}
  \begin{Phonetics}{裤}{ku4}
    \definition[条]{s.}{calças}
  \end{Phonetics}
\end{Entry}

\begin{Entry}{裤子}{12,3}{⾐,⼦}
  \begin{Phonetics}{裤子}{ku4zi5}[][HSK 3]
    \definition[条]{s.}{calças; calções; roupas usadas abaixo da cintura, com cós, virilha e duas pernas}
  \end{Phonetics}
\end{Entry}

%%%%%%%%%% 詈 %%%%%%%%%%
\subsection*{詈}\addcontentsline{loh}{figure}{詈}

\begin{Entry}{詈}{12}{⾔}
  \begin{Phonetics}{詈}{li4}
    \definition{v.}{xingar; usar linguagem severa}
  \end{Phonetics}
\end{Entry}

\begin{Entry}{詈骂}{12,9}{⾔,⾺}
  \begin{Phonetics}{詈骂}{li4ma4}
    \definition{v.}{xingar | abusar}
  \end{Phonetics}
\end{Entry}

%%%%%%%%%% 谢 %%%%%%%%%%
\subsection*{谢}\addcontentsline{loh}{figure}{谢}

\begin{Entry}{谢}{12}{⾔}
  \begin{Phonetics}{谢}{xie4}
    \definition*{s.}{Sobrenome: Xie}
    \definition{v.}{agradecer | desculpar-se; pedir desculpas; admitir a própria culpa | recusar; declinar; renunciar | murchar; perder de flores ou folhas}
  \end{Phonetics}
\end{Entry}

\begin{Entry}{谢天谢地}{12,4,12,6}{⾔,⼤,⾔,⼟}
  \begin{Phonetics}{谢天谢地}{xie4tian1xie4di4}
    \definition{expr.}{agradecer a Deus | agradecer aos céus}
  \end{Phonetics}
\end{Entry}

\begin{Entry}{谢世}{12,5}{⾔,⼀}
  \begin{Phonetics}{谢世}{xie4shi4}
    \definition{v.}{morrer | falecer}
  \end{Phonetics}
\end{Entry}

\begin{Entry}{谢恩}{12,10}{⾔,⼼}
  \begin{Phonetics}{谢恩}{xie4'en1}
    \definition{v.}{agradecer a alguém pelo favor (especialmente imperador ou oficial superior)}
  \end{Phonetics}
\end{Entry}

\begin{Entry}{谢病}{12,10}{⾔,⽧}
  \begin{Phonetics}{谢病}{xie4bing4}
    \definition{v.}{desculpar-se por causa de doença}
  \end{Phonetics}
\end{Entry}

\begin{Entry}{谢媒}{12,12}{⾔,⼥}
  \begin{Phonetics}{谢媒}{xie4mei2}
    \definition{v.}{agradecer ao casamenteiro}
  \end{Phonetics}
\end{Entry}

\begin{Entry}{谢谢}{12,12}{⾔,⾔}
  \begin{Phonetics}{谢谢}{xie4xie5}[][HSK 1]
    \definition{interj.}{``Obrigado!''}
    \definition{v.}{agradecer; agradecer a gentileza dos outros}
  \end{Phonetics}
\end{Entry}

\begin{Entry}{谢意}{12,13}{⾔,⼼}
  \begin{Phonetics}{谢意}{xie4yi4}
    \definition{s.}{gratidão}
  \end{Phonetics}
\end{Entry}

%%%%%%%%%% 谦 %%%%%%%%%%
\subsection*{谦}\addcontentsline{loh}{figure}{谦}

\begin{Entry}{谦}{12}{⾔}
  \begin{Phonetics}{谦}{qian1}
    \definition*{s.}{Sobrenome: Qian}
    \definition{adj.}{modesto}
    \definition{s.}{modéstia}
  \end{Phonetics}
\end{Entry}

\begin{Entry}{谦逊}{12,9}{⾔,⾡}
  \begin{Phonetics}{谦逊}{qian1xun4}[][HSK 7-9]
    \definition{adj.}{humilde; modesto; despretensioso; sem afetação}
  \end{Phonetics}
\end{Entry}

\begin{Entry}{谦虚}{12,11}{⾔,⾌}
  \begin{Phonetics}{谦虚}{qian1xu1}[][HSK 6]
    \definition{adj.}{modesto; não se orgulhe de suas próprias conquistas e esteja disposto a aceitar críticas e opiniões de outras pessoas}
    \definition{v.}{falar modestamente; quando recebo elogios e cumprimentos de outras pessoas, sinto que não sou tão bom}
  \end{Phonetics}
\end{Entry}

%%%%%%%%%% 貂 %%%%%%%%%%
\subsection*{貂}\addcontentsline{loh}{figure}{貂}

\begin{Entry}{貂}{12}{⾘}
  \begin{Phonetics}{貂}{diao1}
    \definition*{s.}{Sobrenome: Diao}
    \definition[只]{s.}{marta; fuinha; arminho}
  \end{Phonetics}
\end{Entry}

%%%%%%%%%% 赋 %%%%%%%%%%
\subsection*{赋}\addcontentsline{loh}{figure}{赋}

\begin{Entry}{赋}{12}{⾙}
  \begin{Phonetics}{赋}{fu4}
    \definition{s.}{dotação (natural) | Obsoleto: imposto territorial | fu, estilo antigo, uma forma literária complexa que combina elementos de poesia e prosa, muito cultivada desde a época Han até o período das Seis Dinastias}
    \definition{v.}{compor (versos); escrever poemas e letras | Literário: conceder; entregar; dotar com}
  \end{Phonetics}
\end{Entry}

\begin{Entry}{赋予}{12,4}{⾙,⼅}
  \begin{Phonetics}{赋予}{fu4yu3}[][HSK 7-9]
    \definition{v.}{conceder; dotar (uma tarefa importante, missão, etc.); dar a alguém uma tarefa, responsabilidade, direito, autoridade, etc. | investir com (significado, característica, etc.); dar a algo uma cor, significado, importância, etc.}
  \end{Phonetics}
\end{Entry}

%%%%%%%%%% 赌 %%%%%%%%%%
\subsection*{赌}\addcontentsline{loh}{figure}{赌}

\begin{Entry}{赌}{12}{⾙}
  \begin{Phonetics}{赌}{du3}[][HSK 6]
    \definition{v.}{jogar | apostar; geralmente se refere à luta pela vitória ou derrota}
  \end{Phonetics}
\end{Entry}

\begin{Entry}{赌博}{12,12}{⾙,⼗}
  \begin{Phonetics}{赌博}{du3bo2}[][HSK 6]
    \definition{v.}{apostar; jogar; usar jogos de cartas, rolagem de dados, etc., para apostar dinheiro}
  \end{Phonetics}
\end{Entry}

%%%%%%%%%% 赎 %%%%%%%%%%
\subsection*{赎}\addcontentsline{loh}{figure}{赎}

\begin{Entry}{赎}{12}{⾙}
  \begin{Phonetics}{赎}{shu2}[][HSK 7-9]
    \definition*{s.}{Sobrenome: Shu}
    \definition{v.}{resgatar; pedir resgate; usar dinheiro para trocar por garantias | expiar (um crime); desviar; inventar}
  \antonymref{当}{dang4}
  \end{Phonetics}
\end{Entry}

%%%%%%%%%% 赏 %%%%%%%%%%
\subsection*{赏}\addcontentsline{loh}{figure}{赏}

\begin{Entry}{赏}{12}{⾙}
  \begin{Phonetics}{赏}{shang3}[][HSK 4]
    \definition*{s.}{Sobrenome: Shang}
    \definition{s.}{recompensa; prêmio}
    \definition{v.}{conceder (outorgar) uma recompensa; recompensar; premiar | admirar; desfrutar; apreciar; valorizar}
  \end{Phonetics}
\end{Entry}

\begin{Entry}{赏心悦目}{12,4,10,5}{⾙,⼼,⼼,⽬}
  \begin{Phonetics}{赏心悦目}{shang3xin1yue4mu4}
    \definition{expr.}{``Aquece o coração e encanta os olhos.''; achar a paisagem agradável tanto aos olhos quanto à mente}
  \end{Phonetics}
\end{Entry}

\begin{Entry}{赏赐}{12,12}{⾙,⾙}
  \begin{Phonetics}{赏赐}{shang3ci4}
    \definition{s.}{recompensa | prêmio}
    \definition{v.}{recompensar | premiar}
  \end{Phonetics}
\end{Entry}

%%%%%%%%%% 赐 %%%%%%%%%%
\subsection*{赐}\addcontentsline{loh}{figure}{赐}

\begin{Entry}{赐}{12}{⾙}
  \begin{Phonetics}{赐}{ci4}[][HSK 7-9]
    \definition{s.}{favor; bênção; presente; uma recompensa, um benefício concedido}
    \definition{v.}{conceder; conferir; favorecer; presentear; dar, antigamente, se referia ao superior dando ao subordinado ou ao mais velho dando ao mais novo | responder; dar conselho; instruir; uma palavra usada para demonstrar respeito; quando alguém lhe dá instruções, responde ou lhe entrega algo}
  \end{Phonetics}
\end{Entry}

\begin{Entry}{赐教}{12,11}{⾙,⽁}
  \begin{Phonetics}{赐教}{ci4jiao4}[][HSK 7-9]
    \definition{v.}{condescender em ensinar; conceder instrução | importa"-se em me esclarecer com suas instruções}
  \end{Phonetics}
\end{Entry}

%%%%%%%%%% 赔 %%%%%%%%%%
\subsection*{赔}\addcontentsline{loh}{figure}{赔}

\begin{Entry}{赔}{12}{⾙}
  \begin{Phonetics}{赔}{pei2}[][HSK 5]
    \definition{v.}{compensar; pagar por; indenizar | sofrer uma perda; fazer negócios e perder dinheiro | desculpar-se | suportar uma perda}
  \end{Phonetics}
\end{Entry}

\begin{Entry}{赔钱}{12,10}{⾙,⾦}
  \begin{Phonetics}{赔钱}{pei2/qian2}[][HSK 7-9]
    \definition{v.+compl.}{perder dinheiro | compensar; compensar com dinheiro os prejuízos causados a terceiros}
  \end{Phonetics}
\end{Entry}

\begin{Entry}{赔偿}{12,11}{⾙,⼈}
  \begin{Phonetics}{赔偿}{pei2chang2}[][HSK 5]
    \definition{v.}{indenizar; compensar; pagar por; indenizar outras pessoas ou grupos por perdas causadas por suas próprias ações}
  \end{Phonetics}
\end{Entry}

%%%%%%%%%% 趁 %%%%%%%%%%
\subsection*{趁}\addcontentsline{loh}{figure}{趁}

\begin{Entry}{趁}{12}{⾛}
  \begin{Phonetics}{趁}{chen4}[][HSK 7-9]
    \definition{prep.}{aproveitar"-se de; tirar vantagem de (tempo, oportunidade, etc.); indica o tempo e as condições de uso}
    \definition{v.}{ser rico em; possuir}
  \end{Phonetics}
\end{Entry}

\begin{Entry}{趁早}{12,6}{⾛,⽇}
  \begin{Phonetics}{趁早}{chen4zao3}[][HSK 7-9]
    \definition{adv.}{o mais cedo possível; antes que seja tarde demais; na primeira oportunidade}
  \end{Phonetics}
\end{Entry}

\begin{Entry}{趁机}{12,6}{⾛,⽊}
  \begin{Phonetics}{趁机}{chen4ji1}[][HSK 7-9]
    \definition{adv./adv.}{aproveitar a ocasião; aproveitar a oportunidade}
  \end{Phonetics}
\end{Entry}

\begin{Entry}{趁着}{12,11}{⾛,⽬}
  \begin{Phonetics}{趁着}{chen4zhe5}[][HSK 7-9]
    \definition{v.}{aproveitar"-se de; tirar vantagem de}
  \end{Phonetics}
\end{Entry}

%%%%%%%%%% 超 %%%%%%%%%%
\subsection*{超}\addcontentsline{loh}{figure}{超}

\begin{Entry}{超}{12}{⾛}
  \begin{Phonetics}{超}{chao1}[][HSK 6]
    \definition{adj.}{super; extremamente; maior (ou menor) que o padrão geral}
    \definition{v.}{exceder; ultrapassar; vir para a frente por trás; prevalecer | transcender; ir além; não ser sujeito a certas restrições; ir além de um certo intervalo | exceder; superar; exceder o limite prescrito}
  \end{Phonetics}
\end{Entry}

\begin{Entry}{超车}{12,4}{⾛,⾞}
  \begin{Phonetics}{超车}{chao1/che1}[][HSK 7-9]
    \definition{v.+compl.}{ultrapassar (um veículo); passar por um veículo que trafega na mesma direção}
  \end{Phonetics}
\end{Entry}

\begin{Entry}{超出}{12,5}{⾛,⼐}
  \begin{Phonetics}{超出}{chao1 chu1}[][HSK 6]
    \definition{v.}{exceder; ultrapassar; ir além (de uma certa quantidade ou intervalo)}
  \end{Phonetics}
\end{Entry}

\begin{Entry}{超市}{12,5}{⾛,⼱}
  \begin{Phonetics}{超市}{chao1shi4}[][HSK 2]
    \definition[家]{s.}{supermercado; abreviação de 超级市场}
  \seealsoref{超级市场}{chao1 ji2 shi4 chang3}
  \end{Phonetics}
\end{Entry}

\begin{Entry}{超级}{12,6}{⾛,⽷}
  \begin{Phonetics}{超级}{chao1ji2}[][HSK 3]
    \definition{adj.}{super; além do nível geral}
    \definition{pref.}{super-; ultra-; hiper-}
  \end{Phonetics}
\end{Entry}

\begin{Entry}{超级市场}{12,6,5,6}{⾛,⽷,⼱,⼟}
  \begin{Phonetics}{超级市场}{chao1 ji2 shi4 chang3}
    \definition[个,间,所,家]{s.}{supermercado; hipermercado}
  \end{Phonetics}
\end{Entry}

\begin{Entry}{超过}{12,6}{⾛,⾡}
  \begin{Phonetics}{超过}{chao1/guo4}[][HSK 2]
    \definition{v.+compl.}{ultrapassar; superar (algo ou alguém); passar de trás para a frente de alguém ou algo | exceder; ser mais do que; ultrapassar (um padrão)}
  \end{Phonetics}
\end{Entry}

\begin{Entry}{超声}{12,7}{⾛,⼠}
  \begin{Phonetics}{超声}{chao1sheng1}
    \definition{adj.}{ultrasônico}
    \definition{s.}{ultrasom}
  \end{Phonetics}
\end{Entry}

\begin{Entry}{超前}{12,9}{⾛,⼑}
  \begin{Phonetics}{超前}{chao1qian2}[][HSK 7-9]
    \definition{v.}{transcender os tempos; estar à frente dos tempos; estar à frente do seu tempo | superar os antecessores; liderar; assumir a liderança}
  \end{Phonetics}
\end{Entry}

\begin{Entry}{超标}{12,9}{⾛,⽊}
  \begin{Phonetics}{超标}{chao1/biao1}[][HSK 7-9]
    \definition{v.+compl.}{exceder uma cota (ou padrão); exceder o limite prescrito}
  \end{Phonetics}
\end{Entry}

\begin{Entry}{超速}{12,10}{⾛,⾡}
  \begin{Phonetics}{超速}{chao1su4}[][HSK 7-9]
    \definition{s.}{Física: hipervelocidade | excesso de velocidade}
    \definition{v.}{exceder o limite de velocidade}
  \end{Phonetics}
\end{Entry}

\begin{Entry}{超越}{12,12}{⾛,⾛}
  \begin{Phonetics}{超越}{chao1yue4}[][HSK 5]
    \definition{v.}{ultrapassar; superar; passar por cima; transcender}
  \end{Phonetics}
\end{Entry}

%%%%%%%%%% 越 %%%%%%%%%%
\subsection*{越}\addcontentsline{loh}{figure}{越}

\begin{Entry}{越}{12}{⾛}
  \begin{Phonetics}{越}{yue4}[][HSK 2]
    \definition{adj.}{superior; excede ou ultrapassa o ordinário}
    \definition{adv.}{quanto mais\dots mais; usados juntos, eles formam o formato de ``越…越…'' para indicar que o grau de uma situação se torna mais sério à medida que se desenvolve; ``成年…'' para indicar que o grau de uma situação se torna mais sério à medida que o tempo passa}
    \definition{v.}{passar por cima; pular; cruzar | exceder; ultrapassar | estar em um tom alto; estar animado | saquear; pilhar; expoliar; apreender; roubar | passar; passar através; atravessar}
  \seealsoref{越……越……}{yue4 yue4}
  \seealsoref{越来越……}{yue4lai2yue4}
  \end{Phonetics}
\end{Entry}

\begin{Entry}{越来越……}{12,7,12}{⾛,⽊,⾛}
  \begin{Phonetics}{越来越……}{yue4lai2yue4}[][HSK 2]
    \definition{adv.}{cada vez mais\dots; isso significa que o grau de algo se aprofunda à medida que o tempo passa}
  \end{Phonetics}
\end{Entry}

\begin{Entry}{越……越……}{12,12}{⾛,⾛}
  \begin{Phonetics}{越……越……}{yue4 yue4}
    \definition{expr.}{quanto mais\dots tanto mais\dots}
  \end{Phonetics}
\end{Entry}

\begin{Entry}{越障}{12,13}{⾛,⾩}
  \begin{Phonetics}{越障}{yue4zhang4}
    \definition{s.}{curso com obstáculos para treinamento de tropas}
    \definition{v.}{superar obstáculos}
  \end{Phonetics}
\end{Entry}

\begin{Entry}{越境}{12,14}{⾛,⼟}
  \begin{Phonetics}{越境}{yue4jing4}
    \definition{v.}{cruzar a fronteira ilegalmente; entrar ou sair clandestinamente de um país}
  \end{Phonetics}
\end{Entry}

%%%%%%%%%% 趋 %%%%%%%%%%
\subsection*{趋}\addcontentsline{loh}{figure}{趋}

\begin{Entry}{趋}{12}{⾛}
  \begin{Phonetics}{趋}{qu1}
    \definition{v.}{apressar-se | tender para; tender a se tornar | (ganso, cobra, etc.) estalar a cabeça e morder as pessoas}
  \end{Phonetics}
\end{Entry}

\begin{Entry}{趋于}{12,3}{⾛,⼆}
  \begin{Phonetics}{趋于}{qu1yu2}[][HSK 7-9]
    \definition{v.}{tender a}
  \end{Phonetics}
\end{Entry}

\begin{Entry}{趋势}{12,8}{⾛,⼒}
  \begin{Phonetics}{趋势}{qu1shi4}[][HSK 4]
    \definition{s.}{rumo; tendência; direção; impulso das coisas que se movem em uma direção ou outra}
  \end{Phonetics}
\end{Entry}

%%%%%%%%%% 跌 %%%%%%%%%%
\subsection*{跌}\addcontentsline{loh}{figure}{跌}

\begin{Entry}{跌}{12}{⾜}
  \begin{Phonetics}{跌}{die1}[][HSK 6]
    \definition{s.}{(de um objeto, etc.) queda; tombo | (de preços, etc.) queda}
    \definition{v.}{cair; tombar; perder o equilíbrio e cair | cair (objetos caindo); descer | cair (queda de preços)}
  \end{Phonetics}
\end{Entry}

%%%%%%%%%% 跑 %%%%%%%%%%
\subsection*{跑}\addcontentsline{loh}{figure}{跑}

\begin{Entry}{跑}{12}{⾜}
  \begin{Phonetics}{跑}{pao2}
    \definition{v.}{(animais) bater com a pata (no chão); (animais) escavar o solo com suas garras ou cascos}
  \end{Phonetics}
  \begin{Phonetics}{跑}{pao3}[][HSK 1]
    \definition{v.}{correr; pessoas ou animais que se movem rapidamente para a frente com as pernas e os pés | caminhar; passear | fugir; escapar | correr de um lado para outro; fazer rondas; correr atrás de algo | de um líquido ou gás) vazar; evaporar | (como complemento de um verbo) fora; longe | participar de uma corrida}
  \end{Phonetics}
\end{Entry}

\begin{Entry}{跑马}{12,3}{⾜,⾺}
  \begin{Phonetics}{跑马}{pao3ma3}
    \definition{s.}{corrida de cavalos}
    \definition{v.}{andar a cavalo em ritmo acelerado}
  \end{Phonetics}
\end{Entry}

\begin{Entry}{跑车}{12,4}{⾜,⾞}
  \begin{Phonetics}{跑车}{pao3che1}[][HSK 7-9]
    \definition{s.}{bicicleta de corrida | \emph{roadster}; carro de corrida | carrinho para transportar toras em uma floresta}
    \definition{v.}{Coloquial: (condutores de trem) estar em serviço | (vagões de carvão em uma mina) deslizar acidentalmente para baixo (desgovernado) | Dialeto: dirigir um veículo de transporte | trabalhar em um trem; atendente de trem trabalhando no trem | escorregar acidentalmente para baixo; isso se refere a um acidente em um poço inclinado de mina, onde o cabo de aço se rompe repentinamente durante o içamento ou o guincho escorrega por outros motivos}
  \end{Phonetics}
\end{Entry}

\begin{Entry}{跑龙套}{12,5,10}{⾜,⿓,⼤}
  \begin{Phonetics}{跑龙套}{pao3 long2tao4}[][HSK 7-9]
    \definition{v.}{Teatro: interpretar um papel secundário | desempenhar um papel secundário; não ser ninguém | desempenhar um papel pequeno}
  \end{Phonetics}
\end{Entry}

\begin{Entry}{跑步}{12,7}{⾜,⽌}
  \begin{Phonetics}{跑步}{pao3/bu4}[][HSK 3]
    \definition{v.+compl.}{correr; trotar}
  \end{Phonetics}
\end{Entry}

\begin{Entry}{跑肚}{12,7}{⾜,⾁}
  \begin{Phonetics}{跑肚}{pao3du4}
    \definition{v.}{Coloquial: ter diarréia}
  \end{Phonetics}
\end{Entry}

\begin{Entry}{跑调}{12,10}{⾜,⾔}
  \begin{Phonetics}{跑调}{pao3diao4}
    \definition{v.}{Coloquial: estar fora do tom ou desafinado (enquanto canta)}
  \end{Phonetics}
\end{Entry}

\begin{Entry}{跑掉}{12,11}{⾜,⼿}
  \begin{Phonetics}{跑掉}{pao3diao4}
    \definition{v.}{fugir}
  \end{Phonetics}
\end{Entry}

\begin{Entry}{跑道}{12,12}{⾜,⾡}
  \begin{Phonetics}{跑道}{pao3dao4}[][HSK 7-9]
    \definition[条]{s.}{pista de decolagem; a pista de taxiagem utilizada pelas aeronaves durante a decolagem e o pouso | pista; pista de atletismo; as linhas brancas desenhadas na pista são usadas para corridas de corrida ou ciclismo}
  \end{Phonetics}
\end{Entry}

\begin{Entry}{跑腿}{12,13}{⾜,⾁}
  \begin{Phonetics}{跑腿}{pao3tui3}
    \definition{v.}{realizar tarefas}
  \end{Phonetics}
\end{Entry}

\begin{Entry}{跑酷}{12,14}{⾜,⾣}
  \begin{Phonetics}{跑酷}{pao3ku4}
    \definition*{s.}{Empréstimo linguístico: Parkour}
  \end{Phonetics}
\end{Entry}

\begin{Entry}{跑题}{12,15}{⾜,⾴}
  \begin{Phonetics}{跑题}{pao3ti2}
    \definition{v.}{divagar | fugir do assunto | tergiversar}
  \end{Phonetics}
\end{Entry}

%%%%%%%%%% 辈 %%%%%%%%%%
\subsection*{辈}\addcontentsline{loh}{figure}{辈}

\begin{Entry}{辈}{12}{⾞}
  \begin{Phonetics}{辈}{bei4}[][HSK 5]
    \definition*{s.}{Sobrenome: Bei}
    \definition{s.}{pessoas de um certo tipo; semelhantes | geração; geração na família | duração da vida | círculo familiar}
  \end{Phonetics}
\end{Entry}

%%%%%%%%%% 辉 %%%%%%%%%%
\subsection*{辉}\addcontentsline{loh}{figure}{辉}

\begin{Entry}{辉}{12}{⾞}
  \begin{Phonetics}{辉}{hui1}
    \definition{s.}{brilho; esplendor; fulgor}
    \definition{v.}{brilhar}
  \end{Phonetics}
\end{Entry}

\begin{Entry}{辉煌}{12,13}{⾞,⽕}
  \begin{Phonetics}{辉煌}{hui1huang2}[][HSK 7-9]
    \definition{adj.}{brilhante; esplêndido; deslumbrante  | brilhante; glorioso; descreve conquistas notáveis}
  \end{Phonetics}
\end{Entry}

%%%%%%%%%% 辜 %%%%%%%%%%
\subsection*{辜}\addcontentsline{loh}{figure}{辜}

\begin{Entry}{辜}{12}{⾟}
  \begin{Phonetics}{辜}{gu1}
    \definition*{s.}{Sobrenome: Gu}
    \definition{s.}{culpa; crime}
  \end{Phonetics}
\end{Entry}

\begin{Entry}{辜负}{12,6}{⾟,⾙}
  \begin{Phonetics}{辜负}{gu1fu4}[][HSK 7-9]
    \definition{v.}{desapontar; decepcionar; ser indigno de; não corresponder a}
  \end{Phonetics}
\end{Entry}

%%%%%%%%%% 逼 %%%%%%%%%%
\subsection*{逼}\addcontentsline{loh}{figure}{逼}

\begin{Entry}{逼}{12}{⾡}
  \begin{Phonetics}{逼}{bi1}[][HSK 6]
    \definition{adj.}{estreito}
    \definition{v.}{forçar; pressionar; compelir | extorquir; pressionar por | fechar em; pressionar em direção a; aproximar"-se}
  \end{Phonetics}
\end{Entry}

\begin{Entry}{逼近}{12,7}{⾡,⾡}
  \begin{Phonetics}{逼近}{bi1jin4}[][HSK 7-9]
    \definition{adj.}{aproximado (valor númerico)}
    \definition{s.}{aproximação (função matemática mais simples)}
    \definition{v.}{avançar em direção a; aproximar"-se de; aproximar"-se | ganhar em (sobre); aglomerar"-se em}
  \end{Phonetics}
\end{Entry}

\begin{Entry}{逼迫}{12,8}{⾡,⾡}
  \begin{Phonetics}{逼迫}{bi1po4}[][HSK 7-9]
    \definition{v.}{forçar; compelir; coagir | restringir; exercer pressão para induzir; forçar}
  \end{Phonetics}
\end{Entry}

\begin{Entry}{逼狭}{12,9}{⾡,⽝}
  \begin{Phonetics}{逼狭}{bi1xia2}
    \definition{adj.}{estreito}
  \end{Phonetics}
\end{Entry}

\begin{Entry}{逼真}{12,10}{⾡,⼗}
  \begin{Phonetics}{逼真}{bi1zhen1}[][HSK 7-9]
    \definition{adj.}{fiel à realidade; realista; muito semelhante à coisa real | claro; distinto; verdadeiro}
  \end{Phonetics}
\end{Entry}

%%%%%%%%%% 遂 %%%%%%%%%%
\subsection*{遂}\addcontentsline{loh}{figure}{遂}

\begin{Entry}{遂}{12}{⾡}
  \begin{Phonetics}{遂}{sui2}
    \definition{adv.}{então; portanto; assim; como resultado; posteriormente}
    \definition{s.}{hemiplegia (paralisia de um lado do corpo)}
    \definition{v.}{ser como desejado; cumprir; satisfazer | ter sucesso; realizar; alcançar; concluir; ser bem"-sucedido}
  \end{Phonetics}
  \begin{Phonetics}{遂}{sui4}
    \definition*{s.}{Sobrenome: Sui}
    \definition{adv.}{Literário: então; em seguida}
    \definition{v.}{satisfazer; realizar | Literário: ter sucesso | cumprir}
  \end{Phonetics}
\end{Entry}

\begin{Entry}{遂心}{12,4}{⾡,⼼}
  \begin{Phonetics}{遂心}{sui4/xin1}[][HSK 7-9]
    \definition{v.+compl.}{ter o desejo do coração realizado; ter um anseio atendido; satisfazer o próprio desejo}
  \end{Phonetics}
\end{Entry}

\begin{Entry}{遂意}{12,13}{⾡,⼼}
  \begin{Phonetics}{遂意}{sui4yi4}
    \definition{adj.}{de acordo com o próprio gosto; do agrado de alguém}
  \end{Phonetics}
\end{Entry}

%%%%%%%%%% 遇 %%%%%%%%%%
\subsection*{遇}\addcontentsline{loh}{figure}{遇}

\begin{Entry}{遇}{12}{⾡}
  \begin{Phonetics}{遇}{yu4}[][HSK 4]
    \definition*{s.}{Sobrenome: Yu}
    \definition{s.}{chance; oportunidade}
    \definition{v.}{encontrar; deparar-se com; encontrar-se | tratar; receber}
  \end{Phonetics}
\end{Entry}

\begin{Entry}{遇见}{12,4}{⾡,⾒}
  \begin{Phonetics}{遇见}{yu4/jian5}[][HSK 4]
    \definition{v.+compl.}{encontrar; deparar-se com}
  \end{Phonetics}
\end{Entry}

\begin{Entry}{遇到}{12,8}{⾡,⼑}
  \begin{Phonetics}{遇到}{yu4/dao4}[][HSK 4]
    \definition{v.+compl.}{esbarrar em; encontrar; deparar-se com; conhecer alguém ou algo (inesperado)}
  \end{Phonetics}
\end{Entry}

%%%%%%%%%% 遍 %%%%%%%%%%
\subsection*{遍}\addcontentsline{loh}{figure}{遍}

\begin{Entry}{遍}{12}{⾡}
  \begin{Phonetics}{遍}{bian4}[][HSK 2]
    \definition{adv.}{por toda parte; em toda parte; em todos os lugares}
    \definition{clas.}{usado para a repetição de ações de leitura, fala ou escrita}
  \end{Phonetics}
\end{Entry}

\begin{Entry}{遍布}{12,5}{⾡,⼱}
  \begin{Phonetics}{遍布}{bian4bu4}[][HSK 7-9]
    \definition{v.}{encontrar em todos os lugares; espalhar por toda parte; distribuir em todos os lugares}
  \end{Phonetics}
\end{Entry}

\begin{Entry}{遍地}{12,6}{⾡,⼟}
  \begin{Phonetics}{遍地}{bian4di4}[][HSK 6]
    \definition{adv.}{em todos os lugares; em toda parte; por toda parte}
  \end{Phonetics}
\end{Entry}

%%%%%%%%%% 遏 %%%%%%%%%%
\subsection*{遏}\addcontentsline{loh}{figure}{遏}

\begin{Entry}{遏}{12}{⾡}
  \begin{Phonetics}{遏}{e4}
    \definition{v.}{reprimir; restringir; reter; impedir; proibir}
  \end{Phonetics}
\end{Entry}

\begin{Entry}{遏制}{12,8}{⾡,⼑}
  \begin{Phonetics}{遏制}{e4zhi4}[][HSK 7-9]
    \definition{v.}{conter; restringir; controlar e prevenir ativamente o desenvolvimento de coisas que possam trazer perigo; usado principalmente para discutir tópicos formais}
  \end{Phonetics}
\end{Entry}

%%%%%%%%%% 道 %%%%%%%%%%
\subsection*{道}\addcontentsline{loh}{figure}{道}

\begin{Entry}{道}{12}{⾡}
  \begin{Phonetics}{道}{dao4}[][HSK 2]
    \definition*{s.}{Taoismo;  Taoista | Sobrenome: Dao}
    \definition{clas.}{usado para pratos em refeições, etapas em um procedimento, etc. | usado para certos objetos longos e estreitos; tira | usado para portas, paredes, etc.; pesado | usado para comandos, títulos, etc.}
    \definition[条]{s.}{estrada; caminho; trilha | curso; canal; o caminho percorrido pelo fluxo da água | maneira; método; princípio; raciocínio | moral; moralidade | habilidade; técnica | doutrina; princípio; sistema de pensamento acadêmico ou religioso; origem de todas as coisas no universo | taoísta; taoísmo; pertencente ao taoísmo | seita supersticiosa; certas organizações reacionárias e supersticiosas | linha; traços finos e alongados | trato; os canais dentro do corpo}
    \definition{v.}{dizer; falar; expressar-se | pensar; supor; considerar; acreditar que}
  \end{Phonetics}
\end{Entry}

\begin{Entry}{道行}{12,6}{⾡,⾏}
  \begin{Phonetics}{道行}{dao4 heng2}
    \definition{s.}{realizações de um monge budista ou sacerdote taoísta | habilidades; capacidades; aptidões | (figurativo) habilidade | habilidades adquiridas através da prática religiosa}
  \end{Phonetics}
\end{Entry}

\begin{Entry}{道具}{12,8}{⾡,⼋}
  \begin{Phonetics}{道具}{dao4ju4}[][HSK 7-9]
    \definition{s.}{adereços; objetos de cena; artigos de palco; objetos usados em apresentações, como mesas e cadeiras, são chamados de grandes adereços, enquanto cigarros e xícaras de chá são chamados de pequenos adereços}
  \end{Phonetics}
\end{Entry}

\begin{Entry}{道教}{12,11}{⾡,⽁}
  \begin{Phonetics}{道教}{dao4jiao4}[][HSK 6]
    \definition*{s.}{Taoísmo (sistema de crenças chinês)}
    \definition{s.}{a religião taoísta; taoísmo}
  \end{Phonetics}
\end{Entry}

\begin{Entry}{道理}{12,11}{⾡,⽟}
  \begin{Phonetics}{道理}{dao4li5}[][HSK 2]
    \definition[个,种]{s.}{verdade; princípio; a lei das coisas | sentido; razão}
  \end{Phonetics}
\end{Entry}

\begin{Entry}{道路}{12,13}{⾡,⾜}
  \begin{Phonetics}{道路}{dao4lu4}[][HSK 2]
    \definition[条,段]{s.}{estrada; caminho; os canais de comunicação entre os dois lugares, incluindo terrestres e aquáticos | caminho; processo; refere"-se à vida, à existência (significado abstrato)}
  \end{Phonetics}
\end{Entry}

\begin{Entry}{道歉}{12,14}{⾡,⽋}
  \begin{Phonetics}{道歉}{dao4/qian4}[][HSK 6]
    \definition{v.+compl.}{pedir desculpas; fazer um pedido de desculpas; dizer aos outros que você estava errado e pedir perdão}
  \end{Phonetics}
\end{Entry}

\begin{Entry}{道德}{12,15}{⾡,⼻}
  \begin{Phonetics}{道德}{dao4de2}[][HSK 5]
    \definition{adj.}{moral; descreve uma pessoa ou comportamento que atende aos requisitos morais; mais usado em situações negativas}
    \definition[种]{s.}{moral; ética; moralidade; regras e normas para que as pessoas vivam juntas e se comportem em comum}
  \end{Phonetics}
\end{Entry}

%%%%%%%%%% 遗 %%%%%%%%%%
\subsection*{遗}\addcontentsline{loh}{figure}{遗}

\begin{Entry}{遗}{12}{⾡}
  \begin{Phonetics}{遗}{yi2}
    \definition*{s.}{Sobrenome: Yi}
    \definition{s.}{descarga involuntária de urina, etc. | algo perdido}
    \definition{v.}{perder | omitir | deixar para trás; guardar; não dar | deixar para trás após a morte; legar; transmitir}
  \end{Phonetics}
\end{Entry}

\begin{Entry}{遗产}{12,6}{⾡,⼇}
  \begin{Phonetics}{遗产}{yi2chan3}[][HSK 4]
    \definition[笔,份]{s.}{legado; herança; patrimônio; propriedade deixada pelo falecido | patrimônio; riqueza cultural ou riqueza material transmitida pela história}
  \end{Phonetics}
\end{Entry}

\begin{Entry}{遗传}{12,6}{⾡,⼈}
  \begin{Phonetics}{遗传}{yi2chuan2}[][HSK 4]
    \definition{v.}{herdar, descender, transmitir, passar adiante}
  \end{Phonetics}
\end{Entry}

\begin{Entry}{遗男}{12,7}{⾡,⽥}
  \begin{Phonetics}{遗男}{yi2nan2}
    \definition{s.}{órfão | filho póstumo}
  \end{Phonetics}
\end{Entry}

\begin{Entry}{遗迹}{12,9}{⾡,⾡}
  \begin{Phonetics}{遗迹}{yi2ji4}
    \definition{s.}{vestígio histórico; sítio; vestígio; traço; ruína; vestígios deixados por tempos antigos ou eras passadas}
  \end{Phonetics}
\end{Entry}

\begin{Entry}{遗案}{12,10}{⾡,⽊}
  \begin{Phonetics}{遗案}{yi2'an4}
    \definition{s.}{(lei) caso não resolvido}
  \end{Phonetics}
\end{Entry}

\begin{Entry}{遗落}{12,12}{⾡,⾋}
  \begin{Phonetics}{遗落}{yi2luo4}
    \definition{v.}{esquecer | deixar para trás (inadvertidamente) | deixar de fora | omitir}
  \end{Phonetics}
\end{Entry}

\begin{Entry}{遗嘱}{12,15}{⾡,⼝}
  \begin{Phonetics}{遗嘱}{yi2zhu3}
    \definition{s.}{testamento}
  \end{Phonetics}
\end{Entry}

\begin{Entry}{遗骸}{12,15}{⾡,⾻}
  \begin{Phonetics}{遗骸}{yi2hai2}
    \definition{v.}{restos mortais}
  \end{Phonetics}
\end{Entry}

\begin{Entry}{遗憾}{12,16}{⾡,⼼}
  \begin{Phonetics}{遗憾}{yi2han4}[][HSK 6]
    \definition{adj.}{triste; arrependido; contrito; sentir pena de situações que estão fora de controle ou são insatisfatórias}
    \definition{s.}{pena; arrependimento; sentindo pena que os desejos não se realizaram}
  \end{Phonetics}
\end{Entry}

%%%%%%%%%% 酢 %%%%%%%%%%
\subsection*{酢}\addcontentsline{loh}{figure}{酢}

\begin{Entry}{酢}{12}{⾣}
  \begin{Phonetics}{酢}{cu4}
    \definition{s.}{vinagre | (figurativo) ciúme (como em um caso de amor)}
    \variantof{醋}
  \end{Phonetics}
  \begin{Phonetics}{酢}{zuo4}
    \definition{s.}{brinde ao anfitrião feito pelo convidado}
  \end{Phonetics}
\end{Entry}

%%%%%%%%%% 酣 %%%%%%%%%%
\subsection*{酣}\addcontentsline{loh}{figure}{酣}

\begin{Entry}{酣}{12}{⾣}
  \begin{Phonetics}{酣}{han1}
    \definition{adj.}{intoxicado}
  \end{Phonetics}
\end{Entry}

\begin{Entry}{酣畅}{12,8}{⾣,⽥}
  \begin{Phonetics}{酣畅}{han1chang4}[][HSK 7-9]
    \definition{adj.}{alegre e animado (com bebida) | profundo (sono profundo)}
    \definition{adv.}{com facilidade e entusiasmo; totalmente; refere"-se a obras literárias e artísticas}
  \end{Phonetics}
\end{Entry}

\begin{Entry}{酣睡}{12,13}{⾣,⽬}
  \begin{Phonetics}{酣睡}{han1shui4}[][HSK 7-9]
    \definition{v.}{dormir profundamente; estar em sono profundo | estar profundamente adormecido; cair em sono profundo}
  \end{Phonetics}
\end{Entry}

%%%%%%%%%% 酥 %%%%%%%%%%
\subsection*{酥}\addcontentsline{loh}{figure}{酥}

\begin{Entry}{酥}{12}{⾣}
  \begin{Phonetics}{酥}{su1}[][HSK 7-9]
    \definition{adj.}{crocante; (alimento) solto e frágil | macio; derretido}
    \definition{s.}{Arcaico: ghee (tipo de manteiga clarificada) | biscoito amanteigado; uma massa leve e quebradiça feita com farinha, óleo e açúcar}
  \end{Phonetics}
\end{Entry}

%%%%%%%%%% 释 %%%%%%%%%%
\subsection*{释}\addcontentsline{loh}{figure}{释}

\begin{Entry}{释}{12}{⾤}
  \begin{Phonetics}{释}{shi4}
    \definition*{s.}{Sakyamuni; refere"-se a Siddhartha Gautama, o fundador do budismo; também se refere ao próprio budismo}
    \definition{s.}{budismo}
    \definition{v.}{explicar; elucidar | esclarecer; dissipar; deixar ir; aliviar | soltar; ser aliviado de; aliviar; deixar ir; colocar no chão | libertar; pôr em liberdade}
  \end{Phonetics}
\end{Entry}

\begin{Entry}{释放}{12,8}{⾤,⽅}
  \begin{Phonetics}{释放}{shi4fang4}[][HSK 7-9]
    \definition{v.}{libertar; pôr em liberdade; restaurar a liberdade pessoal dos detidos | libertar; deixar sair; liberar algo interno, como matéria ou energia; metaforicamente, liberar as emoções internas}
  \synonymref{放走}{fang4zou3}
  \synonymref{排放}{pai2fang4}
  \antonymref{捕捉}{bu3zhuo1}
  \antonymref{逮捕}{dai4bu3}
  \antonymref{拘留}{ju1liu2}
  \antonymref{扣押}{kou4ya1}
  \antonymref{收集}{shou1ji2}
  \antonymref{吸收}{xi1shou1}
  \end{Phonetics}
\end{Entry}

%%%%%%%%%% 量 %%%%%%%%%%
\subsection*{量}\addcontentsline{loh}{figure}{量}

\begin{Entry}{量}{12}{⾥}
  \begin{Phonetics}{量}{liang2}[][HSK 4]
    \definition{v.}{medir | estimar; dimensionar}
  \end{Phonetics}
  \begin{Phonetics}{量}{liang4}
    \definition{s.}{instrumento de medida; antigamente, o termo se referia a objetos como baldes e litros, que medem o volume | capacidade (para tolerância ou ingestão de alimentos ou bebidas); refere"-se ao limite do que pode ser acomodado | quantidade; valor; volume; número}
    \definition{v.}{estimar; medir; pesar}
  \end{Phonetics}
\end{Entry}

%%%%%%%%%% 铺 %%%%%%%%%%
\subsection*{铺}\addcontentsline{loh}{figure}{铺}

\begin{Entry}{铺}{12}{⾦}
  \begin{Phonetics}{铺}{pu1}[][HSK 6]
    \definition{clas.}{usado para kang, etc.; kang, uma plataforma de alvenaria ou de barro em uma extremidade de um cômodo, aquecida no inverno por fogueiras embaixo e coberta com esteiras para dormir}
    \definition{v.}{espalhar; estender; desdobrar | colocar; pavimentar}
  \end{Phonetics}
  \begin{Phonetics}{铺}{pu4}
    \definition{s.}{pequena loja; depósito | uma cama feita de tábuas de madeira; geralmente se refere a uma cama | estação de correios; antiga estação de correios (usada principalmente em nomes de lugares)}
  \end{Phonetics}
\end{Entry}

\begin{Entry}{铺垫}{12,9}{⾦,⼟}
  \begin{Phonetics}{铺垫}{pu1dian4}
    \definition{s.}{cobre leito | colcha | roupa de cama}
    \definition{v.}{pavimentar}
  \end{Phonetics}
\end{Entry}

\begin{Entry}{铺路}{12,13}{⾦,⾜}
  \begin{Phonetics}{铺路}{pu1/lu4}[][HSK 7-9]
    \definition{v.+compl.}{pavimentar (uma estrada); construir uma estrada | Figurativo: preparar o terreno (para algo); criar condições para fazer algo; lançar as bases para | dar um presente a alguém para garantir o sucesso}
  \end{Phonetics}
\end{Entry}

%%%%%%%%%% 销 %%%%%%%%%%
\subsection*{销}\addcontentsline{loh}{figure}{销}

\begin{Entry}{销}{12}{⾦}
  \begin{Phonetics}{销}{xiao1}
    \definition*{s.}{Sobrenome: Xiao}
    \definition{s.}{gasto; despesa | pino}
    \definition{v.}{derreter (metal) | cancelar; anular | vender; comercializar | aferrolhar; fixar; prender; pregar | fixar com um parafuso; parafusar | gastar (consumo) | inserir um pino}
  \end{Phonetics}
\end{Entry}

\begin{Entry}{销售}{12,11}{⾦,⼝}
  \begin{Phonetics}{销售}{xiao1shou4}[][HSK 4]
    \definition{v.}{vender; comercializar}
  \end{Phonetics}
\end{Entry}

%%%%%%%%%% 锁 %%%%%%%%%%
\subsection*{锁}\addcontentsline{loh}{figure}{锁}

\begin{Entry}{锁}{12}{⾦}
  \begin{Phonetics}{锁}{suo3}[][HSK 5]
    \definition[把]{s.}{fechadura; dispositivo que impede que as pessoas abram facilmente a parte que se abre e fecha | correntes; cadeado e correntes | qualquer coisa com a forma de um cadeado antigo}
    \definition{v.}{trancar; trancar com chave | costurar com ponto fixo | tricotar}
  \end{Phonetics}
\end{Entry}

\begin{Entry}{锁定}{12,8}{⾦,⼧}
  \begin{Phonetics}{锁定}{suo3ding4}[][HSK 7-9]
    \definition{v.}{fixar; trancar | assegurar; confirmar; estabelecer; garantir | travar em; acompanhar atentamente}
  \synonymref{冻结}{dong4jie2}
  \synonymref{固定}{gu4ding4}
  \end{Phonetics}
\end{Entry}

%%%%%%%%%% 锅 %%%%%%%%%%
\subsection*{锅}\addcontentsline{loh}{figure}{锅}

\begin{Entry}{锅}{12}{⾦}
  \begin{Phonetics}{锅}{guo1}[][HSK 5]
    \definition[口,个,只]{s.}{panela; frigideira; utensílios de cozinha, redondos e côncavos, feitos principalmente de ferro, alumínio, etc. | parte que se parece com um pote em alguns objetos}
  \end{Phonetics}
\end{Entry}

%%%%%%%%%% 锐 %%%%%%%%%%
\subsection*{锐}\addcontentsline{loh}{figure}{锐}

\begin{Entry}{锐}{12}{⾦}
  \begin{Phonetics}{锐}{rui4}
    \definition*{s.}{Sobrenome: Rui}
    \definition{adj.}{afiado; aguçado | agudo; perspicaz | rápido; ágil; veloz}
    \definition{adv.}{rapidamente; de repente}
    \definition{s.}{vigor; espírito de luta | armas afiadas}
  \antonymref{钝}{dun4}
  \end{Phonetics}
\end{Entry}

%%%%%%%%%% 阔 %%%%%%%%%%
\subsection*{阔}\addcontentsline{loh}{figure}{阔}

\begin{Entry}{阔}{12}{⾨}
  \begin{Phonetics}{阔}{kuo4}[][HSK 6]
    \definition{adj.}{amplo; amplo; vasto | rico | longo, no sentido de ``há muito tempo'' | vazio; impraticável}
  \end{Phonetics}
\end{Entry}

\begin{Entry}{阔绰}{12,11}{⾨,⽷}
  \begin{Phonetics}{阔绰}{kuo4chuo4}[][HSK 7-9]
    \definition{adj.}{ostentoso; generoso com dinheiro; extravagante; luxuoso}
  \end{Phonetics}
\end{Entry}

%%%%%%%%%% 隔 %%%%%%%%%%
\subsection*{隔}\addcontentsline{loh}{figure}{隔}

\begin{Entry}{隔}{12}{⾩}
  \begin{Phonetics}{隔}{ge2}[][HSK 4]
    \definition{adj.}{seguinte; vizinho}
    \definition{v.}{separar; cortar; dividir; particionar | estar a uma distância de, após ou em um intervalo de | ficar de pé ou deitar entre}
  \end{Phonetics}
\end{Entry}

\begin{Entry}{隔开}{12,4}{⾩,⼶}
  \begin{Phonetics}{隔开}{ge2kai1}[][HSK 4]
    \definition{v.}{separar; manter separado; barricar; separar completamente duas pessoas (ou coisas) ou duas partes de uma coisa que estão intimamente unidas}
  \end{Phonetics}
\end{Entry}

\begin{Entry}{隔阂}{12,9}{⾩,⾨}
  \begin{Phonetics}{隔阂}{ge2he2}[][HSK 7-9]
    \definition[层,种,点]{s.}{estranhamento; mal-entendido; há uma falta de conexão emocional e uma distância de pensamento entre eles}
  \end{Phonetics}
\end{Entry}

\begin{Entry}{隔离}{12,10}{⾩,⼇}
  \begin{Phonetics}{隔离}{ge2li2}[][HSK 7-9]
    \definition{v.}{segregar; não permitir que as pessoas se reúnam, cortar o contato | isolar; colocar em quarentena; separar pessoas e animais com doenças infecciosas de pessoas e animais saudáveis para evitar o contato}
  \end{Phonetics}
\end{Entry}

\begin{Entry}{隔壁}{12,16}{⾩,⼟}
  \begin{Phonetics}{隔壁}{ge2bi4}[][HSK 5]
    \definition{s.}{vizinho; casas ou pessoas vizinhas | septo; distante (socialmente distante) | anteparo; partição}
  \end{Phonetics}
\end{Entry}

%%%%%%%%%% 雄 %%%%%%%%%%
\subsection*{雄}\addcontentsline{loh}{figure}{雄}

\begin{Entry}{雄}{12}{⾫}
  \begin{Phonetics}{雄}{xiong2}
    \definition*{s.}{Sobrenome: Xiong}
    \definition{adj.}{masculino | grandioso; imponente; audacioso | poderoso}
    \definition{s.}{uma pessoa ou país com grande poder e influência}
  \end{Phonetics}
\end{Entry}

\begin{Entry}{雄伟}{12,6}{⾫,⼈}
  \begin{Phonetics}{雄伟}{xiong2wei3}[][HSK 5]
    \definition{adj.}{magnífico; magnificente | imponente; magnífico}
  \end{Phonetics}
\end{Entry}

%%%%%%%%%% 集 %%%%%%%%%%
\subsection*{集}\addcontentsline{loh}{figure}{集}

\begin{Entry}{集}{12}{⾫}
  \begin{Phonetics}{集}{ji2}[][HSK 6]
    \definition*{s.}{Sobrenome: Ji}
    \definition{clas.}{parte; volume}
    \definition[个,本]{s.}{mercado; feira rural | coleção; conjunto; antologia | Matemática: conjunto}
    \definition{v.}{reunir; coletar; montar}
  \end{Phonetics}
\end{Entry}

\begin{Entry}{集中}{12,4}{⾫,⼁}
  \begin{Phonetics}{集中}{ji2zhong1}[][HSK 3]
    \definition{adj.}{centralizado; concentrado}
    \definition{v.}{concentrar; centralizar; focar; acumular; reunir | reunir pessoas, coisas, forças, etc. dispersas; resumir opiniões, experiências, etc.}
  \antonymref{分散}{fen1san4}
  \end{Phonetics}
\end{Entry}

\begin{Entry}{集会}{12,6}{⾫,⼈}
  \begin{Phonetics}{集会}{ji2hui4}[][HSK 7-9]
    \definition[个,次]{s.}{assembleia; reunião}
    \definition{v.}{reunir; reunir-se}
  \end{Phonetics}
\end{Entry}

\begin{Entry}{集合}{12,6}{⾫,⼝}
  \begin{Phonetics}{集合}{ji2he2}[][HSK 4]
    \definition{s.}{conjunto; montagem; coleção; agregação}
    \definition{v.}{reunir-se; juntar-se | reunir, juntar, convocar}
  \end{Phonetics}
\end{Entry}

\begin{Entry}{集团}{12,6}{⾫,⼞}
  \begin{Phonetics}{集团}{ji2tuan2}[][HSK 5]
    \definition[个,家,些]{s.}{anel; bloco; grupo; panelinha; círculo; grupo organizado para agir em conjunto com um determinado objetivo | grupo; entidade econômica com uma direção de negócios especializada, liderada por uma grande empresa com forte poder econômico e alta visibilidade, e formada pela combinação ou fusão de empresas relacionadas}
  \end{Phonetics}
\end{Entry}

\begin{Entry}{集体}{12,7}{⾫,⼈}
  \begin{Phonetics}{集体}{ji2ti3}[][HSK 3]
    \definition{s.}{coletivo; comunidade; grupo; equipe; organizações ou grupos em que muitas pessoas trabalham, estudam e vivem juntas}
  \end{Phonetics}
\end{Entry}

\begin{Entry}{集邮}{12,7}{⾫,⾢}
  \begin{Phonetics}{集邮}{ji2/you2}[][HSK 7-9]
    \definition[本]{s.}{filatelia; coleção de selos}
    \definition{v.+compl.}{colecionar selos}
  \end{Phonetics}
\end{Entry}

\begin{Entry}{集结}{12,9}{⾫,⽷}
  \begin{Phonetics}{集结}{ji2jie2}[][HSK 7-9]
    \definition{v.}{(especialmente tropas) reunir; concentrar; estabelecer; fortalecer}
  \end{Phonetics}
\end{Entry}

\begin{Entry}{集资}{12,10}{⾫,⾙}
  \begin{Phonetics}{集资}{ji2zi1}[][HSK 7-9]
    \definition{v.}{angariar fundos; recolher dinheiro; reunir recursos | arrecadar (reunir) dinheiro; concentrar fundos; retirar dinheiro (de muitas fontes); arrecadar fundos; solicitar fundos}
  \end{Phonetics}
\end{Entry}

\begin{Entry}{集装箱}{12,12,15}{⾫,⾐,⾋}
  \begin{Phonetics}{集装箱}{ji2zhuang1xiang1}[][HSK 7-9]
    \definition{s.}{\emph{container}}
  \end{Phonetics}
\end{Entry}

%%%%%%%%%% 雇 %%%%%%%%%%
\subsection*{雇}\addcontentsline{loh}{figure}{雇}

\begin{Entry}{雇}{12}{⾫}
  \begin{Phonetics}{雇}{gu4}[][HSK 7-9]
    \definition{v.}{contratar; empregar; pagar pessoas para fazerem coisas por você | contratar (transporte de aluguel)}
  \end{Phonetics}
\end{Entry}

\begin{Entry}{雇主}{12,5}{⾫,⼂}
  \begin{Phonetics}{雇主}{gu4zhu3}[][HSK 7-9]
    \definition[名]{s.}{empregador; uma pessoa que contrata trabalhadores, veículos ou barcos}
  \end{Phonetics}
\end{Entry}

\begin{Entry}{雇佣}{12,7}{⾫,⼈}
  \begin{Phonetics}{雇佣}{gu4yong1}[][HSK 7-9]
    \definition{v.}{contratar; empregar; comprar mão de obra com dinheiro}
  \end{Phonetics}
\end{Entry}

\begin{Entry}{雇员}{12,7}{⾫,⼝}
  \begin{Phonetics}{雇员}{gu4yuan2}[][HSK 7-9]
    \definition[名,位,个]{s.}{empregado; servo; pessoal contratado ou temporário fora do estabelecimento}
  \end{Phonetics}
\end{Entry}

%%%%%%%%%% 靓 %%%%%%%%%%
\subsection*{靓}\addcontentsline{loh}{figure}{靓}

\begin{Entry}{靓}{12}{⾭}
  \begin{Phonetics}{靓}{jing4}
    \definition{v.}{(referindo"-se a vestimenta de alguém) ficar bonito | vestir"-se | maquiar (o rosto)}
  \end{Phonetics}
  \begin{Phonetics}{靓}{liang4}
    \definition{v.}{Literário: vestir-se bem; maquiar-se}
  \end{Phonetics}
\end{Entry}

\begin{Entry}{靓丽}{12,7}{⾭,⼀}
  \begin{Phonetics}{靓丽}{liang4li4}
    \definition{adj.}{lindo; bonito}
  \end{Phonetics}
\end{Entry}

%%%%%%%%%% 韩 %%%%%%%%%%
\subsection*{韩}\addcontentsline{loh}{figure}{韩}

\begin{Entry}{韩}{12}{⾱}
  \begin{Phonetics}{韩}{han2}
    \definition*{s.}{Um estado durante o Período dos Estados Combatentes nas atuais províncias centrais de Henan e sudeste de Shanxi | O nome de um estado feudal durante a dinastia Zhou, localizado no que hoje é o nordeste de Hejin, província de Shanxi | Coreia do Sul, abreviação de 韩国; República da Coreia (RC) | Sobrenome: Han}
  \seealsoref{韩国}{han2guo2}
  \end{Phonetics}
\end{Entry}

\begin{Entry}{韩国}{12,8}{⾱,⼞}
  \begin{Phonetics}{韩国}{han2guo2}
    \definition*{s.}{Coréia do Sul; República da Coreia}
  \end{Phonetics}
\end{Entry}

\begin{Entry}{韩国人}{12,8,2}{⾱,⼞,⼈}
  \begin{Phonetics}{韩国人}{han2guo2ren2}
    \definition{s.}{coreano | pessoa ou povo da Coréia}
  \end{Phonetics}
\end{Entry}

%%%%%%%%%% 馋 %%%%%%%%%%
\subsection*{馋}\addcontentsline{loh}{figure}{馋}

\begin{Entry}{馋}{12}{⾷}
  \begin{Phonetics}{馋}{chan2}[][HSK 7-9]
    \definition{adj.}{cobiçoso; invejoso | guloso; comilão; glutão}
    \definition{v.}{desejar comida; querer comer (alguma coisa)}
  \end{Phonetics}
\end{Entry}

%%%%%%%%%% 骗 %%%%%%%%%%
\subsection*{骗}\addcontentsline{loh}{figure}{骗}

\begin{Entry}{骗}{12}{⾺}
  \begin{Phonetics}{骗}{pian4}[][HSK 5]
    \definition{v.}{enganar; trapacear; iludir; ludibriar; usar mentiras ou meios fraudulentos para fazer alguém acreditar ou ser enganado | enganar; fraudar | montar (um cavalo); balançar (ou saltar) para a sela}
  \end{Phonetics}
\end{Entry}

\begin{Entry}{骗人}{12,2}{⾺,⼈}
  \begin{Phonetics}{骗人}{pian4 ren2}[][HSK 7-9]
    \definition{v.}{enganar; ludibriar alguém; enganar alguém usando mentiras ou truques}
  \end{Phonetics}
\end{Entry}

\begin{Entry}{骗子}{12,3}{⾺,⼦}
  \begin{Phonetics}{骗子}{pian4zi5}[][HSK 5]
    \definition[个]{s.}{trapaceiro; vigarista; fraudador; impostor; golpista; pessoa que obtém bens de forma fraudulenta}
  \end{Phonetics}
\end{Entry}

%%%%%%%%%% 骚 %%%%%%%%%%
\subsection*{骚}\addcontentsline{loh}{figure}{骚}

\begin{Entry}{骚}{12}{⾺}
  \begin{Phonetics}{骚}{sao1}
    \definition*{s.}{Abreviação de Li Sao (Encontrando a Tristeza), um poema do poeta e estadista do século IV a.C. Qu Yuan (屈原)}
    \definition{adj.}{coquete; (de uma mulher) lasciva | masculino (de alguns animais domésticos)}
    \definition{s.}{escritos literários; geralmente se refere à poesia | o cheiro de urina; mau cheiro}
    \definition{v.}{perturbar}
  \seealsoref{屈原}{qu1yuan2}
  \end{Phonetics}
\end{Entry}

\begin{Entry}{骚乱}{12,7}{⾺,⼄}
  \begin{Phonetics}{骚乱}{sao1luan4}[][HSK 7-9]
    \definition{s.}{rebelião; perturbação; motim; confusão}
    \definition{v.}{criar perturbação; estar em meio a uma turbulência; causar problemas}
  \end{Phonetics}
\end{Entry}

\begin{Entry}{骚扰}{12,7}{⾺,⼿}
  \begin{Phonetics}{骚扰}{sao1rao3}[][HSK 7-9]
    \definition{v.}{assediar; molestar; perturbar}
  \end{Phonetics}
\end{Entry}

%%%%%%%%%% 鲁 %%%%%%%%%%
\subsection*{鲁}\addcontentsline{loh}{figure}{鲁}

\begin{Entry}{鲁}{12}{⿂}
  \begin{Phonetics}{鲁}{lu3}
    \definition*{s.}{Lu, um dos estados beligerantes em que a China foi dividida durante o período Zhou Oriental, localizado na porção sul da atual província de Shandong | outro nome para Província de Shandong | Sobrenome: Lu}
  \seealsoref{山东}{shan1dong1}
  \end{Phonetics}
\end{Entry}

\begin{Entry}{鲁莽}{12,10}{⿂,⾋}
  \begin{Phonetics}{鲁莽}{lu3mang3}[][HSK 7-9]
    \definition{adj.}{imprudente; grosseiro e precipitado; discurso e comportamento rudes e imprudentes}
  \end{Phonetics}
\end{Entry}

%%%%%%%%%% 鹅 %%%%%%%%%%
\subsection*{鹅}\addcontentsline{loh}{figure}{鹅}

\begin{Entry}{鹅}{12}{⿃}
  \begin{Phonetics}{鹅}{e2}[][HSK 7-9]
    \definition[只,群]{s.}{ganso}
  \end{Phonetics}
\end{Entry}

%%%%%%%%%% 黍 %%%%%%%%%%
\subsection*{黍}\addcontentsline{loh}{figure}{黍}

\begin{Entry}{黍}{12}{⿉}[Kangxi 202]
  \begin{Phonetics}{黍}{shu3}
    \definition{s.}{painço}
  \end{Phonetics}
\end{Entry}

%%%%%%%%%% 黑 %%%%%%%%%%
\subsection*{黑}\addcontentsline{loh}{figure}{黑}

\begin{Entry}{黑}{12}{⿊}[Kangxi 203]
  \begin{Phonetics}{黑}{hei1}[][HSK 2]
    \definition*{s.}{Província de Heilongjiang, abreviação de 黑龙江 | Sobrenome: Hei}
    \definition{adj.}{preto; cor semelhante à do carvão | escuro | obscuro; secreto | perverso; sinistro; ruim; cruel | reacionário}
    \definition{s.}{noite}
    \definition{v.}{fazer algo ilegalmente ou de forma desonesta; enganar; desviar dinheiro ilegalmente | invadir (uma rede, sites, computador, etc.)}
  \seealsoref{黑龙江}{hei1long2jiang1}
  \end{Phonetics}
\end{Entry}

\begin{Entry}{黑马}{12,3}{⿊,⾺}
  \begin{Phonetics}{黑马}{hei1ma3}[][HSK 7-9]
    \definition[匹,群]{s.}{azarão (cavalo preto) | Figurativo: pessoa pouco conhecida que alcança sucesso inesperado}
  \end{Phonetics}
\end{Entry}

\begin{Entry}{黑心}{12,4}{⿊,⼼}
  \begin{Phonetics}{黑心}{hei1xin1}[][HSK 7-9]
    \definition{adj.}{malvado; perverso | ganancioso; avarento | (certos bens) de má qualidade | implacável e sem consciência | de mente viciosa cheia de ódio e ciúme}
    \definition{s.}{coração negro; mente maligna | núcleo preto (falha na cerâmica)}
  \end{Phonetics}
\end{Entry}

\begin{Entry}{黑手}{12,4}{⿊,⼿}
  \begin{Phonetics}{黑手}{hei1shou3}[][HSK 7-9]
    \definition{s.}{mão negra; manipulador maligno dos bastidores | uma pessoa cruel manipulando alguém ou algo nos bastidores; uma metáfora para pessoas ou forças que secretamente realizam atividades de conspiração}
  \end{Phonetics}
\end{Entry}

\begin{Entry}{黑白}{12,5}{⿊,⽩}
  \begin{Phonetics}{黑白}{hei1bai2}[][HSK 7-9]
    \definition[只]{s.}{preto e branco | certo e errado; metáfora para o certo e o errado, o bem e o mal}
  \end{Phonetics}
\end{Entry}

\begin{Entry}{黑龙江}{12,5,6}{⿊,⿓,⽔}
  \begin{Phonetics}{黑龙江}{hei1long2jiang1}
    \definition*{s.}{Província de Heilongjiang | Rio Heilong Jiang;  Rio Amur (na Rússia)}
  \end{Phonetics}
\end{Entry}

\begin{Entry}{黑色}{12,6}{⿊,⾊}
  \begin{Phonetics}{黑色}{hei1se4}[][HSK 2]
    \definition{adj.}{metafórico: suspeito, ilegal}
    \definition{s.}{cor preta}
  \end{Phonetics}
\end{Entry}

\begin{Entry}{黑夜}{12,8}{⿊,⼣}
  \begin{Phonetics}{黑夜}{hei1ye4}[][HSK 6]
    \definition[个]{s.}{noite ; uma noite muito escura sem luz}
  \end{Phonetics}
\end{Entry}

\begin{Entry}{黑板}{12,8}{⿊,⽊}
  \begin{Phonetics}{黑板}{hei1ban3}[][HSK 2]
    \definition[块,个]{s.}{quadro negro; quadro de giz; uma placa, na qual se pode escrever com giz}
  \end{Phonetics}
\end{Entry}

\begin{Entry}{黑客}{12,9}{⿊,⼧}
  \begin{Phonetics}{黑客}{hei1ke4}[][HSK 7-9]
    \definition[个,些,位,名]{s.}{Empréstimo linguístico: \emph{hacker}; \emph{cracker}; intruso cibernético; gênio da computação; originalmente se refere a pessoas que não são profissionais de informática, mas são muito proficientes em tecnologia de computadores; agora se refere especificamente a pessoas que podem escrever programas de descriptografia para invadir ilegalmente redes de computadores de outras pessoas para interferir ou destruí-las}
  \end{Phonetics}
\end{Entry}

\begin{Entry}{黑桃}{12,10}{⿊,⽊}
  \begin{Phonetics}{黑桃}{hei1tao2}
    \definition{s.}{espadas ♠ (em jogos de cartas)}
  \seealsoref{方片}{fang1 pian4}
  \seealsoref{红心}{hong2xin1}
  \seealsoref{梅花}{mei2hua1}
  \end{Phonetics}
\end{Entry}

\begin{Entry}{黑暗}{12,13}{⿊,⽇}
  \begin{Phonetics}{黑暗}{hei1'an4}[][HSK 4]
    \definition{adj.}{escuro; sombrio; sem luz | maligno; corrupto; reacionário}
  \end{Phonetics}
\end{Entry}

%%%%%%%%%% 黹 %%%%%%%%%%
\subsection*{黹}\addcontentsline{loh}{figure}{黹}

\begin{Entry}{黹}{12}{⿋}[Kangxi 204]
  \begin{Phonetics}{黹}{zhi3}
    \definition{v.}{costurar; bordar}
  \end{Phonetics}
\end{Entry}

%%%%%%%%%% 鼎 %%%%%%%%%%
\subsection*{鼎}\addcontentsline{loh}{figure}{鼎}

\begin{Entry}{鼎}{12}{⿍}[Kangxi 206]
  \begin{Phonetics}{鼎}{ding3}
    \definition{adj.}{grande; generoso | importante; grandioso}
    \definition{adv.}{exatamente quando; exatamente o momento para}
    \definition[尊]{s.}{um antigo recipiente de cozinha com duas alças e três ou quatro pernas | pote; caldeirão | poder do estado; o trono | como símbolo de dinastia; nos tempos antigos, era considerada uma ferramenta importante para estabelecer um país}
  \end{Phonetics}
\end{Entry}

%%%%% EOF %%%%%


 %%%
%%% 13画
%%%

\section*{13画}\addcontentsline{toc}{section}{13画}

\begin{verbete}{嗄}{a2}{13}[Radical 口]
  \significado{adj.}{rouco}
  \variante{啊}
\end{verbete}

\begin{verbete}{矮}{ai3}{13}[Radical 矢]
  \significado{adj.}{baixo em estatura, dimensão, grau ou ranque | curto (em comprimento)}
\end{verbete}

\begin{verbete}{矮人}{ai3ren2}{13,2}
  \significado{s.}{anão | homúnculo | nanismo}
\end{verbete}

\begin{verbete}{矮小}{ai3xiao3}{13,3}
  \significado{adj.}{baixo e pequeno; curto e pequeno; subdimensionado}
\end{verbete}

\begin{verbete}{矮子}{ai3zi5}{13,3}
  \significado{s.}{pessoa baixa; anão}
\end{verbete}

\begin{verbete}{矮林}{ai3lin2}{13,8}
  \significado{s.}{mato; mata}
\end{verbete}

\begin{verbete}{矮胖}{ai3pang4}{13,9}
  \significado{adj.}{atarracado |  gorducho; rechonchudo; roliço | curto e robusto}
\end{verbete}

\begin{verbete}{矮树}{ai3shu4}{13,9}
  \significado{s.}{arbusto; árvore pequena}
\end{verbete}

\begin{verbete}{矮星}{ai3xing1}{13,9}
  \significado{s.}{estrela anã}
\end{verbete}

\begin{verbete}{矮凳}{ai3deng4}{13,14}
  \significado{s.}{banquinho baixo; banqueta}
\end{verbete}

\begin{verbete}{碍事}{ai4shi4}{13,8}
  \significado{s.}{(usualmente em frases negativas) sem consequência; não importa}
  \significado{v.+compl.}{estar no caminho; ser um obstáculo}
\end{verbete}

\begin{verbete}{暗香}{an4xiang1}{13,9}
  \significado{s.}{fragrância sutil}
\end{verbete}

\begin{verbete}{暗恋}{an4lian4}{13,10}
  \significado{s.}{amor secreto}
  \significado{v.}{estar secretamente apaixonado por}
\end{verbete}

\begin{verbete}{摆手}{bai3shou3}{13,4}
  \significado{v.+compl.}{gesticular com a mão (acenando, acenando adeus, etc.) | balançar os braços | acenar com as mãos}
\end{verbete}

\begin{verbete}{摆烂}{bai3lan4}{13,9}
  \significado{v.}{(neologismo, gíria) parar de lutar (especialmente quando se sabe que não pode ter sucesso);
deixar tudo ir para o inferno}
\end{verbete}

\begin{verbete}{搬}{ban1}{13}[Radical 手]
  \significado{v.}{copiar indiscriminadamente; mover-se (ou seja, mudar-se); mover-se (algo relativamente pesado ou volumoso); mudar; mudar-se}
\end{verbete}

\begin{verbete}{搬口}{ban1kou3}{13,3}
  \significado{v.}{tagarelar; transmitir histórias (idioma); semear dissensão; contar histórias}
\end{verbete}

\begin{verbete}{搬动}{ban1dong4}{13,6}
  \significado{v.}{mover-se (alguma coisa); se mudar}
\end{verbete}

\begin{verbete}{搬弄}{ban1nong4}{13,7}
  \significado{v.}{causar problemas; mexer com alguém; mostrar (o que se pode fazer)}
\end{verbete}

\begin{verbete}{搬运}{ban1yun4}{13,7}
  \significado{s.}{frete; transporte}
  \significado{v.}{carregar; transportar}
\end{verbete}

\begin{verbete}{搬走}{ban1zou3}{13,7}
  \significado{v.}{carregar}
\end{verbete}

\begin{verbete}{搬家}{ban1jia1}{13,10}
  \significado{s.}{mudança}
  \significado{v.+compl.}{mudar-se de casa}
\end{verbete}

\begin{verbete}{酬劳}{chou2lao2}{13,7}
  \significado{s.}{recompensa}
\end{verbete}

\begin{verbete}{锤}{chui2}{13}[Radical 金]
  \significado{s.}{martelo, marreta}
  \significado{s.}{pesos (por exemplo, de uma balança)}
  \significado{v.}{marterlar para dar forma; atacar com um martelo}
\end{verbete}

\begin{verbete}{辞典}{ci2dian3}{13,8}
  \variante{词典}
\end{verbete}

\begin{verbete}{错}{cuo4}{13}[Radical 金]
  \significado*{s.}{sobrenome Cuo}
  \significado{adj.}{errado; enganado}
\end{verbete}

\begin{verbete}{嘟}{du1}{13}[Radical ⼝]
  \significado{s.}{buzina; bip}
  \significado{v.}{fazer beicinho}
\end{verbete}

\begin{verbete}{躲}{duo3}{13}[Radical ⾝]
  \significado{v.}{esconder; esquivar; evitar}
\end{verbete}

\begin{verbete}{躲闪}{duo3shan3}{13,5}
  \significado{v.}{desviar; evadir; esquivar (para fora do caminho)}
\end{verbete}

\begin{verbete}{缝纫}{feng2ren4}{13,6}
  \significado{v.}{costurar}
\end{verbete}

\begin{verbete}{缝纫机}{feng2ren4ji1}{13,6,6}
  \significado[架]{s.}{máquina de costura}
\end{verbete}

\begin{verbete}{福克斯}{fu2ke4si1}{13,7,12}
  \significado*{s.}{Fox (empresa de mídia); Focus (automóvel fabricado pela Ford)}
\end{verbete}

\begin{verbete}{福泽}{fu2ze2}{13,8}
  \significado{s.}{boa sorte}
\end{verbete}

\begin{verbete}{概念}{gai4nian4}{13,8}
  \significado[个]{s.}{conceito; ideia}
\end{verbete}

\begin{verbete}{感动}{gan3dong4}{13,6}
  \significado{v.}{mover (alguém); tocar (alguém emocionalmente)}
\end{verbete}

\begin{verbete}{感到}{gan3dao4}{13,8}
  \significado{v.}{sentir; perceber}
\end{verbete}

\begin{verbete}{感受}{gan3shou4}{13,8}
  \significado{s.}{percepção; um sentimento; uma percepção; uma experiência}
  \significado{v.}{sentir; sentir (através dos sentidos); experimentar}
\end{verbete}

\begin{verbete}{感觉}{gan3jue2}{13,9}
  \significado{s.}{sentimento; impressão; sensação}
  \significado{v.}{sentir; perceber}
\end{verbete}

\begin{verbete}{感冒}{gan3mao4}{13,9}
  \significado{v.}{ficar resfriado; estar com resfriado}
\end{verbete}

\begin{verbete}{感染}{gan3ran3}{13,9}
  \significado{s.}{infecção}
  \significado{v.}{infectar; (fig.) influenciar}
\end{verbete}

\begin{verbete}{感情}{gan3qing2}{13,11}
  \significado{s.}{afeição; emoção; sentimento; sentimento amoroso}
\end{verbete}

\begin{verbete}{感谢}{gan3xie4}{13,12}
  \significado{s.}{gratidão; agradecimento}
\end{verbete}

\begin{verbete}{搞}{gao3}{13}[Radical 手]
  \significado{v.}{fazer}
\end{verbete}

\begin{verbete}{搞好}{gao3hao3}{13,6}
  \significado{v.}{fazer um ótimo trabalho}
\end{verbete}

\begin{verbete}{搞乱}{gao3luan4}{13,7}
  \significado{v.}{estragar; confundir; bagunçar}
\end{verbete}

\begin{verbete}{搞定}{gao3ding4}{13,8}
  \significado{v.}{consertar; resolver}
\end{verbete}

\begin{verbete}{搞鬼}{gao3gui3}{13,9}
  \significado{v.}{fazer travessuras; fazer truques}
\end{verbete}

\begin{verbete}{搞钱}{gao3qian2}{13,10}
  \significado{v.}{fazer dinheiro; acumular dinheiro}
\end{verbete}

\begin{verbete}{搞通}{gao3tong1}{13,10}
  \significado{v.}{entender algo}
\end{verbete}

\begin{verbete}{搞笑}{gao3xiao4}{13,10}
  \significado{adj.}{engraçado; hilário}
  \significado{v.}{fazer as pessoas rirem}
\end{verbete}

\begin{verbete}{搞混}{gao3hun4}{13,11}
  \significado{v.}{confundir}
\end{verbete}

\begin{verbete}{搞错}{gao3cuo4}{13,13}
  \significado{v.}{cometer um erro}
\end{verbete}

\begin{verbete}{跟}{gen1}{13}[Radical ⾜]
  \significado{conj.}{e; com}
  \significado{prep.}{com}
  \significado{v.}{acompanhar junto; seguir de perto; ir com}
\end{verbete}

\begin{verbete}{彀}{gou4}{13}[Radical ⼸]
  \significado{s.}{calcance de um arco e flecha}
  \significado{v.}{puxar um arco ao máximo}
\end{verbete}

\begin{verbete}{鼓掌}{gu3zhang3}{13,12}
  \significado{v.+compl.}{aplaudir; bater palmas}
\end{verbete}

\begin{verbete}{跪拜}{gui4bai4}{13,9}
  \significado{v.}{prostrar-se; ajoelhar-se e adorar}
\end{verbete}

\begin{verbete}{滚轮}{gun3lun2}{13,8}
  \significado{s.}{pneu; dial rotativo; roda de rolagem (\emph{scroll}) (mouse de computador)}
\end{verbete}

\begin{verbete}{滚滚}{gun3gun3}{13,13}
  \significado*{s.}{Apelido para um panda}
  \significado{v.}{mover-se, rolar, fluir continuamente}
\end{verbete}

\begin{verbete}{魂}{hun2}{13}[Radical 鬼]
  \significado{s.}{alma; espírito; alma imortal (que pode ser separada do corpo)}
\end{verbete}

\begin{verbete}{嫉妒}{ji2du4}{13,7}
  \significado{v.}{estar com ciúmes de, invejar}
\end{verbete}

\begin{verbete}{煎}{jian1}{13}[Radical 火]
  \significado{v.}{fritar; refogar}
\end{verbete}

\begin{verbete}{煎饼}{jian1bing3}{13,9}
  \significado[张]{s.}{jianbing, crepe chinês; panqueca}
\end{verbete}

\begin{verbete}{煎蛋}{jian1dan4}{13,11}
  \significado{s.}{ovos fritos}
\end{verbete}

\begin{verbete}{简单}{jian3dan1}{13,8}
  \significado{adj.}{simples; sem complicações}
\end{verbete}

\begin{verbete}{简直}{jian3zhi2}{13,8}
  \significado{adv.}{simplesmente; realmente; absolutamente; em tudo}
\end{verbete}

\begin{verbete}{键}{jian4}{13}[Radical 金]
  \significado{s.}{tecla (em um teclado de piano ou computador); botão (em um mouse ou outro dispositivo); ligação química; cavilha de roda, chaveta}
\end{verbete}

\begin{verbete}{酱}{jiang4}{13}[Radical 酉]
  \significado{s.}{pasta grossa de soja fermentada; marinada em pasta de soja; pasta; geléia}
\end{verbete}

\begin{verbete}{解压}{jie3ya1}{13,6}
  \significado{v.}{aliviar o estresse; (computação) descomprimir}
\end{verbete}

\begin{verbete}{解救}{jie3jiu4}{13,11}
  \significado{v.}{resgatar; ajudar a sair de dificuldades; salvar a situação}
\end{verbete}

\begin{verbete}{解雇}{jie3gu4}{13,12}
  \significado{v.}{demitir}
\end{verbete}

\begin{verbete}{解释}{jie3shi4}{13,12}
  \significado[个]{s.}{explicação}
  \significado{v.}{explicar; interpretar; resolver}
\end{verbete}

\begin{verbete}{锦上添花}{jin3shang4tian1hua1}{13,3,11,7}
  \significado{expr.}{A cereja do bolo; (literalmente) adicione flores ao brocato}
  \significado{v.}{dar a alguém esplendor adicional; fornecer o toque final}
\end{verbete}

\begin{verbete}{赖}{lai4}{13}[Radical 貝]
  \significado*{s.}{sobrenome Lai}
  \significado{v.}{depender; aguentar em um lugar; renegar (promessa); isolar-se; culpar; colocar a culpa em}
\end{verbete}

\begin{verbete}{蓝}{lan2}{13}[Radical 艸]
  \significado*{s.}{sobrenome Lan}
  \significado{adj.}{azul}
\end{verbete}

\begin{verbete}{蓝色}{lan2se4}{13,6}
  \significado{s.}{cor azul}
\end{verbete}

\begin{verbete}{雷亚尔}{lei2ya4'er3}{13,6,5}
  \significado*{s.}{Real Brasileiro}
\end{verbete}

\begin{verbete}{零/〇}{ling2}{13}
  \significado{adj.}{extra}
  \significado{num.}{zero, 0}
  \significado{s.}{matemática:~resto (após a divisão); fração; nada}
\end{verbete}

\begin{verbete}{遛狗}{liu4gou3}{13,8}
  \significado{v.+compl.}{passear com um cachorro}
\end{verbete}

\begin{verbete}{楼}{lou2}{13}[Radical 木]
  \significado*{s.}{sobrenome Lou}
  \significado{clas.}{andar; piso}
  \significado[层,座,栋]{s.}{edifício; prédio; casa com 2 ou mais andares}
\end{verbete}

\begin{verbete}{楼上}{lou2shang4}{13,3}
  \significado{adv.}{no andar de cima; post anterior em um fio de um fórum (gíria da Internet)}
\end{verbete}

\begin{verbete}{楼下}{lou2xia4}{13,3}
  \significado{adv.}{no andar de baixo}
\end{verbete}

\begin{verbete}{楼梯}{lou2ti1}{13,11}
  \significado[个]{s.}{escada; escadaria}
\end{verbete}

\begin{verbete}{路}{lu4}{13}[Radical 足]
  \significado*{s.}{sobrenome Lu}
  \significado[条]{s.}{caminho; estrada; via; jornada; linha (ônibus, etc.); rota}
\end{verbete}

\begin{verbete}{路口}{lu4kou3}{13,3}
  \significado{s.}{cruzamento; interseção (de estradas)}
\end{verbete}

\begin{verbete}{路边}{lu4bian1}{13,5}
  \significado{s.}{meio-fio; acostamento}
\end{verbete}

\begin{verbete}{满}{man3}{13}[Radical 水]
  \significado{adj.}{completo; preenchido; embalado; satisfeito; contente}
  \significado{adv.}{completamente; bastante}
  \significado{v.}{preencher; atingir o limite; satisfazer}
\end{verbete}

\begin{verbete}{满分}{man3fen1}{13,4}
  \significado{s.}{pontuação completa}
\end{verbete}

\begin{verbete}{满满}{man3man3}{13,13}
  \significado{adj.}{completo; densamente empacotado}
\end{verbete}

\begin{verbete}{满意}{man3yi4}{13,13}
  \significado{adj.}{satisfatório}
\end{verbete}

\begin{verbete}{谩骂}{man4ma4}{13,9}
  \significado{v.}{ridicularizar; abusar}
\end{verbete}

\begin{verbete}{蒙面}{meng2mian4}{13,9}
  \significado{adj.}{descarado, desavergonhado, mascarado}
  \significado{v.}{cobrir o rosto, usar uma máscara}
\end{verbete}

\begin{verbete}{幕}{mu4}{13}[Radical 巾]
  \significado{s.}{cortina ou tela; dossel ou tenda; quartel de um general; ato (de uma peça)}
\end{verbete}

\begin{verbete}{暖}{nuan3}{13}[Radical 日]
  \significado{adj.}{quente}
  \significado{v.}{esquentar}
\end{verbete}

\begin{verbete}{暖气}{nuan3qi4}{13,4}
  \significado{s.}{aquecimento central; aquecedor; ar quente}
\end{verbete}

\begin{verbete}{暖和}{nuan3huo5}{13,8}
  \significado{adj.}{morno; agradável e quente}
\end{verbete}

\begin{verbete}{碰头}{peng4tou2}{13,5}
  \significado{s.}{colisão; conflito}
  \significado{v.}{colidir}
  \significado{v.+compl.}{conhecer e discutir; juntar ideias | ver-se}
\end{verbete}

\begin{verbete}{碰运气}{peng4yun4qi5}{13,7,4}
  \significado{v.}{deixar algo ao acaso; tentar a sorte}
\end{verbete}

\begin{verbete}{频道}{pin2dao4}{13,12}
  \significado{s.}{frequência; (televisão) canal}
\end{verbete}

\begin{verbete}{签}{qian1}{13}[Radical 竹]
  \significado{s.}{vara de bambu com inscrição (usada em adivinhação, jogos de azar, sorteios, etc.); rótulo; pequena lasca de madeira; etiqueta}
  \significado{v.}{assinar}
\end{verbete}

\begin{verbete}{签名}{qian1ming2}{13,6}
  \significado{s.}{assinatura}
  \significado{v.+compl.}{autografar; assinar (o nome com uma caneta, etc.)}
\end{verbete}

\begin{verbete}{群山}{qun2shan1}{13,3}
  \significado{s.}{montanhas; uma cadeia de colinas}
\end{verbete}

\begin{verbete}{傻瓜}{sha3gua1}{13,5}
  \significado{adj.}{tolo; burro; simplório; idiota}
  \significado{v.}{enganar; iludir; lograr}
\end{verbete}

\begin{verbete}{傻眼}{sha3yan3}{13,11}
  \significado{adj.}{estupefato; atordoado}
\end{verbete}

\begin{verbete}{摄氏}{she4shi4}{13,4}
  \significado{s.}{graus Celsius (°C), centígrado}
\end{verbete}

\begin{verbete}{数学}{shu4xue2}{13,8}
  \significado{s.}{matemática (disciplina)}
\end{verbete}

\begin{verbete}{睡衣}{shui4yi1}{13,6}
  \significado{s.}{pijamas; roupas de dormir}
\end{verbete}

\begin{verbete}{睡觉}{shui4jiao4}{13,9}
  \significado{v.}{ir para a cama; dormir; deitar-se}
\end{verbete}

\begin{verbete}{睡懒觉}{shui4lan3jiao4}{13,16,9}
  \significado{v.}{levantar-se tarde; passar o tempo a dormir}
\end{verbete}

\begin{verbete}{碎}{sui4}{13}[Radical 石]
  \significado{adj.}{quebrato, fragmentado, espalhado; tagarela}
  \significado{v.}{(transitivo ou intransitivo) quebrar em pedaços, quebrar, desmoronar}
\end{verbete}

\begin{verbete}{滔天}{tao1tian1}{13,4}
  \significado{adj.}{(ondas, raiva, desastres, crimes, etc.) imponente, avassalador, imenso}
\end{verbete}

\begin{verbete}{跳}{tiao4}{13}[Radical 足]
  \significado{v.}{pular; saltar}
\end{verbete}

\begin{verbete}{跳水}{tiao4shui3}{13,4}
  \significado{s.}{mergulho esportivo}
  \significado{v.}{mergulhar (na água); cometer suicídio pulando na água; (fig., preços das ações, etc.) cair dramaticamente}
\end{verbete}

\begin{verbete}{跳电}{tiao4dian4}{13,5}
  \significado{v.}{desarmar (um disjuntor ou interruptor)}
\end{verbete}

\begin{verbete}{跳伞}{tiao4san3}{13,6}
  \significado{s.}{paraquedas}
  \significado{v.}{saltar de paraquedas}
\end{verbete}

\begin{verbete}{跳远}{tiao4yuan3}{13,7}
  \significado{v.+compl.}{salto em distância (atletismo)}
\end{verbete}

\begin{verbete}{跳挡}{tiao4dang3}{13,9}
  \significado{v.}{pular marcha (de um carro); perder a marcha}
\end{verbete}

\begin{verbete}{跳蚤}{tiao4zao5}{13,9}
  \significado{s.}{pulga}
\end{verbete}

\begin{verbete}{跳绳}{tiao4sheng2}{13,11}
  \significado{v.}{pular corda}
\end{verbete}

\begin{verbete}{跳频}{tiao4pin2}{13,13}
  \significado{s.}{FHSS, \emph{Frequency-Hopping Spread Spectrum}, método de transmissão de sinais de rádio}
\end{verbete}

\begin{verbete}{跳跳糖}{tiao4tiao4tang2}{13,13,16}
  \significado{s.}{\emph{Pop Rocks}; \emph{popping candy}}
\end{verbete}

\begin{verbete}{跳舞}{tiao4wu3}{13,14}
  \significado{v.+compl.}{dançar}
\end{verbete}

\begin{verbete}{腿}{tui3}{13}[Radical 肉]
  \significado[条]{s.}{perna; osso do quadril}
\end{verbete}

\begin{verbete}{腿号}{tui3hao4}{13,5}
  \significado{s.}{anilha numerada (por exemplo, usada para identificar pássaros)}
\end{verbete}

\begin{verbete}{腿号箍}{tui3hao4gu1}{13,5,14}
  \veja{腿号}{tui3hao4}
\end{verbete}

\begin{verbete}{碗}{wan3}{13}[Radical 石]
  \significado{clas.}{tigelas}
  \significado[只,个]{s.}{tigela}
\end{verbete}

\begin{verbete}{碗子}{wan3zi5}{13,3}
  \significado{s.}{tigela}
\end{verbete}

\begin{verbete}{碗柜}{wan3gui4}{13,8}
  \significado{s.}{armário}
\end{verbete}

\begin{verbete}{微风}{wei1feng1}{13,4}
  \significado{s.}{brisa, vento leve}
\end{verbete}

\begin{verbete}{微软}{wei1ruan3}{13,8}
  \significado*{s.}{\emph{Microsoft Corporation}}
\end{verbete}

\begin{verbete}{微型}{wei1xing2}{13,9}
  \significado{s.}{miniatura; prefixo micro}
\end{verbete}

\begin{verbete}{嗡嗡}{weng1weng1}{13,13}
  \significado{s.}{zumbido}
  \significado{v.}{zumbir}
\end{verbete}

\begin{verbete}{雾气}{wu4qi4}{13,4}
  \significado{s.}{nevoeiro; névoa; vapor}
\end{verbete}

\begin{verbete}{想}{xiang3}{13}[Radical 心]
  \significado{v./v.o.}{acreditar; sentir falta (sentir-se melancólico com a ausência de alguém ou algo); supor; pensar; querer; desejar}
\end{verbete}

\begin{verbete}{想法}{xiang3fa3}{13,8}
  \significado[个]{s.}{noção; opinião; jeito de pensar}
  \significado{s.}{maneira de pensar, opinião, noção}
  \significado{v.}{pensar em uma maneira (de fazer algo)}
\end{verbete}

\begin{verbete}{想念}{xiang3nian4}{13,8}
  \significado{v.}{perder; sentir falta; lembrar com saudade}
\end{verbete}

\begin{verbete}{想象}{xiang3xiang4}{13,11}
  \significado{v.}{imaginar}
\end{verbete}

\begin{verbete}{想想看}{xiang3xiang3kan4}{13,13,9}
  \significado{v.}{pensar sobre isso}
\end{verbete}

\begin{verbete}{像}{xiang4}{13}[Radical 人]
  \significado{s.}{imagem; retrato; aparência}
  \significado{v.}{assemelhar -se; ser como}
\end{verbete}

\begin{verbete}{新}{xin1}{13}[Radical 斤]
  \significado*{s.}{sobrenome Xin; abreviatura de Xinjiang (新疆); abreviatura de Singapura (新加坡)}
  \significado{adj.}{novo; prefixo meso (química)}
  \significado{adv.}{recentemente}
  \veja{新加坡}{xin1jia1po1}
  \veja{新疆}{xin1jiang1}
\end{verbete}

\begin{verbete}{新加坡}{xin1jia1po1}{13,5,8}
  \significado*{s.}{Singapura}
\end{verbete}

\begin{verbete}{新年}{xin1nian2}{13,6}
  \significado*[个]{s.}{Ano Novo}
\end{verbete}

\begin{verbete}{新闻}{xin1wen2}{13,9}
  \significado[条,个]{s.}{notícia}
\end{verbete}

\begin{verbete}{新娘}{xin1niang2}{13,10}
  \significado{s.}{noiva}
\end{verbete}

\begin{verbete}{新娘子}{xin1niang2zi5}{13,10,3}
  \veja{新娘}{xin1niang2}
\end{verbete}

\begin{verbete}{新娘服装}{xin1niang2 fu2zhuang1}{13,10,8,12}
  \significado{s.}{roupas de noiva}
\end{verbete}

\begin{verbete}{新鲜}{xin1xian1}{13,14}
  \significado{adj.}{fresco (experiência, alimento, etc.)}
  \significado{s.}{frescor}
\end{verbete}

\begin{verbete}{新疆}{xin1jiang1}{13,19}
  \significado*{s.}{Xinjiang}
\end{verbete}

\begin{verbete}{新疆维吾尔自治区}{xin1jiang1 wei2wu2'er3 zi4zhi4qu1}{13,19,11,7,5,6,8,4}
  \significado*{s.}{Região Autônoma Uigur de Xinjiang}
\end{verbete}

\begin{verbete}{腰}{yao1}{13}[Radical 肉]
  \significado{s.}{cintura}
\end{verbete}

\begin{verbete}{腰包}{yao1bao1}{13,5}
  \significado{s.}{pochete; bolso}
\end{verbete}

\begin{verbete}{腰椎}{yao1zhui1}{13,12}
  \significado{s.}{vértebra lombar (espinha dorsal inferior)}
\end{verbete}

\begin{verbete}{摇头}{yao2tou2}{13,5}
  \significado{v.+compl.}{balançar a cabeça de alguém}
\end{verbete}

\begin{verbete}{摇晃}{yao2huang4}{13,10}
  \significado{v.}{sacudir; agitar; balançar; chacoalhar}
\end{verbete}

\begin{verbete}{遥控}{yao2kong4}{13,11}
  \significado{s.}{controle remoto}
  \significado{v.}{dirigir operações de um local remoto, controlar remotamente}
\end{verbete}

\begin{verbete}{颐和园}{yi2he2yuan2}{13,8,7}
  \significado*{s.}{Palácio de Verão}
\end{verbete}

\begin{verbete}{意义}{yi4yi4}{13,3}
  \significado[个]{s.}{importância; significado; senso; ; desejo; força de vontade}
\end{verbete}

\begin{verbete}{意见}{yi4jian4}{13,4}
  \significado[点,条]{s.}{reclamação; ideia; objeção; opinião; sugestão}
\end{verbete}

\begin{verbete}{意外}{yi4wai4}{13,5}
  \significado{adj.}{inesperado}
  \significado{adv.}{acidentalmente}
  \significado[个]{s.}{acidente}
\end{verbete}

\begin{verbete}{意识}{yi4shi2}{13,7}
  \significado{s.}{consciência}
  \significado{v.}{(usualmente seguido de 到) estar ciente, constatar}
\end{verbete}

\begin{verbete}{意译}{yi4yi4}{13,7}
  \significado{s.}{tradução livre; significado (de expressão estrangeira); paráfrase; tradução do significado (em oposição à tradução literal)}
  \veja{直译}{zhi2yi4}
\end{verbete}

\begin{verbete}{意志}{yi4zhi4}{13,7}
  \significado[个]{s.}{determinação; desejo; força de vontade}
\end{verbete}

\begin{verbete}{意思}{yi4si5}{13,9}
  \significado[个]{s.}{interesse}
\end{verbete}

\begin{verbete}{意指}{yi4zhi3}{13,9}
  \significado{v.}{implicar; significar}
\end{verbete}

\begin{verbete}{瑜伽}{yu2jia1}{13,7}
  \significado*{s.}{Ioga}
\end{verbete}

\begin{verbete}{瑜珈}{yu2jia1}{13,9}
  \variante{瑜伽}
\end{verbete}

\begin{verbete}{愈}{yu4}{13}[Radical 心]
  \significado{adv.}{mais e mais; ainda mais}
  \significado{v.}{recuperar; curar}
\end{verbete}

\begin{verbete}{艁}{zao4}{13}
  \variante{造}
\end{verbete}

\begin{verbete}{照}{zhao4}{13}[Radical 火]
  \significado{adv.}{de acordo com; como antes; como pedido; conforme}
  \significado{s.}{foto}
  \significado{v.}{iluminar; olhar (o reflexo de alguém); refletir; brilhar; tirar uma foto}
\end{verbete}

\begin{verbete}{照片}{zhao4pian4}{13,4}
  \significado[张,套,幅]{s.}{fotografia; foto}
\end{verbete}

\begin{verbete}{照片子}{zhao4pian4zi5}{13,4,3}
  \significado{v.}{tirar um raio X}
\end{verbete}

\begin{verbete}{照片底版}{zhao4pian4di3ban3}{13,4,8,8}
  \significado{s.}{placa fotográfica}
\end{verbete}

\begin{verbete}{照亮}{zhao4liang4}{13,9}
  \significado{s.}{iluminação}
  \significado{v.}{iluminar}
\end{verbete}

\begin{verbete}{照相}{zhao4xiang4}{13,9}
  \significado{v.+compl.}{tirar fotografia}
\end{verbete}

\begin{verbete}{照相机}{zhao4xiang4ji1}{13,9,6}
  \significado[个,架,部,台,只]{s.}{câmera/máquina fotográfica}
\end{verbete}

\begin{verbete}{照准}{zhao4zhun3}{13,10}
  \significado{s.}{solicitação concedida (uso formal em documento antigo)}
  \significado{v.}{mirar (arma)}
\end{verbete}

\begin{verbete}{照骗}{zhao4pian4}{13,12}
  \significado{s.}{imagem ``photoshopada''}
\end{verbete}

\begin{verbete}{照像}{zhao4xiang4}{13,13}
  \variante{照相}
\end{verbete}

\begin{verbete}{照像机}{zhao4xiang4ji1}{13,13,6}
  \variante{照相机}
\end{verbete}

\begin{verbete}{置疑}{zhi4yi2}{13,14}
  \significado{v.}{duvidar}
\end{verbete}

\begin{verbete}{罪犯}{zui4fan4}{13,5}
  \significado{s.}{criminoso}
\end{verbete}

\begin{verbete}{罪行}{zui4xing2}{13,6}
  \significado{s.}{crime; ofensa}
\end{verbete}

%%%%% EOF %%%%%


 %%%
%%% 14画
%%%
\section*{14画}\addcontentsline{toc}{section}{14画}\addcontentsline{loh}{figure}{\#\#\#\# 14画}

%%%%%%%%%% 㮸 %%%%%%%%%%
\subsection*{㮸}\addcontentsline{loh}{figure}{㮸}

\begin{Entry}{㮸}{14}{⽊}
  \begin{Phonetics}{㮸}{song4}
    \variantof{送}
  \end{Phonetics}
\end{Entry}

%%%%%%%%%% 僧 %%%%%%%%%%
\subsection*{僧}\addcontentsline{loh}{figure}{僧}

\begin{Entry}{僧}{14}{⼈}
  \begin{Phonetics}{僧}{seng1}
    \definition*{s.}{Sobrenome: Seng}
    \definition[位,名,个]{s.}{monge Budista, abreviação de 僧伽}
  \seealsoref{僧伽}{seng1qie2}
  \end{Phonetics}
\end{Entry}

\begin{Entry}{僧人}{14,2}{⼈,⼈}
  \begin{Phonetics}{僧人}{seng1ren2}[][HSK 7-9]
    \definition{s.}{monge budista | monge}
  \end{Phonetics}
\end{Entry}

\begin{Entry}{僧伽}{14,7}{⼈,⼈}
  \begin{Phonetics}{僧伽}{seng1qie2}
    \definition{s.}{sangha ou sanga (Budismo) | a comunidade monástica | monge}
  \end{Phonetics}
\end{Entry}

%%%%%%%%%% 僮 %%%%%%%%%%
\subsection*{僮}\addcontentsline{loh}{figure}{僮}

\begin{Entry}{僮}{14}{⼈}
  \begin{Phonetics}{僮}{tong2}
    \definition*{s.}{Sobrenome: Tong}
  \end{Phonetics}
  \begin{Phonetics}{僮}{zhuang4}
    \variantof{壮}
  \end{Phonetics}
\end{Entry}

%%%%%%%%%% 兢 %%%%%%%%%%
\subsection*{兢}\addcontentsline{loh}{figure}{兢}

\begin{Entry}{兢}{14}{⼉}
  \begin{Phonetics}{兢}{jing1}
    \definition{adj.}{medroso | cauteloso | forte}
    \definition{v.}{mover}
  \end{Phonetics}
\end{Entry}

\begin{Entry}{兢兢业业}{14,14,5,5}{⼉,⼉,⼀,⼀}
  \begin{Phonetics}{兢兢业业}{jing1jing1ye4ye4}[][HSK 7-9]
    \definition{expr.}{cauteloso e consciencioso; zeloso; aplicado; descreve alguém que é muito cuidadoso, cauteloso e responsável ao fazer as coisas}
  \end{Phonetics}
\end{Entry}

%%%%%%%%%% 凳 %%%%%%%%%%
\subsection*{凳}\addcontentsline{loh}{figure}{凳}

\begin{Entry}{凳}{14}{⼏}
  \begin{Phonetics}{凳}{deng4}
    \definition[条]{s.}{banco; banqueta}
  \end{Phonetics}
\end{Entry}

\begin{Entry}{凳子}{14,3}{⼏,⼦}
  \begin{Phonetics}{凳子}{deng4zi5}[][HSK 7-9]
    \definition[把,条,个]{s.}{banco; banqueta; um móvel que tem pernas para sentar, mas não tem encosto}
  \end{Phonetics}
\end{Entry}

%%%%%%%%%% 嘉 %%%%%%%%%%
\subsection*{嘉}\addcontentsline{loh}{figure}{嘉}

\begin{Entry}{嘉}{14}{⼝}
  \begin{Phonetics}{嘉}{jia1}
    \definition*{s.}{Sobrenome: Jia}
    \definition{adj.}{bom; ótimo | auspicioso | excelente}
    \definition{v.}{elogiar; recomendar}
    \definition{v.}{elogiar}
  \end{Phonetics}
\end{Entry}

\begin{Entry}{嘉年华}{14,6,6}{⼝,⼲,⼗}
  \begin{Phonetics}{嘉年华}{jia1nian2hua2}[][HSK 7-9]
    \definition{s.}{Empréstimo linguístico: carnaval}
  \end{Phonetics}
\end{Entry}

\begin{Entry}{嘉宾}{14,10}{⼝,⼧}
  \begin{Phonetics}{嘉宾}{jia1bin1}[][HSK 6]
    \definition[个,位,名,些]{s.}{convidado}
  \end{Phonetics}
\end{Entry}

%%%%%%%%%% 嘛 %%%%%%%%%%
\subsection*{嘛}\addcontentsline{loh}{figure}{嘛}

\begin{Entry}{嘛}{14}{⼝}
  \begin{Phonetics}{嘛}{ma5}[][HSK 6]
    \definition{part.}{usado no final de uma declaração para expressar que é claro que é verdade que é óbvio | usado no final de uma frase imperativa para expressar expectativa ou dissuasão | usado em uma frase para indicar uma pausa e chamar a atenção da outra pessoa}
  \end{Phonetics}
\end{Entry}

%%%%%%%%%% 境 %%%%%%%%%%
\subsection*{境}\addcontentsline{loh}{figure}{境}

\begin{Entry}{境}{14}{⼟}
  \begin{Phonetics}{境}{jing4}
    \definition{s.}{fronteira; limite | lugar; área; território; região | condição; situação; circunstâncias}
  \end{Phonetics}
\end{Entry}

\begin{Entry}{境内}{14,4}{⼟,⼌}
  \begin{Phonetics}{境内}{jing4nei4}[][HSK 7-9]
    \definition{s.}{área dentro das fronteiras | doméstico | interno (para um país, província, cidade etc.) | dentro das fronteiras}
  \antonymref{境外}{jing4wai4}
  \end{Phonetics}
\end{Entry}

\begin{Entry}{境外}{14,5}{⼟,⼣}
  \begin{Phonetics}{境外}{jing4wai4}[][HSK 7-9]
    \definition{s.}{área fora das fronteiras (ou do território) de um país; além das fronteiras de um país ou região}
  \end{Phonetics}
\end{Entry}

\begin{Entry}{境地}{14,6}{⼟,⼟}
  \begin{Phonetics}{境地}{jing4di4}[][HSK 7-9]
    \definition{s.}{situação; circunstâncias; as circunstâncias ou a situação encontradas (geralmente usadas em um sentido negativo) | reino; estado}
  \end{Phonetics}
\end{Entry}

\begin{Entry}{境界}{14,9}{⼟,⽥}
  \begin{Phonetics}{境界}{jing4jie4}[][HSK 7-9]
    \definition{s.}{limite; limites de terra | estado; nível; extensão alcançada; o grau em que algo é alcançado ou o estado em que se manifesta}
  \end{Phonetics}
\end{Entry}

\begin{Entry}{境遇}{14,12}{⼟,⾡}
  \begin{Phonetics}{境遇}{jing4yu4}[][HSK 7-9]
    \definition{s.}{a sorte de alguém; as circunstâncias; circunstâncias e encontros}
  \end{Phonetics}
\end{Entry}

%%%%%%%%%% 墙 %%%%%%%%%%
\subsection*{墙}\addcontentsline{loh}{figure}{墙}

\begin{Entry}{墙}{14}{⼟}
  \begin{Phonetics}{墙}{qiang2}[][HSK 2]
    \definition[面,堵,道]{s.}{parede; barreira ou perímetro construído com tijolos, pedras, etc. | qualquer coisa com a forma ou função de uma parede; a parte de um objeto que funciona como parede ou divisória}
    \definition{v.}{(gíria) bloquear (um website) (usado geralmente na voz passiva: 被墙)}
  \end{Phonetics}
\end{Entry}

\begin{Entry}{墙纸}{14,7}{⼟,⽷}
  \begin{Phonetics}{墙纸}{qiang2zhi3}
    \definition{s.}{papel de parede}
  \end{Phonetics}
\end{Entry}

\begin{Entry}{墙壁}{14,16}{⼟,⼟}
  \begin{Phonetics}{墙壁}{qiang2bi4}[][HSK 5]
    \definition[面,堵,道]{s.}{parede; barreira ou perímetro construído com tijolos, pedras ou terra}
  \end{Phonetics}
\end{Entry}

%%%%%%%%%% 墬 %%%%%%%%%%
\subsection*{墬}\addcontentsline{loh}{figure}{墬}

\begin{Entry}{墬}{14}{⼟}
  \begin{Phonetics}{墬}{di4}
    \variantof{地}
  \end{Phonetics}
\end{Entry}

%%%%%%%%%% 嫚 %%%%%%%%%%
\subsection*{嫚}\addcontentsline{loh}{figure}{嫚}

\begin{Entry}{嫚}{14}{⼥}
  \begin{Phonetics}{嫚}{man1}
    \definition{s.}{menina bem-comportada}
  \seealsoref{嫚子}{man1zi5}
  \end{Phonetics}
  \begin{Phonetics}{嫚}{man4}
    \definition*{s.}{Sobrenome: Man}
    \definition{s.}{Dialeto: menina}
    \definition{v.}{Literário: desprezar; menosprezar; insultar; humilhar}
  \end{Phonetics}
\end{Entry}

\begin{Entry}{嫚子}{14,3}{⼥,⼦}
  \begin{Phonetics}{嫚子}{man1zi5}
    \definition{s.}{Dialeto: menina}
  \end{Phonetics}
\end{Entry}

\begin{Entry}{嫚骂}{14,9}{⼥,⾺}
  \begin{Phonetics}{嫚骂}{man4ma4}
    \definition{s.}{insultar; repreender; xingar}
  \end{Phonetics}
\end{Entry}

%%%%%%%%%% 嫦 %%%%%%%%%%
\subsection*{嫦}\addcontentsline{loh}{figure}{嫦}

\begin{Entry}{嫦}{14}{⼥}
  \begin{Phonetics}{嫦}{chang2}
    \definition{s.}{uma beleza lendária que voou para a lua | a dama da lua}
  \end{Phonetics}
\end{Entry}

\begin{Entry}{嫦娥}{14,10}{⼥,⼥}
  \begin{Phonetics}{嫦娥}{chang2'e2}[][HSK 7-9]
    \definition*{s.}{Chang'e, a dama da lua (mitologia chinesa); uma fada que voou do mundo humano para o Palácio da Lua na mitologia}
  \end{Phonetics}
\end{Entry}

%%%%%%%%%% 嫩 %%%%%%%%%%
\subsection*{嫩}\addcontentsline{loh}{figure}{嫩}

\begin{Entry}{嫩}{14}{⼥}
  \begin{Phonetics}{嫩}{nen4}[][HSK 7-9]
    \definition{adj.}{terno; delicado; recém-nascido e frágil | macio; malpassado; alguns alimentos são rápidos de cozinhar e fáceis de mastigar | claro; suave; algumas cores são claras | sem habilidade; inexperiente}
  \end{Phonetics}
\end{Entry}

%%%%%%%%%% 孵 %%%%%%%%%%
\subsection*{孵}\addcontentsline{loh}{figure}{孵}

\begin{Entry}{孵}{14}{⼦}
  \begin{Phonetics}{孵}{fu1}
    \definition{v.}{chocar; incubar; (pássaros) sentar em ovos}
  \end{Phonetics}
\end{Entry}

\begin{Entry}{孵化}{14,4}{⼦,⼔}
  \begin{Phonetics}{孵化}{fu1hua4}[][HSK 7-9]
    \definition{v.}{chocar; incubar | incubar; metaforicamente, cultivar e desenvolver coisas novas (agora se refere principalmente ao suporte a empresas de alta tecnologia recém-criadas)}
  \end{Phonetics}
\end{Entry}

%%%%%%%%%% 察 %%%%%%%%%%
\subsection*{察}\addcontentsline{loh}{figure}{察}

\begin{Entry}{察}{14}{⼧}
  \begin{Phonetics}{察}{cha2}
    \definition*{s.}{Sobrenome: Cha}
    \definition{v.}{examinar; investigar; escrutinar | observar; olhar atentamente; investigar}
  \end{Phonetics}
\end{Entry}

\begin{Entry}{察看}{14,9}{⼧,⽬}
  \begin{Phonetics}{察看}{cha2kan4}[][HSK 7-9]
    \definition{v.}{observar; olhar atentamente; inspecionar}
  \end{Phonetics}
\end{Entry}

\begin{Entry}{察觉}{14,9}{⼧,⾒}
  \begin{Phonetics}{察觉}{cha2jue2}[][HSK 7-9]
    \definition{v.}{detectar; perceber; estar ciente de; estar consciente de; descobrir; ver}
  \end{Phonetics}
\end{Entry}

%%%%%%%%%% 寡 %%%%%%%%%%
\subsection*{寡}\addcontentsline{loh}{figure}{寡}

\begin{Entry}{寡}{14}{⼧}
  \begin{Phonetics}{寡}{gua3}
    \definition{adj.}{poucos; escassos | insípido; sem sabor | pouco; escasso | insípido; sem graça}
    \definition{pron.}{eu; título autoproclamado de um antigo monarca}
    \definition{s.}{viúva | viuvez; a natureza ou estado de uma mulher viúva que vive sozinha}
  \antonymref{多}{duo1}
  \antonymref{众}{zhong4}
  \end{Phonetics}
\end{Entry}

\begin{Entry}{寡妇}{14,6}{⼧,⼥}
  \begin{Phonetics}{寡妇}{gua3fu5}[][HSK 7-9]
    \definition[个]{s.}{viúva; uma mulher cujo marido morreu}
  \end{Phonetics}
\end{Entry}

%%%%%%%%%% 寥 %%%%%%%%%%
\subsection*{寥}\addcontentsline{loh}{figure}{寥}

\begin{Entry}{寥}{14}{⼧}
  \begin{Phonetics}{寥}{liao2}
    \definition{adj.}{pouco; escasso | silencioso; deserto | abstruso; vago; amplo e vazio; desolado}
  \end{Phonetics}
\end{Entry}

\begin{Entry}{寥寥无几}{14,14,4,2}{⼧,⼧,⽆,⼏}
  \begin{Phonetics}{寥寥无几}{liao2liao2-wu2ji3}[][HSK 7-9]
    \definition{expr.}{poucos restantes; escasso; esparso; muito poucos; uma quantidade insignificante}
  \end{Phonetics}
\end{Entry}

%%%%%%%%%% 寨 %%%%%%%%%%
\subsection*{寨}\addcontentsline{loh}{figure}{寨}

\begin{Entry}{寨}{14}{⼧}
  \begin{Phonetics}{寨}{zhai4}
    \definition{s.}{fortaleza | paliçada | acampamento | vila (paliçada)}
  \end{Phonetics}
\end{Entry}

%%%%%%%%%% 弊 %%%%%%%%%%
\subsection*{弊}\addcontentsline{loh}{figure}{弊}

\begin{Entry}{弊}{14}{⼶}
  \begin{Phonetics}{弊}{bi4}
    \definition{s.}{fraude; abuso; negligência médica | desvantagem; falha; defeito; dano | trapaça; fraude, engano e falsificação}
  \antonymref{利}{li4}
  \end{Phonetics}
\end{Entry}

\begin{Entry}{弊病}{14,10}{⼶,⽧}
  \begin{Phonetics}{弊病}{bi4bing4}[][HSK 7-9]
    \definition{s.}{doença; mal; negligência | incinveniente; desvantagem; problemas com coisas}
  \end{Phonetics}
\end{Entry}

\begin{Entry}{弊端}{14,14}{⼶,⽴}
  \begin{Phonetics}{弊端}{bi4duan1}[][HSK 7-9]
    \definition{s.}{abuso; negligência; prática corrupta; danos ao interesse público devido a uma lacuna no trabalho}
  \end{Phonetics}
\end{Entry}

%%%%%%%%%% 愿 %%%%%%%%%%
\subsection*{愿}\addcontentsline{loh}{figure}{愿}

\begin{Entry}{愿}{14}{⽕}
  \begin{Phonetics}{愿}{yuan4}[][HSK 5]
    \definition{adj.}{honesto e prudente}
    \definition{s.}{esperança; desejo; vontade; a ideia de alcançar algum objetivo no futuro | voto (feito perante o Buda ou um deus); o desejo de retribuição feito ao rezar para os deuses e Buda}
    \definition{v.}{estar disposto; estar pronto; de bom grado, concordar porque está de acordo com seus desejos | ter esperança; desejar; qerer alcançar algum desejo}
  \end{Phonetics}
\end{Entry}

\begin{Entry}{愿望}{14,11}{⽕,⽉}
  \begin{Phonetics}{愿望}{yuan4wang4}[][HSK 3]
    \definition[个,种]{s.}{desejo; aspiração; a ideia de alcançar algum objetivo no futuro.}
  \end{Phonetics}
\end{Entry}

\begin{Entry}{愿意}{14,13}{⽕,⼼}
  \begin{Phonetics}{愿意}{yuan4yi5}[][HSK 2]
    \definition{v.}{estar disposto; estar pronto | desejar; ter esperança}
  \end{Phonetics}
\end{Entry}

%%%%%%%%%% 慢 %%%%%%%%%%
\subsection*{慢}\addcontentsline{loh}{figure}{慢}

\begin{Entry}{慢}{14}{⼼}
  \begin{Phonetics}{慢}{man4}[][HSK 1]
    \definition*{s.}{Sobrenome: Man}
    \definition{adj.}{lento; devagar; baixa velocidade; longa duração | rude; arrogante; sem educação com as pessoas | frouxo; lento}
    \definition{adv.}{lentamente}
  \antonymref{快}{kuai4}
  \end{Phonetics}
\end{Entry}

\begin{Entry}{慢车}{14,4}{⼼,⾞}
  \begin{Phonetics}{慢车}{man4che1}[][HSK 6]
    \definition{s.}{trem lento com muitas paradas | ônibus ou trem local; parada do trem}
  \antonymref{快车}{kuai4che1}
  \end{Phonetics}
\end{Entry}

\begin{Entry}{慢动作}{14,6,7}{⼼,⼒,⼈}
  \begin{Phonetics}{慢动作}{man4dong4zuo4}
    \definition{s.}{(cinema) câmera lenta}
  \end{Phonetics}
\end{Entry}

\begin{Entry}{慢性}{14,8}{⼼,⼼}
  \begin{Phonetics}{慢性}{man4xing4}[][HSK 7-9]
    \definition{adj.}{crônico; duradouro | lento (em fazer efeito)}
  \end{Phonetics}
\end{Entry}

\begin{Entry}{慢慢}{14,14}{⼼,⼼}
  \begin{Phonetics}{慢慢}{man4man4}[][HSK 3]
    \definition{adv.}{lentamente; vagarosamente; gradualmente | lentamente; vagarosamente; gradualmente; depois de um longo período de tempo}
  \end{Phonetics}
\end{Entry}

\begin{Entry}{慢慢来}{14,14,7}{⼼,⼼,⽊}
  \begin{Phonetics}{慢慢来}{man4man4 lai2}[][HSK 7-9]
    \definition{v.}{ir com calma; não ter pressa; significa não ter impaciência ao fazer as coisas e prosseguir no seu próprio ritmo}
  \end{Phonetics}
\end{Entry}

%%%%%%%%%% 慷 %%%%%%%%%%
\subsection*{慷}\addcontentsline{loh}{figure}{慷}

\begin{Entry}{慷}{14}{⼼}
  \begin{Phonetics}{慷}{kang1}
    \definition{adj.}{generoso | magnânimo}
  \end{Phonetics}
\end{Entry}

\begin{Entry}{慷慨}{14,12}{⼼,⼼}
  \begin{Phonetics}{慷慨}{kang1kai3}[][HSK 7-9]
    \definition{adj.}{generoso; descreve alguém como alguém que não é mesquinho; disposto a ajudar os outros com dinheiro ou bens | veemente; fervoroso; apaixonado; descreve alguém como alguém repleto de senso de justiça e emocionalmente intenso}
  \end{Phonetics}
\end{Entry}

%%%%%%%%%% 截 %%%%%%%%%%
\subsection*{截}\addcontentsline{loh}{figure}{截}

\begin{Entry}{截}{14}{⼽}
  \begin{Phonetics}{截}{jie2}[][HSK 7-9]
    \definition{clas.}{seção; pedaço; comprimento}
    \definition{prep.}{por (um tempo especificado); até}
    \definition{v.}{cortar; romper | parar; verificar; interromper; interceptar}
  \end{Phonetics}
\end{Entry}

\begin{Entry}{截止}{14,4}{⼽,⽌}
  \begin{Phonetics}{截止}{jie2zhi3}[][HSK 6]
    \definition{adv.}{até (um certo limite de tempo); por (um tempo especificado)}
  \end{Phonetics}
\end{Entry}

\begin{Entry}{截至}{14,6}{⼽,⾄}
  \begin{Phonetics}{截至}{jie2zhi4}[][HSK 6]
    \definition{adv.}{a partir de; até (um certo limite de tempo); por (um tempo especificado)}
  \end{Phonetics}
\end{Entry}

\begin{Entry}{截然不同}{14,12,4,6}{⼽,⽕,⼀,⼝}
  \begin{Phonetics}{截然不同}{jie2ran2-bu4tong2}[][HSK 7-9]
    \definition{expr.}{``Completamente diferente.''; tão diferente quanto preto e branco; polos opostos}
  \end{Phonetics}
\end{Entry}

%%%%%%%%%% 摔 %%%%%%%%%%
\subsection*{摔}\addcontentsline{loh}{figure}{摔}

\begin{Entry}{摔}{14}{⼿}
  \begin{Phonetics}{摔}{shuai1}[][HSK 5]
    \definition{v.}{cair; tropeçar; perder o equilíbrio | mergulhar; precipitar-se; cair de uma altura elevada | quebrar; fazer cair e quebrar | lançar; atirar; arremessar; joguar coisas com força e para baixo | bater; golpear; bater com força para que o que está grudado cair}
  \end{Phonetics}
\end{Entry}

\begin{Entry}{摔倒}{14,10}{⼿,⼈}
  \begin{Phonetics}{摔倒}{shuai1dao3}[][HSK 5]
    \definition{v.}{cair; tropeçar; perder o equilíbrio e cair}
  \end{Phonetics}
\end{Entry}

%%%%%%%%%% 摘 %%%%%%%%%%
\subsection*{摘}\addcontentsline{loh}{figure}{摘}

\begin{Entry}{摘}{14}{⼿}
  \begin{Phonetics}{摘}{zhai1}[][HSK 5]
    \definition{v.}{pegar; arrancar; tirar; colher (flores, frutos, folhas de plantas); retirar (coisas que estão sendo usadas ou penduradas) | selecionar; fazer extrações de | pedir dinheiro emprestado em caso de necessidade urgente | vencer; ganhar; alcançar; obter}
  \end{Phonetics}
\end{Entry}

%%%%%%%%%% 摧 %%%%%%%%%%
\subsection*{摧}\addcontentsline{loh}{figure}{摧}

\begin{Entry}{摧}{14}{⼿}
  \begin{Phonetics}{摧}{cui1}
    \definition{v.}{quebrar; destruir}
  \end{Phonetics}
\end{Entry}

\begin{Entry}{摧毁}{14,13}{⼿,⽎}
  \begin{Phonetics}{摧毁}{cui1hui3}[][HSK 7-9]
    \definition{v.}{destruir; esmagar; nocautear; destruir com grande força}
  \end{Phonetics}
\end{Entry}

%%%%%%%%%% 撇 %%%%%%%%%%
\subsection*{撇}\addcontentsline{loh}{figure}{撇}

\begin{Entry}{撇}{14}{⼿}
  \begin{Phonetics}{撇}{pie1}
    \definition{v.}{descartar; jogar ao mar; abandonar | desnatar; retirar delicadamente o líquido da superfície}
  \end{Phonetics}
  \begin{Phonetics}{撇}{pie3}[][HSK 7-9]
    \definition{clas.}{utilizado para coisas sobrancelhas e barbas}
    \definition{s.}{traço descendente à esquerda 丿(em caracteres chineses)}
    \definition{v.}{atirar; arremessar; lançar}
  \end{Phonetics}
\end{Entry}

%%%%%%%%%% 敲 %%%%%%%%%%
\subsection*{敲}\addcontentsline{loh}{figure}{敲}

\begin{Entry}{敲}{14}{⽁}
  \begin{Phonetics}{敲}{qiao1}[][HSK 5]
    \definition{v.}{bater; dar uma pancada; golpear | explorar alguém; cobrar a mais; extorquir; chantagear | lembrar; criticar; alertar; advertir}
  \end{Phonetics}
\end{Entry}

\begin{Entry}{敲门}{14,3}{⽁,⾨}
  \begin{Phonetics}{敲门}{qiao1 men2}[][HSK 5]
    \definition{v.}{bater na porta}
  \end{Phonetics}
\end{Entry}

\begin{Entry}{敲边鼓}{14,5,13}{⽁,⾡,⿎}
  \begin{Phonetics}{敲边鼓}{qiao1 bian1gu3}[][HSK 7-9]
    \definition{v.}{``Tocar trompa.'' | Coloquial: falar ou agir para ajudar alguém à margem; apoiar alguém; apoiar alguém em uma discussão}
  \end{Phonetics}
\end{Entry}

\begin{Entry}{敲诈}{14,7}{⽁,⾔}
  \begin{Phonetics}{敲诈}{qiao1zha4}[][HSK 7-9]
    \definition{v.}{extorquir; chantagear; usar o poder, a intimidação e as ameaças para extorquir dinheiro}
  \end{Phonetics}
\end{Entry}

%%%%%%%%%% 斡 %%%%%%%%%%
\subsection*{斡}\addcontentsline{loh}{figure}{斡}

\begin{Entry}{斡}{14}{⽃}
  \begin{Phonetics}{斡}{wo4}
    \definition{v.}{virar-se}
  \end{Phonetics}
\end{Entry}

\begin{Entry}{斡旋}{14,11}{⽃,⽅}
  \begin{Phonetics}{斡旋}{wo4xuan2}
    \definition{v.}{mediar (um conflito, etc.)}
  \end{Phonetics}
\end{Entry}

%%%%%%%%%% 旗 %%%%%%%%%%
\subsection*{旗}\addcontentsline{loh}{figure}{旗}

\begin{Entry}{旗}{14}{⽅}
  \begin{Phonetics}{旗}{qi2}
    \definition[面]{s.}{bandeira}
  \end{Phonetics}
\end{Entry}

\begin{Entry}{旗帜}{14,8}{⽅,⼱}
  \begin{Phonetics}{旗帜}{qi2zhi4}[][HSK 7-9]
    \definition[面]{s.}{bandeira; estandarte | modelo; bom exemplo; metáfora para modelo ou exemplo a seguir | bandeira (de um pensamento representativo ou posição política); essa metáfora se refere a uma ideologia, doutrina ou força política representativa ou influente}
  \end{Phonetics}
\end{Entry}

\begin{Entry}{旗袍}{14,10}{⽅,⾐}
  \begin{Phonetics}{旗袍}{qi2pao2}[][HSK 7-9]
    \definition[件,个]{s.}{qipao; cheongsam; uma túnica longa usada por mulheres, originalmente usada por mulheres manchus}
  \end{Phonetics}
\end{Entry}

%%%%%%%%%% 榜 %%%%%%%%%%
\subsection*{榜}\addcontentsline{loh}{figure}{榜}

\begin{Entry}{榜}{14}{⽊}
  \begin{Phonetics}{榜}{bang3}
    \definition[块]{s.}{lista publicada de nomes | Literário: placa horizontal inscrita | aviso; anúncio; proclamação antiga}
  \end{Phonetics}
\end{Entry}

\begin{Entry}{榜首}{14,9}{⽊,⾸}
  \begin{Phonetics}{榜首}{bang3shou3}
    \definition{s.}{cabeça da lista de candidatos aprovados; primeiro lugar em um concurso, etc. | topo da lista}
  \end{Phonetics}
\end{Entry}

\begin{Entry}{榜样}{14,10}{⽊,⽊}
  \begin{Phonetics}{榜样}{bang3yang4}[][HSK 7-9]
    \definition[个,位]{s.}{exemplo; modelo; padrão; pessoas ou coisas boas que valem a pena aprender, usado principalmente na linguagem falada}
  \end{Phonetics}
\end{Entry}

%%%%%%%%%% 槃 %%%%%%%%%%
\subsection*{槃}\addcontentsline{loh}{figure}{槃}

\begin{Entry}{槃}{14}{⽊}
  \begin{Phonetics}{槃}{pan2}
    \variantof{盘}
  \end{Phonetics}
\end{Entry}

%%%%%%%%%% 模 %%%%%%%%%%
\subsection*{模}\addcontentsline{loh}{figure}{模}

\begin{Entry}{模}{14}{⽊}
  \begin{Phonetics}{模}{mo2}
    \definition{s.}{padrão | modelo; exemplo | modelo (pessoa) | exame simulado | módulo}
    \definition{v.}{imitar | copiar; emular}
  \end{Phonetics}
  \begin{Phonetics}{模}{mu2}
    \definition*{s.}{Sobrenome: Mu}
    \definition{s.}{molde; padrão; matriz}
  \end{Phonetics}
\end{Entry}

\begin{Entry}{模仿}{14,6}{⽊,⼈}
  \begin{Phonetics}{模仿}{mo2fang3}[][HSK 5]
    \definition{v.}{copiar; imitar; aprender a fazer algo seguindo um modelo pronto}
  \end{Phonetics}
\end{Entry}

\begin{Entry}{模式}{14,6}{⽊,⼷}
  \begin{Phonetics}{模式}{mo2shi4}[][HSK 5]
    \definition{s.}{modelo; modo; padrão; a forma padrão de algo ou o modelo padrão que as pessoas podem seguir}
  \end{Phonetics}
\end{Entry}

\begin{Entry}{模拟}{14,7}{⽊,⼿}
  \begin{Phonetics}{模拟}{mo2ni3}[][HSK 7-9]
    \definition{v.}{ser análogo; imitar; simular; fazer de maneira formal}
  \end{Phonetics}
\end{Entry}

\begin{Entry}{模具}{14,8}{⽊,⼋}
  \begin{Phonetics}{模具}{mu2ju4}
    \definition{s.}{molde | matriz | padrão}
  \end{Phonetics}
\end{Entry}

\begin{Entry}{模型}{14,9}{⽊,⼟}
  \begin{Phonetics}{模型}{mo2xing2}[][HSK 4]
    \definition[个]{s.}{modelo; padrão; itens feitos em escala com base em objetos ou desenhos | molde; padrão; molde para fundir máquinas, objetos, etc.}
  \end{Phonetics}
\end{Entry}

\begin{Entry}{模范}{14,9}{⽊,⾋}
  \begin{Phonetics}{模范}{mo2fan4}[][HSK 5]
    \definition{adj.}{exemplar}
    \definition{s.}{modelo; exemplo excelente; pessoa exemplar; coisa exemplar; pessoas ou coisas exemplares que servem de modelo}
  \end{Phonetics}
\end{Entry}

\begin{Entry}{模样}{14,10}{⽊,⽊}
  \begin{Phonetics}{模样}{mu2yang4}[][HSK 5]
    \definition[副,种]{s.}{aparência; a aparência ou o estilo de vestir de uma pessoa | indicando uma estimativa aproximada de tempo ou idade; expressão de estimativas relativas a tempo, idade, etc. | tendência; situação; inclinação}
  \end{Phonetics}
\end{Entry}

\begin{Entry}{模特儿}{14,10,2}{⽊,⽜,⼉}
  \begin{Phonetics}{模特儿}{mo2te4r5}[][HSK 4]
    \definition[个,名,位]{s.}{modelo (pessoa que posa para um fotógrafo ou pintor ou escultor); objeto de representação ou referência usado por artistas para esboços e esculturas, como o corpo humano, objetos, modelos etc.; também se refere aos arquétipos que os estudiosos da literatura usam para retratar seus personagens | modelo (uma pessoa que usa roupas para exibir modas); pessoa ou manequim usado para exibir estilos de roupas}
  \end{Phonetics}
\end{Entry}

\begin{Entry}{模糊}{14,15}{⽊,⽶}
  \begin{Phonetics}{模糊}{mo2hu5}[][HSK 5]
    \definition{adj.}{vago; confuso; indistinto}
    \definition{v.}{confundir; desorientar}
  \end{Phonetics}
\end{Entry}

%%%%%%%%%% 歉 %%%%%%%%%%
\subsection*{歉}\addcontentsline{loh}{figure}{歉}

\begin{Entry}{歉}{14}{⽋}
  \begin{Phonetics}{歉}{qian4}
    \definition{adj.}{pobre, ruim (colheita) ruim; baixa produtividade (agrícola)}
    \definition{s.}{pedido de desculpas; apologia | quebra de safra}
    \definition{v.}{pedir desculpas; sentir pena dos outros}
  \end{Phonetics}
\end{Entry}

\begin{Entry}{歉意}{14,13}{⽋,⼼}
  \begin{Phonetics}{歉意}{qian4yi4}[][HSK 7-9]
    \definition{s.}{desculpas; arrependimento; pedido de desculpas}
  \end{Phonetics}
\end{Entry}

%%%%%%%%%% 歌 %%%%%%%%%%
\subsection*{歌}\addcontentsline{loh}{figure}{歌}

\begin{Entry}{歌}{14}{⽋}
  \begin{Phonetics}{歌}{ge1}[][HSK 1]
    \definition[首,支,段]{s.}{canção; poesia cantável}
    \definition{v.}{cantar; entoar | louvar; exaltar; cantar louvores a}
  \end{Phonetics}
\end{Entry}

\begin{Entry}{歌手}{14,4}{⽋,⼿}
  \begin{Phonetics}{歌手}{ge1shou3}[][HSK 3]
    \definition[个,位,名]{s.}{cantor; vocalista; pessoa com talento para cantar}
  \end{Phonetics}
\end{Entry}

\begin{Entry}{歌曲}{14,6}{⽋,⽈}
  \begin{Phonetics}{歌曲}{ge1qu3}[][HSK 5]
    \definition[首,支]{s.}{música; obra para as pessoas cantarem, uma combinação de poesia e música}
  \end{Phonetics}
\end{Entry}

\begin{Entry}{歌声}{14,7}{⽋,⼠}
  \begin{Phonetics}{歌声}{ge1sheng1}[][HSK 3]
    \definition{s.}{canto; voz cantada; som do canto}
  \end{Phonetics}
\end{Entry}

\begin{Entry}{歌词}{14,7}{⽋,⾔}
  \begin{Phonetics}{歌词}{ge1ci2}[][HSK 6]
    \definition{s.}{letra da música; libreto}
  \end{Phonetics}
\end{Entry}

\begin{Entry}{歌咏}{14,8}{⽋,⼝}
  \begin{Phonetics}{歌咏}{ge1yong3}[][HSK 7-9]
    \definition{v.}{cantar; cantar canções}
  \end{Phonetics}
\end{Entry}

\begin{Entry}{歌星}{14,9}{⽋,⽇}
  \begin{Phonetics}{歌星}{ge1xing1}[][HSK 6]
    \definition[位,名]{s.}{cantor famoso; estrela da música}
  \end{Phonetics}
\end{Entry}

\begin{Entry}{歌迷}{14,9}{⽋,⾡}
  \begin{Phonetics}{歌迷}{ge1mi2}[][HSK 3]
    \definition{s.}{fã de um cantor; pessoas que gostam de ouvir música ou cantar e ficam fascinadas por isso}
  \end{Phonetics}
\end{Entry}

\begin{Entry}{歌剧}{14,10}{⽋,⼑}
  \begin{Phonetics}{歌剧}{ge1ju4}[][HSK 7-9]
    \definition[场,出]{s.}{ópera | ópera ocidental; um drama que integra poesia, música, dança e outras artes, tendo o canto como principal característica}
  \end{Phonetics}
\end{Entry}

\begin{Entry}{歌颂}{14,10}{⽋,⾴}
  \begin{Phonetics}{歌颂}{ge1song4}[][HSK 7-9]
    \definition{v.}{cantar louvores de; exaltar; elogiar; elogio com poesia, geralmente se refere a elogiar com palavras, etc.}
  \end{Phonetics}
\end{Entry}

\begin{Entry}{歌唱}{14,11}{⽋,⼝}
  \begin{Phonetics}{歌唱}{ge1chang4}[][HSK 6]
    \definition{v.}{cantar | cantar em louvor de; louvor através de cânticos, recitações, etc.}
  \end{Phonetics}
\end{Entry}

\begin{Entry}{歌舞}{14,14}{⽋,⾇}
  \begin{Phonetics}{歌舞}{ge1wu3}[][HSK 7-9]
    \definition{s.}{canto e dança}
  \end{Phonetics}
\end{Entry}

%%%%%%%%%% 滴 %%%%%%%%%%
\subsection*{滴}\addcontentsline{loh}{figure}{滴}

\begin{Entry}{滴}{14}{⽔}
  \begin{Phonetics}{滴}{di1}[][HSK 6]
    \definition{clas.}{gota; quantificador para ``gotejamento''}
    \definition{s.}{uma gota}
    \definition{v.}{pingar}
  \end{Phonetics}
\end{Entry}

%%%%%%%%%% 漂 %%%%%%%%%%
\subsection*{漂}\addcontentsline{loh}{figure}{漂}

\begin{Entry}{漂}{14}{⽔}
  \begin{Phonetics}{漂}{piao1}[][HSK 7-9]
    \definition{v.}{flutuar; derivar; flutuar na superfície de um líquido; flutuar na superfície da água e mover-se com a corrente; mover-se com o vento}
  \end{Phonetics}
  \begin{Phonetics}{漂}{piao3}
    \definition{v.}{branquear (com água sanitária) | enxaguar; enxaguar com água para remover as impurezas}
  \end{Phonetics}
  \begin{Phonetics}{漂}{piao4}
    \definition{adj.}{bonita; usado em 漂亮}
    \definition{v.}{falhar; terminar em fracasso}[这笔投资的钱全都漂了。===Todo o dinheiro desse investimento foi perdido.]
  \seealsoref{漂亮}{piao4liang5}
  \end{Phonetics}
\end{Entry}

\begin{Entry}{漂亮}{14,9}{⽔,⼇}
  \begin{Phonetics}{漂亮}{piao4liang5}[][HSK 2]
    \definition{adj.}{bonito; lindo; atraente; de boa aparência; esteticamente agradável | excelente; notável | não pode ser utilizado para descrever homens}
  \end{Phonetics}
\end{Entry}

\begin{Entry}{漂流}{14,10}{⽔,⽔}
  \begin{Phonetics}{漂流}{piao1liu2}
    \definition{s.}{\emph{rafting}}
    \definition{v.}{derivar; flutuar; flutuar na água e à deriva com a corrente | deixar-se levar; ter vida errante; estar à deriva | praticar rafting em um pequeno barco ou jangada inflável, geralmente como atividade recreativa; descer as corredeiras em pequenos barcos ou jangadas é hoje considerado, em grande parte, uma atividade de recreação aquática}
  \end{Phonetics}
\end{Entry}

%%%%%%%%%% 漆 %%%%%%%%%%
\subsection*{漆}\addcontentsline{loh}{figure}{漆}

\begin{Entry}{漆}{14}{⽔}
  \begin{Phonetics}{漆}{qi1}[][HSK 7-9]
    \definition*{s.}{Sobrenome: Qi}
    \definition[桶,种,层]{s.}{laca; tinta}
    \definition{v.}{aplicar verniz; pintar}
  \end{Phonetics}
\end{Entry}

%%%%%%%%%% 漏 %%%%%%%%%%
\subsection*{漏}\addcontentsline{loh}{figure}{漏}

\begin{Entry}{漏}{14}{⽔}
  \begin{Phonetics}{漏}{lou4}[][HSK 5]
    \definition{s.}{relógio de água; ampulheta | falha; ponto fraco | gonorreia; a medicina tradicional chinesa refere"-se a certas doenças que causam secreção de pus, sangue e muco | unidade de tempo medida por um relógio de água durante a noite}
    \definition{v.}{(líquido, gás, etc.) pingar; vazar; escorrer; cair (de um buraco ou fenda) | vazar; deixar escapar; divulgar | perder; deixar de fora por engano | vazar; o objeto tem poros e pode vazar coisas | há uma fuga de ar}
  \end{Phonetics}
\end{Entry}

\begin{Entry}{漏电}{14,5}{⽔,⽥}
  \begin{Phonetics}{漏电}{lou4dian4}
    \definition{v.}{vazar eletricidade}
  \end{Phonetics}
\end{Entry}

\begin{Entry}{漏洞}{14,9}{⽔,⽔}
  \begin{Phonetics}{漏洞}{lou4dong4}[][HSK 5]
    \definition[个,点]{s.}{vazamento; rachadura; lacunas ou buracos desnecessários que permitem que coisas vazem | falha; defeito; lacuna; (fala, ação, método, etc.) imperfeições}
  \end{Phonetics}
\end{Entry}

%%%%%%%%%% 演 %%%%%%%%%%
\subsection*{演}\addcontentsline{loh}{figure}{演}

\begin{Entry}{演}{14}{⽔}
  \begin{Phonetics}{演}{yan3}[][HSK 3]
    \definition{v.}{desenvolver; evoluir | deduzir; elaborar | exercitar; praticar | representar; atuar; encenar | desempenhar}
  \end{Phonetics}
\end{Entry}

\begin{Entry}{演出}{14,5}{⽔,⼐}
  \begin{Phonetics}{演出}{yan3chu1}[][HSK 3]
    \definition[场,次]{s.}{show; concerto; performance}
    \definition{v.}{apresentar; representar; fazer um show; apresentar peças teatrais, danças, artes cênicas, acrobacias, etc. para o público apreciar}
  \end{Phonetics}
\end{Entry}

\begin{Entry}{演讲}{14,6}{⽔,⾔}
  \begin{Phonetics}{演讲}{yan3jiang3}[][HSK 4]
    \definition[场,次]{s.}{palestra; discurso; ato ou a atividade de apresentar ou expressar ideias, opiniões ou informações oralmente em público ou diante de um público}
    \definition{v.}{dar uma palestra; fazer um discurso; informar o público sobre uma determinada área de conhecimento ou opinião sobre um determinado assunto}
  \end{Phonetics}
\end{Entry}

\begin{Entry}{演员}{14,7}{⽔,⼝}
  \begin{Phonetics}{演员}{yan3yuan2}[][HSK 3]
    \definition[个,位,名]{s.}{ator; artista; pessoas que participam de apresentações teatrais, cinematográficas, de dança, de artes cênicas, de acrobacias, etc.}
  \end{Phonetics}
\end{Entry}

\begin{Entry}{演奏}{14,9}{⽔,⼤}
  \begin{Phonetics}{演奏}{yan3zou4}[][HSK 6]
    \definition{v.}{tocar um instrumento musical; fazer uma apresentação instrumental}
  \end{Phonetics}
\end{Entry}

\begin{Entry}{演唱}{14,11}{⽔,⼝}
  \begin{Phonetics}{演唱}{yan3chang4}[][HSK 3]
    \definition{v.}{cantar em uma performance; apresentar canções, óperas, peças teatrais, etc.}
  \end{Phonetics}
\end{Entry}

\begin{Entry}{演唱会}{14,11,6}{⽔,⼝,⼈}
  \begin{Phonetics}{演唱会}{yan3chang4hui4}[][HSK 3]
    \definition[个,场]{s.}{recital vocal; concerto vocal; uma forma de apresentação centrada no canto, acompanhada por movimentos de dança simples}
  \end{Phonetics}
\end{Entry}

%%%%%%%%%% 漫 %%%%%%%%%%
\subsection*{漫}\addcontentsline{loh}{figure}{漫}

\begin{Entry}{漫}{14}{⽔}
  \begin{Phonetics}{漫}{man4}[][HSK 7-9]
    \definition{adj.}{livre; desimpedido; casual; sem restrições; arbitrário | em todo lugar; por toda parte | longo; extenso; distante}
    \definition{adv.}{não; não há necessidade de; expressa negação, equivalente a 不要}
    \definition{v.}{transbordar; inundar; alagar | estar em todo lugar; estar em todos os lugares}
  \seealsoref{不要}{bu2yao4}
  \end{Phonetics}
\end{Entry}

\begin{Entry}{漫长}{14,4}{⽔,⾧}
  \begin{Phonetics}{漫长}{man4chang2}[][HSK 5]
    \definition{adj.}{muito longo; interminável; (tempo, espaço) dura muito tempo}
  \end{Phonetics}
\end{Entry}

\begin{Entry}{漫画}{14,8}{⽔,⽥}
  \begin{Phonetics}{漫画}{man4hua4}[][HSK 5]
    \definition[幅,本,张,套]{s.}{desenho animado; caricatura; \emph{cartoon}}
  \end{Phonetics}
\end{Entry}

\begin{Entry}{漫骂}{14,9}{⽔,⾺}
  \begin{Phonetics}{漫骂}{man4ma4}
    \definition{v.}{usar linguagem ofensiva contra; insultar; difamar}
    \variantof{谩骂}
  \end{Phonetics}
\end{Entry}

\begin{Entry}{漫游}{14,12}{⽔,⽔}
  \begin{Phonetics}{漫游}{man4you2}[][HSK 7-9]
    \definition{v.}{vagar; perambular; dar voltas; fazer uma viagem de lazer | vagar; navegar; isso se refere à capacidade de telefones celulares e outros dispositivos se conectarem a qualquer terminal em outra área de serviço por meio da rede, após entrarem em uma área de serviço não registrada | (peixes) mover-se livremente; nadar livremente na água}
  \end{Phonetics}
\end{Entry}

%%%%%%%%%% 煽 %%%%%%%%%%
\subsection*{煽}\addcontentsline{loh}{figure}{煽}

\begin{Entry}{煽}{14}{⽕}
  \begin{Phonetics}{煽}{shan1}
    \definition{v.}{abanar (fogo); agitar um leque ou outra folha | incitar; instigar; agitar | vangloriar-se de; esbanjar prêmios em}
  \end{Phonetics}
\end{Entry}

\begin{Entry}{煽动}{14,6}{⽕,⼒}
  \begin{Phonetics}{煽动}{shan1dong4}[][HSK 7-9]
    \definition{v.}{instigar; incitar (alguém a fazer coisas ruins); agitar; inflamar}
  \end{Phonetics}
\end{Entry}

%%%%%%%%%% 熊 %%%%%%%%%%
\subsection*{熊}\addcontentsline{loh}{figure}{熊}

\begin{Entry}{熊}{14}{⽕}
  \begin{Phonetics}{熊}{xiong2}[][HSK 5]
    \definition*{s.}{Sobrenome: Xiong}
    \definition[头,只]{s.}{urso}
    \definition{v.}{repreender; censurar}
  \end{Phonetics}
\end{Entry}

\begin{Entry}{熊猫}{14,11}{⽕,⽝}
  \begin{Phonetics}{熊猫}{xiong2mao1}
    \definition[把,只]{s.}{panda gigante}
  \seealsoref{猫熊}{mao1xiong2}
  \end{Phonetics}
\end{Entry}

%%%%%%%%%% 熏 %%%%%%%%%%
\subsection*{熏}\addcontentsline{loh}{figure}{熏}

\begin{Entry}{熏}{14}{⽕}
  \begin{Phonetics}{熏}{xun1}
    \definition{v.}{expor à fumaça ou vapores; fumigar | tratar (carne, peixe, etc.) com fumaça; defumar | tornar perfumado com incenso, etc. | sufocar (asfixia e envenenamento por gás)}
  \end{Phonetics}
\end{Entry}

\begin{Entry}{熏香}{14,9}{⽕,⾹}
  \begin{Phonetics}{熏香}{xun1xiang1}
    \definition{s.}{incenso}
  \end{Phonetics}
\end{Entry}

%%%%%%%%%% 熬 %%%%%%%%%%
\subsection*{熬}\addcontentsline{loh}{figure}{熬}

\begin{Entry}{熬}{14}{⽕}
  \begin{Phonetics}{熬}{ao1}
    \definition{v.}{ensopar; ferver; cozinhar em água}
  \end{Phonetics}
  \begin{Phonetics}{熬}{ao2}[][HSK 7-9]
    \definition{v.}{ferver; ensopar; fazer uma decocção; cozinhar em fogo baixo por muito tempo | preparar; infundir; extrair a essência por fervura longa | resistir; suportar (angústia, tempos difíceis, etc.)}
  \end{Phonetics}
\end{Entry}

\begin{Entry}{熬夜}{14,8}{⽕,⼣}
  \begin{Phonetics}{熬夜}{ao2/ye4}[][HSK 7-9]
    \definition{v.+compl.}{ficar acordado a noite toda ou até tarde da noite}
  \end{Phonetics}
\end{Entry}

%%%%%%%%%% 疑 %%%%%%%%%%
\subsection*{疑}\addcontentsline{loh}{figure}{疑}

\begin{Entry}{疑}{14}{⽦}
  \begin{Phonetics}{疑}{yi2}
    \definition{adj.}{duvidoso; incerto}
    \definition{v.}{duvidar; desacreditar; suspeitar}
  \end{Phonetics}
\end{Entry}

\begin{Entry}{疑问}{14,6}{⽦,⾨}
  \begin{Phonetics}{疑问}{yi2wen4}[][HSK 4]
    \definition[个,些]{s.}{dúvida; consulta; pergunta; questionamento; coisas que não podem ser determinadas ou explicadas}
  \end{Phonetics}
\end{Entry}

%%%%%%%%%% 瘦 %%%%%%%%%%
\subsection*{瘦}\addcontentsline{loh}{figure}{瘦}

\begin{Entry}{瘦}{14}{⽧}
  \begin{Phonetics}{瘦}{shou4}[][HSK 5]
    \definition{adj.}{magro; esquelético | magro | apertado | infértil; pobre | esquelético; pouca gordura; pouca carne | (roupas, sapatos, meias, etc.) apertado |magra; (carne comestível) com baixo teor de gordura}
    \definition{v.}{perder peso}
  \antonymref{肥}{fei2}
  \antonymref{或}{huo4}
  \antonymref{胖}{pang4}
  \end{Phonetics}
\end{Entry}

%%%%%%%%%% 瞅 %%%%%%%%%%
\subsection*{瞅}\addcontentsline{loh}{figure}{瞅}

\begin{Entry}{瞅}{14}{⽬}
  \begin{Phonetics}{瞅}{chou3}[][HSK 7-9]
    \definition{v.}{Dialeto: olhar para}[让我瞅瞅。===Deixe-me dar uma olhada.]
  \end{Phonetics}
\end{Entry}

%%%%%%%%%% 碧 %%%%%%%%%%
\subsection*{碧}\addcontentsline{loh}{figure}{碧}

\begin{Entry}{碧}{14}{⽯}
  \begin{Phonetics}{碧}{bi4}
    \definition*{s.}{Sobrenome: Bi}
    \definition{adj.}{verde claro | azul claro | azul; verde-azulado; esverdeado; azul-celeste. turquesa}
    \definition{s.}{Literário: jade verde | safira}
  \end{Phonetics}
\end{Entry}

\begin{Entry}{碧绿}{14,11}{⽯,⽷}
  \begin{Phonetics}{碧绿}{bi4lv4}[][HSK 7-9]
    \definition{adj.}{verde jade; verde esmeralda; descreve um verde muito brilhante e profundo}
  \end{Phonetics}
\end{Entry}

%%%%%%%%%% 碳 %%%%%%%%%%
\subsection*{碳}\addcontentsline{loh}{figure}{碳}

\begin{Entry}{碳}{14}{⽯}
  \begin{Phonetics}{碳}{tan4}
    \definition{s.}{carbono (elemento químico)}
  \end{Phonetics}
\end{Entry}

\begin{Entry}{碳足迹}{14,7,9}{⽯,⾜,⾡}
  \begin{Phonetics}{碳足迹}{tan4 zu2ji4}
    \definition{s.}{pegada de carbono}
  \end{Phonetics}
\end{Entry}

%%%%%%%%%% 磁 %%%%%%%%%%
\subsection*{磁}\addcontentsline{loh}{figure}{磁}

\begin{Entry}{磁}{14}{⽯}
  \begin{Phonetics}{磁}{ci2}
    \definition[块]{s.}{porcelana | (física) magnetismo; propriedade de atrair ferro, níquel, etc. | (dialeto)  (de relação) próximo; íntimo}
  \end{Phonetics}
\end{Entry}

\begin{Entry}{磁卡}{14,5}{⽯,⼘}
  \begin{Phonetics}{磁卡}{ci2ka3}[][HSK 7-9]
    \definition[张]{s.}{cartão magnético}
  \end{Phonetics}
\end{Entry}

\begin{Entry}{磁带}{14,9}{⽯,⼱}
  \begin{Phonetics}{磁带}{ci2dai4}[][HSK 7-9]
    \definition[盘,盒,卷]{s.}{fita; fita magnética; cassete; uma fita plástica tratada com material magnético que pode gravar som ou imagens}
  \end{Phonetics}
\end{Entry}

\begin{Entry}{磁铁}{14,10}{⽯,⾦}
  \begin{Phonetics}{磁铁}{ci2tie3}
    \definition{s.}{imã | magneto}
  \seealsoref{吸铁石}{xi1tie3shi2}
  \end{Phonetics}
\end{Entry}

\begin{Entry}{磁盘}{14,11}{⽯,⽫}
  \begin{Phonetics}{磁盘}{ci2pan2}[][HSK 7-9]
    \definition{s.}{Computação: disco; disquete; um disco é um dispositivo de armazenamento que usa tecnologia de gravação magnética para armazenar dados}
  \end{Phonetics}
\end{Entry}

%%%%%%%%%% 磋 %%%%%%%%%%
\subsection*{磋}\addcontentsline{loh}{figure}{磋}

\begin{Entry}{磋}{14}{⽯}
  \begin{Phonetics}{磋}{cuo1}
    \definition{v.}{moer e polir marfim (significado original) | Figurativo: consultar; trocar opiniões | moer; polir}
  \end{Phonetics}
\end{Entry}

\begin{Entry}{磋商}{14,11}{⽯,⼝}
  \begin{Phonetics}{磋商}{cuo1shang1}[][HSK 7-9]
    \definition{v.}{consultar; negociar; trocar pontos de vista; discutir repetidamente; discutir cuidadosamente}
  \end{Phonetics}
\end{Entry}

%%%%%%%%%% 稳 %%%%%%%%%%
\subsection*{稳}\addcontentsline{loh}{figure}{稳}

\begin{Entry}{稳}{14}{⽲}
  \begin{Phonetics}{稳}{wen3}[][HSK 4]
    \definition{adj.}{constante; estável; firme | estável; estático; sedado | seguro; confiável; certo}
    \definition{adv.}{certamente; com certeza; seguramente; sem dúvida}
    \definition{v.}{estabilizar, manter estável; acalmar}
  \end{Phonetics}
\end{Entry}

\begin{Entry}{稳定}{14,8}{⽲,⼧}
  \begin{Phonetics}{稳定}{wen3ding4}[][HSK 4]
    \definition{adj.}{estável; firme; descreve uma natureza, um estado, etc. relativamente fixo; não muda significativamente}
    \definition{v.}{manter estável; estabilizar}
  \end{Phonetics}
\end{Entry}

%%%%%%%%%% 竭 %%%%%%%%%%
\subsection*{竭}\addcontentsline{loh}{figure}{竭}

\begin{Entry}{竭}{14}{⽴}
  \begin{Phonetics}{竭}{jie2}
    \definition*{s.}{Sobrenome: Jie}
    \definition{v.}{esgotar; consumir | Literário: secar; drenar}
  \end{Phonetics}
\end{Entry}

\begin{Entry}{竭力}{14,2}{⽴,⼒}
  \begin{Phonetics}{竭力}{jie2li4}[][HSK 7-9]
    \definition{v.}{fazer o máximo; fazer o máximo; não poupar esforços; tentar por todos os meios possíveis; dar o melhor de si; usar todos os esforços do corpo e da mente para\dots; usar cada grama de sua energia}
  \end{Phonetics}
\end{Entry}

\begin{Entry}{竭尽全力}{14,6,6,2}{⽴,⼫,⼊,⼒}
  \begin{Phonetics}{竭尽全力}{jie2jin4-quan2li4}[][HSK 7-9]
    \definition{expr.}{``Dê o seu melhor.''; não poupar esforços; fazer o máximo possível; com todas as forças; usar todas as suas forças para descrever o ato de fazer o máximo esforço; fazer o máximo possível; fazer tudo o que estiver ao seu alcance}
  \end{Phonetics}
\end{Entry}

%%%%%%%%%% 端 %%%%%%%%%%
\subsection*{端}\addcontentsline{loh}{figure}{端}

\begin{Entry}{端}{14}{⽴}
  \begin{Phonetics}{端}{duan1}[][HSK 6]
    \definition*{s.}{Sobrenome: Duan}
    \definition{adj.}{adequado; próprio | reto; correto}
    \definition{s.}{fim; extremidade | começo | item; ponto; pista, projeto ou aspecto | causa; razão | problema; incidente; coisas (geralmente se refere a coisas ruins, como acidentes, disputas, etc.)}
    \definition{v.}{carregar; segurar algo nivelado com ambas as mãos; segurar algo horizontalmente | erradicar; eliminar; acabar com; remover completamente; varrer | dar ares de superioridade | revelar}
  \end{Phonetics}
\end{Entry}

\begin{Entry}{端午节}{14,4,5}{⽴,⼗,⾋}
  \begin{Phonetics}{端午节}{duan1wu3jie2}[][HSK 6]
    \definition*[个]{s.}{Festa do Duplo Cinco, Festival dos Barcos-Dragão (5º~dia do quinto mês lunar)}
  \end{Phonetics}
\end{Entry}

\begin{Entry}{端正}{14,5}{⽴,⽌}
  \begin{Phonetics}{端正}{duan1zheng4}[][HSK 7-9]
    \definition{adj.}{apropriado; correto; não torto ou inclinado | ereto; integridade; decência}
    \definition{v.}{corrigir; fazer o certo}
  \end{Phonetics}
\end{Entry}

%%%%%%%%%% 算 %%%%%%%%%%
\subsection*{算}\addcontentsline{loh}{figure}{算}

\begin{Entry}{算}{14}{⽵}
  \begin{Phonetics}{算}{suan4}[][HSK 2]
    \definition{adv.}{finalmente; por fim; no final; significa que, após um longo período de tempo ou muitas dificuldades, finalmente se alcançou o objetivo, equivalente a 总算}
    \definition{v.}{calcular; estimar; computar | contar; incluir | planejar; calcular; projetar | pensar; supor; especular | considerar; considerar como; contar como; reconhecer como | (aritmética) contar; ter peso | deixe estar; deixe passar; seguido por 了: desistir, não se importar mais}
  \seealsoref{了}{le5}
  \seealsoref{总算}{zong3suan4}
  \end{Phonetics}
\end{Entry}

\begin{Entry}{算了}{14,2}{⽵,⼅}
  \begin{Phonetics}{算了}{suan4le5}[][HSK 6]
    \definition{part.}{deixe estar; deixe passar; usado no final de uma frase para expressar imperativo, término, etc.}
    \definition{v.}{deixar; deixe estar; deixe passar; esquecer isso; não querer continuar; é usado para persuadir os outros ou para expressar que posso aceitar a situação atual, para encerrar o assunto ou assunto atual, ou para dizer ``esqueça''}
  \end{Phonetics}
\end{Entry}

\begin{Entry}{算命}{14,8}{⽵,⼝}
  \begin{Phonetics}{算命}{suan4ming4}
    \definition{s.}{cartomante}
    \definition{v.}{ler a sorte | fazer advinhações}
  \end{Phonetics}
\end{Entry}

\begin{Entry}{算是}{14,9}{⽵,⽇}
  \begin{Phonetics}{算是}{suan4shi4}[][HSK 6]
    \definition{adv.}{finalmente; por fim; depois de muito tempo, o objetivo foi finalmente alcançado}
    \definition{v.}{contar como; pensar que; ser considerado}
  \end{Phonetics}
\end{Entry}

%%%%%%%%%% 管 %%%%%%%%%%
\subsection*{管}\addcontentsline{loh}{figure}{管}

\begin{Entry}{管}{14}{⽵}
  \begin{Phonetics}{管}{guan3}[][HSK 3]
    \definition*{s.}{Guan, um estado da dinastia Zhou | Sobrenome: Guan}
    \definition{adj.}{estreito; restrito; limitado; pequeno}
    \definition{clas.}{usado para objetos cilíndricos longos e finos}
    \definition{conj.}{não importa (quem, o quê, como, etc.)}
    \definition{prep.}{função semelhante a 把, usada especificamente em conjunto com 叫}
    \definition[根,条,排]{s.}{cano; tubo | instrumento musical de sopro | válvula; tubo | duto; canal; vasos}
    \definition{v.}{administrar; dirigir; controlar; cuidar; ser responsável por | ter jurisdição sobre; administrar | disciplinar (crianças ou alunos) | preocupar-se com; importar-se com; incomodar-se com; intervir | fornecer; garantir | supervisionar | governar | submeter alguém a disciplina | assumir; arcar com | incomodar; interferir | assegurar; garantir}
  \seealsoref{把}{ba3}
  \seealsoref{叫}{jiao4}
  \end{Phonetics}
\end{Entry}

\begin{Entry}{管子}{14,3}{⽵,⼦}
  \begin{Phonetics}{管子}{guan3zi5}[][HSK 7-9]
    \definition*{s.}{Guanzi ou Guan Zhong 管仲 (-645 a.C.), famoso político de Qi (齐国) do período da Primavera e do Outono | Guanzi, livro clássico contendo escritos de Guan Zhong e sua escola}
  \seealsoref{管仲}{guan3 zhong4}
  \seealsoref{齐国}{qi2 guo2}
  \end{Phonetics}
\end{Entry}

\begin{Entry}{管……叫……}{14,5}{⽵,⼝}
  \begin{Phonetics}{管……叫……}{guan3 jiao4}
    \definition{expr.}{chamar alguém (ou algo) de alguém (ou algo)}
  \end{Phonetics}
\end{Entry}

\begin{Entry}{管用}{14,5}{⽵,⽤}
  \begin{Phonetics}{管用}{guan3yong4}[][HSK 7-9]
    \definition{adj.}{eficaz; funcional}
  \end{Phonetics}
\end{Entry}

\begin{Entry}{管仲}{14,6}{⽵,⼈}
  \begin{Phonetics}{管仲}{guan3 zhong4}
    \definition*{s.}{uma visão restrita através de um tubo de bambu | conhecido como tubo de Guangzi 管子}
    \definition*{s.}{Guan Zhong (-645 aC), famoso político do Qi (齐国) do período da Primavera e Outono}
  \seealsoref{管子}{guan3zi5}
  \seealsoref{齐国}{qi2 guo2}
  \end{Phonetics}
\end{Entry}

\begin{Entry}{管家}{14,10}{⽵,⼧}
  \begin{Phonetics}{管家}{guan3jia1}[][HSK 7-9]
    \definition[个]{s.}{mordomo; antigamente, referia"-se a alguém que administrava os negócios de uma família rica | governanta; alguém que gerencia as tarefas domésticas | gerente; governanta; uma pessoa que administra bens ou negócios familiares ou coletivos}
    \definition{v.}{administrar uma casa}
  \end{Phonetics}
\end{Entry}

\begin{Entry}{管教}{14,11}{⽵,⽁}
  \begin{Phonetics}{管教}{guan3jiao4}[][HSK 7-9]
    \definition{adv.}{Dialeto: certamente; seguramente}
    \definition{v.}{corrigir; disciplinar alguém júnior | responsabilizar"-se por | ensinar}
  \end{Phonetics}
\end{Entry}

\begin{Entry}{管理}{14,11}{⽵,⽟}
  \begin{Phonetics}{管理}{guan3li3}[][HSK 3]
    \definition{v.}{gerenciar; executar; administrar; governar; estar encarregado de; responsável por garantir o bom andamento de uma determinada tarefa | controlar; gerenciar; fazer com que pessoas e animais obedeçam ou se comportem de maneira ordeira | cuidar; zelar por; proteger; cuidar, organizar coisas}
  \end{Phonetics}
\end{Entry}

\begin{Entry}{管理费}{14,11,9}{⽵,⽟,⾙}
  \begin{Phonetics}{管理费}{guan3li3fei4}[][HSK 7-9]
    \definition{s.}{despesas de gestão; custos de administração | taxa de administração}
  \end{Phonetics}
\end{Entry}

\begin{Entry}{管道}{14,12}{⽵,⾡}
  \begin{Phonetics}{管道}{guan3dao4}[][HSK 6]
    \definition[根,千米,公里]{s.}{oleoduto; canal; túnel; tubulação; um tubo feito de metal ou outro material usado para transportar ou descarregar fluidos (como vapor, gás, óleo, água, etc.) | caminho; canal; abordagem}
  \end{Phonetics}
\end{Entry}

\begin{Entry}{管辖}{14,14}{⽵,⾞}
  \begin{Phonetics}{管辖}{guan3xia2}[][HSK 7-9]
    \definition{v.}{gerenciar; governar (pessoal, assuntos, áreas, casos, etc.)}
  \end{Phonetics}
\end{Entry}

%%%%%%%%%% 精 %%%%%%%%%%
\subsection*{精}\addcontentsline{loh}{figure}{精}

\begin{Entry}{精}{14}{⽶}
  \begin{Phonetics}{精}{jing1}[][HSK 6]
    \definition{adj.}{refinado; escolhido; purificado ou selecionado | perfeito; excelente; melhor | fino; preciso; meticuloso | inteligente; astuto; esperto | habilidoso; versado; proficiente}
    \definition{adv.}{muito; extremamente; antes de certos adjetivos, significa 十分 ou 非常}
    \definition{s.}{extrato; essência; essência refinada ou selecionada; extraída | energia; espírito | semente; esperma; sêmen | \emph{goblin}; espírito; elfo; demônio}
  \seealsoref{非常}{fei1chang2}
  \seealsoref{十分}{shi2fen1}
  \antonymref{粗}{cu1}
  \end{Phonetics}
\end{Entry}

\begin{Entry}{精力}{14,2}{⽶,⼒}
  \begin{Phonetics}{精力}{jing1li4}[][HSK 4]
    \definition[些]{s.}{energia; vigor; força mental e física}
  \end{Phonetics}
\end{Entry}

\begin{Entry}{精子}{14,3}{⽶,⼦}
  \begin{Phonetics}{精子}{jing1zi3}
    \definition{s.}{espermatozoide; célula germinativa}
  \end{Phonetics}
\end{Entry}

\begin{Entry}{精心}{14,4}{⽶,⼼}
  \begin{Phonetics}{精心}{jing1xin1}[][HSK 7-9]
    \definition{adv.}{meticulosamente; cuidadosamente; elaboradamente; preste muita atenção; concentre-se totalmente}[他们精心设计了这个项目。===Eles planejaram este projeto meticulosamente.]
  \end{Phonetics}
\end{Entry}

\begin{Entry}{精打细算}{14,5,8,14}{⽶,⼿,⽷,⽵}
  \begin{Phonetics}{精打细算}{jing1da3-xi4suan4}[][HSK 7-9]
    \definition{expr.}{seja muito cuidadoso nos cálculos; contagem precisa; seja preciso nos cálculos; faça um orçamento rigoroso; cálculo cuidadoso e detalhado (meticuloso); cálculo cuidadoso e orçamento rigoroso; conte cada centavo e faça cada centavo valer a pena; corte os gastos com precisão; planejar com meticulosidade e cuidado, isso significa calcular com precisão o uso de mão de obra, recursos materiais e recursos financeiros para evitar desperdícios}
  \end{Phonetics}
\end{Entry}

\begin{Entry}{精华}{14,6}{⽶,⼗}
  \begin{Phonetics}{精华}{jing1hua2}[][HSK 7-9]
    \definition{s.}{elite; creme; escolha; essência; quintessência; a melhor e mais refinada parte de tudo | glória; esplendor; brilho; luz (do Sol e da Lua)}
  \end{Phonetics}
\end{Entry}

\begin{Entry}{精妙}{14,7}{⽶,⼥}
  \begin{Phonetics}{精妙}{jing1miao4}[][HSK 7-9]
    \definition{adj.}{requintado | fino e delicado (geralmente de obras de arte)}
  \end{Phonetics}
\end{Entry}

\begin{Entry}{精灵}{14,7}{⽶,⽕}
  \begin{Phonetics}{精灵}{jing1ling2}
    \definition{s.}{espírito | fada | elfo | duende | gênio}
  \end{Phonetics}
\end{Entry}

\begin{Entry}{精明}{14,8}{⽶,⽇}
  \begin{Phonetics}{精明}{jing1ming2}[][HSK 7-9]
    \definition{adj.}{astuto; sagaz; perspicaz; inteligente e brilhante}
  \end{Phonetics}
\end{Entry}

\begin{Entry}{精练}{14,8}{⽶,⽷}
  \begin{Phonetics}{精练}{jing1lian4}[][HSK 7-9]
    \definition{adj.}{conciso; sucinto; lacônico | refinado; palavras e frases redundantes eliminadas}
    \definition{v.}{praticar intensivamente}
  \synonymref{干脆}{gan1cui4}
  \synonymref{简单}{jian3dan1}
  \synonymref{简洁}{jian3jie2}
  \synonymref{精炼}{jing1lian4}
  \antonymref{冗长}{rong3chang2}
  \end{Phonetics}
\end{Entry}

\begin{Entry}{精细}{14,8}{⽶,⽷}
  \begin{Phonetics}{精细}{jing1xi4}[][HSK 7-9]
    \definition{adj.}{fino; cuidadoso; meticuloso; muito delicado | astuto; perspicaz e cuidadoso; muito meticuloso}
  \end{Phonetics}
\end{Entry}

\begin{Entry}{精英}{14,8}{⽶,⾋}
  \begin{Phonetics}{精英}{jing1ying1}[][HSK 7-9]
    \definition{s.}{creme; essência; quintessência | escolhido; elite; pessoa de habilidade excepcional}
  \end{Phonetics}
\end{Entry}

\begin{Entry}{精品}{14,9}{⽶,⼝}
  \begin{Phonetics}{精品}{jing1pin3}[][HSK 6]
    \definition[个]{s.}{belas obras (de arte); objetos de arte | produtos de qualidade; artigos de excelente qualidade; produto \emph{premium}}
  \end{Phonetics}
\end{Entry}

\begin{Entry}{精炼}{14,9}{⽶,⽕}
  \begin{Phonetics}{精炼}{jing1lian4}
    \definition{adj.}{conciso; sucinto; lacônico}
    \definition{s.}{refino; remoção de impurezas}
    \definition{v.}{Metalurgia: refinar; purificar; fundir}
  \end{Phonetics}
\end{Entry}

\begin{Entry}{精神}{14,9}{⽶,⽰}
  \begin{Phonetics}{精神}{jing1shen2}[][HSK 3]
    \definition[种,个,类,股]{s.}{espírito; mente; estado mental; refere"-se à consciência, às atividades mentais e ao estado psicológico geral de uma pessoa | substância; espírito; essência; propósito; significado principal}
  \end{Phonetics}
  \begin{Phonetics}{精神}{jing1shen5}[][HSK 3]
    \definition{adj.}{animado; espirituoso; vigoroso; descreve uma pessoa como cheia de energia | muito bonito; boa aparência, bom físico}
    \definition[种,个,类,股]{s.}{impulso; vigor; vitalidade}
  \end{Phonetics}
\end{Entry}

\begin{Entry}{精神病}{14,9,10}{⽶,⽰,⽧}
  \begin{Phonetics}{精神病}{jing1shen2bing4}[][HSK 7-9]
    \definition{s.}{doença mental; transtorno mental; psicose}[这是幻想型精神病的体现。===Isso é uma manifestação de psicose delirante.]
  \end{Phonetics}
\end{Entry}

\begin{Entry}{精美}{14,9}{⽶,⽺}
  \begin{Phonetics}{精美}{jing1mei3}[][HSK 6]
    \definition{adj.}{elegante; requintado}
  \end{Phonetics}
\end{Entry}

\begin{Entry}{精疲力竭}{14,10,2,14}{⽶,⽧,⼒,⽴}
  \begin{Phonetics}{精疲力竭}{jing1pi2-li4jie2}[][HSK 7-9]
    \definition{expr.}{esgotado; exausto; desgastado; descrevendo fadiga extrema e completa falta de energia}
  \end{Phonetics}
\end{Entry}

\begin{Entry}{精益求精}{14,10,7,14}{⽶,⽫,⽔,⽶}
  \begin{Phonetics}{精益求精}{jing1yi4qiu2jing1}[][HSK 7-9]
    \definition{expr.}{``Busque a excelência.''; esforçar-se pela perfeição; buscar a melhoria constante; perseguir a excelência; almejar a perfeição; já está muito bom, mas você ainda quer que fique ainda melhor; melhorar algo constantemente; continuar melhorando}
  \end{Phonetics}
\end{Entry}

\begin{Entry}{精致}{14,10}{⽶,⾄}
  \begin{Phonetics}{精致}{jing1zhi4}[][HSK 7-9]
    \definition{adj.}{fino; requintado; delicado}[我们欣赏她精致的手工艺品。===Admiramos seu trabalho artesanal requintado.]
  \end{Phonetics}
\end{Entry}

\begin{Entry}{精通}{14,10}{⽶,⾡}
  \begin{Phonetics}{精通}{jing1tong1}[][HSK 7-9]
    \definition{v.}{dominar; ser proficiente em; ter um bom domínio de; ter um profundo entendimento e conhecimento abrangente de uma área específica de estudo, tecnologia ou negócios}
  \end{Phonetics}
\end{Entry}

\begin{Entry}{精密}{14,11}{⽶,⼧}
  \begin{Phonetics}{精密}{jing1mi4}
    \definition{adj.}{preciso; preciso e meticuloso}
  \end{Phonetics}
\end{Entry}

\begin{Entry}{精彩}{14,11}{⽶,⼺}
  \begin{Phonetics}{精彩}{jing1cai3}[][HSK 3]
    \definition{adj.}{brilhante; esplêndido; maravilhoso}
  \end{Phonetics}
\end{Entry}

\begin{Entry}{精确}{14,12}{⽶,⽯}
  \begin{Phonetics}{精确}{jing1que4}[][HSK 7-9]
    \definition{adj.}{exato; preciso; acurado; muito preciso e correto}
  \end{Phonetics}
\end{Entry}

\begin{Entry}{精简}{14,13}{⽶,⽵}
  \begin{Phonetics}{精简}{jing1jian3}[][HSK 7-9]
    \definition{v.}{reduzir; simplificar; cortar; simplificar; eliminar o desnecessário e conservar o necessário}
  \end{Phonetics}
\end{Entry}

\begin{Entry}{精髓}{14,21}{⽶,⾻}
  \begin{Phonetics}{精髓}{jing1sui3}[][HSK 7-9]
    \definition{s.}{medula; medula óssea; quintessência; essência metafórica das coisas}
  \end{Phonetics}
\end{Entry}

%%%%%%%%%% 缩 %%%%%%%%%%
\subsection*{缩}\addcontentsline{loh}{figure}{缩}

\begin{Entry}{缩}{14}{⽷}
  \begin{Phonetics}{缩}{suo1}
    \definition*{s.}{Sobrenome: Suo}
    \definition{v.}{contrair; encolher | recuar; retirar-se | economizar}
  \end{Phonetics}
\end{Entry}

\begin{Entry}{缩小}{14,3}{⽷,⼩}
  \begin{Phonetics}{缩小}{suo1/xiao3}[][HSK 4]
    \definition{v.+compl.}{reduzir, estreitar, encolher;  tornar menor}
  \antonymref{放大}{fang4/da4}
  \end{Phonetics}
\end{Entry}

\begin{Entry}{缩手}{14,4}{⽷,⼿}
  \begin{Phonetics}{缩手}{suo1shou3}
    \definition{v.}{retirar a mão}
  \end{Phonetics}
\end{Entry}

\begin{Entry}{缩短}{14,12}{⽷,⽮}
  \begin{Phonetics}{缩短}{suo1/duan3}[][HSK 4]
    \definition{v.+compl.}{encurtar; reduzir; diminuir}
  \end{Phonetics}
\end{Entry}

\begin{Entry}{缩影卡片}{14,15,5,4}{⽷,⼺,⼘,⽚}
  \begin{Phonetics}{缩影卡片}{suo1ying3 ka3pian4}
    \definition{s.}{cartão em miniatura; microcartão}
  \end{Phonetics}
\end{Entry}

%%%%%%%%%% 翠 %%%%%%%%%%
\subsection*{翠}\addcontentsline{loh}{figure}{翠}

\begin{Entry}{翠}{14}{⽻}
  \begin{Phonetics}{翠}{cui4}
    \definition{adj.}{verde; verde esmeralda}
    \definition{s.}{martim-pescador | jadeíte; jade}
  \end{Phonetics}
\end{Entry}

\begin{Entry}{翠绿}{14,11}{⽻,⽷}
  \begin{Phonetics}{翠绿}{cui4lv4}[][HSK 7-9]
    \definition{adj.}{verde esmeralda; verde jade}
  \end{Phonetics}
\end{Entry}

%%%%%%%%%% 聚 %%%%%%%%%%
\subsection*{聚}\addcontentsline{loh}{figure}{聚}

\begin{Entry}{聚}{14}{⽿}
  \begin{Phonetics}{聚}{ju4}[][HSK 4]
    \definition*{s.}{Sobrenome: Ju}
    \definition{v.}{reunir-se; juntar-se}
  \end{Phonetics}
\end{Entry}

\begin{Entry}{聚会}{14,6}{⽿,⼈}
  \begin{Phonetics}{聚会}{ju4hui4}[][HSK 4]
    \definition[个,次]{s.}{reunião; encontro; confraternização; festa}
    \definition{v.}{encontrar-se; reunir-se}
  \end{Phonetics}
\end{Entry}

\begin{Entry}{聚散}{14,12}{⽿,⽁}
  \begin{Phonetics}{聚散}{ju4san4}
    \definition{s.}{juntos e separados | agregação e dissipação}
  \end{Phonetics}
\end{Entry}

\begin{Entry}{聚集}{14,12}{⽿,⾫}
  \begin{Phonetics}{聚集}{ju4ji2}[][HSK 7-9]
    \definition{v.}{reunir; juntar; coletar; reunir-se; juntar-se}
  \end{Phonetics}
\end{Entry}

\begin{Entry}{聚精会神}{14,14,6,9}{⽿,⽶,⼈,⽰}
  \begin{Phonetics}{聚精会神}{ju4jing1-hui4shen2}[][HSK 7-9]
    \definition{expr.}{concentrado; concentrar a atenção; focar a mente; estar absorto em; estar profundamente concentrado; estar totalmente concentrado}
  \end{Phonetics}
\end{Entry}

%%%%%%%%%% 腐 %%%%%%%%%%
\subsection*{腐}\addcontentsline{loh}{figure}{腐}

\begin{Entry}{腐}{14}{⾁}
  \begin{Phonetics}{腐}{fu3}
    \definition{adj.}{podre; obsoleto; corrupto | corroído; pútrido}
    \definition{s.}{tofu}
    \definition{v.}{apodrecer; corroer; estragar; decair}
  \end{Phonetics}
\end{Entry}

\begin{Entry}{腐化}{14,4}{⾁,⼔}
  \begin{Phonetics}{腐化}{fu3hua4}[][HSK 7-9]
    \definition{adj.}{degenerado; corrupto, dissoluto ou depravado; desmoralizado; decadente}
    \definition{v.}{decompor; apodrecer; tornar-se pútrido | quebrar; corroer}
  \end{Phonetics}
\end{Entry}

\begin{Entry}{腐朽}{14,6}{⾁,⽊}
  \begin{Phonetics}{腐朽}{fu3xiu3}[][HSK 7-9]
    \definition{adj.}{decaído; decadente; degenerado; uma metáfora para as ideias ultrapassadas das pessoas ou para a moral social corrupta}
    \definition{v.}{apodrecer; decair; apodrecimento e deterioração da madeira e outros materiais fibrosos}
  \end{Phonetics}
\end{Entry}

\begin{Entry}{腐败}{14,8}{⾁,⾒}
  \begin{Phonetics}{腐败}{fu3bai4}[][HSK 7-9]
    \definition{adj.}{(ideia) corrupto; decadente; (pensamento) obsoleto; (comportamento) degenerado | (sistema, organização, instituição, medida, etc.) corrupto}
    \definition{s.}{deterioração; podridão}
    \definition{v.}{apodrecer; decair}
  \end{Phonetics}
\end{Entry}

\begin{Entry}{腐烂}{14,9}{⾁,⽕}
  \begin{Phonetics}{腐烂}{fu3lan4}[][HSK 7-9]
    \definition{adj.}{corrupto; depravado | (pensamento) obsoleto; (comportamento) degenerado}
    \definition{v.}{apodrecer; decompor; tornar"-se pútrido}
  \end{Phonetics}
\end{Entry}

\begin{Entry}{腐蚀}{14,9}{⾁,⾷}
  \begin{Phonetics}{腐蚀}{fu3shi2}[][HSK 7-9]
    \definition{v.}{corroer; destruir gradualmente um objeto por meio de reações químicas | corroer; corromper (pensamentos e comportamentos)}
  \end{Phonetics}
\end{Entry}

%%%%%%%%%% 膜 %%%%%%%%%%
\subsection*{膜}\addcontentsline{loh}{figure}{膜}

\begin{Entry}{膜}{14}{⾁}
  \begin{Phonetics}{膜}{mo2}[][HSK 6]
    \definition[张]{s.}{membrana | filme; revestimento fino}
  \end{Phonetics}
\end{Entry}

\begin{Entry}{膜拜}{14,9}{⾁,⼿}
  \begin{Phonetics}{膜拜}{mo2bai4}
    \definition{v.}{ajoelhar-se e se curvar com as mãos unidas no nível da testa | ter ou mostrar sentimentos fortes de respeito e admiração por um deus}
  \end{Phonetics}
\end{Entry}

%%%%%%%%%% 舞 %%%%%%%%%%
\subsection*{舞}\addcontentsline{loh}{figure}{舞}

\begin{Entry}{舞}{14}{⾇}
  \begin{Phonetics}{舞}{wu3}[][HSK 5]
    \definition[支,段,个]{s.}{dança | palco; metáfora do domínio das atividades sociais}
    \definition{v.}{mover-se como numa dança | dançar com algo nas mãos; brincar com | florescer; empunhar; brandir | esvoaçar | fazer malabarismos; brincar com}
  \end{Phonetics}
\end{Entry}

\begin{Entry}{舞厅}{14,4}{⾇,⼚}
  \begin{Phonetics}{舞厅}{wu3ting1}
    \definition[间]{s.}{salão de dança | salão de baile}
  \end{Phonetics}
\end{Entry}

\begin{Entry}{舞厅舞}{14,4,14}{⾇,⼚,⾇}
  \begin{Phonetics}{舞厅舞}{wu3ting1wu3}
    \definition{s.}{dança de salão}
  \end{Phonetics}
\end{Entry}

\begin{Entry}{舞台}{14,5}{⾇,⼝}
  \begin{Phonetics}{舞台}{wu3tai2}[][HSK 3]
    \definition[个]{s.}{palco; plataforma elevada usada exclusivamente para apresentações artísticas, geralmente localizada na parte frontal de teatros e auditórios | palco; metáfora do campo das atividades sociais}
  \end{Phonetics}
\end{Entry}

\begin{Entry}{舞会}{14,6}{⾇,⼈}
  \begin{Phonetics}{舞会}{wu3hui4}
    \definition{s.}{baile}
  \end{Phonetics}
\end{Entry}

\begin{Entry}{舞会舞}{14,6,14}{⾇,⼈,⾇}
  \begin{Phonetics}{舞会舞}{wu3hui4wu3}
    \definition{s.}{baile}
  \end{Phonetics}
\end{Entry}

\begin{Entry}{舞抃}{14,7}{⾇,⼿}
  \begin{Phonetics}{舞抃}{wu3bian4}
    \definition{s.}{dançar por prazer}
  \end{Phonetics}
\end{Entry}

\begin{Entry}{舞蹈}{14,17}{⾇,⾜}
  \begin{Phonetics}{舞蹈}{wu3dao3}[][HSK 6]
    \definition[段,支,场,个]{s.}{dança; uma forma de arte que usa movimentos rítmicos como principal meio de expressão, podendo expressar a vida, os pensamentos e os sentimentos das pessoas, geralmente acompanhada de música}
    \definition{v.}{dançar}
  \end{Phonetics}
\end{Entry}

%%%%%%%%%% 蔓 %%%%%%%%%%
\subsection*{蔓}\addcontentsline{loh}{figure}{蔓}

\begin{Entry}{蔓}{14}{⾋}
  \begin{Phonetics}{蔓}{man2}
    \definition{s.}{couve-chinesa | nabo}
  \end{Phonetics}
  \begin{Phonetics}{蔓}{man4}
    \definition{s.}{uma videira com gavinhas; caule fino que não consegue ficar em pé}
    \definition{v.}{rastejar; espalhar; estender}
  \end{Phonetics}
  \begin{Phonetics}{蔓}{wan4}
    \definition*{s.}{Sobrenome: Wan}
    \definition{s.}{uma videira com gavinhas; caule fino que não consegue ficar em pé}
  \end{Phonetics}
\end{Entry}

\begin{Entry}{蔓延}{14,6}{⾋,⼵}
  \begin{Phonetics}{蔓延}{man4yan2}[][HSK 7-9]
    \definition{v.}{espalhar; esticar; estender | infestar; espalhar; essa metáfora descreve coisas que se estendem e se expandem para fora como trepadeiras rastejantes}
  \end{Phonetics}
\end{Entry}

\begin{Entry}{蔓草}{14,9}{⾋,⾋}
  \begin{Phonetics}{蔓草}{man4cao3}
    \definition{s.}{videira | trepadeira}
  \end{Phonetics}
\end{Entry}

%%%%%%%%%% 蜘 %%%%%%%%%%
\subsection*{蜘}\addcontentsline{loh}{figure}{蜘}

\begin{Entry}{蜘}{14}{⾍}
  \begin{Phonetics}{蜘}{zhi1}
    \definition[只]{s.}{aranha}
  \seealsoref{蜘蛛}{zhi1zhu1}
  \end{Phonetics}
\end{Entry}

\begin{Entry}{蜘蛛}{14,12}{⾍,⾍}
  \begin{Phonetics}{蜘蛛}{zhi1zhu1}
    \definition{s.}{aranha}
  \end{Phonetics}
\end{Entry}

\begin{Entry}{蜘蛛网}{14,12,6}{⾍,⾍,⽹}
  \begin{Phonetics}{蜘蛛网}{zhi1zhu1 wang3}
    \definition{s.}{teia de aranha}
  \end{Phonetics}
\end{Entry}

%%%%%%%%%% 蜜 %%%%%%%%%%
\subsection*{蜜}\addcontentsline{loh}{figure}{蜜}

\begin{Entry}{蜜}{14}{⾍}
  \begin{Phonetics}{蜜}{mi4}[][HSK 7-9]
    \definition{adj.}{melado; doce}
    \definition{s.}{mel | semelhante ao mel | coisas parecidas com mel; melaço}
  \end{Phonetics}
\end{Entry}

\begin{Entry}{蜜月}{14,4}{⾍,⽉}
  \begin{Phonetics}{蜜月}{mi4yue4}[][HSK 7-9]
    \definition{s.}{lua de mel; o primeiro mês após o casamento}
  \end{Phonetics}
\end{Entry}

\begin{Entry}{蜜桃}{14,10}{⾍,⽊}
  \begin{Phonetics}{蜜桃}{mi4tao2}
    \definition{s.}{nectarina | pêssego | pêssego suculento}
  \end{Phonetics}
\end{Entry}

\begin{Entry}{蜜蜂}{14,13}{⾍,⾍}
  \begin{Phonetics}{蜜蜂}{mi4feng1}[][HSK 7-9]
    \definition[只,群,箱,窝]{s.}{abelha; abelha-melífera}
  \end{Phonetics}
\end{Entry}

%%%%%%%%%% 蜡 %%%%%%%%%%
\subsection*{蜡}\addcontentsline{loh}{figure}{蜡}

\begin{Entry}{蜡}{14}{⾍}
  \begin{Phonetics}{蜡}{la4}[][HSK 7-9]
    \definition{s.}{cera; óleos produzidos por animais, minerais ou plantas | vela}
  \end{Phonetics}
  \begin{Phonetics}{蜡}{zha4}
    \definition{s.}{uma antiga cerimônia de sacrifício de fim de ano}
  \end{Phonetics}
\end{Entry}

\begin{Entry}{蜡烛}{14,10}{⾍,⽕}
  \begin{Phonetics}{蜡烛}{la4zhu2}[][HSK 7-9]
    \definition[根,支,包]{s.}{vela; círio; peça cilíndrica, geralmente de cera, que possui um pavio e se utiliza como iluminação}
  \end{Phonetics}
\end{Entry}

%%%%%%%%%% 蜥 %%%%%%%%%%
\subsection*{蜥}\addcontentsline{loh}{figure}{蜥}

\begin{Entry}{蜥}{14}{⾍}
  \begin{Phonetics}{蜥}{xi1}
    \definition{s.}{lagarto}
  \end{Phonetics}
\end{Entry}

\begin{Entry}{蜥易}{14,8}{⾍,⽇}
  \begin{Phonetics}{蜥易}{xi1yi4}
    \variantof{蜥蜴}
  \end{Phonetics}
\end{Entry}

\begin{Entry}{蜥蜴}{14,14}{⾍,⾍}
  \begin{Phonetics}{蜥蜴}{xi1yi4}
    \definition{s.}{lagarto}
  \end{Phonetics}
\end{Entry}

%%%%%%%%%% 蜻 %%%%%%%%%%
\subsection*{蜻}\addcontentsline{loh}{figure}{蜻}

\begin{Entry}{蜻}{14}{⾍}
  \begin{Phonetics}{蜻}{qing1}
    \definition[只]{s.}{libélula, 蜻蜓}
  \seealsoref{蜻蜓}{qing1ting2}
  \end{Phonetics}
\end{Entry}

\begin{Entry}{蜻蜓}{14,12}{⾍,⾍}
  \begin{Phonetics}{蜻蜓}{qing1ting2}
    \definition{s.}{libélula}
  \end{Phonetics}
\end{Entry}

\begin{Entry}{蜻蝏}{14,15}{⾍,⾍}
  \begin{Phonetics}{蜻蝏}{qing1ting2}
    \variantof{蜻蜓}
  \end{Phonetics}
\end{Entry}

%%%%%%%%%% 蝉 %%%%%%%%%%
\subsection*{蝉}\addcontentsline{loh}{figure}{蝉}

\begin{Entry}{蝉}{14}{⾍}
  \begin{Phonetics}{蝉}{chan2}
    \definition[只,个]{s.}{cigarra}
  \seealsoref{知了}{zhi1liao3}
  \end{Phonetics}
\end{Entry}

%%%%%%%%%% 裹 %%%%%%%%%%
\subsection*{裹}\addcontentsline{loh}{figure}{裹}

\begin{Entry}{裹}{14}{⾐}
  \begin{Phonetics}{裹}{guo3}[][HSK 7-9]
    \definition{s.}{pacote; encomenda}
    \definition{v.}{amarrar; embrulhar; envolver | levar embora; varrer com violência | Dialeto: sugar (leite) | pressionar a servir; fugir com (algo)}
  \end{Phonetics}
\end{Entry}

%%%%%%%%%% 褐 %%%%%%%%%%
\subsection*{褐}\addcontentsline{loh}{figure}{褐}

\begin{Entry}{褐}{14}{⾐}
  \begin{Phonetics}{褐}{he4}
    \definition{adj.}{marrom; castanho; pardo}
    \definition{s.}{pano de cânhamo grosso}
  \end{Phonetics}
\end{Entry}

\begin{Entry}{褐色}{14,6}{⾐,⾊}
  \begin{Phonetics}{褐色}{he4 se4}
    \definition{s.}{cor marrom}
  \end{Phonetics}
\end{Entry}

%%%%%%%%%% 褡 %%%%%%%%%%
\subsection*{褡}\addcontentsline{loh}{figure}{褡}

\begin{Entry}{褡}{14}{⾐}
  \begin{Phonetics}{褡}{da1}
    \definition{s.}{bolsa; malote; algibeira | jaqueta sem mangas}
  \end{Phonetics}
\end{Entry}

%%%%%%%%%% 谱 %%%%%%%%%%
\subsection*{谱}\addcontentsline{loh}{figure}{谱}

\begin{Entry}{谱}{14}{⾔}
  \begin{Phonetics}{谱}{pu3}[][HSK 7-9]
    \definition{s.}{registro ou documento de fácil consulta (na forma de gráficos, tabelas, listas, etc.); cronologia; um livro compilado para consulta, organizado de acordo com a categoria ou sistema do assunto e utilizando tabelas ou outros formatos claros | manual; guia; formatos ou diagramas que podem ser usados para orientar a prática | partitura; partitura musical; partitura para música}
    \definition{v.}{compor música | exibir-se; mostrar-se arrogante}
  \end{Phonetics}
\end{Entry}

%%%%%%%%%% 豪 %%%%%%%%%%
\subsection*{豪}\addcontentsline{loh}{figure}{豪}

\begin{Entry}{豪}{14}{⾗}
  \begin{Phonetics}{豪}{hao2}
    \definition*{s.}{Sobrenome: Hao}
    \definition{adj.}{direto; irrestrito; ousado | despótico; intimidador | rico e poderoso}
    \definition{s.}{pessoa com poderes ou dons extraordinários}
  \end{Phonetics}
\end{Entry}

\begin{Entry}{豪华}{14,6}{⾗,⼗}
  \begin{Phonetics}{豪华}{hao2hua2}[][HSK 7-9]
    \definition{adj.}{luxo; luxuoso; (edifício, equipamento ou decoração) magnífico; muito lindo}
  \end{Phonetics}
\end{Entry}

%%%%%%%%%% 赚 %%%%%%%%%%
\subsection*{赚}\addcontentsline{loh}{figure}{赚}

\begin{Entry}{赚}{14}{⾙}
  \begin{Phonetics}{赚}{zhuan4}[][HSK 6]
    \definition{s.}{lucro}
    \definition{v.}{ganhar (dinheiro); obter lucro com o negócio}
  \antonymref{赔}{pei2}
  \end{Phonetics}
\end{Entry}

\begin{Entry}{赚钱}{14,10}{⾙,⾦}
  \begin{Phonetics}{赚钱}{zhuan4 qian2}[][HSK 6]
    \definition{v.}{ganhar dinheiro; obter lucro ou recompensa}
  \end{Phonetics}
\end{Entry}

%%%%%%%%%% 赛 %%%%%%%%%%
\subsection*{赛}\addcontentsline{loh}{figure}{赛}

\begin{Entry}{赛}{14}{⾙}
  \begin{Phonetics}{赛}{sai4}[][HSK 6]
    \definition*{s.}{Sobrenome: Sai}
    \definition{s.}{jogo; partida; competição | sacrifício; cerimônia de sacrifício; antigamente, sacrifícios eram feitos para agradecer aos deuses por suas dádivas}
    \definition{v.}{ter uma competição (comparando alto e baixo, forte e fraco) | superar; ser comparável a; comparar com}
  \end{Phonetics}
\end{Entry}

\begin{Entry}{赛车}{14,4}{⾙,⾞}
  \begin{Phonetics}{赛车}{sai4che1}[][HSK 7-9]
    \definition{s.}{veículo de corrida; carro de corrida; bicicletas, motocicletas ou carros de corrida | corridas de carros}
    \definition{v.}{correr; disputar uma corrida}
  \end{Phonetics}
\end{Entry}

\begin{Entry}{赛场}{14,6}{⾙,⼟}
  \begin{Phonetics}{赛场}{sai4chang3}[][HSK 6]
    \definition{s.}{local de competição; arena; ringue; terreno | campo (para competição de atletismo) | pista de corrida}
  \end{Phonetics}
\end{Entry}

\begin{Entry}{赛跑}{14,12}{⾙,⾜}
  \begin{Phonetics}{赛跑}{sai4pao3}[][HSK 7-9]
    \definition{v.}{correr; disputar uma corrida; esportes que testam a velocidade de corrida incluem provas de curta, média, longa e ultra-longa distância, além de corridas com barreiras, revezamentos, corridas com obstáculos e corridas de cross-country}
  \end{Phonetics}
\end{Entry}

%%%%%%%%%% 赫 %%%%%%%%%%
\subsection*{赫}\addcontentsline{loh}{figure}{赫}

\begin{Entry}{赫}{14}{⾚}
  \begin{Phonetics}{赫}{he4}
    \definition*{s.}{Sobrenome: He}
    \definition{adj.}{conspícuo; grandioso | vermelho brilhante e flamejante; vermelho como fogo}
    \definition{clas.}{Hz, hertz; abreviação de 赫兹}
  \seealsoref{赫兹}{he4zi1}
  \end{Phonetics}
\end{Entry}

\begin{Entry}{赫兹}{14,9}{⾚,⼋}
  \begin{Phonetics}{赫兹}{he4zi1}
    \definition{s.}{hertz (Hz), unidade de frequência}
    \definition{s.}{Heinrich Hertz (1857-1894), físico e meteorologista alemão, pioneiro da radiação eletromagnética}
  \end{Phonetics}
\end{Entry}

\begin{Entry}{赫然}{14,12}{⾚,⽕}
  \begin{Phonetics}{赫然}{he4ran2}[][HSK 7-9]
    \definition{adj.}{inesperado e chocante/impressionante; descreve algo que é muito marcante ou surpreendente | (raiva, etc.) terrível; violento; descreve o olhar de raiva | grande; eminente; florescente; excepcional; descreve a aparência de ser proeminente}
  \end{Phonetics}
\end{Entry}

%%%%%%%%%% 辗 %%%%%%%%%%
\subsection*{辗}\addcontentsline{loh}{figure}{辗}

\begin{Entry}{辗}{14}{⾞}
  \begin{Phonetics}{辗}{zhan3}
    \definition{v.}{(arcaico) virar | (arcaico) rolar para o lado | (arcaico) virar a metade}
  \end{Phonetics}
\end{Entry}

%%%%%%%%%% 辣 %%%%%%%%%%
\subsection*{辣}\addcontentsline{loh}{figure}{辣}

\begin{Entry}{辣}{14}{⾟}
  \begin{Phonetics}{辣}{la4}[][HSK 4]
    \definition{adj.}{apimentado; picante; pungente; quente | cruel; implacável; venenoso; vicioso}
    \definition{v.}{queimar; picar; formigar; ter uma irritação picante (boca, nariz ou olhos)}
  \end{Phonetics}
\end{Entry}

\begin{Entry}{辣椒}{14,12}{⾟,⽊}
  \begin{Phonetics}{辣椒}{la4jiao1}[][HSK 7-9]
    \definition[颗,把,袋,种]{s.}{pimenta; pimenta-malagueta; pimenta caiena; pimentão; pimenta vermelha}
  \end{Phonetics}
\end{Entry}

%%%%%%%%%% 遭 %%%%%%%%%%
\subsection*{遭}\addcontentsline{loh}{figure}{遭}

\begin{Entry}{遭}{14}{⾡}
  \begin{Phonetics}{遭}{zao1}
    \definition{clas.}{tempo; vez; ocasião | rodadas}
    \definition{v.}{encontrar-se com (desastre, infortúnio, etc.); sofrer}
  \end{Phonetics}
\end{Entry}

\begin{Entry}{遭到}{14,8}{⾡,⼑}
  \begin{Phonetics}{遭到}{zao1dao4}[][HSK 6]
    \definition{v.}{sofrer; ser rejeitado; receber crítica; significa sofrer infortúnio ou dano}[我们遭到意外事故。===Nós sofremos um acidente.]
  \end{Phonetics}
\end{Entry}

\begin{Entry}{遭受}{14,8}{⾡,⼜}
  \begin{Phonetics}{遭受}{zao1shou4}[][HSK 6]
    \definition{v.}{sofrer; aguentar; ser submetido a; encontrar ou vivenciar coisas dolorosas que você não quer que aconteçam}
  \end{Phonetics}
\end{Entry}

\begin{Entry}{遭遇}{14,12}{⾡,⾡}
  \begin{Phonetics}{遭遇}{zao1yu4}[][HSK 6]
    \definition[场,次,种,段]{s.}{sorte (difícil); experiência (amarga); encontrando coisas ruins}
    \definition{v.}{encontrar; encontrar-se com; esbarrar em; encontros inesperados com pessoas ou coisas que não são boas para você}
  \end{Phonetics}
\end{Entry}

%%%%%%%%%% 酷 %%%%%%%%%%
\subsection*{酷}\addcontentsline{loh}{figure}{酷}

\begin{Entry}{酷}{14}{⾣}
  \begin{Phonetics}{酷}{ku4}[][HSK 6]
    \definition{adj.}{cruel; opressivo | feroz; escaldante | brutal | Empréstimo linguístico: \emph{cool}; legal; excelente; moderno; ótimo | elegante e sóbrio; gracioso e severo}
    \definition{adv.}{muito; extremamente}
  \end{Phonetics}
\end{Entry}

\begin{Entry}{酷似}{14,6}{⾣,⼈}
  \begin{Phonetics}{酷似}{ku4si4}[][HSK 7-9]
    \definition{v.}{ser a própria imagem de; ser exatamente igual a; apresentar forte semelhança com}
  \end{Phonetics}
\end{Entry}

\begin{Entry}{酷斯拉}{14,12,8}{⾣,⽄,⼿}
  \begin{Phonetics}{酷斯拉}{ku4si1la1}
    \definition*{s.}{Godzilla. do Japonês Gojira, ゴジラ}
  \seealsoref{哥斯拉}{ge1si1la1}
  \end{Phonetics}
\end{Entry}

%%%%%%%%%% 酸 %%%%%%%%%%
\subsection*{酸}\addcontentsline{loh}{figure}{酸}

\begin{Entry}{酸}{14}{⾣}
  \begin{Phonetics}{酸}{suan1}[][HSK 4]
    \definition{adj.}{azedo; ácido | aflito; angustiado; doente do coração | pedante; descreve uma pessoa que finge ser culta e também descreve uma pessoa que é muito inflexível com suas próprias ideias e não está disposta a mudá-las para atender às exigências da época, é usado principalmente para satirizar intelectuais que fingem ser capazes de escrever poemas e artigos | ciumento; invejoso; sentimentos desconfortáveis porque outra pessoa é melhor do que você e, em geral, também apresenta comportamento hostil}
    \definition{s.}{ácido; produto químico que tem um sabor ácido quando misturado com água}
    \definition{v.}{estar dolorido (devido à fadiga ou doença); descreve a sensação de não ter força muscular e um pouco de dor por estar doente ou muito cansado}
  \end{Phonetics}
\end{Entry}

\begin{Entry}{酸奶}{14,5}{⾣,⼥}
  \begin{Phonetics}{酸奶}{suan1nai3}[][HSK 4]
    \definition[瓶,杯,盒,袋]{s.}{iogurte; produto lácteo fermentado por bactérias de ácido láctico}
  \end{Phonetics}
\end{Entry}

\begin{Entry}{酸甜苦辣}{14,11,8,14}{⾣,⽢,⾋,⾟}
  \begin{Phonetics}{酸甜苦辣}{suan1-tian2-ku3-la4}[][HSK 5]
    \definition{expr.}{os altos e baixos da vida; as experiências agridoces da vida; os aspectos doces, azedos, amargos e picantes da vida; refere"-se a todos os tipos de sabores, como metáfora para experiências diversas, como felicidade, sofrimento, etc.; azedo, doce, amargo, picante (alegrias e tristezas da vida)}
  \end{Phonetics}
\end{Entry}

\begin{Entry}{酸辣汤}{14,14,6}{⾣,⾟,⽔}
  \begin{Phonetics}{酸辣汤}{suan1la4tang1}
    \definition{s.}{sopa avinagrada e picante (prato)}
  \end{Phonetics}
\end{Entry}

%%%%%%%%%% 酿 %%%%%%%%%%
\subsection*{酿}\addcontentsline{loh}{figure}{酿}

\begin{Entry}{酿}{14}{⾣}
  \begin{Phonetics}{酿}{niang2}
    \definition{v.}{fermentar; preparar (vinho de arroz)}
  \end{Phonetics}
  \begin{Phonetics}{酿}{niang4}
    \definition{s.}{vinho}
    \definition{v.}{fazer (vinho); fabricar (cerveja) | produzir (mel) | levar a; resultar em; formar gradualmente}
  \end{Phonetics}
\end{Entry}

\begin{Entry}{酿造}{14,10}{⾣,⾡}
  \begin{Phonetics}{酿造}{niang4zao4}[][HSK 7-9]
    \definition{v.}{fabricar (cerveja, etc.); produzir (vinho, vinagre, etc.); fabricar utilizando fermentação}
  \end{Phonetics}
\end{Entry}

%%%%%%%%%% 锺 %%%%%%%%%%
\subsection*{锺}\addcontentsline{loh}{figure}{锺}

\begin{Entry}{锺}{14}{⾦}
  \begin{Phonetics}{锺}{zhong1}
    \variantof{钟}
  \end{Phonetics}
\end{Entry}

%%%%%%%%%% 锻 %%%%%%%%%%
\subsection*{锻}\addcontentsline{loh}{figure}{锻}

\begin{Entry}{锻}{14}{⾦}
  \begin{Phonetics}{锻}{duan4}
    \definition{v.}{forjar; moldar}
  \end{Phonetics}
\end{Entry}

\begin{Entry}{锻炼}{14,9}{⾦,⽕}
  \begin{Phonetics}{锻炼}{duan4lian4}[][HSK 4]
    \definition{v.}{exercitar"-se; fazer (ou fazer) exercícios; submeter"-se a treinamento físico; fortalecer o corpo por meio do esporte | fortalecer; endurecer; aprimorar as habilidades de trabalho e de vida por meio de trabalho e outras atividades | forjar ou moldar metal para torná"-lo mais refinado; refere"-se à transformação de materiais metálicos em objetos de determinada forma e tamanho por meio de aquecimento, batimento, prensagem etc.}
  \end{Phonetics}
\end{Entry}

%%%%%%%%%% 镀 %%%%%%%%%%
\subsection*{镀}\addcontentsline{loh}{figure}{镀}

\begin{Entry}{镀}{14}{⾦}
  \begin{Phonetics}{镀}{du4}
    \definition{v.}{cobrir ou revestir (com um metal)}
  \end{Phonetics}
\end{Entry}

\begin{Entry}{镀金}{14,8}{⾦,⾦}
  \begin{Phonetics}{镀金}{du4jin1}
    \definition{v.}{banhar a ouro | dourar | (figurativo) fazer algo muito comum parecer especial}
  \end{Phonetics}
\end{Entry}

%%%%%%%%%% 隧 %%%%%%%%%%
\subsection*{隧}\addcontentsline{loh}{figure}{隧}

\begin{Entry}{隧}{14}{⾩}
  \begin{Phonetics}{隧}{sui4}
    \definition{s.}{túnel; passagem subterrânea | estrada | subúrbios; áreas suburbanas}
    \definition{v.}{virar}
  \end{Phonetics}
\end{Entry}

\begin{Entry}{隧道}{14,12}{⾩,⾡}
  \begin{Phonetics}{隧道}{sui4dao4}
    \definition{s.}{túnel}
  \end{Phonetics}
\end{Entry}

%%%%%%%%%% 需 %%%%%%%%%%
\subsection*{需}\addcontentsline{loh}{figure}{需}

\begin{Entry}{需}{14}{⾬}
  \begin{Phonetics}{需}{xu1}
    \definition*{s.}{Sobrenome: Xu}
    \definition{s.}{necessidades; bens de primeira necessidade}
    \definition{v.}{precisar; querer; exigir}
  \end{Phonetics}
\end{Entry}

\begin{Entry}{需求}{14,7}{⾬,⽔}
  \begin{Phonetics}{需求}{xu1qiu2}[][HSK 3]
    \definition[种]{s.}{necessidades; demanda; exigência; solicitações decorrentes de necessidades}
  \end{Phonetics}
\end{Entry}

\begin{Entry}{需要}{14,9}{⾬,⾑}
  \begin{Phonetics}{需要}{xu1yao4}[][HSK 3]
    \definition[种]{s.}{necessidade; desejo ou exigência em relação a algo}
    \definition{v.}{precisar; querer; exigir; demandar; solicitar}
  \end{Phonetics}
\end{Entry}

%%%%%%%%%% 静 %%%%%%%%%%
\subsection*{静}\addcontentsline{loh}{figure}{静}

\begin{Entry}{静}{14}{⾭}
  \begin{Phonetics}{静}{jing4}[][HSK 3]
    \definition*{s.}{Sobrenome: Jing}
    \definition{adj.}{tranquilo;  sossegado; calmo; imóvel | silencioso; quieto; sem emitir nenhum som | calmo, sereno; serenidade; (interior) paz}
    \definition{v.}{acalmar-se; aquietar-se; tranquilizar (o coração)}
  \end{Phonetics}
\end{Entry}

\begin{Entry}{静止}{14,4}{⾭,⽌}
  \begin{Phonetics}{静止}{jing4zhi3}[][HSK 7-9]
    \definition{adj.}{estático; imóvel; parado; estacionário}
  \end{Phonetics}
\end{Entry}

%%%%%%%%%% 颗 %%%%%%%%%%
\subsection*{颗}\addcontentsline{loh}{figure}{颗}

\begin{Entry}{颗}{14}{⾴}
  \begin{Phonetics}{颗}{ke1}[][HSK 5]
    \definition{clas.}{usado para grãos, pérolas, dentes, corações, satelites, pequenas esferas, etc.}
    \definition{s.}{grão; partícula; pequenas coisas redondas}
  \end{Phonetics}
\end{Entry}

%%%%%%%%%% 馒 %%%%%%%%%%
\subsection*{馒}\addcontentsline{loh}{figure}{馒}

\begin{Entry}{馒}{14}{⾷}
  \begin{Phonetics}{馒}{man2}
    \definition{s.}{pão cozido no vapor}
  \end{Phonetics}
\end{Entry}

\begin{Entry}{馒头}{14,5}{⾷,⼤}
  \begin{Phonetics}{馒头}{man2tou5}[][HSK 6]
    \definition[个,锅,屉,筐]{s.}{pão cozido no vapor; um alimento cozido no vapor feito de farinha fermentada, geralmente redondo na parte superior e plano na parte inferior, sem recheio}
  \end{Phonetics}
\end{Entry}

%%%%%%%%%% 魄 %%%%%%%%%%
\subsection*{魄}\addcontentsline{loh}{figure}{魄}

\begin{Entry}{魄}{14}{⿁}
  \begin{Phonetics}{魄}{bo2}
    \definition{adj.}{desanimado; aflito; abatido}
  \seealsoref{落魄}{luo4bo2}
  \end{Phonetics}
  \begin{Phonetics}{魄}{po4}
    \definition{s.}{alma | vigor; espírito; coragem; energia}
  \end{Phonetics}
  \begin{Phonetics}{魄}{tuo4}
    \definition{adj.}{desanimado; sem ânimo; mentalmente abatido; outra pronúncia de 魄 em 落魄}
  \seealsoref{落魄}{luo4po4}
  \end{Phonetics}
\end{Entry}

\begin{Entry}{魄力}{14,2}{⿁,⼒}
  \begin{Phonetics}{魄力}{po4li4}[][HSK 7-9]
    \definition[种]{s.}{coragem; ousadia; audácia e resolução; refere"-se à coragem e à determinação com que alguém lida com as situações}
  \end{Phonetics}
\end{Entry}

%%%%%%%%%% 魅 %%%%%%%%%%
\subsection*{魅}\addcontentsline{loh}{figure}{魅}

\begin{Entry}{魅}{14}{⿁}
  \begin{Phonetics}{魅}{mei4}
    \definition{s.}{espírito maligno; demônio | \emph{goblin}; trasgo; gnomo; duende maléfico}
    \definition{v.}{atormentar; cativar}
  \end{Phonetics}
\end{Entry}

\begin{Entry}{魅力}{14,2}{⿁,⼒}
  \begin{Phonetics}{魅力}{mei4li4}[][HSK 7-9]
    \definition[种]{s.}{charme; feitiço; glamour; bruxaria; carisma; feitiçaria; encanto; fascínio; encantamento; o poder de atrair e motivar pessoas}
  \end{Phonetics}
\end{Entry}

%%%%%%%%%% 鲜 %%%%%%%%%%
\subsection*{鲜}\addcontentsline{loh}{figure}{鲜}

\begin{Entry}{鲜}{14}{⿂}
  \begin{Phonetics}{鲜}{xian1}[][HSK 4]
    \definition*{s.}{Sobrenome: Xian}
    \definition{adj.}{fresco; novo; fresco (experiência, comida etc.) |brilhante; de cores vivas | saboroso; delicioso | exuberante; luxuriante}
    \definition{s.}{aves e animais recém-abatidos; vegetais recém-colhidos; frutas, etc. | alimentos aquáticos; geralmente, peixes vivos, camarões, etc., para alimentação}
  \end{Phonetics}
  \begin{Phonetics}{鲜}{xian3}
    \definition{adj.}{raro; pouco; pequeno}
    \definition{adv.}{raramente}
  \end{Phonetics}
\end{Entry}

\begin{Entry}{鲜花}{14,7}{⿂,⾋}
  \begin{Phonetics}{鲜花}{xian1hua1}[][HSK 4]
    \definition[朵,束,支]{s.}{flor; flores frescas; flores bonitas e frescas}
  \end{Phonetics}
\end{Entry}

\begin{Entry}{鲜明}{14,8}{⿂,⽇}
  \begin{Phonetics}{鲜明}{xian1ming2}[][HSK 4]
    \definition{adj.}{brilhante (cor) | distinto; bem definido; nítido; claro; característico}
  \end{Phonetics}
\end{Entry}

\begin{Entry}{鲜艳}{14,10}{⿂,⾊}
  \begin{Phonetics}{鲜艳}{xian1yan4}[][HSK 5]
    \definition{adj.}{de cores alegres; de cores brilhantes}
  \end{Phonetics}
\end{Entry}

%%%%%%%%%% 鼻 %%%%%%%%%%
\subsection*{鼻}\addcontentsline{loh}{figure}{鼻}

\begin{Entry}{鼻}{14}{⿐}[Kangxi 209]
  \begin{Phonetics}{鼻}{bi2}
    \definition{s.}{nariz}
  \end{Phonetics}
\end{Entry}

\begin{Entry}{鼻子}{14,3}{⿐,⼦}
  \begin{Phonetics}{鼻子}{bi2zi5}[][HSK 5]
    \definition[个,只]{s.}{nariz; órgão da face, responsável pela respiração e pelo olfato}
  \end{Phonetics}
\end{Entry}

\begin{Entry}{鼻涕}{14,10}{⿐,⽔}
  \begin{Phonetics}{鼻涕}{bi2ti4}[][HSK 7-9]
    \definition[些,点]{s.}{ranho; muco nasal; secreção nasal; fluido secretado pela mucosa nasal}
  \end{Phonetics}
\end{Entry}

%%%%% EOF %%%%%


 %%%
%%% 15画
%%%
\section*{15画}\addcontentsline{toc}{section}{15画}\addcontentsline{loh}{figure}{\#\#\#\# 15画}

%%%%%%%%%% 僵 %%%%%%%%%%
\subsection*{僵}\addcontentsline{loh}{figure}{僵}

\begin{Entry}{僵}{15}{⼈}
  \begin{Phonetics}{僵}{jiang1}[][HSK 7-9]
    \definition{adj.}{rígido; paralisado; congelado | rígido; austero; duro | Dialeto: em impasse; tenso}
    \definition{v.}{Dialeto: parar de sorrir; ficar com uma expressão séria}
  \end{Phonetics}
\end{Entry}

\begin{Entry}{僵化}{15,4}{⼈、⼔}
  \begin{Phonetics}{僵化}{jiang1hua4}[][HSK 7-9]
    \definition{v.}{tornar-se rígido; ossificar; tornar-se estereotipado; parar de se desenvolver; petrificar; inativar}
  \end{Phonetics}
\end{Entry}

\begin{Entry}{僵局}{15,7}{⼈、⼫}
  \begin{Phonetics}{僵局}{jiang1ju2}[][HSK 7-9]
    \definition[个,种]{s.}{impasse; beco sem saída; a questão é difícil de resolver e o progresso está paralisado}
  \end{Phonetics}
\end{Entry}

%%%%%%%%%% 嘱 %%%%%%%%%%
\subsection*{嘱}\addcontentsline{loh}{figure}{嘱}

\begin{Entry}{嘱}{15}{⼝}
  \begin{Phonetics}{嘱}{zhu3}
    \definition{v.}{juntar-se | implorar | incitar}
  \end{Phonetics}
\end{Entry}

\begin{Entry}{嘱托}{15,6}{⼝、⼿}
  \begin{Phonetics}{嘱托}{zhu3tuo1}
    \definition{v.}{confiar uma tarefa a alguém}
  \end{Phonetics}
\end{Entry}

\begin{Entry}{嘱咐}{15,8}{⼝、⼝}
  \begin{Phonetics}{嘱咐}{zhu3fu5}
    \definition{v.}{ordenar | dizer | exortar}
  \end{Phonetics}
\end{Entry}

%%%%%%%%%% 嘲 %%%%%%%%%%
\subsection*{嘲}\addcontentsline{loh}{figure}{嘲}

\begin{Entry}{嘲}{15}{⼝}
  \begin{Phonetics}{嘲}{chao2}
    \definition{v.}{ridicularizar; zombar; fazer piada de}
  \end{Phonetics}
  \begin{Phonetics}{嘲}{zhao1}
    \definition{s.}{Onomatopéia: barulho clamoroso feito por várias pessoas falando ou cantando, ou por instrumentos musicais, ou pássaros cantando; descreve um som caótico e fragmentado}
  \end{Phonetics}
\end{Entry}

\begin{Entry}{嘲弄}{15,7}{⼝、⼶}
  \begin{Phonetics}{嘲弄}{chao2nong4}[][HSK 7-9]
    \definition{v.}{zombar; zombar de}
  \end{Phonetics}
\end{Entry}

\begin{Entry}{嘲笑}{15,10}{⼝、⽵}
  \begin{Phonetics}{嘲笑}{chao2xiao4}[][HSK 7-9]
    \definition{v.}{ridicularizar; zombar; rir de; zombar de; fazer graça de; usar palavras para zombar de alguém}
  \end{Phonetics}
\end{Entry}

%%%%%%%%%% 嘹 %%%%%%%%%%
\subsection*{嘹}\addcontentsline{loh}{figure}{嘹}

\begin{Entry}{嘹}{15}{⼝}
  \begin{Phonetics}{嘹}{liao2}
    \definition{adj.}{(som) alto e claro | som claro | grito (de guindastes, etc.)}
  \end{Phonetics}
\end{Entry}

\begin{Entry}{嘹亮}{15,9}{⼝、⼇}
  \begin{Phonetics}{嘹亮}{liao2liang4}
    \definition{adj.}{ressonante; alto e claro}
  \end{Phonetics}
\end{Entry}

%%%%%%%%%% 嘿 %%%%%%%%%%
\subsection*{嘿}\addcontentsline{loh}{figure}{嘿}

\begin{Entry}{嘿}{15}{⼝}
  \begin{Phonetics}{嘿}{hei1}[][HSK 7-9]
    \definition{interj.}{Ei!; indicando uma saudação ou chamar a atenção | expressando orgulho ou satisfação | expressando espanto, surpresa}
  \end{Phonetics}
  \begin{Phonetics}{嘿}{mo4}
    \definition{adj.}{quieto; silencioso; tácito}
  \end{Phonetics}
\end{Entry}

%%%%%%%%%% 噎 %%%%%%%%%%
\subsection*{噎}\addcontentsline{loh}{figure}{噎}

\begin{Entry}{噎}{15}{⼝}
  \begin{Phonetics}{噎}{ye1}
    \definition{v.}{engasgar | sufocar}
  \end{Phonetics}
\end{Entry}

%%%%%%%%%% 增 %%%%%%%%%%
\subsection*{增}\addcontentsline{loh}{figure}{增}

\begin{Entry}{增}{15}{⼟}
  \begin{Phonetics}{增}{zeng1}[][HSK 5]
    \definition*{s.}{Sobrenome: Zeng}
    \definition{v.}{aumentar; ganhar; adicionar}
  \end{Phonetics}
\end{Entry}

\begin{Entry}{增大}{15,3}{⼟、⼤}
  \begin{Phonetics}{增大}{zeng1 da4}[][HSK 5]
    \definition{v.}{ampliar; expandir; estender | amplificar}
  \end{Phonetics}
\end{Entry}

\begin{Entry}{增长}{15,4}{⼟、⾧}
  \begin{Phonetics}{增长}{zeng1 zhang3}[][HSK 3]
    \definition{v.}{subir; crescer; aumentar; melhorar a partir da base existente}
  \end{Phonetics}
\end{Entry}

\begin{Entry}{增加}{15,5}{⼟、⼒}
  \begin{Phonetics}{增加}{zeng1jia1}[][HSK 3]
    \definition{v.}{adicionar; aumentar; incrementar; adicionar mais ao que já existe}
  \end{Phonetics}
\end{Entry}

\begin{Entry}{增产}{15,6}{⼟、⼇}
  \begin{Phonetics}{增产}{zeng1/chan3}[][HSK 5]
    \definition{v.+compl.}{aumentar a produção}
  \end{Phonetics}
\end{Entry}

\begin{Entry}{增多}{15,6}{⼟、⼣}
  \begin{Phonetics}{增多}{zeng1 duo1}[][HSK 5]
    \definition{v.}{aumentar; crescer em número ou quantidade}
  \end{Phonetics}
\end{Entry}

\begin{Entry}{增进}{15,7}{⼟、⾡}
  \begin{Phonetics}{增进}{zeng1 jin4}[][HSK 6]
    \definition{v.}{melhorar; promover; aprofundar}
  \end{Phonetics}
\end{Entry}

\begin{Entry}{增值}{15,10}{⼟、⼈}
  \begin{Phonetics}{增值}{zeng1 zhi2}[][HSK 6]
    \definition{s.}{aumento de valor; apreciação; incremento | valor agregado}
  \end{Phonetics}
\end{Entry}

\begin{Entry}{增速}{15,10}{⼟、⾡}
  \begin{Phonetics}{增速}{zeng1su4}
    \definition{s.}{Economia: taxa de crescimento}
    \definition{v.}{acelerar; aumentar a velocidade}
  \end{Phonetics}
\end{Entry}

\begin{Entry}{增强}{15,12}{⼟、⼸}
  \begin{Phonetics}{增强}{zeng1 qiang2}[][HSK 5]
    \definition{v.}{impulsionar; aprimorar; aumentar; fortalecer; tornar mais forte ou mais poderoso}
  \end{Phonetics}
\end{Entry}

%%%%%%%%%% 墨 %%%%%%%%%%
\subsection*{墨}\addcontentsline{loh}{figure}{墨}

\begin{Entry}{墨}{15}{⿊}
  \begin{Phonetics}{墨}{mo4}[][HSK 7-9]
    \definition*{s.}{Escola Moísta; Moísmo | México, abreviação de 墨西哥}
    \definition{adj.}{preto; escuro como breu | corrupto | escuro}
    \definition{s.}{tinta chinesa; bastão de tinta | pigmento; tinta | caligrafia ou pintura | aprendizagem; alfabetização | marcador de linha de carpinteiro; marcador de tinta | tatuar o rosto (um castigo); uma punição na China antiga | corrupção; peculato; fraude}
  \seealsoref{墨西哥}{mo4xi1ge1}
  \end{Phonetics}
\end{Entry}

\begin{Entry}{墨水}{15,4}{⿊、⽔}
  \begin{Phonetics}{墨水}{mo4 shui3}[][HSK 6]
    \definition[瓶]{s.}{tinta chinesa preparada; tinta (para caneta-tinteiro) | aprendizagem; alfabetização; uma metáfora para o conhecimento ou a capacidade de ler e escrever}
  \end{Phonetics}
\end{Entry}

\begin{Entry}{墨西哥}{15,6,10}{⿊、⾑、⼝}
  \begin{Phonetics}{墨西哥}{mo4xi1ge1}
    \definition*{s.}{México; Planalto no México}
  \end{Phonetics}
\end{Entry}

\begin{Entry}{墨镜}{15,16}{⿊、⾦}
  \begin{Phonetics}{墨镜}{mo4jing4}
    \definition[只,双,副]{s.}{óculos escuros}
  \end{Phonetics}
\end{Entry}

%%%%%%%%%% 履 %%%%%%%%%%
\subsection*{履}\addcontentsline{loh}{figure}{履}

\begin{Entry}{履}{15}{⼫}
  \begin{Phonetics}{履}{lv3}
    \definition{s.}{sapato | pegada}
    \definition{v.}{pisar em; caminhar sobre | executar; cumprir; honrar; completar}
  \end{Phonetics}
\end{Entry}

\begin{Entry}{履行}{15,6}{⼫、⾏}
  \begin{Phonetics}{履行}{lv3xing2}[][HSK 7-9]
    \definition{v.}{cumprir; executar; realizar; a execução de contratos, acordos, promessas, responsabilidades, etc.}
  \end{Phonetics}
\end{Entry}

%%%%%%%%%% 影 %%%%%%%%%%
\subsection*{影}\addcontentsline{loh}{figure}{影}

\begin{Entry}{影}{15}{⼺}
  \begin{Phonetics}{影}{ying3}
    \definition*{s.}{Sobrenome: Ying}
    \definition{s.}{sombra | reflexão; imagem | traço; sinal; impressão vaga | fotografia; imagem | filme | jogo de sombras; pantomima de sombra}
    \definition{v.}{(dialeto) esconder; ocultar | copiar; rastrear | fotocopiar}
  \end{Phonetics}
\end{Entry}

\begin{Entry}{影子}{15,3}{⼺、⼦}
  \begin{Phonetics}{影子}{ying3zi5}[][HSK 4]
    \definition[个,片]{s.}{sombra; imagem projetada por um objeto, etc., que bloqueia a luz | reflexão; reflexo; imagem de um objeto, etc., conforme aparece em um refletor, como um espelho, uma superfície de água, etc. | sinal; vestígio; vaga impressão}
  \end{Phonetics}
\end{Entry}

\begin{Entry}{影片}{15,4}{⼺、⽚}
  \begin{Phonetics}{影片}{ying3 pian4}[][HSK 2]
    \definition[部,盘,盒,卷]{s.}{filme; imagem | filme; película usada para reproduzir filmes}
  \end{Phonetics}
\end{Entry}

\begin{Entry}{影视}{15,8}{⼺、⾒}
  \begin{Phonetics}{影视}{ying3 shi4}[][HSK 3]
    \definition{s.}{cinema e televisão combinados; denominação conjunta para cinema e TV}
  \end{Phonetics}
\end{Entry}

\begin{Entry}{影响}{15,9}{⼺、⼝}
  \begin{Phonetics}{影响}{ying3xiang3}[][HSK 2]
    \definition{s.}{efeito; influência; efeitos sobre pessoas ou coisas}
    \definition{v.}{afetar; influenciar; influência sobre os pensamentos ou ações dos outros}
  \end{Phonetics}
\end{Entry}

\begin{Entry}{影响力}{15,9,2}{⼺、⼝、⼒}
  \begin{Phonetics}{影响力}{ying3 xiang3 li4}[][HSK 6]
    \definition{s.}{impacto | influência}
  \end{Phonetics}
\end{Entry}

\begin{Entry}{影星}{15,9}{⼺、⽇}
  \begin{Phonetics}{影星}{ying3 xing1}[][HSK 6]
    \definition{s.}{estrela de cinema}
  \end{Phonetics}
\end{Entry}

\begin{Entry}{影迷}{15,9}{⼺、⾡}
  \begin{Phonetics}{影迷}{ying3 mi2}[][HSK 6]
    \definition[个,名,位]{s.}{fã de cinema; entusiasta de cinema; pessoas viciadas em assistir filmes}
  \end{Phonetics}
\end{Entry}

\begin{Entry}{影像}{15,13}{⼺、⼈}
  \begin{Phonetics}{影像}{ying3xiang4}
    \definition{s.}{imagem}
  \end{Phonetics}
\end{Entry}

%%%%%%%%%% 德 %%%%%%%%%%
\subsection*{德}\addcontentsline{loh}{figure}{德}

\begin{Entry}{德}{15}{⼻}
  \begin{Phonetics}{德}{de2}[][HSK 7-9]
    \definition*{s.}{Alemanha, abreviação de 德国 | Sobrenome: De}
    \definition{s.}{virtude; moral; caráter moral; moralidade; conduta; qualidades políticas | coração; mente; pensamentos | bondade; favor; graça}
  \seealsoref{德国}{de2guo2}
  \end{Phonetics}
\end{Entry}

\begin{Entry}{德国}{15,8}{⼻、⼞}
  \begin{Phonetics}{德国}{de2guo2}
    \definition*{s.}{Alemanha}
  \end{Phonetics}
\end{Entry}

\begin{Entry}{德国人}{15,8,2}{⼻、⼞、⼈}
  \begin{Phonetics}{德国人}{de2guo2ren2}
    \definition{s.}{alemão | pessoa ou povo da Alemanha}
  \end{Phonetics}
\end{Entry}

%%%%%%%%%% 慰 %%%%%%%%%%
\subsection*{慰}\addcontentsline{loh}{figure}{慰}

\begin{Entry}{慰}{15}{⼼}
  \begin{Phonetics}{慰}{wei4}
    \definition{adj.}{aliviado; em paz; confortável}
    \definition{v.}{consolar; confortar | ser (ficar) aliviado}
  \end{Phonetics}
\end{Entry}

\begin{Entry}{慰问}{15,6}{⼼、⾨}
  \begin{Phonetics}{慰问}{wei4wen4}[][HSK 5]
    \definition{v.}{visitar; consolar; expressar simpatia por; confortar e cumprimentar com palavras e presentes;  enfatizar o conforto e o cumprimento, frequentemente usado por superiores para subordinados}
  \end{Phonetics}
\end{Entry}

%%%%%%%%%% 憋 %%%%%%%%%%
\subsection*{憋}\addcontentsline{loh}{figure}{憋}

\begin{Entry}{憋}{15}{⼼}
  \begin{Phonetics}{憋}{bie1}[][HSK 7-9]
    \definition{adj.}{sufocado; oprimido}
    \definition{v.}{suprimir; conter | Dialeto: obrigar | Dialeto: ponderar; contemplar | Dialeto: ficar de olho em | Dialeto: destruir (por pressão interna) | calar a boca; inibir; bloquear | sufocar; abafar}
  \end{Phonetics}
\end{Entry}

%%%%%%%%%% 憧 %%%%%%%%%%
\subsection*{憧}\addcontentsline{loh}{figure}{憧}

\begin{Entry}{憧}{15}{⼼}
  \begin{Phonetics}{憧}{chong1}
    \definition{adj.}{irresoluto; indeciso | estúpido; imbecil; confuso}
  \end{Phonetics}
\end{Entry}

\begin{Entry}{憧憬}{15,15}{⼼、⼼}
  \begin{Phonetics}{憧憬}{chong1jing3}
    \definition{v.}{ansiar por | esperar por}
  \end{Phonetics}
\end{Entry}

%%%%%%%%%% 懂 %%%%%%%%%%
\subsection*{懂}\addcontentsline{loh}{figure}{懂}

\begin{Entry}{懂}{15}{⼼}
  \begin{Phonetics}{懂}{dong3}[][HSK 2]
    \definition*{s.}{Sobrenome: Dong}
    \definition{v.}{compreender; entender}
  \end{Phonetics}
\end{Entry}

\begin{Entry}{懂事}{15,8}{⼼、⼅}
  \begin{Phonetics}{懂事}{dong3shi4}[][HSK 7-9]
    \definition{adj.}{sensato; inteligente; muito compreensivo da natureza e da razão humana}
  \end{Phonetics}
\end{Entry}

\begin{Entry}{懂得}{15,11}{⼼、⼻}
  \begin{Phonetics}{懂得}{dong3 de5}[][HSK 2]
    \definition{v.}{saber (significado, prática, etc.); compreender; entender}
  \end{Phonetics}
\end{Entry}

%%%%%%%%%% 摩 %%%%%%%%%%
\subsection*{摩}\addcontentsline{loh}{figure}{摩}

\begin{Entry}{摩}{15}{⼿}
  \begin{Phonetics}{摩}{mo2}
    \definition{v.}{esfregar; raspar; tocar | refletir; estudar | afagar}
  \end{Phonetics}
\end{Entry}

\begin{Entry}{摩托}{15,6}{⼿、⼿}
  \begin{Phonetics}{摩托}{mo2 tuo1}[][HSK 5]
    \definition[辆]{s.}{Empréstimo linguístico: motor; motor de combustão interna | Empréstimo linguístico: motocicleta, abreviação de 摩托车}
  \seealsoref{摩托车}{mo2tuo1che1}
  \end{Phonetics}
\end{Entry}

\begin{Entry}{摩托车}{15,6,4}{⼿、⼿、⾞}
  \begin{Phonetics}{摩托车}{mo2tuo1che1}
    \definition[辆,部]{s.}{(empréstimo linguístico) motocicleta}
  \end{Phonetics}
\end{Entry}

\begin{Entry}{摩擦}{15,17}{⼿、⼿}
  \begin{Phonetics}{摩擦}{mo2ca1}[][HSK 5]
    \definition{s.}{atrito; desacordo; conflito (entre duas partes); a ação de impedir o movimento relativo entre dois objetos em contato, produzida na superfície de contato | atrito; metáfora para o conflito entre as duas partes}
    \definition{v.}{esfregar}
  \end{Phonetics}
\end{Entry}

%%%%%%%%%% 撑 %%%%%%%%%%
\subsection*{撑}\addcontentsline{loh}{figure}{撑}

\begin{Entry}{撑}{15}{⼿}
  \begin{Phonetics}{撑}{cheng1}[][HSK 6]
    \definition{s.}{suporte; escora;  apoio; esteio}
    \definition{v.}{sustentar; apoiar; resistir a | empurrar (ou mover) com uma vara; usar um mastro para empurrar a margem ou o leito do rio para fazer o barco avançar | manter; manter-se atualizado | abrir; desdobrar; expandir (um objeto contraído) | encher até estourar (inchaço devido a excesso de comida ou alimentação excessiva)}
  \end{Phonetics}
\end{Entry}

%%%%%%%%%% 撒 %%%%%%%%%%
\subsection*{撒}\addcontentsline{loh}{figure}{撒}

\begin{Entry}{撒}{15}{⼿}
  \begin{Phonetics}{撒}{sa1}
    \definition{v.}{lançar; deixar ir; deixar sair; liberar | livrar-se de todas as restrições; deixar-se levar; tentar usá-lo ou exibi-lo o máximo possível}
  \end{Phonetics}
\end{Entry}

\begin{Entry}{撒旦}{15,5}{⼿、⽇}
  \begin{Phonetics}{撒旦}{sa1dan4}
    \definition*{s.}{Satã}
  \end{Phonetics}
\end{Entry}

\begin{Entry}{撒旦主义}{15,5,5,3}{⼿、⽇、⼂、⼂}
  \begin{Phonetics}{撒旦主义}{sa1dan4 zhu3yi4}
    \definition*{s.}{Satanismo}
  \end{Phonetics}
\end{Entry}

\begin{Entry}{撒但}{15,7}{⼿、⼈}
  \begin{Phonetics}{撒但}{sa1dan4}
    \variantof{撒旦}
  \end{Phonetics}
\end{Entry}

%%%%%%%%%% 撞 %%%%%%%%%%
\subsection*{撞}\addcontentsline{loh}{figure}{撞}

\begin{Entry}{撞}{15}{⼿}
  \begin{Phonetics}{撞}{zhuang4}[][HSK 5]
    \definition{v.}{chocar-se contra; chocar-se com; bater; colidir | encontrar-se por acaso; esbarrar em; deparar-se com | apressar; correr; empurrar | aproveitar a chance | esbarrar de repente em |  encontrar | confiar em; tentar | agir precipitadamente; invadir}
  \end{Phonetics}
\end{Entry}

\begin{Entry}{撞车}{15,4}{⼿、⾞}
  \begin{Phonetics}{撞车}{zhuang4/che1}
    \definition{v.+compl.}{(figurativo) colidir (opiniões, cronogramas, etc.) | ser o mesmo (assunto) | colidir (com outro veículo)}
  \end{Phonetics}
\end{Entry}

\begin{Entry}{撞运气}{15,7,4}{⼿、⾡、⽓}
  \begin{Phonetics}{撞运气}{zhuang4yun4qi5}
    \definition{v.}{confiar no destino | tentar a sorte}
  \end{Phonetics}
\end{Entry}

%%%%%%%%%% 撤 %%%%%%%%%%
\subsection*{撤}\addcontentsline{loh}{figure}{撤}

\begin{Entry}{撤}{15}{⼿}
  \begin{Phonetics}{撤}{che4}[][HSK 7-9]
    \definition{v.}{remover, tirar | demitir; liberar | retirar-se; evacuar}
  \end{Phonetics}
\end{Entry}

\begin{Entry}{撤换}{15,10}{⼿、⼿}
  \begin{Phonetics}{撤换}{che4huan4}[][HSK 7-9]
    \definition{v.}{demitir e substituir (alguém); revogar; substituir (alguém ou alguma coisa)}
  \end{Phonetics}
\end{Entry}

\begin{Entry}{撤离}{15,10}{⼿、⼇}
  \begin{Phonetics}{撤离}{che4 li2}[][HSK 6]
    \definition{v.}{retirar-se de; deixar; evacuar}
  \end{Phonetics}
\end{Entry}

\begin{Entry}{撤销}{15,12}{⼿、⾦}
  \begin{Phonetics}{撤销}{che4xiao1}[][HSK 6]
    \definition{v.}{cancelar; rescindir; revogar; remover}
  \end{Phonetics}
\end{Entry}

%%%%%%%%%% 播 %%%%%%%%%%
\subsection*{播}\addcontentsline{loh}{figure}{播}

\begin{Entry}{播}{15}{⼿}
  \begin{Phonetics}{播}{bo1}[][HSK 6]
    \definition{v.}{espalhar; transmitir | semear | mover-se; migrar; ir para o exílio}
  \end{Phonetics}
\end{Entry}

\begin{Entry}{播出}{15,5}{⼿、⼐}
  \begin{Phonetics}{播出}{bo1 chu1}[][HSK 3]
    \definition{v.}{radiodifundir; transmitir; estar no ar; transmitir via rádio e televisão}
  \end{Phonetics}
\end{Entry}

\begin{Entry}{播放}{15,8}{⼿、⽅}
  \begin{Phonetics}{播放}{bo1fang4}[][HSK 3]
    \definition{v.}{ir ao ar; transmitir por rádio | mostrar; exibir; transmitir (um programa de TV)}
  \end{Phonetics}
\end{Entry}

\begin{Entry}{播音}{15,9}{⼿、⾳}
  \begin{Phonetics}{播音}{bo1/yin1}
    \definition{s.}{transmissão}
    \definition{v.+compl.}{transmitir}
  \end{Phonetics}
\end{Entry}

%%%%%%%%%% 擒 %%%%%%%%%%
\subsection*{擒}\addcontentsline{loh}{figure}{擒}

\begin{Entry}{擒}{15}{⼿}
  \begin{Phonetics}{擒}{qin2}
    \definition{v.}{capturar; pegar; apreender}
  \end{Phonetics}
\end{Entry}

\begin{Entry}{擒获}{15,10}{⼿、⾋}
  \begin{Phonetics}{擒获}{qin2huo4}
    \definition{v.}{apreender | capturar}
  \end{Phonetics}
\end{Entry}

%%%%%%%%%% 敷 %%%%%%%%%%
\subsection*{敷}\addcontentsline{loh}{figure}{敷}

\begin{Entry}{敷}{15}{⽁}
  \begin{Phonetics}{敷}{fu1}[][HSK 7-9]
    \definition*{s.}{Sobrenome: Fu}
    \definition{v.}{aplicar (pó, pomada, etc.) | espalhar; dispor | ser suficiente para | espalhar-se}
  \end{Phonetics}
\end{Entry}

%%%%%%%%%% 暴 %%%%%%%%%%
\subsection*{暴}\addcontentsline{loh}{figure}{暴}

\begin{Entry}{暴}{15}{⽇}
  \begin{Phonetics}{暴}{bao4}
    \definition*{s.}{Sobrenome: Bao}
    \definition{adj.}{repentino e violento | cruel; selvagem; feroz | temperamental | severo e tirânico; brutal | irritável; irascível; impaciente}
    \definition{adv.}{de repente e ferozmente}
    \definition{s.}{violência; ferocidade}
    \definition{v.}{sobressair; destacar-se; inchar | expor; transmitir | desperdiçar; arruinar; estragar}
  \end{Phonetics}
\end{Entry}

\begin{Entry}{暴力}{15,2}{⽇、⼒}
  \begin{Phonetics}{暴力}{bao4li4}[][HSK 6]
    \definition{s.}{violência; força (usada em tempos de conflito); poder de coerção}
  \end{Phonetics}
\end{Entry}

\begin{Entry}{暴风雨}{15,4,8}{⽇、⾵、⾬}
  \begin{Phonetics}{暴风雨}{bao4 feng1 yu3}[][HSK 6]
    \definition{s.}{tempestade; tormenta; temporal; borrasca; vento e chuva fortes e violentos}
  \end{Phonetics}
\end{Entry}

\begin{Entry}{暴风骤雨}{15,4,17,8}{⽇、⾵、⾺、⾬}
  \begin{Phonetics}{暴风骤雨}{bao4feng1-zhou4yu3}[][HSK 7-9]
    \definition{expr.}{tempestade violenta; furacão; tempestade | vento violento e tempestade de chuva}
  \end{Phonetics}
\end{Entry}

\begin{Entry}{暴行}{15,6}{⽇、⾏}
  \begin{Phonetics}{暴行}{bao4xing2}
    \definition{s.}{ato selvagem | atrocidade | indignação}
  \end{Phonetics}
\end{Entry}

\begin{Entry}{暴乱}{15,7}{⽇、⼄}
  \begin{Phonetics}{暴乱}{bao4luan4}
    \definition{s.}{rebelião | revolta | tumulto}
  \end{Phonetics}
\end{Entry}

\begin{Entry}{暴利}{15,7}{⽇、⼑}
  \begin{Phonetics}{暴利}{bao4li4}[][HSK 7-9]
    \definition{s.}{lucros enormes repentinos | lucros exorbitantes; lucros extravagantes; lucros excessivos}
  \end{Phonetics}
\end{Entry}

\begin{Entry}{暴雨}{15,8}{⽇、⾬}
  \begin{Phonetics}{暴雨}{bao4yu3}[][HSK 6]
    \definition[场,次,阵]{s.}{tempestade; chuva torrencial; chuva forte com precipitação intensa; em meteorologia, refere-se a chuvas de 16 mm ou mais em uma hora ou 50 mm ou mais em 24 horas}
  \end{Phonetics}
\end{Entry}

\begin{Entry}{暴躁}{15,20}{⽇、⾜}
  \begin{Phonetics}{暴躁}{bao4zao4}[][HSK 7-9]
    \definition{adj.}{irascível; febril; irritável; temperamental; descreve uma pessoa que é impaciente, não consegue controlar suas emoções e fica com raiva facilmente}
  \end{Phonetics}
\end{Entry}

\begin{Entry}{暴露}{15,21}{⽇、⾬}
  \begin{Phonetics}{暴露}{bao4lu4}[][HSK 6]
    \definition{adj.}{reveladoras (roupas inadequadas que expõem muito o corpo)}
    \definition{v.}{expor; desnudar; revelar; tornar público algo oculto}
  \end{Phonetics}
\end{Entry}

%%%%%%%%%% 槽 %%%%%%%%%%
\subsection*{槽}\addcontentsline{loh}{figure}{槽}

\begin{Entry}{槽}{15}{⽊}
  \begin{Phonetics}{槽}{cao2}[][HSK 7-9]
    \definition{clas.}{usado para portas | usado para porcos}
    \definition[个,道]{s.}{cocho | sulco; entalhe | canal | manjedoura (para água, ração animal, vinho, cuba); um recipiente para alimentar o gado, geralmente é retangular, alto em todos os lados e côncavo no meio, como uma caixa sem tampa | tanque de fermentação; cuba de vinho; geralmente se refere a certos utensílios com lados altos e côncavos no meio | leito do rio; fossa; refere-se a certos cursos d'água ou valas com lados altos e um meio côncavo | ranhura; fenda; uma depressão semelhante a um sulco em um objeto}
  \end{Phonetics}
\end{Entry}

%%%%%%%%%% 横 %%%%%%%%%%
\subsection*{横}\addcontentsline{loh}{figure}{横}

\begin{Entry}{横}{15}{⽊}
  \begin{Phonetics}{横}{heng2}[][HSK 6]
    \definition{adj.}{horizontal; transversal; paralelo ao plano horizontal (oposto de 竖 e 直) | em ângulo reto com; direção esquerda-direita (em oposição à 竖, 直 ou 纵) | e leste a oeste ou de oeste a leste; direção leste-oeste (oposta a 纵) | desenfreado; turbulento | violento; feroz; irracional}
    \definition{adv.}{de qualquer forma; em qualquer caso | provavelmente; muito provavelmente}
    \definition{s.}{traço horizontal (em caracteres chineses)}
    \definition{v.}{deitar-se transversalmente; estar de lado | colocar algo transversalmente (ou horizontalmente)}
  \seealsoref{竖}{shu4}
  \seealsoref{直}{zhi2}
  \seealsoref{纵}{zong4}
  \end{Phonetics}
  \begin{Phonetics}{横}{heng4}[][HSK 7-9]
    \definition{adj.}{chocante e irracional; inesperado}
  \end{Phonetics}
\end{Entry}

\begin{Entry}{横七竖八}{15,2,9,2}{⽊、⼀、⽴、⼋}
  \begin{Phonetics}{横七竖八}{heng2qi1-shu4ba1}[][HSK 7-9]
    \definition{expr.}{em desordem; em seis e sete; desorganizado}
  \end{Phonetics}
\end{Entry}

\begin{Entry}{横向}{15,6}{⽊、⼝}
  \begin{Phonetics}{横向}{heng2xiang4}[][HSK 7-9]
    \definition{adj.}{horizontal; transversal (oposto a 竖向,纵向) | lateral | ortogonal | perpendicular}
  \seealsoref{竖向}{shu4xiang4}
  \seealsoref{纵向}{zong4xiang4}
  \end{Phonetics}
\end{Entry}

\begin{Entry}{横竖}{15,9}{⽊、⽴}
  \begin{Phonetics}{横竖}{heng2shu5}
    \definition{adv.}{de qualquer forma; em qualquer maneira; isso significa que não importa o que aconteça, o resultado ou a conclusão não mudará; equivale a 反正}
  \seealsoref{反正}{fan3zheng4}
  \end{Phonetics}
\end{Entry}

%%%%%%%%%% 樱 %%%%%%%%%%
\subsection*{樱}\addcontentsline{loh}{figure}{樱}

\begin{Entry}{樱}{15}{⽊}
  \begin{Phonetics}{樱}{ying1}
    \definition[个,棵,朵]{s.}{cereja | cerejeira oriental; flores de cerejeira}
  \end{Phonetics}
\end{Entry}

\begin{Entry}{樱桃}{15,10}{⽊、⽊}
  \begin{Phonetics}{樱桃}{ying1tao2}
    \definition{s.}{cereja}
  \end{Phonetics}
\end{Entry}

%%%%%%%%%% 橄 %%%%%%%%%%
\subsection*{橄}\addcontentsline{loh}{figure}{橄}

\begin{Entry}{橄}{15}{⽊}
  \begin{Phonetics}{橄}{gan3}
    \definition*{s.}{Sobrenome: Gan}
  \end{Phonetics}
\end{Entry}

\begin{Entry}{橄榄球}{15,13,11}{⽊、⽊、⽟}
  \begin{Phonetics}{橄榄球}{gan3lan3qiu2}
    \definition{s.}{futebol jogado com bola oval (rúgbi, futebol americano, regras australianas, etc.)}
  \end{Phonetics}
\end{Entry}

%%%%%%%%%% 潜 %%%%%%%%%%
\subsection*{潜}\addcontentsline{loh}{figure}{潜}

\begin{Entry}{潜}{15}{⽔}
  \begin{Phonetics}{潜}{qian2}
    \definition*{s.}{Sobrenome: Qian}
    \definition{adj.}{latente; oculto}
    \definition{adv.}{furtivamente; secretamente; às escondidas}
    \definition{v.}{ir para debaixo d'água; esconder-se debaixo d'água; mergulhar | esconder | vadear (atravessar) na água | enterrar | fugir de casa}
  \end{Phonetics}
\end{Entry}

\begin{Entry}{潜力}{15,2}{⽔、⼒}
  \begin{Phonetics}{潜力}{qian2li4}[][HSK 6]
    \definition{s.}{potencial; potencialidade; capacidade latente; as habilidades e possibilidades de desenvolvimento que as pessoas e as coisas ainda não demonstraram}
  \end{Phonetics}
\end{Entry}

\begin{Entry}{潜在}{15,6}{⽔、⼟}
  \begin{Phonetics}{潜在}{qian2zai4}
    \definition{adj.}{oculto | latente}
    \definition{s.}{potencial}
  \end{Phonetics}
\end{Entry}

%%%%%%%%%% 潦 %%%%%%%%%%
\subsection*{潦}\addcontentsline{loh}{figure}{潦}

\begin{Entry}{潦}{15}{⽔}
  \begin{Phonetics}{潦}{lao3}
    \definition{s.}{Literário: água da chuva; chuva forte | Literário: poças nas estradas | Literário: inundado}
  \end{Phonetics}
  \begin{Phonetics}{潦}{liao2}
    \definition{s.}{rabisco; garrancho; garatuja; caligrafia desleixada ou descuidada}
  \end{Phonetics}
\end{Entry}

\begin{Entry}{潦草}{15,9}{⽔、⾋}
  \begin{Phonetics}{潦草}{liao2cao3}[][HSK 7-9]
    \definition{adj.}{ilegível; apressado e descuidado (na caligrafia) | desleixado; descuidado; descuidado e pouco sério ao realizar as coisas}
  \end{Phonetics}
\end{Entry}

%%%%%%%%%% 潮 %%%%%%%%%%
\subsection*{潮}\addcontentsline{loh}{figure}{潮}

\begin{Entry}{潮}{15}{⽔}
  \begin{Phonetics}{潮}{chao2}[][HSK 4]
    \definition{adj.}{úmido; molhado | inferior; de qualidade ruim | inferior; não muito habilidoso}
    \definition{s.}{maré; água da maré | surto; corrente; maré; uma metáfora para mudanças sociais em grande escala ou para os altos e baixos de um movimento (social)}
    \definition{s.}{Chaozhou, uma cidade na província de Guangdong}
  \end{Phonetics}
\end{Entry}

\begin{Entry}{潮流}{15,10}{⽔、⽔}
  \begin{Phonetics}{潮流}{chao2liu2}[][HSK 4]
    \definition[种,股,个]{s.}{maré; corrente de maré; movimento da água devido às marés | tendência; analogia com mudanças sociais ou tendências de desenvolvimento}
  \end{Phonetics}
\end{Entry}

\begin{Entry}{潮绣}{15,10}{⽔、⽷}
  \begin{Phonetics}{潮绣}{chao2xiu4}
    \definition*{s.}{Bordado Chaozhou}
  \end{Phonetics}
\end{Entry}

\begin{Entry}{潮湿}{15,12}{⽔、⽔}
  \begin{Phonetics}{潮湿}{chao2shi1}[][HSK 4]
    \definition{adj.}{molhado; úmido; umedecido; que contém mais água do que o normal}
  \end{Phonetics}
\end{Entry}

%%%%%%%%%% 澄 %%%%%%%%%%
\subsection*{澄}\addcontentsline{loh}{figure}{澄}

\begin{Entry}{澄}{15}{⽔}
  \begin{Phonetics}{澄}{cheng2}
    \definition*{s.}{Sobrenome: Cheng}
    \definition{adj.}{claro; transparente}
    \definition{v.}{esclarecer; purificar}
  \end{Phonetics}
  \begin{Phonetics}{澄}{deng4}
    \definition{adj.}{(água, ar, etc.) claro; transparente; límpido}
    \definition{v.}{esclarecer; aclarar | sedimentar; fazer com que impurezas em um líquido afundem}
  \end{Phonetics}
\end{Entry}

\begin{Entry}{澄清}{15,11}{⽔、⽔}
  \begin{Phonetics}{澄清}{cheng2qing1}[][HSK 7-9]
    \definition{adj.}{claro; transparente}
    \definition{v.}{esclarecer; deixar claro; entender | purificar; limpar; esclarecer a turbidez, uma metáfora para esclarecer uma situação caótica}
  \end{Phonetics}
\end{Entry}

%%%%%%%%%% 澳 %%%%%%%%%%
\subsection*{澳}\addcontentsline{loh}{figure}{澳}

\begin{Entry}{澳}{15}{⽔}
  \begin{Phonetics}{澳}{ao4}
    \definition*{s.}{Abreviação de Austrália, 澳大利亚 | Sobrenome: Ao}
    \definition{s.}{baía; uma entrada do mar; um lugar curvo na costa onde os barcos podem ser atracados, frequentemente usado em nomes de lugares}
  \seealsoref{澳大利亚}{ao4da4li4ya4}
  \end{Phonetics}
\end{Entry}

\begin{Entry}{澳大利亚}{15,3,7,6}{⽔、⼤、⼑、⼆}
  \begin{Phonetics}{澳大利亚}{ao4da4li4ya4}
    \definition*{s.}{Austrália}
  \end{Phonetics}
\end{Entry}

%%%%%%%%%% 熟 %%%%%%%%%%
\subsection*{熟}\addcontentsline{loh}{figure}{熟}

\begin{Entry}{熟}{15}{⽕}
  \begin{Phonetics}{熟}{shu2}[][HSK 2]
    \definition{adj.}{maduro (frutos) | pronto; cozido | processado, fabricado ou exercitado | familiar, bem conhecido; conhecido por ser comum ou frequentemente utilizado | habilidoso;  (trabalho, tecnologia) experiente; não é novato | profundo; sólido}
  \end{Phonetics}
\end{Entry}

\begin{Entry}{熟人}{15,2}{⽕、⼈}
  \begin{Phonetics}{熟人}{shu2 ren2}[][HSK 3]
    \definition[位,名,个,些]{s.}{amigo; conhecido; pessoas que se conhecem há muito tempo; pessoas que são muito familiares}
  \end{Phonetics}
\end{Entry}

\begin{Entry}{熟练}{15,8}{⽕、⽷}
  \begin{Phonetics}{熟练}{shu2lian4}[][HSK 4]
    \definition{adj.}{especializado; proficiente; qualificado; habilidoso}
  \end{Phonetics}
\end{Entry}

\begin{Entry}{熟悉}{15,11}{⽕、⼼}
  \begin{Phonetics}{熟悉}{shu2xi1}[][HSK 5]
    \definition{adj.}{familiarizado com; não ser estranho}
    \definition{v.}{estar familiarizado com; saber claramente que | conhecer bem algo ou alguém; compreender e dominar (a situação) através da observação ou da experiência}
  \end{Phonetics}
\end{Entry}

%%%%%%%%%% 獞 %%%%%%%%%%
\subsection*{獞}\addcontentsline{loh}{figure}{獞}

\begin{Entry}{獞}{15}{⽝}
  \begin{Phonetics}{獞}{tong2}
    \definition{s.}{nome de uma variedade de cão | tribos selvagens no sul da China}
  \end{Phonetics}
  \begin{Phonetics}{獞}{zhuang4}
    \variantof{壮}
  \end{Phonetics}
\end{Entry}

%%%%%%%%%% 瞒 %%%%%%%%%%
\subsection*{瞒}\addcontentsline{loh}{figure}{瞒}

\begin{Entry}{瞒}{15}{⽬}
  \begin{Phonetics}{瞒}{man2}[][HSK 7-9]
    \definition*{s.}{Sobrenome: Man}
    \definition{v.}{ocultar a verdade de; esconder; esconder a verdade de alguém}
  \end{Phonetics}
\end{Entry}

%%%%%%%%%% 碾 %%%%%%%%%%
\subsection*{碾}\addcontentsline{loh}{figure}{碾}

\begin{Entry}{碾}{15}{⽯}
  \begin{Phonetics}{碾}{nian3}
    \definition[台,个]{s.}{rolo e mó; rolo de pedra | rolo compressor}
    \definition{v.}{moer ou descascar com um rolo; esmagar | (literário) cortar e polir (jade, vidro, etc.) | achatar | pisar; pisotear, 轧}
  \seealsoref{辗}{zhan3}
  \end{Phonetics}
\end{Entry}

\begin{Entry}{碾碎}{15,13}{⽯、⽯}
  \begin{Phonetics}{碾碎}{nian3sui4}
    \definition{v.}{pulverizar | esmagar}
  \end{Phonetics}
\end{Entry}

%%%%%%%%%% 磅 %%%%%%%%%%
\subsection*{磅}\addcontentsline{loh}{figure}{磅}

\begin{Entry}{磅}{15}{⽯}
  \begin{Phonetics}{磅}{bang4}[][HSK 7-9]
    \definition{clas.}{libra | Tipografia: pt, ponto (tamanho de letra, por exemplo: 10pt)}
    \definition{s.}{escalas}
    \definition{v.}{pesar com uma balança}
  \end{Phonetics}
  \begin{Phonetics}{磅}{pang2}
    \definition{adj.}{majestoso; abundante; cheio de energia; magnífico}
  \end{Phonetics}
\end{Entry}

%%%%%%%%%% 磕 %%%%%%%%%%
\subsection*{磕}\addcontentsline{loh}{figure}{磕}

\begin{Entry}{磕}{15}{⽯}
  \begin{Phonetics}{磕}{ke1}[][HSK 7-9]
    \definition{v.}{bater (com força em algo); bater em algo duro | derrubar algo de um recipiente, vaso, etc.}
  \end{Phonetics}
\end{Entry}

%%%%%%%%%% 稻 %%%%%%%%%%
\subsection*{稻}\addcontentsline{loh}{figure}{稻}

\begin{Entry}{稻}{15}{⽲}
  \begin{Phonetics}{稻}{dao4}
    \definition{s.}{arroz; arroz com casca}
  \end{Phonetics}
\end{Entry}

\begin{Entry}{稻草}{15,9}{⽲、⾋}
  \begin{Phonetics}{稻草}{dao4cao3}[][HSK 7-9]
    \definition[捆,根,抱,束]{s.}{palha de arroz (pode ser usada para fazer cordas ou esteiras de palha, para fazer papel, ou para ser usada como ração, combustível, etc.)}
  \end{Phonetics}
\end{Entry}

%%%%%%%%%% 稿 %%%%%%%%%%
\subsection*{稿}\addcontentsline{loh}{figure}{稿}

\begin{Entry}{稿}{15}{⽲}
  \begin{Phonetics}{稿}{gao3}
    \definition[篇]{s.}{(significado original) talo de grão; palha | rascunho; esboço; manuscrito | texto original}
  \end{Phonetics}
\end{Entry}

\begin{Entry}{稿子}{15,3}{⽲、⼦}
  \begin{Phonetics}{稿子}{gao3 zi5}[][HSK 6]
    \definition[篇,份,堆,叠]{s.}{rascunho; esboço; rascunhos de poemas, ensaios, desenhos, etc. | rascunho; manuscrito; poemas escritos | ideia; plano; plano preliminar ou conceito de trabalho}
  \end{Phonetics}
\end{Entry}

\begin{Entry}{稿纸}{15,7}{⽲、⽷}
  \begin{Phonetics}{稿纸}{gao3zhi3}
    \definition{s.}{rascunho | manuscrito}
  \end{Phonetics}
\end{Entry}

%%%%%%%%%% 箭 %%%%%%%%%%
\subsection*{箭}\addcontentsline{loh}{figure}{箭}

\begin{Entry}{箭}{15}{⽵}
  \begin{Phonetics}{箭}{jian4}[][HSK 6]
    \definition[支]{s.}{seta | distância percorrida por uma flecha}
  \end{Phonetics}
\end{Entry}

%%%%%%%%%% 箱 %%%%%%%%%%
\subsection*{箱}\addcontentsline{loh}{figure}{箱}

\begin{Entry}{箱}{15}{⾋}
  \begin{Phonetics}{箱}{xiang1}[][HSK 4]
    \definition{s.}{caixa; estojo; baú | qualquer coisa no formato de caixa}
  \end{Phonetics}
\end{Entry}

\begin{Entry}{箱子}{15,3}{⾋、⼦}
  \begin{Phonetics}{箱子}{xiang1 zi5}[][HSK 4]
    \definition[个,只]{s.}{baú; caixa; estojo; maleta; pasta executiva}
  \end{Phonetics}
\end{Entry}

%%%%%%%%%% 篇 %%%%%%%%%%
\subsection*{篇}\addcontentsline{loh}{figure}{篇}

\begin{Entry}{篇}{15}{⽵}
  \begin{Phonetics}{篇}{pian1}[][HSK 2]
    \definition*{s.}{Sobrenome: Pian}
    \definition{clas.}{usado para folhas de papel, páginas de livros, artigos, etc.}
    \definition{s.}{um pedaço de escrita | folha (de papel, etc.) | (para papel, folhas de livros, artigos, etc.) folha; página; pedaço}
  \end{Phonetics}
\end{Entry}

%%%%%%%%%% 糆 %%%%%%%%%%
\subsection*{糆}\addcontentsline{loh}{figure}{糆}

\begin{Entry}{糆}{15}{⽶}
  \begin{Phonetics}{糆}{mian4}
    \variantof{面}
  \end{Phonetics}
\end{Entry}

%%%%%%%%%% 糊 %%%%%%%%%%
\subsection*{糊}\addcontentsline{loh}{figure}{糊}

\begin{Entry}{糊}{15}{⽶}
  \begin{Phonetics}{糊}{hu1}
    \definition{v.}{colar; untar; usar uma pasta mais espessa para revestir costuras, furos ou superfícies planas}
  \end{Phonetics}
  \begin{Phonetics}{糊}{hu2}[][HSK 7-9]
    \definition{adj.}{queimado}
    \definition{s.}{mingau; pasta; papa}
    \definition{v.}{colar com pasta; colar | (comida) ser queimado}
  \end{Phonetics}
  \begin{Phonetics}{糊}{hu4}
    \definition{s.}{pasta; comida que parece mingau}
  \end{Phonetics}
\end{Entry}

\begin{Entry}{糊里糊涂}{15,7,15,10}{⽶、⾥、⽶、⽔}
  \begin{Phonetics}{糊里糊涂}{hu2 li5 hu2tu5}
    \definition{adj.}{desnorteado | perturbado}
  \end{Phonetics}
\end{Entry}

\begin{Entry}{糊涂}{15,10}{⽶、⽔}
  \begin{Phonetics}{糊涂}{hu2tu5}[][HSK 7-9]
    \definition{adj.}{confuso; perplexo; desnorteado; com compreensão pouco clara ou confusa das coisas | confuso; com conteúdo confuso}
  \end{Phonetics}
\end{Entry}

%%%%%%%%%% 聪 %%%%%%%%%%
\subsection*{聪}\addcontentsline{loh}{figure}{聪}

\begin{Entry}{聪}{15}{⽿}
  \begin{Phonetics}{聪}{cong1}
    \definition{adj.}{audição aguçada | brilhante; inteligente; esperto | perspicaz}
    \definition{s.}{(literário) faculdades auditivas}
  \end{Phonetics}
\end{Entry}

\begin{Entry}{聪明}{15,8}{⽿、⽇}
  \begin{Phonetics}{聪明}{cong1ming5}[][HSK 5]
    \definition{adj.}{brilhante; esperto; inteligente; intelecto bem desenvolvido com boa memória e capacidade de compreensão}
  \end{Phonetics}
\end{Entry}

\begin{Entry}{聪慧}{15,15}{⽿、⼼}
  \begin{Phonetics}{聪慧}{cong1hui4}
    \definition{adj.}{inteligente | brilhante}
  \end{Phonetics}
\end{Entry}

%%%%%%%%%% 蔬 %%%%%%%%%%
\subsection*{蔬}\addcontentsline{loh}{figure}{蔬}

\begin{Entry}{蔬}{15}{⾋}
  \begin{Phonetics}{蔬}{shu1}
    \definition{s.}{vegetais}
  \end{Phonetics}
\end{Entry}

\begin{Entry}{蔬菜}{15,11}{⾋、⾋}
  \begin{Phonetics}{蔬菜}{shu1cai4}[][HSK 5]
    \definition[样,种]{s.}{verduras; legumes; vegetais; ervas que podem ser usadas na culinária}
  \end{Phonetics}
\end{Entry}

%%%%%%%%%% 蕃 %%%%%%%%%%
\subsection*{蕃}\addcontentsline{loh}{figure}{蕃}

\begin{Entry}{蕃}{15}{⾋}
  \begin{Phonetics}{蕃}{bo1}
    \definition[种]{s.}{estrangeiros}
  \end{Phonetics}
  \begin{Phonetics}{蕃}{fan1}
    \definition[种]{s.}{estrangeiros; aborígenes}
  \end{Phonetics}
  \begin{Phonetics}{蕃}{fan2}
    \definition{adj.}{exuberante; próspero}
    \definition{v.}{multiplicar; proliferar}
  \end{Phonetics}
\end{Entry}

\begin{Entry}{蕃茄}{15,8}{⾋、⾋}
  \begin{Phonetics}{蕃茄}{fan1 qie2}
    \variantof{番茄}
  \end{Phonetics}
\end{Entry}

%%%%%%%%%% 蝌 %%%%%%%%%%
\subsection*{蝌}\addcontentsline{loh}{figure}{蝌}

\begin{Entry}{蝌}{15}{⾍}
  \begin{Phonetics}{蝌}{ke1}
    \definition[只]{s.}{girino}
  \end{Phonetics}
\end{Entry}

\begin{Entry}{蝌蚪}{15,10}{⾍、⾍}
  \begin{Phonetics}{蝌蚪}{ke1dou3}
    \definition{s.}{girino}
  \end{Phonetics}
\end{Entry}

%%%%%%%%%% 蝲 %%%%%%%%%%
\subsection*{蝲}\addcontentsline{loh}{figure}{蝲}

\begin{Entry}{蝲}{15}{⾍}
  \begin{Phonetics}{蝲}{la4}
    \definition{s.}{lagostim de água doce}
  \seealsoref{蝲蛄}{la4gu3}
  \end{Phonetics}
\end{Entry}

\begin{Entry}{蝲蛄}{15,11}{⾍、⾍}
  \begin{Phonetics}{蝲蛄}{la4gu3}
    \definition{s.}{lagostim; lagostim de água doce}
  \end{Phonetics}
\end{Entry}

\begin{Entry}{蝲蝲蛄}{15,15,11}{⾍、⾍、⾍}
  \begin{Phonetics}{蝲蝲蛄}{la4la4gu3}
    \definition{s.}{grilo toupeira}
  \end{Phonetics}
\end{Entry}

%%%%%%%%%% 蝴 %%%%%%%%%%
\subsection*{蝴}\addcontentsline{loh}{figure}{蝴}

\begin{Entry}{蝴}{15}{⾍}
  \begin{Phonetics}{蝴}{hu2}
    \definition[对]{s.}{borboleta}
  \end{Phonetics}
\end{Entry}

\begin{Entry}{蝴蝶}{15,15}{⾍、⾍}
  \begin{Phonetics}{蝴蝶}{hu2die2}
    \definition[只]{s.}{borboleta}
  \end{Phonetics}
\end{Entry}

%%%%%%%%%% 豌 %%%%%%%%%%
\subsection*{豌}\addcontentsline{loh}{figure}{豌}

\begin{Entry}{豌}{15}{⾖}
  \begin{Phonetics}{豌}{wan1}
    \definition[粒]{s.}{ervilhas}
  \end{Phonetics}
\end{Entry}

\begin{Entry}{豌豆}{15,7}{⾖、⾖}
  \begin{Phonetics}{豌豆}{wan1dou4}
    \definition{s.}{ervilha}
  \end{Phonetics}
\end{Entry}

%%%%%%%%%% 豫 %%%%%%%%%%
\subsection*{豫}\addcontentsline{loh}{figure}{豫}

\begin{Entry}{豫}{15}{⾗}
  \begin{Phonetics}{豫}{yu4}
    \definition*{s.}{Província de Henan, abreviatura de 河南}
    \definition{adj.}{satisfeito; encantado | anterior; preliminar; preparatório}
    \definition{adv.}{com antecedência; antecipadamente}
    \definition{v.}{viver com facilidade e conforto | participar de}
  \seealsoref{河南}{he2nan2}
  \seealsoref{预}{yu4}
  \end{Phonetics}
\end{Entry}

%%%%%%%%%% 趟 %%%%%%%%%%
\subsection*{趟}\addcontentsline{loh}{figure}{趟}

\begin{Entry}{趟}{15}{⾛}
  \begin{Phonetics}{趟}{tang1}
    \definition{v.}{atravessar; andar na grama ou onde não haja caminho | usar arados, capinadores, etc. para virar o solo e remover ervas daninhas | vadear; atravessar a vau; caminhar por águas rasas}[我们趟水去那小岛。===Nós vadeamos até a ilha.]
  \end{Phonetics}
  \begin{Phonetics}{趟}{tang4}[][HSK 6]
    \definition{clas.}{usado para o número de vezes de viagens de ida e volta |  usado para coisas dispostas em fileiras ou tiras | usado para a programação de veículos, navios, etc. que circulam em uma determinada ordem | usado em conjuntos de movimentos de artes marciais}
    \definition{s.}{marcha; procissão; jornada; viagem}
  \end{Phonetics}
\end{Entry}

%%%%%%%%%% 踏 %%%%%%%%%%
\subsection*{踏}\addcontentsline{loh}{figure}{踏}

\begin{Entry}{踏}{15}{⾜}
  \begin{Phonetics}{踏}{ta1}
    \definition{part.}{Caracter formador de palavras}
  \end{Phonetics}
  \begin{Phonetics}{踏}{ta4}[][HSK 6]
    \definition{v.}{por os pés em; pisar em; esmagar com o pé | fazer uma investigação ou levantamento no local}
  \end{Phonetics}
\end{Entry}

\begin{Entry}{踏实}{15,8}{⾜、⼧}
  \begin{Phonetics}{踏实}{ta1shi5}[][HSK 6]
    \definition{adj.}{confiável; sério; estável e seguro; descreve uma atitude séria em relação ao trabalho ou estudo | à vontade; livre de ansiedade; descreve uma mente ou sentimento estável, sem qualquer preocupação ou ansiedade}
  \end{Phonetics}
\end{Entry}

\begin{Entry}{踏板}{15,8}{⾜、⽊}
  \begin{Phonetics}{踏板}{ta4ban3}
    \definition{s.}{pedal (em um carro, em um piano, etc.) |  apoio para os pés | estribo}
  \end{Phonetics}
\end{Entry}

%%%%%%%%%% 踢 %%%%%%%%%%
\subsection*{踢}\addcontentsline{loh}{figure}{踢}

\begin{Entry}{踢}{15}{⾜}
  \begin{Phonetics}{踢}{ti1}[][HSK 6]
    \definition{v.}{chutar | jogar (por exemplo, futebol)}
  \end{Phonetics}
\end{Entry}

\begin{Entry}{踢蹋舞}{15,17,14}{⾜、⾜、⾇}
  \begin{Phonetics}{踢蹋舞}{ti1ta4wu3}
    \definition{s.}{sapateado | passo de dança}
  \end{Phonetics}
\end{Entry}

\begin{Entry}{踢爆}{15,19}{⾜、⽕}
  \begin{Phonetics}{踢爆}{ti1bao4}
    \definition{v.}{expor | revelar}
  \end{Phonetics}
\end{Entry}

%%%%%%%%%% 踩 %%%%%%%%%%
\subsection*{踩}\addcontentsline{loh}{figure}{踩}

\begin{Entry}{踩}{15}{⾜}
  \begin{Phonetics}{踩}{cai3}[][HSK 6]
    \definition{v.}{pisar; pisotear | pisar; metáfora: depreciar ou estragar | rastrear; antigamente significava rastrear (bandidos) ou investigar (casos)}
  \end{Phonetics}
\end{Entry}

%%%%%%%%%% 躺 %%%%%%%%%%
\subsection*{躺}\addcontentsline{loh}{figure}{躺}

\begin{Entry}{躺}{15}{⾝}
  \begin{Phonetics}{躺}{tang3}[][HSK 4]
    \definition{v.}{deitar; reclinar; cair no chão ou sobre um objeto}
  \end{Phonetics}
\end{Entry}

%%%%%%%%%% 遵 %%%%%%%%%%
\subsection*{遵}\addcontentsline{loh}{figure}{遵}

\begin{Entry}{遵}{15}{⾡}
  \begin{Phonetics}{遵}{zun1}
    \definition{v.}{cumprir; obedecer; observar; seguir}
  \end{Phonetics}
\end{Entry}

\begin{Entry}{遵守}{15,6}{⾡、⼧}
  \begin{Phonetics}{遵守}{zun1shou3}[][HSK 5]
    \definition{v.}{obedecer; observar; cumprir; respeitar; atuar de acordo com as regras; não infringir}
  \end{Phonetics}
\end{Entry}

%%%%%%%%%% 醇 %%%%%%%%%%
\subsection*{醇}\addcontentsline{loh}{figure}{醇}

\begin{Entry}{醇}{15}{⾣}
  \begin{Phonetics}{醇}{chun2}
    \definition{adj.}{Literário: puro; puro e suave; não misturado}
    \definition{s.}{Literário: vinho suave; bom vinho ; Química: álcool}
  \end{Phonetics}
\end{Entry}

\begin{Entry}{醇厚}{15,9}{⾣、⼚}
  \begin{Phonetics}{醇厚}{chun2hou4}[][HSK 7-9]
    \definition{adj.}{suave; rico; cheiro e sabor puros e ricos | puro e honesto; simples e gentil}
  \end{Phonetics}
\end{Entry}

%%%%%%%%%% 醉 %%%%%%%%%%
\subsection*{醉}\addcontentsline{loh}{figure}{醉}

\begin{Entry}{醉}{15}{⾣}
  \begin{Phonetics}{醉}{zui4}[][HSK 5]
    \definition{v.}{embriagar-se; ficar bêbado; intoxicar-se; beber em excesso e perder o controle | (de certos alimentos) ser embebido em licor; ser mergulhado em vinho; marinar (alimentos) em vinho | entregar-se a; ser viciado em; gostar demais, a ponto de chegar à obsessão}
  \end{Phonetics}
\end{Entry}

%%%%%%%%%% 醋 %%%%%%%%%%
\subsection*{醋}\addcontentsline{loh}{figure}{醋}

\begin{Entry}{醋}{15}{⾣}
  \begin{Phonetics}{醋}{cu4}[][HSK 6]
    \definition[瓶,坛,碟,碗]{s.}{(condimento) vinagre | ciúme (como em caso de amor); uma metáfora para o ciúme, referindo-se principalmente aos relacionamentos entre pessoas}
  \end{Phonetics}
\end{Entry}

%%%%%%%%%% 镇 %%%%%%%%%%
\subsection*{镇}\addcontentsline{loh}{figure}{镇}

\begin{Entry}{镇}{15}{⾦}
  \begin{Phonetics}{镇}{zhen4}[][HSK 6]
    \definition{adj.}{inteiro; indica um período inteiro de tempo}
    \definition{adv.}{frequentemente; muitas vezes}
    \definition{s.}{posto de guarnição | cidade; divisão administrativa | centro comercial}
    \definition{v.}{suprimir; segurar; manter pressionado |  acalmar-se; recompor-se; estabilizar | guardar; guarnecer; fortalecer; usar a força para manter a estabilidade | resfriar com gelo; esfriar em água fria | acalmar; suprimir; dissuadir | suprimir pela força; sancionar}
  \end{Phonetics}
\end{Entry}

%%%%%%%%%% 震 %%%%%%%%%%
\subsection*{震}\addcontentsline{loh}{figure}{震}

\begin{Entry}{震}{15}{⾬}
  \begin{Phonetics}{震}{zhen4}
    \definition*{s.}{Zhen, um dos Oito Trigramas que representa o trovão | Sobrenome: Zhen}
    \definition{adj.}{(coloquial) muito animado; profundamente surpreso; chocado}
    \definition{s.}{vibração; trepidação; tremor; abalo | terremoto; refere-se especificamente a terremotos | trovão; relâmpago}
    \definition{v.}{sacudir; chocar; vibrar; estremecer | ficar muito animado; ficar profundamente surpreso; ficar chocado | superar; vencer}
  \end{Phonetics}
\end{Entry}

\begin{Entry}{震惊}{15,11}{⾬、⼼}
  \begin{Phonetics}{震惊}{zhen4jing1}[][HSK 5]
    \definition{adj.}{chocado; atordoado; espantado; atônito}
    \definition{v.}{chocar; surpreender; espantar}
  \end{Phonetics}
\end{Entry}

\begin{Entry}{震撼}{15,16}{⾬、⼿}
  \begin{Phonetics}{震撼}{zhen4han4}
    \definition{v.}{sacudir | chocar | atordoar}
  \end{Phonetics}
\end{Entry}

%%%%%%%%%% 靠 %%%%%%%%%%
\subsection*{靠}\addcontentsline{loh}{figure}{靠}

\begin{Entry}{靠}{15}{⾮}
  \begin{Phonetics}{靠}{kao4}[][HSK 2]
    \definition{prep.}{manter (em); aproximar-se (de); ao longo de | por; graças a; com base em; de acordo com}
    \definition{s.}{armadura de palco (feita de seda bordada); armadura usada pelos generais militares antigos nas peças teatrais}
    \definition{v.}{inclinar-se; sentado ou em pé, deixar parte do peso do corpo ser suportado por outra pessoa ou objeto (pessoa) | encostar-se (em); apoiar-se ou levantar-se com a ajuda de alguma coisa | aproximar-se; estar perto de | confiar em; depender de | confiar}
  \end{Phonetics}
\end{Entry}

\begin{Entry}{靠近}{15,7}{⾮、⾡}
  \begin{Phonetics}{靠近}{kao4 jin4}[][HSK 5]
    \definition{adv.}{próximo; perto de; ao lado de}
    \definition{v.}{aproximar-se; chegar perto; avançar em direção a um determinado objetivo de modo que a distância fique cada vez menor}
  \end{Phonetics}
\end{Entry}

\begin{Entry}{靠拢}{15,8}{⾮、⼿}
  \begin{Phonetics}{靠拢}{kao4long3}[][HSK 7-9]
    \definition{v.}{aproximar-se de; encostar-se; reunir-se; aconchegar-se}
  \end{Phonetics}
\end{Entry}

%%%%%%%%%% 鞋 %%%%%%%%%%
\subsection*{鞋}\addcontentsline{loh}{figure}{鞋}

\begin{Entry}{鞋}{15}{⾰}
  \begin{Phonetics}{鞋}{xie2}[][HSK 2]
    \definition[双,只]{s.}{sapatos; usado nos pés; algo que toca o chão ao caminhar; sem cano alto}
  \end{Phonetics}
\end{Entry}

%%%%%%%%%% 题 %%%%%%%%%%
\subsection*{题}\addcontentsline{loh}{figure}{题}

\begin{Entry}{题}{15}{⾴}
  \begin{Phonetics}{题}{ti2}[][HSK 2]
    \definition*{s.}{Sobrenome: Ti}
    \definition[个,道]{s.}{tópico; título; assunto; problema; frases que indicam o conteúdo de poemas ou discursos | questão; questões que devem ser respondidas durante os exercícios ou exames | antigamente, referia-se à testa}
    \definition{v.}{inscrever; escrever; assinar}
  \end{Phonetics}
\end{Entry}

\begin{Entry}{题目}{15,5}{⾴、⽬}
  \begin{Phonetics}{题目}{ti2mu4}[][HSK 3]
    \definition[个,道]{s.}{título; assunto; tópico; o título de um poema ou discurso | quebra-cabeça; problema de exercício; questões a serem respondidas em exercícios ou provas}
  \end{Phonetics}
\end{Entry}

\begin{Entry}{题材}{15,7}{⾴、⽊}
  \begin{Phonetics}{题材}{ti2cai2}[][HSK 5]
    \definition{s.}{tema; assunto; material que compõe as obras literárias e artísticas, ou seja, os eventos ou fenômenos da vida descritos concretamente nas obras}
  \end{Phonetics}
\end{Entry}

%%%%%%%%%% 颜 %%%%%%%%%%
\subsection*{颜}\addcontentsline{loh}{figure}{颜}

\begin{Entry}{颜}{15}{⾴}
  \begin{Phonetics}{颜}{yan2}
    \definition*{s.}{Sobrenome: Yan}
    \definition{s.}{rosto; semblante; expressão facial | rosto; prestígio; dignidade | cor}
  \end{Phonetics}
\end{Entry}

\begin{Entry}{颜色}{15,6}{⾴、⾊}
  \begin{Phonetics}{颜色}{yan2 se4}[][HSK 2]
    \definition[个,种]{s.}{cor; a sensação visual de um objeto é uma impressão diferente produzida pelas diferentes quantidades de luz absorvidas e refletidas pelo objeto | tez; semblante; aparência; geralmente se refere à aparência de uma garota | olhar severo no rosto como um aviso; um olhar ou ação que faz os outros parecerem particularmente ferozes | a expressão mostrada no rosto}
  \end{Phonetics}
\end{Entry}

%%%%%%%%%% 额 %%%%%%%%%%
\subsection*{额}\addcontentsline{loh}{figure}{额}

\begin{Entry}{额}{15}{⾴}
  \begin{Phonetics}{额}{e2}
    \definition*{s.}{Sobrenome: E}
    \definition[块]{s.}{testa; a área abaixo do cabelo e acima das sobrancelhas em humanos; a área aproximadamente equivalente na cabeça de alguns animais | uma tábua horizontal; placa horizontal inscrita; uma placa pendurada no lintel de uma porta ou na parede | um número específico (ou quantidade); limite superior de número; número limitado | a parte superior de algo}
  \end{Phonetics}
\end{Entry}

\begin{Entry}{额外}{15,5}{⾴、⼣}
  \begin{Phonetics}{额外}{e2wai4}[][HSK 7-9]
    \definition{adj.}{extra; adicional; excede a quantidade ou intervalo prescrito}
  \end{Phonetics}
\end{Entry}

%%%%%%%%%% 飘 %%%%%%%%%%
\subsection*{飘}\addcontentsline{loh}{figure}{飘}

\begin{Entry}{飘}{15}{⾵}
  \begin{Phonetics}{飘}{piao1}
    \definition{adj.}{complacente | frívolo | fraco | instável | bambo | cambaleante}
    \definition{v.}{flutuar (no ar) | esvoaçar | tremular}
  \end{Phonetics}
\end{Entry}

%%%%%%%%%% 鲨 %%%%%%%%%%
\subsection*{鲨}\addcontentsline{loh}{figure}{鲨}

\begin{Entry}{鲨}{15}{⿂}
  \begin{Phonetics}{鲨}{sha1}
    \definition[只,条]{s.}{tubarão}
  \end{Phonetics}
\end{Entry}

\begin{Entry}{鲨鱼}{15,8}{⿂、⿂}
  \begin{Phonetics}{鲨鱼}{sha1yu2}
    \definition{s.}{tubarão}
  \end{Phonetics}
\end{Entry}

%%%%%%%%%% 鹤 %%%%%%%%%%
\subsection*{鹤}\addcontentsline{loh}{figure}{鹤}

\begin{Entry}{鹤}{15}{⿃}
  \begin{Phonetics}{鹤}{he4}
    \definition*{s.}{Sobrenome: He}
    \definition[只]{s.}{grou (ave)}
  \end{Phonetics}
\end{Entry}

\begin{Entry}{鹤立鸡群}{15,5,7,13}{⿃、⽴、⿃、⽺}
  \begin{Phonetics}{鹤立鸡群}{he4li4ji1qun2}[][HSK 7-9]
    \definition{expr.}{destaque-se da multidão; manifestamente superior; muito acima do comum; como um guindaste em pé entre galinhas --- fique de pé acima dos outros}
  \end{Phonetics}
\end{Entry}

%%%%%%%%%% 麫 %%%%%%%%%%
\subsection*{麫}\addcontentsline{loh}{figure}{麫}

\begin{Entry}{麫}{15}{⿆}
  \begin{Phonetics}{麫}{mian4}
    \variantof{面}
  \end{Phonetics}
\end{Entry}

%%%%%%%%%% 黎 %%%%%%%%%%
\subsection*{黎}\addcontentsline{loh}{figure}{黎}

\begin{Entry}{黎}{15}{⿉}
  \begin{Phonetics}{黎}{li2}
    \definition*{s.}{Etnia Li, uma das minorias nacionais da província de Hainan | Sobrenome: Li}
    \definition{adj.}{Literário: preto; escuro | Literário: numeroso}
    \definition{s.}{multidão; as massas; a população}
  \end{Phonetics}
\end{Entry}

\begin{Entry}{黎明}{15,8}{⿉、⽇}
  \begin{Phonetics}{黎明}{li2ming2}[][HSK 7-9]
    \definition[个]{s.}{amanhecer; alvorecer; quando está prestes a amanhecer ou logo após o amanhecer}
  \end{Phonetics}
\end{Entry}

%%%%% EOF %%%%%


 %%%
%%% 16画
%%%
\section*{16画}\addcontentsline{toc}{section}{16画}\addcontentsline{loh}{figure}{\#\#\#\# 16画}

%%%%%%%%%% 儒 %%%%%%%%%%
\subsection*{儒}\addcontentsline{loh}{figure}{儒}

\begin{Entry}{儒}{16}{⼈}
  \begin{Phonetics}{儒}{ru2}
    \definition*{s.}{Confucionismo; Confucionista | Sobrenome: Ru}
    \definition{s.}{Obsoleto: erudito; homem culto | confucionismo}
  \end{Phonetics}
\end{Entry}

\begin{Entry}{儒学}{16,8}{⼈,⼦}
  \begin{Phonetics}{儒学}{ru2xue2}[][HSK 7-9]
    \definition{s.}{confucionismo; ensinamentos confucionistas | Obsoleto: escola administrada pelo governo em nível provincial, de prefeitura ou de condado durante as dinastias Yuan, Ming e Qing}
  \end{Phonetics}
\end{Entry}

\begin{Entry}{儒家}{16,10}{⼈,⼧}
  \begin{Phonetics}{儒家}{ru2jia1}[][HSK 7-9]
    \definition*{s.}{Confucionismo; Escola Confucionista; Confucionistas; uma corrente de pensamento do período pré-Qin, representada por Confúcio, que defendia o governo por meio de ritos e enfatizava as relações éticas tradicionais}
  \end{Phonetics}
\end{Entry}

\begin{Entry}{儒教}{16,11}{⼈,⽁}
  \begin{Phonetics}{儒教}{ru2jiao4}
    \definition{s.}{confucionismo; o confucionismo, a partir das Dinastias do Norte e do Sul, era chamado de confucionismo enquanto religião e era mencionado juntamente com o budismo e o taoísmo}
  \seealsoref{儒家}{ru2jia1}
  \end{Phonetics}
\end{Entry}

%%%%%%%%%% 凝 %%%%%%%%%%
\subsection*{凝}\addcontentsline{loh}{figure}{凝}

\begin{Entry}{凝}{16}{⼎}
  \begin{Phonetics}{凝}{ning2}
    \definition{adv.}{atentamente; atenção fixa}
    \definition{v.}{congelar; coagular; coalhar; condensar | contemplar; olhar pensativamente}
  \end{Phonetics}
\end{Entry}

\begin{Entry}{凝固}{16,8}{⼎,⼞}
  \begin{Phonetics}{凝固}{ning2gu4}[][HSK 7-9]
    \definition{v.}{estagnar; parar, não se move mais nem muda de direção}
  \end{Phonetics}
\end{Entry}

\begin{Entry}{凝聚}{16,14}{⼎,⽿}
  \begin{Phonetics}{凝聚}{ning2ju4}[][HSK 7-9]
    \definition{v.}{condensar (o vapor); coagular; coalhar (os fluidos) | reunir; acumular; juntar}
  \end{Phonetics}
\end{Entry}

%%%%%%%%%% 嘴 %%%%%%%%%%
\subsection*{嘴}\addcontentsline{loh}{figure}{嘴}

\begin{Entry}{嘴}{16}{⼝}
  \begin{Phonetics}{嘴}{zui3}[][HSK 2]
    \definition[张]{s.}{boca; boca humana ou animal | qualquer coisa com formato ou função semelhante a uma boca | fala | comida}
  \end{Phonetics}
\end{Entry}

\begin{Entry}{嘴巴}{16,4}{⼝,⼰}
  \begin{Phonetics}{嘴巴}{zui3ba5}[][HSK 4]
    \definition[张]{s.}{boca}
  \end{Phonetics}
\end{Entry}

\begin{Entry}{嘴巴子}{16,4,3}{⼝,⼰,⼦}
  \begin{Phonetics}{嘴巴子}{zui3ba5zi5}
    \definition{s.}{tapa | bofetada}
  \end{Phonetics}
\end{Entry}

%%%%%%%%%% 器 %%%%%%%%%%
\subsection*{器}\addcontentsline{loh}{figure}{器}

\begin{Entry}{器}{16}{⼝}
  \begin{Phonetics}{器}{qi4}
    \definition[台]{s.}{dispositivo | ferramenta | utensílio}
  \end{Phonetics}
\end{Entry}

\begin{Entry}{器材}{16,7}{⼝,⽊}
  \begin{Phonetics}{器材}{qi4cai2}[][HSK 7-9]
    \definition[种,批,套,件]{s.}{material; aparelho; equipamento; ferramentas e materiais}
  \end{Phonetics}
\end{Entry}

\begin{Entry}{器官}{16,8}{⼝,⼧}
  \begin{Phonetics}{器官}{qi4guan1}[][HSK 4]
    \definition[个,种]{s.}{órgão; aparelho; parte de um organismo que consiste em vários tipos de tecidos celulares que podem desempenhar uma função fisiológica separada}
  \end{Phonetics}
\end{Entry}

\begin{Entry}{器械}{16,11}{⼝,⽊}
  \begin{Phonetics}{器械}{qi4xie4}[][HSK 7-9]
    \definition[种,批,些]{s.}{aparelho; dispositivo; instrumento; equipamento; instrumentos com finalidades especiais ou com construção relativamente precisa | arma; armamento; instrumentos e dispositivos usados diretamente para matar pessoal inimigo e destruir instalações de combate inimigas, como facas, armas de fogo, artilharia e mísseis}
  \end{Phonetics}
\end{Entry}

%%%%%%%%%% 壁 %%%%%%%%%%
\subsection*{壁}\addcontentsline{loh}{figure}{壁}

\begin{Entry}{壁}{16}{⼟}
  \begin{Phonetics}{壁}{bi4}
    \definition*{s.}{Bi, a décima quarta das vinte e oito constelações em que a esfera celeste foi dividida, consistindo em duas estrelas em linha reta, uma em Pégaso e a outra em Andrômeda | A Estrela Bìxìu, uma das Vinte e Oito Mansões da astronomia tradicional chinesa}
    \definition[道]{s.}{parede | superfície plana como uma parede | penhasco | muralha; parapeito | barreira}
  \end{Phonetics}
\end{Entry}

\begin{Entry}{壁纸}{16,7}{⼟,⽷}
  \begin{Phonetics}{壁纸}{bi4zhi3}
    \definition{s.}{papel de parede; papel colado em paredes internas para decoração ou proteção, com diversos tipos e cores}
  \end{Phonetics}
\end{Entry}

\begin{Entry}{壁画}{16,8}{⼟,⽥}
  \begin{Phonetics}{壁画}{bi4hua4}[][HSK 7-9]
    \definition[幅]{s.}{mural; afresco; desenhos nas paredes ou tetos de edifícios}
  \end{Phonetics}
\end{Entry}

\begin{Entry}{壁虎}{16,8}{⼟,⾌}
  \begin{Phonetics}{壁虎}{bi4hu3}
    \definition{s.}{lagartixa}
  \end{Phonetics}
\end{Entry}

%%%%%%%%%% 懒 %%%%%%%%%%
\subsection*{懒}\addcontentsline{loh}{figure}{懒}

\begin{Entry}{懒}{16}{⼼}
  \begin{Phonetics}{懒}{lan3}[][HSK 6]
    \definition{adj.}{indolente; preguiçoso | lento; lânguido | ocioso; preguiçoso}
  \antonymref{勤}{qin2}
  \end{Phonetics}
\end{Entry}

\begin{Entry}{懒人}{16,2}{⼼,⼈}
  \begin{Phonetics}{懒人}{lan3ren2}
    \definition{s.}{pessoa preguiçosa}
  \end{Phonetics}
\end{Entry}

\begin{Entry}{懒汉}{16,5}{⼼,⽔}
  \begin{Phonetics}{懒汉}{lan3han4}
    \definition{s.}{sujeito ocioso | vagabundo | preguiçosos}
  \end{Phonetics}
\end{Entry}

\begin{Entry}{懒虫}{16,6}{⼼,⾍}
  \begin{Phonetics}{懒虫}{lan3chong2}
    \definition{s.}{desleixado ocioso | (insulto) sujeito preguiçoso}
  \end{Phonetics}
\end{Entry}

\begin{Entry}{懒怠}{16,9}{⼼,⼼}
  \begin{Phonetics}{懒怠}{lan3dai4}
    \definition{s.}{preguiça}
  \end{Phonetics}
\end{Entry}

\begin{Entry}{懒鬼}{16,9}{⼼,⿁}
  \begin{Phonetics}{懒鬼}{lan3gui3}
    \definition{s.}{cara preguiçoso}
  \end{Phonetics}
\end{Entry}

\begin{Entry}{懒得}{16,11}{⼼,⼻}
  \begin{Phonetics}{懒得}{lan3de5}[][HSK 7-9]
    \definition{v.}{não estar disposto a fazer algo; estar entediado; estar sem vontade (de fazer algo)}
  \end{Phonetics}
\end{Entry}

\begin{Entry}{懒惰}{16,12}{⼼,⼼}
  \begin{Phonetics}{懒惰}{lan3duo4}[][HSK 7-9]
    \definition{adj.}{preguiçoso; ocioso; que não gosta de trabalho e esforço; que não é diligente}
  \end{Phonetics}
\end{Entry}

\begin{Entry}{懒散}{16,12}{⼼,⽁}
  \begin{Phonetics}{懒散}{lan3san3}
    \definition{adj.}{inativo | indolente | preguiçoso | negligente}
  \end{Phonetics}
\end{Entry}

\begin{Entry}{懒腰}{16,13}{⼼,⾁}
  \begin{Phonetics}{懒腰}{lan3yao1}
    \definition[个]{s.}{alongamento (do corpo)}
  \end{Phonetics}
\end{Entry}

%%%%%%%%%% 撼 %%%%%%%%%%
\subsection*{撼}\addcontentsline{loh}{figure}{撼}

\begin{Entry}{撼}{16}{⼿}
  \begin{Phonetics}{撼}{han4}
    \definition{v.}{agitar; sacudir}
  \end{Phonetics}
\end{Entry}

%%%%%%%%%% 擅 %%%%%%%%%%
\subsection*{擅}\addcontentsline{loh}{figure}{擅}

\begin{Entry}{擅}{16}{⼿}
  \begin{Phonetics}{擅}{shan4}
    \definition{adv.}{sem autorização; arbitrariamente | fazer algo por conta própria}
    \definition{v.}{ser bom em; ser especialista em | arrogar-se a si mesmo; fazer algo por conta própria | reivindicar arbitrariamente; ir além do escopo e ajir arbitrariamente}
  \end{Phonetics}
\end{Entry}

\begin{Entry}{擅长}{16,4}{⼿,⾧}
  \begin{Phonetics}{擅长}{shan4chang2}[][HSK 7-9]
    \definition{v.}{ser bom em; ser especialista em; ser habilidoso em; ter um talento especial em determinada área}
  \end{Phonetics}
\end{Entry}

\begin{Entry}{擅自}{16,6}{⼿,⾃}
  \begin{Phonetics}{擅自}{shan4zi4}[][HSK 7-9]
    \definition{adv.}{arbitrariamente; sem permissão ou autorização; agir por iniciativa própria em assuntos que estão fora da sua alçada}
  \end{Phonetics}
\end{Entry}

%%%%%%%%%% 操 %%%%%%%%%%
\subsection*{操}\addcontentsline{loh}{figure}{操}

\begin{Entry}{操}{16}{⼿}
  \begin{Phonetics}{操}{cao1}
    \definition*{s.}{Sobrenome: Cao}
    \definition[节,套]{s.}{exercício; ginástica | conduta; comportamento; moralidade, a moral e o código de conduta que as pessoas seguem}
    \definition{v.}{segurar; agarrar; segurar na mão | fazer algo; envolver"-se em | falar (uma língua ou dialeto) | treinar (tropas); exercitar (corpo); praticar ou treinar de acordo com uma determinada forma ou postura | dirigir; manusear}
  \end{Phonetics}
\end{Entry}

\begin{Entry}{操心}{16,4}{⼿,⼼}
  \begin{Phonetics}{操心}{cao1/xin1}[][HSK 7-9]
    \definition{v.+compl.}{esforçar"-se; preocupar"-se com; incomodar"-se com}
  \end{Phonetics}
\end{Entry}

\begin{Entry}{操场}{16,6}{⼿,⼟}
  \begin{Phonetics}{操场}{cao1chang3}[][HSK 4]
    \definition[个,片,座,处]{s.}{\emph{playground}; campo esportivo; locais para exercícios físicos ou exercícios militares}
  \end{Phonetics}
\end{Entry}

\begin{Entry}{操作}{16,7}{⼿,⼈}
  \begin{Phonetics}{操作}{cao1zuo4}[][HSK 4]
    \definition{v.}{operar; seguir os requisitos e procedimentos prescritos | implementar; realizar; executar; refere"-se à implementação concreta (planos, medidas, etc.)}
  \end{Phonetics}
\end{Entry}

\begin{Entry}{操劳}{16,7}{⼿,⼒}
  \begin{Phonetics}{操劳}{cao1lao2}[][HSK 7-9]
    \definition{v.}{trabalhar duro | cuidar; cuidar de}
  \end{Phonetics}
\end{Entry}

\begin{Entry}{操纵}{16,7}{⼿,⽷}
  \begin{Phonetics}{操纵}{cao1zong4}[][HSK 6]
    \definition{v.}{operar; controlar (uma máquina, instrumento, etc.) | manipular; controlar secretamente; assumir o controle de (uma pessoa, organização, situação, etc.)}
  \end{Phonetics}
\end{Entry}

\begin{Entry}{操控}{16,11}{⼿,⼿}
  \begin{Phonetics}{操控}{cao1kong4}[][HSK 7-9]
    \definition{v.}{controlar; manipular}
  \end{Phonetics}
\end{Entry}

%%%%%%%%%% 整 %%%%%%%%%%
\subsection*{整}\addcontentsline{loh}{figure}{整}

\begin{Entry}{整}{16}{⽁}
  \begin{Phonetics}{整}{zheng3}[][HSK 3]
    \definition*{s.}{Sobrenome: Zheng}
    \definition{adj.}{cheio; integral; inteiro; completo; sem defeitos | limpo; arrumado; organizado; em boa ordem | redondo (não é uma fração)}
    \definition{s.}{número inteiro (não fracionário)}
    \definition{v.}{retificar; corrigir; pôr em ordem | consertar; renovar; reparar | corrigir; punir; causar sofrimento;  fazer alguém sofrer | fazer; realizar; trabalhar; em algumas regiões, significa 做, 搞}
  \seealsoref{搞}{gao3}
  \seealsoref{做}{zuo4}
  \end{Phonetics}
\end{Entry}

\begin{Entry}{整个}{16,3}{⽁,⼈}
  \begin{Phonetics}{整个}{zheng3ge4}[][HSK 3]
    \definition{adj.}{total; inteiro; completo}
  \end{Phonetics}
\end{Entry}

\begin{Entry}{整天}{16,4}{⽁,⼤}
  \begin{Phonetics}{整天}{zheng3tian1}[][HSK 3]
    \definition{s.}{o dia inteiro; o dia todo; durante todo o dia; de manhã à noite}
  \end{Phonetics}
\end{Entry}

\begin{Entry}{整齐}{16,6}{⽁,⿑}
  \begin{Phonetics}{整齐}{zheng3qi2}[][HSK 3]
    \definition{adj.}{arrumado; organizado; em boa ordem | uniforme; regular; tamanho, comprimento, grau, etc. são relativamente consistentes | usado para descrever que todas as coisas necessárias estão prontas}
    \definition{v.}{estar em boas condições; manter a ordem e a organização}
  \end{Phonetics}
\end{Entry}

\begin{Entry}{整体}{16,7}{⽁,⼈}
  \begin{Phonetics}{整体}{zheng3ti3}[][HSK 3]
    \definition[个]{s.}{um todo; totalidade}
  \end{Phonetics}
\end{Entry}

\begin{Entry}{整治}{16,8}{⽁,⽔}
  \begin{Phonetics}{整治}{zheng3zhi4}[][HSK 6]
    \definition{v.}{renovar; consertar; arrumar; dragar (um rio, etc.) | punir; fazer alguém sofrer}
  \end{Phonetics}
\end{Entry}

\begin{Entry}{整顿}{16,10}{⽁,⾴}
  \begin{Phonetics}{整顿}{zheng3dun4}[][HSK 6]
    \definition{v.}{retificar; consolidar; reorganizar; tornar fenômenos, disciplinas e estilos desordenados e irracionais, ordenados e razoáveis}
  \end{Phonetics}
\end{Entry}

\begin{Entry}{整理}{16,11}{⽁,⽟}
  \begin{Phonetics}{整理}{zheng3li3}[][HSK 3]
    \definition{v.}{organizar; reorganizar; classificar; ordenar; colocar em ordem}
  \end{Phonetics}
\end{Entry}

\begin{Entry}{整整}{16,16}{⽁,⽁}
  \begin{Phonetics}{整整}{zheng3zheng3}[][HSK 3]
    \definition{adv.}{inteiramente; completamente; solidamente; continuamente}
  \end{Phonetics}
\end{Entry}

%%%%%%%%%% 橘 %%%%%%%%%%
\subsection*{橘}\addcontentsline{loh}{figure}{橘}

\begin{Entry}{橘}{16}{⽊}
  \begin{Phonetics}{橘}{ju2}
    \definition[只,棵]{s.}{tangerina}
  \end{Phonetics}
\end{Entry}

\begin{Entry}{橘子}{16,3}{⽊,⼦}
  \begin{Phonetics}{橘子}{ju2zi5}[][HSK 7-9]
    \definition[个,堆,箱,筐]{s.}{tangerina; laranja mandarina}
  \end{Phonetics}
\end{Entry}

\begin{Entry}{橘子汁}{16,3,5}{⽊,⼦,⽔}
  \begin{Phonetics}{橘子汁}{ju2zi5zhi1}
    \definition[瓶,杯,罐,盒]{s.}{suco de laranja}
  \seealsoref{橙汁}{cheng2zhi1}
  \seealsoref{柳橙汁}{liu3cheng2zhi1}
  \end{Phonetics}
\end{Entry}

%%%%%%%%%% 橙 %%%%%%%%%%
\subsection*{橙}\addcontentsline{loh}{figure}{橙}

\begin{Entry}{橙}{16}{⽊}
  \begin{Phonetics}{橙}{cheng2}
    \definition{s.}{laranja; fruta da laranjeira | laranjeira; pé de laranja | cor laranja}
  \end{Phonetics}
\end{Entry}

\begin{Entry}{橙汁}{16,5}{⽊,⽔}
  \begin{Phonetics}{橙汁}{cheng2zhi1}[][HSK 7-9]
    \definition[瓶,杯,罐,盒]{s.}{laranjada; suco de laranja}
  \seealsoref{橘子汁}{ju2zi5zhi1}
  \seealsoref{柳橙汁}{liu3cheng2zhi1}
  \end{Phonetics}
\end{Entry}

\begin{Entry}{橙色}{16,6}{⽊,⾊}
  \begin{Phonetics}{橙色}{cheng2 se4}
    \definition{s.}{cor de laranja}
  \end{Phonetics}
\end{Entry}

%%%%%%%%%% 激 %%%%%%%%%%
\subsection*{激}\addcontentsline{loh}{figure}{激}

\begin{Entry}{激}{16}{⽔}
  \begin{Phonetics}{激}{ji1}
    \definition*{s.}{Sobrenome: Ji}
    \definition{adj.}{afiado; feroz; violento | vívido}
    \definition{adv.}{bruscamente; ferozmente; violentamente}
    \definition{s.}{o impacto de ondas fortes contra a costa}
    \definition{v.}{bater; avançar; correr | despertar; estimular; incitar; excitar | ficar doente por se molhar | esfriar (colocando água gelada, etc.)}
  \end{Phonetics}
\end{Entry}

\begin{Entry}{激化}{16,4}{⽔,⼔}
  \begin{Phonetics}{激化}{ji1hua4}[][HSK 7-9]
    \definition{v.}{aguçar; intensificar; tornar agudo}
  \end{Phonetics}
\end{Entry}

\begin{Entry}{激发}{16,5}{⽔,⼜}
  \begin{Phonetics}{激发}{ji1fa1}[][HSK 7-9]
    \definition{v.}{despertar; desencadear; estimular; motivar; inspirar | excitar; mudar moléculas e átomos de um estado de energia mais baixo para um estado de energia mais alto}
  \end{Phonetics}
\end{Entry}

\begin{Entry}{激光}{16,6}{⽔,⼉}
  \begin{Phonetics}{激光}{ji1guang1}[][HSK 7-9]
    \definition*{s.}{LASER; \emph{Light Amplification by Stimulated Emission of Radiation}; amplificação de luz por emissão estimulada de radiação}
  \end{Phonetics}
\end{Entry}

\begin{Entry}{激动}{16,6}{⽔,⼒}
  \begin{Phonetics}{激动}{ji1dong4}[][HSK 4]
    \definition{adj.}{animado; entusiasmado; empolgado}
    \definition{v.}{agitar; excitar; tornar fortes os sentimentos de alguém}
  \end{Phonetics}
\end{Entry}

\begin{Entry}{激励}{16,7}{⽔,⼒}
  \begin{Phonetics}{激励}{ji1li4}[][HSK 7-9]
    \definition{v.}{instar; impelir; inspirar; encorajar; usar palavras ou ações de outras pessoas para encorajar as pessoas a trabalhar mais e fazer melhor | dirigir; excitar; estimular ou excitar uma reação ou atividade}
  \end{Phonetics}
\end{Entry}

\begin{Entry}{激活}{16,9}{⽔,⽔}
  \begin{Phonetics}{激活}{ji1huo2}[][HSK 7-9]
    \definition{v.}{ativar; estimular certas substâncias no corpo para torná-las ativas | colocar em jogo; revigorar; estimular metaforicamente, influenciar algo, torná-lo ativo}
  \end{Phonetics}
\end{Entry}

\begin{Entry}{激烈}{16,10}{⽔,⽕}
  \begin{Phonetics}{激烈}{ji1lie4}[][HSK 4]
    \definition{adj.}{agudo; afiado; feroz; violento; intenso}
  \end{Phonetics}
\end{Entry}

\begin{Entry}{激素}{16,10}{⽔,⽷}
  \begin{Phonetics}{激素}{ji1su4}[][HSK 7-9]
    \definition{s.}{Fisiológico: hormônio; uma substância química que regula a atividade celular}
  \end{Phonetics}
\end{Entry}

\begin{Entry}{激起}{16,10}{⽔,⾛}
  \begin{Phonetics}{激起}{ji1qi3}[][HSK 7-9]
    \definition{v.}{excitar; agitar; despertar; evocar; desencadear}
  \end{Phonetics}
\end{Entry}

\begin{Entry}{激情}{16,11}{⽔,⼼}
  \begin{Phonetics}{激情}{ji1qing2}[][HSK 6]
    \definition{s.}{paixão; emoções fortes e explosivas, como êxtase, raiva, etc.}
  \end{Phonetics}
\end{Entry}

%%%%%%%%%% 燃 %%%%%%%%%%
\subsection*{燃}\addcontentsline{loh}{figure}{燃}

\begin{Entry}{燃}{16}{⽕}
  \begin{Phonetics}{燃}{ran2}
    \definition{v.}{queimar | acender; inflamar}
  \end{Phonetics}
\end{Entry}

\begin{Entry}{燃气}{16,4}{⽕,⽓}
  \begin{Phonetics}{燃气}{ran2qi4}[][HSK 7-9]
    \definition{s.}{combustível gasoso, como gás de carvão, biogás, gás natural e gás liquefeito de petróleo}
  \end{Phonetics}
\end{Entry}

\begin{Entry}{燃放}{16,8}{⽕,⽅}
  \begin{Phonetics}{燃放}{ran2fang4}[][HSK 7-9]
    \definition{v.}{acender (fogos de artifício, etc.); acender fogos de artifício, etc., para causar uma explosão}
  \end{Phonetics}
\end{Entry}

\begin{Entry}{燃油}{16,8}{⽕,⽔}
  \begin{Phonetics}{燃油}{ran2you2}[][HSK 7-9]
    \definition{s.}{óleo combustível}[燃油价格持续上涨了。===Os preços dos combustíveis continuaram a subir.]
  \end{Phonetics}
\end{Entry}

\begin{Entry}{燃料}{16,10}{⽕,⽃}
  \begin{Phonetics}{燃料}{ran2liao4}[][HSK 4]
    \definition[种]{s.}{combustível; carburante; substâncias que podem gerar calor e energia luminosa quando queimadas podem ser divididas em três tipos de acordo com sua forma: combustível sólido (como carvão, carvão vegetal, madeira), combustível líquido (como gasolina, querosene) e combustível gasoso (como gás de carvão, biogás); também se refere a substâncias que podem gerar energia nuclear, como urânio, plutônio, etc.}
  \end{Phonetics}
\end{Entry}

\begin{Entry}{燃烧}{16,10}{⽕,⽕}
  \begin{Phonetics}{燃烧}{ran2shao1}[][HSK 4]
    \definition{v.}{queimar; acender | arder; inflamar; ferver; metáfora para as emoções de uma pessoa serem muito fortes, como um fogo ardente}
  \end{Phonetics}
\end{Entry}

%%%%%%%%%% 犟 %%%%%%%%%%
\subsection*{犟}\addcontentsline{loh}{figure}{犟}

\begin{Entry}{犟}{16}{⽜}
  \begin{Phonetics}{犟}{jiang4}
    \variantof{强}
  \end{Phonetics}
\end{Entry}

%%%%%%%%%% 瞠 %%%%%%%%%%
\subsection*{瞠}\addcontentsline{loh}{figure}{瞠}

\begin{Entry}{瞠}{16}{⽬}
  \begin{Phonetics}{瞠}{cheng1}
    \definition{v.}{Literário: olhar fixamente para algo além do alcance}
  \end{Phonetics}
\end{Entry}

%%%%%%%%%% 磨 %%%%%%%%%%
\subsection*{磨}\addcontentsline{loh}{figure}{磨}

\begin{Entry}{磨}{16}{⽯}
  \begin{Phonetics}{磨}{mo2}[][HSK 6]
    \definition{v.}{esfregar; desgastar | moer; refletir; polir | desgastar; esgotar; cansar; exaurir | incomodar; causar problemas | destruir; obliterar; extinguir-se | ficar ocioso; perder tempo; perder tempo; procrastinar}
  \end{Phonetics}
  \begin{Phonetics}{磨}{mo4}
    \definition[盘]{s.}{mó (pedra pesada e redonda para moinho)}
    \definition{v.}{moer; esfarelar; triturar | virar; inverter a marcha}
  \end{Phonetics}
\end{Entry}

\begin{Entry}{磨合}{16,6}{⽯,⼝}
  \begin{Phonetics}{磨合}{mo2he2}[][HSK 7-9]
    \definition{v.}{amaciar; máquinas e veículos novos ou reformados, após um período de operação, têm suas marcas de usinagem suavizadas, tornando as superfícies de fricção mais bem vedadas | (pessoas) conviver em harmonia; aprender a se dar bem; acomodar-se mutuamente}
  \end{Phonetics}
\end{Entry}

\begin{Entry}{磨损}{16,10}{⽯,⼿}
  \begin{Phonetics}{磨损}{mo2sun3}[][HSK 7-9]
    \definition{v.}{desgastar; causar abrasão; desgastar por fricção e uso}
  \end{Phonetics}
\end{Entry}

\begin{Entry}{磨难}{16,10}{⽯,⾫}
  \begin{Phonetics}{磨难}{mo2nan4}[][HSK 7-9]
    \definition{s.}{tribulação; dificuldade; sofrimento; o tormento sofrido em circunstâncias difíceis também é chamado de tribulação}
  \end{Phonetics}
\end{Entry}

\begin{Entry}{磨菇}{16,11}{⽯,⾋}
  \begin{Phonetics}{磨菇}{mo2gu5}
    \variantof{蘑菇}
  \end{Phonetics}
\end{Entry}

%%%%%%%%%% 穆 %%%%%%%%%%
\subsection*{穆}\addcontentsline{loh}{figure}{穆}

\begin{Entry}{穆}{16}{⽲}
  \begin{Phonetics}{穆}{mu4}
    \definition*{s.}{Sobrenome: Mu}
    \definition{adj.}{solene; reverente | respeitoso}
  \end{Phonetics}
\end{Entry}

\begin{Entry}{穆斯林}{16,12,8}{⽲,⽄,⽊}
  \begin{Phonetics}{穆斯林}{mu4si1lin2}[][HSK 7-9]
    \definition[个,位,名]{s.}{muçulmano}
  \end{Phonetics}
\end{Entry}

\begin{Entry}{穆棱}{16,12}{⽲,⽊}
  \begin{Phonetics}{穆棱}{mu4ling2}
    \definition*{s.}{Cidade no nível do condado de Muling, na província de Mudanjiang 牡丹江,Heilongjiang}
  \seealsoref{牡丹江}{mu3dan1jiang1}
  \end{Phonetics}
\end{Entry}

%%%%%%%%%% 篮 %%%%%%%%%%
\subsection*{篮}\addcontentsline{loh}{figure}{篮}

\begin{Entry}{篮}{16}{⽵}
  \begin{Phonetics}{篮}{lan2}
    \definition[个]{s.}{cesto | o anel de ferro e a rede na cesta de basquete}
  \end{Phonetics}
\end{Entry}

\begin{Entry}{篮球}{16,11}{⽵,⽟}
  \begin{Phonetics}{篮球}{lan2qiu2}[][HSK 2]
    \definition[个,只]{s.}{basquetebol | bola de basquete; refere"-se à bola utilizada no basquetebol}
  \end{Phonetics}
\end{Entry}

%%%%%%%%%% 糕 %%%%%%%%%%
\subsection*{糕}\addcontentsline{loh}{figure}{糕}

\begin{Entry}{糕}{16}{⽶}
  \begin{Phonetics}{糕}{gao1}
    \definition{s.}{bolo; alimentos feitos de farinha de arroz, farinha de trigo, etc.}
  \end{Phonetics}
\end{Entry}

\begin{Entry}{糕点}{16,9}{⽶,⽕}
  \begin{Phonetics}{糕点}{gao1dian3}
    \definition{s.}{bolos | pastéis}
  \end{Phonetics}
\end{Entry}

\begin{Entry}{糕点师}{16,9,6}{⽶,⽕,⼱}
  \begin{Phonetics}{糕点师}{gao1dian3 shi1}
    \definition{s.}{confeiteiro}
  \end{Phonetics}
\end{Entry}

\begin{Entry}{糕点店}{16,9,8}{⽶,⽕,⼴}
  \begin{Phonetics}{糕点店}{gao1dian3 dian4}
    \definition{s.}{confeitaria}
  \end{Phonetics}
\end{Entry}

%%%%%%%%%% 糖 %%%%%%%%%%
\subsection*{糖}\addcontentsline{loh}{figure}{糖}

\begin{Entry}{糖}{16}{⽶}
  \begin{Phonetics}{糖}{tang2}[][HSK 3]
    \definition[包,斤,勺,袋,块]{s.}{açúcar; um tipo de açúcar; um tipo de composto orgânico, que pode ser dividido em três tipos: monossacarídeos, dissacarídeos e polissacarídeos; é a principal substância que produz energia térmica no corpo humano, como glicose, sacarose, lactose, amido, etc. | açúcar; açúcar comestível; termo geral para açúcar | doces; balas | carboidrato; algo doce e calórico}
  \end{Phonetics}
\end{Entry}

\begin{Entry}{糖尿病}{16,7,10}{⽶,⼫,⽧}
  \begin{Phonetics}{糖尿病}{tang2niao4bing4}[][HSK 7-9]
    \definition{s.}{diabetes; diabetes mellitus; doença crônica causada pela secreção insuficiente de insulina, levando a distúrbios no metabolismo da glicose e níveis elevados de açúcar no sangue}[糖尿病会带来严重的问题。===O diabetes pode causar problemas graves.]
  \end{Phonetics}
\end{Entry}

\begin{Entry}{糖果}{16,8}{⽶,⽊}
  \begin{Phonetics}{糖果}{tang2guo3}[][HSK 7-9]
    \definition[个,颗,包,袋]{s.}{doce; alimentos açucarados, que geralmente contêm suco de frutas, especiarias, leite ou café, etc.}
  \end{Phonetics}
\end{Entry}

\begin{Entry}{糖醋鱼}{16,15,8}{⽶,⾣,⿂}
  \begin{Phonetics}{糖醋鱼}{tang2cu4yu2}
    \definition{s.}{peixe guisado em molho agridoce (prato)}
  \end{Phonetics}
\end{Entry}

%%%%%%%%%% 缴 %%%%%%%%%%
\subsection*{缴}\addcontentsline{loh}{figure}{缴}

\begin{Entry}{缴}{16}{⽷}
  \begin{Phonetics}{缴}{jiao3}[][HSK 7-9]
    \definition{v.}{pagar | capturar | entregar (em)}
  \end{Phonetics}
  \begin{Phonetics}{缴}{zhuo2}
    \definition{s.}{corda de seda crua usada na antiguidade para amarrar flechas para caçar pássaros}
  \end{Phonetics}
\end{Entry}

\begin{Entry}{缴纳}{16,7}{⽷,⽷}
  \begin{Phonetics}{缴纳}{jiao3na4}[][HSK 7-9]
    \definition{v.}{pagar algo ao público; pagar; receber; colocar em}
  \end{Phonetics}
\end{Entry}

\begin{Entry}{缴费}{16,9}{⽷,⾙}
  \begin{Phonetics}{缴费}{jiao3fei4}[][HSK 7-9]
    \definition{v.}{pagar taxas; a taxa exigida para pagar por um serviço ou produto}
  \end{Phonetics}
\end{Entry}

%%%%%%%%%% 膨 %%%%%%%%%%
\subsection*{膨}\addcontentsline{loh}{figure}{膨}

\begin{Entry}{膨}{16}{⾁}
  \begin{Phonetics}{膨}{peng2}
    \definition{v.}{inchar; inflar | expandir; aumentar o comprimento ou o volume de um objeto}
  \end{Phonetics}
\end{Entry}

\begin{Entry}{膨胀}{16,8}{⾁,⾁}
  \begin{Phonetics}{膨胀}{peng2zhang4}[][HSK 7-9]
    \definition{v.}{inchar; dilatar; expandir; o volume do objeto aumenta | inflar; essa metáfora descreve algo que foi expandido ou crescido de forma inadequada}
  \end{Phonetics}
\end{Entry}

%%%%%%%%%% 膳 %%%%%%%%%%
\subsection*{膳}\addcontentsline{loh}{figure}{膳}

\begin{Entry}{膳}{16}{⾁}
  \begin{Phonetics}{膳}{shan4}
    \definition{s.}{refeições; comida; alimentação}
  \end{Phonetics}
\end{Entry}

\begin{Entry}{膳食}{16,9}{⾁,⾷}
  \begin{Phonetics}{膳食}{shan4shi2}[][HSK 7-9]
    \definition{s.}{comida; refeições; refeições consumidas todos os dias}
  \end{Phonetics}
\end{Entry}

%%%%%%%%%% 蕹 %%%%%%%%%%
\subsection*{蕹}\addcontentsline{loh}{figure}{蕹}

\begin{Entry}{蕹}{16}{⾋}
  \begin{Phonetics}{蕹}{weng4}
    \definition{s.}{espinafre-d’água ou \emph{ong choy}, usado como vegetal no sul da China e no sudeste da Ásia}
  \end{Phonetics}
\end{Entry}

\begin{Entry}{蕹菜}{16,11}{⾋,⾋}
  \begin{Phonetics}{蕹菜}{weng4cai4}
    \definition{s.}{espinafre aquático | \emph{ong choy} | repolho do pântano | convolvulus aquático | glória-da-manhã aquática}
  \seealsoref{空心菜}{kong1xin1cai4}
  \end{Phonetics}
\end{Entry}

%%%%%%%%%% 薄 %%%%%%%%%%
\subsection*{薄}\addcontentsline{loh}{figure}{薄}

\begin{Entry}{薄}{16}{⾋}
  \begin{Phonetics}{薄}{bao2}[][HSK 4]
    \definition{adj.}{fino; frágil | frio; indiferente; carente de calor | leve; fraco | pobre; infértil}
  \end{Phonetics}
  \begin{Phonetics}{薄}{bo2}
    \definition*{s.}{Sobrenome: Bo}
    \definition{adj.}{pequeno; leve; magro | mau; cruel; mesquinho | frívolo; fútil; não solene | fraco; frágil}
    \definition{v.}{desprezar; tratar com desprezo; menosprezar | aproximar"-se}
  \end{Phonetics}
  \begin{Phonetics}{薄}{bo4}
    \definition{s.}{menta; uma erva perene com aroma refrescante nos caules e folhas}
  \end{Phonetics}
\end{Entry}

\begin{Entry}{薄弱}{16,10}{⾋,⼸}
  \begin{Phonetics}{薄弱}{bo2ruo4}[][HSK 5]
    \definition{adj.}{fraco; frágil; não é firme; não é sólido}
  \end{Phonetics}
\end{Entry}

%%%%%%%%%% 薛 %%%%%%%%%%
\subsection*{薛}\addcontentsline{loh}{figure}{薛}

\begin{Entry}{薛}{16}{⾋}
  \begin{Phonetics}{薛}{xue1}
    \definition*{s.}{Estado vassalo durante a Dinastia Zhou (1046-256 a.C.) | Sobrenome: Xue}
    \definition{s.}{erva semelhante ao absinto (clássico)}
  \end{Phonetics}
\end{Entry}

\begin{Entry}{薛稷}{16,15}{⾋,⽲}
  \begin{Phonetics}{薛稷}{xue1 ji4}
    \definition*{s.}{Xue Ji (649-713), um dos quatro grandes calígrafos do início da dinastia Tang, 唐初四大家}
  \seealsoref{唐初四大家}{tang2 chu1 si4 da4jia1}
  \end{Phonetics}
\end{Entry}

%%%%%%%%%% 薪 %%%%%%%%%%
\subsection*{薪}\addcontentsline{loh}{figure}{薪}

\begin{Entry}{薪}{16}{⾋}
  \begin{Phonetics}{薪}{xin1}
    \definition{s.}{lenha; combustível | salário; ordenado; pagamento}
  \end{Phonetics}
\end{Entry}

\begin{Entry}{薪水}{16,4}{⾋,⽔}
  \begin{Phonetics}{薪水}{xin1shui5}[][HSK 6]
    \definition[份,笔]{s.}{pagamento; salário; ordenados; dinheiro ou bens pagos regularmente aos trabalhadores como compensação pelo seu trabalho}
  \end{Phonetics}
\end{Entry}

%%%%%%%%%% 薯 %%%%%%%%%%
\subsection*{薯}\addcontentsline{loh}{figure}{薯}

\begin{Entry}{薯}{16}{⾋}
  \begin{Phonetics}{薯}{shu3}
    \definition{s.}{batata | inhame}
  \end{Phonetics}
\end{Entry}

\begin{Entry}{薯片}{16,4}{⾋,⽚}
  \begin{Phonetics}{薯片}{shu3pian4}[][HSK 6]
    \definition{s.}{batatas fritas (\emph{chips}); batatas fritas crocantes ; flocos finos feitos de batatas}
  \end{Phonetics}
\end{Entry}

\begin{Entry}{薯条}{16,7}{⾋,⽊}
  \begin{Phonetics}{薯条}{shu3tiao2}[][HSK 6]
    \definition{s.}{batatas fritas (palito)}
  \end{Phonetics}
\end{Entry}

%%%%%%%%%% 融 %%%%%%%%%%
\subsection*{融}\addcontentsline{loh}{figure}{融}

\begin{Entry}{融}{16}{⿀}
  \begin{Phonetics}{融}{rong2}[][HSK 7-9]
    \definition*{s.}{Sobrenome: Rong}
    \definition{adj.}{permanente; longo prazo; duradouro | muito brilhante | circulante; corrente}
    \definition{s.}{fogo | plena luz do dia}
    \definition{v.}{derreter; descongelar | misturar; fundir; estar em harmonia | circular (dinheiro, etc.)}
  \end{Phonetics}
\end{Entry}

\begin{Entry}{融入}{16,2}{⿀,⼊}
  \begin{Phonetics}{融入}{rong2ru4}[][HSK 6]
    \definition{v.}{integrar em; juntar-se, integrar-se ao grupo | misturar-se; enfatizar a mistura e a combinação com o ambiente circundante para se tornar harmonioso e consistente | encher com (um certo sentimento); imbuir com (uma certa qualidade); preparar (chá, ervas, etc.); imergir; infundir (drogas, etc.)}
  \end{Phonetics}
\end{Entry}

\begin{Entry}{融化}{16,4}{⿀,⼔}
  \begin{Phonetics}{融化}{rong2hua4}[][HSK 7-9]
    \definition{v.}{derreter; descongelar}
  \end{Phonetics}
\end{Entry}

\begin{Entry}{融合}{16,6}{⿀,⼝}
  \begin{Phonetics}{融合}{rong2he2}[][HSK 6]
    \definition{v.}{fundir; mesclar; misturar; combinar várias coisas diferentes em uma}
  \end{Phonetics}
\end{Entry}

\begin{Entry}{融洽}{16,9}{⿀,⽔}
  \begin{Phonetics}{融洽}{rong2qia4}[][HSK 7-9]
    \definition{adj.}{harmonioso; em termos amigáveis}
  \end{Phonetics}
\end{Entry}

%%%%%%%%%% 衡 %%%%%%%%%%
\subsection*{衡}\addcontentsline{loh}{figure}{衡}

\begin{Entry}{衡}{16}{⾏}
  \begin{Phonetics}{衡}{heng2}
    \definition*{s.}{Sobrenome: Heng}
    \definition[个]{s.}{braço graduado de uma balança | balança; aparelho de pesagem}
    \definition{v.}{pesar; medir; julgar}
  \end{Phonetics}
\end{Entry}

\begin{Entry}{衡山}{16,3}{⾏,⼭}
  \begin{Phonetics}{衡山}{heng2shan1}
    \definition*{s.}{Monte Heng em Hunan, uma cordilheira ao sul das Cinco Montanhas Sagradas (五岳) | Condado de Hengshan em Hengyang (衡阳)}
  \seealsoref{衡阳}{heng2yang2}
  \seealsoref{五岳}{wu3yue4}
  \end{Phonetics}
\end{Entry}

\begin{Entry}{衡阳}{16,6}{⾏,⾩}
  \begin{Phonetics}{衡阳}{heng2yang2}
    \definition*{s.}{Hengyang, cidade de nível de prefeitura em Hunan situada no lado sul do Monte Heng (衡山)}
  \seealsoref{衡山}{heng2shan1}
  \end{Phonetics}
\end{Entry}

\begin{Entry}{衡量}{16,12}{⾏,⾥}
  \begin{Phonetics}{衡量}{heng2liang5}[][HSK 6]
    \definition{v.}{pesar; medir; comparar; avaliar | considerar; pensar sobre; deliberar}
  \end{Phonetics}
\end{Entry}

%%%%%%%%%% 赞 %%%%%%%%%%
\subsection*{赞}\addcontentsline{loh}{figure}{赞}

\begin{Entry}{赞}{16}{⾙}
  \begin{Phonetics}{赞}{zan4}
    \definition{v.}{patrocinar | apoiar | elogiar | (gíria na \emph{Internet}) para curtir (uma postagem \emph{on-line})}
  \end{Phonetics}
\end{Entry}

\begin{Entry}{赞成}{16,6}{⾙,⼽}
  \begin{Phonetics}{赞成}{zan4cheng2}[][HSK 4]
    \definition{v.}{endossar; favorecer; aprovar; concordar com; concordar ou apoiar as ideias, os planos, as propostas ou o comportamento de outra pessoa}
  \end{Phonetics}
\end{Entry}

\begin{Entry}{赞扬}{16,6}{⾙,⼿}
  \begin{Phonetics}{赞扬}{zan4yang2}
    \definition{v.}{elogiar | aprovar | demonstrar aprovação}
  \end{Phonetics}
\end{Entry}

\begin{Entry}{赞助}{16,7}{⾙,⼒}
  \begin{Phonetics}{赞助}{zan4zhu4}[][HSK 4]
    \definition{v.}{apoiar; patrocinar; concordar e ajudar (refere"-se principalmente a oferecer dinheiro para ajudar)}
  \end{Phonetics}
\end{Entry}

\begin{Entry}{赞赏}{16,12}{⾙,⾙}
  \begin{Phonetics}{赞赏}{zan4shang3}[][HSK 4]
    \definition{v.}{admirar; apreciar; valorizar}
  \end{Phonetics}
\end{Entry}

%%%%%%%%%% 赠 %%%%%%%%%%
\subsection*{赠}\addcontentsline{loh}{figure}{赠}

\begin{Entry}{赠}{16}{⾙}
  \begin{Phonetics}{赠}{zeng4}[][HSK 5]
    \definition{s.}{um presente (de despedida), uma lembrança; honrarias póstumas; uma patente de título}
    \definition{v.}{dar um presente; presentear com um brinde}
  \end{Phonetics}
\end{Entry}

\begin{Entry}{赠送}{16,9}{⾙,⾡}
  \begin{Phonetics}{赠送}{zeng4song4}[][HSK 5]
    \definition{v.}{dar; dar de presente; dar algo de graça a alguém}
  \end{Phonetics}
\end{Entry}

%%%%%%%%%% 踹 %%%%%%%%%%
\subsection*{踹}\addcontentsline{loh}{figure}{踹}

\begin{Entry}{踹}{16}{⾜}
  \begin{Phonetics}{踹}{chuai4}[][HSK 7-9]
    \definition{v.}{chutar (com a sola do pé) | pisar; pisotear; pisar em}
  \end{Phonetics}
\end{Entry}

%%%%%%%%%% 辨 %%%%%%%%%%
\subsection*{辨}\addcontentsline{loh}{figure}{辨}

\begin{Entry}{辨}{16}{⾟}
  \begin{Phonetics}{辨}{bian4}
    \definition{v.}{diferenciar; distinguir; discriminar | reconhecer; distinguir; identificar; discernir}
  \end{Phonetics}
\end{Entry}

\begin{Entry}{辨认}{16,4}{⾟,⾔}
  \begin{Phonetics}{辨认}{bian4ren4}[][HSK 7-9]
    \definition{v.}{identificar; reconhecer; identificar e julgar com base em características para encontrar ou identificar um objeto}
  \end{Phonetics}
\end{Entry}

\begin{Entry}{辨别}{16,7}{⾟,⼑}
  \begin{Phonetics}{辨别}{bian4bie2}[][HSK 7-9]
    \definition{v.}{diferenciar; distinguir; discriminar; encontrar características de diferentes coisas e diferenciá"-las}
  \end{Phonetics}
\end{Entry}

%%%%%%%%%% 辩 %%%%%%%%%%
\subsection*{辩}\addcontentsline{loh}{figure}{辩}

\begin{Entry}{辩}{16}{⾟}
  \begin{Phonetics}{辩}{bian4}
    \definition{v.}{argumentar; disputar; debater}
  \end{Phonetics}
\end{Entry}

\begin{Entry}{辩论}{16,6}{⾟,⾔}
  \begin{Phonetics}{辩论}{bian4lun4}[][HSK 4]
    \definition{v.}{debater; obter um entendimento unificado ou correto, ambos os lados usam linguagem, palavras etc. para explicar seus pontos de vista, apontar os erros ou as contradições do outro lado}
  \end{Phonetics}
\end{Entry}

\begin{Entry}{辩护}{16,7}{⾟,⼿}
  \begin{Phonetics}{辩护}{bian4hu4}[][HSK 7-9]
    \definition{v.}{pleitear; defender | defender; argumentar em favor de}
  \end{Phonetics}
\end{Entry}

\begin{Entry}{辩解}{16,13}{⾟,⾓}
  \begin{Phonetics}{辩解}{bian4jie3}[][HSK 7-9]
    \definition{v.}{fornecer uma explicação; tentar se defender; explicar uma visão ou comportamento criticado; eliminar a crítica ou reduzir sua gravidade}
  \end{Phonetics}
\end{Entry}

%%%%%%%%%% 避 %%%%%%%%%%
\subsection*{避}\addcontentsline{loh}{figure}{避}

\begin{Entry}{避}{16}{⾌}
  \begin{Phonetics}{避}{bi4}[][HSK 4]
    \definition{v.}{evitar; evadir; esquivar"-se; buscar abrigo; fugir | impedir; manter afastado; repelir; previnir}
  \end{Phonetics}
\end{Entry}

\begin{Entry}{避免}{16,7}{⾌,⼉}
  \begin{Phonetics}{避免}{bi4mian3}[][HSK 4]
    \definition{v.}{evitar; desviar; abster"-se de; tentar não fazer com que algo aconteça; prevenir; tentar impedir (que algo ruim aconteça) com antecedência}
  \end{Phonetics}
\end{Entry}

\begin{Entry}{避难}{16,10}{⾌,⾫}
  \begin{Phonetics}{避难}{bi4/nan4}[][HSK 7-9]
    \definition{s.}{refúgio}
    \definition{v.+compl.}{refugiar"-se; buscar asilo (político etc.)}
  \end{Phonetics}
\end{Entry}

\begin{Entry}{避暑}{16,12}{⾌,⽇}
  \begin{Phonetics}{避暑}{bi4/shu3}[][HSK 7-9]
    \definition{v.+compl.}{ir de férias em um resort de verão; ir para um lugar fresco para evitar o calor do verão | prevenir insolação}
  \end{Phonetics}
\end{Entry}

%%%%%%%%%% 邀 %%%%%%%%%%
\subsection*{邀}\addcontentsline{loh}{figure}{邀}

\begin{Entry}{邀}{16}{⾡}
  \begin{Phonetics}{邀}{yao1}
    \definition{v.}{convidar; requerer | (literário)  buscar aprovação; pedir permissão | interceptar}
  \end{Phonetics}
\end{Entry}

\begin{Entry}{邀请}{16,10}{⾡,⾔}
  \begin{Phonetics}{邀请}{yao1qing3}[][HSK 5]
    \definition[份,个]{s.}{convite}
    \definition{v.}{convidar; solicitar; convidar pessoas para irem à sua casa ou a um local combinado}
  \end{Phonetics}
\end{Entry}

%%%%%%%%%% 醒 %%%%%%%%%%
\subsection*{醒}\addcontentsline{loh}{figure}{醒}

\begin{Entry}{醒}{16}{⾣}
  \begin{Phonetics}{醒}{xing3}[][HSK 4]
    \definition{adj.}{impressionante; notável; admirável; atraente; chamativo}
    \definition{v.}{ficar sóbrio; voltar a si; recuperar a consciência; retornar à normalidade após intoxicação, anestesia ou coma | despertar; estar acordado | ter a mente clara; mover a consciência da confusão para a compreensão | vir a entender; tornar-se ciente de; tomar consciência de}
  \end{Phonetics}
\end{Entry}

%%%%%%%%%% 镖 %%%%%%%%%%
\subsection*{镖}\addcontentsline{loh}{figure}{镖}

\begin{Entry}{镖}{16}{⾦}
  \begin{Phonetics}{镖}{biao1}
    \definition{s.}{dardo | arma de arremesso | mercadorias enviadas sob a proteção de uma escolta armada}
  \end{Phonetics}
\end{Entry}

%%%%%%%%%% 镜 %%%%%%%%%%
\subsection*{镜}\addcontentsline{loh}{figure}{镜}

\begin{Entry}{镜}{16}{⾦}
  \begin{Phonetics}{镜}{jing4}
    \definition*{s.}{Sobrenome: Jing}
    \definition[面,副]{s.}{espelho | lente; vidro; dispositivos para auxiliar a visão ou conduzir experimentos ópticos}
    \definition{v.}{espelhar | perceber | usar como referência}
  \end{Phonetics}
\end{Entry}

\begin{Entry}{镜子}{16,3}{⾦,⼦}
  \begin{Phonetics}{镜子}{jing4zi5}[][HSK 4]
    \definition[面,个]{s.}{espelho; instrumento de reflexão de imagem liso e plano, antigamente esmerilhado a partir de um disco grosso de cobre fundido, atualmente feito de vidro plano revestido de prata ou alumínio | óculos; óculos de grau}
  \end{Phonetics}
\end{Entry}

\begin{Entry}{镜头}{16,5}{⾦,⼤}
  \begin{Phonetics}{镜头}{jing4tou2}[][HSK 4]
    \definition[个,组]{s.}{lente de câmera; objetiva; combinação de várias lentes, usada para formar uma imagem | foto; cena}
  \end{Phonetics}
\end{Entry}

%%%%%%%%%% 雕 %%%%%%%%%%
\subsection*{雕}\addcontentsline{loh}{figure}{雕}

\begin{Entry}{雕}{16}{⾫}
  \begin{Phonetics}{雕}{diao1}[][HSK 7-9]
    \definition*{s.}{Sobrenome: Diao}
    \definition{s.}{abutre; águia | escultura ou obras esculpidas}
    \definition{v.}{esculpir; gravar}
  \end{Phonetics}
\end{Entry}

\begin{Entry}{雕刻}{16,8}{⾫,⼑}
  \begin{Phonetics}{雕刻}{diao1ke4}[][HSK 7-9]
    \definition[件,尊,个]{s.}{escultura; entalhe; obras de arte esculpidas}
    \definition{v.}{esculpir; gravar; esculpir uma imagem em metal, marfim, osso ou outros materiais}
  \end{Phonetics}
\end{Entry}

\begin{Entry}{雕塑}{16,13}{⾫,⼟}
  \begin{Phonetics}{雕塑}{diao1su4}[][HSK 7-9]
    \definition[座,件,尊,个]{s.}{escultura; obras de arte tridimensionais feitas de madeira, pedra ou metal por meio de entalhe, empilhamento, batida, etc.}
    \definition{v.}{esculpir; entalhar e moldar; usar madeira, pedra ou metal para criar formas artísticas tridimensionais esculpindo, empilhando, batendo, etc.}
  \end{Phonetics}
\end{Entry}

%%%%%%%%%% 霍 %%%%%%%%%%
\subsection*{霍}\addcontentsline{loh}{figure}{霍}

\begin{Entry}{霍}{16}{⾬}
  \begin{Phonetics}{霍}{huo4}
    \definition*{s.}{Sobrenome: Huo}
    \definition{adv.}{Literário: de repente; rapidamente}
  \end{Phonetics}
\end{Entry}

\begin{Entry}{霍乱}{16,7}{⾬,⼄}
  \begin{Phonetics}{霍乱}{huo4luan4}[][HSK 7-9]
    \definition{s.}{cólera; uma doença altamente contagiosa causada pelo Vibrio Cholerae | gastroenterite aguda (geralmente se refere a sintomas como vômitos intensos, diarreia, dor abdominal e cólicas)}
  \end{Phonetics}
\end{Entry}

%%%%%%%%%% 颠 %%%%%%%%%%
\subsection*{颠}\addcontentsline{loh}{figure}{颠}

\begin{Entry}{颠}{16}{⾴}
  \begin{Phonetics}{颠}{dian1}
    \definition*{s.}{Sobrenome: Dian}
    \definition{adj.}{mentalmente perturbado; insano; o mesmo que 癫}
    \definition{s.}{coroa (da cabeça) | topo; cume}
    \definition{v.}{sacudir; bater | cair; virar; tombar | Dialeto: correr; ir embora}
  \seealsoref{癫}{dian1}
  \end{Phonetics}
\end{Entry}

\begin{Entry}{颠倒}{16,10}{⾴,⼈}
  \begin{Phonetics}{颠倒}{dian1dao3}[][HSK 7-9]
    \definition{adj.}{confuso; desordenado}
    \definition{v.}{inverter; reverter; virar de cabeça para baixo}
  \end{Phonetics}
\end{Entry}

\begin{Entry}{颠覆}{16,18}{⾴,⾑}
  \begin{Phonetics}{颠覆}{dian1fu4}[][HSK 7-9]
    \definition{v.}{derrubar; subverter; virar; tombar | tombar; derrubar (um regime legítimo) por conspiração}
  \end{Phonetics}
\end{Entry}

%%%%%%%%%% 飙 %%%%%%%%%%
\subsection*{飙}\addcontentsline{loh}{figure}{飙}

\begin{Entry}{飙}{16}{⾵}
  \begin{Phonetics}{飙}{biao1}
    \definition{s.}{tempestade; furacão; redemoinho | Literário: vento violento; redemoinho}
  \end{Phonetics}
\end{Entry}

\begin{Entry}{飙升}{16,4}{⾵,⼗}
  \begin{Phonetics}{飙升}{biao1sheng1}[][HSK 7-9]
    \definition{v.}{disparar; subir rapidamente; (preço, quantidade, etc.) aumentam rapidamente}
  \end{Phonetics}
\end{Entry}

%%%%%%%%%% 餐 %%%%%%%%%%
\subsection*{餐}\addcontentsline{loh}{figure}{餐}

\begin{Entry}{餐}{16}{⾷}
  \begin{Phonetics}{餐}{can1}[][HSK 6]
    \definition{clas.}{comer; fazer uma refeição}
    \definition{clas.}{usado para refeições}
    \definition{s.}{comida; refeição}
  \end{Phonetics}
\end{Entry}

\begin{Entry}{餐厅}{16,4}{⾷,⼚}
  \begin{Phonetics}{餐厅}{can1ting1}[][HSK 5]
    \definition[间]{s.}{sala de jantar}
    \definition[间,家,个]{s.}{restaurante; refeitório em um hotel | cantina, refeitório; também é chamado de 食堂}
  \seealsoref{食堂}{shi2tang2}
  \end{Phonetics}
\end{Entry}

\begin{Entry}{餐饮}{16,7}{⾷,⾷}
  \begin{Phonetics}{餐饮}{can1yin3}[][HSK 5]
    \definition[个]{s.}{comidas e bebidas; refere"-se a atividades de bufê em restaurantes e hotéis}
  \end{Phonetics}
\end{Entry}

\begin{Entry}{餐桌}{16,10}{⾷,⽊}
  \begin{Phonetics}{餐桌}{can1zhuo1}[][HSK 7-9]
    \definition[张]{s.}{mesa de jantar}
  \end{Phonetics}
\end{Entry}

\begin{Entry}{餐馆}{16,11}{⾷,⾷}
  \begin{Phonetics}{餐馆}{can1guan3}[][HSK 5]
    \definition[家,个]{s.}{restaurante}
  \end{Phonetics}
\end{Entry}

%%%%%%%%%% 鲸 %%%%%%%%%%
\subsection*{鲸}\addcontentsline{loh}{figure}{鲸}

\begin{Entry}{鲸}{16}{⿂}
  \begin{Phonetics}{鲸}{jing1}
    \definition[头,只,条]{s.}{baleia; cetáceo}
  \end{Phonetics}
\end{Entry}

\begin{Entry}{鲸鱼}{16,8}{⿂,⿂}
  \begin{Phonetics}{鲸鱼}{jing1yu2}
    \definition{s.}{baleia}
  \end{Phonetics}
\end{Entry}

\begin{Entry}{鲸鲨}{16,15}{⿂,⿂}
  \begin{Phonetics}{鲸鲨}{jing1sha1}
    \definition{s.}{tubarão baleia}
  \end{Phonetics}
\end{Entry}

%%%%%%%%%% 鹦 %%%%%%%%%%
\subsection*{鹦}\addcontentsline{loh}{figure}{鹦}

\begin{Entry}{鹦}{16}{⿃}
  \begin{Phonetics}{鹦}{ying1}
    \definition[只]{s.}{papagaio}
  \end{Phonetics}
\end{Entry}

\begin{Entry}{鹦鹉}{16,13}{⿃,⿃}
  \begin{Phonetics}{鹦鹉}{ying1wu3}
    \definition{s.}{papagaio (ave)}
  \end{Phonetics}
\end{Entry}

%%%%%%%%%% 鹾 %%%%%%%%%%
\subsection*{鹾}\addcontentsline{loh}{figure}{鹾}

\begin{Entry}{鹾}{16}{⿄}
  \begin{Phonetics}{鹾}{cuo2}
    \definition{adj.}{salgado}
    \definition{s.}{sal}
  \end{Phonetics}
\end{Entry}

%%%%%%%%%% 默 %%%%%%%%%%
\subsection*{默}\addcontentsline{loh}{figure}{默}

\begin{Entry}{默}{16}{⿊}
  \begin{Phonetics}{默}{mo4}
    \definition*{s.}{Sobrenome: Mo}
    \definition{adj.}{taciturno; reservado | silencioso}
    \definition{v.}{escrever de memória}
  \end{Phonetics}
\end{Entry}

\begin{Entry}{默契}{16,9}{⿊,⼤}
  \begin{Phonetics}{默契}{mo4qi4}[][HSK 7-9]
    \definition{adj.}{bem coordenado; mutuamente e tacitamente compreendido/acordado; descreve uma conexão profunda entre duas pessoas que transcende as palavras}
    \definition[些,种,份,点]{s.}{acordo ou contrato secreto; entendimento tácito}
  \synonymref{理解}{li3jie3}
  \antonymref{分歧}{fen1qi2}
  \end{Phonetics}
\end{Entry}

\begin{Entry}{默读}{16,10}{⿊,⾔}
  \begin{Phonetics}{默读}{mo4du2}[][HSK 7-9]
    \definition{v.}{ler em silêncio | subvocalizar}
  \antonymref{朗读}{lang3du2}
  \antonymref{朗诵}{lang3song4}
  \end{Phonetics}
\end{Entry}

\begin{Entry}{默默}{16,16}{⿊,⿊}
  \begin{Phonetics}{默默}{mo4mo4}[][HSK 4]
    \definition{adj.}{mudo; quieto; silencioso}
    \definition{adv.}{silenciosamente}
  \antonymref{高调}{gao1diao4}
  \end{Phonetics}
\end{Entry}

\begin{Entry}{默默无闻}{16,16,4,9}{⿊,⿊,⽆,⾨}
  \begin{Phonetics}{默默无闻}{mo4mo4-wu2wen2}[][HSK 7-9]
    \definition{expr.}{obscuro; quieto e desconhecido; desconhecido}
  \end{Phonetics}
\end{Entry}

%%%%% EOF %%%%%


 %%%
%%% 17画
%%%

\section*{17画}\addcontentsline{toc}{section}{17画}

\begin{Entry}{戴}{17}{⼽}
  \begin{Phonetics}{戴}{dai4}[][HSK 4]
    \definition*{s.}{Sobrenome Dai}
    \definition{v.}{usar/vestir (óculos, gravata, relógio de pulso, luvas); colocar objetos em sua cabeça, rosto, pescoço, peito, braços etc. | honrar; respeitar;}
  \end{Phonetics}
\end{Entry}

\begin{Entry}{擦}{17}{⼿}
  \begin{Phonetics}{擦}{ca1}[][HSK 4]
    \definition{v.}{enxugar; esfregar; apagar; limpar; limpar esfregando com um pano, toalha de mão, etc. | espalhar sobre; colocar sobre | passar raspando | ralar (em pedaços); ralar frutas em um ralador para fazer fios finos}
  \end{Phonetics}
\end{Entry}

\begin{Entry}{擦拭}{17,9}{⼿、⼿}
  \begin{Phonetics}{擦拭}{ca1shi4}
    \definition{v.}{limpar com um pano}
  \end{Phonetics}
\end{Entry}

\begin{Entry}{癌}{17}{⽧}
  \begin{Phonetics}{癌}{ai2}[][HSK 7,8,9]
    \definition{s.}{câncer; carcinoma; tumor maligno}
  \end{Phonetics}
\end{Entry}

\begin{Entry}{癌症}{17,10}{⽧、⽧}
  \begin{Phonetics}{癌症}{ai2zheng4}[][HSK 7,8,9]
    \definition[种]{s.}{câncer; tumores malignos no corpo}
  \end{Phonetics}
\end{Entry}

\begin{Entry}{瞧}{17}{⽬}
  \begin{Phonetics}{瞧}{qiao2}[][HSK 5]
    \definition{v.}{ver; olhar | tratar; diagnosticar e tratar | ver; visitar; fazer uma visita}
  \end{Phonetics}
\end{Entry}

\begin{Entry}{窾}{17}{⽳}
  \begin{Phonetics}{窾}{cuan4}
    \definition{adj.}{vazio | seco | destituído; pobre}
    \definition{s.}{buraco | lei}
    \definition{v.}{esconder}
  \end{Phonetics}
  \begin{Phonetics}{窾}{kuan3}
    \definition{adj.}{oco}
    \definition{s.}{rachadura; cavidade | (onomatopéia) água batendo na rocha}
    \definition{v.}{escavar um buraco}
  \end{Phonetics}
\end{Entry}

\begin{Entry}{糟}{17}{⽶}
  \begin{Phonetics}{糟}{zao1}[][HSK 5]
    \definition{adj.}{pobre; apodrecido; deteriorado | estragado; em uma bagunça; em um estado miserável (terrível) | (situação ou circunstância) ruim; desfavorável}
    \definition{s.}{resíduos de destilação de bebidas alcoólicas; resíduos do processo de fermentação do vinho}
    \definition{v.}{marinar alimentos em vinho ou mosto | desperdiçar; estragar; destruir}
  \end{Phonetics}
\end{Entry}

\begin{Entry}{糟糕}{17,16}{⽶、⽶}
  \begin{Phonetics}{糟糕}{zao1gao1}[][HSK 5]
    \definition{adj.}{(corpo, situação, etc.) muito ruim, péssimo}
    \definition{interj.}{que terrível; que má sorte; muito ruim}
  \end{Phonetics}
\end{Entry}

\begin{Entry}{繁}{17}{⽷}
  \begin{Phonetics}{繁}{fan2}
    \definition{adj.}{em grande número; numerosos; múltiplos (oposto a 简) | em grande número; numerosos; complexos; complicado}
    \definition{v.}{propagar; multiplicar}
  \seealsoref{简}{jian3}
  \end{Phonetics}
\end{Entry}

\begin{Entry}{繁荣}{17,9}{⽷、⾋}
  \begin{Phonetics}{繁荣}{fan2rong2}[][HSK 5]
    \definition{adj.}{florescente; próspero}
    \definition{v.}{promover; prosperar}
  \end{Phonetics}
\end{Entry}

\begin{Entry}{繁殖}{17,12}{⽷、⽍}
  \begin{Phonetics}{繁殖}{fan2zhi2}[][HSK 6]
    \definition{v.}{criar; reproduzir; propagar; multiplicar; os organismos produzem novos indivíduos}
  \end{Phonetics}
\end{Entry}

\begin{Entry}{藏}{17}{⾋}
  \begin{Phonetics}{藏}{cang2}[][HSK 6]
    \definition*{s.}{Sobrenome Cang}
    \definition{v.}{esconder; ocultar; esconder da vista | armazenar; coletar; colocar de lado}
  \end{Phonetics}
  \begin{Phonetics}{藏}{zang4}
    \definition*{s.}{Escrituras budistas ou taoístas; um termo geral para clássicos budistas ou taoístas | Região Autônoma do Tibete, 西藏}
    \definition{s.}{depósito; local de armazenamento; armazém; local onde grandes quantidades de coisas são armazenadas | os tibetanos, 藏族; grupo étnico Zang (ou tibetano)}
  \seealsoref{西藏}{xi1zang4}
  \seealsoref{藏族}{zang4zu2}
  \end{Phonetics}
\end{Entry}

\begin{Entry}{藏族}{17,11}{⾋、⽅}
  \begin{Phonetics}{藏族}{zang4zu2}
    \definition*{s.}{Etnia Zang (ou tibetana); Os Zangs (ou tibetanos) , distribuídos pela Região Autônoma do Tibete e pelas províncias de Qinghai, Sichuan, Gansu e Yunnan}
  \end{Phonetics}
\end{Entry}

\begin{Entry}{螺}{17}{⾍}
  \begin{Phonetics}{螺}{luo2}
    \definition{s.}{concha em espiral | caracol | búzio}
  \end{Phonetics}
\end{Entry}

\begin{Entry}{螺丝}{17,5}{⾍、⼀}
  \begin{Phonetics}{螺丝}{luo2si1}
    \definition{s.}{parafuso}
  \end{Phonetics}
\end{Entry}

\begin{Entry}{赢}{17}{⾙}
  \begin{Phonetics}{赢}{ying2}[][HSK 3]
    \definition{v.}{vencer; derrotar | ganhar; lucrar}
  \end{Phonetics}
\end{Entry}

\begin{Entry}{赢得}{17,11}{⾙、⼻}
  \begin{Phonetics}{赢得}{ying2 de2}[][HSK 4]
    \definition{v.}{ganhar; obter; conquistar; assegurar; garantir}
  \end{Phonetics}
\end{Entry}

\begin{Entry}{辫}{17}{⾟}
  \begin{Phonetics}{辫}{bian4}
    \definition{s.}{trança; rabo de cavalo | para coisas como uma trança}
  \end{Phonetics}
\end{Entry}

\begin{Entry}{辫子}{17,3}{⾟、⼦}
  \begin{Phonetics}{辫子}{bian4zi5}
    \definition[根,条]{s.}{trança | um erro ou falha que pode ser explorado por um oponente | alça}
  \end{Phonetics}
\end{Entry}

\begin{Entry}{邉}{17}{⾡}
  \begin{Phonetics}{邉}{bian1}
    \variantof{边}
  \end{Phonetics}
\end{Entry}

\begin{Entry}{霜}{17}{⾬}
  \begin{Phonetics}{霜}{shuang1}
    \definition{s.}{geada | pó branco ou creme espalhado por uma superfície | glacê | creme de pele}
  \end{Phonetics}
\end{Entry}

\begin{Entry}{鳄}{17}{⿂}
  \begin{Phonetics}{鳄}{e4}
    \definition{s.}{crocodilo;  jacaré}
  \end{Phonetics}
\end{Entry}

\begin{Entry}{鳄鱼}{17,8}{⿂、⿂}
  \begin{Phonetics}{鳄鱼}{e4yu2}
    \definition[条]{s.}{jacaré | crocodilo}
  \end{Phonetics}
\end{Entry}

\begin{Entry}{龠}{17}{⿕}[Kangxi 214]
  \begin{Phonetics}{龠}{yue4}
    \definition{clas.}{yue, uma unidade de medida seca para grãos (= 0,5 decilitro);}
    \definition{s.}{uma flauta curta antiga}
  \end{Phonetics}
\end{Entry}

%%%%% EOF %%%%%


 %%%
%%% 18画
%%%
\section*{18画}\addcontentsline{toc}{section}{18画}\addcontentsline{loh}{figure}{\#\#\#\# 18画}

%%%%%%%%%% 嚣 %%%%%%%%%%
\subsection*{嚣}\addcontentsline{loh}{figure}{嚣}

\begin{Entry}{嚣}{18}{⼝}
  \begin{Phonetics}{嚣}{xiao1}
    \definition*{s.}{Sobrenome: Xiao}
    \definition{adj.}{lazer}
    \definition{v.}{clamar; fazer barulho}
  \end{Phonetics}
\end{Entry}

\begin{Entry}{嚣张}{18,7}{⼝,⼸}
  \begin{Phonetics}{嚣张}{xiao1zhang1}
    \definition{adj.}{desenfreado | arrogante | agressivo}
  \end{Phonetics}
\end{Entry}

%%%%%%%%%% 懵 %%%%%%%%%%
\subsection*{懵}\addcontentsline{loh}{figure}{懵}

\begin{Entry}{懵}{18}{⼼}
  \begin{Phonetics}{懵}{meng3}
    \definition{adj.}{confuso; ignorante; irracional | inconsciente; entorpecido}
  \end{Phonetics}
\end{Entry}

\begin{Entry}{懵懂}{18,15}{⼼,⼼}
  \begin{Phonetics}{懵懂}{meng3dong3}
    \definition{adj.}{confuso | ignorante}
  \end{Phonetics}
\end{Entry}

%%%%%%%%%% 戳 %%%%%%%%%%
\subsection*{戳}\addcontentsline{loh}{figure}{戳}

\begin{Entry}{戳}{18}{⼽}
  \begin{Phonetics}{戳}{chuo1}[][HSK 7-9]
    \definition{s.}{selo; carimbo, abreviação de 戳记}
    \definition{v.}{cutucar; esfaquear | Dialeto: torcer; embotar | Dialeto: ficar em pé}
  \seealsoref{戳记}{chuo1ji4}
  \end{Phonetics}
\end{Entry}

\begin{Entry}{戳记}{18,5}{⼽,⾔}
  \begin{Phonetics}{戳记}{chuo1ji4}
    \definition{s.}{carimbo; selo}
  \end{Phonetics}
\end{Entry}

%%%%%%%%%% 毉 %%%%%%%%%%
\subsection*{毉}\addcontentsline{loh}{figure}{毉}

\begin{Entry}{毉}{18}{⼖}
  \begin{Phonetics}{毉}{yi1}
    \variantof{医}
  \end{Phonetics}
\end{Entry}

%%%%%%%%%% 瀑 %%%%%%%%%%
\subsection*{瀑}\addcontentsline{loh}{figure}{瀑}

\begin{Entry}{瀑}{18}{⽔}
  \begin{Phonetics}{瀑}{bao4}
    \definition{s.}{chuva torrencial; tempestade}
  \end{Phonetics}
  \begin{Phonetics}{瀑}{pu4}
    \definition{s.}{cachoeira; catarata}
  \end{Phonetics}
\end{Entry}

\begin{Entry}{瀑布}{18,5}{⽔,⼱}
  \begin{Phonetics}{瀑布}{pu4bu4}[][HSK 7-9]
    \definition[道,条]{s.}{queda de água; cachoeira; cascata; catarata}
  \end{Phonetics}
\end{Entry}

%%%%%%%%%% 翻 %%%%%%%%%%
\subsection*{翻}\addcontentsline{loh}{figure}{翻}

\begin{Entry}{翻}{18}{⽻}
  \begin{Phonetics}{翻}{fan1}[][HSK 4]
    \definition{v.}{virar; dar a volta; inverter; mudar de posição; torcer; reverter | vasculhar; procurar; pesquisar; mover objetos para localizar algo | reverter; retrair; retirar | passar por cima; ultrapassar; cruzar | multiplicar | traduzir; decodificar | romper-se; cair; desentender-se com alguém}
  \end{Phonetics}
\end{Entry}

\begin{Entry}{翻天覆地}{18,4,18,6}{⽻,⼤,⾑,⼟}
  \begin{Phonetics}{翻天覆地}{fan1tian1-fu4di4}[][HSK 7-9]
    \definition{expr.}{virar o mundo de cabeça para baixo; uma mudança tremenda; abalar a terra; marcar época; virar o céu e a terra; sacudir o próprio chão (mundo); virar o mundo de cabeça para baixo; mudanças titânicas; ``Céu e terra virados de cabeça para baixo.''}
  \end{Phonetics}
\end{Entry}

\begin{Entry}{翻过}{18,6}{⽻,⾡}
  \begin{Phonetics}{翻过}{fan1guo4}
    \definition{v.}{virar |  transformar}
  \end{Phonetics}
\end{Entry}

\begin{Entry}{翻来覆去}{18,7,18,5}{⽻,⽊,⾑,⼛}
  \begin{Phonetics}{翻来覆去}{fan1lai2-fu4qu4}[][HSK 7-9]
    \definition{expr.}{jogar de um lado para o outro; tocar a mesma corda; repetir várias vezes; virar e se virar; dizer repetidamente; jogar inquieto de um lado para o outro; virar de um lado para o outro}
  \end{Phonetics}
\end{Entry}

\begin{Entry}{翻译}{18,7}{⽻,⾔}
  \begin{Phonetics}{翻译}{fan1yi4}[][HSK 4]
    \definition[个,位,名]{s.}{tradutor; intérprete; pessoas que fazem trabalhos de tradução}
    \definition{v.}{traduzir; interpretar; colocar o significado de palavras de um idioma em palavras de outro idioma (expressão idiomática); expressar um significado em outro idioma}
  \end{Phonetics}
\end{Entry}

\begin{Entry}{翻脸}{18,11}{⽻,⾁}
  \begin{Phonetics}{翻脸}{fan1/lian3}
    \definition{v.+compl.}{brigar com alguém | tornar"-se hostil}
  \end{Phonetics}
\end{Entry}

\begin{Entry}{翻番}{18,12}{⽻,⽥}
  \begin{Phonetics}{翻番}{fan1/fan1}[][HSK 7-9]
    \definition{v.+compl.}{aumentar em um número especificado de vezes; dobrar}
  \end{Phonetics}
\end{Entry}

%%%%%%%%%% 藤 %%%%%%%%%%
\subsection*{藤}\addcontentsline{loh}{figure}{藤}

\begin{Entry}{藤}{18}{⾋}
  \begin{Phonetics}{藤}{teng2}
    \definition*{s.}{Teng}
    \definition[根,条]{s.}{cana; vime | videira}
  \end{Phonetics}
\end{Entry}

\begin{Entry}{藤椅}{18,12}{⾋,⽊}
  \begin{Phonetics}{藤椅}{teng2yi3}[][HSK 7-9]
    \definition{s.}{cadeira de vime}
  \end{Phonetics}
\end{Entry}

%%%%%%%%%% 覆 %%%%%%%%%%
\subsection*{覆}\addcontentsline{loh}{figure}{覆}

\begin{Entry}{覆}{18}{⾑}
  \begin{Phonetics}{覆}{fu4}
    \definition{v.}{cobrir; encapar | derrubar; perturbar; virar de cabeça para baixo}
  \end{Phonetics}
\end{Entry}

\begin{Entry}{覆盆子}{18,9,3}{⾑,⽫,⼦}
  \begin{Phonetics}{覆盆子}{fu4pen2zi5}
    \definition{s.}{framboesa}
  \end{Phonetics}
\end{Entry}

\begin{Entry}{覆盖}{18,11}{⾑,⽫}
  \begin{Phonetics}{覆盖}{fu4gai4}[][HSK 7-9]
    \definition{s.}{vegetação; cobertura vegetal; refere"-se às plantas que cobrem o solo}
    \definition{v.}{cobrir}
  \end{Phonetics}
\end{Entry}

%%%%%%%%%% 蹦 %%%%%%%%%%
\subsection*{蹦}\addcontentsline{loh}{figure}{蹦}

\begin{Entry}{蹦}{18}{⾜}
  \begin{Phonetics}{蹦}{beng4}[][HSK 7-9]
    \definition{v.}{pular; saltar; quicar}
  \end{Phonetics}
\end{Entry}

\begin{Entry}{蹦极}{18,7}{⾜,⽊}
  \begin{Phonetics}{蹦极}{beng4ji2}
    \definition{s.}{\emph{bungee jumping}}
  \end{Phonetics}
\end{Entry}

%%%%%%%%%% 鞭 %%%%%%%%%%
\subsection*{鞭}\addcontentsline{loh}{figure}{鞭}

\begin{Entry}{鞭}{18}{⾰}
  \begin{Phonetics}{鞭}{bian1}
    \definition[条]{s.}{chicote; açoite; chibata | um bastão de ferro usado como arma na China antiga | algo parecido com um chicote | uma série de pequenos fogos de artifício | pênis de animal; refere"-se ao pênis de certos mamíferos usado para fins medicinais ou comestíveis}
    \definition{v.}{açoitar; chicotear; flagelar}
  \end{Phonetics}
\end{Entry}

\begin{Entry}{鞭炮}{18,9}{⾰,⽕}
  \begin{Phonetics}{鞭炮}{bian1pao4}[][HSK 7-9]
    \definition[串,挂,盒,捆,箱,个]{s.}{\emph{maroon}, um tipo de foguete usado como alarme ou aviso; fogos de artifício; um termo geral para fogos de artifício grandes e pequenos}
  \end{Phonetics}
\end{Entry}

\begin{Entry}{鞭策}{18,12}{⾰,⽵}
  \begin{Phonetics}{鞭策}{bian1ce4}[][HSK 7-9]
    \definition{v.}{estimular; incitar; incentivar}
  \end{Phonetics}
\end{Entry}

%%%%%%%%%% 鼂 %%%%%%%%%%
\subsection*{鼂}\addcontentsline{loh}{figure}{鼂}

\begin{Entry}{鼂}{18}{⿌}
  \begin{Phonetics}{鼂}{chao2}
    \definition*{s.}{Sobrenome: Chao}
    \definition{s.}{tartaruga marinha}
  \end{Phonetics}
\end{Entry}

%%%%% EOF %%%%%


 %%%
%%% 19画
%%%
\section*{19画}\addcontentsline{toc}{section}{19画}\addcontentsline{loh}{figure}{\#\#\#\# 19画}

%%%%%%%%%% 巅 %%%%%%%%%%
\subsection*{巅}\addcontentsline{loh}{figure}{巅}

\begin{Entry}{巅}{19}{⼭}
  \begin{Phonetics}{巅}{dian1}
    \definition[个]{s.}{pico da montanha; cume; topo da montanha}
  \end{Phonetics}
\end{Entry}

\begin{Entry}{巅峰}{19,10}{⼭,⼭}
  \begin{Phonetics}{巅峰}{dian1feng1}[][HSK 7-9]
    \definition{s.}{um cume; um pico de montanha}
  \end{Phonetics}
\end{Entry}

%%%%%%%%%% 攀 %%%%%%%%%%
\subsection*{攀}\addcontentsline{loh}{figure}{攀}

\begin{Entry}{攀}{19}{⼿}
  \begin{Phonetics}{攀}{pan1}[][HSK 7-9]
    \definition{v.}{escalar; escalar | buscar conexões em altos cargos | envolver; implicar | agarrar; agarrar-se; segurar-se a}
  \end{Phonetics}
\end{Entry}

\begin{Entry}{攀升}{19,4}{⼿,⼗}
  \begin{Phonetics}{攀升}{pan1sheng1}[][HSK 7-9]
    \definition{v.}{subir para um ponto mais alto | (preços, quantidade, etc.) subir; aumentar; escalar | subir}
  \end{Phonetics}
\end{Entry}

\begin{Entry}{攀岩}{19,8}{⼿,⼭}
  \begin{Phonetics}{攀岩}{pan1yan2}
    \definition{s.}{escalada em rocha; isso se refere a esse tipo de esporte}
    \definition{v.}{escalar uma parede rochosa íngreme com equipamento mínimo}
  \end{Phonetics}
\end{Entry}

\begin{Entry}{攀爬}{19,8}{⼿,⽖}
  \begin{Phonetics}{攀爬}{pan1pa2}
    \definition{v.}{escalar; escalada em rocha; refere"-se ao movimento em uma determinada direção usando apenas as mãos e os pés, com o mínimo uso de ferramentas}
  \end{Phonetics}
\end{Entry}

%%%%%%%%%% 曝 %%%%%%%%%%
\subsection*{曝}\addcontentsline{loh}{figure}{曝}

\begin{Entry}{曝}{19}{⽇}
  \begin{Phonetics}{曝}{bao4}
    \definition{v.}{usado em  曝光}
  \seealsoref{曝光}{bao4/guang1}
  \end{Phonetics}
  \begin{Phonetics}{曝}{pu4}
    \definition{v.}{expor ao sol}
  \end{Phonetics}
\end{Entry}

\begin{Entry}{曝光}{19,6}{⽇,⼉}
  \begin{Phonetics}{曝光}{bao4/guang1}[][HSK 7-9]
    \definition{v.+compl.}{expor; sensibilizar filme fotográfico ou papel fotossensível para formar uma imagem latente | expor; tornar (algo ruim) público; metáfora para revelar coisas secretas (geralmente vergonhosas) ao mundo}
  \end{Phonetics}
\end{Entry}

%%%%%%%%%% 爆 %%%%%%%%%%
\subsection*{爆}\addcontentsline{loh}{figure}{爆}

\begin{Entry}{爆}{19}{⽕}
  \begin{Phonetics}{爆}{bao4}[][HSK 6]
    \definition{v.}{explodir; estourar | fritar rapidamente; ferver rapidamente | aparecer (ou ocorrer) inesperadamente}
  \end{Phonetics}
\end{Entry}

\begin{Entry}{爆发}{19,5}{⽕,⼜}
  \begin{Phonetics}{爆发}{bao4fa1}[][HSK 6]
    \definition{v.}{entrar em erupção; explodir | estourar; irromper; ocorrer de forma repentina e violenta}
  \end{Phonetics}
\end{Entry}

\begin{Entry}{爆竹}{19,6}{⽕,⽵}
  \begin{Phonetics}{爆竹}{bao4 zhu2}[][HSK 7-9]
    \definition[串,个]{s.}{fogo de artifício}
  \end{Phonetics}
\end{Entry}

\begin{Entry}{爆米花}{19,6,7}{⽕,⽶,⾋}
  \begin{Phonetics}{爆米花}{bao4mi3hua1}
    \definition{s.}{pipoca (de milho) | pipoca de arroz}
  \end{Phonetics}
\end{Entry}

\begin{Entry}{爆冷门}{19,7,3}{⽕,⼎,⾨}
  \begin{Phonetics}{爆冷门}{bao4 leng3men2}[][HSK 7-9]
    \definition{s.}{um avanço | uma reviravolta (especialmente nos esportes) | reviravolta inesperada dos acontecimentos}
    \definition{v.}{dar um golpe}
  \end{Phonetics}
\end{Entry}

\begin{Entry}{爆炸}{19,9}{⽕,⽕}
  \begin{Phonetics}{爆炸}{bao4zha4}[][HSK 6]
    \definition{s.}{explosão}
    \definition{v.}{explodir; explodir; detonar | aumentar bruscamente em um curto espaço de tempo (de quantidade)}
  \end{Phonetics}
\end{Entry}

\begin{Entry}{爆满}{19,13}{⽕,⽔}
  \begin{Phonetics}{爆满}{bao4man3}[][HSK 7-9]
    \definition{v.}{(teatro, cinema, estádio, etc.) lotar; ter casa cheia | estar lotado}
  \end{Phonetics}
\end{Entry}

%%%%%%%%%% 聼 %%%%%%%%%%
\subsection*{聼}\addcontentsline{loh}{figure}{聼}

\begin{Entry}{聼}{19}{⼼}
  \begin{Phonetics}{聼}{ting1}
    \variantof{听}
  \end{Phonetics}
\end{Entry}

%%%%%%%%%% 蘑 %%%%%%%%%%
\subsection*{蘑}\addcontentsline{loh}{figure}{蘑}

\begin{Entry}{蘑}{19}{⾋}
  \begin{Phonetics}{蘑}{mo2}
    \definition{s.}{cogumelo}
  \end{Phonetics}
\end{Entry}

\begin{Entry}{蘑菇}{19,11}{⾋,⾋}
  \begin{Phonetics}{蘑菇}{mo2gu5}[][HSK 7-9]
    \definition[个,朵,斤,种]{s.}{cogumelo; termo genérico para fungos em forma de guarda"-chuva; referindo"-se especificamente a cogumelos champignon ou cogumelos shiitake}
    \definition{v.}{afligir; importunar; insistir em | demorar; enrolar; movimentar"-se lenta e arrastadamente}
  \end{Phonetics}
\end{Entry}

%%%%%%%%%% 警 %%%%%%%%%%
\subsection*{警}\addcontentsline{loh}{figure}{警}

\begin{Entry}{警}{19}{⾔}
  \begin{Phonetics}{警}{jing3}
    \definition{s.}{policial}
    \definition{v.}{alertar | avisar}
  \end{Phonetics}
\end{Entry}

\begin{Entry}{警车}{19,4}{⾔,⾞}
  \begin{Phonetics}{警车}{jing3che1}[][HSK 7-9]
    \definition{s.}{carro (ou van) da polícia}
  \end{Phonetics}
\end{Entry}

\begin{Entry}{警告}{19,7}{⾔,⼝}
  \begin{Phonetics}{警告}{jing3gao4}[][HSK 5]
    \definition[个]{s.}{advertência (como medida disciplinar); uma forma de punição}
    \definition{v.}{avisar; advertir; admoestar}
  \end{Phonetics}
\end{Entry}

\begin{Entry}{警官}{19,8}{⾔,⼧}
  \begin{Phonetics}{警官}{jing3guan1}[][HSK 7-9]
    \definition[名]{s.}{policial; guarda; agente; oficial de polícia}
  \end{Phonetics}
\end{Entry}

\begin{Entry}{警钟}{19,9}{⾔,⾦}
  \begin{Phonetics}{警钟}{jing3zhong1}[][HSK 7-9]
    \definition{s.}{sino de alarme; toque de alarme; sirene}
  \end{Phonetics}
\end{Entry}

\begin{Entry}{警惕}{19,11}{⾔,⼼}
  \begin{Phonetics}{警惕}{jing3ti4}[][HSK 7-9]
    \definition{v.}{estar vigilante; ficar atento; estar em alerta; estar em guarda contra; estar muito atento aos perigos potenciais ou tendências errôneas}
  \end{Phonetics}
\end{Entry}

\begin{Entry}{警察}{19,14}{⾔,⼧}
  \begin{Phonetics}{警察}{jing3cha2}[][HSK 3]
    \definition[个,位,名,群,队]{s.}{polícia; policial; oficial de polícia; as forças armadas que mantêm a segurança social do país são uma parte importante do aparato estatal; também se refere aos membros dessas forças armadas}
  \end{Phonetics}
\end{Entry}

%%%%%%%%%% 蹬 %%%%%%%%%%
\subsection*{蹬}\addcontentsline{loh}{figure}{蹬}

\begin{Entry}{蹬}{19}{⾜}
  \begin{Phonetics}{蹬}{deng1}[][HSK 7-9]
    \definition{v.}{pressionar com o pé; pisar; pisar em | Dialeto: calçar (sapatos ou calças); usar (sapatos) | Gíria: despejar (algo)}
  \end{Phonetics}
  \begin{Phonetics}{蹬}{deng4}
    \definition{s.}{lutar; ter dificuldade}
  \seealsoref{蹭蹬}{ceng4deng4}
  \end{Phonetics}
\end{Entry}

%%%%%%%%%% 蹭 %%%%%%%%%%
\subsection*{蹭}\addcontentsline{loh}{figure}{蹭}

\begin{Entry}{蹭}{19}{⾜}
  \begin{Phonetics}{蹭}{ceng4}[][HSK 7-9]
    \definition{v.}{esfregar; raspar; arranhar | esfregar em algo e ficar manchado; ser manchado com; manchar por fricção | mover"-se lentamente; demorar"-se; arrastar"-se | Dialeto: roubar}
  \end{Phonetics}
\end{Entry}

\begin{Entry}{蹭蹬}{19,19}{⾜,⾜}
  \begin{Phonetics}{蹭蹬}{ceng4deng4}
    \definition{interj.}{``Droga!''}
    \definition{v.}{enfrentar contratempos; estar sem sorte; ter má sorte}
  \end{Phonetics}
\end{Entry}

%%%%%%%%%% 蹲 %%%%%%%%%%
\subsection*{蹲}\addcontentsline{loh}{figure}{蹲}

\begin{Entry}{蹲}{19}{⾜}
  \begin{Phonetics}{蹲}{dun1}[][HSK 6]
    \definition{v.}{agachamento sobre os calcanhares; dobrar as pernas o máximo possível, como se estivesse sentado, mas não deixar as nádegas tocarem o chão | ficar; metáfora para ficar ocioso em casa}
  \end{Phonetics}
\end{Entry}

\begin{Entry}{蹲下}{19,3}{⾜,⼀}
  \begin{Phonetics}{蹲下}{dun1xia4}
    \definition{v.}{agachar | agachar-se}
  \end{Phonetics}
\end{Entry}

%%%%%%%%%% 颤 %%%%%%%%%%
\subsection*{颤}\addcontentsline{loh}{figure}{颤}

\begin{Entry}{颤}{19}{⾴}
  \begin{Phonetics}{颤}{chan4}
    \definition{v.}{tremer; estremecer | vibrar; tremer; sacudir}
  \end{Phonetics}
\end{Entry}

\begin{Entry}{颤抖}{19,7}{⾴,⼿}
  \begin{Phonetics}{颤抖}{chan4dou3}[][HSK 7-9]
    \definition{v.}{tremer; estremecer; tremular; tiritar}
  \end{Phonetics}
\end{Entry}

%%%%% EOF %%%%%


 %%%
%%% 20画
%%%

\section*{20画}\addcontentsline{toc}{section}{20画}

\begin{Entry}{嚼}{20}{⼝}
  \begin{Phonetics}{嚼}{jiao2}[][HSK 7-9]
    \definition{v.}{mastigar; mascar; limitado para uso em 过屠门而大嚼}
  \seealsoref{过屠门而大嚼}{guo4 tu2men2 er2 da4 jiao2}
  \end{Phonetics}
  \begin{Phonetics}{嚼}{jiao4}
    \definition{v.}{mascar; ruminar}
  \end{Phonetics}
  \begin{Phonetics}{嚼}{jue2}
    \definition{v.}{mastigar; morder; mastigar completamente; é usado em algumas palavras compostas e expressões idiomáticas; usado em 咀嚼}
  \seealsoref{咀嚼}{ju3jue2}
  \end{Phonetics}
\end{Entry}

\begin{Entry}{壤}{20}{⼟}
  \begin{Phonetics}{壤}{rang3}
    \definition{s.}{solo | terra | (literário) a terra (em contraste com o céu 天)}
  \end{Phonetics}
\end{Entry}

\begin{Entry}{灌}{20}{⽔}
  \begin{Phonetics}{灌}{guan4}[][HSK 7-9]
    \definition*{s.}{Sobrenome Guan}
    \definition{s.}{arbusto; aglomerados de árvores baixas | irrigação}
    \definition{v.}{irrigar (rega e irrigação do solo) | encher; despejar; injetar | gravar; refere-se à gravação (música)}
  \end{Phonetics}
\end{Entry}

\begin{Entry}{灌溉}{20,12}{⽔、⽔}
  \begin{Phonetics}{灌溉}{guan4gai4}[][HSK 7-9]
    \definition{v.}{regar; irrigar}
  \end{Phonetics}
\end{Entry}

\begin{Entry}{灌输}{20,13}{⽔、⾞}
  \begin{Phonetics}{灌输}{guan4shu1}[][HSK 7-9]
    \definition{v.}{implantar; incutir em; inculcar; imbuir com (ideias, conhecimento); transmitir (ideias, conhecimento, etc.) | canalizar água; despejar água em; direcionar a água para onde ela é necessária}
  \end{Phonetics}
\end{Entry}

\begin{Entry}{譬}{20}{⾔}
  \begin{Phonetics}{譬}{pi4}
    \definition{s.}{exemplo; analogia; metáfora}
    \definition{v.}{dar um exemplo; fazer uma analogia}
  \end{Phonetics}
\end{Entry}

\begin{Entry}{譬如}{20,6}{⾔、⼥}
  \begin{Phonetics}{譬如}{pi4ru2}
    \definition{conj.}{por exemplo | como}
  \end{Phonetics}
\end{Entry}

\begin{Entry}{魔}{20}{⿁}
  \begin{Phonetics}{魔}{mo2}
    \definition{adj.}{místico; misterioso; mágico}
    \definition{s.}{espírito maligno; demônio; diabo; monstro | mágico; místico}
  \end{Phonetics}
\end{Entry}

\begin{Entry}{魔头}{20,5}{⿁、⼤}
  \begin{Phonetics}{魔头}{mo2tou2}
    \definition{s.}{monstro | diabo}
  \end{Phonetics}
\end{Entry}

\begin{Entry}{魔术}{20,5}{⿁、⽊}
  \begin{Phonetics}{魔术}{mo2shu4}
    \definition{s.}{magia}
  \end{Phonetics}
\end{Entry}

%%%%% EOF %%%%%


 %%%
%%% 21画
%%%

\section*{21画}\addcontentsline{toc}{section}{21画}

\begin{entry}{露}{21}{⾬}
  \begin{phonetics}{露}{lou4}[][HSK 6]
    \definition{v.}{mostrar; apresentar (uma certa emoção ou olhar no rosto) | mostrar; aparentar; fazer algo visível; as pessoas podem ver}
  \end{phonetics}
  \begin{phonetics}{露}{lu4}[][HSK 6]
    \definition{adj.}{fora de uma casa, tenda, etc., sem cobertura}
    \definition{s.}{orvalho; gotas de água condensadas | xarope; suco de fruta; bebida destilada de flores, folhas ou frutos}
    \definition{v.}{revelar; expor; mostrar; trair}
  \end{phonetics}
\end{entry}

\begin{entry}{露珠}{21,10}{⾬、⽟}
  \begin{phonetics}{露珠}{lu4zhu1}
    \definition{s.}{orvalho}
  \end{phonetics}
\end{entry}

\begin{entry}{霸}{21}{⾬}
  \begin{phonetics}{霸}{ba4}
    \definition*{s.}{Sobrenome Ba}
    \definition{adj.}{arrogante; dominador; tirânico}
    \definition{s.}{líder dos senhores feudais; suserano | tirano; déspota; valentão; \emph{bully} | poder hegemônico; hegemonismo; hegemonia | chefe dos príncipes feudais; líder da antiga aliança feudal}
    \definition{v.}{dominar; tiranizar; governar (ocupar) pela força}
  \end{phonetics}
\end{entry}

\begin{entry}{霸权}{21,6}{⾬、⽊}
  \begin{phonetics}{霸权}{ba4quan2}
    \definition{s.}{hegemonia | supremacia}
  \end{phonetics}
\end{entry}

\begin{entry}{鷄}{21}{⿃}
  \begin{phonetics}{鷄}{ji1}
    \variantof{鸡}
  \end{phonetics}
\end{entry}

%%%%% EOF %%%%%


 %%%
%%% 22画
%%%
\section*{22画}\addcontentsline{toc}{section}{22画}\addcontentsline{loh}{figure}{\#\#\#\# 22画}

%%%%%%%%%% 镶 %%%%%%%%%%
\subsection*{镶}\addcontentsline{loh}{figure}{镶}

\begin{Entry}{镶}{22}{⾦}
  \begin{Phonetics}{镶}{xiang1}
    \definition{v.}{para incrustar; cravar; montar | marginar; orlar; rodear; decorar; colocar uma moldura | inserir; integrar}
  \end{Phonetics}
\end{Entry}

%%%%% EOF %%%%%


 %%%
%%% 23画
%%%
\section*{23画}\addcontentsline{toc}{section}{23画}

%%%%%%%%%% 罐 %%%%%%%%%%
\subsection*{罐}

\begin{Entry}{罐}{23}{⽸}
  \begin{Phonetics}{罐}{guan4}[][HSK 7-9]
    \definition{clas.}{lata; jarra; gavetas e recipientes de água feitos de cerâmica ou metal}[我买了一罐可乐。===Comprei uma lata de Coca-Cola.]
    \definition{s.}{lata; jarra; jarro; pote; tanque | cuba de carvão; vagão de caçamba para carregamento de carvão em minas de carvão}
  \end{Phonetics}
\end{Entry}

\begin{Entry}{罐头}{23,5}{⽸、⼤}
  \begin{Phonetics}{罐头}{guan4tou5}[][HSK 7-9]
    \definition[个,盒,瓶]{s.}{lata; jarra | enlatado; comida enlatada é a abreviação de 罐头食品, que é processada e embalada em latas de ferro seladas ou garrafas de vidro, e pode ser armazenada por um longo tempo}
  \seealsoref{罐头食品}{guan4tou2 shi2pin3}
  \end{Phonetics}
\end{Entry}

\begin{Entry}{罐头食品}{23,5,9,9}{⽸、⼤、⾷、⼝}
  \begin{Phonetics}{罐头食品}{guan4tou2 shi2pin3}
    \definition{s.}{alimentos enlatados; produtos enlatados}
  \end{Phonetics}
\end{Entry}

%%%%% EOF %%%%%


% \input{groups.by.strokes/024.tex}
% \input{groups.by.strokes/025.tex}
% \input{groups.by.strokes/026.tex}
% \input{groups.by.strokes/027.tex}
% \input{groups.by.strokes/028.tex}
% \input{groups.by.strokes/029.tex}
% \input{groups.by.strokes/030.tex}
\end{DictionaryEntries}

\pagestyle{plain}

\ifdraftdoc 
%
%
%
\else

\clearpage
\chapter{Termos Gramaticais Chineses}
\input{include/termos.gramaticais.tex}

\clearpage
\chapter{Classificadores Nominais}
\input{include/classificadores.nominais.tex}

\clearpage
\chapter{Classificadores Verbais}
\input{include/classificadores.verbais.tex}

\clearpage
\chapter{Verbos Direcionais}
\input{include/verbos.direcionais.tex}

\clearpage
\chapter{Locativos}
%%%%%%%%%%%%%%%%%%%%%%%%%%%%%%%%%%%%%%%%%%%%%%%%%%%%%%%%%%%%%%%%%%%%%%%%%%%%%%%
%%%%%%%%%%%%%%%%%%%%%%%%%%%%%%%%%%%%%%%%%%%%%%%%%%%%%%%%%%%%%%%%%%%%%%%%%%%%%%%
%%%%%                                                                     %%%%%
%%%%% locativos.tex:                                                      %%%%%
%%%%% Tabela com os locativos chineses                                    %%%%%
%%%%%                                                                     %%%%%
%%%%%%%%%%%%%%%%%%%%%%%%%%%%%%%%%%%%%%%%%%%%%%%%%%%%%%%%%%%%%%%%%%%%%%%%%%%%%%%
%%%%%%%%%%%%%%%%%%%%%%%%%%%%%%%%%%%%%%%%%%%%%%%%%%%%%%%%%%%%%%%%%%%%%%%%%%%%%%%

%%% Ajustes para a tabela
\DefTblrTemplate{caption}{default}{}
\DefTblrTemplate{capcont}{default}{ \UseTblrTemplate{conthead-text}{default} }
\DefTblrTemplate{contfoot-text}{default}{Continua na próxima página.}
\DefTblrTemplate{conthead-text}{default}{(Continuação)}
\DefTblrTemplate{firsthead}{default}{ \UseTblrTemplate{caption}{default} }
\DefTblrTemplate{middlehead,lasthead}{default}{ \UseTblrTemplate{conthead}{default} }
\DefTblrTemplate{firstfoot,middlefoot}{default}{ \UseTblrTemplate{contfoot}{default} }
\DefTblrTemplate{lastfoot}{default}{ \UseTblrTemplate{note}{default} \UseTblrTemplate{remark}{default} }

%%% Tabela
\begin{longtblr}
{
 colspec = {cccccc},
 width = 1\linewidth,
 hlines = {white},
 vlines = {white},
 rowhead = 1, rowfoot = 0,
 row{1} = {font=\bfseries, bg=gray8, fg=black},
 column{1} = {font=\bfseries, bg=gray8, fg=black},
 cell{1}{1} = {bg=white},
 cell{2-Z}{2-Z} = {bg=gray9},
 cell{6}{5-6} = {bg=white},
 cell{7}{2-4} = {bg=white},
 cell{9}{2-5} = {bg=white},
 cell{10}{3-6} = {bg=white},
 cell{11}{2-5} = {bg=white},
 cell{12}{4-6} = {bg=white},
 cell{13}{4-6} = {bg=white},
}
                                           & {边\\   \normalsize\dpy{bian1}}        & {面\\   \normalsize\dpy{mian4}}        & {头\\   \normalsize\dpy{tou5}}        & {以\\   \normalsize\dpy{yi3}}        & {之\\   \normalsize\dpy{zhi1}}        \\
{上\\ \normalsize\dpy{shang4}\\ sobre}     & {上边\\ \normalsize\dpy{shang4 bian1}} & {上面\\ \normalsize\dpy{shang4 mian4}} & {上头\\ \normalsize\dpy{shang4 tou5}} & {以上\\ \normalsize\dpy{yi3 shang4}} & {之上\\ \normalsize\dpy{zhi1 shang4}} \\
{下\\ \normalsize\dpy{xia4}\\ sob}         & {下边\\ \normalsize\dpy{xia4 bian1}}   & {下面\\ \normalsize\dpy{xia4 mian4}}   & {下头\\ \normalsize\dpy{xia4 tou5}}   & {以下\\ \normalsize\dpy{yi3 xia4}}   & {之下\\ \normalsize\dpy{zhi1 xia4}}   \\
{前\\ \normalsize\dpy{qian2}\\ na frente}  & {前边\\ \normalsize\dpy{qian2 bian1}}  & {前面\\ \normalsize\dpy{qian2 mian4}}  & {前头\\ \normalsize\dpy{qian2 tou5}}  & {以前\\ \normalsize\dpy{yi3 qian2}}  & {之前\\ \normalsize\dpy{zhi1 qian2}}  \\
{后\\ \normalsize\dpy{hou4}\\ atrás}       & {后边\\ \normalsize\dpy{hou4 bian1}}   & {后面\\ \normalsize\dpy{hou4 mian4}}   & {后头\\ \normalsize\dpy{hou4 tou5}}   & {以后\\ \normalsize\dpy{yi3 hou4}}   & {之后\\ \normalsize\dpy{zhi1 hou4}}   \\
{里\\ \normalsize\dpy{li3}\\ dentro}       & {里边\\ \normalsize\dpy{li3 bian1}}    & {里面\\ \normalsize\dpy{li3 mian4}}    & {里头\\ \normalsize\dpy{li3 tou5}}    &                                      &                                       \\
{内\\ \normalsize\dpy{nei4}\\ no interior} &                                        &                                        &                                       & {以内\\ \normalsize\dpy{yi3 nei4}}   & {之内\\ \normalsize\dpy{zhi1 nei4}}   \\
{外\\ \normalsize\dpy{wai4}\\ no exterior} & {外边\\ \normalsize\dpy{wai4 bian1}}   & {外面\\ \normalsize\dpy{wai4 mian4}}   & {外头\\ \normalsize\dpy{wai4 tou5}}   & {以外\\ \normalsize\dpy{yi3 wai4}}   & {之外\\ \normalsize\dpy{zhi1 wai4}}   \\
{间\\ \normalsize\dpy{jian1}\\ entre}      &                                        &                                        &                                       &                                      & {之间\\ \normalsize\dpy{zhi1 jian1}}  \\
{旁\\ \normalsize\dpy{pang2}\\ ao lado}    & {旁边\\ \normalsize\dpy{pang2 bian1}}  &                                        &                                       &                                      &                                       \\
{中\\ \normalsize\dpy{zhong1}\\ no meio}   &                                        &                                        &                                       &                                      & {之中\\ \normalsize\dpy{zhi1 zhong1}} \\
{左\\ \normalsize\dpy{zuo3}\\ à esquerda}  & {左边\\ \normalsize\dpy{zuo3 bian1}}   & {左面\\ \normalsize\dpy{zuo3 mian4}}   &                                       &                                      &                                       \\
{右\\ \normalsize\dpy{you4}\\ à direita}   & {右边\\ \normalsize\dpy{you4 bian1}}   & {右面\\ \normalsize\dpy{you4 mian4}}   &                                       &                                      &                                       \\
\pagebreak
{东\\ \normalsize\dpy{dong1}\\ no leste}   & {东边\\ \normalsize\dpy{dong1 bian1}}  & {东面\\ \normalsize\dpy{dong1 mian4}}  & {东头\\ \normalsize\dpy{dong1 tou5}}  & {以东\\ \normalsize\dpy{yi3 dong1}}  & {之东\\ \normalsize\dpy{zhi1 dong1}}  \\
{南\\ \normalsize\dpy{nan2}\\ no sul}      & {南边\\ \normalsize\dpy{nan2 bian1}}   & {南面\\ \normalsize\dpy{nan2 mian4}}   & {南头\\ \normalsize\dpy{nan2 tou5}}   & {以南\\ \normalsize\dpy{yi3 nan2}}   & {之南\\ \normalsize\dpy{zhi1 nan2}}   \\
{西\\ \normalsize\dpy{xi1}\\ no oeste}     & {西边\\ \normalsize\dpy{xi1 bian1}}    & {西面\\ \normalsize\dpy{xi1 mian4}}    & {西头\\ \normalsize\dpy{xi1 tou5}}    & {以西\\ \normalsize\dpy{yi3 xi1}}    & {之西\\ \normalsize\dpy{zhi1 xi1}}    \\
{北\\ \normalsize\dpy{bei3}\\ n norte}     & {北边\\ \normalsize\dpy{bei3 bian1}}   & {北面\\ \normalsize\dpy{bei3 mian4}}   & {北头\\ \normalsize\dpy{bei3 tou5}}   & {以北\\ \normalsize\dpy{yi3 bei3}}   & {之北\\ \normalsize\dpy{zhi1 bei3}}   \\
\end{longtblr}

%%%%% EOF %%%%%


\clearpage
\chapter{Radicais Kangxi}
%%%%%%%%%%%%%%%%%%%%%%%%%%%%%%%%%%%%%%%%%%%%%%%%%%%%%%%%%%%%%%%%%%%%%%%%%%%%%%%
%%%%%%%%%%%%%%%%%%%%%%%%%%%%%%%%%%%%%%%%%%%%%%%%%%%%%%%%%%%%%%%%%%%%%%%%%%%%%%%
%%%%%                                                                     %%%%%
%%%%% radicais_kangxi.tex:                                                %%%%%
%%%%% Lista dos 214 radicais Kangxi utilizados nos caracteres chineses.   %%%%%
%%%%%                                                                     %%%%%
%%%%%%%%%%%%%%%%%%%%%%%%%%%%%%%%%%%%%%%%%%%%%%%%%%%%%%%%%%%%%%%%%%%%%%%%%%%%%%%
%%%%%%%%%%%%%%%%%%%%%%%%%%%%%%%%%%%%%%%%%%%%%%%%%%%%%%%%%%%%%%%%%%%%%%%%%%%%%%%

%%% Ajustes para a tabela
\DefTblrTemplate{caption}{default}{}
\DefTblrTemplate{capcont}{default}{ \UseTblrTemplate{conthead-text}{default} }
\DefTblrTemplate{contfoot-text}{default}{Continua na próxima página.}
\DefTblrTemplate{conthead-text}{default}{(Continuação)}
\DefTblrTemplate{firsthead}{default}{ \UseTblrTemplate{caption}{default} }
\DefTblrTemplate{middlehead,lasthead}{default}{ \UseTblrTemplate{conthead}{default} }
\DefTblrTemplate{firstfoot,middlefoot}{default}{ \UseTblrTemplate{contfoot}{default} }
\DefTblrTemplate{lastfoot}{default}{ \UseTblrTemplate{note}{default} \UseTblrTemplate{remark}{default} }

%%% Tabela
\begin{longtblr}
{
  colspec = {|r|ll|l|l|}, hlines,
  width = 1\linewidth,
  rowhead = 1, rowfoot = 0,
  row{1} = {font=\bfseries, fg=white, bg=black},
  row{2-Z} = {font=\normalfont},
}
\textbf{Nº} & \SetCell[c=2]{c}\textbf{Radical e\\Variantes} & 2-2 & \textbf{Tradução} & \textbf{Pinyin} \\
1 & 一 & & um & \dictpinyin{yi1} \\
2 & 丨 & & linha & \dictpinyin{shu4} \\
3 & 丶 & & ponto, indica um fim & \dictpinyin{dian3} \\
4 & 丿 & 乀,乁 & cortar, dobrar & \dictpinyin{pie3} \\
5 & 乙 & 乚、乛、⺄ & segundo, anzol & \dictpinyin{yi3} \\
6 & 亅 & & gancho & \dictpinyin{gou1} \\
7 & 二 & & dois & \dictpinyin{er4} \\
8 & 亠 & & tampa & \dictpinyin{tou2} \\
9 & 人 & 亻、𠆢 & pessoa & \dictpinyin{ren2} \\
10 & 儿 & & pernas & \dictpinyin{er2} \\
11 & 入 & & entrar, juntar-se & \dictpinyin{ru4} \\
12 & 八 & 丷 & oito & \dictpinyin{ba1} \\
13 & 冂 & & largo, exterior & \dictpinyin{jiong3} \\
14 & 冖 & & capa de pano & \dictpinyin{mi4} \\
15 & 冫 & & gelo & \dictpinyin{bing1} \\
16 & 几 & & mesa pequena & \dictpinyin{ji1},\dictpinyin{ji3} \\
17 & 凵 & & receptáculo, caixa aberta & \dictpinyin{qu3} \\
18 & 刀 & 刂、⺈ & faca & \dictpinyin{dao1} \\
19 & 力 & & poder, força & \dictpinyin{li4} \\
20 & 勹 & & invólucro & \dictpinyin{bao1} \\
21 & 匕 & & colher & \dictpinyin{bi3} \\
22 & 匚 & & caixa & \dictpinyin{fang1} \\
23 & 匸 & & compartimento oculto & \dictpinyin{xi3} \\
24 & 十 & & dez, completo, perfeito & \dictpinyin{shi2} \\
25 & 卜 & & advinhação, divinação & \dictpinyin{bu3} \\
26 & 卩 & 㔾 & foca & \dictpinyin{jie2} \\
27 & 厂 & & penhasco, precipício & \dictpinyin{han4} \\
28 & 厶 & & privado & \dictpinyin{si1} \\
29 & 又 & & mão direita, e, novamente & \dictpinyin{you4} \\
30 & 口 & & boca & \dictpinyin{kou3} \\
31 & 囗 & & compartimento, recinto & \dictpinyin{wei2} \\
32 & 土 & & terra & \dictpinyin{tu3} \\
33 & 士 & & acadêmico, bacharel & \dictpinyin{shi4} \\
34 & 夂 & & ir & \dictpinyin{zhi1} \\
35 & 夊 & & ir devagar & \dictpinyin{sui1} \\
36 & 夕 & & tarde, pôr do sol & \dictpinyin{xi1} \\
37 & 大 & & grande, muito & \dictpinyin{da4} \\
38 & 女 & & mulher, fêmea & \dictpinyin{nv3} \\
39 & 子 & & criança, semente & \dictpinyin{zi3} \\
40 & 宀 & & teto, telhado & \dictpinyin{mian2} \\
41 & 寸 & & polegar, polegada & \dictpinyin{cun4} \\
42 & 小 & ⺌、⺍ & pequeno, insignificante & \dictpinyin{xiao3} \\
43 & 尢 & 尣 & manco, coxo & \dictpinyin{you2} \\
44 & 尸 & & cadáver & \dictpinyin{shi1} \\
45 & 屮 & & brotar, germinar & \dictpinyin{che4} \\
46 & 山 & & montanha & \dictpinyin{shan1} \\
47 & 巛 & 川 & rio & \dictpinyin{chuan1} \\
48 & 工 & & trabalho & \dictpinyin{gong1} \\
49 & 己 & ⺒ & próprio, a si mesmo & \dictpinyin{ji3} \\
50 & 巾 & & turbante, cachecol & \dictpinyin{jin1} \\
51 & 干 & & oposto, seco & \dictpinyin{gan1} \\
52 & 幺 & 么 & baixo, minúsculo & \dictpinyin{yao1} \\
53 & 广 & & casa em um penhasco & \dictpinyin{guang3} \\
54 & 廴 & & passada longa & \dictpinyin{yin3} \\
55 & 廾 & & duas mãos, vinte, arco & \dictpinyin{gong3} \\
56 & 弋 & & tiro, flecha & \dictpinyin{yi4} \\
57 & 弓 & & arco & \dictpinyin{gong1} \\
58 & 彐 & 彑 & focinho de porco & \dictpinyin{ji4} \\
59 & 彡 & & cerda, barba & \dictpinyin{shan1} \\
60 & 彳 & & passo & \dictpinyin{chi4} \\
61 & 心 & 忄、⺗ & coração, mente & \dictpinyin{xin1} \\
62 & 戈 & & lança & \dictpinyin{ge1} \\
63 & 戶 & 户、戸 & porta, casa & \dictpinyin{hu4} \\
64 & 手 & 扌、龵 & mão & \dictpinyin{shou3} \\
65 & 支 & & ramo & \dictpinyin{zhi1} \\
66 & 攴 & 攵 & tocar, bater levemente & \dictpinyin{pu1} \\
67 & 文 & & escrita, literatura & \dictpinyin{wen2} \\
68 & 斗 & & objeto em forma de concha & \dictpinyin{dou3} \\
69 & 斤 & & machado & \dictpinyin{jin1} \\
70 & 方 & & quadrado & \dictpinyin{fang1} \\
71 & 无 & 旡 & não, nada, negativo & \dictpinyin{wu2} \\
72 & 日 & & sol, dia & \dictpinyin{ri4} \\
73 & 曰 & & dizer, falar & \dictpinyin{yue1} \\
74 & 月 & & lua, mês & \dictpinyin{yue4} \\
75 & 木 & & árvore & \dictpinyin{mu4} \\
76 & 欠 & & falta, não ter, hiato & \dictpinyin{qian4} \\
77 & 止 & & parar & \dictpinyin{zhi3} \\
78 & 歹 & 歺 & morte, decadência & \dictpinyin{dai3} \\
79 & 殳 & & arma, lança & \dictpinyin{shu1} \\
80 & 毋 & 母 & mãe, não faça & \dictpinyin{mu3} \\
81 & 比 & & comparar, competir & \dictpinyin{bi3} \\
82 & 毛 & & pelagem & \dictpinyin{mao2} \\
83 & 氏 & & clã, linhagem & \dictpinyin{shi4} \\
84 & 气 & & ar, vapor, respiração & \dictpinyin{qi4} \\
85 & 水 & 氵、氺 & água & \dictpinyin{shui3} \\
86 & 火 & 灬 & fogo & \dictpinyin{huo3} \\
87 & 爪 & 爫 & garra, unha & \dictpinyin{zhao3} \\
88 & 父 & & pai, luz & \dictpinyin{fu4} \\
89 & 爻 & & duplo x, trigramas & \dictpinyin{yao2} \\
90 & 爿 & 丬 & metade de um tronco, madeira rachada & \dictpinyin{pan2} \\
91 & 片 & & fatia, filme & \dictpinyin{pian4} \\
92 & 牙 & & dente, presa & \dictpinyin{ya2} \\
93 & 牛 & 牜、⺧ & boi, vaca & \dictpinyin{niu2} \\
94 & 犬 & 犭 & cão & \dictpinyin{quan3} \\
95 & 玄 & & escuro, profundo & \dictpinyin{xuan2} \\
96 & 玉 & 王、玊 & jade & \dictpinyin{yu4} \\
97 & 瓜 & & melão & \dictpinyin{gua1} \\
98 & 瓦 & & telha & \dictpinyin{wa3} \\
99 & 甘 & & doce & \dictpinyin{gan1} \\
100 & 生 & & vida & \dictpinyin{sheng1} \\
101 & 用 & & usar & \dictpinyin{yong4} \\
102 & 田 & & campo, arrozal & \dictpinyin{tian2} \\
103 & 疋 & ⺪& pedaço de pano & \dictpinyin{pi3} \\
104 & 疒 & & doença & \dictpinyin{ne4} \\
105 & 癶 & & pegadas, pernas & \dictpinyin{bo1} \\
106 & 白 & & branco & \dictpinyin{bai2} \\
107 & 皮 & & pele, couro & \dictpinyin{pi2} \\
108 & 皿 & & prato & \dictpinyin{min3} \\
109 & 目 & ⺫ & olho & \dictpinyin{mu4} \\
110 & 矛 & & lança & \dictpinyin{mao2} \\
111 & 矢 & & seta, flecha & \dictpinyin{shi3} \\
112 & 石 & & pedra & \dictpinyin{shi2} \\
113 & 示 & 礻& espírito, ancestral, veneração & \dictpinyin{shi4} \\
114 & 禸 & & trilha & \dictpinyin{rou2} \\
115 & 禾 & & grão & \dictpinyin{he2} \\
116 & 穴 & & caverna & \dictpinyin{xue2} \\
117 & 立 & & ficar em pé, ereto & \dictpinyin{li4} \\
118 & 竹 & ⺮ & bambu & \dictpinyin{zhu2} \\
119 & 米 & & arroz & \dictpinyin{mi3} \\
120 & 糸 & 纟、糹 & seda & \dictpinyin{mi4} \\
121 & 缶 & & pote, jarra & \dictpinyin{fou3} \\
122 & 网 & ⺲、罓、⺳ & rede & \dictpinyin{wang3} \\
123 & 羊 & ⺶、⺷ & ovelha, cabra & \dictpinyin{yang2} \\
124 & 羽 & & pena & \dictpinyin{yu3} \\
125 & 老 & 耂 & velho & \dictpinyin{lao3} \\
126 & 而 & & e, mas & \dictpinyin{er2} \\
127 & 耒 & & arado & \dictpinyin{lei3} \\
128 & 耳 & & orelha & \dictpinyin{er3} \\
129 & 聿 & ⺺、⺻ & escova & \dictpinyin{yu4} \\
130 & 肉 & 月、⺼ & carne & \dictpinyin{rou4} \\
131 & 臣 & & ministro, oficial & \dictpinyin{chen2} \\
132 & 自 & & próprio, auto-- & \dictpinyin{zi4} \\
133 & 至 & & chegar & \dictpinyin{zhi4} \\
134 & 臼 & & argamassa, ligação & \dictpinyin{jiu4} \\
135 & 舌 & & língua & \dictpinyin{she2} \\
136 & 舛 & & opor & \dictpinyin{chuan3} \\
137 & 舟 & & barco & \dictpinyin{zhou1} \\
138 & 艮 & & parada, quietude & \dictpinyin{gen3} \\
139 & 色 & & cor, forma & \dictpinyin{se4} \\
140 & 艸 & ⺿ & grama & \dictpinyin{cao3} \\
141 & 虍 & & tigre & \dictpinyin{hu1} \\
142 & 虫 & & inseto, verme & \dictpinyin{chong2} \\
143 & 血 & & sangue & \dictpinyin{xue4} \\
144 & 行 & & andar, ir, fazer & \dictpinyin{xing2} \\
145 & 衣 & ⻂& roupa & \dictpinyin{yi1} \\
146 & 襾 & 西、覀 & capa, oeste & \dictpinyin{ya4} \\
147 & 見 & 见 & ver & \dictpinyin{jian4} \\
148 & 角 & ⻆、⻇ & chifre & \dictpinyin{jiao3} \\
149 & 言 & 讠、訁 & palavra, linguagem & \dictpinyin{yan2} \\
150 & 谷 & & vale & \dictpinyin{gu3} \\
151 & 豆 & & feijão, fava & \dictpinyin{dou4} \\
152 & 豕 & & porco & \dictpinyin{shi3} \\
153 & 豸 & & texugo, inseto sem pernas & \dictpinyin{zhi4} \\
154 & 貝 & 贝 & concha & \dictpinyin{bei4} \\
155 & 赤 & & vermelho, nu & \dictpinyin{chi4} \\
156 & 走 & & correr & \dictpinyin{zou3} \\
157 & 足 & ⻊& pé & \dictpinyin{zu2} \\
158 & 身 & & corpo & \dictpinyin{shen1} \\
159 & 車 & 车 & carroça, carro & \dictpinyin{che1} \\
160 & 辛 & & amargo & \dictpinyin{xin1} \\
161 & 辰 & & manhã & \dictpinyin{chen2} \\
162 & 辵 & ⻌、⻍、⻎ & caminhar & \dictpinyin{chuo4} \\
163 & 邑 & ⻏ & cidade & \dictpinyin{yi4} \\
164 & 酉 & & vinho, álcool & \dictpinyin{you3} \\
165 & 釆 & & distinto & \dictpinyin{bian4} \\
166 & 里 & & aldeia, vila & \dictpinyin{li3} \\
167 & 金 & 钅、釒 & ouro, metal & \dictpinyin{jin1} \\
168 & 長 & 长、镸 & longo, crescer & \dictpinyin{zhang3} \\
169 & 門 & 门 & portão, porta & \dictpinyin{men2} \\
170 & 阜 & ⻖ & monte, barragem & \dictpinyin{fu4} \\
171 & 隶 & & escravo & \dictpinyin{li4} \\
172 & 隹 & & pássaro de cauda curta & \dictpinyin{zhui1} \\
173 & 雨 & & chuva & \dictpinyin{yu3} \\
174 & 靑 & 青 & azul, verde ou preto & \dictpinyin{qing1} \\
175 & 非 & & errado & \dictpinyin{fei1} \\
176 & 面 & 靣 & face & \dictpinyin{mian4} \\
177 & 革 & & couro, couro cru & \dictpinyin{ge2} \\
178 & 韋 & 韦 & couro tingido & \dictpinyin{wei2} \\
179 & 韭 & & alho-poró & \dictpinyin{jiu3} \\
180 & 音 & & som & \dictpinyin{yin1} \\
181 & 頁 & 页 & folha, página & \dictpinyin{ye4} \\
182 & 風 & 风 & vento & \dictpinyin{feng1} \\
183 & 飛 & 飞 & voar & \dictpinyin{fei1} \\
184 & 食 & 饣、飠 & alimento, comer & \dictpinyin{shi2} \\
185 & 首 & & cabeça & \dictpinyin{shou3} \\
186 & 香 & & perfume, aroma & \dictpinyin{xiang1} \\
187 & 馬 & 马 & cavalo & \dictpinyin{ma3} \\
188 & 骨 & ⻣ & osso & \dictpinyin{gu3} \\
189 & 高 & 髙 & alto & \dictpinyin{gao1} \\
190 & 髟 & & cabelo & \dictpinyin{biao1} \\
191 & 鬥 & & luta & \dictpinyin{dou4} \\
192 & 鬯 & & vinho sacrificial & \dictpinyin{chang4} \\
193 & 鬲 & & caldeirão, tripé & \dictpinyin{ge2} \\
194 & 鬼 & & fantasma, demônio & \dictpinyin{gui3} \\
195 & 魚 & 鱼 & peixe & \dictpinyin{yu2} \\
196 & 鳥 & 鸟 & pássaro & \dictpinyin{niao3} \\
197 & 鹵 & 卤 & sal & \dictpinyin{lu3} \\
198 & 鹿 & & corça, veado & \dictpinyin{lu4} \\
199 & 麥 & 麦 & trigo & \dictpinyin{mai4} \\
200 & 麻 & & cânhamo, linho & \dictpinyin{ma2} \\
201 & 黃 & 黄 & amarelo & \dictpinyin{huang4} \\
202 & 黍 & & milhete, painço & \dictpinyin{shu3} \\
203 & 黑 & & preto & \dictpinyin{hei1} \\
204 & 黹 & & bordado & \dictpinyin{zhi3} \\
205 & 黽 & 黾 & sapo, anfíbio & \dictpinyin{mian3} \\
206 & 鼎 & & tripé de sacrifício, caldeirão de três pernas & \dictpinyin{ding3} \\
207 & 鼓 & & tambor & \dictpinyin{gu3} \\
208 & 鼠 & 鼡 & rato, camundongo & \dictpinyin{shu3} \\
209 & 鼻 & & nariz & \dictpinyin{bi2} \\
210 & 齊 & 齐、斉 & mesmo, uniformemente & \dictpinyin{qi2} \\
211 & 齒 & 齿 & dente & \dictpinyin{chi3} \\
212 & 龍 & 龙 & dragão & \dictpinyin{long2} \\
213 & 龜 & 龟 & tartaruga & \dictpinyin{gui1} \\
214 & 龠 & & flauta & \dictpinyin{yue4} \\
\end{longtblr}

%%%%% EOF %%%%%


\fi

\clearpage
\chapter{Lista de Hanzis (somente o primeiro caracter)}                                                           
\begin{multicols}{5}
 \begin{KeepFromToc}
  \listoffirsthanzis
 \end{KeepFromToc}
\end{multicols}

\end{document}

%%%%% EOF %%%%
