%
% Funções que uso no dicionario.tex
%

\newcommand{\sectionbreak}{\phantomsection}

\DeclareRobustCommand{\&}{%
  \ifdim\fontdimen1\font>0pt
    \textsl{\symbol{`\&}}%
  \else
    \symbol{`\&}%
  \fi
}

\NewDocumentCommand{\mypinyin}{m}{«\pinyin{#1}»}

\NewDocumentCommand{\dictpy}{m}{%
  \StrSubstitute{#1}{r5}{r}[\rA]%
  \StrSubstitute{\rA}{t5}{t}[\rZ]%
  \edef\py{\mypinyin{\rZ}}%
  \py
}

\NewDocumentCommand{\myenumerate}{>{\SplitList{;}}m}{
  \begin{enumerate*}[left=0pt,mode=unboxed,itemjoin={{; }}]
  \ProcessList{#1}{\insertitem}
  \end{enumerate*}
}
\NewDocumentCommand{\insertitem}{m}{\item #1}

\ExplSyntaxOn

\NewDocumentEnvironment{verbete}{momom}{%
  \leavevmode%
  \label{\mdfivesum{#1::#3}}
  \markboth{#1{\small\dictpy{#3}}}{#1{\small\dictpy{#3}}}
  \index[pinyin]{#1@\dictpy{#3} #1}
  \index[stroke]{#1@#1 \dictpy{#3}}
  \index[radical]{#1@#1 \dictpy{#3}}
  \tl_set:Nn \l_hanzy_tl {#1}
  \tl_set:Nn \l_pinyin_tl {#3}
  \tl_set:Nn \l_strokes_tl {#5}
  \begin{minipage}[t][][t]{.475\textwidth}
  \begin{flushleft}
  {\tiny\ding{108}}\hspace{1ex}{\Large\textbf{#1}}\hfill\textsuperscript{\tiny(#5画)}\\
  \IfValueTF{#2}{
    \IfValueTF{#4}{\small#2\dictpy{#3}#4}{\small#2\dictpy{#3}}%
  }{%
    \IfValueTF{#4}{\small\dictpy{#3}#4}{\small\dictpy{#3}}%
  }
  \newline%
}{%
  \end{flushleft}
  \end{minipage}\par
}

\NewDocumentEnvironment{extragrande}{momom}{%
  \leavevmode%
  \label{\mdfivesum{#1::#3}}
  \markboth{#1{\small\dictpy{#3}}}{#1{\small\dictpy{#3}}}
  \index[pinyin]{#1@\dictpy{#3} #1}
  \index[stroke]{#1@#1 \dictpy{#3}}
  \index[radical]{#1@#1 \dictpy{#3}}
  \tl_set:Nn \l_hanzy_tl {#1}
  \tl_set:Nn \l_pinyin_tl {#3}
  \tl_set:Nn \l_strokes_tl {#5}
  \begin{minipage}[t][][t]{.475\textwidth}
  \begin{flushleft}
  {\tiny\ding{108}}\hspace{1ex}{\Large\textbf{#1}}\\%
  {\hspace*{\fill}\textsuperscript{\tiny(#5画)}}\\%
  \IfValueTF{#2}{%
    \IfValueTF{#4}{\small#2\dictpy{#3}#4}{\small#2\dictpy{#3}}
  }{%
    \IfValueTF{#4}{\small\dictpy{#3}#4}{\small\dictpy{#3}}
  }
  \newline
}{%
  \end{flushleft}
  \end{minipage}\par
}

\NewDocumentCommand{\significado}{somm}{%
  \IfBooleanTF{#1}{% Substantivo Próprio
    \index[pinyin]{@\dictpy{\l_pinyin_tl} \textcolor{BrickRed}{\l_hanzy_tl}}
    \index[stroke]{\l_hanzy_tl@\textcolor{BrickRed}{\l_hanzy_tl} \dictpy{\l_pinyin_tl}}
    \index[radical]{\l_hanzy_tl@\textcolor{BrickRed}{\l_hanzy_tl} \dictpy{\l_pinyin_tl}}
    $\hookrightarrow$\ (\textit{#3})\ \textsl{(Substantivo\ Próprio)}\\%
    \IfValueT{#2}{[p.c.: #2]\\}%
    {\myenumerate{#4}}%
  }{%
    $\hookrightarrow$\ (\textit{#3})%
    \IfValueT{#2}{[p.c.: #2]\\}%
    {\myenumerate{#4}}%
  }
  \newline
}

\NewDocumentCommand{\veja}{smm}{%
  \IfBooleanTF{#1}{%
    \textit{Veja:} #2\ {\footnotesize(pág.~\pageref{\mdfivesum{#2::#3}})}\\%
    \hspace{6ex}{\small\dictpy{#3}}%
  }{%
    \textit{Veja:} #2 {\small\dictpy{#3}}\ {\footnotesize(pág.~\pageref{\mdfivesum{#2::#3}})}
  }%
  \newline
}

\NewDocumentCommand{\variante}{smm}{%
  \IfBooleanTF{#1}{%
    $\hookrightarrow$\ variante\ de #2 {\footnotesize(pág.~\pageref{\mdfivesum{#2::#3}})}\\
    \hspace{1.5em}{\small\dictpy{#3}}
  }{%
    $\hookrightarrow$\ variante\ de #2 {\small\dictpy{#3}}\ {\footnotesize(pág.~\pageref{\mdfivesum{#2::#3}})}
  }%
  \newline
}

\NewDocumentCommand{\exemplo}{mmm}{%
  \textit{Ex.:} #1\CJKunderdot{\l_hanzy_tl}#2\\
  \hspace{5ex} (#3)
  \newline
}

\ExplSyntaxOff
