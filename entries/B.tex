%%%
%%% B
%%%
\section*{B}
\addcontentsline{toc}{section}{B}
\begin{multicols}{2}

\begin{hanzi}[吧]{ba0}
\entry{ba0}{part.}{sugestão, requisição ou comando leve|consentimento ou aprovação|confirmação de uma suposição|alguma dúvida|pausa entre suposições em alternativas}
\end{hanzi}

\begin{hanzi}[八]{ba1}
\entry{ba1}{num.}{oito|8}
\end{hanzi}

\begin{hanzi}[巴西]{Ba1xi1}
\entry{Ba1xi1}{n.}{Brasil}
\end{hanzi}

\begin{hanzi}[爸爸]{ba4ba0}
\entry{ba4ba0}{n.}{papai; pai|\pc{个/位}}
\end{hanzi}

\begin{hanzi}[白]{bai2}
\entry{bai2}{adj.}{branco|claro|puro; simples; em branco}
\entry{bai2}{adv.}{em vão; sem propósito; por nada|grátis}
\entry{bai2}{n.}{parte falada na ópera; diálogo|dialeto}
\end{hanzi}

\begin{hanzi}[白菜]{bai2cai4}
\entry{bai2cai4}{n.}{repolho chinês}
\end{hanzi}

\begin{hanzi}[白色]{bai2se4}
\entry{bai2se4}{n.}{cor branca}
\end{hanzi}

\begin{hanzi}[白天]{bai2tian1}
\entry{bai2tian1}{p.t.}{dia; de dia}
\end{hanzi}

\begin{hanzi}[百]{bai3}
\entry{bai3}{num.}{centena; cem; cento}
\end{hanzi}

\begin{hanzi}[半]{ban4}
\entry{ban4}{num.}{meio; meia}
\end{hanzi}

\begin{hanzi}[办]{ban4}
\entry{ban4}{v.}{tratar; fazer}
\end{hanzi}

\begin{hanzi}[办公室]{ban4gong1shi4}
\entry{ban4gong1shi4}{n.}{gabinete; escritório}
\end{hanzi}

\begin{hanzi}[帮助]{bang1zhu4}
\entry{bang1zhu4}{n.}{ajuda}
\entry{bang1zhu4}{v.}{ajudar}
\end{hanzi}

\begin{hanzi}[包]{bao1}
\entry{bao1}{p.c.}{pacote; saco; sacola}
\end{hanzi}

\begin{hanzi}[包子]{bao1zi0}
\entry{bao1zi0}{n.}{pão recheado cozido no vapor}
\end{hanzi}

\begin{hanzi}[报酬]{bao4chou2}
\entry{bao4chou2}{n.}{recompensa; remuneração}
\end{hanzi}

\begin{hanzi}[杯]{bei1}
\entry{bei1}{p.c.}{palavra classificadora copo}
\end{hanzi}

\begin{hanzi}[杯子]{bei1zi0}
\entry{bei1zi0}{n.}{copo; caneca; xícara; taça}
\end{hanzi}

\begin{hanzi}[北边]{bei3bian0}
\entry{bei3bian0}{p.l.}{norte}
\end{hanzi}

\begin{hanzi}[北京]{Bei3jing1}
\entry{Bei3jing1}{n.}{Pequim; Capital da China}
\end{hanzi}

\begin{hanzi}[北京]{Bei3jing1}
\entry{Bei3jing1}{n.}{Beijing(Pequim)}
\end{hanzi}

\begin{hanzi}[北面]{bei3mian0}
\entry{bei3mian0}{p.l.}{norte}
\end{hanzi}

\begin{hanzi}[背]{bei4}
\entry{bei4}{n.}{costas}
\end{hanzi}

\begin{hanzi}[被子]{bei4zi0}
\entry{bei4zi0}{n.}{colcha}
\end{hanzi}

\begin{hanzi}[本]{ben3}
\entry{ben3}{p.c.}{para livros, dicionários, etc}
\end{hanzi}

\begin{hanzi}[本子]{ben3zi0}
\entry{ben3zi0}{n.}{caderno}
\end{hanzi}

\begin{hanzi}[鼻子]{bi2zi0}
\entry{bi2zi0}{n.}{nariz}
\end{hanzi}

\begin{hanzi}[笔]{bi3}
\entry{bi3}{n.}{caneta; lápis|\pc{支}}
\end{hanzi}

\begin{hanzi}[比]{bi3}
\entry{bi3}{prep.}{que; do que}
\end{hanzi}

\begin{hanzi}[比较]{bi3jiao4}
\entry{bi3jiao4}{adv.}{comparativamente; relativamente}
\end{hanzi}

\begin{hanzi}[比萨饼]{bi3sa4bing3}
\entry{bi3sa4bing3}{n.}{pizza}
\end{hanzi}

\begin{hanzi}[比赛]{bi3sai4}
\entry{bi3sai4}{n.}{competição; concurso}
\end{hanzi}

\begin{hanzi}[别]{bie2}
\entry{bie2}{adv.}{nada de (pedir a alguém para não fazer); não}
\end{hanzi}

\begin{hanzi}[别的]{bie2de0}
\entry{bie2de0}{pron.}{outro; outra}
\end{hanzi}

\begin{hanzi}[别人]{bie2ren0}
\entry{bie2ren0}{pron.}{outrem; outra pessoa; outras pessoas}
\end{hanzi}

\begin{hanzi}[冰]{bing1}
\entry{bing1}{n.}{gelo}
\end{hanzi}

\begin{hanzi}[冰球]{bing1qiu2}
\entry{bing1qiu2}{n.}{hóquei no gelo}
\end{hanzi}

\begin{hanzi}[病]{bing4}
\entry{bing4}{n.}{doença}
\entry{bing4}{v.}{adoecer; estar doente}
\end{hanzi}

\begin{hanzi}[博物馆]{bo2wu4guan3}
\entry{bo2wu4guan3}{n.}{museu}
\end{hanzi}

\begin{hanzi}[脖子]{bo2zi0}
\entry{bo2zi0}{n.}{pescoço}
\end{hanzi}

\begin{hanzi}[不]{bu0}
\entry{bu0}{adv.}{não (em expressões\\``verbo$+$不$+$verbo'')}
\entry{bu2}{adv.}{não (antes de quarto tom)}
\entry{bu4}{adv.}{não}
\end{hanzi}

\begin{hanzi}[不]{bu2}
\entry{bu2}{adv.}{não (antes de quarto tom)}
\entry{bu0}{adv.}{não (em expressões\\``verbo$+$不$+$verbo'')}
\entry{bu4}{adv.}{não}
\end{hanzi}

\begin{hanzi}[不错]{bu2cuo4}
\entry{bu2cuo4}{adj.}{não (é) mau; bastante bom; bom; boa}
\end{hanzi}

\begin{hanzi}[不过]{bu2guo4}
\entry{bu2guo4}{conj.}{mas, contudo}
\end{hanzi}

\begin{hanzi}[不客气]{bu2ke4qi0}
\entry{bu2ke4qi0}{}{de nada; não há de que}
\end{hanzi}

\begin{hanzi}[不要]{bu2yao4}
\entry{bu2yao4}{v.o.}{nada de (pedir a alguém não fazer); não}
\end{hanzi}

\begin{hanzi}[不用]{bu2yong4}
\entry{bu2yong4}{v.o.}{não precisar}
\end{hanzi}

\begin{hanzi}[不]{bu4}
\entry{bu4}{adv.}{não}
\entry{bu0}{adv.}{não (em expressões\\``verbo$+$不$+$verbo'')}
\entry{bu2}{adv.}{não (antes de quarto tom)}
\end{hanzi}

\begin{hanzi}[不同]{bu4tong2}
\entry{bu4tong2}{adj.}{diferente}
\end{hanzi}


\end{multicols}
