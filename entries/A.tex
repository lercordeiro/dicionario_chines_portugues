%%%
%%% A
%%%
\section*{A}
\addcontentsline{toc}{section}{A}
\begin{multicols}{2}

\begin{hanzi}[啊]{a0}
\entry{a0}{part.}{ah!;oh!|no final da sentença para expressar entusiasmo|%
no final da sentença para expressar impaciência ou o que é óbvio|%
no final de uma ordem, aviso, etc|%
no final da sentença para expressar questionamento|%
para indicar uma pausa deliberada|%
para enumerar itens}
\end{hanzi}

\begin{hanzi}[矮]{ai3}
\entry{ai3}{adj.}{baixo em estatura, dimensão, grau ou ranque}
\end{hanzi}

\begin{hanzi}[爱]{ai4}
\entry{ai4}{n.}{amor; afeição}
\entry{ai4}{v.}{amar; ter afeição; gostar; gostar de|%
inclinado a (fazer alguma coisa); tender (a acontecer)}
\end{hanzi}

\begin{hanzi}[爱好]{ai4hao4}
\entry{ai4hao4}{n.}{passatempo; interesse|\pc{个}}
\entry{ai4hao4}{v.}{ter algo como hobby; ter prazer em fazer algo; gostar de (fazer alguma coisa)}
\end{hanzi}

\begin{hanzi}[爱人]{ai4ren0}
\entry{ai4ren0}{n.}{marido ou esposa|amado ou amada|querido ou querida|\pc{个}}
\end{hanzi}

\end{multicols}
