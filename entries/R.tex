%%%
%%% R
%%%
\section*{R}
\addcontentsline{toc}{section}{R}
\begin{multicols}{2}

\begin{hanzi}[然后]{ran2hou4}
\entry{ran2hou4}{conj.}{depois; logo; portanto}
\end{hanzi}

\begin{hanzi}[让]{rang4}
\entry{rang4}{v.}{deixar; permitir}
\end{hanzi}

\begin{hanzi}[热]{re4}
\entry{re4}{adj.}{quente}
\end{hanzi}

\begin{hanzi}[热闹]{re4nao0}
\entry{re4nao0}{adj.}{animado}
\end{hanzi}

\begin{hanzi}[人]{ren2}
\entry{ren2}{n.}{pessoa}
\end{hanzi}

\begin{hanzi}[人口]{ren2kou3}
\entry{ren2kou3}{n.}{população}
\end{hanzi}

\begin{hanzi}[人民币]{Ren2min2bi4}
\entry{Ren2min2bi4}{n.}{RMB; nome da moeda chinesa}
\end{hanzi}

\begin{hanzi}[认识]{ren4shi0}
\entry{ren4shi0}{v.}{conhecer}
\end{hanzi}

\begin{hanzi}[日]{ri4}
\entry{ri4}{p.c.}{dia (mais usado em escrita)}
\end{hanzi}

\begin{hanzi}[日本]{Ri4ben3}
\entry{Ri4ben3}{n.}{Japão}
\end{hanzi}

\begin{hanzi}[如果]{ru2guo3}
\entry{ru2guo3}{conj.}{se; caso; no caso de}
\end{hanzi}

\begin{hanzi}[乳房]{ru3fang2}
\entry{ru3fang2}{n.}{seio; mama}
\end{hanzi}

\begin{hanzi}[肉]{rou4}
\entry{rou4}{n.}{carne}
\end{hanzi}

\end{multicols}
