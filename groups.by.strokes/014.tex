%%%
%%% 14画
%%%
\section*{14画}\addcontentsline{toc}{section}{14画}\addcontentsline{loh}{figure}{\#\#\#\# 14画}

%%%%%%%%%% 㮸 %%%%%%%%%%
\subsection*{㮸}\addcontentsline{loh}{figure}{㮸}

\begin{Entry}{㮸}{14}{⽊}
  \begin{Phonetics}{㮸}{song4}
    \variantof{送}
  \end{Phonetics}
\end{Entry}

%%%%%%%%%% 僧 %%%%%%%%%%
\subsection*{僧}\addcontentsline{loh}{figure}{僧}

\begin{Entry}{僧}{14}{⼈}
  \begin{Phonetics}{僧}{seng1}
    \definition*{s.}{Sobrenome: Seng}
    \definition[位,名,个]{s.}{monge Budista, abreviação de 僧伽}
  \seealsoref{僧伽}{seng1qie2}
  \end{Phonetics}
\end{Entry}

\begin{Entry}{僧人}{14,2}{⼈,⼈}
  \begin{Phonetics}{僧人}{seng1ren2}[][HSK 7-9]
    \definition{s.}{monge budista | monge}
  \end{Phonetics}
\end{Entry}

\begin{Entry}{僧伽}{14,7}{⼈,⼈}
  \begin{Phonetics}{僧伽}{seng1qie2}
    \definition{s.}{sangha ou sanga (Budismo) | a comunidade monástica | monge}
  \end{Phonetics}
\end{Entry}

%%%%%%%%%% 僮 %%%%%%%%%%
\subsection*{僮}\addcontentsline{loh}{figure}{僮}

\begin{Entry}{僮}{14}{⼈}
  \begin{Phonetics}{僮}{tong2}
    \definition*{s.}{Sobrenome: Tong}
  \end{Phonetics}
  \begin{Phonetics}{僮}{zhuang4}
    \variantof{壮}
  \end{Phonetics}
\end{Entry}

%%%%%%%%%% 兢 %%%%%%%%%%
\subsection*{兢}\addcontentsline{loh}{figure}{兢}

\begin{Entry}{兢}{14}{⼉}
  \begin{Phonetics}{兢}{jing1}
    \definition{adj.}{medroso | cauteloso | forte}
    \definition{v.}{mover}
  \end{Phonetics}
\end{Entry}

\begin{Entry}{兢兢业业}{14,14,5,5}{⼉,⼉,⼀,⼀}
  \begin{Phonetics}{兢兢业业}{jing1jing1ye4ye4}[][HSK 7-9]
    \definition{expr.}{cauteloso e consciencioso; zeloso; aplicado; descreve alguém que é muito cuidadoso, cauteloso e responsável ao fazer as coisas}
  \end{Phonetics}
\end{Entry}

%%%%%%%%%% 凳 %%%%%%%%%%
\subsection*{凳}\addcontentsline{loh}{figure}{凳}

\begin{Entry}{凳}{14}{⼏}
  \begin{Phonetics}{凳}{deng4}
    \definition[条]{s.}{banco; banqueta}
  \end{Phonetics}
\end{Entry}

\begin{Entry}{凳子}{14,3}{⼏,⼦}
  \begin{Phonetics}{凳子}{deng4zi5}[][HSK 7-9]
    \definition[把,条,个]{s.}{banco; banqueta; um móvel que tem pernas para sentar, mas não tem encosto}
  \end{Phonetics}
\end{Entry}

%%%%%%%%%% 嘉 %%%%%%%%%%
\subsection*{嘉}\addcontentsline{loh}{figure}{嘉}

\begin{Entry}{嘉}{14}{⼝}
  \begin{Phonetics}{嘉}{jia1}
    \definition*{s.}{Sobrenome: Jia}
    \definition{adj.}{bom; ótimo | auspicioso | excelente}
    \definition{v.}{elogiar; recomendar}
    \definition{v.}{elogiar}
  \end{Phonetics}
\end{Entry}

\begin{Entry}{嘉年华}{14,6,6}{⼝,⼲,⼗}
  \begin{Phonetics}{嘉年华}{jia1nian2hua2}[][HSK 7-9]
    \definition{s.}{Empréstimo linguístico: carnaval}
  \end{Phonetics}
\end{Entry}

\begin{Entry}{嘉宾}{14,10}{⼝,⼧}
  \begin{Phonetics}{嘉宾}{jia1bin1}[][HSK 6]
    \definition[个,位,名,些]{s.}{convidado}
  \end{Phonetics}
\end{Entry}

%%%%%%%%%% 嘛 %%%%%%%%%%
\subsection*{嘛}\addcontentsline{loh}{figure}{嘛}

\begin{Entry}{嘛}{14}{⼝}
  \begin{Phonetics}{嘛}{ma5}[][HSK 6]
    \definition{part.}{usado no final de uma declaração para expressar que é claro que é verdade que é óbvio | usado no final de uma frase imperativa para expressar expectativa ou dissuasão | usado em uma frase para indicar uma pausa e chamar a atenção da outra pessoa}
  \end{Phonetics}
\end{Entry}

%%%%%%%%%% 境 %%%%%%%%%%
\subsection*{境}\addcontentsline{loh}{figure}{境}

\begin{Entry}{境}{14}{⼟}
  \begin{Phonetics}{境}{jing4}
    \definition{s.}{fronteira; limite | lugar; área; território; região | condição; situação; circunstâncias}
  \end{Phonetics}
\end{Entry}

\begin{Entry}{境内}{14,4}{⼟,⼌}
  \begin{Phonetics}{境内}{jing4nei4}[][HSK 7-9]
    \definition{s.}{área dentro das fronteiras | doméstico | interno (para um país, província, cidade etc.) | dentro das fronteiras}
  \antonymref{境外}{jing4wai4}
  \end{Phonetics}
\end{Entry}

\begin{Entry}{境外}{14,5}{⼟,⼣}
  \begin{Phonetics}{境外}{jing4wai4}[][HSK 7-9]
    \definition{s.}{área fora das fronteiras (ou do território) de um país; além das fronteiras de um país ou região}
  \end{Phonetics}
\end{Entry}

\begin{Entry}{境地}{14,6}{⼟,⼟}
  \begin{Phonetics}{境地}{jing4di4}[][HSK 7-9]
    \definition{s.}{situação; circunstâncias; as circunstâncias ou a situação encontradas (geralmente usadas em um sentido negativo) | reino; estado}
  \end{Phonetics}
\end{Entry}

\begin{Entry}{境界}{14,9}{⼟,⽥}
  \begin{Phonetics}{境界}{jing4jie4}[][HSK 7-9]
    \definition{s.}{limite; limites de terra | estado; nível; extensão alcançada; o grau em que algo é alcançado ou o estado em que se manifesta}
  \end{Phonetics}
\end{Entry}

\begin{Entry}{境遇}{14,12}{⼟,⾡}
  \begin{Phonetics}{境遇}{jing4yu4}[][HSK 7-9]
    \definition{s.}{a sorte de alguém; as circunstâncias; circunstâncias e encontros}
  \end{Phonetics}
\end{Entry}

%%%%%%%%%% 墙 %%%%%%%%%%
\subsection*{墙}\addcontentsline{loh}{figure}{墙}

\begin{Entry}{墙}{14}{⼟}
  \begin{Phonetics}{墙}{qiang2}[][HSK 2]
    \definition[面,堵,道]{s.}{parede; barreira ou perímetro construído com tijolos, pedras, etc. | qualquer coisa com a forma ou função de uma parede; a parte de um objeto que funciona como parede ou divisória}
    \definition{v.}{(gíria) bloquear (um website) (usado geralmente na voz passiva: 被墙)}
  \end{Phonetics}
\end{Entry}

\begin{Entry}{墙纸}{14,7}{⼟,⽷}
  \begin{Phonetics}{墙纸}{qiang2zhi3}
    \definition{s.}{papel de parede}
  \end{Phonetics}
\end{Entry}

\begin{Entry}{墙壁}{14,16}{⼟,⼟}
  \begin{Phonetics}{墙壁}{qiang2bi4}[][HSK 5]
    \definition[面,堵,道]{s.}{parede; barreira ou perímetro construído com tijolos, pedras ou terra}
  \end{Phonetics}
\end{Entry}

%%%%%%%%%% 墬 %%%%%%%%%%
\subsection*{墬}\addcontentsline{loh}{figure}{墬}

\begin{Entry}{墬}{14}{⼟}
  \begin{Phonetics}{墬}{di4}
    \variantof{地}
  \end{Phonetics}
\end{Entry}

%%%%%%%%%% 嫚 %%%%%%%%%%
\subsection*{嫚}\addcontentsline{loh}{figure}{嫚}

\begin{Entry}{嫚}{14}{⼥}
  \begin{Phonetics}{嫚}{man1}
    \definition{s.}{menina bem-comportada}
  \seealsoref{嫚子}{man1zi5}
  \end{Phonetics}
  \begin{Phonetics}{嫚}{man4}
    \definition*{s.}{Sobrenome: Man}
    \definition{s.}{Dialeto: menina}
    \definition{v.}{Literário: desprezar; menosprezar; insultar; humilhar}
  \end{Phonetics}
\end{Entry}

\begin{Entry}{嫚子}{14,3}{⼥,⼦}
  \begin{Phonetics}{嫚子}{man1zi5}
    \definition{s.}{Dialeto: menina}
  \end{Phonetics}
\end{Entry}

\begin{Entry}{嫚骂}{14,9}{⼥,⾺}
  \begin{Phonetics}{嫚骂}{man4ma4}
    \definition{s.}{insultar; repreender; xingar}
  \end{Phonetics}
\end{Entry}

%%%%%%%%%% 嫦 %%%%%%%%%%
\subsection*{嫦}\addcontentsline{loh}{figure}{嫦}

\begin{Entry}{嫦}{14}{⼥}
  \begin{Phonetics}{嫦}{chang2}
    \definition{s.}{uma beleza lendária que voou para a lua | a dama da lua}
  \end{Phonetics}
\end{Entry}

\begin{Entry}{嫦娥}{14,10}{⼥,⼥}
  \begin{Phonetics}{嫦娥}{chang2'e2}[][HSK 7-9]
    \definition*{s.}{Chang'e, a dama da lua (mitologia chinesa); uma fada que voou do mundo humano para o Palácio da Lua na mitologia}
  \end{Phonetics}
\end{Entry}

%%%%%%%%%% 嫩 %%%%%%%%%%
\subsection*{嫩}\addcontentsline{loh}{figure}{嫩}

\begin{Entry}{嫩}{14}{⼥}
  \begin{Phonetics}{嫩}{nen4}[][HSK 7-9]
    \definition{adj.}{terno; delicado; recém-nascido e frágil | macio; malpassado; alguns alimentos são rápidos de cozinhar e fáceis de mastigar | claro; suave; algumas cores são claras | sem habilidade; inexperiente}
  \end{Phonetics}
\end{Entry}

%%%%%%%%%% 孵 %%%%%%%%%%
\subsection*{孵}\addcontentsline{loh}{figure}{孵}

\begin{Entry}{孵}{14}{⼦}
  \begin{Phonetics}{孵}{fu1}
    \definition{v.}{chocar; incubar; (pássaros) sentar em ovos}
  \end{Phonetics}
\end{Entry}

\begin{Entry}{孵化}{14,4}{⼦,⼔}
  \begin{Phonetics}{孵化}{fu1hua4}[][HSK 7-9]
    \definition{v.}{chocar; incubar | incubar; metaforicamente, cultivar e desenvolver coisas novas (agora se refere principalmente ao suporte a empresas de alta tecnologia recém-criadas)}
  \end{Phonetics}
\end{Entry}

%%%%%%%%%% 察 %%%%%%%%%%
\subsection*{察}\addcontentsline{loh}{figure}{察}

\begin{Entry}{察}{14}{⼧}
  \begin{Phonetics}{察}{cha2}
    \definition*{s.}{Sobrenome: Cha}
    \definition{v.}{examinar; investigar; escrutinar | observar; olhar atentamente; investigar}
  \end{Phonetics}
\end{Entry}

\begin{Entry}{察看}{14,9}{⼧,⽬}
  \begin{Phonetics}{察看}{cha2kan4}[][HSK 7-9]
    \definition{v.}{observar; olhar atentamente; inspecionar}
  \end{Phonetics}
\end{Entry}

\begin{Entry}{察觉}{14,9}{⼧,⾒}
  \begin{Phonetics}{察觉}{cha2jue2}[][HSK 7-9]
    \definition{v.}{detectar; perceber; estar ciente de; estar consciente de; descobrir; ver}
  \end{Phonetics}
\end{Entry}

%%%%%%%%%% 寡 %%%%%%%%%%
\subsection*{寡}\addcontentsline{loh}{figure}{寡}

\begin{Entry}{寡}{14}{⼧}
  \begin{Phonetics}{寡}{gua3}
    \definition{adj.}{poucos; escassos | insípido; sem sabor | pouco; escasso | insípido; sem graça}
    \definition{pron.}{eu; título autoproclamado de um antigo monarca}
    \definition{s.}{viúva | viuvez; a natureza ou estado de uma mulher viúva que vive sozinha}
  \antonymref{多}{duo1}
  \antonymref{众}{zhong4}
  \end{Phonetics}
\end{Entry}

\begin{Entry}{寡妇}{14,6}{⼧,⼥}
  \begin{Phonetics}{寡妇}{gua3fu5}[][HSK 7-9]
    \definition[个]{s.}{viúva; uma mulher cujo marido morreu}
  \end{Phonetics}
\end{Entry}

%%%%%%%%%% 寥 %%%%%%%%%%
\subsection*{寥}\addcontentsline{loh}{figure}{寥}

\begin{Entry}{寥}{14}{⼧}
  \begin{Phonetics}{寥}{liao2}
    \definition{adj.}{pouco; escasso | silencioso; deserto | abstruso; vago; amplo e vazio; desolado}
  \end{Phonetics}
\end{Entry}

\begin{Entry}{寥寥无几}{14,14,4,2}{⼧,⼧,⽆,⼏}
  \begin{Phonetics}{寥寥无几}{liao2liao2-wu2ji3}[][HSK 7-9]
    \definition{expr.}{poucos restantes; escasso; esparso; muito poucos; uma quantidade insignificante}
  \end{Phonetics}
\end{Entry}

%%%%%%%%%% 寨 %%%%%%%%%%
\subsection*{寨}\addcontentsline{loh}{figure}{寨}

\begin{Entry}{寨}{14}{⼧}
  \begin{Phonetics}{寨}{zhai4}
    \definition{s.}{fortaleza | paliçada | acampamento | vila (paliçada)}
  \end{Phonetics}
\end{Entry}

%%%%%%%%%% 弊 %%%%%%%%%%
\subsection*{弊}\addcontentsline{loh}{figure}{弊}

\begin{Entry}{弊}{14}{⼶}
  \begin{Phonetics}{弊}{bi4}
    \definition{s.}{fraude; abuso; negligência médica | desvantagem; falha; defeito; dano | trapaça; fraude, engano e falsificação}
  \antonymref{利}{li4}
  \end{Phonetics}
\end{Entry}

\begin{Entry}{弊病}{14,10}{⼶,⽧}
  \begin{Phonetics}{弊病}{bi4bing4}[][HSK 7-9]
    \definition{s.}{doença; mal; negligência | incinveniente; desvantagem; problemas com coisas}
  \end{Phonetics}
\end{Entry}

\begin{Entry}{弊端}{14,14}{⼶,⽴}
  \begin{Phonetics}{弊端}{bi4duan1}[][HSK 7-9]
    \definition{s.}{abuso; negligência; prática corrupta; danos ao interesse público devido a uma lacuna no trabalho}
  \end{Phonetics}
\end{Entry}

%%%%%%%%%% 愿 %%%%%%%%%%
\subsection*{愿}\addcontentsline{loh}{figure}{愿}

\begin{Entry}{愿}{14}{⽕}
  \begin{Phonetics}{愿}{yuan4}[][HSK 5]
    \definition{adj.}{honesto e prudente}
    \definition{s.}{esperança; desejo; vontade; a ideia de alcançar algum objetivo no futuro | voto (feito perante o Buda ou um deus); o desejo de retribuição feito ao rezar para os deuses e Buda}
    \definition{v.}{estar disposto; estar pronto; de bom grado, concordar porque está de acordo com seus desejos | ter esperança; desejar; qerer alcançar algum desejo}
  \end{Phonetics}
\end{Entry}

\begin{Entry}{愿望}{14,11}{⽕,⽉}
  \begin{Phonetics}{愿望}{yuan4wang4}[][HSK 3]
    \definition[个,种]{s.}{desejo; aspiração; a ideia de alcançar algum objetivo no futuro.}
  \end{Phonetics}
\end{Entry}

\begin{Entry}{愿意}{14,13}{⽕,⼼}
  \begin{Phonetics}{愿意}{yuan4yi5}[][HSK 2]
    \definition{v.}{estar disposto; estar pronto | desejar; ter esperança}
  \end{Phonetics}
\end{Entry}

%%%%%%%%%% 慢 %%%%%%%%%%
\subsection*{慢}\addcontentsline{loh}{figure}{慢}

\begin{Entry}{慢}{14}{⼼}
  \begin{Phonetics}{慢}{man4}[][HSK 1]
    \definition*{s.}{Sobrenome: Man}
    \definition{adj.}{lento; devagar; baixa velocidade; longa duração | rude; arrogante; sem educação com as pessoas | frouxo; lento}
    \definition{adv.}{lentamente}
  \antonymref{快}{kuai4}
  \end{Phonetics}
\end{Entry}

\begin{Entry}{慢车}{14,4}{⼼,⾞}
  \begin{Phonetics}{慢车}{man4che1}[][HSK 6]
    \definition{s.}{trem lento com muitas paradas | ônibus ou trem local; parada do trem}
  \antonymref{快车}{kuai4che1}
  \end{Phonetics}
\end{Entry}

\begin{Entry}{慢动作}{14,6,7}{⼼,⼒,⼈}
  \begin{Phonetics}{慢动作}{man4dong4zuo4}
    \definition{s.}{(cinema) câmera lenta}
  \end{Phonetics}
\end{Entry}

\begin{Entry}{慢性}{14,8}{⼼,⼼}
  \begin{Phonetics}{慢性}{man4xing4}[][HSK 7-9]
    \definition{adj.}{crônico; duradouro | lento (em fazer efeito)}
  \end{Phonetics}
\end{Entry}

\begin{Entry}{慢慢}{14,14}{⼼,⼼}
  \begin{Phonetics}{慢慢}{man4man4}[][HSK 3]
    \definition{adv.}{lentamente; vagarosamente; gradualmente | lentamente; vagarosamente; gradualmente; depois de um longo período de tempo}
  \end{Phonetics}
\end{Entry}

\begin{Entry}{慢慢来}{14,14,7}{⼼,⼼,⽊}
  \begin{Phonetics}{慢慢来}{man4man4 lai2}[][HSK 7-9]
    \definition{v.}{ir com calma; não ter pressa; significa não ter impaciência ao fazer as coisas e prosseguir no seu próprio ritmo}
  \end{Phonetics}
\end{Entry}

%%%%%%%%%% 慷 %%%%%%%%%%
\subsection*{慷}\addcontentsline{loh}{figure}{慷}

\begin{Entry}{慷}{14}{⼼}
  \begin{Phonetics}{慷}{kang1}
    \definition{adj.}{generoso | magnânimo}
  \end{Phonetics}
\end{Entry}

\begin{Entry}{慷慨}{14,12}{⼼,⼼}
  \begin{Phonetics}{慷慨}{kang1kai3}[][HSK 7-9]
    \definition{adj.}{generoso; descreve alguém como alguém que não é mesquinho; disposto a ajudar os outros com dinheiro ou bens | veemente; fervoroso; apaixonado; descreve alguém como alguém repleto de senso de justiça e emocionalmente intenso}
  \end{Phonetics}
\end{Entry}

%%%%%%%%%% 截 %%%%%%%%%%
\subsection*{截}\addcontentsline{loh}{figure}{截}

\begin{Entry}{截}{14}{⼽}
  \begin{Phonetics}{截}{jie2}[][HSK 7-9]
    \definition{clas.}{seção; pedaço; comprimento}
    \definition{prep.}{por (um tempo especificado); até}
    \definition{v.}{cortar; romper | parar; verificar; interromper; interceptar}
  \end{Phonetics}
\end{Entry}

\begin{Entry}{截止}{14,4}{⼽,⽌}
  \begin{Phonetics}{截止}{jie2zhi3}[][HSK 6]
    \definition{adv.}{até (um certo limite de tempo); por (um tempo especificado)}
  \end{Phonetics}
\end{Entry}

\begin{Entry}{截至}{14,6}{⼽,⾄}
  \begin{Phonetics}{截至}{jie2zhi4}[][HSK 6]
    \definition{adv.}{a partir de; até (um certo limite de tempo); por (um tempo especificado)}
  \end{Phonetics}
\end{Entry}

\begin{Entry}{截然不同}{14,12,4,6}{⼽,⽕,⼀,⼝}
  \begin{Phonetics}{截然不同}{jie2ran2-bu4tong2}[][HSK 7-9]
    \definition{expr.}{``Completamente diferente.''; tão diferente quanto preto e branco; polos opostos}
  \end{Phonetics}
\end{Entry}

%%%%%%%%%% 摔 %%%%%%%%%%
\subsection*{摔}\addcontentsline{loh}{figure}{摔}

\begin{Entry}{摔}{14}{⼿}
  \begin{Phonetics}{摔}{shuai1}[][HSK 5]
    \definition{v.}{cair; tropeçar; perder o equilíbrio | mergulhar; precipitar-se; cair de uma altura elevada | quebrar; fazer cair e quebrar | lançar; atirar; arremessar; joguar coisas com força e para baixo | bater; golpear; bater com força para que o que está grudado cair}
  \end{Phonetics}
\end{Entry}

\begin{Entry}{摔倒}{14,10}{⼿,⼈}
  \begin{Phonetics}{摔倒}{shuai1dao3}[][HSK 5]
    \definition{v.}{cair; tropeçar; perder o equilíbrio e cair}
  \end{Phonetics}
\end{Entry}

%%%%%%%%%% 摘 %%%%%%%%%%
\subsection*{摘}\addcontentsline{loh}{figure}{摘}

\begin{Entry}{摘}{14}{⼿}
  \begin{Phonetics}{摘}{zhai1}[][HSK 5]
    \definition{v.}{pegar; arrancar; tirar; colher (flores, frutos, folhas de plantas); retirar (coisas que estão sendo usadas ou penduradas) | selecionar; fazer extrações de | pedir dinheiro emprestado em caso de necessidade urgente | vencer; ganhar; alcançar; obter}
  \end{Phonetics}
\end{Entry}

%%%%%%%%%% 摧 %%%%%%%%%%
\subsection*{摧}\addcontentsline{loh}{figure}{摧}

\begin{Entry}{摧}{14}{⼿}
  \begin{Phonetics}{摧}{cui1}
    \definition{v.}{quebrar; destruir}
  \end{Phonetics}
\end{Entry}

\begin{Entry}{摧毁}{14,13}{⼿,⽎}
  \begin{Phonetics}{摧毁}{cui1hui3}[][HSK 7-9]
    \definition{v.}{destruir; esmagar; nocautear; destruir com grande força}
  \end{Phonetics}
\end{Entry}

%%%%%%%%%% 撇 %%%%%%%%%%
\subsection*{撇}\addcontentsline{loh}{figure}{撇}

\begin{Entry}{撇}{14}{⼿}
  \begin{Phonetics}{撇}{pie1}
    \definition{v.}{descartar; jogar ao mar; abandonar | desnatar; retirar delicadamente o líquido da superfície}
  \end{Phonetics}
  \begin{Phonetics}{撇}{pie3}[][HSK 7-9]
    \definition{clas.}{utilizado para coisas sobrancelhas e barbas}
    \definition{s.}{traço descendente à esquerda 丿(em caracteres chineses)}
    \definition{v.}{atirar; arremessar; lançar}
  \end{Phonetics}
\end{Entry}

%%%%%%%%%% 敲 %%%%%%%%%%
\subsection*{敲}\addcontentsline{loh}{figure}{敲}

\begin{Entry}{敲}{14}{⽁}
  \begin{Phonetics}{敲}{qiao1}[][HSK 5]
    \definition{v.}{bater; dar uma pancada; golpear | explorar alguém; cobrar a mais; extorquir; chantagear | lembrar; criticar; alertar; advertir}
  \end{Phonetics}
\end{Entry}

\begin{Entry}{敲门}{14,3}{⽁,⾨}
  \begin{Phonetics}{敲门}{qiao1 men2}[][HSK 5]
    \definition{v.}{bater na porta}
  \end{Phonetics}
\end{Entry}

\begin{Entry}{敲边鼓}{14,5,13}{⽁,⾡,⿎}
  \begin{Phonetics}{敲边鼓}{qiao1 bian1gu3}[][HSK 7-9]
    \definition{v.}{``Tocar trompa.'' | Coloquial: falar ou agir para ajudar alguém à margem; apoiar alguém; apoiar alguém em uma discussão}
  \end{Phonetics}
\end{Entry}

\begin{Entry}{敲诈}{14,7}{⽁,⾔}
  \begin{Phonetics}{敲诈}{qiao1zha4}[][HSK 7-9]
    \definition{v.}{extorquir; chantagear; usar o poder, a intimidação e as ameaças para extorquir dinheiro}
  \end{Phonetics}
\end{Entry}

%%%%%%%%%% 斡 %%%%%%%%%%
\subsection*{斡}\addcontentsline{loh}{figure}{斡}

\begin{Entry}{斡}{14}{⽃}
  \begin{Phonetics}{斡}{wo4}
    \definition{v.}{virar-se}
  \end{Phonetics}
\end{Entry}

\begin{Entry}{斡旋}{14,11}{⽃,⽅}
  \begin{Phonetics}{斡旋}{wo4xuan2}
    \definition{v.}{mediar (um conflito, etc.)}
  \end{Phonetics}
\end{Entry}

%%%%%%%%%% 旗 %%%%%%%%%%
\subsection*{旗}\addcontentsline{loh}{figure}{旗}

\begin{Entry}{旗}{14}{⽅}
  \begin{Phonetics}{旗}{qi2}
    \definition[面]{s.}{bandeira}
  \end{Phonetics}
\end{Entry}

\begin{Entry}{旗帜}{14,8}{⽅,⼱}
  \begin{Phonetics}{旗帜}{qi2zhi4}[][HSK 7-9]
    \definition[面]{s.}{bandeira; estandarte | modelo; bom exemplo; metáfora para modelo ou exemplo a seguir | bandeira (de um pensamento representativo ou posição política); essa metáfora se refere a uma ideologia, doutrina ou força política representativa ou influente}
  \end{Phonetics}
\end{Entry}

\begin{Entry}{旗袍}{14,10}{⽅,⾐}
  \begin{Phonetics}{旗袍}{qi2pao2}[][HSK 7-9]
    \definition[件,个]{s.}{qipao; cheongsam; uma túnica longa usada por mulheres, originalmente usada por mulheres manchus}
  \end{Phonetics}
\end{Entry}

%%%%%%%%%% 榜 %%%%%%%%%%
\subsection*{榜}\addcontentsline{loh}{figure}{榜}

\begin{Entry}{榜}{14}{⽊}
  \begin{Phonetics}{榜}{bang3}
    \definition[块]{s.}{lista publicada de nomes | Literário: placa horizontal inscrita | aviso; anúncio; proclamação antiga}
  \end{Phonetics}
\end{Entry}

\begin{Entry}{榜首}{14,9}{⽊,⾸}
  \begin{Phonetics}{榜首}{bang3shou3}
    \definition{s.}{cabeça da lista de candidatos aprovados; primeiro lugar em um concurso, etc. | topo da lista}
  \end{Phonetics}
\end{Entry}

\begin{Entry}{榜样}{14,10}{⽊,⽊}
  \begin{Phonetics}{榜样}{bang3yang4}[][HSK 7-9]
    \definition[个,位]{s.}{exemplo; modelo; padrão; pessoas ou coisas boas que valem a pena aprender, usado principalmente na linguagem falada}
  \end{Phonetics}
\end{Entry}

%%%%%%%%%% 槃 %%%%%%%%%%
\subsection*{槃}\addcontentsline{loh}{figure}{槃}

\begin{Entry}{槃}{14}{⽊}
  \begin{Phonetics}{槃}{pan2}
    \variantof{盘}
  \end{Phonetics}
\end{Entry}

%%%%%%%%%% 模 %%%%%%%%%%
\subsection*{模}\addcontentsline{loh}{figure}{模}

\begin{Entry}{模}{14}{⽊}
  \begin{Phonetics}{模}{mo2}
    \definition{s.}{padrão | modelo; exemplo | modelo (pessoa) | exame simulado | módulo}
    \definition{v.}{imitar | copiar; emular}
  \end{Phonetics}
  \begin{Phonetics}{模}{mu2}
    \definition*{s.}{Sobrenome: Mu}
    \definition{s.}{molde; padrão; matriz}
  \end{Phonetics}
\end{Entry}

\begin{Entry}{模仿}{14,6}{⽊,⼈}
  \begin{Phonetics}{模仿}{mo2fang3}[][HSK 5]
    \definition{v.}{copiar; imitar; aprender a fazer algo seguindo um modelo pronto}
  \end{Phonetics}
\end{Entry}

\begin{Entry}{模式}{14,6}{⽊,⼷}
  \begin{Phonetics}{模式}{mo2shi4}[][HSK 5]
    \definition{s.}{modelo; modo; padrão; a forma padrão de algo ou o modelo padrão que as pessoas podem seguir}
  \end{Phonetics}
\end{Entry}

\begin{Entry}{模拟}{14,7}{⽊,⼿}
  \begin{Phonetics}{模拟}{mo2ni3}[][HSK 7-9]
    \definition{v.}{ser análogo; imitar; simular; fazer de maneira formal}
  \end{Phonetics}
\end{Entry}

\begin{Entry}{模具}{14,8}{⽊,⼋}
  \begin{Phonetics}{模具}{mu2ju4}
    \definition{s.}{molde | matriz | padrão}
  \end{Phonetics}
\end{Entry}

\begin{Entry}{模型}{14,9}{⽊,⼟}
  \begin{Phonetics}{模型}{mo2xing2}[][HSK 4]
    \definition[个]{s.}{modelo; padrão; itens feitos em escala com base em objetos ou desenhos | molde; padrão; molde para fundir máquinas, objetos, etc.}
  \end{Phonetics}
\end{Entry}

\begin{Entry}{模范}{14,9}{⽊,⾋}
  \begin{Phonetics}{模范}{mo2fan4}[][HSK 5]
    \definition{adj.}{exemplar}
    \definition{s.}{modelo; exemplo excelente; pessoa exemplar; coisa exemplar; pessoas ou coisas exemplares que servem de modelo}
  \end{Phonetics}
\end{Entry}

\begin{Entry}{模样}{14,10}{⽊,⽊}
  \begin{Phonetics}{模样}{mu2yang4}[][HSK 5]
    \definition[副,种]{s.}{aparência; a aparência ou o estilo de vestir de uma pessoa | indicando uma estimativa aproximada de tempo ou idade; expressão de estimativas relativas a tempo, idade, etc. | tendência; situação; inclinação}
  \end{Phonetics}
\end{Entry}

\begin{Entry}{模特儿}{14,10,2}{⽊,⽜,⼉}
  \begin{Phonetics}{模特儿}{mo2te4r5}[][HSK 4]
    \definition[个,名,位]{s.}{modelo (pessoa que posa para um fotógrafo ou pintor ou escultor); objeto de representação ou referência usado por artistas para esboços e esculturas, como o corpo humano, objetos, modelos etc.; também se refere aos arquétipos que os estudiosos da literatura usam para retratar seus personagens | modelo (uma pessoa que usa roupas para exibir modas); pessoa ou manequim usado para exibir estilos de roupas}
  \end{Phonetics}
\end{Entry}

\begin{Entry}{模糊}{14,15}{⽊,⽶}
  \begin{Phonetics}{模糊}{mo2hu5}[][HSK 5]
    \definition{adj.}{vago; confuso; indistinto}
    \definition{v.}{confundir; desorientar}
  \end{Phonetics}
\end{Entry}

%%%%%%%%%% 歉 %%%%%%%%%%
\subsection*{歉}\addcontentsline{loh}{figure}{歉}

\begin{Entry}{歉}{14}{⽋}
  \begin{Phonetics}{歉}{qian4}
    \definition{adj.}{pobre, ruim (colheita) ruim; baixa produtividade (agrícola)}
    \definition{s.}{pedido de desculpas; apologia | quebra de safra}
    \definition{v.}{pedir desculpas; sentir pena dos outros}
  \end{Phonetics}
\end{Entry}

\begin{Entry}{歉意}{14,13}{⽋,⼼}
  \begin{Phonetics}{歉意}{qian4yi4}[][HSK 7-9]
    \definition{s.}{desculpas; arrependimento; pedido de desculpas}
  \end{Phonetics}
\end{Entry}

%%%%%%%%%% 歌 %%%%%%%%%%
\subsection*{歌}\addcontentsline{loh}{figure}{歌}

\begin{Entry}{歌}{14}{⽋}
  \begin{Phonetics}{歌}{ge1}[][HSK 1]
    \definition[首,支,段]{s.}{canção; poesia cantável}
    \definition{v.}{cantar; entoar | louvar; exaltar; cantar louvores a}
  \end{Phonetics}
\end{Entry}

\begin{Entry}{歌手}{14,4}{⽋,⼿}
  \begin{Phonetics}{歌手}{ge1shou3}[][HSK 3]
    \definition[个,位,名]{s.}{cantor; vocalista; pessoa com talento para cantar}
  \end{Phonetics}
\end{Entry}

\begin{Entry}{歌曲}{14,6}{⽋,⽈}
  \begin{Phonetics}{歌曲}{ge1qu3}[][HSK 5]
    \definition[首,支]{s.}{música; obra para as pessoas cantarem, uma combinação de poesia e música}
  \end{Phonetics}
\end{Entry}

\begin{Entry}{歌声}{14,7}{⽋,⼠}
  \begin{Phonetics}{歌声}{ge1sheng1}[][HSK 3]
    \definition{s.}{canto; voz cantada; som do canto}
  \end{Phonetics}
\end{Entry}

\begin{Entry}{歌词}{14,7}{⽋,⾔}
  \begin{Phonetics}{歌词}{ge1ci2}[][HSK 6]
    \definition{s.}{letra da música; libreto}
  \end{Phonetics}
\end{Entry}

\begin{Entry}{歌咏}{14,8}{⽋,⼝}
  \begin{Phonetics}{歌咏}{ge1yong3}[][HSK 7-9]
    \definition{v.}{cantar; cantar canções}
  \end{Phonetics}
\end{Entry}

\begin{Entry}{歌星}{14,9}{⽋,⽇}
  \begin{Phonetics}{歌星}{ge1xing1}[][HSK 6]
    \definition[位,名]{s.}{cantor famoso; estrela da música}
  \end{Phonetics}
\end{Entry}

\begin{Entry}{歌迷}{14,9}{⽋,⾡}
  \begin{Phonetics}{歌迷}{ge1mi2}[][HSK 3]
    \definition{s.}{fã de um cantor; pessoas que gostam de ouvir música ou cantar e ficam fascinadas por isso}
  \end{Phonetics}
\end{Entry}

\begin{Entry}{歌剧}{14,10}{⽋,⼑}
  \begin{Phonetics}{歌剧}{ge1ju4}[][HSK 7-9]
    \definition[场,出]{s.}{ópera | ópera ocidental; um drama que integra poesia, música, dança e outras artes, tendo o canto como principal característica}
  \end{Phonetics}
\end{Entry}

\begin{Entry}{歌颂}{14,10}{⽋,⾴}
  \begin{Phonetics}{歌颂}{ge1song4}[][HSK 7-9]
    \definition{v.}{cantar louvores de; exaltar; elogiar; elogio com poesia, geralmente se refere a elogiar com palavras, etc.}
  \end{Phonetics}
\end{Entry}

\begin{Entry}{歌唱}{14,11}{⽋,⼝}
  \begin{Phonetics}{歌唱}{ge1chang4}[][HSK 6]
    \definition{v.}{cantar | cantar em louvor de; louvor através de cânticos, recitações, etc.}
  \end{Phonetics}
\end{Entry}

\begin{Entry}{歌舞}{14,14}{⽋,⾇}
  \begin{Phonetics}{歌舞}{ge1wu3}[][HSK 7-9]
    \definition{s.}{canto e dança}
  \end{Phonetics}
\end{Entry}

%%%%%%%%%% 滴 %%%%%%%%%%
\subsection*{滴}\addcontentsline{loh}{figure}{滴}

\begin{Entry}{滴}{14}{⽔}
  \begin{Phonetics}{滴}{di1}[][HSK 6]
    \definition{clas.}{gota; quantificador para ``gotejamento''}
    \definition{s.}{uma gota}
    \definition{v.}{pingar}
  \end{Phonetics}
\end{Entry}

%%%%%%%%%% 漂 %%%%%%%%%%
\subsection*{漂}\addcontentsline{loh}{figure}{漂}

\begin{Entry}{漂}{14}{⽔}
  \begin{Phonetics}{漂}{piao1}[][HSK 7-9]
    \definition{v.}{flutuar; derivar; flutuar na superfície de um líquido; flutuar na superfície da água e mover-se com a corrente; mover-se com o vento}
  \end{Phonetics}
  \begin{Phonetics}{漂}{piao3}
    \definition{v.}{branquear (com água sanitária) | enxaguar; enxaguar com água para remover as impurezas}
  \end{Phonetics}
  \begin{Phonetics}{漂}{piao4}
    \definition{adj.}{bonita; usado em 漂亮}
    \definition{v.}{falhar; terminar em fracasso}[这笔投资的钱全都漂了。===Todo o dinheiro desse investimento foi perdido.]
  \seealsoref{漂亮}{piao4liang5}
  \end{Phonetics}
\end{Entry}

\begin{Entry}{漂亮}{14,9}{⽔,⼇}
  \begin{Phonetics}{漂亮}{piao4liang5}[][HSK 2]
    \definition{adj.}{bonito; lindo; atraente; de boa aparência; esteticamente agradável | excelente; notável | não pode ser utilizado para descrever homens}
  \end{Phonetics}
\end{Entry}

\begin{Entry}{漂流}{14,10}{⽔,⽔}
  \begin{Phonetics}{漂流}{piao1liu2}
    \definition{s.}{\emph{rafting}}
    \definition{v.}{derivar; flutuar; flutuar na água e à deriva com a corrente | deixar-se levar; ter vida errante; estar à deriva | praticar rafting em um pequeno barco ou jangada inflável, geralmente como atividade recreativa; descer as corredeiras em pequenos barcos ou jangadas é hoje considerado, em grande parte, uma atividade de recreação aquática}
  \end{Phonetics}
\end{Entry}

%%%%%%%%%% 漆 %%%%%%%%%%
\subsection*{漆}\addcontentsline{loh}{figure}{漆}

\begin{Entry}{漆}{14}{⽔}
  \begin{Phonetics}{漆}{qi1}[][HSK 7-9]
    \definition*{s.}{Sobrenome: Qi}
    \definition[桶,种,层]{s.}{laca; tinta}
    \definition{v.}{aplicar verniz; pintar}
  \end{Phonetics}
\end{Entry}

%%%%%%%%%% 漏 %%%%%%%%%%
\subsection*{漏}\addcontentsline{loh}{figure}{漏}

\begin{Entry}{漏}{14}{⽔}
  \begin{Phonetics}{漏}{lou4}[][HSK 5]
    \definition{s.}{relógio de água; ampulheta | falha; ponto fraco | gonorreia; a medicina tradicional chinesa refere"-se a certas doenças que causam secreção de pus, sangue e muco | unidade de tempo medida por um relógio de água durante a noite}
    \definition{v.}{(líquido, gás, etc.) pingar; vazar; escorrer; cair (de um buraco ou fenda) | vazar; deixar escapar; divulgar | perder; deixar de fora por engano | vazar; o objeto tem poros e pode vazar coisas | há uma fuga de ar}
  \end{Phonetics}
\end{Entry}

\begin{Entry}{漏电}{14,5}{⽔,⽥}
  \begin{Phonetics}{漏电}{lou4dian4}
    \definition{v.}{vazar eletricidade}
  \end{Phonetics}
\end{Entry}

\begin{Entry}{漏洞}{14,9}{⽔,⽔}
  \begin{Phonetics}{漏洞}{lou4dong4}[][HSK 5]
    \definition[个,点]{s.}{vazamento; rachadura; lacunas ou buracos desnecessários que permitem que coisas vazem | falha; defeito; lacuna; (fala, ação, método, etc.) imperfeições}
  \end{Phonetics}
\end{Entry}

%%%%%%%%%% 演 %%%%%%%%%%
\subsection*{演}\addcontentsline{loh}{figure}{演}

\begin{Entry}{演}{14}{⽔}
  \begin{Phonetics}{演}{yan3}[][HSK 3]
    \definition{v.}{desenvolver; evoluir | deduzir; elaborar | exercitar; praticar | representar; atuar; encenar | desempenhar}
  \end{Phonetics}
\end{Entry}

\begin{Entry}{演出}{14,5}{⽔,⼐}
  \begin{Phonetics}{演出}{yan3chu1}[][HSK 3]
    \definition[场,次]{s.}{show; concerto; performance}
    \definition{v.}{apresentar; representar; fazer um show; apresentar peças teatrais, danças, artes cênicas, acrobacias, etc. para o público apreciar}
  \end{Phonetics}
\end{Entry}

\begin{Entry}{演讲}{14,6}{⽔,⾔}
  \begin{Phonetics}{演讲}{yan3jiang3}[][HSK 4]
    \definition[场,次]{s.}{palestra; discurso; ato ou a atividade de apresentar ou expressar ideias, opiniões ou informações oralmente em público ou diante de um público}
    \definition{v.}{dar uma palestra; fazer um discurso; informar o público sobre uma determinada área de conhecimento ou opinião sobre um determinado assunto}
  \end{Phonetics}
\end{Entry}

\begin{Entry}{演员}{14,7}{⽔,⼝}
  \begin{Phonetics}{演员}{yan3yuan2}[][HSK 3]
    \definition[个,位,名]{s.}{ator; artista; pessoas que participam de apresentações teatrais, cinematográficas, de dança, de artes cênicas, de acrobacias, etc.}
  \end{Phonetics}
\end{Entry}

\begin{Entry}{演奏}{14,9}{⽔,⼤}
  \begin{Phonetics}{演奏}{yan3zou4}[][HSK 6]
    \definition{v.}{tocar um instrumento musical; fazer uma apresentação instrumental}
  \end{Phonetics}
\end{Entry}

\begin{Entry}{演唱}{14,11}{⽔,⼝}
  \begin{Phonetics}{演唱}{yan3chang4}[][HSK 3]
    \definition{v.}{cantar em uma performance; apresentar canções, óperas, peças teatrais, etc.}
  \end{Phonetics}
\end{Entry}

\begin{Entry}{演唱会}{14,11,6}{⽔,⼝,⼈}
  \begin{Phonetics}{演唱会}{yan3chang4hui4}[][HSK 3]
    \definition[个,场]{s.}{recital vocal; concerto vocal; uma forma de apresentação centrada no canto, acompanhada por movimentos de dança simples}
  \end{Phonetics}
\end{Entry}

%%%%%%%%%% 漫 %%%%%%%%%%
\subsection*{漫}\addcontentsline{loh}{figure}{漫}

\begin{Entry}{漫}{14}{⽔}
  \begin{Phonetics}{漫}{man4}[][HSK 7-9]
    \definition{adj.}{livre; desimpedido; casual; sem restrições; arbitrário | em todo lugar; por toda parte | longo; extenso; distante}
    \definition{adv.}{não; não há necessidade de; expressa negação, equivalente a 不要}
    \definition{v.}{transbordar; inundar; alagar | estar em todo lugar; estar em todos os lugares}
  \seealsoref{不要}{bu2yao4}
  \end{Phonetics}
\end{Entry}

\begin{Entry}{漫长}{14,4}{⽔,⾧}
  \begin{Phonetics}{漫长}{man4chang2}[][HSK 5]
    \definition{adj.}{muito longo; interminável; (tempo, espaço) dura muito tempo}
  \end{Phonetics}
\end{Entry}

\begin{Entry}{漫画}{14,8}{⽔,⽥}
  \begin{Phonetics}{漫画}{man4hua4}[][HSK 5]
    \definition[幅,本,张,套]{s.}{desenho animado; caricatura; \emph{cartoon}}
  \end{Phonetics}
\end{Entry}

\begin{Entry}{漫骂}{14,9}{⽔,⾺}
  \begin{Phonetics}{漫骂}{man4ma4}
    \definition{v.}{usar linguagem ofensiva contra; insultar; difamar}
    \variantof{谩骂}
  \end{Phonetics}
\end{Entry}

\begin{Entry}{漫游}{14,12}{⽔,⽔}
  \begin{Phonetics}{漫游}{man4you2}[][HSK 7-9]
    \definition{v.}{vagar; perambular; dar voltas; fazer uma viagem de lazer | vagar; navegar; isso se refere à capacidade de telefones celulares e outros dispositivos se conectarem a qualquer terminal em outra área de serviço por meio da rede, após entrarem em uma área de serviço não registrada | (peixes) mover-se livremente; nadar livremente na água}
  \end{Phonetics}
\end{Entry}

%%%%%%%%%% 煽 %%%%%%%%%%
\subsection*{煽}\addcontentsline{loh}{figure}{煽}

\begin{Entry}{煽}{14}{⽕}
  \begin{Phonetics}{煽}{shan1}
    \definition{v.}{abanar (fogo); agitar um leque ou outra folha | incitar; instigar; agitar | vangloriar-se de; esbanjar prêmios em}
  \end{Phonetics}
\end{Entry}

\begin{Entry}{煽动}{14,6}{⽕,⼒}
  \begin{Phonetics}{煽动}{shan1dong4}[][HSK 7-9]
    \definition{v.}{instigar; incitar (alguém a fazer coisas ruins); agitar; inflamar}
  \end{Phonetics}
\end{Entry}

%%%%%%%%%% 熊 %%%%%%%%%%
\subsection*{熊}\addcontentsline{loh}{figure}{熊}

\begin{Entry}{熊}{14}{⽕}
  \begin{Phonetics}{熊}{xiong2}[][HSK 5]
    \definition*{s.}{Sobrenome: Xiong}
    \definition[头,只]{s.}{urso}
    \definition{v.}{repreender; censurar}
  \end{Phonetics}
\end{Entry}

\begin{Entry}{熊猫}{14,11}{⽕,⽝}
  \begin{Phonetics}{熊猫}{xiong2mao1}
    \definition[把,只]{s.}{panda gigante}
  \seealsoref{猫熊}{mao1xiong2}
  \end{Phonetics}
\end{Entry}

%%%%%%%%%% 熏 %%%%%%%%%%
\subsection*{熏}\addcontentsline{loh}{figure}{熏}

\begin{Entry}{熏}{14}{⽕}
  \begin{Phonetics}{熏}{xun1}
    \definition{v.}{expor à fumaça ou vapores; fumigar | tratar (carne, peixe, etc.) com fumaça; defumar | tornar perfumado com incenso, etc. | sufocar (asfixia e envenenamento por gás)}
  \end{Phonetics}
\end{Entry}

\begin{Entry}{熏香}{14,9}{⽕,⾹}
  \begin{Phonetics}{熏香}{xun1xiang1}
    \definition{s.}{incenso}
  \end{Phonetics}
\end{Entry}

%%%%%%%%%% 熬 %%%%%%%%%%
\subsection*{熬}\addcontentsline{loh}{figure}{熬}

\begin{Entry}{熬}{14}{⽕}
  \begin{Phonetics}{熬}{ao1}
    \definition{v.}{ensopar; ferver; cozinhar em água}
  \end{Phonetics}
  \begin{Phonetics}{熬}{ao2}[][HSK 7-9]
    \definition{v.}{ferver; ensopar; fazer uma decocção; cozinhar em fogo baixo por muito tempo | preparar; infundir; extrair a essência por fervura longa | resistir; suportar (angústia, tempos difíceis, etc.)}
  \end{Phonetics}
\end{Entry}

\begin{Entry}{熬夜}{14,8}{⽕,⼣}
  \begin{Phonetics}{熬夜}{ao2/ye4}[][HSK 7-9]
    \definition{v.+compl.}{ficar acordado a noite toda ou até tarde da noite}
  \end{Phonetics}
\end{Entry}

%%%%%%%%%% 疑 %%%%%%%%%%
\subsection*{疑}\addcontentsline{loh}{figure}{疑}

\begin{Entry}{疑}{14}{⽦}
  \begin{Phonetics}{疑}{yi2}
    \definition{adj.}{duvidoso; incerto}
    \definition{v.}{duvidar; desacreditar; suspeitar}
  \end{Phonetics}
\end{Entry}

\begin{Entry}{疑问}{14,6}{⽦,⾨}
  \begin{Phonetics}{疑问}{yi2wen4}[][HSK 4]
    \definition[个,些]{s.}{dúvida; consulta; pergunta; questionamento; coisas que não podem ser determinadas ou explicadas}
  \end{Phonetics}
\end{Entry}

%%%%%%%%%% 瘦 %%%%%%%%%%
\subsection*{瘦}\addcontentsline{loh}{figure}{瘦}

\begin{Entry}{瘦}{14}{⽧}
  \begin{Phonetics}{瘦}{shou4}[][HSK 5]
    \definition{adj.}{magro; esquelético | magro | apertado | infértil; pobre | esquelético; pouca gordura; pouca carne | (roupas, sapatos, meias, etc.) apertado |magra; (carne comestível) com baixo teor de gordura}
    \definition{v.}{perder peso}
  \antonymref{肥}{fei2}
  \antonymref{或}{huo4}
  \antonymref{胖}{pang4}
  \end{Phonetics}
\end{Entry}

%%%%%%%%%% 瞅 %%%%%%%%%%
\subsection*{瞅}\addcontentsline{loh}{figure}{瞅}

\begin{Entry}{瞅}{14}{⽬}
  \begin{Phonetics}{瞅}{chou3}[][HSK 7-9]
    \definition{v.}{Dialeto: olhar para}[让我瞅瞅。===Deixe-me dar uma olhada.]
  \end{Phonetics}
\end{Entry}

%%%%%%%%%% 碧 %%%%%%%%%%
\subsection*{碧}\addcontentsline{loh}{figure}{碧}

\begin{Entry}{碧}{14}{⽯}
  \begin{Phonetics}{碧}{bi4}
    \definition*{s.}{Sobrenome: Bi}
    \definition{adj.}{verde claro | azul claro | azul; verde-azulado; esverdeado; azul-celeste. turquesa}
    \definition{s.}{Literário: jade verde | safira}
  \end{Phonetics}
\end{Entry}

\begin{Entry}{碧绿}{14,11}{⽯,⽷}
  \begin{Phonetics}{碧绿}{bi4lv4}[][HSK 7-9]
    \definition{adj.}{verde jade; verde esmeralda; descreve um verde muito brilhante e profundo}
  \end{Phonetics}
\end{Entry}

%%%%%%%%%% 碳 %%%%%%%%%%
\subsection*{碳}\addcontentsline{loh}{figure}{碳}

\begin{Entry}{碳}{14}{⽯}
  \begin{Phonetics}{碳}{tan4}
    \definition{s.}{carbono (elemento químico)}
  \end{Phonetics}
\end{Entry}

\begin{Entry}{碳足迹}{14,7,9}{⽯,⾜,⾡}
  \begin{Phonetics}{碳足迹}{tan4 zu2ji4}
    \definition{s.}{pegada de carbono}
  \end{Phonetics}
\end{Entry}

%%%%%%%%%% 磁 %%%%%%%%%%
\subsection*{磁}\addcontentsline{loh}{figure}{磁}

\begin{Entry}{磁}{14}{⽯}
  \begin{Phonetics}{磁}{ci2}
    \definition[块]{s.}{porcelana | (física) magnetismo; propriedade de atrair ferro, níquel, etc. | (dialeto)  (de relação) próximo; íntimo}
  \end{Phonetics}
\end{Entry}

\begin{Entry}{磁卡}{14,5}{⽯,⼘}
  \begin{Phonetics}{磁卡}{ci2ka3}[][HSK 7-9]
    \definition[张]{s.}{cartão magnético}
  \end{Phonetics}
\end{Entry}

\begin{Entry}{磁带}{14,9}{⽯,⼱}
  \begin{Phonetics}{磁带}{ci2dai4}[][HSK 7-9]
    \definition[盘,盒,卷]{s.}{fita; fita magnética; cassete; uma fita plástica tratada com material magnético que pode gravar som ou imagens}
  \end{Phonetics}
\end{Entry}

\begin{Entry}{磁铁}{14,10}{⽯,⾦}
  \begin{Phonetics}{磁铁}{ci2tie3}
    \definition{s.}{imã | magneto}
  \seealsoref{吸铁石}{xi1tie3shi2}
  \end{Phonetics}
\end{Entry}

\begin{Entry}{磁盘}{14,11}{⽯,⽫}
  \begin{Phonetics}{磁盘}{ci2pan2}[][HSK 7-9]
    \definition{s.}{Computação: disco; disquete; um disco é um dispositivo de armazenamento que usa tecnologia de gravação magnética para armazenar dados}
  \end{Phonetics}
\end{Entry}

%%%%%%%%%% 磋 %%%%%%%%%%
\subsection*{磋}\addcontentsline{loh}{figure}{磋}

\begin{Entry}{磋}{14}{⽯}
  \begin{Phonetics}{磋}{cuo1}
    \definition{v.}{moer e polir marfim (significado original) | Figurativo: consultar; trocar opiniões | moer; polir}
  \end{Phonetics}
\end{Entry}

\begin{Entry}{磋商}{14,11}{⽯,⼝}
  \begin{Phonetics}{磋商}{cuo1shang1}[][HSK 7-9]
    \definition{v.}{consultar; negociar; trocar pontos de vista; discutir repetidamente; discutir cuidadosamente}
  \end{Phonetics}
\end{Entry}

%%%%%%%%%% 稳 %%%%%%%%%%
\subsection*{稳}\addcontentsline{loh}{figure}{稳}

\begin{Entry}{稳}{14}{⽲}
  \begin{Phonetics}{稳}{wen3}[][HSK 4]
    \definition{adj.}{constante; estável; firme | estável; estático; sedado | seguro; confiável; certo}
    \definition{adv.}{certamente; com certeza; seguramente; sem dúvida}
    \definition{v.}{estabilizar, manter estável; acalmar}
  \end{Phonetics}
\end{Entry}

\begin{Entry}{稳定}{14,8}{⽲,⼧}
  \begin{Phonetics}{稳定}{wen3ding4}[][HSK 4]
    \definition{adj.}{estável; firme; descreve uma natureza, um estado, etc. relativamente fixo; não muda significativamente}
    \definition{v.}{manter estável; estabilizar}
  \end{Phonetics}
\end{Entry}

%%%%%%%%%% 竭 %%%%%%%%%%
\subsection*{竭}\addcontentsline{loh}{figure}{竭}

\begin{Entry}{竭}{14}{⽴}
  \begin{Phonetics}{竭}{jie2}
    \definition*{s.}{Sobrenome: Jie}
    \definition{v.}{esgotar; consumir | Literário: secar; drenar}
  \end{Phonetics}
\end{Entry}

\begin{Entry}{竭力}{14,2}{⽴,⼒}
  \begin{Phonetics}{竭力}{jie2li4}[][HSK 7-9]
    \definition{v.}{fazer o máximo; fazer o máximo; não poupar esforços; tentar por todos os meios possíveis; dar o melhor de si; usar todos os esforços do corpo e da mente para\dots; usar cada grama de sua energia}
  \end{Phonetics}
\end{Entry}

\begin{Entry}{竭尽全力}{14,6,6,2}{⽴,⼫,⼊,⼒}
  \begin{Phonetics}{竭尽全力}{jie2jin4-quan2li4}[][HSK 7-9]
    \definition{expr.}{``Dê o seu melhor.''; não poupar esforços; fazer o máximo possível; com todas as forças; usar todas as suas forças para descrever o ato de fazer o máximo esforço; fazer o máximo possível; fazer tudo o que estiver ao seu alcance}
  \end{Phonetics}
\end{Entry}

%%%%%%%%%% 端 %%%%%%%%%%
\subsection*{端}\addcontentsline{loh}{figure}{端}

\begin{Entry}{端}{14}{⽴}
  \begin{Phonetics}{端}{duan1}[][HSK 6]
    \definition*{s.}{Sobrenome: Duan}
    \definition{adj.}{adequado; próprio | reto; correto}
    \definition{s.}{fim; extremidade | começo | item; ponto; pista, projeto ou aspecto | causa; razão | problema; incidente; coisas (geralmente se refere a coisas ruins, como acidentes, disputas, etc.)}
    \definition{v.}{carregar; segurar algo nivelado com ambas as mãos; segurar algo horizontalmente | erradicar; eliminar; acabar com; remover completamente; varrer | dar ares de superioridade | revelar}
  \end{Phonetics}
\end{Entry}

\begin{Entry}{端午节}{14,4,5}{⽴,⼗,⾋}
  \begin{Phonetics}{端午节}{duan1wu3jie2}[][HSK 6]
    \definition*[个]{s.}{Festa do Duplo Cinco, Festival dos Barcos-Dragão (5º~dia do quinto mês lunar)}
  \end{Phonetics}
\end{Entry}

\begin{Entry}{端正}{14,5}{⽴,⽌}
  \begin{Phonetics}{端正}{duan1zheng4}[][HSK 7-9]
    \definition{adj.}{apropriado; correto; não torto ou inclinado | ereto; integridade; decência}
    \definition{v.}{corrigir; fazer o certo}
  \end{Phonetics}
\end{Entry}

%%%%%%%%%% 算 %%%%%%%%%%
\subsection*{算}\addcontentsline{loh}{figure}{算}

\begin{Entry}{算}{14}{⽵}
  \begin{Phonetics}{算}{suan4}[][HSK 2]
    \definition{adv.}{finalmente; por fim; no final; significa que, após um longo período de tempo ou muitas dificuldades, finalmente se alcançou o objetivo, equivalente a 总算}
    \definition{v.}{calcular; estimar; computar | contar; incluir | planejar; calcular; projetar | pensar; supor; especular | considerar; considerar como; contar como; reconhecer como | (aritmética) contar; ter peso | deixe estar; deixe passar; seguido por 了: desistir, não se importar mais}
  \seealsoref{了}{le5}
  \seealsoref{总算}{zong3suan4}
  \end{Phonetics}
\end{Entry}

\begin{Entry}{算了}{14,2}{⽵,⼅}
  \begin{Phonetics}{算了}{suan4le5}[][HSK 6]
    \definition{part.}{deixe estar; deixe passar; usado no final de uma frase para expressar imperativo, término, etc.}
    \definition{v.}{deixar; deixe estar; deixe passar; esquecer isso; não querer continuar; é usado para persuadir os outros ou para expressar que posso aceitar a situação atual, para encerrar o assunto ou assunto atual, ou para dizer ``esqueça''}
  \end{Phonetics}
\end{Entry}

\begin{Entry}{算命}{14,8}{⽵,⼝}
  \begin{Phonetics}{算命}{suan4ming4}
    \definition{s.}{cartomante}
    \definition{v.}{ler a sorte | fazer advinhações}
  \end{Phonetics}
\end{Entry}

\begin{Entry}{算是}{14,9}{⽵,⽇}
  \begin{Phonetics}{算是}{suan4shi4}[][HSK 6]
    \definition{adv.}{finalmente; por fim; depois de muito tempo, o objetivo foi finalmente alcançado}
    \definition{v.}{contar como; pensar que; ser considerado}
  \end{Phonetics}
\end{Entry}

%%%%%%%%%% 管 %%%%%%%%%%
\subsection*{管}\addcontentsline{loh}{figure}{管}

\begin{Entry}{管}{14}{⽵}
  \begin{Phonetics}{管}{guan3}[][HSK 3]
    \definition*{s.}{Guan, um estado da dinastia Zhou | Sobrenome: Guan}
    \definition{adj.}{estreito; restrito; limitado; pequeno}
    \definition{clas.}{usado para objetos cilíndricos longos e finos}
    \definition{conj.}{não importa (quem, o quê, como, etc.)}
    \definition{prep.}{função semelhante a 把, usada especificamente em conjunto com 叫}
    \definition[根,条,排]{s.}{cano; tubo | instrumento musical de sopro | válvula; tubo | duto; canal; vasos}
    \definition{v.}{administrar; dirigir; controlar; cuidar; ser responsável por | ter jurisdição sobre; administrar | disciplinar (crianças ou alunos) | preocupar-se com; importar-se com; incomodar-se com; intervir | fornecer; garantir | supervisionar | governar | submeter alguém a disciplina | assumir; arcar com | incomodar; interferir | assegurar; garantir}
  \seealsoref{把}{ba3}
  \seealsoref{叫}{jiao4}
  \end{Phonetics}
\end{Entry}

\begin{Entry}{管子}{14,3}{⽵,⼦}
  \begin{Phonetics}{管子}{guan3zi5}[][HSK 7-9]
    \definition*{s.}{Guanzi ou Guan Zhong 管仲 (-645 a.C.), famoso político de Qi (齐国) do período da Primavera e do Outono | Guanzi, livro clássico contendo escritos de Guan Zhong e sua escola}
  \seealsoref{管仲}{guan3 zhong4}
  \seealsoref{齐国}{qi2 guo2}
  \end{Phonetics}
\end{Entry}

\begin{Entry}{管……叫……}{14,5}{⽵,⼝}
  \begin{Phonetics}{管……叫……}{guan3 jiao4}
    \definition{expr.}{chamar alguém (ou algo) de alguém (ou algo)}
  \end{Phonetics}
\end{Entry}

\begin{Entry}{管用}{14,5}{⽵,⽤}
  \begin{Phonetics}{管用}{guan3yong4}[][HSK 7-9]
    \definition{adj.}{eficaz; funcional}
  \end{Phonetics}
\end{Entry}

\begin{Entry}{管仲}{14,6}{⽵,⼈}
  \begin{Phonetics}{管仲}{guan3 zhong4}
    \definition*{s.}{uma visão restrita através de um tubo de bambu | conhecido como tubo de Guangzi 管子}
    \definition*{s.}{Guan Zhong (-645 aC), famoso político do Qi (齐国) do período da Primavera e Outono}
  \seealsoref{管子}{guan3zi5}
  \seealsoref{齐国}{qi2 guo2}
  \end{Phonetics}
\end{Entry}

\begin{Entry}{管家}{14,10}{⽵,⼧}
  \begin{Phonetics}{管家}{guan3jia1}[][HSK 7-9]
    \definition[个]{s.}{mordomo; antigamente, referia"-se a alguém que administrava os negócios de uma família rica | governanta; alguém que gerencia as tarefas domésticas | gerente; governanta; uma pessoa que administra bens ou negócios familiares ou coletivos}
    \definition{v.}{administrar uma casa}
  \end{Phonetics}
\end{Entry}

\begin{Entry}{管教}{14,11}{⽵,⽁}
  \begin{Phonetics}{管教}{guan3jiao4}[][HSK 7-9]
    \definition{adv.}{Dialeto: certamente; seguramente}
    \definition{v.}{corrigir; disciplinar alguém júnior | responsabilizar"-se por | ensinar}
  \end{Phonetics}
\end{Entry}

\begin{Entry}{管理}{14,11}{⽵,⽟}
  \begin{Phonetics}{管理}{guan3li3}[][HSK 3]
    \definition{v.}{gerenciar; executar; administrar; governar; estar encarregado de; responsável por garantir o bom andamento de uma determinada tarefa | controlar; gerenciar; fazer com que pessoas e animais obedeçam ou se comportem de maneira ordeira | cuidar; zelar por; proteger; cuidar, organizar coisas}
  \end{Phonetics}
\end{Entry}

\begin{Entry}{管理费}{14,11,9}{⽵,⽟,⾙}
  \begin{Phonetics}{管理费}{guan3li3fei4}[][HSK 7-9]
    \definition{s.}{despesas de gestão; custos de administração | taxa de administração}
  \end{Phonetics}
\end{Entry}

\begin{Entry}{管道}{14,12}{⽵,⾡}
  \begin{Phonetics}{管道}{guan3dao4}[][HSK 6]
    \definition[根,千米,公里]{s.}{oleoduto; canal; túnel; tubulação; um tubo feito de metal ou outro material usado para transportar ou descarregar fluidos (como vapor, gás, óleo, água, etc.) | caminho; canal; abordagem}
  \end{Phonetics}
\end{Entry}

\begin{Entry}{管辖}{14,14}{⽵,⾞}
  \begin{Phonetics}{管辖}{guan3xia2}[][HSK 7-9]
    \definition{v.}{gerenciar; governar (pessoal, assuntos, áreas, casos, etc.)}
  \end{Phonetics}
\end{Entry}

%%%%%%%%%% 精 %%%%%%%%%%
\subsection*{精}\addcontentsline{loh}{figure}{精}

\begin{Entry}{精}{14}{⽶}
  \begin{Phonetics}{精}{jing1}[][HSK 6]
    \definition{adj.}{refinado; escolhido; purificado ou selecionado | perfeito; excelente; melhor | fino; preciso; meticuloso | inteligente; astuto; esperto | habilidoso; versado; proficiente}
    \definition{adv.}{muito; extremamente; antes de certos adjetivos, significa 十分 ou 非常}
    \definition{s.}{extrato; essência; essência refinada ou selecionada; extraída | energia; espírito | semente; esperma; sêmen | \emph{goblin}; espírito; elfo; demônio}
  \seealsoref{非常}{fei1chang2}
  \seealsoref{十分}{shi2fen1}
  \antonymref{粗}{cu1}
  \end{Phonetics}
\end{Entry}

\begin{Entry}{精力}{14,2}{⽶,⼒}
  \begin{Phonetics}{精力}{jing1li4}[][HSK 4]
    \definition[些]{s.}{energia; vigor; força mental e física}
  \end{Phonetics}
\end{Entry}

\begin{Entry}{精子}{14,3}{⽶,⼦}
  \begin{Phonetics}{精子}{jing1zi3}
    \definition{s.}{espermatozoide; célula germinativa}
  \end{Phonetics}
\end{Entry}

\begin{Entry}{精心}{14,4}{⽶,⼼}
  \begin{Phonetics}{精心}{jing1xin1}[][HSK 7-9]
    \definition{adv.}{meticulosamente; cuidadosamente; elaboradamente; preste muita atenção; concentre-se totalmente}[他们精心设计了这个项目。===Eles planejaram este projeto meticulosamente.]
  \end{Phonetics}
\end{Entry}

\begin{Entry}{精打细算}{14,5,8,14}{⽶,⼿,⽷,⽵}
  \begin{Phonetics}{精打细算}{jing1da3-xi4suan4}[][HSK 7-9]
    \definition{expr.}{seja muito cuidadoso nos cálculos; contagem precisa; seja preciso nos cálculos; faça um orçamento rigoroso; cálculo cuidadoso e detalhado (meticuloso); cálculo cuidadoso e orçamento rigoroso; conte cada centavo e faça cada centavo valer a pena; corte os gastos com precisão; planejar com meticulosidade e cuidado, isso significa calcular com precisão o uso de mão de obra, recursos materiais e recursos financeiros para evitar desperdícios}
  \end{Phonetics}
\end{Entry}

\begin{Entry}{精华}{14,6}{⽶,⼗}
  \begin{Phonetics}{精华}{jing1hua2}[][HSK 7-9]
    \definition{s.}{elite; creme; escolha; essência; quintessência; a melhor e mais refinada parte de tudo | glória; esplendor; brilho; luz (do Sol e da Lua)}
  \end{Phonetics}
\end{Entry}

\begin{Entry}{精妙}{14,7}{⽶,⼥}
  \begin{Phonetics}{精妙}{jing1miao4}[][HSK 7-9]
    \definition{adj.}{requintado | fino e delicado (geralmente de obras de arte)}
  \end{Phonetics}
\end{Entry}

\begin{Entry}{精灵}{14,7}{⽶,⽕}
  \begin{Phonetics}{精灵}{jing1ling2}
    \definition{s.}{espírito | fada | elfo | duende | gênio}
  \end{Phonetics}
\end{Entry}

\begin{Entry}{精明}{14,8}{⽶,⽇}
  \begin{Phonetics}{精明}{jing1ming2}[][HSK 7-9]
    \definition{adj.}{astuto; sagaz; perspicaz; inteligente e brilhante}
  \end{Phonetics}
\end{Entry}

\begin{Entry}{精练}{14,8}{⽶,⽷}
  \begin{Phonetics}{精练}{jing1lian4}[][HSK 7-9]
    \definition{adj.}{conciso; sucinto; lacônico | refinado; palavras e frases redundantes eliminadas}
    \definition{v.}{praticar intensivamente}
  \synonymref{干脆}{gan1cui4}
  \synonymref{简单}{jian3dan1}
  \synonymref{简洁}{jian3jie2}
  \synonymref{精炼}{jing1lian4}
  \antonymref{冗长}{rong3chang2}
  \end{Phonetics}
\end{Entry}

\begin{Entry}{精细}{14,8}{⽶,⽷}
  \begin{Phonetics}{精细}{jing1xi4}[][HSK 7-9]
    \definition{adj.}{fino; cuidadoso; meticuloso; muito delicado | astuto; perspicaz e cuidadoso; muito meticuloso}
  \end{Phonetics}
\end{Entry}

\begin{Entry}{精英}{14,8}{⽶,⾋}
  \begin{Phonetics}{精英}{jing1ying1}[][HSK 7-9]
    \definition{s.}{creme; essência; quintessência | escolhido; elite; pessoa de habilidade excepcional}
  \end{Phonetics}
\end{Entry}

\begin{Entry}{精品}{14,9}{⽶,⼝}
  \begin{Phonetics}{精品}{jing1pin3}[][HSK 6]
    \definition[个]{s.}{belas obras (de arte); objetos de arte | produtos de qualidade; artigos de excelente qualidade; produto \emph{premium}}
  \end{Phonetics}
\end{Entry}

\begin{Entry}{精炼}{14,9}{⽶,⽕}
  \begin{Phonetics}{精炼}{jing1lian4}
    \definition{adj.}{conciso; sucinto; lacônico}
    \definition{s.}{refino; remoção de impurezas}
    \definition{v.}{Metalurgia: refinar; purificar; fundir}
  \end{Phonetics}
\end{Entry}

\begin{Entry}{精神}{14,9}{⽶,⽰}
  \begin{Phonetics}{精神}{jing1shen2}[][HSK 3]
    \definition[种,个,类,股]{s.}{espírito; mente; estado mental; refere"-se à consciência, às atividades mentais e ao estado psicológico geral de uma pessoa | substância; espírito; essência; propósito; significado principal}
  \end{Phonetics}
  \begin{Phonetics}{精神}{jing1shen5}[][HSK 3]
    \definition{adj.}{animado; espirituoso; vigoroso; descreve uma pessoa como cheia de energia | muito bonito; boa aparência, bom físico}
    \definition[种,个,类,股]{s.}{impulso; vigor; vitalidade}
  \end{Phonetics}
\end{Entry}

\begin{Entry}{精神病}{14,9,10}{⽶,⽰,⽧}
  \begin{Phonetics}{精神病}{jing1shen2bing4}[][HSK 7-9]
    \definition{s.}{doença mental; transtorno mental; psicose}[这是幻想型精神病的体现。===Isso é uma manifestação de psicose delirante.]
  \end{Phonetics}
\end{Entry}

\begin{Entry}{精美}{14,9}{⽶,⽺}
  \begin{Phonetics}{精美}{jing1mei3}[][HSK 6]
    \definition{adj.}{elegante; requintado}
  \end{Phonetics}
\end{Entry}

\begin{Entry}{精疲力竭}{14,10,2,14}{⽶,⽧,⼒,⽴}
  \begin{Phonetics}{精疲力竭}{jing1pi2-li4jie2}[][HSK 7-9]
    \definition{expr.}{esgotado; exausto; desgastado; descrevendo fadiga extrema e completa falta de energia}
  \end{Phonetics}
\end{Entry}

\begin{Entry}{精益求精}{14,10,7,14}{⽶,⽫,⽔,⽶}
  \begin{Phonetics}{精益求精}{jing1yi4qiu2jing1}[][HSK 7-9]
    \definition{expr.}{``Busque a excelência.''; esforçar-se pela perfeição; buscar a melhoria constante; perseguir a excelência; almejar a perfeição; já está muito bom, mas você ainda quer que fique ainda melhor; melhorar algo constantemente; continuar melhorando}
  \end{Phonetics}
\end{Entry}

\begin{Entry}{精致}{14,10}{⽶,⾄}
  \begin{Phonetics}{精致}{jing1zhi4}[][HSK 7-9]
    \definition{adj.}{fino; requintado; delicado}[我们欣赏她精致的手工艺品。===Admiramos seu trabalho artesanal requintado.]
  \end{Phonetics}
\end{Entry}

\begin{Entry}{精通}{14,10}{⽶,⾡}
  \begin{Phonetics}{精通}{jing1tong1}[][HSK 7-9]
    \definition{v.}{dominar; ser proficiente em; ter um bom domínio de; ter um profundo entendimento e conhecimento abrangente de uma área específica de estudo, tecnologia ou negócios}
  \end{Phonetics}
\end{Entry}

\begin{Entry}{精密}{14,11}{⽶,⼧}
  \begin{Phonetics}{精密}{jing1mi4}
    \definition{adj.}{preciso; preciso e meticuloso}
  \end{Phonetics}
\end{Entry}

\begin{Entry}{精彩}{14,11}{⽶,⼺}
  \begin{Phonetics}{精彩}{jing1cai3}[][HSK 3]
    \definition{adj.}{brilhante; esplêndido; maravilhoso}
  \end{Phonetics}
\end{Entry}

\begin{Entry}{精确}{14,12}{⽶,⽯}
  \begin{Phonetics}{精确}{jing1que4}[][HSK 7-9]
    \definition{adj.}{exato; preciso; acurado; muito preciso e correto}
  \end{Phonetics}
\end{Entry}

\begin{Entry}{精简}{14,13}{⽶,⽵}
  \begin{Phonetics}{精简}{jing1jian3}[][HSK 7-9]
    \definition{v.}{reduzir; simplificar; cortar; simplificar; eliminar o desnecessário e conservar o necessário}
  \end{Phonetics}
\end{Entry}

\begin{Entry}{精髓}{14,21}{⽶,⾻}
  \begin{Phonetics}{精髓}{jing1sui3}[][HSK 7-9]
    \definition{s.}{medula; medula óssea; quintessência; essência metafórica das coisas}
  \end{Phonetics}
\end{Entry}

%%%%%%%%%% 缩 %%%%%%%%%%
\subsection*{缩}\addcontentsline{loh}{figure}{缩}

\begin{Entry}{缩}{14}{⽷}
  \begin{Phonetics}{缩}{suo1}
    \definition*{s.}{Sobrenome: Suo}
    \definition{v.}{contrair; encolher | recuar; retirar-se | economizar}
  \end{Phonetics}
\end{Entry}

\begin{Entry}{缩小}{14,3}{⽷,⼩}
  \begin{Phonetics}{缩小}{suo1/xiao3}[][HSK 4]
    \definition{v.+compl.}{reduzir, estreitar, encolher;  tornar menor}
  \antonymref{放大}{fang4/da4}
  \end{Phonetics}
\end{Entry}

\begin{Entry}{缩手}{14,4}{⽷,⼿}
  \begin{Phonetics}{缩手}{suo1shou3}
    \definition{v.}{retirar a mão}
  \end{Phonetics}
\end{Entry}

\begin{Entry}{缩短}{14,12}{⽷,⽮}
  \begin{Phonetics}{缩短}{suo1/duan3}[][HSK 4]
    \definition{v.+compl.}{encurtar; reduzir; diminuir}
  \end{Phonetics}
\end{Entry}

\begin{Entry}{缩影卡片}{14,15,5,4}{⽷,⼺,⼘,⽚}
  \begin{Phonetics}{缩影卡片}{suo1ying3 ka3pian4}
    \definition{s.}{cartão em miniatura; microcartão}
  \end{Phonetics}
\end{Entry}

%%%%%%%%%% 翠 %%%%%%%%%%
\subsection*{翠}\addcontentsline{loh}{figure}{翠}

\begin{Entry}{翠}{14}{⽻}
  \begin{Phonetics}{翠}{cui4}
    \definition{adj.}{verde; verde esmeralda}
    \definition{s.}{martim-pescador | jadeíte; jade}
  \end{Phonetics}
\end{Entry}

\begin{Entry}{翠绿}{14,11}{⽻,⽷}
  \begin{Phonetics}{翠绿}{cui4lv4}[][HSK 7-9]
    \definition{adj.}{verde esmeralda; verde jade}
  \end{Phonetics}
\end{Entry}

%%%%%%%%%% 聚 %%%%%%%%%%
\subsection*{聚}\addcontentsline{loh}{figure}{聚}

\begin{Entry}{聚}{14}{⽿}
  \begin{Phonetics}{聚}{ju4}[][HSK 4]
    \definition*{s.}{Sobrenome: Ju}
    \definition{v.}{reunir-se; juntar-se}
  \end{Phonetics}
\end{Entry}

\begin{Entry}{聚会}{14,6}{⽿,⼈}
  \begin{Phonetics}{聚会}{ju4hui4}[][HSK 4]
    \definition[个,次]{s.}{reunião; encontro; confraternização; festa}
    \definition{v.}{encontrar-se; reunir-se}
  \end{Phonetics}
\end{Entry}

\begin{Entry}{聚散}{14,12}{⽿,⽁}
  \begin{Phonetics}{聚散}{ju4san4}
    \definition{s.}{juntos e separados | agregação e dissipação}
  \end{Phonetics}
\end{Entry}

\begin{Entry}{聚集}{14,12}{⽿,⾫}
  \begin{Phonetics}{聚集}{ju4ji2}[][HSK 7-9]
    \definition{v.}{reunir; juntar; coletar; reunir-se; juntar-se}
  \end{Phonetics}
\end{Entry}

\begin{Entry}{聚精会神}{14,14,6,9}{⽿,⽶,⼈,⽰}
  \begin{Phonetics}{聚精会神}{ju4jing1-hui4shen2}[][HSK 7-9]
    \definition{expr.}{concentrado; concentrar a atenção; focar a mente; estar absorto em; estar profundamente concentrado; estar totalmente concentrado}
  \end{Phonetics}
\end{Entry}

%%%%%%%%%% 腐 %%%%%%%%%%
\subsection*{腐}\addcontentsline{loh}{figure}{腐}

\begin{Entry}{腐}{14}{⾁}
  \begin{Phonetics}{腐}{fu3}
    \definition{adj.}{podre; obsoleto; corrupto | corroído; pútrido}
    \definition{s.}{tofu}
    \definition{v.}{apodrecer; corroer; estragar; decair}
  \end{Phonetics}
\end{Entry}

\begin{Entry}{腐化}{14,4}{⾁,⼔}
  \begin{Phonetics}{腐化}{fu3hua4}[][HSK 7-9]
    \definition{adj.}{degenerado; corrupto, dissoluto ou depravado; desmoralizado; decadente}
    \definition{v.}{decompor; apodrecer; tornar-se pútrido | quebrar; corroer}
  \end{Phonetics}
\end{Entry}

\begin{Entry}{腐朽}{14,6}{⾁,⽊}
  \begin{Phonetics}{腐朽}{fu3xiu3}[][HSK 7-9]
    \definition{adj.}{decaído; decadente; degenerado; uma metáfora para as ideias ultrapassadas das pessoas ou para a moral social corrupta}
    \definition{v.}{apodrecer; decair; apodrecimento e deterioração da madeira e outros materiais fibrosos}
  \end{Phonetics}
\end{Entry}

\begin{Entry}{腐败}{14,8}{⾁,⾒}
  \begin{Phonetics}{腐败}{fu3bai4}[][HSK 7-9]
    \definition{adj.}{(ideia) corrupto; decadente; (pensamento) obsoleto; (comportamento) degenerado | (sistema, organização, instituição, medida, etc.) corrupto}
    \definition{s.}{deterioração; podridão}
    \definition{v.}{apodrecer; decair}
  \end{Phonetics}
\end{Entry}

\begin{Entry}{腐烂}{14,9}{⾁,⽕}
  \begin{Phonetics}{腐烂}{fu3lan4}[][HSK 7-9]
    \definition{adj.}{corrupto; depravado | (pensamento) obsoleto; (comportamento) degenerado}
    \definition{v.}{apodrecer; decompor; tornar"-se pútrido}
  \end{Phonetics}
\end{Entry}

\begin{Entry}{腐蚀}{14,9}{⾁,⾷}
  \begin{Phonetics}{腐蚀}{fu3shi2}[][HSK 7-9]
    \definition{v.}{corroer; destruir gradualmente um objeto por meio de reações químicas | corroer; corromper (pensamentos e comportamentos)}
  \end{Phonetics}
\end{Entry}

%%%%%%%%%% 膜 %%%%%%%%%%
\subsection*{膜}\addcontentsline{loh}{figure}{膜}

\begin{Entry}{膜}{14}{⾁}
  \begin{Phonetics}{膜}{mo2}[][HSK 6]
    \definition[张]{s.}{membrana | filme; revestimento fino}
  \end{Phonetics}
\end{Entry}

\begin{Entry}{膜拜}{14,9}{⾁,⼿}
  \begin{Phonetics}{膜拜}{mo2bai4}
    \definition{v.}{ajoelhar-se e se curvar com as mãos unidas no nível da testa | ter ou mostrar sentimentos fortes de respeito e admiração por um deus}
  \end{Phonetics}
\end{Entry}

%%%%%%%%%% 舞 %%%%%%%%%%
\subsection*{舞}\addcontentsline{loh}{figure}{舞}

\begin{Entry}{舞}{14}{⾇}
  \begin{Phonetics}{舞}{wu3}[][HSK 5]
    \definition[支,段,个]{s.}{dança | palco; metáfora do domínio das atividades sociais}
    \definition{v.}{mover-se como numa dança | dançar com algo nas mãos; brincar com | florescer; empunhar; brandir | esvoaçar | fazer malabarismos; brincar com}
  \end{Phonetics}
\end{Entry}

\begin{Entry}{舞厅}{14,4}{⾇,⼚}
  \begin{Phonetics}{舞厅}{wu3ting1}
    \definition[间]{s.}{salão de dança | salão de baile}
  \end{Phonetics}
\end{Entry}

\begin{Entry}{舞厅舞}{14,4,14}{⾇,⼚,⾇}
  \begin{Phonetics}{舞厅舞}{wu3ting1wu3}
    \definition{s.}{dança de salão}
  \end{Phonetics}
\end{Entry}

\begin{Entry}{舞台}{14,5}{⾇,⼝}
  \begin{Phonetics}{舞台}{wu3tai2}[][HSK 3]
    \definition[个]{s.}{palco; plataforma elevada usada exclusivamente para apresentações artísticas, geralmente localizada na parte frontal de teatros e auditórios | palco; metáfora do campo das atividades sociais}
  \end{Phonetics}
\end{Entry}

\begin{Entry}{舞会}{14,6}{⾇,⼈}
  \begin{Phonetics}{舞会}{wu3hui4}
    \definition{s.}{baile}
  \end{Phonetics}
\end{Entry}

\begin{Entry}{舞会舞}{14,6,14}{⾇,⼈,⾇}
  \begin{Phonetics}{舞会舞}{wu3hui4wu3}
    \definition{s.}{baile}
  \end{Phonetics}
\end{Entry}

\begin{Entry}{舞抃}{14,7}{⾇,⼿}
  \begin{Phonetics}{舞抃}{wu3bian4}
    \definition{s.}{dançar por prazer}
  \end{Phonetics}
\end{Entry}

\begin{Entry}{舞蹈}{14,17}{⾇,⾜}
  \begin{Phonetics}{舞蹈}{wu3dao3}[][HSK 6]
    \definition[段,支,场,个]{s.}{dança; uma forma de arte que usa movimentos rítmicos como principal meio de expressão, podendo expressar a vida, os pensamentos e os sentimentos das pessoas, geralmente acompanhada de música}
    \definition{v.}{dançar}
  \end{Phonetics}
\end{Entry}

%%%%%%%%%% 蔓 %%%%%%%%%%
\subsection*{蔓}\addcontentsline{loh}{figure}{蔓}

\begin{Entry}{蔓}{14}{⾋}
  \begin{Phonetics}{蔓}{man2}
    \definition{s.}{couve-chinesa | nabo}
  \end{Phonetics}
  \begin{Phonetics}{蔓}{man4}
    \definition{s.}{uma videira com gavinhas; caule fino que não consegue ficar em pé}
    \definition{v.}{rastejar; espalhar; estender}
  \end{Phonetics}
  \begin{Phonetics}{蔓}{wan4}
    \definition*{s.}{Sobrenome: Wan}
    \definition{s.}{uma videira com gavinhas; caule fino que não consegue ficar em pé}
  \end{Phonetics}
\end{Entry}

\begin{Entry}{蔓延}{14,6}{⾋,⼵}
  \begin{Phonetics}{蔓延}{man4yan2}[][HSK 7-9]
    \definition{v.}{espalhar; esticar; estender | infestar; espalhar; essa metáfora descreve coisas que se estendem e se expandem para fora como trepadeiras rastejantes}
  \end{Phonetics}
\end{Entry}

\begin{Entry}{蔓草}{14,9}{⾋,⾋}
  \begin{Phonetics}{蔓草}{man4cao3}
    \definition{s.}{videira | trepadeira}
  \end{Phonetics}
\end{Entry}

%%%%%%%%%% 蜘 %%%%%%%%%%
\subsection*{蜘}\addcontentsline{loh}{figure}{蜘}

\begin{Entry}{蜘}{14}{⾍}
  \begin{Phonetics}{蜘}{zhi1}
    \definition[只]{s.}{aranha}
  \seealsoref{蜘蛛}{zhi1zhu1}
  \end{Phonetics}
\end{Entry}

\begin{Entry}{蜘蛛}{14,12}{⾍,⾍}
  \begin{Phonetics}{蜘蛛}{zhi1zhu1}
    \definition{s.}{aranha}
  \end{Phonetics}
\end{Entry}

\begin{Entry}{蜘蛛网}{14,12,6}{⾍,⾍,⽹}
  \begin{Phonetics}{蜘蛛网}{zhi1zhu1 wang3}
    \definition{s.}{teia de aranha}
  \end{Phonetics}
\end{Entry}

%%%%%%%%%% 蜜 %%%%%%%%%%
\subsection*{蜜}\addcontentsline{loh}{figure}{蜜}

\begin{Entry}{蜜}{14}{⾍}
  \begin{Phonetics}{蜜}{mi4}[][HSK 7-9]
    \definition{adj.}{melado; doce}
    \definition{s.}{mel | semelhante ao mel | coisas parecidas com mel; melaço}
  \end{Phonetics}
\end{Entry}

\begin{Entry}{蜜月}{14,4}{⾍,⽉}
  \begin{Phonetics}{蜜月}{mi4yue4}[][HSK 7-9]
    \definition{s.}{lua de mel; o primeiro mês após o casamento}
  \end{Phonetics}
\end{Entry}

\begin{Entry}{蜜桃}{14,10}{⾍,⽊}
  \begin{Phonetics}{蜜桃}{mi4tao2}
    \definition{s.}{nectarina | pêssego | pêssego suculento}
  \end{Phonetics}
\end{Entry}

\begin{Entry}{蜜蜂}{14,13}{⾍,⾍}
  \begin{Phonetics}{蜜蜂}{mi4feng1}[][HSK 7-9]
    \definition[只,群,箱,窝]{s.}{abelha; abelha-melífera}
  \end{Phonetics}
\end{Entry}

%%%%%%%%%% 蜡 %%%%%%%%%%
\subsection*{蜡}\addcontentsline{loh}{figure}{蜡}

\begin{Entry}{蜡}{14}{⾍}
  \begin{Phonetics}{蜡}{la4}[][HSK 7-9]
    \definition{s.}{cera; óleos produzidos por animais, minerais ou plantas | vela}
  \end{Phonetics}
  \begin{Phonetics}{蜡}{zha4}
    \definition{s.}{uma antiga cerimônia de sacrifício de fim de ano}
  \end{Phonetics}
\end{Entry}

\begin{Entry}{蜡烛}{14,10}{⾍,⽕}
  \begin{Phonetics}{蜡烛}{la4zhu2}[][HSK 7-9]
    \definition[根,支,包]{s.}{vela; círio; peça cilíndrica, geralmente de cera, que possui um pavio e se utiliza como iluminação}
  \end{Phonetics}
\end{Entry}

%%%%%%%%%% 蜥 %%%%%%%%%%
\subsection*{蜥}\addcontentsline{loh}{figure}{蜥}

\begin{Entry}{蜥}{14}{⾍}
  \begin{Phonetics}{蜥}{xi1}
    \definition{s.}{lagarto}
  \end{Phonetics}
\end{Entry}

\begin{Entry}{蜥易}{14,8}{⾍,⽇}
  \begin{Phonetics}{蜥易}{xi1yi4}
    \variantof{蜥蜴}
  \end{Phonetics}
\end{Entry}

\begin{Entry}{蜥蜴}{14,14}{⾍,⾍}
  \begin{Phonetics}{蜥蜴}{xi1yi4}
    \definition{s.}{lagarto}
  \end{Phonetics}
\end{Entry}

%%%%%%%%%% 蜻 %%%%%%%%%%
\subsection*{蜻}\addcontentsline{loh}{figure}{蜻}

\begin{Entry}{蜻}{14}{⾍}
  \begin{Phonetics}{蜻}{qing1}
    \definition[只]{s.}{libélula, 蜻蜓}
  \seealsoref{蜻蜓}{qing1ting2}
  \end{Phonetics}
\end{Entry}

\begin{Entry}{蜻蜓}{14,12}{⾍,⾍}
  \begin{Phonetics}{蜻蜓}{qing1ting2}
    \definition{s.}{libélula}
  \end{Phonetics}
\end{Entry}

\begin{Entry}{蜻蝏}{14,15}{⾍,⾍}
  \begin{Phonetics}{蜻蝏}{qing1ting2}
    \variantof{蜻蜓}
  \end{Phonetics}
\end{Entry}

%%%%%%%%%% 蝉 %%%%%%%%%%
\subsection*{蝉}\addcontentsline{loh}{figure}{蝉}

\begin{Entry}{蝉}{14}{⾍}
  \begin{Phonetics}{蝉}{chan2}
    \definition[只,个]{s.}{cigarra}
  \seealsoref{知了}{zhi1liao3}
  \end{Phonetics}
\end{Entry}

%%%%%%%%%% 裹 %%%%%%%%%%
\subsection*{裹}\addcontentsline{loh}{figure}{裹}

\begin{Entry}{裹}{14}{⾐}
  \begin{Phonetics}{裹}{guo3}[][HSK 7-9]
    \definition{s.}{pacote; encomenda}
    \definition{v.}{amarrar; embrulhar; envolver | levar embora; varrer com violência | Dialeto: sugar (leite) | pressionar a servir; fugir com (algo)}
  \end{Phonetics}
\end{Entry}

%%%%%%%%%% 褐 %%%%%%%%%%
\subsection*{褐}\addcontentsline{loh}{figure}{褐}

\begin{Entry}{褐}{14}{⾐}
  \begin{Phonetics}{褐}{he4}
    \definition{adj.}{marrom; castanho; pardo}
    \definition{s.}{pano de cânhamo grosso}
  \end{Phonetics}
\end{Entry}

\begin{Entry}{褐色}{14,6}{⾐,⾊}
  \begin{Phonetics}{褐色}{he4 se4}
    \definition{s.}{cor marrom}
  \end{Phonetics}
\end{Entry}

%%%%%%%%%% 褡 %%%%%%%%%%
\subsection*{褡}\addcontentsline{loh}{figure}{褡}

\begin{Entry}{褡}{14}{⾐}
  \begin{Phonetics}{褡}{da1}
    \definition{s.}{bolsa; malote; algibeira | jaqueta sem mangas}
  \end{Phonetics}
\end{Entry}

%%%%%%%%%% 谱 %%%%%%%%%%
\subsection*{谱}\addcontentsline{loh}{figure}{谱}

\begin{Entry}{谱}{14}{⾔}
  \begin{Phonetics}{谱}{pu3}[][HSK 7-9]
    \definition{s.}{registro ou documento de fácil consulta (na forma de gráficos, tabelas, listas, etc.); cronologia; um livro compilado para consulta, organizado de acordo com a categoria ou sistema do assunto e utilizando tabelas ou outros formatos claros | manual; guia; formatos ou diagramas que podem ser usados para orientar a prática | partitura; partitura musical; partitura para música}
    \definition{v.}{compor música | exibir-se; mostrar-se arrogante}
  \end{Phonetics}
\end{Entry}

%%%%%%%%%% 豪 %%%%%%%%%%
\subsection*{豪}\addcontentsline{loh}{figure}{豪}

\begin{Entry}{豪}{14}{⾗}
  \begin{Phonetics}{豪}{hao2}
    \definition*{s.}{Sobrenome: Hao}
    \definition{adj.}{direto; irrestrito; ousado | despótico; intimidador | rico e poderoso}
    \definition{s.}{pessoa com poderes ou dons extraordinários}
  \end{Phonetics}
\end{Entry}

\begin{Entry}{豪华}{14,6}{⾗,⼗}
  \begin{Phonetics}{豪华}{hao2hua2}[][HSK 7-9]
    \definition{adj.}{luxo; luxuoso; (edifício, equipamento ou decoração) magnífico; muito lindo}
  \end{Phonetics}
\end{Entry}

%%%%%%%%%% 赚 %%%%%%%%%%
\subsection*{赚}\addcontentsline{loh}{figure}{赚}

\begin{Entry}{赚}{14}{⾙}
  \begin{Phonetics}{赚}{zhuan4}[][HSK 6]
    \definition{s.}{lucro}
    \definition{v.}{ganhar (dinheiro); obter lucro com o negócio}
  \antonymref{赔}{pei2}
  \end{Phonetics}
\end{Entry}

\begin{Entry}{赚钱}{14,10}{⾙,⾦}
  \begin{Phonetics}{赚钱}{zhuan4 qian2}[][HSK 6]
    \definition{v.}{ganhar dinheiro; obter lucro ou recompensa}
  \end{Phonetics}
\end{Entry}

%%%%%%%%%% 赛 %%%%%%%%%%
\subsection*{赛}\addcontentsline{loh}{figure}{赛}

\begin{Entry}{赛}{14}{⾙}
  \begin{Phonetics}{赛}{sai4}[][HSK 6]
    \definition*{s.}{Sobrenome: Sai}
    \definition{s.}{jogo; partida; competição | sacrifício; cerimônia de sacrifício; antigamente, sacrifícios eram feitos para agradecer aos deuses por suas dádivas}
    \definition{v.}{ter uma competição (comparando alto e baixo, forte e fraco) | superar; ser comparável a; comparar com}
  \end{Phonetics}
\end{Entry}

\begin{Entry}{赛车}{14,4}{⾙,⾞}
  \begin{Phonetics}{赛车}{sai4che1}[][HSK 7-9]
    \definition{s.}{veículo de corrida; carro de corrida; bicicletas, motocicletas ou carros de corrida | corridas de carros}
    \definition{v.}{correr; disputar uma corrida}
  \end{Phonetics}
\end{Entry}

\begin{Entry}{赛场}{14,6}{⾙,⼟}
  \begin{Phonetics}{赛场}{sai4chang3}[][HSK 6]
    \definition{s.}{local de competição; arena; ringue; terreno | campo (para competição de atletismo) | pista de corrida}
  \end{Phonetics}
\end{Entry}

\begin{Entry}{赛跑}{14,12}{⾙,⾜}
  \begin{Phonetics}{赛跑}{sai4pao3}[][HSK 7-9]
    \definition{v.}{correr; disputar uma corrida; esportes que testam a velocidade de corrida incluem provas de curta, média, longa e ultra-longa distância, além de corridas com barreiras, revezamentos, corridas com obstáculos e corridas de cross-country}
  \end{Phonetics}
\end{Entry}

%%%%%%%%%% 赫 %%%%%%%%%%
\subsection*{赫}\addcontentsline{loh}{figure}{赫}

\begin{Entry}{赫}{14}{⾚}
  \begin{Phonetics}{赫}{he4}
    \definition*{s.}{Sobrenome: He}
    \definition{adj.}{conspícuo; grandioso | vermelho brilhante e flamejante; vermelho como fogo}
    \definition{clas.}{Hz, hertz; abreviação de 赫兹}
  \seealsoref{赫兹}{he4zi1}
  \end{Phonetics}
\end{Entry}

\begin{Entry}{赫兹}{14,9}{⾚,⼋}
  \begin{Phonetics}{赫兹}{he4zi1}
    \definition{s.}{hertz (Hz), unidade de frequência}
    \definition{s.}{Heinrich Hertz (1857-1894), físico e meteorologista alemão, pioneiro da radiação eletromagnética}
  \end{Phonetics}
\end{Entry}

\begin{Entry}{赫然}{14,12}{⾚,⽕}
  \begin{Phonetics}{赫然}{he4ran2}[][HSK 7-9]
    \definition{adj.}{inesperado e chocante/impressionante; descreve algo que é muito marcante ou surpreendente | (raiva, etc.) terrível; violento; descreve o olhar de raiva | grande; eminente; florescente; excepcional; descreve a aparência de ser proeminente}
  \end{Phonetics}
\end{Entry}

%%%%%%%%%% 辗 %%%%%%%%%%
\subsection*{辗}\addcontentsline{loh}{figure}{辗}

\begin{Entry}{辗}{14}{⾞}
  \begin{Phonetics}{辗}{zhan3}
    \definition{v.}{(arcaico) virar | (arcaico) rolar para o lado | (arcaico) virar a metade}
  \end{Phonetics}
\end{Entry}

%%%%%%%%%% 辣 %%%%%%%%%%
\subsection*{辣}\addcontentsline{loh}{figure}{辣}

\begin{Entry}{辣}{14}{⾟}
  \begin{Phonetics}{辣}{la4}[][HSK 4]
    \definition{adj.}{apimentado; picante; pungente; quente | cruel; implacável; venenoso; vicioso}
    \definition{v.}{queimar; picar; formigar; ter uma irritação picante (boca, nariz ou olhos)}
  \end{Phonetics}
\end{Entry}

\begin{Entry}{辣椒}{14,12}{⾟,⽊}
  \begin{Phonetics}{辣椒}{la4jiao1}[][HSK 7-9]
    \definition[颗,把,袋,种]{s.}{pimenta; pimenta-malagueta; pimenta caiena; pimentão; pimenta vermelha}
  \end{Phonetics}
\end{Entry}

%%%%%%%%%% 遭 %%%%%%%%%%
\subsection*{遭}\addcontentsline{loh}{figure}{遭}

\begin{Entry}{遭}{14}{⾡}
  \begin{Phonetics}{遭}{zao1}
    \definition{clas.}{tempo; vez; ocasião | rodadas}
    \definition{v.}{encontrar-se com (desastre, infortúnio, etc.); sofrer}
  \end{Phonetics}
\end{Entry}

\begin{Entry}{遭到}{14,8}{⾡,⼑}
  \begin{Phonetics}{遭到}{zao1dao4}[][HSK 6]
    \definition{v.}{sofrer; ser rejeitado; receber crítica; significa sofrer infortúnio ou dano}[我们遭到意外事故。===Nós sofremos um acidente.]
  \end{Phonetics}
\end{Entry}

\begin{Entry}{遭受}{14,8}{⾡,⼜}
  \begin{Phonetics}{遭受}{zao1shou4}[][HSK 6]
    \definition{v.}{sofrer; aguentar; ser submetido a; encontrar ou vivenciar coisas dolorosas que você não quer que aconteçam}
  \end{Phonetics}
\end{Entry}

\begin{Entry}{遭遇}{14,12}{⾡,⾡}
  \begin{Phonetics}{遭遇}{zao1yu4}[][HSK 6]
    \definition[场,次,种,段]{s.}{sorte (difícil); experiência (amarga); encontrando coisas ruins}
    \definition{v.}{encontrar; encontrar-se com; esbarrar em; encontros inesperados com pessoas ou coisas que não são boas para você}
  \end{Phonetics}
\end{Entry}

%%%%%%%%%% 酷 %%%%%%%%%%
\subsection*{酷}\addcontentsline{loh}{figure}{酷}

\begin{Entry}{酷}{14}{⾣}
  \begin{Phonetics}{酷}{ku4}[][HSK 6]
    \definition{adj.}{cruel; opressivo | feroz; escaldante | brutal | Empréstimo linguístico: \emph{cool}; legal; excelente; moderno; ótimo | elegante e sóbrio; gracioso e severo}
    \definition{adv.}{muito; extremamente}
  \end{Phonetics}
\end{Entry}

\begin{Entry}{酷似}{14,6}{⾣,⼈}
  \begin{Phonetics}{酷似}{ku4si4}[][HSK 7-9]
    \definition{v.}{ser a própria imagem de; ser exatamente igual a; apresentar forte semelhança com}
  \end{Phonetics}
\end{Entry}

\begin{Entry}{酷斯拉}{14,12,8}{⾣,⽄,⼿}
  \begin{Phonetics}{酷斯拉}{ku4si1la1}
    \definition*{s.}{Godzilla. do Japonês Gojira, ゴジラ}
  \seealsoref{哥斯拉}{ge1si1la1}
  \end{Phonetics}
\end{Entry}

%%%%%%%%%% 酸 %%%%%%%%%%
\subsection*{酸}\addcontentsline{loh}{figure}{酸}

\begin{Entry}{酸}{14}{⾣}
  \begin{Phonetics}{酸}{suan1}[][HSK 4]
    \definition{adj.}{azedo; ácido | aflito; angustiado; doente do coração | pedante; descreve uma pessoa que finge ser culta e também descreve uma pessoa que é muito inflexível com suas próprias ideias e não está disposta a mudá-las para atender às exigências da época, é usado principalmente para satirizar intelectuais que fingem ser capazes de escrever poemas e artigos | ciumento; invejoso; sentimentos desconfortáveis porque outra pessoa é melhor do que você e, em geral, também apresenta comportamento hostil}
    \definition{s.}{ácido; produto químico que tem um sabor ácido quando misturado com água}
    \definition{v.}{estar dolorido (devido à fadiga ou doença); descreve a sensação de não ter força muscular e um pouco de dor por estar doente ou muito cansado}
  \end{Phonetics}
\end{Entry}

\begin{Entry}{酸奶}{14,5}{⾣,⼥}
  \begin{Phonetics}{酸奶}{suan1nai3}[][HSK 4]
    \definition[瓶,杯,盒,袋]{s.}{iogurte; produto lácteo fermentado por bactérias de ácido láctico}
  \end{Phonetics}
\end{Entry}

\begin{Entry}{酸甜苦辣}{14,11,8,14}{⾣,⽢,⾋,⾟}
  \begin{Phonetics}{酸甜苦辣}{suan1-tian2-ku3-la4}[][HSK 5]
    \definition{expr.}{os altos e baixos da vida; as experiências agridoces da vida; os aspectos doces, azedos, amargos e picantes da vida; refere"-se a todos os tipos de sabores, como metáfora para experiências diversas, como felicidade, sofrimento, etc.; azedo, doce, amargo, picante (alegrias e tristezas da vida)}
  \end{Phonetics}
\end{Entry}

\begin{Entry}{酸辣汤}{14,14,6}{⾣,⾟,⽔}
  \begin{Phonetics}{酸辣汤}{suan1la4tang1}
    \definition{s.}{sopa avinagrada e picante (prato)}
  \end{Phonetics}
\end{Entry}

%%%%%%%%%% 酿 %%%%%%%%%%
\subsection*{酿}\addcontentsline{loh}{figure}{酿}

\begin{Entry}{酿}{14}{⾣}
  \begin{Phonetics}{酿}{niang2}
    \definition{v.}{fermentar; preparar (vinho de arroz)}
  \end{Phonetics}
  \begin{Phonetics}{酿}{niang4}
    \definition{s.}{vinho}
    \definition{v.}{fazer (vinho); fabricar (cerveja) | produzir (mel) | levar a; resultar em; formar gradualmente}
  \end{Phonetics}
\end{Entry}

\begin{Entry}{酿造}{14,10}{⾣,⾡}
  \begin{Phonetics}{酿造}{niang4zao4}[][HSK 7-9]
    \definition{v.}{fabricar (cerveja, etc.); produzir (vinho, vinagre, etc.); fabricar utilizando fermentação}
  \end{Phonetics}
\end{Entry}

%%%%%%%%%% 锺 %%%%%%%%%%
\subsection*{锺}\addcontentsline{loh}{figure}{锺}

\begin{Entry}{锺}{14}{⾦}
  \begin{Phonetics}{锺}{zhong1}
    \variantof{钟}
  \end{Phonetics}
\end{Entry}

%%%%%%%%%% 锻 %%%%%%%%%%
\subsection*{锻}\addcontentsline{loh}{figure}{锻}

\begin{Entry}{锻}{14}{⾦}
  \begin{Phonetics}{锻}{duan4}
    \definition{v.}{forjar; moldar}
  \end{Phonetics}
\end{Entry}

\begin{Entry}{锻炼}{14,9}{⾦,⽕}
  \begin{Phonetics}{锻炼}{duan4lian4}[][HSK 4]
    \definition{v.}{exercitar"-se; fazer (ou fazer) exercícios; submeter"-se a treinamento físico; fortalecer o corpo por meio do esporte | fortalecer; endurecer; aprimorar as habilidades de trabalho e de vida por meio de trabalho e outras atividades | forjar ou moldar metal para torná"-lo mais refinado; refere"-se à transformação de materiais metálicos em objetos de determinada forma e tamanho por meio de aquecimento, batimento, prensagem etc.}
  \end{Phonetics}
\end{Entry}

%%%%%%%%%% 镀 %%%%%%%%%%
\subsection*{镀}\addcontentsline{loh}{figure}{镀}

\begin{Entry}{镀}{14}{⾦}
  \begin{Phonetics}{镀}{du4}
    \definition{v.}{cobrir ou revestir (com um metal)}
  \end{Phonetics}
\end{Entry}

\begin{Entry}{镀金}{14,8}{⾦,⾦}
  \begin{Phonetics}{镀金}{du4jin1}
    \definition{v.}{banhar a ouro | dourar | (figurativo) fazer algo muito comum parecer especial}
  \end{Phonetics}
\end{Entry}

%%%%%%%%%% 隧 %%%%%%%%%%
\subsection*{隧}\addcontentsline{loh}{figure}{隧}

\begin{Entry}{隧}{14}{⾩}
  \begin{Phonetics}{隧}{sui4}
    \definition{s.}{túnel; passagem subterrânea | estrada | subúrbios; áreas suburbanas}
    \definition{v.}{virar}
  \end{Phonetics}
\end{Entry}

\begin{Entry}{隧道}{14,12}{⾩,⾡}
  \begin{Phonetics}{隧道}{sui4dao4}
    \definition{s.}{túnel}
  \end{Phonetics}
\end{Entry}

%%%%%%%%%% 需 %%%%%%%%%%
\subsection*{需}\addcontentsline{loh}{figure}{需}

\begin{Entry}{需}{14}{⾬}
  \begin{Phonetics}{需}{xu1}
    \definition*{s.}{Sobrenome: Xu}
    \definition{s.}{necessidades; bens de primeira necessidade}
    \definition{v.}{precisar; querer; exigir}
  \end{Phonetics}
\end{Entry}

\begin{Entry}{需求}{14,7}{⾬,⽔}
  \begin{Phonetics}{需求}{xu1qiu2}[][HSK 3]
    \definition[种]{s.}{necessidades; demanda; exigência; solicitações decorrentes de necessidades}
  \end{Phonetics}
\end{Entry}

\begin{Entry}{需要}{14,9}{⾬,⾑}
  \begin{Phonetics}{需要}{xu1yao4}[][HSK 3]
    \definition[种]{s.}{necessidade; desejo ou exigência em relação a algo}
    \definition{v.}{precisar; querer; exigir; demandar; solicitar}
  \end{Phonetics}
\end{Entry}

%%%%%%%%%% 静 %%%%%%%%%%
\subsection*{静}\addcontentsline{loh}{figure}{静}

\begin{Entry}{静}{14}{⾭}
  \begin{Phonetics}{静}{jing4}[][HSK 3]
    \definition*{s.}{Sobrenome: Jing}
    \definition{adj.}{tranquilo;  sossegado; calmo; imóvel | silencioso; quieto; sem emitir nenhum som | calmo, sereno; serenidade; (interior) paz}
    \definition{v.}{acalmar-se; aquietar-se; tranquilizar (o coração)}
  \end{Phonetics}
\end{Entry}

\begin{Entry}{静止}{14,4}{⾭,⽌}
  \begin{Phonetics}{静止}{jing4zhi3}[][HSK 7-9]
    \definition{adj.}{estático; imóvel; parado; estacionário}
  \end{Phonetics}
\end{Entry}

%%%%%%%%%% 颗 %%%%%%%%%%
\subsection*{颗}\addcontentsline{loh}{figure}{颗}

\begin{Entry}{颗}{14}{⾴}
  \begin{Phonetics}{颗}{ke1}[][HSK 5]
    \definition{clas.}{usado para grãos, pérolas, dentes, corações, satelites, pequenas esferas, etc.}
    \definition{s.}{grão; partícula; pequenas coisas redondas}
  \end{Phonetics}
\end{Entry}

%%%%%%%%%% 馒 %%%%%%%%%%
\subsection*{馒}\addcontentsline{loh}{figure}{馒}

\begin{Entry}{馒}{14}{⾷}
  \begin{Phonetics}{馒}{man2}
    \definition{s.}{pão cozido no vapor}
  \end{Phonetics}
\end{Entry}

\begin{Entry}{馒头}{14,5}{⾷,⼤}
  \begin{Phonetics}{馒头}{man2tou5}[][HSK 6]
    \definition[个,锅,屉,筐]{s.}{pão cozido no vapor; um alimento cozido no vapor feito de farinha fermentada, geralmente redondo na parte superior e plano na parte inferior, sem recheio}
  \end{Phonetics}
\end{Entry}

%%%%%%%%%% 魄 %%%%%%%%%%
\subsection*{魄}\addcontentsline{loh}{figure}{魄}

\begin{Entry}{魄}{14}{⿁}
  \begin{Phonetics}{魄}{bo2}
    \definition{adj.}{desanimado; aflito; abatido}
  \seealsoref{落魄}{luo4bo2}
  \end{Phonetics}
  \begin{Phonetics}{魄}{po4}
    \definition{s.}{alma | vigor; espírito; coragem; energia}
  \end{Phonetics}
  \begin{Phonetics}{魄}{tuo4}
    \definition{adj.}{desanimado; sem ânimo; mentalmente abatido; outra pronúncia de 魄 em 落魄}
  \seealsoref{落魄}{luo4po4}
  \end{Phonetics}
\end{Entry}

\begin{Entry}{魄力}{14,2}{⿁,⼒}
  \begin{Phonetics}{魄力}{po4li4}[][HSK 7-9]
    \definition[种]{s.}{coragem; ousadia; audácia e resolução; refere"-se à coragem e à determinação com que alguém lida com as situações}
  \end{Phonetics}
\end{Entry}

%%%%%%%%%% 魅 %%%%%%%%%%
\subsection*{魅}\addcontentsline{loh}{figure}{魅}

\begin{Entry}{魅}{14}{⿁}
  \begin{Phonetics}{魅}{mei4}
    \definition{s.}{espírito maligno; demônio | \emph{goblin}; trasgo; gnomo; duende maléfico}
    \definition{v.}{atormentar; cativar}
  \end{Phonetics}
\end{Entry}

\begin{Entry}{魅力}{14,2}{⿁,⼒}
  \begin{Phonetics}{魅力}{mei4li4}[][HSK 7-9]
    \definition[种]{s.}{charme; feitiço; glamour; bruxaria; carisma; feitiçaria; encanto; fascínio; encantamento; o poder de atrair e motivar pessoas}
  \end{Phonetics}
\end{Entry}

%%%%%%%%%% 鲜 %%%%%%%%%%
\subsection*{鲜}\addcontentsline{loh}{figure}{鲜}

\begin{Entry}{鲜}{14}{⿂}
  \begin{Phonetics}{鲜}{xian1}[][HSK 4]
    \definition*{s.}{Sobrenome: Xian}
    \definition{adj.}{fresco; novo; fresco (experiência, comida etc.) |brilhante; de cores vivas | saboroso; delicioso | exuberante; luxuriante}
    \definition{s.}{aves e animais recém-abatidos; vegetais recém-colhidos; frutas, etc. | alimentos aquáticos; geralmente, peixes vivos, camarões, etc., para alimentação}
  \end{Phonetics}
  \begin{Phonetics}{鲜}{xian3}
    \definition{adj.}{raro; pouco; pequeno}
    \definition{adv.}{raramente}
  \end{Phonetics}
\end{Entry}

\begin{Entry}{鲜花}{14,7}{⿂,⾋}
  \begin{Phonetics}{鲜花}{xian1hua1}[][HSK 4]
    \definition[朵,束,支]{s.}{flor; flores frescas; flores bonitas e frescas}
  \end{Phonetics}
\end{Entry}

\begin{Entry}{鲜明}{14,8}{⿂,⽇}
  \begin{Phonetics}{鲜明}{xian1ming2}[][HSK 4]
    \definition{adj.}{brilhante (cor) | distinto; bem definido; nítido; claro; característico}
  \end{Phonetics}
\end{Entry}

\begin{Entry}{鲜艳}{14,10}{⿂,⾊}
  \begin{Phonetics}{鲜艳}{xian1yan4}[][HSK 5]
    \definition{adj.}{de cores alegres; de cores brilhantes}
  \end{Phonetics}
\end{Entry}

%%%%%%%%%% 鼻 %%%%%%%%%%
\subsection*{鼻}\addcontentsline{loh}{figure}{鼻}

\begin{Entry}{鼻}{14}{⿐}[Kangxi 209]
  \begin{Phonetics}{鼻}{bi2}
    \definition{s.}{nariz}
  \end{Phonetics}
\end{Entry}

\begin{Entry}{鼻子}{14,3}{⿐,⼦}
  \begin{Phonetics}{鼻子}{bi2zi5}[][HSK 5]
    \definition[个,只]{s.}{nariz; órgão da face, responsável pela respiração e pelo olfato}
  \end{Phonetics}
\end{Entry}

\begin{Entry}{鼻涕}{14,10}{⿐,⽔}
  \begin{Phonetics}{鼻涕}{bi2ti4}[][HSK 7-9]
    \definition[些,点]{s.}{ranho; muco nasal; secreção nasal; fluido secretado pela mucosa nasal}
  \end{Phonetics}
\end{Entry}

%%%%% EOF %%%%%

