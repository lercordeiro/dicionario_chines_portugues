%%%%%%%%%%%%%%%%%%%%%%%%
% Entrada de um verbete:
%
% \begin{verbete}{hanzi}[antes]{pronúncia}[depois]{traços}
%   \significado[classificadores]{termo_gramatical}{lista_de_significados}
%   \veja{hanzi}{pronúncia}
% \end{verbete}
%
% Podem haver tantos comandos '\significado' e '\veja' quantos necessários
%  dentro de cada bloco de verbete. 
%
% Preferivelmente usar um comando '\significado' para cada termo gramatical ou
%  se houver classificadores específicos para cada significado da palavra.
%
% Existe o comando '\significado*' que deve ser utilizado só e somente só para
%  substantivos prórios.
%
% Podem haver vários comandos '\veja', é aconselhável usar um comando para cada
%  par de hanzy+pronúncia, o sistema cuida do texto e de colocar o número da
%  página automaticamente.
%
%
% Para o environment verbetes
%   Campos Obrigatórios
%     hanzi: Hanzi da palavra
%     pronúncia: pinyin no formato: "ma1", "ma2", "ma3", "ma4", "ma5" 
%                os números correspondem aos tons 
%                "ma" == "ma0" == "ma5" (utilizar "5" para tom neutro)
%     traços: número de traços dos caracteres separados por ';'
%   Campos Opcionais
%     antes: texto adicionado antes do texto da pronúncia
%            se adiciona um '\\' (pula linha) caso a pronúncia
%            fique por cima do texto do número de caracteres,
%            se usado o texto da pronúncia vai para baixo do Hanzi
%     depois: texto adicionado depois do texto da pronúncia,
%             se usado o texto da pronúncia vai para baixo do Hanzi
%
%
% Para o comando \significado:
%   Campos Obrigatórios
%     termo_gramatical: abreviação dos termos, como 's.', 'v.', 'adj.',
%                       'adv.', etc, conforme tabela acima
%     lista_de_significados: significados da palavra separados por ';' ou ','.
%                            Quando se usa ';' os significados são automaticamente
%                            colocados em uma sequência numérica
%   Campos Opcionais
%     classificadores: quando há, lista de classificadores separados por ',',
%                      o sistema adiciona o texto automaticamente
%
% Para o comando \significado*:
%   Campos Obrigatórios
%     termo_gramatical: abreviação dos termos, como 's.', 'v.', 'adj.',
%                       'adv.', etc, conforme tabela acima
%     lista_de_significados: significados da palavra separados por ';' ou ','.
%                            Quando se usa ';' os significados são automaticamente
%                            colocados em uma sequência numérica
%   Campos Opcionais
%     classificadores: quando há, lista de classificadores separados por ',',
%                      o sistema adiciona o texto automaticamente
%
%   Observação: Este comando deve ser utilizado para identificar substantivos
%               próprios, ele adiciona o texto (Substantivo Próprio) na descrição
%               do significado e uma entrada de outra cor nos índices remissivos).
%
% Para o comando \veja:
%   Campos Obrigatórios
%     hanzi: hanzi da palavra
%     pronúncia: pinyin da pronúncia da palavra como em verbetes
%
%
% Para o comando \veja*:
%   Campos Obrigatórios
%     hanzi: hanzi da palavra
%     pronúncia: pinyin da pronúncia da palavra como em verbetes
%   Observação: Este comando joga o pinyin para a linha de baixo,
%               para evitar quebra e deixar mais bonito.
%
% Alguns exemplos de alguns de entradas de verbetes:
%
% \begin{verbete}[4]{不}[\\]{bu2}[ (antes de quarto tom)]{4}
%   \significado{adv.}{não}
%   \veja{不}{bu0}
%   \veja{不}{bu4}
% \end{verbete}
%
% \begin{verbete}{飞机票}{fei1ji1piao4}{4;6;11}
%   \significado[张]{v.}{bilhete de avião}
%   \veja{机票}{ji1piao4}
% \end{verbete}
%
% \begin{verbete}{餐厅}{can1ting1}{16;4}
%   \significado[家]{s.}{restaurante}
%   \significado[间]{s.}{sala de jantar}
% \end{verbete}
%
% \begin{verbete}{重}{chong2}{9}
%   \significado{adv.}{de novo}
%   \significado{p.c.}{camadas}
%   \significado{s.}{repetição}
%   \significado{v.}{repetir}
%   \veja{重}{zhong4}
% \end{verbete}
%
%%%%%%%%%%%%%%%%%%%%%%%%%%%%%%%%%%%%%%%%%%%%%%%%%%%%%%%%%%%%%%%%%%%%%%%

\begin{verbete}{__hanzi__}{__pinyin__}{__strokes__}
  \significado[__classificador__]{__class__}{__significado__}
  \veja{__hanzi__}{__pinyin__}
\end{verbete}
