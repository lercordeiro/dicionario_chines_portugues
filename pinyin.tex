
%%%%%%%%%%%%%%%%%%%%%%%%%%%%%%%%%%%%%%%%%
% LuaLaTex
%
% Dicionário Chinês -> Português
% Autor: Luiz Eduardo Roncato Cordeiro
%
% Licença:
% CC BY-NC-SA 3.0 (http://creativecommons.org/licenses/by-nc-sa/3.0/)
%%%%%%%%%%%%%%%%%%%%%%%%%%%%%%%%%%%%%%%%%

\documentclass[a4paper,9pt,twoside,openany]{memoir}

\usepackage{fontspec}
\usepackage[dvipsnames]{xcolor}
\usepackage{polyglossia}
\usepackage{multicol}
\usepackage{imakeidx}
\usepackage[inline]{enumitem}
\usepackage{tasks}
\usepackage{zhnumber}
\usepackage{tikz}
\usepackage[hyperindex,hidelinks]{hyperref}
\usepackage{pifont}
\usepackage{xstring}
\usepackage{tabularray}
\usepackage{stackengine}
\usepackage{tcolorbox}
\usepackage{luacode}

% Meus Comandos
%%%%%%%%%%%%%%%%%%%%%%%%%%%%%%%%%%%%%%%%%%%%%%%%%%%%%%%%%%%%%%%%%%%%%%%%%%%%%%%
%%%%%%%%%%%%%%%%%%%%%%%%%%%%%%%%%%%%%%%%%%%%%%%%%%%%%%%%%%%%%%%%%%%%%%%%%%%%%%%
%%%%%                                                                     %%%%%
%%%%% Funções e Ajustes dos Documentos do Dicionário                      %%%%%
%%%%%                                                                     %%%%%
%%%%%%%%%%%%%%%%%%%%%%%%%%%%%%%%%%%%%%%%%%%%%%%%%%%%%%%%%%%%%%%%%%%%%%%%%%%%%%%
%%%%%%%%%%%%%%%%%%%%%%%%%%%%%%%%%%%%%%%%%%%%%%%%%%%%%%%%%%%%%%%%%%%%%%%%%%%%%%%

%%% Hyperref em modo 'draft' não gera os hiperlinks
\hypersetup{final}

%%% Largura da entrada do verbete
\def\entrywidth{.49\textwidth}

%%% Estilo do capítulo, o melhor que encontrei
\chapterstyle{verville}

%%% Sem identação
\setlength\parindent{0pt}

%%% Ajuste das margens do documento
\setlrmarginsandblock{3cm}{2cm}{*}
\setulmarginsandblock{2cm}{*}{1}
\checkandfixthelayout

%%% Pra evitar viúvas e órfãs
\clubpenalty=10000
\widowpenalty=10000
\raggedbottom

%%% Espaçamento das linhas, 1 e meio
\OnehalfSpacing

%%% Usando a fonte NoTofu do Google.
\babelfont{rm}[
 Renderer=Node,
 Ligatures=TeX,
 BoldFont={NotoSerifCJKsc-SemiBold},
 BoldSlantedFont={NotoSerifCJKsc-SemiBold},
 AutoFakeSlant=0.25,
 SlantedFeatures={FakeSlant=0.25},
 BoldSlantedFeatures={FakeSlant=0.25}]
 {Noto Serif CJK SC Light}
\babelfont{sf}[Renderer=Harfbuzz,Ligatures=TeX]{Noto Sans CJK SC Light}
\babelfont{tt}[Renderer=Harfbuzz,Ligatures=TeX]{Noto Sans Mono CJK SC}

%%% Ajustes do Sumário
\makeatletter
\renewcommand{\@pnumwidth}{2em} 
\renewcommand{\@tocrmarg}{4em}
\makeatother
\renewcommand\cftbeforechapterskip{5pt plus 1pt}

%%% Ajustes da separação das colunas quando em modo texto de 2 colunas
\setlength{\columnsep}{.8em}
\setlength{\columnseprule}{0.1mm}

%%% Ajustes para o "stackengine"
\renewcommand\stacktype{S}
\renewcommand\stackalignment{c}

%%% Ajustes de Cabeçalhos e Rodapés
\setheadfoot{14pt}{28pt}

% Estilo "plain"
\makefootrule{plain}{\textwidth}{\normalrulethickness}{2pt}
\ifdraftdoc
 \makeevenfoot{plain}{\thepage}{汉葡词典}{Draft}
 \makeoddfoot{plain}{Draft}{汉葡词典}{\thepage}
\else
 \makeevenfoot{plain}{\thepage}{汉葡词典}{}
 \makeoddfoot{plain}{}{汉葡词典}{\thepage}
\fi

% Estilo "dictionary"
\makepagestyle{dictionary}
\makeheadrule{dictionary}{\textwidth}{\normalrulethickness}
\makefootrule{dictionary}{\textwidth}{\normalrulethickness}{2pt}
\ifdraftdoc
 \makeevenhead{dictionary}{\rightmark}{Draft}{\leftmark}
 \makeoddhead{dictionary}{\rightmark}{Draft}{\leftmark}
 \makeevenfoot{dictionary}{\thepage}{汉葡词典}{Draft}
 \makeoddfoot{dictionary}{Draft}{汉葡词典}{\thepage}
\else
 \makeevenhead{dictionary}{\rightmark}{}{\leftmark}
 \makeoddhead{dictionary}{\rightmark}{}{\leftmark}
 \makeevenfoot{dictionary}{\thepage}{汉葡词典}{}
 \makeoddfoot{dictionary}{}{汉葡词典}{\thepage}
\fi

%%% Estilo das Seções
\newcommand{\boxedsec}[1]
 {%
  \begin{tcolorbox}%
   [%
    %hbox,
    %before skip=2sp plus 1sp minus 1sp,
    %after skip=2sp plus 1sp minus 1sp,
    colframe=black,%
    colback=black!15!white,%
    boxrule=2pt,%
    leftrule=2mm,%
    left=0mm,%
    right=0mm,%
    top=0mm,%
    bottom=0mm%
   ]
   \hfill\LARGE\bfseries#1
  \end{tcolorbox}
 }
\setsecheadstyle{\boxedsec}
\setbeforesecskip{-2ex plus -.5ex minus -.25ex}
\setaftersecskip{.5ex plus .25ex}
\newcommand{\sectionbreak}{\phantomsection}

%%% Variáveis tipo "bool" para dizer se tem ou não os campos
%%% "Veja" e "Veja também" nas definições dos verbetes
\newbool{f_see}
\newbool{f_seealso}

%%% Converte os pinyins numéricos em pinyins com marcação de tom
\directlua{dofile "include/tex-sx-pinyin-tonemarks.lua"}

% Função "\pinyin" faz a conversão
\protected\def\pinyin#1{%
 \directlua{packagedata.pinyintones.convert ([==[#1]==])}%
}

% Comando "\dictpinyin", coloca o pinyin entre «»
\NewDocumentCommand{\dictpinyin}{m}{\guillemotleft\pinyin{#1}\guillemotright} 

% Comando "\dpy", gera a string do pinyin utilizada no Dicionário
% Este comando realiza uma série de substituições antes
\NewDocumentCommand{\dpy}{m}%
 {%
  \StrSubstitute{#1}{r5}{r}[\result]%
  \StrSubstitute{\result}{v}{ü}[\result]%
  \StrSubstitute{\result}{V}{Ü}[\result]%
  \edef\py{\dictpinyin{\result}}%
  \mbox{}\py
 }

%%% Comandos genéricos usados no Dicionário

% Comando "\&", insere o caracgter "&"
\DeclareRobustCommand{\&}%
 {%
  \ifdim\fontdimen1\font>0pt%
   \textsl{\symbol{`\&}}%
  \else%
   \symbol{`\&}%
  \fi%
 }

% Comando "\dul{text}", sublinha o texto dado
\NewDocumentCommand{\dul}{m}{\underline{#1}}

% Ambiente "enumerate" especial utilizado no dicionário, coloca as definições 
% do verbete em uma lista numerada em linha
\NewDocumentCommand{\dictenumerate}{>{\SplitList{|}}m}
 {%
  \begin{enumerate*}[nosep,label=(\arabic*),left=0pt,mode=unboxed,font=\bfseries]
   \ProcessList{#1}{\insertitem}
  \end{enumerate*}
 }
\NewDocumentCommand{\insertitem}{>{\TrimSpaces}m}{\item #1}

% Ambiente "enumerate" especial utilizado no dicionário, coloca os exemplos
% das definições do verbete em uma lista numerada em linha, utilizando
% algarismos romanos
\makeatletter
\NewDocumentCommand{\dictexamples}{m>{\SplitList{|}}m}
 {%
  \def\@theword{#1}%
  \begin{enumerate}[nosep,label=(\roman*),left=0pt,mode=unboxed,font=\bfseries]
   \ProcessList{#2}{\insertexample}
  \end{enumerate}
 }
\NewDocumentCommand{\insertexample}{>{\TrimSpaces}m}
 {
  \IfSubStr{#1}{mud::::}
   {% Sublinhado Manual
    \StrBehind{#1}{mud::::}[output]%
    \IfSubStr{\output}{___}
    {% Com traducao
     \StrCut{\output}{___}\csA\csB%
     \item\csA\\{\small``\csB''}
    }
    {% Sem traducao
     \item\output
    }
   }
   {% Sublinhado Automático
    \IfSubStr{#1}{___}
    {% Com traducao
      \StrCut{#1}{___}\csA\csB%
      \item\StrSubstitute{\csA}{\@theword}{\underline{\@theword}}\\{\small``\csB''}
    }
    {% Sem traducao
      \item\StrSubstitute{#1}{\@theword}{\underline{\@theword}}
    }
   }
 }  
\makeatother

%%%%% EOF %%%%%


%%%%%%%%%%%%%%%%%%%%%%%%%%%%%%%%%%%%%%%%%%%%%%%%%%%%%%%%%%%%%%%%%%%%%%%%%%%%%%%
%%%%%%%%%%%%%%%%%%%%%%%%%%%%%%%%%%%%%%%%%%%%%%%%%%%%%%%%%%%%%%%%%%%%%%%%%%%%%%%
%%%%%                                                                     %%%%%
%%%%% pinyincmd.tex:                                                      %%%%%
%%%%% Ambientes para o Dicionário ordenado por pinyins.                   %%%%%
%%%%%                                                                     %%%%%
%%%%%%%%%%%%%%%%%%%%%%%%%%%%%%%%%%%%%%%%%%%%%%%%%%%%%%%%%%%%%%%%%%%%%%%%%%%%%%%
%%%%%%%%%%%%%%%%%%%%%%%%%%%%%%%%%%%%%%%%%%%%%%%%%%%%%%%%%%%%%%%%%%%%%%%%%%%%%%%

\ExplSyntaxOn

%%% Ambiente "entry", para os verbetes
\NewDocumentEnvironment{entry}{mO{}mO{}mooo}%
 {%
  \leavevmode
  \markboth{#1{\tiny\dpy{#3}}}{#1{\tiny\dpy{#3}}}
  \tl_set:Nn \l_hanzi_tl {#1}
  \tl_set:Nn \l_pinyin_tl {#3}
  \tl_set:Nn \l_strokes_tl {#5}
  \boolfalse{f_see}\renewcommand\seerefl{}\listadd{\seerefl}{}% Initialize list
  \boolfalse{f_seealso}\renewcommand\seealsorefl{}\listadd{\seealsorefl}{}% Initialize list
  \begin{minipage}[t][][t]{.49\textwidth}
   \label{#1:#3}
   \begin{tcolorbox}[size=title,colframe=black,colback=white,boxrule=1pt,toprule=2pt,left=0mm,right=0mm,top=0mm,bottom=0mm]
    {\Large#1}\hfill\textsuperscript{\tiny(#5画)}\\
    {\footnotesize#2\ \dpy{#3}\ #4}%
    \IfValueT{#6}{\mbox{}\hfill{\tiny#6}}{}%
    \IfValueT{#7}{\mbox{}\hfill{\tiny#7}}{}%
    \IfValueT{#8}{\mbox{}\hfill{\tiny#8}}{}
   \end{tcolorbox}
 }{%
  \ifbool{f_see}%
   {% Process "see" references
    \RenewDocumentCommand\do{>{\SplitArgument{1}{:}}m}{\item \seeitem ##1}
    \textbf{Veja:\ }%
    \begin{itemize*}[label={}, before={\hspace{2sp}}, itemjoin={{,\ }}, itemjoin*={{\ e~}}]
     \dolistloop{\seerefl}
    \end{itemize*}\par
   }{}%
  \ifbool{f_seealso}%
   {% Process "see also" references
    \RenewDocumentCommand\do{>{\SplitArgument{1}{:}}m}{\item \seeitem ##1}
    \textbf{Veja\ também:\ }%
    \begin{itemize*}[label={}, before={\hspace{2sp}}, itemjoin={{,\ }}, itemjoin*={{\ e~}}]
     \dolistloop{\seealsorefl}
    \end{itemize*}\par
   }{}%
  \end{minipage}
  \vspace{2mm}
}

%%% Ambiente "entry*", para os verbetes muito compridos
\NewDocumentEnvironment{entry*}{mO{}mO{}mooo}
 {%
  \leavevmode
  \markboth{#1{\tiny\dpy{#3}}}{#1{\tiny\dpy{#3}}}
  \tl_set:Nn \l_hanzi_tl {#1}
  \tl_set:Nn \l_pinyin_tl {#3}
  \tl_set:Nn \l_strokes_tl {#5}
  \boolfalse{f_see} \renewcommand\seerefl{} \listadd{\seerefl}{}% Initialize list
  \boolfalse{f_seealso} \renewcommand\seealsorefl{} \listadd{\seealsorefl}{}% Initialize list
  \vspace{\baselineskip}
  \begin{minipage}[t][][t]{.49\textwidth}
   \label{#1:#3}
   \begin{tcolorbox}[size=title,colframe=black,colback=white,boxrule=1pt,toprule=2pt,left=0mm,right=0mm,top=0mm,bottom=0mm]
    \mbox{}\hfill\textsuperscript{\tiny(#5画)}\\
    {\LARGE#1}\\
    {\footnotesize#2\ \dpy{#3}\ #4}\\
    \IfValueT{#6}{\mbox{}\hfill{\tiny#6}}{}%
    \IfValueT{#7}{\mbox{}\hfill{\tiny#7}}{}%
    \IfValueT{#8}{\mbox{}\hfill{\tiny#8}}{}
   \end{tcolorbox}
 }%
 {%
  \ifbool{f_see}%
   {% Há 'veja'
    \RenewDocumentCommand\do{>{\SplitArgument{1}{:}}m}{\item \seeitem ##1}
    \textbf{Veja:\ }%
    \begin{itemize*}[label={}, itemjoin={{,\ }}, itemjoin*={{\ e~}}]
     \dolistloop{\seerefl}
    \end{itemize*}\par
   }{}%
  \ifbool{f_seealso}%
   {% Há 'veja tambem'
    \RenewDocumentCommand\do{>{\SplitArgument{1}{:}}m}{\item \seeitem ##1}
    \textbf{Veja\ também:\ }%
    \begin{itemize*}[label={}, itemjoin={{,\ }}, itemjoin*={{\ e~}}]
     \dolistloop{\seealsorefl}
    \end{itemize*}\par
   }{}%
  \end{minipage}
  \vspace{2mm}
}

\ExplSyntaxOff

%%%%% EOF %%%%%


% Ajustes do PDF
\hypersetup{
  linktoc=page,
  colorlinks=true,
  urlcolor=blue,
  linkcolor=blue,
  citecolor=blue,
  pdftitle={汉葡词典 - Dicionário Chinês-Português},
  pdfsubject={Dicionário Chinês-Português -- Ordenado por Pinyin},
  pdfauthor={Luiz Eduardo Roncato Cordeiro, AKA 罗学凯},
  pdfkeywords={dicionário, chinês, português, instituto confúcio}
}

% Índices Remissivos
\makeindex[title=Índice Remissivo por Traço,intoc=true,columns=3,columnsep=15pt,columnseprule=true,noautomatic=true,name=pstroke]
\makeindex[title=Índice Remissivo por Radical,intoc=true,columns=3,columnsep=15pt,columnseprule=true,noautomatic=true,name=pradical]
\indexsetup{level=\chapter*,toclevel=chapter,headers={\indexname}{\indexname}}

%%%
%%% Documento começa aqui!
%%%

\begin{document}
\addfontfeatures{CharacterWidth=Proportional}

\begin{titlingpage}
  \raggedleft
  \rule{1pt}{\textheight}
  \hspace{0.1\textwidth}
  \parbox[b]{0.75\textwidth}{
    \vspace{0.05\textheight}
    {\HUGE\bfseries 汉葡词典}\\[2\baselineskip] % Title
    {\Large\textsc{Dicionário Chinês-Português}\\%
     \large\textsc{\zhtoday}}\\% Date
    [4\baselineskip]
    {\Large\textsc{罗学凯}\\%
     \small Luiz Eduardo Roncato Cordeiro\\% Author
            Aluno do Instituto Confúcio da UNESP}\\%
    \vspace{0.5\textheight}\\%
    {Instituto Confúcio, Curso de Chinês}\\[\baselineskip] % Publisher?
  }
  \newpage
  \raggedright
  \setlength{\parindent}{0pt}
  \setlength{\parskip}{\baselineskip}
  \mbox{}
  \vfill
  \footnotesize
  \textcopyright{} 2024 por Luiz Eduardo Roncato Cordeiro, está licenciado sob CC BY-NC-SA 4.0\\
  \begin{itemize}
    \item Para visualizar uma cópia desta licença, visite:\\ \url{http://creativecommons.org/licenses/by-nc-sa/4.0/}
    \item Este trabalho ainda está em andamento e o ``código fonte'' está localizado em:\\ \url{https://github.com/lercordeiro/dicionario_chines_portugues}
    \item A última versão compilada também pode ser encontrada em:\\ \url{https://ler.cordeiro.nom.br/}
  \end{itemize}
%   \begin{tabular}{ll}
%   First edition: & T.B.D. \\
%   \end{tabular}
\end{titlingpage}


\clearpage
\pagestyle{empty}
\tableofcontents

\clearpage
\pagestyle{empty}
\chapter{汉葡词典}

%%%%%%%%%%%%%%%%%%%%%%%%
%
% https://en.wikipedia.org/wiki/Chinese_character_orders
%
%%%%%%%%%%%%%%%%%%%%%%%%

Dicionário Chinês-Português ordenado primeiro pelo pinyin de cada
caracter, depois pelo número de traços e, finalmente, pela ordem do
caracter na tabela UTF-8.

\clearpage
\pagestyle{dicionario}
\begin{multicols}{2}
%%%
%%% A
%%%
\section*{A}
\addcontentsline{toc}{section}{A}
\begin{multicols}{2}

\begin{hanzi}[啊]{a0}
\entry{a0}{part.}{ah!;oh!|no final da sentença para expressar entusiasmo|%
no final da sentença para expressar impaciência ou o que é óbvio|%
no final de uma ordem, aviso, etc|%
no final da sentença para expressar questionamento|%
para indicar uma pausa deliberada|%
para enumerar itens}
\end{hanzi}

\begin{hanzi}[矮]{ai3}
\entry{ai3}{adj.}{baixo (estatura, dimensão, grau ou ranque)}
\end{hanzi}

\begin{hanzi}[爱]{ai4}
\entry{ai4}{n.}{amor; afeição}
\entry{ai4}{v.}{amar; gostar|ter afeição}
\end{hanzi}

\begin{hanzi}[爱好]{ai4hao4}
\entry{ai4hao4}{n.}{passatempo; interesse|\pc{个}}
\entry{ai4hao4}{v.}{ter algo como hobby; ter prazer em fazer algo}
\end{hanzi}

\begin{hanzi}[爱人]{ai4ren0}
\entry{ai4ren0}{n.}{marido ou esposa|querido ou querida|\pc{个}}
\end{hanzi}

\end{multicols}

%%%
%%% B
%%%

\section*{B}\addcontentsline{toc}{section}{B}

\begin{entry}{八}{ba1}{2}[HSK 1][Kangxi 12][Radical ⼋]
  \definition{num.}{oito; 8}
\end{entry}

\begin{entry}{八八六}{ba1 ba1 liu4}{2,2,4}[Radicais ⼋、⼋、⼋]
  \definition{expr.}{\emph{Bye bye!} (em salas de bate-papo e mensagens de texto)}
\end{entry}

\begin{entry}{巴勒斯坦}{ba1le4si1tan3}{4,11,12,8}[Radicais ⼰、⼒、⽄、⼟]
  \definition*{s.}{Palestina}
\end{entry}

\begin{entry}{巴士}{ba1 shi4}{4,3}[HSK 4][Radicais ⼰、⼠]
  \definition[辆]{s.}{ônibus; transliteração da palavra inglesa ``bus''}
\end{entry}

\begin{entry}{巴西}{ba1xi1}{4,6}[Radicais ⼰、⾑]
  \definition*{s.}{Brasil}
\end{entry}

\begin{entry}{巴西人}{ba1xi1ren2}{4,6,2}[Radicais ⼰、⾑、⼈]
  \definition[个,位]{s.}{brasileiro | pessoa ou povo do Brasil}
  \example{他是巴西人。}[Ele é brasileiro.]
\end{entry}

\begin{entry}{巴西战舞}{ba1xi1zhan4wu3}{4,6,9,14}[Radicais ⼰、⾑、⼽、⾇]
  \definition{s.}{capoeira}
\end{entry}

\begin{entry}{吧}{ba1}{7}[Radical ⼝]
  \definition{s.}{som de estalo, som crepitante}
  \definition{v.}{puxar o cachimbo; fumar | abreviação de ``bar''}
  \seeref{吧}{ba5}
\end{entry}

\begin{entry}{拔尖}{ba2jian1}{8,6}[Radicais ⼿、⼩]
  \definition{adj.}{topo de linha | fora do comum | o melhor}
  \definition{v.+compl.}{empurrar-se para a frente | sentir que é superior aos outros}
\end{entry}

\begin{entry}{把}{ba3}{7}[HSK 3][Radical ⼿]
  \definition{clas.}{para objetos com alça | para objetos pequenos:~punhado}
  \definition{part.}{partícula tornando o substantivo seguinte um objeto direto}
  \definition{v.}{conter | alcançar | segurar}
  \seeref{把}{ba4}
\end{entry}

\begin{entry}{把柄}{ba3bing3}{7,9}[Radicais ⼿、⽊]
  \definition{s.}{(figurativo) informações que podem ser usadas contra alguém}
\end{entry}

\begin{entry}{把持}{ba3chi2}{7,9}[Radicais ⼿、⼿]
  \definition{v.}{controlar | dominar | monopolizar}
\end{entry}

\begin{entry}{把风}{ba3feng1}{7,4}[Radicais ⼿、⾵]
  \definition{v.}{estar atento | vigiar (durante uma atividade clandestina)}
\end{entry}

\begin{entry}{把关}{ba3guan1}{7,6}[Radicais ⼿、⼋]
  \definition{v.}{verificar estritamente | examinar cuidadosamente para ver se algo é feito de acordo com um padrão fixo | fazer a verificação final | guardar uma passagem, fronteira}
\end{entry}

\begin{entry}{把脉}{ba3mai4}{7,9}[Radicais ⼿、⾁]
  \definition{v.}{sentir ou tomar o pulso de alguém}
\end{entry}

\begin{entry}{把式}{ba3shi4}{7,6}[Radicais ⼿、⼷]
  \definition{s.}{pessoa qualificada em um comércio}
\end{entry}

\begin{entry}{把守}{ba3shou3}{7,6}[Radicais ⼿、⼧]
  \definition{v.}{vigiar | guardar}
\end{entry}

\begin{entry}{把玩}{ba3wan2}{7,8}[Radicais ⼿、⽟]
  \definition{v.}{brincar com | mexer com}
\end{entry}

\begin{entry}{把稳}{ba3wen3}{7,14}[Radicais ⼿、⽲]
  \definition{adj.}{confiável}
\end{entry}

\begin{entry}{把握}{ba3wo4}{7,12}[HSK 3][Radicais ⼿、⼿]
  \definition{s.}{seguro | garantia | certeza}
  \definition{v.}{agarrar | segurar | aproveitar}
\end{entry}

\begin{entry}{把戏}{ba3xi4}{7,6}[Radicais ⼿、⼽]
  \definition{s.}{acrobacia | malabarismo | truque barato}
\end{entry}

\begin{entry}{把}{ba4}{7}[Radical ⼿]
  \definition{v.}{lidar}
  \seeref{把}{ba3}
\end{entry}

\begin{entry}{爸}{ba4}{8}[HSK 1][Radical ⽗]
  \definition[个,位]{s.}{(informal) pai}
  \seeref{爸爸}{ba4ba5}
  \seealsoref{爸爸}{ba4ba5}
\end{entry}

\begin{entry}{爸爸}{ba4ba5}{8,8}[HSK 1][Radicais ⽗、⽗]
  \definition[个,位,名,群]{s.}{(informal) pai; papai; papa}
  \seeref{爸}{ba4}
\end{entry}

\begin{entry}{爸妈}{ba4ma1}{8,6}[Radicais ⽗、⼥]
  \definition{s.}{pai e mãe}
\end{entry}

\begin{entry}{罢}{ba4}{10}[Radical ⽹]
  \definition{v.}{parar | cessar | demitir | suspender | desistir | terminar}
  \seeref{罢}{ba5}
\end{entry}

\begin{entry}{霸权}{ba4quan2}{21,6}[Radicais ⾬、⽊]
  \definition{s.}{hegemonia | supremacia}
\end{entry}

\begin{entry}{吧}{ba5}{7}[HSK 1][Radical ⼝]
  \definition{part.}{indica discussão, sugestão, solicitação ou comando no final de uma frase | indica concordância ou aprovação no final de uma frase | indica uma pergunta ou especulação no final de uma frase | indica incerteza no final de uma frase | em uma frase, indica uma pausa, carrega um tom hipotético, frequentemente apresenta um contraste e implica um dilema}
  \seeref{吧}{ba1}
\end{entry}

\begin{entry}{罢}{ba5}{10}[Radical ⽹]
  \definition{part.}{partícula final, a mesma que 吧}
  \seeref{罢}{ba4}
  \seealsoref{吧}{ba5}
\end{entry}

\begin{entry}{白}{bai2}{5}[HSK 1,3][Kangxi 106][Radical ⽩]
  \definition*{s.}{sobrenome Bai}
  \definition{adj.}{branco | claro | puro; claro; simples; sem mistura; em branco | branco (como símbolo de reação) | escrito incorretamente ou pronunciado incorretamente | grátis; sem custos}
  \definition{adv.}{em vão; sem propósito; sem resultados}
  \definition{s.}{parte falada em ópera, etc.; frases de peças de teatro, etc. | dialeto local | funeral}
  \definition{v.}{explicar; apresentar; esclarecer; declarar | branquear | olhar para as pessoas com o branco dos olhos (olhar vazio, de desaprovação)}
\end{entry}

\begin{entry}{白菜}{bai2 cai4}{5,11}[HSK 3][Radicais ⽩、⾋]
  \definition[棵,个]{s.}{acelga | repolho chinês}
\end{entry}

\begin{entry}{白痴}{bai2chi1}{5,13}[Radicais ⽩、⽧]
  \definition{adj./s.}{estúpido | imbecil}
\end{entry}

\begin{entry}{白蛋白}{bai2dan4bai2}{5,11,5}[Radicais ⽩、⾍、⽩]
  \definition{s.}{albumina}
\end{entry}

\begin{entry}{白鹄}{bai2hu2}{5,12}[Radicais ⽩、⿃]
  \definition{s.}{cisne branco}
\end{entry}

\begin{entry}{白拣}{bai2jian3}{5,8}[Radicais ⽩、⼿]
  \definition{s.}{uma escolha barata}
  \definition{v.}{escolher algo que não custa nada}
\end{entry}

\begin{entry}{白萝卜}{bai2luo2bo5}{5,11,2}[Radicais ⽩、⾋、⼘]
  \definition{s.}{rabanete branco}
\end{entry}

\begin{entry}{白色}{bai2 se4}{5,6}[HSK 2][Radicais ⽩、⾊]
  \definition{s.}{cor branca}
\end{entry}

\begin{entry}{白天}{bai2 tian1}{5,4}[HSK 1][Radicais ⽩、⼤]
  \definition{adv.}{dia | de dia}
  \definition[个]{s.}{dia}
\end{entry}

\begin{entry}{白苋}{bai2xian4}{5,7}[Radicais ⽩、⾋]
  \definition{s.}{amaranto branco | brotos e folhas tenras de espinafre chinês usados como alimento}
\end{entry}

\begin{entry}{百}{bai3}{6}[HSK 1][Radical ⽩]
  \definition*{s.}{sobrenome Bai}
  \definition{num.}{cem; 100 | centena | cento}
\end{entry}

\begin{entry}{百般}{bai3ban1}{6,10}[Radicais ⽩、⾈]
  \definition{adv.}{de todas as maneiras possíveis | por todos os meios}
\end{entry}

\begin{entry}{百分}{bai3fen1}{6,4}[Radicais ⽩、⼑]
  \definition{num.}{por cento}
  \definition{s.}{porcentagem}
\end{entry}

\begin{entry}{百货}{bai3 huo4}{6,8}[HSK 4][Radicais ⽩、⾙]
  \definition{s.}{mercadorias em geral; loja de departamentos; um termo geral para bens que incluem principalmente roupas, utensílios e necessidades diárias gerais}
\end{entry}

\begin{entry}{柏树}{bai3shu4}{9,9}[Radicais ⽊、⽊]
  \definition{s.}{cipreste}
\end{entry}

\begin{entry}{摆}{bai3}{13}[HSK 4][Radical ⼿]
  \definition*{s.}{sobrenome Bai | Festival de Ganbai; uma reunião realizada nas áreas Dai durante festivais religiosos, para celebrar uma boa colheita ou para trocar materiais; geralmente se refere a uma reunião em massa}
  \definition{s.}{pêndulo; um dispositivo mecânico que controla a frequência de vibração em relógios e instrumentos | a bainha inferior de um vestido, jaqueta ou saia}
  \definition{v.}{colocar; organizar | vestir; assumir | balançar; acenar; agitar para frente e para trás | expor; declarar claramente; listar | dizer; falar | libertar-se}
\end{entry}

\begin{entry}{摆动}{bai3 dong4}{13,6}[HSK 4][Radicais ⼿、⼒]
  \definition{v.}{balançar; balançar para frente e para trás; oscilar; vibrar}
\end{entry}

\begin{entry}{摆烂}{bai3lan4}{13,9}[Radicais ⼿、⽕]
  \definition{v.}{(neologismo, gíria) parar de lutar (especialmente quando se sabe que não pode ter sucesso) | deixar tudo ir para o inferno}
\end{entry}

\begin{entry}{摆手}{bai3shou3}{13,4}[Radicais ⼿、⼿]
  \definition{v.+compl.}{gesticular com a mão (acenando, acenando adeus, etc.) | balançar os braços | acenar com as mãos}
\end{entry}

\begin{entry}{摆脱}{bai3tuo1}{13,11}[HSK 4][Radicais ⼿、⾁]
  \definition{v.}{sacudir; rejeitar; romper com; libertar-se (ou desembaraçar-se) de; livrar-se de dificuldades, escravidão, controle, etc.}
\end{entry}

\begin{entry}{败}{bai4}{8}[HSK 4][Radical ⾒]
  \definition{adj.}{dilapidado; decadente; murcho; em declínio}
  \definition{v.}{derrota; bater | falhar | quebrar; neutralizar; dissipar | arruinar; estragar; corromper | ser derrotado; perder}
\end{entry}

\begin{entry}{班}{ban1}{10}[HSK 1][Radical ⽟]
  \definition*{s.}{sobrenome Ban}
  \definition{clas.}{para grupos}
  \definition[个]{s.}{equipe| time | esquadrão | turno de trabalho | classificação}
\end{entry}

\begin{entry}{班级}{ban1 ji2}{10,6}[HSK 3][Radicais ⽟、⽷]
  \definition[个]{s.}{classe | série (na escola)}
\end{entry}

\begin{entry}{班长}{ban1 zhang3}{10,4}[HSK 2][Radicais ⽟、⾧]
  \definition[个]{s.}{monitor de classe | líder de equipe | líder de esquadrão}
\end{entry}

\begin{entry}{般}{ban1}{10}[Radical ⾈]
  \definition{s.}{espécie | tipo | classe | caminho | maneira}
  \seeref{般}{bo1}
  \seeref{般}{pan2}
\end{entry}

\begin{entry}{搬}{ban1}{13}[HSK 3][Radical ⼿]
  \definition{v.}{copiar indiscriminadamente | mover-se (ou seja, mudar-se) | mover-se (algo relativamente pesado ou volumoso) | mudar | mudar-se}
\end{entry}

\begin{entry}{搬动}{ban1dong4}{13,6}[Radicais ⼿、⼒]
  \definition{v.}{mover-se (alguma coisa) | se mudar}
\end{entry}

\begin{entry}{搬家}{ban1jia1}{13,10}[HSK 3][Radicais ⼿、⼧]
  \definition{s.}{mudança}
  \definition{v.+compl.}{mudar-se de casa}
\end{entry}

\begin{entry}{搬口}{ban1kou3}{13,3}[Radicais ⼿、⼝]
  \definition{v.}{tagarelar | (idioma) transmitir histórias | semear dissensão | contar histórias}
\end{entry}

\begin{entry}{搬弄}{ban1nong4}{13,7}[Radicais ⼿、⼶]
  \definition{v.}{causar problemas | mexer com alguém | mostrar (o que se pode fazer)}
\end{entry}

\begin{entry}{搬运}{ban1yun4}{13,7}[Radicais ⼿、⾡]
  \definition{s.}{frete | transporte}
  \definition{v.}{carregar | transportar}
\end{entry}

\begin{entry}{搬走}{ban1zou3}{13,7}[Radicais ⼿、⾛]
  \definition{v.}{carregar}
\end{entry}

\begin{entry}{板}{ban3}{8}[HSK 3][Radical ⽊]
  \definition{adj.}{rígido; não natural | duro}
  \definition{clas.}{para cartões, papéis}
  \definition{s.}{tábua; placa; prato | veneziana; persiana; refere-se especificamente aos painéis de portas de lojas | badalos (instrumento musical que marca o ritmo) | uma batida acentuada (ritmo) na música e na ópera tradicional | chefe}
  \definition{v.}{parecer sério | livrar-se de maus hábitos ou falhas}
\end{entry}

\begin{entry}{办}{ban4}{4}[HSK 2][Radical ⼒]
  \definition{v.}{lidar com | lidar | gerenciar | configurar}
\end{entry}

\begin{entry}{办法}{ban4fa3}{4,8}[HSK 2][Radicais ⼒、⽔]
  \definition[条,个]{s.}{meio (de se fazer alguma coisa) | método | medida}
\end{entry}

\begin{entry}{办公}{ban4gong1}{4,4}[Radicais ⼒、⼋]
  \definition{v.+compl.}{lidar com negócios oficiais | trabalhar (especialmente em um escritório)}
\end{entry}

\begin{entry}{办公室}{ban4gong1shi4}{4,4,9}[HSK 2][Radicais ⼒、⼋、⼧]
  \definition[间]{s.}{gabinete | escritório}
\end{entry}

\begin{entry}{办理}{ban4li3}{4,11}[HSK 3][Radicais ⼒、⽟]
  \definition{v.}{conduzir | manusear | transacionar}
\end{entry}

\begin{entry}{办事}{ban4 shi4}{4,8}[HSK 4][Radicais ⼒、⼅]
  \definition{v.}{trabalhar | lidar com assuntos; manipular transações}
\end{entry}

\begin{entry}{半}{ban4}{5}[HSK 1][Radical ⼗]
  \definition{adj.}{incompleto}
  \definition{num.}{(depois de um número) ``e meio''}
  \definition{pref.}{semi}
  \definition{s.}{metade}
\end{entry}

\begin{entry}{半年}{ban4 nian2}{5,6}[HSK 1][Radicais ⼗、⼲]
  \definition{s.}{meio ano}
\end{entry}

\begin{entry}{半球}{ban4qiu2}{5,11}[Radicais ⼗、⽟]
  \definition{s.}{hemisfério}
\end{entry}

\begin{entry}{半天}{ban4 tian1}{5,4}[HSK 1][Radicais ⼗、⼤]
  \definition{s.}{metade do dia | muito tempo | bastante tempo}
\end{entry}

\begin{entry}{半夜}{ban4 ye4}{5,8}[HSK 2][Radicais ⼗、⼣]
  \definition{adv.}{no meio da noite | metade de uma noite}
  \definition{s.}{meia-noite}
\end{entry}

\begin{entry}{半音}{ban4yin1}{5,9}[Radicais ⼗、⾳]
  \definition{s.}{semitom}
\end{entry}

\begin{entry}{伴侣}{ban4lv3}{7,8}[Radicais ⼈、⼈]
  \definition{s.}{companheiro | parceiro}
\end{entry}

\begin{entry}{帮}{bang1}{9}[HSK 1][Radical ⼱]
  \definition{clas.}{para alguém (como uma ajuda)}
  \definition{s.}{gangue | grupo | contratado (como trabalhador) | camada externa | festa | sociedade secreta}
  \definition{v.}{ajudar | apoiar}
\end{entry}

\begin{entry}{帮教}{bang1jiao4}{9,11}[Radicais ⼱、⽁]
  \definition{v.}{orientar}
\end{entry}

\begin{entry}{帮忙}{bang1 mang2}{9,6}[HSK 1][Radicais ⼱、⼼]
  \definition{v.+compl.}{ajudar | dar uma mão | estender a mão | fazer um favor}
\end{entry}

\begin{entry}{帮佣}{bang1yong1}{9,7}[Radicais ⼱、⼈]
  \definition{s.}{ajudante doméstico | servo}
\end{entry}

\begin{entry}{帮助}{bang1zhu4}{9,7}[HSK 2][Radicais ⼱、⼒]
  \definition[种]{s.}{ajuda | assistência}
  \definition{v.}{ajudar | dar assistência}
\end{entry}

\begin{entry}{棒棒糖}{bang4bang4tang2}{12,12,16}[Radicais ⽊、⽊、⽶]
  \definition[根]{s.}{pirulito}
\end{entry}

\begin{entry}{棒冰}{bang4bing1}{12,6}[Radicais ⽊、⼎]
  \definition{s.}{picolé}
\end{entry}

\begin{entry}{包}{bao1}{5}[HSK 1][Radical ⼓]
  \definition*{s.}{sobrenome Bao}
  \definition{clas.}{pacotes, sacos, sacolas, embrulhos}
  \definition[个,只]{s.}{bolsa | pacote | recipiente | embrulho}
  \definition{v.}{contratar | cobrir | segurar ou abraçar | incluir | assumir o comando | embrulhar}
\end{entry}

\begin{entry}{包办}{bao1ban4}{5,4}[Radicais ⼓、⼒]
  \definition{v.}{comandar todo o show | comprometer-se a fazer tudo sozinho}
\end{entry}

\begin{entry}{包干}{bao1gan1}{5,3}[Radicais ⼓、⼲]
  \definition{s.}{tarefa alocada}
  \definition{v.}{ter a responsabilidade total sobre um trabalho}
\end{entry}

\begin{entry}{包裹}{bao1guo3}{5,14}[HSK 4][Radicais ⼓、⾐]
  \definition[个]{s.}{pacote; embrulho}
  \definition{v.}{embrulhar; amarrar; enrolar coisas em pano ou outra coisa}
\end{entry}

\begin{entry}{包含}{bao1han2}{5,7}[HSK 4][Radicais ⼓、⼝]
  \definition{v.}{conter; implicar; incluir; conter dentro, resumir, enfatizar o que está contido dentro, focar em relações internas, muitas vezes coisas abstratas}
\end{entry}

\begin{entry}{包括}{bao1kuo4}{5,9}[HSK 4][Radicais ⼓、⼿]
  \definition{v.}{incluir; compreender; consistir em; conter, conter dentro, resumir junto, enfatizar a listagem de todas as partes, ou a citação de uma parte delas, que podem ser coisas abstratas ou concretas}
\end{entry}

\begin{entry}{包容}{bao1rong2}{5,10}[Radicais ⼓、⼧]
  \definition{adj.}{inclusivo}
  \definition{v.}{perdoar | mostrar tolerância | conter | segurar}
\end{entry}

\begin{entry}{包子}{bao1 zi5}{5,3}[HSK 1][Radicais ⼓、⼦]
  \definition[个]{s.}{pão recheado cozido no vapor}
\end{entry}

\begin{entry}{包租}{bao1zu1}{5,10}[Radicais ⼓、⽲]
  \definition{s.}{aluguel fixo para terras agrícolas}
  \definition{v.}{fretar | alugar | alugar um terreno ou uma casa para subarrendar}
\end{entry}

\begin{entry}{薄}{bao2}{16}[HSK 4][Radical ⾋]
  \definition{adj.}{fino; frágil; pouca espessura |  frio; indiferente; carente de calor; emocionalmente frio; não profundo | leve; fraco | pobre; infértil}
  \seeref{薄}{bo2}
\end{entry}

\begin{entry}{宝}{bao3}{8}[HSK 4][Radical ⼧]
  \definition{adj.}{antigo; precioso; estimado}
  \definition[个,件]{s.}{tesouro; objeto estimado; coisa preciosa | dispositivo de jogo; ferramenta de jogo | dinheiro; moeda; moeda antiga com furo quadrado no centro; moeda de prata}
  \definition{s.}{sobrenome Bao}
\end{entry}

\begin{entry}{宝宝}{bao3 bao5}{8,8}[HSK 4][Radicais ⼧、⼧]
  \definition[个]{s.}{querida; \emph{darling}; \emph{baby}; apelido para crianças}
\end{entry}

\begin{entry}{宝贝}{bao3bei4}{8,4}[HSK 4][Radicais ⼧、⾙]
  \definition{adj.}{excêntrico; estranho; imprestável; um termo depreciativo para uma pessoa incompetente ou ridícula}
  \definition[个,件]{s.}{tesouro; objeto estimado; coisa preciosa | querida; \emph{darling}; \emph{baby}; apelido para crianças}
\end{entry}

\begin{entry}{宝贵}{bao3gui4}{8,9}[HSK 4][Radicais ⼧、⾙]
  \definition{adj.}{precioso; extremamente valioso, muito raro, pode ser usado para descrever coisas específicas, também pode ser usado para descrever coisas abstratas | valioso; como um tesouro}
\end{entry}

\begin{entry}{宝石}{bao3 shi2}{8,5}[HSK 4][Radicais ⼧、⽯]
  \definition[颗,枚,块]{s.}{gema; jóia; pedra preciosa; mineral precioso que tem um brilho lindo e uma dureza de mais de sete graus, não é afetado pela atmosfera ou por produtos químicos e pode ser usado como decoração, suporte de instrumentos ou abrasivos}
\end{entry}

\begin{entry}{饱}{bao3}{8}[HSK 2][Radical ⾷]
  \definition{adj.}{ter comido até ficar satisfeito | estar cheio | cheio}
  \definition{adv.}{completamente | até estar cheio}
  \definition{v.}{satisfazer}
\end{entry}

\begin{entry}{保}{bao3}{9}[HSK 3][Radical ⼈]
  \definition*{s.}{sobrenome Bao}
  \definition{s.}{fiador
oficial responsável
sistema administrativo}
  \definition{v.}{defender | proteger |manter | preservar | conservar em boas condições | garantir | assegurar | ficar como fiador de alguém.}
\end{entry}

\begin{entry}{保安}{bao3 an1}{9,6}[HSK 3][Radicais ⼈、⼧]
  \definition{s.}{guarda de segurança}
  \definition{v.}{manter seguro | garantir a segurança}
\end{entry}

\begin{entry}{保持}{bao3chi2}{9,9}[HSK 3][Radicais ⼈、⼿]
  \definition{v.}{manter | segurar | reter | preservar}
\end{entry}

\begin{entry}{保存}{bao3cun2}{9,6}[HSK 3][Radicais ⼈、⼦]
  \definition{v.}{conservar | preservar | (computação) salvar (um arquivo, etc.)}
\end{entry}

\begin{entry}{保护}{bao3hu4}{9,7}[HSK 3][Radicais ⼈、⼿]
  \definition{s.}{proteção | salvaguarda}
  \definition{v.}{proteger | defender | salvaguardar}
\end{entry}

\begin{entry}{保护国}{bao3hu4guo2}{9,7,8}[Radicais ⼈、⼿、⼞]
  \definition{s.}{protetorado}
\end{entry}

\begin{entry}{保护剂}{bao3hu4ji4}{9,7,8}[Radicais ⼈、⼿、⼑]
  \definition{s.}{agente protetor}
\end{entry}

\begin{entry}{保护区}{bao3hu4qu1}{9,7,4}[Radicais ⼈、⼿、⼖]
  \definition{s.}{área protegida | área de conservação}
\end{entry}

\begin{entry}{保护色}{bao3hu4se4}{9,7,6}[Radicais ⼈、⼿、⾊]
  \definition{s.}{camuflagem}
\end{entry}

\begin{entry}{保护神}{bao3hu4shen2}{9,7,9}[Radicais ⼈、⼿、⽰]
  \definition{s.}{anjo da guarda | santo patrono}
\end{entry}

\begin{entry}{保护物}{bao3hu4 wu4}{9,7,8}[Radicais ⼈、⼿、⽜]
  \definition{s.}{protetor}
\end{entry}

\begin{entry}{保护性}{bao3hu4xing4}{9,7,8}[Radicais ⼈、⼿、⼼]
  \definition{s.}{proteção}
\end{entry}

\begin{entry}{保护者}{bao3hu4zhe3}{9,7,8}[Radicais ⼈、⼿、⽼]
  \definition{s.}{protetor | segurador}
\end{entry}

\begin{entry}{保护主义}{bao3hu4zhu3yi4}{9,7,5,3}[Radicais ⼈、⼿、⼂、⼂]
  \definition{s.}{protecionismo}
\end{entry}

\begin{entry}{保留}{bao3liu2}{9,10}[HSK 3][Radicais ⼈、⽥]
  \definition{v.}{reter | continuar a ter | segurar | reservar}
\end{entry}

\begin{entry}{保密}{bao3mi4}{9,11}[HSK 4][Radicais ⼈、⼧]
  \definition{v.}{manter segredo; manter algo em segredo; manter a confidencialidade}
\end{entry}

\begin{entry}{保守}{bao3shou3}{9,6}[HSK 4][Radicais ⼈、⼧]
  \definition{adj.}{retrógrado; conservador; pensamentos e conceitos que são retrógrados e não conseguem acompanhar o desenvolvimento da situação}
  \definition{v.}{manter; guardar; evitar perder}
\end{entry}

\begin{entry}{保险}{bao3xian3}{9,9}[HSK 3][Radicais ⼈、⾩]
  \definition[个]{adj./s.}{seguro}
  \definition{v.}{ter certeza | estar vinculado a}
\end{entry}

\begin{entry}{保证}{bao3zheng4}{9,7}[HSK 3][Radicais ⼈、⾔]
  \definition[个]{s.}{garantia}
  \definition{v.}{garantir}
\end{entry}

\begin{entry}{报}{bao4}{7}[HSK 3][Radical ⼿]
  \definition[份,张]{s.}{jornal | recompensa | relatório | vingança}
  \definition{v.}{anunciar | informar}
\end{entry}

\begin{entry}{报酬}{bao4chou5}{7,13}[Radicais ⼿、⾣]
  \definition{s.}{recompensa | remuneração}
\end{entry}

\begin{entry}{报到}{bao4dao4}{7,8}[HSK 3][Radicais ⼿、⼑]
  \definition{v.+compl.}{apresentar-se para o serviço | fazer check-in | registrar-se | assinar}
\end{entry}

\begin{entry}{报道}{bao4dao4}{7,12}[HSK 3][Radicais ⼿、⾡]
  \definition[个,篇,分]{s.}{história | reportagem}
  \definition{v.}{cobrir | relatar (notícias)}
\end{entry}

\begin{entry}{报告}{bao4gao4}{7,7}[HSK 3][Radicais ⼿、⼝]
  \definition[份,篇,分,个,通]{s.}{relatório | discurso | palestra | aconselhamento}
  \definition{v.}{relatar | dar a conhecer | informar}
\end{entry}

\begin{entry}{报名}{bao4ming2}{7,6}[HSK 2][Radicais ⼿、⼝]
  \definition{v.+compl.}{matricular-se | alistar-se | inscrever-se | inserir o nome de alguém}
\end{entry}

\begin{entry}{报纸}{bao4zhi3}{7,7}[HSK 2][Radicais ⼿、⽷]
  \definition[张]{s.}{jornal | diário}
\end{entry}

\begin{entry}{抱}{bao4}{8}[HSK 4][Radical ⼿]
  \definition*{s.}{sobrenome Bao}
  \definition{clas.}{braçada; medida dos dois braços}
  \definition{v.}{carregar no peito; segurar com ambos os braços; abraçar | ter o primeiro filho ou neto | adotar um bebê ou criança | ficar juntos, unidos | encaixar ou servir perfeitamente (roupas e sapatos do tamanho certo) | estimar; nutrir; abrigar; ter em mente | continuar; sobrecarregar com | chocar ovos}
\end{entry}

\begin{entry}{抱怨}{bao4yuan4}{8,9}[Radicais ⼿、⼼]
  \definition{v.}{reclamar | resmungar | abrir uma reclamação | sentir-se insatisfeito}
\end{entry}

\begin{entry}{豹子}{bao4zi5}{10,3}[Radicais ⾘、⼦]
  \definition[头]{s.}{leopardo}
\end{entry}

\begin{entry}{暴力}{bao4li4}{15,2}[Radicais ⽇、⼒]
  \definition{adj.}{violento}
  \definition{s.}{violência}
\end{entry}

\begin{entry}{暴乱}{bao4luan4}{15,7}[Radicais ⽇、⼄]
  \definition{s.}{rebelião | revolta | tumulto}
\end{entry}

\begin{entry}{暴行}{bao4xing2}{15,6}[Radicais ⽇、⾏]
  \definition{s.}{ato selvagem | atrocidade | indignação}
\end{entry}

\begin{entry}{暴雨}{bao4yu3}{15,8}[Radicais ⽇、⾬]
  \definition[场,阵]{s.}{tempestade | chuva torrencial}
\end{entry}

\begin{entry}{暴躁}{bao4zao4}{15,20}[Radicais ⽇、⾜]
  \definition{adj.}{irascível | irritável}
\end{entry}

\begin{entry}{爆米花}{bao4mi3hua1}{19,6,7}[Radicais ⽕、⽶、⾋]
  \definition{s.}{pipoca (de milho) | pipoca de arroz}
\end{entry}

\begin{entry}{爆炸}{bao4zha4}{19,9}[Radicais ⽕、⽕]
  \definition{s.}{explosão}
  \definition{v.}{explodir | detonar}
\end{entry}

\begin{entry}{杯}{bei1}{8}[HSK 1][Radical ⽊]
  \definition{clas.}{para certos recipientes de líquidos: copo, xícara, etc.}
  \definition{s.}{copo | caneca | xícara | taça | troféu}
\end{entry}

\begin{entry}{杯具}{bei1ju4}{8,8}[Radicais ⽊、⼋]
  \definition{s.}{parachoque | fiasco | (gíria) tragédia}
\end{entry}

\begin{entry}{杯子}{bei1 zi5}{8,3}[HSK 1][Radicais ⽊、⼦]
  \definition[个,只]{s.}{copo | caneca | xícara | taça}
\end{entry}

\begin{entry}{背}{bei1}{9}[HSK 2][Radical ⾁]
  \definition{v.}{estar sobrecarregado | carregar nas costas ou no ombro}
  \seeref{背}{bei4}
\end{entry}

\begin{entry}{北}{bei3}{5}[HSK 1][Radical ⼔]
  \definition{s.}{norte}
  \definition{v.}{(clássico) ser derrotado}
\end{entry}

\begin{entry}{北边}{bei3 bian1}{5,5}[HSK 1][Radicais ⼔、⾡]
  \definition{adv.}{lado norte | ao norte de}
\end{entry}

\begin{entry}{北部}{bei3 bu4}{5,10}[HSK 3][Radicais ⼔、⾢]
  \definition{s.}{parte norte}
\end{entry}

\begin{entry}{北大西洋公约组织}{bei3 da4xi1 yang2 gong1 yue1 zu3zhi1}{5,3,6,9,4,6,8,8}[Radicais ⼔、⼤、⾑、⽔、⼋、⽷、⽷、⽷]
  \definition*{s.}{Organização do Tratado do Atlântico Norte, OTAN}
\end{entry}

\begin{entry}{北方}{bei3fang1}{5,4}[HSK 2][Radicais ⼔、⽅]
  \definition{s.}{norte | a parte norte de um país}
\end{entry}

\begin{entry}{北极}{bei3ji2}{5,7}[Radicais ⼔、⽊]
  \definition*{s.}{Ártico | Pólo Norte}
  \definition{s.}{pólo norte magnético}
\end{entry}

\begin{entry}{北京}{bei3 jing1}{5,8}[HSK 1][Radicais ⼔、⼇]
  \definition*{s.}{Beijing (Pequim), Capital da República Popular da China | Beijing (Pequim), governo da RPC}
\end{entry}

\begin{entry}{北面}{bei3mian4}{5,9}[Radicais ⼔、⾯]
  \definition{s.}{lado norte}
\end{entry}

\begin{entry}{北约}{bei3yue1}{5,6}[Radicais ⼔、⽷]
  \definition*{s.}{OTAN (Organização do Tratado do Atlântico Norte), abreviação de 北大西洋公约组织}
  \seeref{北大西洋公约组织}{bei3 da4xi1 yang2 gong1 yue1 zu3zhi1}
\end{entry}

\begin{entry}{备份}{bei4fen4}{8,6}[Radicais ⼡、⼈]
  \definition{s.}{cópia de segurança | \emph{backup}}
\end{entry}

\begin{entry}{备胎}{bei4tai1}{8,9}[Radicais ⼡、⾁]
  \definition{s.}{pneu sobressalente | (gíria) substituto}
\end{entry}

\begin{entry}{背}{bei4}{9}[HSK 3][Radical ⾁]
  \definition{adv.}{a parte de trás de um corpo ou objeto}
  \definition{s.}{costas | (gíria) azarado}
  \definition{v.}{esconder algo de | decorar | recitar de memória | virar as costas}
  \seeref{背}{bei1}
\end{entry}

\begin{entry}{背后}{bei4 hou4}{9,6}[HSK 3][Radicais ⾁、⼝]
  \definition{s.}{parte de trás | traseira | nas costas de alguém}
\end{entry}

\begin{entry}{背景}{bei4jing3}{9,12}[HSK 4][Radicais ⾁、⽇]
  \definition[种]{s.}{pano de fundo; fundo; cenário de teatro, filme ou drama de TV | fundo; cenário que permeia a imagem principal na tela | condições sociais; ambientes históricos (significativamente influentes para algo ou alguém) | poder que dá suporte a alguém}
\end{entry}

\begin{entry}{倍}{bei4}{10}[HSK 4][Radical ⼈]
  \definition{adv.}{mais; especialmente}
  \definition{clas.}{vezes; para obter um número igual ao número original, você pode multiplicar o número por esse múltiplo}
  \definition{s.}{dobro; duas vezes mais}
\end{entry}

\begin{entry}{被}{bei4}{10}[HSK 3][Radical ⾐]
  \definition*{s.}{sobrenome Bei}
  \definition{part.}{usada antes de verbos para formar frases verbais passivas}
  \definition{prep.}{usado em uma frase para indicar que o sujeito é o receptor da ação}
  \definition{s.}{colcha}
  \definition{v.}{cobrir; espalhar
sofrer}
\end{entry}

\begin{entry}{被单}{bei4dan1}{10,8}[Radicais ⾐、⼗]
  \definition[床]{s.}{lençol}
\end{entry}

\begin{entry}{被动}{bei4dong4}{10,6}[Radicais ⾐、⼒]
  \definition{adj.}{passivo}
\end{entry}

\begin{entry}{被告}{bei4gao4}{10,7}[Radicais ⾐、⼝]
  \definition{s.}{réu}
\end{entry}

\begin{entry}{被迫}{bei4 po4}{10,8}[HSK 4][Radicais ⾐、⾡]
  \definition{v.}{ser forçado; ser coagido; ser compelido; ser constrangido; ser forçado a fazer algo por força externa}
\end{entry}

\begin{entry}{被套}{bei4tao4}{10,10}[Radicais ⾐、⼤]
  \definition{s.}{capa de \emph{edredon}}
  \definition{v.}{ter dinheiro preso (em ações, imóveis, etc.)}
\end{entry}

\begin{entry}{被窝}{bei4wo1}{10,12}[Radicais ⾐、⽳]
  \definition{s.}{colcha}
\end{entry}

\begin{entry}{被子}{bei4zi5}{10,3}[HSK 3][Radicais ⾐、⼦]
  \definition[床]{s.}{colcha}
\end{entry}

\begin{entry}{本}{ben3}{5}[HSK 1][Radical ⽊]
  \definition{adj.}{o atual | original | inerente}
  \definition{adv.}{originalmente}
  \definition{clas.}{para livros, dicionários, periódicos, arquivos, etc.}
  \definition{s.}{raiz | caule | origem | fonte}
\end{entry}

\begin{entry}{本金}{ben3 jin1}{5,8}[Radicais ⽊、⾦]
  \definition{s.}{capital; capital para a operação do comércio e da indústria; capital para a operação de negócios |
valor principal; dinheiro retirado ao depositar ou tomar emprestado (diferente de ``利息'')}
  \seealsoref{利息}{li4xi1}
\end{entry}

\begin{entry}{本科}{ben3ke1}{5,9}[HSK 4][Radicais ⽊、⽲]
  \definition{s.}{graduação; bacharelado; o curso básico de uma universidade ou faculdade}
\end{entry}

\begin{entry}{本来}{ben3lai2}{5,7}[HSK 3][Radicais ⽊、⽊]
  \definition{adv.}{originalmente | apropriadamente | legalmente}
\end{entry}

\begin{entry}{本领}{ben3 ling3}{5,11}[HSK 3][Radicais ⽊、⾴]
  \definition[项,个]{s.}{capacidade | faculdade | poder | habilidade | talento}
\end{entry}

\begin{entry}{本事}{ben3shi4}{5,8}[Radicais ⽊、⼅]
  \definition{s.}{habilidade | capacidade | \emph{status} | poder | posição | autoridade}
  \seeref{本事}{ben3shi5}
\end{entry}

\begin{entry}{本事}{ben3shi5}{5,8}[HSK 3][Radicais ⽊、⼅]
  \definition{s.}{habilidade | capacidade |\emph{status} | poder | posição | autoridade}
  \seeref{本事}{ben3shi4}
\end{entry}

\begin{entry}{本子}{ben3 zi5}{5,3}[HSK 1][Radicais ⽊、⼦]
  \definition[本]{s.}{caderno}
\end{entry}

\begin{entry}{笨}{ben4}{11}[HSK 4][Radical ⽵]
  \definition{adj.}{estúpido; sem graça; tolo; de pouca habilidade; sem inteligência | desajeitado; tosco; inflexível | incômodo; pesado; desajeitado; difícil de manejar; trabalhoso}
\end{entry}

\begin{entry}{笨蛋}{ben4dan4}{11,11}[Radicais ⽵、⾍]
  \definition{s.}{bobalhão | cabeça-oca | cabeça-dura}
  \definition{v.}{iludir | enganar}
\end{entry}

\begin{entry}{崩}{beng1}{11}[Radical ⼭]
  \definition{s.}{morte de rei ou imperador | desaparecimento}
  \definition{v.}{entrar em colapso | cair em ruínas}
\end{entry}

\begin{entry}{绷带}{beng1dai4}{11,9}[Radicais ⽷、⼱]
  \definition{s.}{curativo | (empréstimo linguístico) \emph{bandage}}
\end{entry}

\begin{entry}{甭}{beng2}{9}[Radical ⽤]
  \definition{v.}{contração de 不用 | não precisar}
  \seeref{不用}{bu2 yong4}
\end{entry}

\begin{entry}{蹦极}{beng4ji2}{18,7}[Radicais ⾜、⽊]
  \definition{s.}{\emph{bungee jumping}}
\end{entry}

\begin{entry}{鼻子}{bi2zi5}{14,3}[Radicais ⿐、⼦]
  \definition[个,只]{s.}{nariz}
\end{entry}

\begin{entry}{比}{bi3}{4}[HSK 1][Kangxi 81][Radical ⽐]
  \definition*{s.}{Bélgica, abreviação de 比利时}
  \definition{part.}{partícula usada para comparação (superioridade)}
  \definition{prep.}{que | do que | (seguido por um substantivo e adjetivo) mais \{adj.\} do que \{s.\}}
  \definition{s.}{razão (taxa)}
  \definition{v.}{comparar | contrastar | gesticular (com as mãos)}
  \seeref{比利时}{bi3li4shi2}
\end{entry}

\begin{entry}{比分}{bi3 fen1}{4,4}[HSK 4][Radicais ⽐、⼑]
  \definition{s.}{pontuação; comparação de pontuações entre as duas equipes em uma partida}
\end{entry}

\begin{entry}{比较}{bi3jiao4}{4,10}[HSK 3][Radicais ⽐、⾞]
  \definition{adv.}{comparativamente | relativamente}
  \definition{s.}{comparação}
  \definition{v.}{comparar}
\end{entry}

\begin{entry}{比利时}{bi3li4shi2}{4,7,7}[Radicais ⽐、⼑、⽇]
  \definition*{s.}{Bélgica}
\end{entry}

\begin{entry}{比例}{bi3li4}{4,8}[HSK 3][Radicais ⽐、⼈]
  \definition{s.}{escala | razão | proporção}
\end{entry}

\begin{entry}{比拼}{bi3pin1}{4,9}[Radicais ⽐、⼿]
  \definition{s.}{concurso}
  \definition{v.}{competir ferozmente}
\end{entry}

\begin{entry}{比如}{bi3ru2}{4,6}[HSK 2][Radicais ⽐、⼥]
  \definition{conj.}{por exemplo | como}
\end{entry}

\begin{entry}{比如说}{bi3 ru2 shuo1}{4,6,9}[HSK 2][Radicais ⽐、⼥、⾔]
  \definition{adv.}{por exemplo}
\end{entry}

\begin{entry}{比萨饼}{bi3sa4bing3}{4,11,9}[Radicais ⽐、⾋、⾷]
  \definition[张]{s.}{pizza}
\end{entry}

\begin{entry}{比赛}{bi3sai4}{4,14}[HSK 3][Radicais ⽐、⾙]
  \definition[场,次]{s.}{competição | concurso}
  \definition{v.}{competir}
\end{entry}

\begin{entry}{比亚迪}{bi3ya4di2}{4,6,8}[Radicais ⽐、⼆、⾡]
  \definition*{s.}{Montadora BYD}
\end{entry}

\begin{entry}{笔}{bi3}{10}[HSK 2][Radical ⽵]
  \definition{clas.}{para somas de dinheiro, negócios}
  \definition[支,枝]{s.}{caneta | lápis}
\end{entry}

\begin{entry}{笔记}{bi3 ji4}{10,5}[HSK 2][Radicais ⽵、⾔]
  \definition[篇,本,个]{s.}{notas | ensaios | esboços}
  \definition{v.}{tomar nota (por escrito)}
\end{entry}

\begin{entry}{笔记本}{bi3ji4ben3}{10,5,5}[HSK 2][Radicais ⽵、⾔、⽊]
  \definition[本]{s.}{caderno}
  \definition{s.}{\emph{laptop}}
\end{entry}

\begin{entry}{必定}{bi4ding4}{5,8}[Radicais ⼼、⼧]
  \definition{adv.}{sem falta | certamente | com certeza | definitivamente | inevitavelmente | com determinação}
  \definition{v.}{estar vinculado a | ter certeza de}
\end{entry}

\begin{entry}{必然}{bi4ran2}{5,12}[HSK 3][Radicais ⼼、⽕]
  \definition{adj.}{certo | inevitável | necessário}
  \definition{adv.}{inevitavelmente}
  \definition{s.}{necessidade}
\end{entry}

\begin{entry}{必须}{bi4xu1}{5,9}[HSK 2][Radicais ⼼、⾴]
  \definition{adv.}{necessariamente | obrigatoriamente}
\end{entry}

\begin{entry}{必要}{bi4yao4}{5,9}[HSK 3][Radicais ⼼、⾑]
  \definition{adj.}{necessário | essencial | indispensável}
  \definition[个,些]{s.}{necessidade}
\end{entry}

\begin{entry}{毕业}{bi4ye4}{6,5}[HSK 4][Radicais ⽐、⼀]
  \definition{s.}{formatura}
  \definition{v.+compl.}{formar-se}
\end{entry}

\begin{entry}{毕业生}{bi4 ye4 sheng1}{6,5,5}[HSK 4][Radicais ⽐、⼀、⽣]
  \definition[个]{s.}{diplomado; graduado; bacharel; pessoa que recebeu um diploma, grau ou certificado}
\end{entry}

\begin{entry}{闭嘴}{bi4zui3}{6,16}[Radicais ⾨、⼝]
  \definition{expr.}{Cale-se!}
\end{entry}

\begin{entry}{壁虎}{bi4hu3}{16,8}[Radicais ⼟、⾌]
  \definition{s.}{lagartixa}
\end{entry}

\begin{entry}{壁纸}{bi4zhi3}{16,7}[Radicais ⼟、⽷]
  \definition{s.}{papel de parede}
\end{entry}

\begin{entry}{避}{bi4}{16}[HSK 4][Radical ⾌]
  \definition{v.}{evitar; evadir; esquivar-se; buscar abrigo; fugir | impedir; manter afastado; repelir; previnir}
\end{entry}

\begin{entry}{避免}{bi4mian3}{16,7}[HSK 4][Radicais ⾌、⼉]
  \definition{v.}{evitar; desviar; abster-se de; tentar não fazer com que algo aconteça; prevenir; tentar impedir (que algo ruim aconteça) com antecedência}
\end{entry}

\begin{entry}{边}{bian1}{5}[HSK 2][Radical ⾡]
  \definition{adv.}{simultaneamente}
  \definition[个]{s.}{fronteira | limite | borda | margem | lado}
  \seeref{边}{bian5}
\end{entry}

\begin{entry}{边防}{bian1fang2}{5,6}[Radicais ⾡、⾩]
  \definition{s.}{defesa da fronteira}
\end{entry}

\begin{entry}{边关}{bian1guan1}{5,6}[Radicais ⾡、⼋]
  \definition{s.}{posto de fronteira | posição defensiva estratégica na fronteira}
\end{entry}

\begin{entry}{编}{bian1}{12}[HSK 4][Radical ⽷]
  \definition*{s.}{sobrenome Bian}
  \definition{s.}{livro; volume; parte de um livro}
  \definition{v.}{tecer; trançar; entrançar | fazer uma lista; organizar em uma lista; organizar; agrupar | editar; compilar | compor; escrever | fabricar; inventar; fazer; preparar}
\end{entry}

\begin{entry}{编程}{bian1cheng2}{12,12}[Radicais ⽷、⽲]
  \definition{s.}{programa de computador}
  \definition{v.}{programar computador}
\end{entry}

\begin{entry}{邉}{bian1}{17}[Radical ⾡]
  \variantof{边}
\end{entry}

\begin{entry}{变}{bian4}{8}[HSK 2][Radical ⼜]
  \definition{v.}{mudar | transformar | variar}
\end{entry}

\begin{entry}{变成}{bian4 cheng2}{8,6}[HSK 2][Radicais ⼜、⼽]
  \definition{v.}{mudar | transformar-se em | tornar-se}
\end{entry}

\begin{entry}{变更}{bian4geng1}{8,7}[Radicais ⼜、⽈]
  \definition{v.}{alterar | mudar | modificar}
\end{entry}

\begin{entry}{变化}{bian4hua4}{8,4}[HSK 3][Radicais ⼜、⼔]
  \definition[个]{s.}{mudança | variação}
  \definition{v.}{(intransitivo) mudar, variar}
\end{entry}

\begin{entry}{变节}{bian4jie2}{8,5}[Radicais ⼜、⾋]
  \definition{s.}{traição | deserção | vira-casaca}
  \definition{v.}{mudar de lado politicamente}
\end{entry}

\begin{entry}{变迁}{bian4qian1}{8,6}[Radicais ⼜、⾡]
  \definition{s.}{mudanças | vicissitudes}
\end{entry}

\begin{entry}{变数}{bian4shu4}{8,13}[Radicais ⼜、⽁]
  \definition{s.}{(matemática) variável}
\end{entry}

\begin{entry}{变为}{bian4 wei2}{8,4}[HSK 3][Radicais ⼜、⼂]
  \definition{v.}{transformar-se em | tornar-se | mudar para}
\end{entry}

\begin{entry}{变心}{bian4xin1}{8,4}[Radicais ⼜、⼼]
  \definition{v.+compl.}{deixar de ser fiel}
\end{entry}

\begin{entry}{变性}{bian4xing4}{8,8}[Radicais ⼜、⼼]
  \definition{s.}{desnaturação | transexual}
  \definition{v.}{desnaturar | mudar de sexo}
\end{entry}

\begin{entry}{变异}{bian4yi4}{8,6}[Radicais ⼜、⼶]
  \definition{s.}{variação | mutação}
\end{entry}

\begin{entry}{变装}{bian4zhuang1}{8,12}[Radicais ⼜、⾐]
  \definition{v.}{trocar de roupa | vestir-se | vestir uma fantasia | disfarçar-se ou fantasiar-se de personagem real ou ficcional, \emph{cosplay} | travestir-se}
\end{entry}

\begin{entry}{遍}{bian4}{12}[HSK 2][Radical ⾡]
  \definition{adv.}{em todos os lugares | por toda parte}
  \definition{clas.}{para a repetição de ações de leitura, fala ou escrita}
\end{entry}

\begin{entry}{辩论}{bian4lun4}{16,6}[HSK 4][Radicais ⾟、⾔]
  \definition[场,次]{s.}{debate; argumento; a atividade comportamental em si de argumentar ou refutar diferentes pontos de vista ou afirmações, ou uma ocasião ou situação em que tal argumentação ou refutação é feita}
  \definition{v.}{debater; obter um entendimento unificado ou correto, ambos os lados usam linguagem, palavras etc. para explicar seus pontos de vista, apontar os erros ou as contradições do outro lado}
\end{entry}

\begin{entry}{辫子}{bian4zi5}{17,3}[Radicais ⾟、⼦]
  \definition[根,条]{s.}{trança | um erro ou falha que pode ser explorado por um oponente | alça}
\end{entry}

\begin{entry}{边}{bian5}{5}[Radical ⾡]
  \definition{suf.}{sufixo de uma palavra de localidade}
  \seeref{边}{bian1}
\end{entry}

\begin{entry}{标题}{biao1ti2}{9,15}[HSK 3][Radicais ⽊、⾴]
  \definition[个,条,篇]{s.}{título | manchete | cabeçalho}
\end{entry}

\begin{entry}{标志}{biao1zhi4}{9,7}[HSK 4][Radicais ⽊、⼼]
  \definition[个,种]{s.}{sinal; marca; logotipo; símbolo; emblema; marcações que caracterizam um objeto para facilitar a identificação}
  \definition{v.}{marcar; indicar; simbolizar; identificar}
\end{entry}

\begin{entry}{标准}{biao1zhun3}{9,10}[HSK 3][Radicais ⽊、⼎]
  \definition{adj.}{criterioso | padronizado | normatizado}
  \definition[个]{s.}{critério | padrão (oficial) | norma}
\end{entry}

\begin{entry}{镖}{biao1}{16}[Radical ⾦]
  \definition{s.}{dardo | arma de arremesso | mercadorias enviadas sob a proteção de uma escolta armada}
\end{entry}

\begin{entry}{表}{biao3}{8}[HSK 2][Radical ⾐]
  \definition*{s.}{sobrenome Biao}
  \definition{s.}{superfície externa | a relação entre os filhos ou netos de um irmão e uma irmã ou de irmãs | exemplo | modelo | memorial a um imperador dos tempos antigos | gráfico | formulário | lista | tabela | medidor | relógio de pulso}
\end{entry}

\begin{entry}{表白}{biao3bai2}{8,5}[Radicais ⾐、⽩]
  \definition{s.}{declaração | confissão}
  \definition{v.}{confessar a si mesmo | expressar | revelar pensamentos ou sentimentos de alguém}
\end{entry}

\begin{entry}{表达}{biao3da2}{8,6}[HSK 3][Radicais ⾐、⾡]
  \definition{v.}{entregar | expressar | mostrar | transmitir | comunicar}
\end{entry}

\begin{entry}{表格}{biao3ge2}{8,10}[HSK 3][Radicais ⾐、⽊]
  \definition[份,张]{s.}{tabela | formulário}
\end{entry}

\begin{entry}{表面}{biao3mian4}{8,9}[HSK 3][Radicais ⾐、⾯]
  \definition{s.}{superfície | lado de fora | aparência | superficialidade}
\end{entry}

\begin{entry}{表明}{biao3ming2}{8,8}[HSK 3][Radicais ⾐、⽇]
  \definition{v.}{deixar claro | tornar conhecido | declarar claramente}
\end{entry}

\begin{entry}{表情}{biao3qing2}{8,11}[HSK 4][Radicais ⾐、⼼]
  \definition[个,种,幅]{s.}{expressão; expressão facial; expressão de pensamentos e sentimentos internos por meio de mudanças faciais ou de gestos}
  \definition{v.}{expressar pensamentos e sentimentos internos por meio de mudanças faciais ou de gestos}
\end{entry}

\begin{entry}{表示}{biao3shi4}{8,5}[HSK 2][Radicais ⾐、⽰]
  \definition{s.}{expressão | indicação}
  \definition{v.}{expressar | mostrar | indicar | significar}
\end{entry}

\begin{entry}{表现}{biao3xian4}{8,8}[HSK 3][Radicais ⾐、⾒]
  \definition[个,种,份]{s.}{desempenho | expressão  manifestação | comportamento}
  \definition{v.}{mostrar | expressar | exibir | manifestar | descrever}
\end{entry}

\begin{entry}{表演}{biao3yan3}{8,14}[HSK 3][Radicais ⾐、⽔]
  \definition[场]{s.}{representação | atuação | exposição}
  \definition{v.}{executar | atuar | jogar | demonstrar | agir | fingir}
\end{entry}

\begin{entry}{表演赛}{biao3yan3sai4}{8,14,14}[Radicais ⾐、⽔、⾙]
  \definition{s.}{partida ou jogo de exibição}
\end{entry}

\begin{entry}{表演特技}{biao3yan3 te4ji4}{8,14,10,7}[Radicais ⾐、⽔、⽜、⼿]
  \definition{s.}{acrobacia | pirueta | façanha}
\end{entry}

\begin{entry}{表演艺术}{biao3yan3 yi4shu4}{8,14,4,5}[Radicais ⾐、⽔、⾋、⽊]
  \definition{s.}{arte performática}
\end{entry}

\begin{entry}{表演游戏}{biao3yan3 you2xi4}{8,14,12,6}[Radicais ⾐、⽔、⽔、⼽]
  \definition{s.}{exibição dramática}
\end{entry}

\begin{entry}{表演者}{biao3yan3 zhe3}{8,14,8}[Radicais ⾐、⽔、⽼]
  \definition{s.}{ator}
\end{entry}

\begin{entry}{表扬}{biao3yang2}{8,6}[HSK 4][Radicais ⾐、⼿]
  \definition[次,种,份]{s.}{elogios públicos por boas ações}
  \definition{v.}{elogiar; louvar}
\end{entry}

\begin{entry}{表扬信}{biao3yang2 xin4}{8,6,9}[Radicais ⾐、⼿、⼈]
  \definition{s.}{carta de elogio | depoimento}
\end{entry}

\begin{entry}{别}{bie2}{7}[HSK 1,4][Radical ⼑]
  \definition*{s.}{sobrenome Bie}
  \definition{adv.}{não; nada de (pedir a alguém para não fazer); é melhor não | talvez, usado em conjunto com a palavra ``是'' para indicar especulação.}
  \definition{pron.}{outro; algum outro}
  \definition{s.}{distinção; diferença | classificação}
  \definition{v.}{deixar; partir; separar | diferenciar; distinguir; encontrar aspectos diferentes | fixar objetos com pinos | girar; virar | aderir; colar; preder}
  \seeref{别}{bie4}
  \seealsoref{是}{shi4}
\end{entry}

\begin{entry}{别的}{bie2 de5}{7,8}[HSK 1][Radicais ⼑、⽩]
  \definition{pron.}{outro}
\end{entry}

\begin{entry}{别人}{bie2ren5}{7,2}[Radicais ⼑、⼈]
  \definition{pron.}{outra pessoa | outro povo | outros}
\end{entry}

\begin{entry}{别说}{bie2shuo1}{7,9}[Radicais ⼑、⾔]
  \definition{v.}{não falar de | não mencionar}
\end{entry}

\begin{entry}{别}{bie4}{7}[Radical ⼑]
  \definition{v.}{fazer com que alguém mude seus hábitos, opiniões, etc.}
  \seeref{别}{bie2}
\end{entry}

\begin{entry}{宾馆}{bin1guan3}{10,11}[Radicais ⼧、⾷]
  \definition[个,家]{s.}{casa de hóspedes | hotel}
\end{entry}

\begin{entry}{冰}{bing1}{6}[HSK 4][Radical ⼎]
  \definition{adj.}{frio (pessoa)| hostil}
  \definition[块]{s.}{gelo; água em estado sólido |  (gíria) metanfetamina}
  \definition{v.}{colocar gelo; colocar gelo ao redor; colocar no gelo; resfriar objetos com gelo ou água fria | sentir frio}
\end{entry}

\begin{entry}{冰糕}{bing1gao1}{6,16}[Radicais ⼎、⽶]
  \definition{s.}{sorvete | picolé}
\end{entry}

\begin{entry}{冰棍}{bing1gun4}{6,12}[Radicais ⼎、⽊]
  \definition[根]{s.}{picolé}
\end{entry}

\begin{entry}{冰激凌}{bing1ji1ling2}{6,16,10}[Radicais ⼎、⽔、⼎]
  \definition{s.}{sorvete}
\end{entry}

\begin{entry}{冰球}{bing1qiu2}{6,11}[Radicais ⼎、⽟]
  \definition{s.}{hóquei no gelo}
\end{entry}

\begin{entry}{冰天雪地}{bing1tian1-xue3di4}{6,4,11,6}[Radicais ⼎、⼤、⾬、⼟]
  \definition{expr.}{um mundo de gelo e neve}
\end{entry}

\begin{entry}{冰箱}{bing1xiang1}{6,15}[HSK 4][Radicais ⼎、⾋]
  \definition[台,个]{s.}{geladeira; freezer; refrigerador; aparelhos para congelar alimentos ou medicamentos com gelo para mantê-los frios}
\end{entry}

\begin{entry}{冰雪}{bing1 xue3}{6,11}[HSK 4][Radicais ⼎、⾬]
  \definition{adj.}{puro como gelo e neve; descreve uma pessoa como pura}
  \definition{s.}{gelo e neve}
\end{entry}

\begin{entry}{兵}{bing1}{7}[HSK 4][Radical ⼋]
  \definition[名]{s.}{armas; armamentos | soldado; pessoal militar | exército; tropas | soldado raso; membro mais jovem do exército | assuntos militares (estratégia) | peão, uma das peças do xadrez chinês}
\end{entry}

\begin{entry}{兵器}{bing1qi4}{7,16}[Radicais ⼋、⼝]
  \definition{s.}{armas | armamento}
\end{entry}

\begin{entry}{饼}{bing3}{9}[Radical ⾷]
  \definition[张]{s.}{panqueca | biscoito | torta}
\end{entry}

\begin{entry}{饼干}{bing3gan1}{9,3}[Radicais ⾷、⼲]
  \definition[片,块]{s.}{bolacha | biscoito}
\end{entry}

\begin{entry}{并}{bing4}{6}[HSK 3,4][Radical ⼲]
  \definition{adv.}{igualmente; simultaneamente; lado a lado; coisas diferentes existem ao mesmo tempo; coisas diferentes estão acontecendo ao mesmo tempo | em absoluto (usado antes de uma negativa para dar ênfase);  usado antes de uma palavra negativa para reforçar o tom e refutá-la ligeiramente}
  \definition{conj.}{além de; e}
  \definition{v.}{combinar; fundir; incorporar; anexar; juntar}
\end{entry}

\begin{entry}{并排}{bing4pai2}{6,11}[Radicais ⼲、⼿]
  \definition{adv.}{lado a lado}
\end{entry}

\begin{entry}{并且}{bing4qie3}{6,5}[HSK 3][Radicais ⼲、⼀]
  \definition{conj.}{além disso | o que é mais | e}
\end{entry}

\begin{entry}{幷}{bing4}{8}[Radical ⼲]
  \variantof{并}
\end{entry}

\begin{entry}{倂}{bing4}{10}[Radical ⼈]
  \variantof{并}
\end{entry}

\begin{entry}{病}{bing4}{10}[HSK 1][Radical ⽧]
  \definition[场]{s.}{doença}
  \definition{v.}{adoecer | estar doente}
\end{entry}

\begin{entry}{病人}{bing4 ren2}{10,2}[HSK 1][Radicais ⽧、⼈]
  \definition{s.}{doente | paciente}
\end{entry}

\begin{entry}{拨转}{bo1zhuan3}{8,8}[Radicais ⼿、⾞]
  \definition{v.}{transferir (fundos, etc.) | virar | dar a volta}
\end{entry}

\begin{entry}{波}{bo1}{8}[Radical ⽔]
  \definition*{s.}{Polônia, abreviação de 波兰}
  \definition{s.}{onda | ondulação | tempestade | surto}
  \seeref{波兰}{bo1lan2}
\end{entry}

\begin{entry}{波兰}{bo1lan2}{8,5}[Radicais ⽔、⼋]
  \definition*{s.}{Polônia}
\end{entry}

\begin{entry}{波音}{bo1yin1}{8,9}[Radicais ⽔、⾳]
  \definition*{s.}{Boeing (empresa aeroespacial)}
  \definition{s.}{mordente (música)}
\end{entry}

\begin{entry}{玻璃}{bo1li5}{9,14}[Radicais ⽟、⽟]
  \definition[张,塊]{s.}{vidro | (gíria) homossexual masculino}
\end{entry}

\begin{entry}{般}{bo1}{10}[Radical ⾈]
  \definition{s.}{utilizado em 般若 \dpy{bo1re3}}
  \seeref{般若}{bo1re3}
\end{entry}

\begin{entry}{般若}{bo1re3}{10,8}[Radicais ⾈、⾋]
  \definition*{s.}{Prajna (sânscrito), \emph{insight} sobre a verdadeira natureza da realidade | (Budismo) sabedoria}
\end{entry}

\begin{entry}{啵}{bo1}{11}[Radical ⼝]
  \definition{s.}{(onomatopéia) borbulhar}
  \seeref{啵}{bo5}
\end{entry}

\begin{entry}{菠菜}{bo1cai4}{11,11}[Radicais ⾋、⾋]
  \definition[棵]{s.}{espinafre}
\end{entry}

\begin{entry}{播出}{bo1 chu1}{15,5}[HSK 3][Radicais ⼿、⼐]
  \definition{v.}{transmitir | estar no ar}
\end{entry}

\begin{entry}{播放}{bo1fang4}{15,8}[HSK 3][Radicais ⼿、⽅]
  \definition{v.}{ir ao ar | transmitir por rádio | mostrar | transmitir (um programa de TV)}
\end{entry}

\begin{entry}{播音}{bo1yin1}{15,9}[Radicais ⼿、⾳]
  \definition{s.}{transmissão}
  \definition{v.+compl.}{transmitir}
\end{entry}

\begin{entry}{脖子}{bo2zi5}{11,3}[Radicais ⾁、⼦]
  \definition[个]{s.}{pescoço}
\end{entry}

\begin{entry}{博文}{bo2wen2}{12,4}[Radicais ⼗、⽂]
  \definition{s.}{artigo em um blog}
  \definition{v.}{escrever um artigo em um blog}
\end{entry}

\begin{entry}{博物馆}{bo2wu4guan3}{12,8,11}[Radicais ⼗、⽜、⾷]
  \definition{s.}{museu}
\end{entry}

\begin{entry}{博主}{bo2zhu3}{12,5}[Radicais ⼗、⼂]
  \definition{s.}{blogueiro}
\end{entry}

\begin{entry}{薄}{bo2}{16}[Radical ⾋]
  \definition{adj.}{ligeiro; escasso; pequeno | mesquinho; pouco generoso; cruel | frívolo; fútil; leviano}
  \seeref{薄}{bao2}
\end{entry}

\begin{entry}{啵}{bo5}{11}[Radical ⼝]
  \definition{part.}{partícula gramaticalmente equivalente a 吧}
  \seeref{啵}{bo1}
  \seealsoref{吧}{ba5}
\end{entry}

\begin{entry}{不}{bu2}[(antes de quarto tom)]{4}[HSK 1][Radical ⼀]
  \definition{adv.}{não}
  \definition{pref.}{prefixo negativo}
  \seeref{不}{bu4}
  \seeref{不}{bu5}
\end{entry}

\begin{entry}{不必}{bu2 bi4}{4,5}[HSK 3][Radicais ⼀、⼼]
  \definition{adv.}{não precisa | não tem que}
\end{entry}

\begin{entry}{不错}{bu2 cuo4}{4,13}[HSK 2][Radicais ⼀、⾦]
  \definition{adj.}{correto | não (é) mau | bastante bom | certo}
\end{entry}

\begin{entry}{不大}{bu2 da4}{4,3}[HSK 1][Radicais ⼀、⼤]
  \definition{adv.}{não muito | não frequentemente | raramente |dificilmente | escassamente}
\end{entry}

\begin{entry}{不大离}{bu2da4li2}{4,3,10}[Radicais ⼀、⼤、⼇]
  \definition{adj.}{bem perto | quase certo | nada mal}
\end{entry}

\begin{entry}{不但}{bu2 dan4}{4,7}[HSK 2][Radicais ⼀、⼈]
  \definition{conj.}{não somente}
\end{entry}

\begin{entry}{不但……而且……}{bu2 dan4 er2qie3}{4,7,6,5}[HSK 2][Radicais ⼀、⼈、⽽、⼀]
  \definition{conj.}{não só\dots mas também\dots}
\end{entry}

\begin{entry}{不到}{bu2dao4}{4,8}[Radicais ⼀、⼑]
  \definition{adj.}{insuficiente}
  \definition{adv.}{menos que}
  \definition{v.}{não chegar}
\end{entry}

\begin{entry}{不断}{bu2duan4}{4,11}[HSK 3][Radicais ⼀、⽄]
  \definition{adv.}{continuamente | sem fim}
\end{entry}

\begin{entry}{不对}{bu2 dui4}{4,5}[HSK 1][Radicais ⼀、⼨]
  \definition{adj.}{incorreto | errado | anormal | estranho | estar em desacordo com | ser difícil de conviver}
\end{entry}

\begin{entry}{不够}{bu2 gou4}{4,11}[HSK 2][Radicais ⼀、⼣]
  \definition{adv.}{insuficiente}
  \definition{v.}{não ser suficiente}
\end{entry}

\begin{entry}{不过}{bu2guo4}{4,6}[HSK 2][Radicais ⼀、⾡]
  \definition{conj.}{mas | contudo | no entanto}
\end{entry}

\begin{entry}{不客气}{bu2 ke4 qi5}{4,9,4}[HSK 1][Radicais ⼀、⼧、⽓]
  \definition{adj.}{indelicado | rude | brusco}
  \definition{expr.}{de nada | não há de que | não mencione isso}
\end{entry}

\begin{entry}{不论}{bu2 lun4}{4,6}[HSK 3][Radicais ⼀、⾔]
  \definition{conj.}{não importa (o que, quem, como, etc.) | se \dots ou \dots}
\end{entry}

\begin{entry}{不论……都……}{bu2lun4 dou1}{4,6,10}[Radicais ⼀、⾔、⾢]
  \definition{conj.}{não apenas\dots, (o que, quem, como, etc.), \dots}
\end{entry}

\begin{entry}{不论……也……}{bu2lun4 ye3}{4,6,3}[Radicais ⼀、⾔、⼄]
  \definition{conj.}{não apenas\dots, (o que, quem, como, etc.), \dots}
\end{entry}

\begin{entry}{不日}{bu2ri4}{4,4}[Radicais ⼀、⽇]
  \definition{adv.}{em alguns dias}
\end{entry}

\begin{entry}{不是话}{bu2shi4hua4}{4,9,8}[Radicais ⼀、⽇、⾔]
  \definition{expr.}{sem razão | demasiado irracionável}
  \seeref{不像话}{bu2xiang4hua4}
  \seeref{不成话}{bu4cheng2hua4}
\end{entry}

\begin{entry}{不太}{bu2 tai4}{4,4}[HSK 2][Radicais ⼀、⼤]
  \definition{adv.}{não bastante | não muito}
\end{entry}

\begin{entry}{不像话}{bu2xiang4hua4}{4,13,8}[Radicais ⼀、⼈、⾔]
  \definition{expr.}{sem razão | demasiado irracionável}
  \seeref{不是话}{bu2shi4hua4}
  \seeref{不成话}{bu4cheng2hua4}
\end{entry}

\begin{entry}{不要}{bu2 yao4}{4,9}[HSK 2][Radicais ⼀、⾑]
  \definition{adv.}{nada de (pedir a alguém para não fazer) | não}
\end{entry}

\begin{entry}{不要紧}{bu2yao4jin3}{4,9,10}[HSK 4][Radicais ⼀、⾑、⽷]
  \definition{adj.}{sem importância; sem seriedade; não problemático | não importa; não é um obstáculo | parece estar tudo bem, mas | à primeira vista, isso não parece atrapalhar}
\end{entry}

\begin{entry}{不用}{bu2 yong4}{4,5}[HSK 1][Radicais ⼀、⽤]
  \definition{v.}{não precisar}
  \seeref{甭}{beng2}
\end{entry}

\begin{entry}{不在乎}{bu2 zai4 hu1}{4,6,5}[HSK 4][Radicais ⼀、⼟、⼃]
  \definition{v.}{não se importar; não dar a mínima; não dar atenção}
\end{entry}

\begin{entry}{不注意}{bu2zhu4yi4}{4,8,13}[Radicais ⼀、⽔、⼼]
  \definition{adj.}{impensado | distraído}
  \definition{s.}{descuido | distração}
\end{entry}

\begin{entry}{补}{bu3}{7}[HSK 3][Radical ⾐]
  \definition*{s.}{sobrenome Bu}
  \definition{s.}{benefício | ajuda | uso}
  \definition{v.}{consertar | remendar | preencher | adicionar suplemento | suprir | compensar |nutrir}
\end{entry}

\begin{entry}{补充}{bu3chong1}{7,6}[HSK 3][Radicais ⾐、⼉]
  \definition{adj.}{adicional | suplementar}
  \definition[个]{s.}{aditivo | suplemento}
  \definition{v.}{reabastecer | suplementar | complementar}
\end{entry}

\begin{entry}{不}{bu4}{4}[HSK 1][Radical ⼀]
  \definition{adv.}{não}
  \definition{pref.}{prefixo negativo}
  \seeref{不}{bu2}
  \seeref{不}{bu5}
\end{entry}

\begin{entry}{不安}{bu4'an1}{4,6}[HSK 3][Radicais ⼀、⼧]
  \definition{adj.}{inquieto | instável | intranquilo | pesaroso}
\end{entry}

\begin{entry}{不成话}{bu4cheng2hua4}{4,6,8}[Radicais ⼀、⼽、⾔]
  \definition{expr.}{sem razão | demasiado irracionável}
  \seeref{不是话}{bu2shi4hua4}
  \seeref{不像话}{bu2xiang4hua4}
\end{entry}

\begin{entry}{不得不}{bu4de2bu4}{4,11,4}[HSK 3][Radicais ⼀、⼻、⼀]
  \definition{adv.}{tem que | não tem escolha a não ser}
\end{entry}

\begin{entry}{不公}{bu4gong1}{4,4}[Radicais ⼀、⼋]
  \definition{adj.}{injusto}
\end{entry}

\begin{entry}{不管}{bu4guan3}{4,14}[HSK 4][Radicais ⼀、⽵]
  \definition{conj.}{não importa (o que, como, etc.); independentemente de; indica que, embora as condições ou circunstâncias tenham mudado, o resultado permanece o mesmo}
  \seeref{不管……都……}{bu4guan3 dou1}
  \seeref{不管……也……}{bu4guan3 ye3}
\end{entry}

\begin{entry}{不管……都……}{bu4guan3 dou1}{4,14,10}[Radicais ⼀、⽵、⾢]
  \definition{conj.}{não apenas\dots, (o que, quem, como, etc.), \dots}
\end{entry}

\begin{entry}{不管……也……}{bu4guan3 ye3}{4,14,3}[Radicais ⼀、⽵、⼄]
  \definition{conj.}{não apenas\dots, (o que, quem, como, etc.), \dots}
\end{entry}

\begin{entry}{不光}{bu4 guang1}{4,6}[HSK 3][Radicais ⼀、⼉]
  \definition{adv.}{não é o único}
  \definition{conj.}{não somente}
\end{entry}

\begin{entry}{不好意思}{bu4 hao3 yi4 si5}{4,6,13,9}[HSK 2][Radicais ⼀、⼥、⼼、⼼]
  \definition{adj.}{pedir desculpas (por incomodar alguém) | sentir-se envergonhado | achar isso embaraçoso}
\end{entry}

\begin{entry}{不仅}{bu4jin3}{4,4}[HSK 3][Radicais ⼀、⼈]
  \definition{adv.}{não apenas (em número, quantidade ou extensão)}
  \definition{conj.}{não somente}
\end{entry}

\begin{entry}{不久}{bu4 jiu3}{4,3}[HSK 2][Radicais ⼀、⼃]
  \definition{adj.}{em breve | futuro próximo | logo depois | não muito depois | não muito tempo (antes ou depois de algo)}
\end{entry}

\begin{entry}{不可避免}{bu4ke3bi4mian3}{4,5,16,7}[Radicais ⼀、⼝、⾌、⼉]
  \definition{adj./adv.}{inevitável}
\end{entry}

\begin{entry}{不满}{bu4 man3}{4,13}[HSK 2][Radicais ⼀、⽔]
  \definition{adj.}{ressentido | insatisfeito | descontente}
  \definition{v.}{estar descontente com |ser menor que}
\end{entry}

\begin{entry}{不然}{bu4ran2}{4,12}[HSK 4][Radicais ⼀、⽕]
  \definition{adj.}{não é assim; não é o caso}
  \definition{conj.}{se não; caso contrário; indica outra consequência ou circunstância que teria ocorrido se não fosse}
\end{entry}

\begin{entry}{不如}{bu4ru2}{4,6}[HSK 2][Radicais ⼀、⼥]
  \definition{conj.}{em vez de | melhor que | seria melhor}
  \definition{v.}{ser inferior a | não ser igual a | não ser tão bom quanto | não poder fazer melhor que}
\end{entry}

\begin{entry}{不少}{bu4 shao3}{4,4}[HSK 2][Radicais ⼀、⼩]
  \definition{adj.}{muitos | bastante | não poucos}
\end{entry}

\begin{entry}{不是……而是}{bu4shi4 er2 shi4}{4,9,6,9}[Radicais ⼀、⽇、⽽、⽇]
  \definition{conj.}{não somente\dots mas também\dots, expressam um relacionamento mais profundo e avançado em significado, mas as orações antes e depois são consistentes em expressar significados negativos e afirmativos, entretanto, a primeira metade da frase expressa negação, e a segunda metade expressa afirmação, e o significado das orações anteriores e seguintes não pode ser de um nível mais alto}
\end{entry}

\begin{entry}{不同}{bu4 tong2}{4,6}[HSK 2][Radicais ⼀、⼝]
  \definition{adj.}{diferente | distinto}
\end{entry}

\begin{entry}{不行}{bu4 xing2}{4,6}[HSK 2][Radicais ⼀、⾏]
  \definition{adj.}{não funciona | não é bom}
  \definition{adv.}{profundamente | terrivelmente | extremamente}
  \definition{v.}{não fazer | não ser permitido | estar fora de questão | estar à beira da morte}
\end{entry}

\begin{entry}{不一定}{bu4 yi2 ding4}{4,1,8}[HSK 2][Radicais ⼀、⼀、⼧]
  \definition{adv.}{talvez | incerto | não tenho certeza | não necessariamente}
\end{entry}

\begin{entry}{不一会儿}{bu4 yi2 hui4r5}{4,1,6,2}[HSK 2][Radicais ⼀、⼀、⼈、⼉]
  \definition{expr.}{em um momento | em pouco tempo |em breve}
\end{entry}

\begin{entry}{不止}{bu4zhi3}{4,4}[Radicais ⼀、⽌]
  \definition{adv.}{incessantemente | sem fim | mais que | não limitado a}
\end{entry}

\begin{entry}{布}{bu4}{5}[HSK 3][Radical ⼱]
  \definition*{s.}{sobrenome Bu}
  \definition[块,幅,匹]{s.}{pano | tecido | uma moeda de cobre nos tempos antigos}
  \definition{v.}{anunciar | declarar | tornar conhecido | proclamar | publicar | espalhar | disseminar |organizar | implantar | dispor}
\end{entry}

\begin{entry}{布谷鸟}{bu4gu3niao3}{5,7,5}[Radicais ⼱、⾕、⿃]
  \definition{s.}{cuco (pássaro)}
  \seealsoref{杜鹃}{du4juan1}
  \seealsoref{杜鹃鸟}{du4juan1niao3}
  \seealsoref{杜宇}{du4yu3}
\end{entry}

\begin{entry}{布署}{bu4shu3}{5,13}[Radicais ⼱、⽹]
  \variantof{部署}
\end{entry}

\begin{entry}{布置}{bu4zhi4}{5,13}[HSK 4][Radicais ⼱、⽹]
  \definition{v.}{arrumar; organizar; decorar; colocar adequadamente objetos ou paisagismo, conforme necessário | designar; tomar providências para; dar instruções sobre; organizar trabalho, atividades, etc.}
\end{entry}

\begin{entry}{步}{bu4}{7}[HSK 3][Radical ⽌]
  \definition*{s.}{sobrenome Bu}
  \definition{clas.}{uma unidade antiga para medida de comprimento, equivalente a cinco chi}
  \definition{s.}{ritmo | passo | estágio | passo | condição | situação | estado}
  \definition{v.}{ir a pé | andar | pisar | contar passos}
\end{entry}

\begin{entry}{步行}{bu4 xing2}{7,6}[HSK 4][Radicais ⽌、⾏]
  \definition{v.}{caminhar; ir a pé; andar a pé (diferente de andar de carro, a cavalo, etc.)}
\end{entry}

\begin{entry}{部}{bu4}{10}[HSK 3][Radical ⾢]
  \definition{clas.}{para obras de literatura, filmes, máquinas etc.}
  \definition[根]{s.}{departamento | divisão | ministério | seção | parte | tropas}
\end{entry}

\begin{entry}{部队}{bu4dui4}{10,4}[Radicais ⾢、⾩]
  \definition[个]{s.}{exército | forças armadas | tropas | unidades}
\end{entry}

\begin{entry}{部分}{bu4fen5}{10,4}[HSK 2][Radicais ⾢、⼑]
  \definition[个]{s.}{parte | parte de | uma parte de | pedaço | secção}
\end{entry}

\begin{entry}{部门}{bu4men2}{10,3}[HSK 3][Radicais ⾢、⾨]
  \definition[个]{s.}{filial | departamento | divisão | seção}
\end{entry}

\begin{entry}{部属}{bu4shu3}{10,12}[Radicais ⾢、⼫]
  \definition{s.}{afiliado a um ministério | subordinado | tropas sob comando de alguém}
\end{entry}

\begin{entry}{部署}{bu4shu3}{10,13}[Radicais ⾢、⽹]
  \definition{s.}{implantação}
  \definition{v.}{implantar}
\end{entry}

\begin{entry}{部下}{bu4xia4}{10,3}[Radicais ⾢、⼀]
  \definition{s.}{subordinado | tropas sob comando de alguém}
\end{entry}

\begin{entry}{部长}{bu4 zhang3}{10,4}[HSK 3][Radicais ⾢、⾧]
  \definition[个,位,名]{s.}{ministro | chefe de departamento | chefe de seção}
\end{entry}

\begin{entry}{部族}{bu4zu2}{10,11}[Radicais ⾢、⽅]
  \definition{adj.}{tribal}
  \definition{s.}{tribo}
\end{entry}

\begin{entry}{不}{bu5}{4}[HSK 1][Radical ⼀]
  \definition{adv.}{não (em expressões ``v.+不+v.'')}
  \seeref{不}{bu2}
  \seeref{不}{bu4}
\end{entry}

%%%%% EOF %%%%%


%%%
%%% C
%%%
\section*{C}
\addcontentsline{toc}{section}{C}
\begin{multicols}{2}

\entry{菜}{n.}{cai4}{hortaliça; verdura; prato}
\entry{菜单}{n.}{cai4dan1}{menu; ementa; cardápio}

\entry{草地}{n.}{cao3di4}{relva; pastagem}

\entry{参加}{v.}{can1jia1}{juntar; participar}

\entry{餐厅}{n.}{can1ting1}{cantina; sala de jantar}

\entry{厕所}{n.}{ce4suo3}{sanitário; toilette}

\entry{磁带}{n.}{ci2dai4}{cassete|\fbox{盘}}
\entry{磁盘}{n.}{ci2pan2}{disquete}

\entry{词典}{n.}{ci2dian3}{dicionário|\fbox{本}}

\entry{茶}{n.}{cha2}{chá}

\entry{长}{adj.}{chang2}{comprido; longo}
\entry{长成}{n.}{chang2cheng2}{Grande Muralha}

\entry{常常}{adv.}{chang2chang2}{frequentemente}

\entry{炒}{v.}{chao3}{saltear}

\entry{车}{n.}{che1}{veículo; viatura}
\entry{车牌}{n.}{che1pai2}{matrícula; placa de carro}
\entry{车站}{n.}{che1zhan4}{estação; paragem}

\entry{衬衫}{n.}{chen4shan1}{camisa|\fbox{件}}

\entry{成都}{n.}{cheng2du1}{Chengdu}

\entry{城市}{n.}{cheng2shi4}{cidade}

\entry{橙色}{n.}{cheng2se4}{cor de laranja}
\entry{橙汁}{n.}{cheng2zhi1}{suco de laranja}

\entry{惩罚}{v.}{cheng2fa2}{punir; penalizar}
\entry{惩处}{v.}{cheng2chu3}{punir; penalizar}

\entry{吃}{v.}{chi1}{comer}

\entry{迟到}{v.}{chi1dao4}{chegar atrasado; tardar}

\entry{憧憬}{v.}{chong1jing3}{ansiar por; esperar por}

\entry{宠物}{n.}{chong3wu4}{animal de estimação}

\entry{酬劳}{n.}{chou2lao2}{recompensa}

\entry{出}{v.d.}{chu1}{sair}
\entry{出去}{v.d.}{chu1qu0}{sair; ir para fora}
\entry{出口}{n.}{chu1kou3}{exportação}
\entry{出口}{v.}{chu1kou3}{exportar}
\entry{出站}{n.}{chu1zhan4}{saída da estação}
\entry{出租汽车}{n.}{chu1zu1qi4che1}{táxi|\fbox{辆}}

\entry{穿}{v.}{chuan1}{vestir}

\entry{船}{v.}{chuan2}{barco; navio}

\entry{传真}{n.}{chuan2zhen1}{fax; facsímile}

\entry{床}{n.}{chuang2}{cama|\fbox{张}}

\entry{春天}{n.}{chun1tian1}{primavera}

\entry{绰号}{n.}{chuo4hao4}{apelido}

\entry{聪明}{adj.}{cong1ming2}{inteligente; brilhante; esperto}
\entry{聪慧}{adj.}{cong1hui4}{inteligente; brilhante}

\entry{从}{prep.}{cong2}{de; desde; a partir de}

\entry{醋}{n.}{cu4}{vinagre}

\entry{错}{adj.}{cuo4}{errado; enganado}

\end{multicols}

\section*{D}
\addcontentsline{toc}{section}{D}
\begin{multicols}{2}
%%%
%%% D
%%%
% \entry{打}{v.}{jogar}
% \entry{打电话}{v.}{ligar; dar um telefonema}
% \entry{打算}{v.}{pretender}
% \entry{大}{adj.}{grande}
% \entry{大概}{adv.}{aproximadamente; por volta de}
% \entry{大海}{n.}{mar}
% \entry{大家}{pron.}{todos; todas}
% \entry{大学}{n.}{universidade}
% \entry{\xpinyin{大}{Da4}洋洲}{n.}{Oceania}
% \entry{带}{v.}{levar}
% \entry{戴}{v.}{usar; vestir}
% \entry{担心}{v.}{preocupar-se}
% \entry{蛋糕}{n.}{bolo}
% \entry{当然}{adv.}{claro; certamente; com certeza}
% \entry{到}{v.}{chegar}
% \entry{得}{v.}{ganhar; obter}
% \entry{\xpinyin{德}{De2}国}{n.}{Alemanha}
% \entry{的}{part.}{partícula utilizada em possessivos; partícula utilizada entre adjetivos e substantivos, opcional se substantivo possui apenas um caracter}
% \entry{地方}{n.}{lugar; local; sítio}
% \entry{等}{v.}{esperar}
% \entry{第}{num.}{prefixo para expressar números ordinais}
% \entry{弟\xpinyin{弟}{di0}}{n.}{irmão mais novo}
% \entry{\xpinyin{地}{di4}图}{n.}{mapa}
% \entry{点(钟)}{p.c.}{hora}
% \entry{(一)\xpinyin{点}{dianr3}\xpinyin{儿}{}}{p.c.}{um pouco}
% \entry{(商)店}{n.}{loja}
% \entry{电话}{n.}{telefone}
% \entry{电脑}{n.}{computador}
% \entry{电视}{n.}{televisor}
% \entry{电影}{n.}{cinema; filme}
% \entry{电子}{n.}{eletrônico; eletrônica)}
% \entry{电子邮件}{n.}{correio eletrônico; e-mail}
% \entry{东}{n.}{leste}
% \entry{\xpinyin{东}{Dong1}方}{n.}{Oriente}
% \entry{东天}{n.}{inverno}
% \entry{东\xpinyin{西}{xi0}}{n.}{coisa}
% \entry{都}{adv.}{todo; toda; todos; todas}
% \entry{读}{v.}{ler}
% \entry{度}{v.}{passar}
% \entry{锻炼}{v.}{fazer exercício físico}
% \entry{对}{adj.}{correto; sim}
% \entry{对不起}{}{desculpar; pedir desculpa; perdão}
% \entry{多}{adj.}{muito; muita; muitos; muitas}
% \entry{多大}{interr.}{quantos anos; que idade}
% \entry{多少}{interr.}{quanto; quanta; quantos; quantas}
% \entry{打算}{v./n.}{pensar; planear; plano}
% \entry{打球}{v.}{jogar à bola; jogar (futebol; basquetebol; handbol; etc)}
% \entry{冰球}{n.}{hóquei em gelo}
\end{multicols}

%%%
%%% E
%%%
\section*{E}
\addcontentsline{toc}{section}{E}
\begin{multicols}{2}
\entry{俄罗斯}{n.}{É\ luo2si1}{Rússia}
\entry{儿媳}{n.}{er2xi2}{esposa do filho}
\entry{儿子}{n.}{er2zi0}{filho}
\entry{二}{num.}{er4}{dois}
\end{multicols}

%%%
%%% F
%%%

\section*{F}\addcontentsline{toc}{section}{F}

\begin{verbete}{发}{fa1}{5}[Radical ⼜]
  \significado{clas.}{para tiros (rodadas)}
  \significado{v.}{enviar; mandar}
  \veja{发}{fa4}
\end{verbete}

\begin{verbete}{发表}{fa1biao3}{5,8}
  \significado{v.}{emitir; publicar}
\end{verbete}

\begin{verbete}{发财}{fa1cai2}{5,7}
  \significado{v.+compl.}{ficar rico; fazer fortuna}
\end{verbete}

\begin{verbete}{发愁}{fa1chou2}{5,13}
  \significado{v.+compl.}{preocupar-se; ficar ansioso; ficar triste}
\end{verbete}

\begin{verbete}{发动机}{fa1dong4ji1}{5,6,6}
  \significado[台]{s.}{motor}
\end{verbete}

\begin{verbete}{发抖}{fa1dou3}{5,7}
  \significado{v.}{tremer; sacudir; estremecer}
\end{verbete}

\begin{verbete}{发明者}{fa1ming2zhe3}{5,8,8}
  \significado{s.}{inventor}
\end{verbete}

\begin{verbete}{发票}{fa1piao4}{5,11}
  \significado{s.}{fatura; recibo; conta}
\end{verbete}

\begin{verbete}{发烧}{fa1shao1}{5,10}
  \significado{v.}{ter febre}
\end{verbete}

\begin{verbete}{发生}{fa1sheng1}{5,5}
  \significado{v.}{acontecer; ocorrer}
\end{verbete}

\begin{verbete}{发现}{fa1xian4}{5,8}
  \significado{s.}{descoberta}
  \significado{v.}{perceber, tornar-se ciente de; descobrir, encontrar, detectar}
\end{verbete}

\begin{verbete}{发现者}{fa1xian4 zhe3}{5,8,8}
  \significado{s.}{descobridor}
\end{verbete}

\begin{verbete}{发音}{fa1yin1}{5,9}
  \significado{s.}{pronúncia}
  \significado{v.}{pronunciar}
\end{verbete}

\begin{verbete}{发展}{fa1zhan3}{5,10}
  \significado{s.}{desenvolvimento}
  \significado{v.}{desenvolver}
\end{verbete}

\begin{verbete}{罚}{fa2}{9}[Radical 网]
  \significado{v.}{castigar; punir}
\end{verbete}

\begin{verbete}{罚款}{fa2kuan3}{9,12}
  \significado{s.}{multa (monetária); pena}
  \significado{v.+compl.}{aplicar uma multa; multar}
\end{verbete}

\begin{verbete}{筏}{fa2}{12}[Radical 竹]
  \significado{s.}{jangada (de troncos, bambus, etc.)}
\end{verbete}

\begin{verbete}{法}{fa3}{8}[Radical 水]
  \significado*{s.}{França, abreviação de~法国}
  \veja{法国}{fa3guo2}
\end{verbete}

\begin{verbete}{法国}{fa3guo2}{8,8}
  \significado*{s.}{França}
  \veja{法}{fa3}
\end{verbete}

\begin{verbete}{法国人}{fa3guo2ren2}{8,8,2}
  \significado{s.}{francês; nascido na França}
\end{verbete}

\begin{verbete}{法网}{fa3wang3}{8,6}
  \significado*{s.}{Torneio de Roland Garros (French Open), torneio de tênis}
\end{verbete}

\begin{verbete}{法文}{fa3wen2}{8,4}
  \significado*{s.}{françês, língua francesa}
\end{verbete}

\begin{verbete}{法语}{fa3yu3}{8,9}
  \significado{s.}{françês, língua francesa}
\end{verbete}

\begin{verbete}{发}{fa4}{5}[Radical ⼜]
  \significado{s.}{cabelo}
  \veja{发}{fa1}
\end{verbete}

\begin{verbete}{发型}{fa4xing2}{5,9}
  \significado{s.}{penteado}
\end{verbete}

\begin{verbete}{发簪}{fa4zan1}{5,18}
  \significado{s.}{grampo de cabelo}
\end{verbete}

\begin{verbete}{番茄}{fan1qie2}{12,8}
  \significado{s.}{tomate}
\end{verbete}

\begin{verbete}{蕃茄}{fan1qie2}{15,8}
  \variante{番茄}
\end{verbete}

\begin{verbete}{翻过}{fan1guo4}{18,6}
  \significado{v.}{virar; transformar}
\end{verbete}

\begin{verbete}{翻脸}{fan1lian3}{18,11}
  \significado{v.+compl.}{brigar com alguém; tornar-se hostil}
\end{verbete}

\begin{verbete}{翻译}{fan1yi4}{18,7}
  \significado[个,位,名]{s.}{tradução; tradutor; interpretação; intérprete}
  \significado{v.}{traduzir; interpretar}
\end{verbete}

\begin{verbete}{反对}{fan3dui4}{4,5}
  \significado{v.}{contrariar; opor-se; lutar contra}
\end{verbete}

\begin{verbete}{反对党}{fan3dui4dang3}{4,5,10}
  \significado{s.}{partido de oposição}
\end{verbete}

\begin{verbete}{反对派}{fan3dui4pai4}{4,5,9}
  \significado{s.}{facção de oposição}
\end{verbete}

\begin{verbete}{反对票}{fan3dui4piao4}{4,5,11}
  \significado{s.}{voto dissidente}
\end{verbete}

\begin{verbete}{反复}{fan3fu4}{4,9}
  \significado{adv.}{de novo e de novo; repetidamente}
\end{verbete}

\begin{verbete}{反省}{fan3xing3}{4,9}
  \significado{v.}{examinar a consciência; questionar-se; refletir sobre si mesmo; sondar a alma}
\end{verbete}

\begin{verbete}{反应}{fan3ying4}{4,7}
  \significado[个]{s.}{reação; resposta; reação química}
  \significado{v.}{reagir; responder}
\end{verbete}

\begin{verbete}{反正}{fan3zheng4}{4,5}
  \significado{adv.}{de qualquer maneira; em qualquer caso; aconteça o que acontecer}
\end{verbete}

\begin{verbete}{犯法}{fan4fa3}{5,8}
  \significado{v.}{violar (quebrar) a lei}
\end{verbete}

\begin{verbete}{犯罪}{fan4zui4}{5,13}
  \significado{v.+compl.}{cometer  um crime (uma ofensa)}
\end{verbete}

\begin{verbete}{饭}{fan4}{7}[Radical 食]
  \significado[碗]{s.}{arroz cozido}
  \significado[顿]{s.}{refeição}
  \significado{s.}{(empréstimo linguístico) fã; devoto}
\end{verbete}

\begin{verbete}{饭店}{fan4dian4}{7,8}
  \significado[家,个]{s.}{hotel; restaurante}
\end{verbete}

\begin{verbete}{方案}{fang1'an4}{4,10}
  \significado[个,套]{s.}{plano; programa (para uma ação, etc.); proposta; porposta de projeto de lei}
\end{verbete}

\begin{verbete}{方便}{fang1bian4}{4,9}
  \significado{adj.}{conveniente, adequado}
  \significado{v.}{facilitar, facilitar as coisas; ter dinheiro de sobra; (eufemismo) aliviar-se}
\end{verbete}

\begin{verbete}{方法}{fang1fa3}{4,8}
  \significado[个]{s.}{método, meio}
\end{verbete}

\begin{verbete}{方言}{fang1yan2}{4,7}
  \significado*{s.}{o primeiro dicionário de dialeto chinês, editado por Yang Xiong 扬雄 no século I, contendo mais de 9.000 caracteres}
  \significado{s.}{dialeto}
  \veja{扬雄}{yang2xiong2}
\end{verbete}

\begin{verbete}{防护}{fang2hu4}{6,7}
  \significado{v.}{defender; proteger}
\end{verbete}

\begin{verbete}{防晒}{fang2shai4}{6,10}
  \significado{s.}{protetor solar}
\end{verbete}

\begin{verbete}{房东}{fang2dong1}{8,5}
  \significado{s.}{dono; proprietário; senhorio}
\end{verbete}

\begin{verbete}{房间}{fang2jian1}{8,7}
  \significado[间,个]{s.}{quarto}
\end{verbete}

\begin{verbete}{房主}{fang2zhu3}{8,5}
  \significado{s.}{proprietário; dono de um imóvel}
\end{verbete}

\begin{verbete}{房子}{fang2zi5}{8,3}
  \significado[栋,幢,座,套,间,个]{s.}{apartamento; casa; quarto}
\end{verbete}

\begin{verbete}{访问}{fang3wen4}{6,6}
  \significado{v.}{visitar}
\end{verbete}

\begin{verbete}{放}{fang4}{8}[Radical 攴]
  \significado{v.}{liberar; libertar; deixar ir; colocar; por; detonar (fogos de artifício)}
\end{verbete}

\begin{verbete}{放鞭炮}{fang4bian1pao4}{8,18,9}
  \significado{s.}{um conjunto de bombinhas ou traques}
\end{verbete}

\begin{verbete}{放出}{fang4chu1}{8,5}
  \significado{v.}{liberar; libertar}
\end{verbete}

\begin{verbete}{放大}{fang4da4}{8,3}
  \significado{v.}{ampliar}
\end{verbete}

\begin{verbete}{放电}{fang4dian4}{8,5}
  \significado{s.}{descarga elétrica}
\end{verbete}

\begin{verbete}{放飞}{fang4fei1}{8,3}
  \significado{s.}{deixar voar}
\end{verbete}

\begin{verbete}{放过}{fang4guo4}{8,6}
  \significado{v.}{deixar; deixar alguém escapar impune; passar despercebido}
\end{verbete}

\begin{verbete}{放假}{fang4jia4}{8,11}
  \significado{v.}{ter férias ou feriado}
\end{verbete}

\begin{verbete}{放弃}{fang4qi4}{8,7}
  \significado{v.}{abandonar, desistir de, renunciar}
\end{verbete}

\begin{verbete}{放弃权利}{fang4qi4 quan2li4}{8,7,6,7}
  \significado{s.}{renúncia}
\end{verbete}

\begin{verbete}{放弃者}{fang4qi4zhe3}{8,7,8}
  \significado{s.}{desistente}
\end{verbete}

\begin{verbete}{放任}{fang4ren4}{8,6}
  \significado{v.}{ignorar; saciar-se; deixar sozinho}
\end{verbete}

\begin{verbete}{放肆}{fang4si4}{8,13}
  \significado{adj.}{atrevido; pesunçoso; devasso}
\end{verbete}

\begin{verbete}{放松}{fang4song1}{8,8}
  \significado{adj.}{relaxado; afrouxado}
  \significado{v.}{relaxar; afrouxar}
\end{verbete}

\begin{verbete}{放心}{fang4xin1}{8,4}
  \significado{adj.}{despreocupado}
  \significado{v.}{sentir-se aliviado; sentir-se tranquilo; ficar à vontade}
  \significado{v.+compl.}{confiar; ter confiança em alguém; estar à vontade; sentir-se aliviado}
\end{verbete}

\begin{verbete}{放养}{fang4yang3}{8,9}
  \significado{v.}{criar (gado, peixes, culturas, etc.); crescer; criar}
\end{verbete}

\begin{verbete}{放走}{fang4zou3}{8,7}
  \significado{v.}{permitir (uma pessoa ou um animal) ir; liberar; libertar}
\end{verbete}

\begin{verbete}{飞船}{fei1chuan2}{3,11}
  \significado{s.}{espaçonave; dirigível; aeronave}
\end{verbete}

\begin{verbete}{飞碟}{fei1die2}{3,14}
  \significado{s.}{disco-voador, OVNI; \emph{frisbee}}
\end{verbete}

\begin{verbete}{飞机}{fei1ji1}{3,6}
  \significado[架]{s.}{avião}
\end{verbete}

\begin{verbete}{飞机票}{fei1ji1piao4}{3,6,11}
  \significado[张]{s.}{bilhete de avião}
  \veja{机票}{ji1piao4}
\end{verbete}

\begin{verbete}{飞行}{fei1xing2}{3,6}
  \significado{s.}{voo; aviação}
  \significado{v.}{(aviões, etc;) voar}
\end{verbete}

\begin{verbete}{非}{fei1}{8}[Radical ⾮][Kangxi 175]
  \significado*{s.}{África, abreviação de~非洲}
  \significado{adv.}{não ser; não é; não}
  \veja{非洲}{fei1zhou1}
\end{verbete}

\begin{verbete}{非常}{fei1chang2}{8,11}
  \significado{adv.}{extraordinário; altamente; muito}
\end{verbete}

\begin{verbete}{非洲}{fei1zhou1}{8,9}
  \significado*{s.}{África}
  \veja{非}{fei1}
\end{verbete}

\begin{verbete}{非洲人}{fei1zhou1ren2}{8,9,2}
  \significado{s.}{africano; nascido na África}
\end{verbete}

\begin{verbete}{狒狒}{fei4fei4}{8,8}
  \significado{s.}{babuíno}
\end{verbete}

\begin{verbete}{分}{fen1}{4}[Radical ⼑]
  \significado{clas.}{minuto; centavo}
  \significado{s.}{um ponto (em esportes ou jogos); parte ou subdivisão; fração}
\end{verbete}

\begin{verbete}{分公司}{fen1gong1si1}{4,4,5}
  \significado{s.}{sucursal; filial de companhia}
\end{verbete}

\begin{verbete}{分量}{fen1liang4}{4,12}
  \significado{s.}{componente vetorial}
  \veja{分量}{fen4liang4}
  \veja{分量}{fen4liang5}
\end{verbete}

\begin{verbete}{分手}{fen1shou3}{4,4}
  \significado{v.+compl.}{separar; separar-se do companheiro; dizer adeus}
\end{verbete}

\begin{verbete}{分钟}{fen1zhong1}{4,9}
  \significado{s.}{minuto}
\end{verbete}

\begin{verbete}{分子}{fen1zi3}{4,3}
  \significado{s.}{molécula; (matemática) numerador de uma fração}
  \veja{分子}{fen4zi3}
\end{verbete}

\begin{verbete}{焚香}{fen2xiang1}{12,9}
  \significado{v.}{queimar incenso}
\end{verbete}

\begin{verbete}{粉}{fen3}{10}[Radical ⽶]
  \significado{s.}{pó; pó cosmético facial; alimento preparado a partir de amido, macarrão feito de qualquer tipo de farinha}
  \significado{v.}{tornar algo em pó; ser um fã de}
\end{verbete}

\begin{verbete}{粉色}{fen3se4}{10,6}
  \significado{s.}{cor-de-rosa}
\end{verbete}

\begin{verbete}{粉丝}{fen3si1}{10,5}
  \significado[把]{s.}{aletria de amido de feijão; aletria chinesa; macarrão de celofane ou macarrão de vidro (transparente)}
  \significado{s.}{fã (empréstimo linguístico); entusiasta de alguém ou alguma coisa}
\end{verbete}

\begin{verbete}{分量}{fen4liang4}{4,12}
  \significado{s.}{tamanho da porção (comida)}
  \veja{分量}{fen1liang4}
  \veja{分量}{fen4liang5}
\end{verbete}

\begin{verbete}{分量}{fen4liang5}{4,12}
  \significado{s.}{quantidade; peso; medida}
  \veja{分量}{fen1liang4}
  \veja{分量}{fen4liang4}
\end{verbete}

\begin{verbete}{分子}{fen4zi3}{4,3}
  \significado{s.}{membros de uma classe ou grupo; elementos políticos (como intelectuais ou extremistas)}
  \veja{分子}{fen1zi3}
\end{verbete}

\begin{verbete}{份}{fen4}{6}[Radical 人]
  \significado{clas.}{para presentes, jornais, revistas, papéis, relatórios, contratos, etc. ou pratos (refeição)}
\end{verbete}

\begin{verbete}{奋战}{fen4zhan4}{8,9}
  \significado{v.}{lutar bravamente; trabalhar duro}
\end{verbete}

\begin{verbete}{愤怒}{fen4nu4}{12,9}
  \significado{adj.}{zangado; indignado}
  \significado{s.}{ira}
\end{verbete}

\begin{verbete}{愤世嫉俗}{fen4shi4ji2su2}{12,5,13,9}
  \significado{v.}{ser cínico; ser amargurado}
\end{verbete}

\begin{verbete}{丰收}{feng1shou1}{4,6}
  \significado{s.}{colheita abundante}
\end{verbete}

\begin{verbete}{风}{feng1}{4}[Radical 風][Kangxi 182]
  \significado[阵,丝]{s.}{vento}
\end{verbete}

\begin{verbete}{风景}{feng1jing3}{4,12}
  \significado{s.}{cenário, paisagem}
\end{verbete}

\begin{verbete}{风扇}{feng1shan4}{4,10}
  \significado{s.}{ventilador elétrico}
\end{verbete}

\begin{verbete}{风筝}{feng1zheng5}{4,12}
  \significado{s.}{pipa; papagaio; pandorga}
\end{verbete}

\begin{verbete}{枫叶}{feng1ye4}{8,5}
  \significado{s.}{folha de bordo (maple, tipo de árvore)}
\end{verbete}

\begin{verbete}{封}{feng1}{9}[Radical 寸]
  \significado*{s.}{sobrenome Feng}
  \significado{clas.}{para objetos selados, especialmente cartas}
  \significado{v.}{conceder um título; conferir; conceder; selar}
\end{verbete}

\begin{verbete}{封闭}{feng1bi4}{9,6}
  \significado{v.}{fechar; selar; confinado}
\end{verbete}

\begin{verbete}{封底}{feng1di3}{9,8}
  \significado{s.}{contracapa de um livro}
\end{verbete}

\begin{verbete}{封冻}{feng1dong4}{9,7}
  \significado{v.}{congelar (água ou terra)}
\end{verbete}

\begin{verbete}{封盖}{feng1gai4}{9,11}
  \significado{s.}{boné; capa; selo}
  \significado{v.}{cobrir}
\end{verbete}

\begin{verbete}{封建}{feng1jian4}{9,8}
  \significado{adj.}{feudal}
  \significado{s.}{feudalismo}
\end{verbete}

\begin{verbete}{封口}{feng1kou3}{9,3}
  \significado{v.}{selar; fechar; curar (uma ferida); manter os lábios selados}
\end{verbete}

\begin{verbete}{封面}{feng1mian4}{9,9}
  \significado{s.}{capa (de uma publicação); sobrecapa}
\end{verbete}

\begin{verbete}{封印}{feng1yin4}{9,5}
  \significado{s.}{selo (em envelopes)}
\end{verbete}

\begin{verbete}{封斋}{feng1zhai1}{9,10}
  \significado*{s.}{Ramadã (Islã)}
\end{verbete}

\begin{verbete}{疯狂}{feng1kuang2}{9,7}
  \significado{adj.}{louco, frenético, selvagem}
\end{verbete}

\begin{verbete}{缝纫}{feng2ren4}{13,6}
  \significado{v.}{costurar}
\end{verbete}

\begin{verbete}{缝纫机}{feng2ren4ji1}{13,6,6}
  \significado[架]{s.}{máquina de costura}
\end{verbete}

\begin{verbete}{凤凰}{feng4huang2}{4,11}
  \significado{s.}{fênix}
\end{verbete}

\begin{verbete}{佛}{fo2}{7}[Radical 人]
  \significado*{s.}{Buda (abreviatura de 佛陀); Budismo}
  \veja{佛陀}{fo2tuo2}
  \veja{佛}{fu2}
\end{verbete}

\begin{verbete}{佛陀}{fo2tuo2}{7,7}
  \significado{s.}{Buda (uma pessoa que atingiu a Budeidade, ou especificamente Siddhartha Gautama)}
\end{verbete}

\begin{verbete}{否定}{fou3ding4}{7,8}
  \significado{s.}{negativo (resposta); negação}
  \significado{v.}{negar; rejeitar}
\end{verbete}

\begin{verbete}{否则}{fou3ze2}{7,6}
  \significado{conj.}{caso contrário; ou}
\end{verbete}

\begin{verbete}{夫妻}{fu1qi1}{4,8}
  \significado{s.}{casal; marido e eposa}
\end{verbete}

\begin{verbete}{佛}{fu2}{7}[Radical 人]
  \significado{adv.}{aparentemente}
  \significado{s.}{ornamento da cabeça (feminino)}
  \veja{佛}{fo2}
\end{verbete}

\begin{verbete}{扶梯}{fu2ti1}{7,11}
  \significado{s.}{escada rolante}
\end{verbete}

\begin{verbete}{服}{fu2}{8}[Radical ⽉]
  \significado{s.}{roupas; vestido; vestuário; roupa de luto}
  \significado{v.}{servir (nas forças armadas, uma sentença de prisão, etc.); obedecer; ser convencido (por um argumento); convencer; admirar; aclimatar; tomar (medicamento); usar roupas de luto}
  \veja{服}{fu4}
\end{verbete}

\begin{verbete}{服务员}{fu2wu4yuan2}{8,5,7}
  \significado{s.}{atendente; garçom; garçonete; pessoal de atendimento ao cliente}
\end{verbete}

\begin{verbete}{浮力}{fu2li4}{10,2}
  \significado{s.}{flutuabilidade}
\end{verbete}

\begin{verbete}{浮图}{fu2tu2}{10,8}
  \significado*{s.}{Termo alternativo para 佛陀}
  \variante{浮屠}
  \veja{佛陀}{fo2tuo2}
\end{verbete}

\begin{verbete}{浮屠}{fu2tu2}{10,11}
  \significado*{s.}{Buda; Templo (Stupa) Budista (transliteração de Pali Thuo)}
\end{verbete}

\begin{verbete}{符合}{fu2he2}{11,6}
  \significado{conj.}{de acordo com; concordando com; contando com; alinhado com}
  \significado{v.}{concordar com; estar em conformidade com; corresponder com; gerenciar; lidar}
\end{verbete}

\begin{verbete}{福克斯}{fu2ke4si1}{13,7,12}
  \significado*{s.}{Fox (empresa de mídia); Focus (automóvel fabricado pela Ford)}
\end{verbete}

\begin{verbete}{福泽}{fu2ze2}{13,8}
  \significado{s.}{boa sorte}
\end{verbete}

\begin{verbete}{父母亲}{fu4mu3qin1}{4,5,9}
  \significado{s.}{pais}
\end{verbete}

\begin{verbete}{父亲}{fu4qin1}{4,9}
  \significado[个]{s.}{pai}
\end{verbete}

\begin{verbete}{付}{fu4}{5}[Radical 人]
  \significado*{s.}{sobrenome Fu}
  \significado{clas.}{para pares ou conjuntos de coisas}
  \significado{v.}{pagar}
\end{verbete}

\begin{verbete}{付款}{fu4kuan3}{5,12}
  \significado{s.}{pagamento}
  \significado{v.+compl.}{pagar uma quantia em dinheiro}
\end{verbete}

\begin{verbete}{负责}{fu4ze2}{6,8}
  \significado{adj.}{consciencioso}
  \significado{v.}{ser responsável por, estar no comando de, assumir a responsabilidade por}
\end{verbete}

\begin{verbete}{附近}{fu4jin4}{7,7}
  \significado{adv.}{aqui perto; perto daqui}
\end{verbete}

\begin{verbete}{服}{fu4}{8}[Radical ⽉]
  \significado{clas.}{para remédio: dose}
  \veja{服}{fu2}
\end{verbete}

\begin{verbete}{复活节}{fu4huo2jie2}{9,9,5}
  \significado*{s.}{Páscoa}
\end{verbete}

\begin{verbete}{复刻}{fu4ke4}{9,8}
  \significado{v.}{reimprimir (um trabalho que esteve fora do catálogo); reeditar (um disco de vinil, um CD, etc.); replicar; recriar; (computação) \emph{fork} (empréstimo linguístico)}
\end{verbete}

\begin{verbete}{副}{fu4}{11}[Radical 刀]
  \significado{clas.}{para pares, conjuntos de coisas e expressões faciais; para óculos, luvas, etc.}
\end{verbete}

\begin{verbete}{覆盆子}{fu4pen2zi5}{18,9,3}
  \significado{s.}{framboesa}
\end{verbete}

%%%%% EOF %%%%%


\section*{G}
\addcontentsline{toc}{section}{G}
\begin{multicols}{2}
%%%
%%% G
%%%
% \entry{干}{v.}{fazer}
% \entry{高兴}{adj.}{feliz; alegre; contente}
% \entry{哥\xpinyin{哥}{ge0}}{n.}{irmão mais velho}
% \entry{个}{p.c.}{de uso geral}
% \entry{给}{pre.}{a; para}
% \entry{给}{v.}{dar}
% \entry{跟}{prep.}{com}
% \entry{公司}{n.}{empresa; companhia}
% \entry{工作}{n./v.}{trabalho; trabalhar}
% \entry{狗}{n.}{cão; cachorro}
% \entry{刮(风)}{v.}{soprar (vento)}
% \entry{拐}{v.}{virar}
% \entry{光盘}{n.}{disco compacto}
% \entry{贵姓}{interr.}{nome}
% \entry{国}{n.}{país}
% \entry{果酱}{n.}{compota ou doce (de frutas)}
% \entry{过年}{v.}{festejar o Ano Novo Chinês}
% \entry{贵}{adj.}{caro}
% \entry{公克}{n.}{trabalho escolar; trabalho de casa}
% \entry{橄榄球}{n.}{rúgbi}
\end{multicols}

%%%
%%% H
%%%

\section*{H}\addcontentsline{toc}{section}{H}

\begin{verbete}{哈马斯}{ha1ma3si1}{9,3,12}
  \significado*{s.}{Hamas (Grupo Palestino)}
\end{verbete}

\begin{verbete}{还}{hai2}{7}[Radical 辵]
  \significado{adv.}{ainda; também; ainda mais; razoavelmente; bastante}
  \veja{还}{huan2}
\end{verbete}

\begin{verbete}{还是}{hai2shi5}{7,9}
  \significado{adv.}{ainda (como antes); inesperadamente; teve melhor}
  \significado{conj.}{ou (somente para frases interrogativas)}
\end{verbete}

\begin{verbete}{还有}{hai2you3}{7,6}
  \significado{adv.}{além do mais; além disso; ainda permanece; ainda há}
\end{verbete}

\begin{verbete}{孩子}{hai2zi5}{9,3}
  \significado{s.}{criança; filho}
\end{verbete}

\begin{verbete}{海}{hai3}{10}[Radical 水]
  \significado*{s.}{sobrenome Hai}
  \significado[个,片]{s.}{mar; oceano}
\end{verbete}

\begin{verbete}{海边}{hai3bian1}{10,5}
  \significado{s.}{costa marítima; litoral; beira-mar; praia}
\end{verbete}

\begin{verbete}{海底}{hai3di3}{10,8}
  \significado{adj.}{submarino}
  \significado{s.}{fundo do mar; solo oceânico; fundo do oceano}
\end{verbete}

\begin{verbete}{海风}{hai3feng1}{10,4}
  \significado{s.}{brisa do mar; vento que vem do mar}
\end{verbete}

\begin{verbete}{海浪}{hai3lang4}{10,10}
  \significado{s.}{ondas do mar}
\end{verbete}

\begin{verbete}{海里}{hai3li3}{10,7}
  \significado{s.}{milha náutica}
\end{verbete}

\begin{verbete}{海绵}{hai3mian2}{10,11}
  \significado{s.}{(zoologia) esponja do mar; esponja (feita de poliéster ou celulose, etc.); espuma de borracha}
\end{verbete}

\begin{verbete}{海鸥}{hai3'ou1}{10,9}
  \significado{s.}{gaivota}
\end{verbete}

\begin{verbete}{海水}{hai3shui3}{10,4}
  \significado{s.}{água do mar}
\end{verbete}

\begin{verbete}{海棠}{hai3tang2}{10,12}
  \significado{s.}{begônia}
\end{verbete}

\begin{verbete}{害}{hai4}{10}[Radical 宀]
  \significado{s.}{dano; mal; calamidade}
  \significado{v.}{causar danos a; causar problemas para}
\end{verbete}

\begin{verbete}{害怕}{hai4pa4}{10,8}
  \significado{v.}{ter medo, ficar com medo, temer}
\end{verbete}

\begin{verbete}{害羞}{hai4xiu1}{10,10}
  \significado{adj.}{tímido; envergonhado}
\end{verbete}

\begin{verbete}{含金量}{han2jin1liang4}{7,8,12}
  \significado{adj.}{conteúdo de ouro; (fig.) valioso}
\end{verbete}

\begin{verbete}{函数}{han2shu4}{8,13}
  \significado{s.}{função (matemática)}
\end{verbete}

\begin{verbete}{涵}{han2}{11}[Radical 水]
  \significado{s.}{bueiro; galeria}
  \significado{v.}{conter; incluir; entupir}
\end{verbete}

\begin{verbete}{韩国}{han2guo2}{12,8}
  \significado*{s.}{Coréia do Sul}
\end{verbete}

\begin{verbete}{韩国人}{han2guo2ren2}{12,8,2}
  \significado{s.}{coreano; nascido na Coréia}
\end{verbete}

\begin{verbete}{汉堡包}{han4bao3bao1}{5,12,5}
  \significado[个]{s.}{hambúrguer}
\end{verbete}

\begin{verbete}{汉堡王}{han4bao3wang2}{5,12,4}
  \significado*{s.}{Burguer King (restaurante \emph{fast-food})}
\end{verbete}

\begin{verbete}{汉服}{han4fu2}{5,8}
  \significado{s.}{vestido chinês tradicional Han}
\end{verbete}

\begin{verbete}{汉葡词典}{han4-pu2 ci2dian3}{5,12,7,8}
  \significado[部,本]{s.}{dicionário chinês-português}
  \veja{葡汉词典}{pu2-han4 ci2dian3}
\end{verbete}

\begin{verbete}{汉语}{han4yu3}{5,9}
  \significado[门]{s.}{chinês, língua chinesa, mandarim}
\end{verbete}

\begin{verbete}{汉字}{han4zi4}{5,6}
  \significado[个]{s.}{caracter chinês}
\end{verbete}

\begin{verbete}{汗水}{han4shui3}{6,4}
  \significado*{s.}{Rio Han (Hanshui)}
  \significado{s.}{suor; transpiração}
\end{verbete}

\begin{verbete}{汗腺}{han4xian4}{6,13}
  \significado{s.}{glândula sudorípara}
\end{verbete}

\begin{verbete}{汗液}{han4ye4}{6,11}
  \significado{s.}{suor}
\end{verbete}

\begin{verbete}{焊}{han4}{11}[Radical 火]
  \significado{v.}{soldar}
\end{verbete}

\begin{verbete}{撼}{han4}{16}[Radical 手]
  \significado{v.}{sacudir; vibrar}
\end{verbete}

\begin{verbete}{行}{hang2}{6}[Radical 行]
  \significado{s.}{firma comercial; linha de negócio; profissão; linha (de um tema); linha (em tabela de dados)}
  \veja{行}{xing2}
\end{verbete}

\begin{verbete}{航班}{hang2ban1}{10,10}
  \significado{s.}{voo; número de voo}
\end{verbete}

\begin{verbete}{航天员}{hang2tian1yuan2}{10,4,7}
  \significado{s.}{astronauta}
\end{verbete}

\begin{verbete}{号}{hao2}{5}[Radical 口]
  \significado[个]{s.}{rugido; choro}
  \veja{号}{hao4}
\end{verbete}

\begin{verbete}{蚝}{hao2}{10}[Radical 虫]
  \significado{s.}{ostra}
\end{verbete}

\begin{verbete}{毫不费力}{hao2bu2fei4li4}{11,4,9,2}
  \significado{adj.}{sem esforço; não gastando o menor esforço}
\end{verbete}

\begin{verbete}{毫米}{hao2mi3}{11,6}
  \significado{s.}{milímetro}
\end{verbete}

\begin{verbete}{豪华}{hao2hua2}{14,6}
  \significado{adj.}{luxuoso}
\end{verbete}

\begin{verbete}{好}{hao3}{6}[Radical ⼥]
  \significado{adj.}{bom, bem}
  \significado{adv.}{apropriadamente; cuidadosamente; muito (linguagem falada)}
  \veja{好}{hao4}
\end{verbete}

\begin{verbete}{好吃}{hao3chi1}{6,6}
  \significado{adj.}{delicioso; saboroso}
  \veja{好吃}{hao4chi1}
\end{verbete}

\begin{verbete}{好汉}{hao3han4}{6,5}
  \significado[条]{s.}{herói; pessoa forte e corajosa}
\end{verbete}

\begin{verbete}{好久}{hao3jiu3}{6,3}
  \significado{adv.}{por muito tempo; por eras (no passado)}
\end{verbete}

\begin{verbete}{好看}{hao3kan4}{6,9}
  \significado{adj.}{boa aparência; bom (um filme, livro, programa de TV, etc.)}
\end{verbete}

\begin{verbete}{好听}{hao3ting1}{6,7}
  \significado{adj.}{agradável de ouvir}
\end{verbete}

\begin{verbete}{好玩儿}{hao3wan2r5}{6,8,2}
  \significado{adj.}{divertido; prazeroso; interessante}
\end{verbete}

\begin{verbete}{好象}{hao3xiang4}{6,11}
  \variante{好像}
\end{verbete}

\begin{verbete}{好像}{hao3xiang4}{6,13}
  \significado{adv.}{talvez fosse; parecer; ser como}
\end{verbete}

\begin{verbete}{好心}{hao3xin1}{6,4}
  \significado{s.}{bondade; boas intenções}
\end{verbete}

\begin{verbete}{好学}{hao3xue2}{6,8}
  \significado{adj.}{fácil de aprender}
  \veja{好学}{hao4xue2}
\end{verbete}

\begin{verbete}{好用}{hao3yong4}{6,5}
  \significado{adj.}{fácil de usar; adequado ao uso}
\end{verbete}

\begin{verbete}{好友}{hao3you3}{6,4}
  \significado[个]{s.}{amigo próximo; (site de redes sociais) amigo}
\end{verbete}

\begin{verbete}{号}{hao4}{5}[Radical 口]
  \significado{num.}{dia do mês; usado para indicar o número de pessoas}
  \significado[个]{s.}{dia do mês; número}
  \veja{号}{hao2}
\end{verbete}

\begin{verbete}{号角}{hao4jiao3}{5,7}
  \significado{s.}{corneta; trombeta}
\end{verbete}

\begin{verbete}{号码}{hao4ma3}{5,8}
  \significado[堆,个]{s.}{número}
\end{verbete}

\begin{verbete}{好}{hao4}{6}[Radical ⼥]
  \significado{v.}{gostar de; estar propenso a; ter tendência a}
  \veja{好}{hao3}
\end{verbete}

\begin{verbete}{好吃}{hao4chi1}{6,6}
  \significado{v.}{gostar de comer; ser guloso}
  \veja{好吃}{hao3chi1}
\end{verbete}

\begin{verbete}{好奇}{hao4qi2}{6,8}
  \significado{adj.}{curioso}
  \significado{s.}{curiosidade}
\end{verbete}

\begin{verbete}{好学}{hao4xue2}{6,8}
  \significado{s.}{estudioso; erudito}
  \veja{好学}{hao3xue2}
\end{verbete}

\begin{verbete}{呵}{he1}{8}[Radical 口]
  \significado{expr.}{Meu Deus!; expelir a respiração}
  \veja{呵}{a1}
\end{verbete}

\begin{verbete}{欱}{he1}{10}
  \variante{喝}
\end{verbete}

\begin{verbete}{喝}{he1}{12}[Radical 口]
  \significado{interj.}{Meu Deus!}
  \significado{v.}{beber}
  \veja{喝}{he4}
\end{verbete}

\begin{verbete}{喝醉}{he1zui4}{12,15}
  \significado{v.}{ficar bêbado}
\end{verbete}

\begin{verbete}{合同}{he2tong5}{6,6}
  \significado[个]{s.}{contrato (negócio)}
\end{verbete}

\begin{verbete}{合宪性}{he2xian4xing4}{6,9,8}
  \significado{s.}{constitucionalismo}
\end{verbete}

\begin{verbete}{合资}{he2zi1}{6,10}
  \significado{s.}{\emph{joint-venture} com capitais mistos}
\end{verbete}

\begin{verbete}{合作}{he2zuo4}{6,7}
  \significado[个]{s.}{cooperação}
  \significado{v.}{cooperar; colaborar}
\end{verbete}

\begin{verbete}{何况}{he2kuang4}{7,7}
  \significado{conj.}{além disso; muito menos}
\end{verbete}

\begin{verbete}{和}{he2}{8}[Radical 口]
  \significado*{s.}{sobrenome He}
  \significado{conj.}{e (somente para palavras)}
  \veja{和}{he4}
  \veja{和}{hu2}
  \veja{和}{huo2}
  \veja{和}{huo4}
\end{verbete}

\begin{verbete}{和平}{he2ping2}{8,5}
  \significado{adj.}{pacífico}
  \significado{s.}{paz}
\end{verbete}

\begin{verbete}{和平共处}{he2ping2gong4chu3}{8,5,6,5}
  \significado{s.}{coexistência pacífica de nações, sociedades, etc.}
\end{verbete}

\begin{verbete}{和谐}{he2xie2}{8,11}
  \significado{adj.}{harmonioso}
  \significado{s.}{harmonia}
  \significado{v.}{(eufemismo) censurar}
\end{verbete}

\begin{verbete}{河}{he2}{8}[Radical 水]
  \significado[条,道]{s.}{rio}
\end{verbete}

\begin{verbete}{河蚌}{he2bang4}{8,10}
  \significado{s.}{mexilhões; bivalves cultivados em rios e lagos}
\end{verbete}

\begin{verbete}{核}{he2}{10}[Radical 木]
  \significado{adj.}{nuclear}
  \significado{s.}{poço; pedra; núcleo}
  \significado{v.}{examinar; checar; verificar}
\end{verbete}

\begin{verbete}{荷}{he2}{10}[Radical 艸]
  \significado{s.}{lótus}
  \veja{荷}{he4}
\end{verbete}

\begin{verbete}{荷花}{he2hua1}{10,7}
  \significado{s.}{lótus}
\end{verbete}

\begin{verbete}{盒}{he2}{11}[Radical ⽫]
  \significado{clas.}{caixa pequena}
  \significado{s.}{caixa pequena; estojo}
\end{verbete}

\begin{verbete}{和}{he4}{8}[Radical 口]
  \significado{v.}{conversar com os outros; compor um poema em resposta (ao poema de alguém) usando a mesma sequência de rimas; juntar-se ao canto (canção)}
  \veja{和}{he2}
  \veja{和}{hu2}
  \veja{和}{huo2}
  \veja{和}{huo4}
\end{verbete}

\begin{verbete}{贺}{he4}{9}[Radical 貝]
  \significado*{s.}{sobrenome He}
  \significado{v.}{parabenizar; congratular}
\end{verbete}

\begin{verbete}{荷}{he4}{10}[Radical 艸]
  \significado{s.}{carga; responsabilidade}
  \significado{v.}{carregar no ombro ou nas costas}
  \veja{荷}{he2}
\end{verbete}

\begin{verbete}{喝}{he4}{12}[Radical 口]
  \significado{v.}{gritar bem alto}
  \veja{喝}{he1}
\end{verbete}

\begin{verbete}{喝彩}{he4cai3}{12,11}
  \significado{s.}{aclamar; torcer}
\end{verbete}

\begin{verbete}{褐色}{he4se4}{14,6}
  \significado{s.}{cor marrom}
\end{verbete}

\begin{verbete}{鹤}{he4}{15}[Radical 鳥]
  \significado{s.}{grou (ave)}
\end{verbete}

\begin{verbete}{黑}{hei1}{12}[Radical ⿊][Kangxi 203]
  \significado{adj.}{preto; escuro; ilegal; secreto; sombrio; sinistro}
  \significado{v.}{esconder (algo); difamar; hackear (computador, empréstimo linguístico)}
\end{verbete}

\begin{verbete}{黑板}{hei1ban3}{12,8}
  \significado[块,个]{s.}{quadro negro}
\end{verbete}

\begin{verbete}{黑客}{hei1ke4}{12,9}
  \significado{s.}{\emph{hacker} (computação, empréstimo linguístico)}
\end{verbete}

\begin{verbete}{黑色}{hei1se4}{12,6}
  \significado{s.}{cor preta}
\end{verbete}

\begin{verbete}{很}{hen3}{9}[Radical 彳]
  \significado{adv.}{muito; mui; advérbio de grau}
\end{verbete}

\begin{verbete}{恨}{hen4}{9}[Radical 心]
  \significado{s.}{ódio}
  \significado{v.}{odiar}
\end{verbete}

\begin{verbete}{恒星系}{heng2xing1xi4}{9,9,7}
  \significado{s.}{sistema estelar; galáxia}
\end{verbete}

\begin{verbete}{横竖}{heng2shu5}{15,9}
  \significado{adv.}{de qualquer maneira; independentemente (linguagem falada)}
\end{verbete}

\begin{verbete}{轰鸣}{hong1ming2}{8,8}
  \significado{s.}{bum (som de explosão); estrondo}
\end{verbete}

\begin{verbete}{轰炸机}{hong1zha4ji1}{8,9,6}
  \significado{s.}{bombardeiro (aeronave)}
\end{verbete}

\begin{verbete}{哄}{hong1}{9}[Radical 口]
  \significado{s.}{gargalhadas, risadas ruidosas; algazarra; rugido, clamor}
  \veja{哄}{hong3}
  \veja{哄}{hong4}
\end{verbete}

\begin{verbete}{红}{hong2}{6}[Radical 糸]
  \significado*{s.}{sobrenome Hong}
  \significado{adj.}{vermelho; popular; revolucionário}
  \significado{s.}{bônus}
\end{verbete}

\begin{verbete}{红宝石}{hong2bao3shi2}{6,8,5}
  \significado{s.}{rubi}
\end{verbete}

\begin{verbete}{红茶}{hong2cha2}{6,9}
  \significado[杯,壶]{s.}{chá preto}
\end{verbete}

\begin{verbete}{红绿灯}{hong2lv4deng1}{6,11,6}
  \significado[个]{s.}{semáforo; sinal de trânsito}
\end{verbete}

\begin{verbete}{红色}{hong2se4}{6,6}
  \significado{s.}{cor vermelha}
\end{verbete}

\begin{verbete}{红烧}{hong2shao1}{6,10}
  \significado{s.}{guisado em molho de soja (prato)}
\end{verbete}

\begin{verbete}{红薯}{hong2shu3}{6,16}
  \significado{s.}{batata doce}
\end{verbete}

\begin{verbete}{红线}{hong2xian4}{6,8}
  \significado{s.}{linha vermelha}
\end{verbete}

\begin{verbete}{洪水}{hong2shui3}{9,4}
  \significado{s.}{enchente; inundação; dilúvio}
\end{verbete}

\begin{verbete}{哄}{hong3}{9}[Radical 口]
  \significado{v.}{enganar; persuadir; divertir (uma criança)}
  \veja{哄}{hong1}
  \veja{哄}{hong4}
\end{verbete}

\begin{verbete}{哄}{hong4}{9}[Radical 口]
  \significado{s.}{tumulto; agitação; perturbação}
  \veja{哄}{hong1}
  \veja{哄}{hong3}
\end{verbete}

\begin{verbete}{猴子}{hou2zi5}{12,3}
  \significado[只]{s.}{macaco}
\end{verbete}

\begin{verbete}{后边}{hou4bian5}{6,5}
  \significado{adv.}{atrás; detrás}
\end{verbete}

\begin{verbete}{后来}{hou4lai2}{6,7}
  \significado{adv.}{mais tarde}
\end{verbete}

\begin{verbete}{后面}{hou4mian5}{6,9}
  \significado{adv.}{atrás; detrás}
\end{verbete}

\begin{verbete}{后年}{hou4nian2}{6,6}
  \significado{adv.}{daqui a dois anos}
\end{verbete}

\begin{verbete}{后天}{hou4tian1}{6,4}
  \significado{adv.}{depois de amanhã}
\end{verbete}

\begin{verbete}{呼吸}{hu1xi1}{8,6}
  \significado{v.}{respirar}
\end{verbete}

\begin{verbete}{呼啸}{hu1xiao4}{8,11}
  \significado{v.}{assobiar}
\end{verbete}

\begin{verbete}{忽然}{hu1ran2}{8,12}
  \significado{adv.}{de repente}
\end{verbete}

\begin{verbete}{和}{hu2}{8}[Radical 口]
  \significado{v.}{completar um conjunto de Mahjong ou cartas de baralho}
  \veja{和}{he2}
  \veja{和}{he4}
  \veja{和}{huo2}
  \veja{和}{huo4}
\end{verbete}

\begin{verbete}{胡萝卜}{hu2luo2bo5}{9,11,2}
  \significado{s.}{cenoura}
\end{verbete}

\begin{verbete}{湖}{hu2}{12}[Radical 水]
  \significado[个,片]{s.}{lago}
\end{verbete}

\begin{verbete}{湖南}{hu2nan2}{12,9}
  \significado*{s.}{Hunan}
\end{verbete}

\begin{verbete}{葫芦}{hu2lu5}{12,7}
  \significado{adj.}{confuso}
  \significado{s.}{cabaça;  termo genérico para bloco e equipamento (ou partes dele)}
\end{verbete}

\begin{verbete}{糊里糊涂}{hu2li5hu2tu2}{15,7,15,10}
  \significado{adj.}{desnorteado; perturbado}
\end{verbete}

\begin{verbete}{蝴蝶}{hu2die2}{15,15}
  \significado[只]{s.}{borboleta}
\end{verbete}

\begin{verbete}{虎}{hu3}{8}[Radical ⾌]
  \significado{s.}{tigre}
  \veja{老虎}{lao3hu3}
\end{verbete}

\begin{verbete}{虎虎}{hu3hu3}{8,8}
  \significado{adj.}{formidável; forte; vigoroso}
\end{verbete}

\begin{verbete}{虎口}{hu3kou3}{8,3}
  \significado{s.}{lugar perigoso; cova do tigre}
\end{verbete}

\begin{verbete}{虎鼬}{hu3you4}{8,18}
  \significado{s.}{doninha}
\end{verbete}

\begin{verbete}{互}{hu4}{4}[Radical ⼆]
  \significado{adj.}{mútuo; recíproco}
\end{verbete}

\begin{verbete}{互动}{hu4dong4}{4,6}
  \significado{s.}{interativo}
  \significado{v.}{interagir}
\end{verbete}

\begin{verbete}{互利}{hu4li4}{4,7}
  \significado{s.}{benefício mútuo}
\end{verbete}

\begin{verbete}{互相}{hu4xiang1}{4,9}
  \significado{adv.}{mutuamente, um ao outro}
\end{verbete}

\begin{verbete}{护照}{hu4zhao4}{7,13}
  \significado[本,个]{s.}{passaporte}
\end{verbete}

\begin{verbete}{化}{hua1}{4}[Radical 匕]
  \variante{花}
\end{verbete}

\begin{verbete}{花}{hua1}{7}[Radical 艸]
  \significado*{s.}{sobrenome Hua}
  \significado[朵,支,束,把,盆,簇]{s.}{flor}
\end{verbete}

\begin{verbete}{花茶}{hua1cha2}{7,9}
  \significado[杯,壶]{s.}{chá perfumado}
\end{verbete}

\begin{verbete}{花店}{hua1dian4}{7,8}
  \significado{s.}{floricultura}
\end{verbete}

\begin{verbete}{花儿}{hua1r5}{7,2}
  \significado[朵,支,束,把,盆,簇]{s.}{flor}
\end{verbete}

\begin{verbete}{花生}{hua1sheng1}{7,5}
  \significado[粒]{s.}{amendoim}
\end{verbete}

\begin{verbete}{花样游泳}{hua1yang4you2yong3}{7,10,12,8}
  \significado{s.}{nado sincronizado}
\end{verbete}

\begin{verbete}{花椰菜}{hua1ye1cai4}{7,12,11}
  \significado{s.}{couve-flor}
\end{verbete}

\begin{verbete}{划}{hua2}{6}[Radical 刀]
  \significado{adj.}{rentável; vale (o esforço); compensa (fazer alguma coisa)}
  \significado{v.}{remar; cortar; arranhar (cortar a superfície de algo); riscar (um fósforo)}
  \veja{划}{hua4}
\end{verbete}

\begin{verbete}{划船}{hua2chuan2}{6,11}
  \significado{v.}{remar um barco}
\end{verbete}

\begin{verbete}{划艇}{hua2ting3}{6,12}
  \significado{s.}{barco a remo}
\end{verbete}

\begin{verbete}{华盛顿}{hua2sheng4dun4}{6,11,10}
  \significado*{s.}{Washington}
\end{verbete}

\begin{verbete}{华氏}{hua2shi4}{6,4}
  \significado{s.}{graus Fahrenheit (°F)}
\end{verbete}

\begin{verbete}{华夏}{hua2xia4}{6,10}
  \significado*{s.}{Huaxia; nome antigo para a China; Catai}
\end{verbete}

\begin{verbete}{华裔}{hua2yi4}{6,13}
  \significado{s.}{descendente de chinês}
\end{verbete}

\begin{verbete}{滑}{hua2}{12}[Radical 水]
  \significado*{s.}{sobrenome Hua}
  \significado{adj.}{deslizado}
  \significado{v.}{deslizar}
\end{verbete}

\begin{verbete}{滑雪}{hua2xue3}{12,11}
  \significado{v.+compl.}{esquiar; fazer esqui}
\end{verbete}

\begin{verbete}{化学}{hua4xue2}{4,8}
  \significado{s.}{química (disciplina)}
\end{verbete}

\begin{verbete}{划}{hua4}{6}[Radical 刀]
  \significado{s.}{traço de um caracter chinês}
  \significado{v.}{delimitar; transferir; atribuir; planejar; desenhar (uma linha)}
  \veja{划}{hua2}
\end{verbete}

\begin{verbete}{画}{hua4}{8}[Radical 田]
  \significado[幅,张]{s.}{quadro, pintura; traço de um caractere chinês (variante de 划); (caligrafia) traço horizontal (variante de traço 划)}
  \significado{v.}{desenhar, pintar; traçar uma linha (variante de 划)}
  \veja{划}{hua4}
\end{verbete}

\begin{verbete}{画地为牢}{hua4di4wei2lao2}{8,6,4,7}
  \significado{expr.}{(literalmente) ser confinado dentro de um círculo desenhado no chão; (figurativo) limitar-se a uma gama restrita de atividades}
\end{verbete}

\begin{verbete}{话}{hua4}{8}[Radical 言]
  \significado[种,席,句,口,番]{s.}{fala; linguagem; dialeto}
\end{verbete}

\begin{verbete}{怀旧}{huai2jiu4}{7,5}
  \significado{s.}{nostalgia}
  \significado{v.}{sentir-se nostálgico}
\end{verbete}

\begin{verbete}{坏}{huai4}{7}[Radical 土]
  \significado{adj.}{avariado; mau}
  \significado{v.}{perder o controle}
\end{verbete}

\begin{verbete}{坏处}{huai4chu5}{7,5}
  \significado[个]{s.}{dano; problema}
\end{verbete}

\begin{verbete}{坏蛋}{huai4dan4}{7,11}
  \significado{s.}{bastardo; canalha; pessoa má}
\end{verbete}

\begin{verbete}{欢快}{huan1kuai4}{6,7}
  \significado{adj.}{feliz e sem ansiedade; vívido}
\end{verbete}

\begin{verbete}{欢迎}{huan1ying2}{6,7}
  \significado{adj.}{bem-vindo}
  \significado{v.}{dar as boas-vindas; ser bem-vindo}
\end{verbete}

\begin{verbete}{还}{huan2}{7}[Radical 辵]
  \significado*{s.}{sobrenome Huan}
  \significado{v.}{devolver; restituir; pagar de volta}
  \veja{还}{hai2}
\end{verbete}

\begin{verbete}{环境}{huan2jing4}{8,14}
  \significado[个]{s.}{ambiente; arredores; circunstâncias}
\end{verbete}

\begin{verbete}{环境卫生}{huan2jing4wei4sheng1}{8,14,3,5}
  \significado{s.}{saneamento ambiental}
\end{verbete}

\begin{verbete}{环卫}{huan2wei4}{8,3}
  \significado{s.}{limpeza pública; saneamento urbano; saneamento ambiental; abreviatura de 环境卫生}
  \veja{环境卫生}{huan2jing4wei4sheng1}
\end{verbete}

\begin{verbete}{幻觉}{huan4jue2}{4,9}
  \significado{s.}{ilusão; alucinação}
\end{verbete}

\begin{verbete}{换}{huan4}{10}[Radical 手]
  \significado{v.}{mudar; trocar; substituir; converter (moedas)}
\end{verbete}

\begin{verbete}{换钱}{huan4qian2}{10,10}
  \significado{v.+compl.}{trocar dinheiro (em pequenas valores ou em outra moeda)| trocar (mercadorias) por dinheiro; vender}
\end{verbete}

\begin{verbete}{荒芜}{huang1wu2}{9,7}
  \significado{adj.}{estéril}
\end{verbete}

\begin{verbete}{皇帝}{huang2di4}{9,9}
  \significado[个]{s.}{imperador}
\end{verbete}

\begin{verbete}{黄}{huang2}{11}[Radical ⻩][Kangxi 201]
  \significado*{s.}{sobrenome Huang ou Hwang}
  \significado{adj.}{amarelo; pornográfico}
\end{verbete}

\begin{verbete}{黄瓜}{huang2gua1}{11,5}
  \significado[条]{s.}{pepino}
\end{verbete}

\begin{verbete}{黄昏}{huang2hun1}{11,8}
  \significado{s.}{anoitecer}
\end{verbete}

\begin{verbete}{黄色}{huang2se4}{11,6}
  \significado{s.}{cor amarela}
\end{verbete}

\begin{verbete}{黄油}{huang2you2}{11,8}
  \significado[盒]{s.}{manteiga}
\end{verbete}

\begin{verbete}{谎话}{huang3hua4}{11,8}
  \significado{s.}{mentira}
\end{verbete}

\begin{verbete}{灰色}{hui1se4}{6,6}
  \significado{s.}{cor cinza}
\end{verbete}

\begin{verbete}{挥汗如雨}{hui1han4ru2yu3}{9,6,6,8}
  \significado{s.}{suor derramado}
  \significado{v.}{pingar com suor}
\end{verbete}

\begin{verbete}{囘}{hui2}{5}
  \variante{回}
\end{verbete}

\begin{verbete}{回}{hui2}{6}[Radical ⼞]
  \significado{v.}{regressar}
\end{verbete}

\begin{verbete}{回答}{hui2da2}{6,12}
  \significado{v.}{responder}
\end{verbete}

\begin{verbete}{回来}{hui2lai5}{6,7}
  \significado{v.}{regressar; voltar; estar de volta; (para a minha localização)}
\end{verbete}

\begin{verbete}{回去}{hui2qu5}{6,5}
  \significado{v.}{regressar; voltar; estar de volta; (a partir da minha localização)}
\end{verbete}

\begin{verbete}{回信}{hui2xin4}{6,9}
  \significado{s.}{uma carta em resposta | uma mensagem verbal em resposta}
  \significado{v.+compl.}{escrever em resposta; escrever de volta; responder uma carta | responder verbalmente uma mensagem}
\end{verbete}

\begin{verbete}{回旋}{hui2xuan2}{6,11}
  \significado{v.}{circular; rodar; dar a volta}
\end{verbete}

\begin{verbete}{廻}{hui2}{8}
  \variante{回}
\end{verbete}

\begin{verbete}{会}{hui4}{6}[Radical 人]
  \significado{s.}{reunião; (sufixo) união, grupo, associação}
  \significado{v.}{saber, ter habilidade,  saber como fazer; ser provável, ter certeza de; encontrar, reunir-se}
  \veja{会}{kuai4}
\end{verbete}

\begin{verbete}{会首}{hui4shou3}{6,9}
  \significado{s.}{chefe de uma sociedade; patrocinador de uma organização}
\end{verbete}

\begin{verbete}{婚礼}{hun1li3}{11,5}
  \significado[场]{s.}{casamento; núpcias; cerimônia de casamento}
\end{verbete}

\begin{verbete}{魂}{hun2}{13}[Radical 鬼]
  \significado{s.}{alma; espírito; alma imortal (que pode ser separada do corpo)}
\end{verbete}

\begin{verbete}{混饭}{hun4fan4}{11,7}
  \significado{v.+compl.}{trabalhar para viver}
\end{verbete}

\begin{verbete}{混乱}{hun4luan4}{11,7}
  \significado{adj.}{confuso; caótico; desordenado}
  \significado{s.}{caos}
\end{verbete}

\begin{verbete}{和}{huo2}{8}[Radical 口]
  \significado{v.}{combinar uma substância em pó (farinha, gesso, etc.) com água}
  \veja{和}{he2}
  \veja{和}{he4}
  \veja{和}{hu2}
  \veja{和}{huo4}
\end{verbete}

\begin{verbete}{活}{huo2}{9}[Radical 水]
  \significado{adj.}{vivo}
  \significado{s.}{trabalho; mão de obra}
  \significado{v.}{viver}
\end{verbete}

\begin{verbete}{活动}{huo2dong4}{9,6}
  \significado[项,个]{s.}{atividade; evento; campanha}
  \significado{v.}{exercer; operar}
\end{verbete}

\begin{verbete}{活力}{huo2li4}{9,2}
  \significado{s.}{energia; vitalidade; vigor; força vital}
\end{verbete}

\begin{verbete}{活路}{huo2lu4}{9,13}
  \significado{s.}{maneira de sobreviver; meio de subsistência}
  \veja{活路}{huo2lu5}
\end{verbete}

\begin{verbete}{活路}{huo2lu5}{9,13}
  \significado{s.}{labor; trabalho físico}
  \veja{活路}{huo2lu4}
\end{verbete}

\begin{verbete}{活着}{huo2zhe5}{9,11}
  \significado{adj.}{vivo}
\end{verbete}

\begin{verbete}{火}{huo3}{4}[Radical 火][Kangxi 86]
  \significado*{s.}{sobrenome Huo}
  \significado{adj.}{urgente; ardente ou flamejante; quente (popular)}
  \significado{clas.}{para unidades militares (antigo)}
  \significado{s.}{fogo; munição; calor interno (medicina chinesa)}
\end{verbete}

\begin{verbete}{火柴}{huo3chai2}{4,10}
  \significado[根,盒]{s.}{fósforo (palito de fósforo)}
\end{verbete}

\begin{verbete}{火车}{huo3che1}{4,4}
  \significado[列,节,班,趟]{s.}{trem; comboio}
\end{verbete}

\begin{verbete}{火车司机}{huo3che1 si1ji1}{4,4,5,6}
  \significado{s.}{maquinista de trem}
\end{verbete}

\begin{verbete}{火海}{huo3hai3}{4,10}
  \significado{s.}{um mar de chamas}
\end{verbete}

\begin{verbete}{和}{huo4}{8}[Radical 口]
  \significado{clas.}{para fervuras de ervas medicinais; para enxágue de roupas}
  \significado{v.}{misturar; misturar (ingredientes) juntos}
  \veja{和}{he2}
  \veja{和}{he4}
  \veja{和}{hu2}
  \veja{和}{huo2}
\end{verbete}

\begin{verbete}{或}{huo4}{8}[Radical 戈]
  \significado{conj.}{ou; ou\dots ou\dots}
\end{verbete}

\begin{verbete}{或者}{huo4zhe3}{8,8}
  \significado{conj.}{ou (usado em expressões afirmativas)}
\end{verbete}

\begin{verbete}{货车}{huo4che1}{8,4}
  \significado{s.}{caminhão; van; vagão de carga}
\end{verbete}

\begin{verbete}{惑星}{huo4xing1}{12,9}
  \significado{s.}{planeta}
  \veja{行星}{xing2xing1}
\end{verbete}

%%%%% EOF %%%%%


%%%%%%%%%%%%%%%%%%%% Não existem palavras com pinyin iniciado em "I"
%%%
%%% J
%%%
\section*{J}
\addcontentsline{toc}{section}{J}

\begin{verbete}[ji1]{鸡}[7]
\begin{pronuncia}{ji1}
\significado[只]{s.}{ galo, galinha; gíria: prostituta }
\end{pronuncia}
\end{verbete}

\begin{verbete}[ji1dan4]{鸡蛋}[7;11]
\begin{pronuncia}{ji1dan4}
\significado[个,打]{s.}{ ovo de galinha }
\end{pronuncia}
\end{verbete}

\begin{verbete}[ji1chang3]{机场}[6;6]
\begin{pronuncia}{ji1chang3}
\significado[家,处]{s.}{ aeroporto }
\end{pronuncia}
\end{verbete}

\begin{verbete}[...ji2le0]{······极了}[7;2]
\begin{pronuncia}{...ji2le0}
\significado{expr.}{ muito; extremamente }
\end{pronuncia}
\end{verbete}

\begin{verbete}[ji2ge2]{及格}[3;10]
\begin{pronuncia}{ji2ge2}
\significado{v.}{ aprovar; passar no exame }
\end{pronuncia}
\end{verbete}

\begin{verbete}[ji3]{几}[2]
\begin{pronuncia}{ji3}
\significado{interr.}{ quantos? (até 10 itens); alguns? (até 10 itens) }
\end{pronuncia}
\end{verbete}

\begin{verbete}[ji4jie2]{季节}[8;5]
\begin{pronuncia}{ji4jie2}
\significado[个]{s.}{ estação (clima) }
\end{pronuncia}
\end{verbete}

\begin{verbete}[jia1]{家}[10]
\begin{pronuncia}{jia1}
\significado[个]{s.}{ família; casa }
\end{pronuncia}
\end{verbete}

\begin{verbete}[jia1li0]{家里}[10;7]
\begin{pronuncia}{jia1li0}
\significado{p.d.l.}{ em casa }
\end{pronuncia}
\end{verbete}

\begin{verbete}[jia1xiang1]{家乡}[10;3]
\begin{pronuncia}{jia1xiang1}
\significado[个]{s.}{ terra natal }
\end{pronuncia}
\end{verbete}

\begin{verbete}[Jia1na2da4]{加拿大}[5;10;3]
\begin{pronuncia}{Jia1na2da4}
\significado{s.}{ Canadá }
\end{pronuncia}
\end{verbete}

\begin{verbete}[jia1na2da4ren2]{加拿大人}[5;10;3;2]
\begin{pronuncia}[\\]{jia1na2da4ren2}
\significado{s.}{ canadense; pessoa nascida no Canadá }
\end{pronuncia}
\end{verbete}

\begin{verbete}[jian1bang3]{肩膀}[8;14]
\begin{pronuncia}{jian1bang3}
\significado{s.}{ ombro }
\end{pronuncia}
\end{verbete}

\begin{verbete}[jian3cha2]{检查}[11;9]
\begin{pronuncia}{jian3cha2}
\significado[次]{s.}{ inspeção }
\significado{v.}{ examinar; verificar; inspecionar }
\end{pronuncia}
\end{verbete}

\begin{verbete}[jian3dan1]{简单}[13;8]
\begin{pronuncia}{jian3dan1}
\significado{adj.}{ simples }
\end{pronuncia}
\end{verbete}

\begin{verbete}[jian4]{见}[4]
\begin{pronuncia}{jian4}
\significado{v.}{ ver; encontrar alguém }
\end{pronuncia}
\end{verbete}

\begin{verbete}[jian4mian4]{见面}[4;9]
\begin{pronuncia}{jian4mian4}
\significado{v.}{ encontrar-se com alguém }
\end{pronuncia}
\end{verbete}

\begin{verbete}[jian4]{件}[6]
\begin{pronuncia}{jian4}
\significado{p.c.}{ para roupas }
\end{pronuncia}
\end{verbete}

\begin{verbete}[jian4yi4]{建议}[8;5]
\begin{pronuncia}{jian4yi4}
\significado[个,点]{s.}{ sugestão }
\significado{v.}{ sugerir }
\end{pronuncia}
\end{verbete}

\begin{verbete}[Jiang1xi1]{江西}[6;6]
\begin{pronuncia}{Jiang1xi1}
\significado{s.}{ Jiangxi }
\end{pronuncia}
\end{verbete}

\begin{verbete}[jiao1tong1]{交通}[6;10]
\begin{pronuncia}{jiao1tong1}
\significado{s.}{ transporte; tráfego; trânsito }
\end{pronuncia}
\end{verbete}

\begin{verbete}[jiao1qu1]{郊区}
\begin{pronuncia}{jiao1qu1}
\significado[个]{s.}{ subúrbio; arredores }
\end{pronuncia}
\end{verbete}

\begin{verbete}[jiao1juan3]{胶卷}[8;4]
\begin{pronuncia}{jiao1juan3}
\significado{s.}{ filme; película; rolo }
\end{pronuncia}
\end{verbete}

\begin{verbete}[jiao3]{脚}[11]
\begin{pronuncia}{jiao3}
\significado[双,只]{s.}{ pé }
\significado{p.c.}{ para crianças }
\end{pronuncia}
\end{verbete}

\begin{verbete}[jiao3]{角}[7]
\begin{pronuncia}{jiao3}
\significado{p.c.}{ 1 jiao = 10 centavos }
\end{pronuncia}
\end{verbete}

\begin{verbete}[jiao3zi0]{饺子}[9;3]
\begin{pronuncia}{jiao3zi0}
\significado[个,只]{s.}{ jiaozi; raviólis chineses; bolinho de massa }
\end{pronuncia}
\end{verbete}

\begin{verbete}[jiao4]{叫}[5]
\begin{pronuncia}{jiao4}
\significado{v.}{ chamar-se; chamar }
\end{pronuncia}
\end{verbete}

\begin{verbete}[jiao4]{教}[11]
\begin{pronuncia}{jiao4}
\significado{v.}{ ensinar }
\end{pronuncia}
\end{verbete}

\begin{verbete}[jiao4lian4]{教练}[11;8]
\begin{pronuncia}{jiao4lian4}
\significado[个,位,名]{s.}{ treinador }
\end{pronuncia}
\end{verbete}

\begin{verbete}[jiao4shou4]{教授}[11;11]
\begin{pronuncia}{jiao4shou4}
\significado[个,位]{s.}{ professor }
\end{pronuncia}
\end{verbete}

\begin{verbete}[jiao4shi1]{教师}[11;6]
\begin{pronuncia}{jiao4shi1}
\significado[个]{s.}{ professor; mestre }
\end{pronuncia}
\end{verbete}

\begin{verbete}[jiao4shi4]{教室}[11;9]
\begin{pronuncia}{jiao4shi4}
\significado[间]{s.}{ sala de aula }
\end{pronuncia}
\end{verbete}

\begin{verbete}[jiao4xue2lou2]{教学楼}[11;8;13]
\begin{pronuncia}{jiao4xue2lou2}
\significado{s.}{ edifício de salas de aula }
\end{pronuncia}
\end{verbete}

\begin{verbete}[jie1]{街}[12]
\begin{pronuncia}{jie1}
\significado[条]{s.}{ rua }
\end{pronuncia}
\end{verbete}

\begin{verbete}[jie1]{接}[11]
\begin{pronuncia}{jie1}
\significado{v.}{ ir buscar (alguém); ir ao encontro de (alguém); receber }
\end{pronuncia}
\end{verbete}

\begin{verbete}[jie1]{接(电话)}[5;5;8]
\begin{pronuncia}{jie1}
\significado{v.}{ atender (o telefone) }
\end{pronuncia}
\end{verbete}

\begin{verbete}[jie1dai4]{接待}[11;9]
\begin{pronuncia}{jie1dai4}
\significado{v.}{ receber (alguém); acolher; recepcionar }
\end{pronuncia}
\end{verbete}

\begin{verbete}[jie2ri4]{节日}[5;4]
\begin{pronuncia}{jie2ri4}
\significado[个]{s.}{ festa }
\end{pronuncia}
\end{verbete}

\begin{verbete}[jie2guo3]{结果}[9;8]
\begin{pronuncia}{jie2guo3}
\significado{s.}{ resultado; consequência }
\end{pronuncia}
\end{verbete}

\begin{verbete}[jie3jie0]{姐姐}[8;8]
\begin{pronuncia}{jie3jie0}
\significado[个]{s.}{ irmã mais velha }
\end{pronuncia}
\end{verbete}

\begin{verbete}[jie3fu0]{姐夫}[8;4]
\begin{pronuncia}{jie3fu0}
\significado{s.}{ marido da irmã mais velha }
\end{pronuncia}
\end{verbete}

\begin{verbete}[jie4shao4]{介绍}[4;8]
\begin{pronuncia}{jie4shao4}
\significado{s.}{ apresentação }
\significado{v.}{ apresentar }
\end{pronuncia}
\end{verbete}

\begin{verbete}[jie4]{借}[10]
\begin{pronuncia}{jie4}
\significado{v.}{ pedir emprestado; emprestar }
\end{pronuncia}
\end{verbete}

\begin{verbete}[jie4shu1zheng4]{借书证}[10;4;7]
\begin{pronuncia}{jie4shu1zheng4}
\significado{s.}{ cartão de biblioteca; literalmente: cartão para pedir emprestado livros }
\end{pronuncia}
\end{verbete}

\begin{verbete}[jin1nian2]{今年}[4;6]
\begin{pronuncia}{jin1nian2}
\significado{p.t.}{ este ano }
\end{pronuncia}
\end{verbete}

\begin{verbete}[jin1tian1]{今天}[4;4]
\begin{pronuncia}{jin1tian1}
\significado{p.t.}{ hoje }
\end{pronuncia}
\end{verbete}

\begin{verbete}[jin1rong2]{金融}[8;16]
\begin{pronuncia}{jin1rong2}
\significado{s.}{ finança }
\end{pronuncia}
\end{verbete}

\begin{verbete}[jin4]{近}[7]
\begin{pronuncia}{jin4}
\significado{adj.}{ perto; próximo }
\end{pronuncia}
\end{verbete}

\begin{verbete}[jin4]{进}[7]
\begin{pronuncia}{jin4}
\significado{v.d.}{ entrar }
\end{pronuncia}
\end{verbete}

\begin{verbete}[jin4chu1kou3]{进出口}[7;5;3]
\begin{pronuncia}{jin4chu1kou3}
\significado{s.}{ importação e exportação }
\end{pronuncia}
\end{verbete}

\begin{verbete}[jin4kou3]{进口}[7;3]
\begin{pronuncia}{jin4kou3}
\significado{s.}{ importação }
\significado{v.}{ importar }
\end{pronuncia}
\end{verbete}

\begin{verbete}[jin4lai0]{进来}[7;7]
\begin{pronuncia}{jin4lai0}
\significado{v.d.}{ entrar (para a minha localização) }
\end{pronuncia}
\end{verbete}

\begin{verbete}[jin4qu0]{进去}[7;5]
\begin{pronuncia}{jin4qu0}
\significado{v.d.}{ entrar (a partir da minha localização) }
\end{pronuncia}
\end{verbete}

\begin{verbete}[jing1chang2]{经常}[8;11]
\begin{pronuncia}{jing1chang2}
\significado{adv.}{ muitas vezes }
\end{pronuncia}
\end{verbete}

\begin{verbete}[jing1ji4]{经济}[8;9]
\begin{pronuncia}{jing1ji4}
\significado{s.}{ economia }
\end{pronuncia}
\end{verbete}

\begin{verbete}[jing1li3]{经理}[8;11]
\begin{pronuncia}{jing1li3}
\significado[个,位,名]{s.}{ gerente }
\end{pronuncia}
\end{verbete}

\begin{verbete}[jing3cha2]{警察}[19;14]
\begin{pronuncia}{jing3cha2}
\significado[个]{s.}{ polícia; agente de polícia }
\end{pronuncia}
\end{verbete}

\begin{verbete}[jiu3]{九}[2]
\begin{pronuncia}{jiu3}
\significado{num.}{ nove; 9 }
\end{pronuncia}
\end{verbete}

\begin{verbete}[jiu3cai4]{韭菜}[9;11]
\begin{pronuncia}{jiu3cai4}
\significado{s.}{ cebolinha chinesa }
\end{pronuncia}
\end{verbete}

\begin{verbete}[jiu3]{酒}[10]
\begin{pronuncia}{jiu3}
\significado[杯,瓶,罐,桶,缸]{s.}{ bebida alcoólica; vinho; aguardente; licor }
\end{pronuncia}
\end{verbete}

\begin{verbete}[jiu3guan3]{酒馆}[10;11]
\begin{pronuncia}{jiu3guan3}
\significado{s.}{ bar }
\end{pronuncia}
\end{verbete}

\begin{verbete}[jiu4]{就}[12]
\begin{pronuncia}{jiu4}
\significado{adv.}{ exatamente; justamente }
\end{pronuncia}
\end{verbete}

\begin{verbete}[ju4]{句}[5]
\begin{pronuncia}{ju4}
\significado{s.}{ sentença; cláusula; frase }
\significado{p.c.}{ para orações, frases ou linhas de versos}
\end{pronuncia}
\end{verbete}

\begin{verbete}[ju4zi0]{句子}[5;3]
\begin{pronuncia}{ju4zi0}
\significado[个]{n}{ sentença }
\end{pronuncia}
\end{verbete}

\begin{verbete}[ju3xing2]{举行}[9;6]
\begin{pronuncia}{ju3xing2}
\significado{v.}{ realizar; ter lugar }
\end{pronuncia}
\end{verbete}

\begin{verbete}[jue2de2]{觉得}[9;11]
\begin{pronuncia}{jue2de2}
\significado{v.}{ achar; sentir }
\end{pronuncia}
\end{verbete}

%%%%% EOF %%%%

%%%
%%% K
%%%
%\section*{K}
\addcontentsline{toc}{section}{K}

\begin{verbete}{咖啡}{ka1fei1}{8;11}
  \significado[杯]{s.}{café (empréstimo linguístico)}
\end{verbete}

\begin{verbete}{咖啡馆}{ka1fei1guan3}{8;11;11}
  \significado[家]{s.}{cafeteria}
\end{verbete}

\begin{verbete}{咖啡色}{ka1fei1se4}{8;11;6}
  \significado{s.}{cor café}
\end{verbete}

\begin{verbete}{卡车司机}{ka3che1 si1ji1}{5;4;5;6}
  \significado{s.}{motorista de caminhão}
\end{verbete}

\begin{verbete}{卡片}{ka3pian4}{5;4}
  \significado{s.}{cartão}
\end{verbete}

\begin{verbete}{卡片游戏}{ka3pian4 you2xi4}{5;4;12;6}
  \significado{s.}{carta de baralho}
\end{verbete}

\begin{verbete}{卡通}{ka3tong1}{5;10}
  \significado{s.}{\emph{cartoon} (empréstimo linguístico)}
\end{verbete}

\begin{verbete}{开}{kai1}{4}[Radical 廾][Componentes 一廾]
  \significado{p.c.}{quilate (ouro)}
  \significado{v.}{abrir; ligar; dirigir; iniciar (alguma coisa); ferver; escrever  (uma receita, cheque, fatura, etc.)}
\end{verbete}

\begin{verbete}{开车}{kai1che1}{4;4}
  \significado{v.+compl.}{conduzir; dirigir}
\end{verbete}

\begin{verbete}{开尔文}{kai1'er3wen2}{4;5;4}
  \significado{s.}{Kelvin (K, escala de temperatura)}
\end{verbete}

\begin{verbete}{开发区}{kai1fa1qu1}{4;5;4}
  \significado{s.}{zona de desenvolvimento}
\end{verbete}

\begin{verbete}{开花}{kai1hua1}{4;7}
  \significado{v.}{florescer; (fig.) explodir, abrir-se; (fig.) explodir de alegria; (fig.) começar a existir de repente em todos os lugares}
\end{verbete}

\begin{verbete}{开口}{kai1kou3}{4;3}
  \significado{v.}{abrir a boca de alguém; começar a falar}
\end{verbete}

\begin{verbete}{开启}{kai1qi3}{4;7}
  \significado{v.}{abrir; iniciar; (computação) ativar}
\end{verbete}

\begin{verbete}{开始}{kai1shi3}{4;8}
  \significado{adv.}{inicial}
  \significado[个]{s.}{começo; início}
  \significado{v.}{começar; iniciar}
\end{verbete}

\begin{verbete}{开锁}{kai1suo3}{4;12}
  \significado{v.}{desbloquear, destravar}
\end{verbete}

\begin{verbete}{开头}{kai1tou2}{4;5}
  \significado{s.}{início; começo}
  \significado{v.}{iniciar; começar}
\end{verbete}

\begin{verbete}{开心}{kai1xin1}{4;4}
  \significado{v.}{sentir-se feliz; regozijar-se; divertir-se; tirar sarro de alguém}
\end{verbete}

\begin{verbete}{开夜车}{kai1ye4che1}{4;8;4}
  \significado{expr.}{trabalho noturno; literalmente:~``conduzir carro à noite''}
\end{verbete}

\begin{verbete}{看}{kan1}{9}[Radical 目][Componentes 龵目]
  \significado{v.}{cuidar; vigiar}
  \veja{看}{kan4}
\end{verbete}

\begin{verbete}{砍}{kan3}{9}[Radical 石][Componentes 石欠]
  \significado{v.}{cortar}
\end{verbete}

\begin{verbete}{砍刀}{kan3dao1}{9;2}
  \significado{s.}{facão; machete}
\end{verbete}

\begin{verbete}{砍掉}{kan3diao4}{9;11}
  \significado{v.}{amputar}
\end{verbete}

\begin{verbete}{砍断}{kan3duan4}{9;11}
  \significado{v.}{cortar}
\end{verbete}

\begin{verbete}{砍价}{kan3jia4}{9;6}
  \significado{v.}{barganhar; cortar ou derrubar um preço}
\end{verbete}

\begin{verbete}{砍杀}{kan3sha1}{9;6}
  \significado{v.}{atacar com arma branca}
\end{verbete}

\begin{verbete}{砍伤}{kan3shang1}{9;6}
  \significado{v.}{ferir com lâmina ou machado}
\end{verbete}

\begin{verbete}{砍树}{kan3shu4}{9;9}
  \significado{v.}{derrubar árvores}
\end{verbete}

\begin{verbete}{砍死}{kan3si3}{9;6}
  \significado{v.}{matar com um machado}
\end{verbete}

\begin{verbete}{砍头}{kan3tou2}{9;5}
  \significado{v.}{decapitar}
\end{verbete}

\begin{verbete}{看}{kan4}{9}[Radical 目][Componentes 龵目]
  \significado{interj.}{Cuidado! (para um perigo)}
  \significado{part.}{(depois de um verbo) tentar}
  \significado{v.}{olhar; ver; assistir; ler; visitar (pessoas)}
  \veja{看}{kan1}
\end{verbete}

\begin{verbete}{看淡}{kan4dan4}{9;11}
  \significado{v.}{considerar sem importância; ser indiferente a (fama, riqueza, etc.); (de uma economia ou mercado) enfraquecer, ficar mais lento, diminuir a velocidade}
\end{verbete}

\begin{verbete}{看到}{kan4dao4}{9;8}
  \significado{v.}{ver}
\end{verbete}

\begin{verbete}{看法}{kan4fa3}{9;8}
  \significado[个]{s.}{modo de olhar alguma coisa; ponto de vista; opinião}
\end{verbete}

\begin{verbete}{看见}{kan4jian4}{9;4}
  \significado{v.}{encontrar; enxergar; ver; avistar}
\end{verbete}

\begin{verbete}{看起来}{kan4qi3lai5}{9;10;7}
  \significado{adv.}{aparentemente, parece como se, parece ser, dá a impressão de, parece estar em face disso}
\end{verbete}

\begin{verbete}{扛}{kang2}{6}[Radical 手][Componentes ⺘⼯]
  \significado{v.}{carregar no ombro de alguém;  (fig.) assumir (um fardo, dever, etc.)}
  \veja{扛}{gang1}
\end{verbete}

\begin{verbete}{考试}{kao3shi4}{6;8}
  \significado[次]{s.}{teste; prova; exame}
  \significado{v.+compl.}{submeter-se a uma prova; fazer um teste}
\end{verbete}

\begin{verbete}{烤}{kao3}{10}[Radical 火][Componentes 火考]
  \significado{v.}{assar; grelhar}
\end{verbete}

\begin{verbete}{烤肉}{kao3rou4}{10;6}
  \significado{s.}{churrasco}
\end{verbete}

\begin{verbete}{科技}{ke1ji4}{9;7}
  \significado{s.}{ciência e tecnologia}
\end{verbete}

\begin{verbete}{科学家}{ke1xue2jia1}{9;8;10}
  \significado[个]{s.}{cientista}
\end{verbete}

\begin{verbete}{颗}{ke1}{14}[Radical 頁][Componentes 果页]
  \significado{p.c.}{para grãos, pérolas, dentes, corações, satelites, pequenas esferas, etc.}
\end{verbete}

\begin{verbete}{蝌蚪}{ke1dou3}{15;10}
  \significado{s.}{girino}
\end{verbete}

\begin{verbete}{壳}{ke2}{7}[Radical 士][Componentes 士冗]
  \significado{s.}{casca (de ovo, noz, caranguejo, etc.); caixa; invólucro; alojamento (de uma máquina ou dispositivo)}
\end{verbete}

\begin{verbete}{咳嗽}{ke2sou5}{9;14}
  \significado{v.}{ter tosse; tossir}
\end{verbete}

\begin{verbete}{可}{ke3}{5}[Radical 口][Componentes 丁口]
  \significado{adv.}{muito; realmente}
\end{verbete}

\begin{verbete}{可爱}{ke3'ai4}{5;10}
  \significado{adj.}{adorável; querido; fofo}
\end{verbete}

\begin{verbete}{可编程}{ke3bian1cheng2}{5;12;12}
  \significado{adj.}{programável}
\end{verbete}

\begin{verbete*}{可擦写可编程只读存储器}{ke3ca1xie3ke3bian1cheng2zhi1du2cun2chu3qi4}{5;17;5;5;12;12;5;10;6;12;16}
  \significado{s.}{EPROM (\emph{erasable programmable read-only memory})}
\end{verbete*}

\begin{verbete}{可卡因}{ke3ka3yin1}{5;5;6}
  \significado{s.}{cocaína (empréstimo linguístico)}
\end{verbete}

\begin{verbete}{可口可乐}{ke3kou3ke3le4}{5;3;5;5}
  \significado*{s.}{Coca-Cola (empréstimo linguístico)}
\end{verbete}

\begin{verbete}{可能}{ke3neng2}{5;10}
  \significado{adj.}{possível; provável}
  \significado{adv.}{possivelmente; provavelmente}
  \significado[个]{s.}{possibilidade; probabilidade}
\end{verbete}

\begin{verbete}{可怕}{ke3pa4}{5;8}
  \significado{adj.}{horrível; terrível; formidável; assustador; hediondo}
  \significado{adv.}{terrivelmente}
\end{verbete}

\begin{verbete}{可是}{ke3shi4}{5;9}
  \significado{adv.}{(usado para dar ênfase) de fato}
  \significado{conj.}{porém; contudo; mas}
\end{verbete}

\begin{verbete}{可惜}{ke3xi1}{5;11}
  \significado{adj.}{é uma pena; que pena}
  \significado{adv.}{infelizmente; que pena; é uma pena}
\end{verbete}

\begin{verbete}{可以}{ke3yi3}{5;4}
  \significado{v.o.}{ser capaz de; poder}
\end{verbete}

\begin{verbete}{渴}{ke3}{12}[Radical 水][Componentes 氵曷]
  \significado{adj.}{sedento}
\end{verbete}

\begin{verbete}{刻}{ke4}{8}[Radical 刀][Componentes 亥刂]
  \significado{p.c.}{para curtos intervalos de tempo}
  \significado{p.t.}{quarto (de hora)}
  \significado{v.}{esculpir; cortar; gravar}
\end{verbete}

\begin{verbete}{刻画}{ke4hua4}{8;8}
  \significado{v.}{retratar; tirar um retrato}
\end{verbete}

\begin{verbete}{刻钟}{ke4 zhong1}{8;9}
  \significado{p.t.}{um quarto de hora}
\end{verbete}

\begin{verbete}{客气}{ke4qi5}{9;4}
  \significado{adj.}{cortês; delicado; modesto; educado}
  \significado{v.}{fazer cerimônia}
\end{verbete}

\begin{verbete}{客厅}{ke4ting1}{9;4}
  \significado[间]{s.}{sala de estar; sala de visitas}
\end{verbete}

\begin{verbete}{课本}{ke4ben3}{10;5}
  \significado[本]{s.}{livro do aluno; manual}
\end{verbete}

\begin{verbete}{肯定}{ken3ding4}{8;8}
  \significado{adv.}{com certeza; certamente; definitivamente; afirmativo (resposta)}
  \significado{v.}{afirmar; ter a certeza; ser positivo; dar reconhecimento}
\end{verbete}

\begin{verbete}{坑}{keng1}{7}[Radical 土][Componentes 土亢]
  \significado{s.}{poço; depressão; túnel; buraco no chão}
  \significado{v.}{enganar; trapacear}
\end{verbete}

\begin{verbete}{空间}{kong1jian1}{8;7}
  \significado{s.}{espaço; sala; escopo (figurativo); espaço sideral (astronomia); espaço (matemática, física)}
\end{verbete}

\begin{verbete}{空间站}{kong1jian1zhan4}{8;7;10}
  \significado{s.}{estação espacial}
\end{verbete}

\begin{verbete}{空气}{kong1qi4}{8;4}
  \significado{s.}{ar; atmosfera}
\end{verbete}

\begin{verbete}{空调}{kong1tiao2}{8;10}
  \significado[台]{s.}{ar-condicionado; condicionador de ar}
\end{verbete}

\begin{verbete}{空心菜}{kong1xin1cai4}{8;4;11}
  \veja{蕹菜}{weng4cai4}
\end{verbete}

\begin{verbete}{孔}{kong3}{4}[Radical 子][Componentes 子乚]
  \significado*{s.}{sobrenome Kong}
  \significado{p.c.}{para habitações em cavernas}
  \significado[个]{s.}{buraco}
\end{verbete}

\begin{verbete}{孔夫子}{kong3fu1zi3}{4;4;3}
  \significado*{s.}{Confúcio (551-479 aC), pensador e filósofo social chinês}
  \veja{孔子}{kong3zi3}
\end{verbete}

\begin{verbete}{孔雀}{kong3que4}{4;11}
  \significado{s.}{pavão}
\end{verbete}

\begin{verbete}{孔子}{kong3zi3}{4;3}
  \significado*{s.}{Confúcio (551-479 aC), pensador e filósofo social chinês}
  \veja{孔夫子}{kong3fu1zi3}
\end{verbete}

\begin{verbete}{孔子学院}{kong3zi3 xue2yuan4}{4;3;8;9}
  \significado*{s.}{Instituto Confúcio, organização estabelecida internacionalmente pela República Popular da China, que promove a língua e a cultura chinesas}
\end{verbete}

\begin{verbete}{恐怖主义}{kong3bu4zhu3yi4}{10;8;5;3}
  \significado{adj.}{terrorista}
  \significado{s.}{terrorismo}
\end{verbete}

\begin{verbete}{恐龙}{kong3long2}{10;5}
  \significado[头,只]{s.}{dinossauro}
\end{verbete}

\begin{verbete}{恐怕}{kong3pa4}{10;8}
  \significado{adv.}{talvez; possivelmente; provavelmente; (em sentido não tão bom)}
  \significado{v.}{temer}
\end{verbete}

\begin{verbete}{空儿}{kong4r5}{8;2}
  \significado{s.}{tempo livre}
  \significado{v.}{ter tempo livre}
\end{verbete}

\begin{verbete}{控制}{kong4zhi4}{11;8}
  \significado{v.}{controlar}
\end{verbete}

\begin{verbete}{口}{kou3}{3}[Radical 口][Componentes 口][Kangxi 30]
  \significado{p.c.}{para coisas com bocas (pessoas, animais domésticos, canhões, etc.); para mordidas ou bocados}
  \significado{s.}{boca}
\end{verbete}

\begin{verbete}{口袋}{kou3dai4}{3;11}
  \significado{s.}{bolso; saco}
\end{verbete}

\begin{verbete}{口袋妖怪}{kou3dai4 yao1guai4}{3;11;7;8}
  \significado*{s.}{\emph{Pokémon}}
\end{verbete}

\begin{verbete}{口香糖}{kou3xiang1tang2}{3;9;16}
  \significado{s.}{goma de mascar; chiclete}
\end{verbete}

\begin{verbete}{口音}{kou3yin1}{3;9}
  \significado{s.}{sons da fala oral (linguística)}
  \veja{口音}{kou3yin5}
\end{verbete}

\begin{verbete}{口音}{kou3yin5}{3;9}
  \significado{s.}{sotaque; voz}
  \veja{口音}{kou3yin1}
\end{verbete}

\begin{verbete}{口语}{kou3yu3}{3;9}
  \significado[门]{s.}{linguagem oral; linguagem falada; fofoca; calúnia}
\end{verbete}

\begin{verbete}{枯木}{ku1mu4}{9;4}
  \significado{s.}{árvore morta; madeira morta}
\end{verbete}

\begin{verbete}{哭}{ku1}{10}[Radical 口][Componentes 吅犬]
  \significado{v.}{chorar}
\end{verbete}

\begin{verbete}{哭墙}{ku1qiang2}{10;14}
  \significado*{s.}{Muro das Lamentações (Jerusalém)}
\end{verbete}

\begin{verbete}{苦瓜}{ku3gua1}{8;5}
  \significado{s.}{melão amargo (cabaça amarga, pêra bálsamo, maçã bálsamo, pepino amargo)}
\end{verbete}

\begin{verbete}{裤子}{ku4zi5}{12;3}
  \significado[条]{s.}{calças}
\end{verbete}

\begin{verbete}{酷}{ku4}{14}[Radical 酉][Componentes 酉告]
  \significado{adj.}{impiedoso; forte (por exemplo, vinho); (empréstimo linguístico) legal, \emph{cool}}
\end{verbete}

\begin{verbete}{酷斯拉}{ku4si1la1}{14;12;8}
  \significado*{s.}{Godzilla (Japonês ゴジラ Gojira)}
  \veja{哥斯拉}{ge1si1la1}
\end{verbete}

\begin{verbete}{会}{kuai4}{6}[Radical 人][Componentes 人云]
  \significado{s.}{contador; contabilidade}
  \significado{v.}{equilibrar contas}
  \veja{会}{hui4}
\end{verbete}

\begin{verbete}{块}{kuai4}{7}[Radical 土][Componentes 土夬]
  \significado{p.c.}{coloquial:~para dinheiro e unidades monetárias; para peças ou pedaços de roupa, bolos, sabão, etc.}
  \significado{s.}{pedaço; pedaço (de terra); peça}
\end{verbete}

\begin{verbete}{快}{kuai4}{7}[Radical 心][Componentes 忄夬]
  \significado{adj.}{quase; rápido; depressa}
  \significado{v.}{apressar-se}
\end{verbete}

\begin{verbete}{快递}{kuai4di4}{7;10}
  \significado{s.}{entrega expressa}
\end{verbete}

\begin{verbete}{快乐}{kuai4le4}{7;5}
  \significado{adj.}{feliz; alegre}
  \significado{s.}{felicidade; alegria}
\end{verbete}

\begin{verbete}{快速}{kuai4su4}{7;10}
  \significado{adj.}{veloz, de alta velocidade, rápido}
\end{verbete}

\begin{verbete}{宽影片}{kuan1ying3pian4}{10;15;4}
  \significado{s.}{filme \emph{widescreen}}
\end{verbete}

\begin{verbete}{款}{kuan3}{12}[Radical 欠][Componentes 士欠示]
  \significado{p.c.}{para versões ou modelos (de um produto)}
  \significado[笔,个]{s.}{montante de dinheiro; fundos; parágrafo; seção}
\end{verbete}

\begin{verbete}{窾}{kuan3}{17}[Radical 穴][Componentes 穴款]
  \significado{adj.}{oco}
  \significado{s.}{rachadura; cavidade; onomatopéia:~água atingindo a rocha}
  \significado{v.}{escavar um buraco}
  \veja{窾}{cuan4}
\end{verbete}

\begin{verbete}{狂欢节}{kuang2huan1jie2}{7;6;5}
  \significado*{s.}{Carnaval}
\end{verbete}

\begin{verbete}{况且}{kuang4qie3}{7;5}
  \significado{conj.}{além disso; além do mais}
\end{verbete}

\begin{verbete}{旷野}{kuang4ye3}{7;11}
  \significado{s.}{região selvagem}
\end{verbete}

\begin{verbete}{矿泉水}{kuang4quan2shui3}{8;9;4}
  \significado[瓶,杯]{s.}{água mineral}
\end{verbete}

\begin{verbete}{葵花}{kui2hua1}{12;7}
  \significado{s.}{girassol (flor)}
\end{verbete}

\begin{verbete}{困}{kun4}{7}[Radical ⼞][Componentes 囗木]
  \significado{adj.}{pressionado; preso}
  \significado{v.}{prender; cercar}
\end{verbete}

\begin{verbete}{困难}{kun4nan5}{7;10}
  \significado{adj.}{difícil; desafiante}
  \significado{s.}{situação difícil}
\end{verbete}

%%%%% EOF %%%%%

%%%
%%% L
%%%

\section*{L}\addcontentsline{toc}{section}{L}

\begin{EntryWithPhonetic}{垃}{la1}{8}{⼟}
  \definition[堆]{s.}{lixo}
\end{EntryWithPhonetic}

\begin{EntryWithPhonetic}{垃圾}{la1 ji1}{8,6}{⼟、⼟}[HSK 4]
  \definition{adj.}{lixo; inútil, ruim ou prejudicial}
  \definition[袋,桶,堆,车,片]{s.}{entulho; lixo; refugo; rejeito; resíduo; coisa inútil que é jogada fora; metáfora para alguém ou algo que perdeu seu valor ou serve a um propósito ruim}
\end{EntryWithPhonetic}

\begin{EntryWithPhonetic}{垃圾车}{la1ji1che1}{8,6,4}{⼟、⼟、⾞}
  \definition{s.}{caminhão de lixo}
\end{EntryWithPhonetic}

\begin{EntryWithPhonetic}{垃圾电邮}{la1ji1 dian4you2}{8,6,5,7}{⼟、⼟、⽥、⾢}
  \definition{s.}{\emph{e-mail} de \emph{spam}}
  \seealsoref{垃圾邮件}{la1ji1 you2jian4}
\end{EntryWithPhonetic}

\begin{EntryWithPhonetic}{垃圾堆}{la1ji1dui1}{8,6,11}{⼟、⼟、⼟}
  \definition{s.}{depósito de lixo}
\end{EntryWithPhonetic}

\begin{EntryWithPhonetic}{垃圾工}{la1ji1gong1}{8,6,3}{⼟、⼟、⼯}
  \definition{s.}{lixeiro | gari}
\end{EntryWithPhonetic}

\begin{EntryWithPhonetic}{垃圾食品}{la1ji1shi2pin3}{8,6,9,9}{⼟、⼟、⾷、⼝}
  \definition{s.}{\emph{junk food}}
\end{EntryWithPhonetic}

\begin{EntryWithPhonetic}{垃圾筒}{la1ji1tong3}{8,6,12}{⼟、⼟、⽵}
  \definition{s.}{cesto de lixo}
\end{EntryWithPhonetic}

\begin{EntryWithPhonetic}{垃圾箱}{la1ji1xiang1}{8,6,15}{⼟、⼟、⾋}
  \definition{s.}{cesto de lixo}
\end{EntryWithPhonetic}

\begin{EntryWithPhonetic}{垃圾邮件}{la1ji1 you2jian4}{8,6,7,6}{⼟、⼟、⾢、⼈}
  \definition{s.}{\emph{spam}, \emph{e-mail} não solicitado}
  \seealsoref{垃圾电邮}{la1ji1 dian4you2}
\end{EntryWithPhonetic}

\begin{EntryWithPhonetic}{拉}{la1}{8}{⼿}[HSK 2]
  \definition{s.}{abreviação de América Latina, 拉丁美洲}
  \definition{v.}{puxar; arrastar; rebocar | transportar por veículo; rebocar | arrastar (ou puxar) para fora | mover (tropas para um lugar) | dar uma mãozinha; ajudar | arrastar para dentro; implicar; envolver | criar (criança) | atrair; conquistar; solicitar; angariar votos | bater-papo | organizar; preparar | ter dívidas; estar endividado | pressionar; recrutar à força | (no tênis, tênis de mesa, etc.) levantar (a bola) | tocar (certos instrumentos musicais); puxar uma parte do instrumento para que ele emita som | prolongar; espaçar | envolver-se em | (coloquial) esvaziar os intestinos | levantar, uma das técnicas do tênis de mesa | destruir; esmagar; quebrar}
  \seeref{la4}
  \seealsoref{拉丁美洲}{la1ding1 mei3zhou1}
\end{EntryWithPhonetic}

\begin{EntryWithPhonetic}{拉布布}{la1bu4bu4}{8,5,5}{⼿、⼱、⼱}
  \definition*{s.}{Labubu}
\end{EntryWithPhonetic}

\begin{EntryWithPhonetic}{拉丁美洲}{la1ding1 mei3zhou1}{8,2,9,9}{⼿、⼀、⽺、⽔}
  \definition*{s.}{América Latina, nome coletivo dos países da América Central e do Sul, devido ao fato de a maioria de seus habitantes ser descendente de povos latinos e de a língua falada ser do grupo latino}
\end{EntryWithPhonetic}

\begin{EntryWithPhonetic}{拉开}{la1 kai1}{8,4}{⼿、⼶}[HSK 4]
  \definition{v.}{puxar para abrir; recuar| ampliar; espaçar; distanciar; afastar; separar}
\end{EntryWithPhonetic}

\begin{EntryWithPhonetic}{拉拉队}{la1la1dui4}{8,8,4}{⼿、⼿、⾩}
  \definition{s.}{claque | torcida}
\end{EntryWithPhonetic}

\begin{EntryWithPhonetic}{拉萨}{la1sa4}{8,11}{⼿、⾋}
  \definition*{s.}{Lhasa, capital da Região Autônoma do Tibete, 西藏自治区}
  \seealsoref{西藏自治区}{xi1zang4 zi4zhi4qu1}
\end{EntryWithPhonetic}

\begin{EntryWithPhonetic}{啦}{la1}{11}{⼝}
  \definition{s.}{(onomatoméia) som de canto, aplausos etc.; usado para palavras como 呼啦啦, 哗啦啦, 哩哩啦啦, etc.}
  \seeref{la5}
  \seealsoref{呼啦啦}{hu1 la1 la1}
  \seealsoref{哗啦啦}{hua1la1 la5}
  \seealsoref{哩哩啦啦}{li1 li1 la1 la1}
\end{EntryWithPhonetic}

\begin{EntryWithPhonetic}{拉}{la4}{8}{⼿}
  \definition{s.}{usado em 拉拉蛄 \dpy{la4la4gu3}}
  \seeref{la1}
  \seealsoref{拉拉蛄}{la4la4gu3}
\end{EntryWithPhonetic}

\begin{EntryWithPhonetic}{拉拉蛄}{la4la4gu3}{8,8,11}{⼿、⼿、⾍}
  \variantof{蝲蝲蛄}
\end{EntryWithPhonetic}

\begin{EntryWithPhonetic}{落}{la4}{12}{⾋}[HSK 5]
  \definition{v.}{deixar de fora; estar ausente | deixar para trás; esquecer de trazer; deixar algo em algum lugar e esquecer de levar| ficar para trás (ou cair); não conseguir acompanhar}
  \seeref{lao4}
  \seeref{luo4}
\end{EntryWithPhonetic}

\begin{EntryWithPhonetic}{蜡}{la4}{14}{⾍}
  \definition{s.}{cera; óleos produzidos por animais, minerais ou plantas | vela}
\end{EntryWithPhonetic}

\begin{EntryWithPhonetic}{蜡烛}{la4zhu2}{14,10}{⾍、⽕}
  \definition[根,支]{s.}{vela | círio | peça, geralmente de cera, que possui um pavio e se utiliza para iluminar}
\end{EntryWithPhonetic}

\begin{EntryWithPhonetic}{辣}{la4}{14}{⾟}[HSK 4]
  \definition{adj.}{apimentado; picante; pungente; quente | cruel; implacável; venenoso; vicioso}
  \definition{v.}{queimar; picar; formigar; ter uma irritação picante (boca, nariz ou olhos)}
\end{EntryWithPhonetic}

\begin{EntryWithPhonetic}{蝲}{la4}{15}{⾍}
  \definition{s.}{lagostim de água doce}
  \seealsoref{蝲蛄}{la4gu3}
\end{EntryWithPhonetic}

\begin{EntryWithPhonetic}{蝲蛄}{la4gu3}{15,11}{⾍、⾍}
  \definition{s.}{lagostim; lagostim de água doce}
\end{EntryWithPhonetic}

\begin{EntryWithPhonetic}{蝲蝲蛄}{la4la4gu3}{15,15,11}{⾍、⾍、⾍}
  \definition{s.}{grilo toupeira}
\end{EntryWithPhonetic}

\begin{EntryWithPhonetic}{啦}{la5}{11}{⼝}[HSK 6]
  \definition{part.}{uma palavra composta de 了 e 啊, que tem o significado de ambos}
  \seeref{la1}
  \seealsoref{啊}{a5}
  \seealsoref{了}{le5}
\end{EntryWithPhonetic}

\begin{EntryWithPhonetic}{来}{lai2}{7}{⽊}[HSK 1]
  \definition*{s.}{Sobrenome Lai}
  \definition{part.}{usado após uma palavra numérica ou de quantidade; indica uma quantidade aproximada | usado depois de numerais como 一, 二, 三; para listar razões ou fatos, etc.}
  \definition{s.}{usado após uma expressão de tempo para indicar uma duração que vai do passado ao presente}
  \definition{v.}{vir; chegar; de outro lugar para o lugar onde o interlocutor se encontra | aparecer; acontecer; vir; (problemas, coisas, etc.) ocorrerem; surgirem | substitui um verbo com significado específico, indicando a realização de uma ação específica | estar indo para; usado antes de outro verbo, indica que algo será feito | vir para fazer algo; usado após outro verbo, indica que se vai fazer algo | usado para indicar um propósito; expressar o objetivo, fazer algo usando o método, a atitude ou a direção anteriores | usado com 得 ou 不 para indicar possibilidade, capacidade ou hábito}
  \seealsoref{不}{bu4}
  \seealsoref{得}{de5}
\end{EntryWithPhonetic}

\begin{EntryWithPhonetic}{来不及}{lai2bu5ji2}{7,4,3}{⽊、⼀、⼃}[HSK 4]
  \definition{v.}{ser tarde demais; não ter tempo; não ter tempo suficiente (para fazer algo); não ser possível participar ou se atualizar devido a restrições de tempo}
\end{EntryWithPhonetic}

\begin{EntryWithPhonetic}{来到}{lai2 dao4}{7,8}{⽊、⼑}[HSK 1]
  \definition{v.}{chegar; vir}
\end{EntryWithPhonetic}

\begin{EntryWithPhonetic}{来得及}{lai2de5ji2}{7,11,3}{⽊、⼻、⼃}[HSK 4]
  \definition{v.}{ainda ter tempo; ser capaz de fazer isso; ser capaz de fazer algo a tempo; ainda ter tempo de cuidar ou de colocar em dia}
\end{EntryWithPhonetic}

\begin{EntryWithPhonetic}{来往}{lai2 wang3}{7,8}{⽊、⼻}[HSK 6]
  \definition{s.}{negociação; contato com alguém; interações sociais}
  \definition{v.}{ir e vir | ter negócios com alguém}
\end{EntryWithPhonetic}

\begin{EntryWithPhonetic}{来信}{lai2 xin4}{7,9}{⽊、⼈}[HSK 5]
  \definition[封]{s.}{sua carta; carta recebida; carta ao interlocutor}
  \definition{v.}{enviar uma carta para aqui; enviar uma carta para o remetente}
\end{EntryWithPhonetic}

\begin{EntryWithPhonetic}{来源}{lai2yuan2}{7,13}{⽊、⽔}[HSK 4]
  \definition{s.}{origem; causa; fonte; tabula rasa (ou seja, o lugar de onde as coisas vêm)}
  \definition{v.}{originar-se; surgir; ter origem; (algo) originar (seguido de 于)}
  \seealsoref{于}{yu2}
\end{EntryWithPhonetic}

\begin{EntryWithPhonetic}{来自}{lai2zi4}{7,6}{⽊、⾃}[HSK 2]
  \definition{v.}{vir de (um local) | \emph{From:} (cabeçalho de \emph{e -mail})}
\end{EntryWithPhonetic}

\begin{EntryWithPhonetic}{赖}{lai4}{13}{⾙}[HSK 6]
  \definition*{s.}{Sobrenome Lai}
  \definition{adj.}{ruim; pobre; não é bom}
  \definition{v.}{confiar em; depender de | permanecer em um lugar; prolongar a permanência de alguém em um lugar; ficar e recusar-se a sair | negar o próprio erro ou responsabilidade; voltar atrás na palavra; repudiar; negar; não admitir culpa; não assumir responsabilidade | colocar a culpa nos outros; incriminar falsamente (acusar); acusar alguém de algo errado; acusar alguém falsamente | culpar}
\end{EntryWithPhonetic}

\begin{EntryWithPhonetic}{兰}{lan2}{5}{⼋}
  \definition*{s.}{Sobrenome Lan}
  \definition{s.}{orquídea | lírio magnólia}
\end{EntryWithPhonetic}

\begin{EntryWithPhonetic}{兰花}{lan2hua1}{5,7}{⼋、⾋}
  \definition{s.}{orquídea}
\end{EntryWithPhonetic}

\begin{EntryWithPhonetic}{兰州}{lan2zhou1}{5,6}{⼋、⼮}
  \definition*{s.}{Lanzhou. capital da província de Gansu, 甘肃}
  \seealsoref{甘肃}{gan1su4}
\end{EntryWithPhonetic}

\begin{EntryWithPhonetic}{栏}{lan2}{9}{⽊}
  \definition{s.}{cerca; corrimão; balaustrada | curral; galpão; celeiro; chiqueiro | coluna (de uma página ou tabela, ou de um jornal) | quadro (de avisos); prancha; tabuleiro | Esporte: obstáculo}
\end{EntryWithPhonetic}

\begin{EntryWithPhonetic}{栏目}{lan2mu4}{9,5}{⽊、⽬}[HSK 6]
  \definition[个,档]{s.}{coluna; programa; seções nomeadas de jornais, revistas, etc. divididas de acordo com a natureza de seu conteúdo}
\end{EntryWithPhonetic}

\begin{EntryWithPhonetic}{蓝}{lan2}{13}{⾋}[HSK 2]
  \definition*{s.}{Sobrenome Lan}
  \definition{adj.}{azul}
  \definition{s.}{planta índigo; anil | plantas azuis; refere-se a certas plantas que podem ser usadas como corante azul ou certas plantas cujas folhas são azul-esverdeadas}
\end{EntryWithPhonetic}

\begin{EntryWithPhonetic}{蓝领}{lan2 ling3}{13,11}{⾋、⾴}[HSK 6]
  \definition[名,位,个]{s.}{trabalhador braçal}
\end{EntryWithPhonetic}

\begin{EntryWithPhonetic}{蓝色}{lan2 se4}{13,6}{⾋、⾊}[HSK 2]
  \definition[抹,片,缕,团,块]{s.}{cor azul}
\end{EntryWithPhonetic}

\begin{EntryWithPhonetic}{篮}{lan2}{16}{⽵}
  \definition[个]{s.}{cesto | o anel de ferro e a rede na cesta de basquete}
\end{EntryWithPhonetic}

\begin{EntryWithPhonetic}{篮球}{lan2qiu2}{16,11}{⽵、⽟}[HSK 2]
  \definition[个,只]{s.}{basquetebol | bola de basquete; refere-se à bola utilizada no basquetebol}
\end{EntryWithPhonetic}

\begin{EntryWithPhonetic}{懒}{lan3}{16}{⼼}[HSK 6]
  \definition{adj.}{indolente; preguiçoso (oposto de 勤) | lento; lânguido | ocioso; preguiçoso}
  \seealsoref{勤}{qin2}
\end{EntryWithPhonetic}

\begin{EntryWithPhonetic}{懒虫}{lan3chong2}{16,6}{⼼、⾍}
  \definition{s.}{desleixado ocioso | (insulto) sujeito preguiçoso}
\end{EntryWithPhonetic}

\begin{EntryWithPhonetic}{懒怠}{lan3dai4}{16,9}{⼼、⼼}
  \definition{s.}{preguiça}
\end{EntryWithPhonetic}

\begin{EntryWithPhonetic}{懒得}{lan3de5}{16,11}{⼼、⼻}
  \definition{adv.}{demasiado preguiçoso}
  \definition{v.}{não sentir vontade (de fazer algo)}
\end{EntryWithPhonetic}

\begin{EntryWithPhonetic}{懒惰}{lan3duo4}{16,12}{⼼、⼼}
  \definition{adj.}{preguiçoso}
\end{EntryWithPhonetic}

\begin{EntryWithPhonetic}{懒鬼}{lan3gui3}{16,9}{⼼、⿁}
  \definition{s.}{cara preguiçoso}
\end{EntryWithPhonetic}

\begin{EntryWithPhonetic}{懒汉}{lan3han4}{16,5}{⼼、⽔}
  \definition{s.}{sujeito ocioso | vagabundo | preguiçosos}
\end{EntryWithPhonetic}

\begin{EntryWithPhonetic}{懒人}{lan3ren2}{16,2}{⼼、⼈}
  \definition{s.}{pessoa preguiçosa}
\end{EntryWithPhonetic}

\begin{EntryWithPhonetic}{懒散}{lan3san3}{16,12}{⼼、⽁}
  \definition{adj.}{inativo | indolente | preguiçoso | negligente}
\end{EntryWithPhonetic}

\begin{EntryWithPhonetic}{懒腰}{lan3yao1}{16,13}{⼼、⾁}
  \definition[个]{s.}{alongamento (do corpo)}
\end{EntryWithPhonetic}

\begin{EntryWithPhonetic}{烂}{lan4}{9}{⽕}[HSK 5]
  \definition{adj.}{macio; pastoso; amassado | podre; deteriorado | quebrado; esfarrapado; gasto | desorganizado; indigno}
  \definition{adv.}{totalmente; extremamente; completamente; expressa um grau muito profundo}
  \definition{v.}{apodrecer; infeccionar; decompor-se}
\end{EntryWithPhonetic}

\begin{EntryWithPhonetic}{廊}{lang2}{11}{⼴}
  \definition[个]{s.}{varanda; corredor}
\end{EntryWithPhonetic}

\begin{EntryWithPhonetic}{廊坊}{lang2fang2}{11,7}{⼴、⼟}
  \definition*{s.}{Cidade de Langfang em Hebei}
\end{EntryWithPhonetic}

\begin{EntryWithPhonetic}{朗}{lang3}{10}{⽉}
  \definition*{s.}{Sobrenome Lang}
  \definition{adj.}{claro; brilhante | alto e claro (som)}
\end{EntryWithPhonetic}

\begin{EntryWithPhonetic}{朗读}{lang3du2}{10,10}{⽉、⾔}[HSK 5]
  \definition{v.}{ler em voz alta; recitar com voz clara e alta}
\end{EntryWithPhonetic}

\begin{EntryWithPhonetic}{浪}{lang4}{10}{⽔}
  \definition*{s.}{Sobrenome Lang}
  \definition{adj.}{desenfreado; perdulário}
  \definition{adv.}{livremente}
  \definition[朵,阵,波]{s.}{onda; vagalhão; rebentação | algo ondulatório | coisas ondulando como ondas}
  \definition{v.}{passear; divagar}
\end{EntryWithPhonetic}

\begin{EntryWithPhonetic}{浪费}{lang4fei4}{10,9}{⽔、⾙}[HSK 3]
  \definition{adj.}{desperdiçado; extravagante; não econômico}
  \definition{v.}{desperdiçar; dissipar; esbanjar; ser extravagante; uso excessivo ou inadequado de bens, recursos humanos, tempo, etc.}
\end{EntryWithPhonetic}

\begin{EntryWithPhonetic}{浪花}{lang4hua1}{10,7}{⽔、⾋}
  \definition[朵]{s.}{\emph{spray} | \emph{spray} do oceano | (figurativo) acontecimentos de sua vida}
\end{EntryWithPhonetic}

\begin{EntryWithPhonetic}{浪漫}{lang4man4}{10,14}{⽔、⽔}[HSK 5]
  \definition{adj.}{romântico; poético | não convencional; boêmio; abandonado; libertino; devasso; comportar-se de maneira descuidada e descuidada (geralmente se referindo a relacionamentos entre pessoas) | irrealista; impraticável}
\end{EntryWithPhonetic}

\begin{EntryWithPhonetic}{捞}{lao1}{10}{⼿}
  \definition{v.}{pescar | dragar}
\end{EntryWithPhonetic}

\begin{EntryWithPhonetic}{劳}{lao2}{7}{⼒}
  \definition*{s.}{Sobrenome Lao}
  \definition{adj.}{difícil; cansativo; cansado}
  \definition{s.}{fadiga; trabalho árduo | ação meritória; serviço; conquistas | trabalhador | mérito | trabalhador braçal}
  \definition{v.}{trabalho; labor | esforço; exercício intenso | (pedir um favor a alguém, também 有劳) colocar alguém no trabalho de | expressar apreço (ao executor de uma tarefa); recompensar | colocar alguém no trabalho de; incomodar alguém com algo | trazer presentes para}
  \seealsoref{有劳}{you3lao2}
\end{EntryWithPhonetic}

\begin{EntryWithPhonetic}{劳动}{lao2dong4}{7,6}{⼒、⼒}[HSK 5]
  \definition[次]{s.}{trabalho; mão de obra; atividades intelectuais ou físicas que podem criar valor | trabalho físico; trabalho manual; referindo-se especificamente ao trabalho físico}
  \definition{v.}{realizar trabalho físico}
\end{EntryWithPhonetic}

\begin{EntryWithPhonetic}{劳工同事}{lao2gong1 tong2shi4}{7,3,6,8}{⼒、⼯、⼝、⼅}
  \definition{s.}{colaborador | colega de trabalho}
\end{EntryWithPhonetic}

\begin{EntryWithPhonetic}{牢}{lao2}{7}{⼧}[HSK 6]
  \definition*{s.}{Sobrenome Lao}
  \definition{adj.}{firme; durável}
  \definition{s.}{prisão; cadeia | (cercado para animais) curral; baia; galinheiro; estábulo; estrebaria; cocheira | (arcaico) animal de sacrifício}
\end{EntryWithPhonetic}

\begin{EntryWithPhonetic}{老}{lao3}{6}{⽼}[HSK 1,2][Kangxi 125]
  \definition*{s.}{Sobrenome Lao}
  \definition{adj.}{velho; envelhecido; idade avançada | antigo; de longa data; existe há muito tempo | antigo; desatualizado; obsoleto; ultrapassado  | antigo; tradicional; original | coberto de vegetação; os vegetais cresceram além do período ideal para serem consumidos | resistente; endurecido; alimentos muito cozidos | escuro; profundo; (sobre cores) | último nascido; o mais novo | veterano; experiente; sofisticado}
  \definition{adv.}{longo; por muito tempo | sempre (fazendo algo) | muito}
  \definition{pref.}{usado para designar pessoas, ordem de classificação, certos nomes de animais e plantas}
  \definition{s.}{idosos; pessoas mais velhas | ancião; sênior; um título respeitoso para pessoas mais velhas}
  \definition{v.}{envelhecer | morrer; referindo-se à morte de um idoso}
\end{EntryWithPhonetic}

\begin{EntryWithPhonetic}{老百姓}{lao3bai3xing4}{6,6,8}{⽼、⽩、⼥}[HSK 3]
  \definition[些]{s.}{povo; civis; pessoas comuns; residentes (em contraste com militares e funcionários públicos)}
\end{EntryWithPhonetic}

\begin{EntryWithPhonetic}{老板}{lao3ban3}{6,8}{⽼、⽊}[HSK 3]
  \definition[个,位]{s.}{chefe; dono; líder; refere-se ao gerente de uma empresa comercial ou industrial | antigo título honorífico dado a atores famosos de ópera ou atores que também eram diretores de companhias de ópera}
\end{EntryWithPhonetic}

\begin{EntryWithPhonetic}{老兵}{lao3bing1}{6,7}{⽼、⼋}
  \definition{s.}{velho soldado | veterano de guerra | veterano (alguém que tem muita experiência em algum domínio)}
\end{EntryWithPhonetic}

\begin{EntryWithPhonetic}{老公}{lao3 gong1}{6,4}{⽼、⼋}[HSK 4]
  \definition[个,位,名]{s.}{marido; esposo}
\end{EntryWithPhonetic}

\begin{EntryWithPhonetic}{老虎}{lao3hu3}{6,8}{⽼、⾌}
  \definition[只]{s.}{tigre}
  \seealsoref{虎}{hu3}
\end{EntryWithPhonetic}

\begin{EntryWithPhonetic}{老家}{lao3 jia1}{6,10}{⽼、⼧}[HSK 4]
  \definition{s.}{cidade natal; local de origem | lugar nativo; refere-se às gerações anteriores da família ou ao local onde a pessoa nasceu ou viveu}
\end{EntryWithPhonetic}

\begin{EntryWithPhonetic}{老年}{lao3 nian2}{6,6}{⽼、⼲}[HSK 2]
  \definition[个]{s.}{idoso; velhice; idade acima de 60 ou 70 anos}
\end{EntryWithPhonetic}

\begin{EntryWithPhonetic}{老朋友}{lao3 peng2 you3}{6,8,4}{⽼、⽉、⼜}[HSK 2]
  \definition[个,位,名]{s.}{velho amigo; refere-se a amigos que conhecemos há muito tempo e com quem temos uma relação íntima}
\end{EntryWithPhonetic}

\begin{EntryWithPhonetic}{老婆}{lao3po2}{6,11}{⽼、⼥}[HSK 4]
  \definition[个,位,名]{s.}{esposa}
\end{EntryWithPhonetic}

\begin{EntryWithPhonetic}{老人}{lao3 ren2}{6,2}{⽼、⼈}[HSK 1]
  \definition[位]{s.}{homem ou mulher de idade avançada; o idoso; o velho}
\end{EntryWithPhonetic}

\begin{EntryWithPhonetic}{老人家}{lao3 ren2 jia1}{6,2,10}{⽼、⼈、⼧}
  \definition[位,名,个]{s.}{avô; avó; pessoa idosa venerável; um título respeitoso para os idosos | maneira de chamar o pai ou a mãe idosos na frente dos outros; referir-se aos próprios pais ou aos pais, professores, etc. de outras pessoas}
\end{EntryWithPhonetic}

\begin{EntryWithPhonetic}{老师}{lao3shi1}{6,6}{⽼、⼱}[HSK 1]
  \definition[个,位]{s.}{professor; título honorífico para professores; refere-se, de maneira geral, a pessoas que transmitem cultura e tecnologia ou que são dignas de admiração em termos de ideias, moralidade e conhecimentos profissionais}
\end{EntryWithPhonetic}

\begin{EntryWithPhonetic}{老是}{lao3 shi4}{6,9}{⽼、⽇}[HSK 2]
  \definition{adv.}{sempre; indica que a ação continua ou que o estado permanece inalterado, equivalente a 一直}
  \seealsoref{一直}{yi4zhi2}
\end{EntryWithPhonetic}

\begin{EntryWithPhonetic}{老实}{lao3shi5}{6,8}{⽼、⼧}[HSK 4]
  \definition{adj.}{franco; sincero; honesto | bom; bem-comportado | ingênuo; simplório; meio bobo; facilmente enganado; eufemismo para pouco inteligente}
\end{EntryWithPhonetic}

\begin{EntryWithPhonetic}{老太太}{lao3 tai4 tai5}{6,4,4}{⽼、⼤、⼤}[HSK 3]
  \definition[位,名,个]{s.}{velha senhora; (em tratamento direto)Venerável Senhora; uma maneira respeitosa de chamar uma senhora idosa; título honorífico para mulheres idosas | (forma de tratamento) sua velha mãe; minha velha mãe, avó ou sogra; referindo-se à própria mãe, à mãe do outro ou à mãe de outra pessoa, à sogra ou à sogra política}
\end{EntryWithPhonetic}

\begin{EntryWithPhonetic}{老头儿}{lao3 tou2r5}{6,5,2}{⽼、⼤、⼉}[HSK 3]
  \definition{s.}{(coloquial) (com um tom de intimidade) velho; velho amigo}
  \seealsoref{老头子}{lao3 tou2zi5}
\end{EntryWithPhonetic}

\begin{EntryWithPhonetic}{老头子}{lao3 tou2zi5}{6,5,3}{⽼、⼤、⼦}
  \definition{s.}{velho antiquado (ou velho rabugento) | (referindo-se ao marido idoso) meu velho | chefe de uma sociedade secreta | (coloquial) velho; velho rabugento}
  \seealsoref{老头儿}{lao3 tou2r5}
\end{EntryWithPhonetic}

\begin{EntryWithPhonetic}{老乡}{lao3 xiang1}{6,3}{⽼、⼄}[HSK 6]
  \definition[个,位]{s.}{conterrâneo; conterrâneo | uma maneira de chamar um fazendeiro cujo nome você não conhece}
\end{EntryWithPhonetic}

\begin{EntryWithPhonetic}{落}{lao4}{12}{⾋}
  \definition{v.}{cair; cair de uma altura elevada | se abaixar; descer; ir para baixo | permanecer; fazer uma parada; deixar para trás | obter; ter; receber}
  \seeref{la4}
  \seeref{luo4}
\end{EntryWithPhonetic}

\begin{EntryWithPhonetic}{乐}{le4}{5}{⼃}[HSK 3]
  \definition*{s.}{Sobrenome Le}
  \definition{adj.}{feliz; contente; rejubilante; animado; bem disposto}
  \definition{s.}{prazer; diversão; felicidade}
  \definition{v.}{desfrutar; ficar feliz em; amar; encontrar prazer em | rir; divertir-se}
  \seeref{yue4}
\end{EntryWithPhonetic}

\begin{EntryWithPhonetic}{乐高}{le4gao1}{5,10}{⼃、⾼}
  \definition*{s.}{Lego (brinquedo)}
\end{EntryWithPhonetic}

\begin{EntryWithPhonetic}{乐观}{le4guan1}{5,6}{⼃、⾒}[HSK 3]
  \definition{adj.}{esperançoso; otimista; confiante; espírito alegre, confiante no futuro (oposto a 悲观)}
  \seealsoref{悲观}{bei1guan1}
\end{EntryWithPhonetic}

\begin{EntryWithPhonetic}{乐趣}{le4qu4}{5,15}{⼃、⾛}[HSK 4]
  \definition[个,种,些,点]{s.}{alegria; deleite; prazer; implicação de fazer alguém se sentir feliz; um humor de preferência}
\end{EntryWithPhonetic}

\begin{EntryWithPhonetic}{乐园}{le4yuan2}{5,7}{⼃、⼞}
  \definition{s.}{paraíso}
\end{EntryWithPhonetic}

\begin{EntryWithPhonetic}{了}{le5}{2}{⼅}[HSK 1,3]
  \definition{part.}{usada após verbos ou adjetivos para indicar a conclusão de uma ação, em um momento no passado ou antes do início de outra ação, ou uma ação esperada ou presumida | usada para indicar uma mudança de situação ou estado, seja real ou prevista | comandos ou solicitações em resposta a uma situação alterada; usada para xpressar urgência ou dissuadir | usada para indicar que algo chegou ao extremo; usada no final da frase ou em pausas no meio da frase, para expressar um tom de exclamação}
  \seeref{liao3}
\end{EntryWithPhonetic}

\begin{EntryWithPhonetic}{累}{lei2}{11}{⽷}
  \definition*{s.}{Sobrenome Lei}
  \definition{adj.}{incômodo; complicado}
  \definition{s.}{corda; cordão | touro na época de acasalamento}
  \definition{v.}{amarrar; prender; atar | copular}
  \seeref{lei3}
  \seeref{lei4}
\end{EntryWithPhonetic}

\begin{EntryWithPhonetic}{雷}{lei2}{13}{⾬}
  \definition*{s.}{Sobrenome Lei}
  \definition[声,个,颗]{s.}{trovão | (militar) mina}
\end{EntryWithPhonetic}

\begin{EntryWithPhonetic}{雷电}{lei2dian4}{13,5}{⾬、⽥}
  \definition{s.}{trovão e relâmpago; raio}
\end{EntryWithPhonetic}

\begin{EntryWithPhonetic}{雷亚尔}{lei2ya4'er3}{13,6,5}{⾬、⼆、⼩}
  \definition*{s.}{Real Brasileiro}
\end{EntryWithPhonetic}

\begin{EntryWithPhonetic}{累}{lei3}{11}{⽷}
  \definition*{s.}{Sobrenome Lei}
  \definition{adj.}{em andamento; repetido; contínuo}
  \definition{v.}{acumular; empilhar; colocar em cima de outro | envolver; implicar | construir empilhando tijolos, pedras, terra, etc.}
  \seeref{lei2}
  \seeref{lei4}
\end{EntryWithPhonetic}

\begin{EntryWithPhonetic}{絫}{lei3}{12}{⽷}
  \variantof{累}
\end{EntryWithPhonetic}

\begin{EntryWithPhonetic}{泪}{lei4}{8}{⽔}[HSK 4]
  \definition[滴,行]{s.}{lágrima | algo semelhante a uma lágrima}
\end{EntryWithPhonetic}

\begin{EntryWithPhonetic}{泪水}{lei4 shui3}{8,4}{⽔、⽔}[HSK 4]
  \definition[滴,行]{s.}{lágrima}
\end{EntryWithPhonetic}

\begin{EntryWithPhonetic}{类}{lei4}{9}{⽶}[HSK 3]
  \definition*{s.}{Sobrenome Lei}
  \definition{clas.}{tipo; espécie; categoria usada para pessoas ou coisas}
  \definition{s.}{classe; categoria; tipo; variedade; a combinação de muitas coisas semelhantes ou iguais}
  \definition{v.}{assemelhar-se a; ser semelhante a}
\end{EntryWithPhonetic}

\begin{EntryWithPhonetic}{类似}{lei4si4}{9,6}{⽶、⼈}[HSK 3]
  \definition{adj.}{semelhante; análogo}
\end{EntryWithPhonetic}

\begin{EntryWithPhonetic}{类型}{lei4xing2}{9,9}{⽶、⼟}[HSK 4]
  \definition[种,个]{s.}{tipo; espécie; categoria; tipos formados por coisas com características comuns}
\end{EntryWithPhonetic}

\begin{EntryWithPhonetic}{累}{lei4}{11}{⽷}[HSK 1]
  \definition{adj.}{cansado; exausto; fatigado}
  \definition{v.}{cansar; desgastar; fatigar; esgotar | labutar; trabalhar duro}
  \seeref{lei2}
  \seeref{lei3}
\end{EntryWithPhonetic}

\begin{EntryWithPhonetic}{冷}{leng3}{7}{⼎}[HSK 1]
  \definition*{s.}{Sobrenome Leng}
  \definition{adj.}{frio; baixa temperatura; sensação de frio | gelado; frio por natureza; sem entusiasmo; sem gentileza | desolado; pouco frequentado; quieto; sem agitação | negligenciado; indesejável; ignorado por todos | raro; estranho; incomum | feito em segredo; filmado de forma escondida; lançado secretamente}
  \definition{v.}{esfriar; resfriar | esfriar; congelar; tornar-se indiferente, apático | ignorar}
\end{EntryWithPhonetic}

\begin{EntryWithPhonetic}{冷静}{leng3jing4}{7,14}{⼎、⾭}[HSK 4]
  \definition{adj.}{calmo; descreve uma pessoa que consegue ficar atenta em uma situação importante ou de emergência e não toma decisões aleatórias por causa de seus sentimentos no momento | (lugar) tranquilo; quieto; deserto}
\end{EntryWithPhonetic}

\begin{EntryWithPhonetic}{冷门}{leng3men2}{7,3}{⼎、⾨}
  \definition{s.}{uma profissão, ofício ou ramo de aprendizagem que recebe pouca atenção | um vencedor inesperado; azarão}
\end{EntryWithPhonetic}

\begin{EntryWithPhonetic}{冷气}{leng3 qi4}{7,4}{⼎、⽓}[HSK 6]
  \definition[股,阵]{s.}{ar frio (ou fresco); correntes de ar frio | ar condicionado; ar resfriado por equipamento de refrigeração | ar condicionado; equipamentos de ar condicionado}
\end{EntryWithPhonetic}

\begin{EntryWithPhonetic}{冷水}{leng3 shui3}{7,4}{⼎、⽔}[HSK 6]
  \definition[杯,瓶]{s.}{água fria | água não fervida}
\end{EntryWithPhonetic}

\begin{EntryWithPhonetic}{哩哩啦啦}{li1 li1 la1 la1}{10,10,11,11}{⼝、⼝、⼝、⼝}
  \definition{adj.}{espalhado; disperso; disseminado; difuso; esporádico; aqui e ali}
\end{EntryWithPhonetic}

\begin{EntryWithPhonetic}{厘}{li2}{9}{⼚}
  \definition*{s.}{Sobrenome Li}
  \definition{clas.}{li, uma unidade tradicional de comprimento, igual a 0,001 chi (市尺), e equivalente a 0,333 milímetro ou 0,013 polegada | li, uma unidade tradicional de peso, igual a 0,0001 jin (市斤), e equivalente a 5 centigramas ou 0,771 grãos | li, uma unidade tradicional de área, igual a 0,01 mu (市亩), e equivalente a 0,667 metro quadrado ou 0,797 jarda quadrada | li, unidade monetária chinesa, igual a 0,1 fen ou 0,001 yuan | li, unidade de taxa de juros, igual a 0,1\% de juros mensais ou 1\% de juros anuais | quantidade muito pequena; fração; o mínimo}
  \definition{v.}{regular; retificar | administrar}
  \seealsoref{市尺}{shi4 chi3}
  \seealsoref{市斤}{shi4jin1}
  \seealsoref{市亩}{shi4mu3}
\end{EntryWithPhonetic}

\begin{EntryWithPhonetic}{厘米}{li2mi3}{9,6}{⼚、⽶}[HSK 4]
  \definition{clas.}{centímetro; unidade de comprimento, símbolo cm, 1 metro é igual a 100 centímetros}
\end{EntryWithPhonetic}

\begin{EntryWithPhonetic}{离}{li2}{10}{⼇}[HSK 2]
  \definition*{s.}{Um dos Oito Diagramas | Sobrenome Li}
  \definition{prep.}{(ser longe) de\dots até\dots}
  \definition{v.}{partir; separar-se; afastar-se; estar longe de | prescindir; dispensar; ser independente de | mudar de; desviar-se de | mudar de; desviar-se de; trair; ser incompatível}
\end{EntryWithPhonetic}

\begin{EntryWithPhonetic}{离不开}{li2 bu4 kai1}{10,4,4}{⼇、⼀、⼶}[HSK 4]
  \definition{v.}{não pode prescindir; ser inseparável de; não ser capaz de se separar ou deixar uma pessoa, coisa ou circunstância}
\end{EntryWithPhonetic}

\begin{EntryWithPhonetic}{离婚}{li2/hun1}{10,11}{⼇、⼥}[HSK 3]
  \definition{v.+compl.}{divórciar; romper um casamento; obter o divórcio}
\end{EntryWithPhonetic}

\begin{EntryWithPhonetic}{离开}{li2kai1}{10,4}{⼇、⼶}[HSK 2]
  \definition{v.}{deixar; partir; desviar-se; separar-se das pessoas, dos lugares e das coisas}
\end{EntryWithPhonetic}

\begin{EntryWithPhonetic}{梨}{li2}{11}{⽊}[HSK 5]
  \definition*{s.}{Sobrenome Li}
  \definition[个,只,斤,棵,种]{s.}{perira; árvore de pera | pera}
\end{EntryWithPhonetic}

\begin{EntryWithPhonetic}{黎}{li2}{15}{⿉}
  \definition*{s.}{Etnia Li, uma das minorias nacionais da província de Hainan | Sobrenome Li}
  \definition{adj.}{Literário: preto; escuro | Literário: numeroso}
  \definition{s.}{multidão; as massas; a população}
\end{EntryWithPhonetic}

\begin{EntryWithPhonetic}{礼}{li3}{5}{⽰}[HSK 5]
  \definition*{s.}{Sobrenome Li}
  \definition[份]{s.}{observâncias cerimoniais em geral; cerimônia; rito | cortesia; etiqueta; boas maneiras | presente; oferta}
\end{EntryWithPhonetic}

\begin{EntryWithPhonetic}{礼拜}{li3 bai4}{5,9}{⽰、⼿}[HSK 5]
  \definition[个]{s.}{dia da semana; usado em conjunto com 一, 二, 三, 四, 五, 六, 日(或天, indica um dia específico da semana | semana; referência à semana | domingo}
  \definition{v.}{prestar homenagem aos deuses que veneram; rezar; orar}
\end{EntryWithPhonetic}

\begin{EntryWithPhonetic}{礼节}{li3jie2}{5,5}{⽰、⾋}
  \definition{s.}{protocolo | cerimônia | etiqueta}
\end{EntryWithPhonetic}

\begin{EntryWithPhonetic}{礼貌}{li3mao4}{5,14}{⽰、⾘}[HSK 5]
  \definition{adj.}{educado; descreve uma pessoa que fala e age respeitando os outros, sem arrogância, de acordo com as exigências das relações sociais}
  \definition{s.}{cortesia; educação; boas maneiras}
\end{EntryWithPhonetic}

\begin{EntryWithPhonetic}{礼让}{li3rang4}{5,5}{⽰、⾔}
  \definition{s.}{cortesia}
  \definition{v.}{mostrar consideração por (outros) | ceder a (outro veículo, etc.)}
\end{EntryWithPhonetic}

\begin{EntryWithPhonetic}{礼堂}{li3 tang2}{5,11}{⽰、⼟}[HSK 6]
  \definition[个,座,处]{s.}{auditórios; salão de assembleias; um salão para reuniões ou cerimônias}
\end{EntryWithPhonetic}

\begin{EntryWithPhonetic}{礼物}{li3wu4}{5,8}{⽰、⽜}[HSK 2]
  \definition[份,件,个,分,些]{s.}{presente; lembrança; itens oferecidos como forma de respeito ou celebração, referindo-se de maneira geral a itens oferecidos como presente}
\end{EntryWithPhonetic}

\begin{EntryWithPhonetic}{李}{li3}{7}{⽊}
  \definition*{s.}{Sobrenome Li}
  \definition[棵]{s.}{ameixa | ameixeira}
\end{EntryWithPhonetic}

\begin{EntryWithPhonetic}{李四}{li3si4}{7,5}{⽊、⼞}
  \definition*{s.}{Li Si | Zé Ninguém | Nome para uma pessoa não especificada, 2 de 3}
  \seealsoref{王五}{wang2wu3}
  \seealsoref{张三}{zhang1san1}
\end{EntryWithPhonetic}

\begin{EntryWithPhonetic}{李子}{li3zi5}{7,3}{⽊、⼦}
  \definition[个]{s.}{ameixa}
\end{EntryWithPhonetic}

\begin{EntryWithPhonetic}{里}{li3}{7}{⾥}[HSK 1][Kangxi 166]
  \definition*{s.}{Sobrenome Li}
  \definition{clas.}{li, uma unidade chinesa de comprimento (= 1/2 quilômetro)}
  \definition{s.}{forro; revestimento; interior; parte de trás do tecido | interno; dentro; no interior | vizinhança; vizinhos | cidade natal; local de origem}
\end{EntryWithPhonetic}

\begin{EntryWithPhonetic}{里边}{li3 bian5}{7,5}{⾥、⾡}[HSK 1]
  \definition{s.}{em; dentro; no interior}
\end{EntryWithPhonetic}

\begin{EntryWithPhonetic}{里面}{li3 mian4}{7,9}{⾥、⾯}[HSK 3]
  \definition{s.}{dentro; interior}
\end{EntryWithPhonetic}

\begin{EntryWithPhonetic}{里斯本}{li3si1ben3}{7,12,5}{⾥、⽄、⽊}
  \definition*{s.}{Lisboa}
\end{EntryWithPhonetic}

\begin{EntryWithPhonetic}{里斯本大学}{li3si1ben3 da4xue2}{7,12,5,3,8}{⾥、⽄、⽊、⼤、⼦}
  \definition*{s.}{Universidade de Lisboa}
\end{EntryWithPhonetic}

\begin{EntryWithPhonetic}{里头}{li3 tou5}{7,5}{⾥、⼤}[HSK 2]
  \definition{s.}{dentro}
\end{EntryWithPhonetic}

\begin{EntryWithPhonetic}{哩}{li3}{10}{⼝}
  \definition{clas.}{milha (unidade de comprimento igual a 1.609,344 m)}
  \seeref{li5}
\end{EntryWithPhonetic}

\begin{EntryWithPhonetic}{理}{li3}{11}{⽟}[HSK 6]
  \definition*{s.}{Sobrenome Li}
  \definition{s.}{textura; grão (em madeira, pele, etc.) | ordem; sequência | razão; lógica; verdade | ciências naturais (especialmente física)}
  \definition{v.}{gerenciar; executar | colocar em ordem; arrumar | (geralmente no negativo) prestar atenção a; fazer um gesto ou falar com | tratar | colocar em ordem; limpar | tomar conhecimento de; prestar atenção a; expressar uma atitude; expressar uma opinião}
\end{EntryWithPhonetic}

\begin{EntryWithPhonetic}{理财}{li3 cai2}{11,7}{⽟、⾙}[HSK 6]
  \definition{v.}{administrar questões financeiras; conduzir transações financeiras; administrar propriedade; ser responsável pelo trabalho financeiro}
\end{EntryWithPhonetic}

\begin{EntryWithPhonetic}{理发}{li3/fa4}{11,5}{⽟、⼜}[HSK 3]
  \definition{v.+compl.}{cortar e aparar o cabelo; ter (dar) um corte de cabelo}
\end{EntryWithPhonetic}

\begin{EntryWithPhonetic}{理解}{li3jie3}{11,13}{⽟、⾓}[HSK 3]
  \definition{v.}{entender; compreender; compreender o significado por trás de algo através da reflexão e do aprendizado | entender com empatia; achar que os outros não conseguem fazer determinada coisa e demonstrar compaixão, perdão e não crítica}
\end{EntryWithPhonetic}

\begin{EntryWithPhonetic}{理论}{li3lun4}{11,6}{⽟、⾔}[HSK 3]
  \definition[套,个]{s.}{teoria; uma série de conclusões tiradas pelas pessoas sobre atividades naturais ou sociais}
  \definition{v.}{argumentar; raciocinar com alguém; discutir com outras pessoas sobre quem está certo ou errado}
\end{EntryWithPhonetic}

\begin{EntryWithPhonetic}{理想}{li3xiang3}{11,13}{⽟、⼼}[HSK 2]
  \definition{adj.}{ideal; perfeito | conforme o desejado; satisfatório}
  \definition{adv.}{idealmente}
  \definition[个,种]{s.}{ideal; sonho; aspiração}
\end{EntryWithPhonetic}

\begin{EntryWithPhonetic}{理由}{li3you2}{11,5}{⽟、⽥}[HSK 3]
  \definition[个,条,种,堆]{s.}{razão; justificativa; fundamento; a razão pela qual as coisas são feitas desta ou daquela maneira}
\end{EntryWithPhonetic}

\begin{EntryWithPhonetic}{理智}{li3zhi4}{11,12}{⽟、⽇}[HSK 6]
  \definition{adj.}{racional; sensato; cabeça fria; sóbrio; calmo}
  \definition{s.}{sentido; razão; intelecto; a capacidade de distinguir o certo do errado, analisar e julgar e controlar as emoções e o comportamento de acordo}
\end{EntryWithPhonetic}

\begin{EntryWithPhonetic}{力}{li4}{2}{⼒}[HSK 3][Kangxi 19]
  \definition*{s.}{Sobrenome Li}
  \definition{adj.}{forte; eficiente; capaz | forte; poderoso; referência geral à função das coisas}
  \definition{adv.}{energicamente; arduamente; vigorosamente; com todo o esforço; com toda a dedicação}
  \definition{s.}{força; energia; poder; (física) refere-se à ação de alterar o estado de movimento ou a forma de um objeto |poder; força; habilidade; capacidade; funções dos órgãos do corpo humano | força física; resistência física}
\end{EntryWithPhonetic}

\begin{EntryWithPhonetic}{力量}{li4liang5}{2,12}{⼒、⾥}[HSK 3]
  \definition[出]{s.}{força física; força espiritual | habilidade; capacidade | eficácia; efeito | força (pessoa ou grupo que tem muito poder ou influência); referência a uma pessoa ou grupo que pode desempenhar um papel importante}
\end{EntryWithPhonetic}

\begin{EntryWithPhonetic}{力气}{li4qi5}{2,4}{⼒、⽓}[HSK 4]
  \definition[把]{s.}{força física; eficiência muscular; força | esforço}
\end{EntryWithPhonetic}

\begin{EntryWithPhonetic}{历}{li4}{4}{⼚}
  \definition{adj.}{todas as anteriores (ocasiões, sessões, etc.)}
  \definition{adv.}{por toda parte; um por um}
  \definition{s.}{experiência; registro | almanaque; anuário; calendário}
  \definition{v.}{passar por; sofrer; experimentar | passar através; atravessar}
\end{EntryWithPhonetic}

\begin{EntryWithPhonetic}{历史}{li4shi3}{4,5}{⼚、⼝}[HSK 4]
  \definition[段]{s.}{história; registro do passado; processo de desenvolvimento da natureza e da sociedade humana; processo de desenvolvimento de uma coisa ou pessoa | história; eventos passados; experiência | história; refere-se ao tema da história}
\end{EntryWithPhonetic}

\begin{EntryWithPhonetic}{厉}{li4}{5}{⼚}
  \definition*{s.}{Sobrenome Li}
  \definition{adj.}{rigoroso; estrito | severo; sombrio; sério}
\end{EntryWithPhonetic}

\begin{EntryWithPhonetic}{厉害}{li4hai5}{5,10}{⼚、⼧}[HSK 5]
  \definition{adj.}{feroz; severo; descreve uma situação como sendo muito grave | severo; duro; descreve uma pessoa que é exigente com os outros, muito severa, muitas vezes deixando os outros um pouco assustados | incrível; talentoso; impressionante; usado para avaliar a capacidade de uma pessoa ou algo que ela fez que é notável | aterrorizante; assustador; descreve animais ferozes e assustadores}
\end{EntryWithPhonetic}

\begin{EntryWithPhonetic}{立}{li4}{5}{⽴}[HSK 5][Kangxi 117]
  \definition{adj.}{ereto; vertical; na vertical}
  \definition{adv.}{imediatamente; instantaneamente}
  \definition{v.}{ficar em pé, com os pés no chão ou apoiados em algum objeto; o objeto deve estar na vertical | erguer; colocar (ou levantar) algo; colocar em pé | encontrar; criar; elaborar; formular; estabelecer | configurar; fundar; estabelecer | viver; existir | ascender ao trono; antigamente, referia-se à ascensão ao trono de um monarca | nomear; designar; antigamente, significava estabelecer uma determinada posição ou status}
\end{EntryWithPhonetic}

\begin{EntryWithPhonetic}{立场}{li4chang3}{5,6}{⽴、⼟}[HSK 5]
  \definition[个]{s.}{posição; postura; a posição e a atitude adotadas ao reconhecer e lidar com os problemas | ponto de vista; refere-se especificamente à atitude de reconhecer e lidar com questões a partir dos interesses de uma determinada classe, ou seja, a posição de classe}
\end{EntryWithPhonetic}

\begin{EntryWithPhonetic}{立法}{li4fa3}{5,8}{⽴、⽔}
  \definition{s.}{legislação}
  \definition{v.}{promulgar leis | legislar}
\end{EntryWithPhonetic}

\begin{EntryWithPhonetic}{立即}{li4ji2}{5,7}{⽴、⼙}[HSK 4]
  \definition{adv.}{prontamente; imediatamente; de imediato}
\end{EntryWithPhonetic}

\begin{EntryWithPhonetic}{立刻}{li4ke4}{5,8}{⽴、⼑}[HSK 3]
  \definition{adv.}{imediatamente; de ​​uma vez; indica que algo acontecerá imediatamente após um determinado momento}
\end{EntryWithPhonetic}

\begin{EntryWithPhonetic}{利}{li4}{7}{⼑}[HSK 6]
  \definition*{s.}{Sobrenome Li}
  \definition{adj.}{afiado; cortante | favorável; conveniente; sem dificuldades; sem ou com poucas dificuldades}
  \definition{s.}{benefício; vantagem | lucro; ganhos; juros}
  \definition{v.}{beneficiar; tornar vantajoso}
\end{EntryWithPhonetic}

\begin{EntryWithPhonetic}{利润}{li4run4}{7,10}{⼑、⽔}[HSK 5]
  \definition[笔,份]{s.}{lucro; o dinheiro ganho com atividades comerciais e industriais}
\end{EntryWithPhonetic}

\begin{EntryWithPhonetic}{利息}{li4xi1}{7,10}{⼑、⼼}[HSK 4]
  \definition{s.}{acréscimo; juros; dinheiro recebido além do valor principal como resultado de depósitos ou empréstimos (diferenciado de 本金)}
  \seealsoref{本金}{ben3 jin1}
\end{EntryWithPhonetic}

\begin{EntryWithPhonetic}{利益}{li4yi4}{7,10}{⼑、⽫}[HSK 4]
  \definition[个,种]{s.}{ganho; lucro; juros; benefício}
\end{EntryWithPhonetic}

\begin{EntryWithPhonetic}{利用}{li4yong4}{7,5}{⼑、⽤}[HSK 3]
  \definition{v.}{usar; utilizar; fazer uso de; fazer com que algo ou alguém funcione bem| explorar; tirar vantagem de; usar meios para fazer com que pessoas ou coisas sirvam aos seus interesses}
\end{EntryWithPhonetic}

\begin{EntryWithPhonetic}{例}{li4}{8}{⼈}
  \definition{adj.}{regular; rotineiro}
  \definition{s.}{exemplo; instância | precedente | caso; instância | regras; estatutos; regulamentos}
  \definition{v.}{analogizar}
\end{EntryWithPhonetic}

\begin{EntryWithPhonetic}{例如}{li4ru2}{8,6}{⼈、⼥}[HSK 2]
  \definition{conj.}{por exemplo; tal como; como por exemplo; colocado antes do exemplo, indica que o exemplo vem a seguir}
\end{EntryWithPhonetic}

\begin{EntryWithPhonetic}{例外}{li4wai4}{8,5}{⼈、⼣}[HSK 5]
  \definition[个,种]{s.}{exceção; situações que não se enquadram nas regras gerais ou nas leis comuns}
  \definition{v.}{ser excepcional; ser uma exceção}
\end{EntryWithPhonetic}

\begin{EntryWithPhonetic}{例子}{li4 zi5}{8,3}{⼈、⼦}[HSK 2]
  \definition[个]{s.}{exemplo; algo usado para ajudar a explicar ou provar uma determinada situação ou afirmação}
\end{EntryWithPhonetic}

\begin{EntryWithPhonetic}{隶}{li4}{8}{⾪}[Kangxi 171]
  \definition*{s.}{Sobrenome Li}
  \definition{s.}{escravo; pessoa em servidão; pessoas escravizadas | Arcaico: corredor de cargo governamental na China feudal | um dos estilos antigos da caligrafia chinesa}
  \definition{v.}{estar subordinado a; estar afiliado a (ou com)}
\end{EntryWithPhonetic}

\begin{EntryWithPhonetic}{荔}{li4}{9}{⾋}
  \definition[颗]{s.}{lichia | (arcaico) uma espécie de grama semelhante à taboa}
\end{EntryWithPhonetic}

\begin{EntryWithPhonetic}{荔枝}{li4zhi1}{9,8}{⾋、⽊}
  \definition{s.}{lichia}
\end{EntryWithPhonetic}

\begin{EntryWithPhonetic}{鬲}{li4}{10}{⿀}[Kangxi 193]
  \definition{s.}{recipiente de cerâmica antigo com três pernas usado para cozinhar, com marcas de cordão na parte externa e pernas ocas}
  \seeref{ge2}
\end{EntryWithPhonetic}

\begin{EntryWithPhonetic}{詈}{li4}{12}{⾔}
  \definition{v.}{xingar; usar linguagem severa}
\end{EntryWithPhonetic}

\begin{EntryWithPhonetic}{詈骂}{li4ma4}{12,9}{⾔、⾺}
  \definition{v.}{xingar | abusar}
\end{EntryWithPhonetic}

\begin{EntryWithPhonetic}{哩}{li5}{10}{⼝}
  \definition{part.}{(dialeto) final modal semelhante a 呢 ou 啦, usado em um tom definido, mas um tanto exagerado}
  \seeref{li3}
  \seealsoref{啦}{la5}
  \seealsoref{呢}{ne5}
\end{EntryWithPhonetic}

\begin{EntryWithPhonetic}{俩}{lia3}{9}{⼈}[HSK 4]
  \definition{num.}{ambos; dois; contração de 两个 | alguns; vários; refere-se a um pequeno número}
\end{EntryWithPhonetic}

\begin{EntryWithPhonetic}{俩钱}{lia3qian2}{9,10}{⼈、⾦}
  \definition{s.}{uma pequena quantia de dinheiro}
\end{EntryWithPhonetic}

\begin{EntryWithPhonetic}{连}{lian2}{7}{⾡}[HSK 3]
  \definition*{s.}{Sobrenome Lian}
  \definition{adv.}{em sucessão; um após o outro; repetidamente}
  \definition{prep.}{incluindo; incluido | até mesmo}
  \definition[个]{s.}{companhia; unidades organizacionais das forças armadas}
  \definition{v.}{ligar; juntar; conectar | envolver-se (em problemas); implicar; incriminar | costurar; coser}
\end{EntryWithPhonetic}

\begin{EntryWithPhonetic}{连接}{lian2 jie1}{7,11}{⾡、⼿}[HSK 5]
  \definition[条]{s.}{conexão}
  \definition{v.}{ligar; unir; relacionar, conectar; anexar}
\end{EntryWithPhonetic}

\begin{EntryWithPhonetic}{连忙}{lian2mang2}{7,6}{⾡、⼼}[HSK 3]
  \definition{adv.}{imediatamente; de imediato; com pressa; apressadamente}
\end{EntryWithPhonetic}

\begin{EntryWithPhonetic}{连锁反应}{lian2suo3fan3ying4}{7,12,4,7}{⾡、⾦、⼜、⼴}
  \definition{s.}{reação em cadeia}
\end{EntryWithPhonetic}

\begin{EntryWithPhonetic}{连续}{lian2xu4}{7,11}{⾡、⽷}[HSK 3]
  \definition{adv.}{continuamente; sucessivamente; em uma fileira; um após o outro}
\end{EntryWithPhonetic}

\begin{EntryWithPhonetic}{连续剧}{lian2 xu4 ju4}{7,11,10}{⾡、⽷、⼑}[HSK 3]
  \definition[部,集]{s.}{série; novela; drama dividido em vários episódios, transmitido continuamente pela rádio ou televisão, com enredo contínuo}
\end{EntryWithPhonetic}

\begin{EntryWithPhonetic}{帘}{lian2}{8}{⼱}
  \definition[块,个]{s.}{bandeira em mastro sobre adega; bandeira como placa de loja | cortina; tela de bambu ou tecido; objetos para cobrir portas e janelas}
\end{EntryWithPhonetic}

\begin{EntryWithPhonetic}{莲}{lian2}{10}{⾋}
  \definition*{s.}{Sobrenome Lian}
  \definition[粒]{s.}{lótus}
\end{EntryWithPhonetic}

\begin{EntryWithPhonetic}{莲花}{lian2hua1}{10,7}{⾋、⾋}
  \definition{s.}{flor de lótus | lírio aquático}
\end{EntryWithPhonetic}

\begin{EntryWithPhonetic}{莲藕}{lian2'ou3}{10,18}{⾋、⾋}
  \definition{s.}{raiz de Lotus}
\end{EntryWithPhonetic}

\begin{EntryWithPhonetic}{联}{lian2}{12}{⽿}
  \definition{s.}{dísticos (antitéticos)}
  \definition{v.}{aliar-se a; unir-se; juntar-se a}
\end{EntryWithPhonetic}

\begin{EntryWithPhonetic}{联合}{lian2he2}{12,6}{⽿、⼝}[HSK 3]
  \definition{adj.}{conjunto; unido; federal; combinado}
  \definition{s.}{aliado; união; aliança; conectar-se ou unir-se para agir em conjunto}
\end{EntryWithPhonetic}

\begin{EntryWithPhonetic}{联合国}{lian2 he2 guo2}{12,6,8}{⽿、⼝、⼞}[HSK 3]
  \definition*{s.}{Nações Unidas; Organização internacional fundada em 1945, após o fim da Segunda Guerra Mundial, com sede em Nova Iorque, Estados Unidos ; as suas principais instituições são a Assembleia Geral, o Conselho de Segurança, o Conselho Econômico e Social e o Secretariado; de acordo com a Carta das Nações Unidas, os seus principais objetivos são manter a paz e a segurança internacionais, desenvolver relações amigáveis entre os países e promover a cooperação internacional nas áreas econômica e cultural}
\end{EntryWithPhonetic}

\begin{EntryWithPhonetic}{联合会}{lian2he2hui4}{12,6,6}{⽿、⼝、⼈}
  \definition{s.}{federação}
\end{EntryWithPhonetic}

\begin{EntryWithPhonetic}{联络}{lian2luo4}{12,9}{⽿、⽷}[HSK 5]
  \definition{v.}{entrar em contato; comunicar-se; entrar em contato com}
\end{EntryWithPhonetic}

\begin{EntryWithPhonetic}{联盟}{lian2meng2}{12,13}{⽿、⽫}[HSK 6]
  \definition{s.}{aliança; coalizão; liga; união}
\end{EntryWithPhonetic}

\begin{EntryWithPhonetic}{联赛}{lian2 sai4}{12,14}{⽿、⾙}[HSK 6]
  \definition{s.}{jogos da liga | liga (esportiva) | torneio da liga}
\end{EntryWithPhonetic}

\begin{EntryWithPhonetic}{联手}{lian2 shou3}{12,4}{⽿、⼿}[HSK 6]
  \definition{v.}{dar as mãos; cooperar | Literário: dar as mãos | agir em conjunto}
\end{EntryWithPhonetic}

\begin{EntryWithPhonetic}{联系}{lian2xi4}{12,7}{⽿、⽷}[HSK 3]
  \definition[个,种,层]{s.}{relacionamento; relacionamento entre duas coisas}
  \definition{v.}{entrar em contato; contatar; comunicar-se com alguém por telefone, e-mail ou carta | agendar; entrar em contato com; estabelecer algum tipo de relação com a outra parte | relacionar; combinar; integrar}
\end{EntryWithPhonetic}

\begin{EntryWithPhonetic}{联想}{lian2xiang3}{12,13}{⽿、⼼}[HSK 5]
  \definition*{s.}{Lenovo (empresa)}
  \definition{v.}{associar-se a; estabelecer uma conexão mental; lembrar-se de algo; lembrar-se de outras pessoas ou coisas relacionadas devido a alguém ou algo; evocar outros conceitos relacionados devido a um determinado conceito}
\end{EntryWithPhonetic}

\begin{EntryWithPhonetic}{脸}{lian3}{11}{⾁}[HSK 2]
  \definition[张,个]{s.}{rosto (de pessoas ou animais); a parte frontal da cabeça, da testa ao queixo | parte frontal de algo | cara; autoestima; aparência | rosto; expressões faciais}
\end{EntryWithPhonetic}

\begin{EntryWithPhonetic}{脸盆}{lian3 pen2}{11,9}{⾁、⽫}[HSK 5]
  \definition[个]{s.}{lavatório; bacia para lavar as mãos e o rosto}
\end{EntryWithPhonetic}

\begin{EntryWithPhonetic}{脸色}{lian3 se4}{11,6}{⾁、⾊}[HSK 5]
  \definition{s.}{aparência; tez; cor da pele | aparência; expressão facial | (indicando a condição física de alguém) aparência; cor}
\end{EntryWithPhonetic}

\begin{EntryWithPhonetic}{练}{lian4}{8}{⽷}[HSK 2]
  \definition*{s.}{Sobrenome Lian}
  \definition{adj.}{habilidoso; experiente; bem treinado}
  \definition{s.}{seda branca}
  \definition{v.}{tratar, amaciar e branquear a seda por meio de fervura; cozinhar seda crua ou tecidos de seda crua | treinar; praticar; exercitar}
\end{EntryWithPhonetic}

\begin{EntryWithPhonetic}{练习}{lian4xi2}{8,3}{⽷、⼄}[HSK 2]
  \definition[项,次]{s.}{exercício (em livros); tarefas ou exercícios organizados para consolidar os resultados da aprendizagem}
  \definition{v.}{praticar; exercitar; repitir várias vezes até ficar bem treinado}
\end{EntryWithPhonetic}

\begin{EntryWithPhonetic}{炼}{lian4}{9}{⽕}
  \definition{v.}{fundir; refinar | temperar (um metal) com fogo | pesar a palavra; procurar a frase certa; polir | trabalhar; tornar uma substância pura ou resistente por aquecimento, etc. | polir; fazer as palavras bonitas e concisas}
\end{EntryWithPhonetic}

\begin{EntryWithPhonetic}{恋}{lian4}{10}{⼼}
  \definition*{s.}{Sobrenome Lian}
  \definition{v.}{amor (romântico) | ansiar por; sentir-se apegado a | amar; apaixonar-se por | não querendo se separar de; sentir sua falta para sempre; não suportar ficar separado}
\end{EntryWithPhonetic}

\begin{EntryWithPhonetic}{恋爱}{lian4'ai4}{10,10}{⼼、⽖}[HSK 5]
  \definition[个,场,段]{s.}{namoro; afeto; amor romântico; ações que demonstram o amor mútuo}
  \definition{v.}{amar; estar apaixonado}
\end{EntryWithPhonetic}

\begin{EntryWithPhonetic}{良}{liang2}{7}{⾉}
  \definition*{s.}{Sobrenome Liang}
  \definition{adj.}{bom; ótimo; agradável}
  \definition{adv.}{muito; muito mesmo; de fato}
  \definition{s.}{boas pessoas; pessoas gentis; talentos excepcionais}
\end{EntryWithPhonetic}

\begin{EntryWithPhonetic}{良好}{liang2hao3}{7,6}{⾉、⼥}[HSK 4]
  \definition{adj.}{bom; ótimo; bem; satisfatório}
\end{EntryWithPhonetic}

\begin{EntryWithPhonetic}{良田}{liang2tian2}{7,5}{⾉、⽥}
  \definition{s.}{terra agrícola boa | terra fértil}
\end{EntryWithPhonetic}

\begin{EntryWithPhonetic}{良心}{liang2xin1}{7,4}{⾉、⼼}
  \definition{s.}{consciência}
\end{EntryWithPhonetic}

\begin{EntryWithPhonetic}{凉}{liang2}{10}{⼎}[HSK 2]
  \definition{adj.}{frio; gelado; ligeiramente fria (menos do que 冷) | sombrio; desolado; sem animação | desanimado; desapontado | usado para prevenir o calor e manter a temperatura amena; para proteção contra o calor}
  \definition{s.}{frio; refere-se a um ambiente fresco ou a uma brisa fresca}
  \seeref{liang4}
  \seealsoref{冷}{leng3}
\end{EntryWithPhonetic}

\begin{EntryWithPhonetic}{凉快}{liang2kuai5}{10,7}{⼎、⼼}[HSK 2]
  \definition{adj.}{fresco; refrescante}
  \definition{v.}{refrescar; refrescar-se; deixar o corpo fresco e revigorado}
\end{EntryWithPhonetic}

\begin{EntryWithPhonetic}{凉水}{liang2 shui3}{10,4}{⼎、⽔}[HSK 3]
  \definition{s.}{água fria; água não aquecida | água não fervida}
\end{EntryWithPhonetic}

\begin{EntryWithPhonetic}{凉鞋}{liang2 xie2}{10,15}{⼎、⾰}[HSK 6]
  \definition[双,只]{s.}{sandália; alpargata; alpercata; alparca ; sapatos de verão com cabedal ventilado}
\end{EntryWithPhonetic}

\begin{EntryWithPhonetic}{量}{liang2}{12}{⾥}[HSK 4]
  \definition{v.}{medir | estimar; dimensionar}
  \seeref{liang4}
\end{EntryWithPhonetic}

\begin{EntryWithPhonetic}{粮}{liang2}{13}{⽶}
  \definition[斤,粒]{s.}{grãos; alimentos; provisões | imposto sobre grãos | nutrição | imposto agrícola; grãos como imposto agrícola}
\end{EntryWithPhonetic}

\begin{EntryWithPhonetic}{粮食}{liang2shi5}{13,9}{⽶、⾷}[HSK 4]
  \definition[种,吨,袋,颗,粒]{s.}{alimentos; grãos; termo geral para os vários tipos de arroz, feijão, etc. que podem ser consumidos}
\end{EntryWithPhonetic}

\begin{EntryWithPhonetic}{两}{liang3}{7}{⼀}[HSK 1,2]
  \definition*{s.}{Sobrenome Liang}
  \definition{clas.}{liang, uma unidade de peso (=50 gramas)}
  \definition{num.}{dois (sempre usado antes de classificadores) | poucos; alguns; indica um número indeterminado}
  \definition{s.}{ambos (lados); qualquer (lado)}
\end{EntryWithPhonetic}

\begin{EntryWithPhonetic}{两岸}{liang3 an4}{7,8}{⼀、⼭}[HSK 5]
  \definition{s.}{ambos os lados; ambas as margens; ambas as costas; entre os dois lados do estreito; bilateral}
\end{EntryWithPhonetic}

\begin{EntryWithPhonetic}{两边}{liang3 bian1}{7,5}{⼀、⾡}[HSK 4]
  \definition{s.}{ambos os lados; ambas as direções; ambos os lugares | ambas as partes; ambos os lados}
\end{EntryWithPhonetic}

\begin{EntryWithPhonetic}{两侧}{liang3 ce4}{7,8}{⼀、⼈}[HSK 6]
  \definition{s.}{dois flancos; dois (ambos) lados; ambos}
\end{EntryWithPhonetic}

\begin{EntryWithPhonetic}{两码事}{liang3ma3shi4}{7,8,8}{⼀、⽯、⼅}
  \definition{expr.}{duas coisas completamente diferentes; dois assuntos diferentes}
\end{EntryWithPhonetic}

\begin{EntryWithPhonetic}{两手}{liang3 shou3}{7,4}{⼀、⼿}[HSK 6]
  \definition{s.}{ambas as mãos | ambos os aspectos; táticas duplas | Coloquial: habilidade; capacidade}
\end{EntryWithPhonetic}

\begin{EntryWithPhonetic}{亮}{liang4}{9}{⼇}[HSK 2]
  \definition*{s.}{Sobrenome Lian}
  \definition{adj.}{brilhante; claro | alto e claro; retumbante | esclarecido; aberto e claro}
  \definition{s.}{luz}
  \definition{v.}{iluminar; clarear; brilhar | elevar a voz; ressoar; tornar o som mais alto | revelar; mostrar; aparecer; exibir}
\end{EntryWithPhonetic}

\begin{EntryWithPhonetic}{凉}{liang4}{10}{⼎}
  \definition{v.}{deixar algo esfriar; deixar um objeto quente descansar por um tempo para que a temperatura diminua}
  \seeref{liang2}
\end{EntryWithPhonetic}

\begin{EntryWithPhonetic}{辆}{liang4}{11}{⾞}[HSK 2]
  \definition{clas.}{usado para automóveis, veículos, etc.}
\end{EntryWithPhonetic}

\begin{EntryWithPhonetic}{量}{liang4}{12}{⾥}
  \definition{s.}{instrumento de medida; antigamente, o termo se referia a objetos como baldes e litros, que medem o volume | capacidade (para tolerância ou ingestão de alimentos ou bebidas); refere-se ao limite do que pode ser acomodado | quantidade; valor; volume; número}
  \definition{v.}{estimar; medir; pesar}
  \seeref{liang2}
\end{EntryWithPhonetic}

\begin{EntryWithPhonetic}{疗}{liao2}{7}{⽧}
  \definition{v.}{tratar; curar | recuperar}
\end{EntryWithPhonetic}

\begin{EntryWithPhonetic}{疗养}{liao2 yang3}{7,9}{⽧、⼋}[HSK 4]
  \definition{v.}{recuperar; convalescer; tratar pessoas com doenças crônicas ou debilitantes em instituições médicas especializadas com foco na recuperação}
\end{EntryWithPhonetic}

\begin{EntryWithPhonetic}{聊}{liao2}{11}{⽿}[HSK 6]
  \definition*{s.}{Sobrenome Liao}
  \definition{adv.}{apenas; meramente; provisoriamente; por enquanto | um pouco; ligeiramente}
  \definition{v.}{tagarelar; conversar; bater papo | confiar (ou depender, recorrer) a}
\end{EntryWithPhonetic}

\begin{EntryWithPhonetic}{聊天}{liao2/tian1}{11,4}{⽿、⼤}
  \definition{v.+compl.}{papear | bater papo}
\end{EntryWithPhonetic}

\begin{EntryWithPhonetic}{聊天儿}{liao2/tian1r5}{11,4,2}{⽿、⼤、⼉}[HSK 6]
  \definition{v.+compl.}{conversar; fofocar; bater papo; duas ou mais pessoas conversando sem um tópico ou propósito específico}
\end{EntryWithPhonetic}

\begin{EntryWithPhonetic}{了}{liao3}{2}{⼅}
  \definition*{s.}{Sobrenome Liao}
  \definition{adv.}{inteiramente; um pouco; totalmente (mais usado em negativas)}
  \definition{v.}{terminar; concluir; encerrar; cumprir; eliminar; resolver | compreender; saber; perceber; saber claramente | expressar possibilidade ou impossibilidade; usado com 得 ou 不 após o verbo, indica possibilidade ou impossibilidade}
  \seeref{le5}
  \seealsoref{不}{bu4}
  \seealsoref{得}{de5}
\end{EntryWithPhonetic}

\begin{EntryWithPhonetic}{了不起}{liao3bu5qi3}{2,4,10}{⼅、⼀、⾛}[HSK 4]
  \definition{adj.}{incrível; fantástico; extraordinário | sério; grave}
\end{EntryWithPhonetic}

\begin{EntryWithPhonetic}{了解}{liao3jie3}{2,13}{⼅、⾓}[HSK 4]
  \definition{v.}{entender; compreender | investigar; indagar sobre}
\end{EntryWithPhonetic}

\begin{EntryWithPhonetic}{料}{liao4}{10}{⽃}[HSK 6]
  \definition{clas.}{usado na medicina tradicional chinesa para preparar pílulas | unidade usada para calcular um pedaço de madeira, é a seção transversal em ambas as extremidades, que é de 1 pé (quadrado) com 7 pés de comprimento}
  \definition{s.}{material; coisa | (grão) alimento; forragem | artigos de vidro; vidros coloridos opacos | (para pílulas de medicina chinesa) prescrição}
  \definition{v.}{supor; esperar; antecipar | gerenciar; cuidar de | prever}
\end{EntryWithPhonetic}

\begin{EntryWithPhonetic}{列}{lie4}{6}{⼑}[HSK 4]
  \definition*{s.}{Sobrenome Lie}
  \definition{clas.}{usado para coisas em linhas e colunas}
  \definition{pron.}{cada um e todos; cada; muito}
  \definition{s.}{linha; arquivo; classificação (oposto a 行) | classificação; escopo | ranque | tipo}
  \definition{v.}{organizar; alinhar; colocar em ordem | listar; inserir em uma lista; classificar | formar uma linha}
  \seealsoref{行}{hang2}
\end{EntryWithPhonetic}

\begin{EntryWithPhonetic}{列车}{lie4che1}{6,4}{⼑、⾞}[HSK 4]
  \definition[列,班,趟,辆,节]{s.}{trem; trem em uma composição contínua, puxado por uma locomotiva e equipado com uma tripulação e marcações prescritas; geralmente um trem de passageiros}
\end{EntryWithPhonetic}

\begin{EntryWithPhonetic}{列入}{lie4 ru4}{6,2}{⼑、⼊}[HSK 4]
  \definition{v.}{listar; entrar em uma lista; ser incluído em | incluir em uma lista; juntar-se; registrar-se}
\end{EntryWithPhonetic}

\begin{EntryWithPhonetic}{列为}{lie4 wei2}{6,4}{⼑、⼂}[HSK 4]
  \definition{v.}{ser classificado como; ser listado como}
\end{EntryWithPhonetic}

\begin{EntryWithPhonetic}{烈}{lie4}{10}{⽕}
  \definition*{s.}{Sobrenome Lie}
  \definition{adj.}{forte; violento; intenso; feroz | justo; severo | firme; convicto; rigoroso}
  \definition{s.}{pessoa que morreu por uma causa justa | conquistas; façanhas | mártir sacrificando-se por uma causa justa}
\end{EntryWithPhonetic}

\begin{EntryWithPhonetic}{烈士}{lie4shi4}{10,3}{⽕、⼠}
  \definition{s.}{mártir}
\end{EntryWithPhonetic}

\begin{EntryWithPhonetic}{猎}{lie4}{11}{⽝}
  \definition[个]{s.}{traje de caça}
  \definition{v.}{caçar | procurar; perseguir}
\end{EntryWithPhonetic}

\begin{EntryWithPhonetic}{猎物}{lie4wu4}{11,8}{⽝、⽜}
  \definition{s.}{presa (vítima de um predador)}
\end{EntryWithPhonetic}

\begin{EntryWithPhonetic}{裂}{lie4}{12}{⾐}[HSK 6]
  \definition{s.}{entalhe; incisão; entalhes grandes e profundos nas bordas das folhas ou corolas | brecha; lacuna; rachadura; refere-se à rachadura ou divisão que aparece na superfície ou no interior de um objeto}
  \definition{v.}{dividir; rachar; rasgar | (figurativo) quebrar; esmagar; arruinar}
\end{EntryWithPhonetic}

\begin{EntryWithPhonetic}{邻}{lin2}{7}{⾢}
  \definition{adj.}{vizinho; perto; adjacente; perto; próximo}
  \definition{s.}{vizinho | bairro; vizinhança}
\end{EntryWithPhonetic}

\begin{EntryWithPhonetic}{邻居}{lin2ju1}{7,8}{⾢、⼫}[HSK 5]
  \definition[个,位,名,家]{s.}{vizinho; pessoas ou famílias que moram muito perto}
\end{EntryWithPhonetic}

\begin{EntryWithPhonetic}{临}{lin2}{9}{⼁}
  \definition*{s.}{Sobrenome Lin}
  \definition{adv.}{pouco antes; prestes a; no ponto de; indica que uma ação está prestes a ocorrer}
  \definition{v.}{encarar; enfrentar; aproximar-se | chegar; estar presente | copiar (um modelo de caligrafia ou pintura); traçar sobre as palavras ou figuras | olhar de cima para baixo | ir de cima para baixo}
\end{EntryWithPhonetic}

\begin{EntryWithPhonetic}{临近}{lin2jin4}{9,7}{⼁、⾡}
  \definition{v.}{aproximar-se; estar perto de}
\end{EntryWithPhonetic}

\begin{EntryWithPhonetic}{临时}{lin2shi2}{9,7}{⼁、⽇}[HSK 4]
  \definition{adj.}{temporário; provisório; por um breve período}
  \definition{adv.}{no momento em que algo acontece (quando as coisas dão errado)}
\end{EntryWithPhonetic}

\begin{EntryWithPhonetic}{淋}{lin2}{11}{⽔}
  \definition{v.}{borrifar | pingar | derramar | encharcar}
  \seeref{lin4}
\end{EntryWithPhonetic}

\begin{EntryWithPhonetic}{淋}{lin4}{11}{⽔}
  \definition{s.}{gonorréia}
  \definition{v.}{filtrar | coar | drenar}
  \seeref{lin2}
\end{EntryWithPhonetic}

\begin{EntryWithPhonetic}{令}{ling2}{5}{⼈}
  \definition*{s.}{Antigo nome geográfico, na região atual de Linyi, província de Shanxi | Sobrenome Ling}
  \seeref{ling3}
  \seeref{ling4}
\end{EntryWithPhonetic}

\begin{EntryWithPhonetic}{灵}{ling2}{7}{⽕}
  \definition*{s.}{Sobrenome Ling}
  \definition{adj.}{rápido; inteligente; afiado | eficaz; efetivo | flexível; hábil}
  \definition{s.}{espírito; alma | inteligência; mente | fada; duende; elfo | restos mortais do falecido; esquife | carro funerário; caixão ou algo relacionado aos mortos}
\end{EntryWithPhonetic}

\begin{EntryWithPhonetic}{灵感}{ling2gan3}{7,13}{⽕、⼼}
  \definition{s.}{inspiração | explosão de criatividade em empreendimento científico ou artístico}
\end{EntryWithPhonetic}

\begin{EntryWithPhonetic}{灵魂}{ling2hun2}{7,13}{⽕、⿁}
  \definition{s.}{alma | espírito}
\end{EntryWithPhonetic}

\begin{EntryWithPhonetic}{灵活}{ling2huo2}{7,9}{⽕、⽔}[HSK 6]
  \definition[种,点,些]{adj.}{ágil; rápido; ligeiro; descreve a capacidade de fazer rapidamente mudanças apropriadas com base na situação ao lidar com as coisas | flexível; elástico; descreve reações rápidas, como movimentos e funções cerebrais}
\end{EntryWithPhonetic}

\begin{EntryWithPhonetic}{铃}{ling2}{10}{⾦}[HSK 5]
  \definition[串,个]{s.}{sino; instrumento musical feito de metal | objetos em forma de sino | cápsula; botão; broto}
\end{EntryWithPhonetic}

\begin{EntryWithPhonetic}{铃声}{ling2 sheng1}{10,7}{⾦、⼠}[HSK 5]
  \definition{s.}{o tilintar de sinos; o som de um sino tocando}
\end{EntryWithPhonetic}

\begin{EntryWithPhonetic}{陵}{ling2}{10}{⾩}
  \definition*{s.}{Sobrenome Ling}
  \definition{s.}{colina; monte | túmulo imperial; mausoléu}
  \definition{v.}{(literário) intimidar; violar}
\end{EntryWithPhonetic}

\begin{EntryWithPhonetic}{陵园}{ling2yuan2}{10,7}{⾩、⼞}
  \definition{s.}{cemitério}
\end{EntryWithPhonetic}

\begin{EntryWithPhonetic}{菱}{ling2}{11}{⾋}
  \definition{s.}{maruca; caltrop aquático; castanha d'água}
\end{EntryWithPhonetic}

\begin{EntryWithPhonetic}{菱角}{ling2jiao5}{11,7}{⾋、⾓}
  \definition{s.}{castanha d'água}
\end{EntryWithPhonetic}

\begin{EntryWithPhonetic}{零}{ling2}{13}{⾬}[HSK 1]
  \definition*{s.}{Sobrenome Ling}
  \definition{adj.}{ímpar; dispersos; fragmentados (em oposição a 整)}
  \definition{num.}{zero; 0; também grafado como 〇; representa um número menor que qualquer número positivo e maior que qualquer número negativo; representa a ausência de quantidade | zero grau no termômetro | usado para indicar qualidade, comprimento, tempo, idade, etc. Entre dois dígitos, indica que a quantidade da unidade mais alta é acompanhada pela quantidade da unidade mais baixa | sinal de zero (0); nulo; espaço em branco para indicar números em caracteres chineses maiúsculos}
  \definition{s.}{fragmento; fração; lote ímpar; um número fracionário que não é suficiente para uma determinada unidade; um ponto decimal diferente de um inteiro}
  \definition{v.}{(de chuva, lágrimas, etc.) cair | murchar e cair}
  \seealsoref{整}{zheng3}
\end{EntryWithPhonetic}

\begin{EntryWithPhonetic}{零散}{ling2san3}{13,12}{⾬、⽁}
  \definition{adj.}{espalhado; disperso}
\end{EntryWithPhonetic}

\begin{EntryWithPhonetic}{零食}{ling2shi2}{13,9}{⾬、⾷}[HSK 4]
  \definition[包,袋,盒,箱,堆]{s.}{lanches; refrescos; petiscos entre as refeições; alimentação esporádica, além das refeições normais}
\end{EntryWithPhonetic}

\begin{EntryWithPhonetic}{零下}{ling2 xia4}{13,3}{⾬、⼀}[HSK 2]
  \definition{s.}{abaixo de zero; negativo}
\end{EntryWithPhonetic}

\begin{EntryWithPhonetic}{令}{ling3}{5}{⼈}
  \definition{clas.}{resma (de papel); unidade de medida de papel: 500 folhas inteiras de papel original produzidas mecanicamente equivalem a 1 resma}
  \seeref{ling2}
  \seeref{ling4}
\end{EntryWithPhonetic}

\begin{EntryWithPhonetic}{岭}{ling3}{8}{⼭}
  \definition{s.}{cordilheira}
\end{EntryWithPhonetic}

\begin{EntryWithPhonetic}{领}{ling3}{11}{⾴}[HSK 3]
  \definition{clas.}{usado para roupas, mantos, esteiras, tapetes, telas, etc.}
  \definition{s.}{pescoço; gargalo | gola; colarinho; faixa de pescoço | esboço; ponto principal; essência}
  \definition{v.}{conduzir; guiar; orientar | possuir; ser o possuidor de; ter jurisdição sobre | obter; conseguir; receber (o que foi distribuído) | aceitar; tomar |entender; compreender (o significado)}
\end{EntryWithPhonetic}

\begin{EntryWithPhonetic}{领带}{ling3 dai4}{11,9}{⾴、⼱}[HSK 5]
  \definition[条]{s.}{colar; gargantilha; gravata}
\end{EntryWithPhonetic}

\begin{EntryWithPhonetic}{领导}{ling3dao3}{11,6}{⾴、⼨}[HSK 3]
  \definition[个,位,名,些]{s.}{líder; liderança; pessoa que ocupa uma posição de liderança}
  \definition{v.}{liderar; exercer liderança; (elogio) liderar, gerenciar outras pessoas;  trabalhar com outras pessoas ou avançar em direção a um objetivo}
\end{EntryWithPhonetic}

\begin{EntryWithPhonetic}{领情}{ling3/qing2}{11,11}{⾴、⼼}
  \definition{v.+compl.}{sentir-se grato a alguém}
\end{EntryWithPhonetic}

\begin{EntryWithPhonetic}{领取}{ling3 qu3}{11,8}{⾴、⼜}[HSK 6]
  \definition{v.}{sacar; receber; obter; receber o que lhe é enviado}
\end{EntryWithPhonetic}

\begin{EntryWithPhonetic}{领先}{ling3xian1}{11,6}{⾴、⼉}[HSK 3]
  \definition{v.}{liderar; assumir a liderança; estar na liderança; (velocidade, desempenho, etc.) superar pessoas ou coisas semelhantes, estar na vanguarda}
\end{EntryWithPhonetic}

\begin{EntryWithPhonetic}{领袖}{ling3xiu4}{11,10}{⾴、⾐}[HSK 6]
  \definition[个,位,名]{s.}{líder de estados, grupos políticos, organizações de massa, etc.}
\end{EntryWithPhonetic}

\begin{EntryWithPhonetic}{令}{ling4}{5}{⼈}[HSK 5]
  \definition{adj.}{bom; excelente | termos de cortesia usados para se referir aos familiares e parentes da outra pessoa}
  \definition{s.}{ordem; decreto; comando; ordem emitida pela autoridade superior | um título oficial; administradores de certos departamentos governamentais na antiguidade | temporada; estação; clima e fenologia de uma determinada estação | poema-canção; letra curta}
  \definition{v.}{ordenar; comandar | fazer com que alguém; fazer com que; permitir que}
  \seeref{ling2}
  \seeref{ling3}
\end{EntryWithPhonetic}

\begin{EntryWithPhonetic}{令人}{ling4ren2}{5,2}{⼈、⼈}
  \definition{v.}{causar alguém (a fazer alguma coisa) | fazer alguém ficar zangado, encantado, etc.}
\end{EntryWithPhonetic}

\begin{EntryWithPhonetic}{另}{ling4}{5}{⼝}[HSK 6]
  \definition*{s.}{Sobrenome Ling}
  \definition{adv.}{além disso; indica que está fora do escopo da declaração | no lugar de; em vez de}
  \definition{pron.}{(com substantivos) outro; diferente; refere-se a pessoas ou coisas fora do escopo do que é dito}
\end{EntryWithPhonetic}

\begin{EntryWithPhonetic}{另外}{ling4wai4}{5,5}{⼝、⼣}[HSK 3]
  \definition{adv.}{além disso; em adição; ademais; além do mais; além de que; além do que já foi dito}
  \definition{conj.}{além disso; usada entre duas ou mais frases, indica algo além do que foi mencionado anteriormente}
  \definition{pron.}{outro; além das pessoas ou coisas mencionadas anteriormente}
\end{EntryWithPhonetic}

\begin{EntryWithPhonetic}{另一方面}{ling4 yi4 fang1 mian4}{5,1,4,9}{⼝、⼀、⽅、⾯}[HSK 3]
  \definition{adv./conj.}{outro aspecto | por outro lado; por sua vez; em contrapartida}
\end{EntryWithPhonetic}

\begin{EntryWithPhonetic}{刘}{liu2}{6}{⼑}
  \definition*{s.}{Sobrenome Liu}
  \definition{s.}{Clássico: um tipo de machado de batalha}
  \definition{v.}{matar; massacrar}
\end{EntryWithPhonetic}

\begin{EntryWithPhonetic}{流}{liu2}{10}{⽔}[HSK 2]
  \definition*{s.}{Sobrenome Liu}
  \definition{adj.}{fluente; tão suave quanto a água corrente}
  \definition{clas.}{lúmen; abreviação de lumens, 流明}
  \definition[名,个]{s.}{corrente de água | corrente; algo que se assemelha a um fluxo de água | razão; taxa; classe; grau; ramificação; facção; hierarquia}
  \definition{v.}{(de líquido) fluir | vaguear; vagar; mover-se de um lugar para outro; movimentar-se sem direção fixa | espalhar; circular; transmitir; divulgar | degenerar; mudar para pior; tender (aspecto negativo) | banir; enviar para o exílio | correr (ou fluir) como líquido; refere-se à parte do rio após deixar sua nascente (em contraste com a 源)}
  \seealsoref{流明}{liu2ming2}
  \seealsoref{源}{yuan2}
\end{EntryWithPhonetic}

\begin{EntryWithPhonetic}{流传}{liu2chuan2}{10,6}{⽔、⼈}[HSK 4]
  \definition[间]{v.}{espalhar; circular; passar adiante}
\end{EntryWithPhonetic}

\begin{EntryWithPhonetic}{流动}{liu2 dong4}{10,6}{⽔、⼒}[HSK 5]
  \definition{v.}{(água, ar, etc.) fluir; correr; circular | ir de um lugar para outro; estar em movimento; ser móvel (oposto a 固定)}
  \seealsoref{固定}{gu4ding4}
\end{EntryWithPhonetic}

\begin{EntryWithPhonetic}{流感}{liu2 gan3}{10,13}{⽔、⼼}[HSK 6]
  \definition{s.}{gripe; influenza; abreviação de 流行性感冒}
  \seealsoref{流行性感冒}{liu2xing2 xing4 gan3mao4}
\end{EntryWithPhonetic}

\begin{EntryWithPhonetic}{流利}{liu2li4}{10,7}{⽔、⼑}[HSK 2]
  \definition{adj.}{fluente; suave; lúcido; falar e escrever com fluência e clareza | com fluência; sem dificuldades}
\end{EntryWithPhonetic}

\begin{EntryWithPhonetic}{流明}{liu2ming2}{10,8}{⽔、⽇}
  \definition{s.}{(empréstimo linguístico) lúmen (unidade de fluxo luminoso)}
\end{EntryWithPhonetic}

\begin{EntryWithPhonetic}{流水}{liu2shui3}{10,4}{⽔、⽔}
  \definition{s.}{água corrente | (negócio) rotatividade}
\end{EntryWithPhonetic}

\begin{EntryWithPhonetic}{流通}{liu2tong1}{10,10}{⽔、⾡}[HSK 5]
  \definition{v.}{(ar, dinheiro, mercadorias, etc.) fluir; circular}
\end{EntryWithPhonetic}

\begin{EntryWithPhonetic}{流星}{liu2xing1}{10,9}{⽔、⽇}
  \definition{s.}{meteoro | estrela cadente}
\end{EntryWithPhonetic}

\begin{EntryWithPhonetic}{流行}{liu2xing2}{10,6}{⽔、⾏}[HSK 2]
  \definition{adj.}{popular; na moda; muito popular}
  \definition{v.}{ser popular; prevalecer; espalhar-se amplamente; divulgar amplamente}
\end{EntryWithPhonetic}

\begin{EntryWithPhonetic}{流行性感冒}{liu2xing2 xing4 gan3mao4}{10,6,8,13,9}{⽔、⾏、⼼、⼼、⽇}
  \definition{s.}{gripe muito forte; influenza}
\end{EntryWithPhonetic}

\begin{EntryWithPhonetic}{留}{liu2}{10}{⽥}[HSK 2]
  \definition*{s.}{Sobrenome Liu}
  \definition{v.}{ficar; permanecer; parar em um determinado local ou posição; não se afastar | estudar no exterior (geralmente seguido pelo nome de um país com uma sílaba) | pedir a alguém para ficar; manter alguém onde está | concentrar-se em; concentrar a atenção em algo | manter; guardar; reservar; não joger fora | acumular; deixar crescer | aceitar; receber | transmitir (legado); deixar para trás}
\end{EntryWithPhonetic}

\begin{EntryWithPhonetic}{留神}{liu2/shen2}{10,9}{⽥、⽰}
  \definition{v.+compl.}{tomar cuidado | prestar atenção | manter os olhos abertos}
\end{EntryWithPhonetic}

\begin{EntryWithPhonetic}{留下}{liu2 xia4}{10,3}{⽥、⼀}[HSK 2]
  \definition{v.}{deixar; parar em algum lugar}
\end{EntryWithPhonetic}

\begin{EntryWithPhonetic}{留学}{liu2xue2}{10,8}{⽥、⼦}[HSK 3]
  \definition{v.}{estudar no exterior; permanecer no estrangeiro para estudar ou pesquisar}
\end{EntryWithPhonetic}

\begin{EntryWithPhonetic}{留学生}{liu2 xue2 sheng1}{10,8,5}{⽥、⼦、⽣}[HSK 2]
  \definition[个,位,名,批,帮]{s.}{estudante estrangeiro; estudante que retornou; estudante que estuda no exterior}
\end{EntryWithPhonetic}

\begin{EntryWithPhonetic}{留言}{liu2 yan2}{10,7}{⽥、⾔}[HSK 6]
  \definition[条]{s.}{mensagem; recado}
  \definition{v.}{deixar uma mensagem; deixar seus comentários}
\end{EntryWithPhonetic}

\begin{EntryWithPhonetic}{柳}{liu3}{9}{⽊}
  \definition*{s.}{Liu, a vigésima quarta das vinte e oito constelações, consistindo de oito estrelas em Hydra | Liu, uma das mansões lunares | Sobrenome Liu}
  \definition[棵]{s.}{salgueiro}
\end{EntryWithPhonetic}

\begin{EntryWithPhonetic}{柳橙汁}{liu3cheng2zhi1}{9,16,5}{⽊、⽊、⽔}
  \definition[瓶,杯,罐,盒]{s.}{suco de laranja}
  \seealsoref{橙汁}{cheng2zhi1}
  \seealsoref{橘子汁}{ju2zi5zhi1}
\end{EntryWithPhonetic}

\begin{EntryWithPhonetic}{六}{liu4}{4}{⼋}[HSK 1]
  \definition*{s.}{Sobrenome Liu}
  \definition{num.}{seis; 6}
  \definition{s.}{símbolo musical utilizado na partitura da música tradicional chinesa, representando o primeiro grau da escala musical, equivalente ao ``5'' da notação musical simplificada}
\end{EntryWithPhonetic}

\begin{EntryWithPhonetic}{陆}{liu4}{7}{⾩}
  \definition{num.}{seis, usado para o numeral 六 em cheques, etc. para evitar erros ou alterações}
  \seeref{lu4}
  \seealsoref{六}{liu4}
\end{EntryWithPhonetic}

\begin{EntryWithPhonetic}{遛}{liu4}{13}{⾡}
  \definition{v.}{passear | andar (um animal) | caminhar conduzindo um animal doméstico}
\end{EntryWithPhonetic}

\begin{EntryWithPhonetic}{遛狗}{liu4/gou3}{13,8}{⾡、⽝}
  \definition{v.+compl.}{passear com um cachorro}
\end{EntryWithPhonetic}

\begin{EntryWithPhonetic}{龙}{long2}{5}{⿓}[HSK 3][Kangxi 212]
  \definition*{s.}{Sobrenome Long}
  \definition[条]{s.}{dragão; animal mítico e sobrenatural, com chifres, escamas, garras e bigodes, capaz de voar e mergulhar na água, provocar nuvens e chuva | dinossauro; um enorme réptil extinto; referência a certos répteis gigantes da antiguidade | do imperador; dragão como símbolo do imperador; usado na era feudal como símbolo do imperador; também se refere a coisas pertencentes ao imperador | em forma de dragão; com um desenho de dragão; refere-se a certos objetos que formam uma sequência semelhante a um dragão ou decorados com motivos de dragões}
\end{EntryWithPhonetic}

\begin{EntryWithPhonetic}{龙山}{long2shan1}{5,3}{⿓、⼭}
  \definition*{s.}{Longshan}
\end{EntryWithPhonetic}

\begin{EntryWithPhonetic}{龙虾}{long2xia1}{5,9}{⿓、⾍}
  \definition{s.}{lagosta}
\end{EntryWithPhonetic}

\begin{EntryWithPhonetic}{笼}{long2}{11}{⽵}
  \definition{s.}{armação fechada de bambu, arame, etc. | jaula | gaiola}
  \seeref{long3}
\end{EntryWithPhonetic}

\begin{EntryWithPhonetic}{笼子}{long2zi5}{11,3}{⽵、⼦}
  \definition{s.}{jaula | cesta | gaiola | recipiente}
  \seeref{long3zi5}
\end{EntryWithPhonetic}

\begin{EntryWithPhonetic}{笼}{long3}{11}{⽵}
  \definition{v.}{envolver | cobrir}
  \seeref{long2}
\end{EntryWithPhonetic}

\begin{EntryWithPhonetic}{笼子}{long3zi5}{11,3}{⽵、⼦}
  \definition{s.}{caixa grande | porta-malas}
  \seeref{long2zi5}
\end{EntryWithPhonetic}

\begin{EntryWithPhonetic}{弄}{long4}{7}{⼶}
  \definition{s.}{rua estreita; beco; viela; travessa}
  \seeref{nong4}
\end{EntryWithPhonetic}

\begin{EntryWithPhonetic}{楼}{lou2}{13}{⽊}[HSK 1]
  \definition*{s.}{Sobrenome Lou}
  \definition{clas.}{andar, piso}
  \definition[层,座,栋]{s.}{um prédio com muitos andares | piso; andar | superestrutura; uma estrutura com um convés superior; um andar adicional construído sobre uma casa ou outro edifício | nome usado para certas lojas ou locais de entretenimento | arco ornamental; certas construções decorativas altas com passagens por baixo}
\end{EntryWithPhonetic}

\begin{EntryWithPhonetic}{楼道}{lou2 dao4}{13,12}{⽊、⾡}[HSK 6]
  \definition[个]{s.}{corredor; passagem | passagem (em edifício de vários andares)}
\end{EntryWithPhonetic}

\begin{EntryWithPhonetic}{楼房}{lou2 fang2}{13,8}{⽊、⼾}[HSK 6]
  \definition[栋,幢,座,套,层]{s.}{um edifício de dois ou mais andares}
\end{EntryWithPhonetic}

\begin{EntryWithPhonetic}{楼上}{lou2 shang4}{13,3}{⽊、⼀}[HSK 1]
  \definition{s.}{no andar de cima | autor anterior em um tópico do fórum; em plataformas como fóruns na internet, refere-se à pessoa que se manifesta antes de você.}
\end{EntryWithPhonetic}

\begin{EntryWithPhonetic}{楼梯}{lou2 ti1}{13,11}{⽊、⽊}[HSK 4]
  \definition[个,层,段,阶]{s.}{escada; escadaria; degraus no meio de dois andares para permitir que as pessoas subam ou desçam as escadas}
\end{EntryWithPhonetic}

\begin{EntryWithPhonetic}{楼下}{lou2 xia4}{13,3}{⽊、⼀}[HSK 1]
  \definition{s.}{no andar de baixo}
\end{EntryWithPhonetic}

\begin{EntryWithPhonetic}{漏}{lou4}{14}{⽔}[HSK 5]
  \definition{s.}{relógio de água; ampulheta | falha; ponto fraco | gonorreia; a medicina tradicional chinesa refere-se a certas doenças que causam secreção de pus, sangue e muco | unidade de tempo medida por um relógio de água durante a noite}
  \definition{v.}{(líquido, gás, etc.) pingar; vazar; escorrer; cair (de um buraco ou fenda) | vazar; deixar escapar; divulgar | perder; deixar de fora por engano | vazar; o objeto tem poros e pode vazar coisas | há uma fuga de ar}
\end{EntryWithPhonetic}

\begin{EntryWithPhonetic}{漏电}{lou4dian4}{14,5}{⽔、⽥}
  \definition{v.}{vazar eletricidade}
\end{EntryWithPhonetic}

\begin{EntryWithPhonetic}{漏洞}{lou4 dong4}{14,9}{⽔、⽔}[HSK 5]
  \definition[个,点]{s.}{vazamento; rachadura; lacunas ou buracos desnecessários que permitem que coisas vazem | falha; defeito; lacuna; (fala, ação, método, etc.) imperfeições}
\end{EntryWithPhonetic}

\begin{EntryWithPhonetic}{露}{lou4}{21}{⾬}[HSK 6]
  \definition{v.}{mostrar; apresentar (uma certa emoção ou olhar no rosto) | mostrar; aparentar; fazer algo visível; as pessoas podem ver}
  \seeref{lu4}
\end{EntryWithPhonetic}

\begin{EntryWithPhonetic}{卢}{lu2}{5}{⼘}
  \definition*{s.}{Luxemburgo, abreviação de 卢森堡 | Sobrenome Lu}
  \definition{s.}{Aarcaico: preta (cor)}
  \seealsoref{卢森堡}{lu2sen1bao3}
\end{EntryWithPhonetic}

\begin{EntryWithPhonetic}{卢森堡}{lu2sen1bao3}{5,12,12}{⼘、⽊、⼟}
  \definition*{s.}{Luxemburgo}
\end{EntryWithPhonetic}

\begin{EntryWithPhonetic}{卢旺达}{lu2wang4da2}{5,8,6}{⼘、⽇、⾡}
  \definition*{s.}{Ruanda}
\end{EntryWithPhonetic}

\begin{EntryWithPhonetic}{芦}{lu2}{7}{⾋}
  \definition*{s.}{Sobrenome Lu}
  \definition{s.}{junco}
\end{EntryWithPhonetic}

\begin{EntryWithPhonetic}{芦笋}{lu2sun3}{7,10}{⾋、⽵}
  \definition{s.}{aspargos}
\end{EntryWithPhonetic}

\begin{EntryWithPhonetic}{陆}{lu4}{7}{⾩}
  \definition*{s.}{Sobrenome Lu}
  \definition[个]{s.}{terra; terreno | rota terrestre; por terra}
  \seeref{liu4}
\end{EntryWithPhonetic}

\begin{EntryWithPhonetic}{陆地}{lu4di4}{7,6}{⾩、⼟}[HSK 4]
  \definition[块,片]{s.}{terra; terra seca (em oposição ao mar); superfície da Terra, excluindo os oceanos (e, às vezes, rios e lagos)}
\end{EntryWithPhonetic}

\begin{EntryWithPhonetic}{陆军}{lu4 jun1}{7,6}{⾩、⼍}[HSK 6]
  \definition{s.}{força terrestre; exército}
\end{EntryWithPhonetic}

\begin{EntryWithPhonetic}{陆路}{lu4lu4}{7,13}{⾩、⾜}
  \definition{s.}{rota terrestre}
\end{EntryWithPhonetic}

\begin{EntryWithPhonetic}{陆续}{lu4xu4}{7,11}{⾩、⽷}[HSK 4]
  \definition{adv.}{sucessivamente; um após o outro; intermitentemente}
\end{EntryWithPhonetic}

\begin{EntryWithPhonetic}{录}{lu4}{8}{⼹}[HSK 3]
  \definition{s.}{registro; cadastro; coleção; seleções}
  \definition{v.}{copiar; gravar; escrever; copiar; registrar | contratar; selecionar; empregar; adotar ou nomear | gravar em fita magnética}
\end{EntryWithPhonetic}

\begin{EntryWithPhonetic}{录取}{lu4qu3}{8,8}{⼹、⼜}[HSK 4]
  \definition{v.}{aceitar; admitir; recrutar; entrar; matricular (os aprovados no exame)}
\end{EntryWithPhonetic}

\begin{EntryWithPhonetic}{录像}{lu4/xiang4}{8,13}{⼹、⼈}[HSK 6]
  \definition[段,个,些,盘]{s.}{vídeo; gravação; fita de vídeo; imagens gravadas com celulares, câmeras, etc.}
  \definition{v.+compl.}{gravar bídeo; gravar em fita de vídeo | usar celulares, câmeras e outros dispositivos para salvar registros de vídeo}
\end{EntryWithPhonetic}

\begin{EntryWithPhonetic}{录像带}{lu4xiang4dai4}{8,13,9}{⼹、⼈、⼱}
  \definition[盘]{s.}{video-cassete}
\end{EntryWithPhonetic}

\begin{EntryWithPhonetic}{录像机}{lu4xiang4ji1}{8,13,6}{⼹、⼈、⽊}
  \definition[台]{s.}{gravador de vídeo | VCR}
\end{EntryWithPhonetic}

\begin{EntryWithPhonetic}{录音}{lu4/yin1}{8,9}{⼹、⾳}[HSK 3]
  \definition[段,个]{s.}{gravação de som; som gravado com equipamento especializado}
  \definition{v.+compl.}{gravar; converter o som em sinal elétrico e, em seguida, gravá-lo por meios mecânicos, ópticos ou eletromagnéticos}
\end{EntryWithPhonetic}

\begin{EntryWithPhonetic}{录音机}{lu4 yin1 ji1}{8,9,6}{⼹、⾳、⽊}[HSK 6]
  \definition[台]{s.}{gravador de som; máquina de gravação (de fita)}
\end{EntryWithPhonetic}

\begin{EntryWithPhonetic}{鹿}{lu4}{11}{⿅}[Kangxi 198]
  \definition*{s.}{Sobrenome Lu}
  \definition[只,头,群]{s.}{cervo | veado}
\end{EntryWithPhonetic}

\begin{EntryWithPhonetic}{路}{lu4}{13}{⾜}[HSK 1]
  \definition*{s.}{Sobrenome Lu}
  \definition{clas.}{tipo; classe | linha; coluna; usado para um grupo de pessoas ou uma equipe; para organizar em ordem}
  \definition[条]{s.}{estrada; caminho; via | viagem; jornada; distância | maneira; meios | sequência; linha; lógica | região; distrito | rota | classe; classificação; grau | linha; fileira}
\end{EntryWithPhonetic}

\begin{EntryWithPhonetic}{路边}{lu4 bian1}{13,5}{⾜、⾡}[HSK 2]
  \definition{s.}{calçada; beira da estrada; margem da rua}
\end{EntryWithPhonetic}

\begin{EntryWithPhonetic}{路过}{lu4 guo4}{13,6}{⾜、⾡}[HSK 6]
  \definition{v.}{passar por (algum lugar); atravessar}
\end{EntryWithPhonetic}

\begin{EntryWithPhonetic}{路口}{lu4 kou3}{13,3}{⾜、⼝}[HSK 1]
  \definition[个]{s.}{cruzamento; intersecção; onde as estradas se encontram}
\end{EntryWithPhonetic}

\begin{EntryWithPhonetic}{路上}{lu4 shang5}{13,3}{⾜、⼀}[HSK 1]
  \definition{s.}{na estrada | a caminho; na rota; em processo de mudança de um lugar para outro}
\end{EntryWithPhonetic}

\begin{EntryWithPhonetic}{路线}{lu4 xian4}{13,8}{⾜、⽷}[HSK 3]
  \definition[条]{s.}{rota; caminho; linha; a estrada percorrida de um lugar a outro | linha; diretriz (de política, ideologia, campo de trabalho); a via fundamental a seguir em termos ideológicos, políticos ou profissionais}
\end{EntryWithPhonetic}

\begin{EntryWithPhonetic}{露}{lu4}{21}{⾬}[HSK 6]
  \definition{adj.}{fora de uma casa, tenda, etc., sem cobertura}
  \definition{s.}{orvalho; gotas de água condensadas | xarope; suco de fruta; bebida destilada de flores, folhas ou frutos}
  \definition{v.}{revelar; expor; mostrar; trair}
  \seeref{lou4}
\end{EntryWithPhonetic}

\begin{EntryWithPhonetic}{露珠}{lu4zhu1}{21,10}{⾬、⽟}
  \definition{s.}{orvalho}
\end{EntryWithPhonetic}

\begin{EntryWithPhonetic}{乱}{luan4}{7}{⼄}[HSK 3]
  \definition{adj.}{em desordem; em confusão; em desarrumação; sem ordem nem organização | em um estado mental confuso | (de uma sociedade) turbulento; agitado | (de relações sexuais) impróprio; promíscuo}
  \definition{adv.}{aleatoriamente; arbitrariamente; indiscriminadamente; sem restrições; à vontade}
  \definition{s.}{motim; agitação; tumulto; revolta; guerra; calamidade}
  \definition{v.}{confundir; embaralhar; misturar; causar desordem}
\end{EntryWithPhonetic}

\begin{EntryWithPhonetic}{伦}{lun2}{6}{⼈}
  \definition*{s.}{Sobrenome Lun}
  \definition{s.}{relações humanas (especialmente como concebidas pela ética feudal) | lógica; ordem | par; correspondência; (mesma) classe | ética; relações humanas | sequência lógica; ordem | o mesmo tipo; semelhante; igual}
\end{EntryWithPhonetic}

\begin{EntryWithPhonetic}{伦敦}{lun2dun1}{6,12}{⼈、⽁}
  \definition*{s.}{Londres}
\end{EntryWithPhonetic}

\begin{EntryWithPhonetic}{论}{lun2}{6}{⾔}
  \definition*{s.}{Os Analectos de Confúcio, registro dos ditos e feitos de Confúcio e seus discípulos}
  \seeref{lun4}
\end{EntryWithPhonetic}

\begin{EntryWithPhonetic}{轮}{lun2}{8}{⾞}[HSK 4]
  \definition{clas.}{usado para sol vermelho, lua brilhante, etc. | usado para rodadas | doze anos de idade (os doze ramos terrestres são usados para lembrar o gênero humano e cada doze anos de idade é um ciclo)}
  \definition{s.}{roda | anel; disco; objeto semelhante a uma roda | navio a vapor; barco a vapor}
  \definition{v.}{revezar; substituir um ao outro em sequência (para fazer algo)}
\end{EntryWithPhonetic}

\begin{EntryWithPhonetic}{轮船}{lun2chuan2}{8,11}{⾞、⾈}[HSK 4]
  \definition[艘,班]{s.}{vapor; navio a vapor; barco a vapor}
\end{EntryWithPhonetic}

\begin{EntryWithPhonetic}{轮回}{lun2hui2}{8,6}{⾞、⼞}
  \definition[个]{s.}{reencarnação (Budismo) | ciclo}
  \definition{v.}{reencarnar}
\end{EntryWithPhonetic}

\begin{EntryWithPhonetic}{轮椅}{lun2 yi3}{8,12}{⾞、⽊}[HSK 4]
  \definition{s.}{cadeira de rodas; dispositivo de assento especialmente projetado com rodas para pessoas com dificuldade de locomoção, que pode ser acionado por um disco de roda ou manivela operados manualmente}
\end{EntryWithPhonetic}

\begin{EntryWithPhonetic}{轮子}{lun2 zi5}{8,3}{⾞、⼦}[HSK 4]
  \definition[个,只]{s.}{roda; peças circulares de veículos ou máquinas com capacidade de rotação}
\end{EntryWithPhonetic}

\begin{EntryWithPhonetic}{论}{lun4}{6}{⾔}
  \definition*{s.}{Sobrenome Lun}
  \definition{prep.}{por (uma certa unidade de medida) | de acordo com (um certo sistema ou princípio)}
  \definition{s.}{visão; opinião; declaração | (frequentemente em títulos) dissertação; ensaio; tratado | teoria; doutrina | ideia; palavras ou artigos que analisam e explicam coisas}
  \definition{v.}{discutir; falar sobre; discursar sobre; comentar | mencionar; considerar; falar de | decidir sobre; determinar | decidir sobre a natureza da culpa; punir | argumentar; analisar e explicar coisas | considerar; ponderar; medir; avaliar}
  \seeref{lun2}
\end{EntryWithPhonetic}

\begin{EntryWithPhonetic}{论文}{lun4wen2}{6,4}{⾔、⽂}[HSK 4]
  \definition[篇]{s.}{tese; redação; artigo; artigo que discute ou examina uma questão}
\end{EntryWithPhonetic}

\begin{EntryWithPhonetic}{罗}{luo2}{8}{⽹}
  \definition*{s.}{Sobrenome Luo}
  \definition{clas.}{uma grosa; uma bruta; doze dúzias; 144 unidades}
  \definition{s.}{uma rede para capturar pássaros | peneira; tela | uma espécie de gaze de seda}
  \definition{v.}{pegar pássaros com uma rede | espalhar; exibir; mostrar | coletar; reunir; recrutar | peneirar}
\end{EntryWithPhonetic}

\begin{EntryWithPhonetic}{逻}{luo2}{11}{⾡}
  \definition{s.}{patrulha | (literário) a beira de um riacho de montanha}
  \definition{v.}{patrulhar; fazer rondas}
\end{EntryWithPhonetic}

\begin{EntryWithPhonetic}{逻辑}{luo2ji5}{11,13}{⾡、⾞}[HSK 5]
  \definition[套,条,种]{s.}{lógica; lei objetiva; a objetividade das leis que regem o desenvolvimento das coisas | lógica; razão; regras para o pensamento | lógica como ciência do raciocínio, do pensamento; disciplina que estuda a lógica}
\end{EntryWithPhonetic}

\begin{EntryWithPhonetic}{螺}{luo2}{17}{⾍}
  \definition{s.}{concha em espiral | caracol | búzio}
\end{EntryWithPhonetic}

\begin{EntryWithPhonetic}{螺丝}{luo2si1}{17,5}{⾍、⼀}
  \definition{s.}{parafuso}
\end{EntryWithPhonetic}

\begin{EntryWithPhonetic}{骆}{luo4}{9}{⾺}
  \definition*{s.}{Sobrenome Luo}
  \definition[只]{s.}{Arcaico: um cavalo branco com crina preta, mencionado em antigos livros chineses}
\end{EntryWithPhonetic}

\begin{EntryWithPhonetic}{骆驼}{luo4tuo5}{9,8}{⾺、⾺}
  \definition[头,只,匹]{s.}{camelo | coloquial: cabeça-dura, idiota}
\end{EntryWithPhonetic}

\begin{EntryWithPhonetic}{落}{luo4}{12}{⾋}[HSK 4]
  \definition*{s.}{Sobrenome Luo}
  \definition{s.}{paradeiro; lugar para ficar; local de descanso | assentamento; local de reunião | parte curta; área pequena; refere-se a um pequeno lugar ou área}
  \definition{v.}{cair; cair de uma altura elevada | se abaixar; descer; ir para baixo | abaixar; deixar cair (ou descer); fazer descer | afundar; declinar; descer | ficar para trás; ficar para trás ou ficar de fora | permanecer; fazer uma parada; deixar para trás | cair sobre; repousar com | obter; ter; receber | anotar; escrever no papel | cair em; entrar em; ficar preso}
  \seeref{la4}
  \seeref{lao4}
\end{EntryWithPhonetic}

\begin{EntryWithPhonetic}{落后}{luo4hou4}{12,6}{⾋、⼝}[HSK 3]
  \definition{adj.}{atrasado; trabalho em atraso, nível de desenvolvimento ou grau de reconhecimento (em oposição a 进步)}
  \definition{v.}{ficar para trás; ficar atrasado; ficar para trás em relação aos outros durante o avanço ou o progresso do trabalho}
  \seealsoref{进步}{jin4bu4}
\end{EntryWithPhonetic}

\begin{EntryWithPhonetic}{落花生}{luo4 hua1 sheng1}{12,7,5}{⾋、⾋、⽣}
  \definition{s.}{amendoim | noz de macaco}
\end{EntryWithPhonetic}

\begin{EntryWithPhonetic}{落日}{luo4ri4}{12,4}{⾋、⽇}
  \definition{s.}{pôr do sol}
\end{EntryWithPhonetic}

\begin{EntryWithPhonetic}{落实}{luo4shi2}{12,8}{⾋、⼧}[HSK 5]
  \definition{adj.}{sentimento de tranquilidade; (humor) estável; seguro}
  \definition{v.}{implementar; ser praticável; tornar os planos, políticas, medidas, etc. específicos e compreensíveis, de modo a que possam ser realizados | implementar; colocar em prática; pôr em prática significa que os planos, políticas e medidas são específicos e claros, e podem ser realizados}
\end{EntryWithPhonetic}

\begin{EntryWithPhonetic}{落汤鸡}{luo4tang1ji1}{12,6,7}{⾋、⽔、⿃}
  \definition{s.}{uma pessoa que parece encharcada e acamada| sofrimento profundo}
\end{EntryWithPhonetic}

\begin{EntryWithPhonetic}{驴}{lv2}{7}{⾺}
  \definition[头,只]{s.}{burro; asno; jumento; jegue}
\end{EntryWithPhonetic}

\begin{EntryWithPhonetic}{旅}{lv3}{10}{⽅}
  \definition{adv.}{juntos; conjuntamente}
  \definition[个]{s.}{brigada; unidade organizacional militar, abaixo do nível de divisão e acima do nível de regimento ou batalhão | força; tropas; geralmente se refere aos militares | viajante; passageiro; turista | viagem; jornada | pessoas}
  \definition{v.}{viajar; ficar longe de casa; ir para longe; morar longe de casa}
\end{EntryWithPhonetic}

\begin{EntryWithPhonetic}{旅程}{lv3cheng2}{10,12}{⽅、⽲}
  \definition{s.}{jornada | viagem}
\end{EntryWithPhonetic}

\begin{EntryWithPhonetic}{旅店}{lv3 dian5}{10,8}{⽅、⼴}[HSK 6]
  \definition[家,个]{s.}{pousada; albergue; hotel}
\end{EntryWithPhonetic}

\begin{EntryWithPhonetic}{旅馆}{lv3 guan3}{10,11}{⽅、⾷}[HSK 3]
  \definition[家,个,所]{s.}{pousada; hotel; local comercial destinado ao alojamento de viajantes}
\end{EntryWithPhonetic}

\begin{EntryWithPhonetic}{旅客}{lv3 ke4}{10,9}{⽅、⼧}[HSK 2]
  \definition[名,位,个,些]{s.}{viajante; passageiro; as agências de transporte e turismo referem-se às pessoas que viajam}
\end{EntryWithPhonetic}

\begin{EntryWithPhonetic}{旅行}{lv3xing2}{10,6}{⽅、⾏}[HSK 2]
  \definition{v.}{viajar; passear; para tratar de assuntos ou passear, ir de um lugar para outro (geralmente se refere a distâncias longas)}
\end{EntryWithPhonetic}

\begin{EntryWithPhonetic}{旅行社}{lv3 xing2 she4}{10,6,7}{⽅、⾏、⽰}[HSK 3]
  \definition[家]{s.}{agência de viagens; agência especializada em serviços relacionados a viagens, que providencia hospedagem, transporte e outros serviços para viajantes}
\end{EntryWithPhonetic}

\begin{EntryWithPhonetic}{旅游}{lv3you2}{10,12}{⽅、⽔}[HSK 2]
  \definition{v.}{viajar para outros lugares para passear e fazer turismo}
\end{EntryWithPhonetic}

\begin{EntryWithPhonetic}{屡}{lv3}{12}{⼫}
  \definition{adv.}{uma e outra vez; repetidamente | frequentemente}
\end{EntryWithPhonetic}

\begin{EntryWithPhonetic}{屡次}{lv3ci4}{12,6}{⼫、⽋}
  \definition{adv.}{repetidamente | uma e outra vez | muitas vezes}
\end{EntryWithPhonetic}

\begin{EntryWithPhonetic}{律}{lv4}{9}{⼻}
  \definition*{s.}{Sobrenome Lü}
  \definition{s.}{lei; regra; estatuto; regulamento}
  \definition{v.}{restringir; disciplinar; manter sob controle}
\end{EntryWithPhonetic}

\begin{EntryWithPhonetic}{律师}{lv4shi1}{9,6}{⼻、⼱}[HSK 4]
  \definition[名,个,位]{s.}{advogado; procurador; profissionais encarregados pelas partes ou nomeados pelo tribunal para auxiliar as partes no litígio, para comparecer ao tribunal para defesa e para tratar de assuntos jurídicos relacionados, de acordo com a lei}
\end{EntryWithPhonetic}

\begin{EntryWithPhonetic}{率}{lv4}{11}{⽞}
  \definition{s.}{taxa; razão; proporção; a relação proporcional entre duas grandezas relacionadas}
  \seeref{shuai4}
\end{EntryWithPhonetic}

\begin{EntryWithPhonetic}{绿}{lv4}{11}{⽷}[HSK 2]
  \definition*{s.}{Sobrenome Lü}
  \definition{adj.}{verde}
  \definition{v.}{tornar-se verde; ficar verde}
\end{EntryWithPhonetic}

\begin{EntryWithPhonetic}{绿茶}{lv4 cha2}{11,9}{⽷、⾋}[HSK 3]
  \definition{s.}{chá verde; chá produzido apenas através dos processos de maturação, enrolamento (ou sem enrolamento) e secagem, sem passar por fermentação, com cor verde-claro}
\end{EntryWithPhonetic}

\begin{EntryWithPhonetic}{绿豆}{lv4dou4}{11,7}{⽷、⾖}
  \definition{s.}{vagens}
\end{EntryWithPhonetic}

\begin{EntryWithPhonetic}{绿豆芽}{lv4dou4 ya2}{11,7,7}{⽷、⾖、⾋}
  \definition{s.}{broto de feijão verde}
\end{EntryWithPhonetic}

\begin{EntryWithPhonetic}{绿化}{lv4 hua4}{11,4}{⽷、⼔}[HSK 6]
  \definition{v.}{tornar verde plantando árvores, flores, etc.; arborizar; reflorestar; plantar árvores, flores e plantas para embelezar o ambiente ou prevenir a erosão do solo}
\end{EntryWithPhonetic}

\begin{EntryWithPhonetic}{绿色}{lv4 se4}{11,6}{⽷、⾊}[HSK 2]
  \definition{adj.}{verde; ecológico; sem poluição; em conformidade com os requisitos ambientais}
  \definition{s.}{cor verde}
\end{EntryWithPhonetic}

\begin{EntryWithPhonetic}{略}{lve4}{11}{⽥}
  \definition{adv.}{ligeiramente | marginalmente | aproximadamente}
\end{EntryWithPhonetic}

\begin{EntryWithPhonetic}{略微}{lve4wei1}{11,13}{⽥、⼻}
  \definition{adv.}{ligeiramente | marginalmente | aproximadamente}
\end{EntryWithPhonetic}

%%%%% EOF %%%%%


%%%
%%% M
%%%
\section*{M}
\addcontentsline{toc}{section}{M}
\begin{multicols}{2}
% \entry{妈\xpinyin{妈}{ma0}}{n.}{mamãe; mãe}
% \entry{马上}{adv.}{já; imediatamente}
\entry{吗}{part.}{ma0}{partícula interrogativa|perguntas ``sim-não''}
% \entry{买}{v.}{comprar}
% \entry{买东\xpinyin{西}{xi0}}{v.}{fazer compras}
% \entry{卖}{v.}{vender}
% \entry{忙}{adj.}{ocupado; ocupada}
% \entry{猫}{n.}{gato}
% \entry{没关\xpinyin{系}{xi0}}{}{não ter problema; não ter importância; não fazer mal}
% \entry{\xpinyin{美}{Mei2}国}{n.}{Estados Unidos da América}
% \entry{没有}{v.}{não há; não tem}
% \entry{每次}{}{todas as vezes; sempre}
% \entry{每天}{}{todos os dias}
% \entry{\xpinyin{美}{Mei3}洲}{n.}{América}
% \entry{妹\xpinyin{妹}{mei0}}{n.}{irmã mais nova}
\entry{们}{sufixo}{men0}{sufixo para plural}
% \entry{米饭}{n.}{arroz cozido}
% \entry{面包}{n.}{pão}
% \entry{明天}{p.t.}{amanhã}
% \entry{名字}{n.}{nome}
% \entry{明恋}{adj.}{claro; clara}
% \entry{明年}{n.}{próximo ano}
% \entry{墨镜}{n.}{óculos escuros}
\end{multicols}

%%%
%%% N
%%%

\section*{N}\addcontentsline{toc}{section}{N}

\begin{entry}{那}{na1}{6}{⾢}
  \definition*{s.}{sobrenome Na}
\end{entry}

\begin{entry}{拿}{na2}{10}{⼿}[HSK 1]
  \definition{part.}{usado da mesma forma que 把: para marcar o seguinte substantivo seguinte como objeto direto}
  \definition{v.}{segurar | tomar | pegar em}
\end{entry}

\begin{entry}{拿出}{na2 chu1}{10,5}{⼿、⼐}[HSK 2]
  \definition{v.}{apresentar (evidências) | prover | apresentar (uma proposta) | colocar para fora | retirar}
\end{entry}

\begin{entry}{拿到}{na2 dao4}{10,8}{⼿、⼑}[HSK 2]
  \definition{v.}{pegar | obter}
\end{entry}

\begin{entry}{那}{na3}{6}{⾢}
  \variantof{哪}
\end{entry}

\begin{entry}{哪}{na3}{9}{⼝}[HSK 1,4]
  \definition{adv.}{para expressar uma pergunta retórica}
  \definition{pron.}{qual?; o que? | qualquer; ser usado em um sentido geral}
  \seeref{哪}{na5}
  \seeref{哪}{nei3}
\end{entry}

\begin{entry}{哪国人}{na3 guo2ren2}{9,8,2}{⼝、⼞、⼈}
  \definition{expr.}{de qual país?}
\end{entry}

\begin{entry}{哪里}{na3 li3}{9,7}{⼝、⾥}[HSK 1]
  \definition{adv.}{onde?}
\end{entry}

\begin{entry}{哪怕}{na3pa4}{9,8}{⼝、⼼}[HSK 4]
  \definition{conj.}{mesmo; mesmo se; mesmo que; não importa o quão}
\end{entry}

\begin{entry}{哪儿}{na3r5}{9,2}{⼝、⼉}[HSK 1]
  \definition{adv.}{onde?}
\end{entry}

\begin{entry}{哪些}{na3xie1}{9,8}{⼝、⼆}[HSK 1]
  \definition{pron.}{quais?}
\end{entry}

\begin{entry}{那}{na4}{6}{⾢}[HSK 1,2]
  \definition{conj.}{nessa situação | nesse caso}
  \definition{pron.}{aquele | aquilo}
\end{entry}

\begin{entry}{那边}{na4bian5}{6,5}{⾢、⾡}[HSK 1]
  \definition{pron.}{ali | acolá}
\end{entry}

\begin{entry}{那会儿}{na4 hui4r5}{6,6,2}{⾢、⼈、⼉}[HSK 2]
  \definition{pron.}{então | naquela época}
\end{entry}

\begin{entry}{那里}{na4 li3}{6,7}{⾢、⾥}[HSK 1]
  \definition{pron.}{lá | ali}
\end{entry}

\begin{entry}{那么}{na4 me5}{6,3}{⾢、⼃}[HSK 2]
  \definition{adv.}{então | como aquele | dessa maneira}
\end{entry}

\begin{entry}{那麽}{na4 me5}{6,14}{⾢、⿇}
  \variantof{那么}
\end{entry}

\begin{entry}{那儿}{na4r5}{6,2}{⾢、⼉}[HSK 1]
  \definition{pron.}{lá | ali}
\end{entry}

\begin{entry}{那时}{na4 shi2}{6,7}{⾢、⽇}[HSK 2]
  \definition{pron.}{então | naquela época | naqueles dias}
\end{entry}

\begin{entry}{那时候}{na4 shi2 hou5}{6,7,10}{⾢、⽇、⼈}[HSK 2]
  \definition{adv.}{naquela hora}
\end{entry}

\begin{entry}{那些}{na4xie1}{6,8}{⾢、⼆}[HSK 1]
  \definition{pron.}{aqueles}
\end{entry}

\begin{entry}{那样}{na4 yang4}{6,10}{⾢、⽊}[HSK 2]
  \definition{pron.}{assim | tal | como esse | desse tipo}
\end{entry}

\begin{entry}{哪}{na5}{9}{⼝}
  \definition{part.}{usado depois de uma palavra com a terminação -n, é equivalente a ``啊''}
  \seeref{哪}{na3}
  \seeref{哪}{nei3}
  \seealsoref{啊}{a5}
\end{entry}

\begin{entry}{奶}{nai3}{5}{⼥}[HSK 1]
  \definition[杯,滴,瓶,只,桶]{s.}{seios | leite}
  \definition{v.}{amamentar}
\end{entry}

\begin{entry}{奶茶}{nai3 cha2}{5,9}{⼥、⾋}[HSK 3]
  \definition[杯]{s.}{chá com leite}
\end{entry}

\begin{entry}{奶奶}{nai3nai5}{5,5}{⼥、⼥}[HSK 1]
  \definition[位]{s.}{avó (paterna) | (respeitoso) dona da casa}
\end{entry}

\begin{entry}{耐心}{nai4xin1}{9,4}{⽽、⼼}
  \definition{s.}{paciência}
  \definition{v.}{ser paciente}
\end{entry}

\begin{entry}{男}{nan2}{7}{⽥}[HSK 1]
  \definition{adj.}{masculino}
  \definition{s.}{Barão, o mais baixo de cinco ordens de nobreza}
\end{entry}

\begin{entry}{男孩儿}{nan2hai2r5}{7,9,2}{⽥、⼦、⼉}[HSK 1]
  \definition{s.}{menino | rapaz}
\end{entry}

\begin{entry}{男女}{nan2 nv3}{7,3}{⽥、⼥}[HSK 4]
  \definition{s.}{homens e mulheres; masculino e feminino}
\end{entry}

\begin{entry}{男朋友}{nan2peng2you5}{7,8,4}{⽥、⽉、⼜}[HSK 1]
  \definition{s.}{namorado}
\end{entry}

\begin{entry}{男人}{nan2ren2}{7,2}{⽥、⼈}[HSK 1]
  \definition[个]{s.}{um homem | um macho | cavalheiro | marido}
\end{entry}

\begin{entry}{男生}{nan2sheng1}{7,5}{⽥、⽣}[HSK 1]
  \definition[个]{s.}{aluno | estudante do sexo masculino}
\end{entry}

\begin{entry}{男士}{nan2 shi4}{7,3}{⽥、⼠}[HSK 4]
  \definition{s.}{cavalheiro; \emph{gentleman}}
\end{entry}

\begin{entry}{男子}{nan2zi3}{7,3}{⽥、⼦}[HSK 3]
  \definition[名]{s.}{homem; macho}
\end{entry}

\begin{entry}{南}{nan2}{9}{⼗}[HSK 1]
  \definition*{s.}{sobrenome Nan}
  \definition{s.}{sul}
\end{entry}

\begin{entry}{南边}{nan2bian5}{9,5}{⼗、⾡}[HSK 1]
  \definition{adv.}{sul | lado sul | parte sul | ao sul de}
\end{entry}

\begin{entry}{南部}{nan2 bu4}{9,10}{⼗、⾢}[HSK 3]
  \definition{s.}{parte sul; sul | a parte sul}
\end{entry}

\begin{entry}{南方}{nan2 fang1}{9,4}{⼗、⽅}[HSK 2]
  \definition{s.}{sul | o Sul | a parte sul do país}
\end{entry}

\begin{entry}{南极}{nan2ji2}{9,7}{⼗、⽊}
  \definition*{s.}{Antártico | Pólo Sul}
  \definition{s.}{pólo sul magnético}
\end{entry}

\begin{entry}{南面}{nan2mian4}{9,9}{⼗、⾯}
  \definition{s.}{sul | lado sul}
\end{entry}

\begin{entry}{难}{nan2}{10}{⾫}[HSK 1]
  \definition{adj.}{difícil}
  \definition{s.}{dificuldade}
  \seeref{难}{nan4}
\end{entry}

\begin{entry}{难道}{nan2dao4}{10,12}{⾫、⾡}[HSK 3]
  \definition{adv.}{indica uma pergunta retórica | certamente não significa que\dots?; é possível que\dots?; não me diga\dots; poderia ser que\dots?}
\end{entry}

\begin{entry}{难度}{nan2 du4}{10,9}{⾫、⼴}[HSK 3]
  \definition{s.}{dificuldade; grau de dificuldade}
\end{entry}

\begin{entry}{难过}{nan2guo4}{10,6}{⾫、⾡}[HSK 2]
  \definition{adj.}{triste | ruim | pesaroso | arrependido | difícil}
\end{entry}

\begin{entry}{难看}{nan2 kan4}{10,9}{⾫、⽬}[HSK 2]
  \definition{adj.}{feio | antiestético | vergonhoso | embaraçoso | vergonhoso}
\end{entry}

\begin{entry}{难受}{nan2shou4}{10,8}{⾫、⼜}[HSK 2]
  \definition{adj.}{sofrer dor | sentir-se mal | desconfortável | sentir-se infeliz}
\end{entry}

\begin{entry}{难题}{nan2 ti2}{10,15}{⾫、⾴}[HSK 2]
  \definition[出]{s.}{desafio | problema difícil | pergunta difícil}
\end{entry}

\begin{entry}{难听}{nan2 ting1}{10,7}{⾫、⼝}[HSK 2]
  \definition{adj.}{desagradável de ouvir | ofensivo | grosseiro | escandaloso}
\end{entry}

\begin{entry}{难}{nan4}{10}{⾫}
  \definition{s.}{desastre}
  \definition{v.}{repreender}
  \seeref{难}{nan2}
\end{entry}

\begin{entry}{难免}{nan4mian3}{10,7}{⾫、⼉}[HSK 4]
  \definition{adj.}{inevitável; difícil de evitar}
\end{entry}

\begin{entry}{孬}{nao1}{10}{⼥}
  \definition{adj.}{(dialeto) não (é) bom (contração de 不+好)}
\end{entry}

\begin{entry}{脑袋}{nao3dai5}{10,11}{⾁、⾐}[HSK 4]
  \definition[颗,个]{s.}{cabeça; a parte mais alta do corpo humano ou a parte mais alta de um animal que contém órgãos como a boca, o nariz, os olhos etc. | mente; cérebro; capacidade de pensar, lembrar, etc.}
\end{entry}

\begin{entry}{脑瓜}{nao3gua1}{10,5}{⾁、⽠}
  \definition{s.}{crânio | cérebro | cabeça | mente | mentalidade | ideia}
  \seealsoref{脑瓜子}{nao3gua1zi5}
\end{entry}

\begin{entry}{脑瓜子}{nao3gua1zi5}{10,5,3}{⾁、⽠、⼦}
  \definition{s.}{crânio | cérebro | cabeça | mente | mentalidade | ideia}
  \seealsoref{脑瓜}{nao3gua1}
\end{entry}

\begin{entry}{闹}{nao4}{8}{⾾}[HSK 4]
  \definition{adj.}{barulhento}
  \definition{v.}{fazer barulho; provocar problemas | dar vazão (à sua raiva, ressentimento, etc.) | sofrer de; ser incomodado por; ocorrer (um desastre ou coisa ruim) | fazer;  entrar em ação | agitar; perturbar | brincar; fazer bagunça}
\end{entry}

\begin{entry}{闹钟}{nao4 zhong1}{8,9}{⾾、⾦}[HSK 4]
  \definition[个,台,只]{s.}{despertador; relógios capazes de tocar alarmes em horários predeterminados}
\end{entry}

\begin{entry}{呢}{ne5}{8}{⼝}[HSK 1]
  \definition{part.}{(no final de uma frase declarativa) partícula que indica a continuação de um estado ou ação |  partícula para perguntar sobre a localização (``Onde está\dots?'') | partícula indicando  afirmação forte | partícula indicando que uma pergunta feita anteriormente deve ser aplicada à palavra anterior (``E quanto a\dots?'', ``E\dots?'') | partícula sinalizando uma pausa, para enfatizar as palavras anteriores e permitir que o ouvinte tenha tempo para compreendê-las (``ok?'', ``você está comigo ?'')}
  \seeref{呢}{ni2}
\end{entry}

\begin{entry}{哪}{nei3}{9}{⼝}
  \definition{part.}{qual? (interrogativo, seguido de classificador ou numeral-classificador)}
  \seeref{哪}{na3}
  \seeref{哪}{na5}
\end{entry}

\begin{entry}{内}{nei4}{4}{⼌}[HSK 3]
  \definition*{s.}{sobrenome Nei}
  \definition{adj.}{interno; interior}
  \definition{prep.}{dentro}
  \definition{s.}{interior; lado de dentro; parte de dentro | a esposa ou parentes dela}
\end{entry}

\begin{entry}{内部}{nei4bu4}{4,10}{⼌、⾢}[HSK 4]
  \definition{s.}{interior; dentro; interno; dentro de um determinado intervalo}
\end{entry}

\begin{entry}{内存}{nei4cun2}{4,6}{⼌、⼦}
  \definition{s.}{armazenamento interno | memória do computador | RAM (\emph{random access memory})}
  \seealsoref{随机存取存储器}{sui2ji1cun2qu3cun2chu3qi4}
  \seealsoref{随机存取记忆体}{sui2ji1cun2qu3ji4yi4ti3}
\end{entry}

\begin{entry}{内科}{nei4ke1}{4,9}{⼌、⽲}[HSK 4]
  \definition{s.}{medicina geral; clínica geral; clínica médica}
\end{entry}

\begin{entry}{内燃机}{nei4ran2ji1}{4,16,6}{⼌、⽕、⽊}
  \definition{s.}{motor de combustão interna}
\end{entry}

\begin{entry}{内容}{nei4rong2}{4,10}{⼌、⼧}[HSK 3]
  \definition[个]{s.}{conteúdo; substância}
\end{entry}

\begin{entry}{内心}{nei4 xin1}{4,4}{⼌、⼼}[HSK 3]
  \definition{s.}{coração; interior; íntimo do ser}
\end{entry}

\begin{entry}{内省}{nei4xing3}{4,9}{⼌、⽬}
  \definition{s.}{introspecção}
  \definition{v.}{refletir sobre si mesmo}
\end{entry}

\begin{entry}{能}{neng2}{10}{⾁}[HSK 1]
  \definition*{s.}{sobrenome Neng}
  \definition{adv.}{talvez}
  \definition{s.}{(física)nenergia | habilidade}
  \definition{v.}{poder | ser capaz de}
\end{entry}

\begin{entry}{能不能}{neng2 bu4 neng2}{10,4,10}{⾁、⼀、⾁}[HSK 3]
  \definition{adv.}{pode ou não pode\dots?}
\end{entry}

\begin{entry}{能干}{neng2gan4}{10,3}{⾁、⼲}[HSK 4]
  \definition{adj.}{apto; capaz; competente}
\end{entry}

\begin{entry}{能够}{neng2 gou4}{10,11}{⾁、⼣}[HSK 2]
  \definition{v.}{ser capaz de}
\end{entry}

\begin{entry}{能力}{neng2li4}{10,2}{⾁、⼒}[HSK 3]
  \definition{s.}{habilidade; capacidade; aptidão}
\end{entry}

\begin{entry}{能上能下}{neng2shang4neng2xia4}{10,3,10,3}{⾁、⼀、⾁、⼀}
  \definition{s.}{pronto para aceitar qualquer trabalho, alto ou baixo}
\end{entry}

\begin{entry}{呢}{ni2}{8}{⼝}
  \definition{s.}{material de lã}
  \seeref{呢}{ne5}
\end{entry}

\begin{entry}{泥}{ni2}{8}{⽔}
  \definition{s.}{lama | argila | pasta | polpa}
  \seeref{泥}{ni4}
\end{entry}

\begin{entry}{泥潭}{ni2tan2}{8,15}{⽔、⽔}
  \definition{s.}{atoleiro | lamaçal | charco | pântano}
\end{entry}

\begin{entry}{你}{ni3}{7}{⼈}[HSK 1]
  \definition{pron.}{você (informal) | tu | te | ti | contigo}
  \seeref{您}{nin2}
\end{entry}

\begin{entry}{你的}{ni3 de5}{7,8}{⼈、⽩}
  \definition{pron.}{seu}
\end{entry}

\begin{entry}{你好}{ni3hao3}{7,6}{⼈、⼥}
  \definition{interj.}{Olá! | Oi!}
\end{entry}

\begin{entry}{你们}{ni3men5}{7,5}{⼈、⼈}[HSK 1]
  \definition{pron.}{vocês (informal) | vós | vos | convosco}
\end{entry}

\begin{entry}{你们的}{ni3men5 de5}{7,5,8}{⼈、⼈、⽩}
  \definition{pron.}{vossos}
\end{entry}

\begin{entry}{伲}{ni4}{7}{⼈}
  \definition{pron.}{(dialeto) eu | meu | nosso | nós}
  \seeref{你}{ni3}
\end{entry}

\begin{entry}{泥}{ni4}{8}{⽔}
  \definition{adj.}{contido}
  \seeref{泥}{ni2}
\end{entry}

\begin{entry}{逆境}{ni4jing4}{9,14}{⾡、⼟}
  \definition{s.}{adversidade | tribulação}
\end{entry}

\begin{entry}{年}{nian2}{6}{⼲}[HSK 1]
  \definition*{s.}{sobrenome Nian}
  \definition[个]{clas./s.}{ano}
\end{entry}

\begin{entry}{年初}{nian2 chu1}{6,7}{⼲、⾐}[HSK 3]
  \definition{s.}{o começo do ano}
\end{entry}

\begin{entry}{年代}{nian2dai4}{6,5}{⼲、⼈}[HSK 3]
  \definition[个]{s.}{idade; anos; tempo | uma década de um século}
\end{entry}

\begin{entry}{年底}{nian2 di3}{6,8}{⼲、⼴}[HSK 3]
  \definition[个]{s.}{fim de ano; o fim do ano}
\end{entry}

\begin{entry}{年货}{nian2huo4}{6,8}{⼲、⾙}
  \definition{s.}{mercadorias vendidas no Ano Novo Chinês}
\end{entry}

\begin{entry}{年级}{nian2ji2}{6,6}{⼲、⽷}[HSK 2]
  \definition[个]{s.}{classe | ano (escola)}
\end{entry}

\begin{entry}{年纪}{nian2ji4}{6,6}{⼲、⽷}[HSK 3]
  \definition{s.}{era; época; idade}
\end{entry}

\begin{entry}{年轻}{nian2qing1}{6,9}{⼲、⾞}[HSK 2]
  \definition{adj.}{jovem}
\end{entry}

\begin{entry}{碾碎}{nian3sui4}{15,13}{⽯、⽯}
  \definition{v.}{pulverizar | esmagar}
\end{entry}

\begin{entry}{念}{nian4}{8}{⼼}[HSK 3]
  \definition*{s.}{sobrenome Nian}
  \definition{num.}{vinte; 20}
  \definition{s.}{ideia; pensamento}
  \definition{v.}{ler em voz alta | estudar; frequentar a escola | considerar; levar em conta | sentir falta; pensar em}
\end{entry}

\begin{entry}{鸟}{niao3}{5}{⿃}[HSK 2][Kangxi 196]
  \definition[只,群]{s.}{pássaro}
  \seeref{鸟}{diao3}
\end{entry}

\begin{entry}{鸟儿}{niao3r5}{5,2}{⿃、⼉}
  \definition[只]{s.}{pássaro | ave}
\end{entry}

\begin{entry}{尿}{niao4}{7}{⼫}
  \definition[泡]{s.}{urina}
  \definition{v.}{urinar}
  \seeref{尿}{sui1}
\end{entry}

\begin{entry}{您}{nin2}{11}{⼼}[HSK 1]
  \definition{pron.}{você (formal) | tu | te | ti | contigo}
  \seeref{你}{ni3}
\end{entry}

\begin{entry}{宁}{ning2}{5}{⼧}
  \definition*{s.}{sobrenome Ning}
  \definition{adj.}{calmo, pacífico, sereno | saudável}
  \seeref{宁}{ning4}
\end{entry}

\begin{entry}{宁静}{ning2 jing4}{5,14}{⼧、⾭}[HSK 4]
  \definition{adj.}{calmo; tranquilo; pacífico}
\end{entry}

\begin{entry}{柠檬}{ning2meng2}{9,17}{⽊、⽊}
  \definition{s.}{limão}
\end{entry}

\begin{entry}{拧开}{ning3kai1}{8,4}{⼿、⼶}
  \definition{v.}{desaparafusar | desatarrachar | torcer (uma tampa) | abrir (uma torneira) | ligar (girando um botão) | girar (maçaneta da porta)}
\end{entry}

\begin{entry}{宁}{ning4}{5}{⼧}
  \definition{conj.}{mais\dots do que\dots, melhor\dots do que\dots}
  \seeref{宁}{ning2}
\end{entry}

\begin{entry}{宁可}{ning4ke3}{5,5}{⼧、⼝}
  \definition{conj.}{mais\dots do que\dots | melhor\dots do que\dots}
\end{entry}

\begin{entry}{宁可……也不……}{ning4ke3 ye3bu4}{5,5,3,4}{⼧、⼝、⼄、⼀}
  \definition{conj.}{em vez de\dots}
\end{entry}

\begin{entry}{宁可……也要……}{ning4ke3 ye3yao4}{5,5,3,9}{⼧、⼝、⼄、⾑}
  \definition{conj.}{mesmo que tenhamos que\dots nós iremos\dots}
\end{entry}

\begin{entry}{宁肯}{ning4ken3}{5,8}{⼧、⾁}
  \definition{conj.}{mais\dots do que\dots, melhor\dots do que\dots}
\end{entry}

\begin{entry}{宁愿}{ning4yuan4}{5,14}{⼧、⽕}
  \definition{conj.}{mais\dots do que\dots, melhor\dots do que\dots}
\end{entry}

\begin{entry}{牛}{niu2}{4}{⽜}[HSK 3][Kangxi 93]
  \definition*{s.}{sobrenome Niu}
  \definition{adj.}{muito capaz ou bom
teimoso; arrogante}
  \definition{clas.}{Newton (medida física de força)}
  \definition[头]{s.}{gado; boi | niu (nona das vinte e oito constelações em que a esfera celeste foi dividida, consistindo de seis estrelas, três em Áries e três em Sagitário)}
\end{entry}

\begin{entry}{牛顿}{niu2dun4}{4,10}{⽜、⾴}
  \definition*{s.}{Newton (nome) | newton (N, unidade de força do SI)}
\end{entry}

\begin{entry}{牛郎织女}{niu2lang2zhi1nv3}{4,8,8,3}{⽜、⾢、⽷、⼥}
  \definition*{s.}{Vaqueiro e Tecelã (personagens de contos folclóricos) | amantes separados | Altair e Vega (estrelas)}
\end{entry}

\begin{entry}{牛奶}{niu2nai3}{4,5}{⽜、⼥}[HSK 1]
  \definition[瓶,杯]{s.}{leite de vaca}
\end{entry}

\begin{entry}{牛人}{niu2ren2}{4,2}{⽜、⼈}
  \definition{s.}{(coloquial) o cara | verdadeiro especialista | \emph{badass}}
\end{entry}

\begin{entry}{牛肉}{niu2rou4}{4,6}{⽜、⾁}
  \definition{s.}{carne de vaca | bife}
\end{entry}

\begin{entry}{牛仔裤}{niu2zai3ku4}{4,5,12}{⽜、⼈、⾐}
  \definition[条]{s.}{calça de ganga, jeans}
\end{entry}

\begin{entry}{农村}{nong2cun1}{6,7}{⼍、⽊}[HSK 3]
  \definition[个]{s.}{aldeia; campo; área rural}
\end{entry}

\begin{entry}{农民}{nong2min2}{6,5}{⼍、⽒}[HSK 3]
  \definition[个,位]{s.}{fazendeiro; camponês; campesinato}
\end{entry}

\begin{entry}{农业}{nong2ye4}{6,5}{⼍、⼀}[HSK 3]
  \definition{s.}{agricultura; lavoura}
\end{entry}

\begin{entry}{浓}{nong2}{9}{⽔}[HSK 4]
  \definition{adj.}{denso; espesso; concentrado; um líquido ou gás que contém mais de um determinado ingrediente | grande; forte; profundo (de grau ou extensão) | profundo; (algumas cores) escuro}
\end{entry}

\begin{entry}{弄}{nong4}{7}{⼶}[HSK 2]
  \definition{s.}{beco | viela | travessa}
  \seeref{弄}{long4}
\end{entry}

\begin{entry}{努力}{nu3li4}{7,2}{⼒、⼒}[HSK 2]
  \definition{adj.}{diligente | aplicado}
  \definition{s.}{esforçar-se | se esforçar}
\end{entry}

\begin{entry}{怒骂}{nu4ma4}{9,9}{⼼、⾺}
  \definition{v.}{praguejar de raiva}
\end{entry}

\begin{entry}{暖}{nuan3}{13}{⽇}
  \definition{adj.}{quente}
  \definition{v.}{esquentar}
\end{entry}

\begin{entry}{暖和}{nuan3huo5}{13,8}{⽇、⼝}[HSK 3]
  \definition{adj.}{morno; agradável e quente}
  \definition{v.}{aquecer}
\end{entry}

\begin{entry}{暖气}{nuan3qi4}{13,4}{⽇、⽓}[HSK 4]
  \definition[个]{s.}{aquecedor; aquecimento; aquecimento central}
\end{entry}

\begin{entry}{那}{nuo2}{6}{⾢}
  \definition*{s.}{sobrenome Nuo}
\end{entry}

\begin{entry}{诺贝尔奖}{nuo4bei4'er3 jiang3}{10,4,5,9}{⾔、⾙、⼩、⼤}
  \definition*{s.}{Prêmio Nobel}
\end{entry}

\begin{entry}{诺奖}{nuo4jiang3}{10,9}{⾔、⼤}
  \definition*{s.}{Prêmio Nobel, abreviação de 诺贝尔奖}
  \seeref{诺贝尔奖}{nuo4bei4'er3 jiang3}
\end{entry}

\begin{entry}{女}{nv3}{3}{⼥}[HSK 1][Kangxi 38]
  \definition{adj.}{feminino}
\end{entry}

\begin{entry}{女儿}{nv3'er2}{3,2}{⼥、⼉}[HSK 1]
  \definition{s.}{filha}
  \seealsoref{儿子}{er2zi5}
\end{entry}

\begin{entry}{女孩}{nv3hai2}{3,9}{⼥、⼦}
  \definition{s.}{menina | garota}
\end{entry}

\begin{entry}{女孩儿}{nv3hai2r5}{3,9,2}{⼥、⼦、⼉}[HSK 1]
\end{entry}

\begin{entry}{女朋友}{nv3peng2you5}{3,8,4}{⼥、⽉、⼜}[HSK 1]
  \definition{s.}{namorada}
\end{entry}

\begin{entry}{女人}{nv3ren2}{3,2}{⼥、⼈}[HSK 1]
  \definition[个,位]{s.}{mulher}
\end{entry}

\begin{entry}{女生}{nv3sheng1}{3,5}{⼥、⽣}[HSK 1]
  \definition[个]{s.}{aluna | estudante so sexo feminino}
\end{entry}

\begin{entry}{女士}{nv3shi4}{3,3}{⼥、⼠}[HSK 4]
  \definition{pron.}{Sra.; Senhorita; Senhora; título honorífico para mulheres (agora usado em contextos diplomáticos)}
  \definition[位,个]{s.}{senhora; madame}
\end{entry}

\begin{entry}{女王}{nv3wang2}{3,4}{⼥、⽟}
  \definition{s.}{rainha}
\end{entry}

\begin{entry}{女婿}{nv3xu5}{3,12}{⼥、⼥}
  \definition{s.}{marido da filha}
\end{entry}

\begin{entry}{女子}{nv3 zi3}{3,3}{⼥、⼦}[HSK 3]
  \definition[位]{s.}{mulher; feminino}
\end{entry}

%%%%% EOF %%%%%


%%%
%%% O
%%%
%\section*{O}
\addcontentsline{toc}{section}{O}

\begin{verbete}{欧}{ou1}{8}
  \significado*{s.}{Europa, abreviação de~欧洲; sobrenome Ou}
  \veja{欧洲}{ou1zhou1}
\end{verbete}

\begin{verbete}{欧盟}{ou1meng2}{8;13}
  \significado*{s.}{Uniáo Europeia}
\end{verbete}

\begin{verbete}{欧洲}{ou1zhou1}{8;9}
  \significado*{s.}{Europa}
  \veja{欧}{ou1}
\end{verbete}

\begin{verbete}{欧洲共同体}{ou1zhou1 gong4tong2ti3}{8;9;6;6;7}
  \significado*{s.}{Comunidade Europeia}
\end{verbete}

\begin{verbete}{欧洲人}{ou1zhou1ren2}{8;9;2}
  \significado{s.}{europeu; nascido na Europa}
\end{verbete}

\begin{verbete}{偶然}{ou3ran2}{11;12}
  \significado{adv.}{por acaso; fortuitamente}
\end{verbete}

%%%%% EOF %%%%%

%%%
%%% P
%%%
\section*{P}
\addcontentsline{toc}{section}{P}
\begin{multicols}{2}
\entry{爬}{v.}{pa2}{escalar; trepar}

\entry{怕}{v.}{pa4}{ter medo de}

\entry{排球}{n.}{pai2qiu2}{voleibol}

\entry{盘}{p.c.}{pan2}{para cassete, video-cassete}

\entry{胖}{adj.}{pang4}{gordo}

\entry{跑步}{v.}{pao3bu4}{correr}

\entry{配}{v.}{pei3}{combinar}

\entry{朋友}{n.}{peng2you0}{amigo; amiga|namorado; namorada}

\entry{啤酒}{n.}{pi2jiu3}{cerveja}
\entry{啤酒馆}{n.}{pi2jiu3guan3}{cervejaria}

\entry{漂亮}{adj.}{piao4liang0}{linda|bonito;lindo (para objetos inanimados)}

\entry{瓶}{n.}{ping2}{garrafa}
\entry{瓶}{p.c.}{ping2}{palavra classificadora, garrafa}

\entry{平时}{p.t.}{ping2shi2}{normalmente; numa época normal}

\entry{苹果}{n.}{ping2guo3}{maçã}

\entry{葡汉词典}{n.}{pu2han4ci2dian3}{dicionário português-chinês}
\entry{葡萄牙}{n.}{Pu2tao2ya2}{Portugal}
\entry{葡萄牙语}{n.}{Pu2tao2ya2yu3}{português; língua portuguesa}
\entry{葡语}{n.}{Pu2yu3}{português; língua portuguesa}
\entry{葡文}{n.}{Pu2wen2}{português; língua portuguesa}

\entry{普通话}{n.}{pu3tong1hua4}{mandarim (lit. ``linguagem comum'')}

\entry{便宜}{adj.}{pian2yi0}{barato}

\entry{乒乓球}{n.}{ping1pang1qui2}{tênis de mesa; ping-pong}

\end{multicols}

%%%
%%% Q
%%%

\section*{Q}\addcontentsline{toc}{section}{Q}

\begin{entry}{七}{qi1}{2}{⼀}[HSK 1]
  \definition{num.}{sete; 7}
\end{entry}

\begin{entry}{七夕}{qi1xi1}{2,3}{⼀、⼣}
  \definition*{s.}{Dia dos Namorados Chinês, quando o vaqueiro e a tecelã (牛郎织女) têm permissão para se reunirem anualmente | Festival das Meninas | Festival Duplo Sete, noite do sétimo mês lunar}
  \seeref{牛郎织女}{niu2lang2zhi1nv3}
\end{entry}

\begin{entry}{妻子}{qi1zi3}{8,3}{⼥、⼦}
  \definition{s.}{esposa e filhos; (chinês antigo) refere-se a esposas, filhos e filhas}
  \seeref{妻子}{qi1zi5}
\end{entry}

\begin{entry}{妻子}{qi1zi5}{8,3}{⼥、⼦}[HSK 4]
  \definition{s.}{esposa (não é usado como um termo carinhoso)}
  \seeref{妻子}{qi1zi3}
\end{entry}

\begin{entry}{期}{qi1}{12}{⽉}[HSK 3]
  \definition{clas.}{questão; número; termo}
  \definition{s.}{tempo designado (programado) | um período de tempo; fase; estágio}
  \definition{v.}{marcar uma consulta | esperar; supor; imaginar}
\end{entry}

\begin{entry}{期待}{qi1dai4}{12,9}{⽉、⼻}[HSK 4]
  \definition{v.}{aguardar; esperar; aguardar ansiosamente; ter em mente a realização de um determinado fim ou a ocorrência de uma determinada situação}
\end{entry}

\begin{entry}{期间}{qi1jian1}{12,7}{⽉、⾨}[HSK 4]
  \definition{s.}{prazo; tempo; período}
\end{entry}

\begin{entry}{期末}{qi1 mo4}{12,5}{⽉、⽊}[HSK 4]
  \definition{s.}{terminal; final do prazo; fim do período}
\end{entry}

\begin{entry}{期望}{qi1wang4}{12,11}{⽉、⽉}[HSK 5]
  \definition{s.}{esperança; expectativa}
  \definition{v.}{esperar; ter esperança}
\end{entry}

\begin{entry}{期限}{qi1xian4}{12,8}{⽉、⾩}[HSK 4]
  \definition{s.}{prazo; limite de tempo; tempo alocado; período de tempo limitado, também o limite final do limite de tempo}
\end{entry}

\begin{entry}{期中}{qi1 zhong1}{12,4}{⽉、⼁}[HSK 4]
  \definition{adj.}{provisório; interino; intermediário}
\end{entry}

\begin{entry}{齐}{qi2}{6}{⿑}[HSK 3][Kangxi 210]
  \definition*{s.}{sobrenome Qi | Qi, um estado da Dinastia Zhou | Dinastia Qi do Sul (479-502), uma das Dinastias do Sul | Dinastia Qi do Norte (550-577), uma das Dinastias do Norte}
  \definition{adj.}{arrumado; uniforme; regular | semelhante; similar | tudo pronto; todos presentes}
  \definition{adv.}{juntos; simultaneamente}
  \definition{prep.}{ao longo de; junto a; paralelo a}
  \definition{v.}{atingir a altura de; em um nível com; estar nivelado com; no mesmo plano com}
\end{entry}

\begin{entry}{齐全}{qi2quan2}{6,6}{⿑、⼊}[HSK 5]
  \definition{adj.}{completo; tudo pronto}
\end{entry}

\begin{entry}{其}{qi2}{8}{⼋}[HSK 5]
  \definition*{s.}{sobrenome Qi}
  \definition{adv.}{fazer uma suposição ou uma réplica | expressar comando, ordem}
  \definition{pron.}{dele (dela, deles, delas) | ele, ela, isso, eles; elas | isso; tal | isso (referindo-se a nenhuma pessoa ou coisa específica)}
  \definition{suf.}{sufixo de palavra, anexado ao advérbio}
\end{entry}

\begin{entry}{其次}{qi2ci4}{8,6}{⼋、⽋}[HSK 3]
  \definition{adj.}{secundário}
  \definition{conj.}{próximo; então; em segundo lugar}
\end{entry}

\begin{entry}{其实}{qi2shi2}{8,8}{⼋、⼧}[HSK 3]
  \definition{adv.}{na verdade; na realidade; de fato}
\end{entry}

\begin{entry}{其他}{qi2ta1}{8,5}{⼋、⼈}[HSK 2]
  \definition{pron.}{todos os outro(s) | o resto}
\end{entry}

\begin{entry}{其余}{qi2yu2}{8,7}{⼋、⼈}[HSK 4]
  \definition{pron.}{o restante; os outros}
\end{entry}

\begin{entry}{其中}{qi2zhong1}{8,4}{⼋、⼁}[HSK 2]
  \definition{pron.}{dentro | entre (o qual, eles, etc.) | em (o qual, isso, etc.)}
\end{entry}

\begin{entry}{奇怪}{qi2guai4}{8,8}{⼤、⼼}[HSK 3]
  \definition{adj.}{estranho; esquisito}
  \definition{v.}{ficar perplexo; maravilhar-se; sentir-se surpreso}
\end{entry}

\begin{entry}{奇迹}{qi2ji4}{8,9}{⼤、⾡}
  \definition{adj.}{milagroso}
  \definition{s.}{milagre}
\end{entry}

\begin{entry}{骑}{qi2}{11}{⾺}[HSK 2]
  \definition{clas.}{para cavalos de sela}
  \definition{v.}{andar (cavalo, bicicleta, etc.) | sentar-se montado | montar}
\end{entry}

\begin{entry}{骑车}{qi2 che1}{11,4}{⾺、⾞}[HSK 2]
  \definition{v.}{andar de bicicleta | pedalar}
\end{entry}

\begin{entry}{旗}{qi2}{14}{⽅}
  \definition[面]{s.}{bandeira}
\end{entry}

\begin{entry}{企业}{qi3ye4}{6,5}{⼈、⼀}[HSK 4]
  \definition[家,个]{s.}{empresa; estabelecimento; empreendimento; negócio; setores envolvidos em atividades econômicas como produção, transporte, comércio, etc., como fábricas, minas, ferrovias, empresas comerciais, etc.}
\end{entry}

\begin{entry}{岂}{qi3}{6}{⼭}
  \definition*{s.}{sobrenome Qi}
  \definition{adv.}{expressa uma pergunta retórica, equivalente a ``哪里'', ``怎么'' e ``难道''}
  \seealsoref{哪里}{na3 li3}
  \seealsoref{难道}{nan2dao4}
  \seealsoref{怎么}{zen3me5}
\end{entry}

\begin{entry}{岂有此理}{qi3you3ci3li3}{6,6,6,11}{⼭、⽉、⽌、⽟}
  \definition{interj.}{Que exorbitante! | Absurdo! | Como isso pode ser assim? | Ridículo!}
\end{entry}

\begin{entry}{启动}{qi3 dong4}{7,6}{⼝、⼒}[HSK 5]
  \definition{v.}{ligar (uma máquina); acionar; ligar máquinas, equipamentos elétricos, etc., para começar a trabalhar | entrar em vigor; começar a vigorar e a ser implementados planos, projetos, documentos jurídicos, etc.}
\end{entry}

\begin{entry}{启发}{qi3fa1}{7,5}{⼝、⼜}[HSK 5]
  \definition{s.}{iluminação; esclarecimento; fenômenos e princípios que levam as pessoas a refletir e a abrir suas mentes}
  \definition{v.}{despertar; inspirar; esclarecer; orientar, fazer com que compreendam}
\end{entry}

\begin{entry}{启事}{qi3shi4}{7,8}{⼝、⼅}[HSK 5]
  \definition{s.}{aviso; anúncio; texto publicado em jornais ou afixado em paredes com o objetivo de divulgar publicamente algo}
\end{entry}

\begin{entry}{起}{qi3}{10}{⾛}[HSK 1]
  \definition*{s.}{sobrenome Qi}
  \definition{clas.}{caso; instância | lote; grupo}
  \definition{v.}{levantar | levantar-se | extrair| remover | puxar | aparecer | crescer | construir | configurar | começar | iniciar}
\end{entry}

\begin{entry}{起床}{qi3 chuang2}{10,7}{⾛、⼴}[HSK 1]
  \definition{v.+compl.}{sair da cama | levantar-se}
\end{entry}

\begin{entry}{起到}{qi3 dao4}{10,8}{⾛、⼑}[HSK 5]
  \definition{v.}{ter (um efeito motivador, etc.); desempenhar (um papel estabilizador, etc.)}
\end{entry}

\begin{entry}{起飞}{qi3fei1}{10,3}{⾛、⾶}[HSK 2]
  \definition{v.}{decolar}
\end{entry}

\begin{entry}{起来}{qi3 lai2}{10,7}{⾛、⽊}[HSK 1]
  \definition{v.+compl.}{levantar-se}
\end{entry}

\begin{entry}{起码}{qi3ma3}{10,8}{⾛、⽯}[HSK 5]
  \definition{adj.}{mínimo; elementar; rudimentar}
  \definition{adv.}{mínimamente; pelo menos;}
\end{entry}

\begin{entry}{起跳}{qi3tiao4}{10,13}{⾛、⾜}
  \definition{v.}{(atletismo) decolar (no início de um salto) | (de preço, salário, etc.) começar (de um determinado nível)}
\end{entry}

\begin{entry}{气}{qi4}{4}{⽓}[HSK 2][Kangxi 84]
  \definition[口]{s.}{gás | ar | respiração | clima | cheiro | odor | espírito | moral | ares | maneira | estilo | insulto | intimidação | energia vital | energia da vida}
  \definition{v.}{ficar bravo | ficar enfurecido | irritar | enfurecer}
\end{entry}

\begin{entry}{气候}{qi4hou4}{4,10}{⽓、⼈}[HSK 3]
  \definition[种]{s.}{clima; tempo
tendência; situação
resultado; efeito; conquista}
\end{entry}

\begin{entry}{气球}{qi4qiu2}{4,11}{⽓、⽟}[HSK 4]
  \definition{s.}{balão; bolas feitas de borracha, plástico, etc., que podem ser aumentadas soprando ar nelas e podem ser usadas como brinquedos, decorações ou meios de transporte}
\end{entry}

\begin{entry}{气体}{qi4 ti3}{4,7}{⽓、⼈}[HSK 5]
  \definition[种]{s.}{gás; não têm forma nem volume definidos e podem fluir.; o ar, o oxigênio, o gás metano e outros são gases}
\end{entry}

\begin{entry}{气温}{qi4 wen1}{4,12}{⽓、⽔}[HSK 2]
  \definition[个]{s.}{temperatura do ar}
\end{entry}

\begin{entry}{气象}{qi4xiang4}{4,11}{⽓、⾗}[HSK 5]
  \definition[个]{s.}{fenômenos meteorológicos; condições e fenômenos atmosféricos, como vento, relâmpagos, trovões, geadas, neve, etc. | meteorologia | situação; atmosfera; cena; circunstância | maneira imponente}
\end{entry}

\begin{entry}{气质}{qi4zhi4}{4,8}{⽓、⾙}
  \definition{s.}{traços de personalidade, temperamento, disposição | aura, ar, sentimento, \emph{vibe} | refinamento, sofisticação, classe}
\end{entry}

\begin{entry}{汽车}{qi4che1}{7,4}{⽔、⾞}[HSK 1]
  \definition[辆]{s.}{automóvel | carro | veículo motorizado}
\end{entry}

\begin{entry}{汽水}{qi4 shui3}{7,4}{⽔、⽔}[HSK 4]
  \definition[罐,瓶]{s.}{refrigerante; refrigerante gaseificado; bebida refrescante, feita com a pressão de dióxido de carbono para dissolver na água e adicionar açúcar, suco de frutas, especiarias etc.}
\end{entry}

\begin{entry}{汽油}{qi4you2}{7,8}{⽔、⽔}[HSK 4]
  \definition{s.}{gasolina; mistura líquida de hidrocarbonetos com volatilidade e combustibilidade, que é usada como combustível a partir do fracionamento ou craqueamento do petróleo}
\end{entry}

\begin{entry}{器}{qi4}{16}{⼝}
  \definition[台]{s.}{dispositivo | ferramenta | utensílio}
\end{entry}

\begin{entry}{器官}{qi4guan1}{16,8}{⼝、⼧}[HSK 4]
  \definition[个]{s.}{órgão; aparelho; parte de um organismo que consiste em vários tipos de tecidos celulares que podem desempenhar uma função fisiológica separada}
\end{entry}

\begin{entry}{卡}{qia3}{5}{⼘}
  \definition[张]{s.}{grampo | prendedor}
  \definition{s.}{posto de controle}
  \definition{v.}{cunhar | ficar preso | encravar}
  \seeref{卡}{ka3}
\end{entry}

\begin{entry}{恰}{qia4}{9}{⼼}
  \definition{adv.}{exatamente | apenas}
\end{entry}

\begin{entry}{恰到好处}{qia4dao4hao3chu4}{9,8,6,5}{⼼、⼑、⼥、⼡}
  \definition{expr.}{é simplesmente perfeito | é simplesmente correto}
\end{entry}

\begin{entry}{恰好}{qia4hao3}{9,6}{⼼、⼥}
  \definition{adv.}{certo | por sorte | ao que parece | por sorte coincidência}
\end{entry}

\begin{entry}{千}{qian1}{3}{⼗}[HSK 2]
  \definition{num.}{mil; 1.000; 1000}
\end{entry}

\begin{entry}{千古}{qian1gu3}{3,5}{⼗、⼝}
  \definition{adv.}{por toda a eternidade | em todas as idades}
  \definition{s.}{eternidade (usada em um dístico elegíaco, coroa de flores, etc., dedicada aos mortos)}
\end{entry}

\begin{entry}{千克}{qian1 ke4}{3,7}{⼗、⼗}[HSK 2]
  \definition{clas.}{kg | quilo | quilograma}
\end{entry}

\begin{entry}{千年}{qian1nian2}{3,6}{⼗、⼲}
  \definition{s.}{milênio}
\end{entry}

\begin{entry}{千千万万}{qian1qian1wan4wan4}{3,3,3,3}{⼗、⼗、⼀、⼀}
  \definition{num.}{inumerável | números incontáveis | milhares e milhares}
\end{entry}

\begin{entry}{千万}{qian1wan4}{3,3}{⼗、⼀}[HSK 3]
  \definition{adv.}{(usado para indicar desejos fortes) por todos os meios; sob quaisquer circunstâncias}
  \definition{num.}{dez milhões; milhões e milhões}
\end{entry}

\begin{entry}{签}{qian1}{13}{⽵}[HSK 5]
  \definition{s.}{tiras de bambu usadas para adivinhação ou sorteio; pPequenas tiras de bambu ou varas finas com caracteres e símbolos gravados, usadas para adivinhação, jogos de azar ou como fichas para contagem, etc. | etiqueta; adesivo; pequena tira usada como marca | um pedaço fino e pontiagudo de bambu ou madeira; pequeno bastão pontiagudo}
  \definition{v.}{assinar; autografar; escrever o nome, palavras ou fazer marcas em documentos ou recibos | fazer comentários breves em um documento; escrever brevemente (pontos principais ou opiniões) | (em costura) alinhavar; costura grosseira}
\end{entry}

\begin{entry}{签订}{qian1 ding4}{13,4}{⽵、⾔}[HSK 5]
  \definition{v.}{concluir e assinar (um tratado, etc.)}
\end{entry}

\begin{entry}{签名}{qian1 ming2}{13,6}{⽵、⼝}[HSK 5]
  \definition[个,次]{s.}{assinatura; autógrafo}
  \definition{v.+compl.}{assinar o próprio nome; autografar; escrever seu nome para indicar concordância, apoio ou homenagem, etc.}
\end{entry}

\begin{entry}{签约}{qian1 yue1}{13,6}{⽵、⽷}[HSK 5]
  \definition{v.}{assinar um contrato; assinar contratos e tratados, frequentemente utilizado no trabalho e em cooperações comerciais}
\end{entry}

\begin{entry}{签证}{qian1zheng4}{13,7}{⽵、⾔}[HSK 5]
  \definition[张,个]{s.}{visto; visto de entrada em um país}
\end{entry}

\begin{entry}{签字}{qian1 zi4}{13,6}{⽵、⼦}[HSK 5]
  \definition{v.}{assinar; colocar a assinatura; escrever seu nome à mão em documentos, recibos, etc., para demonstrar responsabilidade}
\end{entry}

\begin{entry}{前}{qian2}{9}{⼑}[HSK 1]
  \definition{adv.}{frente; em frente de | A.C. (Antes de~Cristo)}[前293年  (293 a.C.)]
  \seealsoref{公元}{gong1yuan2}
\end{entry}

\begin{entry}{前边}{qian2bian5}{9,5}{⼑、⾡}[HSK 1]
  \definition{adv.}{à frente | da frente}
\end{entry}

\begin{entry}{前后}{qian2 hou4}{9,6}{⼑、⼝}[HSK 3]
  \definition{s.}{em volta; sobre | do início ao fim | frente e verso}
\end{entry}

\begin{entry}{前进}{qian2 jin4}{9,7}{⼑、⾡}[HSK 3]
  \definition{v.}{marchar; avançar; para ir em frente; seguir em frente}
\end{entry}

\begin{entry}{前景}{qian2jing3}{9,12}{⼑、⽇}[HSK 5]
  \definition{s.}{primeiro plano (de uma vista, imagem, foto, etc.); as imagens que parecem mais próximas do espectador em pinturas, palcos e telas | vista; perspectiva; prospecto; ponto de vista; situações que podem ocorrer no trabalho, na carreira, etc.}
\end{entry}

\begin{entry}{前面}{qian2mian4}{9,9}{⼑、⾯}[HSK 3]
  \definition{s.}{frente | parte anterior; acima}
\end{entry}

\begin{entry}{前年}{qian2 nian2}{9,6}{⼑、⼲}[HSK 2]
  \definition{adv.}{há dois anos}
\end{entry}

\begin{entry}{前提}{qian2ti2}{9,12}{⼑、⼿}[HSK 5]
  \definition[个,项]{s.}{premissa; pressuposto | pré-requisito; pressuposição; condições prévias para que algo aconteça ou se desenvolva}
\end{entry}

\begin{entry}{前天}{qian2tian1}{9,4}{⼑、⼤}[HSK 1]
  \definition{adv.}{anteontem}
\end{entry}

\begin{entry}{前头}{qian2 tou5}{9,5}{⼑、⼤}[HSK 4]
  \definition{s.}{à frente; na frente; adiante}
\end{entry}

\begin{entry}{前途}{qian2tu2}{9,10}{⼑、⾡}[HSK 4]
  \definition[片,段,种]{s.}{futuro; perspectiva; prospecto; originalmente, refere-se à jornada à frente, mas, metaforicamente, refere-se ao futuro.}
\end{entry}

\begin{entry}{前往}{qian2 wang3}{9,8}{⼑、⼻}[HSK 3]
  \definition{v.}{ir para; prosseguir para; partir para}
\end{entry}

\begin{entry}{钱}{qian2}{10}{⾦}[HSK 1]
  \definition*{s.}{sobrenome Qian}
  \definition[笔]{s.}{moeda | dinheiro}
\end{entry}

\begin{entry}{钱包}{qian2bao1}{10,5}{⾦、⼓}[HSK 1]
  \definition{s.}{carteira | bolsa}
\end{entry}

\begin{entry}{潜在}{qian2zai4}{15,6}{⽔、⼟}
  \definition{adj.}{oculto | latente}
  \definition{s.}{potencial}
\end{entry}

\begin{entry}{浅}{qian3}{8}{⽔}[HSK 4]
  \definition{adj.}{raso; superficial;  (em oposição a ``深'') | fácil; simples; redação, conteúdo, etc. simples e fáceis de entender | superficial; não é profundo em aprendizado, percepção e sabedoria | não próximo; não íntimo; sentimentos não profundos | (cor) claro; pálido;  cor pouco intensa; leve |experiência breve; duração de tempo breve | baixo grau; peso leve; nível baixo}
  \seeref{浅}{jian1}
  \seealsoref{深}{shen1}
\end{entry}

\begin{entry}{欠}{qian4}{4}{⽋}[HSK 5]
  \definition{v.}{bocejar | levantar ligeiramente (uma parte do corpo) | estar em dívida; estar atrasado com; não devolver o que pediu emprestado a outra pessoa, ou não dar o que deveria ter dado a outra pessoa | faltar; não ser suficiente}
\end{entry}

\begin{entry}{抢}{qiang1}{7}{⼿}
  \definition{prep.}{contra; direção relativa inversa}
  \definition{v.}{bater; tocar}
  \seeref{抢}{qiang3}
\end{entry}

\begin{entry}{枪}{qiang1}{8}{⽊}[HSK 5]
  \definition*{s.}{sobrenome Qiang}
  \definition{s.}{lança | arma; rifle; arma de fogo | uma coisa em forma de arma | enxada; ferramenta para cavar a terra}
  \definition{v.}{escrever artigos ou responder perguntas para outras pessoas}
\end{entry}

\begin{entry}{将}{qiang1}{9}{⼨}
  \definition{v.}{pedir; apelar para}
  \seeref{将}{jiang1}
  \seeref{将}{jiang4}
\end{entry}

\begin{entry}{强}{qiang2}{12}{⼸}[HSK 3]
  \definition*{s.}{sobrenome Qiang}
  \definition{adj.}{forte; poderoso | melhor; superior | mais; extra; adicional; um pouco mais que | resoluto; firme | decidido; resolvido | violento; impetuoso | alto padrão}
  \definition{v.}{fortalecer; tornar forte}
  \seeref{强}{jiang4}
  \seeref{强}{qiang3}
\end{entry}

\begin{entry}{强大}{qiang2 da4}{12,3}{⼸、⼤}[HSK 3]
  \definition{adj.}{forte; poderoso; potente; possante}
\end{entry}

\begin{entry}{强调}{qiang2diao4}{12,10}{⼸、⾔}[HSK 3]
  \definition{v.}{salientar; sublinhar; enfatizar; dar ênfase a; vincar}
\end{entry}

\begin{entry}{强度}{qiang2 du4}{12,9}{⼸、⼴}[HSK 5]
  \definition[个]{s.}{intensidade; força | magnitude; rigor; avidez}
\end{entry}

\begin{entry}{强烈}{qiang2lie4}{12,10}{⼸、⽕}[HSK 3]
  \definition{adj.}{forte; intenso | violento; impetuoso | afiado; marcante}
\end{entry}

\begin{entry}{墙}{qiang2}{14}{⼟}[HSK 2]
  \definition[面,堵]{s.}{parede}
  \definition{v.}{(gíria) bloquear (um website) (usado geralmente na voz passiva: 被墙)}
\end{entry}

\begin{entry}{墙壁}{qiang2 bi4}{14,16}{⼟、⼟}[HSK 5]
  \definition[堵]{s.}{parede; barreira ou perímetro construído com tijolos, pedras ou terra}
\end{entry}

\begin{entry}{墙纸}{qiang2zhi3}{14,7}{⼟、⽷}
  \definition{s.}{papel de parede}
\end{entry}

\begin{entry}{抢}{qiang3}{7}{⼿}[HSK 5]
  \definition{v.}{roubar; saquear | agarrar; apanhar; arrebatar | disputar; lutar por; ser o primeiro; competir para ser o primeiro | correr; apressar-se; fazer uma incursão | raspar; arranhar; raspar ou esfregar uma camada da superfície de um objeto}
  \seeref{抢}{qiang1}
\end{entry}

\begin{entry}{抢救}{qiang3jiu4}{7,11}{⼿、⽁}[HSK 5]
  \definition{v.}{salvar; resgatar; prestar de socorro ou assistência rápidos em situações de emergência | salvar; tomar medidas rápidas para evitar ou minimizar perdas iminentes.}
\end{entry}

\begin{entry}{抢掠}{qiang3lve4}{7,11}{⼿、⼿}
  \definition{s.}{saque | pilhagem}
  \definition{v.}{saquear | pilhar}
\end{entry}

\begin{entry}{强}{qiang3}{12}{⼸}
  \definition{v.}{fazer um esforço; esforçar-se}
  \seeref{强}{jiang4}
  \seeref{强}{qiang2}
\end{entry}

\begin{entry}{强迫}{qiang3po4}{12,8}{⼸、⾡}[HSK 5]
  \definition{v.}{impelir; forçar; impor; compelir; aplicar pessão para obedecer}
\end{entry}

\begin{entry}{悄悄}{qiao1qiao1}{10,10}{⼼、⼼}[HSK 5]
  \definition{adv.}{silenciosamente; em silêncio; aos sussuros; sem som ou em voz baixa; com o mínimo de ruído possível}
\end{entry}

\begin{entry}{敲}{qiao1}{14}{⽁}[HSK 5]
  \definition{v.}{bater; dar uma pancada; golpear | explorar alguém; cobrar a mais; extorquir; chantagear | lembrar; criticar; alertar; advertir}
\end{entry}

\begin{entry}{敲门}{qiao1 men2}{14,3}{⽁、⾨}[HSK 5]
  \definition{v.}{bater na porta}
\end{entry}

\begin{entry}{桥}{qiao2}{10}{⽊}[HSK 3]
  \definition*{s.}{sobrenome Qiao}
  \definition[座]{s.}{ponte}
\end{entry}

\begin{entry}{瞧}{qiao2}{17}{⽬}[HSK 5]
  \definition{v.}{ver; olhar | tratar; diagnosticar e tratar | ver; visitar; fazer uma visita}
\end{entry}

\begin{entry}{巧}{qiao3}{5}{⼯}[HSK 3]
  \definition{adj.}{habilidoso; engenhoso; esperto | oportuno; coincidente; fortuito | astuto; enganoso; enganador; traiçoeiro; ardiloso}
\end{entry}

\begin{entry}{巧合}{qiao3he2}{5,6}{⼯、⼝}
  \definition{s.}{coincidência}
  \definition{v.}{coincidir}
\end{entry}

\begin{entry}{巧克力}{qiao3ke4li4}{5,7,2}{⼯、⼗、⼒}[HSK 4]
  \definition[块]{s.}{(empréstimo linguístico) chocolate}
\end{entry}

\begin{entry}{切}{qie1}{4}{⼑}[HSK 4]
  \definition{v.}{cortar; fatiar; separar itens com uma faca | cortar ou romper; truncar | (geometria) geometria, refere-se a quando uma linha, círculo ou superfície intercepta um círculo, arco ou esfera em apenas um ponto}
  \seeref{切}{qie4}
\end{entry}

\begin{entry}{切割}{qie1ge1}{4,12}{⼑、⼑}
  \definition{v.}{cortar}
\end{entry}

\begin{entry}{茄子}{qie2zi5}{8,3}{⾋、⼦}
  \definition{s.}{berinjela chinesa | ``xis'' fonético (ao ser fotografado), equivale ao ``diga xis''}
\end{entry}

\begin{entry}{且}{qie3}{5}{⼀}
  \definition*{s.}{sobrenome Qie}
  \definition{adv.}{apenas; por enquanto | por um longo tempo}
  \definition{conj.}{mesmo; até; até mesmo | ambos\dots e\dots}
\end{entry}

\begin{entry}{切}{qie4}{4}{⼑}
  \definition{adj.}{ansioso; sério | duro; severo; rude; áspero}
  \definition{adv.}{com certeza; certamente}
  \definition{s.}{limiar; degrau}
  \definition{v.}{ser prático ou realista | ajustar-se ou corresponder | ser próximo ou íntimo | cortar algo em pedaços com uma faca | (medicina tradicional chinesa) tomar o pulso}
  \seeref{切}{qie1}
\end{entry}

\begin{entry}{亲}{qin1}{9}{⼇}[HSK 3]
  \definition{adj.}{parente próximo; relacionado por sangue; de ​​relação de sangue | querido; próximo; íntimo | em si mesmo; pessoalmente}
  \definition[位]{s.}{pais | parente | casal; casamento | noiva}
  \definition{v.}{beijar | (países, partidos, etc.) a favor de; apoiar; estar perto de}
  \seeref{亲}{qing4}
\end{entry}

\begin{entry}{亲爱}{qin1'ai4}{9,10}{⼇、⽖}[HSK 4]
  \definition{adj.}{querido; amado; termo carinhoso que expressa intimidade e afeto}
\end{entry}

\begin{entry}{亲密}{qin1mi4}{9,11}{⼇、⼧}[HSK 4]
  \definition{adj.}{próximo; íntimo; relacionamento afetuoso e próximo}
\end{entry}

\begin{entry}{亲切}{qin1qie4}{9,4}{⼇、⼑}[HSK 3]
  \definition{adj.}{gentil; cordial | próximo; íntimo}
\end{entry}

\begin{entry}{亲人}{qin1 ren2}{9,2}{⼇、⼈}[HSK 3]
  \definition{s.}{um membro da família; os pais, o cônjuge, os filhos, etc. | queridos; entes queridos; aqueles queridos para alguém}
\end{entry}

\begin{entry}{亲自}{qin1zi4}{9,6}{⼇、⾃}[HSK 3]
  \definition{adv.}{pessoalmente; em pessoa; si mesmo}
\end{entry}

\begin{entry}{侵略}{qin1lve4}{9,11}{⼈、⽥}
  \definition{s.}{invasão}
  \definition{v.}{invadir}
\end{entry}

\begin{entry}{芹菜}{qin2cai4}{7,11}{⾋、⾋}
  \definition{s.}{salsão}
\end{entry}

\begin{entry}{琴}{qin2}{12}{⽟}[HSK 5]
  \definition*{s.}{sobrenome Qin}
  \definition[架,台]{s.}{cítara; qin; guqin (um instrumento de cordas dedilhadas com sete cordas, em alguns aspectos semelhante à cítara)  | nome genérico para certos instrumentos musicais}
\end{entry}

\begin{entry}{琴键}{qin2jian4}{12,13}{⽟、⾦}
  \definition{s.}{tecla de piano}
\end{entry}

\begin{entry}{禽}{qin2}{12}{⽱}
  \definition*{s.}{sobrenome Qin}
  \definition[只]{s.}{aves; pássaros | termo genérico para aves e animais}
\end{entry}

\begin{entry}{勤奋}{qin2fen4}{13,8}{⼒、⼤}[HSK 5]
  \definition{adj.}{diligente; assíduo; trabalhador; descreve alguém que se esforça continuamente nos estudos ou no trabalho}
\end{entry}

\begin{entry}{擒获}{qin2huo4}{15,10}{⼿、⾋}
  \definition{v.}{apreender | capturar}
\end{entry}

\begin{entry}{青}{qing1}{8}{⾭}[HSK 5]
  \definition*{s.}{abreviação de Província de Qinghai | sobrenome Qing}
  \definition{adj.}{azul ou verde | preto | jovens (pessoas)}
  \definition{s.}{grama verde | colheitas jovens (não maduras) | tiras de bambu verde}
\end{entry}

\begin{entry}{青菜}{qing1cai4}{8,11}{⾭、⾋}
  \definition{s.}{verduras}
\end{entry}

\begin{entry}{青春}{qing1chun1}{8,9}{⾭、⽇}[HSK 4]
  \definition[个]{s.}{juventude; jovialidade}
\end{entry}

\begin{entry}{青椒}{qing1jiao1}{8,12}{⾭、⽊}
  \definition{s.}{pimenta verde}
\end{entry}

\begin{entry}{青年}{qing1 nian2}{8,6}{⾭、⼲}[HSK 2]
  \definition[个,名,位]{s.}{juventude | jovem}
\end{entry}

\begin{entry}{青年节}{qing1nian2jie2}{8,6,5}{⾭、⼲、⾋}
  \definition*{s.}{Dia da Juventude (4 de maio)}
\end{entry}

\begin{entry}{青少年}{qing1shao4nian2}{8,4,6}{⾭、⼩、⼲}[HSK 2]
  \definition[位,个]{s.}{adolescentes}
\end{entry}

\begin{entry}{青天}{qing1tian1}{8,4}{⾭、⼤}
  \definition{s.}{céu claro, limpo ou azul}
\end{entry}

\begin{entry}{青铜}{qing1tong2}{8,11}{⾭、⾦}
  \definition{s.}{bronze (liga de cobre, 銅, e estanho, 锡)}
\end{entry}

\begin{entry}{青蛙}{qing1wa1}{8,12}{⾭、⾍}
  \definition{adj.}{(gíria velha) cara feio}
  \definition[只]{s.}{sapo}
\end{entry}

\begin{entry}{青玉米}{qing1yu4mi3}{8,5,6}{⾭、⽟、⽶}
  \definition{s.}{milho verde}
\end{entry}

\begin{entry}{轻}{qing1}{9}{⾞}[HSK 2]
  \definition{adj.}{leve | pequeno em número, grau, etc. | não importante | relaxado}
  \definition{adv.}{suavemente | levemente | precipitadamente}
  \definition{v.}{menosprezar}
\end{entry}

\begin{entry}{轻松}{qing1song1}{9,8}{⾞、⽊}[HSK 4]
  \definition{adj.}{leve; relaxado; livre de fardos; não se sentir nervoso ou cansado}
  \definition{v.}{relaxar; levar as coisas menos a sério}
\end{entry}

\begin{entry}{轻易}{qing1yi4}{9,8}{⾞、⽇}[HSK 4]
  \definition{adj.}{fácil; simples}
  \definition{adv.}{facilmente; prontamente | facilmente; precipitadamente; indica que uma ação é realizada casualmente, geralmente usado em frases negativas}
\end{entry}

\begin{entry}{倾城}{qing1cheng2}{10,9}{⼈、⼟}
  \definition{adj.}{sedutora (mulher)}
  \definition{adv.}{de todo o lugar | vindo de todos os lugares}
  \definition{v.}{arruinar e derrubar o estado}
\end{entry}

\begin{entry}{清}{qing1}{11}{⽔}
  \definition*{s.}{sobrenome Qing}
  \definition{adj.}{claro | limpo (água, etc.) | tranquilo | quieto | puro | não corrompido | distinto}
  \definition{v.}{limpar | resolver (contas)}
\end{entry}

\begin{entry}{清唱}{qing1chang4}{11,11}{⽔、⼝}
  \definition{v.}{cantar à capela}
\end{entry}

\begin{entry}{清彻}{qing1che4}{11,7}{⽔、⼻}
  \variantof{清澈}
\end{entry}

\begin{entry}{清澈}{qing1che4}{11,15}{⽔、⽔}
  \definition{adj.}{claro | límpido}
\end{entry}

\begin{entry}{清晨}{qing1chen2}{11,11}{⽔、⽇}[HSK 5]
  \definition{s.}{matinal; manhã cedo; geralmente se refere ao período do amanhecer até logo após o nascer do sol}
\end{entry}

\begin{entry}{清楚}{qing1chu5}{11,13}{⽔、⽊}[HSK 2]
  \definition{adj.}{claro | límpido}
  \definition{v.}{ser claro sobre | entender completamente}
\end{entry}

\begin{entry}{清理}{qing1li3}{11,11}{⽔、⽟}[HSK 5]
  \definition{v.}{esclarecer; resolver; verificar; colocar em ordem; organizar tudo e jogar fora o que não for útil}
\end{entry}

\begin{entry}{清凉}{qing1liang2}{11,10}{⽔、⼎}
  \definition{adj.}{fresco | refrescante | (roupa) ousada, reveladora}
\end{entry}

\begin{entry}{清明节}{qing1ming2jie2}{11,8,5}{⽔、⽇、⾋}
  \definition*{s.}{Dia Qingming, Dia dos Finados (uma das 24~divisões do ano solar no calendário lunar chinês:~dia~4 ou 5~de~abril solar)}
\end{entry}

\begin{entry}{清爽}{qing1shuang3}{11,11}{⽔、⽘}
  \definition{adj.}{refrescante | relaxado}
\end{entry}

\begin{entry}{清晰}{qing1xi1}{11,12}{⽔、⽇}
  \definition{adj.}{claro | distinto}
\end{entry}

\begin{entry}{清醒}{qing1xing3}{11,16}{⽔、⾣}[HSK 4]
  \definition{adj.}{sóbrio; lúcido; totalmente acordado}
\end{entry}

\begin{entry}{蜻蜓}{qing1ting2}{14,12}{⾍、⾍}
  \definition{s.}{libélula}
\end{entry}

\begin{entry}{蜻蝏}{qing1ting2}{14,15}{⾍、⾍}
  \variantof{蜻蜓}
\end{entry}

\begin{entry}{情感}{qing2 gan3}{11,13}{⼼、⼼}[HSK 3]
  \definition[份]{s.}{emoção; sentimento | afeição; apego}
  \definition{v.}{mover-se (emocionalmente)}
\end{entry}

\begin{entry}{情节}{qing2jie2}{11,5}{⼼、⾋}[HSK 5]
  \definition{s.}{enredo; trama; desenrolar específico dos acontecimentos | circunstância; detalhes do crime ou erro | enredo; roteiro; refere-se especificamente ao processo de desenvolvimento e evolução dos conflitos e contradições em obras literárias narrativas}
\end{entry}

\begin{entry}{情景}{qing2jing3}{11,12}{⼼、⽇}[HSK 4]
  \definition[个]{s.}{cena; vista; circunstâncias}
\end{entry}

\begin{entry}{情况}{qing2kuang4}{11,7}{⼼、⼎}[HSK 3]
  \definition[种,个,些]{s.}{condição; situação; circunstâncias; estado das coisas | mudança notável}
\end{entry}

\begin{entry}{情形}{qing2xing2}{11,7}{⼼、⼺}[HSK 5]
  \definition[个]{s.}{situação; condição; circunstâncias; estado de coisas; a situação específica das coisas}
\end{entry}

\begin{entry}{情绪}{qing2xu4}{11,11}{⼼、⽷}
  \definition[种]{s.}{humor | estado da mente | mau humor}
\end{entry}

\begin{entry}{晴}{qing2}{12}{⽇}[HSK 2]
  \definition{adj.}{ensolarado | claro}
\end{entry}

\begin{entry}{晴朗}{qing2lang3}{12,10}{⽇、⽉}[HSK 5]
  \definition{adj.}{bom; claro; ensolarado; céu limpo e sem nuvens}
\end{entry}

\begin{entry}{晴天}{qing2 tian1}{12,4}{⽇、⼤}[HSK 2]
  \definition[个]{s.}{dia ensolarado}
\end{entry}

\begin{entry}{请}{qing3}{10}{⾔}[HSK 1]
  \definition{v.}{por favor (fazer alguma coisa) | perguntar | convidar | solicitar}
\end{entry}

\begin{entry}{请假}{qing3 jia4}{10,11}{⾔、⼈}[HSK 1]
  \definition{v.+compl.}{pedir licença para sair}
\end{entry}

\begin{entry}{请假条}{qing3jia4tiao2}{10,11,7}{⾔、⼈、⽊}
  \definition{s.}{pedido de licença de ausência (do trabalho ou da escola)}
\end{entry}

\begin{entry}{请教}{qing3jiao4}{10,11}{⾔、⽁}[HSK 3]
  \definition{v.}{consultar; pedir conselho}
\end{entry}

\begin{entry}{请进}{qing3 jin4}{10,7}{⾔、⾡}[HSK 1]
  \definition{v.}{por favor entre}
\end{entry}

\begin{entry}{请客}{qing3ke4}{10,9}{⾔、⼧}[HSK 2]
  \definition{v.+compl.}{entreter os convidados | dar um jantar | convidar para jantar}
\end{entry}

\begin{entry}{请求}{qing3qiu2}{10,7}{⾔、⽔}[HSK 2]
  \definition[个]{s.}{solicitação}
  \definition{v.}{solicitar | perguntar}
\end{entry}

\begin{entry}{请问}{qing3wen4}{10,6}{⾔、⾨}[HSK 1]
  \definition{expr.}{Com licença, posso perguntar\dots? (para perguntar por qualquer coisa)}
\end{entry}

\begin{entry}{请坐}{qing3 zuo4}{10,7}{⾔、⼟}[HSK 1]
  \definition{v.}{por favor, sente-se}
\end{entry}

\begin{entry}{庆祝}{qing4zhu4}{6,9}{⼴、⽰}[HSK 3]
  \definition{v.}{celebrar; comemorar; festejar}
\end{entry}

\begin{entry}{亲}{qing4}{9}{⼇}
  \definition{s.}{parentes por afinidade; parentes por casamento}
  \seeref{亲}{qin1}
\end{entry}

\begin{entry}{穷}{qiong2}{7}{⽳}[HSK 4]
  \definition{adj.}{remoto; isolado; de difícil acesso | pobre; atingido pela pobreza | situação difícil, sem saída}
  \definition{adv.}{completamente | extremamente}
  \definition{v.}{exaurir; esgotar; consmir | ir até o fim; perseguir completamente perseguido; sondar profundamente | gastar}
\end{entry}

\begin{entry}{穷人}{qiong2 ren2}{7,2}{⽳、⼈}[HSK 4]
  \definition{s.}{os pobres; pessoas pobres}
\end{entry}

\begin{entry}{丘陵}{qiu1ling2}{5,10}{⼀、⾩}
  \definition{s.}{colinas}
\end{entry}

\begin{entry}{秋}{qiu1}{9}{⽲}
  \definition*{s.}{sobrenome Qiu}
  \definition{s.}{outono | colheita}
\end{entry}

\begin{entry}{秋季}{qiu1 ji4}{9,8}{⽲、⼦}[HSK 4]
  \definition[个]{s.}{outono; terceiro trimestre do ano, segundo o costume chinês, refere-se ao período de três meses entre o outono e o inverno, também se refere aos sétimo, oitavo e nono meses do calendário lunar}
\end{entry}

\begin{entry}{秋天}{qiu1 tian1}{9,4}{⽲、⼤}[HSK 2]
  \definition[个]{s.}{outono}
\end{entry}

\begin{entry}{求}{qiu2}{7}{⽔}[HSK 2]
  \definition*{s.}{sobrenome Qiu}
  \definition{s.}{demanda}
  \definition{v.}{pedir | implorar | solicitar | suplicar | esforçar-se por | procurar | tentar}
\end{entry}

\begin{entry}{球}{qiu2}{11}{⽟}[HSK 1]
  \definition[个]{s.}{bola | esfera | globo}
  \definition[场]{s.}{jogo | partida de bola}
\end{entry}

\begin{entry}{球场}{qiu2 chang3}{11,6}{⽟、⼟}[HSK 2]
  \definition{s.}{um campo onde são jogados jogos de bola | tribunal | campo | curso | \emph{links}}
\end{entry}

\begin{entry}{球队}{qiu2 dui4}{11,4}{⽟、⾩}[HSK 2]
  \definition{s.}{time (basquete, futebol, etc.)}
\end{entry}

\begin{entry}{球迷}{qiu2mi2}{11,9}{⽟、⾡}[HSK 3]
  \definition[个]{s.}{fã (de esportes de bola)}
\end{entry}

\begin{entry}{球拍}{qiu2pai1}{11,8}{⽟、⼿}
  \definition{s.}{raquete}
\end{entry}

\begin{entry}{球鞋}{qiu2 xie2}{11,15}{⽟、⾰}[HSK 2]
  \definition{s.}{calçados esportivos | tênis | tênis de ginástica}
\end{entry}

\begin{entry}{区}{qu1}{4}{⼖}[HSK 3]
  \definition[个]{s.}{área; distrito; região; zona | uma divisão administrativa}
  \definition{v.}{distinguir; classificar; subdividir}
  \seeref{区}{ou1}
\end{entry}

\begin{entry}{区别}{qu1bie2}{4,7}{⼖、⼑}[HSK 3]
  \definition[个]{s.}{diferença; distinção; discriminação}
  \definition{v.}{distinguir; diferenciar; fazer distinção entre}
\end{entry}

\begin{entry}{区域}{qu1yu4}{4,11}{⼖、⼟}[HSK 5]
  \definition[片,块,个]{s.}{área; setor; região; faixa; inclui áreas regionais com condições naturais, culturais, administrativas, etc.}
\end{entry}

\begin{entry}{曲棍球}{qu1gun4qiu2}{6,12,11}{⽈、⽊、⽟}
  \definition{s.}{hóquei em campo}
\end{entry}

\begin{entry}{驱}{qu1}{7}{⾺}
  \definition{v.}{expulsar | repelir}
\end{entry}

\begin{entry}{趋势}{qu1shi4}{12,8}{⾛、⼒}[HSK 4]
  \definition{s.}{tendência; tendência; direção; impulso das coisas que se movem em uma direção ou outra}
\end{entry}

\begin{entry}{取}{qu3}{8}{⼜}[HSK 2]
  \definition{v.}{buscar | obter | escolher}
\end{entry}

\begin{entry}{取得}{qu3 de2}{8,11}{⼜、⼻}[HSK 2]
  \definition{v.}{ganhar | adquirir | obter}
\end{entry}

\begin{entry}{取胜}{qu3sheng4}{8,9}{⼜、⾁}
  \definition{v.}{prevalecer sobre os oponentes | marcar uma vitória}
\end{entry}

\begin{entry}{取水}{qu3shui3}{8,4}{⼜、⽔}
  \definition{v.}{obter água (de um poço, etc.)}
\end{entry}

\begin{entry}{取现}{qu3xian4}{8,8}{⼜、⾒}
  \definition{v.}{sacar dinheiro}
\end{entry}

\begin{entry}{取消}{qu3xiao1}{8,10}{⼜、⽔}[HSK 3]
  \definition{v.}{cancelar; suspender; anular; abolir; revogar; rescindir}
\end{entry}

\begin{entry}{取悦}{qu3yue4}{8,10}{⼜、⼼}
  \definition{v.}{tentar agradar}
\end{entry}

\begin{entry}{厺}{qu4}{5}{⼤}
  \variantof{去}
\end{entry}

\begin{entry}{去}{qu4}{5}{⼛}[HSK 1]
  \definition{v.}{ir | (eufenismo) morrer}
\end{entry}

\begin{entry}{去年}{qu4nian2}{5,6}{⼛、⼲}[HSK 1]
  \definition{s.}{ano passado}
\end{entry}

\begin{entry}{去世}{qu4shi4}{5,5}{⼛、⼀}[HSK 3]
  \definition{v.}{morrer; falecer (um adulto)}
\end{entry}

\begin{entry}{去死}{qu4si3}{5,6}{⼛、⽍}
  \definition{interj.}{Caia morto! | Vá para o Inferno!}
\end{entry}

\begin{entry}{圈}{quan1}{11}{⼞}[HSK 4]
  \definition[个]{s.}{anel; círculo; refere-se a algo em forma de anel | domínio; grupo; escopo; círculo(s)}
  \definition{v.}{cercar; rodear; circundar | marcar com um círculo}
  \seeref{圈}{juan1}
  \seeref{圈}{juan4}
\end{entry}

\begin{entry}{圈粉}{quan1fen3}{11,10}{⼞、⽶}
  \definition{s.}{(neologismo, coloquial) ganhar alguém como fã, obter novos fãs}
\end{entry}

\begin{entry}{全}{quan2}{6}{⼊}[HSK 2]
  \definition*{s.}{sobrenome Quan}
  \definition{adv.}{completamente | totalmente}
\end{entry}

\begin{entry}{全部}{quan2bu4}{6,10}{⼊、⾢}[HSK 2]
  \definition{adv.}{todo, todos}
\end{entry}

\begin{entry}{全场}{quan2 chang3}{6,6}{⼊、⼟}[HSK 3]
  \definition{s.}{toda a audiência; todos os presentes | corte (de justiça) inteira}
\end{entry}

\begin{entry}{全都}{quan2 dou1}{6,10}{⼊、⾢}[HSK 5]
  \definition{adv.}{tudo; todos; sem exceção}
\end{entry}

\begin{entry}{全都不}{quan2dou1 bu4}{6,10,4}{⼊、⾢、⼀}
  \definition{adj.}{nada; nenhum; nenhum deles; nada disso}
\end{entry}

\begin{entry}{全国}{quan2 guo2}{6,8}{⼊、⼞}[HSK 2]
  \definition{s.}{nação | a nação inteira | o país inteiro | todo o país}
\end{entry}

\begin{entry}{全家}{quan2 jia1}{6,10}{⼊、⼧}[HSK 2]
  \definition{s.}{a família inteira | toda a família}
\end{entry}

\begin{entry}{全面}{quan2mian4}{6,9}{⼊、⾯}[HSK 3]
  \definition{adj.}{geral; completo}
  \definition{s.}{todos os aspectos; cada aspecto}
  \seealsoref{片面}{pian4mian4}
\end{entry}

\begin{entry}{全年}{quan2 nian2}{6,6}{⼊、⼲}[HSK 2]
  \definition{s.}{anual}
\end{entry}

\begin{entry}{全球}{quan2 qiu2}{6,11}{⼊、⽟}[HSK 3]
  \definition{adj.}{global}
  \definition{s.}{o mundo inteiro}
\end{entry}

\begin{entry}{全身}{quan2 shen1}{6,7}{⼊、⾝}[HSK 2]
  \definition{s.}{corpo inteiro | todo (o corpo)}
\end{entry}

\begin{entry}{全世界}{quan2 shi4 jie4}{6,5,9}{⼊、⼀、⽥}[HSK 5]
  \definition[种]{s.}{mundo inteiro; mundo todo | em todo o mundo}
\end{entry}

\begin{entry}{全体}{quan2 ti3}{6,7}{⼊、⼈}[HSK 2]
  \definition{s.}{tudo | todo | inteiro}
\end{entry}

\begin{entry}{全职}{quan2zhi2}{6,11}{⼊、⽿}
  \definition{s.}{período integral | tempo inteiro | (trabalho) \emph{full-time}}
\end{entry}

\begin{entry}{权利}{quan2li4}{6,7}{⽊、⼑}[HSK 4]
  \definition[项,种,个,条,份]{s.}{direito; interesse; os poderes e benefícios (em oposição a “义务”) exercidos por um cidadão ou pessoa jurídica de acordo com a lei}
  \seealsoref{义务}{yi4wu4}
\end{entry}

\begin{entry}{泉}{quan2}{9}{⽔}[HSK 5]
  \definition*{s.}{sobrenome Quan}
  \definition[股,眼,汪]{s.}{fonte (de água mineral) | a nascente de um rio | termo antigo para moeda}
\end{entry}

\begin{entry}{拳法}{quan2fa3}{10,8}{⼿、⽔}
  \definition{s.}{boxe | luta}
\end{entry}

\begin{entry}{拳王}{quan2wang2}{10,4}{⼿、⽟}
  \definition{s.}{pugilista | boxeador}
\end{entry}

\begin{entry}{犬}{quan3}{4}{⽝}[Kangxi 94]
  \definition{s.}{cachorro}
\end{entry}

\begin{entry}{劝}{quan4}{4}{⼒}[HSK 5]
  \definition*{s.}{sobrenome Quan}
  \definition{v.}{insistir; aconselhar; tentar persuadir; persuadir, argumentar para que as pessoas obedeçam | incentivar; encorajar}
\end{entry}

\begin{entry}{缺}{que1}{10}{⽸}[HSK 3]
  \definition{adj.}{incompleto; imperfeito}
  \definition[种]{s.}{vaga; abertura; falta}
  \definition{v.}{estar com falta de; faltar | estar faltando; estar incompleto | estar ausente}
\end{entry}

\begin{entry}{缺点}{que1dian3}{10,9}{⽸、⽕}[HSK 3]
  \definition[个]{s.}{desvantagem; deficiência; inconveniência; ponto fraco}
\end{entry}

\begin{entry}{缺乏}{que1fa2}{10,4}{⽸、⼃}[HSK 5]
  \definition{v.}{faltar; estar em falta de; não ter ou não ter totalmente (algo que deveria possuir ou é desejaria possuir)}
\end{entry}

\begin{entry}{缺勤}{que1qin2}{10,13}{⽸、⼒}
  \definition{v.+compl.}{ausentar-se do dever (trabalho)}
\end{entry}

\begin{entry}{缺少}{que1shao3}{10,4}{⽸、⼩}[HSK 3]
  \definition{v.}{falta; estar com falta de; estar em falta de}
\end{entry}

\begin{entry}{却}{que4}{7}{⼙}[HSK 4]
  \definition{adv.}{mas; contudo; no entanto; enquanto; indica um ponto de virada}
  \definition{v.}{recuar; retroceder | afastar; repelir; desencorajar | declinar; recusar; rejeitar | usado depois de certos verbos para indicar a conclusão de uma ação}
\end{entry}

\begin{entry}{却是}{que4shi4}{7,9}{⼙、⽇}
  \definition{conj.}{no entanto | realmente | o fato é\dots | mas isso é\dots}
\end{entry}

\begin{entry}{确}{que4}{12}{⽯}
  \definition{adj.}{autenticado | sólido | firme | real | verdadeiro}
\end{entry}

\begin{entry}{确保}{que4bao3}{12,9}{⽯、⼈}[HSK 3]
  \definition{v.}{assegurar; garantir}
\end{entry}

\begin{entry}{确定}{que4ding4}{12,8}{⽯、⼧}[HSK 3]
  \definition{adj.}{definido; certo}
  \definition{v.}{consertar; definir; determinar}
\end{entry}

\begin{entry}{确立}{que4li4}{12,5}{⽯、⽴}[HSK 5]
  \definition{v.}{estabelecer; criar; construir; estabelecer ou consolidar firmemente}
\end{entry}

\begin{entry}{确认}{que4ren4}{12,4}{⽯、⾔}[HSK 4]
  \definition{v.}{afirmar; confirmar; reconhecer; confirmar explicitamente (fatos, princípios, etc.)}
\end{entry}

\begin{entry}{确实}{que4shi2}{12,8}{⽯、⼧}[HSK 3]
  \definition{adj.}{verdadeiro; confiável}
  \definition{adv.}{verdadeiramente; realmente; de ​​fato}
\end{entry}

\begin{entry}{裙子}{qun2zi5}{12,3}{⾐、⼦}[HSK 3]
  \definition[条,件]{s.}{saia (peça de vestuário)}
\end{entry}

\begin{entry}{群}{qun2}{13}{⽺}[HSK 3]
  \definition{clas.}{grupo; rebanho; manada}
  \definition{s.}{multidão; grupo}
\end{entry}

\begin{entry}{群山}{qun2shan1}{13,3}{⽺、⼭}
  \definition{s.}{montanhas | uma cadeia de colinas}
\end{entry}

\begin{entry}{群体}{qun2 ti3}{13,7}{⽺、⼈}[HSK 5]
  \definition{s.}{colônia; um conjunto composto por muitos indivíduos da mesma espécie que estão fisicamente conectados, exemplos incluem corais entre os animais e certas algas entre as plantas | grupos; refere-se, de maneira geral, ao conjunto formado por muitos indivíduos interligados que compartilham características essenciais em comum}
\end{entry}

\begin{entry}{群众}{qun2zhong4}{13,6}{⽺、⼈}[HSK 5]
  \definition[个,名,位]{s.}{as massas; refere-se ao povo em geral | não filiado; apartidário; refere-se a pessoas que não são membros do Partido Comunista Chinês nem da Liga da Juventude Comunista |
alguém que não ocupa uma posição de liderança}
\end{entry}

%%%%% EOF %%%%%


%%%
%%% R
%%%

\section*{R}\addcontentsline{toc}{section}{R}

\begin{verbete}{儿}{r5}{2}[Radical 儿]
  \significado{conj.}{sufixo diminutivo não silábico; final retroflexo}
  \veja{儿}{er2}
  \veja{儿}{ren2}
\end{verbete}

\begin{verbete}{然}{ran2}{12}[Radical 火]
  \significado{conj.}{mas; no entanto}
\end{verbete}

\begin{verbete}{然而}{ran2'er2}{12,6}
  \significado{conj.}{mas; no entanto}
\end{verbete}

\begin{verbete}{然后}{ran2hou4}{12,6}
  \significado{conj.}{depois; logo; portanto}
\end{verbete}

\begin{verbete}{燃烧}{ran2shao1}{16,10}
  \significado{s.}{combustão; flama}
  \significado{v.}{queimar; acender}
\end{verbete}

\begin{verbete}{壤}{rang3}{20}[Radical 土]
  \significado{s.}{solo; terra; (literário) a terra (em contraste com o céu 天)}
\end{verbete}

\begin{verbete}{让}{rang4}{5}[Radical 言]
  \significado{v.}{deixar alguém fazer alguma coisa; fazer alguém (sentir-se triste, etc.); permitir; conceder}
\end{verbete}

\begin{verbete}{让步}{rang4bu4}{5,7}
  \significado{v.+compl.}{fazer uma concessão; entregar; desistir; comprometer}
\end{verbete}

\begin{verbete}{热}{re4}{10}[Radical 火]
  \significado{adj.}{quente (clima); fervente; ardente; fervoroso}
  \significado{v.}{aquecer; ferver}
\end{verbete}

\begin{verbete}{热爱}{re4'ai4}{10,10}
  \significado{v.}{amar ardentemente; adorar}
\end{verbete}

\begin{verbete}{热泪盈眶}{re4lei4ying2kuang4}{10,8,9,11}
  \significado{expr.}{olhos cheios de lágrimas de emoção; extremamente emocionado}
\end{verbete}

\begin{verbete}{热闹}{re4nao5}{10,8}
  \significado{adj.}{animado; movimentado com barulho e excitação}
\end{verbete}

\begin{verbete}{热心}{re4xin1}{10,4}
  \significado{adj.}{entusiasmado, ardente, zeloso}
\end{verbete}

\begin{verbete}{热血沸腾}{re4xue4fei4teng2}{10,6,8,13}
  \significado{expr.}{ferver o sangue; apaixonar-se}
\end{verbete}

\begin{verbete}{人}{ren2}{2}[Radical 人][Kangxi 9]
  \significado[个,位]{s.}{pessoa; gente}
\end{verbete}

\begin{verbete}{人才}{ren2cai2}{2,3}
  \significado{s.}{talento; pessoa talentosa}
\end{verbete}

\begin{verbete}{人材}{ren2cai2}{2,7}
  \variante{人才}
\end{verbete}

\begin{verbete}{人道}{ren2dao4}{2,12}
  \significado{s.}{solidariedade humana; humanitarismo; humano; a ``maneira humana'', um dos estágios do ciclo de reencarnação (budismo); relação sexual}
\end{verbete}

\begin{verbete}{人海}{ren2hai3}{2,10}
  \significado{s.}{uma multidão; um mar de pessoas}
\end{verbete}

\begin{verbete}{人间}{ren2jian1}{2,7}
  \significado{s.}{o mundo humano; a Terra}
\end{verbete}

\begin{verbete}{人口}{ren2kou3}{2,3}
  \significado{s.}{pessoas; população}
\end{verbete}

\begin{verbete}{人类}{ren2lei4}{2,9}
  \significado{s.}{humanidade; raça humana}
\end{verbete}

\begin{verbete}{人民}{ren2min2}{2,5}
  \significado[个]{s.}{povo; população}
\end{verbete}

\begin{verbete}{人民币}{ren2min2bi4}{2,5,4}
  \significado*{s.}{Renminbi (RMB); Yuan Chinês (CYN); nome da moeda chinesa}
\end{verbete}

\begin{verbete}{人权}{ren2quan2}{2,6}
  \significado*{s.}{Direitos Humanos}
  \veja{人权法}{ren2quan2fa3}
\end{verbete}

\begin{verbete}{人权法}{ren2quan2fa3}{2,6,8}
  \significado*{s.}{Direitos Humanos}
  \veja{人权}{ren2quan2}
\end{verbete}

\begin{verbete}{人生}{ren2sheng1}{2,5}
  \significado{s.}{vida (tempo de alguém na Terra)}
\end{verbete}

\begin{verbete}{人像}{ren2xiang4}{2,13}
  \significado{s.}{``retrato'' de uma pessoa (esboço, foto, escultura, etc.)}
\end{verbete}

\begin{verbete}{人行道}{ren2xing2dao4}{2,6,12}
  \significado{s.}{calçada}
\end{verbete}

\begin{verbete}{人鱼}{ren2yu2}{2,8}
  \significado{s.}{sereia; peixe-boi; salamandra gigante}
\end{verbete}

\begin{verbete}{儿}{ren2}{2}[Radical 儿]
  \significado{s.}{pessoa, radical em caracteres chineses}
  \variante{人}
  \veja{儿}{er2}
  \veja{儿}{r5}
\end{verbete}

\begin{verbete}{忍耐}{ren3nai4}{7,9}
  \significado{s.}{paciência; resistência}
  \significado{v.}{suportar; resistir; exercer paciência}
\end{verbete}

\begin{verbete}{认识}{ren4shi5}{4,7}
  \significado{s.}{conhecimento; saber; entendimento}
  \significado{v.}{estar familiarizado com; conhecer alguém; saber; reconhecer}
\end{verbete}

\begin{verbete}{认真}{ren4zhen1}{4,10}
  \significado{adj.}{sério; consciencioso}
  \significado{adv.}{seriamente}
  \significado{v.}{levar a sério}
\end{verbete}

\begin{verbete}{任务}{ren4wu5}{6,5}
  \significado[项,个]{s.}{missão, atribuição, tarefa, obrigação, papel}
\end{verbete}

\begin{verbete}{扔}{reng1}{5}[Radical 手]
  \significado{v.}{lançar; atirar}
\end{verbete}

\begin{verbete}{扔掉}{reng1diao4}{5,11}
  \significado{v.}{jogar fora}
\end{verbete}

\begin{verbete}{扔弃}{reng1qi4}{5,7}
  \significado{v.}{abandonar; descartar; jogar fora}
\end{verbete}

\begin{verbete}{扔下}{reng1xia4}{5,3}
  \significado{v.}{lançar (uma bomba); derrubar}
\end{verbete}

\begin{verbete}{仍然}{reng2ran2}{4,12}
  \significado{adv.}{ainda}
\end{verbete}

\begin{verbete}{日}{ri4}{4}[Radical 日][Kangxi 72]
  \significado*{s.}{Japão, abreviação de~日本}
  \significado{clas.}{dia (mais usado em escrita); data, dia do mês}
  \veja{日本}{ri4ben3}
\end{verbete}

\begin{verbete}{日本}{ri4ben3}{4,5}
  \significado*{s.}{Japão}
  \veja{日}{ri4}
\end{verbete}

\begin{verbete}{日本人}{ri4ben3ren2}{4,5,2}
  \significado{s.}{japonês; nascido no Japão}
\end{verbete}

\begin{verbete}{日常}{ri4chang2}{4,11}
  \significado{adv.}{diariamente; dia-a-dia; todo dia}
\end{verbete}

\begin{verbete}{日出}{ri4chu1}{4,5}
  \significado{s.}{nascer do sol}
  \veja{夕阳}{xi1yang2}
\end{verbete}

\begin{verbete}{日光灯}{ri4guang1deng1}{4,6,6}
  \significado{s.}{lâmpada fluorescente}
\end{verbete}

\begin{verbete}{日子}{ri4zi5}{4,3}
  \significado{s.}{dia; uma data (calendário); dias de vida de alguém}
\end{verbete}

\begin{verbete}{容貌}{rong2mao4}{10,14}
  \significado{s.}{aparência; aspecto; características}
\end{verbete}

\begin{verbete}{容易}{rong2yi4}{10,8}
  \significado{adj.}{fácil; responsável (por); provável}
\end{verbete}

\begin{verbete}{柔软}{rou2ruan3}{9,8}
  \significado{adj.}{macio; suave}
\end{verbete}

\begin{verbete}{揉}{rou2}{12}[Radical 手]
  \significado{v.}{amassar; massagear; esfregar}
\end{verbete}

\begin{verbete}{揉碎}{rou2sui4}{12,13}
  \significado{v.}{esmagar; desintegrar-se em pedaços}
\end{verbete}

\begin{verbete}{肉}{rou4}{6}[Radical 肉][Kangxi 130]
  \significado{s.}{carne; polpa de uma fruta}
\end{verbete}

\begin{verbete}{肉桂}{rou4gui4}{6,10}
  \significado{s.}{canela}
  \veja{官桂}{guan1gui4}
\end{verbete}

\begin{verbete}{如}{ru2}{6}[Radical 女]
  \significado{conj.}{por exemplo}
\end{verbete}

\begin{verbete}{如此}{ru2ci3}{6,6}
  \significado{adv.}{assim, então, tal}
\end{verbete}

\begin{verbete}{如果}{ru2guo3}{6,8}
  \significado{conj.}{se; caso; no caso de; no evento de; supondo que}
\end{verbete}

\begin{verbete}{如画}{ru2hua4}{6,8}
  \significado{adj.}{pitoresco}
\end{verbete}

\begin{verbete}{儒教}{ru2jiao4}{16,11}
  \significado*{s.}{Confucionismo}
\end{verbete}

\begin{verbete}{乳房}{ru3fang2}{8,8}
  \significado{s.}{seio; mama; úbere}
\end{verbete}

\begin{verbete}{辱骂}{ru3ma4}{10,9}
  \significado{v.}{insultar; abusar}
\end{verbete}

\begin{verbete}{入党}{ru4dang3}{2,10}
  \significado{v.}{ingressar em um partido político (especialmente o partido comunista)}
\end{verbete}

\begin{verbete}{入境}{ru4jing4}{2,14}
  \significado{s.}{imigração}
  \significado{v.+compl.}{entrar em um país; imigrar}
\end{verbete}

\begin{verbete}{入门}{ru4men2}{2,3}
  \significado{s.}{curso elementar; ABC; guia}
  \significado{v.+compl.}{atravessar o limiar; aprender o ABC de; introduzir um assunto; aprender os rudimentos de um assunto |}
\end{verbete}

\begin{verbete}{入乡随俗}{ru4xiang1-sui2su2}{2,3,11,9}
  \significado{expr.}{Em roma, faça como os romanos!}
\end{verbete}

\begin{verbete}{软件}{ruan3jian4}{8,6}
  \significado{v.}{\emph{software}}
\end{verbete}

%%%%% EOF %%%%%


%%%
%%% S
%%%

\section*{S}\addcontentsline{toc}{section}{S}

\begin{verbete}{撒旦}{sa1dan4}{15,5}
  \significado*{s.}{Satã}
\end{verbete}

\begin{verbete}{撒旦主义}{sa1dan4 zhu3yi4}{15,5,5,3}
  \significado*{s.}{Satanismo}
\end{verbete}

\begin{verbete}{撒但}{sa1dan4}{15,7}
  \variante{撒旦}
\end{verbete}

\begin{verbete}{洒水}{sa3shui3}{9,4}
  \significado{v.}{borrifar}
\end{verbete}

\begin{verbete}{飒飒}{sa4sa4}{9,9}
  \significado{s.}{o farfalhar; sussurro, murmúrio (do vento nas árvores, o mar, etc.)}
\end{verbete}

\begin{verbete}{赛}{sai4}{14}[Radical 貝]
  \significado{s.}{competição}
  \significado{v.}{competir; superar; destacar-se}
\end{verbete}

\begin{verbete}{赛车}{sai4che1}{14,4}
  \significado{s.}{corrida de automóvel; corrida de bicicleta; carro de corrida}
\end{verbete}

\begin{verbete}{三}{san1}{3}[Radical 一]
  \significado*{s.}{sobrenome San}
  \significado{num.}{três, 3}
\end{verbete}

\begin{verbete}{三角}{san1jiao3}{3,7}
  \significado{s.}{triângulo}
\end{verbete}

\begin{verbete}{三角恋爱}{san1jiao3lian4'ai4}{3,7,10,10}
  \significado{s.}{triângulo amoroso}
\end{verbete}

\begin{verbete}{三轮车}{san1lun2che1}{3,8,4}
  \significado{s.}{triciclo}
\end{verbete}

\begin{verbete}{三明治}{san1ming2zhi4}{3,8,8}
  \significado{s.}{sanduíche (empréstimo linguístico)}
\end{verbete}

\begin{verbete}{散}{san3}{12}[Radical 攴]
  \significado{adj.}{disseminado; dispersado; solto}
  \significado{s.}{medicamento em pó}
  \significado{v.}{soltar-se; desmoronar}
  \veja{散}{san4}
\end{verbete}

\begin{verbete}{散}{san4}{12}[Radical 攴]
  \significado{v.}{terminar (uma reunião, etc.); dispersar; disseminar; dissipar}
  \veja{散}{san3}
\end{verbete}

\begin{verbete}{散步}{san4bu4}{12,7}
  \significado{v.+compl.}{dar um passeio; passear | dar uma caminhada}
\end{verbete}

\begin{verbete}{散心}{san4xin1}{12,4}
  \significado{v.+compl.}{aliviar o tédio; desfrutar de uma diversão; estar despreocupado}
\end{verbete}

\begin{verbete}{丧钟}{sang1zhong1}{8,9}
  \significado{s.}{sentença de morte}
\end{verbete}

\begin{verbete}{桑}{sang1}{10}[Radical 木]
  \significado*{s.}{sobrenome Sang}
  \significado{s.}{amoreira}
\end{verbete}

\begin{verbete}{桑巴舞}{sang1ba1wu3}{10,4,14}
  \significado{s.}{samba}
\end{verbete}

\begin{verbete}{桑树}{sang1shu4}{10,9}
  \significado{s.}{amoreira, suas folhas são utilizadas para alimentar bichos-da-seda}
\end{verbete}

\begin{verbete}{骚乱}{sao1luan4}{12,7}
  \significado{s.}{rebelião; perturbação; tumulto}
  \significado{v.}{criar um distúrbio}
\end{verbete}

\begin{verbete}{扫兴}{sao3xing4}{6,6}
  \significado{v.+compl.}{sentir-se decepcionado; entristecer alguém}
\end{verbete}

\begin{verbete}{嫂子}{sao3zi5}{12,3}
  \significado{s.}{esposa do irmão mais velho}
\end{verbete}

\begin{verbete}{色狼}{se4lang2}{6,10}
  \significado*{s.}{Sátiro}
  \significado{adj.}{depravado; tarado}
\end{verbete}

\begin{verbete}{森林}{sen1lin2}{12,8}
  \significado{s.}{floresta}
\end{verbete}

\begin{verbete}{僧}{seng1}{14}[Radical 人]
  \significado{s.}{monge Budista (abreviatura de 僧伽)}
  \veja{僧伽}{seng1qie2}
\end{verbete}

\begin{verbete}{僧伽}{seng1qie2}{14,7}
  \significado{s.}{sangha ou sanga (Budismo); a comunidade monástica; monge}
\end{verbete}

\begin{verbete}{杀气}{sha1qi4}{6,4}
  \significado{s.}{espírito assassino; aura de morte}
  \significado{v.}{desabafar a raiva de alguém}
\end{verbete}

\begin{verbete}{沙}{sha1}{7}[Radical 水]
  \significado*{s.}{sobrenome Sha}
  \significado[粒]{s.}{areia; cascalho; grânulo; pó}
\end{verbete}

\begin{verbete}{沙漠}{sha1mo4}{7,13}
  \significado[个]{s.}{deserto}
\end{verbete}

\begin{verbete}{沙特}{sha1te4}{7,10}
  \significado*{s.}{Saudita; abreviação de 沙特阿拉伯}
  \veja{沙特阿拉伯}{sha1te4 a1la1bo2}
\end{verbete}

\begin{verbete}{沙特阿拉伯}{sha1te4 a1la1bo2}{7,10,7,8,7}
  \significado*{s.}{Arábia Saudita}
\end{verbete}

\begin{verbete}{沙鱼}{sha1yu2}{7,8}
  \variante{鲨鱼}
\end{verbete}

\begin{verbete}{刹}{sha1}{8}[Radical 刀]
  \significado{v.}{frear}
  \veja{刹}{cha4}
\end{verbete}

\begin{verbete}{砂}{sha1}{9}[Radical 石]
  \variante{沙}
\end{verbete}

\begin{verbete}{莎莎舞}{sha1sha1wu3}{10,10,14}
  \significado{s.}{salsa (dança)}
\end{verbete}

\begin{verbete}{鲨鱼}{sha1yu2}{15,8}
  \significado{s.}{tubarão}
\end{verbete}

\begin{verbete}{啥}{sha2}{11}[Radical 口]
  \significado{interr.}{Equivalente a 什么 (dialeto), também pronunciado como \dpy{sha4}}
  \veja{什么}{shen2me5}
\end{verbete}

\begin{verbete}{傻瓜}{sha3gua1}{13,5}
  \significado{adj.}{tolo; burro; simplório; idiota}
  \significado{v.}{enganar; iludir; lograr}
\end{verbete}

\begin{verbete}{傻眼}{sha3yan3}{13,11}
  \significado{adj.}{estupefato; atordoado}
\end{verbete}

\begin{verbete}{啥}{sha4}{11}[Radical 口]
  \veja{啥}{sha2}
\end{verbete}

\begin{verbete}{晒干}{shai4gan1}{10,3}
  \significado{v.}{secar ao sol}
\end{verbete}

\begin{verbete}{山}{shan1}{3}[Radical 山][Kangxi 46]
  \significado*{s.}{sobrenome Shan}
  \significado[座]{s.}{montanha; monte; qualquer coisa que se assemelhe a uma montanha}
\end{verbete}

\begin{verbete}{山顶}{shan1ding3}{3,8}
  \significado{s.}{cume da montanha}
\end{verbete}

\begin{verbete}{山东}{shan1dong1}{3,5}
  \significado*{s.}{Shandong}
\end{verbete}

\begin{verbete}{山谷}{shan1gu3}{3,7}
  \significado{s.}{vale; ravina}
\end{verbete}

\begin{verbete}{山区}{shan1qu1}{3,4}
  \significado[个]{s.}{área montanhosa; montanhas}
\end{verbete}

\begin{verbete}{山体}{shan1ti3}{3,7}
  \significado{s.}{forma de uma montanha}
\end{verbete}

\begin{verbete}{山羊}{shan1yang2}{3,6}
  \significado{s.}{cabra; (ginástica) cavalo de salto de pequeno porte}
\end{verbete}

\begin{verbete}{山寨}{shan1zhai4}{3,14}
  \significado{s.}{fortaleza fortificada da vila; fortaleza da montanha (especialmente de bandidos); falsificação; imitação; (fig.) pechincha}
\end{verbete}

\begin{verbete}{闪存盘}{shan3cun2pan2}{5,6,11}
  \significado{s.}{unidade de memória USB; cartão de memória; \emph{pen drive}}
\end{verbete}

\begin{verbete}{单}{shan4}{8}[Radical 十]
  \significado*{s.}{sobrenome Shan}
  \veja{单}{chan2}
  \veja{单}{dan1}
\end{verbete}

\begin{verbete}{扇子}{shan4zi5}{10,3}
  \significado[把]{s.}{leque; abano; abanador}
\end{verbete}

\begin{verbete}{善意}{shan4yi4}{12,13}
  \significado{s.}{boa vontade; benevolência; bondade}
\end{verbete}

\begin{verbete}{禅}{shan4}{12}[Radical 示]
  \significado{v.}{abdicar}
  \veja{禅}{chan2}
\end{verbete}

\begin{verbete}{擅自}{shan4zi4}{16,6}
  \significado{adv.}{sem permissão ou autorização; por iniciativa própria}
\end{verbete}

\begin{verbete}{伤}{shang1}{6}[Radical 人]
  \significado{s.}{ferida; ferimento}
  \significado{v.}{ferir; ferir-se}
\end{verbete}

\begin{verbete}{伤心}{shang1xin1}{6,4}
  \significado{v.}{sofrer; ter o coração partido; sentir-se profundamente magoado}
\end{verbete}

\begin{verbete}{汤}{shang1}{6}[Radical 水]
  \significado{s.}{correnteza forte}
  \veja{汤}{tang1}
\end{verbete}

\begin{verbete}{商店}{shang1dian4}{11,8}
  \significado[家,个]{s.}{loja}
\end{verbete}

\begin{verbete}{商贸}{shang1mao4}{11,9}
  \significado{s.}{comércio}
\end{verbete}

\begin{verbete}{上声}{shang3sheng1}{3,7}
  \significado{s.}{tom descendente e ascendente; terceiro tom no mandarim moderno}
\end{verbete}

\begin{verbete}{赏赐}{shang3ci4}{12,12}
  \significado{s.}{recompensa; prêmio}
  \significado{v.}{recompensar; premiar}
\end{verbete}

\begin{verbete}{赏心悦目}{shang3xin1yue4mu4}{12,4,10,5}
  \significado{expr.}{"Aquece o coração e encanta os olhos."}
\end{verbete}

\begin{verbete}{上}{shang4}{3}[Radical 一]
  \significado{adv.}{acima; em cima; sobre}
  \significado{v.}{subir; entrar em; frequentar (aula ou universidade)}
\end{verbete}

\begin{verbete}{上班}{shang4ban1}{3,10}
  \significado{v.+compl.}{ir para o trabalho; ir para o emprego; estar de plantão}
\end{verbete}

\begin{verbete}{上边}{shang4bian5}{3,5}
  \significado{adv.}{acima de; parte de cima; por cima}
\end{verbete}

\begin{verbete}{上车}{shang4che1}{3,4}
  \significado{v.}{entrar (em ônibus, trem, carro, etc.)}
\end{verbete}

\begin{verbete}{上当}{shang4dang4}{3,6}
  \significado{v.+compl.}{ser enganado; morder uma isca; ser manipulado; ser joguete nas mãos de alguém}
\end{verbete}

\begin{verbete}{上访}{shang4fang3}{3,6}
  \significado{v.}{buscar uma audiência com superiores (especialmente funcionários do governo) para fazer uma petição por algo}
\end{verbete}

\begin{verbete}{上古}{shang4gu3}{3,5}
  \significado{s.}{o passado distante; tempos antigos ; antiguidade}
\end{verbete}

\begin{verbete}{上海}{shang4hai3}{3,10}
  \significado*{s.}{Shangai (Xangai)}
\end{verbete}

\begin{verbete}{上课}{shang4ke4}{3,10}
  \significado{v.}{assistir à aula; ir para a aula; ir dar uma aula}
\end{verbete}

\begin{verbete}{上来}{shang4lai2}{3,7}
  \significado{v.}{subir (para a minha localização)}
\end{verbete}

\begin{verbete}{上面}{shang4mian4}{3,9}
  \significado{adv.}{acima de; parte de cima; por cima}
\end{verbete}

\begin{verbete}{上坡路}{shang4po1lu4}{3,8,13}
  \significado{s.}{aclive; progresso; (fig.) tendência ascendente}
\end{verbete}

\begin{verbete}{上去}{shang4qu4}{3,5}
  \significado{v.}{subir (a partir da minha localização)}
\end{verbete}

\begin{verbete}{上网}{shang4wang3}{3,6}
  \significado{v.}{conectar à \emph{Internet}; fazer \emph{upload}; ficar \emph{online}}
\end{verbete}

\begin{verbete}{上午}{shang4wu3}{3,4}
  \significado{adv.}{manhã; de manhã; período antes do meio-dia}
\end{verbete}

\begin{verbete}{上询}{shang4 xun2}{3,8}
  \significado{adv.}{primeira dezena do mês}
\end{verbete}

\begin{verbete}{上演}{shang4yan3}{3,14}
  \significado{s.}{exibição; encenação}
  \significado{v.}{exibir (um filme); encenar (uma peça)}
\end{verbete}

\begin{verbete}{尚且}{shang4qie3}{8,5}
  \significado{conj.}{até; ainda}
\end{verbete}

\begin{verbete}{尚且……何况……}{shang4qie3 he2kuang4}{8,5,7,7}
  \significado{conj.}{ainda que\dots, \dots}
\end{verbete}

\begin{verbete}{烧}{shao1}{10}[Radical 火]
  \significado{s.}{febre}
  \significado{v.}{queimar; cozinhar; cozer; assar; aquecer; ferver (chá, água, etc.); ter febre; (coloquial) deixar as coisas subirem à cabeça}
\end{verbete}

\begin{verbete}{烧烤}{shao1kao3}{10,10}
  \significado{s.}{churrasco}
  \significado{v.}{assar}
\end{verbete}

\begin{verbete}{稍}{shao1}{12}[Radical 禾]
  \significado{adv.}{um pouco; ligeiramente; em vez de}
\end{verbete}

\begin{verbete}{稍微}{shao1wei1}{12,13}
  \significado{adv.}{um pouco}
\end{verbete}

\begin{verbete}{少}{shao3}{4}[Radical 小]
  \significado{adj.}{pouco, poucos}
  \significado{v.}{sentir falta; faltar; parar (de fazer algo)}
  \veja{少}{shao4}
\end{verbete}

\begin{verbete}{少}{shao4}{4}[Radical 小]
  \significado{s.}{jovem}
  \veja{少}{shao3}
\end{verbete}

\begin{verbete}{舌头}{she2tou5}{6,5}
  \significado[个]{s.}{língua; soldado inimigo capturado com o propósito de extrair informações}
\end{verbete}

\begin{verbete}{蛇}{she2}{11}[Radical 虫]
  \significado[条]{s.}{cobra; serpente}
\end{verbete}

\begin{verbete}{设备}{she4bei4}{6,8}
  \significado[个]{s.}{equipamento; instalações}
\end{verbete}

\begin{verbete}{设计}{she4ji4}{6,4}
  \significado{s.}{projeto; planejamento}
  \significado{v.}{projetar; planejar}
\end{verbete}

\begin{verbete}{射}{she4}{10}[Radical 寸]
  \significado{v.}{atirar; lançar}
\end{verbete}

\begin{verbete}{摄氏}{she4shi4}{13,4}
  \significado{s.}{graus Celsius (°C), centígrado}
\end{verbete}

\begin{verbete}{谁}{shei2}{10}[Radical 言]
  \significado{interr.}{quem?}
  \veja{谁}{shui2}
\end{verbete}

\begin{verbete}{身体}{shen1ti3}{7,7}
  \significado[具,个]{s.}{em pessoa; saúde de alguém; o corpo}
\end{verbete}

\begin{verbete}{身体能力}{shen1ti3 neng2li4}{7,7,10,2}
  \significado{s.}{habilidade física}
\end{verbete}

\begin{verbete}{身体乳}{shen1ti3 ru3}{7,7,8}
  \significado{s.}{loção corporal}
\end{verbete}

\begin{verbete}{身亡}{shen1wang2}{7,3}
  \significado{v.}{morrer}
\end{verbete}

\begin{verbete}{深}{shen1}{11}[Radical 水]
  \significado{adj.}{profundo}
\end{verbete}

\begin{verbete}{深厚}{shen1hou4}{11,9}
  \significado{adj.}{profundo}
\end{verbete}

\begin{verbete}{深深}{shen1shen1}{11,11}
  \significado{adj.}{profundo}
  \significado{adv.}{profundamente}
\end{verbete}

\begin{verbete}{深夜}{shen1ye4}{11,8}
  \significado{adv.}{tarde da noite}
\end{verbete}

\begin{verbete}{什么}{shen2me5}{4,3}
  \significado{pron.}{que?; o que?}
  \significado{pron.}{algo; qualquer coisa}
\end{verbete}

\begin{verbete}{什么时候}{shen2me5shi2hou5}{4,3,7,10}
  \significado{adv.}{quando?; a que horas?}
\end{verbete}

\begin{verbete}{神}{shen2}{9}[Radical 示]
  \significado*{s.}{Deus}
  \significado{s.}{deus; divindade}
\end{verbete}

\begin{verbete}{神话}{shen2hua4}{9,8}
  \significado{s.}{lenda; conto de fadas; mito; mitologia}
\end{verbete}

\begin{verbete}{神经}{shen2jing1}{9,8}
  \significado{adj.}{desequilibrado; louco; insano}
  \significado{s.}{nervo}
\end{verbete}

\begin{verbete}{神经病的}{shen2jing1bing4de5}{9,8,10,8}
  \significado{adj.}{neurótico}
\end{verbete}

\begin{verbete}{神经病学}{shen2jing1bing4xue2}{9,8,10,8}
  \significado{s.}{neurologia}
\end{verbete}

\begin{verbete}{神明}{shen2ming2}{9,8}
  \significado{s.}{divindades; deuses}
\end{verbete}

\begin{verbete}{神奇}{shen2qi2}{9,8}
  \significado{adj.}{mágico; místico; milagroso}
  \significado{s.}{mágica; milagre}
\end{verbete}

\begin{verbete}{神器}{shen2qi4}{9,16}
  \significado{s.}{objeto mágico; objeto simbólico do poder imperial; arma fina; ferramenta muito útil}
\end{verbete}

\begin{verbete}{神兽}{shen2shou4}{9,11}
  \significado{s.}{animal mitológico; fera}
\end{verbete}

\begin{verbete}{甚而}{shen4'er2}{9,6}
  \significado{conj.}{(ir) tão longe quanto; tanto que}
\end{verbete}

\begin{verbete}{甚或}{shen4huo4}{9,8}
  \significado{conj.}{(ir) tão longe quanto; tanto que}
\end{verbete}

\begin{verbete}{甚至}{shen4zhi4}{9,6}
  \significado{conj.}{(ir) tão longe quanto; tanto que; mesmo (na medida em que)}
\end{verbete}

\begin{verbete}{升起}{sheng1qi3}{4,10}
  \significado{v.}{levantar; içar; subir}
\end{verbete}

\begin{verbete}{生}{sheng1}{5}[Radical 生][Kangxi 100]
  \significado{adj.}{vida; estudante; cru; não cozido}
  \significado{v.}{nascer; dar a luz; crescer}
\end{verbete}

\begin{verbete}{生菜}{sheng1cai4}{5,11}
  \significado{s.}{alface}
\end{verbete}

\begin{verbete}{生的}{sheng1de5}{5,8}
  \significado{conj.}{para evitar isso; para que\dots não\dots}
\end{verbete}

\begin{verbete}{生活}{sheng1huo2}{5,9}
  \significado[道]{s.}{vida; atividade; meios de subsistência}
  \significado{v.}{viver}
\end{verbete}

\begin{verbete}{生活垃圾}{sheng1huo2la1ji1}{5,9,8,6}
  \significado{s.}{lixo doméstico}
\end{verbete}

\begin{verbete}{生活型}{sheng1huo2 xing2}{5,9,9}
  \significado{s.}{forma de vida}
\end{verbete}

\begin{verbete}{生理}{sheng1li3}{5,11}
  \significado{adj.}{fisiológico}
  \significado{s.}{fisiologia}
\end{verbete}

\begin{verbete}{生气}{sheng1qi4}{5,4}
  \significado{s.}{vitalidade; vigor}
  \significado{v.+compl.}{irritar-se; zangar-se; ofender-se; ficar com raiva}
\end{verbete}

\begin{verbete}{生日}{sheng1ri4}{5,4}
  \significado[个]{s.}{aniversário}
\end{verbete}

\begin{verbete}{生态}{sheng1tai4}{5,8}
  \significado{adj.}{ecológico}
  \significado{s.}{ecologia}
\end{verbete}

\begin{verbete}{生物}{sheng1wu4}{5,8}
  \significado{adj.}{biológico}
  \significado{s.}{biologia (disciplina); organismo; ser vivo}
\end{verbete}

\begin{verbete}{生意}{sheng1yi4}{5,13}
  \significado{s.}{força vital; vitalidade}
  \veja{生意}{sheng1yi5}
\end{verbete}

\begin{verbete}{生意}{sheng1yi5}{5,13}
  \significado{s.}{negócio}
  \veja{生意}{sheng1yi4}
\end{verbete}

\begin{verbete}{生鱼片}{sheng1yu2pian4}{5,8,4}
  \significado{s.}{fatias de peixe cru, \emph{sashimi}}
\end{verbete}

\begin{verbete}{生长}{sheng1zhang3}{5,4}
  \significado{v.}{crescer; amadurecer; ser criado}
\end{verbete}

\begin{verbete}{声明}{sheng1ming2}{7,8}
  \significado[项,份]{s.}{declaração}
  \significado{v.}{declarar}
\end{verbete}

\begin{verbete}{绳子}{sheng2zi5}{11,3}
  \significado[条]{s.}{corda; cordão}
\end{verbete}

\begin{verbete}{省}{sheng3}{9}[Radical 目]
  \significado{s.}{província; capital provincial}
  \significado{v.}{economizar, guardar, ser frugal; omitir, excluir, deixar de fora}
  \veja{省}{xing3}
\end{verbete}

\begin{verbete}{省城}{sheng3cheng2}{9,9}
  \significado{s.}{capital da província}
\end{verbete}

\begin{verbete}{省会}{sheng3hui4}{9,6}
  \significado{s.}{capital da província}
\end{verbete}

\begin{verbete}{省俭}{sheng3jian3}{9,9}
  \significado{s.}{econômico; frugal}
  \significado{v.}{economizar}
\end{verbete}

\begin{verbete}{省力}{sheng3li4}{9,2}
  \significado{v.}{economizar esforço ou trabalho}
\end{verbete}

\begin{verbete}{省钱}{sheng3qian2}{9,10}
  \significado{v.}{economizar dinheiro}
\end{verbete}

\begin{verbete}{省却}{sheng3que4}{9,7}
  \significado{v.}{livrar-se (para economizar espaço); salvar}
\end{verbete}

\begin{verbete}{省心}{sheng3xin1}{9,4}
  \significado{adj.}{despreocupado}
  \significado{v.}{ser poupado de preocupações; despreocupar-se}
\end{verbete}

\begin{verbete}{省长}{sheng3zhang3}{9,4}
  \significado*{s.}{Governador; governador de uma província}
\end{verbete}

\begin{verbete}{圣诞节}{sheng4dan4jie2}{5,8,5}
  \significado*{s.}{Natal}
\end{verbete}

\begin{verbete}{圣地}{sheng4di4}{5,6}
  \significado{s.}{terra santa (de uma religião); lugar sagrado; santuário; cidade santa (como Jerusalém, Meca, etc.); centro de interesse histórico}
\end{verbete}

\begin{verbete}{胜利}{sheng4li4}{9,7}
  \significado[个]{s.}{vitória}
\end{verbete}

\begin{verbete}{胜算}{sheng4suan4}{9,14}
  \significado{s.}{probabilidade de sucesso; estratégia que garante o sucesso}
  \significado{v.}{ter certeza do sucesso}
\end{verbete}

\begin{verbete}{盛宴}{sheng4yan4}{11,10}
  \significado{s.}{celebração}
\end{verbete}

\begin{verbete}{失落}{shi1luo4}{5,12}
  \significado{s.}{frustração; decepção; perda}
  \significado{v.}{perder (algo); cair (algo); sentir uma sensação de perda}
\end{verbete}

\begin{verbete}{失眠}{shi1mian2}{5,10}
  \significado{s.}{insônia}
  \significado{v.}{ter insônia}
\end{verbete}

\begin{verbete}{失去}{shi1qu4}{5,5}
  \significado{v.}{perder}
\end{verbete}

\begin{verbete}{失望}{shi1wang4}{5,11}
  \significado{adj.}{desapontado}
  \significado{v.}{perder a esperança; desesperar}
\end{verbete}

\begin{verbete}{失意}{shi1yi4}{5,13}
  \significado{adj.}{desapontado, frustrado}
\end{verbete}

\begin{verbete}{师}{shi1}{6}[Radical 巾]
  \significado*{s.}{sobrenome Shi}
  \significado{s.}{professor; mestre; especialista; modelo; divisão do exército}
  \significado{v.}{despachar tropas}
\end{verbete}

\begin{verbete}{师傅}{shi1fu5}{6,12}
  \significado[个,位,名]{s.}{técnico; mestre-trabalhador; forma respeitosa de tratamento para homens mais velhos}
\end{verbete}

\begin{verbete}{诗词}{shi1ci2}{8,7}
  \significado{s.}{verso}
\end{verbete}

\begin{verbete}{诗句}{shi1ju4}{8,5}
  \significado[行]{s.}{verso; versículo}
\end{verbete}

\begin{verbete}{诗意}{shi1yi4}{8,13}
  \significado{adj.}{poético}
  \significado{s.}{poesia}
\end{verbete}

\begin{verbete}{十}{shi2}{2}[Radical 十][Kangxi 24]
  \significado{num.}{dez, 10; dezena}
\end{verbete}

\begin{verbete}{十分}{shi2fen1}{2,4}
  \significado{adv.}{muito; extremamente; totalmente; absolutamente}
\end{verbete}

\begin{verbete}{十足}{shi2zu2}{2,7}
  \significado{adj.}{amplo; completo; cento por cento; tom puro (de alguma cor)}
\end{verbete}

\begin{verbete}{时差}{shi2cha1}{7,9}
  \significado{s.}{diferença de tempo; \emph{jet lag}}
\end{verbete}

\begin{verbete}{时光}{shi2guang1}{7,6}
  \significado{s.}{tempo; época; período de tempo}
\end{verbete}

\begin{verbete}{时候}{shi2hou5}{7,10}
  \significado{adv.}{quando?}
  \significado{s.}{duração de tempo; momento; período; tempo}
\end{verbete}

\begin{verbete}{时间}{shi2jian1}{7,7}
  \significado{s.}{(conceito de, duração de, um ponto no) tempo}
\end{verbete}

\begin{verbete}{时刻}{shi2ke4}{7,8}
  \significado{adv.}{constantemente; sempre}
  \significado[个,段]{s.}{tempo; conjuntura; momento; período de tempo}
\end{verbete}

\begin{verbete}{时时}{shi2shi2}{7,7}
  \significado{adv.}{muitas vezes; constantemente}
\end{verbete}

\begin{verbete}{实力}{shi2li4}{8,2}
  \significado{s.}{força}
\end{verbete}

\begin{verbete}{实现}{shi2xian4}{8,8}
  \significado{v.}{alcançar, implementar, constatar}
\end{verbete}

\begin{verbete}{实在}{shi2zai4}{8,6}
  \significado{adv.}{realmente; verdadeiramente; de fato; na verdade}
\end{verbete}

\begin{verbete}{食品}{shi2pin3}{9,9}
  \significado[种]{s.}{comida; alimento; produtos alimentícios; provisões}
\end{verbete}

\begin{verbete}{食堂}{shi2tang2}{9,11}
  \significado[个,间]{s.}{sala de jantar}
\end{verbete}

\begin{verbete}{食物}{shi2wu4}{9,8}
  \significado[种]{s.}{comida}
\end{verbete}

\begin{verbete}{屎}{shi3}{9}[Radical 尸]
  \significado{s.}{fezes, excrementos; (forma ligada) secreção (do ouvido, olho, etc.)}
\end{verbete}

\begin{verbete}{世代}{shi4dai4}{5,5}
  \significado{adv.}{por muitas gerações}
  \significado{s.}{geração; era}
\end{verbete}

\begin{verbete}{世界}{shi4jie4}{5,9}
  \significado[个]{s.}{mundo}
\end{verbete}

\begin{verbete}{世界杯}{shi4jie4bei1}{5,9,8}
  \significado*{s.}{Copa do Mundo}
\end{verbete}

\begin{verbete}{世锦赛}{shi4jin3sai4}{5,13,14}
  \significado*{s.}{Campeonato Mundial}
\end{verbete}

\begin{verbete}{市场}{shi4chang3}{5,6}
  \significado{s.}{mercado (também no abstrato)}
\end{verbete}

\begin{verbete}{市区}{shi4qu1}{5,4}
  \significado{s.}{centro da cidade; distrito urbano}
\end{verbete}

\begin{verbete}{市中心}{shi4zhong1xin1}{5,4,4}
  \significado{s.}{centro da cidade}
\end{verbete}

\begin{verbete}{式}{shi4}{6}[Radical 弋]
  \significado{s.}{tipo; forma; padrão; estilo}
\end{verbete}

\begin{verbete}{事}{shi4}{8}[Radical 亅]
  \significado[件,桩,回]{s.}{coisa; assunto; item; matéria; coisa de trabalho; caso}
\end{verbete}

\begin{verbete}{事故}{shi4gu4}{8,9}
  \significado[桩,起,次]{s.}{acidente}
\end{verbete}

\begin{verbete}{事儿}{shi4r5}{8,2}
  \significado[件,桩]{s.}{o emprego; negócio; afazeres; assunto que precisa ser resolvido; matéria}
\end{verbete}

\begin{verbete}{视角}{shi4jiao3}{8,7}
  \significado{s.}{ângulo do qual se observa um objeto; (fig.) perspectiva, ponto de vista, quadro de referência; (cinematografia) ângulo da câmera; (percepção visual) ângulo visual (o ângulo que um objeto visto subtende no olho); (fotografia) ângulo de visão}
\end{verbete}

\begin{verbete}{视频}{shi4pin2}{8,13}
  \significado{s.}{vídeo}
\end{verbete}

\begin{verbete}{试}{shi4}{8}[Radical 言]
  \significado{s.}{exame; experimento; prova; teste}
  \significado{v.}{experimentar; provar; testar}
\end{verbete}

\begin{verbete}{室}{shi4}{9}[Radical 宀]
  \significado*{s.}{sobrenome Shi}
  \significado{s.}{família ou clã; cova; cômodo; bainha; unidade de trabalho}
\end{verbete}

\begin{verbete}{是}{shi4}{9}[Radical 日]
  \significado{adj.}{correto, certo, verdadeiro; (reconhecimento respeitoso de um comando) muito bem}
  \significado{adv.}{(advérbio para afirmação enfática)}
  \significado{v.}{ser (somente seguido por substantivos)}
\end{verbete}

\begin{verbete}{是的}{shi4de5}{9,8}
  \significado{adv.}{sim; está certo}
\end{verbete}

\begin{verbete}{适合}{shi4he2}{9,6}
  \significado{v.}{servir (uma roupa); adequar}
\end{verbete}

\begin{verbete}{收}{shou1}{6}[Radical 攴]
  \significado{expr.}{aos cuidados de (usado na linha de endereço após o nome)}
  \significado{v.}{receber; aceitar; coletar; colher; guardar}
\end{verbete}

\begin{verbete}{收到}{shou1dao4}{6,8}
  \significado{v.}{receber}
\end{verbete}

\begin{verbete}{收据}{shou1ju4}{6,11}
  \significado[张]{s.}{recibo; \emph{voucher}}
\end{verbete}

\begin{verbete}{收看}{shou1kan4}{6,9}
  \significado{v.}{assistir (a um programa de TV)}
\end{verbete}

\begin{verbete}{收敛}{shou1lian3}{6,11}
  \significado{v.}{diminuir, desaparecer, fazer desaparecer; exercer restrição, conter (alegria, arrogância, etc.); constringir; (matemática) convergir}
\end{verbete}

\begin{verbete}{收买}{shou1mai3}{6,6}
  \significado{v.}{subornar; comprar}
\end{verbete}

\begin{verbete}{手}{shou3}{4}[Radical 手][Kangxi 64]
  \significado{adj.}{conveniente}
  \significado{clas.}{de habilidade}
  \significado[双,只]{s.}{mão; pessoa envolvida em certos tipos de trabalho; pessoa qualificada em certos tipos de trabalho}
  \significado{v.}{segurar (formal)}
\end{verbete}

\begin{verbete}{手臂}{shou3bi4}{4,17}
  \significado{s.}{braço}
\end{verbete}

\begin{verbete}{手边}{shou3bian1}{4,5}
  \significado{adv.}{à mão; na mão}
\end{verbete}

\begin{verbete}{手工}{shou3gong1}{4,3}
  \significado{s.}{trabalho manual; artesanato}
\end{verbete}

\begin{verbete}{手工艺人}{shou3gong1 yi4ren2}{4,3,4,2}
  \significado{s.}{artesão}
\end{verbete}

\begin{verbete}{手机}{shou3ji1}{4,6}
  \significado[部,支]{s.}{telefone celular, móvel}
\end{verbete}

\begin{verbete}{手刹}{shou3sha1}{4,8}
  \significado{s.}{freio de mão}
\end{verbete}

\begin{verbete}{手指}{shou3zhi3}{4,9}
  \significado[个,只]{s.}{dedo}
\end{verbete}

\begin{verbete}{守门员}{shou3men2yuan2}{6,3,7}
  \significado{s.}{goleiro}
\end{verbete}

\begin{verbete}{首席执行官}{shou3xi2 zhi2xing2 guan1}{9,10,6,6,8}
  \significado{s.}{\emph{chief executive officer}, CEO}
\end{verbete}

\begin{verbete}{首相}{shou3xiang4}{9,9}
  \significado*{s.}{Primeiro-Ministro (Japão, UK, etc.)}
\end{verbete}

\begin{verbete}{掱}{shou3}{12}[Radical 手]
  \variante{手}
\end{verbete}

\begin{verbete}{受到}{shou4dao4}{8,8}
  \significado{v.}{receber (elogio, educação, punição, etc.); ser elogiado, educado, punido, etc.}
\end{verbete}

\begin{verbete}{受得了}{shou4de5liao3}{8,11,2}
  \significado{v.}{suportar, aguentar}
\end{verbete}

\begin{verbete}{受限}{shou4xian4}{8,8}
  \significado{v.}{ser limitado; ser restrito; ser constrangido}
\end{verbete}

\begin{verbete}{瘦}{shou4}{14}[Radical 疒]
  \significado{adj.}{magro; emagrecido; apertado (roupas); improdutivo (terras); magro (carne)}
  \significado{v.}{perder peso}
\end{verbete}

\begin{verbete}{书}{shu1}{4}[Radical 乙]
  \significado[本,册,部]{s.}{livro; carta; documento}
\end{verbete}

\begin{verbete}{书店}{shu1dian4}{4,8}
  \significado[家]{s.}{livraria}
\end{verbete}

\begin{verbete}{书记}{shu1ji5}{4,5}
  \significado{s.}{secretário (chefe de um ramo de um partido socialista ou comunista); atendente; balconista; escriturário}
\end{verbete}

\begin{verbete}{舒服}{shu1fu5}{12,8}
  \significado{adj.}{estar confortável; bem disposto; (sentir-se) bem}
\end{verbete}

\begin{verbete}{熟练}{shu2lian4}{15,8}
  \significado{adj.}{especializado; proficiente; qualificado; habilidoso}
\end{verbete}

\begin{verbete}{熟悉}{shu2xi1}{15,11}
  \significado{v.}{conhecer bem; estar familiarizado com}
\end{verbete}

\begin{verbete}{属}{shu3}{12}[Radical 尸]
  \significado{s.}{categoria; gênero (taxonomia); familiares; dependentes}
  \significado{v.}{pertencer; subordinar; nascer no ano do signo de (um dos doze animais zodiacais); provar ser; constituir}
  \veja{属}{zhu3}
\end{verbete}

\begin{verbete}{属于}{shu3yu2}{12,3}
  \significado{v.}{ser classificado como; pertencer a; fazer parte de}
\end{verbete}

\begin{verbete}{暑假}{shu3jia4}{12,11}
  \significado[个]{s.}{férias de verão}
\end{verbete}

\begin{verbete}{薯}{shu3}{16}[Radical 艸]
  \significado{s.}{batata; inhame}
\end{verbete}

\begin{verbete}{束}{shu4}{7}[Radical 木]
  \significado*{s.}{sobrenome Shu}
  \significado{clas.}{para cachos, feixes, feixes de luz, etc.}
  \significado{s.}{monte; pacote; maço; feixe; cacho}
  \significado{v.}{vincular; controlar}
\end{verbete}

\begin{verbete}{束腰}{shu4yao1}{7,13}
  \significado{s.}{cinto, cinta, cinturão}
\end{verbete}

\begin{verbete}{树}{shu4}{9}[Radical 木]
  \significado[棵]{s.}{árvore}
  \significado{v.}{cultivar}
\end{verbete}

\begin{verbete}{树莓}{shu4mei2}{9,10}
  \significado{s.}{framboesa}
\end{verbete}

\begin{verbete}{树木}{shu4mu4}{9,4}
  \significado{s.}{árvore}
\end{verbete}

\begin{verbete}{树叶}{shu4ye4}{9,5}
  \significado{s.}{folhas de árvores}
\end{verbete}

\begin{verbete}{数学}{shu4xue2}{13,8}
  \significado{s.}{matemática (disciplina)}
\end{verbete}

\begin{verbete}{刷子}{shua1zi5}{8,3}
  \significado[把]{s.}{pincel; escova; escovão}
\end{verbete}

\begin{verbete}{耍}{shua3}{9}[Radical 而]
  \significado{v.}{brincar com; empunhar; agir (legal, calmo, tranquilo, descolado, etc.); exibir (uma habilidade, o temperamento de alguém, etc.)}
\end{verbete}

\begin{verbete}{耍赖}{shua3lai4}{9,13}
  \significado{v.}{agir descaradamente; recusar -se a reconhecer que alguém perdeu o jogo ou fez uma promessa, etc.; agir como um idiota; agir como se algo nunca tivesse acontecido}
\end{verbete}

\begin{verbete}{摔}{shuai1}{14}[Radical 手]
  \significado{v.}{cair; cair e quebrar; partir}
\end{verbete}

\begin{verbete}{帅}{shuai4}{5}[Radical 巾]
  \significado*{s.}{sobrenome Shuai}
  \significado{adj.}{elegante; agradável à vista; gracioso; inteligente}
  \significado{interj.}{Legal!}
  \significado{s.}{comandante em chefe}
\end{verbete}

\begin{verbete}{双}{shuang1}{4}[Radical 又]
  \significado*{s.}{sobrenome Shuang}
  \significado{s.}{dobro; par; dupla; ambos; número par}
\end{verbete}

\begin{verbete}{双层床}{shuang1ceng2chuang2}{4,7,7}
  \significado{s.}{beliche}
\end{verbete}

\begin{verbete}{双打}{shuang1da3}{4,5}
  \significado[场]{s.}{duplas (em esportes)}
\end{verbete}

\begin{verbete}{双方同意}{shuang1fang1tong2yi4}{4,4,6,13}
  \significado{s.}{acordo bilateral}
\end{verbete}

\begin{verbete}{霜}{shuang1}{17}[Radical 雨]
  \significado{s.}{geada; pó branco ou creme espalhado por uma superfície; glacê; creme de pele}
\end{verbete}

\begin{verbete}{谁}{shui2}{10}[Radical 言]
  \significado{interr.}{quem?}
  \veja{谁}{shei2}
\end{verbete}

\begin{verbete}{水}{shui3}{4}[Radical 水][Kangxi 85]
  \significado*{s.}{sobrenome Shui}
  \significado{clas.}{para número de lavagens}
  \significado{s.}{água; líquido; encargos ou receitas adicionais}
\end{verbete}

\begin{verbete}{水边}{shui3bian1}{4,5}
  \significado{s.}{beira d'água; beira-mar; costa (de mar, lago ou rio)}
\end{verbete}

\begin{verbete}{水波}{shui3bo1}{4,8}
  \significado{s.}{ondulação (na água); onda}
\end{verbete}

\begin{verbete}{水槽}{shui3cao2}{4,15}
  \significado{s.}{pia (de cozinha)}
\end{verbete}

\begin{verbete}{水果}{shui3guo3}{4,8}
  \significado[个]{s.}{fruta}
\end{verbete}

\begin{verbete}{水饺}{shui3jiao3}{4,9}
  \significado{s.}{\emph{dumplings}; pastéis chineses cozidos}
\end{verbete}

\begin{verbete}{水灵}{shui3ling2}{4,7}
  \significado{adj.}{cheio de vida (sobre uma pessoa, etc.); úmido e brilhante (sobre os olhos); fresco (sobre frutas; etc.); brilhante; aparência saudável}
\end{verbete}

\begin{verbete}{水路}{shui3lu4}{4,13}
  \significado{s.}{hidrovia}
\end{verbete}

\begin{verbete}{水培}{shui3pei2}{4,11}
  \significado{v.}{cultivar plantas hidroponicamente}
\end{verbete}

\begin{verbete}{水平}{shui3ping2}{4,5}
  \significado{s.}{nível (de realização, etc.); padrão; nível horizontal}
\end{verbete}

\begin{verbete}{水平尺}{shui3ping2chi3}{4,5,4}
  \significado{s.}{nível espiritual}
\end{verbete}

\begin{verbete}{水平度}{shui3ping2 du4}{4,5,9}
  \significado{s.}{nivelamento}
\end{verbete}

\begin{verbete}{水平面}{shui3ping2mian4}{4,5,9}
  \significado{s.}{plano horizontal; nível-da-água; superfície horizontal}
\end{verbete}

\begin{verbete}{水平视差}{shui3ping2 shi4cha1}{4,5,8,9}
  \significado{s.}{paralaxe horizontal}
\end{verbete}

\begin{verbete}{水平仪}{shui3ping2yi2}{4,5,5}
  \significado{s.}{nível (dispositivo para determinar horizontal); nível espiritual; nível de topógrafo}
\end{verbete}

\begin{verbete}{水平以下}{shui3ping2 yi3xia4}{4,5,4,3}
  \significado{s.}{sub-nível}
\end{verbete}

\begin{verbete}{水平轴}{shui3ping2zhou2}{4,5,9}
  \significado{s.}{eixo horizontal}
\end{verbete}

\begin{verbete}{水瓶}{shui3 ping2}{4,10}
  \significado{s.}{garrada de água}
\end{verbete}

\begin{verbete}{水豚}{shui3tun2}{4,11}
  \significado{s.}{capivara}
\end{verbete}

\begin{verbete}{水污染}{shui3wu1ran3}{4,6,9}
  \significado{s.}{poluição da água}
\end{verbete}

\begin{verbete}{说}{shui4}{9}[Radical 言]
  \significado{v.}{persuadir}
  \veja{说}{shuo1}
\end{verbete}

\begin{verbete}{税}{shui4}{12}[Radical 禾]
  \significado{s.}{taxas; impostos}
\end{verbete}

\begin{verbete}{睡觉}{shui4jiao4}{13,9}
  \significado{v.}{ir para a cama; dormir; deitar-se}
\end{verbete}

\begin{verbete}{睡懒觉}{shui4lan3jiao4}{13,16,9}
  \significado{v.}{levantar-se tarde; passar o tempo a dormir}
\end{verbete}

\begin{verbete}{睡衣}{shui4yi1}{13,6}
  \significado{s.}{pijamas; roupas de dormir}
\end{verbete}

\begin{verbete}{顺}{shun4}{9}[Radical 頁]
  \significado{adj.}{correr bem; favorável}
\end{verbete}

\begin{verbete}{顺便}{shun4bian4}{9,9}
  \significado{adv.}{convenientemente; de passagem; sem muito esforço extra}
\end{verbete}

\begin{verbete}{顺畅}{shun4chang4}{9,8}
  \significado{adj.}{liso e sem obstáculos; fluente}
\end{verbete}

\begin{verbete}{顺从}{shun4cong2}{9,4}
  \significado{v.}{obedecer; submeter-se}
\end{verbete}

\begin{verbete}{顺当}{shun4dang5}{9,6}
  \significado{adv.}{suavemente}
\end{verbete}

\begin{verbete}{顺耳}{shun4'er3}{9,6}
  \significado{adj.}{agradável ao ouvido}
\end{verbete}

\begin{verbete}{顺境}{shun4jing4}{9,14}
  \significado{s.}{circunstâncias favoráveis}
\end{verbete}

\begin{verbete}{顺利}{shun4li4}{9,7}
  \significado{adv.}{suavemente; sem problemas}
\end{verbete}

\begin{verbete}{顺水}{shun4shui3}{9,4}
  \significado{v.}{ir com o fluxo}
\end{verbete}

\begin{verbete}{顺心}{shun4xin1}{9,4}
  \significado{adj.}{satisfatório; satisfeito}
\end{verbete}

\begin{verbete}{顺叙}{shun4xu4}{9,9}
  \significado{s.}{narrativa cronológica}
\end{verbete}

\begin{verbete}{顺延}{shun4yan2}{9,6}
  \significado{v.}{adiar; procrastinar}
\end{verbete}

\begin{verbete}{顺眼}{shun4yan3}{9,11}
  \significado{adj.}{agradável aos olhos}
\end{verbete}

\begin{verbete}{顺嘴}{shun4zui3}{9,16}
  \significado{v.}{deixar escapar (sem pensar); ler suavemente (texto); adequar-se  ao gosto (comida)}
\end{verbete}

\begin{verbete}{说}{shuo1}{9}[Radical 言]
  \significado{s.}{uma teoria (normalmente o último caractere, como em 日心说, teoria heliocêntrica)}
  \significado{v.}{falar; dizer; explicar; contar}
  \veja{说}{shui4}
\end{verbete}

\begin{verbete}{说好}{shuo1hao3}{9,6}
  \significado{v.}{chegar a um acordo; concluir negociações}
\end{verbete}

\begin{verbete}{说谎}{shuo1huang3}{9,11}
  \significado{v.+compl.}{mentir; contar uma mentira}
\end{verbete}

\begin{verbete}{说理}{shuo1li3}{9,11}
  \significado{v.}{racionalizar; discutir logicamente}
\end{verbete}

\begin{verbete}{说完}{shuo1-wan2}{9,7}
  \significado{expr.}{acabar/terminar palavras}
\end{verbete}

\begin{verbete}{丝}{si1}{5}[Radical 一]
  \significado{adj.}{filiforme; delgado como um fio; que se assemelha a um fio}
  \significado{clas.}{um traço (de fumaça, etc.), um pouquinho, etc.}
  \significado{s.}{seda;  (cozinha) pedaços ou tiras de julienne, tiras cortadas finas}
\end{verbete}

\begin{verbete}{司机}{si1ji1}{5,6}
  \significado{s.}{condutor; motorista; chofer}
\end{verbete}

\begin{verbete}{私人}{si1ren2}{7,2}
  \significado{adj.}{privado; interpessoal}
  \significado[些]{s.}{alguém com quem se tem um relacionamento pessoal próximo}
\end{verbete}

\begin{verbete}{私人信件}{si1ren2 xin4jian4}{7,2,9,6}
  \significado{s.}{carta pessoal}
\end{verbete}

\begin{verbete}{私人钥匙}{si1ren2yao4shi5}{7,2,9,11}
  \significado{s.}{criptografia:~chave privada}
\end{verbete}

\begin{verbete}{私人诊所}{si1ren2 zhen3suo3}{7,2,7,8}
  \significado[些]{s.}{clínica privada}
\end{verbete}

\begin{verbete}{私生活}{si1sheng1huo2}{7,5,9}
  \significado{s.}{vida privada}
\end{verbete}

\begin{verbete}{私自}{si1zi4}{7,6}
  \significado{adj.}{privado, pessoal;}
  \significado{adv.}{secretamente, sem aprovação explícita}
\end{verbete}

\begin{verbete}{思想}{si1xiang3}{9,13}
  \significado[个]{s.}{pensamento; ideia; ideologia}
\end{verbete}

\begin{verbete}{斯巴达}{si1ba1da2}{12,4,6}
  \significado*{s.}{Esparta}
\end{verbete}

\begin{verbete}{死}{si3}{6}[Radical 歹]
  \significado{adj.}{maldito; intransitável; inflexível; rígido; intransponível}
  \significado{adv.}{extremamente}
  \significado{v.}{morrer; falecer}
\end{verbete}

\begin{verbete}{死亡}{si3wang2}{6,3}
  \significado{s.}{morte}
  \significado{v.}{morrer}
\end{verbete}

\begin{verbete}{四}{si4}{5}[Radical 囗]
  \significado{num.}{quatro, 4}
\end{verbete}

\begin{verbete}{四川}{si4chuan1}{5,3}
  \significado*{s.}{Sichuan}
\end{verbete}

\begin{verbete}{四季分明}{si4ji4-fen1ming2}{5,8,4,8}
  \significado{expr.}{as quatro estações são muito distintas}
\end{verbete}

\begin{verbete}{四季如春}{si4ji4-ru2chun1}{5,8,6,9}
  \significado{expr.}{é primavera todo o ano; clima favorável durante todo o ano; quatro estações como a primavera}
\end{verbete}

\begin{verbete}{似曾相识}{si4ceng2xiang1shi2}{6,12,9,7}
  \significado{s.}{\emph{déjà vu} (a experiência de ver exatamente a mesma situação pela segunda vez); situação aparentemente familiar}
\end{verbete}

\begin{verbete}{寺}{si4}{6}[Radical 寸]
  \significado{s.}{Templo Budista; Mesquita}
\end{verbete}

\begin{verbete}{寺庙}{si4miao4}{6,8}
  \significado{s.}{templo; mosteiro; santuário}
\end{verbete}

\begin{verbete}{松木}{song1mu4}{8,4}
  \significado{s.}{pinheiro}
\end{verbete}

\begin{verbete}{宋}{song4}{7}[Radical 宀]
  \significado*{s.}{sobrenome Song}
  \significado{s.}{Dinastia Song (960-1279); Song das dinastias do sul (420-479)}
\end{verbete}

\begin{verbete}{送}{song4}{9}[Radical 辵]
  \significado{v.}{distribuir; entregar; dar; oferecer (alguma coisa como presente); enviar; remeter}
\end{verbete}

\begin{verbete}{㮸}{song4}{14}
  \variante{送}
\end{verbete}

\begin{verbete}{苏格兰}{su1ge2lan2}{7,10,5}
  \significado*{s.}{Escócia}
\end{verbete}

\begin{verbete}{宿舍}{su4she4}{11,8}
  \significado[间]{s.}{dormitório; quarto de dormir; hostel}
\end{verbete}

\begin{verbete}{痠}{suan1}{12}
  \significado{v.}{doer; estar dolorido}
  \variante{酸}
\end{verbete}

\begin{verbete}{酸}{suan1}{14}[Radical 酉]
  \significado{adj.}{ácido; avinagrado}
\end{verbete}

\begin{verbete}{酸辣汤}{suan1la4tang1}{14,14,6}
  \significado{s.}{sopa avinagrada e picante (prato)}
\end{verbete}

\begin{verbete}{算了}{suan4le5}{14,2}
  \significado{v.}{deixar; deixe estar; deixe passar; esqueça isso}
\end{verbete}

\begin{verbete}{算命}{suan4ming4}{14,8}
  \significado{s.}{cartomante}
  \significado{v.}{ler a sorte; fazer advinhações}
\end{verbete}

\begin{verbete}{尿}{sui1}{7}[Radical 尸]
  \significado{s.}{(coloquial) urina}
  \veja{尿}{niao4}
\end{verbete}

\begin{verbete}{虽}{sui1}{9}[Radical 虫]
  \significado{conj.}{no entanto; embora; mesmo se/embora}
\end{verbete}

\begin{verbete}{虽然}{sui1ran2}{9,12}
  \significado{conj.}{embora (frequentemente usado correlativamente com 可是, 但是, etc)}
  \veja{但是}{dan4shi4}
  \veja{可是}{ke3shi4}
\end{verbete}

\begin{verbete}{随便}{sui2bian4}{11,9}
  \significado{adj.}{à vontade; como queira; como desejar; casual; negligente; devasso}
  \significado{adv.}{aleatoriamente}
\end{verbete}

\begin{verbete}{随处}{sui2chu4}{11,5}
  \significado{adv.}{em qualquer lugar}
\end{verbete}

\begin{verbete}{随地}{sui2di4}{11,6}
  \significado{adv.}{qualquer lugar; todo lugar}
\end{verbete}

\begin{verbete}{随机存取存储器}{sui2ji1cun2qu3cun2chu3qi4}{11,6,6,8,6,12,16}
  \significado{s.}{RAM (\emph{random access memory})}
  \veja{内存}{nei4cun2}
  \veja{随机存取记忆体}{sui2ji1cun2qu3ji4yi4ti3}
\end{verbete}

\begin{verbete}{随机存取记忆体}{sui2ji1cun2qu3ji4yi4ti3}{11,6,6,8,5,4,7}
  \significado{s.}{RAM (\emph{random access memory})}
  \veja{内存}{nei4cun2}
  \veja{随机存取存储器}{sui2ji1cun2qu3cun2chu3qi4}
\end{verbete}

\begin{verbete}{随时}{sui2shi2}{11,7}
  \significado{adv.}{a qualquer momento; sempre que necessário}
\end{verbete}

\begin{verbete}{岁}{sui4}{6}[Radical 山]
  \significado{clas.}{para anos (de idade)}
  \significado{s.}{idade; ano (idade ou colheita)}
\end{verbete}

\begin{verbete}{碎}{sui4}{13}[Radical 石]
  \significado{adj.}{quebrato, fragmentado, espalhado; tagarela}
  \significado{v.}{(transitivo ou intransitivo) quebrar em pedaços, quebrar, desmoronar}
\end{verbete}

\begin{verbete}{隧道}{sui4dao4}{14,12}
  \significado{s.}{túnel}
\end{verbete}

\begin{verbete}{孙女}{sun1nv3}{6,3}
  \significado{s.}{filha do filho}
\end{verbete}

\begin{verbete}{孙武}{sun1wu3}{6,8}
  \significado*{s.}{Sun Wu, também conhecido por Sun Tzu (孙子), general, estrategista e filósofo autor do ``Arte da Guerra'' (孙子兵法)}
  \veja{孙子}{sun1zi3}
  \veja{孙子兵法}{sun1zi3 bing1fa3}
\end{verbete}

\begin{verbete}{孙子}{sun1zi3}{6,3}
  \significado*{s.}{Sun Tzu, também conhecido por Sun Wu (孙武), general, estrategista e filósofo autor do ``Arte da Guerra'' (孙子兵法)}
  \veja{孙武}{sun1wu3}
  \veja{孙子兵法}{sun1zi3 bing1fa3}
\end{verbete}

\begin{verbete}{孙子兵法}{sun1zi3 bing1fa3}{6,3,7,8}
  \significado*{s.}{``Arte da Guerra'', escrito por Sun Tzu (孫子)}
  \veja{孙武}{sun1wu3}
  \veja{孙子}{sun1zi3}
\end{verbete}

\begin{verbete}{孙子}{sun1zi5}{6,3}
  \significado{s.}{filho do filho}
\end{verbete}

\begin{verbete}{笋}{sun3}{10}[Radical 竹]
  \significado{s.}{broto de bambu}
\end{verbete}

\begin{verbete}{缩影卡片}{suo1ying3 ka3pian4}{14,15,5,4}
  \significado{s.}{cartão em miniatura}
\end{verbete}

\begin{verbete}{所以}{suo3yi3}{8,4}
  \significado{adv.}{portanto; então; como resultado}
  \significado{conj.}{por isso; como resultado; a razão porque}
\end{verbete}

\begin{verbete}{索性}{suo3xing4}{10,8}
  \significado{adv.}{poderia muito bem; simplesmente; apenas}
\end{verbete}

%%%%% EOF %%%%%


%%%
%%% T
%%%

\section*{T}\addcontentsline{toc}{section}{T}

\begin{entry}{T-恤}{[t]-xu4}{0,9}
  \definition{s.}{camiseta | pulôver | suéter}
\end{entry}

\begin{entry}{㐌}{ta1}{5}[Radical 乙]
  \variantof{它}
\end{entry}

\begin{entry}{他}{ta1}{5}[Radical 人][HSK 1]
  \definition{pron.}{ele | se, o, lhe | si, consigo, ele}
  \seeref{怹}{tan1}
\end{entry}

\begin{entry}{他的}{ta1 de5}{5,8}
  \definition{pron.}{dele}
\end{entry}

\begin{entry}{他妈的}{ta1ma1de5}{5,6,8}
  \definition{interj.}{Dane-se! | Foda-se!}
\end{entry}

\begin{entry}{他们}{ta1men5}{5,5}[HSK 1]
  \definition{pron.}{eles | se, os, lhes | si, consigo, eles}
\end{entry}

\begin{entry}{他们的}{ta1men5 de5}{5,5,8}
  \definition{pron.}{deles}
\end{entry}

\begin{entry}{它}{ta1}{5}[Radical 宀][HSK 2]
  \definition{pron.}{ele (para objetos inanimados) | se, o, lhe | si, consigo, eles}
\end{entry}

\begin{entry}{它们}{ta1 men5}{5,5}[HSK 2]
  \definition{pron.}{eles (para objetos inanimados) | se, os, lhes | si, consigo, eles}
\end{entry}

\begin{entry}{她}{ta1}{6}[Radical 女][HSK 1]
  \definition{pron.}{ela | se, a, lhe | si, consigo, ela}
\end{entry}

\begin{entry}{她的}{ta1 de5}{6,8}
  \definition{pron.}{dela}
\end{entry}

\begin{entry}{她们}{ta1men5}{6,5}[HSK 1]
  \definition{pron.}{elas | se, as, lhes | si, consigo, elas}
\end{entry}

\begin{entry}{她们的}{ta1men5 de5}{6,5,8}
  \definition{pron.}{delas}
\end{entry}

\begin{entry}{踏板}{ta4ban3}{15,8}
  \definition{s.}{pedal (em um carro, em um piano, etc.) |  apoio para os pés | estribo}
\end{entry}

\begin{entry}{台}{tai2}{5}[Radical 口]
  \definition*{s.}{sobrenome Tai}
  \definition{clas.}{para aparelhos e máquinas}
  \definition{s.}{estação de transmissão | contador | \emph{help desk} | suporte técnico | plataforma | terraço | tufão}
\end{entry}

\begin{entry}{台风}{tai2feng1}{5,4}
  \definition{s.}{tufão}
\end{entry}

\begin{entry}{台下}{tai2xia4}{5,3}
  \definition{s.}{platéia | fora do palco}
\end{entry}

\begin{entry}{抬杠}{tai2gang4}{8,7}
  \definition{v.+compl.}{discutir pelo prazer em discutir | discutir obstinadamente | brigar}
\end{entry}

\begin{entry}{太}{tai4}{4}[Radical 大][HSK 1]
  \definition{adv.}{excessivamente | demais | muito}
\end{entry}

\begin{entry}{太极拳}{tai4ji2quan2}{4,7,10}
  \definition*{s.}{Tai Chi Chuan, Taiji, T'aichi ou T'aichichuan; forma tradicional de exercício físico ou relaxamento}
\end{entry}

\begin{entry}{太空}{tai4kong1}{4,8}
  \definition{s.}{espaço sideral | espaço exterior}
\end{entry}

\begin{entry}{太平洋}{tai4ping2 yang2}{4,5,9}
  \definition*{s.}{Oceano Pacífico}
\end{entry}

\begin{entry}{太太}{tai4tai5}{4,4}[HSK 2]
  \definition[个,位]{s.}{esposa | madame| mulher casada}
\end{entry}

\begin{entry}{太阳窗}{tai4yang2chuang1}{4,6,12}
  \definition{s.}{teto solar (de veículos)}
\end{entry}

\begin{entry}{太阳灯}{tai4yang2deng1}{4,6,6}
  \definition{s.}{lâmpada solar (com células fotovoltaicas)}
\end{entry}

\begin{entry}{太阳风}{tai4yang2feng1}{4,6,4}
  \definition{s.}{vento solar}
\end{entry}

\begin{entry}{太阳镜}{tai4yang2jing4}{4,6,16}
  \definition{s.}{óculos de sol}
\end{entry}

\begin{entry}{太阳日}{tai4yang2ri4}{4,6,4}
  \definition{s.}{dia solar}
\end{entry}

\begin{entry}{太阳穴}{tai4yang2xue2}{4,6,5}
  \definition{s.}{têmpora (nas laterais da cabeça humana)}
\end{entry}

\begin{entry}{太阳翼}{tai4yang2yi4}{4,6,17}
  \definition{s.}{painel solar}
\end{entry}

\begin{entry}{太阳雨}{tai4yang2yu3}{4,6,8}
  \definition{s.}{banho de sol}
\end{entry}

\begin{entry}{太阳}{tai4yang5}{4,6}[HSK 2]
  \definition[个]{s.}{sol | abreviação de 太阳穴}
  \seeref{太阳穴}{tai4yang2xue2}
\end{entry}

\begin{entry}{态度}{tai4du5}{8,9}[HSK 2]
  \definition[个]{s.}{maneira | comportamento | atitude | atitude | abordagem}
\end{entry}

\begin{entry}{贪婪}{tan1lan2}{8,11}
  \definition{adj.}{avaro | ambicioso | voraz | insaciável}
\end{entry}

\begin{entry}{怹}{tan1}{9}[Radical 心]
  \definition{pron.}{ele, ela (cortês, em oposição a 他)}
  \seeref{他}{ta1}
\end{entry}

\begin{entry}{谈话}{tan2hua4}{10,8}
  \definition[次]{s.}{conversa | fala | papo | declaração}
  \definition{v.+compl.}{conversar | falar | declarar}
\end{entry}

\begin{entry}{谈恋爱}{tan2lian4'ai4}{10,10,10}
  \definition{v.}{namorar | apaixonar-se}
\end{entry}

\begin{entry}{坦克}{tan3ke4}{8,7}
  \definition{s.}{(empréstimo linguístico) tanque (veículo militar)}
\end{entry}

\begin{entry}{探亲}{tan4qin1}{11,9}
  \definition{v.+compl.}{ir para casa para visitar a família}
\end{entry}

\begin{entry}{碳}{tan4}{14}[Radical 石]
  \definition{s.}{carbono (elemento químico)}
\end{entry}

\begin{entry}{汤}{tang1}{6}[Radical 水]
  \definition*{s.}{sobrenome Tang}
  \definition{s.}{sopa | caldo | decocção de ervas medicinais | água quente ou fervente | água em que algo foi fervido}
  \seeref{汤}{shang1}
\end{entry}

\begin{entry}{唐人街}{tang2ren2 jie1}{10,2,12}
  \definition*{s.}{Bairro Chinês | \emph{Chinatown}}
  \seealsoref{中国城}{zhong1guo2cheng2}
\end{entry}

\begin{entry}{糖}{tang2}{16}[Radical 米]
  \definition[颗,块]{s.}{açúcar | doces}
\end{entry}

\begin{entry}{糖醋鱼}{tang2cu4yu2}{16,15,8}
  \definition{s.}{peixe guisado em molho agridoce (prato)}
\end{entry}

\begin{entry}{倘或}{tang3huo4}{10,8}
  \definition{conj.}{se | supondo que | no caso}
\end{entry}

\begin{entry}{倘若}{tang3ruo4}{10,8}
  \definition{conj.}{se | supondo que | no caso}
\end{entry}

\begin{entry}{倘使}{tang3shi3}{10,8}
  \definition{conj.}{se | supondo que | no caso}
\end{entry}

\begin{entry}{滔天}{tao1tian1}{13,4}
  \definition{adj.}{(ondas, raiva, desastres, crimes, etc.) imponente, avassalador, imenso}
\end{entry}

\begin{entry}{逃}{tao2}{9}[Radical 辵]
  \definition{v.}{escapar | fugir}
\end{entry}

\begin{entry}{桃}{tao2}{10}[Radical 木]
  \definition{s.}{pêssego}
\end{entry}

\begin{entry}{讨论}{tao3lun4}{5,6}[HSK 2]
  \definition{v.}{discutir | falar sobre}
\end{entry}

\begin{entry}{讨生活}{tao3sheng1huo2}{5,5,9}
  \definition{v.}{ganhar a vida}
\end{entry}

\begin{entry}{套}{tao4}{10}[Radical 大][HSK 2]
  \definition{clas.}{para conjuntos, coleções}
  \definition{s.}{cobertura | fórmula | laço de corda}
  \definition{v.}{cobrir | envolver | intercalar | sobrepor}
\end{entry}

\begin{entry}{套问}{tao4wen4}{10,6}
  \definition{s.}{retórica}
  \definition{v.}{descobrir por meio de questionamento indireto diplomático}
\end{entry}

\begin{entry}{特别}{te4bie2}{10,7}[HSK 2]
  \definition{adj.}{especial | paricular | incomum}
  \definition{adv.}{especialmente | particularmente | propositalmente}
\end{entry}

\begin{entry}{特地}{te4di4}{10,6}
  \definition{adv.}{especialmente | propositalmente}
\end{entry}

\begin{entry}{特点}{te4dian3}{10,9}[HSK 2]
  \definition[个]{s.}{característica | peculiaridade | característica distintiva}
\end{entry}

\begin{entry}{特技}{te4ji4}{10,7}
  \definition{s.}{efeito especial | dublê}
\end{entry}

\begin{entry}{疼}{teng2}{10}[Radical 疒][HSK 2]
  \definition{adj.}{dolorido | doído}
  \definition{v.}{doer | amar ternamente}
\end{entry}

\begin{entry}{梯恩梯}{ti1'en1ti1}{11,10,11}
  \definition{s.}{(empréstimo linguístico) TNT, trinitrotolueno}
\end{entry}

\begin{entry}{踢}{ti1}{15}[Radical 足]
  \definition{v.}{chutar | jogar (por exemplo, futebol) | dar pontapés em}
\end{entry}

\begin{entry}{踢爆}{ti1bao4}{15,19}
  \definition{v.}{expor | revelar}
\end{entry}

\begin{entry}{踢蹋舞}{ti1ta4wu3}{15,17,14}
  \definition{s.}{sapateado | passo de dança}
\end{entry}

\begin{entry}{提}{ti2}{12}[Radical 手][HSK 2]
  \definition*{s.}{sobrenome Ti}
  \definition{s.}{concha | traço ascendente (em caracteres chineses)}
  \definition{v.}{carregar (na mão com o braço para baixo) | levantar | elevar | promover | avançar | mudar para um momento anterior | mover uma data para a frente | trazer à tona | apresentar | extrair | tirar | trazer | entregar | mencionar | referir-se a}
\end{entry}

\begin{entry}{提出}{ti2 chu1}{12,5}[HSK 2]
  \definition{v.}{levantar | propor | expor | apresentar}
\end{entry}

\begin{entry}{提到}{ti2 dao4}{12,8}[HSK 2]
  \definition{v.}{mencionar | referir-se a | levantar (assunto)}
\end{entry}

\begin{entry}{提高}{ti2gao1}{12,10}[HSK 2]
  \definition{v.}{melhorar | aumentar | elevar}
\end{entry}

\begin{entry}{提及}{ti2ji2}{12,3}
  \definition{v.}{mencionar | levantar (um assunto) | chamar a atenção de alguém}
\end{entry}

\begin{entry}{提升}{ti2sheng1}{12,4}
  \definition{v.}{promover (para uma posição de classificação mais alta) | levantar | içar | (figurativo) elevar, levantar, melhorar}
\end{entry}

\begin{entry}{题}{ti2}{15}[Radical 頁][HSK 2]
  \definition*{s.}{sobrenome Ti}
  \definition[道]{s.}{assunto | título | tópico | problema}
  \definition{v.}{inscrever | escrever}
\end{entry}

\begin{entry}{体内}{ti3nei4}{7,4}
  \definition{adj.}{dentro do corpo | \emph{in vivo} (versus \emph{in vitro} | interno a}
\end{entry}

\begin{entry}{体验}{ti3yan4}{7,10}
  \definition{v.}{vivenciar | experimentar por si mesmo}
\end{entry}

\begin{entry}{体育}{ti3yu4}{7,8}[HSK 2]
  \definition{s.}{treinamento físico | esportes | atividades esportivas}
\end{entry}

\begin{entry}{体育场}{ti3 yu4 chang3}{7,8,6}[HSK 2]
  \definition[个,座]{s.}{estádio | campo de esportes}
\end{entry}

\begin{entry}{体育馆}{ti3 yu4 guan3}{7,8,11}[HSK 2]
  \definition[个]{s.}{ginásio | estádio}
\end{entry}

\begin{entry}{天}{tian1}{4}[Radical 大][HSK 1]
  \definition{s.}{dia | céu | paraíso}
\end{entry}

\begin{entry}{天才}{tian1cai2}{4,3}
  \definition{adj.}{talentoso | superdotado | genial}
  \definition{s.}{talento | dom | gênio}
\end{entry}

\begin{entry}{天鹅}{tian1'e2}{4,12}
  \definition{s.}{cisne}
\end{entry}

\begin{entry}{天公}{tian1gong1}{4,4}
  \definition{s.}{céu, paraíso | senhor do céu}
\end{entry}

\begin{entry}{天花板}{tian1hua1ban3}{4,7,8}
  \definition{s.}{teto}
\end{entry}

\begin{entry}{天气}{tian1qi4}{4,4}[HSK 1]
  \definition{s.}{clima, tempo}
\end{entry}

\begin{entry}{天然}{tian1ran2}{4,12}
  \definition{adj.}{natural}
\end{entry}

\begin{entry}{天上}{tian1 shang4}{4,3}[HSK 2]
  \definition{s.}{o céu | paraíso}
\end{entry}

\begin{entry}{天使}{tian1shi3}{4,8}
  \definition{s.}{anjo}
\end{entry}

\begin{entry}{天堂}{tian1tang2}{4,11}
  \definition{s.}{paraíso, céu}
\end{entry}

\begin{entry}{天天}{tian1tian1}{4,4}
  \definition{adv.}{todo dia}
\end{entry}

\begin{entry}{天下}{tian1xia4}{4,3}
  \definition{s.}{terra sob o céu | o mundo todo | toda a China | reino}
\end{entry}

\begin{entry}{天择}{tian1ze2}{4,8}
  \definition{s.}{seleção natural}
\end{entry}

\begin{entry}{天柱}{tian1zhu4}{4,9}
  \definition{s.}{pilar celestial, que sustenta o céu}
\end{entry}

\begin{entry}{兲}{tian1}{6}[Radical 八]
  \variantof{天}
\end{entry}

\begin{entry}{田}{tian2}{5}[Radical 田][Kangxi 102]
  \definition*{s.}{sobrenome Tian}
  \definition[片]{s.}{fazenda | campo}
\end{entry}

\begin{entry}{田园}{tian2yuan2}{5,7}
  \definition{adj.}{bucólico}
  \definition{s.}{campo | interior | rural}
\end{entry}

\begin{entry}{钿}{tian2}{10}[Radical 金]
  \definition{s.}{(dialeto) moeda, dinheiro}
  \seeref{钿}{dian4}
\end{entry}

\begin{entry}{甜}{tian2}{11}[Radical 甘]
  \definition{adj.}{doce}
\end{entry}

\begin{entry}{甜酒}{tian2jiu3}{11,10}
  \definition{s.}{licor doce}
\end{entry}

\begin{entry}{甜菊}{tian2ju2}{11,11}
  \definition{s.}{estévia, arbusto cujas folhas produzem um substituto para o açúcar}
\end{entry}

\begin{entry}{甜品}{tian2pin3}{11,9}
  \definition{s.}{sobremesa}
\end{entry}

\begin{entry}{甜食}{tian2shi2}{11,9}
  \definition{s.}{doces | sobremesa}
\end{entry}

\begin{entry}{甜酸}{tian2suan1}{11,14}
  \definition{adj.}{agridoce}
\end{entry}

\begin{entry}{甜甜圈}{tian2tian2quan1}{11,11,11}
  \definition{s.}{rosquinha | \emph{doughnut}}
\end{entry}

\begin{entry}{甜筒}{tian2tong3}{11,12}
  \definition{s.}{sorvete de casquinha}
\end{entry}

\begin{entry}{甜头}{tian2tou5}{11,5}
  \definition{s.}{benefício | sabor doce (de poder, sucesso, etc.)}
\end{entry}

\begin{entry}{甜心}{tian2xin1}{11,4}
  \definition{s.}{querido}
\end{entry}

\begin{entry}{甜言}{tian2yan2}{11,7}
  \definition{s.}{boa conversa | palavras amáveis}
\end{entry}

\begin{entry}{甜玉米}{tian2 yu4mi3}{11,5,6}
  \definition{s.}{milho doce}
\end{entry}

\begin{entry}{甜稚}{tian2zhi4}{11,13}
  \definition{s.}{doce e inocente}
\end{entry}

\begin{entry}{条}{tiao2}{7}[Radical 木][HSK 2]
  \definition{clas.}{para coisas longas e finas (fita, rio, estrada, calças, etc.)}
  \definition{s.}{artigo | cláusula (de lei ou tratado) | item | faixa}
\end{entry}

\begin{entry}{条幅}{tiao2fu2}{7,12}
  \definition{s.}{faixa | banner | pergaminho de parede (para pintura ou caligrafia)}
\end{entry}

\begin{entry}{条贯}{tiao2guan4}{7,8}
  \definition{s.}{ordem | procedimentos | sequência | sistema}
\end{entry}

\begin{entry}{条件}{tiao2jian4}{7,6}[HSK 2]
  \definition[个]{s.}{circunstâncias | condição | fator | pré-requisito | qualificação | requisito}
\end{entry}

\begin{entry}{条例}{tiao2li4}{7,8}
  \definition{s.}{código de conduta | ordenanças | regulamentos | regras | estatutos}
\end{entry}

\begin{entry}{条目}{tiao2mu4}{7,5}
  \definition{s.}{cláusulas e subcláusulas (em documento formal) | verbete (em um dicionário, enciclopédia, etc.)}
\end{entry}

\begin{entry}{调律}{tiao2lv4}{10,9}
  \definition{v.}{afinar (por exemplo, um piano)}
\end{entry}

\begin{entry}{挑衅}{tiao3xin4}{9,11}
  \definition{s.}{provocação}
  \definition{v.}{provocar}
\end{entry}

\begin{entry}{跳}{tiao4}{13}[Radical 足]
  \definition{v.}{pular | saltar}
\end{entry}

\begin{entry}{跳挡}{tiao4dang3}{13,9}
  \definition{v.}{pular marcha (de um carro) | perder a marcha}
\end{entry}

\begin{entry}{跳电}{tiao4dian4}{13,5}
  \definition{v.}{desarmar (um disjuntor ou interruptor)}
\end{entry}

\begin{entry}{跳频}{tiao4pin2}{13,13}
  \definition{s.}{FHSS, \emph{Frequency-Hopping Spread Spectrum}, método de transmissão de sinais de rádio}
\end{entry}

\begin{entry}{跳伞}{tiao4san3}{13,6}
  \definition{s.}{paraquedas}
  \definition{v.}{saltar de paraquedas}
\end{entry}

\begin{entry}{跳绳}{tiao4sheng2}{13,11}
  \definition{v.}{pular corda}
\end{entry}

\begin{entry}{跳水}{tiao4shui3}{13,4}
  \definition{s.}{mergulho esportivo}
  \definition{v.}{mergulhar (na água) | cometer suicídio pulando na água | (figurativo, preços das ações, etc.) cair dramaticamente}
\end{entry}

\begin{entry}{跳跳糖}{tiao4tiao4tang2}{13,13,16}
  \definition{s.}{\emph{Pop Rocks}, \emph{popping candy}}
\end{entry}

\begin{entry}{跳舞}{tiao4wu3}{13,14}
  \definition{v.+compl.}{dançar}
\end{entry}

\begin{entry}{跳远}{tiao4yuan3}{13,7}
  \definition{v.+compl.}{salto em distância (atletismo)}
\end{entry}

\begin{entry}{跳蚤}{tiao4zao5}{13,9}
  \definition{s.}{pulga}
\end{entry}

\begin{entry}{铁}{tie3}{10}[Radical 金]
  \definition*{s.}{sobrenome Tie}
  \definition{adj.}{duro | forte | violento | inabalável | determinado | (gíria) apertado}
  \definition{s.}{ferro (metal) | arma}
\end{entry}

\begin{entry}{铁轨}{tie3gui3}{10,6}
  \definition[根]{s.}{trilho | trilho ferroviário}
\end{entry}

\begin{entry}{铁路}{tie3lu4}{10,13}
  \definition[条]{s.}{ferrovia}
\end{entry}

\begin{entry}{听}{ting1}{7}[Radical 口][HSK 1]
  \definition{clas.}{para bebidas enlatadas}
  \definition{s.}{lata de bebida (empréstimo linguístico, do inglês ``\emph{tin}'')}
  \definition{v.}{ouvir | escutar | obedecer}
\end{entry}

\begin{entry}{听到}{ting1dao4}{7,8}[HSK 1]
  \definition{v.}{ouvir | notar}
\end{entry}

\begin{entry}{听断}{ting1duan4}{7,11}
  \definition{v.}{ouvir e decidir | julgar (ou seja, ouvir e julgar em um tribunal)}
\end{entry}

\begin{entry}{听骨}{ting1gu3}{7,9}
  \definition{s.}{ossículos (do ouvido médio)}
  \seealsoref{听小骨}{ting1xiao3gu3}
\end{entry}

\begin{entry}{听会}{ting1hui4}{7,6}
  \definition{v.}{participar de uma reunião (e ouvir o que é discutido)}
\end{entry}

\begin{entry}{听见}{ting1 jian4}{7,4}[HSK 1]
  \definition{v.}{ouvir}
\end{entry}

\begin{entry}{听讲}{ting1 jiang3}{7,6}[HSK 2]
  \definition{v.+compl.}{assistir a uma palestra; ouvir uma conversa}
\end{entry}

\begin{entry}{听来}{ting1lai2}{7,7}
  \definition{v.}{ouvir de algum lugar | soar (antigo, estrangeiro, excitante, certo, etc.) | soar como se (ou seja, dar uma impressão ao ouvinte)}
\end{entry}

\begin{entry}{听力}{ting1li4}{7,2}
  \definition{s.}{audição | capacidade de compreensão oral}
\end{entry}

\begin{entry}{听力理解}{ting1li4li3jie3}{7,2,11,13}
  \definition{s.}{compreensão auditiva}
\end{entry}

\begin{entry}{听命}{ting1ming4}{7,8}
  \definition{v.}{obedecer ordens | receber ordens}
\end{entry}

\begin{entry}{听凭}{ting1ping2}{7,8}
  \definition{v.}{permitir (alguém a fazer o que desejar)}
\end{entry}

\begin{entry}{听说}{ting1 shuo1}{7,9}[HSK 2]
  \definition{v.}{ouvir dizer}
\end{entry}

\begin{entry}{听随}{ting1sui2}{7,11}
  \definition{v.}{permitir | obedecer}
\end{entry}

\begin{entry}{听戏}{ting1xi4}{7,6}
  \definition{v.}{assistir a uma ópera | ver uma ópera}
\end{entry}

\begin{entry}{听小骨}{ting1xiao3gu3}{7,3,9}
  \definition{s.}{ossículos (do ouvido médio)}
  \seealsoref{听骨}{ting1gu3}
\end{entry}

\begin{entry}{听写}{ting1xie3}{7,5}[HSK 1]
  \definition{s.}{ditado}
  \definition{v.}{transcrever música de ouvido | escrever (em um exercício de ditado)}
\end{entry}

\begin{entry}{聼}{ting1}{19}[Radical 耳]
  \variantof{听}
\end{entry}

\begin{entry}{亭}{ting2}{9}[Radical 亠]
  \definition{s.}{pavilhão | cabine | quiosque}
\end{entry}

\begin{entry}{停}{ting2}{11}[Radical 人][HSK 2]
  \definition{v.}{parar | estacionar (um carro)}
\end{entry}

\begin{entry}{停办}{ting2ban4}{11,4}
  \definition{v.}{cancelar | sair do negócio | desligar | terminar}
\end{entry}

\begin{entry}{停车}{ting2 che1}{11,4}[HSK 2]
  \definition{v.}{parar de trabalhar (uma máquina) | estacionar | parar (um veículo) | paralisar}
\end{entry}

\begin{entry}{停车场}{ting2 che1 chang3}{11,4,6}[HSK 2]
  \definition{s.}{parque de estacionamento}
\end{entry}

\begin{entry}{停当}{ting2dang5}{11,6}
  \definition{adj.}{realizado | preparado | assentado}
\end{entry}

\begin{entry}{停电}{ting2dian4}{11,5}
  \definition{s.}{corte de energia}
  \definition{v.}{ter uma falha de energia}
\end{entry}

\begin{entry}{停工}{ting2gong1}{11,3}
  \definition{v.}{parar de trabalhar | parar a produção}
\end{entry}

\begin{entry}{停火}{ting2huo3}{11,4}
  \definition{s.}{cessar-fogo}
  \definition{v.+compl.}{cessar fogo}
\end{entry}

\begin{entry}{停课}{ting2ke4}{11,10}
  \definition{v.}{fechar (escola) | parar as aulas}
\end{entry}

\begin{entry}{停留}{ting2liu2}{11,10}
  \definition{v.}{ficar em algum lugar temporariamente | demorar | permanecer}
\end{entry}

\begin{entry}{停息}{ting2xi1}{11,10}
  \definition{v.}{cessar | parar}
\end{entry}

\begin{entry}{停歇}{ting2xie1}{11,13}
  \definition{v.}{parar para descansar}
\end{entry}

\begin{entry}{停业}{ting2ye4}{11,5}
  \definition{v.}{cessar a negociação (temporária ou permanentemente) | fechar}
\end{entry}

\begin{entry}{停用}{ting2yong4}{11,5}
  \definition{v.}{desabilitar | descontinuar | parar de usar | suspender}
\end{entry}

\begin{entry}{停止}{ting2zhi3}{11,4}
  \definition{v.}{cessar | encerrar | parar}
\end{entry}

\begin{entry}{挺}{ting3}{9}[Radical 手][HSK 2]
  \definition{adj.}{ereto | fora do comum | direto}
  \definition{adv.}{bastante, ou melhor, bonito | muito (coloquial)}
  \definition{clas.}{para metralhadoras}
  \definition{v.}{endireitar (fisicamente) | sobressair (uma parte do corpo) | dar suporte | resistir}
\end{entry}

\begin{entry}{挺拔}{ting3ba2}{9,8}
  \definition{adj.}{alto e reto}
\end{entry}

\begin{entry}{挺杆}{ting3gan3}{9,7}
  \definition{s.}{tucho (peça de máquina)}
\end{entry}

\begin{entry}{挺过}{ting3guo4}{9,6}
  \definition{s.}{sobreviver}
\end{entry}

\begin{entry}{挺好}{ting3 hao3}{9,6}[HSK 2]
  \definition{adj.}{muito bom}
\end{entry}

\begin{entry}{挺进}{ting3jin4}{9,7}
  \definition{s.}{progresso | avanço}
  \definition{v.}{progredir | avançar}
\end{entry}

\begin{entry}{挺立}{ting3li4}{9,5}
  \definition{v.}{ficar ereto | ficar de pé}
\end{entry}

\begin{entry}{挺身}{ting3shen1}{9,7}
  \definition{v.}{endireitar as costas}
\end{entry}

\begin{entry}{挺尸}{ting3shi1}{9,3}
  \definition{v.}{(coloquial) dormir | (literalmente) ficar deitado duro como um cadáver}
\end{entry}

\begin{entry}{挺腰}{ting3yao1}{9,13}
  \definition{v.}{arquear as costas | endireitar as costas}
\end{entry}

\begin{entry}{挺住}{ting3zhu4}{9,7}
  \definition{v.}{permanecer firme | manter-se firme (diante da adversidade ou da dor)}
\end{entry}

\begin{entry}{通}{tong1}{10}[Radical 辵][HSK 2]
  \definition{clas.}{para cartas, telegramas, telefonemas, etc.}
  \definition{suf.}{especialista}
  \definition{v.}{ligar para | conseguir a ligação}
  \seeref{通}{tong4}
\end{entry}

\begin{entry}{通牒}{tong1die2}{10,13}
  \definition{s.}{nota diplomática}
\end{entry}

\begin{entry}{通观}{tong1guan1}{10,6}
  \definition{v.}{ter uma visão geral de algo}
\end{entry}

\begin{entry}{通过}{tong1guo4}{10,6}[HSK 2]
  \definition{adv.}{por meio de | através de | via}
  \definition{v.}{passar por | adotar (uma resolução), aprovar (legislação) | passar (em um teste)}
\end{entry}

\begin{entry}{通识}{tong1shi2}{10,7}
  \definition{s.}{conhecimento comum | erudição | conhecimento geral | amplamente conhecido}
\end{entry}

\begin{entry}{通知}{tong1zhi1}{10,8}[HSK 2]
  \definition[份,个,张]{s.}{aviso | circular}
  \definition{v.}{aconselhar | notificar | informar | dar aviso}
\end{entry}

\begin{entry}{同}{tong2}{6}[Radical 口]
  \definition{adj.}{junto}
  \definition{adv.}{junto com}
\end{entry}

\begin{entry}{同伙}{tong2huo3}{6,6}
  \definition[个]{s.}{cúmplice | colega}
\end{entry}

\begin{entry}{同流合污}{tong2liu2he2wu1}{6,10,6,6}
  \definition{expr.}{chafurdar na lama com alguém | seguir o mau exemplo dos outros}
\end{entry}

\begin{entry}{同情}{tong2qing2}{6,11}
  \definition{s.}{simpatia}
  \definition{v.}{simpatizar com}
\end{entry}

\begin{entry}{同时}{tong2shi2}{6,7}[HSK 2]
  \definition{conj.}{além disso}
  \definition{s.}{enquanto isso | ao mesmo tempo}
\end{entry}

\begin{entry}{同事}{tong2shi4}{6,8}[HSK 2]
  \definition{s.}{colega | colega de trabalho | companheiro}
\end{entry}

\begin{entry}{同屋}{tong2wu1}{6,9}
  \definition[个]{s.}{companheiro de quarto | colega de quarto}
\end{entry}

\begin{entry}{同性恋}{tong2xing4lian4}{6,8,10}
  \definition{s.}{homossexualidade | pessoa gay | amor gay}
\end{entry}

\begin{entry}{同学}{tong2xue2}{6,8}[HSK 1]
  \definition[位,个]{s.}{colega de classe | colega estudante}
\end{entry}

\begin{entry}{同砚}{tong2yan4}{6,9}
  \definition[位,个]{s.}{colega de classe | colega estudante}
\end{entry}

\begin{entry}{同样}{tong2 yang4}{6,10}[HSK 2]
  \definition{adj.}{igual | similar}
\end{entry}

\begin{entry}{同意}{tong2yi4}{6,13}
  \definition{v.}{concordar | aprovar | consentir}
\end{entry}

\begin{entry}{童年}{tong2nian2}{12,6}
  \definition{s.}{infância}
\end{entry}

\begin{entry}{通}{tong4}{10}[Radical 辵]
  \definition{clas.}{para uma atividade, tomada em sua totalidade (discurso de abuso, período de reprodução de música, bebedeira, etc.)}
  \seeref{通}{tong1}
\end{entry}

\begin{entry}{痛骂}{tong4ma4}{12,9}
  \definition{v.}{repreender severamente}
\end{entry}

\begin{entry}{偷}{tou1}{11}[Radical 人]
  \definition{adv.}{furtivamente}
  \definition{v.}{furtar | roubar}
\end{entry}

\begin{entry}{偷安}{tou1'an1}{11,6}
  \definition{v.}{buscar facilidade temporária}
\end{entry}

\begin{entry}{偷渡}{tou1du4}{11,12}
  \definition{s.}{contrabando | imigração ilegal | clandestino (em um navio)}
  \definition{v.}{executar um bloqueio | roubar através da fronteira internacional}
\end{entry}

\begin{entry}{偷窃}{tou1qie4}{11,9}
  \definition{v.}{furtar | roubar}
\end{entry}

\begin{entry}{偷情}{tou1qing2}{11,11}
  \definition{v.}{manter um caso de amor clandestino}
\end{entry}

\begin{entry}{偷税}{tou1shui4}{11,12}
  \definition{s.}{evasão fiscal}
\end{entry}

\begin{entry}{偷听}{tou1ting1}{11,7}
  \definition{v.}{bisbilhotar; monitorar (secretamente)}
\end{entry}

\begin{entry}{偷袭}{tou1xi2}{11,11}
  \definition{s.}{ataque surpresa}
  \definition{v.}{montar um ataque furtivo | invadir}
\end{entry}

\begin{entry}{偸}{tou1}{11}[Radical 亻]
  \variantof{偷}
\end{entry}

\begin{entry}{头}{tou2}{5}[Radical 大][HSK 2]
  \definition{clas.}{para suínos ou gado}
  \definition[个]{s.}{cabeça}
  \seeref{头}{tou5}
\end{entry}

\begin{entry}{头发}{tou2fa5}{5,5}[HSK 2]
  \definition{s.}{cabelo}
\end{entry}

\begin{entry}{头号}{tou2hao4}{5,5}
  \definition{adj.}{primeira classe | número um | \emph{top rank}}
\end{entry}

\begin{entry}{头脑风暴}{tou2nao3feng1bao4}{5,10,4,15}
  \definition{s.}{\emph{brainstorm}}
\end{entry}

\begin{entry}{头头}{tou2tou2}{5,5}
  \definition{s.}{chefe | o cabeça}
\end{entry}

\begin{entry}{头像}{tou2xiang4}{5,13}
  \definition{s.}{retrato | busto | avatar | imagem de perfil (computação)}
\end{entry}

\begin{entry}{投递}{tou2di4}{7,10}
  \definition{v.}{despachar | enviar}
\end{entry}

\begin{entry}{投票}{tou2piao4}{7,11}
  \definition{v.+compl.}{votar | depositar um voto}
\end{entry}

\begin{entry}{投资}{tou2zi1}{7,10}
  \definition{s.}{investimento}
  \definition{v.}{investir}
\end{entry}

\begin{entry}{投资风险}{tou2zi1feng1xian3}{7,10,4,9}
  \definition{s.}{risco de investimento}
\end{entry}

\begin{entry}{投资回报率}{tou2zi1hui2bao4lv4}{7,10,6,7,11}
  \definition{s.}{retorno sobre o investimento (ROI)}
\end{entry}

\begin{entry}{投资家}{tou2zi1jia1}{7,10,10}
  \definition{s.}{investidor}
  \seealsoref{投资人}{tou2zi1ren2}
  \seealsoref{投资者}{tou2zi1zhe3}
\end{entry}

\begin{entry}{投资人}{tou2zi1ren2}{7,10,2}
  \definition{s.}{investidor}
  \seealsoref{投资家}{tou2zi1jia1}
  \seealsoref{投资者}{tou2zi1zhe3}
\end{entry}

\begin{entry}{投资者}{tou2zi1zhe3}{7,10,8}
  \definition{s.}{investidor}
  \seealsoref{投资家}{tou2zi1jia1}
  \seealsoref{投资人}{tou2zi1ren2}
\end{entry}

\begin{entry}{透}{tou4}{10}[Radical 辵]
  \definition{adj.}{completo | total}
  \definition{adv.}{completamente | totalmente}
  \definition{v.}{aparecer | passar através | penetrar}
\end{entry}

\begin{entry}{透彻}{tou4che4}{10,7}
  \definition{adj.}{minucioso | incisivo | penetrante}
\end{entry}

\begin{entry}{透澈}{tou4che4}{10,15}
  \variantof{透彻}
\end{entry}

\begin{entry}{透顶}{tou4ding3}{10,8}
  \definition{adv.}{completamente}
\end{entry}

\begin{entry}{透过}{tou4guo4}{10,6}
  \definition{v.}{passar através | penetrar}
\end{entry}

\begin{entry}{透亮}{tou4liang4}{10,9}
  \definition{adj.}{brilhante | claro como cristal}
\end{entry}

\begin{entry}{透露}{tou4lu4}{10,21}
  \definition{v.}{divulgar | vazar | revelar}
\end{entry}

\begin{entry}{透明}{tou4ming2}{10,8}
  \definition{adj.}{transparente | (figurativo) transparente, aberto a escrutínio}
\end{entry}

\begin{entry}{透辟}{tou4pi4}{10,13}
  \definition{adj.}{incisivo | penetrante}
\end{entry}

\begin{entry}{透气}{tou4qi4}{10,4}
  \definition{v.}{respirar (sobre tecido, etc.) | fluir livremente (sobre ar) | respirar ar fresco | ventilar}
\end{entry}

\begin{entry}{透水}{tou4shui3}{10,4}
  \definition{adj.}{permeável}
  \definition{s.}{vazamento de água}
\end{entry}

\begin{entry}{透支}{tou4zhi1}{10,4}
  \definition{v.}{cheque especial (bancário) | saque a descoberto}
\end{entry}

\begin{entry}{头}{tou5}{5}[Radical 大]
  \definition{suf.}{sufixo para nomes}
  \seeref{头}{tou2}
\end{entry}

\begin{entry}{突然}{tu1ran2}{9,12}
  \definition{adv.}{de repente | abruptamente | inesperadamente}
\end{entry}

\begin{entry}{图}{tu2}{8}[Radical 囗]
  \definition[张]{s.}{diagrama | imagem | desenho | gráfico | mapa}
  \definition{v.}{planejar | esquematizar | tentar | perseguir | procurar}
\end{entry}

\begin{entry}{图片}{tu2 pian4}{8,4}[HSK 2]
  \definition[张,幅]{s.}{imagem | fotografia}
\end{entry}

\begin{entry}{图书馆}{tu2shu1guan3}{8,4,11}[HSK 1]
  \definition[家,个]{s.}{biblioteca}
\end{entry}

\begin{entry}{徒手}{tu2shou3}{10,4}
  \definition{adj.}{com as mãos vazias | desarmado | mão livre (desenho) | lutando mão-a-mão}
\end{entry}

\begin{entry}{土地}{tu3di4}{3,6}
  \definition[片,块]{s.}{terra | solo | território}
  \seeref{土地}{tu3di4}
\end{entry}

\begin{entry}{土地}{tu3di5}{3,6}
  \definition{s.}{deus local | \emph{genius loci} deidade protetora de um local}
  \seeref{土地}{tu3di4}
\end{entry}

\begin{entry}{土豆}{tu3dou4}{3,7}
  \definition[个,颗]{s.}{batata}
\end{entry}

\begin{entry}{土豆泥}{tu3dou4ni2}{3,7,8}
  \definition{s.}{purê de batatas}
\end{entry}

\begin{entry}{土鸡}{tu3ji1}{3,7}
  \definition{s.}{galinha caipira}
\end{entry}

\begin{entry}{吐}{tu3}{6}[Radical 口]
  \definition{v.}{cuspir | enviar (seda de um bicho-da-seda, cápsulas de flores de algodão etc.) | dizer | despejar (suas queixas)}
  \seeref{吐}{tu4}
\end{entry}

\begin{entry}{吐}{tu4}{6}[Radical 口]
  \definition{v.}{vomitar}
  \seeref{吐}{tu3}
\end{entry}

\begin{entry}{兔子}{tu4zi5}{8,3}
  \definition[只]{s.}{coelho | lebre}
\end{entry}

\begin{entry}{团队}{tuan2dui4}{6,4}
  \definition{s.}{equipe}
\end{entry}

\begin{entry}{团结}{tuan2jie2}{6,9}
  \definition{adj.}{unido}
  \definition{s.}{unidade | solidariedade}
  \definition{v.}{unir}
\end{entry}

\begin{entry}{推}{tui1}{11}[Radical 手][HSK 2]
  \definition{v.}{empurrar | girar um moinho ou uma pedra de amolar | moer | impulsionar | promover | avançar | estender | deduzir | inferir | declinar | empurrar para longe | deslocar | adiar | diferir | eleger | selecionar | escolher | ter em alta estima | elogiar muito}
\end{entry}

\begin{entry}{推迟}{tui1chi2}{11,7}
  \definition{v.}{adiar | deixar para mais tarde | tardar}
\end{entry}

\begin{entry}{推介}{tui1jie4}{11,4}
  \definition{s.}{promoção}
  \definition{v.}{promover | introduzir e recomendar}
\end{entry}

\begin{entry}{腿}{tui3}{13}[Radical 肉][HSK 2]
  \definition[条]{s.}{perna | osso do quadril}
\end{entry}

\begin{entry}{腿号}{tui3hao4}{13,5}
  \definition{s.}{anilha numerada (por exemplo, usada para identificar pássaros)}
  \seealsoref{腿号箍}{tui3hao4gu1}
\end{entry}

\begin{entry}{腿号箍}{tui3hao4gu1}{13,5,14}
  \definition{s.}{anilha numerada (por exemplo, usada para identificar pássaros)}
  \seealsoref{腿号}{tui3hao4}
\end{entry}

\begin{entry}{退休}{tui4xiu1}{9,6}
  \definition{v.+compl.}{aposentar-se}
\end{entry}

\begin{entry}{拖拉机}{tuo1la1ji1}{8,8,6}
  \definition[台]{s.}{trator}
\end{entry}

\begin{entry}{拖鞋}{tuo1xie2}{8,15}
  \definition[双,只]{s.}{chinelos | sandálias}
\end{entry}

\begin{entry}{脱毛}{tuo1mao2}{11,4}
  \definition{s.}{depilação}
  \definition{v.}{perder cabelo ou penas | depilar | fazer a barba}
\end{entry}

\begin{entry}{脱险}{tuo1xian3}{11,9}
  \definition{v.}{sair do perigo}
\end{entry}

\begin{entry}{鸵鸟}{tuo2niao3}{10,5}
  \definition{s.}{avestruz}
\end{entry}

\begin{entry}{唾骂}{tuo4ma4}{11,9}
  \definition{v.}{insultar | amaldiçoar}
\end{entry}

%%%%% EOF %%%%%


%%%%%%%%%%%%%%%%%%%% Não existem palavras com pinyin iniciado em "U"
%%%%%%%%%%%%%%%%%%%% Não existem palavras com pinyin iniciado em "V"
%%%
%%% W
%%%
\section*{W}
\addcontentsline{toc}{section}{W}

\begin{verbete}[wai4bian0]{外边}[5;5]
\begin{pronuncia}{wai4bian0}
\significado{p.l.}{ fora; por fora; exterior }
\end{pronuncia}
\end{verbete}

\begin{verbete}[wai4gong1]{外公}[5;4]
\begin{pronuncia}{wai4gong1}
\significado{s.}{ avô materno }
\end{pronuncia}
\end{verbete}

\begin{verbete}[wai4guo2]{外国}[5;8]
\begin{pronuncia}{wai4guo2}
\significado[个]{s.}{ país estrangeiro }
\end{pronuncia}
\end{verbete}

\begin{verbete}[wai4hao4]{外号}[5;5]
\begin{pronuncia}{wai4hao4}
\significado{s.}{ apelido }
\end{pronuncia}
\end{verbete}

\begin{verbete}[wai4mao4]{外贸}[5;9]
\begin{pronuncia}{wai4mao4}
\significado{s.}{ comércio exterior }
\end{pronuncia}
\end{verbete}

\begin{verbete}[wai4mian0]{外面}[5;9]
\begin{pronuncia}{wai4mian0}
\significado{p.l.}{ fora; por fora; exterior }
\end{pronuncia}
\end{verbete}

\begin{verbete}[wai4po2]{外婆}[5;11]
\begin{pronuncia}{wai4po2}
\significado{s.}{ avó materna }
\end{pronuncia}
\end{verbete}

\begin{verbete}[wai4shi4]{外事}[5;8]
\begin{pronuncia}{wai4shi4}
\significado{s.}{ assuntos ou relações exteriores }
\end{pronuncia}
\end{verbete}

\begin{verbete}[wai4sun1]{外孙}[5;6]
\begin{pronuncia}{wai4sun1}
\significado{s.}{ filho da filha }
\end{pronuncia}
\end{verbete}

\begin{verbete}[wai4sun1nv3]{外孙女}[5;6;3]
\begin{pronuncia}{wai4sun1nv3}
\significado{s.}{ filha da filha }
\end{pronuncia}
\end{verbete}

\begin{verbete}[wai4yu3]{外语}[5;9]
\begin{pronuncia}{wai4yu3}
\significado[门]{s.}{ língua estrangeira }
\end{pronuncia}
\end{verbete}

\begin{verbete}[wan1dou4]{豌豆}[15;7]
\begin{pronuncia}{wan1dou4}
\significado{s.}{ ervilha }
\end{pronuncia}
\end{verbete}

\begin{verbete}[wan2]{完}[7]
\begin{pronuncia}{wan2}
\significado{v.}{ acabar; terminar }
\end{pronuncia}
\end{verbete}

\begin{verbete}[wan2]{玩}[8]
\begin{pronuncia}{wan2}
\significado{v.}{ brincar; tocar (intrumento musical) }
\end{pronuncia}
\end{verbete}

\begin{verbete}[wanr2]{玩儿}[8;2]
\begin{pronuncia}{wanr2}
\significado{v.}{ divertir-se }
\end{pronuncia}
\end{verbete}

\begin{verbete}[wan3]{晚}[11]
\begin{pronuncia}{wan3}
\significado{adj.}{ tarde }
\end{pronuncia}
\end{verbete}

\begin{verbete}[wan3fan4]{晚饭}[11;7]
\begin{pronuncia}{wan3fan4}
\significado[份,顿,次,餐]{s.}{ jantar }
\end{pronuncia}
\end{verbete}

\begin{verbete}[wan3shang0]{晚上}[11;3]
\begin{pronuncia}{wan3shang0}
\significado{p.t.}{ noite; à noite }
\end{pronuncia}
\end{verbete}

\begin{verbete}[wan3]{碗}[13]
\begin{pronuncia}{wan3}
\significado[只,个]{n}{ tigela }
\significado{p.c.}{ tigelas }
\end{pronuncia}
\end{verbete}

\begin{verbete}[wan3zi0]{碗子}[13;3]
\begin{pronuncia}{wan3zi0}
\significado{n}{ tigela }
\end{pronuncia}
\end{verbete}

\begin{verbete}[wan4]{万}[3]
\begin{pronuncia}{wan4}
\significado{num.}{ dez mil; 10.000 }
\end{pronuncia}
\end{verbete}

\begin{verbete}[wang3]{往}[8]
\begin{pronuncia}{wang3}
\significado{prep.}{ para; em direção a }
\end{pronuncia}
\end{verbete}

\begin{verbete}[wang3qiu2]{网球}[6;11]
\begin{pronuncia}{wang3qiu2}
\significado[个]{s.}{ tênis (esporte); bola de tênis }
\end{pronuncia}
\end{verbete}

\begin{verbete}[wang4]{忘}[7]
\begin{pronuncia}{wang4}
\significado{v.}{ esquecer }
\end{pronuncia}
\end{verbete}

\begin{verbete}[wen1du4]{温度}[12;9]
\begin{pronuncia}{wen1du4}
\significado[个]{s.}{ temperatura }
\end{pronuncia}
\end{verbete}

\begin{verbete}[wei2]{喂}[12]
\begin{pronuncia}{wei2}
\significado{interj.}{ ei!; chamar atenção (alô, telefone) }
\end{pronuncia}
\begin{pronuncia}{wei4}
\significado*{}{ 喂\p{wei4} }
\end{pronuncia}
\end{verbete}

\begin{verbete}[wei4sheng1jian1]{卫生间}[3;5;7]
\begin{pronuncia}{wei4sheng1jian1}
\significado[间]{s.}{ banheiro; toilette }
\end{pronuncia}
\end{verbete}

\begin{verbete}[wei4]{为}[4]
\begin{pronuncia}{wei4}
\significado{prep.}{ para }
\end{pronuncia}
\end{verbete}

\begin{verbete}[wei4shen2me0]{为什么}[4;4;3]
\begin{pronuncia}{wei4shen2me0}
\significado{interr.}{ por que? }
\end{pronuncia}
\end{verbete}

\begin{verbete}[wei4]{位}[7]
\begin{pronuncia}{wei4}
\significado{p.c.}{ para pessoas (com cortesia) }
\end{pronuncia}
\end{verbete}

\begin{verbete}[wei4dao0]{味道}[8;12]
\begin{pronuncia}{wei4dao0}
\significado{s.}{ sabor }
\end{pronuncia}
\end{verbete}

\begin{verbete}[wei4]{喂}
\begin{pronuncia}{wei4}
\significado{interj.}{ ei!; chamar atenção (alô, telefone) }
\end{pronuncia}
\begin{pronuncia}{wei2}
\significado*{}{ 喂\p{wei2} }
\end{pronuncia}
\end{verbete}

\begin{verbete}[wen2hua4]{文化}[4;4]
\begin{pronuncia}{wen2hua4}
\significado[个,种]{s.}{ cultura; civilização }
\end{pronuncia}
\end{verbete}

\begin{verbete}[Wen2xue2xi4]{文学系}[4;8;7]
\begin{pronuncia}{Wen2xue2xi4}
\significado{s.}{ Faculdade de Letras }
\end{pronuncia}
\end{verbete}

\begin{verbete}[wen4]{问}[6]
\begin{pronuncia}{wen4}
\significado{v.}{ perguntar }
\end{pronuncia}
\end{verbete}

\begin{verbete}[wen4ti2]{问题}[6;15]
\begin{pronuncia}{wen4ti2}
\significado[个]{s.}{ pergunta; questão; problema }
\end{pronuncia}
\end{verbete}

\begin{verbete}[wo3]{我}[7]
\begin{pronuncia}{wo3}
\significado{pron.}{ eu }
\end{pronuncia}
\end{verbete}

\begin{verbete}[wo3de0]{我的}[7;8]
\begin{pronuncia}{wo3de0}
\significado{pron.}{ meu, meus }
\end{pronuncia}
\end{verbete}

\begin{verbete}[wo3men0]{我们}[7;5]
\begin{pronuncia}{wo3men0}
\significado{pron.}{ nós }
\end{pronuncia}
\end{verbete}

\begin{verbete}[wo3men0de0]{我们的}[7;5;8]
\begin{pronuncia}{wo3men0de0}
\significado{pron.}{ nosso, nossos }
\end{pronuncia}
\end{verbete}

\begin{verbete}[wo3shi4]{卧室}[8;9]
\begin{pronuncia}{wo3shi4}
\significado{s.}{ quarto de dormir }
\end{pronuncia}
\end{verbete}

\begin{verbete}[wu3fan4]{午饭}[4;7]
\begin{pronuncia}{wu3fan4}
\significado[份,顿,次,餐]{s.}{ almoço }
\end{pronuncia}
\end{verbete}

\begin{verbete}[wu3]{五}[4]
\begin{pronuncia}{wu3}
\significado{num.}{ cinco; 5 }
\end{pronuncia}
\end{verbete}

\begin{verbete}[wu3]{舞}[14]
\begin{pronuncia}{wu3}
\significado{s.}{ dança }
\end{pronuncia}
\end{verbete}

%%%%% EOF %%%%%

%%%
%%% X
%%%

\section*{X}\addcontentsline{toc}{section}{X}

\begin{verbete}{夕阳}{xi1yang2}{3,6}
  \significado{s.}{pôr do sol}
  \veja{日出}{ri4chu1}
\end{verbete}

\begin{verbete}{吸管}{xi1guan3}{6,14}
  \significado[支]{s.}{canudo para beber; pipeta; conta-gotas; \emph{snorkel}}
\end{verbete}

\begin{verbete}{吸铁石}{xi1tie3shi2}{6,10,5}
  \significado{s.}{imã; magneto}
  \veja{磁铁}{ci2tie3}
\end{verbete}

\begin{verbete}{吸烟}{xi1yan1}{6,10}
  \significado{v.+compl.}{fumar}
\end{verbete}

\begin{verbete}{西}{xi1}{6}[Radical 襾]
  \significado{p.l.}{oeste}
\end{verbete}

\begin{verbete}{西安}{xi1'an1}{6,6}
  \significado*{s.}{Xi'an}
\end{verbete}

\begin{verbete}{西班牙文}{xi1ban1ya2wen2}{6,10,4,4}
  \significado{s.}{espanhol, língua espanhola}
  \veja{西文}{xi1wen2}
\end{verbete}

\begin{verbete}{西班牙语}{xi1ban1ya2yu3}{6,10,4,9}
  \significado{s.}{espanhol, língua espanhola}
  \veja{西语}{xi1yu3}
\end{verbete}

\begin{verbete}{西半球}{xi1ban4qiu2}{6,5,11}
  \significado{s.}{hemisfério oeste}
\end{verbete}

\begin{verbete}{西边}{xi1bian1}{6,5}
  \significado{adv.}{ao oeste de; oeste; lado oeste; parte ocidental}
\end{verbete}

\begin{verbete}{西部}{xi1bu4}{6,10}
  \significado{s.}{parte ocidental}
\end{verbete}

\begin{verbete}{西方}{xi1fang1}{6,4}
  \significado{s.}{países ocidentais; o Ocidente; o Oeste}
\end{verbete}

\begin{verbete}{西瓜}{xi1gua1}{6,5}
  \significado[颗,个]{s.}{melancia}
\end{verbete}

\begin{verbete}{西兰花}{xi1lan2hua1}{6,5,7}
  \significado{s.}{brócolis}
\end{verbete}

\begin{verbete}{西蓝花}{xi1lan2hua1}{6,13,7}
  \variante{西兰花}
\end{verbete}

\begin{verbete}{西面}{xi1mian4}{6,9}
  \significado{s.}{oeste; lado oeste}
\end{verbete}

\begin{verbete}{西文}{xi1wen2}{6,4}
  \significado{s.}{espanhol, língua espanhola}
  \veja{西班牙文}{xi1ban1ya2wen2}
\end{verbete}

\begin{verbete}{西西}{xi1xi1}{6,6}
  \significado{num.}{centímetro cúbico}
\end{verbete}

\begin{verbete}{西语}{xi1yu3}{6,9}
  \significado{s.}{espanhol, língua espanhola}
  \veja{西班牙语}{xi1ban1ya2yu3}
\end{verbete}

\begin{verbete}{希望}{xi1wang4}{7,11}
  \significado[个]{s.}{desejo}
  \significado{v.}{desejar}
\end{verbete}

\begin{verbete}{昔日}{xi1ri4}{8,4}
  \significado{adj.}{passado}
\end{verbete}

\begin{verbete}{牺牲}{xi1sheng1}{10,9}
  \significado{s.}{abate de um animal como sacrifício}
  \significado{v.}{sacrificar a vida de alguém; sacrificar (algo de valor)}
\end{verbete}

\begin{verbete}{悉尼}{xi1ni2}{11,5}
  \significado*{s.}{Sidney}
\end{verbete}

\begin{verbete}{悉数}{xi1shu3}{11,13}
  \significado{adv.}{enumerar em detalhes; explicar claramente}
  \veja{悉数}{xi1shu4}
\end{verbete}

\begin{verbete}{悉数}{xi1shu4}{11,13}
  \significado{adv.}{todos; cada um; toda a soma}
  \veja{悉数}{xi1shu3}
\end{verbete}

\begin{verbete}{悉心}{xi1xin1}{11,4}
  \significado{adv.}{colocar o coração (e a alma) em algo; com muito cuidado}
\end{verbete}

\begin{verbete}{蜥易}{xi1yi4}{14,8}
  \variante{蜥蜴}
\end{verbete}

\begin{verbete}{蜥蜴}{xi1yi4}{14,14}
  \significado{s.}{lagarto}
\end{verbete}

\begin{verbete}{席卷}{xi2juan3}{10,8}
  \significado{v.}{engolfar; varrer; levar tudo para fora}
\end{verbete}

\begin{verbete}{袭击}{xi2ji1}{11,5}
  \significado{s.}{ataque (especialmente um ataque surpresa); invasão}
  \significado{v.}{atacar}
\end{verbete}

\begin{verbete}{洗}{xi3}{9}[Radical 水]
  \significado{v.}{lavar; revelar (fotos); tomar banho}
\end{verbete}

\begin{verbete}{洗涤}{xi3di2}{9,10}
  \significado{s.}{enxágue; lava}
  \significado{v.}{enxaguar; lavar}
\end{verbete}

\begin{verbete}{洗涤间}{xi3di2jian1}{9,10,7}
  \significado{s.}{lavanderia}
\end{verbete}

\begin{verbete}{洗劫}{xi3jie2}{9,7}
  \significado{v.}{saquear; pilhar; roubar}
\end{verbete}

\begin{verbete}{洗净}{xi3jing4}{9,8}
  \significado{v.}{lavar (limpeza)}
\end{verbete}

\begin{verbete}{洗礼}{xi3li3}{9,5}
  \significado{s.}{batismo}
  \significado{v.}{batizar}
\end{verbete}

\begin{verbete}{洗手}{xi3shou3}{9,4}
  \significado{v.}{ir ao banheiro; lavar as mãos}
\end{verbete}

\begin{verbete}{洗手不干}{xi3shou3bu2gan4}{9,4,4,3}
  \significado{v.}{parar totalmente de fazer algo}
\end{verbete}

\begin{verbete}{洗手池}{xi3shou3chi2}{9,4,6}
  \significado{s.}{pia de banheiro; lavatório}
  \veja{洗手盆}{xi3shou3pen2}
\end{verbete}

\begin{verbete}{洗手间}{xi3shou3jian1}{9,4,7}
  \significado{s.}{sanitário; toilette; banheiro}
\end{verbete}

\begin{verbete}{洗手盆}{xi3shou3pen2}{9,4,9}
  \significado{s.}{pia de banheiro; lavatório}
  \veja{洗手池}{xi3shou3chi2}
\end{verbete}

\begin{verbete}{洗手乳}{xi3shou3ru3}{9,4,8}
  \significado{s.}{sabonete líquido para lavar as mãos}
  \veja{洗手液}{xi3shou3ye4}
\end{verbete}

\begin{verbete}{洗手液}{xi3shou3ye4}{9,4,11}
  \significado{s.}{sabonete líquido para lavar as mãos}
  \veja{洗手乳}{xi3shou3ru3}
\end{verbete}

\begin{verbete}{洗脱}{xi3tuo1}{9,11}
  \significado{v.}{limpar; purgar; lavar}
\end{verbete}

\begin{verbete}{洗碗}{xi3wan3}{9,13}
  \significado{v.}{lavar pratos}
\end{verbete}

\begin{verbete}{洗胃}{xi3wei4}{9,9}
  \significado{s.}{medicina:~lavagem gástrica}
  \significado{v.}{ter o estômago lavado}
\end{verbete}

\begin{verbete}{洗衣机}{xi3yi1ji1}{9,6,6}
  \significado[台]{s.}{máquina de lavar roupa}
\end{verbete}

\begin{verbete}{洗澡}{xi3zao3}{9,16}
  \significado{v.+compl.}{tomar banho; duchar-se; lavar-se}
\end{verbete}

\begin{verbete}{洗澡间}{xi3zao3jian1}{9,16,7}
  \significado[间]{s.}{banheiro}
\end{verbete}

\begin{verbete}{喜欢}{xi3huan5}{12,6}
  \significado{v.}{gostar}
\end{verbete}

\begin{verbete}{喜剧}{xi3ju4}{12,10}
  \significado[部,出]{s.}{uma comédia}
\end{verbete}

\begin{verbete}{戏}{xi4}{6}[Radical 戈]
  \significado[出,场,台]{s.}{drama; peça de teatro; \emph{show}}
\end{verbete}

\begin{verbete}{戏法}{xi4fa3}{6,8}
  \significado{s.}{truque de mágica; prestidigitação}
\end{verbete}

\begin{verbete}{戏剧}{xi4ju4}{6,10}
  \significado{s.}{drama; suspense; teatro}
\end{verbete}

\begin{verbete}{戏剧般}{xi4ju4ban1}{6,10,10}
  \significado{adj.}{melodramático}
\end{verbete}

\begin{verbete}{戏剧编剧}{xi4ju4bian1ju4}{6,10,12,10}
  \significado{s.}{dramaturgo}
\end{verbete}

\begin{verbete}{戏剧化地}{xi4ju4hua4di4}{6,10,4,6}
  \significado{adv.}{dramaticamente; teatralmente}
\end{verbete}

\begin{verbete}{戏剧家}{xi4ju4jia1}{6,10,10}
  \significado{s.}{dramaturgo}
\end{verbete}

\begin{verbete}{戏剧效果}{xi4ju4xiao4guo3}{6,10,10,8}
  \significado{s.}{efeito dramático}
\end{verbete}

\begin{verbete}{戏剧性}{xi4ju4xing4}{6,10,8}
  \significado{adj.}{dramático}
\end{verbete}

\begin{verbete}{戏剧演出}{xi4ju4yan3chu1}{6,10,14,5}
  \significado{s.}{performance dramática}
\end{verbete}

\begin{verbete}{戏弄}{xi4nong4}{6,7}
  \significado{v.}{zombar de; pregar peças; provocar}
\end{verbete}

\begin{verbete}{戏耍}{xi4shua3}{6,9}
  \significado{v.}{divertir-me; brincar com; provocar}
\end{verbete}

\begin{verbete}{戏谑}{xi4xue4}{6,11}
  \significado{v.}{brincar; fazer piadas; ridicularizar}
\end{verbete}

\begin{verbete}{戏院}{xi4yuan4}{6,9}
  \significado{s.}{teatro}
\end{verbete}

\begin{verbete}{系}{xi4}{7}[Radical 糸]
  \significado{s.}{faculdade (da universidade); departamento}
  \significado{v.}{prender; vincular; conectar; relacionar com; amarrar; se preocupar}
\end{verbete}

\begin{verbete}{系列}{xi4lie4}{7,6}
  \significado{s.}{série; conjunto}
\end{verbete}

\begin{verbete}{系囚}{xi4qiu2}{7,5}
  \significado{s.}{prisioneiro}
\end{verbete}

\begin{verbete}{系统}{xi4tong3}{7,9}
  \significado[个]{s.}{sistema}
\end{verbete}

\begin{verbete}{细节}{xi4jie2}{8,5}
  \significado{s.}{detalhe, particularidade}
\end{verbete}

\begin{verbete}{细菌战}{xi4jun1zhan4}{8,11,9}
  \significado{s.}{guerra biológica}
\end{verbete}

\begin{verbete}{虾}{xia1}{9}[Radical 虫]
  \significado{s.}{camarão}
\end{verbete}

\begin{verbete}{下}{xia4}{3}[Radical 一]
  \significado{adv.}{abaixo; em baixo de}
  \significado{v.}{descer; chegar a (uma decisão, conclusão, etc.); recusar}
\end{verbete}

\begin{verbete}{下巴}{xia4ba5}{3,4}
  \significado[个]{s.}{queixo}
\end{verbete}

\begin{verbete}{下边}{xia4bian5}{3,5}
  \significado{adv.}{em baixo; abaixo; parte de baixo}
\end{verbete}

\begin{verbete}{下车}{xia4che1}{3,4}
  \significado{v.}{descer; sair (de ônibus, carro, etc.)}
\end{verbete}

\begin{verbete}{下蛋}{xia4dan4}{3,11}
  \significado{v.}{botar ovos}
\end{verbete}

\begin{verbete}{下海}{xia4hai3}{3,10}
  \significado{v.+compl.}{ir para o mar; (barco) deixar o porto e iniciar uma jornada | ir pescar no mar | tornar-se ator profissional}
\end{verbete}

\begin{verbete}{下课}{xia4ke4}{3,10}
  \significado{v.+compl.}{acabar a aula; terminar a aula}
\end{verbete}

\begin{verbete}{下来}{xia4lai5}{3,7}
  \significado{v.}{descer (para a minha localização)}
\end{verbete}

\begin{verbete}{下面}{xia4mian4}{3,9}
  \significado{adv.}{em baixo; abaixo; parte de baixo}
  \significado{v.}{cozinhar macarrão}
\end{verbete}

\begin{verbete}{下去}{xia4qu4}{3,5}
  \significado{v.}{descer (a partir da minha localização)}
\end{verbete}

\begin{verbete}{下水道}{xia4shui3dao4}{3,4,12}
  \significado{s.}{esgoto}
\end{verbete}

\begin{verbete}{下午}{xia4wu3}{3,4}
  \significado{adv.}{tarde; à tarde; período logo após o meio-dia}
\end{verbete}

\begin{verbete}{下午茶}{xia4wu3cha2}{3,4,9}
  \significado{s.}{chá da tarde (normalmente chás com doces);}
\end{verbete}

\begin{verbete}{下线}{xia4xian4}{3,8}
  \significado{v.}{ficar \emph{offline}; (um produto) sair da linha de montagem; pessoa abaixo de si em um esquema de pirâmide}
\end{verbete}

\begin{verbete}{下旬}{xia4xun2}{3,6}
  \significado{adv.}{última dezena do mês}
\end{verbete}

\begin{verbete}{下雨}{xia4yu3}{3,8}
  \significado{v.+compl.}{chover}
\end{verbete}

\begin{verbete}{下载}{xia4zai3}{3,10}
  \significado{v.}{baixar; \emph{download}}
\end{verbete}

\begin{verbete}{下崽}{xia4zai3}{3,12}
  \significado{v.}{dar à luz (animais); parir}
\end{verbete}

\begin{verbete}{吓人}{xia4ren2}{6,2}
  \significado{adj.}{apavorante; assustador}
  \significado{v.+compl.}{assustar-se; tomar um susto}
\end{verbete}

\begin{verbete}{夏日}{xia4ri4}{10,4}
  \significado{s.}{horário de verão}
\end{verbete}

\begin{verbete}{夏天}{xia4tian1}{10,4}
  \significado[个]{s.}{verão}
\end{verbete}

\begin{verbete}{仙}{xian1}{5}[Radical 人]
  \significado{s.}{imortal}
\end{verbete}

\begin{verbete}{先}{xian1}{6}[Radical 儿]
  \significado{adv.}{em primeiro lugar; primeiramente; antes do tempo; de antemão}
\end{verbete}

\begin{verbete}{先不先}{xian1bu4xian1}{6,4,6}
  \significado{adv.}{dialeto:~antes de tudo; em primeiro lugar}
\end{verbete}

\begin{verbete}{先到先得}{xian1dao4xian1de2}{6,8,6,11}
  \significado{expr.}{primeiro a chegar, primeiro a ser servido}
\end{verbete}

\begin{verbete}{先烈}{xian1lie4}{6,10}
  \significado{s.}{mártir}
\end{verbete}

\begin{verbete}{先期}{xian1qi1}{6,12}
  \significado{adv.}{antecipadamente}
  \significado{s.}{prematuro; \emph{front-end}}
\end{verbete}

\begin{verbete}{先生}{xian1sheng5}{6,5}
  \significado[位]{s.}{senhor; marido; professor; dialeto:~doutor}
\end{verbete}

\begin{verbete}{先天}{xian1tian1}{6,4}
  \significado{adj.}{congênito; inato; natural}
  \significado{s.}{período embrionário}
\end{verbete}

\begin{verbete}{先验}{xian1yan4}{6,10}
  \significado{adj.}{filosofia:~a priori}
\end{verbete}

\begin{verbete}{先有}{xian1you3}{6,6}
  \significado{adj.}{preexistente; anterior}
\end{verbete}

\begin{verbete}{咸}{xian2}{9}[Radical 口]
  \significado*{s.}{sobrenome Xian}
  \significado{adj.}{salgado}
\end{verbete}

\begin{verbete}{咸菜}{xian2cai4}{9,11}
  \significado{s.}{legumes salgados; \emph{pickles}}
\end{verbete}

\begin{verbete}{咸淡}{xian2dan4}{9,11}
  \significado{s.}{água salobra; grau de salinidade; salgado e sem sal (sabores)}
\end{verbete}

\begin{verbete}{咸肉}{xian2rou4}{9,6}
  \significado{s.}{\emph{bacon}; carne curada com sal}
\end{verbete}

\begin{verbete}{咸涩}{xian2se4}{9,10}
  \significado{s.}{ácido; salgado e amargo}
\end{verbete}

\begin{verbete}{咸水}{xian2shui3}{9,4}
  \significado{s.}{salmora; água salgada}
\end{verbete}

\begin{verbete}{咸盐}{xian2yan2}{9,10}
  \significado{s.}{coloquial:~sal; sal de mesa}
\end{verbete}

\begin{verbete}{咸鱼}{xian2yu2}{9,8}
  \significado{s.}{peixe salgado}
\end{verbete}

\begin{verbete}{猃狁}{xian3yun3}{10,7}
  \significado*{s.}{Termo da dinastia Zhou para uma tribo nômade do norte mais tarde chamou o Xiongnu (匈奴) nas dinastias Qin e Han}
  \veja{匈奴}{xiong1nu2}
\end{verbete}

\begin{verbete}{见}{xian4}{4}[Radical 見]
  \significado{v.}{aparecer; também escrito como 现}
  \veja{见}{jian4}
  \veja{现}{xian4}
\end{verbete}

\begin{verbete}{现}{xian4}{8}[Radical 見]
  \significado{adj.}{presente; atual}
  \significado{v.}{aparecer}
  \veja{见}{xian4}
\end{verbete}

\begin{verbete}{现代}{xian4dai4}{8,5}
  \significado*{s.}{Hyundai, empresa sul-coreana}
  \significado{s.}{tempos modernos; era moderna}
\end{verbete}

\begin{verbete}{现货}{xian4huo4}{8,8}
  \significado{s.}{produtos à vista}
\end{verbete}

\begin{verbete}{现货的}{xian4huo4 de5}{8,8,8}
  \significado{s.}{produtos em estoque}
\end{verbete}

\begin{verbete}{现实}{xian4shi2}{8,8}
  \significado{adj.}{real; realístico}
  \significado{s.}{realidade}
\end{verbete}

\begin{verbete}{现象}{xian4xiang4}{8,11}
  \significado[个,种]{s.}{fenômeno}
\end{verbete}

\begin{verbete}{现有}{xian4you3}{8,6}
  \significado{adj.}{disponível atualmente; atualmente existente}
\end{verbete}

\begin{verbete}{现在}{xian4zai4}{8,6}
  \significado{adv.}{agora; neste momento}
\end{verbete}

\begin{verbete}{现抓}{xian4zhua1}{8,7}
  \significado{v.}{improvisar}
\end{verbete}

\begin{verbete}{现做}{xian4zuo4}{8,11}
  \significado{adj.}{fresco}
  \significado{v.}{fazer (comida) no local}
\end{verbete}

\begin{verbete}{线香}{xian4xiang1}{8,9}
  \significado{s.}{bastão ou vareta de incenso}
\end{verbete}

\begin{verbete}{宪法法院}{xian4fa3fa3yuan4}{9,8,8,9}
  \significado{s.}{tribunal constitucional}
\end{verbete}

\begin{verbete}{宪政}{xian4zheng4}{9,9}
  \significado{s.}{governo constitucional}
\end{verbete}

\begin{verbete}{宪制}{xian4zhi4}{9,8}
  \significado{adj.}{constitucional}
  \significado{s.}{sistema de governo constitucional}
\end{verbete}

\begin{verbete}{陷入}{xian4ru4}{10,2}
  \significado{v.}{afundar; ser pego em; pousar (em uma situação)}
\end{verbete}

\begin{verbete}{羡慕}{xian4mu4}{12,14}
  \significado{v.}{invejar; admirar}
\end{verbete}

\begin{verbete}{乡巴佬}{xiang1ba1lao3}{3,4,8}
  \significado{s.}{aldeão; caipira}
\end{verbete}

\begin{verbete}{乡村}{xiang1cun1}{3,7}
  \significado{adj.}{rural; rústico}
  \significado{s.}{vila; campo}
\end{verbete}

\begin{verbete}{相处}{xiang1chu3}{9,5}
  \significado{v.}{entrar em contato (com alguém); associar; interagir; se dar bem (bem, mal)}
\end{verbete}

\begin{verbete}{相当}{xiang1dang1}{9,6}
  \significado{adv.}{bastante; consideravelmente}
\end{verbete}

\begin{verbete}{相聚}{xiang1ju4}{9,14}
  \significado{v.}{reunir-se; montar}
\end{verbete}

\begin{verbete}{相亲}{xiang1qin1}{9,9}
  \significado{s.}{encontro às cegas; entrevista arranjada para avaliar a proposta de um parceiro de casamento; apegar-se profundamente um ao outro}
\end{verbete}

\begin{verbete}{相思病}{xiang1si1bing4}{9,9,10}
  \significado{s.}{saudade de amor}
\end{verbete}

\begin{verbete}{相信}{xiang1xin4}{9,9}
  \significado{v.}{acreditar, estar convencido. aceitar como verdadeiro}
\end{verbete}

\begin{verbete}{相宜}{xiang1yi2}{9,8}
  \significado{adj.}{adequado; apropriado}
  \significado{v.}{ser adequado ou apropriado}
\end{verbete}

\begin{verbete}{相遇}{xiang1yu4}{9,12}
  \significado{v.}{encontrar (reunião, encontro, etc.)}
\end{verbete}

\begin{verbete}{香}{xiang1}{9}[Radical 香]
  \significado{adj.}{perfumado; com cheiro doce; aromático; saboroso ou apetitoso;  (comer) com prazer; (para dormir) som}
  \significado[根]{s.}{perfume ou especiarias; bastão ou vareta de incenso}
\end{verbete}

\begin{verbete}{香槟酒}{xiang1bin1jiu3}{9,14,10}
  \significado[杯]{s.}{\emph{champagne} (empréstimo linguístico)}
\end{verbete}

\begin{verbete}{香波}{xiang1bo1}{9,8}
  \significado{s.}{xampu}
\end{verbete}

\begin{verbete}{香肠}{xiang1chang2}{9,7}
  \significado[根]{s.}{salsicha}
\end{verbete}

\begin{verbete}{香港}{xiang1gang3}{9,12}
  \significado*{s.}{Hong Kong}
  \veja{香港岛}{xiang1gang3 dao3}
\end{verbete}

\begin{verbete}{香港岛}{xiang1gang3 dao3}{9,12,7}
  \significado*{s.}{Ilha de Hong Kong}
  \veja{香港}{xiang1gang3}
\end{verbete}

\begin{verbete}{香蕉}{xiang1jiao1}{9,15}
  \significado[枝,根,个,把]{s.}{banana}
\end{verbete}

\begin{verbete}{香炉}{xiang1lu2}{9,8}
  \significado{s.}{incensário (para queimar incenso); queimador de incenso; insensório, turíbulo}
\end{verbete}

\begin{verbete}{香气}{xiang1qi4}{9,4}
  \significado{s.}{fragrância; aroma; incenso}
\end{verbete}

\begin{verbete}{香味}{xiang1wei4}{9,8}
  \significado[股]{s.}{fragrância; cheiro doce}
\end{verbete}

\begin{verbete}{香蕈}{xiang1xun4}{9,15}
  \significado{s.}{\emph{shiitake}, cogumelo comestível}
\end{verbete}

\begin{verbete}{香烟}{xiang1yan1}{9,10}
  \significado[支,条]{s.}{cigarro; fumaça de incenso queimado}
\end{verbete}

\begin{verbete}{香艳}{xiang1yan4}{9,10}
  \significado{adj.}{atraente; erótico; romântico}
\end{verbete}

\begin{verbete}{香皂}{xiang1zao4}{9,7}
  \significado{s.}{sabonete; sabonete perfumado}
\end{verbete}

\begin{verbete}{享受}{xiang3shou4}{8,8}
  \significado[种]{s.}{prazer}
  \significado{v.}{desfrutar; viver}
\end{verbete}

\begin{verbete}{想}{xiang3}{13}[Radical 心]
  \significado{v./v.o.}{acreditar; sentir falta (sentir-se melancólico com a ausência de alguém ou algo); supor; pensar; querer; desejar}
\end{verbete}

\begin{verbete}{想法}{xiang3fa3}{13,8}
  \significado[个]{s.}{noção; opinião; jeito de pensar}
  \significado{s.}{maneira de pensar, opinião, noção}
  \significado{v.}{pensar em uma maneira (de fazer algo)}
\end{verbete}

\begin{verbete}{想念}{xiang3nian4}{13,8}
  \significado{v.}{perder; sentir falta; lembrar com saudade}
\end{verbete}

\begin{verbete}{想想看}{xiang3xiang3kan4}{13,13,9}
  \significado{v.}{pensar sobre isso}
\end{verbete}

\begin{verbete}{想象}{xiang3xiang4}{13,11}
  \significado{v.}{imaginar}
\end{verbete}

\begin{verbete}{向}{xiang4}{6}[Radical 口]
  \significado*{s.}{sobrenome Xiang}
  \significado{prep.}{para}
  \significado{v.}{enfrentar; virar para; apoiar}
\end{verbete}

\begin{verbete}{向汪}{xiang4wang1}{6,7}
  \significado{v.}{esperar que}
\end{verbete}

\begin{verbete}{向往}{xiang4wang3}{6,8}
  \significado{v.}{ansiar por; esperar ansiosamente por}
\end{verbete}

\begin{verbete}{像}{xiang4}{13}[Radical 人]
  \significado{s.}{imagem; retrato; aparência}
  \significado{v.}{assemelhar -se; ser como}
\end{verbete}

\begin{verbete}{消防}{xiao1fang2}{10,6}
  \significado{s.}{combate a incêncios; controle de incêndios}
\end{verbete}

\begin{verbete}{消防员}{xiao1fang2yuan2}{10,6,7}
  \significado{s.}{bombeiro}
\end{verbete}

\begin{verbete}{消失}{xiao1shi1}{10,5}
  \significado{v.}{desaparecer; desvanecer}
\end{verbete}

\begin{verbete}{嚣张}{xiao1zhang1}{18,7}
  \significado{adj.}{desenfreado; arrogante; agressivo}
\end{verbete}

\begin{verbete}{小}{xiao3}{3}[Radical 小][Kangxi 42]
  \significado{adj.}{pequeno; jovem}
\end{verbete}

\begin{verbete}{小白菜}{xiao3bai2cai4}{3,5,11}
  \significado[棵]{s.}{\emph{bok choy}; couve chinesa}
\end{verbete}

\begin{verbete}{小吃}{xiao3chi1}{3,6}
  \significado{s.}{refeição leve; petisco}
\end{verbete}

\begin{verbete}{小狗}{xiao3 gou3}{3,8}
  \significado{s.}{filhote de cachorro}
\end{verbete}

\begin{verbete}{小姐}{xiao3jie5}{3,8}
  \significado[个,位]{s.}{senhorita; jovem senhora; gíria:~prostituta}
\end{verbete}

\begin{verbete}{小气鬼}{xiao3qi4gui3}{3,4,9}
  \significado{adj.}{avarento; mão-de-vaca; miserável; pão-duro}
\end{verbete}

\begin{verbete}{小区}{xiao3qu1}{3,4}
  \significado{s.}{conjunto habitacional, comunidade, bairro; célula (telecomunicações)}
\end{verbete}

\begin{verbete}{小时}{xiao3shi2}{3,7}
  \significado{adv.}{hora; para horas}
  \significado[个]{s.}{hora}
\end{verbete}

\begin{verbete}{小树}{xiao3shu4}{3,9}
  \significado[棵]{s.}{muda; arbusto; árvore pequena}
\end{verbete}

\begin{verbete}{小说}{xiao3shuo1}{3,9}
  \significado[本,部]{s.}{romance; ficção}
\end{verbete}

\begin{verbete}{小腿}{xiao3tui3}{3,13}
  \significado{s.}{perna (do joelho ao calcanhar); haste}
\end{verbete}

\begin{verbete}{小屋}{xiao3wu1}{3,9}
  \significado{s.}{cabana; chalé; cabine}
\end{verbete}

\begin{verbete}{小小}{xiao3xiao3}{3,3}
  \significado{adj.}{muito pequeno}
\end{verbete}

\begin{verbete}{小心}{xiao3xin1}{3,4}
  \significado{adj.}{cuidado}
\end{verbete}

\begin{verbete}{小学}{xiao3xue2}{3,8}
  \significado{s.}{escola ensino fundamental}
\end{verbete}

\begin{verbete}{小洋白菜}{xiao3 yang2bai2cai4}{3,9,5,11}
  \significado{s.}{couve de bruxelas}
\end{verbete}

\begin{verbete}{小众}{xiao3zhong4}{3,6}
  \significado{s.}{minoria da população; nicho (mercado, etc.)}
\end{verbete}

\begin{verbete}{哮喘}{xiao4chuan3}{10,12}
  \significado{s.}{asma}
\end{verbete}

\begin{verbete}{效果}{xiao4guo3}{10,8}
  \significado{s.}{resultado; efeito; eficácia; (teatro/cinema) efeitos sonoros ou visuais}
\end{verbete}

\begin{verbete}{校}{xiao4}{10}[Radical 木]
  \significado[所]{s.}{oficial militar; escola}
  \veja{校}{jiao4}
\end{verbete}

\begin{verbete}{校服}{xiao4fu2}{10,8}
  \significado{s.}{uniforme escolar}
\end{verbete}

\begin{verbete}{校规}{xiao4gui1}{10,8}
  \significado{s.}{regras e regulamentos escolares}
\end{verbete}

\begin{verbete}{校监}{xiao4jian1}{10,10}
  \significado{s.}{diretor; supervisor (de escola)}
\end{verbete}

\begin{verbete}{校园}{xiao4yuan2}{10,7}
  \significado{s.}{campus}
\end{verbete}

\begin{verbete}{校长}{xiao4zhang3}{10,4}
  \significado[个,位,名]{s.}{diretor de escola; reitor (universidade)}
\end{verbete}

\begin{verbete}{笑}{xiao4}{10}[Radical 竹]
  \significado{v.}{sorrir, rir; rir de}
\end{verbete}

\begin{verbete}{笑话}{xiao4hua5}{10,8}
  \significado{adj.}{absurdo, ridículo}
  \significado[个]{s.}{piada, brincadeira}
  \significado{v.}{rir de algo, zombar, ridicularizar}
\end{verbete}

\begin{verbete}{笑容}{xiao4rong2}{10,10}
  \significado[副]{s.}{sorriso; expressão sorridente}
\end{verbete}

\begin{verbete}{些}{xie1}{8}[Radical 二]
  \significado{adv.}{uns; alguns; vários}
  \significado{clas.}{que indica uma pequena quantidade ou pequeno número maior que 1}
\end{verbete}

\begin{verbete}{些许}{xie1xu3}{8,6}
  \significado{num.}{um pouco}
\end{verbete}

\begin{verbete}{斜阳}{xie2yang2}{11,6}
  \significado{s.}{sol poente}
\end{verbete}

\begin{verbete}{谐}{xie2}{11}[Radical 言]
  \significado{adj.}{harmonioso; humorístico}
\end{verbete}

\begin{verbete}{鞋}{xie2}{15}[Radical 革]
  \significado[双,只]{s.}{sapatos}
\end{verbete}

\begin{verbete}{写}{xie3}{5}[Radical 冖]
  \significado{v.}{escrever}
\end{verbete}

\begin{verbete}{写意}{xie3yi4}{5,13}
  \significado{s.}{estilo de pintura chinesa à mão livre, caracterizado por traços ousados em vez de detalhes precisos}
  \significado{v.}{sugerir (em vez de descrever em detalhes)}
  \veja{写意}{xie4yi4}
\end{verbete}

\begin{verbete}{写照}{xie3zhao4}{5,13}
  \significado{s.}{retrato}
\end{verbete}

\begin{verbete}{写真}{xie3zhen1}{5,10}
  \significado{s.}{retrato}
  \significado{v.}{descrever algo com precisão}
\end{verbete}

\begin{verbete}{写字}{xie3zi4}{5,6}
  \significado{v.}{escrever (à mão); praticar caligrafia}
\end{verbete}

\begin{verbete}{写字匠}{xie3zi4 jiang4}{5,6,6}
  \significado{s.}{calígrafo}
\end{verbete}

\begin{verbete}{写作}{xie3zuo4}{5,7}
  \significado{s.}{escrita; redação; composição}
  \significado{v.}{escrever}
\end{verbete}

\begin{verbete}{血}{xie3}{6}
  \veja{血}{xue4}
\end{verbete}

\begin{verbete}{写意}{xie4yi4}{5,13}
  \significado{adj.}{confortável; agradável; relaxado}
  \veja{写意}{xie3yi4}
\end{verbete}

\begin{verbete}{泄气}{xie4qi4}{8,4}
  \significado{adj.}{decepcionante; frustrante; patético}
  \significado{v.+compl.}{perder o coração; sintir-se desencorajado; ficar desanimado}
\end{verbete}

\begin{verbete}{谢病}{xie4bing4}{12,10}
  \significado{v.}{desculpar-se por causa de doença}
\end{verbete}

\begin{verbete}{谢恩}{xie4'en1}{12,10}
  \significado{v.}{agradecer a alguém pelo favor (especialmente imperador ou oficial superior)}
\end{verbete}

\begin{verbete}{谢媒}{xie4mei2}{12,12}
  \significado{v.}{agradecer ao casamenteiro}
\end{verbete}

\begin{verbete}{谢世}{xie4shi4}{12,5}
  \significado{v.}{morrer; falecer}
\end{verbete}

\begin{verbete}{谢天谢地}{xie4tian1xie4di4}{12,4,12,6}
  \significado{expr.}{agradecer a Deus; agradecer aos céus}
\end{verbete}

\begin{verbete}{谢谢}{xie4xie5}{12,12}
  \significado{interj.}{Obrigado!}
  \significado{v.}{agradecer}
\end{verbete}

\begin{verbete}{谢意}{xie4yi4}{12,13}
  \significado{s.}{gratidão}
\end{verbete}

\begin{verbete}{心机}{xin1ji1}{4,6}
  \significado{s.}{pensamento; esquema}
\end{verbete}

\begin{verbete}{心声}{xin1sheng1}{4,7}
  \significado{s.}{desejo sincero, voz interior, aspiração}
\end{verbete}

\begin{verbete}{心疼}{xin1teng2}{4,10}
  \significado{adj.}{angustiado}
  \significado{v.}{sentir pena de alguém; arrepender-se; ressentir-se; ficar angustiado}
\end{verbete}

\begin{verbete}{心中}{xin1zhong1}{4,4}
  \significado{adv.}{nos pensamentos; no coração}
  \significado{s.}{ponto central}
\end{verbete}

\begin{verbete}{芯片}{xin1pian4}{7,4}
  \significado{s.}{chip de computador; microchip}
\end{verbete}

\begin{verbete}{辛苦}{xin1ku3}{7,8}
  \significado{adj.}{exaustivo; duro; árduo}
  \significado{s.}{dificuldades}
  \significado{v.}{trabalhar duro; ter muitos problemas}
\end{verbete}

\begin{verbete}{新}{xin1}{13}[Radical 斤]
  \significado*{s.}{sobrenome Xin; abreviatura de Xinjiang (新疆); abreviatura de Singapura (新加坡)}
  \significado{adj.}{novo; prefixo meso (química)}
  \significado{adv.}{recentemente}
  \veja{新加坡}{xin1jia1po1}
  \veja{新疆}{xin1jiang1}
\end{verbete}

\begin{verbete}{新加坡}{xin1jia1po1}{13,5,8}
  \significado*{s.}{Singapura}
\end{verbete}

\begin{verbete}{新疆}{xin1jiang1}{13,19}
  \significado*{s.}{Xinjiang}
\end{verbete}

\begin{verbete}{新疆维吾尔自治区}{xin1jiang1 wei2wu2'er3 zi4zhi4qu1}{13,19,11,7,5,6,8,4}
  \significado*{s.}{Região Autônoma Uigur de Xinjiang}
\end{verbete}

\begin{verbete}{新年}{xin1nian2}{13,6}
  \significado*[个]{s.}{Ano Novo}
\end{verbete}

\begin{verbete}{新娘}{xin1niang2}{13,10}
  \significado{s.}{noiva}
\end{verbete}

\begin{verbete}{新娘服装}{xin1niang2 fu2zhuang1}{13,10,8,12}
  \significado{s.}{roupas de noiva}
\end{verbete}

\begin{verbete}{新娘子}{xin1niang2zi5}{13,10,3}
  \veja{新娘}{xin1niang2}
\end{verbete}

\begin{verbete}{新闻}{xin1wen2}{13,9}
  \significado[条,个]{s.}{notícia}
\end{verbete}

\begin{verbete}{新鲜}{xin1xian1}{13,14}
  \significado{adj.}{fresco (experiência, alimento, etc.)}
  \significado{s.}{frescor}
\end{verbete}

\begin{verbete}{信}{xin4}{9}[Radical 人]
  \significado[封]{s.}{carta; correspondência}
\end{verbete}

\begin{verbete}{信访}{xin4fang3}{9,6}
  \significado{s.}{carta de reclamação; carta de petição}
  \veja{上访}{shang4fang3}
\end{verbete}

\begin{verbete}{信封}{xin4feng1}{9,9}
  \significado[个]{s.}{envelope}
\end{verbete}

\begin{verbete}{信经}{xin4jing1}{9,8}
  \significado[个]{s.}{crença; credo (seção da missa católica)}
\end{verbete}

\begin{verbete}{信心}{xin4xin1}{9,4}
  \significado[个]{s.}{confiança; fé (em alguém ou algo)}
\end{verbete}

\begin{verbete}{信用}{xin4yong4}{9,5}
  \significado{s.}{crédito (comércio)}
\end{verbete}

\begin{verbete}{信用卡}{xin4yong4ka3}{9,5,5}
  \significado[些]{s.}{cartão de crédito}
\end{verbete}

\begin{verbete}{星表}{xing1biao3}{9,8}
  \significado{s.}{catálogo de estrelas}
\end{verbete}

\begin{verbete}{星辰}{xing1chen2}{9,7}
  \significado{s.}{estrelas}
\end{verbete}

\begin{verbete}{星火}{xing1huo3}{9,4}
  \significado{s.}{trilha de meteoro (usada principalmente em expressões como 急如星火); faísca}
\end{verbete}

\begin{verbete}{星期}{xing1qi1}{9,12}
  \significado[个]{s.}{semana}
\end{verbete}

\begin{verbete}{星期二}{xing1qi1'er4}{9,12,2}
  \significado{s.}{terça-feira}
\end{verbete}

\begin{verbete}{星期六}{xing1qi1liu4}{9,12,4}
  \significado{s.}{sábado}
\end{verbete}

\begin{verbete}{星期日}{xing1qi1ri4}{9,12,4}
  \significado{s.}{domingo}
  \veja{星期天}{xing1qi1tian1}
\end{verbete}

\begin{verbete}{星期三}{xing1qi1san1}{9,12,3}
  \significado{s.}{quarta-feira}
\end{verbete}

\begin{verbete}{星期四}{xing1qi1si4}{9,12,5}
  \significado{s.}{quinta-feira}
\end{verbete}

\begin{verbete}{星期天}{xing1qi1tian1}{9,12,4}
  \significado{s.}{domingo}
  \veja{星期日}{xing1qi1ri4}
\end{verbete}

\begin{verbete}{星期五}{xing1qi1wu3}{9,12,4}
  \significado{s.}{sexta-feira}
\end{verbete}

\begin{verbete}{星期一}{xing1qi1yi1}{9,12,1}
  \significado{s.}{segunda-feira}
\end{verbete}

\begin{verbete}{星星}{xing1xing5}{9,9}
  \significado{s.}{estrela}
\end{verbete}

\begin{verbete}{星座}{xing1zuo4}{9,10}
  \significado[张]{s.}{signo astrológico; constelação}
\end{verbete}

\begin{verbete}{猩猩}{xing1xing5}{12,12}
  \significado{s.}{orangotango}
\end{verbete}

\begin{verbete}{行}{xing2}{6}[Radical 行][Kangxi 144]
  \significado{adj.}{capaz; competente}
  \significado{expr.}{claro que sim; de acordo; está bem}
  \significado{interj.}{OK!}
  \significado{v.}{caminhar; ir; viajar; atuar}
  \veja{行}{hang2}
\end{verbete}

\begin{verbete}{行动}{xing2dong4}{6,6}
  \significado[个]{s.}{ação; operação}
  \significado{v.}{mover}
\end{verbete}

\begin{verbete}{行进}{xing2jin4}{6,7}
  \significado{s.}{avançar; movimentar-se para frente}
\end{verbete}

\begin{verbete}{行礼}{xing2li3}{6,5}
  \significado{v.}{saudar; fazer saudação}
\end{verbete}

\begin{verbete}{行李}{xing2li5}{6,7}
  \significado[件]{s.}{bagagem}
\end{verbete}

\begin{verbete}{行人}{xing2ren2}{6,2}
  \significado{s.}{transeunte; pedestre; viajante à pé}
\end{verbete}

\begin{verbete}{行驶}{xing2shi3}{6,8}
  \significado{v.}{viajar ao longo de uma rota (veículos, etc.)}
\end{verbete}

\begin{verbete}{行星}{xing2xing1}{6,9}
  \significado[颗]{s.}{planeta}
  \veja{惑星}{huo4xing1}
\end{verbete}

\begin{verbete}{行凶}{xing2xiong1}{6,4}
  \significado{v.+compl.}{cometer agressão física ou assassinato; fazer algo violento}
\end{verbete}

\begin{verbete}{形而上学}{xing2'er2shang4xue2}{7,6,3,8}
  \significado{s.}{metafísica}
\end{verbete}

\begin{verbete}{形容}{xing2rong2}{7,10}
  \significado{s.}{descrever}
  \significado{s.}{semblante (literário), aparência}
\end{verbete}

\begin{verbete}{形象}{xing2xiang4}{7,11}
  \significado[个]{s.}{imagem; forma; figura; vializuação}
\end{verbete}

\begin{verbete}{省}{xing3}{9}[Radical 目]
  \significado[个]{s.}{governadoria}
  \significado{v.}{examinar minuciosamente, refletir (sobre a conduta de alguém); realizar; fazer uma visita (aos pais ou idosos)}
  \veja{省}{sheng3}
\end{verbete}

\begin{verbete}{省悟}{xing3wu4}{9,10}
  \significado{v.}{voltar a si; constatar; ver a verdade; acordar para a realidade}
\end{verbete}

\begin{verbete}{兴趣}{xing4qu4}{6,15}
  \significado[个]{s.}{interesse (desejo de conhecer sobre alguma coisa ou coisa no qual está interessado); hobby}
\end{verbete}

\begin{verbete}{姓}{xing4}{8}[Radical 女]
  \significado[个]{s.}{sobrenome}
  \significado{v.}{ter o sobrenome}
\end{verbete}

\begin{verbete}{姓名}{xing4ming2}{8,6}
  \significado{s.}{nome completo}
\end{verbete}

\begin{verbete}{姓氏}{xing4shi4}{8,4}
  \significado{s.}{sobrenome}
\end{verbete}

\begin{verbete}{幸福}{xing4fu2}{8,13}
  \significado{adj.}{feliz; abençoado}
  \significado{s.}{felicidade}
\end{verbete}

\begin{verbete}{幸亏}{xing4kui1}{8,3}
  \significado{adv.}{felizmente}
\end{verbete}

\begin{verbete}{幸运}{xing4yun4}{8,7}
  \significado{adj.}{afortunado; feliz; sortudo}
\end{verbete}

\begin{verbete}{幸运抽奖}{xing4yun4chou1jiang3}{8,7,8,9}
  \significado{s.}{loteria; sorteio}
\end{verbete}

\begin{verbete}{幸运儿}{xing4yun4'er2}{8,7,2}
  \significado{s.}{pessoa de sorte}
\end{verbete}

\begin{verbete}{性}{xing4}{8}[Radical 心]
  \significado{adj.}{sufixo formando adjetivo a partir de verbo}
  \significado[个]{s.}{natureza; carácter; propriedade; qualidade; atributo; sexualidade; sexo; gênero; essência}
  \significado{s.}{sufixo formando substantivo a partir de adjetivo}
\end{verbete}

\begin{verbete}{性侵}{xing4qin1}{8,9}
  \significado{s.}{agressão sexual}
  \significado{v.}{agredir sexualmente}
\end{verbete}

\begin{verbete}{性生活}{xing4sheng1huo2}{8,5,9}
  \significado{s.}{vida sexual}
\end{verbete}

\begin{verbete}{兄弟}{xiong1di4}{5,7}
  \significado{adj.}{fraternal; \emph{brotherly}}
  \significado{pron.}{eu, me (termo de uso humilde por homens em discurso público)}
  \significado[个]{s.}{irmãos; irmão mais novo; \emph{brothers}}
\end{verbete}

\begin{verbete}{匈奴}{xiong1nu2}{6,5}
  \significado*{s.}{Xiongnu, um povo da estepe oriental que criou um império que floresceu na época das dinastias Qin e Han}
\end{verbete}

\begin{verbete}{汹涌}{xiong1yong3}{7,10}
  \significado{adj.}{turbulento}
  \significado{v.}{aumentar ou emergir violentamente (oceano, rio, lago, etc.)}
\end{verbete}

\begin{verbete}{胸}{xiong1}{10}[Radical 肉]
  \significado{s.}{peito; tórax}
\end{verbete}

\begin{verbete}{熊}{xiong2}{14}[Radical 火]
  \significado*{s.}{sobrenome Xiong}
  \significado{adj.}{incapaz}
  \significado[把]{s.}{urso}
  \significado{v.}{repreender}
\end{verbete}

\begin{verbete}{熊猫}{xiong2mao1}{14,11}
  \significado[把,只]{s.}{panda gigante}
\end{verbete}

\begin{verbete}{休兵}{xiu1bing1}{6,7}
  \significado{s.}{armistício}
  \significado{v.}{cessar fogo}
\end{verbete}

\begin{verbete}{休假}{xiu1jia4}{6,11}
  \significado{v.+compl.}{ter um feriado; tirar férias; sair de férias}
\end{verbete}

\begin{verbete}{休憩}{xiu1qi4}{6,16}
  \significado{v.}{relaxar; descansar; dar um tempo}
\end{verbete}

\begin{verbete}{休息室}{xiu1xi1shi4}{6,10,9}
  \significado{s.}{saguão; salão}
\end{verbete}

\begin{verbete}{休息}{xiu1xi5}{6,10}
  \significado{s.}{descanço}
  \significado{v.}{descansar}
\end{verbete}

\begin{verbete}{休闲}{xiu1xian2}{6,7}
  \significado{s.}{ócio; lazer}
  \significado{v.}{desfrutar do lazer}
\end{verbete}

\begin{verbete}{休整}{xiu1zheng3}{6,16}
  \significado{v.}{militar:~descansar e reorganizar}
\end{verbete}

\begin{verbete}{修}{xiu1}{9}[Radical 人]
  \significado*{s.}{sobrenome Xiu}
  \significado{v.}{reparar; consertar; construir}
\end{verbete}

\begin{verbete}{修改}{xiu1gai3}{9,7}
  \significado{v.}{alterar; modificar; complementar}
\end{verbete}

\begin{verbete}{修规}{xiu1gui1}{9,8}
  \significado{s.}{plano de construção}
\end{verbete}

\begin{verbete}{绣}{xiu4}{10}[Radical 糸]
  \significado{s.}{bordado}
  \significado{v.}{bordar}
\end{verbete}

\begin{verbete}{臭}{xiu4}{10}[Radical ⾃]
  \significado{s.}{olfato; cheiro ruim}
  \veja{臭}{chou4}
\end{verbete}

\begin{verbete}{袖}{xiu4}{10}[Radical 衣]
  \significado{s.}{manga (de camisa, de camiseta, etc.)}
\end{verbete}

\begin{verbete}{虚伪}{xu1wei3}{11,6}
  \significado{adj.}{falso; hipócrita; artificial}
\end{verbete}

\begin{verbete}{需要}{xu1yao4}{14,9}
  \significado{s.}{necessidade}
  \significado{v.}{precisar; necessitar}
\end{verbete}

\begin{verbete}{许}{xu3}{6}[Radical 言]
  \significado*{s.}{sobrenome Xu}
  \significado{adv.}{um pouco; talvez}
  \significado{v.}{permitir; prometer; elogiar}
\end{verbete}

\begin{verbete}{畜}{xu4}{10}[Radical ⽥]
  \significado{v.}{criar (animais)}
  \veja{畜}{chu4}
\end{verbete}

\begin{verbete}{宣布}{xuan1bu4}{9,5}
  \significado{v.}{declarar; anunciar; proclamar}
\end{verbete}

\begin{verbete}{宣扬}{xuan1yang2}{9,6}
  \significado{v.}{divulgar; anunciar; espalhar por toda parte}
\end{verbete}

\begin{verbete}{玄学}{xuan2xue2}{5,8}
  \significado{s.}{Escola Philosófica Wei e Jin amalgamando os ideais daoísta e confucionistas; tradução da metafísica (形而上学)}
  \veja{形而上学}{xing2'er2shang4xue2}
\end{verbete}

\begin{verbete}{悬挂}{xuan2gua4}{11,9}
  \significado{s.}{(veículo) suspensão}
  \significado{v.}{suspender}
\end{verbete}

\begin{verbete}{悬崖}{xuan2ya2}{11,11}
  \significado{s.}{precipício; penhasco}
\end{verbete}

\begin{verbete}{旋转}{xuan2zhuan3}{11,8}
  \significado{v.}{girar}
\end{verbete}

\begin{verbete}{选手}{xuan3shou3}{9,4}
  \significado{s.}{jogador; atleta; competidor}
\end{verbete}

\begin{verbete}{选择}{xuan3ze2}{9,8}
  \significado{s.}{escolha, opção, alternativa}
  \significado{v.}{selecionar, escolher}
\end{verbete}

\begin{verbete}{学}{xue2}{8}[Radical 子]
  \significado{v.}{aprender; estudar}
\end{verbete}

\begin{verbete}{学费}{xue2fei4}{8,9}
  \significado[个]{s.}{mensalidade}
\end{verbete}

\begin{verbete}{学分}{xue2fen1}{8,4}
  \significado{s.}{créditos de um curso}
\end{verbete}

\begin{verbete}{学好}{xue2hao3}{8,6}
  \significado{v.}{seguir bons exemplos; aprender bem}
\end{verbete}

\begin{verbete}{学会}{xue2hui4}{8,6}
  \significado{s.}{instituto; associação (acadêmica); sociedade científica, douta ou erudita}
  \significado{v.}{aprender; dominar (um assunto)}
\end{verbete}

\begin{verbete}{学期}{xue2qi1}{8,12}
  \significado[个]{s.}{semestre}
\end{verbete}

\begin{verbete}{学生}{xue2sheng5}{8,5}
  \significado{s.}{estudante; aluno}
\end{verbete}

\begin{verbete}{学生证}{xue2sheng5zheng4}{8,5,7}
  \significado{s.}{cartão de identidade de estudante}
\end{verbete}

\begin{verbete}{学术}{xue2shu4}{8,5}
  \significado[个]{s.}{aprendizagem; ciência}
\end{verbete}

\begin{verbete}{学问}{xue2wen4}{8,6}
  \significado[个]{s.}{conhecimento; aprendizagem}
\end{verbete}

\begin{verbete}{学习}{xue2xi2}{8,3}
  \significado{v.}{estudar; aprender}
\end{verbete}

\begin{verbete}{学校}{xue2xiao4}{8,10}
  \significado{s.}{escola; instituição de ensino}
\end{verbete}

\begin{verbete}{学院}{xue2yuan4}{8,9}
  \significado[所]{s.}{instituto}
\end{verbete}

\begin{verbete}{雪}{xue3}{11}[Radical 雨]
  \significado*{s.}{sobrenome Xue}
  \significado[场]{s.}{neve}
\end{verbete}

\begin{verbete}{雪板}{xue3ban3}{11,8}
  \significado{s.}{prancha de \emph{snowboard}}
  \significado{v.}{praticar \textit{snowboard}}
\end{verbete}

\begin{verbete}{雪糕}{xue3gao1}{11,16}
  \significado{s.}{picolé}
\end{verbete}

\begin{verbete}{雪花}{xue3hua1}{11,7}
  \significado{s.}{floco de neve}
\end{verbete}

\begin{verbete}{雪葩}{xue3pa1}{11,12}
  \significado{s.}{sorvete}
\end{verbete}

\begin{verbete}{雪人}{xue3ren2}{11,2}
  \significado{s.}{boneco de neve; \emph{Yeti}}
\end{verbete}

\begin{verbete}{雪山}{xue3shan1}{11,3}
  \significado{s.}{montanha coberta de neve}
\end{verbete}

\begin{verbete}{雪鞋}{xue3xie2}{11,15}
  \significado[双]{s.}{sapatos de neve}
\end{verbete}

\begin{verbete}{血}{xue4}{6}[Radical 血]
  \significado[片]{s.}{sangue}
\end{verbete}

\begin{verbete}{血汗}{xue4han4}{6,6}
  \significado{s.}{(fig.) suor e labuta, trabalho duro}
\end{verbete}

\begin{verbete}{熏香}{xun1xiang1}{14,9}
  \significado{s.}{incenso}
\end{verbete}

\begin{verbete}{巡逻}{xun2luo2}{6,11}
  \significado{s.}{patrulha}
  \significado{v.}{patrulhar (polícia, exército ou marinha)}
\end{verbete}

%%%%% EOF %%%%%


%%%
%%% Y
%%%
\section*{Y}
\addcontentsline{toc}{section}{Y}
%\begin{multicols*}{2}

\begin{verbete}[ya1sui4qian2]{压岁钱}
\begin{pronuncia}{ya1sui4qian2}
\significado{n.}{
dinheiro da sorte|
dinheiro dado às crianças como presente no Ano Novo Chinês
}
\end{pronuncia}
\end{verbete}

\begin{verbete}[ya1]{鸭}
\begin{pronuncia}{ya1}
\significado[只]{n.}{
pato|
gíria: prostituto
}
\end{pronuncia}
\end{verbete}

\begin{verbete}[ya2]{牙}
\begin{pronuncia}{ya2}
\significado[颗]{n.}{
dente; marfim
}
\end{pronuncia}
\end{verbete}

\begin{verbete}[ya2chi3]{牙齿}
\begin{pronuncia}{ya2chi3}
\significado{adv.}{
dental
}
\significado[颗]{n.}{
dente
}
\end{pronuncia}
\end{verbete}

\begin{verbete}[Ya4zhou1]{亚洲}
\begin{pronuncia}{Ya4zhou1}
\significado{n.}{
Ásia
}
\end{pronuncia}
\end{verbete}

\begin{verbete}[yan2se4]{颜色}
\begin{pronuncia}{yan2se4}
\significado{n.}{
cor; pigmento; tintura
}
\end{pronuncia}
\end{verbete}

\begin{verbete}[yan3jing4]{眼镜}
\begin{pronuncia}{yan3jing4}
\significado[副]{n.}{
óculos
}
\end{pronuncia}
\end{verbete}

\begin{verbete}[yan3jing0]{眼睛}
\begin{pronuncia}{yan3jing0}
\significado[只,双]{n.}{
olho(s)
}
\end{pronuncia}
\end{verbete}

\begin{verbete}[yang3]{养}
\begin{pronuncia}{yang3}
\significado{v.}{
criar (animais ou filhos), plantar (flores), etc
}
\end{pronuncia}
\end{verbete}

\begin{verbete}[yang4zi0]{样子}
\begin{pronuncia}{yang4zi0}
\significado{n.}{
aparência; forma; modelo
}
\end{pronuncia}
\end{verbete}

\begin{verbete}[yao1]{腰}
\begin{pronuncia}{yao1}
\significado{n.}{
cintura
}
\end{pronuncia}
\end{verbete}

\begin{verbete}[yao4]{药}
\begin{pronuncia}{yao4}
\significado[种,服,味]{n.}{
medicamento; remédio; droga
}
\end{pronuncia}
\end{verbete}

\begin{verbete}[yao4]{要}
\begin{pronuncia}{yao4}
\significado{v./v.o.}{
querer; precisar
}
\end{pronuncia}
\end{verbete}

\begin{verbete}[yao4shi0]{要是}
\begin{pronuncia}{yao4shi0}
\significado{conj.}{
se
}
\end{pronuncia}
\end{verbete}

\begin{verbete}[yao4shi0 ...\  de0hua0]{要是······的话}
\begin{pronuncia}[\\]{yao4shi0 ...\  de0hua0}
\significado{conj.}{
se ... no caso de
}
\end{pronuncia}
\end{verbete}

\begin{verbete}[ye2ye0]{爷爷}
\begin{pronuncia}{ye2ye0}
\significado[个]{n.}{
avô (paterno)
}
\end{pronuncia}
\end{verbete}

\begin{verbete}[ye3]{也}
\begin{pronuncia}{ye3}
\significado{adv.}{
também
}
\end{pronuncia}
\end{verbete}

\begin{verbete}[ye4li0]{夜里}
\begin{pronuncia}{ye4li0}
\significado{p.t.}{
noite
}
\end{pronuncia}
\end{verbete}

\begin{verbete}[yi1]{一}
\begin{pronuncia}{yi1}[(quando usado sozinho)]
\significado{num.}{
um, uma; 1
}
\end{pronuncia}
\begin{pronuncia}{yi2}[(antes de quarto tom)]
\significado{num.}{
um, uma; 1|
um, uma (artigo)
}
\end{pronuncia}
\begin{pronuncia}{yi4}
\significado{num.}{
um, uma; 1|
um, uma (artigo)
}
\end{pronuncia}
\end{verbete}

\begin{verbete}[yi2]{一}
\begin{pronuncia}{yi2}[(antes de quarto tom)]
\significado{num.}{
um, uma; 1|
um, uma (artigo)
}
\end{pronuncia}
\begin{pronuncia}{yi1}[(quando usado sozinho)]
\significado{num.}{
um, uma; 1
}
\end{pronuncia}
\begin{pronuncia}{yi4}
\significado{num.}{
um, uma; 1|
um, uma (artigo)
}
\end{pronuncia}
\end{verbete}

\begin{verbete}[yi2ban4]{一半}
\begin{pronuncia}{yi2ban4}
\significado{adj.}{
metade
}
\end{pronuncia}
\end{verbete}

\begin{verbete}[yi2ding4]{一定}
\begin{pronuncia}{yi2ding4}
\significado{adv.}{
certamente; definitivamente
}
\end{pronuncia}
\end{verbete}

\begin{verbete}[yi2gong4]{一共}
\begin{pronuncia}{yi2gong4}
\significado{adv.}{
tudo; no local
}
\end{pronuncia}
\end{verbete}

\begin{verbete}[yi2xia4]{一下}
\begin{pronuncia}{yi2xia4}
\significado{adv.}{
em um curto tempo; rapidamente
}
\end{pronuncia}
\end{verbete}

\begin{verbete}[yi2yang4]{一样}
\begin{pronuncia}{yi2yang4}
\significado{adj.}{
igual; mesmo, mesma
}
\end{pronuncia}
\end{verbete}

\begin{verbete}[yi4]{一}
\begin{pronuncia}{yi4}
\significado{num.}{
um, uma; 1|
um, uma (artigo)
}
\end{pronuncia}
\begin{pronuncia}{yi2}[(antes de quarto tom)]
\significado{num.}{
um, uma; 1|
um, uma (artigo)
}
\end{pronuncia}
\begin{pronuncia}{yi1}[(quando usado sozinho)]
\significado{num.}{
um, uma; 1
}
\end{pronuncia}
\end{verbete}

\begin{verbete}[yi4ban1]{一般}
\begin{pronuncia}{yi4ban1}
\significado{adj.}{
geral; comum; normal
}
\significado{adv.}{
normalmente
}
\end{pronuncia}
\end{verbete}

\begin{verbete}[yi4dianr3]{一点儿}
\begin{pronuncia}{yi4dianr3}
\significado{adv.}{
um pouco (``adj.+一点儿'' ou ``一点儿+n.'')
}
\end{pronuncia}
\end{verbete}

\begin{verbete}[yi4huir4]{一会儿}
\begin{pronuncia}{yi4huir4}
\significado{adv.}{
daqui a pouco tempo; pouco tempo
}
\end{pronuncia}
\end{verbete}

\begin{verbete}[yi4qi3]{一起}
\begin{pronuncia}{yi4qi3}
\significado{adv.}{
juntamente; em conjunto
}
\end{pronuncia}
\end{verbete}

\begin{verbete}[yi4zhi2]{一直}
\begin{pronuncia}{yi4zhi2}
\significado{adv.}{
diretamente; sempre
}
\end{pronuncia}
\end{verbete}

\begin{verbete}[yi4xie1]{一些}
\begin{pronuncia}{yi4xie1}
\significado{pron.}{
uns, umas; alguns, algumas
}
\end{pronuncia}
\end{verbete}

\begin{verbete}[yi1fu0]{衣服}
\begin{pronuncia}{yi1fu0}
\significado[件,套]{n.}{
roupa, vestuário
}
\end{pronuncia}
\end{verbete}

\begin{verbete}[yi1sheng1]{医生}
\begin{pronuncia}{yi1sheng1}
\significado[个,位,名]{n.}{
médico; clínico
}
\end{pronuncia}
\end{verbete}

\begin{verbete}[yi1yuan0]{医院}
\begin{pronuncia}{yi1yuan0}
\significado[所,家,座]{n.}{
hospital
}
\end{pronuncia}
\end{verbete}

\begin{verbete}[yi2he2yuan2]{颐和园}
\begin{pronuncia}{yi2he2yuan2}
\significado{n.}{
Palácio de Verão
}
\end{pronuncia}
\end{verbete}

\begin{verbete}[yi2han4]{遗憾}
\begin{pronuncia}{yi2han4}
\significado{v.}{
ter pena de
}
\end{pronuncia}
\end{verbete}

\begin{verbete}[yi3jing1]{已经}
\begin{pronuncia}{yi3jing1}
\significado{adv.}{
já
}
\end{pronuncia}
\end{verbete}

\begin{verbete}[yi3hou4]{以后}
\begin{pronuncia}{yi3hou4}
\significado{p.t.}{
depois de; depois; após
}
\end{pronuncia}
\end{verbete}

\begin{verbete}[yi3qian2]{以前}
\begin{pronuncia}{yi3qian2}
\significado{p.t.}{
antes de; antes
}
\end{pronuncia}
\end{verbete}

\begin{verbete}[yi4]{亿}
\begin{pronuncia}{yi4}
\significado{num.}{
cem milhões; 100.000.000
}
\end{pronuncia}
\end{verbete}

\begin{verbete}[yi4si0]{意思}
\begin{pronuncia}{yi4si0}
\significado[个]{n.}{
interesse
}
\end{pronuncia}
\end{verbete}

\begin{verbete}[yin1tian1]{阴天}
\begin{pronuncia}{yin1tian1}
\significado{adj.}{
céu muito nublado; céu cinzento
}
\end{pronuncia}
\end{verbete}

\begin{verbete}[yin1wei4]{因为}
\begin{pronuncia}{yin1wei4}
\significado{conj.}{
porque
}
\end{pronuncia}
\end{verbete}

\begin{verbete}[yin1yue4音乐]{音乐}
\begin{pronuncia}{yin1yue4}
\significado[张,曲,段]{n.}{
música
}
\end{pronuncia}
\end{verbete}

\begin{verbete}[yin2hang2]{银行}
\begin{pronuncia}{yin2hang2}
\significado[家,个]{n.}{
banco; agência bancária
}
\end{pronuncia}
\end{verbete}

\begin{verbete}[yin3liao4]{饮料}
\begin{pronuncia}{yin3liao4}
\significado{n.}{
bebida
}
\end{pronuncia}
\end{verbete}

\begin{verbete}[ying1gai1]{应该}
\begin{pronuncia}{ying1gai1}
\significado{v.}{
dever; ter de
}
\end{pronuncia}
\end{verbete}

\begin{verbete}[Ying1guo2]{英国}
\begin{pronuncia}{Ying1guo2}
\significado{n.}{
Reino Unido
}
\end{pronuncia}
\end{verbete}

\begin{verbete}[ying1yu3]{英语}
\begin{pronuncia}{ying1yu3}
\significado{n.}{
inglês, língua inglesa
}
\end{pronuncia}
\end{verbete}

\begin{verbete}[ying1wen2]{英文}
\begin{pronuncia}{ying1wen2}
\significado{n.}{
inglês, língua inglesa
}
\end{pronuncia}
\end{verbete}

\begin{verbete}[you1mei3]{优美}
\begin{pronuncia}{you1mei3}
\significado{adj.}{
gracioso; fino; elegante
}
\end{pronuncia}
\end{verbete}

\begin{verbete}[you2jian4]{邮件}
\begin{pronuncia}{you2jian4}
\significado{n.}{
correspondência; \emph{email}
}
\end{pronuncia}
\end{verbete}

\begin{verbete}[you2ju4]{邮局}
\begin{pronuncia}{you2ju4}
\significado[家,个]{n.}{
correio; agência dos correios
}
\end{pronuncia}
\end{verbete}

\begin{verbete}[you2]{游}
\begin{pronuncia}{you2}
\significado{v.}{
nadar
}
\end{pronuncia}
\end{verbete}

\begin{verbete}[you2yong3]{游泳}
\begin{pronuncia}{you2yong3}
\significado{v.+compl.}{
nadar
}
\end{pronuncia}
\end{verbete}

\begin{verbete}[you2yong3chi2]{游泳池}
\begin{pronuncia}{you2yong3chi2}
\significado[场]{n.}{
piscina
}
\end{pronuncia}
\end{verbete}

\begin{verbete}[you3]{有}
\begin{pronuncia}{you3}
\significado{v.}{
ter; haver
}
\end{pronuncia}
\end{verbete}

\begin{verbete}[you3de0]{有的}
\begin{pronuncia}{you3de0}
\significado{pron.}{
algum, alguma, alguns, algumas
}
\end{pronuncia}
\end{verbete}

\begin{verbete}[you3de0\ shi2hou0]{有的时候}
\begin{pronuncia}{you3de0\ shi2hou0}
\significado{expr.}{
às vezes;
de vez em quando;
de quando em quando
}
\end{pronuncia}
\end{verbete}

\begin{verbete}[you3dianr3]{有点儿}
\begin{pronuncia}{you3dianr3}
\significado{adv.}{
um pouco (``有点儿+n. ou v. mental'')
}
\end{pronuncia}
\end{verbete}

\begin{verbete}[you3ming2]{有名}
\begin{pronuncia}{you3ming2}
\significado{adj.}{
famoso, famosa
}
\end{pronuncia}
\end{verbete}

\begin{verbete}[you3shi2]{有时}
\begin{pronuncia}{you3shi2}
\significado{expr.}{
às vezes;
de vez em quando;
de quando em quando
}
\end{pronuncia}
\end{verbete}

\begin{verbete}[you3shi2hou0]{有时候}
\begin{pronuncia}{you3shi2hou0}
\significado{expr.}{
às vezes;
de vez em quando;
de quando em quando
}
\end{pronuncia}
\end{verbete}

\begin{verbete}[you3yi2si0]{有意思}
\begin{pronuncia}{you3yi2si0}
\significado{adj.}{
interessante
}
\end{pronuncia}
\end{verbete}

\begin{verbete}[you3yong4]{有用}
\begin{pronuncia}{you3yong4}
\significado{adj.}{
útil
}
\end{pronuncia}
\end{verbete}

\begin{verbete}[you4]{右}
\begin{pronuncia}{you4}
\significado{p.l.}{
direita
}
\end{pronuncia}
\end{verbete}

\begin{verbete}[you4bian0]{右边}
\begin{pronuncia}{you4bian0}
\significado{p.l.}{
à direita; ao lado direito
}
\end{pronuncia}
\end{verbete}

\begin{verbete}[you4mian0]{右面}
\begin{pronuncia}{you4mian0}
\significado{p.l.}{
à direita; ao lado direito
}
\end{pronuncia}
\end{verbete}

\begin{verbete}[yong4]{用}
\begin{pronuncia}{yong4}
\significado{v.}{
usar
}
\end{pronuncia}
\end{verbete}

\begin{verbete}[yu2]{鱼}
\begin{pronuncia}{yu2}
\significado[条,尾]{n.}{
peixe
}
\end{pronuncia}
\end{verbete}

\begin{verbete}[yu2pian4]{鱼片}
\begin{pronuncia}{yu2pian4}
\significado{n.}{
fatia de peixe; filé de peixe
}
\end{pronuncia}
\end{verbete}

\begin{verbete}[yu2xiang1rou4si1]{鱼香肉丝}
\begin{pronuncia}{yu2xiang1rou4si1}
\significado{n.}{
tiras de carne de porco salteadas com molho picante
}
\end{pronuncia}
\end{verbete}

\begin{verbete}[yu3]{玉}
\begin{pronuncia}{yu3}
\significado[块]{n.}{
jade
}
\end{pronuncia}
\end{verbete}

\begin{verbete}[yu3mao2qiu2]{羽毛球}
\begin{pronuncia}{yu3mao2qiu2}
\significado{n.}{
badminton
}
\end{pronuncia}
\end{verbete}

\begin{verbete}[yu3]{雨}
\begin{pronuncia}{yu3}
\significado[阵,场]{n.}{
chuva
}
\end{pronuncia}
\end{verbete}

\begin{verbete}[yu3san3]{雨伞}
\begin{pronuncia}{yu3san3}
\significado[把]{n.}{
guarda-chuva
}
\end{pronuncia}
\end{verbete}

\begin{verbete}[yu3yi1]{雨衣}
\begin{pronuncia}{yu3yi1}
\significado[件]{n.}{
impermeável
}
\end{pronuncia}
\end{verbete}

\begin{verbete}[yu3fa3]{语法}
\begin{pronuncia}{yu3fa3}
\significado{n.}{
gramática
}
\end{pronuncia}
\end{verbete}

\begin{verbete}[yu3yan2shi2yan4shi4]{语言实验室}
\begin{pronuncia}[\\]{yu3yan2shi2yan4shi4}
\significado{n.}{
laboratório de línguas
}
\end{pronuncia}
\end{verbete}

\begin{verbete}[yu4bao4]{预报}
\begin{pronuncia}{yu4bao4}
\significado{n.}{
previsão (meteorológica); boletim meteorológico
}
\significado{v.}{
prever (o tempo)
}
\end{pronuncia}
\end{verbete}

\begin{verbete}[yuan2]{元}
\begin{pronuncia}{yuan2}
\significado{p.c.}{
unidade monetária da China
}
\end{pronuncia}
\end{verbete}

\begin{verbete}[yuan3]{远}
\begin{pronuncia}{yuan3}
\significado{adj.}{
longe; longo, longa
}
\end{pronuncia}
\end{verbete}

\begin{verbete}[yuan4zi0]{院子}
\begin{pronuncia}{yuan4zi0}
\significado[个]{n.}{
pátio; jardim; quintal
}
\end{pronuncia}
\end{verbete}

\begin{verbete}[yue1hui4]{约会}
\begin{pronuncia}{yue1hui4}
\significado[次,个]{n.}{
compromisso; encontro marcado
}
\end{pronuncia}
\end{verbete}

\begin{verbete}[yue4]{月}
\begin{pronuncia}{yue4}
\significado[个,轮]{n.}{
mês
}
\end{pronuncia}
\end{verbete}

\begin{verbete}[yue4liang0]{月亮}
\begin{pronuncia}{yue4liang0}
\significado{n.}{
lua
}
\end{pronuncia}
\end{verbete}

\begin{verbete}[yue4du2]{阅读}
\begin{pronuncia}{yue4du2}
\significado{n.}{
leitura
}
\significado{v.}{
ler
}
\end{pronuncia}
\end{verbete}

\begin{verbete}[yue4...\ yue4...]{越······越······}
\begin{pronuncia}{yue4...\ yue4...}
\significado{expr.}{
quanto mais... tanto mais...
}
\end{pronuncia}
\end{verbete}

\begin{verbete}[yue4lai2yue4...]{越来越······}
\begin{pronuncia}{yue4lai2yue4...}
\significado{expr.}{
cada vez mais...
}
\end{pronuncia}
\end{verbete}

\begin{verbete}[yue4lan3shi4]{阅览室}
\begin{pronuncia}{yue4lan3shi4}
\significado[间]{n.}{
sala de leitura
}
\end{pronuncia}
\end{verbete}

\begin{verbete}[Yun2nan2]{云南}
\begin{pronuncia}{Yun2nan2}
\significado{n.}{
Yunnan
}
\end{pronuncia}
\end{verbete}

\begin{verbete}[yun4dong4]{运动}
\begin{pronuncia}{yun4dong4}
\significado[场]{n.}{
esporte; desporto
}
\end{pronuncia}
\end{verbete}

\begin{verbete}[yun4dong4chang3]{运动场}
\begin{pronuncia}{yun4dong4chang3}
\significado{n.}{
campo desportivo; campo de jogos
}
\end{pronuncia}
\end{verbete}

\begin{verbete}[yun4dong4hui4]{运动会}
\begin{pronuncia}{yun4dong4hui4}
\significado[个]{n.}{
jogos desportivos
}
\end{pronuncia}
\end{verbete}

\begin{verbete}[yun4dong4yuan2]{运动员}
\begin{pronuncia}{yun4dong4yuan2}
\significado[名,个]{n.}{
jogador, jogadora; atleta
}
\end{pronuncia}
\end{verbete}

%\end{multicols*}

%%%
%%% Z
%%%
\section*{Z}
\addcontentsline{toc}{section}{Z}
\begin{multicols*}{2}

\begin{verbete}[zai4]{在}
\begin{pronuncia}{zai4}
\significado{adv.}{
para designar ações que estão passando
}
\significado{prep.}{
em
}
\significado{v.}{
estar; ficar
}
\end{pronuncia}
\end{verbete}

\begin{verbete}[zai4]{再}
\begin{pronuncia}{zai4}
\significado{adv.}{
de novo; outra vez
}
\end{pronuncia}
\end{verbete}

\begin{verbete}[zai4jian4]{再见}
\begin{pronuncia}{zai4jian4}
\significado{v.}{
adeus; até à vista; até à próxima; até logo
}
\end{pronuncia}
\end{verbete}

\begin{verbete}[zan2men0]{咱们}
\begin{pronuncia}{zan2men0}
\significado{pron.}{
nós (eu e você)
}
\end{pronuncia}
\end{verbete}

\begin{verbete}[zang1]{脏}
\begin{pronuncia}{zang1}
\significado{adj.}{
sujo
}
\end{pronuncia}
\end{verbete}

\begin{verbete}[zao3]{早}
\begin{pronuncia}{zao3}
\significado{adj.}{
cedo
}
\end{pronuncia}
\end{verbete}

\begin{verbete}[zao3fan4]{早反}
\begin{pronuncia}{zao3fan4}
\significado{n.}{
café da manhã
}
\end{pronuncia}
\end{verbete}

\begin{verbete}[zao3shang0]{早上}
\begin{pronuncia}{zao3shang0}
\significado{p.t.}{
manhã cedo; manhãzinha
}
\end{pronuncia}
\end{verbete}

\begin{verbete}[zen3me0]{怎么}
\begin{pronuncia}{zen3me0}
\significado{interr.}{
como?
}
\end{pronuncia}
\end{verbete}

\begin{verbete}[zen3me0yang4]{怎么样}
\begin{pronuncia}{zen3me0yang4}
\significado{interr.}{
como?; que tal?
}
\end{pronuncia}
\end{verbete}

\begin{verbete}[zhan4]{站}
\begin{pronuncia}{zhan4}
\significado{n.}{
estação; ponto; paragem
}
\end{pronuncia}
\end{verbete}

\begin{verbete}[zhang1]{张}
\begin{pronuncia}{zhang1}
\significado{p.c.}{
para folha de papéis, mapas, etc
}
\end{pronuncia}
\end{verbete}

\begin{verbete}[zhao2ji2]{着急}
\begin{pronuncia}{zhao2ji2}
\significado{adj.}{
inquieto; ansioso
}
\end{pronuncia}
\end{verbete}

\begin{verbete}[zhao3]{找}
\begin{pronuncia}{zhao3}
\significado{v.}{
andar à procura de; procurar; dar troco
}
\end{pronuncia}
\end{verbete}

\begin{verbete}[zhao4pian4]{照片}
\begin{pronuncia}{zhao4pian4}
\significado[张,套,幅]{n.}{
fotografia; foto
}
\end{pronuncia}
\end{verbete}

\begin{verbete}[zhao4xiang4]{照相}
\begin{pronuncia}{zhao4xiang4}
\significado{v.+compl.}{
tirar fotografia
}
\end{pronuncia}
\end{verbete}

\begin{verbete}[zhao4xiang4ji1]{照相机}
\begin{pronuncia}{zhao4xiang4ji1}
\significado[个,架,部,台,只]{n.}{
câmera/máquina fotográfica
}
\end{pronuncia}
\end{verbete}

\begin{verbete}[zhe4]{这}
\begin{pronuncia}{zhe4}
\significado{pron.}{
este, esta, isto
}
\end{pronuncia}
\end{verbete}

\begin{verbete}[zhe4li0]{这里}
\begin{pronuncia}{zhe4li0}
\significado{pron.}{
aqui
}
\end{pronuncia}
\end{verbete}

\begin{verbete}[zhe4me0]{这么}
\begin{pronuncia}{zhe4me0}
\significado{adv.}{
como este; desta maneira
}
\end{pronuncia}
\end{verbete}

\begin{verbete}[zhe4xie1]{这些}
\begin{pronuncia}{zhe4xie1}
\significado{pron.}{
estes, estas
}
\end{pronuncia}
\end{verbete}

\begin{verbete}[zher4]{这儿}
\begin{pronuncia}{zher4}
\significado{pron.}{
aqui
}
\end{pronuncia}
\end{verbete}

\begin{verbete}[Zhe4jiang1]{浙江}
\begin{pronuncia}{Zhe4jiang1}
\significado{n.}{
Zhejiang
}
\end{pronuncia}
\end{verbete}

\begin{verbete}[zhen1]{真}
\begin{pronuncia}{zhen1}
\significado{adv.}{
que...tão...!|
realmente
}
\end{pronuncia}
\end{verbete}

\begin{verbete}[zheng4qian2]{挣钱}
\begin{pronuncia}{zheng4qian2}
\significado{v.+compl.}{
ganhar dinheiro
}
\end{pronuncia}
\end{verbete}

\begin{verbete}[zheng4zai4]{正在}
\begin{pronuncia}{zheng4zai4}
\significado{adv.}{
estar a + inf.; estar + ger.
}
\end{pronuncia}
\end{verbete}

\begin{verbete}[zhi1]{支}
\begin{pronuncia}{zhi1}
\significado{p.c.}{
para caneta, lápis, etc
}
\end{pronuncia}
\end{verbete}

\begin{verbete}[zhi1]{只}
\begin{pronuncia}{zhi1}
\significado{p.c.}{
para pássaros, gatos, cãezinhos, etc
}
\end{pronuncia}
\begin{pronuncia}{zhi3}
\significado{adv.}{
apenas; só
}
\end{pronuncia}
\end{verbete}

\begin{verbete}[zhi1dao0]{知道}
\begin{pronuncia}{zhi1dao0}
\significado{v.}{
conhecer; saber
}
\end{pronuncia}
\end{verbete}

\begin{verbete}[zhi2yuan2]{职员}
\begin{pronuncia}{zhi2yuan2}
\significado{n.}{
empregado, empregada
}
\end{pronuncia}
\end{verbete}

\begin{verbete}[zhi3]{只}
\begin{pronuncia}{zhi3}
\significado{adv.}{
apenas; só
}
\end{pronuncia}
\begin{pronuncia}{zhi1}
\significado{p.c.}{
para pássaros, gatos, cãezinhos, etc
}
\end{pronuncia}
\end{verbete}

\begin{verbete}[Zhong1guo2]{中国}
\begin{pronuncia}{Zhong1guo2}
\significado{n.}{
China
}
\end{pronuncia}
\end{verbete}

\begin{verbete}[Zhong1guo2tong1]{中国通}
\begin{pronuncia}{Zhong1guo2tong1}
\significado{n.}{
conhecedor da China; especialista em tudo sobre a China
}
\end{pronuncia}
\end{verbete}

\begin{verbete}[zhong1jian1]{中间}
\begin{pronuncia}{zhong1jian1}
\significado{p.l.}{
central; centro; no meio
}
\end{pronuncia}
\end{verbete}

\begin{verbete}[zhong1wen2]{中文}
\begin{pronuncia}{zhong1wen2}
\significado{n.}{
chinês; língua chinesa
}
\end{pronuncia}
\end{verbete}

\begin{verbete}[zhong1xue2]{中学}
\begin{pronuncia}{zhong1xue2}
\significado[个]{n.}{
escola ensino médio
}
\end{pronuncia}
\end{verbete}

\begin{verbete}[zhong1xue2sheng1]{中学生}
\begin{pronuncia}{zhong1xue2sheng1}
\significado{n.}{
estudante da escola ensino médio
}
\end{pronuncia}
\end{verbete}

\begin{verbete}[zhong1xun2]{中询}
\begin{pronuncia}{zhong1xun2}
\significado{p.t.}{
segunda dezena do mês; meio do mês; em meados do mês
}
\end{pronuncia}
\end{verbete}

\begin{verbete}[zhong1]{钟}
\begin{pronuncia}{zhong1}
\significado{p.c.}{
hora
}
\end{pronuncia}
\end{verbete}

\begin{verbete}[zhong3]{种}
\begin{pronuncia}{zhong3}
\significado{p.c.}{
para tipos, espécies e gêneros
}
\end{pronuncia}
\end{verbete}

\begin{verbete}[zhong4]{重}
\begin{pronuncia}{zhong4}
\significado{adj.}{
pesado, pesada
}
\end{pronuncia}
\end{verbete}

\begin{verbete}[zhong4liang4]{重量}
\begin{pronuncia}{zhong4liang4}
\significado[个]{n.}{
peso
}
\end{pronuncia}
\end{verbete}

\begin{verbete}[zhou1mo4]{周末}
\begin{pronuncia}{zhou1mo4}
\significado{n.}{
final-de-semana
}
\end{pronuncia}
\end{verbete}

\begin{verbete}[zhu1]{猪}
\begin{pronuncia}{zhu1}
\significado[口,头]{n.}{
porco, porca
}
\end{pronuncia}
\end{verbete}

\begin{verbete}[Zhu3xi2]{主席}
\begin{pronuncia}{Zhu3xi2}
\significado[个,位]{n.}{
Presidente (da China); Primeiro-Ministro
}
\end{pronuncia}
\end{verbete}

\begin{verbete}[zhu4]{住}
\begin{pronuncia}{zhu4}
\significado{v.}{
morar; viver; alojar-se
}
\end{pronuncia}
\end{verbete}

\begin{verbete}[zhu4zhai2]{住宅}
\begin{pronuncia}{zhu4zhai2}
\significado{n.}{
residência
}
\end{pronuncia}
\end{verbete}

\begin{verbete}[zhu4ce4]{注册}
\begin{pronuncia}{zhu4ce4}
\significado{v.}{
inscrever-se; matricular-se
}
\end{pronuncia}
\end{verbete}

\begin{verbete}[zhu4]{祝}
\begin{pronuncia}{zhu4}
\significado{v.}{
desejar (exprimir um bom desejo);
congratular
}
\end{pronuncia}
\end{verbete}

\begin{verbete}[zhu4fu4]{嘱咐}
\begin{pronuncia}{zhu4fu4}
\significado{v.}{
ordenar; dizer; exortar
}
\end{pronuncia}
\end{verbete}

\begin{verbete}[zhuan1ye4]{专业}
\begin{pronuncia}{zhuan1ye4}
\significado[门,个]{n.}{
área de atuação; especialidade
}
\end{pronuncia}
\end{verbete}

\begin{verbete}[zhuo1zi0]{桌子}
\begin{pronuncia}{zhuo1zi0}
\significado[张,套]{n.}{
mesa
}
\end{pronuncia}
\end{verbete}

\begin{verbete}[zi3se4]{紫色}
\begin{pronuncia}{zi3se4}
\significado{n.}{
cor roxa
}
\end{pronuncia}
\end{verbete}

\begin{verbete}[zi4]{字}
\begin{pronuncia}{zi4}
\significado[个]{n.}{
carácter; letra; símbolo; palavra
}
\end{pronuncia}
\end{verbete}

\begin{verbete}[zi4ji3]{自己}
\begin{pronuncia}{zi4ji3}
\significado{pron.}{
a si próprio; próprio
}
\end{pronuncia}
\end{verbete}

\begin{verbete}[zi4xing2che1]{自行车}
\begin{pronuncia}{zi4xing2che1}
\significado[辆]{n.}{
bicicleta
}
\end{pronuncia}
\end{verbete}

\begin{verbete}[zi4wo3]{自我}
\begin{pronuncia}{zi4wo3}
\significado{pron.}{
a si mesmo; eu próprio|
auto-...
}
\end{pronuncia}
\end{verbete}

\begin{verbete}[Zong3du1]{总督}
\begin{pronuncia}{Zong3du1}
\significado{n.}{
Governador; Governador-Geral; Vice-Rei
}
\end{pronuncia}
\end{verbete}

\begin{verbete}[Zong3li3]{总理}
\begin{pronuncia}{Zong3li3}
\significado[个,位,名]{n.}{
Primeiro-Ministro
}
\end{pronuncia}
\end{verbete}

\begin{verbete}[Zong3tong3]{总统}
\begin{pronuncia}{Zong3tong3}
\significado[个,位,名,届]{n.}{
Presidente (de um país)
}
\end{pronuncia}
\end{verbete}

\begin{verbete}[zou3]{走}
\begin{pronuncia}{zou3}
\significado{v.}{
andar; caminhar
}
\end{pronuncia}
\end{verbete}

\begin{verbete}[zu2qiu2]{足球}
\begin{pronuncia}{zu2qiu2}
\significado[个]{n.}{
futebol; bola de futebol
}
\end{pronuncia}
\end{verbete}

\begin{verbete}[zui3ba0]{嘴巴}
\begin{pronuncia}{zui3ba0}
\significado[张]{n.}{
boca
}
\significado[个]{n.}{
bofetada na cara
}
\end{pronuncia}
\end{verbete}

\begin{verbete}[zui4]{最}
\begin{pronuncia}{zui4}
\significado{adv.}{
o mais, a mais|
grau superlativo relativo de superioridade
}
\end{pronuncia}
\end{verbete}

\begin{verbete}[zui4hou4]{最后}
\begin{pronuncia}{zui4hou4}
\significado{adj.}{
final; último
}
\end{pronuncia}
\end{verbete}

\begin{verbete}[zui4jin4]{最近}
\begin{pronuncia}{zui4jin4}
\significado{adv.}{
ultimamente; recentemente
}
\end{pronuncia}
\end{verbete}

\begin{verbete}[zuo2tian1]{昨天}
\begin{pronuncia}{zuo2tian1}
\significado{p.t.}{
ontem
}
\end{pronuncia}
\end{verbete}

\begin{verbete}[zuo3]{左}
\begin{pronuncia}{zuo3}
\significado{p.l.}{
esquerda
}
\end{pronuncia}
\end{verbete}

\begin{verbete}[zuo3bian0]{左边}
\begin{pronuncia}{zuo3bian0}
\significado{p.l.}{
esquerda; lado esquerdo
}
\end{pronuncia}
\end{verbete}

\begin{verbete}[zuo3mian0]{左面}
\begin{pronuncia}{zuo3mian0}
\significado{p.l.}{
esquerda; lado esquerdo
}
\end{pronuncia}
\end{verbete}

\begin{verbete}[zuo3you4]{左右}
\begin{pronuncia}{zuo3you4}
\significado{part.}{
cerca de; aproximadamente
}
\end{pronuncia}
\end{verbete}

\begin{verbete}[zuo4]{坐}
\begin{pronuncia}{zuo4}
\significado{v.}{
sentar-se|
andar de carro, ônibus, trem, avião, etc
}
\end{pronuncia}
\end{verbete}

\begin{verbete}[zuo4]{做}
\begin{pronuncia}{zuo4}
\significado{v.}{
fazer
}
\end{pronuncia}
\end{verbete}

\end{multicols*}

\end{multicols}

\clearpage
\pagestyle{plain}
\chapter{Termos Gramaticais Chineses}
\begin{tabular}{lll}
substantivo/nome       & \textbf{s.}        & 名词 \\
palavra de lugar       & \textbf{p.d.l.}    & 处所词 \\
palavra de localização & \textbf{p.l.}      & 方位词 \\
palavra de tempo       & \textbf{p.t.}      & 时间词 \\
verbo                  & \textbf{v.}        & 动词 \\
verbo direcional       & \textbf{v.d.}      & 趣向\hspace{1em}动词 \\
verbo optativo         & \textbf{v.o.}      & 能缘\hspace{1em}动词 \\
adjetivo               & \textbf{adj.}      & 形容词 \\
numeral                & \textbf{num.}      & 数词 \\
palavra classificadora & \textbf{p.c.}      & 两量词 \\
pronome                & \textbf{pron.}     & 代词 \\
interrogativo          & \textbf{interr.}   & 疑问词 \\
advérbio               & \textbf{adv.}      & 副词 \\
preposição             & \textbf{prep.}     & 介词 \\
conjunção              & \textbf{conj.}     & 连词 \\
partícula              & \textbf{part.}     & 助词 \\
sujeito                & \textbf{suj.}      & 主语 \\
objeto                 & \textbf{obj.}      & 宾语 \\
atributo               & \textbf{atrib.}    & 定语 \\
adjunto adverbial      & \textbf{a.adv.}    & 状语 \\
complemento            & \textbf{compl.}    & 补语 \\
verbo+complemento      & \textbf{v.+compl.} & 动宾式\hspace{1em}离合词 \\
expressão idiomática   & \textbf{expr.}     & \\
interjeição            & \textbf{interj.}   & \\
\end{tabular}


\clearpage
\pagestyle{plain}
\chapter{Radicais Kangxi}
\chapter{Radicais Chineses}

\begin{multicols}{3}
\begin{tabular}{rllll}
\hline
  Nº & Radical & Variante & Tradução & Pinyin \\
\hline
  1  & 一 && um           & \pinyin{yi1}         \\
  2  & 丨 && linha        & \pinyin{shu4}        \\
  3  & 丶 && ponto        & \pinyin{dian3}       \\
  4  & 丿 &乀,乁 & golpear & \pinyin{pie3}       \\
  5  & 乙 &乚,乛 & segundo & \pinyin{yi3}         \\
  6  & 亅 && gancho       & \pinyin{gou1}        \\
  7  & 二 && dois         & \pinyin{er4}         \\
  8  & 亠 && membro       & \pinyin{tou2}        \\
  9  & 人 &亻 & homem     & \pinyin{ren2}        \\
 10  & 儿 && pernas       & \pinyin{er2}         \\
 11  & 入 && entra        & \pinyin{ru4}         \\
 12  & 八 &丷 & oito      & \pinyin{ba1}         \\
 13  & 冂 && caixa de baixo & \pinyin{jiong3}    \\
 14  & 冖 && sobre        & \pinyin{mi4}         \\
 15  & 冫 && gelo         & \pinyin{bing1}       \\
 16  & 几 && mesa         & \pinyin{ji1},\pinyin{ji3} \\
 17  & 凵 && caixa aberta & \pinyin{qu3}         \\
 18  & 刀 &刂 & faca      & \pinyin{dao1}        \\
 19  & 力 && poder        & \pinyin{li4}         \\
 20  & 勹 && embrulho     & \pinyin{bao1}        \\
 21  & 匕 && colher       & \pinyin{bi3}         \\
 22  & 匚 && caixa aberta & \pinyin{fang1}       \\
 23  & 匸 && esconderijo anexo & \pinyin{xi3}    \\
 24  & 十 && dez          & \pinyin{shi2}        \\
 25  & 卜 && místico      & \pinyin{bu3}         \\
 26  & 卩 && foca         & \pinyin{jie2}        \\
 27  & 厂 && penhasco     & \pinyin{han4}        \\
 28  & 厶 && privado      & \pinyin{si1}         \\
 29  & 又 && novamente    & \pinyin{you4}        \\
 30  & 口 && boca         & \pinyin{kou3}        \\
 31  & 囗 && lugar        & \pinyin{wei2}        \\
 32  & 土 && Terra        & \pinyin{tu3}         \\
 33  & 士 && guerreiro    & \pinyin{shi4}        \\
 34  & 夂 && ir           & \pinyin{zhi1}        \\
 35  & 夊 && devagar      & \pinyin{sui1}        \\
 36  & 夕 && tarde        & \pinyin{xi1}         \\
 37  & 大 && grande       & \pinyin{da4}         \\
 38  & 女 && mulher       & \pinyin{nv3}         \\
 39  & 子 && criança      & \pinyin{zi3}         \\
 40  & 宀 && cobertura    & \pinyin{mian2}       \\
 41  & 寸 && polegada     & \pinyin{cun4}        \\
 42  & 小 && pequeno      & \pinyin{xiao3}       \\
 43  & 尢 &尣 & coxo      & \pinyin{you2}        \\
 44  & 尸 && cadáver      & \pinyin{shi1}        \\
 45  & 屮 && brotar       & \pinyin{che4}        \\
 46  & 山 && montanha     & \pinyin{shan1}       \\
 47  & 川 &巛,巜& rio     & \pinyin{chuan1}      \\
 48  & 工 && trabalho     & \pinyin{gong1}       \\
 49  & 己 && a si mesmo   & \pinyin{ji3}         \\
 50  & 巾 && turbante     & \pinyin{jin1}        \\
 51  & 干 && seco         & \pinyin{gan1}        \\
 52  & 幺 && fio curto    & \pinyin{yao1}        \\
 53  & 广 && vasto        & \pinyin{guang3}      \\
 54  & 廴 && passo longo  & \pinyin{yin3}        \\
 55  & 廾 && duas mãos    & \pinyin{gong3}       \\
 56  & 弋 && atirar flecha & \pinyin{yi4}        \\
 57  & 弓 && arco         & \pinyin{gong1}       \\
 58  & 彐 &彑 & focinho   & \pinyin{ji4}         \\
 59  & 彡 && cerdas       & \pinyin{shan1}       \\
 60  & 彳 && dupla        & \pinyin{chi4}        \\
 61  & 心 &忄& coração    & \pinyin{xin1}        \\
 62  & 戈 && lança        & \pinyin{ge1}         \\
 63  & 户 && por          & \pinyin{hu4}         \\
 64  & 手 &扌& mão        & \pinyin{shou3}       \\
 65  & 支 && ramo         & \pinyin{zhi1}        \\
 66  & 攴 &攵 & batida    & \pinyin{pu1}         \\
 67  & 文 && escrita      & \pinyin{wen2}        \\
 68  & 斗 && mergulhador  & \pinyin{dou3}        \\
 69  & 斤 && eixo         & \pinyin{jin1}        \\
 70  & 方 && quadrado     & \pinyin{fang1}       \\
 71  & 无 && não          & \pinyin{wu2}         \\
 72  & 日 && sol          & \pinyin{ri4}         \\
 73  & 曰 && dizer        & \pinyin{yue1}        \\
 74  & 月 && lua          & \pinyin{yue4}        \\
 75  & 木 && árvore       & \pinyin{mu4}         \\
 76  & 欠 && falta        & \pinyin{qian4}       \\
 77  & 止 && parar        & \pinyin{zhi3}        \\
 78  & 歹 && morte        & \pinyin{dai3}        \\
 79  & 殳 && arma         & \pinyin{shu1}        \\
 80  & 母 && mãe          & \pinyin{mu3}         \\
 81  & 比 && comparar     & \pinyin{bi3}         \\
 82  & 毛 && pelo & \pinyin{mao2}        \\
 83  & 氏 && clã & \pinyin{shi4}        \\
 84  & 气 && ar & \pinyin{qi4}         \\
 85  & 水 &氵 & água & \pinyin{shui3}       \\
 86  & 火 &灬 & fogo & \pinyin{huo3}        \\
 87  & 爪 &爫 & garra & \pinyin{zhao3}       \\
 88  & 父 && pai & \pinyin{fu4}         \\
 89  & 爻 && linha & \pinyin{yao2}        \\
 90  & 爿 &丬 & meio tronco & \pinyin{pan2}      \\
 91  & 片 && fatia & \pinyin{pian4}       \\
 92  & 牙 && dente & \pinyin{ya2}         \\
 93  & 牛 &牜 & vaca & \pinyin{niu2}        \\
 94  & 犬 &犭 & cão & \pinyin{quan3}       \\
 95  & 玄 && profundo & \pinyin{xuan2}       \\
 96  & 玉 &王 & jade & \pinyin{yu4}         \\
 97  & 瓜 && melão & \pinyin{gua1}         \\
 98  & 瓦 && telha & \pinyin{wa3}         \\
 99  & 甘 && doce & \pinyin{gan1}         \\
100  & 生 && vida & \pinyin{sheng1}         \\
101  & 用 && usar & \pinyin{yong4}         \\
102  & 田 && campo & \pinyin{tian2}         \\
103  & 疋 && roupa & \pinyin{pi3} \\
104  & 疒 && doença & \pinyin{ne4} \\
105  & 癶 && pegadas & \pinyin{bo1} \\
106  & 白 && branco & \pinyin{bai2} \\
107  & 皮 && pele & \pinyin{pi2} \\
108  & 皿 && prato & \pinyin{min3} \\
109  & 目 && olho & \pinyin{mu4} \\
110  & 矛 && lança & \pinyin{mao2} \\
111  & 矢 && seta & \pinyin{shi3} \\
112  & 石 && pedra & \pinyin{shi2} \\
113  & 示 &礻 & espírito & \pinyin{shi4} \\
114  & 禸 && rastrear & \pinyin{rou2} \\
115  & 禾 && grão & \pinyin{he2} \\
116  & 穴 && caverna & \pinyin{xue2} \\
117  & 立 && ficar em pé & \pinyin{li4} \\
118  & 竹 &⺮ & bambu & \pinyin{zhu2} \\
119  & 米 && arroz & \pinyin{mi3} \\
120  & 糸 &纟& seda & \pinyin{mi4} \\
121  & 缶 && pote & \pinyin{fou3} \\
122  & 网 &罒 & rede & \pinyin{wang3} \\
123  & 羊 && ovelha & \pinyin{yang2} \\
124  & 羽 && pena & \pinyin{yu3} \\
125  & 老 && velho & \pinyin{lao3} \\
126  & 而 && e & \pinyin{er2} \\
127  & 耒 && arado & \pinyin{lei3} \\
128  & 耳 && orelha & \pinyin{er3} \\
129  & 聿 && escova & \pinyin{yu4} \\
130  & 肉 && carne & \pinyin{rou4} \\
131  & 臣 && ministro & \pinyin{chen2} \\
132  & 自 && auto- & \pinyin{zi4} \\
133  & 至 && chegar & \pinyin{zhi4} \\
134  & 臼 && argamassa & \pinyin{jiu4} \\
135  & 舌 && língua & \pinyin{she2} \\
136  & 舛 && opor & \pinyin{chuan3} \\
137  & 舟 && barco & \pinyin{zhou1} \\
138  & 艮 && pausa & \pinyin{gen3} \\
139  & 色 && cor & \pinyin{se4} \\
140  & 艸 &艹 & grama & \pinyin{cao3} \\
141  & 虍 && tigre & \pinyin{hu1} \\
142  & 虫 && inseto & \pinyin{chong2} \\
143  & 血 && sangue & \pinyin{xue4} \\
144  & 行 && andar & \pinyin{xing2} \\
145  & 衣 &衤 & roupa & \pinyin{yi1} \\
146  & 襾 &覀 & oeste & \pinyin{ya4} \\
147  & 見 &见 & ver & \pinyin{jian4} \\
148  & 角 && chifre & \pinyin{jiao3} \\
149  & 言 &讠 & palavra & \pinyin{yan2} \\
150  & 谷 && vale & \pinyin{gu3} \\
151  & 豆 && grão & \pinyin{dou4} \\
152  & 豕 && porco & \pinyin{shi3} \\
153  & 豸 && texugo & \pinyin{zhi4} \\
154  & 貝 &贝 & concha & \pinyin{bei4} \\
155  & 赤 && vermelho & \pinyin{chi4} \\
156  & 走 && andar & \pinyin{zou3} \\
157  & 足 &⻊ & pé & \pinyin{zu2} \\
158  & 身 && corpo & \pinyin{shen1} \\
159  & 車 &车 & carro & \pinyin{che1} \\
160  & 辛 && amargo & \pinyin{xin1} \\
161  & 辰 && manhã & \pinyin{chen2} \\
162  & 辵 &辶 & caminhar & \pinyin{chuo4} \\
163  & 邑 &阝 & cidade & \pinyin{yi4} \\
164  & 酉 && vinho & \pinyin{you3} \\
165  & 釆 && distinto & \pinyin{bian4} \\
166  & 里 && aldeia & \pinyin{li3} \\
167  & 金 && ouro & \pinyin{jin1} \\
168  & 長 &长 & longo & \pinyin{zhang3} \\
169  & 門 &门 & portão & \pinyin{men2} \\
170  & 阜 &阝 & monte & \pinyin{fu4} \\
171  & 隶 && escravo & \pinyin{li4} \\
172  & 隹 && pássaro de cauda curta & \pinyin{zhui1} \\
173  & 雨 && chuva & \pinyin{yu3} \\
174  & 青 && azul & \pinyin{qing1} \\
175  & 非 && errado & \pinyin{fei1} \\
176  & 面 && face & \pinyin{mian4} \\
177  & 革 && couro & \pinyin{ge2} \\
178  & 韋 &韦 & couro tingido & \pinyin{wei2} \\
179  & 韭 && parecia & \pinyin{jiu3} \\
180  & 音 && som & \pinyin{yin1} \\
181  & 頁 &页 & folha & \pinyin{ye4} \\
182  & 風 &风 & vento & \pinyin{feng1} \\
183  & 飛 &飞 & mosca & \pinyin{fei1} \\
184  & 食 &饣,飠 & alimento & \pinyin{shi2} \\
185  & 首 && cabeça & \pinyin{shou3} \\
186  & 香 && perfume & \pinyin{xiang1} \\
187  & 馬 &马 & cavalo & \pinyin{ma3} \\
188  & 骨 && osso & \pinyin{gu3} \\
189  & 高 && alto & \pinyin{gao1} \\
190  & 髟 && cabelo & \pinyin{biao1} \\
191  & 鬥 && luta & \pinyin{dou4} \\
192  & 鬯 && vinho & \pinyin{chang4} \\
193  & 鬲 && separado & \pinyin{ge2} \\
194  & 鬼 && fantasma & \pinyin{gui3} \\
195  & 魚 &鱼 & peixe & \pinyin{yu2} \\
196  & 鳥 &鸟 & pássaro & \pinyin{niao3} \\
197  & 鹵 && sal & \pinyin{lu3} \\
198  & 鹿 && veado & \pinyin{lu4} \\
199  & 麥 &麦 & trigo & \pinyin{mai4} \\
200  & 麻 && cânhamo & \pinyin{ma2} \\
201  & 黃 && amarelo & \pinyin{huang4} \\
202  & 黍 && nação & \pinyin{shu3} \\
203  & 黑 && preto & \pinyin{hei1} \\
204  & 黹 && costura & \pinyin{zhi3} \\
205  & 黽 &黾 & rã & \pinyin{mian3} \\
206  & 鼎 && tripé & \pinyin{ding3} \\
207  & 鼓 && tambor & \pinyin{gu3} \\
208  & 鼠 &鼡 & rato & \pinyin{shu3} \\
209  & 鼻 && nariz & \pinyin{bi2} \\
210  & 齊 &齐 & até & \pinyin{qi2} \\
211  & 齒 &齿 & dente & \pinyin{chi3} \\
212  & 龍 &龙 & dragão & \pinyin{long2} \\
213  & 龜 &龟 & tartaruga & \pinyin{gui1} \\
214  & 龠 && flauta & \pinyin{yue4} \\
\end{tabular}
\end{multicols}


\printindex[pstroke]
\printindex[pradical]

\end{document}

%%%%% EOF %%%%
