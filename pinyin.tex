
%%%%%%%%%%%%%%%%%%%%%%%%%%%%%%%%%%%%%%%%%
% LuaLaTex
%
% Dicionário Chinês -> Português
% Autor: Luiz Eduardo Roncato Cordeiro
%
% Licença:
% CC BY-NC-SA 3.0 (http://creativecommons.org/licenses/by-nc-sa/3.0/)
%%%%%%%%%%%%%%%%%%%%%%%%%%%%%%%%%%%%%%%%%

\documentclass[a4paper,9pt,twoside,openright,book]{memoir}

\usepackage[brazilian]{babel}
\usepackage{fontspec}
\usepackage[dvipsnames]{xcolor}
\usepackage{imakeidx}
\usepackage[inline]{enumitem}
\usepackage{zhnumber}
\usepackage{tikz}
\usepackage[hyperindex]{hyperref}
\usepackage{pifont}
\usepackage{xstring}
\usepackage{xifthen}
\usepackage{tabularray}
\usepackage[most]{tcolorbox}
\usepackage{luacode}
\usepackage{stackengine}

% Meus Comandos
%%%%%%%%%%%%%%%%%%%%%%%%%%%%%%%%%%%%%%%%%%%%%%%%%%%%%%%%%%%%%%%%%%%%%%%%%%%%%%%
%%%%%%%%%%%%%%%%%%%%%%%%%%%%%%%%%%%%%%%%%%%%%%%%%%%%%%%%%%%%%%%%%%%%%%%%%%%%%%%
%%%%%                                                                     %%%%%
%%%%% Funções e Ajustes dos Documentos do Dicionário                      %%%%%
%%%%%                                                                     %%%%%
%%%%%%%%%%%%%%%%%%%%%%%%%%%%%%%%%%%%%%%%%%%%%%%%%%%%%%%%%%%%%%%%%%%%%%%%%%%%%%%
%%%%%%%%%%%%%%%%%%%%%%%%%%%%%%%%%%%%%%%%%%%%%%%%%%%%%%%%%%%%%%%%%%%%%%%%%%%%%%%

%%% Espaçamento das linhas normal
\SingleSpacing

%%% Hyperref em modo 'draft' não gera os hiperlinks
\hypersetup{final}

%%% Largura da entrada do verbete
\def\entrywidth{.49\textwidth}

%%% Estilo do capítulo, o melhor que encontrei
\chapterstyle{verville}

%%% Sem identação
\setlength{\parindent}{0cm}
\setlength{\parskip}{0.6\baselineskip}

%%% Ajuste das margens do documento
\setlrmarginsandblock{3cm}{2cm}{*}
\setulmarginsandblock{2cm}{*}{1}
\checkandfixthelayout

%%% Pra evitar viúvas e órfãs
\clubpenalty=10000
\widowpenalty=10000
\raggedbottom

%%% Usando a fonte NoTofu do Google.
\babelfont{rm}[
 Renderer=Node,
 Ligatures=TeX,
 BoldFont={NotoSerifCJKsc-SemiBold},
 BoldSlantedFont={NotoSerifCJKsc-SemiBold},
 AutoFakeSlant=0.25,
 SlantedFeatures={FakeSlant=0.25},
 BoldSlantedFeatures={FakeSlant=0.25}]
 {Noto Serif CJK SC Light}
\babelfont{sf}[Renderer=Harfbuzz,Ligatures=TeX]{Noto Sans CJK SC Light}
\babelfont{tt}[Renderer=Harfbuzz,Ligatures=TeX]{Noto Sans Mono CJK SC}

%%% Ajustes do MultiCol: parar com a indentação do primeiro parágrafo
\AddToHook{env/multicols/begin}{\AddToHookNext{para/begin}{\OmitIndent}}

%%% Ajustes do Sumário
\makeatletter
\renewcommand{\@pnumwidth}{2em} 
\renewcommand{\@tocrmarg}{4em}
\makeatother
\renewcommand\cftbeforechapterskip{5pt plus 1pt}

%%% Ajustes da separação das colunas quando em modo texto de 2 colunas
\setlength{\columnsep}{0.8em}
\setlength{\columnseprule}{0.1mm}

%%% Ajustes para o "stackengine"
\renewcommand\stacktype{S}
\renewcommand\stackalignment{c}

%%% Ajustes de Cabeçalhos e Rodapés
\setheadfoot{14pt}{28pt}

% Estilo "plain"
\makefootrule{plain}{\textwidth}{\normalrulethickness}{2pt}
\ifdraftdoc
 \makeevenfoot{plain}{\thepage}{汉葡词典}{Draft}
 \makeoddfoot{plain}{Draft}{汉葡词典}{\thepage}
\else
 \makeevenfoot{plain}{\thepage}{汉葡词典}{}
 \makeoddfoot{plain}{}{汉葡词典}{\thepage}
\fi

% Estilo "dictionary"
\makepagestyle{dictionary}
\makeheadrule{dictionary}{\textwidth}{\normalrulethickness}
\makefootrule{dictionary}{\textwidth}{\normalrulethickness}{2pt}
\ifdraftdoc
 \makeevenhead{dictionary}{\rightmark}{Draft}{\leftmark}
 \makeoddhead{dictionary}{\rightmark}{Draft}{\leftmark}
 \makeevenfoot{dictionary}{\thepage}{汉葡词典}{Draft}
 \makeoddfoot{dictionary}{Draft}{汉葡词典}{\thepage}
\else
 \makeevenhead{dictionary}{\rightmark}{}{\leftmark}
 \makeoddhead{dictionary}{\rightmark}{}{\leftmark}
 \makeevenfoot{dictionary}{\thepage}{汉葡词典}{}
 \makeoddfoot{dictionary}{}{汉葡词典}{\thepage}
\fi

%%% Estilo das Seções
\newcommand{\boxedsec}[1]
 {%
  \begin{tcolorbox}%
   [%
    enhanced,%
    nobeforeafter,%
    before={\noindent},%
    colframe=black,%
    colback=black!15!white,%
    boxrule=2pt,%
    leftrule=2mm,%
    left=0mm,%
    right=0mm,%
    top=0mm,%
    bottom=0mm%
   ]
   \hfill\LARGE\bfseries#1
  \end{tcolorbox}
 }
\setsecheadstyle{\boxedsec}
\setbeforesecskip{1ex plus .25ex minus .25ex}
\setaftersecskip{.25ex plus .25ex minus .25ex}
\newcommand{\sectionbreak}{\phantomsection}

%%% Estilo das caixas dos verbetes
\newtcolorbox{lightbox}%
 {%
  enhanced,%
  size=fbox,%
  colframe=black,%
  colback=white,%
  boxrule=1pt,%
  toprule=3pt,%
  left=0mm,%
  right=0mm,%
  top=0mm,%
  bottom=0mm,%
  middle=0mm,%
  nobeforeafter,%
  segmentation empty,%
  before={\noindent}%
 }
\newtcolorbox{darkbox}%
 {%
  enhanced,%
  size=fbox,%
  colframe=black,%
  colback=black!5!white,%
  boxrule=1pt,%
  toprule=3pt,%
  left=0mm,%
  right=0mm,%
  top=0mm,%
  bottom=0mm,%
  middle=0mm,%
  nobeforeafter,%
  segmentation empty,%
  before={\noindent}%
 }


%%% Variáveis tipo "bool" para dizer se tem ou não os campos
%%% "Veja" e "Veja também" nas definições dos verbetes
\newbool{f_see}
\newbool{f_seealso}

%%% Converte os pinyins numéricos em pinyins com marcação de tom
\directlua{dofile "include/tex-sx-pinyin-tonemarks.lua"}

%%% Comandos genéricos usados no Dicionário

% Função "\pinyin" faz a conversão
\protected\def\pinyin#1{%
 \directlua{packagedata.pinyintones.convert ([==[#1]==])}%
}

% Comando "\dictpinyin", coloca o pinyin entre «»
\NewDocumentCommand{\dictpinyin}{m}{\guillemotleft\pinyin{#1}\guillemotright} 

% Comando "\dpy", gera a string do pinyin utilizada no Dicionário
% Este comando realiza uma série de substituições antes
\NewDocumentCommand{\dpy}{m}%
 {%
  \StrSubstitute{#1}{5}{}[\result]%
  \StrSubstitute{\result}{v}{ü}[\result]%
  \StrSubstitute{\result}{V}{Ü}[\result]%
  \edef\py{\dictpinyin{\result}}%
  \mbox{}\py
 }

% Comando "\&", insere o caracgter "&"
\DeclareRobustCommand{\&}%
 {%
  \ifdim\fontdimen1\font>0pt%
   \textsl{\symbol{`\&}}%
  \else%
   \symbol{`\&}%
  \fi%
 }

% Comando "\dul{text}", sublinha o texto dado
\NewDocumentCommand{\dul}{m}{\underline{#1}}

% Ambiente "enumerate" especial utilizado no dicionário, coloca as definições 
% do verbete em uma lista numerada em linha
\NewDocumentCommand{\dictenumerate}{>{\SplitList{|}}m}
 {%
  \begin{enumerate*}[nosep,label=(\arabic*),left=0pt,mode=unboxed,font=\bfseries]
   \ProcessList{#1}{\insertitem}
  \end{enumerate*}
 }
\NewDocumentCommand{\insertitem}{>{\TrimSpaces}m}{\item #1}

% Ambiente "enumerate" especial utilizado no dicionário, coloca os exemplos
% das definições do verbete em uma lista numerada em linha, utilizando
% algarismos romanos
\makeatletter
\NewDocumentCommand{\dictexamples}{m>{\SplitList{|}}m}
 {%
  \def\@theword{#1}%
  \begin{enumerate}[nosep,label=(\roman*),left=0pt,mode=unboxed,font=\bfseries]
   \ProcessList{#2}{\insertexample}
  \end{enumerate}
 }
\NewDocumentCommand{\insertexample}{>{\TrimSpaces}m}
 {
  \IfSubStr{#1}{mud::::}
   {% Sublinhado Manual
    \StrBehind{#1}{mud::::}[output]%
    \IfSubStr{\output}{___}
    {% Com traducao
     \StrCut{\output}{___}\csA\csB%
     \item\csA\\{\small``\csB''}
    }
    {% Sem traducao
     \item\output
    }
   }
   {% Sublinhado Automático
    \IfSubStr{#1}{___}
    {% Com traducao
      \StrCut{#1}{___}\csA\csB%
      \item\StrSubstitute{\csA}{\@theword}{\underline{\@theword}}\\{\small``\csB''}
    }
    {% Sem traducao
      \item\StrSubstitute{#1}{\@theword}{\underline{\@theword}}
    }
   }
 }  
\makeatother

\ExplSyntaxOn

%%% Cria listas especializadas (seelist e seealsolist)
\newlist{seelist}{enumerate}{1}
\newlist{seealsolist}{enumerate}{1}

\setlist[seelist]{label={(\alph*)},topsep=0pt,nosep,noitemsep}%,leftmargin=\parindent}
\setlist[seealsolist]{label={(\alph*)},topsep=0pt,nosep,noitemsep}%,leftmargin=\parindent}

%%% Cria e inicializa a lista "\seerefl", "Veja"
\newcommand\seerefl{}
\listadd{\seerefl}{}% Inicializa a lista

%%% Cria e inicializa a lista "\seealsorefl", "Veja também"
\newcommand\seealsorefl{}
\listadd{\seealsorefl}{}% Inicializa a lista

%%% Comando "\seeitem", adiciona um item "Veja" ou "Veja também" na lista,
%%% com os pinyins abaixo dos caracteres
\newcommand{\seeitem}[2]{#1~\dpy{#2}\ (p.~\pageref{#1:#2})}

%%% Comando "\definition", gera o texto da definição
\NewDocumentCommand{\definition}{sommo}
 {%
  \IfBooleanTF{#1}%
   {% Substantivo Próprio
    {\small\ding{108}}\ (\textit{S.P.})\IfValueT{#2}{~[clas.:~#2]}{\ \dictenumerate{#4}}\par
   }%
   {%
    {\small\ding{108}}\ (\textit{#3})\IfValueT{#2}{~[clas.: #2]}{\ \dictenumerate{#4}}\par
   }%
  \IfValueT{#5}%
   {%
    \IfSubStr{#5}{|}{\textbf{Exemplos:}}{\textbf{Exemplo:}}\dictexamples{\l_hanzi_tl}{#5}
   }%
 }

%%% Comando "Variante de"
\NewDocumentCommand{\variantof}{m}
 {
  {\small\ding{108}}\ Variante\ de\ #1\ (p.~\pageref{#1:\l_pinyin_tl})\par
 }

%%% Comando "Veja"
\NewDocumentCommand{\seeref}{mm}
 {%
  \booltrue{f_see}
  \listgadd{\seerefl}{#1:#2}
 }

%%% Comando "Veja também"
\NewDocumentCommand{\seealsoref}{mm}
 {%
  \booltrue{f_seealso}
  \listgadd{\seealsorefl}{#1:#2}
 }

\ExplSyntaxOff

%%%%% EOF %%%%%


%%%%%%%%%%%%%%%%%%%%%%%%%%%%%%%%%%%%%%%%%%%%%%%%%%%%%%%%%%%%%%%%%%%%%%%%%%%%%%%
%%%%%%%%%%%%%%%%%%%%%%%%%%%%%%%%%%%%%%%%%%%%%%%%%%%%%%%%%%%%%%%%%%%%%%%%%%%%%%%
%%%%%                                                                     %%%%%
%%%%% pinyincmd.tex:                                                      %%%%%
%%%%% Ambientes para o Dicionário ordenado por pinyins.                   %%%%%
%%%%%                                                                     %%%%%
%%%%%%%%%%%%%%%%%%%%%%%%%%%%%%%%%%%%%%%%%%%%%%%%%%%%%%%%%%%%%%%%%%%%%%%%%%%%%%%
%%%%%%%%%%%%%%%%%%%%%%%%%%%%%%%%%%%%%%%%%%%%%%%%%%%%%%%%%%%%%%%%%%%%%%%%%%%%%%%

\ExplSyntaxOn

%%% Ambiente "entry", para os verbetes
\NewDocumentEnvironment{entry}{mO{}mO{}mooo}%
 {%
  \leavevmode
  \markboth{#1{\tiny\dpy{#3}}}{#1{\tiny\dpy{#3}}}
  \tl_set:Nn \l_hanzi_tl {#1}
  \tl_set:Nn \l_pinyin_tl {#3}
  \tl_set:Nn \l_strokes_tl {#5}
  \boolfalse{f_see}\renewcommand\seerefl{}\listadd{\seerefl}{}% Initialize list
  \boolfalse{f_seealso}\renewcommand\seealsorefl{}\listadd{\seealsorefl}{}% Initialize list
  \begin{minipage}[t][][t]{.49\textwidth}
   \label{#1:#3}
   \begin{tcolorbox}[size=title,colframe=black,colback=white,boxrule=1pt,toprule=2pt,left=0mm,right=0mm,top=0mm,bottom=0mm]
    {\Large#1}\hfill\textsuperscript{\tiny(#5画)}\\
    {\footnotesize#2\ \dpy{#3}\ #4}%
    \IfValueT{#6}{\mbox{}\hfill{\tiny#6}}{}%
    \IfValueT{#7}{\mbox{}\hfill{\tiny#7}}{}%
    \IfValueT{#8}{\mbox{}\hfill{\tiny#8}}{}
   \end{tcolorbox}
 }{%
  \ifbool{f_see}%
   {% Process "see" references
    \RenewDocumentCommand\do{>{\SplitArgument{1}{:}}m}{\item \seeitem ##1}
    \textbf{Veja:\ }%
    \begin{itemize*}[label={}, before={\hspace{2sp}}, itemjoin={{,\ }}, itemjoin*={{\ e~}}]
     \dolistloop{\seerefl}
    \end{itemize*}\par
   }{}%
  \ifbool{f_seealso}%
   {% Process "see also" references
    \RenewDocumentCommand\do{>{\SplitArgument{1}{:}}m}{\item \seeitem ##1}
    \textbf{Veja\ também:\ }%
    \begin{itemize*}[label={}, before={\hspace{2sp}}, itemjoin={{,\ }}, itemjoin*={{\ e~}}]
     \dolistloop{\seealsorefl}
    \end{itemize*}\par
   }{}%
  \end{minipage}
  \vspace{2mm}
}

%%% Ambiente "entry*", para os verbetes muito compridos
\NewDocumentEnvironment{entry*}{mO{}mO{}mooo}
 {%
  \leavevmode
  \markboth{#1{\tiny\dpy{#3}}}{#1{\tiny\dpy{#3}}}
  \tl_set:Nn \l_hanzi_tl {#1}
  \tl_set:Nn \l_pinyin_tl {#3}
  \tl_set:Nn \l_strokes_tl {#5}
  \boolfalse{f_see} \renewcommand\seerefl{} \listadd{\seerefl}{}% Initialize list
  \boolfalse{f_seealso} \renewcommand\seealsorefl{} \listadd{\seealsorefl}{}% Initialize list
  \vspace{\baselineskip}
  \begin{minipage}[t][][t]{.49\textwidth}
   \label{#1:#3}
   \begin{tcolorbox}[size=title,colframe=black,colback=white,boxrule=1pt,toprule=2pt,left=0mm,right=0mm,top=0mm,bottom=0mm]
    \mbox{}\hfill\textsuperscript{\tiny(#5画)}\\
    {\LARGE#1}\\
    {\footnotesize#2\ \dpy{#3}\ #4}\\
    \IfValueT{#6}{\mbox{}\hfill{\tiny#6}}{}%
    \IfValueT{#7}{\mbox{}\hfill{\tiny#7}}{}%
    \IfValueT{#8}{\mbox{}\hfill{\tiny#8}}{}
   \end{tcolorbox}
 }%
 {%
  \ifbool{f_see}%
   {% Há 'veja'
    \RenewDocumentCommand\do{>{\SplitArgument{1}{:}}m}{\item \seeitem ##1}
    \textbf{Veja:\ }%
    \begin{itemize*}[label={}, itemjoin={{,\ }}, itemjoin*={{\ e~}}]
     \dolistloop{\seerefl}
    \end{itemize*}\par
   }{}%
  \ifbool{f_seealso}%
   {% Há 'veja tambem'
    \RenewDocumentCommand\do{>{\SplitArgument{1}{:}}m}{\item \seeitem ##1}
    \textbf{Veja\ também:\ }%
    \begin{itemize*}[label={}, itemjoin={{,\ }}, itemjoin*={{\ e~}}]
     \dolistloop{\seealsorefl}
    \end{itemize*}\par
   }{}%
  \end{minipage}
  \vspace{2mm}
}

\ExplSyntaxOff

%%%%% EOF %%%%%


% Ajustes do PDF
\hypersetup{
  linktoc=page,
  colorlinks=true,
  urlcolor=blue,
  linkcolor=blue,
  citecolor=blue,
  pdftitle={汉葡词典 - Dicionário Chinês-Português},
  pdfsubject={Dicionário Chinês-Português -- Ordenado por Pinyin},
  pdfauthor={罗学凯, Luiz Eduardo Roncato Cordeiro},
  pdfkeywords={dicionário, chinês, português, instituto confúcio}
}

%%%
%%% Documento começa aqui!
%%%

\begin{document}
\addfontfeatures{CharacterWidth=Proportional}

\input{title.tex}

\clearpage
\pagestyle{empty}
\tableofcontents

\clearpage
\pagestyle{empty}
\chapter{汉葡词典}

%%%%%%%%%%%%%%%%%%%%%%%%
%
% https://en.wikipedia.org/wiki/Chinese_character_orders
%
%%%%%%%%%%%%%%%%%%%%%%%%

Dicionário Chinês-Português ordenado primeiro pelo pinyin de cada
caracter, depois pelo número de traços e, finalmente, pela ordem do
caracter na tabela UTF-8.

\clearpage
\pagestyle{dictionary}
\twocolumn
%%%
%%% A
%%%

\section*{A}\addcontentsline{toc}{section}{A}

\begin{entry}{阿}{a1}{7}{⾩}
  \definition{pref.}{em dialetos do sul para formar termos carinhosos, antes de nomes de animais de estimação, sobrenomes monossilábicos ou números que denotam ordem de antiguidade em uma; anexado a 大, 二, 三,\dots\ para indicar classificação (e, às vezes, intimidade) | antes dos termos de parentesco; na frente de um sobrenome, de um nome próprio ou de um determinado título, com uma conotação de intimidade | em alguns contextos, pode soar infantil ou muito informal (por exemplo, chamar um colega de trabalho por ``阿 + Nome'' sem intimidade)}[阿妈___mamãe | 阿明 ___forma carinhosa de chamar alguém chamado Ming]
  \seeref{阿}{e1}
\end{entry}

\begin{entry}{阿哥}{a1ge1}{7,10}{⾩、⼝}
  \definition{s.}{irmão mais velho (afetivo)}[阿哥,帮我拿一下书包___Irmão, ajude-me com minha mochila escolar!]
\end{entry}

\begin{entry}{阿姨}{a1yi2}{7,9}{⾩、⼥}[HSK 4]
  \definition[个,位]{s.}{tia; uma forma de tratamento para uma mulher da geração dos pais; dirigir-se a uma mulher que tem aproximadamente a mesma idade da sua mãe, geralmente não é parente | babá em uma família; professora em um jardim de infância | tia; irmã da mãe (mais comum no sul da China)}[阿姨,生日快乐!___Tia, feliz aniversário! | 阿姨,这个苹果多少钱一斤?___Tia/Senhora, quanto custa o quilo dessas maçãs? | 阿姨,我想喝水。___Tia/Babá, eu quero beber água.]
\end{entry}

\begin{entry}{呵}{a1}{8}{⼝}
  \variantof{啊}
  \seeref{呵}{he1}
\end{entry}

\begin{entry}{啊}{a1}{10}{⼝}[HSK 2]
  \definition{interj.}{Ah!; Oh!; expressar surpresa ou admiração}
  \seeref{啊}{a2}
  \seeref{啊}{a3}
  \seeref{啊}{a4}
  \seeref{啊}{a5}
\end{entry}

\begin{entry}{啊呀}{a1ya1}{10,7}{⼝、⼝}
  \definition{interj.}{Oh meu Deus! | interjeição de surpresa}
\end{entry}

\begin{entry}{啊哟}{a1yo5}{10,9}{⼝、⼝}
  \definition{interj.}{Meu Deus! | Oh! | Ai! | interjeição de surpresa ou dor}
\end{entry}

\begin{entry}{啊}{a2}{10}{⼝}[HSK 2]
  \definition{interj.}{Eh?; Ei?; Que?; Por que?; expressar questionamento, dúvida ou solicitar opinião}
  \seeref{啊}{a1}
  \seeref{啊}{a3}
  \seeref{啊}{a4}
  \seeref{啊}{a5}
\end{entry}

\begin{entry}{嗄}{a2}{13}{⼝}
  \definition{adj.}{rouco}
  \variantof{啊}
\end{entry}

\begin{entry}{啊}{a3}{10}{⼝}[HSK 2]
  \definition{interj.}{Eh?; Meu!; E aí?; Que?; expressar surpresa e dúvida}
  \seeref{啊}{a1}
  \seeref{啊}{a2}
  \seeref{啊}{a4}
  \seeref{啊}{a5}
\end{entry}

\begin{entry}{啊}{a4}{10}{⼝}[HSK 2]
  \definition{interj.}{Bem!; Sim!; expressa concordância, pronúncia mais curta | Oh!; Ah!; indica que compreendeu, com pronúncia mais longa | Oh!; expressa surpresa ou admiração, com pronúncia mais longa | Desgraça!; expressar tristeza, pesar ou pesar}
  \seeref{啊}{a1}
  \seeref{啊}{a2}
  \seeref{啊}{a3}
  \seeref{啊}{a5}
\end{entry}

\begin{entry}{啊}{a5}{10}{⼝}[HSK 2,4]
  \definition{part.}{usado no final da frase para expressar admiração | usado no final da frase para expressar afirmação, justificativa, insistência, recomendação, etc. | usado no final da frase para indicar dúvida | usado para fazer uma pequena pausa na frase, chamando a atenção para o que vem a seguir | usado após os itens enumerados | usado após verbos repetitivos, indica um processo longo}
  \seeref{啊}{a1}
  \seeref{啊}{a2}
  \seeref{啊}{a3}
  \seeref{啊}{a4}
\end{entry}

\begin{entry}{矮}{ai3}{13}{⽮}[HSK 4]
  \definition{adj.}{baixo em estatura, dimensão, grau ou ranque | curto (em comprimento)}[他比我矮。___Ele é mais baixo que eu. | 这栋楼很矮,只有三层。___Esse prédio é baixo, tem só três andares. | 她虽然矮,但是跑得很快!___Ela pode ser baixinha, mas corre muito rápido!]
\end{entry}

\begin{entry}{矮凳}{ai3deng4}{13,14}{⽮、⼏}
  \definition{s.}{banquinho baixo | banqueta}[这个矮凳是木制的,很结实。___Este banquinho é de madeira e bem resistente.]
\end{entry}

\begin{entry}{矮林}{ai3lin2}{13,8}{⽮、⽊}
  \definition{s.}{mata rasteira | bosque baixo}[这片矮林里有很多野兔和鸟类。___Neste bosque baixo há muitos coelhos selvagens e pássaros. | 山坡上长满了矮林,远看像绿色的地毯。___A encosta está coberta de mata rasteira, que de longe parece um tapete verde.]
\end{entry}

\begin{entry}{矮胖}{ai3pang4}{13,9}{⽮、⾁}
  \definition{adj.}{atarracado; gorducho; rechonchudo; roliço; baixo e robusto | chamar alguém diretamente de 矮胖 pode ser ofensivo}[我家猫矮胖矮胖的,像个毛球。___Meu gato é baixinho e gordinho, parece uma bolinha de pelo.]
\end{entry}

\begin{entry}{矮人}{ai3ren2}{13,2}{⽮、⼈}
  \definition{s.}{anão; pessoa de baixa estatura (indivíduo) | homúnculo; figuras criadas artificialmente pelos alquimistas em frascos de destilação | nanismo}[他虽然是矮人,但很有力气。___Embora ele seja baixo, é muito forte. | 北欧神话中的矮人是技艺高超的工匠。___Na mitologia nórdica, os anões são artesãos habilidosos. | 他因为身高被嘲笑为‘矮人’,这让他很伤心。___Ele foi zombado por ser chamado de ‘anão’ devido à sua altura, o que o magoou.]
\end{entry}

\begin{entry}{矮树}{ai3shu4}{13,9}{⽮、⽊}
  \definition{s.}{arbusto | árvore pequena, baixa}[矮树比高树更容易修剪。___Árvores baixas são mais fáceis de podar do que árvores altas. | 我们种了些矮树作为花园的边界。___Plantamos alguns arbustos como cerca natural do jardim.]
\end{entry}

\begin{entry}{矮小}{ai3 xiao3}{13,3}{⽮、⼩}[HSK 4]
  \definition{adj.}{subdimensionado; curto e pequeno; baixo e pequeno | quando usado para pessoas, pode soar depreciativo se não for em contexto neutro ou afetuoso}[这位矮小的老人是村里的智者。___Este idoso baixinho é o sábio da vila. | 这种矮小的灌木适合盆栽。___Este tipo de arbusto pequeno é ideal para vasos. | 山脚下有一片矮小的房屋,显得格外宁静。___Ao pé da montanha, havia casas baixas que transmitiam uma tranquilidade única.]
\end{entry}

\begin{entry}{矮星}{ai3xing1}{13,9}{⽮、⽇}
  \definition{s.}{estrela anã}[白矮星是恒星演化的最终阶段之一。___Anãs brancas são um dos estágios finais da evolução estelar.]
\end{entry}

\begin{entry}{矮子}{ai3zi5}{13,3}{⽮、⼦}
  \definition{s.}{pessoa baixa; anão; baixinho}[白雪公主和七个小矮子住在森林里。___Branca de Neve e os sete anões vivem na floresta. | 用`矮子'称呼他人是不礼貌的。___Chamar alguém de ``baixinho'' é falta de educação.]
\end{entry}

\begin{entry}{爱}{ai4}{10}{⽖}[HSK 1]
  \definition*{s.}{sobrenome Ai}
  \definition[个]{s.}{amor; afeição; afeição profunda; preocupação profunda; especialmente amor entre pessoas}[爱是理解和包容。___O amor é compreensão e tolerância.]
  \definition{v.}{amar; ter sentimentos profundos por pessoas ou coisas | gostar; gostar de; estar interessado em |  cuidar; valorizar; ter em alta estima; cuidar bem de | estar apto a; ter o hábito de}[他们深深爱着对方。___Eles se amam profundamente. | 我爱我的家人。___Eu amo minha família. | 我爱旅行。___Eu adoro viajar.]
\end{entry}

\begin{entry}{爱爱}{ai4'ai5}{10,10}{⽖、⽖}
  \definition{v.}{(coloquial) fazer amor ou relações íntimas | pode ser usado como um apelido entre casais, transmitindo ternura | pode soar vulgar se usado em contextos inadequados}[他们俩刚结婚,天天都想爱爱。___Eles acabaram de se casar e querem fazer amor todo dia. | 爱爱,你今天好漂亮!___Amor, você está linda hoje!]
\end{entry}

\begin{entry}{爱抚}{ai4fu3}{10,7}{⽖、⼿}
  \definition{v.}{acariciar; afagar; cuidar (com ternura)}[他轻轻爱抚她的头发。___Ele afagou suavemente o cabelo dela. | 母亲爱抚婴儿的脸颊。___A mãe acaricia a bochecha do bebê. | 她爱抚着小猫的耳朵。___Ela acariciou as orelhas do gatinho.]
\end{entry}

\begin{entry}{爱国}{ai4 guo2}{10,8}{⽖、⼞}[HSK 4]
  \definition{adj.}{patriótico; patriotismo}[爱国是每个公民的责任。___O patriotismo é o dever de todo cidadão. | 这部电影讲述了英雄的爱国故事。___Este filme conta a história patriótica de um herói.]
  \definition{v.}{ser patriota; amar o seu país}
\end{entry}

\begin{entry}{爱好}{ai4 hao4}{10,6}{⽖、⼥}[HSK 1]
  \definition[个,种]{s.}{passatempo; interesse; \emph{hobby}; sentimentos de interesse especial ou afeição por algo | 爱好 é mais usado para atividades regulares (esportes, música), enquanto 喜欢 é para preferências gerais}[他的爱好是收集邮票。___Seu hobby era colecionar selos.  | 我的爱好是读书和旅行。___Meus hobbies são ler e viajar.]
  \definition{v.}{estar interessado em; ter prazer em; ter um forte interesse em algo; ter sentimentos profundos por alguém ou algo}
  \seealsoref{喜欢}{xi3huan5}
\end{entry}

\begin{entry}{爱好者}{ai4 hao4 zhe3}{10,6,8}{⽖、⼥、⽼}
  \definition{s.}{hobbista; amador; entusiasta; fã; amante (de arte, esportes, etc.)}[他是一位摄影爱好者。___Ele é um entusiasta de fotografia. | 她是位潜水爱好者,经常去东南亚潜水。___Ela é uma mergulhadora amadora e frequentemente mergulha no Sudeste Asiático.  | 我们为书法爱好者创建了一个微信群。___Criamos um grupo no WeChat para amantes de caligrafia.]
\end{entry}

\begin{entry}{爱护}{ai4hu4}{10,7}{⽖、⼿}[HSK 4]
  \definition{v.}{acalentar; valorizar; salvaguardar; cuidar bem de}[全社会都应爱护老年人。___Toda a sociedade deve tratar os idosos com cuidado e respeito. | 请爱护公园里的小动物。___Por favor, tratem os animais do parque com cuidado.]
\end{entry}

\begin{entry}{爱情}{ai4qing2}{10,11}{⽖、⼼}[HSK 2]
  \definition{s.}{amor (entre pessoas); afeição}[爱情是盲目的。___O amor é cego. | 爱情如同玫瑰,美丽却带刺。___O amor é como uma rosa, bela mas com espinhos.  | 这首歌讲述了破碎的爱情故事。___Esta música conta uma história de amor fracassado.]
\end{entry}

\begin{entry}{爱人}{ai4 ren5}{10,2}{⽖、⼈}[HSK 2]
  \definition[个]{s.}{amante; \emph{dollbaby}; namorado(a) | marido ou esposa; mais usado em ocasiões formais}[这是我的爱人。___Este é o meu/minha esposo/companheiro. | 她是我一生的爱人。___Ela é o amor da minha vida. | 请携带爱人出席晚宴。___Por favor, traga seu cônjuge para o jantar.]
\end{entry}

\begin{entry}{爱上}{ai4shang4}{10,3}{⽖、⼀}
  \definition{v.}{perder o coração por; apaixonar-se por}[他在旅行时爱上了一位法国女孩。___Ele se apaixonou por uma garota francesa durante a viagem.  | 来到杭州后,我爱上了龙井茶。___Depois de chegar em Hangzhou, me apaixonei pelo chá Longjing. | 我从来没想过自己会爱上健身。___Eu nunca imaginei que iria me apaixonar por academia.]
\end{entry}

\begin{entry}{爱心}{ai4xin1}{10,4}{⽖、⼼}[HSK 3]
  \definition[片]{s.}{amor; carinho; compaixão; um sentimento de preocupação e carinho por outras pessoas ou animais}
\end{entry}

\begin{entry}{碍事}{ai4shi4}{13,8}{⽯、⼅}
  \definition{s.}{(usualmente em frases negativas) sem consequência, não importa}
  \definition{v.+compl.}{estar no caminho | ser um obstáculo}
\end{entry}

\begin{entry}{厂}{an1}{2}{⼚}
  \definition{s.}{usado principalmente em nomes pessoais}[他名中有个厂字。___O nome dele contém a palavra ``fábrica''.]
  \seeref{厂}{chang3}
  \seeref{厂}{han3}
\end{entry}

\begin{entry}{广}{an1}{3}{⼴}
  \definition{s.}{mais comum em nomes de pessoas; o mesmo que 庵}[广安是我的朋友。___An'an é meu amigo.]
  \seeref{广}{guang3}
  \seeref{广}{yan3}
  \seealsoref{庵}{an1}
\end{entry}

\begin{entry}{安}{an1}{6}{⼧}[HSK 4]
  \definition{adj.}{pacífico; quieto; tranquilo; calmo; estáve; sem perturbação | seguro; protegido; com boa saúde; em paz; bem}
  \definition{pron.}{onde; como}
  \definition{s.}{segurança; proteção; paz; conforto | ampère; (eletricidade) abreviação de ampère}
  \definition{v.}{deixar (a mente de alguém) à vontade; acalmar; estabilizar | satisfazer; estar satisfeito; sentir-se satisfeito e à vontade | colocar em uma posição adequada; encontrar um lugar para | instalar; consertar; encaixar; configurar | trazer (uma acusação contra alguém); dar (a alguém um apelido) | abrigar (uma intenção); manter; segurar}
\end{entry}

\begin{entry}{安家}{an1jia1}{6,10}{⼧、⼧}
  \definition{v.+compl.}{montar uma casa | estabelecer-se}
\end{entry}

\begin{entry}{安静}{an1jing4}{6,14}{⼧、⾭}[HSK 2]
  \definition{adj.}{silencioso; tranquilo; sem som; sem barulho e sem algazarra}
\end{entry}

\begin{entry}{安排}{an1pai2}{6,11}{⼧、⼿}[HSK 3]
  \definition{s.}{plano; programação; organização; tabela do plano de atividades ou horários}
  \definition{v.}{organizar (assuntos) de acordo com a sequência ou regras; tratar as coisas de acordo com uma determinada ordem ou regras | atribuir tarefas a alguém; colocar as pessoas nos cargos de trabalho determinados, conforme planejado}
\end{entry}

\begin{entry}{安全}{an1quan2}{6,6}{⼧、⼊}[HSK 2]
  \definition{adj.}{seguro; protegido; sem perigo; sem ameaças; sem acidentes}
  \definition{s.}{segurança; proteção; refere-se a um estado ou conceito, geralmente indicando ausência de ameaças ou perigo}
\end{entry}

\begin{entry}{安神}{an1shen2}{6,9}{⼧、⽰}
  \definition{v.+compl.}{acalmar os nervos | aliviar a inquietação pela tranquilização da mente e do corpo}
\end{entry}

\begin{entry}{安慰}{an1wei4}{6,15}{⼧、⼼}[HSK 5]
  \definition{adj.}{confortar; tranquilizar; consolar; apaziguar;}
  \definition[个]{s.}{conforto; consolo; comportamento que alivia a dor de alguém e o acalma com palavras ou gestos}
  \definition{v.}{confortar; consolar; acalmar e confortar; deixar a mente tranquila}
\end{entry}

\begin{entry}{安置}{an1zhi4}{6,13}{⼧、⽹}[HSK 4]
  \definition{v.}{providenciar; encontrar um lugar para; ajudar a estabelecer-se; colocar pessoas ou coisas em uma determinada posição ou organizá-las adequadamente}
\end{entry}

\begin{entry}{安装}{an1zhuang1}{6,12}{⼧、⾐}[HSK 3]
  \definition{v.}{instalar; consertar; configurar; fixar máquinas ou equipamentos (geralmente conjuntos) em um determinado local, de acordo com métodos e especificações específicos}
\end{entry}

\begin{entry}{庵}{an1}{11}{⼴}
  \definition*{s.}{sobrenome An}
  \definition[个,座]{s.}{cabana | convento de freiras; templos budistas, principalmente onde vivem as freiras}
\end{entry}

\begin{entry}{岸}{an4}{8}{⼭}[HSK 5]
  \definition{adj.}{elevado; grandioso (de maneira sombria ou condescendente)}
  \definition[个]{s.}{margem; costa; litoral; terreno à beira da água}
\end{entry}

\begin{entry}{岸上}{an4 shang4}{8,3}{⼭、⼀}[HSK 5]
  \definition{s.}{em terra; costa; margem | na margem do rio; na beira do rio}
\end{entry}

\begin{entry}{按}{an4}{9}{⼿}[HSK 3]
  \definition{prep.}{de acordo com; à luz de; com base em; em conformidade com}
  \definition{v.}{pressionar; empurrar para baixo; pressionar ou apertar com a mão ou os dedos | arquivar; deixar de lado | restringir; controlar; inibir; parar | verificar; consultar | comentar ou anotar (por um editor ou autor)}
\end{entry}

\begin{entry}{按摩}{an4mo2}{9,15}{⼿、⼿}[HSK 5]
  \definition{s.}{massagem; empurrar, pressionar, beliscar e amassar o corpo de uma pessoa com as mãos para promover a circulação sanguínea, aumentar a resistência da pele e regular a função dos nervos}
\end{entry}

\begin{entry}{按时}{an4shi2}{9,7}{⼿、⽇}[HSK 4]
  \definition{adv.}{na hora; no horário; pontualmente; de acordo com o tempo estipulado}
\end{entry}

\begin{entry}{按照}{an4zhao4}{9,13}{⼿、⽕}[HSK 3]
  \definition{prep.}{de acordo com; em conformidade com; à luz de; com base em; apresentar os fundamentos ou critérios de julgamento para fazer algo}
\end{entry}

\begin{entry}{暗}{an4}{13}{⽇}[HSK 4]
  \definition{adj.}{escuro; opaco; sem graça; pouca luz | escondido; secreto; não revelado | pouco claro; nebuloso; vago; confuso | subterrâneo}
  \definition{adv.}{secretamente | no escuro}
\end{entry}

\begin{entry}{暗恋}{an4lian4}{13,10}{⽇、⼼}
  \definition{s.}{amor secreto}
  \definition{v.}{estar secretamente apaixonado por}
\end{entry}

\begin{entry}{暗示}{an4shi4}{13,5}{⽇、⽰}[HSK 4]
  \definition[个]{s.}{sugestão; insinuação; intimação; (psicologia) refere-se ao uso de palavras, gestos, expressões, etc. para fazer as pessoas aceitarem involuntariamente uma determinada opinião ou fazerem algo}
  \definition{v.}{dar uma dica; sugerir secretamente; indicar algo a alguém usando outras palavras, expressões faciais ou gestos sem dizer em voz alta}
\end{entry}

\begin{entry}{暗香}{an4xiang1}{13,9}{⽇、⾹}
  \definition{s.}{fragrância sutil}
\end{entry}

\begin{entry}{奥}{ao4}{12}{⼤}
  \definition{adj.}{obscuro | misterioso}
\end{entry}

\begin{entry}{奥林匹克运动会}{ao4lin2pi3ke4 yun4dong4hui4}{12,8,4,7,7,6,6}{⼤、⽊、⼖、⼗、⾡、⼒、⼈}
  \definition*{s.}{Jogos Olímpicos, Olimpíadas}
\end{entry}

\begin{entry}{奥特曼}{ao4te4man4}{12,10,11}{⼤、⽜、⽈}
  \definition*{s.}{\emph{Ultraman},  super-herói de ficção científica japonesa}
\end{entry}

\begin{entry}{奥运}{ao4yun4}{12,7}{⼤、⾡}
  \definition*{s.}{Jogos Olímpicos, Olimpíadas, abreviação de 奥林匹克运动会}
  \seealsoref{奥林匹克运动会}{ao4lin2pi3ke4 yun4dong4hui4}
\end{entry}

\begin{entry}{奥运会}{ao4yun4hui4}{12,7,6}{⼤、⾡、⼈}
  \definition*{s.}{Jogos Olímpicos, Olimpíadas, abreviação de 奥林匹克运动会}
  \seealsoref{奥林匹克运动会}{ao4lin2pi3ke4 yun4dong4hui4}
\end{entry}

\begin{entry}{澳}{ao4}{15}{⽔}
  \definition*{s.}{Austrália, abreviação de 澳大利亚}
  \seealsoref{澳大利亚}{ao4da4li4ya4}
\end{entry}

\begin{entry}{澳大利亚}{ao4da4li4ya4}{15,3,7,6}{⽔、⼤、⼑、⼆}
  \definition*{s.}{Austrália}
\end{entry}

%%%%% EOF %%%%%


%%%
%%% B
%%%

\section*{B}\addcontentsline{toc}{section}{B}

\begin{entry}{八}{ba1}{2}[HSK 1][Kangxi 12][Radical ⼋]
  \definition{num.}{oito; 8}
\end{entry}

\begin{entry}{八八六}{ba1 ba1 liu4}{2,2,4}[Radicais ⼋、⼋、⼋]
  \definition{expr.}{\emph{Bye bye!} (em salas de bate-papo e mensagens de texto)}
\end{entry}

\begin{entry}{巴勒斯坦}{ba1le4si1tan3}{4,11,12,8}[Radicais ⼰、⼒、⽄、⼟]
  \definition*{s.}{Palestina}
\end{entry}

\begin{entry}{巴士}{ba1 shi4}{4,3}[HSK 4][Radicais ⼰、⼠]
  \definition[辆]{s.}{ônibus; transliteração da palavra inglesa ``bus''}
\end{entry}

\begin{entry}{巴西}{ba1xi1}{4,6}[Radicais ⼰、⾑]
  \definition*{s.}{Brasil}
\end{entry}

\begin{entry}{巴西人}{ba1xi1ren2}{4,6,2}[Radicais ⼰、⾑、⼈]
  \definition[个,位]{s.}{brasileiro | pessoa ou povo do Brasil}
  \example{他是巴西人。}[Ele é brasileiro.]
\end{entry}

\begin{entry}{巴西战舞}{ba1xi1zhan4wu3}{4,6,9,14}[Radicais ⼰、⾑、⼽、⾇]
  \definition{s.}{capoeira}
\end{entry}

\begin{entry}{吧}{ba1}{7}[Radical ⼝]
  \definition{s.}{som de estalo, som crepitante}
  \definition{v.}{puxar o cachimbo; fumar | abreviação de ``bar''}
  \seeref{吧}{ba5}
\end{entry}

\begin{entry}{拔尖}{ba2jian1}{8,6}[Radicais ⼿、⼩]
  \definition{adj.}{topo de linha | fora do comum | o melhor}
  \definition{v.+compl.}{empurrar-se para a frente | sentir que é superior aos outros}
\end{entry}

\begin{entry}{把}{ba3}{7}[HSK 3][Radical ⼿]
  \definition{clas.}{para objetos com alça | para objetos pequenos:~punhado}
  \definition{part.}{partícula tornando o substantivo seguinte um objeto direto}
  \definition{v.}{conter | alcançar | segurar}
  \seeref{把}{ba4}
\end{entry}

\begin{entry}{把柄}{ba3bing3}{7,9}[Radicais ⼿、⽊]
  \definition{s.}{(figurativo) informações que podem ser usadas contra alguém}
\end{entry}

\begin{entry}{把持}{ba3chi2}{7,9}[Radicais ⼿、⼿]
  \definition{v.}{controlar | dominar | monopolizar}
\end{entry}

\begin{entry}{把风}{ba3feng1}{7,4}[Radicais ⼿、⾵]
  \definition{v.}{estar atento | vigiar (durante uma atividade clandestina)}
\end{entry}

\begin{entry}{把关}{ba3guan1}{7,6}[Radicais ⼿、⼋]
  \definition{v.}{verificar estritamente | examinar cuidadosamente para ver se algo é feito de acordo com um padrão fixo | fazer a verificação final | guardar uma passagem, fronteira}
\end{entry}

\begin{entry}{把脉}{ba3mai4}{7,9}[Radicais ⼿、⾁]
  \definition{v.}{sentir ou tomar o pulso de alguém}
\end{entry}

\begin{entry}{把式}{ba3shi4}{7,6}[Radicais ⼿、⼷]
  \definition{s.}{pessoa qualificada em um comércio}
\end{entry}

\begin{entry}{把守}{ba3shou3}{7,6}[Radicais ⼿、⼧]
  \definition{v.}{vigiar | guardar}
\end{entry}

\begin{entry}{把玩}{ba3wan2}{7,8}[Radicais ⼿、⽟]
  \definition{v.}{brincar com | mexer com}
\end{entry}

\begin{entry}{把稳}{ba3wen3}{7,14}[Radicais ⼿、⽲]
  \definition{adj.}{confiável}
\end{entry}

\begin{entry}{把握}{ba3wo4}{7,12}[HSK 3][Radicais ⼿、⼿]
  \definition{s.}{seguro | garantia | certeza}
  \definition{v.}{agarrar | segurar | aproveitar}
\end{entry}

\begin{entry}{把戏}{ba3xi4}{7,6}[Radicais ⼿、⼽]
  \definition{s.}{acrobacia | malabarismo | truque barato}
\end{entry}

\begin{entry}{把}{ba4}{7}[Radical ⼿]
  \definition{v.}{lidar}
  \seeref{把}{ba3}
\end{entry}

\begin{entry}{爸}{ba4}{8}[HSK 1][Radical ⽗]
  \definition[个,位]{s.}{(informal) pai}
  \seeref{爸爸}{ba4ba5}
  \seealsoref{爸爸}{ba4ba5}
\end{entry}

\begin{entry}{爸爸}{ba4ba5}{8,8}[HSK 1][Radicais ⽗、⽗]
  \definition[个,位,名,群]{s.}{(informal) pai; papai; papa}
  \seeref{爸}{ba4}
\end{entry}

\begin{entry}{爸妈}{ba4ma1}{8,6}[Radicais ⽗、⼥]
  \definition{s.}{pai e mãe}
\end{entry}

\begin{entry}{罢}{ba4}{10}[Radical ⽹]
  \definition{v.}{parar | cessar | demitir | suspender | desistir | terminar}
  \seeref{罢}{ba5}
\end{entry}

\begin{entry}{霸权}{ba4quan2}{21,6}[Radicais ⾬、⽊]
  \definition{s.}{hegemonia | supremacia}
\end{entry}

\begin{entry}{吧}{ba5}{7}[HSK 1][Radical ⼝]
  \definition{part.}{indica discussão, sugestão, solicitação ou comando no final de uma frase | indica concordância ou aprovação no final de uma frase | indica uma pergunta ou especulação no final de uma frase | indica incerteza no final de uma frase | em uma frase, indica uma pausa, carrega um tom hipotético, frequentemente apresenta um contraste e implica um dilema}
  \seeref{吧}{ba1}
\end{entry}

\begin{entry}{罢}{ba5}{10}[Radical ⽹]
  \definition{part.}{partícula final, a mesma que 吧}
  \seeref{罢}{ba4}
  \seealsoref{吧}{ba5}
\end{entry}

\begin{entry}{白}{bai2}{5}[HSK 1,3][Kangxi 106][Radical ⽩]
  \definition*{s.}{sobrenome Bai}
  \definition{adj.}{branco | claro | puro; claro; simples; sem mistura; em branco | branco (como símbolo de reação) | escrito incorretamente ou pronunciado incorretamente | grátis; sem custos}
  \definition{adv.}{em vão; sem propósito; sem resultados}
  \definition{s.}{parte falada em ópera, etc.; frases de peças de teatro, etc. | dialeto local | funeral}
  \definition{v.}{explicar; apresentar; esclarecer; declarar | branquear | olhar para as pessoas com o branco dos olhos (olhar vazio, de desaprovação)}
\end{entry}

\begin{entry}{白菜}{bai2 cai4}{5,11}[HSK 3][Radicais ⽩、⾋]
  \definition[棵,个]{s.}{acelga | repolho chinês}
\end{entry}

\begin{entry}{白痴}{bai2chi1}{5,13}[Radicais ⽩、⽧]
  \definition{adj./s.}{estúpido | imbecil}
\end{entry}

\begin{entry}{白蛋白}{bai2dan4bai2}{5,11,5}[Radicais ⽩、⾍、⽩]
  \definition{s.}{albumina}
\end{entry}

\begin{entry}{白鹄}{bai2hu2}{5,12}[Radicais ⽩、⿃]
  \definition{s.}{cisne branco}
\end{entry}

\begin{entry}{白拣}{bai2jian3}{5,8}[Radicais ⽩、⼿]
  \definition{s.}{uma escolha barata}
  \definition{v.}{escolher algo que não custa nada}
\end{entry}

\begin{entry}{白萝卜}{bai2luo2bo5}{5,11,2}[Radicais ⽩、⾋、⼘]
  \definition{s.}{rabanete branco}
\end{entry}

\begin{entry}{白色}{bai2 se4}{5,6}[HSK 2][Radicais ⽩、⾊]
  \definition{s.}{cor branca}
\end{entry}

\begin{entry}{白天}{bai2 tian1}{5,4}[HSK 1][Radicais ⽩、⼤]
  \definition{adv.}{dia | de dia}
  \definition[个]{s.}{dia}
\end{entry}

\begin{entry}{白苋}{bai2xian4}{5,7}[Radicais ⽩、⾋]
  \definition{s.}{amaranto branco | brotos e folhas tenras de espinafre chinês usados como alimento}
\end{entry}

\begin{entry}{百}{bai3}{6}[HSK 1][Radical ⽩]
  \definition*{s.}{sobrenome Bai}
  \definition{num.}{cem; 100 | centena | cento}
\end{entry}

\begin{entry}{百般}{bai3ban1}{6,10}[Radicais ⽩、⾈]
  \definition{adv.}{de todas as maneiras possíveis | por todos os meios}
\end{entry}

\begin{entry}{百分}{bai3fen1}{6,4}[Radicais ⽩、⼑]
  \definition{num.}{por cento}
  \definition{s.}{porcentagem}
\end{entry}

\begin{entry}{百货}{bai3 huo4}{6,8}[HSK 4][Radicais ⽩、⾙]
  \definition{s.}{mercadorias em geral; loja de departamentos; um termo geral para bens que incluem principalmente roupas, utensílios e necessidades diárias gerais}
\end{entry}

\begin{entry}{柏树}{bai3shu4}{9,9}[Radicais ⽊、⽊]
  \definition{s.}{cipreste}
\end{entry}

\begin{entry}{摆}{bai3}{13}[HSK 4][Radical ⼿]
  \definition*{s.}{sobrenome Bai | Festival de Ganbai; uma reunião realizada nas áreas Dai durante festivais religiosos, para celebrar uma boa colheita ou para trocar materiais; geralmente se refere a uma reunião em massa}
  \definition{s.}{pêndulo; um dispositivo mecânico que controla a frequência de vibração em relógios e instrumentos | a bainha inferior de um vestido, jaqueta ou saia}
  \definition{v.}{colocar; organizar | vestir; assumir | balançar; acenar; agitar para frente e para trás | expor; declarar claramente; listar | dizer; falar | libertar-se}
\end{entry}

\begin{entry}{摆动}{bai3 dong4}{13,6}[HSK 4][Radicais ⼿、⼒]
  \definition{v.}{balançar; balançar para frente e para trás; oscilar; vibrar}
\end{entry}

\begin{entry}{摆烂}{bai3lan4}{13,9}[Radicais ⼿、⽕]
  \definition{v.}{(neologismo, gíria) parar de lutar (especialmente quando se sabe que não pode ter sucesso) | deixar tudo ir para o inferno}
\end{entry}

\begin{entry}{摆手}{bai3shou3}{13,4}[Radicais ⼿、⼿]
  \definition{v.+compl.}{gesticular com a mão (acenando, acenando adeus, etc.) | balançar os braços | acenar com as mãos}
\end{entry}

\begin{entry}{摆脱}{bai3tuo1}{13,11}[HSK 4][Radicais ⼿、⾁]
  \definition{v.}{sacudir; rejeitar; romper com; libertar-se (ou desembaraçar-se) de; livrar-se de dificuldades, escravidão, controle, etc.}
\end{entry}

\begin{entry}{败}{bai4}{8}[HSK 4][Radical ⾒]
  \definition{adj.}{dilapidado; decadente; murcho; em declínio}
  \definition{v.}{derrota; bater | falhar | quebrar; neutralizar; dissipar | arruinar; estragar; corromper | ser derrotado; perder}
\end{entry}

\begin{entry}{班}{ban1}{10}[HSK 1][Radical ⽟]
  \definition*{s.}{sobrenome Ban}
  \definition{clas.}{para grupos}
  \definition[个]{s.}{equipe| time | esquadrão | turno de trabalho | classificação}
\end{entry}

\begin{entry}{班级}{ban1 ji2}{10,6}[HSK 3][Radicais ⽟、⽷]
  \definition[个]{s.}{classe | série (na escola)}
\end{entry}

\begin{entry}{班长}{ban1 zhang3}{10,4}[HSK 2][Radicais ⽟、⾧]
  \definition[个]{s.}{monitor de classe | líder de equipe | líder de esquadrão}
\end{entry}

\begin{entry}{般}{ban1}{10}[Radical ⾈]
  \definition{s.}{espécie | tipo | classe | caminho | maneira}
  \seeref{般}{bo1}
  \seeref{般}{pan2}
\end{entry}

\begin{entry}{搬}{ban1}{13}[HSK 3][Radical ⼿]
  \definition{v.}{copiar indiscriminadamente | mover-se (ou seja, mudar-se) | mover-se (algo relativamente pesado ou volumoso) | mudar | mudar-se}
\end{entry}

\begin{entry}{搬动}{ban1dong4}{13,6}[Radicais ⼿、⼒]
  \definition{v.}{mover-se (alguma coisa) | se mudar}
\end{entry}

\begin{entry}{搬家}{ban1jia1}{13,10}[HSK 3][Radicais ⼿、⼧]
  \definition{s.}{mudança}
  \definition{v.+compl.}{mudar-se de casa}
\end{entry}

\begin{entry}{搬口}{ban1kou3}{13,3}[Radicais ⼿、⼝]
  \definition{v.}{tagarelar | (idioma) transmitir histórias | semear dissensão | contar histórias}
\end{entry}

\begin{entry}{搬弄}{ban1nong4}{13,7}[Radicais ⼿、⼶]
  \definition{v.}{causar problemas | mexer com alguém | mostrar (o que se pode fazer)}
\end{entry}

\begin{entry}{搬运}{ban1yun4}{13,7}[Radicais ⼿、⾡]
  \definition{s.}{frete | transporte}
  \definition{v.}{carregar | transportar}
\end{entry}

\begin{entry}{搬走}{ban1zou3}{13,7}[Radicais ⼿、⾛]
  \definition{v.}{carregar}
\end{entry}

\begin{entry}{板}{ban3}{8}[HSK 3][Radical ⽊]
  \definition{adj.}{rígido; não natural | duro}
  \definition{clas.}{para cartões, papéis}
  \definition{s.}{tábua; placa; prato | veneziana; persiana; refere-se especificamente aos painéis de portas de lojas | badalos (instrumento musical que marca o ritmo) | uma batida acentuada (ritmo) na música e na ópera tradicional | chefe}
  \definition{v.}{parecer sério | livrar-se de maus hábitos ou falhas}
\end{entry}

\begin{entry}{办}{ban4}{4}[HSK 2][Radical ⼒]
  \definition{v.}{lidar com | lidar | gerenciar | configurar}
\end{entry}

\begin{entry}{办法}{ban4fa3}{4,8}[HSK 2][Radicais ⼒、⽔]
  \definition[条,个]{s.}{meio (de se fazer alguma coisa) | método | medida}
\end{entry}

\begin{entry}{办公}{ban4gong1}{4,4}[Radicais ⼒、⼋]
  \definition{v.+compl.}{lidar com negócios oficiais | trabalhar (especialmente em um escritório)}
\end{entry}

\begin{entry}{办公室}{ban4gong1shi4}{4,4,9}[HSK 2][Radicais ⼒、⼋、⼧]
  \definition[间]{s.}{gabinete | escritório}
\end{entry}

\begin{entry}{办理}{ban4li3}{4,11}[HSK 3][Radicais ⼒、⽟]
  \definition{v.}{conduzir | manusear | transacionar}
\end{entry}

\begin{entry}{办事}{ban4 shi4}{4,8}[HSK 4][Radicais ⼒、⼅]
  \definition{v.}{trabalhar | lidar com assuntos; manipular transações}
\end{entry}

\begin{entry}{半}{ban4}{5}[HSK 1][Radical ⼗]
  \definition{adj.}{incompleto}
  \definition{num.}{(depois de um número) ``e meio''}
  \definition{pref.}{semi}
  \definition{s.}{metade}
\end{entry}

\begin{entry}{半年}{ban4 nian2}{5,6}[HSK 1][Radicais ⼗、⼲]
  \definition{s.}{meio ano}
\end{entry}

\begin{entry}{半球}{ban4qiu2}{5,11}[Radicais ⼗、⽟]
  \definition{s.}{hemisfério}
\end{entry}

\begin{entry}{半天}{ban4 tian1}{5,4}[HSK 1][Radicais ⼗、⼤]
  \definition{s.}{metade do dia | muito tempo | bastante tempo}
\end{entry}

\begin{entry}{半夜}{ban4 ye4}{5,8}[HSK 2][Radicais ⼗、⼣]
  \definition{adv.}{no meio da noite | metade de uma noite}
  \definition{s.}{meia-noite}
\end{entry}

\begin{entry}{半音}{ban4yin1}{5,9}[Radicais ⼗、⾳]
  \definition{s.}{semitom}
\end{entry}

\begin{entry}{伴侣}{ban4lv3}{7,8}[Radicais ⼈、⼈]
  \definition{s.}{companheiro | parceiro}
\end{entry}

\begin{entry}{帮}{bang1}{9}[HSK 1][Radical ⼱]
  \definition{clas.}{para alguém (como uma ajuda)}
  \definition{s.}{gangue | grupo | contratado (como trabalhador) | camada externa | festa | sociedade secreta}
  \definition{v.}{ajudar | apoiar}
\end{entry}

\begin{entry}{帮教}{bang1jiao4}{9,11}[Radicais ⼱、⽁]
  \definition{v.}{orientar}
\end{entry}

\begin{entry}{帮忙}{bang1 mang2}{9,6}[HSK 1][Radicais ⼱、⼼]
  \definition{v.+compl.}{ajudar | dar uma mão | estender a mão | fazer um favor}
\end{entry}

\begin{entry}{帮佣}{bang1yong1}{9,7}[Radicais ⼱、⼈]
  \definition{s.}{ajudante doméstico | servo}
\end{entry}

\begin{entry}{帮助}{bang1zhu4}{9,7}[HSK 2][Radicais ⼱、⼒]
  \definition[种]{s.}{ajuda | assistência}
  \definition{v.}{ajudar | dar assistência}
\end{entry}

\begin{entry}{棒棒糖}{bang4bang4tang2}{12,12,16}[Radicais ⽊、⽊、⽶]
  \definition[根]{s.}{pirulito}
\end{entry}

\begin{entry}{棒冰}{bang4bing1}{12,6}[Radicais ⽊、⼎]
  \definition{s.}{picolé}
\end{entry}

\begin{entry}{包}{bao1}{5}[HSK 1][Radical ⼓]
  \definition*{s.}{sobrenome Bao}
  \definition{clas.}{pacotes, sacos, sacolas, embrulhos}
  \definition[个,只]{s.}{bolsa | pacote | recipiente | embrulho}
  \definition{v.}{contratar | cobrir | segurar ou abraçar | incluir | assumir o comando | embrulhar}
\end{entry}

\begin{entry}{包办}{bao1ban4}{5,4}[Radicais ⼓、⼒]
  \definition{v.}{comandar todo o show | comprometer-se a fazer tudo sozinho}
\end{entry}

\begin{entry}{包干}{bao1gan1}{5,3}[Radicais ⼓、⼲]
  \definition{s.}{tarefa alocada}
  \definition{v.}{ter a responsabilidade total sobre um trabalho}
\end{entry}

\begin{entry}{包裹}{bao1guo3}{5,14}[HSK 4][Radicais ⼓、⾐]
  \definition[个]{s.}{pacote; embrulho}
  \definition{v.}{embrulhar; amarrar; enrolar coisas em pano ou outra coisa}
\end{entry}

\begin{entry}{包含}{bao1han2}{5,7}[HSK 4][Radicais ⼓、⼝]
  \definition{v.}{conter; implicar; incluir; conter dentro, resumir, enfatizar o que está contido dentro, focar em relações internas, muitas vezes coisas abstratas}
\end{entry}

\begin{entry}{包括}{bao1kuo4}{5,9}[HSK 4][Radicais ⼓、⼿]
  \definition{v.}{incluir; compreender; consistir em; conter, conter dentro, resumir junto, enfatizar a listagem de todas as partes, ou a citação de uma parte delas, que podem ser coisas abstratas ou concretas}
\end{entry}

\begin{entry}{包容}{bao1rong2}{5,10}[Radicais ⼓、⼧]
  \definition{adj.}{inclusivo}
  \definition{v.}{perdoar | mostrar tolerância | conter | segurar}
\end{entry}

\begin{entry}{包子}{bao1 zi5}{5,3}[HSK 1][Radicais ⼓、⼦]
  \definition[个]{s.}{pão recheado cozido no vapor}
\end{entry}

\begin{entry}{包租}{bao1zu1}{5,10}[Radicais ⼓、⽲]
  \definition{s.}{aluguel fixo para terras agrícolas}
  \definition{v.}{fretar | alugar | alugar um terreno ou uma casa para subarrendar}
\end{entry}

\begin{entry}{薄}{bao2}{16}[HSK 4][Radical ⾋]
  \definition{adj.}{fino; frágil; pouca espessura |  frio; indiferente; carente de calor; emocionalmente frio; não profundo | leve; fraco | pobre; infértil}
  \seeref{薄}{bo2}
\end{entry}

\begin{entry}{宝}{bao3}{8}[HSK 4][Radical ⼧]
  \definition{adj.}{antigo; precioso; estimado}
  \definition[个,件]{s.}{tesouro; objeto estimado; coisa preciosa | dispositivo de jogo; ferramenta de jogo | dinheiro; moeda; moeda antiga com furo quadrado no centro; moeda de prata}
  \definition{s.}{sobrenome Bao}
\end{entry}

\begin{entry}{宝宝}{bao3 bao5}{8,8}[HSK 4][Radicais ⼧、⼧]
  \definition[个]{s.}{querida; \emph{darling}; \emph{baby}; apelido para crianças}
\end{entry}

\begin{entry}{宝贝}{bao3bei4}{8,4}[HSK 4][Radicais ⼧、⾙]
  \definition{adj.}{excêntrico; estranho; imprestável; um termo depreciativo para uma pessoa incompetente ou ridícula}
  \definition[个,件]{s.}{tesouro; objeto estimado; coisa preciosa | querida; \emph{darling}; \emph{baby}; apelido para crianças}
\end{entry}

\begin{entry}{宝贵}{bao3gui4}{8,9}[HSK 4][Radicais ⼧、⾙]
  \definition{adj.}{precioso; extremamente valioso, muito raro, pode ser usado para descrever coisas específicas, também pode ser usado para descrever coisas abstratas | valioso; como um tesouro}
\end{entry}

\begin{entry}{宝石}{bao3 shi2}{8,5}[HSK 4][Radicais ⼧、⽯]
  \definition[颗,枚,块]{s.}{gema; jóia; pedra preciosa; mineral precioso que tem um brilho lindo e uma dureza de mais de sete graus, não é afetado pela atmosfera ou por produtos químicos e pode ser usado como decoração, suporte de instrumentos ou abrasivos}
\end{entry}

\begin{entry}{饱}{bao3}{8}[HSK 2][Radical ⾷]
  \definition{adj.}{ter comido até ficar satisfeito | estar cheio | cheio}
  \definition{adv.}{completamente | até estar cheio}
  \definition{v.}{satisfazer}
\end{entry}

\begin{entry}{保}{bao3}{9}[HSK 3][Radical ⼈]
  \definition*{s.}{sobrenome Bao}
  \definition{s.}{fiador
oficial responsável
sistema administrativo}
  \definition{v.}{defender | proteger |manter | preservar | conservar em boas condições | garantir | assegurar | ficar como fiador de alguém.}
\end{entry}

\begin{entry}{保安}{bao3 an1}{9,6}[HSK 3][Radicais ⼈、⼧]
  \definition{s.}{guarda de segurança}
  \definition{v.}{manter seguro | garantir a segurança}
\end{entry}

\begin{entry}{保持}{bao3chi2}{9,9}[HSK 3][Radicais ⼈、⼿]
  \definition{v.}{manter | segurar | reter | preservar}
\end{entry}

\begin{entry}{保存}{bao3cun2}{9,6}[HSK 3][Radicais ⼈、⼦]
  \definition{v.}{conservar | preservar | (computação) salvar (um arquivo, etc.)}
\end{entry}

\begin{entry}{保护}{bao3hu4}{9,7}[HSK 3][Radicais ⼈、⼿]
  \definition{s.}{proteção | salvaguarda}
  \definition{v.}{proteger | defender | salvaguardar}
\end{entry}

\begin{entry}{保护国}{bao3hu4guo2}{9,7,8}[Radicais ⼈、⼿、⼞]
  \definition{s.}{protetorado}
\end{entry}

\begin{entry}{保护剂}{bao3hu4ji4}{9,7,8}[Radicais ⼈、⼿、⼑]
  \definition{s.}{agente protetor}
\end{entry}

\begin{entry}{保护区}{bao3hu4qu1}{9,7,4}[Radicais ⼈、⼿、⼖]
  \definition{s.}{área protegida | área de conservação}
\end{entry}

\begin{entry}{保护色}{bao3hu4se4}{9,7,6}[Radicais ⼈、⼿、⾊]
  \definition{s.}{camuflagem}
\end{entry}

\begin{entry}{保护神}{bao3hu4shen2}{9,7,9}[Radicais ⼈、⼿、⽰]
  \definition{s.}{anjo da guarda | santo patrono}
\end{entry}

\begin{entry}{保护物}{bao3hu4 wu4}{9,7,8}[Radicais ⼈、⼿、⽜]
  \definition{s.}{protetor}
\end{entry}

\begin{entry}{保护性}{bao3hu4xing4}{9,7,8}[Radicais ⼈、⼿、⼼]
  \definition{s.}{proteção}
\end{entry}

\begin{entry}{保护者}{bao3hu4zhe3}{9,7,8}[Radicais ⼈、⼿、⽼]
  \definition{s.}{protetor | segurador}
\end{entry}

\begin{entry}{保护主义}{bao3hu4zhu3yi4}{9,7,5,3}[Radicais ⼈、⼿、⼂、⼂]
  \definition{s.}{protecionismo}
\end{entry}

\begin{entry}{保留}{bao3liu2}{9,10}[HSK 3][Radicais ⼈、⽥]
  \definition{v.}{reter | continuar a ter | segurar | reservar}
\end{entry}

\begin{entry}{保密}{bao3mi4}{9,11}[HSK 4][Radicais ⼈、⼧]
  \definition{v.}{manter segredo; manter algo em segredo; manter a confidencialidade}
\end{entry}

\begin{entry}{保守}{bao3shou3}{9,6}[HSK 4][Radicais ⼈、⼧]
  \definition{adj.}{retrógrado; conservador; pensamentos e conceitos que são retrógrados e não conseguem acompanhar o desenvolvimento da situação}
  \definition{v.}{manter; guardar; evitar perder}
\end{entry}

\begin{entry}{保险}{bao3xian3}{9,9}[HSK 3][Radicais ⼈、⾩]
  \definition[个]{adj./s.}{seguro}
  \definition{v.}{ter certeza | estar vinculado a}
\end{entry}

\begin{entry}{保证}{bao3zheng4}{9,7}[HSK 3][Radicais ⼈、⾔]
  \definition[个]{s.}{garantia}
  \definition{v.}{garantir}
\end{entry}

\begin{entry}{报}{bao4}{7}[HSK 3][Radical ⼿]
  \definition[份,张]{s.}{jornal | recompensa | relatório | vingança}
  \definition{v.}{anunciar | informar}
\end{entry}

\begin{entry}{报酬}{bao4chou5}{7,13}[Radicais ⼿、⾣]
  \definition{s.}{recompensa | remuneração}
\end{entry}

\begin{entry}{报到}{bao4dao4}{7,8}[HSK 3][Radicais ⼿、⼑]
  \definition{v.+compl.}{apresentar-se para o serviço | fazer check-in | registrar-se | assinar}
\end{entry}

\begin{entry}{报道}{bao4dao4}{7,12}[HSK 3][Radicais ⼿、⾡]
  \definition[个,篇,分]{s.}{história | reportagem}
  \definition{v.}{cobrir | relatar (notícias)}
\end{entry}

\begin{entry}{报告}{bao4gao4}{7,7}[HSK 3][Radicais ⼿、⼝]
  \definition[份,篇,分,个,通]{s.}{relatório | discurso | palestra | aconselhamento}
  \definition{v.}{relatar | dar a conhecer | informar}
\end{entry}

\begin{entry}{报名}{bao4ming2}{7,6}[HSK 2][Radicais ⼿、⼝]
  \definition{v.+compl.}{matricular-se | alistar-se | inscrever-se | inserir o nome de alguém}
\end{entry}

\begin{entry}{报纸}{bao4zhi3}{7,7}[HSK 2][Radicais ⼿、⽷]
  \definition[张]{s.}{jornal | diário}
\end{entry}

\begin{entry}{抱}{bao4}{8}[HSK 4][Radical ⼿]
  \definition*{s.}{sobrenome Bao}
  \definition{clas.}{braçada; medida dos dois braços}
  \definition{v.}{carregar no peito; segurar com ambos os braços; abraçar | ter o primeiro filho ou neto | adotar um bebê ou criança | ficar juntos, unidos | encaixar ou servir perfeitamente (roupas e sapatos do tamanho certo) | estimar; nutrir; abrigar; ter em mente | continuar; sobrecarregar com | chocar ovos}
\end{entry}

\begin{entry}{抱怨}{bao4yuan4}{8,9}[Radicais ⼿、⼼]
  \definition{v.}{reclamar | resmungar | abrir uma reclamação | sentir-se insatisfeito}
\end{entry}

\begin{entry}{豹子}{bao4zi5}{10,3}[Radicais ⾘、⼦]
  \definition[头]{s.}{leopardo}
\end{entry}

\begin{entry}{暴力}{bao4li4}{15,2}[Radicais ⽇、⼒]
  \definition{adj.}{violento}
  \definition{s.}{violência}
\end{entry}

\begin{entry}{暴乱}{bao4luan4}{15,7}[Radicais ⽇、⼄]
  \definition{s.}{rebelião | revolta | tumulto}
\end{entry}

\begin{entry}{暴行}{bao4xing2}{15,6}[Radicais ⽇、⾏]
  \definition{s.}{ato selvagem | atrocidade | indignação}
\end{entry}

\begin{entry}{暴雨}{bao4yu3}{15,8}[Radicais ⽇、⾬]
  \definition[场,阵]{s.}{tempestade | chuva torrencial}
\end{entry}

\begin{entry}{暴躁}{bao4zao4}{15,20}[Radicais ⽇、⾜]
  \definition{adj.}{irascível | irritável}
\end{entry}

\begin{entry}{爆米花}{bao4mi3hua1}{19,6,7}[Radicais ⽕、⽶、⾋]
  \definition{s.}{pipoca (de milho) | pipoca de arroz}
\end{entry}

\begin{entry}{爆炸}{bao4zha4}{19,9}[Radicais ⽕、⽕]
  \definition{s.}{explosão}
  \definition{v.}{explodir | detonar}
\end{entry}

\begin{entry}{杯}{bei1}{8}[HSK 1][Radical ⽊]
  \definition{clas.}{para certos recipientes de líquidos: copo, xícara, etc.}
  \definition{s.}{copo | caneca | xícara | taça | troféu}
\end{entry}

\begin{entry}{杯具}{bei1ju4}{8,8}[Radicais ⽊、⼋]
  \definition{s.}{parachoque | fiasco | (gíria) tragédia}
\end{entry}

\begin{entry}{杯子}{bei1 zi5}{8,3}[HSK 1][Radicais ⽊、⼦]
  \definition[个,只]{s.}{copo | caneca | xícara | taça}
\end{entry}

\begin{entry}{背}{bei1}{9}[HSK 2][Radical ⾁]
  \definition{v.}{estar sobrecarregado | carregar nas costas ou no ombro}
  \seeref{背}{bei4}
\end{entry}

\begin{entry}{北}{bei3}{5}[HSK 1][Radical ⼔]
  \definition{s.}{norte}
  \definition{v.}{(clássico) ser derrotado}
\end{entry}

\begin{entry}{北边}{bei3 bian1}{5,5}[HSK 1][Radicais ⼔、⾡]
  \definition{adv.}{lado norte | ao norte de}
\end{entry}

\begin{entry}{北部}{bei3 bu4}{5,10}[HSK 3][Radicais ⼔、⾢]
  \definition{s.}{parte norte}
\end{entry}

\begin{entry}{北大西洋公约组织}{bei3 da4xi1 yang2 gong1 yue1 zu3zhi1}{5,3,6,9,4,6,8,8}[Radicais ⼔、⼤、⾑、⽔、⼋、⽷、⽷、⽷]
  \definition*{s.}{Organização do Tratado do Atlântico Norte, OTAN}
\end{entry}

\begin{entry}{北方}{bei3fang1}{5,4}[HSK 2][Radicais ⼔、⽅]
  \definition{s.}{norte | a parte norte de um país}
\end{entry}

\begin{entry}{北极}{bei3ji2}{5,7}[Radicais ⼔、⽊]
  \definition*{s.}{Ártico | Pólo Norte}
  \definition{s.}{pólo norte magnético}
\end{entry}

\begin{entry}{北京}{bei3 jing1}{5,8}[HSK 1][Radicais ⼔、⼇]
  \definition*{s.}{Beijing (Pequim), Capital da República Popular da China | Beijing (Pequim), governo da RPC}
\end{entry}

\begin{entry}{北面}{bei3mian4}{5,9}[Radicais ⼔、⾯]
  \definition{s.}{lado norte}
\end{entry}

\begin{entry}{北约}{bei3yue1}{5,6}[Radicais ⼔、⽷]
  \definition*{s.}{OTAN (Organização do Tratado do Atlântico Norte), abreviação de 北大西洋公约组织}
  \seeref{北大西洋公约组织}{bei3 da4xi1 yang2 gong1 yue1 zu3zhi1}
\end{entry}

\begin{entry}{备份}{bei4fen4}{8,6}[Radicais ⼡、⼈]
  \definition{s.}{cópia de segurança | \emph{backup}}
\end{entry}

\begin{entry}{备胎}{bei4tai1}{8,9}[Radicais ⼡、⾁]
  \definition{s.}{pneu sobressalente | (gíria) substituto}
\end{entry}

\begin{entry}{背}{bei4}{9}[HSK 3][Radical ⾁]
  \definition{adv.}{a parte de trás de um corpo ou objeto}
  \definition{s.}{costas | (gíria) azarado}
  \definition{v.}{esconder algo de | decorar | recitar de memória | virar as costas}
  \seeref{背}{bei1}
\end{entry}

\begin{entry}{背后}{bei4 hou4}{9,6}[HSK 3][Radicais ⾁、⼝]
  \definition{s.}{parte de trás | traseira | nas costas de alguém}
\end{entry}

\begin{entry}{背景}{bei4jing3}{9,12}[HSK 4][Radicais ⾁、⽇]
  \definition[种]{s.}{pano de fundo; fundo; cenário de teatro, filme ou drama de TV | fundo; cenário que permeia a imagem principal na tela | condições sociais; ambientes históricos (significativamente influentes para algo ou alguém) | poder que dá suporte a alguém}
\end{entry}

\begin{entry}{倍}{bei4}{10}[HSK 4][Radical ⼈]
  \definition{adv.}{mais; especialmente}
  \definition{clas.}{vezes; para obter um número igual ao número original, você pode multiplicar o número por esse múltiplo}
  \definition{s.}{dobro; duas vezes mais}
\end{entry}

\begin{entry}{被}{bei4}{10}[HSK 3][Radical ⾐]
  \definition*{s.}{sobrenome Bei}
  \definition{part.}{usada antes de verbos para formar frases verbais passivas}
  \definition{prep.}{usado em uma frase para indicar que o sujeito é o receptor da ação}
  \definition{s.}{colcha}
  \definition{v.}{cobrir; espalhar
sofrer}
\end{entry}

\begin{entry}{被单}{bei4dan1}{10,8}[Radicais ⾐、⼗]
  \definition[床]{s.}{lençol}
\end{entry}

\begin{entry}{被动}{bei4dong4}{10,6}[Radicais ⾐、⼒]
  \definition{adj.}{passivo}
\end{entry}

\begin{entry}{被告}{bei4gao4}{10,7}[Radicais ⾐、⼝]
  \definition{s.}{réu}
\end{entry}

\begin{entry}{被迫}{bei4 po4}{10,8}[HSK 4][Radicais ⾐、⾡]
  \definition{v.}{ser forçado; ser coagido; ser compelido; ser constrangido; ser forçado a fazer algo por força externa}
\end{entry}

\begin{entry}{被套}{bei4tao4}{10,10}[Radicais ⾐、⼤]
  \definition{s.}{capa de \emph{edredon}}
  \definition{v.}{ter dinheiro preso (em ações, imóveis, etc.)}
\end{entry}

\begin{entry}{被窝}{bei4wo1}{10,12}[Radicais ⾐、⽳]
  \definition{s.}{colcha}
\end{entry}

\begin{entry}{被子}{bei4zi5}{10,3}[HSK 3][Radicais ⾐、⼦]
  \definition[床]{s.}{colcha}
\end{entry}

\begin{entry}{本}{ben3}{5}[HSK 1][Radical ⽊]
  \definition{adj.}{o atual | original | inerente}
  \definition{adv.}{originalmente}
  \definition{clas.}{para livros, dicionários, periódicos, arquivos, etc.}
  \definition{s.}{raiz | caule | origem | fonte}
\end{entry}

\begin{entry}{本金}{ben3 jin1}{5,8}[Radicais ⽊、⾦]
  \definition{s.}{capital; capital para a operação do comércio e da indústria; capital para a operação de negócios |
valor principal; dinheiro retirado ao depositar ou tomar emprestado (diferente de ``利息'')}
  \seealsoref{利息}{li4xi1}
\end{entry}

\begin{entry}{本科}{ben3ke1}{5,9}[HSK 4][Radicais ⽊、⽲]
  \definition{s.}{graduação; bacharelado; o curso básico de uma universidade ou faculdade}
\end{entry}

\begin{entry}{本来}{ben3lai2}{5,7}[HSK 3][Radicais ⽊、⽊]
  \definition{adv.}{originalmente | apropriadamente | legalmente}
\end{entry}

\begin{entry}{本领}{ben3 ling3}{5,11}[HSK 3][Radicais ⽊、⾴]
  \definition[项,个]{s.}{capacidade | faculdade | poder | habilidade | talento}
\end{entry}

\begin{entry}{本事}{ben3shi4}{5,8}[Radicais ⽊、⼅]
  \definition{s.}{habilidade | capacidade | \emph{status} | poder | posição | autoridade}
  \seeref{本事}{ben3shi5}
\end{entry}

\begin{entry}{本事}{ben3shi5}{5,8}[HSK 3][Radicais ⽊、⼅]
  \definition{s.}{habilidade | capacidade |\emph{status} | poder | posição | autoridade}
  \seeref{本事}{ben3shi4}
\end{entry}

\begin{entry}{本子}{ben3 zi5}{5,3}[HSK 1][Radicais ⽊、⼦]
  \definition[本]{s.}{caderno}
\end{entry}

\begin{entry}{笨}{ben4}{11}[HSK 4][Radical ⽵]
  \definition{adj.}{estúpido; sem graça; tolo; de pouca habilidade; sem inteligência | desajeitado; tosco; inflexível | incômodo; pesado; desajeitado; difícil de manejar; trabalhoso}
\end{entry}

\begin{entry}{笨蛋}{ben4dan4}{11,11}[Radicais ⽵、⾍]
  \definition{s.}{bobalhão | cabeça-oca | cabeça-dura}
  \definition{v.}{iludir | enganar}
\end{entry}

\begin{entry}{崩}{beng1}{11}[Radical ⼭]
  \definition{s.}{morte de rei ou imperador | desaparecimento}
  \definition{v.}{entrar em colapso | cair em ruínas}
\end{entry}

\begin{entry}{绷带}{beng1dai4}{11,9}[Radicais ⽷、⼱]
  \definition{s.}{curativo | (empréstimo linguístico) \emph{bandage}}
\end{entry}

\begin{entry}{甭}{beng2}{9}[Radical ⽤]
  \definition{v.}{contração de 不用 | não precisar}
  \seeref{不用}{bu2 yong4}
\end{entry}

\begin{entry}{蹦极}{beng4ji2}{18,7}[Radicais ⾜、⽊]
  \definition{s.}{\emph{bungee jumping}}
\end{entry}

\begin{entry}{鼻子}{bi2zi5}{14,3}[Radicais ⿐、⼦]
  \definition[个,只]{s.}{nariz}
\end{entry}

\begin{entry}{比}{bi3}{4}[HSK 1][Kangxi 81][Radical ⽐]
  \definition*{s.}{Bélgica, abreviação de 比利时}
  \definition{part.}{partícula usada para comparação (superioridade)}
  \definition{prep.}{que | do que | (seguido por um substantivo e adjetivo) mais \{adj.\} do que \{s.\}}
  \definition{s.}{razão (taxa)}
  \definition{v.}{comparar | contrastar | gesticular (com as mãos)}
  \seeref{比利时}{bi3li4shi2}
\end{entry}

\begin{entry}{比分}{bi3 fen1}{4,4}[HSK 4][Radicais ⽐、⼑]
  \definition{s.}{pontuação; comparação de pontuações entre as duas equipes em uma partida}
\end{entry}

\begin{entry}{比较}{bi3jiao4}{4,10}[HSK 3][Radicais ⽐、⾞]
  \definition{adv.}{comparativamente | relativamente}
  \definition{s.}{comparação}
  \definition{v.}{comparar}
\end{entry}

\begin{entry}{比利时}{bi3li4shi2}{4,7,7}[Radicais ⽐、⼑、⽇]
  \definition*{s.}{Bélgica}
\end{entry}

\begin{entry}{比例}{bi3li4}{4,8}[HSK 3][Radicais ⽐、⼈]
  \definition{s.}{escala | razão | proporção}
\end{entry}

\begin{entry}{比拼}{bi3pin1}{4,9}[Radicais ⽐、⼿]
  \definition{s.}{concurso}
  \definition{v.}{competir ferozmente}
\end{entry}

\begin{entry}{比如}{bi3ru2}{4,6}[HSK 2][Radicais ⽐、⼥]
  \definition{conj.}{por exemplo | como}
\end{entry}

\begin{entry}{比如说}{bi3 ru2 shuo1}{4,6,9}[HSK 2][Radicais ⽐、⼥、⾔]
  \definition{adv.}{por exemplo}
\end{entry}

\begin{entry}{比萨饼}{bi3sa4bing3}{4,11,9}[Radicais ⽐、⾋、⾷]
  \definition[张]{s.}{pizza}
\end{entry}

\begin{entry}{比赛}{bi3sai4}{4,14}[HSK 3][Radicais ⽐、⾙]
  \definition[场,次]{s.}{competição | concurso}
  \definition{v.}{competir}
\end{entry}

\begin{entry}{比亚迪}{bi3ya4di2}{4,6,8}[Radicais ⽐、⼆、⾡]
  \definition*{s.}{Montadora BYD}
\end{entry}

\begin{entry}{笔}{bi3}{10}[HSK 2][Radical ⽵]
  \definition{clas.}{para somas de dinheiro, negócios}
  \definition[支,枝]{s.}{caneta | lápis}
\end{entry}

\begin{entry}{笔记}{bi3 ji4}{10,5}[HSK 2][Radicais ⽵、⾔]
  \definition[篇,本,个]{s.}{notas | ensaios | esboços}
  \definition{v.}{tomar nota (por escrito)}
\end{entry}

\begin{entry}{笔记本}{bi3ji4ben3}{10,5,5}[HSK 2][Radicais ⽵、⾔、⽊]
  \definition[本]{s.}{caderno}
  \definition{s.}{\emph{laptop}}
\end{entry}

\begin{entry}{必定}{bi4ding4}{5,8}[Radicais ⼼、⼧]
  \definition{adv.}{sem falta | certamente | com certeza | definitivamente | inevitavelmente | com determinação}
  \definition{v.}{estar vinculado a | ter certeza de}
\end{entry}

\begin{entry}{必然}{bi4ran2}{5,12}[HSK 3][Radicais ⼼、⽕]
  \definition{adj.}{certo | inevitável | necessário}
  \definition{adv.}{inevitavelmente}
  \definition{s.}{necessidade}
\end{entry}

\begin{entry}{必须}{bi4xu1}{5,9}[HSK 2][Radicais ⼼、⾴]
  \definition{adv.}{necessariamente | obrigatoriamente}
\end{entry}

\begin{entry}{必要}{bi4yao4}{5,9}[HSK 3][Radicais ⼼、⾑]
  \definition{adj.}{necessário | essencial | indispensável}
  \definition[个,些]{s.}{necessidade}
\end{entry}

\begin{entry}{毕业}{bi4ye4}{6,5}[HSK 4][Radicais ⽐、⼀]
  \definition{s.}{formatura}
  \definition{v.+compl.}{formar-se}
\end{entry}

\begin{entry}{毕业生}{bi4 ye4 sheng1}{6,5,5}[HSK 4][Radicais ⽐、⼀、⽣]
  \definition[个]{s.}{diplomado; graduado; bacharel; pessoa que recebeu um diploma, grau ou certificado}
\end{entry}

\begin{entry}{闭嘴}{bi4zui3}{6,16}[Radicais ⾨、⼝]
  \definition{expr.}{Cale-se!}
\end{entry}

\begin{entry}{壁虎}{bi4hu3}{16,8}[Radicais ⼟、⾌]
  \definition{s.}{lagartixa}
\end{entry}

\begin{entry}{壁纸}{bi4zhi3}{16,7}[Radicais ⼟、⽷]
  \definition{s.}{papel de parede}
\end{entry}

\begin{entry}{避}{bi4}{16}[HSK 4][Radical ⾌]
  \definition{v.}{evitar; evadir; esquivar-se; buscar abrigo; fugir | impedir; manter afastado; repelir; previnir}
\end{entry}

\begin{entry}{避免}{bi4mian3}{16,7}[HSK 4][Radicais ⾌、⼉]
  \definition{v.}{evitar; desviar; abster-se de; tentar não fazer com que algo aconteça; prevenir; tentar impedir (que algo ruim aconteça) com antecedência}
\end{entry}

\begin{entry}{边}{bian1}{5}[HSK 2][Radical ⾡]
  \definition{adv.}{simultaneamente}
  \definition[个]{s.}{fronteira | limite | borda | margem | lado}
  \seeref{边}{bian5}
\end{entry}

\begin{entry}{边防}{bian1fang2}{5,6}[Radicais ⾡、⾩]
  \definition{s.}{defesa da fronteira}
\end{entry}

\begin{entry}{边关}{bian1guan1}{5,6}[Radicais ⾡、⼋]
  \definition{s.}{posto de fronteira | posição defensiva estratégica na fronteira}
\end{entry}

\begin{entry}{编}{bian1}{12}[HSK 4][Radical ⽷]
  \definition*{s.}{sobrenome Bian}
  \definition{s.}{livro; volume; parte de um livro}
  \definition{v.}{tecer; trançar; entrançar | fazer uma lista; organizar em uma lista; organizar; agrupar | editar; compilar | compor; escrever | fabricar; inventar; fazer; preparar}
\end{entry}

\begin{entry}{编程}{bian1cheng2}{12,12}[Radicais ⽷、⽲]
  \definition{s.}{programa de computador}
  \definition{v.}{programar computador}
\end{entry}

\begin{entry}{邉}{bian1}{17}[Radical ⾡]
  \variantof{边}
\end{entry}

\begin{entry}{变}{bian4}{8}[HSK 2][Radical ⼜]
  \definition{v.}{mudar | transformar | variar}
\end{entry}

\begin{entry}{变成}{bian4 cheng2}{8,6}[HSK 2][Radicais ⼜、⼽]
  \definition{v.}{mudar | transformar-se em | tornar-se}
\end{entry}

\begin{entry}{变更}{bian4geng1}{8,7}[Radicais ⼜、⽈]
  \definition{v.}{alterar | mudar | modificar}
\end{entry}

\begin{entry}{变化}{bian4hua4}{8,4}[HSK 3][Radicais ⼜、⼔]
  \definition[个]{s.}{mudança | variação}
  \definition{v.}{(intransitivo) mudar, variar}
\end{entry}

\begin{entry}{变节}{bian4jie2}{8,5}[Radicais ⼜、⾋]
  \definition{s.}{traição | deserção | vira-casaca}
  \definition{v.}{mudar de lado politicamente}
\end{entry}

\begin{entry}{变迁}{bian4qian1}{8,6}[Radicais ⼜、⾡]
  \definition{s.}{mudanças | vicissitudes}
\end{entry}

\begin{entry}{变数}{bian4shu4}{8,13}[Radicais ⼜、⽁]
  \definition{s.}{(matemática) variável}
\end{entry}

\begin{entry}{变为}{bian4 wei2}{8,4}[HSK 3][Radicais ⼜、⼂]
  \definition{v.}{transformar-se em | tornar-se | mudar para}
\end{entry}

\begin{entry}{变心}{bian4xin1}{8,4}[Radicais ⼜、⼼]
  \definition{v.+compl.}{deixar de ser fiel}
\end{entry}

\begin{entry}{变性}{bian4xing4}{8,8}[Radicais ⼜、⼼]
  \definition{s.}{desnaturação | transexual}
  \definition{v.}{desnaturar | mudar de sexo}
\end{entry}

\begin{entry}{变异}{bian4yi4}{8,6}[Radicais ⼜、⼶]
  \definition{s.}{variação | mutação}
\end{entry}

\begin{entry}{变装}{bian4zhuang1}{8,12}[Radicais ⼜、⾐]
  \definition{v.}{trocar de roupa | vestir-se | vestir uma fantasia | disfarçar-se ou fantasiar-se de personagem real ou ficcional, \emph{cosplay} | travestir-se}
\end{entry}

\begin{entry}{遍}{bian4}{12}[HSK 2][Radical ⾡]
  \definition{adv.}{em todos os lugares | por toda parte}
  \definition{clas.}{para a repetição de ações de leitura, fala ou escrita}
\end{entry}

\begin{entry}{辩论}{bian4lun4}{16,6}[HSK 4][Radicais ⾟、⾔]
  \definition[场,次]{s.}{debate; argumento; a atividade comportamental em si de argumentar ou refutar diferentes pontos de vista ou afirmações, ou uma ocasião ou situação em que tal argumentação ou refutação é feita}
  \definition{v.}{debater; obter um entendimento unificado ou correto, ambos os lados usam linguagem, palavras etc. para explicar seus pontos de vista, apontar os erros ou as contradições do outro lado}
\end{entry}

\begin{entry}{辫子}{bian4zi5}{17,3}[Radicais ⾟、⼦]
  \definition[根,条]{s.}{trança | um erro ou falha que pode ser explorado por um oponente | alça}
\end{entry}

\begin{entry}{边}{bian5}{5}[Radical ⾡]
  \definition{suf.}{sufixo de uma palavra de localidade}
  \seeref{边}{bian1}
\end{entry}

\begin{entry}{标题}{biao1ti2}{9,15}[HSK 3][Radicais ⽊、⾴]
  \definition[个,条,篇]{s.}{título | manchete | cabeçalho}
\end{entry}

\begin{entry}{标志}{biao1zhi4}{9,7}[HSK 4][Radicais ⽊、⼼]
  \definition[个,种]{s.}{sinal; marca; logotipo; símbolo; emblema; marcações que caracterizam um objeto para facilitar a identificação}
  \definition{v.}{marcar; indicar; simbolizar; identificar}
\end{entry}

\begin{entry}{标准}{biao1zhun3}{9,10}[HSK 3][Radicais ⽊、⼎]
  \definition{adj.}{criterioso | padronizado | normatizado}
  \definition[个]{s.}{critério | padrão (oficial) | norma}
\end{entry}

\begin{entry}{镖}{biao1}{16}[Radical ⾦]
  \definition{s.}{dardo | arma de arremesso | mercadorias enviadas sob a proteção de uma escolta armada}
\end{entry}

\begin{entry}{表}{biao3}{8}[HSK 2][Radical ⾐]
  \definition*{s.}{sobrenome Biao}
  \definition{s.}{superfície externa | a relação entre os filhos ou netos de um irmão e uma irmã ou de irmãs | exemplo | modelo | memorial a um imperador dos tempos antigos | gráfico | formulário | lista | tabela | medidor | relógio de pulso}
\end{entry}

\begin{entry}{表白}{biao3bai2}{8,5}[Radicais ⾐、⽩]
  \definition{s.}{declaração | confissão}
  \definition{v.}{confessar a si mesmo | expressar | revelar pensamentos ou sentimentos de alguém}
\end{entry}

\begin{entry}{表达}{biao3da2}{8,6}[HSK 3][Radicais ⾐、⾡]
  \definition{v.}{entregar | expressar | mostrar | transmitir | comunicar}
\end{entry}

\begin{entry}{表格}{biao3ge2}{8,10}[HSK 3][Radicais ⾐、⽊]
  \definition[份,张]{s.}{tabela | formulário}
\end{entry}

\begin{entry}{表面}{biao3mian4}{8,9}[HSK 3][Radicais ⾐、⾯]
  \definition{s.}{superfície | lado de fora | aparência | superficialidade}
\end{entry}

\begin{entry}{表明}{biao3ming2}{8,8}[HSK 3][Radicais ⾐、⽇]
  \definition{v.}{deixar claro | tornar conhecido | declarar claramente}
\end{entry}

\begin{entry}{表情}{biao3qing2}{8,11}[HSK 4][Radicais ⾐、⼼]
  \definition[个,种,幅]{s.}{expressão; expressão facial; expressão de pensamentos e sentimentos internos por meio de mudanças faciais ou de gestos}
  \definition{v.}{expressar pensamentos e sentimentos internos por meio de mudanças faciais ou de gestos}
\end{entry}

\begin{entry}{表示}{biao3shi4}{8,5}[HSK 2][Radicais ⾐、⽰]
  \definition{s.}{expressão | indicação}
  \definition{v.}{expressar | mostrar | indicar | significar}
\end{entry}

\begin{entry}{表现}{biao3xian4}{8,8}[HSK 3][Radicais ⾐、⾒]
  \definition[个,种,份]{s.}{desempenho | expressão  manifestação | comportamento}
  \definition{v.}{mostrar | expressar | exibir | manifestar | descrever}
\end{entry}

\begin{entry}{表演}{biao3yan3}{8,14}[HSK 3][Radicais ⾐、⽔]
  \definition[场]{s.}{representação | atuação | exposição}
  \definition{v.}{executar | atuar | jogar | demonstrar | agir | fingir}
\end{entry}

\begin{entry}{表演赛}{biao3yan3sai4}{8,14,14}[Radicais ⾐、⽔、⾙]
  \definition{s.}{partida ou jogo de exibição}
\end{entry}

\begin{entry}{表演特技}{biao3yan3 te4ji4}{8,14,10,7}[Radicais ⾐、⽔、⽜、⼿]
  \definition{s.}{acrobacia | pirueta | façanha}
\end{entry}

\begin{entry}{表演艺术}{biao3yan3 yi4shu4}{8,14,4,5}[Radicais ⾐、⽔、⾋、⽊]
  \definition{s.}{arte performática}
\end{entry}

\begin{entry}{表演游戏}{biao3yan3 you2xi4}{8,14,12,6}[Radicais ⾐、⽔、⽔、⼽]
  \definition{s.}{exibição dramática}
\end{entry}

\begin{entry}{表演者}{biao3yan3 zhe3}{8,14,8}[Radicais ⾐、⽔、⽼]
  \definition{s.}{ator}
\end{entry}

\begin{entry}{表扬}{biao3yang2}{8,6}[HSK 4][Radicais ⾐、⼿]
  \definition[次,种,份]{s.}{elogios públicos por boas ações}
  \definition{v.}{elogiar; louvar}
\end{entry}

\begin{entry}{表扬信}{biao3yang2 xin4}{8,6,9}[Radicais ⾐、⼿、⼈]
  \definition{s.}{carta de elogio | depoimento}
\end{entry}

\begin{entry}{别}{bie2}{7}[HSK 1,4][Radical ⼑]
  \definition*{s.}{sobrenome Bie}
  \definition{adv.}{não; nada de (pedir a alguém para não fazer); é melhor não | talvez, usado em conjunto com a palavra ``是'' para indicar especulação.}
  \definition{pron.}{outro; algum outro}
  \definition{s.}{distinção; diferença | classificação}
  \definition{v.}{deixar; partir; separar | diferenciar; distinguir; encontrar aspectos diferentes | fixar objetos com pinos | girar; virar | aderir; colar; preder}
  \seeref{别}{bie4}
  \seealsoref{是}{shi4}
\end{entry}

\begin{entry}{别的}{bie2 de5}{7,8}[HSK 1][Radicais ⼑、⽩]
  \definition{pron.}{outro}
\end{entry}

\begin{entry}{别人}{bie2ren5}{7,2}[Radicais ⼑、⼈]
  \definition{pron.}{outra pessoa | outro povo | outros}
\end{entry}

\begin{entry}{别说}{bie2shuo1}{7,9}[Radicais ⼑、⾔]
  \definition{v.}{não falar de | não mencionar}
\end{entry}

\begin{entry}{别}{bie4}{7}[Radical ⼑]
  \definition{v.}{fazer com que alguém mude seus hábitos, opiniões, etc.}
  \seeref{别}{bie2}
\end{entry}

\begin{entry}{宾馆}{bin1guan3}{10,11}[Radicais ⼧、⾷]
  \definition[个,家]{s.}{casa de hóspedes | hotel}
\end{entry}

\begin{entry}{冰}{bing1}{6}[HSK 4][Radical ⼎]
  \definition{adj.}{frio (pessoa)| hostil}
  \definition[块]{s.}{gelo; água em estado sólido |  (gíria) metanfetamina}
  \definition{v.}{colocar gelo; colocar gelo ao redor; colocar no gelo; resfriar objetos com gelo ou água fria | sentir frio}
\end{entry}

\begin{entry}{冰糕}{bing1gao1}{6,16}[Radicais ⼎、⽶]
  \definition{s.}{sorvete | picolé}
\end{entry}

\begin{entry}{冰棍}{bing1gun4}{6,12}[Radicais ⼎、⽊]
  \definition[根]{s.}{picolé}
\end{entry}

\begin{entry}{冰激凌}{bing1ji1ling2}{6,16,10}[Radicais ⼎、⽔、⼎]
  \definition{s.}{sorvete}
\end{entry}

\begin{entry}{冰球}{bing1qiu2}{6,11}[Radicais ⼎、⽟]
  \definition{s.}{hóquei no gelo}
\end{entry}

\begin{entry}{冰天雪地}{bing1tian1-xue3di4}{6,4,11,6}[Radicais ⼎、⼤、⾬、⼟]
  \definition{expr.}{um mundo de gelo e neve}
\end{entry}

\begin{entry}{冰箱}{bing1xiang1}{6,15}[HSK 4][Radicais ⼎、⾋]
  \definition[台,个]{s.}{geladeira; freezer; refrigerador; aparelhos para congelar alimentos ou medicamentos com gelo para mantê-los frios}
\end{entry}

\begin{entry}{冰雪}{bing1 xue3}{6,11}[HSK 4][Radicais ⼎、⾬]
  \definition{adj.}{puro como gelo e neve; descreve uma pessoa como pura}
  \definition{s.}{gelo e neve}
\end{entry}

\begin{entry}{兵}{bing1}{7}[HSK 4][Radical ⼋]
  \definition[名]{s.}{armas; armamentos | soldado; pessoal militar | exército; tropas | soldado raso; membro mais jovem do exército | assuntos militares (estratégia) | peão, uma das peças do xadrez chinês}
\end{entry}

\begin{entry}{兵器}{bing1qi4}{7,16}[Radicais ⼋、⼝]
  \definition{s.}{armas | armamento}
\end{entry}

\begin{entry}{饼}{bing3}{9}[Radical ⾷]
  \definition[张]{s.}{panqueca | biscoito | torta}
\end{entry}

\begin{entry}{饼干}{bing3gan1}{9,3}[Radicais ⾷、⼲]
  \definition[片,块]{s.}{bolacha | biscoito}
\end{entry}

\begin{entry}{并}{bing4}{6}[HSK 3,4][Radical ⼲]
  \definition{adv.}{igualmente; simultaneamente; lado a lado; coisas diferentes existem ao mesmo tempo; coisas diferentes estão acontecendo ao mesmo tempo | em absoluto (usado antes de uma negativa para dar ênfase);  usado antes de uma palavra negativa para reforçar o tom e refutá-la ligeiramente}
  \definition{conj.}{além de; e}
  \definition{v.}{combinar; fundir; incorporar; anexar; juntar}
\end{entry}

\begin{entry}{并排}{bing4pai2}{6,11}[Radicais ⼲、⼿]
  \definition{adv.}{lado a lado}
\end{entry}

\begin{entry}{并且}{bing4qie3}{6,5}[HSK 3][Radicais ⼲、⼀]
  \definition{conj.}{além disso | o que é mais | e}
\end{entry}

\begin{entry}{幷}{bing4}{8}[Radical ⼲]
  \variantof{并}
\end{entry}

\begin{entry}{倂}{bing4}{10}[Radical ⼈]
  \variantof{并}
\end{entry}

\begin{entry}{病}{bing4}{10}[HSK 1][Radical ⽧]
  \definition[场]{s.}{doença}
  \definition{v.}{adoecer | estar doente}
\end{entry}

\begin{entry}{病人}{bing4 ren2}{10,2}[HSK 1][Radicais ⽧、⼈]
  \definition{s.}{doente | paciente}
\end{entry}

\begin{entry}{拨转}{bo1zhuan3}{8,8}[Radicais ⼿、⾞]
  \definition{v.}{transferir (fundos, etc.) | virar | dar a volta}
\end{entry}

\begin{entry}{波}{bo1}{8}[Radical ⽔]
  \definition*{s.}{Polônia, abreviação de 波兰}
  \definition{s.}{onda | ondulação | tempestade | surto}
  \seeref{波兰}{bo1lan2}
\end{entry}

\begin{entry}{波兰}{bo1lan2}{8,5}[Radicais ⽔、⼋]
  \definition*{s.}{Polônia}
\end{entry}

\begin{entry}{波音}{bo1yin1}{8,9}[Radicais ⽔、⾳]
  \definition*{s.}{Boeing (empresa aeroespacial)}
  \definition{s.}{mordente (música)}
\end{entry}

\begin{entry}{玻璃}{bo1li5}{9,14}[Radicais ⽟、⽟]
  \definition[张,塊]{s.}{vidro | (gíria) homossexual masculino}
\end{entry}

\begin{entry}{般}{bo1}{10}[Radical ⾈]
  \definition{s.}{utilizado em 般若 \dpy{bo1re3}}
  \seeref{般若}{bo1re3}
\end{entry}

\begin{entry}{般若}{bo1re3}{10,8}[Radicais ⾈、⾋]
  \definition*{s.}{Prajna (sânscrito), \emph{insight} sobre a verdadeira natureza da realidade | (Budismo) sabedoria}
\end{entry}

\begin{entry}{啵}{bo1}{11}[Radical ⼝]
  \definition{s.}{(onomatopéia) borbulhar}
  \seeref{啵}{bo5}
\end{entry}

\begin{entry}{菠菜}{bo1cai4}{11,11}[Radicais ⾋、⾋]
  \definition[棵]{s.}{espinafre}
\end{entry}

\begin{entry}{播出}{bo1 chu1}{15,5}[HSK 3][Radicais ⼿、⼐]
  \definition{v.}{transmitir | estar no ar}
\end{entry}

\begin{entry}{播放}{bo1fang4}{15,8}[HSK 3][Radicais ⼿、⽅]
  \definition{v.}{ir ao ar | transmitir por rádio | mostrar | transmitir (um programa de TV)}
\end{entry}

\begin{entry}{播音}{bo1yin1}{15,9}[Radicais ⼿、⾳]
  \definition{s.}{transmissão}
  \definition{v.+compl.}{transmitir}
\end{entry}

\begin{entry}{脖子}{bo2zi5}{11,3}[Radicais ⾁、⼦]
  \definition[个]{s.}{pescoço}
\end{entry}

\begin{entry}{博文}{bo2wen2}{12,4}[Radicais ⼗、⽂]
  \definition{s.}{artigo em um blog}
  \definition{v.}{escrever um artigo em um blog}
\end{entry}

\begin{entry}{博物馆}{bo2wu4guan3}{12,8,11}[Radicais ⼗、⽜、⾷]
  \definition{s.}{museu}
\end{entry}

\begin{entry}{博主}{bo2zhu3}{12,5}[Radicais ⼗、⼂]
  \definition{s.}{blogueiro}
\end{entry}

\begin{entry}{薄}{bo2}{16}[Radical ⾋]
  \definition{adj.}{ligeiro; escasso; pequeno | mesquinho; pouco generoso; cruel | frívolo; fútil; leviano}
  \seeref{薄}{bao2}
\end{entry}

\begin{entry}{啵}{bo5}{11}[Radical ⼝]
  \definition{part.}{partícula gramaticalmente equivalente a 吧}
  \seeref{啵}{bo1}
  \seealsoref{吧}{ba5}
\end{entry}

\begin{entry}{不}{bu2}[(antes de quarto tom)]{4}[HSK 1][Radical ⼀]
  \definition{adv.}{não}
  \definition{pref.}{prefixo negativo}
  \seeref{不}{bu4}
  \seeref{不}{bu5}
\end{entry}

\begin{entry}{不必}{bu2 bi4}{4,5}[HSK 3][Radicais ⼀、⼼]
  \definition{adv.}{não precisa | não tem que}
\end{entry}

\begin{entry}{不错}{bu2 cuo4}{4,13}[HSK 2][Radicais ⼀、⾦]
  \definition{adj.}{correto | não (é) mau | bastante bom | certo}
\end{entry}

\begin{entry}{不大}{bu2 da4}{4,3}[HSK 1][Radicais ⼀、⼤]
  \definition{adv.}{não muito | não frequentemente | raramente |dificilmente | escassamente}
\end{entry}

\begin{entry}{不大离}{bu2da4li2}{4,3,10}[Radicais ⼀、⼤、⼇]
  \definition{adj.}{bem perto | quase certo | nada mal}
\end{entry}

\begin{entry}{不但}{bu2 dan4}{4,7}[HSK 2][Radicais ⼀、⼈]
  \definition{conj.}{não somente}
\end{entry}

\begin{entry}{不但……而且……}{bu2 dan4 er2qie3}{4,7,6,5}[HSK 2][Radicais ⼀、⼈、⽽、⼀]
  \definition{conj.}{não só\dots mas também\dots}
\end{entry}

\begin{entry}{不到}{bu2dao4}{4,8}[Radicais ⼀、⼑]
  \definition{adj.}{insuficiente}
  \definition{adv.}{menos que}
  \definition{v.}{não chegar}
\end{entry}

\begin{entry}{不断}{bu2duan4}{4,11}[HSK 3][Radicais ⼀、⽄]
  \definition{adv.}{continuamente | sem fim}
\end{entry}

\begin{entry}{不对}{bu2 dui4}{4,5}[HSK 1][Radicais ⼀、⼨]
  \definition{adj.}{incorreto | errado | anormal | estranho | estar em desacordo com | ser difícil de conviver}
\end{entry}

\begin{entry}{不够}{bu2 gou4}{4,11}[HSK 2][Radicais ⼀、⼣]
  \definition{adv.}{insuficiente}
  \definition{v.}{não ser suficiente}
\end{entry}

\begin{entry}{不过}{bu2guo4}{4,6}[HSK 2][Radicais ⼀、⾡]
  \definition{conj.}{mas | contudo | no entanto}
\end{entry}

\begin{entry}{不客气}{bu2 ke4 qi5}{4,9,4}[HSK 1][Radicais ⼀、⼧、⽓]
  \definition{adj.}{indelicado | rude | brusco}
  \definition{expr.}{de nada | não há de que | não mencione isso}
\end{entry}

\begin{entry}{不论}{bu2 lun4}{4,6}[HSK 3][Radicais ⼀、⾔]
  \definition{conj.}{não importa (o que, quem, como, etc.) | se \dots ou \dots}
\end{entry}

\begin{entry}{不论……都……}{bu2lun4 dou1}{4,6,10}[Radicais ⼀、⾔、⾢]
  \definition{conj.}{não apenas\dots, (o que, quem, como, etc.), \dots}
\end{entry}

\begin{entry}{不论……也……}{bu2lun4 ye3}{4,6,3}[Radicais ⼀、⾔、⼄]
  \definition{conj.}{não apenas\dots, (o que, quem, como, etc.), \dots}
\end{entry}

\begin{entry}{不日}{bu2ri4}{4,4}[Radicais ⼀、⽇]
  \definition{adv.}{em alguns dias}
\end{entry}

\begin{entry}{不是话}{bu2shi4hua4}{4,9,8}[Radicais ⼀、⽇、⾔]
  \definition{expr.}{sem razão | demasiado irracionável}
  \seeref{不像话}{bu2xiang4hua4}
  \seeref{不成话}{bu4cheng2hua4}
\end{entry}

\begin{entry}{不太}{bu2 tai4}{4,4}[HSK 2][Radicais ⼀、⼤]
  \definition{adv.}{não bastante | não muito}
\end{entry}

\begin{entry}{不像话}{bu2xiang4hua4}{4,13,8}[Radicais ⼀、⼈、⾔]
  \definition{expr.}{sem razão | demasiado irracionável}
  \seeref{不是话}{bu2shi4hua4}
  \seeref{不成话}{bu4cheng2hua4}
\end{entry}

\begin{entry}{不要}{bu2 yao4}{4,9}[HSK 2][Radicais ⼀、⾑]
  \definition{adv.}{nada de (pedir a alguém para não fazer) | não}
\end{entry}

\begin{entry}{不要紧}{bu2yao4jin3}{4,9,10}[HSK 4][Radicais ⼀、⾑、⽷]
  \definition{adj.}{sem importância; sem seriedade; não problemático | não importa; não é um obstáculo | parece estar tudo bem, mas | à primeira vista, isso não parece atrapalhar}
\end{entry}

\begin{entry}{不用}{bu2 yong4}{4,5}[HSK 1][Radicais ⼀、⽤]
  \definition{v.}{não precisar}
  \seeref{甭}{beng2}
\end{entry}

\begin{entry}{不在乎}{bu2 zai4 hu1}{4,6,5}[HSK 4][Radicais ⼀、⼟、⼃]
  \definition{v.}{não se importar; não dar a mínima; não dar atenção}
\end{entry}

\begin{entry}{不注意}{bu2zhu4yi4}{4,8,13}[Radicais ⼀、⽔、⼼]
  \definition{adj.}{impensado | distraído}
  \definition{s.}{descuido | distração}
\end{entry}

\begin{entry}{补}{bu3}{7}[HSK 3][Radical ⾐]
  \definition*{s.}{sobrenome Bu}
  \definition{s.}{benefício | ajuda | uso}
  \definition{v.}{consertar | remendar | preencher | adicionar suplemento | suprir | compensar |nutrir}
\end{entry}

\begin{entry}{补充}{bu3chong1}{7,6}[HSK 3][Radicais ⾐、⼉]
  \definition{adj.}{adicional | suplementar}
  \definition[个]{s.}{aditivo | suplemento}
  \definition{v.}{reabastecer | suplementar | complementar}
\end{entry}

\begin{entry}{不}{bu4}{4}[HSK 1][Radical ⼀]
  \definition{adv.}{não}
  \definition{pref.}{prefixo negativo}
  \seeref{不}{bu2}
  \seeref{不}{bu5}
\end{entry}

\begin{entry}{不安}{bu4'an1}{4,6}[HSK 3][Radicais ⼀、⼧]
  \definition{adj.}{inquieto | instável | intranquilo | pesaroso}
\end{entry}

\begin{entry}{不成话}{bu4cheng2hua4}{4,6,8}[Radicais ⼀、⼽、⾔]
  \definition{expr.}{sem razão | demasiado irracionável}
  \seeref{不是话}{bu2shi4hua4}
  \seeref{不像话}{bu2xiang4hua4}
\end{entry}

\begin{entry}{不得不}{bu4de2bu4}{4,11,4}[HSK 3][Radicais ⼀、⼻、⼀]
  \definition{adv.}{tem que | não tem escolha a não ser}
\end{entry}

\begin{entry}{不公}{bu4gong1}{4,4}[Radicais ⼀、⼋]
  \definition{adj.}{injusto}
\end{entry}

\begin{entry}{不管}{bu4guan3}{4,14}[HSK 4][Radicais ⼀、⽵]
  \definition{conj.}{não importa (o que, como, etc.); independentemente de; indica que, embora as condições ou circunstâncias tenham mudado, o resultado permanece o mesmo}
  \seeref{不管……都……}{bu4guan3 dou1}
  \seeref{不管……也……}{bu4guan3 ye3}
\end{entry}

\begin{entry}{不管……都……}{bu4guan3 dou1}{4,14,10}[Radicais ⼀、⽵、⾢]
  \definition{conj.}{não apenas\dots, (o que, quem, como, etc.), \dots}
\end{entry}

\begin{entry}{不管……也……}{bu4guan3 ye3}{4,14,3}[Radicais ⼀、⽵、⼄]
  \definition{conj.}{não apenas\dots, (o que, quem, como, etc.), \dots}
\end{entry}

\begin{entry}{不光}{bu4 guang1}{4,6}[HSK 3][Radicais ⼀、⼉]
  \definition{adv.}{não é o único}
  \definition{conj.}{não somente}
\end{entry}

\begin{entry}{不好意思}{bu4 hao3 yi4 si5}{4,6,13,9}[HSK 2][Radicais ⼀、⼥、⼼、⼼]
  \definition{adj.}{pedir desculpas (por incomodar alguém) | sentir-se envergonhado | achar isso embaraçoso}
\end{entry}

\begin{entry}{不仅}{bu4jin3}{4,4}[HSK 3][Radicais ⼀、⼈]
  \definition{adv.}{não apenas (em número, quantidade ou extensão)}
  \definition{conj.}{não somente}
\end{entry}

\begin{entry}{不久}{bu4 jiu3}{4,3}[HSK 2][Radicais ⼀、⼃]
  \definition{adj.}{em breve | futuro próximo | logo depois | não muito depois | não muito tempo (antes ou depois de algo)}
\end{entry}

\begin{entry}{不可避免}{bu4ke3bi4mian3}{4,5,16,7}[Radicais ⼀、⼝、⾌、⼉]
  \definition{adj./adv.}{inevitável}
\end{entry}

\begin{entry}{不满}{bu4 man3}{4,13}[HSK 2][Radicais ⼀、⽔]
  \definition{adj.}{ressentido | insatisfeito | descontente}
  \definition{v.}{estar descontente com |ser menor que}
\end{entry}

\begin{entry}{不然}{bu4ran2}{4,12}[HSK 4][Radicais ⼀、⽕]
  \definition{adj.}{não é assim; não é o caso}
  \definition{conj.}{se não; caso contrário; indica outra consequência ou circunstância que teria ocorrido se não fosse}
\end{entry}

\begin{entry}{不如}{bu4ru2}{4,6}[HSK 2][Radicais ⼀、⼥]
  \definition{conj.}{em vez de | melhor que | seria melhor}
  \definition{v.}{ser inferior a | não ser igual a | não ser tão bom quanto | não poder fazer melhor que}
\end{entry}

\begin{entry}{不少}{bu4 shao3}{4,4}[HSK 2][Radicais ⼀、⼩]
  \definition{adj.}{muitos | bastante | não poucos}
\end{entry}

\begin{entry}{不是……而是}{bu4shi4 er2 shi4}{4,9,6,9}[Radicais ⼀、⽇、⽽、⽇]
  \definition{conj.}{não somente\dots mas também\dots, expressam um relacionamento mais profundo e avançado em significado, mas as orações antes e depois são consistentes em expressar significados negativos e afirmativos, entretanto, a primeira metade da frase expressa negação, e a segunda metade expressa afirmação, e o significado das orações anteriores e seguintes não pode ser de um nível mais alto}
\end{entry}

\begin{entry}{不同}{bu4 tong2}{4,6}[HSK 2][Radicais ⼀、⼝]
  \definition{adj.}{diferente | distinto}
\end{entry}

\begin{entry}{不行}{bu4 xing2}{4,6}[HSK 2][Radicais ⼀、⾏]
  \definition{adj.}{não funciona | não é bom}
  \definition{adv.}{profundamente | terrivelmente | extremamente}
  \definition{v.}{não fazer | não ser permitido | estar fora de questão | estar à beira da morte}
\end{entry}

\begin{entry}{不一定}{bu4 yi2 ding4}{4,1,8}[HSK 2][Radicais ⼀、⼀、⼧]
  \definition{adv.}{talvez | incerto | não tenho certeza | não necessariamente}
\end{entry}

\begin{entry}{不一会儿}{bu4 yi2 hui4r5}{4,1,6,2}[HSK 2][Radicais ⼀、⼀、⼈、⼉]
  \definition{expr.}{em um momento | em pouco tempo |em breve}
\end{entry}

\begin{entry}{不止}{bu4zhi3}{4,4}[Radicais ⼀、⽌]
  \definition{adv.}{incessantemente | sem fim | mais que | não limitado a}
\end{entry}

\begin{entry}{布}{bu4}{5}[HSK 3][Radical ⼱]
  \definition*{s.}{sobrenome Bu}
  \definition[块,幅,匹]{s.}{pano | tecido | uma moeda de cobre nos tempos antigos}
  \definition{v.}{anunciar | declarar | tornar conhecido | proclamar | publicar | espalhar | disseminar |organizar | implantar | dispor}
\end{entry}

\begin{entry}{布谷鸟}{bu4gu3niao3}{5,7,5}[Radicais ⼱、⾕、⿃]
  \definition{s.}{cuco (pássaro)}
  \seealsoref{杜鹃}{du4juan1}
  \seealsoref{杜鹃鸟}{du4juan1niao3}
  \seealsoref{杜宇}{du4yu3}
\end{entry}

\begin{entry}{布署}{bu4shu3}{5,13}[Radicais ⼱、⽹]
  \variantof{部署}
\end{entry}

\begin{entry}{布置}{bu4zhi4}{5,13}[HSK 4][Radicais ⼱、⽹]
  \definition{v.}{arrumar; organizar; decorar; colocar adequadamente objetos ou paisagismo, conforme necessário | designar; tomar providências para; dar instruções sobre; organizar trabalho, atividades, etc.}
\end{entry}

\begin{entry}{步}{bu4}{7}[HSK 3][Radical ⽌]
  \definition*{s.}{sobrenome Bu}
  \definition{clas.}{uma unidade antiga para medida de comprimento, equivalente a cinco chi}
  \definition{s.}{ritmo | passo | estágio | passo | condição | situação | estado}
  \definition{v.}{ir a pé | andar | pisar | contar passos}
\end{entry}

\begin{entry}{步行}{bu4 xing2}{7,6}[HSK 4][Radicais ⽌、⾏]
  \definition{v.}{caminhar; ir a pé; andar a pé (diferente de andar de carro, a cavalo, etc.)}
\end{entry}

\begin{entry}{部}{bu4}{10}[HSK 3][Radical ⾢]
  \definition{clas.}{para obras de literatura, filmes, máquinas etc.}
  \definition[根]{s.}{departamento | divisão | ministério | seção | parte | tropas}
\end{entry}

\begin{entry}{部队}{bu4dui4}{10,4}[Radicais ⾢、⾩]
  \definition[个]{s.}{exército | forças armadas | tropas | unidades}
\end{entry}

\begin{entry}{部分}{bu4fen5}{10,4}[HSK 2][Radicais ⾢、⼑]
  \definition[个]{s.}{parte | parte de | uma parte de | pedaço | secção}
\end{entry}

\begin{entry}{部门}{bu4men2}{10,3}[HSK 3][Radicais ⾢、⾨]
  \definition[个]{s.}{filial | departamento | divisão | seção}
\end{entry}

\begin{entry}{部属}{bu4shu3}{10,12}[Radicais ⾢、⼫]
  \definition{s.}{afiliado a um ministério | subordinado | tropas sob comando de alguém}
\end{entry}

\begin{entry}{部署}{bu4shu3}{10,13}[Radicais ⾢、⽹]
  \definition{s.}{implantação}
  \definition{v.}{implantar}
\end{entry}

\begin{entry}{部下}{bu4xia4}{10,3}[Radicais ⾢、⼀]
  \definition{s.}{subordinado | tropas sob comando de alguém}
\end{entry}

\begin{entry}{部长}{bu4 zhang3}{10,4}[HSK 3][Radicais ⾢、⾧]
  \definition[个,位,名]{s.}{ministro | chefe de departamento | chefe de seção}
\end{entry}

\begin{entry}{部族}{bu4zu2}{10,11}[Radicais ⾢、⽅]
  \definition{adj.}{tribal}
  \definition{s.}{tribo}
\end{entry}

\begin{entry}{不}{bu5}{4}[HSK 1][Radical ⼀]
  \definition{adv.}{não (em expressões ``v.+不+v.'')}
  \seeref{不}{bu2}
  \seeref{不}{bu4}
\end{entry}

%%%%% EOF %%%%%


%%%
%%% C
%%%

\section*{C}\addcontentsline{toc}{section}{C}

\begin{entry}{擦拭}{ca1shi4}{17,9}
  \definition{v.}{limpar com um pano}
\end{entry}

\begin{entry}{猜}{cai1}{11}[Radical 犬]
  \definition{v.}{advinhar}
\end{entry}

\begin{entry}{才}{cai2}{3}[Radical 手][HSK 2]
  \definition{adv.}{apenas (seguido por uma cláusula numérica) | meramente | só (indicando que algo está acontecendo mais tarde do que o esperado) | só depois | só então | não\dots até (precedido por uma cláusula de condição ou razão) | há um momento atrás | há pouco tempo}
  \definition{conj.}{apenas quando}
  \definition{s.}{um indivíduo capaz | habilidade | talento}
\end{entry}

\begin{entry}{才华}{cai2hua2}{3,6}
  \definition[份]{s.}{talento}
\end{entry}

\begin{entry}{才略}{cai2lve4}{3,11}
  \definition{s.}{habilidade e sagacidade}
\end{entry}

\begin{entry}{才能}{cai2 neng2}{3,10}[HSK 3]
  \definition[间]{s.}{talento | habilidade | dom | capacidade}
\end{entry}

\begin{entry}{裁}{cai2}{12}[Radical 衣]
  \definition{s.}{decisão | julgamento}
  \definition{v.}{recortar (tecido de uma roupa) | cortar | aparar | reduzir | diminuir | cortar pessoal de uma equipe}
\end{entry}

\begin{entry}{采访}{cai3fang3}{8,6}
  \definition{s.}{entrevista}
  \definition{v.}{entrevistar | reunir notícias | cobrir (eventos)}
\end{entry}

\begin{entry}{采取}{cai3qu3}{8,8}[HSK 3]
  \definition{v.}{adotar | reunir | coletar | tomar | assumir}
\end{entry}

\begin{entry}{采用}{cai3 yong4}{8,5}[HSK 3]
  \definition{v.}{selecionar e usar | adotar}
\end{entry}

\begin{entry}{彩虹}{cai3hong2}{11,9}
  \definition[道]{s.}{arco-íris}
\end{entry}

\begin{entry}{彩色}{cai3 se4}{11,6}[HSK 3]
  \definition{s.}{multicolorido; cor}
\end{entry}

\begin{entry}{菜}{cai4}{11}[Radical 艸][HSK 1]
  \definition[棵]{s.}{hortaliça | verdura | legume}
  \definition[样,道,盘]{s.}{prato (tipo de alimento) | o tipo (de alguém) | (características de alguém, etc.) fraco, pobre}
\end{entry}

\begin{entry}{菜单}{cai4dan1}{11,8}[HSK 2]
  \definition[份,张]{s.}{menu | cardápio}
\end{entry}

\begin{entry}{菜刀}{cai4dao1}{11,2}
  \definition[把]{s.}{faca de vegetais | faca de cozinha | cutelo}
\end{entry}

\begin{entry}{参观}{can1guan1}{8,6}[HSK 2]
  \definition{v.}{visitar}
\end{entry}

\begin{entry}{参加}{can1jia1}{8,5}[HSK 2]
  \definition{v.}{participar de | tomar parte em | assistir}
\end{entry}

\begin{entry}{餐厅}{can1ting1}{16,4}
  \definition[家]{s.}{restaurante}
  \definition[间]{s.}{sala de jantar}
\end{entry}

\begin{entry}{残疾人}{can2ji2ren2}{9,10,2}
  \definition{s.}{pessoa com deficiência}
\end{entry}

\begin{entry}{残酷}{can2ku4}{9,14}
  \definition{adj.}{cruel}
  \definition{s.}{crueldade}
\end{entry}

\begin{entry}{蚕纸}{can2zhi3}{10,7}
  \definition{s.}{papel onde o bicho-da-seda põe seus ovos}
\end{entry}

\begin{entry}{惨}{can3}{11}[Radical 心]
  \definition{adj.}{miserável | cruel | desumano | desastroso | trágico | sombrio}
\end{entry}

\begin{entry}{舱}{cang1}{10}[Radical ⾈]
  \definition{s.}{cabine | porão (de carga) de um navio ou avião}
\end{entry}

\begin{entry}{操心}{cao1xin1}{16,4}
  \definition{v.+compl.}{preocupar-se com}
\end{entry}

\begin{entry}{操作}{cao1zuo4}{16,7}
  \definition{s.}{operação}
  \definition{v.}{trabalhar | operar | manipular}
\end{entry}

\begin{entry}{槽}{cao2}{15}[Radical ⽊]
  \definition{s.}{calha | canal | sulco | manjedoura}
\end{entry}

\begin{entry}{草}{cao3}{9}[Radical 艸][HSK 2]
  \definition[棵,撮,株,根]{s.}{erva | grama}
\end{entry}

\begin{entry}{草地}{cao3 di4}{9,6}[HSK 2]
  \definition[片]{s.}{relva | pastagem}
\end{entry}

\begin{entry}{草莓}{cao3mei2}{9,10}
  \definition[颗]{s.}{morango}
\end{entry}

\begin{entry}{草纸}{cao3zhi3}{9,7}
  \definition{s.}{papel pardo | pergaminho | papel de palha áspero | papel higiênico}
\end{entry}

\begin{entry}{肏}{cao4}{8}[Radical 肉]
  \definition{v.}{(vulgar) foder}
\end{entry}

\begin{entry}{厕所}{ce4suo3}{8,8}
  \definition[间,处]{s.}{lavatório | \emph{toilette}}
\end{entry}

\begin{entry}{厕纸}{ce4zhi3}{8,7}
  \definition{s.}{papel higiênico}
\end{entry}

\begin{entry}{策划}{ce4hua4}{12,6}
  \definition{s.}{planejador | produtor | plano}
  \definition{v.}{esquematizar | engenhar | planejar}
\end{entry}

\begin{entry}{层}{ceng2}{7}[Radical ⼫][HSK 2]
  \definition{clas.}{para andar, piso}
\end{entry}

\begin{entry}{层层}{ceng2ceng2}{7,7}
  \definition{s.}{camada sobre camada}
\end{entry}

\begin{entry}{层次}{ceng2ci4}{7,6}
  \definition{s.}{camada | nível | graduação | arranjo de ideias}
\end{entry}

\begin{entry}{曾经}{ceng2jing1}{12,8}[HSK 3]
  \definition{adv.}{uma vez | antes | costumava | no passado}
\end{entry}

\begin{entry}{插话}{cha1hua4}{12,8}
  \definition{s.}{interrupção | digressão}
  \definition{v.+compl.}{interromper (a fala de alguém)}
\end{entry}

\begin{entry}{插手}{cha1shou3}{12,4}
  \definition{v.+compl.}{envolver-se em | dar uma mão | ter (tomar) uma mão | cutucar o nariz de alguém | intrometer-se}
\end{entry}

\begin{entry}{查}{cha2}{9}[Radical 木][HSK 2]
  \definition{v.}{verificar | examinar | investigar |consultar}
  \seeref{查}{zha1}
\end{entry}

\begin{entry}{茶}{cha2}{9}[Radical 艸][HSK 1]
  \definition[杯,壶]{s.}{chá | pé (planta) de chá}
\end{entry}

\begin{entry}{刹}{cha4}{8}[Radical 刀]
  \definition{s.}{mosteiro, templo ou santuário budista | abreviação de 刹多罗 | sânscrito "ksetra"}
  \seeref{刹多罗}{cha4duo1luo2}
  \seeref{刹}{sha1}
\end{entry}

\begin{entry}{刹多罗}{cha4duo1luo2}{8,6,8}
  \definition*{s.}{Kshatara, sânscrito ``ksetra''}
\end{entry}

\begin{entry}{差}{cha4}{9}[Radical 工][HSK 1]
  \definition{adv.}{ligeiramente | comparativamente | um pouco}
  \definition{s.}{differença | dissimilaridade | engano | equívoco}
\end{entry}

\begin{entry}{差不多}{cha4bu5duo1}{9,4,6}[HSK 2]
  \definition{adj.}{mais ou menos}
  \definition{adv.}{quase perto}
\end{entry}

\begin{entry}{差点儿}{cha4dian3r5}{9,9,2}
  \definition{adv.}{por pouco | por um triz | quase}
\end{entry}

\begin{entry}{拆}{chai1}{8}[Radical 手]
  \definition{v.}{remover | tirar do seu lugar | desfazer | desmontar}
\end{entry}

\begin{entry}{单}{chan2}{8}[Radical 十]
  \definition{s.}{usado em 单于 \dpy{chan2yu2}}
  \seeref{单于}{chan2yu2}
  \seeref{单}{dan1}
  \seeref{单}{shan4}
\end{entry}

\begin{entry}{单于}{chan2yu2}{8,3}
  \definition{s.}{rei de Xiongnu (匈奴)}
  \seealsoref{匈奴}{xiong1nu2}
\end{entry}

\begin{entry}{禅}{chan2}{12}[Radical 示]
  \definition*{s.}{Zen}
  \definition{s.}{meditação (Budismo)}
  \seeref{禅}{shan4}
\end{entry}

\begin{entry}{蝉}{chan2}{14}[Radical 虫]
  \definition{s.}{cigarra}
\end{entry}

\begin{entry}{产后}{chan3hou4}{6,6}
  \definition{s.}{pós-parto}
\end{entry}

\begin{entry}{产生}{chan3sheng1}{6,5}[HSK 3]
  \definition{v.}{produzir; evoluir; emergir; provocar; vir a ser; dar origem a}
\end{entry}

\begin{entry}{铲车}{chan3che1}{11,4}
  \definition[台]{s.}{empilhadeira}
\end{entry}

\begin{entry}{长}{chang2}{4}[Radical 長][Kangxi 168][HSK 2]
  \definition{adj.}{comprido | longo}
  \seeref{长}{zhang3}
\end{entry}

\begin{entry}{长城}{chang2cheng2}{4,9}[HSK 3]
  \definition*{s.}{A Grande Muralha}
\end{entry}

\begin{entry}{长处}{chang2 chu4}{4,5}[HSK 3]
  \definition{s.}{força; boas qualidades; pontos fortes}
\end{entry}

\begin{entry}{长颈鹿}{chang2jing3lu4}{4,11,11}
  \definition[只]{s.}{girafa}
\end{entry}

\begin{entry}{长期}{chang2 qi1}{4,12}[HSK 3]
  \definition{adj.}{secular; longo prazo; longo alcance; durante um longo período de tempo}
  \definition{s.}{longo prazo}
\end{entry}

\begin{entry}{常}{chang2}{11}[Radical 巾][HSK 1]
  \definition*{s.}{sobrenome Chang}
  \definition{adv.}{muitas vezes | frequentemente}
\end{entry}

\begin{entry}{常常}{chang2chang2}{11,11}[HSK 1]
  \definition{adv.}{frequentemente | com frequência}
\end{entry}

\begin{entry}{常见}{chang2 jian4}{11,4}[HSK 2]
  \definition{adj.}{comum}
\end{entry}

\begin{entry}{常问问题}{chang2wen4wen4ti2}{11,6,6,15}
  \definition{s.}{FAQ; perguntas frequentes}
\end{entry}

\begin{entry}{常用}{chang2 yong4}{11,5}[HSK 2]
  \definition{adj.}{em uso comum}
\end{entry}

\begin{entry}{厂}{chang3}{2}[Radical 厂][HSK 3]
  \definition[家]{s.}{fábrica; moinho; planta; obra | pátio; depósito}
  \seeref{厂}{han3}
\end{entry}

\begin{entry}{场}{chang3}{6}[Radical ⼟][HSK 2]
  \definition{clas.}{para número de exames | para atividades esportivas ou recreativas}
  \definition{s.}{local grande usado para um propósito específico | cena (de uma peça) | palco}
\end{entry}

\begin{entry}{场合}{chang3he2}{6,6}[HSK 3]
  \definition[种]{s.}{ocasião; situação}
\end{entry}

\begin{entry}{场景}{chang3jing3}{6,12}
  \definition{s.}{cena | cenário | situação | contexto}
\end{entry}

\begin{entry}{场面}{chang3mian4}{6,9}
  \definition{s.}{cena | espetáculo | ocasião | situação}
\end{entry}

\begin{entry}{场所}{chang3suo3}{6,8}[HSK 3]
  \definition{s.}{lugar; sítio; arena}
\end{entry}

\begin{entry}{唱}{chang4}{11}[Radical ⼝][HSK 1]
  \definition{v.}{cantar}
\end{entry}

\begin{entry}{唱歌}{chang4ge1}{11,14}[HSK 1]
  \definition{v.+compl.}{cantar}
\end{entry}

\begin{entry}{超过}{chao1guo4}{12,6}[HSK 2]
  \definition{v.}{passar | ultrapassar (alguém ou algo) | exceder | ser mais do que | estar acima de (um padrão)}
\end{entry}

\begin{entry}{超级}{chao1ji2}{12,6}[HSK 3]
  \definition{adj.}{super}
  \definition{pref.}{``super'' | ``ultra'' | ``hiper''}
\end{entry}

\begin{entry}{超声}{chao1sheng1}{12,7}
  \definition{adj.}{ultrasônico}
  \definition{s.}{ultrasom}
\end{entry}

\begin{entry}{超市}{chao1shi4}{12,5}[HSK 2]
  \definition[家]{s.}{supermercado}
\end{entry}

\begin{entry}{巢}{chao2}{11}[Radical ⼮]
  \definition*{s.}{sobrenome Chao}
  \definition{s.}{ninho (de aves, etc.)}
\end{entry}

\begin{entry}{朝}{chao2}{12}[Radical ⽉][HSK 3]
  \definition*{s.}{sobrenome Chao}
  \definition{prep.}{para; em direção a}
  \definition{s.}{tribunal; governo | dinastia | o reino de um imperador}
  \definition{v.}{ter uma audiência com (um rei, um imperador, etc.); fazer uma peregrinação a | encarar; olhar}
  \seeref{朝}{zhao1}
\end{entry}

\begin{entry}{朝廷}{chao2ting2}{12,6}
  \definition{s.}{corte imperial | dinastia}
\end{entry}

\begin{entry}{朝鲜}{chao2xian3}{12,14}
  \definition*{s.}{Coréia do Norte}
\end{entry}

\begin{entry}{潮流}{chao2liu2}{15,10}
  \definition{s.}{tendência | onda | corrente}
\end{entry}

\begin{entry}{吵}{chao3}{7}[Radical ⼝][HSK 3]
  \definition{adj.}{barulhento; ruidoso}
  \definition{v.}{perturbar fazendo barulho; fazer barulho | discutir; brigar; disputar}
\end{entry}

\begin{entry}{吵架}{chao3jia4}{7,9}[HSK 3]
  \definition{v.+compl.}{brigar; discutir; ter uma briga}
\end{entry}

\begin{entry}{炒}{chao3}{8}[Radical ⽕]
  \definition{v.}{saltear | demitir (alguém)}
\end{entry}

\begin{entry}{车}{che1}{4}[Radical 車][Kangxi 159][HSK 1]
  \definition*{s.}{sobrenome Che}
  \definition[辆]{s.}{carro | veículo | viatura}
  \seeref{车}{ju1}
\end{entry}

\begin{entry}{车次}{che1ci4}{4,6}
  \definition{s.}{número do trem}
\end{entry}

\begin{entry}{车库}{che1ku4}{4,7}
  \definition{s.}{garagem}
\end{entry}

\begin{entry}{车辆}{che1 liang4}{4,11}[HSK 2]
  \definition{s.}{veículo | carro}
\end{entry}

\begin{entry}{车牌}{che1pai2}{4,12}
  \definition{s.}{matrícula | placa de carro}
\end{entry}

\begin{entry}{车票}{che1piao4}{4,11}[HSK 1]
  \definition{s.}{bilhete (de ônibus, trem, metrô)}
\end{entry}

\begin{entry}{车上}{che1 shang5}{4,3}[HSK 1]
  \definition{adv.}{no carro | dentro do veículo}
\end{entry}

\begin{entry}{车水马龙}{che1shui3-ma3long2}{4,4,3,5}
  \definition{expr.}{tráfego engarrafado | engarrafamento | (literalmente) ``fluxo interminável de cavalos e carruagens''}
\end{entry}

\begin{entry}{车站}{che1zhan4}{4,10}[HSK 1]
  \definition[处,个]{s.}{estação | ponto de ônibus}
\end{entry}

\begin{entry}{车主}{che1zhu3}{4,5}
  \definition{s.}{proprietário do carro}
\end{entry}

\begin{entry}{车子}{che1zi5}{4,3}
  \definition{s.}{qualquer veículo (carro, bicicleta, caminhão, etc)}
\end{entry}

\begin{entry}{撤}{che4}{15}[Radical 手]
  \definition{v.}{remover, tirar}
\end{entry}

\begin{entry}{沉}{chen2}{7}[Radical 水]
  \definition{adj.}{profundo}
  \definition{v.}{submergir | imergir | mergulhar | afundar}
\end{entry}

\begin{entry}{沉默}{chen2mo4}{7,16}
  \definition{adj.}{taciturno | não comunicativo | silencioso}
\end{entry}

\begin{entry}{衬衫}{chen4shan1}{8,8}[HSK 3]
  \definition[件]{s.}{camisa; blusa}
\end{entry}

\begin{entry}{衬衣}{chen4 yi1}{8,6}[HSK 3]
  \definition[件]{s.}{camisa}
\end{entry}

\begin{entry}{称}{chen4}{10}[Radical 禾]
  \definition{v.}{ajustar | combinar}
  \seeref{称}{cheng1}
\end{entry}

\begin{entry}{称}{cheng1}{10}[Radical 禾][HSK 2]
  \definition*{s.}{sobrenome Cheng}
  \definition{s.}{nome}
  \definition{v.}{chamar | dizer | elogiar | louvar | pesar | levantar | começar}
  \seeref{称}{chen4}
\end{entry}

\begin{entry}{称为}{cheng1 wei2}{10,4}[HSK 3]
  \definition{v.}{chamar; ser chamado; ser conhecido como}
\end{entry}

\begin{entry}{成}{cheng2}{6}[Radical ⼽][HSK 2]
  \definition*{s.}{sobrenome Cheng}
  \definition{v.}{sair-se bem | ser bem sucedido}
\end{entry}

\begin{entry}{成都}{cheng2du1}{6,10}
  \definition*{s.}{Chengdu}
\end{entry}

\begin{entry}{成功}{cheng2gong1}{6,5}[HSK 3]
  \definition{adj.}{bem-sucedido | frutífero}
  \definition[个,次]{s.}{sucesso}
  \definition{v.}{ter sucesso}
\end{entry}

\begin{entry}{成果}{cheng2guo3}{6,8}[HSK 3]
  \definition{s.}{realização; resultado}
\end{entry}

\begin{entry}{成婚}{cheng2hun1}{6,11}
  \definition{v.}{casar-se}
\end{entry}

\begin{entry}{成活}{cheng2huo2}{6,9}
  \definition{v.}{sobreviver}
\end{entry}

\begin{entry}{成吉思汗}{cheng2ji2si1han2}{6,6,9,6}
  \definition*{s.}{Genghis Khan (1162-1227), fundador e governante do Império Mongol}
\end{entry}

\begin{entry}{成绩}{cheng2ji4}{6,11}[HSK 2]
  \definition[项,个]{s.}{nota | classificação}
\end{entry}

\begin{entry}{成家}{cheng2jia1}{6,10}
  \definition{v.}{tornar-se um especialista reconhecido | estabelecer-se e casar-se (de um homem)}
\end{entry}

\begin{entry}{成就}{cheng2jiu4}{6,12}[HSK 3]
  \definition[个]{s.}{realização; sucesso}
  \definition{v.}{realizar; atingir; completar}
\end{entry}

\begin{entry}{成立}{cheng2li4}{6,5}[HSK 3]
  \definition{v.}{fundar; estabelecer; montar | ser válido; ser sustentável; reter água}
\end{entry}

\begin{entry}{成批}{cheng2pi1}{6,7}
  \definition{s.}{em lotes | a granel}
\end{entry}

\begin{entry}{成器}{cheng2qi4}{6,16}
  \definition{v.}{tornar-se uma pessoa digna de respeito | fazer algo de si mesmo}
\end{entry}

\begin{entry}{成色}{cheng2se4}{6,6}
  \definition{v.}{sair-se bem | ser bem sucedido}
\end{entry}

\begin{entry}{成熟}{cheng2shu2}{6,15}[HSK 3]
  \definition{adj./s.}{maduro; totalmente crescido}
  \definition{v.}{amadurecer; estar maduro; estar totalmente crescido}
\end{entry}

\begin{entry}{成为}{cheng2wei2}{6,4}[HSK 2]
  \definition{s.}{tornar-se | transformar-se em}
\end{entry}

\begin{entry}{成员}{cheng2yuan2}{6,7}[HSK 3]
  \definition[个]{s.}{membro}
\end{entry}

\begin{entry}{成长}{cheng2zhang3}{6,4}[HSK 3]
  \definition{v.}{crescer; amadurecer; amadurar}
\end{entry}

\begin{entry}{承认}{cheng2ren4}{8,4}
  \definition{s.}{reconhecimento (diplomático, artístico, etc.)}
  \definition{v.}{admitir | conceder | reconhecer}
\end{entry}

\begin{entry}{诚实}{cheng2shi2}{8,8}
  \definition{adj.}{honesto}
\end{entry}

\begin{entry}{诚实地}{cheng2shi2 di4}{8,8,6}
  \definition{adv.}{honestamente}
\end{entry}

\begin{entry}{城}{cheng2}{9}[Radical 土][HSK 3]
  \definition*{s.}{sobrenome Cheng}
  \definition[座,道,个]{s.}{muralha da cidade; muro | cidade}
\end{entry}

\begin{entry}{城堡}{cheng2bao3}{9,12}
  \definition*{s.}{castelo | torre (peça de xadrez)}
\end{entry}

\begin{entry}{城度}{cheng2du4}{9,9}[HSK 3]
  \definition*{s.}{Cidade}
\end{entry}

\begin{entry}{城市}{cheng2shi4}{9,5}[HSK 3]
  \definition[个,座]{s.}{cidade}
\end{entry}

\begin{entry}{乘客}{cheng2ke4}{10,9}
  \definition{s.}{passageiro}
\end{entry}

\begin{entry}{乘客数}{cheng2ke4 shu4}{10,9,13}
  \definition{s.}{número de passageiros}
\end{entry}

\begin{entry}{惩处}{cheng2chu3}{12,5}
  \definition{v.}{administrar justiça | punir}
\end{entry}

\begin{entry}{惩罚}{cheng2fa2}{12,9}
  \definition{v.}{punir | penalizar}
\end{entry}

\begin{entry}{程控}{cheng2kong4}{12,11}
  \definition{s.}{programado | sob controle automático}
\end{entry}

\begin{entry}{程序}{cheng2xu4}{12,7}
  \definition{s.}{procedimento | sequência | ordem | programa de computador}
\end{entry}

\begin{entry}{程序库}{cheng2xu4ku4}{12,7,7}
  \definition{s.}{biblioteca de funções e procedimentos para programas de computador}
\end{entry}

\begin{entry}{程序设计}{cheng2xu4she4ji4}{12,7,6,4}
  \definition{s.}{programação de computadores}
\end{entry}

\begin{entry}{橙色}{cheng2 se4}{16,6}
  \definition{s.}{cor de laranja}
\end{entry}

\begin{entry}{橙汁}{cheng2zhi1}{16,5}
  \definition[瓶,杯,罐,盒]{s.}{suco de laranja}
  \seealsoref{橘子汁}{ju2zi5zhi1}
  \seealsoref{柳橙汁}{liu3cheng2zhi1}
\end{entry}

\begin{entry}{吃}{chi1}{6}[Radical ⼝][HSK 1]
  \definition{v.}{comer | consumir | comer em (uma cafeteria, etc.) | erradicar | destruir | absorver}
\end{entry}

\begin{entry}{吃饭}{chi1fan4}{6,7}[HSK 1]
  \definition{v.+compl.}{comer | ter (comer) uma refeição | manter vivo | ganhar a vida}
\end{entry}

\begin{entry}{吃屎}{chi1 shi3}{6,9}
  \definition{expr.}{Coma merda!}
\end{entry}

\begin{entry}{池}{chi2}{6}[Radical 水]
  \definition*{s.}{sobrenome Chi}
  \definition{s.}{lagoa | reservatório | fosso}
\end{entry}

\begin{entry}{迟到}{chi2dao4}{7,8}
  \definition{v.}{chegar atrasado | tardar}
\end{entry}

\begin{entry}{持续}{chi2xu4}{9,11}[HSK 3]
  \definition{v.}{durar; continuar; sustentar}
\end{entry}

\begin{entry}{斥骂}{chi4ma4}{5,9}
  \definition{v.}{repreender}
\end{entry}

\begin{entry}{充满}{chong1man3}{6,13}[HSK 3]
  \definition{v.}{preencher | encher-se de; transbordar de; permear-se de}
\end{entry}

\begin{entry}{冲}{chong1}{6}[Radical ⼎]
  \definition{s.}{via pública}
  \definition{v.}{(água) correr contra | misturar com água | infundir | enxaguar | dar a descarga | revelar (um filme) | subir no ar | chocar-se | colidir com | ir em frente | apressar-se}
  \seeref{冲}{chong4}
\end{entry}

\begin{entry}{冲锋}{chong1feng1}{6,12}
  \definition{v.}{cobrar | tomar de assalto}
\end{entry}

\begin{entry}{冲浪}{chong1lang4}{6,10}
  \definition{s.}{surfe}
  \definition{v.}{surfar}
\end{entry}

\begin{entry}{冲突}{chong1tu1}{6,9}
  \definition{s.}{conflito | choque de forças opostas | colisão (de interesses)}
\end{entry}

\begin{entry}{憧憬}{chong1jing3}{15,15}
  \definition{v.}{ansiar por | esperar por}
\end{entry}

\begin{entry}{重}{chong2}{9}[Radical ⾥]
  \definition*{s.}{sobrenome Chong}
  \definition{adv.}{novamente; mais uma vez}
  \definition{clas.}{para camadas}
  \definition{v.}{repetir; duplicar}
  \seeref{重}{zhong4}
\end{entry}

\begin{entry}{重重}{chong2chong2}{9,9}
  \definition{adv.}{camada após camada | um após o outro}
  \seeref{重重}{zhong4zhong4}
\end{entry}

\begin{entry}{重点}{chong2dian3}{9,9}
  \definition{adj./adv./s.}{nota-chave | ponto-chave | ponto focal | ênfase}
  \seeref{重点}{zhong4dian3}
\end{entry}

\begin{entry}{重逢}{chong2feng2}{9,10}
  \definition{s.}{reunião}
  \definition{v.}{encontrar-se novamente | reunir-se}
\end{entry}

\begin{entry}{重复}{chong2fu4}{9,9}[HSK 2]
  \definition{v.}{repetir | iterar | duplicar | reduplicar | fazer algo de novo}
\end{entry}

\begin{entry}{重新}{chong2xin1}{9,13}[HSK 2]
  \definition{adv.}{de novo | novamente}
\end{entry}

\begin{entry}{重阳节}{chong2yang2jie2}{9,6,5}
  \definition*{s.}{Festa do Duplo Nove, Festival Yang, dia de subir aos lugares mais altos para evitar calamidades e dia do culto aos antepassados (9º dia do nono mês lunar)}
\end{entry}

\begin{entry}{崇}{chong2}{11}[Radical ⼭]
  \definition*{s.}{sobrenome Chong}
  \definition{adj.}{alto | sublime | elevado}
  \definition{v.}{estimar | adorar}
\end{entry}

\begin{entry}{宠物}{chong3wu4}{8,8}
  \definition{s.}{animal de estimação}
\end{entry}

\begin{entry}{冲}{chong4}{6}[Radical ⼎]
  \definition{adj.}{poderoso | vigoroso | pungente}
  \definition{adv.}{em direção | em vista de}
  \seeref{冲}{chong1}
\end{entry}

\begin{entry}{酬劳}{chou2lao2}{13,7}
  \definition{s.}{recompensa}
\end{entry}

\begin{entry}{臭}{chou4}{10}[Radical ⾃]
  \definition{adj.}{fétido | repulsivo | repugnante | malcheiroso}
  \definition{s.}{fedor}
  \definition{v.}{feder}
  \seeref{臭}{xiu4}
\end{entry}

\begin{entry}{臭气}{chou4qi4}{10,4}
  \definition{s.}{fedor}
\end{entry}

\begin{entry}{出}{chu1}{5}[Radical ⼐][HSK 1]
  \definition{clas.}{para dramas, peças, óperas, etc.}
  \definition{v.}{sair | ir para fora | vir para fora}
\end{entry}

\begin{entry}{出版}{chu1ban3}{5,8}
  \definition{v.}{publicar | editar}
\end{entry}

\begin{entry}{出版社}{chu1ban3she4}{5,8,7}
  \definition{s.}{editora}
\end{entry}

\begin{entry}{出差}{chu1chai1}{5,9}
  \definition{v.+compl.}{fazer uma viagem oficial ou de negócios}
\end{entry}

\begin{entry}{出发}{chu1fa1}{5,5}[HSK 2]
  \definition{v.}{partir | começar (uma jornada)}
\end{entry}

\begin{entry}{出国}{chu1 guo2}{5,8}[HSK 2]
  \definition{v.+compl.}{ir para o exterior | deixar a terra natal}
\end{entry}

\begin{entry}{出汗}{chu1han4}{5,6}
  \definition{v.}{transpirar | suar}
\end{entry}

\begin{entry}{出击}{chu1ji1}{5,5}
  \definition{v.}{atacar}
\end{entry}

\begin{entry}{出口}{chu1kou3}{5,3}[HSK 2]
  \definition[个]{s.}{exportação}
  \definition{v.+compl.}{exportar}
\end{entry}

\begin{entry}{出来}{chu1 lai2}{5,7}[HSK 1]
  \definition{v.}{sair | vir para fora (para a minha localização)}
\end{entry}

\begin{entry}{出门}{chu1 men2}{5,3}[HSK 2]
  \definition{v.+compl.}{sair | sair de casa | estar longe de casa | fazer uma viagem | casar}
\end{entry}

\begin{entry}{出去}{chu1 qu4}{5,5}[HSK 1]
  \definition{v.}{sair | ir para fora (a partir da minha localização)}
\end{entry}

\begin{entry}{出生}{chu1sheng1}{5,5}[HSK 2]
  \definition{v.}{nascer}
\end{entry}

\begin{entry}{出现}{chu1xian4}{5,8}[HSK 2]
  \definition{v.}{aparecer | surgir | emergir | crescer}
\end{entry}

\begin{entry}{出行}{chu1xing2}{5,6}
  \definition{v.}{sair para algum lugar (viagem relativamente curta) | partir em uma viagem (viagem mais longa)}
\end{entry}

\begin{entry}{出院}{chu1 yuan4}{5,9}[HSK 2]
  \definition{v.}{deixar o hospital | estar fora do hospital | ter alta do hospital}
\end{entry}

\begin{entry}{出站}{chu1 zhan4}{5,10}
  \definition{s.}{saída da estação}
\end{entry}

\begin{entry}{出租}{chu1 zu1}{5,10}[HSK 2]
  \definition{v.}{alugar | arrendar}
\end{entry}

\begin{entry}{出租车}{chu1zu1che1}{5,10,4}[HSK 2]
  \definition{s.}{táxi}
  \seealsoref{出租汽车}{chu1zu1qi4che1}
\end{entry}

\begin{entry}{出租汽车}{chu1zu1qi4che1}{5,10,7,4}
  \definition[辆]{s.}{táxi}
  \seealsoref{出租车}{chu1zu1che1}
\end{entry}

\begin{entry}{出租司机}{chu1zu1si1ji1}{5,10,5,6}
  \definition{s.}{motorista de táxi}
\end{entry}

\begin{entry}{初}{chu1}{7}[Radical 衣][HSK 3]
  \definition*{s.}{sobrenome Chu}
  \definition{adj.}{primeiro (em ordem) | elementar; rudimentar | original}
  \definition{adv.}{pela primeira vez}
  \definition{pref.}{anexado aos numerais de um a dez para indicar ordem (primeiro ao décimo)}
  \definition{s.}{no início de; na primeira parte de | o estágio júnior (pleno; sênior)}
\end{entry}

\begin{entry}{初步}{chu1bu4}{7,7}[HSK 3]
  \definition{adj.}{inicial; preliminar}
\end{entry}

\begin{entry}{初级}{chu1ji2}{7,6}[HSK 3]
  \definition{adj.}{elementar; primário; júnior; inicial}
\end{entry}

\begin{entry}{初心}{chu1xin1}{7,4}
  \definition{s.}{intenção original de alguém, aspiração, etc. | (budismo) ``mente do iniciante'' (ter a mente aberta quando estudando um assunto como um iniciante no assunto teria)}
\end{entry}

\begin{entry}{初中}{chu1 zhong1}{7,4}[HSK 3]
  \definition[所,个]{s.}{ensino médio; ensino fundamental}
\end{entry}

\begin{entry}{除非}{chu2fei1}{9,8}
  \definition{conj.}{a menos que | somente se}
\end{entry}

\begin{entry}{除了}{chu2le5}{9,2}[HSK 3]
  \definition{prep.}{exceto; à parte | além disso; além de | ou \dots ou \dots}
\end{entry}

\begin{entry}{厨房}{chu2fang2}{12,8}
  \definition[间]{s.}{cozinha}
\end{entry}

\begin{entry}{处}{chu3}{5}[Radical ⼡]
  \definition{v.}{residir | viver | habitar | estar dentro | estar situado em | ficar | se dar bem com | estar em uma posição de | lidar com | disciplinar | punir}
  \seeref{处}{chu4}
\end{entry}

\begin{entry}{处罚}{chu3fa2}{5,9}
  \definition{v.}{penalizar | punir}
\end{entry}

\begin{entry}{处理}{chu3li3}{5,11}[HSK 3]
  \definition{s.}{manuseio; descarte}
  \definition{v.}{lidar com; dispor de | resolver; punir; lidar | vender a preços reduzidos; liquidar | lidar com; processar}
\end{entry}

\begin{entry}{处}{chu4}{5}[Radical ⼡]
  \definition{clas.}{para locais ou itens de danos: lugar, local}
  \definition{s.}{local | localização | lugar | ponto | escritório | departamento}
  \seeref{处}{chu3}
\end{entry}

\begin{entry}{处处}{chu4chu4}{5,5}
  \definition{adv.}{em todos os lugares | em todos os aspectos}
\end{entry}

\begin{entry}{畜}{chu4}{10}[Radical ⽥]
  \definition{s.}{gado | animal domesticado | animal doméstico}
  \seeref{畜}{xu4}
\end{entry}

\begin{entry}{穿}{chuan1}{9}[Radical ⽳][HSK 1]
  \definition{v.}{vestir}
\end{entry}

\begin{entry}{传}{chuan2}{6}[Radical 人][HSK 3]
  \definition{v.}{passar; passar adiante | passar adiante; legar; passar de \dots para \dots | transmitir (conhecimento, habilidade, etc.); comunicar; ensinar | espalhar; propagar | transmitir; conduzir; transferir | transmitir; expressar |convocar | infectar; ser contagioso}
  \seeref{传}{zhuan4}
\end{entry}

\begin{entry}{传播}{chuan2bo1}{6,15}[HSK 3]
  \definition{v.}{espalhar; difundir; propagar; disseminar}
\end{entry}

\begin{entry}{传承}{chuan2cheng2}{6,8}
  \definition{s.}{herança | tradição continuada}
  \definition{v.}{transmitir (para as gerações futuras) | passar adiante (desde os tempos antigos)}
\end{entry}

\begin{entry}{传给}{chuan2gei3}{6,9}
  \definition{v.}{passar para | transferir para | entregar a}
\end{entry}

\begin{entry}{传来}{chuan2 lai2}{6,7}[HSK 3]
  \definition{v.}{(um som) passar | (notícias) chegar}
\end{entry}

\begin{entry}{传说}{chuan2shuo1}{6,9}[HSK 3]
  \definition{s.}{lenda | conto popular | folclore}
  \definition{v.}{dizer que; ser dito; passar de boca em boca}
\end{entry}

\begin{entry}{传统}{chuan2tong3}{6,9}
  \definition{adj.}{tradicional | convencional}
  \definition[个]{s.}{tradição | convenção}
\end{entry}

\begin{entry}{传真}{chuan2zhen1}{6,10}
  \definition{s.}{fax, facsímile}
\end{entry}

\begin{entry}{船}{chuan2}{11}[Radical ⾈][HSK 2]
  \definition[条,艘,只]{s.}{barco | navio}
\end{entry}

\begin{entry}{创作}{chuan4zuo4}{6,7}[HSK 3]
  \definition[个]{s.}{criação; trabalho criativo}
  \definition{v.}{escrever; criar; produzir; compor}
\end{entry}

\begin{entry}{窗帘}{chuang1lian2}{12,8}
  \definition{s.}{cortina}
\end{entry}

\begin{entry}{床}{chuang2}{7}[Radical ⼴][HSK 1]
  \definition{clas.}{para camas}
  \definition[张]{s.}{cama}
\end{entry}

\begin{entry}{创新}{chuang4xin1}{6,13}[HSK 3]
  \definition[个,种,次]{s.}{inovação}
  \definition{v.}{trazer novas ideias; inovar; abrir novos caminhos; criar algo novo}
\end{entry}

\begin{entry}{创业}{chuang4ye4}{6,5}[HSK 3]
  \definition{s.}{empreendedorismo}
  \definition{v.}{começar um empreendimento; iniciar um negócio, uma empresa | esculpir}
\end{entry}

\begin{entry}{创意}{chuang4yi4}{6,13}
  \definition{adj.}{criativo}
  \definition{s.}{criatividade}
\end{entry}

\begin{entry}{创造}{chuang4zao4}{6,10}[HSK 3]
  \definition{s.}{criação; inovação}
  \definition{v.}{criar; produzir; trazer à tona}
\end{entry}

\begin{entry}{吹}{chui1}{7}[Radical 口][HSK 2]
  \definition{v.}{soprar | tocar (instrumentos de sopro) | bajular |  louvar aos céus | separar (casal)  | fracassar}
\end{entry}

\begin{entry}{吹牛}{chui1niu2}{7,4}
  \definition{v.+compl.}{ogulhar-se | gabar-se | destacar-se}
\end{entry}

\begin{entry}{锤}{chui2}{13}[Radical 金]
  \definition{s.}{martelo | marreta}
  \definition{s.}{pesos (por exemplo, de uma balança)}
  \definition{v.}{marterlar para dar forma | atacar com um martelo}
\end{entry}

\begin{entry}{春}{chun1}{9}[Radical 日]
  \definition*{s.}{sobrenome Chun}
  \definition{s.}{primavera | amor | luxúria | vida | vitalidade}
\end{entry}

\begin{entry}{春节}{chun1 jie2}{9,5}[HSK 2]
  \definition*{s.}{Festival da Primavera (Ano Novo Chinês)}
\end{entry}

\begin{entry}{春天}{chun1 tian1}{9,4}
  \definition[个]{s.}{primavera}
\end{entry}

\begin{entry}{纯真}{chun2zhen1}{7,10}
  \definition{adj.}{inocente e não afetado | puro e não adulterado}
  \definition{s.}{inocência}
\end{entry}

\begin{entry}{唇}{chun2}{10}[Radical ⼝]
  \definition{s.}{lábios}
\end{entry}

\begin{entry}{绰号}{chuo4hao4}{11,5}
  \definition{s.}{apelido}
\end{entry}

\begin{entry}{词}{ci2}{7}[Radical 言][HSK 2]
  \definition[个,组]{s.}{discurso | declaração | linhas de jogo | um tipo de poesia clássica chinesa, originária da Dinastia Tang e totalmente desenvolvida na Dinastia Song | palavra  | termo}
\end{entry}

\begin{entry}{词典}{ci2dian3}{7,8}[HSK 2]
  \definition[部,本]{s.}{dicionário}
  \seealsoref{字典}{zi4dian3}
\end{entry}

\begin{entry}{词语}{ci2yu3}{7,9}[HSK 2]
  \definition{s.}{palavra (termo geral, incluindo desdemonossilábicas até frases curtas) | termo (por exemplo, termo técnico) | expressão}
\end{entry}

\begin{entry}{瓷}{ci2}{10}[Radical ⽡]
  \definition{s.}{artigos de porcelana}
\end{entry}

\begin{entry}{辞典}{ci2dian3}{13,8}
  \variantof{词典}
\end{entry}

\begin{entry}{磁带}{ci2dai4}{14,9}
  \definition[盘,盒]{s.}{cassete | fita magnética}
\end{entry}

\begin{entry}{磁盘}{ci2pan2}{14,11}
  \definition{s.}{disquete}
\end{entry}

\begin{entry}{磁铁}{ci2tie3}{14,10}
  \definition{s.}{imã | magneto}
  \seealsoref{吸铁石}{xi1tie3shi2}
\end{entry}

\begin{entry}{次}{ci4}{6}[Radical ⽋][HSK 1]
  \definition{clas.}{para frequência (número de vezes)}
\end{entry}

\begin{entry}{刺}{ci4}{8}[Radical 刀]
  \definition{s.}{espinho | picada}
  \definition{v.}{picar | perfurar | esfaquear | assassinar}
\end{entry}

\begin{entry}{刺猬}{ci4wei5}{8,12}
  \definition{s.}{porco-espinho | ouriço}
\end{entry}

\begin{entry}{匆匆}{cong1cong1}{5,5}
  \definition{adv.}{apressadamente}
\end{entry}

\begin{entry}{葱}{cong1}{12}[Radical 艸]
  \definition{s.}{cebolinha}
\end{entry}

\begin{entry}{聪慧}{cong1hui4}{15,15}
  \definition{adj.}{inteligente | brilhante}
\end{entry}

\begin{entry}{聪明}{cong1ming5}{15,8}
  \definition{adj.}{inteligente | brilhante | esperto}
\end{entry}

\begin{entry}{从}{cong2}{4}[Radical ⼈][HSK 1]
  \definition*{s.}{sobrenome Cong}
  \definition{prep.}{de | desde | a partir de}
\end{entry}

\begin{entry}{从不}{cong2bu4}{4,4}
  \definition{adv.}{nunca}
\end{entry}

\begin{entry}{从而}{cong2'er2}{4,6}
  \definition{conj.}{assim | desse modo}
\end{entry}

\begin{entry}{从来}{cong2lai2}{4,7}[HSK 3]
  \definition{adv.}{sempre; o tempo todo; em todos os momentos}
\end{entry}

\begin{entry}{从前}{cong2qian2}{4,9}[HSK 3]
  \definition{s.}{antes; antigamente; no passado | era uma vez; há muito tempo atrás}
\end{entry}

\begin{entry}{从事}{cong2shi4}{4,8}[HSK 3]
  \definition{v.}{trabalhar; empreender; empenhar-se em; envolver-se em | lidar com; manusear}
\end{entry}

\begin{entry}{从未}{cong2wei4}{4,5}
  \definition{adv.}{nunca}
\end{entry}

\begin{entry}{从小}{cong2 xiao3}{4,3}[HSK 2]
  \definition{adv.}{desde a infância | desde muito jovem | quando criança}
\end{entry}

\begin{entry}{粗糙}{cu1cao1}{11,16}
  \definition{adj.}{áspero | grosseiro}
\end{entry}

\begin{entry}{粗心}{cu1xin1}{11,4}
  \definition{adj.}{descuido}
\end{entry}

\begin{entry}{粗心地做}{cu1xin1 di4 zuo4}{11,4,6,11}
  \definition{adj.}{feito descuidadamente}
\end{entry}

\begin{entry}{酢}{cu4}{12}[Radical 酉]
  \variantof{醋}
\end{entry}

\begin{entry}{醋}{cu4}{15}[Radical ⾣]
  \definition{s.}{vinagre}
\end{entry}

\begin{entry}{窾}{cuan4}{17}[Radical 穴]
  \definition{v.}{esconder}
  \seeref{窾}{kuan3}
\end{entry}

\begin{entry}{村}{cun1}{7}[Radical ⽊][HSK 3]
  \definition{adj.}{rústico; grosseiro}
  \definition{s.}{aldeia; vila}
\end{entry}

\begin{entry}{存}{cun2}{6}[Radical 子][HSK 3]
  \definition{v.}{existir; viver; sobreviver | armazenar; manter | acumular; coletar | depositar | sair com; verificar |reservar; reter | permanecer em equilíbrio; estar em estoque | estimar; abrigar}
\end{entry}

\begin{entry}{存在}{cun2zai4}{6,6}[HSK 3]
  \definition{s.}{existência; ser; ente}
  \definition{v.}{existir; ser}
\end{entry}

\begin{entry}{搓}{cuo1}{12}[Radical 手]
  \definition{s.}{torção}
  \definition{v.}{esfregar ou rolar entre as mãos ou dedos | torcer}
\end{entry}

\begin{entry}{挫折}{cuo4zhe2}{10,7}
  \definition{s.}{revés | reverso | derrota | frustração | decepção}
  \definition{v.}{frustrar | desencorajar | subjugar}
\end{entry}

\begin{entry}{错}{cuo4}{13}[Radical 金][HSK 1]
  \definition*{s.}{sobrenome Cuo}
  \definition{adj.}{errado | enganado}
\end{entry}

\begin{entry}{错误}{cuo4wu4}{13,9}[HSK 3]
  \definition{adj.}{equivocado; errado; errôneo}
  \definition[个,次]{s.}{engano; erro; erro grosseiro; falha}
\end{entry}

%%%%% EOF %%%%%


%%%
%%% D
%%%

\section*{D}\addcontentsline{toc}{section}{D}

\begin{entry}{搭配}{da1pei4}{12,10}[Radicais ⼿、⾣]
  \definition{v.}{emparelhar | combinar | organizar em pares | adicionar alguém em um grupo}
\end{entry}

\begin{entry}{搭讪}{da1shan4}{12,5}[Radicais ⼿、⾔]
  \definition{v.}{bater em alguém | incitar uma conversa | começar a conversar para acabar com um silêncio constrangedor ou uma situação embaraçosa}
\end{entry}

\begin{entry}{答应}{da1ying5}{12,7}[HSK 2][Radicais ⽵、⼴]
  \definition{v.}{responder | concordar | prometer | cumprir com}
\end{entry}

\begin{entry}{打}{da2}{5}[Radical ⼿]
  \definition{clas./s.}{(empréstimo linguístico) dúzia}
  \seeref{打}{da3}
\end{entry}

\begin{entry}{达到}{da2dao4}{6,8}[HSK 3][Radicais ⾡、⼑]
  \definition{v.}{alcançar; atingir; atender o padrão}
\end{entry}

\begin{entry}{答案}{da2'an4}{12,10}[HSK 4][Radicais ⽵、⽊]
  \definition[个]{s.}{chave; resposta; solução}
\end{entry}

\begin{entry}{打}{da3}{5}[HSK 1,4][Radical ⼿]
  \definition{prep.}{de; desde; ponto de partida que indica lugar, tempo ou extensão | indica rotas e locais percorridos}
  \definition{v.}{golpear; acertar; bater | quebrar; esmagar | lutar; atacar; espancar | entrar com uma ação judicial; negociar; fazer representações | construir; edificar | fabricar (em uma ferraria); forjar | misturar; mexer; bater | amarrar; embalar | tricotar; tecer | desenhar; pintar; deixar uma marca; imprimir | abrir; perfurar; cavar | içar; levantar
enviar; despachar; projetar | emitir ou receber (um certificado, etc.) | remover; livrar-se de}
  \seeref{打}{da2}
\end{entry}

\begin{entry}{打败}{da3 bai4}{5,8}[HSK 4][Radicais ⼿、⾒]
  \definition{v.}{derrotar; vencer; piorar | sofrer uma derrota; ser derrotado}
\end{entry}

\begin{entry}{打扮}{da3ban5}{5,7}[Radicais ⼿、⼿]
  \definition{v.}{arranjar-se | enfeitar-se}
\end{entry}

\begin{entry}{打车}{da3 che1}{5,4}[HSK 1][Radicais ⼿、⾞]
  \definition{v.}{pegar um táxi | chamar um táxi}
\end{entry}

\begin{entry}{打的}{da3di1}{5,8}[Radicais ⼿、⽩]
  \definition{v.+compl.}{(coloquial) pegar um táxi | ir de táxi}
\end{entry}

\begin{entry}{打电话}{da3 dian4 hua4}{5,5,8}[HSK 1][Radicais ⼿、⽥、⾔]
  \definition{v.}{telefonar | fazer uma chamada telefônica | dar um telefonema}
  \seealsoref{给……打电话}{gei3 da3 dian4 hua4}
\end{entry}

\begin{entry}{打工}{da3gong1}{5,3}[HSK 2][Radicais ⼿、⼯]
  \definition{v.}{(para alunos) ter um emprego fora do horário de aula ou durante as férias | trabalhar em um emprego temporá rio ou casual}
\end{entry}

\begin{entry}{打工人}{da3gong1ren2}{5,3,2}[Radicais ⼿、⼯、⼈]
  \definition{s.}{trabalhador}
\end{entry}

\begin{entry}{打架}{da3jia4}{5,9}[Radicais ⼿、⽊]
  \definition{v.+compl.}{lutar | brigar | participar de lutas, brigas}
\end{entry}

\begin{entry}{打搅}{da3jiao3}{5,12}[Radicais ⼿、⼿]
  \definition{v.}{perturbar | incomodar}
\end{entry}

\begin{entry}{打结}{da3jie2}{5,9}[Radicais ⼿、⽷]
  \definition{v.}{dar um nó | amarrar}
\end{entry}

\begin{entry}{打开}{da3 kai1}{5,4}[HSK 1][Radicais ⼿、⼶]
  \definition{v.}{abrir | desdobrar | ligar | avançar | espalhar}
\end{entry}

\begin{entry}{打瞌睡}{da3ke1shui4}{5,15,13}[Radicais ⼿、⽬、⽬]
  \definition{v.}{cochilar}
\end{entry}

\begin{entry}{打雷}{da3 lei2}{5,13}[HSK 4][Radicais ⼿、⾬]
  \definition{v.}{trovejar; produzir ruídos altos quando as nuvens descarregam eletricidade}
\end{entry}

\begin{entry}{打猎}{da3lie4}{5,11}[Radicais ⼿、⽝]
  \definition{v.}{ir caçar}
\end{entry}

\begin{entry}{打骂}{da3ma4}{5,9}[Radicais ⼿、⾺]
  \definition{v.}{bater e repreender}
\end{entry}

\begin{entry}{打磨}{da3mo2}{5,16}[Radicais ⼿、⽯]
  \definition{v.}{polir | fazer brilhar}
\end{entry}

\begin{entry}{打屁股}{da3pi4gu5}{5,7,8}[Radicais ⼿、⼫、⾁]
  \definition{v.}{dar um tapa no bumbum de alguém}
\end{entry}

\begin{entry}{打破}{da3 po4}{5,10}[HSK 3][Radicais ⼿、⽯]
  \definition{v.}{quebrar; esmagar}
\end{entry}

\begin{entry}{打球}{da3 qiu2}{5,11}[HSK 1][Radicais ⼿、⽟]
  \definition{v.}{jogar bola (com as mãos) | jogar (basquetebol, handbol, etc.)}
\end{entry}

\begin{entry}{打扰}{da3rao3}{5,7}[Radicais ⼿、⼿]
  \definition{v.}{perturbar | incomodar}
\end{entry}

\begin{entry}{打扫}{da3sao3}{5,6}[HSK 4][Radicais ⼿、⼿]
  \definition{v.}{varrer; limpar; varrer para limpar}
\end{entry}

\begin{entry}{打算}{da3suan4}{5,14}[HSK 2][Radicais ⼿、⽵]
  \definition[个]{s.}{plano | intenção}
  \definition{v.}{pensar | planejar | pretender}
\end{entry}

\begin{entry}{打听}{da3ting5}{5,7}[HSK 3][Radicais ⼿、⼝]
  \definition{v.}{perguntar sobre; indagar sobre; obter uma linha sobre}
\end{entry}

\begin{entry}{打压}{da3ya1}{5,6}[Radicais ⼿、⼚]
  \definition{v.}{reprimir | derrotar}
\end{entry}

\begin{entry}{打印}{da3yin4}{5,5}[HSK 2][Radicais ⼿、⼙]
  \definition{v.}{imprimir}
\end{entry}

\begin{entry}{打折}{da3zhe2}{5,7}[HSK 4][Radicais ⼿、⼿]
  \definition{v.+compl.}{dar desconto; dar um desconto; vender produtos a um preço reduzido em uma determinada porcentagem do preço original; metáfora para não cumprir 100\% do que foi originalmente padronizado ou prometido}
\end{entry}

\begin{entry}{打针}{da3zhen1}{5,7}[HSK 4][Radicais ⼿、⾦]
  \definition{v.+compl.}{dar ou receber uma injeção; injetar um medicamento líquido em um organismo com uma seringa}
\end{entry}

\begin{entry}{大}{da4}{3}[HSK 1][Kangxi 37][Radical ⼤]
  \definition{adj.}{grande | enorme | maior | largo | profundo | mais velho (que) | mais antigo | mais velho | muito}
  \definition{s.}{(dialeto) pai | irmão mais velho ou mais novo do pai}
  \seeref{大}{dai4}
\end{entry}

\begin{entry}{大巴}{da4 ba1}{3,4}[HSK 4][Radicais ⼤、⼰]
  \definition{s.}{ônibus}
\end{entry}

\begin{entry}{大部分}{da4 bu4 fen4}{3,10,4}[HSK 2][Radicais ⼤、⾢、⼑]
  \definition{s.}{a maioria | a maior parte}
\end{entry}

\begin{entry}{大大}{da4 da4}{3,3}[HSK 2][Radicais ⼤、⼤]
  \definition{adv.}{muito; enormemente}
\end{entry}

\begin{entry}{大胆}{da4dan3}{3,9}[Radicais ⼤、⾁]
  \definition{adj.}{audacioso | ousado | destemido}
\end{entry}

\begin{entry}{大豆}{da4dou4}{3,7}[Radicais ⼤、⾖]
  \definition{s.}{soja}
\end{entry}

\begin{entry}{大多}{da4 duo1}{3,6}[HSK 4][Radicais ⼤、⼣]
  \definition{adv.}{majoritariamente; em sua maior parte; em sua maioria; em grande parte}
\end{entry}

\begin{entry}{大多数}{da4 duo1 shu4}{3,6,13}[HSK 2][Radicais ⼤、⼣、⽁]
  \definition{s.}{a grande maioria | a vasta maioria | a maior parte}
\end{entry}

\begin{entry}{大方}{da4fang5}{3,4}[HSK 4][Radicais ⼤、⽅]
  \definition{adj.}{generoso | não afetado; natural e equilibrado |  de bom gosto}
\end{entry}

\begin{entry}{大夫}{da4fu1}{3,4}[Radicais ⼤、⼤]
  \definition{s.}{oficial sênior (na China Imperial)}
  \seeref{大夫}{dai4fu5}
\end{entry}

\begin{entry}{大概}{da4gai4}{3,13}[HSK 3][Radicais ⼤、⽊]
  \definition{adj.}{geral; grosseiro; aproximado}
  \definition{adv.}{sobre; provavelmente
geralmente; brevemente}
  \definition{s.}{ideia geral; esboço geral}
\end{entry}

\begin{entry}{大哥}{da4 ge1}{3,10}[HSK 4][Radicais ⼤、⼝]
  \definition{s.}{irmão mais velho | \emph{big brother} (endereço educado para um homem da mesma idade que você) | líder de gangue; pessoa mais poderosa em uma organização que realiza atividades ilegais na sociedade}
\end{entry}

\begin{entry}{大规模}{da4 gui1 mo2}{3,8,14}[HSK 4][Radicais ⼤、⾒、⽊]
  \definition{adj.}{em larga escala; extensivo; maciço; massa}
  \definition{adj.}{em larga escala; extensivo; maciço; massivo}
\end{entry}

\begin{entry}{大海}{da4 hai3}{3,10}[HSK 2][Radicais ⼤、⽔]
  \definition{s.}{mar | oceano}
\end{entry}

\begin{entry}{大后天}{da4 hou4 tian1}{3,6,4}[Radicais ⼤、⼝、⼤]
  \definition{adv.}{daqui a três dias}
\end{entry}

\begin{entry}{大会}{da4 hui4}{3,6}[HSK 4][Radicais ⼤、⼈]
  \definition{s.}{sessão plenária; reunião geral de membros; reuniões convocadas por partidos políticos socialistas | reunião de massa; comício de massa}
\end{entry}

\begin{entry}{大家}{da4jia1}{3,10}[HSK 2][Radicais ⼤、⼧]
  \definition{pron.}{todos}
\end{entry}

\begin{entry}{大姐}{da4 jie3}{3,8}[HSK 4][Radicais ⼤、⼥]
  \definition[个]{s.}{irmã mais velha (também um termo educado para se dirigir a uma garota ou mulher um pouco mais velha do que a pessoa que fala)}
\end{entry}

\begin{entry}{大口}{da4kou3}{3,3}[Radicais ⼤、⼝]
  \definition{s.}{grande bocado (de comida, bebida, fumo, etc.)}
\end{entry}

\begin{entry}{大量}{da4 liang4}{3,12}[HSK 2][Radicais ⼤、⾥]
  \definition{adj.}{numeroso | em massa | grande em número ou quantidade | generoso | magnânimo}
\end{entry}

\begin{entry}{大楼}{da4 lou2}{3,13}[HSK 4][Radicais ⼤、⽊]
  \definition[座,幢]{s.}{edifício; mansão; edifício de vários andares disponível para uso residencial e comercial}
\end{entry}

\begin{entry}{大陆}{da4 lu4}{3,7}[HSK 4][Radicais ⼤、⾩]
  \definition*{s.}{China continental; refere-se especificamente à vasta porção terrestre do território da China}
  \definition[个,块]{s.}{terra firme; continente; vasta extensão de terra}
\end{entry}

\begin{entry}{大妈}{da4 ma1}{3,6}[HSK 4][Radicais ⼤、⼥]
  \definition{s.}{tia; esposa do irmão mais velho do pai | tia (homenagem às mulheres idosas)}
\end{entry}

\begin{entry}{大马}{da4ma3}{3,3}[Radicais ⼤、⾺]
  \definition*{s.}{Malásia}
\end{entry}

\begin{entry}{大门}{da4 men2}{3,3}[HSK 2][Radicais ⼤、⾨]
  \definition{s.}{portão | entrada}
\end{entry}

\begin{entry}{大前天}{da4qian2tian1}{3,9,4}[Radicais ⼤、⼑、⼤]
  \definition{adv.}{três dias atrás}
\end{entry}

\begin{entry}{大全}{da4quan2}{3,6}[Radicais ⼤、⼊]
  \definition{s.}{coleção abrangente}
\end{entry}

\begin{entry}{大人}{da4 ren2}{3,2}[HSK 2][Radicais ⼤、⼈]
  \definition{s.}{adulto}
\end{entry}

\begin{entry}{大赛}{da4sai4}{3,14}[Radicais ⼤、⾙]
  \definition{s.}{grande concurso, competição}
\end{entry}

\begin{entry}{大神}{da4shen2}{3,9}[Radicais ⼤、⽰]
  \definition{s.}{deidade | (gíria da Internet) guru | \emph{expert} | gênio}
\end{entry}

\begin{entry}{大声}{da4 sheng1}{3,7}[HSK 2][Radicais ⼤、⼠]
  \definition{adj.}{alto volume | em voz alta}
\end{entry}

\begin{entry}{大使馆}{da4shi3guan3}{3,8,11}[HSK 3][Radicais ⼤、⼈、⾷]
  \definition[座,个]{s.}{embaixada}
\end{entry}

\begin{entry}{大蒜}{da4suan4}{3,13}[Radicais ⼤、⾋]
  \definition[瓣,头]{s.}{alho}
\end{entry}

\begin{entry}{大腿}{da4tui3}{3,13}[Radicais ⼤、⾁]
  \definition{s.}{coxa}
\end{entry}

\begin{entry}{大戏}{da4xi4}{3,6}[Radicais ⼤、⼽]
  \definition*{s.}{Drama, Ópera Chinesa}
\end{entry}

\begin{entry}{大小}{da4 xiao3}{3,3}[HSK 2][Radicais ⼤、⼩]
  \definition{adv.}{no mínimo}
  \definition[家]{s.}{tamanho | grau de antiguidade | adultos e crianças | grande ou pequeno}
\end{entry}

\begin{entry}{大猩猩}{da4xing1xing5}{3,12,12}[Radicais ⼤、⽝、⽝]
  \definition{s.}{gorila}
\end{entry}

\begin{entry}{大型}{da4xing2}{3,9}[HSK 4][Radicais ⼤、⼟]
  \definition{adj.}{grande; em larga escala; tamanho e volume grandes | larga escala (importante e influente)}
\end{entry}

\begin{entry}{大学}{da4 xue2}{3,8}[HSK 1][Radicais ⼤、⼦]
  \definition[所]{s.}{faculdade | universidade}
\end{entry}

\begin{entry}{大学生}{da4 xue2 sheng1}{3,8,5}[HSK 1][Radicais ⼤、⼦、⽣]
  \definition{s.}{estudante universitário}
\end{entry}

\begin{entry}{大洋洲}{da4yang2zhou1}{3,9,9}[Radicais ⼤、⽔、⽔]
  \definition*{s.}{Oceania}
\end{entry}

\begin{entry}{大爷}{da4 ye5}{3,6}[HSK 4][Radicais ⼤、⽗]
  \definition{s.}{irmão mais velho do pai; tio | tio (homenagem aos homens mais velhos)}
\end{entry}

\begin{entry}{大衣}{da4 yi1}{3,6}[HSK 2][Radicais ⼤、⾐]
  \definition{s.}{sobretudo}
\end{entry}

\begin{entry}{大雨}{da4yu3}{3,8}[Radicais ⼤、⾬]
  \definition[场]{s.}{chuva pesada, forte}
\end{entry}

\begin{entry}{大约}{da4yue1}{3,6}[HSK 3][Radicais ⼤、⽷]
  \definition{adv.}{aproximadamente; sobre | provavelmente}
\end{entry}

\begin{entry}{大战}{da4zhan4}{3,9}[Radicais ⼤、⼽]
  \definition{s.}{guerra}
  \definition{v.}{guerrear | lutar em uma guerra}
\end{entry}

\begin{entry}{大众}{da4 zhong4}{3,6}[HSK 4][Radicais ⼤、⼈]
  \definition{s.}{massas; população; pessoas comuns; público em geral}
\end{entry}

\begin{entry}{大自然}{da4 zi4 ran2}{3,6,12}[HSK 2][Radicais ⼤、⾃、⽕]
  \definition{s.}{natureza}
\end{entry}

\begin{entry}{歹徒}{dai3tu2}{4,10}[Radicais ⽍、⼻]
  \definition{s.}{malfeitor | gangster | bandido}
\end{entry}

\begin{entry}{逮}{dai3}{11}[Radical ⾡]
  \definition{v.}{(coloquial) pegar, aproveitar, capturar}
  \seeref{逮}{dai4}
\end{entry}

\begin{entry}{大}{dai4}{3}[Radical ⼤]
  \definition{s.}{usado em 大夫 \dpy{dai4fu5}: médico, doutor}
  \seeref{大}{da4}
  \seeref{大夫}{dai4fu5}
\end{entry}

\begin{entry}{大夫}{dai4fu5}{3,4}[HSK 3][Radicais ⼤、⼤]
  \definition{s.}{médico, doutor}
  \seeref{大夫}{da4fu1}
\end{entry}

\begin{entry}{代}{dai4}{5}[HSK 3][Radical ⼈]
  \definition*{s.}{sobrenome Dai}
  \definition{s.}{período histórico | dinastia | geração | era}
  \definition{v.}{tomar o lugar de; estar no lugar de
agir em nome de; exercer}
\end{entry}

\begin{entry}{代表}{dai4biao3}{5,8}[HSK 3][Radicais ⼈、⾐]
  \definition[位,个,名]{s.}{deputado; delegado; representante | representante oficial}
  \definition{v.}{representar; defender}
\end{entry}

\begin{entry}{代表团}{dai4 biao3 tuan2}{5,8,6}[HSK 3][Radicais ⼈、⾐、⼞]
  \definition[个]{s.}{delegação; contingente}
\end{entry}

\begin{entry}{代称}{dai4cheng1}{5,10}[Radicais ⼈、⽲]
  \definition{s.}{nome alternativo | antonomásia}
  \definition{v.}{referir-se a algo ou alguém por outro nome}
\end{entry}

\begin{entry}{代价}{dai4jia4}{5,6}[Radicais ⼈、⼈]
  \definition{s.}{preço | custo}
\end{entry}

\begin{entry}{代替}{dai4ti4}{5,12}[HSK 4][Radicais ⼈、⽈]
  \definition{v.}{substituir; substituir por; tomar o lugar de}
\end{entry}

\begin{entry}{代言}{dai4yan2}{5,7}[Radicais ⼈、⾔]
  \definition{v.}{ser um porta-voz | ser um embaixador (para uma marca) | endossar}
\end{entry}

\begin{entry}{带}{dai4}{9}[HSK 2][Radical ⼱]
  \definition{v.}{levar | trazer}
\end{entry}

\begin{entry}{带动}{dai4 dong4}{9,6}[HSK 3][Radicais ⼱、⼒]
  \definition{v.}{dirigir; ativar; fazer algo funcionar; acionar | liderar; trazer; estimular; motivar; atrair}
\end{entry}

\begin{entry}{带来}{dai4 lai2}{9,7}[HSK 2][Radicais ⼱、⽊]
  \definition{v.}{trazer | (figurativo) provocar, produzir}
\end{entry}

\begin{entry}{带领}{dai4ling3}{9,11}[HSK 3][Radicais ⼱、⾴]
  \definition{v.}{guiar | liderar}
\end{entry}

\begin{entry}{待遇}{dai4yu4}{9,12}[HSK 4][Radicais ⼻、⾡]
  \definition[种,项,份]{s.}{tratamento; refere-se a direitos, status social, etc. | salário; ordenado; remuneração}
\end{entry}

\begin{entry}{袋}{dai4}{11}[HSK 4][Radical ⾐]
  \definition{clas.}{para armazenamento em sacolas | para cachimbos, cigarros ou tabaco seco}
  \definition[口]{s.}{saco; sacola; bolso; bolsa}
\end{entry}

\begin{entry}{逮}{dai4}{11}[Radical ⾡]
  \definition{v.}{(literário) alcançar, usado em 逮捕}
  \seeref{逮}{dai3}
  \seealsoref{逮捕}{dai4bu3}
\end{entry}

\begin{entry}{逮捕}{dai4bu3}{11,10}[Radicais ⾡、⼿]
  \definition{v.}{prender | apreender | levar sob custódia}
\end{entry}

\begin{entry}{戴}{dai4}{17}[HSK 4][Radical ⼽]
  \definition*{s.}{sobrenome Dai}
  \definition[条]{s.}{respeito; honra}
  \definition{v.}{usar/vestir (óculos, gravata, relógio de pulso, luvas); colocar objetos em sua cabeça, rosto, pescoço, peito, braços etc.}
\end{entry}

\begin{entry}{单}{dan1}{8}[HSK 4][Radical ⼗]
  \definition{adj.}{sozinho; único | ímpar | sem forro (vestuário) | simples; poucos itens ou categorias; não é complexo | fino; fraco; frágil}
  \definition{adv.}{isoladamente; sozinho | somente; sozinho; unicamente | somente; apenas}
  \definition[个]{s.}{lençol; um pano grande para cobrir a cama | conta; lista; pedaços de papel que detalham coisas}
  \seeref{单}{chan2}
  \seeref{单}{shan4}
\end{entry}

\begin{entry}{单纯}{dan1chun2}{8,7}[HSK 4][Radicais ⼗、⽷]
  \definition{adj.}{puro; simples; descomplicado}
  \definition{adv.}{sozinho; puramente; meramente}
\end{entry}

\begin{entry}{单调}{dan1diao4}{8,10}[HSK 4][Radicais ⼗、⾔]
  \definition{adj.}{maçante; monótono}
\end{entry}

\begin{entry}{单独}{dan1du2}{8,9}[HSK 4][Radicais ⼗、⽝]
  \definition{adv.}{solo; sozinho; por si mesmo; por conta própria}
\end{entry}

\begin{entry}{单脚滑行车}{dan1jiao3hua2xing2che1}{8,11,12,6,4}[Radicais ⼗、⾁、⽔、⾏、⾞]
  \definition{s.}{\emph{scooter}}
\end{entry}

\begin{entry}{单位}{dan1wei4}{8,7}[HSK 2][Radicais ⼗、⼈]
  \definition[个]{s.}{unidade (como padrão de medida) | unidade (como uma organização, departamento, divisão, seção, etc.)}
\end{entry}

\begin{entry}{单元}{dan1yuan2}{8,4}[HSK 3][Radicais ⼗、⼉]
  \definition[个,组,套]{s.}{unidade (de algo)}
\end{entry}

\begin{entry}{单质}{dan1zhi4}{8,8}[Radicais ⼗、⾙]
  \definition{s.}{substância simples (consistindo puramente de um elemento, como diamante, grafite, etc.)}
\end{entry}

\begin{entry}{担保}{dan1bao3}{8,9}[HSK 4][Radicais ⼿、⼈]
  \definition{v.}{garantir; atestar; expressar responsabilidade e garantir que não haverá problemas ou que eles serão resolvidos}
\end{entry}

\begin{entry}{担任}{dan1ren4}{8,6}[HSK 4][Radicais ⼿、⼈]
  \definition{v.}{servir como; assumir o cargo de; ocupar o posto de; ocupar um determinado cargo ou emprego}
\end{entry}

\begin{entry}{担心}{dan1xin1}{8,4}[HSK 4][Radicais ⼿、⼼]
  \definition{v.}{preocupar-se; ficar ansioso; sentir-se desconfortável com algo}
\end{entry}

\begin{entry}{耽心}{dan1xin1}{10,4}[Radicais ⽿、⼼]
  \variantof{担心}
\end{entry}

\begin{entry}{胆小鬼}{dan3xiao3gui3}{9,3,9}[Radicais ⾁、⼩、⿁]
  \definition{adj.}{covarde | medroso}
\end{entry}

\begin{entry}{但}{dan4}{7}[HSK 2][Radical ⼈]
  \definition{conj.}{mas | ainda | no entanto | apenas}
\end{entry}

\begin{entry}{但是}{dan4 shi4}{7,9}[HSK 2][Radicais ⼈、⽇]
  \definition{conj.}{mas | ainda | no entanto}
\end{entry}

\begin{entry}{淡}{dan4}{11}[HSK 4][Radical ⽔]
  \definition*{s.}{sobrenome Dan}
  \definition{adj.}{fino; leve | sem gosto; fraco; não tem sabor forte; não é salgado | leve; fraco; pálido | indiferente; frio; sem entusiasmo | frouxo; sem brilho | sem sentido; trivial}
\end{entry}

\begin{entry}{蛋}{dan4}{11}[HSK 2][Radical ⾍]
  \definition[个,打]{s.}{ovo | objeto de formato oval}
\end{entry}

\begin{entry}{蛋糕}{dan4gao1}{11,16}[Radicais ⾍、⽶]
  \definition[块,个]{s.}{bolo}
\end{entry}

\begin{entry}{当}{dang1}{6}[HSK 2][Radical ⼹]
  \definition*{s.}{sobrenome Dang}
  \definition{adj.}{igual}
  \definition{interj.}{(onomatopéia) barulho metálico, clangor}
  \definition{prep.}{na presença de alguém | na cara de alguém | exatamente em (um momento ou lugar)}
  \definition{v.}{trabalhar como | servir como | ser | suportar | aceitar | merecer | dirigir | gerir | estar encarregado de | deveria | deve}
  \seeref{当}{dang4}
\end{entry}

\begin{entry}{当初}{dang1chu1}{6,7}[HSK 3][Radicais ⼹、⾐]
  \definition{s.}{no começo; originalmente; no início; em primeiro lugar}
\end{entry}

\begin{entry}{当地}{dang1di4}{6,6}[Radicais ⼹、⼟]
  \definition{s.}{local}
\end{entry}

\begin{entry}{当然}{dang1ran2}{6,12}[HSK 3][Radicais ⼹、⽕]
  \definition{adj.}{natural; verdadeiro; espontâneo}
  \definition{adv.}{sem dúvida; certamente; claro}
\end{entry}

\begin{entry}{当时}{dang1shi2}{6,7}[HSK 2][Radicais ⼹、⽇]
  \definition{s.}{aquela época}
  \definition{v.}{ser o momento certo; acontecer na hora certa}
  \seeref{当时}{dang4shi2}
\end{entry}

\begin{entry}{当中}{dang1 zhong1}{6,4}[HSK 3][Radicais ⼹、⼁]
  \definition{prep.}{no meio; no centro | entre}
\end{entry}

\begin{entry}{挡风玻璃}{dang3feng1bo1li5}{9,4,9,14}[Radicais ⼿、⾵、⽟、⽟]
  \definition{s.}{parabrisa}
\end{entry}

\begin{entry}{当}{dang4}{6}[Radical ⼹]
  \definition{adj.}{próprio; certo}
  \definition{pron.}{naquele mesmo (dia, etc.)}
  \definition{s.}{algo penhorado | penhor | promessa}
  \definition{v.}{combinar | igualar a tratar como | considerar como | tomar para pensar | penhorar}
  \seeref{当}{dang1}
\end{entry}

\begin{entry}{当时}{dang4shi2}{6,7}[Radicais ⼹、⽇]
  \definition{adv.}{imediatamente}
  \seeref{当时}{dang1shi2}
\end{entry}

\begin{entry}{刀}{dao1}{2}[HSK 3][Kangxi 18][Radical ⼑]
  \definition*{s.}{sobrenome Dao}
  \definition{clas.}{para cortes de faca ou facadas | para cem folhas (de papel)}
  \definition[把]{s.}{faca; espada | algo em forma de faca | moeda antiga em forma de faca}
\end{entry}

\begin{entry}{导弹}{dao3dan4}{6,11}[Radicais ⼨、⼸]
  \definition[枚]{s.}{míssil (guiado)}
\end{entry}

\begin{entry}{导演}{dao3yan3}{6,14}[HSK 3][Radicais ⼨、⽔]
  \definition[位,名,个]{s.}{diretor}
  \definition{v.}{dirigir (um filme, peça, etc.)}
\end{entry}

\begin{entry}{导游}{dao3you2}{6,12}[HSK 4][Radicais ⼨、⽔]
  \definition[个,位,名]{s.}{guia turístico; pessoas que trabalham como guias turísticos}
  \definition{v.}{guiar; conduzir um passeio turístico}
\end{entry}

\begin{entry}{导致}{dao3zhi4}{6,10}[HSK 4][Radicais ⼨、⾄]
  \definition{v.}{causar; levar a; dar origem a (um resultado ruim)}
\end{entry}

\begin{entry}{倒}{dao3}{10}[HSK 2][Radical ⼈]
  \definition{v.}{cair no chão | deitar-se no chão | colapsar | ir à falência}
  \seeref{倒}{dao4}
\end{entry}

\begin{entry}{倒闭}{dao3bi4}{10,6}[HSK 4][Radicais ⼈、⾨]
  \definition{v.}{fechar; ir à falência; entrar em liquidação; sair do negócio}
\end{entry}

\begin{entry}{倒车}{dao3che1}{10,4}[HSK 4][Radicais ⼈、⾞]
  \definition{v.}{mudar de trem ou ônibus; trocar de trem ou ônibus no meio do caminho}
  \seeref{倒车}{dao4che1}
\end{entry}

\begin{entry}{倒地}{dao3di4}{10,6}[Radicais ⼈、⼟]
  \definition{v.}{cair no chão}
\end{entry}

\begin{entry}{倒楣}{dao3mei2}{10,13}[Radicais ⼈、⽊]
  \variantof{倒霉}
\end{entry}

\begin{entry}{倒霉}{dao3mei2}{10,15}[Radicais ⼈、⾬]
  \definition{adj.}{azarado}
  \definition{s.}{azar | má sorte}
  \definition{v.}{estar sem sorte | ter azar}
  \seealsoref{倒血霉}{dao3xue4mei2}
\end{entry}

\begin{entry}{倒血霉}{dao3xue4mei2}{10,6,15}[Radicais ⼈、⾎、⾬]
  \definition{v.}{ter muito azar (versão mais forte de 倒霉)}
  \seealsoref{倒霉}{dao3mei2}
\end{entry}

\begin{entry}{到}{dao4}{8}[HSK 1][Radical ⼑]
  \definition{prep.}{a | até | para}
  \definition{v.}{chegar}
\end{entry}

\begin{entry}{到处}{dao4chu4}{8,5}[HSK 2][Radicais ⼑、⼡]
  \definition{adv.}{em todos os lugares}
\end{entry}

\begin{entry}{到达}{dao4da2}{8,6}[HSK 3][Radicais ⼑、⾡]
  \definition{v.}{chegar; alcançar}
\end{entry}

\begin{entry}{到底}{dao4di3}{8,8}[HSK 3][Radicais ⼑、⼴]
  \definition{adv.}{na terra (usado em frases interrogativas para expressar a determinação de alguém em encontrar uma resposta definitiva) | afinal | finalmente; por fim; no fim}
\end{entry}

\begin{entry}{倒}{dao4}{10}[HSK 2][Radical ⼈]
  \definition{adv.}{ao contrário da expectativa | ao contrário}
  \definition{v.}{inverter | colocar de cabeça para baixo ou de frente para trás | derramar | tombar}
  \seeref{倒}{dao3}
\end{entry}

\begin{entry}{倒车}{dao4che1}{10,4}[HSK 4][Radicais ⼈、⾞]
  \definition{v.}{dar marcha à ré (em um veículo)}
  \seeref{倒车}{dao3che1}
\end{entry}

\begin{entry}{道}{dao4}{12}[HSK 2][Radical ⾡]
  \definition*{s.}{Taoism | Taoist}
  \definition*{s.}{sobrenome Dao}
  \definition{s.}{estrada | caminho | rota | caminho | canal | curso | maneira | método | moral | moralidade | doutrina | corpo de ensinamentos morais | o Caminho da Natureza que não pode receber um nome | princípio | seita supersticiosa | linha | trato | habilidade}
\end{entry}

\begin{entry}{道理}{dao4li5}{12,11}[HSK 2][Radicais ⾡、⽟]
  \definition[个]{s.}{razão | argumento | sentido | princípio | base | justificativa}
\end{entry}

\begin{entry}{道路}{dao4 lu4}{12,13}[HSK 2][Radicais ⾡、⾜]
  \definition{s.}{estrada | caminho | processo}
\end{entry}

\begin{entry}{道歉}{dao4qian4}{12,14}[Radicais ⾡、⽋]
  \definition{v.+compl.}{desculpar-se | fazer um pedido de desculpas}
\end{entry}

\begin{entry}{得}{de2}{11}[HSK 2][Radical ⼻]
  \definition{adj.}{adequado; apropriado | satisfeito; complacente; orgulhoso de si mesmo}
  \definition{interj.}{usado para encerrar uma conversa para indicar concordância ou proibição | usado quando a situação não é a esperada, para expressar impotência}
  \definition{v.}{obter (em vez de ``perder''); conseguir; ganhar | (de um cálculo) igual; resultar em; efetuar cálculos para produzir resultados | estar terminado; estar pronto; cumprir | contrair uma doença}
  \definition{v.aux.}{usado antes de outros verbos para expressar permissão | usado antes de outros verbos para indicar que é possível (usado principalmente na forma negativa) | usado em conversas para indicar que não há necessidade de dizer mais nada}
  \seeref{得}{de5}
  \seeref{得}{dei3}
\end{entry}

\begin{entry}{得出}{de2 chu1}{11,5}[HSK 2][Radicais ⼻、⼐]
  \definition{v.}{chegar (a uma conclusão) | obter (a um resultado)}
\end{entry}

\begin{entry}{得到}{de2 dao4}{11,8}[HSK 1][Radicais ⼻、⼑]
  \definition{v.}{obter | receber}
\end{entry}

\begin{entry}{得分}{de2 fen1}{11,4}[HSK 3][Radicais ⼻、⼑]
  \definition{v.}{fazer pontos; pontuar}
\end{entry}

\begin{entry}{得了}{de2le5}{11,2}[Radicais ⼻、⼅]
  \definition{expr.}{Tudo bem!; É o bastante!}
  \seeref{得了}{de2liao3}
\end{entry}

\begin{entry}{得了}{de2liao3}{11,2}[Radicais ⼻、⼅]
  \definition{adj.}{(enfaticamente, em perguntas retóricas) possível}
  \seeref{得了}{de2le5}
\end{entry}

\begin{entry}{得意}{de2yi4}{11,13}[HSK 4][Radicais ⼻、⼼]
  \definition{adj.}{complacente; orgulhoso de si mesmo; satisfeito consigo mesmo}
  \definition{v.+compl.}{orgulhar-se de si mesmo; ter satisfação consigo mesmo; ser complacente}
\end{entry}

\begin{entry}{德}{de2}{15}[Radical ⼻]
  \definition*{s.}{Alemanha, abreviação de 德国}
  \definition{s.}{virtude | bondade | moralidade | ética | personagem | tipo}
  \seealsoref{德国}{de2guo2}
\end{entry}

\begin{entry}{德国}{de2guo2}{15,8}[Radicais ⼻、⼞]
  \definition*{s.}{Alemanha}
\end{entry}

\begin{entry}{德国人}{de2guo2ren2}{15,8,2}[Radicais ⼻、⼞、⼈]
  \definition{s.}{alemão | pessoa ou povo da Alemanha}
\end{entry}

\begin{entry}{地}{de5}{6}[HSK 1][Radical ⼟]
  \definition{part.}{(estrutural) utilizada antes de um verbo ou adjetivo, ligando-o ao adjunto adverbial modificador precedente}
  \seeref{地}{di4}
\end{entry}

\begin{entry}{底}{de5}{8}[Radical ⼴]
  \definition{part.}{usada após uma palavra ou frase que é usada como determinante para indicar subordinação à palavra central}
  \seeref{底}{di3}
\end{entry}

\begin{entry}{的}{de5}{8}[Radical ⽩]
  \definition{part.}{de | partícula usada em possessivos | utilizada entre adjetivos e substantivos (opcional se o adjetivo possui apenas um caracter) | usado após um atributo | usado para formar uma expressão nominal | usado no final de uma frase declarativa para dar ênfase}
  \seeref{的}{di1}
  \seeref{的}{di2}
  \seeref{的}{di4}
\end{entry}

\begin{entry}{的话}{de5 hua4}{8,8}[HSK 2][Radicais ⽩、⾔]
  \definition{part.}{se | no caso | suponha que}
\end{entry}

\begin{entry}{得}{de5}{11}[HSK 2][Radical ⼻]
  \definition{part.}{depois de um verbo ou adjetivo para expressar possibilidade ou capacidade | entre um verbo e seu complemento para expressar possibilidade | ligando um verbo ou um adjetivo a um complemento que descreve a maneira ou o grau}
  \seeref{得}{de2}
  \seeref{得}{dei3}
\end{entry}

\begin{entry}{得}{dei3}{11}[HSK 4][Radical ⼻]
  \definition{v.}{precisar; dever; expressa uma necessidade racional, factual ou subjetiva | dever; ter que; necessidade expressa, volitiva ou factual}
  \seeref{得}{de2}
  \seeref{得}{de5}
\end{entry}

\begin{entry}{灯}{deng1}{6}[HSK 2][Radical ⽕]
  \definition[盏]{s.}{lâmpada | lanterna | luz}
\end{entry}

\begin{entry}{灯标}{deng1biao1}{6,9}[Radicais ⽕、⽊]
  \definition{s.}{farol | luz de farol}
\end{entry}

\begin{entry}{灯光}{deng1 guang1}{6,6}[HSK 4][Radicais ⽕、⼉]
  \definition{s.}{iluminação; luminosidade da lâmpada | luminação (palco); equipamento de iluminação para palco ou estúdio}
\end{entry}

\begin{entry}{灯号}{deng1hao4}{6,5}[Radicais ⽕、⼝]
  \definition{s.}{sinal luminoso | luz indicadora}
\end{entry}

\begin{entry}{灯泡}{deng1pao4}{6,8}[Radicais ⽕、⽔]
  \definition[个]{s.}{lâmpada | (gíria) terceiro indesejado estragando encontro de casal}
  \seealsoref{电灯泡}{dian4deng1pao4}
\end{entry}

\begin{entry}{灯丝}{deng1si1}{6,5}[Radicais ⽕、⼀]
  \definition{s.}{filamento (de uma lâmpada)}
\end{entry}

\begin{entry}{登}{deng1}{12}[HSK 4][Radical ⽨]
  \definition*{s.}{sobrenome Deng}
  \definition{v.}{subir; montar; escalar (uma altura) | publicar; registrar; inserir | ser colhidas e levadas para a eira | pressionar com o pé; pedalar; pisar | pisar em; pisar | calçar (calçados, etc.)}
\end{entry}

\begin{entry}{登记}{deng1ji4}{12,5}[HSK 4][Radicais ⽨、⾔]
  \definition{v.+compl.}{registrar-se; fazer o \emph{check-in} | registrar; reportar; informar; relatar por escrito a um superior ou autoridade relevante (usado principalmente para documentos legais)}
\end{entry}

\begin{entry}{登录}{deng1lu4}{12,8}[HSK 4][Radicais ⽨、⼹]
  \definition{v.}{fazer \emph{logon}; fazer \emph{login} | gravar; registrar; computadores eletrônicos e sua terminologia de rede, referindo-se ao acesso ao sistema operacional ou ao site a ser visitado}
\end{entry}

\begin{entry}{登山}{deng1 shan1}{12,3}[HSK 4][Radicais ⽨、⼭]
  \definition{s.}{escalar; fazer alpinismo; subir uma montanha}
\end{entry}

\begin{entry}{等}{deng3}{12}[HSK 1,2][Radical ⽵]
  \definition{adj.}{igual}
  \definition{clas.}{para classe, grau, classificação | para tipo}
  \definition{prep.}{quando | até}
  \definition{v.}{esperar | aguardar}
\end{entry}

\begin{entry}{等待}{deng3dai4}{12,9}[HSK 3][Radicais ⽵、⼻]
  \definition{v.}{esperar; aguardar}
\end{entry}

\begin{entry}{等到}{deng3 dao4}{12,8}[HSK 2][Radicais ⽵、⼑]
  \definition{prep.}{pelo tempo | quando | espere até}
\end{entry}

\begin{entry}{等于}{deng3yu2}{12,3}[HSK 2][Radicais ⽵、⼆]
  \definition{adv.}{igual a | equivalente a}
  \definition{v.}{equivaler a | ser equivalente a}
\end{entry}

\begin{entry}{低}{di1}{7}[HSK 2][Radical ⼈]
  \definition{adj.}{baixo}
  \definition{adv.}{abaixo}
  \definition{v.}{abaixar (a cabeça) | deixar cair | pendurar | inclinar}
\end{entry}

\begin{entry}{低潮}{di1chao2}{7,15}[Radicais ⼈、⽔]
  \definition{s.}{maré baixa/vazante; o nível mais baixo da maré durante um ciclo de maré (distinto da ``高潮'') | vazante baixa; o ponto mais baixo; uma metáfora para o baixo estágio de desenvolvimento das coisas}
  \seealsoref{高潮}{gao1chao2}
\end{entry}

\begin{entry}{的}{di1}{8}[Radical ⽩]
  \definition{s.}{abreviação de 的士: um táxi}
  \seeref{的}{de5}
  \seeref{的}{di2}
  \seeref{的}{di4}
  \seealsoref{的士}{di1shi4}
\end{entry}

\begin{entry}{的士}{di1shi4}{8,3}[Radicais ⽩、⼠]
  \definition{s.}{(empréstimo linguístico) táxi}
\end{entry}

\begin{entry}{堤坝}{di1ba4}{12,7}[Radicais ⼟、⼟]
  \definition{s.}{represa | dique | barragem}
\end{entry}

\begin{entry}{滴}{di1}{14}[Radical ⽔]
  \definition{s.}{uma gota}
  \definition{v.}{pingar}
\end{entry}

\begin{entry}{的}{di2}{8}[Radical ⽩]
  \definition{adv.}{realmente e verdadeiramente}
  \seeref{的}{de5}
  \seeref{的}{di1}
  \seeref{的}{di4}
\end{entry}

\begin{entry}{的确}{di2que4}{8,12}[HSK 4][Radicais ⽩、⽯]
  \definition{adv.}{realmente; de fato, ao expressar certeza sobre a situação}
\end{entry}

\begin{entry}{敌人}{di2ren2}{10,2}[HSK 4][Radicais ⾆、⼈]
  \definition[群,伙,股,批,帮,个]{s.}{inimigo; pessoa hostil; parte hostil}
\end{entry}

\begin{entry}{笛}{di2}{11}[Radical ⽵]
  \definition{s.}{flauta}
\end{entry}

\begin{entry}{底}{di3}{8}[HSK 4][Radical ⼴]
  \definition*{s.}{sobrenome Di}
  \definition{pron.}{o que |  isto; isso; aqui | tal | dessa forma}
  \definition{s.}{base; fundo; parte inferior de um objeto | detalhes; o cerne da questão; base, fonte ou contexto de uma coisa | rascunho; cópia mantida como registro; rascunho que pode ser usado como base | final de um ano ou mês | chão; fundo; fundação | a última parte de algo}
  \seeref{底}{de5}
\end{entry}

\begin{entry}{底气}{di3qi4}{8,4}[Radicais ⼴、⽓]
  \definition{s.}{capacidade pulmonar | ousadia | confiança | autoconfiança | vigor}
\end{entry}

\begin{entry}{底下}{di3 xia4}{8,3}[HSK 3][Radicais ⼴、⼀]
  \definition{adv.}{em baixo; abaixo; sob | próximo; mais tarde; depois}
\end{entry}

\begin{entry}{抵抗}{di3kang4}{8,7}[Radicais ⼿、⼿]
  \definition{s.}{resistência}
  \definition{v.}{resistir}
\end{entry}

\begin{entry}{地}{di4}{6}[HSK 1][Radical ⼟]
  \definition[个,片]{s.}{mundo | campo | chão | terra | lugar}
  \seeref{地}{de5}
\end{entry}

\begin{entry}{地点}{di4dian3}{6,9}[HSK 1][Radicais ⼟、⽕]
  \definition[个]{s.}{localização | lugar | local}
\end{entry}

\begin{entry}{地方}{di4fang1}{6,4}[Radicais ⼟、⽅]
  \definition[个]{s.}{distrito; localidade;  em oposição a ``中央'', o número total de unidades administrativas em todos os níveis abaixo do centro | governo local e população; refere-se a outros setores que não o militar}
  \seeref{地方}{di4fang5}
  \seealsoref{中央}{zhong1yang1}
\end{entry}

\begin{entry}{地方}{di4fang5}{6,4}[HSK 1,4][Radicais ⼟、⽅]
  \definition[个,处,块]{s.}{lugar; cômodo; área; refere-se a um espaço específico
parte}
  \seeref{地方}{di4fang1}
\end{entry}

\begin{entry}{地核}{di4he2}{6,10}[Radicais ⼟、⽊]
  \definition{s.}{(geologia) núcleo da Terra}
\end{entry}

\begin{entry}{地理}{di4li3}{6,11}[Radicais ⼟、⽟]
  \definition{s.}{geografia}
\end{entry}

\begin{entry}{地面}{di4 mian4}{6,9}[HSK 4][Radicais ⼟、⾯]
  \definition{s.}{a superfície da Terra | térreo; piso; camada de material colocada no chão dentro e ao redor dos edifícios | localidade; chão | região; território; principalmente áreas administrativas}
\end{entry}

\begin{entry}{地球}{di4qiu2}{6,11}[HSK 2][Radicais ⼟、⽟]
  \definition{s.}{o planeta terra}
\end{entry}

\begin{entry}{地区}{di4qu1}{6,4}[HSK 3][Radicais ⼟、⼖]
  \definition{adj.}{regional}
  \definition[个]{s.}{área; distrito; região | prefeitura | latitudes; localidade; lado}
  \definition{suf.}{como sufixo do nome da cidade, significa prefeitura ou condado}
\end{entry}

\begin{entry}{地上}{di4 shang5}{6,3}[HSK 1][Radicais ⼟、⼀]
  \definition{adv.}{no chão}
\end{entry}

\begin{entry}{地铁}{di4tie3}{6,10}[HSK 2][Radicais ⼟、⾦]
  \definition{s.}{metrô, metropolitano}
\end{entry}

\begin{entry}{地铁站}{di4 tie3 zhan4}{6,10,10}[HSK 2][Radicais ⼟、⾦、⽴]
  \definition{s.}{estação de metrô}
\end{entry}

\begin{entry}{地图}{di4tu2}{6,8}[HSK 1][Radicais ⼟、⼞]
  \definition[张,本]{s.}{mapa}
\end{entry}

\begin{entry}{地位}{di4wei4}{6,7}[HSK 4][Radicais ⼟、⼈]
  \definition{s.}{lugar; status; posição; posição da pessoa ou do grupo nas relações sociais | lugar; posição (ocupada por uma pessoa ou coisa); espaço ocupado por uma pessoa ou coisa}
\end{entry}

\begin{entry}{地下}{di4 xia4}{6,3}[HSK 4][Radicais ⼟、⼀]
  \definition{s.}{subterrâneo | secreta (atividade) | recursos ocultos}
\end{entry}

\begin{entry}{地下室}{di4xia4shi4}{6,3,9}[Radicais ⼟、⼀、⼧]
  \definition{s.}{subterrâneo | porão}
\end{entry}

\begin{entry}{地狱}{di4yu4}{6,9}[Radicais ⼟、⽝]
  \definition*{s.}{\emph{Naraka} (Budismo)}
  \definition{adj.}{infernal}
  \definition{s.}{inferno | submundo}
\end{entry}

\begin{entry}{地震}{di4zhen4}{6,15}[Radicais ⼟、⾬]
  \definition{s.}{terremoto | tremor de terra}
\end{entry}

\begin{entry}{地址}{di4zhi3}{6,7}[HSK 4][Radicais ⼟、⼟]
  \definition[个]{s.}{endereço; local de residência ou correspondência}
\end{entry}

\begin{entry}{地砖}{di4zhuan1}{6,9}[Radicais ⼟、⽯]
  \definition{s.}{ladrilho de piso}
\end{entry}

\begin{entry}{弟}{di4}{7}[HSK 1][Radical ⼸]
  \definition{s.}{irmão mais novo | júnior}
\end{entry}

\begin{entry}{弟弟}{di4 di5}{7,7}[HSK 1][Radicais ⼸、⼸]
  \definition[个,位]{s.}{irmão mais novo}
\end{entry}

\begin{entry}{弟妹}{di4mei4}{7,8}[Radicais ⼸、⼥]
  \definition{s.}{esposa do irmão mais novo}
\end{entry}

\begin{entry}{的}{di4}{8}[Radical ⽩]
  \definition{adj.}{objetivo | claro}
  \seeref{的}{de5}
  \seeref{的}{di1}
  \seeref{的}{di2}
\end{entry}

\begin{entry}{帝国}{di4guo2}{9,8}[Radicais ⼱、⼞]
  \definition{adj.}{imperial}
  \definition{s.}{império}
\end{entry}

\begin{entry}{第}{di4}{11}[HSK 1][Radical ⽵]
  \definition{num.}{prefixo para expressar números ordinais}
\end{entry}

\begin{entry}{墬}{di4}{14}[Radical ⼟]
  \variantof{地}
\end{entry}

\begin{entry}{典型}{dian3xing2}{8,9}[HSK 4][Radicais ⼋、⼟]
  \definition{adj.}{típico; representativo}
  \definition[个]{s.}{modelo; caso típico; indivíduo ou evento representativo | personagens típicos; personalidades modelo (em obras literárias); personagens na literatura e na arte que refletem a natureza de uma determinada sociedade e têm uma personalidade distinta}
\end{entry}

\begin{entry}{点}{dian3}{9}[HSK 1][Radical ⽕]
  \definition{clas.}{para itens | hora cheia}
  \definition{s.}{ponto | gota | mancha | horas | ponto (no espaço ou no tempo) | traço de ponto em caracteres chineses}
  \definition{v.}{desenhar um ponto | verificar uma lista | escolher | pedir (comida em um restaurante) | tocar brevemente | sugerir | acender | derramar um líquido gota a gota}
\end{entry}

\begin{entry}{点火}{dian3huo3}{9,4}[Radicais ⽕、⽕]
  \definition{s.}{ignição}
  \definition{v.}{inflamar | acender um fogo | agitar | dar partida em um motor | (figurativo) provocar problemas}
\end{entry}

\begin{entry}{点名}{dian3 ming2}{9,6}[HSK 4][Radicais ⽕、⼝]
  \definition{v.}{fazer a lista de chamada; manter o controle da presença de alguém; chamar nomes para controle de presença | mencionar alguém pelo nome}
\end{entry}

\begin{entry}{点燃}{dian3ran2}{9,16}[Radicais ⽕、⽕]
  \definition{v.}{inflamar | incendiar}
\end{entry}

\begin{entry}{点头}{dian3 tou2}{9,5}[HSK 2][Radicais ⽕、⼤]
  \definition{v.}{acenar com a cabeça}
\end{entry}

\begin{entry}{电}{dian4}{5}[HSK 1][Radical ⽥]
  \definition{s.}{eletricidade | telegrama | cabo}
  \definition{v.}{dar ou receber um choque elétrico | enviar um telegrama | telegrafar}
\end{entry}

\begin{entry}{电冰箱}{dian4bing1xiang1}{5,6,15}[Radicais ⽥、⼎、⾋]
  \definition[台]{s.}{frigorífico | refrigerador}
\end{entry}

\begin{entry}{电车司机}{dian4che1 si1ji1}{5,4,5,6}[Radicais ⽥、⾞、⼝、⽊]
  \definition{s.}{motorista de bonde}
\end{entry}

\begin{entry}{电灯}{dian4 deng1}{5,6}[HSK 4][Radicais ⽥、⽕]
  \definition[盏,个]{s.}{luz elétrica; lâmpada elétrica; lâmpadas que usam eletricidade como fonte de energia}
\end{entry}

\begin{entry}{电灯泡}{dian4deng1pao4}{5,6,8}[Radicais ⽥、⽕、⽔]
  \definition{s.}{lâmpada elétrica | (gíria) terceiro convidado indesejado}
\end{entry}

\begin{entry}{电动}{dian4dong4}{5,6}[Radicais ⽥、⼒]
  \definition{adj.}{movido a eletricidade | elétrico}
\end{entry}

\begin{entry}{电动车}{dian4 dong4 che1}{5,6,4}[HSK 4][Radicais ⽥、⼒、⾞]
  \definition{s.}{veículo elétrico (\emph{scooter}, bicicleta, carro, etc.)}
\end{entry}

\begin{entry}{电话}{dian4 hua4}{5,8}[HSK 1][Radicais ⽥、⾔]
  \definition[部]{s.}{telefone}
  \definition[通]{s.}{chamada telefônica}
\end{entry}

\begin{entry}{电脑}{dian4nao3}{5,10}[HSK 1][Radicais ⽥、⾁]
  \definition[台]{s.}{computador}
\end{entry}

\begin{entry}{电脑语言}{dian4nao3yu3yan2}{5,10,9,7}[Radicais ⽥、⾁、⾔、⾔]
  \definition{s.}{linguagem de programação | linguagem de computador}
\end{entry}

\begin{entry}{电器}{dian4qi4}{5,16}[Radicais ⽥、⼝]
  \definition{s.}{aparelho elétrico}
\end{entry}

\begin{entry}{电视}{dian4shi4}{5,8}[HSK 1][Radicais ⽥、⾒]
  \definition[台,个]{s.}{televisão | TV | televisor}
\end{entry}

\begin{entry}{电视机}{dian4 shi4 ji1}{5,8,6}[HSK 1][Radicais ⽥、⾒、⽊]
  \definition[台]{s.}{aparelho de televisão | televisor}
\end{entry}

\begin{entry}{电视剧}{dian4 shi4 ju4}{5,8,10}[HSK 3][Radicais ⽥、⾒、⼑]
  \definition[部]{s.}{série de TV; drama de TV; novela}
\end{entry}

\begin{entry}{电视台}{dian4 shi4 tai2}{5,8,5}[HSK 3][Radicais ⽥、⾒、⼝]
  \definition[个]{s.}{canal de TV | estação de televisão}
\end{entry}

\begin{entry}{电台}{dian4 tai2}{5,5}[HSK 3][Radicais ⽥、⼝]
  \definition[个,家]{s.}{transceptor; transmissor-receptor | aparelho de rádio; estação de rádio; estação de transmissão}
\end{entry}

\begin{entry}{电梯}{dian4ti1}{5,11}[HSK 4][Radicais ⽥、⽊]
  \definition[部,台,架]{s.}{elevador}
\end{entry}

\begin{entry}{电梯司机}{dian4ti1 si1ji1}{5,11,5,6}[Radicais ⽥、⽊、⼝、⽊]
  \definition{s.}{ascensorista}
\end{entry}

\begin{entry}{电影}{dian4ying3}{5,15}[HSK 1][Radicais ⽥、⼺]
  \definition[部,片,幕,场]{s.}{filme}
\end{entry}

\begin{entry}{电影奖}{dian4ying3jiang3}{5,15,9}[Radicais ⽥、⼺、⼤]
  \definition{s.}{premiações de cinema}
\end{entry}

\begin{entry}{电影节}{dian4ying3jie2}{5,15,5}[Radicais ⽥、⼺、⾋]
  \definition{s.}{festival de cinema}
\end{entry}

\begin{entry}{电影界}{dian4ying3jie4}{5,15,9}[Radicais ⽥、⼺、⽥]
  \definition{s.}{indústria cinematográfica}
\end{entry}

\begin{entry}{电影票}{dian4ying3piao4}{5,15,11}[Radicais ⽥、⼺、⽰]
  \definition{s.}{ingresso de filme}
\end{entry}

\begin{entry}{电影术}{dian4ying3 shu4}{5,15,5}[Radicais ⽥、⼺、⽊]
  \definition{s.}{cinematografia}
\end{entry}

\begin{entry}{电影艺术}{dian4ying3 yi4shu4}{5,15,4,5}[Radicais ⽥、⼺、⾋、⽊]
  \definition{s.}{arte cinematográfica}
\end{entry}

\begin{entry}{电影音乐}{dian4ying3 yin1yue4}{5,15,9,5}[Radicais ⽥、⼺、⾳、⼃]
  \definition{s.}{música cinematográfica}
\end{entry}

\begin{entry}{电影院}{dian4 ying3 yuan4}{5,15,9}[HSK 1][Radicais ⽥、⼺、⾩]
  \definition[次,家,座]{s.}{sala de cinema}
\end{entry}

\begin{entry}{电邮}{dian4you2}{5,7}[Radicais ⽥、⾢]
  \definition{s.}{correio eletrônico, \emph{e-mail} | abreviação de~电子邮件}
  \seealsoref{电子邮件}{dian4zi3you2jian4}
\end{entry}

\begin{entry}{电源}{dian4yuan2}{5,13}[HSK 4][Radicais ⽥、⽔]
  \definition{s.}{fonte de alimentação; fonte de energia; fonte de energia elétrica; dispositivo que fornece energia elétrica a um aparelho, como uma bateria, um gerador, etc.}
\end{entry}

\begin{entry}{电子}{dian4zi3}{5,3}[Radicais ⽥、⼦]
  \definition{s.}{eletrônico | elétron}
\end{entry}

\begin{entry}{电子名片}{dian4zi3 ming2pian4}{5,3,6,4}[Radicais ⽥、⼦、⼝、⽚]
  \definition{s.}{cartão de visita eletrônico}
\end{entry}

\begin{entry}{电子邮件}{dian4zi3you2jian4}{5,3,7,6}[HSK 3][Radicais ⽥、⼦、⾢、⼈]
  \definition[封,份]{s.}{correio eletrônico; \emph{e-mail}}
  \seealsoref{电邮}{dian4you2}
\end{entry}

\begin{entry}{店}{dian4}{8}[HSK 2][Radical ⼴]
  \definition{s.}{loja | pousada}
\end{entry}

\begin{entry}{店员}{dian4yuan2}{8,7}[Radicais ⼴、⼝]
  \definition{s.}{assistente de loja | balconista | vendedor}
\end{entry}

\begin{entry}{店主}{dian4zhu3}{8,5}[Radicais ⼴、⼂]
  \definition{s.}{lojista | dono de loja}
\end{entry}

\begin{entry}{垫子}{dian4zi5}{9,3}[Radicais ⼟、⼦]
  \definition{s.}{colchão | esteira | almofada}
\end{entry}

\begin{entry}{钿}{dian4}{10}[Radical ⾦]
  \definition{s.}{ornamento incrustado antigo em forma de flor}
  \definition{v.}{incrustar com ouro, prata, etc.}
  \seeref{钿}{tian2}
\end{entry}

\begin{entry}{淀}{dian4}{11}[Radical ⽔]
  \definition{adj.}{pantanoso}
  \definition{s.}{lago raso | pântano}
  \definition{v.}{formar sedimentos | precipitar}
\end{entry}

\begin{entry}{貂}{diao1}{12}[Radical ⾘]
  \definition{s.}{marta | fuinha}
\end{entry}

\begin{entry}{雕刻}{diao1ke4}{16,8}[Radicais ⾫、⼑]
  \definition{s.}{escultura}
  \definition{v.}{esculpir | gravar}
\end{entry}

\begin{entry}{鸟}{diao3}{5}[Radical ⿃]
  \definition{s.}{pênis | órgão genital masculino | aves | aviário}
  \seeref{鸟}{niao3}
\end{entry}

\begin{entry}{屌丝}{diao3si1}{9,5}[Radicais ⼫、⼀]
  \definition{adj.}{panaca | zé-ninguém | (gíria de \emph{Internet}) \emph{looser}}
\end{entry}

\begin{entry}{钓鱼}{diao4yu2}{8,8}[Radicais ⾦、⿂]
  \definition{v.}{pescar (com linha e anzol) | (figurativo) aprisionar}
\end{entry}

\begin{entry}{调}{diao4}{10}[HSK 3][Radical ⾔]
  \definition{s.}{sotaque | nota (musical) | melodia; tom}
  \definition{v.}{transferir; deslocar; mover | distribuir; alocar | trocar; permutar; comutar}
  \seeref{调}{tiao2}
\end{entry}

\begin{entry}{调查}{diao4cha2}{10,9}[HSK 3][Radicais ⾔、⽊]
  \definition[项,个]{s.}{pesquisa; investigação}
  \definition{v.}{investigar; indagar; inquerir}
\end{entry}

\begin{entry}{掉}{diao4}{11}[HSK 2][Radical ⼿]
  \definition{v.}{cair | deixar cair}
\end{entry}

\begin{entry}{掉包}{diao4bao1}{11,5}[Radicais ⼿、⼓]
  \definition{v.}{vender uma falsificação pelo artigo genuíno | roubar o item valioso de alguém e substituí-lo por um item de aparência semelhante, mas sem valor}
\end{entry}

\begin{entry}{掉膘}{diao4biao1}{11,15}[Radicais ⼿、⾁]
  \definition{v.}{perder peso (gado)}
\end{entry}

\begin{entry}{掉队}{diao4dui4}{11,4}[Radicais ⼿、⾩]
  \definition{v.}{abandonar | ficar para trás}
\end{entry}

\begin{entry}{掉落}{diao4luo4}{11,12}[Radicais ⼿、⾋]
  \definition{v.}{derrubar}
\end{entry}

\begin{entry}{掉线}{diao4xian4}{11,8}[Radicais ⼿、⽷]
  \definition{v.}{desconectar-se (da \emph{Internet})}
\end{entry}

\begin{entry}{掉转}{diao4zhuan3}{11,8}[Radicais ⼿、⾞]
  \definition{v.}{dar a volta}
\end{entry}

\begin{entry}{叮嘱}{ding1zhu3}{5,15}[Radicais ⼝、⼝]
  \definition{v.}{exortar | avisar | insistir de novo e de novo}
\end{entry}

\begin{entry}{顶}{ding3}{8}[HSK 4][Radical ⾴]
  \definition{adv.}{muito (linguagem falada); a maioria; extremamente; expressa o grau mais alto, equivalente a ``最'' e ``极''}
  \definition{clas.}{para coisas que têm um topo}
  \definition{prep.}{até}
  \definition{s.}{coroa da cabeça; parte mais alta do corpo ou objeto | topo; limite superior; ponto mais alto}
  \definition{v.}{carregar na cabeça; carregar em sua cabeça | empurrar (ou apoiar) para cima; empurrar por baixo (ou por trás) |
dar cabeçadas; dar uma coronhada | sustentar; apoiar; suportar | resistir; ir contra; enfrentar | rebater; retorquir; responder de volta | cooperar; enfrentar; apoiar; dar suporte | igualar; ser equivalente a | substituir; tomar o lugar de | assumir o controle; transferir ou adquirir o direito de administrar um negócio ou alugar uma casa ou terreno}
  \seealsoref{极}{ji2}
  \seealsoref{最}{zui4}
\end{entry}

\begin{entry}{订}{ding4}{4}[HSK 3][Radical ⾔]
  \definition{v.}{concluir; elaborar; concordar com |assinar (um jornal, etc.); reservar (assentos, ingressos, etc.); encomendar (mercadorias, etc.) |fazer correções; revisar | grampear junto; unir}
\end{entry}

\begin{entry}{定}{ding4}{8}[HSK 4][Radical ⼧]
  \definition{adj.}{calmo; estável | fixo; estabelecido; fixado; inalterado}
  \definition{adv.}{certamente; com certeza; definitivamente}
  \definition{s.}{sobrenome Ding}
  \definition{v.}{decidir; fixar; definir; determinar; ter certeza | consertar; fazer com que seja consertado | acalmar; estabilizar; tornar estável | assinar (um jornal, etc.); reservar (assentos, ingressos, etc.); encomendar (mercadorias, etc.)}
\end{entry}

\begin{entry}{定期}{ding4qi1}{8,12}[HSK 3][Radicais ⼧、⽉]
  \definition{adj.}{regular; periódico; em intervalos regulares}
  \definition{v.}{fixar (definir) uma data}
\end{entry}

\begin{entry}{丢}{diu1}{6}[Radical ⼛]
  \definition{v.}{perder | perder-se}
\end{entry}

\begin{entry}{丢掉}{diu1diao4}{6,11}[Radicais ⼛、⼿]
  \definition{v.}{jogar fora | descartar |perder}
\end{entry}

\begin{entry}{丢官}{diu1guan1}{6,8}[Radicais ⼛、⼧]
  \definition{v.}{perder um cargo oficial}
\end{entry}

\begin{entry}{丢开}{diu1kai1}{6,4}[Radicais ⼛、⼶]
  \definition{v.}{jogar fora ou deixar de lado | esquecer por um tempo}
\end{entry}

\begin{entry}{丢脸}{diu1lian3}{6,11}[Radicais ⼛、⾁]
  \definition{adj.}{vergonhoso}
\end{entry}

\begin{entry}{丢弃}{diu1qi4}{6,7}[Radicais ⼛、⼶]
  \definition{v.}{jogar fora | descartar}
\end{entry}

\begin{entry}{丢失}{diu1shi1}{6,5}[Radicais ⼛、⼤]
  \definition{v.}{perder}
\end{entry}

\begin{entry}{丢下}{diu1xia4}{6,3}[Radicais ⼛、⼀]
  \definition{v.}{abandonar}
\end{entry}

\begin{entry}{东}{dong1}{5}[HSK 1][Radical ⼀]
  \definition*{s.}{sobrenome Dong}
  \definition{s.}{leste}
\end{entry}

\begin{entry}{东半球}{dong1ban4qiu2}{5,5,11}[Radicais ⼀、⼗、⽟]
  \definition*{s.}{Hemisfério Oriental}
\end{entry}

\begin{entry}{东北}{dong1 bei3}{5,5}[HSK 2][Radicais ⼀、⼔]
  \definition*{s.}{Nordeste da China | Manchúria}
  \definition{s.}{nordeste}
\end{entry}

\begin{entry}{东边}{dong1 bian5}{5,5}[HSK 1][Radicais ⼀、⾡]
  \definition{s.}{este | leste | lado leste | oriente}
\end{entry}

\begin{entry}{东部}{dong1 bu4}{5,10}[HSK 3][Radicais ⼀、⾢]
  \definition{s.}{o leste; parte oriental}
\end{entry}

\begin{entry}{东方}{dong1 fang1}{5,4}[HSK 2][Radicais ⼀、⽅]
  \definition*{s.}{sobrenome Dongfang}
  \definition{s.}{leste | oriente}
\end{entry}

\begin{entry}{东方学院}{dong1fang1 xue2yuan4}{5,4,8,9}[Radicais ⼀、⽅、⼦、⾩]
  \definition*{s.}{Instituto Oriental}
\end{entry}

\begin{entry}{东面}{dong1mian4}{5,9}[Radicais ⼀、⾯]
  \definition{s.}{lado leste (de algo)}
\end{entry}

\begin{entry}{东南}{dong1 nan2}{5,9}[HSK 2][Radicais ⼀、⼗]
  \definition{s.}{sudeste | sudeste da China | o Sudeste}
\end{entry}

\begin{entry}{东西}{dong1xi1}{5,6}[Radicais ⼀、⾑]
  \definition{s.}{leste e oeste}
  \seeref{东西}{dong1xi5}
\end{entry}

\begin{entry}{东西}{dong1xi5}{5,6}[HSK 1][Radicais ⼀、⾑]
  \definition[个,件]{s.}{coisa | material | pessoa}
  \seeref{东西}{dong1xi1}
\end{entry}

\begin{entry}{冬}{dong1}{5}[Radical ⼎]
  \definition*{s.}{sobrenome Dong}
  \definition{s.}{inverno}
\end{entry}

\begin{entry}{冬瓜}{dong1gua1}{5,5}[Radicais ⼎、⽠]
  \definition{s.}{melão de inverno}
\end{entry}

\begin{entry}{冬季}{dong1 ji4}{5,8}[HSK 4][Radicais ⼎、⼦]
  \definition[个]{s.}{inverno; o quarto trimestre do ano, habitualmente referido na China como o período de três meses entre o início do inverno e o início da primavera, e também referido como ``décimo, décimo primeiro e décimo segundo'' meses do calendário lunar}
\end{entry}

\begin{entry}{冬天}{dong1 tian1}{5,4}[HSK 2][Radicais ⼎、⼤]
  \definition{s.}{inverno}
\end{entry}

\begin{entry}{懂}{dong3}{15}[HSK 2][Radical ⼼]
  \definition{v.}{compreender | entender}
\end{entry}

\begin{entry}{懂得}{dong3 de5}{15,11}[HSK 2][Radicais ⼼、⼻]
  \definition{v.}{saber | entender | compreender}
\end{entry}

\begin{entry}{动}{dong4}{6}[HSK 1][Radical ⼒]
  \definition{v.}{mover | movimentar}
\end{entry}

\begin{entry}{动感}{dong4gan3}{6,13}[Radicais ⼒、⼼]
  \definition{adj.}{dinâmica | vívida}
  \definition{adv.}{dinamicamente}
  \definition{s.}{senso de movimento (geralmente em uma obra de arte estática)}
\end{entry}

\begin{entry}{动画片}{dong4hua4pian4}{6,8,4}[HSK 4][Radicais ⼒、⽥、⽚]
  \definition[部]{s.}{desenho animado; animações; filme de animação}
\end{entry}

\begin{entry}{动力}{dong4li4}{6,2}[Radicais ⼒、⼒]
  \definition{s.}{poder; força motriz | ímpeto; força motriz (ou propulsora)}
\end{entry}

\begin{entry}{动漫}{dong4man4}{6,14}[Radicais ⼒、⽔]
  \definition{s.}{desenhos animados | quadrinhos | anime | mangá}
\end{entry}

\begin{entry}{动人}{dong4 ren2}{6,2}[HSK 3][Radicais ⼒、⼈]
  \definition{adj.}{em movimento; tocando}
\end{entry}

\begin{entry}{动身}{dong4shen1}{6,7}[Radicais ⼒、⾝]
  \definition{v.+compl.}{fazer uma jornada | começar uma jornada | partir | partir em uma jornada | sair (para um lugar distante)}
\end{entry}

\begin{entry}{动物}{dong4wu4}{6,8}[HSK 2][Radicais ⼒、⽜]
  \definition[只,群,个]{s.}{animal}
\end{entry}

\begin{entry}{动物园}{dong4 wu4 yuan2}{6,8,7}[HSK 2][Radicais ⼒、⽜、⼞]
  \definition[个]{s.}{jardim zoológico | zoo}
\end{entry}

\begin{entry}{动摇}{dong4 yao2}{6,13}[HSK 4][Radicais ⼒、⼿]
  \definition{adj.}{instável}
  \definition{v.}{ondular; pairar; agitar; balançar; sacudir | hesitar; vacilar; esmorecer; abalar}
\end{entry}

\begin{entry}{动作}{dong4zuo4}{6,7}[HSK 1][Radicais ⼒、⼈]
  \definition[个]{s.}{movimento | ação}
  \definition{v.}{mover | agir}
\end{entry}

\begin{entry}{洞穴}{dong4xue2}{9,5}[Radicais ⽔、⽳]
  \definition{s.}{caverna}
\end{entry}

\begin{entry}{都}{dou1}{10}[HSK 1][Radical ⾢]
  \definition{adv.}{todos | ambos | inteiramente | até | já (usado para dar ênfase) | (não) em tudo}
  \seeref{都}{du1}
\end{entry}

\begin{entry}{豆腐}{dou4fu5}{7,14}[HSK 4][Radicais ⾖、⾁]
  \definition[块,盒,斤,盘,锅]{s.}{\emph{tofu}}
\end{entry}

\begin{entry}{豆荚}{dou4jia2}{7,9}[Radicais ⾖、⾋]
  \definition{s.}{vagem (de legumes)}
\end{entry}

\begin{entry}{豆角}{dou4jiao3}{7,7}[Radicais ⾖、⾓]
  \definition{s.}{feijão verde}
\end{entry}

\begin{entry}{读}{dou4}{10}[Radical ⾔]
  \definition{s.}{vírgula | frase marcada por pausa}
  \seeref{读}{du2}
\end{entry}

\begin{entry}{都}{du1}{10}[Radical ⾢]
  \definition*{s.}{sobrenome Du}
  \definition{s.}{capital | metrópole}
  \seeref{都}{dou1}
\end{entry}

\begin{entry}{嘟}{du1}{13}[Radical ⼝]
  \definition{s.}{buzina | bip}
  \definition{v.}{fazer beicinho}
\end{entry}

\begin{entry}{毒}{du2}{9}[Radical ⽏]
  \definition{adj.}{venenoso | tóxico}
  \definition{s.}{veneno | tóxico}
  \definition{v.}{intoxicar}
\end{entry}

\begin{entry}{毒害}{du2hai4}{9,10}[Radicais ⽏、⼧]
  \definition{s.}{envenenamento}
  \definition{v.}{envenenar (prejudicar com uma substância tóxica) | envenenar (as mentes das pessoas)}
\end{entry}

\begin{entry}{毒杀}{du2sha1}{9,6}[Radicais ⽏、⽊]
  \definition{v.}{matar por envenenamento}
\end{entry}

\begin{entry}{毒蛇}{du2she2}{9,11}[Radicais ⽏、⾍]
  \definition{s.}{víbora | cobra venenosa}
\end{entry}

\begin{entry}{毒物}{du2wu4}{9,8}[Radicais ⽏、⽜]
  \definition{s.}{substância venenosa | toxina}
\end{entry}

\begin{entry}{独}{du2}{9}[Radical ⽝]
  \definition{adj.}{sozinho | solitário | solteiro}
  \definition{adv.}{apenas}
\end{entry}

\begin{entry}{独立}{du2li4}{9,5}[HSK 4][Radicais ⽝、⽴]
  \definition{adj.}{independente; por conta própria | separado; respectivo; descreve algo que é separado e não está em contato com outra coisa}
  \definition{prep.}{independente de; separado de; não mais anexado à unidade original, mas uma unidade separada}
  \definition{v.}{ficar sozinho | alcançar a independência; tornar-se um país independente; liberdade de um Estado, regime ou organização contra interferência, controle e dominação por forças externas}
\end{entry}

\begin{entry}{独特}{du2te4}{9,10}[HSK 4][Radicais ⽝、⽜]
  \definition{adj.}{único; distinto; original; especial}
\end{entry}

\begin{entry}{独自}{du2 zi4}{9,6}[HSK 4][Radicais ⽝、⾃]
  \definition{adj.}{sozinho; por si mesmo; por conta própria}
\end{entry}

\begin{entry}{读}{du2}{10}[HSK 1][Radical ⾔]
  \definition{v.}{ler em voz alta | ler | frequentar (escola) | estudar (uma matéria na escola) | pronunciar}
  \seeref{读}{dou4}
\end{entry}

\begin{entry}{读书}{du2 shu1}{10,4}[HSK 1][Radicais ⾔、⼄]
  \definition{v.+compl.}{ler | estudar | frequentar a escola}
\end{entry}

\begin{entry}{读音}{du2 yin1}{10,9}[HSK 2][Radicais ⾔、⾳]
  \definition{s.}{pronúncia}
\end{entry}

\begin{entry}{读者}{du2 zhe3}{10,8}[HSK 3][Radicais ⾔、⽼]
  \definition[个,位]{s.}{leitor}
\end{entry}

\begin{entry}{肚}{du3}{7}[Radical ⾁]
  \definition{s.}{tripas | entranhas}
  \seeref{肚}{du4}
\end{entry}

\begin{entry}{堵}{du3}{11}[HSK 4][Radical ⼟]
  \definition{adj.}{asfixiado; abafado; sufocado; oprimido}
  \definition{clas.}{para paredes}
  \definition{s.}{parede}
  \definition{s.}{sobrenome Du}
  \definition{v.}{impedir; bloquear}
\end{entry}

\begin{entry}{堵车}{du3che1}{11,4}[HSK 4][Radicais ⼟、⾞]
  \definition{v.}{congestionar (trânsito)}
  \definition{v.+compl.}{congestionamento; tráfego intenso; ficar congestionado (no tráfego); bloqueio de vias devido ao excesso de tráfego, etc.}
\end{entry}

\begin{entry}{杜鹃}{du4juan1}{7,12}[Radicais ⽊、⿃]
  \definition{s.}{cuco (pássaro)}
  \seealsoref{布谷鸟}{bu4gu3niao3}
  \seealsoref{杜鹃鸟}{du4juan1niao3}
  \seealsoref{杜宇}{du4yu3}
\end{entry}

\begin{entry}{杜鹃鸟}{du4juan1niao3}{7,12,5}[Radicais ⽊、⿃、⿃]
  \definition{s.}{cuco (pássaro)}
  \seealsoref{布谷鸟}{bu4gu3niao3}
  \seealsoref{杜鹃}{du4juan1}
  \seealsoref{杜宇}{du4yu3}
\end{entry}

\begin{entry}{杜宇}{du4yu3}{7,6}[Radicais ⽊、⼧]
  \definition{s.}{cuco (pássaro)}
  \seealsoref{布谷鸟}{bu4gu3niao3}
  \seealsoref{杜鹃}{du4juan1}
  \seealsoref{杜鹃鸟}{du4juan1niao3}
\end{entry}

\begin{entry}{肚}{du4}{7}[Radical ⾁]
  \definition{s.}{barriga}
  \seeref{肚}{du3}
\end{entry}

\begin{entry}{肚子}{du4zi5}{7,3}[HSK 4][Radicais ⾁、⼦]
  \definition[个,只]{s.}{abdômen; barriguinha; ventre; barriga}
\end{entry}

\begin{entry}{度}{du4}{9}[HSK 2][Radical ⼴]
  \definition{clas.}{para temperatura, etc. | para eventos e ocorrências}
  \definition{s.}{grau (ângulo, temperatura, etc.) | kilowatt-hora}
  \seeref{度}{duo2}
\end{entry}

\begin{entry}{度过}{du4guo4}{9,6}[HSK 4][Radicais ⼴、⾡]
  \definition{s.}{passar o tempo; fazer o tempo desaparecer no trabalho, na vida, no lazer e no descanso}
\end{entry}

\begin{entry}{渡过}{du4guo4}{12,6}[Radicais ⽔、⾡]
  \definition{v.}{atravessar | passar por}
\end{entry}

\begin{entry}{镀金}{du4jin1}{14,8}[Radicais ⾦、⾦]
  \definition{v.}{banhar a ouro | dourar | (figurativo) fazer algo muito comum parecer especial}
\end{entry}

\begin{entry}{端午节}{duan1wu3jie2}{14,4,5}[Radicais ⽴、⼗、⾋]
  \definition*{s.}{Festa do Duplo Cinco, Festival dos Barcos-Dragão (5º~dia do quinto mês lunar)}
\end{entry}

\begin{entry}{短}{duan3}{12}[HSK 2][Radical ⽮]
  \definition{adj.}{curto | breve}
\end{entry}

\begin{entry}{短处}{duan3 chu4}{12,5}[HSK 3][Radicais ⽮、⼡]
  \definition{s.}{deficiência; ponto fraco; defeito; fraqueza}
\end{entry}

\begin{entry}{短促}{duan3cu4}{12,9}[Radicais ⽮、⼈]
  \definition{adj.}{curto (tom de voz) | fugaz | ofegante (respiração) | curto no tempo}
\end{entry}

\begin{entry}{短裤}{duan3 ku4}{12,12}[HSK 3][Radicais ⽮、⾐]
  \definition[条]{s.}{calças curtas; calção; \emph{shorts}}
\end{entry}

\begin{entry}{短跑}{duan3pao3}{12,12}[Radicais ⽮、⾜]
  \definition{s.}{corrida}
\end{entry}

\begin{entry}{短期}{duan3 qi1}{12,12}[HSK 3][Radicais ⽮、⽉]
  \definition{adj.}{curto prazo}
  \definition[个]{s.}{curto período}
\end{entry}

\begin{entry}{短缺}{duan3que1}{12,10}[Radicais ⽮、⽸]
  \definition{s.}{escassez}
\end{entry}

\begin{entry}{短少}{duan3shao3}{12,4}[Radicais ⽮、⼩]
  \definition{v.}{estar aquém do valor total}
\end{entry}

\begin{entry}{短视}{duan3shi4}{12,8}[Radicais ⽮、⾒]
  \definition{adj.}{míope}
\end{entry}

\begin{entry}{短信}{duan3xin4}{12,9}[HSK 2][Radicais ⽮、⼈]
  \definition{s.}{mensagem de texto}
\end{entry}

\begin{entry}{短暂}{duan3zan4}{12,12}[Radicais ⽮、⽇]
  \definition{adj.}{momentâneo | de curta duração}
\end{entry}

\begin{entry}{段}{duan4}{9}[HSK 2][Radical ⽎]
  \definition*{s.}{sobrenome Duan}
  \definition{clas.}{para histórias, períodos de tempo, desenvolvimento de um tópico, etc.}
  \definition{s.}{parágrafo | seção | segmento | estágio (de um processo)}
\end{entry}

\begin{entry}{断}{duan4}{11}[HSK 3][Radical ⽄]
  \definition*{s.}{sobrenome Duan}
  \definition{adv.}{absolutamente; decididamente}
  \definition{v.}{quebrar; estalar | interromper; cortar; parar | desistir; abster-se de | julgar; decidir}
\end{entry}

\begin{entry}{断交}{duan4jiao1}{11,6}[Radicais ⽄、⼇]
  \definition{v.+compl.}{terminar uma amizade | romper relações diplomáticas}
\end{entry}

\begin{entry}{锻炼}{duan4lian4}{14,9}[HSK 4][Radicais ⾦、⽕]
  \definition{v.}{exercitar-se; fazer (ou fazer) exercícios; submeter-se a treinamento físico; fortalecer o corpo por meio do esporte | fortalecer; endurecer; aprimorar as habilidades de trabalho e de vida por meio de trabalho e outras atividades | forjar ou moldar metal para torná-lo mais refinado; refere-se à transformação de materiais metálicos em objetos de determinada forma e tamanho por meio de aquecimento, batimento, prensagem etc.}
\end{entry}

\begin{entry}{队}{dui4}{4}[HSK 2][Radical ⾩]
  \definition[个]{s.}{esquadrão | equipe | grupo}
\end{entry}

\begin{entry}{队友}{dui4you3}{4,4}[Radicais ⾩、⼜]
  \definition{s.}{companheiro de equipe}
\end{entry}

\begin{entry}{队员}{dui4 yuan2}{4,7}[HSK 3][Radicais ⾩、⼝]
  \definition{s.}{membro da equipe}
\end{entry}

\begin{entry}{队长}{dui4 zhang3}{4,4}[HSK 2][Radicais ⾩、⾧]
  \definition{s.}{capitão (de equipe) | líder da equipe}
\end{entry}

\begin{entry}{对}{dui4}{5}[HSK 1,2][Radical ⼨]
  \definition{adj.}{correto | sim}
  \definition{clas.}{para casais}
  \definition{prep.}{com | para | para com}
\end{entry}

\begin{entry}{对比}{dui4bi3}{5,4}[HSK 4][Radicais ⼨、⽐]
  \definition{s.}{razão; proporção | contraste; comparação; diferenças ou lacunas encontradas após comparação}
  \definition{v.}{contrastar; comparar}
\end{entry}

\begin{entry}{对不起}{dui4bu5qi3}{5,4,10}[HSK 1][Radicais ⼨、⼀、⾛]
  \definition{interj.}{Desculpe! | Desculpe-me! | Perdoe-me! | Desculpe? (por favor, repita)}
  \definition{v.}{desculpar | pedir desculpas | perdoar}
\end{entry}

\begin{entry}{对待}{dui4dai4}{5,9}[HSK 3][Radicais ⼨、⼻]
  \definition{v.}{tratar; abordar; manusear; estar em uma posição relacionada ou comparada a outra}
\end{entry}

\begin{entry}{对得起}{dui4de5qi3}{5,11,10}[Radicais ⼨、⼻、⾛]
  \definition{v.}{não decepcionar alguém | tratar alguém de maneira justa | ser digno de}
\end{entry}

\begin{entry}{对方}{dui4fang1}{5,4}[HSK 3][Radicais ⼨、⽅]
  \definition{s.}{outro lado; lado oposto; outra parte}
\end{entry}

\begin{entry}{对付}{dui4fu5}{5,5}[HSK 4][Radicais ⼨、⼈]
  \definition{adj.}{em bons termos; estar em termos agradáveis ​​(frequentemente usado em negativas); dialeto usado para descrever duas pessoas que têm um bom relacionamento e se dão bem, frequentemente usado para negar}
  \definition{v.}{enfrentar; tratar; lidar com | fazer acontecer; (informal) fazer algo que você não quer fazer; aceitar algo que você não gosta}
\end{entry}

\begin{entry}{对……感兴趣}{dui4 gan3xing4qu4}{5,13,6,15}[Radicais ⼨、⼼、⼋、⾛]
  \definition{expr.}{estar interessado em\dots | ter interesse em\dots | interessar-se por\dots}
  \seeref{对……有兴趣}{dui4 you3xing4qu4}
\end{entry}

\begin{entry}{对话}{dui4hua4}{5,8}[HSK 2][Radicais ⼨、⾔]
  \definition[个]{s.}{diálogo | conversa}
  \definition{v.}{dialogar | conversar}
\end{entry}

\begin{entry}{对面}{dui4mian4}{5,9}[HSK 2][Radicais ⼨、⾯]
  \definition{s.}{lado oposto}
\end{entry}

\begin{entry}{对手}{dui4shou3}{5,4}[HSK 3][Radicais ⼨、⼿]
  \definition[个]{s.}{oponente; adversário}
\end{entry}

\begin{entry}{对……熟悉}{dui4 shu2xi1}{5,15,11}[Radicais ⼨、⽕、⼼]
  \definition{expr.}{estar familiarizado com\dots}
\end{entry}

\begin{entry}{对……说}{dui4 shuo5}{5,9}[Radicais ⼨、⾔]
  \definition{v.}{dizer a alguém}
\end{entry}

\begin{entry}{对象}{dui4xiang4}{5,11}[HSK 3][Radicais ⼨、⾗]
  \definition[个]{s.}{alvo; objeto | parceiro; namorado; namorada}
\end{entry}

\begin{entry}{对……有兴趣}{dui4 you3xing4qu4}{5,6,6,15}[Radicais ⼨、⽉、⼋、⾛]
  \definition{expr.}{estar interessado em\dots | ter interesse em\dots | interessar-se por\dots}
  \seeref{对……感兴趣}{dui4 gan3xing4qu4}
\end{entry}

\begin{entry}{对于}{dui4yu2}{5,3}[HSK 4][Radicais ⼨、⼆]
  \definition{prep.}{para; relativo a; no que diz respeito a; a respeito de}
\end{entry}

\begin{entry}{蹲下}{dun1xia4}{19,3}[Radicais ⾜、⼀]
  \definition{v.}{agachar | agachar-se}
\end{entry}

\begin{entry}{顿}{dun4}{10}[HSK 3][Radical ⾴]
  \definition*{s.}{sobrenome Dun}
  \definition{adj.}{cansado; fatigado}
  \definition{adv.}{de repente; imediatamente}
  \definition{clas.}{para refeições | para surras, repreensões, etc.}
  \definition{s.}{um lugar para ficar}
  \definition{v.}{pausar
pausar na escrita para reforçar o início ou o fim de um traço
tocar o chão (com a cabeça)
pisar (o pé)
resolver; arranjar
montar acampamento; ficar temporariamente}
\end{entry}

\begin{entry}{多}{duo1}{6}[HSK 1,2][Radical ⼣]
  \definition{adv.}{muitos | muito | muitas vezes | um monte de | numerosos | mais | em excesso | como (até que ponto)}
  \definition{num.}{(após um número) ímpar}
  \definition{pref.}{multi | poli}
\end{entry}

\begin{entry}{多重}{duo1chong2}{6,9}[Radicais ⼣、⾥]
  \definition{pref.}{multi (facetado, cultural, étnico, etc.)}
\end{entry}

\begin{entry}{多次}{duo1 ci4}{6,6}[HSK 4][Radicais ⼣、⽋]
  \definition{adv.}{muitas vezes; de vez em quando; repetidamente; em muitas ocasiões}
\end{entry}

\begin{entry}{多大}{duo1da4}{6,3}[Radicais ⼣、⼤]
  \definition{adj.}{quantos anos? | que idade? | quão grande?}
\end{entry}

\begin{entry}{多久}{duo1 jiu3}{6,3}[HSK 2][Radicais ⼣、⼃]
  \definition{pron.}{quanto tempo? | quanto tempo}
\end{entry}

\begin{entry}{多么}{duo1me5}{6,3}[HSK 2][Radicais ⼣、⼃]
  \definition{adv.}{(em exclamações) como |  o que | até que ponto | em uma extensão não especificada | como (usado em uma frase interrogativa para perguntar sobre grau ou número)}
\end{entry}

\begin{entry}{多年}{duo1 nian2}{6,6}[HSK 4][Radicais ⼣、⼲]
  \definition{adv.}{por muitos anos; durante muitos anos}
\end{entry}

\begin{entry}{多少}{duo1shao3}{6,4}[Radicais ⼣、⼩]
  \definition{num.}{número | quantidade | um pouco}
  \seeref{多少}{duo1shao5}
\end{entry}

\begin{entry}{多少}{duo1shao5}{6,4}[HSK 1][Radicais ⼣、⼩]
  \definition{adv.}{quanto? | quantos? | (número de telefone, ID de estudante, etc.) qual o número?}
  \seeref{多少}{duo1shao3}
\end{entry}

\begin{entry}{多数}{duo1 shu4}{6,13}[HSK 2][Radicais ⼣、⽁]
  \definition{adj.}{maioria | plural}
  \definition{pref.}{pluri-}
\end{entry}

\begin{entry}{多样}{duo1 yang4}{6,10}[HSK 4][Radicais ⼣、⽊]
  \definition{adj.}{diversos; variados; diversificado}
  \definition{s.}{diversidade}
\end{entry}

\begin{entry}{多云}{duo1 yun2}{6,4}[HSK 2][Radicais ⼣、⼆]
  \definition{adj.}{céu nublado}
\end{entry}

\begin{entry}{多种}{duo1 zhong3}{6,9}[HSK 4][Radicais ⼣、⽲]
  \definition{adj.}{diverso; vários tipos de; múltiplo; diversificado}
\end{entry}

\begin{entry}{夺冠}{duo2guan4}{6,9}[Radicais ⼤、⼍]
  \definition{v.}{apoderar-se da coroa | (fig.) ganhar um campeonato | ganhar a medalha de ouro}
\end{entry}

\begin{entry}{度}{duo2}{9}[Radical ⼴]
  \definition{v.}{estimar}
  \seeref{度}{du4}
\end{entry}

\begin{entry}{躲}{duo3}{13}[Radical ⾝]
  \definition{v.}{esconder | esquivar | evitar}
\end{entry}

\begin{entry}{躲闪}{duo3shan3}{13,5}[Radicais ⾝、⾨]
  \definition{v.}{desviar | evadir | esquivar (para fora do caminho)}
\end{entry}

%%%%% EOF %%%%%


%%%
%%% E
%%%

\section*{E}\addcontentsline{toc}{section}{E}

\begin{entry}{阿}{e1}{7}[Radical ⾩]
  \definition{adj.}{gracioso}
  \definition{pron.}{monte grande | canto, esquina}
  \definition{v.}{jogar | agradar | atender | ser injustamente parcial com | ser dobrado}
  \seeref{阿}{a1}
\end{entry}

\begin{entry}{俄}{e2}{9}[Radical ⼈]
  \definition*{s.}{Rússia, abreviação de 俄罗斯}
  \seealsoref{俄罗斯}{e2luo2si1}
\end{entry}

\begin{entry}{俄罗斯}{e2luo2si1}{9,8,12}[Radicais ⼈、⽹、⽄]
  \definition*{s.}{Rússia}
\end{entry}

\begin{entry}{俄罗斯人}{e2luo2si1ren2}{9,8,12,2}[Radicais ⼈、⽹、⽄、⼈]
  \definition{s.}{russo | pessoa ou povo da Rússia}
\end{entry}

\begin{entry}{哦}{e2}{10}[Radical ⼝]
  \definition{v.}{entoar cântico}
  \seeref{哦}{o2}
  \seeref{哦}{o4}
  \seeref{哦}{o5}
\end{entry}

\begin{entry}{恶心}{e3xin1}{10,4}[HSK 4][Radicais ⼼、⼼]
  \definition{adj.}{nauseante; repugnante}
  \definition{s.}{enjoo; náusea; repugnância; sensação de enjoo; vontade de vomitar}
  \definition{v.}{repugnar; ser nauseante; vomitar}
  \seeref{恶心}{e4xin1}
\end{entry}

\begin{entry}{恶心}{e4xin1}{10,4}[Radicais ⼼、⼼]
  \definition{s.}{mau hábito | hábito vicioso | vício}
  \seeref{恶心}{e3xin1}
\end{entry}

\begin{entry}{饿}{e4}{10}[HSK 1][Radical ⾷]
  \definition{adj.}{faminto}
  \definition{s.}{fome}
  \definition{v.}{morrer de fome}
\end{entry}

\begin{entry}{鳄鱼}{e4yu2}{17,8}[Radicais ⿂、⿂]
  \definition[条]{s.}{jacaré | crocodilo}
\end{entry}

\begin{entry}{恩赐}{en1ci4}{10,12}[Radicais ⼼、⾙]
  \definition{s.}{favor | caridade}
  \definition{v.}{conceder (favor, caridade)}
\end{entry}

\begin{entry}{儿}{er2}{2}[Kangxi 10][Radical ⼉]
  \definition{s.}{criança | filho}
  \seeref{儿}{r5}
  \seeref{儿}{ren2}
\end{entry}

\begin{entry}{儿童}{er2tong2}{2,12}[HSK 4][Radicais ⼉、⽴]
  \definition[个,群]{s.}{criança; menor de idade (mais jovem do que ``少年'')}
  \seealsoref{少年}{shao4nian2}
\end{entry}

\begin{entry}{儿媳}{er2xi2}{2,13}[Radicais ⼉、⼥]
  \definition{s.}{esposa do filho}
\end{entry}

\begin{entry}{儿子}{er2zi5}{2,3}[Radicais ⼉、⼦]
  \definition{s.}{filho}
  \seealsoref{女儿}{nv3'er2}
\end{entry}

\begin{entry}{而}{er2}{6}[HSK 4][Kangxi 126][Radical ⽽]
  \definition{conj.}{e (coordenação) | e ainda (restrição) | conexão de componentes com continuidade semântica | conecxão de componentes afirmativos e negativos que se complementam | conexão de componentes com significados opostos para indicar um contraste |  conexão de componentes de causa e efeito no raciocínio | significa “chegar” ou “alcançar” | conexão de componentes que indicam tempo ou modo ao verbo | inserido entre o sujeito e o predicado, significa ``如果'' (se)}
  \seealsoref{如果}{ru2guo3}
\end{entry}

\begin{entry}{而况}{er2kuang4}{6,7}[Radicais ⽽、⼎]
  \definition{conj.}{além disso | além do mais}
\end{entry}

\begin{entry}{而且}{er2 qie3}{6,5}[HSK 2][Radicais ⽽、⼀]
  \definition{conj.}{muito menos | além disso | além do mais}
\end{entry}

\begin{entry}{而是}{er2 shi4}{6,9}[HSK 4][Radicais ⽽、⽇]
  \definition{conj.}{mas; em vez disso; geralmente usada em conjunto com ``不是'' para formar o correlativo ``不是……而是'', indicando uma relação paralela}
  \seealsoref{不是……而是}{bu4shi4 er2 shi4}
\end{entry}

\begin{entry}{耳朵}{er3duo5}{6,6}[Radicais ⽿、⽊]
  \definition[只,个,对]{s.}{orelha}
\end{entry}

\begin{entry}{耳机}{er3 ji1}{6,6}[HSK 4][Radicais ⽿、⽊]
  \definition[副,个,对]{s.}{fone de ouvido; receptor (de telefone); dispositivos que permitem que uma pessoa ouça sons sozinha, como ouvir música, histórias, chamadas telefônicas etc., usados na cabeça ou inseridos nos ouvidos}
\end{entry}

\begin{entry}{二}{er4}{2}[HSK 1][Kangxi 7][Radical ⼆]
  \definition{num.}{dois; 2 | (dialeto de Pequim) estúpido}
\end{entry}

\begin{entry}{二手}{er4 shou3}{2,4}[HSK 4][Radicais ⼆、⼿]
  \definition{adj.}{usado; de segunda mão; refere-se especificamente a usados e revendidos}
\end{entry}

\begin{entry}{二战}{er4zhan4}{2,9}[Radicais ⼆、⼽]
  \definition*{s.}{Segunda Guerra Mundial}
\end{entry}

%%%%% EOF %%%%%


%%%
%%% F
%%%

\section*{F}\addcontentsline{toc}{section}{F}

\begin{entry}{发}{fa1}{5}[Radical ⼜][HSK 2]
  \definition{clas.}{para tiros (rodadas)}
  \definition{v.}{enviar | mandar}
  \seeref{发}{fa4}
\end{entry}

\begin{entry}{发表}{fa1biao3}{5,8}[HSK 3]
  \definition{v.}{publicar; entregar; emitir; expressar; anunciar | publicar}
\end{entry}

\begin{entry}{发财}{fa1cai2}{5,7}
  \definition{v.+compl.}{ficar rico | fazer fortuna}
\end{entry}

\begin{entry}{发愁}{fa1chou2}{5,13}
  \definition{v.+compl.}{preocupar-se | ficar ansioso | ficar triste}
\end{entry}

\begin{entry}{发出}{fa1 chu1}{5,5}[HSK 3]
  \definition{v.}{fazer; produzir; deixar sair | emitir; anunciar | enviar; partir | dar; emitir}
\end{entry}

\begin{entry}{发达}{fa1da2}{5,6}[HSK 3]
  \definition{adj.}{desenvolvido; florescente}
  \definition{v.}{desenvolver; promover; florescer}
\end{entry}

\begin{entry}{发动}{fa1dong4}{5,6}[HSK 3]
  \definition{v.}{iniciar; lançar; ligar motor; dar a partida (motor de combustão interna) | chamar à ação; mobilizar; estimular; despertar}
\end{entry}

\begin{entry}{发动机}{fa1dong4ji1}{5,6,6}
  \definition[台]{s.}{motor}
\end{entry}

\begin{entry}{发抖}{fa1dou3}{5,7}
  \definition{v.}{tremer | sacudir | estremecer}
\end{entry}

\begin{entry}{发明}{fa1ming2}{5,8}[HSK 3]
  \definition[个]{s.}{invenção}
  \definition{v.}{inventar | expor; explicar}
\end{entry}

\begin{entry}{发明者}{fa1ming2zhe3}{5,8,8}
  \definition{s.}{inventor}
\end{entry}

\begin{entry}{发票}{fa1piao4}{5,11}
  \definition{s.}{fatura | recibo | conta}
\end{entry}

\begin{entry}{发烧}{fa1shao1}{5,10}
  \definition{v.}{ter febre}
\end{entry}

\begin{entry}{发生}{fa1sheng1}{5,5}[HSK 3]
  \definition{v.}{ocorrer; acontecer; tomar lugar}
\end{entry}

\begin{entry}{发送}{fa1 song4}{5,9}[HSK 3]
  \definition{v.}{enviar; despachar | transmitir; enviar}
\end{entry}

\begin{entry}{发现}{fa1xian4}{5,8}[HSK 2]
  \definition{s.}{descoberta}
  \definition{v.}{perceber, tornar-se ciente de | descobrir, encontrar, detectar}
\end{entry}

\begin{entry}{发现者}{fa1xian4 zhe3}{5,8,8}
  \definition{s.}{descobridor}
\end{entry}

\begin{entry}{发言}{fa1yan2}{5,7}[HSK 3]
  \definition[个]{s.}{discurso; declaração; palestra}
  \definition{v.+compl.}{falar; fazer uma declaração (discurso)}
\end{entry}

\begin{entry}{发音}{fa1yin1}{5,9}
  \definition{s.}{pronúncia}
  \definition{v.}{pronunciar}
\end{entry}

\begin{entry}{发展}{fa1zhan3}{5,10}[HSK 3]
  \definition{s.}{desenvolvimento}
  \definition{v.}{crescer; expandir; avançar; desenvolver | recrutar; expandir; admitir}
\end{entry}

\begin{entry}{罚}{fa2}{9}[Radical 网]
  \definition{v.}{castigar | punir}
\end{entry}

\begin{entry}{罚款}{fa2kuan3}{9,12}
  \definition{s.}{multa (monetária) | pena}
  \definition{v.+compl.}{aplicar uma multa | multar}
\end{entry}

\begin{entry}{筏}{fa2}{12}[Radical 竹]
  \definition{s.}{jangada (de troncos, bambus, etc.)}
\end{entry}

\begin{entry}{法}{fa3}{8}[Radical 水]
  \definition*{s.}{França, abreviação de~法国}
  \seealsoref{法国}{fa3guo2}
\end{entry}

\begin{entry}{法国}{fa3guo2}{8,8}
  \definition*{s.}{França}
\end{entry}

\begin{entry}{法国人}{fa3guo2ren2}{8,8,2}
  \definition{s.}{francês | pessoa ou povo da França}
\end{entry}

\begin{entry}{法网}{fa3wang3}{8,6}
  \definition*{s.}{Torneio de Roland Garros (French Open), torneio de tênis}
\end{entry}

\begin{entry}{法文}{fa3wen2}{8,4}
  \definition*{s.}{françês, língua francesa}
\end{entry}

\begin{entry}{法语}{fa3yu3}{8,9}
  \definition{s.}{françês, língua francesa}
\end{entry}

\begin{entry}{发}{fa4}{5}[Radical ⼜]
  \definition{s.}{cabelo}
  \seeref{发}{fa1}
\end{entry}

\begin{entry}{发型}{fa4xing2}{5,9}
  \definition{s.}{penteado}
\end{entry}

\begin{entry}{发簪}{fa4zan1}{5,18}
  \definition{s.}{grampo de cabelo}
\end{entry}

\begin{entry}{番茄}{fan1qie2}{12,8}
  \definition{s.}{tomate}
\end{entry}

\begin{entry}{蕃茄}{fan1qie2}{15,8}
  \variantof{番茄}
\end{entry}

\begin{entry}{翻过}{fan1guo4}{18,6}
  \definition{v.}{virar |  transformar}
\end{entry}

\begin{entry}{翻脸}{fan1lian3}{18,11}
  \definition{v.+compl.}{brigar com alguém | tornar-se hostil}
\end{entry}

\begin{entry}{翻译}{fan1yi4}{18,7}
  \definition[个,位,名]{s.}{tradução | tradutor | interpretação | intérprete}
  \definition{v.}{traduzir; interpretar}
\end{entry}

\begin{entry}{反对}{fan3dui4}{4,5}[HSK 3]
  \definition{v.}{lutar; opor-se; objetar a; ser contra}
\end{entry}

\begin{entry}{反对党}{fan3dui4dang3}{4,5,10}
  \definition{s.}{partido de oposição}
\end{entry}

\begin{entry}{反对派}{fan3dui4pai4}{4,5,9}
  \definition{s.}{facção de oposição}
\end{entry}

\begin{entry}{反对票}{fan3dui4piao4}{4,5,11}
  \definition{s.}{voto dissidente}
\end{entry}

\begin{entry}{反复}{fan3fu4}{4,9}[HSK 3]
  \definition{adv.}{repetidamente; de ​​novo e de novo}
  \definition{s.}{reversão; recaída}
  \definition{v.}{recuar; cortar e mudar}
\end{entry}

\begin{entry}{反省}{fan3xing3}{4,9}
  \definition{v.}{examinar a consciência | questionar-se | refletir sobre si mesmo | sondar a alma}
\end{entry}

\begin{entry}{反应}{fan3ying4}{4,7}[HSK 3]
  \definition[个]{s.}{reação; resposta}
  \definition{v.}{reagir; responder}
\end{entry}

\begin{entry}{反正}{fan3zheng4}{4,5}[HSK 3]
  \definition{adv.}{de qualquer forma | tudo igual; em qualquer caso}
\end{entry}

\begin{entry}{犯法}{fan4fa3}{5,8}
  \definition{v.}{violar (quebrar) a lei}
\end{entry}

\begin{entry}{犯罪}{fan4zui4}{5,13}
  \definition{v.+compl.}{cometer  um crime (uma ofensa)}
\end{entry}

\begin{entry}{饭}{fan4}{7}[Radical 食][HSK 1]
  \definition[碗]{s.}{arroz cozido}
  \definition[顿]{s.}{refeição}
  \definition{s.}{(empréstimo linguístico) fã, devoto}
\end{entry}

\begin{entry}{饭店}{fan4dian4}{7,8}[HSK 1]
  \definition[家,个]{s.}{restaurante | hotel}
\end{entry}

\begin{entry}{饭馆}{fan4 guan3}{7,11}[HSK 2]
  \definition[家,个]{s.}{restaurante | lanchonete}
\end{entry}

\begin{entry}{范围}{fan4wei2}{9,7}[HSK 3]
  \definition[个]{s.}{escopo; limite; alcance}
  \definition{v.}{estabelecer limites para; limitar o escopo de}
\end{entry}

\begin{entry}{方案}{fang1'an4}{4,10}
  \definition[个,套]{s.}{plano | programa (para uma ação, etc.) | proposta | proposta de projeto de lei}
\end{entry}

\begin{entry}{方便}{fang1bian4}{4,9}[HSK 2]
  \definition{adj.}{conveniente | adequado}
  \definition{v.}{facilitar, facilitar as coisas | ter dinheiro de sobra | (eufemismo) aliviar-se}
\end{entry}

\begin{entry}{方便面}{fang1 bian4 mian4}{4,9,9}[HSK 2]
  \definition{s.}{macarrão instantâneo}
\end{entry}

\begin{entry}{方法}{fang1fa3}{4,8}[HSK 2]
  \definition[个]{s.}{método | meio}
\end{entry}

\begin{entry}{方面}{fang1mian4}{4,9}[HSK 2]
  \definition[个]{s.}{lado | campo | aspecto}
\end{entry}

\begin{entry}{方式}{fang1shi4}{4,6}[HSK 3]
  \definition[种,个]{s.}{maneira; método}
\end{entry}

\begin{entry}{方向}{fang1xiang4}{4,6}[HSK 2]
  \definition[个]{s.}{direção | orientação | alvo | meta | objetivo}
\end{entry}

\begin{entry}{方言}{fang1yan2}{4,7}
  \definition*{s.}{o primeiro dicionário de dialeto chinês, editado por Yang Xiong 扬雄 no século I, contendo mais de 9.000 caracteres}
  \definition{s.}{dialeto}
  \seealsoref{扬雄}{yang2xiong2}
\end{entry}

\begin{entry}{防}{fang2}{6}[HSK 3]
  \definition*{s.}{sobrenome Fang}
  \definition{s.}{defesa | barragem; dique; aterro}
  \definition{v.}{prover contra; defender contra; proteger contra}
\end{entry}

\begin{entry}{防护}{fang2hu4}{6,7}
  \definition{v.}{defender | proteger}
\end{entry}

\begin{entry}{防晒}{fang2shai4}{6,10}
  \definition{s.}{protetor solar}
\end{entry}

\begin{entry}{防止}{fang2zhi3}{6,4}[HSK 3]
  \definition{v.}{evitar; prevenir; prevenir; proteger contra}
\end{entry}

\begin{entry}{房东}{fang2dong1}{8,5}[HSK 3]
  \definition[个,位]{s.}{dono;  proprietário; senhorio}
\end{entry}

\begin{entry}{房间}{fang2jian1}{8,7}[HSK 1]
  \definition[间,个]{s.}{quarto}
\end{entry}

\begin{entry}{房屋}{fang2 wu1}{8,9}[HSK 3]
  \definition[间,所,套]{s.}{casas; habitação; edifícios}
\end{entry}

\begin{entry}{房主}{fang2zhu3}{8,5}
  \definition{s.}{proprietário | dono de um imóvel}
\end{entry}

\begin{entry}{房子}{fang2zi5}{8,3}[HSK 1]
  \definition[栋,幢,座,套,间,个]{s.}{apartamento | casa | quarto}
\end{entry}

\begin{entry}{房租}{fang2 zu1}{8,10}[HSK 3]
  \definition[笔]{s.}{aluguel}
\end{entry}

\begin{entry}{访问}{fang3wen4}{6,6}[HSK 3]
  \definition{v.}{visitar; ligar; entrevistar | visitar um \emph{site}}
\end{entry}

\begin{entry}{放}{fang4}{8}[Radical 攴][HSK 1]
  \definition{v.}{liberar | libertar | deixar ir | colocar | por | detonar (fogos de artifício)}
\end{entry}

\begin{entry}{放鞭炮}{fang4bian1pao4}{8,18,9}
  \definition{s.}{um conjunto de bombinhas ou traques}
\end{entry}

\begin{entry}{放出}{fang4chu1}{8,5}
  \definition{v.}{liberar | libertar}
\end{entry}

\begin{entry}{放大}{fang4da4}{8,3}
  \definition{v.}{ampliar}
\end{entry}

\begin{entry}{放到}{fang4 dao4}{8,8}[HSK 3]
  \definition{v.}{colocar em; meter}
\end{entry}

\begin{entry}{放电}{fang4dian4}{8,5}
  \definition{s.}{descarga elétrica}
\end{entry}

\begin{entry}{放飞}{fang4fei1}{8,3}
  \definition{s.}{deixar voar}
\end{entry}

\begin{entry}{放过}{fang4guo4}{8,6}
  \definition{v.}{deixar | deixar alguém escapar impune | passar despercebido}
\end{entry}

\begin{entry}{放假}{fang4 jia4}{8,11}[HSK 1]
  \definition{v.}{ter férias ou feriado}
\end{entry}

\begin{entry}{放弃}{fang4qi4}{8,7}
  \definition{v.}{abandonar | desistir de | renunciar}
\end{entry}

\begin{entry}{放弃权利}{fang4qi4 quan2li4}{8,7,6,7}
  \definition{s.}{renúncia}
\end{entry}

\begin{entry}{放弃者}{fang4qi4zhe3}{8,7,8}
  \definition{s.}{desistente}
\end{entry}

\begin{entry}{放任}{fang4ren4}{8,6}
  \definition{v.}{ignorar | saciar-se | deixar sozinho}
\end{entry}

\begin{entry}{放肆}{fang4si4}{8,13}
  \definition{adj.}{atrevido | pesunçoso | devasso}
\end{entry}

\begin{entry}{放松}{fang4song1}{8,8}
  \definition{adj.}{relaxado | afrouxado}
  \definition{v.}{relaxar | afrouxar}
\end{entry}

\begin{entry}{放下}{fang4 xia4}{8,3}[HSK 2]
  \definition{v.}{deitar | colocar para baixo | deixar ir | liberar | desistir | colocar em algum lugar}
\end{entry}

\begin{entry}{放心}{fang4xin1}{8,4}[HSK 2]
  \definition{adj.}{despreocupado}
  \definition{v.}{sentir-se aliviado | sentir-se tranquilo | ficar à vontade}
  \definition{v.+compl.}{confiar | ter confiança em alguém | estar à vontade | sentir-se aliviado}
\end{entry}

\begin{entry}{放学}{fang4 xue2}{8,8}[HSK 1]
  \definition{v.+compl.}{sair da escola | acabar as aulas | terminar a aula (por hoje)}
\end{entry}

\begin{entry}{放养}{fang4yang3}{8,9}
  \definition{v.}{criar (gado, peixes, culturas, etc.) | crescer | criar}
\end{entry}

\begin{entry}{放走}{fang4zou3}{8,7}
  \definition{v.}{permitir (uma pessoa ou um animal) ir | liberar | libertar}
\end{entry}

\begin{entry}{飞}{fei1}{3}[Radical 飛][HSK 1]
  \definition*{s.}{sobrenome Fei}
  \definition{adj.}{inesperado | acidental | infundado | sem fundamento}
  \definition{adv.}{rapidamente}
  \definition{s.}{roda livre de uma bicicleta}
  \definition{v.}{voar | esvoaçar | flutuar no ar | volatilizar}
\end{entry}

\begin{entry}{飞船}{fei1chuan2}{3,11}
  \definition{s.}{espaçonave | dirigível | aeronave}
\end{entry}

\begin{entry}{飞碟}{fei1die2}{3,14}
  \definition{s.}{disco-voador, OVNI, \emph{UFO} | \emph{frisbee}}
\end{entry}

\begin{entry}{飞机}{fei1ji1}{3,6}[HSK 1]
  \definition[架]{s.}{avião}
\end{entry}

\begin{entry}{飞机票}{fei1ji1piao4}{3,6,11}
  \definition[张]{s.}{bilhete de avião}
  \seealsoref{机票}{ji1piao4}
\end{entry}

\begin{entry}{飞行}{fei1 xing2}{3,6}[HSK 3]
  \definition{s.}{voo | aviação}
  \definition{v.}{voar; fazer um voo | (aviões, foguetes, etc.) voar no ar}
\end{entry}

\begin{entry}{非}{fei1}{8}[Radical ⾮][Kangxi 175]
  \definition*{s.}{África, abreviação de 非洲}
  \definition{adv.}{não ser | não é | não}
  \seealsoref{非洲}{fei1zhou1}
\end{entry}

\begin{entry}{非常}{fei1chang2}{8,11}[HSK 1]
  \definition{adv.}{extraordinário | altamente | muito}
\end{entry}

\begin{entry}{非洲}{fei1zhou1}{8,9}
  \definition*{s.}{África}
\end{entry}

\begin{entry}{非洲人}{fei1zhou1ren2}{8,9,2}
  \definition{s.}{africano | pessoa ou povo da África}
\end{entry}

\begin{entry}{狒狒}{fei4fei4}{8,8}
  \definition{s.}{babuíno}
\end{entry}

\begin{entry}{费}{fei4}{9}[Radical 貝][HSK 3]
  \definition*{s.}{Fei}
  \definition{s.}{taxa; despesa; encargo}
  \definition{v.}{custar; gastar; desperdiçar}
\end{entry}

\begin{entry}{费用}{fei4 yong4}{9,5}[HSK 3]
  \definition[笔,个]{s.}{custo; despesa; desembolso}
\end{entry}

\begin{entry}{分}{fen1}{4}[Radical ⼑][HSK 1]
  \definition{s.}{parte ou subdivisão | fração | um décimo (de certas unidades) | unidade de comprimento equivalente a 0,33cm | minuto (unidade de tempo) | minuto (unidade de medida angular) | um ponto (em esportes e jogos) | 0,01 yuan (unidade de dinheiro)}
  \definition{v.}{dividir | separar | distribuir | atribuir | distinguir (bom e mau)}
  \seeref{分}{fen4}
\end{entry}

\begin{entry}{分别}{fen1bie2}{4,7}[HSK 3]
  \definition{adv.}{diferentemente; de ​​maneiras diferentes}
  \definition{s.}{diferença}
  \definition{v.}{partir; deixar um ao outro | distinguir; diferenciar}
\end{entry}

\begin{entry}{分公司}{fen1gong1si1}{4,4,5}
  \definition{s.}{sucursal | filial de companhia}
\end{entry}

\begin{entry}{分开}{fen1 kai1}{4,4}[HSK 2]
  \definition{v.+compl.}{separar | dividir | desacoplar | desempacotar | quebrar | desmembrar | romper | desfazer | desvincular | distribuir | separar de (em) | dividir \dots de \dots | separar de}
\end{entry}

\begin{entry}{分量}{fen1liang4}{4,12}
  \definition{s.}{componente vetorial}
  \seeref{分量}{fen4liang4}
  \seeref{分量}{fen4liang5}
\end{entry}

\begin{entry}{分配}{fen1pei4}{4,10}[HSK 3]
  \definition{v.}{atribuir; dispor | atribuir; compartilhar; distribuir}
\end{entry}

\begin{entry}{分手}{fen1shou3}{4,4}
  \definition{v.+compl.}{separar | separar-se do companheiro | dizer adeus}
\end{entry}

\begin{entry}{分数}{fen1 shu4}{4,13}[HSK 2]
  \definition{s.}{fração | número fracionário | marca | nota | ponto}
\end{entry}

\begin{entry}{分钟}{fen1zhong1}{4,9}[HSK 2]
  \definition{s.}{minuto (usado em intervalos de tempo)}
\end{entry}

\begin{entry}{分子}{fen1zi3}{4,3}
  \definition{s.}{molécula | (matemática) numerador de uma fração}
  \seeref{分子}{fen4zi3}
\end{entry}

\begin{entry}{分组}{fen1 zu3}{4,8}[HSK 3]
  \definition{v.}{agrupar; dividir em grupos}
\end{entry}

\begin{entry}{焚香}{fen2xiang1}{12,9}
  \definition{v.}{queimar incenso}
\end{entry}

\begin{entry}{粉}{fen3}{10}[Radical ⽶]
  \definition{s.}{pó | pó cosmético facial | alimento preparado a partir de amido | macarrão feito de qualquer tipo de farinha}
  \definition{v.}{tornar algo em pó | ser um fã de}
\end{entry}

\begin{entry}{粉色}{fen3 se4}{10,6}
  \definition{s.}{cor-de-rosa}
\end{entry}

\begin{entry}{粉丝}{fen3si1}{10,5}
  \definition{s.}{(empréstimo linguístico) fã | entusiasta de alguém ou alguma coisa}
  \definition[把]{s.}{aletria de amido de feijão | aletria chinesa | macarrão de celofane ou macarrão de vidro (transparente)}
\end{entry}

\begin{entry}{分}{fen4}{4}[Radical 刀][HSK 2]
  \definition{s.}{parte | ingrediente | componente}
  \seeref{分}{fen1}
\end{entry}

\begin{entry}{分量}{fen4liang4}{4,12}
  \definition{s.}{tamanho da porção (comida)}
  \seeref{分量}{fen1liang4}
  \seeref{分量}{fen4liang5}
\end{entry}

\begin{entry}{分量}{fen4liang5}{4,12}
  \definition{s.}{quantidade | peso | medida}
  \seeref{分量}{fen1liang4}
  \seeref{分量}{fen4liang4}
\end{entry}

\begin{entry}{分子}{fen4zi3}{4,3}
  \definition{s.}{membros de uma classe ou grupo | elementos políticos (como intelectuais ou extremistas)}
  \seeref{分子}{fen1zi3}
\end{entry}

\begin{entry}{份}{fen4}{6}[Radical 人][HSK 2]
  \definition{clas.}{para presentes, jornais, revistas, papéis, relatórios, contratos, etc. ou pratos (refeição)}
\end{entry}

\begin{entry}{奋战}{fen4zhan4}{8,9}
  \definition{v.}{lutar bravamente | trabalhar duro}
\end{entry}

\begin{entry}{愤怒}{fen4nu4}{12,9}
  \definition{adj.}{zangado | indignado}
  \definition{s.}{ira}
\end{entry}

\begin{entry}{愤世嫉俗}{fen4shi4ji2su2}{12,5,13,9}
  \definition{v.}{ser cínico | ser amargurado}
\end{entry}

\begin{entry}{丰富}{feng1fu4}{4,12}[HSK 3]
  \definition{adj.}{rico; abundante; pleno}
  \definition{v.}{enriquecer}
\end{entry}

\begin{entry}{丰收}{feng1shou1}{4,6}
  \definition{s.}{colheita abundante}
\end{entry}

\begin{entry}{风}{feng1}{4}[Radical 風][Kangxi 182][HSK 1]
  \definition[阵,丝]{s.}{vento}
\end{entry}

\begin{entry}{风景}{feng1jing3}{4,12}
  \definition{s.}{cenário | paisagem}
\end{entry}

\begin{entry}{风扇}{feng1shan4}{4,10}
  \definition{s.}{ventilador elétrico}
\end{entry}

\begin{entry}{风险}{feng1xian3}{4,9}[HSK 3]
  \definition[个,种,项,类]{s.}{risco; perigo}
\end{entry}

\begin{entry}{风筝}{feng1zheng5}{4,12}
  \definition{s.}{pipa | papagaio | pandorga}
\end{entry}

\begin{entry}{枫叶}{feng1ye4}{8,5}
  \definition{s.}{folha de bordo (maple, tipo de árvore)}
\end{entry}

\begin{entry}{封}{feng1}{9}[Radical 寸][HSK 2]
  \definition*{s.}{sobrenome Feng}
  \definition{clas.}{para objetos selados, especialmente cartas}
  \definition{v.}{conceder um título | conferir | conceder | selar}
\end{entry}

\begin{entry}{封闭}{feng1bi4}{9,6}
  \definition{v.}{fechar | selar | confinado}
\end{entry}

\begin{entry}{封底}{feng1di3}{9,8}
  \definition{s.}{contracapa de um livro}
\end{entry}

\begin{entry}{封冻}{feng1dong4}{9,7}
  \definition{v.}{congelar (água ou terra)}
\end{entry}

\begin{entry}{封盖}{feng1gai4}{9,11}
  \definition{s.}{boné | capa | selo}
  \definition{v.}{cobrir}
\end{entry}

\begin{entry}{封建}{feng1jian4}{9,8}
  \definition{adj.}{feudal}
  \definition{s.}{feudalismo}
\end{entry}

\begin{entry}{封口}{feng1kou3}{9,3}
  \definition{v.}{selar | fechar | curar (uma ferida) | manter os lábios selados}
\end{entry}

\begin{entry}{封面}{feng1mian4}{9,9}
  \definition{s.}{capa (de uma publicação) | sobrecapa}
\end{entry}

\begin{entry}{封印}{feng1yin4}{9,5}
  \definition{s.}{selo (em envelopes)}
\end{entry}

\begin{entry}{封斋}{feng1zhai1}{9,10}
  \definition*{s.}{Ramadã (Islã)}
\end{entry}

\begin{entry}{疯狂}{feng1kuang2}{9,7}
  \definition{adj.}{louco | frenético | selvagem}
\end{entry}

\begin{entry}{缝纫}{feng2ren4}{13,6}
  \definition{v.}{costurar}
\end{entry}

\begin{entry}{缝纫机}{feng2ren4ji1}{13,6,6}
  \definition[架]{s.}{máquina de costura}
\end{entry}

\begin{entry}{凤凰}{feng4huang2}{4,11}
  \definition{s.}{fênix}
\end{entry}

\begin{entry}{佛}{fo2}{7}[Radical 人]
  \definition*{s.}{Buda, abreviação de 佛陀 | Budismo}
  \seeref{佛}{fu2}
  \seealsoref{佛陀}{fo2tuo2}
\end{entry}

\begin{entry}{佛陀}{fo2tuo2}{7,7}
  \definition{s.}{Buda (uma pessoa que atingiu a Budeidade, ou especificamente Siddhartha Gautama)}
\end{entry}

\begin{entry}{否定}{fou3ding4}{7,8}[HSK 3]
  \definition{adj.}{negativo}
  \definition{s.}{negativo (resposta); negação}
  \definition{v.}{rejeitar; negar}
\end{entry}

\begin{entry}{否认}{fou3ren4}{7,4}[HSK 3]
  \definition{v.}{negar; repudiar}
\end{entry}

\begin{entry}{否则}{fou3ze2}{7,6}
  \definition{conj.}{caso contrário | ou}
\end{entry}

\begin{entry}{夫妻}{fu1qi1}{4,8}
  \definition{s.}{casal | marido e eposa}
\end{entry}

\begin{entry}{佛}{fu2}{7}[Radical 人]
  \definition{adv.}{aparentemente}
  \definition{s.}{ornamento da cabeça (feminino)}
  \seeref{佛}{fo2}
\end{entry}

\begin{entry}{扶梯}{fu2ti1}{7,11}
  \definition{s.}{escada rolante}
\end{entry}

\begin{entry}{服}{fu2}{8}[Radical ⽉]
  \definition{s.}{roupas | vestido | vestuário | roupa de luto}
  \definition{v.}{servir (nas forças armadas, uma sentença de prisão, etc.) | obedecer | ser convencido (por um argumento) | convencer | admirar | aclimatar | tomar (medicamento) | usar roupas de luto}
  \seeref{服}{fu4}
\end{entry}

\begin{entry}{服务}{fu2 wu4}{8,5}[HSK 2]
  \definition{v.}{prestar serviço a | estar a serviço de | servir | trabalhar | servir}
\end{entry}

\begin{entry}{服务员}{fu2wu4yuan2}{8,5,7}
  \definition{s.}{atendente | garçom | garçonete | pessoal de atendimento ao cliente}
\end{entry}

\begin{entry}{服装}{fu2zhuang1}{8,12}[HSK 3]
  \definition[套,件,身]{s.}{roupas; trajes; fantasias}
\end{entry}

\begin{entry}{浮力}{fu2li4}{10,2}
  \definition{s.}{flutuabilidade}
\end{entry}

\begin{entry}{浮图}{fu2tu2}{10,8}
  \definition*{s.}{Termo alternativo para 佛陀}
  \variantof{浮屠}
  \seealsoref{佛陀}{fo2tuo2}
\end{entry}

\begin{entry}{浮屠}{fu2tu2}{10,11}
  \definition*{s.}{Buda | Templo (Stupa) Budista (transliteração de Pali Thuo)}
\end{entry}

\begin{entry}{符合}{fu2he2}{11,6}
  \definition{conj.}{de acordo com | concordando com | contando com | alinhado com}
  \definition{v.}{concordar com | estar em conformidade com | corresponder com | gerenciar | lidar}
\end{entry}

\begin{entry}{福}{fu2}{13}[Radical 示][HSK 3]
  \definition*{s.}{sobrenome Fu}
  \definition{s.}{benção; felicidade; boa sorte; boa fortuna}
  \definition{v.}{curvar-se; reverenciar}
\end{entry}

\begin{entry}{福克斯}{fu2ke4si1}{13,7,12}
  \definition*{s.}{Fox (empresa de mídia) | Focus (automóvel fabricado pela Ford)}
\end{entry}

\begin{entry}{福泽}{fu2ze2}{13,8}
  \definition{s.}{boa sorte}
\end{entry}

\begin{entry}{父母}{fu4 mu3}{4,5}[HSK 3]
  \definition{s.}{pai e mãe; pais}
\end{entry}

\begin{entry}{父母亲}{fu4mu3qin1}{4,5,9}
  \definition{s.}{pais}
\end{entry}

\begin{entry}{父亲}{fu4qin1}{4,9}[HSK 3]
  \definition[个,位]{s.}{pai}
\end{entry}

\begin{entry}{付}{fu4}{5}[Radical 人][HSK 3]
  \definition*{s.}{sobrenome Fu}
  \definition{clas.}{para pares ou conjuntos de coisas | para expressões faciais}
  \definition{v.}{comprometer-se a; entregar (entregar) a; entregar | pagar}
\end{entry}

\begin{entry}{付款}{fu4kuan3}{5,12}
  \definition{s.}{pagamento}
  \definition{v.+compl.}{pagar uma quantia em dinheiro}
\end{entry}

\begin{entry}{负责}{fu4ze2}{6,8}[HSK 3]
  \definition{adj.}{consciencioso}
  \definition{v.}{ser responsável por; estar encarregado de}
\end{entry}

\begin{entry}{附近}{fu4jin4}{7,7}
  \definition{adv.}{aqui perto | perto daqui}
\end{entry}

\begin{entry}{服}{fu4}{8}[Radical ⽉]
  \definition{clas.}{(para remédio) dose}
  \seeref{服}{fu2}
\end{entry}

\begin{entry}{复活节}{fu4huo2jie2}{9,9,5}
  \definition*{s.}{Páscoa}
\end{entry}

\begin{entry}{复刻}{fu4ke4}{9,8}
  \definition{v.}{reimprimir (um trabalho que esteve fora do catálogo) | reeditar (um disco de vinil, um CD, etc.) | replicar | recriar | (empréstimo linguístico) (computação) \emph{fork}}
\end{entry}

\begin{entry}{复习}{fu4xi2}{9,3}[HSK 2]
  \definition{s.}{revisão}
  \definition{v.}{rever | revisar}
\end{entry}

\begin{entry}{复印}{fu4yin4}{9,5}[HSK 3]
  \definition{v.}{fotografar; fotocopiar; duplicar}
\end{entry}

\begin{entry}{复杂}{fu4za1}{9,6}[HSK 3]
  \definition{adj.}{complexo; complicado}
\end{entry}

\begin{entry}{副}{fu4}{11}[Radical 刀]
  \definition{clas.}{para pares, conjuntos de coisas e expressões faciais | para óculos, luvas, etc.}
\end{entry}

\begin{entry}{富}{fu4}{12}[Radical 宀][HSK 3]
  \definition*{s.}{sobrenome Fu}
  \definition{adj.}{rico; póspero | rico; abundante}
  \definition{s.}{fortuna; riqueza}
\end{entry}

\begin{entry}{覆盆子}{fu4pen2zi5}{18,9,3}
  \definition{s.}{framboesa}
\end{entry}

%%%%% EOF %%%%%


%%%
%%% G
%%%

\section*{G}\addcontentsline{toc}{section}{G}

\begin{entry}{夹}{ga1}{6}{⼤}
  \definition{s.}{axila; sovaco; atualmente, costuma-se escrever ``胳肢窝'' (axila)}
  \seeref{夹}{jia1}
  \seeref{夹}{jia2}
  \seealsoref{胳肢窝}{ga1 zhi1 wo1}
\end{entry}

\begin{entry}{胳肢窝}{ga1 zhi1 wo1}{10,8,12}{⾁、⾁、⽳}
  \definition{s.}{axila; sovaco; também escrito ``夹肢窝''}
  \seealsoref{夹肢窝}{jia1 zhi1 wo1}
\end{entry}

\begin{entry}{该}{gai1}{8}{⾔}[HSK 2]
  \definition{v.}{deveria | é a vez de alguém fazer algo | merecer | dever}
\end{entry}

\begin{entry}{改}{gai3}{7}{⽁}[HSK 2]
  \definition*{s.}{sobrenome Gai}
  \definition{v.}{mudar | transformar | revisar | alterar | modificar | retificar | corrigir | mudar para (fazer outra coisa)}
\end{entry}

\begin{entry}{改变}{gai3bian4}{7,8}{⽁、⼜}[HSK 2]
  \definition{v.}{mudar | alterar | transformar | virar | converter | moldar | modificar}
\end{entry}

\begin{entry}{改革}{gai3ge2}{7,9}{⽁、⾰}[HSK 5]
  \definition[项,次,种]{s.}{reforma; reformação; iniciativas para aprimorar a inovação}
  \definition{v.}{reformar; transformar as antigas partes irracionais das coisas em novas que possam ser adaptadas à situação objetiva}
\end{entry}

\begin{entry}{改进}{gai3jin4}{7,7}{⽁、⾡}[HSK 3]
  \definition[个]{s.}{melhoria}
  \definition{v.}{aprimorar; aperfeiçoar; melhorar; tornar melhor
modificar}
\end{entry}

\begin{entry}{改良}{gai3liang2}{7,7}{⽁、⾉}
  \definition{v.}{melhorar (algo) | reformar (um sistema)}
\end{entry}

\begin{entry}{改善}{gai3shan4}{7,12}{⽁、⼝}[HSK 4]
  \definition{v.}{melhorar; amenizar; mudar a situação original para torná-la melhor}
\end{entry}

\begin{entry}{改善关系}{gai3shan4guan1xi5}{7,12,6,7}{⽁、⼝、⼋、⽷}
  \definition{v.}{melhorar a relação}
\end{entry}

\begin{entry}{改善通讯}{gai3shan4tong1xun4}{7,12,10,5}{⽁、⼝、⾡、⾔}
  \definition{v.}{melhorar a comunicação}
\end{entry}

\begin{entry}{改造}{gai3 zao4}{7,10}{⽁、⾡}[HSK 3]
  \definition{v.}{transformar; renovar | remodelar}
\end{entry}

\begin{entry}{改正}{gai3 zheng4}{7,5}{⽁、⽌}[HSK 4]
  \definition{v.}{corrigir; emendar; mudar o errado para o correto}
\end{entry}

\begin{entry}{芥}{gai4}{7}{⾋}
  \definition{s.}{usado em 芥蓝 \dpy{gai4lan2}}
  \seeref{芥蓝}{gai4lan2}
  \seeref{芥}{jie4}
\end{entry}

\begin{entry}{芥兰}{gai4lan2}{7,5}{⾋、⼋}
  \variantof{芥蓝}
\end{entry}

\begin{entry}{芥蓝}{gai4lan2}{7,13}{⾋、⾋}
  \definition{s.}{brócolis chinês | couve chinesa | mostarda}
  \seeref{格兰菜}{ge2lan2cai4}
\end{entry}

\begin{entry}{盖}{gai4}{11}{⽫}[HSK 4]
  \definition*{s.}{sobrenome Gai}
  \definition{adj.}{excelente; soberbo; fantástico}
  \definition{adv.}{cerca de; ao redor; aproximadamente; expressa um julgamento especulativo sobre algo, ou uma explicação da causa, o que é equivalente a ``大概'' ou ``原来''}
  \definition{conj.}{para; porque; dando continuidade à frase anterior, afirmando a razão ou causa, com tom incerto}
  \definition{s.}{tampa; capa; cobertura; algo que cobre ou sela a parte superior de um objeto | carapaça; concha (de tartaruga, caranguejo, etc.); ossos em formato de crânio em certas partes do corpo humano; as conchas nas costas de certos animais | dossel; capota; toldo | nivelador (uma ferramenta agrícola usada para nivelar terras)}
  \definition{v.}{cobrir; proteger; colocar uma capa em; colocar uma tampa em um objeto | selar; afixar um selo em | superar; sobressair; sobrepujar; ultrapassar | construir; colocar para cima | esconder; ocultar; encobrir | nivelar o terreno com um nivelador (ferramenta agrícola)}
  \seeref{盖}{ge3}
  \seealsoref{大概}{da4gai4}
  \seealsoref{原来}{yuan2lai2}
\end{entry}

\begin{entry}{概括}{gai4kuo4}{13,9}{⽊、⼿}[HSK 4]
  \definition{adj.}{genérico; simples e claro, captando o conteúdo principal}
  \definition{s.}{generalização}
  \definition{v.}{generalizar; resumir}
\end{entry}

\begin{entry}{概念}{gai4nian4}{13,8}{⽊、⼼}[HSK 3]
  \definition[个]{s.}{ideia; noção; conceito; concepção}
\end{entry}

\begin{entry}{干}{gan1}{3}{⼲}[HSK 1][Kangxi 51]
  \definition*{s.}{sobrenome Gan}
  \definition{v.}{preocupar | ignorar | interferir}
  \seeref{干}{gan4}
\end{entry}

\begin{entry}{干杯}{gan1bei1}{3,8}{⼲、⽊}[HSK 2]
  \definition{interj.}{Saúde!}
  \definition{v.+compl.}{fazer um brinde | brindar até a última gota}
\end{entry}

\begin{entry}{干脆}{gan1cui4}{3,10}{⼲、⾁}[HSK 5]
  \definition{adj.}{claro; direto; (falando, fazendo coisas) sem hesitação; atitude clara}
  \definition{adv.}{justamente; diretamente; sem maiores considerações}
\end{entry}

\begin{entry}{干净}{gan1jing4}{3,8}{⼲、⼎}[HSK 1]
  \definition{adj.}{limpo | arrumado}
\end{entry}

\begin{entry}{干你屁事}{gan1 ni3 pi4shi4}{3,7,7,8}{⼲、⼈、⼫、⼅}
  \definition{interj.}{Foda-se!}
\end{entry}

\begin{entry}{干扰}{gan1rao3}{3,7}{⼲、⼿}[HSK 5]
  \definition{v.}{perturbar; incomodar | interferir; interromper o funcionamento adequado de equipamentos eletrônicos com sinais eletrônicos dispersos}
\end{entry}

\begin{entry}{干与}{gan1yu4}{3,3}{⼲、⼀}
  \variantof{干预}
\end{entry}

\begin{entry}{干预}{gan1yu4}{3,10}{⼲、⾴}[HSK 5]
  \definition{s.}{intromissão; intervenção}
  \definition{v.}{intrometer-se; intervir; interpor-se;}
\end{entry}

\begin{entry}{甘薯}{gan1shu3}{5,16}{⽢、⾋}
  \definition{s.}{batata doce}
\end{entry}

\begin{entry}{甘心}{gan1xin1}{5,4}{⽢、⼼}
  \definition{v.}{estar disposto a | resignar-se a}
\end{entry}

\begin{entry}{赶}{gan3}{10}{⾛}[HSK 3]
  \definition*{s.}{sobrenome Gan}
  \definition{prep.}{por; até}
  \definition{v.}{ultrapassar; alcançar | perseguir; correr para; correr atrás; tentar pegar | dirigir | expulsar; afastar | encontrar; deparar-se com; esbarrar em; acontecer com; encontrar-se em (uma situação); aproveitar-se de (uma oportunidade) | ir para}
\end{entry}

\begin{entry}{赶到}{gan3 dao4}{10,8}{⾛、⼑}[HSK 3]
  \definition{v.}{apressar (para algum lugar); avançar de súbito}
\end{entry}

\begin{entry}{赶赴}{gan3fu4}{10,9}{⾛、⾛}
  \definition{v.}{apressar}
\end{entry}

\begin{entry}{赶集}{gan3ji2}{10,12}{⾛、⾫}
  \definition{v.}{ir a uma feira | ir ao mercado}
\end{entry}

\begin{entry}{赶脚}{gan3jiao3}{10,11}{⾛、⾁}
  \definition{v.}{transportar mercadorias para ganhar a vida (especialmente de burro) | trabalhar como carroceiro ou porteiro}
\end{entry}

\begin{entry}{赶紧}{gan3jin3}{10,10}{⾛、⽷}[HSK 3]
  \definition{adv.}{apressadamente; sem demora}
\end{entry}

\begin{entry}{赶快}{gan3kuai4}{10,7}{⾛、⼼}[HSK 3]
  \definition{adv.}{rapidamente; imediatamente}
\end{entry}

\begin{entry}{赶路}{gan3lu4}{10,13}{⾛、⾜}
  \definition{v.}{apressar a jornada | apressar-se}
\end{entry}

\begin{entry}{赶忙}{gan3mang2}{10,6}{⾛、⼼}
  \definition{v.}{acelerar | apressar | se apressar}
\end{entry}

\begin{entry}{赶跑}{gan3pao3}{10,12}{⾛、⾜}
  \definition{v.}{afastar | forçar a saída | repelir}
\end{entry}

\begin{entry}{赶上}{gan3shang4}{10,3}{⾛、⼀}
  \definition{adv.}{a tempo para}
  \definition{v.}{alcançar | ultrapassar}
\end{entry}

\begin{entry}{赶早}{gan3zao3}{10,6}{⾛、⽇}
  \definition{adv.}{o mais breve possível | na primeira oportunidade | antes que seja tarde | quanto antes melhor}
\end{entry}

\begin{entry}{赶走}{gan3zou3}{10,7}{⾛、⾛}
  \definition{v.}{expulsar | voltar atrás}
\end{entry}

\begin{entry}{敢}{gan3}{11}{⽁}[HSK 3]
  \definition{adj.}{ousado; corajoso; bravo}
  \definition{adv.}{talvez; provavelmente}
  \definition{v.}{ousar; aventurar-se | ter a confiança de; ter certeza | tornar ousado; aventurar-se}
\end{entry}

\begin{entry}{敢情}{gan3qing5}{11,11}{⽁、⼼}
  \definition{adv.}{claro | acontece que\dots}
\end{entry}

\begin{entry}{感到}{gan3 dao4}{13,8}{⼼、⼑}[HSK 2]
  \definition{v.}{sentir | perceber}
\end{entry}

\begin{entry}{感动}{gan3dong4}{13,6}{⼼、⼒}[HSK 2]
  \definition{v.}{mover (alguém) | tocar (alguém emocionalmente)}
\end{entry}

\begin{entry}{感觉}{gan3jue2}{13,9}{⼼、⾒}[HSK 2]
  \definition{s.}{sentimento | impressão | sensação}
  \definition{v.}{sentir | perceber}
\end{entry}

\begin{entry}{感冒}{gan3mao4}{13,9}{⼼、⽇}[HSK 3]
  \definition{adj.}{interessado}
  \definition[场,次]{s.}{resfriado; resfriado comum; gripe}
  \definition{v.}{pegar (ter) um resfriado}
\end{entry}

\begin{entry}{感情}{gan3qing2}{13,11}{⼼、⼼}[HSK 3]
  \definition[份,个,种]{s.}{emoção; sentimento | amor; afeição; apego}
\end{entry}

\begin{entry}{感染}{gan3ran3}{13,9}{⼼、⽊}
  \definition{s.}{infecção}
  \definition{v.}{infectar | (figurativo) influenciar}
\end{entry}

\begin{entry}{感受}{gan3shou4}{13,8}{⼼、⼜}[HSK 3]
  \definition{s.}{percepção ; sentimento; experiência}
  \definition{v.}{sentir; sentir (através dos sentidos); experimentar}
\end{entry}

\begin{entry}{感想}{gan3xiang3}{13,13}{⼼、⼼}[HSK 5]
  \definition[个,条]{s.}{pensamentos; impressões; reflexões; resposta do pensamento decorrente da exposição ao mundo exterior}
\end{entry}

\begin{entry}{感谢}{gan3xie4}{13,12}{⼼、⾔}[HSK 2]
  \definition{s.}{gratidão | agradecimento}
\end{entry}

\begin{entry}{感兴趣}{gan3xing4qu4}{13,6,15}{⼼、⼋、⾛}[HSK 4]
  \definition{v.}{estar interessado}
  \seeref{对……感兴趣}{dui4 gan3xing4qu4}
\end{entry}

\begin{entry}{橄榄球}{gan3lan3qiu2}{15,13,11}{⽊、⽊、⽟}
  \definition{s.}{futebol jogado com bola oval (rúgbi, futebol americano, regras australianas, etc.)}
\end{entry}

\begin{entry}{干}{gan4}{3}{⼲}[HSK 1]
  \definition{v.}{fazer | gerenciar | trabalhar | (gíria) matar | (vulgar) foder}
  \seeref{干}{gan1}
\end{entry}

\begin{entry}{干活}{gan4huo2}{3,9}{⼲、⽔}
  \definition{v.+compl.}{trabalhar | trabalhar em um emprego}
\end{entry}

\begin{entry}{干活儿}{gan4huo2r5}{3,9,2}{⼲、⽔、⼉}[HSK 2]
  \definition{v.}{trabalhar em um emprego}
\end{entry}

\begin{entry}{干吗}{gan4 ma2}{3,6}{⼲、⼝}[HSK 3]
  \definition{pron.}{por que?}
  \definition{v.}{o que fazer?}
\end{entry}

\begin{entry}{干什么}{gan4 shen2 me5}{3,4,3}{⼲、⼈、⼃}[HSK 1]
  \definition{v.}{o que fazer? | o que está fazendo?}
\end{entry}

\begin{entry}{刚}{gang1}{6}{⼑}[HSK 2]
  \definition{adj.}{duro (sentido de difícil) | forte}
  \definition{adv.}{apenas | exatamente | há pouco tempo | por muito pouco | assim que}
\end{entry}

\begin{entry}{刚才}{gang1cai2}{6,3}{⼑、⼿}[HSK 2]
  \definition{adv.}{ainda agora | há pouco tempo}
\end{entry}

\begin{entry}{刚刚}{gang1 gang1}{6,6}{⼑、⼑}[HSK 2]
  \definition{adv.}{apenas | apenas agora | um momento atrás | por muito pouco}
\end{entry}

\begin{entry}{扛}{gang1}{6}{⼿}
  \definition{v.}{levantar com as duas mãos | carregar alguma coisa juntos (duas ou mais pessoas)}
  \seeref{扛}{kang2}
\end{entry}

\begin{entry}{杠}{gang1}{7}{⽊}
  \definition{s.}{mastro de bandeira | poste | passarela}
  \seeref{杠}{gang4}
\end{entry}

\begin{entry}{钢}{gang1}{9}{⾦}
  \definition{s.}{aço}
\end{entry}

\begin{entry}{钢笔}{gang1 bi3}{9,10}{⾦、⽵}[HSK 5]
  \definition[支]{s.}{caneta-tinteiro; canetas com ponta metálica}
\end{entry}

\begin{entry}{钢琴}{gang1qin2}{9,12}{⾦、⽟}[HSK 5]
  \definition[架]{s.}{piano}
\end{entry}

\begin{entry}{钢丝}{gang1si1}{9,5}{⾦、⼀}
  \definition{s.}{cabo de aço | corda bamba}
\end{entry}

\begin{entry}{杠}{gang4}{7}{⽊}
  \definition{s.}{vara grossa | barra | linha grossa | padrão, critério | hífen, traço}
  \definition{v.}{marcar com uma linha grossa | afiar (faca, navalha, etc.)}
  \seeref{杠}{gang1}
\end{entry}

\begin{entry}{高}{gao1}{10}{⾼}[HSK 1][Kangxi 189]
  \definition*{s.}{sobrenome Gao}
  \definition{adj.}{alto | acima da média}
  \definition{pron.}{Seu (honorífico)}
\end{entry}

\begin{entry}{高潮}{gao1chao2}{10,15}{⾼、⽔}[HSK 4]
  \definition[个,场]{s.}{maré alta; o nível mais alto da maré em um ciclo de maré | pico; aumento; maré alta; uma metáfora para o estágio mais próspero de desenvolvimento das coisas (diferente de ``低潮'') | (ficção, drama e filmes) clímax}
  \seealsoref{低潮}{di1chao2}
\end{entry}

\begin{entry}{高大}{gao1 da4}{10,3}{⾼、⼤}[HSK 5]
  \definition{adj.}{alto e grande; alto | elevado; sublime; nobre}
\end{entry}

\begin{entry}{高度}{gao1 du4}{10,9}{⾼、⼴}[HSK 5]
  \definition{adj.}{alto; elevado; avançado; alto grau | alta concentração; intenso}
  \definition[个]{s.}{altura; altitude; elevação; distância de baixo para cima; o grau e o nível em que as coisas se desenvolveram}
\end{entry}

\begin{entry}{高尔夫}{gao1'er3fu1}{10,5,4}{⾼、⼩、⼤}
  \definition{s.}{(empréstimo linguístico) \emph{golf}}
\end{entry}

\begin{entry}{高跟鞋}{gao1 gen1 xie2}{10,13,15}{⾼、⾜、⾰}[HSK 5]
  \definition{s.}{salto alto; sapatos de salto alto; sapato feminino com salto mais alto e mais distante do chão}
\end{entry}

\begin{entry}{高级}{gao1ji2}{10,6}{⾼、⽷}[HSK 2]
  \definition{adj.}{sênior | alto escalão | alto nível | alto grau | grau superior | alta qualidade | avançado}
\end{entry}

\begin{entry}{高价}{gao1 jia4}{10,6}{⾼、⼈}[HSK 4]
  \definition{s.}{preço alto; bilhete caro; custo elevado; dispendioso}
\end{entry}

\begin{entry}{高楼}{gao1lou2}{10,13}{⾼、⽊}
  \definition[座]{s.}{edifício alto | edifício de muitos andares | arranha-céu}
\end{entry}

\begin{entry}{高尚}{gao1shang4}{10,8}{⾼、⼩}[HSK 4]
  \definition{adj.}{nobre; elevado; descreve um alto padrão moral e uma boa qualidade de pensamento | significativo e não de mau gosto}
\end{entry}

\begin{entry}{高手}{gao1shou3}{10,4}{⾼、⼿}
  \definition{s.}{\emph{expert} | mestre}
\end{entry}

\begin{entry}{高速}{gao1 su4}{10,10}{⾼、⾡}[HSK 3]
  \definition{adj.}{alta velocidade}
  \definition{s.}{auto-estrada; via expressa}
\end{entry}

\begin{entry}{高速公路}{gao1su4gong1lu4}{10,10,4,13}{⾼、⾡、⼋、⾜}[HSK 3]
  \definition[条]{s.}{via expressa; rodovia; auto-estrada}
\end{entry}

\begin{entry}{高铁}{gao1 tie3}{10,10}{⾼、⾦}[HSK 4]
  \definition{s.}{trem de alta velocidade; trem bala}
\end{entry}

\begin{entry}{高温}{gao1 wen1}{10,12}{⾼、⽔}[HSK 5]
  \definition{s.}{alta temperatura; temperatura elevada; hipertermia; megatemperatura; inferno;}
\end{entry}

\begin{entry}{高效}{gao1xiao4}{10,10}{⾼、⽁}
  \definition{adj.}{eficiente | altamente eficaz}
\end{entry}

\begin{entry}{高兴}{gao1xing4}{10,6}{⾼、⼋}[HSK 1]
  \definition{adj.}{feliz | contente | disposto (a fazer alguma coisa) | de bom humor}
\end{entry}

\begin{entry}{高于}{gao1 yu2}{10,3}{⾼、⼆}[HSK 5]
  \definition{v.}{ser mais alto do que; sobrepujar}
\end{entry}

\begin{entry}{高原}{gao1 yuan2}{10,10}{⾼、⼚}[HSK 5]
  \definition[片]{s.}{planalto continental; planalto | platô}
\end{entry}

\begin{entry}{高中}{gao1 zhong1}{10,4}{⾼、⼁}[HSK 2]
  \definition{s.}{escola secundária | escola de segundo grau}
\end{entry}

\begin{entry}{糕点}{gao1dian3}{16,9}{⽶、⽕}
  \definition{s.}{bolos | pastéis}
\end{entry}

\begin{entry}{糕点店}{gao1dian3 dian4}{16,9,8}{⽶、⽕、⼴}
  \definition{s.}{confeitaria}
\end{entry}

\begin{entry}{糕点师}{gao1dian3 shi1}{16,9,6}{⽶、⽕、⼱}
  \definition{s.}{confeiteiro}
\end{entry}

\begin{entry}{搞}{gao3}{13}{⼿}[HSK 5]
  \definition{v.}{fazer; realizar; estar envolvido em; engajar-se em um estudo, fazer algo em relação a, etc. | fazer; produzir; gerar; trabalhar | iniciar; estabelecer; organizar; configurar | consertar (mudar) alguém; fazer alguém sofrer | obter; assegurar; agarrar |  (seguido de um complemento) fazer com que se torne; produzir um determinado efeito ou resultado}
\end{entry}

\begin{entry}{搞错}{gao3cuo4}{13,13}{⼿、⾦}
  \definition{v.}{cometer um erro}
\end{entry}

\begin{entry}{搞定}{gao3ding4}{13,8}{⼿、⼧}
  \definition{v.}{consertar | resolver}
\end{entry}

\begin{entry}{搞鬼}{gao3gui3}{13,9}{⼿、⿁}
  \definition{v.}{fazer travessuras | fazer truques}
\end{entry}

\begin{entry}{搞好}{gao3 hao3}{13,6}{⼿、⼥}[HSK 5]
  \definition{v.}{fazer um bom trabalho; fazer bem; suar; tornar submisso, tornar útil, por meio de solicitações e presentes amigáveis; amolecer}
\end{entry}

\begin{entry}{搞混}{gao3hun4}{13,11}{⼿、⽔}
  \definition{v.}{confundir}
\end{entry}

\begin{entry}{搞乱}{gao3luan4}{13,7}{⼿、⼄}
  \definition{v.}{estragar | confundir | bagunçar}
\end{entry}

\begin{entry}{搞钱}{gao3qian2}{13,10}{⼿、⾦}
  \definition{v.}{fazer dinheiro | acumular dinheiro}
\end{entry}

\begin{entry}{搞通}{gao3tong1}{13,10}{⼿、⾡}
  \definition{v.}{entender algo}
\end{entry}

\begin{entry}{搞笑}{gao3xiao4}{13,10}{⼿、⽵}
  \definition{adj.}{engraçado | hilário}
  \definition{v.}{fazer as pessoas rirem}
\end{entry}

\begin{entry}{稿纸}{gao3zhi3}{15,7}{⽲、⽷}
  \definition{s.}{rascunho | manuscrito}
\end{entry}

\begin{entry}{告别}{gao4bie2}{7,7}{⼝、⼑}[HSK 3]
  \definition{v.+compl.}{dizer adeus a | deixar; partir de | prestar as últimas homenagens ao falecido}
\end{entry}

\begin{entry}{告急}{gao4ji2}{7,9}{⼝、⼼}
  \definition{v.}{estar em estado de emergência | relatar uma emergência | solicitar assistência de emergência}
\end{entry}

\begin{entry}{告诉}{gao4su4}{7,7}{⼝、⾔}
  \definition{v.}{apresentar queixa | registar uma reclamação}
  \seeref{告诉}{gao4su5}
\end{entry}

\begin{entry}{告诉}{gao4su5}{7,7}{⼝、⾔}[HSK 1]
  \definition{v.}{contar | dar a conhecer | informar}
  \seeref{告诉}{gao4su4}
\end{entry}

\begin{entry}{哥}{ge1}{10}{⼝}[HSK 1]
  \definition{s.}{irmão mais velho}
  \seeref{哥哥}{ge1 ge5}
\end{entry}

\begin{entry}{哥哥}{ge1 ge5}{10,10}{⼝、⼝}[HSK 1]
  \definition[个,位]{s.}{irmão mais velho}
\end{entry}

\begin{entry}{哥们}{ge1men5}{10,5}{⼝、⼈}
  \definition{expr.}{\emph{Brothers!}}
  \definition{s.}{(coloquial) cara | irmão (forma diminuta de tratamento entre homens)}
\end{entry}

\begin{entry}{哥斯拉}{ge1si1la1}{10,12,8}{⼝、⽄、⼿}
  \definition*{s.}{Godzilla}
  \seealsoref{酷斯拉}{ku4si1la1}
\end{entry}

\begin{entry}{鸽子}{ge1zi5}{11,3}{⿃、⼦}
  \definition{s.}{pombo}
\end{entry}

\begin{entry}{搁浅}{ge1qian3}{12,8}{⼿、⽔}
  \definition{v.}{ficar encalhado (navio) | encalhar | (figurativo) encontrar dificuldades e parar}
\end{entry}

\begin{entry}{歌}{ge1}{14}{⽋}[HSK 1]
  \definition[支,首]{s.}{canção | canto}
\end{entry}

\begin{entry}{歌迷}{ge1 mi2}{14,9}{⽋、⾡}
  \definition{s.}{fã de um cantor}
\end{entry}

\begin{entry}{歌曲}{ge1 qu3}{14,6}{⽋、⽈}[HSK 5]
  \definition{s.}{música; obra para as pessoas cantarem, uma combinação de poesia e música}
\end{entry}

\begin{entry}{歌声}{ge1 sheng1}{14,7}{⽋、⼠}[HSK 3]
  \definition{s.}{voz cantada; som de canto}
\end{entry}

\begin{entry}{歌手}{ge1 shou3}{14,4}{⽋、⼿}[HSK 3]
  \definition[个,位,名]{s.}{cantor; vocalista}
\end{entry}

\begin{entry}{阁下}{ge2xia4}{9,3}{⾨、⼀}
  \definition{pron.}{Sua Excelência | Sua Majestade | \emph{Sire}}
\end{entry}

\begin{entry}{格兰菜}{ge2lan2cai4}{10,5,11}{⽊、⼋、⾋}
  \definition{s.}{brócolis chinês | couve chinesa | mostarda}
  \seeref{芥蓝}{gai4lan2}
\end{entry}

\begin{entry}{格外}{ge2wai4}{10,5}{⽊、⼣}[HSK 4]
  \definition{adv.}{especialmente; particularmente; ainda mais; indica mais do que a média | adicionalmente; indica adicional ou extra}
\end{entry}

\begin{entry}{鬲}{ge2}{10}{⿀}
  \definition{s.}{um antigo tripé de cozinha com pernas ocas; uma grande panela de barro}
  \seeref{鬲}{li4}
\end{entry}

\begin{entry}{隔}{ge2}{12}{⾩}[HSK 4]
  \definition{adj.}{seguinte; vizinho}
  \definition{v.}{dividir; separar; bloquear; obstruir | estar a uma distância de, após ou em um intervalo de}
\end{entry}

\begin{entry}{隔壁}{ge2bi4}{12,16}{⾩、⼟}[HSK 5]
  \definition{s.}{vizinho; casas ou pessoas vizinhas | septo; distante (socialmente distante) | anteparo; partição}
\end{entry}

\begin{entry}{隔开}{ge2 kai1}{12,4}{⾩、⼶}[HSK 4]
  \definition{v.}{separar; manter separado; barricar; separar completamente duas pessoas (ou coisas) ou duas partes de uma coisa que estão intimamente unidas}
\end{entry}

\begin{entry}{个}{ge3}{3}{⼈}
  \definition{pron.}{usado em 自个儿}
  \seeref{个}{ge4}
  \seeref{自个儿}{zi4ge3r5}
\end{entry}

\begin{entry}{盖}{ge3}{11}{⽫}
  \definition*{s.}{sobrenome Ge}
  \seeref{盖}{gai4}
\end{entry}

\begin{entry}{个}{ge4}{3}{⼈}[HSK 1]
  \definition{clas.}{para objetos e pessoas em geral}
  \definition{pron.}{isto | aquilo}
  \definition{s.}{indivíduo | tamanho}
  \seeref{个}{ge3}
\end{entry}

\begin{entry}{个别}{ge4bie2}{3,7}{⼈、⼑}[HSK 4]
  \definition{adj.}{muito poucos; excepcionais}
  \definition{adv.}{separadamente; individualmente; isoladamente}
\end{entry}

\begin{entry}{个儿}{ge4r5}{3,2}{⼈、⼉}[HSK 5]
  \definition{s.}{tamanho; altura; estatura; tamanho do corpo ou do objeto |
pessoas ou coisas consideradas isoladamente; referir-se a uma pessoa ou coisa individualmente}
\end{entry}

\begin{entry}{个人}{ge4ren2}{3,2}{⼈、⼈}[HSK 3]
  \definition{pron.}{pessoal; si mesmo}
  \definition[个]{s.}{indivíduo}
\end{entry}

\begin{entry}{个体}{ge4ti3}{3,7}{⼈、⼈}[HSK 4]
  \definition{s.}{pessoa ou organismo individual}
\end{entry}

\begin{entry}{个性}{ge4xing4}{3,8}{⼈、⼼}[HSK 3]
  \definition{s.}{caráter individual; individualidade; personalidade}
\end{entry}

\begin{entry}{个子}{ge4zi5}{3,3}{⼈、⼦}[HSK 2]
  \definition{s.}{altura | estatura}
\end{entry}

\begin{entry}{各}{ge4}{6}{⼝}[HSK 3]
  \definition{adv.}{indica que mais de uma pessoa ou coisa está fazendo algo ou tem um determinado atributo}
  \definition{pron.}{todo; todos; cada | diferentes entre si; vários}
\end{entry}

\begin{entry}{各地}{ge4 di4}{6,6}{⼝、⼟}[HSK 3]
  \definition{s.}{todos os lugares; vários lugares}
\end{entry}

\begin{entry}{各个}{ge4 ge4}{6,3}{⼝、⼈}[HSK 4]
  \definition{adv./pron.}{cada | um a um; um após o outro}
\end{entry}

\begin{entry}{各位}{ge4 wei4}{6,7}{⼝、⼈}[HSK 3]
  \definition{pron.}{todos | cada}
\end{entry}

\begin{entry}{各种}{ge4 zhong3}{6,9}{⼝、⽲}[HSK 3]
  \definition{adv.}{todos os tipos; vários; cada tipo}
\end{entry}

\begin{entry}{各自}{ge4zi4}{6,6}{⼝、⾃}[HSK 3]
  \definition{pron.}{cada; respectivo; por si mesmo}
\end{entry}

\begin{entry}{给}{gei3}{9}{⽷}[HSK 1]
  \definition{prep.}{a | para}
  \definition{v.}{dar | permitir | fazer alguma coisa (para alguém)}
  \seeref{给}{ji3}
\end{entry}

\begin{entry}{给……打电话}{gei3 da3 dian4 hua4}{9,5,5,8}{⽷、⼿、⽥、⾔}
  \definition{expr.}{telefonar para alguém}
  \seeref{打电话}{da3 dian4 hua4}
\end{entry}

\begin{entry}{根}{gen1}{10}{⽊}[HSK 4]
  \definition*{s.}{sobrenome Gen}
  \definition{adv.}{completamente; minuciosamente; radicalmente}
  \definition{clas.}{para objetos finos, alongados}
  \definition{s.}{raiz (de uma planta) | descendentes; posteridade; analogia com as gerações futuras | raiz (abreviação de raiz quadrada) | radical (química, refere-se a radicais carregados) | base; pé; raiz; parte inferior, base ou parte de um objeto que está presa a outra coisa | a parte de baixo das coisas; fonte; a origem  das coisas | base; fundamento}
\end{entry}

\begin{entry}{根本}{gen1ben3}{10,5}{⽊、⽊}[HSK 3]
  \definition{adj.}{básico; essencial; fundamental}
  \definition{adv.}{sempre; simplesmente; absolutamente; de qualquer modo | radically; thoroughly}
  \definition[个]{s.}{base; raiz; fundação}
\end{entry}

\begin{entry}{根据}{gen1ju4}{10,11}{⽊、⼿}[HSK 4]
  \definition[个]{prep.}{com base em; de acordo com; à luz de}
  \definition{s.}{base; fundamentos; razão; fundo; alicerce}
  \definition{v.}{basear; usar algo como premissa para uma conclusão ou como base para uma ação verbal}
\end{entry}

\begin{entry}{跟}{gen1}{13}{⾜}[HSK 1]
  \definition{conj.}{e; com}
  \definition{prep.}{com}
  \definition{v.}{acompanhar junto | seguir de perto | ir com}
\end{entry}

\begin{entry}{跟前}{gen1qian2}{13,9}{⾜、⼑}[HSK 5]
  \definition{s.}{próximo; perto de; na frente de; (na ou para) a presença de alguém | o tempo imediatamente anterior a algum evento; tempo que se aproxima}
  \seeref{跟前}{gen1qian5}
\end{entry}

\begin{entry}{跟前}{gen1qian5}{13,9}{⾜、⼑}
  \definition{v.}{(dos filhos de alguém) viver com alguém (exclusivamente com relação à presença ou ausência de crianças)}
  \seeref{跟前}{gen1qian2}
\end{entry}

\begin{entry}{跟随}{gen1sui2}{13,11}{⾜、⾩}[HSK 5]
  \definition{s.}{seguidor; usado para se referir a alguém que seguiu}
  \definition{v.}{seguir; ir atrás; acompanhar}
\end{entry}

\begin{entry}{更}{geng1}{7}{⽈}
  \definition{s.}{vigia (por exemplo, de sentinela ou guarda)}
  \definition{v.}{alterar ou substituir | experimentar}
  \seeref{更}{geng4}
\end{entry}

\begin{entry}{更换}{geng1 huan4}{7,10}{⽈、⼿}[HSK 5]
  \definition{v.}{alterar; mudar; substituir; comutar}
\end{entry}

\begin{entry}{更新}{geng1xin1}{7,13}{⽈、⽄}[HSK 5]
  \definition{v.}{renovar; atualizar; substituir; remover o antigo e substituir pelo novo}
\end{entry}

\begin{entry}{耕}{geng1}{10}{⽾}
  \definition{v.}{lavrar; arar; cultivar | ganhar a vida; buscar o próprio sustento}
\end{entry}

\begin{entry}{更}{geng4}{7}{⽈}[HSK 2]
  \definition{adv.}{mais | ainda mais}
  \seeref{更}{geng1}
\end{entry}

\begin{entry}{更加}{geng4 jia1}{7,5}{⽈、⼒}[HSK 3]
  \definition{adv.}{mais; ainda mais; em maior grau}
\end{entry}

\begin{entry}{工}{gong1}{3}{⼯}[Kangxi 48]
  \definition{s.}{trabalho | trabalhador | habilidade | profissão | comércio | ofício}
\end{entry}

\begin{entry}{工厂}{gong1chang3}{3,2}{⼯、⼚}[HSK 3]
  \definition[家,座,个]{s.}{fábrica; moinho; planta; obras}
\end{entry}

\begin{entry}{工尺谱}{gong1 che3 pu3}{3,4,14}{⼯、⼫、⾔}
  \definition{s.}{notação musical tradicional chinesa que usa caracteres chineses para representar notas musicais}
\end{entry}

\begin{entry}{工程}{gong1 cheng2}{3,12}{⼯、⽲}[HSK 4]
  \definition[个,项]{s.}{projeto; programa; trabalhos que utilizam equipamentos grandes e complexos, como projetos de reconstrução urbana e projetos de cestas de alimentos, etc. | engenharia; departamentos de produção e manufatura usam equipamentos grandes e complexos para realizar seu trabalho}
\end{entry}

\begin{entry}{工程师}{gong1cheng2shi1}{3,12,6}{⼯、⽲、⼱}[HSK 3]
  \definition[个,名]{s.}{engenheiro}
\end{entry}

\begin{entry}{工夫}{gong1 fu1}{3,4}{⼯、⼤}
  \definition[个]{s.}{tempo | tempo livre; lazer}
  \seeref{工夫}{gong1 fu5}
\end{entry}

\begin{entry}{工夫}{gong1 fu5}{3,4}{⼯、⼤}[HSK 3]
  \definition{s.}{(um período de) tempo | tempo livre}
  \seeref{工夫}{gong1 fu1}
\end{entry}

\begin{entry}{工具}{gong1ju4}{3,8}{⼯、⼋}[HSK 3]
  \definition[个]{s.}{ferramenta; implemento | ferramenta; meio; instrumento}
\end{entry}

\begin{entry}{工龄}{gong1ling2}{3,13}{⼯、⿒}
  \definition{s.}{tempo de serviço | senioridade}
\end{entry}

\begin{entry}{工人}{gong1ren2}{3,2}{⼯、⼈}[HSK 1]
  \definition{s.}{trabalhador | operário | mão de obra de fábrica}
\end{entry}

\begin{entry}{工业}{gong1ye4}{3,5}{⼯、⼀}[HSK 3]
  \definition{s.}{indústria}
\end{entry}

\begin{entry}{工艺}{gong1 yi4}{3,4}{⼯、⾋}[HSK 5]
  \definition{s.}{técnica; tecnologia; arte industrial; técnicas ou métodos de fabricação e processamento de produtos | artesanato; arte artesanal}
\end{entry}

\begin{entry}{工艺品}{gong1 yi4 pin3}{3,4,9}{⼯、⾋、⼝}[HSK 5]
  \definition[个,件]{s.}{trabalho manual; artesanato; habilidade manual; artigo artesanal; itens delicados produzidos com técnicas artesanais. Por exemplo, esculturas em jade, esmaltes Jingtailan, bordados, etc.}
\end{entry}

\begin{entry}{工资}{gong1zi1}{3,10}{⼯、⾙}[HSK 3]
  \definition[份,个,年,月,天]{s.}{pagamento; salário}
\end{entry}

\begin{entry}{工作}{gong1zuo4}{3,7}{⼯、⼈}[HSK 1]
  \definition[个,份,项]{s.}{trabalho | tarefa}
  \definition{v.}{trabalhar | operar (uma máquina)}
\end{entry}

\begin{entry}{工作日}{gong1 zuo4 ri4}{3,7,4}{⼯、⼈、⽇}[HSK 5]
  \definition{s.}{dia de trabalho; dia útil; dias em que você deveria estar trabalhando de acordo com as regras | horas de trabalho por dia; horas do dia para fazer o trabalho necessário}
\end{entry}

\begin{entry}{公布}{gong1bu4}{4,5}{⼋、⼱}[HSK 3]
  \definition{v.}{promulgar; anunciar; publicar; tornar público}
\end{entry}

\begin{entry}{公车}{gong1che1}{4,4}{⼋、⾞}
  \definition{s.}{abreviação de~公共汽车, ônibus}
  \seeref{公共汽车}{gong1gong4qi4che1}
  \seealsoref{公共}{gong1 gong4}
\end{entry}

\begin{entry}{公告}{gong1gao4}{4,7}{⼋、⼝}[HSK 5]
  \definition{s.}{anúncio; notificação de assuntos importantes ao público em geral pelo governo ou por um órgão importante}
  \definition{v.}{anunciar; o governo ou órgão governamental informa publicamente às pessoas algo importante}
\end{entry}

\begin{entry}{公共}{gong1 gong4}{4,6}{⼋、⼋}[HSK 3]
  \definition{adj.}{público; comum; comunal}
  \definition{s.}{ônibus}
  \seealsoref{公车}{gong1che1}
  \seealsoref{公共汽车}{gong1gong4qi4che1}
\end{entry}

\begin{entry}{公共汽车}{gong1gong4qi4che1}{4,6,7,4}{⼋、⼋、⽔、⾞}[HSK 2]
  \definition[辆,班]{s.}{ônibus}
  \seeref{公车}{gong1che1}
  \seealsoref{公共}{gong1 gong4}
\end{entry}

\begin{entry}{公交车}{gong1 jiao1 che1}{4,6,4}{⼋、⼇、⾞}[HSK 2]
  \definition[辆]{s.}{ônibus urbano | veículo de transporte público}
\end{entry}

\begin{entry}{公斤}{gong1jin1}{4,4}{⼋、⽄}[HSK 2]
  \definition{clas.}{quilograma (kg)}
\end{entry}

\begin{entry}{公开}{gong1kai1}{4,4}{⼋、⼶}[HSK 3]
  \definition{adj.}{aberto; público}
  \definition{v.}{tornar público}
\end{entry}

\begin{entry}{公克}{gong1ke4}{4,7}{⼋、⼗}
  \definition{s.}{grama (medida de peso)}
\end{entry}

\begin{entry}{公里}{gong1li3}{4,7}{⼋、⾥}[HSK 2]
  \definition{s.}{quilômetro}
\end{entry}

\begin{entry}{公路}{gong1 lu4}{4,13}{⼋、⾜}[HSK 2]
  \definition[条]{s.}{rodovia | via de trânsito | estrada | auto-estrada}
\end{entry}

\begin{entry}{公民}{gong1min2}{4,5}{⼋、⽒}[HSK 3]
  \definition{s.}{cidadão; civil}
\end{entry}

\begin{entry}{公平}{gong1ping2}{4,5}{⼋、⼲}[HSK 2]
  \definition{adj.}{justo | imparcial | equitativo}
\end{entry}

\begin{entry}{公认}{gong1ren4}{4,4}{⼋、⾔}[HSK 5]
  \definition{v.}{(geralmente) reconhecer; (universalmente) aceitar}
\end{entry}

\begin{entry}{公式}{gong1shi4}{4,6}{⼋、⼷}[HSK 5]
  \definition[个]{s.}{fórmula; expressão}
\end{entry}

\begin{entry}{公司}{gong1si1}{4,5}{⼋、⼝}[HSK 2]
  \definition[家]{s.}{empresa | companhia | corporação | firma}
\end{entry}

\begin{entry}{公司治理}{gong1si1zhi4li3}{4,5,8,11}{⼋、⼝、⽔、⽟}
  \definition{s.}{governança corporativa}
\end{entry}

\begin{entry}{公务员}{gong1 wu4 yuan2}{4,5,7}{⼋、⼒、⼝}[HSK 3]
  \definition[个,名]{s.}{funcionário público}
\end{entry}

\begin{entry}{公用电话}{gong1yong4dian4hua4}{4,5,5,8}{⼋、⽤、⽥、⾔}
  \definition[部]{s.}{telefone público}
\end{entry}

\begin{entry}{公寓}{gong1yu4}{4,12}{⼋、⼧}
  \definition[套]{s.}{prédio de apartamentos | pensão}
\end{entry}

\begin{entry}{公元}{gong1yuan2}{4,4}{⼋、⼉}[HSK 4]
  \definition{s.}{D.C. (Depois de~Cristo); a era cristã; um método internacionalmente aceito de registro de datas, o ano lendário do nascimento de Jesus é 1 d.C., também conhecido como o primeiro ano da Era Comum, e é denotado por D.C.}
  \seealsoref{前}{qian2}
\end{entry}

\begin{entry}{公园}{gong1yuan2}{4,7}{⼋、⼞}[HSK 2]
  \definition[座]{s.}{parque (para recreação pública)}
\end{entry}

\begin{entry}{公正}{gong1zheng4}{4,5}{⼋、⽌}[HSK 5]
  \definition{adj.}{justo; equitativo; imparcial; de mente justa; equidade e integridade sem favoritismo}
\end{entry}

\begin{entry}{功臣}{gong1chen2}{5,6}{⼒、⾂}
  \definition{s.}{oficial meritório | pessoa que presta serviço excepcional, herói | (fig.) alguém que desempenha um papel vital}
\end{entry}

\begin{entry}{功夫}{gong1fu5}{5,4}{⼒、⼤}[HSK 3]
  \definition*{s.}{Gongfu (Kung Fu), arte marcial}
  \definition[番]{s.}{habilidade; feitura | luta acrobática; habilidade em artes marciais | esforço; tempo e energia}
\end{entry}

\begin{entry}{功课}{gong1 ke4}{5,10}{⼒、⾔}[HSK 3]
  \definition[份,门]{s.}{trabalho escolar; dever de casa | tarefa; lições; lição escolar}
\end{entry}

\begin{entry}{功能}{gong1neng2}{5,10}{⼒、⾁}[HSK 3]
  \definition[种,项]{s.}{função}
\end{entry}

\begin{entry}{供应}{gong1 ying4}{8,7}{⼈、⼴}[HSK 4]
  \definition{v.}{fornecer; acomodar}
\end{entry}

\begin{entry}{共}{gong4}{6}{⼋}[HSK 4]
  \definition*{s.}{Abreviação de Partido Comunista | sobrenome Gong}
  \definition{adj.}{conjunto; mútuo; geral; comum; o mesmo para todos}
  \definition{adv.}{juntos; juntamente; conjuntamente | em sua totalidade; em todos}
  \definition{v.}{compartilhar com; empreender ou realizar em conjunto}
\end{entry}

\begin{entry}{共产}{gong4chan3}{6,6}{⼋、⼇}
  \definition{adj.}{comunista}
  \definition{s.}{comunismo}
\end{entry}

\begin{entry}{共产党}{gong4chan3dang3}{6,6,10}{⼋、⼇、⼉}
  \definition*{s.}{Partido Comunista}
\end{entry}

\begin{entry}{共计}{gong4ji4}{6,4}{⼋、⾔}[HSK 5]
  \definition{s.}{total; total geral; agregado; montante}
  \definition{v.}{contar até; somar até; totalizar}
\end{entry}

\begin{entry}{共同}{gong4tong2}{6,6}{⼋、⼝}[HSK 3]
  \definition{adj.}{comum; compartilhado; colaborativo}
  \definition{adv.}{juntos; conjuntamente}
\end{entry}

\begin{entry}{共同体}{gong4tong2ti3}{6,6,7}{⼋、⼝、⼈}
  \definition{s.}{comunidade}
\end{entry}

\begin{entry}{共享}{gong4 xiang3}{6,8}{⼋、⼇}[HSK 5]
  \definition{v.}{compartilhar; desfrutar juntos; aproveitar as coisas boas juntos}
\end{entry}

\begin{entry}{共有}{gong4 you3}{6,6}{⼋、⽉}[HSK 3]
  \definition{v.}{ter completamente; compartilhar; possuir (por todos)}
\end{entry}

\begin{entry}{勾}{gou1}{4}{⼓}
  \definition*{s.}{sobrenome Gou}
  \definition{v.}{atrair | excitar | marcar | atacar | delinear | conspirar}
  \variantof{钩}
  \seeref{勾}{gou4}
\end{entry}

\begin{entry}{沟}{gou1}{7}{⽔}[HSK 5]
  \definition[条]{s.}{canal; vala; sarjeta; trincheira; cursos d'água ou fortificações escavados | ranhura; sulco raso; uma depressão que se assemelha a uma vala | ravina; barranco; cursos d'água}
\end{entry}

\begin{entry}{沟通}{gou1tong1}{7,10}{⽔、⾡}[HSK 5]
  \definition{v.}{comunicar; comunicar-se para entender as ideias, opiniões, etc. | conectar; ligar; estabelecer um paralelo entre os dois}
\end{entry}

\begin{entry}{钩}{gou1}{9}{⾦}
  \definition{s.}{gancho | \emph{check mark} | \emph{tick}}
  \definition{v.}{enganchar | costurar}
\end{entry}

\begin{entry}{狗}{gou3}{8}{⽝}[HSK 2]
  \definition[条,只]{s.}{cão | cachorro}
\end{entry}

\begin{entry}{勾}{gou4}{4}{⼓}
  \definition{s.}{usado em 勾当}
  \seeref{勾}{gou1}
  \seeref{勾当}{gou4dang4}
\end{entry}

\begin{entry}{勾当}{gou4dang4}{4,6}{⼓、⼹}
  \definition{s.}{negócio obscuro}
\end{entry}

\begin{entry}{句}{gou4}{5}{⼝}
  \variantof{勾}
  \seeref{句}{ju4}
\end{entry}

\begin{entry}{构}{gou4}{8}{⽊}
  \definition{s.}{composição literária}
  \definition{v.}{construir | formar | compor}
  \variantof{够}
\end{entry}

\begin{entry}{构成}{gou4cheng2}{8,6}{⽊、⼽}[HSK 4]
  \definition{s.}{parte; componente; composição; estrutura}
  \definition{v.}{formar; compor; constituir; compor; encaixar muitas partes para formar um todo | consistir; causar; formar (principalmente em termos jurídicos)}
\end{entry}

\begin{entry}{构造}{gou4 zao4}{8,10}{⽊、⾡}[HSK 4]
  \definition[种]{s.}{estrutura; construção; disposição, organização e inter-relação dos componentes}
  \definition{v.}{formar; construir}
\end{entry}

\begin{entry}{诟骂}{gou4ma4}{8,9}{⾔、⾺}
  \definition{v.}{abusar verbalmente | insultar | criticar}
\end{entry}

\begin{entry}{购买}{gou4 mai3}{8,6}{⾙、⼄}[HSK 4]
  \definition{v.}{comprar; adquirir; empobrecer}
\end{entry}

\begin{entry}{购物}{gou4wu4}{8,8}{⾙、⽜}[HSK 4]
  \definition{s.}{compras; itens comprados; \emph{shopping}}
\end{entry}

\begin{entry}{够}{gou4}{11}{⼣}[HSK 2]
  \definition{adj.}{suficiente}
  \definition{adv.}{(antes do adj.) realmente}
  \definition{v.}{bastar | chegar}
\end{entry}

\begin{entry}{够本}{gou4ben3}{11,5}{⼣、⽊}
  \definition{v.}{empatar | fazer valer o dinheiro}
\end{entry}

\begin{entry}{够不着}{gou4bu5zhao2}{11,4,11}{⼣、⼀、⽬}
  \definition{v.}{ser incapaz de alcançar}
\end{entry}

\begin{entry}{够得着}{gou4de5zhao2}{11,11,11}{⼣、⼻、⽬}
  \definition{v.}{estar à altura | alcançar}
\end{entry}

\begin{entry}{够格}{gou4ge2}{11,10}{⼣、⽊}
  \definition{adj.}{apto | qualificado | apresentável}
\end{entry}

\begin{entry}{够朋友}{gou4peng2you5}{11,8,4}{⼣、⽉、⼜}
  \definition{v.}{ser um amigo verdadeiro}
\end{entry}

\begin{entry}{够呛}{gou4qiang4}{11,7}{⼣、⼝}
  \definition{adj.}{suficiente | terrível | insuportável | improvável}
\end{entry}

\begin{entry}{够戗}{gou4qiang4}{11,8}{⼣、⼽}
  \variantof{够呛}
\end{entry}

\begin{entry}{够味}{gou4wei4}{11,8}{⼣、⼝}
  \definition{adj.}{excelente | na medida}
\end{entry}

\begin{entry}{彀}{gou4}{13}{⼸}
  \definition{s.}{calcance de um arco e flecha}
  \definition{v.}{puxar um arco ao máximo}
\end{entry}

\begin{entry}{估计}{gu1ji4}{7,4}{⼈、⾔}[HSK 5]
  \definition{v.}{fazer contas; estimar; calcular; julgar a natureza, quantidade, mudança, etc. de uma coisa em uma determinada situação | parecer; parecer como se; aparentar; fazer inferências aproximadas sobre a natureza, a quantidade e a mudança das coisas com base em determinadas circunstâncias}
\end{entry}

\begin{entry}{姑娘}{gu1niang5}{8,10}{⼥、⼥}[HSK 3]
  \definition[位,个]{s.}{menina; jovem senhora; mulher solteira | filha}
\end{entry}

\begin{entry}{姑且}{gu1qie3}{8,5}{⼥、⼀}
  \definition{adv.}{provisoriamente | por enquanto}
\end{entry}

\begin{entry}{孤独}{gu1du2}{8,9}{⼦、⽝}
  \definition{adj.}{solitário}
\end{entry}

\begin{entry}{古}{gu3}{5}{⼝}[HSK 3]
  \definition*{s.}{sobrenome Gu}
  \definition{adj.}{arcaico; antigo; antiquíssimo}
  \definition{pref.}{``paleo''; ``arqueo''}
  \definition{s.}{antiguidade; arcaísmo | livros ou ortodoxias de antigos sábios | uma forma de poesia pré-Tang}
\end{entry}

\begin{entry}{古城}{gu3cheng2}{5,9}{⼝、⼟}
  \definition{s.}{cidade antiga}
\end{entry}

\begin{entry}{古代}{gu3dai4}{5,5}{⼝、⼈}[HSK 3]
  \definition{s.}{tempos antigos | sociedade antiga; sociedade primitiva | antigamente}
\end{entry}

\begin{entry}{古老}{gu3 lao3}{5,6}{⼝、⽼}[HSK 5]
  \definition{adj.}{antigo; antiquado; histórico}
\end{entry}

\begin{entry}{古人}{gu3ren2}{5,2}{⼝、⼈}
  \definition{s.}{pessoas dos tempos antigos | os antigos | espécies humanas extintas, como \emph{Homo erectus} ou \emph{Homo neanderthalensis} | (literário) pessoa falecida}
\end{entry}

\begin{entry}{古铜色}{gu3tong2 se4}{5,11,6}{⼝、⾦、⾊}
  \definition{s.}{cor bronze}
\end{entry}

\begin{entry}{谷}{gu3}{7}{⾕}[Kangxi 150]
  \definition{adj.}{bom; gentil;}
  \definition{s.}{vale; ravina; desfiladeiro; garganta; faixa estreita de terra com uma saída no meio de duas colinas ou dois platôs | arroz não descascado | salário de funcionário (na época feudal) |calha; cocho; canal | fossa sob o cerebelo (anatomia); valécula | dificuldade; dilema}
  \definition{v.}{criar (filhos) | crescer}
\end{entry}

\begin{entry}{骨}{gu3}{9}{⾻}[Kangxi 188]
  \definition{s.}{osso}
\end{entry}

\begin{entry}{骨头}{gu3tou5}{9,5}{⾻、⼤}[HSK 4]
  \definition[根,块]{s.}{osso; tecidos mais duros no corpo de uma pessoa ou de alguns animais que sustentam o corpo ou protegem os órgãos do corpo | caráter de uma pessoa; refere-se à qualidade do caráter de uma pessoa}
\end{entry}

\begin{entry}{鼓}{gu3}{13}{⿎}[HSK 5]
  \definition*{s.}{sobrenome Gu}
  \definition{adj.}{abaulado; inchado; saliente; protuberante}
  \definition{clas.}{unidades antigas de cronometragem noturna; vigílias da noite}
  \definition{s.}{tambor; instrumento de percussão |
coisas semelhantes a tambores; formato, som e função semelhantes aos de um tambor |}
  \definition{v.}{soar; bater; golpear; fazer um objeto soar | ventilar; soprar com fole | agitar; despertar; ativar; incitar; revigorar | bater asas | aumentar; fazer beicinho}
\end{entry}

\begin{entry}{鼓励}{gu3li4}{13,7}{⿎、⼒}[HSK 5]
  \definition{v.}{incitar; encorajar; provocar e incentivar}
\end{entry}

\begin{entry}{鼓掌}{gu3zhang3}{13,12}{⿎、⼿}[HSK 5]
  \definition{v.+compl.}{aplaudir; bater palmas, principalmente para expressar felicidade, aprovação ou boas-vindas}
\end{entry}

\begin{entry}{固定}{gu4ding4}{8,8}{⼞、⼧}[HSK 4]
  \definition{adj.}{fixo; regular; inalterado ou imóvel}
  \definition{v.}{consertar; tornar fixo, não mover novamente; colocar as coisas em ordem, não mudá-las novamente}
\end{entry}

\begin{entry}{故}{gu4}{9}{⽁}
  \definition{conj.}{por isso | portanto | então}
\end{entry}

\begin{entry}{故宫}{gu4gong1}{9,9}{⽁、⼧}
  \definition*{s.}{Palácio Imperial | Cidade Proibida}
\end{entry}

\begin{entry}{故事}{gu4shi4}{9,8}{⽁、⼅}
  \definition{s.}{prática antiga}
  \seeref{故事}{gu4shi5}
\end{entry}

\begin{entry}{故事}{gu4shi5}{9,8}{⽁、⼅}[HSK 2]
  \definition{s.}{narrativa | história | conto}
  \seeref{故事}{gu4shi4}
\end{entry}

\begin{entry}{故乡}{gu4xiang1}{9,3}{⽁、⼄}[HSK 3]
  \definition[个]{s.}{cidade natal; terra natal}
\end{entry}

\begin{entry}{故意}{gu4yi4}{9,13}{⽁、⼼}[HSK 2]
  \definition{adv.}{intencionalmente | deliberadamente | propositalmente}
\end{entry}

\begin{entry}{顾客}{gu4ke4}{10,9}{⾴、⼧}[HSK 2]
  \definition[位]{s.}{cliente}
\end{entry}

\begin{entry}{顾问}{gu4wen4}{10,6}{⾴、⾨}[HSK 5]
  \definition{s.}{conselheiro; consultor; assessor; pessoas com conhecimento especializado ou experiência contratadas para prestar consultoria a organizações ou indivíduos}
\end{entry}

\begin{entry}{瓜}{gua1}{5}{⽠}[HSK 4][Kangxi 97]
  \definition*{s.}{sobrenome Gua}
  \definition[个]{s.}{qualquer tipo de melão ou cabaça | companheiro (termo depreciativo para uma pessoa)}
\end{entry}

\begin{entry}{刮}{gua1}{8}{⼑}
  \definition{v.}{ventar | soprar (vento)}
\end{entry}

\begin{entry}{刮风}{gua1feng1}{8,4}{⼑、⾵}
  \definition{v.+compl.}{ventar | fazer vento}
\end{entry}

\begin{entry}{挂}{gua4}{9}{⼿}[HSK 3]
  \definition{clas.}{para conjuntos ou sequência de itens}
  \definition{v.}{pendurar; colocar; suspender | interromper chamada (telefônica) | colocar alguém em contato com; ligar; telefonar
pegar carona; ser pego | ter em mente; estar preocupado com | ser revestido com; ser coberto com | colocar em registro; registrar}
\end{entry}

\begin{entry}{挂号}{gua4hao4}{9,5}{⼿、⼝}
  \definition{v.+compl.}{registrar-se (em um hospital, etc.) | enviar através de carta registrada}
\end{entry}

\begin{entry}{挂号信}{gua4hao4xin4}{9,5,9}{⼿、⼝、⼈}
  \definition{s.}{carta registrada}
\end{entry}

\begin{entry}{乖乖}{guai1guai1}{8,8}{⼃、⼃}
  \definition{adj.}{bem-comportado (criança) | obediente}
  \seeref{乖乖}{guai1guai5}
\end{entry}

\begin{entry}{乖乖}{guai1guai5}{8,8}{⼃、⼃}
  \definition{expr.}{Graças a Deus! | Oh meu Deus!}
  \seeref{乖乖}{guai1guai1}
\end{entry}

\begin{entry}{拐}{guai3}{8}{⼿}
  \definition{s.}{bengala | muleta}
  \definition{v.}{virar (uma esquina, etc.) | cortar | sequestrar | fraudar | apropriar-se indevidamente}
\end{entry}

\begin{entry}{怪}{guai4}{8}{⼼}[HSK 4,5]
  \definition*{s.}{sobrenome Guai}
  \definition{adj.}{estranho; esquisito; peculiar; excêntrico; pitoresco; monstruoso; desconcertante; anormal; incomum}
  \definition{adv.}{bastante; muito}
  \definition{s.}{monstro; demônio; diabo; ser maligno}
  \definition{v.}{achar algo estranho; admirar; ficar surpreso | culpar; repreender}
\end{entry}

\begin{entry}{怪癖}{guai4pi3}{8,18}{⼼、⽧}
  \definition{adj.}{peculiar}
  \definition{s.}{excentricidade | peculiaridade | hobby estranho}
\end{entry}

\begin{entry}{怪兽}{guai4shou4}{8,11}{⼼、⼋}
  \definition{s.}{animal raro | animal mítico | monstro}
\end{entry}

\begin{entry}{关}{guan1}{6}{⼋}[HSK 1,4]
  \definition*{s.}{sobrenome Guan}
  \definition{s.}{passagem; ponto de controle | alfândega; escritórios de cobrança de impostos para exportação e importação de mercadorias | ponto de inflexão ou barreira; ponto de virada ou dificuldade | momento crítico; mecanismo}
  \definition{v.}{fechar; encerrar; amarrar algo | fechar; trancar | encerrar; sair do mercado; falir | conceder ou sacar o pagamento de um salário | desligar | envolver; preocupar-se; conectar-se}
\end{entry}

\begin{entry}{关闭}{guan1bi4}{6,6}{⼋、⾨}[HSK 4]
  \definition{v.}{fechar | (empresa) falir}
\end{entry}

\begin{entry}{关怀}{guan1huai2}{6,7}{⼋、⼼}[HSK 5]
  \definition{v.}{mostrar cuidado amoroso por; mostrar solicitude por; cuidar, amar, apoiar ou ajudar os fracos ou grupos em dificuldade | geralmente usado para superiores para subordinados, anciãos para juniores ou organizações para indivíduos}
\end{entry}

\begin{entry}{关机}{guan1 ji1}{6,6}{⼋、⽊}[HSK 2]
  \definition{v.}{encerrar | finalizar | desligar}
\end{entry}

\begin{entry}{关键}{guan1jian4}{6,13}{⼋、⾦}[HSK 5]
  \definition{adj.}{crucial; decisivo; importante; que pode determinar o curso e o resultado dos eventos}
  \definition[个]{s.}{chave; ponto crucial; aspectos ou condições mais importantes que determinam o desenvolvimento e o resultado de algo}
\end{entry}

\begin{entry}{关上}{guan1 shang5}{6,3}{⼋、⼀}[HSK 1]
  \definition{v.}{fechar (uma porta) | fechar | desligar (luz, equipamento elétrico etc.)}
\end{entry}

\begin{entry}{关系}{guan1xi5}{6,7}{⼋、⽷}[HSK 3]
  \definition[个,种]{s.}{relações; conexões; relacionamento | consequência; impacto; significado | causa; razão (geralmente usado com 由于 ou 因为) | credenciais que mostram filiação a uma organização}
  \definition{v.}{preocupar; afetar; ter influência sobre; ter a ver com}
  \seealsoref{因为}{yin1wei4}
  \seealsoref{由于}{you2yu2}
\end{entry}

\begin{entry}{关心}{guan1xin1}{6,4}{⼋、⼼}[HSK 2]
  \definition{v.}{cuidar de | preocupar-se com | expressar interesse em | mostrar solicitude por}
\end{entry}

\begin{entry}{关于}{guan1yu2}{6,3}{⼋、⼆}[HSK 4]
  \definition{prep.}{sobre; relativo a; pertencente a; uma questão de; com relação a}
\end{entry}

\begin{entry}{关注}{guan1 zhu4}{6,8}{⼋、⽔}[HSK 3]
  \definition{s.}{preocupação; interesse; atenção}
  \definition{v.}{prestar atenção em; seguir algo de perto; seguir (nas redes sociais)}
\end{entry}

\begin{entry}{观察}{guan1cha2}{6,14}{⾒、⼧}[HSK 3]
  \definition{v.}{assistir; pesquisar; observar}
\end{entry}

\begin{entry}{观点}{guan1dian3}{6,9}{⾒、⽕}[HSK 2]
  \definition{s.}{ponto de vista | perspectiva}
\end{entry}

\begin{entry}{观看}{guan1 kan4}{6,9}{⾒、⽬}[HSK 3]
  \definition{v.}{assistir; ver}
\end{entry}

\begin{entry}{观念}{guan1nian4}{6,8}{⾒、⼼}[HSK 3]
  \definition[个]{s.}{ideia; conceito}
\end{entry}

\begin{entry}{观众}{guan1zhong4}{6,6}{⾒、⼈}[HSK 3]
  \definition[位,名,批,个]{s.}{espectador; audiência}
\end{entry}

\begin{entry}{官}{guan1}{8}{⼧}[HSK 4]
  \definition*{s.}{sobrenome Guan}
  \definition{adj.}{propriedade do governo; pertencente ao governo ou ao público | público}
  \definition[个,位]{s.}{funcionário do governo; oficial; servidor público; titular de cargo; funcionário público nomeado acima de um determinado nível | órgão (parte do tecido do corpo)}
\end{entry}

\begin{entry}{官方}{guan1fang1}{8,4}{⼧、⽅}[HSK 4]
  \definition{s.}{autoridade; (do ou pelo) governo | oficial (de uma organização ou instituição)}
\end{entry}

\begin{entry}{官桂}{guan1gui4}{8,10}{⼧、⽊}
  \definition{s.}{canela}
  \seealsoref{肉桂}{rou4gui4}
\end{entry}

\begin{entry}{冠}{guan1}{9}{⼍}
  \definition{s.}{chapéu | coroa | brasão | boné}
  \seeref{冠}{guan4}
\end{entry}

\begin{entry}{棺}{guan1}{12}{⽊}
  \definition{s.}{caixão | esquife | ataúde}
\end{entry}

\begin{entry}{管}{guan3}{14}{⽵}[HSK 3]
  \definition*{s.}{sobrenome Guan}
  \definition{adj.}{estreito; restrito; limitado; pequeno}
  \definition{clas.}{para objetos cilíndricos finos}
  \definition{conj.}{não importa (o que, como, etc.)}
  \definition{prep.}{a função é semelhante a ``把'', usada especificamente em conjunto com ``叫''.}
  \definition[根,条,排]{s.}{cano; tubo | instrumento musical de sopro | válvula; tubo | duto; canal; vasos}
  \definition{v.}{estar encarregado de; gerenciar; executar; supervisionar | administrar; governar | sujeitar alguém à disciplina | assumir; arcar | interferir; incomodar | garantir; assegurar; fornecer}
\end{entry}

\begin{entry}{管家}{guan3jia1}{14,10}{⽵、⼧}
  \definition{s.}{mordomo | governanta}
  \definition{v.}{administrar uma casa}
\end{entry}

\begin{entry}{管理}{guan3li3}{14,11}{⽵、⽟}[HSK 3]
  \definition{v.}{gerenciar; executar; administrar; governar; estar encarregado de | controlar; gerenciar | cuidar de}
\end{entry}

\begin{entry}{冠}{guan4}{9}{⼍}
  \definition*{s.}{sobrenome Guan}
  \definition{v.}{colocar um chapéu | ser o primeiro | dublar}
  \seeref{冠}{guan1}
\end{entry}

\begin{entry}{冠军}{guan4jun1}{9,6}{⼍、⼍}[HSK 5]
  \definition[个]{s.}{campeão; medalhista de ouro; primeiro lugar em esportes e outras competições}
\end{entry}

\begin{entry}{光}{guang1}{6}{⼉}[HSK 3]
  \definition*{s.}{sobrenome Guang}
  \definition{adj.}{suave; brilhante | nu; despido; descoberto | esgotado; sem nada sobrando | glorioso; gracioso | brilhante}
  \definition{adv.}{somente; sozinho; meramente}
  \definition{s.}{luz; raio | cenário | honra; glória; brilho | claridade | favor; graça | momento | corpo celeste}
  \definition{v.}{glorificar; recuperar; reconquistar | estar nu | brilhar}
\end{entry}

\begin{entry}{光临}{guang1lin2}{6,9}{⼉、⼁}[HSK 4]
  \definition{v.}{honrar com sua presença, uma palavra de honra, usada para dizer que um convidado chegou}
\end{entry}

\begin{entry}{光明}{guang1ming2}{6,8}{⼉、⽇}[HSK 3]
  \definition{adj.}{brilhante | ingênuo | justo; honesto}
  \definition{s.}{luz}
\end{entry}

\begin{entry}{光盘}{guang1pan2}{6,11}{⼉、⽫}[HSK 4]
  \definition[片,张]{s.}{CD; disco compacto; um disco circular feito de plástico rígido composto que usa um laser para registrar e ler informações}
\end{entry}

\begin{entry}{光槃}{guang1pan2}{6,14}{⼉、⽊}
  \variantof{光盘}
\end{entry}

\begin{entry}{光荣}{guang1rong2}{6,9}{⼉、⾋}[HSK 5]
  \definition{adj.}{honroso; honrado; glorioso; por fazer algo que é benéfico para o país ou para a coletividade e que é considerado por todos como digno de respeito ou elogio}
  \definition{s.}{honra; glória; crédito; sentimento de honra decorrente do fato de ser respeitado ou elogiado por fazer algo importante ou grandioso}
\end{entry}

\begin{entry}{光污染}{guang1 wu1ran3}{6,6,9}{⼉、⽔、⽊}
  \definition{s.}{poluição luminosa}
\end{entry}

\begin{entry}{光线}{guang1 xian4}{6,8}{⼉、⽷}[HSK 5]
  \definition[条,道]{s.}{luz; feixe luminoso; raio de luz}
\end{entry}

\begin{entry}{广}{guang3}{3}{⼴}[HSK 5]
  \definition{adj.}{largo; vasto; amplo; extenso; (oposto a ``狭'')}
  \seeref{广}{an1}
  \seeref{广}{yan3}
  \seealsoref{狭}{xia2}
\end{entry}

\begin{entry}{广播}{guang3bo1}{3,15}{⼴、⼿}[HSK 3]
  \definition[个]{s.}{programa de rádio; transmissão (de rádio)}
  \definition{v.}{transmitir; estar no ar | espalhar-se amplamente; ser conhecido em toda parte}
\end{entry}

\begin{entry}{广场}{guang3chang3}{3,6}{⼴、⼟}[HSK 2]
  \definition{s.}{praça | praça pública | esplanada}
\end{entry}

\begin{entry}{广场舞}{guang3chang3wu3}{3,6,14}{⼴、⼟、⾇}
  \definition{s.}{quadrilha, uma rotina de exercícios tocada com música em quadrados públicos, parques e praças, popular especialmente entre mulheres de meia-idade e aposentados na China}
\end{entry}

\begin{entry}{广大}{guang3da4}{3,3}{⼴、⼤}[HSK 3]
  \definition{adj.}{muito difundido | (uma área ou espaço) vasto; extenso; em grande escala | numeroso}
\end{entry}

\begin{entry}{广东}{guang3dong1}{3,5}{⼴、⼀}
  \definition*{s.}{Guangdong}
\end{entry}

\begin{entry}{广泛}{guang3fan4}{3,7}{⼴、⽔}[HSK 5]
  \definition{adj.}{amplo; extenso; de grande alcance; disseminado; escopo e cobertura amplos}
\end{entry}

\begin{entry}{广告}{guang3gao4}{3,7}{⼴、⼝}[HSK 2]
  \definition[项]{s.}{publicidade | anúncio publicitário}
\end{entry}

\begin{entry}{逛}{guang4}{10}{⾡}[HSK 4]
  \definition{v.}{perambular; passear; vaguear}
\end{entry}

\begin{entry}{归}{gui1}{5}{⼹}[HSK 4]
  \definition*{s.}{sobrenome Gui}
  \definition{s.}{divisão no ábaco com divisor de um dígito}
  \definition{v.}{retornar; voltar para; voltar (ou ir) | devolver algo a; dar de volta a | convergir; juntar-se | encarregar alguém de algo | atribuir a; pertencer a | usado entre os mesmos verbos, indicando que a ação não levou ao resultado correspondente}
\end{entry}

\begin{entry}{龟速}{gui1su4}{7,10}{⿔、⾡}
  \definition{adv.}{tão lento quanto uma tartaruga}
\end{entry}

\begin{entry}{规定}{gui1ding4}{8,8}{⾒、⼧}[HSK 3]
  \definition[个,条,项]{s.}{regra; regulamento; estipulação}
  \definition{v.}{estipular; prover; prescrever}
\end{entry}

\begin{entry}{规范}{gui1fan4}{8,9}{⾒、⾋}[HSK 3]
  \definition{adj.}{regular; normal; padrão; atendendo às especificações}
  \definition{s.}{norma; padrão}
  \definition{v.}{regular; padronizar}
\end{entry}

\begin{entry}{规划}{gui1hua4}{8,6}{⾒、⼑}[HSK 5]
  \definition{s.}{plano; projeto; planejamento; programa; programação; esquematização; plano de desenvolvimento de longo prazo mais abrangente |}
  \definition{v.}{planejar; programar;}
\end{entry}

\begin{entry}{规律}{gui1lv4}{8,9}{⾒、⼻}[HSK 4]
  \definition{adj.}{estável; regular; coisas, comportamentos, fenômenos, etc. que ocorrem em um determinado momento}
  \definition{s.}{lei; padrão regular; conexão essencial e recorrente entre as coisas}
\end{entry}

\begin{entry}{规模}{gui1mo2}{8,14}{⾒、⽊}[HSK 4]
  \definition[个]{s.}{escala; escopo; dimensões; padrão, forma ou escopo (de um empreendimento, instituição, projeto, movimento, etc.)}
\end{entry}

\begin{entry}{规则}{gui1ze2}{8,6}{⾒、⼑}[HSK 4]
  \definition{adj.}{ordenado; regular; descreve a forma, estrutura, arranjo, etc., que se conformam a uma determinada maneira organizada}
  \definition{s.}{regra; regulamento; sistema ou código de conduta prescrito para observância comum | lei; norma}
\end{entry}

\begin{entry}{鬼}{gui3}{9}{⿁}[HSK 5]
  \definition*{s.}{Gui, uma das mansões lunares | sobrenome Gui}
  \definition{adj.}{evasivo; furtivo; sub-reptício; ardiloso; enganoso, malicioso; obscuro | terrível; ruim; severo; vil | esperto; astuto; inteligente}
  \definition{s.}{espírito; fantasma; aparição; refere-se à alma de uma pessoa após a morte | usado para formar um termo de abuso para caráter ignóbil; refere-se a pessoas que têm maus hábitos ou cujo comportamento é repugnante | companheiro; pessoa que é considerada divertida}
\end{entry}

\begin{entry}{鬼怪}{gui3guai4}{9,8}{⿁、⼼}
  \definition{s.}{\emph{hobgoblin} | bicho-papão | fantasma}
\end{entry}

\begin{entry}{鬼火}{gui3huo3}{9,4}{⿁、⽕}
  \definition{s.}{fogo-fátuo | boitatá | fogo corredor | fogo de santelmo}
\end{entry}

\begin{entry}{柜子}{gui4 zi5}{8,3}{⽊、⼦}[HSK 5]
  \definition[个]{s.}{gabinete; armário; dispositivo para guardar roupas, documentos, livros, etc.}
\end{entry}

\begin{entry}{贵}[⻉]{gui4}[中一⻉]{9}{⾙}[HSK 1]
  \definition{adj.}{caro | nobre | precioso}
\end{entry}

\begin{entry}{贵姓}{gui4xing4}{9,8}{⾙、⼥}
  \definition{expr.}{qual seu sobrenome?}
\end{entry}

\begin{entry}{跪拜}{gui4bai4}{13,9}{⾜、⼿}
  \definition{v.}{prostrar-se | ajoelhar-se e adorar}
\end{entry}

\begin{entry}{滚}{gun3}{13}{⽔}[HSK 5]
  \definition*{s.}{sobrenome Gun}
  \definition{adj.}{rolante | fervente | precipitado; torrencial}
  \definition{adv.}{muito; em um grau elevado}
  \definition{v.}{rolar; girar; virar | escapar; fugir; ir embora | ferver | amarrar; aparar; fazer bainha}
\end{entry}

\begin{entry}{滚滚}{gun3gun3}{13,13}{⽔、⽔}
  \definition*{s.}{Apelido para um panda}
  \definition{v.}{mover-se | rolar | fluir continuamente}
\end{entry}

\begin{entry}{滚轮}{gun3lun2}{13,8}{⽔、⾞}
  \definition{s.}{pneu | dial rotativo | roda de rolagem (\emph{scroll})  (mouse de computador)}
\end{entry}

\begin{entry}{过}{guo1}{6}{⾡}
  \definition*{s.}{sobrenome Guo}
  \seeref{过}{guo4}
  \seeref{过}{guo5}
\end{entry}

\begin{entry}{锅}{guo1}{12}{⾦}[HSK 5]
  \definition[口,只]{s.}{panela; frigideira; utensílios de cozinha, redondos e côncavos, feitos principalmente de ferro, alumínio, etc. | parte que se parece com um pote em alguns objetos}
\end{entry}

\begin{entry}{囯}{guo2}{7}{⼞}
  \variantof{国}
\end{entry}

\begin{entry}{国}{guo2}{8}{⼞}[HSK 1]
  \definition*{s.}{sobrenome Guo}
  \definition[个]{s.}{país | nação}
\end{entry}

\begin{entry}{国宾馆}{guo2bin1guan3}{8,10,11}{⼞、⼧、⾷}
  \definition{s.}{pousada estadual}
\end{entry}

\begin{entry}{国歌}{guo2ge1}{8,14}{⼞、⽋}
  \definition{s.}{hino nacional}
\end{entry}

\begin{entry}{国籍}{guo2ji2}{8,20}{⼞、⽵}[HSK 5]
  \definition{s.}{nacionalidade; cidadania; refere-se à identidade de um indivíduo como pertencente a um Estado | identidade nacional (de um avião, navio, etc.)}
\end{entry}

\begin{entry}{国际}{guo2ji4}{8,7}{⼞、⾩}[HSK 2]
  \definition{adj.}{internacional}
\end{entry}

\begin{entry}{国际儿童节}{guo2ji4 er2tong2jie2}{8,7,2,12,5}{⼞、⾩、⼉、⽴、⾋}
  \definition*{s.}{Dia Internacional das Crianças (1~de~junho)}
\end{entry}

\begin{entry}{国际妇女节}{guo2ji4 fu4nv3jie2}{8,7,6,3,5}{⼞、⾩、⼥、⼥、⾋}
  \definition*{s.}{Dia Internacional das Mulheres (8~de~março)}
\end{entry}

\begin{entry}{国际劳动节}{guo2ji4 lao2dong4 jie2}{8,7,7,6,5}{⼞、⾩、⼒、⼒、⾋}
  \definition*{s.}{Dia Internacional dos Trabalhadores (1~de~maio)}
\end{entry}

\begin{entry}{国家}{guo2jia1}{8,10}{⼞、⼧}[HSK 1]
  \definition[个]{s.}{país | nação | estado}
\end{entry}

\begin{entry}{国民}{guo2 min2}{8,5}{⼞、⽒}[HSK 5]
  \definition[个]{s.}{membro de uma nação; povo de uma nação}
\end{entry}

\begin{entry}{国内}{guo2 nei4}{8,4}{⼞、⼌}[HSK 3]
  \definition{s.}{interno (a um país); doméstico; lar}
\end{entry}

\begin{entry}{国旗}{guo2qi2}{8,14}{⼞、⽅}
  \definition[面]{s.}{bandeira (de um país)}
\end{entry}

\begin{entry}{国庆}{guo2 qing4}{8,6}{⼞、⼴}[HSK 3]
  \definition*{s.}{Dia Nacional}
\end{entry}

\begin{entry}{国庆节}{guo2qing4jie2}{8,6,5}{⼞、⼴、⾋}
  \definition*{s.}{Dia Nacional (1~de~outubro)}
\end{entry}

\begin{entry}{国人}{guo2ren2}{8,2}{⼞、⼈}
  \definition{s.}{compatriota}
\end{entry}

\begin{entry}{国外}{guo2 wai4}{8,5}{⼞、⼣}[HSK 1]
  \definition{adj.}{no exterior | externo (assuntos) | estrangeiro}
\end{entry}

\begin{entry}{国语}{guo2yu3}{8,9}{⼞、⾔}
  \definition*{s.}{Língua Chinesa (Mandarim), enfatizando sua natureza nacional}
\end{entry}

\begin{entry}{果酱}{guo3jiang4}{8,13}{⽊、⾣}
  \definition{s.}{geléia | compota ou doce (de frutas)}
\end{entry}

\begin{entry}{果然}{guo3ran2}{8,12}{⽊、⽕}[HSK 3]
  \definition{adv.}{realmente; como esperado; com certeza}
  \definition{conj.}{se realmente; se de fato}
\end{entry}

\begin{entry}{果实}{guo3shi2}{8,8}{⽊、⼧}[HSK 4]
  \definition[种]{s.}{fruta; o órgão que se desenvolve a partir do ovário ou com outras partes da flor após a fertilização da flor | ganhos; frutos;  uma metáfora para conquista ou recompensa por trabalho árduo}
\end{entry}

\begin{entry}{果汁}{guo3zhi1}{8,5}{⽊、⽔}[HSK 3]
  \definition[杯,瓶,种]{s.}{suco; suco de fruta}
\end{entry}

\begin{entry}{果子}{guo3zi5}{8,3}{⽊、⼦}
  \definition{s.}{fruta}
\end{entry}

\begin{entry}{过}{guo4}{6}{⾡}[HSK 1,2]
  \definition{part.}{passado}
  \definition{v.}{atravessar | passar (tempo)}
  \seeref{过}{guo1}
  \seeref{过}{guo5}
\end{entry}

\begin{entry}{过不惯}{guo4 bu5 guan4}{6,4,11}{⾡、⼀、⼼}
  \definition{v.}{não se acostumar | não se habituar}
  \seeref{过惯}{guo4guan4}
\end{entry}

\begin{entry}{过程}{guo4cheng2}{6,12}{⾡、⽲}[HSK 3]
  \definition[个]{s.}{curso dos eventos; processo}
\end{entry}

\begin{entry}{过度}{guo4du4}{6,9}{⾡、⼴}[HSK 5]
  \definition{adj.}{excessivo; acima do limite; além do limite; além do que é apropriado}
\end{entry}

\begin{entry}{过分}{guo4fen4}{6,4}{⾡、⼑}[HSK 4]
  \definition{adj.}{excessivo; muito longe; demais; falar ou agir além dos limites ou graus adequados}
  \definition{adv.}{excessivamente; indevidamente; muito mesmo}
\end{entry}

\begin{entry}{过关}{guo4guan1}{6,6}{⾡、⼋}
  \definition{v.+compl.}{passar uma barreira | passar por uma provação | passar em um teste | atingir um padrão | passar pela alfândega}
\end{entry}

\begin{entry}{过惯}{guo4guan4}{6,11}{⾡、⼼}
  \definition{v.}{estar acostumado (a um certo estilo de vida, etc.)}
  \seeref{过不惯}{guo4 bu5 guan4}
\end{entry}

\begin{entry}{过节}{guo4jie2}{6,5}{⾡、⾋}
  \definition{v.+compl.}{celebrar festividades | comemorar um festival}
\end{entry}

\begin{entry}{过来}{guo4 lai2}{6,7}{⾡、⽊}[HSK 2]
  \definition{v.}{atravessar (para a minha localização) | vir até aqui}
\end{entry}

\begin{entry}{过敏}{guo4min3}{6,11}{⾡、⽁}[HSK 5]
  \definition{adj.}{sensível; excessivamente sensível; resposta acima do normal; ceticismo excessivo}
  \definition{v.}{ser alérgico a}
\end{entry}

\begin{entry}{过年}{guo4 nian2}{6,6}{⾡、⼲}[HSK 2]
  \definition{v.+compl.}{comemorar o Ano Novo | comemorar o Festival da Primavera | passar o Ano Novo | passar o Festival da Primavera}
\end{entry}

\begin{entry}{过期}{guo4qi1}{6,12}{⾡、⽉}
  \definition{v.+compl.}{exceder a data | passar a data | expirar (passar a data de expiração)}
\end{entry}

\begin{entry}{过去}{guo4 qu4}{6,5}{⾡、⼛}[HSK 2,3]
  \definition{v.}{atravessar, passar por (a partir da minha localização) | falecer}
\end{entry}

\begin{entry}{过瘾}{guo4yin3}{6,16}{⾡、⽧}
  \definition{adj.}{gratificante | imensamente agradável | satisfatório}
  \definition{v.+compl.}{satisfazer um desejo | se divertir com algo}
\end{entry}

\begin{entry}{过于}{guo4yu2}{6,3}{⾡、⼆}[HSK 5]
  \definition{adv.}{demais; indevidamente; excessivamente; advérbios de grau ou quantidade excessiva}
\end{entry}

\begin{entry}{过}{guo5}{6}{⾡}
  \definition{part.}{(marcador de ação experiente)}
  \seeref{过}{guo1}
  \seeref{过}{guo4}
\end{entry}

%%%%% EOF %%%%%


%%%
%%% H
%%%

\section*{H}\addcontentsline{toc}{section}{H}

\begin{entry}{哈哈}{ha1 ha1}{9,9}{⼝、⼝}[HSK 3]
  \definition{expr.}{(onomatopéia)  ha ha; o som de uma risada alta}
\end{entry}

\begin{entry}{哈马斯}{ha1ma3si1}{9,3,12}{⼝、⾺、⽄}
  \definition*{s.}{Hamas (Grupo Palestino)}
\end{entry}

\begin{entry}{咳}{hai1}{9}{⼝}
  \definition{interj.}{expressa tristeza, arrependimento ou espanto}
\end{entry}

\begin{entry}{还}{hai2}{7}{⾡}[HSK 1]
  \definition{adv.}{ainda; indica que a ação ou estado permanece inalterado, equivalente a 仍然 | também; além disso; em adição; indica que há um aumento ou suplemento além do escopo já indicado | ainda mais; usado com 比 para indicar que as características e o grau das coisas comparadas aumentaram, o que é equivalente a 更加 razoavelmente; medianamente; usado antes de um adjetivo, indica que algo atinge apenas o nível mínimo exigido | mesmo; usado na primeira parte da frase como complemento, e na segunda parte como conclusão, equivalente a 尚且 | que expressa realização ou descoberta; expressa surpresa por algo que não se esperava, mas que acabou acontecendo | tão cedo quanto; por um curto período de tempo; indica que já era assim há muito tempo | para dar ênfase; para reforçar o tom}
  \seeref{还}{huan2}
  \seealsoref{比}{bi3}
  \seealsoref{更加}{geng4 jia1}
  \seealsoref{仍然}{reng2ran2}
  \seealsoref{尚且}{shang4qie3}
\end{entry}

\begin{entry}{还是}{hai2shi5}{7,9}{⾡、⽇}[HSK 1]
  \definition{adv.}{ainda; ainda assim; não é a continuação de um determinado estado, fenômeno ou ação; o resultado é o mesmo de antes, sem mudanças  |que expressa uma preferência por uma alternativa; expressa comparação ou escolha feita após consideração cuidadosa, frequentemente usado para fazer sugestões | que expressa realização ou descoberta; indica que o resultado final foi inesperado}
  \definition{conj.}{ou (somente para frases interrogativas); indica várias opções, geralmente usado em perguntas | tudo; se; não importa; independentemente de; significa que, independentemente das mudanças que ocorram, o resultado permanecerá o mesmo}
\end{entry}

\begin{entry}{还有}{hai2 you3}{7,6}{⾡、⽉}[HSK 1]
  \definition{adv.}{também; ainda; além disso; então novamente; enfatizar as partes complementares, excedentes ou não mencionadas além do que já é conhecido}
\end{entry}

\begin{entry}{孩子}{hai2 zi5}{9,3}{⼦、⼦}[HSK 1]
  \definition[个]{s.}{criança; crianças; pessoas com idade entre alguns anos ou na adolescência, geralmente com menos de 14 anos | crianças; filho ou filha}
\end{entry}

\begin{entry}{海}{hai3}{10}{⽔}[HSK 2]
  \definition*{s.}{sobrenome Hai}
  \definition{adj.}{extragrande; de grande capacidade; descreve capacidade, tom de voz, etc.}
  \definition{adv.}{aleatoriamente; sem rumo; sem limites; sem restrições}
  \definition[片]{s.}{mar; grande lago; a parte do oceano próxima à costa, alguns grandes lagos também são chamados de mar | grande número de pessoas ou coisas reunidas; metáfora para muitas coisas semelhantes que formam um grande conjunto}
\end{entry}

\begin{entry}{海边}{hai3 bian1}{10,5}{⽔、⾡}[HSK 2]
  \definition{s.}{praia; costa; litoral; orla marítima; a parte marginal do oceano e as grandes áreas de água salgada cercadas por terra firme, onde a terra e a água se encontram, formam a costa}
\end{entry}

\begin{entry}{海底}{hai3di3}{10,8}{⽔、⼴}
  \definition{adj.}{submarino}
  \definition{s.}{fundo do mar | solo oceânico | fundo do oceano}
\end{entry}

\begin{entry}{海风}{hai3feng1}{10,4}{⽔、⾵}
  \definition{s.}{brisa do mar | vento que vem do mar}
\end{entry}

\begin{entry}{海关}{hai3guan1}{10,6}{⽔、⼋}[HSK 3]
  \definition{s.}{alfândega}
\end{entry}

\begin{entry}{海浪}{hai3lang4}{10,10}{⽔、⽔}
  \definition{s.}{ondas do mar}
\end{entry}

\begin{entry}{海里}{hai3li3}{10,7}{⽔、⾥}
  \definition{s.}{milha náutica}
\end{entry}

\begin{entry}{海绵}{hai3mian2}{10,11}{⽔、⽷}
  \definition{s.}{(zoologia) esponja do mar | esponja (feita de poliéster ou celulose, etc.) | espuma de borracha}
\end{entry}

\begin{entry}{海鸥}{hai3'ou1}{10,9}{⽔、⿃}
  \definition{s.}{gaivota}
\end{entry}

\begin{entry}{海水}{hai3 shui3}{10,4}{⽔、⽔}[HSK 4]
  \definition[把]{s.}{água do mar; salmoura}
\end{entry}

\begin{entry}{海棠}{hai3tang2}{10,12}{⽔、⽊}
  \definition{s.}{begônia}
\end{entry}

\begin{entry}{海鲜}{hai3xian1}{10,14}{⽔、⿂}[HSK 4]
  \definition[种,份,桌,批,些]{s.}{frutos do mar; mariscos; peixes marinhos frescos, camarões, etc., para consumo |}
\end{entry}

\begin{entry}{害}{hai4}{10}{⼧}[HSK 5]
  \definition{adj.}{prejudicial; destrutivo; injurioso; nocivo}
  \definition{s.}{mal; maldade; dano; calamidade}
  \definition{v.}{prejudicar; fazer mal a; causar problemas a | matar; assassinar | sofrer de; contrair (uma doença) | sentir-se (envergonhado, com medo, etc.); despertar (um sentimento ou uma emoção)}
\end{entry}

\begin{entry}{害怕}{hai4pa4}{10,8}{⼧、⼼}[HSK 3]
  \definition{v.}{estar assustado; ter medo}
\end{entry}

\begin{entry}{害羞}{hai4xiu1}{10,10}{⼧、⽺}
  \definition{adj.}{tímido | envergonhado}
\end{entry}

\begin{entry}{汗}{han2}{6}{⽔}
  \definition*{s.}{abreviação de Khan}[他是成吉思汗。(Ele é Genghis Khan.)]
  \seeref{汗}{han4}
\end{entry}

\begin{entry}{含}{han2}{7}{⼝}[HSK 4]
  \definition{v.}{manter na boca (sem engolir ou cuspir) | conter; incluir | cuidar; acalentar; abrigar}
\end{entry}

\begin{entry}{含金量}{han2jin1liang4}{7,8,12}{⼝、⾦、⾥}
  \definition{adj.}{conteúdo de ouro | (fig.) valioso}
\end{entry}

\begin{entry}{含量}{han2 liang4}{7,12}{⼝、⾥}[HSK 4]
  \definition{s.}{conteúdo; a quantidade de um componente contido em uma substância}
\end{entry}

\begin{entry}{含义}{han2yi4}{7,3}{⼝、⼂}[HSK 4]
  \definition[个,种,层]{s.}{sentido; mensagem; significado; implicação}
\end{entry}

\begin{entry}{含有}{han2 you3}{7,6}{⼝、⽉}[HSK 4]
  \definition{v.}{conter; ter; incluir}
\end{entry}

\begin{entry}{函数}{han2shu4}{8,13}{⼐、⽁}
  \definition{s.}{função (matemática)}
\end{entry}

\begin{entry}{涵}{han2}{11}{⽔}
  \definition{s.}{bueiro | galeria}
  \definition{v.}{conter | incluir | entupir}
\end{entry}

\begin{entry}{寒假}{han2jia4}{12,11}{⼧、⼈}[HSK 4]
  \definition[个]{s.}{férias de inverno (feriados); férias escolares no meio do inverno, em janeiro e fevereiro (na China)}
\end{entry}

\begin{entry}{寒冷}{han2 leng3}{12,7}{⼧、⼎}[HSK 4]
  \definition{adj.}{frio; frígido; gélido; gelado}
\end{entry}

\begin{entry}{韩国}{han2guo2}{12,8}{⾱、⼞}
  \definition*{s.}{Coréia do Sul}
\end{entry}

\begin{entry}{韩国人}{han2guo2ren2}{12,8,2}{⾱、⼞、⼈}
  \definition{s.}{coreano | pessoa ou povo da Coréia}
\end{entry}

\begin{entry}{厂}{han3}{2}{⼚}[Kangxi 27]
  \definition{s.}{radical ``penhasco'' em caracteres chineses (radical Kangxi 27)}
  \seeref{厂}{chang3}
\end{entry}

\begin{entry}{喊}{han3}{12}{⼝}[HSK 2]
  \definition{v.}{gritar; clamar; berrar | chamar (uma pessoa) | chamar; dirigir-se a}
\end{entry}

\begin{entry}{汉}{han4}{5}{⽔}
  \definition{s.}{grupo étnico Han | chinês (língua) | dinastia Han (206 a.C.-220d.C.) | homem}
\end{entry}

\begin{entry}{汉堡包}{han4bao3bao1}{5,12,5}{⽔、⼟、⼓}
  \definition[个]{s.}{hambúrguer}
\end{entry}

\begin{entry}{汉堡王}{han4bao3wang2}{5,12,4}{⽔、⼟、⽟}
  \definition*{s.}{Burguer King (restaurante \emph{fast-food})}
\end{entry}

\begin{entry}{汉服}{han4fu2}{5,8}{⽔、⽉}
  \definition{s.}{vestido chinês tradicional Han}
\end{entry}

\begin{entry}{汉葡词典}{han4-pu2 ci2dian3}{5,12,7,8}{⽔、⾋、⾔、⼋}
  \definition[部,本]{s.}{dicionário chinês-português}
  \seealsoref{葡汉词典}{pu2-han4 ci2dian3}
\end{entry}

\begin{entry}{汉语}{han4yu3}{5,9}{⽔、⾔}[HSK 1]
  \definition[门]{s.}{língua chinesa, mandarim}
\end{entry}

\begin{entry}{汉字}{han4 zi4}{5,6}{⽔、⼦}[HSK 1]
  \definition[个]{s.}{caractere chinês; ideograma chinês; sinograma; com pouquíssimas exceções, os caracteres chineses representam uma sílaba cada um}
\end{entry}

\begin{entry}{汗}{han4}{6}{⽔}[HSK 5]
  \definition{s.}{suor; transpiração; perspiração}
  \seeref{汗}{han2}
\end{entry}

\begin{entry}{汗水}{han4shui3}{6,4}{⽔、⽔}
  \definition*{s.}{Rio Han (Hanshui)}
  \definition{s.}{suor | transpiração}
\end{entry}

\begin{entry}{汗腺}{han4xian4}{6,13}{⽔、⾁}
  \definition{s.}{glândula sudorípara}
\end{entry}

\begin{entry}{汗液}{han4ye4}{6,11}{⽔、⽔}
  \definition{s.}{suor}
\end{entry}

\begin{entry}{焊}{han4}{11}{⽕}
  \definition{v.}{soldar}
\end{entry}

\begin{entry}{撼}{han4}{16}{⼿}
  \definition{v.}{sacudir | vibrar}
\end{entry}

\begin{entry}{行}{hang2}{6}{⾏}[HSK 3]
  \definition{adj.}{temporário; improvisado | capaz; competente}
  \definition{adv.}{logo; em breve}
  \definition{clas.}{linha; fileira; coisas usadas para formar filas, linhas}
  \definition{s.}{comportamento; conduta | linha; fileira | empresa comercial; certas instituições comerciais | comércio; profissão; ramo de atividade | especialista; conhecedor; refere-se ao conhecimento e experiência em um determinado setor}
  \definition{v.}{ir; caminhar; viajar | estar atualizado; circular | fazer; executar; realizar | (antes de um verbo dissílabo, indicando a realização de alguma ação) | ficar bem; vai dar certo | (remédio) fazer efeito | classificar (entre irmãos e irmãs por ordem de idade)}
  \seeref{行}{heng2}
  \seeref{行}{xing2}
\end{entry}

\begin{entry}{行业}{hang2ye4}{6,5}{⾏、⼀}[HSK 4]
  \definition[种,个]{s.}{comércio; indústria; setor; profissão; categorias em negócios e indústria referem-se a ocupações em geral}
\end{entry}

\begin{entry}{航班}{hang2ban1}{10,10}{⾈、⽟}[HSK 4]
  \definition[个,次]{s.}{número do voo; voo programado}
\end{entry}

\begin{entry}{航空}{hang2kong1}{10,8}{⾈、⽳}[HSK 4]
  \definition{s.}{viagem; aviação; refere-se ao voo de uma aeronave no ar}
\end{entry}

\begin{entry}{航天员}{hang2tian1yuan2}{10,4,7}{⾈、⼤、⼝}
  \definition{s.}{astronauta}
\end{entry}

\begin{entry}{号}{hao2}{5}{⼝}
  \definition[个]{s.}{rugido | choro}
  \definition{v.}{uivar; gritar; gritar em voz alta e prolongada |
lamentar; chorar alto |
uivar; (vento) assobiar, assoviar}
  \seeref{号}{hao4}
\end{entry}

\begin{entry}{蚝}{hao2}{10}{⾍}
  \definition{s.}{ostra}
\end{entry}

\begin{entry}{毫不费力}{hao2bu2fei4li4}{11,4,9,2}{⽊、⼀、⾙、⼒}
  \definition{adj.}{sem esforço | não gastando o menor esforço}
\end{entry}

\begin{entry}{毫米}{hao2mi3}{11,6}{⽊、⽶}[HSK 4]
  \definition{clas.}{milímetro; unidade legal de medida de comprimento, 1 mm equivale a 0,1 cm}
\end{entry}

\begin{entry}{毫升}{hao2 sheng1}{11,4}{⽊、⼗}[HSK 4]
  \definition{clas.}{mililitro; unidade de volume, milésimo de um litro (ml)}
\end{entry}

\begin{entry}{豪华}{hao2hua2}{14,6}{⾗、⼗}
  \definition{adj.}{luxuoso}
\end{entry}

\begin{entry}{好}{hao3}{6}{⼥}[HSK 1,2,4]
  \definition{adj.}{bom; ótimo; agradável; vantajoso; satisfatório | amigável; gentil; amistoso; amável | saudável; bem | pronto; concluído; usado após um verbo para indicar conclusão ou perfeição | fácil (de fazer); conveniente; responsável (por)}
  \definition{adv.}{muito; bastante; tão; usado na frente de uma palavra de quantidade ou uma palavra de tempo para indicar muito ou por muito tempo | em que medida; como; usado antes de adjetivos e verbos para indicar profundidade e com exclamação}
  \definition{interj.}{O.K.; tudo bem; aprovação, acordo ou encerramento | (no início de uma frase ou oração) expressa concordância (ou desaprovação, surpresa, etc.)}
  \definition{prep.}{de modo a; para que}
  \definition{s.}{referindo-se a palavras de elogio ou aplauso | saudações; cumprimentos}
  \definition{suf.}{sufixo que indica conclusão ou prontidão | depois de um pronome significa ``olá''}
  \definition{v.}{dever; precisar; ter que | apaixonar-se}
  \seeref{好}{hao4}
\end{entry}

\begin{entry}{好吃}{hao3chi1}{6,6}{⼥、⼝}[HSK 1]
  \definition{adj.}{bom; saboroso; delicioso; descreve o sabor agradável de algo, que as pessoas gostam de comer}
  \seeref{好吃}{hao4chi1}
\end{entry}

\begin{entry}{好处}{hao3chu4}{6,5}{⼥、⼡}[HSK 2]
  \definition[个]{s.}{bom; benefício; vantagem; fatores favoráveis a pessoas ou coisas | ganho; lucro; algo que não se deveria receber, dado por outra pessoa ou obtido através de uma oportunidade; geralmente tem conotação pejorativa}
\end{entry}

\begin{entry}{好多}{hao3 duo1}{6,6}{⼥、⼣}[HSK 2]
  \definition{adj.}{muitos; uma boa quantidade; uma grande quantidade; uma quantidade enorme}
  \definition{pron.}{quantos?; quanto?; frequentemente usado para perguntar sobre quantidade}
\end{entry}

\begin{entry}{好汉}{hao3han4}{6,5}{⼥、⽔}
  \definition[条]{s.}{herói | pessoa forte e corajosa}
\end{entry}

\begin{entry}{好好}{hao3 hao3}{6,6}{⼥、⼥}[HSK 3]
  \definition{adj.}{realmente bom/bem; em perfeitas condições; quando tudo está bem}
  \definition{adv.}{diretamente; seriamente; cuidadosamente}
\end{entry}

\begin{entry}{好久}{hao3jiu3}{6,3}{⼥、⼃}[HSK 2]
  \definition{adv.}{por muito tempo | por eras (no passado)}
\end{entry}

\begin{entry}{好看}{hao3 kan4}{6,9}{⼥、⽬}[HSK 1]
  \definition{adj.}{de boa aparência; agradável; bonito | interessante; descreve o enredo ou conteúdo de filmes, romances, performances, etc., como sendo cativante, agradável ou apreciável}
\end{entry}

\begin{entry}{好人}{hao3 ren2}{6,2}{⼥、⼈}[HSK 2]
  \definition[个,位,名]{s.}{pessoa boa (ou excelente) (oposto de 坏人) | pessoa saudável | pessoa gentil que tenta se dar bem com todos (muitas vezes em detrimento dos princípios)}
  \seealsoref{坏人}{huai4 ren2}
\end{entry}

\begin{entry}{好生}{hao3sheng1}{6,5}{⼥、⽣}
  \definition{adv.}{bastante; extremamente | cuidadosamente; apropriadamente}
\end{entry}

\begin{entry}{好事}{hao3 shi4}{6,8}{⼥、⼅}[HSK 2]
  \definition[个,件]{s.}{boa ação; gentileza | (antigo) obra de caridade | acontecimento feliz; evento festivo}
  \seeref{好事}{hao4 shi4}
\end{entry}

\begin{entry}{好听}{hao3 ting1}{6,7}{⼥、⼝}[HSK 1]
  \definition{adj.}{agradável de ouvir (de som ou voz) | bom; palatável; satisfatório (de palavras)  | decente; honrado (de ações, etc.); descreve uma coisa que parece prestigiosa | interessante; descreve palavras, histórias e outras coisas interessantes}
\end{entry}

\begin{entry}{好玩儿}{hao3 wan2r5}{6,8,2}{⼥、⽟、⼉}[HSK 1]
  \definition{adj.}{divertido; interessante; capaz de despertar interesse}
\end{entry}

\begin{entry}{好象}{hao3xiang4}{6,11}{⼥、⾗}
  \variantof{好像}
\end{entry}

\begin{entry}{好像}{hao3xiang4}{6,13}{⼥、⼈}[HSK 2]
  \definition{adv.}{como se; um pouco parecido; como se fosse}
  \definition{v.}{parecer; ser como; parecer-se com}
\end{entry}

\begin{entry}{好心}{hao3xin1}{6,4}{⼥、⼼}
  \definition{s.}{bondade | boas intenções}
\end{entry}

\begin{entry}{好学}{hao3xue2}{6,8}{⼥、⼦}
  \definition{adj.}{fácil de aprender}
  \seeref{好学}{hao4xue2}
\end{entry}

\begin{entry}{好用}{hao3yong4}{6,5}{⼥、⽤}
  \definition{adj.}{fácil de usar | adequado ao uso}
\end{entry}

\begin{entry}{好友}{hao3you3}{6,4}{⼥、⼜}[HSK 4]
  \definition[位,个]{s.}{bom amigo; amigo próximo}
\end{entry}

\begin{entry}{好运}{hao3 yun4}{6,7}{⼥、⾡}[HSK 5]
  \definition{s.}{boa sorte, fortuna ou oportunidade}
\end{entry}

\begin{entry}{号}{hao4}{5}{⼝}[HSK 1]
  \definition{clas.}{usado para o número de pessoas |  tipo; espécie; classificação | para pessoas ou negócios; número de vezes utilizado para transações}
  \definition{num.}{dia do mês | usado para indicar o número de pessoas}
  \definition[个]{s.}{nome | nome presumido; nome alternativo; pseudônimo; apelido | casa de negócios; loja | marca; sinal; sinalização | número | data | ordem; no exército, as ordens são transmitidas verbalmente ou por meio de clarins | qualquer instrumento de sopro e latão; trombeta usada no exército ou em bandas | qualquer coisa usada como buzina | chamada de corneta; qualquer chamada feita em uma corneta; usar um apito para emitir um som com um significado específico | pessoa em uma condição especial; pessoas que se encontram em uma situação especial}
  \definition{suf.}{sufixo de navio}
  \definition{v.}{marcar; fazer uma marca | sentir; colocar a mão no pulso do paciente e avaliar a situação através do fluxo sanguíneo}
  \seeref{号}{hao2}
\end{entry}

\begin{entry}{号角}{hao4jiao3}{5,7}{⼝、⾓}
  \definition{s.}{corneta | trombeta}
\end{entry}

\begin{entry}{号码}{hao4ma3}{5,8}{⼝、⽯}[HSK 4]
  \definition[个,组,串]{s.}{número}
\end{entry}

\begin{entry}{号召}{hao4zhao4}{5,5}{⼝、⼝}[HSK 5]
  \definition{s.}{chamado; apelo; desejo ou pedido solene (de um governo, partido político, organização etc.) para que as massas façam algo}
  \definition{v.}{chamar;  (governo, partido político, organização, etc.) fazer um pedido solene às massas para que façam algo, na esperança de que todos se esforcem para alcançá-lo}
\end{entry}

\begin{entry}{好}{hao4}{6}{⼥}
  \definition*{s.}{sobrenome Hao}
  \definition{adv.}{algo que acontece com frequência, que é fácil de acontecer}
  \definition{v.}{gostar; amar; ter afeição por}
  \seeref{好}{hao3}
\end{entry}

\begin{entry}{好吃}{hao4chi1}{6,6}{⼥、⼝}
  \definition{v.}{ser guloso; gostar de comer boa comida}
  \seeref{好吃}{hao3chi1}
\end{entry}

\begin{entry}{好奇}{hao4qi2}{6,8}{⼥、⼤}[HSK 3]
  \definition{adj.}{curioso}
  \definition{s.}{curiosidade}
  \definition{v.}{ser ou estar curioso}
\end{entry}

\begin{entry}{好事}{hao4 shi4}{6,8}{⼥、⼅}
  \definition[个,件]{s.}{intrometido; gostar de se meter na vida dos outros}
  \seeref{好事}{hao3 shi4}
\end{entry}

\begin{entry}{好学}{hao4xue2}{6,8}{⼥、⼦}
  \definition{s.}{estudioso | erudito}
  \seeref{好学}{hao3xue2}
\end{entry}

\begin{entry}{呵}{he1}{8}{⼝}
  \definition{expr.}{Meu Deus! | expelir a respiração}
  \seeref{呵}{a1}
\end{entry}

\begin{entry}{欱}{he1}{10}{⽋}
  \variantof{喝}
\end{entry}

\begin{entry}{喝}{he1}{12}{⼝}[HSK 1]
  \definition{interj.}{Meu Deus!; Oh!; Ah!; Uau!}
  \definition{s.}{bebida; especificamente, vinho}
  \definition{v.}{beber; engolir líquidos ou alimentos líquidos | beber bebida alcoólica; referência específica ao consumo de álcool}
  \seeref{喝}{he4}
\end{entry}

\begin{entry}{喝醉}{he1zui4}{12,15}{⼝、⾣}
  \definition{v.}{ficar bêbado}
\end{entry}

\begin{entry}{合}{he2}{6}{⼝}[HSK 3]
  \definition*{s.}{sobrenome He}
  \definition{adj.}{todo; completo; inteiro}
  \definition{clas.}{para rodadas}
  \definition{s.}{conjunção}
  \definition{v.}{fechar | juntar; combinar | adequar-se; concordar; conformar-se a | ser igual a; somar}
\end{entry}

\begin{entry}{合并}{he2bing4}{6,6}{⼝、⼲}[HSK 5]
  \definition{v.}{fundir; amalgamar; combinar várias coisas em uma coisa só | (doença) ser complicada por outra doença; uma doença levar a outra, ataques simultâneos (de várias doenças)}
\end{entry}

\begin{entry}{合成}{he2cheng2}{6,6}{⼝、⼽}[HSK 5]
  \definition{s.}{compor; integrar; combinar; misturar | sintetizar, reação química para transformar uma substância com uma composição simples em uma substância com uma composição complexa}
\end{entry}

\begin{entry}{合法}{he2fa3}{6,8}{⼝、⽔}[HSK 3]
  \definition{adj.}{legal; legítimo; correto}
\end{entry}

\begin{entry}{合格}{he2ge2}{6,10}{⼝、⽊}[HSK 3]
  \definition{adj.}{qualificado; de acordo com o padrão}
\end{entry}

\begin{entry}{合理}{he2li3}{6,11}{⼝、⽟}[HSK 3]
  \definition{adj.}{racional; razoável; equitativo}
\end{entry}

\begin{entry}{合适}{he2shi4}{6,9}{⼝、⾡}[HSK 2]
  \definition{adj.}{correto; adequado; apropriado; conveniente; em conformidade com a realidade ou com os requisitos objetivos}
\end{entry}

\begin{entry}{合同}{he2tong5}{6,6}{⼝、⼝}[HSK 4]
  \definition[个,份]{s.}{contrato; acordo; uma disposição para observância mútua por duas ou mais partes na condução de um assunto com o objetivo de determinar seus respectivos direitos e obrigações.}
\end{entry}

\begin{entry}{合宪性}{he2xian4xing4}{6,9,8}{⼝、⼧、⼼}
  \definition{s.}{constitucionalismo}
\end{entry}

\begin{entry}{合资}{he2zi1}{6,10}{⼝、⾙}
  \definition{s.}{\emph{joint-venture} com capitais mistos}
\end{entry}

\begin{entry}{合作}{he2zuo4}{6,7}{⼝、⼈}[HSK 3]
  \definition[个]{s.}{cooperação; colaboração}
  \definition{v.}{cooperar; colaborar; trabalhar em conjunto}
\end{entry}

\begin{entry}{何}{he2}{7}{⼈}
  \definition*{s.}{sobrenome He}
  \definition{adv.}{expressa exclamação, equivalente a 多么}
  \definition{pron.}{O que?; Onde?; Por que? | expressa uma pergunta retórica, equivalente a 岂, 怎}
  \seealsoref{多么}{duo1me5}
  \seealsoref{岂}{qi3}
  \seealsoref{怎}{zen3}
\end{entry}

\begin{entry}{何不}{he2bu4}{7,4}{⼈、⼀}
  \definition{adv.}{por que não?; use o tom interrogativo para expressar "deveria" ou "pode"}
\end{entry}

\begin{entry}{何况}{he2kuang4}{7,7}{⼈、⼎}
  \definition{conj.}{além disso | muito menos}
\end{entry}

\begin{entry}{和}{he2}{8}{⼝}[HSK 1]
  \definition*{s.}{sobrenome He}
  \definition{adj.}{gentil; suave; amável | harmonioso; em boas condições}
  \definition{conj.}{e (somente para palavras) | junto com}
  \definition{prep.}{relacionado com | para; com; indica relação; comparação, etc.}
  \definition{s.}{soma; soma total | japonês; refere-se ao Japão}
  \definition{v.}{disputar; reconciliar; acabar com a guerra ou a disputa | empatar; (próxima edição ou torneio) sem vencedor}
  \seeref{和}{he4}
  \seeref{和}{hu2}
  \seeref{和}{huo2}
  \seeref{和}{huo4}
\end{entry}

\begin{entry}{和平}{he2ping2}{8,5}{⼝、⼲}[HSK 3]
  \definition{adj.}{pacífico; não violento}
  \definition{s.}{paz}
\end{entry}

\begin{entry}{和平共处}{he2ping2gong4chu3}{8,5,6,5}{⼝、⼲、⼋、⼡}
  \definition{s.}{coexistência pacífica de nações, sociedades, etc.}
\end{entry}

\begin{entry}{和谐}{he2xie2}{8,11}{⼝、⾔}
  \definition{adj.}{harmonioso}
  \definition{s.}{harmonia}
  \definition{v.}{(eufemismo) censurar}
\end{entry}

\begin{entry}{河}{he2}{8}{⽔}[HSK 2]
  \definition*{s.}{o sistema da Via Láctea | o rio Amarelo; o rio Huanghe}
  \definition*{s.}{sobrenome He}
  \definition[条,道]{s.}{rio; refere-se a grandes cursos de água}
\end{entry}

\begin{entry}{河蚌}{he2bang4}{8,10}{⽔、⾍}
  \definition{s.}{mexilhões | bivalves cultivados em rios e lagos}
\end{entry}

\begin{entry}{核}{he2}{10}{⽊}
  \definition{adj.}{nuclear}
  \definition{s.}{poço | pedra | núcleo}
  \definition{v.}{examinar | checar | verificar}
\end{entry}

\begin{entry}{荷}{he2}{10}{⾋}
  \definition{s.}{lótus}
  \seeref{荷}{he4}
\end{entry}

\begin{entry}{荷花}{he2hua1}{10,7}{⾋、⾋}
  \definition{s.}{lótus}
\end{entry}

\begin{entry}{盒}{he2}{11}{⽫}[HSK 5]
  \definition{clas.}{caixa (de pequena dimensão)}
  \definition{s.}{caixa; estojo; recipiente; receptáculo}
\end{entry}

\begin{entry}{盒饭}{he2 fan4}{11,7}{⽫、⾷}[HSK 5]
  \definition{s.}{refeição embalada; marmita; \emph{fast-food} vendida em caixas}
\end{entry}

\begin{entry}{盒子}{he2zi5}{11,3}{⽫、⼦}[HSK 5]
  \definition[个]{s.}{caixa; recipiente que têm tampas na parte superior e podem conter coisas dentro, geralmente é pequeno e plano}
\end{entry}

\begin{entry}{和}{he4}{8}{⼝}
  \definition{v.}{compor um poema em resposta (ao poema de alguém) usando a mesma sequência de rimas | juntar-se à cantoria | cantar junto com outros}
  \seeref{和}{he2}
  \seeref{和}{hu2}
  \seeref{和}{huo2}
  \seeref{和}{huo4}
\end{entry}

\begin{entry}{贺}{he4}{9}{⾙}
  \definition*{s.}{sobrenome He}
  \definition{v.}{parabenizar | congratular}
\end{entry}

\begin{entry}{贺卡}{he4 ka3}{9,5}{⾙、⼘}[HSK 5]
  \definition[张]{s.}{cartão de felicitações; pedaço de papel para parabenizar amigos e parentes em seu casamento, aniversário ou festivais, geralmente impresso com palavras e desenhos de felicitações}
\end{entry}

\begin{entry}{荷}{he4}{10}{⾋}
  \definition{s.}{carga | responsabilidade}
  \definition{v.}{carregar no ombro ou nas costas}
  \seeref{荷}{he2}
\end{entry}

\begin{entry}{喝}{he4}{12}{⼝}
  \definition{v.}{gritar bem alto}
  \seeref{喝}{he1}
\end{entry}

\begin{entry}{喝彩}{he4cai3}{12,11}{⼝、⼺}
  \definition{s.}{aclamar | torcer}
\end{entry}

\begin{entry}{褐色}{he4 se4}{14,6}{⾐、⾊}
  \definition{s.}{cor marrom}
\end{entry}

\begin{entry}{鹤}{he4}{15}{⿃}
  \definition{s.}{grou (ave)}
\end{entry}

\begin{entry}{黑}{hei1}{12}{⿊}[HSK 2][Kangxi 203]
  \definition*{s.}{sobrenome Hei}
  \definition*{s.}{abreviação de Província de Heilongjiang, 黑龙江}
  \definition{adj.}{preto; cor semelhante à do carvão | escuro | obscuro; secreto | perverso; sinistro; ruim; cruel | reacionário}
  \definition{s.}{noite}
  \definition{v.}{fazer algo ilegalmente ou de forma desonesta; enganar; desviar dinheiro ilegalmente | invadir (uma rede, sites, computador, etc.)}
  \seealsoref{黑龙江}{hei1long2jiang1}
\end{entry}

\begin{entry}{黑暗}{hei1 an4}{12,13}{⿊、⽇}[HSK 4]
  \definition{adj.}{escuro; sombrio; sem luz | maligno; corrupto; reacionário}
\end{entry}

\begin{entry}{黑板}{hei1ban3}{12,8}{⿊、⽊}[HSK 2]
  \definition[块,个]{s.}{quadro negro; quadro de giz; uma placa, na qual se pode escrever com giz}
\end{entry}

\begin{entry}{黑客}{hei1ke4}{12,9}{⿊、⼧}
  \definition{s.}{(empréstimo linguístico) (computação) \emph{hacker}}
\end{entry}

\begin{entry}{黑龙江}{hei1long2jiang1}{12,5,6}{⿊、⿓、⽔}
  \definition*{s.}{Heilongjiang (Província)}
  \definition*{s.}{Rio Heilong Jiang; (na Rússia) o rio Amur}
\end{entry}

\begin{entry}{黑色}{hei1 se4}{12,6}{⿊、⾊}[HSK 2]
  \definition{adj.}{metafórico: suspeito, ilegal}
  \definition{s.}{cor preta}
\end{entry}

\begin{entry}{很}{hen3}{9}{⼻}[HSK 1]
  \definition{adv.}{muito; bastante; terrivelmente; indica um grau bastante elevado; definitivo; o mais alto}
\end{entry}

\begin{entry}{恨}{hen4}{9}{⼼}[HSK 5]
  \definition{s.}{ódio; resentimento}
  \definition{v.}{odiar}
\end{entry}

\begin{entry}{行}{heng2}{6}{⾏}
  \definition{s.}{usado em 道行}
  \seeref{行}{hang2}
  \seeref{行}{xing2}
  \seealsoref{道行}{dao4 heng2}
\end{entry}

\begin{entry}{恒星系}{heng2xing1xi4}{9,9,7}{⼼、⽇、⽷}
  \definition{s.}{sistema estelar | galáxia}
\end{entry}

\begin{entry}{横竖}{heng2shu5}{15,9}{⽊、⽴}
  \definition{adv.}{de qualquer maneira | independentemente (linguagem falada)}
\end{entry}

\begin{entry}{轰鸣}{hong1ming2}{8,8}{⾞、⿃}
  \definition{s.}{bum (som de explosão) | estrondo}
\end{entry}

\begin{entry}{轰炸机}{hong1zha4ji1}{8,9,6}{⾞、⽕、⽊}
  \definition{s.}{bombardeiro (aeronave)}
\end{entry}

\begin{entry}{哄}{hong1}{9}{⼝}
  \definition{s.}{gargalhadas | risadas ruidosas | algazarra | rugido | clamor}
  \seeref{哄}{hong3}
  \seeref{哄}{hong4}
\end{entry}

\begin{entry}{红}{hong2}{6}{⽷}[HSK 2]
  \definition*{s.}{sobrenome Hong}
  \definition{adj.}{vermelho | popular; bem-sucedido; símbolo de sucesso ou valorização | vermelho; revolucionário; símbolo da revolução | festivo; símbolo de alegria}
  \definition{s.}{tecido vermelho, bandeirinhas, etc. usados em ocasiões festivas | bônus; dividendo}
\end{entry}

\begin{entry}{红包}{hong2 bao1}{6,5}{⽷、⼓}[HSK 4]
  \definition[个]{s.}{saco de papel vermelho ou envelope contendo dinheiro como presente, gorjeta ou bônus | suborno; propina}
\end{entry}

\begin{entry}{红宝石}{hong2bao3shi2}{6,8,5}{⽷、⼧、⽯}
  \definition{s.}{rubi}
\end{entry}

\begin{entry}{红茶}{hong2 cha2}{6,9}{⽷、⾋}[HSK 3]
  \definition[杯,壶,斤,种]{s.}{chá preto}
\end{entry}

\begin{entry}{红酒}{hong2 jiu3}{6,10}{⽷、⾣}[HSK 3]
  \definition{s.}{vinho tinto}
\end{entry}

\begin{entry}{红绿灯}{hong2lv4deng1}{6,11,6}{⽷、⽷、⽕}
  \definition[个]{s.}{semáforo | sinal de trânsito}
\end{entry}

\begin{entry}{红色}{hong2 se4}{6,6}{⽷、⾊}[HSK 2]
  \definition{adj.}{vermelho; revolucionário; símbolo da revolução ou da consciência política elevada}
  \definition{s.}{cor vermelha}
\end{entry}

\begin{entry}{红烧}{hong2shao1}{6,10}{⽷、⽕}
  \definition{s.}{guisado em molho de soja (prato)}
\end{entry}

\begin{entry}{红薯}{hong2shu3}{6,16}{⽷、⾋}
  \definition{s.}{batata doce}
\end{entry}

\begin{entry}{红线}{hong2xian4}{6,8}{⽷、⽷}
  \definition{s.}{linha vermelha}
\end{entry}

\begin{entry}{洪水}{hong2shui3}{9,4}{⽔、⽔}
  \definition{s.}{enchente | inundação | dilúvio}
\end{entry}

\begin{entry}{哄}{hong3}{9}{⼝}
  \definition{v.}{enganar | persuadir | divertir (uma criança)}
  \seeref{哄}{hong1}
  \seeref{哄}{hong4}
\end{entry}

\begin{entry}{哄}{hong4}{9}{⼝}
  \definition{s.}{tumulto | agitação | perturbação}
  \seeref{哄}{hong1}
  \seeref{哄}{hong3}
\end{entry}

\begin{entry}{猴}{hou2}{12}{⽝}[HSK 5]
  \definition{adj.}{esperto; inteligente; perspicaz}
  \definition[只,群]{s.}{macaco}
\end{entry}

\begin{entry}{猴子}{hou2zi5}{12,3}{⽝、⼦}
  \definition[只]{s.}{macaco}
\end{entry}

\begin{entry}{后}{hou4}{6}{⼝}[HSK 1]
  \definition*{s.}{sobrenome Hou}
  \definition{s.}{atrás; traseiro; a direção oposta àquela para a qual a pessoa está voltada; a direção oposta àquela para a qual a parte de trás de uma casa está voltada (o oposto de 前)  | depois; mais tarde no tempo; futuro (em oposição a 先 ou 前) | último | posteridade; descendência | rainha; imperatriz | governante; soberano; monarca antigo}
  \seealsoref{前}{qian2}
  \seealsoref{先}{xian1}
\end{entry}

\begin{entry}{后边}{hou4 bian5}{6,5}{⼝、⾡}[HSK 1]
  \definition{adv.}{costas; traseira; atrás}
\end{entry}

\begin{entry}{后果}{hou4guo3}{6,8}{⼝、⽊}[HSK 3]
  \definition{s.}{consequência; resultado}
\end{entry}

\begin{entry}{后悔}{hou4hui3}{6,10}{⼝、⼼}[HSK 5]
  \definition{v.}{lamentar; ter remorso; arrepender-se}
\end{entry}

\begin{entry}{后来}{hou4lai2}{6,7}{⼝、⽊}[HSK 2]
  \definition{adv.}{mais tarde; depois; refere-se a um período posterior a um determinado momento no passado}
\end{entry}

\begin{entry}{后面}{hou4mian4}{6,9}{⼝、⾯}[HSK 3]
  \definition{adv.}{parte de trás; retaguarda; atrás | atrás; perto do fim; na parte de trás | mais tarde; depois}
  \seeref{后面}{hou4mian5}
\end{entry}

\begin{entry}{后面}{hou4mian5}{6,9}{⼝、⾯}[HSK 3]
  \definition{adv.}{parte de trás; retaguarda; atrás | atrás; perto do fim; na parte de trás | mais tarde; depois}
  \seeref{后面}{hou4mian4}
\end{entry}

\begin{entry}{后年}{hou4nian2}{6,6}{⼝、⼲}[HSK 3]
  \definition{s.}{o ano que vem; daqui a dois anos}
\end{entry}

\begin{entry}{后天}{hou4 tian1}{6,4}{⼝、⼤}[HSK 1]
  \definition{s.}{depois de amanhã; período em que uma pessoa ou animal vive e cresce sozinho após deixar o útero materno (em oposição a 先天)}
  \seealsoref{先天}{xian1tian1}
\end{entry}

\begin{entry}{后头}{hou4 tou5}{6,5}{⼝、⼤}[HSK 4]
  \definition{adv.}{posteriormente | atrás | mais tarde}
  \definition{s.}{a parte de trás | a parte traseira}
\end{entry}

\begin{entry}{厚}{hou4}{9}{⼚}[HSK 4]
  \definition*{s.}{sobrenome Hou}
  \definition{adj.}{grosso; espesso | profundo | bondoso; gentil; magnânimo | grande; generoso | rico ou forte em sabor}
  \definition{s.}{espessura; profundidade}
  \definition{v.}{favorecer; enfatizar}
\end{entry}

\begin{entry}{呼吸}{hu1xi1}{8,6}{⼝、⼝}[HSK 4]
  \definition{s.}{um suspiro; metáfora para um período de tempo muito curto}
  \definition{v.}{respirar}
\end{entry}

\begin{entry}{呼啸}{hu1xiao4}{8,11}{⼝、⼝}
  \definition{v.}{assobiar}
\end{entry}

\begin{entry}{忽然}{hu1ran2}{8,12}{⼼、⽕}[HSK 2]
  \definition{adv.}{repentinamente; de repente; sem aviso prévio; significa que algo aconteceu de forma rápida e inesperada}
\end{entry}

\begin{entry}{忽视}{hu1shi4}{8,8}{⼼、⾒}[HSK 4]
  \definition{v.}{ignorar; negligenciar; menosprezar; desprezar; dar de ombros}
\end{entry}

\begin{entry}{和}{hu2}{8}{⼝}
  \definition{v.}{completar um conjunto de Mahjong ou cartas de baralho}
  \seeref{和}{he2}
  \seeref{和}{he4}
  \seeref{和}{huo2}
  \seeref{和}{huo4}
\end{entry}

\begin{entry}{胡萝卜}{hu2luo2bo5}{9,11,2}{⾁、⾋、⼘}
  \definition{s.}{cenoura}
\end{entry}

\begin{entry}{胡同儿}{hu2 tong4r5}{9,6,2}{⾁、⼝、⼉}[HSK 5]
  \definition{s.}{beco; via; rua}
\end{entry}

\begin{entry}{胡子}{hu2 zi5}{9,3}{⾁、⼦}[HSK 5]
  \definition[团,根,个]{s.}{barba; bigode | bandido; salteador}
\end{entry}

\begin{entry}{湖}{hu2}{12}{⽔}[HSK 2]
  \definition*{s.}{abreviação de Huzhou, 湖州 | um nome que se refere às províncias de Hunan, 湖南,  e Hubei, 湖北}
  \definition[个,片]{s.}{lago}
  \seealsoref{湖北}{hu2bei3}
  \seealsoref{湖南}{hu2nan2}
  \seealsoref{湖州}{hu2zhou1}
\end{entry}

\begin{entry}{湖北}{hu2bei3}{12,5}{⽔、⼔}
  \definition*{s.}{Província de Hubei (Hupeh), na China central}
\end{entry}

\begin{entry}{湖南}{hu2nan2}{12,9}{⽔、⼗}
  \definition*{s.}{Província de Hunan (província do sul da China)}
\end{entry}

\begin{entry}{湖州}{hu2zhou1}{12,6}{⽔、⼮}
  \definition*{s.}{Cidade de Huzhou, em Zhejiang}
\end{entry}

\begin{entry}{葫芦}{hu2lu5}{12,7}{⾋、⾋}
  \definition{adj.}{confuso}
  \definition{s.}{cabaça | termo genérico para bloco e equipamento (ou partes dele)}
\end{entry}

\begin{entry}{糊里糊涂}{hu2li5hu2tu5}{15,7,15,10}{⽶、⾥、⽶、⽔}
  \definition{adj.}{desnorteado | perturbado}
\end{entry}

\begin{entry}{蝴蝶}{hu2die2}{15,15}{⾍、⾍}
  \definition[只]{s.}{borboleta}
\end{entry}

\begin{entry}{虎}{hu3}{8}{⾌}[HSK 5]
  \definition*{s.}{sobrenome Hu}
  \definition{adj.}{corajoso; bravo; valente; vigoroso}
  \definition{s.}{tigre}
  \definition{v.}{blefar; o mesmo que 唬 | parecer feroz; mostrar a aparência feroz de alguém}
  \seealsoref{唬}{hu3}
  \seealsoref{老虎}{lao3hu3}
\end{entry}

\begin{entry}{虎虎}{hu3hu3}{8,8}{⾌、⾌}
  \definition{adj.}{formidável | forte | vigoroso}
\end{entry}

\begin{entry}{虎口}{hu3kou3}{8,3}{⾌、⼝}
  \definition{s.}{lugar perigoso | cova do tigre}
\end{entry}

\begin{entry}{虎鼬}{hu3you4}{8,18}{⾌、⿏}
  \definition{s.}{doninha}
\end{entry}

\begin{entry}{唬}{hu3}{11}{⼝}
  \definition{v.}{blefar, exagerar para assustar ou confundir}
\end{entry}

\begin{entry}{互}{hu4}{4}{⼆}
  \definition{adj.}{mútuo | recíproco}
\end{entry}

\begin{entry}{互动}{hu4dong4}{4,6}{⼆、⼒}
  \definition{s.}{interativo}
  \definition{v.}{interagir}
\end{entry}

\begin{entry}{互利}{hu4li4}{4,7}{⼆、⼑}
  \definition{s.}{benefício mútuo}
\end{entry}

\begin{entry}{互联网}{hu4lian2wang3}{4,12,6}{⼆、⽿、⽹}[HSK 3]
  \definition{s.}{\emph{Internet}}
  \seealsoref{网际网路}{wang3ji4wang3lu4}
  \seealsoref{网际网络}{wang3ji4wang3luo4}
  \seealsoref{网路}{wang3lu4}
\end{entry}

\begin{entry}{互相}{hu4xiang1}{4,9}{⼆、⽬}[HSK 3]
  \definition{adv.}{mutuamente; um ao outro}
\end{entry}

\begin{entry}{户}{hu4}{4}{⼾}[HSK 4][Kangxi 63]
  \definition*{s.}{sobrenome Hu}
  \definition[个]{s.}{porta com um painel; porta | domicílio; residência; família | status familiar | conta (banco)}
\end{entry}

\begin{entry}{护士}{hu4shi5}{7,3}{⼿、⼠}[HSK 4]
  \definition[名,位]{s.}{enfermeiro; pessoas especializadas em enfermagem em hospitais ou instituições epidemiológicas}
\end{entry}

\begin{entry}{护照}{hu4zhao4}{7,13}{⼿、⽕}[HSK 2]
  \definition[本,个]{s.}{passaporte; documento emitido pela autoridade competente do país para comprovar a nacionalidade e a identidade dos cidadãos que viajam para o exterior}
\end{entry}

\begin{entry}{化}{hua1}{4}{⼔}[HSK 3]
  \definition*{s.}{sobrenome Hua}
  \definition{s.}{químico}
  \definition{suf.}{modernizar; modernização}
  \definition{v.}{mudar; converter; transformar | converter; influenciar | derreter; dissolver | digerir | queimar | morrer | pedir esmola}
  \variantof{花}
\end{entry}

\begin{entry}{花}{hua1}{7}{⾋}[HSK 1,2,4]
  \definition*{s.}{sobrenome Hua}
  \definition{adj.}{multicolorido; colorido | embaçado; obscuro; deslumbrado e confuso | extravagante; florido; vistoso}
  \definition[朵,支,束,把,盆,簇]{s.}{flor; órgãos de reprodução sexual de plantas com sementes | flor; planta ornamental |  qualquer coisa que se assemelhe a uma flor | fogos de artifício | padrão; design; design decorativo | flor; metáfora para a essência de uma causa | prostituta; cortesã; referindo-se a prostitutas ou a assuntos relacionados a prostitutas | algodão | varíola | ferimento; ferida; lesões traumáticas sofridas em combate}
  \definition{v.}{gastar; despender; consumir}
\end{entry}

\begin{entry}{花茶}{hua1cha2}{7,9}{⾋、⾋}
  \definition[杯,壶]{s.}{chá perfumado}
\end{entry}

\begin{entry}{花店}{hua1dian4}{7,8}{⾋、⼴}
  \definition{s.}{floricultura}
\end{entry}

\begin{entry}{花儿}{hua1r5}{7,2}{⾋、⼉}
  \definition[朵,支,束,把,盆,簇]{s.}{flor}
\end{entry}

\begin{entry}{花生}{hua1sheng1}{7,5}{⾋、⽣}
  \definition[粒]{s.}{amendoim}
\end{entry}

\begin{entry}{花样游泳}{hua1yang4you2yong3}{7,10,12,8}{⾋、⽊、⽔、⽔}
  \definition{s.}{nado sincronizado}
\end{entry}

\begin{entry}{花椰菜}{hua1ye1cai4}{7,12,11}{⾋、⽊、⾋}
  \definition{s.}{couve-flor}
\end{entry}

\begin{entry}{花园}{hua1 yuan2}{7,7}{⾋、⼞}[HSK 2]
  \definition[个,座]{s.}{jardim; um local onde se plantam flores e árvores para passear e descansar}
\end{entry}

\begin{entry}{划}{hua2}{6}{⼑}[HSK 4]
  \definition{adj.}{rentável; vale (o esforço); compensa (fazer alguma coisa)}
  \definition{v.}{remar | ser vantajoso para alguém; ser uma pechincha | arranhar; cortar a superfície de; cortar em outra coisa com um objeto pontiagudo | arranhar; golpear;  esfregar uma coisa ou varrer sobre outra}
  \seeref{划}{hua4}
\end{entry}

\begin{entry}{划船}{hua2 chuan2}{6,11}{⼑、⾈}[HSK 3]
  \definition[次,回]{s.}{remo (ato de remar); passeios de barco}
  \definition{v.}{remar um barco}
\end{entry}

\begin{entry}{划艇}{hua2ting3}{6,12}{⼑、⾈}
  \definition{s.}{barco a remo}
\end{entry}

\begin{entry}{华人}{hua2 ren2}{6,2}{⼗、⼈}[HSK 3]
  \definition{s.}{Chinês; chinês étnico | cidadãos estrangeiros de ascendência chinesa que adquiriram nacionalidade no seu país de residência}
\end{entry}

\begin{entry}{华盛顿}{hua2sheng4dun4}{6,11,10}{⼗、⽫、⾴}
  \definition*{s.}{Washington}
\end{entry}

\begin{entry}{华氏}{hua2shi4}{6,4}{⼗、⽒}
  \definition{s.}{graus Fahrenheit (°F)}
\end{entry}

\begin{entry}{华夏}{hua2xia4}{6,10}{⼗、⼢}
  \definition*{s.}{Huaxia, nome antigo da China | Catai}
\end{entry}

\begin{entry}{华裔}{hua2yi4}{6,13}{⼗、⾐}
  \definition{s.}{descendente de chinês}
\end{entry}

\begin{entry}{华语}{hua2 yu3}{6,9}{⼗、⾔}[HSK 5]
  \definition*{s.}{Chinês (idioma)}
\end{entry}

\begin{entry}{滑}{hua2}{12}{⽔}[HSK 5]
  \definition*{s.}{sobrenome Hua}
  \definition{adj.}{escorregadio; liso; objetos com superfícies lisas e baixo atrito | astuto; ardiloso; escorregadio}
  \definition{v.}{escorregar; deslizar | se atrapalhar; se safar de algo}
\end{entry}

\begin{entry}{滑雪}{hua2xue3}{12,11}{⽔、⾬}
  \definition{v.+compl.}{esquiar | praticar esqui}
\end{entry}

\begin{entry}{化石}{hua4shi2}{4,5}{⼔、⽯}[HSK 5]
  \definition{s.}{fóssil; restos, relíquias ou vestígios de organismos antigos enterrados no solo e transformados em objetos semelhantes a pedras}
\end{entry}

\begin{entry}{化学}{hua4xue2}{4,8}{⼔、⼦}
  \definition{s.}{química (disciplina)}
\end{entry}

\begin{entry}{划}{hua4}{6}{⼑}[HSK 4]
  \definition{s.}{traço de um caracter chinês}
  \definition{v.}{delimitar; diferenciar; delinear | transferir; ceder | planejar; programar | desenhar; marcar; delinear; fazer linhas ou escrever como marcadores com uma caneta ou objeto semelhante a uma caneta}
  \seeref{划}{hua2}
\end{entry}

\begin{entry}{划分}{hua4fen1}{6,4}{⼑、⼑}[HSK 5]
  \definition{v.}{dividir; particionar; reparticionar | diferenciar; encontrar aspectos diferentes}
\end{entry}

\begin{entry}{画}{hua4}{8}{⽥}[HSK 2]
  \definition*{s.}{sobrenome Hua}
  \definition{clas.}{traços (de um caractere chinês)}
  \definition[张,幅]{s.}{desenho; pintura; imagem; figura desenhada | traço horizontal (em caracteres chineses)}
  \definition{v.}{desenhar; pintar | desenhar; marcar; assinar}
  \seealsoref{划}{hua4}
\end{entry}

\begin{entry}{画地为牢}{hua4di4wei2lao2}{8,6,4,7}{⽥、⼟、⼂、⼧}
  \definition{expr.}{(literalmente) ser confinado dentro de um círculo desenhado no chão | (figurativo) limitar-se a uma gama restrita de atividades}
\end{entry}

\begin{entry}{画家}{hua4 jia1}{8,10}{⽥、⼧}[HSK 2]
  \definition[个,位,名,些]{s.}{pintor; pessoa com talento para pintura}
\end{entry}

\begin{entry}{画面}{hua4 mian4}{8,9}{⽥、⾯}[HSK 5]
  \definition[个,幅]{s.}{quadro; aparência geral de uma imagem; imagem apresentada no quadro, na tela, etc.}
\end{entry}

\begin{entry}{画儿}{hua4r5}{8,2}{⽥、⼉}[HSK 2]
  \definition[幅,张]{s.}{imagem; desenho; pintura; obra de arte pintada}
\end{entry}

\begin{entry}{话}{hua4}{8}{⾔}[HSK 1]
  \definition[种,席,句,口,番]{s.}{palavra; conversa; a voz que expressa os pensamentos quando falada, ou os caracteres que registram essa voz}
  \definition{v.}{falar sobre; falar a respeito}
\end{entry}

\begin{entry}{话剧}{hua4 ju4}{8,10}{⾔、⼑}[HSK 3]
  \definition[台,部]{s.}{drama moderno; peça de teatro}
\end{entry}

\begin{entry}{话题}{hua4ti2}{8,15}{⾔、⾴}[HSK 3]
  \definition[个,种,项]{s.}{assunto de uma palestra; tópico de uma conversa}
\end{entry}

\begin{entry}{怀旧}{huai2jiu4}{7,5}{⼼、⽇}
  \definition{s.}{nostalgia}
  \definition{v.}{sentir-se nostálgico}
\end{entry}

\begin{entry}{怀念}{huai2nian4}{7,8}{⼼、⼼}[HSK 4]
  \definition{v.}{pensar em; valorizar a memória de}
\end{entry}

\begin{entry}{怀疑}{huai2yi2}{7,14}{⼼、⽦}[HSK 4]
  \definition{v.}{duvidar; suspeitar | supor}
\end{entry}

\begin{entry}{坏}{huai4}{7}{⼟}[HSK 1]
  \definition{adj.}{ruim; prejudicial; insatisfatório; péssimo | mal; extremamente; indica um grau profundo, geralmente usado após verbos ou adjetivos que expressam estado psicológico | podre; estragado; impróprio; prejudicial ao uso}
  \definition[种]{s.}{ideia maligna; truque sujo; péssima ideia}
  \definition{v.}{estragar; destruir; corromper}
\end{entry}

\begin{entry}{坏处}{huai4 chu4}{7,5}{⼟、⼡}[HSK 2]
  \definition[个]{s.}{dano; prejuízo; desvantagem; fatores prejudiciais a pessoas ou coisas}
\end{entry}

\begin{entry}{坏蛋}{huai4dan4}{7,11}{⼟、⾍}
  \definition{s.}{bastardo | canalha | pessoa má}
\end{entry}

\begin{entry}{坏人}{huai4 ren2}{7,2}{⼟、⼈}[HSK 2]
  \definition[个,种]{s.}{malfeitor; canalha; pessoa má; pessoa de má qualidade; pessoa que faz coisas ruins}
\end{entry}

\begin{entry}{欢快}{huan1kuai4}{6,7}{⽋、⼼}
  \definition{adj.}{feliz e sem ansiedade | vívido}
\end{entry}

\begin{entry}{欢乐}{huan1le4}{6,5}{⽋、⼃}[HSK 3]
  \definition{adj.}{feliz; alegre}
\end{entry}

\begin{entry}{欢迎}{huan1ying2}{6,7}{⽋、⾡}[HSK 2]
  \definition{adj.}{bem-vindo}
  \definition{v.}{dar as boas-vindas; cumprimentar; receber com alegria | dar as boas-vindas; receber favoravelmente (bem)}
\end{entry}

\begin{entry}{还}{huan2}{7}{⾡}[HSK 1]
  \definition*{s.}{sobrenome Huan}
  \definition{v.}{voltar; retornar; voltar ao lugar original ou restaurar o estado original | retribuir; devolver; reembolsar; devolver o dinheiro ou os bens emprestados ao seu proprietário | dar ou fazer algo em troca; retribuir as ações dos outros}
  \seeref{还}{hai2}
\end{entry}

\begin{entry}{环}{huan2}{8}{⽟}[HSK 3]
  \definition*{s.}{sobrenome Huan}
  \definition{clas.}{para anéis}
  \definition{s.}{anel; arco | \emph{link}; ligação}
  \definition{v.}{cercar; rodear; circular; circundar}
\end{entry}

\begin{entry}{环保}{huan2 bao3}{8,9}{⽟、⼈}[HSK 3]
  \definition{adj.}{bom para o meio ambiente; não danifica o meio ambiente}
  \definition{s.}{proteção ambiental}
\end{entry}

\begin{entry}{环节}{huan2jie2}{8,5}{⽟、⾋}[HSK 5]
  \definition{s.}{\emph{link}; ligação; vínculo; uma das muitas coisas que estão inter-relacionadas | segmento; estrutura anelar de alguns animais inferiores}
\end{entry}

\begin{entry}{环境}{huan2jing4}{8,14}{⽟、⼟}[HSK 3]
  \definition[个]{s.}{ambiente | arredores; circunstâncias}
\end{entry}

\begin{entry}{环境卫生}{huan2jing4wei4sheng1}{8,14,3,5}{⽟、⼟、⼙、⽣}
  \definition{s.}{saneamento ambiental}
  \seealsoref{环卫}{huan2wei4}
\end{entry}

\begin{entry}{环卫}{huan2wei4}{8,3}{⽟、⼙}
  \definition{s.}{limpeza pública | saneamento urbano | saneamento ambiental | abreviação de 环境卫生}
  \seealsoref{环境卫生}{huan2jing4wei4sheng1}
\end{entry}

\begin{entry}{缓}{huan3}{12}{⽶}
  \definition{adj.}{lento; sem pressa | sem tensão; relaxado}
  \definition{v.}{atrasar; adiar; protelar | recuperar; reviver; voltar a si}
\end{entry}

\begin{entry}{缓解}{huan3jie3}{12,13}{⽶、⾓}[HSK 4]
  \definition{v.}{facilitar; aliviar; atenuar; amenizar; reduzir}
\end{entry}

\begin{entry}{幻觉}{huan4jue2}{4,9}{⼳、⾒}
  \definition{s.}{ilusão | alucinação}
\end{entry}

\begin{entry}{换}{huan4}{10}{⼿}[HSK 2]
  \definition{v.}{negociar; trocar; permutar; dar algo a alguém e, ao mesmo tempo, obter algo dele em troca | mudar; transformar; substituir | trocar dinheiro (câmbio) | transfundir (sangue) | transplantar (um órgão)}
\end{entry}

\begin{entry}{换钱}{huan4qian2}{10,10}{⼿、⾦}
  \definition{v.+compl.}{trocar dinheiro (em pequenas valores ou em outra moeda) | trocar (mercadorias) por dinheiro | vender}
\end{entry}

\begin{entry}{荒芜}{huang1wu2}{9,7}{⾋、⾋}
  \definition{adj.}{estéril}
\end{entry}

\begin{entry}{慌}{huang1}{12}{⼼}[HSK 5]
  \definition{adj.}{agitado; confuso; que inspira terror}
  \definition{v.}{ficar com medo; ficar nervoso}
\end{entry}

\begin{entry}{慌忙}{huang1 mang2}{12,6}{⼼、⼼}[HSK 5]
  \definition{adj.}{apressado; afobado; com muita pressa}
  \definition{adv.}{apressadamente}
\end{entry}

\begin{entry}{皇帝}{huang2di4}{9,9}{⽩、⼱}
  \definition[个]{s.}{imperador}
\end{entry}

\begin{entry}{黄}{huang2}{11}{⿈}[HSK 2][Kangxi 201]
  \definition*{s.}{sobrenome Huang ou Hwang}
  \definition*{s.}{abreviação de Rio Huanghe | refere-se ao Imperador Amarelo, um imperador da mitologia chinesa antiga}
  \definition{adj.}{amarelo | obsceno; indecente; pornográfico; símbolo de corrupção e decadência, referindo-se especificamente à pornografia}
  \definition{s.}{gema; ovas de caranguejo; refere-se a certas coisas de cor amarela}
  \definition{v.}{fracassar; dar errado}
  \seealsoref{黄河}{huang2he2}
\end{entry}

\begin{entry}{黄瓜}{huang2 gua1}{11,5}{⿈、⽠}[HSK 4]
  \definition[根,棵,株]{s.}{pepino}
\end{entry}

\begin{entry}{黄河}{huang2he2}{11,8}{⿈、⽔}
  \definition*{s.}{Rio Amarelo | Rio Huang He}
\end{entry}

\begin{entry}{黄昏}{huang2hun1}{11,8}{⿈、⽇}
  \definition{s.}{anoitecer}
\end{entry}

\begin{entry}{黄金}{huang2jin1}{11,8}{⿈、⾦}[HSK 4]
  \definition{adj.}{de primeira qualidade; dourado;}
  \definition[块,克,两]{s.}{ouro; \emph{aurum}; um tipo de metal, de cor amarela, mais precioso, abreviação de 金, frequentemente falado como 金子}
  \seealsoref{金}{jin1}
  \seealsoref{金子}{jin1zi5}
\end{entry}

\begin{entry}{黄色}{huang2 se4}{11,6}{⿈、⾊}[HSK 2]
  \definition{adj.}{decadente; obsceno; erótico; pornográfico; símbolo de corrupção e decadência, referindo-se especificamente à pornografia}
  \definition[种]{s.}{cor amarela}
\end{entry}

\begin{entry}{黄油}{huang2you2}{11,8}{⿈、⽔}
  \definition[盒]{s.}{manteiga}
\end{entry}

\begin{entry}{谎话}{huang3hua4}{11,8}{⾔、⾔}
  \definition{s.}{mentira}
\end{entry}

\begin{entry}{灰色}{hui1 se4}{6,6}{⽕、⾊}[HSK 5]
  \definition{adj.}{obscuro; ambíguo | sombrio; pessimista}
  \definition[个]{s.}{cor cinza; acinzentado}
\end{entry}

\begin{entry}{恢复}{hui1fu4}{9,9}{⼼、⼢}[HSK 5]
  \definition{v.}{retomar; renovar; restaurar; voltar a | reviver; recuperar; reencontrar | restaurar; restabelecer; reabilitar; regenerar; ressurgir; restabelecer alguém em; recuperar o que foi perdido}
\end{entry}

\begin{entry}{挥汗如雨}{hui1han4ru2yu3}{9,6,6,8}{⼿、⽔、⼥、⾬}
  \definition{s.}{suor derramado}
  \definition{v.}{pingar com suor}
\end{entry}

\begin{entry}{囘}{hui2}{5}{⼞}
  \variantof{回}
\end{entry}

\begin{entry}{回}{hui2}{6}{⼞}[HSK 1,2]
  \definition{clas.}{usado para coisas, ações, número de vezes |  um trecho de um conto; um capítulo de um romance em capítulos}
  \definition{s.}{sobrenome Hui | seção ou capítulo (de um livro clássico) | grupo étnico Hui (mulçumanos chineses)}
  \definition{v.}{circular; enrolar | retornar; voltar; voltar ao lugar de origem | dar meia-volta | responder; contestar | relatar; reportar; responder}
\end{entry}

\begin{entry}{回报}{hui2bao4}{6,7}{⼞、⼿}[HSK 5]
  \definition{s.}{recompensa; pagamento; benefícios recebidos como resultado de assistência, esforço ou afeto | retornos; benefícios recebidos por meio de investimentos}
  \definition{v.}{pagar de volta; beneficiar pessoas ou organizações que os ajudaram ou cuidaram deles de alguma forma}
\end{entry}

\begin{entry}{回避}{hui2bi4}{6,16}{⼞、⾌}
  \definition{v.}{fugir (de um problema); em direito, refere-se especificamente à não participação nos procedimentos de um caso de um oficial de justiça, etc., que tenha interesse no caso ou nas partes do caso | esquivar-se; evadir-se; evitar (encontrar alguém)}
\end{entry}

\begin{entry}{回答}{hui2da2}{6,12}{⼞、⽵}[HSK 1]
  \definition[个]{s.}{resposta}
  \definition{v.}{responder; explicar a questão; expressar opinião sobre a solicitação}
\end{entry}

\begin{entry}{回到}{hui2 dao4}{6,8}{⼞、⼑}[HSK 1]
  \definition{v.}{retornar para; voltar e chegar (ao lugar onde estava originalmente); (após uma mudança nas circunstâncias) retornar ao estado original}
\end{entry}

\begin{entry}{回复}{hui2 fu4}{6,9}{⼞、⼢}[HSK 4]
  \definition{v.}{responder (a uma carta) | retornar ao estado normal; restaurar algo ao seu estado original}
\end{entry}

\begin{entry}{回顾}{hui2gu4}{6,10}{⼞、⾴}[HSK 5]
  \definition{v.}{olhar para trás | revisar; fazer uma retrospectiva; olhar para trás, pensar no passado}
\end{entry}

\begin{entry}{回国}{hui2 guo2}{6,8}{⼞、⼞}[HSK 2]
  \definition{v.}{regressar ao seu país (terra natal); referindo-se a voltar do exterior}
\end{entry}

\begin{entry}{回家}{hui2 jia1}{6,10}{⼞、⼧}[HSK 1]
  \definition{v.}{ir (voltar) para casa; estar em casa; voltar para casa}
\end{entry}

\begin{entry}{回来}{hui2 lai5}{6,7}{⼞、⽊}[HSK 1]
  \definition{v.}{voltar; regressar (para a minha localização) | retornar; usado após um verbo, significa ``vir ao lugar original''}
\end{entry}

\begin{entry}{回去}{hui2 qu4}{6,5}{⼞、⼛}[HSK 1]
  \definition{v.}{retornar; voltar; estar de volta ; (a partir da minha localização)}
\end{entry}

\begin{entry}{回收}{hui2shou1}{6,6}{⼞、⽁}[HSK 5]
  \definition{v.}{reciclar; reciclar itens (geralmente resíduos ou produtos antigos) para reutilização | recuperar; recolher; recuperar o que foi emitido ou demitido}
\end{entry}

\begin{entry}{回头}{hui2 tou2}{6,5}{⼞、⼤}[HSK 5]
  \definition{adv.}{mais tarde; depois de um tempo}
  \definition{conj.}{ou então; usado no início da segunda metade de uma frase para indicar o que acontecerá se você não fizer o que fez na primeira metade da frase}
  \definition{v.}{dar a meia-volta; virar a cabeça; virar a cabeça para trás | retornar; voltar | arrepender-se; corrigir seu caminho; reconhecer e corrigir erros}
\end{entry}

\begin{entry}{回信}{hui2 xin4}{6,9}{⼞、⼈}[HSK 5]
  \definition[封]{s.}{uma carta em resposta; uma mensagem verbal em resposta}
  \definition{v.+compl.}{escrever em resposta; escrever de volta; responder uma carta; responder verbalmente uma mensagem}
\end{entry}

\begin{entry}{回旋}{hui2xuan2}{6,11}{⼞、⽅}
  \definition{v.}{circular | rodar | dar a volta}
\end{entry}

\begin{entry}{回忆}{hui2yi4}{6,4}{⼞、⼼}[HSK 5]
  \definition[个,段]{s.}{memória; lembrança de eventos ou experiências passadas}
  \definition{v.}{lembrar; recordar}
\end{entry}

\begin{entry}{廻}{hui2}{8}{⼵}
  \variantof{回}
\end{entry}

\begin{entry}{汇}{hui4}{5}{⽔}[HSK 4]
  \definition{s.}{coisas coletadas; conjunto; coleção}
  \definition{v.}{convergir | reunir-se | remeter; transferir por meio de agências postais e telegráficas, bancos}
\end{entry}

\begin{entry}{汇报}{hui4bao4}{5,7}{⽔、⼿}[HSK 4]
  \definition[份,次]{s.}{relatório; referindo-se ao conteúdo de declarações escritas ou orais feitas a um superior ou pessoa relevante para apresentar uma situação ou refletir um problema}
  \definition{v.}{relatar; fazer um relato de}
\end{entry}

\begin{entry}{汇款}{hui4 kuan3}{5,12}{⽔、⽋}[HSK 5]
  \definition[笔,个]{s.}{remessa; dinheiro enviado ou recebido}
  \definition{v.+compl.}{remeter dinheiro; fazer uma remessa; enviar dinheiro}
\end{entry}

\begin{entry}{汇率}{hui4lv4}{5,11}{⽔、⽞}[HSK 4]
  \definition[个]{s.}{taxa de câmbio; relação entre a moeda de um país e a de outro}
\end{entry}

\begin{entry}{会}{hui4}{6}{⼈}[HSK 1,2]
  \definition{adv.}{um momento}
  \definition{clas.}{momento; um curto período de tempo}
  \definition{s.}{reunião; festa; conferência; reunião com um objetivo específico | reunião; reunião no trabalho | feira do templo; festival religioso | associação; sociedade; sindicato; certas organizações públicas | oportunidade; ocasião; momento oportuno | cidade principal; capital; cidade central}
  \definition{suf.}{união; grupo; associação}
  \definition{v.}{ser provável que; ter certeza de; indica que é possível realizar (é possível responder à pergunta separadamente) |  poder; ser capaz de; significa saber como fazer ou ter a capacidade de fazer (geralmente se refere a coisas que precisam ser aprendidas) | saber; compreender; entender | encontrar; ver | reunir-se; reunir; agregar; juntar | destacar-se em; ser bom em; ser hábil em; indica proficiência | pagar (ou custear) contas}
  \seeref{会}{kuai4}
\end{entry}

\begin{entry}{会首}{hui4shou3}{6,9}{⼈、⾸}
  \definition{s.}{chefe de uma sociedade | patrocinador de uma organização}
\end{entry}

\begin{entry}{会谈}{hui4 tan2}{6,10}{⼈、⾔}[HSK 5]
  \definition{v.}{manter conversações; comumente usado em assuntos internacionais ou atividades diplomáticas}
\end{entry}

\begin{entry}{会议}{hui4yi4}{6,5}{⼈、⾔}[HSK 3]
  \definition[场,届,个]{s.}{reunião; conferência | conselho; congresso}
\end{entry}

\begin{entry}{会员}{hui4 yuan2}{6,7}{⼈、⼝}[HSK 3]
  \definition[位]{s.}{membro; associado | filiação}
\end{entry}

\begin{entry}{婚礼}{hun1li3}{11,5}{⼥、⽰}[HSK 4]
  \definition[场]{s.}{casamento; núpcias; cerimônia de casamento}
\end{entry}

\begin{entry}{魂}{hun2}{13}{⿁}
  \definition{s.}{alma | espírito | alma imortal (que pode ser separada do corpo)}
\end{entry}

\begin{entry}{混饭}{hun4fan4}{11,7}{⽔、⾷}
  \definition{v.+compl.}{trabalhar para viver}
\end{entry}

\begin{entry}{混乱}{hun4luan4}{11,7}{⽔、⼄}
  \definition{adj.}{confuso | caótico | desordenado}
  \definition{s.}{caos}
\end{entry}

\begin{entry}{和}{huo2}{8}{⼝}
  \definition{v.}{combinar uma substância em pó (farinha, gesso, etc.) com água; adicionar líquido ao pó e mexer ou amassar até ficar pegajoso}
  \seeref{和}{he2}
  \seeref{和}{he4}
  \seeref{和}{hu2}
  \seeref{和}{huo4}
\end{entry}

\begin{entry}{活}{huo2}{9}{⽔}[HSK 3]
  \definition{adj.}{vivo; vivendo | vívido; animado; ativo | móvel; em movimento}
  \definition{adv.}{exatamente; simplesmente}
  \definition{s.}{trabalho | produto}
  \definition{v.}{viver | salvar (a vida de uma pessoa)}
\end{entry}

\begin{entry}{活动}{huo2dong4}{9,6}{⽔、⼒}[HSK 2]
  \definition{adj.}{móvel; flexível para alterações ou mudanças}
  \definition[些,个,种,类,次]{s.}{atividade; ação tomada com o objetivo de alcançar um determinado objetivo}
  \definition{v.}{fazer exercício; movimentar-se | usar influência pessoal; usar meios irregulares | mover-se}
\end{entry}

\begin{entry}{活力}{huo2li4}{9,2}{⽔、⼒}[HSK 5]
  \definition{s.}{vigor; vitalidade; energia; muito forte, geralmente usado para descrever pessoas, cidades, empresas, economias, etc.}
\end{entry}

\begin{entry}{活路}{huo2lu4}{9,13}{⽔、⾜}
  \definition{s.}{maneira de sobreviver | meio de subsistência}
  \seeref{活路}{huo2lu5}
\end{entry}

\begin{entry}{活路}{huo2lu5}{9,13}{⽔、⾜}
  \definition{s.}{labor | trabalho físico}
  \seeref{活路}{huo2lu4}
\end{entry}

\begin{entry}{活泼}{huo2po1}{9,8}{⽔、⽔}[HSK 5]
  \definition{adj.}{vívido; ativo; animado; brilhante; vivaz; cheio de vida | reativo; (química) significa que a substância é ativa e reage facilmente com outras substâncias}
\end{entry}

\begin{entry}{活着}{huo2zhe5}{9,11}{⽔、⽬}
  \definition{adj.}{vivo}
\end{entry}

\begin{entry}{火}{huo3}{4}{⽕}[HSK 3,4][Kangxi 86]
  \definition*{s.}{sobrenome Huo}
  \definition{adj.}{ardente; flamejante; vermelho como fogo | efervescente; próspero}
  \definition{adv.}{urgentemente}
  \definition{clas.}{para unidades militares (antigo)}
  \definition{s.}{fogo | armas de fogo; munições | calor interno (uma das seis causas de doenças) | a ação de lutar}
  \definition{v.}{ficar com raiva; perder a paciência}
\end{entry}

\begin{entry}{火柴}{huo3chai2}{4,10}{⽕、⽊}[HSK 5]
  \definition[根,盒]{s.}{fósforo (palito de fósforo); fósforo de segurança; iniciador de fogo feito de uma tira fina de madeira mergulhada em um composto de fósforo ou enxofre}
\end{entry}

\begin{entry}{火车}{huo3 che1}{4,4}{⽕、⾞}[HSK 1]
  \definition[个,列,节,班,趟]{s.}{trem; comboio}
\end{entry}

\begin{entry}{火车司机}{huo3che1 si1ji1}{4,4,5,6}{⽕、⾞、⼝、⽊}
  \definition{s.}{maquinista de trem}
\end{entry}

\begin{entry}{火海}{huo3hai3}{4,10}{⽕、⽔}
  \definition{s.}{um mar de chamas}
\end{entry}

\begin{entry}{火腿}{huo3 tui3}{4,13}{⽕、⾁}[HSK 5]
  \definition[道,个]{s.}{presunto; as pernas de porco marinadas mais famosas são produzidas em Jinhua, na província de Zhejiang, e em Xuanwei, na província de Yunnan.}
\end{entry}

\begin{entry}{火灾}{huo3 zai1}{4,7}{⽕、⽕}[HSK 5]
  \definition[场]{s.}{fogo (como um desastre); conflagração; desastres causados por incêndios}
\end{entry}

\begin{entry}{伙}{huo3}{6}{⼈}[HSK 4]
  \definition{clas.}{grupo; multidão; banda}
  \definition{s.}{iguaria; alimentação; refeições | parceiro; companheiro | coletivo de colegas}
  \definition{v.}{combinar; unir}
\end{entry}

\begin{entry}{伙伴}{huo3ban4}{6,7}{⼈、⼈}[HSK 4]
  \definition[个,位,群]{s.}{parceiro; companheiro; antigo sistema militar de dez pessoas para uma fogueira, o chefe da fogueira, uma pessoa encarregada de cozinhar, com a fogueira é chamado de parceiro da fogueira, agora se refere à participação comum em uma determinada organização ou engajada em certas atividades}
\end{entry}

\begin{entry}{和}{huo4}{8}{⼝}
  \definition{clas.}{para enxágues de roupas | para fervuras de ervas medicinais}
  \definition{v.}{misturar (ingredientes); misturar pós ou grãos; misturar com água para obter uma consistência mais líquida}
  \seeref{和}{he2}
  \seeref{和}{he4}
  \seeref{和}{hu2}
  \seeref{和}{huo2}
\end{entry}

\begin{entry}{或}{huo4}{8}{⼽}[HSK 2]
  \definition{adv.}{talvez; possivelmente; provavelmente | (geralmente na forma negativa) um pouco; ligeiramente}
  \definition{conj.}{ou (indicando escolha); ou\dots ou\dots}
  \definition{pron.}{alguém; algumas pessoas; refere-se a alguém ou algo, equivalente a 有人 ou 有的}
  \seealsoref{有的}{you3 de5}
  \seealsoref{有人}{you3 ren2}
\end{entry}

\begin{entry}{或是}{huo4 shi4}{8,9}{⼽、⽇}[HSK 5]
  \definition{adv.}{um ou outro; o outro}
  \definition{conj.}{ou; às vezes, é apenas uma de duas coisas}
\end{entry}

\begin{entry}{或许}{huo4xu3}{8,6}{⼽、⾔}[HSK 4]
  \definition{adv.}{talvez; possivelmente; receio; não tenho certeza}
\end{entry}

\begin{entry}{或者}{huo4zhe3}{8,8}{⼽、⽼}[HSK 2]
  \definition{adv.}{talvez; possivelmente}
  \definition{conj.}{ou (usado em expressões afirmativas); ou\dots ou\dots; usado em frases narrativas para indicar uma relação de escolha | ou (usado para indicar equação); indica relação de equivalência, indicando que os objetos anterior e posterior são iguais}
\end{entry}

\begin{entry}{货}{huo4}{8}{⾙}[HSK 4]
  \definition{s.}{dinheiro; moeda | bens; mercadorias; \emph{commodity} | palavras insultuosas dirigidas a alguém; maldição; xingamento}
\end{entry}

\begin{entry}{货车}{huo4che1}{8,4}{⾙、⾞}
  \definition{s.}{caminhão | van | vagão de carga}
\end{entry}

\begin{entry}{获}{huo4}{10}{⾋}[HSK 4]
  \definition*{s.}{sobrenome Huo}
  \definition{v.}{capturar; pegar | obter; ganhar; colher | colher; ceifar}
\end{entry}

\begin{entry}{获得}{huo4de2}{10,11}{⾋、⼻}[HSK 4]
  \definition{v.}{adquirir; ganhar; obter; alcançar}
\end{entry}

\begin{entry}{获奖}{huo4 jiang3}{10,9}{⾋、⼤}[HSK 4]
  \definition{v.}{ganhar prêmio; ser recompensado; ganhar um prêmio; receber um prêmio}
\end{entry}

\begin{entry}{获取}{huo4 qu3}{10,8}{⾋、⼜}[HSK 4]
  \definition{v.}{adquirir; obter; ganhar; colher}
\end{entry}

\begin{entry}{惑星}{huo4xing1}{12,9}{⼼、⽇}
  \definition{s.}{planeta}
  \seealsoref{行星}{xing2xing1}
\end{entry}

%%%%% EOF %%%%%


%%%%%%%%%%%%%%%%%%%%%%%%%%%%%% Não existem palavras com pinyin iniciado em "I"
%%%
%%% J
%%%

\section*{J}\addcontentsline{toc}{section}{J}

\begin{entry}{几}{ji1}{2}{⼏}[Kangxi 16]
  \definition{adv.}{quase}
  \definition{s.}{mesa pequena}
  \seeref{几}{ji3}
\end{entry}

\begin{entry}{几乎}{ji1hu1}{2,5}{⼏、⼃}[HSK 4]
  \definition{adv.}{quase; praticamente; próximo a | perto de; quase; à beira de}
\end{entry}

\begin{entry}{机场}{ji1chang3}{6,6}{⽊、⼟}[HSK 1]
  \definition[家,处]{s.}{aeroporto | aeródromo}
\end{entry}

\begin{entry}{机构}{ji1gou4}{6,8}{⽊、⽊}[HSK 4]
  \definition[所]{s.}{órgão; organização; instituição; instalações; aparelhamento; configuração | mecanismo; funcionamento interno de uma máquina ou unidade | estrutura interna de uma organização}
\end{entry}

\begin{entry}{机会}{ji1hui4}{6,6}{⽊、⼈}[HSK 2]
  \definition{s.}{chance | oportunidade}
\end{entry}

\begin{entry}{机甲}{ji1jia3}{6,5}{⽊、⽥}
  \definition{s.}{\emph{mecha} (robôs operados pelo homem em mangá japonês)}
\end{entry}

\begin{entry}{机票}{ji1 piao4}{6,11}{⽊、⽰}[HSK 1]
  \definition[张]{s.}{bilhete de avião}
  \seealsoref{飞机票}{fei1ji1 piao4}
\end{entry}

\begin{entry}{机器}{ji1qi4}{6,16}{⽊、⼝}[HSK 3]
  \definition[台,部,个]{s.}{máquina; maquinário; motor | aparelho; dispositivo}
\end{entry}

\begin{entry}{机器人}{ji1 qi4 ren2}{6,16,2}{⽊、⼝、⼈}[HSK 5]
  \definition[个]{s.}{androide; golem | pessoa mecânica | robô}
\end{entry}

\begin{entry}{机械}{ji1xie4}{6,11}{⽊、⽊}
  \definition{s.}{máquina | maquinaria | mecânica}
\end{entry}

\begin{entry}{机遇}{ji1yu4}{6,12}{⽊、⾡}[HSK 4]
  \definition[个]{s.}{chance; oportunidade; circunstâncias favoráveis}
\end{entry}

\begin{entry}{机制}{ji1 zhi4}{6,8}{⽊、⼑}[HSK 5]
  \definition{s.}{mecanismo; processado por máquina; feito por máquina}
\end{entry}

\begin{entry}{肌肉}{ji1rou4}{6,6}{⾁、⾁}[HSK 5]
  \definition[身,块]{s.}{músculo; um dos tecidos básicos dos músculos humanos e de alguns animais, composto principalmente de células musculares fibrosas, pode se contrair, é o movimento do corpo e o corpo de digestão, respiração, circulação, excreção e outros processos fisiológicos da fonte de energia; pode ser dividido em três tipos: músculo liso, músculo esquelético e músculo cardíaco}
\end{entry}

\begin{entry}{鸡}{ji1}{7}{⿃}[HSK 2]
  \definition[只]{s.}{galo, galinha | (gíria) prostituta}
\end{entry}

\begin{entry}{鸡蛋}{ji1dan4}{7,11}{⿃、⾍}[HSK 1]
  \definition[个,打]{s.}{ovo de galinha}
\end{entry}

\begin{entry}{积极}{ji1ji2}{10,7}{⽲、⽊}[HSK 3]
  \definition{adj.}{ativo | positivo}
\end{entry}

\begin{entry}{积累}{ji1lei3}{10,11}{⽲、⽷}[HSK 4]
  \definition{s.}{acúmulo; acumulação}
  \definition{v.}{acumular}
\end{entry}

\begin{entry}{积木}{ji1mu4}{10,4}{⽲、⽊}
  \definition{s.}{blocos de montar (brinquedo)}
\end{entry}

\begin{entry}{基本}{ji1ben3}{11,5}{⼟、⽊}[HSK 3]
  \definition{adj.}{básico; fundamental; elementar | principal}
  \definition{adv.}{basicamente; em geral}
  \definition{s.}{fundação}
\end{entry}

\begin{entry}{基本法}{ji1ben3fa3}{11,5,8}{⼟、⽊、⽔}
  \definition{s.}{lei básica (constituição)}
\end{entry}

\begin{entry}{基本功}{ji1ben3gong1}{11,5,5}{⼟、⽊、⼒}
  \definition{s.}{habilidades | fundamentos básicos}
\end{entry}

\begin{entry}{基本上}{ji1 ben3 shang4}{11,5,3}{⼟、⽊、⼀}[HSK 3]
  \definition{adv.}{basicamente; no principal | em geral}
\end{entry}

\begin{entry}{基础}{ji1chu3}{11,10}{⼟、⽯}[HSK 3]
  \definition{adj.}{básico; fundamental}
  \definition[个]{s.}{base; fundamento; fundação}
\end{entry}

\begin{entry}{基地}{ji1di4}{11,6}{⼟、⼟}[HSK 5]
  \definition{s.}{base; como base para alguns negócios | base; um local dedicado à realização de um negócio}
\end{entry}

\begin{entry}{基督教}{ji1du1jiao4}{11,13,11}{⼟、⽬、⽁}
  \definition*{s.}{Cristianismo | Cristão}
\end{entry}

\begin{entry}{基金}{ji1jin1}{11,8}{⼟、⾦}[HSK 5]
  \definition[项,支,种,个]{s.}{fundo; fundos reservados ou destinados ao estabelecimento ou desenvolvimento de uma empresa}
\end{entry}

\begin{entry}{基因}{ji1yin1}{11,6}{⼟、⼞}
  \definition{s.}{gene}
\end{entry}

\begin{entry}{激动}{ji1dong4}{16,6}{⽔、⼒}[HSK 4]
  \definition{adj.}{animado; entusiasmado; empolgado}
  \definition{v.}{agitar; excitar; tornar fortes os sentimentos de alguém}
\end{entry}

\begin{entry}{激烈}{ji1lie4}{16,10}{⽔、⽕}[HSK 4]
  \definition{adj.}{agudo; afiado; feroz; violento; intenso}
\end{entry}

\begin{entry}{鷄}{ji1}{21}{⿃}
  \variantof{鸡}
\end{entry}

\begin{entry}{及}{ji2}{3}{⼃}
  \definition{conj.}{e | bem como}
\end{entry}

\begin{entry}{及格}{ji2ge2}{3,10}{⼃、⽊}[HSK 4]
  \definition{v.+compl.}{passar; passar em um teste, exame, etc.}
\end{entry}

\begin{entry}{及时}{ji2shi2}{3,7}{⼃、⽇}[HSK 3]
  \definition{adj.}{oportuno; a tempo; sazonal}
  \definition{adv.}{prontamente; sem demora}
\end{entry}

\begin{entry}{吉他}{ji2ta1}{6,5}{⼝、⼈}
  \definition[把]{s.}{(empréstimo linguístico) guitarra}
\end{entry}

\begin{entry}{级}{ji2}{6}{⽷}[HSK 2]
  \definition{clas.}{para passo, estágio}
  \definition{s.}{nível | classificação | grau | qualquer uma das divisões anuais de um curso escolar: série, classe, etc. | etapa}
\end{entry}

\begin{entry}{即}{ji2}{7}{⼙}
  \definition{conj.}{e | até | mesmo se/embora}
\end{entry}

\begin{entry}{即便}{ji2bian4}{7,9}{⼙、⼈}
  \definition{conj.}{mesmo se/embora}
\end{entry}

\begin{entry}{即或}{ji2huo4}{7,8}{⼙、⼽}
  \definition{conj.}{mesmo se/embora}
\end{entry}

\begin{entry}{即将}{ji2jiang1}{7,9}{⼙、⼨}[HSK 4]
  \definition{adv.}{em breve; estar prestes a; estar a ponto de}
\end{entry}

\begin{entry}{即若}{ji2ruo4}{7,8}{⼙、⾋}
  \definition{conj.}{mesmo se/embora}
\end{entry}

\begin{entry}{即使}{ji2shi3}{7,8}{⼙、⼈}[HSK 5]
  \definition{conj.}{mesmo; mesmo que; mesmo se; apesar de; expressando uma concessão hipotética}
\end{entry}

\begin{entry}{即是}{ji2shi4}{7,9}{⼙、⽇}
  \definition{conj.}{aquilo é}
\end{entry}

\begin{entry}{极}{ji2}{7}{⽊}[HSK 4]
  \definition*{s.}{sobrenome Ji}
  \definition{adj.}{máximo; extremo; final; supremo}
  \definition{adv.}{extremamente; excessivamente}
  \definition{s.}{o ponto máximo, mais alto; extremo; ápice; ponto culminante |
pólo; as extremidades norte e sul da Terra; as extremidades de um ímã; a extremidade de uma fonte de alimentação ou de um aparelho elétrico onde a corrente entra ou sai do aparelho}
  \definition{v.}{chegar ao fim de; levar a extremos}
\end{entry}

\begin{entry}{……极了}{ji2le5}{7,2}{⽊、⼅}[HSK 3]
  \definition{expr.}{extremamente}
\end{entry}

\begin{entry}{极其}{ji2qi2}{7,8}{⽊、⼋}[HSK 4]
  \definition{adv.}{mais; extremamente; excessivamente}
\end{entry}

\begin{entry}{急}{ji2}{9}{⼼}[HSK 2]
  \definition{adj.}{impaciente |ansioso | irritado | aborrecido |violento | urgente | premente}
  \definition{s.}{urgência | emergência}
  \definition{v.}{preocupar | estar ansioso para ajudar}
\end{entry}

\begin{entry}{急救}{ji2jiu4}{9,11}{⼼、⽁}
  \definition{s.}{primeiros socorros}
  \definition{v.}{dar tratamento de emergência}
\end{entry}

\begin{entry}{急忙}{ji2mang2}{9,6}{⼼、⼼}[HSK 4]
  \definition{adv.}{apressadamente; com pressa}
\end{entry}

\begin{entry}{集合}{ji2he2}{12,6}{⾫、⼝}[HSK 4]
  \definition{v.}{reunir-se; juntar-se | reunir, juntar, convocar}
\end{entry}

\begin{entry}{集体}{ji2ti3}{12,7}{⾫、⼈}[HSK 3]
  \definition{s.}{coletivo; comunidade; grupo; equipe}
\end{entry}

\begin{entry}{集团}{ji2tuan2}{12,6}{⾫、⼞}[HSK 5]
  \definition[个]{s.}{anel; bloco; grupo; panelinha; círculo; grupo organizado para agir em conjunto com um determinado objetivo | grupo; entidade econômica com uma direção de negócios especializada, liderada por uma grande empresa com forte poder econômico e alta visibilidade, e formada pela combinação ou fusão de empresas relacionadas}
\end{entry}

\begin{entry}{集中}{ji2zhong1}{12,4}{⾫、⼁}[HSK 3]
  \definition{adj.}{centralizado; concentrado}
  \definition{v.}{concentrar; juntar}
\end{entry}

\begin{entry}{嫉妒}{ji2du4}{13,7}{⼥、⼥}
  \definition{v.}{estar com ciúmes de | invejar}
\end{entry}

\begin{entry}{几}{ji3}{2}{⼏}[HSK 1]
  \definition{adv.}{quantos?, (até 10 itens) | vários | alguns}
  \seeref{几}{ji1}
\end{entry}

\begin{entry}{几何}{ji3he2}{2,7}{⼏、⼈}
  \definition{s.}{geometria}
\end{entry}

\begin{entry}{挤}{ji3}{9}{⼿}[HSK 5]
  \definition{adj.}{lotado; congestionado; descreve um grande número de pessoas ou coisas e muito pouco espaço}
  \definition{v.}{empacotar; amontoar; aglomerar | sacudir; empurrar contra; empurrar alguém ou algo para longe com seu corpo com toda a força que puder| pressionar; apertar; expulsar por pressão}
\end{entry}

\begin{entry}{给}{ji3}{9}{⽷}
  \definition{v.}{fornecer | prover}
  \seeref{给}{gei3}
\end{entry}

\begin{entry}{计划}{ji4hua4}{4,6}{⾔、⼑}[HSK 2]
  \definition[个,项]{s.}{plano | projeto | programa}
  \definition{v.}{planejar | mapear}
\end{entry}

\begin{entry}{计算}{ji4suan4}{4,14}{⾔、⽵}[HSK 3]
  \definition{v.}{contar; calcular; computar; enumerar | planejar; considerar | conspirar secretamente contra os outros}
\end{entry}

\begin{entry}{计算机}{ji4 suan4 ji1}{4,14,6}{⾔、⽵、⽊}[HSK 2]
  \definition[部,台]{s.}{computador | calculadora}
\end{entry}

\begin{entry}{记}{ji4}{5}{⾔}[HSK 1]
  \definition{clas.}{para tapas, palmadas, bofetadas, etc.}
  \definition{s.}{nota | registro | marca | sinal |marca de nascença}
  \definition{v.}{lembrar | ter em mente | memorizar | escrever (anotar) | registrar}
\end{entry}

\begin{entry}{记得}{ji4de5}{5,11}{⾔、⼻}[HSK 1]
  \definition{v.}{lembrar | lembrar-se}
\end{entry}

\begin{entry}{记录}{ji4lu4}{5,8}{⾔、⼹}[HSK 3]
  \definition[个,位]{s.}{notas; registro | anotador; registrador}
  \definition{v.}{tomar notas; registrar}
\end{entry}

\begin{entry}{记性}{ji4xing5}{5,8}{⾔、⼼}
  \definition{s.}{memória (habilidade em reter informações)}
\end{entry}

\begin{entry}{记忆}{ji4yi4}{5,4}{⾔、⼼}[HSK 5]
  \definition[段]{s.}{memória; manter em sua mente uma imagem do passado}
  \definition{v.}{recordar; lembrar; lembrar-se ou recordar alguém ou algo do passado}
\end{entry}

\begin{entry}{记载}{ji4zai3}{5,10}{⾔、⾞}[HSK 4]
  \definition[段,份]{s.}{registro; conta; artigos e materiais que registram eventos}
  \definition{v.}{registrar; colocar por escrito}
\end{entry}

\begin{entry}{记者}{ji4zhe3}{5,8}{⾔、⽼}[HSK 3]
  \definition[群,名,位]{s.}{repórter; correspondente; jornalista}
\end{entry}

\begin{entry}{记住}{ji4 zhu5}{5,7}{⾔、⼈}[HSK 1]
  \definition{v.}{decorar | memorizar | ter em mente}
\end{entry}

\begin{entry}{纪录}{ji4lu4}{6,8}{⽷、⼹}[HSK 3]
  \definition{s.}{recorde (esportes)}
\end{entry}

\begin{entry}{纪律}{ji4lv4}{6,9}{⽷、⼻}[HSK 4]
  \definition{s.}{disciplina; código de conduta que cada membro da vida coletiva deve observar}
\end{entry}

\begin{entry}{纪念}{ji4nian4}{6,8}{⽷、⼼}[HSK 3]
  \definition[个]{s.}{lembrança | aniversário (comemoração)}
  \definition{v.}{comemorar}
\end{entry}

\begin{entry}{技俩}{ji4liang3}{7,9}{⼿、⼈}
  \definition{s.}{truque | estratagema | ardil | esquema | estratégia | tática}
\end{entry}

\begin{entry}{技能}{ji4 neng2}{7,10}{⼿、⾁}[HSK 5]
  \definition[种,项]{s.}{habilidade técnica; domínio de uma habilidade ou técnica; capacidade de adquirir e aplicar conhecimento}
\end{entry}

\begin{entry}{技巧}{ji4qiao3}{7,5}{⼿、⼯}[HSK 4]
  \definition{s.}{habilidade; técnica; habilidades engenhosas expressas em artes, artesanato, esportes, etc.}
\end{entry}

\begin{entry}{技术}{ji4shu4}{7,5}{⼿、⽊}[HSK 3]
  \definition[种,门,项]{s.}{habilidade; técnica; tecnologia}
\end{entry}

\begin{entry}{系}{ji4}{7}{⽷}
  \definition{v.}{amarrar; prender; abotoar}
  \seealsoref{系}{xi4}
\end{entry}

\begin{entry}{季}{ji4}{8}{⼦}[HSK 4]
  \definition*{s.}{sobrenome Ji}
  \definition{s.}{estação; o ano é dividido em quatro estações, primavera, verão, outono e inverno, e uma estação dura três meses | temporada | o fim de uma era | o último mês de uma temporada | o quarto ou mais novo entre irmãos; último na ordem de precedência}
\end{entry}

\begin{entry}{季度}{ji4du4}{8,9}{⼦、⼴}[HSK 4]
  \definition[个]{s.}{trimestre; período de tempo trimestral}
\end{entry}

\begin{entry}{季节}{ji4jie2}{8,5}{⼦、⾋}[HSK 4]
  \definition[个]{s.}{estação (clima); um período característico do ano}
\end{entry}

\begin{entry}{既}{ji4}{9}{⽆}[HSK 4]
  \definition*{s.}{sobrenome Ji}
  \definition{adv.}{já}
  \definition{conj.}{desde; como; agora que | assim como; e também; ambos\dots e\dots; usado em conjunto com advérbios como ``且、又、也'' para indicar uma combinação de ambas as situações}
  \seealsoref{且}{qie3}
  \seealsoref{也}{ye3}
  \seealsoref{又}{you4}
\end{entry}

\begin{entry}{既不……又不……}{ji4bu4 you4bu4}{9,4,2,4}{⽆、⼀、⼜、⼀}
  \definition{conj.}{nem mesmo\dots}
\end{entry}

\begin{entry}{既然}{ji4ran2}{9,12}{⽆、⽕}[HSK 4]
  \definition{conj.}{como; desde; agora que; usado na primeira metade de uma frase, muitas vezes repetido na segunda metade pelos advérbios ``就、也、还'' para indicar que a premissa é primeiro declarada e depois inferida}
  \seealsoref{还}{hai2}
  \seealsoref{就}{jiu4}
  \seealsoref{也}{ye3}
\end{entry}

\begin{entry}{既又}{ji4you4}{9,2}{⽆、⼜}
  \definition{conj.}{desde | como | agora isso | os dois e | assim como}
\end{entry}

\begin{entry}{继承}{ji4cheng2}{10,8}{⽷、⼿}[HSK 5]
  \definition{v.}{herdar (o patrimônio de uma pessoa falecida, etc.) de acordo com a lei | continuar; geralmente se refere à aceitação do estilo, da cultura, do conhecimento, etc., daqueles que nos precederam | continuar; os descendentes continuam o trabalho deixado por seus antecessores.}
\end{entry}

\begin{entry}{继续}{ji4xu4}{10,11}{⽷、⽷}[HSK 3]
  \definition{v.}{continuar; prosseguir}
\end{entry}

\begin{entry}{寂寥}{ji4liao2}{11,14}{⼧、⼧}
  \definition{s.}{solidão | vasto e vazio | quieto e desolado (literário)}
\end{entry}

\begin{entry}{寂寞}{ji4mo4}{11,13}{⼧、⼧}
  \definition{adj.}{sozinho | solitário | (de um lugar) silencioso}
\end{entry}

\begin{entry}{寄}{ji4}{11}{⼧}[HSK 4]
  \definition{adj.}{adotado; fomentado; promovido}
  \definition{v.}{enviar; postar; remeter | confiar; depositar; colocar | depender de; apegar-se a}
\end{entry}

\begin{entry}{寄存}{ji4cun2}{11,6}{⼧、⼦}
  \definition{v.}{depositar | deixar algo com alguém | armazenar}
\end{entry}

\begin{entry}{寄递}{ji4di4}{11,10}{⼧、⾡}
  \definition{s.}{entrega de correspondência}
\end{entry}

\begin{entry}{寄放}{ji4fang4}{11,8}{⼧、⽅}
  \definition{v.}{deixar algo com alguém}
\end{entry}

\begin{entry}{寄居}{ji4ju1}{11,8}{⼧、⼫}
  \definition{s.}{morar longe de casa}
\end{entry}

\begin{entry}{寄卖}{ji4mai4}{11,8}{⼧、⼗}
  \definition{v.}{consignar para venda}
\end{entry}

\begin{entry}{寄生}{ji4sheng1}{11,5}{⼧、⽣}
  \definition{s.}{parasita | parasitismo}
  \definition{v.}{viver tirando vantagem dos outros | viver dentro ou sobre outro organismo como um parasita}
\end{entry}

\begin{entry}{寄生生活}{ji4sheng1sheng1huo2}{11,5,5,9}{⼧、⽣、⽣、⽔}
  \definition{s.}{parasitismo | vida parasitária}
\end{entry}

\begin{entry}{寄售}{ji4shou4}{11,11}{⼧、⼝}
  \definition{v.}{venda em consignação}
\end{entry}

\begin{entry}{寄送}{ji4song4}{11,9}{⼧、⾡}
  \definition{v.}{enviar | transmitir}
\end{entry}

\begin{entry}{寄宿}{ji4su4}{11,11}{⼧、⼧}
  \definition{s.}{embarque}
  \definition{v.}{embarcar}
\end{entry}

\begin{entry}{寄托}{ji4tuo1}{11,6}{⼧、⼿}
  \definition{v.}{investir (sua esperança, energia, etc.) em algo | confiar (a alguém) | colocar (a esperança, a energia, etc.) em}
\end{entry}

\begin{entry}{寄望}{ji4wang4}{11,11}{⼧、⽉}
  \definition{v.}{depositar esperanças em}
\end{entry}

\begin{entry}{寄养}{ji4yang3}{11,9}{⼧、⼋}
  \definition{v.}{embarcar | promover | colocar sob os cuidados de alguém (uma criança, animal de estimação, etc.)}
\end{entry}

\begin{entry}{寄予}{ji4yu3}{11,4}{⼧、⼅}
  \definition{v.}{expressar | colocar (esperança, importância, etc.) em | mostrar}
\end{entry}

\begin{entry}{旣}{ji4}{11}{⽆}
  \variantof{既}
\end{entry}

\begin{entry}{加}{jia1}{5}{⼒}[HSK 2]
  \definition*{s.}{Canadá, abreviação de~加拿大 | sobrenome Jia}
  \seeref{加拿大}{jia1na2da4}
\end{entry}

\begin{entry}{加班}{jia1ban1}{5,10}{⼒、⽟}[HSK 4]
  \definition{v.+compl.}{fazer horas extras; trabalhar horas extras}
\end{entry}

\begin{entry}{加工}{jia1gong1}{5,3}{⼒、⼯}[HSK 3]
  \definition{s.}{processo | trabalho (de uma máquina)}
  \definition{v.}{processar | melhorar; polir}
\end{entry}

\begin{entry}{加快}{jia1 kuai4}{5,7}{⼒、⼼}[HSK 3]
  \definition{v.}{acelerar; aumentar a velocidade}
\end{entry}

\begin{entry}{加拿大}{jia1na2da4}{5,10,3}{⼒、⼿、⼤}
  \definition{s.}{Canadá}
\end{entry}

\begin{entry}{加拿大人}{jia1na2da4ren2}{5,10,3,2}{⼒、⼿、⼤、⼈}
  \definition{s.}{canadense | pessoa ou povo do Canadá}
\end{entry}

\begin{entry}{加强}{jia1 qiang2}{5,12}{⼒、⼸}[HSK 3]
  \definition{v.}{fortalecer; engrandecer; reforçar}
\end{entry}

\begin{entry}{加热}{jia1 re4}{5,10}{⼒、⽕}[HSK 5]
  \definition{v.}{aquecer; esquentar; aumentar a temperatura de um objeto}
\end{entry}

\begin{entry}{加入}{jia1ru4}{5,2}{⼒、⼊}[HSK 4]
  \definition{v.}{juntar-se; unir-se; aderir a; tornar-se um membro de uma organização, grupo | adicionar; colocar em}
\end{entry}

\begin{entry}{加上}{jia1 shang4}{5,3}{⼒、⼀}[HSK 5]
  \definition{conj.}{além disso; em adição}
  \definition{v.}{adicionar; acrescentar; dar; aumentar}
\end{entry}

\begin{entry}{加速}{jia1 su4}{5,10}{⼒、⾡}[HSK 5]
  \definition{v.}{acelerar; agilizar}
\end{entry}

\begin{entry}{加速度}{jia1su4du4}{5,10,9}{⼒、⾡、⼴}
  \definition{s.}{aceleração}
\end{entry}

\begin{entry}{加以}{jia1 yi3}{5,4}{⼒、⼈}[HSK 5]
  \definition{conj.}{além disso; em adição; indica outras razões ou condições}
  \definition{v.}{usado na frente de palavras dissilábicas para indicar como um objeto mencionado deve ser tratado ou descartado | usado antes de um verbo polifônico ou de um substantivo formado a partir de um verbo para indicar como tratar ou lidar com o que foi mencionado anteriormente}
\end{entry}

\begin{entry}{加油}{jia1you2}{5,8}{⼒、⽔}[HSK 2]
  \definition{v.+compl.}{lubrificar | encher o tanque de combustível | fazer um esforço maior | fazer um esforço extra}
\end{entry}

\begin{entry}{加油站}{jia1you2zhan4}{5,8,10}{⼒、⽔、⽴}[HSK 4]
  \definition[个,家]{s.}{posto de gasolina; posto de combustível; postos de abastecimento para venda a varejo de gasolina e óleo para carros e outros veículos motorizados}
\end{entry}

\begin{entry}{夹}{jia1}{6}{⼤}[HSK 5]
  \definition{s.}{clipe, grampo, pasta, etc.}
  \definition{v.}{colocar no meio; pressionar de ambos os lados; aplicar força ou ação ao mesmo objeto de ambos os lados ao mesmo tempo | misturar; mesclar; intercalar}
  \seeref{夹}{ga1}
  \seeref{夹}{jia2}
\end{entry}

\begin{entry}{夹肢窝}{jia1 zhi1 wo1}{6,8,12}{⼤、⾁、⽳}
  \definition{s.}{axila; sovaco; também escrito como ``胳肢窝''}
  \seealsoref{胳肢窝}{ga1 zhi1 wo1}
\end{entry}

\begin{entry}{家}{jia1}{10}{⼧}[HSK 1,2]
  \definition{clas.}{para famílias ou empresas}
  \definition{pron.}{(educado) meu (irmã, tio, etc.)}
  \definition[个]{s.}{casa | família}
  \definition{suf.}{sufixo substantivo para designar um especialista em alguma atividade, como um músico ou revolucionário, para designar uma profissão como em -eiro, -ista}
\end{entry}

\begin{entry}{家伙}{jia1huo5}{10,6}{⼧、⼈}
  \definition{s.}{prato, implemento ou móvel doméstico | animal doméstico | (coloquial) o cara | indivíduo | arma}
\end{entry}

\begin{entry}{家具}{jia1ju4}{10,8}{⼧、⼋}[HSK 3]
  \definition[件,套]{s.}{móveis; mobiliário de casa}
\end{entry}

\begin{entry}{家俱}{jia1ju4}{10,10}{⼧、⼈}
  \variantof{家具}
\end{entry}

\begin{entry}{家里}{jia1 li3}{10,7}{⼧、⾥}[HSK 1]
  \definition{adv.}{em casa}
\end{entry}

\begin{entry}{家人}{jia1ren2}{10,2}{⼧、⼈}[HSK 1]
  \definition{s.}{(a) família | membro da família}
\end{entry}

\begin{entry}{家属}{jia1shu3}{10,12}{⼧、⼫}[HSK 3]
  \definition{s.}{membros da família; dependentes (familiares)}
\end{entry}

\begin{entry}{家庭}{jia1ting2}{10,9}{⼧、⼴}[HSK 2]
  \definition[个,户]{s.}{família}
\end{entry}

\begin{entry}{家务}{jia1wu4}{10,5}{⼧、⼒}[HSK 4]
  \definition[堆,次,件]{s.}{trabalho doméstico; tarefas domésticas}
\end{entry}

\begin{entry}{家乡}{jia1xiang1}{10,3}{⼧、⼄}[HSK 3]
  \definition[个]{s.}{cidade natal}
\end{entry}

\begin{entry}{家长}{jia1 zhang3}{10,4}{⼧、⾧}[HSK 2]
  \definition[位,名,个]{s.}{pais | patriarca | guardião}
\end{entry}

\begin{entry}{傢具}{jia1ju4}{12,8}{⼈、⼋}
  \variantof{家具}
\end{entry}

\begin{entry}{嘉年华}{jia1nian2hua2}{14,6,6}{⼝、⼲、⼗}
  \definition{s.}{(empréstimo linguístico) carnaval}
\end{entry}

\begin{entry}{夹}{jia2}{6}{⼤}
  \definition{adj.}{forrado; com camada dupla; duas camadas (roupas, colchas, etc.) |
pinçado; voz deliberadamente engraçada}
  \seeref{夹}{ga1}
  \seeref{夹}{jia1}
\end{entry}

\begin{entry}{甲}{jia3}{5}{⽥}[HSK 5]
  \definition*{s.}{sobrenome Jia}
  \definition{s.}{alfa; primeiro lugar; o primeiro dos caules celestiais, geralmente usado para indicar o primeiro em ordem ou classificação | concha; carapaça; crustáceos | unha; crostas queratinosas nos dedos das mãos e dos pés | armadura; equipamento de proteção feito de metal | unidade de administração civil composta por 10 residências | uma palavra substituta para uma pessoa ou coisa indefinida; usado como pronome |}
  \definition{v.}{ocupar o primeiro lugar; ser melhor do que}
\end{entry}

\begin{entry}{甲骨文}{jia3gu3wen2}{5,9,4}{⽥、⾻、⽂}
  \definition{s.}{escrituras de oráculos | inscrições em ossos de oráculos (forma original de escritura chinesa)}
\end{entry}

\begin{entry}{假}{jia3}{11}{⼈}[HSK 2]
  \definition{adj.}{falso | artificial}
  \definition{v.}{emprestar}
  \seeref{假}{jia4}
\end{entry}

\begin{entry}{假的}{jia3de5}{11,8}{⼈、⽩}
  \definition{adj.}{falso | substituto | simulado}
\end{entry}

\begin{entry}{假如}{jia3ru2}{11,6}{⼈、⼥}[HSK 4]
  \definition{conj.}{se; supondo; no caso}
\end{entry}

\begin{entry}{假声}{jia3sheng1}{11,7}{⼈、⼠}
  \definition{s.}{falsete}
  \seealsoref{真声}{zhen1sheng1}
\end{entry}

\begin{entry}{假使}{jia3shi3}{11,8}{⼈、⼈}
  \definition{conj.}{se | supondo | em caso}
\end{entry}

\begin{entry}{假证件}{jia3zheng4jian4}{11,7,6}{⼈、⾔、⼈}
  \definition{s.}{documentos falsos}
\end{entry}

\begin{entry}{价}{jia4}{6}{⼈}[HSK 5]
  \definition{s.}{preço | valor; (figurativo) valores (éticos, culturais etc.) | (química) valência}
\end{entry}

\begin{entry}{价格}{jia4ge2}{6,10}{⼈、⽊}[HSK 3]
  \definition[个]{s.}{preço; tarifa}
\end{entry}

\begin{entry}{价钱}{jia4 qian2}{6,10}{⼈、⾦}[HSK 3]
  \definition[些]{s.}{preço}
\end{entry}

\begin{entry}{价值}{jia4zhi2}{6,10}{⼈、⼈}[HSK 3]
  \definition{s.}{valor}
\end{entry}

\begin{entry}{驾驶}{jia4shi3}{8,8}{⾺、⾺}[HSK 5]
  \definition{v.}{dirigir; pilotar; conduzir; guiar; operar (um carro, navio, avião, trator, etc.) para fazê-lo mover}
\end{entry}

\begin{entry}{驾照}{jia4 zhao4}{8,13}{⾺、⽕}[HSK 5]
  \definition[本,张]{s.}{carteira de motorista}
\end{entry}

\begin{entry}{架}{jia4}{9}{⽊}[HSK 3]
  \definition{clas.}{para coisas com pilares ou componentes mecânicos | quadrado (usado para montanhas)}
  \definition{s.}{quadro; prateleira; suporte | briga; discussão}
  \definition{v.}{colocar para cima; erigir | afastar; resistir | suportar; ajudar | sequestrar; levar alguém embora à força}
\end{entry}

\begin{entry}{架式}{jia4shi5}{9,6}{⽊、⼷}
  \variantof{架势}
\end{entry}

\begin{entry}{架势}{jia4shi5}{9,8}{⽊、⼒}
  \definition{s.}{postura | atitude | posição (sobre um assunto, etc.)}
\end{entry}

\begin{entry}{假}{jia4}{11}{⼈}
  \definition{s.}{férias}
  \seeref{假}{jia3}
\end{entry}

\begin{entry}{假期}{jia4 qi1}{11,12}{⼈、⽉}[HSK 2]
  \definition[个]{s.}{férias | feriados | período de licença}
\end{entry}

\begin{entry}{奸夫}{jian1fu1}{6,4}{⼥、⼤}
  \definition{s.}{homem adúltero}
\end{entry}

\begin{entry}{坚持}{jian1chi2}{7,9}{⼟、⼿}[HSK 3]
  \definition{v.}{persistir e; perseverar em; sustentar; insistir em; manter-se fiel a; aderir a}
\end{entry}

\begin{entry}{坚定}{jian1ding4}{7,8}{⼟、⼧}[HSK 5]
  \definition{adj.}{firme; inabalável; inamovível; (posição, opinião, vontade, etc.) firme e estável, inabalável}
  \definition{v.}{fortalecer}
\end{entry}

\begin{entry}{坚固}{jian1gu4}{7,8}{⼟、⼞}[HSK 4]
  \definition{adj.}{firme; sólido; robusto; forte; durável; firmemente unidos e inquebráveis}
\end{entry}

\begin{entry}{坚决}{jian1jue2}{7,6}{⼟、⼎}[HSK 3]
  \definition{adj.}{firme; resoluto}
\end{entry}

\begin{entry}{坚强}{jian1qiang2}{7,12}{⼟、⼸}[HSK 3]
  \definition{adj.}{forte; firme; convicto}
  \definition{v.}{fortalecer; tornar forte}
\end{entry}

\begin{entry}{坚守}{jian1shou3}{7,6}{⼟、⼧}
  \definition{v.}{agarrar-se}
\end{entry}

\begin{entry}{间}{jian1}{7}{⾨}
  \definition{adv.}{entre | dentro de um tempo ou espaço definidos}
  \definition{clas.}{para salas}
  \definition{s.}{sala | seção de uma sala ou espaço lateral entre dois pares de pilares}
  \seeref{间}{jian4}
\end{entry}

\begin{entry}{浅}{jian1}{8}{⽔}
  \definition{adj.}{murmurando, fluindo suavemente, gorgolejando suavemente}
  \definition{s.}{(onomatopéia) som de água em movimento |}
\end{entry}

\begin{entry}{肩}{jian1}{8}{⾁}[HSK 5]
  \definition*{s.}{sobrenome Jian}
  \definition{s.}{ombro; torso}
  \definition{v.}{assumir; empreender; carregar; suportar; suportar um fardo}
\end{entry}

\begin{entry}{肩膀}{jian1bang3}{8,14}{⾁、⾁}
  \definition{s.}{ombro}
\end{entry}

\begin{entry}{艰苦}{jian1ku3}{8,8}{⾉、⾋}[HSK 5]
  \definition{adj.}{duro; resistente; árduo; difícil; condições de trabalho ou de vida ruins que tornam as pessoas miseráveis}
\end{entry}

\begin{entry}{艰难}{jian1nan2}{8,10}{⾉、⾫}[HSK 5]
  \definition{adj.}{duro; árduo; difícil}
\end{entry}

\begin{entry}{兼}{jian1}{10}{⼋}
  \definition{conj.}{e (ocupando dois ou mais cargos (oficiais) ao mesmo tempo)}
\end{entry}

\begin{entry}{监狱}{jian1yu4}{10,9}{⽫、⽝}
  \definition{s.}{prisão}
\end{entry}

\begin{entry}{煎}{jian1}{13}{⽕}
  \definition{v.}{fritar | refogar}
\end{entry}

\begin{entry}{煎饼}{jian1bing3}{13,9}{⽕、⾷}
  \definition[张]{s.}{jianbing, crepe chinês | panqueca}
\end{entry}

\begin{entry}{煎蛋}{jian1dan4}{13,11}{⽕、⾍}
  \definition{s.}{ovos fritos}
\end{entry}

\begin{entry}{俭省}{jian3sheng3}{9,9}{⼈、⽬}
  \definition{adj.}{econômico}
\end{entry}

\begin{entry}{柬埔寨}{jian3pu3zhai4}{9,10,14}{⽊、⼟、⼧}
  \definition*{s.}{Camboja}
\end{entry}

\begin{entry}{捡}{jian3}{10}{⼿}
  \definition{v.}{apanhar | recolher | coletar}
\end{entry}

\begin{entry}{减}{jian3}{11}{⼎}[HSK 4]
  \definition*{s.}{sobrenome Jian}
  \definition{v.}{subtrair; remover uma parte da quantidade original | reduzir; diminuir; cortar}
\end{entry}

\begin{entry}{减肥}{jian3fei2}{11,8}{⼎、⾁}[HSK 4]
  \definition[次]{v.+compl.}{perder peso; dieta, exercícios, medicamentos, massagem, cirurgia, etc., para reduzir o excesso de gordura corporal, de modo que o grau de obesidade seja reduzido}
\end{entry}

\begin{entry}{减轻}{jian3 qing1}{11,9}{⼎、⾞}[HSK 5]
  \definition{v.}{aliviar; remeter; clarear; facilitar; mitigar}
\end{entry}

\begin{entry}{减少}{jian3shao3}{11,4}{⼎、⼩}[HSK 4]
  \definition{v.}{cair; reduzir; diminuir; subtrair uma parte}
\end{entry}

\begin{entry}{剪}{jian3}{11}{⼑}[HSK 5]
  \definition[把]{s.}{tesouras; tesouras de poda; cortadores | pinças; tenazes}
  \definition{v.}{cortar; aparar; tosquiar; cortar (com uma tesoura) | exterminar; eliminar; acabar com}
\end{entry}

\begin{entry}{剪刀}{jian3dao1}{11,2}{⼑、⼑}[HSK 5]
  \definition[把]{s.}{tesoura; instrumento de ferro para cortar tecido, papel, barbante, etc., com duas lâminas interligadas que podem ser abertas e fechadas}
\end{entry}

\begin{entry}{剪子}{jian3 zi5}{11,3}{⼑、⼦}[HSK 5]
  \definition[把]{s.}{cortador; tosquiadeira | tesoura}
\end{entry}

\begin{entry}{检测}{jian3 ce4}{11,9}{⽊、⽔}[HSK 4]
  \definition{v.}{testar; detectar; verificar}
\end{entry}

\begin{entry}{检查}{jian3cha2}{11,9}{⽊、⽊}[HSK 2]
  \definition[次]{s.}{inspeção}
  \definition{v.}{examinar | inspecionar}
\end{entry}

\begin{entry}{检验}{jian3yan4}{11,10}{⽊、⾺}[HSK 5]
  \definition{v.}{testar; examinar; inspecionar}
\end{entry}

\begin{entry}{简单}{jian3dan1}{13,8}{⽵、⼗}[HSK 3]
  \definition{adj.}{simples; descomplicado | comum; lugar-comum | casual; simplificado}
\end{entry}

\begin{entry}{简历}{jian3li4}{13,4}{⽵、⼚}[HSK 4]
  \definition[个,份]{s.}{currículo; \emph{curriculum vitae} (CV); notas biográficas}
\end{entry}

\begin{entry}{简直}{jian3zhi2}{13,8}{⽵、⽬}[HSK 3]
  \definition{adv.}{simplesmente; em tudo; virtualmente}
\end{entry}

\begin{entry}{见}{jian4}{4}{⾒}[HSK 1]
  \definition{s.}{opinião, visão}
  \definition{v.}{ver | entrevistar | encontrar alguém | parecer (ser alguma coisa)}
  \seeref{见}{xian4}
\end{entry}

\begin{entry}{见到}{jian4 dao4}{4,8}{⾒、⼑}[HSK 2]
  \definition{v.}{ver | esbarrar em | encontrar-se com}
\end{entry}

\begin{entry}{见过}{jian4 guo4}{4,6}{⾒、⾡}[HSK 2]
  \definition{s.}{visto (ver)}
\end{entry}

\begin{entry}{见面}{jian4 mian4}{4,9}{⾒、⾯}[HSK 1]
  \definition{v.+compl.}{encontrar-se com alguém | ver alguém face-a-face}
\end{entry}

\begin{entry}{件}{jian4}{6}{⼈}[HSK 2]
  \definition{clas.}{para eventos, coisas, roupas etc.}
  \definition{s.}{item | componente}
\end{entry}

\begin{entry}{间}{jian4}{7}{⾨}[HSK 1]
  \definition{s.}{lacuna}
  \definition{v.}{separar | podar (mudas) | semear descontentamento}
  \seeref{间}{jian1}
\end{entry}

\begin{entry}{间或}{jian4huo4}{7,8}{⾨、⼽}
  \definition{adv.}{às vezes | ocasionalmente | de vez em quando}
\end{entry}

\begin{entry}{间接}{jian4jie1}{7,11}{⾨、⼿}[HSK 5]
  \definition{adj.}{indireto; de segunda mão; em oposição a ``直接''}
  \seealsoref{直接}{zhi2jie1}
\end{entry}

\begin{entry}{建}{jian4}{8}{⼵}[HSK 3]
  \definition*{s.}{sobrenome Jian}
  \definition{v.}{construir; construir; erigir | estabelecer; configurar; fundar}
\end{entry}

\begin{entry}{建成}{jian4 cheng2}{8,6}{⼵、⼽}[HSK 3]
  \definition{v.}{terminar a construção}
\end{entry}

\begin{entry}{建立}{jian4li4}{8,5}{⼵、⽴}[HSK 3]
  \definition{v.}{estabelecer; construir | vir a ser}
\end{entry}

\begin{entry}{建立者}{jian4li4zhe3}{8,5,8}{⼵、⽴、⽼}
  \definition{s.}{fundador}
\end{entry}

\begin{entry}{建设}{jian4she4}{8,6}{⼵、⾔}[HSK 3]
  \definition{s.}{reconstrução; desenvolvimento}
  \definition{v.}{construir}
\end{entry}

\begin{entry}{建设性}{jian4she4xing4}{8,6,8}{⼵、⾔、⼼}
  \definition{adj.}{construtivo}
  \definition{s.}{construtividade}
\end{entry}

\begin{entry}{建设者}{jian4she4zhe3}{8,6,8}{⼵、⾔、⽼}
  \definition{s.}{construtor}
\end{entry}

\begin{entry}{建议}{jian4yi4}{8,5}{⼵、⾔}[HSK 3]
  \definition[个,点,条]{s.}{proposta; sugestão; recomendação}
  \definition{v.}{propor; sugerir; recomendar}
\end{entry}

\begin{entry}{建造}{jian4 zao4}{8,10}{⼵、⾡}[HSK 5]
  \definition{adj.}{indireto; de segunda mão; ter um relacionamento por meio de um terceiro (em oposição a ``直接'')}
  \seealsoref{直接}{zhi2jie1}
\end{entry}

\begin{entry}{建筑}{jian4zhu4}{8,12}{⼵、⽵}[HSK 5]
  \definition[座,幢,排]{s.}{construção; estrutura; edifício; prédio}
  \definition{v.}{construir; erguer; edificar; construir casas, estradas, pontes, etc.}
\end{entry}

\begin{entry}{剑}{jian4}{9}{⼑}
  \definition{clas.}{para golpes de uma espada}
  \definition[口,把]{s.}{espada de dois gumes}
\end{entry}

\begin{entry}{剑客}{jian4ke4}{9,9}{⼑、⼧}
  \definition{s.}{espada | esgrimista, espadachim}
\end{entry}

\begin{entry}{健康}{jian4kang1}{10,11}{⼈、⼴}[HSK 2]
  \definition{adj.}{em forma | saudável | curado}
  \definition{s.}{saúde | físico}
\end{entry}

\begin{entry}{健全}{jian4quan2}{10,6}{⼈、⼊}[HSK 5]
  \definition{adj.}{saudável; íntegro; capaz; apto; robusto e sem mácula | sólido; completo; perfeito}
  \definition{v.}{aperfeiçoar; melhorar; fortalecer; reforçar}
\end{entry}

\begin{entry}{健身}{jian4shen1}{10,7}{⼈、⾝}[HSK 4]
  \definition{s.}{exercício físico | \emph{fitness}}
  \definition{v.+compl.}{exercitar-se; manter a forma; praticar um esporte, especialmente a ginástica, inclusive em aparelhos, para desenvolver força, flexibilidade, aumentar a resistência, melhorar a coordenação e o controle de todas as partes do corpo}
\end{entry}

\begin{entry}{渐渐}{jian4 jian4}{11,11}{⽔、⽔}[HSK 4]
  \definition{adv.}{gradualmente; pouco a pouco; passo a passo; indica um aumento ou diminuição gradual em grau ou quantidade}
\end{entry}

\begin{entry}{键}{jian4}{13}{⾦}[HSK 5]
  \definition[个]{s.}{pino (para máquinas) | tecla (de uma máquina de escrever, piano, etc.) | chave | etapa crucial}
\end{entry}

\begin{entry}{键盘}{jian4pan2}{13,11}{⾦、⽫}[HSK 5]
  \definition[个]{s.}{braço; teclado; cravo; painel de teclas; porta-chaves}
\end{entry}

\begin{entry}{江}{jiang1}{6}{⽔}[HSK 4]
  \definition*{s.}{Rio Changjiang | sobrenome Jiang}
  \definition[条,道]{s.}{rio grande}
\end{entry}

\begin{entry}{江南水乡}{jiang1nan2shui3xiang1}{6,9,4,3}{⽔、⼗、⽔、⼄}
  \definition*{s.}{Vila Aquática de Jiangnan | Cidades Aquáticas}
\end{entry}

\begin{entry}{江水}{jiang1shui3}{6,4}{⽔、⽔}
  \definition{s.}{água do rio}
\end{entry}

\begin{entry}{江西}{jiang1xi1}{6,6}{⽔、⾑}
  \definition*{s.}{Jiangxi}
\end{entry}

\begin{entry}{姜}{jiang1}{9}{⼥}
  \definition*{s.}{sobrenome Jiang}
  \definition{s.}{gengibre}
\end{entry}

\begin{entry}{将}{jiang1}{9}{⼨}[HSK 5]
  \definition*{s.}{sobrenome Jiang}
  \definition{adv.}{estar indo para; parcialmente\dots parcialmente\dots}
  \definition{part.}{expressar uma direção, como ``进来'', ``出去''; usado no meio de verbos e complementos que indicam tendência, como ``进来'', ``出去'' etc.}
  \definition{prep.}{com; por meio de; por | usado da mesma forma que ``把''}
  \definition{v.}{fazer algo; lidar com (um assunto) | dar um cheque-mate | cuidar (da saúde) | incitar alguém a agir; desafiar; estimular | segurar; pegar | colocar; tirar | levar; trazer | dar suporte; dar apoio}
  \seeref{将}{jiang4}
  \seeref{将}{qiang1}
  \seealsoref{把}{ba3}
  \seealsoref{出去}{chu1 qu4}
  \seealsoref{进来}{jin4 lai2}
\end{entry}

\begin{entry}{将近}{jiang1jin4}{9,7}{⼨、⾡}[HSK 3]
  \definition{adv.}{quase}
\end{entry}

\begin{entry}{将来}{jiang1lai2}{9,7}{⼨、⽊}[HSK 3]
  \definition[个]{s.}{futuro}
\end{entry}

\begin{entry}{将要}{jiang1 yao4}{9,9}{⼨、⾑}[HSK 5]
  \definition{adv.}{irá; deverá; estará prestes a; irá a; indica que um ato ou situação ocorre logo em seguida}
\end{entry}

\begin{entry}{讲}{jiang3}{6}{⾔}[HSK 2]
  \definition{v.}{falar (de) | falar (sobre) | relacionar | dizer | contar | explicar | explicitar | elaborar (sobre) | tornar claro |interpretar | discutir | consultar | negociar}
\end{entry}

\begin{entry}{讲话}{jiang3 hua4}{6,8}{⾔、⾔}[HSK 2]
  \definition{s.}{discurso | guia | introdução}
  \definition{v.+compl.}{falar | conversar | abordar}
\end{entry}

\begin{entry}{讲究}{jiang3jiu5}{6,7}{⾔、⽳}[HSK 4]
  \definition{adj.}{requintado; elegante; de bom gosto; exigente com a vida e com outros aspectos, buscando alto nível, qualidade e detalhes}
  \definition{s.}{estudo cuidadoso; algo que merece atenção; elementos e aspectos que merecem atenção especial}
  \definition{v.}{dar ênfase a; ser específico sobre; prestar atenção a}
\end{entry}

\begin{entry}{讲述}{jiang3shu4}{6,8}{⾔、⾡}
  \definition{v.}{falar sobre | narrar | descrever}
\end{entry}

\begin{entry}{讲座}{jiang3zuo4}{6,10}{⾔、⼴}[HSK 4]
  \definition[个]{s.}{palestra; um curso de palestras; a forma de instrução usada para ensinar um determinado assunto ou tópico, geralmente por meio de palestras ao vivo, seriados de rádio ou televisão ou seriados de jornal.}
\end{entry}

\begin{entry}{奖}{jiang3}{9}{⼤}[HSK 4]
  \definition[个,次]{s.}{prêmio; recompensa | elogio; loa}
  \definition{v.}{elogiar; recompensar; recomendar; incentivar}
\end{entry}

\begin{entry}{奖金}{jiang3jin1}{9,8}{⼤、⾦}[HSK 4]
  \definition[个,笔]{s.}{bônus; recompensa; prêmio; prêmio em dinheiro; dinheiro de recompensa, dinheiro dado às pessoas para incentivá-las ou elogiá-las por terem se saído bem em alguma coisa}
\end{entry}

\begin{entry}{奖励}{jiang3li4}{9,7}{⼤、⼒}[HSK 5]
  \definition{s.}{prêmio; recompensa; dinheiro ou honras dadas em troca de elogios ou incentivos}
  \definition{v.}{recompensar; incentivar; encorajar}
\end{entry}

\begin{entry}{奖学金}{jiang3 xue2 jin1}{9,8,8}{⼤、⼦、⾦}[HSK 4]
  \definition[笔]{s.}{bolsa de estudos; exposição; prêmios concedidos por escolas, organizações ou indivíduos a alunos com bom desempenho acadêmico}
\end{entry}

\begin{entry}{匠}{jiang4}{6}{⼕}
  \definition{s.}{artesão}
\end{entry}

\begin{entry}{降}{jiang4}{8}{⾩}[HSK 4]
  \definition*{s.}{sobrenome Jiang}
  \definition{v.}{cair; descer | diminuir; reduzir | nascer}
\end{entry}

\begin{entry}{降低}{jiang4di1}{8,7}{⾩、⼈}[HSK 4]
  \definition{v.}{reduzir; cortar; diminuir; rebaixar; cair; abaixar}
\end{entry}

\begin{entry}{降价}{jiang4 jia4}{8,6}{⾩、⼈}[HSK 4]
  \definition{v.}{ficar mais barato; cortar o preço; reduzir o preço}
\end{entry}

\begin{entry}{降落}{jiang4luo4}{8,12}{⾩、⾋}[HSK 4]
  \definition{v.}{aterrissar; descer; descer do céu}
\end{entry}

\begin{entry}{降温}{jiang4 wen1}{8,12}{⾩、⽔}[HSK 4]
  \definition{v.}{baixar a temperatura (como em uma oficina);  recusar | cair a temperatura | esfriar; resfriar; metáfora para um declínio no entusiasmo ou uma diminuição no ímpeto de algo}
\end{entry}

\begin{entry}{将}{jiang4}{9}{⼨}
  \definition{s.}{general; nome do posto; abaixo de marechal de campo; acima de coronel}
  \definition{v.}{comandar; liderar}
  \seeref{将}{jiang1}
  \seeref{将}{qiang1}
\end{entry}

\begin{entry}{强}{jiang4}{12}{⼸}
  \definition{adj.}{teimoso; inflexível}
  \seeref{强}{qiang2}
  \seeref{强}{qiang3}
\end{entry}

\begin{entry}{酱}{jiang4}{13}{⾣}
  \definition{s.}{pasta grossa de soja fermentada | marinada em pasta de soja | pasta | geléia}
\end{entry}

\begin{entry}{犟}{jiang4}{16}{⽜}
  \variantof{强}
\end{entry}

\begin{entry}{交}{jiao1}{6}{⼇}[HSK 2]
  \definition{v.}{entregar | dar}
\end{entry}

\begin{entry}{交班}{jiao1ban1}{6,10}{⼇、⽟}
  \definition{v.}{passar para o próximo turno de trabalho}
\end{entry}

\begin{entry}{交杯酒}{jiao1bei1jiu3}{6,8,10}{⼇、⽊、⾣}
  \definition{s.}{copo de vinho nupcial}
\end{entry}

\begin{entry}{交叉}{jiao1cha1}{6,3}{⼇、⼜}
  \definition{v.}{cruzar | sobrepor}
\end{entry}

\begin{entry}{交叉点}{jiao1cha1dian3}{6,3,9}{⼇、⼜、⽕}
  \definition{s.}{encruzilhada | cruzamento | junção}
\end{entry}

\begin{entry}{交叉口}{jiao1cha1kou3}{6,3,3}{⼇、⼜、⼝}
  \definition{s.}{intersecção (rodovia)}
\end{entry}

\begin{entry}{交代}{jiao1dai4}{6,5}{⼇、⼈}[HSK 5]
  \definition{v.}{contar; entregar | ordenar; insistir; contar aos outros sobre suas intenções, instruções | contar; admitir}
\end{entry}

\begin{entry}{交叠}{jiao1die2}{6,13}{⼇、⼜}
  \definition{s.}{sobreposição}
\end{entry}

\begin{entry}{交费}{jiao1 fei4}{6,9}{⼇、⾙}[HSK 3]
  \definition{v.}{pagar taxas; pagar uma taxa}
\end{entry}

\begin{entry}{交给}{jiao1 gei3}{6,9}{⼇、⽷}[HSK 2]
  \definition{v.}{entregar algo | dar algo}
\end{entry}

\begin{entry}{交媾}{jiao1gou4}{6,13}{⼇、⼥}
  \definition{v.}{copular | ter relações sexuais}
\end{entry}

\begin{entry}{交换}{jiao1huan4}{6,10}{⼇、⼿}[HSK 4]
  \definition{v.}{trocar; permutar; comutar; intercambiar}
\end{entry}

\begin{entry}{交际}{jiao1ji4}{6,7}{⼇、⾩}[HSK 4]
  \definition{s.}{contato; comunicação; relações sociais; contato interpessoal, socialização}
\end{entry}

\begin{entry}{交界}{jiao1jie4}{6,9}{⼇、⽥}
  \definition{s.}{fronteira comum | limite comum | interface}
\end{entry}

\begin{entry}{交警}{jiao1 jing3}{6,19}{⼇、⾔}[HSK 3]
  \definition{s.}{policial de trânsito, abreviação de 交通警察}
  \seeref{交通警察}{jiao1tong1jing3cha2}
\end{entry}

\begin{entry}{交流}{jiao1liu2}{6,10}{⼇、⽔}[HSK 3]
  \definition{v.}{trocar; permutar; cambiar}
\end{entry}

\begin{entry}{交朋友}{jiao1 peng2 you3}{6,8,4}{⼇、⽉、⼜}[HSK 2]
  \definition{v.}{faça amizade com alguém}
\end{entry}

\begin{entry}{交通}{jiao1tong1}{6,10}{⼇、⾡}[HSK 2]
  \definition{s.}{transporte | tráfego | trânsito | comunicação | conexão}
  \definition{v.}{estar conectado | ser conectado}
\end{entry}

\begin{entry}{交通警察}{jiao1tong1jing3cha2}{6,10,19,14}{⼇、⾡、⾔、⼧}
  \definition{s.}{policial de trânsito}
  \seealsoref{交警}{jiao1 jing3}
\end{entry}

\begin{entry}{交往}{jiao1wang3}{6,8}{⼇、⼻}[HSK 3]
  \definition{v.}{estar em contato com; associar-se a}
\end{entry}

\begin{entry}{交响}{jiao1xiang3}{6,9}{⼇、⼝}
  \definition{s.}{sinfonia}
\end{entry}

\begin{entry}{交易}{jiao1yi4}{6,8}{⼇、⽇}[HSK 3]
  \definition[笔,桩,个,场]{s.}{acordo; negócio; transação}
  \definition{v.}{negociar}
\end{entry}

\begin{entry}{交运}{jiao1yun4}{6,7}{⼇、⾡}
  \definition{v.}{despachar (bagagem em um aeroporto, etc.) | entregar para transporte}
\end{entry}

\begin{entry}{郊区}{jiao1qu1}{8,4}{⾢、⼖}[HSK 5]
  \definition[个,片,块]{s.}{subúrbios; arredores; periferia; área ao redor da cidade que está administrativamente sob a jurisdição da cidade}
\end{entry}

\begin{entry}{胶带}{jiao1 dai4}{10,9}{⾁、⼱}[HSK 5]
  \definition[卷]{s.}{fita adesiva | fita de gravação | correia de borracha}
\end{entry}

\begin{entry}{胶卷}{jiao1juan3}{10,8}{⾁、⼙}
  \definition{s.}{filme | rolo de filme}
\end{entry}

\begin{entry}{胶水}{jiao1shui3}{10,4}{⾁、⽔}[HSK 5]
  \definition[瓶]{s.}{cola; mucilagem; cola líquida}
\end{entry}

\begin{entry}{教}{jiao1}{11}{⽁}[HSK 1]
  \definition{v.}{ensinar | lecionar}
  \seeref{教}{jiao4}
\end{entry}

\begin{entry}{教会}{jiao1hui4}{11,6}{⽁、⼈}
  \definition{v.}{mostrar | ensinar}
  \seeref{教会}{jiao4hui4}
\end{entry}

\begin{entry}{教学}{jiao1xue2}{11,8}{⽁、⼦}
  \definition{v.}{ensinar (como um professor)}
  \seeref{教学}{jiao4 xue2}
\end{entry}

\begin{entry}{焦虑}{jiao1lv4}{12,10}{⽕、⾌}
  \definition{adj.}{ansioso | preocupado | apreensivo}
\end{entry}

\begin{entry}{角}{jiao3}{7}{⾓}[HSK 2][Kangxi 148]
  \definition{clas.}{1 jiao = 10 centavos}
  \definition[个]{s.}{ângulo | esquina | chifre | em forma de chifre}
  \seeref{角}{jue2}
\end{entry}

\begin{entry}{角度}{jiao3du4}{7,9}{⾓、⼴}[HSK 2]
  \definition{s.}{ângulo | ponto de vista}
\end{entry}

\begin{entry}{饺子}{jiao3zi5}{9,3}{⾷、⼦}[HSK 2]
  \definition[个,只]{s.}{jiaozi | bolinhos chineses | bolinho de massa}
\end{entry}

\begin{entry}{脚}{jiao3}{11}{⾁}[HSK 2]
  \definition{clas.}{para chutes}
  \definition[双,只]{s.}{pé | base (de um objeto) | perna (de um animal ou objeto)}
\end{entry}

\begin{entry}{脚步}{jiao3 bu4}{11,7}{⾁、⽌}[HSK 5]
  \definition{s.}{pé; passo; pisada; refere-se ao movimento das pernas ao caminhar | ritmo; passo; distância entre os pés dianteiros e traseiros ao caminhar}
\end{entry}

\begin{entry}{叫}{jiao4}{5}{⼝}[HSK 1,3]
  \definition{adj.}{macho (animal)}
  \definition{prep.}{apresenta a voz ativa na construção passiva.}
  \definition{v.}{chorar; gritar | chamar; cumprimentar | contratar; encomendar | nomear; chamar | pedir; licitar | anunciar-se}
\end{entry}

\begin{entry}{叫作}{jiao4 zuo4}{5,7}{⼝、⼈}[HSK 2]
  \definition{v.}{ser chamado de | ser conhecido como}
\end{entry}

\begin{entry}{校}{jiao4}{10}{⽊}
  \definition{v.}{verificar | comparar | revisar}
  \seeref{校}{xiao4}
\end{entry}

\begin{entry}{较}{jiao4}{10}{⾞}[HSK 3]
  \definition{adj.}{claro; óbvio; marcado}
  \definition{adv.}{comparativamente; relativamente; razoavelmente; bastante; bastante}
  \definition{prep.}{usado para comparar características e graus}
  \definition{v.}{comparar | disputar}
\end{entry}

\begin{entry}{敎}{jiao4}{11}{⽁}
  \variantof{教}
\end{entry}

\begin{entry}{教}{jiao4}{11}{⽁}
  \definition*{s.}{sobrenome Jiao}
  \definition{s.}{religião | ensinamento}
  \definition{v.}{causar | como fazer algo | contar (explicar como fazer algo)}
  \seeref{教}{jiao1}
\end{entry}

\begin{entry}{教材}{jiao4cai2}{11,7}{⽁、⽊}[HSK 3]
  \definition[本,套]{s.}{livro didático; material didático}
\end{entry}

\begin{entry}{教导}{jiao4dao3}{11,6}{⽁、⼨}
  \definition{s.}{instrução | orientação | ensino}
  \definition{v.}{instruir | orientar | ensinar}
\end{entry}

\begin{entry}{教官}{jiao4guan1}{11,8}{⽁、⼧}
  \definition{s.}{instrutor militar}
\end{entry}

\begin{entry}{教会}{jiao4hui4}{11,6}{⽁、⼈}
  \definition{s.}{igreja cristã}
  \seeref{教会}{jiao1hui4}
\end{entry}

\begin{entry}{教练}{jiao4lian4}{11,8}{⽁、⽷}[HSK 3]
  \definition[个,位,名]{s.}{instrutor; treinador (esportes)}
  \definition{v.}{treinar}
\end{entry}

\begin{entry}{教师}{jiao4 shi1}{11,6}{⽁、⼱}[HSK 2]
  \definition[个]{s.}{professor | mestre}
\end{entry}

\begin{entry}{教室}{jiao4shi4}{11,9}{⽁、⼧}[HSK 2]
  \definition[间]{s.}{sala de aula}
\end{entry}

\begin{entry}{教授}{jiao4shou4}{11,11}{⽁、⼿}[HSK 4]
  \definition[个,位]{s.}{professor (universitário)}
  \definition{v.}{ensinar; instruir; dar aulas; dar palestras}
\end{entry}

\begin{entry}{教堂}{jiao4tang2}{11,11}{⽁、⼟}
  \definition[间]{s.}{igreja | capela}
\end{entry}

\begin{entry}{教学}{jiao4 xue2}{11,8}{⽁、⼦}[HSK 2]
  \definition[次]{s.}{ensino | instrução}
  \seeref{教学}{jiao1xue2}
\end{entry}

\begin{entry}{教学楼}{jiao4 xue2 lou2}{11,8,13}{⽁、⼦、⽊}[HSK 1]
  \definition{s.}{edifício de salas de aula}
\end{entry}

\begin{entry}{教训}{jiao4xun4}{11,5}{⽁、⾔}[HSK 4]
  \definition{s.}{moral; lição}
  \definition{v.}{repreender; educar; ensinar uma lição a alguém; dar uma bronca em alguém; dar um sermão em alguém (por ter cometido um erro, etc.)}
\end{entry}

\begin{entry}{教育}{jiao4yu4}{11,8}{⽁、⾁}[HSK 2]
  \definition{s.}{educação}
  \definition{v.}{ensinar | educar}
\end{entry}

\begin{entry}{教长}{jiao4zhang3}{11,4}{⽁、⾧}
  \definition{s.}{imã (Islã) | mulá}
\end{entry}

\begin{entry}{阶段}{jie1duan4}{6,9}{⾩、⽎}[HSK 4]
  \definition{s.}{estágio; fase; período; bancada; gradação}
\end{entry}

\begin{entry}{皆}{jie1}{9}{⽩}
  \definition{adv.}{todos | em todos os casos}
\end{entry}

\begin{entry}{结}{jie1}{9}{⽷}
  \definition{v.}{dar (frutos); formar (sementes); produzir frutos ou sementes (uma planta)}
  \seeref{结}{jie2}
\end{entry}

\begin{entry}{结果}{jie1guo3}{9,8}{⽷、⽊}
  \definition{v.}{dar frutos}
  \seeref{结果}{jie2guo3}
\end{entry}

\begin{entry}{结实}{jie1shi5}{9,8}{⽷、⼧}[HSK 3]
  \definition{adj.}{sólido; resistente; durável | forte; resistente; robusto}
\end{entry}

\begin{entry}{接}{jie1}{11}{⼿}[HSK 2]
  \definition{v.}{ir buscar (alguém) |  ir ao encontro de (alguém) | receber}
\end{entry}

\begin{entry}{接班人}{jie1ban1ren2}{11,10,2}{⼿、⽟、⼈}
  \definition{s.}{sucessor}
\end{entry}

\begin{entry}{接触}{jie1chu4}{11,13}{⼿、⾓}[HSK 5]
  \definition{v.}{entrar em contato com | entrar em contato; tocar; interagir | engajar; o termo militar refere-se a fogo cruzado}
\end{entry}

\begin{entry}{接待}{jie1dai4}{11,9}{⼿、⼻}[HSK 3]
  \definition{v.}{receber (alguém); acolher; recepcionar}
\end{entry}

\begin{entry}{接到}{jie1 dao4}{11,8}{⼿、⼑}[HSK 2]
  \definition{v.}{receber (carta, etc.)}
\end{entry}

\begin{entry}{接(电话)}{jie1(dian4hua4)}{11,5,8}{⼿、⽥、⾔}
  \definition{v.}{atender (o telefone) | receber (uma ligação telefônica)}
\end{entry}

\begin{entry}{接近}{jie1jin4}{11,7}{⼿、⾡}[HSK 3]
  \definition{adj.}{perto; próximo}
  \definition{v.}{estar perto de; aproximar; aproximar-se}
\end{entry}

\begin{entry}{接连}{jie1lian2}{11,7}{⼿、⾡}[HSK 5]
  \definition{adv.}{no final; em sucessão; em uma fileira; um após o outro; seguindo o anterior; continuando}
\end{entry}

\begin{entry}{接受}{jie1shou4}{11,8}{⼿、⼜}[HSK 2]
  \definition{v.}{aceitar | concordar}
\end{entry}

\begin{entry}{接下来}{jie1 xia4 lai2}{11,3,7}{⼿、⼀、⽊}[HSK 2]
  \definition{expr.}{próximo | seguinte | aceitar}
\end{entry}

\begin{entry}{接着}{jie1zhe5}{11,11}{⼿、⽬}[HSK 2]
  \definition{adv.}{por sua vez | com um seguindo o outro}
  \definition{v.}{seguir | prosseguir | continuar | prosseguir | pegar}
\end{entry}

\begin{entry}{街}{jie1}{12}{⾏}[HSK 2]
  \definition[条]{s.}{rua}
\end{entry}

\begin{entry}{街道}{jie1dao4}{12,12}{⾏、⾡}[HSK 4]
  \definition[条]{s.}{caminho; rua; estrada; via pública com casas em ambos os lados, relativamente larga | escritório do subdistrito; tipo de organização responsável por gerenciar determinados aspectos da rua}
\end{entry}

\begin{entry}{街舞}{jie1wu3}{12,14}{⾏、⾇}
  \definition{s.}{dança de rua, \emph{street dance} (por exemplo, \emph{breakdance})}
\end{entry}

\begin{entry}{节}{jie2}{5}{⾋}[HSK 2]
  \definition*{s.}{sobrenome Jie}
  \definition{clas.}{para nós, seções, comprimentos}
  \definition{s.}{junta | botão | nó | divisão | parte | festival | feriado | item | integridade moral | castidade}
\end{entry}

\begin{entry}{节目}{jie2mu4}{5,5}{⾋、⽬}[HSK 2]
  \definition{s.}{programa | item (em um programa)}
\end{entry}

\begin{entry}{节日}{jie2ri4}{5,4}{⾋、⽇}[HSK 2]
  \definition[个]{s.}{festival | feriado}
\end{entry}

\begin{entry}{节省}{jie2sheng3}{5,9}{⾋、⽬}[HSK 4]
  \definition{adj.}{econômico; parcimonioso}
  \definition{v.}{economizar; conservar; usar com moderação; reduzir; eliminar ou minimizar o esgotamento de itens potencialmente esgotáveis}
\end{entry}

\begin{entry}{节约}{jie2yue1}{5,6}{⾋、⽷}[HSK 3]
  \definition{adj.}{econômico}
  \definition{v.}{guardar; economizar}
\end{entry}

\begin{entry}{节奏}{jie2zou4}{5,9}{⾋、⼤}
  \definition{s.}{ritmo | cadência | batida}
\end{entry}

\begin{entry}{结}{jie2}{9}{⽷}[HSK 4]
  \definition*{s.}{sobrenome Jie}
  \definition{s.}{nó | declaração juramentada; garantia por escrito; documento que, antigamente, significava um reconhecimento de encerramento ou uma garantia de responsabilidade}
  \definition{v.}{amarrar; tricotar; dar nó; tecer | formar; forjar; cimentar; solidificar | resolver; concluir | combinar; formar um relacionamento}
  \seeref{结}{jie1}
\end{entry}

\begin{entry}{结构}{jie2gou4}{9,8}{⽷、⽊}[HSK 4]
  \definition[个,座]{s.}{estrutura; composição; construção; formação; constituição; tecido; forma; sistematização; mecânica; organização | arquitetura; estrutura; construção; construção de partes de edifícios com suporte de carga ou com carga externa | textura (geológico)}
\end{entry}

\begin{entry}{结果}{jie2guo3}{9,8}{⽷、⽊}[HSK 2]
  \definition{s.}{resultado | conclusão}
  \definition{v.}{despachar | matar}
  \seeref{结果}{jie1guo3}
\end{entry}

\begin{entry}{结合}{jie2he2}{9,6}{⽷、⼝}[HSK 3]
  \definition{v.}{ligar; unir; combinar; integrar | casar-se; unir-se em matrimônio}
\end{entry}

\begin{entry}{结婚}{jie2hun1}{9,11}{⽷、⼥}[HSK 3]
  \definition{v.+compl.}{casar; casar-se}
\end{entry}

\begin{entry}{结婚礼服}{jie2hun1 li3 fu2}{9,11,5,8}{⽷、⼥、⽰、⽉}
  \definition{s.}{vestido de casamento}
\end{entry}

\begin{entry}{结局}{jie2ju2}{9,7}{⽷、⼫}
  \definition{s.}{conclusão | fim | final}
\end{entry}

\begin{entry}{结论}{jie2lun4}{9,6}{⽷、⾔}[HSK 4]
  \definition[个]{s.}{conclusão; palavra final sobre uma pessoa ou coisa após investigação e pesquisa | veredito; julgamento deduzido de premissas também é chamado de conclusão}
\end{entry}

\begin{entry}{结社自由}{jie2she4zi4you2}{9,7,6,5}{⽷、⽰、⾃、⽥}
  \definition{s.}{(constitucional) liberdade de associação}
\end{entry}

\begin{entry}{结束}{jie2shu4}{9,7}{⽷、⽊}[HSK 3]
  \definition{v.}{finalizar; fechar; terminar; concluir; encerrar}
\end{entry}

\begin{entry}{结束辩论}{jie2shu4 bian4 lun4}{9,7,16,6}{⽷、⽊、⾟、⾔}
  \definition{s.}{debate de encerramento}
\end{entry}

\begin{entry}{结束工作}{jie2shu4gong1zuo4}{9,7,3,7}{⽷、⽊、⼯、⼈}
  \definition{s.}{trabalho final}
  \definition{v.}{terminar o trabalho}
\end{entry}

\begin{entry}{结束剂}{jie2shu4 ji4}{9,7,8}{⽷、⽊、⼑}
  \definition{s.}{finalizador}
\end{entry}

\begin{entry}{结束区}{jie2shu4 qu1}{9,7,4}{⽷、⽊、⼖}
  \definition{s.}{zona final}
\end{entry}

\begin{entry}{结束文本}{jie2shu4 wen2ben3}{9,7,4,5}{⽷、⽊、⽂、⽊}
  \definition{s.}{texto final}
\end{entry}

\begin{entry}{结束语}{jie2shu4yu3}{9,7,9}{⽷、⽊、⾔}
  \definition{s.}{conclusões finais | considerações finais}
\end{entry}

\begin{entry}{捷径}{jie2jing4}{11,8}{⼿、⼻}
  \definition{s.}{atalho}
\end{entry}

\begin{entry}{姐}{jie3}{8}{⼥}[HSK 1]
  \definition{s.}{irmã mais velha | um termo geral para mulheres jovens}
  \seeref{姐姐}{jie3 jie5}
\end{entry}

\begin{entry}{姐夫}{jie3fu5}{8,4}{⼥、⼤}
  \definition{s.}{marido da irmã mais velha}
\end{entry}

\begin{entry}{姐姐}{jie3 jie5}{8,8}{⼥、⼥}[HSK 1]
  \definition[个]{s.}{irmã mais velha}
\end{entry}

\begin{entry}{姐妹}{jie3 mei4}{8,8}{⼥、⼥}[HSK 4]
  \definition[个]{s.}{irmãs}
\end{entry}

\begin{entry}{解除}{jie3chu2}{13,9}{⾓、⾩}[HSK 5]
  \definition{v.}{remover; aliviar; livrar-se de; eliminar}
\end{entry}

\begin{entry}{解放}{jie3fang4}{13,8}{⾓、⽅}[HSK 5]
  \definition{v.}{libertar; emancipar; eliminar as restrições para permitir o desenvolvimento da liberdade}
\end{entry}

\begin{entry}{解雇}{jie3gu4}{13,12}{⾓、⾫}
  \definition{v.}{demitir}
\end{entry}

\begin{entry}{解救}{jie3jiu4}{13,11}{⾓、⽁}
  \definition{v.}{resgatar | ajudar a sair de dificuldades | salvar a situação}
\end{entry}

\begin{entry}{解决}{jie3jue2}{13,6}{⾓、⼎}[HSK 3]
  \definition{v.}{solucionar; resolver; liquidar | acabar com; descartar}
\end{entry}

\begin{entry}{解开}{jie3 kai1}{13,4}{⾓、⼶}[HSK 3]
  \definition{v.}{desatar; desfazer; desamarrar; desabotoar}
\end{entry}

\begin{entry}{解释}{jie3shi4}{13,12}{⾓、⾤}[HSK 4]
  \definition{v.}{explicar; expor; interpretar | analisar; explicaro significado, razões, justificativas, etc.}
\end{entry}

\begin{entry}{解压}{jie3ya1}{13,6}{⾓、⼚}
  \definition{v.}{aliviar o estresse | (computação) descomprimir}
\end{entry}

\begin{entry}{介绍}{jie4shao4}{4,8}{⼈、⽷}[HSK 1]
  \definition{s.}{introdução | apresentação}
  \definition{v.}{fazer uma apresentação | apresentar (alguém para alguém) | apresentar (alguém para um emprego, etc.)}
\end{entry}

\begin{entry}{戒}{jie4}{7}{⼽}[HSK 5]
  \definition[个]{s.}{advertência; exortação | disciplina monástica budista; preceitos budistas | anel (dedo)}
  \definition{v.}{proteger-se contra; estar preparado; estar atento | advertir; exortar; admoestar | abandonar; parar; desistir; desistir (de um hábito ruim)}
\end{entry}

\begin{entry}{芥}{jie4}{7}{⾋}
  \definition{s.}{mostarda}
  \seeref{芥}{gai4}
\end{entry}

\begin{entry}{芥兰}{jie4lan2}{7,5}{⾋、⼋}
  \definition{s.}{couve}
\end{entry}

\begin{entry}{届}{jie4}{8}{⼫}[HSK 5]
  \definition{clas.}{sessões (de uma conferência); anos (de graduação); quantificador, ligeiramente equivalente a ``次'', usado para reuniões regulares ou turmas de formandos, etc.}
  \definition{v.}{vencer o prazo}
  \seealsoref{次}{ci4}
\end{entry}

\begin{entry}{界碑}{jie4bei1}{9,13}{⽥、⽯}
  \definition{s.}{marco de fronteira}
\end{entry}

\begin{entry}{借}{jie4}{10}{⼈}[HSK 2]
  \definition{adv.}{por meio de}
  \definition{v.}{pedir emprestado | emprestar | aproveitar (uma oportunidade)}
\end{entry}

\begin{entry}{借书证}{jie4shu1zheng4}{10,4,7}{⼈、⼄、⾔}
  \definition{s.}{cartão de biblioteca | (literalmente) cartão para pedir emprestado livros}
\end{entry}

\begin{entry}{今后}{jin1 hou4}{4,6}{⼈、⼝}[HSK 2]
  \definition{s.}{de agora em diante | daqui em diante | no futuro}
\end{entry}

\begin{entry}{今年}{jin1 nian2}{4,6}{⼈、⼲}[HSK 1]
  \definition{adv.}{este ano}
\end{entry}

\begin{entry}{今日}{jin1 ri4}{4,4}{⼈、⽇}[HSK 5]
  \definition{s.}{hoje}
\end{entry}

\begin{entry}{今天}{jin1tian1}{4,4}{⼈、⼤}[HSK 1]
  \definition{adv.}{hoje | no presente | agora}
\end{entry}

\begin{entry}{斤}{jin1}{4}{⽄}[HSK 2][Kangxi 69]
  \definition{clas.}{peso igual a 500 g}
\end{entry}

\begin{entry}{金}{jin1}{8}{⾦}[HSK 3][Kangxi 167]
  \definition*{s.}{sobrenome Jin}
  \definition{adj.}{dourado | altamente respeitado; precioso}
  \definition[锭,块]{s.}{ouro | metal | dinheiro | instrumento antigo de percussão de metal | a Dinastia Jin (1115-1234)}
\end{entry}

\begin{entry}{金刚石}{jin1gang1shi2}{8,6,5}{⾦、⼑、⽯}
  \definition{s.}{diamante, também chamado de 钻石}
  \seeref{钻石}{zuan4shi2}
\end{entry}

\begin{entry}{金牌}{jin1 pai2}{8,12}{⾦、⽚}[HSK 3]
  \definition[枚]{s.}{medalha de ouro | ficha de ouro}
\end{entry}

\begin{entry}{金融}{jin1rong2}{8,16}{⾦、⿀}
  \definition{s.}{finança}
\end{entry}

\begin{entry}{金色}{jin1 se4}{8,6}{⾦、⾊}
  \definition{s.}{cor ouro; dourado}
\end{entry}

\begin{entry}{金子}{jin1zi5}{8,3}{⾦、⼦}
  \definition{s.}{ouro; elemento metálico, símbolo Au (aurum) amarelo-avermelhado, macio, dúctil, quimicamente estável é um metal precioso, usado para fabricar dinheiro, ornamentos etc.}
\end{entry}

\begin{entry}{矜}{jin1}{9}{⽭}
  \definition{adj.}{presunçoso; vaidoso | contido; reservado; determinado}
  \definition{v.}{ter pena; simpatizar com; compadecer-se}
\end{entry}

\begin{entry}{仅}{jin3}{4}{⼈}[HSK 3]
  \definition{adv.}{somente; meramente; por muito pouco}
\end{entry}

\begin{entry}{仅仅}{jin3 jin3}{4,4}{⼈、⼈}[HSK 3]
  \definition{adv.}{somente; meramente; por muito pouco}
\end{entry}

\begin{entry}{尽管}{jin3guan3}{6,14}{⼫、⽵}[HSK 5]
  \definition{adv.}{justo; livremente; faça o que quiser, não se preocupe, não há restrições de movimento ou comportamento}
  \definition{conj.}{no entanto; embora; apesar de ; normalmente usado no início de uma frase anterior para introduzir um fato, seguido de ``但是'' etc. para introduzir um resultado que o fato não deveria ter; às vezes, também pode ser usado no início de uma frase posterior.}
  \seealsoref{但是}{dan4 shi4}
\end{entry}

\begin{entry}{尽可能}{jin3 ke3 neng2}{6,5,10}{⼫、⼝、⾁}[HSK 5]
  \definition{adv.}{na medida do possível; com o melhor de sua capacidade; tentar fazer algo, atingir um determinado nível ou extensão}
\end{entry}

\begin{entry}{尽快}{jin3kuai4}{6,7}{⼫、⼼}[HSK 4]
  \definition{adv.}{com toda a velocidade; o mais rápido possível; o mais breve possível}
\end{entry}

\begin{entry}{尽量}{jin3liang4}{6,12}{⼫、⾥}[HSK 3]
  \definition{adv.}{tanto quanto possível; da melhor maneira possível}
\end{entry}

\begin{entry}{紧}{jin3}{10}{⽷}[HSK 3]
  \definition{adj.}{tenso; apertado | seguro; firme | cerrado; apertado | urgente; premente; tenso | rigoroso; rígido; severo | difícil; sem dinheiro}
  \definition{v.}{apertar}
\end{entry}

\begin{entry}{紧急}{jin3ji2}{10,9}{⽷、⼼}[HSK 3]
  \definition{adj.}{urgente}
  \definition{adj.}{urgente; premente; crítico}
  \definition{s.}{emergência}
\end{entry}

\begin{entry}{紧紧}{jin3 jin3}{10,10}{⽷、⽷}[HSK 5]
  \definition{adv.}{firmemente; estreitamente; apertadamente; prestar muita atenção (em algo)}
\end{entry}

\begin{entry}{紧密}{jin3 mi4}{10,11}{⽷、⼧}[HSK 4]
  \definition{adj.}{próximos; inseparáveis | incessante; rápido e intenso}
\end{entry}

\begin{entry}{紧张}{jin3zhang1}{10,7}{⽷、⼸}[HSK 3]
  \definition{adj.}{nervoso; tenso | apertado; em falta | tenso; intenso; coado}
\end{entry}

\begin{entry}{锦上添花}{jin3shang4tian1hua1}{13,3,11,7}{⾦、⼀、⽔、⾋}
  \definition{expr.}{A cereja do bolo | (literalmente) adicione flores ao brocato}
  \definition{v.}{dar a alguém esplendor adicional | fornecer o toque final}
\end{entry}

\begin{entry}{尽力}{jin4li4}{6,2}{⼫、⼒}[HSK 4]
  \definition{v.+compl.}{esforçar-se ao máximo; esforçar-se ao máximo; usar toda a sua força; fazer algo com seu melhor esforço}
\end{entry}

\begin{entry}{近}{jin4}{7}{⾡}[HSK 2]
  \definition{adj.}{perto | próximo}
\end{entry}

\begin{entry}{近代}{jin4dai4}{7,5}{⾡、⼈}[HSK 4]
  \definition{s.}{tempos modernos; era passada relativamente próxima à era moderna, geralmente referida na história chinesa como 1840 a 1919 | na história mundial, geralmente se refere à era capitalista}
\end{entry}

\begin{entry}{近来}{jin4lai2}{7,7}{⾡、⽊}[HSK 5]
  \definition{adv.}{ultimamente; recentemente; de ​​tarde; refere-se a um período de tempo entre o passado imediato e o presente}
\end{entry}

\begin{entry}{近期}{jin4 qi1}{7,12}{⾡、⽉}[HSK 3]
  \definition{adv.}{num futuro próximo; brevemente}
\end{entry}

\begin{entry}{进}{jin4}{7}{⾡}[HSK 1]
  \definition{clas.}{para seções em um edifício ou complexo residencial}
  \definition{s.}{(matemática) base de um sistema numérico}
  \definition{v.}{entrar}
\end{entry}

\begin{entry}{进步}{jin4bu4}{7,7}{⾡、⽌}[HSK 3]
  \definition{adj.}{progressivo}
  \definition[个]{s.}{avanço; progresso; melhora}
  \definition{v.}{avançar; progredir; melhorar}
\end{entry}

\begin{entry}{进出口}{jin4chu1kou3}{7,5,3}{⾡、⼐、⼝}
  \definition{s.}{importação e exportação}
  \definition{v.}{importar e exportar}
\end{entry}

\begin{entry}{进化}{jin4hua4}{7,4}{⾡、⼔}[HSK 5]
  \definition[个]{s.}{evolução; os organismos se desenvolvem e evoluem do simples para o complexo e de níveis baixos para altos}
  \definition{v.}{evoluir; um termo geral usado para descrever uma mudança gradual para melhor}
\end{entry}

\begin{entry}{进口}{jin4kou3}{7,3}{⾡、⼝}[HSK 4]
  \definition{adj.}{importado}
  \definition{s.}{importação; entrada de um edifício ou local, também chamada de `` 入口''}
  \definition{v.+compl.}{importar; comprar ou transportar mercadorias de outro país ou região | entrar no porto; navegar em direção a um porto}
  \seealsoref{入口}{ru4kou3}
\end{entry}

\begin{entry}{进来}{jin4 lai2}{7,7}{⾡、⽊}[HSK 1]
  \definition{v.}{entrar (para a minha localização)}
\end{entry}

\begin{entry}{进去}{jin4 qu4}{7,5}{⾡、⼛}[HSK 1]
  \definition{v.}{entrar (a partir da minha localização)}
\end{entry}

\begin{entry}{进入}{jin4 ru4}{7,2}{⾡、⼊}[HSK 2]
  \definition{v.}{entrar | juntar-se}
\end{entry}

\begin{entry}{进行}{jin4xing2}{7,6}{⾡、⾏}[HSK 2]
  \definition{v.}{continuar | estar em andamento | fazer | conduzir | continuar | executar | marchar | avançar | prosseguir | estar em marcha}
\end{entry}

\begin{entry}{进行编程}{jin4xing2bian1cheng2}{7,6,12,12}{⾡、⾏、⽷、⽲}
  \definition{s.}{programa de computador executável}
\end{entry}

\begin{entry}{进一步}{jin4 yi2 bu4}{7,1,7}{⾡、⼀、⽌}[HSK 3]
  \definition{adv.}{mais; dar um passo adiante; avançar um passo}
\end{entry}

\begin{entry}{进展}{jin4zhan3}{7,10}{⾡、⼫}[HSK 3]
  \definition{v.}{fazer progresso; progredir}
\end{entry}

\begin{entry}{禁止}{jin4zhi3}{13,4}{⽰、⽌}[HSK 4]
  \definition{v.}{banir; proibir; interditar}
\end{entry}

\begin{entry}{京}{jing1}{8}{⼇}
  \definition*{s.}{Beijing, abreviação de~北京 | sobrenome Jing}
  \definition{s.}{capital de um país}
  \seeref{北京}{bei3 jing1}
\end{entry}

\begin{entry}{京剧}{jing1ju4}{8,10}{⼇、⼑}[HSK 3]
  \definition*[场,段]{s.}{Ópera de Beijing (Pequim)}
\end{entry}

\begin{entry}{经}{jing1}{8}{⽷}
  \definition*{s.}{sobrenome Jing}
  \definition{s.}{livro sagrado | escritura | clássicos | longitude | menstruação | canal}
  \definition{v.}{passar | sofrer | suportar | deformar (têxtil)}
\end{entry}

\begin{entry}{经常}{jing1chang2}{8,11}{⽷、⼱}[HSK 2]
  \definition{adv.}{constantemente | diariamente | dia-a-dia | todo dia | frequentemente | sempre | regularmente}
\end{entry}

\begin{entry}{经典}{jing1dian3}{8,8}{⽷、⼋}[HSK 4]
  \definition{adj.}{clássico; (escritos ou obras, etc.) que são típicos, autorizados}
  \definition{s.}{clássicos; escritos tradicionais e valiosos; os livros mais importantes e fundamentais da religião | escrituras; escritos de doutrinas religiosas}
\end{entry}

\begin{entry}{经费}{jing1fei4}{8,9}{⽷、⾙}[HSK 5]
  \definition[笔]{s.}{fundos; desembolso; gastos | despesas; gastos}
\end{entry}

\begin{entry}{经过}{jing1guo4}{8,6}{⽷、⾡}[HSK 2]
  \definition[个]{s.}{processo | curso}
  \definition{v.}{passar | passar por}
\end{entry}

\begin{entry}{经济}{jing1ji4}{8,9}{⽷、⽔}[HSK 3]
  \definition{adj.}{econômico;  parcimonioso}
  \definition{s.}{economia |economia nacional; setor da economia nacional | renda; condição financeira}
  \definition{v.}{governar o país}
\end{entry}

\begin{entry}{经理}{jing1li3}{8,11}{⽷、⽟}[HSK 2]
  \definition[个,位,名]{s.}{diretor | gerente}
\end{entry}

\begin{entry}{经历}{jing1li4}{8,4}{⽷、⼚}[HSK 3]
  \definition[个,次,段,种]{s.}{experiência}
  \definition{v.}{passar por}
\end{entry}

\begin{entry}{经验}{jing1yan4}{8,10}{⽷、⾺}[HSK 3]
  \definition[个,次,种]{s.}{experiência}
  \definition{v.}{experimentar; passar por}
\end{entry}

\begin{entry}{经营}{jing1ying2}{8,11}{⽷、⾋}[HSK 3]
  \definition{v.}{executar; gerenciar; operar; envolver-se em | gerenciar}
\end{entry}

\begin{entry}{惊呆}{jing1dai1}{11,7}{⼼、⼝}
  \definition{adj.}{estupefato | chocado}
\end{entry}

\begin{entry}{惊喜}{jing1xi3}{11,12}{⼼、⼝}
  \definition{s.}{boa surpresa}
  \definition{v.}{ser agradavelmente surpreendido}
\end{entry}

\begin{entry}{精彩}{jing1cai3}{14,11}{⽶、⼺}[HSK 3]
  \definition{adj.}{brilhante; esplêndido; maravilhoso}
\end{entry}

\begin{entry}{精力}{jing1li4}{14,2}{⽶、⼒}[HSK 4]
  \definition[些]{s.}{energia; vigor; força mental e física}
\end{entry}

\begin{entry}{精灵}{jing1ling2}{14,7}{⽶、⽕}
  \definition{s.}{espírito | fada | elfo | duende | gênio}
\end{entry}

\begin{entry}{精品}{jing1pin3}{14,9}{⽶、⼝}
  \definition{s.}{produtos de qualidade | produto premium | bom trabalho (de arte)}
\end{entry}

\begin{entry}{精神}{jing1shen2}{14,9}{⽶、⽰}[HSK 3]
  \definition[个]{s.}{espírito; mente; estado mental | substância; espírito; essência}
  \seeref{精神}{jing1shen5}
\end{entry}

\begin{entry}{精神}{jing1shen5}{14,9}{⽶、⽰}[HSK 3]
  \definition{adj.}{animado; espirituoso; vigoroso
bonito}
  \definition[种,个,类,股]{s.}{impulso; vigor; vitalidade}
  \seeref{精神}{jing1shen2}
\end{entry}

\begin{entry}{精致}{jing1zhi4}{14,10}{⽶、⾄}
  \definition{adj.}{delicado | exótico | refinado}
\end{entry}

\begin{entry}{鲸鲨}{jing1sha1}{16,15}{⿂、⿂}
  \definition{s.}{tubarão baleia}
\end{entry}

\begin{entry}{鲸鱼}{jing1yu2}{16,8}{⿂、⿂}
  \definition{s.}{baleia}
\end{entry}

\begin{entry}{井}{jing3}{4}{⼆}
  \definition{adj.}{puro | ordenado}
  \definition[口]{s.}{poço}
\end{entry}

\begin{entry}{景色}{jing3se4}{12,6}{⽇、⾊}[HSK 3]
  \definition[片,幅,道,处]{s.}{vista; cena; cenário; paisagem}
\end{entry}

\begin{entry}{景象}{jing3 xiang4}{12,11}{⽇、⾗}[HSK 5]
  \definition[个]{s.}{cena; visão; vista; quadro}
\end{entry}

\begin{entry}{警}{jing3}{19}{⾔}
  \definition{s.}{policial}
  \definition{v.}{alertar | avisar}
\end{entry}

\begin{entry}{警察}{jing3cha2}{19,14}{⾔、⼧}[HSK 3]
  \definition[个,位,名,群,队]{s.}{polícia; policial; oficial de polícia}
\end{entry}

\begin{entry}{警告}{jing3gao4}{19,7}{⾔、⼝}[HSK 5]
  \definition[个]{s.}{advertência (como medida disciplinar); uma forma de punição}
  \definition{v.}{avisar; advertir; admoestar}
\end{entry}

\begin{entry}{警官}{jing3guan1}{19,8}{⾔、⼧}
  \definition{s.}{polícia | policial}
\end{entry}

\begin{entry}{竞赛}{jing4sai4}{10,14}{⽴、⾙}[HSK 5]
  \definition[个]{s.}{concurso; competição; partida; corrida}
  \definition{v.}{correr; competir; competir uns com os outros por superioridade; em esportes, produção e outras atividades, para comparar competência, habilidade etc., usado principalmente na linguagem falada}
\end{entry}

\begin{entry}{竞争}{jing4zheng1}{10,6}{⽴、⼑}[HSK 5]
  \definition{v.}{competir; disputar; lutar; entre duas ou mais partes; em prol de seus próprios interesses; lutar pela vitória por meio de uma disputa de sua própria força contra outra}
\end{entry}

\begin{entry}{竟然}{jing4ran2}{11,12}{⾳、⽕}[HSK 4]
  \definition{adv.}{de fato; inesperadamente; para surpresa de alguém; chegar ao ponto de; indica que algo é um pouco inesperado}
\end{entry}

\begin{entry}{敬礼}{jing4li3}{12,5}{⽁、⽰}
  \definition{s.}{saudação}
  \definition{v.}{saudar}
\end{entry}

\begin{entry}{静}{jing4}{14}{⾭}[HSK 3]
  \definition*{s.}{sobrenome Jing}
  \definition{adj.}{tranquilo;  sossegado; calmo; imóvel | silencioso; quieto}
\end{entry}

\begin{entry}{镜头}{jing4tou2}{16,5}{⾦、⼤}[HSK 4]
  \definition[个]{s.}{lente de câmera; objetiva; combinação de várias lentes, usada para formar uma imagem | foto; cena}
\end{entry}

\begin{entry}{镜子}{jing4zi5}{16,3}{⾦、⼦}[HSK 4]
  \definition[面,个]{s.}{espelho; instrumento de reflexão de imagem liso e plano, antigamente esmerilhado a partir de um disco grosso de cobre fundido, atualmente feito de vidro plano revestido de prata ou alumínio |
óculos; óculos de grau;}
\end{entry}

\begin{entry}{纠葛}{jiu1ge2}{5,12}{⽷、⾋}
  \definition{s.}{emaranhado | disputa}
\end{entry}

\begin{entry}{究竟}{jiu1jing4}{7,11}{⽳、⾳}[HSK 4]
  \definition{adv.}{de fato; exatamente; usado em frases interrogativas para buscar | afinal de contas, no final; ênfase em fatos ou motivos}
  \definition{s.}{resultado; desfecho; a causa, o efeito ou a história completa do que aconteceu}
\end{entry}

\begin{entry}{九}{jiu3}{2}{⼄}[HSK 1]
  \definition{num.}{nove; 9}
\end{entry}

\begin{entry}{久}{jiu3}{3}{⼃}[HSK 3]
  \definition{adj.}{por muito tempo | duração de tempo especificada}
\end{entry}

\begin{entry}{韭菜}{jiu3cai4}{9,11}{⾲、⾋}
  \definition{s.}{cebolinha chinesa | (figurativo) investidores de varejo que perdem seu dinheiro para operadores mais experientes (ou seja, são ``colhidos'' como cebolinhas)}
\end{entry}

\begin{entry}{酒}{jiu3}{10}{⾣}[HSK 2]
  \definition[杯,瓶,罐,桶,缸]{s.}{bebida alcoólica | vinho (especialmente vinho de arroz) | aguardente | licor | espíritos}
\end{entry}

\begin{entry}{酒吧}{jiu3ba1}{10,7}{⾣、⼝}[HSK 4]
  \definition[家,个]{s.}{bar; \emph{pub}; um local onde são vendidas bebidas alcoólicas e onde as pessoas podem beber e conversar, referindo-se principalmente a um restaurante ou hotel de estilo ocidental especializado na venda de bebidas alcoólicas.}
\end{entry}

\begin{entry}{酒店}{jiu3 dian4}{10,8}{⾣、⼴}[HSK 2]
  \definition[家]{s.}{hotel | restaurante}
\end{entry}

\begin{entry}{酒馆}{jiu3guan3}{10,11}{⾣、⾷}
  \definition{s.}{bar | taverna | adega}
\end{entry}

\begin{entry}{酒鬼}{jiu3gui3}{10,9}{⾣、⿁}[HSK 5]
  \definition{s.}{bebedor de vinho; beberrão; ébrio | alcoólatra}
\end{entry}

\begin{entry}{旧}{jiu4}{5}{⽇}[HSK 3]
  \definition*{s.}{sobrenome Jiu}
  \definition{adj.}{passado; antigo; velho | usado; desgastado; velho}
  \definition{s.}{velha amizade; velho amigo}
\end{entry}

\begin{entry}{救}{jiu4}{11}{⽁}[HSK 3]
  \definition*{s.}{sobrenome Jiu}
  \definition{v.}{resgatar; salvar | ajudar; aliviar; socorrer}
\end{entry}

\begin{entry}{救出}{jiu4chu1}{11,5}{⽁、⼐}
  \definition{v.}{resgatar | tirar do perigo}
\end{entry}

\begin{entry}{救护车}{jiu4hu4che1}{11,7,4}{⽁、⼿、⾞}
  \definition[辆]{s.}{ambulância}
\end{entry}

\begin{entry}{救命}{jiu4ming4}{11,8}{⽁、⼝}
  \definition{interj.}{Socorro! | Salve-me!}
  \definition{v.+compl.}{salvar a vida de alguém}
\end{entry}

\begin{entry}{救灾}{jiu4 zai1}{11,7}{⽁、⽕}[HSK 5]
  \definition{v.}{ajudar as vítimas de desastres, aliviar o desastre; resgatar pessoas afetadas por desastres; recuperar danos causados por desastres}
\end{entry}

\begin{entry}{就}{jiu4}{12}{⼪}[HSK 1]
  \definition{adv.}{exatamente | justamente}
  \definition{v.}{realizar | se envolver em | acompanhar (em alimentos) | aproveitar | avançar | empreender}
\end{entry}

\begin{entry}{就是}{jiu4 shi4}{12,9}{⼪、⽇}[HSK 3]
  \definition{adv.}{exatamente; precisamente | apenas; simplesmente | usado para indicar escolha}
  \definition{conj.}{ainda que}
  \definition{part.}{usado no final de uma frase para expressar afirmação}
\end{entry}

\begin{entry}{就要}{jiu4 yao4}{12,9}{⼪、⾑}[HSK 2]
  \definition{adv.}{estar prestes a | estar indo para | estar a ponto de}
\end{entry}

\begin{entry}{就业}{jiu4ye4}{12,5}{⼪、⼀}[HSK 3]
  \definition{v.+compl.}{conseguir um emprego; obter emprego; assumir uma ocupação}
\end{entry}

\begin{entry}{就职}{jiu4zhi2}{12,11}{⼪、⽿}
  \definition{v.}{assumir o cargo | assumir um posto}
\end{entry}

\begin{entry}{车}{ju1}{4}{⾞}
  \definition{s.}{(arcaico) carruagem de guerra | torre (no xadrez)}
  \seeref{车}{che1}
\end{entry}

\begin{entry}{居民}{ju1min2}{8,5}{⼫、⽒}[HSK 4]
  \definition[个,户,位]{s.}{residente; habitante; pessoas que estão fixas em um único lugar}
\end{entry}

\begin{entry}{居然}{ju1ran2}{8,12}{⼫、⽕}[HSK 5]
  \definition{adv.}{inesperadamente; para surpresa de alguém; além da expectativa (expressão idiomática) |}
  \definition{v.}{ir tão longe a ponto de; ter a impudência de; ter o descaramento de;}
\end{entry}

\begin{entry}{居住}{ju1zhu4}{8,7}{⼫、⼈}[HSK 4]
  \definition{v.}{viver; residir; morar; habitar}
\end{entry}

\begin{entry}{局}{ju2}{7}{⼫}[HSK 4]
  \definition{s.}{tabuleiro de xadrez | jogo; turno; \emph{set} | situação; estado das coisas | tolerância; grandeza ou pequenez da mente; grau de tolerância de uma pessoa em relação às outras | reunião de pessoas em festas | ardil; artidício; estratagema; armadilha | parte; porção; parcela | nome de determinadas lojas}
\end{entry}

\begin{entry}{局面}{ju2mian4}{7,9}{⼫、⾯}[HSK 5]
  \definition[种]{s.}{aspecto; fase; situação; o estado das coisas em um período de tempo, em sua maior parte abstraído | escopo; escala}
\end{entry}

\begin{entry}{局长}{ju2 zhang3}{7,4}{⼫、⾧}[HSK 5]
  \definition[位,个]{s.}{comissário; diretor; principais chefes de gabinete do governo}
\end{entry}

\begin{entry}{橘子汁}{ju2zi5zhi1}{16,3,5}{⽊、⼦、⽔}
  \definition[瓶,杯,罐,盒]{s.}{suco de laranja}
  \seealsoref{橙汁}{cheng2zhi1}
  \seealsoref{柳橙汁}{liu3cheng2zhi1}
\end{entry}

\begin{entry}{举}{ju3}{9}{⼂}[HSK 2]
  \definition{v.}{levantar | segurar | iniciar | começar | dar à luz a | eleger | escolher | citar| enumerar}
\end{entry}

\begin{entry}{举办}{ju3ban4}{9,4}{⼂、⼒}[HSK 3]
  \definition{v.}{segurar; conduzir}
\end{entry}

\begin{entry}{举动}{ju3dong4}{9,6}{⼂、⼒}[HSK 5]
  \definition{s.}{ato; atividade; movimento; ação}
\end{entry}

\begin{entry}{举手}{ju3 shou3}{9,4}{⼂、⼿}[HSK 2]
  \definition{v.}{levantar (colocar) a mão ou mãos}
\end{entry}

\begin{entry}{举行}{ju3xing2}{9,6}{⼂、⾏}[HSK 2]
  \definition{v.}{realizar (uma reunião, cerimônia, etc.) | ter lugar}
\end{entry}

\begin{entry}{巨大}{ju4da4}{4,3}{⼯、⼤}[HSK 4]
  \definition{adj.}{enorme; tremendo; enorme; gigantesco; imenso}
\end{entry}

\begin{entry}{句}{ju4}{5}{⼝}[HSK 2]
  \definition{clas.}{para orações, frases ou linhas de versos}
  \definition{s.}{sentença | cláusula | frase}
  \seeref{句}{gou4}
\end{entry}

\begin{entry}{句子}{ju4zi5}{5,3}{⼝、⼦}[HSK 2]
  \definition[个]{s.}{sentença | frase | oração}
\end{entry}

\begin{entry}{拒绝}{ju4jue2}{7,9}{⼿、⽷}[HSK 5]
  \definition{v.}{recusar; rejeitar; declinar; não aceitar (pedidos, sugestões ou presentes)}
\end{entry}

\begin{entry}{足}{ju4}{7}{⾜}
  \definition{adj.}{excessivo}
  \seeref{足}{zu2}
\end{entry}

\begin{entry}{具备}{ju4bei4}{8,8}{⼋、⼡}[HSK 4]
  \definition{v.}{ter; possuir; ser provido de}
\end{entry}

\begin{entry}{具体}{ju4ti3}{8,7}{⼋、⼈}[HSK 3]
  \definition{adj.}{específico; particular | concreto; específico | concreto; real}
  \definition{v.}{incorporar; objetivar}
\end{entry}

\begin{entry}{具有}{ju4 you3}{8,6}{⼋、⽉}[HSK 3]
  \definition{v.}{ter; possuir; ser provido de}
\end{entry}

\begin{entry}{俱乐部}{ju4le4bu4}{10,5,10}{⼈、⼃、⾢}[HSK 5]
  \definition[个]{s.}{clube; grupos e locais para atividades sociais, políticas, literárias, recreativas e outras}
\end{entry}

\begin{entry}{剧本}{ju4ben3}{10,5}{⼑、⽊}[HSK 5]
  \definition{s.}{cenário; roteiro (para drama, filme, etc.); gênero de obra literária que consiste em diálogos entre personagens (às vezes cantados) e indicações de palco}
\end{entry}

\begin{entry}{剧场}{ju4 chang3}{10,6}{⼑、⼟}[HSK 3]
  \definition[个,坐]{s.}{teatro; um lugar para apresentações teatrais, canto e dança, etc.}
\end{entry}

\begin{entry}{据说}{ju4shuo1}{11,9}{⼿、⾔}[HSK 3]
  \definition{v.}{ser dito; ser relatado}
\end{entry}

\begin{entry}{距离}{ju4li2}{11,10}{⾜、⼇}[HSK 4]
  \definition[个]{s.}{distância}
  \definition{v.}{estar distante de}
\end{entry}

\begin{entry}{聚}{ju4}{14}{⽿}[HSK 4]
  \definition{v.}{reunir-se; juntar-se}
\end{entry}

\begin{entry}{聚会}{ju4hui4}{14,6}{⽿、⼈}[HSK 4]
  \definition[个,次]{s.}{reunião; encontro; confraternização; festa}
  \definition{v.}{encontrar-se; reunir-se}
\end{entry}

\begin{entry}{聚散}{ju4san4}{14,12}{⽿、⽁}
  \definition{s.}{juntos e separados | agregação e dissipação}
\end{entry}

\begin{entry}{圈}{juan1}{11}{⼞}
  \definition{v.}{prender aves e animais de criação | prender; colocar na cadeia, prisão | confinar}
  \seeref{圈}{juan4}
  \seeref{圈}{quan1}
\end{entry}

\begin{entry}{卷}{juan3}{8}{⼙}[HSK 4]
  \definition{clas.}{para pequenas coisas enroladas (maço de papel dinheiro, carretel de filme, etc.) | para rolos, carretéis, bobinas, etc.}
  \definition[张]{s.}{rolo; carretel; bobina}
  \definition{v.}{enrolar; dobrar algo em um cilindro ou semicírculo | varrer; carregar; levar junto | envolver-se; participar}
  \seeref{卷}{juan4}
\end{entry}

\begin{entry}{卷}{juan4}{8}{⼙}[HSK 4]
  \definition{clas.}{para capítulos, seções ou volumes; fascículos}
  \definition{s.}{livro; livros e pinturas que são enrolados para coleção; geralmente se refere a pinturas e caligrafia | papel de exame | arquivo; dossiê}
  \seeref{卷}{juan3}
\end{entry}

\begin{entry}{圈}{juan4}{11}{⼞}
  \definition*{s.}{sobrenome Juan}
  \definition{s.}{curral; local onde o gado ou as aves são mantidos, geralmente cercado ou murado, alguns com galpões}
  \seeref{圈}{juan1}
  \seeref{圈}{quan1}
\end{entry}

\begin{entry}{决不}{jue2 bu4}{6,4}{⼎、⼀}[HSK 5]
  \definition{adv.}{definitivamente não; certamente não; sob nenhuma circunstância; de forma alguma}
\end{entry}

\begin{entry}{决定}{jue2ding4}{6,8}{⼎、⼧}[HSK 3]
  \definition{adj.}{decisivo}
  \definition{s.}{decisão; resolução}
  \definition{v.}{decidir; determinar | decidir; resolver; tomar uma decisão}
\end{entry}

\begin{entry}{决赛}{jue2sai4}{6,14}{⼎、⾙}[HSK 3]
  \definition{s.}{finais (de uma competição)}
\end{entry}

\begin{entry}{决心}{jue2xin1}{6,4}{⼎、⼼}[HSK 3]
  \definition{s.}{resolução; determinação}
  \definition{v.}{secidir-se; decidir fazer algo e não vacilar nem mudar de ideia}
\end{entry}

\begin{entry}{角}{jue2}{7}{⾓}
  \definition*{s.}{sobrenome Jue}
  \definition{s.}{papel (teatro)}
  \definition{v.}{competir}
  \seeref{角}{jiao3}
\end{entry}

\begin{entry}{角色}{jue2se4}{7,6}{⾓、⾊}[HSK 4]
  \definition{s.}{papel; personagem em uma peça; personagem representado por um ator | papel; função; parte}
\end{entry}

\begin{entry}{绝版}{jue2ban3}{9,8}{⽷、⽚}
  \definition{adj.}{esgotado | fora de catálogo}
\end{entry}

\begin{entry}{绝不}{jue2bu4}{9,4}{⽷、⼀}
  \definition{adv.}{definitivamente não | de forma alguma | sob nenhuma circunstância}
\end{entry}

\begin{entry}{绝对}{jue2dui4}{9,5}{⽷、⼨}[HSK 3]
  \definition{adj.}{absoluto; extremo}
  \definition{adv.}{absolutamente}
\end{entry}

\begin{entry}{绝望}{jue2 wang4}{9,11}{⽷、⽉}[HSK 5]
  \definition{v.+compl.}{desesperar; desistir de toda esperança; perder toda esperança de}
\end{entry}

\begin{entry}{绝招}{jue2zhao1}{9,8}{⽷、⼿}
  \definition{s.}{habilidade única | movimento delicado inesperado (como último recurso) | golpe de mestre | golpe final}
\end{entry}

\begin{entry}{觉得}{jue2de5}{9,11}{⾒、⼻}[HSK 1]
  \definition{v.}{pensar que\dots | sentir que\dots | sentir (desconfortável, etc.)}
\end{entry}

\begin{entry}{脚}{jue2}{11}{⾁}
  \variantof{角}
\end{entry}

\begin{entry}{军人}{jun1 ren2}{6,2}{⼍、⼈}[HSK 5]
  \definition{s.}{soldado; militar; pessoal militar; pessoas com status militar; pessoas servindo nas forças armadas}
\end{entry}

\begin{entry}{军装}{jun1zhuang1}{6,12}{⼍、⾐}
  \definition{s.}{uniforme militar}
\end{entry}

\begin{entry}{君主立宪制}{jun1zhu3li4xian4zhi4}{7,5,5,9,8}{⼝、⼂、⽴、⼧、⼑}
  \definition{s.}{monarquia constitucional}
\end{entry}

%%%%% EOF %%%%%


%%%
%%% K
%%%

\section*{K}\addcontentsline{toc}{section}{K}

\begin{entry}{咖啡}{ka1fei1}{8,11}[HSK 3][Radicais ⼝、⼝]
  \definition[杯]{s.}{(empréstimo linguístico) café}
\end{entry}

\begin{entry}{咖啡馆}{ka1fei1guan3}{8,11,11}[Radicais ⼝、⼝、⾷]
  \definition[家]{s.}{cafeteria}
\end{entry}

\begin{entry}{咖啡色}{ka1fei1 se4}{8,11,6}[Radicais ⼝、⼝、⾊]
  \definition{s.}{cor café}
\end{entry}

\begin{entry}{卡}{ka3}{5}[HSK 2][Radical ⼘]
  \definition{clas.}{para calorias}
  \definition{s.}{cartão}
  \definition{v.}{bloquear | verificar | agarrar}
  \seeref{卡}{qia3}
\end{entry}

\begin{entry}{卡车司机}{ka3che1 si1ji1}{5,4,5,6}[Radicais ⼘、⾞、⼝、⽊]
  \definition{s.}{motorista de caminhão}
\end{entry}

\begin{entry}{卡片}{ka3pian4}{5,4}[Radicais ⼘、⽚]
  \definition{s.}{cartão}
\end{entry}

\begin{entry}{卡片游戏}{ka3pian4 you2xi4}{5,4,12,6}[Radicais ⼘、⽚、⽔、⼽]
  \definition{s.}{carta de baralho}
\end{entry}

\begin{entry}{卡通}{ka3tong1}{5,10}[Radicais ⼘、⾡]
  \definition{s.}{(empréstimo linguístico) \emph{cartoon}}
\end{entry}

\begin{entry}{开}{kai1}{4}[HSK 1][Radical ⼶]
  \definition{clas.}{quilate (ouro)}
  \definition{v.}{abrir | ligar | dirigir | iniciar (alguma coisa) | começar | ferver | escrever  (uma receita, cheque, fatura, etc.) | operar (um veículo) | abreviação de Kelvin 开尔文}
  \seeref{开尔文}{kai1'er3wen2}
\end{entry}

\begin{entry}{开车}{kai1 che1}{4,4}[HSK 1][Radicais ⼶、⾞]
  \definition{v.+compl.}{conduzir | dirigir}
\end{entry}

\begin{entry}{开尔文}{kai1'er3wen2}{4,5,4}[Radicais ⼶、⼩、⽂]
  \definition{s.}{Kelvin, temperatura absoluta | K, escala de temperatura}
\end{entry}

\begin{entry}{开发}{kai1fa1}{4,5}[HSK 3][Radicais ⼶、⼜]
  \definition{v.}{explorar | tornar acessível}
\end{entry}

\begin{entry}{开发区}{kai1fa1qu1}{4,5,4}[Radicais ⼶、⼜、⼖]
  \definition{s.}{zona de desenvolvimento}
\end{entry}

\begin{entry}{开放}{kai1fang4}{4,8}[HSK 3][Radicais ⼶、⽅]
  \definition{adj.}{de mente aberta; sem restrições por convenções}
  \definition{v.}{florescer | abrir (para o público) | diminuir uma proibição, restrição, etc. (de política)}
\end{entry}

\begin{entry}{开花}{kai1hua1}{4,7}[Radicais ⼶、⾋]
  \definition{v.}{florescer | (fig.) explodir, abrir-se | (fig.) explodir de alegria | (fig.) começar a existir de repente em todos os lugares}
\end{entry}

\begin{entry}{开会}{kai1 hui4}{4,6}[HSK 1][Radicais ⼶、⼈]
  \definition{v.+compl.}{realizar uma reunião | ter uma reunião | participar de uma reunião (conferência)}
\end{entry}

\begin{entry}{开机}{kai1 ji1}{4,6}[HSK 2][Radicais ⼶、⽊]
  \definition{v.}{começar a filmar um filme ou programa de TV | iniciar uma máquina}
\end{entry}

\begin{entry}{开口}{kai1kou3}{4,3}[Radicais ⼶、⼝]
  \definition{v.}{abrir a boca de alguém | começar a falar}
\end{entry}

\begin{entry}{开启}{kai1qi3}{4,7}[Radicais ⼶、⼝]
  \definition{v.}{abrir | iniciar | (computação) ativar}
\end{entry}

\begin{entry}{开始}{kai1shi3}{4,8}[HSK 3][Radicais ⼶、⼥]
  \definition{adv.}{inicial}
  \definition[个]{s.}{começo; início; estágio inicial}
  \definition{v.}{começar; iniciar}
\end{entry}

\begin{entry}{开锁}{kai1suo3}{4,12}[Radicais ⼶、⾦]
  \definition{v.}{desbloquear | destravar}
\end{entry}

\begin{entry}{开头}{kai1tou2}{4,5}[Radicais ⼶、⼤]
  \definition{s.}{início | começo}
  \definition{v.+compl.}{iniciar | começar | fazer um começo}
\end{entry}

\begin{entry}{开玩笑}{kai1 wan2xiao4}{4,8,10}[HSK 1][Radicais ⼶、⽟、⽵]
  \definition{v.}{contar uma piada | brincar | fazer piada de | pregar uma peça | provocar}
\end{entry}

\begin{entry}{开心}{kai1xin1}{4,4}[HSK 2][Radicais ⼶、⼼]
  \definition{v.}{sentir-se feliz | regozijar-se | divertir-se | tirar sarro de alguém}
\end{entry}

\begin{entry}{开学}{kai1 xue2}{4,8}[HSK 2][Radicais ⼶、⼦]
  \definition{v.}{iniciar as aulas | iniciar o semestre | começar as aulas}
\end{entry}

\begin{entry}{开业}{kai1 ye4}{4,5}[HSK 3][Radicais ⼶、⼀]
  \definition{v.}{iniciar um negócio; abrir para negócios | abrir um consultório particular}
\end{entry}

\begin{entry}{开夜车}{kai1ye4che1}{4,8,4}[Radicais ⼶、⼣、⾞]
  \definition{expr.}{trabalho noturno | (literalmente) ``conduzir carro à noite''}
\end{entry}

\begin{entry}{开展}{kai1zhan3}{4,10}[HSK 3][Radicais ⼶、⼫]
  \definition{v.}{lançar; desenvolver | abrir; inaugurar}
\end{entry}

\begin{entry}{看}{kan1}{9}[Radical ⽬]
  \definition{v.}{cuidar | vigiar}
  \seeref{看}{kan4}
\end{entry}

\begin{entry}{砍}{kan3}{9}[Radical ⽯]
  \definition{v.}{cortar}
\end{entry}

\begin{entry}{砍刀}{kan3dao1}{9,2}[Radicais ⽯、⼑]
  \definition{s.}{facão | machete}
\end{entry}

\begin{entry}{砍掉}{kan3diao4}{9,11}[Radicais ⽯、⼿]
  \definition{v.}{amputar}
\end{entry}

\begin{entry}{砍断}{kan3duan4}{9,11}[Radicais ⽯、⽄]
  \definition{v.}{cortar}
\end{entry}

\begin{entry}{砍价}{kan3jia4}{9,6}[Radicais ⽯、⼈]
  \definition{v.}{barganhar | cortar ou derrubar um preço}
\end{entry}

\begin{entry}{砍杀}{kan3sha1}{9,6}[Radicais ⽯、⽊]
  \definition{v.}{atacar com arma branca}
\end{entry}

\begin{entry}{砍伤}{kan3shang1}{9,6}[Radicais ⽯、⼈]
  \definition{v.}{ferir com lâmina ou machado}
\end{entry}

\begin{entry}{砍树}{kan3shu4}{9,9}[Radicais ⽯、⽊]
  \definition{v.}{derrubar árvores}
\end{entry}

\begin{entry}{砍死}{kan3si3}{9,6}[Radicais ⽯、⽍]
  \definition{v.}{matar com um machado}
\end{entry}

\begin{entry}{砍头}{kan3tou2}{9,5}[Radicais ⽯、⼤]
  \definition{v.}{decapitar}
\end{entry}

\begin{entry}{看}{kan4}{9}[HSK 1][Radical ⽬]
  \definition{interj.}{Cuidado! (para um perigo)}
  \definition{part.}{(depois de um verbo) tentar}
  \definition{v.}{olhar | ver | assistir | ler | visitar (pessoas)}
  \seeref{看}{kan1}
\end{entry}

\begin{entry}{看病}{kan4 bing4}{9,10}[HSK 1][Radicais ⽬、⽧]
  \definition{v.+compl.}{(médico) ver um paciente | (paciente) consultar (ver) um médico}
\end{entry}

\begin{entry}{看淡}{kan4dan4}{9,11}[Radicais ⽬、⽔]
  \definition{v.}{considerar sem importância | ser indiferente a (fama, riqueza, etc.) | (de uma economia ou mercado) enfraquecer, ficar mais lento, diminuir a velocidade}
\end{entry}

\begin{entry}{看到}{kan4 dao4}{9,8}[HSK 1][Radicais ⽬、⼑]
  \definition{v.}{ver}
\end{entry}

\begin{entry}{看法}{kan4fa3}{9,8}[HSK 2][Radicais ⽬、⽔]
  \definition[个]{s.}{modo de olhar alguma coisa | ponto de vista | opinião}
\end{entry}

\begin{entry}{看见}{kan4 jian4}{9,4}[HSK 1][Radicais ⽬、⾒]
  \definition{v.}{encontrar | enxergar | ver | avistar}
\end{entry}

\begin{entry}{看起来}{kan4 qi3 lai5}{9,10,7}[HSK 3][Radicais ⽬、⾛、⽊]
  \definition{v.}{parecer; parecer com}
\end{entry}

\begin{entry}{看上去}{kan4 shang4 qu4}{9,3,5}[HSK 3][Radicais ⽬、⼀、⼛]
  \definition{adv.}{parece que}
\end{entry}

\begin{entry}{扛}{kang2}{6}[Radical ⼿]
  \definition{v.}{carregar no ombro de alguém |  (fig.) assumir (um fardo, dever, etc.)}
  \seeref{扛}{gang1}
\end{entry}

\begin{entry}{考}{kao3}{6}[HSK 1][Radical ⽼]
  \definition*{s.}{sobrenome Kao}
  \definition{s.}{o pai falecido de alguém}
  \definition{v.}{examinar | dar (fazer) um exame, prova ou teste | verificar | inspecionar |estudar | investigar}
\end{entry}

\begin{entry}{考生}{kao3 sheng1}{6,5}[HSK 2][Radicais ⽼、⽣]
  \definition{s.}{candidato a exame}
\end{entry}

\begin{entry}{考试}{kao3shi4}{6,8}[HSK 1][Radicais ⽼、⾔]
  \definition[次]{s.}{teste | prova | exame}
  \definition{v.+compl.}{submeter-se a uma prova | fazer um teste}
\end{entry}

\begin{entry}{考验}{kao3yan4}{6,10}[HSK 3][Radicais ⽼、⾺]
  \definition[场,个,种]{s.}{teste; julgamento}
  \definition{v.}{testar}
\end{entry}

\begin{entry}{烤}{kao3}{10}[Radical ⽕]
  \definition{v.}{assar | grelhar}
\end{entry}

\begin{entry}{烤肉}{kao3rou4}{10,6}[Radicais ⽕、⾁]
  \definition{s.}{churrasco}
\end{entry}

\begin{entry}{靠}{kao4}{15}[HSK 2][Radical ⾮]
  \definition{prep.}{para | (chegar) perto |por | por força de | em}
  \definition{v.}{encostar-se (em) | chegar perto de | estar perto de | depender de | confiar em |confiar}
\end{entry}

\begin{entry}{科}{ke1}{9}[HSK 2][Radical ⽲]
  \definition*{s.}{sobrenome Ke}
  \definition{s.}{um ramo de estudo acadêmico ou profissional |uma divisão ou subdivisão de uma unidade administrativa | família | instruções de palco no drama chinês clássico}
\end{entry}

\begin{entry}{科技}{ke1 ji4}{9,7}[HSK 3][Radicais ⽲、⼿]
  \definition{s.}{ciência e tecnologia}
\end{entry}

\begin{entry}{科学}{ke1xue2}{9,8}[HSK 2][Radicais ⽲、⼦]
  \definition{adj.}{científico}
  \definition[门]{s.}{ciência}
\end{entry}

\begin{entry}{科学家}{ke1xue2jia1}{9,8,10}[Radicais ⽲、⼦、⼧]
  \definition[个]{s.}{cientista}
\end{entry}

\begin{entry}{颗}{ke1}{14}[Radical ⾴]
  \definition{clas.}{para grãos, pérolas, dentes, corações, satelites, pequenas esferas, etc.}
\end{entry}

\begin{entry}{蝌蚪}{ke1dou3}{15,10}[Radicais ⾍、⾍]
  \definition{s.}{girino}
\end{entry}

\begin{entry}{壳}{ke2}{7}[Radical ⼠]
  \definition{s.}{casca (de ovo, noz, caranguejo, etc.) | caixa | invólucro | alojamento (de uma máquina ou dispositivo)}
\end{entry}

\begin{entry}{咳嗽}{ke2sou5}{9,14}[Radicais ⼝、⼝]
  \definition{v.}{ter tosse | tossir}
\end{entry}

\begin{entry}{可}{ke3}{5}[Radical ⼝]
  \definition{adv.}{muito | realmente}
\end{entry}

\begin{entry}{可爱}{ke3'ai4}{5,10}[HSK 2][Radicais ⼝、⽖]
  \definition{adj.}{adorável | querido | fofo}
\end{entry}

\begin{entry}{可编程}{ke3bian1cheng2}{5,12,12}[Radicais ⼝、⽷、⽲]
  \definition{adj.}{programável}
\end{entry}

\begin{entry*}{可擦写可编程只读存储器}{ke3ca1xie3ke3bian1cheng2zhi1du2cun2chu3qi4}{5,17,5,5,12,12,5,10,6,12,16}[Radicais ⼝、⼿、⼍、⼝、⽷、⽲、⼝、⾔、⼦、⼈、⼝]
  \definition{s.}{EPROM (\emph{erasable programmable read-only memory})}
\end{entry*}

\begin{entry}{可卡因}{ke3ka3yin1}{5,5,6}[Radicais ⼝、⼘、⼞]
  \definition{s.}{(empréstimo linguístico) cocaína}
\end{entry}

\begin{entry}{可靠}{ke3kao4}{5,15}[HSK 3][Radicais ⼝、⾮]
  \definition{adj.}{confiável | verdadeiro; autêntico}
\end{entry}

\begin{entry}{可口可乐}{ke3kou3ke3le4}{5,3,5,5}[Radicais ⼝、⼝、⼝、⼃]
  \definition*{s.}{(empréstimo linguístico) Coca-Cola}
\end{entry}

\begin{entry}{可乐}{ke3 le4}{5,5}[HSK 3][Radicais ⼝、⼃]
  \definition*{s.}{\emph{coke}; coca; coca-cola}
\end{entry}

\begin{entry}{可能}{ke3neng2}{5,10}[HSK 2][Radicais ⼝、⾁]
  \definition{adj.}{possível | provável}
  \definition{adv.}{possivelmente | provavelmente}
  \definition[个]{s.}{possibilidade | probabilidade}
\end{entry}

\begin{entry}{可怕}{ke3pa4}{5,8}[HSK 2][Radicais ⼝、⼼]
  \definition{adj.}{horrível | terrível | formidável | assustador | hediondo}
  \definition{adv.}{terrivelmente}
\end{entry}

\begin{entry}{可是}{ke3shi4}{5,9}[HSK 2][Radicais ⼝、⽇]
  \definition{adv.}{(usado para dar ênfase) de fato}
  \definition{conj.}{porém | contudo | mas}
\end{entry}

\begin{entry}{可惜}{ke3xi1}{5,11}[Radicais ⼝、⼼]
  \definition{adj.}{é uma pena | que pena}
  \definition{adv.}{infelizmente | que pena | é uma pena}
\end{entry}

\begin{entry}{可以}{ke3yi3}{5,4}[HSK 2][Radicais ⼝、⼈]
  \definition{v.}{ser capaz de | poder}
\end{entry}

\begin{entry}{渴}{ke3}{12}[HSK 1][Radical ⽔]
  \definition{adj.}{sedento}
\end{entry}

\begin{entry}{克}{ke4}{7}[HSK 2][Radical ⼗]
  \definition{clas.}{grama (g)}
  \definition{v.}{pode | ser capaz de | restringir | controlar | superar | subjugar | capturar (uma cidade, etc.) | digerir | cortar | reduzir | definir um limite de tempo}
\end{entry}

\begin{entry}{克服}{ke4fu2}{7,8}[HSK 3][Radicais ⼗、⽉]
  \definition{v.}{sobrepujar; superar; conquistar | suportar (dificuldades, inconveniências, etc.)}
\end{entry}

\begin{entry}{刻}{ke4}{8}[HSK 2][Radical ⼑]
  \definition{clas.}{para curtos intervalos de tempo}
  \definition{s.}{quarto (de hora)}
  \definition{v.}{esculpir | cortar | gravar}
\end{entry}

\begin{entry}{刻画}{ke4hua4}{8,8}[Radicais ⼑、⽥]
  \definition{v.}{retratar | tirar um retrato}
\end{entry}

\begin{entry}{刻钟}{ke4 zhong1}{8,9}[Radicais ⼑、⾦]
  \definition{s.}{um quarto de hora}
\end{entry}

\begin{entry}{客观}{ke4guan1}{9,6}[HSK 3][Radicais ⼧、⾒]
  \definition{adj.}{objetivo; justo e razoável; imparcial}
  \definition{s.}{objetivo}
\end{entry}

\begin{entry}{客气}{ke4qi5}{9,4}[Radicais ⼧、⽓]
  \definition{adj.}{cortês | delicado | modesto | educado}
  \definition{v.}{fazer cerimônia}
\end{entry}

\begin{entry}{客人}{ke4ren2}{9,2}[HSK 2][Radicais ⼧、⼈]
  \definition{s.}{visitante | convidado | cliente | passageiro | viajante}
\end{entry}

\begin{entry}{客厅}{ke4ting1}{9,4}[Radicais ⼧、⼚]
  \definition[间]{s.}{sala de estar | sala de visitas}
\end{entry}

\begin{entry}{课}{ke4}{10}[HSK 1][Radical ⾔]
  \definition{s.}{aula | curso | lição | imposto | taxa |seção}
\end{entry}

\begin{entry}{课本}{ke4 ben3}{10,5}[HSK 1][Radicais ⾔、⽊]
  \definition[本]{s.}{livro do aluno | manual}
\end{entry}

\begin{entry}{课程}{ke4cheng2}{10,12}[HSK 3][Radicais ⾔、⽲]
  \definition[个,堂,节,门]{s.}{curso; currículo}
\end{entry}

\begin{entry}{课堂}{ke4 tang2}{10,11}[HSK 2][Radicais ⾔、⼟]
  \definition[间]{s.}{sala de aula}
\end{entry}

\begin{entry}{课文}{ke4 wen2}{10,4}[HSK 1][Radicais ⾔、⽂]
  \definition{s.}{texto (de uma lição)}
\end{entry}

\begin{entry}{肯定}{ken3ding4}{8,8}[Radicais ⾁、⼧]
  \definition{adv.}{com certeza | certamente | definitivamente | afirmativo (resposta)}
  \definition{v.}{afirmar | ter a certeza | ser positivo | dar reconhecimento}
\end{entry}

\begin{entry}{坑}{keng1}{7}[Radical ⼟]
  \definition{s.}{poço | depressão | túnel | buraco no chão}
  \definition{v.}{enganar | trapacear}
\end{entry}

\begin{entry}{坑人}{keng1ren2}{7,2}[Radicais ⼟、⼈]
  \definition{v.+compl.}{trapacear alguém}
\end{entry}

\begin{entry}{空}{kong1}{8}[HSK 3][Radical ⽳]
  \definition*{s.}{sobrenome Kong}
  \definition{adj.}{vazio; oco; nulo}
  \definition{adv.}{por nada; em vão}
  \definition{s.}{céu; ar | vazio; vazio do mundo dos sentidos}
  \seeref{空}{kong4}
\end{entry}

\begin{entry}{空间}{kong1jian1}{8,7}[Radicais ⽳、⾨]
  \definition{s.}{espaço | sala | (figurativo) escopo | (astronomia) espaço sideral | (matemática, física) espaço}
\end{entry}

\begin{entry}{空间站}{kong1jian1zhan4}{8,7,10}[Radicais ⽳、⾨、⽴]
  \definition{s.}{estação espacial}
\end{entry}

\begin{entry}{空姐}{kong1jie3}{8,8}[Radicais ⽳、⼥]
  \definition{s.}{aeromoça | comissária de bordo | abreviação de 空中小姐}
  \seeref{空中小姐}{kong1zhong1xiao3jie3}
\end{entry}

\begin{entry}{空气}{kong1qi4}{8,4}[HSK 2][Radicais ⽳、⽓]
  \definition{s.}{ar | atmosfera}
\end{entry}

\begin{entry}{空调}{kong1tiao2}{8,10}[HSK 3][Radicais ⽳、⾔]
  \definition[台]{s.}{ar-condicionado;  condicionador de ar}
\end{entry}

\begin{entry}{空心菜}{kong1xin1cai4}{8,4,11}[Radicais ⽳、⼼、⾋]
  \definition{s.}{espinafre aquático | \emph{ong choy} | repolho do pântano | convolvulus aquático | glória-da-manhã aquática}
  \seealsoref{蕹菜}{weng4cai4}
\end{entry}

\begin{entry}{空中小姐}{kong1zhong1xiao3jie3}{8,4,3,8}[Radicais ⽳、⼁、⼩、⼥]
  \definition{s.}{aeromoça}
\end{entry}

\begin{entry}{孔}{kong3}{4}[Radical ⼦]
  \definition*{s.}{sobrenome Kong}
  \definition{clas.}{para habitações em cavernas}
  \definition[个]{s.}{buraco}
\end{entry}

\begin{entry}{孔夫子}{kong3fu1zi3}{4,4,3}[Radicais ⼦、⼤、⼦]
  \definition*{s.}{Confúcio (551-479 aC), pensador e filósofo social chinês}
  \seealsoref{孔子}{kong3zi3}
\end{entry}

\begin{entry}{孔雀}{kong3que4}{4,11}[Radicais ⼦、⾫]
  \definition{s.}{pavão}
\end{entry}

\begin{entry}{孔子}{kong3zi3}{4,3}[Radicais ⼦、⼦]
  \definition*{s.}{Confúcio (551-479 aC), pensador e filósofo social chinês}
  \seealsoref{孔夫子}{kong3fu1zi3}
\end{entry}

\begin{entry}{孔子学院}{kong3zi3 xue2yuan4}{4,3,8,9}[Radicais ⼦、⼦、⼦、⾩]
  \definition*{s.}{Instituto Confúcio, organização estabelecida internacionalmente pela República Popular da China, que promove a língua e a cultura chinesas}
\end{entry}

\begin{entry}{恐怖主义}{kong3bu4zhu3yi4}{10,8,5,3}[Radicais ⼼、⼼、⼂、⼂]
  \definition{adj.}{terrorista}
  \definition{s.}{terrorismo}
\end{entry}

\begin{entry}{恐龙}{kong3long2}{10,5}[Radicais ⼼、⿓]
  \definition[头,只]{s.}{dinossauro}
\end{entry}

\begin{entry}{恐怕}{kong3pa4}{10,8}[HSK 3][Radicais ⼼、⼼]
  \definition{adv.}{talvez; provavelmente; pode ser | por medo de}
  \definition{v.}{ter medo de; temer; recear}
\end{entry}

\begin{entry}{空}{kong4}{8}[Radical ⽳]
  \definition{adj.}{desocupado; vago; em branco}
  \definition{s.}{espaço vazio | tempo livre}
  \definition{v.}{deixar em branco ou vazio; esvaziar; desocupar}
  \seeref{空}{kong1}
\end{entry}

\begin{entry}{空儿}{kong4r5}{8,2}[HSK 3][Radicais ⽳、⼉]
  \definition{s.}{tempo livre | espaço (não utilizado)}
  \definition{v.}{ter tempo livre}
\end{entry}

\begin{entry}{控制}{kong4zhi4}{11,8}[Radicais ⼿、⼑]
  \definition{v.}{controlar}
\end{entry}

\begin{entry}{口}{kou3}{3}[HSK 1][Kangxi 30][Radical ⼝]
  \definition{clas.}{para coisas com bocas (pessoas, animais domésticos, canhões, etc.) | para mordidas ou bocados}
  \definition{s.}{boca}
\end{entry}

\begin{entry}{口袋}{kou3dai4}{3,11}[Radicais ⼝、⾐]
  \definition{s.}{bolso | saco}
\end{entry}

\begin{entry}{口袋妖怪}{kou3dai4 yao1guai4}{3,11,7,8}[Radicais ⼝、⾐、⼥、⼼]
  \definition*{s.}{\emph{Pokémon}}
\end{entry}

\begin{entry}{口香糖}{kou3xiang1tang2}{3,9,16}[Radicais ⼝、⾹、⽶]
  \definition{s.}{goma de mascar | chiclete}
\end{entry}

\begin{entry}{口音}{kou3yin1}{3,9}[Radicais ⼝、⾳]
  \definition{s.}{sons da fala oral (linguística)}
  \seeref{口音}{kou3yin5}
\end{entry}

\begin{entry}{口音}{kou3yin5}{3,9}[Radicais ⼝、⾳]
  \definition{s.}{sotaque | voz}
  \seeref{口音}{kou3yin1}
\end{entry}

\begin{entry}{口语}{kou3yu3}{3,9}[Radicais ⼝、⾔]
  \definition[门]{s.}{linguagem oral | linguagem falada | fofoca | calúnia}
\end{entry}

\begin{entry}{枯木}{ku1mu4}{9,4}[Radicais ⽊、⽊]
  \definition{s.}{árvore morta | madeira morta}
\end{entry}

\begin{entry}{哭}{ku1}{10}[HSK 2][Radical ⼝]
  \definition{v.}{chorar}
\end{entry}

\begin{entry}{哭墙}{ku1qiang2}{10,14}[Radicais ⼝、⼟]
  \definition*{s.}{Muro das Lamentações (Jerusalém)}
\end{entry}

\begin{entry}{苦瓜}{ku3gua1}{8,5}[Radicais ⾋、⽠]
  \definition{s.}{melão amargo (cabaça amarga, pêra bálsamo, maçã bálsamo, pepino amargo)}
\end{entry}

\begin{entry}{裤子}{ku4zi5}{12,3}[HSK 3][Radicais ⾐、⼦]
  \definition[条]{s.}{calças}
\end{entry}

\begin{entry}{酷}{ku4}{14}[Radical ⾣]
  \definition{adj.}{impiedoso | forte (por exemplo, vinho) | (empréstimo linguístico) legal, \emph{cool}}
\end{entry}

\begin{entry}{酷斯拉}{ku4si1la1}{14,12,8}[Radicais ⾣、⽄、⼿]
  \definition*{s.}{Godzilla (Japonês ゴジラ Gojira)}
  \seealsoref{哥斯拉}{ge1si1la1}
\end{entry}

\begin{entry}{会}{kuai4}{6}[Radical ⼈]
  \definition{s.}{contabilidade}
  \definition{v.}{equilibrar uma conta}
  \seeref{会}{hui4}
\end{entry}

\begin{entry}{块}{kuai4}{7}[HSK 1][Radical ⼟]
  \definition{clas.}{(coloquial) para dinheiro e unidades monetárias | para peças ou pedaços de roupa, bolos, sabão, etc.}
  \definition{s.}{pedaço | pedaço (de terra) | peça}
\end{entry}

\begin{entry}{快}{kuai4}{7}[HSK 1][Radical ⼼]
  \definition{adj.}{quase | rápido | depressa}
  \definition{v.}{apressar-se}
\end{entry}

\begin{entry}{快餐}{kuai4 can1}{7,16}[HSK 2][Radicais ⼼、⾷]
  \definition[份,顿]{s.}{comida rápida | \emph{fast food}}
\end{entry}

\begin{entry}{快递}{kuai4di4}{7,10}[Radicais ⼼、⾡]
  \definition{s.}{entrega expressa}
\end{entry}

\begin{entry}{快点儿}{kuai4 dian3r5}{7,9,2}[HSK 2][Radicais ⼼、⽕、⼉]
  \definition{v.}{apressar-se}
\end{entry}

\begin{entry}{快乐}{kuai4le4}{7,5}[HSK 2][Radicais ⼼、⼃]
  \definition{adj.}{feliz | alegre}
  \definition{s.}{felicidade | alegria}
\end{entry}

\begin{entry}{快速}{kuai4 su4}{7,10}[HSK 3][Radicais ⼼、⾡]
  \definition{adj.}{rápido; veloz; de alta velocidade}
\end{entry}

\begin{entry}{快要}{kuai4 yao4}{7,9}[HSK 2][Radicais ⼼、⾑]
  \definition{adv.}{estar prestes a | estar indo para | estar à beira de | em breve | em nenhum momento}
\end{entry}

\begin{entry}{筷子}{kuai4zi5}{13,3}[HSK 2][Radicais ⽵、⼦]
  \definition[对,根,把,双]{s.}{pauzinhos | \emph{chopsticks}}
\end{entry}

\begin{entry}{宽影片}{kuan1ying3pian4}{10,15,4}[Radicais ⼧、⼺、⽚]
  \definition{s.}{filme \emph{widescreen}}
\end{entry}

\begin{entry}{款}{kuan3}{12}[Radical ⽋]
  \definition{clas.}{para versões ou modelos (de um produto)}
  \definition[笔,个]{s.}{montante de dinheiro | fundos | parágrafo | seção}
\end{entry}

\begin{entry}{窾}{kuan3}{17}[Radical ⽳]
  \definition{adj.}{oco}
  \definition{s.}{rachadura | cavidade | (onomatopéia) água atingindo a rocha}
  \definition{v.}{escavar um buraco}
  \seeref{窾}{cuan4}
\end{entry}

\begin{entry}{狂欢节}{kuang2huan1jie2}{7,6,5}[Radicais ⽝、⽋、⾋]
  \definition*{s.}{Carnaval}
\end{entry}

\begin{entry}{况且}{kuang4qie3}{7,5}[Radicais ⼎、⼀]
  \definition{conj.}{além disso | além do mais}
\end{entry}

\begin{entry}{旷野}{kuang4ye3}{7,11}[Radicais ⽇、⾥]
  \definition{s.}{região selvagem}
\end{entry}

\begin{entry}{矿泉水}{kuang4quan2shui3}{8,9,4}[Radicais ⽯、⽔、⽔]
  \definition[瓶,杯]{s.}{água mineral}
\end{entry}

\begin{entry}{葵花}{kui2hua1}{12,7}[Radicais ⾋、⾋]
  \definition{s.}{girassol (flor)}
\end{entry}

\begin{entry}{困}{kun4}{7}[HSK 3][Radical ⼞]
  \definition{adj.}{cansado | sonolento}
  \definition{v.}{estar encalhado; estar em grande pressão | cercar; prender; sitiar; cercar; rodear}
\end{entry}

\begin{entry}{困难}{kun4nan5}{7,10}[HSK 3][Radicais ⼞、⾫]
  \definition{adj.}{dificuldades financeiras; circunstâncias difíceis | complicado; nodoso; difícil; duro;}
  \definition[种]{s.}{dificuldade; situação difícil}
\end{entry}

%%%%% EOF %%%%%


%%%
%%% L
%%%

\section*{L}\addcontentsline{toc}{section}{L}

\begin{entry}{垃圾}{la1 ji1}{8,6}{⼟、⼟}[HSK 4]
  \definition{adj.}{lixo; inútil, ruim ou prejudicial}
  \definition[个]{s.}{entulho; lixo; refugo; rejeito; resíduo; coisa inútil que é jogada fora; metáfora para alguém ou algo que perdeu seu valor ou serve a um propósito ruim}
\end{entry}

\begin{entry}{垃圾车}{la1ji1che1}{8,6,4}{⼟、⼟、⾞}
  \definition{s.}{caminhão de lixo}
\end{entry}

\begin{entry}{垃圾电邮}{la1ji1dian4you2}{8,6,5,7}{⼟、⼟、⽥、⾢}
  \definition{s.}{\emph{e-mail} de \emph{spam}}
\end{entry}

\begin{entry}{垃圾堆}{la1ji1dui1}{8,6,11}{⼟、⼟、⼟}
  \definition{s.}{depósito de lixo}
\end{entry}

\begin{entry}{垃圾工}{la1ji1gong1}{8,6,3}{⼟、⼟、⼯}
  \definition{s.}{lixeiro | gari}
\end{entry}

\begin{entry}{垃圾食品}{la1ji1shi2pin3}{8,6,9,9}{⼟、⼟、⾷、⼝}
  \definition{s.}{\emph{junk food}}
\end{entry}

\begin{entry}{垃圾筒}{la1ji1tong3}{8,6,12}{⼟、⼟、⽵}
  \definition{s.}{cesto de lixo}
\end{entry}

\begin{entry}{垃圾箱}{la1ji1xiang1}{8,6,15}{⼟、⼟、⾋}
  \definition{s.}{cesto de lixo}
\end{entry}

\begin{entry}{垃圾邮件}{la1ji1you2jian4}{8,6,7,6}{⼟、⼟、⾢、⼈}
  \definition{s.}{\emph{spam}, \emph{e-mail} não solicitado}
\end{entry}

\begin{entry}{拉}{la1}{8}{⼿}[HSK 2]
  \definition{s.}{abreviação de América Latina, 拉丁美洲}
  \definition{v.}{puxar; arrastar; rebocar | transportar por veículo; rebocar | arrastar (ou puxar) para fora | mover (tropas para um lugar) | dar uma mãozinha; ajudar | arrastar para dentro; implicar; envolver | criar (criança) | atrair; conquistar; solicitar; angariar votos | bater-papo | organizar; preparar | ter dívidas; estar endividado | pressionar; recrutar à força | (no tênis, tênis de mesa, etc.) levantar (a bola) | tocar (certos instrumentos musicais); puxar uma parte do instrumento para que ele emita som | prolongar; espaçar | envolver-se em | (coloquial) esvaziar os intestinos | levantar, uma das técnicas do tênis de mesa | destruir; esmagar; quebrar}
  \seeref{拉}{la4}
  \seealsoref{拉丁美洲}{la1ding1 mei3zhou1}
\end{entry}

\begin{entry}{拉丁美洲}{la1ding1 mei3zhou1}{8,2,9,9}{⼿、⼀、⽺、⽔}
  \definition*{s.}{América Latina; o nome coletivo dos países da América Central e do Sul é ``América Latina'', devido ao fato de a maioria de seus habitantes ser descendente de povos latinos e de a língua falada ser do grupo latino}
\end{entry}

\begin{entry}{拉开}{la1 kai1}{8,4}{⼿、⼶}[HSK 4]
  \definition{v.}{puxar para abrir; recuar| ampliar; espaçar; distanciar; afastar; separar}
\end{entry}

\begin{entry}{拉拉队}{la1la1dui4}{8,8,4}{⼿、⼿、⾩}
  \definition{s.}{claque | torcida}
\end{entry}

\begin{entry}{拉}{la4}{8}{⼿}
  \definition{s.}{usado em 拉拉蛄 \dpy{la4la4gu3}}
  \seeref{拉}{la1}
  \seealsoref{拉拉蛄}{la4la4gu3}
\end{entry}

\begin{entry}{拉拉蛄}{la4la4gu3}{8,8,11}{⼿、⼿、⾍}
  \variantof{蝲蝲蛄}
\end{entry}

\begin{entry}{落}{la4}{12}{⾋}[HSK 5]
  \definition{v.}{deixar de fora; estar ausente | deixar para trás; esquecer de trazer; deixar algo em algum lugar e esquecer de levar| ficar para trás (ou cair); não conseguir acompanhar}
  \seeref{落}{lao4}
  \seeref{落}{luo4}
\end{entry}

\begin{entry}{蜡烛}{la4zhu2}{14,10}{⾍、⽕}
  \definition[根,支]{s.}{vela | círio | peça, geralmente de cera, que possui um pavio e se utiliza para iluminar}
\end{entry}

\begin{entry}{辣}{la4}{14}{⾟}[HSK 4]
  \definition{adj.}{apimentado; picante; pungente; quente | cruel; implacável; venenoso; vicioso}
  \definition{v.}{queimar; picar; formigar; ter uma irritação picante (boca, nariz ou olhos)}
\end{entry}

\begin{entry}{蝲蝲蛄}{la4la4gu3}{15,15,11}{⾍、⾍、⾍}
  \definition{s.}{grilo toupeira}
\end{entry}

\begin{entry}{来}{lai2}{7}{⽊}[HSK 1]
  \definition*{s.}{sobrenome Lai}
  \definition{part.}{usado após uma palavra numérica ou de quantidade; indica uma quantidade aproximada | usado depois de numerais como 一, 二, 三; para listar razões ou fatos, etc.}
  \definition{s.}{usado após uma expressão de tempo para indicar uma duração que vai do passado ao presente}
  \definition{v.}{vir; chegar; de outro lugar para o lugar onde o interlocutor se encontra | aparecer; acontecer; vir; (problemas, coisas, etc.) ocorrerem; surgirem | substitui um verbo com significado específico, indicando a realização de uma ação específica | estar indo para; usado antes de outro verbo, indica que algo será feito | vir para fazer algo; usado após outro verbo, indica que se vai fazer algo | usado para indicar um propósito; expressar o objetivo, fazer algo usando o método, a atitude ou a direção anteriores | usado com 得 ou 不 para indicar possibilidade, capacidade ou hábito}
  \seealsoref{不}{bu4}
  \seealsoref{得}{de5}
\end{entry}

\begin{entry}{来不及}{lai2bu5ji2}{7,4,3}{⽊、⼀、⼃}[HSK 4]
  \definition{v.}{ser tarde demais; não ter tempo; não ter tempo suficiente (para fazer algo); não ser possível participar ou se atualizar devido a restrições de tempo}
\end{entry}

\begin{entry}{来到}{lai2 dao4}{7,8}{⽊、⼑}[HSK 1]
  \definition{v.}{chegar; vir}
\end{entry}

\begin{entry}{来得及}{lai2de5ji2}{7,11,3}{⽊、⼻、⼃}[HSK 4]
  \definition{v.}{ainda ter tempo; ser capaz de fazê-lo; ser capaz de fazer algo a tempo; ainda ter tempo de chegar lá ou de se atualizar}
\end{entry}

\begin{entry}{来信}{lai2 xin4}{7,9}{⽊、⼈}[HSK 5]
  \definition{s.}{sua carta; carta recebida; carta ao interlocutor}
  \definition{v.}{enviar uma carta para aqui; enviar uma carta para o remetente}
\end{entry}

\begin{entry}{来源}{lai2yuan2}{7,13}{⽊、⽔}[HSK 4]
  \definition{s.}{origem; causa; fonte; tabula rasa (ou seja, o lugar de onde as coisas vêm)}
  \definition{v.}{originar-se; surgir; ter origem; (algo) originar (seguido de 于)}
  \seealsoref{于}{yu2}
\end{entry}

\begin{entry}{来自}{lai2zi4}{7,6}{⽊、⾃}[HSK 2]
  \definition{v.}{vir de (um local) | \emph{From:} (cabeçalho de \emph{e -mail})}
\end{entry}

\begin{entry}{赖}{lai4}{13}{⾙}
  \definition*{s.}{sobrenome Lai}
  \definition{v.}{depender | aguentar em um lugar | renegar (promessa) | isolar-se | culpar | colocar a culpa em}
\end{entry}

\begin{entry}{兰花}{lan2hua1}{5,7}{⼋、⾋}
  \definition{s.}{orquídea}
\end{entry}

\begin{entry}{蓝}{lan2}{13}{⾋}[HSK 2]
  \definition*{s.}{sobrenome Lan}
  \definition{adj.}{azul}
  \definition{s.}{planta índigo; anil | plantas azuis; refere-se a certas plantas que podem ser usadas como corante azul ou certas plantas cujas folhas são azul-esverdeadas}
\end{entry}

\begin{entry}{蓝色}{lan2 se4}{13,6}{⾋、⾊}[HSK 2]
  \definition[抹,片,缕,团,块]{s.}{cor azul}
\end{entry}

\begin{entry}{篮球}{lan2qiu2}{16,11}{⽵、⽟}[HSK 2]
  \definition[个,只]{s.}{basquetebol | bola de basquete; refere-se à bola utilizada no basquetebol}
\end{entry}

\begin{entry}{懒}{lan3}{16}{⼼}
  \definition{adj.}{preguiçoso | indolente | vadio}
\end{entry}

\begin{entry}{懒虫}{lan3chong2}{16,6}{⼼、⾍}
  \definition{s.}{desleixado ocioso | (insulto) sujeito preguiçoso}
\end{entry}

\begin{entry}{懒怠}{lan3dai4}{16,9}{⼼、⼼}
  \definition{s.}{preguiça}
\end{entry}

\begin{entry}{懒得}{lan3de5}{16,11}{⼼、⼻}
  \definition{adv.}{demasiado preguiçoso}
  \definition{v.}{não sentir vontade (de fazer algo)}
\end{entry}

\begin{entry}{懒惰}{lan3duo4}{16,12}{⼼、⼼}
  \definition{adj.}{preguiçoso}
\end{entry}

\begin{entry}{懒鬼}{lan3gui3}{16,9}{⼼、⿁}
  \definition{s.}{cara preguiçoso}
\end{entry}

\begin{entry}{懒汉}{lan3han4}{16,5}{⼼、⽔}
  \definition{s.}{sujeito ocioso | vagabundo | preguiçosos}
\end{entry}

\begin{entry}{懒人}{lan3ren2}{16,2}{⼼、⼈}
  \definition{s.}{pessoa preguiçosa}
\end{entry}

\begin{entry}{懒散}{lan3san3}{16,12}{⼼、⽁}
  \definition{adj.}{inativo | indolente | preguiçoso | negligente}
\end{entry}

\begin{entry}{懒腰}{lan3yao1}{16,13}{⼼、⾁}
  \definition[个]{s.}{alongamento (do corpo)}
\end{entry}

\begin{entry}{烂}{lan4}{9}{⽕}[HSK 5]
  \definition{adj.}{macio; pastoso; amassado | podre; deteriorado | quebrado; esfarrapado; gasto | desorganizado; indigno}
  \definition{adv.}{totalmente; extremamente; completamente; expressa um grau muito profundo}
  \definition{v.}{apodrecer; infeccionar; decompor-se}
\end{entry}

\begin{entry}{廊坊}{lang2fang2}{11,7}{⼴、⼟}
  \definition*{s.}{Cidade de nível de prefeitura de Langfang em Hebei}
\end{entry}

\begin{entry}{朗读}{lang3du2}{10,10}{⽉、⾔}[HSK 5]
  \definition{v.}{ler em voz alta; recitar com voz clara e alta}
\end{entry}

\begin{entry}{浪费}{lang4fei4}{10,9}{⽔、⾙}[HSK 3]
  \definition{adj.}{desperdiçado; extravagante; não econômico}
  \definition{v.}{desperdiçar; dissipar; esbanjar; ser extravagante; uso excessivo ou inadequado de bens, recursos humanos, tempo, etc.}
\end{entry}

\begin{entry}{浪花}{lang4hua1}{10,7}{⽔、⾋}
  \definition[朵]{s.}{\emph{spray} | \emph{spray} do oceano | (figurativo) acontecimentos de sua vida}
\end{entry}

\begin{entry}{浪漫}{lang4man4}{10,14}{⽔、⽔}[HSK 5]
  \definition{adj.}{romântico; poético | não convencional; boêmio; abandonado; libertino; devasso; comportar-se de maneira descuidada e descuidada (geralmente se referindo a relacionamentos entre homens e mulheres) | irrealista; impraticável}
\end{entry}

\begin{entry}{捞}{lao1}{10}{⼿}
  \definition{v.}{pescar | dragar}
\end{entry}

\begin{entry}{劳动}{lao2dong4}{7,6}{⼒、⼒}[HSK 5]
  \definition[次]{s.}{trabalho; mão de obra; atividades intelectuais ou físicas que podem criar valor | trabalho físico; trabalho manual; referindo-se especificamente ao trabalho físico}
  \definition{v.}{realizar trabalho físico}
\end{entry}

\begin{entry}{劳工同事}{lao2gong1 tong2shi4}{7,3,6,8}{⼒、⼯、⼝、⼅}
  \definition{s.}{colaborador | colega de trabalho}
\end{entry}

\begin{entry}{老}{lao3}{6}{⽼}[HSK 1,2][Kangxi 125]
  \definition*{s.}{sobrenome Lao}
  \definition{adj.}{velho; envelhecido; idade avançada | antigo; de longa data; existe há muito tempo | antigo; desatualizado; obsoleto; ultrapassado  | antigo; tradicional; original | coberto de vegetação; os vegetais cresceram além do período ideal para serem consumidos | resistente; endurecido; alimentos muito cozidos | escuro; profundo; (sobre cores) | último nascido; o mais novo | veterano; experiente; sofisticado}
  \definition{adv.}{longo; por muito tempo | sempre (fazendo algo) | muito}
  \definition{pref.}{usado para designar pessoas, ordem de classificação, certos nomes de animais e plantas}
  \definition{s.}{idosos; pessoas mais velhas | ancião; sênior; um título respeitoso para pessoas mais velhas}
  \definition{v.}{envelhecer | morrer; referindo-se à morte de um idoso}
\end{entry}

\begin{entry}{老百姓}{lao3bai3xing4}{6,6,8}{⽼、⽩、⼥}[HSK 3]
  \definition[些]{s.}{povo; civis; pessoas comuns; residentes (em contraste com militares e funcionários públicos)}
\end{entry}

\begin{entry}{老板}{lao3ban3}{6,8}{⽼、⽊}[HSK 3]
  \definition[个,位]{s.}{chefe; dono; líder; refere-se ao gerente de uma empresa comercial ou industrial | antigo título honorífico dado a atores famosos de ópera ou atores que também eram diretores de companhias de ópera}
\end{entry}

\begin{entry}{老兵}{lao3bing1}{6,7}{⽼、⼋}
  \definition{s.}{velho soldado | veterano de guerra | veterano (alguém que tem muita experiência em algum domínio)}
\end{entry}

\begin{entry}{老公}{lao3 gong1}{6,4}{⽼、⼋}[HSK 4]
  \definition[个]{s.}{marido; esposo}
\end{entry}

\begin{entry}{老虎}{lao3hu3}{6,8}{⽼、⾌}
  \definition[只]{s.}{tigre}
  \seealsoref{虎}{hu3}
\end{entry}

\begin{entry}{老家}{lao3 jia1}{6,10}{⽼、⼧}[HSK 4]
  \definition{s.}{cidade natal; local de origem | lugar nativo; refere-se às gerações anteriores da família ou ao local onde a pessoa nasceu ou viveu}
\end{entry}

\begin{entry}{老年}{lao3 nian2}{6,6}{⽼、⼲}[HSK 2]
  \definition[个]{s.}{idoso; velhice; idade acima de 60 ou 70 anos}
\end{entry}

\begin{entry}{老朋友}{lao3 peng2 you3}{6,8,4}{⽼、⽉、⼜}[HSK 2]
  \definition[个,位,名]{s.}{velho amigo; refere-se a amigos que conhecemos há muito tempo e com quem temos uma relação íntima}
\end{entry}

\begin{entry}{老婆}{lao3po2}{6,11}{⽼、⼥}[HSK 4]
  \definition[个]{s.}{esposa}
\end{entry}

\begin{entry}{老人}{lao3 ren2}{6,2}{⽼、⼈}[HSK 1]
  \definition[位]{s.}{homem ou mulher de idade avançada; o idoso; o velho}
\end{entry}

\begin{entry}{老人家}{lao3 ren2 jia1}{6,2,10}{⽼、⼈、⼧}
  \definition{s.}{senhor ancião | madame | senhora | termo educado para mulher ou homem velho}
\end{entry}

\begin{entry}{老师}{lao3shi1}{6,6}{⽼、⼱}[HSK 1]
  \definition[个,位]{s.}{professor; título honorífico para professores; refere-se, de maneira geral, a pessoas que transmitem cultura e tecnologia ou que são dignas de admiração em termos de ideias, moralidade e conhecimentos profissionais}
\end{entry}

\begin{entry}{老是}{lao3 shi4}{6,9}{⽼、⽇}[HSK 2]
  \definition{adv.}{sempre; indica que a ação continua ou que o estado permanece inalterado, equivalente a 一直}
  \seealsoref{一直}{yi4zhi2}
\end{entry}

\begin{entry}{老实}{lao3shi5}{6,8}{⽼、⼧}[HSK 4]
  \definition{adj.}{franco; sincero; honesto | bom; bem-comportado | ingênuo; simplório; meio bobo; facilmente enganado; eufemismo para pouco inteligente}
\end{entry}

\begin{entry}{老太太}{lao3 tai4 tai5}{6,4,4}{⽼、⼤、⼤}[HSK 3]
  \definition[位,名,个]{s.}{velha senhora; (em tratamento direto)Venerável Senhora; uma maneira respeitosa de chamar uma senhora idosa; título honorífico para mulheres idosas | (forma de tratamento) sua velha mãe; minha velha mãe, avó ou sogra; referindo-se à própria mãe, à mãe do outro ou à mãe de outra pessoa, à sogra ou à sogra política}
\end{entry}

\begin{entry}{老头儿}{lao3 tou2r5}{6,5,2}{⽼、⼤、⼉}[HSK 3]
  \definition{s.}{(coloquial) (com um tom de intimidade) velho; velho amigo}
  \seealsoref{老头子}{lao3 tou2zi5}
\end{entry}

\begin{entry}{老头子}{lao3 tou2zi5}{6,5,3}{⽼、⼤、⼦}
  \definition{s.}{velho antiquado (ou velho rabugento) | (referindo-se ao marido idoso) meu velho | chefe de uma sociedade secreta | (coloquial) velho; velho rabugento}
  \seealsoref{老头儿}{lao3 tou2r5}
\end{entry}

\begin{entry}{落}{lao4}{12}{⾋}
  \definition{v.}{cair; cair de uma altura elevada | se abaixar; descer; ir para baixo | permanecer; fazer uma parada; deixar para trás | obter; ter; receber}
  \seeref{落}{la4}
  \seeref{落}{lao4}
\end{entry}

\begin{entry}{乐}{le4}{5}{⼃}[HSK 3]
  \definition*{s.}{sobrenome Le}
  \definition{adj.}{feliz; contente; rejubilante; animado; bem disposto}
  \definition{s.}{prazer; diversão}
  \definition{v.}{desfrutar; ficar feliz em; amar; encontrar prazer em | rir; divertir-se}
  \seeref{乐}{yue4}
\end{entry}

\begin{entry}{乐高}{le4gao1}{5,10}{⼃、⾼}
  \definition*{s.}{Lego (brinquedo)}
\end{entry}

\begin{entry}{乐观}{le4guan1}{5,6}{⼃、⾒}[HSK 3]
  \definition{adj.}{esperançoso; otimista; confiante; espírito alegre, confiante no futuro (oposto a 悲观)}
  \seealsoref{悲观}{bei1guan1}
\end{entry}

\begin{entry}{乐趣}{le4qu4}{5,15}{⼃、⾛}[HSK 4]
  \definition[个,种,些,点]{s.}{alegria; deleite; prazer; implicação de fazer alguém se sentir feliz; um humor de preferência}
\end{entry}

\begin{entry}{乐园}{le4yuan2}{5,7}{⼃、⼞}
  \definition{s.}{paraíso}
\end{entry}

\begin{entry}{了}{le5}{2}{⼅}[HSK 1,3]
  \definition{part.}{usada após verbos ou adjetivos para indicar a conclusão de uma ação, em um momento no passado ou antes do início de outra ação, ou uma ação esperada ou presumida | usada para indicar uma mudança de situação ou estado, seja real ou prevista | comandos ou solicitações em resposta a uma situação alterada; usada para xpressar urgência ou dissuadir | usada para indicar que algo chegou ao extremo; usada no final da frase ou em pausas no meio da frase, para expressar um tom de exclamação}
  \seeref{了}{liao3}
\end{entry}

\begin{entry}{累}{lei2}{11}{⽷}
  \definition*{s.}{sobrenome Lei}
  \definition{adj.}{incômodo; complicado}
  \definition{s.}{corda; cordão | touro na época de acasalamento}
  \definition{v.}{amarrar; prender; atar | copular}
  \seeref{累}{lei3}
  \seeref{累}{lei4}
\end{entry}

\begin{entry}{雷电}{lei2dian4}{13,5}{⾬、⽥}
  \definition{s.}{trovão e relâmpago; raio}
\end{entry}

\begin{entry}{雷亚尔}{lei2ya4'er3}{13,6,5}{⾬、⼆、⼩}
  \definition*{s.}{Real Brasileiro}
\end{entry}

\begin{entry}{累}{lei3}{11}{⽷}
  \definition*{s.}{sobrenome Lei}
  \definition{adj.}{em andamento; repetido; contínuo}
  \definition{v.}{acumular; empilhar; colocar em cima de outro | envolver; implicar | construir empilhando tijolos, pedras, terra, etc.}
  \seeref{累}{lei2}
  \seeref{累}{lei4}
\end{entry}

\begin{entry}{絫}{lei3}{12}{⽷}
  \variantof{累}
\end{entry}

\begin{entry}{泪}{lei4}{8}{⽔}[HSK 4]
  \definition[滴]{s.}{lágrima}
\end{entry}

\begin{entry}{泪水}{lei4 shui3}{8,4}{⽔、⽔}[HSK 4]
  \definition{s.}{lágrima}
\end{entry}

\begin{entry}{类}{lei4}{9}{⽶}[HSK 3]
  \definition*{s.}{sobrenome Lei}
  \definition{clas.}{tipo; espécie; categoria usada para pessoas ou coisas}
  \definition{s.}{classe; categoria; tipo; variedade; a combinação de muitas coisas semelhantes ou iguais}
  \definition{v.}{assemelhar-se a; ser semelhante a}
\end{entry}

\begin{entry}{类似}{lei4si4}{9,6}{⽶、⼈}[HSK 3]
  \definition{adj.}{semelhante; análogo}
\end{entry}

\begin{entry}{类型}{lei4xing2}{9,9}{⽶、⼟}[HSK 4]
  \definition[种,个]{s.}{tipo; espécie; categoria; tipos formados por coisas com características comuns}
\end{entry}

\begin{entry}{累}{lei4}{11}{⽷}[HSK 1]
  \definition{adj.}{cansado; exausto; fatigado}
  \definition{v.}{cansar; desgastar; fatigar; esgotar | labutar; trabalhar duro}
  \seeref{累}{lei2}
  \seeref{累}{lei3}
\end{entry}

\begin{entry}{冷}{leng3}{7}{⼎}[HSK 1]
  \definition*{s.}{sobrenome Leng}
  \definition{adj.}{frio; baixa temperatura; sensação de frio | gelado; frio por natureza; sem entusiasmo; sem gentileza | desolado; pouco frequentado; quieto; sem agitação | negligenciado; indesejável; ignorado por todos | raro; estranho; incomum | feito em segredo; filmado de forma escondida; lançado secretamente}
  \definition{v.}{esfriar; resfriar | esfriar; congelar; tornar-se indiferente, apático | ignorar}
\end{entry}

\begin{entry}{冷静}{leng3jing4}{7,14}{⼎、⾭}[HSK 4]
  \definition{adj.}{calmo; descreve uma pessoa que consegue ficar atenta em uma situação importante ou de emergência e não toma decisões aleatórias por causa de seus sentimentos no momento | (lugar) tranquilo; quieto; deserto}
\end{entry}

\begin{entry}{冷门}{leng3men2}{7,3}{⼎、⾨}
  \definition{s.}{uma profissão, ofício ou ramo de aprendizagem que recebe pouca atenção | um vencedor inesperado; azarão}
\end{entry}

\begin{entry}{厘米}{li2mi3}{9,6}{⼚、⽶}[HSK 4]
  \definition{clas.}{centímetro; unidade de comprimento, símbolo cm, 1 metro é igual a 100 centímetros}
\end{entry}

\begin{entry}{离}{li2}{10}{⼇}[HSK 2]
  \definition*{s.}{sobrenome Li}
  \definition*{s.}{um dos Oito Diagramas}
  \definition{prep.}{(ser longe) de\dots até\dots}
  \definition{v.}{partir; separar-se; afastar-se; estar longe de | prescindir; dispensar; ser independente de | mudar de; desviar-se de | mudar de; desviar-se de; trair; ser incompatível}
\end{entry}

\begin{entry}{离不开}{li2 bu4 kai1}{10,4,4}{⼇、⼀、⼶}[HSK 4]
  \definition{v.}{não pode prescindir; ser inseparável de; não ser capaz de se separar ou deixar uma pessoa, coisa ou circunstância}
\end{entry}

\begin{entry}{离婚}{li2hun1}{10,11}{⼇、⼥}[HSK 3]
  \definition{v.+compl.}{divórciar; romper um casamento; obter o divórcio}
\end{entry}

\begin{entry}{离开}{li2kai1}{10,4}{⼇、⼶}[HSK 2]
  \definition{v.}{deixar; partir; desviar-se; separar-se das pessoas, dos lugares e das coisas}
\end{entry}

\begin{entry}{梨}{li2}{11}{⽊}[HSK 5]
  \definition*{s.}{sobrenome Li}
  \definition[个,颗]{s.}{perira; árvore de pera | pera}
\end{entry}

\begin{entry}{黎}{li2}{15}{⿉}
  \definition*{s.}{sobrenome Li}
  \definition*{s.}{a nacionalidade Li, uma das minorias nacionais da província de Hainan}
  \definition{adj.}{numeroso}
  \definition{s.}{multidão}
\end{entry}

\begin{entry}{礼}{li3}{5}{⽰}[HSK 5]
  \definition*{s.}{sobrenome Li}
  \definition[份]{s.}{observâncias cerimoniais em geral; cerimônia; rito | cortesia; etiqueta; boas maneiras | presente; oferta}
\end{entry}

\begin{entry}{礼拜}{li3 bai4}{5,9}{⽰、⼿}[HSK 5]
  \definition[个]{s.}{dia da semana; usado em conjunto com 一, 二, 三, 四, 五, 六, 日(或天, indica um dia específico da semana | semana; referência à semana | domingo}
  \definition{v.}{prestar homenagem aos deuses que veneram; rezar; orar}
\end{entry}

\begin{entry}{礼节}{li3jie2}{5,5}{⽰、⾋}
  \definition{s.}{protocolo | cerimônia | etiqueta}
\end{entry}

\begin{entry}{礼貌}{li3mao4}{5,14}{⽰、⾘}[HSK 5]
  \definition{adj.}{educado; descreve uma pessoa que fala e age respeitando os outros, sem arrogância, de acordo com as exigências das relações sociais}
  \definition{s.}{cortesia; educação; boas maneiras}
\end{entry}

\begin{entry}{礼让}{li3rang4}{5,5}{⽰、⾔}
  \definition{s.}{cortesia}
  \definition{v.}{mostrar consideração por (outros) | ceder a (outro veículo, etc.)}
\end{entry}

\begin{entry}{礼物}{li3wu4}{5,8}{⽰、⽜}[HSK 2]
  \definition[份,件,个,分,些]{s.}{presente; lembrança; itens oferecidos como forma de respeito ou celebração, referindo-se de maneira geral a itens oferecidos como presente}
\end{entry}

\begin{entry}{李}{li3}{7}{⽊}
  \definition*{s.}{sobrenome Li}
  \definition{s.}{ameixa}
\end{entry}

\begin{entry}{李四}{li3si4}{7,5}{⽊、⼞}
  \definition*{s.}{Li Si | Zé Ninguém | nome para uma pessoa não especificada, 2 de 3}
  \seealsoref{王五}{wang2wu3}
  \seealsoref{张三}{zhang1san1}
\end{entry}

\begin{entry}{李子}{li3zi5}{7,3}{⽊、⼦}
  \definition[个]{s.}{ameixa}
\end{entry}

\begin{entry}{里}{li3}{7}{⾥}[HSK 1][Kangxi 166]
  \definition*{s.}{sobrenome Li}
  \definition{clas.}{li, uma unidade chinesa de comprimento (= 1/2 quilômetro)}
  \definition{s.}{forro; revestimento; interior; parte de trás do tecido | interno; dentro; no interior | vizinhança; vizinhos | cidade natal; local de origem}
\end{entry}

\begin{entry}{里边}{li3 bian5}{7,5}{⾥、⾡}[HSK 1]
  \definition{s.}{em; dentro; no interior}
\end{entry}

\begin{entry}{里面}{li3 mian4}{7,9}{⾥、⾯}[HSK 3]
  \definition{s.}{dentro; interior}
\end{entry}

\begin{entry}{里斯本}{li3si1ben3}{7,12,5}{⾥、⽄、⽊}
  \definition*{s.}{Lisboa}
\end{entry}

\begin{entry}{里斯本大学}{li3si1ben3 da4xue2}{7,12,5,3,8}{⾥、⽄、⽊、⼤、⼦}
  \definition*{s.}{Universidade de Lisboa}
\end{entry}

\begin{entry}{里头}{li3 tou5}{7,5}{⾥、⼤}[HSK 2]
  \definition{s.}{dentro}
\end{entry}

\begin{entry}{理发}{li3fa4}{11,5}{⽟、⼜}[HSK 3]
  \definition{v.+compl.}{cortar e aparar o cabelo; ter (dar) um corte de cabelo}
\end{entry}

\begin{entry}{理解}{li3jie3}{11,13}{⽟、⾓}[HSK 3]
  \definition{v.}{entender; compreender; compreender o significado por trás de algo através da reflexão e do aprendizado | entender com empatia; achar que os outros não conseguem fazer determinada coisa e demonstrar compaixão, perdão e não crítica}
\end{entry}

\begin{entry}{理论}{li3lun4}{11,6}{⽟、⾔}[HSK 3]
  \definition[套,个]{s.}{teoria; uma série de conclusões tiradas pelas pessoas sobre atividades naturais ou sociais}
  \definition{v.}{argumentar; raciocinar com alguém; discutir com outras pessoas sobre quem está certo ou errado}
\end{entry}

\begin{entry}{理想}{li3xiang3}{11,13}{⽟、⼼}[HSK 2]
  \definition{adj.}{ideal; perfeito | conforme o desejado; satisfatório}
  \definition{adv.}{idealmente}
  \definition[个,种]{s.}{ideal; sonho; aspiração}
\end{entry}

\begin{entry}{理由}{li3you2}{11,5}{⽟、⽥}[HSK 3]
  \definition[个,条,种,堆]{s.}{razão; justificativa; fundamento; a razão pela qual as coisas são feitas desta ou daquela maneira}
\end{entry}

\begin{entry}{力}{li4}{2}{⼒}[HSK 3][Kangxi 19]
  \definition*{s.}{sobrenome Li}
  \definition{adj.}{forte; eficiente; capaz | forte; poderoso; referência geral à função das coisas}
  \definition{adv.}{energicamente; arduamente; vigorosamente; com todo o esforço; com toda a dedicação}
  \definition{s.}{força; energia; poder; (física) refere-se à ação de alterar o estado de movimento ou a forma de um objeto |poder; força; habilidade; capacidade; funções dos órgãos do corpo humano | força física; resistência física}
\end{entry}

\begin{entry}{力量}{li4liang5}{2,12}{⼒、⾥}[HSK 3]
  \definition[出]{s.}{força física; força espiritual | habilidade; capacidade | eficácia; efeito | força (pessoa ou grupo que tem muito poder ou influência); referência a uma pessoa ou grupo que pode desempenhar um papel importante}
\end{entry}

\begin{entry}{力气}{li4qi5}{2,4}{⼒、⽓}[HSK 4]
  \definition[点,把]{s.}{força física | esforço}
\end{entry}

\begin{entry}{历史}{li4shi3}{4,5}{⼚、⼝}[HSK 4]
  \definition[个,门,段]{s.}{histórico; registro do passado; processo de desenvolvimento da natureza e da sociedade humana; processo de desenvolvimento de uma coisa ou pessoa | história; eventos passados; experiência | história; refere-se ao tema da história}
\end{entry}

\begin{entry}{厉害}{li4hai5}{5,10}{⼚、⼧}[HSK 5]
  \definition{adj.}{feroz; severo; descreve uma situação como sendo muito grave | severo; duro; descreve uma pessoa que é exigente com os outros, muito severa, muitas vezes deixando os outros um pouco assustados | incrível; talentoso; impressionante; usado para avaliar a capacidade de uma pessoa ou algo que ela fez que é notável | aterrorizante; assustador; descreve animais ferozes e assustadores.}
\end{entry}

\begin{entry}{立}{li4}{5}{⽴}[HSK 5]
  \definition{adj.}{ereto; vertical; na vertical}
  \definition{adv.}{imediatamente; instantaneamente}
  \definition{v.}{ficar em pé, com os pés no chão ou apoiados em algum objeto; o objeto deve estar na vertical | erguer; colocar (ou levantar) algo; colocar em pé | encontrar; criar; elaborar; formular; estabelecer | configurar; fundar; estabelecer | viver; existir | ascender ao trono; antigamente, referia-se à ascensão ao trono de um monarca | nomear; designar; antigamente, significava estabelecer uma determinada posição ou status}
\end{entry}

\begin{entry}{立场}{li4chang3}{5,6}{⽴、⼟}[HSK 5]
  \definition[个]{s.}{posição; postura; a posição e a atitude adotadas ao reconhecer e lidar com os problemas | ponto de vista; refere-se especificamente à atitude de reconhecer e lidar com questões a partir dos interesses de uma determinada classe, ou seja, a posição de classe}
\end{entry}

\begin{entry}{立法}{li4fa3}{5,8}{⽴、⽔}
  \definition{s.}{legislação}
  \definition{v.}{promulgar leis | legislar}
\end{entry}

\begin{entry}{立即}{li4ji2}{5,7}{⽴、⼙}[HSK 4]
  \definition{adv.}{prontamente; imediatamente; de imediato}
\end{entry}

\begin{entry}{立刻}{li4ke4}{5,8}{⽴、⼑}[HSK 3]
  \definition{adv.}{imediatamente; de ​​uma vez; indica que algo acontecerá imediatamente após um determinado momento}
\end{entry}

\begin{entry}{利}{li4}{7}{⼑}
  \definition*{s.}{sobrenome Li}
  \definition{adj.}{afiado; cortante | favorável; conveniente; sem dificuldades; sem ou com poucas dificuldades}
  \definition{s.}{benefício; vantagem | lucro; ganhos; juros}
  \definition{v.}{beneficiar; tornar vantajoso}
\end{entry}

\begin{entry}{利润}{li4run4}{7,10}{⼑、⽔}[HSK 5]
  \definition[笔]{s.}{lucro; o dinheiro ganho com atividades comerciais e industriais}
\end{entry}

\begin{entry}{利息}{li4xi1}{7,10}{⼑、⼼}[HSK 4]
  \definition{s.}{acréscimo; juros; dinheiro recebido além do valor principal como resultado de depósitos ou empréstimos (diferenciado de 本金)}
  \seealsoref{本金}{ben3 jin1}
\end{entry}

\begin{entry}{利益}{li4yi4}{7,10}{⼑、⽫}[HSK 4]
  \definition[个,种]{s.}{ganho; lucro; juros; benefício}
\end{entry}

\begin{entry}{利用}{li4yong4}{7,5}{⼑、⽤}[HSK 3]
  \definition{v.}{usar; utilizar; fazer uso de; fazer com que algo ou alguém funcione bem| explorar; tirar vantagem de; usar meios para fazer com que pessoas ou coisas sirvam aos seus interesses}
\end{entry}

\begin{entry}{例如}{li4ru2}{8,6}{⼈、⼥}[HSK 2]
  \definition{conj.}{por exemplo; tal como; como por exemplo; colocado antes do exemplo, indica que o exemplo vem a seguir}
\end{entry}

\begin{entry}{例外}{li4wai4}{8,5}{⼈、⼣}[HSK 5]
  \definition[个]{s.}{exceção; situações que não se enquadram nas regras gerais ou nas leis comuns}
  \definition{v.}{ser excepcional; ser uma exceção}
\end{entry}

\begin{entry}{例子}{li4 zi5}{8,3}{⼈、⼦}[HSK 2]
  \definition[个]{s.}{exemplo; algo usado para ajudar a explicar ou provar uma determinada situação ou afirmação}
\end{entry}

\begin{entry}{隶}{li4}{8}{⾪}[Kangxi 171]
  \definition*{s.}{sobrenome Li}
  \definition{s.}{escravo; uma pessoa em servidão; uma pessoa humilde | um dos estilos antigos da caligrafia chinesa}
  \definition{v.}{estar subordinado a}
\end{entry}

\begin{entry}{荔枝}{li4zhi1}{9,8}{⾋、⽊}
  \definition{s.}{lichia}
\end{entry}

\begin{entry}{鬲}{li4}{10}{⿀}[Kangxi 193]
  \definition{s.}{um antigo tripé de cozinha com pernas ocas; uma grande panela de barro}
  \seeref{鬲}{ge2}
\end{entry}

\begin{entry}{詈骂}{li4ma4}{12,9}{⾔、⾺}
  \definition{v.}{xingar | abusar}
\end{entry}

\begin{entry}{俩}{lia3}{9}{⼈}[HSK 4]
  \definition{num.}{ambos; dois; contração de 两个 | alguns; vários; refere-se a um pequeno número}
\end{entry}

\begin{entry}{俩钱}{lia3qian2}{9,10}{⼈、⾦}
  \definition{s.}{uma pequena quantia de dinheiro}
\end{entry}

\begin{entry}{连}{lian2}{7}{⾡}[HSK 3]
  \definition*{s.}{sobrenome Lian}
  \definition{adv.}{em sucessão; um após o outro; repetidamente}
  \definition{prep.}{incluindo; incluido | até mesmo}
  \definition[个]{s.}{companhia; unidades organizacionais das forças armadas}
  \definition{v.}{ligar; juntar; conectar | envolver-se (em problemas); implicar; incriminar | costurar; coser}
\end{entry}

\begin{entry}{连接}{lian2 jie1}{7,11}{⾡、⼿}[HSK 5]
  \definition[条]{s.}{conexão}
  \definition{v.}{ligar; unir; relacionar, conectar; anexar}
\end{entry}

\begin{entry}{连忙}{lian2mang2}{7,6}{⾡、⼼}[HSK 3]
  \definition{adv.}{imediatamente; de imediato; com pressa; apressadamente}
\end{entry}

\begin{entry}{连锁反应}{lian2suo3fan3ying4}{7,12,4,7}{⾡、⾦、⼜、⼴}
  \definition{s.}{reação em cadeia}
\end{entry}

\begin{entry}{连续}{lian2xu4}{7,11}{⾡、⽷}[HSK 3]
  \definition{adv.}{continuamente; sucessivamente; em uma fileira; um após o outro}
\end{entry}

\begin{entry}{连续剧}{lian2 xu4 ju4}{7,11,10}{⾡、⽷、⼑}[HSK 3]
  \definition[部,集]{s.}{série; novela; drama dividido em vários episódios, transmitido continuamente pela rádio ou televisão, com enredo contínuo}
\end{entry}

\begin{entry}{帘}{lian2}{8}{⼱}
  \definition{s.}{cortina | tela (pendurada) | bandeira usada como placa de loja}
\end{entry}

\begin{entry}{莲花}{lian2hua1}{10,7}{⾋、⾋}
  \definition{s.}{flor de lótus | lírio aquático}
\end{entry}

\begin{entry}{莲藕}{lian2'ou3}{10,18}{⾋、⾋}
  \definition{s.}{raiz de Lotus}
\end{entry}

\begin{entry}{联合}{lian2he2}{12,6}{⽿、⼝}[HSK 3]
  \definition{adj.}{conjunto; unido; federal; combinado}
  \definition{s.}{aliado; união; aliança; conectar-se ou unir-se para agir em conjunto}
\end{entry}

\begin{entry}{联合国}{lian2 he2 guo2}{12,6,8}{⽿、⼝、⼞}[HSK 3]
  \definition*{s.}{Nações Unidas; Organização internacional fundada em 1945, após o fim da Segunda Guerra Mundial, com sede em Nova Iorque, Estados Unidos ; as suas principais instituições são a Assembleia Geral, o Conselho de Segurança, o Conselho Econômico e Social e o Secretariado; de acordo com a Carta das Nações Unidas, os seus principais objetivos são manter a paz e a segurança internacionais, desenvolver relações amigáveis entre os países e promover a cooperação internacional nas áreas econômica e cultural}
\end{entry}

\begin{entry}{联合会}{lian2he2hui4}{12,6,6}{⽿、⼝、⼈}
  \definition{s.}{federação}
\end{entry}

\begin{entry}{联络}{lian2luo4}{12,9}{⽿、⽷}[HSK 5]
  \definition{v.}{entrar em contato; comunicar-se; entrar em contato com}
\end{entry}

\begin{entry}{联系}{lian2xi4}{12,7}{⽿、⽷}[HSK 3]
  \definition[个,种,层]{s.}{relacionamento; relacionamento entre duas coisas}
  \definition{v.}{entrar em contato; contatar; comunicar-se com alguém por telefone, e-mail ou carta | agendar; entrar em contato com; estabelecer algum tipo de relação com a outra parte | relacionar; combinar; integrar}
\end{entry}

\begin{entry}{联想}{lian2xiang3}{12,13}{⽿、⼼}[HSK 5]
  \definition*{s.}{Lenovo (empresa)}
  \definition{v.}{associar-se a; estabelecer uma conexão mental; lembrar-se de algo; lembrar-se de outras pessoas ou coisas relacionadas devido a alguém ou algo; evocar outros conceitos relacionados devido a um determinado conceito |}
\end{entry}

\begin{entry}{脸}{lian3}{11}{⾁}[HSK 2]
  \definition[张,个]{s.}{rosto (de pessoas ou animais); a parte frontal da cabeça, da testa ao queixo | parte frontal de algo | cara; autoestima; aparência | rosto; expressões faciais}
\end{entry}

\begin{entry}{脸盆}{lian3 pen2}{11,9}{⾁、⽫}[HSK 5]
  \definition[个]{s.}{lavatório; bacia para lavar as mãos e o rosto}
\end{entry}

\begin{entry}{脸色}{lian3 se4}{11,6}{⾁、⾊}[HSK 5]
  \definition{s.}{aparência; tez; cor da pele | aparência; expressão facial | (indicando a condição física de alguém) aparência; cor}
\end{entry}

\begin{entry}{练}{lian4}{8}{⽷}[HSK 2]
  \definition*{s.}{sobrenome Lian}
  \definition{adj.}{habilidoso; experiente; bem treinado}
  \definition{s.}{seda branca}
  \definition{v.}{tratar, amaciar e branquear a seda por meio de fervura; cozinhar seda crua ou tecidos de seda crua | treinar; praticar; exercitar}
\end{entry}

\begin{entry}{练习}{lian4xi2}{8,3}{⽷、⼄}[HSK 2]
  \definition[项,次]{s.}{exercício (em livros); tarefas ou exercícios organizados para consolidar os resultados da aprendizagem}
  \definition{v.}{praticar; exercitar; repitir várias vezes até ficar bem treinado}
\end{entry}

\begin{entry}{恋爱}{lian4'ai4}{10,10}{⼼、⽖}[HSK 5]
  \definition[个,场,段]{s.}{namoro; afeto; amor romântico; ações que demonstram o amor mútuo}
  \definition{v.}{amar; estar apaixonado}
\end{entry}

\begin{entry}{良好}{liang2hao3}{7,6}{⾉、⼥}[HSK 4]
  \definition{adj.}{bom; ótimo; bem}
\end{entry}

\begin{entry}{良田}{liang2tian2}{7,5}{⾉、⽥}
  \definition{s.}{terra agrícola boa | terra fértil}
\end{entry}

\begin{entry}{良心}{liang2xin1}{7,4}{⾉、⼼}
  \definition{s.}{consciência}
\end{entry}

\begin{entry}{凉}{liang2}{10}{⼎}[HSK 2]
  \definition{adj.}{frio; gelado; ligeiramente fria (menos do que 冷) | sombrio; desolado; sem animação | desanimado; desapontado | usado para prevenir o calor e manter a temperatura amena; para proteção contra o calor}
  \definition{s.}{frio; refere-se a um ambiente fresco ou a uma brisa fresca}
  \seeref{凉}{liang4}
  \seealsoref{冷}{leng3}
\end{entry}

\begin{entry}{凉快}{liang2kuai5}{10,7}{⼎、⼼}[HSK 2]
  \definition{adj.}{fresco; refrescante}
  \definition{v.}{refrescar; refrescar-se; deixar o corpo fresco e revigorado}
\end{entry}

\begin{entry}{凉水}{liang2 shui3}{10,4}{⼎、⽔}[HSK 3]
  \definition{s.}{água fria; água não aquecida | água não fervida}
\end{entry}

\begin{entry}{凉鞋}{liang2xie2}{10,15}{⼎、⾰}
  \definition{s.}{sandália | alpargata | alpercata | alparca}
\end{entry}

\begin{entry}{量}{liang2}{12}{⾥}[HSK 4]
  \definition{v.}{medir | estimar; dimensionar}
  \seeref{量}{liang4}
\end{entry}

\begin{entry}{粮食}{liang2shi5}{13,9}{⽶、⾷}[HSK 4]
  \definition[种,斤,吨,袋]{s.}{alimentos; grãos; termo geral para os vários tipos de arroz, feijão, etc. que podem ser consumidos}
\end{entry}

\begin{entry}{两}{liang3}{7}{⼀}[HSK 1,2]
  \definition*{s.}{sobrenome Liang}
  \definition{clas.}{liang, uma unidade de peso (=50 gramas)}
  \definition{num.}{dois (sempre usado antes de classificadores) | poucos; alguns; indica um número indeterminado}
  \definition{s.}{ambos (lados); qualquer (lado)}
\end{entry}

\begin{entry}{两岸}{liang3 an4}{7,8}{⼀、⼭}[HSK 5]
  \definition{s.}{ambos os lados; ambas as margens; ambas as costas; entre os dois lados do estreito; bilateral}
\end{entry}

\begin{entry}{两边}{liang3 bian1}{7,5}{⼀、⾡}[HSK 4]
  \definition{s.}{ambos os lados; ambas as direções; ambos os lugares | ambas as partes; ambos os lados}
\end{entry}

\begin{entry}{两码事}{liang3ma3shi4}{7,8,8}{⼀、⽯、⼅}
  \definition{expr.}{duas coisas completamente diferentes}
\end{entry}

\begin{entry}{亮}{liang4}{9}{⼇}[HSK 2]
  \definition*{s.}{sobrenome Lian}
  \definition{adj.}{brilhante; claro | alto e claro; retumbante | esclarecido; aberto e claro}
  \definition{s.}{luz}
  \definition{v.}{iluminar; clarear; brilhar | elevar a voz; ressoar; tornar o som mais alto | revelar; mostrar; aparecer; exibir}
\end{entry}

\begin{entry}{凉}{liang4}{10}{⼎}
  \definition{v.}{deixar algo esfriar; deixar um objeto quente descansar por um tempo para que a temperatura diminua}
  \seeref{凉}{liang2}
\end{entry}

\begin{entry}{辆}{liang4}{11}{⾞}[HSK 2]
  \definition{clas.}{usado para automóveis, veículos, etc.}
\end{entry}

\begin{entry}{量}{liang4}{12}{⾥}
  \definition{s.}{instrumento de medida; antigamente, o termo se referia a objetos como baldes e litros, que medem o volume | capacidade (para tolerância ou ingestão de alimentos ou bebidas); refere-se ao limite do que pode ser acomodado | quantidade; valor; volume; número}
  \definition{v.}{estimar; medir; pesar}
  \seeref{量}{liang2}
\end{entry}

\begin{entry}{疗养}{liao2 yang3}{7,9}{⽧、⼋}[HSK 4]
  \definition{v.}{recuperar; convalescer; tratar pessoas com doenças crônicas ou debilitantes em instituições médicas especializadas com foco na recuperação}
\end{entry}

\begin{entry}{聊天}{liao2tian1}{11,4}{⽿、⼤}
  \definition{v.+compl.}{papear | bater papo}
\end{entry}

\begin{entry}{了}{liao3}{2}{⼅}
  \definition*{s.}{sobrenome Liao}
  \definition{adv.}{inteiramente; um pouco; totalmente (mais usado em negativas)}
  \definition{v.}{terminar; concluir; encerrar; cumprir; eliminar; resolver | compreender; saber; perceber; saber claramente | expressar possibilidade ou impossibilidade; usado com 得 ou 不 após o verbo, indica possibilidade ou impossibilidade}
  \seeref{了}{le5}
  \seealsoref{不}{bu4}
  \seealsoref{得}{de5}
\end{entry}

\begin{entry}{了不起}{liao3bu5qi3}{2,4,10}{⼅、⼀、⾛}[HSK 4]
  \definition{adj.}{incrível; fantástico; extraordinário | sério; grave}
\end{entry}

\begin{entry}{了解}{liao3jie3}{2,13}{⼅、⾓}[HSK 4]
  \definition{v.}{entender; compreender | investigar; indagar sobre}
\end{entry}

\begin{entry}{列}{lie4}{6}{⼑}[HSK 4]
  \definition{v.}{organizar; formar uma linha; alinhar | listar; inserir em uma lista}
\end{entry}

\begin{entry}{列车}{lie4che1}{6,4}{⼑、⾞}[HSK 4]
  \definition{s.}{trem; trem em uma composição contínua, puxado por uma locomotiva e equipado com uma tripulação e marcações prescritas; geralmente um trem de passageiros}
\end{entry}

\begin{entry}{列入}{lie4 ru4}{6,2}{⼑、⼊}[HSK 4]
  \definition{v.}{incluir em uma lista}
\end{entry}

\begin{entry}{列为}{lie4 wei2}{6,4}{⼑、⼂}[HSK 4]
  \definition{v.}{ser classificado como; ser listado como}
\end{entry}

\begin{entry}{烈士}{lie4shi4}{10,3}{⽕、⼠}
  \definition{s.}{mártir}
\end{entry}

\begin{entry}{猎物}{lie4wu4}{11,8}{⽝、⽜}
  \definition{s.}{presa (vítima de um predador)}
\end{entry}

\begin{entry}{邻居}{lin2ju1}{7,8}{⾢、⼫}[HSK 5]
  \definition[个,位,家]{s.}{vizinho; pessoas ou famílias que moram muito perto}
\end{entry}

\begin{entry}{临}{lin2}{9}{⼁}
  \definition*{s.}{sobrenome Lin}
  \definition{adv.}{pouco antes; prestes a; no ponto de}
  \definition{v.}{encarar; enfrentar; aproximar-se | chegar; estar presente | copiar (um modelo de caligrafia ou pintura); traçar sobre as palavras ou figuras | olhar de cima para baixo | ir de cima para baixo}
\end{entry}

\begin{entry}{临近}{lin2jin4}{9,7}{⼁、⾡}
  \definition{v.}{aproximar-se; estar perto de}
\end{entry}

\begin{entry}{临时}{lin2shi2}{9,7}{⼁、⽇}[HSK 4]
  \definition{adj.}{temporário; provisório; por um breve período}
  \definition{adv.}{no momento em que algo acontece (quando as coisas dão errado)}
\end{entry}

\begin{entry}{淋}{lin2}{11}{⽔}
  \definition{v.}{borrifar | pingar | derramar | encharcar}
  \seeref{淋}{lin4}
\end{entry}

\begin{entry}{淋}{lin4}{11}{⽔}
  \definition{s.}{gonorréia}
  \definition{v.}{filtrar | coar | drenar}
  \seeref{淋}{lin2}
\end{entry}

\begin{entry}{令}{ling2}{5}{⼈}
  \definition*{s.}{sobrenome Ling}
  \definition*{s.}{Antigo nome geográfico, na região atual de Linyi, província de Shanxi.}
  \seeref{令}{ling3}
  \seeref{令}{ling4}
\end{entry}

\begin{entry}{灵感}{ling2gan3}{7,13}{⽕、⼼}
  \definition{s.}{inspiração | explosão de criatividade em empreendimento científico ou artístico}
\end{entry}

\begin{entry}{灵魂}{ling2hun2}{7,13}{⽕、⿁}
  \definition{s.}{alma | espírito}
\end{entry}

\begin{entry}{铃}{ling2}{10}{⾦}[HSK 5]
  \definition{s.}{sino; instrumento musical feito de metal | objetos em forma de sino | cápsula; botão; broto}
\end{entry}

\begin{entry}{铃声}{ling2 sheng1}{10,7}{⾦、⼠}[HSK 5]
  \definition{s.}{o tilintar de sinos; o som de um sino tocando}
\end{entry}

\begin{entry}{陵园}{ling2yuan2}{10,7}{⾩、⼞}
  \definition{s.}{cemitério}
\end{entry}

\begin{entry}{菱角}{ling2jiao5}{11,7}{⾋、⾓}
  \definition{s.}{castanha d'água}
\end{entry}

\begin{entry}{零/〇}{ling2 ling2}{13,13}{⾬、⾬}[HSK 1]
  \definition*{s.}{sobrenome Ling}
  \definition{adj.}{ímpar; dispersos; fragmentados (em oposição a 整)}
  \definition{num.}{zero; 0; representa um número menor que qualquer número positivo e maior que qualquer número negativo; representa a ausência de quantidade | zero grau no termômetro | usado para indicar qualidade, comprimento, tempo, idade, etc. Entre dois dígitos, indica que a quantidade da unidade mais alta é acompanhada pela quantidade da unidade mais baixa | sinal de zero (0); nulo; espaço em branco para indicar números em caracteres chineses maiúsculos}
  \definition{s.}{fragmento; fração; lote ímpar; um número fracionário que não é suficiente para uma determinada unidade; um ponto decimal diferente de um inteiro}
  \definition{v.}{(de chuva, lágrimas, etc.) cair | murchar e cair}
  \seealsoref{整}{zheng3}
\end{entry}

\begin{entry}{零散}{ling2san3}{13,12}{⾬、⽁}
  \definition{adj.}{espalhado; disperso}
\end{entry}

\begin{entry}{零食}{ling2shi2}{13,9}{⾬、⾷}[HSK 4]
  \definition[包,袋,盒,箱,堆]{s.}{lanches; refrescos; petiscos entre as refeições; alimentação esporádica, além das refeições normais}
\end{entry}

\begin{entry}{零下}{ling2 xia4}{13,3}{⾬、⼀}[HSK 2]
  \definition{s.}{abaixo de zero; negativo}
\end{entry}

\begin{entry}{令}{ling3}{5}{⼈}
  \definition{clas.}{resma (de papel); unidade de medida de papel: 500 folhas inteiras de papel original produzidas mecanicamente equivalem a 1 resma}
\end{entry}

\begin{entry}{岭}{ling3}{8}{⼭}
  \definition{s.}{cordilheira}
\end{entry}

\begin{entry}{领}{ling3}{11}{⾴}[HSK 3]
  \definition{clas.}{usado para roupas, mantos, esteiras, tapetes, telas, etc.}
  \definition{s.}{pescoço; gargalo | gola; colarinho; faixa de pescoço | esboço; ponto principal; essência}
  \definition{v.}{conduzir; guiar; orientar | possuir; ser o possuidor de; ter jurisdição sobre | obter; conseguir; receber (o que foi distribuído) | aceitar; tomar |entender; compreender (o significado)}
\end{entry}

\begin{entry}{领带}{ling3 dai4}{11,9}{⾴、⼱}[HSK 5]
  \definition[条]{s.}{colar; gargantilha; gravata}
\end{entry}

\begin{entry}{领导}{ling3dao3}{11,6}{⾴、⼨}[HSK 3]
  \definition[个,位,名,些]{s.}{líder; liderança; pessoa que ocupa uma posição de liderança}
  \definition{v.}{liderar; exercer liderança; (elogio) liderar, gerenciar outras pessoas;  trabalhar com outras pessoas ou avançar em direção a um objetivo}
\end{entry}

\begin{entry}{领情}{ling3qing2}{11,11}{⾴、⼼}
  \definition{v.+compl.}{sentir-se grato a alguém}
\end{entry}

\begin{entry}{领先}{ling3xian1}{11,6}{⾴、⼉}[HSK 3]
  \definition{v.}{liderar; assumir a liderança; estar na liderança; (velocidade, desempenho, etc.) superar pessoas ou coisas semelhantes, estar na vanguarda}
\end{entry}

\begin{entry}{令}{ling4}{5}{⼈}[HSK 5]
  \definition{adj.}{bom; excelente | termos de cortesia usados para se referir aos familiares e parentes da outra pessoa}
  \definition{s.}{ordem; decreto; comando; ordem emitida pela autoridade superior | um título oficial; administradores de certos departamentos governamentais na antiguidade | temporada; estação; clima e fenologia de uma determinada estação | poema-canção; letra curta}
  \definition{v.}{ordenar; comandar | fazer com que alguém; fazer com que; permitir que}
  \seeref{令}{ling2}
  \seeref{令}{ling3}
\end{entry}

\begin{entry}{令人}{ling4ren2}{5,2}{⼈、⼈}
  \definition{v.}{causar alguém (a fazer alguma coisa) | fazer alguém ficar zangado, encantado, etc.}
\end{entry}

\begin{entry}{另外}{ling4wai4}{5,5}{⼝、⼣}[HSK 3]
  \definition{adv.}{além disso; em adição; ademais; além do mais; além de que; além do que já foi dito}
  \definition{conj.}{além disso; usada entre duas ou mais frases, indica algo além do que foi mencionado anteriormente}
  \definition{pron.}{outro; além das pessoas ou coisas mencionadas anteriormente}
\end{entry}

\begin{entry}{另一方面}{ling4 yi4 fang1 mian4}{5,1,4,9}{⼝、⼀、⽅、⾯}[HSK 3]
  \definition{adv./conj.}{outro aspecto | por outro lado; por sua vez; em contrapartida}
\end{entry}

\begin{entry}{刘}{liu2}{6}{⼑}
  \definition*{s.}{sobrenome Liu}
  \definition{s.}{(clássico) um tipo de machado de batalha}
  \definition{v.}{matar}
\end{entry}

\begin{entry}{流}{liu2}{10}{⽔}[HSK 2]
  \definition*{s.}{sobrenome Liu}
  \definition{adj.}{fluente; tão suave quanto a água corrente}
  \definition{clas.}{lúmen; abreviação de lumens, 流明}
  \definition[名,个]{s.}{corrente de água | corrente; algo que se assemelha a um fluxo de água | razão; taxa; classe; grau; ramificação; facção; hierarquia}
  \definition{v.}{(de líquido) fluir | vaguear; vagar; mover-se de um lugar para outro; movimentar-se sem direção fixa | espalhar; circular; transmitir; divulgar | degenerar; mudar para pior; tender (aspecto negativo) | banir; enviar para o exílio | correr (ou fluir) como líquido; refere-se à parte do rio após deixar sua nascente (em contraste com a 源)}
  \seealsoref{流明}{liu2ming2}
  \seealsoref{源}{yuan2}
\end{entry}

\begin{entry}{流传}{liu2chuan2}{10,6}{⽔、⼈}[HSK 4]
  \definition{v.}{espalhar; circular; passar adiante}
\end{entry}

\begin{entry}{流动}{liu2 dong4}{10,6}{⽔、⼒}[HSK 5]
  \definition{v.}{fluir; correr; circular | ir de um lugar para outro; mover-se; mudar frequentemente de posição}
\end{entry}

\begin{entry}{流利}{liu2li4}{10,7}{⽔、⼑}[HSK 2]
  \definition{adj.}{fluente; suave; lúcido; falar e escrever com fluência e clareza | com fluência; sem dificuldades}
\end{entry}

\begin{entry}{流明}{liu2ming2}{10,8}{⽔、⽇}
  \definition{s.}{(empréstimo linguístico) lúmen (unidade de fluxo luminoso)}
\end{entry}

\begin{entry}{流水}{liu2shui3}{10,4}{⽔、⽔}
  \definition{s.}{água corrente | (negócio) rotatividade}
\end{entry}

\begin{entry}{流通}{liu2tong1}{10,10}{⽔、⾡}[HSK 5]
  \definition{v.}{(de ar, dinheiro, mercadorias, etc.) fluir; circular}
\end{entry}

\begin{entry}{流星}{liu2xing1}{10,9}{⽔、⽇}
  \definition{s.}{meteoro | estrela cadente}
\end{entry}

\begin{entry}{流行}{liu2xing2}{10,6}{⽔、⾏}[HSK 2]
  \definition{adj.}{popular; na moda; muito popular}
  \definition{v.}{ser popular; prevalecer; espalhar-se amplamente; divulgar amplamente}
\end{entry}

\begin{entry}{留}{liu2}{10}{⽥}[HSK 2]
  \definition*{s.}{sobrenome Liu}
  \definition{v.}{ficar; permanecer; parar em um determinado local ou posição; não se afastar | estudar no exterior (geralmente seguido pelo nome de um país com uma sílaba) | pedir a alguém para ficar; manter alguém onde está | concentrar-se em; concentrar a atenção em algo | manter; guardar; reservar; não joger fora | acumular; deixar crescer | aceitar; receber | transmitir (legado); deixar para trás}
\end{entry}

\begin{entry}{留神}{liu2shen2}{10,9}{⽥、⽰}
  \definition{v.+compl.}{tomar cuidado | prestar atenção | manter os olhos abertos}
\end{entry}

\begin{entry}{留下}{liu2 xia4}{10,3}{⽥、⼀}[HSK 2]
  \definition{v.}{deixar; parar em algum lugar}
\end{entry}

\begin{entry}{留学}{liu2xue2}{10,8}{⽥、⼦}[HSK 3]
  \definition{v.}{estudar no exterior; permanecer no estrangeiro para estudar ou pesquisar}
\end{entry}

\begin{entry}{留学生}{liu2 xue2 sheng1}{10,8,5}{⽥、⼦、⽣}[HSK 2]
  \definition[个,位,名,批,帮]{s.}{estudante estrangeiro; estudante que retornou; estudante que estuda no exterior}
\end{entry}

\begin{entry}{柳}{liu3}{9}{⽊}
  \definition*{s.}{sobrenome Liu}
  \definition{s.}{salgueiro}
\end{entry}

\begin{entry}{柳橙汁}{liu3cheng2zhi1}{9,16,5}{⽊、⽊、⽔}
  \definition[瓶,杯,罐,盒]{s.}{suco de laranja}
  \seealsoref{橙汁}{cheng2zhi1}
  \seealsoref{橘子汁}{ju2zi5zhi1}
\end{entry}

\begin{entry}{六}{liu4}{4}{⼋}[HSK 1]
  \definition*{s.}{sobrenome Liu}
  \definition{num.}{seis; 6}
  \definition{s.}{símbolo musical utilizado na partitura da música tradicional chinesa, representando o primeiro grau da escala musical, equivalente ao ``5'' da notação musical simplificada}
\end{entry}

\begin{entry}{遛狗}{liu4gou3}{13,8}{⾡、⽝}
  \definition{v.+compl.}{passear com um cachorro}
\end{entry}

\begin{entry}{龙}{long2}{5}{⿓}[HSK 3][Kangxi 212]
  \definition*{s.}{sobrenome Long}
  \definition[条]{s.}{dragão; animal mítico e sobrenatural, com chifres, escamas, garras e bigodes, capaz de voar e mergulhar na água, provocar nuvens e chuva | dinossauro; um enorme réptil extinto; referência a certos répteis gigantes da antiguidade | do imperador; dragão como símbolo do imperador; usado na era feudal como símbolo do imperador; também se refere a coisas pertencentes ao imperador | em forma de dragão; com um desenho de dragão; refere-se a certos objetos que formam uma sequência semelhante a um dragão ou decorados com motivos de dragões}
\end{entry}

\begin{entry}{龙山}{long2shan1}{5,3}{⿓、⼭}
  \definition*{s.}{Longshan}
\end{entry}

\begin{entry}{龙虾}{long2xia1}{5,9}{⿓、⾍}
  \definition{s.}{lagosta}
\end{entry}

\begin{entry}{笼}{long2}{11}{⽵}
  \definition{s.}{armação fechada de bambu, arame, etc. | jaula | gaiola}
  \seeref{笼}{long3}
\end{entry}

\begin{entry}{笼子}{long2zi5}{11,3}{⽵、⼦}
  \definition{s.}{jaula | cesta | gaiola | recipiente}
  \seeref{笼子}{long3zi5}
\end{entry}

\begin{entry}{笼}{long3}{11}{⽵}
  \definition{v.}{envolver | cobrir}
  \seeref{笼}{long2}
\end{entry}

\begin{entry}{笼子}{long3zi5}{11,3}{⽵、⼦}
  \definition{s.}{caixa grande | porta-malas}
  \seeref{笼子}{long2zi5}
\end{entry}

\begin{entry}{弄}{long4}{7}{⼶}
  \definition{s.}{rua estreita; beco; viela; travessa}
  \seeref{弄}{nong4}
\end{entry}

\begin{entry}{楼}{lou2}{13}{⽊}[HSK 1]
  \definition*{s.}{sobrenome Lou}
  \definition{clas.}{andar, piso}
  \definition[层,座,栋]{s.}{um prédio com muitos andares | piso; andar | superestrutura; uma estrutura com um convés superior; um andar adicional construído sobre uma casa ou outro edifício | nome usado para certas lojas ou locais de entretenimento | arco ornamental; certas construções decorativas altas com passagens por baixo}
\end{entry}

\begin{entry}{楼上}{lou2 shang4}{13,3}{⽊、⼀}[HSK 1]
  \definition{s.}{no andar de cima | autor anterior em um tópico do fórum; em plataformas como fóruns na internet, refere-se à pessoa que se manifesta antes de você.}
\end{entry}

\begin{entry}{楼梯}{lou2 ti1}{13,11}{⽊、⽊}[HSK 4]
  \definition[个]{s.}{escada; escadaria; degraus no meio de dois andares para permitir que as pessoas subam ou desçam as escadas}
\end{entry}

\begin{entry}{楼下}{lou2 xia4}{13,3}{⽊、⼀}[HSK 1]
  \definition{s.}{no andar de baixo}
\end{entry}

\begin{entry}{漏}{lou4}{14}{⽔}[HSK 5]
  \definition{s.}{relógio de água; ampulheta | falha; ponto fraco | gonorreia; a medicina tradicional chinesa refere-se a certas doenças que causam secreção de pus, sangue e muco | unidade de tempo medida por um relógio de água durante a noite}
  \definition{v.}{(líquido, gás, etc.) pingar; vazar; escorrer; cair (de um buraco ou fenda) | vazar; deixar escapar; divulgar | perder; deixar de fora por engano | vazar; o objeto tem poros e pode vazar coisas | há uma fuga de ar}
\end{entry}

\begin{entry}{漏电}{lou4dian4}{14,5}{⽔、⽥}
  \definition{v.}{vazar eletricidade}
\end{entry}

\begin{entry}{漏洞}{lou4 dong4}{14,9}{⽔、⽔}[HSK 5]
  \definition[个]{s.}{vazamento; rachadura; lacunas ou buracos desnecessários que permitem que coisas vazem | falha; defeito; lacuna; (fala, ação, método, etc.) imperfeições}
\end{entry}

\begin{entry}{卢旺达}{lu2wang4da2}{5,8,6}{⼘、⽇、⾡}
  \definition*{s.}{Ruanda}
\end{entry}

\begin{entry}{芦笋}{lu2sun3}{7,10}{⾋、⽵}
  \definition{s.}{aspargos}
\end{entry}

\begin{entry}{陆地}{lu4di4}{7,6}{⾩、⼟}[HSK 4]
  \definition[块,片]{s.}{terra; terra seca (em oposição ao mar); superfície da Terra, excluindo os oceanos (e, às vezes, rios e lagos)}
\end{entry}

\begin{entry}{陆路}{lu4lu4}{7,13}{⾩、⾜}
  \definition{s.}{rota terrestre}
\end{entry}

\begin{entry}{陆续}{lu4xu4}{7,11}{⾩、⽷}[HSK 4]
  \definition{adv.}{sucessivamente; um após o outro; intermitentemente}
\end{entry}

\begin{entry}{录}{lu4}{8}{⼹}[HSK 3]
  \definition{s.}{registro; cadastro; coleção; seleções}
  \definition{v.}{copiar; gravar; escrever; copiar; registrar | contratar; selecionar; empregar; adotar ou nomear | gravar em fita magnética}
\end{entry}

\begin{entry}{录取}{lu4qu3}{8,8}{⼹、⼜}[HSK 4]
  \definition{v.}{aceitar; admitir; recrutar; entrar; matricular (os aprovados no exame)}
\end{entry}

\begin{entry}{录像带}{lu4xiang4dai4}{8,13,9}{⼹、⼈、⼱}
  \definition[盘]{s.}{video-cassete}
\end{entry}

\begin{entry}{录像机}{lu4xiang4ji1}{8,13,6}{⼹、⼈、⽊}
  \definition[台]{s.}{gravador de vídeo | VCR}
\end{entry}

\begin{entry}{录音}{lu4yin1}{8,9}{⼹、⾳}[HSK 3]
  \definition[段,个]{s.}{gravação de som; som gravado com equipamento especializado}
  \definition{v.+compl.}{gravar; converter o som em sinal elétrico e, em seguida, gravá-lo por meios mecânicos, ópticos ou eletromagnéticos}
\end{entry}

\begin{entry}{录音机}{lu4yin1ji1}{8,9,6}{⼹、⾳、⽊}
  \definition[台]{s.}{gravador de áudio}
\end{entry}

\begin{entry}{鹿}{lu4}{11}{⿅}[Kangxi 198]
  \definition{s.}{cervo | veado}
\end{entry}

\begin{entry}{路}{lu4}{13}{⾜}[HSK 1]
  \definition*{s.}{sobrenome Lu}
  \definition{clas.}{tipo; classe | linha; coluna; usado para um grupo de pessoas ou uma equipe; para organizar em ordem}
  \definition[条]{s.}{estrada; caminho; via | viagem; jornada; distância | maneira; meios | sequência; linha; lógica | região; distrito | rota | classe; classificação; grau | linha; fileira}
\end{entry}

\begin{entry}{路边}{lu4 bian1}{13,5}{⾜、⾡}[HSK 2]
  \definition{s.}{calçada; beira da estrada; margem da rua}
\end{entry}

\begin{entry}{路口}{lu4 kou3}{13,3}{⾜、⼝}[HSK 1]
  \definition[个]{s.}{cruzamento; intersecção; onde as estradas se encontram}
\end{entry}

\begin{entry}{路上}{lu4 shang5}{13,3}{⾜、⼀}[HSK 1]
  \definition{s.}{na estrada | a caminho; na rota; em processo de mudança de um lugar para outro}
\end{entry}

\begin{entry}{路线}{lu4 xian4}{13,8}{⾜、⽷}[HSK 3]
  \definition[条]{s.}{rota; caminho; linha; a estrada percorrida de um lugar a outro | linha; diretriz (de política, ideologia, campo de trabalho); a via fundamental a seguir em termos ideológicos, políticos ou profissionais}
\end{entry}

\begin{entry}{露珠}{lu4zhu1}{21,10}{⾬、⽟}
  \definition{s.}{orvalho}
\end{entry}

\begin{entry}{乱}{luan4}{7}{⼄}[HSK 3]
  \definition{adj.}{em desordem; em confusão; em desarrumação; sem ordem nem organização | em um estado mental confuso | (de uma sociedade) turbulento; agitado | (de relações sexuais) impróprio; promíscuo}
  \definition{adv.}{aleatoriamente; arbitrariamente; indiscriminadamente; sem restrições; à vontade}
  \definition{s.}{motim; agitação; tumulto; revolta; guerra; calamidade}
  \definition{v.}{confundir; embaralhar; misturar; causar desordem}
\end{entry}

\begin{entry}{伦敦}{lun2dun1}{6,12}{⼈、⽁}
  \definition*{s.}{Londres}
\end{entry}

\begin{entry}{轮}{lun2}{8}{⾞}[HSK 4]
  \definition{clas.}{para sol vermelho, lua brilhante, etc. | para rodadas | doze anos de idade (os doze ramos terrestres são usados para lembrar o gênero humano e cada doze anos de idade é um ciclo)}
  \definition{s.}{roda | anel; disco; objeto semelhante a uma roda | navio a vapor; barco a vapor}
  \definition{v.}{revezar; substituir um ao outro em sequência (para fazer algo)}
\end{entry}

\begin{entry}{轮船}{lun2chuan2}{8,11}{⾞、⾈}[HSK 4]
  \definition[艘]{s.}{navio}
\end{entry}

\begin{entry}{轮回}{lun2hui2}{8,6}{⾞、⼞}
  \definition[个]{s.}{reencarnação (Budismo) | ciclo}
  \definition{v.}{reencarnar}
\end{entry}

\begin{entry}{轮椅}{lun2 yi3}{8,12}{⾞、⽊}[HSK 4]
  \definition{s.}{cadeira de rodas; dispositivo de assento especialmente projetado com rodas para pessoas com dificuldade de locomoção, que pode ser acionado por um disco de roda ou manivela operados manualmente}
\end{entry}

\begin{entry}{轮子}{lun2 zi5}{8,3}{⾞、⼦}[HSK 4]
  \definition[个]{s.}{roda; peças circulares de veículos ou máquinas com capacidade de rotação}
\end{entry}

\begin{entry}{论文}{lun4wen2}{6,4}{⾔、⽂}[HSK 4]
  \definition[篇]{s.}{tese; redação; artigo; artigo que discute ou examina uma questão}
\end{entry}

\begin{entry}{罗}{luo2}{8}{⽹}
  \definition*{s.}{sobrenome Luo}
  \definition{v.}{coletar | juntar | pegar | peneirar}
\end{entry}

\begin{entry}{逻辑}{luo2ji5}{11,13}{⾡、⾞}[HSK 5]
  \definition{s.}{lógica; lei objetiva; a objetividade das leis que regem o desenvolvimento das coisas | lógica; razão; regras para o pensamento | lógica como ciência do raciocínio, do pensamento; disciplina que estuda a lógica}
\end{entry}

\begin{entry}{螺}{luo2}{17}{⾍}
  \definition{s.}{concha em espiral | caracol | búzio}
\end{entry}

\begin{entry}{螺丝}{luo2si1}{17,5}{⾍、⼀}
  \definition{s.}{parafuso}
\end{entry}

\begin{entry}{骆驼}{luo4tuo5}{9,8}{⾺、⾺}
  \definition[峰,匹,头]{s.}{camelo | (coloquial) cabeça-dura, idiota}
\end{entry}

\begin{entry}{落}{luo4}{12}{⾋}[HSK 4]
  \definition*{s.}{sobrenome Luo}
  \definition{s.}{paradeiro; lugar para ficar; local de descanso | assentamento; local de reunião | parte curta; área pequena; refere-se a um pequeno lugar ou área}
  \definition{v.}{cair; cair de uma altura elevada | se abaixar; descer; ir para baixo | abaixar; deixar cair (ou descer); fazer descer | afundar; declinar; descer | ficar para trás; ficar para trás ou ficar de fora | permanecer; fazer uma parada; deixar para trás | cair sobre; repousar com | obter; ter; receber | anotar; escrever no papel | cair em; entrar em; ficar preso}
  \seeref{落}{la4}
  \seeref{落}{lao4}
\end{entry}

\begin{entry}{落后}{luo4hou4}{12,6}{⾋、⼝}[HSK 3]
  \definition{adj.}{atrasado; trabalho em atraso, nível de desenvolvimento ou grau de reconhecimento (em oposição a 进步)}
  \definition{v.}{ficar para trás; ficar atrasado; ficar para trás em relação aos outros durante o avanço ou o progresso do trabalho}
  \seealsoref{进步}{jin4bu4}
\end{entry}

\begin{entry}{落日}{luo4ri4}{12,4}{⾋、⽇}
  \definition{s.}{pôr do sol}
\end{entry}

\begin{entry}{落实}{luo4shi2}{12,8}{⾋、⼧}[HSK 5]
  \definition{adj.}{sentimento de tranquilidade; (humor) estável; seguro}
  \definition{v.}{implementar; ser praticável; tornar os planos, políticas, medidas, etc. específicos e compreensíveis, de modo a que possam ser realizados | implementar; colocar em prática; pôr em prática significa que os planos, políticas e medidas são específicos e claros, e podem ser realizados}
\end{entry}

\begin{entry}{落汤鸡}{luo4tang1ji1}{12,6,7}{⾋、⽔、⿃}
  \definition{s.}{uma pessoa que parece encharcada e acamada| sofrimento profundo}
\end{entry}

\begin{entry}{驴}{lv2}{7}{⾺}
  \definition[头]{s.}{burro | asno | jumento | jegue}
\end{entry}

\begin{entry}{旅程}{lv3cheng2}{10,12}{⽅、⽲}
  \definition{s.}{jornada | viagem}
\end{entry}

\begin{entry}{旅馆}{lv3 guan3}{10,11}{⽅、⾷}[HSK 3]
  \definition[家,个,所]{s.}{pousada; hotel; local comercial destinado ao alojamento de viajantes}
\end{entry}

\begin{entry}{旅客}{lv3 ke4}{10,9}{⽅、⼧}[HSK 2]
  \definition[名,位,个,些]{s.}{viajante; passageiro; as agências de transporte e turismo referem-se às pessoas que viajam}
\end{entry}

\begin{entry}{旅行}{lv3xing2}{10,6}{⽅、⾏}[HSK 2]
  \definition{v.}{viajar; passear; para tratar de assuntos ou passear, ir de um lugar para outro (geralmente se refere a distâncias longas)}
\end{entry}

\begin{entry}{旅行社}{lv3 xing2 she4}{10,6,7}{⽅、⾏、⽰}[HSK 3]
  \definition[家]{s.}{agência de viagens; agência especializada em serviços relacionados a viagens, que providencia hospedagem, transporte e outros serviços para viajantes}
\end{entry}

\begin{entry}{旅游}{lv3you2}{10,12}{⽅、⽔}[HSK 2]
  \definition{v.}{viajar para outros lugares para passear e fazer turismo}
\end{entry}

\begin{entry}{屡次}{lv3ci4}{12,6}{⼫、⽋}
  \definition{adv.}{repetidamente | uma e outra vez | muitas vezes}
\end{entry}

\begin{entry}{律师}{lv4shi1}{9,6}{⼻、⼱}[HSK 4]
  \definition[名,个,位]{s.}{advogado; procurador; profissionais encarregados pelas partes ou nomeados pelo tribunal para auxiliar as partes no litígio, para comparecer ao tribunal para defesa e para tratar de assuntos jurídicos relacionados, de acordo com a lei}
\end{entry}

\begin{entry}{绿}{lv4}{11}{⽷}[HSK 2]
  \definition*{s.}{sobrenome Lü}
  \definition{adj.}{verde}
  \definition{v.}{tornar-se verde; ficar verde}
\end{entry}

\begin{entry}{绿茶}{lv4 cha2}{11,9}{⽷、⾋}[HSK 3]
  \definition{s.}{chá verde; chá produzido apenas através dos processos de maturação, enrolamento (ou sem enrolamento) e secagem, sem passar por fermentação, com cor verde-claro}
\end{entry}

\begin{entry}{绿豆}{lv4dou4}{11,7}{⽷、⾖}
  \definition{s.}{vagens}
\end{entry}

\begin{entry}{绿豆芽}{lv4dou4 ya2}{11,7,7}{⽷、⾖、⾋}
  \definition{s.}{broto de feijão verde}
\end{entry}

\begin{entry}{绿色}{lv4 se4}{11,6}{⽷、⾊}[HSK 2]
  \definition{adj.}{verde; ecológico; sem poluição; em conformidade com os requisitos ambientais}
  \definition{s.}{cor verde}
\end{entry}

\begin{entry}{略}{lve4}{11}{⽥}
  \definition{adv.}{ligeiramente | marginalmente | aproximadamente}
\end{entry}

\begin{entry}{略微}{lve4wei1}{11,13}{⽥、⼻}
  \definition{adv.}{ligeiramente | marginalmente | aproximadamente}
\end{entry}

%%%%% EOF %%%%%


%%%
%%% M
%%%

\section*{M}\addcontentsline{toc}{section}{M}

\begin{entry}{妈}{ma1}{6}{⼥}[HSK 1]
  \definition[个,位]{s.}{mãe | mamãe | uma forma de tratamento para uma mulher casada uma geração mais velha | uma forma de tratamento para uma empregada doméstica de meia-idade ou velha}
  \seeref{妈妈}{ma1ma5}
\end{entry}

\begin{entry}{妈妈}{ma1ma5}{6,6}{⼥、⼥}[HSK 1]
  \definition[个,位]{s.}{mamãe | mãe}
\end{entry}

\begin{entry}{吗}{ma2}{6}{⼝}
  \definition{adv.}{(coloquial) que?}
  \seeref{吗}{ma3}
  \seeref{吗}{ma5}
\end{entry}

\begin{entry}{麻烦}{ma2fan5}{11,10}{⿇、⽕}[HSK 3]
  \definition{adj.}{problemático; inconveniente}
  \definition[个]{s.}{problema; inconveniência}
  \definition{v.}{preocupar; incomodar; amolar; azucrinar; incomodar alguém; enfadar; aborrecer}
\end{entry}

\begin{entry}{麻将}{ma2jiang4}{11,9}{⿇、⼨}
  \definition[副]{s.}{\emph{mahjong}}
\end{entry}

\begin{entry}{麻辣豆腐}{ma2la4 dou4fu5}{11,14,7,14}{⿇、⾟、⾖、⾁}
  \definition{s.}{tofú guisado em molho picante (prato)}
\end{entry}

\begin{entry}{马}{ma3}{3}{⾺}[HSK 3][Kangxi 187]
  \definition*{s.}{sobrenome Ma}
  \definition{adj.}{grande}
  \definition[匹]{s.}{cavalo | a peça do cavalo no xadrez chinês}
\end{entry}

\begin{entry}{马耳他}{ma3'er3ta1}{3,6,5}{⾺、⽿、⼈}
  \definition*{s.}{Malta}
\end{entry}

\begin{entry}{马克思列宁主义}{ma3ke4si1 lie4ning2zhu3yi4}{3,7,9,6,5,5,3}{⾺、⼗、⼼、⼑、⼧、⼂、⼂}
  \definition*{s.}{Marxismo-Leninismo}
\end{entry}

\begin{entry}{马路}{ma3lu4}{3,13}{⾺、⾜}[HSK 1]
  \definition[条]{s.}{rua | estrada}
\end{entry}

\begin{entry}{马马虎虎}{ma3ma3hu3hu3}{3,3,8,8}{⾺、⾺、⾌、⾌}
  \definition{adj.}{descuidado | casual | tolerável | vago | mais ou menos}
\end{entry}

\begin{entry}{马上}{ma3shang4}{3,3}{⾺、⼀}[HSK 1]
  \definition{adv.}{já | imediatamente | de imediato | sem demora}
\end{entry}

\begin{entry}{马尾}{ma3wei3}{3,7}{⾺、⼫}
  \definition{s.}{(penteado) rabo de cavalo | cauda de cavalo}
\end{entry}

\begin{entry}{吗}{ma3}{6}{⼝}
  \definition{s.}{usada em 吗啡}
  \seeref{吗啡}{ma3fei1}
\end{entry}

\begin{entry}{吗啡}{ma3fei1}{6,11}{⼝、⼝}
  \definition{s.}{morfina (empréstimo linguístico)}
\end{entry}

\begin{entry}{蚂蚁}{ma3yi3}{9,9}{⾍、⾍}
  \definition{s.}{formiga}
\end{entry}

\begin{entry}{骂}{ma4}{9}{⾺}
  \definition{v.}{insultar | maldizer | ralhar}
\end{entry}

\begin{entry}{骂街}{ma4jie1}{9,12}{⾺、⾏}
  \definition{v.}{gritar abusos na rua}
\end{entry}

\begin{entry}{骂名}{ma4ming2}{9,6}{⾺、⼝}
  \definition{s.}{infâmia}
\end{entry}

\begin{entry}{吗}{ma5}{6}{⼝}[HSK 1]
  \definition{part.}{partícula interrogativa, usada em perguntas ``sim-não''}
  \seeref{吗}{ma2}
  \seeref{吗}{ma3}
\end{entry}

\begin{entry}{埋伏}{mai2fu2}{10,6}{⼟、⼈}
  \definition{s.}{emboscada}
  \definition{v.}{emboscar}
\end{entry}

\begin{entry}{买}{mai3}{6}{⼄}[HSK 1]
  \definition{v.}{comprar}
\end{entry}

\begin{entry}{买东西}{mai3dong1xi5}{6,5,6}{⼄、⼀、⾑}
  \definition{v.}{fazer compras}
\end{entry}

\begin{entry}{麦当劳}{mai4dang1lao2}{7,6,7}{⿆、⼹、⼒}
  \definition*{s.}{McDonald's (empresa de \emph{fast-food})}
  \seealsoref{麦当劳叔叔}{mai4dang1lao2 shu1shu5}
\end{entry}

\begin{entry}{麦当劳叔叔}{mai4dang1lao2 shu1shu5}{7,6,7,8,8}{⿆、⼹、⼒、⼜、⼜}
  \definition*{s.}{Ronald McDonald}
  \seealsoref{麦当劳}{mai4dang1lao2}
\end{entry}

\begin{entry}{麦淇淋}{mai4qi2lin2}{7,11,11}{⿆、⽔、⽔}
  \definition{s.}{(empréstimo linguístico) margarina}
\end{entry}

\begin{entry}{卖}{mai4}{8}{⼗}[HSK 2]
  \definition{v.}{vender}
\end{entry}

\begin{entry}{馒头}{man2tou5}{14,5}{⾷、⼤}
  \definition{s.}{pão cozido no vapor}
\end{entry}

\begin{entry}{满}{man3}{13}{⽔}[HSK 2]
  \definition{adj.}{completo | preenchido | embalado | satisfeito | contente}
  \definition{adv.}{completamente | bastante}
  \definition{v.}{preencher | atingir o limite | satisfazer}
\end{entry}

\begin{entry}{满分}{man3fen1}{13,4}{⽔、⼑}
  \definition{s.}{pontuação completa}
\end{entry}

\begin{entry}{满满}{man3man3}{13,13}{⽔、⽔}
  \definition{adj.}{completo | densamente empacotado}
\end{entry}

\begin{entry}{满意}{man3yi4}{13,13}{⽔、⼼}[HSK 2]
  \definition{adj.}{satisfatório}
\end{entry}

\begin{entry}{满足}{man3zu2}{13,7}{⽔、⾜}[HSK 3]
  \definition{v.}{estar satisfeito; contentar-se | satisfazer; causar satisfação; contentar}
\end{entry}

\begin{entry}{谩骂}{man4ma4}{13,9}{⾔、⾺}
  \definition{v.}{ridicularizar | abusar}
\end{entry}

\begin{entry}{慢}{man4}{14}{⼼}[HSK 1]
  \definition{adj.}{devagar}
\end{entry}

\begin{entry}{慢动作}{man4dong4zuo4}{14,6,7}{⼼、⼒、⼈}
  \definition{s.}{(cinema) câmera lenta}
\end{entry}

\begin{entry}{慢慢}{man4 man4}{14,14}{⼼、⼼}[HSK 3]
  \definition{adv.}{lentamente; vagarosamente; gradualmente}
\end{entry}

\begin{entry}{漫骂}{man4ma4}{14,9}{⽔、⾺}
  \variantof{谩骂}
\end{entry}

\begin{entry}{蔓草}{man4cao3}{14,9}{⾋、⾋}
  \definition{s.}{videira | trepadeira}
\end{entry}

\begin{entry}{忙}{mang2}{6}{⼼}[HSK 1]
  \definition{adj.}{ocupado}
  \definition{s.}{apressar}
\end{entry}

\begin{entry}{盲目}{mang2mu4}{8,5}{⽬、⽬}
  \definition{adj.}{ignorante | sem compreensão}
  \definition{adv.}{cegamente}
  \definition{s.}{cego}
\end{entry}

\begin{entry}{猫}{mao1}{11}{⽝}[HSK 2]
  \definition[只]{s.}{gato |  (empréstimo linguístico) (coloquial) MODEM}
  \definition{v.}{(dialeto) esconder-se}
  \seeref{猫}{mao2}
\end{entry}

\begin{entry}{猫熊}{mao1xiong2}{11,14}{⽝、⽕}
  \definition[把,只]{s.}{panda gigante}
  \seealsoref{熊猫}{xiong2mao1}
\end{entry}

\begin{entry}{毛}{mao2}{4}{⽑}[HSK 1,3][Kangxi 82]
  \definition*{s.}{sobrenome Mao}
  \definition{adj.}{bronco; bruto; semi-acabado | grosseiro | pequeno | descuidado; precipitado; impetuoso | agitado; assustado; amedrontado}
  \definition{clas.}{1 mao corresponde a 10 centavos}
  \definition[根]{s.}{cabelo; penugem; pena; pele | mofo; bolor | mao, uma unidade fracionária de dinheiro na China; uma moeda de dez centavos | lã}
  \definition{v.}{depreciar; desvalorizar; rebaixar
ficar bravo; explodir}
\end{entry}

\begin{entry}{毛病}{mao2bing4}{4,10}{⽑、⽧}[HSK 3]
  \definition[个]{s.}{doença ou deficiência | problema; fracasso | mau hábito; deficiência}
\end{entry}

\begin{entry}{毛巾}{mao2jin1}{4,3}{⽑、⼱}[HSK 4]
  \definition[条]{s.}{toalha; toalha de banho}
\end{entry}

\begin{entry}{毛衣}{mao2 yi1}{4,6}{⽑、⾐}[HSK 4]
  \definition[件]{s.}{suéter; blusa feita de lã}
\end{entry}

\begin{entry}{矛}{mao2}{5}{⽭}[Kangxi 110]
  \definition{s.}{lança; lanceta}
\end{entry}

\begin{entry}{牦牛}{mao2niu2}{8,4}{⽜、⽜}
  \definition{s.}{iaque}
\end{entry}

\begin{entry}{猫}{mao2}{11}{⽝}
  \definition{v.}{utilizado em 猫腰 \dpy{mao2yao1}}
  \seeref{猫腰}{mao2yao1}
\end{entry}

\begin{entry}{猫腰}{mao2yao1}{11,13}{⽝、⾁}
  \definition{v.}{curvar-se}
\end{entry}

\begin{entry}{冒险}{mao4xian3}{9,9}{⽇、⾩}
  \definition{adj.}{corajoso}
  \definition{s.}{risco | aventura}
  \definition{v.+compl.}{correr risco | arriscar-se | aventurar-se em}
\end{entry}

\begin{entry}{贸易}{mao4yi4}{9,8}{⾙、⽇}
  \definition[个]{s.}{transação comercial}
  \definition{v.}{fazer uma transação comercial}
\end{entry}

\begin{entry}{帽子}{mao4zi5}{12,3}{⼱、⼦}[HSK 4]
  \definition[顶,个,种]{s.}{boné; chapéu; capacete | etiqueta; rótulo; marca}
\end{entry}

\begin{entry}{没}{mei2}{7}{⽔}[HSK 1]
  \definition{adv.}{não ter | não há | ficar sem}
  \definition{pref.}{não (prefixo negativo para verbos, traduzido para outras línguas com verbos no pretérito)}
  \seeref{没}{mo4}
\end{entry}

\begin{entry}{没错}{mei2 cuo4}{7,13}{⽔、⾦}[HSK 4]
  \definition{adv.}{está certo; é isso mesmo; não há como errar}
\end{entry}

\begin{entry}{没法儿}{mei2 fa3r5}{7,8,2}{⽔、⽔、⼉}[HSK 4]
  \definition{adv.}{não pode; sem chance}
\end{entry}

\begin{entry}{没关系}{mei2 guan1xi5}{7,6,7}{⽔、⼋、⽷}[HSK 1]
  \definition{v.}{não ter problema | não ter importância | não fazer mal}
  \seeref{没有关系}{mei2you3guan1xi5}
\end{entry}

\begin{entry}{没了}{mei2le5}{7,2}{⽔、⼅}
  \definition{v.}{estar morto | deixar de existir}
\end{entry}

\begin{entry}{没什么}{mei2 shen2me5}{7,4,3}{⽔、⼈、⼃}[HSK 1]
  \definition{expr.}{não é nada | está tudo bem | não importa | não importa}
\end{entry}

\begin{entry}{没事儿}{mei2 shi4r5}{7,8,2}{⽔、⼅、⼉}[HSK 1]
  \definition{expr.}{livre de trabalho | sem problemas | não é importante |não é nada |deixa para lá}
  \definition{v.}{ter tempo livre}
\end{entry}

\begin{entry}{没想到}{mei2 xiang3 dao4}{7,13,8}{⽔、⼼、⼑}[HSK 4]
  \definition{expr.}{não esperava; inesperado}
\end{entry}

\begin{entry}{没用}{mei2 yong4}{7,5}{⽔、⽤}[HSK 3]
  \definition{adj.}{inútil; imprestável; sem valor; sem préstimo; vão; que não serve para nada}
\end{entry}

\begin{entry}{没有}{mei2you3}{7,6}{⽔、⽉}[HSK 1]
  \definition{v.}{não há | não tem | não existe}
\end{entry}

\begin{entry}{没有关系}{mei2you3guan1xi5}{7,6,6,7}{⽔、⽉、⼋、⽷}
  \definition{v.}{não ter problema | não ter importância | não fazer mal}
  \seeref{没关系}{mei2 guan1xi5}
\end{entry}

\begin{entry}{没有意思}{mei2you3yi4si5}{7,6,13,9}{⽔、⽉、⼼、⼼}
  \definition{adj.}{tedioso | chato | sem interesse}
\end{entry}

\begin{entry}{眉}{mei2}{9}{⽬}
  \definition{s.}{sobrancelha | margem superior}
\end{entry}

\begin{entry}{眉毛}{mei2mao5}{9,4}{⽬、⽑}
  \definition[根]{s.}{sobrancelha}
\end{entry}

\begin{entry}{眉头}{mei2tou2}{9,5}{⽬、⼤}
  \definition{s.}{testa}
\end{entry}

\begin{entry}{媒体}{mei2ti3}{12,7}{⼥、⼈}[HSK 3]
  \definition[家,个,种]{s.}{mídia; mídia de massa}
\end{entry}

\begin{entry}{每}{mei3}{7}{⽏}[HSK 3]
  \definition*{s.}{sobrenome Mei}
  \definition{adv.}{frequentemente; todo}
  \definition{pron.}{cada; cada um; cada qual;  todo}
\end{entry}

\begin{entry}{每次}{mei3ci4}{7,6}{⽏、⽋}
  \definition{adv.}{toda vez | cada vez}
\end{entry}

\begin{entry}{每个人}{mei3ge5ren2}{7,3,2}{⽏、⼈、⼈}
  \definition{pron.}{todo mundo | todos}
\end{entry}

\begin{entry}{每天}{mei3tian1}{7,4}{⽏、⼤}
  \definition{adv.}{todo dia | cada dia}
\end{entry}

\begin{entry}{美}{mei3}{9}{⽺}[HSK 3]
  \definition*{s.}{Abreviatura de América (美洲) | Abreviatura de Estados Unidos da América (美国)}
  \definition{adj.}{lindo; bonito; belo; atraente | satisfatório; bom; agradável}
  \definition{v.}{embelezar; enfeitar | orgulhar-se de; estar satisfeito consigo mesmo}
  \seealsoref{美国}{mei3guo2}
  \seealsoref{美洲}{mei3zhou1}
\end{entry}

\begin{entry}{美国}{mei3guo2}{9,8}{⽺、⼞}
  \definition*{s.}{Estados Unidos da América}
\end{entry}

\begin{entry}{美国人}{mei3guo2ren2}{9,8,2}{⽺、⼞、⼈}
  \definition{s.}{americano | pessoa ou povo dos Estados Unidos da América}
\end{entry}

\begin{entry}{美好}{mei3 hao3}{9,6}{⽺、⼥}[HSK 3]
  \definition{adj.}{bem; feliz; glorioso}
\end{entry}

\begin{entry}{美甲}{mei3jia3}{9,5}{⽺、⽥}
  \definition{s.}{manicure e/ou pedicure}
\end{entry}

\begin{entry}{美金}{mei3 jin1}{9,8}{⽺、⾦}[HSK 4]
  \definition{s.}{USD; dólar americano: a moeda local dos Estados Unidos}
\end{entry}

\begin{entry}{美丽}{mei3li4}{9,7}{⽺、⼀}[HSK 3]
  \definition{adj.}{bonito; lindo}
\end{entry}

\begin{entry}{美女}{mei3 nv3}{9,3}{⽺、⼥}[HSK 4]
  \definition[个,位]{s.}{beldade; mulher bonita; uma jovem linda}
\end{entry}

\begin{entry}{美食}{mei3 shi2}{9,9}{⽺、⾷}[HSK 3]
  \definition[种,道,桌]{s.}{iguaria; comida deliciosa}
\end{entry}

\begin{entry}{美术}{mei3shu4}{9,5}{⽺、⽊}[HSK 3]
  \definition[种]{s.}{arte; belas artes | pintura}
\end{entry}

\begin{entry}{美味}{mei3wei4}{9,8}{⽺、⼝}
  \definition{adj.}{delicioso}
  \definition{s.}{comida deliciosa | delicadeza (\emph{delicacy})}
\end{entry}

\begin{entry}{美学}{mei3xue2}{9,8}{⽺、⼦}
  \definition{s.}{estética}
\end{entry}

\begin{entry}{美元}{mei3yuan2}{9,4}{⽺、⼉}[HSK 3]
  \definition*[元,笔,沓]{s.}{Dólar Americano}
\end{entry}

\begin{entry}{美洲}{mei3zhou1}{9,9}{⽺、⽔}
  \definition*{s.}{América (incluindo Norte, Central e Sul)}
\end{entry}

\begin{entry}{美洲人}{mei3zhou1ren2}{9,9,2}{⽺、⽔、⼈}
  \definition{s.}{americano | pessoa ou povo do continente Americano}
\end{entry}

\begin{entry}{妹}{mei4}{8}{⼥}[HSK 1]
  \definition[个]{s.}{irmã mais nova}
  \seeref{妹妹}{mei4mei5}
\end{entry}

\begin{entry}{妹夫}{mei4fu5}{8,4}{⼥、⼤}
  \definition{s.}{marido da irmã mais nova}
\end{entry}

\begin{entry}{妹妹}{mei4mei5}{8,8}{⼥、⼥}[HSK 1]
  \definition[个]{s.}{irmã mais nova | mulher jovem}
\end{entry}

\begin{entry}{魅力}{mei4li4}{14,2}{⿁、⼒}
  \definition{s.}{charme | fascínio | glamour | carisma}
\end{entry}

\begin{entry}{闷热}{men1re4}{7,10}{⾨、⽕}
  \definition{adj.}{abafado | quente e abafado | sufocantemente quente | quente e sensual}
\end{entry}

\begin{entry}{门}{men2}{3}{⾨}[HSK 1][Kangxi 169]
  \definition*{s.}{sobrenome Men}
  \definition{clas.}{para canhão | para lição de casa, tecnologia, etc.}
  \definition{s.}{porta | portão | entrada; saída | interruptor | válvula |maneira | método | acesso | família | casa | escola (de pensamento) | seita (religiosa) | ramo de estudo | categoria; classe | filo}
\end{entry}

\begin{entry}{门口}{men2kou3}{3,3}{⾨、⼝}[HSK 1]
  \definition[个]{s.}{porta | portão}
\end{entry}

\begin{entry}{门票}{men2piao4}{3,11}{⾨、⽰}[HSK 1]
  \definition{s.}{bilhete de entrada | bilhete de admissão}
\end{entry}

\begin{entry}{们}{men5}{5}{⼈}[HSK 1]
  \definition{part.}{sufixo para plural de pronomes e substantivos referentes a indivíduos}
\end{entry}

\begin{entry}{蒙面}{meng2mian4}{13,9}{⾋、⾯}
  \definition{adj.}{descarado | desavergonhado | mascarado}
  \definition{v.}{cobrir o rosto | usar uma máscara}
\end{entry}

\begin{entry}{猛}{meng3}{11}{⽝}
  \definition{adj.}{feroz | violento | corajoso | abrupto | (gíria) incrível}
  \definition{adv.}{de repente}
\end{entry}

\begin{entry}{猛然}{meng3ran2}{11,12}{⽝、⽕}
  \definition{adv.}{de repente | abruptamente}
\end{entry}

\begin{entry}{懵懂}{meng3dong3}{18,15}{⼼、⼼}
  \definition{adj.}{confuso | ignorante}
\end{entry}

\begin{entry}{梦}{meng4}{11}{⼣}[HSK 4]
  \definition*{s.}{sobrenome Meng}
  \definition[场,个]{s.}{sonho; atividade de representação no cérebro durante o sono}
  \definition{v.}{sonhar; ter um sonho}
\end{entry}

\begin{entry}{梦见}{meng4 jian4}{11,4}{⼣、⾒}[HSK 4]
  \definition{v.}{sonhar; sonhar com; ver em um sonho}
\end{entry}

\begin{entry}{梦想}{meng4xiang3}{11,13}{⼣、⼼}[HSK 4]
  \definition[个]{s.}{sonhar; esperança vã; sonho inalcançável}
  \definition{v.}{sonhar; sonhar com carinho; desejar ardentemente}
\end{entry}

\begin{entry}{眯}{mi1}{11}{⽬}
  \definition{v.}{estreitar os olhos | esmagar | (dialeto) tirar uma soneca}
  \seeref{眯}{mi2}
\end{entry}

\begin{entry}{迷}{mi2}{9}{⾡}[HSK 3]
  \definition*{s.}{sobrenome Mi}
  \definition{adj.}{perdido; confuso}
  \definition{s.}{fã; entusiasta; fanático}
  \definition{v.}{estar confuso; perder o rumo; se perder-se | ficar fascinado por; entregar-se a; ficar encantado com (por); ser louco por | confundir; desorientar; fascinar; encantar}
\end{entry}

\begin{entry}{迷宫}{mi2gong1}{9,9}{⾡、⼧}
  \definition{s.}{labirinto}
\end{entry}

\begin{entry}{迷恋}{mi2lian4}{9,10}{⾡、⼼}
  \definition{adj.}{obcecado}
  \definition{v.}{estar/ser apaixonado por | ficar encantado por | estar/ser obcecado por}
\end{entry}

\begin{entry}{迷路}{mi2lu4}{9,13}{⾡、⾜}
  \definition{s.}{labirinto | ouvido interno}
  \definition{v.+compl.}{perder o caminho | perder-se | seguir pelo caminho errado | não conseguir encontrar o caminho}
\end{entry}

\begin{entry}{迷你}{mi2ni3}{9,7}{⾡、⼈}
  \definition{adj.}{(empréstimo linguístico) mini, como em minissaia ou \emph{Mini Cooper}}
\end{entry}

\begin{entry}{迷人}{mi2ren2}{9,2}{⾡、⼈}
  \definition{adj.}{fascinante | encantador | tentador}
\end{entry}

\begin{entry}{眯}{mi2}{11}{⽬}
  \definition{v.}{cegar (como com poeira)}
  \seeref{眯}{mi1}
\end{entry}

\begin{entry}{米}{mi3}{6}{⽶}[HSK 2,3][Kangxi 119]
  \definition*{s.}{sobrenome Mi}
  \definition{clas.}{metro (m)}
  \definition{s.}{arroz | semente descascada}
\end{entry}

\begin{entry}{米饭}{mi3fan4}{6,7}{⽶、⾷}[HSK 1]
  \definition{s.}{arroz cozido}
\end{entry}

\begin{entry}{秘密}{mi4mi4}{10,11}{⽲、⼧}[HSK 4]
  \definition{adj.}{secreto}
  \definition[个]{s.}{segredo; algo secreto; coisas que você não quer que as pessoas saibam}
\end{entry}

\begin{entry}{秘书}{mi4shu1}{10,4}{⽲、⼄}[HSK 4]
  \definition[个,位,名]{s.}{o cargo de secretário; funções de secretariado | secretário; pessoas encarregadas da correspondência e que auxiliam o chefe do órgão ou departamento na condução diária de seu trabalho}
\end{entry}

\begin{entry}{密}{mi4}{11}{⼧}[HSK 4]
  \definition*{s.}{sobrenome Mi}
  \definition{adj.}{fechado; denso; espesso | íntimo; próximo; afetuoso | delicado; fino; cuidadoso; meticuloso}
  \definition{adv.}{secretamente}
  \definition{s.}{segredo | densidade}
\end{entry}

\begin{entry}{密码}{mi4ma3}{11,8}{⼧、⽯}[HSK 4]
  \definition[个]{s.}{código; senha;}
\end{entry}

\begin{entry}{密切}{mi4qie4}{11,4}{⼧、⼑}[HSK 4]
  \definition{adj.}{próximo; íntimo; relacionamento próximo}
  \definition{adv.}{cuidadosamente; atentamente; intimamente}
  \definition{v.}{tornar-se próximo; tornar-se íntimo; conectar-se}
\end{entry}

\begin{entry}{蜜桃}{mi4tao2}{14,10}{⾍、⽊}
  \definition{s.}{pêssego suculento}
\end{entry}

\begin{entry}{棉}{mian2}{12}{⽊}
  \definition{s.}{termo genérico para algodão ou paina | algodão | acolchoado ou estofado com algodão}
\end{entry}

\begin{entry}{免得}{mian3de5}{7,11}{⼉、⼻}
  \definition{conj.}{de modo a não | para evitar | para que não}
\end{entry}

\begin{entry}{免费}{mian3fei4}{7,9}{⼉、⾙}[HSK 4]
  \definition{s.}{gratuito; sem custo}
  \definition{v.+compl.}{isentar de taxas; tonar grátis}
\end{entry}

\begin{entry}{免税}{mian3shui4}{7,12}{⼉、⽲}
  \definition{adj.}{isento de impostos (tributação)}
  \definition{s.}{livre de impostos | isenção de impostos}
  \definition{v.+compl.}{isentar impostos}
\end{entry}

\begin{entry}{靣}{mian4}{8}{⼀}
  \variantof{面}
\end{entry}

\begin{entry}{面}{mian4}{9}{⾯}[HSK 2][Kangxi 176]
  \definition{clas.}{para objetos com superfície plana como tambores, espelhos, bandeiras, etc.}
  \definition{s.}{farinha | massa | (gíria) (uma pessoa) ineficaz | face | superfície | lado | lado de fora}
\end{entry}

\begin{entry}{面包}{mian4bao1}{9,5}{⾯、⼓}[HSK 1]
  \definition[个,片,袋,块]{s.}{pão}[我买八个面包了。(Comprei oito pães.) | 他在吃两片面包。(Ele está comendo duas fatias de pão.) | 我在家里带了一袋面包。(Trouxe um saco de pão para casa.) | 我拿了一块面包。(Peguei um pedaço de pão.)]
\end{entry}

\begin{entry}{面对}{mian4dui4}{9,5}{⾯、⼨}[HSK 3]
  \definition{v.}{enfrentar; defrontar | confrontar (problema)}
\end{entry}

\begin{entry}{面对面}{mian4dui4mian4}{9,5,9}{⾯、⼨、⾯}
  \definition{expr.}{cara a cara}
\end{entry}

\begin{entry}{面对面吃面}{mian4dui4mian4 chi1 mian4}{9,5,9,6,9}{⾯、⼨、⾯、⼝、⾯}
  \definition{expr.}{Comer macarrão cara a cara; indica que o seu estado atual, ou algumas das posições em que você está, ou algumas das coisas que você fez são muito claras}
\end{entry}

\begin{entry}{面积}{mian4ji1}{9,10}{⾯、⽲}[HSK 3]
  \definition{s.}{área (de um andar, pedaço de terreno, etc.); área de uma superfície}
\end{entry}

\begin{entry}{面临}{mian4lin2}{9,9}{⾯、⼁}[HSK 4]
  \definition{v.}{ser confrontado com; encontrar (uma situação) na frente de}
\end{entry}

\begin{entry}{面前}{mian4 qian2}{9,9}{⾯、⼑}[HSK 2]
  \definition{adv.}{antes | na frente de | na (frente) de}
\end{entry}

\begin{entry}{面试}{mian4 shi4}{9,8}{⾯、⾔}[HSK 4]
  \definition[次]{s.}{entrevista; audição}
\end{entry}

\begin{entry}{面条}{mian4tiao2}{9,7}{⾯、⽊}
  \definition{s.}{macarrão | espaguete}
\end{entry}

\begin{entry}{面条儿}{mian4tiao2r5}{9,7,2}{⾯、⽊、⼉}[HSK 1]
  \definition{s.}{macarrão | \emph{noodles}}
\end{entry}

\begin{entry}{面团}{mian4tuan2}{9,6}{⾯、⼞}
  \definition{s.}{massa | pasta}
\end{entry}

\begin{entry}{糆}{mian4}{15}{⽶}
  \variantof{面}
\end{entry}

\begin{entry}{麫}{mian4}{15}{⿆}
  \variantof{面}
\end{entry}

\begin{entry}{描述}{miao2 shu4}{11,8}{⼿、⾡}[HSK 4]
  \definition[段,种]{s.}{descrição; trecho que descreve um evento ou uma cena}
  \definition{v.}{descrever; representar}
\end{entry}

\begin{entry}{描写}{miao2xie3}{11,5}{⼿、⼍}[HSK 4]
  \definition{v.}{representar; retratar; descrever; usar a linguagem e as palavras para transmitir uma imagem concreta de uma pessoa, evento ou situação}
\end{entry}

\begin{entry}{秒}{miao3}{9}{⽲}
  \definition{adv.}{(coloquial) instantaneamente}
  \definition{s.}{segundo (unidade de tempo) | segundo (unidade de medida angular)}
\end{entry}

\begin{entry}{妙招}{miao4zhao1}{7,8}{⼥、⼿}
  \definition{adj.}{escorregadio}
  \definition{s.}{movimento inteligente | maneira inteligente de fazer algo}
\end{entry}

\begin{entry}{灭火}{mie4huo3}{5,4}{⽕、⽕}
  \definition{s.}{combate a incêndios}
  \definition{v.}{extinguir um incêndio}
\end{entry}

\begin{entry}{民间}{min2jian1}{5,7}{⽒、⾨}[HSK 3]
  \definition{s.}{povo (entre as pessoas) | não governamental; de pessoa para pessoa}
\end{entry}

\begin{entry}{民众}{min2zhong4}{5,6}{⽒、⼈}
  \definition{s.}{a população | as massas | as pessoas comuns}
\end{entry}

\begin{entry}{民主}{min2zhu3}{5,5}{⽒、⼂}
  \definition{adj.}{democrático}
  \definition{s.}{democracia}
\end{entry}

\begin{entry}{民族}{min2zu2}{5,11}{⽒、⽅}[HSK 3]
  \definition[个]{s.}{nação | grupo étnico}
\end{entry}

\begin{entry}{名}{ming2}{6}{⼝}[HSK 2]
  \definition*{s.}{sobrenome Ming}
  \definition{s.}{nome | denominação | fama | reputação}
\end{entry}

\begin{entry}{名称}{ming2 cheng1}{6,10}{⼝、⽲}[HSK 2]
  \definition[个,种]{s.}{nome | designação}
\end{entry}

\begin{entry}{名单}{ming2 dan1}{6,8}{⼝、⼗}[HSK 2]
  \definition[个]{s.}{lista | lista de nomes}
\end{entry}

\begin{entry}{名牌儿}{ming2 pai2r5}{6,12,2}{⼝、⽚、⼉}[HSK 4]
  \definition*{s.}{Marca famosa}
\end{entry}

\begin{entry}{名片}{ming2pian4}{6,4}{⼝、⽚}[HSK 4]
  \definition[张,盒,叠]{s.}{cartão de visita; um pedaço de papel retangular com o nome, o cargo, o endereço etc. impressos}
\end{entry}

\begin{entry}{名人}{ming2 ren2}{6,2}{⼝、⼈}[HSK 4]
  \definition{s.}{celebridade; pessoa famosa}
\end{entry}

\begin{entry}{名字}{ming2zi5}{6,6}{⼝、⼦}[HSK 1]
  \definition[个]{s.}{nome (de uma pessoa ou coisa)}
\end{entry}

\begin{entry}{明白}{ming2bai5}{8,5}{⽇、⽩}[HSK 1]
  \definition{adj.}{compreendido | percebido | óbvio | inequívoco}
  \definition{v.}{compreender | perceber}
\end{entry}

\begin{entry}{明明}{ming2ming2}{8,8}{⽇、⽇}
  \definition{adv.}{obviamente | claramente}
\end{entry}

\begin{entry}{明年}{ming2nian2}{8,6}{⽇、⼲}[HSK 1]
  \definition{adv.}{próximo ano}
\end{entry}

\begin{entry}{明确}{ming2que4}{8,12}{⽇、⽯}[HSK 3]
  \definition{adj.}{claro; definido; específico}
  \definition{v.}{deixar claro; tornar definitivo}
\end{entry}

\begin{entry}{明天}{ming2tian1}{8,4}{⽇、⼤}[HSK 1]
  \definition{adv.}{amanhã}
\end{entry}

\begin{entry}{明显}{ming2xian3}{8,9}{⽇、⽇}[HSK 3]
  \definition{adj.}{claro; óbvio; distinto}
\end{entry}

\begin{entry}{明星}{ming2xing1}{8,9}{⽇、⽇}[HSK 2]
  \definition[个,位,颗]{s.}{estrela | talento de ponta | estrela (artista) | estrela brilhante | estrela brilhante}
\end{entry}

\begin{entry}{明珠}{ming2zhu1}{8,10}{⽇、⽟}
  \definition{s.}{pérola | jóia (de grande valor)}
\end{entry}

\begin{entry}{鸣}{ming2}{8}{⿃}
  \definition{v.}{chorar (pássaros, animais e insetos) | fazer um som | dar voz (gratidão, queixas, etc.)}
\end{entry}

\begin{entry}{命运}{ming4yun4}{8,7}{⼝、⾡}[HSK 3]
  \definition[个]{s.}{tendência de desenvolvimento; tendência de futuro | destino; sina; sorte}
\end{entry}

\begin{entry}{摸}{mo1}{13}{⼿}[HSK 4]
  \definition{v.}{sentir; acariciar; tocar; tocar (um objeto) levemente com a mão e depois removê-lo ou mover a mão suavemente sobre a superfície do objeto | sentir para; tatear para; sentir algo com as mãos | descobrir; sentir; sondar; explorar; tentar fazer ou entender | sentir o caminho; tatear no escuro; andar por estradas que você não consegue reconhecer | furtar; roubar}
\end{entry}

\begin{entry}{模仿}{mo2fang3}{14,6}{⽊、⼈}
  \definition{v.}{imitar | copiar}
\end{entry}

\begin{entry}{模式}{mo2shi4}{14,6}{⽊、⼷}
  \definition{s.}{modelo | modo | método | padrão}
\end{entry}

\begin{entry}{模特儿}{mo2 te4r5}{14,10,2}{⽊、⽜、⼉}[HSK 4]
  \definition[个]{s.}{modelo (pessoa que posa para um fotógrafo ou pintor ou escultor); objeto de representação ou referência usado por artistas para esboços e esculturas, como o corpo humano, objetos, modelos etc.; também se refere aos arquétipos que os estudiosos da literatura usam para retratar seus personagens | modelo (uma pessoa que usa roupas para exibir modas); pessoa ou manequim usado para exibir estilos de roupas}
\end{entry}

\begin{entry}{模型}{mo2xing2}{14,9}{⽊、⼟}[HSK 4]
  \definition[个]{s.}{modelo; padrão; itens feitos em escala com base em objetos ou desenhos | molde; padrão; molde para fundir máquinas, objetos, etc.}
\end{entry}

\begin{entry}{膜拜}{mo2bai4}{14,9}{⾁、⼿}
  \definition{v.}{ajoelhar-se e se curvar com as mãos unidas no nível da testa | ter ou mostrar sentimentos fortes de respeito e admiração por um deus}
\end{entry}

\begin{entry}{摩托}{mo2tuo1}{15,6}{⼿、⼿}
  \definition{s.}{(empréstimo linguístico) motor | (empréstimo linguístico) motocicleta, abreviação de 摩托车}
  \seeref{摩托车}{mo2tuo1che1}
\end{entry}

\begin{entry}{摩托车}{mo2tuo1che1}{15,6,4}{⼿、⼿、⾞}
  \definition[辆,部]{s.}{(empréstimo linguístico) motocicleta}
\end{entry}

\begin{entry}{磨}{mo2}{16}{⽯}
  \definition{v.}{moer | polir | afiar | desgastar | esfregar}
  \seeref{磨}{mo4}
\end{entry}

\begin{entry}{磨菇}{mo2gu5}{16,11}{⽯、⾋}
  \variantof{蘑菇}
\end{entry}

\begin{entry}{蘑菇}{mo2gu5}{19,11}{⾋、⾋}
  \definition{s.}{cogumelo}
  \definition{v.}{mandriar | embromar | amofinar | incomodar alguém com solicitações ou interrupções frequentes ou persistentes}
\end{entry}

\begin{entry}{魔术}{mo2shu4}{20,5}{⿁、⽊}
  \definition{s.}{magia}
\end{entry}

\begin{entry}{魔头}{mo2tou2}{20,5}{⿁、⼤}
  \definition{s.}{monstro | diabo}
\end{entry}

\begin{entry}{抹泪}{mo3lei4}{8,8}{⼿、⽔}
  \definition{v.}{limpar as lágrimas | (figurativo) derramar lágrimas}
\end{entry}

\begin{entry}{末}{mo4}{5}{⽊}[HSK 4]
  \definition{adj.}{último; final}
  \definition{s.}{ponta; terminal; extremidade; o final de algo | não essenciais; detalhes secundários | fim; final | pó; poeira | um papel na ópera tradicional}
\end{entry}

\begin{entry}{没}{mo4}{7}{⽔}
  \definition{adj.}{afogado}
  \definition{v.}{acabar | morrer | inundar}
  \variantof{没}
\end{entry}

\begin{entry}{莫名其妙}{mo4ming2qi2miao4}{10,6,8,7}{⾋、⼝、⼋、⼥}
  \definition{adj.}{desconcertante | bizzaro | inexplicável | perplexo}
\end{entry}

\begin{entry}{墨镜}{mo4jing4}{15,16}{⿊、⾦}
  \definition[只,双,副]{s.}{óculos escuros}
\end{entry}

\begin{entry}{磨}{mo4}{16}{⽯}
  \definition{s.}{mó (pedra pesada e redonda para moinho)}
  \definition{v.}{moer}
\end{entry}

\begin{entry}{默默}{mo4mo4}{16,16}{⿊、⿊}[HSK 4]
  \definition{adj.}{mudo; silencioso}
  \definition{adv.}{silenciosamente}
\end{entry}

\begin{entry}{默契}{mo4qi4}{16,9}{⿊、⼤}
  \definition{adj.}{(de membros da equipe) bem coordenados}
  \definition{s.}{entendimento tácito | entendimento mútuo | conectado em um nível mútuo profundo | (de membros da equipe) bem coordenados}
\end{entry}

\begin{entry}{某}{mou3}{9}{⽊}[HSK 3]
  \definition{pron.}{um certo alguém ou coisa; algum | usado para substituir seu próprio nome}
\end{entry}

\begin{entry}{模具}{mu2ju4}{14,8}{⽊、⼋}
  \definition{s.}{molde | matriz | padrão}
\end{entry}

\begin{entry}{母亲}{mu3qin1}{5,9}{⽏、⼇}[HSK 3]
  \definition[位,个]{s.}{mãe}
\end{entry}

\begin{entry}{母语}{mu3yu3}{5,9}{⽏、⾔}
  \definition{s.}{língua materna | língua nativa}
\end{entry}

\begin{entry}{木偶}{mu4'ou3}{4,11}{⽊、⼈}
  \definition{s.}{fantoche, marionete}
\end{entry}

\begin{entry}{木头}{mu4tou5}{4,5}{⽊、⼤}[HSK 3]
  \definition{adj.}{estúpido; cabeça-dura}
  \definition[块,根]{s.}{tronco; madeira; viga; prancha}
\end{entry}

\begin{entry}{目标}{mu4biao1}{5,9}{⽬、⽊}[HSK 3]
  \definition[个]{s.}{alvo; objetivo | objetivo; destino}
\end{entry}

\begin{entry}{目的}{mu4di4}{5,8}{⽬、⽩}[HSK 2]
  \definition[个]{s.}{objetivo | meta | alvo | propósito}
\end{entry}

\begin{entry}{目前}{mu4qian2}{5,9}{⽬、⼑}[HSK 3]
  \definition{adv.}{agora; recentemente; no momento; no presente}
\end{entry}

\begin{entry}{幕}{mu4}{13}{⼱}
  \definition{s.}{cortina ou tela | dossel ou tenda | quartel de um general | ato (de uma peça)}
\end{entry}

%%%%% EOF %%%%%


%%%
%%% N
%%%

\section*{N}\addcontentsline{toc}{section}{N}

\begin{entry}{那}{na1}{6}{⾢}
  \definition*{s.}{sobrenome Na}
\end{entry}

\begin{entry}{拿}{na2}{10}{⼿}[HSK 1]
  \definition{part.}{usado da mesma forma que 把: para marcar o seguinte substantivo seguinte como objeto direto}
  \definition{v.}{segurar | tomar | pegar em}
\end{entry}

\begin{entry}{拿出}{na2 chu1}{10,5}{⼿、⼐}[HSK 2]
  \definition{v.}{apresentar (evidências) | prover | apresentar (uma proposta) | colocar para fora | retirar}
\end{entry}

\begin{entry}{拿到}{na2 dao4}{10,8}{⼿、⼑}[HSK 2]
  \definition{v.}{pegar | obter}
\end{entry}

\begin{entry}{那}{na3}{6}{⾢}
  \variantof{哪}
\end{entry}

\begin{entry}{哪}{na3}{9}{⼝}[HSK 1,4]
  \definition{adv.}{para expressar uma pergunta retórica}
  \definition{pron.}{qual?; o que? | qualquer; ser usado em um sentido geral}
  \seeref{哪}{na5}
  \seeref{哪}{nei3}
\end{entry}

\begin{entry}{哪国人}{na3 guo2ren2}{9,8,2}{⼝、⼞、⼈}
  \definition{expr.}{de qual país?}
\end{entry}

\begin{entry}{哪里}{na3 li3}{9,7}{⼝、⾥}[HSK 1]
  \definition{adv.}{onde?}
\end{entry}

\begin{entry}{哪怕}{na3pa4}{9,8}{⼝、⼼}[HSK 4]
  \definition{conj.}{mesmo; mesmo se; mesmo que; não importa o quão}
\end{entry}

\begin{entry}{哪儿}{na3r5}{9,2}{⼝、⼉}[HSK 1]
  \definition{adv.}{onde?}
\end{entry}

\begin{entry}{哪些}{na3xie1}{9,8}{⼝、⼆}[HSK 1]
  \definition{pron.}{quais?}
\end{entry}

\begin{entry}{那}{na4}{6}{⾢}[HSK 1,2]
  \definition{conj.}{nessa situação | nesse caso}
  \definition{pron.}{aquele | aquilo}
\end{entry}

\begin{entry}{那边}{na4bian5}{6,5}{⾢、⾡}[HSK 1]
  \definition{pron.}{ali | acolá}
\end{entry}

\begin{entry}{那会儿}{na4 hui4r5}{6,6,2}{⾢、⼈、⼉}[HSK 2]
  \definition{pron.}{então | naquela época}
\end{entry}

\begin{entry}{那里}{na4 li3}{6,7}{⾢、⾥}[HSK 1]
  \definition{pron.}{lá | ali}
\end{entry}

\begin{entry}{那么}{na4 me5}{6,3}{⾢、⼃}[HSK 2]
  \definition{adv.}{então | como aquele | dessa maneira}
\end{entry}

\begin{entry}{那麽}{na4 me5}{6,14}{⾢、⿇}
  \variantof{那么}
\end{entry}

\begin{entry}{那儿}{na4r5}{6,2}{⾢、⼉}[HSK 1]
  \definition{pron.}{lá | ali}
\end{entry}

\begin{entry}{那时}{na4 shi2}{6,7}{⾢、⽇}[HSK 2]
  \definition{pron.}{então | naquela época | naqueles dias}
\end{entry}

\begin{entry}{那时候}{na4 shi2 hou5}{6,7,10}{⾢、⽇、⼈}[HSK 2]
  \definition{adv.}{naquela hora}
\end{entry}

\begin{entry}{那些}{na4xie1}{6,8}{⾢、⼆}[HSK 1]
  \definition{pron.}{aqueles}
\end{entry}

\begin{entry}{那样}{na4 yang4}{6,10}{⾢、⽊}[HSK 2]
  \definition{pron.}{assim | tal | como esse | desse tipo}
\end{entry}

\begin{entry}{哪}{na5}{9}{⼝}
  \definition{part.}{usado depois de uma palavra com a terminação -n, é equivalente a ``啊''}
  \seeref{哪}{na3}
  \seeref{哪}{nei3}
  \seealsoref{啊}{a5}
\end{entry}

\begin{entry}{奶}{nai3}{5}{⼥}[HSK 1]
  \definition[杯,滴,瓶,只,桶]{s.}{seios | leite}
  \definition{v.}{amamentar}
\end{entry}

\begin{entry}{奶茶}{nai3 cha2}{5,9}{⼥、⾋}[HSK 3]
  \definition[杯]{s.}{chá com leite}
\end{entry}

\begin{entry}{奶奶}{nai3nai5}{5,5}{⼥、⼥}[HSK 1]
  \definition[位]{s.}{avó (paterna) | (respeitoso) dona da casa}
\end{entry}

\begin{entry}{耐心}{nai4xin1}{9,4}{⽽、⼼}
  \definition{s.}{paciência}
  \definition{v.}{ser paciente}
\end{entry}

\begin{entry}{男}{nan2}{7}{⽥}[HSK 1]
  \definition{adj.}{masculino}
  \definition{s.}{Barão, o mais baixo de cinco ordens de nobreza}
\end{entry}

\begin{entry}{男孩儿}{nan2hai2r5}{7,9,2}{⽥、⼦、⼉}[HSK 1]
  \definition{s.}{menino | rapaz}
\end{entry}

\begin{entry}{男女}{nan2 nv3}{7,3}{⽥、⼥}[HSK 4]
  \definition{s.}{homens e mulheres; masculino e feminino}
\end{entry}

\begin{entry}{男朋友}{nan2peng2you5}{7,8,4}{⽥、⽉、⼜}[HSK 1]
  \definition{s.}{namorado}
\end{entry}

\begin{entry}{男人}{nan2ren2}{7,2}{⽥、⼈}[HSK 1]
  \definition[个]{s.}{um homem | um macho | cavalheiro | marido}
\end{entry}

\begin{entry}{男生}{nan2sheng1}{7,5}{⽥、⽣}[HSK 1]
  \definition[个]{s.}{aluno | estudante do sexo masculino}
\end{entry}

\begin{entry}{男士}{nan2 shi4}{7,3}{⽥、⼠}[HSK 4]
  \definition{s.}{cavalheiro; \emph{gentleman}}
\end{entry}

\begin{entry}{男子}{nan2zi3}{7,3}{⽥、⼦}[HSK 3]
  \definition[名]{s.}{homem; macho}
\end{entry}

\begin{entry}{南}{nan2}{9}{⼗}[HSK 1]
  \definition*{s.}{sobrenome Nan}
  \definition{s.}{sul}
\end{entry}

\begin{entry}{南边}{nan2bian5}{9,5}{⼗、⾡}[HSK 1]
  \definition{adv.}{sul | lado sul | parte sul | ao sul de}
\end{entry}

\begin{entry}{南部}{nan2 bu4}{9,10}{⼗、⾢}[HSK 3]
  \definition{s.}{parte sul; sul | a parte sul}
\end{entry}

\begin{entry}{南方}{nan2 fang1}{9,4}{⼗、⽅}[HSK 2]
  \definition{s.}{sul | o Sul | a parte sul do país}
\end{entry}

\begin{entry}{南极}{nan2ji2}{9,7}{⼗、⽊}
  \definition*{s.}{Antártico | Pólo Sul}
  \definition{s.}{pólo sul magnético}
\end{entry}

\begin{entry}{南面}{nan2mian4}{9,9}{⼗、⾯}
  \definition{s.}{sul | lado sul}
\end{entry}

\begin{entry}{难}{nan2}{10}{⾫}[HSK 1]
  \definition{adj.}{difícil}
  \definition{s.}{dificuldade}
  \seeref{难}{nan4}
\end{entry}

\begin{entry}{难道}{nan2dao4}{10,12}{⾫、⾡}[HSK 3]
  \definition{adv.}{indica uma pergunta retórica | certamente não significa que\dots?; é possível que\dots?; não me diga\dots; poderia ser que\dots?}
\end{entry}

\begin{entry}{难度}{nan2 du4}{10,9}{⾫、⼴}[HSK 3]
  \definition{s.}{dificuldade; grau de dificuldade}
\end{entry}

\begin{entry}{难过}{nan2guo4}{10,6}{⾫、⾡}[HSK 2]
  \definition{adj.}{triste | ruim | pesaroso | arrependido | difícil}
\end{entry}

\begin{entry}{难看}{nan2 kan4}{10,9}{⾫、⽬}[HSK 2]
  \definition{adj.}{feio | antiestético | vergonhoso | embaraçoso | vergonhoso}
\end{entry}

\begin{entry}{难受}{nan2shou4}{10,8}{⾫、⼜}[HSK 2]
  \definition{adj.}{sofrer dor | sentir-se mal | desconfortável | sentir-se infeliz}
\end{entry}

\begin{entry}{难题}{nan2 ti2}{10,15}{⾫、⾴}[HSK 2]
  \definition[出]{s.}{desafio | problema difícil | pergunta difícil}
\end{entry}

\begin{entry}{难听}{nan2 ting1}{10,7}{⾫、⼝}[HSK 2]
  \definition{adj.}{desagradável de ouvir | ofensivo | grosseiro | escandaloso}
\end{entry}

\begin{entry}{难}{nan4}{10}{⾫}
  \definition{s.}{desastre}
  \definition{v.}{repreender}
  \seeref{难}{nan2}
\end{entry}

\begin{entry}{难免}{nan4mian3}{10,7}{⾫、⼉}[HSK 4]
  \definition{adj.}{inevitável; difícil de evitar}
\end{entry}

\begin{entry}{孬}{nao1}{10}{⼥}
  \definition{adj.}{(dialeto) não (é) bom (contração de 不+好)}
\end{entry}

\begin{entry}{脑袋}{nao3dai5}{10,11}{⾁、⾐}[HSK 4]
  \definition[颗,个]{s.}{cabeça; a parte mais alta do corpo humano ou a parte mais alta de um animal que contém órgãos como a boca, o nariz, os olhos etc. | mente; cérebro; capacidade de pensar, lembrar, etc.}
\end{entry}

\begin{entry}{脑瓜}{nao3gua1}{10,5}{⾁、⽠}
  \definition{s.}{crânio | cérebro | cabeça | mente | mentalidade | ideia}
  \seealsoref{脑瓜子}{nao3gua1zi5}
\end{entry}

\begin{entry}{脑瓜子}{nao3gua1zi5}{10,5,3}{⾁、⽠、⼦}
  \definition{s.}{crânio | cérebro | cabeça | mente | mentalidade | ideia}
  \seealsoref{脑瓜}{nao3gua1}
\end{entry}

\begin{entry}{闹}{nao4}{8}{⾾}[HSK 4]
  \definition{adj.}{barulhento}
  \definition{v.}{fazer barulho; provocar problemas | dar vazão (à sua raiva, ressentimento, etc.) | sofrer de; ser incomodado por; ocorrer (um desastre ou coisa ruim) | fazer;  entrar em ação | agitar; perturbar | brincar; fazer bagunça}
\end{entry}

\begin{entry}{闹钟}{nao4 zhong1}{8,9}{⾾、⾦}[HSK 4]
  \definition[个,台,只]{s.}{despertador; relógios capazes de tocar alarmes em horários predeterminados}
\end{entry}

\begin{entry}{呢}{ne5}{8}{⼝}[HSK 1]
  \definition{part.}{(no final de uma frase declarativa) partícula que indica a continuação de um estado ou ação |  partícula para perguntar sobre a localização (``Onde está\dots?'') | partícula indicando  afirmação forte | partícula indicando que uma pergunta feita anteriormente deve ser aplicada à palavra anterior (``E quanto a\dots?'', ``E\dots?'') | partícula sinalizando uma pausa, para enfatizar as palavras anteriores e permitir que o ouvinte tenha tempo para compreendê-las (``ok?'', ``você está comigo ?'')}
  \seeref{呢}{ni2}
\end{entry}

\begin{entry}{哪}{nei3}{9}{⼝}
  \definition{part.}{qual? (interrogativo, seguido de classificador ou numeral-classificador)}
  \seeref{哪}{na3}
  \seeref{哪}{na5}
\end{entry}

\begin{entry}{内}{nei4}{4}{⼌}[HSK 3]
  \definition*{s.}{sobrenome Nei}
  \definition{adj.}{interno; interior}
  \definition{prep.}{dentro}
  \definition{s.}{interior; lado de dentro; parte de dentro | a esposa ou parentes dela}
\end{entry}

\begin{entry}{内部}{nei4bu4}{4,10}{⼌、⾢}[HSK 4]
  \definition{s.}{interior; dentro; interno; dentro de um determinado intervalo}
\end{entry}

\begin{entry}{内存}{nei4cun2}{4,6}{⼌、⼦}
  \definition{s.}{armazenamento interno | memória do computador | RAM (\emph{random access memory})}
  \seealsoref{随机存取存储器}{sui2ji1cun2qu3cun2chu3qi4}
  \seealsoref{随机存取记忆体}{sui2ji1cun2qu3ji4yi4ti3}
\end{entry}

\begin{entry}{内科}{nei4ke1}{4,9}{⼌、⽲}[HSK 4]
  \definition{s.}{medicina geral; clínica geral; clínica médica}
\end{entry}

\begin{entry}{内燃机}{nei4ran2ji1}{4,16,6}{⼌、⽕、⽊}
  \definition{s.}{motor de combustão interna}
\end{entry}

\begin{entry}{内容}{nei4rong2}{4,10}{⼌、⼧}[HSK 3]
  \definition[个]{s.}{conteúdo; substância}
\end{entry}

\begin{entry}{内心}{nei4 xin1}{4,4}{⼌、⼼}[HSK 3]
  \definition{s.}{coração; interior; íntimo do ser}
\end{entry}

\begin{entry}{内省}{nei4xing3}{4,9}{⼌、⽬}
  \definition{s.}{introspecção}
  \definition{v.}{refletir sobre si mesmo}
\end{entry}

\begin{entry}{能}{neng2}{10}{⾁}[HSK 1]
  \definition*{s.}{sobrenome Neng}
  \definition{adv.}{talvez}
  \definition{s.}{(física)nenergia | habilidade}
  \definition{v.}{poder | ser capaz de}
\end{entry}

\begin{entry}{能不能}{neng2 bu4 neng2}{10,4,10}{⾁、⼀、⾁}[HSK 3]
  \definition{adv.}{pode ou não pode\dots?}
\end{entry}

\begin{entry}{能干}{neng2gan4}{10,3}{⾁、⼲}[HSK 4]
  \definition{adj.}{apto; capaz; competente}
\end{entry}

\begin{entry}{能够}{neng2 gou4}{10,11}{⾁、⼣}[HSK 2]
  \definition{v.}{ser capaz de}
\end{entry}

\begin{entry}{能力}{neng2li4}{10,2}{⾁、⼒}[HSK 3]
  \definition{s.}{habilidade; capacidade; aptidão}
\end{entry}

\begin{entry}{能上能下}{neng2shang4neng2xia4}{10,3,10,3}{⾁、⼀、⾁、⼀}
  \definition{s.}{pronto para aceitar qualquer trabalho, alto ou baixo}
\end{entry}

\begin{entry}{呢}{ni2}{8}{⼝}
  \definition{s.}{material de lã}
  \seeref{呢}{ne5}
\end{entry}

\begin{entry}{泥}{ni2}{8}{⽔}
  \definition{s.}{lama | argila | pasta | polpa}
  \seeref{泥}{ni4}
\end{entry}

\begin{entry}{泥潭}{ni2tan2}{8,15}{⽔、⽔}
  \definition{s.}{atoleiro | lamaçal | charco | pântano}
\end{entry}

\begin{entry}{你}{ni3}{7}{⼈}[HSK 1]
  \definition{pron.}{você (informal) | tu | te | ti | contigo}
  \seeref{您}{nin2}
\end{entry}

\begin{entry}{你的}{ni3 de5}{7,8}{⼈、⽩}
  \definition{pron.}{seu}
\end{entry}

\begin{entry}{你好}{ni3hao3}{7,6}{⼈、⼥}
  \definition{interj.}{Olá! | Oi!}
\end{entry}

\begin{entry}{你们}{ni3men5}{7,5}{⼈、⼈}[HSK 1]
  \definition{pron.}{vocês (informal) | vós | vos | convosco}
\end{entry}

\begin{entry}{你们的}{ni3men5 de5}{7,5,8}{⼈、⼈、⽩}
  \definition{pron.}{vossos}
\end{entry}

\begin{entry}{伲}{ni4}{7}{⼈}
  \definition{pron.}{(dialeto) eu | meu | nosso | nós}
  \seeref{你}{ni3}
\end{entry}

\begin{entry}{泥}{ni4}{8}{⽔}
  \definition{adj.}{contido}
  \seeref{泥}{ni2}
\end{entry}

\begin{entry}{逆境}{ni4jing4}{9,14}{⾡、⼟}
  \definition{s.}{adversidade | tribulação}
\end{entry}

\begin{entry}{年}{nian2}{6}{⼲}[HSK 1]
  \definition*{s.}{sobrenome Nian}
  \definition[个]{clas./s.}{ano}
\end{entry}

\begin{entry}{年初}{nian2 chu1}{6,7}{⼲、⾐}[HSK 3]
  \definition{s.}{o começo do ano}
\end{entry}

\begin{entry}{年代}{nian2dai4}{6,5}{⼲、⼈}[HSK 3]
  \definition[个]{s.}{idade; anos; tempo | uma década de um século}
\end{entry}

\begin{entry}{年底}{nian2 di3}{6,8}{⼲、⼴}[HSK 3]
  \definition[个]{s.}{fim de ano; o fim do ano}
\end{entry}

\begin{entry}{年货}{nian2huo4}{6,8}{⼲、⾙}
  \definition{s.}{mercadorias vendidas no Ano Novo Chinês}
\end{entry}

\begin{entry}{年级}{nian2ji2}{6,6}{⼲、⽷}[HSK 2]
  \definition[个]{s.}{classe | ano (escola)}
\end{entry}

\begin{entry}{年纪}{nian2ji4}{6,6}{⼲、⽷}[HSK 3]
  \definition{s.}{era; época; idade}
\end{entry}

\begin{entry}{年轻}{nian2qing1}{6,9}{⼲、⾞}[HSK 2]
  \definition{adj.}{jovem}
\end{entry}

\begin{entry}{碾碎}{nian3sui4}{15,13}{⽯、⽯}
  \definition{v.}{pulverizar | esmagar}
\end{entry}

\begin{entry}{念}{nian4}{8}{⼼}[HSK 3]
  \definition*{s.}{sobrenome Nian}
  \definition{num.}{vinte; 20}
  \definition{s.}{ideia; pensamento}
  \definition{v.}{ler em voz alta | estudar; frequentar a escola | considerar; levar em conta | sentir falta; pensar em}
\end{entry}

\begin{entry}{鸟}{niao3}{5}{⿃}[HSK 2][Kangxi 196]
  \definition[只,群]{s.}{pássaro}
  \seeref{鸟}{diao3}
\end{entry}

\begin{entry}{鸟儿}{niao3r5}{5,2}{⿃、⼉}
  \definition[只]{s.}{pássaro | ave}
\end{entry}

\begin{entry}{尿}{niao4}{7}{⼫}
  \definition[泡]{s.}{urina}
  \definition{v.}{urinar}
  \seeref{尿}{sui1}
\end{entry}

\begin{entry}{您}{nin2}{11}{⼼}[HSK 1]
  \definition{pron.}{você (formal) | tu | te | ti | contigo}
  \seeref{你}{ni3}
\end{entry}

\begin{entry}{宁}{ning2}{5}{⼧}
  \definition*{s.}{sobrenome Ning}
  \definition{adj.}{calmo, pacífico, sereno | saudável}
  \seeref{宁}{ning4}
\end{entry}

\begin{entry}{宁静}{ning2 jing4}{5,14}{⼧、⾭}[HSK 4]
  \definition{adj.}{calmo; tranquilo; pacífico}
\end{entry}

\begin{entry}{柠檬}{ning2meng2}{9,17}{⽊、⽊}
  \definition{s.}{limão}
\end{entry}

\begin{entry}{拧开}{ning3kai1}{8,4}{⼿、⼶}
  \definition{v.}{desaparafusar | desatarrachar | torcer (uma tampa) | abrir (uma torneira) | ligar (girando um botão) | girar (maçaneta da porta)}
\end{entry}

\begin{entry}{宁}{ning4}{5}{⼧}
  \definition{conj.}{mais\dots do que\dots, melhor\dots do que\dots}
  \seeref{宁}{ning2}
\end{entry}

\begin{entry}{宁可}{ning4ke3}{5,5}{⼧、⼝}
  \definition{conj.}{mais\dots do que\dots | melhor\dots do que\dots}
\end{entry}

\begin{entry}{宁可……也不……}{ning4ke3 ye3bu4}{5,5,3,4}{⼧、⼝、⼄、⼀}
  \definition{conj.}{em vez de\dots}
\end{entry}

\begin{entry}{宁可……也要……}{ning4ke3 ye3yao4}{5,5,3,9}{⼧、⼝、⼄、⾑}
  \definition{conj.}{mesmo que tenhamos que\dots nós iremos\dots}
\end{entry}

\begin{entry}{宁肯}{ning4ken3}{5,8}{⼧、⾁}
  \definition{conj.}{mais\dots do que\dots, melhor\dots do que\dots}
\end{entry}

\begin{entry}{宁愿}{ning4yuan4}{5,14}{⼧、⽕}
  \definition{conj.}{mais\dots do que\dots, melhor\dots do que\dots}
\end{entry}

\begin{entry}{牛}{niu2}{4}{⽜}[HSK 3][Kangxi 93]
  \definition*{s.}{sobrenome Niu}
  \definition{adj.}{muito capaz ou bom
teimoso; arrogante}
  \definition{clas.}{Newton (medida física de força)}
  \definition[头]{s.}{gado; boi | niu (nona das vinte e oito constelações em que a esfera celeste foi dividida, consistindo de seis estrelas, três em Áries e três em Sagitário)}
\end{entry}

\begin{entry}{牛顿}{niu2dun4}{4,10}{⽜、⾴}
  \definition*{s.}{Newton (nome) | newton (N, unidade de força do SI)}
\end{entry}

\begin{entry}{牛郎织女}{niu2lang2zhi1nv3}{4,8,8,3}{⽜、⾢、⽷、⼥}
  \definition*{s.}{Vaqueiro e Tecelã (personagens de contos folclóricos) | amantes separados | Altair e Vega (estrelas)}
\end{entry}

\begin{entry}{牛奶}{niu2nai3}{4,5}{⽜、⼥}[HSK 1]
  \definition[瓶,杯]{s.}{leite de vaca}
\end{entry}

\begin{entry}{牛人}{niu2ren2}{4,2}{⽜、⼈}
  \definition{s.}{(coloquial) o cara | verdadeiro especialista | \emph{badass}}
\end{entry}

\begin{entry}{牛肉}{niu2rou4}{4,6}{⽜、⾁}
  \definition{s.}{carne de vaca | bife}
\end{entry}

\begin{entry}{牛仔裤}{niu2zai3ku4}{4,5,12}{⽜、⼈、⾐}
  \definition[条]{s.}{calça de ganga, jeans}
\end{entry}

\begin{entry}{农村}{nong2cun1}{6,7}{⼍、⽊}[HSK 3]
  \definition[个]{s.}{aldeia; campo; área rural}
\end{entry}

\begin{entry}{农民}{nong2min2}{6,5}{⼍、⽒}[HSK 3]
  \definition[个,位]{s.}{fazendeiro; camponês; campesinato}
\end{entry}

\begin{entry}{农业}{nong2ye4}{6,5}{⼍、⼀}[HSK 3]
  \definition{s.}{agricultura; lavoura}
\end{entry}

\begin{entry}{浓}{nong2}{9}{⽔}[HSK 4]
  \definition{adj.}{denso; espesso; concentrado; um líquido ou gás que contém mais de um determinado ingrediente | grande; forte; profundo (de grau ou extensão) | profundo; (algumas cores) escuro}
\end{entry}

\begin{entry}{弄}{nong4}{7}{⼶}[HSK 2]
  \definition{s.}{beco | viela | travessa}
  \seeref{弄}{long4}
\end{entry}

\begin{entry}{努力}{nu3li4}{7,2}{⼒、⼒}[HSK 2]
  \definition{adj.}{diligente | aplicado}
  \definition{s.}{esforçar-se | se esforçar}
\end{entry}

\begin{entry}{怒骂}{nu4ma4}{9,9}{⼼、⾺}
  \definition{v.}{praguejar de raiva}
\end{entry}

\begin{entry}{暖}{nuan3}{13}{⽇}
  \definition{adj.}{quente}
  \definition{v.}{esquentar}
\end{entry}

\begin{entry}{暖和}{nuan3huo5}{13,8}{⽇、⼝}[HSK 3]
  \definition{adj.}{morno; agradável e quente}
  \definition{v.}{aquecer}
\end{entry}

\begin{entry}{暖气}{nuan3qi4}{13,4}{⽇、⽓}[HSK 4]
  \definition[个]{s.}{aquecedor; aquecimento; aquecimento central}
\end{entry}

\begin{entry}{那}{nuo2}{6}{⾢}
  \definition*{s.}{sobrenome Nuo}
\end{entry}

\begin{entry}{诺贝尔奖}{nuo4bei4'er3 jiang3}{10,4,5,9}{⾔、⾙、⼩、⼤}
  \definition*{s.}{Prêmio Nobel}
\end{entry}

\begin{entry}{诺奖}{nuo4jiang3}{10,9}{⾔、⼤}
  \definition*{s.}{Prêmio Nobel, abreviação de 诺贝尔奖}
  \seeref{诺贝尔奖}{nuo4bei4'er3 jiang3}
\end{entry}

\begin{entry}{女}{nv3}{3}{⼥}[HSK 1][Kangxi 38]
  \definition{adj.}{feminino}
\end{entry}

\begin{entry}{女儿}{nv3'er2}{3,2}{⼥、⼉}[HSK 1]
  \definition{s.}{filha}
  \seealsoref{儿子}{er2zi5}
\end{entry}

\begin{entry}{女孩}{nv3hai2}{3,9}{⼥、⼦}
  \definition{s.}{menina | garota}
\end{entry}

\begin{entry}{女孩儿}{nv3hai2r5}{3,9,2}{⼥、⼦、⼉}[HSK 1]
\end{entry}

\begin{entry}{女朋友}{nv3peng2you5}{3,8,4}{⼥、⽉、⼜}[HSK 1]
  \definition{s.}{namorada}
\end{entry}

\begin{entry}{女人}{nv3ren2}{3,2}{⼥、⼈}[HSK 1]
  \definition[个,位]{s.}{mulher}
\end{entry}

\begin{entry}{女生}{nv3sheng1}{3,5}{⼥、⽣}[HSK 1]
  \definition[个]{s.}{aluna | estudante so sexo feminino}
\end{entry}

\begin{entry}{女士}{nv3shi4}{3,3}{⼥、⼠}[HSK 4]
  \definition{pron.}{Sra.; Senhorita; Senhora; título honorífico para mulheres (agora usado em contextos diplomáticos)}
  \definition[位,个]{s.}{senhora; madame}
\end{entry}

\begin{entry}{女王}{nv3wang2}{3,4}{⼥、⽟}
  \definition{s.}{rainha}
\end{entry}

\begin{entry}{女婿}{nv3xu5}{3,12}{⼥、⼥}
  \definition{s.}{marido da filha}
\end{entry}

\begin{entry}{女子}{nv3 zi3}{3,3}{⼥、⼦}[HSK 3]
  \definition[位]{s.}{mulher; feminino}
\end{entry}

%%%%% EOF %%%%%


%%%
%%% O
%%%

\section*{O}\addcontentsline{toc}{section}{O}

\begin{entry}{喔}{o1}{12}[Radical 口]
  \definition{interj.}{Oh!, Entendi!, usado para indicar realização, compreensão}
\end{entry}

\begin{entry}{哦}{o2}{10}[Radical ⼝]
  \definition{interj.}{Oh! (indicando dúvida ou surpresa)}
  \seeref{哦}{e2}
  \seeref{哦}{o4}
  \seeref{哦}{o5}
\end{entry}

\begin{entry}{哦}{o4}{10}[Radical ⼝]
  \definition{interj.}{Oh! (indicando que acabou de aprender algo)}
  \seeref{哦}{e2}
  \seeref{哦}{o2}
  \seeref{哦}{o5}
\end{entry}

\begin{entry}{哦}{o5}{10}[Radical ⼝]
  \definition{part.}{final da frase que transmite informalidade, calor, simpatia ou intimidade; também pode indicar que alguém está declarando um fato de que a outra pessoa não está ciente}
  \seeref{哦}{e2}
  \seeref{哦}{o2}
  \seeref{哦}{o4}
\end{entry}

\begin{entry}{区}{ou1}{4}[Radical 匸]
  \definition*{s.}{sobrenome Ou}
  \seeref{区}{qu1}
\end{entry}

\begin{entry}{欧}{ou1}{8}[Radical 欠]
  \definition*{s.}{Europa, abreviação de~欧洲 | sobrenome Ou}
  \seeref{欧洲}{ou1zhou1}
\end{entry}

\begin{entry}{欧盟}{ou1meng2}{8,13}
  \definition*{s.}{Uniáo Europeia}
\end{entry}

\begin{entry}{欧洲}{ou1zhou1}{8,9}
  \definition*{s.}{Europa}
\end{entry}

\begin{entry}{欧洲共同体}{ou1zhou1 gong4tong2ti3}{8,9,6,6,7}
  \definition*{s.}{Comunidade Europeia}
\end{entry}

\begin{entry}{欧洲人}{ou1zhou1ren2}{8,9,2}
  \definition{s.}{europeu | pessoa ou povo da Europa}
\end{entry}

\begin{entry}{偶然}{ou3ran2}{11,12}
  \definition{adv.}{por acaso | fortuitamente}
\end{entry}

%%%%% EOF %%%%%


%%%
%%% P
%%%

\section*{P}\addcontentsline{toc}{section}{P}

\begin{entry}{扒犁}{pa2li2}{5,11}{⼿、⽜}
  \definition{s.}{trenó}
  \seeref{爬犁}{pa2li2}
\end{entry}

\begin{entry}{爬}{pa2}{8}{⽖}[HSK 2]
  \definition{v.}{escalar | subir | trepar | rastejar}
\end{entry}

\begin{entry}{爬杆}{pa2gan1}{8,7}{⽖、⽊}
  \definition{s.}{escalada em poste}
  \definition{v.}{escalar um poste}
\end{entry}

\begin{entry}{爬竿}{pa2gan1}{8,9}{⽖、⽵}
  \definition{s.}{poste de escalada | escalada em poste (como ginástica ou ato de circo)}
\end{entry}

\begin{entry}{爬犁}{pa2li2}{8,11}{⽖、⽜}
  \definition{s.}{trenó}
  \seeref{扒犁}{pa2li2}
\end{entry}

\begin{entry}{爬墙}{pa2qiang2}{8,14}{⽖、⼟}
  \definition{v.}{escalar uma parede}
\end{entry}

\begin{entry}{爬山}{pa2shan1}{8,3}{⽖、⼭}[HSK 2]
  \definition{s.}{alpinista | montanhismo}
  \definition{v.}{escalar uma montanha}
\end{entry}

\begin{entry}{爬上}{pa2shang4}{8,3}{⽖、⼀}
  \definition{v.}{escalar}
\end{entry}

\begin{entry}{爬升}{pa2sheng1}{8,4}{⽖、⼗}
  \definition{v.}{ascender | ganhar promoção | subir (números de vendas, etc.) | aumentar}
\end{entry}

\begin{entry}{爬梳}{pa2shu1}{8,11}{⽖、⽊}
  \definition{v.}{vasculhar (documentos históricos, etc.) | desvendar}
\end{entry}

\begin{entry}{爬行}{pa2xing2}{8,6}{⽖、⾏}
  \definition{v.}{rastejar | arrastar | engatinhar}
\end{entry}

\begin{entry}{怕}{pa4}{8}{⼼}[HSK 2]
  \definition*{s.}{sobrenome Pa}
  \definition{adv.}{por medo; talvez; suponho}
  \definition{v.}{recear; temer; ter medo de | ser incapaz de suportar (ficar de pé, suportar) | ter medo de; ter medo de}
\end{entry}

\begin{entry}{拍}{pai1}{8}{⼿}[HSK 3]
  \definition[只,把]{s.}{bastão; raquete | batida; tempo}
  \definition{v.}{bater palmas; bater; dar um tapa | chicotear; açoitar; bater | enviar (um telegrama, etc.) | tirar (uma foto); fotografar | bajular; lisonjear; adular}
\end{entry}

\begin{entry}{拍马}{pai1ma3}{8,3}{⼿、⾺}
  \definition{v.}{instigar um cavalo dando tapinhas em seu traseiro | lisonjear | bajular}
  \seealsoref{拍马屁}{pai1ma3pi4}
\end{entry}

\begin{entry}{拍马屁}{pai1ma3pi4}{8,3,7}{⼿、⾺、⼫}
  \definition{s.}{puxa-saco | bajulador}
  \definition{v.}{puxar o saco | bajular}
  \seealsoref{拍马}{pai1ma3}
\end{entry}

\begin{entry}{拍照}{pai1 zhao4}{8,13}{⼿、⽕}[HSK 4]
  \definition{v.+compl.}{fotografar; tirar uma foto}
\end{entry}

\begin{entry}{排}{pai2}{11}{⼿}[HSK 2,3]
  \definition{clas.}{para linhas}
  \definition{s.}{linha | pelotão | jangada; balsa | torta}
  \definition{v.}{organizar; colocar em ordem | ensaiar | excluir; ejetar; descarregar | empurrar}
\end{entry}

\begin{entry}{排队}{pai2dui4}{11,4}{⼿、⾩}[HSK 2]
  \definition{v.+compl.}{formar uma fila | alinhar | listar | classificar}
\end{entry}

\begin{entry}{排列}{pai2lie4}{11,6}{⼿、⼑}[HSK 4]
  \definition{v.}{classificar; colocar; variar; organizar; pôr em ordem}
\end{entry}

\begin{entry}{排名}{pai2 ming2}{11,6}{⼿、⼝}[HSK 3]
  \definition{s.}{classificação; resultado}
\end{entry}

\begin{entry}{排球}{pai2 qiu2}{11,11}{⼿、⽟}[HSK 2]
  \definition[个]{s.}{voleibol}
\end{entry}

\begin{entry}{排水}{pai2shui3}{11,4}{⼿、⽔}
  \definition{v.}{drenar}
\end{entry}

\begin{entry}{牌}{pai2}{12}{⽚}[HSK 4]
  \definition[块]{s.}{placa; tabuleta; quadro; placar | marca; marca registrada; marca comercial | cartas, dominó, etc. | a tonalidade de uma música}
\end{entry}

\begin{entry}{牌子}{pai2 zi5}{12,3}{⽚、⼦}[HSK 3]
  \definition[个,种,块]{s.}{sinal; placa | marca; marca registrada}
\end{entry}

\begin{entry}{派}{pai4}{9}{⽔}[HSK 3]
  \definition{adj.}{elegante; bonito}
  \definition{clas.}{para grupos, escolas de pensamento ou arte, etc. | para um discursos, atmosferas, cenas, etc.}
  \definition{s.}{panelinha; grupo exclusivo; facção | torta | estilo | afluente; braço de rio}
  \definition{v.}{enviar; despachar | alocar; repartir; distribuir}
\end{entry}

\begin{entry}{攀爬}{pan1pa2}{19,8}{⼿、⽖}
  \definition{v.}{escalar}
\end{entry}

\begin{entry}{攀岩}{pan1yan2}{19,8}{⼿、⼭}
  \definition{s.}{alpinista}
  \definition{v.}{escalar uma montanha}
\end{entry}

\begin{entry}{爿}{pan2}{4}{⽙}[Kangxi 90]
  \definition{clas.}{para faixas de terra ou bambu, lojas, fábricas etc.}
\end{entry}

\begin{entry}{胖}{pan2}{9}{⾁}
  \definition{adj.}{saudável}
  \seeref{胖}{pang4}
\end{entry}

\begin{entry}{般}{pan2}{10}{⾈}
  \definition{s.}{utilizado em 般乐 \dpy{pan2le4}}
  \seeref{般乐}{pan2le4}
\end{entry}

\begin{entry}{般乐}{pan2le4}{10,5}{⾈、⼃}
  \definition{v.}{jogar | divertir-se}
\end{entry}

\begin{entry}{盘}{pan2}{11}{⽫}[HSK 4]
  \definition*{s.}{sobrenome Pan}
  \definition{clas.}{para pratos, pedras de moer, etc. | para jogos de xadrez e de bola | para as coisas que estão entrelaçadas, emaranhadas}
  \definition{s.}{bandeja; tabuleiro | recipiente plano e raso, como uma bandeja, prato, travessa etc.  | preço atual; cotação de mercado; refere-se ao preço básico pelo qual as commodities são negociadas}
  \definition{v.}{enrolar; torcer; enrolar (para cima); entrelaçar; cercar | construir (assentando tijolos, pedras, etc.) | checar; examinar; interrogar; verificar um por um ou repetidamente (quantidade, situação, etc.) | transferir a propriedade de; passar para outra pessoa | carregar; transportar}
\end{entry}

\begin{entry}{盘子}{pan2zi5}{11,3}{⽫、⼦}[HSK 4]
  \definition[个,叠,套,只]{s.}{prato; utensílio de fundo raso para guardar objetos, maior do que um pires, geralmente de formato redondo | situação de mercado; taxa de mercado; transação comercial}
\end{entry}

\begin{entry}{槃}{pan2}{14}{⽊}
  \variantof{盘}
\end{entry}

\begin{entry}{判断}{pan4duan4}{7,11}{⼑、⽄}[HSK 3]
  \definition[个]{s.}{julgamento}
  \definition{v.}{julgar; decidir}
\end{entry}

\begin{entry}{旁边}{pang2bian1}{10,5}{⽅、⾡}[HSK 1]
  \definition{adv.}{junto a | próximo de | ao lado}
\end{entry}

\begin{entry}{胖}{pang4}{9}{⾁}[HSK 3]
  \definition{adj.}{gordo; robusto; rechonchudo}
  \seeref{胖}{pan2}
\end{entry}

\begin{entry}{胖子}{pang4 zi5}{9,3}{⾁、⼦}[HSK 4]
  \definition{s.}{obeso; gordo; pessoa gorda}
\end{entry}

\begin{entry}{泡}{pao1}{8}{⽔}
  \definition{adj.}{estufado | inchado | esponjoso}
  \definition{clas.}{para urina ou fezes}
  \definition{s.}{pequeno lago (especialmente em nomes de lugares)}
  \seeref{泡}{pao4}
\end{entry}

\begin{entry}{跑}{pao2}{12}{⾜}
  \definition{v.}{(de um animal) dar patadas (no chão)}
  \seeref{跑}{pao3}
\end{entry}

\begin{entry}{跑}{pao3}{12}{⾜}[HSK 1]
  \definition{v.}{vazar ou evaporar (sobre um gás ou líquido) | escapar | correr | correr (em tarefas, etc.) | fugir}
  \seeref{跑}{pao2}
\end{entry}

\begin{entry}{跑步}{pao3bu4}{12,7}{⾜、⽌}[HSK 3]
  \definition{s.}{corrida}
  \definition{v.+compl.}{correr; trotar}
\end{entry}

\begin{entry}{跑调}{pao3diao4}{12,10}{⾜、⾔}
  \definition{v.}{(coloquial) estar fora do tom ou desafinado (enquanto canta)}
\end{entry}

\begin{entry}{跑掉}{pao3diao4}{12,11}{⾜、⼿}
  \definition{v.}{fugir}
\end{entry}

\begin{entry}{跑肚}{pao3du4}{12,7}{⾜、⾁}
  \definition{v.}{(coloquial) ter diarréia}
\end{entry}

\begin{entry}{跑酷}{pao3ku4}{12,14}{⾜、⾣}
  \definition*{s.}{(empréstimo linguístico) \emph{Parkour}}
\end{entry}

\begin{entry}{跑马}{pao3ma3}{12,3}{⾜、⾺}
  \definition{s.}{corrida de cavalos}
  \definition{v.}{andar a cavalo em ritmo acelerado}
\end{entry}

\begin{entry}{跑题}{pao3ti2}{12,15}{⾜、⾴}
  \definition{v.}{divagar | fugir do assunto | tergiversar}
\end{entry}

\begin{entry}{跑腿}{pao3tui3}{12,13}{⾜、⾁}
  \definition{v.}{realizar tarefas}
\end{entry}

\begin{entry}{泡}{pao4}{8}{⽔}
  \definition{clas.}{para ocorrências de uma ação | para número de infusões}
  \definition{s.}{bolha | espuma}
  \definition{v.}{encharcar | infundir | pegar (uma garota) | sair com (um parceiro sexual)}
  \seeref{泡}{pao1}
\end{entry}

\begin{entry}{胚}{pei1}{9}{⾁}
  \definition{s.}{embrião}
\end{entry}

\begin{entry}{陪}{pei2}{10}{⾩}
  \definition{v.}{acompanhar | ajudar | fazer companhia a alguém}
\end{entry}

\begin{entry}{培训}{pei2xun4}{11,5}{⼟、⾔}[HSK 4]
  \definition{v.}{treinar (trabalhadores técnicos, quadros profissionais, etc.)}
\end{entry}

\begin{entry}{培训班}{pei2 xun4 ban1}{11,5,10}{⼟、⾔、⽟}[HSK 4]
  \definition{s.}{aula de treinamento; curso de treinamento}
\end{entry}

\begin{entry}{培养}{pei2yang3}{11,9}{⼟、⼋}[HSK 4]
  \definition{v.}{cultivar (plantas, microorganismos) | promover; treinar ou desenvolver; educar e treinar para um determinado propósito durante um longo período de tempo; fazer crescer | progredir gradualmente; desenvolver ou cultivar gradualmente (hábito, qualidade, caráter, emoção, estilo, interesse, habilidade, etc.)}
\end{entry}

\begin{entry}{培育}{pei2yu4}{11,8}{⼟、⾁}[HSK 4]
  \definition{v.}{criar; fomentar; educar; procriar; nutrir; cultivar}
\end{entry}

\begin{entry}{赔钱}{pei2qian2}{12,10}{⾙、⾦}
  \definition{v.+compl.}{perder dinheiro | pagar pelos danos}
\end{entry}

\begin{entry}{佩服}{pei4fu2}{8,8}{⼈、⽉}
  \definition{v.}{admirar}
\end{entry}

\begin{entry}{配}{pei4}{10}{⾣}[HSK 3]
  \definition{s.}{esposa}
  \definition{v.}{unir-se em matrimônio | acasalar (animais) | compor; combinar; mesclar; amalgamar; misturar |distribuir de acordo com o plano; repartir | encontrar algo para encaixar ou substituir outra coisa | corresponder; combinar; equiparar | merecer; ser digno de; ser qualificado}
\end{entry}

\begin{entry}{配合}{pei4he2}{10,6}{⾣、⼝}[HSK 3]
  \definition{s.}{coordenação}
  \definition{v.}{cooperar; coordenar}
\end{entry}

\begin{entry}{盆}{pen2}{9}{⽫}
  \definition[个]{s.}{panela | bacia | vaso de flores}
\end{entry}

\begin{entry}{盆友}{pen2you3}{9,4}{⽫、⼜}
  \definition{s.}{(gíria na \emph{Internet}) amigo (trocadilho com 朋友)}
  \seeref{朋友}{peng2you5}
\end{entry}

\begin{entry}{朋友}{peng2you5}{8,4}{⽉、⼜}[HSK 1]
  \definition[个,位]{s.}{amigo}
\end{entry}

\begin{entry}{膨胀}{peng2zhang4}{16,8}{⾁、⾁}
  \definition{v.}{expandir | inflar | inchar}
\end{entry}

\begin{entry}{碰}{peng4}{13}{⽯}[HSK 2]
  \definition{v.}{tocar | bater | encontrar | correr para | tentar a sorte | arriscar | encontrar para discutir}
\end{entry}

\begin{entry}{碰到}{peng4 dao4}{13,8}{⽯、⼑}[HSK 2]
  \definition{v.}{encontrar (com) | esbarrar em | deparar-se com}
\end{entry}

\begin{entry}{碰见}{peng4 jian4}{13,4}{⽯、⾒}[HSK 2]
  \definition{v.}{reunir-se | encontrar}
\end{entry}

\begin{entry}{碰头}{peng4tou2}{13,5}{⽯、⼤}
  \definition{s.}{colisão | conflito}
  \definition{v.}{colidir}
  \definition{v.+compl.}{conhecer e discutir | juntar ideias | ver-se}
\end{entry}

\begin{entry}{碰运气}{peng4yun4qi5}{13,7,4}{⽯、⾡、⽓}
  \definition{v.}{deixar algo ao acaso | tentar a sorte}
\end{entry}

\begin{entry}{批}{pi1}{7}{⼿}[HSK 4]
  \definition{adj.}{(compra ou venda) atacado; a granel; em grandes quantidades}
  \definition{clas.}{para mercadorias a granel, grande número de pessoas}
  \definition{s.}{fibras de algodão, linho, etc., prontas para serem estiradas e torcidas | anotação; comentário}
  \definition{v.}{escrever comentários ou críticas sobre documentos subordinados, textos de outras pessoas, tarefas etc. | refutar; criticar | dar um tapa}
\end{entry}

\begin{entry}{批评}{pi1ping2}{7,7}{⼿、⾔}[HSK 3]
  \definition{s.}{crítica}
  \definition{v.}{criticar; comentar sobre}
\end{entry}

\begin{entry}{批准}{pi1zhun3}{7,10}{⼿、⼎}[HSK 3]
  \definition{v.}{aprovar}
\end{entry}

\begin{entry}{皮}{pi2}{5}{⽪}[HSK 3][Kangxi 107]
  \definition*{s.}{sobrenome Pi}
  \definition{adj.}{macios e encharcados; não mais crocantes | danadinho; travesso | endurecido; não se importa mais}
  \definition{pref.}{pico- (um trilhonésimo)}
  \definition[张]{s.}{pele | couro cru; couro | pelagem | capa; envoltório | superfície | uma peça larga e plana (de algum material fino) | borracha}
\end{entry}

\begin{entry}{皮包}{pi2 bao1}{5,5}{⽪、⼓}[HSK 3]
  \definition[个,只,款]{s.}{bolsa; pasta; portfólio}
\end{entry}

\begin{entry}{皮肤}{pi2fu1}{5,8}{⽪、⾁}
  \definition[层,块]{s.}{pele}
\end{entry}

\begin{entry}{皮卡}{pi2ka3}{5,5}{⽪、⼘}
  \definition{s.}{(empréstimo linguístico) \emph{pick-up} | caminhonete}
\end{entry}

\begin{entry}{皮卡丘}{pi2ka3qiu1}{5,5,5}{⽪、⼘、⼀}
  \definition*{s.}{\emph{Pikachu} (Pokémon, 口袋妖怪)}
  \seealsoref{口袋妖怪}{kou3dai4 yao1guai4}
\end{entry}

\begin{entry}{皮下}{pi2xia4}{5,3}{⽪、⼀}
  \definition{adj.}{(injeção) subcutâneo | sob a pele}
\end{entry}

\begin{entry}{皮鞋}{pi2xie2}{5,15}{⽪、⾰}
  \definition[双,只,款]{s.}{sapatos de couro}
\end{entry}

\begin{entry}{啤酒}{pi2jiu3}{11,10}{⼝、⾣}[HSK 3]
  \definition[杯,瓶,罐,桶,缸]{s.}{(empréstimo linguístico) cerveja}
\end{entry}

\begin{entry}{啤酒馆}{pi2jiu3guan3}{11,10,11}{⼝、⾣、⾷}
  \definition{s.}{cervejaria}
\end{entry}

\begin{entry}{脾气}{pi2qi5}{12,4}{⾁、⽓}
  \definition{s.}{temperamento | humor | disposição | caráter}
\end{entry}

\begin{entry}{屁股}{pi4gu5}{7,8}{⼫、⾁}
  \definition{s.}{nádega | quadris}
\end{entry}

\begin{entry}{屁话}{pi4hua4}{7,8}{⼫、⾔}
  \definition{s.}{absurdo | tolice | besteira}
\end{entry}

\begin{entry}{譬如}{pi4ru2}{20,6}{⾔、⼥}
  \definition{conj.}{por exemplo | como}
\end{entry}

\begin{entry}{偏偏}{pian1pian1}{11,11}{⼈、⼈}
  \definition{adv.}{voluntariamente | insistentemente | persistentemente | ao contrário da expectativa | infelizmente (indicando que alguma coisa aconteceu ao contrário do que se esperava) | teimosamente (indicando que algo é o oposto ao que seria normal ou razoável) | precisamente (indicando que alguém ou um grupo é escolhido)}
\end{entry}

\begin{entry}{篇}{pian1}{15}{⽵}[HSK 2]
  \definition*{s.}{sobrenome Pian}
  \definition{clas.}{para pedaços, folhas}
  \definition{s.}{um pedaço de escrita
folha (de papel, etc.)}
\end{entry}

\begin{entry}{便宜}{pian2yi5}{9,8}{⼈、⼧}[HSK 2]
  \definition{adj.}{barato}
  \definition{v.}{deixar alguém levemente de lado}
\end{entry}

\begin{entry}{片}{pian4}{4}{⽚}[HSK 2][Kangxi 91]
  \definition{adj.}{parcial | incompleto | que só tem um lado}
  \definition{clas.}{para CDs, filmes, DVDs, etc. | para fatias, comprimidos, extensão de terra, área de água | usado com numeral~一:~para  cenário, cena, sentimento, atmosfera, som etc.}
  \definition{s.}{uma fatia | floco | filme | pedaço fino}
  \definition{v.}{fatiar | esculpir fino}
\end{entry}

\begin{entry}{片面}{pian4mian4}{4,9}{⽚、⾯}[HSK 4]
  \definition{adj.}{unilateral (em oposição a ``全面'')}
  \seealsoref{全面}{quan2mian4}
\end{entry}

\begin{entry}{漂}{piao1}{14}{⽔}
  \definition{v.}{flutuar | estar a deriva}
  \seeref{漂}{piao3}
  \seeref{漂}{piao4}
\end{entry}

\begin{entry}{漂流}{piao1liu2}{14,10}{⽔、⽔}
  \definition{s.}{\emph{rafting}}
  \definition{v.}{ser levado pela correnteza | flutuar ao longo ou sobre}
\end{entry}

\begin{entry}{飘}{piao1}{15}{⾵}
  \definition{adj.}{complacente | frívolo | fraco | instável | bambo | cambaleante}
  \definition{v.}{flutuar (no ar) | esvoaçar | tremular}
\end{entry}

\begin{entry}{漂}{piao3}{14}{⽔}
  \definition{v.}{alvejar | branquear}
  \seeref{漂}{piao1}
  \seeref{漂}{piao4}
\end{entry}

\begin{entry}{票}{piao4}{11}{⽰}[HSK 1]
  \definition{clas.}{para grupos, lotes, transações comerciais}
  \definition[张]{s.}{performance amadora de ópera chinesa | cédula eleitoral | nota | bilhete | pessoa detida por resgate | refém}
\end{entry}

\begin{entry}{票价}{piao4 jia4}{11,6}{⽰、⼈}[HSK 3]
  \definition[个]{s.}{o preço de um bilhete; taxa de admissão; taxa de entrada}
\end{entry}

\begin{entry}{漂}{piao4}{14}{⽔}
  \definition{adj.}{usado em 漂亮}
  \seeref{漂}{piao1}
  \seeref{漂}{piao3}
  \seeref{漂亮}{piao4liang5}
\end{entry}

\begin{entry}{漂亮}{piao4liang5}{14,9}{⽔、⼇}[HSK 2]
  \definition{adj.}{bonita, linda | bonito, lindo (para objetos inanimados)}
\end{entry}

\begin{entry}{拼}{pin1}{9}{⼿}
  \definition{v.}{soletrar | juntar | unir}
\end{entry}

\begin{entry}{拼命}{pin1ming4}{9,8}{⼿、⼝}
  \definition{adv.}{com toda a força | desesperadamente}
  \definition{v.+compl.}{arriscar a vida de alguém | desafiar a morte | colocar-se em uma luta desesperada | fazer algo desesperadamente | exercer a maior força}
\end{entry}

\begin{entry}{拼音}{pin1yin1}{9,9}{⼿、⾳}
  \definition{s.}{escrita fonética | pinyin (romanização chinesa)}
\end{entry}

\begin{entry}{贫民窟}{pin2min2ku1}{8,5,13}{⾙、⽒、⽳}
  \definition{s.}{favela}
\end{entry}

\begin{entry}{频道}{pin2dao4}{13,12}{⾴、⾡}
  \definition{s.}{frequência | (televisão) canal}
\end{entry}

\begin{entry}{品德}{pin3de2}{9,15}{⼝、⼻}
  \definition{s.}{caráter moral | moralidade}
\end{entry}

\begin{entry}{品质}{pin3zhi4}{9,8}{⼝、⾙}[HSK 4]
  \definition[个,种]{s.}{qualidade; caráter; natureza do pensamento, da compreensão, do caráter, etc., conforme expresso no comportamento, no estilo, etc. | qualidade (de produtos, mercadorias, etc.)}
\end{entry}

\begin{entry}{乒乓球}{ping1pang1qiu2}{6,6,11}{⼃、⼃、⽟}
  \definition[个]{s.}{tênis de mesa |ping-pong}
\end{entry}

\begin{entry}{平}{ping2}{5}{⼲}[HSK 2]
  \definition*{s.}{sobrenome Ping}
  \definition{adj.}{calmo | pacífico}
  \definition{s.}{plano | nível}
  \definition{v.}{fazer a mesma pontuação | marcar uma pontuação}
\end{entry}

\begin{entry}{平安}{ping2'an1}{5,6}{⼲、⼧}[HSK 2]
  \definition{s.}{seguro | bem | sem contratempos | são e salvo}
\end{entry}

\begin{entry}{平常}{ping2chang2}{5,11}{⼲、⼱}[HSK 2]
  \definition{adj.}{comum | ordinário | usual}
  \definition{adv.}{usualmente | geralmente | ordinariamente | como regra}
\end{entry}

\begin{entry}{平等}{ping2deng3}{5,12}{⼲、⽵}[HSK 2]
  \definition{adj.}{igual | igualdade}
\end{entry}

\begin{entry}{平地}{ping2di4}{5,6}{⼲、⼟}
  \definition{v.}{nivelar a terra | aplanar}
\end{entry}

\begin{entry}{平方}{ping2fang1}{5,4}{⼲、⽅}[HSK 4]
  \definition{s.}{quadrado}
\end{entry}

\begin{entry}{平方米}{ping2fang1 mi3}{5,4,6}{⼲、⽅、⽶}
  \definition{clas.}{unidade de medida de área, 1 metro quadrado equivale a 10.000 centímetros quadrados}
\end{entry}

\begin{entry}{平静}{ping2jing4}{5,14}{⼲、⾭}[HSK 4]
  \definition{adj.}{(humor, ambiente, etc.) calmo; quieto; pacífico; tranquilo}
\end{entry}

\begin{entry}{平均}{ping2jun1}{5,7}{⼲、⼟}[HSK 4]
  \definition{adj.}{igual; médio}
  \definition{s.}{média}
  \definition{v.}{calcular a média de um conjunto de números}
\end{entry}

\begin{entry}{平时}{ping2shi2}{5,7}{⼲、⽇}[HSK 2]
  \definition{adv.}{normalmente | em tempos normais | em tempos de paz}
\end{entry}

\begin{entry}{平台}{ping2tai2}{5,5}{⼲、⼝}
  \definition{s.}{plataforma | terraço | edifício de telhado plano}
\end{entry}

\begin{entry}{平稳}{ping2 wen3}{5,14}{⼲、⽲}[HSK 4]
  \definition{adj.}{firme; estável; suave e constante; sem oscilações ou flutuações}
\end{entry}

\begin{entry}{评价}{ping2jia4}{7,6}{⾔、⼈}[HSK 3]
  \definition[个,项,条,份]{s.}{avaliação; apreciação}
  \definition{v.}{estimar; avaliar}
\end{entry}

\begin{entry}{评论}{ping2lun4}{7,6}{⾔、⾔}
  \definition[篇]{s.}{comentário}
  \definition{v.}{comentar | discutir}
\end{entry}

\begin{entry}{苹果}{ping2guo3}{8,8}{⾋、⽊}[HSK 3]
  \definition[个,颗]{s.}{maçã}
\end{entry}

\begin{entry}{瓶}{ping2}{10}{⽡}[HSK 2]
  \definition{clas.}{para vinho ou líquidos}
  \definition[个]{s.}{garrafa | jarro| vaso}
\end{entry}

\begin{entry}{瓶盖}{ping2gai4}{10,11}{⽡、⽫}
  \definition{s.}{tampa de garrafa}
\end{entry}

\begin{entry}{瓶装}{ping2zhuang1}{10,12}{⽡、⾐}
  \definition{adj.}{engarrafado}
\end{entry}

\begin{entry}{瓶子}{ping2zi5}{10,3}{⽡、⼦}[HSK 2]
  \definition[个]{s.}{garrafa}
\end{entry}

\begin{entry}{甁}{ping2}{12}{⽡}
  \variantof{瓶}
\end{entry}

\begin{entry}{颇}{po1}{11}{⽪}
  \definition*{s.}{sobrenome Po}
  \definition{adv.}{muito, bastante (linguagem escrita)}
\end{entry}

\begin{entry}{迫切}{po4qie4}{8,4}{⾡、⼑}[HSK 4]
  \definition{adj.}{urgente; premente; muito ansiosamente, a ponto de ser difícil esperar}
\end{entry}

\begin{entry}{破}{po4}{10}{⽯}[HSK 3]
  \definition{adj.}{quebrado; danificado; rasgado; desgastado | pobre; ruim; insignificante; péssimo; miserável}
  \definition{v.}{estar quebrado; estar danificado | quebrar; avariar; danificar | quebrar; dividir; cortar; cinzelar | trocar (dinheiro) | romper; quebrar (avanço) | livrar-se de; destruir; romper com
derrotar; capturar (uma cidade, etc.) | despender; gastar (dinheiro) | expor a verdade de; desnudar}
\end{entry}

\begin{entry}{破产}{po4chan3}{10,6}{⽯、⼇}[HSK 4]
  \definition{v.+compl.}{falir; ir à falência; tornar-se insolvente; entrar em liquidação; perder todo o patrimônio | falhar; fracassar; não dar em nada; figura de linguagem (geralmente com uma conotação depreciativa)}
\end{entry}

\begin{entry}{破坏}{po4huai4}{10,7}{⽯、⼟}[HSK 3]
  \definition{s.}{destruição | dano}
  \definition{v.}{demolir; naufragar; soçobrar; destruir; obliterar | quebrar; violar (um acordo, regulamento, etc.) | prejudicar; perturbar; sabotar; causar grande dano | reverter; mudar (um sistema social, costume, etc.) completamente ou violentamente | destruir; decompor}
\end{entry}

\begin{entry}{破坏性}{po4huai4xing4}{10,7,8}{⽯、⼟、⼼}
  \definition{adj.}{destrutivo}
  \definition{s.}{poder destrutivo}
\end{entry}

\begin{entry}{扑克}{pu1ke4}{5,7}{⼿、⼗}
  \definition{s.}{(empréstimo linguístico) (jogo) \emph{poker}  | baralho}
\end{entry}

\begin{entry}{铺}{pu1}{12}{⾦}
  \definition{v.}{espalhar | exibir | montar}
  \seeref{铺}{pu4}
\end{entry}

\begin{entry}{铺垫}{pu1dian4}{12,9}{⾦、⼟}
  \definition{s.}{cobre leito | colcha | roupa de cama}
  \definition{v.}{pavimentar}
\end{entry}

\begin{entry}{葡}{pu2}{12}{⾋}
  \definition*{s.}{Portugal, abreviação de 葡萄牙}
  \seeref{葡萄牙}{pu2tao2ya2}
\end{entry}

\begin{entry}{葡汉词典}{pu2-han4 ci2dian3}{12,5,7,8}{⾋、⽔、⾔、⼋}
  \definition{s.}{dicionário português-chinês}
  \seealsoref{汉葡词典}{han4-pu2 ci2dian3}
\end{entry}

\begin{entry}{葡萄牙}{pu2tao2ya2}{12,11,4}{⾋、⾋、⽛}
  \definition{s.}{Portugal}
  \seeref{葡}{pu2}
\end{entry}

\begin{entry}{葡萄牙文}{pu2tao2ya2wen2}{12,11,4,4}{⾋、⾋、⽛、⽂}
  \definition{s.}{português, língua portuguesa}
  \seeref{葡文}{pu2wen2}
\end{entry}

\begin{entry}{葡萄牙语}{pu2tao2ya2yu3}{12,11,4,9}{⾋、⾋、⽛、⾔}
  \definition{s.}{português, língua portuguesa}
  \seeref{葡语}{pu2yu3}
\end{entry}

\begin{entry}{葡萄}{pu2tao5}{12,11}{⾋、⾋}
  \definition{s.}{uva}
\end{entry}

\begin{entry}{葡文}{pu2wen2}{12,4}{⾋、⽂}
  \definition{s.}{português, língua portuguesa}
  \seeref{葡萄牙文}{pu2tao2ya2wen2}
\end{entry}

\begin{entry}{葡语}{pu2yu3}{12,9}{⾋、⾔}
  \definition{s.}{português, língua portuguesa}
  \seeref{葡萄牙语}{pu2tao2ya2yu3}
\end{entry}

\begin{entry}{普遍}{pu3bian4}{12,12}{⽇、⾡}[HSK 3]
  \definition{adj.}{geral; comum; universal; difundido}
\end{entry}

\begin{entry}{普及}{pu3ji2}{12,3}{⽇、⼃}[HSK 3]
  \definition{adj.}{popular; universal; onipresente}
  \definition{v.}{popularizar; disseminar; espalhar entre o povo}
\end{entry}

\begin{entry}{普通}{pu3 tong1}{12,10}{⽇、⾡}[HSK 2]
  \definition{adj.}{ordinário | comum | geral | médio}
\end{entry}

\begin{entry}{普通话}{pu3tong1hua4}{12,10,8}{⽇、⾡、⾔}[HSK 2]
  \definition*{s.}{Mandarim (literalmente ``linguagem comum'') | Putonghua (fala comum da língua chinesa) | discurso comum}
\end{entry}

\begin{entry}{铺}{pu4}{12}{⾦}
  \definition{s.}{cama de tábua | lugar para dormir | loja | depósito}
  \seeref{铺}{pu1}
\end{entry}

\begin{entry}{瀑布}{pu4bu4}{18,5}{⽔、⼱}
  \definition{s.}{queda de água | cachoeira | cascata | catarata}
\end{entry}

%%%%% EOF %%%%%


%%%
%%% Q
%%%

\section*{Q}\addcontentsline{toc}{section}{Q}

\begin{entry}{七}{qi1}{2}{⼀}[HSK 1]
  \definition{num.}{sete; 7}
\end{entry}

\begin{entry}{七夕}{qi1xi1}{2,3}{⼀、⼣}
  \definition*{s.}{Dia dos Namorados Chinês, quando o vaqueiro e a tecelã (牛郎织女) têm permissão para se reunirem anualmente | Festival das Meninas | Festival Duplo Sete, noite do sétimo mês lunar}
  \seeref{牛郎织女}{niu2lang2zhi1nv3}
\end{entry}

\begin{entry}{妻子}{qi1zi3}{8,3}{⼥、⼦}
  \definition{s.}{esposa e filhos; (chinês antigo) refere-se a esposas, filhos e filhas}
  \seeref{妻子}{qi1zi5}
\end{entry}

\begin{entry}{妻子}{qi1zi5}{8,3}{⼥、⼦}[HSK 4]
  \definition{s.}{esposa (não é usado como um termo carinhoso)}
  \seeref{妻子}{qi1zi3}
\end{entry}

\begin{entry}{期}{qi1}{12}{⽉}[HSK 3]
  \definition{clas.}{questão; número; termo}
  \definition{s.}{tempo designado (programado) | um período de tempo; fase; estágio}
  \definition{v.}{marcar uma consulta | esperar; supor; imaginar}
\end{entry}

\begin{entry}{期待}{qi1dai4}{12,9}{⽉、⼻}[HSK 4]
  \definition{v.}{aguardar; esperar; aguardar ansiosamente; ter em mente a realização de um determinado fim ou a ocorrência de uma determinada situação}
\end{entry}

\begin{entry}{期间}{qi1jian1}{12,7}{⽉、⾨}[HSK 4]
  \definition{s.}{prazo; tempo; período}
\end{entry}

\begin{entry}{期末}{qi1 mo4}{12,5}{⽉、⽊}[HSK 4]
  \definition{s.}{terminal; final do prazo; fim do período}
\end{entry}

\begin{entry}{期望}{qi1wang4}{12,11}{⽉、⽉}[HSK 5]
  \definition{s.}{esperança; expectativa}
  \definition{v.}{esperar; ter esperança}
\end{entry}

\begin{entry}{期限}{qi1xian4}{12,8}{⽉、⾩}[HSK 4]
  \definition{s.}{prazo; limite de tempo; tempo alocado; período de tempo limitado, também o limite final do limite de tempo}
\end{entry}

\begin{entry}{期中}{qi1 zhong1}{12,4}{⽉、⼁}[HSK 4]
  \definition{adj.}{provisório; interino; intermediário}
\end{entry}

\begin{entry}{齐}{qi2}{6}{⿑}[HSK 3][Kangxi 210]
  \definition*{s.}{sobrenome Qi | Qi, um estado da Dinastia Zhou | Dinastia Qi do Sul (479-502), uma das Dinastias do Sul | Dinastia Qi do Norte (550-577), uma das Dinastias do Norte}
  \definition{adj.}{arrumado; uniforme; regular | semelhante; similar | tudo pronto; todos presentes}
  \definition{adv.}{juntos; simultaneamente}
  \definition{prep.}{ao longo de; junto a; paralelo a}
  \definition{v.}{atingir a altura de; em um nível com; estar nivelado com; no mesmo plano com}
\end{entry}

\begin{entry}{齐全}{qi2quan2}{6,6}{⿑、⼊}[HSK 5]
  \definition{adj.}{completo; tudo pronto}
\end{entry}

\begin{entry}{其}{qi2}{8}{⼋}[HSK 5]
  \definition*{s.}{sobrenome Qi}
  \definition{adv.}{fazer uma suposição ou uma réplica | expressar comando, ordem}
  \definition{pron.}{dele (dela, deles, delas) | ele, ela, isso, eles; elas | isso; tal | isso (referindo-se a nenhuma pessoa ou coisa específica)}
  \definition{suf.}{sufixo de palavra, anexado ao advérbio}
\end{entry}

\begin{entry}{其次}{qi2ci4}{8,6}{⼋、⽋}[HSK 3]
  \definition{adj.}{secundário}
  \definition{conj.}{próximo; então; em segundo lugar}
\end{entry}

\begin{entry}{其实}{qi2shi2}{8,8}{⼋、⼧}[HSK 3]
  \definition{adv.}{na verdade; na realidade; de fato}
\end{entry}

\begin{entry}{其他}{qi2ta1}{8,5}{⼋、⼈}[HSK 2]
  \definition{pron.}{todos os outro(s) | o resto}
\end{entry}

\begin{entry}{其余}{qi2yu2}{8,7}{⼋、⼈}[HSK 4]
  \definition{pron.}{o restante; os outros}
\end{entry}

\begin{entry}{其中}{qi2zhong1}{8,4}{⼋、⼁}[HSK 2]
  \definition{pron.}{dentro | entre (o qual, eles, etc.) | em (o qual, isso, etc.)}
\end{entry}

\begin{entry}{奇怪}{qi2guai4}{8,8}{⼤、⼼}[HSK 3]
  \definition{adj.}{estranho; esquisito}
  \definition{v.}{ficar perplexo; maravilhar-se; sentir-se surpreso}
\end{entry}

\begin{entry}{奇迹}{qi2ji4}{8,9}{⼤、⾡}
  \definition{adj.}{milagroso}
  \definition{s.}{milagre}
\end{entry}

\begin{entry}{骑}{qi2}{11}{⾺}[HSK 2]
  \definition{clas.}{para cavalos de sela}
  \definition{v.}{andar (cavalo, bicicleta, etc.) | sentar-se montado | montar}
\end{entry}

\begin{entry}{骑车}{qi2 che1}{11,4}{⾺、⾞}[HSK 2]
  \definition{v.}{andar de bicicleta | pedalar}
\end{entry}

\begin{entry}{旗}{qi2}{14}{⽅}
  \definition[面]{s.}{bandeira}
\end{entry}

\begin{entry}{企业}{qi3ye4}{6,5}{⼈、⼀}[HSK 4]
  \definition[家,个]{s.}{empresa; estabelecimento; empreendimento; negócio; setores envolvidos em atividades econômicas como produção, transporte, comércio, etc., como fábricas, minas, ferrovias, empresas comerciais, etc.}
\end{entry}

\begin{entry}{岂}{qi3}{6}{⼭}
  \definition*{s.}{sobrenome Qi}
  \definition{adv.}{expressa uma pergunta retórica, equivalente a ``哪里'', ``怎么'' e ``难道''}
  \seealsoref{哪里}{na3 li3}
  \seealsoref{难道}{nan2dao4}
  \seealsoref{怎么}{zen3me5}
\end{entry}

\begin{entry}{岂有此理}{qi3you3ci3li3}{6,6,6,11}{⼭、⽉、⽌、⽟}
  \definition{interj.}{Que exorbitante! | Absurdo! | Como isso pode ser assim? | Ridículo!}
\end{entry}

\begin{entry}{启动}{qi3 dong4}{7,6}{⼝、⼒}[HSK 5]
  \definition{v.}{ligar (uma máquina); acionar; ligar máquinas, equipamentos elétricos, etc., para começar a trabalhar | entrar em vigor; começar a vigorar e a ser implementados planos, projetos, documentos jurídicos, etc.}
\end{entry}

\begin{entry}{启发}{qi3fa1}{7,5}{⼝、⼜}[HSK 5]
  \definition{s.}{iluminação; esclarecimento; fenômenos e princípios que levam as pessoas a refletir e a abrir suas mentes}
  \definition{v.}{despertar; inspirar; esclarecer; orientar, fazer com que compreendam}
\end{entry}

\begin{entry}{启事}{qi3shi4}{7,8}{⼝、⼅}[HSK 5]
  \definition{s.}{aviso; anúncio; texto publicado em jornais ou afixado em paredes com o objetivo de divulgar publicamente algo}
\end{entry}

\begin{entry}{起}{qi3}{10}{⾛}[HSK 1]
  \definition*{s.}{sobrenome Qi}
  \definition{clas.}{caso; instância | lote; grupo}
  \definition{v.}{levantar | levantar-se | extrair| remover | puxar | aparecer | crescer | construir | configurar | começar | iniciar}
\end{entry}

\begin{entry}{起床}{qi3 chuang2}{10,7}{⾛、⼴}[HSK 1]
  \definition{v.+compl.}{sair da cama | levantar-se}
\end{entry}

\begin{entry}{起到}{qi3 dao4}{10,8}{⾛、⼑}[HSK 5]
  \definition{v.}{ter (um efeito motivador, etc.); desempenhar (um papel estabilizador, etc.)}
\end{entry}

\begin{entry}{起飞}{qi3fei1}{10,3}{⾛、⾶}[HSK 2]
  \definition{v.}{decolar}
\end{entry}

\begin{entry}{起来}{qi3 lai2}{10,7}{⾛、⽊}[HSK 1]
  \definition{v.+compl.}{levantar-se}
\end{entry}

\begin{entry}{起码}{qi3ma3}{10,8}{⾛、⽯}[HSK 5]
  \definition{adj.}{mínimo; elementar; rudimentar}
  \definition{adv.}{mínimamente; pelo menos;}
\end{entry}

\begin{entry}{起跳}{qi3tiao4}{10,13}{⾛、⾜}
  \definition{v.}{(atletismo) decolar (no início de um salto) | (de preço, salário, etc.) começar (de um determinado nível)}
\end{entry}

\begin{entry}{气}{qi4}{4}{⽓}[HSK 2][Kangxi 84]
  \definition[口]{s.}{gás | ar | respiração | clima | cheiro | odor | espírito | moral | ares | maneira | estilo | insulto | intimidação | energia vital | energia da vida}
  \definition{v.}{ficar bravo | ficar enfurecido | irritar | enfurecer}
\end{entry}

\begin{entry}{气候}{qi4hou4}{4,10}{⽓、⼈}[HSK 3]
  \definition[种]{s.}{clima; tempo
tendência; situação
resultado; efeito; conquista}
\end{entry}

\begin{entry}{气球}{qi4qiu2}{4,11}{⽓、⽟}[HSK 4]
  \definition{s.}{balão; bolas feitas de borracha, plástico, etc., que podem ser aumentadas soprando ar nelas e podem ser usadas como brinquedos, decorações ou meios de transporte}
\end{entry}

\begin{entry}{气体}{qi4 ti3}{4,7}{⽓、⼈}[HSK 5]
  \definition[种]{s.}{gás; não têm forma nem volume definidos e podem fluir.; o ar, o oxigênio, o gás metano e outros são gases}
\end{entry}

\begin{entry}{气温}{qi4 wen1}{4,12}{⽓、⽔}[HSK 2]
  \definition[个]{s.}{temperatura do ar}
\end{entry}

\begin{entry}{气象}{qi4xiang4}{4,11}{⽓、⾗}[HSK 5]
  \definition[个]{s.}{fenômenos meteorológicos; condições e fenômenos atmosféricos, como vento, relâmpagos, trovões, geadas, neve, etc. | meteorologia | situação; atmosfera; cena; circunstância | maneira imponente}
\end{entry}

\begin{entry}{气质}{qi4zhi4}{4,8}{⽓、⾙}
  \definition{s.}{traços de personalidade, temperamento, disposição | aura, ar, sentimento, \emph{vibe} | refinamento, sofisticação, classe}
\end{entry}

\begin{entry}{汽车}{qi4che1}{7,4}{⽔、⾞}[HSK 1]
  \definition[辆]{s.}{automóvel | carro | veículo motorizado}
\end{entry}

\begin{entry}{汽水}{qi4 shui3}{7,4}{⽔、⽔}[HSK 4]
  \definition[罐,瓶]{s.}{refrigerante; refrigerante gaseificado; bebida refrescante, feita com a pressão de dióxido de carbono para dissolver na água e adicionar açúcar, suco de frutas, especiarias etc.}
\end{entry}

\begin{entry}{汽油}{qi4you2}{7,8}{⽔、⽔}[HSK 4]
  \definition{s.}{gasolina; mistura líquida de hidrocarbonetos com volatilidade e combustibilidade, que é usada como combustível a partir do fracionamento ou craqueamento do petróleo}
\end{entry}

\begin{entry}{器}{qi4}{16}{⼝}
  \definition[台]{s.}{dispositivo | ferramenta | utensílio}
\end{entry}

\begin{entry}{器官}{qi4guan1}{16,8}{⼝、⼧}[HSK 4]
  \definition[个]{s.}{órgão; aparelho; parte de um organismo que consiste em vários tipos de tecidos celulares que podem desempenhar uma função fisiológica separada}
\end{entry}

\begin{entry}{卡}{qia3}{5}{⼘}
  \definition[张]{s.}{grampo | prendedor}
  \definition{s.}{posto de controle}
  \definition{v.}{cunhar | ficar preso | encravar}
  \seeref{卡}{ka3}
\end{entry}

\begin{entry}{恰}{qia4}{9}{⼼}
  \definition{adv.}{exatamente | apenas}
\end{entry}

\begin{entry}{恰到好处}{qia4dao4hao3chu4}{9,8,6,5}{⼼、⼑、⼥、⼡}
  \definition{expr.}{é simplesmente perfeito | é simplesmente correto}
\end{entry}

\begin{entry}{恰好}{qia4hao3}{9,6}{⼼、⼥}
  \definition{adv.}{certo | por sorte | ao que parece | por sorte coincidência}
\end{entry}

\begin{entry}{千}{qian1}{3}{⼗}[HSK 2]
  \definition{num.}{mil; 1.000; 1000}
\end{entry}

\begin{entry}{千古}{qian1gu3}{3,5}{⼗、⼝}
  \definition{adv.}{por toda a eternidade | em todas as idades}
  \definition{s.}{eternidade (usada em um dístico elegíaco, coroa de flores, etc., dedicada aos mortos)}
\end{entry}

\begin{entry}{千克}{qian1 ke4}{3,7}{⼗、⼗}[HSK 2]
  \definition{clas.}{kg | quilo | quilograma}
\end{entry}

\begin{entry}{千年}{qian1nian2}{3,6}{⼗、⼲}
  \definition{s.}{milênio}
\end{entry}

\begin{entry}{千千万万}{qian1qian1wan4wan4}{3,3,3,3}{⼗、⼗、⼀、⼀}
  \definition{num.}{inumerável | números incontáveis | milhares e milhares}
\end{entry}

\begin{entry}{千万}{qian1wan4}{3,3}{⼗、⼀}[HSK 3]
  \definition{adv.}{(usado para indicar desejos fortes) por todos os meios; sob quaisquer circunstâncias}
  \definition{num.}{dez milhões; milhões e milhões}
\end{entry}

\begin{entry}{签}{qian1}{13}{⽵}[HSK 5]
  \definition{s.}{tiras de bambu usadas para adivinhação ou sorteio; pPequenas tiras de bambu ou varas finas com caracteres e símbolos gravados, usadas para adivinhação, jogos de azar ou como fichas para contagem, etc. | etiqueta; adesivo; pequena tira usada como marca | um pedaço fino e pontiagudo de bambu ou madeira; pequeno bastão pontiagudo}
  \definition{v.}{assinar; autografar; escrever o nome, palavras ou fazer marcas em documentos ou recibos | fazer comentários breves em um documento; escrever brevemente (pontos principais ou opiniões) | (em costura) alinhavar; costura grosseira}
\end{entry}

\begin{entry}{签订}{qian1 ding4}{13,4}{⽵、⾔}[HSK 5]
  \definition{v.}{concluir e assinar (um tratado, etc.)}
\end{entry}

\begin{entry}{签名}{qian1 ming2}{13,6}{⽵、⼝}[HSK 5]
  \definition[个,次]{s.}{assinatura; autógrafo}
  \definition{v.+compl.}{assinar o próprio nome; autografar; escrever seu nome para indicar concordância, apoio ou homenagem, etc.}
\end{entry}

\begin{entry}{签约}{qian1 yue1}{13,6}{⽵、⽷}[HSK 5]
  \definition{v.}{assinar um contrato; assinar contratos e tratados, frequentemente utilizado no trabalho e em cooperações comerciais}
\end{entry}

\begin{entry}{签证}{qian1zheng4}{13,7}{⽵、⾔}[HSK 5]
  \definition[张,个]{s.}{visto; visto de entrada em um país}
\end{entry}

\begin{entry}{签字}{qian1 zi4}{13,6}{⽵、⼦}[HSK 5]
  \definition{v.}{assinar; colocar a assinatura; escrever seu nome à mão em documentos, recibos, etc., para demonstrar responsabilidade}
\end{entry}

\begin{entry}{前}{qian2}{9}{⼑}[HSK 1]
  \definition{adv.}{frente; em frente de | A.C. (Antes de~Cristo)}[前293年  (293 a.C.)]
  \seealsoref{公元}{gong1yuan2}
\end{entry}

\begin{entry}{前边}{qian2bian5}{9,5}{⼑、⾡}[HSK 1]
  \definition{adv.}{à frente | da frente}
\end{entry}

\begin{entry}{前后}{qian2 hou4}{9,6}{⼑、⼝}[HSK 3]
  \definition{s.}{em volta; sobre | do início ao fim | frente e verso}
\end{entry}

\begin{entry}{前进}{qian2 jin4}{9,7}{⼑、⾡}[HSK 3]
  \definition{v.}{marchar; avançar; para ir em frente; seguir em frente}
\end{entry}

\begin{entry}{前景}{qian2jing3}{9,12}{⼑、⽇}[HSK 5]
  \definition{s.}{primeiro plano (de uma vista, imagem, foto, etc.); as imagens que parecem mais próximas do espectador em pinturas, palcos e telas | vista; perspectiva; prospecto; ponto de vista; situações que podem ocorrer no trabalho, na carreira, etc.}
\end{entry}

\begin{entry}{前面}{qian2mian4}{9,9}{⼑、⾯}[HSK 3]
  \definition{s.}{frente | parte anterior; acima}
\end{entry}

\begin{entry}{前年}{qian2 nian2}{9,6}{⼑、⼲}[HSK 2]
  \definition{adv.}{há dois anos}
\end{entry}

\begin{entry}{前提}{qian2ti2}{9,12}{⼑、⼿}[HSK 5]
  \definition[个,项]{s.}{premissa; pressuposto | pré-requisito; pressuposição; condições prévias para que algo aconteça ou se desenvolva}
\end{entry}

\begin{entry}{前天}{qian2tian1}{9,4}{⼑、⼤}[HSK 1]
  \definition{adv.}{anteontem}
\end{entry}

\begin{entry}{前头}{qian2 tou5}{9,5}{⼑、⼤}[HSK 4]
  \definition{s.}{à frente; na frente; adiante}
\end{entry}

\begin{entry}{前途}{qian2tu2}{9,10}{⼑、⾡}[HSK 4]
  \definition[片,段,种]{s.}{futuro; perspectiva; prospecto; originalmente, refere-se à jornada à frente, mas, metaforicamente, refere-se ao futuro.}
\end{entry}

\begin{entry}{前往}{qian2 wang3}{9,8}{⼑、⼻}[HSK 3]
  \definition{v.}{ir para; prosseguir para; partir para}
\end{entry}

\begin{entry}{钱}{qian2}{10}{⾦}[HSK 1]
  \definition*{s.}{sobrenome Qian}
  \definition[笔]{s.}{moeda | dinheiro}
\end{entry}

\begin{entry}{钱包}{qian2bao1}{10,5}{⾦、⼓}[HSK 1]
  \definition{s.}{carteira | bolsa}
\end{entry}

\begin{entry}{潜在}{qian2zai4}{15,6}{⽔、⼟}
  \definition{adj.}{oculto | latente}
  \definition{s.}{potencial}
\end{entry}

\begin{entry}{浅}{qian3}{8}{⽔}[HSK 4]
  \definition{adj.}{raso; superficial;  (em oposição a ``深'') | fácil; simples; redação, conteúdo, etc. simples e fáceis de entender | superficial; não é profundo em aprendizado, percepção e sabedoria | não próximo; não íntimo; sentimentos não profundos | (cor) claro; pálido;  cor pouco intensa; leve |experiência breve; duração de tempo breve | baixo grau; peso leve; nível baixo}
  \seeref{浅}{jian1}
  \seealsoref{深}{shen1}
\end{entry}

\begin{entry}{欠}{qian4}{4}{⽋}[HSK 5]
  \definition{v.}{bocejar | levantar ligeiramente (uma parte do corpo) | estar em dívida; estar atrasado com; não devolver o que pediu emprestado a outra pessoa, ou não dar o que deveria ter dado a outra pessoa | faltar; não ser suficiente}
\end{entry}

\begin{entry}{抢}{qiang1}{7}{⼿}
  \definition{prep.}{contra; direção relativa inversa}
  \definition{v.}{bater; tocar}
  \seeref{抢}{qiang3}
\end{entry}

\begin{entry}{枪}{qiang1}{8}{⽊}[HSK 5]
  \definition*{s.}{sobrenome Qiang}
  \definition{s.}{lança | arma; rifle; arma de fogo | uma coisa em forma de arma | enxada; ferramenta para cavar a terra}
  \definition{v.}{escrever artigos ou responder perguntas para outras pessoas}
\end{entry}

\begin{entry}{将}{qiang1}{9}{⼨}
  \definition{v.}{pedir; apelar para}
  \seeref{将}{jiang1}
  \seeref{将}{jiang4}
\end{entry}

\begin{entry}{强}{qiang2}{12}{⼸}[HSK 3]
  \definition*{s.}{sobrenome Qiang}
  \definition{adj.}{forte; poderoso | melhor; superior | mais; extra; adicional; um pouco mais que | resoluto; firme | decidido; resolvido | violento; impetuoso | alto padrão}
  \definition{v.}{fortalecer; tornar forte}
  \seeref{强}{jiang4}
  \seeref{强}{qiang3}
\end{entry}

\begin{entry}{强大}{qiang2 da4}{12,3}{⼸、⼤}[HSK 3]
  \definition{adj.}{forte; poderoso; potente; possante}
\end{entry}

\begin{entry}{强调}{qiang2diao4}{12,10}{⼸、⾔}[HSK 3]
  \definition{v.}{salientar; sublinhar; enfatizar; dar ênfase a; vincar}
\end{entry}

\begin{entry}{强度}{qiang2 du4}{12,9}{⼸、⼴}[HSK 5]
  \definition[个]{s.}{intensidade; força | magnitude; rigor; avidez}
\end{entry}

\begin{entry}{强烈}{qiang2lie4}{12,10}{⼸、⽕}[HSK 3]
  \definition{adj.}{forte; intenso | violento; impetuoso | afiado; marcante}
\end{entry}

\begin{entry}{墙}{qiang2}{14}{⼟}[HSK 2]
  \definition[面,堵]{s.}{parede}
  \definition{v.}{(gíria) bloquear (um website) (usado geralmente na voz passiva: 被墙)}
\end{entry}

\begin{entry}{墙壁}{qiang2 bi4}{14,16}{⼟、⼟}[HSK 5]
  \definition[堵]{s.}{parede; barreira ou perímetro construído com tijolos, pedras ou terra}
\end{entry}

\begin{entry}{墙纸}{qiang2zhi3}{14,7}{⼟、⽷}
  \definition{s.}{papel de parede}
\end{entry}

\begin{entry}{抢}{qiang3}{7}{⼿}[HSK 5]
  \definition{v.}{roubar; saquear | agarrar; apanhar; arrebatar | disputar; lutar por; ser o primeiro; competir para ser o primeiro | correr; apressar-se; fazer uma incursão | raspar; arranhar; raspar ou esfregar uma camada da superfície de um objeto}
  \seeref{抢}{qiang1}
\end{entry}

\begin{entry}{抢救}{qiang3jiu4}{7,11}{⼿、⽁}[HSK 5]
  \definition{v.}{salvar; resgatar; prestar de socorro ou assistência rápidos em situações de emergência | salvar; tomar medidas rápidas para evitar ou minimizar perdas iminentes.}
\end{entry}

\begin{entry}{抢掠}{qiang3lve4}{7,11}{⼿、⼿}
  \definition{s.}{saque | pilhagem}
  \definition{v.}{saquear | pilhar}
\end{entry}

\begin{entry}{强}{qiang3}{12}{⼸}
  \definition{v.}{fazer um esforço; esforçar-se}
  \seeref{强}{jiang4}
  \seeref{强}{qiang2}
\end{entry}

\begin{entry}{强迫}{qiang3po4}{12,8}{⼸、⾡}[HSK 5]
  \definition{v.}{impelir; forçar; impor; compelir; aplicar pessão para obedecer}
\end{entry}

\begin{entry}{悄悄}{qiao1qiao1}{10,10}{⼼、⼼}[HSK 5]
  \definition{adv.}{silenciosamente; em silêncio; aos sussuros; sem som ou em voz baixa; com o mínimo de ruído possível}
\end{entry}

\begin{entry}{敲}{qiao1}{14}{⽁}[HSK 5]
  \definition{v.}{bater; dar uma pancada; golpear | explorar alguém; cobrar a mais; extorquir; chantagear | lembrar; criticar; alertar; advertir}
\end{entry}

\begin{entry}{敲门}{qiao1 men2}{14,3}{⽁、⾨}[HSK 5]
  \definition{v.}{bater na porta}
\end{entry}

\begin{entry}{桥}{qiao2}{10}{⽊}[HSK 3]
  \definition*{s.}{sobrenome Qiao}
  \definition[座]{s.}{ponte}
\end{entry}

\begin{entry}{瞧}{qiao2}{17}{⽬}[HSK 5]
  \definition{v.}{ver; olhar | tratar; diagnosticar e tratar | ver; visitar; fazer uma visita}
\end{entry}

\begin{entry}{巧}{qiao3}{5}{⼯}[HSK 3]
  \definition{adj.}{habilidoso; engenhoso; esperto | oportuno; coincidente; fortuito | astuto; enganoso; enganador; traiçoeiro; ardiloso}
\end{entry}

\begin{entry}{巧合}{qiao3he2}{5,6}{⼯、⼝}
  \definition{s.}{coincidência}
  \definition{v.}{coincidir}
\end{entry}

\begin{entry}{巧克力}{qiao3ke4li4}{5,7,2}{⼯、⼗、⼒}[HSK 4]
  \definition[块]{s.}{(empréstimo linguístico) chocolate}
\end{entry}

\begin{entry}{切}{qie1}{4}{⼑}[HSK 4]
  \definition{v.}{cortar; fatiar; separar itens com uma faca | cortar ou romper; truncar | (geometria) geometria, refere-se a quando uma linha, círculo ou superfície intercepta um círculo, arco ou esfera em apenas um ponto}
  \seeref{切}{qie4}
\end{entry}

\begin{entry}{切割}{qie1ge1}{4,12}{⼑、⼑}
  \definition{v.}{cortar}
\end{entry}

\begin{entry}{茄子}{qie2zi5}{8,3}{⾋、⼦}
  \definition{s.}{berinjela chinesa | ``xis'' fonético (ao ser fotografado), equivale ao ``diga xis''}
\end{entry}

\begin{entry}{且}{qie3}{5}{⼀}
  \definition*{s.}{sobrenome Qie}
  \definition{adv.}{apenas; por enquanto | por um longo tempo}
  \definition{conj.}{mesmo; até; até mesmo | ambos\dots e\dots}
\end{entry}

\begin{entry}{切}{qie4}{4}{⼑}
  \definition{adj.}{ansioso; sério | duro; severo; rude; áspero}
  \definition{adv.}{com certeza; certamente}
  \definition{s.}{limiar; degrau}
  \definition{v.}{ser prático ou realista | ajustar-se ou corresponder | ser próximo ou íntimo | cortar algo em pedaços com uma faca | (medicina tradicional chinesa) tomar o pulso}
  \seeref{切}{qie1}
\end{entry}

\begin{entry}{亲}{qin1}{9}{⼇}[HSK 3]
  \definition{adj.}{parente próximo; relacionado por sangue; de ​​relação de sangue | querido; próximo; íntimo | em si mesmo; pessoalmente}
  \definition[位]{s.}{pais | parente | casal; casamento | noiva}
  \definition{v.}{beijar | (países, partidos, etc.) a favor de; apoiar; estar perto de}
  \seeref{亲}{qing4}
\end{entry}

\begin{entry}{亲爱}{qin1'ai4}{9,10}{⼇、⽖}[HSK 4]
  \definition{adj.}{querido; amado; termo carinhoso que expressa intimidade e afeto}
\end{entry}

\begin{entry}{亲密}{qin1mi4}{9,11}{⼇、⼧}[HSK 4]
  \definition{adj.}{próximo; íntimo; relacionamento afetuoso e próximo}
\end{entry}

\begin{entry}{亲切}{qin1qie4}{9,4}{⼇、⼑}[HSK 3]
  \definition{adj.}{gentil; cordial | próximo; íntimo}
\end{entry}

\begin{entry}{亲人}{qin1 ren2}{9,2}{⼇、⼈}[HSK 3]
  \definition{s.}{um membro da família; os pais, o cônjuge, os filhos, etc. | queridos; entes queridos; aqueles queridos para alguém}
\end{entry}

\begin{entry}{亲自}{qin1zi4}{9,6}{⼇、⾃}[HSK 3]
  \definition{adv.}{pessoalmente; em pessoa; si mesmo}
\end{entry}

\begin{entry}{侵略}{qin1lve4}{9,11}{⼈、⽥}
  \definition{s.}{invasão}
  \definition{v.}{invadir}
\end{entry}

\begin{entry}{芹菜}{qin2cai4}{7,11}{⾋、⾋}
  \definition{s.}{salsão}
\end{entry}

\begin{entry}{琴}{qin2}{12}{⽟}[HSK 5]
  \definition*{s.}{sobrenome Qin}
  \definition[架,台]{s.}{cítara; qin; guqin (um instrumento de cordas dedilhadas com sete cordas, em alguns aspectos semelhante à cítara)  | nome genérico para certos instrumentos musicais}
\end{entry}

\begin{entry}{琴键}{qin2jian4}{12,13}{⽟、⾦}
  \definition{s.}{tecla de piano}
\end{entry}

\begin{entry}{禽}{qin2}{12}{⽱}
  \definition*{s.}{sobrenome Qin}
  \definition[只]{s.}{aves; pássaros | termo genérico para aves e animais}
\end{entry}

\begin{entry}{勤奋}{qin2fen4}{13,8}{⼒、⼤}[HSK 5]
  \definition{adj.}{diligente; assíduo; trabalhador; descreve alguém que se esforça continuamente nos estudos ou no trabalho}
\end{entry}

\begin{entry}{擒获}{qin2huo4}{15,10}{⼿、⾋}
  \definition{v.}{apreender | capturar}
\end{entry}

\begin{entry}{青}{qing1}{8}{⾭}[HSK 5]
  \definition*{s.}{abreviação de Província de Qinghai | sobrenome Qing}
  \definition{adj.}{azul ou verde | preto | jovens (pessoas)}
  \definition{s.}{grama verde | colheitas jovens (não maduras) | tiras de bambu verde}
\end{entry}

\begin{entry}{青菜}{qing1cai4}{8,11}{⾭、⾋}
  \definition{s.}{verduras}
\end{entry}

\begin{entry}{青春}{qing1chun1}{8,9}{⾭、⽇}[HSK 4]
  \definition[个]{s.}{juventude; jovialidade}
\end{entry}

\begin{entry}{青椒}{qing1jiao1}{8,12}{⾭、⽊}
  \definition{s.}{pimenta verde}
\end{entry}

\begin{entry}{青年}{qing1 nian2}{8,6}{⾭、⼲}[HSK 2]
  \definition[个,名,位]{s.}{juventude | jovem}
\end{entry}

\begin{entry}{青年节}{qing1nian2jie2}{8,6,5}{⾭、⼲、⾋}
  \definition*{s.}{Dia da Juventude (4 de maio)}
\end{entry}

\begin{entry}{青少年}{qing1shao4nian2}{8,4,6}{⾭、⼩、⼲}[HSK 2]
  \definition[位,个]{s.}{adolescentes}
\end{entry}

\begin{entry}{青天}{qing1tian1}{8,4}{⾭、⼤}
  \definition{s.}{céu claro, limpo ou azul}
\end{entry}

\begin{entry}{青铜}{qing1tong2}{8,11}{⾭、⾦}
  \definition{s.}{bronze (liga de cobre, 銅, e estanho, 锡)}
\end{entry}

\begin{entry}{青蛙}{qing1wa1}{8,12}{⾭、⾍}
  \definition{adj.}{(gíria velha) cara feio}
  \definition[只]{s.}{sapo}
\end{entry}

\begin{entry}{青玉米}{qing1yu4mi3}{8,5,6}{⾭、⽟、⽶}
  \definition{s.}{milho verde}
\end{entry}

\begin{entry}{轻}{qing1}{9}{⾞}[HSK 2]
  \definition{adj.}{leve | pequeno em número, grau, etc. | não importante | relaxado}
  \definition{adv.}{suavemente | levemente | precipitadamente}
  \definition{v.}{menosprezar}
\end{entry}

\begin{entry}{轻松}{qing1song1}{9,8}{⾞、⽊}[HSK 4]
  \definition{adj.}{leve; relaxado; livre de fardos; não se sentir nervoso ou cansado}
  \definition{v.}{relaxar; levar as coisas menos a sério}
\end{entry}

\begin{entry}{轻易}{qing1yi4}{9,8}{⾞、⽇}[HSK 4]
  \definition{adj.}{fácil; simples}
  \definition{adv.}{facilmente; prontamente | facilmente; precipitadamente; indica que uma ação é realizada casualmente, geralmente usado em frases negativas}
\end{entry}

\begin{entry}{倾城}{qing1cheng2}{10,9}{⼈、⼟}
  \definition{adj.}{sedutora (mulher)}
  \definition{adv.}{de todo o lugar | vindo de todos os lugares}
  \definition{v.}{arruinar e derrubar o estado}
\end{entry}

\begin{entry}{清}{qing1}{11}{⽔}
  \definition*{s.}{sobrenome Qing}
  \definition{adj.}{claro | limpo (água, etc.) | tranquilo | quieto | puro | não corrompido | distinto}
  \definition{v.}{limpar | resolver (contas)}
\end{entry}

\begin{entry}{清唱}{qing1chang4}{11,11}{⽔、⼝}
  \definition{v.}{cantar à capela}
\end{entry}

\begin{entry}{清彻}{qing1che4}{11,7}{⽔、⼻}
  \variantof{清澈}
\end{entry}

\begin{entry}{清澈}{qing1che4}{11,15}{⽔、⽔}
  \definition{adj.}{claro | límpido}
\end{entry}

\begin{entry}{清晨}{qing1chen2}{11,11}{⽔、⽇}[HSK 5]
  \definition{s.}{matinal; manhã cedo; geralmente se refere ao período do amanhecer até logo após o nascer do sol}
\end{entry}

\begin{entry}{清楚}{qing1chu5}{11,13}{⽔、⽊}[HSK 2]
  \definition{adj.}{claro | límpido}
  \definition{v.}{ser claro sobre | entender completamente}
\end{entry}

\begin{entry}{清理}{qing1li3}{11,11}{⽔、⽟}[HSK 5]
  \definition{v.}{esclarecer; resolver; verificar; colocar em ordem; organizar tudo e jogar fora o que não for útil}
\end{entry}

\begin{entry}{清凉}{qing1liang2}{11,10}{⽔、⼎}
  \definition{adj.}{fresco | refrescante | (roupa) ousada, reveladora}
\end{entry}

\begin{entry}{清明节}{qing1ming2jie2}{11,8,5}{⽔、⽇、⾋}
  \definition*{s.}{Dia Qingming, Dia dos Finados (uma das 24~divisões do ano solar no calendário lunar chinês:~dia~4 ou 5~de~abril solar)}
\end{entry}

\begin{entry}{清爽}{qing1shuang3}{11,11}{⽔、⽘}
  \definition{adj.}{refrescante | relaxado}
\end{entry}

\begin{entry}{清晰}{qing1xi1}{11,12}{⽔、⽇}
  \definition{adj.}{claro | distinto}
\end{entry}

\begin{entry}{清醒}{qing1xing3}{11,16}{⽔、⾣}[HSK 4]
  \definition{adj.}{sóbrio; lúcido; totalmente acordado}
\end{entry}

\begin{entry}{蜻蜓}{qing1ting2}{14,12}{⾍、⾍}
  \definition{s.}{libélula}
\end{entry}

\begin{entry}{蜻蝏}{qing1ting2}{14,15}{⾍、⾍}
  \variantof{蜻蜓}
\end{entry}

\begin{entry}{情感}{qing2 gan3}{11,13}{⼼、⼼}[HSK 3]
  \definition[份]{s.}{emoção; sentimento | afeição; apego}
  \definition{v.}{mover-se (emocionalmente)}
\end{entry}

\begin{entry}{情节}{qing2jie2}{11,5}{⼼、⾋}[HSK 5]
  \definition{s.}{enredo; trama; desenrolar específico dos acontecimentos | circunstância; detalhes do crime ou erro | enredo; roteiro; refere-se especificamente ao processo de desenvolvimento e evolução dos conflitos e contradições em obras literárias narrativas}
\end{entry}

\begin{entry}{情景}{qing2jing3}{11,12}{⼼、⽇}[HSK 4]
  \definition[个]{s.}{cena; vista; circunstâncias}
\end{entry}

\begin{entry}{情况}{qing2kuang4}{11,7}{⼼、⼎}[HSK 3]
  \definition[种,个,些]{s.}{condição; situação; circunstâncias; estado das coisas | mudança notável}
\end{entry}

\begin{entry}{情形}{qing2xing2}{11,7}{⼼、⼺}[HSK 5]
  \definition[个]{s.}{situação; condição; circunstâncias; estado de coisas; a situação específica das coisas}
\end{entry}

\begin{entry}{情绪}{qing2xu4}{11,11}{⼼、⽷}
  \definition[种]{s.}{humor | estado da mente | mau humor}
\end{entry}

\begin{entry}{晴}{qing2}{12}{⽇}[HSK 2]
  \definition{adj.}{ensolarado | claro}
\end{entry}

\begin{entry}{晴朗}{qing2lang3}{12,10}{⽇、⽉}[HSK 5]
  \definition{adj.}{bom; claro; ensolarado; céu limpo e sem nuvens}
\end{entry}

\begin{entry}{晴天}{qing2 tian1}{12,4}{⽇、⼤}[HSK 2]
  \definition[个]{s.}{dia ensolarado}
\end{entry}

\begin{entry}{请}{qing3}{10}{⾔}[HSK 1]
  \definition{v.}{por favor (fazer alguma coisa) | perguntar | convidar | solicitar}
\end{entry}

\begin{entry}{请假}{qing3 jia4}{10,11}{⾔、⼈}[HSK 1]
  \definition{v.+compl.}{pedir licença para sair}
\end{entry}

\begin{entry}{请假条}{qing3jia4tiao2}{10,11,7}{⾔、⼈、⽊}
  \definition{s.}{pedido de licença de ausência (do trabalho ou da escola)}
\end{entry}

\begin{entry}{请教}{qing3jiao4}{10,11}{⾔、⽁}[HSK 3]
  \definition{v.}{consultar; pedir conselho}
\end{entry}

\begin{entry}{请进}{qing3 jin4}{10,7}{⾔、⾡}[HSK 1]
  \definition{v.}{por favor entre}
\end{entry}

\begin{entry}{请客}{qing3ke4}{10,9}{⾔、⼧}[HSK 2]
  \definition{v.+compl.}{entreter os convidados | dar um jantar | convidar para jantar}
\end{entry}

\begin{entry}{请求}{qing3qiu2}{10,7}{⾔、⽔}[HSK 2]
  \definition[个]{s.}{solicitação}
  \definition{v.}{solicitar | perguntar}
\end{entry}

\begin{entry}{请问}{qing3wen4}{10,6}{⾔、⾨}[HSK 1]
  \definition{expr.}{Com licença, posso perguntar\dots? (para perguntar por qualquer coisa)}
\end{entry}

\begin{entry}{请坐}{qing3 zuo4}{10,7}{⾔、⼟}[HSK 1]
  \definition{v.}{por favor, sente-se}
\end{entry}

\begin{entry}{庆祝}{qing4zhu4}{6,9}{⼴、⽰}[HSK 3]
  \definition{v.}{celebrar; comemorar; festejar}
\end{entry}

\begin{entry}{亲}{qing4}{9}{⼇}
  \definition{s.}{parentes por afinidade; parentes por casamento}
  \seeref{亲}{qin1}
\end{entry}

\begin{entry}{穷}{qiong2}{7}{⽳}[HSK 4]
  \definition{adj.}{remoto; isolado; de difícil acesso | pobre; atingido pela pobreza | situação difícil, sem saída}
  \definition{adv.}{completamente | extremamente}
  \definition{v.}{exaurir; esgotar; consmir | ir até o fim; perseguir completamente perseguido; sondar profundamente | gastar}
\end{entry}

\begin{entry}{穷人}{qiong2 ren2}{7,2}{⽳、⼈}[HSK 4]
  \definition{s.}{os pobres; pessoas pobres}
\end{entry}

\begin{entry}{丘陵}{qiu1ling2}{5,10}{⼀、⾩}
  \definition{s.}{colinas}
\end{entry}

\begin{entry}{秋}{qiu1}{9}{⽲}
  \definition*{s.}{sobrenome Qiu}
  \definition{s.}{outono | colheita}
\end{entry}

\begin{entry}{秋季}{qiu1 ji4}{9,8}{⽲、⼦}[HSK 4]
  \definition[个]{s.}{outono; terceiro trimestre do ano, segundo o costume chinês, refere-se ao período de três meses entre o outono e o inverno, também se refere aos sétimo, oitavo e nono meses do calendário lunar}
\end{entry}

\begin{entry}{秋天}{qiu1 tian1}{9,4}{⽲、⼤}[HSK 2]
  \definition[个]{s.}{outono}
\end{entry}

\begin{entry}{求}{qiu2}{7}{⽔}[HSK 2]
  \definition*{s.}{sobrenome Qiu}
  \definition{s.}{demanda}
  \definition{v.}{pedir | implorar | solicitar | suplicar | esforçar-se por | procurar | tentar}
\end{entry}

\begin{entry}{球}{qiu2}{11}{⽟}[HSK 1]
  \definition[个]{s.}{bola | esfera | globo}
  \definition[场]{s.}{jogo | partida de bola}
\end{entry}

\begin{entry}{球场}{qiu2 chang3}{11,6}{⽟、⼟}[HSK 2]
  \definition{s.}{um campo onde são jogados jogos de bola | tribunal | campo | curso | \emph{links}}
\end{entry}

\begin{entry}{球队}{qiu2 dui4}{11,4}{⽟、⾩}[HSK 2]
  \definition{s.}{time (basquete, futebol, etc.)}
\end{entry}

\begin{entry}{球迷}{qiu2mi2}{11,9}{⽟、⾡}[HSK 3]
  \definition[个]{s.}{fã (de esportes de bola)}
\end{entry}

\begin{entry}{球拍}{qiu2pai1}{11,8}{⽟、⼿}
  \definition{s.}{raquete}
\end{entry}

\begin{entry}{球鞋}{qiu2 xie2}{11,15}{⽟、⾰}[HSK 2]
  \definition{s.}{calçados esportivos | tênis | tênis de ginástica}
\end{entry}

\begin{entry}{区}{qu1}{4}{⼖}[HSK 3]
  \definition[个]{s.}{área; distrito; região; zona | uma divisão administrativa}
  \definition{v.}{distinguir; classificar; subdividir}
  \seeref{区}{ou1}
\end{entry}

\begin{entry}{区别}{qu1bie2}{4,7}{⼖、⼑}[HSK 3]
  \definition[个]{s.}{diferença; distinção; discriminação}
  \definition{v.}{distinguir; diferenciar; fazer distinção entre}
\end{entry}

\begin{entry}{区域}{qu1yu4}{4,11}{⼖、⼟}[HSK 5]
  \definition[片,块,个]{s.}{área; setor; região; faixa; inclui áreas regionais com condições naturais, culturais, administrativas, etc.}
\end{entry}

\begin{entry}{曲棍球}{qu1gun4qiu2}{6,12,11}{⽈、⽊、⽟}
  \definition{s.}{hóquei em campo}
\end{entry}

\begin{entry}{驱}{qu1}{7}{⾺}
  \definition{v.}{expulsar | repelir}
\end{entry}

\begin{entry}{趋势}{qu1shi4}{12,8}{⾛、⼒}[HSK 4]
  \definition{s.}{tendência; tendência; direção; impulso das coisas que se movem em uma direção ou outra}
\end{entry}

\begin{entry}{取}{qu3}{8}{⼜}[HSK 2]
  \definition{v.}{buscar | obter | escolher}
\end{entry}

\begin{entry}{取得}{qu3 de2}{8,11}{⼜、⼻}[HSK 2]
  \definition{v.}{ganhar | adquirir | obter}
\end{entry}

\begin{entry}{取胜}{qu3sheng4}{8,9}{⼜、⾁}
  \definition{v.}{prevalecer sobre os oponentes | marcar uma vitória}
\end{entry}

\begin{entry}{取水}{qu3shui3}{8,4}{⼜、⽔}
  \definition{v.}{obter água (de um poço, etc.)}
\end{entry}

\begin{entry}{取现}{qu3xian4}{8,8}{⼜、⾒}
  \definition{v.}{sacar dinheiro}
\end{entry}

\begin{entry}{取消}{qu3xiao1}{8,10}{⼜、⽔}[HSK 3]
  \definition{v.}{cancelar; suspender; anular; abolir; revogar; rescindir}
\end{entry}

\begin{entry}{取悦}{qu3yue4}{8,10}{⼜、⼼}
  \definition{v.}{tentar agradar}
\end{entry}

\begin{entry}{厺}{qu4}{5}{⼤}
  \variantof{去}
\end{entry}

\begin{entry}{去}{qu4}{5}{⼛}[HSK 1]
  \definition{v.}{ir | (eufenismo) morrer}
\end{entry}

\begin{entry}{去年}{qu4nian2}{5,6}{⼛、⼲}[HSK 1]
  \definition{s.}{ano passado}
\end{entry}

\begin{entry}{去世}{qu4shi4}{5,5}{⼛、⼀}[HSK 3]
  \definition{v.}{morrer; falecer (um adulto)}
\end{entry}

\begin{entry}{去死}{qu4si3}{5,6}{⼛、⽍}
  \definition{interj.}{Caia morto! | Vá para o Inferno!}
\end{entry}

\begin{entry}{圈}{quan1}{11}{⼞}[HSK 4]
  \definition[个]{s.}{anel; círculo; refere-se a algo em forma de anel | domínio; grupo; escopo; círculo(s)}
  \definition{v.}{cercar; rodear; circundar | marcar com um círculo}
  \seeref{圈}{juan1}
  \seeref{圈}{juan4}
\end{entry}

\begin{entry}{圈粉}{quan1fen3}{11,10}{⼞、⽶}
  \definition{s.}{(neologismo, coloquial) ganhar alguém como fã, obter novos fãs}
\end{entry}

\begin{entry}{全}{quan2}{6}{⼊}[HSK 2]
  \definition*{s.}{sobrenome Quan}
  \definition{adv.}{completamente | totalmente}
\end{entry}

\begin{entry}{全部}{quan2bu4}{6,10}{⼊、⾢}[HSK 2]
  \definition{adv.}{todo, todos}
\end{entry}

\begin{entry}{全场}{quan2 chang3}{6,6}{⼊、⼟}[HSK 3]
  \definition{s.}{toda a audiência; todos os presentes | corte (de justiça) inteira}
\end{entry}

\begin{entry}{全都}{quan2 dou1}{6,10}{⼊、⾢}[HSK 5]
  \definition{adv.}{tudo; todos; sem exceção}
\end{entry}

\begin{entry}{全都不}{quan2dou1 bu4}{6,10,4}{⼊、⾢、⼀}
  \definition{adj.}{nada; nenhum; nenhum deles; nada disso}
\end{entry}

\begin{entry}{全国}{quan2 guo2}{6,8}{⼊、⼞}[HSK 2]
  \definition{s.}{nação | a nação inteira | o país inteiro | todo o país}
\end{entry}

\begin{entry}{全家}{quan2 jia1}{6,10}{⼊、⼧}[HSK 2]
  \definition{s.}{a família inteira | toda a família}
\end{entry}

\begin{entry}{全面}{quan2mian4}{6,9}{⼊、⾯}[HSK 3]
  \definition{adj.}{geral; completo}
  \definition{s.}{todos os aspectos; cada aspecto}
  \seealsoref{片面}{pian4mian4}
\end{entry}

\begin{entry}{全年}{quan2 nian2}{6,6}{⼊、⼲}[HSK 2]
  \definition{s.}{anual}
\end{entry}

\begin{entry}{全球}{quan2 qiu2}{6,11}{⼊、⽟}[HSK 3]
  \definition{adj.}{global}
  \definition{s.}{o mundo inteiro}
\end{entry}

\begin{entry}{全身}{quan2 shen1}{6,7}{⼊、⾝}[HSK 2]
  \definition{s.}{corpo inteiro | todo (o corpo)}
\end{entry}

\begin{entry}{全世界}{quan2 shi4 jie4}{6,5,9}{⼊、⼀、⽥}[HSK 5]
  \definition[种]{s.}{mundo inteiro; mundo todo | em todo o mundo}
\end{entry}

\begin{entry}{全体}{quan2 ti3}{6,7}{⼊、⼈}[HSK 2]
  \definition{s.}{tudo | todo | inteiro}
\end{entry}

\begin{entry}{全职}{quan2zhi2}{6,11}{⼊、⽿}
  \definition{s.}{período integral | tempo inteiro | (trabalho) \emph{full-time}}
\end{entry}

\begin{entry}{权利}{quan2li4}{6,7}{⽊、⼑}[HSK 4]
  \definition[项,种,个,条,份]{s.}{direito; interesse; os poderes e benefícios (em oposição a “义务”) exercidos por um cidadão ou pessoa jurídica de acordo com a lei}
  \seealsoref{义务}{yi4wu4}
\end{entry}

\begin{entry}{泉}{quan2}{9}{⽔}[HSK 5]
  \definition*{s.}{sobrenome Quan}
  \definition[股,眼,汪]{s.}{fonte (de água mineral) | a nascente de um rio | termo antigo para moeda}
\end{entry}

\begin{entry}{拳法}{quan2fa3}{10,8}{⼿、⽔}
  \definition{s.}{boxe | luta}
\end{entry}

\begin{entry}{拳王}{quan2wang2}{10,4}{⼿、⽟}
  \definition{s.}{pugilista | boxeador}
\end{entry}

\begin{entry}{犬}{quan3}{4}{⽝}[Kangxi 94]
  \definition{s.}{cachorro}
\end{entry}

\begin{entry}{劝}{quan4}{4}{⼒}[HSK 5]
  \definition*{s.}{sobrenome Quan}
  \definition{v.}{insistir; aconselhar; tentar persuadir; persuadir, argumentar para que as pessoas obedeçam | incentivar; encorajar}
\end{entry}

\begin{entry}{缺}{que1}{10}{⽸}[HSK 3]
  \definition{adj.}{incompleto; imperfeito}
  \definition[种]{s.}{vaga; abertura; falta}
  \definition{v.}{estar com falta de; faltar | estar faltando; estar incompleto | estar ausente}
\end{entry}

\begin{entry}{缺点}{que1dian3}{10,9}{⽸、⽕}[HSK 3]
  \definition[个]{s.}{desvantagem; deficiência; inconveniência; ponto fraco}
\end{entry}

\begin{entry}{缺乏}{que1fa2}{10,4}{⽸、⼃}[HSK 5]
  \definition{v.}{faltar; estar em falta de; não ter ou não ter totalmente (algo que deveria possuir ou é desejaria possuir)}
\end{entry}

\begin{entry}{缺勤}{que1qin2}{10,13}{⽸、⼒}
  \definition{v.+compl.}{ausentar-se do dever (trabalho)}
\end{entry}

\begin{entry}{缺少}{que1shao3}{10,4}{⽸、⼩}[HSK 3]
  \definition{v.}{falta; estar com falta de; estar em falta de}
\end{entry}

\begin{entry}{却}{que4}{7}{⼙}[HSK 4]
  \definition{adv.}{mas; contudo; no entanto; enquanto; indica um ponto de virada}
  \definition{v.}{recuar; retroceder | afastar; repelir; desencorajar | declinar; recusar; rejeitar | usado depois de certos verbos para indicar a conclusão de uma ação}
\end{entry}

\begin{entry}{却是}{que4shi4}{7,9}{⼙、⽇}
  \definition{conj.}{no entanto | realmente | o fato é\dots | mas isso é\dots}
\end{entry}

\begin{entry}{确}{que4}{12}{⽯}
  \definition{adj.}{autenticado | sólido | firme | real | verdadeiro}
\end{entry}

\begin{entry}{确保}{que4bao3}{12,9}{⽯、⼈}[HSK 3]
  \definition{v.}{assegurar; garantir}
\end{entry}

\begin{entry}{确定}{que4ding4}{12,8}{⽯、⼧}[HSK 3]
  \definition{adj.}{definido; certo}
  \definition{v.}{consertar; definir; determinar}
\end{entry}

\begin{entry}{确立}{que4li4}{12,5}{⽯、⽴}[HSK 5]
  \definition{v.}{estabelecer; criar; construir; estabelecer ou consolidar firmemente}
\end{entry}

\begin{entry}{确认}{que4ren4}{12,4}{⽯、⾔}[HSK 4]
  \definition{v.}{afirmar; confirmar; reconhecer; confirmar explicitamente (fatos, princípios, etc.)}
\end{entry}

\begin{entry}{确实}{que4shi2}{12,8}{⽯、⼧}[HSK 3]
  \definition{adj.}{verdadeiro; confiável}
  \definition{adv.}{verdadeiramente; realmente; de ​​fato}
\end{entry}

\begin{entry}{裙子}{qun2zi5}{12,3}{⾐、⼦}[HSK 3]
  \definition[条,件]{s.}{saia (peça de vestuário)}
\end{entry}

\begin{entry}{群}{qun2}{13}{⽺}[HSK 3]
  \definition{clas.}{grupo; rebanho; manada}
  \definition{s.}{multidão; grupo}
\end{entry}

\begin{entry}{群山}{qun2shan1}{13,3}{⽺、⼭}
  \definition{s.}{montanhas | uma cadeia de colinas}
\end{entry}

\begin{entry}{群体}{qun2 ti3}{13,7}{⽺、⼈}[HSK 5]
  \definition{s.}{colônia; um conjunto composto por muitos indivíduos da mesma espécie que estão fisicamente conectados, exemplos incluem corais entre os animais e certas algas entre as plantas | grupos; refere-se, de maneira geral, ao conjunto formado por muitos indivíduos interligados que compartilham características essenciais em comum}
\end{entry}

\begin{entry}{群众}{qun2zhong4}{13,6}{⽺、⼈}[HSK 5]
  \definition[个,名,位]{s.}{as massas; refere-se ao povo em geral | não filiado; apartidário; refere-se a pessoas que não são membros do Partido Comunista Chinês nem da Liga da Juventude Comunista |
alguém que não ocupa uma posição de liderança}
\end{entry}

%%%%% EOF %%%%%


%%%
%%% R
%%%

\section*{R}\addcontentsline{toc}{section}{R}

\begin{entry}{儿}{r5}{2}{⼉}
  \definition{suf.}{sufixo diminutivo não silábico | final retroflexo}
  \seeref{儿}{er2}
  \seeref{儿}{ren2}
\end{entry}

\begin{entry}{然}{ran2}{12}{⽕}
  \definition{conj.}{mas | no entanto}
\end{entry}

\begin{entry}{然而}{ran2'er2}{12,6}{⽕、⽽}[HSK 4]
  \definition{conj.}{ainda; mas; contudo; todavia; usado no início de uma frase para indicar uma transição; para indicar uma transição, geralmente é precedido por uma conjunção como 虽然 para indicar concessão}
  \seealsoref{虽然}{sui1 ran2}
\end{entry}

\begin{entry}{然后}{ran2hou4}{12,6}{⽕、⼝}[HSK 2]
  \definition{conj.}{então; depois disso; posteriormente; indica que algo segue após uma ação ou situação}
\end{entry}

\begin{entry}{燃料}{ran2 liao4}{16,10}{⽕、⽃}[HSK 4]
  \definition{s.}{combustível; carburante; substâncias que podem gerar calor e energia luminosa quando queimadas podem ser divididas em três tipos de acordo com sua forma: combustível sólido (como carvão, carvão vegetal, madeira), combustível líquido (como gasolina, querosene) e combustível gasoso (como gás de carvão, biogás); também se refere a substâncias que podem gerar energia nuclear, como urânio, plutônio, etc.}
\end{entry}

\begin{entry}{燃烧}{ran2shao1}{16,10}{⽕、⽕}[HSK 4]
  \definition{s.}{combustão | flama}
  \definition{v.}{queimar; acender | arder; inflamar; ferver; metáfora para as emoções de uma pessoa serem muito fortes, como um fogo ardente}
\end{entry}

\begin{entry}{染}{ran3}{9}{⽊}[HSK 5]
  \definition*{s.}{sobrenome Ran}
  \definition{s.}{soja fermentada e temperada em forma de pasta}
  \definition{v.}{tingir; pintar | pegar (uma doença); cair em (um mau hábito, etc.) | sujar; contaminar | pegar (contrair) (uma doença) | adquirir (um mau hábito, etc.); contaminar}
\end{entry}

\begin{entry}{壤}{rang3}{20}{⼟}
  \definition{s.}{solo | terra | (literário) a terra (em contraste com o céu 天)}
\end{entry}

\begin{entry}{让}{rang4}{5}{⾔}[HSK 2]
  \definition*{s.}{sobrenome Rang}
  \definition{prep.}{em uma frase passiva para introduzir o executor da ação | de acordo com; em conformidade com; à luz de; com base em; usado para expressar a opinião subjetiva de alguém}
  \definition{v.}{ceder; recuar; render-se; desistir; admitir | convidar; oferecer | deixar; permitir; fazer | deixar alguém ter algo por um preço justo | ser inferior a; não ser tão bom quanto | ceder; afastar-se | expressar desejos | esquivar-se; evitar; fugir | Usado antes de 我们, indica uma ordem ou sugestão para que todos façam algo juntos}
  \seealsoref{我们}{wo3men5}
\end{entry}

\begin{entry}{让步}{rang4bu4}{5,7}{⾔、⽌}
  \definition{v.+compl.}{fazer uma concessão | entregar | desistir | comprometer}
\end{entry}

\begin{entry}{绕}{rao4}{9}{⽷}[HSK 5]
  \definition*{s.}{sobrenome Rao}
  \definition{v.}{enrolar; bobinar; rebobinar | mover-se em círculo; girar; revolver | fazer um desvio; contornar; dar a volta | confundir; desorientar}
\end{entry}

\begin{entry}{热}{re4}{10}{⽕}[HSK 1]
  \definition{adj.}{quente; temperatura elevada | ardente; caloroso; profundamente afetuoso | ansioso; invejoso; descreve inveja e desejo de possuir algo | térmico; altamente radioativo | popular; muito procurado; muito apreciado por muitas pessoas}
  \definition{s.}{calor; energia liberada pelo movimento irregular das moléculas dentro de um objeto | febre; febre alta causada por doença | moda passageira; mania; febre}
  \definition{v.}{aquecer (geralmente se refere a alimentos)}
\end{entry}

\begin{entry}{热爱}{re4'ai4}{10,10}{⽕、⽖}[HSK 3]
  \definition{v.}{amar ardentemente; amar de coração; ter amor profundo por}
\end{entry}

\begin{entry}{热泪盈眶}{re4lei4ying2kuang4}{10,8,9,11}{⽕、⽔、⽫、⽬}
  \definition{expr.}{olhos cheios de lágrimas de emoção | extremamente emocionado}
\end{entry}

\begin{entry}{热量}{re4 liang4}{10,12}{⽕、⾥}[HSK 5]
  \definition{s.}{calor; quantidade de calor; calorias; em física, refere-se à energia transferida entre objetos com temperaturas diferentes, do objeto com temperatura mais alta para o objeto com temperatura mais baixa}
\end{entry}

\begin{entry}{热烈}{re4lie4}{10,10}{⽕、⽕}[HSK 3]
  \definition{adj.}{caloroso; calorosamente; fervoroso; ardente; entusiasmado}
\end{entry}

\begin{entry}{热门}{re4men2}{10,3}{⽕、⾨}[HSK 5]
  \definition{adj.}{popular}
  \definition{s.}{algo que desperta o interesse popular; metáfora para algo que está na moda e recebe a atenção de todos (em contraste com 冷门)}
  \seealsoref{冷门}{leng3men2}
\end{entry}

\begin{entry}{热闹}{re4nao5}{10,8}{⽕、⾾}[HSK 4]
  \definition{adj.}{animado; agitado; movimentado com barulho e excitação; descreve uma cena animada com uma atmosfera calorosa}
  \definition{s.}{uma vista emocionante; uma cena de agitação e excitação; atmosfera acolhedora}
  \definition{v.}{animar; divertir-se}
\end{entry}

\begin{entry}{热情}{re4qing2}{10,11}{⽕、⼼}[HSK 2]
  \definition{adj.}{caloroso; fervoroso; entusiasmado; cordial; descreve sentimentos calorosos por alguém}
  \definition{s.}{entusiasmo; ardor; devoção; calor humano; zelo; sentimentos calorosos}
\end{entry}

\begin{entry}{热心}{re4xin1}{10,4}{⽕、⼼}[HSK 4]
  \definition{adj.}{ardente; sincero; entusiasmado; afetuoso; apaixonado; interessado}
  \definition{v.}{ser entusiasmado com alguma coisa}
\end{entry}

\begin{entry}{热血沸腾}{re4xue4fei4teng2}{10,6,8,13}{⽕、⾎、⽔、⾁}
  \definition{expr.}{ferver o sangue | apaixonar-se}
\end{entry}

\begin{entry}{人}{ren2}{2}{⼈}[HSK 1][Kangxi 9]
  \definition*{s.}{sobrenome Ren}
  \definition[个,名,位]{s.}{homem; pessoa; pessoas; ser humano | todos; cada um; todo mundo | adulto; crescido | uma pessoa envolvida em uma atividade específica | pessoas; outras pessoas | caráter; personalidade; qualidade, caráter ou reputação de uma pessoa | como alguém se sente; estado de saúde de alguém | mão de obra; força de trabalho}
\end{entry}

\begin{entry}{人才}{ren2cai2}{2,3}{⼈、⼿}[HSK 3]
  \definition{adj.}{aparência bonita, elegante}
  \definition[个]{s.}{talento; pessoal qualificado; pessoa com capacidade}
\end{entry}

\begin{entry}{人材}{ren2cai2}{2,7}{⼈、⽊}
  \variantof{人才}
\end{entry}

\begin{entry}{人道}{ren2dao4}{2,12}{⼈、⾡}
  \definition{s.}{solidariedade humana | humanitarismo | humano | a ``maneira humana'', um dos estágios do ciclo de reencarnação (budismo) | relação sexual}
\end{entry}

\begin{entry}{人工}{ren2gong1}{2,3}{⼈、⼯}[HSK 3]
  \definition{adj.}{feito pelo homem; artificial}
  \definition[个]{s.}{trabalho manual; trabalho feito à mão | mão de obra; homem-dia; uma unidade de cálculo da quantidade de trabalho realizado}
\end{entry}

\begin{entry}{人海}{ren2hai3}{2,10}{⼈、⽔}
  \definition{s.}{uma multidão | um mar de pessoas}
\end{entry}

\begin{entry}{人家}{ren2jia1}{2,10}{⼈、⼧}[HSK 4]
  \definition[对]{s.}{lar; família; família do noivo; casa do futuro marido}
  \seeref{人家}{ren2jia5}
\end{entry}

\begin{entry}{人家}{ren2jia5}{2,10}{⼈、⼧}
  \definition{pron.}{outros; uma pessoa ou pessoas diferentes do falante ou ouvinte; refere-se a alguém diferente de si mesmo ou de outra pessoa | certa pessoa ou pessoas (a pessoa ou pessoas mencionadas em um contexto próximo, aproximadamente equivalente ao pronome de terceira pessoa);  refere-se a uma pessoa ou algumas pessoas, com significado semelhante a 他 | eu; mim (usado retoricamente no lugar do primeiro pronome pessoal, muitas vezes expressando descontentamento de forma jocosa; geralmente usado quando se fala com pessoas próximas, para significar 自己, usado principamente por meninas)}
  \seeref{人家}{ren2jia1}
  \seealsoref{他}{ta1}
  \seealsoref{自己}{zi4ji3}
\end{entry}

\begin{entry}{人间}{ren2jian1}{2,7}{⼈、⾨}[HSK 5]
  \definition{s.}{o mundo humano | o Mundo; a Terra}
\end{entry}

\begin{entry}{人口}{ren2kou3}{2,3}{⼈、⼝}[HSK 2]
  \definition[个,群]{s.}{população; o número total de pessoas que vivem em uma determinada região durante um determinado período de tempo | número de membros da família; o número total de pessoas em uma família | pessoas; público; população; referência geral a pessoas | rumores do povo; referindo-se à opinião pública}
\end{entry}

\begin{entry}{人类}{ren2lei4}{2,9}{⼈、⽶}[HSK 3]
  \definition[种]{s.}{humano; humanidade; raça humana}
\end{entry}

\begin{entry}{人力}{ren2 li4}{2,2}{⼈、⼒}[HSK 5]
  \definition{s.}{mão de obra; trabalho manual; força de trabalho}
\end{entry}

\begin{entry}{人们}{ren2 men5}{2,5}{⼈、⼈}[HSK 2]
  \definition{s.}{homens; pessoas; o público; referindo-se a muitas pessoas; todos}
\end{entry}

\begin{entry}{人民}{ren2 min2}{2,5}{⼈、⽒}[HSK 3]
  \definition[群,批,个]{s.}{o povo}
\end{entry}

\begin{entry}{人民币}{ren2min2bi4}{2,5,4}{⼈、⽒、⼱}[HSK 3]
  \definition*[块,张,元]{s.}{Renminbi (RMB); Yuan Chinês (CYN); nome da moeda chinesa}
\end{entry}

\begin{entry}{人权}{ren2quan2}{2,6}{⼈、⽊}
  \definition*{s.}{Direitos Humanos}
  \seealsoref{人权法}{ren2quan2fa3}
\end{entry}

\begin{entry}{人权法}{ren2quan2fa3}{2,6,8}{⼈、⽊、⽔}
  \definition*{s.}{Direitos Humanos}
  \seealsoref{人权}{ren2quan2}
\end{entry}

\begin{entry}{人群}{ren2 qun2}{2,13}{⼈、⽺}[HSK 3]
  \definition{s.}{multidão; ajuntamento; torpel; aglomeração; um grupo de pessoas}
\end{entry}

\begin{entry}{人生}{ren2sheng1}{2,5}{⼈、⽣}[HSK 3]
  \definition{s.}{vida (tempo de alguém na Terra)}
\end{entry}

\begin{entry}{人士}{ren2shi4}{2,3}{⼈、⼠}[HSK 5]
  \definition{s.}{pessoa; figura; personalidade; figura pública; pessoas com certa influência social}
\end{entry}

\begin{entry}{人数}{ren2 shu4}{2,13}{⼈、⽁}[HSK 2]
  \definition{s.}{número de pessoas; significa o número total de pessoas, uma quantidade de pessoas; normalmente, usa-se números para fazer estatísticas específicas, mas às vezes também se usa um intervalo aproximado para fazer estimativas}
\end{entry}

\begin{entry}{人物}{ren2wu4}{2,8}{⼈、⽜}[HSK 5]
  \definition[个,位,名]{s.}{personagem; personagens criados em obras literárias e artísticas | figura; personalidade; homem influente; refere-se a pessoas com grande talento e status; também se refere a pessoas com certas características ou que são representativas em algum aspecto | pintura figurativa; um tipo de pintura tradicional chinesa com personagens como tema}
\end{entry}

\begin{entry}{人像}{ren2xiang4}{2,13}{⼈、⼈}
  \definition{s.}{``retrato'' de uma pessoa (esboço, foto, escultura, etc.)}
\end{entry}

\begin{entry}{人行道}{ren2xing2dao4}{2,6,12}{⼈、⾏、⾡}
  \definition{s.}{calçada}
\end{entry}

\begin{entry}{人鱼}{ren2yu2}{2,8}{⼈、⿂}
  \definition{s.}{sereia | peixe-boi | salamandra gigante}
\end{entry}

\begin{entry}{人员}{ren2yuan2}{2,7}{⼈、⼝}[HSK 3]
  \definition[个,位,名]{s.}{funcionários | pessoal}
\end{entry}

\begin{entry}{儿}{ren2}{2}{⼉}
  \definition{s.}{pessoa, radical em caracteres chineses}
  \variantof{人}
  \seeref{儿}{er2}
  \seeref{儿}{r5}
\end{entry}

\begin{entry}{忍}{ren3}{7}{⼼}[HSK 5]
  \definition{v.}{suportar; aguentar; tolerar; aturar | ter coragem para; ser insensível o suficiente para; ser capaz de endurecer o coração e fazer coisas que não se devem fazer por uma questão de razão}
\end{entry}

\begin{entry}{忍不住}{ren3bu5zhu4}{7,4,7}{⼼、⼀、⼈}[HSK 5]
  \definition{v.}{incapaz de suportar; não conseguir evitar fazer algo; não conseguir se controlar}
\end{entry}

\begin{entry}{忍耐}{ren3nai4}{7,9}{⼼、⽽}
  \definition{s.}{paciência | resistência}
  \definition{v.}{suportar | resistir | exercer paciência}
\end{entry}

\begin{entry}{忍受}{ren3shou4}{7,8}{⼼、⼜}[HSK 5]
  \definition{v.}{suportar; sofrer; aguentar; tolerar; suportar com dificuldade o sofrimento, as dificuldades e as adversidades da vida}
\end{entry}

\begin{entry}{认}{ren4}{4}{⾔}[HSK 5]
  \definition{v.}{reconhecer; saber; distinguir; identificar | estabelecer uma determinada relação com; adotar | admitir; reconhecer; assumir | comprometer-se a fazer algo | (frequentemente seguido por 了) aceitar como inevitável; resignar-se}
  \seealsoref{了}{le5}
\end{entry}

\begin{entry}{认出}{ren4 chu1}{4,5}{⾔、⼐}[HSK 3]
  \definition{v.}{reconhecer; decifrar; identificar}
\end{entry}

\begin{entry}{认得}{ren4 de5}{4,11}{⾔、⼻}[HSK 3]
  \definition{v.}{saber; reconhecer}
\end{entry}

\begin{entry}{认定}{ren4ding4}{4,8}{⾔、⼧}[HSK 5]
  \definition{v.}{afirmar; manter; acreditar firmemente; considerar com certeza | decidir-se por algo; confirmar; chegar a uma conclusão afirmativa}
\end{entry}

\begin{entry}{认可}{ren4ke3}{4,5}{⾔、⼝}[HSK 3]
  \definition{v.}{aceitar; aprovar; confirmar; dar força legal a | permitir; concordar}
\end{entry}

\begin{entry}{认识}{ren4shi5}{4,7}{⾔、⾔}[HSK 1]
  \definition[份]{s.}{cognição; conhecimento; compreensão; refere-se à reflexão da mente humana sobre o mundo objetivo}
  \definition{v.}{saber; compreender; reconhecer}
\end{entry}

\begin{entry}{认为}{ren4wei2}{4,4}{⾔、⼂}[HSK 2]
  \definition{v.}{pensar; considerar; manter; julgar; formar uma opinião sobre uma pessoa ou coisa, fazer um julgamento}
\end{entry}

\begin{entry}{认真}{ren4zhen1}{4,10}{⾔、⼗}[HSK 1]
  \definition{adj.}{sério; sério e meticuloso}
  \definition{adv.}{seriamente}
  \definition{v.}{levar algo a sério; considerar como verdadeiro; levar a sério}
\end{entry}

\begin{entry}{任}{ren4}{6}{⼈}[HSK 3]
  \definition{clas.}{número de vezes que serviu em uma posição}
  \definition{conj.}{não importa (como, o que, etc.)}
  \definition{s.}{correio oficial; escritório}
  \definition{v.}{nomear; designar | assumir um posto; assumir um emprego | deixar; permitir; dar rédea solta a}
\end{entry}

\begin{entry}{任何}{ren4he2}{6,7}{⼈、⼈}[HSK 3]
  \definition{pron.}{qualquer; qualquer que seja; o que for}
\end{entry}

\begin{entry}{任务}{ren4wu5}{6,5}{⼈、⼒}[HSK 3]
  \definition[项,个]{s.}{tarefa; dever; missão; designação}
\end{entry}

\begin{entry}{扔}{reng1}{5}{⼿}[HSK 5]
  \definition{v.}{arremessar; lançar; atirar; jogar | esquecer; jogar fora; descartar | colocar casualmente; deixar as pessoas ou as coisas de lado, não se importar}
\end{entry}

\begin{entry}{扔掉}{reng1diao4}{5,11}{⼿、⼿}
  \definition{v.}{jogar fora}
\end{entry}

\begin{entry}{扔弃}{reng1qi4}{5,7}{⼿、⼶}
  \definition{v.}{abandonar | descartar | jogar fora}
\end{entry}

\begin{entry}{扔下}{reng1xia4}{5,3}{⼿、⼀}
  \definition{v.}{lançar (uma bomba) | derrubar}
\end{entry}

\begin{entry}{仍}{reng2}{4}{⼈}[HSK 3]
  \definition*{s.}{sobrenome Reng}
  \definition{adv.}{ainda}
  \definition{conj.}{por isso}
  \definition{v.}{permanecer}
\end{entry}

\begin{entry}{仍旧}{reng2jiu4}{4,5}{⼈、⽇}[HSK 5]
  \definition{adv.}{ainda; ainda assim; contudo}
  \definition{v.}{permanecer igual; continuar sendo}
\end{entry}

\begin{entry}{仍然}{reng2ran2}{4,12}{⼈、⽕}[HSK 3]
  \definition{adv.}{ainda; como antes}
\end{entry}

\begin{entry}{日}{ri4}{4}{⽇}[HSK 1][Kangxi 72]
  \definition*{s.}{Japão, abreviação de 日本}
  \definition{clas.}{usado para contar o número de dias}
  \definition{s.}{sol | dia (em oposição a 夜); período diurno | diariamente; todos os dias; a cada dia que passa | um dia específico; dia especial | tempo; refere-se a um período de tempo | dia; uma rotação da Terra}
  \seealsoref{日本}{ri4ben3}
  \seealsoref{夜}{ye4}
\end{entry}

\begin{entry}{日报}{ri4 bao4}{4,7}{⽇、⼿}[HSK 2]
  \definition[份,种]{s.}{diário; jornais diários; jornal publicado todas as manhãs}
\end{entry}

\begin{entry}{日本}{ri4ben3}{4,5}{⽇、⽊}
  \definition*{s.}{Japão}
\end{entry}

\begin{entry}{日本人}{ri4ben3ren2}{4,5,2}{⽇、⽊、⼈}
  \definition{s.}{japonês | pessoa ou povo do Japão}
\end{entry}

\begin{entry}{日常}{ri4chang2}{4,11}{⽇、⼱}[HSK 3]
  \definition{adv.}{usual; diário; cotidiano; dia a dia}
\end{entry}

\begin{entry}{日出}{ri4chu1}{4,5}{⽇、⼐}
  \definition{s.}{nascer do sol}
  \seealsoref{夕阳}{xi1yang2}
\end{entry}

\begin{entry}{日光灯}{ri4guang1deng1}{4,6,6}{⽇、⼉、⽕}
  \definition{s.}{lâmpada fluorescente}
\end{entry}

\begin{entry}{日记}{ri4ji4}{4,5}{⽇、⾔}[HSK 4]
  \definition[本,篇,册]{s.}{diário; artigo que registra eventos e pensamentos diários}
\end{entry}

\begin{entry}{日历}{ri4li4}{4,4}{⽇、⼚}[HSK 4]
  \definition[张,本]{s.}{caledário; livro com o ano, mês, dia, semana, termo solar, aniversário, etc. registrados, um livro por ano, uma página por dia, aberto diariamente}
\end{entry}

\begin{entry}{日期}{ri4qi1}{4,12}{⽇、⽉}[HSK 1]
  \definition[个,段]{s.}{data; a data ou período específico em que algo aconteceu}
\end{entry}

\begin{entry}{日心说}{ri4 xin1 shuo1}{4,4,9}{⽇、⼼、⾔}
  \definition{s.}{teoria heliocêntrica | a teoria de que o sol está no centro do universo}
\end{entry}

\begin{entry}{日子}{ri4zi5}{4,3}{⽇、⼦}[HSK 2]
  \definition[个,段,些,番]{s.}{dia; data; referência a uma data específica | dias; tempo; referência ao número de dias e horas | vida; subsistência; refere-se à vida ou ao sustento}
\end{entry}

\begin{entry}{容貌}{rong2mao4}{10,14}{⼧、⾘}
  \definition{s.}{aparência | aspecto | características}
\end{entry}

\begin{entry}{容易}{rong2yi4}{10,8}{⼧、⽇}[HSK 3]
  \definition{adj.}{fácil; simples | provável; responsável; apto}
\end{entry}

\begin{entry}{柔软}{rou2ruan3}{9,8}{⽊、⾞}
  \definition{adj.}{macio | suave}
\end{entry}

\begin{entry}{揉}{rou2}{12}{⼿}
  \definition{v.}{amassar | massagear | esfregar}
\end{entry}

\begin{entry}{揉碎}{rou2sui4}{12,13}{⼿、⽯}
  \definition{v.}{esmagar | desintegrar-se em pedaços}
\end{entry}

\begin{entry}{肉}{rou4}{6}{⾁}[HSK 1][Kangxi 130]
  \definition{adj.}{não crocante; mole | lento (em movimento); preguiçoso | carnal; erótico}
  \definition[块]{s.}{carne (especialmente carne de porco) | carne | polpa (da fruta)}
\end{entry}

\begin{entry}{肉桂}{rou4gui4}{6,10}{⾁、⽊}
  \definition{s.}{canela}
  \seealsoref{官桂}{guan1gui4}
\end{entry}

\begin{entry}{如}{ru2}{6}{⼥}
  \definition{conj.}{por exemplo}
\end{entry}

\begin{entry}{如此}{ru2 ci3}{6,6}{⼥、⽌}[HSK 5]
  \definition{adv.}{assim; tal; dessa forma; dessa maneira; refere-se a uma situação mencionada anteriormente, equivalente a 这样}
  \seealsoref{这样}{zhe4 yang4}
\end{entry}

\begin{entry}{如果}{ru2guo3}{6,8}{⼥、⽊}[HSK 2]
  \definition{conj.}{se; no caso de; na eventualidade de; supondo que; para expressar suposições, pode-se usar 要是 na linguagem falada.}
  \seealsoref{要是}{yao4shi5}
\end{entry}

\begin{entry}{如何}{ru2he2}{6,7}{⼥、⼈}[HSK 3]
  \definition{adv.}{como?; o que?}
\end{entry}

\begin{entry}{如画}{ru2hua4}{6,8}{⼥、⽥}
  \definition{adj.}{pitoresco}
\end{entry}

\begin{entry}{如今}{ru2jin1}{6,4}{⼥、⼈}[HSK 4]
  \definition{s.}{agora; hoje em dia; atualmente; no presente}
\end{entry}

\begin{entry}{如同}{ru2 tong2}{6,6}{⼥、⼝}[HSK 5]
  \definition{v.}{parecer que. usado principalmente em metáforas}
\end{entry}

\begin{entry}{如下}{ru2 xia4}{6,3}{⼥、⼀}[HSK 5]
  \definition{adv.}{como descrito ou listado abaixo; conforme segue; conforme abaixo}
\end{entry}

\begin{entry}{儒教}{ru2jiao4}{16,11}{⼈、⽁}
  \definition*{s.}{Confucionismo}
\end{entry}

\begin{entry}{乳房}{ru3fang2}{8,8}{⼄、⼾}
  \definition{s.}{seio | mama | úbere}
\end{entry}

\begin{entry}{辱骂}{ru3ma4}{10,9}{⾠、⾺}
  \definition{v.}{insultar | abusar}
\end{entry}

\begin{entry}{入党}{ru4dang3}{2,10}{⼊、⼉}
  \definition{v.}{ingressar em um partido político (especialmente o partido comunista)}
\end{entry}

\begin{entry}{入境}{ru4jing4}{2,14}{⼊、⼟}
  \definition{s.}{imigração}
  \definition{v.+compl.}{entrar em um país | imigrar}
\end{entry}

\begin{entry}{入口}{ru4kou3}{2,3}{⼊、⼝}[HSK 2]
  \definition[个]{s.}{entrada; entrada em locais, edifícios, estradas, etc., através de portões ou portas}
  \definition{v.}{entrar na boca | importar; mercadorias estrangeiras importadas, às vezes também se refere a mercadorias de outras regiões importadas para esta região}
\end{entry}

\begin{entry}{入门}{ru4 men2}{2,3}{⼊、⾨}[HSK 5]
  \definition{s.}{(geralmente em títulos de livros) curso básico; manual introdutório | ABC; guia; refere-se a leituras básicas; conhecimentos básicos}
  \definition{v.+compl.}{ultrapassar o limiar; aprender os rudimentos de um assunto | aprender o ABC de; ser introduzido a um assunto; aprender o básico}
\end{entry}

\begin{entry}{入乡随俗}{ru4xiang1-sui2su2}{2,3,11,9}{⼊、⼄、⾩、⼈}
  \definition{expr.}{Em roma, faça como os romanos!}
\end{entry}

\begin{entry}{软}{ruan3}{8}{⾞}[HSK 5]
  \definition*{s.}{sobrenome Ruan}
  \definition{adj.}{macio; flexível; maleável; maleável (oposto de 硬) | suave; brando; delicado | fraco; débil | de baixa qualidade, capacidade, etc. | facilmente movido (ou influenciado) | de maneira suave (ou gentil) | indulgente; tolerante | maleável; flexível | fácil de se emocionar ou abalar}
  \seealsoref{硬}{ying4}
\end{entry}

\begin{entry}{软件}{ruan3jian4}{8,6}{⾞、⼈}[HSK 5]
  \definition[款,个]{s.}{(computador) \emph{software}; programas de computador, procedimentos, regras e quaisquer arquivos, documentos e dados relacionados à operação do sistema de computador}
\end{entry}

\begin{entry}{锐}{rui4}{12}{⾦}
  \definition*{s.}{sobrenome Rui}
  \definition{adj.}{afiado; aguçado (oposto a 钝) | agudo; perspicaz | rápido; ágil; veloz}
  \definition{adv.}{rapidamente; de ​​repente}
  \definition{s.}{vigor; espírito de luta | armas afiadas}
  \seealsoref{钝}{dun4}
\end{entry}

\begin{entry}{若}{ruo4}{8}{⾋}
  \definition*{s.}{sobrenome Ruo}
  \definition{adv.}{como se; como se fosse; usado antes do verbo para indicar que o que foi dito é mais ou menos assim, equivalente a 好像}
  \definition{conj.}{se; usado na primeira parte de uma frase composta, expressa uma relação hipotética, equivalente a 如果}
  \definition{pron.}{você; referir-se ao interlocutor como 你 ou 你的}
  \definition{v.}{parecer}
  \seealsoref{好像}{hao3xiang4}
  \seealsoref{你}{ni3}
  \seealsoref{你的}{ni3 de5}
  \seealsoref{如果}{ru2guo3}
\end{entry}

\begin{entry}{弱}{ruo4}{10}{⼸}[HSK 4]
  \definition{adj.}{fraco; debilitado | jovem | inferior; pior | colocado depois de uma fração ou decimal para indicar que é um pouco menor que esse número}
  \definition{v.}{perder (através da morte)}
\end{entry}

%%%%% EOF %%%%%


%%%
%%% S
%%%

\section*{S}\addcontentsline{toc}{section}{S}

\begin{entry}{撒旦}{sa1dan4}{15,5}
  \definition*{s.}{Satã}
\end{entry}

\begin{entry}{撒旦主义}{sa1dan4 zhu3yi4}{15,5,5,3}
  \definition*{s.}{Satanismo}
\end{entry}

\begin{entry}{撒但}{sa1dan4}{15,7}
  \variantof{撒旦}
\end{entry}

\begin{entry}{洒水}{sa3shui3}{9,4}
  \definition{v.}{borrifar}
\end{entry}

\begin{entry}{飒飒}{sa4sa4}{9,9}
  \definition{s.}{o farfalhar | sussurro | murmúrio (do vento nas árvores, o mar, etc.)}
\end{entry}

\begin{entry}{赛}{sai4}{14}[Radical 貝]
  \definition{s.}{competição}
  \definition{v.}{competir | superar | destacar-se}
\end{entry}

\begin{entry}{赛车}{sai4che1}{14,4}
  \definition{s.}{corrida de automóvel | corrida de bicicleta | carro de corrida}
\end{entry}

\begin{entry}{三}{san1}{3}[Radical 一]
  \definition*{s.}{sobrenome San}
  \definition{num.}{três; 3}
\end{entry}

\begin{entry}{三角}{san1jiao3}{3,7}
  \definition{s.}{triângulo}
\end{entry}

\begin{entry}{三角恋爱}{san1jiao3lian4'ai4}{3,7,10,10}
  \definition{s.}{triângulo amoroso}
\end{entry}

\begin{entry}{三轮车}{san1lun2che1}{3,8,4}
  \definition{s.}{triciclo}
\end{entry}

\begin{entry}{三明治}{san1ming2zhi4}{3,8,8}
  \definition{s.}{(empréstimo linguístico) sanduíche}
\end{entry}

\begin{entry}{散}{san3}{12}[Radical 攴]
  \definition{adj.}{disseminado | dispersado | solto}
  \definition{s.}{medicamento em pó}
  \definition{v.}{soltar-se | desmoronar}
  \seeref{散}{san4}
\end{entry}

\begin{entry}{散}{san4}{12}[Radical 攴]
  \definition{v.}{terminar (uma reunião, etc.) | dispersar | disseminar | dissipar}
  \seeref{散}{san3}
\end{entry}

\begin{entry}{散步}{san4bu4}{12,7}
  \definition{v.+compl.}{dar um passeio | passear | dar uma caminhada}
\end{entry}

\begin{entry}{散心}{san4xin1}{12,4}
  \definition{v.+compl.}{aliviar o tédio | desfrutar de uma diversão | estar despreocupado}
\end{entry}

\begin{entry}{丧钟}{sang1zhong1}{8,9}
  \definition{s.}{sentença de morte}
\end{entry}

\begin{entry}{桑}{sang1}{10}[Radical 木]
  \definition*{s.}{sobrenome Sang}
  \definition{s.}{amoreira}
\end{entry}

\begin{entry}{桑巴舞}{sang1ba1wu3}{10,4,14}
  \definition{s.}{samba}
\end{entry}

\begin{entry}{桑树}{sang1shu4}{10,9}
  \definition{s.}{amoreira, suas folhas são utilizadas para alimentar bichos-da-seda}
\end{entry}

\begin{entry}{骚乱}{sao1luan4}{12,7}
  \definition{s.}{rebelião | perturbação | tumulto}
  \definition{v.}{criar um distúrbio}
\end{entry}

\begin{entry}{扫兴}{sao3xing4}{6,6}
  \definition{v.+compl.}{sentir-se decepcionado | entristecer alguém}
\end{entry}

\begin{entry}{嫂子}{sao3zi5}{12,3}
  \definition{s.}{esposa do irmão mais velho}
\end{entry}

\begin{entry}{色狼}{se4lang2}{6,10}
  \definition*{s.}{Sátiro}
  \definition{adj.}{depravado | tarado}
\end{entry}

\begin{entry}{森林}{sen1lin2}{12,8}
  \definition{s.}{floresta}
\end{entry}

\begin{entry}{僧}{seng1}{14}[Radical 人]
  \definition{s.}{monge Budista, abreviação de 僧伽}
  \seeref{僧伽}{seng1qie2}
\end{entry}

\begin{entry}{僧伽}{seng1qie2}{14,7}
  \definition{s.}{sangha ou sanga (Budismo) | a comunidade monástica | monge}
\end{entry}

\begin{entry}{杀气}{sha1qi4}{6,4}
  \definition{s.}{espírito assassino | aura de morte}
  \definition{v.}{desabafar a raiva de alguém}
\end{entry}

\begin{entry}{沙}{sha1}{7}[Radical 水]
  \definition*{s.}{sobrenome Sha}
  \definition[粒]{s.}{areia | cascalho | grânulo | pó}
\end{entry}

\begin{entry}{沙漠}{sha1mo4}{7,13}
  \definition[个]{s.}{deserto}
\end{entry}

\begin{entry}{沙特}{sha1te4}{7,10}
  \definition*{s.}{Saudita | abreviação de 沙特阿拉伯}
  \seeref{沙特阿拉伯}{sha1te4 a1la1bo2}
\end{entry}

\begin{entry}{沙特阿拉伯}{sha1te4 a1la1bo2}{7,10,7,8,7}
  \definition*{s.}{Arábia Saudita}
\end{entry}

\begin{entry}{沙鱼}{sha1yu2}{7,8}
  \variantof{鲨鱼}
\end{entry}

\begin{entry}{刹}{sha1}{8}[Radical 刀]
  \definition{v.}{frear}
  \seeref{刹}{cha4}
\end{entry}

\begin{entry}{砂}{sha1}{9}[Radical 石]
  \variantof{沙}
\end{entry}

\begin{entry}{莎莎舞}{sha1sha1wu3}{10,10,14}
  \definition{s.}{salsa (dança)}
\end{entry}

\begin{entry}{鲨鱼}{sha1yu2}{15,8}
  \definition{s.}{tubarão}
\end{entry}

\begin{entry}{啥}{sha2}{11}[Radical 口]
  \definition{adv.}{Equivalente a 什么 (dialeto), também pronunciado como \dpy{sha4}}
  \seeref{什么}{shen2me5}
\end{entry}

\begin{entry}{傻瓜}{sha3gua1}{13,5}
  \definition{adj.}{tolo | burro | simplório | idiota}
  \definition{v.}{enganar | iludir | lograr}
\end{entry}

\begin{entry}{傻眼}{sha3yan3}{13,11}
  \definition{adj.}{estupefato | atordoado}
\end{entry}

\begin{entry}{啥}{sha4}{11}[Radical 口]
  \definition{adv.}{Equivalente a 什么 (dialeto), também pronunciado como \dpy{sha2}}
  \seeref{啥}{sha2}
\end{entry}

\begin{entry}{晒干}{shai4gan1}{10,3}
  \definition{v.}{secar ao sol}
\end{entry}

\begin{entry}{山}{shan1}{3}[Radical 山][Kangxi 46]
  \definition*{s.}{sobrenome Shan}
  \definition[座]{s.}{montanha | monte | qualquer coisa que se assemelhe a uma montanha}
\end{entry}

\begin{entry}{山顶}{shan1ding3}{3,8}
  \definition{s.}{cume da montanha}
\end{entry}

\begin{entry}{山东}{shan1dong1}{3,5}
  \definition*{s.}{Shandong}
\end{entry}

\begin{entry}{山谷}{shan1gu3}{3,7}
  \definition{s.}{vale | ravina}
\end{entry}

\begin{entry}{山区}{shan1qu1}{3,4}
  \definition[个]{s.}{área montanhosa | montanhas}
\end{entry}

\begin{entry}{山体}{shan1ti3}{3,7}
  \definition{s.}{forma de uma montanha}
\end{entry}

\begin{entry}{山羊}{shan1yang2}{3,6}
  \definition{s.}{cabra | (ginástica) cavalo de salto de pequeno porte}
\end{entry}

\begin{entry}{山寨}{shan1zhai4}{3,14}
  \definition{s.}{fortaleza fortificada da vila | fortaleza da montanha (especialmente de bandidos) | falsificação | imitação | (fig.) pechincha}
\end{entry}

\begin{entry}{闪存盘}{shan3cun2pan2}{5,6,11}
  \definition{s.}{unidade de memória \emph{USB} | cartão de memória}
  \seealsoref{优盘}{you1pan2}
\end{entry}

\begin{entry}{单}{shan4}{8}[Radical 十]
  \definition*{s.}{sobrenome Shan}
  \seeref{单}{chan2}
  \seeref{单}{dan1}
\end{entry}

\begin{entry}{扇子}{shan4zi5}{10,3}
  \definition[把]{s.}{leque | abano | abanador}
\end{entry}

\begin{entry}{善意}{shan4yi4}{12,13}
  \definition{s.}{boa vontade | benevolência | bondade}
\end{entry}

\begin{entry}{禅}{shan4}{12}[Radical 示]
  \definition{v.}{abdicar}
  \seeref{禅}{chan2}
\end{entry}

\begin{entry}{擅自}{shan4zi4}{16,6}
  \definition{adv.}{sem permissão ou autorização | por iniciativa própria}
\end{entry}

\begin{entry}{伤}{shang1}{6}[Radical 人]
  \definition{s.}{ferida | ferimento}
  \definition{v.}{ferir | ferir-se}
\end{entry}

\begin{entry}{伤心}{shang1xin1}{6,4}
  \definition{v.}{sofrer | ter o coração partido | sentir-se profundamente magoado}
\end{entry}

\begin{entry}{汤}{shang1}{6}[Radical 水]
  \definition{s.}{correnteza forte}
  \seeref{汤}{tang1}
\end{entry}

\begin{entry}{商店}{shang1dian4}{11,8}
  \definition[家,个]{s.}{loja}
\end{entry}

\begin{entry}{商贸}{shang1mao4}{11,9}
  \definition{s.}{comércio}
\end{entry}

\begin{entry}{上声}{shang3sheng1}{3,7}
  \definition{s.}{tom descendente e ascendente | terceiro tom no mandarim moderno}
\end{entry}

\begin{entry}{赏赐}{shang3ci4}{12,12}
  \definition{s.}{recompensa | prêmio}
  \definition{v.}{recompensar | premiar}
\end{entry}

\begin{entry}{赏心悦目}{shang3xin1yue4mu4}{12,4,10,5}
  \definition{expr.}{"Aquece o coração e encanta os olhos."}
\end{entry}

\begin{entry}{上}{shang4}{3}[Radical 一]
  \definition{adv.}{acima | em cima | sobre}
  \definition{v.}{subir | entrar em | frequentar (aula ou universidade)}
\end{entry}

\begin{entry}{上班}{shang4ban1}{3,10}
  \definition{v.+compl.}{ir para o trabalho | ir para o emprego | estar de plantão}
\end{entry}

\begin{entry}{上边}{shang4bian5}{3,5}
  \definition{adv.}{acima de | parte de cima | por cima}
\end{entry}

\begin{entry}{上车}{shang4che1}{3,4}
  \definition{v.}{entrar (em ônibus, trem, carro, etc.)}
\end{entry}

\begin{entry}{上当}{shang4dang4}{3,6}
  \definition{v.+compl.}{ser enganado | morder uma isca | ser manipulado | ser joguete nas mãos de alguém}
\end{entry}

\begin{entry}{上访}{shang4fang3}{3,6}
  \definition{v.}{buscar uma audiência com superiores (especialmente funcionários do governo) para fazer uma petição por algo}
\end{entry}

\begin{entry}{上古}{shang4gu3}{3,5}
  \definition{s.}{o passado distante | tempos antigos | antiguidade}
\end{entry}

\begin{entry}{上海}{shang4hai3}{3,10}
  \definition*{s.}{Shangai (Xangai)}
\end{entry}

\begin{entry}{上课}{shang4ke4}{3,10}
  \definition{v.}{assistir à aula | ir para a aula | ir dar uma aula}
\end{entry}

\begin{entry}{上来}{shang4lai2}{3,7}
  \definition{v.}{subir (para a minha localização)}
\end{entry}

\begin{entry}{上面}{shang4mian4}{3,9}
  \definition{adv.}{acima de | parte de cima | por cima}
\end{entry}

\begin{entry}{上坡路}{shang4po1lu4}{3,8,13}
  \definition{s.}{aclive | progresso | (fig.) tendência ascendente}
\end{entry}

\begin{entry}{上去}{shang4qu4}{3,5}
  \definition{v.}{subir (a partir da minha localização)}
\end{entry}

\begin{entry}{上网}{shang4wang3}{3,6}
  \definition{v.}{conectar à \emph{Internet} | fazer \emph{upload} | ficar \emph{online}}
\end{entry}

\begin{entry}{上午}{shang4wu3}{3,4}
  \definition{adv.}{manhã | de manhã | período antes do meio-dia}
\end{entry}

\begin{entry}{上询}{shang4 xun2}{3,8}
  \definition{adv.}{primeira dezena do mês}
\end{entry}

\begin{entry}{上演}{shang4yan3}{3,14}
  \definition{s.}{exibição | encenação}
  \definition{v.}{exibir (um filme) | encenar (uma peça)}
\end{entry}

\begin{entry}{尚且}{shang4qie3}{8,5}
  \definition{conj.}{até | ainda}
\end{entry}

\begin{entry}{尚且……何况……}{shang4qie3 he2kuang4}{8,5,7,7}
  \definition{conj.}{ainda que\dots, \dots}
\end{entry}

\begin{entry}{烧}{shao1}{10}[Radical 火]
  \definition{s.}{febre}
  \definition{v.}{queimar | cozinhar | cozer | assar | aquecer | ferver (chá, água, etc.) | ter febre | (coloquial) deixar as coisas subirem à cabeça}
\end{entry}

\begin{entry}{烧烤}{shao1kao3}{10,10}
  \definition{s.}{churrasco}
  \definition{v.}{assar}
\end{entry}

\begin{entry}{稍}{shao1}{12}[Radical 禾]
  \definition{adv.}{um pouco | ligeiramente | em vez de}
\end{entry}

\begin{entry}{稍微}{shao1wei1}{12,13}
  \definition{adv.}{um pouco}
\end{entry}

\begin{entry}{少}{shao3}{4}[Radical 小]
  \definition{adj.}{pouco, poucos}
  \definition{v.}{sentir falta | faltar | parar (de fazer algo)}
  \seeref{少}{shao4}
\end{entry}

\begin{entry}{少}{shao4}{4}[Radical 小]
  \definition{s.}{jovem}
  \seeref{少}{shao3}
\end{entry}

\begin{entry}{舌头}{she2tou5}{6,5}
  \definition[个]{s.}{língua | soldado inimigo capturado com o propósito de extrair informações}
\end{entry}

\begin{entry}{蛇}{she2}{11}[Radical 虫]
  \definition[条]{s.}{cobra | serpente}
\end{entry}

\begin{entry}{设备}{she4bei4}{6,8}
  \definition[个]{s.}{equipamento | instalações}
\end{entry}

\begin{entry}{设计}{she4ji4}{6,4}
  \definition{s.}{projeto | planejamento}
  \definition{v.}{projetar | planejar}
\end{entry}

\begin{entry}{射}{she4}{10}[Radical 寸]
  \definition{v.}{atirar | lançar}
\end{entry}

\begin{entry}{摄氏}{she4shi4}{13,4}
  \definition{s.}{graus Celsius (°C), centígrado}
\end{entry}

\begin{entry}{谁}{shei2}{10}[Radical 言]
  \definition{interr.}{quem?}
  \seeref{谁}{shui2}
\end{entry}

\begin{entry}{身体}{shen1ti3}{7,7}
  \definition[具,个]{s.}{em pessoa | saúde de alguém | o corpo}
\end{entry}

\begin{entry}{身体能力}{shen1ti3 neng2li4}{7,7,10,2}
  \definition{s.}{habilidade física}
\end{entry}

\begin{entry}{身体乳}{shen1ti3 ru3}{7,7,8}
  \definition{s.}{loção corporal}
\end{entry}

\begin{entry}{身亡}{shen1wang2}{7,3}
  \definition{v.}{morrer}
\end{entry}

\begin{entry}{深}{shen1}{11}[Radical 水]
  \definition{adj.}{profundo}
\end{entry}

\begin{entry}{深厚}{shen1hou4}{11,9}
  \definition{adj.}{profundo}
\end{entry}

\begin{entry}{深深}{shen1shen1}{11,11}
  \definition{adj.}{profundo}
  \definition{adv.}{profundamente}
\end{entry}

\begin{entry}{深夜}{shen1ye4}{11,8}
  \definition{adv.}{tarde da noite}
\end{entry}

\begin{entry}{什么}{shen2me5}{4,3}
  \definition{pron.}{que? | o que?}
  \definition{pron.}{algo | qualquer coisa}
\end{entry}

\begin{entry}{什么时候}{shen2me5shi2hou5}{4,3,7,10}
  \definition{adv.}{quando? | a que horas?}
\end{entry}

\begin{entry}{神}{shen2}{9}[Radical 示]
  \definition*{s.}{Deus}
  \definition{s.}{deus | divindade}
\end{entry}

\begin{entry}{神话}{shen2hua4}{9,8}
  \definition{s.}{lenda | conto de fadas | mito | mitologia}
\end{entry}

\begin{entry}{神经}{shen2jing1}{9,8}
  \definition{adj.}{desequilibrado | louco | insano}
  \definition{s.}{nervo}
\end{entry}

\begin{entry}{神经病的}{shen2jing1bing4de5}{9,8,10,8}
  \definition{adj.}{neurótico}
\end{entry}

\begin{entry}{神经病学}{shen2jing1bing4xue2}{9,8,10,8}
  \definition{s.}{neurologia}
\end{entry}

\begin{entry}{神明}{shen2ming2}{9,8}
  \definition{s.}{divindades | deuses}
\end{entry}

\begin{entry}{神奇}{shen2qi2}{9,8}
  \definition{adj.}{mágico | místico | milagroso}
  \definition{s.}{mágica | milagre}
\end{entry}

\begin{entry}{神器}{shen2qi4}{9,16}
  \definition{s.}{objeto mágico | objeto simbólico do poder imperial | arma fina | ferramenta muito útil}
\end{entry}

\begin{entry}{神兽}{shen2shou4}{9,11}
  \definition{s.}{animal mitológico | fera}
\end{entry}

\begin{entry}{甚而}{shen4'er2}{9,6}
  \definition{conj.}{(ir) tão longe quanto | tanto que}
\end{entry}

\begin{entry}{甚或}{shen4huo4}{9,8}
  \definition{conj.}{(ir) tão longe quanto | tanto que}
\end{entry}

\begin{entry}{甚至}{shen4zhi4}{9,6}
  \definition{conj.}{(ir) tão longe quanto | tanto que | mesmo (na medida em que)}
\end{entry}

\begin{entry}{升起}{sheng1qi3}{4,10}
  \definition{v.}{levantar | içar | subir}
\end{entry}

\begin{entry}{生}{sheng1}{5}[Radical 生][Kangxi 100]
  \definition{adj.}{vida | estudante | cru | não cozido}
  \definition{v.}{nascer | dar a luz | crescer}
\end{entry}

\begin{entry}{生菜}{sheng1cai4}{5,11}
  \definition{s.}{alface}
\end{entry}

\begin{entry}{生的}{sheng1de5}{5,8}
  \definition{conj.}{para evitar isso | para que\dots não\dots}
\end{entry}

\begin{entry}{生活}{sheng1huo2}{5,9}
  \definition[道]{s.}{vida | atividade | meios de subsistência}
  \definition{v.}{viver}
\end{entry}

\begin{entry}{生活垃圾}{sheng1huo2la1ji1}{5,9,8,6}
  \definition{s.}{lixo doméstico}
\end{entry}

\begin{entry}{生活型}{sheng1huo2 xing2}{5,9,9}
  \definition{s.}{forma de vida}
\end{entry}

\begin{entry}{生理}{sheng1li3}{5,11}
  \definition{adj.}{fisiológico}
  \definition{s.}{fisiologia}
\end{entry}

\begin{entry}{生气}{sheng1qi4}{5,4}
  \definition{s.}{vitalidade | vigor}
  \definition{v.+compl.}{irritar-se | zangar-se | ofender-se | ficar com raiva}
\end{entry}

\begin{entry}{生日}{sheng1ri4}{5,4}
  \definition[个]{s.}{aniversário}
\end{entry}

\begin{entry}{生态}{sheng1tai4}{5,8}
  \definition{adj.}{ecológico}
  \definition{s.}{ecologia}
\end{entry}

\begin{entry}{生物}{sheng1wu4}{5,8}
  \definition{adj.}{biológico}
  \definition{s.}{biologia (disciplina) | organismo | ser vivo}
\end{entry}

\begin{entry}{生意}{sheng1yi4}{5,13}
  \definition{s.}{força vital | vitalidade}
  \seealsoref{生意}{sheng1yi5}
\end{entry}

\begin{entry}{生意}{sheng1yi5}{5,13}
  \definition{s.}{negócio}
  \seealsoref{生意}{sheng1yi4}
\end{entry}

\begin{entry}{生鱼片}{sheng1yu2pian4}{5,8,4}
  \definition{s.}{fatias de peixe cru, \emph{sashimi}}
\end{entry}

\begin{entry}{生长}{sheng1zhang3}{5,4}
  \definition{v.}{crescer | amadurecer | ser criado}
\end{entry}

\begin{entry}{声明}{sheng1ming2}{7,8}
  \definition[项,份]{s.}{declaração}
  \definition{v.}{declarar}
\end{entry}

\begin{entry}{绳子}{sheng2zi5}{11,3}
  \definition[条]{s.}{corda | cordão}
\end{entry}

\begin{entry}{省}{sheng3}{9}[Radical 目]
  \definition{s.}{província | capital provincial}
  \definition{v.}{economizar | guardar | ser frugal | omitir | excluir | deixar de fora}
  \seeref{省}{xing3}
\end{entry}

\begin{entry}{省城}{sheng3cheng2}{9,9}
  \definition{s.}{capital da província}
\end{entry}

\begin{entry}{省会}{sheng3hui4}{9,6}
  \definition{s.}{capital da província}
\end{entry}

\begin{entry}{省俭}{sheng3jian3}{9,9}
  \definition{s.}{econômico | frugal}
  \definition{v.}{economizar}
\end{entry}

\begin{entry}{省力}{sheng3li4}{9,2}
  \definition{v.}{economizar esforço ou trabalho}
\end{entry}

\begin{entry}{省钱}{sheng3qian2}{9,10}
  \definition{v.}{economizar dinheiro}
\end{entry}

\begin{entry}{省却}{sheng3que4}{9,7}
  \definition{v.}{livrar-se (para economizar espaço) | salvar}
\end{entry}

\begin{entry}{省心}{sheng3xin1}{9,4}
  \definition{adj.}{despreocupado}
  \definition{v.}{ser poupado de preocupações | despreocupar-se}
\end{entry}

\begin{entry}{省长}{sheng3zhang3}{9,4}
  \definition*{s.}{Governador | governador de uma província}
\end{entry}

\begin{entry}{圣诞节}{sheng4dan4jie2}{5,8,5}
  \definition*{s.}{Natal}
\end{entry}

\begin{entry}{圣地}{sheng4di4}{5,6}
  \definition{s.}{terra santa (de uma religião) | lugar sagrado | santuário | cidade santa (como Jerusalém, Meca, etc.) | centro de interesse histórico}
\end{entry}

\begin{entry}{胜利}{sheng4li4}{9,7}
  \definition[个]{s.}{vitória}
\end{entry}

\begin{entry}{胜算}{sheng4suan4}{9,14}
  \definition{s.}{probabilidade de sucesso | estratégia que garante o sucesso}
  \definition{v.}{ter certeza do sucesso}
\end{entry}

\begin{entry}{盛宴}{sheng4yan4}{11,10}
  \definition{s.}{celebração}
\end{entry}

\begin{entry}{失落}{shi1luo4}{5,12}
  \definition{s.}{frustração | decepção | perda}
  \definition{v.}{perder (algo) | cair (algo) | sentir uma sensação de perda}
\end{entry}

\begin{entry}{失眠}{shi1mian2}{5,10}
  \definition{s.}{insônia}
  \definition{v.}{ter insônia}
\end{entry}

\begin{entry}{失去}{shi1qu4}{5,5}
  \definition{v.}{perder}
\end{entry}

\begin{entry}{失望}{shi1wang4}{5,11}
  \definition{adj.}{desapontado}
  \definition{v.}{perder a esperança | desesperar}
\end{entry}

\begin{entry}{失意}{shi1yi4}{5,13}
  \definition{adj.}{desapontado | frustrado}
\end{entry}

\begin{entry}{师}{shi1}{6}[Radical 巾]
  \definition*{s.}{sobrenome Shi}
  \definition{s.}{professor | mestre | especialista | modelo | divisão do exército}
  \definition{v.}{despachar tropas}
\end{entry}

\begin{entry}{师傅}{shi1fu5}{6,12}
  \definition[个,位,名]{s.}{técnico | mestre-trabalhador | forma respeitosa de tratamento para homens mais velhos}
\end{entry}

\begin{entry}{诗词}{shi1ci2}{8,7}
  \definition{s.}{verso}
\end{entry}

\begin{entry}{诗句}{shi1ju4}{8,5}
  \definition[行]{s.}{verso | versículo}
\end{entry}

\begin{entry}{诗意}{shi1yi4}{8,13}
  \definition{adj.}{poético}
  \definition{s.}{poesia}
\end{entry}

\begin{entry}{十}{shi2}{2}[Radical 十][Kangxi 24]
  \definition{num.}{dez; 10 | dezena}
\end{entry}

\begin{entry}{十分}{shi2fen1}{2,4}
  \definition{adv.}{muito | extremamente | totalmente | absolutamente}
\end{entry}

\begin{entry}{十足}{shi2zu2}{2,7}
  \definition{adj.}{amplo | completo | cento por cento | tom puro (de alguma cor)}
\end{entry}

\begin{entry}{时差}{shi2cha1}{7,9}
  \definition{s.}{diferença de tempo | \emph{jet lag}}
\end{entry}

\begin{entry}{时光}{shi2guang1}{7,6}
  \definition{s.}{tempo | época | período de tempo}
\end{entry}

\begin{entry}{时候}{shi2hou5}{7,10}
  \definition{adv.}{quando?}
  \definition{s.}{duração de tempo | momento | período | tempo}
\end{entry}

\begin{entry}{时间}{shi2jian1}{7,7}
  \definition{s.}{(conceito de, duração de, um ponto no) tempo}
\end{entry}

\begin{entry}{时刻}{shi2ke4}{7,8}
  \definition{adv.}{constantemente | sempre}
  \definition[个,段]{s.}{tempo | conjuntura | momento | período de tempo}
\end{entry}

\begin{entry}{时时}{shi2shi2}{7,7}
  \definition{adv.}{muitas vezes | constantemente}
\end{entry}

\begin{entry}{实力}{shi2li4}{8,2}
  \definition{s.}{força}
\end{entry}

\begin{entry}{实现}{shi2xian4}{8,8}
  \definition{v.}{alcançar | implementar | constatar}
\end{entry}

\begin{entry}{实在}{shi2zai4}{8,6}
  \definition{adv.}{realmente | verdadeiramente | de fato | na verdade}
\end{entry}

\begin{entry}{食品}{shi2pin3}{9,9}
  \definition[种]{s.}{comida | alimento | produtos alimentícios | provisões}
\end{entry}

\begin{entry}{食堂}{shi2tang2}{9,11}
  \definition[个,间]{s.}{sala de jantar}
\end{entry}

\begin{entry}{食物}{shi2wu4}{9,8}
  \definition[种]{s.}{comida}
\end{entry}

\begin{entry}{屎}{shi3}{9}[Radical 尸]
  \definition{s.}{fezes | excrementos | (forma ligada) secreção (do ouvido, olho, etc.)}
\end{entry}

\begin{entry}{世代}{shi4dai4}{5,5}
  \definition{adv.}{por muitas gerações, eras}
  \definition{s.}{geração | era}
\end{entry}

\begin{entry}{世界}{shi4jie4}{5,9}
  \definition[个]{s.}{mundo}
\end{entry}

\begin{entry}{世界杯}{shi4jie4bei1}{5,9,8}
  \definition*{s.}{Copa do Mundo}
\end{entry}

\begin{entry}{世锦赛}{shi4jin3sai4}{5,13,14}
  \definition*{s.}{Campeonato Mundial}
\end{entry}

\begin{entry}{市场}{shi4chang3}{5,6}
  \definition{s.}{mercado (também no abstrato)}
\end{entry}

\begin{entry}{市区}{shi4qu1}{5,4}
  \definition{s.}{centro da cidade | distrito urbano}
\end{entry}

\begin{entry}{市中心}{shi4zhong1xin1}{5,4,4}
  \definition{s.}{centro da cidade}
\end{entry}

\begin{entry}{式}{shi4}{6}[Radical 弋]
  \definition{s.}{tipo | forma | padrão | estilo}
\end{entry}

\begin{entry}{事}{shi4}{8}[Radical 亅]
  \definition[件,桩,回]{s.}{coisa | assunto | item | matéria | coisa de trabalho | caso}
\end{entry}

\begin{entry}{事故}{shi4gu4}{8,9}
  \definition[桩,起,次]{s.}{acidente}
\end{entry}

\begin{entry}{事儿}{shi4r5}{8,2}
  \definition[件,桩]{s.}{o emprego | negócio | afazeres | assunto que precisa ser resolvido | matéria}
\end{entry}

\begin{entry}{视角}{shi4jiao3}{8,7}
  \definition{s.}{ângulo do qual se observa um objeto | (figurativo) perspectiva, ponto de vista, quadro de referência | (cinematografia) ângulo da câmera | (percepção visual) ângulo visual (o ângulo que um objeto visto subtende no olho) | (fotografia) ângulo de visão}
\end{entry}

\begin{entry}{视频}{shi4pin2}{8,13}
  \definition{s.}{vídeo}
\end{entry}

\begin{entry}{试}{shi4}{8}[Radical 言]
  \definition{s.}{exame | experimento | prova | teste}
  \definition{v.}{experimentar | provar | testar}
\end{entry}

\begin{entry}{室}{shi4}{9}[Radical 宀]
  \definition*{s.}{sobrenome Shi}
  \definition{s.}{família ou clã | cova | cômodo | bainha | unidade de trabalho}
\end{entry}

\begin{entry}{是}{shi4}{9}[Radical 日]
  \definition{adj.}{correto | certo | verdadeiro | (reconhecimento respeitoso de um comando) muito bem}
  \definition{adv.}{(advérbio para afirmação enfática)}
  \definition{v.}{ser (somente seguido por substantivos)}
\end{entry}

\begin{entry}{是的}{shi4de5}{9,8}
  \definition{adv.}{sim | está certo}
\end{entry}

\begin{entry}{适合}{shi4he2}{9,6}
  \definition{v.}{servir (uma roupa) | adequar}
\end{entry}

\begin{entry}{收}{shou1}{6}[Radical 攴]
  \definition{expr.}{aos cuidados de (usado na linha de endereço após o nome)}
  \definition{v.}{receber | aceitar | coletar | colher | guardar}
\end{entry}

\begin{entry}{收到}{shou1dao4}{6,8}
  \definition{v.}{receber}
\end{entry}

\begin{entry}{收据}{shou1ju4}{6,11}
  \definition[张]{s.}{recibo | \emph{voucher}}
\end{entry}

\begin{entry}{收看}{shou1kan4}{6,9}
  \definition{v.}{assistir (a um programa de TV)}
\end{entry}

\begin{entry}{收敛}{shou1lian3}{6,11}
  \definition{v.}{diminuir | desaparecer | fazer desaparecer | exercer restrição | conter (alegria, arrogância, etc.) | constringir | (matemática) convergir}
\end{entry}

\begin{entry}{收买}{shou1mai3}{6,6}
  \definition{v.}{subornar | comprar}
\end{entry}

\begin{entry}{手}{shou3}{4}[Radical 手][Kangxi 64]
  \definition{adj.}{conveniente}
  \definition{clas.}{de habilidade}
  \definition[双,只]{s.}{mão | pessoa envolvida em certos tipos de trabalho | pessoa qualificada para certos tipos de trabalho}
  \definition{v.}{segurar (formal)}
\end{entry}

\begin{entry}{手臂}{shou3bi4}{4,17}
  \definition{s.}{braço}
\end{entry}

\begin{entry}{手边}{shou3bian1}{4,5}
  \definition{adv.}{à mão | na mão}
\end{entry}

\begin{entry}{手工}{shou3gong1}{4,3}
  \definition{s.}{trabalho manual | artesanato}
\end{entry}

\begin{entry}{手工艺人}{shou3gong1 yi4ren2}{4,3,4,2}
  \definition{s.}{artesão}
\end{entry}

\begin{entry}{手机}{shou3ji1}{4,6}
  \definition[部,支]{s.}{telefone celular ou móvel}
\end{entry}

\begin{entry}{手刹}{shou3sha1}{4,8}
  \definition{s.}{freio de mão}
\end{entry}

\begin{entry}{手指}{shou3zhi3}{4,9}
  \definition[个,只]{s.}{dedo}
\end{entry}

\begin{entry}{守门员}{shou3men2yuan2}{6,3,7}
  \definition{s.}{goleiro}
\end{entry}

\begin{entry}{首席执行官}{shou3xi2 zhi2xing2 guan1}{9,10,6,6,8}
  \definition{s.}{\emph{chief executive officer}, CEO}
\end{entry}

\begin{entry}{首相}{shou3xiang4}{9,9}
  \definition*{s.}{Primeiro-Ministro (Japão, UK, etc.)}
\end{entry}

\begin{entry}{掱}{shou3}{12}[Radical 手]
  \variantof{手}
\end{entry}

\begin{entry}{受到}{shou4dao4}{8,8}
  \definition{v.}{receber (elogio, educação, punição, etc.) | ser elogiado, educado, punido, etc.}
\end{entry}

\begin{entry}{受得了}{shou4de5liao3}{8,11,2}
  \definition{v.}{suportar | aguentar}
\end{entry}

\begin{entry}{受限}{shou4xian4}{8,8}
  \definition{v.}{ser limitado | ser restrito | ser constrangido}
\end{entry}

\begin{entry}{瘦}{shou4}{14}[Radical 疒]
  \definition{adj.}{magro | emagrecido | apertado (roupas) | improdutivo (terras) | magro (carne)}
  \definition{v.}{perder peso}
\end{entry}

\begin{entry}{书}{shu1}{4}[Radical 乙]
  \definition[本,册,部]{s.}{livro | carta | documento}
\end{entry}

\begin{entry}{书店}{shu1dian4}{4,8}
  \definition[家]{s.}{livraria}
\end{entry}

\begin{entry}{书记}{shu1ji5}{4,5}
  \definition{s.}{secretário (chefe de um ramo de um partido socialista ou comunista) | atendente | balconista | escriturário}
\end{entry}

\begin{entry}{舒服}{shu1fu5}{12,8}
  \definition{adj.}{estar confortável | bem disposto | sentir-se bem}
\end{entry}

\begin{entry}{熟练}{shu2lian4}{15,8}
  \definition{adj.}{especializado | proficiente | qualificado | habilidoso}
\end{entry}

\begin{entry}{熟悉}{shu2xi1}{15,11}
  \definition{v.}{conhecer bem | estar familiarizado com}
\end{entry}

\begin{entry}{属}{shu3}{12}[Radical 尸]
  \definition{s.}{categoria | gênero (taxonomia) | familiares | dependentes}
  \definition{v.}{pertencer | subordinar | nascer no ano do signo de (um dos doze animais zodiacais) | provar ser | constituir}
  \seeref{属}{zhu3}
\end{entry}

\begin{entry}{属于}{shu3yu2}{12,3}
  \definition{v.}{ser classificado como | pertencer a | fazer parte de}
\end{entry}

\begin{entry}{暑假}{shu3jia4}{12,11}
  \definition[个]{s.}{férias de verão}
\end{entry}

\begin{entry}{薯}{shu3}{16}[Radical 艸]
  \definition{s.}{batata | inhame}
\end{entry}

\begin{entry}{束}{shu4}{7}[Radical 木]
  \definition*{s.}{sobrenome Shu}
  \definition{clas.}{para cachos, feixes, feixes de luz, etc.}
  \definition{s.}{monte | pacote | maço | feixe | cacho}
  \definition{v.}{vincular | controlar}
\end{entry}

\begin{entry}{束腰}{shu4yao1}{7,13}
  \definition{s.}{cinto | cinta | cinturão}
\end{entry}

\begin{entry}{树}{shu4}{9}[Radical 木]
  \definition[棵]{s.}{árvore}
  \definition{v.}{cultivar}
\end{entry}

\begin{entry}{树莓}{shu4mei2}{9,10}
  \definition{s.}{framboesa}
\end{entry}

\begin{entry}{树木}{shu4mu4}{9,4}
  \definition{s.}{árvore}
\end{entry}

\begin{entry}{树叶}{shu4ye4}{9,5}
  \definition{s.}{folhas de árvores}
\end{entry}

\begin{entry}{数学}{shu4xue2}{13,8}
  \definition{s.}{matemática (disciplina)}
\end{entry}

\begin{entry}{刷子}{shua1zi5}{8,3}
  \definition[把]{s.}{pincel | escova | escovão}
\end{entry}

\begin{entry}{耍}{shua3}{9}[Radical 而]
  \definition{v.}{brincar com | empunhar | agir (legal, calmo, tranquilo, descolado, etc.) | exibir (uma habilidade, o temperamento de alguém, etc.)}
\end{entry}

\begin{entry}{耍赖}{shua3lai4}{9,13}
  \definition{v.}{agir descaradamente | recusar -se a reconhecer que alguém perdeu o jogo ou fez uma promessa, etc. | agir como um idiota | agir como se algo nunca tivesse acontecido}
\end{entry}

\begin{entry}{摔}{shuai1}{14}[Radical 手]
  \definition{v.}{cair | cair e quebrar | partir}
\end{entry}

\begin{entry}{帅}{shuai4}{5}[Radical 巾]
  \definition*{s.}{sobrenome Shuai}
  \definition{adj.}{elegante | agradável à vista | gracioso | inteligente}
  \definition{interj.}{Legal!}
  \definition{s.}{comandante em chefe}
\end{entry}

\begin{entry}{双}{shuang1}{4}[Radical 又]
  \definition*{s.}{sobrenome Shuang}
  \definition{s.}{dobro | par | dupla | ambos | número par}
\end{entry}

\begin{entry}{双层床}{shuang1ceng2chuang2}{4,7,7}
  \definition{s.}{beliche}
\end{entry}

\begin{entry}{双打}{shuang1da3}{4,5}
  \definition[场]{s.}{duplas (em esportes)}
\end{entry}

\begin{entry}{双方同意}{shuang1fang1tong2yi4}{4,4,6,13}
  \definition{s.}{acordo bilateral}
\end{entry}

\begin{entry}{霜}{shuang1}{17}[Radical 雨]
  \definition{s.}{geada | pó branco ou creme espalhado por uma superfície | glacê | creme de pele}
\end{entry}

\begin{entry}{谁}{shui2}{10}[Radical 言]
  \definition{interr.}{quem?}
  \seeref{谁}{shei2}
\end{entry}

\begin{entry}{水}{shui3}{4}[Radical 水][Kangxi 85]
  \definition*{s.}{sobrenome Shui}
  \definition{clas.}{para número de lavagens}
  \definition{s.}{água | líquido | encargos ou receitas adicionais}
\end{entry}

\begin{entry}{水边}{shui3bian1}{4,5}
  \definition{s.}{beira d'água | beira-mar | costa (de mar, lago ou rio)}
\end{entry}

\begin{entry}{水波}{shui3bo1}{4,8}
  \definition{s.}{ondulação (na água) | onda}
\end{entry}

\begin{entry}{水槽}{shui3cao2}{4,15}
  \definition{s.}{pia (de cozinha)}
\end{entry}

\begin{entry}{水果}{shui3guo3}{4,8}
  \definition[个]{s.}{fruta}
\end{entry}

\begin{entry}{水饺}{shui3jiao3}{4,9}
  \definition{s.}{\emph{dumplings} | pastéis chineses cozidos}
\end{entry}

\begin{entry}{水灵}{shui3ling2}{4,7}
  \definition{adj.}{cheio de vida (sobre uma pessoa, etc.) | úmido e brilhante (sobre os olhos) | fresco (sobre frutas, etc.) | brilhante | aparência saudável}
\end{entry}

\begin{entry}{水路}{shui3lu4}{4,13}
  \definition{s.}{hidrovia}
\end{entry}

\begin{entry}{水培}{shui3pei2}{4,11}
  \definition{v.}{cultivar plantas hidroponicamente}
\end{entry}

\begin{entry}{水平}{shui3ping2}{4,5}
  \definition{s.}{nível (de realização, etc.) | padrão | nível horizontal}
\end{entry}

\begin{entry}{水平尺}{shui3ping2chi3}{4,5,4}
  \definition{s.}{nível espiritual}
\end{entry}

\begin{entry}{水平度}{shui3ping2 du4}{4,5,9}
  \definition{s.}{nivelamento}
\end{entry}

\begin{entry}{水平面}{shui3ping2mian4}{4,5,9}
  \definition{s.}{plano horizontal | nível-da-água | superfície horizontal}
\end{entry}

\begin{entry}{水平视差}{shui3ping2 shi4cha1}{4,5,8,9}
  \definition{s.}{paralaxe horizontal}
\end{entry}

\begin{entry}{水平仪}{shui3ping2yi2}{4,5,5}
  \definition{s.}{nível (dispositivo para determinar horizontal) | nível espiritual | nível de topógrafo}
\end{entry}

\begin{entry}{水平以下}{shui3ping2 yi3xia4}{4,5,4,3}
  \definition{s.}{sub-nível}
\end{entry}

\begin{entry}{水平轴}{shui3ping2zhou2}{4,5,9}
  \definition{s.}{eixo horizontal}
\end{entry}

\begin{entry}{水瓶}{shui3 ping2}{4,10}
  \definition{s.}{garrada de água}
\end{entry}

\begin{entry}{水豚}{shui3tun2}{4,11}
  \definition{s.}{capivara}
\end{entry}

\begin{entry}{水污染}{shui3wu1ran3}{4,6,9}
  \definition{s.}{poluição da água}
\end{entry}

\begin{entry}{说}{shui4}{9}[Radical 言]
  \definition{v.}{persuadir}
  \seeref{说}{shuo1}
\end{entry}

\begin{entry}{税}{shui4}{12}[Radical 禾]
  \definition{s.}{taxas | impostos}
\end{entry}

\begin{entry}{睡觉}{shui4jiao4}{13,9}
  \definition{v.+compl.}{ir para a cama | dormir | deitar-se}
\end{entry}

\begin{entry}{睡懒觉}{shui4lan3jiao4}{13,16,9}
  \definition{v.}{levantar-se tarde | passar o tempo a dormir}
\end{entry}

\begin{entry}{睡衣}{shui4yi1}{13,6}
  \definition{s.}{pijamas | roupas de dormir}
\end{entry}

\begin{entry}{顺}{shun4}{9}[Radical 頁]
  \definition{adj.}{correr bem | favorável}
\end{entry}

\begin{entry}{顺便}{shun4bian4}{9,9}
  \definition{adv.}{convenientemente | de passagem | sem muito esforço extra}
\end{entry}

\begin{entry}{顺畅}{shun4chang4}{9,8}
  \definition{adj.}{liso e sem obstáculos | fluente}
\end{entry}

\begin{entry}{顺从}{shun4cong2}{9,4}
  \definition{v.}{obedecer | submeter-se}
\end{entry}

\begin{entry}{顺当}{shun4dang5}{9,6}
  \definition{adv.}{suavemente}
\end{entry}

\begin{entry}{顺耳}{shun4'er3}{9,6}
  \definition{adj.}{agradável ao ouvido}
\end{entry}

\begin{entry}{顺境}{shun4jing4}{9,14}
  \definition{s.}{circunstâncias favoráveis}
\end{entry}

\begin{entry}{顺利}{shun4li4}{9,7}
  \definition{adv.}{suavemente | sem problemas}
\end{entry}

\begin{entry}{顺水}{shun4shui3}{9,4}
  \definition{v.}{ir com o fluxo}
\end{entry}

\begin{entry}{顺心}{shun4xin1}{9,4}
  \definition{adj.}{satisfatório | satisfeito}
\end{entry}

\begin{entry}{顺叙}{shun4xu4}{9,9}
  \definition{s.}{narrativa cronológica}
\end{entry}

\begin{entry}{顺延}{shun4yan2}{9,6}
  \definition{v.}{adiar | procrastinar}
\end{entry}

\begin{entry}{顺眼}{shun4yan3}{9,11}
  \definition{adj.}{agradável aos olhos}
\end{entry}

\begin{entry}{顺嘴}{shun4zui3}{9,16}
  \definition{v.}{deixar escapar (sem pensar) | ler suavemente (texto) | adequar-se  ao gosto (comida)}
\end{entry}

\begin{entry}{说}{shuo1}{9}[Radical 言]
  \definition{s.}{uma teoria (normalmente o último caractere, como em 日心说, teoria heliocêntrica)}
  \definition{v.}{falar | dizer | explicar | contar}
  \seeref{说}{shui4}
\end{entry}

\begin{entry}{说好}{shuo1hao3}{9,6}
  \definition{v.}{chegar a um acordo | concluir negociações}
\end{entry}

\begin{entry}{说谎}{shuo1huang3}{9,11}
  \definition{v.+compl.}{mentir | contar uma mentira}
\end{entry}

\begin{entry}{说理}{shuo1li3}{9,11}
  \definition{v.}{racionalizar | discutir logicamente}
\end{entry}

\begin{entry}{说完}{shuo1-wan2}{9,7}
  \definition{expr.}{acabar/terminar palavras}
\end{entry}

\begin{entry}{丝}{si1}{5}[Radical 一]
  \definition{adj.}{filiforme | delgado como um fio | que se assemelha a um fio}
  \definition{clas.}{um traço (de fumaça, etc.) | um pouquinho, etc.}
  \definition{s.}{seda | (cozinha) pedaços ou tiras de julienne, tiras cortadas finas}
\end{entry}

\begin{entry}{司机}{si1ji1}{5,6}
  \definition{s.}{condutor | motorista | chofer}
\end{entry}

\begin{entry}{私人}{si1ren2}{7,2}
  \definition{adj.}{privado | interpessoal}
  \definition[些]{s.}{alguém com quem se tem um relacionamento pessoal próximo}
\end{entry}

\begin{entry}{私人信件}{si1ren2 xin4jian4}{7,2,9,6}
  \definition{s.}{carta pessoal}
\end{entry}

\begin{entry}{私人钥匙}{si1ren2yao4shi5}{7,2,9,11}
  \definition{s.}{(criptografia) chave privada}
\end{entry}

\begin{entry}{私人诊所}{si1ren2 zhen3suo3}{7,2,7,8}
  \definition[些]{s.}{clínica privada}
\end{entry}

\begin{entry}{私生活}{si1sheng1huo2}{7,5,9}
  \definition{s.}{vida privada}
\end{entry}

\begin{entry}{私自}{si1zi4}{7,6}
  \definition{adj.}{privado | pessoal}
  \definition{adv.}{secretamente | sem aprovação explícita}
\end{entry}

\begin{entry}{思想}{si1xiang3}{9,13}
  \definition[个]{s.}{pensamento | ideia | ideologia}
\end{entry}

\begin{entry}{斯巴达}{si1ba1da2}{12,4,6}
  \definition*{s.}{Esparta}
\end{entry}

\begin{entry}{死}{si3}{6}[Radical 歹]
  \definition{adj.}{maldito | intransitável | inflexível | rígido | intransponível}
  \definition{adv.}{extremamente}
  \definition{v.}{morrer | falecer}
\end{entry}

\begin{entry}{死亡}{si3wang2}{6,3}
  \definition{s.}{morte}
  \definition{v.}{morrer}
\end{entry}

\begin{entry}{四}{si4}{5}[Radical 囗]
  \definition{num.}{quatro; 4}
\end{entry}

\begin{entry}{四川}{si4chuan1}{5,3}
  \definition*{s.}{Sichuan}
\end{entry}

\begin{entry}{四季分明}{si4ji4-fen1ming2}{5,8,4,8}
  \definition{expr.}{as quatro estações são muito distintas}
\end{entry}

\begin{entry}{四季如春}{si4ji4-ru2chun1}{5,8,6,9}
  \definition{expr.}{é primavera todo o ano | clima favorável durante todo o ano | quatro estações como a primavera}
\end{entry}

\begin{entry}{似曾相识}{si4ceng2xiang1shi2}{6,12,9,7}
  \definition{s.}{\emph{déjà vu} (a experiência de ver exatamente a mesma situação pela segunda vez) | situação aparentemente familiar}
\end{entry}

\begin{entry}{寺}{si4}{6}[Radical 寸]
  \definition{s.}{Templo Budista | Mesquita}
\end{entry}

\begin{entry}{寺庙}{si4miao4}{6,8}
  \definition{s.}{templo | mosteiro | santuário}
\end{entry}

\begin{entry}{松木}{song1mu4}{8,4}
  \definition{s.}{pinheiro}
\end{entry}

\begin{entry}{宋}{song4}{7}[Radical 宀]
  \definition*{s.}{sobrenome Song}
  \definition{s.}{Dinastia Song (960-1279) | Song das dinastias do sul (420-479)}
\end{entry}

\begin{entry}{送}{song4}{9}[Radical 辵]
  \definition{v.}{distribuir | entregar | dar | oferecer (alguma coisa como presente) | enviar | remeter}
\end{entry}

\begin{entry}{㮸}{song4}{14}
  \variantof{送}
\end{entry}

\begin{entry}{苏格兰}{su1ge2lan2}{7,10,5}
  \definition*{s.}{Escócia}
\end{entry}

\begin{entry}{宿舍}{su4she4}{11,8}
  \definition[间]{s.}{dormitório | quarto de dormir | hostel}
\end{entry}

\begin{entry}{痠}{suan1}{12}
  \definition{v.}{doer | estar dolorido}
  \variantof{酸}
\end{entry}

\begin{entry}{酸}{suan1}{14}[Radical 酉]
  \definition{adj.}{ácido | avinagrado}
\end{entry}

\begin{entry}{酸辣汤}{suan1la4tang1}{14,14,6}
  \definition{s.}{sopa avinagrada e picante (prato)}
\end{entry}

\begin{entry}{算了}{suan4le5}{14,2}
  \definition{v.}{deixar | deixe estar | deixe passar | esqueça isso}
\end{entry}

\begin{entry}{算命}{suan4ming4}{14,8}
  \definition{s.}{cartomante}
  \definition{v.}{ler a sorte | fazer advinhações}
\end{entry}

\begin{entry}{尿}{sui1}{7}[Radical 尸]
  \definition{s.}{(coloquial) urina}
  \seeref{尿}{niao4}
\end{entry}

\begin{entry}{虽}{sui1}{9}[Radical 虫]
  \definition{conj.}{no entanto | embora | mesmo se/embora}
\end{entry}

\begin{entry}{虽然}{sui1ran2}{9,12}
  \definition{conj.}{embora (frequentemente usado correlativamente com 可是, 但是, etc)}
  \seealsoref{但是}{dan4shi4}
  \seealsoref{可是}{ke3shi4}
\end{entry}

\begin{entry}{随便}{sui2bian4}{11,9}
  \definition{adj.}{à vontade | como queira | como desejar | casual | negligente | devasso}
  \definition{adv.}{aleatoriamente}
\end{entry}

\begin{entry}{随处}{sui2chu4}{11,5}
  \definition{adv.}{em qualquer lugar}
\end{entry}

\begin{entry}{随地}{sui2di4}{11,6}
  \definition{adv.}{qualquer lugar | todo lugar}
\end{entry}

\begin{entry}{随机存取存储器}{sui2ji1cun2qu3cun2chu3qi4}{11,6,6,8,6,12,16}
  \definition{s.}{RAM (\emph{random access memory})}
  \seealsoref{内存}{nei4cun2}
  \seealsoref{随机存取记忆体}{sui2ji1cun2qu3ji4yi4ti3}
\end{entry}

\begin{entry}{随机存取记忆体}{sui2ji1cun2qu3ji4yi4ti3}{11,6,6,8,5,4,7}
  \definition{s.}{RAM (\emph{random access memory})}
  \seealsoref{内存}{nei4cun2}
  \seealsoref{随机存取存储器}{sui2ji1cun2qu3cun2chu3qi4}
\end{entry}

\begin{entry}{随时}{sui2shi2}{11,7}
  \definition{adv.}{a qualquer momento | sempre que necessário}
\end{entry}

\begin{entry}{岁}{sui4}{6}[Radical 山]
  \definition{clas.}{para anos (de idade)}
  \definition{s.}{idade | ano (idade ou colheita)}
\end{entry}

\begin{entry}{碎}{sui4}{13}[Radical 石]
  \definition{adj.}{quebrado | fragmentado | espalhado | tagarela}
  \definition{v.}{(transitivo ou intransitivo) quebrar em pedaços, quebrar, desmoronar}
\end{entry}

\begin{entry}{隧道}{sui4dao4}{14,12}
  \definition{s.}{túnel}
\end{entry}

\begin{entry}{孙女}{sun1nv3}{6,3}
  \definition{s.}{filha do filho}
\end{entry}

\begin{entry}{孙武}{sun1wu3}{6,8}
  \definition*{s.}{Sun Wu, também conhecido por Sun Tzu (孙子), general, estrategista e filósofo autor do ``Arte da Guerra'' (孙子兵法)}
  \seeref{孙子}{sun1zi3}
  \seealsoref{孙子兵法}{sun1zi3 bing1fa3}
\end{entry}

\begin{entry}{孙子}{sun1zi3}{6,3}
  \definition*{s.}{Sun Tzu, também conhecido por Sun Wu (孙武), general, estrategista e filósofo autor do ``Arte da Guerra'' (孙子兵法)}
  \seeref{孙武}{sun1wu3}
  \seealsoref{孙子兵法}{sun1zi3 bing1fa3}
\end{entry}

\begin{entry}{孙子兵法}{sun1zi3 bing1fa3}{6,3,7,8}
  \definition*{s.}{``Arte da Guerra'', escrito por Sun Tzu (孫子)}
  \seeref{孙武}{sun1wu3}
  \seeref{孙子}{sun1zi3}
\end{entry}

\begin{entry}{孙子}{sun1zi5}{6,3}
  \definition{s.}{filho do filho}
\end{entry}

\begin{entry}{笋}{sun3}{10}[Radical 竹]
  \definition{s.}{broto de bambu}
\end{entry}

\begin{entry}{缩影卡片}{suo1ying3 ka3pian4}{14,15,5,4}
  \definition{s.}{cartão em miniatura}
\end{entry}

\begin{entry}{所以}{suo3yi3}{8,4}
  \definition{adv.}{portanto | então | como resultado}
  \definition{conj.}{por isso | como resultado | a razão porque}
\end{entry}

\begin{entry}{索性}{suo3xing4}{10,8}
  \definition{adv.}{poderia muito bem | simplesmente | apenas}
\end{entry}

%%%%% EOF %%%%%


%%%
%%% T
%%%

\section*{T}\addcontentsline{toc}{section}{T}

\begin{entry}{T-恤}{[t]-xu4}{0,9}
  \definition{s.}{camiseta | pulôver | suéter}
\end{entry}

\begin{entry}{㐌}{ta1}{5}[Radical 乙]
  \variantof{它}
\end{entry}

\begin{entry}{他}{ta1}{5}[Radical 人][HSK 1]
  \definition{pron.}{ele | se, o, lhe | si, consigo, ele}
  \seeref{怹}{tan1}
\end{entry}

\begin{entry}{他的}{ta1 de5}{5,8}
  \definition{pron.}{dele}
\end{entry}

\begin{entry}{他妈的}{ta1ma1de5}{5,6,8}
  \definition{interj.}{Dane-se! | Foda-se!}
\end{entry}

\begin{entry}{他们}{ta1men5}{5,5}[HSK 1]
  \definition{pron.}{eles | se, os, lhes | si, consigo, eles}
\end{entry}

\begin{entry}{他们的}{ta1men5 de5}{5,5,8}
  \definition{pron.}{deles}
\end{entry}

\begin{entry}{它}{ta1}{5}[Radical 宀][HSK 2]
  \definition{pron.}{ele (para objetos inanimados) | se, o, lhe | si, consigo, eles}
\end{entry}

\begin{entry}{它们}{ta1 men5}{5,5}[HSK 2]
  \definition{pron.}{eles (para objetos inanimados) | se, os, lhes | si, consigo, eles}
\end{entry}

\begin{entry}{她}{ta1}{6}[Radical 女][HSK 1]
  \definition{pron.}{ela | se, a, lhe | si, consigo, ela}
\end{entry}

\begin{entry}{她的}{ta1 de5}{6,8}
  \definition{pron.}{dela}
\end{entry}

\begin{entry}{她们}{ta1men5}{6,5}[HSK 1]
  \definition{pron.}{elas | se, as, lhes | si, consigo, elas}
\end{entry}

\begin{entry}{她们的}{ta1men5 de5}{6,5,8}
  \definition{pron.}{delas}
\end{entry}

\begin{entry}{踏板}{ta4ban3}{15,8}
  \definition{s.}{pedal (em um carro, em um piano, etc.) |  apoio para os pés | estribo}
\end{entry}

\begin{entry}{台}{tai2}{5}[Radical 口]
  \definition*{s.}{sobrenome Tai}
  \definition{clas.}{para aparelhos e máquinas}
  \definition{s.}{estação de transmissão | contador | \emph{help desk} | suporte técnico | plataforma | terraço | tufão}
\end{entry}

\begin{entry}{台风}{tai2feng1}{5,4}
  \definition{s.}{tufão}
\end{entry}

\begin{entry}{台下}{tai2xia4}{5,3}
  \definition{s.}{platéia | fora do palco}
\end{entry}

\begin{entry}{抬杠}{tai2gang4}{8,7}
  \definition{v.+compl.}{discutir pelo prazer em discutir | discutir obstinadamente | brigar}
\end{entry}

\begin{entry}{太}{tai4}{4}[Radical 大][HSK 1]
  \definition{adv.}{excessivamente | demais | muito}
\end{entry}

\begin{entry}{太极拳}{tai4ji2quan2}{4,7,10}
  \definition*{s.}{Tai Chi Chuan, Taiji, T'aichi ou T'aichichuan; forma tradicional de exercício físico ou relaxamento}
\end{entry}

\begin{entry}{太空}{tai4kong1}{4,8}
  \definition{s.}{espaço sideral | espaço exterior}
\end{entry}

\begin{entry}{太平洋}{tai4ping2 yang2}{4,5,9}
  \definition*{s.}{Oceano Pacífico}
\end{entry}

\begin{entry}{太太}{tai4tai5}{4,4}[HSK 2]
  \definition[个,位]{s.}{esposa | madame| mulher casada}
\end{entry}

\begin{entry}{太阳窗}{tai4yang2chuang1}{4,6,12}
  \definition{s.}{teto solar (de veículos)}
\end{entry}

\begin{entry}{太阳灯}{tai4yang2deng1}{4,6,6}
  \definition{s.}{lâmpada solar (com células fotovoltaicas)}
\end{entry}

\begin{entry}{太阳风}{tai4yang2feng1}{4,6,4}
  \definition{s.}{vento solar}
\end{entry}

\begin{entry}{太阳镜}{tai4yang2jing4}{4,6,16}
  \definition{s.}{óculos de sol}
\end{entry}

\begin{entry}{太阳日}{tai4yang2ri4}{4,6,4}
  \definition{s.}{dia solar}
\end{entry}

\begin{entry}{太阳穴}{tai4yang2xue2}{4,6,5}
  \definition{s.}{têmpora (nas laterais da cabeça humana)}
\end{entry}

\begin{entry}{太阳翼}{tai4yang2yi4}{4,6,17}
  \definition{s.}{painel solar}
\end{entry}

\begin{entry}{太阳雨}{tai4yang2yu3}{4,6,8}
  \definition{s.}{banho de sol}
\end{entry}

\begin{entry}{太阳}{tai4yang5}{4,6}[HSK 2]
  \definition[个]{s.}{sol | abreviação de 太阳穴}
  \seeref{太阳穴}{tai4yang2xue2}
\end{entry}

\begin{entry}{态度}{tai4du5}{8,9}[HSK 2]
  \definition[个]{s.}{maneira | comportamento | atitude | atitude | abordagem}
\end{entry}

\begin{entry}{贪婪}{tan1lan2}{8,11}
  \definition{adj.}{avaro | ambicioso | voraz | insaciável}
\end{entry}

\begin{entry}{怹}{tan1}{9}[Radical 心]
  \definition{pron.}{ele, ela (cortês, em oposição a 他)}
  \seeref{他}{ta1}
\end{entry}

\begin{entry}{谈话}{tan2hua4}{10,8}
  \definition[次]{s.}{conversa | fala | papo | declaração}
  \definition{v.+compl.}{conversar | falar | declarar}
\end{entry}

\begin{entry}{谈恋爱}{tan2lian4'ai4}{10,10,10}
  \definition{v.}{namorar | apaixonar-se}
\end{entry}

\begin{entry}{坦克}{tan3ke4}{8,7}
  \definition{s.}{(empréstimo linguístico) tanque (veículo militar)}
\end{entry}

\begin{entry}{探亲}{tan4qin1}{11,9}
  \definition{v.+compl.}{ir para casa para visitar a família}
\end{entry}

\begin{entry}{碳}{tan4}{14}[Radical 石]
  \definition{s.}{carbono (elemento químico)}
\end{entry}

\begin{entry}{汤}{tang1}{6}[Radical 水]
  \definition*{s.}{sobrenome Tang}
  \definition{s.}{sopa | caldo | decocção de ervas medicinais | água quente ou fervente | água em que algo foi fervido}
  \seeref{汤}{shang1}
\end{entry}

\begin{entry}{唐人街}{tang2ren2 jie1}{10,2,12}
  \definition*{s.}{Bairro Chinês | \emph{Chinatown}}
  \seealsoref{中国城}{zhong1guo2cheng2}
\end{entry}

\begin{entry}{糖}{tang2}{16}[Radical 米]
  \definition[颗,块]{s.}{açúcar | doces}
\end{entry}

\begin{entry}{糖醋鱼}{tang2cu4yu2}{16,15,8}
  \definition{s.}{peixe guisado em molho agridoce (prato)}
\end{entry}

\begin{entry}{倘或}{tang3huo4}{10,8}
  \definition{conj.}{se | supondo que | no caso}
\end{entry}

\begin{entry}{倘若}{tang3ruo4}{10,8}
  \definition{conj.}{se | supondo que | no caso}
\end{entry}

\begin{entry}{倘使}{tang3shi3}{10,8}
  \definition{conj.}{se | supondo que | no caso}
\end{entry}

\begin{entry}{滔天}{tao1tian1}{13,4}
  \definition{adj.}{(ondas, raiva, desastres, crimes, etc.) imponente, avassalador, imenso}
\end{entry}

\begin{entry}{逃}{tao2}{9}[Radical 辵]
  \definition{v.}{escapar | fugir}
\end{entry}

\begin{entry}{桃}{tao2}{10}[Radical 木]
  \definition{s.}{pêssego}
\end{entry}

\begin{entry}{讨论}{tao3lun4}{5,6}[HSK 2]
  \definition{v.}{discutir | falar sobre}
\end{entry}

\begin{entry}{讨生活}{tao3sheng1huo2}{5,5,9}
  \definition{v.}{ganhar a vida}
\end{entry}

\begin{entry}{套}{tao4}{10}[Radical 大][HSK 2]
  \definition{clas.}{para conjuntos, coleções}
  \definition{s.}{cobertura | fórmula | laço de corda}
  \definition{v.}{cobrir | envolver | intercalar | sobrepor}
\end{entry}

\begin{entry}{套问}{tao4wen4}{10,6}
  \definition{s.}{retórica}
  \definition{v.}{descobrir por meio de questionamento indireto diplomático}
\end{entry}

\begin{entry}{特别}{te4bie2}{10,7}[HSK 2]
  \definition{adj.}{especial | paricular | incomum}
  \definition{adv.}{especialmente | particularmente | propositalmente}
\end{entry}

\begin{entry}{特地}{te4di4}{10,6}
  \definition{adv.}{especialmente | propositalmente}
\end{entry}

\begin{entry}{特点}{te4dian3}{10,9}[HSK 2]
  \definition[个]{s.}{característica | peculiaridade | característica distintiva}
\end{entry}

\begin{entry}{特技}{te4ji4}{10,7}
  \definition{s.}{efeito especial | dublê}
\end{entry}

\begin{entry}{疼}{teng2}{10}[Radical 疒][HSK 2]
  \definition{adj.}{dolorido | doído}
  \definition{v.}{doer | amar ternamente}
\end{entry}

\begin{entry}{梯恩梯}{ti1'en1ti1}{11,10,11}
  \definition{s.}{(empréstimo linguístico) TNT, trinitrotolueno}
\end{entry}

\begin{entry}{踢}{ti1}{15}[Radical 足]
  \definition{v.}{chutar | jogar (por exemplo, futebol) | dar pontapés em}
\end{entry}

\begin{entry}{踢爆}{ti1bao4}{15,19}
  \definition{v.}{expor | revelar}
\end{entry}

\begin{entry}{踢蹋舞}{ti1ta4wu3}{15,17,14}
  \definition{s.}{sapateado | passo de dança}
\end{entry}

\begin{entry}{提}{ti2}{12}[Radical 手][HSK 2]
  \definition*{s.}{sobrenome Ti}
  \definition{s.}{concha | traço ascendente (em caracteres chineses)}
  \definition{v.}{carregar (na mão com o braço para baixo) | levantar | elevar | promover | avançar | mudar para um momento anterior | mover uma data para a frente | trazer à tona | apresentar | extrair | tirar | trazer | entregar | mencionar | referir-se a}
\end{entry}

\begin{entry}{提出}{ti2 chu1}{12,5}[HSK 2]
  \definition{v.}{levantar | propor | expor | apresentar}
\end{entry}

\begin{entry}{提到}{ti2 dao4}{12,8}[HSK 2]
  \definition{v.}{mencionar | referir-se a | levantar (assunto)}
\end{entry}

\begin{entry}{提高}{ti2gao1}{12,10}[HSK 2]
  \definition{v.}{melhorar | aumentar | elevar}
\end{entry}

\begin{entry}{提及}{ti2ji2}{12,3}
  \definition{v.}{mencionar | levantar (um assunto) | chamar a atenção de alguém}
\end{entry}

\begin{entry}{提升}{ti2sheng1}{12,4}
  \definition{v.}{promover (para uma posição de classificação mais alta) | levantar | içar | (figurativo) elevar, levantar, melhorar}
\end{entry}

\begin{entry}{题}{ti2}{15}[Radical 頁][HSK 2]
  \definition*{s.}{sobrenome Ti}
  \definition[道]{s.}{assunto | título | tópico | problema}
  \definition{v.}{inscrever | escrever}
\end{entry}

\begin{entry}{体内}{ti3nei4}{7,4}
  \definition{adj.}{dentro do corpo | \emph{in vivo} (versus \emph{in vitro} | interno a}
\end{entry}

\begin{entry}{体验}{ti3yan4}{7,10}
  \definition{v.}{vivenciar | experimentar por si mesmo}
\end{entry}

\begin{entry}{体育}{ti3yu4}{7,8}[HSK 2]
  \definition{s.}{treinamento físico | esportes | atividades esportivas}
\end{entry}

\begin{entry}{体育场}{ti3 yu4 chang3}{7,8,6}[HSK 2]
  \definition[个,座]{s.}{estádio | campo de esportes}
\end{entry}

\begin{entry}{体育馆}{ti3 yu4 guan3}{7,8,11}[HSK 2]
  \definition[个]{s.}{ginásio | estádio}
\end{entry}

\begin{entry}{天}{tian1}{4}[Radical 大][HSK 1]
  \definition{s.}{dia | céu | paraíso}
\end{entry}

\begin{entry}{天才}{tian1cai2}{4,3}
  \definition{adj.}{talentoso | superdotado | genial}
  \definition{s.}{talento | dom | gênio}
\end{entry}

\begin{entry}{天鹅}{tian1'e2}{4,12}
  \definition{s.}{cisne}
\end{entry}

\begin{entry}{天公}{tian1gong1}{4,4}
  \definition{s.}{céu, paraíso | senhor do céu}
\end{entry}

\begin{entry}{天花板}{tian1hua1ban3}{4,7,8}
  \definition{s.}{teto}
\end{entry}

\begin{entry}{天气}{tian1qi4}{4,4}[HSK 1]
  \definition{s.}{clima, tempo}
\end{entry}

\begin{entry}{天然}{tian1ran2}{4,12}
  \definition{adj.}{natural}
\end{entry}

\begin{entry}{天上}{tian1 shang4}{4,3}[HSK 2]
  \definition{s.}{o céu | paraíso}
\end{entry}

\begin{entry}{天使}{tian1shi3}{4,8}
  \definition{s.}{anjo}
\end{entry}

\begin{entry}{天堂}{tian1tang2}{4,11}
  \definition{s.}{paraíso, céu}
\end{entry}

\begin{entry}{天天}{tian1tian1}{4,4}
  \definition{adv.}{todo dia}
\end{entry}

\begin{entry}{天下}{tian1xia4}{4,3}
  \definition{s.}{terra sob o céu | o mundo todo | toda a China | reino}
\end{entry}

\begin{entry}{天择}{tian1ze2}{4,8}
  \definition{s.}{seleção natural}
\end{entry}

\begin{entry}{天柱}{tian1zhu4}{4,9}
  \definition{s.}{pilar celestial, que sustenta o céu}
\end{entry}

\begin{entry}{兲}{tian1}{6}[Radical 八]
  \variantof{天}
\end{entry}

\begin{entry}{田}{tian2}{5}[Radical 田][Kangxi 102]
  \definition*{s.}{sobrenome Tian}
  \definition[片]{s.}{fazenda | campo}
\end{entry}

\begin{entry}{田园}{tian2yuan2}{5,7}
  \definition{adj.}{bucólico}
  \definition{s.}{campo | interior | rural}
\end{entry}

\begin{entry}{钿}{tian2}{10}[Radical 金]
  \definition{s.}{(dialeto) moeda, dinheiro}
  \seeref{钿}{dian4}
\end{entry}

\begin{entry}{甜}{tian2}{11}[Radical 甘]
  \definition{adj.}{doce}
\end{entry}

\begin{entry}{甜酒}{tian2jiu3}{11,10}
  \definition{s.}{licor doce}
\end{entry}

\begin{entry}{甜菊}{tian2ju2}{11,11}
  \definition{s.}{estévia, arbusto cujas folhas produzem um substituto para o açúcar}
\end{entry}

\begin{entry}{甜品}{tian2pin3}{11,9}
  \definition{s.}{sobremesa}
\end{entry}

\begin{entry}{甜食}{tian2shi2}{11,9}
  \definition{s.}{doces | sobremesa}
\end{entry}

\begin{entry}{甜酸}{tian2suan1}{11,14}
  \definition{adj.}{agridoce}
\end{entry}

\begin{entry}{甜甜圈}{tian2tian2quan1}{11,11,11}
  \definition{s.}{rosquinha | \emph{doughnut}}
\end{entry}

\begin{entry}{甜筒}{tian2tong3}{11,12}
  \definition{s.}{sorvete de casquinha}
\end{entry}

\begin{entry}{甜头}{tian2tou5}{11,5}
  \definition{s.}{benefício | sabor doce (de poder, sucesso, etc.)}
\end{entry}

\begin{entry}{甜心}{tian2xin1}{11,4}
  \definition{s.}{querido}
\end{entry}

\begin{entry}{甜言}{tian2yan2}{11,7}
  \definition{s.}{boa conversa | palavras amáveis}
\end{entry}

\begin{entry}{甜玉米}{tian2 yu4mi3}{11,5,6}
  \definition{s.}{milho doce}
\end{entry}

\begin{entry}{甜稚}{tian2zhi4}{11,13}
  \definition{s.}{doce e inocente}
\end{entry}

\begin{entry}{条}{tiao2}{7}[Radical 木][HSK 2]
  \definition{clas.}{para coisas longas e finas (fita, rio, estrada, calças, etc.)}
  \definition{s.}{artigo | cláusula (de lei ou tratado) | item | faixa}
\end{entry}

\begin{entry}{条幅}{tiao2fu2}{7,12}
  \definition{s.}{faixa | banner | pergaminho de parede (para pintura ou caligrafia)}
\end{entry}

\begin{entry}{条贯}{tiao2guan4}{7,8}
  \definition{s.}{ordem | procedimentos | sequência | sistema}
\end{entry}

\begin{entry}{条件}{tiao2jian4}{7,6}[HSK 2]
  \definition[个]{s.}{circunstâncias | condição | fator | pré-requisito | qualificação | requisito}
\end{entry}

\begin{entry}{条例}{tiao2li4}{7,8}
  \definition{s.}{código de conduta | ordenanças | regulamentos | regras | estatutos}
\end{entry}

\begin{entry}{条目}{tiao2mu4}{7,5}
  \definition{s.}{cláusulas e subcláusulas (em documento formal) | verbete (em um dicionário, enciclopédia, etc.)}
\end{entry}

\begin{entry}{调律}{tiao2lv4}{10,9}
  \definition{v.}{afinar (por exemplo, um piano)}
\end{entry}

\begin{entry}{挑衅}{tiao3xin4}{9,11}
  \definition{s.}{provocação}
  \definition{v.}{provocar}
\end{entry}

\begin{entry}{跳}{tiao4}{13}[Radical 足]
  \definition{v.}{pular | saltar}
\end{entry}

\begin{entry}{跳挡}{tiao4dang3}{13,9}
  \definition{v.}{pular marcha (de um carro) | perder a marcha}
\end{entry}

\begin{entry}{跳电}{tiao4dian4}{13,5}
  \definition{v.}{desarmar (um disjuntor ou interruptor)}
\end{entry}

\begin{entry}{跳频}{tiao4pin2}{13,13}
  \definition{s.}{FHSS, \emph{Frequency-Hopping Spread Spectrum}, método de transmissão de sinais de rádio}
\end{entry}

\begin{entry}{跳伞}{tiao4san3}{13,6}
  \definition{s.}{paraquedas}
  \definition{v.}{saltar de paraquedas}
\end{entry}

\begin{entry}{跳绳}{tiao4sheng2}{13,11}
  \definition{v.}{pular corda}
\end{entry}

\begin{entry}{跳水}{tiao4shui3}{13,4}
  \definition{s.}{mergulho esportivo}
  \definition{v.}{mergulhar (na água) | cometer suicídio pulando na água | (figurativo, preços das ações, etc.) cair dramaticamente}
\end{entry}

\begin{entry}{跳跳糖}{tiao4tiao4tang2}{13,13,16}
  \definition{s.}{\emph{Pop Rocks}, \emph{popping candy}}
\end{entry}

\begin{entry}{跳舞}{tiao4wu3}{13,14}
  \definition{v.+compl.}{dançar}
\end{entry}

\begin{entry}{跳远}{tiao4yuan3}{13,7}
  \definition{v.+compl.}{salto em distância (atletismo)}
\end{entry}

\begin{entry}{跳蚤}{tiao4zao5}{13,9}
  \definition{s.}{pulga}
\end{entry}

\begin{entry}{铁}{tie3}{10}[Radical 金]
  \definition*{s.}{sobrenome Tie}
  \definition{adj.}{duro | forte | violento | inabalável | determinado | (gíria) apertado}
  \definition{s.}{ferro (metal) | arma}
\end{entry}

\begin{entry}{铁轨}{tie3gui3}{10,6}
  \definition[根]{s.}{trilho | trilho ferroviário}
\end{entry}

\begin{entry}{铁路}{tie3lu4}{10,13}
  \definition[条]{s.}{ferrovia}
\end{entry}

\begin{entry}{听}{ting1}{7}[Radical 口][HSK 1]
  \definition{clas.}{para bebidas enlatadas}
  \definition{s.}{lata de bebida (empréstimo linguístico, do inglês ``\emph{tin}'')}
  \definition{v.}{ouvir | escutar | obedecer}
\end{entry}

\begin{entry}{听到}{ting1dao4}{7,8}[HSK 1]
  \definition{v.}{ouvir | notar}
\end{entry}

\begin{entry}{听断}{ting1duan4}{7,11}
  \definition{v.}{ouvir e decidir | julgar (ou seja, ouvir e julgar em um tribunal)}
\end{entry}

\begin{entry}{听骨}{ting1gu3}{7,9}
  \definition{s.}{ossículos (do ouvido médio)}
  \seealsoref{听小骨}{ting1xiao3gu3}
\end{entry}

\begin{entry}{听会}{ting1hui4}{7,6}
  \definition{v.}{participar de uma reunião (e ouvir o que é discutido)}
\end{entry}

\begin{entry}{听见}{ting1 jian4}{7,4}[HSK 1]
  \definition{v.}{ouvir}
\end{entry}

\begin{entry}{听讲}{ting1 jiang3}{7,6}[HSK 2]
  \definition{v.+compl.}{assistir a uma palestra; ouvir uma conversa}
\end{entry}

\begin{entry}{听来}{ting1lai2}{7,7}
  \definition{v.}{ouvir de algum lugar | soar (antigo, estrangeiro, excitante, certo, etc.) | soar como se (ou seja, dar uma impressão ao ouvinte)}
\end{entry}

\begin{entry}{听力}{ting1li4}{7,2}
  \definition{s.}{audição | capacidade de compreensão oral}
\end{entry}

\begin{entry}{听力理解}{ting1li4li3jie3}{7,2,11,13}
  \definition{s.}{compreensão auditiva}
\end{entry}

\begin{entry}{听命}{ting1ming4}{7,8}
  \definition{v.}{obedecer ordens | receber ordens}
\end{entry}

\begin{entry}{听凭}{ting1ping2}{7,8}
  \definition{v.}{permitir (alguém a fazer o que desejar)}
\end{entry}

\begin{entry}{听说}{ting1 shuo1}{7,9}[HSK 2]
  \definition{v.}{ouvir dizer}
\end{entry}

\begin{entry}{听随}{ting1sui2}{7,11}
  \definition{v.}{permitir | obedecer}
\end{entry}

\begin{entry}{听戏}{ting1xi4}{7,6}
  \definition{v.}{assistir a uma ópera | ver uma ópera}
\end{entry}

\begin{entry}{听小骨}{ting1xiao3gu3}{7,3,9}
  \definition{s.}{ossículos (do ouvido médio)}
  \seealsoref{听骨}{ting1gu3}
\end{entry}

\begin{entry}{听写}{ting1xie3}{7,5}[HSK 1]
  \definition{s.}{ditado}
  \definition{v.}{transcrever música de ouvido | escrever (em um exercício de ditado)}
\end{entry}

\begin{entry}{聼}{ting1}{19}[Radical 耳]
  \variantof{听}
\end{entry}

\begin{entry}{亭}{ting2}{9}[Radical 亠]
  \definition{s.}{pavilhão | cabine | quiosque}
\end{entry}

\begin{entry}{停}{ting2}{11}[Radical 人][HSK 2]
  \definition{v.}{parar | estacionar (um carro)}
\end{entry}

\begin{entry}{停办}{ting2ban4}{11,4}
  \definition{v.}{cancelar | sair do negócio | desligar | terminar}
\end{entry}

\begin{entry}{停车}{ting2 che1}{11,4}[HSK 2]
  \definition{v.}{parar de trabalhar (uma máquina) | estacionar | parar (um veículo) | paralisar}
\end{entry}

\begin{entry}{停车场}{ting2 che1 chang3}{11,4,6}[HSK 2]
  \definition{s.}{parque de estacionamento}
\end{entry}

\begin{entry}{停当}{ting2dang5}{11,6}
  \definition{adj.}{realizado | preparado | assentado}
\end{entry}

\begin{entry}{停电}{ting2dian4}{11,5}
  \definition{s.}{corte de energia}
  \definition{v.}{ter uma falha de energia}
\end{entry}

\begin{entry}{停工}{ting2gong1}{11,3}
  \definition{v.}{parar de trabalhar | parar a produção}
\end{entry}

\begin{entry}{停火}{ting2huo3}{11,4}
  \definition{s.}{cessar-fogo}
  \definition{v.+compl.}{cessar fogo}
\end{entry}

\begin{entry}{停课}{ting2ke4}{11,10}
  \definition{v.}{fechar (escola) | parar as aulas}
\end{entry}

\begin{entry}{停留}{ting2liu2}{11,10}
  \definition{v.}{ficar em algum lugar temporariamente | demorar | permanecer}
\end{entry}

\begin{entry}{停息}{ting2xi1}{11,10}
  \definition{v.}{cessar | parar}
\end{entry}

\begin{entry}{停歇}{ting2xie1}{11,13}
  \definition{v.}{parar para descansar}
\end{entry}

\begin{entry}{停业}{ting2ye4}{11,5}
  \definition{v.}{cessar a negociação (temporária ou permanentemente) | fechar}
\end{entry}

\begin{entry}{停用}{ting2yong4}{11,5}
  \definition{v.}{desabilitar | descontinuar | parar de usar | suspender}
\end{entry}

\begin{entry}{停止}{ting2zhi3}{11,4}
  \definition{v.}{cessar | encerrar | parar}
\end{entry}

\begin{entry}{挺}{ting3}{9}[Radical 手][HSK 2]
  \definition{adj.}{ereto | fora do comum | direto}
  \definition{adv.}{bastante, ou melhor, bonito | muito (coloquial)}
  \definition{clas.}{para metralhadoras}
  \definition{v.}{endireitar (fisicamente) | sobressair (uma parte do corpo) | dar suporte | resistir}
\end{entry}

\begin{entry}{挺拔}{ting3ba2}{9,8}
  \definition{adj.}{alto e reto}
\end{entry}

\begin{entry}{挺杆}{ting3gan3}{9,7}
  \definition{s.}{tucho (peça de máquina)}
\end{entry}

\begin{entry}{挺过}{ting3guo4}{9,6}
  \definition{s.}{sobreviver}
\end{entry}

\begin{entry}{挺好}{ting3 hao3}{9,6}[HSK 2]
  \definition{adj.}{muito bom}
\end{entry}

\begin{entry}{挺进}{ting3jin4}{9,7}
  \definition{s.}{progresso | avanço}
  \definition{v.}{progredir | avançar}
\end{entry}

\begin{entry}{挺立}{ting3li4}{9,5}
  \definition{v.}{ficar ereto | ficar de pé}
\end{entry}

\begin{entry}{挺身}{ting3shen1}{9,7}
  \definition{v.}{endireitar as costas}
\end{entry}

\begin{entry}{挺尸}{ting3shi1}{9,3}
  \definition{v.}{(coloquial) dormir | (literalmente) ficar deitado duro como um cadáver}
\end{entry}

\begin{entry}{挺腰}{ting3yao1}{9,13}
  \definition{v.}{arquear as costas | endireitar as costas}
\end{entry}

\begin{entry}{挺住}{ting3zhu4}{9,7}
  \definition{v.}{permanecer firme | manter-se firme (diante da adversidade ou da dor)}
\end{entry}

\begin{entry}{通}{tong1}{10}[Radical 辵][HSK 2]
  \definition{clas.}{para cartas, telegramas, telefonemas, etc.}
  \definition{suf.}{especialista}
  \definition{v.}{ligar para | conseguir a ligação}
  \seeref{通}{tong4}
\end{entry}

\begin{entry}{通牒}{tong1die2}{10,13}
  \definition{s.}{nota diplomática}
\end{entry}

\begin{entry}{通观}{tong1guan1}{10,6}
  \definition{v.}{ter uma visão geral de algo}
\end{entry}

\begin{entry}{通过}{tong1guo4}{10,6}[HSK 2]
  \definition{adv.}{por meio de | através de | via}
  \definition{v.}{passar por | adotar (uma resolução), aprovar (legislação) | passar (em um teste)}
\end{entry}

\begin{entry}{通识}{tong1shi2}{10,7}
  \definition{s.}{conhecimento comum | erudição | conhecimento geral | amplamente conhecido}
\end{entry}

\begin{entry}{通知}{tong1zhi1}{10,8}[HSK 2]
  \definition[份,个,张]{s.}{aviso | circular}
  \definition{v.}{aconselhar | notificar | informar | dar aviso}
\end{entry}

\begin{entry}{同}{tong2}{6}[Radical 口]
  \definition{adj.}{junto}
  \definition{adv.}{junto com}
\end{entry}

\begin{entry}{同伙}{tong2huo3}{6,6}
  \definition[个]{s.}{cúmplice | colega}
\end{entry}

\begin{entry}{同流合污}{tong2liu2he2wu1}{6,10,6,6}
  \definition{expr.}{chafurdar na lama com alguém | seguir o mau exemplo dos outros}
\end{entry}

\begin{entry}{同情}{tong2qing2}{6,11}
  \definition{s.}{simpatia}
  \definition{v.}{simpatizar com}
\end{entry}

\begin{entry}{同时}{tong2shi2}{6,7}[HSK 2]
  \definition{conj.}{além disso}
  \definition{s.}{enquanto isso | ao mesmo tempo}
\end{entry}

\begin{entry}{同事}{tong2shi4}{6,8}[HSK 2]
  \definition{s.}{colega | colega de trabalho | companheiro}
\end{entry}

\begin{entry}{同屋}{tong2wu1}{6,9}
  \definition[个]{s.}{companheiro de quarto | colega de quarto}
\end{entry}

\begin{entry}{同性恋}{tong2xing4lian4}{6,8,10}
  \definition{s.}{homossexualidade | pessoa gay | amor gay}
\end{entry}

\begin{entry}{同学}{tong2xue2}{6,8}[HSK 1]
  \definition[位,个]{s.}{colega de classe | colega estudante}
\end{entry}

\begin{entry}{同砚}{tong2yan4}{6,9}
  \definition[位,个]{s.}{colega de classe | colega estudante}
\end{entry}

\begin{entry}{同样}{tong2 yang4}{6,10}[HSK 2]
  \definition{adj.}{igual | similar}
\end{entry}

\begin{entry}{同意}{tong2yi4}{6,13}
  \definition{v.}{concordar | aprovar | consentir}
\end{entry}

\begin{entry}{童年}{tong2nian2}{12,6}
  \definition{s.}{infância}
\end{entry}

\begin{entry}{通}{tong4}{10}[Radical 辵]
  \definition{clas.}{para uma atividade, tomada em sua totalidade (discurso de abuso, período de reprodução de música, bebedeira, etc.)}
  \seeref{通}{tong1}
\end{entry}

\begin{entry}{痛骂}{tong4ma4}{12,9}
  \definition{v.}{repreender severamente}
\end{entry}

\begin{entry}{偷}{tou1}{11}[Radical 人]
  \definition{adv.}{furtivamente}
  \definition{v.}{furtar | roubar}
\end{entry}

\begin{entry}{偷安}{tou1'an1}{11,6}
  \definition{v.}{buscar facilidade temporária}
\end{entry}

\begin{entry}{偷渡}{tou1du4}{11,12}
  \definition{s.}{contrabando | imigração ilegal | clandestino (em um navio)}
  \definition{v.}{executar um bloqueio | roubar através da fronteira internacional}
\end{entry}

\begin{entry}{偷窃}{tou1qie4}{11,9}
  \definition{v.}{furtar | roubar}
\end{entry}

\begin{entry}{偷情}{tou1qing2}{11,11}
  \definition{v.}{manter um caso de amor clandestino}
\end{entry}

\begin{entry}{偷税}{tou1shui4}{11,12}
  \definition{s.}{evasão fiscal}
\end{entry}

\begin{entry}{偷听}{tou1ting1}{11,7}
  \definition{v.}{bisbilhotar; monitorar (secretamente)}
\end{entry}

\begin{entry}{偷袭}{tou1xi2}{11,11}
  \definition{s.}{ataque surpresa}
  \definition{v.}{montar um ataque furtivo | invadir}
\end{entry}

\begin{entry}{偸}{tou1}{11}[Radical 亻]
  \variantof{偷}
\end{entry}

\begin{entry}{头}{tou2}{5}[Radical 大][HSK 2]
  \definition{clas.}{para suínos ou gado}
  \definition[个]{s.}{cabeça}
  \seeref{头}{tou5}
\end{entry}

\begin{entry}{头发}{tou2fa5}{5,5}[HSK 2]
  \definition{s.}{cabelo}
\end{entry}

\begin{entry}{头号}{tou2hao4}{5,5}
  \definition{adj.}{primeira classe | número um | \emph{top rank}}
\end{entry}

\begin{entry}{头脑风暴}{tou2nao3feng1bao4}{5,10,4,15}
  \definition{s.}{\emph{brainstorm}}
\end{entry}

\begin{entry}{头头}{tou2tou2}{5,5}
  \definition{s.}{chefe | o cabeça}
\end{entry}

\begin{entry}{头像}{tou2xiang4}{5,13}
  \definition{s.}{retrato | busto | avatar | imagem de perfil (computação)}
\end{entry}

\begin{entry}{投递}{tou2di4}{7,10}
  \definition{v.}{despachar | enviar}
\end{entry}

\begin{entry}{投票}{tou2piao4}{7,11}
  \definition{v.+compl.}{votar | depositar um voto}
\end{entry}

\begin{entry}{投资}{tou2zi1}{7,10}
  \definition{s.}{investimento}
  \definition{v.}{investir}
\end{entry}

\begin{entry}{投资风险}{tou2zi1feng1xian3}{7,10,4,9}
  \definition{s.}{risco de investimento}
\end{entry}

\begin{entry}{投资回报率}{tou2zi1hui2bao4lv4}{7,10,6,7,11}
  \definition{s.}{retorno sobre o investimento (ROI)}
\end{entry}

\begin{entry}{投资家}{tou2zi1jia1}{7,10,10}
  \definition{s.}{investidor}
  \seealsoref{投资人}{tou2zi1ren2}
  \seealsoref{投资者}{tou2zi1zhe3}
\end{entry}

\begin{entry}{投资人}{tou2zi1ren2}{7,10,2}
  \definition{s.}{investidor}
  \seealsoref{投资家}{tou2zi1jia1}
  \seealsoref{投资者}{tou2zi1zhe3}
\end{entry}

\begin{entry}{投资者}{tou2zi1zhe3}{7,10,8}
  \definition{s.}{investidor}
  \seealsoref{投资家}{tou2zi1jia1}
  \seealsoref{投资人}{tou2zi1ren2}
\end{entry}

\begin{entry}{透}{tou4}{10}[Radical 辵]
  \definition{adj.}{completo | total}
  \definition{adv.}{completamente | totalmente}
  \definition{v.}{aparecer | passar através | penetrar}
\end{entry}

\begin{entry}{透彻}{tou4che4}{10,7}
  \definition{adj.}{minucioso | incisivo | penetrante}
\end{entry}

\begin{entry}{透澈}{tou4che4}{10,15}
  \variantof{透彻}
\end{entry}

\begin{entry}{透顶}{tou4ding3}{10,8}
  \definition{adv.}{completamente}
\end{entry}

\begin{entry}{透过}{tou4guo4}{10,6}
  \definition{v.}{passar através | penetrar}
\end{entry}

\begin{entry}{透亮}{tou4liang4}{10,9}
  \definition{adj.}{brilhante | claro como cristal}
\end{entry}

\begin{entry}{透露}{tou4lu4}{10,21}
  \definition{v.}{divulgar | vazar | revelar}
\end{entry}

\begin{entry}{透明}{tou4ming2}{10,8}
  \definition{adj.}{transparente | (figurativo) transparente, aberto a escrutínio}
\end{entry}

\begin{entry}{透辟}{tou4pi4}{10,13}
  \definition{adj.}{incisivo | penetrante}
\end{entry}

\begin{entry}{透气}{tou4qi4}{10,4}
  \definition{v.}{respirar (sobre tecido, etc.) | fluir livremente (sobre ar) | respirar ar fresco | ventilar}
\end{entry}

\begin{entry}{透水}{tou4shui3}{10,4}
  \definition{adj.}{permeável}
  \definition{s.}{vazamento de água}
\end{entry}

\begin{entry}{透支}{tou4zhi1}{10,4}
  \definition{v.}{cheque especial (bancário) | saque a descoberto}
\end{entry}

\begin{entry}{头}{tou5}{5}[Radical 大]
  \definition{suf.}{sufixo para nomes}
  \seeref{头}{tou2}
\end{entry}

\begin{entry}{突然}{tu1ran2}{9,12}
  \definition{adv.}{de repente | abruptamente | inesperadamente}
\end{entry}

\begin{entry}{图}{tu2}{8}[Radical 囗]
  \definition[张]{s.}{diagrama | imagem | desenho | gráfico | mapa}
  \definition{v.}{planejar | esquematizar | tentar | perseguir | procurar}
\end{entry}

\begin{entry}{图片}{tu2 pian4}{8,4}[HSK 2]
  \definition[张,幅]{s.}{imagem | fotografia}
\end{entry}

\begin{entry}{图书馆}{tu2shu1guan3}{8,4,11}[HSK 1]
  \definition[家,个]{s.}{biblioteca}
\end{entry}

\begin{entry}{徒手}{tu2shou3}{10,4}
  \definition{adj.}{com as mãos vazias | desarmado | mão livre (desenho) | lutando mão-a-mão}
\end{entry}

\begin{entry}{土地}{tu3di4}{3,6}
  \definition[片,块]{s.}{terra | solo | território}
  \seeref{土地}{tu3di4}
\end{entry}

\begin{entry}{土地}{tu3di5}{3,6}
  \definition{s.}{deus local | \emph{genius loci} deidade protetora de um local}
  \seeref{土地}{tu3di4}
\end{entry}

\begin{entry}{土豆}{tu3dou4}{3,7}
  \definition[个,颗]{s.}{batata}
\end{entry}

\begin{entry}{土豆泥}{tu3dou4ni2}{3,7,8}
  \definition{s.}{purê de batatas}
\end{entry}

\begin{entry}{土鸡}{tu3ji1}{3,7}
  \definition{s.}{galinha caipira}
\end{entry}

\begin{entry}{吐}{tu3}{6}[Radical 口]
  \definition{v.}{cuspir | enviar (seda de um bicho-da-seda, cápsulas de flores de algodão etc.) | dizer | despejar (suas queixas)}
  \seeref{吐}{tu4}
\end{entry}

\begin{entry}{吐}{tu4}{6}[Radical 口]
  \definition{v.}{vomitar}
  \seeref{吐}{tu3}
\end{entry}

\begin{entry}{兔子}{tu4zi5}{8,3}
  \definition[只]{s.}{coelho | lebre}
\end{entry}

\begin{entry}{团队}{tuan2dui4}{6,4}
  \definition{s.}{equipe}
\end{entry}

\begin{entry}{团结}{tuan2jie2}{6,9}
  \definition{adj.}{unido}
  \definition{s.}{unidade | solidariedade}
  \definition{v.}{unir}
\end{entry}

\begin{entry}{推}{tui1}{11}[Radical 手][HSK 2]
  \definition{v.}{empurrar | girar um moinho ou uma pedra de amolar | moer | impulsionar | promover | avançar | estender | deduzir | inferir | declinar | empurrar para longe | deslocar | adiar | diferir | eleger | selecionar | escolher | ter em alta estima | elogiar muito}
\end{entry}

\begin{entry}{推迟}{tui1chi2}{11,7}
  \definition{v.}{adiar | deixar para mais tarde | tardar}
\end{entry}

\begin{entry}{推介}{tui1jie4}{11,4}
  \definition{s.}{promoção}
  \definition{v.}{promover | introduzir e recomendar}
\end{entry}

\begin{entry}{腿}{tui3}{13}[Radical 肉][HSK 2]
  \definition[条]{s.}{perna | osso do quadril}
\end{entry}

\begin{entry}{腿号}{tui3hao4}{13,5}
  \definition{s.}{anilha numerada (por exemplo, usada para identificar pássaros)}
  \seealsoref{腿号箍}{tui3hao4gu1}
\end{entry}

\begin{entry}{腿号箍}{tui3hao4gu1}{13,5,14}
  \definition{s.}{anilha numerada (por exemplo, usada para identificar pássaros)}
  \seealsoref{腿号}{tui3hao4}
\end{entry}

\begin{entry}{退休}{tui4xiu1}{9,6}
  \definition{v.+compl.}{aposentar-se}
\end{entry}

\begin{entry}{拖拉机}{tuo1la1ji1}{8,8,6}
  \definition[台]{s.}{trator}
\end{entry}

\begin{entry}{拖鞋}{tuo1xie2}{8,15}
  \definition[双,只]{s.}{chinelos | sandálias}
\end{entry}

\begin{entry}{脱毛}{tuo1mao2}{11,4}
  \definition{s.}{depilação}
  \definition{v.}{perder cabelo ou penas | depilar | fazer a barba}
\end{entry}

\begin{entry}{脱险}{tuo1xian3}{11,9}
  \definition{v.}{sair do perigo}
\end{entry}

\begin{entry}{鸵鸟}{tuo2niao3}{10,5}
  \definition{s.}{avestruz}
\end{entry}

\begin{entry}{唾骂}{tuo4ma4}{11,9}
  \definition{v.}{insultar | amaldiçoar}
\end{entry}

%%%%% EOF %%%%%


%%%%%%%%%%%%%%%%%%%%%%%%%%%%%% Não existem palavras com pinyin iniciado em "U"
%%%%%%%%%%%%%%%%%%%%%%%%%%%%%% Não existem palavras com pinyin iniciado em "V"
%%%
%%% W
%%%

\section*{W}\addcontentsline{toc}{section}{W}

\begin{entry}{哇塞}{wa1sai1}{9,13}[Radicais ⼝、⼟]
  \definition{interj.}{(gíria) Uau!}
\end{entry}

\begin{entry}{哇噻}{wa1sai1}{9,16}[Radicais ⼝、⼝]
  \variantof{哇塞}
\end{entry}

\begin{entry}{挖}{wa1}{9}[Radical ⼿]
  \definition{v.}{cavar | escavar}
\end{entry}

\begin{entry}{挖掘机}{wa1jue2ji1}{9,11,6}[Radicais ⼿、⼿、⽊]
  \definition{s.}{escavadeira | escavador | escavadora | pá mecânica}
\end{entry}

\begin{entry}{瓦}{wa3}{4}[Kangxi 98][Radical ⽡]
  \definition{s.}{telha | abreviação de 瓦特}
  \seeref{瓦特}{wa3te4}
\end{entry}

\begin{entry}{瓦努阿图}{wa3nu3'a1tu2}{4,7,7,8}[Radicais ⽡、⼒、⾩、⼞]
  \definition*{s.}{Vanuatu, país do sudoeste do Oceano Pacífico}
\end{entry}

\begin{entry}{瓦特}{wa3te4}{4,10}[Radicais ⽡、⽜]
  \definition{s.}{(empréstimo linguístico) watt | medida de potência}
\end{entry}

\begin{entry}{歪}{wai1}{9}[Radical ⽌]
  \definition{adj.}{torto | tortuoso | nocivo}
\end{entry}

\begin{entry}{歪果仁}{wai1guo3ren2}{9,8,4}[Radicais ⽌、⽊、⼈]
  \definition{s.}{gíria na \emph{Internet} para 外国人}
  \seeref{外国人}{wai4guo2ren2}
\end{entry}

\begin{entry}{外}{wai4}{5}[HSK 1][Radical ⼣]
  \definition{s.}{fora | por fora | exterior | estrangeiro}
\end{entry}

\begin{entry}{外边}{wai4bian5}{5,5}[HSK 1][Radicais ⼣、⾡]
  \definition{adv.}{fora do país | superfície externa | fora | lugar diferente de sua casa}
\end{entry}

\begin{entry}{外插}{wai4cha1}{5,12}[Radicais ⼣、⼿]
  \definition{v.}{extrapolar | (computação) conectar (um dispositivo periférico, etc.)}
\end{entry}

\begin{entry}{外地}{wai4 di4}{5,6}[HSK 2][Radicais ⼣、⼟]
  \definition{s.}{não local | outros lugares}
\end{entry}

\begin{entry}{外公}{wai4gong1}{5,4}[Radicais ⼣、⼋]
  \definition{s.}{avô materno}
\end{entry}

\begin{entry}{外国}{wai4guo2}{5,8}[HSK 1][Radicais ⼣、⼞]
  \definition[个]{s.}{país estrangeiro}
\end{entry}

\begin{entry}{外国人}{wai4guo2ren2}{5,8,2}[Radicais ⼣、⼞、⼈]
  \definition{s.}{estrangeiro | pessoa de fora do país}
\end{entry}

\begin{entry}{外海}{wai4hai3}{5,10}[Radicais ⼣、⽔]
  \definition{s.}{mar aberto}
\end{entry}

\begin{entry}{外号}{wai4hao4}{5,5}[Radicais ⼣、⼝]
  \definition{s.}{apelido}
\end{entry}

\begin{entry}{外积}{wai4ji1}{5,10}[Radicais ⼣、⽲]
  \definition{s.}{produto exterior | (matemática) o produto vetorial de dois vetores}
\end{entry}

\begin{entry}{外交}{wai4jiao1}{5,6}[HSK 3][Radicais ⼣、⼇]
  \definition{adj.}{diplomático}
  \definition[个]{s.}{diplomacia; relações exteriores}
\end{entry}

\begin{entry}{外卖}{wai4 mai4}{5,8}[HSK 2][Radicais ⼣、⼗]
  \definition{s.}{para viagem | para fora}
  \definition{v.}{entregar | oferecer}
\end{entry}

\begin{entry}{外贸}{wai4mao4}{5,9}[Radicais ⼣、⾙]
  \definition{s.}{comércio exterior}
\end{entry}

\begin{entry}{外貌协会}{wai4mao4xie2hui4}{5,14,6,6}[Radicais ⼣、⾘、⼗、⼈]
  \definition{s.}{``o clube da boa aparência'': pessoas que dão grande importância à aparência de uma pessoa}
  \seealsoref{外协}{wai4xie2}
\end{entry}

\begin{entry}{外面}{wai4 mian4}{5,9}[HSK 3][Radicais ⼣、⾯]
  \definition{s.}{o lado de fora | exterior; aparência externa}
\end{entry}

\begin{entry}{外婆}{wai4po2}{5,11}[Radicais ⼣、⼥]
  \definition{s.}{avó materna}
\end{entry}

\begin{entry}{外事}{wai4shi4}{5,8}[Radicais ⼣、⼅]
  \definition{s.}{assuntos ou relações exteriores}
\end{entry}

\begin{entry}{外水}{wai4shui3}{5,4}[Radicais ⼣、⽔]
  \definition{s.}{renda extra}
\end{entry}

\begin{entry}{外孙}{wai4sun1}{5,6}[Radicais ⼣、⼦]
  \definition{s.}{filho da filha}
\end{entry}

\begin{entry}{外孙女}{wai4sun1nv3}{5,6,3}[Radicais ⼣、⼦、⼥]
  \definition{s.}{filha da filha}
\end{entry}

\begin{entry}{外围}{wai4wei2}{5,7}[Radicais ⼣、⼞]
  \definition{adv.}{arredores}
\end{entry}

\begin{entry}{外文}{wai4 wen2}{5,4}[HSK 3][Radicais ⼣、⽂]
  \definition{s.}{língua estrangeira (escrita)}
\end{entry}

\begin{entry}{外协}{wai4xie2}{5,6}[Radicais ⼣、⼗]
  \definition{s.}{terceirização | pessoas que julgam os outros pela aparência}
  \seealsoref{外貌协会}{wai4mao4xie2hui4}
\end{entry}

\begin{entry}{外衣}{wai4yi1}{5,6}[Radicais ⼣、⾐]
  \definition{s.}{aparência | roupa de cima}
\end{entry}

\begin{entry}{外语}{wai4yu3}{5,9}[HSK 1][Radicais ⼣、⾔]
  \definition[门]{s.}{língua estrangeira}
\end{entry}

\begin{entry}{豌豆}{wan1dou4}{15,7}[Radicais ⾖、⾖]
  \definition{s.}{ervilha}
\end{entry}

\begin{entry}{完}{wan2}{7}[HSK 2][Radical ⼧]
  \definition{adj.}{completo | inteiro}
  \definition{adv.}{todo}
  \definition{v.}{acabar | completar | terminar}
\end{entry}

\begin{entry}{完备}{wan2bei4}{7,8}[Radicais ⼧、⼡]
  \definition{adj.}{completo | impecável | perfeito}
  \definition{v.}{não deixar nada a desejar}
\end{entry}

\begin{entry}{完毕}{wan2bi4}{7,6}[Radicais ⼧、⽐]
  \definition{v.}{completar | terminar | acabar}
\end{entry}

\begin{entry}{完成}{wan2cheng2}{7,6}[HSK 2][Radicais ⼧、⼽]
  \definition{v.}{realizar | completar}
\end{entry}

\begin{entry}{完满}{wan2man3}{7,13}[Radicais ⼧、⽔]
  \definition{adj.}{satisfatório | bem-sucedido}
\end{entry}

\begin{entry}{完美}{wan2mei3}{7,9}[HSK 3][Radicais ⼧、⽺]
  \definition{adj.}{perfeito; impecável; consumado}
  \definition{adv.}{perfeitamente}
  \definition{s.}{perfeição}
\end{entry}

\begin{entry}{完全}{wan2quan2}{7,6}[HSK 2][Radicais ⼧、⼊]
  \definition{adj.}{completo | todo}
  \definition{adv.}{inteiramente | totalmente}
\end{entry}

\begin{entry}{完人}{wan2ren2}{7,2}[Radicais ⼧、⼈]
  \definition{s.}{pessoa perfeita}
\end{entry}

\begin{entry}{完善}{wan2shan4}{7,12}[HSK 3][Radicais ⼧、⼝]
  \definition{adj.}{perfeito; consumado}
  \definition{v.}{refinar; melhorar; tornar perfeito}
\end{entry}

\begin{entry}{完税}{wan2shui4}{7,12}[Radicais ⼧、⽲]
  \definition{v.}{pagar imposto}
\end{entry}

\begin{entry}{完完全全}{wan2wan2quan2quan2}{7,7,6,6}[Radicais ⼧、⼧、⼊、⼊]
  \definition{adv.}{completamente}
\end{entry}

\begin{entry}{完整}{wan2zheng3}{7,16}[HSK 3][Radicais ⼧、⽁]
  \definition{adj.}{intacto; inteiro; completo; integrado}
\end{entry}

\begin{entry}{玩}{wan2}{8}[Radical ⽟]
  \definition{s.}{brinquedo | algo usado para diversão}
  \definition{v.}{divertir-se | manter algo para entretenimento | brincar com}
\end{entry}

\begin{entry}{玩伴}{wan2ban4}{8,7}[Radicais ⽟、⼈]
  \definition{s.}{parceiro de brincadeira}
\end{entry}

\begin{entry}{玩遍}{wan2bian4}{8,12}[Radicais ⽟、⾡]
  \definition{v.}{passear (todo o país, toda a cidade, etc.) | visitar (um grande número de lugares)}
\end{entry}

\begin{entry}{玩家}{wan2jia1}{8,10}[Radicais ⽟、⼧]
  \definition{s.}{entusiasta (áudio, modelos de aviões, etc.) | jogador (de um jogo)}
\end{entry}

\begin{entry}{玩具}{wan2ju4}{8,8}[HSK 3][Radicais ⽟、⼋]
  \definition[个,件,套,只,辆]{s.}{brinquedo; brincadeira}
\end{entry}

\begin{entry}{玩具厂}{wan2ju4chang3}{8,8,2}[Radicais ⽟、⼋、⼚]
  \definition{s.}{fábrica de brinquedos}
\end{entry}

\begin{entry}{玩具车}{wan2ju4 che1}{8,8,4}[Radicais ⽟、⼋、⾞]
  \definition{s.}{carrinho de brinquedo}
\end{entry}

\begin{entry}{玩偶}{wan2'ou3}{8,11}[Radicais ⽟、⼈]
  \definition{s.}{estatueta de brinquedo | boneco de ação | bicho de pelúcia | boneca}
\end{entry}

\begin{entry}{玩儿}{wan2r5}{8,2}[HSK 1][Radicais ⽟、⼉]
  \definition{v.}{divertir-se}
\end{entry}

\begin{entry}{玩耍}{wan2shua3}{8,9}[Radicais ⽟、⽽]
  \definition{v.}{divertir-me | brincar (como as crianças fazem)}
\end{entry}

\begin{entry}{玩味}{wan2wei4}{8,8}[Radicais ⽟、⼝]
  \definition{v.}{ponderar sutilezas | ruminar (pensamentos)}
\end{entry}

\begin{entry}{玩艺}{wan2yi4}{8,4}[Radicais ⽟、⾋]
  \variantof{玩意}
\end{entry}

\begin{entry}{玩意}{wan2yi4}{8,13}[Radicais ⽟、⼼]
  \definition{s.}{ato | brinquedo | coisa | truque (em uma performance, show de palco, acrobacias, etc.)}
\end{entry}

\begin{entry}{玩者}{wan2zhe3}{8,8}[Radicais ⽟、⽼]
  \definition{s.}{jogador}
\end{entry}

\begin{entry}{顽强}{wan2qiang2}{10,12}[Radicais ⾴、⼸]
  \definition{adj.}{persistente | tenaz | difícil de derrotar}
\end{entry}

\begin{entry}{埦}{wan3}{11}[Radical ⼟]
  \variantof{碗}
\end{entry}

\begin{entry}{晚}{wan3}{11}[HSK 1][Radical ⽇]
  \definition{adj.}{tarde | noite}
\end{entry}

\begin{entry}{晚安}{wan3'an1}{11,6}[HSK 2][Radicais ⽇、⼧]
  \definition{v.}{boa noite}
\end{entry}

\begin{entry}{晚报}{wan3 bao4}{11,7}[HSK 2][Radicais ⽇、⼿]
  \definition{s.}{jornal da noite}
\end{entry}

\begin{entry}{晚餐}{wan3can1}{11,16}[HSK 2][Radicais ⽇、⾷]
  \definition[份,顿,次]{s.}{jantar | refeição noturna}
\end{entry}

\begin{entry}{晚点}{wan3dian3}{11,9}[Radicais ⽇、⽕]
  \definition{adj.}{atrasado}
  \definition{s.}{jantar leve}
\end{entry}

\begin{entry}{晚饭}{wan3fan4}{11,7}[HSK 1][Radicais ⽇、⾷]
  \definition[份,顿,次,餐]{s.}{jantar}
\end{entry}

\begin{entry}{晚会}{wan3hui4}{11,6}[HSK 2][Radicais ⽇、⼈]
  \definition[个]{s.}{festa noturna}
\end{entry}

\begin{entry}{晚近}{wan3jin4}{11,7}[Radicais ⽇、⾡]
  \definition{adj.}{recente | mais recente no passado}
  \definition{adv.}{ultimamente | recentemente}
\end{entry}

\begin{entry}{晚景}{wan3jing3}{11,12}[Radicais ⽇、⽇]
  \definition{s.}{circunstâncias dos anos de declínio de alguém | cena noturna}
\end{entry}

\begin{entry}{晚上}{wan3shang5}{11,3}[HSK 1][Radicais ⽇、⼀]
  \definition{adv.}{noite | à noite}
\end{entry}

\begin{entry}{晚育}{wan3yu4}{11,8}[Radicais ⽇、⾁]
  \definition{s.}{parto tardio}
  \definition{v.}{ter um filho mais tarde}
\end{entry}

\begin{entry}{碗}{wan3}{13}[HSK 2][Radical ⽯]
  \definition{clas.}{tigelas}
  \definition[只,个]{s.}{tigela}
\end{entry}

\begin{entry}{碗柜}{wan3gui4}{13,8}[Radicais ⽯、⽊]
  \definition{s.}{armário}
\end{entry}

\begin{entry}{碗子}{wan3zi5}{13,3}[Radicais ⽯、⼦]
  \definition{s.}{tigela}
\end{entry}

\begin{entry}{万}{wan4}{3}[HSK 2][Radical ⼀]
  \definition*{s.}{sobrenome Wan}
  \definition{adj.}{um grande número}
  \definition{num.}{dez mil; 10.000; 1.0000}
\end{entry}

\begin{entry}{万圣节}{wan4sheng4jie2}{3,5,5}[Radicais ⼀、⼟、⾋]
  \definition*{s.}{Dia de Todos os Santos}
  \seealsoref{万圣节前夕}{wan4sheng4jie2qian2xi1}
\end{entry}

\begin{entry}{万圣节前夕}{wan4sheng4jie2qian2xi1}{3,5,5,9,3}[Radicais ⼀、⼟、⾋、⼑、⼣]
  \definition*{s.}{Véspera do Dia de Todos os Santos | \emph{Halloween}}
  \seealsoref{万圣节}{wan4sheng4jie2}
\end{entry}

\begin{entry}{万万}{wan4wan4}{3,3}[Radicais ⼀、⼀]
  \definition{adv.}{absolutamente | totalmente}
\end{entry}

\begin{entry}{王}{wang2}{4}[Radical ⽟]
  \definition*{s.}{sobrenome Wang}
  \definition{adj.}{grande | ótimo}
  \definition{s.}{rei ou monarca | melhor ou mais forte do seu tipo}
  \seeref{王}{wang4}
\end{entry}

\begin{entry}{王朝}{wang2chao2}{4,12}[Radicais ⽟、⽉]
  \definition{s.}{dinastia}
\end{entry}

\begin{entry}{王五}{wang2wu3}{4,4}[Radicais ⽟、⼆]
  \definition{s.}{Wang Wu | Zé Ninguém | nome para uma pessoa não especificada, 3 de 3}
  \seealsoref{李四}{li3si4}
  \seealsoref{张三}{zhang1san1}
\end{entry}

\begin{entry}{网}{wang3}{6}[HSK 2][Kangxi 122][Radical ⽹]
  \definition{s.}{rede}
\end{entry}

\begin{entry}{网罟}{wang3gu3}{6,10}[Radicais ⽹、⽹]
  \definition{s.}{(fig.) a rede da justiça | rede usada para capturar peixes (ou outros animais, como pássaros)}
\end{entry}

\begin{entry}{网际网路}{wang3ji4wang3lu4}{6,7,6,13}[Radicais ⽹、⾩、⽹、⾜]
  \definition*{s.}{\emph{Internet}}
  \seealsoref{互联网}{hu4lian2wang3}
  \seealsoref{网际网络}{wang3ji4wang3luo4}
  \seealsoref{网路}{wang3lu4}
\end{entry}

\begin{entry}{网际网络}{wang3ji4wang3luo4}{6,7,6,9}[Radicais ⽹、⾩、⽹、⽷]
  \definition*{s.}{\emph{Internet}}
  \seealsoref{互联网}{hu4lian2wang3}
  \seealsoref{网际网路}{wang3ji4wang3lu4}
  \seealsoref{网路}{wang3lu4}
\end{entry}

\begin{entry}{网路}{wang3lu4}{6,13}[Radicais ⽹、⾜]
  \definition{s.}{\emph{Internet}}
  \seealsoref{互联网}{hu4lian2wang3}
  \seealsoref{网际网路}{wang3ji4wang3lu4}
  \seealsoref{网际网络}{wang3ji4wang3luo4}
\end{entry}

\begin{entry}{网球}{wang3qiu2}{6,11}[HSK 2][Radicais ⽹、⽟]
  \definition{s.}{tênis (esporte)}
  \definition[个]{s.}{bola de tênis}
\end{entry}

\begin{entry}{网上}{wang3 shang4}{6,3}[HSK 1][Radicais ⽹、⼀]
  \definition{s.}{\emph{online}}
\end{entry}

\begin{entry}{网上银行}{wang3shang4yin2hang2}{6,3,11,6}[Radicais ⽹、⼀、⾦、⾏]
  \definition[个]{s.}{banco \emph{online} | acesso a operações bancárias via \emph{Internet}}
  \seealsoref{网银}{wang3yin2}
\end{entry}

\begin{entry}{网银}{wang3yin2}{6,11}[Radicais ⽹、⾦]
  \definition{s.}{banco \emph{online} | acesso a operações bancárias via \emph{Internet}}
  \seealsoref{网上银行}{wang3shang4yin2hang2}
\end{entry}

\begin{entry}{网友}{wang3you3}{6,4}[HSK 1][Radicais ⽹、⼜]
  \definition{s.}{internauta | usuário da \emph{Internet}}
\end{entry}

\begin{entry}{网站}{wang3zhan4}{6,10}[HSK 2][Radicais ⽹、⽴]
  \definition[个,家]{s.}{\emph{website}}
\end{entry}

\begin{entry}{往}{wang3}{8}[HSK 2][Radical ⼻]
  \definition{prep.}{para | em direção a}
\end{entry}

\begin{entry}{往程}{wang3cheng2}{8,12}[Radicais ⼻、⽲]
  \definition{s.}{saída (de uma viagem de ônibus ou trem, etc.)}
\end{entry}

\begin{entry}{往返}{wang3fan3}{8,7}[Radicais ⼻、⾡]
  \definition{s.}{ida e volta}
  \definition{v.}{ir e voltar | ir e vir}
\end{entry}

\begin{entry}{往复}{wang3fu4}{8,9}[Radicais ⼻、⼢]
  \definition{s.}{para trás e para frente (por exemplo, da ação do pistão ou da bomba)}
  \definition{v.}{ir e voltar | fazer uma viagem de volta}
\end{entry}

\begin{entry}{往迹}{wang3ji4}{8,9}[Radicais ⼻、⾡]
  \definition{s.}{eventos passados}
\end{entry}

\begin{entry}{往来}{wang3lai2}{8,7}[Radicais ⼻、⽊]
  \definition{s.}{contatos | negociações}
\end{entry}

\begin{entry}{往例}{wang3li4}{8,8}[Radicais ⼻、⼈]
  \definition{s.}{prática (habitual) do passado | precedente}
\end{entry}

\begin{entry}{往日}{wang3ri4}{8,4}[Radicais ⼻、⽇]
  \definition{adv.}{dias passados}
  \definition{s.}{o passado}
\end{entry}

\begin{entry}{往生}{wang3sheng1}{8,5}[Radicais ⼻、⽣]
  \definition{v.}{renascer | morrer | (Budismo) viver no paraíso}
\end{entry}

\begin{entry}{往事}{wang3shi4}{8,8}[Radicais ⼻、⼅]
  \definition{s.}{acontecimentos anteriores | eventos passados}
\end{entry}

\begin{entry}{往往}{wang3wang3}{8,8}[HSK 3][Radicais ⼻、⼻]
  \definition{adv.}{frequentemente; muitas vezes; mais frequentemente do que não}
\end{entry}

\begin{entry}{往昔}{wang3xi1}{8,8}[Radicais ⼻、⽇]
  \definition{s.}{o passado}
\end{entry}

\begin{entry}{罔}{wang3}{8}[Radical ⼌]
  \definition{v.}{enganar}
\end{entry}

\begin{entry}{王}{wang4}{4}[Radical ⽟]
  \definition{v.}{(literário) (um monarca) reinar (um reino)}
  \seeref{王}{wang2}
\end{entry}

\begin{entry}{忘}{wang4}{7}[HSK 1][Radical ⼼]
  \definition{v.}{esquecer | negligenciar | ignorar}
\end{entry}

\begin{entry}{忘本}{wang4ben3}{7,5}[Radicais ⼼、⽊]
  \definition{v.}{esquecer as próprias raízes}
\end{entry}

\begin{entry}{忘餐}{wang4can1}{7,16}[Radicais ⼼、⾷]
  \definition{v.}{esquecer as refeições}
\end{entry}

\begin{entry}{忘掉}{wang4diao4}{7,11}[Radicais ⼼、⼿]
  \definition{v.}{esquecer}
\end{entry}

\begin{entry}{忘恩}{wang4'en1}{7,10}[Radicais ⼼、⼼]
  \definition{v.}{ser ingrato}
\end{entry}

\begin{entry}{忘怀}{wang4huai2}{7,7}[Radicais ⼼、⼼]
  \definition{v.}{esquecer}
\end{entry}

\begin{entry}{忘记}{wang4ji4}{7,5}[HSK 1][Radicais ⼼、⾔]
  \definition{v.}{esquecer}
\end{entry}

\begin{entry}{忘却}{wang4que4}{7,7}[Radicais ⼼、⼙]
  \definition{v.}{esquecer}
\end{entry}

\begin{entry}{危害}{wei1hai4}{6,10}[HSK 3][Radicais ⼙、⼧]
  \definition{s.}{prejuízo; perigo; dano}
  \definition{v.}{prejudicar; pôr em perigo; pôr em risco}
\end{entry}

\begin{entry}{危急}{wei1ji2}{6,9}[Radicais ⼙、⼼]
  \definition{adj.}{crítico | desesperadora (situação)}
\end{entry}

\begin{entry}{危难}{wei1nan4}{6,10}[Radicais ⼙、⾫]
  \definition{s.}{calamidade}
\end{entry}

\begin{entry}{危险}{wei1xian3}{6,9}[HSK 3][Radicais ⼙、⾩]
  \definition{adj.}{arriscado; perigoso}
\end{entry}

\begin{entry}{微风}{wei1feng1}{13,4}[Radicais ⼻、⾵]
  \definition{s.}{brisa | vento leve}
\end{entry}

\begin{entry}{微软}{wei1ruan3}{13,8}[Radicais ⼻、⾞]
  \definition*{s.}{\emph{Microsoft Corporation}}
\end{entry}

\begin{entry}{微型}{wei1xing2}{13,9}[Radicais ⼻、⼟]
  \definition{pref.}{``micro''}
  \definition{s.}{miniatura}
\end{entry}

\begin{entry}{为}{wei2}{4}[HSK 3][Radical ⼂]
  \definition*{s.}{sobrenome Wei}
  \definition{part.}{frequentemente usado com “何” para expressar dúvida}
  \definition{prep.}{como (na capacidade de) | por (na voz passiva)}
  \definition{suf.}{anexado a certos adjetivos monossilábicos, indicando grau ou alcançe | anexado a certos advérbios de grau para fortalecer o tom}
  \definition{v.}{fazer; agir | servir como; agir como; desempenhar o papel de | tornar-se; transformar-se em | ser; significar}
  \seeref{为}{wei4}
  \seealsoref{何}{he2}
\end{entry}

\begin{entry}{围}{wei2}{7}[HSK 3][Radical ⼞]
  \definition*{s.}{sobrenome Wei}
  \definition{clas.}{o comprimento dos dois polegares e indicadores ou o comprimento de ambos os braços quando unidos}
  \definition{s.}{em volta de tudo; ao redor}
  \definition{v.}{cercar; rodear; circundar; encurralar | enrolar; envolver}
\end{entry}

\begin{entry}{违规}{wei2gui1}{7,8}[Radicais ⾡、⾒]
  \definition{v.}{violar as regras}
\end{entry}

\begin{entry}{违宪}{wei2xian4}{7,9}[Radicais ⾡、⼧]
  \definition{adj.}{inconstitucional}
\end{entry}

\begin{entry}{维吾尔}{wei2wu2'er3}{11,7,5}[Radicais ⽷、⼝、⼩]
  \definition*{s.}{Grupo étnico Uigur de Xinjiang}
\end{entry}

\begin{entry}{伟}{wei3}{6}[Radical ⼈]
  \definition{adj.}{grande | ótimo}
\end{entry}

\begin{entry}{伟大}{wei3da4}{6,3}[HSK 3][Radicais ⼈、⼤]
  \definition{adj.}{ótimo; importante (contribuição, etc.) | ótimo; magnífico; digno da maior admiração}
\end{entry}

\begin{entry}{尾巴}{wei3ba5}{7,4}[Radicais ⼫、⼰]
  \definition{s.}{cauda}
\end{entry}

\begin{entry}{委内瑞拉}{wei3nei4rui4la1}{8,4,13,8}[Radicais ⼥、⼌、⽟、⼿]
  \definition*{s.}{Venezuela}
\end{entry}

\begin{entry}{卫生}{wei4 sheng1}{3,5}[HSK 3][Radicais ⼙、⽣]
  \definition{adj.}{bom para a saúde; higiênico}
  \definition{s.}{higiene; saneamento}
\end{entry}

\begin{entry}{卫生部}{wei4sheng1bu4}{3,5,10}[Radicais ⼙、⽣、⾢]
  \definition*{s.}{Ministério da Saúde}
\end{entry}

\begin{entry}{卫生防疫}{wei4sheng1 fang2yi4}{3,5,6,9}[Radicais ⼙、⽣、⾩、⽧]
  \definition{s.}{prevenção contra a epidemia}
\end{entry}

\begin{entry}{卫生间}{wei4sheng1jian1}{3,5,7}[HSK 3][Radicais ⼙、⽣、⾨]
  \definition[间,个]{s.}{banheiro; sanitário; \emph{toilette}}
\end{entry}

\begin{entry}{卫生巾}{wei4sheng1jin1}{3,5,3}[Radicais ⼙、⽣、⼱]
  \definition{s.}{absorvente higiênico}
\end{entry}

\begin{entry}{卫生局}{wei4sheng1ju2}{3,5,7}[Radicais ⼙、⽣、⼫]
  \definition*{s.}{Departamento de Saúde | Escritório de Saúde}
\end{entry}

\begin{entry}{卫生棉}{wei4sheng1mian2}{3,5,12}[Radicais ⼙、⽣、⽊]
  \definition{s.}{absorvente | algodão absorvente esterilizado (usado para curativos ou limpeza de feridas) | absorvente tampão}
\end{entry}

\begin{entry}{卫生球}{wei4sheng1qiu2}{3,5,11}[Radicais ⼙、⽣、⽟]
  \definition{s.}{naftalina}
\end{entry}

\begin{entry}{卫生署}{wei4sheng1shu3}{3,5,13}[Radicais ⼙、⽣、⽹]
  \definition*{s.}{Agência de Saúde (ou Escritório, ou Departamento)}
\end{entry}

\begin{entry}{卫生套}{wei4sheng1tao4}{3,5,10}[Radicais ⼙、⽣、⼤]
  \definition[只]{s.}{preservativo | camisinha}
\end{entry}

\begin{entry}{卫生厅}{wei4sheng1ting1}{3,5,4}[Radicais ⼙、⽣、⼚]
  \definition*{s.}{Departamento de Saúde (da província)}
\end{entry}

\begin{entry}{卫生纸}{wei4sheng1zhi3}{3,5,7}[Radicais ⼙、⽣、⽷]
  \definition{s.}{papel higiênico}
\end{entry}

\begin{entry}{为}{wei4}{4}[HSK 2,3][Radical ⼂]
  \definition{prep.}{objeto da ação | indicando propósito | indicando razões | para; em direção a}
  \definition{v.}{apoiar; defender}
  \seeref{为}{wei2}
\end{entry}

\begin{entry}{为了}{wei4le5}{4,2}[HSK 3][Radicais ⼂、⼅]
  \definition{conj.}{para; por causa de; a fim de}
\end{entry}

\begin{entry}{为什么}{wei4shen2me5}{4,4,3}[HSK 2][Radicais ⼂、⼈、⼃]
  \definition{adv.}{por que?}
\end{entry}

\begin{entry}{未}{wei4}{5}[Radical ⽊]
  \definition{adv.}{não ter | ainda não}
\end{entry}

\begin{entry}{未必}{wei4bi4}{5,5}[Radicais ⽊、⼼]
  \definition{adv.}{não pode | não necessariamente}
\end{entry}

\begin{entry}{位}{wei4}{7}[HSK 2][Radical ⼈]
  \definition{clas.}{para pessoas (com cortesia) | para bits binários}
  \definition{s.}{(física) potencial | localização | lugar | posição | assento}
  \example{十六位}[16 bits]
\end{entry}

\begin{entry}{位居}{wei4ju1}{7,8}[Radicais ⼈、⼫]
  \definition{v.}{estar localizado em}
\end{entry}

\begin{entry}{位置}{wei4zhi5}{7,13}[Radicais ⼈、⽹]
  \definition[个]{s.}{lugar | posição | assento}
\end{entry}

\begin{entry}{位子}{wei4zi5}{7,3}[Radicais ⼈、⼦]
  \definition{s.}{lugar | assento}
\end{entry}

\begin{entry}{味}{wei4}{8}[Radical ⼝]
  \definition{clas.}{para medicamentos}
  \definition{s.}{cheiro | gosto}
\end{entry}

\begin{entry}{味道}{wei4dao5}{8,12}[HSK 2][Radicais ⼝、⾡]
  \definition{s.}{sabor | (dialeto) odor, cheiro | (figurativo) sentimento (de…), dica (de…) | (figurativo) interesse, prazer}
\end{entry}

\begin{entry}{味儿}{wei4r5}{8,2}[Radicais ⼝、⼉]
  \definition{s.}{sabor}
\end{entry}

\begin{entry}{胃口}{wei4kou3}{9,3}[Radicais ⾁、⼝]
  \definition{s.}{apetite}
\end{entry}

\begin{entry}{喂}{wei4}{12}[HSK 2][Radical ⼝]
  \definition{interj.}{Ei!, para chamar atenção | Alô? (quando respondendo uma chamada telefônica, pronuncia-se como \dpy{wei2})}
  \definition{v.}{alimentar | alimentar (um animal, bebê, inválido, etc.)}
\end{entry}

\begin{entry}{喂哺}{wei4bu3}{12,10}[Radicais ⼝、⼝]
  \definition{v.}{alimentar (um bebê)}
\end{entry}

\begin{entry}{喂料}{wei4liao4}{12,10}[Radicais ⼝、⽃]
  \definition{v.}{alimentar (também no sentido figurativo)}
\end{entry}

\begin{entry}{喂母乳}{wei4mu3ru3}{12,5,8}[Radicais ⼝、⽏、⼄]
  \definition{s.}{amamentação}
\end{entry}

\begin{entry}{喂奶}{wei4nai3}{12,5}[Radicais ⼝、⼥]
  \definition{v.}{amamentar}
\end{entry}

\begin{entry}{喂食}{wei4shi2}{12,9}[Radicais ⼝、⾷]
  \definition{v.}{alimentar}
\end{entry}

\begin{entry}{喂养}{wei4yang3}{12,9}[Radicais ⼝、⼋]
  \definition{v.}{alimentar (uma criança, animal doméstico, etc.) | manter | criar (um animal)}
\end{entry}

\begin{entry}{温度}{wen1du4}{12,9}[HSK 2][Radicais ⽔、⼴]
  \definition[个]{s.}{temperatura}
\end{entry}

\begin{entry}{温度表}{wen1du4biao3}{12,9,8}[Radicais ⽔、⼴、⾐]
  \definition{s.}{termômetro}
\end{entry}

\begin{entry}{温度计}{wen1du4ji4}{12,9,4}[Radicais ⽔、⼴、⾔]
  \definition{s.}{termógrafo | termômetro}
\end{entry}

\begin{entry}{温度梯度}{wen1du4ti1du4}{12,9,11,9}[Radicais ⽔、⼴、⽊、⼴]
  \definition{s.}{gradiente de temperatura}
\end{entry}

\begin{entry}{温暖}{wen1nuan3}{12,13}[HSK 3][Radicais ⽔、⽇]
  \definition{adj.}{caloroso; gentil}
  \definition{v.}{aquecer (fazer você se sentir aquecido)}
\end{entry}

\begin{entry}{温柔}{wen1rou2}{12,9}[Radicais ⽔、⽊]
  \definition{adj.}{gentil e suave | terno | doce (comumente usado para descrever uma menina ou mulher)}
\end{entry}

\begin{entry}{文化}{wen2hua4}{4,4}[HSK 3][Radicais ⽂、⼔]
  \definition[个,种]{s.}{cultura; civilização | cultura; alfabetização; escolaridade; educação}
\end{entry}

\begin{entry}{文化层}{wen2hua4ceng2}{4,4,7}[Radicais ⽂、⼔、⼫]
  \definition{s.}{nível de cultura (em sítio arqueológico)}
\end{entry}

\begin{entry}{文化宫}{wen2hua4gong1}{4,4,9}[Radicais ⽂、⼔、⼧]
  \definition{s.}{palácio cultural}
\end{entry}

\begin{entry}{文化圈}{wen2hua4quan1}{4,4,11}[Radicais ⽂、⼔、⼞]
  \definition{s.}{esfera de influência cultural}
\end{entry}

\begin{entry}{文化热}{wen2hua4re4}{4,4,10}[Radicais ⽂、⼔、⽕]
  \definition{s.}{mania cultural | febre cultural}
\end{entry}

\begin{entry}{文化史}{wen2hua4shi3}{4,4,5}[Radicais ⽂、⼔、⼝]
  \definition*{s.}{História Cultural}
\end{entry}

\begin{entry}{文化水平}{wen2hua4 shui3ping2}{4,4,4,5}[Radicais ⽂、⼔、⽔、⼲]
  \definition{s.}{nível educacional}
\end{entry}

\begin{entry}{文化障碍}{wen2hua4zhang4'ai4}{4,4,13,13}[Radicais ⽂、⼔、⾩、⽯]
  \definition{s.}{barreira cultural}
\end{entry}

\begin{entry}{文件}{wen2jian4}{4,6}[HSK 3][Radicais ⽂、⼈]
  \definition[份,分]{s.}{documentos oficiais; papéis; instrumentos | os arquivos no computador | artigos ou trabalhos sobre teorias políticas, atualidades, pesquisas acadêmicas, etc.}
\end{entry}

\begin{entry}{文明}{wen2ming2}{4,8}[HSK 3][Radicais ⽂、⽇]
  \definition{adj.}{civilizado}
  \definition[个]{s.}{cultura; civilização}
\end{entry}

\begin{entry}{文学}{wen2xue2}{4,8}[HSK 3][Radicais ⽂、⼦]
  \definition[个,种]{s.}{literatura}
\end{entry}

\begin{entry}{文学系}{wen2xue2 xi4}{4,8,7}[Radicais ⽂、⼦、⽷]
  \definition*{s.}{Faculdade de Letras}
\end{entry}

\begin{entry}{文章}{wen2zhang1}{4,11}[HSK 3][Radicais ⽂、⾳]
  \definition[篇,段,页]{s.}{ensaio; dissertação; artigo | significado oculto; significado implícito | trabalho (coisas para fazer)}
\end{entry}

\begin{entry}{文字}{wen2zi4}{4,6}[HSK 3][Radicais ⽂、⼦]
  \definition[种,类,段,行,篇]{s.}{personagens; roteiro; escrita
linguagem escrita}
\end{entry}

\begin{entry}{纹路}{wen2lu4}{7,13}[Radicais ⽷、⾜]
  \definition{s.}{padrão de linhas | rugas | veias | veias (em mármore ou impressão digital) | grãos (em madeira, etc.)}
\end{entry}

\begin{entry}{闻}{wen2}{9}[HSK 2][Radical ⾨]
  \definition*{s.}{sobrenome Wen}
  \definition{s.}{notícias | reputação | fama}
  \definition{v.}{ouvir | cheirar | farejar}
\end{entry}

\begin{entry}{蚊香}{wen2xiang1}{10,9}[Radicais ⾍、⾹]
  \definition{s.}{incenso ou espiral repelente de mosquitos}
\end{entry}

\begin{entry}{蚊子}{wen2zi5}{10,3}[Radicais ⾍、⼦]
  \definition{s.}{pernilongo}
\end{entry}

\begin{entry}{稳定}{wen3ding4}{14,8}[Radicais ⽲、⼧]
  \definition{adj.}{estável}
  \definition{s.}{estabilidade}
  \definition{v.}{estabilizar | pacificar}
\end{entry}

\begin{entry}{问}{wen4}{6}[HSK 1][Radical ⾨]
  \definition{v.}{perguntar}
\end{entry}

\begin{entry}{问安}{wen4'an1}{6,6}[Radicais ⾨、⼧]
  \definition{s.}{saudações}
  \definition{v.}{dar cumprimentos a | prestar homenagem}
\end{entry}

\begin{entry}{问鼎}{wen4ding3}{6,12}[Radicais ⾨、⿍]
  \definition{v.}{visar (o primeiro lugar, etc.) | aspirar ao trono}
\end{entry}

\begin{entry}{问候}{wen4hou4}{6,10}[Radicais ⾨、⼈]
  \definition{s.}{homenagem | saudação}
  \definition{v.}{prestar homenagem |enviar uma saudação | (fig.) (coloquial) fazer referência ofensiva a (alguém querido pela pessoa com quem se está falando)}
\end{entry}

\begin{entry}{问卷}{wen4juan4}{6,8}[Radicais ⾨、⼙]
  \definition[份]{s.}{questionário}
\end{entry}

\begin{entry}{问路}{wen4 lu4}{6,13}[HSK 2][Radicais ⾨、⾜]
  \definition{v.}{perguntar sobre o caminho | pedir por direções}
\end{entry}

\begin{entry}{问市}{wen4shi4}{6,5}[Radicais ⾨、⼱]
  \definition{v.}{chegar ao mercado | bater o mercado | atingir o mercado}
\end{entry}

\begin{entry}{问题}{wen4ti2}{6,15}[HSK 2][Radicais ⾨、⾴]
  \definition[个]{s.}{pergunta | questão | problema}
\end{entry}

\begin{entry}{嗡嗡}{weng1weng1}{13,13}[Radicais ⼝、⼝]
  \definition{s.}{zumbido}
  \definition{v.}{zumbir}
\end{entry}

\begin{entry}{蕹菜}{weng4cai4}{16,11}[Radicais ⾋、⾋]
  \definition{s.}{espinafre aquático | \emph{ong choy} | repolho do pântano | convolvulus aquático | glória-da-manhã aquática}
  \seealsoref{空心菜}{kong1xin1cai4}
\end{entry}

\begin{entry}{我}{wo3}{7}[HSK 1][Radical ⼽]
  \definition{pron.}{eu | me | mim | comigo}
\end{entry}

\begin{entry}{我的}{wo3 de5}{7,8}[Radicais ⼽、⽩]
  \definition{pron.}{meu, meus}
\end{entry}

\begin{entry}{我们}{wo3men5}{7,5}[HSK 1][Radicais ⼽、⼈]
  \definition{pron.}{nós | nos | conosco}
\end{entry}

\begin{entry}{我们的}{wo3men5 de5}{7,5,8}[Radicais ⼽、⼈、⽩]
  \definition{pron.}{nosso, nossos}
\end{entry}

\begin{entry}{我去}{wo3qu4}{7,5}[Radicais ⼽、⼛]
  \definition{interj.}{(gíria) O que\dots!! | Oh meu Deus! | Isso é insano!}
\end{entry}

\begin{entry}{卧}{wo4}{8}[Radical ⾂]
  \definition{v.}{agachar | deitar}
\end{entry}

\begin{entry}{卧病}{wo4bing4}{8,10}[Radicais ⾂、⽧]
  \definition{s.}{acamado | doente na cama}
\end{entry}

\begin{entry}{卧舱}{wo4cang1}{8,10}[Radicais ⾂、⾈]
  \definition{s.}{cabine de dormir em um barco ou trem}
\end{entry}

\begin{entry}{卧车}{wo4che1}{8,4}[Radicais ⾂、⾞]
  \definition{s.}{um carro-leito | vagão-leito}
\end{entry}

\begin{entry}{卧床}{wo4chuang2}{8,7}[Radicais ⾂、⼴]
  \definition{adj.}{acamado}
  \definition{s.}{cama}
  \definition{v.}{deitar na cama}
\end{entry}

\begin{entry}{卧倒}{wo4dao3}{8,10}[Radicais ⾂、⼈]
  \definition{v.}{cair no chão | deitar-se}
\end{entry}

\begin{entry}{卧式}{wo4shi4}{8,6}[Radicais ⾂、⼷]
  \definition{adj.}{horizontal}
\end{entry}

\begin{entry}{卧室}{wo4shi4}{8,9}[Radicais ⾂、⼧]
  \definition[间]{s.}{quarto de dormir}
\end{entry}

\begin{entry}{卧榻}{wo4ta4}{8,14}[Radicais ⾂、⽊]
  \definition{s.}{um sofá | uma cama estreita}
\end{entry}

\begin{entry}{卧推}{wo4tui1}{8,11}[Radicais ⾂、⼿]
  \definition{s.}{supino}
\end{entry}

\begin{entry}{握手}{wo4shou3}{12,4}[HSK 3][Radicais ⼿、⼿]
  \definition{v.+compl.}{apertar as mãos}
\end{entry}

\begin{entry}{斡旋}{wo4xuan2}{14,11}[Radicais ⽃、⽅]
  \definition{v.}{mediar (um conflito, etc.)}
\end{entry}

\begin{entry}{乌龟}{wu1gui1}{4,7}[Radicais ⼃、⿔]
  \definition{s.}{tartaruga}
\end{entry}

\begin{entry}{乌克兰}{wu1ke4lan2}{4,7,5}[Radicais ⼃、⼗、⼋]
  \definition*{s.}{Ucrânia}
\end{entry}

\begin{entry}{污染}{wu1ran3}{6,9}[Radicais ⽔、⽊]
  \definition{s.}{poluição}
  \definition{v.}{poluir}
\end{entry}

\begin{entry}{污染区}{wu1ran3qu1}{6,9,4}[Radicais ⽔、⽊、⼖]
  \definition{s.}{área contaminada}
\end{entry}

\begin{entry}{污染物}{wu1ran3wu4}{6,9,8}[Radicais ⽔、⽊、⽜]
  \definition{s.}{poluente}
  \seealsoref{污染物质}{wu1ran3 wu4zhi4}
\end{entry}

\begin{entry}{污染物质}{wu1ran3 wu4zhi4}{6,9,8,8}[Radicais ⽔、⽊、⽜、⾙]
  \definition{s.}{poluente}
  \seealsoref{污染物}{wu1ran3wu4}
\end{entry}

\begin{entry}{污水}{wu1shui3}{6,4}[Radicais ⽔、⽔]
  \definition{s.}{esgoto}
\end{entry}

\begin{entry}{屋子}{wu1zi5}{9,3}[HSK 3][Radicais ⼫、⼦]
  \definition[间,座,栋]{s.}{casa}
\end{entry}

\begin{entry}{无}{wu2}{4}[Kangxi 71][Radical ⽆]
  \definition{adv.}{não ter algo | não há\dots}
\end{entry}

\begin{entry}{无敌}{wu2di2}{4,10}[Radicais ⽆、⾆]
  \definition{adj.}{invencível | inigualável}
\end{entry}

\begin{entry}{无骨}{wu2 gu3}{4,9}[Radicais ⽆、⾻]
  \definition{adj.}{desossado}
\end{entry}

\begin{entry}{无故}{wu2gu4}{4,9}[Radicais ⽆、⽁]
  \definition{adv.}{sem causa ou razão | sem motivo}
\end{entry}

\begin{entry}{无论……也……}{wu2lun4 ye3}{4,6,3}[Radicais ⽆、⾔、⼄]
  \definition{conj.}{não apenas\dots, (o que, quem, como, etc.), \dots}
\end{entry}

\begin{entry}{无人}{wu2ren2}{4,2}[Radicais ⽆、⼈]
  \definition{adj.}{não tripulado | desabitado}
\end{entry}

\begin{entry}{无人机}{wu2ren2ji1}{4,2,6}[Radicais ⽆、⼈、⽊]
  \definition{s.}{\emph{drone} | veículo aéreo não tripulado}
\end{entry}

\begin{entry}{无视}{wu2shi4}{4,8}[Radicais ⽆、⾒]
  \definition{v.}{ignorar | desconsiderar}
\end{entry}

\begin{entry}{无氧}{wu2yang3}{4,10}[Radicais ⽆、⽓]
  \definition{adj.}{anaeróbico}
\end{entry}

\begin{entry}{吾}{wu2}{7}[Radical ⼝]
  \definition{pron.}{eu | (antigo) meu}
  \definition{s.}{sobrenome Wu}
\end{entry}

\begin{entry}{五}{wu3}{4}[HSK 1][Radical ⼆]
  \definition{num.}{cinco; 5}
\end{entry}

\begin{entry}{五体投地}{wu3ti3tou2di4}{4,7,7,6}[Radicais ⼆、⼈、⼿、⼟]
  \definition{expr.}{prostrar-se em admiração | adular alguém}
\end{entry}

\begin{entry}{五五}{wu3wu3}{4,4}[Radicais ⼆、⼆]
  \definition{num.}{50-50}
  \definition{s.}{igual (partilha, parceria, etc.)}
\end{entry}

\begin{entry}{午}{wu3}{4}[Radical ⼗]
  \definition{s.}{período entre 11h00 e 13h00, meio-dia}
\end{entry}

\begin{entry}{午餐}{wu3 can1}{4,16}[HSK 2][Radicais ⼗、⾷]
  \definition[份,顿,次]{s.}{almoço}
  \seealsoref{午饭}{wu3fan4}
\end{entry}

\begin{entry}{午饭}{wu3fan4}{4,7}[HSK 1][Radicais ⼗、⾷]
  \definition[份,顿,次,餐]{s.}{almoço}
  \seealsoref{午餐}{wu3 can1}
\end{entry}

\begin{entry}{午后}{wu3hou4}{4,6}[Radicais ⼗、⼝]
  \definition{s.}{tarde | período da tarde}
\end{entry}

\begin{entry}{午前}{wu3qian2}{4,9}[Radicais ⼗、⼑]
  \definition{s.}{\emph{A.M.} | manhã | período da manhã}
\end{entry}

\begin{entry}{午睡}{wu3 shui4}{4,13}[HSK 2][Radicais ⼗、⽬]
  \definition{s.}{siesta}
  \definition{v.}{tirar uma soneca}
\end{entry}

\begin{entry}{午休}{wu3xiu1}{4,6}[Radicais ⼗、⼈]
  \definition{s.}{pausa para almoço | cochilo na hora do almoço | intervalo do meio-dia}
\end{entry}

\begin{entry}{午宴}{wu3yan4}{4,10}[Radicais ⼗、⼧]
  \definition{s.}{banquete de almoço}
\end{entry}

\begin{entry}{午夜}{wu3ye4}{4,8}[Radicais ⼗、⼣]
  \definition{s.}{meia-noite}
\end{entry}

\begin{entry}{武}{wu3}{8}[Radical ⽌]
  \definition*{s.}{sobrenome Wu}
  \definition{s.}{arte marcial}
\end{entry}

\begin{entry}{武大戏}{wu3 da4xi4}{8,3,6}[Radicais ⽌、⼤、⼽]
  \definition*{s.}{Drama de Luta Acrobática | Drama Wu}
\end{entry}

\begin{entry}{武断}{wu3duan4}{8,11}[Radicais ⽌、⽄]
  \definition{adj.}{arbitrário | dogmático | subjetivo}
\end{entry}

\begin{entry}{武官}{wu3guan1}{8,8}[Radicais ⽌、⼧]
  \definition{s.}{oficial militar}
\end{entry}

\begin{entry}{武力}{wu3li4}{8,2}[Radicais ⽌、⼒]
  \definition{s.}{forças armadas | militares}
\end{entry}

\begin{entry}{武器}{wu3qi4}{8,16}[HSK 3][Radicais ⽌、⼝]
  \definition[批,种]{s.}{arma; armamento}
\end{entry}

\begin{entry}{武士}{wu3shi4}{8,3}[Radicais ⽌、⼠]
  \definition{s.}{samurai | guerreiro}
\end{entry}

\begin{entry}{武术}{wu3shu4}{8,5}[HSK 3][Radicais ⽌、⽊]
  \definition[种,套,门]{s.}{arte marcial; autodefesa; \emph{wushu}}
\end{entry}

\begin{entry}{武艺}{wu3yi4}{8,4}[Radicais ⽌、⾋]
  \definition{s.}{arte marcial | habilidade militar}
\end{entry}

\begin{entry}{武装}{wu3zhuang1}{8,12}[Radicais ⽌、⾐]
  \definition{s.}{forças armadas | militar | arma}
  \definition{v.}{armar}
\end{entry}

\begin{entry}{舞}{wu3}{14}[Radical ⾇]
  \definition{s.}{dança}
\end{entry}

\begin{entry}{舞抃}{wu3bian4}{14,7}[Radicais ⾇、⼿]
  \definition{s.}{dançar por prazer}
\end{entry}

\begin{entry}{舞蹈}{wu3dao3}{14,17}[Radicais ⾇、⾜]
  \definition{s.}{dança (ato performático)}
\end{entry}

\begin{entry}{舞会}{wu3hui4}{14,6}[Radicais ⾇、⼈]
  \definition{s.}{baile}
\end{entry}

\begin{entry}{舞会舞}{wu3hui4wu3}{14,6,14}[Radicais ⾇、⼈、⾇]
  \definition{s.}{baile}
\end{entry}

\begin{entry}{舞台}{wu3 tai2}{14,5}[HSK 3][Radicais ⾇、⼝]
  \definition[个]{s.}{palco; arena}
\end{entry}

\begin{entry}{舞厅}{wu3ting1}{14,4}[Radicais ⾇、⼚]
  \definition[间]{s.}{salão de dança | salão de baile}
\end{entry}

\begin{entry}{舞厅舞}{wu3ting1wu3}{14,4,14}[Radicais ⾇、⼚、⾇]
  \definition{s.}{dança de salão}
\end{entry}

\begin{entry}{务实}{wu4shi2}{5,8}[Radicais ⼒、⼧]
  \definition{adj.}{pragmático}
  \definition{v.}{lidar com assuntos concretos}
\end{entry}

\begin{entry}{物理}{wu4li3}{8,11}[Radicais ⽜、⽟]
  \definition{s.}{física (disciplina)}
\end{entry}

\begin{entry}{误点}{wu4dian3}{9,9}[Radicais ⾔、⽕]
  \definition{v.+compl.}{atrasar | chegar tarde}
\end{entry}

\begin{entry}{雾气}{wu4qi4}{13,4}[Radicais ⾬、⽓]
  \definition{s.}{nevoeiro | névoa | vapor}
\end{entry}

%%%%% EOF %%%%%


%%%
%%% X
%%%

\section*{X}\addcontentsline{toc}{section}{X}

\begin{entry}{夕阳}{xi1yang2}{3,6}{⼣、⾩}
  \definition{s.}{pôr do sol}
  \seealsoref{日出}{ri4chu1}
\end{entry}

\begin{entry}{吸}{xi1}{6}{⼝}[HSK 4]
  \definition{v.}{inalar; inspirar; aspirar; itroduzir líquidos, gases, etc. no corpo | absorver; sugar | atrair; atrair para si mesmo; atrair (interesse, investimento etc.)}
\end{entry}

\begin{entry}{吸管}{xi1 guan3}{6,14}{⼝、⽵}[HSK 4]
  \definition[根,个]{s.}{tubo de sucção; sugador; canudo (para beber); refere-se ao tubo fino usado para sugar bebidas | conta-gotas; pipeta; cateter para bombeamento de líquidos usando pressão de ar}
\end{entry}

\begin{entry}{吸收}{xi1shou1}{6,6}{⼝、⽁}[HSK 4]
  \definition{v.}{imbuir; absorver; assimilar; sugar;  chupar; (animais, plantas, etc.) extrair material de fora dos tecidos para o interior dos tecidos | absorver; chupar;  sugar alguma substância de fora para dentro | recrutar; alistar; inscrever-se; matricular-se; admitir; (organizações ou coletivos) aceitar novos membros | absorver; aproveitar e usar a experiência, o conhecimento, o dinheiro e outras coisas valiosas de outras pessoas | absorver; diminuir, atenuar ou eliminar determinados efeitos ou fenômenos}
\end{entry}

\begin{entry}{吸铁石}{xi1tie3shi2}{6,10,5}{⼝、⾦、⽯}
  \definition{s.}{imã | magneto}
  \seealsoref{磁铁}{ci2tie3}
\end{entry}

\begin{entry}{吸烟}{xi1yan1}{6,10}{⼝、⽕}[HSK 4]
  \definition{v.+compl.}{fumar}
\end{entry}

\begin{entry}{吸引}{xi1yin3}{6,4}{⼝、⼸}[HSK 4]
  \definition{v.}{atrair; apelar para; chamar a atenção de outros objetos, forças ou pessoas para si mesmo}
\end{entry}

\begin{entry}{西}{xi1}{6}{⾑}[HSK 1]
  \definition{s.}{oeste}
\end{entry}

\begin{entry}{西安}{xi1'an1}{6,6}{⾑、⼧}
  \definition*{s.}{Xi'an}
\end{entry}

\begin{entry}{西班牙文}{xi1ban1ya2wen2}{6,10,4,4}{⾑、⽟、⽛、⽂}
  \definition{s.}{espanhol, língua espanhola}
  \seealsoref{西文}{xi1wen2}
\end{entry}

\begin{entry}{西班牙语}{xi1ban1ya2yu3}{6,10,4,9}{⾑、⽟、⽛、⾔}
  \definition{s.}{espanhol | língua espanhola}
  \seealsoref{西语}{xi1yu3}
\end{entry}

\begin{entry}{西半球}{xi1ban4qiu2}{6,5,11}{⾑、⼗、⽟}
  \definition{s.}{hemisfério oeste}
\end{entry}

\begin{entry}{西北}{xi1 bei3}{6,5}{⾑、⼔}[HSK 2]
  \definition{s.}{noroeste | noroeste da China}
\end{entry}

\begin{entry}{西边}{xi1bian1}{6,5}{⾑、⾡}[HSK 1]
  \definition{adv.}{ao oeste de | oeste | lado oeste | parte ocidental}
\end{entry}

\begin{entry}{西部}{xi1 bu4}{6,10}{⾑、⾢}[HSK 3]
  \definition{s.}{parte ocidental}
\end{entry}

\begin{entry}{西餐}{xi1 can1}{6,16}{⾑、⾷}[HSK 2]
  \definition[分,顿]{s.}{comida ocidental}
\end{entry}

\begin{entry}{西方}{xi1 fang1}{6,4}{⾑、⽅}[HSK 2]
  \definition{s.}{países ocidentais | o Ocidente | o Oeste}
\end{entry}

\begin{entry}{西瓜}{xi1gua1}{6,5}{⾑、⽠}[HSK 4]
  \definition[个,颗,粒]{s.}{melancia; fruto que é uma baga de formato grande, globular ou oval, com muita polpa aguada e doce}
\end{entry}

\begin{entry}{西兰花}{xi1lan2hua1}{6,5,7}{⾑、⼋、⾋}
  \definition{s.}{brócolis}
\end{entry}

\begin{entry}{西蓝花}{xi1lan2hua1}{6,13,7}{⾑、⾋、⾋}
  \variantof{西兰花}
\end{entry}

\begin{entry}{西面}{xi1mian4}{6,9}{⾑、⾯}
  \definition{s.}{oeste | lado oeste}
\end{entry}

\begin{entry}{西南}{xi1 nan2}{6,9}{⾑、⼗}[HSK 2]
  \definition{s.}{sudoeste | sudoeste da China}
\end{entry}

\begin{entry}{西文}{xi1wen2}{6,4}{⾑、⽂}
  \definition{s.}{espanhol | língua espanhola}
  \seealsoref{西班牙文}{xi1ban1ya2wen2}
\end{entry}

\begin{entry}{西西}{xi1xi1}{6,6}{⾑、⾑}
  \definition{num.}{centímetro cúbico}
\end{entry}

\begin{entry}{西医}{xi1 yi1}{6,7}{⾑、⼖}[HSK 2]
  \definition{s.}{medicina ocidental | um médico treinado em medicina ocidental}
\end{entry}

\begin{entry}{西语}{xi1yu3}{6,9}{⾑、⾔}
  \definition{s.}{espanhol | língua espanhola}
  \seealsoref{西班牙语}{xi1ban1ya2yu3}
\end{entry}

\begin{entry}{希望}{xi1wang4}{7,11}{⼱、⽉}[HSK 3]
  \definition[个]{s.}{esperança; desejo; expectativa | aquilo em que a esperança é depositada}
  \definition{v.}{ter esperança; desejar; esperar}
\end{entry}

\begin{entry}{昔日}{xi1ri4}{8,4}{⽇、⽇}
  \definition{adj.}{passado}
\end{entry}

\begin{entry}{牺牲}{xi1sheng1}{10,9}{⽜、⽜}
  \definition{s.}{abate de um animal como sacrifício}
  \definition{v.}{sacrificar a vida de alguém | sacrificar (algo de valor)}
\end{entry}

\begin{entry}{悉尼}{xi1ni2}{11,5}{⼼、⼫}
  \definition*{s.}{Sidney}
\end{entry}

\begin{entry}{悉数}{xi1shu3}{11,13}{⼼、⽁}
  \definition{adv.}{enumerar em detalhes | explicar claramente}
  \seeref{悉数}{xi1shu4}
\end{entry}

\begin{entry}{悉数}{xi1shu4}{11,13}{⼼、⽁}
  \definition{adv.}{todos | cada um | toda a soma}
  \seeref{悉数}{xi1shu3}
\end{entry}

\begin{entry}{悉心}{xi1xin1}{11,4}{⼼、⼼}
  \definition{adv.}{colocar o coração (e a alma) em algo | com muito cuidado}
\end{entry}

\begin{entry}{蜥易}{xi1yi4}{14,8}{⾍、⽇}
  \variantof{蜥蜴}
\end{entry}

\begin{entry}{蜥蜴}{xi1yi4}{14,14}{⾍、⾍}
  \definition{s.}{lagarto}
\end{entry}

\begin{entry}{习惯}{xi2guan4}{3,11}{⼄、⼼}[HSK 2]
  \definition[个]{s.}{hábito | costume | prática usual}
  \definition{v.}{ser acostumado a | ter o hábito de}
\end{entry}

\begin{entry}{席卷}{xi2juan3}{10,8}{⼱、⼙}
  \definition{v.}{engolfar | varrer | levar tudo para fora}
\end{entry}

\begin{entry}{袭击}{xi2ji1}{11,5}{⾐、⼐}
  \definition{s.}{ataque (especialmente um ataque surpresa) | invasão}
  \definition{v.}{atacar}
\end{entry}

\begin{entry}{洗}{xi3}{9}{⽔}[HSK 1]
  \definition{v.}{lavar | revelar (fotos) | tomar banho}
\end{entry}

\begin{entry}{洗涤}{xi3di2}{9,10}{⽔、⽔}
  \definition{s.}{enxágue | lava}
  \definition{v.}{enxaguar | lavar}
\end{entry}

\begin{entry}{洗涤间}{xi3di2jian1}{9,10,7}{⽔、⽔、⾨}
  \definition{s.}{lavanderia}
\end{entry}

\begin{entry}{洗劫}{xi3jie2}{9,7}{⽔、⼒}
  \definition{v.}{saquear | pilhar | roubar}
\end{entry}

\begin{entry}{洗净}{xi3jing4}{9,8}{⽔、⼎}
  \definition{v.}{lavar (limpeza)}
\end{entry}

\begin{entry}{洗礼}{xi3li3}{9,5}{⽔、⽰}
  \definition{s.}{batismo}
  \definition{v.}{batizar}
\end{entry}

\begin{entry}{洗手}{xi3shou3}{9,4}{⽔、⼿}
  \definition{v.}{ir ao banheiro | lavar as mãos}
\end{entry}

\begin{entry}{洗手不干}{xi3shou3bu2gan4}{9,4,4,3}{⽔、⼿、⼀、⼲}
  \definition{v.}{parar totalmente de fazer algo}
\end{entry}

\begin{entry}{洗手池}{xi3shou3chi2}{9,4,6}{⽔、⼿、⽔}
  \definition{s.}{pia de banheiro | lavatório}
  \seealsoref{洗手盆}{xi3shou3pen2}
\end{entry}

\begin{entry}{洗手间}{xi3shou3jian1}{9,4,7}{⽔、⼿、⾨}[HSK 1]
  \definition{s.}{sanitário | toilette | banheiro}
\end{entry}

\begin{entry}{洗手盆}{xi3shou3pen2}{9,4,9}{⽔、⼿、⽫}
  \definition{s.}{pia de banheiro | lavatório}
  \seealsoref{洗手池}{xi3shou3chi2}
\end{entry}

\begin{entry}{洗手乳}{xi3shou3ru3}{9,4,8}{⽔、⼿、⼄}
  \definition{s.}{sabonete líquido para lavar as mãos}
  \seealsoref{洗手液}{xi3shou3ye4}
\end{entry}

\begin{entry}{洗手液}{xi3shou3ye4}{9,4,11}{⽔、⼿、⽔}
  \definition{s.}{sabonete líquido para lavar as mãos}
  \seealsoref{洗手乳}{xi3shou3ru3}
\end{entry}

\begin{entry}{洗脱}{xi3tuo1}{9,11}{⽔、⾁}
  \definition{v.}{limpar | purgar | lavar}
\end{entry}

\begin{entry}{洗碗}{xi3wan3}{9,13}{⽔、⽯}
  \definition{v.}{lavar pratos}
\end{entry}

\begin{entry}{洗胃}{xi3wei4}{9,9}{⽔、⾁}
  \definition{s.}{(medicina) lavagem gástrica}
  \definition{v.}{ter o estômago lavado}
\end{entry}

\begin{entry}{洗衣机}{xi3 yi1 ji1}{9,6,6}{⽔、⾐、⽊}[HSK 2]
  \definition[台]{s.}{máquina de lavar roupa}
\end{entry}

\begin{entry}{洗澡}{xi3zao3}{9,16}{⽔、⽔}[HSK 2]
  \definition{v.+compl.}{tomar banho | duchar-se | lavar-se}
\end{entry}

\begin{entry}{洗澡间}{xi3zao3jian1}{9,16,7}{⽔、⽔、⾨}
  \definition[间]{s.}{banheiro}
\end{entry}

\begin{entry}{喜爱}{xi3 ai4}{12,10}{⼝、⽖}[HSK 4]
  \definition{v.}{gostar; amar; ter afeição por; estar interessado em; ter uma queda ou sentir interesse por pessoas ou coisas}
\end{entry}

\begin{entry}{喜欢}{xi3huan5}{12,6}{⼝、⽋}[HSK 1]
  \definition{v.}{gostar}
\end{entry}

\begin{entry}{喜剧}{xi3ju4}{12,10}{⼝、⼑}
  \definition[部,出]{s.}{uma comédia}
\end{entry}

\begin{entry}{戏}{xi4}{6}{⼽}
  \definition[出,场,台]{s.}{drama | peça de teatro | \emph{show}}
\end{entry}

\begin{entry}{戏法}{xi4fa3}{6,8}{⼽、⽔}
  \definition{s.}{truque de mágica | prestidigitação}
\end{entry}

\begin{entry}{戏剧}{xi4ju4}{6,10}{⼽、⼑}
  \definition{s.}{drama | suspense | teatro}
\end{entry}

\begin{entry}{戏剧般}{xi4ju4ban1}{6,10,10}{⼽、⼑、⾈}
  \definition{adj.}{melodramático}
\end{entry}

\begin{entry}{戏剧编剧}{xi4ju4bian1ju4}{6,10,12,10}{⼽、⼑、⽷、⼑}
  \definition{s.}{dramaturgo}
\end{entry}

\begin{entry}{戏剧化地}{xi4ju4hua4di4}{6,10,4,6}{⼽、⼑、⼔、⼟}
  \definition{adv.}{dramaticamente | teatralmente}
\end{entry}

\begin{entry}{戏剧家}{xi4ju4jia1}{6,10,10}{⼽、⼑、⼧}
  \definition{s.}{dramaturgo}
\end{entry}

\begin{entry}{戏剧效果}{xi4ju4xiao4guo3}{6,10,10,8}{⼽、⼑、⽁、⽊}
  \definition{s.}{efeito dramático}
\end{entry}

\begin{entry}{戏剧性}{xi4ju4xing4}{6,10,8}{⼽、⼑、⼼}
  \definition{adj.}{dramático}
\end{entry}

\begin{entry}{戏剧演出}{xi4ju4yan3chu1}{6,10,14,5}{⼽、⼑、⽔、⼐}
  \definition{s.}{performance dramática}
\end{entry}

\begin{entry}{戏弄}{xi4nong4}{6,7}{⼽、⼶}
  \definition{v.}{zombar de | pregar peças | provocar}
\end{entry}

\begin{entry}{戏耍}{xi4shua3}{6,9}{⼽、⽽}
  \definition{v.}{divertir-me | brincar com | provocar}
\end{entry}

\begin{entry}{戏谑}{xi4xue4}{6,11}{⼽、⾔}
  \definition{v.}{brincar | fazer piadas | ridicularizar}
\end{entry}

\begin{entry}{戏院}{xi4yuan4}{6,9}{⼽、⾩}
  \definition{s.}{teatro}
\end{entry}

\begin{entry}{系}{xi4}{7}{⽷}[HSK 3,4]
  \definition*{s.}{sobrenome Xi}
  \definition{s.}{faculdade (da universidade) | departamento}
  \definition{v.}{sistema; série | departamento; faculdade}
  \definition{v.}{relacionar-se com; suportar; depender de | sentir-se ansioso; estar preocupado | amarrar; prender | ser}
  \seeref{系}{ji4}
\end{entry}

\begin{entry}{系列}{xi4lie4}{7,6}{⽷、⼑}[HSK 4]
  \definition{s.}{série; conjunto; conjunto de coisas relacionadas (matemática)}
\end{entry}

\begin{entry}{系囚}{xi4qiu2}{7,5}{⽷、⼞}
  \definition{s.}{prisioneiro}
\end{entry}

\begin{entry}{系统}{xi4tong3}{7,9}{⽷、⽷}[HSK 4]
  \definition{adj.}{sistemático; organizado}
  \definition[个]{s.}{sistema; relação de tipos semelhantes (ou seja, grupo de coisas semelhantes)}
\end{entry}

\begin{entry}{细}{xi4}{8}{⽷}[HSK 4]
  \definition{adj.}{fino; delgado; esguio; esbelto; em oposição a ``粗'' | fino; em partículas pequenas; grãos pequenos | fino e macio;  um sussuro | fino; requintado; delicado | cuidadoso; detalhado; meticuloso | ínfimo; minúsculo; insignificante; diminuto | jovem; pequeno}
  \seealsoref{粗}{cu1}
\end{entry}

\begin{entry}{细节}{xi4jie2}{8,5}{⽷、⾋}[HSK 4]
  \definition{s.}{detalhe; particularidade; aspectos secundários ou partes sutis de um enredo ou episódios secundários usados em uma obra literária para expressar o caráter de uma pessoa ou as características essenciais de uma coisa}
\end{entry}

\begin{entry}{细菌战}{xi4jun1zhan4}{8,11,9}{⽷、⾋、⼽}
  \definition{s.}{guerra biológica}
\end{entry}

\begin{entry}{细致}{xi4zhi4}{8,10}{⽷、⾄}[HSK 4]
  \definition{adj.}{meticuloso; cuidadoso; minucioso | intrincado; delicado}
\end{entry}

\begin{entry}{虾}{xia1}{9}{⾍}
  \definition{s.}{camarão}
\end{entry}

\begin{entry}{下}{xia4}{3}{⼀}[HSK 1,2]
  \definition{adv.}{abaixo | em baixo de}
  \definition{clas.}{para número de vezes para ações}
  \definition{v.}{descer | chegar a (uma decisão, conclusão, etc.) | recusar}
\end{entry}

\begin{entry}{下巴}{xia4ba5}{3,4}{⼀、⼰}
  \definition[个]{s.}{queixo}
\end{entry}

\begin{entry}{下班}{xia4 ban1}{3,10}{⼀、⽟}[HSK 1]
  \definition{v.+compl.}{sair do trabalho}
\end{entry}

\begin{entry}{下边}{xia4bian5}{3,5}{⼀、⾡}[HSK 1]
  \definition{adv.}{em baixo | abaixo | parte de baixo}
\end{entry}

\begin{entry}{下车}{xia4 che1}{3,4}{⼀、⾞}[HSK 1]
  \definition{v.}{descer ou sair (de ônibus, carro, etc.)}
\end{entry}

\begin{entry}{下次}{xia4 ci4}{3,6}{⼀、⽋}[HSK 1]
  \definition{s.}{próxima vez}
\end{entry}

\begin{entry}{下蛋}{xia4dan4}{3,11}{⼀、⾍}
  \definition{v.}{botar ovos}
\end{entry}

\begin{entry}{下个月}{xia4 ge4 yue4}{3,3,4}{⼀、⼈、⽉}[HSK 4]
  \definition{s.}{próximo mês; mês que vem; refere-se ao próximo mês do mês atual}
\end{entry}

\begin{entry}{下海}{xia4hai3}{3,10}{⼀、⽔}
  \definition{v.+compl.}{ir para o mar; (barco) deixar o porto e iniciar uma jornada | ir pescar no mar | tornar-se ator profissional}
\end{entry}

\begin{entry}{下降}{xia4 jiang4}{3,8}{⼀、⾩}[HSK 4]
  \definition{v.}{cair; despencar; declinar; descer; diminuir; ir para baixo}
\end{entry}

\begin{entry}{下课}{xia4 ke4}{3,10}{⼀、⾔}[HSK 1]
  \definition{v.+compl.}{acabar a aula | terminar a aula}
\end{entry}

\begin{entry}{下来}{xia4 lai5}{3,7}{⼀、⽊}[HSK 3]
  \definition{part.}{usado depois de um verbo para indicar que uma ação ou comportamento está se movendo em direção ao falante ou que a ação está continuando ou sendo concluída | usado depois de um adjetivo para indicar que um certo estado começou a aparecer e continuará a se desenvolver.}
  \definition{v.}{descer (para a minha localização) | (colheitas/frutas/vegetais, etc.) ser colhido; estar maduro o suficiente para ser colhido | (período de tempo) acabar; passar; chegar ao fim}
\end{entry}

\begin{entry}{下楼}{xia4 lou2}{3,13}{⼀、⽊}[HSK 4]
  \definition{v.}{descer as escadas}
\end{entry}

\begin{entry}{下面}{xia4 mian4}{3,9}{⼀、⾯}[HSK 3]
  \definition{s.}{em baixo; abaixo; parte de baixo | próximo; seguinte | subordinado; o nível inferior; homens nos níveis inferiores}
  \definition{v.}{cozinhar macarrão}
\end{entry}

\begin{entry}{下去}{xia4 qu4}{3,5}{⼀、⼛}[HSK 3]
  \definition{part.}{usado depois de verbos para indicar de alto a baixo | usado depois de um verbo para indicar continuação}
  \definition{v.}{descer (a partir da minha localização)
continuar
obter; crescer; tornar-se}
\end{entry}

\begin{entry}{下水道}{xia4shui3dao4}{3,4,12}{⼀、⽔、⾡}
  \definition{s.}{esgoto}
\end{entry}

\begin{entry}{下午}{xia4wu3}{3,4}{⼀、⼗}[HSK 1]
  \definition{adv.}{tarde | à tarde | período logo após o meio-dia}
\end{entry}

\begin{entry}{下午茶}{xia4wu3cha2}{3,4,9}{⼀、⼗、⾋}
  \definition{s.}{chá da tarde (normalmente chás com doces)}
\end{entry}

\begin{entry}{下线}{xia4xian4}{3,8}{⼀、⽷}
  \definition{v.}{ficar \emph{offline} | (um produto) sair da linha de montagem | pessoa abaixo de si em um esquema de pirâmide}
\end{entry}

\begin{entry}{下雪}{xia4 xue3}{3,11}{⼀、⾬}[HSK 2]
  \definition[场,次]{s.}{neve}
  \definition{v.+compl.}{nevar}
\end{entry}

\begin{entry}{下旬}{xia4xun2}{3,6}{⼀、⽇}
  \definition{adv.}{última dezena do mês}
\end{entry}

\begin{entry}{下雨}{xia4 yu3}{3,8}{⼀、⾬}[HSK 1]
  \definition{v.+compl.}{chover}
\end{entry}

\begin{entry}{下载}{xia4zai3}{3,10}{⼀、⾞}[HSK 4]
  \definition{v.}{\emph{download}; baixar; salvar informações da \emph{Web} em um dispositivo, como um computador}
\end{entry}

\begin{entry}{下崽}{xia4zai3}{3,12}{⼀、⼭}
  \definition{v.}{dar à luz (animais) | parir}
\end{entry}

\begin{entry}{下周}{xia4 zhou1}{3,8}{⼀、⼝}[HSK 2]
  \definition{s.}{próxima semana}
\end{entry}

\begin{entry}{吓人}{xia4ren2}{6,2}{⼝、⼈}
  \definition{adj.}{apavorante | assustador}
  \definition{v.+compl.}{assustar-se | tomar um susto}
\end{entry}

\begin{entry}{夏季}{xia4 ji4}{10,8}{⼢、⼦}[HSK 4]
  \definition{s.}{verão; segundo trimestre do ano, habitualmente chamado na China de período de três meses, do início do verão ao início do outono, também chamado de ``quarto, quinto e sexto'' meses do calendário lunar}
\end{entry}

\begin{entry}{夏日}{xia4ri4}{10,4}{⼢、⽇}
  \definition{s.}{horário de verão}
\end{entry}

\begin{entry}{夏天}{xia4 tian1}{10,4}{⼢、⼤}[HSK 2]
  \definition[个]{s.}{verão}
\end{entry}

\begin{entry}{仙}{xian1}{5}{⼈}
  \definition{s.}{imortal}
\end{entry}

\begin{entry}{先}{xian1}{6}{⼉}[HSK 1]
  \definition{adv.}{em primeiro lugar | primeiramente | antes do tempo | de antemão}
\end{entry}

\begin{entry}{先不先}{xian1bu4xian1}{6,4,6}{⼉、⼀、⼉}
  \definition{adv.}{(dialeto) antes de tudo | em primeiro lugar}
\end{entry}

\begin{entry}{先到先得}{xian1dao4xian1de2}{6,8,6,11}{⼉、⼑、⼉、⼻}
  \definition{expr.}{primeiro a chegar | primeiro a ser servido}
\end{entry}

\begin{entry}{先进}{xian1jin4}{6,7}{⼉、⾡}[HSK 3]
  \definition{adj.}{avançado}
  \definition{s.}{indivíduo avançado; grupo avançado}
\end{entry}

\begin{entry}{先烈}{xian1lie4}{6,10}{⼉、⽕}
  \definition{s.}{mártir}
\end{entry}

\begin{entry}{先期}{xian1qi1}{6,12}{⼉、⽉}
  \definition{adv.}{antecipadamente}
  \definition{s.}{prematuro | \emph{front-end}}
\end{entry}

\begin{entry}{先生}{xian1sheng5}{6,5}{⼉、⽣}[HSK 1]
  \definition[位]{s.}{senhor | marido | professor | (dialeto) doutor}
\end{entry}

\begin{entry}{先天}{xian1tian1}{6,4}{⼉、⼤}
  \definition{adj.}{congênito | inato | natural}
  \definition{s.}{período embrionário}
\end{entry}

\begin{entry}{先验}{xian1yan4}{6,10}{⼉、⾺}
  \definition{adj.}{(filosofia) a priori}
\end{entry}

\begin{entry}{先有}{xian1you3}{6,6}{⼉、⽉}
  \definition{adj.}{preexistente | anterior}
\end{entry}

\begin{entry}{鲜}{xian1}{14}{⿂}[HSK 4]
  \definition*{s.}{sobrenome Xian}
  \definition{adj.}{fresco; novo; fresco (experiência, comida etc.) |brilhante; de cores vivas | saboroso; delicioso | exuberante; luxuriante}
  \definition{s.}{aves e animais recém-abatidos; vegetais recém-colhidos; frutas, etc. | alimentos aquáticos; geralmente, peixes vivos, camarões, etc., para alimentação}
  \seeref{鲜}{xian3}
\end{entry}

\begin{entry}{鲜花}{xian1 hua1}{14,7}{⿂、⾋}[HSK 4]
  \definition[朵,束,支,捧]{s.}{flor; flores frescas; flores bonitas e frescas}
\end{entry}

\begin{entry}{鲜明}{xian1ming2}{14,8}{⿂、⽇}[HSK 4]
  \definition{adj.}{brilhante (cor) | distinto; bem definido; nítido; claro; característico}
\end{entry}

\begin{entry}{咸}{xian2}{9}{⼝}[HSK 4]
  \definition*{s.}{sobrenome Xian}
  \definition{adj.}{salgado; em conserva; sabor salgado}
  \definition{adv.}{todos; indica a totalidade de um intervalo, equivalente a ``全'' e ``都''}
  \seealsoref{都}{dou1}
  \seealsoref{全}{quan2}
\end{entry}

\begin{entry}{咸菜}{xian2cai4}{9,11}{⼝、⾋}
  \definition{s.}{legumes salgados | \emph{pickles}}
\end{entry}

\begin{entry}{咸淡}{xian2dan4}{9,11}{⼝、⽔}
  \definition{s.}{água salobra | grau de salinidade | salgado e sem sal (sabores)}
\end{entry}

\begin{entry}{咸肉}{xian2rou4}{9,6}{⼝、⾁}
  \definition{s.}{\emph{bacon} | carne curada com sal}
\end{entry}

\begin{entry}{咸涩}{xian2se4}{9,10}{⼝、⽔}
  \definition{s.}{ácido | salgado e amargo}
\end{entry}

\begin{entry}{咸水}{xian2shui3}{9,4}{⼝、⽔}
  \definition{s.}{salmora | água salgada}
\end{entry}

\begin{entry}{咸盐}{xian2yan2}{9,10}{⼝、⽫}
  \definition{s.}{(coloquial) sal | sal de mesa}
\end{entry}

\begin{entry}{咸鱼}{xian2yu2}{9,8}{⼝、⿂}
  \definition{s.}{peixe salgado}
\end{entry}

\begin{entry}{显得}{xian3de5}{9,11}{⽇、⼻}[HSK 3]
  \definition{v.}{parecer; aparecer}
\end{entry}

\begin{entry}{显然}{xian3ran2}{9,12}{⽇、⽕}[HSK 3]
  \definition{adj.}{claro; evidente; óbvio}
  \definition{adv.}{claramente; evidentemente; obviamente}
\end{entry}

\begin{entry}{显示}{xian3shi4}{9,5}{⽇、⽰}[HSK 3]
  \definition{v.}{mostrar | exibir}
\end{entry}

\begin{entry}{显著}{xian3zhu4}{9,11}{⽇、⽬}[HSK 4]
  \definition{adj.}{notável; significativo; notável; extraordinário; muito óbvio; muito claramente demonstrado; muito facilmente visto ou sentido}
\end{entry}

\begin{entry}{猃狁}{xian3yun3}{10,7}{⽝、⽝}
  \definition*{s.}{Termo da dinastia Zhou para uma tribo nômade do norte mais tarde chamou o Xiongnu (匈奴) nas dinastias Qin e Han}
  \seealsoref{匈奴}{xiong1nu2}
\end{entry}

\begin{entry}{鲜}{xian3}{14}{⿂}
  \definition{adj.}{raro; pouco; pequeno;}
  \definition{adv.}{raramente}
  \seeref{鲜}{xian1}
\end{entry}

\begin{entry}{见}{xian4}{4}{⾒}
  \definition{v.}{aparecer | também escrito como 现}
  \seeref{见}{jian4}
  \seeref{现}{xian4}
\end{entry}

\begin{entry}{县}{xian4}{7}{⼛}[HSK 4]
  \definition[个]{s.}{condado; unidade de divisão administrativa}
\end{entry}

\begin{entry}{现}{xian4}{8}{⾒}
  \definition{adj.}{presente | atual}
  \definition{v.}{aparecer}
  \seeref{见}{xian4}
\end{entry}

\begin{entry}{现场}{xian4chang3}{8,6}{⾒、⼟}[HSK 3]
  \definition[个,处]{s.}{cena (de um incidente) | local; lugar; sítio}
\end{entry}

\begin{entry}{现代}{xian4dai4}{8,5}{⾒、⼈}[HSK 3]
  \definition*{s.}{Hyundai, empresa sul-coreana}
  \definition{adj.}{moderno; contemporâneo}
  \definition{s.}{tempos modernos; era contemporânea}
\end{entry}

\begin{entry}{现货}{xian4huo4}{8,8}{⾒、⾙}
  \definition{s.}{produtos à vista}
\end{entry}

\begin{entry}{现货的}{xian4huo4 de5}{8,8,8}{⾒、⾙、⽩}
  \definition{s.}{produtos em estoque}
\end{entry}

\begin{entry}{现金}{xian4jin1}{8,8}{⾒、⾦}[HSK 3]
  \definition[笔]{s.}{dinheiro; dinheiro vivo | reserva de dinheiro em um banco}
\end{entry}

\begin{entry}{现实}{xian4shi2}{8,8}{⾒、⼧}[HSK 3]
  \definition{adj.}{real; atual}
  \definition[个]{s.}{realidade; atualidade}
\end{entry}

\begin{entry}{现象}{xian4xiang4}{8,11}{⾒、⾗}[HSK 3]
  \definition[个,种]{s.}{aparência (das coisas); fenômeno}
\end{entry}

\begin{entry}{现有}{xian4you3}{8,6}{⾒、⽉}
  \definition{adj.}{disponível atualmente | atualmente existente}
\end{entry}

\begin{entry}{现在}{xian4zai4}{8,6}{⾒、⼟}[HSK 1]
  \definition{adv.}{agora | neste momento}
\end{entry}

\begin{entry}{现抓}{xian4zhua1}{8,7}{⾒、⼿}
  \definition{v.}{improvisar}
\end{entry}

\begin{entry}{现做}{xian4zuo4}{8,11}{⾒、⼈}
  \definition{adj.}{fresco}
  \definition{v.}{fazer (comida) no local}
\end{entry}

\begin{entry}{线}{xian4}{8}{⽷}[HSK 3]
  \definition{clas.}{para coisas abstratas, o número é limitado a ``一''}
  \definition{s.}{fio; corda; arame | linha | feito de fio de algodão | algo em forma de linha, fio, etc. | rota; linha | linha de demarcação; limite | beira; borda | linha ideológica e política | pista; fio}
\end{entry}

\begin{entry}{线香}{xian4xiang1}{8,9}{⽷、⾹}
  \definition{s.}{bastão ou vareta de incenso}
\end{entry}

\begin{entry}{限制}{xian4zhi4}{8,8}{⾩、⼑}[HSK 4]
  \definition{s.}{limite; restrição; confinamento}
  \definition{v.}{limitar; adstringir; restringir; confinar; fechar em (sobre)}
\end{entry}

\begin{entry}{宪法法院}{xian4fa3fa3yuan4}{9,8,8,9}{⼧、⽔、⽔、⾩}
  \definition{s.}{tribunal constitucional}
\end{entry}

\begin{entry}{宪政}{xian4zheng4}{9,9}{⼧、⽁}
  \definition{s.}{governo constitucional}
\end{entry}

\begin{entry}{宪制}{xian4zhi4}{9,8}{⼧、⼑}
  \definition{adj.}{constitucional}
  \definition{s.}{sistema de governo constitucional}
\end{entry}

\begin{entry}{陷入}{xian4ru4}{10,2}{⾩、⼊}
  \definition{v.}{afundar | ser pego em | pousar (em uma situação)}
\end{entry}

\begin{entry}{羡慕}{xian4mu4}{12,14}{⽺、⼼}
  \definition{v.}{invejar | admirar}
\end{entry}

\begin{entry}{乡巴佬}{xiang1ba1lao3}{3,4,8}{⼄、⼰、⼈}
  \definition{s.}{aldeão | caipira}
\end{entry}

\begin{entry}{乡村}{xiang1cun1}{3,7}{⼄、⽊}
  \definition{adj.}{rural | rústico}
  \definition{s.}{vila | campo}
\end{entry}

\begin{entry}{相比}{xiang1 bi3}{9,4}{⽬、⽐}[HSK 3]
  \definition{v.}{combinar; comparar com |comparar uma coisa com outra, usar uma coisa como padrão para ver as características de outra coisa ou para obter um ponto de vista}
\end{entry}

\begin{entry}{相处}{xiang1chu3}{9,5}{⽬、⼡}[HSK 4]
  \definition{v.}{dar-se bem; viver juntos; dar-se bem (uns com os outros); viver uns com os outros; entrar em contato uns com os outros, tratar uns aos outros}
\end{entry}

\begin{entry}{相当}{xiang1dang1}{9,6}{⽬、⼹}[HSK 3]
  \definition{adj.}{adequado; ajustado; apropriado}
  \definition{adv.}{bastante; razoavelmente; consideravelmente}
  \definition{v.}{combinar; equilibrar; corresponder a; ser aproximadamente igual a; ser compatível com}
\end{entry}

\begin{entry}{相反}{xiang1fan3}{9,4}{⽬、⼜}[HSK 4]
  \definition{adj.}{oposto; contrário; dois aspectos das coisas são contraditórios e mutuamente exclusivos}
  \definition{conj.}{pelo contrário; usado no início ou no meio de uma frase para indicar uma contradição de significado com o que foi dito anteriormente.}
\end{entry}

\begin{entry}{相关}{xiang1guan1}{9,6}{⽬、⼋}[HSK 3]
  \definition{v.}{mutuamente relacionados; inter-relacionados}
\end{entry}

\begin{entry}{相互}{xiang1 hu4}{9,4}{⽬、⼆}[HSK 3]
  \definition{adj.}{mútuo; recíproco}
  \definition{adv.}{mutuamente; um ao outro}
\end{entry}

\begin{entry}{相聚}{xiang1ju4}{9,14}{⽬、⽿}
  \definition{v.}{reunir-se | montar}
\end{entry}

\begin{entry}{相亲}{xiang1qin1}{9,9}{⽬、⼇}
  \definition{s.}{encontro às cegas | entrevista arranjada para avaliar a proposta de um parceiro de casamento | apegar-se profundamente um ao outro}
\end{entry}

\begin{entry}{相思病}{xiang1si1bing4}{9,9,10}{⽬、⼼、⽧}
  \definition{s.}{saudade de amor}
\end{entry}

\begin{entry}{相似}{xiang1si4}{9,6}{⽬、⼈}[HSK 3]
  \definition{v.}{assemelhar-se; ser semelhante; ser igual}
\end{entry}

\begin{entry}{相同}{xiang1tong2}{9,6}{⽬、⼝}[HSK 2]
  \definition{adj.}{igual | idêntico | o mesmo}
\end{entry}

\begin{entry}{相信}{xiang1xin4}{9,9}{⽬、⼈}[HSK 2]
  \definition{v.}{acreditar | estar convencido | aceitar como verdadeiro}
\end{entry}

\begin{entry}{相宜}{xiang1yi2}{9,8}{⽬、⼧}
  \definition{adj.}{adequado | apropriado}
  \definition{v.}{ser adequado ou apropriado}
\end{entry}

\begin{entry}{相遇}{xiang1yu4}{9,12}{⽬、⾡}
  \definition{v.}{encontrar (reunião, encontro, etc.)}
\end{entry}

\begin{entry}{香}{xiang1}{9}{⾹}[HSK 3][Kangxi 186]
  \definition*{s.}{sobrenome Xiang}
  \definition{adj.}{aromático; perfumado; fragrante; cheiroso | saboroso; saboroso; delicioso; apetitoso | com gosto; com bom apetite | (sono) profundo | popular; bem-vindo}
  \definition[束,根,炷]{s.}{especiaria; perfume; fragrância; aromatizante | incenso | relacionado a mulheres ou às próprias mulheres}
\end{entry}

\begin{entry}{香槟酒}{xiang1bin1jiu3}{9,14,10}{⾹、⽊、⾣}
  \definition[杯]{s.}{(empréstimo linguístico) \emph{champagne}}
\end{entry}

\begin{entry}{香波}{xiang1bo1}{9,8}{⾹、⽔}
  \definition{s.}{xampu}
\end{entry}

\begin{entry}{香肠}{xiang1chang2}{9,7}{⾹、⾁}
  \definition[根]{s.}{salsicha}
\end{entry}

\begin{entry}{香港}{xiang1gang3}{9,12}{⾹、⽔}
  \definition*{s.}{Hong Kong}
  \seealsoref{香港岛}{xiang1gang3 dao3}
\end{entry}

\begin{entry}{香港岛}{xiang1gang3 dao3}{9,12,7}{⾹、⽔、⼭}
  \definition*{s.}{Ilha de Hong Kong}
  \seealsoref{香港}{xiang1gang3}
\end{entry}

\begin{entry}{香蕉}{xiang1jiao1}{9,15}{⾹、⾋}[HSK 3]
  \definition[枝,根,个,把,串,束,弓]{s.}{banana}
\end{entry}

\begin{entry}{香炉}{xiang1lu2}{9,8}{⾹、⽕}
  \definition{s.}{incensário (para queimar incenso) | queimador de incenso | insensório, turíbulo}
\end{entry}

\begin{entry}{香气}{xiang1qi4}{9,4}{⾹、⽓}
  \definition{s.}{fragrância | aroma | incenso}
\end{entry}

\begin{entry}{香味}{xiang1wei4}{9,8}{⾹、⼝}
  \definition[股]{s.}{fragrância | cheiro doce}
\end{entry}

\begin{entry}{香蕈}{xiang1xun4}{9,15}{⾹、⾋}
  \definition{s.}{\emph{shiitake}, cogumelo comestível}
\end{entry}

\begin{entry}{香烟}{xiang1yan1}{9,10}{⾹、⽕}
  \definition[支,条]{s.}{cigarro | fumaça de incenso queimado}
\end{entry}

\begin{entry}{香艳}{xiang1yan4}{9,10}{⾹、⾊}
  \definition{adj.}{atraente | erótico | romântico}
\end{entry}

\begin{entry}{香皂}{xiang1zao4}{9,7}{⾹、⽩}
  \definition{s.}{sabonete | sabonete perfumado}
\end{entry}

\begin{entry}{箱}{xiang1}{15}{⾋}[HSK 4]
  \definition{s.}{caixa; estojo; baú | qualquer coisa no formato de caixa}
\end{entry}

\begin{entry}{箱子}{xiang1 zi5}{15,3}{⾋、⼦}[HSK 4]
  \definition[个,只]{s.}{baú; caixa; estojo; maleta; pasta executiva}
\end{entry}

\begin{entry}{享受}{xiang3shou4}{8,8}{⼇、⼜}
  \definition[种]{s.}{prazer}
  \definition{v.}{desfrutar | viver}
\end{entry}

\begin{entry}{响}{xiang3}{9}{⼝}[HSK 2]
  \definition{adj.}{barulhento}
  \definition[声,阵]{s.}{som | barulho | eco}
  \definition{v.}{fazer um som | soar | tocar}
\end{entry}

\begin{entry}{想}{xiang3}{13}{⼼}[HSK 1]
  \definition{v.}{acreditar | sentir falta (sentir-se melancólico com a ausência de alguém ou algo) | supor | pensar | querer | desejar}
\end{entry}

\begin{entry}{想到}{xiang3 dao4}{13,8}{⼼、⼑}[HSK 2]
  \definition{v.}{pensar em | trazer à mente | ter no coração}
\end{entry}

\begin{entry}{想法}{xiang3 fa3}{13,8}{⼼、⽔}[HSK 2]
  \definition[个]{s.}{noção | opinião | jeito de pensar}
  \definition{s.}{maneira de pensar | opinião | noção}
  \definition{v.}{pensar em uma maneira (de fazer algo)}
\end{entry}

\begin{entry}{想念}{xiang3nian4}{13,8}{⼼、⼼}[HSK 4]
  \definition{v.}{sentir falta; pensar em; lembrar com carinho; ficar doente por; desejar ver novamente; lembrar com saudade}
\end{entry}

\begin{entry}{想起}{xiang3 qi3}{13,10}{⼼、⾛}[HSK 2]
  \definition{v.}{recordar | lembrar | pensar em | trazer à mente | cruzar pelos pensamentos de alguém | passar pelo pensamento de alguém}
\end{entry}

\begin{entry}{想想看}{xiang3xiang3kan4}{13,13,9}{⼼、⼼、⽬}
  \definition{v.}{pensar sobre isso}
\end{entry}

\begin{entry}{想象}{xiang3xiang4}{13,11}{⼼、⾗}[HSK 4]
  \definition[个]{s.}{imaginação; refere-se ao processo mental de processamento e transformação de representações armazenadas na mente para formar novas imagens}
  \definition{v.}{imaginar; vislumbrar; visualizar; refere-se a ter uma imagem concreta de algo que não está na frente dos olhos}
\end{entry}

\begin{entry}{向}{xiang4}{6}{⼝}[HSK 2]
  \definition*{s.}{sobrenome Xiang}
  \definition{prep.}{para}
  \definition{v.}{enfrentar | virar para | apoiar}
\end{entry}

\begin{entry}{向汪}{xiang4wang1}{6,7}{⼝、⽔}
  \definition{v.}{esperar que}
\end{entry}

\begin{entry}{向往}{xiang4wang3}{6,8}{⼝、⼻}
  \definition{v.}{ansiar por | esperar ansiosamente por}
\end{entry}

\begin{entry}{相机}{xiang4 ji1}{9,6}{⽬、⽊}[HSK 2]
  \definition[台,个]{s.}{câmera | máquina fotográfica}
  \definition{v.}{ficar atento a uma oportunidade}
\end{entry}

\begin{entry}{相片}{xiang4 pian4}{9,4}{⽬、⽚}[HSK 4]
  \definition[张]{s.}{foto; fotografia; uma imagem de uma pessoa ou objeto feita pela exposição de papel fotográfico a um negativo fotográfico e, em seguida, revelando e fixando a imagem.}
\end{entry}

\begin{entry}{项}{xiang4}{9}{⾴}[HSK 4]
  \definition*{s.}{sobrenome Xiang}
  \definition{clas.}{para itens discriminados; taxonomia}
  \definition{s.}{nuca (do pescoço); a parte de trás do pescoço |
soma (de dinheiro); fundos para fins especiais |
termo; em álgebra, significa uma única fórmula que não é unida por um sinal de mais ou de menos | item}
\end{entry}

\begin{entry}{项目}{xiang4mu4}{9,5}{⾴、⽬}[HSK 4]
  \definition{s.}{evento | item; projeto; trabalhos de engenharia, acadêmicos, etc., de conteúdo específico}
\end{entry}

\begin{entry}{像}{xiang4}{13}{⼈}[HSK 2]
  \definition{s.}{imagem | retrato | aparência}
  \definition{v.}{assemelhar-se | ser como}
\end{entry}

\begin{entry}{消防}{xiao1fang2}{10,6}{⽔、⾩}
  \definition{s.}{combate a incêncios | controle de incêndios}
\end{entry}

\begin{entry}{消防员}{xiao1fang2yuan2}{10,6,7}{⽔、⾩、⼝}
  \definition{s.}{bombeiro}
\end{entry}

\begin{entry}{消费}{xiao1fei4}{10,9}{⽔、⾙}[HSK 3]
  \definition{v.}{gastar; consumir | consumir (recursos naturais)}
\end{entry}

\begin{entry}{消化}{xiao1hua4}{10,4}{⽔、⼔}[HSK 4]
  \definition{v.}{digerir (alimentos) | digerir (conhecimento); pensar e absorver; uma metáfora para a compreensão total de novos conhecimentos ou informações e a capacidade de transformá-los em algo que possa ser usado}
\end{entry}

\begin{entry}{消失}{xiao1shi1}{10,5}{⽔、⼤}[HSK 3]
  \definition{v.}{desaparecer; desvanecer; dissolver; dissipar; evaporar; sumir}
\end{entry}

\begin{entry}{消息}{xiao1xi5}{10,10}{⽔、⼼}[HSK 3]
  \definition[个,条,篇]{s.}{notícias; informação}
\end{entry}

\begin{entry}{销售}{xiao1shou4}{12,11}{⾦、⼝}[HSK 4]
  \definition{v.}{vender; comercializar}
\end{entry}

\begin{entry}{嚣张}{xiao1zhang1}{18,7}{⼝、⼸}
  \definition{adj.}{desenfreado | arrogante | agressivo}
\end{entry}

\begin{entry}{小}{xiao3}{3}{⼩}[HSK 1,2][Kangxi 42]
  \definition{adj.}{pequeno | jovem}
\end{entry}

\begin{entry}{小白菜}{xiao3bai2cai4}{3,5,11}{⼩、⽩、⾋}
  \definition[棵]{s.}{\emph{bok choy} | couve chinesa}
\end{entry}

\begin{entry}{小吃}{xiao3chi1}{3,6}{⼩、⼝}[HSK 4]
  \definition{s.}{lanche; petiscos; comida com especialidades locais, não muito para uma porção | prato frio; prato feito; cortes de frios na culinária ocidental | pratos pequenos e baratos; pratos simples em restaurantes com porções pequenas e preços baixos}
\end{entry}

\begin{entry}{小狗}{xiao3 gou3}{3,8}{⼩、⽝}
  \definition{s.}{filhote de cachorro}
\end{entry}

\begin{entry}{小孩儿}{xiao3hai2r5}{3,9,2}{⼩、⼦、⼉}[HSK 1]
  \definition[个]{s.}{criança | bebê}
\end{entry}

\begin{entry}{小伙子}{xiao3huo3zi5}{3,6,3}{⼩、⼈、⼦}[HSK 4]
  \definition[个]{s.}{rapaz jovem; jovem colega}
\end{entry}

\begin{entry}{小姐}{xiao3jie5}{3,8}{⼩、⼥}[HSK 1]
  \definition[个,位]{s.}{senhorita | jovem senhora | (gíria) prostituta}
\end{entry}

\begin{entry}{小朋友}{xiao3peng2you3}{3,8,4}{⼩、⽉、⼜}[HSK 1]
  \definition{s.}{criança | [termo de tratamento usado por um adulto para uma criança] amiguinho}
\end{entry}

\begin{entry}{小气鬼}{xiao3qi4gui3}{3,4,9}{⼩、⽓、⿁}
  \definition{adj.}{avarento | mão-de-vaca | miserável | pão-duro}
\end{entry}

\begin{entry}{小区}{xiao3qu1}{3,4}{⼩、⼖}
  \definition{s.}{conjunto habitacional, comunidade, bairro | célula (telecomunicações)}
\end{entry}

\begin{entry}{小声}{xiao3 sheng1}{3,7}{⼩、⼠}[HSK 2]
  \definition{v.}{falar em voz baixa | sussurar}
\end{entry}

\begin{entry}{小时}{xiao3shi2}{3,7}{⼩、⽇}[HSK 1]
  \definition{adv.}{hora | para horas}
  \definition[个]{s.}{hora}
\end{entry}

\begin{entry}{小时候}{xiao3 shi2 hou5}{3,7,10}{⼩、⽇、⼈}[HSK 2]
  \definition{s.}{na infância | quando alguém era jovem}
\end{entry}

\begin{entry}{小树}{xiao3shu4}{3,9}{⼩、⽊}
  \definition[棵]{s.}{muda | arbusto | árvore pequena}
\end{entry}

\begin{entry}{小说}{xiao3shuo1}{3,9}{⼩、⾔}[HSK 2]
  \definition[本,部]{s.}{romance | ficção}
\end{entry}

\begin{entry}{小腿}{xiao3tui3}{3,13}{⼩、⾁}
  \definition{s.}{perna (do joelho ao calcanhar) | haste}
\end{entry}

\begin{entry}{小屋}{xiao3wu1}{3,9}{⼩、⼫}
  \definition{s.}{cabana | chalé | cabine}
\end{entry}

\begin{entry}{小小}{xiao3xiao3}{3,3}{⼩、⼩}
  \definition{adj.}{muito pequeno}
\end{entry}

\begin{entry}{小心}{xiao3xin1}{3,4}{⼩、⼼}[HSK 2]
  \definition{adj.}{cuidado}
\end{entry}

\begin{entry}{小型}{xiao3 xing2}{3,9}{⼩、⼟}[HSK 4]
  \definition{adj.}{de tamanho pequeno; em pequena escala; miniatura; tipo pequeno; tamanho de bolso; tipo compacto}
  \definition{s.}{(Mediterrâneo) escunas, pequenos veleiros de pesca ou turismo | pequeno \emph{rover} lunar (duas pessoas)}
\end{entry}

\begin{entry}{小学}{xiao3xue2}{3,8}{⼩、⼦}[HSK 1]
  \definition{s.}{escola ensino fundamental}
\end{entry}

\begin{entry}{小学生}{xiao3xue2sheng1}{3,8,5}{⼩、⼦、⽣}[HSK 1]
  \definition{s.}{aluno, estudante de escola primária}
\end{entry}

\begin{entry}{小洋白菜}{xiao3 yang2bai2cai4}{3,9,5,11}{⼩、⽔、⽩、⾋}
  \definition{s.}{couve de bruxelas}
\end{entry}

\begin{entry}{小众}{xiao3zhong4}{3,6}{⼩、⼈}
  \definition{s.}{minoria da população | nicho (mercado, etc.)}
\end{entry}

\begin{entry}{小组}{xiao3 zu3}{3,8}{⼩、⽷}[HSK 2]
  \definition[个]{s.}{grupo}
\end{entry}

\begin{entry}{哮喘}{xiao4chuan3}{10,12}{⼝、⼝}
  \definition{s.}{asma}
\end{entry}

\begin{entry}{效果}{xiao4guo3}{10,8}{⽁、⽊}[HSK 3]
  \definition[种,个]{s.}{efeito; resultado | efeitos sonoros; vários sons ou fenômenos naturais criados para combinar com o enredo em dramas e filmes, como vento e chuva, tiros, fogo, neve, etc.}
\end{entry}

\begin{entry}{效率}{xiao4lv4}{10,11}{⽁、⽞}[HSK 4]
  \definition[种]{s.}{eficiência; produtividade}
\end{entry}

\begin{entry}{校}{xiao4}{10}{⽊}
  \definition[所]{s.}{oficial militar | escola}
  \seeref{校}{jiao4}
\end{entry}

\begin{entry}{校服}{xiao4fu2}{10,8}{⽊、⽉}
  \definition{s.}{uniforme escolar}
\end{entry}

\begin{entry}{校规}{xiao4gui1}{10,8}{⽊、⾒}
  \definition{s.}{regras e regulamentos escolares}
\end{entry}

\begin{entry}{校监}{xiao4jian1}{10,10}{⽊、⽫}
  \definition{s.}{diretor | supervisor (de escola)}
\end{entry}

\begin{entry}{校园}{xiao4 yuan2}{10,7}{⽊、⼞}[HSK 2]
  \definition{s.}{campus}
\end{entry}

\begin{entry}{校长}{xiao4zhang3}{10,4}{⽊、⾧}[HSK 2]
  \definition[个,位,名]{s.}{diretor de escola | reitor (universidade)}
\end{entry}

\begin{entry}{笑}{xiao4}{10}{⽵}[HSK 1]
  \definition{v.}{sorrir | rir | rir de}
\end{entry}

\begin{entry}{笑话儿}{xiao4 hua4r5}{10,8,2}{⽵、⾔、⼉}[HSK 2]
  \definition{s.}{piada | gracejo}
\end{entry}

\begin{entry}{笑话}{xiao4hua5}{10,8}{⽵、⾔}[HSK 2]
  \definition{adj.}{absurdo | ridículo}
  \definition[个]{s.}{piada | brincadeira}
  \definition{v.}{rir de algo | zombar | ridicularizar}
\end{entry}

\begin{entry}{笑容}{xiao4rong2}{10,10}{⽵、⼧}
  \definition[副]{s.}{sorriso | expressão sorridente}
\end{entry}

\begin{entry}{些}{xie1}{8}{⼆}[HSK 4]
  \definition{adv.}{um pouco; um pouco mais; usado após um adjetivo ou parte de um verbo para indicar uma pequena quantidade, equivalente a ``一点儿''}
  \definition{clas.}{alguns; um pouco; denota uma quantidade indefinida}
  \seealsoref{一点儿}{yi4dian3r5}
\end{entry}

\begin{entry}{些许}{xie1xu3}{8,6}{⼆、⾔}
  \definition{num.}{um pouco}
\end{entry}

\begin{entry}{斜阳}{xie2yang2}{11,6}{⽃、⾩}
  \definition{s.}{sol poente}
\end{entry}

\begin{entry}{谐}{xie2}{11}{⾔}
  \definition{adj.}{harmonioso | humorístico}
\end{entry}

\begin{entry}{鞋}{xie2}{15}{⾰}[HSK 2]
  \definition[双,只]{s.}{sapatos}
\end{entry}

\begin{entry}{写}{xie3}{5}{⼍}[HSK 1]
  \definition{v.}{escrever}
\end{entry}

\begin{entry}{写意}{xie3yi4}{5,13}{⼍、⼼}
  \definition{s.}{estilo de pintura chinesa à mão livre, caracterizado por traços ousados em vez de detalhes precisos}
  \definition{v.}{sugerir (em vez de descrever em detalhes)}
  \seeref{写意}{xie4yi4}
\end{entry}

\begin{entry}{写照}{xie3zhao4}{5,13}{⼍、⽕}
  \definition{s.}{retrato}
\end{entry}

\begin{entry}{写真}{xie3zhen1}{5,10}{⼍、⼗}
  \definition{s.}{retrato}
  \definition{v.}{descrever algo com precisão}
\end{entry}

\begin{entry}{写字}{xie3zi4}{5,6}{⼍、⼦}
  \definition{v.}{escrever (à mão) | praticar caligrafia}
\end{entry}

\begin{entry}{写字匠}{xie3zi4 jiang4}{5,6,6}{⼍、⼦、⼕}
  \definition{s.}{calígrafo}
\end{entry}

\begin{entry}{写作}{xie3zuo4}{5,7}{⼍、⼈}[HSK 3]
  \definition{s.}{escrita; redação; composição}
  \definition{v.}{escrever artigos; escrever livros, etc.; também se refere especificamente à criação de obras literárias}
\end{entry}

\begin{entry}{血}{xie3}{6}{⾎}[Kangxi 143]
  \seeref{血}{xue4}
\end{entry}

\begin{entry}{写意}{xie4yi4}{5,13}{⼍、⼼}
  \definition{adj.}{confortável | agradável | relaxado}
  \seeref{写意}{xie3yi4}
\end{entry}

\begin{entry}{泄气}{xie4qi4}{8,4}{⽔、⽓}
  \definition{adj.}{decepcionante | frustrante | patético}
  \definition{v.+compl.}{perder o coração | sentir-se desencorajado | ficar desanimado}
\end{entry}

\begin{entry}{谢病}{xie4bing4}{12,10}{⾔、⽧}
  \definition{v.}{desculpar-se por causa de doença}
\end{entry}

\begin{entry}{谢恩}{xie4'en1}{12,10}{⾔、⼼}
  \definition{v.}{agradecer a alguém pelo favor (especialmente imperador ou oficial superior)}
\end{entry}

\begin{entry}{谢媒}{xie4mei2}{12,12}{⾔、⼥}
  \definition{v.}{agradecer ao casamenteiro}
\end{entry}

\begin{entry}{谢世}{xie4shi4}{12,5}{⾔、⼀}
  \definition{v.}{morrer | falecer}
\end{entry}

\begin{entry}{谢天谢地}{xie4tian1xie4di4}{12,4,12,6}{⾔、⼤、⾔、⼟}
  \definition{expr.}{agradecer a Deus | agradecer aos céus}
\end{entry}

\begin{entry}{谢谢}{xie4xie5}{12,12}{⾔、⾔}[HSK 1]
  \definition{interj.}{Obrigado!}
  \definition{v.}{agradecer}
\end{entry}

\begin{entry}{谢意}{xie4yi4}{12,13}{⾔、⼼}
  \definition{s.}{gratidão}
\end{entry}

\begin{entry}{心}{xin1}{4}{⼼}[HSK 3][Kangxi 61]
  \definition*{s.}{Xin, uma das mansões lunares}
  \definition[颗]{s.}{o coração | coração; mente; sentimento; intenção | centro; núcleo}
\end{entry}

\begin{entry}{心机}{xin1ji1}{4,6}{⼼、⽊}
  \definition{s.}{pensamento | esquema}
\end{entry}

\begin{entry}{心里}{xin1 li3}{4,7}{⼼、⾥}[HSK 2]
  \definition[把]{s.}{no coração | no coração de alguém | na mente}
\end{entry}

\begin{entry}{心理}{xin1li3}{4,11}{⼼、⽟}[HSK 4]
  \definition{adj.}{psicológico}
  \definition{s.}{mentalidade; refere-se à reflexão da mente humana sobre coisas objetivas, incluindo sensação, percepção, memória, pensamento e emoções | psicologia}
\end{entry}

\begin{entry}{心情}{xin1qing2}{4,11}{⼼、⼼}[HSK 2]
  \definition{s.}{humor | sentimento | estado de espírito}
\end{entry}

\begin{entry}{心声}{xin1sheng1}{4,7}{⼼、⼠}
  \definition{s.}{desejo sincero | voz interior | aspiração}
\end{entry}

\begin{entry}{心疼}{xin1teng2}{4,10}{⼼、⽧}
  \definition{adj.}{angustiado}
  \definition{v.}{sentir pena de alguém | arrepender-se | ressentir-se | ficar angustiado}
\end{entry}

\begin{entry}{心中}{xin1zhong1}{4,4}{⼼、⼁}[HSK 2]
  \definition{adv.}{nos pensamentos | no coração}
  \definition{s.}{ponto central}
\end{entry}

\begin{entry}{芯片}{xin1pian4}{7,4}{⾋、⽚}
  \definition{s.}{chip de computador | microchip}
\end{entry}

\begin{entry}{辛苦}{xin1ku3}{7,8}{⾟、⾋}
  \definition{adj.}{exaustivo | duro | árduo}
  \definition{s.}{dificuldades}
  \definition{v.}{trabalhar duro | ter muitos problemas}
\end{entry}

\begin{entry}{新}{xin1}{13}{⽄}[HSK 1]
  \definition*{s.}{sobrenome Xin | abreviação de Xinjiang (新疆) | abreviação de Singapura (新加坡)}
  \definition{adj.}{novo}
  \definition{adv.}{recentemente}
  \definition{pref.}{(química) meso-}
  \seealsoref{新加坡}{xin1jia1po1}
  \seealsoref{新疆}{xin1jiang1}
\end{entry}

\begin{entry}{新加坡}{xin1jia1po1}{13,5,8}{⽄、⼒、⼟}
  \definition*{s.}{Singapura}
\end{entry}

\begin{entry}{新疆}{xin1jiang1}{13,19}{⽄、⼸}
  \definition*{s.}{Xinjiang}
\end{entry}

\begin{entry}{新疆维吾尔自治区}{xin1jiang1 wei2wu2'er3 zi4zhi4qu1}{13,19,11,7,5,6,8,4}{⽄、⼸、⽷、⼝、⼩、⾃、⽔、⼖}
  \definition*{s.}{Região Autônoma Uigur de Xinjiang}
\end{entry}

\begin{entry}{新郎}{xin1lang2}{13,8}{⽄、⾢}[HSK 4]
  \definition[位,个]{s.}{noivo; homens no momento do casamento}
\end{entry}

\begin{entry}{新年}{xin1nian2}{13,6}{⽄、⼲}[HSK 1]
  \definition*[个]{s.}{Ano Novo}
\end{entry}

\begin{entry}{新娘}{xin1niang2}{13,10}{⽄、⼥}[HSK 4]
  \definition[位,个]{s.}{noiva; a mulher no momento do casamento}
  \seealsoref{新娘子}{xin1niang2zi5}
\end{entry}

\begin{entry}{新娘服装}{xin1niang2 fu2zhuang1}{13,10,8,12}{⽄、⼥、⽉、⾐}
  \definition{s.}{roupas de noiva}
\end{entry}

\begin{entry}{新娘子}{xin1niang2zi5}{13,10,3}{⽄、⼥、⼦}
  \definition{s.}{noiva}
  \seealsoref{新娘}{xin1niang2}
\end{entry}

\begin{entry}{新闻}{xin1wen2}{13,9}{⽄、⾨}[HSK 2]
  \definition[条,个]{s.}{notícia}
\end{entry}

\begin{entry}{新鲜}{xin1xian1}{13,14}{⽄、⿂}
  \definition{adj.}{fresco (experiência, alimento, etc.)}
  \definition{s.}{frescor}
\end{entry}

\begin{entry}{新型}{xin1 xing2}{13,9}{⽄、⼟}[HSK 4]
  \definition[种]{s.}{ultimo modelo; novo tipo; novo padrão; novo estilo}
\end{entry}

\begin{entry}{信}{xin4}{9}{⼈}[HSK 2,3]
  \definition*{s.}{sobrenome Xin}
  \definition{adj.}{verdade}
  \definition{adv.}{à vontade; ao acaso; sem plano}
  \definition[封,个,张]{s.}{carta; correio
mensagem; palavra; informação
sinal; evidência
confiança; fé
fusível
arsênico}
  \definition{v.}{acreditar; fazer um balanço; dar crédito | professar fé em; acreditar em}
\end{entry}

\begin{entry}{信访}{xin4fang3}{9,6}{⼈、⾔}
  \definition{s.}{carta de reclamação | carta de petição}
  \seealsoref{上访}{shang4fang3}
\end{entry}

\begin{entry}{信封}{xin4feng1}{9,9}{⼈、⼨}[HSK 3]
  \definition[个]{s.}{envelope de carta}
\end{entry}

\begin{entry}{信号}{xin4hao4}{9,5}{⼈、⼝}[HSK 2]
  \definition[个]{s.}{sinal | ponte de sinalização}
\end{entry}

\begin{entry}{信经}{xin4jing1}{9,8}{⼈、⽷}
  \definition[个]{s.}{crença | credo (seção da missa católica)}
\end{entry}

\begin{entry}{信任}{xin4ren4}{9,6}{⼈、⼈}[HSK 3]
  \definition[个]{s.}{confiança; certeza; convicção}
  \definition{v.}{confiar; ter confiança em}
\end{entry}

\begin{entry}{信息}{xin4xi1}{9,10}{⼈、⼼}[HSK 2]
  \definition[个,条]{s.}{notícias | informação | mensagem}
\end{entry}

\begin{entry}{信心}{xin4xin1}{9,4}{⼈、⼼}[HSK 2]
  \definition[个]{s.}{confiança | fé (em alguém ou algo)}
\end{entry}

\begin{entry}{信用}{xin4yong4}{9,5}{⼈、⽤}
  \definition{s.}{crédito (comércio)}
\end{entry}

\begin{entry}{信用卡}{xin4yong4ka3}{9,5,5}{⼈、⽤、⼘}[HSK 2]
  \definition[些]{s.}{cartão de crédito}
\end{entry}

\begin{entry}{兴}{xing1}{6}{⼋}
  \definition*{s.}{sobrenome Xing}
  \definition{adv.}{talvez (dialeto)}
  \definition{v.}{subir | florescer | tornar-se popular | começar | encorajar | levantar-se | (frequentemente usado em negativas) permitir (dialeto)}
  \seeref{兴}{xing4}
\end{entry}

\begin{entry}{兴奋}{xing1fen4}{6,8}{⼋、⼤}[HSK 4]
  \definition{adj.}{animado; excitante; empolgante;}
  \definition{s.}{excitação; empolgação}
  \definition{v.}{excitar; intoxicar}
\end{entry}

\begin{entry}{星表}{xing1biao3}{9,8}{⽇、⾐}
  \definition{s.}{catálogo de estrelas}
\end{entry}

\begin{entry}{星辰}{xing1chen2}{9,7}{⽇、⾠}
  \definition{s.}{estrelas}
\end{entry}

\begin{entry}{星火}{xing1huo3}{9,4}{⽇、⽕}
  \definition{s.}{trilha de meteoro (usada principalmente em expressões como 急如星火) | faísca}
\end{entry}

\begin{entry}{星期}{xing1qi1}{9,12}{⽇、⽉}[HSK 1]
  \definition[个]{s.}{semana}
\end{entry}

\begin{entry}{星期二}{xing1qi1'er4}{9,12,2}{⽇、⽉、⼆}[HSK 1]
  \definition{s.}{terça-feira}
\end{entry}

\begin{entry}{星期六}{xing1qi1liu4}{9,12,4}{⽇、⽉、⼋}[HSK 1]
  \definition{s.}{sábado}
\end{entry}

\begin{entry}{星期日}{xing1qi1ri4}{9,12,4}{⽇、⽉、⽇}[HSK 1]
  \definition{s.}{domingo}
  \seealsoref{星期天}{xing1qi1tian1}
\end{entry}

\begin{entry}{星期三}{xing1qi1san1}{9,12,3}{⽇、⽉、⼀}[HSK 1]
  \definition{s.}{quarta-feira}
\end{entry}

\begin{entry}{星期四}{xing1qi1si4}{9,12,5}{⽇、⽉、⼞}[HSK 1]
  \definition{s.}{quinta-feira}
\end{entry}

\begin{entry}{星期天}{xing1qi1tian1}{9,12,4}{⽇、⽉、⼤}[HSK 1]
  \definition{s.}{domingo}
  \seealsoref{星期日}{xing1qi1ri4}
\end{entry}

\begin{entry}{星期五}{xing1qi1wu3}{9,12,4}{⽇、⽉、⼆}[HSK 1]
  \definition{s.}{sexta-feira}
\end{entry}

\begin{entry}{星期一}{xing1qi1yi1}{9,12,1}{⽇、⽉、⼀}[HSK 1]
  \definition{s.}{segunda-feira}
\end{entry}

\begin{entry}{星星}{xing1 xing5}{9,9}{⽇、⽇}[HSK 2]
  \definition{s.}{estrela}
\end{entry}

\begin{entry}{星座}{xing1zuo4}{9,10}{⽇、⼴}
  \definition[张]{s.}{signo astrológico | constelação}
\end{entry}

\begin{entry}{猩猩}{xing1xing5}{12,12}{⽝、⽝}
  \definition{s.}{orangotango}
\end{entry}

\begin{entry}{行}{xing2}{6}{⾏}[HSK 1][Kangxi 144]
  \definition{adj.}{capaz | competente}
  \definition{expr.}{claro que sim | de acordo | está bem}
  \definition{interj.}{OK!}
  \definition{v.}{caminhar | ir | viajar | atuar}
  \seeref{行}{hang2}
\end{entry}

\begin{entry}{行动}{xing2dong4}{6,6}{⾏、⼒}[HSK 2]
  \definition[个]{s.}{ação | operação}
  \definition{v.}{mover}
\end{entry}

\begin{entry}{行进}{xing2jin4}{6,7}{⾏、⾡}
  \definition{s.}{avançar | movimentar-se para frente}
\end{entry}

\begin{entry}{行礼}{xing2li3}{6,5}{⾏、⽰}
  \definition{v.}{saudar | fazer saudação}
\end{entry}

\begin{entry}{行李}{xing2li5}{6,7}{⾏、⽊}[HSK 3]
  \definition[个,件]{s.}{bagagem | pacotes, caixas, cestas, etc. que você carrega quando sai}
\end{entry}

\begin{entry}{行人}{xing2ren2}{6,2}{⾏、⼈}[HSK 2]
  \definition{s.}{transeunte | pedestre | viajante à pé}
\end{entry}

\begin{entry}{行驶}{xing2shi3}{6,8}{⾏、⾺}
  \definition{v.}{viajar ao longo de uma rota (veículos, etc.)}
\end{entry}

\begin{entry}{行为}{xing2wei2}{6,4}{⾏、⼂}[HSK 2]
  \definition[个]{s.}{ação | comportamento | conduta}
\end{entry}

\begin{entry}{行星}{xing2xing1}{6,9}{⾏、⽇}
  \definition[颗]{s.}{planeta}
  \seealsoref{惑星}{huo4xing1}
\end{entry}

\begin{entry}{行凶}{xing2xiong1}{6,4}{⾏、⼐}
  \definition{v.+compl.}{cometer agressão física ou assassinato | fazer algo violento}
\end{entry}

\begin{entry}{形成}{xing2cheng2}{7,6}{⼺、⼽}[HSK 3]
  \definition{v.}{moldar; formar; tomar forma | tornar-se algo ou algo através do desenvolvimento e da mudança}
\end{entry}

\begin{entry}{形而上学}{xing2'er2shang4xue2}{7,6,3,8}{⼺、⽽、⼀、⼦}
  \definition{s.}{metafísica}
\end{entry}

\begin{entry}{形容}{xing2rong2}{7,10}{⼺、⼧}[HSK 4]
  \definition{s.}{aparência; semblante}
  \definition{v.}{descrever}
\end{entry}

\begin{entry}{形式}{xing2shi4}{7,6}{⼺、⼷}[HSK 3]
  \definition[种,个]{s.}{forma; formato; modalidade | aparência, estrutura ou estado de algo}
\end{entry}

\begin{entry}{形势}{xing2shi4}{7,8}{⼺、⼒}[HSK 4]
  \definition[个]{s.}{terreno; características topográficas; situação geográfica, principalmente de uma perspectiva militar | situação; circunstâncias; a situação geral, a tendência de como as coisas estão se desenvolvendo e mudando | geralmente não é usado em situações pessoais}
\end{entry}

\begin{entry}{形象}{xing2xiang4}{7,11}{⼺、⾗}[HSK 3]
  \definition{adj.}{vívido}
  \definition[个]{s.}{imagem; forma; figura | uma forma ou gesto específico que pode despertar os pensamentos ou emoções das pessoas | imagem literária; imagem artística | pessoas ou coisas com características diferentes criadas na literatura, no cinema e em outras artes}
\end{entry}

\begin{entry}{形状}{xing2zhuang4}{7,7}{⼺、⽝}[HSK 3]
  \definition[个]{s.}{forma; aparência | a aparência de um objeto ou figura formada pela combinação de superfícies ou linhas externas}
\end{entry}

\begin{entry}{型}{xing2}{9}{⼟}[HSK 4]
  \definition{s.}{molde; modelo | modelo; tipo; padrão}
\end{entry}

\begin{entry}{型号}{xing2 hao4}{9,5}{⼟、⼝}[HSK 4]
  \definition[个,种]{s.}{modelo; tipo; refere-se ao desempenho, às especificações e ao tamanho de aeronaves, máquinas, implementos agrícolas, etc.}
\end{entry}

\begin{entry}{省}{xing3}{9}{⽬}
  \definition[个]{s.}{governadoria}
  \definition{v.}{examinar minuciosamente | refletir (sobre a conduta de alguém) | realizar | fazer uma visita (aos pais ou idosos)}
  \seeref{省}{sheng3}
\end{entry}

\begin{entry}{省悟}{xing3wu4}{9,10}{⽬、⼼}
  \definition{v.}{voltar a si | constatar | ver a verdade | acordar para a realidade}
\end{entry}

\begin{entry}{醒}{xing3}{16}{⾣}[HSK 4]
  \definition{adj.}{impressionante; notável; admirável; atraente; chamativo}
  \definition{v.}{ficar sóbrio; voltar a si; recuperar a consciência; retornar à normalidade após intoxicação, anestesia ou coma | despertar; estar acordado | ter a mente clara; mover a consciência da confusão para a compreensão | vir a entender; tornar-se ciente de; tomar consciência de}
\end{entry}

\begin{entry}{兴}{xing4}{6}{⼋}
  \definition{s.}{sentimento ou desejo de fazer algo | interesse em algo | excitação}
  \seeref{兴}{xing1}
\end{entry}

\begin{entry}{兴趣}{xing4 qu4}{6,15}{⼋、⾛}[HSK 4]
  \definition[个]{s.}{interesse (desejo de conhecer sobre alguma coisa ou coisa no qual está interessado) | \emph{hobby}}
\end{entry}

\begin{entry}{姓}{xing4}{8}{⼥}[HSK 2]
  \definition[个]{s.}{sobrenome}
  \definition{v.}{ter o sobrenome}
\end{entry}

\begin{entry}{姓名}{xing4ming2}{8,6}{⼥、⼝}[HSK 2]
  \definition{s.}{nome completo}
\end{entry}

\begin{entry}{姓氏}{xing4shi4}{8,4}{⼥、⽒}
  \definition{s.}{sobrenome}
\end{entry}

\begin{entry}{幸福}{xing4fu2}{8,13}{⼲、⽰}[HSK 3]
  \definition{adj.}{feliz | a vida, a família, etc. deixam as pessoas satisfeitas e felizes}
  \definition{s.}{felicidade; bem estar | uma sensação ou experiência satisfatória e feliz}
\end{entry}

\begin{entry}{幸亏}{xing4kui1}{8,3}{⼲、⼆}
  \definition{adv.}{felizmente}
\end{entry}

\begin{entry}{幸运}{xing4yun4}{8,7}{⼲、⾡}[HSK 3]
  \definition{adj.}{sortudo; feliz; afortunado}
  \definition[个]{s.}{boa sorte; boa fortuna}
\end{entry}

\begin{entry}{幸运抽奖}{xing4yun4chou1jiang3}{8,7,8,9}{⼲、⾡、⼿、⼤}
  \definition{s.}{loteria | sorteio}
\end{entry}

\begin{entry}{幸运儿}{xing4yun4'er2}{8,7,2}{⼲、⾡、⼉}
  \definition{s.}{pessoa de sorte}
\end{entry}

\begin{entry}{性}{xing4}{8}{⼼}[HSK 3]
  \definition*{s.}{sobrenome Xing}
  \definition[个]{s.}{natureza; caráter; disposição | propriedade; qualidade | sexo; gênero}
  \definition{suf.}{forma substantivo a partir de adjetivo | indica natureza, escopo ou maneira}
\end{entry}

\begin{entry}{性别}{xing4bie2}{8,7}{⼼、⼑}[HSK 3]
  \definition[种]{s.}{sexo; gênero}
\end{entry}

\begin{entry}{性格}{xing4ge2}{8,10}{⼼、⽊}[HSK 3]
  \definition[种,个]{s.}{caráter; temperamento}
\end{entry}

\begin{entry}{性侵}{xing4qin1}{8,9}{⼼、⼈}
  \definition{s.}{agressão sexual}
  \definition{v.}{agredir sexualmente}
\end{entry}

\begin{entry}{性生活}{xing4sheng1huo2}{8,5,9}{⼼、⽣、⽔}
  \definition{s.}{vida sexual}
\end{entry}

\begin{entry}{性质}{xing4zhi4}{8,8}{⼼、⾙}[HSK 4]
  \definition[个,种,类]{s.}{natureza; qualidade; caráter; propriedade; propriedade fundamental que distingue uma coisa de outra}
\end{entry}

\begin{entry}{兄弟}{xiong1di4}{5,7}{⼉、⼸}[HSK 4]
  \definition{adj.}{fraternal}
  \definition{pron.}{eu, me (termo de uso humilde por homens em discurso público)}
  \definition[个,对]{s.}{irmãos; irmão}
\end{entry}

\begin{entry}{匈奴}{xiong1nu2}{6,5}{⼓、⼥}
  \definition*{s.}{Xiongnu, um povo da estepe oriental que criou um império que floresceu na época das dinastias Qin e Han}
\end{entry}

\begin{entry}{汹涌}{xiong1yong3}{7,10}{⽔、⽔}
  \definition{adj.}{turbulento}
  \definition{v.}{aumentar ou emergir violentamente (oceano, rio, lago, etc.)}
\end{entry}

\begin{entry}{胸}{xiong1}{10}{⾁}
  \definition{s.}{peito | tórax}
\end{entry}

\begin{entry}{胸部}{xiong1 bu4}{10,10}{⾁、⾢}[HSK 4]
  \definition{s.}{peito; tórax; seios}
\end{entry}

\begin{entry}{熊}{xiong2}{14}{⽕}
  \definition*{s.}{sobrenome Xiong}
  \definition{adj.}{incapaz}
  \definition[把]{s.}{urso}
  \definition{v.}{repreender}
\end{entry}

\begin{entry}{熊猫}{xiong2mao1}{14,11}{⽕、⽝}
  \definition[把,只]{s.}{panda gigante}
  \seealsoref{猫熊}{mao1xiong2}
\end{entry}

\begin{entry}{休兵}{xiu1bing1}{6,7}{⼈、⼋}
  \definition{s.}{armistício}
  \definition{v.}{cessar fogo}
\end{entry}

\begin{entry}{休假}{xiu1 jia4}{6,11}{⼈、⼈}[HSK 2]
  \definition{v.+compl.}{ter um feriado | tirar férias | sair de férias}
\end{entry}

\begin{entry}{休憩}{xiu1qi4}{6,16}{⼈、⼼}
  \definition{v.}{relaxar | descansar | dar um tempo}
\end{entry}

\begin{entry}{休息室}{xiu1xi1shi4}{6,10,9}{⼈、⼼、⼧}
  \definition{s.}{saguão | salão}
\end{entry}

\begin{entry}{休息}{xiu1xi5}{6,10}{⼈、⼼}[HSK 1]
  \definition{s.}{descanço}
  \definition{v.}{descansar}
\end{entry}

\begin{entry}{休闲}{xiu1xian2}{6,7}{⼈、⾨}
  \definition{s.}{ócio | lazer}
  \definition{v.}{desfrutar do lazer}
\end{entry}

\begin{entry}{休整}{xiu1zheng3}{6,16}{⼈、⽁}
  \definition{v.}{(militar) descansar e reorganizar}
\end{entry}

\begin{entry}{修}{xiu1}{9}{⼈}[HSK 3]
  \definition*{s.}{sobrenome Xiu}
  \definition{adj.}{comprido; alto e esbelto}
  \definition{s.}{revisionismo}
  \definition{v.}{embelezar; decorar
consertar; reparar; reformar
escrever; compilar
estudar; cultivar
construir; edificar
aparar; podar}
\end{entry}

\begin{entry}{修改}{xiu1gai3}{9,7}{⼈、⽁}[HSK 3]
  \definition{v.}{revisar; alterar}
\end{entry}

\begin{entry}{修规}{xiu1gui1}{9,8}{⼈、⾒}
  \definition{s.}{plano de construção}
\end{entry}

\begin{entry}{修理}{xiu1li3}{9,11}{⼈、⽟}[HSK 4]
  \definition{v.}{consertar; reparar; restaurar algo danificado à sua forma ou função original | aparar; podar; cortar com tesouras e outras ferramentas para deixar árvores, flores, cabelos, etc. arrumados | culpar; punir; criticar ou punir uma pessoa para mostrar que ela está errada}
\end{entry}

\begin{entry}{绣}{xiu4}{10}{⽷}
  \definition{s.}{bordado}
  \definition{v.}{bordar}
\end{entry}

\begin{entry}{臭}{xiu4}{10}{⾃}
  \definition{s.}{odor; cheiro;}
  \definition{v.}{cheirar; farejar; o mesmo que "嗅"}
  \seeref{臭}{chou4}
  \seealsoref{嗅}{xiu4}
\end{entry}

\begin{entry}{袖}{xiu4}{10}{⾐}
  \definition{s.}{manga (de camisa, de camiseta, etc.)}
\end{entry}

\begin{entry}{嗅}{xiu4}{13}{⼝}
  \definition{v.}{cheirar; farejar; identificar odores pelo nariz}
\end{entry}

\begin{entry}{虚伪}{xu1wei3}{11,6}{⾌、⼈}
  \definition{adj.}{falso | hipócrita | artificial}
\end{entry}

\begin{entry}{需求}{xu1qiu2}{14,7}{⾬、⽔}[HSK 3]
  \definition{s.}{necessidades; demanda; requisito; requerimento; exigência | solicitações decorrentes de necessidades}
\end{entry}

\begin{entry}{需要}{xu1yao4}{14,9}{⾬、⾑}[HSK 3]
  \definition{s.}{necessidade | desejo ou solicitação de algo}
  \definition{v.}{precisar; querer; requerer; demandar}
\end{entry}

\begin{entry}{许}{xu3}{6}{⾔}
  \definition*{s.}{sobrenome Xu}
  \definition{adv.}{um pouco | talvez}
  \definition{v.}{permitir | prometer | elogiar}
\end{entry}

\begin{entry}{许多}{xu3duo1}{6,6}{⾔、⼣}[HSK 2]
  \definition{num.}{muitos | muito | numerosos | uma grande quantidade de}
\end{entry}

\begin{entry}{T-恤}{xu4}{9}{⼼}
  \definition{s.}{camiseta | pulôver | suéter}
\end{entry}

\begin{entry}{畜}{xu4}{10}{⽥}
  \definition{v.}{criar (animais)}
  \seeref{畜}{chu4}
\end{entry}

\begin{entry}{宣布}{xuan1bu4}{9,5}{⼧、⼱}[HSK 3]
  \definition{v.}{declarar; proclamar; pronunciar; anunciar | anunciar oficialmente e publicamente as últimas decisões e situações a todos}
\end{entry}

\begin{entry}{宣传}{xuan1chuan2}{9,6}{⼧、⼈}[HSK 3]
  \definition{v.}{propagar; disseminar; conduzir propaganda | explicar às massas para que elas possam acreditar e agir de acordo}
\end{entry}

\begin{entry}{宣扬}{xuan1yang2}{9,6}{⼧、⼿}
  \definition{v.}{divulgar | anunciar | espalhar por toda parte}
\end{entry}

\begin{entry}{玄学}{xuan2xue2}{5,8}{⽞、⼦}
  \definition{s.}{Escola Philosófica Wei e Jin amalgamando os ideais daoísta e confucionistas | tradução da metafísica (形而上学)}
  \seeref{形而上学}{xing2'er2shang4xue2}
\end{entry}

\begin{entry}{悬挂}{xuan2gua4}{11,9}{⼼、⼿}
  \definition{s.}{(veículo) suspensão}
  \definition{v.}{suspender}
\end{entry}

\begin{entry}{悬崖}{xuan2ya2}{11,11}{⼼、⼭}
  \definition{s.}{precipício | penhasco}
\end{entry}

\begin{entry}{旋转}{xuan2zhuan3}{11,8}{⽅、⾞}
  \definition{v.}{girar}
\end{entry}

\begin{entry}{选}{xuan3}{9}{⾡}[HSK 2]
  \definition{s.}{seleções | antologia}
  \definition{v.}{selecionar | escolher | eleger}
\end{entry}

\begin{entry}{选手}{xuan3shou3}{9,4}{⾡、⼿}[HSK 3]
  \definition[位]{s.}{jogador; competidor (selecionado); atleta selecionado para uma competição esportiva}
\end{entry}

\begin{entry}{选择}{xuan3ze2}{9,8}{⾡、⼿}[HSK 4]
  \definition[个,种,次]{s.}{escolha; opção; resultado da escolha; possibilidade de escolha}
  \definition{v.}{selecionar; escolher}
\end{entry}

\begin{entry}{学}{xue2}{8}{⼦}[HSK 1]
  \definition{v.}{aprender | estudar}
\end{entry}

\begin{entry}{学费}{xue2 fei4}{8,9}{⼦、⾙}[HSK 3]
  \definition[个]{s.}{mensalidade (taxa); prêmio; taxas que os alunos devem pagar para estudar na escola, conforme estipulado pela escola | preço pelo que se aprendeu ao custo de cada um; uma metáfora para o preço pago para ganhar uma certa experiência | custo; preço; todas as despesas necessárias para os alunos estudarem}
\end{entry}

\begin{entry}{学分}{xue2fen1}{8,4}{⼦、⼑}[HSK 4]
  \definition{s.}{créditos de um curso; uma unidade de medida do peso e do tempo do curso no ensino superior; cada curso vale um crédito para uma aula por semana durante um semestre; alunos devem concluir o número necessário de créditos para se formar}
\end{entry}

\begin{entry}{学好}{xue2hao3}{8,6}{⼦、⼥}
  \definition{v.}{seguir bons exemplos | aprender bem}
\end{entry}

\begin{entry}{学会}{xue2hui4}{8,6}{⼦、⼈}
  \definition{s.}{instituto | associação (acadêmica) | sociedade científica, douta ou erudita}
  \definition{v.}{aprender | dominar (um assunto)}
\end{entry}

\begin{entry}{学年}{xue2 nian2}{8,6}{⼦、⼲}[HSK 4]
  \definition{s.}{ano letivo; ano acadêmico}
\end{entry}

\begin{entry}{学期}{xue2qi1}{8,12}{⼦、⽉}[HSK 2]
  \definition[个]{s.}{semestre}
\end{entry}

\begin{entry}{学生}{xue2sheng5}{8,5}{⼦、⽣}[HSK 1]
  \definition{s.}{estudante | aluno}
\end{entry}

\begin{entry}{学生证}{xue2sheng5zheng4}{8,5,7}{⼦、⽣、⾔}
  \definition{s.}{cartão de identidade de estudante}
\end{entry}

\begin{entry}{学时}{xue2 shi2}{8,7}{⼦、⽇}[HSK 4]
  \definition{s.}{hora-aula; hora de aula | período}
\end{entry}

\begin{entry}{学术}{xue2shu4}{8,5}{⼦、⽊}[HSK 4]
  \definition[个]{s.}{aprendizagem; aprendizado; ciências; aprendizado sistemático e especializado}
\end{entry}

\begin{entry}{学问}{xue2wen4}{8,6}{⼦、⾨}[HSK 4]
  \definition[个]{s.}{aprendizado, conhecimento, erudição; a compreensão correta do mundo objetivo que alguém tem | conhecimento; aprendizado sistemático; conhecimento sistemático sobre algo ou uma ciência que pode ser aprendido em um livro ou em uma experiência prática}
\end{entry}

\begin{entry}{学习}{xue2xi2}{8,3}{⼦、⼄}[HSK 1]
  \definition{v.}{estudar | aprender}
\end{entry}

\begin{entry}{学校}{xue2xiao4}{8,10}{⼦、⽊}[HSK 1]
  \definition{s.}{escola | instituição de ensino}
\end{entry}

\begin{entry}{学院}{xue2yuan4}{8,9}{⼦、⾩}[HSK 1]
  \definition[所]{s.}{instituto}
\end{entry}

\begin{entry}{雪}{xue3}{11}{⾬}[HSK 2]
  \definition*{s.}{sobrenome Xue}
  \definition[场]{s.}{neve}
\end{entry}

\begin{entry}{雪板}{xue3ban3}{11,8}{⾬、⽊}
  \definition{s.}{prancha de \emph{snowboard}}
  \definition{v.}{praticar \textit{snowboard}}
\end{entry}

\begin{entry}{雪糕}{xue3gao1}{11,16}{⾬、⽶}
  \definition{s.}{picolé}
\end{entry}

\begin{entry}{雪花}{xue3hua1}{11,7}{⾬、⾋}
  \definition{s.}{floco de neve}
\end{entry}

\begin{entry}{雪葩}{xue3pa1}{11,12}{⾬、⾋}
  \definition{s.}{sorvete}
\end{entry}

\begin{entry}{雪人}{xue3ren2}{11,2}{⾬、⼈}
  \definition{s.}{boneco de neve | \emph{Yeti}}
\end{entry}

\begin{entry}{雪山}{xue3shan1}{11,3}{⾬、⼭}
  \definition{s.}{montanha coberta de neve}
\end{entry}

\begin{entry}{雪鞋}{xue3xie2}{11,15}{⾬、⾰}
  \definition[双]{s.}{sapatos de neve}
\end{entry}

\begin{entry}{血}{xue4}{6}{⾎}[HSK 3][Kangxi 143]
  \definition{adj.}{relacionado por sangue; parente consanguíneo}
  \definition[片]{s.}{sangue}
  \seeref{血}{xie3}
\end{entry}

\begin{entry}{血汗}{xue4han4}{6,6}{⾎、⽔}
  \definition{s.}{(fig.) suor e labuta, trabalho duro}
\end{entry}

\begin{entry}{熏香}{xun1xiang1}{14,9}{⽕、⾹}
  \definition{s.}{incenso}
\end{entry}

\begin{entry}{寻找}{xun2zhao3}{6,7}{⼨、⼿}[HSK 4]
  \definition{v.}{buscar; procurar; pesquisar; encontrar, que pode ser usado tanto para coisas concretas quanto para coisas abstratas}
\end{entry}

\begin{entry}{巡逻}{xun2luo2}{6,11}{⾡、⾡}
  \definition{s.}{patrulha}
  \definition{v.}{patrulhar (polícia, exército ou marinha)}
\end{entry}

\begin{entry}{训练}{xun4lian4}{5,8}{⾔、⽷}[HSK 3]
  \definition{v.}{treinar; exercitar; adquirir certas especialidades ou habilidades de forma planejada e passo a passo}
\end{entry}

\begin{entry}{迅速}{xun4su4}{6,10}{⾡、⾡}[HSK 4]
  \definition{adv.}{rapidamente; velozmente; prontamente}
\end{entry}

%%%%% EOF %%%%%


%%%
%%% Y
%%%

\section*{Y}\addcontentsline{toc}{section}{Y}

\begin{entry}{压岁钱}{ya1sui4qian2}{6,6,10}
  \definition{s.}{dinheiro da sorte | dinheiro dado às crianças como presente no Ano Novo Chinês}
\end{entry}

\begin{entry}{压碎}{ya1sui4}{6,13}
  \definition{v.}{esmagar em pedaços}
\end{entry}

\begin{entry}{压韵}{ya1yun4}{6,13}
  \variantof{押韵}
\end{entry}

\begin{entry}{押}{ya1}{8}[Radical 手]
  \definition{v.}{deter sob custódia | escoltar e proteger | hipotecar | penhorar}
\end{entry}

\begin{entry}{押后}{ya1hou4}{8,6}
  \definition{v.}{encerrar | adiar}
\end{entry}

\begin{entry}{押金}{ya1jin1}{8,8}
  \definition{s.}{caução | sinal | depósito}
\end{entry}

\begin{entry}{押送}{ya1song4}{8,9}
  \definition{v.}{enviar sob escolta | transportar um detido}
\end{entry}

\begin{entry}{押运}{ya1yun4}{8,7}
  \definition{v.}{escoltar sob guarda | escoltar (bens ou fundos)}
\end{entry}

\begin{entry}{押韵}{ya1yun4}{8,13}
  \definition{v.}{rimar}
\end{entry}

\begin{entry}{押注}{ya1zhu4}{8,8}
  \definition{v.}{apostar}
\end{entry}

\begin{entry}{押租}{ya1zu1}{8,10}
  \definition{s.}{depósito de aluguel}
\end{entry}

\begin{entry}{鸭}{ya1}{10}[Radical 鳥]
  \definition[只]{s.}{pato | (gíria) prostituto}
\end{entry}

\begin{entry}{鸭子}{ya1zi5}{10,3}
  \definition[只]{s.}{pato | (gíria) prostituto}
\end{entry}

\begin{entry}{牙}{ya2}{4}[Radical 牙][Kangxi 92]
  \definition[颗]{s.}{dente | marfim}
\end{entry}

\begin{entry}{牙齿}{ya2chi3}{4,8}
  \definition{adv.}{dental}
  \definition[颗]{s.}{dente}
\end{entry}

\begin{entry}{牙膏}{ya2gao1}{4,14}
  \definition[管]{s.}{pasta de dente}
\end{entry}

\begin{entry}{牙行}{ya2hang2}{4,6}
  \definition{s.}{corretor | \emph{broker}}
\end{entry}

\begin{entry}{牙刷}{ya2shua1}{4,8}
  \definition[把]{s.}{escova de dentes}
\end{entry}

\begin{entry}{牙线}{ya2xian4}{4,8}
  \definition[条]{s.}{fio dental}
\end{entry}

\begin{entry}{牙医}{ya2yi1}{4,7}
  \definition{s.}{dentista}
\end{entry}

\begin{entry}{崖}{ya2}{11}[Radical 山]
  \definition{s.}{precipício | penhasco}
\end{entry}

\begin{entry}{亚细亚洲}{ya4xi4ya4zhou1}{6,8,6,9}
  \definition*{s.}{Ásia}
\end{entry}

\begin{entry}{亚洲}{ya4zhou1}{6,9}
  \definition*{s.}{Ásia, abreviação de~亚细亚洲}
  \seeref{亚细亚洲}{ya4xi4ya4zhou1}
\end{entry}

\begin{entry}{亚洲人}{ya4zhou1ren2}{6,9,2}
  \definition{s.}{asiático | pessoa ou povo da Ásia}
\end{entry}

\begin{entry}{烟}{yan1}{10}[Radical 火]
  \definition[根]{s.}{cigarro ou cachimbo}
  \definition[竜]{s.}{fumaça | névoa |  vapor}
  \definition{s.}{planta de tabaco}
  \definition{v.}{ficar irritado com a fumaça (olhos)}
\end{entry}

\begin{entry}{烟草}{yan1cao3}{10,9}
  \definition{s.}{tabaco}
\end{entry}

\begin{entry}{烟囱}{yan1cong1}{10,7}
  \definition{s.}{chaminé}
\end{entry}

\begin{entry}{烟花}{yan1hua1}{10,7}
  \definition{s.}{fogos de artifício}
\end{entry}

\begin{entry}{烟火}{yan1huo3}{10,4}
  \definition{s.}{fogo de artifício}
\end{entry}

\begin{entry}{烟头}{yan1tou2}{10,5}
  \definition[根]{s.}{bituca de cigarro}
\end{entry}

\begin{entry}{烟叶}{yan1ye4}{10,5}
  \definition{s.}{folha de tabaco}
\end{entry}

\begin{entry}{烟雨}{yan1yu3}{10,8}
  \definition{s.}{chuvisco | garoa}
\end{entry}

\begin{entry}{严重}{yan2zhong4}{7,9}
  \definition{adj.}{crítico | grave | sério | severo}
\end{entry}

\begin{entry}{严重打伤}{yan2zhong4 da3 shang1}{7,9,5,6}
  \definition{s.}{gravemente ferido}
\end{entry}

\begin{entry}{严重地}{yan2zhong4 di4}{7,9,6}
  \definition{adv.}{seriamente | gravemente}
\end{entry}

\begin{entry}{严重关切}{yan2zhong4guan1qie4}{7,9,6,4}
  \definition{s.}{preocupação séria}
\end{entry}

\begin{entry}{严重后果}{yan2zhong4hou4guo3}{7,9,6,8}
  \definition{s.}{consequências sérias | repercursões graves}
\end{entry}

\begin{entry}{严重破坏}{yan2zhong4 po4huai4}{7,9,10,7}
  \definition{s.}{destruição grave}
\end{entry}

\begin{entry}{严重伤害}{yan2zhong4 shang1hai4}{7,9,6,10}
  \definition{s.}{ferimento grave}
\end{entry}

\begin{entry}{严重危害}{yan2zhong4wei1hai4}{7,9,6,10}
  \definition{s.}{danos graves}
\end{entry}

\begin{entry}{严重问题}{yan2zhong4wen4ti2}{7,9,6,15}
  \definition{s.}{problema sério}
\end{entry}

\begin{entry}{严重性}{yan2zhong4xing4}{7,9,8}
  \definition{s.}{seriedade | gravidade}
\end{entry}

\begin{entry}{言论}{yan2lun4}{7,6}
  \definition{s.}{expressão de opinião |  visualizações | comentários | argumentos}
\end{entry}

\begin{entry}{炎热}{yan2re4}{8,10}
  \definition{adj.}{extremamente quente | escaldante (clima)}
\end{entry}

\begin{entry}{颜}{yan2}{15}[Radical 頁]
  \definition*{s.}{sobrenome Yan}
  \definition{s.}{cor | face | semblante}
\end{entry}

\begin{entry}{颜色}{yan2se4}{15,6}[HSK 2]
  \definition{s.}{cor | pigmento | tintura}
\end{entry}

\begin{entry}{眼}{yan3}{11}[Radical 目][HSK 2]
  \definition{clas.}{para grandes coisas ocas: poços, fogões, panelas, etc.}
  \definition[只,双]{s.}{ponto crucial (de um assunto) | olho | pequeno buraco}
\end{entry}

\begin{entry}{眼柄}{yan3bing3}{11,9}
  \definition{s.}{pedúnculo ocular (de crustáceo, etc.)}
\end{entry}

\begin{entry}{眼袋}{yan3dai4}{11,11}
  \definition{s.}{inchaço sob os olhos}
\end{entry}

\begin{entry}{眼花缭乱}{yan3hua1liao2luan4}{11,7,15,7}
  \definition{v.}{ficar deslumbrado | deslumbrar}
\end{entry}

\begin{entry}{眼镜}{yan3jing4}{11,16}
  \definition[副]{s.}{óculos}
\end{entry}

\begin{entry}{眼睛}{yan3jing5}{11,13}[HSK 2]
  \definition[只,双]{s.}{olho(s)}
\end{entry}

\begin{entry}{眼泪}{yan3lei4}{11,8}
  \definition[滴]{s.}{choro | lágrimas}
\end{entry}

\begin{entry}{眼证}{yan3zheng4}{11,7}
  \definition{s.}{testemunha ocular}
\end{entry}

\begin{entry}{演}{yan3}{14}[Radical 水]
  \definition{v.}{atuar, encenar (em uma peça, show, etc.)}
\end{entry}

\begin{entry}{演员}{yan3yuan2}{14,7}
  \definition{s.}{ator | artista}
\end{entry}

\begin{entry}{扬雄}{yang2xiong2}{6,12}
  \definition*{s.}{Yang Xiong (53 AC-18 DC), estudioso, poeta e lexicógrafo, autor do primeiro dicionário de dialeto chinês 方言}
  \seealsoref{方言}{fang1yan2}
\end{entry}

\begin{entry}{阳}{yang2}{6}[Radical 阜]
  \definition*{s.}{Yang (o princípio positivo de Yin e Yang)}
  \definition{s.}{positivo (eletricidade) | sol}
  \seealsoref{阴}{yin1}
  \seealsoref{阴阳}{yin1yang2}
\end{entry}

\begin{entry}{阳台}{yang2tai2}{6,5}
  \definition{s.}{varanda | sacada}
\end{entry}

\begin{entry}{洋葱}{yang2cong1}{9,12}
  \definition{s.}{cebola}
\end{entry}

\begin{entry}{养}{yang3}{9}[Radical 八][HSK 2]
  \definition{v.}{criar (animais ou filhos), plantar (flores), etc. | dar a luz}
\end{entry}

\begin{entry}{养分}{yang3fen4}{9,4}
  \definition{s.}{nutriente}
\end{entry}

\begin{entry}{养料}{yang3liao4}{9,10}
  \definition{s.}{nutrição}
\end{entry}

\begin{entry}{氧}{yang3}{10}[Radical 气]
  \definition{s.}{oxigênio}
\end{entry}

\begin{entry}{样}{yang4}{10}[Radical 木]
  \definition{s.}{aparência | forma | modelo}
\end{entry}

\begin{entry}{样品}{yang4pin3}{10,9}
  \definition{s.}{amostra | espécime}
\end{entry}

\begin{entry}{样儿}{yang4r5}{10,2}
  \definition{s.}{aparência | forma | modelo}
  \seealsoref{样子}{yang4zi5}
\end{entry}

\begin{entry}{样样}{yang4yang4}{10,10}
  \definition{adv.}{todos os tipos}
\end{entry}

\begin{entry}{样章}{yang4zhang1}{10,11}
  \definition{s.}{capítulo de amostra}
\end{entry}

\begin{entry}{样子}{yang4zi5}{10,3}[HSK 2]
  \definition{s.}{aparência | forma | modelo}
  \seealsoref{样儿}{yang4r5}
\end{entry}

\begin{entry}{妖}{yao1}{7}[Radical 女]
  \definition{adj.}{enfeitiçante | encantador}
  \definition{s.}{\emph{goblin} | bruxa | diabo | monstro | fantasma | demônio}
\end{entry}

\begin{entry}{祅}{yao1}{8}[Radical 礻]
  \definition{s.}{espírito maligno | \emph{goblin} | bruxaria}
  \variantof{妖}
\end{entry}

\begin{entry}{要}{yao1}{9}[Radical 襾]
  \definition{v.}{(forma ligada) demandar, coagir}
  \seeref{要}{yao4}
\end{entry}

\begin{entry}{要求}{yao1qiu2}{9,7}[HSK 2]
  \definition[点]{s.}{requerimento}
  \definition{v.}{pedir | exigir | solicitar | fazer uma reivindicação}
\end{entry}

\begin{entry}{要挟}{yao1xie2}{9,9}
  \definition{v.}{chantagear | ameaçar}
\end{entry}

\begin{entry}{腰}{yao1}{13}[Radical 肉]
  \definition{s.}{cintura}
\end{entry}

\begin{entry}{腰包}{yao1bao1}{13,5}
  \definition{s.}{pochete | bolso}
\end{entry}

\begin{entry}{腰椎}{yao1zhui1}{13,12}
  \definition{s.}{vértebra lombar (espinha dorsal inferior)}
\end{entry}

\begin{entry}{摇晃}{yao2huang4}{13,10}
  \definition{v.}{sacudir | agitar | balançar | chacoalhar}
\end{entry}

\begin{entry}{摇头}{yao2tou2}{13,5}
  \definition{v.+compl.}{balançar a cabeça de alguém}
\end{entry}

\begin{entry}{遥控}{yao2kong4}{13,11}
  \definition{s.}{controle remoto}
  \definition{v.}{dirigir operações de um local remoto | controlar remotamente}
\end{entry}

\begin{entry}{药}{yao4}{9}[Radical 艸][HSK 2]
  \definition[种,服,味]{s.}{medicamento | remédio | droga}
\end{entry}

\begin{entry}{药补}{yao4bu3}{9,7}
  \definition{s.}{suplemento dietético medicinal que ajuda a melhorar a saúde}
\end{entry}

\begin{entry}{药典}{yao4dian3}{9,8}
  \definition{s.}{farmacopéia}
\end{entry}

\begin{entry}{药店}{yao4 dian4}{9,8}[HSK 2]
  \definition{s.}{farmácia | drogaria | loja de produtos químicos}
\end{entry}

\begin{entry}{药房}{yao4fang2}{9,8}
  \definition{s.}{farmácia | drogaria}
\end{entry}

\begin{entry}{药罐}{yao4guan4}{9,23}
  \definition{s.}{frasco de remédio}
\end{entry}

\begin{entry}{药片}{yao4 pian4}{9,4}[HSK 2]
  \definition[片]{s.}{uma pílula ou comprimido (remédio)}
\end{entry}

\begin{entry}{药品}{yao4pin3}{9,9}
  \definition{s.}{medicamento | remédio | droga}
\end{entry}

\begin{entry}{药签}{yao4qian1}{9,13}
  \definition{s.}{cotonete médico}
\end{entry}

\begin{entry}{药膳}{yao4shan4}{9,16}
  \definition{s.}{dieta medicinal}
\end{entry}

\begin{entry}{药水}{yao4 shui3}{9,4}[HSK 2]
  \definition{s.}{remédio engarrafado | loção | medicamento em forma líquida}
\end{entry}

\begin{entry}{药丸}{yao4wan2}{9,3}
  \definition[粒]{s.}{pílula}
\end{entry}

\begin{entry}{要}{yao4}{9}[Radical 襾][HSK 1]
  \definition{adj.}{(forma ligada) importante}
  \definition{conj.}{se (o mesmo que  要是)}
  \definition{v.}{querer | precisar | pedir por | precisar de}
  \seeref{要}{yao1}
  \seealsoref{要是}{yao4shi5}
\end{entry}

\begin{entry}{要不}{yao4bu4}{9,4}
  \definition{conj.}{de outra forma | se não | outro | ou}
\end{entry}

\begin{entry}{要不然}{yao4bu4ran2}{9,4,12}
  \definition{conj.}{de outra forma | se não | outro | ou}
\end{entry}

\begin{entry}{要点}{yao4dian3}{9,9}
  \definition{s.}{pontos principais | essencial}
\end{entry}

\begin{entry}{要好}{yao4hao3}{9,6}
  \definition{v.}{ser amigos íntimos | estar em boas condições}
\end{entry}

\begin{entry}{要谎}{yao4huang3}{9,11}
  \definition{v.}{pedir um preço enorme (como primeiro passo de negociação)}
\end{entry}

\begin{entry}{要么……要么……}{yao4me5 yao4me5}{9,3,9,3}
  \definition{conj.}{ou\dots ou\dots}
\end{entry}

\begin{entry}{要强}{yao4qiang2}{9,12}
  \definition{adj.}{ansioso para se destacar | ansioso para progredir na vida | obstinado}
\end{entry}

\begin{entry}{要是}{yao4shi5}{9,9}
  \definition{conj.}{se | no caso de | no evento de | supondo que}
\end{entry}

\begin{entry}{要是……的话}{yao4shi5 de5hua4}{9,9,8,8}
  \definition{conj.}{se\dots no caso de}
\end{entry}

\begin{entry}{要死}{yao4si3}{9,6}
  \definition{adv.}{extremamente | muito}
\end{entry}

\begin{entry}{要义}{yao4yi4}{9,3}
  \definition{s.}{resumo | o essencial}
\end{entry}

\begin{entry}{钥匙}{yao4shi5}{9,11}
  \definition[把]{s.}{chave}
\end{entry}

\begin{entry}{钥匙洞孔}{yao4shi5dong4kong3}{9,11,9,4}
  \definition{s.}{buraco da fechadura}
\end{entry}

\begin{entry}{钥匙卡}{yao4shi5ka3}{9,11,5}
  \definition{s.}{cartão de acesso}
\end{entry}

\begin{entry}{钥匙孔}{yao4shi5kong3}{9,11,4}
  \definition{s.}{buraco da fechadura}
\end{entry}

\begin{entry}{钥匙圈}{yao4shi5quan1}{9,11,11}
  \definition{s.}{chaveiro}
\end{entry}

\begin{entry}{椰汁}{ye1zhi1}{12,5}
  \definition{s.}{água de coco}
\end{entry}

\begin{entry}{噎}{ye1}{15}[Radical 口]
  \definition{v.}{engasgar | sufocar}
\end{entry}

\begin{entry}{爷爷}{ye2ye5}{6,6}[HSK 1]
  \definition[个]{s.}{avô (paterno)}
\end{entry}

\begin{entry}{也}{ye3}{3}[Radical 乙][HSK 1]
  \definition*{s.}{sobrenome Ye}
  \definition{adv.}{também | (em frases negativas) nem, tampouco}
\end{entry}

\begin{entry}{也就是}{ye3jiu4shi4}{3,12,9}
  \definition{adv.}{i.e., isso é | ou seja}
\end{entry}

\begin{entry}{也就是说}{ye3jiu4shi4shuo1}{3,12,9,9}
  \definition{adv.}{em outras palavras | então | isto é | por isso}
\end{entry}

\begin{entry}{也许}{ye3xu3}{3,6}[HSK 2]
  \definition{adv.}{possivelmente | talvez}
\end{entry}

\begin{entry}{也有今天}{ye3you3jin1tian1}{3,6,4,4}
  \definition{expr.}{obter apenas o que merece | todo cachorro tem seu dia | obter a sua parte (coisas boas ou ruins) | servir alguém bem}
\end{entry}

\begin{entry}{野}{ye3}{11}[Radical 里]
  \definition{adj.}{selvagem | rude}
  \definition{s.}{campo | espaço aberto | limite}
\end{entry}

\begin{entry}{野生}{ye3sheng1}{11,5}
  \definition{adj.}{selvagem | não domesticado}
\end{entry}

\begin{entry}{页}{ye4}{6}[Radical 頁][HSK 1]
  \definition{clas.}{páginas}
  \definition{s.}{página | folha}
  \seeref{页}{xie2}
\end{entry}

\begin{entry}{夜}{ye4}{8}[Radical 夕][HSK 2]
  \definition{s.}{noite}
\end{entry}

\begin{entry}{夜场}{ye4chang3}{8,6}
  \definition{s.}{show noturno (em um teatro, etc.) | local de entretenimento noturno (bar, boate, discoteca, etc.)}
\end{entry}

\begin{entry}{夜店}{ye4dian4}{8,8}
  \definition{s.}{boate | \emph{nightclub}}
\end{entry}

\begin{entry}{夜里}{ye4li5}{8,7}[HSK 2]
  \definition{adv.}{à noite | durante a noite | período noturno}
\end{entry}

\begin{entry}{夜幕}{ye4mu4}{8,13}
  \definition{s.}{cortina da noite}
\end{entry}

\begin{entry}{夜鸟}{ye4niao3}{8,5}
  \definition{s.}{ave noturna}
\end{entry}

\begin{entry}{夜深人静}{ye4shen1ren2jing4}{8,11,2,14}
  \definition{expr.}{"Na calada da noite."}
\end{entry}

\begin{entry}{夜生活}{ye4sheng1huo2}{8,5,9}
  \definition{s.}{vida noturna}
\end{entry}

\begin{entry}{夜晚}{ye4wan3}{8,11}
  \definition[个]{s.}{noite}
\end{entry}

\begin{entry}{夜夜}{ye4ye4}{8,8}
  \definition{adv.}{toda noite}
\end{entry}

\begin{entry}{液体}{ye4ti3}{11,7}
  \definition{adj./s.}{líquido}
\end{entry}

\begin{entry}{一}{yi1}[(quando usado sozinho)]{1}[Radical 一][Kangxi 1][HSK 1]
  \definition{num.}{um; 1 | pronunciado como \dpy{yao1} quando dito número a número}
  \seeref{一}{yi2}
  \seeref{一}{yi4}
\end{entry}

\begin{entry}{一……就……}{yi1 jiu4}{1,12}
  \definition{expr.}{logo que |  uma vez que}
\end{entry}

\begin{entry}{一行}{yi1xing2}{1,6}
  \definition{s.}{festa | delegação}
\end{entry}

\begin{entry}{伊朗}{yi1lang3}{6,10}
  \definition*{s.}{Irã}
\end{entry}

\begin{entry}{伊马姆}{yi1ma3mu3}{6,3,8}
  \definition*{s.}{Islã}
  \seealsoref{伊玛目}{yi1ma3mu4}
  \seealsoref{伊曼}{yi1man4}
  \seealsoref{伊斯兰}{yi1si1lan2}
\end{entry}

\begin{entry}{伊玛目}{yi1ma3mu4}{6,7,5}
  \definition*{s.}{Islã}
  \seealsoref{伊马姆}{yi1ma3mu3}
  \seealsoref{伊曼}{yi1man4}
  \seealsoref{伊斯兰}{yi1si1lan2}
\end{entry}

\begin{entry}{伊曼}{yi1man4}{6,11}
  \definition*{s.}{Islã}
  \seealsoref{伊马姆}{yi1ma3mu3}
  \seealsoref{伊玛目}{yi1ma3mu4}
  \seealsoref{伊斯兰}{yi1si1lan2}
\end{entry}

\begin{entry}{伊斯兰}{yi1si1lan2}{6,12,5}
  \definition*{s.}{Islã}
  \seealsoref{伊马姆}{yi1ma3mu3}
  \seealsoref{伊玛目}{yi1ma3mu4}
  \seealsoref{伊曼}{yi1man4}
\end{entry}

\begin{entry}{衣}{yi1}{6}[Radical 衣][Kangxi 145]
  \definition[件]{s.}{roupa}
  \seeref{衣}{yi4}
\end{entry}

\begin{entry}{衣服}{yi1fu5}{6,8}[HSK 1]
  \definition[件,套]{s.}{roupa | vestuário}
\end{entry}

\begin{entry}{衣柜}{yi1gui4}{6,8}
  \definition[个]{s.}{armário | guarda-roupa}
\end{entry}

\begin{entry}{衣甲}{yi1jia3}{6,5}
  \definition{s.}{armadura}
\end{entry}

\begin{entry}{医}{yi1}{7}[Radical 匸]
  \definition{s.}{médico | medicina}
  \definition{v.}{curar | tratar}
\end{entry}

\begin{entry}{医生}{yi1sheng1}{7,5}[HSK 1]
  \definition[个,位,名]{s.}{médico | clínico}
\end{entry}

\begin{entry}{医院}{yi1yuan4}{7,9}[HSK 1]
  \definition[所,家,座]{s.}{hospital}
\end{entry}

\begin{entry}{依然}{yi1ran2}{8,12}
  \definition{adv.}{como era antes | ainda}
\end{entry}

\begin{entry}{依偎}{yi1wei1}{8,11}
  \definition{v.}{aninhar-se | aconchegar-se}
\end{entry}

\begin{entry}{毉}{yi1}{18}[Radical 殳]
  \variantof{医}
\end{entry}

\begin{entry}{一}{yi2}[(antes de quarto tom)]{1}[Radical 一][HSK 1]
  \definition{num.}{um; 1 | um (artigo)}
  \seeref{一}{yi1}
  \seeref{一}{yi4}
\end{entry}

\begin{entry}{一半}{yi2ban4}{1,5}[HSK 1]
  \definition{num.}{meio | metade}
\end{entry}

\begin{entry}{一部分}{yi2 bu4 fen4}{1,10,4}[HSK 2]
  \definition{adv.}{parcialmente}
  \definition{num.}{parte | porção | seção | fração}
  \definition[把]{s.}{parcial}
\end{entry}

\begin{entry}{一道}{yi2dao4}{1,12}
  \definition{adv.}{juntos | ao lado}
\end{entry}

\begin{entry}{一定}{yi2ding4}{1,8}[HSK 2]
  \definition{adv.}{certamente | definitivamente}
\end{entry}

\begin{entry}{一个样}{yi2ge5yang4}{1,3,10}
  \definition{adj.}{igual | mesmo}
  \seeref{一样}{yi2yang4}
\end{entry}

\begin{entry}{一共}{yi2gong4}{1,6}[HSK 2]
  \definition{adv.}{completamente | no total | no todo | em suma}
\end{entry}

\begin{entry}{一会儿}{yi2 hui4r5}{1,6,2}[HSK 1]
  \definition{adv.}{daqui a pouco tempo | pouco tempo}
\end{entry}

\begin{entry}{一块}{yi2kuai4}{1,7}
  \definition{adv.}{(principalmente mandarim) juntos}
\end{entry}

\begin{entry}{一块儿}{yi2kuai4r5}{1,7,2}[HSK 1]
  \definition{adv.}{juntos}
\end{entry}

\begin{entry}{一路平安}{yi2 lu4 ping2 an1}{1,13,5,6}[HSK 2]
  \definition{expr.}{Boa viagem!}
  \definition{v.}{ter uma viagem agradável}
\end{entry}

\begin{entry}{一路顺风}{yi2 lu4 shun4 feng1}{1,13,9,4}[HSK 2]
  \definition{expr.}{ter uma viagem agradável}
\end{entry}

\begin{entry}{一下}{yi2xia4}{1,3}
  \definition{adv.}{em um curto tempo | rapidamente}
\end{entry}

\begin{entry}{一下儿}{yi2xia4r5}{1,3,2}[HSK 1]
  \definition{adv.}{um pouco}
\end{entry}

\begin{entry}{一样}{yi2yang4}{1,10}[HSK 1]
  \definition{adj.}{igual | mesmo}
\end{entry}

\begin{entry}{一战}{yi2zhan4}{1,9}
  \definition*{s.}{Primeira Guerra Mundial}
\end{entry}

\begin{entry}{仪式}{yi2shi4}{5,6}
  \definition{s.}{cerimônia}
\end{entry}

\begin{entry}{遗案}{yi2'an4}{12,10}
  \definition{s.}{(lei) caso não resolvido}
\end{entry}

\begin{entry}{遗产}{yi2chan3}{12,6}
  \definition[笔]{s.}{legado | herança}
\end{entry}

\begin{entry}{遗骸}{yi2hai2}{12,15}
  \definition{v.}{restos mortais}
\end{entry}

\begin{entry}{遗憾}{yi2han4}{12,16}
  \definition{v.}{ter pena de | lamentar}
\end{entry}

\begin{entry}{遗迹}{yi2ji4}{12,9}
  \definition{s.}{vestígios históricos | remanescente | vestígio}
\end{entry}

\begin{entry}{遗落}{yi2luo4}{12,12}
  \definition{v.}{esquecer | deixar para trás (inadvertidamente) | deixar de fora | omitir}
\end{entry}

\begin{entry}{遗男}{yi2nan2}{12,7}
  \definition{s.}{órfão | filho póstumo}
\end{entry}

\begin{entry}{遗嘱}{yi2zhu3}{12,15}
  \definition{s.}{testamento}
\end{entry}

\begin{entry}{颐和园}{yi2he2yuan2}{13,8,7}
  \definition*{s.}{Palácio de Verão}
\end{entry}

\begin{entry}{已}{yi3}{3}[Radical 己]
  \definition{adv.}{já | após | então}
\end{entry}

\begin{entry}{已故}{yi3gu4}{3,9}
  \definition{adj.}{morto | atrasado}
\end{entry}

\begin{entry}{已婚}{yi3hun1}{3,11}
  \definition{adj.}{casado}
\end{entry}

\begin{entry}{已经}{yi3jing1}{3,8}[HSK 2]
  \definition{adv.}{já}
\end{entry}

\begin{entry}{已久}{yi3jiu3}{3,3}
  \definition{adv.}{já faz muito tempo}
\end{entry}

\begin{entry}{已灭}{yi3mie4}{3,5}
  \definition{adj.}{extinto}
\end{entry}

\begin{entry}{已然}{yi3ran2}{3,12}
  \definition{adv.}{já | já ser assim}
\end{entry}

\begin{entry}{已知}{yi3zhi1}{3,8}
  \definition{v.}{conhecido (ter ciência)}
\end{entry}

\begin{entry}{以便}{yi3bian4}{4,9}
  \definition{conj.}{a fim de | para que | assim como}
\end{entry}

\begin{entry}{以此}{yi3ci3}{4,6}
  \definition{adv.}{devido a esta | deste modo | por isso | com isso}
\end{entry}

\begin{entry}{以后}{yi3 hou4}{4,6}[HSK 2]
  \definition{adv.}{depois de | depois | após}
\end{entry}

\begin{entry}{以及}{yi3ji2}{4,3}
  \definition{conj.}{assim como | juntamente como}
\end{entry}

\begin{entry}{以来}{yi3lai2}{4,7}
  \definition{prep.}{desde (um evento anterior)}
\end{entry}

\begin{entry}{以免}{yi3mian3}{4,7}
  \definition{conj.}{para evitar isso}
\end{entry}

\begin{entry}{以期}{yi3qi1}{4,12}
  \definition{v.}{tentando | esperando | esperando por}
\end{entry}

\begin{entry}{以前}{yi3qian2}{4,9}[HSK 2]
  \definition{adv.}{antes de | antes}
\end{entry}

\begin{entry}{以求}{yi3qiu2}{4,7}
  \definition{conj.}{a fim de}
\end{entry}

\begin{entry}{以色列}{yi3se4lie4}{4,6,6}
  \definition*{s.}{Israel}
\end{entry}

\begin{entry}{以上}{yi3 shang4}{4,3}[HSK 2]
  \definition{s.}{mais que | sobre | acima | o acima | o precedente | o acima mencionado}
\end{entry}

\begin{entry}{以外}{yi3 wai4}{4,5}[HSK 2]
  \definition{s.}{além | exceto | fora | diferente de}
\end{entry}

\begin{entry}{以为}{yi3wei2}{4,4}[HSK 2]
  \definition{v.}{pensar, ou seja, considerar que\dots (geralmente há uma implicação de que a noção está errada --- exceto ao expressar a própria opnião atual)}
\end{entry}

\begin{entry}{以下}{yi3 xia4}{4,3}[HSK 2]
  \definition[所]{s.}{abaixo | sob | seguinte}
\end{entry}

\begin{entry}{以至}{yi3zhi4}{4,6}
  \definition{adv.}{até}
  \definition{conj.}{a tal ponto que\dots}
  \seealsoref{以至于}{yi3zhi4yu2}
\end{entry}

\begin{entry}{以至于}{yi3zhi4yu2}{4,6,3}
  \definition{adv.}{até}
  \definition{conj.}{na medida em que\dots}
  \seealsoref{以至}{yi3zhi4}
\end{entry}

\begin{entry}{椅子}{yi3zi5}{12,3}[HSK 2]
  \definition[把,套]{s.}{cadeira}
\end{entry}

\begin{entry}{一}{yi4}{1}[Radical 一][HSK 1]
  \definition{adv.}{uma vez | assim que | ao}
  \definition{num.}{um; 1 | um (artigo)}
  \seeref{一}{yi1}
  \seeref{一}{yi2}
\end{entry}

\begin{entry}{一般}{yi4ban1}{1,10}[HSK 2]
  \definition{adj.}{geral | comum | normal}
  \definition{adv.}{normalmente}
\end{entry}

\begin{entry}{一边}{yi4bian1}{1,5}[HSK 1]
  \definition{adj.}{mesmo | igual}
  \definition{adv.}{enquanto | como | ao mesmo tempo | simultaneamente}
  \definition{s.}{lado | um lado | cada lado | ao lado de}
\end{entry}

\begin{entry}{一点点}{yi4 dian3 dian3}{1,9,9}[HSK 2]
  \definition{adj.}{um pouco}
\end{entry}

\begin{entry}{一点儿}{yi4dian3r5}{1,9,2}[HSK 1]
  \definition{adv.}{um pouco (``{adj.}+一点儿'' ou ``一点儿+{s.}'') | um ponto}
\end{entry}

\begin{entry}{一齐}{yi4qi2}{1,6}
  \definition{adv.}{tudo ao mesmo tempo | em uníssono | junto}
\end{entry}

\begin{entry}{一起}{yi4qi3}{1,10}[HSK 1]
  \definition{adv.}{juntamente | em conjunto | no mesmo lugar | completamente | em todos}
\end{entry}

\begin{entry}{一生}{yi4 sheng1}{1,5}[HSK 2]
  \definition{s.}{toda a vida | ao longo da vida | a vida de alguém}
\end{entry}

\begin{entry}{一时}{yi4shi2}{1,7}
  \definition{adv.}{por pouco tempo | por um tempo | temporariamente | momentaneamente | uma vez | de tempos em tempos | ocasionalmente}
\end{entry}

\begin{entry}{一同}{yi4tong2}{1,6}
  \definition{adv.}{juntos, ao mesmo tempo}
\end{entry}

\begin{entry}{一些}{yi4xie1}{1,8}[HSK 1]
  \definition{pron.}{uns | alguns}
\end{entry}

\begin{entry}{一直}{yi4zhi2}{1,8}[HSK 2]
  \definition{adv.}{diretamente | sempre em frente | o tempo todo | sempre | constantemente}
\end{entry}

\begin{entry}{亿}{yi4}{3}[Radical 人][HSK 2]
  \definition{num.}{cem milhões; 100.000.000; 1.0000.0000}
\end{entry}

\begin{entry}{艺人}{yi4ren2}{4,2}
  \definition{s.}{artista | ator}
\end{entry}

\begin{entry}{亦}{yi4}{6}[Radical 亠]
  \definition{adv.}{também | igualmente | apenas | embora | já}
\end{entry}

\begin{entry}{异常}{yi4chang2}{6,11}
  \definition{adj.}{extraordinário | anormal}
  \definition{adv.}{extraordinariamente | excepcionalmente}
  \definition{s.}{anormalidade}
\end{entry}

\begin{entry}{衣}{yi4}{6}[Radical 衣]
  \definition{v.}{vestir | vestir-se}
  \seeref{衣}{yi1}
\end{entry}

\begin{entry}{意见}{yi4jian4}{13,4}[HSK 2]
  \definition[点,条]{s.}{reclamação | ideia | objeção | opinião | sugestão}
\end{entry}

\begin{entry}{意识}{yi4shi2}{13,7}
  \definition{s.}{consciência}
  \definition{v.}{(usualmente seguido de 到) estar ciente, constatar}
\end{entry}

\begin{entry}{意思}{yi4si5}{13,9}[HSK 2]
  \definition[个]{s.}{interesse}
\end{entry}

\begin{entry}{意外}{yi4wai4}{13,5}
  \definition{adj.}{inesperado}
  \definition{adv.}{acidentalmente}
  \definition[个]{s.}{acidente}
\end{entry}

\begin{entry}{意义}{yi4yi4}{13,3}
  \definition[个]{s.}{importância | significado | senso | desejo | força de vontade}
\end{entry}

\begin{entry}{意译}{yi4yi4}{13,7}
  \definition{s.}{tradução livre | significado (de expressão estrangeira) | paráfrase | tradução do significado (em oposição à tradução literal)}
  \seealsoref{直译}{zhi2yi4}
\end{entry}

\begin{entry}{意指}{yi4zhi3}{13,9}
  \definition{v.}{implicar | significar}
\end{entry}

\begin{entry}{意志}{yi4zhi4}{13,7}
  \definition[个]{s.}{determinação | desejo | força de vontade}
\end{entry}

\begin{entry}{因此}{yin1ci3}{6,6}
  \definition{conj.}{então | portanto | por esta razão | consequentemente}
\end{entry}

\begin{entry}{因而}{yin1'er2}{6,6}
  \definition{conj.}{então | portanto | por esta razão | consequentemente}
\end{entry}

\begin{entry}{因为}{yin1wei4}{6,4}[HSK 2]
  \definition{conj.}{porque | devido a | por conta de}
\end{entry}

\begin{entry}{因为……所以……}{yin1wei4 suo3yi3}{6,4,8,4}
  \definition{conj.}{porque\dots portanto\dots}
\end{entry}

\begin{entry}{阴}{yin1}{6}[Radical 阜][HSK 2]
  \definition*{s.}{Yin (o princípio negativo de Yin e Yang) | sobrenome Yin}
  \definition{adj.}{nublado | sombrio | escondido | implícito}
  \definition{s.}{negativo (eletricidade) | lua}
  \seealsoref{阳}{yang2}
  \seealsoref{阴阳}{yin1yang2}
\end{entry}

\begin{entry}{阴天}{yin1 tian1}{6,4}[HSK 2]
  \definition{adj.}{céu nublado | céu cinzento}
\end{entry}

\begin{entry}{阴阳}{yin1yang2}{6,6}
  \definition*{s.}{Yin e Yang}
  \seealsoref{阳}{yang2}
  \seealsoref{阴}{yin1}
\end{entry}

\begin{entry}{音节}{yin1 jie2}{9,5}[HSK 2]
  \definition{s.}{sílaba}
\end{entry}

\begin{entry}{音乐}{yin1yue4}{9,5}[HSK 2]
  \definition[张,曲,段]{s.}{música}
\end{entry}

\begin{entry}{音乐光碟}{yin1yue4guang1die2}{9,5,6,14}
  \definition{s.}{CD de música}
\end{entry}

\begin{entry}{音乐会}{yin1 yue4 hui4}{9,5,6}[HSK 2]
  \definition[场]{s.}{concerto}
\end{entry}

\begin{entry}{音乐家}{yin1yue4jia1}{9,5,10}
  \definition{s.}{músico}
\end{entry}

\begin{entry}{音乐节}{yin1yue4jie2}{9,5,5}
  \definition{s.}{festival de música}
\end{entry}

\begin{entry}{音乐厅}{yin1yue4ting1}{9,5,4}
  \definition{s.}{auditório | teatro | \emph{concert hall}}
\end{entry}

\begin{entry}{音乐学}{yin1yue4xue2}{9,5,8}
  \definition{s.}{musicologia}
\end{entry}

\begin{entry}{音乐学院}{yin1yue4xue2yuan4}{9,5,8,9}
  \definition{s.}{conservatório | academia de música}
\end{entry}

\begin{entry}{音乐院}{yin1yue4yuan4}{9,5,9}
  \definition{s.}{conservatório | instituto de música}
\end{entry}

\begin{entry}{吟诗}{yin2shi1}{7,8}
  \definition{v.}{recitar poesia}
\end{entry}

\begin{entry}{银行}{yin2hang2}{11,6}[HSK 2]
  \definition[家,个]{s.}{banco | agência bancária}
\end{entry}

\begin{entry}{银行卡}{yin2 hang2 ka3}{11,6,5}[HSK 2]
  \definition{s.}{cartão bancário}
\end{entry}

\begin{entry}{银河}{yin2he2}{11,8}
  \definition*{s.}{Via Láctea}
  \seealsoref{银河系}{yin2he2xi4}
\end{entry}

\begin{entry}{银河系}{yin2he2xi4}{11,8,7}
  \definition*{s.}{Galáxia Via Láctea}
  \seealsoref{银河}{yin2he2}
\end{entry}

\begin{entry}{银色}{yin2se4}{11,6}
  \definition{s.}{prateado}
\end{entry}

\begin{entry}{引擎}{yin3qing2}{4,16}
  \definition[台]{s.}{motor | (empréstimo linguístico) \emph{engine}}
\end{entry}

\begin{entry}{饮料}{yin3liao4}{7,10}
  \definition{s.}{bebida}
\end{entry}

\begin{entry}{应该}{ying1gai1}{7,8}[HSK 2]
  \definition{v.}{dever | ter de}
\end{entry}

\begin{entry}{英国}{ying1guo2}{8,8}
  \definition*{s.}{Reino Unido}
\end{entry}

\begin{entry}{英国人}{ying1guo2ren2}{8,8,2}
  \definition{s.}{inglês | pessoa ou povo do Reino Unido}
\end{entry}

\begin{entry}{英文}{ying1 wen2}{8,4}[HSK 2]
  \definition{s.}{inglês, língua inglesa}
\end{entry}

\begin{entry}{英雄}{ying1xiong2}{8,12}
  \definition[个]{s.}{herói}
\end{entry}

\begin{entry}{英语}{ying1 yu3}{8,9}[HSK 2]
  \definition{s.}{inglês, língua inglesa}
\end{entry}

\begin{entry}{樱桃}{ying1tao2}{15,10}
  \definition{s.}{cereja}
\end{entry}

\begin{entry}{鹦鹉}{ying1wu3}{16,13}
  \definition{s.}{papagaio (ave)}
\end{entry}

\begin{entry}{影片}{ying3 pian4}{15,4}[HSK 2]
  \definition[部]{s.}{filme | imagem}
\end{entry}

\begin{entry}{影响}{ying3xiang3}{15,9}[HSK 2]
  \definition{s.}{efeito | influência}
  \definition{v.}{afetar | influenciar}
\end{entry}

\begin{entry}{影像}{ying3xiang4}{15,13}
  \definition{s.}{imagem}
\end{entry}

\begin{entry}{应对}{ying4dui4}{7,5}
  \definition{v.}{responder | manusear | lidar}
\end{entry}

\begin{entry}{应用程序}{ying4yong4cheng2xu4}{7,5,12,7}
  \definition{s.}{aplicativo | programa de computador}
\end{entry}

\begin{entry}{应用程序编程接口}{ying4yong4cheng2xu4bian1cheng2jie1kou3}{7,5,12,7,12,12,11,3}
  \definition{s.}{API (\emph{application programming interface})}
  \seealsoref{应用程序接口}{ying4yong4cheng2xu4jie1kou3}
\end{entry}

\begin{entry}{应用程序接口}{ying4yong4cheng2xu4jie1kou3}{7,5,12,7,11,3}
  \definition{s.}{API (\emph{application programming interface})}
  \seealsoref{应用程序编程接口}{ying4yong4cheng2xu4bian1cheng2jie1kou3}
\end{entry}

\begin{entry}{硬件}{ying4jian4}{12,6}
  \definition{s.}{\emph{hardware}}
\end{entry}

\begin{entry}{永不}{yong3bu4}{5,4}
  \definition{adv.}{nunca}
\end{entry}

\begin{entry}{永远}{yong3yuan3}{5,7}[HSK 2]
  \definition{adv.}{para sempre, sempre | permanentemente}
\end{entry}

\begin{entry}{泳池}{yong3chi2}{8,6}
  \definition{s.}{piscina}
  \seealsoref{游泳池}{you2yong3chi2}
  \seealsoref{游泳馆}{you2yong3guan3}
\end{entry}

\begin{entry}{泳衣}{yong3yi1}{8,6}
  \definition{s.}{roupa de banho | maiô}
  \seealsoref{游泳衣}{you2yong3yi1}
\end{entry}

\begin{entry}{勇敢}{yong3gan3}{9,11}
  \definition{adj.}{bravo | corajoso}
\end{entry}

\begin{entry}{勇气}{yong3qi4}{9,4}
  \definition{adj.}{coragem | valor}
\end{entry}

\begin{entry}{勇士}{yong3shi4}{9,3}
  \definition{s.}{um guerreiro | uma pessoa corajosa}
\end{entry}

\begin{entry}{用}{yong4}{5}[Radical 用][Kangxi 101][HSK 1]
  \definition{v.}{usar}
\end{entry}

\begin{entry}{用处}{yong4chu5}{5,5}
  \definition[个]{s.}{usabilidade | utilidade}
\end{entry}

\begin{entry}{用料}{yong4liao4}{5,10}
  \definition{s.}{ingredientes | materiais}
\end{entry}

\begin{entry}{用心}{yong4xin1}{5,4}
  \definition{s.}{motivo | intenção}
  \definition{v.+compl.}{ser diligente ou atencioso}
\end{entry}

\begin{entry}{优}{you1}{6}[Radical 人]
  \definition{adj.}{excelente | superior}
\end{entry}

\begin{entry}{优等}{you1deng3}{6,12}
  \definition{adj.}{excelente | de primeira linha | alta classe | da mais alta ordem, superior}
\end{entry}

\begin{entry}{优点}{you1dian3}{6,9}
  \definition[个,项]{s.}{vantagem | benefício | mérito | ponto forte}
\end{entry}

\begin{entry}{优格}{you1ge2}{6,10}
  \definition{s.}{iogurte}
\end{entry}

\begin{entry}{优厚}{you1hou4}{6,9}
  \definition{adj.}{generoso}
\end{entry}

\begin{entry}{优伶}{you1ling2}{6,7}
  \definition{s.}{ator}
\end{entry}

\begin{entry}{优美}{you1mei3}{6,9}
  \definition{adj.}{gracioso | fino | elegante}
\end{entry}

\begin{entry}{优盘}{you1pan2}{6,11}
  \definition{s.}{unidade de memória USB}
  \seealsoref{闪存盘}{shan3cun2pan2}
\end{entry}

\begin{entry}{优先}{you1xian1}{6,6}
  \definition{v.}{ter prioridade | ter precedência}
\end{entry}

\begin{entry}{优秀}{you1xiu4}{6,7}
  \definition{adj.}{excelente | fora do comum}
\end{entry}

\begin{entry}{优选}{you1xuan3}{6,9}
  \definition{v.}{otimizar}
\end{entry}

\begin{entry}{优于}{you1yu2}{6,3}
  \definition{v.}{superar}
\end{entry}

\begin{entry}{优裕}{you1yu4}{6,12}
  \definition{adj.}{abundante | bastante}
  \definition{s.}{abundância}
\end{entry}

\begin{entry}{优质}{you1zhi4}{6,8}
  \definition{adj.}{excelente qualidade}
\end{entry}

\begin{entry}{忧郁}{you1yu4}{7,8}
  \definition{adj.}{deprimido | melancólico | desanimado}
  \definition{s.}{depressão | melancolia}
\end{entry}

\begin{entry}{尤其}{you2qi2}{4,8}
  \definition{adv.}{especialmente | particularmente}
\end{entry}

\begin{entry}{邮包}{you2bao1}{7,5}
  \definition{s.}{encomenda postal}
\end{entry}

\begin{entry}{邮递}{you2di4}{7,10}
  \definition{v.}{enviar por correio}
\end{entry}

\begin{entry}{邮电}{you2dian4}{7,5}
  \definition*{s.}{Correios e Telecomunicações}
\end{entry}

\begin{entry}{邮费}{you2fei4}{7,9}
  \definition{s.}{postagem}
  \definition{v.}{postar}
\end{entry}

\begin{entry}{邮件}{you2jian4}{7,6}
  \definition{s.}{correspondência | \emph{e-mail}}
\end{entry}

\begin{entry}{邮局}{you2ju2}{7,7}
  \definition[家,个]{s.}{correio | agência dos correios}
\end{entry}

\begin{entry}{邮迷}{you2mi2}{7,9}
  \definition{s.}{filatelista | colecionador de selos}
\end{entry}

\begin{entry}{邮票}{you2piao4}{7,11}
  \definition[枚,张]{s.}{selo postal}
\end{entry}

\begin{entry}{邮市}{you2shi4}{7,5}
  \definition{s.}{mercado postal}
\end{entry}

\begin{entry}{邮资}{you2zi1}{7,10}
  \definition{s.}{postagem}
\end{entry}

\begin{entry}{油}{you2}{8}[Radical 水][HSK 2]
  \definition{adj.}{oleoso | gorduroso | superficial | astuto}
  \definition{s.}{óleo | gordura | graxa | petróleo}
  \definition{v.}{aplicar óleo de tungue, tinta ou verniz}
\end{entry}

\begin{entry}{游客}{you2 ke4}{12,9}[HSK 2]
  \definition{s.}{viajante | turista | (jogo online) jogador convidado}
\end{entry}

\begin{entry}{游艇}{you2ting3}{12,12}
  \definition[只]{s.}{barcaça | iate}
\end{entry}

\begin{entry}{游戏}{you2xi4}{12,6}
  \definition[场]{s.}{jogo}
  \definition{v.}{jogar}
\end{entry}

\begin{entry}{游泳}{you2yong3}{12,8}
  \definition{s.}{natação}
  \definition{v.+compl.}{nadar}
\end{entry}

\begin{entry}{游泳池}{you2yong3chi2}{12,8,6}
  \definition[场]{s.}{piscina}
  \seealsoref{泳池}{yong3chi2}
  \seealsoref{游泳馆}{you2yong3guan3}
\end{entry}

\begin{entry}{游泳馆}{you2yong3guan3}{12,8,11}
  \definition[场]{s.}{piscina}
  \seealsoref{泳池}{yong3chi2}
  \seealsoref{游泳池}{you2yong3chi2}
\end{entry}

\begin{entry}{游泳镜}{you2yong3jing4}{12,8,16}
  \definition{s.}{óculos de natação}
\end{entry}

\begin{entry}{游泳衣}{you2yong3yi1}{12,8,6}
  \definition{s.}{roupa de banho}
  \seealsoref{泳衣}{yong3yi1}
\end{entry}

\begin{entry}{友好}{you3hao3}{4,6}[HSK 2]
  \definition{adj.}{amigável}
  \definition{s.}{amigo próximo, íntimo}
\end{entry}

\begin{entry}{有}{you3}{6}[Radical 月][HSK 1]
  \definition{v.}{ter | haver | existir}
\end{entry}

\begin{entry}{有道理}{you3dao4li5}{6,12,11}
  \definition{adj.}{razoável}
  \definition{v.}{fazer sentido}
\end{entry}

\begin{entry}{有的}{you3de5}{6,8}[HSK 1]
  \definition{pron.}{algum, alguns}
\end{entry}

\begin{entry}{有的时候}{you3de5shi2hou4}{6,8,7,10}
  \definition{adv.}{às vezes | de vez em quando | de quando em quando}
  \seealsoref{有时}{you3shi2}
  \seealsoref{有时候}{you3shi2hou5}
\end{entry}

\begin{entry}{有点儿}{you3dian3r5}{6,9,2}[HSK 2]
  \definition{adv.}{um pouco (``有点儿+s. ou v. mental'')}
\end{entry}

\begin{entry}{有空儿}{you3 kong4r5}{6,8,2}[HSK 2]
  \definition{v.}{ser livre}
\end{entry}

\begin{entry}{有名}{you3 ming2}{6,6}[HSK 1]
  \definition{adj.}{famoso | conhecido}
\end{entry}

\begin{entry}{有名无实}{you3ming2wu2shi2}{6,6,4,8}
  \definition{v.}{(literal) tem um nome, mas não tem realidade | existe apenas no nome}
\end{entry}

\begin{entry}{有人}{you3 ren2}{6,2}[HSK 2]
  \definition{pron.}{qualquer um | alguém}
  \definition[所]{s.}{ocupado (como no banheiro) | pessoas}
\end{entry}

\begin{entry}{有时}{you3shi2}{6,7}[HSK 1]
  \definition{expr.}{às vezes | de vez em quando | de quando em quando}
  \seealsoref{有的时候}{you3de5shi2hou4}
  \seealsoref{有时候}{you3shi2hou5}
\end{entry}

\begin{entry}{有时候}{you3shi2hou5}{6,7,10}[HSK 1]
  \definition{adv.}{às vezes | de vez em quando | de quando em quando}
  \seealsoref{有的时候}{you3de5shi2hou4}
  \seealsoref{有时}{you3shi2}
\end{entry}

\begin{entry}{有限公司}{you3xian4gong1si1}{6,8,4,5}
  \definition{s.}{companhia limitada | corporação}
\end{entry}

\begin{entry}{有些}{you3 xie1}{6,8}[HSK 1]
  \definition{adv.}{um pouco | ao invés}
  \definition{pron.}{parte | algum}
  \definition{v.}{usado para indicar que há alguns, mas não muitos}
\end{entry}

\begin{entry}{有意思}{you3 yi4 si5}{6,13,9}[HSK 2]
  \definition{adj.}{interessante | agradável | significativo | divertido}
\end{entry}

\begin{entry}{有用}{you3yong4}{6,5}[HSK 1]
  \definition{adj.}{útil}
\end{entry}

\begin{entry}{又}{you4}{2}[Radical 又][Kangxi 29][HSK 2]
  \definition{adv.}{mais uma vez | (usado para dar ênfase) de qualquer maneira | e ainda | e também}
\end{entry}

\begin{entry}{又称}{you4cheng1}{2,10}
  \definition{s.}{também conhecido como}
\end{entry}

\begin{entry}{又及}{you4ji2}{2,3}
  \definition{s.}{P.S., \emph{postscript}}
\end{entry}

\begin{entry}{又名}{you4ming2}{2,6}
  \definition{s.}{também conhecido como | nome alternativo}
\end{entry}

\begin{entry}{又一次}{you4yi2ci4}{2,1,6}
  \definition{adv.}{outra vez | mais uma vez | de novo}
\end{entry}

\begin{entry}{右}{you4}{5}[Radical 口][HSK 1]
  \definition{s.}{(política) a Direita}
  \definition{s.}{direita}
\end{entry}

\begin{entry}{右边}{you4bian5}{5,5}[HSK 1]
  \definition{adv.}{à direita | ao lado direito}
\end{entry}

\begin{entry}{右侧}{you4ce4}{5,8}
  \definition{s.}{lateral direita | lado direito}
\end{entry}

\begin{entry}{右面}{you4mian4}{5,9}
  \definition{s.}{lado direito}
\end{entry}

\begin{entry}{右倾}{you4qing1}{5,10}
  \definition{adj.}{conservador | reacionário}
\end{entry}

\begin{entry}{右手}{you4shou3}{5,4}
  \definition{s.}{mão direita | lado direito}
\end{entry}

\begin{entry}{右袒}{you4tan3}{5,10}
  \definition{v.}{ser tendencioso | ser parcial | favorecer um lado | tomar partido}
\end{entry}

\begin{entry}{右转}{you4zhuan3}{5,8}
  \definition{v.}{virar à direita}
\end{entry}

\begin{entry}{幼儿园}{you4'er2yuan2}{5,2,7}
  \definition{s.}{jardim de infância | berçário}
\end{entry}

\begin{entry}{诱人}{you4ren2}{9,2}
  \definition{adj.}{atraente | cativante}
\end{entry}

\begin{entry}{淤泥}{yu1ni2}{11,8}
  \definition{s.}{lodo}
\end{entry}

\begin{entry}{于是}{yu2shi4}{3,9}
  \definition{conj.}{então | portanto | é por isso}
\end{entry}

\begin{entry}{鱼}{yu2}{8}[Radical 魚][HSK 2]
  \definition*{s.}{sobrenome Yu}
  \definition[条,尾]{s.}{peixe}
\end{entry}

\begin{entry}{鱼船}{yu2chuan2}{8,11}
  \definition{s.}{barco de pesca}
  \seealsoref{渔船}{yu2chuan2}
\end{entry}

\begin{entry}{鱼具}{yu2ju4}{8,8}
  \variantof{渔具}
\end{entry}

\begin{entry}{鱼片}{yu2pian4}{8,4}
  \definition{s.}{fatia de peixe | filé de peixe}
\end{entry}

\begin{entry}{鱼网}{yu2wang3}{8,6}
  \variantof{渔网}
\end{entry}

\begin{entry}{鱼香}{yu2xiang1}{8,9}
  \definition{s.}{um tempero da culinária chinesa que normalmente contém alho, cebolinha, gengibre, açúcar, sal, pimenta, etc. (Embora 鱼香 signifique literalmente ``fragrância de peixe'', não contém frutos do mar.)}
\end{entry}

\begin{entry}{鱼香肉丝}{yu2xiang1rou4si1}{8,9,6,5}
  \definition{s.}{tiras de carne de porco salteadas com molho picante (prato)}
  \seealsoref{鱼香}{yu2xiang1}
\end{entry}

\begin{entry}{鱼汛}{yu2xun4}{8,6}
  \variantof{渔汛}
\end{entry}

\begin{entry}{语}{yu2}{9}[Radical 言]
  \seeref{语}{yu3}
  \seeref{语}{yu4}
\end{entry}

\begin{entry}{渔}{yu2}{11}[Radical 水]
  \definition[条]{s.}{pescador}
  \definition{v.}{pescar}
\end{entry}

\begin{entry}{渔场}{yu2chang3}{11,6}
  \definition{s.}{área de pesca}
\end{entry}

\begin{entry}{渔船}{yu2chuan2}{11,11}
  \definition[条]{s.}{barco de pesca}
  \seealsoref{鱼船}{yu2chuan2}
\end{entry}

\begin{entry}{渔船队}{yu2chuan2dui4}{11,11,4}
  \definition{s.}{frota pesqueira}
\end{entry}

\begin{entry}{渔夫}{yu2fu1}{11,4}
  \definition{s.}{pescador}
\end{entry}

\begin{entry}{渔具}{yu2ju4}{11,8}
  \definition{s.}{equipamento de pesca}
\end{entry}

\begin{entry}{渔捞}{yu2lao1}{11,10}
  \definition{s.}{pesca (como atividade comercial)}
\end{entry}

\begin{entry}{渔猎}{yu2lie4}{11,11}
  \definition{s.}{pesca e caça}
  \definition{v.}{saquear | pilhar}
\end{entry}

\begin{entry}{渔笼}{yu2long2}{11,11}
  \definition{s.}{gaiola de pesca | armadilha de pesca}
\end{entry}

\begin{entry}{渔轮}{yu2lun2}{11,8}
  \definition{s.}{navio de pesca}
\end{entry}

\begin{entry}{渔民}{yu2min2}{11,5}
  \definition{s.}{pescadores | povo pescador}
\end{entry}

\begin{entry}{渔网}{yu2wang3}{11,6}
  \definition{s.}{rede de pesca}
\end{entry}

\begin{entry}{渔汛}{yu2xun4}{11,6}
  \definition{s.}{temporada de pesca}
\end{entry}

\begin{entry}{愉快}{yu2kuai4}{12,7}
  \definition{adj.}{alegre | delicioso | prazeroso | agradável | feliz | encantado}
  \definition{adv.}{alegremente | agradavelmente}
\end{entry}

\begin{entry}{瑜伽}{yu2jia1}{13,7}
  \definition*{s.}{Ioga}
\end{entry}

\begin{entry}{瑜珈}{yu2jia1}{13,9}
  \variantof{瑜伽}
\end{entry}

\begin{entry}{与}{yu3}{3}[Radical 一]
  \definition{conj.}{e, com}
  \seeref{与}{yu4}
\end{entry}

\begin{entry}{与其}{yu3qi2}{3,8}
  \definition{conj.}{mais do que}
\end{entry}

\begin{entry}{与其……不如……}{yu3qi2 bu4ru2}{3,8,4,6}
  \definition{conj.}{ao invés de\dots melhor que\dots}
\end{entry}

\begin{entry}{与其……宁可……}{yu3qi2 ning4ke3}{3,8,5,5}
  \definition{conj.}{ao invés de\dots melhor que\dots}
\end{entry}

\begin{entry}{宇航员}{yu3hang2yuan2}{6,10,7}
  \definition{s.}{astronauta}
\end{entry}

\begin{entry}{宇宙}{yu3zhou4}{6,8}
  \definition{s.}{universo | cosmos}
\end{entry}

\begin{entry}{羽冠}{yu3guan1}{6,9}
  \definition{s.}{crista emplumada (de pássaro)}
\end{entry}

\begin{entry}{羽林}{yu3lin2}{6,8}
  \definition{s.}{escolta armada}
\end{entry}

\begin{entry}{羽流}{yu3liu2}{6,10}
  \definition{s.}{pluma}
\end{entry}

\begin{entry}{羽毛}{yu3mao2}{6,4}
  \definition{s.}{pena | plumagem | pluma}
\end{entry}

\begin{entry}{羽毛笔}{yu3mao2bi3}{6,4,10}
  \definition{s.}{caneta de pena}
\end{entry}

\begin{entry}{羽毛球}{yu3mao2qiu2}{6,4,11}
  \definition[个]{s.}{\emph{badminton}}
\end{entry}

\begin{entry}{雨}{yu3}{8}[Radical 雨][Kangxi 173][HSK 1]
  \definition[阵,场]{s.}{chuva}
  \seeref{雨}{yu4}
\end{entry}

\begin{entry}{雨伞}{yu3san3}{8,6}
  \definition[把]{s.}{guarda-chuva}
\end{entry}

\begin{entry}{雨蚀}{yu3shi2}{8,9}
  \definition{s.}{erosão da chuva}
\end{entry}

\begin{entry}{雨靴}{yu3xue1}{8,13}
  \definition[双]{s.}{botas de chuva}
\end{entry}

\begin{entry}{雨衣}{yu3yi1}{8,6}
  \definition[件]{s.}{impermeável}
\end{entry}

\begin{entry}{语}{yu3}{9}[Radical 言]
  \definition{s.}{dialeto | linguagem | discurso}
  \seeref{语}{yu2}
  \seeref{语}{yu4}
\end{entry}

\begin{entry}{语调}{yu3diao4}{9,10}
  \definition[个]{s.}{entonação}
\end{entry}

\begin{entry}{语法}{yu3fa3}{9,8}
  \definition{s.}{gramática}
\end{entry}

\begin{entry}{语法术语}{yu3fa3shu4yu3}{9,8,5,9}
  \definition{s.}{termo gramatical}
\end{entry}

\begin{entry}{语气}{yu3qi4}{9,4}
  \definition[个]{s.}{maneira de falar | tom}
\end{entry}

\begin{entry}{语言}{yu3yan2}{9,7}[HSK 2]
  \definition[门,种]{s.}{linguagem | língua}
\end{entry}

\begin{entry}{语言实验室}{yu3yan2shi2yan4shi4}{9,7,8,10,9}
  \definition{s.}{laboratório de línguas}
\end{entry}

\begin{entry}{与}{yu4}{3}[Radical 一]
  \definition{v.}{fazer parte de}
  \seeref{与}{yu3}
\end{entry}

\begin{entry}{玉}{yu4}{5}[Radical 玉][Kangxi 96]
  \definition[块]{s.}{jade}
\end{entry}

\begin{entry}{玉米}{yu4mi3}{5,6}
  \definition[粒]{s.}{milho}
\end{entry}

\begin{entry}{玉米饼}{yu4mi3bing3}{5,6,9}
  \definition{s.}{tortilha mexicana | bolo de milho}
\end{entry}

\begin{entry}{玉米粉}{yu4mi3fen3}{5,6,10}
  \definition{s.}{amido de milho | farinha de milho}
\end{entry}

\begin{entry}{玉米糕}{yu4mi3gao1}{5,6,16}
  \definition{s.}{bolo de milho | polenta}
\end{entry}

\begin{entry}{玉米花}{yu4mi3hua1}{5,6,7}
  \definition{s.}{pipoca}
\end{entry}

\begin{entry}{玉米面}{yu4mi3mian4}{5,6,9}
  \definition{s.}{fubá | farinha de milho}
\end{entry}

\begin{entry}{玉米片}{yu4mi3pian4}{5,6,4}
  \definition{s.}{flocos de milho | chips de tortilha}
\end{entry}

\begin{entry}{玉米糁}{yu4mi3san3}{5,6,14}
  \definition{s.}{grãos de milho}
\end{entry}

\begin{entry}{玉米笋}{yu4mi3sun3}{5,6,10}
  \definition{s.}{broto de milho}
\end{entry}

\begin{entry}{芋头}{yu4tou5}{6,5}
  \definition{s.}{taro, similar ao inhame e batata doce}
\end{entry}

\begin{entry}{芋头色}{yu4tou5se4}{6,5,6}
  \definition{s.}{lilás (cor)}
\end{entry}

\begin{entry}{郁郁葱葱}{yu4yu4cong1cong1}{8,8,12,12}
  \definition{expr.}{verdejante e exuberante}
\end{entry}

\begin{entry}{雨}{yu4}{8}[Radical 雨]
  \definition{v.}{cair (chuva, neve, etc.) | precipitar | chover | molhar}
  \seeref{雨}{yu3}
\end{entry}

\begin{entry}{语}{yu4}{9}[Radical 言]
  \definition{v.}{dizer a}
  \seeref{语}{yu2}
  \seeref{语}{yu3}
\end{entry}

\begin{entry}{预}{yu4}{10}[Radical 頁]
  \definition{adv.}{antecipadamente}
  \definition{v.}{avançar | preparar}
\end{entry}

\begin{entry}{预报}{yu4bao4}{10,7}
  \definition{s.}{previsão (meteorológica) | boletim meteorológico}
  \definition{v.}{prever (o tempo)}
\end{entry}

\begin{entry}{预定}{yu4ding4}{10,8}
  \definition{v.}{agendar com antecedência}
\end{entry}

\begin{entry}{预付}{yu4fu4}{10,5}
  \definition{s.}{pré-pago}
  \definition{v.}{pagar antecipadamente}
\end{entry}

\begin{entry}{预感}{yu4gan3}{10,13}
  \definition{s.}{premonição}
  \definition{v.}{ter uma premonição}
\end{entry}

\begin{entry}{预购}{yu4gou4}{10,8}
  \definition{s.}{compra antecipada}
  \definition{v.}{comprar antecipadamente}
\end{entry}

\begin{entry}{预见}{yu4jian4}{10,4}
  \definition{s.}{previsão; intuição; vislumbre}
  \definition{v.}{prever}
\end{entry}

\begin{entry}{预警}{yu4jing3}{10,19}
  \definition{s.}{aviso | aviso antecipado}
\end{entry}

\begin{entry}{预览}{yu4lan3}{10,9}
  \definition{s.}{visualização}
  \definition{v.}{visualizar}
\end{entry}

\begin{entry}{预留}{yu4liu2}{10,10}
  \definition{v.}{separar | reservar}
\end{entry}

\begin{entry}{预谋}{yu4mou2}{10,11}
  \definition{adj.}{premeditado}
  \definition{v.}{planejar algo com antecedência (especialmente um crime)}
\end{entry}

\begin{entry}{预判}{yu4pan4}{10,7}
  \definition{v.}{prever | antecipar}
\end{entry}

\begin{entry}{预配}{yu4pei4}{10,10}
  \definition{s.}{pré-alocado | pré-cabeado}
  \definition{v.}{pré-alocar | pré-cabear}
\end{entry}

\begin{entry}{预提}{yu4ti2}{10,12}
  \definition{s.}{retenção}
  \definition{v.}{reter (imposto)}
\end{entry}

\begin{entry}{预约}{yu4yue1}{10,6}
  \definition{s.}{reserva}
  \definition{v.}{agendar | marcar compromisso}
\end{entry}

\begin{entry}{预祝}{yu4zhu4}{10,9}
  \definition{v.}{parabenizar de antemão | oferecer os melhores votos para}
\end{entry}

\begin{entry}{欲}{yu4}{11}[Radical 欠]
  \definition{adj.}{desejo | apetite | paixão | luxúria | ganância}
  \definition{v.}{desejar}
\end{entry}

\begin{entry}{喻}{yu4}{12}[Radical 口]
  \definition{s.}{analogia | símile | metáfora | alegoria}
  \definition{v.}{descrever algo como}
\end{entry}

\begin{entry}{寓意}{yu4yi4}{12,13}
  \definition{s.}{moral (de uma história),  lição a ser aprendida, implicação, mensagem, significado metafórico}
\end{entry}

\begin{entry}{愈}{yu4}{13}[Radical 心]
  \definition{adv.}{mais e mais | ainda mais}
  \definition{v.}{recuperar | curar}
\end{entry}

\begin{entry}{豫}{yu4}{15}[Radical 豕]
  \definition{adj.}{feliz despreocupado | à vontade}
  \seeref{预}{yu4}
\end{entry}

\begin{entry}{元}{yuan2}{4}[Radical 儿][HSK 1]
  \definition*{s.}{sobrenome Yuan | Dinastia Yuan (1279-1368)}
  \definition{clas.}{unidade monetária da China}
\end{entry}

\begin{entry}{元旦}{yuan2dan4}{4,5}
  \definition*{s.}{Dia de Ano Novo (1 de janeiro)}
\end{entry}

\begin{entry}{元来}{yuan2lai2}{4,7}
  \variantof{原来}
\end{entry}

\begin{entry}{元气}{yuan2qi4}{4,4}
  \definition{s.}{força | vigor | vitalidade | energial vital}
\end{entry}

\begin{entry}{元宵}{yuan2xiao1}{4,10}
  \definition*{s.}{Festival das Lanternas}
  \seealsoref{元宵节}{yuan2xiao1jie2}
  \seealsoref{元夜}{yuan2ye4}
\end{entry}

\begin{entry}{元宵节}{yuan2xiao1jie2}{4,10,5}
  \definition*{s.}{Festival das Lanternas (15º~dia do primeiro mês lunar)}
  \seealsoref{元宵}{yuan2xiao1}
  \seealsoref{元夜}{yuan2ye4}
\end{entry}

\begin{entry}{元夜}{yuan2ye4}{4,8}
  \definition*{s.}{Festival das Lanternas}
  \seealsoref{元宵}{yuan2xiao1}
  \seealsoref{元宵节}{yuan2xiao1jie2}
\end{entry}

\begin{entry}{原来}{yuan2lai2}{10,7}[HSK 2]
  \definition{adv.}{originalmente | como se vê | na verdade}
  \definition{v.}{vir a ser}
\end{entry}

\begin{entry}{原理}{yuan2li3}{10,11}
  \definition{s.}{princípio | teoria}
\end{entry}

\begin{entry}{原木}{yuan2mu4}{10,4}
  \definition{s.}{registro | \emph{logs}}
\end{entry}

\begin{entry}{原色}{yuan2se4}{10,6}
  \definition{s.}{cor primária}
\end{entry}

\begin{entry}{原因}{yuan2yin1}{10,6}[HSK 2]
  \definition[个]{s.}{causa | razão | motivo}
\end{entry}

\begin{entry}{援助}{yuan2zhu4}{12,7}
  \definition{s.}{assistência}
  \definition{v.}{ajudar | apoiar | assistir}
\end{entry}

\begin{entry}{缘}{yuan2}{12}[Radical 糸]
  \definition{s.}{causa | razão | karma | destino | predestinação}
\end{entry}

\begin{entry}{缘分}{yuan2fen4}{12,4}
  \definition{s.}{destino ou acaso que une as pessoas | afinidade ou relacionamento predestinado | destino (Budismo)}
\end{entry}

\begin{entry}{远}{yuan3}{7}[Radical 辵][HSK 1]
  \definition{adj.}{longe | distante | remoto}
  \seeref{远}{yuan4}
\end{entry}

\begin{entry}{远方}{yuan3fang1}{7,4}
  \definition{s.}{longe | um local distante}
\end{entry}

\begin{entry}{远天}{yuan3tian1}{7,4}
  \definition{s.}{paraíso | o céu distante}
\end{entry}

\begin{entry}{远远}{yuan3yuan3}{7,7}
  \definition{adv.}{de longe}
\end{entry}

\begin{entry}{远征}{yuan3zheng1}{7,8}
  \definition{s.}{uma expedição militar | marcha para regiões remotas}
\end{entry}

\begin{entry}{远}{yuan4}{7}[Radical 辵]
  \definition{v.}{distanciar-se de (clássico)}
  \seeref{远}{yuan3}
\end{entry}

\begin{entry}{院}{yuan4}{9}[Radical 阜][HSK 2]
  \definition[个]{s.}{pátio | instituição}
\end{entry}

\begin{entry}{院长}{yuan4zhang3}{9,4}[HSK 2]
  \definition[个]{s.}{presidente de um conselho | reitor | chefe de departamento | primeiro-ministro da República da China | presidente de uma universidade}
\end{entry}

\begin{entry}{院子}{yuan4zi5}{9,3}[HSK 2]
  \definition[个]{s.}{pátio | jardim | quintal}
\end{entry}

\begin{entry}{愿}{yuan4}{14}[Radical 心]
  \definition{adj.}{honesto e prudente}
\end{entry}

\begin{entry}{愿意}{yuan4yi4}{14,13}[HSK 2]
  \definition{s.}{desejo | esperança}
  \definition{v.}{estar disposto | estar pronto}
\end{entry}

\begin{entry}{约会}{yue1hui4}{6,6}
  \definition[次,个]{s.}{compromisso | encontro marcado}
\end{entry}

\begin{entry}{月}{yue4}{4}[Radical 月][Kangxi 74][HSK 1]
  \definition[个,轮]{s.}{mês}
\end{entry}

\begin{entry}{月饼}{yue4bing3}{4,9}
  \definition[张]{s.}{bolo da lua}
\end{entry}

\begin{entry}{月份}{yue4 fen4}{4,6}[HSK 2]
  \definition{s.}{mês}
\end{entry}

\begin{entry}{月径}{yue4jing4}{4,8}
  \definition{s.}{diâmetro da lua | diâmetro da órbita da lua | caminho iluminado pela lua}
\end{entry}

\begin{entry}{月亮}{yue4liang5}{4,9}[HSK 2]
  \definition{s.}{lua}
\end{entry}

\begin{entry}{月球}{yue4qiu2}{4,11}
  \definition{s.}{a lua}
\end{entry}

\begin{entry}{月壤}{yue4rang3}{4,20}
  \definition{s.}{solo lunar}
\end{entry}

\begin{entry}{月相}{yue4xiang4}{4,9}
  \definition{s.}{fases da lua, a saber: lua nova 朔, lua crescente 上弦, lua cheia 望 e lua minguante 下弦}
\end{entry}

\begin{entry}{月月}{yue4yue4}{4,4}
  \definition{adv.}{todo mês}
\end{entry}

\begin{entry}{阅兵式}{yue4bing1shi4}{10,7,6}
  \definition{s.}{parada militar}
\end{entry}

\begin{entry}{阅读}{yue4du2}{10,10}
  \definition{s.}{leitura}
  \definition{v.}{ler}
\end{entry}

\begin{entry}{阅读广度}{yue4du2guang3du4}{10,10,3,9}
  \definition{s.}{intervalo de leitura}
\end{entry}

\begin{entry}{阅读理解}{yue4du2li3jie3}{10,10,11,13}
  \definition{s.}{compreensão de leitura}
\end{entry}

\begin{entry}{阅读器}{yue4du2qi4}{10,10,16}
  \definition{s.}{leitor (\emph{software})}
\end{entry}

\begin{entry}{阅读时间}{yue4du2shi2jian1}{10,10,7,7}
  \definition{s.}{tempo de leitura}
\end{entry}

\begin{entry}{阅读障碍}{yue4du2zhang4ai4}{10,10,13,13}
  \definition{s.}{dislexia}
\end{entry}

\begin{entry}{阅读装置}{yue4du2zhuang1zhi4}{10,10,12,13}
  \definition{s.}{dispositivo de leitura (por exemplo, para códigos de barras, etiquetas RFID, etc.)}
\end{entry}

\begin{entry}{阅览室}{yue4lan3shi4}{10,9,9}
  \definition[间]{s.}{sala de leitura}
\end{entry}

\begin{entry}{粤语}{yue4yu3}{12,9}
  \definition{s.}{cantonês | língua cantonesa}
\end{entry}

\begin{entry}{越}{yue4}{12}[Radical 走][HSK 2]
  \definition{adv.}{quanto mais\dots mais}
  \definition{v.}{subir | exceder | superar}
\end{entry}

\begin{entry}{越境}{yue4jing4}{12,14}
  \definition{v.}{cruzar uma fronteira (geralmente ilegalmente) | entrar ou sair furtivamente de um país}
\end{entry}

\begin{entry}{越来越……}{yue4lai2yue4}{12,7,12}[HSK 2]
  \definition{adv.}{cada vez mais\dots}
\end{entry}

\begin{entry}{越……越……}{yue4 yue4}{12,12}
  \definition{expr.}{quanto mais\dots tanto mais\dots}
\end{entry}

\begin{entry}{越障}{yue4zhang4}{12,13}
  \definition{s.}{curso com obstáculos para treinamento de tropas}
  \definition{v.}{superar obstáculos}
\end{entry}

\begin{entry}{云}{yun2}{4}[Radical 二][HSK 2]
  \definition*{s.}{sobrenome Yun}
  \definition[朵]{s.}{nuvem}
\end{entry}

\begin{entry}{云端}{yun2duan1}{4,14}
  \definition{s.}{alto nas nuvens | (computação) a nuvem}
\end{entry}

\begin{entry}{云南}{yun2nan2}{4,9}
  \definition*{s.}{Yunnan}
\end{entry}

\begin{entry}{云云}{yun2yun2}{4,4}
  \definition{adv.}{e assim por diante | assim e assim}
\end{entry}

\begin{entry}{运动}{yun4dong4}{7,6}[HSK 2]
  \definition[场]{s.}{esporte | desporto}
  \definition{v.}{exercitar | mover-se}
\end{entry}

\begin{entry}{运动病}{yun4dong4bing4}{7,6,10}
  \definition{s.}{enjôo (movimento, carro, etc.)}
\end{entry}

\begin{entry}{运动场}{yun4dong4chang3}{7,6,6}
  \definition{s.}{campo desportivo | campo de jogos}
\end{entry}

\begin{entry}{运动服}{yun4dong4fu2}{7,6,8}
  \definition{s.}{roupa para prática de esporte}
\end{entry}

\begin{entry}{运动会}{yun4dong4hui4}{7,6,6}
  \definition[个]{s.}{competição esportiva}
\end{entry}

\begin{entry}{运动家}{yun4dong4jia1}{7,6,10}
  \definition{s.}{ativista | atleta | esportista}
\end{entry}

\begin{entry}{运动衫}{yun4dong4shan1}{7,6,8}
  \definition[件]{s.}{moletom | camisa esportiva}
\end{entry}

\begin{entry}{运动鞋}{yun4dong4xie2}{7,6,15}
  \definition{s.}{tênis | sapatos esportivos}
\end{entry}

\begin{entry}{运动学}{yun4dong4xue2}{7,6,8}
  \definition{s.}{cinemática}
\end{entry}

\begin{entry}{运动员}{yun4dong4yuan2}{7,6,7}
  \definition[名,个]{s.}{jogador | atleta}
\end{entry}

\begin{entry}{运河}{yun4he2}{7,8}
  \definition{s.}{canal (em um rio)}
\end{entry}

\begin{entry}{运气}{yun4qi5}{7,4}
  \definition{s.}{sorte (boa ou má)}
\end{entry}

\begin{entry}{运行}{yun4xing2}{7,6}
  \definition{v.}{(corpos celestes, etc.) mover-se ao longo do curso | (figurativo) funcionar, estar em operação | (serviço de trem, etc.) operar | (computador) executar um programa}
\end{entry}

%%%%% EOF %%%%%


%%%
%%% Z
%%%

\section*{Z}\addcontentsline{toc}{section}{Z}

\begin{entry}{杂技}{za2ji4}{6,7}
  \definition[场]{s.}{acrobacia}
\end{entry}

\begin{entry}{杂志}{za2zhi4}{6,7}
  \definition[本,份,期]{s.}{revista}
\end{entry}

\begin{entry}{杂志社}{za2zhi4she4}{6,7,7}
  \definition{s.}{editora de revista}
\end{entry}

\begin{entry}{砸}{za2}{10}[Radical 石]
  \definition{v.}{esmagar | bater | falhar | estragar}
\end{entry}

\begin{entry}{栽}{zai1}{10}[Radical 木]
  \definition{v.}{cultivar | plantar}
\end{entry}

\begin{entry}{栽倒}{zai1dao3}{10,10}
  \definition{v.}{cair | sofrer uma queda}
\end{entry}

\begin{entry}{栽培}{zai1pei2}{10,11}
  \definition{v.}{cultivar | educar | patrocinar | treinar}
\end{entry}

\begin{entry}{栽培种}{zai1pei2 zhong3}{10,11,9}
  \definition{s.}{espécies cultivadas}
\end{entry}

\begin{entry}{栽赃}{zai1zang1}{10,10}
  \definition{v.}{enquadrar alguém (plantar provas nele)}
\end{entry}

\begin{entry}{栽植}{zai1zhi2}{10,12}
  \definition{v.}{plantar | transplantar}
\end{entry}

\begin{entry}{栽种}{zai1zhong4}{10,9}
  \definition{v.}{plantar}
\end{entry}

\begin{entry}{再}{zai4}{6}[Radical 冂][HSK 1]
  \definition{adv.}{de novo | outra vez | uma segunda vez | não importa como\dots (seguido por um adjetivo ou verbo, e então (normalmente) 也 ou 都 para dar ênfase)}
\end{entry}

\begin{entry}{再不}{zai4bu4}{6,4}
  \definition{adv.}{nunca mais}
\end{entry}

\begin{entry}{再读}{zai4du2}{6,10}
  \definition{v.}{ler novamente | rever (uma lição, etc.)}
\end{entry}

\begin{entry}{再度}{zai4du4}{6,9}
  \definition{adv.}{outra vez | mais uma vez}
\end{entry}

\begin{entry}{再发}{zai4fa1}{6,5}
  \definition{v.}{reenviar}
\end{entry}

\begin{entry}{再见}{zai4jian4}{6,4}[HSK 1]
  \definition{v.}{adeus | até à vista | até à próxima | até logo}
\end{entry}

\begin{entry}{再临}{zai4lin2}{6,9}
  \definition{v.}{vir de novo}
\end{entry}

\begin{entry}{再三}{zai4san1}{6,3}
  \definition{adv.}{de novo e de novo | repetidamente}
\end{entry}

\begin{entry}{再审}{zai4shen3}{6,8}
  \definition{s.}{novo julgamento | revisão}
  \definition{v.}{ouvir um caso novamente}
\end{entry}

\begin{entry}{再生}{zai4sheng1}{6,5}
  \definition{s.}{reciclagem | regeneração}
  \definition{v.}{reciclar | renascer | regenerar}
\end{entry}

\begin{entry}{再说}{zai4shuo1}{6,9}
  \definition{conj.}{além do mais | além disso | o que mais}
  \definition{v.}{adiar uma discussão para mais tarde | dizer novamente}
\end{entry}

\begin{entry}{再育}{zai4yu4}{6,8}
  \definition{v.}{aumentar | multiplicar | proliferar}
\end{entry}

\begin{entry}{再者}{zai4zhe3}{6,8}
  \definition{conj.}{além do mais | além disso}
\end{entry}

\begin{entry}{在}{zai4}{6}[Radical 土][HSK 1]
  \definition{adv.}{para designar ações que estão passando | durante}
  \definition{prep.}{em}
  \definition{v.}{estar | ficar}
\end{entry}

\begin{entry}{在此}{zai4ci3}{6,6}
  \definition{adv.}{aqui}
\end{entry}

\begin{entry}{在地}{zai4di4}{6,6}
  \definition{s.}{local}
\end{entry}

\begin{entry}{在行}{zai4hang2}{6,6}
  \definition{v.}{ser adepto de algo | ser um especialista em um comércio ou profissão}
\end{entry}

\begin{entry}{在乎}{zai4hu5}{6,5}
  \definition{v.}{preocupar-se com}
\end{entry}

\begin{entry}{在家}{zai4jia1}{6,10}[HSK 1]
  \definition{v.}{estar em casa | permanecer um leigo}
\end{entry}

\begin{entry}{在教}{zai4jiao4}{6,11}
  \definition{v.}{ser um crente (em uma religião)}
\end{entry}

\begin{entry}{在下}{zai4xia4}{6,3}
  \definition{pron.}{eu mesmo (humildemente)}
\end{entry}

\begin{entry}{在线}{zai4xian4}{6,8}
  \definition{s.}{\emph{online}}
\end{entry}

\begin{entry}{在意}{zai4yi4}{6,13}
  \definition{v.+compl.}{preocupar-se | importar-se | levar a sério}
\end{entry}

\begin{entry}{在于}{zai4yu2}{6,3}
  \definition{v.}{descansar | deitar | ser devido a (um determinado atributo)/(de um assunto) a ser determinado | estar à altura de alguém}
\end{entry}

\begin{entry}{咱家}{zan2jia1}{9,10}
  \definition{pron.}{eu (frequentemente usado na literatura vernácula antiga) | me | mim | comigo}
\end{entry}

\begin{entry}{咱俩}{zan2lia3}{9,9}
  \definition{pron.}{nós dois}
\end{entry}

\begin{entry}{咱们}{zan2men5}{9,5}[HSK 2]
  \definition{pron.}{nós (incluindo o orador e a(s) pessoa(s) com quem se fala)}
\end{entry}

\begin{entry}{咱}{zan4}{9}[HSK 2]
  \definition{pron.}{eu}
\end{entry}

\begin{entry}{赞}{zan4}{16}[Radical 貝]
  \definition{v.}{patrocinar | apoiar | elogiar | (gíria na \emph{Internet}) para curtir (uma postagem \emph{on-line})}
\end{entry}

\begin{entry}{赞扬}{zan4yang2}{16,6}
  \definition{v.}{elogiar | aprovar | demonstrar aprovação}
\end{entry}

\begin{entry}{赞助}{zan4zhu4}{16,7}
  \definition{s.}{patrocinador}
  \definition{v.}{apoiar | auxiliar | patrocinar}
\end{entry}

\begin{entry}{脏}{zang1}{10}[Radical 肉][HSK 2]
  \definition{adj.}{sujo | imundo}
  \seeref{脏}{zang4}
\end{entry}

\begin{entry}{脏辫}{zang1bian4}{10,17}
  \definition{s.}{\emph{dreadlocks}}
\end{entry}

\begin{entry}{脏病}{zang1bing4}{10,10}
  \definition{s.}{doença venérea}
\end{entry}

\begin{entry}{脏煤}{zang1mei2}{10,13}
  \definition{s.}{carvão sujo | sujeira (de uma mina de carvão)}
\end{entry}

\begin{entry}{脏土}{zang1tu3}{10,3}
  \definition{s.}{solo sujo | lama | lixo}
\end{entry}

\begin{entry}{脏脏}{zang1zang1}{10,10}
  \definition{adj.}{sujo}
\end{entry}

\begin{entry}{脏字}{zang1zi4}{10,6}
  \definition{s.}{obscenidade}
\end{entry}

\begin{entry}{脏}{zang4}{10}[Radical 肉]
  \definition{s.}{órgão (anatomia) | víscera}
  \seeref{脏}{zang1}
\end{entry}

\begin{entry}{脏器}{zang4qi4}{10,16}
  \definition{s.}{órgãos internos}
\end{entry}

\begin{entry}{葬}{zang4}{12}[Radical 艸]
  \definition{v.}{enterrar (os mortos) | sepultar}
\end{entry}

\begin{entry}{遭到}{zao1dao4}{14,8}
  \definition{v.}{sofrer | encontrar-se com (algo infeliz)}
\end{entry}

\begin{entry}{遭受}{zao1shou4}{14,8}
  \definition{v.}{sofrer | suportar (perda, infornúnio)}
\end{entry}

\begin{entry}{遭遇}{zao1yu4}{14,12}
  \definition{s.}{experiência (amarga)}
  \definition{v.}{encontrar-se com (algo infeliz)}
\end{entry}

\begin{entry}{糟糕}{zao1gao1}{17,16}
  \definition{adj.}{muito mau | péssimo}
\end{entry}

\begin{entry}{早}{zao3}{6}[Radical 日][HSK 1]
  \definition{adj.}{prematuramente}
  \definition{adv.}{cedo | antecipadamente | breve}
  \definition{s.}{manhã}
\end{entry}

\begin{entry}{早安}{zao3'an1}{6,6}
  \definition{interj.}{Bom dia!}
\end{entry}

\begin{entry}{早餐}{zao3 can1}{6,16}[HSK 2]
  \definition[份,顿,次]{s.}{café da manhã}
\end{entry}

\begin{entry}{早车}{zao3che1}{6,4}
  \definition{s.}{trem matutino | ônibus matutino}
\end{entry}

\begin{entry}{早晨}{zao3 chen2}{6,11}
  \definition{adv.}{manhã cedo | manhãzinha}
  \definition[个]{s.}{manhã}
\end{entry}

\begin{entry}{早饭}{zao3fan4}{6,7}[HSK 1]
  \definition[份,顿,次,餐]{s.}{café da manhã}
\end{entry}

\begin{entry}{早就}{zao3 jiu4}{6,12}[HSK 2]
  \definition{adv.}{já em um momento anterior}
\end{entry}

\begin{entry}{早前}{zao3qian2}{6,9}
  \definition{adv.}{previamente}
\end{entry}

\begin{entry}{早上}{zao3shang5}{6,3}[HSK 1]
  \definition{adv.}{manhã cedo | manhãzinha}
  \definition[个]{s.}{manhã}
\end{entry}

\begin{entry}{早亡}{zao3wang2}{6,3}
  \definition[个]{s.}{morte prematura}
  \definition{v.}{morrer prematuramente}
\end{entry}

\begin{entry}{早早儿}{zao3zao3r5}{6,6,2}
  \definition{adv.}{o mais cedo possível | o mais breve possível}
\end{entry}

\begin{entry}{早知}{zao3zhi1}{6,8}
  \definition{v.}{prever | se alguém soubesse antes, \dots}
\end{entry}

\begin{entry}{灶台}{zao4tai2}{7,5}
  \definition{s.}{fogão}
\end{entry}

\begin{entry}{造}{zao4}{10}[Radical 辵]
  \definition{clas.}{para colheitas, cultivos}
  \definition{v.}{criar | construir | fabricar | inventar}
\end{entry}

\begin{entry}{艁}{zao4}{13}[Radical 舟]
  \variantof{造}
\end{entry}

\begin{entry}{责怪}{ze2guai4}{8,8}
  \definition{v.}{repreender | censurar}
\end{entry}

\begin{entry}{怎}{zen3}{9}[Radical 心]
  \definition{adv.}{como}
\end{entry}

\begin{entry}{怎么}{zen3me5}{9,3}[HSK 1]
  \definition{pron.}{como? | o que?}
\end{entry}

\begin{entry}{怎么办}{zen3 me5 ban4}{9,3,4}[HSK 2]
  \definition{adv.}{o que fazer?}
\end{entry}

\begin{entry}{怎么得了}{zen3me5de2liao3}{9,3,11,2}
  \definition{expr.}{Como isso pode ser? | Que bagunça horrível! | O que deve ser feito?}
\end{entry}

\begin{entry}{怎么搞的}{zen3me5gao3de5}{9,3,13,8}
  \definition{expr.}{Como isso aconteceu? | O que deu errado? | E aí? | O que está errado?}
\end{entry}

\begin{entry}{怎么回事}{zen3me5hui2shi4}{9,3,6,8}
  \definition{expr.}{O que aconteceu? | O que se passou?}
\end{entry}

\begin{entry}{怎么了}{zen3me5le5}{9,3,2}
  \definition{expr.}{O que aconteceu? | O que está acontecendo? | E aí?}
\end{entry}

\begin{entry}{怎么样}{zen3me5yang4}{9,3,10}[HSK 2]
  \definition{adv.}{como? | que tal?}
\end{entry}

\begin{entry}{怎样}{zen3 yang4}{9,10}[HSK 2]
  \definition{pron.}{como | o que | de uma certa maneira | de qualquer maneira | não importa o quão}
\end{entry}

\begin{entry}{增速}{zeng1su4}{15,10}
  \definition{s.}{(economia) taxa de crescimento}
  \definition{v.}{acelerar;}
\end{entry}

\begin{entry}{查}{zha1}{9}[Radical 木]
  \definition*{s.}{sobrenome Zha}
  \definition{s.}{espinheiro}
  \seeref{查}{cha2}
\end{entry}

\begin{entry}{闸门}{zha2men2}{8,3}
  \definition{s.}{eclusa | comporta}
\end{entry}

\begin{entry}{寨}{zhai4}{14}[Radical 宀]
  \definition{s.}{fortaleza | paliçada | acampamento | vila (paliçada)}
\end{entry}

\begin{entry}{占}{zhan1}{5}[Radical 卜]
  \definition*{s.}{sobrenome Zhan}
  \definition{v.}{praticar adivinhação | advinhar}
  \seeref{占}{zhan4}
\end{entry}

\begin{entry}{斩获}{zhan3huo4}{8,10}
  \definition{v.}{matar ou capturar (em batalha) | (figurativo) (esportes) marcar (um gol), ganhar (uma medalha) | (figurativo) colher recompensas, obter ganhos}
\end{entry}

\begin{entry}{展示}{zhan3shi4}{10,5}
  \definition{v.}{revelar | mostrar | exibir}
\end{entry}

\begin{entry}{盏}{zhan3}{10}[Radical 皿]
  \definition{clas.}{para lâmpadas}
  \definition{s.}{copo pequeno}
\end{entry}

\begin{entry}{占}{zhan4}{5}[Radical 卜][HSK 2]
  \definition{v.}{ocupar | apreender | tomar | constituir | manter | compor | dar conta de}
  \seeref{占}{zhan1}
\end{entry}

\begin{entry}{战}{zhan4}{9}[Radical 戈]
  \definition{s.}{luta | guerra | batalha}
  \definition{v.}{lutar}
\end{entry}

\begin{entry}{战士}{zhan4shi4}{9,3}
  \definition[个]{s.}{lutador | soldado | guerreiro}
\end{entry}

\begin{entry}{战争}{zhan4zheng1}{9,6}
  \definition[場,次]{s.}{guerra | conflito}
\end{entry}

\begin{entry}{站}{zhan4}{10}[Radical 立][HSK 1]
  \definition{s.}{estação | ponto | parada}
\end{entry}

\begin{entry}{站点}{zhan4dian3}{10,9}
  \definition{s.}{\emph{website}}
\end{entry}

\begin{entry}{站台}{zhan4tai2}{10,5}
  \definition{s.}{plataforma (em uma estação ferroviária)}
\end{entry}

\begin{entry}{站长}{zhan4zhang3}{10,4}
  \definition{s.}{pessoa responsável pela estação de trem | chefe da estação | \emph{webmaster} | gerente de centro de voluntariado}
\end{entry}

\begin{entry}{站住}{zhan4 zhu4}{10,7}[HSK 2]
  \definition{v.}{parar | deter | ficar firme em pé | manter os pés firmes | manter a própria posição | consolidar a própria posição | reter água | ser sustentável}
\end{entry}

\begin{entry}{站姿}{zhan4zi1}{10,9}
  \definition{s.}{postura}
\end{entry}

\begin{entry}{张}{zhang1}{7}[Radical 弓]
  \definition*{s.}{sobrenome Zhang}
  \definition{clas.}{para folha de papéis, mapas, etc. | para votos}
  \definition{s.}{folha de papel}
  \definition{v.}{abrir | espalhar}
\end{entry}

\begin{entry}{张狂}{zhang1kuang2}{7,7}
  \definition{adj.}{impetuoso | frenético | insolente}
\end{entry}

\begin{entry}{张三}{zhang1san1}{7,3}
  \definition*{s.}{Zhang San | Zé Ninguém | nome para uma pessoa não especificada, 1 de 3}
  \seealsoref{李四}{li3si4}
  \seealsoref{王五}{wang2wu3}
\end{entry}

\begin{entry}{章}{zhang1}{11}[Radical 音]
  \definition*{s.}{sobrenome Zhang}
  \definition{s.}{capítulo | seção | cláusula |  movimento (de sinfonia) | selo | crachá | regulamento}
\end{entry}

\begin{entry}{章鱼}{zhang1yu2}{11,8}
  \definition{s.}{polvo | octópode}
\end{entry}

\begin{entry}{长}{zhang3}{4}[Radical 長][HSK 2]
  \definition{s.}{chefe | ancião}
  \definition{v.}{crescer | desenvolver | aumentar | melhorar}
  \seeref{长}{chang2}
\end{entry}

\begin{entry}{长大}{zhang3 da4}{4,3}[HSK 2]
  \definition{v.}{crescer | ser criado}
\end{entry}

\begin{entry}{涨价}{zhang3jia4}{10,6}
  \definition{s.}{aumento de preços}
  \definition{v.+compl.}{avaliar (em valor) | dar preço | aumentar o preço}
\end{entry}

\begin{entry}{掌}{zhang3}{12}[Radical 手]
  \definition{s.}{palma da mão | sola do pé | pata | ferradura}
  \definition{v.}{dar um tapa | segurar na mão | empunhar}
\end{entry}

\begin{entry}{招}{zhao1}{8}[Radical 手]
  \definition{adj.}{contagioso}
  \definition{s.}{um movimento (xadrez) | uma manobra | dispositivo | truque}
  \definition{v.}{recrutar | provocar | acenar | incorrer | infectar | confessar}
\end{entry}

\begin{entry}{招手}{zhao1shou3}{8,4}
  \definition{v.+compl.}{acenar}
\end{entry}

\begin{entry}{招数}{zhao1shu4}{8,13}
  \definition{s.}{estratégia | movimento (no xadrez, no palco, nas artes marciais) | esquema | truque}
\end{entry}

\begin{entry}{着}{zhao1}{11}[Radical 目]
  \definition{interj.}{Tudo bem!}
  \definition{s.}{movimento (xadrez) | truque}
  \seeref{着}{zhao2}
  \seeref{着}{zhe5}
  \seeref{着}{zhuo2}
\end{entry}

\begin{entry}{着数}{zhao1shu4}{11,13}
  \definition{s.}{estratégia | movimento (no xadrez, no palco, nas artes marciais) | esquema | truque}
\end{entry}

\begin{entry}{朝}{zhao1}{12}[Radical 月]
  \definition{s.}{manhã cedo; manhã | dia}
  \seeref{朝}{chao2}
\end{entry}

\begin{entry}{着}{zhao2}{11}[Radical 目]
  \definition{v.}{ser afetado por | queimar | pegar fogo | entrar em contato com | sentir | tocar}
  \seeref{着}{zhao1}
  \seeref{着}{zhe5}
  \seeref{着}{zhuo2}
\end{entry}

\begin{entry}{着地}{zhao2di4}{11,6}
  \definition{v.}{pousar | tocar o chão}
\end{entry}

\begin{entry}{着花}{zhao2hua1}{11,7}
  \definition{v.}{florescer}
  \seeref{着花}{zhuo2hua1}
\end{entry}

\begin{entry}{着急}{zhao2ji2}{11,9}
  \definition{adj.}{inquieto | ansioso}
  \definition{s.}{preocupação | ansiedade}
  \definition{v.+compl.}{preocupar-se | sentir-se ansioso | sentir uma sensação de urgência}
\end{entry}

\begin{entry}{着凉}{zhao2liang2}{11,10}
  \definition{v.}{pegar um resfriado}
\end{entry}

\begin{entry}{找}{zhao3}{7}[Radical 手][HSK 1]
  \definition{v.}{andar à procura de | procurar | tentar procurar | dar troco | retornar algo}
\end{entry}

\begin{entry}{找遍}{zhao3bian4}{7,12}
  \definition{v.}{pentear | pesquisar em todos os lugares}
\end{entry}

\begin{entry}{找出}{zhao3 chu1}{7,5}[HSK 2]
  \definition{v.}{encontrar | procurar}
\end{entry}

\begin{entry}{找到}{zhao3dao4}{7,8}[HSK 1]
  \definition{v.}{encontrar}
\end{entry}

\begin{entry}{找回}{zhao3hui2}{7,6}
  \definition{v.}{recuperar algo}
\end{entry}

\begin{entry}{找见}{zhao3jian4}{7,4}
  \definition{v.}{encontrar (algo que está procurando)}
\end{entry}

\begin{entry}{找零}{zhao3ling2}{7,13}
  \definition{v.}{trocar dinheiro | dar troco}
\end{entry}

\begin{entry}{找钱}{zhao3qian2}{7,10}
  \definition{v.}{dar troco}
\end{entry}

\begin{entry}{找事}{zhao3shi4}{7,8}
  \definition{v.}{procurar emprego | começar uma briga}
\end{entry}

\begin{entry}{找寻}{zhao3xun2}{7,6}
  \definition{v.}{encontrar falhas | procurar | buscar}
\end{entry}

\begin{entry}{找着}{zhao3zhao2}{7,11}
  \definition{v.}{encontrar}
\end{entry}

\begin{entry}{找辙}{zhao3zhe2}{7,16}
  \definition{v.}{procurar um pretexto}
\end{entry}

\begin{entry}{兆}{zhao4}{6}[Radical 儿]
  \definition{num.}{trilhão}
\end{entry}

\begin{entry}{照}{zhao4}{13}[Radical 火]
  \definition{adv.}{de acordo com | como antes | como pedido | conforme}
  \definition{s.}{foto}
  \definition{v.}{iluminar | olhar (o reflexo de alguém) | refletir | brilhar | tirar uma foto}
\end{entry}

\begin{entry}{照顾}{zhao4gu4}{13,10}[HSK 2]
  \definition{v.}{cuidar de | atender a | oferecer tratamento preferencial | (de um cliente) patrocinar | fazer compras em | dar consideração a | mostrar consideração por | levar em conta | fazer concessões para}
\end{entry}

\begin{entry}{照亮}{zhao4liang4}{13,9}
  \definition{s.}{iluminação}
  \definition{v.}{iluminar}
\end{entry}

\begin{entry}{照片}{zhao4pian4}{13,4}[HSK 2]
  \definition[张,套,幅]{s.}{fotografia | foto}
\end{entry}

\begin{entry}{照片底版}{zhao4pian4di3ban3}{13,4,8,8}
  \definition{s.}{placa fotográfica}
\end{entry}

\begin{entry}{照片子}{zhao4pian4zi5}{13,4,3}
  \definition{v.}{tirar um raio X}
\end{entry}

\begin{entry}{照骗}{zhao4pian4}{13,12}
  \definition{s.}{imagem ``photoshopada''}
\end{entry}

\begin{entry}{照相}{zhao4 xiang4}{13,9}[HSK 2]
  \definition{v.+compl.}{tirar fotografia}
\end{entry}

\begin{entry}{照相机}{zhao4xiang4ji1}{13,9,6}
  \definition[个,架,部,台,只]{s.}{câmera/máquina fotográfica}
\end{entry}

\begin{entry}{照像}{zhao4xiang4}{13,13}
  \variantof{照相}
\end{entry}

\begin{entry}{照像机}{zhao4xiang4ji1}{13,13,6}
  \variantof{照相机}
\end{entry}

\begin{entry}{照准}{zhao4zhun3}{13,10}
  \definition{s.}{solicitação concedida (uso formal em documento antigo)}
  \definition{v.}{mirar (arma)}
\end{entry}

\begin{entry}{折转}{zhe2zhuan3}{7,8}
  \definition{s.}{reflexo (ângulo)}
  \definition{v.}{voltar atrás}
\end{entry}

\begin{entry}{哲理}{zhe2li3}{10,11}
  \definition{s.}{filosofia | teoria filosófica}
\end{entry}

\begin{entry}{这}{zhe4}{7}[Radical 辵][HSK 1]
  \definition{pron.}{este, isto}
  \seeref{这}{zhei4}
\end{entry}

\begin{entry}{这边}{zhe4bian5}{7,5}[HSK 1]
  \definition{pron.}{aqui | este lado}
\end{entry}

\begin{entry}{这里}{zhe4li3}{7,7}[HSK 1]
  \definition{pron.}{aqui}
\end{entry}

\begin{entry}{这么}{zhe4 me5}{7,3}[HSK 2]
  \definition{adv.}{como este | desta maneira}
\end{entry}

\begin{entry}{这末}{zhe4me5}{7,5}
  \variantof{这么}
\end{entry}

\begin{entry}{这麽}{zhe4me5}{7,14}
  \variantof{这么}
\end{entry}

\begin{entry}{这儿}{zhe4r5}{7,2}[HSK 1]
  \definition{pron.}{aqui}
\end{entry}

\begin{entry}{这时}{zhe4 shi2}{7,7}[HSK 2]
  \definition{adv.}{neste momento}
\end{entry}

\begin{entry}{这时候}{zhe4 shi2 hou5}{7,7,10}
  \definition{adv.}{neste momento}
\end{entry}

\begin{entry}{这些}{zhe4xie1}{7,8}[HSK 1]
  \definition{pron.}{estes}
\end{entry}

\begin{entry}{这样}{zhe4 yang4}{7,10}[HSK 2]
  \definition{adv.}{assim | dessa maneira | deste modo}
\end{entry}

\begin{entry}{浙江}{zhe4jiang1}{10,6}
  \definition*{s.}{Zhejiang}
\end{entry}

\begin{entry}{着}{zhe5}{11}[Radical 目][HSK 1]
  \definition{part.}{indicando ação em andamento ou estado em andamento}
  \seeref{着}{zhao1}
  \seeref{着}{zhao2}
  \seeref{着}{zhuo2}
\end{entry}

\begin{entry}{这}{zhei4}{7}[Radical 辵]
  \definition{pron.}{(coloquial) este}
  \seeref{这}{zhe4}
\end{entry}

\begin{entry}{珍贵}{zhen1gui4}{9,9}
  \definition{adj.}{precioso}
\end{entry}

\begin{entry}{珍珠}{zhen1zhu1}{9,10}
  \definition[颗]{s.}{pérola}
\end{entry}

\begin{entry}{眞}{zhen1}{10}[Radical 目]
  \variantof{真}
\end{entry}

\begin{entry}{真}{zhen1}{10}[Radical 目][HSK 1]
  \definition{adj.}{genuíno}
  \definition{adv.}{que\dots tão\dots! | realmente}
\end{entry}

\begin{entry}{真的}{zhen1 de5}{10,8}[HSK 1]
  \definition{adv.}{realmente | verdadeiramente}
\end{entry}

\begin{entry}{真理}{zhen1li3}{10,11}
  \definition[个]{s.}{verdade}
\end{entry}

\begin{entry}{真牛}{zhen1niu2}{10,4}
  \definition{adj.}{(gíria) muito legal, incrível}
\end{entry}

\begin{entry}{真切}{zhen1qie4}{10,4}
  \definition{adj.}{claro | distinto | honesto | sincero | vívido}
\end{entry}

\begin{entry}{真声}{zhen1sheng1}{10,7}
  \definition{s.}{voz natural | voz verdadeira}
  \seeref{假声}{jia3sheng1}
\end{entry}

\begin{entry}{真释}{zhen1shi4}{10,12}
  \definition{s.}{razão genuína | explicação verdadeira}
\end{entry}

\begin{entry}{真心}{zhen1xin1}{10,4}
  \definition{adj.}{sincero}
  \definition[片]{s.}{sinceridade}
\end{entry}

\begin{entry}{真真}{zhen1zhen1}{10,10}
  \definition{adv.}{genuinamente | realmente | escrupulosamente}
\end{entry}

\begin{entry}{真正}{zhen1zheng4}{10,5}[HSK 2]
  \definition{adj.}{verdadeiro | real | genuíno}
  \definition{adv.}{realmente | de ​​fato}
\end{entry}

\begin{entry}{真珠}{zhen1zhu1}{10,10}
  \variantof{珍珠}
\end{entry}

\begin{entry}{枕}{zhen3}{8}[Radical 木]
  \definition{s.}{travesseiro | almofada}
\end{entry}

\begin{entry}{阵地}{zhen4di4}{6,6}
  \definition{s.}{posição (militar) | frente de batalha | \emph{front}}
\end{entry}

\begin{entry}{震撼}{zhen4han4}{15,16}
  \definition{v.}{sacudir | chocar | atordoar}
\end{entry}

\begin{entry}{正}{zheng1}{5}[Radical 止]
  \definition{s.}{primeiro mês do ano lunar}
  \seeref{正}{zheng4}
\end{entry}

\begin{entry}{争霸}{zheng1ba4}{6,21}
  \definition{s.}{hegemonia | uma luta de poder}
  \definition{v.}{disputar a hegemonia}
\end{entry}

\begin{entry}{争风吃醋}{zheng1feng1chi1cu4}{6,4,6,15}
  \definition{v.}{rivalizar com alguém pelo carinho de um homem ou mulher |estar com ciúmes de um rival em um caso de amor}
\end{entry}

\begin{entry}{争先}{zheng1xian1}{6,6}
  \definition{v.}{competir para ser o primeiro |contestar o primeiro lugar}
\end{entry}

\begin{entry}{挣扎}{zheng1zha2}{9,4}
  \definition{v.}{lutar}
\end{entry}

\begin{entry}{整天}{zheng3tian1}{16,4}
  \definition{adv.}{dia todo | o dia inteiro}
\end{entry}

\begin{entry}{正}{zheng4}{5}[Radical 止][HSK 1]
  \definition{adj.}{reto | vertical | adequado | principal | (matemática) positivo}
  \definition{adv.}{agora mesmo | no processo de}
  \definition{v.}{corrigir | retificar}
  \seeref{正}{zheng1}
\end{entry}

\begin{entry}{正常}{zheng4chang2}{5,11}[HSK 2]
  \definition{adj.}{regular | normal | ordinário}
\end{entry}

\begin{entry}{正好}{zheng4hao3}{5,6}[HSK 2]
  \definition{adj.}{na medida certa | na hora certa | o suficiente}
  \definition{adv.}{acontecer com | chance de | como acontece}
\end{entry}

\begin{entry}{正确}{zheng4que4}{5,12}[HSK 2]
  \definition{adj.}{correto | certo | próprio}
\end{entry}

\begin{entry}{正是}{zheng4 shi4}{5,9}[HSK 2]
  \definition{adv.}{precisamente | exatamente}
\end{entry}

\begin{entry}{正在}{zheng4zai4}{5,6}[HSK 1]
  \definition{adv.}{no processo de | atualmente | em andamento}
  \definition{v.}{estar a~+~v.inf. | estar~+~v.ger.}
\end{entry}

\begin{entry}{正正}{zheng4zheng4}{5,5}
  \definition{adv.}{na hora certa | ordenadamente}
\end{entry}

\begin{entry}{正宗}{zheng4zong1}{5,8}
  \definition{adj.}{autêntico | genuíno | \emph{old school} | (fig.) tradicional}
\end{entry}

\begin{entry}{证件}{zheng4jian4}{7,6}
  \definition{s.}{documento de identificação | credencial | certificado | comprovante}
\end{entry}

\begin{entry}{证据}{zheng4ju4}{7,11}
  \definition{s.}{evidência | prova | testemunho}
\end{entry}

\begin{entry}{证实}{zheng4shi2}{7,8}
  \definition{v.}{confirmar (algo como verdadeiro) | verificar}
\end{entry}

\begin{entry}{挣}{zheng4}{9}[Radical 手]
  \definition{v.}{ganhar dinheiro | esforçar-se para adquirir | lutar para se libertar}
\end{entry}

\begin{entry}{挣得}{zheng4de2}{9,11}
  \definition{v.}{ganhar renda ou dinheiro}
\end{entry}

\begin{entry}{挣钱}{zheng4qian2}{9,10}
  \definition{v.+compl.}{ganhar dinheiro}
\end{entry}

\begin{entry}{政府}{zheng4fu3}{9,8}
  \definition[个]{s.}{governo}
\end{entry}

\begin{entry}{政纲}{zheng4gang1}{9,7}
  \definition{s.}{programa ou plataforma política}
\end{entry}

\begin{entry}{政治局}{zheng4zhi4ju2}{9,8,7}
  \definition{s.}{o principal comitê de políticas de um partido comunista}
\end{entry}

\begin{entry}{之外}{zhi1wai4}{3,5}
  \definition{adv.}{lado de fora}
\end{entry}

\begin{entry}{支}{zhi1}{4}[Radical 支][Kangxi 65]
  \definition*{s.}{sobrenome Zhi}
  \definition{clas.}{para varetas como canetas e armas | para divisões do exército e para canções ou composições}
  \definition{v.}{sacar dinheiro | erguer | criar | suportar | sustentar}
\end{entry}

\begin{entry}{支承}{zhi1cheng2}{4,8}
  \definition{v.}{suportar o peso de (um edifício) | suportar}
\end{entry}

\begin{entry}{支持}{zhi1chi2}{4,9}
  \definition[个]{s.}{apoio | suporte}
  \definition{v.}{apoiar | ser a favor de | suportar}
\end{entry}

\begin{entry}{支根}{zhi1gen1}{4,10}
  \definition{s.}{raiz ramificada | raízes de apoio | radícula}
\end{entry}

\begin{entry}{支票}{zhi1piao4}{4,11}
  \definition[本]{s.}{cheque (banco)}
\end{entry}

\begin{entry}{支应}{zhi1ying4}{4,7}
  \definition{v.}{lidar com | fornecer}
\end{entry}

\begin{entry}{支支吾吾}{zhi1zhi1wu2wu2}{4,4,7,7}
  \definition{v.}{falhar | murmurar | paralisar | gaguejar}
\end{entry}

\begin{entry}{只}{zhi1}{5}[Radical 口]
  \definition{clas.}{para pássaros, gatos, cãezinhos, etc.}
  \seeref{只}{zhi3}
\end{entry}

\begin{entry}{只身}{zhi1shen1}{5,7}
  \definition{adv.}{sozinho | por si só}
\end{entry}

\begin{entry}{芝麻}{zhi1ma5}{6,11}
  \definition{s.}{semente de gergelim}
\end{entry}

\begin{entry}{知道}{zhi1dao4}{8,12}[HSK 1]
  \definition{v.}{conhecer | saber}
\end{entry}

\begin{entry}{知道了}{zhi1dao4le5}{8,12,2}
  \definition{interj.}{Entendi! | OK!}
\end{entry}

\begin{entry}{知识}{zhi1shi5}{8,7}[HSK 1]
  \definition[门]{s.}{conhecimento}
  \definition{s.}{intelectual}
\end{entry}

\begin{entry}{织}{zhi1}{8}[Radical 糸]
  \definition{v.}{tecer | tricotar}
\end{entry}

\begin{entry}{脂麻}{zhi1ma5}{10,11}
  \variantof{芝麻}
\end{entry}

\begin{entry}{蜘蛛}{zhi1zhu1}{14,12}
  \definition{s.}{aranha}
\end{entry}

\begin{entry}{蜘蛛网}{zhi1zhu1wang3}{14,12,6}
  \definition{s.}{teia de aranha}
\end{entry}

\begin{entry}{执着}{zhi2zhuo2}{6,11}
  \definition{s.}{(budismo) apego}
  \definition{v.}{estar fortemente apegado a | ser dedicado | apegar-se a}
\end{entry}

\begin{entry}{直播}{zhi2bo1}{8,15}
  \definition{s.}{transmissão ao vivo | (agricultura) semeadura direta}
  \definition{v.}{(TV, rádio, Internet) transmitir ao vivo}
\end{entry}

\begin{entry}{直接}{zhi2jie1}{8,11}[HSK 2]
  \definition{adj.}{direto (oposto: indireto 间接) | imediato}
  \seeref{间接}{jian4jie1}
\end{entry}

\begin{entry}{直译}{zhi2yi4}{8,7}
  \definition{s.}{tradução literal}
  \seealsoref{意译}{yi4yi4}
\end{entry}

\begin{entry}{直译器}{zhi2yi4qi4}{8,7,16}
  \definition{s.}{(computação) interpretador}
\end{entry}

\begin{entry}{职业}{zhi2ye4}{11,5}
  \definition{adj.}{profissional}
  \definition{s.}{ocupação | profissão | vocação}
\end{entry}

\begin{entry}{职员}{zhi2yuan2}{11,7}
  \definition[个,位]{s.}{empregado | trabalhador de escritório | membro da equipe}
\end{entry}

\begin{entry}{殖}{zhi2}{12}[Radical 歹]
  \definition{v.}{crescer | reproduzir}
\end{entry}

\begin{entry}{只}{zhi3}{5}[Radical 口][HSK 2]
  \definition{adv.}{apenas | só}
  \seeref{只}{zhi1}
\end{entry}

\begin{entry}{只得}{zhi3de5}{5,11}
  \definition{v.}{ser obrigado a | não ter outra alternativa senão}
\end{entry}

\begin{entry}{只读}{zhi3du2}{5,10}
  \definition{s.}{somente leitura (computação) | \emph{read-only}}
\end{entry}

\begin{entry}{只顾}{zhi3gu4}{5,10}
  \definition{adv.}{exclusivamente preocupado (com uma coisa)}
  \definition{v.}{cuidar de apenas um aspecto}
\end{entry}

\begin{entry}{只好}{zhi3hao3}{5,6}
  \definition{adv.}{ser forçado a | ter que | sem nenhuma opção melhor | não ter outro remédio senão}
\end{entry}

\begin{entry}{只能}{zhi3 neng2}{5,10}[HSK 2]
  \definition{adv.}{só pode | obrigado a fazer algo}
\end{entry}

\begin{entry}{只怕}{zhi3pa4}{5,8}
  \definition{adv.}{receio que\dots | talvez | muito provavelmente}
\end{entry}

\begin{entry}{只消}{zhi3xiao1}{5,10}
  \definition{conj.}{desde que}
\end{entry}

\begin{entry}{只要}{zhi3yao4}{5,9}[HSK 2]
  \definition{conj.}{se apenas | contanto que}
\end{entry}

\begin{entry}{只要……就……}{zhi3yao4 jiu4}{5,9,12}
  \definition{conj.}{contanto que/desde que/se somente\dots, então\dots}
\end{entry}

\begin{entry}{只有……才……}{zhi3you3 cai2}{5,6,3}
  \definition{conj.}{só se\dots então\dots}
\end{entry}

\begin{entry}{纸}{zhi3}{7}[Radical 糸][HSK 2]
  \definition{clas.}{para documentos, cartas, etc.}
  \definition[张,沓]{s.}{papel}
\end{entry}

\begin{entry}{纸币}{zhi3bi4}{7,4}
  \definition[张]{s.}{nota (dinheiro) | cédula}
\end{entry}

\begin{entry}{纸巾}{zhi3jin1}{7,3}
  \definition[张,包]{s.}{lenço | guardanapo | papel toalha}
\end{entry}

\begin{entry}{纸尿裤}{zhi3niao4ku4}{7,7,12}
  \definition{s.}{fralda descartável}
\end{entry}

\begin{entry}{纸烟}{zhi3yan1}{7,10}
  \definition{s.}{cigarro}
\end{entry}

\begin{entry}{纸张}{zhi3zhang1}{7,7}
  \definition{s.}{papel}
\end{entry}

\begin{entry}{指挥}{zhi3hui1}{9,9}
  \definition[个]{s.}{condutor (de uma orquestra)}
  \definition{v.}{conduzir | comandar | direcionar}
\end{entry}

\begin{entry}{指甲}{zhi3jia5}{9,5}
  \definition{s.}{unha da mão}
\end{entry}

\begin{entry}{指南针}{zhi3nan2zhen1}{9,9,7}
  \definition{s.}{bússola}
\end{entry}

\begin{entry}{至于}{zhi4yu2}{6,3}
  \definition{conj.}{para | quanto a | a respeiro de}
\end{entry}

\begin{entry}{志愿}{zhi4yuan4}{7,14}
  \definition{s.}{aspiração | ambição}
  \definition{v.}{ser voluntário}
\end{entry}

\begin{entry}{制裁}{zhi4cai2}{8,12}
  \definition{s.}{punição | sanção (inclusive econômica)}
  \definition{v.}{punir}
\end{entry}

\begin{entry}{治理}{zhi4li3}{8,11}
  \definition{s.}{governança | governo}
  \definition{v.}{gerir para melhor | administrar | por em ordem}
\end{entry}

\begin{entry}{治愈}{zhi4yu4}{8,13}
  \definition{v.}{curar | restaurar a saúde}
\end{entry}

\begin{entry}{致敬}{zhi4jing4}{10,12}
  \definition{v.}{saudar | prestar respeitos a | prestar homenagem a}
\end{entry}

\begin{entry}{智慧}{zhi4hui4}{12,15}
  \definition{s.}{sabedoria | inteligência}
\end{entry}

\begin{entry}{智商}{zhi4shang1}{12,11}
  \definition{s.}{quociente de inteligência, QI}
\end{entry}

\begin{entry}{智障}{zhi4zhang4}{12,13}
  \definition{adj./s.}{retardado}
\end{entry}

\begin{entry}{置疑}{zhi4yi2}{13,14}
  \definition{v.}{duvidar}
\end{entry}

\begin{entry}{中}{zhong1}{4}[Radical 丨][HSK 1]
  \definition*{s.}{China}
  \definition*{s.}{sobrenome Zhong}
  \definition{s.}{centro | meio | médio | intermediário | média | meio caminho entre dois extremos | intermediador}
  \seeref{中}{zhong4}
  \seealsoref{中国}{zhong1guo2}
\end{entry}

\begin{entry}{中餐}{zhong1 can1}{4,16}[HSK 2]
  \definition[分,顿]{s.}{comida chinesa | almoço}
\end{entry}

\begin{entry}{中东}{zhong1dong1}{4,5}
  \definition*{s.}{Oriente Médio}
\end{entry}

\begin{entry}{中国}{zhong1guo2}{4,8}[HSK 1]
  \definition*{s.}{China}
\end{entry}

\begin{entry}{中国城}{zhong1guo2cheng2}{4,8,9}
  \definition*{s.}{Bairro Chinês, \emph{Chinatown}}
  \seeref{唐人街}{tang2ren2 jie1}
\end{entry}

\begin{entry}{中国科学院}{zhong1guo2 ke1xue2yuan4}{4,8,9,8,9}
  \definition*{s.}{Academia Chinesa de Ciências}
\end{entry}

\begin{entry}{中国人}{zhong1guo2ren2}{4,8,2}
  \definition{s.}{chinês | pessoa ou povo da China}
\end{entry}

\begin{entry}{中国通}{zhong1guo2tong1}{4,8,10}
  \definition*{s.}{Conhecedor da China, especialista em tudo sobre a China}
\end{entry}

\begin{entry}{中级}{zhong1 ji2}{4,6}[HSK 2]
  \definition{adj.}{nível médio | nível intermediário}
\end{entry}

\begin{entry}{中间}{zhong1jian1}{4,7}[HSK 1]
  \definition{adv.}{central | centro | no meio}
\end{entry}

\begin{entry}{中年}{zhong1 nian2}{4,6}[HSK 2]
  \definition{s.}{meia-idade}
\end{entry}

\begin{entry}{中情局}{zhong1qing2ju2}{4,11,7}
  \definition*{s.}{Agência Central de Inteligência dos EUA, CIA (abreviação de 中央情报局)}
  \seeref{中央情报局}{zhong1yang1 qing2bao4ju2}
\end{entry}

\begin{entry}{中秋节}{zhong1qiu1jie2}{4,9,5}
  \definition*{s.}{Festival do Meio-Outono | Festival do Bolo Lunar (15º dia do oitavo mês lunar)}
\end{entry}

\begin{entry}{中文}{zhong1wen2}{4,4}[HSK 1]
  \definition{s.}{chinês, língua chinesa}
\end{entry}

\begin{entry}{中午}{zhong1wu3}{4,4}[HSK 1]
  \definition[个]{s.}{meio-dia}
\end{entry}

\begin{entry}{中小学}{zhong1 xiao3 xue2}{4,3,8}[HSK 2]
  \definition{s.}{escolas primárias e secundárias}
\end{entry}

\begin{entry}{中心}{zhong1xin1}{4,4}[HSK 2]
  \definition[个]{s.}{núcleo | coração | meio | centro |chave}
\end{entry}

\begin{entry}{中性}{zhong1xing4}{4,8}
  \definition{adj.}{neutro}
\end{entry}

\begin{entry}{中学}{zhong1xue2}{4,8}[HSK 1]
  \definition[个]{s.}{escola ensino médio}
\end{entry}

\begin{entry}{中学生}{zhong1xue2sheng1}{4,8,5}[HSK 1]
  \definition{s.}{aluno, estudante de escola ensino médio}
\end{entry}

\begin{entry}{中询}{zhong1 xun2}{4,8}
  \definition{adv.}{segunda dezena do mês | meio do mês | em meados do mês}
\end{entry}

\begin{entry}{中央情报局}{zhong1yang1 qing2bao4ju2}{4,5,11,7,7}
  \definition*{s.}{Agência Central de Inteligência dos EUA, CIA}
\end{entry}

\begin{entry}{中药}{zhong1yao4}{4,9}
  \definition[服,种]{s.}{medicina tradicional chinesa}
\end{entry}

\begin{entry}{中医}{zhong1 yi1}{4,7}[HSK 2]
  \definition{s.}{ciência médica tradicional chinesa | médico de medicina tradicional chinesa | praticante de medicina chinesa}
\end{entry}

\begin{entry}{钟}{zhong1}{9}[Radical 金]
  \definition*{s.}{sobrenome Zhong}
  \definition{s.}{copo sem pega | taça}
  \definition{v.}{concentrar}
\end{entry}

\begin{entry}{钟室}{zhong1shi4}{9,9}
  \definition{s.}{campanário | sala do relógio}
\end{entry}

\begin{entry}{钟罩}{zhong1zhao4}{9,13}
  \definition{s.}{redoma | dossel de sino}
\end{entry}

\begin{entry}{锺}{zhong1}{14}[Radical 金]
  \variantof{钟}
\end{entry}

\begin{entry}{种}{zhong3}{9}[Radical 禾]
  \definition{clas.}{para tipos, espécies e gêneros}
  \definition{s.}{tipo | espécie}
\end{entry}

\begin{entry}{种麻}{zhong3ma2}{9,11}
  \definition{s.}{planta de cânhamo (feminina)}
\end{entry}

\begin{entry}{种薯}{zhong3shu3}{9,16}
  \definition{s.}{tubérculo semente}
\end{entry}

\begin{entry}{种种}{zhong3zhong3}{9,9}
  \definition{adj.}{todos os tipos de}
\end{entry}

\begin{entry}{种子}{zhong3zi5}{9,3}
  \definition[颗,粒]{s.}{semente}
\end{entry}

\begin{entry}{种族灭绝}{zhong3zu2mie4jue2}{9,11,5,9}
  \definition{s.}{genocídio | extinção étnica}
\end{entry}

\begin{entry}{中}{zhong4}{4}[Radical 丨]
  \definition{v.}{acertar | encaixar exatamente |ser atingido por | cair em | ser afetado por | sofrer | sustentar}
  \seeref{中}{zhong1}
\end{entry}

\begin{entry}{中意}{zhong4yi4}{4,13}
  \definition{s.}{ser do seu agrado | começar a gostar muito de algo ou de alguém}
\end{entry}

\begin{entry}{众}{zhong4}{6}[Radical 人]
  \definition*{s.}{Câmara dos Deputados, abreviação de 众议院}
  \definition{adj.}{numeroso}
  \definition{adv.}{muitos}
  \definition{s.}{multidão}
  \seeref{众议院}{zhong4yi4yuan4}
\end{entry}

\begin{entry}{众议院}{zhong4yi4yuan4}{6,5,9}
  \definition*{s.}{Casa baixa da Assembléia Bicameral | Câmara dos Deputados}
\end{entry}

\begin{entry}{种地}{zhong4di4}{9,6}
  \definition{v.}{cultivar | trabalhar a terra}
\end{entry}

\begin{entry}{重}{zhong4}{9}[Radical ⾥][HSK 1]
  \definition{adj.}{pesado | profundo; sério | importante; momentoso | discreto; prudente | considerável em quantidade ou valor}
  \definition{adv.}{pesadamente; severamente}
  \definition{v.}{colocar (pôr) ênfase em; dar valor a; atribuir importância a}
  \seeref{重}{chong2}
\end{entry}

\begin{entry}{重点}{zhong4dian3}{9,9}[HSK 2]
  \definition{s.}{nota-chave | ponto-chave | ponto focal | ênfase}
  \seeref{重点}{chong2dian3}
\end{entry}

\begin{entry}{重量}{zhong4liang4}{9,12}
  \definition[个]{s.}{peso}
\end{entry}

\begin{entry}{重视}{zhong4shi4}{9,8}[HSK 2]
  \definition{v.}{atribuir valor a | dar peso a | atribuir importância a | prestar atenção a}
\end{entry}

\begin{entry}{重要}{zhong4yao4}{9,9}[HSK 1]
  \definition{adj.}{importante | significativo | principal}
\end{entry}

\begin{entry}{重重}{zhong4zhong4}{9,9}
  \definition{adv.}{fortemente | severamente}
  \seeref{重重}{chong2chong2}
\end{entry}

\begin{entry}{周}{zhou1}{8}[Radical 口][HSK 2]
  \definition*{s.}{sobrenome Zhou | Dinastia Zhou (1046-256 BC)}
  \definition{adv.}{semanalmente}
  \definition{s.}{círculo | circunferência | ciclo | uma volta (em um circuito) | semana}
  \definition{v.}{fazer um circuito |circular | ajudar financeiramente}
\end{entry}

\begin{entry}{周末}{zhou1mo4}{8,5}[HSK 2]
  \definition{s.}{final-de-semana}
\end{entry}

\begin{entry}{周年}{zhou1nian2}{8,6}[HSK 2]
  \definition{s.}{aniversário}
\end{entry}

\begin{entry}{洲}{zhou1}{9}[Radical 水]
  \definition{s.}{continente | ilha em um rio}
\end{entry}

\begin{entry}{轴承}{zhou2cheng2}{9,8}
  \definition{s.}{(mecânico) rolamento}
\end{entry}

\begin{entry}{咒骂}{zhou4ma4}{8,9}
  \definition{v.}{xingar | amaldiçoar | execrar}
\end{entry}

\begin{entry}{珠子}{zhu1zi5}{10,3}
  \definition[粒,颗]{s.}{pérola | contas}
\end{entry}

\begin{entry}{猪}{zhu1}{11}[Radical 犬]
  \definition[口,头]{s.}{porco | suíno}
\end{entry}

\begin{entry}{猪窠}{zhu1ke1}{11,13}
  \definition{s.}{chiqueiro}
\end{entry}

\begin{entry}{猪柳}{zhu1liu3}{11,9}
  \definition{s.}{filé de porco}
\end{entry}

\begin{entry}{猪笼}{zhu1long2}{11,11}
  \definition{s.}{estrutura cilíndrica de bambu ou arame usada para restringir um porco durante o transporte}
\end{entry}

\begin{entry}{猪头}{zhu1tou2}{11,5}
  \definition{s.}{tolo | idiota}
\end{entry}

\begin{entry}{竹编}{zhu2bian1}{6,12}
  \definition{s.}{vime | tecelagem de bambu}
\end{entry}

\begin{entry}{竹马}{zhu2ma3}{6,3}
  \definition{s.}{cavalo de bambu | vara de bambu usada como cavalo de brinquedo}
\end{entry}

\begin{entry}{竹排}{zhu2pai2}{6,11}
  \definition{s.}{jangada de bambu}
\end{entry}

\begin{entry}{竹子}{zhu2zi5}{6,3}
  \definition[棵,支,根]{s.}{bambu}
\end{entry}

\begin{entry}{逐步}{zhu2bu4}{10,7}
  \definition{adv.}{pouco a pouco; passo a passo; progressivamente}
\end{entry}

\begin{entry}{逐渐}{zhu2jian4}{10,11}
  \definition{adv.}{pouco a pouco; passo a passo; progressivamente}
\end{entry}

\begin{entry}{主人}{zhu3ren2}{5,2}[HSK 2]
  \definition[个,位]{s.}{mestre | anfitrião | proprietário | uma pessoa que tem um certo tipo de bens ou poder}
\end{entry}

\begin{entry}{主席}{zhu3xi2}{5,10}
  \definition*[个,位]{s.}{Presidente (da China) | Primeiro-Ministro}
\end{entry}

\begin{entry}{主席台}{zhu3xi2tai2}{5,10,5}
  \definition[个]{s.}{plataforma | tribuna}
\end{entry}

\begin{entry}{主席团}{zhu3xi2tuan2}{5,10,6}
  \definition{s.}{presídio}
\end{entry}

\begin{entry}{主要}{zhu3yao4}{5,9}[HSK 2]
  \definition{adj.}{principal}
\end{entry}

\begin{entry}{主义}{zhu3yi4}{5,3}
  \definition{s.}{ideologia}
  \definition{suf.}{"ismo"}
\end{entry}

\begin{entry}{属}{zhu3}{12}[Radical 尸]
  \definition{v.}{juntar-se | fixar a atenção em | concentrar-se em}
  \seeref{属}{shu3}
\end{entry}

\begin{entry}{嘱}{zhu3}{15}[Radical 口]
  \definition{v.}{juntar-se | implorar | incitar}
\end{entry}

\begin{entry}{嘱咐}{zhu3fu5}{15,8}
  \definition{v.}{ordenar | dizer | exortar}
\end{entry}

\begin{entry}{嘱托}{zhu3tuo1}{15,6}
  \definition{v.}{confiar uma tarefa a alguém}
\end{entry}

\begin{entry}{住}{zhu4}{7}[Radical 人][HSK 1]
  \definition{v.}{habitar | residir | morar | alojar-se}
\end{entry}

\begin{entry}{住处}{zhu4chu4}{7,5}
  \definition{s.}{morada | habitação | residência}
\end{entry}

\begin{entry}{住房}{zhu4fang2}{7,8}[HSK 2]
  \definition{s.}{habitação}
\end{entry}

\begin{entry}{住所}{zhu4suo3}{7,8}
  \definition[处]{s.}{morada | habitação | residência}
\end{entry}

\begin{entry}{住院}{zhu4 yuan4}{7,9}[HSK 2]
  \definition{v.}{estar hospitalizado | estar no hospital}
\end{entry}

\begin{entry}{住宅}{zhu4zhai2}{7,6}
  \definition{s.}{residência}
\end{entry}

\begin{entry}{住嘴}{zhu4zui3}{7,16}
  \definition{interj.}{Cale-se!}
  \definition{v.}{calar | calar-se}
\end{entry}

\begin{entry}{助兴}{zhu4xing4}{7,6}
  \definition{v.+compl.}{animar as coisas | juntar-se à diversão}
\end{entry}

\begin{entry}{注册}{zhu4ce4}{8,5}
  \definition{v.}{inscrever-se | matricular-se | registrar-se}
\end{entry}

\begin{entry}{注册表}{zhu4ce4biao3}{8,5,8}
  \definition*{s.}{Registro do Windows}
\end{entry}

\begin{entry}{注册人}{zhu4ce4ren2}{8,5,2}
  \definition{s.}{registrante}
\end{entry}

\begin{entry}{注册商标}{zhu4ce4shang1biao1}{8,5,11,9}
  \definition{s.}{marca registrada}
\end{entry}

\begin{entry}{注意}{zhu4yi4}{8,13}
  \definition{v.}{tomar nota de | prestar atenção em}
\end{entry}

\begin{entry}{注意地}{zhu4yi4di4}{8,13,6}
  \definition{s.}{área de cuidado, de observação}
\end{entry}

\begin{entry}{注意力}{zhu4yi4li4}{8,13,2}
  \definition{s.}{atenção}
\end{entry}

\begin{entry}{注意力缺失症}{zhu4yi4li4que1shi1zheng4}{8,13,2,10,5,10}
  \definition{s.}{transtorno de déficit de atenção}
\end{entry}

\begin{entry}{驻军}{zhu4jun1}{8,6}
  \definition{s.}{guarnição}
  \definition{v.}{guarcener ou prover uma tropa}
\end{entry}

\begin{entry}{祝}{zhu4}{9}[Radical 示]
  \definition*{s.}{sobrenome Zhu}
  \definition{v.}{desejar (exprimir um bom desejo) | congratular | rezar}
\end{entry}

\begin{entry}{祝祷}{zhu4dao3}{9,11}
  \definition{v.}{rezar | orar}
\end{entry}

\begin{entry}{祝福}{zhu4fu2}{9,13}
  \definition{s.}{bênçãos}
  \definition{v.}{desejar boa sorte a alguém}
\end{entry}

\begin{entry}{祝好}{zhu4hao3}{9,6}
  \definition{expr.}{desejo-lhe tudo de melhor! (ao encerrar uma correspondência)}
\end{entry}

\begin{entry}{祝贺}{zhu4he4}{9,9}
  \definition[个]{s.}{congratulações}
  \definition{v.}{congratular}
\end{entry}

\begin{entry}{祝酒}{zhu4jiu3}{9,10}
  \definition{v.}{parabenizar e fazer um brinde | brindar}
\end{entry}

\begin{entry}{祝寿}{zhu4shou4}{9,7}
  \definition{v.}{dar parabéns pelo aniversário (a uma pessoa idosa)}
\end{entry}

\begin{entry}{祝颂}{zhu4song4}{9,10}
  \definition{v.}{expressar bons desejos}
\end{entry}

\begin{entry}{祝谢}{zhu4xie4}{9,12}
  \definition{v.}{agradecer | dar parabéns}
\end{entry}

\begin{entry}{祝愿}{zhu4yuan4}{9,14}
  \definition{v.}{desejar}
\end{entry}

\begin{entry}{专业}{zhuan1ye4}{4,5}
  \definition[门,个]{s.}{área de atuação | especialidade}
\end{entry}

\begin{entry}{专业户}{zhuan1ye4hu4}{4,5,4}
  \definition{s.}{indústria caseira | empresa familiar produzindo um produto especial}
\end{entry}

\begin{entry}{专业化}{zhuan1ye4hua4}{4,5,4}
  \definition{s.}{especialização}
\end{entry}

\begin{entry}{专业教育}{zhuan1ye4jiao4yu4}{4,5,11,8}
  \definition{s.}{educação especializada | escola técnica}
\end{entry}

\begin{entry}{专业人才}{zhuan1ye4ren2cai2}{4,5,2,3}
  \definition{s.}{especialista (em uma área)}
\end{entry}

\begin{entry}{专业人士}{zhuan1ye4ren2shi4}{4,5,2,3}
  \definition{s.}{profissional}
\end{entry}

\begin{entry}{专业性}{zhuan1ye4xing4}{4,5,8}
  \definition{s.}{profissionalismo | expertise}
\end{entry}

\begin{entry}{砖}{zhuan1}{9}[Radical 石]
  \definition[块]{s.}{tijolo}
\end{entry}

\begin{entry}{转}{zhuan3}{8}[Radical 車]
  \definition{v.}{mudar de direção | transferir | encaminhar (correio) | virar}
  \seeref{转}{zhuan4}
\end{entry}

\begin{entry}{转产}{zhuan3chan3}{8,6}
  \definition{v.}{mudar a produção | mudar para novos produtos}
\end{entry}

\begin{entry}{转递}{zhuan3di4}{8,10}
  \definition{v.}{passar | retransmitir}
\end{entry}

\begin{entry}{转告}{zhuan3gao4}{8,7}
  \definition{v.}{comunicar | transmitir}
\end{entry}

\begin{entry}{转念}{zhuan3nian4}{8,8}
  \definition{v.}{ter dúvidas sobre algo | pensar melhor}
\end{entry}

\begin{entry}{转账}{zhuan3zhang4}{8,8}
  \definition{v.+compl.}{transferir entre contas | trazer à frente | incluir uma soma de dinheiro do balanço anterior no seguinte}
\end{entry}

\begin{entry}{传}{zhuan4}{6}[Radical 人]
  \definition{s.}{comentários sobre clássicos | biografia | romances sobre eventos históricos}
  \seeref{传}{chuan2}
\end{entry}

\begin{entry}{转}{zhuan4}{8}[Radical 車]
  \definition{clas.}{para ações repetidas | para rotações (por minuto, etc.): RPM}
  \definition{v.}{circular sobre | dar voltas | andar por aí}
  \seeref{转}{zhuan3}
\end{entry}

\begin{entry}{转悠}{zhuan4you5}{8,11}
  \definition{v.}{aparecer repetidamente | rolar | passear por aí}
\end{entry}

\begin{entry}{转游}{zhuan4you5}{8,12}
  \variantof{转悠}
\end{entry}

\begin{entry}{妆}{zhuang1}{6}[Radical 女]
  \definition{s.}{maquiagem | adorno | enxoval | maquiagem e figurino de palco}
  \definition{v.}{maquiar-se | enfeitar-se}
\end{entry}

\begin{entry}{妆扮}{zhuang1ban4}{6,7}
  \variantof{装扮}
\end{entry}

\begin{entry}{桩}{zhuang1}{10}[Radical 木]
  \definition{clas.}{para eventos, casos, transações, assuntos, etc.}
  \definition{s.}{toco | estaca | pilha}
\end{entry}

\begin{entry}{装}{zhuang1}{12}[Radical 衣][HSK 2]
  \definition{s.}{adorno | roupa | traje (de um ator em uma peça)}
  \definition{v.}{adornar | vestir | desepenhar um papel | fingir | instalar | consertar | embrulhar (algo em um saco) | empacotar}
\end{entry}

\begin{entry}{装扮}{zhuang1ban4}{12,7}
  \definition{v.}{enfeitar | decorar | disfarçar-me | vestir-se}
\end{entry}

\begin{entry}{装备}{zhuang1bei4}{12,8}
  \definition{s.}{equipamento}
  \definition{v.}{equipar}
\end{entry}

\begin{entry}{装配}{zhuang1pei4}{12,10}
  \definition{v.}{montar | encaixar}
\end{entry}

\begin{entry}{撞车}{zhuang4che1}{15,4}
  \definition{v.+compl.}{(figurativo) colidir (opiniões, cronogramas, etc.) | ser o mesmo (assunto) | colidir (com outro veículo)}
\end{entry}

\begin{entry}{撞运气}{zhuang4yun4qi5}{15,7,4}
  \definition{v.}{confiar no destino | tentar a sorte}
\end{entry}

\begin{entry}{追赶}{zhui1gan3}{9,10}
  \definition{v.}{perseguir | acelerar | alcançar | ultrapassar}
\end{entry}

\begin{entry}{坠}{zhui4}{7}[Radical 土]
  \definition{v.}{cair | pesar | fazer vergar com o peso}
\end{entry}

\begin{entry}{坠落}{zhui4luo4}{7,12}
  \definition{v.}{cair}
\end{entry}

\begin{entry}{准}{zhun3}{10}[Radical 冫]
  \definition{adv.}{certamente | de acordo com | à luz de}
  \definition{v.}{permitir | conceder}
\end{entry}

\begin{entry}{准备}{zhun3bei4}{10,8}[HSK 1]
  \definition{v.}{preparar | ficar pronto | pretender | planejar}
\end{entry}

\begin{entry}{准确}{zhun3que4}{10,12}[HSK 2]
  \definition{adj.}{exato | preciso | acurado}
\end{entry}

\begin{entry}{桌}{zhuo1}{10}[Radical 木]
  \definition{clas.}{para mesas de convidados em um banquete etc.}
  \definition{s.}{mesa}
\end{entry}

\begin{entry}{桌布}{zhuo1bu4}{10,5}
  \definition[条,块,张]{s.}{(computação) plano de fundo da área de trabalho | toalha de mesa | papel de parede}
\end{entry}

\begin{entry}{桌灯}{zhuo1deng1}{10,6}
  \definition{s.}{luminária | lâmpada de mesa}
\end{entry}

\begin{entry}{桌机}{zhuo1ji1}{10,6}
  \definition{s.}{computador \emph{desktop}}
\end{entry}

\begin{entry}{桌面}{zhuo1mian4}{10,9}
  \definition{s.}{área de trabalho | mesa}
\end{entry}

\begin{entry}{桌球}{zhuo1qiu2}{10,11}
  \definition{s.}{bilhar | sinuca | mesa de ping-pong}
\end{entry}

\begin{entry}{桌游}{zhuo1you2}{10,12}
  \definition{s.}{jogo de tabuleiro}
\end{entry}

\begin{entry}{桌子}{zhuo1zi5}{10,3}[HSK 1]
  \definition[张,套]{s.}{mesa}
\end{entry}

\begin{entry}{棹}{zhuo1}{12}[Radical 木]
  \variantof{桌}
\end{entry}

\begin{entry}{着}{zhuo2}{11}[Radical 目]
  \definition{v.}{aplicar | contactar | usar | vestir (roupas)}
  \seeref{着}{zhao1}
  \seeref{着}{zhao2}
  \seeref{着}{zhe5}
\end{entry}

\begin{entry}{着花}{zhuo2hua1}{11,7}
  \definition{s.}{floração}
  \definition{v.}{florescer}
  \seeref{着花}{zhao2hua1}
\end{entry}

\begin{entry}{着手}{zhuo2shou3}{11,4}
  \definition{v.}{colocar a mão nisso | estabelecer | começar uma tarefa}
\end{entry}

\begin{entry}{着想}{zhuo2xiang3}{11,13}
  \definition{v.}{considerar (as necessidades de outras pessoas) | pensar (para os outros)}
\end{entry}

\begin{entry}{着眼}{zhuo2yan3}{11,11}
  \definition{v.}{ter seus olhos em (um objetivo) | ter algo em mente | concentrar-se}
\end{entry}

\begin{entry}{着装}{zhuo2zhuang1}{11,12}
  \definition{s.}{roupa | vestimenta}
  \definition{v.}{vestir}
\end{entry}

\begin{entry}{资}{zi1}{10}[Radical 貝]
  \definition{s.}{recursos | capital | dinheiro | despesa}
  \definition{v.}{fornecer | suprir}
\end{entry}

\begin{entry}{资助}{zi1zhu4}{10,7}
  \definition{s.}{subsídio}
  \definition{v.}{subsidiar | fornecer ajuda financeira}
\end{entry}

\begin{entry}{子}{zi3}{3}[Radical 子]
  \definition{adj.}{jovem | pequeno | tenro}
  \definition{clas.}{para objetos finos que podem ser pinçados com os dedos}
  \definition{pron.}{você}
  \definition{s.}{filho | pessoa | antigo título de respeito para um homem culto ou virtuoso | semente | ovo; ova | coisas pequenas e duras | moeda de cobre; cobre | o quarto título da classificação dos cinco títulos feudais de nobreza; visconde}
  \seeref{子}{zi5}
\end{entry}

\begin{entry}{子弹}{zi3dan4}{3,11}
  \definition[粒,颗,发]{s.}{bala (de revólver)}
\end{entry}

\begin{entry}{紫}{zi3}{12}[Radical 糸]
  \definition{adj.}{púrpura | violeta}
\end{entry}

\begin{entry}{紫色}{zi3 se4}{12,6}
  \definition{s.}{cor púrpura | cor violeta}
\end{entry}

\begin{entry}{字}{zi4}{6}[Radical 子][HSK 1]
  \definition[个]{s.}{carácter | letra | símbolo | palavra}
\end{entry}

\begin{entry}{字典}{zi4 dian3}{6,8}[HSK 2]
  \definition[本]{s.}{dicionário de caracteres chineses (contendo verbetes de caracteres únicos, em contraste com 词典 que contém verbetes para palavras com um ou mais caracteres)}
  \seeref{词典}{ci2dian3}
\end{entry}

\begin{entry}{字脚}{zi4jiao3}{6,11}
  \definition[典]{s.}{gancho no final da pincelada | serifa}
\end{entry}

\begin{entry}{字母}{zi4mu3}{6,5}
  \definition[个]{s.}{letra (do alfabeto)}
\end{entry}

\begin{entry}{字眼}{zi4yan3}{6,11}
  \definition[个]{s.}{palavras | redação}
\end{entry}

\begin{entry}{字字珠玉}{zi4zi4zhu1yu4}{6,6,10,5}
  \definition{expr.}{cada palavra é uma jóia}
  \definition{s.}{escrita magnífica}
\end{entry}

\begin{entry}{自动化}{zi4dong4hua4}{6,6,4}
  \definition{s.}{automação}
\end{entry}

\begin{entry}{自个儿}{zi4ge3r5}{6,3,2}
  \definition{pron.}{(dialeto) a si mesmo, por si mesmo}
\end{entry}

\begin{entry}{自己}{zi4ji3}{6,3}[HSK 2]
  \definition{pron.}{a si próprio | próprio}
\end{entry}

\begin{entry}{自己动手}{zi4ji3dong4shou3}{6,3,6,4}
  \definition{v.}{fazer (algo) sozinho | ajudar-se a}
\end{entry}

\begin{entry}{自救}{zi4jiu4}{6,11}
  \definition{v.}{sair a si mesmo de problemas}
\end{entry}

\begin{entry}{自来水}{zi4lai2shui3}{6,7,4}
  \definition{s.}{água corrente | água da torneira}
\end{entry}

\begin{entry}{自然}{zi4ran2}{6,12}
  \definition{adj.}{natural}
  \definition{adv.}{naturalmente | de maneira natural}
  \definition{s.}{natureza}
\end{entry}

\begin{entry}{自燃}{zi4ran2}{6,16}
  \definition{s.}{combustão espontânea}
\end{entry}

\begin{entry}{自我}{zi4wo3}{6,7}
  \definition{pref.}{auto}
  \definition{pron.}{a si mesmo | eu próprio | (psicologia) ego}
\end{entry}

\begin{entry}{自我安慰}{zi4wo3'an1wei4}{6,7,6,15}
  \definition{v.}{confortar-se | consolar-se | tranquilizar-se}
\end{entry}

\begin{entry}{自我保存}{zi4wo3 bao3cun2}{6,7,9,6}
  \definition{v.}{autopreservação}
\end{entry}

\begin{entry}{自我吹嘘}{zi4wo3chui1xu1}{6,7,7,14}
  \definition{expr.}{tocar a própria buzina}
\end{entry}

\begin{entry}{自我催眠}{zi4wo3cui1mian2}{6,7,13,10}
  \definition{v.}{consolar-me | tranquilizar-me}
\end{entry}

\begin{entry}{自我的人}{zi4wo3de5ren2}{6,7,8,2}
  \definition{s.}{(minha, sua) própria pessoa | (afirmar) a própria personalidade}
\end{entry}

\begin{entry}{自我防卫}{zi4wo3fang2wei4}{6,7,6,3}
  \definition{s.}{defesa pessoal | auto-defesa}
\end{entry}

\begin{entry}{自我解嘲}{zi4wo3jie3chao2}{6,7,13,15}
  \definition{s.}{referir-se às próprias fraquezas ou falhas com humor autodepreciativo}
\end{entry}

\begin{entry}{自我介绍}{zi4wo3jie4shao4}{6,7,4,8}
  \definition{s.}{defesa pessoal | auto-defesa}
\end{entry}

\begin{entry}{自我批评}{zi4wo3pi1ping2}{6,7,7,7}
  \definition{s.}{autocrítica}
\end{entry}

\begin{entry}{自我实现}{zi4wo3shi2xian4}{6,7,8,8}
  \definition{s.}{(psicologia) auto-atualização, auto-realização}
\end{entry}

\begin{entry}{自我陶醉}{zi4wo3tao2zui4}{6,7,10,15}
  \definition{s.}{narcisista | auto-imbuído | satisfeito consigo mesmo}
\end{entry}

\begin{entry}{自我意识}{zi4wo3yi4shi2}{6,7,13,7}
  \definition{s.}{autoapresentação}
  \definition{v.}{apresentar-se}
\end{entry}

\begin{entry}{自行车}{zi4xing2che1}{6,6,4}[HSK 2]
  \definition[辆]{s.}{bicicleta}
\end{entry}

\begin{entry}{自行车馆}{zi4xing2che1guan3}{6,6,4,11}
  \definition{s.}{estádio de ciclismo | velódromo}
\end{entry}

\begin{entry}{自行车架}{zi4xing2che1jia4}{6,6,4,9}
  \definition{s.}{suporte para bicicleta | bicicletário}
\end{entry}

\begin{entry}{自行车赛}{zi4xing2che1sai4}{6,6,4,14}
  \definition{s.}{corrida de bicicleta}
\end{entry}

\begin{entry}{自由}{zi4you2}{6,5}[HSK 2]
  \definition{adj.}{livre, irrestrito}
  \definition[种]{s.}{liberdade}
\end{entry}

\begin{entry}{自由泳}{zi4you2yong3}{6,5,8}
  \definition{s.}{natação de estilo livre}
\end{entry}

\begin{entry}{自责}{zi4ze2}{6,8}
  \definition{v.}{culpar-se}
\end{entry}

\begin{entry}{子}{zi5}{3}[Radical 子][HSK 1]
  \definition{clas.}{sufixos de palavras de medida individuais}
  \definition{suf.}{sufixo para substantivos}
  \seeref{子}{zi3}
\end{entry}

\begin{entry}{棕褐色}{zong1he4 se4}{12,14,6}
  \definition{s.}{cor sépia | bronzeado}
\end{entry}

\begin{entry}{总}{zong3}{9}[Radical 心]
  \definition{adv.}{em geral | completamente}
\end{entry}

\begin{entry}{总长}{zong3chang2}{9,4}
  \definition{s.}{comprimento total}
\end{entry}

\begin{entry}{总得}{zong3dei3}{9,11}
  \definition{adv.}{prestes a}
  \definition{v.}{dever | precisar}
\end{entry}

\begin{entry}{总督}{zong3du1}{9,13}
  \definition*{s.}{Governador-Geral | Governador | Vice-Rei}
\end{entry}

\begin{entry}{总价}{zong3jia4}{9,6}
  \definition{s.}{preço total}
\end{entry}

\begin{entry}{总结}{zong3jie2}{9,9}
  \definition[个]{s.}{currículo | resumo}
  \definition{v.}{concluir | resumir}
\end{entry}

\begin{entry}{总理}{zong3li3}{9,11}
  \definition*{s.}{Primeiro-Ministro}
\end{entry}

\begin{entry}{总台}{zong3tai2}{9,5}
  \definition{s.}{recepção | balcão de recepção}
\end{entry}

\begin{entry}{总统}{zong3tong3}{9,9}
  \definition*[个,位,名,届]{s.}{Presidente (de um país)}
\end{entry}

\begin{entry}{总务}{zong3wu4}{9,5}
  \definition{s.}{divisão de assuntos gerais | assuntos gerais | pessoa responsável geral}
\end{entry}

\begin{entry}{总线}{zong3xian4}{9,8}
  \definition{s.}{barramento (computador) | \emph{computer bus}}
\end{entry}

\begin{entry}{总站}{zong3zhan4}{9,10}
  \definition{s.}{terminal}
\end{entry}

\begin{entry}{总值}{zong3zhi2}{9,10}
  \definition{s.}{valor total}
\end{entry}

\begin{entry}{赱}{zou3}{6}[Radical 土]
  \variantof{走}
\end{entry}

\begin{entry}{走}{zou3}{7}[Radical 走][Kangxi 156][HSK 1]
  \definition{v.}{andar | caminhar}
\end{entry}

\begin{entry}{走鬼}{zou3gui3}{7,9}
  \definition{s.}{vendedor ambulante sem licença}
\end{entry}

\begin{entry}{走过}{zou3 guo4}{7,6}[HSK 2]
  \definition{v.}{passar}
\end{entry}

\begin{entry}{走进}{zou3 jin4}{7,7}[HSK 2]
  \definition{v.}{entrar}
\end{entry}

\begin{entry}{走开}{zou3 kai1}{7,4}[HSK 2]
  \definition{v.}{ir embora | fugir | ir para outro lugar}
\end{entry}

\begin{entry}{走路}{zou3lu4}{7,13}[HSK 1]
  \definition{v.}{andar | ir a pé | sair | ir embora}
\end{entry}

\begin{entry}{走去}{zou3qu4}{7,5}
  \definition{v.}{caminhar até (para)}
\end{entry}

\begin{entry}{走绳}{zou3sheng2}{7,11}
  \definition{v.}{andar na corda bamba}
  \seeref{走索}{zou3suo3}
\end{entry}

\begin{entry}{走势}{zou3shi4}{7,8}
  \definition{s.}{caminho | tendência}
\end{entry}

\begin{entry}{走索}{zou3suo3}{7,10}
  \definition{v.}{andar na corda bamba}
  \seeref{走绳}{zou3sheng2}
\end{entry}

\begin{entry}{走秀}{zou3xiu4}{7,7}
  \definition{s.}{desfile de moda}
  \definition{v.}{andar na passarela (em um desfile de moda)}
\end{entry}

\begin{entry}{走卒}{zou3zu2}{7,8}
  \definition{s.}{lacaio (masculino) | peão (isto é, soldado de infantaria) | servo}
\end{entry}

\begin{entry}{奏效}{zou4xiao4}{9,10}
  \definition{v.}{mostrar resultados | ser eficaz}
\end{entry}

\begin{entry}{租}{zu1}{10}[Radical 禾][HSK 2]
  \definition{s.}{imposto sobre propriedade urbana ou rural}
  \definition{v.}{alugar | tomar de aluguel}
\end{entry}

\begin{entry}{租船}{zu1chuan2}{10,11}
  \definition{v.}{fretar um navio | alugar um navio}
\end{entry}

\begin{entry}{租房}{zu1fang2}{10,8}
  \definition{v.}{alugar um apartamento}
\end{entry}

\begin{entry}{租金}{zu1jin1}{10,8}
  \definition{s.}{aluguel}
  \seeref{租钱}{zu1qian5}
\end{entry}

\begin{entry}{租赁}{zu1lin4}{10,10}
  \definition{v.}{contratar | alugar}
\end{entry}

\begin{entry}{租钱}{zu1qian5}{10,10}
  \definition{s.}{aluguel}
  \seeref{租金}{zu1jin1}
\end{entry}

\begin{entry}{租让}{zu1rang4}{10,5}
  \definition{v.}{alugar | alugar (a propriedade de alguém para outra pessoa)}
\end{entry}

\begin{entry}{租用}{zu1yong4}{10,5}
  \definition{v.}{contratar | alugar | alugar (algo de alguém)}
\end{entry}

\begin{entry}{租约}{zu1yue1}{10,6}
  \definition{s.}{aluguel}
\end{entry}

\begin{entry}{足}{zu2}{7}[Radical 足][Kangxi 157]
  \definition{adj.}{amplo}
  \definition{s.}{pé}
  \definition{v.}{ser suficiente}
  \seeref{足}{ju4}
\end{entry}

\begin{entry}{足球}{zu2qiu2}{7,11}
  \definition[个]{s.}{futebol | bola de futebol}
\end{entry}

\begin{entry}{足球场}{zu2qiu2chang3}{7,11,6}
  \definition{s.}{campo de futebol}
\end{entry}

\begin{entry}{足球队}{zu2qiu2dui4}{7,11,4}
  \definition{s.}{time de futebol}
\end{entry}

\begin{entry}{足球迷}{zu2qiu2mi2}{7,11,9}
  \definition{s.}{fã de futebol}
\end{entry}

\begin{entry}{足球赛}{zu2qiu2sai4}{7,11,14}
  \definition{s.}{competição de futebol | partida de futebol}
\end{entry}

\begin{entry}{足球协会}{zu2qiu2xie2hui4}{7,11,6,6}
  \definition*{s.}{Associação de Futebol}
\end{entry}

\begin{entry}{足月}{zu2yue4}{7,4}
  \definition{s.}{gestação completa}
\end{entry}

\begin{entry}{足足}{zu2zu2}{7,7}
  \definition{adv.}{tanto quanto | extremamente | completamente | não menos que}
\end{entry}

\begin{entry}{族}{zu2}{11}[Radical 方]
  \definition{s.}{raça | nacionalidade | etnia | clã | por extensão, grupo social}
\end{entry}

\begin{entry}{诅咒}{zu3zhou4}{7,8}
  \definition{v.}{amaldiçoar}
\end{entry}

\begin{entry}{阻击}{zu3ji1}{7,5}
  \definition{v.}{verificar | parar}
\end{entry}

\begin{entry}{组}{zu3}{8}[Radical 糸][HSK 2]
  \definition*{s.}{sobrenome Zu}
  \definition{clas.}{para conjuntos, séries, suítes, baterias}
  \definition{s.}{grupo}
  \definition{v.}{organizar | formar}
\end{entry}

\begin{entry}{组成}{zu3cheng2}{8,6}[HSK 2]
  \definition{v.}{formar | compor | inventar}
\end{entry}

\begin{entry}{组长}{zu3 zhang3}{8,4}[HSK 2]
  \definition[名,位,个]{s.}{líder de grupo}
\end{entry}

\begin{entry}{祖国}{zu3guo2}{9,8}
  \definition{s.}{pátria | terra natal}
\end{entry}

\begin{entry}{钻戒}{zuan4jie4}{10,7}
  \definition[只]{s.}{anel de diamante}
\end{entry}

\begin{entry}{钻石}{zuan4shi2}{10,5}
  \definition[颗]{s.}{diamante}
\end{entry}

\begin{entry}{嘴}{zui3}{16}[Radical 口][HSK 2]
  \definition[张]{s.}{boca | qualquer coisa com formato ou função semelhante a uma boca}
  \definition{v.}{falar}
\end{entry}

\begin{entry}{嘴巴}{zui3ba5}{16,4}
  \definition[张]{s.}{boca}
  \definition[个]{s.}{bofetada na cara}
\end{entry}

\begin{entry}{嘴巴子}{zui3ba5zi5}{16,4,3}
  \definition{s.}{tapa | bofetada}
\end{entry}

\begin{entry}{最}{zui4}{12}[Radical 冂][HSK 1]
  \definition{adv.}{o mais | o melhor | a coisa mais\dots | grau superlativo relativo de superioridade}
\end{entry}

\begin{entry}{最初}{zui4chu1}{12,7}
  \definition{adj.}{inicial | original | primário}
  \definition{adv.}{inicialmente | originalmente}
\end{entry}

\begin{entry}{最多}{zui4duo1}{12,6}
  \definition{adv.}{no máximo | máximo}
\end{entry}

\begin{entry}{最高}{zui4gao1}{12,10}
  \definition{adj.}{altíssimo | supremo | mais alto}
\end{entry}

\begin{entry}{最好}{zui4hao3}{12,6}[HSK 1]
  \definition{adv.}{ser melhor que}
  \definition{v.}{(você) estar melhor (faça o que sugerimos) | querer ser o melhor}
\end{entry}

\begin{entry}{最后}{zui4hou4}{12,6}[HSK 1]
  \definition{adj.}{final | último}
  \definition{adv.}{finalmente}
\end{entry}

\begin{entry}{最佳}{zui4jia1}{12,8}
  \definition{adj.}{melhor (atleta, filme etc) | ótimo}
\end{entry}

\begin{entry}{最近}{zui4jin4}{12,7}[HSK 2]
  \definition{adv.}{ultimamente | recentemente}
\end{entry}

\begin{entry}{最善}{zui4shan4}{12,12}
  \definition{adj.}{ótimo | o melhor}
\end{entry}

\begin{entry}{最少}{zui4shao3}{12,4}
  \definition{adv.}{finalmente}
\end{entry}

\begin{entry}{最先}{zui4xian1}{12,6}
  \definition{adv.}{o primeiro}
\end{entry}

\begin{entry}{最新}{zui4xin1}{12,13}
  \definition{adv.}{mais recente | mais novo}
\end{entry}

\begin{entry}{最优}{zui4you1}{12,6}
  \definition{adj.}{ótimo}
\end{entry}

\begin{entry}{最远}{zui4yuan3}{12,7}
  \definition{adv.}{mais distante | mais longe}
\end{entry}

\begin{entry}{最终}{zui4zhong1}{12,8}
  \definition{adv.}{pelo menos | finalmente}
  \definition{s.}{final | ultimato}
\end{entry}

\begin{entry}{罪犯}{zui4fan4}{13,5}
  \definition{s.}{criminoso}
\end{entry}

\begin{entry}{罪行}{zui4xing2}{13,6}
  \definition{s.}{crime | ofensa}
\end{entry}

\begin{entry}{醉}{zui4}{15}[Radical 酉]
  \definition{v.}{embriagar-se | ficar bêbado}
\end{entry}

\begin{entry}{作}{zuo1}{7}[Radical 人]
  \definition{adj.}{(gíria) incômodo}
  \definition{s.}{trabalhador | oficina | (pessoa) de alta manutenção}
  \seeref{作}{zuo4}
\end{entry}

\begin{entry}{昨}{zuo2}{9}[Radical 日]
  \definition{s.}{ontem}
\end{entry}

\begin{entry}{昨日}{zuo2ri4}{9,4}
  \definition{adv.}{ontem}
\end{entry}

\begin{entry}{昨天}{zuo2tian1}{9,4}[HSK 1]
  \definition{adv.}{ontem}
\end{entry}

\begin{entry}{昨晚}{zuo2wan3}{9,11}
  \definition{adv.}{noite passada | ontem à noite}
\end{entry}

\begin{entry}{昨夜}{zuo2ye4}{9,8}
  \definition{adv.}{noite passada}
\end{entry}

\begin{entry}{左}{zuo3}{5}[Radical 工][HSK 1]
  \definition*{s.}{sobrenome Zuo}
  \definition{s.}{esquerda}
\end{entry}

\begin{entry}{左边}{zuo3bian5}{5,5}[HSK 1]
  \definition{s.}{esquerda | lado esquerdo}
\end{entry}

\begin{entry}{左面}{zuo3mian4}{5,9}
  \definition{s.}{esquerda | lado esquerdo}
\end{entry}

\begin{entry}{左派}{zuo3pai4}{5,9}
  \definition{s.}{(política) esquerda | esquerdista}
\end{entry}

\begin{entry}{左倾}{zuo3qing1}{5,10}
  \definition{s.}{esquerdista | progressivo}
\end{entry}

\begin{entry}{左袒}{zuo3tan3}{5,10}
  \definition{v.}{ser tendencioso | ser parcial para | favorecer um lado | tomar partido com}
\end{entry}

\begin{entry}{左舷}{zuo3xian2}{5,11}
  \definition{s.}{porto (lado de um navio)}
\end{entry}

\begin{entry}{左翼}{zuo3yi4}{5,17}
  \definition{s.}{esquerda (política)}
\end{entry}

\begin{entry}{左右}{zuo3you4}{5,5}
  \definition{adv.}{cerca de | aproximadamente}
\end{entry}

\begin{entry}{作}{zuo4}{7}[Radical 人]
  \definition{s.}{escritos ou obras}
  \definition{v.}{fazer | crescer | escrever ou compor | fingir | considerar como | sentir}
  \seeref{作}{zuo1}
\end{entry}

\begin{entry}{作家}{zuo4jia1}{7,10}[HSK 2]
  \definition[位,个]{s.}{autor | escritor}
\end{entry}

\begin{entry}{作文}{zuo4wen2}{7,4}[HSK 2]
  \definition[篇]{s.}{ensaio |  composição | redação}
  \definition{v.+compl.}{(de alunos) para escrever uma redação}
\end{entry}

\begin{entry}{作业}{zuo4ye4}{7,5}[HSK 2]
  \definition[份,个]{s.}{tarefa escolar | trabalho | tarefa | operação}
\end{entry}

\begin{entry}{作用}{zuo4yong4}{7,5}[HSK 2]
  \definition{s.}{efeito | ação | função}
  \definition{v.}{afetar | agir em}
\end{entry}

\begin{entry}{坐}{zuo4}{7}[Radical 土][HSK 1]
  \definition*{s.}{sobrenome Zuo}
  \definition{v.}{sentar-se | andar de carro, ônibus, trem, avião, etc.}
\end{entry}

\begin{entry}{坐标}{zuo4biao1}{7,9}
  \definition{s.}{coordenada (geometria)}
\end{entry}

\begin{entry}{坐车}{zuo4che1}{7,4}
  \definition{v.}{andar de carro, ônibus, trem, etc.}
\end{entry}

\begin{entry}{坐垫}{zuo4dian4}{7,9}
  \definition[块]{s.}{assento (motocicleta) | almofada}
\end{entry}

\begin{entry}{坐好}{zuo4hao3}{7,6}
  \definition{v.}{sentar-se corretamente | sentar direito}
\end{entry}

\begin{entry}{坐下}{zuo4xia5}{7,3}[HSK 1]
  \definition{v.}{sentar-se | tomar um assento}
\end{entry}

\begin{entry}{坐享}{zuo4xiang3}{7,8}
  \definition{v.}{curtir algo sem levantar um dedo}
\end{entry}

\begin{entry}{座}{zuo4}{10}[Radical 广][HSK 2]
  \definition{clas.}{frequentemente usado para objetos maiores ou fixos}
  \definition{s.}{assento | lugar | base | suporte | pedestal | constelação}
\end{entry}

\begin{entry}{座标}{zuo4biao1}{10,9}
  \variantof{坐标}
\end{entry}

\begin{entry}{座位}{zuo4wei4}{10,7}[HSK 2]
  \definition[个]{s.}{assento | lugar}
\end{entry}

\begin{entry}{座子}{zuo4zi5}{10,3}
  \definition{s.}{soquete | pedestal | sela}
\end{entry}

\begin{entry}{做}{zuo4}{11}[Radical 人][HSK 1]
  \definition{v.}{fazer}
\end{entry}

\begin{entry}{做到}{zuo4 dao4}{11,8}[HSK 2]
  \definition{v.}{realizar | alcançar}
\end{entry}

\begin{entry}{做法}{zuo4fa3}{11,8}[HSK 2]
  \definition[个]{s.}{método para fazer | prática | receita | maneira de lidar com algo | método de trabalho}
\end{entry}

\begin{entry}{做饭}{zuo4 fan4}{11,7}[HSK 2]
  \definition{v.}{preparar uma refeição | cozinhar}
\end{entry}

\begin{entry}{做活}{zuo4huo2}{11,9}
  \definition{v.}{trabalhar para ganhar a vida (especialmente de mulher costureira)}
\end{entry}

\begin{entry}{做生活}{zuo4sheng1huo2}{11,5,9}
  \definition{v.}{fazer tabalhos manuais}
\end{entry}

\begin{entry}{做戏}{zuo4xi4}{11,6}
  \definition{v.}{atuar em uma peça | fazer uma peça}
\end{entry}

\begin{entry}{做眼}{zuo4yan3}{11,11}
  \definition{v.}{agir como um guia | trabalhar como espião}
\end{entry}

\begin{entry}{做作}{zuo4zuo5}{11,7}
  \definition{adj.}{afetado | artificial}
\end{entry}

\begin{entry}{酢}{zuo4}{12}[Radical 酉]
  \definition{v.}{brindar o anfitrião com vinho}
\end{entry}

%%%%% EOF %%%%%


\onecolumn

\ifdraftdoc
%%%
\else

\clearpage
\pagestyle{plain}
\chapter{Termos Gramaticais Chineses}
%%%%%%%%%%%%%%%%%%%%%%%%%%%%%%%%%%%%%%%%%%%%%%%%%%%%%%%%%%%%%%%%%%%%%%%%%%%%%%%
%%%%%%%%%%%%%%%%%%%%%%%%%%%%%%%%%%%%%%%%%%%%%%%%%%%%%%%%%%%%%%%%%%%%%%%%%%%%%%%
%%%%%                                                                     %%%%%
%%%%% termos_gramaticais.tex:                                             %%%%%
%%%%% Tabela dos termos gramaticais utilizados neste Dicionário.          %%%%%
%%%%%                                                                     %%%%%
%%%%%%%%%%%%%%%%%%%%%%%%%%%%%%%%%%%%%%%%%%%%%%%%%%%%%%%%%%%%%%%%%%%%%%%%%%%%%%%
%%%%%%%%%%%%%%%%%%%%%%%%%%%%%%%%%%%%%%%%%%%%%%%%%%%%%%%%%%%%%%%%%%%%%%%%%%%%%%%

%%% Tabela
\begin{center}
\begin{tblr}[m]{lll}
substantivo/nome & \textbf{s.} & 名词 \\
pronome & \textbf{pron.} & 代词 \\
numeral & \textbf{num.} & 数词 \\
classificador & \textbf{clas.} & 量词 \\
verbo & \textbf{v.} & 动词 \\
verbo auxiliar & \textbf{v.aux.} & 助动词 \\
verbo+complemento & \textbf{v.+compl.} & 动宾式\hspace{1em}离合词 \\
adjetivo & \textbf{adj.} & 形容词 \\
advérbio & \textbf{adv.} & 副词 \\
preposição & \textbf{prep.} & 介词 \\
conjunção & \textbf{conj.} & 连词 \\
partícula & \textbf{part.} & 助词 \\
interjeição & \textbf{interj.} & 叹词 \\
prefixo & \textbf{pref.} & 前缀 \\
sufixo & \textbf{suf.} & 后缀 \\
expressão & \textbf{expr.} & 熟语 \\
\end{tblr}
\end{center}

%%%%% EOF %%%%%


\clearpage
\pagestyle{plain}
\chapter{Classificadores Nominais}
%%%%%%%%%%%%%%%%%%%%%%%%%%%%%%%%%%%%%%%%%%%%%%%%%%%%%%%%%%%%%%%%%%%%%%%%%%%%%%%
%%%%%%%%%%%%%%%%%%%%%%%%%%%%%%%%%%%%%%%%%%%%%%%%%%%%%%%%%%%%%%%%%%%%%%%%%%%%%%%
%%%%%                                                                     %%%%%
%%%%% classificadores_nominais.tex:                                       %%%%%
%%%%% Tabela com as palavras classificadoras para substantivos            %%%%%
%%%%%                                                                     %%%%%
%%%%%%%%%%%%%%%%%%%%%%%%%%%%%%%%%%%%%%%%%%%%%%%%%%%%%%%%%%%%%%%%%%%%%%%%%%%%%%%
%%%%%%%%%%%%%%%%%%%%%%%%%%%%%%%%%%%%%%%%%%%%%%%%%%%%%%%%%%%%%%%%%%%%%%%%%%%%%%%

%%% Ajustes para a tabela
\DefTblrTemplate{caption}{default}{}
\DefTblrTemplate{capcont}{default}{ \UseTblrTemplate{conthead-text}{default} }
\DefTblrTemplate{contfoot-text}{default}{Continua na próxima página.}
\DefTblrTemplate{conthead-text}{default}{(Continuação)}
\DefTblrTemplate{firsthead}{default}{ \UseTblrTemplate{caption}{default} }
\DefTblrTemplate{middlehead,lasthead}{default}{ \UseTblrTemplate{conthead}{default} }
\DefTblrTemplate{firstfoot,middlefoot}{default}{ \UseTblrTemplate{contfoot}{default} }
\DefTblrTemplate{lastfoot}{default}{ \UseTblrTemplate{note}{default} \UseTblrTemplate{remark}{default} }

%%% Tabela
\begin{longtblr}
{
 colspec = {|c|c|X|X|}, hlines,
 width = 1\linewidth,
 rowhead = 1, rowfoot = 0,
 row{1} = {font=\bfseries, fg=white, bg=black},
}
\textbf{Hanzi} & \textbf{Pinyin} & \textbf{Descrição} & \textbf{Exemplos}\\
 把 & \dpy{ba3}     & mão cheia --- objetos que podem ser segurados, objetos relativamente longos e planos & faca, tesoura, espada, chave, guarda-chuva, leque, escova de dentes, colher, garfo, martelo, cadeado, pistola, rifle, carne, punhado de arroz, punhado de areia, ramo de flores, punhado de sementes, ramo de pauzinhos, esqueleto, fogo, bule\\
 班 & \dpy{ban1}    & serviços programados (trens, aviões, etc), grupos de pessoas, uma classe como em alunos & \\
 包 & \dpy{bao1}    & pacote & doces, cigarros, açúcar, biscoitos\\
 杯 & \dpy{bei1}    & copo --- bebidas & chá, vinho, álcool, água, leite, suco de fruta, refrigerante\\
 本 & \dpy{ben3}    & volume --- material impresso encadernado & livro, revista, romance, escritura, dicionário, bloco de notas, livro didático\\
 笔 & \dpy{bi3}     & traços de caracteres; grandes quantidades de dinheiro & \\
 部 & \dpy{bu4}     & máquinas, veículos; produções & celular, telefone, carro, jogo, romance, filme/imagem em movimento, ópera, obra literária\\
 册 & \dpy{ce4}     & volumes de livros & \\
 层 & \dpy{ceng2}   & andar, piso; camada & andares (em um prédio); camada de poeira, bolo, tinta\\
 场 & \dpy{chang3}  & evento de curta duração; precipitação & espetáculo público, jogo, crise, guerra, catástrofe, uma doença, performance, jogo, chuva, queda de neve\\
 串 & \dpy{chuan4}  & conjuntos de números & telefone celular/número de celular\\
 床 & \dpy{chuang4} & cama & cobertores, lençóis\\
 次 & \dpy{ci4}     & tempo, repetições --- oportunidades, acidentes & \\
 出 & \dpy{chu1}    & atuação em uma peça & \\
 打 & \dpy{da2}     & dúzia & lápis, ovos\\
 贷 & \dpy{dai4}    & sacos ou bolsos cheios & açúcar, farinha, arroz\\
 道 & \dpy{dao4}    & projeções lineares (raios de luz, etc.); ordem dada por uma figura de autoridade; pergunta, memorando; curso (de comida); coisas longas e tortas (cume da montanha, relâmpago) & \\
 滴 & \dpy{di1}     & gotículas de água, sangue, outros fluidos semelhantes & \\
 点 & \dpy{dian3}   & ideias, sugestões; um pouco/algum (somente com 一) & \\
 碟 & \dpy{die2}    & pires (molho de soja) & \\
 顶 & \dpy{ding3}   & objetos com topo saliente, algo para colocar sobre a cabeça & chapéu, barraca, mosquiteiro\\
 栋 & \dpy{dong4}   & pilar (edifício menor, casa) & \\
 堵 & \dpy{du3}     & luminárias abrangentes (parede sem teto) & \\
 段 & \dpy{duan4}   & comprimento adjacente --- cabos, estradas, pedaço de giz, parte de uma música & \\
 对 & \dpy{dui4}    & casal, par combinado (para certas coisas), dísticos & casal, brincos, vasos\\
 顿 & \dpy{dun4}    & ações de curta duração & refeição, conflito, espancamento, briga, repreensão\\
 朵 & \dpy{duo3}    & coisas parecidas com flores & flor, nuvem, cogumelo\\
 发 & \dpy{fa1}     & coisas redondas & bala, munição\\
 方 & \dpy{fang4}   & coisas quadradas & pedra de tinta, bacon\\
 份 & \dpy{fen4}    & porções, documentos de várias páginas & porção de comida, jornal, emprego\\
 封 & \dpy{feng1}   & coisas que podem ser seladas & carta, correio, telegrama\\
 峰 & \dpy{feng1}   & animais com corcundas & camelo\\
 幅 & \dpy{fu2}     & objetos semelhantes a imagens & foto, pintura, desenho, banner, obra de arte, quadro, cartaz\\
 服 & \dpy{fu4}     & dose (de remédio)\\
 副 & \dpy{fu4}     & objetos que vêm em pares (luvas, óculos, etc.), baralhos, mahjong & \\
 个 & {\dpy{ge5}\\ \dpy{ge4}} & coisas individuais, pessoas, classificador de uso geral (o uso desse classificador em conjunto com qualquer substantivo é geralmente aceito se a pessoa não souber o classificador adequado) & pessoa, irmão mais velho, estudante, parente, modo de pensar, sugestão, pergunta, nação\\
 根 & \dpy{gen1}    & objetos finos e esguios; fios finos e flexíveis & agulha, pilar, banana, palito de massa frita, coxa de frango frita, picolé, pirulito, pauzinho, vela, incenso, cabelo, linha, barbante\\
 股 & \dpy{gu3}     & fluxos (de ar, cheiro, influência, etc.) & \\
 挂 & \dpy{gua4}    & coisas multi-componentes & cavalo e carroça \\
 罐 & \dpy{guan4}   & lata pequena a média & refrigerante, suco, comida, feijão, óleo, doce\\
 行 & \dpy{hang2}   & objetos que formam linhas (palavras, etc.) & \\
 盒 & \dpy{he2}     & caixa pequena & fita, comida, bolo, doces, chocolate, brinquedos, livros, cigarros, detergente, roupas\\
 户 & \dpy{hu4}     & famílias & \\
 伙 & \dpy{huo3}    & classificador geralmente depreciativo para bandos de pessoas, como gangues ou bandidos & \\
 家 & \dpy{jia1}    & reunião de pessoas, estabelecimentos & famílias, empresas, lojas, restaurantes\\
 架 & \dpy{jia4}    & maquinário, veículo & aeronave, avião, piano, máquinas, computadores\\
 间 & \dpy{jian1}   & quartos, espaços & quarto, dormitório, cozinha, sala de aula, casa, escola, empresa, capela\\
 件 & \dpy{jian4}   & assuntos, roupas (tops), móveis & roupa (top), camiseta, camisa, casaco, lençol, bagagem, presente, questão/matéria/coisa\\
 讲 & \dpy{jiang3}  & longos períodos de aula & \\
 节 & \dpy{jie2}    & seção (de bambu, período curto de aula) & \\
 届 & \dpy{jie4}    & sessões ou reuniões agendadas regularmente, grupos de anos em uma escola (por exemplo, Turma de 2025) & \\
 句 & \dpy{ju4}     & linhas de poemas, frases & \\
 棵 & \dpy{ke1}     & árvores ou outra flora semelhante & pinheiro, rosa\\
 颗 & \dpy{ke1}     & pequenos objetos, objetos que parecem pequenos & corações, pérolas, dentes, diamantes, sementes, estrelas distantes, planetas distantes\\
 课 & \dpy{ke4}     & lições em um texto & \\
 口 & \dpy{kou3}    & população de aldeias (número inferior a 100), familiares, poços, bocados de comida & \\
 块 & \dpy{kuai4}   & pedaço de forma irregular & terra, pedra, tofu, sabonete, pedaço/fatia de bolo, pão (não fatias), melancia, carne, queijo, pizza, chiclete, toalha de mesa, relógio de pulso, bloco de incenso\\
 类 & \dpy{lei4}    & objetos do mesmo tipo ou natureza & \\
 粒 & \dpy{li4}     & grão & um (único) amendoim, uva, arroz cru, semenete, doce, bala, chocolate\\
 辆 & \dpy{liang4}  & veículos com rodas (não trens) & automóvel, bicicleta, carro\\
 列 & \dpy{lie4}    & trens & \\
 轮 & \dpy{lun2}    & lua & \\
 枚 & \dpy{mei2}    & medalhas, pequenas coisas planas como selos, cascas de banana, anéis, distintivos, foguetes, mísseis & \\
 门 & \dpy{men2}    & objetos pertencentes a acadêmicos (cursos, disciplinas, etc.); artilharia (canhão) & \\
 面 & \dpy{mian4}   & objetos planos e lisos & espelho, bandeira, parede com telhado, escudo\\
 名 & \dpy{ming4}   & pessoas de alto escalão (médicos, advogados, políticos, realeza, etc.), membros; em linguagem formal pode ser utilizado para qualquer pessoa não necessariamente de alto escalão & \\
 排 & \dpy{pai2}    & linhas --- objetos agrupados em linhas & cadeiras, assentos, mesas, filas de pessoas\\
 盘 & \dpy{pan2}    & objetos planos ou bobinas & cassete de vídeo ou áudio, pratinho, bobina de incenso\\
 批 & \dpy{pi1}     & pessoas, bens, etc. & \\
 匹 & \dpy{pi3}     & cavalos e outras montarias; rolos/pedaços de panos & cavalo, lobo\\
 篇 & \dpy{pian1}   & escritos & papel, artigo, ensaio, relatório\\
 片 & \dpy{pian4}   & fatia - objetos finos, planos, às vezes irregulares & cartão, lábio, nuvem, praia, chiclete, fatia de pão, fatia de carne, biscoito, queijo, fatia de melancia, folha, pétala de flor, campo, lago, pílula (comprimido de remédio), DVD\\
 瓶 & \dpy{ping2}   & garrafa & álcool, água, óleo, cerveja, bebida, vinho, refrigerante, leite, shampoo\\
 期 & \dpy{qi1}     & revistas & \\
 群 & \dpy{qun2}    & grupo (incl. pessoas), rebanho, multidão, enxame, etc. & pessoas, rebanho de gado, bando de pássaros, bando de cães, enxame de mosquitos, colônia de abelhas/formigas\\
 任 & \dpy{ren4}    & mandato (presidente, senador, deputado, etc.) & \\
 扇 & \dpy{shan4}   & coisas que abrem e fecham com dobradiças & janela, porta\\
 首 & \dpy{shou1}   & coisas com versos & canção, poema \\
 束 & \dpy{shu4}    & cachos & flores, uvas \\
 双 & \dpy{shuang1} & par de objetos que naturalmente vêm em pares & pauzinhos, sapatos, luvas, olhos\\
 艘 & \dpy{sou1}    & navios & \\
 所 & \dpy{suo3}    & pequenos edifícios, instituições & universidade, casa independente\\
 台 & \dpy{tai2}    & objetos pesados, especialmente máquinas; apresentações & TV, computador, piano, aparelho, avião, trem, carro, elevador; apresentação de teatro, jogo\\
 躺 & \dpy{tang2}   & períodos de aulas, suítes de imóveis & aulas, lições, leituras\\
 趟 & \dpy{tang4}   & viagens, serviços de transportes programados & \\
 套 & \dpy{tao4}    & conjuntos & livros, revistas, colecionáveis, roupas, casas/apartamentos com vários cômodos, suítes, selos, móveis, quartos, ternos\\
 题 & \dpy{ti2}     & classificador de perguntas & \\
 条 & \dpy{tiao2}   & objetos longos, estreitos e flexíveis & peixe, cobra, dragão, minhoca, cachorro, cachecol, estrada, fita, rio, raiz, caule, corda, edredom, toalha, fio dental, calças, gravata, saia, sofá/banco, pessoa heróica, barco pequeno\\
 帖 & \dpy{tie4}    & bandagens adesivas & \\
 通 & \dpy{tong1}   & conversa, palestra & \\
 桶 & \dpy{tong3}   & jarro, balde, barril & jarro de leite, barril de óleo\\
 头 & \dpy{tou2}    & cabeça de gado & porco, vaca, bois, iaques, ovelhas, burros, mulas, leopardos, dinossauros\\
 团 & \dpy{tuan2}   & bola --- objetos redondos e enrolados & bola de lã, etc. \\
 碗 & \dpy{wan3}    & tigela & de arroz, de macarrão, de sopa\\
 位 & \dpy{wei4}    & classificador educado e respeitoso para pessoas & professor, cliente, colega\\
 项 & \dpy{xiang4}  & projetos & \\
 些 & \dpy{xie1}    & alguns & somente com 一,这,那,哪\\
 样 & \dpy{yang4}   & itens gerais de diferentes atributos & \\
 页 & \dpy{ye4}     & página & \\
 则 & \dpy{ze2}     & diário, registro do dia & \\
 扎 & \dpy{zha1}    & jarra, caneca --- usado em cantonês no lugar de 束 \dpy{shu4} (por exemplo, um pacote de flores) & bebidas como cerveja, refrigerante, suco, etc. (pint/jar: empréstimo linguístico do inglês, pode ser considerado informal ou gíria)\\
 盏 & \dpy{zhan3}   & luminárias (geralmente lâmpadas), bule de chá, etc. & \\
 站 & \dpy{zhan4}   & paradas (de ônibus ou trens) & \\
 张 & \dpy{zhang1}  & folha --- objetos planos ou de papel & papel, mesa, cama, cartão, sofá, CD/DVD, guardanapo, fotografia, ingresso, pintura, constelação, rosto, boca, arco\\
 阵 & \dpy{zhen4}   & rajada, estouro --- eventos com duração curtas & tempestades com raios, rajadas de vento, ocorrência de chuva\\
 支 & \dpy{zhi1}    & objetos bastante longos, semelhantes a bastões & caneta, lápis, pauzinho, canudo, bambu, rosa, rifle, flecha, lança, projétil de artilharia, míssil, cantigas\\
 只 & \dpy{zhi1}    & um de um par; animais & mão, dedo, olho, pé, cabeça, meia, luva, sapato, brinco, óculos; pássaro, frango, gato, tigre, cachorro, macaco, elefante, ovelha, rato, borboleta, rã, inseto\\
 种 & \dpy{zhong3}  & tipos & de coisas, de livros, de pessoas\\
 株 & \dpy{zhu1}    & árvores/plantas menores & arbusto, planta de arroz, planta de trigo\\
 幢 & \dpy{zhuang2} & edifício de vários andares & \\
 组 & \dpy{zu3}     & conjuntos, linhas, séries, baterias (militares) & \\
 座 & \dpy{zuo4}    & montanha, edifício & montanha, colina, estrutura, grande edifício, cidade, ponte, vila, arranha-céu, torre, templo, bloco de apartamentos\\
\end{longtblr}

%%%%% EOF %%%%%


\clearpage
\pagestyle{plain}
\chapter{Classificadores Verbais}
\DefTblrTemplate{caption}{default}{}
\DefTblrTemplate{capcont}{default}{ \UseTblrTemplate{conthead-text}{default} }
\DefTblrTemplate{contfoot-text}{default}{Continua na próxima página.}
\DefTblrTemplate{conthead-text}{default}{(Continuação)}
\DefTblrTemplate{firsthead}{default}{ \UseTblrTemplate{caption}{default} }
\DefTblrTemplate{middlehead,lasthead}{default}{ \UseTblrTemplate{conthead}{default} }
\DefTblrTemplate{firstfoot,middlefoot}{default}{ \UseTblrTemplate{contfoot}{default} }
\DefTblrTemplate{lastfoot}{default}{ \UseTblrTemplate{note}{default} \UseTblrTemplate{remark}{default} }

\begin{longtblr}
{
  colspec = {|c|c|X|X|}, hlines,
  width = 1\linewidth,
  rowhead = 1, rowfoot = 0,
  row{1} = {font=\bfseries, fg=white, bg=black},
}
\textbf{Hanzi} & \textbf{Pinyin} & \textbf{Descrição}\\
    遍 & \dpy{bian4}  & o número de vezes que uma ação foi concluída \\
    场 & \dpy{chang3} & a duração de um evento ocorrendo dentro de outro evento\\
    次 & \dpy{ci4}    & vezes (ao contrário de 遍, 次 refere-se ao número de vezes, independente de ter sido concluído ou não)\\
    顿 & \dpy{dun4}   & ações sem repetição\\
    回 & \dpy{hui2}   & ocorrências (usado coloquialmente)\\
    声 & \dpy{sheng1} & gritos, expressões\\
    趟 & \dpy{tang4}  & viagens, visitas\\
    下 & \dpy{xia4}   & ações breves e frequentemente repentinas, muito mais comum em cantonês do que em dialetos do norte\\
\end{longtblr}


\clearpage
\pagestyle{plain}
\chapter{Verbos Direcionais}
%%%%%%%%%%%%%%%%%%%%%%%%%%%%%%%%%%%%%%%%%%%%%%%%%%%%%%%%%%%%%%%%%%%%%%%%%%%%%%%
%%%%%%%%%%%%%%%%%%%%%%%%%%%%%%%%%%%%%%%%%%%%%%%%%%%%%%%%%%%%%%%%%%%%%%%%%%%%%%%
%%%%%                                                                     %%%%%
%%%%% verbos_direcionais.tex:                                             %%%%%
%%%%% Tabela com os verbos direcionais chineses.                          %%%%%
%%%%%                                                                     %%%%%
%%%%%%%%%%%%%%%%%%%%%%%%%%%%%%%%%%%%%%%%%%%%%%%%%%%%%%%%%%%%%%%%%%%%%%%%%%%%%%%
%%%%%%%%%%%%%%%%%%%%%%%%%%%%%%%%%%%%%%%%%%%%%%%%%%%%%%%%%%%%%%%%%%%%%%%%%%%%%%%

%%% Ajustes para a tabela
\DefTblrTemplate{caption}{default}{}
\DefTblrTemplate{capcont}{default}{ \UseTblrTemplate{conthead-text}{default} }
\DefTblrTemplate{contfoot-text}{default}{Continua na próxima página.}
\DefTblrTemplate{conthead-text}{default}{(Continuação)}
\DefTblrTemplate{firsthead}{default}{ \UseTblrTemplate{caption}{default} }
\DefTblrTemplate{middlehead,lasthead}{default}{ \UseTblrTemplate{conthead}{default} }
\DefTblrTemplate{firstfoot,middlefoot}{default}{ \UseTblrTemplate{contfoot}{default} }
\DefTblrTemplate{lastfoot}{default}{ \UseTblrTemplate{note}{default} \UseTblrTemplate{remark}{default} }

%%% Tabela
\begin{longtblr}
{
  colspec = {cccccccc},
  width = 1\linewidth,
  hlines = {white},
  vlines = {white},
  rowhead = 1, rowfoot = 0,
  row{1} = {font=\bfseries, bg=gray8, fg=black},
  column{1} = {font=\bfseries, bg=gray8, fg=black},
  cell{1}{1} = {bg=white},
  cell{2-Z}{2-Z} = {bg=gray9},
  cell{3}{8} = {bg=white},
}
 & {上\\ \normalsize descer} & {下\\ \normalsize subir} & {进\\ \normalsize entrar} & {出\\ \normalsize sair} & {回\\ \normalsize retornar} & {过\\ \normalsize atravessar} & {起\\ \normalsize levantar} \\
{来\\ \normalsize vir}  &  上来 &  下来 &  进来 &  出来 &  回来 &  过来 &  起来 \\
{去\\ \normalsize ir }  &  上去 &  下去 &  进去 &  出去 &  回去 &  过去 &  \\ 
\end{longtblr}

%%%%% EOF %%%%%


\clearpage
\pagestyle{plain}
\chapter{Locativos}
%%%%%%%%%%%%%%%%%%%%%%%%%%%%%%%%%%%%%%%%%%%%%%%%%%%%%%%%%%%%%%%%%%%%%%%%%%%%%%%
%%%%%%%%%%%%%%%%%%%%%%%%%%%%%%%%%%%%%%%%%%%%%%%%%%%%%%%%%%%%%%%%%%%%%%%%%%%%%%%
%%%%%                                                                     %%%%%
%%%%% locativos.tex:                                                      %%%%%
%%%%% Tabela com os locativos chineses                                    %%%%%
%%%%%                                                                     %%%%%
%%%%%%%%%%%%%%%%%%%%%%%%%%%%%%%%%%%%%%%%%%%%%%%%%%%%%%%%%%%%%%%%%%%%%%%%%%%%%%%
%%%%%%%%%%%%%%%%%%%%%%%%%%%%%%%%%%%%%%%%%%%%%%%%%%%%%%%%%%%%%%%%%%%%%%%%%%%%%%%

%%% Ajustes para a tabela
\DefTblrTemplate{caption}{default}{}
\DefTblrTemplate{capcont}{default}{ \UseTblrTemplate{conthead-text}{default} }
\DefTblrTemplate{contfoot-text}{default}{Continua na próxima página.}
\DefTblrTemplate{conthead-text}{default}{(Continuação)}
\DefTblrTemplate{firsthead}{default}{ \UseTblrTemplate{caption}{default} }
\DefTblrTemplate{middlehead,lasthead}{default}{ \UseTblrTemplate{conthead}{default} }
\DefTblrTemplate{firstfoot,middlefoot}{default}{ \UseTblrTemplate{contfoot}{default} }
\DefTblrTemplate{lastfoot}{default}{ \UseTblrTemplate{note}{default} \UseTblrTemplate{remark}{default} }

%%% Tabela
\begin{longtblr}
{
 colspec = {cccccc},
 width = 1\linewidth,
 hlines = {white},
 vlines = {white},
 rowhead = 1, rowfoot = 0,
 row{1} = {font=\bfseries, bg=gray8, fg=black},
 column{1} = {font=\bfseries, bg=gray8, fg=black},
 cell{1}{1} = {bg=white},
 cell{2-Z}{2-Z} = {bg=gray9},
 cell{6}{5-6} = {bg=white},
 cell{7}{2-4} = {bg=white},
 cell{9}{2-5} = {bg=white},
 cell{10}{3-6} = {bg=white},
 cell{11}{2-5} = {bg=white},
 cell{12}{4-6} = {bg=white},
 cell{13}{4-6} = {bg=white},
}
                                           & {边\\   \normalsize\dpy{bian1}}        & {面\\   \normalsize\dpy{mian4}}        & {头\\   \normalsize\dpy{tou5}}        & {以\\   \normalsize\dpy{yi3}}        & {之\\   \normalsize\dpy{zhi1}}        \\
{上\\ \normalsize\dpy{shang4}\\ sobre}     & {上边\\ \normalsize\dpy{shang4 bian1}} & {上面\\ \normalsize\dpy{shang4 mian4}} & {上头\\ \normalsize\dpy{shang4 tou5}} & {以上\\ \normalsize\dpy{yi3 shang4}} & {之上\\ \normalsize\dpy{zhi1 shang4}} \\
{下\\ \normalsize\dpy{xia4}\\ sob}         & {下边\\ \normalsize\dpy{xia4 bian1}}   & {下面\\ \normalsize\dpy{xia4 mian4}}   & {下头\\ \normalsize\dpy{xia4 tou5}}   & {以下\\ \normalsize\dpy{yi3 xia4}}   & {之下\\ \normalsize\dpy{zhi1 xia4}}   \\
{前\\ \normalsize\dpy{qian2}\\ na frente}  & {前边\\ \normalsize\dpy{qian2 bian1}}  & {前面\\ \normalsize\dpy{qian2 mian4}}  & {前头\\ \normalsize\dpy{qian2 tou5}}  & {以前\\ \normalsize\dpy{yi3 qian2}}  & {之前\\ \normalsize\dpy{zhi1 qian2}}  \\
{后\\ \normalsize\dpy{hou4}\\ atrás}       & {后边\\ \normalsize\dpy{hou4 bian1}}   & {后面\\ \normalsize\dpy{hou4 mian4}}   & {后头\\ \normalsize\dpy{hou4 tou5}}   & {以后\\ \normalsize\dpy{yi3 hou4}}   & {之后\\ \normalsize\dpy{zhi1 hou4}}   \\
{里\\ \normalsize\dpy{li3}\\ dentro}       & {里边\\ \normalsize\dpy{li3 bian1}}    & {里面\\ \normalsize\dpy{li3 mian4}}    & {里头\\ \normalsize\dpy{li3 tou5}}    &                                      &                                       \\
{内\\ \normalsize\dpy{nei4}\\ no interior} &                                        &                                        &                                       & {以内\\ \normalsize\dpy{yi3 nei4}}   & {之内\\ \normalsize\dpy{zhi1 nei4}}   \\
{外\\ \normalsize\dpy{wai4}\\ no exterior} & {外边\\ \normalsize\dpy{wai4 bian1}}   & {外面\\ \normalsize\dpy{wai4 mian4}}   & {外头\\ \normalsize\dpy{wai4 tou5}}   & {以外\\ \normalsize\dpy{yi3 wai4}}   & {之外\\ \normalsize\dpy{zhi1 wai4}}   \\
{间\\ \normalsize\dpy{jian1}\\ entre}      &                                        &                                        &                                       &                                      & {之间\\ \normalsize\dpy{zhi1 jian1}}  \\
{旁\\ \normalsize\dpy{pang2}\\ ao lado}    & {旁边\\ \normalsize\dpy{pang2 bian1}}  &                                        &                                       &                                      &                                       \\
{中\\ \normalsize\dpy{zhong1}\\ no meio}   &                                        &                                        &                                       &                                      & {之中\\ \normalsize\dpy{zhi1 zhong1}} \\
{左\\ \normalsize\dpy{zuo3}\\ à esquerda}  & {左边\\ \normalsize\dpy{zuo3 bian1}}   & {左面\\ \normalsize\dpy{zuo3 mian4}}   &                                       &                                      &                                       \\
{右\\ \normalsize\dpy{you4}\\ à direita}   & {右边\\ \normalsize\dpy{you4 bian1}}   & {右面\\ \normalsize\dpy{you4 mian4}}   &                                       &                                      &                                       \\
\pagebreak
{东\\ \normalsize\dpy{dong1}\\ no leste}   & {东边\\ \normalsize\dpy{dong1 bian1}}  & {东面\\ \normalsize\dpy{dong1 mian4}}  & {东头\\ \normalsize\dpy{dong1 tou5}}  & {以东\\ \normalsize\dpy{yi3 dong1}}  & {之东\\ \normalsize\dpy{zhi1 dong1}}  \\
{南\\ \normalsize\dpy{nan2}\\ no sul}      & {南边\\ \normalsize\dpy{nan2 bian1}}   & {南面\\ \normalsize\dpy{nan2 mian4}}   & {南头\\ \normalsize\dpy{nan2 tou5}}   & {以南\\ \normalsize\dpy{yi3 nan2}}   & {之南\\ \normalsize\dpy{zhi1 nan2}}   \\
{西\\ \normalsize\dpy{xi1}\\ no oeste}     & {西边\\ \normalsize\dpy{xi1 bian1}}    & {西面\\ \normalsize\dpy{xi1 mian4}}    & {西头\\ \normalsize\dpy{xi1 tou5}}    & {以西\\ \normalsize\dpy{yi3 xi1}}    & {之西\\ \normalsize\dpy{zhi1 xi1}}    \\
{北\\ \normalsize\dpy{bei3}\\ n norte}     & {北边\\ \normalsize\dpy{bei3 bian1}}   & {北面\\ \normalsize\dpy{bei3 mian4}}   & {北头\\ \normalsize\dpy{bei3 tou5}}   & {以北\\ \normalsize\dpy{yi3 bei3}}   & {之北\\ \normalsize\dpy{zhi1 bei3}}   \\
\end{longtblr}

%%%%% EOF %%%%%


\clearpage
\pagestyle{plain}
\chapter{Radicais Kangxi}
%%%%%%%%%%%%%%%%%%%%%%%%%%%%%%%%%%%%%%%%%%%%%%%%%%%%%%%%%%%%%%%%%%%%%%%%%%%%%%%
%%%%%%%%%%%%%%%%%%%%%%%%%%%%%%%%%%%%%%%%%%%%%%%%%%%%%%%%%%%%%%%%%%%%%%%%%%%%%%%
%%%%%                                                                     %%%%%
%%%%% radicais_kangxi.tex:                                                %%%%%
%%%%% Lista dos 214 radicais Kangxi utilizados nos caracteres chineses.   %%%%%
%%%%%                                                                     %%%%%
%%%%%%%%%%%%%%%%%%%%%%%%%%%%%%%%%%%%%%%%%%%%%%%%%%%%%%%%%%%%%%%%%%%%%%%%%%%%%%%
%%%%%%%%%%%%%%%%%%%%%%%%%%%%%%%%%%%%%%%%%%%%%%%%%%%%%%%%%%%%%%%%%%%%%%%%%%%%%%%

%%% Ajustes para a tabela
\DefTblrTemplate{caption}{default}{}
\DefTblrTemplate{capcont}{default}{ \UseTblrTemplate{conthead-text}{default} }
\DefTblrTemplate{contfoot-text}{default}{Continua na próxima página.}
\DefTblrTemplate{conthead-text}{default}{(Continuação)}
\DefTblrTemplate{firsthead}{default}{ \UseTblrTemplate{caption}{default} }
\DefTblrTemplate{middlehead,lasthead}{default}{ \UseTblrTemplate{conthead}{default} }
\DefTblrTemplate{firstfoot,middlefoot}{default}{ \UseTblrTemplate{contfoot}{default} }
\DefTblrTemplate{lastfoot}{default}{ \UseTblrTemplate{note}{default} \UseTblrTemplate{remark}{default} }

%%% Tabela
\begin{longtblr}
{
  colspec = {|r|XX|X|X|}, hlines,
  width = 1\linewidth,
  rowhead = 1, rowfoot = 0,
  row{1} = {font=\bfseries, fg=white, bg=black},
  row{2-Z} = {font=\normalfont},
}
\textbf{Nº} & \SetCell[c=2]{c}\textbf{Radical e Variantes} & 2-2 & \textbf{Tradução} & \textbf{Pinyin} \\
  1  & 一 &          & um                     & \dictpinyin{yi1}                  \\
  2  & 丨 &          & linha                  & \dictpinyin{shu4}                 \\
  3  & 丶 &          & ponto                  & \dictpinyin{dian3}                \\
  4  & 丿 & 乀,乁    & golpear                & \dictpinyin{pie3}                 \\
  5  & 乙 & 乚,乛    & segundo                & \dictpinyin{yi3}                  \\
  6  & 亅 &          & gancho                 & \dictpinyin{gou1}                 \\
  7  & 二 &          & dois                   & \dictpinyin{er4}                  \\
  8  & 亠 &          & membro                 & \dictpinyin{tou2}                 \\
  9  & 人 & 亻       & homem                  & \dictpinyin{ren2}                 \\
 10  & 儿 &          & pernas                 & \dictpinyin{er2}                  \\
 11  & 入 &          & entra                  & \dictpinyin{ru4}                  \\
 12  & 八 & 丷       & oito                   & \dictpinyin{ba1}                  \\
 13  & 冂 &          & caixa de baixo         & \dictpinyin{jiong3}               \\
 14  & 冖 &          & sobre                  & \dictpinyin{mi4}                  \\
 15  & 冫 &          & gelo                   & \dictpinyin{bing1}                \\
 16  & 几 &          & mesa                   & \dictpinyin{ji1},\dictpinyin{ji3} \\
 17  & 凵 &          & caixa aberta           & \dictpinyin{qu3}                  \\
 18  & 刀 & 刂,⺈    & faca                   & \dictpinyin{dao1}                 \\
 19  & 力 &          & poder                  & \dictpinyin{li4}                  \\
 20  & 勹 &          & embrulho               & \dictpinyin{bao1}                 \\
 21  & 匕 &          & colher                 & \dictpinyin{bi3}                  \\
 22  & 匚 &          & caixa aberta           & \dictpinyin{fang1}                \\
 23  & 匸 &          & esconderijo anexo      & \dictpinyin{xi3}                  \\
 24  & 十 &          & dez                    & \dictpinyin{shi2}                 \\
 25  & 卜 &          & místico                & \dictpinyin{bu3}                  \\
 26  & 卩 & 㔾       & foca                   & \dictpinyin{jie2}                 \\
 27  & 厂 &          & penhasco               & \dictpinyin{han4}                 \\
 28  & 厶 &          & privado                & \dictpinyin{si1}                  \\
 29  & 又 &          & novamente              & \dictpinyin{you4}                 \\
 30  & 口 &          & boca                   & \dictpinyin{kou3}                 \\
 31  & 囗 &          & lugar                  & \dictpinyin{wei2}                 \\
 32  & 土 &          & Terra                  & \dictpinyin{tu3}                  \\
 33  & 士 &          & guerreiro              & \dictpinyin{shi4}                 \\
 34  & 夂 &          & ir                     & \dictpinyin{zhi1}                 \\
 35  & 夊 &          & devagar                & \dictpinyin{sui1}                 \\
 36  & 夕 &          & tarde                  & \dictpinyin{xi1}                  \\
 37  & 大 &          & grande                 & \dictpinyin{da4}                  \\
 38  & 女 &          & mulher                 & \dictpinyin{nv3}                  \\
 39  & 子 &          & criança                & \dictpinyin{zi3}                  \\
 40  & 宀 &          & cobertura              & \dictpinyin{mian2}                \\
 41  & 寸 &          & polegada               & \dictpinyin{cun4}                 \\
 42  & 小 & ⺌,⺍    & pequeno                & \dictpinyin{xiao3}                \\
 43  & 尢 & 尣       & coxo                   & \dictpinyin{you2}                 \\
 44  & 尸 &          & cadáver                & \dictpinyin{shi1}                 \\
 45  & 屮 &          & brotar                 & \dictpinyin{che4}                 \\
 46  & 山 &          & montanha               & \dictpinyin{shan1}                \\
 47  & 川 & 巛,巜    & rio                    & \dictpinyin{chuan1}               \\
 48  & 工 &          & trabalho               & \dictpinyin{gong1}                \\
 49  & 己 &          & a si mesmo             & \dictpinyin{ji3}                  \\
 50  & 巾 &          & turbante               & \dictpinyin{jin1}                 \\
 51  & 干 &          & seco                   & \dictpinyin{gan1}                 \\
 52  & 幺 & 么       & fio curto              & \dictpinyin{yao1}                 \\
 53  & 广 &          & vasto                  & \dictpinyin{guang3}               \\
 54  & 廴 &          & passo longo            & \dictpinyin{yin3}                 \\
 55  & 廾 &          & duas mãos              & \dictpinyin{gong3}                \\
 56  & 弋 &          & atirar flecha          & \dictpinyin{yi4}                  \\
 57  & 弓 &          & arco                   & \dictpinyin{gong1}                \\
 58  & 彐 & 彑       & focinho                & \dictpinyin{ji4}                  \\
 59  & 彡 &          & cerdas                 & \dictpinyin{shan1}                \\
 60  & 彳 &          & dupla                  & \dictpinyin{chi4}                 \\
 61  & 心 & 忄,⺗    & coração                & \dictpinyin{xin1}                 \\
 62  & 戈 &          & lança                  & \dictpinyin{ge1}                  \\
 63  & 户 & 戶,戸    & por                    & \dictpinyin{hu4}                  \\
 64  & 手 & 扌,龵    & mão                    & \dictpinyin{shou3}                \\
 65  & 支 &          & ramo                   & \dictpinyin{zhi1}                 \\
 66  & 攴 & 攵       & batida                 & \dictpinyin{pu1}                  \\
 67  & 文 &          & escrita                & \dictpinyin{wen2}                 \\
 68  & 斗 &          & mergulhador            & \dictpinyin{dou3}                 \\
 69  & 斤 &          & eixo                   & \dictpinyin{jin1}                 \\
 70  & 方 &          & quadrado               & \dictpinyin{fang1}                \\
 71  & 无 & 旡       & não                    & \dictpinyin{wu2}                  \\
 72  & 日 &          & sol                    & \dictpinyin{ri4}                  \\
 73  & 曰 &          & dizer                  & \dictpinyin{yue1}                 \\
 74  & 月 &          & lua                    & \dictpinyin{yue4}                 \\
 75  & 木 &          & árvore                 & \dictpinyin{mu4}                  \\
 76  & 欠 &          & falta                  & \dictpinyin{qian4}                \\
 77  & 止 &          & parar                  & \dictpinyin{zhi3}                 \\
 78  & 歹 & 歺       & morte                  & \dictpinyin{dai3}                 \\
 79  & 殳 &          & arma                   & \dictpinyin{shu1}                 \\
 80  & 母 & 毋       & mãe                    & \dictpinyin{mu3}                  \\
 81  & 比 &          & comparar               & \dictpinyin{bi3}                  \\
 82  & 毛 &          & pelo                   & \dictpinyin{mao2}                 \\
 83  & 氏 &          & clã                    & \dictpinyin{shi4}                 \\
 84  & 气 &          & ar                     & \dictpinyin{qi4}                  \\
 85  & 水 & 氵,氺    & água                   & \dictpinyin{shui3}                \\
 86  & 火 & 灬       & fogo                   & \dictpinyin{huo3}                 \\
 87  & 爪 & 爫       & garra                  & \dictpinyin{zhao3}                \\
 88  & 父 &          & pai                    & \dictpinyin{fu4}                  \\
 89  & 爻 &          & linha                  & \dictpinyin{yao2}                 \\
 90  & 爿 & 丬       & meio tronco            & \dictpinyin{pan2}                 \\
 91  & 片 &          & fatia                  & \dictpinyin{pian4}                \\
 92  & 牙 &          & dente                  & \dictpinyin{ya2}                  \\
 93  & 牛 & 牜,⺧    & vaca                   & \dictpinyin{niu2}                 \\
 94  & 犬 & 犭       & cão                    & \dictpinyin{quan3}                \\
 95  & 玄 &          & profundo               & \dictpinyin{xuan2}                \\
 96  & 玉 & 王,玊    & jade                   & \dictpinyin{yu4}                  \\
 97  & 瓜 &          & melão                  & \dictpinyin{gua1}                 \\
 98  & 瓦 &          & telha                  & \dictpinyin{wa3}                  \\
 99  & 甘 &          & doce                   & \dictpinyin{gan1}                 \\
100  & 生 &          & vida                   & \dictpinyin{sheng1}               \\
101  & 用 &          & usar                   & \dictpinyin{yong4}                \\
102  & 田 &          & campo                  & \dictpinyin{tian2}                \\
103  & 疋 & ⺪       & roupa                  & \dictpinyin{pi3}                  \\
104  & 疒 &          & doença                 & \dictpinyin{ne4}                  \\
105  & 癶 &          & pegadas                & \dictpinyin{bo1}                  \\
106  & 白 &          & branco                 & \dictpinyin{bai2}                 \\
107  & 皮 &          & pele                   & \dictpinyin{pi2}                  \\
108  & 皿 &          & prato                  & \dictpinyin{min3}                 \\
109  & 目 & ⺫       & olho                   & \dictpinyin{mu4}                  \\
110  & 矛 &          & lança                  & \dictpinyin{mao2}                 \\
111  & 矢 &          & seta                   & \dictpinyin{shi3}                 \\
112  & 石 &          & pedra                  & \dictpinyin{shi2}                 \\
113  & 示 & 礻       & espírito               & \dictpinyin{shi4}                 \\
114  & 禸 &          & rastrear               & \dictpinyin{rou2}                 \\
115  & 禾 &          & grão                   & \dictpinyin{he2}                  \\
116  & 穴 &          & caverna                & \dictpinyin{xue2}                 \\
117  & 立 &          & ficar em pé            & \dictpinyin{li4}                  \\
118  & 竹 & ⺮       & bambu                  & \dictpinyin{zhu2}                 \\
119  & 米 &          & arroz                  & \dictpinyin{mi3}                  \\
120  & 糸 & 纟       & seda                   & \dictpinyin{mi4}                  \\
121  & 缶 &          & pote                   & \dictpinyin{fou3}                 \\
122  & 网 & 罒,罓,⺳ & rede                   & \dictpinyin{wang3}                \\
123  & 羊 & ⺶,⺷    & ovelha                 & \dictpinyin{yang2}                \\
124  & 羽 &          & pena                   & \dictpinyin{yu3}                  \\
125  & 老 & 耂       & velho                  & \dictpinyin{lao3}                 \\
126  & 而 &          & e                      & \dictpinyin{er2}                  \\
127  & 耒 &          & arado                  & \dictpinyin{lei3}                 \\
128  & 耳 &          & orelha                 & \dictpinyin{er3}                  \\
129  & 聿 & ⺺,⺻    & escova                 & \dictpinyin{yu4}                  \\
130  & 肉 & ⺼       & carne                  & \dictpinyin{rou4}                 \\
131  & 臣 &          & ministro               & \dictpinyin{chen2}                \\
132  & 自 &          & auto--                 & \dictpinyin{zi4}                  \\
133  & 至 &          & chegar                 & \dictpinyin{zhi4}                 \\
134  & 臼 &          & argamassa              & \dictpinyin{jiu4}                 \\
135  & 舌 &          & língua                 & \dictpinyin{she2}                 \\
136  & 舛 &          & opor                   & \dictpinyin{chuan3}               \\
137  & 舟 &          & barco                  & \dictpinyin{zhou1}                \\
138  & 艮 &          & pausa                  & \dictpinyin{gen3}                 \\
139  & 色 &          & cor                    & \dictpinyin{se4}                  \\
140  & 艸 & 艹       & grama                  & \dictpinyin{cao3}                 \\
141  & 虍 &          & tigre                  & \dictpinyin{hu1}                  \\
142  & 虫 &          & inseto                 & \dictpinyin{chong2}               \\
143  & 血 &          & sangue                 & \dictpinyin{xue4}                 \\
144  & 行 &          & andar                  & \dictpinyin{xing2}                \\
145  & 衣 & 衤       & roupa                  & \dictpinyin{yi1}                  \\
146  & 襾 & 西,覀    & oeste                  & \dictpinyin{ya4}                  \\
147  & 見 & 见       & ver                    & \dictpinyin{jian4}                \\
148  & 角 & ⻇       & chifre                 & \dictpinyin{jiao3}                \\
149  & 言 & 訁       & palavra                & \dictpinyin{yan2}                 \\
150  & 谷 &          & vale                   & \dictpinyin{gu3}                  \\
151  & 豆 &          & grão                   & \dictpinyin{dou4}                 \\
152  & 豕 &          & porco                  & \dictpinyin{shi3}                 \\
153  & 豸 &          & texugo                 & \dictpinyin{zhi4}                 \\
154  & 貝 & 贝       & concha                 & \dictpinyin{bei4}                 \\
155  & 赤 &          & vermelho               & \dictpinyin{chi4}                 \\
156  & 走 &          & andar                  & \dictpinyin{zou3}                 \\
157  & 足 & ⻊       & pé                     & \dictpinyin{zu2}                  \\
158  & 身 &          & corpo                  & \dictpinyin{shen1}                \\
159  & 車 & 车       & carro                  & \dictpinyin{che1}                 \\
160  & 辛 &          & amargo                 & \dictpinyin{xin1}                 \\
161  & 辰 &          & manhã                  & \dictpinyin{chen2}                \\
162  & 辵 & 辶,⻍,⻎ & caminhar               & \dictpinyin{chuo4}                \\
163  & 邑 & 阝       & cidade                 & \dictpinyin{yi4}                  \\
164  & 酉 &          & vinho                  & \dictpinyin{you3}                 \\
165  & 釆 &          & distinto               & \dictpinyin{bian4}                \\
166  & 里 &          & aldeia                 & \dictpinyin{li3}                  \\
167  & 金 & 釒       & ouro                   & \dictpinyin{jin1}                 \\
168  & 長 & 镸       & longo                  & \dictpinyin{zhang3}               \\
169  & 門 & 门       & portão                 & \dictpinyin{men2}                 \\
170  & 阜 & ⻖       & monte                  & \dictpinyin{fu4}                  \\
171  & 隶 &          & escravo                & \dictpinyin{li4}                  \\
172  & 隹 &          & pássaro de cauda curta & \dictpinyin{zhui1}                \\
173  & 雨 &          & chuva                  & \dictpinyin{yu3}                  \\
174  & 青 & 靑       & azul                   & \dictpinyin{qing1}                \\
175  & 非 &          & errado                 & \dictpinyin{fei1}                 \\
176  & 面 & 靣       & face                   & \dictpinyin{mian4}                \\
177  & 革 &          & couro                  & \dictpinyin{ge2}                  \\
178  & 韋 & 韦       & couro tingido          & \dictpinyin{wei2}                 \\
179  & 韭 &          & parecia                & \dictpinyin{jiu3}                 \\
180  & 音 &          & som                    & \dictpinyin{yin1}                 \\
181  & 頁 & 页       & folha                  & \dictpinyin{ye4}                  \\
182  & 風 & 风       & vento                  & \dictpinyin{feng1}                \\
183  & 飛 & 飞       & mosca                  & \dictpinyin{fei1}                 \\
184  & 食 & 饣,飠    & alimento               & \dictpinyin{shi2}                 \\
185  & 首 &          & cabeça                 & \dictpinyin{shou3}                \\
186  & 香 &          & perfume                & \dictpinyin{xiang1}               \\
187  & 馬 & 马       & cavalo                 & \dictpinyin{ma3}                  \\
188  & 骨 &          & osso                   & \dictpinyin{gu3}                  \\
189  & 高 & 髙       & alto                   & \dictpinyin{gao1}                 \\
190  & 髟 &          & cabelo                 & \dictpinyin{biao1}                \\
191  & 鬥 &          & luta                   & \dictpinyin{dou4}                 \\
192  & 鬯 &          & vinho                  & \dictpinyin{chang4}               \\
193  & 鬲 &          & separado               & \dictpinyin{ge2}                  \\
194  & 鬼 &          & fantasma               & \dictpinyin{gui3}                 \\
195  & 魚 & 鱼       & peixe                  & \dictpinyin{yu2}                  \\
196  & 鳥 & 鸟       & pássaro                & \dictpinyin{niao3}                \\
197  & 鹵 &          & sal                    & \dictpinyin{lu3}                  \\
198  & 鹿 &          & veado                  & \dictpinyin{lu4}                  \\
199  & 麥 & 麦       & trigo                  & \dictpinyin{mai4}                 \\
200  & 麻 &          & cânhamo                & \dictpinyin{ma2}                  \\
201  & 黃 &          & amarelo                & \dictpinyin{huang4}               \\
202  & 黍 &          & nação                  & \dictpinyin{shu3}                 \\
203  & 黑 &          & preto                  & \dictpinyin{hei1}                 \\
204  & 黹 &          & costura                & \dictpinyin{zhi3}                 \\
205  & 黽 & 黾       & rã                     & \dictpinyin{mian3}                \\
206  & 鼎 &          & tripé                  & \dictpinyin{ding3}                \\
207  & 鼓 &          & tambor                 & \dictpinyin{gu3}                  \\
208  & 鼠 & 鼡       & rato                   & \dictpinyin{shu3}                 \\
209  & 鼻 &          & nariz                  & \dictpinyin{bi2}                  \\
210  & 齊 & 齐,斉    & até                    & \dictpinyin{qi2}                  \\
211  & 齒 & 齿       & dente                  & \dictpinyin{chi3}                 \\
212  & 龍 & 龙       & dragão                 & \dictpinyin{long2}                \\
213  & 龜 & 龟       & tartaruga              & \dictpinyin{gui1}                 \\
214  & 龠 &          & flauta                 & \dictpinyin{yue4}                 \\
\end{longtblr}

%%%%% EOF %%%%%


\fi

\end{document}

%%%%% EOF %%%%
