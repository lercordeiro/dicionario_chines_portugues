\begin{EntryWithPhonetic}{说法}{shuo1 fa3}{9,8}{⾔、⽔}[HSK 5]
  \definition[种,个]{s.}{formulação; maneira de dizer uma coisa; formas de expressar opiniões | versão; argumento; declaração; opinião | explicação; acordo; palavras justas; razões ou fundamentos legítimos}
\end{EntryWithPhonetic}
