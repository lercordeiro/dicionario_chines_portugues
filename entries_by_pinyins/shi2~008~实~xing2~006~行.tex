\begin{EntryWithPhonetic}{实行}{shi2xing2}{8,6}{⼧、⾏}[HSK 3]
  \definition{v.}{praticar; implementar; executar; colocar em prática; realizar (programa, política, plano, etc.) por meio de ação}
\end{EntryWithPhonetic}

