\begin{EntryWithPhonetic}{一点儿}{yi4dian3r5}{1,9,2}{⼀、⽕、⼉}[HSK 1]
  \definition{adv.}{um pouco; uma pitada; uma gota; uma amostra; uma pequena quantidade; ({adj.} + (一)点儿, 一点儿 + {s.} ou 有 + (一)点儿 + {s.})}
\end{EntryWithPhonetic}

