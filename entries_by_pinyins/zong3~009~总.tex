\begin{EntryWithPhonetic}{总}{zong3}{9}{⼼}[HSK 3]
  \definition{adj.}{total; geral; global | responsável (liderança)}
  \definition{adv.}{sempre; invariavelmente | de qualquer forma; afinal; eventualmente; mais cedo ou mais tarde; no fim das contas | certamente; provavelmente; com certeza; expressa estimativa; suposição; equivalente a 大概}
  \definition{v.}{reunir; resumir; juntar; compilar}
  \seealsoref{大概}{da4gai4}
\end{EntryWithPhonetic}
