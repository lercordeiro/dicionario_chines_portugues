\begin{EntryWithPhonetic}{去年}{qu4nian2}{5,6}{⼛、⼲}[HSK 1]
  \definition{s.}{ano passado}
\end{EntryWithPhonetic}
