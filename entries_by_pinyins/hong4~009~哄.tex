\begin{EntryWithPhonetic}{哄}{hong4}{9}{⼝}
  \definition{s.}{comoção; tumulto}
  \seeref{hong1}
  \seeref{hong3}
\end{EntryWithPhonetic}
