
%%%%%%%%%%%%%%%%%%%%%%%%%%%%%%%%%%%%%%%%%
% XeTeX
%
% Vocabulário
% Autor: Luiz Eduardo Roncato Cordeiro
%
% Licença:
% CC BY-NC-SA 3.0 (http://creativecommons.org/licenses/by-nc-sa/3.0/)
%%%%%%%%%%%%%%%%%%%%%%%%%%%%%%%%%%%%%%%%%

%\documentclass[a4paper,12pt,twoside,openany,draft]{memoir}
\documentclass[a4paper,12pt,twoside,openany]{memoir}

\usepackage{xltxtra}
\usepackage[brazil]{babel}
\usepackage{xeCJK}
\usepackage{xpinyin}
\usepackage{fontspec}
\usepackage{xunicode}
\usepackage{xltxtra}
\usepackage[usenames,dvipsnames]{color}
\usepackage{multicol}
\usepackage{fancyhdr}
\usepackage{imakeidx}
\usepackage{ifthen}
\usepackage{tocloft}
\usepackage{xparse}
\usepackage{enumitem}

%\setlength{\cftbeforesecskip}{2pt}
%\renewcommand{\cftsecleader}{\cftdotfill{\cftdotsep}}

\setCJKmainfont{AR PL UKai CN}
\setCJKsansfont{AR PL UMing CN}

\makeindex[columns=3, title=Índice, intoc]

\setlength{\parindent}{0em}
\setlength{\parskip}{0.5em}
\setlength{\columnseprule}{0.5pt}

\xpinyinsetup{ratio={.5},vsep={1.2em},multiple={\color{Sepia}}}

% Headers & footers
\fancyhead[L]{\textsf{\rightmark}} % Top left header
\fancyhead[R]{\textsf{\leftmark}} % Top right header
\renewcommand{\headrulewidth}{1.4pt} % Rule under the header
\fancyfoot[C]{{\thepage}} % Bottom center footer
\renewcommand{\footrulewidth}{1.4pt} % Rule under the header
\pagestyle{fancy} % Use the custom headers and footers throughout the document

\setlength{\headheight}{16pt}
\addtolength{\topmargin}{-0.5pt}

\NewDocumentCommand\mylist{>{\SplitList{|}}m}
{
\setlength{\leftmargin}{0em}
\begin{enumerate}[nosep,left=0em,label=\arabic*.]
  \ProcessList{#1}{ \insertitem }
\end{enumerate}
}
\newcommand\insertitem[1]{\item #1}

\newenvironment{hanzi}[2][none]
  {\vspace{1ex}
  \markboth{#1«\pinyin{#2}»}{#1«\pinyin{#2}»}\index{#1«\pinyin{#2}»}
  \begin{minipage}[t][][t]{\linewidth}
  {\Huge \textbf{#1}} \\
  }
  {\end{minipage}
  }

\newcommand{\entry}[3]{%
«\pinyin{#1}» \if\relax\detokenize{#2}\relax\else\ \textit{(#2)}\fi%
{\small \mylist{#3}}%
\vspace{1ex}
}

\newcommand{\e}[1]{\textcolor{OliveGreen}{#1}}
\newcommand{\pc}[1]{P.C.:#1}

\makeatletter
\let\old@makechapterhead\@makechapterhead
% Taken from http://mirrors.ctan.org/macros/latex/unpacked/report.cls
\def\fake@makechapterhead#1{%
  \vspace*{50\p@}%
  {\parindent \z@ \raggedright \normalfont
    \ifnum \c@secnumdepth >\m@ne
        \huge\bfseries \strut%\@chapapp\space \thechapter
        \par\nobreak
        \vskip 20\p@
    \fi
    \interlinepenalty\@M
    \Huge \bfseries #1\par\nobreak
    \vskip 40\p@
  }
  \markboth{#1}{\thechapter}
}
\newcommand{\newchapterhead}{\let\@makechapterhead\fake@makechapterhead}
\newcommand{\restorechapterhead}{\let\@makechapterhead\old@makechapterhead}
\makeatother

%%%
%%% Documento começa aqui!
%%%

\begin{document}

\newchapterhead

\begin{titlingpage} % Suppresses displaying the page number on the title page and the subsequent page counts as page 1
	
	\raggedleft % Right align the title page
	
	\rule{1pt}{\textheight} % Vertical line
	\hspace{0.05\textwidth} % Whitespace between the vertical line and title page text
	\parbox[b]{0.75\textwidth}{ % Paragraph box for holding the title page text, adjust the width to move the title page left or right on the page
		
		{\Huge\bfseries 汉葡词典}\\[2\baselineskip] % Title
		{\large\textit{O Minúsculo Dicionário do Curso de Chinês}}\\[4\baselineskip] % Subtitle or further description
		{\Large\textsc{罗学凯}} % Author name, lower case for consistent small caps
		
		\vspace{0.5\textheight} % Whitespace between the title block and the publisher
		
		{\noindent \today}\\[\baselineskip] % Publisher and logo
	}

\end{titlingpage}

\tableofcontents

\newpage

\chapter{Termos Gramaticais Chineses}

\begin{tabular}{lll}
nome/substantivo       & \textbf{n.}        & 名词 \\
palavra de lugar       & \textbf{p.d.l.}    & 处所词 \\
palavra de localização & \textbf{p.l.}      & 方位词 \\
palavra de tempo       & \textbf{p.t.}      & 时间词 \\
verbo                  & \textbf{v.}        & 动词 \\
verbo direcional       & \textbf{v.d.}      & 趣向\hspace{1em}动词 \\
verbo optativo         & \textbf{v.o.}      & 能缘\hspace{1em}动词 \\
adjetivo               & \textbf{adj.}      & 形容词 \\
numeral                & \textbf{num.}      & 数词 \\
palavra classificadora & \textbf{p.c.}      & 两量词 \\
pronome                & \textbf{pron.}     & 代词 \\
interrogativo          & \textbf{interr.}   & 疑问词 \\
advérbio               & \textbf{adv.}      & 副词 \\
preposição             & \textbf{prep.}     & 介词 \\
conjunção              & \textbf{conj.}     & 连词 \\
partícula              & \textbf{part.}     & 助词 \\
sujeito                & \textbf{suj.}      & 主语 \\
objeto                 & \textbf{obj.}      & 宾语 \\
atributo               & \textbf{atrib.}    & 定语 \\
adjunto adverbial      & \textbf{a.adv.}    & 状语 \\
complemento            & \textbf{compl.}    & 补语 \\
verbo+complemento      & \textbf{v.+compl.} & 动宾式\hspace{1em}离合词 \\
\end{tabular}

\newpage

\chapter{汉葡词典}

%%%
%%% A
%%%
\section*{A}
\addcontentsline{toc}{section}{A}
\begin{multicols}{2}

\begin{hanzi}[啊]{a0}
\entry{a0}{part.}{interjeição de surpresa ou admiração|ah!|oh!}
\end{hanzi}

\begin{hanzi}[矮]{ai3}
\entry{ai3}{adj.}{baixo (estatura, dimensão, grau ou ranque)}
\end{hanzi}

\begin{hanzi}[爱]{ai4}
\entry{ai4}{n.}{amor; afeição}
\entry{ai4}{v.}{amar; gostar|ter afeição}
\end{hanzi}

\begin{hanzi}[爱好]{ai4hao4}
\entry{ai4hao4}{n.}{passatempo; interesse|\fbox{个}}
\entry{ai4hao4}{v.}{ter algo como hobby; ter prazer em fazer algo}
\end{hanzi}

\begin{hanzi}[爱人]{ai4ren0}
\entry{ai4ren0}{n.}{marido ou esposa|querido ou querida|\fbox{个}}
\end{hanzi}

\end{multicols}

%%%
%%% B
%%%
\section*{B}
\addcontentsline{toc}{section}{B}
\begin{multicols}{2}
\entry{八}{num.}{ba1}{oito}
\entry{巴西}{n.}{Ba1xi1}{Brasil}
\entry{爸爸}{n.}{ba4ba0}{papai; pai}
\entry{吧}{part.}{ba0}{partícula usada para sugestão ou suposição}
% \entry{白}{adj.}{branco; branca}
% \entry{白色}{n.}{cor branca}
% \entry{白菜}{n.}{repolho chinês}
\entry{百}{num.}{bai3}{centena; cem; cento}
\entry{北京}{n.}{bei3jing1}{Pequim; Capital da China}
% \entry{半}{num.}{meio; meia}
% \entry{包}{p.c.}{pacote; saco}
% \entry{比萨饼}{n.}{pizza}
% \entry{比赛}{n.}{competição}
% \entry{杯子}{n.}{copo; caneca; xícara; taça}
% \entry{\xpinyin{北}{Bei3}京}{n.}{Beijing(Pequim)}
% \entry{被子}{n.}{colcha}
% \entry{本}{p.c.}{p.c. para livros, dicionários, etc}
% \entry{笔}{n.}{caneta; lápis}
% \entry{比较}{adv.}{comparativamente}
% \entry{别}{adv.}{não}
% \entry{别的}{pron.}{outro; outra}
% \entry{\xpinyin{不}{bu2}错}{adj.}{bom}
% \entry{\xpinyin{不}{bu2}客气}{}{de nada; não há de que}
\entry{不}{adv.}{bu4}{não|«\pinyin{bu2}» antes de quarto tom|«\pinyin{bu0}» em expressões\\``verbo$+$不$+$verbo''}
% \entry{白天}{p.t.}{dia; de dia}
\end{multicols}

%%%
%%% C
%%%
\section*{C}
\addcontentsline{toc}{section}{C}
\begin{multicols}{2}

\begin{hanzi}[菜]{cai4}
\entry{cai4}{n.}{hortaliça; verdura; prato}
\end{hanzi}

\begin{hanzi}[菜单]{cai4dan1}
\entry{cai4dan1}{n.}{menu; ementa; cardápio}
\end{hanzi}

\begin{hanzi}[草地]{cao3di4}
\entry{cao3di4}{n.}{relva; pastagem}
\end{hanzi}

\begin{hanzi}[参观]{can1guan3}
\entry{can1guan3}{v.}{visitar}
\end{hanzi}

\begin{hanzi}[参加]{can1jia1}
\entry{can1jia1}{v.}{juntar; participar}
\end{hanzi}

\begin{hanzi}[餐厅]{can1ting1}
\entry{can1ting1}{n.}{cantina; sala de jantar}
\end{hanzi}

\begin{hanzi}[厕所]{ce4suo3}
\entry{ce4suo3}{n.}{sanitário; toilette}
\end{hanzi}

\begin{hanzi}[磁带]{ci2dai4}
\entry{ci2dai4}{n.}{cassete|\pc{盘}}
\end{hanzi}

\begin{hanzi}[磁盘]{ci2pan2}
\entry{ci2pan2}{n.}{disquete}
\end{hanzi}

\begin{hanzi}[词典]{ci2dian3}
\entry{ci2dian3}{n.}{dicionário|\pc{本}}
\end{hanzi}

\begin{hanzi}[次]{ci4}
\entry{ci4}{p.c.}{para frequência (número de vezes)}
\end{hanzi}

\begin{hanzi}[茶]{cha2}
\entry{cha2}{n.}{chá}
\end{hanzi}

\begin{hanzi}[差不多]{cha4bu0duo1}
\entry{cha4bu0duo1}{adj.}{mais ou menos}
\end{hanzi}

\begin{hanzi}[长]{chang2}
\entry{chang2}{adj.}{comprido; longo}
\end{hanzi}

\begin{hanzi}[长成]{chang2cheng2}
\entry{chang2cheng2}{n.}{Grande Muralha}
\end{hanzi}

\begin{hanzi}[常常]{chang2chang2}
\entry{chang2chang2}{adv.}{frequentemente}
\end{hanzi}

\begin{hanzi}[唱]{chang4}
\entry{chang4}{v.}{cantar}
\end{hanzi}

\begin{hanzi}[唱歌]{chang4ge1}
\entry{chang4ge1}{v.+compl.}{cantar}
\end{hanzi}

\begin{hanzi}[超市]{chao1shi4}
\entry{chao1shi4}{n.}{supermercado}
\end{hanzi}

\begin{hanzi}[炒]{chao3}
\entry{chao3}{v.}{saltear}
\end{hanzi}

\begin{hanzi}[车]{che1}
\entry{che1}{n.}{veículo; viatura}
\end{hanzi}

\begin{hanzi}[车牌]{che1pai2}
\entry{che1pai2}{n.}{matrícula; placa de carro}
\end{hanzi}

\begin{hanzi}[车站]{che1zhan4}
\entry{che1zhan4}{n.}{estação; paragem}
\end{hanzi}

\begin{hanzi}[衬衫]{chen4shan1}
\entry{chen4shan1}{n.}{camisa|\pc{件}}
\end{hanzi}

\begin{hanzi}[成都]{cheng2du1}
\entry{cheng2du1}{n.}{Chengdu}
\end{hanzi}

\begin{hanzi}[城市]{cheng2shi4}
\entry{cheng2shi4}{n.}{cidade}
\end{hanzi}

\begin{hanzi}[橙色]{cheng2se4}
\entry{cheng2se4}{n.}{cor de laranja}
\end{hanzi}

\begin{hanzi}[橙汁]{cheng2zhi1}
\entry{cheng2zhi1}{n.}{suco de laranja}
\end{hanzi}

\begin{hanzi}[惩罚]{cheng2fa2}
\entry{cheng2fa2}{v.}{punir; penalizar}
\end{hanzi}

\begin{hanzi}[惩处]{cheng2chu3}
\entry{cheng2chu3}{v.}{punir; penalizar}
\end{hanzi}

\begin{hanzi}[吃]{chi1}
\entry{chi1}{v.}{comer}
\end{hanzi}

\begin{hanzi}[迟到]{chi1dao4}
\entry{chi1dao4}{v.}{chegar atrasado; tardar}
\end{hanzi}

\begin{hanzi}[憧憬]{chong1jing3}
\entry{chong1jing3}{v.}{ansiar por; esperar por}
\end{hanzi}

\begin{hanzi}[宠物]{chong3wu4}
\entry{chong3wu4}{n.}{animal de estimação}
\end{hanzi}

\begin{hanzi}[酬劳]{chou2lao2}
\entry{chou2lao2}{n.}{}{recompensa}
\end{hanzi}

\begin{hanzi}[出]{chu1}
\entry{chu1}{v.d.}{sair}
\end{hanzi}

\begin{hanzi}[出版]{chu1ban3}
\entry{chu1ban3}{v.}{publicar; editar}
\end{hanzi}

\begin{hanzi}[出版社]{chu1ban3she4}
\entry{chu1ban3she4}{n.}{editora}
\end{hanzi}

\begin{hanzi}[出去]{chu1qu0}
\entry{chu1qu0}{v.d.}{sair; ir para fora}
\end{hanzi}

\begin{hanzi}[出口]{chu1kou3}
\entry{chu1kou3}{n.}{exportação}
\entry{chu1kou3}{v.}{exportar}
\end{hanzi}

\begin{hanzi}[出站]{chu1zhan4}
\entry{chu1zhan4}{n.}{saída da estação}
\end{hanzi}

\begin{hanzi}[出租汽车]{chu1zu1qi4che1}
\entry{chu1zu1qi4che1}{n.}{táxi|\pc{辆}}
\end{hanzi}

\begin{hanzi}[穿]{chuan1}
\entry{chuan1}{v.}{vestir}
\end{hanzi}

\begin{hanzi}[船]{chuan2}
\entry{chuan2}{v.}{barco; navio}
\end{hanzi}

\begin{hanzi}[传真]{chuan2zhen1}
\entry{chuan2zhen1}{n.}{fax; facsímile}
\end{hanzi}

\begin{hanzi}[床]{chuang2}
\entry{chuang2}{n.}{cama|\pc{张}}
\end{hanzi}

\begin{hanzi}[春天]{chun1tian1}
\entry{chun1tian1}{n.}{primavera}
\entry{chun1tian1}{p.t.}{Primavera}
\end{hanzi}

\begin{hanzi}[绰号]{chuo4hao4}
\entry{chuo4hao4}{n.}{apelido}
\end{hanzi}

\begin{hanzi}[聪明]{cong1ming2}
\entry{cong1ming2}{adj.}{inteligente; brilhante; esperto}
\end{hanzi}

\begin{hanzi}[聪慧]{cong1hui4}
\entry{cong1hui4}{adj.}{inteligente; brilhante}
\end{hanzi}

\begin{hanzi}[从]{cong2}
\entry{cong2}{prep.}{de; desde; a partir de}
\end{hanzi}

\begin{hanzi}[醋]{cu4}
\entry{cu4}{n.}{vinagre}
\end{hanzi}

\begin{hanzi}[错]{cuo4}
\entry{cuo4}{adj.}{errado; enganado}
\end{hanzi}

\end{multicols}

%%%
%%% D
%%%
\section*{D}
\addcontentsline{toc}{section}{D}
\begin{multicols}{2}

\begin{hanzi}[打]{da3}
\entry{da3}{v.}{jogar}
\end{hanzi}

\begin{hanzi}[打扮]{da3ban0}
\entry{da3ban0}{v.}{arranjar-se; enfeitar-se}
\end{hanzi}

\begin{hanzi}[打电话]{da3dian4hua4}
\entry{da3dian4hua4}{v.}{ligar; dar um telefonema}
\end{hanzi}

\begin{hanzi}[打工]{da3gong1}
\entry{da3gong1}{v.}{trabalhar temporariamente para alguém; trabalhar por conta de alguém}
\end{hanzi}

\begin{hanzi}[打搅]{da3jiao3}
\entry{da3jiao3}{v.}{perturbar; incomodar}
\end{hanzi}

\begin{hanzi}[打球]{da3qiu2}
\entry{da3qiu2}{v.}{jogar bola; jogar (futebol; basquetebol; handbol; etc)}
\end{hanzi}

\begin{hanzi}[打扰]{da3rao3}
\entry{da3rao3}{v.}{perturbar; incomodar}
\end{hanzi}

\begin{hanzi}[打算]{da3suan4}
\entry{da3suan4}{n.}{plano}
\entry{da3suan4}{v.}{pensar; planejar; pretender}
\end{hanzi}

\begin{hanzi}[打针]{da3zhen1}
\entry{da3zhen1}{v.+compl.}{dar injeção; levar injeção}
\end{hanzi}

\begin{hanzi}[大]{da4}
\entry{da4}{adj.}{da4}{grande}
\end{hanzi}

\begin{hanzi}[大夫]{da4fu0}
\entry{da4fu0}{n.}{da4fu0}{médico; doutor}
\end{hanzi}

\begin{hanzi}[大概]{da4gai4}
\entry{da4gai4}{adv.}{aproximadamente; por volta de}
\end{hanzi}

\begin{hanzi}[大海]{da4hai3}
\entry{da4hai3}{n.}{mar}
\end{hanzi}

\begin{hanzi}[大海]{da4jia1}
\entry{da4jia1}{pron.}{todos; todas}
\end{hanzi}

\begin{hanzi}[大蒜]{da4suan4}
\entry{da4suan4}{n.}{alho}
\end{hanzi}

\begin{hanzi}[大腿]{da4tui3}
\entry{da4tui3}{n.}{coxa}
\end{hanzi}

\begin{hanzi}[大学]{da4xue2}
\entry{da4xue2}{n.}{universidade}
\end{hanzi}

\begin{hanzi}[大洋洲]{Da4yang2zhou1}
\entry{Da4yang2zhou1}{n.}{Oceania}
\end{hanzi}

\begin{hanzi}[带]{dai4}
\entry{dai4}{v.}{trazer}
\end{hanzi}

\begin{hanzi}[戴]{dai4}
\entry{dai4}{v.}{usar/vestir (óculos, gravata, relógio de pulso, luvas); trazer}
\end{hanzi}

\begin{hanzi}[担心]{dan1xin1}
\entry{dan1xin1}{v.}{preocupar-se; estar preocupado}
\end{hanzi}

\begin{hanzi}[蛋糕]{dan4gao1}
\entry{dan4gao1}{n.}{bolo}
\end{hanzi}

\begin{hanzi}[但是]{dan4shi4}
\entry{dan4shi4}{conj.}{mas; contudo}
\end{hanzi}

\begin{hanzi}[当然]{dang1ran2}
\entry{dang1ran2}{adv.}{claro; certamente; com certeza}
\end{hanzi}

\begin{hanzi}[到]{dao4}
\entry{dao4}{prep.}{a; até; para}
\entry{dao4}{v.}{chegar}
\end{hanzi}

\begin{hanzi}[的]{de0}
\entry{de0}{part.}{partícula utilizada em possessivos|partícula utilizada entre adjetivos e substantivos|opcional se substantivo possui apenas um caracter}
\end{hanzi}

\begin{hanzi}[得]{de0}
\entry{de0}{part.}{}
\entry{de2}{v.}{ter que}
\entry{dei3}{v.}{haver de; ter de}
\end{hanzi}


\begin{hanzi}[德国]{De2guo2}
\entry{De2guo2}{n.}{De2guo2}{Alemanha}
\end{hanzi}

\begin{hanzi}[得]{de2}
\entry{de2}{v.}{ter que}
\entry{de0}{part.}{}
\entry{dei3}{v.}{haver de; ter de}
\end{hanzi}

\begin{hanzi}[得到]{de2dao4}
\entry{de2dao4}{v.}{de2dao4}{obter}
\end{hanzi}

\begin{hanzi}[得]{dei3}
\entry{dei3}{v.}{haver de; ter de}
\entry{de0}{part.}{}
\entry{de2}{v.}{ter que}
\end{hanzi}

\begin{hanzi}[登]{deng1}
\entry{deng1}{v.}{subir (montanha; cume)}
\end{hanzi}

\begin{hanzi}[等]{deng3}
\entry{deng3}{v.}{esperar}
\end{hanzi}

\begin{hanzi}[第]{di4}
\entry{di4}{num.}{prefixo para expressar números ordinais}
\end{hanzi}

\begin{hanzi}[弟弟]{di4di0}
\entry{di4di0}{n.}{irmão mais novo}
\end{hanzi}

\begin{hanzi}[弟妹]{di4mei4}
\entry{di4mei4}{n.}{esposa do irmão mais novo}
\end{hanzi}

\begin{hanzi}[地方]{di4fang1}
\entry{di4fang1}{n.}{lugar; local; sítio}
\end{hanzi}

\begin{hanzi}[地址]{di4zhi3}
\entry{di4zhi3}{n.}{endereço}
\end{hanzi}

\begin{hanzi}[地铁]{di4tie3}
\entry{di4tie3}{n.}{Metrô; metropolitano}
\end{hanzi}

\begin{hanzi}[地图]{di4tu2}
\entry{di4tu2}{n.}{mapa|\pc{张}}
\end{hanzi}

\begin{hanzi}[点]{dian3}
\entry{dian3}{p.c.}{hora}
\end{hanzi}

\begin{hanzi}[电话]{dian4hua4}
\entry{dian4hua4}{n.}{telefone}
\end{hanzi}

\begin{hanzi}[电脑]{dian4nao3}
\entry{dian4nao3}{n.}{computador|\pc{台}}
\end{hanzi}

\begin{hanzi}[电视]{dian4shi4}
\entry{dian4shi4}{n.}{televisão; TV; televisor}
\end{hanzi}

\begin{hanzi}[电梯]{dian4ti1}
\entry{dian4ti1}{n.}{elevador}
\end{hanzi}

\begin{hanzi}[电影]{dian4ying3}
\entry{dian4ying3}{n.}{cinema; filme}
\end{hanzi}

\begin{hanzi}[电子]{dian4zi3}
\entry{dian4zi3}{n.}{eletrônico; eletrônica}
\end{hanzi}

\begin{hanzi}[电子邮件]{dian4zi3you2jian4}
\entry{dian4zi3you2jian4}{n.}{correio eletrônico; \textit{e-mail}}
\end{hanzi}

\begin{hanzi}[叮嘱]{ding1zhu3}
\entry{ding1zhu3}{v.}{exortar; avisar; insistir de novo e de novo}
\end{hanzi}

\begin{hanzi}[东]{dong1}
\entry{dong1}{n.}{leste}
\end{hanzi}

\begin{hanzi}[东半球]{dong1ban4qiu2}
\entry{dong1ban4qiu2}{n.}{hemisfério leste}
\end{hanzi}

\begin{hanzi}[东边]{dong1bian0}
\entry{dong1bian0}{p.l.}{este; leste; oriente}
\end{hanzi}

\begin{hanzi}[东方]{dong1fang1}
\entry{dong1fang1}{n.}{Oriente}
\end{hanzi}

\begin{hanzi}[东天]{dong1tian1}
\entry{dong1tian1}{n.}{inverno}
\end{hanzi}

\begin{hanzi}[东面]{dong1mian0}
\entry{dong1mian0}{p.l.}{este; leste; oriente}
\end{hanzi}

\begin{hanzi}[东西]{dong1xi0}
\entry{dong1xi0}{n.}{coisa}
\end{hanzi}

\begin{hanzi}[动物]{dong4wu4}
\entry{dong4wu4}{n.}{animal}
\end{hanzi}

\begin{hanzi}[动物园]{dong4wu4yuan2}
\entry{dong4wu4yuan2}{n.}{jardim zoológico}
\end{hanzi}

\begin{hanzi}[都]{dou1}
\entry{dou1}{adv.}{todo; toda; todos; todas}
\end{hanzi}

\begin{hanzi}[读]{du2}
\entry{du2}{v.}{ler}
\end{hanzi}

\begin{hanzi}[度]{du4}
\entry{du4}{v.}{grau}
\end{hanzi}

\begin{hanzi}[肚子]{du4zi0}
\entry{du4zi0}{n.}{abdómen, barriga}
\end{hanzi}

\begin{hanzi}[对]{dui4}
\entry{dui4}{adj.}{correto; sim}
\entry{dui4}{prep.}{com; para; para com}
\end{hanzi}

\begin{hanzi}[对······感兴趣]{dui4 ...\  gan3shing4qu4}
\entry{dui4 ...\  gan3shing4qu4}{}{estar interessado em, ter interesse em, interessar-se por}
\end{hanzi}

\begin{hanzi}[对······熟悉]{dui4 ...\  shu2xi1}
\entry{dui4 ...\  shu2xi1}{}{estar familiarizado com...}
\end{hanzi}

\begin{hanzi}[对······有兴趣]{dui4 ...\  you3xing4qu4}
\entry{dui4 ...\  you3shing4qu4}{}{estar interessado em, ter interesse em, interessar-se por}
\end{hanzi}

\begin{hanzi}[对不起]{dui4bu0qi3}
\entry{dui4bu0qi3}{v.}{desculpar; pedir desculpa; perdão}
\end{hanzi}

\begin{hanzi}[对面]{dui4mian4}
\entry{dui4mian4}{p.l.}{lado oposto}
\end{hanzi}

\begin{hanzi}[短]{duan3}
\entry{duan3}{adj.}{curto}
\end{hanzi}

\begin{hanzi}[短裤]{duan3ku4}
\entry{duan3ku4}{n.}{calção; shorts}
\end{hanzi}

\begin{hanzi}[锻炼]{duan4lian4}
\entry{duan4lian4}{v.}{fazer exercício físico}
\end{hanzi}

\begin{hanzi}[多]{duo1}
\entry{duo1}{adj.}{muito; muita; muitos; muitas}
\end{hanzi}

\begin{hanzi}[多大]{duo1da4}
\entry{duo1da4}{interr.}{quantos anos; que idade}
\end{hanzi}

\begin{hanzi}[多少]{duo1shao0}
\entry{duo1shao0}{interr.}{quanto; quanta; quantos; quantas (para mais de 10 itens)}
\end{hanzi}

\end{multicols}

%%%%
%%% E
%%%
\section*{E}
\addcontentsline{toc}{section}{E}
\begin{multicols}{2}

\begin{hanzi}[俄罗斯]{É\ luo2si1}
\entry{É\ luo2si1}{n.}{
    Rússia
}
\end{hanzi}

\begin{hanzi}[恩赐]{en1ci4}
\entry{en1ci4}{n.}{
    favor; caridade
}
\entry{en1ci4}{v.}{
    conceder (favor, caridade);
}
\end{hanzi}

\begin{hanzi}[儿媳]{er2xi2}
\entry{er2xi2}{n.}{
    esposa do filho
}
\end{hanzi}

\begin{hanzi}[儿子]{er2zi0}
\entry{er2zi0}{n.}{
    filho
}
\end{hanzi}

\begin{hanzi}[耳朵]{er3duo0}
\entry{er3duo0}{n.}{
    orelha
}
\end{hanzi}

\begin{hanzi}[二]{er4}
\entry{er4}{num.}{
    2|
    dois
}
\end{hanzi}

\end{multicols}

%%%%
%%% F
%%%
\section*{F}
\addcontentsline{toc}{section}{F}
\begin{multicols}{2}

\begin{hanzi}[发]{fa1}
\entry{fa1}{v.}{
    enviar; mandar
}
\end{hanzi}

\begin{hanzi}[发烧]{fa1shao1}
\entry{fa1shao1}{v.}{
    ter febre
}
\end{hanzi}

\begin{hanzi}[罚]{fa2}
\entry{fa2}{v.}{
    castigar; punir
}
\end{hanzi}

\begin{hanzi}[罚款]{fa2kuan3}
\entry{fa2kuan3}{v.+compl.}{
    aplicar uma multa; multar
}
\end{hanzi}

\begin{hanzi}[法国]{Fa3guo2}
\entry{Fa3guo2}{n.}{
    França
}
\end{hanzi}

\begin{hanzi}[法语]{fa3yu3}
\entry{fa3yu3}{n.}{
    françês, língua francesa
}
\end{hanzi}

\begin{hanzi}[法文]{fa3wen2}
\entry{fa3wen2}{n.}{
    françês, língua francesa
}
\end{hanzi}

\begin{hanzi}[番茄]{fan4qie2}
\entry{fan4qie2}{n.}{
    tomate
}
\end{hanzi}

\begin{hanzi}[饭店]{fan4dian4}
\entry{fan4dian4}{n.}{
    restaurante
}
\end{hanzi}

\begin{hanzi}[方便]{fang1bian4}
\entry{fang1bian4}{adj.}{
    conveniente
}
\end{hanzi}

\begin{hanzi}[访问]{fang3wen4}
\entry{fang3wen4}{v.}{
    visitar
}
\end{hanzi}

\begin{hanzi}[放假]{fang4jia4}
\entry{fang4jia4}{v.}{
    ter férias
}
\end{hanzi}

\begin{hanzi}[房间]{fang4jian1}
\entry{fang4jian1}{n.}{
    quarto
}
\end{hanzi}

\begin{hanzi}[放心]{fang4xin1}
\entry{fang4xin1}{adj.}{
    descansado; despreocupado
}
\end{hanzi}

\begin{hanzi}[非]{fei1}
\entry{fei1}{adv.}{
    não; nem
}
\end{hanzi}

\begin{hanzi}[非常]{fei1chang2}
\entry{fei1chang2}{adv.}{
    muito
}
\end{hanzi}

\begin{hanzi}[非洲]{Fei1zhou1}
\entry{Fei1zhou1}{n.}{
    África
}
\end{hanzi}

\begin{hanzi}[飞机]{fei1ji1}
\entry{fei1ji1}{n.}{
    avião
}
\end{hanzi}

\begin{hanzi}[(飞)机票]{(fei1)\ ji1piao4}
\entry{(fei1)\ ji1piao4}{n.}{
    bilhete de avião
}
\end{hanzi}

\begin{hanzi}[分]{fen1}
\entry{fen1}{p.c.}{fen1}{
    minuto|
    centavo
}
\end{hanzi}

\begin{hanzi}[分公司]{fen1gong1si1}
\entry{fen1gong1si1}{n.}{fen1gong1si1}{
    sucursal; filial de companhia
}
\end{hanzi}

\begin{hanzi}[分钟]{fen1zhong1}
\entry{fen1zhong1}{n.}{fen1zhong1}{
    minuto
}
\end{hanzi}

\begin{hanzi}[份]{fen4}
\entry{fen4}{p.c.}{fen4}{
    dose
}
\end{hanzi}

\begin{hanzi}[分量]{fen4liang4}
\entry{fen4liang4}{p.}{fen4liang4}{
    peso; componente vetorial; física
}
\end{hanzi}

\begin{hanzi}[风]{feng1}
\entry{feng1}{n.}{
    vento
}
\end{hanzi}

\begin{hanzi}[枫叶]{feng1ye4}
\entry{feng1ye4}{n.}{
    folha de bordo (tipo de árvore/arbusto)
}
\end{hanzi}

\begin{hanzi}[副]{fu4}
\entry{fu4}{p.c.}{
    par; para óculos, luvas, etc
}
\end{hanzi}

\begin{hanzi}[父亲]{fu4qin0}
\entry{fu4quin0}{n.}{
    pai
}
\end{hanzi}

\begin{hanzi}[父母亲]{fu4mu3qin0}
\entry{fu4mu3quin0}{n.}{
    pais
}
\end{hanzi}

\begin{hanzi}[附近]{fu4jin4}
\entry{fu4jin4}{p.l.}{
    aqui perto; perto daqui
}
\end{hanzi}

\end{multicols}

%%%%
%%% G
%%%
\section*{G}
\addcontentsline{toc}{section}{G}
\begin{multicols}{2}

\begin{hanzi}[干杯]{gan1bei1}
\entry{gan1bei1}{v.+compl.}{
    brindar até a última gota; ``saúde!''
}
\end{hanzi}

\begin{hanzi}[干净]{gan1jing4}
\entry{gan1jing4}{adj.}{
    limpo
}
\end{hanzi}

\begin{hanzi}[赶快]{gan3kuai4}
\entry{gan3kuai4}{adv.}{
    rapidamente, imediatamente
}
\end{hanzi}

\begin{hanzi}[橄榄球]{gan3lan3qiu2}
\entry{gan3lan3qiu2}{n.}{
    rúgbi
}
\end{hanzi}

\begin{hanzi}[感冒]{gan3mao4}
\entry{gan3mao4}{v.}{
    ficar resfriado; estar com resfriado
}
\end{hanzi}

\begin{hanzi}[干]{gan4}
\entry{gan4}{v.}{
    fazer
}
\end{hanzi}

\begin{hanzi}[刚]{gang1}
\entry{gang1}{adv.}{
    acabar de
}
\end{hanzi}

\begin{hanzi}[高]{gao1}
\entry{gao1}{adj.}{
    alto
}
\end{hanzi}

\begin{hanzi}[高兴]{gao1xing4}
\entry{gao1xing4}{adj.}{
    feliz; alegre; contente
}
\end{hanzi}

\begin{hanzi}[告诉]{gao4su0}
\entry{gao4su0}{v.}{
    contar; dar a conhecer; dizer
}
\end{hanzi}

\begin{hanzi}[歌]{ge1}
\entry{ge1}{n.}{
    canção; canto
}
\end{hanzi}

\begin{hanzi}[哥哥]{ge1ge0}
\entry{ge1ge0}{n.}{
    irmão mais velho
}
\end{hanzi}

\begin{hanzi}[个]{ge4}
\entry{ge4}{p.c.}{
    de uso geral
}
\end{hanzi}

\begin{hanzi}[给]{gei3}
\entry{gei3}{pre.}{
    a; para
}
\entry{gei3}{v.}{
    dar
}
\end{hanzi}

\begin{hanzi}[给\ ······\ 打\ 电话]{gei3\ ...\ da3\ dian4hua4}
\entry{gei3\ ...\ da3\ dian4hua4}{}{
    telefonar para alguém
}
\end{hanzi}

\begin{hanzi}[跟]{gen1}
\entry{gen1}{prep.}{
    com
}
\end{hanzi}

\begin{hanzi}[根据]{gen1ju4}
\entry{gen1ju4}{prep.}{
    de acordo com
}
\end{hanzi}

\begin{hanzi}[更]{geng4}
\entry{geng4}{adv.}{
    mais
}
\end{hanzi}

\begin{hanzi}[工作]{gong1zuo4}
\entry{gong1zuo4}{n.}{
    trabalho
}
\entry{gong1zuo4}{v.}{
    trabalhar
}
\end{hanzi}

\begin{hanzi}[公共汽车]{gong1gong4qi4che1}
\entry{gong1gong4qi4che1}{n.}{
    ônibus
}
\end{hanzi}

\begin{hanzi}[公克]{gong1ke4}
\entry{gong1ke4}{n.}{
    trabalho escolar; trabalho de casa
}
\end{hanzi}

\begin{hanzi}[公司]{gong1si1}
\entry{gong1si1}{n.}{
    empresa; companhia
}
\end{hanzi}

\begin{hanzi}[公园]{gong1yuan2}
\entry{gong1yuan2}{n.}{
    parque
}
\end{hanzi}

\begin{hanzi}[狗]{gou3}
\entry{gou3}{n.}{
    cão; cachorro|
    \pc{条/只}
}
\end{hanzi}

\begin{hanzi}[故宫]{Gu4gong1}
\entry{Gu4gong1}{n.}{
    Palácio Imperial
}
\end{hanzi}

\begin{hanzi}[刮]{gua1}
\entry{gua1}{v.}{
    ventar, soprar (vento)
}
\end{hanzi}

\begin{hanzi}[刮风]{gua1feng1}
\entry{gua1feng1}{v.+compl.}{
    ventanejar; fazer vento
}
\end{hanzi}

\begin{hanzi}[拐]{guai3}
\entry{guai3}{v.}{
    virar; cortar
}
\end{hanzi}

\begin{hanzi}[光盘]{guang1pan2}
\entry{guang1pan2}{n.}{
    CD; disco compacto
}
\end{hanzi}

\begin{hanzi}[广东]{guang3dong1}
\entry{guang3dong1}{n.}{
    Guangdong
}
\end{hanzi}

\begin{hanzi}[规定]{gui1ding4}
\entry{gui1ding4}{n.}{
    regulamento
}
\entry{gui1ding4}{v.}{
    estipular
}
\end{hanzi}

\begin{hanzi}[贵]{gui4}
\entry{gui4}{adj.}{
    caro
}
\end{hanzi}

\begin{hanzi}[贵姓]{gui4xing4}
\entry{gui4xing4}{interr.}{
    seu sobrenome
}
\end{hanzi}

\begin{hanzi}[国]{guo2}
\entry{guo2}{n.}{
    país
}
\end{hanzi}

\begin{hanzi}[国家]{guo2jia1}
\entry{guo2jia1}{n.}{
    país
}
\end{hanzi}

\begin{hanzi}[果酱]{guo3jiang4}
\entry{guo3jiang4}{n.}{
    geléia; compota ou doce (de frutas)
}
\end{hanzi}

\begin{hanzi}[过]{guo4}
\entry{guo4}{v.}{
    passar
}
\entry{guo4}{part.}{
    passado
}
\end{hanzi}

\begin{hanzi}[过年]{guo4nian2}
\entry{guo4nian2}{v.}{
    festejar o Ano Novo Chinês
}
\end{hanzi}

\begin{hanzi}[过期]{guo4qi1}
\entry{guo4qi1}{v.+compl.}{
    exceder a data; passar a data
}
\end{hanzi}

\end{multicols}

%%%%
%%% H
%%%
\section*{H}
\addcontentsline{toc}{section}{H}
\begin{multicols}{2}

\entry{还}{adv.}{hai2}{ainda; também}
\entry{还是}{conj.}{hai2shi0}{ou|somente para frases interrogativas}

\entry{孩子}{n.}{hai2zi0}{criança; filho; filha}

\entry{海边}{n.}{hai3bian1}{praia}

\entry{汉国}{n.}{Han2guo2}{Coréia do Sul}
\entry{汉葡词典}{n.}{han4pu2ci2dian3}{dicionário chinês-português}
\entry{汉语}{n.}{Han4yu3}{chinês; língua chinesa; mandarim}

\entry{航班}{n.}{hang2ban1}{número de voo}

\entry{好}{adj.}{hao3}{bom; boa; bem}
\entry{好吃}{adj.}{hao3chi1}{delicioso; saboroso}
\entry{好汉}{n.}{hao3han4}{herói; homem verdadeiro}
\entry{好学}{adj.}{hao3xue2}{fácil de aprender}

\entry{号}{n.}{hao4}{dia; número}
\entry{号}{p.c.}{hao4}{dia}
\entry{号码}{n.}{hao4ma3}{número}

\entry{喝}{v.}{he1}{beber}

\entry{和}{conj.}{he2}{e|somente para palavras}

\entry{河}{n.}{he2}{rio}

\entry{盒}{p.c.}{he2}{caixa}

\entry{黑}{n.}{hei1}{preto; preta}
\entry{黑板}{n.}{hei1ban3}{quadro preto}
\entry{黑色}{n.}{hei1se4}{cor preta}

\entry{很}{adv.}{hen3}{muito; mui}

\entry{红}{adj.}{hong2}{vermelho; vermelha}
\entry{红色}{n.}{hong2se4}{cor vermelha}
\entry{红烧}{n.}{hong2shao1}{guisado em molho de soja}

\entry{后边}{p.l.}{hou4bian0}{atrás; detrás}
\entry{后面}{p.l.}{hou4mian0}{atrás; detrás}
\entry{后年}{p.t.}{hou4nian2}{daqui a dois anos}
\entry{后天}{p.t.}{hou4tian1}{depois de amanhã}

\entry{湖南}{n.}{Hu2nan2}{Hunan}

\entry{华盛顿}{n.}{Hua2sheng4dun4}{Washington}
\entry{华裔}{n.}{hua2yi4}{descendente de chinês}

\entry{话}{n.}{hua4}{palavra; fala}

\entry{欢迎}{v.}{huan1ying2}{dar as boas-vindas; ser bem-vindo}

\entry{换}{v.}{huan4}{mudar}

\entry{黄}{adj.}{huang2}{amarelo; amarela}
\entry{黄色}{n.}{huang2se4}{cor amarela}
\entry{黄油}{n.}{huang2you2}{manteiga}

\entry{花儿}{n.}{huar1}{flor}

\entry{回}{v.d.}{hui2}{regressar}
\entry{回来}{v.d.}{hui2lai0}{regressar; voltar; estar de volta}

\entry{会}{v.}{hui4}{saber}

\entry{火车}{n.}{huo3che1}{trem}

\entry{或者}{conj.}{huo4zhe3}{ou|usado em expressões afirmativas}

\end{multicols}

%%%%
%%% I
%%%
%\section*{I}
%\addcontentsline{toc}{section}{I}
%\begin{multicols}{2}
%\end{multicols}

%%%%
%%% J
%%%
\section*{J}
\addcontentsline{toc}{section}{J}
\begin{multicols}{2}

\begin{hanzi}[鸡]{ji1}
\entry{ji1}{n.}{galo; galinha}
\end{hanzi}

\begin{hanzi}[鸡蛋]{ji1dan4}
\entry{ji1dan4}{n.}{ovo de galinha}
\end{hanzi}

\begin{hanzi}[机场]{ji1chang3}
\entry{ji1chang3}{n.}{aeroporto}
\end{hanzi}

\begin{hanzi}[······极了]{...ji2le0}
\entry{...ji2le0}{}{muito; extremamente}
\end{hanzi}

\begin{hanzi}[几]{ji3}
\entry{ji3}{interr.}{quantos; quantas; alguns; algumas|para quantidades até 10 itens}
\end{hanzi}

\begin{hanzi}[季节]{ji4jie2}
\entry{ji4jie2}{n.}{estação (clima)}
\end{hanzi}

\begin{hanzi}[家]{jia1}
\entry{jia1}{n.}{família; casa}
\end{hanzi}

\begin{hanzi}[家里]{jia1li0}
\entry{jia1li0}{p.d.l.}{em casa}
\end{hanzi}

\begin{hanzi}[家乡]{jia1xiang1}
\entry{jia1xiang1}{n.}{terra natal}
\end{hanzi}

\begin{hanzi}[加拿大]{Jia1na2da4}
\entry{Jia1na2da4}{n.}{Canadá}
\end{hanzi}

\begin{hanzi}[检查]{jia3cha2}
\entry{jia3cha2}{v.}{examinar}
\end{hanzi}

\begin{hanzi}[肩膀]{jian1bang3}
\entry{jian1bang3}{n.}{ombro}
\end{hanzi}

\begin{hanzi}[件]{jian4}
\entry{jian4}{p.c.}{para roupas}
\end{hanzi}

\begin{hanzi}[见]{jian4}
\entry{jian4}{v.}{ver; encontrar alguém}
\end{hanzi}

\begin{hanzi}[见面]{jian4mian4}
\entry{jian4mian4}{v.}{encontar-se com alguém}
\end{hanzi}

\begin{hanzi}[建议]{jian4yi4}
\entry{jian4yi4}{n.}{sugestão}
\entry{jian4yi4}{v.}{sugerir}
\end{hanzi}

\begin{hanzi}[江西]{Jiang1xi1}
\entry{Jiang1xi1}{n.}{Jiangxi}
\end{hanzi}

\begin{hanzi}[胶卷]{jiao1juan3}
\entry{jiao1juan3}{n.}{filme; película; rolo}
\end{hanzi}

\begin{hanzi}[脚]{jiao3}
\entry{jiao3}{n.}{pé}
\end{hanzi}

\begin{hanzi}[角]{jiao3}
\entry{jiao3}{p.c.}{1 jiao = 10 centavos}
\end{hanzi}

\begin{hanzi}[饺子]{jiao3zi0}
\entry{jiao3zi0}{n.}{jiaozi; raviólis chineses; bolinho de massa}
\end{hanzi}

\begin{hanzi}[叫]{jiao4}
\entry{jiao4}{v.}{chamar-se; chamar}
\end{hanzi}

\begin{hanzi}[教]{jiao4}
\entry{jiao4}{v.}{ensinar}
\end{hanzi}

\begin{hanzi}[教练]{jiao4lian4}
\entry{jiao4lian4}{n.}{treinador}
\end{hanzi}

\begin{hanzi}[教授]{jiao4shou4}
\entry{jiao4shou4}{n.}{professor; professora}
\end{hanzi}

\begin{hanzi}[教室]{jiao4shi1}
\entry{jiao4shi1}{n.}{professor; professora; mestre; mestra}
\end{hanzi}

\begin{hanzi}[教室]{jiao4shi4}
\entry{jiao4shi4}{n.}{sala de aula}
\end{hanzi}

\begin{hanzi}[教学楼]{jiao4xue2lou2}
\entry{jiao4xue2lou2}{n.}{edifício de salas de aula}
\end{hanzi}

\begin{hanzi}[街]{jie1}
\entry{jie1}{n.}{rua}
\end{hanzi}

\begin{hanzi}[接]{jie1}
\entry{jie1}{v.}{ir buscar (alguém); ir ao encontro de (alguém); receber}
\end{hanzi}

\begin{hanzi}[节日]{jie2ri4}
\entry{jie2ri4}{n.}{festa}
\end{hanzi}

\begin{hanzi}[姐姐]{jie3jie0}
\entry{jie3jie0}{n.}{irmã mais velha}
\end{hanzi}

\begin{hanzi}[姐姐]{jie3fu0}
\entry{jie3fu0}{n.}{marido da irmã mais velha}
\end{hanzi}

\begin{hanzi}[借]{jie4} 
\entry{jie4}{v.}{pedir emprestado; emprestar}
\end{hanzi}

\begin{hanzi}[借书证]{jie4shu1zheng4} 
\entry{jie4shu1zheng4}{n.}{cartão de biblioteca|(literalmente: cartão para pedir emprestado livros)}
\end{hanzi}

\begin{hanzi}[介绍]{jie4shao4} 
\entry{jie4shao4}{n.}{apresentação}
\entry{jie4shao4}{v.}{apresentar}
\end{hanzi}

\begin{hanzi}[今年]{jin1nian2}
\entry{jin1nian2}{p.t.}{este ano}
\end{hanzi}

\begin{hanzi}[今天]{jin1tian1}
\entry{jin1tian1}{p.t.}{hoje}
\end{hanzi}

\begin{hanzi}[近]{jin4}
\entry{jin4}{adj.}{perto; próximo}
\end{hanzi}

\begin{hanzi}[进]{jin4}
\entry{jin4}{v.d.}{entrar}
\end{hanzi}

\begin{hanzi}[进出口]{jin4chu1kou3}
\entry{jin4chu1kou3}{n.}{importação e exportação}
\end{hanzi}

\begin{hanzi}[进口]{jin4kou3}
\entry{jin4kou3}{n.}{importação}
\end{hanzi}

\begin{hanzi}[进口]{jin4kou3}
\entry{jin4kou3}{v.}{importar}
\end{hanzi}

\begin{hanzi}[进来]{jin4lai0}
\entry{jin4lai0}{v.d.}{entrar}
\end{hanzi}

\begin{hanzi}[经常]{jing1chang2}
\entry{jing1chang2}{adv.}{muitas vezes}
\end{hanzi}

\begin{hanzi}[经济]{jing1ji4}
\entry{jing1ji4}{n.}{economia}
\end{hanzi}

\begin{hanzi}[酒]{jiu3}
\entry{jiu3}{n.}{bebida alcoólica; vinho; aguardente; licor}
\end{hanzi}

\begin{hanzi}[九]{jiu3}
\entry{jiu3}{num.}{nove}
\end{hanzi}

\begin{hanzi}[酒馆]{jiu3guan3}
\entry{jiu3guan3}{n.}{bar}
\end{hanzi}

\begin{hanzi}[就]{jiu4}
\entry{jiu4}{adv.}{exatamente; justamente}
\end{hanzi}

\begin{hanzi}[句]{ju4}
\entry{ju4}{p.c.}{para oração, frase}
\end{hanzi}

\begin{hanzi}[举行]{ju3xing2}
\entry{ju3xing2}{v.}{realizar; ter lugar}
\end{hanzi}

\begin{hanzi}[觉得]{jue2de2}
\entry{jue2de2}{v.}{achar; sentir}
\end{hanzi}

\end{multicols}

%%%%
%%% K
%%%
\section*{K}
\addcontentsline{toc}{section}{K}
\begin{multicols}{2}

\entry{咖啡}{n.}{ka1fei1}{café}
\entry{咖啡馆}{n.}{ka1fei1guan3}{cafeteria}

\entry{开}{v.}{kai1}{abrir; dirigir}
\entry{开车}{v.}{kai1che1}{conduzir; dirigir}

\entry{看}{v.}{kan4}{olhar; ver; assistir}
\entry{看见}{v.}{kan4jian4}{encontrar; enxergar; ver}

\entry{烤}{v.}{kao3}{assar}

\entry{科技}{n.}{Ke1ji4}{Ciência e Tecnologia}

\entry{颗}{p.c.}{ke3}{para grãos e coisas semelhantes}

\entry{咳嗽}{v.}{ke3sou0}{ter tosse; tussir}

\entry{可爱}{adj.}{ke3'ai4}{querido; querida; fofo; fofa}
\entry{可口可乐}{n.}{ke3kou3ke3le3}{coca-cola}
\entry{可能}{adj.}{ke3neng2}{possível}
\entry{可能}{adv.}{ke3neng2}{possivelmente; provavelmente}
\entry{可是}{conj.}{ke3shi4}{porém; contudo; mas}
\entry{可以}{v.o.}{ke3yi3}{poder}

\entry{刻}{n.}{ke4}{quarto (de hora)}

\entry{客气}{adj.}{ke4qi0}{cortês; delicado; educado}
\entry{客气}{v.}{ke4qi0}{fazer cerimônia}

\entry{课本}{n.}{ke4ben3}{livro do aluno; manual}

\entry{肯定}{adj.}{ken3ding4}{com certeza; certamente}

\entry{空儿}{n.}{kongr4}{tempo livre}

\entry{口}{p.c.}{kou3}{para membros da família}
\entry{口香糖}{n.}{kou3xiang1tang2}{goma de mascar; chiclete}

\entry{裤子}{n.}{ku4zi0}{calças|\fbox{条}}

\entry{块}{p.c.}{kuai4}{unidade de Reminbi}

\entry{快乐}{n.}{kuai4le2}{felicidade}

\end{multicols}

%%%%
%%% L
%%%
\section*{L}
\addcontentsline{toc}{section}{L}
\begin{multicols}{2}

\begin{hanzi}[拉拉队]{la1la1dui4}
\entry{la1la1dui4}{n.}{claque; torcida}
\end{hanzi}

\begin{hanzi}[辣]{la4}
\entry{la4}{adj.}{picante}
\end{hanzi}

\begin{hanzi}[来]{lai2}
\entry{lai2}{v.}{vir; trazer}
\end{hanzi}

\begin{hanzi}[蓝]{lan2}
\entry{lan2}{adj.}{azul}
\end{hanzi}

\begin{hanzi}[蓝色]{lan2se4}
\entry{lan2se4}{n.}{cor azul}
\end{hanzi}

\begin{hanzi}[篮球]{lan2qiu2}
\entry{lan2qiu2}{n.}{basquetebol}
\end{hanzi}

\begin{hanzi}[老板]{lao3ban3}
\entry{lao3ban3}{n.}{patrão; patroa}
\end{hanzi}

\begin{hanzi}[老人家]{lao3ren2jia0}
\entry{lao3ren2jia0}{n.}{senhor ancião; madame; senhora}
\end{hanzi}

\begin{hanzi}[老师]{lao3shi1}
\entry{lao3shi1}{n.}{professor; professora}
\end{hanzi}

\begin{hanzi}[了]{le0}
\entry{le0}{part.}{partícula para denotar mudança}
\end{hanzi}

\begin{hanzi}[冷]{leng3}
\entry{leng3}{adj.}{frio}
\end{hanzi}

\begin{hanzi}[累]{lei4}
\entry{lei4}{adj.}{cansado}
\end{hanzi}

\begin{hanzi}[离]{li2}
\entry{li2}{n.}{(ser longe) de ... até...}
\end{hanzi}

\begin{hanzi}[里]{li3}
\entry{li3}{p.l.}{em; dentro; interior}
\end{hanzi}

\begin{hanzi}[里斯本]{Li3si1ben3}
\entry{Li3si1ben3}{n.}{Lisboa}
\end{hanzi}

\begin{hanzi}[礼节]{li3jie2}
\entry{li3jie2}{n.}{cortesia; protocolo; cerimônia}
\end{hanzi}

\begin{hanzi}[礼物]{li3wu4}
\entry{li3wu4}{n.}{prenda; lembrança; presente|\pc{件}}
\end{hanzi}

\begin{hanzi}[厉害]{li4hai0}
\entry{li4hai0}{adj.}{severo; rigoroso; exigente}
\end{hanzi}

\begin{hanzi}[历史]{li4shi3}
\entry{li4shi3}{n.}{história}
\end{hanzi}

\begin{hanzi}[脸]{lian3}
\entry{lian3}{n.}{cara; rosto; face}
\end{hanzi}

\begin{hanzi}[凉快]{liang2kuai0}
\entry{liang2kuai0}{adj.}{legal}
\end{hanzi}

\begin{hanzi}[两]{liang3}
\entry{liang3}{num.}{dois; duas|sempre usado antes de p.c.}
\end{hanzi}

\begin{hanzi}[辆]{liang4}
\entry{liang4}{p.c.}{palavra classificadora para automóveis}
\end{hanzi}

\begin{hanzi}[邻居]{lin2ju1}
\entry{lin2ju1}{n.}{vizinho}
\end{hanzi}

\begin{hanzi}[零/\Circle]{ling2}
\entry{ling2}{num.}{0; zero}
\end{hanzi}

\begin{hanzi}[领导]{ling3dao3}
\entry{ling3dao3}{n.}{chefe; dirigente}
\end{hanzi}

\begin{hanzi}[流利]{liu2li4}
\entry{liu2li4}{adj.}{fluente}
\end{hanzi}

\begin{hanzi}[六]{liu4}
\entry{liu4}{num.}{6; seis}
\end{hanzi}

\begin{hanzi}[遛狗]{liu4gou3}
\entry{liu4gou3}{v.+compl.}{passear o cão}
\end{hanzi}

\begin{hanzi}[伦敦]{lun2dun1}
\entry{lun2dun1}{n.}{Londres}
\end{hanzi}

\begin{hanzi}[龙]{long2}
\entry{long2}{n.}{dragão}
\end{hanzi}

\begin{hanzi}[龙山]{Long2shan1}
\entry{Long2shan1}{n.}{Longshan}
\end{hanzi}

\begin{hanzi}[路]{lu4}
\entry{lu4}{n.}{caminho; via}
\end{hanzi}

\begin{hanzi}[路口]{lu4kou3}
\entry{lu4kou3}{n.}{cruzamento; interseção}
\end{hanzi}

\begin{hanzi}[录像带]{lu4xiang4dai4}
\entry{lu4xiang4dai4}{n.}{video-cassete|\pc{盘}}
\end{hanzi}

\begin{hanzi}[录像机]{lu4xiang4ji1}
\entry{lu4xiang4ji1}{n.}{gravador de vídeo|\pc{台}}
\end{hanzi}

\begin{hanzi}[录音机]{lu4yin1ji1}
\entry{lu4yin1ji1}{n.}{gravador|\pc{台}}
\end{hanzi}

\begin{hanzi}[旅行]{lv3xing2}
\entry{lv3xing2}{v.}{viajar}
\end{hanzi}

\begin{hanzi}[旅游]{lv3you2}
\entry{lv3you2}{v.}{viajar}
\end{hanzi}

\begin{hanzi}[绿]{lv4}
\entry{lv4}{adj.}{verde}
\end{hanzi}

\begin{hanzi}[绿色]{lv4se4}
\entry{lv4se4}{n.}{lv4se4}{cor verde}
\end{hanzi}

\end{multicols}

%%%%
%%% M
%%%
\section*{M}
\addcontentsline{toc}{section}{M}
\begin{multicols}{2}

\begin{hanzi}[吗]{ma0}
\entry{ma0}{part.}{ma0}{partícula interrogativa|perguntas ``sim-não''}
\end{hanzi}

\begin{hanzi}[妈妈]{ma1ma0}
\entry{ma1ma0}{n.}{ma1ma0}{mamãe; mãe}
\end{hanzi}

\begin{hanzi}[麻烦]{ma2fan0}
\entry{ma2fan0}{adj.}{enfastidioso; maçante}
\entry{ma2fan0}{n.}{incômodo}
\entry{ma2fan0}{v.}{incomodar}
\end{hanzi}

\begin{hanzi}[麻辣豆腐]{ma2la4dou4fu0}
\entry{ma2la4dou4fu0}{n.}{tofú guisado em molho picante}
\end{hanzi}

\begin{hanzi}[马路]{ma3lu4}
\entry{ma3lu4}{n.}{rua|\pc{条}}
\end{hanzi}

\begin{hanzi}[马上]{ma3shang4}
\entry{ma3shang4}{adv.}{já; imediatamente}
\end{hanzi}

\begin{hanzi}[买]{mai3}
\entry{mai3}{v.}{comprar}
\end{hanzi}

\begin{hanzi}[买东西]{mai3dong1xi0}
\entry{mai3dong1xi0}{v.}{fazer compras}
\end{hanzi}

\begin{hanzi}[卖]{mai4}
\entry{mai4}{v.}{vender}
\end{hanzi}

\begin{hanzi}[猫]{mao1}
\entry{mao1}{n.}{gato|\pc{只}}
\end{hanzi}

\begin{hanzi}[毛]{mao2}
\entry{mao2}{p.c.}{1 mao = 10 centavos}
\end{hanzi}

\begin{hanzi}[忙]{mang1}
\entry{mang1}{adj.}{ocupado; ocupada}
\end{hanzi}

\begin{hanzi}[忙]{mei2mao0}
\entry{mei2mao0}{n.}{sobrancelha}
\end{hanzi}

\begin{hanzi}[没关系]{mei2guan1xi0}
\entry{mei2guan1xi0}{v.}{não ter problema; não ter importância; não fazer mal}
\end{hanzi}

\begin{hanzi}[没有]{mei2you3}
\entry{mei2you3}{v.}{não há; não tem}
\end{hanzi}

\begin{hanzi}[美国]{Mei2guo1}
\entry{Mei2guo1}{n.}{Estados Unidos da América}
\end{hanzi}

\begin{hanzi}[每]{mei3}
\entry{mei3}{pron.}{cada}
\end{hanzi}

\begin{hanzi}[每次]{mei3ci4}
\entry{mei3ci4}{adv.}{toda vez}
\end{hanzi}

\begin{hanzi}[每天]{mei3tian1}
\entry{mei3tian1}{adv.}{todo dia}
\end{hanzi}

\begin{hanzi}[美洲]{Mei3zhou1}
\entry{Mei3zhou1}{n.}{América}
\end{hanzi}

\begin{hanzi}[妹夫]{mei4fu0}
\entry{mei4fu0}{n.}{marido da irmã mais nova}
\end{hanzi}

\begin{hanzi}[妹妹]{mei4mei0}
\entry{mei4mei0}{n.}{irmã mais nova}
\end{hanzi}

\begin{hanzi}[门口]{men2kou3}
\entry{men2kou3}{p.d.l.}{entrada; porta}
\end{hanzi}

\begin{hanzi}[们]{men0}
\entry{men0}{sufixo}{sufixo para plural}
\end{hanzi}

\begin{hanzi}[面]{mian4}
\entry{mian4}{n.}{farinha; massa}
\end{hanzi}

\begin{hanzi}[面条]{mian4tiao2}
\entry{mian4tiao2}{n.}{massa; espaguete}
\end{hanzi}

\begin{hanzi}[面包]{mian4bao1}
\entry{mian4bao1}{n.}{pão|\pc{个/块/片}}
\end{hanzi}

\begin{hanzi}[名片]{ming2pian4}
\entry{ming2pian4}{n.}{cartão de visita}
\end{hanzi}

\begin{hanzi}[名字]{ming2zi0}
\entry{ming2zi0}{n.}{nome}
\end{hanzi}

\begin{hanzi}[明白]{ming2bai0}
\entry{ming2bai0}{adj.}{compreendido; percebido}
\entry{ming2bai0}{v.}{compreender; perceber}
\end{hanzi}

\begin{hanzi}[明天]{ming2tian1}
\entry{ming2tian1}{p.t.}{amanhã}
\end{hanzi}

\begin{hanzi}[明年]{ming2nian2}
\entry{ming2nian2}{n.}{próximo ano}
\end{hanzi}

\begin{hanzi}[墨镜]{mo4jing4}
\entry{mo4jing4}{n.}{óculos escuros}
\end{hanzi}

\begin{hanzi}[母亲]{mu3qin0}                                               
\entry{mu3quin0}{n.}{mãe}                                                   
\end{hanzi}                                                                     

\begin{hanzi}[米饭]{mv3fan4}
\entry{mv3fan4}{n.}{arroz cozido}
\end{hanzi}

\end{multicols}

%%%%
%%% N
%%%
\section*{N}
\addcontentsline{toc}{section}{N}
\begin{multicols}{2}

\entry{哪}{interr.}{na3}{que, qual}
\entry{哪儿}{interr.}{nar3}{onde}
\entry{哪国人}{}{na3guo2ren2}{de qual país?}
\entry{哪里}{interr.}{na3li0}{onde}
\entry{哪些}{interr.}{na3xie1}{quais}

\entry{那}{conj.}{na4}{nessa situação; nesse caso}
\entry{那}{pron.}{na4}{aquele; aquilo}
\entry{那儿}{pron.}{nar4}{lá; ali}
\entry{那里}{pron.}{na4li0}{lá; ali}
\entry{那些}{pron.}{na4xie1}{aqueles}

\entry{奶奶}{n.}{nai3nai0}{avó(paterna)}

\entry{男}{adj.}{nan2}{masculino}
\entry{男朋友}{n.}{nan2peng2you0}{namorada}
\entry{男孩儿}{n.}{nan2hair2}{menino; rapaz}

\entry{南边}{p.l.}{nan2bian0}{sul}
\entry{南面}{p.l.}{nan2mian0}{sul}

\entry{呢}{interr.}{ne0}{partícula interrogativa enfática}

\entry{能}{v.}{neng2}{poder}

\entry{你}{pron.}{ni3}{você (informal); tu}
\entry{你的}{pron.}{ni3de0}{seu; sua}
\entry{你们}{pron.}{ni3men0}{vocês (informal); vós}
\entry{你们的}{pron.}{ni3men0de0}{seus; suas}

\entry{年}{p.t.}{nian2}{ano}
\entry{年纪}{n.}{nian2ji4}{idade}
\entry{年货}{n.}{nian2huo4}{mercadorias de Ano Novo Chinês}

\entry{鸟儿}{n.}{niaor3}{pássaro|\fbox{只}}

\entry{您}{pron.}{nin2}{você (formal); tu}

\entry{牛}{n.}{niu2}{boi; vaca}
\entry{牛奶}{n.}{niu2nai3}{leite|\fbox{杯}}
\entry{牛肉}{n.}{niu2rou4}{carne de vaca}
\entry{牛仔裤}{n.}{niu2zai3ku4}{calça de ganga; jeans|\fbox{条}}

\entry{农村}{n.}{nong2cun1}{campo rural}

\entry{女}{adj.}{nv3}{feminino}
\entry{女朋友}{n.}{nv3peng2you0}{namorada}
\entry{女儿}{n.}{nv3'er2}{filha}
\entry{女孩儿}{n.}{nv3hair2}{menina; rapariga}
\entry{女儿}{n.}{nv3xu4}{marido da filha}

\end{multicols}

%%%%
%%% O
%%%
\section*{O}
\addcontentsline{toc}{section}{O}
\begin{multicols}{2}

\begin{hanzi}[欧洲]{Ou1zhou1}
\entry{Ou1zhou1}{n.}{
    Europa
}
\end{hanzi}

\end{multicols}

%%%%
%%% P
%%%
\section*{P}
\addcontentsline{toc}{section}{P}
\begin{multicols}{2}

\begin{hanzi}[爬]{pa2}
\entry{pa2}{v.}{
    escalar; trepar
}
\end{hanzi}

\begin{hanzi}[怕]{pa4}
\entry{pa4}{v.}{
    ter medo de
}
\end{hanzi}

\begin{hanzi}[拍照]{pai1zhao4}
\entry{pai1zhao4}{v.+compl.}{
    tirar fotografia
}
\end{hanzi}

\begin{hanzi}[排球]{pai2qiu2}
\entry{pai2qiu2}{n.}{
    voleibol
}
\end{hanzi}

\begin{hanzi}[盘]{pan2}
\entry{pan2}{p.c.}{
    para cassete; video-cassete
}
\end{hanzi}

\begin{hanzi}[旁边]{pang2bian1}
\entry{pang2bian1}{p.l.}{
    junto a; próximo de; ao lado
}
\end{hanzi}

\begin{hanzi}[胖]{pang4}
\entry{pang4}{adj.}{
    gordo, gorda
}
\end{hanzi}

\begin{hanzi}[跑步]{pao3bu4}
\entry{pao3bu4}{v.}{
    correr
}
\end{hanzi}

\begin{hanzi}[陪]{pei2}
\entry{pei2}{v.}{
    acompanhar
}
\end{hanzi}

\begin{hanzi}[配]{pei3}
\entry{pei3}{v.}{
    combinar
}
\end{hanzi}

\begin{hanzi}[朋友]{peng2you0}
\entry{peng2you0}{n.}{
    namorado, namorada|
    amigo, amiga
}
\end{hanzi}

\begin{hanzi}[啤酒]{pi2jiu3}
\entry{pi2jiu3}{n.}{
    cerveja
}
\end{hanzi}

\begin{hanzi}[啤酒馆]{pi2jiu3guan3}
\entry{pi2jiu3guan3}{n.}{
    cervejaria
}
\end{hanzi}

\begin{hanzi}[屁股]{pi4gu}
\entry{pi4gu}{n.}{
    nádega; quadris
}
\end{hanzi}

\begin{hanzi}[票]{piao4}
\entry{piao4}{n.}{
    bilhete|
    \pc{张}
}
\end{hanzi}

\begin{hanzi}[漂亮]{piao4liang0}
\entry{piao4liang0}{adj.}{
    bonita, linda|
    bonito, lindo (para objetos inanimados)
}
\end{hanzi}

\begin{hanzi}[片]{pian4}
\entry{pian4}{p.c.}{
    palavra classificadora, para algumas coisas finas, com pouca espessura; 
    fatia, rodela
}
\end{hanzi}

\begin{hanzi}[瓶]{ping2}
\entry{ping2}{n.}{garrafa}
\entry{ping2}{p.c.}{
    palavra classificadora, garrafa
}
\end{hanzi}

\begin{hanzi}[平时]{ping2shi2}
\entry{ping2shi2}{p.t.}{
    normalmente; numa época normal
}
\end{hanzi}

\begin{hanzi}[苹果]{ping2guo3}
\entry{ping2guo3}{n.}{
    maçã
}
\end{hanzi}

\begin{hanzi}[葡汉词典]{pu2han4ci2dian3}
\entry{pu2han4ci2dian3}{n.}{
    dicionário português-chinês
}
\end{hanzi}

\begin{hanzi}[葡萄牙]{Pu2tao2ya2}
\entry{Pu2tao2ya2}{n.}{
    Portugal
}
\end{hanzi}

\begin{hanzi}[葡萄牙语]{pu2tao2ya2yu3}
\entry{pu2tao2ya2yu3}{n.}{
    português, língua portuguesa
}
\end{hanzi}

\begin{hanzi}[葡语]{pu2yu3}
\entry{pu2yu3}{n.}{
    português, língua portuguesa
}
\end{hanzi}

\begin{hanzi}[葡文]{pu2wen2}
\entry{pu2wen2}{n.}{
    português, língua portuguesa
}
\end{hanzi}

\begin{hanzi}[普通话]{pu3tong1hua4}
\entry{pu3tong1hua4}{n.}{
    mandarim (lit. ``linguagem comum'')
}
\end{hanzi}

\begin{hanzi}[便宜]{pian2yi0}
\entry{pian2yi0}{adj.}{
    barato
}
\end{hanzi}

\begin{hanzi}[乒乓球]{ping1pang1qiu2}
\entry{ping1pang1qiu2}{n.}{
    tênis de mesa; ping-pong
}
\end{hanzi}

\end{multicols}

%\section*{Q}
\addcontentsline{toc}{section}{Q}
\begin{multicols}{2}
%%%
%%% Q
%%%
% \entry{七}{num.}{sete}
% \entry{起床}{v.}{levantar-se}
% \entry{千}{num.}{mil}
% \entry{钱}{n.}{dinheiro}
% \entry{钱包}{n.}{carteira}
% \entry{前面}{p.l.}{em frente; na frente}
% \entry{巧克力}{n.}{chocolate}
% \entry{请}{v.}{fazer o favor}
% \entry{请问}{}{Desculpe...(para perguntar por qualquer coisa)}
% \entry{秋天}{s.}{outono}
% \entry{去}{v.}{ir}
% \entry{去年}{n.}{ano passado}
% \entry{裙子}{n.}{saia}
% \entry{球}{n.}{bola (futebol; basquetebol; handbol; etc)}
% \entry{前天}{p.t.}{anteontem}
% \entry{曲棍球}{n.}{hóquei em campo}
\end{multicols}

%%%%
%%% R
%%%
\section*{R}
\addcontentsline{toc}{section}{R}
\begin{multicols}{2}

\begin{hanzi}[然后]{ran2hou4}
\entry{ran2hou4}{conj.}{
    depois; logo; portanto
}
\end{hanzi}

\begin{hanzi}[让]{rang4}
\entry{rang4}{v.}{
    deixar; permitir
}
\end{hanzi}

\begin{hanzi}[热]{re4}
\entry{re4}{adj.}{
    quente
}
\end{hanzi}

\begin{hanzi}[热闹]{re4nao0}
\entry{re4nao0}{adj.}{
    animado; movimentado
}
\end{hanzi}

\begin{hanzi}[人]{ren2}
\entry{ren2}{n.}{
    pessoa; gente
}
\end{hanzi}

\begin{hanzi}[人口]{ren2kou3}
\entry{ren2kou3}{n.}{
    população
}
\end{hanzi}

\begin{hanzi}[人民币]{Ren2min2bi4}
\entry{Ren2min2bi4}{n.}{
    RMB|
    nome da moeda chinesa
}
\end{hanzi}

\begin{hanzi}[认识]{ren4shi0}
\entry{ren4shi0}{v.}{
    conhecer
}
\end{hanzi}

\begin{hanzi}[日]{ri4}
\entry{ri4}{p.c.}{
    dia (mais usado em escrita)
}
\end{hanzi}

\begin{hanzi}[日本]{Ri4ben3}
\entry{Ri4ben3}{n.}{
    Japão
}
\end{hanzi}

\begin{hanzi}[如果]{ru2guo3}
\entry{ru2guo3}{conj.}{
    se; caso; no caso de
}
\end{hanzi}

\begin{hanzi}[乳房]{ru3fang2}
\entry{ru3fang2}{n.}{
    seio; mama
}
\end{hanzi}

\begin{hanzi}[肉]{rou4}
\entry{rou4}{n.}{
    carne
}
\end{hanzi}

\end{multicols}

%%%%
%%% S
%%%
\section*{S}
\addcontentsline{toc}{section}{S}
\begin{multicols}{2}

\entry{三}{num.}{san1}{três}

\entry{嫂子}{n.}{sao3zi0}{esposa do irmão mais velho}

\entry{四川}{n.}{Si4chuan1}{Sichuan}

\entry{孙女}{n.}{sun1nur3}{filha do filho}
\entry{孙子}{n.}{sun1zi0}{filho do filho}

\entry{山}{n.}{shan1}{montanha; monte}
\entry{山东}{n.}{Shan4dong3}{Shandong}
\entry{山区}{n.}{shan1qu1}{área montanhosa; montanhas}

\entry{商店}{n.}{shang1dian4}{loja}

\entry{赏赐}{n.}{shang3ci4}{recompensa; prêmio}
\entry{赏赐}{v.}{shang3ci4}{recompensar; premiar}

\entry{上}{p.l.}{shang4}{acima; em cima de}
\entry{上}{v.d.}{shang4}{subir}
\entry{上海}{n.}{Shang4hai3}{Shangai (Xangai)}
\entry{上课}{v.}{shang4ke4}{ter aulas}
\entry{上面}{n.}{shang4mian4}{acima de; parte de cima}
\entry{上网}{v.}{shang4wang3}{acessar a Internet}
\entry{上午}{p.t.}{shang4wu3}{manhã; de manhã; período antes do meio-dia}
\entry{上询}{p.t.}{shang4xun2}{primeira dezena do mês}

\entry{少}{adj.}{shao3}{pouco; poucos}

\entry{谁}{interr.}{shei2}{quem}
\entry{谁}{interr.}{shui2}{quem}

\entry{身体}{n.}{shen1ti3}{corpo}

\entry{什么}{interr.}{shen2me0}{que; o que}
\entry{什么时候}{shen2me0shi2hou0}{interr.}{quando; a que horas}

\entry{生\xpinyin{日}{ri0}}{n.}{sheng1ri4}{aniversário}

\entry{圣诞节}{n.}{Sheng4dan4jie2}{Natal}

\entry{十}{num.}{shi2}{dez; dezena}

\entry{时候}{n.}{shi2hou0}{horas; tempo}
\entry{时候}{interr.}{shi2hou0}{quando}
\entry{时间}{n.}{shi2jian1}{tempo}

\entry{事}{n.}{shi4}{coisa; assunto}
\entry{事儿}{n.}{shir4}{afazeres; assunto; coisa; matéria|\fbox{件}}

\entry{试}{v.}{shi4}{experimentar; provar}

\entry{室}{n.}{shi4}{quarto}

\entry{是}{v.}{shi4}{ser}
\entry{是的}{adv.}{shi4de}{sim}

\entry{收到}{v.}{shou1dao4}{receber}

\entry{瘦}{adj.}{shou4}{magro; emagrecido}

\entry{书}{n.}{shu1}{livro|\fbox{本}}

\entry{舒\xpinyin{服}{fu0}}{adj.}{shu1fu2}{estar confortável; bem disposto; (sentir-se) bem}

\entry{暑假}{n.}{shu3jia4}{férias de verão}

\entry{树木}{n.}{shu4mu4}{árvore}

\entry{食品}{n.}{shi2拼}{comida}

\entry{水}{n.}{shui3}{água}
\entry{水果}{n.}{shui3guo3}{fruta}
\entry{水饺}{n.}{shui3jiao3}{dumplings; raviólis chineses}

\entry{睡觉}{v.}{shui4jiao4}{ir para a cama; dormir}

\entry{说}{v.}{shuo1}{falar; dizer}

\entry{四}{num.}{si4}{quatro}

\entry{送}{v.}{song4}{distribuir; entregar}

\entry{酸}{adj.}{suan1}{ácido; ácida; avinagrado; avinagrada}
\entry{酸辣汤}{n.}{suan1la4tang1}{sopa avinagrada e picante}

\entry{岁}{n.}{sui4}{anos de idade}

\end{multicols}

%%%%
%%% T
%%%
\section*{T}
\addcontentsline{toc}{section}{T}
\begin{multicols}{2}

\entry{它}{pron.}{ta1}{ele; ela|objetos e animais}
\entry{它们}{pron.}{ta1men0}{eles; elas|objetos e animais}

\entry{她}{pron.}{ta1}{ela}
\entry{她的}{pron.}{ta1de0}{dela}
\entry{她们}{pron.}{ta1men0}{elas}
\entry{她们的}{pron.}{ta1men0de0}{delas}

\entry{他}{pron.}{ta1}{ele}
\entry{他的}{pron.}{ta1de0}{dele}
\entry{他们}{pron.}{ta1men0}{eles}
\entry{他们的}{pron.}{ta1men0de0}{deles}

\entry{台}{p.c.}{tai2}{para computador, gravador, gravador de vídeo, etc}

\entry{太}{adv.}{tai4}{excessivamente; demais; muito}
\entry{太太}{n.}{tai4tai0}{esposa; mulher}

\entry{汤}{n.}{tang1}{sopa; caldo}

\entry{糖}{n.}{tang2}{açúcar}
\entry{糖醋鱼}{n.}{tang2cu4yu2}{peixe guisado em molho agridoce}

\entry{特别}{adv.}{te4bie2}{especialmente}

\entry{疼}{v.}{teng2}{doer}

\entry{踢}{v.}{ti1}{jogar; dar pontapés em}

\entry{天}{n.}{tian1}{dia}
\entry{天气}{n.}{tian1qi4}{clima; tempo}

\entry{甜}{adj.}{tian2}{doce}

\entry{条}{p.c.}{tiao2}{para calças, saia, rio, peixe, etc}

\entry{听}{v.}{ting1}{ouvir; escutar}

\entry{同学}{n.}{tong2xue2}{aluno; colega de classe}

\entry{头}{n.}{tou2}{cabeça}
\entry{头发}{n.}{tou2fa0}{cabelo}

\entry{腿}{n.}{tui3}{perna}

\end{multicols}

%%%%
%%% U
%%%
%\section*{U}
%\addcontentsline{toc}{section}{U}
%\begin{multicols}{2}
%\end{multicols}

%%%%
%%% V
%%%
%\section*{V}
%\addcontentsline{toc}{section}{V}
%\begin{multicols}{2}
%\end{multicols}

%%%%
%%% W
%%%
\section*{W}
\addcontentsline{toc}{section}{W}
\begin{multicols}{2}

\begin{hanzi}[外号]{wai4hao4}
\entry{wai4hao4}{n.}{apelido}
\end{hanzi}

\begin{hanzi}[外边]{wai4bian0}
\entry{wai4bian0}{n.}{fora; por fora; exterior}
\end{hanzi}

\begin{hanzi}[外公]{wai4gong1}
\entry{wai4gong1}{n.}{avô materno}
\end{hanzi}

\begin{hanzi}[外面]{wai4mian0}
\entry{wai4mian0}{n.}{fora; por fora; exterior}
\end{hanzi}

\begin{hanzi}[外婆]{wai4po2}
\entry{wai4po2}{n.}{avó materna}
\end{hanzi}

\begin{hanzi}[外孙]{wai4sun1}
\entry{wai4sun1}{n.}{filho da filha}
\end{hanzi}

\begin{hanzi}[外孙女]{wai4sun1nv3}
\entry{wai4sun1nv3}{n.}{filha da filha}
\end{hanzi}

\begin{hanzi}[玩]{wan2}
\entry{wan2}{v.}{brincar; tocar (intrumento musical)}
\end{hanzi}

\begin{hanzi}[玩儿]{wanr2}
\entry{wanr2}{v.}{divertir-se}
\end{hanzi}

\begin{hanzi}[完]{wan2}
\entry{wan2}{v.}{acabar; palavras}
\end{hanzi}

\begin{hanzi}[晚饭]{wan3fan4}
\entry{wan3fan4}{n.}{jantar}
\end{hanzi}

\begin{hanzi}[晚上]{wan3shang0}
\entry{wan3shang0}{p.t.}{noite; à noite}
\end{hanzi}

\begin{hanzi}[碗]{wan3}
\entry{wan3}{n}{tigela}
\entry{wan3}{p.c.}{palavra classificadora, tigela}
\end{hanzi}

\begin{hanzi}[碗子]{wan3zi0}
\entry{wan3zi0}{n}{tigela}
\end{hanzi}

\begin{hanzi}[万]{wan4}
\entry{wan4}{num.}{dez mil}
\end{hanzi}

\begin{hanzi}[往]{wang3}
\entry{wang3}{prep.}{para; em direção a}
\end{hanzi}

\begin{hanzi}[网球]{wang3qiu2}
\entry{wang3qiu2}{n.}{tênis (esporte)}
\end{hanzi}

\begin{hanzi}[喂]{wei2}
\entry{wei2}{interjeição}{ei, chamar atenção (alô, telefone)}
\entry{wei4}{interjeição}{ei, chamar atenção (alô, telefone)}
\end{hanzi}

\begin{hanzi}[喂]{wei4}
\entry{wei4}{interjeição}{ei, chamar atenção (alô, telefone)}
\entry{wei2}{interjeição}{ei, chamar atenção (alô, telefone)}
\end{hanzi}

\begin{hanzi}[为]{wei4}
\entry{wei4}{prep.}{para}
\end{hanzi}

\begin{hanzi}[位]{wei4}
\entry{wei4}{p.c.}{para pessoas (com cortesia)}
\end{hanzi}

\begin{hanzi}[味道]{wei4dao0}
\entry{wei4dao0}{n.}{sabor}
\end{hanzi}

\begin{hanzi}[为什么]{wei4shen2me0}
\entry{wei4shen2me0}{interr.}{por que?}
\end{hanzi}

\begin{hanzi}[卫生间]{wei4sheng1jian1}
\entry{wei4sheng1jian1}{n.}{banheiro; toilette}
\end{hanzi}

\begin{hanzi}[问]{wen4}
\entry{wen4}{v.}{perguntar}
\end{hanzi}

\begin{hanzi}[问题]{wen4ti2}
\entry{wen4ti2}{n.}{pergunta}
\end{hanzi}

\begin{hanzi}[我]{wo3}
\entry{wo3}{pron.}{eu}
\end{hanzi}

\begin{hanzi}[我的]{wo3de0}
\entry{wo3de0}{pron.}{meu(s); minha(s)}
\end{hanzi}

\begin{hanzi}[我们]{wo3men0}
\entry{wo3men0}{pron.}{nós}
\end{hanzi}

\begin{hanzi}[我们的]{wo3men0de0}
\entry{wo3men0de0}{pron.}{nosso(s); nossa(s)}
\end{hanzi}

\begin{hanzi}[午饭]{wu3fan4}
\entry{wu3fan4}{n.}{almoço}
\end{hanzi}

\begin{hanzi}[五]{wu3}
\entry{wu3}{num.}{cinco}
\end{hanzi}

\end{multicols}

%%%%
%%% X
%%%
\section*{X}
\addcontentsline{toc}{section}{X}
\begin{multicols}{2}

\begin{hanzi}[西]{xi1}
\entry{xi1}{n.}{oeste}
\end{hanzi}

\begin{hanzi}[西安]{xi1'an1}
\entry{xi1'an1}{n.}{Xi'an}
\end{hanzi}

\begin{hanzi}[西半球]{xi1ban4qiu2}
\entry{xi1ban4qiu2}{n.}{hemisfério oeste}
\end{hanzi}

\begin{hanzi}[西边]{xi1bian0}
\entry{xi1bian0}{p.l.}{oeste, ocidente}
\end{hanzi}

\begin{hanzi}[西方]{Xi1fang1}
\entry{Xi1fang1}{n.}{Ocidente}
\end{hanzi}

\begin{hanzi}[西面]{xi1mian0}
\entry{xi1mian0}{p.l.}{oeste, ocidente}
\end{hanzi}

\begin{hanzi}[西语]{xi1yu3}
\entry{xi1yu3}{n.}{espanhol, língua espanhola}
\end{hanzi}

\begin{hanzi}[西文]{xi1wen2}
\entry{xi1wen2}{n.}{espanhol, língua espanhola}
\end{hanzi}

\begin{hanzi}[悉尼]{Xi1ni2}
\entry{Xi1ni2}{n.}{Sidney}
\end{hanzi}

\begin{hanzi}[喜欢]{xi3huan0}
\entry{xi3huan0}{v.}{gostar}
\end{hanzi}

\begin{hanzi}[洗手间]{xi3shou3jian1}
\entry{xi3shou3jian1}{n.}{sanitário, toilette}
\end{hanzi}

\begin{hanzi}[系]{xi4}
\entry{xi4}{n.}{faculdade (da universidade)}
\end{hanzi}

\begin{hanzi}[下]{xia4}
\entry{xia4}{p.l.}{abaixo, em baixo de}
\entry{xia4}{v.d.}{descer}
\end{hanzi}

\begin{hanzi}[下巴]{xia4ba0}
\entry{xia4ba0}{n.}{queixo}
\end{hanzi}

\begin{hanzi}[下边]{xia4bian0}
\entry{xia4bian0}{p.l.}{em baixo, abaixo, parte de baixo}
\end{hanzi}

\begin{hanzi}[下车]{xia4che1}
\entry{xia4che1}{n.}{descer, sair (de ônibus)}
\end{hanzi}

\begin{hanzi}[下课]{xia4ke4}
\entry{xia4ke4}{v.+compl.}{acabar a aula, terminar a aula}
\end{hanzi}

\begin{hanzi}[下面]{xia4mian0}
\entry{xia4mian0}{p.l.}{em baixo, abaixo, parte de baixo}
\end{hanzi}

\begin{hanzi}[下午]{xia4wu3}
\entry{xia4wu3}{p.t.}{tarde, à tarde, período logo após o meio-dia}
\end{hanzi}

\begin{hanzi}[下旬]{xia4xun2}
\entry{xia4xun2}{p.t.}{última dezena do mês}
\end{hanzi}

\begin{hanzi}[下雨]{xia4yu3}
\entry{xia4yu3}{v.+compl.}{chover}
\end{hanzi}

\begin{hanzi}[夏天]{xia4tian1}
\entry{xia4tian1}{n.}{verão}
\entry{xia4tian1}{p.t.}{Verão}
\end{hanzi}

\begin{hanzi}[先]{xian1}
\entry{xian1}{adv.}{em primeiro lugar, primeiramente}
\end{hanzi}

\begin{hanzi}[先生]{xian1sheng0}
\entry{xian1sheng0}{n.}{senhor, marido}
\end{hanzi}

\begin{hanzi}[咸]{xian2}
\entry{xian2}{adj.}{salgado, salgada}
\end{hanzi}

\begin{hanzi}[现在]{xian4zai4}
\entry{xian4zai4}{p.t.}{agora}
\end{hanzi}

\begin{hanzi}[香港]{xiang1gang3}
\entry{xiang1gang3}{n.}{Hong Kong}
\end{hanzi}

\begin{hanzi}[想]{xiang3}
\entry{xiang3}{v./v.o.}{pensar, querer, achar}
\end{hanzi}

\begin{hanzi}[向]{xiang4}
\entry{xiang4}{prep.}{para}
\end{hanzi}

\begin{hanzi}[向汪]{xiang4wang3}
\entry{xiang4wang3}{v.}{esperar que}
\end{hanzi}

\begin{hanzi}[小]{xiao3}
\entry{xiao3}{adj.}{pequeno, pequena}
\end{hanzi}

\begin{hanzi}[小姐]{xiao3jie0}
\entry{xiao3jie0}{n.}{senhorita|empregada}
\end{hanzi}

\begin{hanzi}[小腿]{xiao3tui3}
\entry{xiao3tui3}{n.}{perna}
\end{hanzi}

\begin{hanzi}[小学]{xiao3xue2}
\entry{xiao3xue2}{n.}{escola ensino fundamental}
\end{hanzi}

\begin{hanzi}[校长]{xiao4zhang3}
\entry{xiao4zhang3}{n.}{diretor de escola|reitor (universidade)}
\end{hanzi}

\begin{hanzi}[些]{xie1}
\entry{xie1}{adv.}{uns, umas, alguns, algumas}
\end{hanzi}

\begin{hanzi}[写]{xie3}
\entry{xie3}{v.}{escrever}
\end{hanzi}

\begin{hanzi}[谢谢]{xie4xie0}
\entry{xie4xie0}{v.}{agradecer|obrigado, obrigada}
\end{hanzi}

\begin{hanzi}[谢天谢地]{xie4tian1xie4di4}
\entry{xie4tian1xie4di4}{}{agradecer a Deus, agradecer aos céus}
\end{hanzi}

\begin{hanzi}[新]{xin1}
\entry{xin1}{adj.}{novo}
\end{hanzi}

\begin{hanzi}[新年]{xin1nian2}
\entry{xin1nian2}{n.}{Ano Novo}
\end{hanzi}

\begin{hanzi}[星期]{xing1qi1}
\entry{xing1qi1}{n.}{semana}
\end{hanzi}

\begin{hanzi}[星期一]{xing1qi1yi4}
\entry{xing1qi1yi4}{n.}{segunda-feira}
\end{hanzi}

\begin{hanzi}[星期二]{xing1qi1}
\entry{xing1qi1}{n.}{terça-feira}
\end{hanzi}

\begin{hanzi}[星期三]{xing1qi1san1}
\entry{xing1qi1san1}{n.}{quarta-feira}
\end{hanzi}

\begin{hanzi}[星期四]{xing1qi1si4}
\entry{xing1qi1si4}{n.}{quinta-feira}
\end{hanzi}

\begin{hanzi}[星期五]{xing1qi1wu3}
\entry{xing1qi1wu3}{n.}{sexta-feira}
\end{hanzi}

\begin{hanzi}[星期六]{xing1qi1liu4}
\entry{xing1qi1liu4}{n.}{sábado}
\end{hanzi}

\begin{hanzi}[星期天]{xing1qi1tian1}
\entry{xing1qi1tian1}{n.}{domingo}
\end{hanzi}

\begin{hanzi}[星期日]{xing1qi1ri4}
\entry{xing1qi1ri4}{n.}{domingo}
\end{hanzi}

\begin{hanzi}[星星]{xing1xing0}
\entry{xing1xing0}{n.}{estrela}
\end{hanzi}

\begin{hanzi}[行]{xing2}
\entry{xing2}{v.}{claro que sim, de acordo, está bem}
\end{hanzi}

\begin{hanzi}[行人]{xing2ren2}
\entry{xing2ren2}{n.}{transeunte|peão}
\end{hanzi}

\begin{hanzi}[新鲜]{xin1xian1}
\entry{xin1xian1}{adj.}{fresco}
\end{hanzi}

\begin{hanzi}[信]{xin4}
\entry{xin4}{n.}{carta}
\end{hanzi}

\begin{hanzi}[姓]{xing4}
\entry{xing4}{n.}{sobrenome}
\entry{xing4}{v.}{ter o sobrenome}
\end{hanzi}

\begin{hanzi}[兴趣]{xing4qu4}
\entry{xing4qu4}{n.}{interesse}
\end{hanzi}

\begin{hanzi}[胸]{xiong1}
\entry{xiong1}{n.}{peito}
\end{hanzi}

\begin{hanzi}[休息]{xiu1xi0}
\entry{xiu1xi0}{v.}{descansar}
\end{hanzi}

\begin{hanzi}[学]{xue2}
\entry{xue2}{v.}{estudar}
\end{hanzi}

\begin{hanzi}[学生]{xue2sheng0}
\entry{xue2sheng0}{n.}{estudante, aluno, aluna}
\end{hanzi}

\begin{hanzi}[学生证]{xue2sheng0zheng4}
\entry{xue2sheng0zheng4}{n.}{cartão de estudante}
\end{hanzi}

\begin{hanzi}[学习]{xue2xi2}
\entry{xue2xi2}{v.}{estudar, aprender}
\end{hanzi}

\begin{hanzi}[学校]{xue2xiao4}
\entry{xue2xiao4}{n.}{escola, instituição de ensino}
\end{hanzi}

\begin{hanzi}[学院]{xue2yuan4}
\entry{xue2yuan4}{n.}{instituto}
\end{hanzi}

\begin{hanzi}[雪]{xue3}
\entry{xue3}{n.}{neve}
\end{hanzi}

\end{multicols}

%%%%
%%% Y
%%%
\section*{Y}
\addcontentsline{toc}{section}{Y}
\begin{multicols}{2}
\entry{鸭}{n.}{ya1}{pato}

\entry{压岁钱}{n.}{ya1sui4qian2}{dinheiro da sorte|dinheiro dado às crianças como presente no Ano Novo Chinês}

\entry{牙}{n.}{ya2}{dente}

\entry{亚洲}{n.}{Ya4zhou1}{Ásia}


\entry{颜色}{n.}{yan2se4}{cor}

\entry{眼镜}{n.}{yan3jing4}{óculos|\fbox{副}}
\entry{眼睛}{n.}{yan3jing0}{olho(s)}

\entry{样子}{n.}{yang4zi0}{aparência; forma}

\entry{药}{n.}{yao4}{remédio}

\entry{要}{v./v.o.}{yao4}{querer; precisar}
\entry{要是}{conj.}{yao4shi0}{se; caso}

\entry{爷爷}{n.}{ye2ye0}{avô (paterno)}

\entry{也}{adv.}{ye3}{também}

\entry{夜里}{p.t.}{ye4li0}{noite}

\entry{一}{num.}{yi1}{um; uma|«\pinyin{yi2}» antes de quarto tom|«\pinyin{yi4}» antes de outros tons}
\entry{一半}{adj.}{yi2ban4}{metade}
\entry{一共}{adv.}{yi2gong4}{tudo; no local}
\entry{一下}{adv.}{yi2xia4}{rapidamente; em um curto tempo}
\entry{一样}{adj.}{yi2yang4}{mesmo}
\entry{一般}{adj.}{yi4ban1}{geral}
\entry{一点儿}{adv.}{yi4dianr3}{um pouco}
\entry{一会儿}{adv.}{yi4huir4}{pouco tempo; daqui a pouco tempo}
\entry{一起}{adv.}{yi4qi3}{juntamente; em conjunto}
\entry{一直}{adv.}{yi4zhi2}{diretamente; sempre}

\entry{衣服}{n.}{yi1fu0}{roupa; vestiário|\fbox{件}}

\entry{医生}{n.}{yi1sheng1}{médico}

\entry{遗憾}{adj.}{yi2han4}{ter pena de}

\entry{以后}{n.}{yi3hou4}{depois de; depois; após}
\entry{以前}{p.t.}{yi3qian2}{antes de; antes}

\entry{亿}{num.}{yi4}{cem milhões}

\entry{意思}{n.}{yi4si0}{interesse}

\entry{因为}{conj.}{yin1wei4}{porque}

\entry{音乐}{n.}{yin1yue4}{música}

\entry{饮料}{n.}{yin3liao4}{bebida}

\entry{应该}{v.}{ying1gai1}{dever}

\entry{英国}{n.}{Ying1guo2}{Reino Unido}
\entry{英语}{n.}{ying1yu3}{inglês; língua inglesa}
\entry{英文}{n.}{ying1wen2}{inglês; língua inglesa}

\entry{邮件}{n.}{you2jian4}{correio}

\entry{游泳}{v.}{you2yong3}{nadar}
\entry{游泳池}{n.}{you2yong3chi2}{piscina}

\entry{有}{v.}{you3}{ter; haver}
\entry{有的}{pron.}{you3de0}{algum; alguma; alguns; algumas}
\entry{有点儿}{adv.}{you3dianr3}{um pouco}
\entry{有意思}{adj.}{you3yi2si0}{interessante}
\entry{有用}{adj.}{you3yong4}{útil}

\entry{右}{n.}{you4}{direita}
\entry{右边}{n.}{you4bian0}{direita}

\entry{用}{v.}{yong4}{usar}

\entry{鱼}{n.}{yu2}{peixe|\fbox{条}}
\entry{鱼香肉丝}{n.}{yu2xiang1rou4si1}{tiras de carne de porco salteadas com molho picante}

\entry{雨伞}{n.}{yu3san3}{guarda-chuva}
\entry{雨衣}{n.}{yu3yi1}{impermeável}

\entry{羽毛球}{n.}{yu3mao2qiu2}{badminton}

\entry{语言实验室}{n.}{yu3yan2shi2yan4shi4}{laboratório de línguas}

\entry{元}{p.c.}{yuan2}{unidade monetária da China}

\entry{约会}{n.}{yue1hui4}{compromisso; encontro marcado}

\entry{月}{n.}{yue4}{mês}

\entry{云南}{n.}{Yun2nan2}{Yunnan}

\entry{运动}{n.}{yun4dong4}{esporte; desporto}
\entry{运动场}{n.}{yun4dong4chang3}{campo desportivo; campo de jogos}
\entry{运动会}{n.}{yun4dong4hui4}{jogos desportivos}
\entry{运动员}{n.}{yun4dong4yuan2}{jogador; jogadora; atleta}

\end{multicols}

%%%%
%%% Z
%%%
\section*{Z}
\addcontentsline{toc}{section}{Z}
\begin{multicols}{2}

\entry{在}{adv.}{zai4}{para designar ações que estão passando}
\entry{在}{prep.}{zai4}{em}
\entry{在}{v.}{zai4}{estar; ficar}

\entry{再}{adv.}{zai4}{de novo; outra vez}
\entry{再见}{v.}{zai4jian4}{adeus; até à vista; até à próxima; até logo}

\entry{脏}{adj.}{zang1}{sujo}

\entry{造反}{n.}{zao3fan4}{café da manhã}

\entry{早上}{p.t.}{zao3shang0}{manhã cedo; manhãzinha}

\entry{怎么}{interr.}{zen3me0}{como}
\entry{怎么样}{interr.}{zen3me0yang4}{como; que tal}

\entry{站}{n.}{zhan4}{estação; ponto; paragem}

\entry{张}{p.c.}{zhang1}{para folha de papéis, mapa, etc}

\entry{着急}{adj.}{zhao2ji2}{inquieto; ansioso}

\entry{找}{v.}{zhao3}{andar à procura de; procurar; dar troco}

\entry{照片}{n.}{zhao4pian4}{fotografia; foto|\fbox{张}}
\entry{照相}{v.+compl.}{zhao4xiang4}{tirar fotografia}
\entry{照相机}{n.}{zhao4xiang4ji1}{câmera/máquina fotográfica}

\entry{这}{pron.}{zhe4}{este; esta; isto}
\entry{这儿}{pron.}{zher4}{aqui}
\entry{这里}{pron.}{zhe4li0}{aqui}
\entry{这些}{pron.}{zhe4xie1}{estes}

\entry{浙江}{n.}{Zhe4jiang1}{Zhejiang}

\entry{真}{adv.}{zhen1}{que...tão...!; realmente}

\entry{挣钱}{v.+compl.}{zheng4qian2}{ganhar dinheiro}

\entry{正在}{adv.}{zheng4zai4}{estar a + inf.; estar + ger.}

\entry{支}{p.c.}{zhi1}{para caneta, lápis, etc}

\entry{只}{p.c.}{zhi1}{para pássaros, gatos, cãezinhos, etc}

\entry{知道}{v.}{zhi1dao0}{conhecer; saber}

\entry{职员}{n.}{zhi2yuan2}{empregado; empregada}

\entry{只}{adv.}{zhi3}{apenas; só}

\entry{钟}{p.c.}{zhong1}{hora}

\entry{中国}{n.}{Zhong1guo2}{China}
\entry{中间}{p.l.}{zhong1jian1}{central; centro; no meio}
\entry{中文}{n.}{zhong1wen2}{chinês; língua chinesa}
\entry{中学}{n.}{zhong1xue2}{escola ensino médio}
\entry{中学生}{n.}{zhong1xue2sheng1}{estudante da escola ensino médio}
\entry{中询}{p.t.}{zhong1xun2}{segunda dezena do mês; meio do mês; em meados do mês}

\entry{种}{p.c.}{zhong3}{palavra classificadora, tipo}

\entry{重量}{n.}{zhong4liang4}{peso}

\entry{猪}{n.}{zhu1}{porco; porca}

\entry{住}{v.}{zhu4}{morar; viver; alojar-se}

\entry{祝}{v.}{zhu4}{desejar}

\entry{嘱咐}{v.}{zhu4fu4}{ordenar; dizer; exortar}

\entry{桌子}{n.}{zhuo1zi0}{mesa}

\entry{紫色}{n.}{zi3se4}{cor roxa}

\entry{自己}{pron.}{zi4ji3}{a si próprio; próprio}

\entry{自行车}{n.}{zi4xing2che1}{bicicleta}

\entry{走}{v.}{zou3}{andar; caminhar}

\entry{足球}{n.}{zu2qiu2}{futebol}

\entry{嘴巴}{n.}{zui3ba0}{boca}

\entry{最}{adv.}{zui4}{o mais; a mais|grau superlativo relativo de superioridade}
\entry{最近}{adv.}{zui4jin4}{ultimamente; recentemente}

\entry{昨天}{p.t.}{zuo2tian1}{ontem}

\entry{左}{p.l.}{zuo3}{esquerda}
\entry{左边}{p.l.}{zuo3bian0}{esquerda; lado esquerdo}
\entry{左面}{p.l.}{zuo3mian0}{esquerda; lado esquerdo}

\entry{坐}{v.}{zuo4}{sentar-se}

\entry{做}{v.}{zuo4}{fazer}

\end{multicols}


\printindex

\end{document}
