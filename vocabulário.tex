
%%%%%%%%%%%%%%%%%%%%%%%%%%%%%%%%%%%%%%%%%
% XeTeX
%
% Vocabulário
% Autor: Luiz Eduardo Roncato Cordeiro
%
% Licença:
% CC BY-NC-SA 3.0 (http://creativecommons.org/licenses/by-nc-sa/3.0/)
%%%%%%%%%%%%%%%%%%%%%%%%%%%%%%%%%%%%%%%%%

%\documentclass[a4paper,12pt,twoside,openany,draft]{memoir}
\documentclass[a4paper,12pt,twoside,openany]{memoir}

\usepackage{xltxtra}
\usepackage[brazil]{babel}
\usepackage{xeCJK}
\usepackage{xpinyin}
\usepackage{fontspec}
\usepackage{xunicode}
\usepackage{xltxtra}
\usepackage[usenames,dvipsnames]{color}
\usepackage{multicol}
\usepackage{fancyhdr}
\usepackage{imakeidx}
\usepackage{ifthen}
\usepackage{tocloft}
\usepackage{xparse}
\usepackage{enumitem}

%\setlength{\cftbeforesecskip}{2pt}
%\renewcommand{\cftsecleader}{\cftdotfill{\cftdotsep}}

\setCJKmainfont{AR PL UKai CN}
\setCJKsansfont{AR PL UMing CN}

\makeindex[columns=3, title=Índice, intoc]

\setlength{\parindent}{0em}
\setlength{\parskip}{0.5em}
\setlength{\columnseprule}{0.5pt}

\xpinyinsetup{ratio={.5},vsep={1.2em},multiple={\color{Sepia}}}

% Headers & footers
\fancyhead[L]{\textsf{\rightmark}} % Top left header
\fancyhead[R]{\textsf{\leftmark}} % Top right header
\renewcommand{\headrulewidth}{1.4pt} % Rule under the header
\fancyfoot[C]{{\thepage}} % Bottom center footer
\renewcommand{\footrulewidth}{1.4pt} % Rule under the header
\pagestyle{fancy} % Use the custom headers and footers throughout the document

\setlength{\headheight}{16pt}
\addtolength{\topmargin}{-0.5pt}

\NewDocumentCommand\mylist{>{\SplitList{|}}m}
{
\setlength{\leftmargin}{0em}
\begin{enumerate}[nosep,left=0em,label=\arabic*.]
  \ProcessList{#1}{ \insertitem }
\end{enumerate}
}
\newcommand\insertitem[1]{\item #1}

\newenvironment{hanzi}[2][none]
  {\vspace{1ex}
  \markboth{#1«\pinyin{#2}»}{#1«\pinyin{#2}»}\index{#1«\pinyin{#2}»}
  \begin{minipage}[t][][t]{\linewidth}
  {\Huge \textbf{#1}} \\
  }
  {\end{minipage}
  }

\newcommand{\entry}[3]{%
«\pinyin{#1}» (#2)%
{\small \mylist{#3}}%
\vspace{1ex}
}

\newcommand{\e}[1]{\textcolor{OliveGreen}{#1}}

\makeatletter
\let\old@makechapterhead\@makechapterhead
% Taken from http://mirrors.ctan.org/macros/latex/unpacked/report.cls
\def\fake@makechapterhead#1{%
  \vspace*{50\p@}%
  {\parindent \z@ \raggedright \normalfont
    \ifnum \c@secnumdepth >\m@ne
        \huge\bfseries \strut%\@chapapp\space \thechapter
        \par\nobreak
        \vskip 20\p@
    \fi
    \interlinepenalty\@M
    \Huge \bfseries #1\par\nobreak
    \vskip 40\p@
  }
  \markboth{#1}{\thechapter}
}
\newcommand{\newchapterhead}{\let\@makechapterhead\fake@makechapterhead}
\newcommand{\restorechapterhead}{\let\@makechapterhead\old@makechapterhead}
\makeatother

%%%
%%% Documento começa aqui!
%%%

\begin{document}

\newchapterhead

\begin{titlingpage} % Suppresses displaying the page number on the title page and the subsequent page counts as page 1
	
	\raggedleft % Right align the title page
	
	\rule{1pt}{\textheight} % Vertical line
	\hspace{0.05\textwidth} % Whitespace between the vertical line and title page text
	\parbox[b]{0.75\textwidth}{ % Paragraph box for holding the title page text, adjust the width to move the title page left or right on the page
		
		{\Huge\bfseries 汉葡词典}\\[2\baselineskip] % Title
		{\large\textit{O Minúsculo Dicionário do Curso de Chinês}}\\[4\baselineskip] % Subtitle or further description
		{\Large\textsc{罗学凯}} % Author name, lower case for consistent small caps
		
		\vspace{0.5\textheight} % Whitespace between the title block and the publisher
		
		{\noindent \today}\\[\baselineskip] % Publisher and logo
	}

\end{titlingpage}

\tableofcontents

\newpage

\chapter{Termos Gramaticais Chineses}

\begin{tabular}{lll}
nome/substantivo       & \textbf{n.}        & 名词 \\
palavra de lugar       & \textbf{p.d.l.}    & 处所词 \\
palavra de localização & \textbf{p.l.}      & 方位词 \\
palavra de tempo       & \textbf{p.t.}      & 时间词 \\
verbo                  & \textbf{v.}        & 动词 \\
verbo direcional       & \textbf{v.d.}      & 趣向\hspace{1em}动词 \\
verbo optativo         & \textbf{v.o.}      & 能缘\hspace{1em}动词 \\
adjetivo               & \textbf{adj.}      & 形容词 \\
numeral                & \textbf{num.}      & 数词 \\
palavra classificadora & \textbf{p.c.}      & 两量词 \\
pronome                & \textbf{pron.}     & 代词 \\
interrogativo          & \textbf{interr.}   & 疑问词 \\
advérbio               & \textbf{adv.}      & 副词 \\
preposição             & \textbf{prep.}     & 介词 \\
conjunção              & \textbf{conj.}     & 连词 \\
partícula              & \textbf{part.}     & 助词 \\
sujeito                & \textbf{suj.}      & 主语 \\
objeto                 & \textbf{obj.}      & 宾语 \\
atributo               & \textbf{atrib.}    & 定语 \\
adjunto adverbial      & \textbf{a.adv.}    & 状语 \\
complemento            & \textbf{compl.}    & 补语 \\
verbo+complemento      & \textbf{v.+compl.} & 动宾式\hspace{1em}离合词 \\
\end{tabular}

\newpage

\chapter{汉葡词典}

%%%
%%% A
%%%
\section*{A}
\addcontentsline{toc}{section}{A}
\begin{multicols}{2}

\begin{hanzi}[啊]{a0}
\entry{a0}{part.}{ah!;oh!|no final da sentença para expressar entusiasmo|%
no final da sentença para expressar impaciência ou o que é óbvio|%
no final de uma ordem, aviso, etc|%
no final da sentença para expressar questionamento|%
para indicar uma pausa deliberada|%
para enumerar itens}
\end{hanzi}

\begin{hanzi}[矮]{ai3}
\entry{ai3}{adj.}{baixo (estatura, dimensão, grau ou ranque)}
\end{hanzi}

\begin{hanzi}[爱]{ai4}
\entry{ai4}{n.}{amor; afeição}
\entry{ai4}{v.}{amar; gostar|ter afeição}
\end{hanzi}

\begin{hanzi}[爱好]{ai4hao4}
\entry{ai4hao4}{n.}{passatempo; interesse|\pc{个}}
\entry{ai4hao4}{v.}{ter algo como hobby; ter prazer em fazer algo}
\end{hanzi}

\begin{hanzi}[爱人]{ai4ren0}
\entry{ai4ren0}{n.}{marido ou esposa|querido ou querida|\pc{个}}
\end{hanzi}

\end{multicols}

%%%
%%% B
%%%
\section*{B}
\addcontentsline{toc}{section}{B}
\begin{multicols}{2}

\entry{吧}{part.}{ba0}{partícula usada para sugestão ou suposição}

\entry{八}{num.}{ba1}{oito}

\entry{巴西}{n.}{Ba1xi1}{Brasil}

\entry{爸爸}{n.}{ba4ba0}{papai; pai}

\entry{白}{adj.}{bai2}{branco; branca}
\entry{白菜}{n.}{bai2cai4}{repolho chinês}
\entry{白色}{n.}{bai2se4}{cor branca}
\entry{白天}{p.t.}{bai2tian1}{dia; de dia}

\entry{百}{num.}{bai3}{centena; cem; cento}

\entry{半}{num.}{ban4}{meio; meia}

\entry{帮助}{n.}{bang1zhu4}{ajuda}
\entry{帮助}{v.}{bang1zhu4}{ajudar}

\entry{包}{p.c.}{bao1}{pacote; saco; sacola}

\entry{杯子}{n.}{bei1zi0}{copo; caneca; xícara; taça}

\entry{北京}{n.}{Bei3jing1}{Pequim; Capital da China}
\entry{北京}{n.}{Bei3jing1}{Beijing(Pequim)}

\entry{被子}{n.}{bei4zi0}{colcha}

\entry{本}{p.c.}{ben3}{para livros, dicionários, etc}
\entry{本}{n.}{ben3zi0}{caderno}

\entry{笔}{n.}{bi3}{caneta; lápis|\fbox{支}}

\entry{比较}{adv.}{bi3jiao4}{comparativamente; relativamente}
\entry{比萨饼}{n.}{bi3sa4bing3}{pizza}
\entry{比赛}{n.}{bi3sai4}{competição; concurso}

\entry{别}{adv.}{bie2}{nada de (pedir a alguém para não fazer); não}
\entry{别的}{pron.}{bie2de0}{outro; outra}

\entry{冰球}{n.}{bing1qiu2}{hóquei no gelo}

\entry{不}{adv.}{bu4}{não|«\pinyin{bu2}» antes de quarto tom|«\pinyin{bu0}» em expressões\\``verbo$+$不$+$verbo''}
\entry{不同}{adj.}{bu4tong2}{diferente}

\entry{不错}{adj.}{bu2cuo4}{bom; boa}
\entry{不客气}{}{bu2ke4qi4}{de nada; não há de que}
\entry{不要}{v.o.}{bu2yao4}{nada de (pedir a alguém não fazer); não}
\entry{不用}{v.o.}{bu2yong4}{não precisar}

\end{multicols}

%%%%
%%% C
%%%
\section*{C}
\addcontentsline{toc}{section}{C}
\begin{multicols}{2}

\entry{菜}{n.}{cai4}{hortaliça; verdura; prato}
\entry{菜单}{n.}{cai4dan1}{menu; ementa; cardápio}

\entry{草地}{n.}{cao3di4}{relva; pastagem}

\entry{参加}{v.}{can1jia1}{juntar; participar}

\entry{餐厅}{n.}{can1ting1}{cantina; sala de jantar}

\entry{厕所}{n.}{ce4suo3}{sanitário; toilette}

\entry{磁带}{n.}{ci2dai4}{cassete|\fbox{盘}}
\entry{磁盘}{n.}{ci2pan2}{disquete}

\entry{词典}{n.}{ci2dian3}{dicionário|\fbox{本}}

\entry{茶}{n.}{cha2}{chá}

\entry{长}{adj.}{chang2}{comprido; longo}
\entry{长成}{n.}{chang2cheng2}{Grande Muralha}

\entry{常常}{adv.}{chang2chang2}{frequentemente}

\entry{炒}{v.}{chao3}{saltear}

\entry{车}{n.}{che1}{veículo; viatura}
\entry{车牌}{n.}{che1pai2}{matrícula; placa de carro}
\entry{车站}{n.}{che1zhan4}{estação; paragem}

\entry{衬衫}{n.}{chen4shan1}{camisa|\fbox{件}}

\entry{成都}{n.}{cheng2du1}{Chengdu}

\entry{城市}{n.}{cheng2shi4}{cidade}

\entry{橙色}{n.}{cheng2se4}{cor de laranja}
\entry{橙汁}{n.}{cheng2zhi1}{suco de laranja}

\entry{惩罚}{v.}{cheng2fa2}{punir; penalizar}
\entry{惩处}{v.}{cheng2chu3}{punir; penalizar}

\entry{吃}{v.}{chi1}{comer}

\entry{迟到}{v.}{chi1dao4}{chegar atrasado; tardar}

\entry{憧憬}{v.}{chong1jing3}{ansiar por; esperar por}

\entry{宠物}{n.}{chong3wu4}{animal de estimação}

\entry{酬劳}{n.}{chou2lao2}{recompensa}

\entry{出}{v.d.}{chu1}{sair}
\entry{出去}{v.d.}{chu1qu0}{sair; ir para fora}
\entry{出口}{n.}{chu1kou3}{exportação}
\entry{出口}{v.}{chu1kou3}{exportar}
\entry{出站}{n.}{chu1zhan4}{saída da estação}
\entry{出租汽车}{n.}{chu1zu1qi4che1}{táxi|\fbox{辆}}

\entry{穿}{v.}{chuan1}{vestir}

\entry{船}{v.}{chuan2}{barco; navio}

\entry{传真}{n.}{chuan2zhen1}{fax; facsímile}

\entry{床}{n.}{chuang2}{cama|\fbox{张}}

\entry{春天}{n.}{chun1tian1}{primavera}

\entry{绰号}{n.}{chuo4hao4}{apelido}

\entry{聪明}{adj.}{cong1ming2}{inteligente; brilhante; esperto}
\entry{聪慧}{adj.}{cong1hui4}{inteligente; brilhante}

\entry{从}{prep.}{cong2}{de; desde; a partir de}

\entry{醋}{n.}{cu4}{vinagre}

\entry{错}{adj.}{cuo4}{errado; enganado}

\end{multicols}

%\section*{D}
\addcontentsline{toc}{section}{D}
\begin{multicols}{2}
%%%
%%% D
%%%
% \entry{打}{v.}{jogar}
% \entry{打电话}{v.}{ligar; dar um telefonema}
% \entry{打算}{v.}{pretender}
% \entry{大}{adj.}{grande}
% \entry{大概}{adv.}{aproximadamente; por volta de}
% \entry{大海}{n.}{mar}
% \entry{大家}{pron.}{todos; todas}
% \entry{大学}{n.}{universidade}
% \entry{\xpinyin{大}{Da4}洋洲}{n.}{Oceania}
% \entry{带}{v.}{levar}
% \entry{戴}{v.}{usar; vestir}
% \entry{担心}{v.}{preocupar-se}
% \entry{蛋糕}{n.}{bolo}
% \entry{当然}{adv.}{claro; certamente; com certeza}
% \entry{到}{v.}{chegar}
% \entry{得}{v.}{ganhar; obter}
% \entry{\xpinyin{德}{De2}国}{n.}{Alemanha}
% \entry{的}{part.}{partícula utilizada em possessivos; partícula utilizada entre adjetivos e substantivos, opcional se substantivo possui apenas um caracter}
% \entry{地方}{n.}{lugar; local; sítio}
% \entry{等}{v.}{esperar}
% \entry{第}{num.}{prefixo para expressar números ordinais}
% \entry{弟\xpinyin{弟}{di0}}{n.}{irmão mais novo}
% \entry{\xpinyin{地}{di4}图}{n.}{mapa}
% \entry{点(钟)}{p.c.}{hora}
% \entry{(一)\xpinyin{点}{dianr3}\xpinyin{儿}{}}{p.c.}{um pouco}
% \entry{(商)店}{n.}{loja}
% \entry{电话}{n.}{telefone}
% \entry{电脑}{n.}{computador}
% \entry{电视}{n.}{televisor}
% \entry{电影}{n.}{cinema; filme}
% \entry{电子}{n.}{eletrônico; eletrônica)}
% \entry{电子邮件}{n.}{correio eletrônico; e-mail}
% \entry{东}{n.}{leste}
% \entry{\xpinyin{东}{Dong1}方}{n.}{Oriente}
% \entry{东天}{n.}{inverno}
% \entry{东\xpinyin{西}{xi0}}{n.}{coisa}
% \entry{都}{adv.}{todo; toda; todos; todas}
% \entry{读}{v.}{ler}
% \entry{度}{v.}{passar}
% \entry{锻炼}{v.}{fazer exercício físico}
% \entry{对}{adj.}{correto; sim}
% \entry{对不起}{}{desculpar; pedir desculpa; perdão}
% \entry{多}{adj.}{muito; muita; muitos; muitas}
% \entry{多大}{interr.}{quantos anos; que idade}
% \entry{多少}{interr.}{quanto; quanta; quantos; quantas}
% \entry{打算}{v./n.}{pensar; planear; plano}
% \entry{打球}{v.}{jogar à bola; jogar (futebol; basquetebol; handbol; etc)}
% \entry{冰球}{n.}{hóquei em gelo}
\end{multicols}

%%%%
%%% E
%%%
\section*{E}
\addcontentsline{toc}{section}{E}
\begin{multicols}{2}
\entry{俄罗斯}{n.}{É\ luo2si1}{Rússia}
\entry{儿媳}{n.}{er2xi2}{esposa do filho}
\entry{儿子}{n.}{er2zi0}{filho}
\entry{二}{num.}{er4}{dois}
\end{multicols}

%%%%
%%% F
%%%
\section*{F}
\addcontentsline{toc}{section}{F}
\begin{multicols}{2}
\entry{发}{v.}{fa1}{enviar; mandar}
\entry{发国}{n.}{Fa3guo2}{França}
\entry{法语}{n.}{Fa3yu3}{françês; língua francesa}
\entry{法文}{n.}{Fa3wen2}{françês; língua francesa}
% \entry{饭店}{n.}{restaurante; hotel}
\entry{非常}{adv.}{fei1chang2}{muito}
% \entry{飞机}{n.}{avião}
\entry{非洲}{n.}{Fei1zhou1}{África}
% \entry{分}{p.c.}{centavo}
% \entry{分}{p.c.}{minuto}
% \entry{分钟}{n.}{minuto}
\entry{分公司}{n.}{fen1gong1si1}{sucursal; filial de companhia}
% \entry{份}{p.c.}{dose}
% \entry{风}{n.}{vento}
% \entry{副}{p.c.}{par}
\entry{贵姓}{}{gui4xing4}{Qual é o nome?}
\end{multicols}

%\section*{G}
\addcontentsline{toc}{section}{G}
\begin{multicols}{2}
%%%
%%% G
%%%
% \entry{干}{v.}{fazer}
% \entry{高兴}{adj.}{feliz; alegre; contente}
% \entry{哥\xpinyin{哥}{ge0}}{n.}{irmão mais velho}
% \entry{个}{p.c.}{de uso geral}
% \entry{给}{pre.}{a; para}
% \entry{给}{v.}{dar}
% \entry{跟}{prep.}{com}
% \entry{公司}{n.}{empresa; companhia}
% \entry{工作}{n./v.}{trabalho; trabalhar}
% \entry{狗}{n.}{cão; cachorro}
% \entry{刮(风)}{v.}{soprar (vento)}
% \entry{拐}{v.}{virar}
% \entry{光盘}{n.}{disco compacto}
% \entry{贵姓}{interr.}{nome}
% \entry{国}{n.}{país}
% \entry{果酱}{n.}{compota ou doce (de frutas)}
% \entry{过年}{v.}{festejar o Ano Novo Chinês}
% \entry{贵}{adj.}{caro}
% \entry{公克}{n.}{trabalho escolar; trabalho de casa}
% \entry{橄榄球}{n.}{rúgbi}
\end{multicols}

%%%%
%%% H
%%%
\section*{H}
\addcontentsline{toc}{section}{H}
\begin{multicols}{2}
\entry{还}{adv.}{hai2}{ainda; também}
\entry{还是}{conj.}{hai2shi0}{ou|somente para frases interrogativas}

\entry{孩子}{n.}{hai2zi0}{criança; filho; filha}

\entry{海边}{n.}{hai3bian1}{praia}

\entry{汉国}{n.}{Han2guo2}{Coréia do Sul}
\entry{汉葡词典}{n.}{han4pu2ci2dian3}{dicionário chinês-português}
\entry{汉语}{n.}{Han4yu3}{chinês; língua chinesa; mandarim}

\entry{航班}{n.}{hang2ban1}{número de voo}

\entry{好}{adj.}{hao3}{bom; boa; bem}
\entry{好吃}{adj.}{hao3chi1}{delicioso; saboroso}
\entry{好学}{adj.}{hao3xue2}{fácil de aprender}

\entry{号}{n.}{hao4}{dia; número}
\entry{号}{p.c.}{hao4}{dia}
\entry{号码}{n.}{hao4ma3}{número}

\entry{喝}{v.}{he1}{beber}

\entry{和}{conj.}{he2}{e|somente para palavras}

\entry{河}{n.}{he2}{rio}

\entry{盒}{p.c.}{he2}{caixa}

\entry{黑}{n.}{hei1}{preto; preta}
\entry{黑板}{n.}{hei1ban3}{quadro preto}
\entry{黑色}{n.}{hei1se4}{cor preta}

\entry{很}{adv.}{hen3}{muito; mui}

\entry{红}{adj.}{hong2}{vermelho; vermelha}
\entry{红色}{n.}{hong2se4}{cor vermelha}
\entry{红烧}{n.}{hong2shao1}{guisado em molho de soja}

\entry{后面}{p.l.}{hou4mian4}{atrás}
\entry{后年}{p.t.}{hou4nian2}{daqui a dois anos}
\entry{后天}{p.t.}{hou4tian1}{depois de amanhã}

\entry{湖南}{n.}{Hu2nan2}{Hunan}

\entry{华盛顿}{n.}{Hua2sheng4dun4}{Washington}
\entry{华裔}{n.}{hua2yi4}{descendente de chinês}

\entry{话}{n.}{hua4}{palavra; fala}

\entry{欢迎}{v.}{huan1ying2}{bem vindo}

\entry{黄}{adj.}{huang2}{amarelo; amarela}
\entry{黄色}{n.}{huang2se4}{cor amarela}
\entry{黄油}{n.}{huang2you2}{manteiga}

\entry{回}{v.d.}{hui2}{regressar}
\entry{回来}{v.d.}{hui2lai0}{regressar; voltar; estar de volta}

\entry{会}{v.}{hui4}{saber}

\entry{火车}{n.}{huo3che1}{trem}

\entry{或者}{conj.}{huo4zhe3}{ou|usado em expressões afirmativas}

\end{multicols}

%%\section*{I}
%\addcontentsline{toc}{section}{I}
%\begin{multicols}{2}
%%%
%%% I
%%%
%\end{multicols}

%%%%
%%% J
%%%
\section*{J}
\addcontentsline{toc}{section}{J}
\begin{multicols}{2}

\entry{鸡}{n.}{ji1}{galo; galinha}
\entry{鸡蛋}{n.}{ji1dan4}{ovo de galinha}

\entry{机场}{n.}{ji1chang3}{aeroporto}

\entry{几}{interr.}{ji3}{quantos; quantas; alguns; algumas|para quantidades até 10 itens}

\entry{家}{n.}{jia1}{família; casa}
\entry{家里}{p.d.l.}{jia1li0}{em casa}

\entry{加拿大}{n.}{Jia1na2da4}{Canadá}

\entry{检查}{v.}{jia3cha2}{examinar}

\entry{肩膀}{n.}{jian1bang3}{ombro}

\entry{件}{p.c.}{jian4}{para roupas}

\entry{见}{v.}{jian4}{ver}
\entry{见面}{v.}{jian4mian4}{encontar-se com alguém}

\entry{江西}{n.}{Jiang1xi1}{Jiangxi}

\entry{胶卷}{n.}{jiao1juan3}{filme; película; rolo}

\entry{脚}{n.}{jiao3}{pé}

\entry{角}{p.c.}{jiao3}{1 jiao = 10 centavos}

\entry{饺子}{n.}{jiao3zi0}{jiaozi; raviólis chineses; bolinho de massa}

\entry{叫}{v.}{jiao4}{chamar-se; chamar}

\entry{教}{v.}{jiao4}{ensinar}
\entry{教练}{n.}{jiao4lian4}{treinador}
\entry{教授}{n.}{jiao4shou4}{professor; professora}
\entry{教室}{n.}{jiao4shi1}{professor; professora; mestre; mestra}
\entry{教室}{n.}{jiao4shi4}{sala de aula}

\entry{街}{n.}{jie1}{rua}

\entry{接}{v.}{jie1}{ir buscar (alguém); ir ao encontro de (alguém); receber}

\entry{节日}{n.}{jie2ri4}{festa}

\entry{姐姐}{n.}{jie3jie0}{irmã mais velha}
\entry{姐夫}{n.}{jie3fu0}{marido da irmã mais velha}

\entry{今年}{p.t.}{jin1nian2}{este ano}
\entry{今天}{p.t.}{jin1tian1}{hoje}

\entry{近}{adj.}{jin4}{perto; próximo}

\entry{进}{v.d.}{jin4}{entrar}
\entry{进出口}{n.}{jin4chu1kou3}{importação e exportação}
\entry{进口}{n.}{jin4kou3}{importação}
\entry{进口}{v.}{jin4kou3}{importar}
\entry{进来}{v.d.}{jin4lai0}{entrar}

\entry{经常}{adv.}{jing1chang2}{muitas vezes}

\entry{酒}{n.}{jiu3}{bebida alcoólica}

\entry{九}{num.}{jiu3}{nove}

\entry{酒馆}{n.}{jiu3guan3}{bar}

\entry{就}{adv.}{jiu4}{exatamente; justamente}

\entry{句}{p.c.}{ju4}{para oração, frase}

\entry{举行}{v.}{ju3xing2}{realizar; ter lugar}

\entry{觉得}{v.}{jue2de2}{achar; sentir}

\end{multicols}

%%%%
%%% K
%%%
\section*{K}
\addcontentsline{toc}{section}{K}
\begin{multicols}{2}
\entry{咖啡}{n.}{ka1fei1}{café}
\entry{咖啡馆}{n.}{ka1fei1guan3}{cafeteria}
% \entry{开}{v.}{abrir; dirigir}
% \entry{开车}{v.}{conduzir; dirigir}
% \entry{看}{v.}{olhar; ver; assistir}
% \entry{看见}{v.}{encontrar; enxergar; ver}
% \entry{颗}{p.c.}{para grãos e coisas semelhantes}
\entry{可爱}{adj.}{ke3'ai4}{querido; querida; fofo; fofa}
\entry{可口可乐}{n.}{ke3kou3ke3le3}{coca-cola}
% \entry{可能}{adv.}{provavelmente}
% \entry{可是}{conj.}{mas}
\entry{可以}{v.o.}{ke3yi3}{poder}
% \entry{刻(钟)}{n.}{um quarto (de hora)}
\entry{口}{p.c.}{kou3}{para membros da família}
% \entry{口香糖}{n.}{pastilha elástica; goma de mascar; chiclete}
% \entry{块(元)}{p.c.}{unidade de Reminbi}
% \entry{快乐}{adj.}{feliz}
% \entry{裤子}{n.}{calças}
% \entry{\xpinyin{空}{kongr4}\xpinyin{儿}{}}{n.}{tempo livre}
\end{multicols}

%%%%
%%% L
%%%
\section*{L}
\addcontentsline{toc}{section}{L}
\begin{multicols}{2}
% \entry{拉拉队}{n.}{claque}
% \entry{来}{v.}{vir}
% \entry{蓝}{adj.}{azul}
% \entry{蓝色}{n.}{cor azul}
% \entry{篮球}{n.}{basquetebol}
% \entry{老板}{n.}{patrão; patroa}
\entry{老师}{n.}{lao3shi1}{professor; professora}
% \entry{了}{part.}{partícula para denotar mudança}
% \entry{冷}{adj.}{frio}
% \entry{里}{p.l.}{em; dentro; interior}
% \entry{\xpinyin{里}{Li3}斯本}{n.}{Lisboa}
% \entry{礼物}{n.}{presente}
% \entry{凉快}{adj.}{fresco}
% \entry{两}{num.}{dois; duas (sempre usado antes de palavra classificadora)}
% \entry{辆}{p.c.}{para automóveis}
% \entry{邻居}{n.}{vizinho}
% \entry{零}{num.}{zero}
% \entry{六}{num.}{seis}
% \entry{龙}{n.}{dragão}
% \entry{龙山}{n.}{Longshan}
% \entry{路口}{n.}{cruzamento}
% \entry{录像带}{n.}{video-cassete}
% \entry{录像机}{n.}{gravador de vídeo}
% \entry{录音机}{n.}{gravador}
% \entry{绿}{adj.}{verde}
% \entry{绿色}{n.}{cor verde}
% \entry{辣}{adj.}{picante}
\end{multicols}

%%%%
%%% M
%%%
\section*{M}
\addcontentsline{toc}{section}{M}
\begin{multicols}{2}
% \entry{妈\xpinyin{妈}{ma0}}{n.}{mamãe; mãe}
% \entry{马上}{adv.}{já; imediatamente}
\entry{吗}{part.}{ma0}{partícula interrogativa|perguntas ``sim-não''}
% \entry{买}{v.}{comprar}
% \entry{买东\xpinyin{西}{xi0}}{v.}{fazer compras}
% \entry{卖}{v.}{vender}
% \entry{忙}{adj.}{ocupado; ocupada}
% \entry{猫}{n.}{gato}
% \entry{没关\xpinyin{系}{xi0}}{}{não ter problema; não ter importância; não fazer mal}
% \entry{\xpinyin{美}{Mei2}国}{n.}{Estados Unidos da América}
% \entry{没有}{v.}{não há; não tem}
% \entry{每次}{}{todas as vezes; sempre}
% \entry{每天}{}{todos os dias}
% \entry{\xpinyin{美}{Mei3}洲}{n.}{América}
% \entry{妹\xpinyin{妹}{mei0}}{n.}{irmã mais nova}
\entry{们}{sufixo}{men0}{sufixo para plural}
% \entry{米饭}{n.}{arroz cozido}
% \entry{面包}{n.}{pão}
% \entry{明天}{p.t.}{amanhã}
% \entry{名字}{n.}{nome}
% \entry{明恋}{adj.}{claro; clara}
% \entry{明年}{n.}{próximo ano}
% \entry{墨镜}{n.}{óculos escuros}
\end{multicols}

%%%%
%%% N
%%%
\section*{N}
\addcontentsline{toc}{section}{N}
\begin{multicols}{2}
% \entry{哪}{interr.}{que, qual}
% \entry{\xpinyin{哪}{nar3}\xpinyin{儿}{}}{interr.}{onde}
% \entry{哪国人}{interr.}{de qual país?}
% \entry{哪里}{interr.}{onde}
% \entry{那}{conj.}{nessa situação; nesse caso}
% \entry{那}{pron.}{aquele}
% \entry{\xpinyin{那}{nar4}\xpinyin{儿}{}}{pron.}{lá; ali}
% \entry{那里}{pron.}{lá; ali}
% \entry{奶\xpinyin{奶}{nai}}{n.}{avó(paterna)}
% \entry{男}{adj.}{masculino}
% \entry{男孩儿}{n.}{menino; rapaz}
\entry{呢}{interr.}{ne0}{partícula interrogativa enfática}
% \entry{能}{v.}{poder}
\entry{你}{pron.}{ni3}{você (informal); tu}
\entry{你的}{pron.}{ni3de0}{seu; sua}
\entry{你们}{pron.}{ni3men0}{vocês (informal); vós}
\entry{你们的}{pron.}{ni3men0de0}{seus; suas}
% \entry{年}{p.t.}{ano}
% \entry{年纪}{n.}{idade}
% \entry{年货}{n.}{compras especiais para o Ano Novo Chinês}
\entry{您}{pron.}{nin2}{você (formal); tu}
% \entry{牛奶}{n.}{leite}
% \entry{女}{adj.}{feminino}
% \entry{女儿}{n.}{filha}
% \entry{女孩儿}{n.}{menina; rapariga}
% \entry{牛}{n.}{boi; vaca}
% \entry{牛肉}{n.}{carne de vaca}
\end{multicols}

%%%%
%%% O
%%%
\section*{O}
\addcontentsline{toc}{section}{O}
\begin{multicols}{2}

\entry{欧洲}{n.}{Ou1zhou1}{Europa}

\end{multicols}

%%%%
%%% P
%%%
\section*{P}
\addcontentsline{toc}{section}{P}
\begin{multicols}{2}
\entry{爬}{v.}{pa2}{escalar; trepar}

\entry{怕}{v.}{pa4}{ter medo de}

\entry{排球}{n.}{pai2qiu2}{voleibol}

\entry{盘}{p.c.}{pan2}{para cassete, video-cassete}

\entry{胖}{adj.}{pang4}{gordo}

\entry{跑步}{v.}{pao3bu4}{correr}

\entry{配}{v.}{pei3}{combinar}

\entry{朋友}{n.}{peng2you0}{amigo; amiga|namorado; namorada}

\entry{啤酒}{n.}{pi2jiu3}{cerveja}
\entry{啤酒馆}{n.}{pi2jiu3guan3}{cervejaria}

\entry{漂亮}{adj.}{piao4liang0}{linda|bonito;lindo (para objetos inanimados)}

\entry{瓶}{n.}{ping2}{garrafa}
\entry{瓶}{p.c.}{ping2}{palavra classificadora, garrafa}

\entry{平时}{p.t.}{ping2shi2}{normalmente; numa época normal}

\entry{苹果}{n.}{ping2guo3}{maçã}

\entry{葡汉词典}{n.}{pu2han4ci2dian3}{dicionário português-chinês}
\entry{葡萄牙}{n.}{Pu2tao2ya2}{Portugal}
\entry{葡萄牙语}{n.}{Pu2tao2ya2yu3}{português; língua portuguesa}
\entry{葡语}{n.}{Pu2yu3}{português; língua portuguesa}
\entry{葡文}{n.}{Pu2wen2}{português; língua portuguesa}

\entry{普通话}{n.}{pu3tong1hua4}{mandarim (lit. ``linguagem comum'')}

\entry{便宜}{adj.}{pian2yi0}{barato}

\entry{乒乓球}{n.}{ping1pang1qui2}{tênis de mesa; ping-pong}

\end{multicols}

%%%%
%%% Q
%%%
\section*{Q}
\addcontentsline{toc}{section}{Q}
\begin{multicols}{2}

\entry{七}{num.}{qi1}{sete}

\entry{起床}{v.}{qi3chuang2}{levantar-se}

\entry{千}{num.}{qian1}{mil}

\entry{钱}{n.}{qian2}{dinheiro}

\entry{前年}{p.t.}{qian2nian2}{há dois anos}
\entry{前面}{p.l.}{qian2mian4}{frente; em frente; na frente}
\entry{前天}{p.t.}{qian2tian1}{anteontem}

\entry{钱包}{n.}{qian2bao1}{carteira}

\entry{强}{adj.}{qiang2}{forte}

\entry{巧克力}{n.}{qiao3ke4li4}{chocolate}

\entry{青菜}{n.}{qing1cai4}{verduras}
\entry{青椒}{n.}{qing1jiao1}{pimenta verde}

\entry{请}{v.}{qing3}{fazer o favor}
\entry{请问}{}{qing3wen4}{Desculpe...|para perguntar por qualquer coisa}

\entry{秋天}{n.}{qiu1tian1}{outono}

\entry{球}{n.}{qiu2}{bola (futebol; basquetebol; handbol; etc)}

\entry{曲棍球}{n.}{qu1gun4qiu2}{hóquei em campo}

\entry{去}{v.}{qu4}{ir}
\entry{去年}{n.}{qu4nian2}{ano passado}

\entry{裙子}{n.}{qun2zi0}{saia; vestido}

\end{multicols}

%%%%
%%% R
%%%
\section*{R}
\addcontentsline{toc}{section}{R}
\begin{multicols}{2}
% \entry{让}{v.}{deixar; permitir}
% \entry{热}{adj.}{calor; quente}
% \entry{热\xpinyin{闹}{nao0}}{adj.}{animado}
\entry{人}{n.}{ren2}{pessoa}
% \entry{认识}{v.}{conhecer}
% \entry{\xpinyin{日}{Ri4}本}{n.}{Japão}
% \entry{如果}{conj.}{se; caso}
% \entry{肉}{n.}{carne}
% \entry{人民币}{n.}{nome da moeda chinesa}
\end{multicols}

%%%%
%%% S
%%%
\section*{S}
\addcontentsline{toc}{section}{S}
\begin{multicols}{2}
% \entry{三}{num.}{três}
% \entry{山}{n.}{montanha; monte}
% \entry{山区}{n.}{área montanhosa}
% \entry{商店}{n.}{loja}
% \entry{上}{p.l.}{acima; em cima; subir}
% \entry{\xpinyin{上}{Shang4}海}{}{Shangai (Xangai)}
% \entry{上课}{}{ter aulas}
% \entry{上面}{n.}{parte de cima}
% \entry{上午}{p.t.}{manhã; de manhã; período antes do meio-dia}
% \entry{少}{adj.}{pouco; poucos}
% \entry{谁}{interr.}{quem}
% \entry{身体}{n.}{corpo}
% \entry{什么}{interr.}{que; o que}
% \entry{什么时\xpinyin{候}{hou}}{interr.}{quando; a que horas}
% \entry{生\xpinyin{日}{ri0}}{n.}{aniversário}
% \entry{\xpinyin{圣}{Sheng4}诞节}{n.}{Natal}
% \entry{十}{num.}{dez; dezena}
% \entry{时候}{interr.}{quando}
% \entry{时间}{n.}{tempo}
% \entry{事}{n.}{assunto}
\entry{是}{v.}{shi4}{ser}
% \entry{是的}{}{sim}
% \entry{收到}{}{receber}
% \entry{书}{n.}{livro}
% \entry{舒\xpinyin{服}{fu0}}{adj.}{bem disposto; (sentir-se) bem}
% \entry{暑假}{n.}{férias de verão}
% \entry{树木}{n.}{árvore}
% \entry{\xpinyin{谁}{shui2}}{pron.}{quem}
% \entry{睡觉}{v.}{ir para a cama; dormir}
% \entry{水}{n.}{água}
% \entry{水果}{n.}{fruta}
% \entry{睡\xpinyin{觉}{jiao4}}{v.}{dormir}
% \entry{说}{v.}{falar; dizer}
% \entry{四}{num.}{quatro}
% \entry{送}{v.}{distribuir; entregar}
% \entry{岁}{n.}{anos de idade}
% \entry{试}{v.}{experimentar; provar}
% \entry{酸}{adj.}{ácido; ácida; avinagrado; avinagrada}
% \entry{酸辣汤}{n.}{sopa avinagrada e picante}
% \entry{睡觉(饺子)}{n.}{raviólis chineses}
% \entry{时\xpinyin{候}{hou0}}{n.}{horas; tempo}
% \entry{上网}{v.}{acessar a Internet}
\end{multicols}

%%%%
%%% T
%%%
\section*{T}
\addcontentsline{toc}{section}{T}
\begin{multicols}{2}

\entry{它}{pron.}{ta1}{ele; ela|objetos e animais}
\entry{它们}{pron.}{ta1men0}{eles; elas|objetos e animais}

\entry{她}{pron.}{ta1}{ela}
\entry{她的}{pron.}{ta1de0}{dela}
\entry{她们}{pron.}{ta1men0}{elas}
\entry{她们的}{pron.}{ta1men0de0}{delas}

\entry{他}{pron.}{ta1}{ele}
\entry{他的}{pron.}{ta1de0}{dele}
\entry{他们}{pron.}{ta1men0}{eles}
\entry{他们的}{pron.}{ta1men0de0}{deles}

\entry{台}{p.c.}{tai2}{para computador, gravador, gravador de vídeo, etc}

\entry{太}{adv.}{tai4}{excessivamente; demais; muito}
\entry{太太}{n.}{tai4tai0}{esposa; mulher}

\entry{汤}{n.}{tang1}{sopa; caldo}

\entry{糖}{n.}{tang2}{açúcar}
\entry{糖醋鱼}{n.}{tang2cu4yu2}{peixe guisado em molho agridoce}

\entry{特别}{adv.}{te4bie2}{especialmente}

\entry{疼}{v.}{teng2}{doer}

\entry{踢}{v.}{ti1}{jogar; dar pontapés em}

\entry{天}{n.}{tian1}{dia}
\entry{天气}{n.}{tian1qi4}{clima; tempo}

\entry{甜}{adj.}{tian2}{doce}

\entry{条}{p.c.}{tiao2}{para calças, saia, rio, peixe, etc}

\entry{听}{v.}{ting1}{ouvir; escutar}

\entry{同学}{n.}{tong2xue2}{aluno; colega de classe}

\entry{头}{n.}{tou2}{cabeça}
\entry{头发}{n.}{tou2fa0}{cabelo}

\entry{腿}{n.}{tui3}{perna}

\end{multicols}

%%\section*{U}
%\addcontentsline{toc}{section}{U}
%\begin{multicols}{2}
%%%
%%% U
%%%
%\end{multicols}

%%%%
%%% V
%%%
%\section*{V}
%\addcontentsline{toc}{section}{V}
%\begin{multicols}{2}
%\end{multicols}

%%%%
%%% W
%%%
\section*{W}
\addcontentsline{toc}{section}{W}
\begin{multicols}{2}

\entry{外号}{n.}{wai4hao4}{apelido}
\entry{外边}{n.}{wai4bian0}{fora; por fora; exterior}
\entry{外公}{n.}{wai4gong1}{avô materno}
\entry{外面}{n.}{wai4mian0}{fora; por fora; exterior}
\entry{外婆}{n.}{wai4po2}{avó materna}
\entry{外孙}{n.}{wai4sun1}{filho da filha}
\entry{外孙女}{n.}{wai4sun1nv3}{filha da filha}

\entry{玩}{v.}{wan2}{brincar; tocar (intrumento musical)}
\entry{玩儿}{v.}{wanr2}{divertir-se}

\entry{完}{v.}{wan2}{acabar; palavras}

\entry{晚饭}{n.}{wan3fan4}{jantar}
\entry{晚上}{p.t.}{wan3shang0}{noite; à noite}

\entry{碗}{n}{wan3}{tigela}
\entry{碗}{p.c.}{wan3}{palavra classificadora, tigela}
\entry{碗子}{n}{wan3zi0}{tigela}

\entry{万}{num.}{wan4}{dez mil}

\entry{往}{prep.}{wang3}{para; em direção a}

\entry{网球}{n.}{wang3qui2}{tênis (esporte)}

\entry{喂}{interjeição}{wei2}{ei, chamar atenção (alô, telefone)}
\entry{喂}{interjeição}{wei4}{ei, chamar atenção (alô, telefone)}

\entry{为}{prep.}{wei4}{para}

\entry{位}{p.c.}{wei4}{para pessoas (com cortesia)}

\entry{味道}{n.}{wei4dao0}{sabor}

\entry{为什么}{interr.}{wei4shen2me0}{por que?}

\entry{卫生间}{n.}{wei4sheng1jian1}{banheiro; toilette}

\entry{问}{v.}{wen4}{perguntar}
\entry{问题}{n.}{wen4ti2}{pergunta}

\entry{我}{pron.}{wo3}{eu}
\entry{我的}{pron.}{wo3de0}{meu(s); minha(s)}
\entry{我们}{pron.}{wo3men0}{nós}
\entry{我们的}{pron.}{wo3men0de0}{nosso(s); nossa(s)}

\entry{午饭}{n.}{wu3fan4}{almoço}

\entry{五}{num.}{wu3}{cinco}

\end{multicols}

%%%%
%%% X
%%%
\section*{X}
\addcontentsline{toc}{section}{X}
\begin{multicols}{2}
% \entry{西部}{n.}{oeste}
\entry{西方}{n.}{Xi1fang1}{Ocidente}
\entry{悉尼}{n.}{Xi1ni2}{Sidney}
\entry{西语}{n.}{xi1yu3}{espanhol; língua espanhola}
\entry{西文}{n.}{xi1wen2}{espanhol; língua espanhola}
\entry{喜欢}{v.}{xi3huan0}{gostar}
\entry{系}{n.}{xi4}{faculdade (da universidade)}
% \entry{下}{p.l.}{abaixo; em baixo}
% \entry{下面}{p.l.}{parte de baixo}
% \entry{夏天}{n.}{verão}
% \entry{下午}{p.t.}{tarde; período logo após o meio-dia}
% \entry{下雨}{v.}{chover}
\entry{先生}{n.}{xian1sheng0}{senhor; marido}
% \entry{先生}{pron.}{senhor(a)}
% \entry{现在}{p.t.}{agora}
\entry{想}{v./v.o.}{xiang3}{pensar; querer; achar}
% \entry{向}{prep.}{para}
\entry{小}{adj.}{xiao3}{pequeno; pequena}
% \entry{小i\xpinyin{姐}{jie0}}{n.}{senhorita; empregada}
\entry{小学}{n.}{xiao3xue2}{escola ensino fundamental}
% \entry{校长}{n.}{diretor de escola; reitor (universidade)}
% \entry{些}{adv.}{uns; umas; alguns; algumas}
\entry{写}{v.}{xie3}{escrever}
\entry{谢谢}{v.}{xie4xie0}{agradecer; obrigado; obrigada}
% \entry{新年}{n.}{Ano Novo}
% \entry{星期}{n.}{semana}
% \entry{行}{v.}{claro que sim; de acordo; está bem}
\entry{信}{n.}{xin4}{carta}
\entry{姓}{v./n.}{xing4}{ter o sobrenome/sobrenome}
% \entry{休\xpinyin{息}{xi0}}{v.}{descansar}
\entry{学}{v.}{xue2}{estudar}
\entry{学生}{n.}{xue2sheng0}{estudante; aluno; aluna}
\entry{学习}{v.}{xue2xi2}{estudar; aprender}
\entry{学校}{n.}{xue2xiao4}{escola; instituição de ensino}
\entry{学院}{n.}{xue2yuan4}{instituto}
% \entry{咸}{adj.}{salgado; salgada}
% \entry{\xpinyin{事}{shir4}\xpinyin{儿}{}}{n.}{afazeres; assunto; coisa; matéria}
\end{multicols}

%\section*{Y}
\addcontentsline{toc}{section}{Y}
\begin{multicols}{2}
%%%
%%% Y
%%%
% % % \entry{压岁钱}{n.}{dinheiro dado às crianças como presente no Ano Novo Chinês}
% % % \entry{牙}{n.}{dente}
% % % \entry{\xpinyin{亚}{Ya4}洲}{n.}{Ásia}
% % % \entry{颜色}{n.}{cor}
% % % \entry{样子}{n.}{aparência; forma}
% % % \entry{药}{n.}{remédio}
% % % \entry{要}{v./v.o.}{querer; precisar}
% % % \entry{要是}{conj.}{se; caso}
% % % \entry{爷\xpinyin{爷}{ye0}}{n.}{avô (paterno)}
% % % \entry{也}{adv.}{também}
% % \entry{一}{num.}{um; uma}
% % \entry{衣\xpinyin{服}{fu0}}{n.}{roupa; vestiário}
% % \entry{医生}{n.}{médico}
% % \entry{\xpinyin{一}{yi2}共}{adv.}{tudo; no local}
% % \entry{\xpinyin{一}{yi2}下}{}{rapidamente; em um curto tempo}
% % \entry{以后}{n.}{depois}
% % \entry{\xpinyin{一}{yi4}起}{adv.}{juntamente; em conjunto}
% % \entry{亿}{num.}{cem milhões}
% % \entry{意思}{n.}{interesse}
% % \entry{因为}{conj.}{porque}
% % \entry{音\xpinyin{乐}{yue4}}{n.}{música}
% % \entry{饮料}{n.}{bebida}
% % \entry{应该}{v.}{dever}
% % \entry{\xpinyin{英}{Ying1}国}{n.}{Reino Unido}
% % \entry{英语}{n.}{chinês; língua chinesa}
% % \entry{英文}{n.}{inglês; língua inglesa}
% % \entry{邮件}{n.}{correio}
% \entry{游泳}{v.}{nadar}
% \entry{游泳池}{n.}{piscina}
% \entry{有}{v.}{ter; haver}
% \entry{有的}{pron.}{algum; alguma; alguns; algumas}
% \entry{有\xpinyin{点}{dianr3}\xpinyin{儿}{}}{adv.}{um pouco}
% \entry{有意思}{adj.}{interessante}
% \entry{有用}{adj.}{útil}
% \entry{右}{n.}{direita}
% \entry{右\xpinyin{边}{bian}}{n.}{direita}
% \entry{用}{v.}{usar}
% \entry{雨伞}{n.}{guarda-chuva}
% \entry{雨衣}{n.}{impermeável}
% \entry{语言实验室}{n.}{laboratório de línguas}
% \entry{元}{p.c.}{palavra classificadora, unidade monetária da China}
% \entry{月}{n.}{mês}
% \entry{运动}{n.}{esporte; desporto}
% \entry{运动场}{n.}{campo desportivo; campo de jogos}
% \entry{运动会}{n.}{jogos desportivos}
% \entry{运动员}{n.}{jogador; jogadora; atleta}
% \entry{鱼}{n.}{peixe}
% \entry{颜色}{n.}{cor}
% \entry{星期六}{p.t.}{sábado}
% \entry{星期}{n.}{semana}
% \entry{约会}{n.}{compromisso; encontro marcado}
% \entry{羽毛球}{n.}{badminton}
% \entry{夜里}{p.t.}{noite}
\end{multicols}

%%%%
%%% Z
%%%
\section*{Z}
\addcontentsline{toc}{section}{Z}
\begin{multicols}{2}
\entry{在}{prep.}{zai4}{em}
\entry{在}{v.}{zai4}{estar; ficar}

\entry{再}{adv.}{zai4}{de novo; outra vez}
\entry{再见}{v.}{zai4jian4}{adeus; até à vista; até à próxima; até logo}

\entry{早上}{p.t.}{zao3shang0}{manhã cedo; manhãzinha}

\entry{怎么}{interr.}{zen3me0}{como}
\entry{怎么样}{interr.}{zen3me0yang4}{como; que tal}

\entry{张}{p.c.}{zhang1}{para folha de papéis, mapa, etc}

\entry{找}{v.}{zhao3}{andar à procura de; procurar; dar troco}

\entry{这}{pron.}{zhe4}{este; esta; isto}
\entry{这儿}{pron.}{zher4}{aqui}
\entry{这里}{pron.}{zhe4li0}{aqui}

\entry{浙江}{n.}{Zhe4jiang1}{Zhejiang}

\entry{真}{adv.}{zhen1}{que...tão...!; realmente}

\entry{挣钱}{v.+compl.}{zheng4qian2}{ganhar dinheiro}

\entry{支}{p.c.}{zhi1}{para caneta, lápis, etc}

\entry{知道}{v.}{zhi1dao0}{conhecer; saber}

\entry{职员}{n.}{zhi2yuan2}{empregado; empregada}

\entry{只}{adv.}{zhi3}{apenas; só}

\entry{钟}{p.c.}{zhong1}{hora}

\entry{中国}{n.}{Zhong1guo2}{China}
\entry{中文}{n.}{zhong1wen2}{chinês; língua chinesa}
\entry{中学}{n.}{zhong1xue2}{escola ensino médio}
\entry{中学生}{n.}{zhong1xue2sheng1}{estudante da escola ensino médio}
\entry{中询}{p.t.}{zhong1xun2}{segunda dezena do mês; meio do mês; em meados do mês}

\entry{种}{p.c.}{zhong3}{palavra classificadora, tipo}

\entry{重量}{n.}{zhong4liang4}{peso}

\entry{猪}{n.}{zhu1}{porco; porca}

\entry{住}{v.}{zhu4}{morar}

\entry{祝}{v.}{zhu4}{desejar}

\entry{嘱咐}{v.}{zhu4fu4}{ordenar; dizer; exortar}

\entry{桌子}{n.}{zhuo1zi0}{mesa}

\entry{紫色}{n.}{zi3se4}{cor roxa}

\entry{走}{v.}{zou3}{andar; ir}

\entry{足球}{n.}{zu2qiu2}{futebol}

\entry{最}{adv.}{zui4}{o mais; a mais|grau superlativo relativo de superioridade}
\entry{最近}{adv.}{zui4jin4}{ultimamente; recentemente}

\entry{昨天}{p.t.}{zuo2tian1}{ontem}

\entry{左}{n.}{zuo3}{esquerda}

\entry{坐}{v.}{zuo4}{sentar-se}

\entry{做}{v.}{zuo4}{fazer}

\end{multicols}


\printindex

\end{document}
