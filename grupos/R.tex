%%%
%%% R
%%%
%\section*{R}
\addcontentsline{toc}{section}{R}

\begin{verbete}{儿}{r5}{2}[Radical 儿][Componentes ⼃乚]
  \significado{conj.}{sufixo diminutivo não silábico; final retroflexo}
  \veja{儿}{er2}
  \veja{儿}{ren2}
\end{verbete}

\begin{verbete}{然}{ran2}{12}[Radical 火][Componentes 冫灬犬𠂊]
  \significado{conj.}{mas; no entanto}
\end{verbete}

\begin{verbete}{然而}{ran2'er2}{12;6}
  \significado{conj.}{mas; no entanto}
\end{verbete}

\begin{verbete}{然后}{ran2hou4}{12;6}
  \significado{conj.}{depois; logo; portanto}
\end{verbete}

\begin{verbete}{燃烧}{ran2shao1}{16;10}
  \significado{s.}{combustão; flama}
  \significado{v.}{queimar; acender}
\end{verbete}

\begin{verbete}{壤}{rang3}{20}[Radical 土][Componentes 土襄]
  \significado{s.}{solo; terra; (literário) a terra (em contraste com o céu 天)}
\end{verbete}

\begin{verbete}{让}{rang4}{5}[Radical 言][Componentes 讠上]
  \significado{v.}{deixar alguém fazer alguma coisa; fazer alguém (sentir-se triste, etc.); permitir; conceder}
\end{verbete}

\begin{verbete}{热}{re4}{10}[Radical 火][Componentes 执灬]
  \significado{adj.}{quente (clima); fervente; ardente; fervoroso}
  \significado{v.}{aquecer; ferver}
\end{verbete}

\begin{verbete}{热爱}{re4'ai4}{10;10}
  \significado{v.}{amar ardentemente; adorar}
\end{verbete}

\begin{verbete}{热泪盈眶}{re4lei4ying2kuang4}{10;8;9;11}
  \significado{expr.}{olhos cheios de lágrimas de emoção; extremamente emocionado}
\end{verbete}

\begin{verbete}{热闹}{re4nao5}{10;8}
  \significado{adj.}{animado; movimentado com barulho e excitação}
\end{verbete}

\begin{verbete}{热心}{re4xin1}{10;4}
  \significado{adj.}{entusiasmado, ardente, zeloso}
\end{verbete}

\begin{verbete}{热血沸腾}{re4xue4fei4teng2}{10;6;8;13}
  \significado{expr.}{ferver o sangue; apaixonar-se}
\end{verbete}

\begin{verbete}{人}{ren2}{2}[Radical 人][Componentes ㇒][Kangxi 9]
  \significado[个,位]{s.}{pessoa; gente}
\end{verbete}

\begin{verbete}{人才}{ren2cai2}{2;3}
  \significado{s.}{talento; pessoa talentosa}
\end{verbete}

\begin{verbete}{人材}{ren2cai2}{2;7}
  \variante{人才}
\end{verbete}

\begin{verbete}{人道}{ren2dao4}{2;12}
  \significado{s.}{solidariedade humana; humanitarismo; humano; a ``maneira humana'', um dos estágios do ciclo de reencarnação (budismo); relação sexual}
\end{verbete}

\begin{verbete}{人海}{ren2hai3}{2;10}
  \significado{s.}{uma multidão; um mar de pessoas}
\end{verbete}

\begin{verbete}{人间}{ren2jian1}{2;7}
  \significado{s.}{o mundo humano; a Terra}
\end{verbete}

\begin{verbete}{人口}{ren2kou3}{2;3}
  \significado{s.}{pessoas; população}
\end{verbete}

\begin{verbete}{人类}{ren2lei4}{2;9}
  \significado{s.}{humanidade; raça humana}
\end{verbete}

\begin{verbete}{人民}{ren2min2}{2;5}
  \significado[个]{s.}{povo; população}
\end{verbete}

\begin{verbete}{人民币}{ren2min2bi4}{2;5;4}
  \significado*{s.}{Renminbi (RMB); Yuan Chinês (CYN); nome da moeda chinesa}
\end{verbete}

\begin{verbete}{人权}{ren2quan2}{2;6}
  \significado*{s.}{Direitos Humanos}
  \veja{人权法}{ren2quan2fa3}
\end{verbete}

\begin{verbete}{人权法}{ren2quan2fa3}{2;6;8}
  \significado*{s.}{Direitos Humanos}
  \veja{人权}{ren2quan2}
\end{verbete}

\begin{verbete}{人生}{ren2sheng1}{2;5}
  \significado{s.}{vida (tempo de alguém na Terra)}
\end{verbete}

\begin{verbete}{人像}{ren2xiang4}{2;13}
  \significado{s.}{``retrato'' de uma pessoa (esboço, foto, escultura, etc.)}
\end{verbete}

\begin{verbete}{人行道}{ren2xing2dao4}{2;6;12}
  \significado{s.}{calçada}
\end{verbete}

\begin{verbete}{人鱼}{ren2yu2}{2;8}
  \significado{s.}{sereia; peixe-boi; salamandra gigante}
\end{verbete}

\begin{verbete}{儿}{ren2}{2}[Radical 儿][Componentes ⼃乚]
  \significado{s.}{pessoa, radical em caracteres chineses}
  \variante{人}
  \veja{儿}{er2}
  \veja{儿}{r5}
\end{verbete}

\begin{verbete}{忍耐}{ren3nai4}{7;9}
  \significado{s.}{paciência; resistência}
  \significado{v.}{suportar; resistir; exercer paciência}
\end{verbete}

\begin{verbete}{认识}{ren4shi5}{4;7}
  \significado{s.}{conhecimento; saber; entendimento}
  \significado{v.}{estar familiarizado com; conhecer alguém; saber; reconhecer}
\end{verbete}

\begin{verbete}{认真}{ren4zhen1}{4;10}
  \significado{adj.}{sério; consciencioso}
  \significado{adv.}{seriamente}
  \significado{v.}{levar a sério}
\end{verbete}

\begin{verbete}{任务}{ren4wu5}{6;5}
  \significado[项,个]{s.}{missão, atribuição, tarefa, obrigação, papel}
\end{verbete}

\begin{verbete}{扔}{reng1}{5}[Radical 手][Componentes 扌乃]
  \significado{v.}{lançar; atirar}
\end{verbete}

\begin{verbete}{扔掉}{reng1diao4}{5;11}
  \significado{v.}{jogar fora}
\end{verbete}

\begin{verbete}{扔弃}{reng1qi4}{5;7}
  \significado{v.}{abandonar; descartar; jogar fora}
\end{verbete}

\begin{verbete}{扔下}{reng1xia4}{5;3}
  \significado{v.}{lançar (uma bomba); derrubar}
\end{verbete}

\begin{verbete}{仍然}{reng2ran2}{4;12}
  \significado{adv.}{ainda}
\end{verbete}

\begin{verbete}{日}{ri4}{4}[Radical 日][Componentes 口一][Kangxi 72]
  \significado*{s.}{Japão, abreviação de~日本}
  \significado{p.c.}{dia (mais usado em escrita); data, dia do mês}
  \veja{日本}{ri4ben3}
\end{verbete}

\begin{verbete}{日本}{ri4ben3}{4;5}
  \significado*{s.}{Japão}
  \veja{日}{ri4}
\end{verbete}

\begin{verbete}{日本人}{ri4ben3ren2}{4;5;2}
  \significado{s.}{japonês; nascido no Japão}
\end{verbete}

\begin{verbete}{日常}{ri4chang2}{4;11}
  \significado{adv.}{diariamente; dia-a-dia; todo dia}
\end{verbete}

\begin{verbete}{日出}{ri4chu1}{4;5}
  \significado{s.}{nascer do sol}
  \veja{夕阳}{xi1yang2}
\end{verbete}

\begin{verbete}{日光灯}{ri4guang1deng1}{4;6;6}
  \significado{s.}{lâmpada fluorescente}
\end{verbete}

\begin{verbete}{日子}{ri4zi5}{4;3}
  \significado{s.}{dia; uma data (calendário); dias de vida de alguém}
\end{verbete}

\begin{verbete}{容貌}{rong2mao4}{10;14}
  \significado{s.}{aparência; aspecto; características}
\end{verbete}

\begin{verbete}{容易}{rong2yi4}{10;8}
  \significado{adj.}{fácil; responsável (por); provável}
\end{verbete}

\begin{verbete}{柔软}{rou2ruan3}{9;8}
  \significado{adj.}{macio; suave}
\end{verbete}

\begin{verbete}{揉}{rou2}{12}[Radical 手][Componentes 扌柔]
  \significado{v.}{amassar; massagear; esfregar}
\end{verbete}

\begin{verbete}{揉碎}{rou2sui4}{12;13}
  \significado{v.}{esmagar; desintegrar-se em pedaços}
\end{verbete}

\begin{verbete}{肉}{rou4}{6}[Radical 肉][Componentes 仌冂][Kangxi 130]
  \significado{s.}{carne; polpa de uma fruta}
\end{verbete}

\begin{verbete}{肉桂}{rou4gui4}{6;10}
  \significado{s.}{canela}
  \veja{官桂}{guan1gui4}
\end{verbete}

\begin{verbete}{如}{ru2}{6}[Radical 女][Componentes 女口]
  \significado{conj.}{por exemplo}
\end{verbete}

\begin{verbete}{如此}{ru2ci3}{6;6}
  \significado{adv.}{assim, então, tal}
\end{verbete}

\begin{verbete}{如果}{ru2guo3}{6;8}
  \significado{conj.}{se; caso; no caso de; no evento de; supondo que}
\end{verbete}

\begin{verbete}{如画}{ru2hua4}{6;8}
  \significado{adj.}{pitoresco}
\end{verbete}

\begin{verbete}{儒教}{ru2jiao4}{16;11}
  \significado*{s.}{Confucionismo}
\end{verbete}

\begin{verbete}{乳房}{ru3fang2}{8;8}
  \significado{s.}{seio; mama; úbere}
\end{verbete}

\begin{verbete}{辱骂}{ru3ma4}{10;9}
  \significado{v.}{insultar; abusar}
\end{verbete}

\begin{verbete}{入党}{ru4dang3}{2;10}
  \significado{v.}{ingressar em um partido político (especialmente o partido comunista)}
\end{verbete}

\begin{verbete}{入乡随俗}{ru4xiang1-sui2su2}{2;3;11;9}
  \significado{expr.}{Em roma, faça como os romanos!}
\end{verbete}

\begin{verbete}{软件}{ruan3jian4}{8;6}
  \significado{v.}{\emph{software}}
\end{verbete}

%%%%% EOF %%%%%
