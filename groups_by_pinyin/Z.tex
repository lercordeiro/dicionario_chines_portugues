%%%
%%% Z
%%%

\section*{Z}\addcontentsline{toc}{section}{Z}

\begin{entry}{杂技}{za2ji4}{6,7}{⽊、⼿}
  \definition[场]{s.}{acrobacia}
\end{entry}

\begin{entry}{杂志}{za2zhi4}{6,7}{⽊、⼼}[HSK 3]
  \definition[本,份,期]{s.}{diário; revista}
\end{entry}

\begin{entry}{杂志社}{za2zhi4she4}{6,7,7}{⽊、⼼、⽰}
  \definition{s.}{editora de revista}
\end{entry}

\begin{entry}{咱家}{za2jia1}{9,10}{⼝、⼧}
  \definition{pron.}{eu (frequentemente usado na literatura vernácula antiga) | me | mim | comigo}
\end{entry}

\begin{entry}{砸}{za2}{10}{⽯}
  \definition{v.}{esmagar | bater | falhar | estragar}
\end{entry}

\begin{entry}{栽}{zai1}{10}{⽊}
  \definition{v.}{cultivar | plantar}
\end{entry}

\begin{entry}{栽倒}{zai1dao3}{10,10}{⽊、⼈}
  \definition{v.}{cair | sofrer uma queda}
\end{entry}

\begin{entry}{栽培}{zai1pei2}{10,11}{⽊、⼟}
  \definition{v.}{cultivar | educar | patrocinar | treinar}
\end{entry}

\begin{entry}{栽培种}{zai1pei2 zhong3}{10,11,9}{⽊、⼟、⽲}
  \definition{s.}{espécies cultivadas}
\end{entry}

\begin{entry}{栽赃}{zai1zang1}{10,10}{⽊、⾙}
  \definition{v.}{enquadrar alguém (plantar provas nele)}
\end{entry}

\begin{entry}{栽植}{zai1zhi2}{10,12}{⽊、⽊}
  \definition{v.}{plantar | transplantar}
\end{entry}

\begin{entry}{栽种}{zai1zhong4}{10,9}{⽊、⽲}
  \definition{v.}{plantar}
\end{entry}

\begin{entry}{再}{zai4}{6}{⼌}[HSK 1]
  \definition{adv.}{de novo | outra vez | uma segunda vez | não importa como\dots (seguido por um adjetivo ou verbo, e então (normalmente) 也 ou 都 para dar ênfase)}
\end{entry}

\begin{entry}{再不}{zai4bu4}{6,4}{⼌、⼀}
  \definition{adv.}{nunca mais}
\end{entry}

\begin{entry}{再读}{zai4du2}{6,10}{⼌、⾔}
  \definition{v.}{ler novamente | rever (uma lição, etc.)}
\end{entry}

\begin{entry}{再度}{zai4du4}{6,9}{⼌、⼴}
  \definition{adv.}{outra vez | mais uma vez}
\end{entry}

\begin{entry}{再发}{zai4fa1}{6,5}{⼌、⼜}
  \definition{v.}{reenviar}
\end{entry}

\begin{entry}{再见}{zai4jian4}{6,4}{⼌、⾒}[HSK 1]
  \definition{v.}{adeus | até à vista | até à próxima | até logo}
\end{entry}

\begin{entry}{再临}{zai4lin2}{6,9}{⼌、⼁}
  \definition{v.}{vir de novo}
\end{entry}

\begin{entry}{再三}{zai4san1}{6,3}{⼌、⼀}[HSK 4]
  \definition{adv.}{repetidamente; repetidas vezes; de novo e de novo}
\end{entry}

\begin{entry}{再审}{zai4shen3}{6,8}{⼌、⼧}
  \definition{s.}{novo julgamento | revisão}
  \definition{v.}{ouvir um caso novamente}
\end{entry}

\begin{entry}{再生}{zai4sheng1}{6,5}{⼌、⽣}
  \definition{s.}{reciclagem | regeneração}
  \definition{v.}{reciclar | renascer | regenerar}
\end{entry}

\begin{entry}{再说}{zai4shuo1}{6,9}{⼌、⾔}
  \definition{conj.}{além do mais | além disso | o que mais}
  \definition{v.}{adiar uma discussão para mais tarde | dizer novamente}
\end{entry}

\begin{entry}{再育}{zai4yu4}{6,8}{⼌、⾁}
  \definition{v.}{aumentar | multiplicar | proliferar}
\end{entry}

\begin{entry}{再者}{zai4zhe3}{6,8}{⼌、⽼}
  \definition{conj.}{além do mais | além disso}
\end{entry}

\begin{entry}{在}{zai4}{6}{⼟}[HSK 1]
  \definition{adv.}{para designar ações que estão passando | durante}
  \definition{prep.}{em}
  \definition{v.}{estar | ficar}
\end{entry}

\begin{entry}{在此}{zai4ci3}{6,6}{⼟、⽌}
  \definition{adv.}{aqui}
\end{entry}

\begin{entry}{在地}{zai4di4}{6,6}{⼟、⼟}
  \definition{s.}{local}
\end{entry}

\begin{entry}{在行}{zai4hang2}{6,6}{⼟、⾏}
  \definition{v.}{ser adepto de algo | ser um especialista em um comércio ou profissão}
\end{entry}

\begin{entry}{在乎}{zai4hu5}{6,5}{⼟、⼃}[HSK 4]
  \definition{v.}{preocupar-se; preocupar-se com; levar a sério | ser responsável por; caber ao; ser da competência de}
\end{entry}

\begin{entry}{在家}{zai4jia1}{6,10}{⼟、⼧}[HSK 1]
  \definition{v.}{estar em casa | permanecer um leigo}
\end{entry}

\begin{entry}{在教}{zai4jiao4}{6,11}{⼟、⽁}
  \definition{v.}{ser um crente (em uma religião)}
\end{entry}

\begin{entry}{在下}{zai4xia4}{6,3}{⼟、⼀}
  \definition{pron.}{eu mesmo (humildemente)}
\end{entry}

\begin{entry}{在线}{zai4xian4}{6,8}{⼟、⽷}
  \definition{s.}{\emph{online}}
\end{entry}

\begin{entry}{在意}{zai4yi4}{6,13}{⼟、⼼}
  \definition{v.+compl.}{preocupar-se | importar-se | levar a sério}
\end{entry}

\begin{entry}{在于}{zai4yu2}{6,3}{⼟、⼆}[HSK 4]
  \definition{v.}{ser responsável por; caber a;  ser da competência de;  apontar a essência das coisas, ou do que elas se tratam | depender de; ser determinado por;  ser devido a (um determinado atributo)/(de um assunto a ser determinado)}
\end{entry}

\begin{entry}{咱}{zan2}{9}{⼝}[HSK 2]
  \definition{pron.}{eu}
\end{entry}

\begin{entry}{咱俩}{zan2lia3}{9,9}{⼝、⼈}
  \definition{pron.}{nós dois}
\end{entry}

\begin{entry}{咱们}{zan2men5}{9,5}{⼝、⼈}[HSK 2]
  \definition{pron.}{nós (incluindo o orador e a(s) pessoa(s) com quem se fala)}
\end{entry}

\begin{entry}{赞}{zan4}{16}{⾙}
  \definition{v.}{patrocinar | apoiar | elogiar | (gíria na \emph{Internet}) para curtir (uma postagem \emph{on-line})}
\end{entry}

\begin{entry}{赞成}{zan4cheng2}{16,6}{⾙、⼽}[HSK 4]
  \definition{v.}{endossar; favorecer; aprovar; concordar com; concordar ou apoiar as ideias, os planos, as propostas ou o comportamento de outra pessoa}
\end{entry}

\begin{entry}{赞赏}{zan4 shang3}{16,12}{⾙、⾙}[HSK 4]
  \definition{v.}{admirar; apreciar; valorizar}
\end{entry}

\begin{entry}{赞扬}{zan4yang2}{16,6}{⾙、⼿}
  \definition{v.}{elogiar | aprovar | demonstrar aprovação}
\end{entry}

\begin{entry}{赞助}{zan4zhu4}{16,7}{⾙、⼒}[HSK 4]
  \definition{s.}{patrocinador}
  \definition{v.}{apoiar; patrocinar; concordar e ajudar (refere-se principalmente a oferecer dinheiro para ajudar)}
\end{entry}

\begin{entry}{脏}{zang1}{10}{⾁}[HSK 2]
  \definition{adj.}{sujo | imundo}
  \seeref{脏}{zang4}
\end{entry}

\begin{entry}{脏辫}{zang1bian4}{10,17}{⾁、⾟}
  \definition{s.}{\emph{dreadlocks}}
\end{entry}

\begin{entry}{脏病}{zang1bing4}{10,10}{⾁、⽧}
  \definition{s.}{doença venérea}
\end{entry}

\begin{entry}{脏煤}{zang1mei2}{10,13}{⾁、⽕}
  \definition{s.}{carvão sujo | sujeira (de uma mina de carvão)}
\end{entry}

\begin{entry}{脏土}{zang1tu3}{10,3}{⾁、⼟}
  \definition{s.}{solo sujo | lama | lixo}
\end{entry}

\begin{entry}{脏脏}{zang1zang1}{10,10}{⾁、⾁}
  \definition{adj.}{sujo}
\end{entry}

\begin{entry}{脏字}{zang1zi4}{10,6}{⾁、⼦}
  \definition{s.}{obscenidade}
\end{entry}

\begin{entry}{脏}{zang4}{10}{⾁}
  \definition{s.}{órgão (anatomia) | víscera}
  \seeref{脏}{zang1}
\end{entry}

\begin{entry}{脏器}{zang4qi4}{10,16}{⾁、⼝}
  \definition{s.}{órgãos internos}
\end{entry}

\begin{entry}{葬}{zang4}{12}{⾋}
  \definition{v.}{enterrar (os mortos) | sepultar}
\end{entry}

\begin{entry}{遭到}{zao1dao4}{14,8}{⾡、⼑}
  \definition{v.}{sofrer | encontrar-se com (algo infeliz)}
\end{entry}

\begin{entry}{遭受}{zao1shou4}{14,8}{⾡、⼜}
  \definition{v.}{sofrer | suportar (perda, infornúnio)}
\end{entry}

\begin{entry}{遭遇}{zao1yu4}{14,12}{⾡、⾡}
  \definition{s.}{experiência (amarga)}
  \definition{v.}{encontrar-se com (algo infeliz)}
\end{entry}

\begin{entry}{糟糕}{zao1gao1}{17,16}{⽶、⽶}
  \definition{adj.}{muito mau | péssimo}
\end{entry}

\begin{entry}{早}{zao3}{6}{⽇}[HSK 1]
  \definition{adj.}{prematuramente}
  \definition{adv.}{cedo | antecipadamente | breve}
  \definition{s.}{manhã}
\end{entry}

\begin{entry}{早安}{zao3'an1}{6,6}{⽇、⼧}
  \definition{interj.}{Bom dia!}
\end{entry}

\begin{entry}{早餐}{zao3 can1}{6,16}{⽇、⾷}[HSK 2]
  \definition[份,顿,次]{s.}{café da manhã}
\end{entry}

\begin{entry}{早车}{zao3che1}{6,4}{⽇、⾞}
  \definition{s.}{trem matutino | ônibus matutino}
\end{entry}

\begin{entry}{早晨}{zao3 chen2}{6,11}{⽇、⽇}[HSK 2]
  \definition{adv.}{manhã cedo | manhãzinha}
  \definition[个]{s.}{manhã}
\end{entry}

\begin{entry}{早饭}{zao3fan4}{6,7}{⽇、⾷}[HSK 1]
  \definition[份,顿,次,餐]{s.}{café da manhã}
\end{entry}

\begin{entry}{早就}{zao3 jiu4}{6,12}{⽇、⼪}[HSK 2]
  \definition{adv.}{já em um momento anterior}
\end{entry}

\begin{entry}{早前}{zao3qian2}{6,9}{⽇、⼑}
  \definition{adv.}{previamente}
\end{entry}

\begin{entry}{早上}{zao3shang5}{6,3}{⽇、⼀}[HSK 1]
  \definition{adv.}{manhã cedo | manhãzinha}
  \definition[个]{s.}{manhã}
\end{entry}

\begin{entry}{早亡}{zao3wang2}{6,3}{⽇、⼇}
  \definition[个]{s.}{morte prematura}
  \definition{v.}{morrer prematuramente}
\end{entry}

\begin{entry}{早已}{zao3 yi3}{6,3}{⽇、⼰}[HSK 3]
  \definition{adv.}{há muito tempo; por muito tempo}
\end{entry}

\begin{entry}{早早儿}{zao3zao3r5}{6,6,2}{⽇、⽇、⼉}
  \definition{adv.}{o mais cedo possível | o mais breve possível}
\end{entry}

\begin{entry}{早知}{zao3zhi1}{6,8}{⽇、⽮}
  \definition{v.}{prever | se alguém soubesse antes, \dots}
\end{entry}

\begin{entry}{灶台}{zao4tai2}{7,5}{⽕、⼝}
  \definition{s.}{fogão}
\end{entry}

\begin{entry}{造}{zao4}{10}{⾡}[HSK 3]
  \definition*{s.}{sobrenome Zao}
  \definition{clas.}{para colheitas ou número de colheitas de safras}
  \definition{s.}{uma das duas partes em um acordo legal ou uma ação judicial | colheita; safra}
  \definition{v.}{fazer; construir; criar; produzir | cozinhar; fabricar; inventar | ir para; chegar a | alcançar; atingir | treinar; educar}
\end{entry}

\begin{entry}{造成}{zao4cheng2}{10,6}{⾡、⼽}[HSK 3]
  \definition{v.}{criar; causar; acarretar; dar origem a; formar; levar a (principalmente resultados ruins)}
\end{entry}

\begin{entry}{造型}{zao4xing2}{10,9}{⾡、⼟}[HSK 4]
  \definition{s.}{modelo; formato; forma; moldagem}
  \definition{v.}{modelar; moldar}
\end{entry}

\begin{entry}{艁}{zao4}{13}{⾈}
  \variantof{造}
\end{entry}

\begin{entry}{责怪}{ze2guai4}{8,8}{⾙、⼼}
  \definition{v.}{repreender | censurar}
\end{entry}

\begin{entry}{责任}{ze2ren4}{8,6}{⾙、⼈}[HSK 3]
  \definition{s.}{dever; responsabilidade; de acordo com sua ocupação, posição, identidade, etc., as coisas que você deve fazer ou as tarefas que você deve realizar | culpa; responsabilidade por uma falha ou erro; não fazer o que se deve fazer e assumir a culpa}
\end{entry}

\begin{entry}{怎}{zen3}{9}{⼼}
  \definition{adv.}{como}
\end{entry}

\begin{entry}{怎么}{zen3me5}{9,3}{⼼、⼃}[HSK 1]
  \definition{pron.}{como? | o que?}
\end{entry}

\begin{entry}{怎么办}{zen3 me5 ban4}{9,3,4}{⼼、⼃、⼒}[HSK 2]
  \definition{adv.}{o que fazer?}
\end{entry}

\begin{entry}{怎么得了}{zen3me5de2liao3}{9,3,11,2}{⼼、⼃、⼻、⼅}
  \definition{expr.}{Como isso pode ser? | Que bagunça horrível! | O que deve ser feito?}
\end{entry}

\begin{entry}{怎么搞的}{zen3me5gao3de5}{9,3,13,8}{⼼、⼃、⼿、⽩}
  \definition{expr.}{Como isso aconteceu? | O que deu errado? | E aí? | O que está errado?}
\end{entry}

\begin{entry}{怎么回事}{zen3me5hui2shi4}{9,3,6,8}{⼼、⼃、⼞、⼅}
  \definition{expr.}{O que aconteceu? | O que se passou?}
\end{entry}

\begin{entry}{怎么了}{zen3me5le5}{9,3,2}{⼼、⼃、⼅}
  \definition{expr.}{O que aconteceu? | O que está acontecendo? | E aí?}
\end{entry}

\begin{entry}{怎么样}{zen3me5yang4}{9,3,10}{⼼、⼃、⽊}[HSK 2]
  \definition{adv.}{como? | que tal?}
\end{entry}

\begin{entry}{怎样}{zen3 yang4}{9,10}{⼼、⽊}[HSK 2]
  \definition{pron.}{como | o que | de uma certa maneira | de qualquer maneira | não importa o quão}
\end{entry}

\begin{entry}{曾}{zeng1}{12}{⽈}
  \definition*{s.}{sobrenome Zeng}
  \definition{s.}{relacionamento entre bisnetos e bisavós}
  \seeref{曾}{ceng2}
\end{entry}

\begin{entry}{增加}{zeng1jia1}{15,5}{⼟、⼒}[HSK 3]
  \definition{v.}{adicionar; aumentar; incrementar}
\end{entry}

\begin{entry}{增速}{zeng1su4}{15,10}{⼟、⾡}
  \definition{s.}{(economia) taxa de crescimento}
  \definition{v.}{acelerar;}
\end{entry}

\begin{entry}{增长}{zeng1 zhang3}{15,4}{⼟、⾧}[HSK 3]
  \definition{v.}{subir; crescer; aumentar}
\end{entry}

\begin{entry}{查}{zha1}{9}{⽊}
  \definition*{s.}{sobrenome Zha}
  \definition{s.}{espinheiro}
  \seeref{查}{cha2}
\end{entry}

\begin{entry}{闸门}{zha2men2}{8,3}{⾨、⾨}
  \definition{s.}{eclusa | comporta}
\end{entry}

\begin{entry}{寨}{zhai4}{14}{⼧}
  \definition{s.}{fortaleza | paliçada | acampamento | vila (paliçada)}
\end{entry}

\begin{entry}{占}{zhan1}{5}{⼘}
  \definition*{s.}{sobrenome Zhan}
  \definition{v.}{praticar adivinhação | advinhar}
  \seeref{占}{zhan4}
\end{entry}

\begin{entry}{斩获}{zhan3huo4}{8,10}{⽄、⾋}
  \definition{v.}{matar ou capturar (em batalha) | (figurativo) (esportes) marcar (um gol), ganhar (uma medalha) | (figurativo) colher recompensas, obter ganhos}
\end{entry}

\begin{entry}{展开}{zhan3kai1}{10,4}{⼫、⼶}[HSK 3]
  \definition{s.}{desenvolvimento; expansão; explosão; evolução}
  \definition{v.}{desenvolver; espalhar; desdobrar; abrir; desenrolar; amplificar; esticar; ventilar | lançar; desdobrar; desenvolver; executar}
\end{entry}

\begin{entry}{展示}{zhan3shi4}{10,5}{⼫、⽰}
  \definition{v.}{revelar | mostrar | exibir}
\end{entry}

\begin{entry}{盏}{zhan3}{10}{⽫}
  \definition{clas.}{para lâmpadas}
  \definition{s.}{copo pequeno}
\end{entry}

\begin{entry}{占}{zhan4}{5}{⼘}[HSK 2]
  \definition{v.}{ocupar | apreender | tomar | constituir | manter | compor | dar conta de}
  \seeref{占}{zhan1}
\end{entry}

\begin{entry}{战}{zhan4}{9}{⼽}
  \definition{s.}{luta | guerra | batalha}
  \definition{v.}{lutar}
\end{entry}

\begin{entry}{战斗}{zhan4dou4}{9,4}{⼽、⽃}[HSK 4]
  \definition[场,次]{s.}{luta; batalha; combate; ação; conflito armado entre as partes oponentes}
  \definition{v.}{lutar | trabalhar sob pressão}
\end{entry}

\begin{entry}{战胜}{zhan4 sheng4}{9,9}{⼽、⾁}[HSK 4]
  \definition{v.}{derrotar; vencer; superar; triunfar sobre; metáfora para superar dificuldades e alcançar o sucesso}
\end{entry}

\begin{entry}{战士}{zhan4shi4}{9,3}{⼽、⼠}[HSK 4]
  \definition[个]{s.}{soldado; membros mais jovens do exército | campeão; guerreiro; lutador; geralmente, uma pessoa que se engaja em alguma causa justa ou participa de alguma luta justa}
\end{entry}

\begin{entry}{战争}{zhan4zheng1}{9,6}{⼽、⼑}[HSK 4]
  \definition[场,次]{s.}{guerra; conflito; luta armada entre povos, entre nações, entre classes ou entre grupos políticos}
\end{entry}

\begin{entry}{站}{zhan4}{10}{⽴}[HSK 1]
  \definition{s.}{estação | ponto | parada}
\end{entry}

\begin{entry}{站点}{zhan4dian3}{10,9}{⽴、⽕}
  \definition{s.}{\emph{website}}
\end{entry}

\begin{entry}{站台}{zhan4tai2}{10,5}{⽴、⼝}
  \definition{s.}{plataforma (em uma estação ferroviária)}
\end{entry}

\begin{entry}{站长}{zhan4zhang3}{10,4}{⽴、⾧}
  \definition{s.}{pessoa responsável pela estação de trem | chefe da estação | \emph{webmaster} | gerente de centro de voluntariado}
\end{entry}

\begin{entry}{站住}{zhan4 zhu4}{10,7}{⽴、⼈}[HSK 2]
  \definition{v.}{parar | deter | ficar firme em pé | manter os pés firmes | manter a própria posição | consolidar a própria posição | reter água | ser sustentável}
\end{entry}

\begin{entry}{站姿}{zhan4zi1}{10,9}{⽴、⼥}
  \definition{s.}{postura}
\end{entry}

\begin{entry}{张}{zhang1}{7}{⼸}[HSK 3]
  \definition*{s.}{Zhang, uma das mansões lunares | sobrenome Zhang}
  \definition{adj.}{nervoso; tenso}
  \definition{clas.}{para papel, couro, etc. | para camas, mesas, etc. | para a boca e o rosto | para arcos}
  \definition{s.}{folha de papel}
  \definition{v.}{consertar (uma corda de arco); encordoar (um instrumento musical ou um arco) | abrir; espalhar; esticar | expor; exibir |expandir; estender | ampliar; exagerar | olhar | dar rédea solta a; satisfazer | iniciar um negócio; abrir uma loja | colocar em bom uso; dar liberdade para | pegar com uma rede; montar armadilhas para capturar pássaros e animais}
\end{entry}

\begin{entry}{张狂}{zhang1kuang2}{7,7}{⼸、⽝}
  \definition{adj.}{impetuoso | frenético | insolente}
\end{entry}

\begin{entry}{张三}{zhang1san1}{7,3}{⼸、⼀}
  \definition*{s.}{Zhang San | Zé Ninguém | nome para uma pessoa não especificada, 1 de 3}
  \seealsoref{李四}{li3si4}
  \seealsoref{王五}{wang2wu3}
\end{entry}

\begin{entry}{章}{zhang1}{11}{⾳}
  \definition*{s.}{sobrenome Zhang}
  \definition{s.}{capítulo | seção | cláusula |  movimento (de sinfonia) | selo | crachá | regulamento}
\end{entry}

\begin{entry}{章鱼}{zhang1yu2}{11,8}{⾳、⿂}
  \definition{s.}{polvo | octópode}
\end{entry}

\begin{entry}{长}{zhang3}{4}{⾧}[HSK 2]
  \definition{s.}{chefe | ancião}
  \definition{v.}{crescer | desenvolver | aumentar | melhorar}
  \seeref{长}{chang2}
\end{entry}

\begin{entry}{长大}{zhang3 da4}{4,3}{⾧、⼤}[HSK 2]
  \definition{v.}{crescer | ser criado}
\end{entry}

\begin{entry}{涨价}{zhang3jia4}{10,6}{⽔、⼈}
  \definition{s.}{aumento de preços}
  \definition{v.+compl.}{avaliar (em valor) | dar preço | aumentar o preço}
\end{entry}

\begin{entry}{掌}{zhang3}{12}{⼿}
  \definition{s.}{palma da mão | sola do pé | pata | ferradura}
  \definition{v.}{dar um tapa | segurar na mão | empunhar}
\end{entry}

\begin{entry}{丈夫}{zhang4fu5}{3,4}{⼀、⼤}[HSK 4]
  \definition[个,位,名]{s.}{marido; esposo}
\end{entry}

\begin{entry}{招}{zhao1}{8}{⼿}
  \definition{adj.}{contagioso}
  \definition{s.}{um movimento (xadrez) | uma manobra | dispositivo | truque}
  \definition{v.}{recrutar | provocar | acenar | incorrer | infectar | confessar}
\end{entry}

\begin{entry}{招呼}{zhao1 hu5}{8,8}{⼿、⼝}[HSK 4]
  \definition{v.}{chamar; chamar a atenção com palavras ou gestos | cumprimentar; saudar; cumprimentar ou despedir-se das pessoas com palavras ou gestos | pedir a alguém para fazer algo; fazer solicitações, pedir ajuda ou fazer coisas | receber e dar boas-vindas aos convidados}
\end{entry}

\begin{entry}{招手}{zhao1shou3}{8,4}{⼿、⼿}
  \definition{v.+compl.}{acenar}
\end{entry}

\begin{entry}{招数}{zhao1shu4}{8,13}{⼿、⽁}
  \definition{s.}{estratégia | movimento (no xadrez, no palco, nas artes marciais) | esquema | truque}
\end{entry}

\begin{entry}{着}{zhao1}{11}{⽬}
  \definition{interj.}{Tudo bem!}
  \definition{s.}{movimento (xadrez) | truque}
  \seeref{着}{zhao2}
  \seeref{着}{zhe5}
  \seeref{着}{zhuo2}
\end{entry}

\begin{entry}{着数}{zhao1shu4}{11,13}{⽬、⽁}
  \definition{s.}{estratégia | movimento (no xadrez, no palco, nas artes marciais) | esquema | truque}
\end{entry}

\begin{entry}{朝}{zhao1}{12}{⽉}
  \definition{s.}{manhã cedo; manhã | dia}
  \seeref{朝}{chao2}
\end{entry}

\begin{entry}{着}{zhao2}{11}{⽬}
  \definition{v.}{ser afetado por | queimar | pegar fogo | entrar em contato com | sentir | tocar}
  \seeref{着}{zhao1}
  \seeref{着}{zhe5}
  \seeref{着}{zhuo2}
\end{entry}

\begin{entry}{着地}{zhao2di4}{11,6}{⽬、⼟}
  \definition{v.}{pousar | tocar o chão}
\end{entry}

\begin{entry}{着花}{zhao2hua1}{11,7}{⽬、⾋}
  \definition{v.}{florescer}
  \seeref{着花}{zhuo2hua1}
\end{entry}

\begin{entry}{着火}{zhao2huo3}{11,4}{⽬、⽕}[HSK 4]
  \definition{v.}{pegar fogo; estar em chamas}
\end{entry}

\begin{entry}{着急}{zhao2ji2}{11,9}{⽬、⼼}[HSK 4]
  \definition{adj.}{ansioso; preocupado |}
  \definition{s.}{preocupação; ansiedade}
  \definition{v.+compl.}{preocupar-se | sentir-se ansioso | sentir uma sensação de urgência}
\end{entry}

\begin{entry}{着凉}{zhao2liang2}{11,10}{⽬、⼎}
  \definition{v.}{pegar um resfriado}
\end{entry}

\begin{entry}{找}{zhao3}{7}{⼿}[HSK 1]
  \definition{v.}{andar à procura de | procurar | tentar procurar | dar troco | retornar algo}
\end{entry}

\begin{entry}{找遍}{zhao3bian4}{7,12}{⼿、⾡}
  \definition{v.}{pentear | pesquisar em todos os lugares}
\end{entry}

\begin{entry}{找出}{zhao3 chu1}{7,5}{⼿、⼐}[HSK 2]
  \definition{v.}{encontrar | procurar}
\end{entry}

\begin{entry}{找到}{zhao3dao4}{7,8}{⼿、⼑}[HSK 1]
  \definition{v.}{encontrar}
\end{entry}

\begin{entry}{找回}{zhao3hui2}{7,6}{⼿、⼞}
  \definition{v.}{recuperar algo}
\end{entry}

\begin{entry}{找见}{zhao3jian4}{7,4}{⼿、⾒}
  \definition{v.}{encontrar (algo que está procurando)}
\end{entry}

\begin{entry}{找零}{zhao3ling2}{7,13}{⼿、⾬}
  \definition{v.}{trocar dinheiro | dar troco}
\end{entry}

\begin{entry}{找钱}{zhao3qian2}{7,10}{⼿、⾦}
  \definition{v.}{dar troco}
\end{entry}

\begin{entry}{找事}{zhao3shi4}{7,8}{⼿、⼅}
  \definition{v.}{procurar emprego | começar uma briga}
\end{entry}

\begin{entry}{找寻}{zhao3xun2}{7,6}{⼿、⼨}
  \definition{v.}{encontrar falhas | procurar | buscar}
\end{entry}

\begin{entry}{找着}{zhao3zhao2}{7,11}{⼿、⽬}
  \definition{v.}{encontrar}
\end{entry}

\begin{entry}{找辙}{zhao3zhe2}{7,16}{⼿、⾞}
  \definition{v.}{procurar um pretexto}
\end{entry}

\begin{entry}{召开}{zhao4kai1}{5,4}{⼝、⼶}[HSK 4]
  \definition{v.}{convocar; chamar pessoas para uma reunião; realizar (uma reunião)}
\end{entry}

\begin{entry}{兆}{zhao4}{6}{⼉}
  \definition{num.}{trilhão}
\end{entry}

\begin{entry}{照}{zhao4}{13}{⽕}[HSK 3]
  \definition{adv.}{de acordo com; indica agir de acordo com o original ou um certo padrão}
  \definition{prep.}{em direção a; na direção de | de acordo com; conforme}
  \definition{s.}{imagem; fotografia | permissão; licença | brilho; iluminação}
  \definition{v.}{brilhar; acender; iluminar | refletir; espelhar; olhar para sua própria imagem em um espelho, etc. | filmar; fotografar; tirar uma foto (fotografia) | cuidar de; tomar conta de | notificar | contrastar | entender}
\end{entry}

\begin{entry}{照顾}{zhao4gu4}{13,10}{⽕、⾴}[HSK 2]
  \definition{v.}{cuidar de | atender a | oferecer tratamento preferencial | (de um cliente) patrocinar | fazer compras em | dar consideração a | mostrar consideração por | levar em conta | fazer concessões para}
\end{entry}

\begin{entry}{照亮}{zhao4liang4}{13,9}{⽕、⼇}
  \definition{s.}{iluminação}
  \definition{v.}{iluminar}
\end{entry}

\begin{entry}{照片}{zhao4pian4}{13,4}{⽕、⽚}[HSK 2]
  \definition[张,套,幅]{s.}{fotografia | foto}
\end{entry}

\begin{entry}{照片底版}{zhao4pian4di3ban3}{13,4,8,8}{⽕、⽚、⼴、⽚}
  \definition{s.}{placa fotográfica}
\end{entry}

\begin{entry}{照片子}{zhao4pian4zi5}{13,4,3}{⽕、⽚、⼦}
  \definition{v.}{tirar um raio X}
\end{entry}

\begin{entry}{照骗}{zhao4pian4}{13,12}{⽕、⾺}
  \definition{s.}{imagem ``photoshopada''}
\end{entry}

\begin{entry}{照相}{zhao4 xiang4}{13,9}{⽕、⽬}[HSK 2]
  \definition{v.+compl.}{tirar fotografia}
\end{entry}

\begin{entry}{照相机}{zhao4xiang4ji1}{13,9,6}{⽕、⽬、⽊}
  \definition[个,架,部,台,只]{s.}{câmera/máquina fotográfica}
\end{entry}

\begin{entry}{照像}{zhao4 xiang4}{13,13}{⽕、⼈}
  \variantof{照相}
\end{entry}

\begin{entry}{照像机}{zhao4xiang4ji1}{13,13,6}{⽕、⼈、⽊}
  \variantof{照相机}
\end{entry}

\begin{entry}{照准}{zhao4zhun3}{13,10}{⽕、⼎}
  \definition{s.}{solicitação concedida (uso formal em documento antigo)}
  \definition{v.}{mirar (arma)}
\end{entry}

\begin{entry}{折}{zhe1}{7}{⼿}
  \definition{v.}{rolar; virar | despejar algo de um recipiente em outro; ficar despejando algo entre dois recipientes}
  \seeref{折}{she2}
  \seeref{折}{zhe2}
\end{entry}

\begin{entry}{折}{zhe2}{7}{⼿}[HSK 4]
  \definition*{s.}{sobrenome Zhe}
  \definition{clas.}{uma passagem em um roteiro de ópera miscelânea de Yuan, aproximadamente equivalente a uma cena ou ato em uma ópera moderna}
  \definition[张,个,些]{s.}{fratura; quebra | abatimento; desconto | traços dos caracteres chineses que têm o formato de "𠃍" e "乚", etc. | pasta; livreto; \emph{folder}}
  \definition{v.}{estalar; quebrar; fazer quebrar | perder; sofrer a perda de | voltar para trás; mudar de direção; retornar |ser convencido; estar cheio de admiração | equivaler a; converter em | dobrar}
  \seeref{折}{she2}
  \seeref{折}{zhe1}
\end{entry}

\begin{entry}{折转}{zhe2zhuan3}{7,8}{⼿、⾞}
  \definition{s.}{reflexo (ângulo)}
  \definition{v.}{voltar atrás}
\end{entry}

\begin{entry}{哲理}{zhe2li3}{10,11}{⼝、⽟}
  \definition{s.}{filosofia | teoria filosófica}
\end{entry}

\begin{entry}{者}{zhe3}{8}{⽼}[HSK 3]
  \definition{part.}{usado depois de um adjetivo ou verbo, ou depois de uma frase com um adjetivo ou verbo, para indicar uma pessoa ou coisa que tem esse atributo ou realiza essa ação | usado depois de ``trabalho'' e ``ismo'' para se referir a pessoas que fazem um determinado trabalho ou acreditam em uma determinada ideologia | usado depois de numerais como ``dois'', ``três'' etc., referindo-se a vários itens mencionados no contexto | usado depois de palavras, frases ou cláusulas para indicar uma pausa}
  \definition{pron.}{usado principalmente no vernáculo antigo, significando o mesmo que ``这''}
  \definition{suf.}{voluntário}
  \seealsoref{这}{zhe4}
\end{entry}

\begin{entry}{这}{zhe4}{7}{⾡}[HSK 1]
  \definition{pron.}{este, isto}
  \seeref{这}{zhei4}
\end{entry}

\begin{entry}{这边}{zhe4bian5}{7,5}{⾡、⾡}[HSK 1]
  \definition{pron.}{aqui | este lado}
\end{entry}

\begin{entry}{这里}{zhe4li3}{7,7}{⾡、⾥}[HSK 1]
  \definition{pron.}{aqui}
\end{entry}

\begin{entry}{这么}{zhe4 me5}{7,3}{⾡、⼃}[HSK 2]
  \definition{adv.}{como este | desta maneira}
\end{entry}

\begin{entry}{这麽}{zhe4 me5}{7,14}{⾡、⿇}
  \variantof{这么}
\end{entry}

\begin{entry}{这儿}{zhe4r5}{7,2}{⾡、⼉}[HSK 1]
  \definition{pron.}{aqui}
\end{entry}

\begin{entry}{这时}{zhe4 shi2}{7,7}{⾡、⽇}[HSK 2]
  \definition{adv.}{neste momento}
\end{entry}

\begin{entry}{这时候}{zhe4 shi2 hou5}{7,7,10}{⾡、⽇、⼈}[HSK 2]
  \definition{adv.}{neste momento}
\end{entry}

\begin{entry}{这些}{zhe4xie1}{7,8}{⾡、⼆}[HSK 1]
  \definition{pron.}{estes}
\end{entry}

\begin{entry}{这样}{zhe4 yang4}{7,10}{⾡、⽊}[HSK 2]
  \definition{adv.}{assim | dessa maneira | deste modo}
\end{entry}

\begin{entry}{浙江}{zhe4jiang1}{10,6}{⽔、⽔}
  \definition*{s.}{Zhejiang}
\end{entry}

\begin{entry}{着}{zhe5}{11}{⽬}[HSK 1,4]
  \definition{interj.}{O.K.!; Tudo bem!; Tudo certo!}
  \definition{part.}{indicando ação em andamento ou estado em andamento}
  \definition{s.}{um movimento no xadrez |movimento; estratégia; estratagema}
  \definition{v.}{colocar; deixar de lado}
  \seeref{着}{zhao1}
  \seeref{着}{zhao2}
  \seeref{着}{zhuo2}
\end{entry}

\begin{entry}{这}{zhei4}{7}{⾡}
  \definition{pron.}{(coloquial) este}
  \seeref{这}{zhe4}
\end{entry}

\begin{entry}{针}{zhen1}{7}{⾦}[HSK 4]
  \definition*{s.}{sobrenome Zhen}
  \definition[根]{s.}{agulha; ferramentas para costura de roupas | objetos semelhantes a agulhas; algo longo e fino como uma agulha | injeção | ponto de costura | pontos de acupuntura na medicina chinesa}
\end{entry}

\begin{entry}{针对}{zhen1dui4}{7,5}{⾦、⼨}[HSK 4]
  \definition{prep.}{em conexão com; de acordo com; à luz de; introdução de objetos de comportamento com uma finalidade clara}
  \definition{v.}{contrariar; apontar para; ter como objetivo; ser direcionado contra; fazer algo especificamente sobre um problema ou uma pessoa}
\end{entry}

\begin{entry}{珍贵}{zhen1gui4}{9,9}{⽟、⾙}
  \definition{adj.}{precioso}
\end{entry}

\begin{entry}{珍珠}{zhen1zhu1}{9,10}{⽟、⽟}
  \definition[颗]{s.}{pérola}
\end{entry}

\begin{entry}{眞}{zhen1}{10}{⽬}
  \variantof{真}
\end{entry}

\begin{entry}{真}{zhen1}{10}{⼗}[HSK 1]
  \definition{adj.}{genuíno}
  \definition{adv.}{que\dots tão\dots! | realmente}
\end{entry}

\begin{entry}{真的}{zhen1 de5}{10,8}{⼗、⽩}[HSK 1]
  \definition{adv.}{realmente | verdadeiramente}
\end{entry}

\begin{entry}{真理}{zhen1li3}{10,11}{⼗、⽟}
  \definition[个]{s.}{verdade}
\end{entry}

\begin{entry}{真牛}{zhen1niu2}{10,4}{⼗、⽜}
  \definition{adj.}{(gíria) muito legal, incrível}
\end{entry}

\begin{entry}{真切}{zhen1qie4}{10,4}{⼗、⼑}
  \definition{adj.}{claro | distinto | honesto | sincero | vívido}
\end{entry}

\begin{entry}{真声}{zhen1sheng1}{10,7}{⼗、⼠}
  \definition{s.}{voz natural | voz verdadeira}
  \seeref{假声}{jia3sheng1}
\end{entry}

\begin{entry}{真实}{zhen1shi2}{10,8}{⼗、⼧}[HSK 3]
  \definition{adj.}{verdadeiro; real; autêntico}
\end{entry}

\begin{entry}{真释}{zhen1shi4}{10,12}{⼗、⾤}
  \definition{s.}{razão genuína | explicação verdadeira}
\end{entry}

\begin{entry}{真心}{zhen1xin1}{10,4}{⼗、⼼}
  \definition{adj.}{sincero}
  \definition[片]{s.}{sinceridade}
\end{entry}

\begin{entry}{真真}{zhen1zhen1}{10,10}{⼗、⼗}
  \definition{adv.}{genuinamente | realmente | escrupulosamente}
\end{entry}

\begin{entry}{真正}{zhen1zheng4}{10,5}{⼗、⽌}[HSK 2]
  \definition{adj.}{verdadeiro | real | genuíno}
  \definition{adv.}{realmente | de ​​fato}
\end{entry}

\begin{entry}{真珠}{zhen1zhu1}{10,10}{⼗、⽟}
  \variantof{珍珠}
\end{entry}

\begin{entry}{枕}{zhen3}{8}{⽊}
  \definition{s.}{travesseiro | almofada}
\end{entry}

\begin{entry}{阵}{zhen4}{6}{⾩}[HSK 4]
  \definition{clas.}{passagens que expressam a passagem de eventos ou ações}
  \definition{s.}{matriz de batalha (formação); termo tático antigo para as fileiras ou formações de uma equipe de combate | \emph{front}; frente de batalha; posição | um período de tempo}
\end{entry}

\begin{entry}{阵地}{zhen4di4}{6,6}{⾩、⼟}
  \definition{s.}{posição (militar) | frente de batalha | \emph{front}}
\end{entry}

\begin{entry}{震撼}{zhen4han4}{15,16}{⾬、⼿}
  \definition{v.}{sacudir | chocar | atordoar}
\end{entry}

\begin{entry}{正}{zheng1}{5}{⽌}
  \definition{s.}{o primeiro mês do ano lunar; a primeira lua}
  \seeref{正}{zheng4}
\end{entry}

\begin{entry}{争}{zheng1}{6}{⼑}[HSK 3]
  \definition*{s.}{sobrenome Zheng}
  \definition{adv.}{como; por que}
  \definition{v.}{contender; competir; lutar por; esforçar-se | argumentar; disputar; debater | faltar; estar aquém de}
\end{entry}

\begin{entry}{争霸}{zheng1ba4}{6,21}{⼑、⾬}
  \definition{s.}{hegemonia | uma luta de poder}
  \definition{v.}{disputar a hegemonia}
\end{entry}

\begin{entry}{争风吃醋}{zheng1feng1chi1cu4}{6,4,6,15}{⼑、⾵、⼝、⾣}
  \definition{v.}{rivalizar com alguém pelo carinho de um homem ou mulher |estar com ciúmes de um rival em um caso de amor}
\end{entry}

\begin{entry}{争论}{zheng1lun4}{6,6}{⼑、⾔}[HSK 4]
  \definition{s.}{debate; discussão; argumentação; disputa}
  \definition{v.}{discutir; disputar; debater; argumentar; contestar}
\end{entry}

\begin{entry}{争取}{zheng1qu3}{6,8}{⼑、⼜}[HSK 3]
  \definition{v.}{esforçar-se por; lutar por; vencer}
\end{entry}

\begin{entry}{争先}{zheng1xian1}{6,6}{⼑、⼉}
  \definition{v.}{competir para ser o primeiro |contestar o primeiro lugar}
\end{entry}

\begin{entry}{征服}{zheng1fu2}{8,8}{⼻、⽉}[HSK 4]
  \definition{v.}{conquistar; cativar | subjugar; dominar}
\end{entry}

\begin{entry}{征求}{zheng1qiu2}{8,7}{⼻、⽔}[HSK 4]
  \definition{v.}{procurar; buscar; solicitar; pedir abertamente opiniões, pontos de vista, etc.}
\end{entry}

\begin{entry}{挣扎}{zheng1zha2}{9,4}{⼿、⼿}
  \definition{v.}{lutar}
\end{entry}

\begin{entry}{整}{zheng3}{16}{⽁}[HSK 3]
  \definition*{s.}{sobrenome Zheng}
  \definition{adj.}{cheio; integral; inteiro; completo; sem defeitos | limpo; arrumado; em boa ordem | redondo (não é uma fração)}
  \definition{s.}{inteiro (número)}
  \definition{v.}{retificar; pôr em ordem | consertar; renovar; reparar |consertar; punir; fazer alguém sofrer | fazer; produzir; trabalhar}
\end{entry}

\begin{entry}{整个}{zheng3ge4}{16,3}{⽁、⼈}[HSK 3]
  \definition{adj.}{total; inteiro; completo}
\end{entry}

\begin{entry}{整理}{zheng3li3}{16,11}{⽁、⽟}[HSK 3]
  \definition{v.}{organizar; reorganizar; classificar; ordenar; colocar em ordem}
\end{entry}

\begin{entry}{整齐}{zheng3qi2}{16,6}{⽁、⿑}[HSK 3]
  \definition{adj.}{limpo; arrumado; em boa ordem | uniforme; regular; relativamente consistente em tamanho, comprimento, grau, etc. | usado para descrever que as coisas necessárias estão todas prontas}
  \definition{v.}{estar em boas condições; deixar as coisas organizadas e arrumadas}
\end{entry}

\begin{entry}{整体}{zheng3ti3}{16,7}{⽁、⼈}[HSK 3]
  \definition[个]{s.}{todo; totalidade}
\end{entry}

\begin{entry}{整天}{zheng3 tian1}{16,4}{⽁、⼤}[HSK 3]
  \definition{s.}{o dia todo; de manhã até a noite}
\end{entry}

\begin{entry}{整整}{zheng3 zheng3}{16,16}{⽁、⽁}[HSK 3]
  \definition{adv.}{inteiramente; completamente; solidamente; continuamente}
\end{entry}

\begin{entry}{正}{zheng4}{5}{⽌}[HSK 1,3]
  \definition*{s.}{sobrenome Zheng}
  \definition{adj.}{reto; ereto; devido; orientação vertical ou padrão | principal; situado no meio; centralizado | anverso; frente | honesto; correto | certo; correto | puro (cor ou sabor); não misturado | padronizado; regular | básico; principal
regular; os comprimentos dos lados e ângulos de uma forma são iguais | positivo; maior que zero; íon positivo; cátion | afiado; exato; usado para tempo, significando naquele ponto ou no meio de um período}
  \definition{adv.}{apenas; exatamente; certo; precisamente | indica o progresso de uma ação ou a continuação de um estado}
  \definition{v.}{definir (colocar) certo; tornar algo reto; não tornar algo torto | retificar; corrigir; ajustar}
  \seeref{正}{zheng1}
\end{entry}

\begin{entry}{正常}{zheng4chang2}{5,11}{⽌、⼱}[HSK 2]
  \definition{adj.}{regular | normal | ordinário}
\end{entry}

\begin{entry}{正好}{zheng4hao3}{5,6}{⽌、⼥}[HSK 2]
  \definition{adj.}{na medida certa | na hora certa | o suficiente}
  \definition{adv.}{acontecer com | chance de | como acontece}
\end{entry}

\begin{entry}{正确}{zheng4que4}{5,12}{⽌、⽯}[HSK 2]
  \definition{adj.}{correto | certo | próprio}
\end{entry}

\begin{entry}{正式}{zheng4shi4}{5,6}{⽌、⼷}[HSK 3]
  \definition{adj.}{formal; oficial; descreve uma atmosfera, atitude ou comportamento sério que não é fácil ou relaxado; descreve o cumprimento de certas formalidades e procedimentos}
\end{entry}

\begin{entry}{正是}{zheng4 shi4}{5,9}{⽌、⽇}[HSK 2]
  \definition{adv.}{precisamente | exatamente}
\end{entry}

\begin{entry}{正在}{zheng4zai4}{5,6}{⽌、⼟}[HSK 1]
  \definition{adv.}{no processo de | atualmente | em andamento}
  \definition{v.}{estar a~+~v.inf. | estar~+~v.ger.}
\end{entry}

\begin{entry}{正正}{zheng4zheng4}{5,5}{⽌、⽌}
  \definition{adv.}{na hora certa | ordenadamente}
\end{entry}

\begin{entry}{正宗}{zheng4zong1}{5,8}{⽌、⼧}
  \definition{adj.}{autêntico | genuíno | \emph{old school} | (fig.) tradicional}
\end{entry}

\begin{entry}{证}{zheng4}{7}{⾔}[HSK 3]
  \definition{s.}{evidência; prova; testemunho; testemunha | certificado; cartão | doença; enfermidade}
  \definition{v.}{provar; verificar; demonstrar}
\end{entry}

\begin{entry}{证件}{zheng4jian4}{7,6}{⾔、⼈}[HSK 3]
  \definition[个,本,张]{s.}{documentos; credenciais; certificado}
\end{entry}

\begin{entry}{证据}{zheng4ju4}{7,11}{⾔、⼿}[HSK 3]
  \definition{s.}{prova; evidência; testemunho; fatos ou materiais relevantes que podem provar a autenticidade de algo}
\end{entry}

\begin{entry}{证明}{zheng4ming2}{7,8}{⾔、⽇}[HSK 3]
  \definition[个,份]{s.}{certificado; testemunho; identificação; certificado ou carta de certificação; documentos que comprovem identidade, experiência, etc., como carteira de estudante, carteira de trabalho, certificado de graduação, etc.}
  \definition{v.}{provar; testemunhar; sustentar; usar materiais confiáveis ​​para mostrar ou determinar a autenticidade de uma pessoa ou coisa}
\end{entry}

\begin{entry}{证实}{zheng4shi2}{7,8}{⾔、⼧}
  \definition{v.}{confirmar (algo como verdadeiro) | verificar}
\end{entry}

\begin{entry}{挣}{zheng4}{9}{⼿}
  \definition{v.}{ganhar dinheiro | esforçar-se para adquirir | lutar para se libertar}
\end{entry}

\begin{entry}{挣得}{zheng4de2}{9,11}{⼿、⼻}
  \definition{v.}{ganhar renda ou dinheiro}
\end{entry}

\begin{entry}{挣钱}{zheng4qian2}{9,10}{⼿、⾦}
  \definition{v.+compl.}{ganhar dinheiro}
\end{entry}

\begin{entry}{政府}{zheng4fu3}{9,8}{⽁、⼴}[HSK 4]
  \definition[个]{s.}{governo;  órgãos executivos do poder do Estado, ou seja, órgãos administrativos do Estado, como o Conselho de Estado (Governo Popular Central) e os governos populares locais em todos os níveis na China}
\end{entry}

\begin{entry}{政纲}{zheng4gang1}{9,7}{⽁、⽷}
  \definition{s.}{programa ou plataforma política}
\end{entry}

\begin{entry}{政治}{zheng4zhi4}{9,8}{⽁、⽔}[HSK 4]
  \definition{s.}{política; assuntos políticos; questões políticas}
\end{entry}

\begin{entry}{政治局}{zheng4zhi4ju2}{9,8,7}{⽁、⽔、⼫}
  \definition{s.}{o principal comitê de políticas de um partido comunista}
\end{entry}

\begin{entry}{之后}{zhi1 hou4}{3,6}{⼂、⼝}[HSK 4]
  \definition{adv.}{mais tarde; posteriormente; depois; desde então; para indicar que é depois de um determinado tempo ou de uma determinada coisa, ``以后'' é usado com frequência na linguagem falada; às vezes, também pode indicar que é depois de um determinado lugar ou local,  ``后面'' é usado com frequência na linguagem falada}
  \seealsoref{后面}{hou4mian4}
  \seealsoref{以后}{yi3 hou4}
\end{entry}

\begin{entry}{之间}{zhi1 jian1}{3,7}{⼂、⾨}[HSK 4]
  \definition{prep.}{(depois de um substantivo) entre; dentro de duas delimitações de tempo, local ou quantitativas | colocado após certos verbos ou advérbios de duas sílabas para indicar um curto período de tempo}
\end{entry}

\begin{entry}{之前}{zhi1 qian2}{3,9}{⼂、⼑}[HSK 4]
  \definition{adv.}{(referindo-se ao tempo) antes, antes de, atrás | (referindo-se ao local físico) na frente de | (usado independentemente) no passado, antes disso}
\end{entry}

\begin{entry}{之外}{zhi1wai4}{3,5}{⼂、⼣}
  \definition{adv.}{lado de fora}
\end{entry}

\begin{entry}{之一}{zhi1 yi1}{3,1}{⼂、⼀}[HSK 4]
  \definition{s.}{um de (algo); pertence a um ou a todo um grupo de coisas com as mesmas características}
\end{entry}

\begin{entry}{支}{zhi1}{4}{⽀}[HSK 3,4][Kangxi 65]
  \definition*{s.}{sobrenome Zhi}
  \definition{clas.}{para equipes, etc. | para canções ou peças musicais | intensidade luminosa para luzes elétricas (velas, watts) | para objetos finos (canetas, armas, etc.) | (fiação) unidade de medida para espessura do fio}
  \definition{s.}{ramo; ramificação; tribo | os doze ramos terrestres}
  \definition{v.}{apoiar; sustentar | esticar-se; levantar | ordenar; mandar embora; colocar alguém para fora; despachar | pagar; desembolsar | receber; obter pagamento; sacar (dinheiro)}
\end{entry}

\begin{entry}{支承}{zhi1cheng2}{4,8}{⽀、⼿}
  \definition{v.}{suportar o peso de (um edifício) | suportar}
\end{entry}

\begin{entry}{支持}{zhi1chi2}{4,9}{⽀、⼿}[HSK 3]
  \definition[个]{s.}{apoio | suporte}
  \definition{v.}{suportar; sustentar; resistir | patrocinar; apoiar; incentivar}
\end{entry}

\begin{entry}{支付}{zhi1 fu4}{4,5}{⽀、⼈}[HSK 3]
  \definition{v.}{pagar (dinheiro); custear}
\end{entry}

\begin{entry}{支根}{zhi1gen1}{4,10}{⽀、⽊}
  \definition{s.}{raiz ramificada | raízes de apoio | radícula}
\end{entry}

\begin{entry}{支票}{zhi1piao4}{4,11}{⽀、⽰}
  \definition[本]{s.}{cheque (banco)}
\end{entry}

\begin{entry}{支应}{zhi1ying4}{4,7}{⽀、⼴}
  \definition{v.}{lidar com | fornecer}
\end{entry}

\begin{entry}{支支吾吾}{zhi1zhi1wu2wu2}{4,4,7,7}{⽀、⽀、⼝、⼝}
  \definition{v.}{falhar | murmurar | paralisar | gaguejar}
\end{entry}

\begin{entry}{只}{zhi1}{5}{⼝}[HSK 3]
  \definition*{s.}{sobrenome Zhi}
  \definition{adj.}{solteiro; solitário}
  \definition{clas.}{para um de um par | para animais pequenos (pássaros, gatos, cães, etc.) | para certos utensílios, aparelhos | para navios}
  \seeref{只}{zhi3}
\end{entry}

\begin{entry}{只身}{zhi1shen1}{5,7}{⼝、⾝}
  \definition{adv.}{sozinho | por si só}
\end{entry}

\begin{entry}{芝麻}{zhi1ma5}{6,11}{⾋、⿇}
  \definition{s.}{semente de gergelim}
\end{entry}

\begin{entry}{知道}{zhi1dao4}{8,12}{⽮、⾡}[HSK 1]
  \definition{v.}{conhecer | saber}
\end{entry}

\begin{entry}{知道了}{zhi1dao4le5}{8,12,2}{⽮、⾡、⼅}
  \definition{interj.}{Entendi! | OK!}
\end{entry}

\begin{entry}{知识}{zhi1shi5}{8,7}{⽮、⾔}[HSK 1]
  \definition[门]{s.}{conhecimento}
  \definition{s.}{intelectual}
\end{entry}

\begin{entry}{织}{zhi1}{8}{⽷}
  \definition{v.}{tecer | tricotar}
\end{entry}

\begin{entry}{脂麻}{zhi1ma5}{10,11}{⾁、⿇}
  \variantof{芝麻}
\end{entry}

\begin{entry}{蜘蛛}{zhi1zhu1}{14,12}{⾍、⾍}
  \definition{s.}{aranha}
\end{entry}

\begin{entry}{蜘蛛网}{zhi1zhu1wang3}{14,12,6}{⾍、⾍、⽹}
  \definition{s.}{teia de aranha}
\end{entry}

\begin{entry}{执着}{zhi2zhuo2}{6,11}{⼿、⽬}
  \definition{s.}{(budismo) apego}
  \definition{v.}{estar fortemente apegado a | ser dedicado | apegar-se a}
\end{entry}

\begin{entry}{直}{zhi2}{8}{⽬}[HSK 3]
  \definition*{s.}{sobrenome Zhi}
  \definition{adj.}{reto; rígido | ereto; vertical; perpendicular; vertical ao solo; de cima para baixo; da frente para trás | justo; honesto; correto | franco; direto}
  \definition{adv.}{diretamente; sempre; reto | continuamente; constantemente | apenas; simplesmente; de ​​fato}
  \definition[条]{s.}{traço vertical (em caracteres chineses, ``竖'')}
  \definition{v.}{endireitar; esticar}
\end{entry}

\begin{entry}{直播}{zhi2bo1}{8,15}{⽬、⼿}[HSK 3]
  \definition{s.}{transmissão ao vivo; transmissão sem gravação por estações de rádio ou sem gravação por canais de televisão | (agricultura) semeadura direta}
  \definition[次]{v.}{(TV, rádio, Internet) transmitir ao vivo}
\end{entry}

\begin{entry}{直到}{zhi2 dao4}{8,8}{⽬、⼑}[HSK 3]
  \definition{adv.}{até (na maior parte do tempo)}
\end{entry}

\begin{entry}{直接}{zhi2jie1}{8,11}{⽬、⼿}[HSK 2]
  \definition{adj.}{direto (oposto: indireto 间接) | imediato}
  \seeref{间接}{jian4jie1}
\end{entry}

\begin{entry}{直译}{zhi2yi4}{8,7}{⽬、⾔}
  \definition{s.}{tradução literal}
  \seealsoref{意译}{yi4yi4}
\end{entry}

\begin{entry}{直译器}{zhi2yi4qi4}{8,7,16}{⽬、⾔、⼝}
  \definition{s.}{(computação) interpretador}
\end{entry}

\begin{entry}{值}{zhi2}{10}{⼈}[HSK 3]
  \definition{s.}{preço; valor; valor numérico}
  \definition{v.}{valer a pena | acontecer com; ir de encontro | estar de serviço; ter sua vez em algo; assumir a posição de turno}
\end{entry}

\begin{entry}{值得}{zhi2de5}{10,11}{⼈、⼻}[HSK 3]
  \definition{adj.}{que vale a pena}
  \definition{v.}{merecer; valer a pena; ser digno; significa que fazer isso terá bons resultados; é valioso e significativo}
\end{entry}

\begin{entry}{职工}{zhi2 gong1}{11,3}{⽿、⼯}[HSK 3]
  \definition[个]{s.}{pessoal; trabalhadores e pessoal de escritório}
\end{entry}

\begin{entry}{职业}{zhi2ye4}{11,5}{⽿、⼀}[HSK 3]
  \definition{adj.}{profissional; não amador}
  \definition[种,份,个]{s.}{ocupação; profissão; vocação; o trabalho que um indivíduo realiza na sociedade como sua principal fonte de subsistência}
\end{entry}

\begin{entry}{职员}{zhi2yuan2}{11,7}{⽿、⼝}
  \definition[个,位]{s.}{empregado | trabalhador de escritório | membro da equipe}
\end{entry}

\begin{entry}{植物}{zhi2wu4}{12,8}{⽊、⽜}[HSK 4]
  \definition[种,株,盆,棵]{s.}{planta; vegetação; flora}
\end{entry}

\begin{entry}{殖}{zhi2}{12}{⽍}
  \definition{v.}{crescer | reproduzir}
\end{entry}

\begin{entry}{只}{zhi3}{5}{⼝}[HSK 2]
  \definition{adv.}{só; somente; apenas; simplesmente; meramente}
  \seeref{只}{zhi1}
\end{entry}

\begin{entry}{只得}{zhi3de5}{5,11}{⼝、⼻}
  \definition{v.}{ser obrigado a | não ter outra alternativa senão}
\end{entry}

\begin{entry}{只读}{zhi3du2}{5,10}{⼝、⾔}
  \definition{s.}{somente leitura (computação) | \emph{read-only}}
\end{entry}

\begin{entry}{只顾}{zhi3gu4}{5,10}{⼝、⾴}
  \definition{adv.}{exclusivamente preocupado (com uma coisa)}
  \definition{v.}{cuidar de apenas um aspecto}
\end{entry}

\begin{entry}{只好}{zhi3hao3}{5,6}{⼝、⼥}[HSK 3]
  \definition{v.}{ter que; ser forçado a; não ter escolha a não ser}
\end{entry}

\begin{entry}{只能}{zhi3 neng2}{5,10}{⼝、⾁}[HSK 2]
  \definition{adv.}{só pode | obrigado a fazer algo}
\end{entry}

\begin{entry}{只怕}{zhi3pa4}{5,8}{⼝、⼼}
  \definition{adv.}{receio que\dots | talvez | muito provavelmente}
\end{entry}

\begin{entry}{只是}{zhi3 shi4}{5,9}{⼝、⽇}[HSK 3]
  \definition{adv.}{somente; meramente; apenas; mas; para enfatizar que é limitado a uma determinada situação ou escopo}
  \definition{conj.}{somente; mas; exceto que; orações de conexão, indicando uma ligeira transição, equivalente a ``不过''}
  \seealsoref{不过}{bu2guo4}
\end{entry}

\begin{entry}{只消}{zhi3xiao1}{5,10}{⼝、⽔}
  \definition{conj.}{desde que}
\end{entry}

\begin{entry}{只要}{zhi3yao4}{5,9}{⼝、⾑}[HSK 2]
  \definition{conj.}{se apenas | contanto que}
\end{entry}

\begin{entry}{只要……就……}{zhi3yao4 jiu4}{5,9,12}{⼝、⾑、⼪}
  \definition{conj.}{contanto que/desde que/se somente\dots, então\dots}
\end{entry}

\begin{entry}{只有}{zhi3 you3}{5,6}{⼝、⽉}[HSK 3]
  \definition{adv.}{somente; tem que; forçado a}
  \definition{conj.}{somente se; conecta cláusulas, expressa condições necessárias, geralmente corresponde a ``才'' (apenas) e ``方'' (que significa apenas)}
  \seealsoref{才}{cai2}
  \seealsoref{方}{fang1}
\end{entry}

\begin{entry}{只有……才……}{zhi3you3 cai2}{5,6,3}{⼝、⽉、⼿}
  \definition{conj.}{só se\dots então\dots}
\end{entry}

\begin{entry}{纸}{zhi3}{7}{⽷}[HSK 2]
  \definition{clas.}{para documentos, cartas, etc.}
  \definition[张,沓]{s.}{papel}
\end{entry}

\begin{entry}{纸币}{zhi3bi4}{7,4}{⽷、⼱}
  \definition[张]{s.}{nota (dinheiro) | cédula}
\end{entry}

\begin{entry}{纸巾}{zhi3jin1}{7,3}{⽷、⼱}
  \definition[张,包]{s.}{lenço | guardanapo | papel toalha}
\end{entry}

\begin{entry}{纸尿裤}{zhi3niao4ku4}{7,7,12}{⽷、⼫、⾐}
  \definition{s.}{fralda descartável}
\end{entry}

\begin{entry}{纸烟}{zhi3yan1}{7,10}{⽷、⽕}
  \definition{s.}{cigarro}
\end{entry}

\begin{entry}{纸张}{zhi3zhang1}{7,7}{⽷、⼸}
  \definition{s.}{papel}
\end{entry}

\begin{entry}{指}{zhi3}{9}{⼿}[HSK 3]
  \definition*{s.}{sobrenome Zhi}
  \definition{clas.}{dígito; largura do dedo; a largura de um dedo é chamada de ``一指'', que é usado para medir profundidade, largura, etc.}
  \definition{s.}{dedo}
  \definition{v.}{apontar para | (pelo) eriçar | indicar; mostrar-se; apontar; demonstrar | referir-se a; dirigir-se a | confiar em; contar com; depender de | criticar; repreender}
\end{entry}

\begin{entry}{指出}{zhi3 chu1}{9,5}{⼿、⼐}[HSK 3]
  \definition{v.}{apontar; indicar}
\end{entry}

\begin{entry}{指导}{zhi3dao3}{9,6}{⼿、⼨}[HSK 3]
  \definition{s.}{guia; pessoa que faz trabalho de orientação}
  \definition{v.}{guiar; dirigir; instruir}
\end{entry}

\begin{entry}{指挥}{zhi3hui1}{9,9}{⼿、⼿}[HSK 4]
  \definition[个]{s.}{diretor; comandante; despachante; operador | maestro; condutor; pessoa na frente de uma orquestra ou coro que dá instruções sobre como tocar ou cantar}
  \definition{v.}{dirigir; conduzir; comandar; direcionar; emitir ordens de agendamento}
\end{entry}

\begin{entry}{指甲}{zhi3jia5}{9,5}{⼿、⽥}
  \definition{s.}{unha da mão}
\end{entry}

\begin{entry}{指南针}{zhi3nan2zhen1}{9,9,7}{⼿、⼗、⾦}
  \definition{s.}{bússola}
\end{entry}

\begin{entry}{黹}{zhi3}{12}{⿋}[Kangxi 204]
  \definition{v.}{costurar; bordar}
\end{entry}

\begin{entry}{至今}{zhi4jin1}{6,4}{⾄、⼈}[HSK 3]
  \definition{adv.}{até agora; até hoje; até este dia}
\end{entry}

\begin{entry}{至少}{zhi4shao3}{6,4}{⾄、⼩}[HSK 3]
  \definition{adv.}{pelo menos}
\end{entry}

\begin{entry}{至于}{zhi4yu2}{6,3}{⾄、⼆}
  \definition{conj.}{para | quanto a | a respeiro de}
\end{entry}

\begin{entry}{志愿}{zhi4 yuan4}{7,14}{⼼、⽕}[HSK 3]
  \definition{s.}{desejo; ideal; aspiração; meta que se espera alcançar}
  \definition{v.}{ser voluntário; tomar a iniciativa e esteja disposto a fazer um trabalho que não gere renda ou que tenha renda muito baixa, mas que possa ajudar outras pessoas}
\end{entry}

\begin{entry}{志愿者}{zhi4yuan4zhe3}{7,14,8}{⼼、⽕、⽼}[HSK 3]
  \definition{s.}{voluntário; pessoa que se voluntaria para servir em atividades de assistência social, eventos de grande porte, conferências, etc.}
\end{entry}

\begin{entry}{制裁}{zhi4cai2}{8,12}{⼑、⾐}
  \definition{s.}{punição | sanção (inclusive econômica)}
  \definition{v.}{punir}
\end{entry}

\begin{entry}{制订}{zhi4 ding4}{8,4}{⼑、⾔}[HSK 4]
  \definition{v.}{esboçar; formular; elaborar; mapear}
\end{entry}

\begin{entry}{制定}{zhi4ding4}{8,8}{⼑、⼧}[HSK 3]
  \definition{v.}{rascunhar; formular; elaborar; estabelecer}
\end{entry}

\begin{entry}{制度}{zhi4du4}{8,9}{⼑、⼴}[HSK 3]
  \definition[个]{s.}{regulamento; regulação; regulamentação | sistema; um sistema político, econômico, cultural e outro formado sob certas condições históricas}
\end{entry}

\begin{entry}{制造}{zhi4zao4}{8,10}{⼑、⾡}[HSK 3]
  \definition{v.}{fazer; produzir; manufaturar; transformar matérias-primas em produtos acabados | criar; agitar; criar artificialmente uma situação ou atmosfera ruim}
\end{entry}

\begin{entry}{制作}{zhi4zuo4}{8,7}{⼑、⼈}[HSK 3]
  \definition{v.}{fazer; fabricar; fazer a partir de matérias-primas, usado principalmente para itens menores e artesanais; criar designs com palavras, imagens, sons, etc. para criar gráficos, anúncios, filmes, jogos, etc.}
\end{entry}

\begin{entry}{治}{zhi4}{8}{⽔}[HSK 4]
  \definition*{s.}{sobrenome Zhi}
  \definition{adj.}{calmo e pacífico}
  \definition{s.}{sede de um antigo governo local}
  \definition{v.}{reger; administrar; governar; gerenciar; gerir | tratar (uma doença); curar; sarar | eliminar; controlar pragas | controlar (um rio); restaurar um curso d'água por meio de dragagem | punir; castigar | estudar; pesquisar; explorar}
\end{entry}

\begin{entry}{治理}{zhi4li3}{8,11}{⽔、⽟}
  \definition{s.}{governança | governo}
  \definition{v.}{gerir para melhor | administrar | por em ordem}
\end{entry}

\begin{entry}{治疗}{zhi4liao2}{8,7}{⽔、⽧}[HSK 4]
  \definition{s.}{diagnóstico; tratamento}
  \definition{v.}{tratar; curar; remediar; eliminar doenças por meio de medicamentos, cirurgia, etc.}
\end{entry}

\begin{entry}{治愈}{zhi4yu4}{8,13}{⽔、⼼}
  \definition{v.}{curar | restaurar a saúde}
\end{entry}

\begin{entry}{质量}{zhi4liang4}{8,12}{⾙、⾥}[HSK 4]
  \definition{s.}{qualidade; o quão bom ou ruim é o produto ou o trabalho}
\end{entry}

\begin{entry}{致敬}{zhi4jing4}{10,12}{⾄、⽁}
  \definition{v.}{saudar | prestar respeitos a | prestar homenagem a}
\end{entry}

\begin{entry}{智慧}{zhi4hui4}{12,15}{⽇、⼼}
  \definition{s.}{sabedoria | inteligência}
\end{entry}

\begin{entry}{智力}{zhi4li4}{12,2}{⽇、⼒}[HSK 4]
  \definition{s.}{inteligência; refere-se à capacidade de uma pessoa de conhecer e entender coisas objetivas e aplicar o conhecimento e a experiência para resolver problemas, incluindo memória, observação, imaginação, pensamento e julgamento}
\end{entry}

\begin{entry}{智能}{zhi4neng2}{12,10}{⽇、⾁}[HSK 4]
  \definition{adj.}{inteligente (telefone, sistema, etc.); descreve máquinas, equipamentos, tecnologia, etc. que foram processados com alta tecnologia e têm a capacidade de falar, pensar, calcular, resolver problemas, etc., como um ser humano}
  \definition{s.}{intelecto; a capacidade de aprender, agir, pensar, inventar, criar, resolver problemas, etc.}
\end{entry}

\begin{entry}{智商}{zhi4shang1}{12,11}{⽇、⼝}
  \definition{s.}{quociente de inteligência, QI}
\end{entry}

\begin{entry}{智障}{zhi4zhang4}{12,13}{⽇、⾩}
  \definition{adj./s.}{retardado}
\end{entry}

\begin{entry}{置疑}{zhi4yi2}{13,14}{⽹、⽦}
  \definition{v.}{duvidar}
\end{entry}

\begin{entry}{中}{zhong1}{4}{⼁}[HSK 1]
  \definition*{s.}{China}
  \definition*{s.}{sobrenome Zhong}
  \definition{s.}{centro | meio | médio | intermediário | média | meio caminho entre dois extremos | intermediador}
  \seeref{中}{zhong4}
  \seealsoref{中国}{zhong1guo2}
\end{entry}

\begin{entry}{中部}{zhong1 bu4}{4,10}{⼁、⾢}[HSK 3]
  \definition{s.}{parte do meio; região central; seção central; a área ou parte do meio | parte do meio; parte central; refere-se à parte central de um romance, filme ou obra televisiva}
\end{entry}

\begin{entry}{中餐}{zhong1 can1}{4,16}{⼁、⾷}[HSK 2]
  \definition[分,顿]{s.}{comida chinesa | almoço}
\end{entry}

\begin{entry}{中东}{zhong1dong1}{4,5}{⼁、⼀}
  \definition*{s.}{Oriente Médio}
\end{entry}

\begin{entry}{中国}{zhong1guo2}{4,8}{⼁、⼞}[HSK 1]
  \definition*{s.}{China}
\end{entry}

\begin{entry}{中国城}{zhong1guo2cheng2}{4,8,9}{⼁、⼞、⼟}
  \definition*{s.}{Bairro Chinês, \emph{Chinatown}}
  \seeref{唐人街}{tang2ren2 jie1}
\end{entry}

\begin{entry}{中国科学院}{zhong1guo2 ke1xue2yuan4}{4,8,9,8,9}{⼁、⼞、⽲、⼦、⾩}
  \definition*{s.}{Academia Chinesa de Ciências}
\end{entry}

\begin{entry}{中国人}{zhong1guo2ren2}{4,8,2}{⼁、⼞、⼈}
  \definition{s.}{chinês | pessoa ou povo da China}
\end{entry}

\begin{entry}{中国通}{zhong1guo2tong1}{4,8,10}{⼁、⼞、⾡}
  \definition*{s.}{Conhecedor da China, especialista em tudo sobre a China}
\end{entry}

\begin{entry}{中华民族}{zhong1 hua2 min2 zu2}{4,6,5,11}{⼁、⼗、⽒、⽅}[HSK 3]
  \definition*{s.}{O Povo Chinês; termo geral para todos os grupos étnicos na China, incluindo 56 grupos étnicos, com uma longa história, esplêndida herança cultural e gloriosas tradições revolucionárias}
\end{entry}

\begin{entry}{中级}{zhong1 ji2}{4,6}{⼁、⽷}[HSK 2]
  \definition{adj.}{nível médio | nível intermediário}
\end{entry}

\begin{entry}{中间}{zhong1jian1}{4,7}{⼁、⾨}[HSK 1]
  \definition{adv.}{central | centro | no meio}
\end{entry}

\begin{entry}{中介}{zhong1jie4}{4,4}{⼁、⼈}[HSK 4]
  \definition[个]{s.}{agente; intermediário}
\end{entry}

\begin{entry}{中年}{zhong1 nian2}{4,6}{⼁、⼲}[HSK 2]
  \definition{s.}{meia-idade}
\end{entry}

\begin{entry}{中情局}{zhong1qing2ju2}{4,11,7}{⼁、⼼、⼫}
  \definition*{s.}{Agência Central de Inteligência dos EUA, CIA (abreviação de 中央情报局)}
  \seeref{中央情报局}{zhong1yang1 qing2bao4ju2}
\end{entry}

\begin{entry}{中秋节}{zhong1qiu1jie2}{4,9,5}{⼁、⽲、⾋}
  \definition*{s.}{Festival do Meio-Outono | Festival do Bolo Lunar (15º dia do oitavo mês lunar)}
\end{entry}

\begin{entry}{中文}{zhong1wen2}{4,4}{⼁、⽂}[HSK 1]
  \definition{s.}{chinês, língua chinesa}
\end{entry}

\begin{entry}{中午}{zhong1wu3}{4,4}{⼁、⼗}[HSK 1]
  \definition[个]{s.}{meio-dia}
\end{entry}

\begin{entry}{中小学}{zhong1 xiao3 xue2}{4,3,8}{⼁、⼩、⼦}[HSK 2]
  \definition{s.}{escolas primárias e secundárias}
\end{entry}

\begin{entry}{中心}{zhong1xin1}{4,4}{⼁、⼼}[HSK 2]
  \definition[个]{s.}{núcleo | coração | meio | centro |chave}
\end{entry}

\begin{entry}{中性}{zhong1xing4}{4,8}{⼁、⼼}
  \definition{adj.}{neutro}
\end{entry}

\begin{entry}{中学}{zhong1xue2}{4,8}{⼁、⼦}[HSK 1]
  \definition[个]{s.}{escola ensino médio}
\end{entry}

\begin{entry}{中学生}{zhong1xue2sheng1}{4,8,5}{⼁、⼦、⽣}[HSK 1]
  \definition{s.}{aluno, estudante de escola ensino médio}
\end{entry}

\begin{entry}{中询}{zhong1 xun2}{4,8}{⼁、⾔}
  \definition{adv.}{segunda dezena do mês | meio do mês | em meados do mês}
\end{entry}

\begin{entry}{中央}{zhong1yang1}{4,5}{⼁、⼤}
  \definition{s.}{centro; meio | autoridades centrais; o mais alto órgão de governo de um país ou partido, etc.}
\end{entry}

\begin{entry}{中央情报局}{zhong1yang1 qing2bao4ju2}{4,5,11,7,7}{⼁、⼤、⼼、⼿、⼫}
  \definition*{s.}{Agência Central de Inteligência dos EUA, CIA}
\end{entry}

\begin{entry}{中药}{zhong1yao4}{4,9}{⼁、⾋}
  \definition[服,种]{s.}{medicina tradicional chinesa}
\end{entry}

\begin{entry}{中医}{zhong1 yi1}{4,7}{⼁、⼖}[HSK 2]
  \definition{s.}{ciência médica tradicional chinesa | médico de medicina tradicional chinesa | praticante de medicina chinesa}
\end{entry}

\begin{entry}{终于}{zhong1yu2}{8,3}{⽷、⼆}[HSK 3]
  \definition{adv.}{finalmente; eventualmente; no final; indica uma situação que ocorre após várias mudanças ou esperas}
\end{entry}

\begin{entry}{钟}{zhong1}{9}{⾦}[HSK 3]
  \definition*{s.}{sobrenome Zhong}
  \definition[顶,个,口]{s.}{sino; campainha; um chocalho feito de cobre ou ferro | relógio; temporizador | tempo; refere-se à hora, ao tempo | copo sem alça; xícara sem alça}
  \definition{v.}{concentrar (as afeições de alguém, etc.)}
\end{entry}

\begin{entry}{钟室}{zhong1shi4}{9,9}{⾦、⼧}
  \definition{s.}{campanário | sala do relógio}
\end{entry}

\begin{entry}{钟罩}{zhong1zhao4}{9,13}{⾦、⽹}
  \definition{s.}{redoma | dossel de sino}
\end{entry}

\begin{entry}{锺}{zhong1}{14}{⾦}
  \variantof{钟}
\end{entry}

\begin{entry}{种}{zhong3}{9}{⽲}[HSK 3,4]
  \definition{clas.}{para tipos, espécies e gêneros de pessoas ou qualquer coisa}
  \definition{s.}{espécie | semente; estirpe; linhagem | entranhas; intestino; espinha dorsal | tipo; variedade; indica tipo, usado para pessoas e qualquer coisa}
  \seeref{种}{zhong4}
\end{entry}

\begin{entry}{种类}{zhong3lei4}{9,9}{⽲、⽶}[HSK 4]
  \definition{s.}{espécie; classe; tipo; variedade; categoria; classificação de alguma coisa de acordo com sua natureza e características}
\end{entry}

\begin{entry}{种麻}{zhong3ma2}{9,11}{⽲、⿇}
  \definition{s.}{planta de cânhamo (feminina)}
\end{entry}

\begin{entry}{种薯}{zhong3shu3}{9,16}{⽲、⾋}
  \definition{s.}{tubérculo semente}
\end{entry}

\begin{entry}{种种}{zhong3zhong3}{9,9}{⽲、⽲}
  \definition{adj.}{todos os tipos de}
\end{entry}

\begin{entry}{种子}{zhong3zi5}{9,3}{⽲、⼦}[HSK 3]
  \definition[颗,粒]{s.}{semente; um órgão exclusivo de certas plantas, geralmente composto de três partes: tegumento, embrião e endosperma, as sementes podem germinar e se tornar novas plantas sob certas condições | jogador cabeça de chave; na competição, quando é realizada a fase eliminatória, são escolhidos os jogadores mais fortes de cada equipe}
\end{entry}

\begin{entry}{种族灭绝}{zhong3zu2mie4jue2}{9,11,5,9}{⽲、⽅、⽕、⽷}
  \definition{s.}{genocídio | extinção étnica}
\end{entry}

\begin{entry}{中}{zhong4}{4}{⼁}
  \definition{v.}{acertar | encaixar exatamente |ser atingido por | cair em | ser afetado por | sofrer | sustentar}
  \seeref{中}{zhong1}
\end{entry}

\begin{entry}{中奖}{zhong4 jiang3}{4,9}{⼁、⼤}[HSK 4]
  \definition{v.}{ganhar um prêmio (em uma loteria, etc.)}
\end{entry}

\begin{entry}{中意}{zhong4yi4}{4,13}{⼁、⼼}
  \definition{s.}{ser do seu agrado | começar a gostar muito de algo ou de alguém}
\end{entry}

\begin{entry}{众}{zhong4}{6}{⼈}
  \definition*{s.}{Câmara dos Deputados, abreviação de 众议院}
  \definition{adj.}{numeroso}
  \definition{adv.}{muitos}
  \definition{s.}{multidão}
  \seeref{众议院}{zhong4yi4yuan4}
\end{entry}

\begin{entry}{众议院}{zhong4yi4yuan4}{6,5,9}{⼈、⾔、⾩}
  \definition*{s.}{Casa baixa da Assembléia Bicameral | Câmara dos Deputados}
\end{entry}

\begin{entry}{种}{zhong4}{9}{⽲}
  \definition{v.}{plantar; semear; crescer; cultivar}
  \seeref{种}{zhong3}
\end{entry}

\begin{entry}{种地}{zhong4di4}{9,6}{⽲、⼟}
  \definition{v.}{cultivar | trabalhar a terra}
\end{entry}

\begin{entry}{种植}{zhong4zhi2}{9,12}{⽲、⽊}[HSK 4]
  \definition{v.}{plantar; crescer; cultivar; enterrar as sementes de uma planta no solo; plantar as mudas de uma planta no solo}
\end{entry}

\begin{entry}{重}{zhong4}{9}{⾥}[HSK 1,3]
  \definition{adj.}{pesado | profundo; sério | importante; momentoso | discreto; prudente | considerável em quantidade ou valor}
  \definition{adv.}{pesadamente; severamente}
  \definition{v.}{colocar (pôr) ênfase em; dar valor a; atribuir importância a}
  \seeref{重}{chong2}
\end{entry}

\begin{entry}{重大}{zhong4da4}{9,3}{⾥、⼤}[HSK 3]
  \definition{adj.}{grande; importante; significativo; de grande importância}
\end{entry}

\begin{entry}{重点}{zhong4dian3}{9,9}{⾥、⽕}[HSK 2]
  \definition{s.}{nota-chave | ponto-chave | ponto focal | ênfase}
  \seeref{重点}{chong2dian3}
\end{entry}

\begin{entry}{重量}{zhong4liang4}{9,12}{⾥、⾥}[HSK 4]
  \definition[个]{s.}{peso; a magnitude da força da gravidade em um objeto}
\end{entry}

\begin{entry}{重视}{zhong4shi4}{9,8}{⾥、⾒}[HSK 2]
  \definition{v.}{atribuir valor a | dar peso a | atribuir importância a | prestar atenção a}
\end{entry}

\begin{entry}{重要}{zhong4yao4}{9,9}{⾥、⾑}[HSK 1]
  \definition{adj.}{importante | significativo | principal}
\end{entry}

\begin{entry}{重重}{zhong4zhong4}{9,9}{⾥、⾥}
  \definition{adv.}{fortemente | severamente}
  \seeref{重重}{chong2chong2}
\end{entry}

\begin{entry}{周}{zhou1}{8}{⼝}[HSK 2]
  \definition*{s.}{sobrenome Zhou | Dinastia Zhou (1046-256 BC)}
  \definition{adv.}{semanalmente}
  \definition{s.}{círculo | circunferência | ciclo | uma volta (em um circuito) | semana}
  \definition{v.}{fazer um circuito |circular | ajudar financeiramente}
\end{entry}

\begin{entry}{周末}{zhou1mo4}{8,5}{⼝、⽊}[HSK 2]
  \definition{s.}{final-de-semana}
\end{entry}

\begin{entry}{周年}{zhou1nian2}{8,6}{⼝、⼲}[HSK 2]
  \definition{s.}{aniversário}
\end{entry}

\begin{entry}{周围}{zhou1wei2}{8,7}{⼝、⼞}[HSK 3]
  \definition{s.}{ao redor; em torno; vizinhança; a parte que circunda o centro}
\end{entry}

\begin{entry}{洲}{zhou1}{9}{⽔}
  \definition{s.}{continente | ilha em um rio}
\end{entry}

\begin{entry}{轴承}{zhou2cheng2}{9,8}{⾞、⼿}
  \definition{s.}{(mecânico) rolamento}
\end{entry}

\begin{entry}{咒骂}{zhou4ma4}{8,9}{⼝、⾺}
  \definition{v.}{xingar | amaldiçoar | execrar}
\end{entry}

\begin{entry}{珠子}{zhu1zi5}{10,3}{⽟、⼦}
  \definition[粒,颗]{s.}{pérola | contas}
\end{entry}

\begin{entry}{猪}{zhu1}{11}{⽝}[HSK 3]
  \definition[口,头]{s.}{porco; suíno}
\end{entry}

\begin{entry}{猪窠}{zhu1ke1}{11,13}{⽝、⽳}
  \definition{s.}{chiqueiro}
\end{entry}

\begin{entry}{猪柳}{zhu1liu3}{11,9}{⽝、⽊}
  \definition{s.}{filé de porco}
\end{entry}

\begin{entry}{猪笼}{zhu1long2}{11,11}{⽝、⽵}
  \definition{s.}{estrutura cilíndrica de bambu ou arame usada para restringir um porco durante o transporte}
\end{entry}

\begin{entry}{猪头}{zhu1tou2}{11,5}{⽝、⼤}
  \definition{s.}{tolo | idiota}
\end{entry}

\begin{entry}{竹编}{zhu2bian1}{6,12}{⽵、⽷}
  \definition{s.}{vime | tecelagem de bambu}
\end{entry}

\begin{entry}{竹马}{zhu2ma3}{6,3}{⽵、⾺}
  \definition{s.}{cavalo de bambu | vara de bambu usada como cavalo de brinquedo}
\end{entry}

\begin{entry}{竹排}{zhu2pai2}{6,11}{⽵、⼿}
  \definition{s.}{jangada de bambu}
\end{entry}

\begin{entry}{竹子}{zhu2zi5}{6,3}{⽵、⼦}
  \definition[棵,支,根]{s.}{bambu}
\end{entry}

\begin{entry}{逐步}{zhu2bu4}{10,7}{⾡、⽌}[HSK 4]
  \definition{adv.}{gradualmente; passo a passo; progressivamente}
\end{entry}

\begin{entry}{逐渐}{zhu2jian4}{10,11}{⾡、⽔}[HSK 4]
  \definition{adv.}{gradualmente; aos poucos; por etapas; indica mudanças lentas e ordenadas no grau, na quantidade, etc.}
\end{entry}

\begin{entry}{主持}{zhu3chi2}{5,9}{⼂、⼿}[HSK 3]
  \definition{s.}{anfitrião; uma pessoa que é responsável ou lida com uma atividade}
  \definition{v.}{dirigir; gerir; tomar conta de | defender, manter}
\end{entry}

\begin{entry}{主动}{zhu3dong4}{5,6}{⼂、⼒}[HSK 3]
  \definition{adj.}{ativo; positivo; agir sem ser pressionado por forças externas (em vez de ser ``passivo'') | iniciativo; capaz de impulsionar as coisas por vontade própria; capaz de criar uma situação favorável e fazer as coisas acontecerem de acordo com suas próprias intenções (em oposição a ``passivo'')}
\end{entry}

\begin{entry}{主人}{zhu3ren2}{5,2}{⼂、⼈}[HSK 2]
  \definition[个,位]{s.}{mestre | anfitrião | proprietário | uma pessoa que tem um certo tipo de bens ou poder}
\end{entry}

\begin{entry}{主任}{zhu3ren4}{5,6}{⼂、⼈}[HSK 3]
  \definition[个,位]{s.}{chefe; diretor; presidente; o chefe de um departamento ou organização}
\end{entry}

\begin{entry}{主题}{zhu3ti2}{5,15}{⼂、⾴}[HSK 4]
  \definition[个]{s.}{tema; assunto; motivo; lema; ideias básicas expressas em toda a obra de literatura e arte por meio de imagens artísticas concretas | pontos/conteúdos principais; referência geral ao conteúdo principal de artigos, discursos, conferências, etc.}
\end{entry}

\begin{entry}{主席}{zhu3xi2}{5,10}{⼂、⼱}[HSK 4]
  \definition*[个,位]{s.}{Presidente (da China)}
  \definition[个,位]{s.}{presidente, \emph{chairman}  (de uma reunião) |
chefe; presidente (de uma organização ou estado)}
\end{entry}

\begin{entry}{主席台}{zhu3xi2tai2}{5,10,5}{⼂、⼱、⼝}
  \definition[个]{s.}{plataforma | tribuna}
\end{entry}

\begin{entry}{主席团}{zhu3xi2tuan2}{5,10,6}{⼂、⼱、⼞}
  \definition{s.}{presídio}
\end{entry}

\begin{entry}{主要}{zhu3yao4}{5,9}{⼂、⾑}[HSK 2]
  \definition{adj.}{principal}
\end{entry}

\begin{entry}{主义}{zhu3yi4}{5,3}{⼂、⼂}
  \definition{s.}{ideologia}
  \definition{suf.}{-ismo}
\end{entry}

\begin{entry}{主意}{zhu3yi5}{5,13}{⼂、⼼}[HSK 3]
  \definition[个]{s.}{ideia; plano; decisão}
\end{entry}

\begin{entry}{主张}{zhu3zhang1}{5,7}{⼂、⼸}[HSK 3]
  \definition[个]{s.}{vista; posição; proposição}
  \definition{v.}{segurar; advogar; manter; defender}
\end{entry}

\begin{entry}{属}{zhu3}{12}{⼫}
  \definition{v.}{juntar; combinar | fixar (a mente) em; centrar (a atenção, etc.) em}
  \seeref{属}{shu3}
\end{entry}

\begin{entry}{嘱}{zhu3}{15}{⼝}
  \definition{v.}{juntar-se | implorar | incitar}
\end{entry}

\begin{entry}{嘱咐}{zhu3fu5}{15,8}{⼝、⼝}
  \definition{v.}{ordenar | dizer | exortar}
\end{entry}

\begin{entry}{嘱托}{zhu3tuo1}{15,6}{⼝、⼿}
  \definition{v.}{confiar uma tarefa a alguém}
\end{entry}

\begin{entry}{住}{zhu4}{7}{⼈}[HSK 1]
  \definition{v.}{habitar | residir | morar | alojar-se}
\end{entry}

\begin{entry}{住处}{zhu4chu4}{7,5}{⼈、⼡}
  \definition{s.}{morada | habitação | residência}
\end{entry}

\begin{entry}{住房}{zhu4fang2}{7,8}{⼈、⼾}[HSK 2]
  \definition{s.}{habitação}
\end{entry}

\begin{entry}{住所}{zhu4suo3}{7,8}{⼈、⼾}
  \definition[处]{s.}{morada | habitação | residência}
\end{entry}

\begin{entry}{住院}{zhu4 yuan4}{7,9}{⼈、⾩}[HSK 2]
  \definition{v.}{estar hospitalizado | estar no hospital}
\end{entry}

\begin{entry}{住宅}{zhu4zhai2}{7,6}{⼈、⼧}
  \definition{s.}{residência}
\end{entry}

\begin{entry}{住嘴}{zhu4zui3}{7,16}{⼈、⼝}
  \definition{interj.}{Cale-se!}
  \definition{v.}{calar | calar-se}
\end{entry}

\begin{entry}{助兴}{zhu4xing4}{7,6}{⼒、⼋}
  \definition{v.+compl.}{animar as coisas | juntar-se à diversão}
\end{entry}

\begin{entry}{注册}{zhu4ce4}{8,5}{⽔、⼌}
  \definition{v.}{inscrever-se | matricular-se | registrar-se}
\end{entry}

\begin{entry}{注册表}{zhu4ce4biao3}{8,5,8}{⽔、⼌、⾐}
  \definition*{s.}{Registro do Windows}
\end{entry}

\begin{entry}{注册人}{zhu4ce4ren2}{8,5,2}{⽔、⼌、⼈}
  \definition{s.}{registrante}
\end{entry}

\begin{entry}{注册商标}{zhu4ce4shang1biao1}{8,5,11,9}{⽔、⼌、⼝、⽊}
  \definition{s.}{marca registrada}
\end{entry}

\begin{entry}{注意}{zhu4yi4}{8,13}{⽔、⼼}[HSK 3]
  \definition{v.}{prestar atenção em; tomar nota de; ficar de olho em; concentrar seus pensamentos em uma coisa}
\end{entry}

\begin{entry}{注意地}{zhu4yi4di4}{8,13,6}{⽔、⼼、⼟}
  \definition{s.}{área de cuidado, de observação}
\end{entry}

\begin{entry}{注意力}{zhu4yi4li4}{8,13,2}{⽔、⼼、⼒}
  \definition{s.}{atenção}
\end{entry}

\begin{entry}{注意力缺失症}{zhu4yi4li4que1shi1zheng4}{8,13,2,10,5,10}{⽔、⼼、⼒、⽸、⼤、⽧}
  \definition{s.}{transtorno de déficit de atenção}
\end{entry}

\begin{entry}{驻军}{zhu4jun1}{8,6}{⾺、⼍}
  \definition{s.}{guarnição}
  \definition{v.}{guarcener ou prover uma tropa}
\end{entry}

\begin{entry}{祝}{zhu4}{9}{⽰}[HSK 3]
  \definition*{s.}{sobrenome Zhu}
  \definition{v.}{expressar bons desejos; desejar; abençoar | orar aos deuses ou espíritos por bênçãos}
\end{entry}

\begin{entry}{祝祷}{zhu4dao3}{9,11}{⽰、⽰}
  \definition{v.}{rezar | orar}
\end{entry}

\begin{entry}{祝福}{zhu4fu2}{9,13}{⽰、⽰}[HSK 4]
  \definition[个]{s.}{bênção; benzedura; benzimento; originalmente, referia-se à oração para obter as bênçãos de Deus, mas, mais tarde, refere-se a desejar paz e felicidade às pessoas}
  \definition{v.}{desejar boa sorte a alguém}
\end{entry}

\begin{entry}{祝好}{zhu4hao3}{9,6}{⽰、⼥}
  \definition{expr.}{desejo-lhe tudo de melhor! (ao encerrar uma correspondência)}
\end{entry}

\begin{entry}{祝贺}{zhu4he4}{9,9}{⽰、⾙}
  \definition[个]{s.}{congratulações}
  \definition{v.}{congratular}
\end{entry}

\begin{entry}{祝酒}{zhu4jiu3}{9,10}{⽰、⾣}
  \definition{v.}{parabenizar e fazer um brinde | brindar}
\end{entry}

\begin{entry}{祝寿}{zhu4shou4}{9,7}{⽰、⼨}
  \definition{v.}{dar parabéns pelo aniversário (a uma pessoa idosa)}
\end{entry}

\begin{entry}{祝颂}{zhu4song4}{9,10}{⽰、⾴}
  \definition{v.}{expressar bons desejos}
\end{entry}

\begin{entry}{祝谢}{zhu4xie4}{9,12}{⽰、⾔}
  \definition{v.}{agradecer | dar parabéns}
\end{entry}

\begin{entry}{祝愿}{zhu4yuan4}{9,14}{⽰、⽕}
  \definition{v.}{desejar}
\end{entry}

\begin{entry}{著名}{zhu4ming2}{11,6}{⽬、⼝}[HSK 4]
  \definition{adj.}{famoso; bem conhecido; célebre}
\end{entry}

\begin{entry}{著作}{zhu4zuo4}{11,7}{⽬、⼈}[HSK 4]
  \definition[部]{s.}{obra; livro; escritos}
  \definition{v.}{escrever; usar palavras para expressar opiniões, conhecimentos, ideias, sentimentos, etc.}
\end{entry}

\begin{entry}{抓}{zhua1}{7}{⼿}[HSK 3]
  \definition{v.}{agarrar | arranhar | capturar | compreender; conhecer a chave ou a chave das coisas ou problemas | focar em algo; fortalecer o poder de fazer (algo) ou administrar (algum aspecto) | atrair a atenção de alguém}
\end{entry}

\begin{entry}{抓紧}{zhua1jin3}{7,10}{⼿、⽷}[HSK 4]
  \definition{v.}{agarrar com firmeza; segurar firme e não soltar | prestar muita atenção a}
\end{entry}

\begin{entry}{抓住}{zhua1 zhu4}{7,7}{⼿、⼈}[HSK 3]
  \definition{v.}{apanhar; prender; capturar (uma pessoa ou animal) e ter sucesso | segurar; agarrar; segurar algo e deixá-lo imóvel}
\end{entry}

\begin{entry}{转}{zhuai3}{8}{⾞}
  \seeref{转}{zhuan3}
  \seeref{转}{zhuan4}
\end{entry}

\begin{entry}{专家}{zhuan1jia1}{4,10}{⼀、⼧}[HSK 3]
  \definition[个]{s.}{perito; especialista; proficiente; uma pessoa que é especialista em um determinado assunto; uma pessoa que é boa em uma determinada tecnologia}
\end{entry}

\begin{entry}{专门}{zhuan1men2}{4,3}{⼀、⾨}[HSK 3]
  \definition{adj.}{especializado}
  \definition{adv.}{especialmente}
\end{entry}

\begin{entry}{专题}{zhuan1ti2}{4,15}{⼀、⾴}[HSK 3]
  \definition{s.}{assunto especial; tópico especial}
\end{entry}

\begin{entry}{专心}{zhuan1xin1}{4,4}{⼀、⼼}[HSK 4]
  \definition{adj.}{absorto; concentrado}
\end{entry}

\begin{entry}{专业}{zhuan1ye4}{4,5}{⼀、⼀}[HSK 3]
  \definition{adj.}{profissional; descreve uma pessoa que tem um alto nível ou rico conhecimento em uma determinada área}
  \definition[个,门]{s.}{profissão; linha especial; comércio especializado; unidades de negócios no departamento de produção | especialidade; disciplina; assunto especializado; campo especial de estudo; um departamento em uma faculdade ou escola profissional secundária}
\end{entry}

\begin{entry}{专业户}{zhuan1ye4hu4}{4,5,4}{⼀、⼀、⼾}
  \definition{s.}{indústria caseira | empresa familiar produzindo um produto especial}
\end{entry}

\begin{entry}{专业化}{zhuan1ye4hua4}{4,5,4}{⼀、⼀、⼔}
  \definition{s.}{especialização}
\end{entry}

\begin{entry}{专业教育}{zhuan1ye4jiao4yu4}{4,5,11,8}{⼀、⼀、⽁、⾁}
  \definition{s.}{educação especializada | escola técnica}
\end{entry}

\begin{entry}{专业人才}{zhuan1ye4ren2cai2}{4,5,2,3}{⼀、⼀、⼈、⼿}
  \definition{s.}{especialista (em uma área)}
\end{entry}

\begin{entry}{专业人士}{zhuan1ye4ren2shi4}{4,5,2,3}{⼀、⼀、⼈、⼠}
  \definition{s.}{profissional}
\end{entry}

\begin{entry}{专业性}{zhuan1ye4xing4}{4,5,8}{⼀、⼀、⼼}
  \definition{s.}{profissionalismo | expertise}
\end{entry}

\begin{entry}{砖}{zhuan1}{9}{⽯}
  \definition[块]{s.}{tijolo}
\end{entry}

\begin{entry}{转}{zhuan3}{8}{⾞}
  \definition{v.}{mudar; deslocar; transferir; virar; mudar de direção, posição, situação, circunstâncias, etc. | transmitir; transferir; passar adiante}
  \seeref{转}{zhuai3}
  \seeref{转}{zhuan4}
\end{entry}

\begin{entry}{转变}{zhuan3bian4}{8,8}{⾞、⼜}[HSK 3]
  \definition{v.}{mudar; converter; transformar}
\end{entry}

\begin{entry}{转产}{zhuan3chan3}{8,6}{⾞、⼇}
  \definition{v.}{mudar a produção | mudar para novos produtos}
\end{entry}

\begin{entry}{转递}{zhuan3di4}{8,10}{⾞、⾡}
  \definition{v.}{passar | retransmitir}
\end{entry}

\begin{entry}{转动}{zhuan3 dong4}{8,6}{⾞、⼒}[HSK 4]
  \definition{v.}{girar; rodar; dar voltas; torcer | dar a volta em algo}
  \seeref{转动}{zhuan4 dong4}
\end{entry}

\begin{entry}{转告}{zhuan3gao4}{8,7}{⾞、⼝}[HSK 4]
  \definition{v.}{passar adiante; comunicar; transmitir; ser instruído a dizer a outra parte o que uma pessoa diz, o que está acontecendo, etc.}
\end{entry}

\begin{entry}{转念}{zhuan3nian4}{8,8}{⾞、⼼}
  \definition{v.}{ter dúvidas sobre algo | pensar melhor}
\end{entry}

\begin{entry}{转身}{zhuan3 shen1}{8,7}{⾞、⾝}[HSK 4]
  \definition{adv.}{em um instante; em um piscar de olhos}
  \definition{v.}{dar a volta; dar meia-volta; dar a volta por cima | virar; girar; refere-se a uma mudança de direção, localização, natureza, etc.}
\end{entry}

\begin{entry}{转移}{zhuan3yi2}{8,11}{⾞、⽲}[HSK 4]
  \definition{v.}{deslocar; desviar; transferir; redirecionar; reposicionar; reorientar | mudar; transformar}
\end{entry}

\begin{entry}{转账}{zhuan3zhang4}{8,8}{⾞、⾙}
  \definition{v.+compl.}{transferir entre contas | trazer à frente | incluir uma soma de dinheiro do balanço anterior no seguinte}
\end{entry}

\begin{entry}{传}{zhuan4}{6}{⼈}
  \definition{s.}{comentários sobre clássicos | biografia | romances sobre eventos históricos}
  \seeref{传}{chuan2}
\end{entry}

\begin{entry}{转}{zhuan4}{8}{⾞}[HSK 3]
  \definition{clas.}{para ações repetidas | para rotações (por minuto, etc.): RPM}
  \definition{v.}{rodar; girar; virar; dar voltas | passear; andar por aí}
  \seeref{转}{zhuai3}
  \seeref{转}{zhuan3}
\end{entry}

\begin{entry}{转动}{zhuan4 dong4}{8,6}{⾞、⼒}[HSK 4]
  \definition{v.}{girar; correr; rolar; revolver; rotacionar; torcer}
  \seeref{转动}{zhuan3 dong4}
\end{entry}

\begin{entry}{转弯}{zhuan4 wan1}{8,9}{⾞、⼸}[HSK 4]
  \definition{v.}{rodar; desviar; virar uma esquina; fazer uma curva; fazer uma curva de 180º}
\end{entry}

\begin{entry}{转悠}{zhuan4you5}{8,11}{⾞、⼼}
  \definition{v.}{aparecer repetidamente | rolar | passear por aí}
\end{entry}

\begin{entry}{转游}{zhuan4you5}{8,12}{⾞、⽔}
  \variantof{转悠}
\end{entry}

\begin{entry}{妆}{zhuang1}{6}{⼥}
  \definition{s.}{maquiagem | adorno | enxoval | maquiagem e figurino de palco}
  \definition{v.}{maquiar-se | enfeitar-se}
\end{entry}

\begin{entry}{妆扮}{zhuang1ban4}{6,7}{⼥、⼿}
  \variantof{装扮}
\end{entry}

\begin{entry}{桩}{zhuang1}{10}{⽊}
  \definition{clas.}{para eventos, casos, transações, assuntos, etc.}
  \definition{s.}{toco | estaca | pilha}
\end{entry}

\begin{entry}{装}{zhuang1}{12}{⾐}[HSK 2]
  \definition{s.}{adorno | roupa | traje (de um ator em uma peça)}
  \definition{v.}{adornar | vestir | desepenhar um papel | fingir | instalar | consertar | embrulhar (algo em um saco) | empacotar}
\end{entry}

\begin{entry}{装扮}{zhuang1ban4}{12,7}{⾐、⼿}
  \definition{v.}{enfeitar | decorar | disfarçar-me | vestir-se}
\end{entry}

\begin{entry}{装备}{zhuang1bei4}{12,8}{⾐、⼡}
  \definition{s.}{equipamento}
  \definition{v.}{equipar}
\end{entry}

\begin{entry}{装配}{zhuang1pei4}{12,10}{⾐、⾣}
  \definition{v.}{montar | encaixar}
\end{entry}

\begin{entry}{装修}{zhuang1 xiu1}{12,9}{⾐、⼈}[HSK 4]
  \definition{v.}{equipar; renovar; decorar (equipar uma sala ou prédio com equipamentos ou decorações)}
\end{entry}

\begin{entry}{装置}{zhuang1 zhi4}{12,13}{⾐、⽹}[HSK 4]
  \definition{s.}{dispositivo; equipamento; máquinas, instrumentos ou outros equipamentos de construção mais complexa e com alguma função independente}
  \definition{v.}{instalar; ajustar; configurar; equipar; montar}
\end{entry}

\begin{entry}{状况}{zhuang4kuang4}{7,7}{⽝、⼎}[HSK 3]
  \definition[个,种]{s.}{estado; \emph{status}; condição; estado de coisas}
\end{entry}

\begin{entry}{状态}{zhuang4tai4}{7,8}{⽝、⼼}[HSK 3]
  \definition[种,个]{s.}{\emph{status}; estado; condição; estado de coisas; a forma em que uma pessoa ou coisa aparece}
\end{entry}

\begin{entry}{撞车}{zhuang4che1}{15,4}{⼿、⾞}
  \definition{v.+compl.}{(figurativo) colidir (opiniões, cronogramas, etc.) | ser o mesmo (assunto) | colidir (com outro veículo)}
\end{entry}

\begin{entry}{撞运气}{zhuang4yun4qi5}{15,7,4}{⼿、⾡、⽓}
  \definition{v.}{confiar no destino | tentar a sorte}
\end{entry}

\begin{entry}{追}{zhui1}{9}{⾡}[HSK 3]
  \definition*{s.}{sobrenome Zhui}
  \definition{v.}{perseguir; correr atrás; ir atrás de; alcançar | rastrear; investigar; chegar ao fundo de | ansiar por (depois); ir atrás; procurar | recordar; relembrar; lembrar | agir retroativamente; fazer postumamente}
\end{entry}

\begin{entry}{追赶}{zhui1gan3}{9,10}{⾡、⾛}
  \definition{v.}{perseguir | acelerar | alcançar | ultrapassar}
\end{entry}

\begin{entry}{追求}{zhui1qiu2}{9,7}{⾡、⽔}[HSK 4]
  \definition{s.}{perseguição (ações e metas positivas)}[她的追求是获得成功。(Sua meta é alcançar o sucesso.)]
  \definition{v.}{buscar; aspirar; perseguir | cortejar, uma referência especial ao namoro}
\end{entry}

\begin{entry}{坠}{zhui4}{7}{⼟}
  \definition{v.}{cair | pesar | fazer vergar com o peso}
\end{entry}

\begin{entry}{坠落}{zhui4luo4}{7,12}{⼟、⾋}
  \definition{v.}{cair}
\end{entry}

\begin{entry}{屯}{zhun1}{4}{⼬}
  \definition{adj.}{difícil; árduo;}
  \seeref{屯}{tun2}
\end{entry}

\begin{entry}{准}{zhun3}{10}{⼎}[HSK 3]
  \definition{adj.}{exato; preciso; algo determinado a ser imutável | preciso; exato; correto | perto; parcialmente; quase; próximo de algo em termos de padrão}
  \definition{adv.}{definitivamente; certamente}
  \definition{pref.}{quasi-; para-}
  \definition{prep.}{de acordo com; baseado em}
  \definition{s.}{norma; padrão; critério | confiança certa; uma ideia definida, certeza, etc. (geralmente usada depois de ``有'' ou ``没有'')}
  \definition{v.}{autorizar; conceder; consentir; permitir}
  \seealsoref{没有}{mei2you3}
  \seealsoref{有}{you3}
\end{entry}

\begin{entry}{准备}{zhun3bei4}{10,8}{⼎、⼡}[HSK 1]
  \definition{v.}{preparar | ficar pronto | pretender | planejar}
\end{entry}

\begin{entry}{准确}{zhun3que4}{10,12}{⼎、⽯}[HSK 2]
  \definition{adj.}{exato | preciso | acurado}
\end{entry}

\begin{entry}{准时}{zhun3shi2}{10,7}{⼎、⽇}[HSK 4]
  \definition{adj.}{pontual}
  \definition{adv.}{na hora certa; dentro do prazo; no horário especificado}
\end{entry}

\begin{entry}{桌}{zhuo1}{10}{⽊}
  \definition{clas.}{para mesas de convidados em um banquete etc.}
  \definition{s.}{mesa}
\end{entry}

\begin{entry}{桌布}{zhuo1bu4}{10,5}{⽊、⼱}
  \definition[条,块,张]{s.}{(computação) plano de fundo da área de trabalho | toalha de mesa | papel de parede}
\end{entry}

\begin{entry}{桌灯}{zhuo1deng1}{10,6}{⽊、⽕}
  \definition{s.}{luminária | lâmpada de mesa}
\end{entry}

\begin{entry}{桌机}{zhuo1ji1}{10,6}{⽊、⽊}
  \definition{s.}{computador \emph{desktop}}
\end{entry}

\begin{entry}{桌面}{zhuo1mian4}{10,9}{⽊、⾯}
  \definition{s.}{área de trabalho | mesa}
\end{entry}

\begin{entry}{桌球}{zhuo1qiu2}{10,11}{⽊、⽟}
  \definition{s.}{bilhar | sinuca | mesa de ping-pong}
\end{entry}

\begin{entry}{桌游}{zhuo1you2}{10,12}{⽊、⽔}
  \definition{s.}{jogo de tabuleiro}
\end{entry}

\begin{entry}{桌子}{zhuo1zi5}{10,3}{⽊、⼦}[HSK 1]
  \definition[张,套]{s.}{mesa}
\end{entry}

\begin{entry}{棹}{zhuo1}{12}{⽊}
  \variantof{桌}
\end{entry}

\begin{entry}{着}{zhuo2}{11}{⽬}
  \definition{v.}{aplicar | contactar | usar | vestir (roupas)}
  \seeref{着}{zhao1}
  \seeref{着}{zhao2}
  \seeref{着}{zhe5}
\end{entry}

\begin{entry}{着花}{zhuo2hua1}{11,7}{⽬、⾋}
  \definition{s.}{floração}
  \definition{v.}{florescer}
  \seeref{着花}{zhao2hua1}
\end{entry}

\begin{entry}{着手}{zhuo2shou3}{11,4}{⽬、⼿}
  \definition{v.}{colocar a mão nisso | estabelecer | começar uma tarefa}
\end{entry}

\begin{entry}{着想}{zhuo2xiang3}{11,13}{⽬、⼼}
  \definition{v.}{considerar (as necessidades de outras pessoas) | pensar (para os outros)}
\end{entry}

\begin{entry}{着眼}{zhuo2yan3}{11,11}{⽬、⽬}
  \definition{v.}{ter seus olhos em (um objetivo) | ter algo em mente | concentrar-se}
\end{entry}

\begin{entry}{着装}{zhuo2zhuang1}{11,12}{⽬、⾐}
  \definition{s.}{roupa | vestimenta}
  \definition{v.}{vestir}
\end{entry}

\begin{entry}{资}{zi1}{10}{⾙}
  \definition{s.}{recursos | capital | dinheiro | despesa}
  \definition{v.}{fornecer | suprir}
\end{entry}

\begin{entry}{资格}{zi1ge2}{10,10}{⾙、⽊}[HSK 3]
  \definition{s.}{qualificação; as condições e identidades necessárias para exercer uma determinada atividade | senioridade; uma identidade formada pelo tempo gasto realizando um determinado trabalho ou atividade}
\end{entry}

\begin{entry}{资金}{zi1jin1}{10,8}{⾙、⾦}[HSK 3]
  \definition[笔]{s.}{fundo; capital; capital para atividades empresariais}
\end{entry}

\begin{entry}{资料}{zi1liao4}{10,10}{⾙、⽃}[HSK 4]
  \definition[份,个]{s.}{dados; material; material informativo para referência ou para ser considerado confiável | material de produção; meios de subsistência; requisitos de produção ou subsistência}
\end{entry}

\begin{entry}{资源}{zi1yuan2}{10,13}{⾙、⽔}[HSK 4]
  \definition{s.}{recurso; fontes naturais de meios de produção ou subsistência}
\end{entry}

\begin{entry}{资助}{zi1zhu4}{10,7}{⾙、⼒}
  \definition{s.}{subsídio}
  \definition{v.}{subsidiar | fornecer ajuda financeira}
\end{entry}

\begin{entry}{子}{zi3}{3}{⼦}[Kangxi 39]
  \definition{adj.}{jovem | pequeno | tenro}
  \definition{clas.}{para objetos finos que podem ser pinçados com os dedos}
  \definition{pron.}{você}
  \definition{s.}{filho | pessoa | antigo título de respeito para um homem culto ou virtuoso | semente | ovo; ova | coisas pequenas e duras | moeda de cobre; cobre | o quarto título da classificação dos cinco títulos feudais de nobreza; visconde}
  \seeref{子}{zi5}
\end{entry}

\begin{entry}{子弹}{zi3dan4}{3,11}{⼦、⼸}
  \definition[粒,颗,发]{s.}{bala (de revólver)}
\end{entry}

\begin{entry}{子女}{zi3 nv3}{3,3}{⼦、⼥}[HSK 3]
  \definition{s.}{crianças; descendência; filhos e filhas}
\end{entry}

\begin{entry}{紫}{zi3}{12}{⽷}
  \definition{adj.}{púrpura | violeta}
\end{entry}

\begin{entry}{紫色}{zi3 se4}{12,6}{⽷、⾊}
  \definition{s.}{cor púrpura | cor violeta}
\end{entry}

\begin{entry}{字}{zi4}{6}{⼦}[HSK 1]
  \definition[个]{s.}{carácter | letra | símbolo | palavra}
\end{entry}

\begin{entry}{字典}{zi4 dian3}{6,8}{⼦、⼋}[HSK 2]
  \definition[本]{s.}{dicionário de caracteres chineses (contendo verbetes de caracteres únicos, em contraste com 词典 que contém verbetes para palavras com um ou mais caracteres)}
  \seeref{词典}{ci2dian3}
\end{entry}

\begin{entry}{字脚}{zi4jiao3}{6,11}{⼦、⾁}
  \definition[典]{s.}{gancho no final da pincelada | serifa}
\end{entry}

\begin{entry}{字母}{zi4mu3}{6,5}{⼦、⽏}[HSK 4]
  \definition[个]{s.}{letra; letras de um alfabeto | caractere que representa uma consoante inicial (em fonologia)}
\end{entry}

\begin{entry}{字眼}{zi4yan3}{6,11}{⼦、⽬}
  \definition[个]{s.}{palavras | redação}
\end{entry}

\begin{entry}{字字珠玉}{zi4zi4zhu1yu4}{6,6,10,5}{⼦、⼦、⽟、⽟}
  \definition{expr.}{cada palavra é uma jóia}
  \definition{s.}{escrita magnífica}
\end{entry}

\begin{entry}{自}{zi4}{6}{⾃}[HSK 4][Kangxi 132]
  \definition*{s.}{sobrenome Zi}
  \definition{adv.}{certamente; com certeza; é claro; naturalmente}
  \definition{prep.}{de; desde; a partir de; apresenta o ponto de partida, a fonte ou o horário de início do comportamento da ação, equivalente a ``从'' e ``由''}
  \definition{pron.}{si mesmo; próprio | próprio; indica que a ação é iniciada por e direcionada a si mesmo | por si mesmo; indica que a ação é autoiniciada e não é causada por uma força externa}
  \definition{v.}{iniciar}
  \seealsoref{从}{cong2}
  \seealsoref{由}{you2}
\end{entry}

\begin{entry}{自从}{zi4cong2}{6,4}{⾃、⼈}[HSK 3]
  \definition{prep.}{de; desde; apresentando o ponto de partida de um determinado tempo ou evento no passado}
\end{entry}

\begin{entry}{自动}{zi4dong4}{6,6}{⾃、⼒}[HSK 3]
  \definition{adj.}{automático; auto-atuante; uso de dispositivos mecânicos, elétricos e outros para operar de forma independente, sem controle humano}
  \definition{adv.}{voluntariamente; por vontade própria | automaticamente; espontaneamente; refere-se ao movimento, mudança, etc. que não é causado pelo poder humano, mas pelo próprio objeto}
\end{entry}

\begin{entry}{自动化}{zi4dong4hua4}{6,6,4}{⾃、⼒、⼔}
  \definition{s.}{automação}
\end{entry}

\begin{entry}{自个儿}{zi4ge3r5}{6,3,2}{⾃、⼈、⼉}
  \definition{pron.}{(dialeto) a si mesmo, por si mesmo}
\end{entry}

\begin{entry}{自己}{zi4ji3}{6,3}{⾃、⼰}[HSK 2]
  \definition{pron.}{a si próprio | próprio}
\end{entry}

\begin{entry}{自己动手}{zi4ji3dong4shou3}{6,3,6,4}{⾃、⼰、⼒、⼿}
  \definition{v.}{fazer (algo) sozinho | ajudar-se a}
\end{entry}

\begin{entry}{自救}{zi4jiu4}{6,11}{⾃、⽁}
  \definition{v.}{sair a si mesmo de problemas}
\end{entry}

\begin{entry}{自觉}{zi4jue2}{6,9}{⾃、⾒}[HSK 3]
  \definition{adj.}{autoconsciente; de ​​livre e espontânea vontade; tomar a iniciativa de fazer as coisas}
  \definition{v.}{estar ciente de}
\end{entry}

\begin{entry}{自来水}{zi4lai2shui3}{6,7,4}{⾃、⽊、⽔}
  \definition{s.}{água corrente | água da torneira}
\end{entry}

\begin{entry}{自然}{zi4ran2}{6,12}{⾃、⽕}[HSK 3]
  \definition{adj.}{natural; no curso normal dos eventos; formado ou desenvolvido sem intervenção humana; algo que se desenvolve livremente}
  \definition{adv.}{naturalmente; certamente; definitivamente}
  \definition{conj.}{usado para ligar duas cláusulas ou frases, com a segunda introduzindo informações adicionais ou adversativas; indica explicação adicional ou uma mudança de significado}
  \definition{s.}{natureza; mundo natural; tudo o que não é criado pelos humanos}
\end{entry}

\begin{entry}{自燃}{zi4ran2}{6,16}{⾃、⽕}
  \definition{s.}{combustão espontânea}
\end{entry}

\begin{entry}{自身}{zi4 shen1}{6,7}{⾃、⾝}[HSK 3]
  \definition{pron.}{eu mesmo; si mesmo}
\end{entry}

\begin{entry}{自我}{zi4wo3}{6,7}{⾃、⼽}
  \definition{pref.}{auto}
  \definition{pron.}{a si mesmo | eu próprio | (psicologia) ego}
\end{entry}

\begin{entry}{自我安慰}{zi4wo3'an1wei4}{6,7,6,15}{⾃、⼽、⼧、⼼}
  \definition{v.}{confortar-se | consolar-se | tranquilizar-se}
\end{entry}

\begin{entry}{自我保存}{zi4wo3 bao3cun2}{6,7,9,6}{⾃、⼽、⼈、⼦}
  \definition{v.}{autopreservação}
\end{entry}

\begin{entry}{自我吹嘘}{zi4wo3chui1xu1}{6,7,7,14}{⾃、⼽、⼝、⼝}
  \definition{expr.}{tocar a própria buzina}
\end{entry}

\begin{entry}{自我催眠}{zi4wo3cui1mian2}{6,7,13,10}{⾃、⼽、⼈、⽬}
  \definition{v.}{consolar-me | tranquilizar-me}
\end{entry}

\begin{entry}{自我的人}{zi4wo3de5ren2}{6,7,8,2}{⾃、⼽、⽩、⼈}
  \definition{s.}{(minha, sua) própria pessoa | (afirmar) a própria personalidade}
\end{entry}

\begin{entry}{自我防卫}{zi4wo3fang2wei4}{6,7,6,3}{⾃、⼽、⾩、⼙}
  \definition{s.}{defesa pessoal | auto-defesa}
\end{entry}

\begin{entry}{自我解嘲}{zi4wo3jie3chao2}{6,7,13,15}{⾃、⼽、⾓、⼝}
  \definition{s.}{referir-se às próprias fraquezas ou falhas com humor autodepreciativo}
\end{entry}

\begin{entry}{自我介绍}{zi4wo3jie4shao4}{6,7,4,8}{⾃、⼽、⼈、⽷}
  \definition{s.}{defesa pessoal | auto-defesa}
\end{entry}

\begin{entry}{自我批评}{zi4wo3pi1ping2}{6,7,7,7}{⾃、⼽、⼿、⾔}
  \definition{s.}{autocrítica}
\end{entry}

\begin{entry}{自我实现}{zi4wo3shi2xian4}{6,7,8,8}{⾃、⼽、⼧、⾒}
  \definition{s.}{(psicologia) auto-atualização, auto-realização}
\end{entry}

\begin{entry}{自我陶醉}{zi4wo3tao2zui4}{6,7,10,15}{⾃、⼽、⾩、⾣}
  \definition{s.}{narcisista | auto-imbuído | satisfeito consigo mesmo}
\end{entry}

\begin{entry}{自我意识}{zi4wo3yi4shi2}{6,7,13,7}{⾃、⼽、⼼、⾔}
  \definition{s.}{autoapresentação}
  \definition{v.}{apresentar-se}
\end{entry}

\begin{entry}{自信}{zi4xin4}{6,9}{⾃、⼈}[HSK 4]
  \definition{adj.}{confiante; descreve a crença em suas próprias habilidades, decisões, etc., tendo confiança em si mesmo}
  \definition[份,种]{s.}{autoconfiança; confiança em si mesmo}
  \definition{v.}{acreditar em si mesmo;}
\end{entry}

\begin{entry}{自行车}{zi4xing2che1}{6,6,4}{⾃、⾏、⾞}[HSK 2]
  \definition[辆]{s.}{bicicleta}
\end{entry}

\begin{entry}{自行车馆}{zi4xing2che1guan3}{6,6,4,11}{⾃、⾏、⾞、⾷}
  \definition{s.}{estádio de ciclismo | velódromo}
\end{entry}

\begin{entry}{自行车架}{zi4xing2che1jia4}{6,6,4,9}{⾃、⾏、⾞、⽊}
  \definition{s.}{suporte para bicicleta | bicicletário}
\end{entry}

\begin{entry}{自行车赛}{zi4xing2che1sai4}{6,6,4,14}{⾃、⾏、⾞、⾙}
  \definition{s.}{corrida de bicicleta}
\end{entry}

\begin{entry}{自由}{zi4you2}{6,5}{⾃、⽥}[HSK 2]
  \definition{adj.}{livre, irrestrito}
  \definition[种]{s.}{liberdade}
\end{entry}

\begin{entry}{自由泳}{zi4you2yong3}{6,5,8}{⾃、⽥、⽔}
  \definition{s.}{natação de estilo livre}
\end{entry}

\begin{entry}{自责}{zi4ze2}{6,8}{⾃、⾙}
  \definition{v.}{culpar-se}
\end{entry}

\begin{entry}{自主}{zi4zhu3}{6,5}{⾃、⼂}[HSK 3]
  \definition{v.}{agir por conta própria; decidir por si mesmo; manter a iniciativa em suas próprias mãos}
\end{entry}

\begin{entry}{子}{zi5}{3}{⼦}[HSK 1]
  \definition{clas.}{sufixos de palavras de medida individuais}
  \definition{suf.}{sufixo para substantivos}
  \seeref{子}{zi3}
\end{entry}

\begin{entry}{综合}{zong1he2}{11,6}{⽷、⼝}[HSK 4]
  \definition{s.}{síntese}
  \definition{v.}{sintetizar; resumir as partes de uma coisa em um todo unificado após análise (em oposição a ``分析''); reunir coisas de um tipo ou natureza diferente}
  \seealsoref{分析}{fen1xi1}
\end{entry}

\begin{entry}{棕褐色}{zong1he4 se4}{12,14,6}{⽊、⾐、⾊}
  \definition{s.}{cor sépia | bronzeado}
\end{entry}

\begin{entry}{总}{zong3}{9}{⼼}[HSK 3]
  \definition{adj.}{total; geral; global | responsável (liderança)}
  \definition{adv.}{sempre; invariavelmente | de qualquer forma; afinal; eventualmente; mais cedo ou mais tarde | seguramente; provavelmente; certamente}
  \definition{v.}{resumir; juntar; reunir}
\end{entry}

\begin{entry}{总长}{zong3chang2}{9,4}{⼼、⾧}
  \definition{s.}{comprimento total}
\end{entry}

\begin{entry}{总得}{zong3dei3}{9,11}{⼼、⼻}
  \definition{adv.}{prestes a}
  \definition{v.}{dever | precisar}
\end{entry}

\begin{entry}{总督}{zong3du1}{9,13}{⼼、⽬}
  \definition*{s.}{Governador-Geral | Governador | Vice-Rei}
\end{entry}

\begin{entry}{总共}{zong3gong4}{9,6}{⼼、⼋}[HSK 4]
  \definition{adv.}{em tudo; em todos; no total; completamente; totalmente; em conjunto}
\end{entry}

\begin{entry}{总价}{zong3jia4}{9,6}{⼼、⼈}
  \definition{s.}{preço total}
\end{entry}

\begin{entry}{总结}{zong3jie2}{9,9}{⼼、⽷}[HSK 3]
  \definition[个,份]{s.}{resumo; conclusão obtida}
  \definition{v.}{resumir; sumariar; analisar a experiência da pesquisa e tirar conclusões}
\end{entry}

\begin{entry}{总理}{zong3li3}{9,11}{⼼、⽟}[HSK 4]
  \definition*[个,位,名]{s.}{Primeiro-Ministro do Conselho de Estado; Título do líder do Conselho de Estado da China | Título do chefe de governo em determinados países | Primeiro-Ministro; Título de líderes de determinados partidos políticos | Título dos chefes de determinadas instituições e empresas nos velhos tempos}
  \definition{v.}{assumir a responsabilidade total;}
\end{entry}

\begin{entry}{总是}{zong3shi4}{9,9}{⼼、⽇}[HSK 3]
  \definition{adv.}{sempre; indica que algo está acontecendo por um período de tempo; um certo estado permanece inalterado
 | afinal; significa que não importa o que aconteça, haverá um resultado.}
\end{entry}

\begin{entry}{总台}{zong3tai2}{9,5}{⼼、⼝}
  \definition{s.}{recepção | balcão de recepção}
\end{entry}

\begin{entry}{总统}{zong3tong3}{9,9}{⼼、⽷}[HSK 4]
  \definition*[个,位,名]{s.}{Presidente (de um país); Título dos líderes de determinadas repúblicas}
\end{entry}

\begin{entry}{总务}{zong3wu4}{9,5}{⼼、⼒}
  \definition{s.}{divisão de assuntos gerais | assuntos gerais | pessoa responsável geral}
\end{entry}

\begin{entry}{总线}{zong3xian4}{9,8}{⼼、⽷}
  \definition{s.}{barramento (computador) | \emph{computer bus}}
\end{entry}

\begin{entry}{总站}{zong3zhan4}{9,10}{⼼、⽴}
  \definition{s.}{terminal}
\end{entry}

\begin{entry}{总之}{zong3zhi1}{9,3}{⼼、⼂}[HSK 4]
  \definition{conj.}{em uma palavra; em suma; em resumo; indica que a declaração seguinte é uma declaração geral}
\end{entry}

\begin{entry}{总值}{zong3zhi2}{9,10}{⼼、⼈}
  \definition{s.}{valor total}
\end{entry}

\begin{entry}{赱}{zou3}{6}{⼟}
  \variantof{走}
\end{entry}

\begin{entry}{走}{zou3}{7}{⾛}[HSK 1][Kangxi 156]
  \definition{v.}{andar | caminhar}
\end{entry}

\begin{entry}{走鬼}{zou3gui3}{7,9}{⾛、⿁}
  \definition{s.}{vendedor ambulante sem licença}
\end{entry}

\begin{entry}{走过}{zou3 guo4}{7,6}{⾛、⾡}[HSK 2]
  \definition{v.}{passar}
\end{entry}

\begin{entry}{走进}{zou3 jin4}{7,7}{⾛、⾡}[HSK 2]
  \definition{v.}{entrar}
\end{entry}

\begin{entry}{走开}{zou3 kai1}{7,4}{⾛、⼶}[HSK 2]
  \definition{v.}{ir embora | fugir | ir para outro lugar}
\end{entry}

\begin{entry}{走路}{zou3lu4}{7,13}{⾛、⾜}[HSK 1]
  \definition{v.}{andar | ir a pé | sair | ir embora}
\end{entry}

\begin{entry}{走去}{zou3qu4}{7,5}{⾛、⼛}
  \definition{v.}{caminhar até (para)}
\end{entry}

\begin{entry}{走绳}{zou3sheng2}{7,11}{⾛、⽷}
  \definition{v.}{andar na corda bamba}
  \seeref{走索}{zou3suo3}
\end{entry}

\begin{entry}{走势}{zou3shi4}{7,8}{⾛、⼒}
  \definition{s.}{caminho | tendência}
\end{entry}

\begin{entry}{走索}{zou3suo3}{7,10}{⾛、⽷}
  \definition{v.}{andar na corda bamba}
  \seeref{走绳}{zou3sheng2}
\end{entry}

\begin{entry}{走秀}{zou3xiu4}{7,7}{⾛、⽲}
  \definition{s.}{desfile de moda}
  \definition{v.}{andar na passarela (em um desfile de moda)}
\end{entry}

\begin{entry}{走卒}{zou3zu2}{7,8}{⾛、⼗}
  \definition{s.}{lacaio (masculino) | peão (isto é, soldado de infantaria) | servo}
\end{entry}

\begin{entry}{奏效}{zou4xiao4}{9,10}{⼤、⽁}
  \definition{v.}{mostrar resultados | ser eficaz}
\end{entry}

\begin{entry}{租}{zu1}{10}{⽲}[HSK 2]
  \definition{s.}{imposto sobre propriedade urbana ou rural}
  \definition{v.}{alugar | tomar de aluguel}
\end{entry}

\begin{entry}{租船}{zu1chuan2}{10,11}{⽲、⾈}
  \definition{v.}{fretar um navio | alugar um navio}
\end{entry}

\begin{entry}{租房}{zu1fang2}{10,8}{⽲、⼾}
  \definition{v.}{alugar um apartamento}
\end{entry}

\begin{entry}{租金}{zu1jin1}{10,8}{⽲、⾦}
  \definition{s.}{aluguel}
  \seeref{租钱}{zu1qian5}
\end{entry}

\begin{entry}{租赁}{zu1lin4}{10,10}{⽲、⾙}
  \definition{v.}{contratar | alugar}
\end{entry}

\begin{entry}{租钱}{zu1qian5}{10,10}{⽲、⾦}
  \definition{s.}{aluguel}
  \seeref{租金}{zu1jin1}
\end{entry}

\begin{entry}{租让}{zu1rang4}{10,5}{⽲、⾔}
  \definition{v.}{alugar | alugar (a propriedade de alguém para outra pessoa)}
\end{entry}

\begin{entry}{租用}{zu1yong4}{10,5}{⽲、⽤}
  \definition{v.}{contratar | alugar | alugar (algo de alguém)}
\end{entry}

\begin{entry}{租约}{zu1yue1}{10,6}{⽲、⽷}
  \definition{s.}{aluguel}
\end{entry}

\begin{entry}{足}{zu2}{7}{⾜}[Kangxi 157]
  \definition{adj.}{amplo}
  \definition{s.}{pé}
  \definition{v.}{ser suficiente}
  \seeref{足}{ju4}
\end{entry}

\begin{entry}{足够}{zu2 gou4}{7,11}{⾜、⼣}[HSK 3]
  \definition{adj.}{bastante; amplo; suficiente; na medida em que deve ser ou pode atender às necessidades}
  \definition{v.}{satisfazer; ser suficiente; estar a contento}
\end{entry}

\begin{entry}{足球}{zu2qiu2}{7,11}{⾜、⽟}[HSK 3]
  \definition[个,只,颗,袋]{s.}{futebol | bola de futebol}
\end{entry}

\begin{entry}{足球场}{zu2qiu2chang3}{7,11,6}{⾜、⽟、⼟}
  \definition{s.}{campo de futebol}
\end{entry}

\begin{entry}{足球队}{zu2qiu2dui4}{7,11,4}{⾜、⽟、⾩}
  \definition{s.}{time de futebol}
\end{entry}

\begin{entry}{足球迷}{zu2qiu2mi2}{7,11,9}{⾜、⽟、⾡}
  \definition{s.}{fã de futebol}
\end{entry}

\begin{entry}{足球赛}{zu2qiu2sai4}{7,11,14}{⾜、⽟、⾙}
  \definition{s.}{competição de futebol | partida de futebol}
\end{entry}

\begin{entry}{足球协会}{zu2qiu2xie2hui4}{7,11,6,6}{⾜、⽟、⼗、⼈}
  \definition*{s.}{Associação de Futebol}
\end{entry}

\begin{entry}{足月}{zu2yue4}{7,4}{⾜、⽉}
  \definition{s.}{gestação completa}
\end{entry}

\begin{entry}{足足}{zu2zu2}{7,7}{⾜、⾜}
  \definition{adv.}{tanto quanto | extremamente | completamente | não menos que}
\end{entry}

\begin{entry}{族}{zu2}{11}{⽅}
  \definition{s.}{raça | nacionalidade | etnia | clã | por extensão, grupo social}
\end{entry}

\begin{entry}{诅咒}{zu3zhou4}{7,8}{⾔、⼝}
  \definition{v.}{amaldiçoar}
\end{entry}

\begin{entry}{阻击}{zu3ji1}{7,5}{⾩、⼐}
  \definition{v.}{verificar | parar}
\end{entry}

\begin{entry}{阻止}{zu3zhi3}{7,4}{⾩、⽌}[HSK 4]
  \definition{v.}{parar; reter; conter; interromper; impedir o avanço; impedir o movimento; obstruir}
\end{entry}

\begin{entry}{组}{zu3}{8}{⽷}[HSK 2]
  \definition*{s.}{sobrenome Zu}
  \definition{clas.}{para conjuntos, séries, suítes, baterias}
  \definition{s.}{grupo}
  \definition{v.}{organizar | formar}
\end{entry}

\begin{entry}{组成}{zu3cheng2}{8,6}{⽷、⼽}[HSK 2]
  \definition{v.}{formar | compor | inventar}
\end{entry}

\begin{entry}{组合}{zu3he2}{8,6}{⽷、⼝}[HSK 3]
  \definition{s.}{associação; combinação
combinação; pegue n elementos de m elementos diferentes e agrupe-os em grupos, independentemente da ordem, onde cada grupo contém pelo menos um componente diferente, o resultado é chamado de combinação de n de m.}
  \definition{v.}{criar; compor; constituir}
\end{entry}

\begin{entry}{组长}{zu3 zhang3}{8,4}{⽷、⾧}[HSK 2]
  \definition[名,位,个]{s.}{líder de grupo}
\end{entry}

\begin{entry}{祖国}{zu3guo2}{9,8}{⽰、⼞}
  \definition{s.}{pátria | terra natal}
\end{entry}

\begin{entry}{钻戒}{zuan4jie4}{10,7}{⾦、⼽}
  \definition[只]{s.}{anel de diamante}
\end{entry}

\begin{entry}{钻石}{zuan4shi2}{10,5}{⾦、⽯}
  \definition[颗]{s.}{diamante}
\end{entry}

\begin{entry}{嘴}{zui3}{16}{⼝}[HSK 2]
  \definition[张]{s.}{boca | qualquer coisa com formato ou função semelhante a uma boca}
  \definition{v.}{falar}
\end{entry}

\begin{entry}{嘴巴}{zui3 ba5}{16,4}{⼝、⼰}[HSK 4]
  \definition[张]{s.}{boca}
\end{entry}

\begin{entry}{嘴巴子}{zui3ba5zi5}{16,4,3}{⼝、⼰、⼦}
  \definition{s.}{tapa | bofetada}
\end{entry}

\begin{entry}{最}{zui4}{12}{⽈}[HSK 1]
  \definition{adv.}{o mais | o melhor | a coisa mais\dots | grau superlativo relativo de superioridade}
\end{entry}

\begin{entry}{最初}{zui4chu1}{12,7}{⽈、⾐}[HSK 4]
  \definition{adj.}{primordial; inicial; primeiro}
  \definition{adv.}{inicialmente; originalmente}
  \definition{s.}{o período mais antigo; início; começo}
\end{entry}

\begin{entry}{最多}{zui4duo1}{12,6}{⽈、⼣}
  \definition{adv.}{no máximo | máximo}
\end{entry}

\begin{entry}{最高}{zui4gao1}{12,10}{⽈、⾼}
  \definition{adj.}{altíssimo | supremo | mais alto}
\end{entry}

\begin{entry}{最好}{zui4hao3}{12,6}{⽈、⼥}[HSK 1]
  \definition{adv.}{ser melhor que}
  \definition{v.}{(você) estar melhor (faça o que sugerimos) | querer ser o melhor}
\end{entry}

\begin{entry}{最后}{zui4hou4}{12,6}{⽈、⼝}[HSK 1]
  \definition{adj.}{final | último}
  \definition{adv.}{finalmente}
\end{entry}

\begin{entry}{最佳}{zui4jia1}{12,8}{⽈、⼈}
  \definition{adj.}{melhor (atleta, filme etc) | ótimo}
\end{entry}

\begin{entry}{最近}{zui4jin4}{12,7}{⽈、⾡}[HSK 2]
  \definition{adv.}{ultimamente | recentemente}
\end{entry}

\begin{entry}{最善}{zui4shan4}{12,12}{⽈、⼝}
  \definition{adj.}{ótimo | o melhor}
\end{entry}

\begin{entry}{最少}{zui4shao3}{12,4}{⽈、⼩}
  \definition{adv.}{finalmente}
\end{entry}

\begin{entry}{最先}{zui4xian1}{12,6}{⽈、⼉}
  \definition{adv.}{o primeiro}
\end{entry}

\begin{entry}{最新}{zui4xin1}{12,13}{⽈、⽄}
  \definition{adv.}{mais recente | mais novo}
\end{entry}

\begin{entry}{最优}{zui4you1}{12,6}{⽈、⼈}
  \definition{adj.}{ótimo}
\end{entry}

\begin{entry}{最远}{zui4yuan3}{12,7}{⽈、⾡}
  \definition{adv.}{mais distante | mais longe}
\end{entry}

\begin{entry}{最终}{zui4zhong1}{12,8}{⽈、⽷}
  \definition{adv.}{pelo menos | finalmente}
  \definition{s.}{final | ultimato}
\end{entry}

\begin{entry}{罪犯}{zui4fan4}{13,5}{⽹、⽝}
  \definition{s.}{criminoso}
\end{entry}

\begin{entry}{罪行}{zui4xing2}{13,6}{⽹、⾏}
  \definition{s.}{crime | ofensa}
\end{entry}

\begin{entry}{醉}{zui4}{15}{⾣}
  \definition{v.}{embriagar-se | ficar bêbado}
\end{entry}

\begin{entry}{作}{zuo1}{7}{⼈}
  \definition{adj.}{(gíria) incômodo}
  \definition{s.}{trabalhador | oficina | (pessoa) de alta manutenção}
  \seeref{作}{zuo4}
\end{entry}

\begin{entry}{昨}{zuo2}{9}{⽇}
  \definition{s.}{ontem}
\end{entry}

\begin{entry}{昨日}{zuo2ri4}{9,4}{⽇、⽇}
  \definition{adv.}{ontem}
\end{entry}

\begin{entry}{昨天}{zuo2tian1}{9,4}{⽇、⼤}[HSK 1]
  \definition{adv.}{ontem}
\end{entry}

\begin{entry}{昨晚}{zuo2wan3}{9,11}{⽇、⽇}
  \definition{adv.}{noite passada | ontem à noite}
\end{entry}

\begin{entry}{昨夜}{zuo2ye4}{9,8}{⽇、⼣}
  \definition{adv.}{noite passada}
\end{entry}

\begin{entry}{左}{zuo3}{5}{⼯}[HSK 1]
  \definition*{s.}{sobrenome Zuo}
  \definition{s.}{esquerda}
\end{entry}

\begin{entry}{左边}{zuo3bian5}{5,5}{⼯、⾡}[HSK 1]
  \definition{s.}{esquerda | lado esquerdo}
\end{entry}

\begin{entry}{左面}{zuo3mian4}{5,9}{⼯、⾯}
  \definition{s.}{esquerda | lado esquerdo}
\end{entry}

\begin{entry}{左派}{zuo3pai4}{5,9}{⼯、⽔}
  \definition{s.}{(política) esquerda | esquerdista}
\end{entry}

\begin{entry}{左倾}{zuo3qing1}{5,10}{⼯、⼈}
  \definition{s.}{esquerdista | progressivo}
\end{entry}

\begin{entry}{左袒}{zuo3tan3}{5,10}{⼯、⾐}
  \definition{v.}{ser tendencioso | ser parcial para | favorecer um lado | tomar partido com}
\end{entry}

\begin{entry}{左舷}{zuo3xian2}{5,11}{⼯、⾈}
  \definition{s.}{porto (lado de um navio)}
\end{entry}

\begin{entry}{左翼}{zuo3yi4}{5,17}{⼯、⽻}
  \definition{s.}{esquerda (política)}
\end{entry}

\begin{entry}{左右}{zuo3you4}{5,5}{⼯、⼝}[HSK 3]
  \definition{adv.}{aproximadamente; ou mais ou menos; por aí; usado depois de um número para indicar um número aproximado, o mesmo que ``上下''}
  \definition{s.}{os lados esquerdo e direito; esquerda e direita, também significa circundar
atendentes; pessoas que te seguem}
  \definition{v.}{controlar; manipular; influenciar}
\end{entry}

\begin{entry}{作}{zuo4}{7}{⼈}
  \definition{s.}{escritos ou obras}
  \definition{v.}{fazer | crescer | escrever ou compor | fingir | considerar como | sentir}
  \seeref{作}{zuo1}
\end{entry}

\begin{entry}{作出}{zuo4 chu1}{7,5}{⼈、⼐}[HSK 4]
  \definition{v.}{mostrar; tomar (decisões, conclusões, etc. por meio de consideração ou discussão); formar (uma conclusão, decisão, etc.) por meio de consideração ou discussão}
\end{entry}

\begin{entry}{作家}{zuo4jia1}{7,10}{⼈、⼧}[HSK 2]
  \definition[位,个]{s.}{autor | escritor}
\end{entry}

\begin{entry}{作品}{zuo4pin3}{7,9}{⼈、⼝}[HSK 3]
  \definition[个,部,篇,幅]{s.}{obra de arte; obras concluídas de literatura e arte}
\end{entry}

\begin{entry}{作为}{zuo4wei2}{7,4}{⼈、⼂}[HSK 4]
  \definition{prep.}{como; na capacidade de; no caráter de; no papel de; em termos de uma certa identidade de uma pessoa ou de uma certa natureza de uma coisa}
  \definition{s.}{ato; ação; conduta; feito; comportamento | conquista; realização; especificamente, uma boa ação}
  \definition{v.}{considerar como; tomar por; olhar como; tratar como | realizar; fazer conquistas; deixar uma marca}
\end{entry}

\begin{entry}{作文}{zuo4wen2}{7,4}{⼈、⽂}[HSK 2]
  \definition[篇]{s.}{ensaio |  composição | redação}
  \definition{v.+compl.}{(de alunos) para escrever uma redação}
\end{entry}

\begin{entry}{作业}{zuo4ye4}{7,5}{⼈、⼀}[HSK 2]
  \definition[份,个]{s.}{tarefa escolar | trabalho | tarefa | operação}
\end{entry}

\begin{entry}{作用}{zuo4yong4}{7,5}{⼈、⽤}[HSK 2]
  \definition{s.}{efeito | ação | função}
  \definition{v.}{afetar | agir em}
\end{entry}

\begin{entry}{作者}{zuo4zhe3}{7,8}{⼈、⽼}[HSK 3]
  \definition[位,名,个]{s.}{autor; escritor; uma pessoa que escreve artigos ou cria obras de arte}
\end{entry}

\begin{entry}{坐}{zuo4}{7}{⼟}[HSK 1]
  \definition*{s.}{sobrenome Zuo}
  \definition{v.}{sentar-se | andar de carro, ônibus, trem, avião, etc.}
\end{entry}

\begin{entry}{坐标}{zuo4biao1}{7,9}{⼟、⽊}
  \definition{s.}{coordenada (geometria)}
\end{entry}

\begin{entry}{坐车}{zuo4che1}{7,4}{⼟、⾞}
  \definition{v.}{andar de carro, ônibus, trem, etc.}
\end{entry}

\begin{entry}{坐垫}{zuo4dian4}{7,9}{⼟、⼟}
  \definition[块]{s.}{assento (motocicleta) | almofada}
\end{entry}

\begin{entry}{坐好}{zuo4hao3}{7,6}{⼟、⼥}
  \definition{v.}{sentar-se corretamente | sentar direito}
\end{entry}

\begin{entry}{坐下}{zuo4xia5}{7,3}{⼟、⼀}[HSK 1]
  \definition{v.}{sentar-se | tomar um assento}
\end{entry}

\begin{entry}{坐享}{zuo4xiang3}{7,8}{⼟、⼇}
  \definition{v.}{curtir algo sem levantar um dedo}
\end{entry}

\begin{entry}{座}{zuo4}{10}{⼴}[HSK 2]
  \definition{clas.}{frequentemente usado para objetos maiores ou fixos}
  \definition{s.}{assento | lugar | base | suporte | pedestal | constelação}
\end{entry}

\begin{entry}{座标}{zuo4biao1}{10,9}{⼴、⽊}
  \variantof{坐标}
\end{entry}

\begin{entry}{座位}{zuo4wei4}{10,7}{⼴、⼈}[HSK 2]
  \definition[个]{s.}{assento | lugar}
\end{entry}

\begin{entry}{座子}{zuo4zi5}{10,3}{⼴、⼦}
  \definition{s.}{soquete | pedestal | sela}
\end{entry}

\begin{entry}{做}{zuo4}{11}{⼈}[HSK 1]
  \definition{v.}{fazer}
\end{entry}

\begin{entry}{做到}{zuo4 dao4}{11,8}{⼈、⼑}[HSK 2]
  \definition{v.}{realizar | alcançar}
\end{entry}

\begin{entry}{做法}{zuo4fa3}{11,8}{⼈、⽔}[HSK 2]
  \definition[个]{s.}{método para fazer | prática | receita | maneira de lidar com algo | método de trabalho}
\end{entry}

\begin{entry}{做饭}{zuo4 fan4}{11,7}{⼈、⾷}[HSK 2]
  \definition{v.}{preparar uma refeição | cozinhar}
\end{entry}

\begin{entry}{做活}{zuo4huo2}{11,9}{⼈、⽔}
  \definition{v.}{trabalhar para ganhar a vida (especialmente de mulher costureira)}
\end{entry}

\begin{entry}{做客}{zuo4 ke4}{11,9}{⼈、⼧}[HSK 3]
  \definition{v.}{visitar; ser um convidado; visitar outras pessoas e ser você mesmo o convidado}
\end{entry}

\begin{entry}{做梦}{zuo4 meng4}{11,11}{⼈、⼣}[HSK 4]
  \definition{s.}{sonho; ilusões e visões na consciência durante o sono}
  \definition{v.}{sonhar; ter um sonho | sonhar acordado, ter um sonho impossível (parábola de fantasias irrealistas)}[别​做​梦​了​,她​不​会​嫁​给​你​的​。(Pare de sonhar, ela não se casará com você.)]
\end{entry}

\begin{entry}{做生活}{zuo4sheng1huo2}{11,5,9}{⼈、⽣、⽔}
  \definition{v.}{fazer tabalhos manuais}
\end{entry}

\begin{entry}{做戏}{zuo4xi4}{11,6}{⼈、⼽}
  \definition{v.}{atuar em uma peça | fazer uma peça}
\end{entry}

\begin{entry}{做眼}{zuo4yan3}{11,11}{⼈、⽬}
  \definition{v.}{agir como um guia | trabalhar como espião}
\end{entry}

\begin{entry}{做作}{zuo4zuo5}{11,7}{⼈、⼈}
  \definition{adj.}{afetado | artificial}
\end{entry}

\begin{entry}{酢}{zuo4}{12}{⾣}
  \definition{v.}{brindar o anfitrião com vinho}
\end{entry}

%%%%% EOF %%%%%

