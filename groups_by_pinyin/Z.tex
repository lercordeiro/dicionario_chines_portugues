%%%
%%% Z
%%%

\section*{Z}\addcontentsline{toc}{section}{Z}

\begin{entry}{杂技}{za2ji4}{6,7}{⽊、⼿}
  \definition[场]{s.}{acrobacia}
\end{entry}

\begin{entry}{杂志}{za2zhi4}{6,7}{⽊、⼼}[HSK 3]
  \definition[本,期,份,种]{s.}{jornal; revista; publicação}
\end{entry}

\begin{entry}{杂志社}{za2zhi4she4}{6,7,7}{⽊、⼼、⽰}
  \definition{s.}{editora de revista}
\end{entry}

\begin{entry}{咱}{za2}{9}{⼝}
  \seeref{咱}{zan2}
  \seeref{咱}{zan5}
\end{entry}

\begin{entry}{咱家}{za2jia1}{9,10}{⼝、⼧}
  \definition{pron.}{eu (frequentemente usado na literatura vernácula antiga) | me | mim | comigo}
\end{entry}

\begin{entry}{砸}{za2}{10}{⽯}
  \definition{v.}{esmagar | bater | falhar | estragar}
\end{entry}

\begin{entry}{灾}{zai1}{7}{⽕}[HSK 5]
  \definition[个,场]{s.}{calamidade; desastre | infortúnio pessoal; adversidade | azar}
\end{entry}

\begin{entry}{灾害}{zai1hai4}{7,10}{⽕、⼧}[HSK 5]
  \definition[个]{s.}{desastre; calamidade; danos causados pela seca, inundações, pragas, granizo, guerras, etc.}
\end{entry}

\begin{entry}{灾难}{zai1nan4}{7,10}{⽕、⾫}[HSK 5]
  \definition[场,次]{s.}{desastre; sofrimento; calamidade; catástrofe; danos e sofrimentos causados por desastres naturais ou guerras}
\end{entry}

\begin{entry}{灾区}{zai1 qu1}{7,4}{⽕、⼖}[HSK 5]
  \definition{s.}{área de desastre; área afetada por catástrofes}
\end{entry}

\begin{entry}{栽}{zai1}{10}{⽊}
  \definition{v.}{cultivar | plantar}
\end{entry}

\begin{entry}{栽倒}{zai1dao3}{10,10}{⽊、⼈}
  \definition{v.}{cair | sofrer uma queda}
\end{entry}

\begin{entry}{栽培}{zai1pei2}{10,11}{⽊、⼟}
  \definition{v.}{cultivar | educar | patrocinar | treinar}
\end{entry}

\begin{entry}{栽培种}{zai1pei2 zhong3}{10,11,9}{⽊、⼟、⽲}
  \definition{s.}{espécies cultivadas}
\end{entry}

\begin{entry}{栽赃}{zai1zang1}{10,10}{⽊、⾙}
  \definition{v.}{enquadrar alguém (plantar provas nele)}
\end{entry}

\begin{entry}{栽植}{zai1zhi2}{10,12}{⽊、⽊}
  \definition{v.}{plantar | transplantar}
\end{entry}

\begin{entry}{栽种}{zai1zhong4}{10,9}{⽊、⽲}
  \definition{v.}{plantar}
\end{entry}

\begin{entry}{再}{zai4}{6}{⼌}[HSK 1]
  \definition{adv.}{mais uma vez; além disso; ainda mais; indica a repetição ou continuação de uma mesma ação ou comportamento; refere-se principalmente a ações ou comportamentos não realizados ou contínuos | usado antes do adjetivo, indica intensificação, equivalente a 更 ou 更加 | (para uma ação adiada, precedida por uma expressão de tempo ou condição) então; somente então; depois de algo; indica que a ação ocorrerá após a conclusão de outra ação | além disso; indica um complemento, equivalente a 另外 ou 又 | próxima vez; indica que a ação ocorrerá após um determinado período de tempo | novamente; de novo}
  \seealsoref{更}{geng4}
  \seealsoref{更加}{geng4 jia1}
  \seealsoref{另外}{ling4wai4}
  \seealsoref{又}{you4}
\end{entry}

\begin{entry}{再不}{zai4bu4}{6,4}{⼌、⼀}
  \definition{adv.}{nunca mais}
\end{entry}

\begin{entry}{再次}{zai4 ci4}{6,6}{⼌、⽋}[HSK 5]
  \definition{adv.}{mais uma vez; uma segunda vez; outra vez}
\end{entry}

\begin{entry}{再读}{zai4du2}{6,10}{⼌、⾔}
  \definition{v.}{ler novamente | rever (uma lição, etc.)}
\end{entry}

\begin{entry}{再度}{zai4du4}{6,9}{⼌、⼴}
  \definition{adv.}{outra vez | mais uma vez}
\end{entry}

\begin{entry}{再发}{zai4fa1}{6,5}{⼌、⼜}
  \definition{v.}{reenviar}
\end{entry}

\begin{entry}{再见}{zai4jian4}{6,4}{⼌、⾒}[HSK 1]
  \definition{v.}{adeus; tchau; até logo; até mais; até mais tarde}
\end{entry}

\begin{entry}{再临}{zai4lin2}{6,9}{⼌、⼁}
  \definition{v.}{vir de novo}
\end{entry}

\begin{entry}{再三}{zai4san1}{6,3}{⼌、⼀}[HSK 4]
  \definition{adv.}{repetidamente; repetidas vezes; de novo e de novo}
\end{entry}

\begin{entry}{再审}{zai4shen3}{6,8}{⼌、⼧}
  \definition{s.}{novo julgamento | revisão}
  \definition{v.}{ouvir um caso novamente}
\end{entry}

\begin{entry}{再生}{zai4sheng1}{6,5}{⼌、⽣}
  \definition{s.}{reciclagem | regeneração}
  \definition{v.}{reciclar | renascer | regenerar}
\end{entry}

\begin{entry}{再说}{zai4shuo1}{6,9}{⼌、⾔}
  \definition{conj.}{além do mais | além disso | o que mais}
  \definition{v.}{adiar uma discussão para mais tarde | dizer novamente}
\end{entry}

\begin{entry}{再也}{zai4 ye3}{6,3}{⼌、⼄}[HSK 5]
  \definition{adv.}{não mais; nunca mais; uma determinada situação ou ação nunca mais ocorrerá}
\end{entry}

\begin{entry}{再育}{zai4yu4}{6,8}{⼌、⾁}
  \definition{v.}{aumentar | multiplicar | proliferar}
\end{entry}

\begin{entry}{再者}{zai4zhe3}{6,8}{⼌、⽼}
  \definition{conj.}{além do mais | além disso}
\end{entry}

\begin{entry}{在}{zai4}{6}{⼟}[HSK 1]
  \definition{adv.}{em processo de; em curso de}
  \definition{prep.}{em; no (um lugar ou momento); indica tempo, local, âmbito, etc.}
  \definition{v.}{existir; estar vivo | estar em; estar no; estar em (um lugar); indica a localização de pessoas ou coisas | permanecer; ficar | depender de; residir em; repousar com | ingressar ou pertencer a uma organização; ser membro de uma organização}
\end{entry}

\begin{entry}{在场}{zai4 chang3}{6,6}{⼟、⼟}[HSK 5]
  \definition{v.}{estar presente; estar no local; estar em cena; estar presente onde as coisas estão acontecendo}
\end{entry}

\begin{entry}{在此}{zai4ci3}{6,6}{⼟、⽌}
  \definition{adv.}{aqui}
\end{entry}

\begin{entry}{在地}{zai4di4}{6,6}{⼟、⼟}
  \definition{s.}{local}
\end{entry}

\begin{entry}{在行}{zai4hang2}{6,6}{⼟、⾏}
  \definition{v.}{ser adepto de algo | ser um especialista em um comércio ou profissão}
\end{entry}

\begin{entry}{在乎}{zai4hu5}{6,5}{⼟、⼃}[HSK 4]
  \definition{v.}{preocupar-se; preocupar-se com; levar a sério | ser responsável por; caber ao; ser da competência de}
\end{entry}

\begin{entry}{在家}{zai4 jia1}{6,10}{⼟、⼧}[HSK 1]
  \definition{v.}{estar em; estar em casa; estar no local de trabalho ou alojamento; sem sair de casa | continuar sendo um leigo; permanecer leigo; para monges, freiras, taoístas e outros que 出家, as pessoas comuns são consideradas leigas}
  \seealsoref{出家}{chu1 jia1}
\end{entry}

\begin{entry}{在教}{zai4jiao4}{6,11}{⼟、⽁}
  \definition{v.}{ser um crente (em uma religião)}
\end{entry}

\begin{entry}{在内}{zai4 nei4}{6,4}{⼟、⼌}[HSK 5]
  \definition{adj.}{incluido}
  \definition{adv.}{dentro; internamente; entre eles}
  \definition{v.}{ser incluído}
\end{entry}

\begin{entry}{在下}{zai4xia4}{6,3}{⼟、⼀}
  \definition{pron.}{eu mesmo (humildemente)}
\end{entry}

\begin{entry}{在线}{zai4xian4}{6,8}{⼟、⽷}
  \definition{s.}{\emph{online}}
\end{entry}

\begin{entry}{在意}{zai4yi4}{6,13}{⼟、⼼}
  \definition{v.+compl.}{preocupar-se | importar-se | levar a sério}
\end{entry}

\begin{entry}{在于}{zai4yu2}{6,3}{⼟、⼆}[HSK 4]
  \definition{v.}{ser responsável por; caber a;  ser da competência de;  apontar a essência das coisas, ou do que elas se tratam | depender de; ser determinado por;  ser devido a (um determinado atributo)/(de um assunto a ser determinado)}
\end{entry}

\begin{entry}{咱}{zan2}{9}{⼝}[HSK 2]
  \definition{pron.}{nós; nos (incluindo tanto o falante quanto a pessoa ou pessoas às quais se dirige) | eu; mim |}
  \seeref{咱}{za2}
  \seeref{咱}{zan5}
\end{entry}

\begin{entry}{咱俩}{zan2lia3}{9,9}{⼝、⼈}
  \definition{pron.}{nós dois}
\end{entry}

\begin{entry}{咱们}{zan2men5}{9,5}{⼝、⼈}[HSK 2]
  \definition{pron.}{dirige-se tanto ao falante (eu, nós) quanto ao ouvinte (você, vocês) | eu; mim; refere-se ao próprio orador, eu}
\end{entry}

\begin{entry}{暂时}{zan4shi2}{12,7}{⽇、⽇}[HSK 5]
  \definition{adj.}{transitório; temporário}
  \definition{adv.}{por enquanto; em pouco tempo}
\end{entry}

\begin{entry}{暂停}{zan4 ting2}{12,11}{⽇、⼈}[HSK 5]
  \definition{s.}{suspensão temporária; refere-se especificamente à suspensão temporária de certas competições desportivas de acordo com as regras}
  \definition{v.}{pausar; suspender; esgotar o tempo}
\end{entry}

\begin{entry}{赞}{zan4}{16}{⾙}
  \definition{v.}{patrocinar | apoiar | elogiar | (gíria na \emph{Internet}) para curtir (uma postagem \emph{on-line})}
\end{entry}

\begin{entry}{赞成}{zan4cheng2}{16,6}{⾙、⼽}[HSK 4]
  \definition{v.}{endossar; favorecer; aprovar; concordar com; concordar ou apoiar as ideias, os planos, as propostas ou o comportamento de outra pessoa}
\end{entry}

\begin{entry}{赞赏}{zan4 shang3}{16,12}{⾙、⾙}[HSK 4]
  \definition{v.}{admirar; apreciar; valorizar}
\end{entry}

\begin{entry}{赞扬}{zan4yang2}{16,6}{⾙、⼿}
  \definition{v.}{elogiar | aprovar | demonstrar aprovação}
\end{entry}

\begin{entry}{赞助}{zan4zhu4}{16,7}{⾙、⼒}[HSK 4]
  \definition{s.}{patrocinador}
  \definition{v.}{apoiar; patrocinar; concordar e ajudar (refere-se principalmente a oferecer dinheiro para ajudar)}
\end{entry}

\begin{entry}{咱}{zan5}{9}{⼝}
  \definition{adv.}{quando; agora; então; naquele momento; usado em 这咱, 那咱, 多咱, uma combinação das duas palavras 早晚}
  \seeref{咱}{za2}
  \seeref{咱}{zan2}
  \seealsoref{多咱}{duo1 zan5}
  \seealsoref{那咱}{na4 zan5}
  \seealsoref{早晚}{zao3 wan3}
  \seealsoref{这咱}{zhe4 zan5}
\end{entry}

\begin{entry}{脏}{zang1}{10}{⾁}[HSK 2]
  \definition{adj.}{sujo; imundo | imundo; metáfora para vulgaridade e obscenidade}
  \definition{v.}{tornar algo sujo ou impuro}
  \seeref{脏}{zang4}
\end{entry}

\begin{entry}{脏辫}{zang1bian4}{10,17}{⾁、⾟}
  \definition{s.}{\emph{dreadlocks}}
\end{entry}

\begin{entry}{脏病}{zang1bing4}{10,10}{⾁、⽧}
  \definition{s.}{doença venérea}
\end{entry}

\begin{entry}{脏煤}{zang1mei2}{10,13}{⾁、⽕}
  \definition{s.}{carvão sujo | sujeira (de uma mina de carvão)}
\end{entry}

\begin{entry}{脏土}{zang1tu3}{10,3}{⾁、⼟}
  \definition{s.}{solo sujo | lama | lixo}
\end{entry}

\begin{entry}{脏脏}{zang1zang1}{10,10}{⾁、⾁}
  \definition{adj.}{sujo}
\end{entry}

\begin{entry}{脏字}{zang1zi4}{10,6}{⾁、⼦}
  \definition{s.}{obscenidade}
\end{entry}

\begin{entry}{脏}{zang4}{10}{⾁}
  \definition[处]{s.}{vísceras; órgãos internos do corpo, geralmente o coração, o fígado, o baço, os pulmões e os rins; um termo geral para órgãos nas cavidades torácica e abdominal de humanos ou animais | (anatomia) órgão; a medicina tradicional chinesa chama o coração, o fígado, o baço, os pulmões e os rins de órgãos internos}
  \seeref{脏}{zang1}
\end{entry}

\begin{entry}{脏器}{zang4qi4}{10,16}{⾁、⼝}
  \definition{s.}{órgãos internos}
\end{entry}

\begin{entry}{葬}{zang4}{12}{⾋}
  \definition{v.}{enterrar (os mortos) | sepultar}
\end{entry}

\begin{entry}{遭到}{zao1dao4}{14,8}{⾡、⼑}
  \definition{v.}{sofrer | encontrar-se com (algo infeliz)}
\end{entry}

\begin{entry}{遭受}{zao1shou4}{14,8}{⾡、⼜}
  \definition{v.}{sofrer | suportar (perda, infornúnio)}
\end{entry}

\begin{entry}{遭遇}{zao1yu4}{14,12}{⾡、⾡}
  \definition{s.}{experiência (amarga)}
  \definition{v.}{encontrar-se com (algo infeliz)}
\end{entry}

\begin{entry}{糟}{zao1}{17}{⽶}[HSK 5]
  \definition{adj.}{pobre; apodrecido; deteriorado | estragado; em uma bagunça; em um estado miserável (terrível) | (situação ou circunstância) ruim; desfavorável}
  \definition{s.}{resíduos de destilação de bebidas alcoólicas; resíduos do processo de fermentação do vinho}
  \definition{v.}{marinar alimentos em vinho ou mosto}
\end{entry}

\begin{entry}{糟糕}{zao1gao1}{17,16}{⽶、⽶}[HSK 5]
  \definition{adj.}{(corpo, situação, etc.) muito ruim, péssimo}
  \definition{interj.}{que terrível; que má sorte; muito ruim}
\end{entry}

\begin{entry}{早}{zao3}{6}{⽇}[HSK 1]
  \definition{adj.}{precoce; antes do previsto ou planejado; antes do tempo; antes de um determinado momento |}
  \definition{adv.}{há muito tempo; desde cedo; por muito tempo; há muito tempo atrás}
  \definition{interj.}{bom dia; saudações, usadas para cumprimentar uns aos outros ao se encontrarem pela manhã}
  \definition[个]{s.}{manhã}
\end{entry}

\begin{entry}{早安}{zao3'an1}{6,6}{⽇、⼧}
  \definition{interj.}{Bom dia!}
\end{entry}

\begin{entry}{早餐}{zao3 can1}{6,16}{⽇、⾷}[HSK 2]
  \definition[份,桌,顿]{s.}{café da manhã; desejum}
\end{entry}

\begin{entry}{早车}{zao3che1}{6,4}{⽇、⾞}
  \definition{s.}{trem matutino | ônibus matutino}
\end{entry}

\begin{entry}{早晨}{zao3 chen2}{6,11}{⽇、⽇}[HSK 2]
  \definition[个,段,番]{s.}{manhã cedo; manhãzinha; o período do amanhecer às oito ou nove horas; às vezes, o período da meia-noite ao meio-dia}
\end{entry}

\begin{entry}{早饭}{zao3 fan4}{6,7}{⽇、⾷}[HSK 1]
  \definition[份,顿]{s.}{o café da manhã}
\end{entry}

\begin{entry}{早就}{zao3 jiu4}{6,12}{⽇、⼪}[HSK 2]
  \definition{adv.}{já; há muito tempo; há muito tempo atrás}
\end{entry}

\begin{entry}{早期}{zao3 qi1}{6,12}{⽇、⽉}[HSK 5]
  \definition{s.}{prófase; estágio inicial; fase inicial; a fase inicial de uma determinada época, processo ou vida de uma pessoa}
\end{entry}

\begin{entry}{早前}{zao3qian2}{6,9}{⽇、⼑}
  \definition{adv.}{previamente}
\end{entry}

\begin{entry}{早上}{zao3shang5}{6,3}{⽇、⼀}[HSK 1]
  \definition[个]{s.}{de manhã cedo; madrugada; o período antes e depois do nascer do sol; geralmente, desde o amanhecer até às 8h ou 9h da manhã; às vezes também se refere ao período entre o amanhecer e o meio-dia}
\end{entry}

\begin{entry}{早晚}{zao3 wan3}{6,11}{⽇、⽇}
  \definition{adv./s.}{manhã e noite | mais cedo ou mais tarde; cedo ou tarde | algum tempo no futuro; algum dia; em algum momento no futuro}
\end{entry}

\begin{entry}{早亡}{zao3wang2}{6,3}{⽇、⼇}
  \definition[个]{s.}{morte prematura}
  \definition{v.}{morrer prematuramente}
\end{entry}

\begin{entry}{早已}{zao3 yi3}{6,3}{⽇、⼰}[HSK 3]
  \definition{adv.}{há muito tempo; por muito tempo | (dialeto) no passado}
\end{entry}

\begin{entry}{早早儿}{zao3zao3r5}{6,6,2}{⽇、⽇、⼉}
  \definition{adv.}{o mais cedo possível | o mais breve possível}
\end{entry}

\begin{entry}{早知}{zao3zhi1}{6,8}{⽇、⽮}
  \definition{v.}{prever | se alguém soubesse antes, \dots}
\end{entry}

\begin{entry}{灶台}{zao4tai2}{7,5}{⽕、⼝}
  \definition{s.}{fogão}
\end{entry}

\begin{entry}{造}{zao4}{10}{⾡}[HSK 3]
  \definition*{s.}{sobrenome Zao}
  \definition{clas.}{para colheitas ou número de colheitas de safras}
  \definition{s.}{uma das duas partes em um acordo legal ou um processo judicial | (dialeto) colheita; safra | realizações; conquistas |}
  \definition{v.}{fazer; construir; criar; produzir | forjar; inventar | correr solto; bagunçar as coisas | expor sem restrições |  treinar; educar | fabricar | alcançar; atingir}
\end{entry}

\begin{entry}{造成}{zao4cheng2}{10,6}{⾡、⼽}[HSK 3]
  \definition{v.}{criar; dar origem a; provocar; causar (geralmente se refere a resultados negativos)}
\end{entry}

\begin{entry}{造型}{zao4xing2}{10,9}{⾡、⼟}[HSK 4]
  \definition{s.}{modelo; formato; forma; moldagem}
  \definition{v.}{modelar; moldar}
\end{entry}

\begin{entry}{艁}{zao4}{13}{⾈}
  \variantof{造}
\end{entry}

\begin{entry}{责怪}{ze2guai4}{8,8}{⾙、⼼}
  \definition{v.}{repreender | censurar}
\end{entry}

\begin{entry}{责任}{ze2ren4}{8,6}{⾙、⼈}[HSK 3]
  \definition[个,种,份]{s.}{dever; responsabilidade; de acordo com a profissão, cargo, identidade, etc., as coisas que você deve fazer ou as tarefas que deve assumir | culpa; responsabilidade por uma falha ou erro; não ter feito o que era sua obrigação e, portanto, ser responsável pela falha}
\end{entry}

\begin{entry}{怎}{zen3}{9}{⼼}
  \definition{adv.}{como}
\end{entry}

\begin{entry}{怎么}{zen3me5}{9,3}{⼼、⼃}[HSK 1]
  \definition{pron.}{como?; o quê?; perguntas sobre natureza, situação, método, motivo, etc. | de qualquer maneira; não importa como; de uma certa maneira; referência geral à natureza, condição ou modo | que? (usado sozinho no início de uma frase para expressar surpresa) | usado após 不 e 没, indica um grau baixo e é uma forma mais educada de se expressar | usado em perguntas retóricas}
  \seealsoref{不}{bu4}
  \seealsoref{没}{mei2}
\end{entry}

\begin{entry}{怎么办}{zen3 me5 ban4}{9,3,4}{⼼、⼃、⼒}[HSK 2]
  \definition{adv.}{o que fazer?; o que deve ser feito?}
\end{entry}

\begin{entry}{怎么得了}{zen3me5de2liao3}{9,3,11,2}{⼼、⼃、⼻、⼅}
  \definition{expr.}{Como isso pode ser? | Que bagunça horrível! | O que deve ser feito?}
\end{entry}

\begin{entry}{怎么搞的}{zen3me5gao3de5}{9,3,13,8}{⼼、⼃、⼿、⽩}
  \definition{expr.}{Como isso aconteceu? | O que deu errado? | E aí? | O que está errado?}
\end{entry}

\begin{entry}{怎么回事}{zen3me5hui2shi4}{9,3,6,8}{⼼、⼃、⼞、⼅}
  \definition{expr.}{O que aconteceu? | O que se passou?}
\end{entry}

\begin{entry}{怎么了}{zen3me5le5}{9,3,2}{⼼、⼃、⼅}
  \definition{expr.}{O que aconteceu? | O que está acontecendo? | E aí?}
\end{entry}

\begin{entry}{怎么样}{zen3me5yang4}{9,3,10}{⼼、⼃、⽊}[HSK 2]
  \definition{adv.}{como?; o que?; como é?; como estão as coisas?; o que você acha?; pergunte sobre o método, natureza, situação, opinião, etc. | substitui uma ação ou situação não dita (usado apenas na forma negativa, mais eufemístico do que uma declaração direta); indaga sobre a natureza, condição, método, razão, etc.}
\end{entry}

\begin{entry}{怎样}{zen3 yang4}{9,10}{⼼、⽊}[HSK 2]
  \definition{pron.}{como?; o que?; indagar sobre a natureza, condição ou método, etc. | como?; indica uma referência virtual | de uma certa maneira; de qualquer maneira; não importa como; indica qualquer | como?; usado como predicado, objeto ou complemento para indagar sobre uma situação}
\end{entry}

\begin{entry}{曾}{zeng1}{12}{⽈}
  \definition*{s.}{sobrenome Zeng}
  \definition{s.}{relacionamento entre bisnetos e bisavós; (parentesco) duas gerações de diferença}
  \seeref{曾}{ceng2}
\end{entry}

\begin{entry}{增}{zeng1}{15}{⼟}[HSK 5]
  \definition*{s.}{sobrenome Zeng}
  \definition{v.}{aumentar; ganhar; adicionar}
\end{entry}

\begin{entry}{增产}{zeng1 chan3}{15,6}{⼟、⼇}[HSK 5]
  \definition{v.+compl.}{aumentar a produção}
\end{entry}

\begin{entry}{增大}{zeng1 da4}{15,3}{⼟、⼤}[HSK 5]
  \definition{v.}{ampliar; expandir; estender | amplificar}
\end{entry}

\begin{entry}{增多}{zeng1 duo1}{15,6}{⼟、⼣}[HSK 5]
  \definition{v.}{aumentar; crescer em número ou quantidade}
\end{entry}

\begin{entry}{增加}{zeng1jia1}{15,5}{⼟、⼒}[HSK 3]
  \definition{v.}{adicionar; aumentar; incrementar; adicionar mais ao que já existe}
\end{entry}

\begin{entry}{增强}{zeng1 qiang2}{15,12}{⼟、⼸}[HSK 5]
  \definition{v.}{impulsionar; aprimorar; aumentar; fortalecer; tornar mais forte ou mais poderoso}
\end{entry}

\begin{entry}{增速}{zeng1su4}{15,10}{⼟、⾡}
  \definition{s.}{(economia) taxa de crescimento}
  \definition{v.}{acelerar;}
\end{entry}

\begin{entry}{增长}{zeng1 zhang3}{15,4}{⼟、⾧}[HSK 3]
  \definition{v.}{subir; crescer; aumentar; melhorar a partir da base existente}
\end{entry}

\begin{entry}{赠}{zeng4}{16}{⾙}[HSK 5]
  \definition{v.}{dar um presente; presentear com um brinde}
\end{entry}

\begin{entry}{赠送}{zeng4song4}{16,9}{⾙、⾡}[HSK 5]
  \definition{v.}{dar; dar de presente; dar algo de graça a alguém}
\end{entry}

\begin{entry}{查}{zha1}{9}{⽊}
  \definition*{s.}{sobrenome Zha}
  \definition{s.}{espinheiro-chinês}
  \seeref{查}{cha2}
\end{entry}

\begin{entry}{闸门}{zha2men2}{8,3}{⾨、⾨}
  \definition{s.}{eclusa | comporta}
\end{entry}

\begin{entry}{摘}{zhai1}{14}{⼿}[HSK 5]
  \definition{v.}{pegar; arrancar; tirar; colher (flores, frutos, folhas de plantas); retirar (coisas que estão sendo usadas ou penduradas) | selecionar; fazer extrações de | pedir dinheiro emprestado em caso de necessidade urgente | vencer; ganhar; alcançar; obter}
\end{entry}

\begin{entry}{寨}{zhai4}{14}{⼧}
  \definition{s.}{fortaleza | paliçada | acampamento | vila (paliçada)}
\end{entry}

\begin{entry}{占}{zhan1}{5}{⼘}
  \definition*{s.}{sobrenome Zhan}
  \definition{v.}{praticar adivinhação; antigamente, as pessoas usavam cascos de tartaruga e mil-folhas para prever boa ou má sorte; mais tarde, a palavra passou a se referir à previsão de boa ou má sorte por vários meios}
  \seeref{占}{zhan4}
\end{entry}

\begin{entry}{斩获}{zhan3huo4}{8,10}{⽄、⾋}
  \definition{v.}{matar ou capturar (em batalha) | (figurativo) (esportes) marcar (um gol), ganhar (uma medalha) | (figurativo) colher recompensas, obter ganhos}
\end{entry}

\begin{entry}{展开}{zhan3kai1}{10,4}{⼫、⼶}[HSK 3]
  \definition{s.}{desenvolvimento; expansão; explosão; evolução}
  \definition{v.}{espalhar; desdobrar; abrir | lançar; desdobrar; desenvolver; realizar em grande escala | espalhar; desenrolar; amplificar; desenvolver; expandir; explodir; evoluir; alongar}
\end{entry}

\begin{entry}{展览}{zhan3lan3}{10,9}{⼫、⾒}[HSK 5]
  \definition[个,次,场]{s.}{exposição; exibição; atividades expostas; itens expostos}
  \definition{v.}{mostrar; exibir; expor; expor algo para que as pessoas vejam}
\end{entry}

\begin{entry}{展示}{zhan3shi4}{10,5}{⼫、⽰}[HSK 5]
  \definition{v.}{mostrar; revelar; pôr a nu; abrir diante de alguém; expor claramente; manifestar de forma evidente}
\end{entry}

\begin{entry}{展现}{zhan3xian4}{10,8}{⼫、⾒}[HSK 5]
  \definition{v.}{mostrar; surgir; manifestar}
\end{entry}

\begin{entry}{盏}{zhan3}{10}{⽫}
  \definition{clas.}{para lâmpadas}
  \definition{s.}{copo pequeno}
\end{entry}

\begin{entry}{占}{zhan4}{5}{⼘}[HSK 2]
  \definition{v.}{tomar; apreender; ocupar; obter e possuir (terra, lugar, etc.) pela força ou outros meios impróprios | manter; inventar; constituir; explicar; estar em (uma certa posição); pertencer a (uma certa situação) | usar; ocupar; tomar; possuir}
  \seeref{占}{zhan1}
\end{entry}

\begin{entry}{占领}{zhan4ling3}{5,11}{⼘、⾴}[HSK 5]
  \definition{v.}{manter; tomar; ocupar; capturar; conquistar (posições ou territórios) com forças armadas | ocupar; capturar; possuir}
\end{entry}

\begin{entry}{占有}{zhan4 you3}{5,6}{⼘、⽉}[HSK 5]
  \definition{v.}{possuir; ter; ocupar e possuir | manter; ocupar; estar em (uma determinada posição) | possuir; deter; ter; dominar}
\end{entry}

\begin{entry}{战}{zhan4}{9}{⼽}
  \definition{s.}{luta | guerra | batalha}
  \definition{v.}{lutar}
\end{entry}

\begin{entry}{战斗}{zhan4dou4}{9,4}{⼽、⽃}[HSK 4]
  \definition[场,次]{s.}{luta; batalha; combate; ação; conflito armado entre as partes oponentes}
  \definition{v.}{lutar | trabalhar sob pressão}
\end{entry}

\begin{entry}{战胜}{zhan4 sheng4}{9,9}{⼽、⾁}[HSK 4]
  \definition{v.}{derrotar; vencer; superar; triunfar sobre; metáfora para superar dificuldades e alcançar o sucesso}
\end{entry}

\begin{entry}{战士}{zhan4shi4}{9,3}{⼽、⼠}[HSK 4]
  \definition[个]{s.}{soldado; membros mais jovens do exército | campeão; guerreiro; lutador; geralmente, uma pessoa que se engaja em alguma causa justa ou participa de alguma luta justa}
\end{entry}

\begin{entry}{战争}{zhan4zheng1}{9,6}{⼽、⼑}[HSK 4]
  \definition[场,次]{s.}{guerra; conflito; luta armada entre povos, entre nações, entre classes ou entre grupos políticos}
\end{entry}

\begin{entry}{站}{zhan4}{10}{⽴}[HSK 1,2]
  \definition*{s.}{sobrenome Zhan}
  \definition{s.}{parada; estação; ponto de parada | central; estação; instituição criada para um determinado tipo de atividade | filial de uma empresa ou organização; local de trabalho criado para realizar uma determinada tarefa | \emph{website}; na rede de computadores, refere-se a um \emph{site}}
  \definition{v.}{ficar em pé; estar em pé | parar; interromper; fazer uma pausa}
\end{entry}

\begin{entry}{站点}{zhan4dian3}{10,9}{⽴、⽕}
  \definition{s.}{\emph{website}}
\end{entry}

\begin{entry}{站台}{zhan4tai2}{10,5}{⽴、⼝}
  \definition{s.}{plataforma (em uma estação ferroviária)}
\end{entry}

\begin{entry}{站长}{zhan4zhang3}{10,4}{⽴、⾧}
  \definition{s.}{pessoa responsável pela estação de trem | chefe da estação | \emph{webmaster} | gerente de centro de voluntariado}
\end{entry}

\begin{entry}{站住}{zhan4 zhu4}{10,7}{⽴、⼈}[HSK 2]
  \definition{v.}{parar; deter; parar enquanto se move | ficar firme nos pés; manter os pés; permanecer firme | manter-se firme; consolidar a posição de alguém; estabelecer-se em uma determinada unidade ou lugar | sustentar a opinião}
\end{entry}

\begin{entry}{站姿}{zhan4zi1}{10,9}{⽴、⼥}
  \definition{s.}{postura}
\end{entry}

\begin{entry}{张}{zhang1}{7}{⼸}[HSK 3]
  \definition*{s.}{sobrenome Zhang}
  \definition*{s.}{Zhang, uma das mansões lunares; uma das 28 constelações}
  \definition{adj.}{indulgente; desenfreado; devasso; libertino}
  \definition{clas.}{usado para papel, couro, etc. | usado para cama, mesa, etc. | usado para o rosto, boca, etc. | usado para arco}
  \definition{v.}{abrir; espalhar; esticar | expor; exibir | (de uma loja) iniciar atividades comerciais; abrir | olhar; contemplar | expandir; estender; ampliar; exagerar | fixar (uma corda de arco); encordoar (um instrumento musical); puxar a corda do arco}
\end{entry}

\begin{entry}{张狂}{zhang1kuang2}{7,7}{⼸、⽝}
  \definition{adj.}{impetuoso | frenético | insolente}
\end{entry}

\begin{entry}{张三}{zhang1san1}{7,3}{⼸、⼀}
  \definition*{s.}{Zhang San | Zé Ninguém | nome para uma pessoa não especificada, 1 de 3}
  \seealsoref{李四}{li3si4}
  \seealsoref{王五}{wang2wu3}
\end{entry}

\begin{entry}{章}{zhang1}{11}{⾳}
  \definition*{s.}{sobrenome Zhang}
  \definition{s.}{capítulo | seção | cláusula |  movimento (de sinfonia) | selo | crachá | regulamento}
\end{entry}

\begin{entry}{章鱼}{zhang1yu2}{11,8}{⾳、⿂}
  \definition{s.}{polvo | octópode}
\end{entry}

\begin{entry}{长}{zhang3}{4}{⾧}[HSK 2]
  \definition{adj.}{mais velho; ancião; sênior; mais antigo; o primeiro da fila}
  \definition{s.}{ancião; pessoas mais velhas ou de posição social mais elevada | chefe; líder; dirigente; responsável}
  \definition{v.}{crescer; desenvolver-se | surgir; começar a crescer; formar-se; nascer em pessoas, animais, plantas ou objetos (algo) | adquirir; melhorar; aumentar; aumentar conhecimento, capacidade, etc.; tornar-se cada vez mais ou cada vez melhor}
  \seeref{长}{chang2}
\end{entry}

\begin{entry}{长大}{zhang3 da4}{4,3}{⾧、⼤}[HSK 2]
  \definition{v.}{crescer; ser criado; um estado de maturidade física ou mental completa}
\end{entry}

\begin{entry}{涨}{zhang3}{10}{⽔}[HSK 5]
  \definition{v.}{subir; inchar; aumentar; elevar}
\end{entry}

\begin{entry}{涨价}{zhang3 jia4}{10,6}{⽔、⼈}[HSK 5]
  \definition{s.}{aumento de preços}
  \definition{v.+compl.}{(de preços) subir; aumentar o preço}
\end{entry}

\begin{entry}{掌}{zhang3}{12}{⼿}
  \definition{s.}{palma da mão | sola do pé | pata | ferradura}
  \definition{v.}{dar um tapa | segurar na mão | empunhar}
\end{entry}

\begin{entry}{掌握}{zhang3wo4}{12,12}{⼿、⼿}[HSK 5]
  \definition{v.}{compreender; dominar; conhecer bem; compreender as coisas; ser capaz de dominar ou utilizar plenamente | segurar; controlar; ter em mãos; tomar nas mãos}
\end{entry}

\begin{entry}{丈夫}{zhang4fu5}{3,4}{⼀、⼤}[HSK 4]
  \definition[个,位,名]{s.}{marido; esposo}
\end{entry}

\begin{entry}{招}{zhao1}{8}{⼿}
  \definition{adj.}{contagioso}
  \definition{s.}{um movimento (xadrez) | uma manobra | dispositivo | truque}
  \definition{v.}{recrutar | provocar | acenar | incorrer | infectar | confessar}
\end{entry}

\begin{entry}{招呼}{zhao1 hu5}{8,8}{⼿、⼝}[HSK 4]
  \definition{v.}{chamar; chamar a atenção com palavras ou gestos | cumprimentar; saudar; cumprimentar ou despedir-se das pessoas com palavras ou gestos | pedir a alguém para fazer algo; fazer solicitações, pedir ajuda ou fazer coisas | receber e dar boas-vindas aos convidados}
\end{entry}

\begin{entry}{招生}{zhao1 sheng1}{8,5}{⼿、⽣}[HSK 5]
  \definition{v.+compl.}{conseguir alunos; matricular novos alunos; recrutar novos alunos}
\end{entry}

\begin{entry}{招手}{zhao1 shou3}{8,4}{⼿、⼿}[HSK 5]
  \definition{v.+compl.}{acenar; chamar a atenção; levantar a mão e acenar com a palma, para indicar que a outra pessoa se aproxime ou para cumprimentá-la}
\end{entry}

\begin{entry}{招数}{zhao1shu4}{8,13}{⼿、⽁}
  \definition{s.}{estratégia | movimento (no xadrez, no palco, nas artes marciais) | esquema | truque}
\end{entry}

\begin{entry}{着}{zhao1}{11}{⽬}
  \definition{interj.}{tudo bem; tudo certo; \emph{O.K.}}
  \definition{s.}{uma jogada no xadrez | truque; meio; artifício; manobra; estratégia}
  \definition{v.}{colocar dentro; guardar}
  \seeref{着}{zhao2}
  \seeref{着}{zhe5}
  \seeref{着}{zhuo2}
\end{entry}

\begin{entry}{着数}{zhao1shu4}{11,13}{⽬、⽁}
  \definition{s.}{estratégia | movimento (no xadrez, no palco, nas artes marciais) | esquema | truque}
\end{entry}

\begin{entry}{朝}{zhao1}{12}{⽉}
  \definition{s.}{manhã cedo; manhã | dia}
  \seeref{朝}{chao2}
\end{entry}

\begin{entry}{着}{zhao2}{11}{⽬}
  \definition{v.}{tocar (contato físico) | sentir; ser afetado por | queimar; acender | adormecer; cair no sono | acertar em cheio; ter sucesso em; usado após o verbo, indica que o objetivo foi alcançado ou que houve um resultado}
  \seeref{着}{zhao1}
  \seeref{着}{zhe5}
  \seeref{着}{zhuo2}
\end{entry}

\begin{entry}{着地}{zhao2di4}{11,6}{⽬、⼟}
  \definition{v.}{pousar | tocar o chão}
\end{entry}

\begin{entry}{着花}{zhao2hua1}{11,7}{⽬、⾋}
  \definition{v.}{florescer}
  \seeref{着花}{zhuo2hua1}
\end{entry}

\begin{entry}{着火}{zhao2huo3}{11,4}{⽬、⽕}[HSK 4]
  \definition{v.}{pegar fogo; estar em chamas}
\end{entry}

\begin{entry}{着急}{zhao2ji2}{11,9}{⽬、⼼}[HSK 4]
  \definition{adj.}{ansioso; preocupado |}
  \definition{s.}{preocupação; ansiedade}
  \definition{v.+compl.}{preocupar-se | sentir-se ansioso | sentir uma sensação de urgência}
\end{entry}

\begin{entry}{着凉}{zhao2liang2}{11,10}{⽬、⼎}
  \definition{v.}{pegar um resfriado}
\end{entry}

\begin{entry}{找}{zhao3}{7}{⼿}[HSK 1]
  \definition{v.}{procurar; tentar encontrar; buscar | querer ver; visitar; abordar; solicitar | dar troco | descobrir; esforçar-se para ver ou obter a pessoa ou coisa desejada | examinar; investigar; completar as partes que faltam | causar intencionalmente (um resultado indesejável, negativo)}
\end{entry}

\begin{entry}{找遍}{zhao3bian4}{7,12}{⼿、⾡}
  \definition{v.}{pentear | pesquisar em todos os lugares}
\end{entry}

\begin{entry}{找出}{zhao3 chu1}{7,5}{⼿、⼐}[HSK 2]
  \definition{v.}{encontrar | procurar}
\end{entry}

\begin{entry}{找到}{zhao3 dao4}{7,8}{⼿、⼑}[HSK 1]
  \definition{v.}{encontrar; procurar; achar; encontar através de pesquisa, exploração, etc.;  ver ou encontrar coisas ou padrões que os antepassados não viram}
\end{entry}

\begin{entry}{找回}{zhao3hui2}{7,6}{⼿、⼞}
  \definition{v.}{recuperar algo}
\end{entry}

\begin{entry}{找见}{zhao3jian4}{7,4}{⼿、⾒}
  \definition{v.}{encontrar (algo que está procurando)}
\end{entry}

\begin{entry}{找零}{zhao3ling2}{7,13}{⼿、⾬}
  \definition{v.}{trocar dinheiro | dar troco}
\end{entry}

\begin{entry}{找钱}{zhao3qian2}{7,10}{⼿、⾦}
  \definition{v.}{dar troco}
\end{entry}

\begin{entry}{找事}{zhao3shi4}{7,8}{⼿、⼅}
  \definition{v.}{procurar emprego | começar uma briga}
\end{entry}

\begin{entry}{找寻}{zhao3xun2}{7,6}{⼿、⼨}
  \definition{v.}{encontrar falhas | procurar | buscar}
\end{entry}

\begin{entry}{找着}{zhao3zhao2}{7,11}{⼿、⽬}
  \definition{v.}{encontrar}
\end{entry}

\begin{entry}{找辙}{zhao3zhe2}{7,16}{⼿、⾞}
  \definition{v.}{procurar um pretexto}
\end{entry}

\begin{entry}{召开}{zhao4kai1}{5,4}{⼝、⼶}[HSK 4]
  \definition{v.}{convocar; chamar pessoas para uma reunião; realizar (uma reunião)}
\end{entry}

\begin{entry}{兆}{zhao4}{6}{⼉}
  \definition{num.}{trilhão}
\end{entry}

\begin{entry}{照}{zhao4}{13}{⽕}[HSK 3]
  \definition{adv.}{de acordo com; significa agir de acordo com o original ou com determinados padrões}
  \definition{prep.}{em direção a; na direção de | de acordo com; em conformidade com}
  \definition{s.}{imagem; fotografia | permissão; licença; autorização | brilho; iluminação}
  \definition{v.}{brilhar; iluminar | refletir; espelhar; olhar para sua própria imagem em um espelho, etc. | filmar; fotografar; tirar uma foto (fotografia) | cuidar de; tomar conta de; zelar por | notificar; informar | contrastar; comparar; verificar | entender; compreender}
\end{entry}

\begin{entry}{照顾}{zhao4gu4}{13,10}{⽕、⾴}[HSK 2]
  \definition{v.}{cuidar; cuidar de; atender | oferecer tratamento preferencial; prestar atenção especial e dar tratamento preferencial | (de um cliente) patrocinar; comprar em; cuidar de clientes que vêm comprar coisas ou solicitar serviços em lojas ou indústrias de serviços | dar consideração a; mostrar consideração por; levar em conta; fazer concessões a}
\end{entry}

\begin{entry}{照亮}{zhao4liang4}{13,9}{⽕、⼇}
  \definition{s.}{iluminação}
  \definition{v.}{iluminar}
\end{entry}

\begin{entry}{照片}{zhao4pian4}{13,4}{⽕、⽚}[HSK 2]
  \definition[张,套,幅]{s.}{fotografia, foto, imagem}
\end{entry}

\begin{entry}{照片底版}{zhao4pian4di3ban3}{13,4,8,8}{⽕、⽚、⼴、⽚}
  \definition{s.}{placa fotográfica}
\end{entry}

\begin{entry}{照片子}{zhao4pian4zi5}{13,4,3}{⽕、⽚、⼦}
  \definition{v.}{tirar um raio X}
\end{entry}

\begin{entry}{照骗}{zhao4pian4}{13,12}{⽕、⾺}
  \definition{s.}{imagem alterada digitalmente; ``photoshopada''}
\end{entry}

\begin{entry}{照相}{zhao4 xiang4}{13,9}{⽕、⽬}[HSK 2]
  \definition{v.+compl.}{fotografar; tirar fotos; tirar uma foto; tirar uma fotografia}
\end{entry}

\begin{entry}{照相机}{zhao4xiang4ji1}{13,9,6}{⽕、⽬、⽊}
  \definition[个,架,部,台,只]{s.}{câmera/máquina fotográfica}
\end{entry}

\begin{entry}{照像}{zhao4 xiang4}{13,13}{⽕、⼈}
  \variantof{照相}
\end{entry}

\begin{entry}{照像机}{zhao4xiang4ji1}{13,13,6}{⽕、⼈、⽊}
  \variantof{照相机}
\end{entry}

\begin{entry}{照准}{zhao4zhun3}{13,10}{⽕、⼎}
  \definition{s.}{solicitação concedida (uso formal em documento antigo)}
  \definition{v.}{mirar (arma)}
\end{entry}

\begin{entry}{折}{zhe1}{7}{⼿}
  \definition{v.}{rolar; virar | despejar algo de um recipiente em outro; ficar despejando algo entre dois recipientes}
  \seeref{折}{she2}
  \seeref{折}{zhe2}
\end{entry}

\begin{entry}{折}{zhe2}{7}{⼿}[HSK 4]
  \definition*{s.}{sobrenome Zhe}
  \definition{clas.}{uma passagem em um roteiro de ópera miscelânea de Yuan, aproximadamente equivalente a uma cena ou ato em uma ópera moderna}
  \definition[张,个,些]{s.}{fratura; quebra | abatimento; desconto | traços dos caracteres chineses que têm o formato de "𠃍" e "乚", etc. | pasta; livreto; \emph{folder}}
  \definition{v.}{estalar; quebrar; fazer quebrar | perder; sofrer a perda de | voltar para trás; mudar de direção; retornar |ser convencido; estar cheio de admiração | equivaler a; converter em | dobrar}
  \seeref{折}{she2}
  \seeref{折}{zhe1}
\end{entry}

\begin{entry}{折转}{zhe2zhuan3}{7,8}{⼿、⾞}
  \definition{s.}{reflexo (ângulo)}
  \definition{v.}{voltar atrás}
\end{entry}

\begin{entry}{哲理}{zhe2li3}{10,11}{⼝、⽟}
  \definition{s.}{filosofia | teoria filosófica}
\end{entry}

\begin{entry}{者}{zhe3}{8}{⽼}[HSK 3]
  \definition{part.}{significa 是 e é usado após palavras, frases e orações para indicar uma pausa}
  \definition{pron.}{usado para se referir à pessoa, coisa ou assunto que realiza uma ação ou possui um determinado atributo | pessoas; caras (Usado para se referir a alguém envolvido em uma determinada profissão, que acredita em uma determinada ideologia ou que tem uma forte tendência para algo) | usado após certos números ou palavras direcionais para se referir a coisas mencionadas anteriormente | significado semelhante a 这 (mais comum na linguagem coloquial antiga)}
  \definition{s.}{sobrenome Zhe}
  \seealsoref{是}{shi4}
  \seealsoref{这}{zhe4}
\end{entry}

\begin{entry}{这}{zhe4}{7}{⾡}[HSK 1]
  \definition{pron.}{este, isto; substitui pessoas ou coisas que estão mais próximas | agora; em vez de 这时候, tem o efeito de reforçar a ênfase}
  \seeref{这}{zhei4}
  \seealsoref{这时候}{zhe4 shi2 hou5}
\end{entry}

\begin{entry}{这边}{zhe4 bian1}{7,5}{⾡、⾡}[HSK 1]
  \definition{pron.}{aqui; deste lado; refere-se a um lugar próximo}
\end{entry}

\begin{entry}{这个}{zhe4ge5}{7,3}{⾡、⼈}
  \definition{pron.}{isto; este | isso; em vez das coisas mencionadas anteriormente | assim; tal; usado antes de verbos e adjetivos, indica um grau muito profundo, com um sentido exagerado | usado junto com 那个 para indicar pessoas ou objetos indefinidos}
  \seealsoref{那个}{na4ge5}
\end{entry}

\begin{entry}{这会儿}{zhe4 hui4r5}{7,6,2}{⾡、⼈、⼉}
  \definition{adv./pron./s.}{agora; no momento; no presente}
\end{entry}

\begin{entry}{这里}{zhe4 li3}{7,7}{⾡、⾥}[HSK 1]
  \definition{pron.}{aqui; pronomes demonstrativo, indicando locais próximos}
\end{entry}

\begin{entry}{这么}{zhe4 me5}{7,3}{⾡、⼃}[HSK 2]
  \definition{pron.}{tal (usado para mostrar o grau) | então (usado para mostrar exagero e exclamação) | desta forma; assim; formas de expressar ações | tal; indica quantidade}
\end{entry}

\begin{entry}{这麽}{zhe4 me5}{7,14}{⾡、⿇}
  \variantof{这么}
\end{entry}

\begin{entry}{这儿}{zhe4r5}{7,2}{⾡、⼉}[HSK 1]
  \definition{pron.}{aqui | agora; neste momento (utilizado apenas após 打, 从, 由)}
  \seealsoref{从}{cong2}
  \seealsoref{打}{da3}
  \seealsoref{由}{you2}
\end{entry}

\begin{entry}{这时}{zhe4 shi2}{7,7}{⾡、⽇}[HSK 2]
  \definition{adv.}{neste momento}
\end{entry}

\begin{entry}{这时候}{zhe4 shi2 hou5}{7,7,10}{⾡、⽇、⼈}[HSK 2]
  \definition{adv.}{neste momento}
\end{entry}

\begin{entry}{这些}{zhe4 xie1}{7,8}{⾡、⼆}[HSK 1]
  \definition{pron.}{estes; pronome demonstrativo, que indicam duas ou mais pessoas ou coisas que estão próximas}
\end{entry}

\begin{entry}{这样}{zhe4 yang4}{7,10}{⾡、⽊}[HSK 2]
  \definition{pron.}{assim; tal; assim; deste jeito; pronome demonstrativo, que indica a natureza, estado, maneira, grau, etc.}
\end{entry}

\begin{entry}{这咱}{zhe4 zan5}{7,9}{⾡、⼝}
  \definition{s.}{agora; no momento; no presente | neste momento}
\end{entry}

\begin{entry}{浙江}{zhe4jiang1}{10,6}{⽔、⽔}
  \definition*{s.}{Zhejiang}
\end{entry}

\begin{entry}{着}{zhe5}{11}{⽬}[HSK 1,4]
  \definition{part.}{adicionar a um verbo ou adjetivo para indicar uma ação ou estado contínuo | em frases que começam com uma palavra que indica um lugar, acrescente ao verbo para indicar um estado resultante | em frases imperativas, usado após verbos ou adjetivos para dar ênfase | adicionado após certos verbos, transforma-se em preposição}
  \definition{s.}{um movimento no xadrez |movimento; estratégia; estratagema}
  \seeref{着}{zhao1}
  \seeref{着}{zhao2}
  \seeref{着}{zhuo2}
\end{entry}

\begin{entry}{这}{zhei4}{7}{⾡}
  \definition{pron.}{(coloquial) este}
  \seeref{这}{zhe4}
\end{entry}

\begin{entry}{针}{zhen1}{7}{⾦}[HSK 4]
  \definition*{s.}{sobrenome Zhen}
  \definition[根]{s.}{agulha; ferramentas para costura de roupas | objetos semelhantes a agulhas; algo longo e fino como uma agulha | injeção | ponto de costura | pontos de acupuntura na medicina chinesa}
\end{entry}

\begin{entry}{针对}{zhen1dui4}{7,5}{⾦、⼨}[HSK 4]
  \definition{prep.}{em conexão com; de acordo com; à luz de; introdução de objetos de comportamento com uma finalidade clara}
  \definition{v.}{contrariar; apontar para; ter como objetivo; ser direcionado contra; fazer algo especificamente sobre um problema ou uma pessoa}
\end{entry}

\begin{entry}{珍贵}{zhen1gui4}{9,9}{⽟、⾙}[HSK 5]
  \definition{adj.}{raro; valioso; precioso; de grande valor; profundo significado}
\end{entry}

\begin{entry}{珍惜}{zhen1xi1}{9,11}{⽟、⼼}[HSK 5]
  \definition{v.}{valorizar; estimar; valorizar e evitar o desperdício}
\end{entry}

\begin{entry}{珍珠}{zhen1zhu1}{9,10}{⽟、⽟}[HSK 5]
  \definition[颗,串]{s.}{pérola; grânulos redondos produzidos nas conchas de certos animais aquáticos, de cor branca, rosa, etc., bonitos e brilhantes, frequentemente usados como adornos}
\end{entry}

\begin{entry}{眞}{zhen1}{10}{⽬}
  \variantof{真}
\end{entry}

\begin{entry}{真}{zhen1}{10}{⼗}[HSK 1]
  \definition*{s.}{sobrenome Zhen}
  \definition{adj.}{verdadeiro; real; genuíno (oposto de 假, 伪) | claro; inequívoco | genuíno; conforme os fatos objetivos (em oposição a 假 e 伪) | sincero}
  \definition{adv.}{realmente; verdadeiramente; de fato}
  \definition{s.}{escrita regular | retrato; imagem; cópia exata de algo | instintos naturais (ou caráter, disposição); natureza; qualidade inerente; origem | estado original; refere-se à forma original das coisas}
  \seealsoref{假}{jia4}
  \seealsoref{伪}{wei3}
\end{entry}

\begin{entry}{真诚}{zhen1 cheng2}{10,8}{⼗、⾔}[HSK 5]
  \definition{adj.}{dadeiro; honesto; sério; sincero; genuíno; descreve uma pessoa que fala e age com sinceridade, de coração, fazendo com que os outros acreditem nela}
\end{entry}

\begin{entry}{真的}{zhen1 de5}{10,8}{⼗、⽩}[HSK 1]
  \definition{adv.}{realmente; salientar que a situação existe realmente | verdadeiramente; realmente; existente na realidade; consistente com os fatos objetivos}
\end{entry}

\begin{entry}{真理}{zhen1li3}{10,11}{⼗、⽟}[HSK 5]
  \definition[个]{s.}{verdade; o reflexo correto das coisas objetivas e suas leis no cérebro humano}
\end{entry}

\begin{entry}{真牛}{zhen1niu2}{10,4}{⼗、⽜}
  \definition{adj.}{(gíria) muito legal, incrível}
\end{entry}

\begin{entry}{真切}{zhen1qie4}{10,4}{⼗、⼑}
  \definition{adj.}{claro | distinto | honesto | sincero | vívido}
\end{entry}

\begin{entry}{真声}{zhen1sheng1}{10,7}{⼗、⼠}
  \definition{s.}{voz natural | voz verdadeira}
  \seealsoref{假声}{jia3sheng1}
\end{entry}

\begin{entry}{真实}{zhen1shi2}{10,8}{⼗、⼧}[HSK 3]
  \definition{adj.}{verdadeiro; real; autêntico; de acordo com fatos objetivos}
\end{entry}

\begin{entry}{真释}{zhen1shi4}{10,12}{⼗、⾤}
  \definition{s.}{razão genuína | explicação verdadeira}
\end{entry}

\begin{entry}{真相}{zhen1xiang4}{10,9}{⼗、⽬}[HSK 5]
  \definition{s.}{face; verdade; verdade nua e crua; a situação real; o estado real das coisas; a verdadeira situação}
\end{entry}

\begin{entry}{真心}{zhen1xin1}{10,4}{⼗、⼼}
  \definition{adj.}{sincero}
  \definition[片]{s.}{sinceridade}
\end{entry}

\begin{entry}{真真}{zhen1zhen1}{10,10}{⼗、⼗}
  \definition{adv.}{genuinamente | realmente | escrupulosamente}
\end{entry}

\begin{entry}{真正}{zhen1zheng4}{10,5}{⼗、⽌}[HSK 2]
  \definition{adj.}{verdadeiro; real; genuíno}
  \definition{adv.}{realmente; de fato; expressa afirmação de uma ação ou situação, equivalente a 确实}
  \seealsoref{确实}{que4shi2}
\end{entry}

\begin{entry}{真珠}{zhen1zhu1}{10,10}{⼗、⽟}
  \variantof{珍珠}
\end{entry}

\begin{entry}{诊断}{zhen3duan4}{7,11}{⾔、⽄}[HSK 5]
  \definition{s.}{diagnóstico; diacrisis}
  \definition{v.}{diagnosticar; após examinar os sintomas do paciente, determinar a doença e seu desenvolvimento}
\end{entry}

\begin{entry}{枕}{zhen3}{8}{⽊}
  \definition{s.}{travesseiro | almofada}
\end{entry}

\begin{entry}{阵}{zhen4}{6}{⾩}[HSK 4]
  \definition{clas.}{passagens que expressam a passagem de eventos ou ações}
  \definition{s.}{matriz de batalha (formação); termo tático antigo para as fileiras ou formações de uma equipe de combate | \emph{front}; frente de batalha; posição | um período de tempo}
\end{entry}

\begin{entry}{阵地}{zhen4di4}{6,6}{⾩、⼟}
  \definition{s.}{posição (militar) | frente de batalha | \emph{front}}
\end{entry}

\begin{entry}{振动}{zhen4dong4}{10,6}{⼿、⼒}[HSK 5]
  \definition{s.}{vibração}
  \definition{v.}{sacudir; balançar; tremer; roncar; tagarelar; vibrar; oscilar; a física se refere ao movimento contínuo de um objeto em torno de um determinado ponto no espaço, como o movimento de um pêndulo, um diapasão ou uma corda de violão}
\end{entry}

\begin{entry}{震撼}{zhen4han4}{15,16}{⾬、⼿}
  \definition{v.}{sacudir | chocar | atordoar}
\end{entry}

\begin{entry}{震惊}{zhen4jing1}{15,11}{⾬、⼼}[HSK 5]
  \definition{adj.}{cado; atordoado; espantado; atônito}
  \definition{v.}{chocar; surpreender; espantar}
\end{entry}

\begin{entry}{正}{zheng1}{5}{⽌}
  \definition{s.}{o primeiro mês do ano lunar; a primeira lua}
  \seeref{正}{zheng4}
\end{entry}

\begin{entry}{争}{zheng1}{6}{⼑}[HSK 3]
  \definition*{s.}{sobrenome Zheng}
  \definition{adv.}{como; por que}
  \definition{v.}{competir; disputar; lutar; esforçar-se para obter ou alcançar | discutir; argumentar; contestar; debater | faltar; estar em falta}
\end{entry}

\begin{entry}{争霸}{zheng1ba4}{6,21}{⼑、⾬}
  \definition{s.}{hegemonia | uma luta de poder}
  \definition{v.}{disputar a hegemonia}
\end{entry}

\begin{entry}{争风吃醋}{zheng1feng1chi1cu4}{6,4,6,15}{⼑、⾵、⼝、⾣}
  \definition{v.}{rivalizar com alguém pelo carinho de um homem ou mulher |estar com ciúmes de um rival em um caso de amor}
\end{entry}

\begin{entry}{争论}{zheng1lun4}{6,6}{⼑、⾔}[HSK 4]
  \definition{s.}{debate; discussão; argumentação; disputa}
  \definition{v.}{discutir; disputar; debater; argumentar; contestar}
\end{entry}

\begin{entry}{争取}{zheng1qu3}{6,8}{⼑、⼜}[HSK 3]
  \definition{v.}{lutar por; conquistar; vencer; se esforçar para conseguir}
\end{entry}

\begin{entry}{争先}{zheng1xian1}{6,6}{⼑、⼉}
  \definition{v.}{competir para ser o primeiro |contestar o primeiro lugar}
\end{entry}

\begin{entry}{争议}{zheng1yi4}{6,5}{⼑、⾔}[HSK 5]
  \definition{s.}{disputa; controvérsia; situações e questões em que há divergências de opinião}
  \definition{v.}{debater; discutir}
\end{entry}

\begin{entry}{征服}{zheng1fu2}{8,8}{⼻、⽉}[HSK 4]
  \definition{v.}{conquistar; cativar | subjugar; dominar}
\end{entry}

\begin{entry}{征求}{zheng1qiu2}{8,7}{⼻、⽔}[HSK 4]
  \definition{v.}{procurar; buscar; solicitar; pedir abertamente opiniões, pontos de vista, etc.}
\end{entry}

\begin{entry}{挣扎}{zheng1zha2}{9,4}{⼿、⼿}
  \definition{v.}{lutar}
\end{entry}

\begin{entry}{整}{zheng3}{16}{⽁}[HSK 3]
  \definition*{s.}{sobrenome Zheng}
  \definition{adj.}{cheio; integral; inteiro; completo; sem defeitos | limpo; arrumado; organizado; em boa ordem | redondo (não é uma fração)}
  \definition{s.}{número inteiro (não fracionário)}
  \definition{v.}{retificar; corrigir; pôr em ordem | consertar; renovar; reparar | corrigir; punir; causar sofrimento;  fazer alguém sofrer | fazer; realizar; trabalhar; em algumas regiões, significa 做, 搞}
  \seealsoref{搞}{gao3}
  \seealsoref{做}{zuo4}
\end{entry}

\begin{entry}{整个}{zheng3ge4}{16,3}{⽁、⼈}[HSK 3]
  \definition{adj.}{total; inteiro; completo}
\end{entry}

\begin{entry}{整理}{zheng3li3}{16,11}{⽁、⽟}[HSK 3]
  \definition{v.}{organizar; reorganizar; classificar; ordenar; colocar em ordem}
\end{entry}

\begin{entry}{整齐}{zheng3qi2}{16,6}{⽁、⿑}[HSK 3]
  \definition{adj.}{arrumado; organizado; em boa ordem | uniforme; regular; tamanho, comprimento, grau, etc. são relativamente consistentes | usado para descrever que todas as coisas necessárias estão prontas}
  \definition{v.}{estar em boas condições; manter a ordem e a organização}
\end{entry}

\begin{entry}{整体}{zheng3ti3}{16,7}{⽁、⼈}[HSK 3]
  \definition[个]{s.}{um todo; totalidade}
\end{entry}

\begin{entry}{整天}{zheng3 tian1}{16,4}{⽁、⼤}[HSK 3]
  \definition{s.}{o dia inteiro; o dia todo; durante todo o dia; de manhã à noite}
\end{entry}

\begin{entry}{整整}{zheng3 zheng3}{16,16}{⽁、⽁}[HSK 3]
  \definition{adv.}{inteiramente; completamente; solidamente; continuamente}
\end{entry}

\begin{entry}{正}{zheng4}{5}{⽌}[HSK 1,3]
  \definition*{s.}{sobrenome Zheng}
  \definition{adj.}{reto; ereto; vertical | principal; posicionado no meio | direito; anverso | honesto; íntegro; justo | puro; sem mistura (de cor ou sabor) | regular; padronizado; de acordo com a lei; correto | chefe; comandante; diretor | regular; as laterais e os ângulos do gráfico têm comprimentos e tamanhos iguais | positivo; (matemática), significa maior que zero; (física) significa perda de elétrons (oposto de 负) | exato; preciso; usado para indicar tempo, refere-se ao momento exato ou ao ponto médio de um período}
  \definition{adv.}{apenas; certo; exatamente; precisamente | agora mesmo; neste momento; indica a continuidade de uma ação ou a permanência de um estado}
  \definition{v.}{definir (colocar) corretamente; alinhar; endireitar | ajustar; corrigir; retificar}
  \seeref{正}{zheng1}
  \seealsoref{负}{fu4}
\end{entry}

\begin{entry}{正版}{zheng4 ban3}{5,8}{⽌、⽚}[HSK 5]
  \definition{s.}{versão genuína; versão autorizada; versão publicada e distribuída oficialmente por uma editora legal (em contraste com a 盗版)}
  \seealsoref{盗版}{dao4ban3}
\end{entry}

\begin{entry}{正常}{zheng4chang2}{5,11}{⽌、⼱}[HSK 2]
  \definition{adj.}{normal; regular; conforma-se com regras ou circunstâncias gerais}
\end{entry}

\begin{entry}{正规}{zheng4gui1}{5,8}{⽌、⾒}[HSK 5]
  \definition{adj.}{normal; regular; padrão; está em conformidade com padrões formalmente definidos ou geralmente reconhecidos}
\end{entry}

\begin{entry}{正好}{zheng4hao3}{5,6}{⽌、⼥}[HSK 2]
  \definition{adj.}{na hora certa; na hora certa; o suficiente}
  \definition{adv.}{acontecer com; chance de; como acontece}
\end{entry}

\begin{entry}{正确}{zheng4que4}{5,12}{⽌、⽯}[HSK 2]
  \definition{adj.}{correto; certo; próprio; conforma-se com fatos, razão ou algum padrão geralmente aceito}
\end{entry}

\begin{entry}{正如}{zheng4 ru2}{5,6}{⽌、⼥}[HSK 5]
  \definition{adv.}{exatamente como; assim como}
\end{entry}

\begin{entry}{正式}{zheng4shi4}{5,6}{⽌、⼷}[HSK 3]
  \definition{adj.}{formal; oficial; descreve uma atmosfera séria, atitudes ou comportamentos que não são fáceis ou descontraídos | formal; oficial; descreve o cumprimento de determinados trâmites e procedimentos}
\end{entry}

\begin{entry}{正是}{zheng4 shi4}{5,9}{⽌、⽇}[HSK 2]
  \definition{v.}{ser precisamente; ser exatamente}
\end{entry}

\begin{entry}{正义}{zheng4yi4}{5,3}{⽌、⼂}[HSK 5]
  \definition{adj.}{justo; íntegro}
  \definition{s.}{justiça; o que é certo; o que é benéfico para o povo | (frequentemente em títulos de livros) interpretação ortodoxa ou retificada (de textos antigos)}
\end{entry}

\begin{entry}{正在}{zheng4zai4}{5,6}{⽌、⼟}[HSK 1]
  \definition{adv.}{em processo de; em andamento; indica que uma ação está em andamento ou que uma situação está em curso.}
  \definition{v.}{estar a + {v.inf.} | estar + {v.ger.}}
\end{entry}

\begin{entry}{正正}{zheng4zheng4}{5,5}{⽌、⽌}
  \definition{adv.}{na hora certa | ordenadamente}
\end{entry}

\begin{entry}{正宗}{zheng4zong1}{5,8}{⽌、⼧}
  \definition{adj.}{autêntico | genuíno | \emph{old school} | (fig.) tradicional}
\end{entry}

\begin{entry}{证}{zheng4}{7}{⾔}[HSK 3]
  \definition{s.}{evidência; prova; testemunho | certificado; cartão | evidência; testemunha | doença; enfermidade}
  \definition{v.}{provar; demonstrar | verificar}
\end{entry}

\begin{entry}{证件}{zheng4jian4}{7,6}{⾔、⼈}[HSK 3]
  \definition[个,本,张,份]{s.}{documentos; credenciais; certificado; documentos que comprovem a identidade, experiência, etc., tais como carteira de estudante, carteira de trabalho, diploma de graduação, etc.}
\end{entry}

\begin{entry}{证据}{zheng4ju4}{7,11}{⾔、⼿}[HSK 3]
  \definition{s.}{prova; evidência; testemunho; fatos ou materiais que comprovam a veracidade de algo}
\end{entry}

\begin{entry}{证明}{zheng4ming2}{7,8}{⾔、⽇}[HSK 3]
  \definition[个,份]{s.}{certificado; atestado; identificação; certificado ou carta de referência; documentos que comprovem identidade, experiência, etc., tais como carteira de estudante, carteira de trabalho, diploma de graduação, etc.}
  \definition{v.}{provar; testemunhar; sustentar; usar materiais confiáveis para demonstrar ou determinar a autenticidade de pessoas ou coisas}
\end{entry}

\begin{entry}{证实}{zheng4shi2}{7,8}{⾔、⼧}[HSK 5]
  \definition{v.}{verificar; afirmar; confirmar; corroborar; demonstrar; autenticar; provar que é verdadeiro}
\end{entry}

\begin{entry}{证书}{zheng4shu1}{7,4}{⾔、⼄}[HSK 5]
  \definition[张,份,些]{s.}{certificado; documentos emitidos por instituições, grupos, etc., que comprovem experiência, nível, honras, poderes, etc.}
\end{entry}

\begin{entry}{挣}{zheng4}{9}{⼿}[HSK 5]
  \definition{v.}{empurrar e puxar; tentar se livrar; lutar para se libertar; esforçar-se para se libertar das amarras | ganhar; fazer; trabalhar para; trocar trabalho por}
\end{entry}

\begin{entry}{挣得}{zheng4de2}{9,11}{⼿、⼻}
  \definition{v.}{ganhar renda ou dinheiro}
\end{entry}

\begin{entry}{挣钱}{zheng4qian2}{9,10}{⼿、⾦}[HSK 5]
  \definition{v.+compl.}{ganhar dinheiro; fazer dinheiro; lucrar; trabalhar para ganhar dinheiro}
\end{entry}

\begin{entry}{政府}{zheng4fu3}{9,8}{⽁、⼴}[HSK 4]
  \definition[个]{s.}{governo;  órgãos executivos do poder do Estado, ou seja, órgãos administrativos do Estado, como o Conselho de Estado (Governo Popular Central) e os governos populares locais em todos os níveis na China}
\end{entry}

\begin{entry}{政纲}{zheng4gang1}{9,7}{⽁、⽷}
  \definition{s.}{programa ou plataforma política}
\end{entry}

\begin{entry}{政治}{zheng4zhi4}{9,8}{⽁、⽔}[HSK 4]
  \definition{s.}{política; assuntos políticos; questões políticas}
\end{entry}

\begin{entry}{政治局}{zheng4zhi4ju2}{9,8,7}{⽁、⽔、⼫}
  \definition{s.}{o principal comitê de políticas de um partido comunista}
\end{entry}

\begin{entry}{之后}{zhi1 hou4}{3,6}{⼂、⼝}[HSK 4]
  \definition{adv.}{mais tarde; posteriormente; depois; desde então; para indicar que é depois de um determinado tempo ou de uma determinada coisa, 以后 é usado com frequência na linguagem falada; às vezes, também pode indicar que é depois de um determinado lugar ou local,  后面 é usado com frequência na linguagem falada}
  \seealsoref{后面}{hou4mian4}
  \seealsoref{以后}{yi3 hou4}
\end{entry}

\begin{entry}{之间}{zhi1 jian1}{3,7}{⼂、⾨}[HSK 4]
  \definition{prep.}{(depois de um substantivo) entre; dentro de duas delimitações de tempo, local ou quantitativas | colocado após certos verbos ou advérbios de duas sílabas para indicar um curto período de tempo}
\end{entry}

\begin{entry}{之类}{zhi1 lei4}{3,9}{⼂、⽶}
  \definition{s.}{usado para dar exemplos (coisas do tipo, desse tipo, assim); uma categoria de pessoas ou coisas que compartilham as mesmas características das pessoas ou coisas mencionadas anteriormente}[我喜欢香蕉、苹果之类的水果。___Eu gosto de frutas como bananas e maçãs.]
\end{entry}

\begin{entry}{之内}{zhi1 nei4}{3,4}{⼂、⼌}[HSK 5]
  \definition{adv.}{em; dentro de; indica dentro de um determinado intervalo, limite ou período de tempo, etc.}
\end{entry}

\begin{entry}{之前}{zhi1 qian2}{3,9}{⼂、⼑}[HSK 4]
  \definition{adv.}{(referindo-se ao tempo) antes, antes de, atrás | (referindo-se ao local físico) na frente de | (usado independentemente) no passado, antes disso}
\end{entry}

\begin{entry}{之外}{zhi1 wai4}{3,5}{⼂、⼣}[HSK 5]
  \definition{adv.}{lado de fora; exceto; além de; além disso; refere-se a algo que excede um determinado limite}
\end{entry}

\begin{entry}{之下}{zhi1 xia4}{3,3}{⼂、⼀}[HSK 5]
  \definition{s.}{usado para indicar algo abaixo de um determinado intervalo, posição, grau, etc.; indica um aspecto inferior em termos de alcance, posição, status, nível, Chengdu, etc. | usado para indicar as condições sob as quais algo acontece | usado para indicar o humor, estado em que alguém faz algo; expressa um determinado comportamento em um determinado estado de espírito ou situação}
\end{entry}

\begin{entry}{之一}{zhi1 yi1}{3,1}{⼂、⼀}[HSK 4]
  \definition{s.}{um de (algo); pertence a um ou a todo um grupo de coisas com as mesmas características}
\end{entry}

\begin{entry}{之中}{zhi1 zhong1}{3,4}{⼂、⼁}[HSK 5]
  \definition{prep.}{em; no meio de; entre}
\end{entry}

\begin{entry}{支}{zhi1}{4}{⽀}[HSK 3,4][Kangxi 65]
  \definition*{s.}{sobrenome Zhi}
  \definition{clas.}{usado para equipes, etc. | usado em canções ou peças musicais | intensidade luminosa utilizada para luzes elétricas | usado para objetos longos e finos}
  \definition{s.}{ramo; ramificação; tribo | os doze ramos terrestres}
  \definition{v.}{segurar; apoiar; sustentar | sobressair; esticar; erguer; estender | ordenar; enviar | receber; pagar; obter pagamento (dinheiro)}
\end{entry}

\begin{entry}{支承}{zhi1cheng2}{4,8}{⽀、⼿}
  \definition{v.}{suportar o peso de (um edifício) | suportar}
\end{entry}

\begin{entry}{支持}{zhi1chi2}{4,9}{⽀、⼿}[HSK 3]
  \definition{v.}{suportar; aguentar; resistir; manter-se com dificuldade | apoiar; dar incentivo ou patrocínio}
\end{entry}

\begin{entry}{支出}{zhi1chu1}{4,5}{⽀、⼐}[HSK 5]
  \definition{s.}{despesas; gastos}
  \definition{v.}{pagar; gastar; desembolsar; efetuar o pagamento}
\end{entry}

\begin{entry}{支付}{zhi1 fu4}{4,5}{⽀、⼈}[HSK 3]
  \definition{v.}{pagar (dinheiro); custear; efetuar pagamento}
\end{entry}

\begin{entry}{支根}{zhi1gen1}{4,10}{⽀、⽊}
  \definition{s.}{raiz ramificada | raízes de apoio | radícula}
\end{entry}

\begin{entry}{支配}{zhi1pei4}{4,10}{⽀、⾣}[HSK 5]
  \definition{v.}{organizar; alocar; orçar; distribuir | controlar; dominar; governar; exercer influência e controle sobre pessoas ou coisas}
\end{entry}

\begin{entry}{支票}{zhi1piao4}{4,11}{⽀、⽰}
  \definition[本]{s.}{cheque (banco)}
\end{entry}

\begin{entry}{支应}{zhi1ying4}{4,7}{⽀、⼴}
  \definition{v.}{lidar com | fornecer}
\end{entry}

\begin{entry}{支支吾吾}{zhi1zhi1wu2wu2}{4,4,7,7}{⽀、⽀、⼝、⼝}
  \definition{v.}{falhar | murmurar | paralisar | gaguejar}
\end{entry}

\begin{entry}{只}{zhi1}{5}{⼝}[HSK 3]
  \definition{adj.}{solteiro; solitário; único; muito raro}
  \definition{clas.}{usado para um de um par | usado para animais pequenos (pássaros, gatos, cães, etc.) | usado para certos utensílios, aparelhos | usado para navios}
  \seeref{只}{zhi3}
\end{entry}

\begin{entry}{只身}{zhi1shen1}{5,7}{⼝、⾝}
  \definition{adv.}{sozinho | por si só}
\end{entry}

\begin{entry}{芝麻}{zhi1ma5}{6,11}{⾋、⿇}
  \definition{s.}{semente de gergelim}
\end{entry}

\begin{entry}{知道}{zhi1dao4}{8,12}{⽮、⾡}[HSK 1]
  \definition{v.}{saber; perceber; estar ciente de; ter conhecimento dos fatos ou da razão; ser sensato}
\end{entry}

\begin{entry}{知道了}{zhi1dao4le5}{8,12,2}{⽮、⾡、⼅}
  \definition{interj.}{Entendi! | OK!}
\end{entry}

\begin{entry}{知了}{zhi1liao3}{8,2}{⽮、⼅}
  \definition[通]{s.}{cigarra}
\end{entry}

\begin{entry}{知识}{zhi1shi5}{8,7}{⽮、⾔}[HSK 1]
  \definition[个,门,种]{s.}{conhecimento; conjunto de conhecimentos e experiências adquiridos pelas pessoas na prática de transformar o mundo | intelectual; refere-se à cultura acadêmica}
\end{entry}

\begin{entry}{织}{zhi1}{8}{⽷}
  \definition{v.}{tecer | tricotar}
\end{entry}

\begin{entry}{脂麻}{zhi1ma5}{10,11}{⾁、⿇}
  \variantof{芝麻}
\end{entry}

\begin{entry}{蜘蛛}{zhi1zhu1}{14,12}{⾍、⾍}
  \definition{s.}{aranha}
\end{entry}

\begin{entry}{蜘蛛网}{zhi1zhu1wang3}{14,12,6}{⾍、⾍、⽹}
  \definition{s.}{teia de aranha}
\end{entry}

\begin{entry}{执行}{zhi2xing2}{6,6}{⼿、⾏}[HSK 5]
  \definition{v.}{executar; implementar; realizar}
\end{entry}

\begin{entry}{执着}{zhi2zhuo2}{6,11}{⼿、⽬}
  \definition{s.}{(budismo) apego}
  \definition{v.}{estar fortemente apegado a | ser dedicado | apegar-se a}
\end{entry}

\begin{entry}{直}{zhi2}{8}{⽬}[HSK 3]
  \definition*{s.}{sobrenome Zhi}
  \definition{adj.}{reto (oposto a 弯,曲) | vertical; perpendicular (oposto a 横) | justo; íntegro; imparcial | franco; sincero; direto ao ponto | rígido; entorpecido | direto; em linha reta; rígido | ereto; perpendicular ao solo}
  \definition{adv.}{diretamente; sempre; reto | continuamente; constantemente | apenas; simplesmente | de ​​fato}
  \definition[条]{s.}{traço vertical (em caracteres chineses, 竖)}
  \definition{v.}{endireitar; tornar reto | alongar}
  \seealsoref{横}{heng2}
  \seealsoref{曲}{qu1}
  \seealsoref{竖}{shu4}
  \seealsoref{弯}{wan1}
\end{entry}

\begin{entry}{直播}{zhi2bo1}{8,15}{⽬、⼿}[HSK 3]
  \definition{v.}{transmissão ao vivo; transmitir diretamente, sem gravar}
\end{entry}

\begin{entry}{直到}{zhi2 dao4}{8,8}{⽬、⼑}[HSK 3]
  \definition{adv.}{até (geralmente se refere ao tempo); até que}
\end{entry}

\begin{entry}{直接}{zhi2jie1}{8,11}{⽬、⼿}[HSK 2]
  \definition{adj.}{direto (oposto: indireto 间接) | imediato}
  \seealsoref{间接}{jian4jie1}
\end{entry}

\begin{entry}{直线}{zhi2 xian4}{8,8}{⽬、⽷}[HSK 5]
  \definition{adj.}{direto; retilíneo | íngreme; acentuada (subida ou descida)}
  \definition[条]{s.}{linha reta}
\end{entry}

\begin{entry}{直译}{zhi2yi4}{8,7}{⽬、⾔}
  \definition{s.}{tradução literal}
  \seealsoref{意译}{yi4yi4}
\end{entry}

\begin{entry}{直译器}{zhi2yi4qi4}{8,7,16}{⽬、⾔、⼝}
  \definition{s.}{(computação) interpretador}
\end{entry}

\begin{entry}{值}{zhi2}{10}{⼈}[HSK 3]
  \definition{adj.}{significativo; valioso; digno de nota}
  \definition{prep.}{quando; introduz o momento em que algo acontece ou existe, equivalente a 当 ou 在}
  \definition{s.}{preço; valor | valor de um número, de uma variável}
  \definition{v.}{valer; custar; a mercadoria é adequada ao preço | ir de encontro; encontrar; cruzar | estar de serviço; ter sua vez em algo; assumir o cargo que lhe cabe | é a vez de (executar uma determinada função pública)}
  \seealsoref{当}{dang1}
  \seealsoref{在}{zai4}
\end{entry}

\begin{entry}{值班}{zhi2ban1}{10,10}{⼈、⽟}[HSK 5]
  \definition{v.}{estar em serviço ou plantão; trabalhar em um turno; (em rodízio) desempenhar funções durante um período de tempo determinado}
\end{entry}

\begin{entry}{值得}{zhi2de5}{10,11}{⼈、⼻}[HSK 3]
  \definition{adj.}{que tem valor; (fazer algo) é vantajoso, sem prejuízos}
  \definition{v.}{merecer; ter valor; significa que fazer isso terá bons resultados; que é valioso e significativo}
\end{entry}

\begin{entry}{职工}{zhi2 gong1}{11,3}{⽿、⼯}[HSK 3]
  \definition[个,位,名,些]{s.}{pessoal; trabalhadores e funcionários administrativos}
\end{entry}

\begin{entry}{职能}{zhi2neng2}{11,10}{⽿、⾁}[HSK 5]
  \definition{s.}{função; funções ou papéis que as organizações, instituições, etc. devem desempenhar}
\end{entry}

\begin{entry}{职位}{zhi2wei4}{11,7}{⽿、⼈}[HSK 5]
  \definition[个]{s.}{posto; posição; cargo que exerce determinadas funções em órgãos ou entidades}
\end{entry}

\begin{entry}{职务}{zhi2wu4}{11,5}{⽿、⼒}[HSK 5]
  \definition{s.}{cargo; posto; deveres; função; funções que devem ser desempenhadas de acordo com as especificações do cargo}
\end{entry}

\begin{entry}{职业}{zhi2ye4}{11,5}{⽿、⼀}[HSK 3]
  \definition{adj.}{profissional; não amador}
  \definition[种,份,个]{s.}{ocupação; profissão; vocação; o trabalho que um indivíduo realiza na sociedade como sua principal fonte de subsistência}
\end{entry}

\begin{entry}{职员}{zhi2yuan2}{11,7}{⽿、⼝}
  \definition[个,位]{s.}{empregado | trabalhador de escritório | membro da equipe}
\end{entry}

\begin{entry}{植物}{zhi2wu4}{12,8}{⽊、⽜}[HSK 4]
  \definition[种,株,盆,棵]{s.}{planta; vegetação; flora}
\end{entry}

\begin{entry}{殖}{zhi2}{12}{⽍}
  \definition{v.}{crescer | reproduzir}
\end{entry}

\begin{entry}{只}{zhi3}{5}{⼝}[HSK 2]
  \definition{adv.}{somente; apenas; meramente | simplesmente; usado para limitar o escopo, indicando que não há nada além disso, equivalente a 仅仅}
  \seeref{只}{zhi1}
  \seealsoref{仅仅}{jin3 jin3}
\end{entry}

\begin{entry}{只不过}{zhi3 bu2 guo4}{5,4,6}{⼝、⼀、⾡}[HSK 5]
  \definition{adv.}{somente; apenas; meramente; não mais do que}
\end{entry}

\begin{entry}{只得}{zhi3de5}{5,11}{⼝、⼻}
  \definition{v.}{ser obrigado a | não ter outra alternativa senão}
\end{entry}

\begin{entry}{只读}{zhi3du2}{5,10}{⼝、⾔}
  \definition{s.}{somente leitura (computação) | \emph{read-only}}
\end{entry}

\begin{entry}{只顾}{zhi3gu4}{5,10}{⼝、⾴}
  \definition{adv.}{exclusivamente preocupado (com uma coisa)}
  \definition{v.}{cuidar de apenas um aspecto}
\end{entry}

\begin{entry}{只管}{zhi3 guan3}{5,14}{⼝、⽵}
  \definition{adv.}{por todos os meios; expressa incentivo para que os outros façam algo com confiança, sem se preocuparem com outras coisas | apenas; simplesmente; significa fazer uma coisa com seriedade, sem se preocupar com outras coisas}
\end{entry}

\begin{entry}{只好}{zhi3hao3}{5,6}{⼝、⼥}[HSK 3]
  \definition{v.}{ter que; ser forçado a; não ter escolha a não ser; significa que só pode ser assim, não há outra opção}
\end{entry}

\begin{entry}{只见}{zhi3 jian4}{5,4}{⼝、⾒}[HSK 5]
  \definition{v.}{somente ver; ver; só vi, e de repente percebi uma certa situação}
\end{entry}

\begin{entry}{只能}{zhi3 neng2}{5,10}{⼝、⾁}[HSK 2]
  \definition{adv.}{só pode; obrigado a fazer algo; isso significa que devido à limitação da capacidade pessoal ou às condições objetivas, não há outra escolha senão esta}
\end{entry}

\begin{entry}{只怕}{zhi3pa4}{5,8}{⼝、⼼}
  \definition{adv.}{receio que\dots | talvez | muito provavelmente}
\end{entry}

\begin{entry}{只是}{zhi3 shi4}{5,9}{⼝、⽇}[HSK 3]
  \definition{adv.}{somente; meramente; apenas; expressa ênfase limitada a uma determinada situação ou âmbito}
  \definition{conj.}{somente; mas; exceto que; conecta frases, indicando uma ligeira transição, equivalente a 不过}
  \seealsoref{不过}{bu2guo4}
\end{entry}

\begin{entry}{只消}{zhi3xiao1}{5,10}{⼝、⽔}
  \definition{conj.}{desde que}
\end{entry}

\begin{entry}{只要}{zhi3yao4}{5,9}{⼝、⾑}[HSK 2]
  \definition{conj.}{desde que; se apenas; contanto que; indica condições necessárias (就 ou 可 são frequentemente usados depois)}
  \seealsoref{便}{bian4}
  \seealsoref{就}{jiu4}
\end{entry}

\begin{entry}{只要……就……}{zhi3yao4 jiu4}{5,9,12}{⼝、⾑、⼪}
  \definition{conj.}{contanto que/desde que/se somente\dots, então\dots}
\end{entry}

\begin{entry}{只有}{zhi3 you3}{5,6}{⼝、⽉}[HSK 3]
  \definition{adv.}{somente; tem que; forçado a}
  \definition{conj.}{somente se; conecta frases, expressa condições necessárias, geralmente corresponde a 才 e 方}
  \seealsoref{才}{cai2}
  \seealsoref{方}{fang1}
\end{entry}

\begin{entry}{只有……才……}{zhi3you3 cai2}{5,6,3}{⼝、⽉、⼿}
  \definition{conj.}{só se\dots então\dots}
\end{entry}

\begin{entry}{纸}{zhi3}{7}{⽷}[HSK 2]
  \definition{clas.}{usado para documentos, cartas, etc.}
  \definition[张,沓]{s.}{papel; uma folha fina de material usada para escrever, pintar, imprimir, embalar, etc., feita principalmente de fibras vegetais | papel joss; papel de incenso; refere-se especificamente a itens supersticiosos, como papel-moeda}
\end{entry}

\begin{entry}{纸币}{zhi3bi4}{7,4}{⽷、⼱}
  \definition[张]{s.}{nota (dinheiro) | cédula}
\end{entry}

\begin{entry}{纸巾}{zhi3jin1}{7,3}{⽷、⼱}
  \definition[张,包]{s.}{lenço | guardanapo | papel toalha}
\end{entry}

\begin{entry}{纸尿裤}{zhi3niao4ku4}{7,7,12}{⽷、⼫、⾐}
  \definition{s.}{fralda descartável}
\end{entry}

\begin{entry}{纸烟}{zhi3yan1}{7,10}{⽷、⽕}
  \definition{s.}{cigarro}
\end{entry}

\begin{entry}{纸张}{zhi3zhang1}{7,7}{⽷、⼸}
  \definition{s.}{papel}
\end{entry}

\begin{entry}{指}{zhi3}{9}{⼿}[HSK 3]
  \definition*{s.}{sobrenome Zhi}
  \definition{clas.}{dígito; largura do dedo; a largura de um dedo é chamada de 一指, que é usado para medir profundidade, largura, etc.}
  \definition{s.}{dedo}
  \definition{v.}{apontar para; indicar; usar o dedo ou a ponta de um objeto para apontar | (pelo) eriçar;  (cabelo) ficar em pé | indicar; mostrar; apontar; demonstrar | referir-se a; dirigir-se a | confiar em; contar com; depender de | criticar; repreender}
\end{entry}

\begin{entry}{指标}{zhi3biao1}{9,9}{⼿、⽊}[HSK 5]
  \definition{s.}{meta; cota; norma; índice; objetivos a serem alcançados | alvo; índice; refletir os requisitos de desenvolvimento em determinados aspectos através de números absolutos ou percentagens de aumento ou diminuição, inclui indicadores quantitativos e qualitativos}
\end{entry}

\begin{entry}{指出}{zhi3 chu1}{9,5}{⼿、⼐}[HSK 3]
  \definition{v.}{apontar; indicar}
\end{entry}

\begin{entry}{指导}{zhi3dao3}{9,6}{⼿、⼨}[HSK 3]
  \definition[位]{s.}{guia; diretor; pessoa que dá orientações}
  \definition{v.}{orientar; dirigir; instruir}
\end{entry}

\begin{entry}{指挥}{zhi3hui1}{9,9}{⼿、⼿}[HSK 4]
  \definition[个]{s.}{diretor; comandante; despachante; operador | maestro; condutor; pessoa na frente de uma orquestra ou coro que dá instruções sobre como tocar ou cantar}
  \definition{v.}{dirigir; conduzir; comandar; direcionar; emitir ordens de agendamento}
\end{entry}

\begin{entry}{指甲}{zhi3jia5}{9,5}{⼿、⽥}[HSK 5]
  \definition{s.}{unha; unha de agulha; unha de dedo; camada córnea na ponta dos dedos}
\end{entry}

\begin{entry}{指南针}{zhi3nan2zhen1}{9,9,7}{⼿、⼗、⾦}
  \definition{s.}{bússola}
\end{entry}

\begin{entry}{指示}{zhi3shi4}{9,5}{⼿、⽰}[HSK 5]
  \definition{s.}{diretriz; instruções; para orientar o trabalho, os superiores emitem opiniões verbais ou escritas aos subordinados}
  \definition{v.}{indicar; apontar; apontar para alguém | instruir; superiores emitem opiniões verbais ou escritas para orientar o trabalho dos subordinados}
\end{entry}

\begin{entry}{指责}{zhi3ze2}{9,8}{⼿、⾙}[HSK 5]
  \definition{v.}{censurar; criticar; encontrar falhas; repreender}
\end{entry}

\begin{entry}{黹}{zhi3}{12}{⿋}[Kangxi 204]
  \definition{v.}{costurar; bordar}
\end{entry}

\begin{entry}{至}{zhi4}{6}{⾄}[HSK 5]
  \definition{adv.}{a maior parte; extremamente; indica o grau mais alto, equivalente a 极 ou 最}
  \definition{prep.}{para; até; chegar a um determinado ponto}
  \definition{s.}{extremo, máximo}
  \definition{v.}{chegar; alcançar}
  \seealsoref{极}{ji2}
  \seealsoref{最}{zui4}
\end{entry}

\begin{entry}{至今}{zhi4jin1}{6,4}{⾄、⼈}[HSK 3]
  \definition{adv.}{até agora; até o momento; até hoje}
\end{entry}

\begin{entry}{至少}{zhi4shao3}{6,4}{⾄、⼩}[HSK 3]
  \definition{adv.}{pelo menos; indica o limite mínimo}
\end{entry}

\begin{entry}{至于}{zhi4yu2}{6,3}{⾄、⼆}
  \definition{conj.}{para | quanto a | a respeiro de}
\end{entry}

\begin{entry}{志愿}{zhi4 yuan4}{7,14}{⼼、⽕}[HSK 3]
  \definition{s.}{desejo; ideal; aspiração; os ideais, desejos ou objetivos que se deseja realizar no coração}
  \definition{v.}{ser voluntário; ser proativo e disposto a realizar trabalhos sem remuneração ou com remuneração baixa, mas que possam ajudar outras pessoas}
\end{entry}

\begin{entry}{志愿书}{zhi4yuan4shu1}{7,14,4}{⼼、⽕、⼄}
  \definition{s.}{formulário de inscrição; formulário de adesão; carta de intenções}
\end{entry}

\begin{entry}{志愿者}{zhi4yuan4zhe3}{7,14,8}{⼼、⽕、⽼}[HSK 3]
  \definition[名,位,个]{s.}{voluntário; pessoas que se voluntariam para prestar serviços em atividades sociais, grandes eventos esportivos, conferências, etc.}
\end{entry}

\begin{entry}{制裁}{zhi4cai2}{8,12}{⼑、⾐}
  \definition{s.}{punição | sanção (inclusive econômica)}
  \definition{v.}{punir}
\end{entry}

\begin{entry}{制成}{zhi4 cheng2}{8,6}{⼑、⼽}[HSK 5]
  \definition{v.}{fabricar; ser feito de; produzir}
\end{entry}

\begin{entry}{制订}{zhi4 ding4}{8,4}{⼑、⾔}[HSK 4]
  \definition{v.}{esboçar; formular; elaborar; mapear}
\end{entry}

\begin{entry}{制定}{zhi4ding4}{8,8}{⼑、⼧}[HSK 3]
  \definition{v.}{rascunhar; formular; elaborar; estabelecer (leis, regulamentos, planos, etc.)}
\end{entry}

\begin{entry}{制度}{zhi4du4}{8,9}{⼑、⼴}[HSK 3]
  \definition[项,条,套,种]{s.}{regulamentação; regulamento; procedimentos operacionais ou diretrizes de conduta que todos devem seguir | sistema; o sistema político, econômico e cultural formado sob determinadas condições históricas}
\end{entry}

\begin{entry}{制约}{zhi4yue1}{8,6}{⼑、⽷}[HSK 5]
  \definition{v.}{limitar; verificar; restringir; a existência e a mudança de uma coisa determinam a existência e a mudança de outra coisa}
\end{entry}

\begin{entry}{制造}{zhi4zao4}{8,10}{⼑、⾡}[HSK 3]
  \definition{v.}{fazer; produzir; manufaturar; transformar matérias-primas em produtos acabados | criar; agitar; criar artificialmente uma situação ou atmosfera desfavorável}
\end{entry}

\begin{entry}{制作}{zhi4zuo4}{8,7}{⼑、⼈}[HSK 3]
  \definition{v.}{fazer; produzir; itens feitos com matérias-primas, geralmente pequenos e feitos à mão | fazer; produzir; criar gráficos, anúncios, filmes, jogos, etc., utilizando texto, imagens, sons, imagens, etc.}
\end{entry}

\begin{entry}{治}{zhi4}{8}{⽔}[HSK 4]
  \definition*{s.}{sobrenome Zhi}
  \definition{adj.}{calmo e pacífico}
  \definition{s.}{sede de um antigo governo local}
  \definition{v.}{reger; administrar; governar; gerenciar; gerir | tratar (uma doença); curar; sarar | eliminar; controlar pragas | controlar (um rio); restaurar um curso d'água por meio de dragagem | punir; castigar | estudar; pesquisar; explorar}
\end{entry}

\begin{entry}{治安}{zhi4'an1}{8,6}{⽔、⼧}[HSK 5]
  \definition{s.}{ordem pública; segurança pública; ordem social estável}
\end{entry}

\begin{entry}{治理}{zhi4li3}{8,11}{⽔、⽟}[HSK 5]
  \definition{s.}{governo | governança}
  \definition{v.}{dirigir; gerenciar; governar; administrar | tratar; aproveitar; colocar sob controle; colocar em ordem}
\end{entry}

\begin{entry}{治疗}{zhi4liao2}{8,7}{⽔、⽧}[HSK 4]
  \definition{s.}{diagnóstico; tratamento}
  \definition{v.}{tratar; curar; remediar; eliminar doenças por meio de medicamentos, cirurgia, etc.}
\end{entry}

\begin{entry}{治愈}{zhi4yu4}{8,13}{⽔、⼼}
  \definition{v.}{curar | restaurar a saúde}
\end{entry}

\begin{entry}{质量}{zhi4liang4}{8,12}{⾙、⾥}[HSK 4]
  \definition{s.}{qualidade; o quão bom ou ruim é o produto ou o trabalho}
\end{entry}

\begin{entry}{致敬}{zhi4jing4}{10,12}{⾄、⽁}
  \definition{v.}{saudar | prestar respeitos a | prestar homenagem a}
\end{entry}

\begin{entry}{智慧}{zhi4hui4}{12,15}{⽇、⼼}
  \definition{s.}{sabedoria | inteligência}
\end{entry}

\begin{entry}{智力}{zhi4li4}{12,2}{⽇、⼒}[HSK 4]
  \definition{s.}{inteligência; refere-se à capacidade de uma pessoa de conhecer e entender coisas objetivas e aplicar o conhecimento e a experiência para resolver problemas, incluindo memória, observação, imaginação, pensamento e julgamento}
\end{entry}

\begin{entry}{智能}{zhi4neng2}{12,10}{⽇、⾁}[HSK 4]
  \definition{adj.}{inteligente (telefone, sistema, etc.); descreve máquinas, equipamentos, tecnologia, etc. que foram processados com alta tecnologia e têm a capacidade de falar, pensar, calcular, resolver problemas, etc., como um ser humano}
  \definition{s.}{intelecto; a capacidade de aprender, agir, pensar, inventar, criar, resolver problemas, etc.}
\end{entry}

\begin{entry}{智商}{zhi4shang1}{12,11}{⽇、⼝}
  \definition{s.}{quociente de inteligência, QI}
\end{entry}

\begin{entry}{智障}{zhi4zhang4}{12,13}{⽇、⾩}
  \definition{adj./s.}{retardado}
\end{entry}

\begin{entry}{置疑}{zhi4yi2}{13,14}{⽹、⽦}
  \definition{v.}{duvidar}
\end{entry}

\begin{entry}{中}{zhong1}{4}{⼁}[HSK 1]
  \definition*{s.}{sobrenome Zhong}
  \definition*{s.}{China; referindo-se à China}
  \definition{adj.}{\emph{O.K.}; tudo bem, ótimo, adequado}
  \definition{s.}{centro; meio; a parte que está à mesma distância de todos os lados, acima e abaixo ou nas duas extremidades | em; entre (usado para indicar coisas dentro de um determinado intervalo) | meio; centro; localizado entre os dois extremos | médio; intermediário; classificação entre os dois extremos | médio; a meio caminho entre dois extremos; imparcial | intermediário | enquanto; durante (use após um verbo para mostrar que a ação está em andamento)}
  \definition{v.}{ser adequado para; ser compatível com}
  \seeref{中}{zhong4}
  \seealsoref{中国}{zhong1guo2}
\end{entry}

\begin{entry}{中部}{zhong1 bu4}{4,10}{⼁、⾢}[HSK 3]
  \definition{s.}{parte do meio; região central; seção central; região ou parte intermediária | parte do meio; parte central; refere-se à parte intermediária de uma série de três partes, como romances, obras cinematográficas e televisivas}
\end{entry}

\begin{entry}{中餐}{zhong1 can1}{4,16}{⼁、⾷}[HSK 2]
  \definition[份,顿,桌]{s.}{refeição chinesa; comida chinesa; comida de estilo chinês (diferente de 西餐) | almoço}
  \seealsoref{西餐}{xi1 can1}
\end{entry}

\begin{entry}{中东}{zhong1dong1}{4,5}{⼁、⼀}
  \definition*{s.}{Oriente Médio}
\end{entry}

\begin{entry}{中断}{zhong1duan4}{4,11}{⼁、⽄}[HSK 5]
  \definition{v.}{suspender; romper; descontinuar; interromper; quebrar | dividir; quebrar; ser quebrado}
\end{entry}

\begin{entry}{中国}{zhong1guo2}{4,8}{⼁、⼞}[HSK 1]
  \definition*{s.}{China; os povos Huaxia e Han estabeleceram suas capitais principalmente ao sul e ao norte do rio Huang He, e por isso chamaram essa região de 中国, com o mesmo significado de 中土, 中原l, 中州 e 中华}
  \seealsoref{中土}{zhong1 tu3}
  \seealsoref{中华}{zhong1hua2}
  \seealsoref{中原}{zhong1yuan2}
  \seealsoref{中州}{zhong1zhou1}
\end{entry}

\begin{entry}{中国城}{zhong1guo2cheng2}{4,8,9}{⼁、⼞、⼟}
  \definition*{s.}{Bairro Chinês, \emph{Chinatown}}
  \seealsoref{唐人街}{tang2ren2 jie1}
\end{entry}

\begin{entry}{中国科学院}{zhong1guo2 ke1xue2yuan4}{4,8,9,8,9}{⼁、⼞、⽲、⼦、⾩}
  \definition*{s.}{Academia Chinesa de Ciências}
\end{entry}

\begin{entry}{中国人}{zhong1guo2ren2}{4,8,2}{⼁、⼞、⼈}
  \definition{s.}{chinês | pessoa ou povo da China}
\end{entry}

\begin{entry}{中国通}{zhong1guo2tong1}{4,8,10}{⼁、⼞、⾡}
  \definition*{s.}{Conhecedor da China, especialista em tudo sobre a China}
\end{entry}

\begin{entry}{中华}{zhong1hua2}{4,6}{⼁、⼗}
  \definition{s.}{China; na antiguidade, a região do rio Amarelo era chamada de Zhonghua, sendo o local onde a etnia Han surgiu inicialmente. Posteriormente, passou a designar a China}
  \seealsoref{中国}{zhong1guo2}
\end{entry}

\begin{entry}{中华民族}{zhong1 hua2 min2 zu2}{4,6,5,11}{⼁、⼗、⽒、⽅}[HSK 3]
  \definition*{s.}{O Povo Chinês; o nome genérico para todas as etnias da China, incluindo 56 etnias, com uma longa história, um rico patrimônio cultural e uma gloriosa tradição revolucionária | A Nação Chinesa}
\end{entry}

\begin{entry}{中级}{zhong1 ji2}{4,6}{⼁、⽷}[HSK 2]
  \definition{adj.}{nível médio; nível intermediário; entre avançado e iniciante}
\end{entry}

\begin{entry}{中间}{zhong1jian1}{4,7}{⼁、⾨}[HSK 1]
  \definition[本]{s.}{em meio a; entre; dentro de um determinado intervalo | meio; centro; posição entre os dois extremos de uma coisa ou entre duas coisas | posição intermediária; espaço entre duas extremidades; posição entre os dois extremos de uma coisa, dois momentos, duas coisas, etc.}
\end{entry}

\begin{entry}{中介}{zhong1jie4}{4,4}{⼁、⼈}[HSK 4]
  \definition[个]{s.}{agente; intermediário}
\end{entry}

\begin{entry}{中年}{zhong1 nian2}{4,6}{⼁、⼲}[HSK 2]
  \definition{s.}{meia-idade; na casa dos quarenta ou cinquenta anos}
\end{entry}

\begin{entry}{中情局}{zhong1qing2ju2}{4,11,7}{⼁、⼼、⼫}
  \definition*{s.}{Agência Central de Inteligência dos EUA, CIA (abreviação de 中央情报局)}
  \seealsoref{中央情报局}{zhong1yang1 qing2bao4ju2}
\end{entry}

\begin{entry}{中秋节}{zhong1 qiu1 jie2}{4,9,5}{⼁、⽲、⾋}[HSK 5]
  \definition*{s.}{Festival do Meio-Outono | Festival do Bolo Lunar (15º dia do oitavo mês lunar)}
\end{entry}

\begin{entry}{中土}{zhong1 tu3}{4,3}{⼁、⼟}
  \definition*{s.}{(fora de uso) China (Terra Média) | Sino-turco}
\end{entry}

\begin{entry}{中文}{zhong1wen2}{4,4}{⼁、⽂}[HSK 1]
  \definition{s.}{a língua chinesa; chinês; a língua e a escrita da China; especificamente, o chinês e os caracteres chineses}
\end{entry}

\begin{entry}{中午}{zhong1wu3}{4,4}{⼁、⼗}[HSK 1]
  \definition[个]{s.}{meio-dia; refere-se ao período entre 12 e 13 horas}
\end{entry}

\begin{entry}{中小学}{zhong1 xiao3 xue2}{4,3,8}{⼁、⼩、⼦}[HSK 2]
  \definition{s.}{escolas primárias e secundárias}
\end{entry}

\begin{entry}{中心}{zhong1xin1}{4,4}{⼁、⼼}[HSK 2]
  \definition[个]{s.}{núcleo; coração; meio; centro; posições com distâncias iguais de todas as áreas circundantes | chave; coração; a parte principal de algo; a pessoa ou coisa que desempenha um papel importante | centro; uma cidade ou lugar que tem um impacto significativo ou desempenha um papel importante em uma determinada área | centro; uma instituição com equipamentos e tecnologia relativamente completos e avançados em uma determinada área}
\end{entry}

\begin{entry}{中性}{zhong1xing4}{4,8}{⼁、⼼}
  \definition{adj.}{neutro}
\end{entry}

\begin{entry}{中学}{zhong1 xue2}{4,8}{⼁、⼦}[HSK 1]
  \definition[个]{s.}{escola ensino médio; escolas onde os jovens recebem educação secundária geralmente incluem o ensino fundamental II e o ensino médio | aprendizagem chinesa (um termo da dinastia Qing tardia para a aprendizagem tradicional chinesa); antigamente, referia-se à academia tradicional da China, como filosofia, linguística, medicina tradicional chinesa, etc.}
\end{entry}

\begin{entry}{中学生}{zhong1 xue2 sheng1}{4,8,5}{⼁、⼦、⽣}[HSK 1]
  \definition{s.}{aluno, estudante do ensino médio; alunos matriculados no ensino médio. Inclui alunos do ensino fundamental II e do ensino médio}
\end{entry}

\begin{entry}{中询}{zhong1 xun2}{4,8}{⼁、⾔}
  \definition{adv.}{segunda dezena do mês | meio do mês | em meados do mês}
\end{entry}

\begin{entry}{中央}{zhong1yang1}{4,5}{⼁、⼤}[HSK 5]
  \definition{s.}{centro; meio; localização central | autoridades centrais; refere-se especificamente ao órgão máximo de liderança de um país ou partido político}
\end{entry}

\begin{entry}{中央情报局}{zhong1yang1 qing2bao4ju2}{4,5,11,7,7}{⼁、⼤、⼼、⼿、⼫}
  \definition*{s.}{Agência Central de Inteligência dos EUA, CIA}
\end{entry}

\begin{entry}{中药}{zhong1 yao4}{4,9}{⼁、⾋}[HSK 5]
  \definition[服,种]{s.}{medicina herbal; medicina tradicional chinesa; fitoterapia; medicamentos utilizados na medicina tradicional chinesa; incluem medicamentos naturais e seus produtos processados (em contraste com 西药)}
  \seealsoref{西药}{xi1 yao4}
\end{entry}

\begin{entry}{中医}{zhong1 yi1}{4,7}{⼁、⼖}[HSK 2]
  \definition[位,个,名,些,群]{s.}{ciência médica tradicional chinesa; medicina chinesa | médico de medicina tradicional chinesa; praticante de medicina chinesa}
\end{entry}

\begin{entry}{中原}{zhong1yuan2}{4,10}{⼁、⼚}
  \definition*{s.}{Planícies Centrais (compreendendo os trechos médio e inferior do rio Huang He) | Planície Central, regiões média e baixa do rio Amarelo, incluindo Henan, oeste de Shandong, sul de Shanxi e Hebei}
\end{entry}

\begin{entry}{中州}{zhong1zhou1}{4,6}{⼁、⼮}
  \definition*{s.}{Região Central (ou seja, província de Henan, devido à sua localização central no país)}
\end{entry}

\begin{entry}{终点}{zhong1dian3}{8,9}{⽷、⽕}[HSK 5]
  \definition[个]{s.}{destino; ponto terminal; ponto de chegada; lugar onde uma jornada termina | final; refere-se especificamente ao local onde a corrida é interrompida}
\end{entry}

\begin{entry}{终究}{zhong1jiu1}{8,7}{⽷、⽳}
  \definition{adv.}{afinal de contas; enfatiza que, não importa o que aconteça, a natureza das pessoas e das coisas não mudará e que as características básicas devem ser reconhecidas (tem o efeito de fortalecer o tom) |  no final; indica que um determinado resultado ocorrerá ou não, frequentemente usado em especulações, julgamentos etc. | afinal de contas; indica que, apesar do grande esforço ou da grande esperança, o resultado objetivo ainda é insatisfatório, geralmente com o significado de pesar ou pena | afinal de contas; indica que um resultado desejado finalmente aparece}
\end{entry}

\begin{entry}{终身}{zhong1shen1}{8,7}{⽷、⾝}[HSK 5]
  \definition{s.}{vida inteira; por toda a vida; por toda a vida}
\end{entry}

\begin{entry}{终于}{zhong1yu2}{8,3}{⽷、⼆}[HSK 3]
  \definition{adv.}{finalmente; por fim; eventualmente; no final; indica uma situação que surge após várias mudanças ou espera}
\end{entry}

\begin{entry}{终止}{zhong1zhi3}{8,4}{⽷、⽌}[HSK 5]
  \definition{v.}{parar; terminar | anular; encerrar; expirar; revogar}
\end{entry}

\begin{entry}{钟}{zhong1}{9}{⾦}[HSK 3]
  \definition*{s.}{sobrenome Zhong}
  \definition[顶,个,口]{s.}{sino; campainha; um instrumento de percussão antigo, oco, feito de cobre ou ferro | relógio; um aparelho para medir o tempo que não se leva consigo | tempo medido em horas e minutos; referindo-se ao tempo ou momento| um recipiente antigo para guardar vinho, com barriga grande e gargalo pequeno | sino; refere-se especificamente aos sinos pendurados em templos ou outros locais, cujo som é usado para marcar as horas, alertar ou convocar pessoas}
  \definition{v.}{focar; concentrar (as afeições de alguém, etc.)}
\end{entry}

\begin{entry}{钟室}{zhong1shi4}{9,9}{⾦、⼧}
  \definition{s.}{campanário | sala do relógio}
\end{entry}

\begin{entry}{钟罩}{zhong1zhao4}{9,13}{⾦、⽹}
  \definition{s.}{redoma | dossel de sino}
\end{entry}

\begin{entry}{锺}{zhong1}{14}{⾦}
  \variantof{钟}
\end{entry}

\begin{entry}{种}{zhong3}{9}{⽲}[HSK 3,4]
  \definition{clas.}{indica tipo, usado para pessoas e qualquer coisa}
  \definition{s.}{espécie | etnia | semente; estirpe; linhagem; material para reprodução biológica em cadeia | coragem; determinação; garra; força de caráter; refere-se à coragem ou determinação}
  \seeref{种}{zhong4}
\end{entry}

\begin{entry}{种类}{zhong3lei4}{9,9}{⽲、⽶}[HSK 4]
  \definition{s.}{espécie; classe; tipo; variedade; categoria; classificação de alguma coisa de acordo com sua natureza e características}
\end{entry}

\begin{entry}{种麻}{zhong3ma2}{9,11}{⽲、⿇}
  \definition{s.}{planta de cânhamo (feminina)}
\end{entry}

\begin{entry}{种薯}{zhong3shu3}{9,16}{⽲、⾋}
  \definition{s.}{tubérculo semente}
\end{entry}

\begin{entry}{种种}{zhong3zhong3}{9,9}{⽲、⽲}
  \definition{adj.}{todos os tipos de}
\end{entry}

\begin{entry}{种子}{zhong3zi5}{9,3}{⽲、⼦}[HSK 3]
  \definition[颗,粒,个,号]{s.}{semente; um órgão exclusivo de certas plantas, geralmente composto de três partes: tegumento, embrião e endosperma, as sementes podem germinar e se tornar novas plantas sob certas condições | jogador classificado; durante a competição, nas eliminatórias, os jogadores mais fortes de cada equipe são escalados}
\end{entry}

\begin{entry}{种族灭绝}{zhong3zu2mie4jue2}{9,11,5,9}{⽲、⽅、⽕、⽷}
  \definition{s.}{genocídio | extinção étnica}
\end{entry}

\begin{entry}{中}{zhong4}{4}{⼁}
  \definition{v.}{acertar | encaixar exatamente |ser atingido por | cair em | ser afetado por | sofrer | sustentar}
  \seeref{中}{zhong1}
\end{entry}

\begin{entry}{中毒}{zhong4 du2}{4,9}{⼁、⽏}[HSK 5]
  \definition{v.}{envenenar; intoxicar | ser envenenado; ideia de que o pensamento foi contaminado}
\end{entry}

\begin{entry}{中奖}{zhong4 jiang3}{4,9}{⼁、⼤}[HSK 4]
  \definition{v.}{ganhar um prêmio (em uma loteria, etc.)}
\end{entry}

\begin{entry}{中意}{zhong4yi4}{4,13}{⼁、⼼}
  \definition{s.}{ser do seu agrado | começar a gostar muito de algo ou de alguém}
\end{entry}

\begin{entry}{众}{zhong4}{6}{⼈}
  \definition*{s.}{Câmara dos Deputados, abreviação de 众议院}
  \definition{adj.}{numeroso}
  \definition{adv.}{muitos}
  \definition{s.}{multidão}
  \seealsoref{众议院}{zhong4yi4yuan4}
\end{entry}

\begin{entry}{众多}{zhong4 duo1}{6,6}{⼈、⼣}[HSK 5]
  \definition{adj.}{muitos; numerosos; multitudinários}
\end{entry}

\begin{entry}{众议院}{zhong4yi4yuan4}{6,5,9}{⼈、⾔、⾩}
  \definition*{s.}{Casa baixa da Assembléia Bicameral | Câmara dos Deputados}
\end{entry}

\begin{entry}{种}{zhong4}{9}{⽲}
  \definition{v.}{semear; cultivar; plantar}
  \seeref{种}{zhong3}
\end{entry}

\begin{entry}{种地}{zhong4di4}{9,6}{⽲、⼟}
  \definition{v.}{cultivar | trabalhar a terra}
\end{entry}

\begin{entry}{种植}{zhong4zhi2}{9,12}{⽲、⽊}[HSK 4]
  \definition{v.}{plantar; crescer; cultivar; enterrar as sementes de uma planta no solo; plantar as mudas de uma planta no solo}
\end{entry}

\begin{entry}{重}{zhong4}{9}{⾥}[HSK 1,3]
  \definition{adj.}{pesado; densidade elevada | profundo; sério; grau profundo | importante; significativo | discreto; prudente | considerável em quantidade ou valor}
  \definition[斤,公,斤,吨]{s.}{peso}
  \definition{v.}{colocar (colocar, pôr) ênfase em; dar valor a; atribuir importância a}
  \seeref{重}{chong2}
\end{entry}

\begin{entry}{重大}{zhong4da4}{9,3}{⾥、⼤}[HSK 3]
  \definition{adj.}{excelente; importante; significativo; de grande importância}
\end{entry}

\begin{entry}{重点}{zhong4dian3}{9,9}{⾥、⽕}[HSK 2]
  \definition[个]{s.}{nota principal; ponto-chave; ponto}
  \seeref{重点}{chong2dian3}
\end{entry}

\begin{entry}{重量}{zhong4liang4}{9,12}{⾥、⾥}[HSK 4]
  \definition[个]{s.}{peso; a magnitude da força da gravidade em um objeto}
\end{entry}

\begin{entry}{重视}{zhong4shi4}{9,8}{⾥、⾒}[HSK 2]
  \definition{v.}{valorizar; dar peso a; atribuir importância a; prestar atenção a; considerar a virtude ou o talento de uma pessoa ou o papel de algo como importante e levá-lo a sério}
\end{entry}

\begin{entry}{重要}{zhong4yao4}{9,9}{⾥、⾑}[HSK 1]
  \definition{adj.}{importante; significativo; relevante; de grande importância, função e impacto}
\end{entry}

\begin{entry}{重重}{zhong4zhong4}{9,9}{⾥、⾥}
  \definition{adv.}{fortemente | severamente}
  \seeref{重重}{chong2chong2}
\end{entry}

\begin{entry}{周}{zhou1}{8}{⼝}[HSK 2]
  \definition*{s.}{sobrenome Zhou}
  \definition*{s.}{Dinastia Zhou (1046-256 BC) | Dinastia Zhou do Norte (557-581), uma das Dinastias do Norte |  A Dinastia Zhou Posterior (951-960), uma das Cinco Dinastias}
  \definition{adj.}{universal; inteiro; por toda parte | atencioso; pensativo; completo; minucioso}
  \definition{adv.}{semanalmente}
  \definition{clas.}{usado para rodadas, voltas}
  \definition{s.}{periferia; arredores; círculo | semana | ciclo}
  \definition{v.}{fazer um circuito; mover-se em um curso circular | ajudar alguém}
\end{entry}

\begin{entry}{周末}{zhou1mo4}{8,5}{⼝、⽊}[HSK 2]
  \definition[个]{s.}{final-de-semana}
\end{entry}

\begin{entry}{周年}{zhou1nian2}{8,6}{⼝、⼲}[HSK 2]
  \definition{s.}{aniversário}
\end{entry}

\begin{entry}{周期}{zhou1qi1}{8,12}{⼝、⽉}[HSK 5]
  \definition{s.}{período; ciclo; no processo de mudança e movimento das coisas, certas características se repetem várias vezes, com um intervalo de tempo entre cada repetição | período; ciclo; refere-se a um processo em que certas características se repetem várias vezes, e o tempo decorrido entre duas ocorrências consecutivas | classificação dos elementos na tabela periódica}
\end{entry}

\begin{entry}{周围}{zhou1wei2}{8,7}{⼝、⼞}[HSK 3]
  \definition{s.}{ao redor; redondeza; vizinhança; a parte ao redor do centro}
\end{entry}

\begin{entry}{洲}{zhou1}{9}{⽔}
  \definition{s.}{continente | ilha em um rio}
\end{entry}

\begin{entry}{轴承}{zhou2cheng2}{9,8}{⾞、⼿}
  \definition{s.}{(mecânico) rolamento}
\end{entry}

\begin{entry}{咒骂}{zhou4ma4}{8,9}{⼝、⾺}
  \definition{v.}{xingar | amaldiçoar | execrar}
\end{entry}

\begin{entry}{昼}{zhou4}{9}{⽇}
  \definition*{s.}{sobrenome Zhou}
  \definition{s.}{diurno; luz do dia; dia (oposição à 夜) | dia; o período do amanhecer ao anoitecer; diurno}
  \seealsoref{夜}{ye4}
\end{entry}

\begin{entry}{珠子}{zhu1zi5}{10,3}{⽟、⼦}
  \definition[粒,颗]{s.}{pérola | contas}
\end{entry}

\begin{entry}{猪}{zhu1}{11}{⽝}[HSK 3]
  \definition[头,只,口]{s.}{porco; suíno}
\end{entry}

\begin{entry}{猪窠}{zhu1ke1}{11,13}{⽝、⽳}
  \definition{s.}{chiqueiro}
\end{entry}

\begin{entry}{猪柳}{zhu1liu3}{11,9}{⽝、⽊}
  \definition{s.}{filé de porco}
\end{entry}

\begin{entry}{猪笼}{zhu1long2}{11,11}{⽝、⽵}
  \definition{s.}{estrutura cilíndrica de bambu ou arame usada para restringir um porco durante o transporte}
\end{entry}

\begin{entry}{猪头}{zhu1tou2}{11,5}{⽝、⼤}
  \definition{s.}{tolo | idiota}
\end{entry}

\begin{entry}{竹编}{zhu2bian1}{6,12}{⽵、⽷}
  \definition{s.}{vime | tecelagem de bambu}
\end{entry}

\begin{entry}{竹马}{zhu2ma3}{6,3}{⽵、⾺}
  \definition{s.}{cavalo de bambu | vara de bambu usada como cavalo de brinquedo}
\end{entry}

\begin{entry}{竹排}{zhu2pai2}{6,11}{⽵、⼿}
  \definition{s.}{jangada de bambu}
\end{entry}

\begin{entry}{竹子}{zhu2zi5}{6,3}{⽵、⼦}[HSK 5]
  \definition[块,株]{s.}{bambu; nome genérico para os tipos de bambu}
\end{entry}

\begin{entry}{逐步}{zhu2bu4}{10,7}{⾡、⽌}[HSK 4]
  \definition{adv.}{gradualmente; passo a passo; progressivamente}
\end{entry}

\begin{entry}{逐渐}{zhu2jian4}{10,11}{⾡、⽔}[HSK 4]
  \definition{adv.}{gradualmente; aos poucos; por etapas; indica mudanças lentas e ordenadas no grau, na quantidade, etc.}
\end{entry}

\begin{entry}{主办}{zhu3ban4}{5,4}{⼂、⼒}[HSK 5]
  \definition{v.}{manter; hospedar; dirigir; patrocinar}
\end{entry}

\begin{entry}{主持}{zhu3chi2}{5,9}{⼂、⼿}[HSK 3]
  \definition[位,名]{s.}{anfitrião; a pessoa responsável por administrar e lidar com uma determinada atividade}
  \definition{v.}{dirigir; administrar; assumir o comando; encarregar-se de; ser responsável por gerenciar, organizar uma determinada atividade ou lidar com um determinado assunto | defender; apoiar; preservar; manter}
\end{entry}

\begin{entry}{主导}{zhu3dao3}{5,6}{⼂、⼨}[HSK 5]
  \definition{adj.}{líder; dominante; guiado; principais e guias para que as coisas se desenvolvam em uma determinada direção}
  \definition{s.}{fator principal (ou orientador)}
\end{entry}

\begin{entry}{主动}{zhu3dong4}{5,6}{⼂、⼒}[HSK 3]
  \definition{adj.}{ativo; positivo; agir sem esperar por um impulso externo (em oposição a 被动) | iniciativo; capaz de impulsionar as coisas por vontade própria; capaz de criar uma situação favorável e fazer as coisas acontecerem de acordo com suas próprias intenções (em oposição a 被动)}
  \seealsoref{被动}{bei4dong4}
\end{entry}

\begin{entry}{主观}{zhu3guan1}{5,6}{⼂、⾒}[HSK 5]
  \definition{adj.}{subjetivo; não com base nas condições reais, mas com base nos próprios desejos | subjetivo; filosoficamente, refere-se à consciência e aos aspectos espirituais dos seres humanos}
\end{entry}

\begin{entry}{主管}{zhu3guan3}{5,14}{⼂、⽵}[HSK 5]
  \definition[门]{s.}{pessoa responsável, como supervisor, gerente, diretor, etc.}
  \definition{v.}{estar encarregado de; ser responsável por; ser o principal responsável pela gestão de um trabalho; assumir a responsabilidade primária pela gestão (um certo aspecto)}
\end{entry}

\begin{entry}{主人}{zhu3ren2}{5,2}{⼂、⼈}[HSK 2]
  \definition[个,位]{s.}{mestre; uma pessoa que empregava tutores, contadores, etc. antigamente; uma pessoa que empregava empregados domésticos | anfitrião; Aaguém que entretém convidados (em oposição a 客人) | proprietário; uma pessoa que possui um certo tipo de bens ou poder}
  \seealsoref{客人}{ke4ren2}
\end{entry}

\begin{entry}{主任}{zhu3ren4}{5,6}{⼂、⼈}[HSK 3]
  \definition[个,位,名]{s.}{chefe; diretor; presidente; o principal responsável por um departamento ou instituição}
\end{entry}

\begin{entry}{主题}{zhu3ti2}{5,15}{⼂、⾴}[HSK 4]
  \definition[个]{s.}{tema; assunto; motivo; lema; ideias básicas expressas em toda a obra de literatura e arte por meio de imagens artísticas concretas | pontos/conteúdos principais; referência geral ao conteúdo principal de artigos, discursos, conferências, etc.}
\end{entry}

\begin{entry}{主体}{zhu3 ti3}{5,7}{⼂、⼈}[HSK 5]
  \definition{s.}{corpo principal; parte principal; parte principal; esteio; a parte principal das coisas | (filosofia) sujeito}
\end{entry}

\begin{entry}{主席}{zhu3xi2}{5,10}{⼂、⼱}[HSK 4]
  \definition*[个,位]{s.}{Presidente (da China)}
  \definition[个,位]{s.}{presidente, \emph{chairman} (de uma reunião) | chefe; presidente (de uma organização ou estado)}
\end{entry}

\begin{entry}{主席台}{zhu3xi2tai2}{5,10,5}{⼂、⼱、⼝}
  \definition[个]{s.}{plataforma | tribuna}
\end{entry}

\begin{entry}{主席团}{zhu3xi2tuan2}{5,10,6}{⼂、⼱、⼞}
  \definition{s.}{presídio}
\end{entry}

\begin{entry}{主要}{zhu3yao4}{5,9}{⼂、⾑}[HSK 2]
  \definition{adj.}{principal; chefe; o mais importante na questão; o decisivo | principal; núcleo; a raiz ou parte mais importante de algo}
\end{entry}

\begin{entry}{主义}{zhu3yi4}{5,3}{⼂、⼂}
  \definition{s.}{ideologia}
  \definition{suf.}{-ismo}
\end{entry}

\begin{entry}{主意}{zhu3yi5}{5,13}{⼂、⼼}[HSK 3]
  \definition[个,种]{s.}{ideia; plano; decisão; método}
\end{entry}

\begin{entry}{主张}{zhu3zhang1}{5,7}{⼂、⼸}[HSK 3]
  \definition[个,项,些,种]{s.}{vista; posição; proposição}
  \definition{v.}{defender; apoiar; manter; representar; ter uma opinião sobre como agir, fazer uma sugestão}
\end{entry}

\begin{entry}{属}{zhu3}{12}{⼫}
  \definition{v.}{juntar; combinar | fixar (a mente) em; centrar (a atenção, etc.) em}
  \seeref{属}{shu3}
\end{entry}

\begin{entry}{嘱}{zhu3}{15}{⼝}
  \definition{v.}{juntar-se | implorar | incitar}
\end{entry}

\begin{entry}{嘱咐}{zhu3fu5}{15,8}{⼝、⼝}
  \definition{v.}{ordenar | dizer | exortar}
\end{entry}

\begin{entry}{嘱托}{zhu3tuo1}{15,6}{⼝、⼿}
  \definition{v.}{confiar uma tarefa a alguém}
\end{entry}

\begin{entry}{住}{zhu4}{7}{⼈}[HSK 1]
  \definition{adv.}{firmemente; indica estabilidade ou firmeza}
  \definition{v.}{viver; residir; morar; ficar | parar; cessar | (após um verbo) com firmeza; até parar | hospedar; acomodar | parar; interromper | ser competente; ser qualificado; estar à altura; usado com 得 ou 不, indica que a força é suficiente (ou insuficiente)}
  \seealsoref{不}{bu4}
  \seealsoref{得}{de5}
\end{entry}

\begin{entry}{住处}{zhu4chu4}{7,5}{⼈、⼡}
  \definition{s.}{morada | habitação | residência}
\end{entry}

\begin{entry}{住房}{zhu4fang2}{7,8}{⼈、⼾}[HSK 2]
  \definition[套,处]{s.}{habitação; alojamento; casas para as pessoas morarem}
\end{entry}

\begin{entry}{住所}{zhu4suo3}{7,8}{⼈、⼾}
  \definition[处]{s.}{morada | habitação | residência}
\end{entry}

\begin{entry}{住院}{zhu4 yuan4}{7,9}{⼈、⾩}[HSK 2]
  \definition{v.}{estar hospitalizado; estar no hospital; ser internado no hospital para tratamento}
\end{entry}

\begin{entry}{住宅}{zhu4zhai2}{7,6}{⼈、⼧}
  \definition{s.}{residência}
\end{entry}

\begin{entry}{住嘴}{zhu4zui3}{7,16}{⼈、⼝}
  \definition{interj.}{Cale-se!}
  \definition{v.}{calar | calar-se}
\end{entry}

\begin{entry}{助理}{zhu4li3}{7,11}{⼒、⽟}[HSK 5]
  \definition[个,名,位]{s.}{deputado; assistente; auxiliar do diretor responsável (geralmente usado em cargos) | ajudante; assistente; pessoa que auxilia o responsável a fazer as coisas}
\end{entry}

\begin{entry}{助手}{zhu4shou3}{7,4}{⼒、⼿}[HSK 5]
  \definition[个]{s.}{ajudante; auxiliar; assistente; alguém que ajuda os outros com seu trabalho}
\end{entry}

\begin{entry}{助兴}{zhu4xing4}{7,6}{⼒、⼋}
  \definition{v.+compl.}{animar as coisas | juntar-se à diversão}
\end{entry}

\begin{entry}{注}{zhu4}{8}{⽔}
  \definition{s.}{apostas (em jogos de azar) | notas (em um texto)}
  \definition{v.}{derramar; encher | concentrar-se em; fixar-se em; focar em  | anotar; explicar com notas | registrar; gravar | irrigar | dar exegese ou explicação}
\end{entry}

\begin{entry}{注册}{zhu4ce4}{8,5}{⽔、⼌}[HSK 5]
  \definition{v.}{inscrever-se; matricular-se; registrar-se; registrar-se junto à autoridade ou escola competente para obter status legal; refere-se especificamente ao usuário de uma determinada rede de computadores que insere o nome de usuário, senha, etc. na rede para obter permissão para usar a rede}
\end{entry}

\begin{entry}{注册表}{zhu4ce4biao3}{8,5,8}{⽔、⼌、⾐}
  \definition*{s.}{Registro do \emph{Windows}}
\end{entry}

\begin{entry}{注册人}{zhu4ce4ren2}{8,5,2}{⽔、⼌、⼈}
  \definition{s.}{registrante}
\end{entry}

\begin{entry}{注册商标}{zhu4ce4shang1biao1}{8,5,11,9}{⽔、⼌、⼝、⽊}
  \definition{s.}{marca registrada}
\end{entry}

\begin{entry}{注射}{zhu4she4}{8,10}{⽔、⼨}[HSK 5]
  \definition{v.}{injetar; usar uma seringa para administrar medicamento líquido em um organismo}
\end{entry}

\begin{entry}{注视}{zhu4shi4}{8,8}{⽔、⾒}[HSK 5]
  \definition{v.}{olhar atentamente para; observar atentamente}
\end{entry}

\begin{entry}{注意}{zhu4yi4}{8,13}{⽔、⼼}[HSK 3]
  \definition{v.}{prestar atenção; notar; ficar de olho; concentrar os pensamentos em um aspecto específico}
\end{entry}

\begin{entry}{注意地}{zhu4yi4di4}{8,13,6}{⽔、⼼、⼟}
  \definition{s.}{área de cuidado, de observação}
\end{entry}

\begin{entry}{注意力}{zhu4yi4li4}{8,13,2}{⽔、⼼、⼒}
  \definition{s.}{atenção}
\end{entry}

\begin{entry}{注意力缺失症}{zhu4yi4li4que1shi1zheng4}{8,13,2,10,5,10}{⽔、⼼、⼒、⽸、⼤、⽧}
  \definition{s.}{transtorno de déficit de atenção}
\end{entry}

\begin{entry}{注重}{zhu4zhong4}{8,9}{⽔、⾥}[HSK 5]
  \definition{v.}{enfatizar; dar ênfase a; dar ênfase a; prestar atenção a; dar importância a}
\end{entry}

\begin{entry}{驻军}{zhu4jun1}{8,6}{⾺、⼍}
  \definition{s.}{guarnição}
  \definition{v.}{guarcener ou prover uma tropa}
\end{entry}

\begin{entry}{祝}{zhu4}{9}{⽰}[HSK 3]
  \definition*{s.}{sobrenome Zhu}
  \definition{v.}{expressar bons votos; desejar; abençoar | rezar aos deuses ou espíritos para obter bênçãos}
\end{entry}

\begin{entry}{祝祷}{zhu4dao3}{9,11}{⽰、⽰}
  \definition{v.}{rezar | orar}
\end{entry}

\begin{entry}{祝福}{zhu4fu2}{9,13}{⽰、⽰}[HSK 4]
  \definition[个]{s.}{bênção; benzedura; benzimento; originalmente, referia-se à oração para obter as bênçãos de Deus, mas, mais tarde, refere-se a desejar paz e felicidade às pessoas}
  \definition{v.}{desejar boa sorte a alguém}
\end{entry}

\begin{entry}{祝好}{zhu4hao3}{9,6}{⽰、⼥}
  \definition{expr.}{desejo-lhe tudo de melhor! (ao encerrar uma correspondência)}
\end{entry}

\begin{entry}{祝贺}{zhu4he4}{9,9}{⽰、⾙}[HSK 5]
  \definition[个]{s.}{congratulações; felicitações}
  \definition{v.}{congratular; felicitar; parabenizar}
\end{entry}

\begin{entry}{祝酒}{zhu4jiu3}{9,10}{⽰、⾣}
  \definition{v.}{parabenizar e fazer um brinde | brindar}
\end{entry}

\begin{entry}{祝寿}{zhu4shou4}{9,7}{⽰、⼨}
  \definition{v.}{dar parabéns pelo aniversário (a uma pessoa idosa)}
\end{entry}

\begin{entry}{祝颂}{zhu4song4}{9,10}{⽰、⾴}
  \definition{v.}{expressar bons desejos}
\end{entry}

\begin{entry}{祝谢}{zhu4xie4}{9,12}{⽰、⾔}
  \definition{v.}{agradecer | dar parabéns}
\end{entry}

\begin{entry}{祝愿}{zhu4yuan4}{9,14}{⽰、⽕}
  \definition{v.}{desejar}
\end{entry}

\begin{entry}{著名}{zhu4ming2}{11,6}{⽬、⼝}[HSK 4]
  \definition{adj.}{famoso; bem conhecido; célebre}
\end{entry}

\begin{entry}{著作}{zhu4zuo4}{11,7}{⽬、⼈}[HSK 4]
  \definition[部]{s.}{obra; livro; escritos}
  \definition{v.}{escrever; usar palavras para expressar opiniões, conhecimentos, ideias, sentimentos, etc.}
\end{entry}

\begin{entry}{抓}{zhua1}{7}{⼿}[HSK 3]
  \definition{v.}{agarrar; segurar; obter; apreender; juntar os dedos para segurar o objeto na mão | riscar; arranhar; usar as unhas, objetos com dentes ou garras de animais para riscar a superfície de um objeto | apanhar; capturar; controlar pessoas ou animais; fazer com que pessoas ou animais caiam nas mãos de alguém | compreender; saber onde está o ponto principal ou a chave de uma questão ou problema | concentrar-se em algo; reforçar a força para fazer (alguma coisa), controlar (algum aspecto) | chamar a atenção de alguém; atrair a atenção}
\end{entry}

\begin{entry}{抓紧}{zhua1jin3}{7,10}{⼿、⽷}[HSK 4]
  \definition{v.}{agarrar com firmeza; segurar firme e não soltar | prestar muita atenção a}
\end{entry}

\begin{entry}{抓住}{zhua1 zhu4}{7,7}{⼿、⼈}[HSK 3]
  \definition{v.}{prender; deter; capturar (pessoas ou animais) e ter sucesso | segurar; agarrar; apreender; agarrar algo para que não se mova}
\end{entry}

\begin{entry}{转}{zhuai3}{8}{⾞}
  \seeref{转}{zhuan3}
  \seeref{转}{zhuan4}
\end{entry}

\begin{entry}{专辑}{zhuan1 ji2}{4,13}{⼀、⾞}[HSK 5]
  \definition[张]{s.}{álbum (música) | registro (música) | coleção especial de material impresso ou transmitido}
\end{entry}

\begin{entry}{专家}{zhuan1jia1}{4,10}{⼀、⼧}[HSK 3]
  \definition[个,位]{s.}{perito; especialista; profissional; pessoa que se dedica ao estudo aprofundado de uma determinada disciplina; pessoa especializada em uma determinada técnica}
\end{entry}

\begin{entry}{专利}{zhuan1li4}{4,7}{⼀、⼑}[HSK 5]
  \definition{s.}{patente; a garantia de que os criadores e inventores desfrutem exclusivamente dos benefícios decorrentes de suas criações e invenções durante um determinado período | direitos de patente; referência à patente}
\end{entry}

\begin{entry}{专门}{zhuan1men2}{4,3}{⼀、⾨}[HSK 3]
  \definition{adj.}{especializado; dedicar-se exclusivamente a uma determinada tarefa; expressa ênfase em fazer frequentemente um determinado tipo de coisa}
  \definition{adv.}{especialmente}
\end{entry}

\begin{entry}{专题}{zhuan1ti2}{4,15}{⼀、⾴}[HSK 3]
  \definition[个,些,种]{s.}{assunto especial; tópico especial; questões específicas}
\end{entry}

\begin{entry}{专心}{zhuan1xin1}{4,4}{⼀、⼼}[HSK 4]
  \definition{adj.}{absorto; concentrado}
\end{entry}

\begin{entry}{专业}{zhuan1ye4}{4,5}{⼀、⼀}[HSK 3]
  \definition{adj.}{profissional; descreve uma pessoa que possui um alto nível ou conhecimento profundo em determinada área}
  \definition[个,门]{s.}{profissão; área específica; comércio especializado; departamentos operacionais da divisão de produção | especialidade; disciplina; matéria especializada; área de estudo específica; em um departamento de uma instituição de ensino superior ou em uma escola profissionalizante de nível médio}
\end{entry}

\begin{entry}{专业户}{zhuan1ye4hu4}{4,5,4}{⼀、⼀、⼾}
  \definition{s.}{indústria caseira | empresa familiar produzindo um produto especial}
\end{entry}

\begin{entry}{专业化}{zhuan1ye4hua4}{4,5,4}{⼀、⼀、⼔}
  \definition{s.}{especialização}
\end{entry}

\begin{entry}{专业教育}{zhuan1ye4jiao4yu4}{4,5,11,8}{⼀、⼀、⽁、⾁}
  \definition{s.}{educação especializada | escola técnica}
\end{entry}

\begin{entry}{专业人才}{zhuan1ye4ren2cai2}{4,5,2,3}{⼀、⼀、⼈、⼿}
  \definition{s.}{especialista (em uma área)}
\end{entry}

\begin{entry}{专业人士}{zhuan1ye4ren2shi4}{4,5,2,3}{⼀、⼀、⼈、⼠}
  \definition{s.}{profissional}
\end{entry}

\begin{entry}{专业性}{zhuan1ye4xing4}{4,5,8}{⼀、⼀、⼼}
  \definition{s.}{profissionalismo | expertise}
\end{entry}

\begin{entry}{砖}{zhuan1}{9}{⽯}
  \definition[块]{s.}{tijolo}
\end{entry}

\begin{entry}{转}{zhuan3}{8}{⾞}
  \definition{v.}{mudar; deslocar; transferir; virar; mudar de direção, posição, situação, circunstâncias, etc. | transmitir; transferir; passar adiante}
  \seeref{转}{zhuai3}
  \seeref{转}{zhuan4}
\end{entry}

\begin{entry}{转变}{zhuan3bian4}{8,8}{⾞、⼜}[HSK 3]
  \definition{v.}{mudar; converter; transformar}
\end{entry}

\begin{entry}{转产}{zhuan3chan3}{8,6}{⾞、⼇}
  \definition{v.}{mudar a produção | mudar para novos produtos}
\end{entry}

\begin{entry}{转递}{zhuan3di4}{8,10}{⾞、⾡}
  \definition{v.}{passar | retransmitir}
\end{entry}

\begin{entry}{转动}{zhuan3 dong4}{8,6}{⾞、⼒}[HSK 4]
  \definition{v.}{girar; rodar; dar voltas; torcer | dar a volta em algo}
  \seeref{转动}{zhuan4 dong4}
\end{entry}

\begin{entry}{转告}{zhuan3gao4}{8,7}{⾞、⼝}[HSK 4]
  \definition{v.}{passar adiante; comunicar; transmitir; ser instruído a dizer a outra parte o que uma pessoa diz, o que está acontecendo, etc.}
\end{entry}

\begin{entry}{转化}{zhuan3 hua4}{8,4}{⾞、⼔}[HSK 5]
  \definition{v.}{mudar; transformar | inverter; converter}
\end{entry}

\begin{entry}{转换}{zhuan3 huan4}{8,10}{⾞、⼿}[HSK 5]
  \definition{v.}{mudar; trocar; converter; transformar; alterar}
\end{entry}

\begin{entry}{转念}{zhuan3nian4}{8,8}{⾞、⼼}
  \definition{v.}{ter dúvidas sobre algo | pensar melhor}
\end{entry}

\begin{entry}{转让}{zhuan3rang4}{8,5}{⾞、⾔}[HSK 5]
  \definition{v.}{ceder; fazer a entrega; transferir a posse de; ceder seus bens ou direitos a outra pessoa}
\end{entry}

\begin{entry}{转身}{zhuan3 shen1}{8,7}{⾞、⾝}[HSK 4]
  \definition{adv.}{em um instante; em um piscar de olhos}
  \definition{v.}{dar a volta; dar meia-volta; dar a volta por cima | virar; girar; refere-se a uma mudança de direção, localização, natureza, etc.}
\end{entry}

\begin{entry}{转向}{zhuan3 xiang4}{8,6}{⾞、⼝}[HSK 5]
  \definition{v.}{desviar; desviar-se; mudar a direção do avanço | mudar a posição política de alguém | mudar de direção; virar-se para (a outra parte)}
  \seeref{转向}{zhuan4 xiang4}
\end{entry}

\begin{entry}{转移}{zhuan3yi2}{8,11}{⾞、⽲}[HSK 4]
  \definition{v.}{deslocar; desviar; transferir; redirecionar; reposicionar; reorientar | mudar; transformar}
\end{entry}

\begin{entry}{转账}{zhuan3zhang4}{8,8}{⾞、⾙}
  \definition{v.+compl.}{transferir entre contas | trazer à frente | incluir uma soma de dinheiro do balanço anterior no seguinte}
\end{entry}

\begin{entry}{传}{zhuan4}{6}{⼈}
  \definition{s.}{comentários sobre clássicos; obras que explicam as escrituras| biografia | romances sobre eventos históricos; obras que narram histórias históricas}
  \seeref{传}{chuan2}
\end{entry}

\begin{entry}{转}{zhuan4}{8}{⾞}[HSK 3]
  \definition{clas.}{usado para rotações (por minuto, por segundo, etc.): RPM}
  \definition{v.}{girar; rodar; revolver; movimento em torno de um centro | passear; dar uma volta}
  \seeref{转}{zhuai3}
  \seeref{转}{zhuan3}
\end{entry}

\begin{entry}{转动}{zhuan4 dong4}{8,6}{⾞、⼒}[HSK 4]
  \definition{v.}{girar; correr; rolar; revolver; rotacionar; torcer}
  \seeref{转动}{zhuan3 dong4}
\end{entry}

\begin{entry}{转弯}{zhuan4 wan1}{8,9}{⾞、⼸}[HSK 4]
  \definition{v.}{rodar; desviar; virar uma esquina; fazer uma curva; fazer uma curva de 180º}
\end{entry}

\begin{entry}{转向}{zhuan4 xiang4}{8,6}{⾞、⼝}
  \definition{v.+compl.}{perder-se; perder o rumo; não consiguir distinguir a direção; estar perdido}
  \seeref{转向}{zhuan3 xiang4}
\end{entry}

\begin{entry}{转悠}{zhuan4you5}{8,11}{⾞、⼼}
  \definition{v.}{aparecer repetidamente | rolar | passear por aí}
\end{entry}

\begin{entry}{转游}{zhuan4you5}{8,12}{⾞、⽔}
  \variantof{转悠}
\end{entry}

\begin{entry}{妆}{zhuang1}{6}{⼥}
  \definition{s.}{maquiagem | adorno | enxoval | maquiagem e figurino de palco}
  \definition{v.}{maquiar-se | enfeitar-se}
\end{entry}

\begin{entry}{妆扮}{zhuang1ban4}{6,7}{⼥、⼿}
  \variantof{装扮}
\end{entry}

\begin{entry}{桩}{zhuang1}{10}{⽊}
  \definition{clas.}{para eventos, casos, transações, assuntos, etc.}
  \definition{s.}{toco | estaca | pilha}
\end{entry}

\begin{entry}{装}{zhuang1}{12}{⾐}[HSK 2]
  \definition*{s.}{sobrenome Zhuang}
  \definition{s.}{vestido; traje; vestimenta; roupa | maquiagem e figurino de palco; maquiagem de ator}
  \definition{v.}{enfeitar; adornar; vestir; decorar; vestir-se; vestir-se bem | fingir; fazer de conta | segurar; embalar; carregar; colocar as coisas em recipientes; colocar as coisas no transporte | encaixar; instalar; equipar; aparelhar; montar | embalar; encaixotar; embrulhar produtos ou colocá-los em caixas, garrafas, etc.}
\end{entry}

\begin{entry}{装扮}{zhuang1ban4}{12,7}{⾐、⼿}
  \definition{v.}{enfeitar | decorar | disfarçar-me | vestir-se}
\end{entry}

\begin{entry}{装备}{zhuang1bei4}{12,8}{⾐、⼡}
  \definition{s.}{equipamento}
  \definition{v.}{equipar}
\end{entry}

\begin{entry}{装配}{zhuang1pei4}{12,10}{⾐、⾣}
  \definition{v.}{montar | encaixar}
\end{entry}

\begin{entry}{装饰}{zhuang1shi4}{12,8}{⾐、⾷}[HSK 5]
  \definition[件,个]{s.}{decoração; acessórios decorativos}
  \definition{v.}{enfeitar; adornar; decorar; ornamentar; embelezar; destacar}
\end{entry}

\begin{entry}{装修}{zhuang1 xiu1}{12,9}{⾐、⼈}[HSK 4]
  \definition{v.}{equipar; renovar; decorar (equipar uma sala ou prédio com equipamentos ou decorações)}
\end{entry}

\begin{entry}{装置}{zhuang1 zhi4}{12,13}{⾐、⽹}[HSK 4]
  \definition{s.}{dispositivo; equipamento; máquinas, instrumentos ou outros equipamentos de construção mais complexa e com alguma função independente}
  \definition{v.}{instalar; ajustar; configurar; equipar; montar}
\end{entry}

\begin{entry}{状况}{zhuang4kuang4}{7,7}{⽝、⼎}[HSK 3]
  \definition[个,种]{s.}{estado; \emph{status}; situação; condição; estado de coisas; a aparência ou o estado em que as coisas se apresentam}
\end{entry}

\begin{entry}{状态}{zhuang4tai4}{7,8}{⽝、⼼}[HSK 3]
  \definition[种,个]{s.}{\emph{status}; estado; condição; situação; estado de coisas; a forma manifestada por pessoas ou coisas}
\end{entry}

\begin{entry}{撞}{zhuang4}{15}{⼿}[HSK 5]
  \definition{v.}{chocar-se contra; chocar-se com; bater; colidir | encontrar-se por acaso; esbarrar em; deparar-se com | apressar; correr; empurrar | aproveitar a chance | esbarrar de repente em |  encontrar | confiar em; tentar | agir precipitadamente; invadir}
\end{entry}

\begin{entry}{撞车}{zhuang4che1}{15,4}{⼿、⾞}
  \definition{v.+compl.}{(figurativo) colidir (opiniões, cronogramas, etc.) | ser o mesmo (assunto) | colidir (com outro veículo)}
\end{entry}

\begin{entry}{撞运气}{zhuang4yun4qi5}{15,7,4}{⼿、⾡、⽓}
  \definition{v.}{confiar no destino | tentar a sorte}
\end{entry}

\begin{entry}{追}{zhui1}{9}{⾡}[HSK 3]
  \definition{v.}{perseguir; correr atrás; seguir de perto | rastrear; investigar; chegar ao fundo de | procurar; ir atrás; esforçar-se para alcançar um determinado objetivo | recordar; relembrar | fazer depois do ocorrido; retrabalhar | cortejar (uma mulher)}
\end{entry}

\begin{entry}{追赶}{zhui1gan3}{9,10}{⾡、⾛}
  \definition{v.}{perseguir | acelerar | alcançar | ultrapassar}
\end{entry}

\begin{entry}{追求}{zhui1qiu2}{9,7}{⾡、⽔}[HSK 4]
  \definition{s.}{perseguição (ações e metas positivas)}[她的追求是获得成功。___Sua meta é alcançar o sucesso.]
  \definition{v.}{buscar; aspirar; perseguir | cortejar, uma referência especial ao namoro}
\end{entry}

\begin{entry}{坠}{zhui4}{7}{⼟}
  \definition{v.}{cair | pesar | fazer vergar com o peso}
\end{entry}

\begin{entry}{坠落}{zhui4luo4}{7,12}{⼟、⾋}
  \definition{v.}{cair}
\end{entry}

\begin{entry}{屯}{zhun1}{4}{⼬}
  \definition{adj.}{difícil; árduo;}
  \seeref{屯}{tun2}
\end{entry}

\begin{entry}{准}{zhun3}{10}{⼎}[HSK 3]
  \definition{adj.}{exato; preciso; algo determinado a ser imutável | preciso; exato; correto | perto; parcialmente; quase; próximo de algo em termos de padrão}
  \definition{adv.}{definitivamente; certamente}
  \definition{pref.}{quasi-; para-}
  \definition{prep.}{de acordo com; baseado em}
  \definition{s.}{norma; padrão; critério | confiança certa; uma ideia definida, certeza, etc. (geralmente usada depois de 有 ou 没有)}
  \definition{v.}{autorizar; conceder; consentir; permitir}
  \seealsoref{没有}{mei2 you3}
  \seealsoref{有}{you3}
\end{entry}

\begin{entry}{准备}{zhun3bei4}{10,8}{⼎、⼡}[HSK 1]
  \definition{v.}{preparar; ficar pronto; planejar ou organizar com antecedência | pretender; planejar}
\end{entry}

\begin{entry}{准确}{zhun3que4}{10,12}{⼎、⽯}[HSK 2]
  \definition{adj.}{exato; preciso; acurado; os resultados da ação são completamente consistentes com os resultados reais ou esperados}
\end{entry}

\begin{entry}{准时}{zhun3shi2}{10,7}{⼎、⽇}[HSK 4]
  \definition{adj.}{pontual}
  \definition{adv.}{na hora certa; dentro do prazo; no horário especificado}
\end{entry}

\begin{entry}{桌}{zhuo1}{10}{⽊}
  \definition{clas.}{para mesas de convidados em um banquete etc.}
  \definition{s.}{mesa}
\end{entry}

\begin{entry}{桌布}{zhuo1bu4}{10,5}{⽊、⼱}
  \definition[条,块,张]{s.}{(computação) plano de fundo da área de trabalho | toalha de mesa | papel de parede}
\end{entry}

\begin{entry}{桌灯}{zhuo1deng1}{10,6}{⽊、⽕}
  \definition{s.}{luminária | lâmpada de mesa}
\end{entry}

\begin{entry}{桌机}{zhuo1ji1}{10,6}{⽊、⽊}
  \definition{s.}{computador \emph{desktop}}
\end{entry}

\begin{entry}{桌面}{zhuo1mian4}{10,9}{⽊、⾯}
  \definition{s.}{área de trabalho | mesa}
\end{entry}

\begin{entry}{桌球}{zhuo1qiu2}{10,11}{⽊、⽟}
  \definition{s.}{bilhar | sinuca | mesa de ping-pong}
\end{entry}

\begin{entry}{桌游}{zhuo1you2}{10,12}{⽊、⽔}
  \definition{s.}{jogo de tabuleiro}
\end{entry}

\begin{entry}{桌子}{zhuo1zi5}{10,3}{⽊、⼦}[HSK 1]
  \definition[张,套]{s.}{mesa; escrivaninha; móveis, com uma superfície plana na parte superior e uma estrutura de suporte na parte inferior, para colocar objetos ou realizar atividades}
\end{entry}

\begin{entry}{棹}{zhuo1}{12}{⽊}
  \variantof{桌}
\end{entry}

\begin{entry}{着}{zhuo2}{11}{⽬}
  \definition{v.}{vestir (roupas); vestir-se | tocar; entrar em contato com; aproximar-se de; (contato físico) | enviar; despachar | expressão usada em documentos oficiais antigos, indicando um tom de ordem | aplicar; usar; adicionar; anexar}
  \seeref{着}{zhao1}
  \seeref{着}{zhao2}
  \seeref{着}{zhe5}
\end{entry}

\begin{entry}{着花}{zhuo2hua1}{11,7}{⽬、⾋}
  \definition{s.}{floração}
  \definition{v.}{florescer}
  \seeref{着花}{zhao2hua1}
\end{entry}

\begin{entry}{着手}{zhuo2shou3}{11,4}{⽬、⼿}
  \definition{v.}{colocar a mão nisso | estabelecer | começar uma tarefa}
\end{entry}

\begin{entry}{着想}{zhuo2xiang3}{11,13}{⽬、⼼}
  \definition{v.}{considerar (as necessidades de outras pessoas) | pensar (para os outros)}
\end{entry}

\begin{entry}{着眼}{zhuo2yan3}{11,11}{⽬、⽬}
  \definition{v.}{ter seus olhos em (um objetivo) | ter algo em mente | concentrar-se}
\end{entry}

\begin{entry}{着装}{zhuo2zhuang1}{11,12}{⽬、⾐}
  \definition{s.}{roupa | vestimenta}
  \definition{v.}{vestir}
\end{entry}

\begin{entry}{资}{zi1}{10}{⾙}
  \definition{s.}{recursos | capital | dinheiro | despesa}
  \definition{v.}{fornecer | suprir}
\end{entry}

\begin{entry}{资本}{zi1ben3}{10,5}{⾙、⽊}[HSK 5]
  \definition{s.}{capital; meios de produção ou moeda utilizados para fins lucrativos | o que é capitalizado; algo usado em benefício próprio; metáfora para obter benefícios}
\end{entry}

\begin{entry}{资产}{zi1chan3}{10,6}{⾙、⼇}[HSK 5]
  \definition{s.}{propriedade; bens; patrimônio | capital; fundo de capital; recursos financeiros da empresa | ativos; na contabilidade, refere-se à utilização de fundos}
\end{entry}

\begin{entry}{资格}{zi1ge2}{10,10}{⾙、⽊}[HSK 3]
  \definition{s.}{qualificação; condições e identidades necessárias para exercer uma determinada atividade | senioridade; identidade formada pelo tempo dedicado a um determinado trabalho ou atividade}
\end{entry}

\begin{entry}{资金}{zi1jin1}{10,8}{⾙、⾦}[HSK 3]
  \definition[笔]{s.}{fundo; capital; capital necessário para atividades comerciais, etc.}
\end{entry}

\begin{entry}{资料}{zi1liao4}{10,10}{⾙、⽃}[HSK 4]
  \definition[份,个]{s.}{dados; material; material informativo para referência ou para ser considerado confiável | material de produção; meios de subsistência; requisitos de produção ou subsistência}
\end{entry}

\begin{entry}{资源}{zi1yuan2}{10,13}{⾙、⽔}[HSK 4]
  \definition{s.}{recurso; fontes naturais de meios de produção ou subsistência}
\end{entry}

\begin{entry}{资助}{zi1zhu4}{10,7}{⾙、⼒}[HSK 5]
  \definition{s.}{subsídio}
  \definition{v.}{subsidiar; patrocinar; ajudar financeiramente; ajudar com recursos financeiros}
\end{entry}

\begin{entry}{子}{zi3}{3}{⼦}[Kangxi 39]
  \definition*{s.}{sobrenome Zi}
  \definition{adj.}{pequeno; jovem; tenro | subsidiário; subordinado; derivado}
  \definition{clas.}{usado para objetos finos que podem ser pinçados com os dedos}
  \definition{pron.}{você;  antigamente, era uma forma de tratamento respeitosa para se referir a outras pessoas, equivalente a 您}
  \definition[个,位,名]{s.}{filho, criança; antigamente, referia-se aos filhos, mas atualmente refere-se especificamente aos filhos homens | pessoa | antigo título de respeito para um homem culto ou virtuoso; na antiguidade, referia-se especificamente a homens eruditos | visconde; o quarto posto na hierarquia dos cinco títulos feudais da nobreza | ovo | semente | coisas pequenas e duras; pequenos fragmentos ou grãos duros e sólidos | cobre; moeda de cobre | o primeiro dos doze ramos terrestres}
  \seeref{子}{zi5}
  \seealsoref{您}{nin2}
\end{entry}

\begin{entry}{子弹}{zi3dan4}{3,11}{⼦、⼸}[HSK 5]
  \definition[粒,颗,发]{s.}{bala; cartucho; munição;}
\end{entry}

\begin{entry}{子女}{zi3 nv3}{3,3}{⼦、⼥}[HSK 3]
  \definition[个]{s.}{crianças; descendentes; filhos e filhas}
\end{entry}

\begin{entry}{仔细}{zi3xi4}{5,8}{⼈、⽷}[HSK 5]
  \definition{adj.}{cuidadoso; atencioso; descreve alguém que é cuidadoso e meticuloso ao fazer as coisas; não é descuidado | frugal; econômico; descreve o uso moderado de dinheiro ou bens, sem desperdício}
  \definition{v.}{ter cuidado; prestar atenção; ter muito cuidado e evitar que aconteçam coisas ruins}
\end{entry}

\begin{entry}{紫}{zi3}{12}{⽷}[HSK 5]
  \definition*{s.}{sobrenome Zi}
  \definition{adj.}{roxo; púrpura; violeta; cor resultante da combinação do vermelho e do azul}
\end{entry}

\begin{entry}{紫色}{zi3 se4}{12,6}{⽷、⾊}
  \definition{s.}{cor púrpura | cor violeta}
\end{entry}

\begin{entry}{字}{zi4}{6}{⼦}[HSK 1]
  \definition[个]{s.}{palavra; caractere; texto | pronúncia (de uma palavra ou caractere); som do caractere | tipo de impressão; estilo de caligrafia; forma de um caractere escrito ou impresso; refere-se às diferentes formas dos caracteres chineses; também se refere às diferentes escolas de caligrafia | escritas; obras de caligrafia | recibo; compromisso por escrito; documento | nome de estilo masculino adotado aos vinte anos de idade | sobrenome | um número indicado num contador elétrico, contador de água, etc.; registrar dos números dos medidores de consumo de água e eletricidade}
  \definition{v.}{ficar noiva (nos tempos antigos)}
\end{entry}

\begin{entry}{字典}{zi4 dian3}{6,8}{⼦、⼋}[HSK 2]
  \definition[本,册,部]{s.}{dicionário de caracteres chineses (contendo verbetes de caracteres únicos, em contraste com 词典 que contém verbetes para palavras com um ou mais caracteres)}
  \seealsoref{词典}{ci2dian3}
\end{entry}

\begin{entry}{字脚}{zi4jiao3}{6,11}{⼦、⾁}
  \definition[典]{s.}{gancho no final da pincelada | serifa}
\end{entry}

\begin{entry}{字母}{zi4mu3}{6,5}{⼦、⽏}[HSK 4]
  \definition[个]{s.}{letra; letras de um alfabeto | caractere que representa uma consoante inicial (em fonologia)}
\end{entry}

\begin{entry}{字眼}{zi4yan3}{6,11}{⼦、⽬}
  \definition[个]{s.}{palavras | redação}
\end{entry}

\begin{entry}{字字珠玉}{zi4zi4zhu1yu4}{6,6,10,5}{⼦、⼦、⽟、⽟}
  \definition{expr.}{cada palavra é uma jóia}
  \definition{s.}{escrita magnífica}
\end{entry}

\begin{entry}{自}{zi4}{6}{⾃}[HSK 4][Kangxi 132]
  \definition*{s.}{sobrenome Zi}
  \definition{adv.}{certamente; com certeza; é claro; naturalmente}
  \definition{prep.}{de; desde; a partir de; apresenta o ponto de partida, a fonte ou o horário de início do comportamento da ação, equivalente a 从 e 由}
  \definition{pron.}{si mesmo; próprio | próprio; indica que a ação é iniciada por e direcionada a si mesmo | por si mesmo; indica que a ação é autoiniciada e não é causada por uma força externa}
  \definition{v.}{iniciar}
  \seealsoref{从}{cong2}
  \seealsoref{由}{you2}
\end{entry}

\begin{entry}{自从}{zi4cong2}{6,4}{⾃、⼈}[HSK 3]
  \definition{prep.}{de; desde; a partir de; referir-se a um momento ou evento específico no passado}
\end{entry}

\begin{entry}{自动}{zi4dong4}{6,6}{⾃、⼒}[HSK 3]
  \definition{adj.}{automático; auto-atuante; uso de dispositivos mecânicos, elétricos, etc, para funcionar automaticamente, sem necessidade de controle humano}
  \definition{adv.}{voluntariamente; por vontade própria; por iniciativa própria | automaticamente; espontaneamente; refere-se a movimentos, mudanças, etc., que não são causados pela ação humana, mas sim pelo próprio objeto}
\end{entry}

\begin{entry}{自动化}{zi4dong4hua4}{6,6,4}{⾃、⼒、⼔}
  \definition{s.}{automação}
\end{entry}

\begin{entry}{自个儿}{zi4ge3r5}{6,3,2}{⾃、⼈、⼉}
  \definition{pron.}{(dialeto) a si mesmo, por si mesmo}
\end{entry}

\begin{entry}{自豪}{zi4hao2}{6,14}{⾃、⾗}[HSK 5]
  \definition{adj.}{orgulhar-se de; ter orgulho de; sentir-se honrado por possuir qualidades excelentes ou ter alcançado grandes conquistas, seja por si mesmo ou por um grupo ou indivíduo relacionado a si}
\end{entry}

\begin{entry}{自己}{zi4ji3}{6,3}{⾃、⼰}[HSK 2]
  \definition{pron.}{a si próprio; a si mesmo; refere-se ao substantivo ou pronome precedente (enfatiza principalmente que não é devido a forças externas)}
\end{entry}

\begin{entry}{自己动手}{zi4ji3dong4shou3}{6,3,6,4}{⾃、⼰、⼒、⼿}
  \definition{v.}{fazer (algo) sozinho | ajudar-se a}
\end{entry}

\begin{entry}{自救}{zi4jiu4}{6,11}{⾃、⽁}
  \definition{v.}{sair a si mesmo de problemas}
\end{entry}

\begin{entry}{自觉}{zi4jue2}{6,9}{⾃、⾒}[HSK 3]
  \definition{adj.}{autoconsciente; de ​​livre e espontânea vontade; controlar o próprio comportamento e agir por iniciativa própria}
  \definition{v.}{estar ciente de}
\end{entry}

\begin{entry}{自来水}{zi4lai2shui3}{6,7,4}{⾃、⽊、⽔}
  \definition{s.}{água corrente | água da torneira}
\end{entry}

\begin{entry}{自然}{zi4ran2}{6,12}{⾃、⽕}[HSK 3]
  \definition{adj.}{natural; no curso normal dos eventos; formado ou desenvolvido sem intervenção humana; algo que se desenvolve livremente}
  \definition{adv.}{naturalmente; definitivamente; certamente, isso significa que, de acordo com a lógica, deve ser assim}
  \definition{conj.}{usado para ligar duas frases, com a segunda introduzindo informações adicionais ou adversativas; indica explicação complementar ou uma mudança de significado}
  \definition{s.}{natureza; mundo natural; tudo o que não foi criado pelo ser humano}
\end{entry}

\begin{entry}{自燃}{zi4ran2}{6,16}{⾃、⽕}
  \definition{s.}{combustão espontânea}
\end{entry}

\begin{entry}{自杀}{zi4 sha1}{6,6}{⾃、⽊}[HSK 5]
  \definition{s.}{suicídio; auto-assassinato; auto-sacrifício}
  \definition{v.}{cometer suicídio; tentar suicídio; matar-se}
\end{entry}

\begin{entry}{自身}{zi4 shen1}{6,7}{⾃、⾝}[HSK 3]
  \definition{pron.}{eu mesmo (enfatizando que não é outra pessoa ou outra coisa)}
\end{entry}

\begin{entry}{自我}{zi4wo3}{6,7}{⾃、⼽}
  \definition{pref.}{auto}
  \definition{pron.}{a si mesmo | eu próprio | (psicologia) ego}
\end{entry}

\begin{entry}{自我安慰}{zi4wo3'an1wei4}{6,7,6,15}{⾃、⼽、⼧、⼼}
  \definition{v.}{confortar-se | consolar-se | tranquilizar-se}
\end{entry}

\begin{entry}{自我保存}{zi4wo3 bao3cun2}{6,7,9,6}{⾃、⼽、⼈、⼦}
  \definition{v.}{autopreservação}
\end{entry}

\begin{entry}{自我吹嘘}{zi4wo3chui1xu1}{6,7,7,14}{⾃、⼽、⼝、⼝}
  \definition{expr.}{tocar a própria buzina}
\end{entry}

\begin{entry}{自我催眠}{zi4wo3cui1mian2}{6,7,13,10}{⾃、⼽、⼈、⽬}
  \definition{v.}{consolar-me | tranquilizar-me}
\end{entry}

\begin{entry}{自我的人}{zi4wo3de5ren2}{6,7,8,2}{⾃、⼽、⽩、⼈}
  \definition{s.}{(minha, sua) própria pessoa | (afirmar) a própria personalidade}
\end{entry}

\begin{entry}{自我防卫}{zi4wo3fang2wei4}{6,7,6,3}{⾃、⼽、⾩、⼙}
  \definition{s.}{defesa pessoal | auto-defesa}
\end{entry}

\begin{entry}{自我解嘲}{zi4wo3jie3chao2}{6,7,13,15}{⾃、⼽、⾓、⼝}
  \definition{s.}{referir-se às próprias fraquezas ou falhas com humor autodepreciativo}
\end{entry}

\begin{entry}{自我介绍}{zi4wo3jie4shao4}{6,7,4,8}{⾃、⼽、⼈、⽷}
  \definition{s.}{defesa pessoal | auto-defesa}
\end{entry}

\begin{entry}{自我批评}{zi4wo3pi1ping2}{6,7,7,7}{⾃、⼽、⼿、⾔}
  \definition{s.}{autocrítica}
\end{entry}

\begin{entry}{自我实现}{zi4wo3shi2xian4}{6,7,8,8}{⾃、⼽、⼧、⾒}
  \definition{s.}{(psicologia) auto-atualização, auto-realização}
\end{entry}

\begin{entry}{自我陶醉}{zi4wo3tao2zui4}{6,7,10,15}{⾃、⼽、⾩、⾣}
  \definition{s.}{narcisista | auto-imbuído | satisfeito consigo mesmo}
\end{entry}

\begin{entry}{自我意识}{zi4wo3yi4shi2}{6,7,13,7}{⾃、⼽、⼼、⾔}
  \definition{s.}{autoapresentação}
  \definition{v.}{apresentar-se}
\end{entry}

\begin{entry}{自信}{zi4xin4}{6,9}{⾃、⼈}[HSK 4]
  \definition{adj.}{confiante; descreve a crença em suas próprias habilidades, decisões, etc., tendo confiança em si mesmo}
  \definition[份,种]{s.}{autoconfiança; confiança em si mesmo}
  \definition{v.}{acreditar em si mesmo;}
\end{entry}

\begin{entry}{自行车}{zi4xing2che1}{6,6,4}{⾃、⾏、⾞}[HSK 2]
  \definition[辆]{s.}{bicicleta; um veículo de duas rodas que é impulsionado para a frente com os pedais}
\end{entry}

\begin{entry}{自行车馆}{zi4xing2che1guan3}{6,6,4,11}{⾃、⾏、⾞、⾷}
  \definition{s.}{estádio de ciclismo | velódromo}
\end{entry}

\begin{entry}{自行车架}{zi4xing2che1jia4}{6,6,4,9}{⾃、⾏、⾞、⽊}
  \definition{s.}{suporte para bicicleta | bicicletário}
\end{entry}

\begin{entry}{自行车赛}{zi4xing2che1sai4}{6,6,4,14}{⾃、⾏、⾞、⾙}
  \definition{s.}{corrida de bicicleta}
\end{entry}

\begin{entry}{自由}{zi4you2}{6,5}{⾃、⽥}[HSK 2]
  \definition{adj.}{livre; irrestrito}
  \definition[个]{s.}{liberdade; o direito de agir de acordo com a própria vontade dentro do âmbito da lei | liberdade; filosoficamente, liberdade é definida como o processo de as pessoas reconhecerem as leis que governam o desenvolvimento das coisas e aplicá-las conscientemente na prática}
\end{entry}

\begin{entry}{自由泳}{zi4you2yong3}{6,5,8}{⾃、⽥、⽔}
  \definition{s.}{natação de estilo livre}
\end{entry}

\begin{entry}{自愿}{zi4yuan4}{6,14}{⾃、⽕}[HSK 5]
  \definition{adv.}{voluntariamente; por iniciativa própria; por vontade própria}
  \definition{s.}{voluntário}
\end{entry}

\begin{entry}{自责}{zi4ze2}{6,8}{⾃、⾙}
  \definition{v.}{culpar-se}
\end{entry}

\begin{entry}{自主}{zi4zhu3}{6,5}{⾃、⼂}[HSK 3]
  \definition{v.}{agir por conta própria; decidir por si mesmo; manter a iniciativa em suas próprias mãos; tomar suas próprias decisões}
\end{entry}

\begin{entry}{子}{zi5}{3}{⼦}[HSK 1]
  \definition{suf.}{sufixo para substantivos | sufixos de palavras de medida individuais; anexado a certas palavras classificadoras}
  \seeref{子}{zi3}
\end{entry}

\begin{entry}{综合}{zong1he2}{11,6}{⽷、⼝}[HSK 4]
  \definition{s.}{síntese}
  \definition{v.}{sintetizar; resumir as partes de uma coisa em um todo unificado após análise (em oposição a 分析); reunir coisas de um tipo ou natureza diferente}
  \seealsoref{分析}{fen1xi1}
\end{entry}

\begin{entry}{棕褐色}{zong1he4 se4}{12,14,6}{⽊、⾐、⾊}
  \definition{s.}{cor sépia | bronzeado}
\end{entry}

\begin{entry}{总}{zong3}{9}{⼼}[HSK 3]
  \definition{adj.}{total; geral; global | responsável (liderança)}
  \definition{adv.}{sempre; invariavelmente | de qualquer forma; afinal; eventualmente; mais cedo ou mais tarde; no fim das contas | certamente; provavelmente; com certeza; expressa estimativa; suposição; equivalente a 大概}
  \definition{v.}{reunir; resumir; juntar; compilar}
  \seealsoref{大概}{da4gai4}
\end{entry}

\begin{entry}{总裁}{zong3cai2}{9,12}{⼼、⾐}[HSK 5]
  \definition[位,名]{s.}{presidente (de uma empresa); nomes de certos líderes de partidos políticos ou grandes empresas}
\end{entry}

\begin{entry}{总长}{zong3chang2}{9,4}{⼼、⾧}
  \definition{s.}{comprimento total}
\end{entry}

\begin{entry}{总得}{zong3dei3}{9,11}{⼼、⼻}
  \definition{adv.}{prestes a}
  \definition{v.}{dever | precisar}
\end{entry}

\begin{entry}{总督}{zong3du1}{9,13}{⼼、⽬}
  \definition*{s.}{Governador-Geral | Governador | Vice-Rei}
\end{entry}

\begin{entry}{总共}{zong3gong4}{9,6}{⼼、⼋}[HSK 4]
  \definition{adv.}{em tudo; em todos; no total; completamente; totalmente; em conjunto}
\end{entry}

\begin{entry}{总价}{zong3jia4}{9,6}{⼼、⼈}
  \definition{s.}{preço total}
\end{entry}

\begin{entry}{总结}{zong3jie2}{9,9}{⼼、⽷}[HSK 3]
  \definition[个,篇]{s.}{resumo; síntese; conclusão resumida}
  \definition{v.}{resumir; sumariar; sintetizar; analisar e estudar as experiências para chegar a conclusões}
\end{entry}

\begin{entry}{总理}{zong3li3}{9,11}{⼼、⽟}[HSK 4]
  \definition*[个,位,名]{s.}{Primeiro-Ministro do Conselho de Estado; Título do líder do Conselho de Estado da China | Título do chefe de governo em determinados países | Primeiro-Ministro; Título de líderes de determinados partidos políticos | Título dos chefes de determinadas instituições e empresas nos velhos tempos}
  \definition{v.}{assumir a responsabilidade total;}
\end{entry}

\begin{entry}{总是}{zong3shi4}{9,9}{⼼、⽇}[HSK 3]
  \definition{adv.}{sempre; indica como tem sido durante um determinado período de tempo; um determinado estado permanece inalterado | afinal; significa que, independentemente do que acontecer, haverá ou será um resultado}
\end{entry}

\begin{entry}{总数}{zong3 shu4}{9,13}{⼼、⽁}[HSK 5]
  \definition{s.}{soma; total; totalidade; inventário; número total; soma total}
\end{entry}

\begin{entry}{总算}{zong3suan4}{9,14}{⼼、⽵}[HSK 5]
  \definition{adv.}{finalmente; por fim; indica que, após um longo período de tempo, um desejo finalmente se tornou realidade | suficiente; considerando tudo; no geral; considerando todos os aspectos; significa que, em geral, está tudo bem}
\end{entry}

\begin{entry}{总台}{zong3tai2}{9,5}{⼼、⼝}
  \definition{s.}{recepção | balcão de recepção}
\end{entry}

\begin{entry}{总体}{zong3 ti3}{9,7}{⼼、⼈}[HSK 5]
  \definition{s.}{total; geral; conjunto; totalidade; massa; população; o todo formado pela união de vários indivíduos; a totalidade das coisas}
\end{entry}

\begin{entry}{总统}{zong3tong3}{9,9}{⼼、⽷}[HSK 4]
  \definition*[个,位,名]{s.}{Presidente (de um país); Título dos líderes de determinadas repúblicas}
\end{entry}

\begin{entry}{总务}{zong3wu4}{9,5}{⼼、⼒}
  \definition{s.}{divisão de assuntos gerais | assuntos gerais | pessoa responsável geral}
\end{entry}

\begin{entry}{总线}{zong3xian4}{9,8}{⼼、⽷}
  \definition{s.}{barramento (computador) | \emph{computer bus}}
\end{entry}

\begin{entry}{总站}{zong3zhan4}{9,10}{⼼、⽴}
  \definition{s.}{terminal}
\end{entry}

\begin{entry}{总之}{zong3zhi1}{9,3}{⼼、⼂}[HSK 4]
  \definition{conj.}{em uma palavra; em suma; em resumo; indica que a declaração seguinte é uma declaração geral}
\end{entry}

\begin{entry}{总值}{zong3zhi2}{9,10}{⼼、⼈}
  \definition{s.}{valor total}
\end{entry}

\begin{entry}{纵}{zong4}{7}{⽷}
  \definition{adj.}{de norte a sul; geograficamente norte-sul | longitudinal | vertical; horizontal; paralelo ao lado longo do objeto | amassado; com rugas}
  \definition{conj.}{embora; mesmo que}
  \definition{v.}{libertar; deixar ir | entregar-se a; deixar-se levar | pular; saltar}
\end{entry}

\begin{entry}{赱}{zou3}{6}{⼟}
  \variantof{走}
\end{entry}

\begin{entry}{走}{zou3}{7}{⾛}[HSK 1][Kangxi 156]
  \definition{v.}{andar; caminhar | correr | mover; movimentar; deslocar | sair; partir; ir embora | visitar; fazer uma visita; (entre amigos e familiares) troca de visitas | passar por; atravessar; ultrapassar | vazar; revelar; divulgar | afastar-se do original; alterar ou perder a forma, o sabor, a cor, etc. originais}
\end{entry}

\begin{entry}{走鬼}{zou3gui3}{7,9}{⾛、⿁}
  \definition{s.}{vendedor ambulante sem licença}
\end{entry}

\begin{entry}{走过}{zou3 guo4}{7,6}{⾛、⾡}[HSK 2]
  \definition{v.}{passar por; perambular}
\end{entry}

\begin{entry}{走进}{zou3 jin4}{7,7}{⾛、⾡}[HSK 2]
  \definition{v.}{entrar}
\end{entry}

\begin{entry}{走开}{zou3 kai1}{7,4}{⾛、⼶}[HSK 2]
  \definition{v.}{ir embora; fugir; ir para outro lugar}
\end{entry}

\begin{entry}{走路}{zou3 lu4}{7,13}{⾛、⾜}[HSK 1]
  \definition{v.}{caminhar; ir a pé; andar em pé sobre a terra | sair; ir embora; partir}
\end{entry}

\begin{entry}{走去}{zou3qu4}{7,5}{⾛、⼛}
  \definition{v.}{caminhar até (para)}
\end{entry}

\begin{entry}{走绳}{zou3sheng2}{7,11}{⾛、⽷}
  \definition{v.}{andar na corda bamba}
  \seealsoref{走索}{zou3suo3}
\end{entry}

\begin{entry}{走势}{zou3shi4}{7,8}{⾛、⼒}
  \definition{s.}{caminho | tendência}
\end{entry}

\begin{entry}{走索}{zou3suo3}{7,10}{⾛、⽷}
  \definition{v.}{andar na corda bamba}
  \seealsoref{走绳}{zou3sheng2}
\end{entry}

\begin{entry}{走秀}{zou3xiu4}{7,7}{⾛、⽲}
  \definition{s.}{desfile de moda}
  \definition{v.}{andar na passarela (em um desfile de moda)}
\end{entry}

\begin{entry}{走卒}{zou3zu2}{7,8}{⾛、⼗}
  \definition{s.}{lacaio (masculino) | peão (isto é, soldado de infantaria) | servo}
\end{entry}

\begin{entry}{奏效}{zou4xiao4}{9,10}{⼤、⽁}
  \definition{v.}{mostrar resultados | ser eficaz}
\end{entry}

\begin{entry}{租}{zu1}{10}{⽲}[HSK 2]
  \definition{s.}{aluguel | imposto sobre a terra; tributação; (antigo) refere-se ao imposto predial}
  \definition{v.}{contratar; alugar; fretar | alugar; arrendar}
\end{entry}

\begin{entry}{租船}{zu1chuan2}{10,11}{⽲、⾈}
  \definition{v.}{fretar um navio | alugar um navio}
\end{entry}

\begin{entry}{租房}{zu1fang2}{10,8}{⽲、⼾}
  \definition{v.}{alugar um apartamento}
\end{entry}

\begin{entry}{租金}{zu1jin1}{10,8}{⽲、⾦}
  \definition{s.}{aluguel}
  \seealsoref{租钱}{zu1qian5}
\end{entry}

\begin{entry}{租赁}{zu1lin4}{10,10}{⽲、⾙}
  \definition{v.}{contratar | alugar}
\end{entry}

\begin{entry}{租钱}{zu1qian5}{10,10}{⽲、⾦}
  \definition{s.}{aluguel}
  \seealsoref{租金}{zu1jin1}
\end{entry}

\begin{entry}{租让}{zu1rang4}{10,5}{⽲、⾔}
  \definition{v.}{alugar | alugar (a propriedade de alguém para outra pessoa)}
\end{entry}

\begin{entry}{租用}{zu1yong4}{10,5}{⽲、⽤}
  \definition{v.}{contratar | alugar | alugar (algo de alguém)}
\end{entry}

\begin{entry}{租约}{zu1yue1}{10,6}{⽲、⽷}
  \definition{s.}{aluguel}
\end{entry}

\begin{entry}{足}{zu2}{7}{⾜}[Kangxi 157]
  \definition{adj.}{amplo}
  \definition{s.}{pé}
  \definition{v.}{ser suficiente}
  \seeref{足}{ju4}
\end{entry}

\begin{entry}{足够}{zu2 gou4}{7,11}{⾜、⼣}[HSK 3]
  \definition{adj.}{bastante; amplo; suficiente; atingir o nível adequado ou capaz de satisfazer as necessidades}
  \definition{v.}{satisfazer; ser suficiente; estar a contento}
\end{entry}

\begin{entry}{足球}{zu2qiu2}{7,11}{⾜、⽟}[HSK 3]
  \definition[个,只,颗,袋]{s.}{futebol | bola de futebol}
\end{entry}

\begin{entry}{足球场}{zu2qiu2chang3}{7,11,6}{⾜、⽟、⼟}
  \definition{s.}{campo de futebol}
\end{entry}

\begin{entry}{足球队}{zu2qiu2dui4}{7,11,4}{⾜、⽟、⾩}
  \definition{s.}{time de futebol}
\end{entry}

\begin{entry}{足球迷}{zu2qiu2mi2}{7,11,9}{⾜、⽟、⾡}
  \definition{s.}{fã (ou entusiasta) de futebol}
\end{entry}

\begin{entry}{足球赛}{zu2qiu2sai4}{7,11,14}{⾜、⽟、⾙}
  \definition{s.}{competição de futebol | partida de futebol}
\end{entry}

\begin{entry}{足球协会}{zu2qiu2xie2hui4}{7,11,6,6}{⾜、⽟、⼗、⼈}
  \definition*{s.}{Associação de Futebol}
\end{entry}

\begin{entry}{足月}{zu2yue4}{7,4}{⾜、⽉}
  \definition{s.}{gestação completa}
\end{entry}

\begin{entry}{足足}{zu2zu2}{7,7}{⾜、⾜}
  \definition{adv.}{tanto quanto | extremamente | completamente | não menos que}
\end{entry}

\begin{entry}{族}{zu2}{11}{⽅}
  \definition{s.}{raça | nacionalidade | etnia | clã | por extensão, grupo social}
\end{entry}

\begin{entry}{诅咒}{zu3zhou4}{7,8}{⾔、⼝}
  \definition{v.}{amaldiçoar}
\end{entry}

\begin{entry}{阻碍}{zu3'ai4}{7,13}{⾩、⽯}[HSK 5]
  \definition{s.}{obstáculo; impedimento; barreira}
  \definition{v.}{bloquear; impedir; obstruir; impedir o bom andamento ou desenvolvimento}
\end{entry}

\begin{entry}{阻击}{zu3ji1}{7,5}{⾩、⼐}
  \definition{v.}{verificar | parar}
\end{entry}

\begin{entry}{阻止}{zu3zhi3}{7,4}{⾩、⽌}[HSK 4]
  \definition{v.}{parar; reter; conter; interromper; impedir o avanço; impedir o movimento; obstruir}
\end{entry}

\begin{entry}{组}{zu3}{8}{⽷}[HSK 2]
  \definition{clas.}{usado para conjuntos, séries, suítes, baterias}
  \definition[个]{s.}{grupo; uma unidade composta por um pequeno número de pessoas}
  \definition{v.}{formar; organizar; combinar pessoas ou coisas dispersas em um todo ou sistema}
\end{entry}

\begin{entry}{组成}{zu3cheng2}{8,6}{⽷、⼽}[HSK 2]
  \definition{v.}{formar; compor; inventar}
\end{entry}

\begin{entry}{组合}{zu3he2}{8,6}{⽷、⼝}[HSK 3]
  \definition{s.}{associação; combinação; o todo organizado | combinação; retirar n elementos diferentes de m elementos e agrupá-los, independentemente da ordem, em que cada grupo contenha pelo menos um elemento diferente, o resultado obtido é chamado de combinação de n elementos de m}
  \definition{v.}{compor; constituir; formar}
\end{entry}

\begin{entry}{组长}{zu3 zhang3}{8,4}{⽷、⾧}[HSK 2]
  \definition[名,位,个]{s.}{líder de grupo; um supervisor de grupo}
\end{entry}

\begin{entry}{组织}{zu3zhi1}{8,8}{⽷、⽷}[HSK 5]
  \definition{s.}{organização; um coletivo ou grupo estabelecido de acordo com determinados objetivos e princípios | sistema organizado; vários fatores interligados de determinada maneira, formando um sistema | tecer; a combinação de linhas horizontais e verticais nos têxteis | tecido; os seres humanos, os animais, as plantas e outros seres vivos são compostos por uma combinação de células com formas e funções semelhantes, que formam os tecidos; os tecidos são as unidades que compõem os diversos órgãos}
\end{entry}

\begin{entry}{祖国}{zu3guo2}{9,8}{⽰、⼞}
  \definition{s.}{pátria | terra natal}
\end{entry}

\begin{entry}{钻戒}{zuan4jie4}{10,7}{⾦、⼽}
  \definition[只]{s.}{anel de diamante}
\end{entry}

\begin{entry}{钻石}{zuan4shi2}{10,5}{⾦、⽯}
  \definition[颗]{s.}{diamante}
\end{entry}

\begin{entry}{嘴}{zui3}{16}{⼝}[HSK 2]
  \definition[张]{s.}{boca; boca humana ou animal | qualquer coisa com formato ou função semelhante a uma boca | fala | comida}
\end{entry}

\begin{entry}{嘴巴}{zui3 ba5}{16,4}{⼝、⼰}[HSK 4]
  \definition[张]{s.}{boca}
\end{entry}

\begin{entry}{嘴巴子}{zui3ba5zi5}{16,4,3}{⼝、⼰、⼦}
  \definition{s.}{tapa | bofetada}
\end{entry}

\begin{entry}{最}{zui4}{12}{⽈}[HSK 1]
  \definition{adv.}{(diante de um adjetivo ou verbo) o mais | (colocado antes de um substantivo de localidade ou de uma palavra que indica um lugar)  mais distante ou mais próximo de (um lugar) | mais; melhor; pior; primeiro; muito; menos; acima de tudo; indica que uma determinada característica excede todas as outras pessoas ou coisas do mesmo tipo}
  \definition{s.}{o máximo; o melhor (ou o mais alto, o maior, etc.)}
\end{entry}

\begin{entry}{最初}{zui4chu1}{12,7}{⽈、⾐}[HSK 4]
  \definition{adj.}{primordial; inicial; primeiro}
  \definition{adv.}{inicialmente; originalmente}
  \definition{s.}{o período mais antigo; início; começo}
\end{entry}

\begin{entry}{最多}{zui4duo1}{12,6}{⽈、⼣}
  \definition{adv.}{no máximo | máximo}
\end{entry}

\begin{entry}{最高}{zui4gao1}{12,10}{⽈、⾼}
  \definition{adj.}{altíssimo | supremo | mais alto}
\end{entry}

\begin{entry}{最好}{zui4hao3}{12,6}{⽈、⼥}[HSK 1]
  \definition{adj.}{melhor; de primeira qualidade; excelente}
  \definition{adv.}{seria melhor; seria o ideal; indica a escolha mais adequada entre várias possibilidades}
\end{entry}

\begin{entry}{最后}{zui4hou4}{12,6}{⽈、⼝}[HSK 1]
  \definition{s.}{último; final; definitivo; refere-se ao tempo, local, etc. que vem depois de outros tempos, locais, etc. na ordem sequencial}
\end{entry}

\begin{entry}{最佳}{zui4jia1}{12,8}{⽈、⼈}
  \definition{adj.}{melhor (atleta, filme etc) | ótimo}
\end{entry}

\begin{entry}{最近}{zui4jin4}{12,7}{⽈、⾡}[HSK 2]
  \definition{adj.}{mais próximo}
  \definition{s.}{recentemente; ultimamente; de tarde; refere-se aos dias antes ou logo depois de um discurso | em breve; no futuro próximo; o futuro próximo}
\end{entry}

\begin{entry}{最善}{zui4shan4}{12,12}{⽈、⼝}
  \definition{adj.}{ótimo | o melhor}
\end{entry}

\begin{entry}{最少}{zui4shao3}{12,4}{⽈、⼩}
  \definition{adv.}{finalmente}
\end{entry}

\begin{entry}{最先}{zui4xian1}{12,6}{⽈、⼉}
  \definition{adv.}{o primeiro}
\end{entry}

\begin{entry}{最新}{zui4xin1}{12,13}{⽈、⽄}
  \definition{adv.}{mais recente | mais novo}
\end{entry}

\begin{entry}{最优}{zui4you1}{12,6}{⽈、⼈}
  \definition{adj.}{ótimo}
\end{entry}

\begin{entry}{最远}{zui4yuan3}{12,7}{⽈、⾡}
  \definition{adv.}{mais distante | mais longe}
\end{entry}

\begin{entry}{最终}{zui4zhong1}{12,8}{⽈、⽷}
  \definition{adv.}{pelo menos | finalmente}
  \definition{s.}{final | ultimato}
\end{entry}

\begin{entry}{罪犯}{zui4fan4}{13,5}{⽹、⽝}
  \definition{s.}{criminoso}
\end{entry}

\begin{entry}{罪行}{zui4xing2}{13,6}{⽹、⾏}
  \definition{s.}{crime | ofensa}
\end{entry}

\begin{entry}{醉}{zui4}{15}{⾣}[HSK 5]
  \definition{v.}{embriagar-se; ficar bêbado; intoxicar-se; beber em excesso e perder o controle | (de certos alimentos) ser embebido em licor; ser mergulhado em vinho; marinar (alimentos) em vinho | entregar-se a; ser viciado em; gostar demais, a ponto de chegar à obsessão}
\end{entry}

\begin{entry}{尊敬}{zun1jing4}{12,12}{⼨、⽁}[HSK 5]
  \definition{adj.}{respeitoso; respeitável}
  \definition{v.}{respeitar; honrar; estimar}
\end{entry}

\begin{entry}{尊重}{zun1zhong4}{12,9}{⼨、⾥}[HSK 5]
  \definition{adj.}{sério; adequado; correto; (linguagem, comportamento) não ser descuidado; não ser leviano}
  \definition{v.}{respeitar; valorizar; estimar; tratar com educação; valorizar | tratar com seriedade; levar a sério e tratar com seriedade}
\end{entry}

\begin{entry}{遵守}{zun1shou3}{15,6}{⾡、⼧}[HSK 5]
  \definition{v.}{obedecer; observar; cumprir; respeitar; atuar de acordo com as regras; não infringir}
\end{entry}

\begin{entry}{作}{zuo1}{7}{⼈}
  \definition{adj.}{(gíria) incômodo}
  \definition{s.}{trabalhador | oficina | (pessoa) de alta manutenção}
  \seeref{作}{zuo4}
\end{entry}

\begin{entry}{昨}{zuo2}{9}{⽇}
  \definition{s.}{ontem}
\end{entry}

\begin{entry}{昨日}{zuo2ri4}{9,4}{⽇、⽇}
  \definition{adv.}{ontem}
\end{entry}

\begin{entry}{昨天}{zuo2tian1}{9,4}{⽇、⼤}[HSK 1]
  \definition{s.}{ontem}
\end{entry}

\begin{entry}{昨晚}{zuo2wan3}{9,11}{⽇、⽇}
  \definition{adv.}{noite passada | ontem à noite}
\end{entry}

\begin{entry}{昨夜}{zuo2ye4}{9,8}{⽇、⼣}
  \definition{adv.}{noite passada}
\end{entry}

\begin{entry}{左}{zuo3}{5}{⼯}[HSK 1]
  \definition*{s.}{sobrenome Zuo}
  \definition{adj.}{estranho; herético; não ortodoxo | errado; incorreto | diferente; contrário; oposto | progressista; revolucionário; politicamente e ideologicamente progressista; radical}
  \definition{s.}{a esquerda; o lado esquerdo | leste; na antiguidade, referia-se especificamente à direção leste (com base na orientação para o sul) | a esquerda; ala esquerda; refere-se a uma posição inferior (na antiguidade, a direita era considerada superior e a esquerda, inferior)}
  \definition{v.}{assistir; auxiliar}
\end{entry}

\begin{entry}{左边}{zuo3bian5}{5,5}{⼯、⾡}[HSK 1]
  \definition{s.}{esquerda; o lado esquerdo}
\end{entry}

\begin{entry}{左面}{zuo3mian4}{5,9}{⼯、⾯}
  \definition{s.}{esquerda | lado esquerdo}
\end{entry}

\begin{entry}{左派}{zuo3pai4}{5,9}{⼯、⽔}
  \definition{s.}{(política) esquerda | esquerdista}
\end{entry}

\begin{entry}{左倾}{zuo3qing1}{5,10}{⼯、⼈}
  \definition{s.}{esquerdista | progressivo}
\end{entry}

\begin{entry}{左袒}{zuo3tan3}{5,10}{⼯、⾐}
  \definition{v.}{ser tendencioso | ser parcial para | favorecer um lado | tomar partido com}
\end{entry}

\begin{entry}{左舷}{zuo3xian2}{5,11}{⼯、⾈}
  \definition{s.}{porto (lado de um navio)}
\end{entry}

\begin{entry}{左翼}{zuo3yi4}{5,17}{⼯、⽻}
  \definition{s.}{esquerda (política)}
\end{entry}

\begin{entry}{左右}{zuo3you4}{5,5}{⼯、⼝}[HSK 3]
  \definition{s.}{os lados esquerdo e direito; esquerda e direita, também indicam os arredores | atendentes; acompanhantes; as pessoas que o acompanham | aproximadamente; mais ou menos; por aí; usado após números para indicar uma estimativa, com o mesmo significado de 上下}
  \definition{v.}{controlar; manipular; influenciar; dominar}
  \seealsoref{上下}{shang4 xia4}
\end{entry}

\begin{entry}{作}{zuo4}{7}{⼈}
  \definition{s.}{escritos ou obras}
  \definition{v.}{fazer | crescer | escrever ou compor | fingir | considerar como | sentir}
  \seeref{作}{zuo1}
\end{entry}

\begin{entry}{作出}{zuo4 chu1}{7,5}{⼈、⼐}[HSK 4]
  \definition{v.}{mostrar; tomar (decisões, conclusões, etc. por meio de consideração ou discussão); formar (uma conclusão, decisão, etc.) por meio de consideração ou discussão}
\end{entry}

\begin{entry}{作家}{zuo4jia1}{7,10}{⼈、⼧}[HSK 2]
  \definition[位,名,个,些]{s.}{escritor; autor; pessoas que alcançaram sucesso na criação literária}
\end{entry}

\begin{entry}{作品}{zuo4pin3}{7,9}{⼈、⼝}[HSK 3]
  \definition[个,部,篇,幅]{s.}{obra de arte; obras literárias e artísticas}
\end{entry}

\begin{entry}{作为}{zuo4wei2}{7,4}{⼈、⼂}[HSK 4]
  \definition{prep.}{como; na capacidade de; no caráter de; no papel de; em termos de uma certa identidade de uma pessoa ou de uma certa natureza de uma coisa}
  \definition{s.}{ato; ação; conduta; feito; comportamento | conquista; realização; especificamente, uma boa ação}
  \definition{v.}{considerar como; tomar por; olhar como; tratar como | realizar; fazer conquistas; deixar uma marca}
\end{entry}

\begin{entry}{作文}{zuo4wen2}{7,4}{⼈、⽂}[HSK 2]
  \definition[篇]{s.}{ensaio; composição; redação}
  \definition{v.+compl.}{(de alunos) escrever uma redação, artigo ou ensaio}
\end{entry}

\begin{entry}{作业}{zuo4ye4}{7,5}{⼈、⼀}[HSK 2]
  \definition[份,个]{s.}{tarefa escolar; tarefa de casa atribuída pelos professores aos alunos}
  \definition{v.}{trabalhar; executar tarefa}
\end{entry}

\begin{entry}{作用}{zuo4yong4}{7,5}{⼈、⽤}[HSK 2]
  \definition[副]{s.}{efeito; ação; função; a influência sobre as coisas; o efeito; a utilidade}
  \definition{v.}{afetar; agir sobre; realizar atividades que têm algum impacto nas coisas}
\end{entry}

\begin{entry}{作者}{zuo4zhe3}{7,8}{⼈、⽼}[HSK 3]
  \definition[位,名,个]{s.}{autor; escritor; pessoas que escrevem artigos ou criam obras de arte}
\end{entry}

\begin{entry}{坐}{zuo4}{7}{⼟}[HSK 1]
  \definition*{s.}{sobrenome Zuo}
  \definition{adv.}{sem motivo algum; sem causa ou razão; sem motivo aparente}
  \definition{prep.}{porque; pelo fato de que; pela razão de que; pelo motivo de que}
  \definition{s.}{assento; lugar; posição}
  \definition{v.}{sentar; sentar-se; ocupar um lugar; colocar os glúteos sobre um objeto para apoiar o peso corporal | pegar; viajar de; pegar carona | ter as costas voltadas para | colocar (uma panela, chaleira, etc.) no fogo | recuo; coice (de rifles, armas, etc.)  | produzir frutos; formar sementes | ser punido; ser acusado de crime | contrair (ou ter) uma doença; sofrer de uma doença | (um edifício) afundar; ceder}
\end{entry}

\begin{entry}{坐标}{zuo4biao1}{7,9}{⼟、⽊}
  \definition{s.}{coordenada (geometria)}
\end{entry}

\begin{entry}{坐车}{zuo4che1}{7,4}{⼟、⾞}
  \definition{v.}{andar de carro, ônibus, trem, etc.}
\end{entry}

\begin{entry}{坐垫}{zuo4dian4}{7,9}{⼟、⼟}
  \definition[块]{s.}{assento (motocicleta) | almofada}
\end{entry}

\begin{entry}{坐好}{zuo4hao3}{7,6}{⼟、⼥}
  \definition{v.}{sentar-se corretamente | sentar direito}
\end{entry}

\begin{entry}{坐下}{zuo4 xia5}{7,3}{⼟、⼀}[HSK 1]
  \definition{v.}{sentar-se; tomar um assento; passar da posição em pé para a posição sentada}
\end{entry}

\begin{entry}{坐享}{zuo4xiang3}{7,8}{⼟、⼇}
  \definition{v.}{curtir algo sem levantar um dedo}
\end{entry}

\begin{entry}{座}{zuo4}{10}{⼴}[HSK 2]
  \definition{clas.}{usado para montanhas, edifícios e objetos imóveis semelhantes}
  \definition{s.}{assento; lugar | suporte; pedestal; base | (astronomia) constalação | (antigo) forma de tratamento a altos funcionários |}
\end{entry}

\begin{entry}{座标}{zuo4biao1}{10,9}{⼴、⽊}
  \variantof{坐标}
\end{entry}

\begin{entry}{座位}{zuo4wei4}{10,7}{⼴、⼈}[HSK 2]
  \definition[个,排]{s.}{assento; lugar}
\end{entry}

\begin{entry}{座子}{zuo4zi5}{10,3}{⼴、⼦}
  \definition{s.}{soquete | pedestal | sela}
\end{entry}

\begin{entry}{做}{zuo4}{11}{⼈}[HSK 1]
  \definition{v.}{fabricar; produzir; criar | escrever; compor | fazer; trabalhar em; dedicar-se a; exercer uma determinada profissão ou atividade | realizar uma festa em família; comemorar | ser; tornar-se; agir como; atuar como | ser usado como | formar ou estabelecer um relacionamento; conectar-se (em algum tipo de relação) | fingir (alguma coisa) | cozinhar; preparar}
\end{entry}

\begin{entry}{做到}{zuo4 dao4}{11,8}{⼈、⼑}[HSK 2]
  \definition{v.}{alcançar; realizar; atingir um determinado objetivo; atingir um determinado padrão}
\end{entry}

\begin{entry}{做法}{zuo4fa3}{11,8}{⼈、⽔}[HSK 2]
  \definition[种,个]{s.}{método; maneira de fazer algo; métodos de lidar com coisas ou fazer coisas}
\end{entry}

\begin{entry}{做饭}{zuo4 fan4}{11,7}{⼈、⾷}[HSK 2]
  \definition{v.}{cozinhar; preparar uma refeição; cozinhar refeições e transformar alimentos crus em alimentos cozidos}
\end{entry}

\begin{entry}{做活}{zuo4huo2}{11,9}{⼈、⽔}
  \definition{v.}{trabalhar para ganhar a vida (especialmente de mulher costureira)}
\end{entry}

\begin{entry}{做客}{zuo4 ke4}{11,9}{⼈、⼧}[HSK 3]
  \definition{v.}{visitar; ser um convidado; ser hóspede}
\end{entry}

\begin{entry}{做梦}{zuo4 meng4}{11,11}{⼈、⼣}[HSK 4]
  \definition{s.}{sonho; ilusões e visões na consciência durante o sono}
  \definition{v.}{sonhar; ter um sonho | sonhar acordado, ter um sonho impossível (parábola de fantasias irrealistas)}[别​做​梦​了​,她​不​会​嫁​给​你​的​。___Pare de sonhar, ela não se casará com você.]
\end{entry}

\begin{entry}{做生活}{zuo4sheng1huo2}{11,5,9}{⼈、⽣、⽔}
  \definition{v.}{fazer tabalhos manuais}
\end{entry}

\begin{entry}{做戏}{zuo4xi4}{11,6}{⼈、⼽}
  \definition{v.}{atuar em uma peça | fazer uma peça}
\end{entry}

\begin{entry}{做眼}{zuo4yan3}{11,11}{⼈、⽬}
  \definition{v.}{agir como um guia | trabalhar como espião}
\end{entry}

\begin{entry}{做作}{zuo4zuo5}{11,7}{⼈、⼈}
  \definition{adj.}{afetado | artificial}
\end{entry}

\begin{entry}{酢}{zuo4}{12}{⾣}
  \definition{s.}{brinde ao anfitrião feito pelo convidado}
\end{entry}

%%%%% EOF %%%%%

