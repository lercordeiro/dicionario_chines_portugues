%%%
%%% X
%%%

\section*{X}\addcontentsline{toc}{section}{X}

\begin{entry}{夕}{xi1}{3}{⼣}
  \definition*{s.}{sobrenome Xi}
  \definition{s.}{pôr do sol; crepúsculo | tarde; noite}
\end{entry}

\begin{entry}{夕阳}{xi1yang2}{3,6}{⼣、⾩}
  \definition{s.}{pôr do sol}
  \seealsoref{日出}{ri4chu1}
\end{entry}

\begin{entry}{吸}{xi1}{6}{⼝}[HSK 4]
  \definition{v.}{inalar; inspirar; aspirar; itroduzir líquidos, gases, etc. no corpo | absorver; sugar | atrair; atrair para si mesmo; atrair (interesse, investimento etc.)}
\end{entry}

\begin{entry}{吸管}{xi1 guan3}{6,14}{⼝、⽵}[HSK 4]
  \definition[根,个]{s.}{tubo de sucção; sugador; canudo (para beber); refere-se ao tubo fino usado para sugar bebidas | conta-gotas; pipeta; cateter para bombeamento de líquidos usando pressão de ar}
\end{entry}

\begin{entry}{吸收}{xi1shou1}{6,6}{⼝、⽁}[HSK 4]
  \definition{v.}{imbuir; absorver; assimilar; sugar;  chupar; (animais, plantas, etc.) extrair material de fora dos tecidos para o interior dos tecidos | absorver; chupar;  sugar alguma substância de fora para dentro | recrutar; alistar; inscrever-se; matricular-se; admitir; (organizações ou coletivos) aceitar novos membros | absorver; aproveitar e usar a experiência, o conhecimento, o dinheiro e outras coisas valiosas de outras pessoas | absorver; diminuir, atenuar ou eliminar determinados efeitos ou fenômenos}
\end{entry}

\begin{entry}{吸铁石}{xi1tie3shi2}{6,10,5}{⼝、⾦、⽯}
  \definition{s.}{imã | magneto}
  \seealsoref{磁铁}{ci2tie3}
\end{entry}

\begin{entry}{吸烟}{xi1yan1}{6,10}{⼝、⽕}[HSK 4]
  \definition{v.+compl.}{fumar}
\end{entry}

\begin{entry}{吸引}{xi1yin3}{6,4}{⼝、⼸}[HSK 4]
  \definition{v.}{atrair; apelar para; chamar a atenção de outros objetos, forças ou pessoas para si mesmo}
\end{entry}

\begin{entry}{西}{xi1}{6}{⾑}[HSK 1]
  \definition*{s.}{sobrenome Xi}
  \definition*{s.}{abreviatura de Espanha | Paraíso Ocidental}
  \definition{s.}{oeste; uma das quatro direções básicas, o lado onde o sol se põe (oposto ao 东) | ocidental; refere-se ao Ocidente (principalmente aos países europeus e americanos) | aqui e ali; em contraposição a 东, significa 到处 ou 零散, 没有次序}
  \seealsoref{到处}{dao4chu4}
  \seealsoref{东}{dong1}
  \seealsoref{零散}{ling2san3}
  \seealsoref{没有次序}{mei2you3 ci4xu4}
\end{entry}

\begin{entry}{西安}{xi1'an1}{6,6}{⾑、⼧}
  \definition*{s.}{Xi'an}
\end{entry}

\begin{entry}{西班牙文}{xi1ban1ya2wen2}{6,10,4,4}{⾑、⽟、⽛、⽂}
  \definition{s.}{espanhol, língua espanhola}
  \seealsoref{西文}{xi1wen2}
\end{entry}

\begin{entry}{西班牙语}{xi1ban1ya2yu3}{6,10,4,9}{⾑、⽟、⽛、⾔}
  \definition{s.}{espanhol | língua espanhola}
  \seealsoref{西语}{xi1yu3}
\end{entry}

\begin{entry}{西半球}{xi1ban4qiu2}{6,5,11}{⾑、⼗、⽟}
  \definition{s.}{hemisfério oeste}
\end{entry}

\begin{entry}{西北}{xi1 bei3}{6,5}{⾑、⼔}[HSK 2]
  \definition{s.}{noroeste | noroeste da China; o Noroeste}
\end{entry}

\begin{entry}{西边}{xi1bian1}{6,5}{⾑、⾡}[HSK 1]
  \definition{s.}{lado oeste; (oeste) Uma das quatro direções principais; uma das direções cardeais, oposta ao 东方}
  \seealsoref{东方}{dong1 fang1}
\end{entry}

\begin{entry}{西部}{xi1 bu4}{6,10}{⾑、⾢}[HSK 3]
  \definition{s.}{(EUA) filme de faroeste; filme de \emph{cowboys}; um faroeste | filme da região ocidental (China) | parte ocidental; região oeste da China}
\end{entry}

\begin{entry}{西餐}{xi1 can1}{6,16}{⾑、⾷}[HSK 2]
  \definition[份,顿,桌]{s.}{comida ocidental; comida de estilo ocidental, comida com garfo e faca (diferente da 中餐)}
  \seealsoref{中餐}{zhong1 can1}
\end{entry}

\begin{entry}{西方}{xi1 fang1}{6,4}{⾑、⽅}[HSK 2]
  \definition{s.}{oeste | o Ocidente; o Oeste; países europeus e americanos | Paraíso Ocidental, termo budista}
\end{entry}

\begin{entry}{西瓜}{xi1gua1}{6,5}{⾑、⽠}[HSK 4]
  \definition[个,颗,粒]{s.}{melancia; fruto que é uma baga de formato grande, globular ou oval, com muita polpa aguada e doce}
\end{entry}

\begin{entry}{西红柿}{xi1hong2shi4}{6,6,9}{⾑、⽷、⽊}[HSK 5]
  \definition[种,只]{s.}{tomate;}
\end{entry}

\begin{entry}{西兰花}{xi1lan2hua1}{6,5,7}{⾑、⼋、⾋}
  \definition{s.}{brócolis}
\end{entry}

\begin{entry}{西蓝花}{xi1lan2hua1}{6,13,7}{⾑、⾋、⾋}
  \variantof{西兰花}
\end{entry}

\begin{entry}{西面}{xi1mian4}{6,9}{⾑、⾯}
  \definition{s.}{oeste | lado oeste}
\end{entry}

\begin{entry}{西南}{xi1 nan2}{6,9}{⾑、⼗}[HSK 2]
  \definition{s.}{sudoeste | o Sudoeste; Sudoeste da China}
\end{entry}

\begin{entry}{西文}{xi1wen2}{6,4}{⾑、⽂}
  \definition{s.}{espanhol | língua espanhola}
  \seealsoref{西班牙文}{xi1ban1ya2wen2}
\end{entry}

\begin{entry}{西西}{xi1xi1}{6,6}{⾑、⾑}
  \definition{num.}{centímetro cúbico}
\end{entry}

\begin{entry}{西药}{xi1 yao4}{6,9}{⾑、⾋}
  \definition{s.}{medicina ocidental; refere-se aos medicamentos usados ​​na medicina ocidental, geralmente feitos por métodos sintéticos ou extraídos de produtos naturais, como comprimidos anti-inflamatórios, aspirina, tintura de iodo, penicilina, etc.}
\end{entry}

\begin{entry}{西医}{xi1 yi1}{6,7}{⾑、⼖}[HSK 2]
  \definition[名,位]{s.}{medicina ocidental; medicina introduzida na China a partir da Europa e da América | um médico treinado em medicina ocidental}
\end{entry}

\begin{entry}{西语}{xi1yu3}{6,9}{⾑、⾔}
  \definition{s.}{espanhol | língua espanhola}
  \seealsoref{西班牙语}{xi1ban1ya2yu3}
\end{entry}

\begin{entry}{西装}{xi1 zhuang1}{6,12}{⾑、⾐}[HSK 5]
  \definition[件,套,个]{s.}{terno; roupas de estilo ocidental; roupas ocidentais, divididas em masculinas e femininas}
\end{entry}

\begin{entry}{希}{xi1}{7}{⼱}
  \definition*{s.}{sobrenome Xi}
  \definition{v.}{ter esperança}
\end{entry}

\begin{entry}{希望}{xi1wang4}{7,11}{⼱、⽉}[HSK 3]
  \definition[个,丝,点]{s.}{esperança; desejo; expectativa; a possibilidade de alcançar um determinado objetivo ou de ocorrer uma determinada situação ideal no futuro | aquilo em que a esperança é depositada; o objeto da esperança}
  \definition{v.}{ter esperança; desejar; esperar; pensar em alcançar algum objetivo ou que alguma situação ocorra}
\end{entry}

\begin{entry}{昔}{xi1}{8}{⽇}
  \definition{s.}{tempos antigos; o passado; era uma vez}
\end{entry}

\begin{entry}{昔日}{xi1ri4}{8,4}{⽇、⽇}
  \definition{adj.}{passado}
\end{entry}

\begin{entry}{牺}{xi1}{10}{⽜}
  \definition{s.}{um animal de cor uniforme para sacrifício; sacrifício; gado com pelagem pura usado para sacrifício}
\end{entry}

\begin{entry}{牺牲}{xi1sheng1}{10,9}{⽜、⽜}
  \definition{s.}{abate de um animal como sacrifício}
  \definition{v.}{sacrificar a vida de alguém | sacrificar (algo de valor)}
\end{entry}

\begin{entry}{悉}{xi1}{11}{⼼}
  \definition*{s.}{sobrenome Xi}
  \definition{adj.}{tudo; inteiro; total | detalhado}
  \definition{v.}{saber; aprender; ser informado de}
\end{entry}

\begin{entry}{悉尼}{xi1ni2}{11,5}{⼼、⼫}
  \definition*{s.}{Sidney}
\end{entry}

\begin{entry}{悉数}{xi1shu3}{11,13}{⼼、⽁}
  \definition{adv.}{enumerar em detalhes | explicar claramente}
  \seeref{悉数}{xi1shu4}
\end{entry}

\begin{entry}{悉数}{xi1shu4}{11,13}{⼼、⽁}
  \definition{adv.}{todos | cada um | toda a soma}
  \seeref{悉数}{xi1shu3}
\end{entry}

\begin{entry}{悉心}{xi1xin1}{11,4}{⼼、⼼}
  \definition{adv.}{colocar o coração (e a alma) em algo | com muito cuidado}
\end{entry}

\begin{entry}{蜥}{xi1}{14}{⾍}
  \definition{s.}{lagarto}
\end{entry}

\begin{entry}{蜥易}{xi1yi4}{14,8}{⾍、⽇}
  \variantof{蜥蜴}
\end{entry}

\begin{entry}{蜥蜴}{xi1yi4}{14,14}{⾍、⾍}
  \definition{s.}{lagarto}
\end{entry}

\begin{entry}{习}{xi2}{3}{⼄}
  \definition*{s.}{sobrenome Xi}
  \definition{s.}{hábito; costume; prática usual; um comportamento que se desenvolve inconscientemente por meio de ações repetidas ao longo de um longo período de tempo}
  \definition{v.}{revisar; praticar; exercitar | acostumado a; familiarizado com; familiarizado com algo por meio de contato frequente | estudar; aprender (pássaro)}
\end{entry}

\begin{entry}{习惯}{xi2guan4}{3,11}{⼄、⼼}[HSK 2]
  \definition[个,种]{s.}{hábito; costume; prática usual; comportamentos, tendências ou tendências sociais que se desenvolvem gradualmente ao longo de um longo período de tempo e são difíceis de mudar}
  \definition{v.}{estar acostumado a; ter o hábito de}
\end{entry}

\begin{entry}{席}{xi2}{10}{⼱}
  \definition*{s.}{sobrenome Xi}
  \definition[卷,张]{s.}{esteira | assento; lugar; caixa | assento (em uma assembleia legislativa) | festim; banquete; jantar}
\end{entry}

\begin{entry}{席卷}{xi2juan3}{10,8}{⼱、⼙}
  \definition{v.}{engolfar | varrer | levar tudo para fora}
\end{entry}

\begin{entry}{袭}{xi2}{11}{⾐}
  \definition*{s.}{sobrenome Xi}
  \definition{clas.}{usado para conjuntos completos de roupas}
  \definition{v.}{fazer um ataque surpresa a; invadir | seguir o padrão de; continuar como antes; fazer o mesmo}
\end{entry}

\begin{entry}{袭击}{xi2ji1}{11,5}{⾐、⼐}
  \definition{s.}{ataque (especialmente um ataque surpresa) | invasão}
  \definition{v.}{atacar}
\end{entry}

\begin{entry}{洗}{xi3}{9}{⽔}[HSK 1]
  \definition[个]{s.}{pequeno recipiente contendo água para enxaguar os pincéis de escrever | batismo}
  \definition{v.}{lavar; tomar banho; remover a sujeira do objeto com água ou outro solvente | batizar | eliminar; corrigir; reparar | saquear; matar e pilhar; matar ou roubar tudo, como se tivesse sido lavado | revelar filmes; imprimir fotos | apagar; limpar (uma gravação, etc.) | embaralhar (cartas, etc.)}
\end{entry}

\begin{entry}{洗涤}{xi3di2}{9,10}{⽔、⽔}
  \definition{s.}{enxágue | lava}
  \definition{v.}{enxaguar | lavar}
\end{entry}

\begin{entry}{洗涤间}{xi3di2jian1}{9,10,7}{⽔、⽔、⾨}
  \definition{s.}{lavanderia}
\end{entry}

\begin{entry}{洗劫}{xi3jie2}{9,7}{⽔、⼒}
  \definition{v.}{saquear | pilhar | roubar}
\end{entry}

\begin{entry}{洗净}{xi3jing4}{9,8}{⽔、⼎}
  \definition{v.}{lavar (limpeza)}
\end{entry}

\begin{entry}{洗礼}{xi3li3}{9,5}{⽔、⽰}
  \definition{s.}{batismo}
  \definition{v.}{batizar}
\end{entry}

\begin{entry}{洗手}{xi3shou3}{9,4}{⽔、⼿}
  \definition{v.}{ir ao banheiro | lavar as mãos}
\end{entry}

\begin{entry}{洗手不干}{xi3shou3bu2gan4}{9,4,4,3}{⽔、⼿、⼀、⼲}
  \definition{v.}{parar totalmente de fazer algo}
\end{entry}

\begin{entry}{洗手池}{xi3shou3chi2}{9,4,6}{⽔、⼿、⽔}
  \definition{s.}{pia de banheiro | lavatório}
  \seealsoref{洗手盆}{xi3shou3pen2}
\end{entry}

\begin{entry}{洗手间}{xi3shou3jian1}{9,4,7}{⽔、⼿、⾨}[HSK 1]
  \definition[个]{s.}{banheiro; lavatório; lavabo}
\end{entry}

\begin{entry}{洗手盆}{xi3shou3pen2}{9,4,9}{⽔、⼿、⽫}
  \definition{s.}{pia de banheiro | lavatório}
  \seealsoref{洗手池}{xi3shou3chi2}
\end{entry}

\begin{entry}{洗手乳}{xi3shou3ru3}{9,4,8}{⽔、⼿、⼄}
  \definition{s.}{sabonete líquido para lavar as mãos}
  \seealsoref{洗手液}{xi3shou3ye4}
\end{entry}

\begin{entry}{洗手液}{xi3shou3ye4}{9,4,11}{⽔、⼿、⽔}
  \definition{s.}{sabonete líquido para lavar as mãos}
  \seealsoref{洗手乳}{xi3shou3ru3}
\end{entry}

\begin{entry}{洗脱}{xi3tuo1}{9,11}{⽔、⾁}
  \definition{v.}{limpar | purgar | lavar}
\end{entry}

\begin{entry}{洗碗}{xi3wan3}{9,13}{⽔、⽯}
  \definition{v.}{lavar pratos}
\end{entry}

\begin{entry}{洗胃}{xi3wei4}{9,9}{⽔、⾁}
  \definition{s.}{(medicina) lavagem gástrica}
  \definition{v.}{ter o estômago lavado}
\end{entry}

\begin{entry}{洗衣机}{xi3 yi1 ji1}{9,6,6}{⽔、⾐、⽊}[HSK 2]
  \definition[台]{s.}{máquina de lavar roupa; eletrodomésticos para lavagem automática ou semiautomática de roupas}
\end{entry}

\begin{entry}{洗澡}{xi3zao3}{9,16}{⽔、⽔}[HSK 2]
  \definition{v.+compl.}{tomar banho; tomar banho de chuveiro; lavar-se}
\end{entry}

\begin{entry}{洗澡间}{xi3zao3jian1}{9,16,7}{⽔、⽔、⾨}
  \definition[间]{s.}{banheiro}
\end{entry}

\begin{entry}{喜}{xi3}{12}{⼝}
  \definition{adj.}{feliz; satisfeito; encantado}
  \definition[桩,件]{s.}{evento feliz (especialmente casamento); ocasião para celebração; algo para comemorar | gravidez | casamento ou coisas relacionadas a ele}
  \definition{v.}{gostar; fonte de; ter inclinação para | precisa; requer; combina melhor com; (um certo organismo) precisa ou é adequado para (um certo ambiente ou algo)}
\end{entry}

\begin{entry}{喜爱}{xi3 ai4}{12,10}{⼝、⽖}[HSK 4]
  \definition{v.}{gostar; amar; ter afeição por; estar interessado em; ter uma queda ou sentir interesse por pessoas ou coisas}
\end{entry}

\begin{entry}{喜欢}{xi3huan5}{12,6}{⼝、⽋}[HSK 1]
  \definition{adj.}{feliz; encantado; exultante; cheio de alegria}
  \definition{v.}{gostar; amar; ter afeição por; estar interessado em; ter uma boa impressão ou interesse por alguém ou algo}
\end{entry}

\begin{entry}{喜剧}{xi3 ju4}{12,10}{⼝、⼑}[HSK 5]
  \definition[部,出]{s.}{comédia (oposto de 悲剧) | comédia; uma das principais categorias do teatro; usa o exagero para satirizar e ridicularizar o feio; fenômenos retrógrados; destaca as contradições inerentes a esses fenômenos e seu conflito com coisas saudáveis; costuma provocar risadas; o final geralmente é feliz}
  \seealsoref{悲剧}{bei1 ju4}
\end{entry}

\begin{entry}{戏}{xi4}{6}{⼽}[HSK 5]
  \definition*{s.}{sobrenome Xi}
  \definition[场,部,出,台]{s.}{drama; peça; espetáculo; \emph{show}}
  \definition{v.}{brincar; praticar esportes; jogar | zombar; brincar; provocar}
\end{entry}

\begin{entry}{戏法}{xi4fa3}{6,8}{⼽、⽔}
  \definition{s.}{truque de mágica | prestidigitação}
\end{entry}

\begin{entry}{戏剧}{xi4ju4}{6,10}{⼽、⼑}[HSK 5]
  \definition{s.}{drama; peça; teatro | roteiro; peça; cenário}
\end{entry}

\begin{entry}{戏剧般}{xi4ju4ban1}{6,10,10}{⼽、⼑、⾈}
  \definition{adj.}{melodramático}
\end{entry}

\begin{entry}{戏剧编剧}{xi4ju4bian1ju4}{6,10,12,10}{⼽、⼑、⽷、⼑}
  \definition{s.}{dramaturgo}
\end{entry}

\begin{entry}{戏剧化地}{xi4ju4hua4di4}{6,10,4,6}{⼽、⼑、⼔、⼟}
  \definition{adv.}{dramaticamente | teatralmente}
\end{entry}

\begin{entry}{戏剧家}{xi4ju4jia1}{6,10,10}{⼽、⼑、⼧}
  \definition{s.}{dramaturgo}
\end{entry}

\begin{entry}{戏剧效果}{xi4ju4xiao4guo3}{6,10,10,8}{⼽、⼑、⽁、⽊}
  \definition{s.}{efeito dramático}
\end{entry}

\begin{entry}{戏剧性}{xi4ju4xing4}{6,10,8}{⼽、⼑、⼼}
  \definition{adj.}{dramático}
\end{entry}

\begin{entry}{戏剧演出}{xi4ju4yan3chu1}{6,10,14,5}{⼽、⼑、⽔、⼐}
  \definition{s.}{performance dramática}
\end{entry}

\begin{entry}{戏弄}{xi4nong4}{6,7}{⼽、⼶}
  \definition{v.}{zombar de | pregar peças | provocar}
\end{entry}

\begin{entry}{戏耍}{xi4shua3}{6,9}{⼽、⽽}
  \definition{v.}{divertir-me | brincar com | provocar}
\end{entry}

\begin{entry}{戏谑}{xi4xue4}{6,11}{⼽、⾔}
  \definition{v.}{brincar | fazer piadas | ridicularizar}
\end{entry}

\begin{entry}{戏院}{xi4yuan4}{6,9}{⼽、⾩}
  \definition{s.}{teatro}
\end{entry}

\begin{entry}{系}{xi4}{7}{⽷}[HSK 3,4]
  \definition*{s.}{sobrenome Xi}
  \definition{s.}{sistema; série | departamento; faculdade; unidades administrativas de ensino divididas por disciplina nas instituições de ensino superior}
  \definition{v.}{relacionar-se com; suportar; depender de | sentir-se ansioso; estar preocupado | amarrar; prender | ser; expressa julgamento, equivalente a 是}
  \seeref{系}{ji4}
  \seealsoref{是}{shi4}
\end{entry}

\begin{entry}{系列}{xi4lie4}{7,6}{⽷、⼑}[HSK 4]
  \definition{s.}{série; conjunto; conjunto de coisas relacionadas (matemática)}
\end{entry}

\begin{entry}{系囚}{xi4qiu2}{7,5}{⽷、⼞}
  \definition{s.}{prisioneiro}
\end{entry}

\begin{entry}{系统}{xi4tong3}{7,9}{⽷、⽷}[HSK 4]
  \definition{adj.}{sistemático; organizado}
  \definition[个]{s.}{sistema; relação de tipos semelhantes (ou seja, grupo de coisas semelhantes)}
\end{entry}

\begin{entry}{细}{xi4}{8}{⽷}[HSK 4]
  \definition{adj.}{fino; delgado; esguio; esbelto; em oposição a 粗 | fino; em partículas pequenas; grãos pequenos | fino e macio;  um sussuro | fino; requintado; delicado | cuidadoso; detalhado; meticuloso | ínfimo; minúsculo; insignificante; diminuto | jovem; pequeno}
  \seealsoref{粗}{cu1}
\end{entry}

\begin{entry}{细节}{xi4jie2}{8,5}{⽷、⾋}[HSK 4]
  \definition{s.}{detalhe; particularidade; aspectos secundários ou partes sutis de um enredo ou episódios secundários usados em uma obra literária para expressar o caráter de uma pessoa ou as características essenciais de uma coisa}
\end{entry}

\begin{entry}{细菌战}{xi4jun1zhan4}{8,11,9}{⽷、⾋、⼽}
  \definition{s.}{guerra biológica}
\end{entry}

\begin{entry}{细致}{xi4zhi4}{8,10}{⽷、⾄}[HSK 4]
  \definition{adj.}{meticuloso; cuidadoso; minucioso | intrincado; delicado}
\end{entry}

\begin{entry}{虾}{xia1}{9}{⾍}
  \definition{s.}{camarão}
\end{entry}

\begin{entry}{狭}{xia2}{9}{⽝}
  \definition{adj.}{estreito}
\end{entry}

\begin{entry}{下}{xia4}{3}{⼀}[HSK 1,2]
  \definition{clas.}{número de vezes usado para a ação | volume de um contêiner; quantidade de objetos que cabem em um utensílio | usado depois de 两 e 几 para expressar habilidade, capacidade, destreza}
  \definition{s.}{abaixo | próximo; último; segundo; referindo-se ao que está por vir ou ao que vem em seguida | mais baixo; inferior; de baixo nível ou grau | próximo; último; segundo; em ordem ou em ordem cronológica | indica pertencer a uma determinada faixa, situação, condição, etc. | indica uma determinada época ou estação | usado após um número para indicar posição ou direção | para baixo (após uma preposição) | sob (depois de um substantivo) | para baixo (antes de um verbo)}
  \definition{v.}{desembarcar; descer; sair | cair (chuva, neve, etc.) | enviar; emitir; entregar | ir para | sair; partir; retirar-se | lançar; colocar | descarregar; desmontar; tirar (fora) | formar (uma opinião, ideia, etc.); tomar decisões, fazer julgamentos, etc. | usar; aplicar | dar à luz (animais) | tomar; capturar; conquistar | ceder | terminar; deixar de lado; terminar o trabalho ou os estudos diários na hora prevista | para negação; ser inferior a; ser menor que}
  \seealsoref{几}{ji3}
  \seealsoref{两}{liang3}
\end{entry}

\begin{entry}{下巴}{xia4ba5}{3,4}{⼀、⼰}
  \definition[个]{s.}{queixo}
\end{entry}

\begin{entry}{下班}{xia4 ban1}{3,10}{⼀、⽟}[HSK 1]
  \definition{v.+compl.}{sair do trabalho; bater ponto; terminar o trabalho na hora prevista e sair do local de trabalho}
\end{entry}

\begin{entry}{下边}{xia4 bian5}{3,5}{⼀、⾡}[HSK 1]
  \definition{s.}{abaixo; sob; por baixo | próximo em ordem; seguinte | nível inferior; subordinado | a parte inferior}
\end{entry}

\begin{entry}{下车}{xia4 che1}{3,4}{⼀、⾞}[HSK 1]
  \definition{v.}{descer ou sair de (um ônibus, trem, carro etc.)}
\end{entry}

\begin{entry}{下次}{xia4 ci4}{3,6}{⼀、⽋}[HSK 1]
  \definition{s.}{na próxima vez; na próxima oportunidade ou no próximo evento}
\end{entry}

\begin{entry}{下蛋}{xia4dan4}{3,11}{⼀、⾍}
  \definition{v.}{botar ovos}
\end{entry}

\begin{entry}{下个月}{xia4 ge4 yue4}{3,3,4}{⼀、⼈、⽉}[HSK 4]
  \definition{s.}{próximo mês; mês que vem; refere-se ao próximo mês do mês atual}
\end{entry}

\begin{entry}{下海}{xia4hai3}{3,10}{⼀、⽔}
  \definition{v.+compl.}{ir para o mar; (barco) deixar o porto e iniciar uma jornada | ir pescar no mar | tornar-se ator profissional}
\end{entry}

\begin{entry}{下降}{xia4 jiang4}{3,8}{⼀、⾩}[HSK 4]
  \definition{v.}{cair; despencar; declinar; descer; diminuir; ir para baixo}
\end{entry}

\begin{entry}{下课}{xia4 ke4}{3,10}{⼀、⾔}[HSK 1]
  \definition{v.+compl.}{terminar a aula; sair da aula}
\end{entry}

\begin{entry}{下来}{xia4 lai5}{3,7}{⼀、⽊}[HSK 3]
  \definition{part.}{usado após o verbo, indica que a ação ou o comportamento se dirige para a posição do falante ou que a ação é contínua ou concluída | usado após um adjetivo, indica que uma determinada situação começou a ocorrer e continuará a se desenvolver}
  \definition{v.}{descer (para a minha localização) | (colheitas/frutas/vegetais, etc.) ser colhido; estar maduro o suficiente para ser colhido | (período de tempo) acabar; passar; chegar ao fim; indicar o fim de um período de tempo}
\end{entry}

\begin{entry}{下楼}{xia4 lou2}{3,13}{⼀、⽊}[HSK 4]
  \definition{v.}{descer as escadas}
\end{entry}

\begin{entry}{下面}{xia4 mian4}{3,9}{⼀、⾯}[HSK 3]
  \definition{s.}{em baixo; abaixo; parte de baixo | próximo; seguinte; a parte posterior; a parte posterior de um artigo ou discurso em relação ao que está sendo narrado no momento | subordinado; o nível inferior; homens nos níveis inferiores | por baixo}
\end{entry}

\begin{entry}{下去}{xia4 qu4}{3,5}{⼀、⼛}[HSK 3]
  \definition{part.}{usado depois de verbos para indicar de alto a baixo | usado depois de um verbo para indicar continuação}
  \definition{v.}{descer; baixar (a partir da minha localização) | (após um verbo) continuar (fazendo algo); prosseguir | usado após o verbo, indica uma descida de um ponto alto para um ponto baixo | usado após o verbo, indica continuidade | usado após um adjetivo, indica que o grau continua aumentando}
\end{entry}

\begin{entry}{下水道}{xia4shui3dao4}{3,4,12}{⼀、⽔、⾡}
  \definition{s.}{esgoto}
\end{entry}

\begin{entry}{下午}{xia4wu3}{3,4}{⼀、⼗}[HSK 1]
  \definition[个]{s.}{tarde; \emph{post meridiem} (p.m.); refere-se ao período entre o meio-dia e o pôr do sol}
\end{entry}

\begin{entry}{下午茶}{xia4wu3cha2}{3,4,9}{⼀、⼗、⾋}
  \definition{s.}{chá da tarde (normalmente chás com doces)}
\end{entry}

\begin{entry}{下线}{xia4xian4}{3,8}{⼀、⽷}
  \definition{v.}{ficar \emph{offline} | (um produto) sair da linha de montagem | pessoa abaixo de si em um esquema de pirâmide}
\end{entry}

\begin{entry}{下雪}{xia4 xue3}{3,11}{⼀、⾬}[HSK 2]
  \definition{v.+compl.}{nevar}
\end{entry}

\begin{entry}{下旬}{xia4xun2}{3,6}{⼀、⽇}
  \definition{adv.}{última dezena do mês}
\end{entry}

\begin{entry}{下雨}{xia4 yu3}{3,8}{⼀、⾬}[HSK 1]
  \definition{v.+compl.}{chover}
\end{entry}

\begin{entry}{下载}{xia4zai3}{3,10}{⼀、⾞}[HSK 4]
  \definition{v.}{\emph{download}; baixar; salvar informações da \emph{Web} em um dispositivo, como um computador}
\end{entry}

\begin{entry}{下崽}{xia4zai3}{3,12}{⼀、⼭}
  \definition{v.}{dar à luz (animais) | parir}
\end{entry}

\begin{entry}{下周}{xia4 zhou1}{3,8}{⼀、⼝}[HSK 2]
  \definition{s.}{próxima semana}
\end{entry}

\begin{entry}{吓}{xia4}{6}{⼝}[HSK 5]
  \definition{interj.}{interjeição que demonstra espanto; Interjeição que expressa insatisfação}
  \definition{v.}{ameaçar; intimidar; usar ameaças ou meios coercitivos para intimidar ou assustar}
\end{entry}

\begin{entry}{吓人}{xia4ren2}{6,2}{⼝、⼈}
  \definition{adj.}{apavorante | assustador}
  \definition{v.+compl.}{assustar-se | tomar um susto}
\end{entry}

\begin{entry}{夏}{xia4}{10}{⼢}
  \definition*{s.}{sobrenome Xia}
  \definition*{s.}{Dinastia Xia (c.2070-1600 a.C.) | China; refere-se à China}
  \definition{s.}{verão}
\end{entry}

\begin{entry}{夏季}{xia4 ji4}{10,8}{⼢、⼦}[HSK 4]
  \definition{s.}{verão; segundo trimestre do ano, habitualmente chamado na China de período de três meses, do início do verão ao início do outono, também chamado de ``quarto, quinto e sexto'' meses do calendário lunar}
\end{entry}

\begin{entry}{夏日}{xia4ri4}{10,4}{⼢、⽇}
  \definition{s.}{horário de verão}
\end{entry}

\begin{entry}{夏天}{xia4 tian1}{10,4}{⼢、⼤}[HSK 2]
  \definition[个]{s.}{verão}
\end{entry}

\begin{entry}{仙}{xian1}{5}{⼈}
  \definition{s.}{imortal}
\end{entry}

\begin{entry}{先}{xian1}{6}{⼉}[HSK 1]
  \definition*{s.}{sobrenome Xian}
  \definition{adv.}{primeiro; antes; mais cedo; com antecedência | no momento; por enquanto; em um curto espaço de tempo; temporariamente}
  \definition{s.}{início; começo; em ordem cronológica ou de precedência | ancestral; geração mais velha; antepassado | tardio; falecido; morto (honrar os mortos)}
\end{entry}

\begin{entry}{先不先}{xian1bu4xian1}{6,4,6}{⼉、⼀、⼉}
  \definition{adv.}{(dialeto) antes de tudo | em primeiro lugar}
\end{entry}

\begin{entry}{先到先得}{xian1dao4xian1de2}{6,8,6,11}{⼉、⼑、⼉、⼻}
  \definition{expr.}{primeiro a chegar | primeiro a ser servido}
\end{entry}

\begin{entry}{先后}{xian1 hou4}{6,6}{⼉、⼝}[HSK 5]
  \definition{adv.}{sucessivamente; um após o outro}
  \definition{s.}{prioridade; ordem; cedo ou tarde; primeiro e último}
\end{entry}

\begin{entry}{先进}{xian1jin4}{6,7}{⼉、⾡}[HSK 3]
  \definition{adj.}{avançado; progressos rápidos e nível elevado, podendo servir de exemplo a seguir}
  \definition{s.}{indivíduos ou grupos avançados}
\end{entry}

\begin{entry}{先烈}{xian1lie4}{6,10}{⼉、⽕}
  \definition{s.}{mártir}
\end{entry}

\begin{entry}{先期}{xian1qi1}{6,12}{⼉、⽉}
  \definition{adv.}{antecipadamente}
  \definition{s.}{prematuro | \emph{front-end}}
\end{entry}

\begin{entry}{先前}{xian1qian2}{6,9}{⼉、⼑}[HSK 5]
  \definition[出]{s.}{antes; anteriormente; geralmente se refere ao passado ou a um certo tempo anterior}
\end{entry}

\begin{entry}{先生}{xian1sheng5}{6,5}{⼉、⽣}[HSK 1]
  \definition[个,位]{s.}{professor; títulos honoríficos para professores, médicos, etc. | marido; antigamente, referia-se ao marido de outra pessoa ou ao próprio marido (ambos com pronomes pessoais como determinantes) | médico; títulos usados para se referir aos médicos no passado | refere-se a pessoas cuja profissão envolve contar histórias, adivinhação, etc.; antigamente, era chamado de contador | senhor; \emph{sir}; título dado aos intelectuais}
\end{entry}

\begin{entry}{先天}{xian1tian1}{6,4}{⼉、⼤}
  \definition{adj.}{congênito | inato | natural}
  \definition{s.}{período embrionário}
\end{entry}

\begin{entry}{先验}{xian1yan4}{6,10}{⼉、⾺}
  \definition{adj.}{(filosofia) a priori}
\end{entry}

\begin{entry}{先有}{xian1you3}{6,6}{⼉、⽉}
  \definition{adj.}{preexistente | anterior}
\end{entry}

\begin{entry}{鲜}{xian1}{14}{⿂}[HSK 4]
  \definition*{s.}{sobrenome Xian}
  \definition{adj.}{fresco; novo; fresco (experiência, comida etc.) |brilhante; de cores vivas | saboroso; delicioso | exuberante; luxuriante}
  \definition{s.}{aves e animais recém-abatidos; vegetais recém-colhidos; frutas, etc. | alimentos aquáticos; geralmente, peixes vivos, camarões, etc., para alimentação}
  \seeref{鲜}{xian3}
\end{entry}

\begin{entry}{鲜花}{xian1 hua1}{14,7}{⿂、⾋}[HSK 4]
  \definition[朵,束,支,捧]{s.}{flor; flores frescas; flores bonitas e frescas}
\end{entry}

\begin{entry}{鲜明}{xian1ming2}{14,8}{⿂、⽇}[HSK 4]
  \definition{adj.}{brilhante (cor) | distinto; bem definido; nítido; claro; característico}
\end{entry}

\begin{entry}{鲜艳}{xian1yan4}{14,10}{⿂、⾊}[HSK 5]
  \definition{adj.}{de cores alegres; de cores brilhantes}
\end{entry}

\begin{entry}{闲}{xian2}{7}{⾨}[HSK 5]
  \definition{adj.}{ocioso; não ocupado; desocupado; sem coisas para fazer; sem atividades; tempo livre | desocupado; (casa, objeto, etc.) não em uso; ocioso | não oficial; não sério; não relacionado ao negócio}
  \definition{s.}{lazer; tempo livre}
\end{entry}

\begin{entry}{咸}{xian2}{9}{⼝}[HSK 4]
  \definition*{s.}{sobrenome Xian}
  \definition{adj.}{salgado; em conserva; sabor salgado}
  \definition{adv.}{todos; indica a totalidade de um intervalo, equivalente a 全 e 都}
  \seealsoref{都}{dou1}
  \seealsoref{全}{quan2}
\end{entry}

\begin{entry}{咸菜}{xian2cai4}{9,11}{⼝、⾋}
  \definition{s.}{legumes salgados | \emph{pickles}}
\end{entry}

\begin{entry}{咸淡}{xian2dan4}{9,11}{⼝、⽔}
  \definition{s.}{água salobra | grau de salinidade | salgado e sem sal (sabores)}
\end{entry}

\begin{entry}{咸肉}{xian2rou4}{9,6}{⼝、⾁}
  \definition{s.}{\emph{bacon} | carne curada com sal}
\end{entry}

\begin{entry}{咸涩}{xian2se4}{9,10}{⼝、⽔}
  \definition{s.}{ácido | salgado e amargo}
\end{entry}

\begin{entry}{咸水}{xian2shui3}{9,4}{⼝、⽔}
  \definition{s.}{salmora | água salgada}
\end{entry}

\begin{entry}{咸盐}{xian2yan2}{9,10}{⼝、⽫}
  \definition{s.}{(coloquial) sal | sal de mesa}
\end{entry}

\begin{entry}{咸鱼}{xian2yu2}{9,8}{⼝、⿂}
  \definition{s.}{peixe salgado}
\end{entry}

\begin{entry}{显}{xian3}{9}{⽇}[HSK 5]
  \definition*{s.}{sobrenome Xian}
  \definition{adj.}{aparente; óbvio; perceptível | ilustre e influente | evidente; óbvio}
  \definition{v.}{mostrar; exibir; manifestar | aparecer; mostrar; revelar}
\end{entry}

\begin{entry}{显得}{xian3de5}{9,11}{⽇、⼻}[HSK 3]
  \definition{v.}{parecer; aparecer; manifestar (alguma situação)}
\end{entry}

\begin{entry}{显然}{xian3ran2}{9,12}{⽇、⽕}[HSK 3]
  \definition{adj.}{claro; evidente; óbvio; fatos, verdades e outras coisas que são fáceis de descobrir, perceber ou sentir claramente}
\end{entry}

\begin{entry}{显示}{xian3shi4}{9,5}{⽇、⽰}[HSK 3]
  \definition{v.}{mostrar; manifestar-se claramente| exibir; ostentar}
\end{entry}

\begin{entry}{显著}{xian3zhu4}{9,11}{⽇、⽬}[HSK 4]
  \definition{adj.}{notável; significativo; notável; extraordinário; muito óbvio; muito claramente demonstrado; muito facilmente visto ou sentido}
\end{entry}

\begin{entry}{猃}{xian3}{10}{⽝}
  \definition{s.}{(arcaico) um tipo de cão com focinho longo}
\end{entry}

\begin{entry}{猃狁}{xian3yun3}{10,7}{⽝、⽝}
  \definition*{s.}{Termo da dinastia Zhou para uma tribo nômade do norte mais tarde chamou o Xiongnu (匈奴) nas dinastias Qin e Han}
  \seealsoref{匈奴}{xiong1nu2}
\end{entry}

\begin{entry}{鲜}{xian3}{14}{⿂}
  \definition{adj.}{raro; pouco; pequeno;}
  \definition{adv.}{raramente}
  \seeref{鲜}{xian1}
\end{entry}

\begin{entry}{见}{xian4}{4}{⾒}
  \definition{v.}{aparecer | também escrito como 现}
  \seeref{见}{jian4}
  \seealsoref{现}{xian4}
\end{entry}

\begin{entry}{县}{xian4}{7}{⼛}[HSK 4]
  \definition[个]{s.}{condado; unidade de divisão administrativa}
\end{entry}

\begin{entry}{现}{xian4}{8}{⾒}
  \definition{adj.}{presente | atual}
  \definition{v.}{aparecer}
  \seealsoref{见}{xian4}
\end{entry}

\begin{entry}{现场}{xian4chang3}{8,6}{⾒、⼟}[HSK 3]
  \definition[个,处]{s.}{local onde ocorreu o acidente, incidente ou desastre| local; ponto; local onde se realizam diretamente atividades como produção, apresentações e competições}
\end{entry}

\begin{entry}{现代}{xian4dai4}{8,5}{⾒、⼈}[HSK 3]
  \definition*{s.}{Hyundai, empresa sul-coreana}
  \definition{adj.}{moderno; contemporâneo; com características, estilo e conceitos modernos, refletindo a vanguarda, a moda e a inovação da atualidade}
  \definition{s.}{tempos modernos; era contemporânea; atualmente, na divisão cronológica da história da China, refere-se principalmente ao período desde o Movimento 4 de Maio até os dias atuais}
\end{entry}

\begin{entry}{现货}{xian4huo4}{8,8}{⾒、⾙}
  \definition{s.}{produtos à vista}
\end{entry}

\begin{entry}{现货的}{xian4huo4 de5}{8,8,8}{⾒、⾙、⽩}
  \definition{s.}{produtos em estoque}
\end{entry}

\begin{entry}{现金}{xian4jin1}{8,8}{⾒、⾦}[HSK 3]
  \definition[笔]{s.}{dinheiro; dinheiro vivo; moeda que pode ser usada diretamente | reserva de dinheiro em um banco; o dinheiro guardado no cofre do banco}
\end{entry}

\begin{entry}{现实}{xian4shi2}{8,8}{⾒、⼧}[HSK 3]
  \definition{adj.}{real; efetivo; verdadeiro; de acordo com circunstâncias objetivas}
  \definition[个]{s.}{realidade; factualidade; coisas que existem objetivamente}
\end{entry}

\begin{entry}{现象}{xian4xiang4}{8,11}{⾒、⾗}[HSK 3]
  \definition[个,种]{s.}{aparência (das coisas); fenômeno; a forma externa e as relações manifestadas pelas coisas em seu desenvolvimento e mudança}
\end{entry}

\begin{entry}{现有}{xian4 you3}{8,6}{⾒、⽉}[HSK 5]
  \definition{adj.}{agora disponível; existente}
  \definition{v.}{estar disponível agora; existir | (literário) ter em mãos; ter em posse}
\end{entry}

\begin{entry}{现在}{xian4zai4}{8,6}{⾒、⼟}[HSK 1]
  \definition{adv.}{agora; no momento; atualmente; neste momento, quando se fala, às vezes inclui um período de tempo mais ou menos longo antes ou depois da fala (diferente de 过去 ou 将来)}
  \seealsoref{过去}{guo4 qu4}
  \seealsoref{将来}{jiang1lai2}
\end{entry}

\begin{entry}{现抓}{xian4zhua1}{8,7}{⾒、⼿}
  \definition{v.}{improvisar}
\end{entry}

\begin{entry}{现状}{xian4zhuang4}{8,7}{⾒、⽝}[HSK 5]
  \definition{s.}{situação atual; situação atual}
\end{entry}

\begin{entry}{现做}{xian4zuo4}{8,11}{⾒、⼈}
  \definition{adj.}{fresco}
  \definition{v.}{fazer (comida) no local}
\end{entry}

\begin{entry}{线}{xian4}{8}{⽷}[HSK 3]
  \definition{clas.}{usado para coisas abstratas, o número é limitado a ``一''}
  \definition[根,个]{s.}{fio; corda; arame; objetos finos e longos feitos de seda, algodão, metal, etc. | linha; figura formada pelo movimento arbitrário de um ponto| feito de fio de algodão | algo em forma de linha, fio, etc. | rota de transporte; linha | linha de demarcação; limite; zona de fronteira; zona de transição | beira; borda | linha ideológica e política | pista; fio}
\end{entry}

\begin{entry}{线索}{xian4suo3}{8,10}{⽷、⽷}[HSK 5]
  \definition[条,个]{s.}{pista; fio; metáfora para o desenvolvimento das coisas ou a maneira de explorar um problema | fio; linha; refere-se ao contexto de desenvolvimento do enredo em obras literárias}
\end{entry}

\begin{entry}{线香}{xian4xiang1}{8,9}{⽷、⾹}
  \definition{s.}{bastão ou vareta de incenso}
\end{entry}

\begin{entry}{限制}{xian4zhi4}{8,8}{⾩、⼑}[HSK 4]
  \definition{s.}{limite; restrição; confinamento}
  \definition{v.}{limitar; adstringir; restringir; confinar; fechar em (sobre)}
\end{entry}

\begin{entry}{宪}{xian4}{9}{⼧}
  \definition*{s.}{sobrenome Xian}
  \definition{s.}{estatuto; decreto | constituição}
\end{entry}

\begin{entry}{宪法法院}{xian4fa3fa3yuan4}{9,8,8,9}{⼧、⽔、⽔、⾩}
  \definition{s.}{tribunal constitucional}
\end{entry}

\begin{entry}{宪政}{xian4zheng4}{9,9}{⼧、⽁}
  \definition{s.}{governo constitucional}
\end{entry}

\begin{entry}{宪制}{xian4zhi4}{9,8}{⼧、⼑}
  \definition{adj.}{constitucional}
  \definition{s.}{sistema de governo constitucional}
\end{entry}

\begin{entry}{陷}{xian4}{10}{⾩}
  \definition[个]{s.}{armadilha; cilada | defeito | deficiência; desvantagem}
  \definition{v.}{ficar preso (ou atolado); enredar | afundar; desabar | acusar falsamente; incriminar; armar | (de uma cidade, etc.) ser capturado; cair | ser enquadrado; ser capturado}
\end{entry}

\begin{entry}{陷入}{xian4ru4}{10,2}{⾩、⼊}
  \definition{v.}{afundar | ser pego em | pousar (em uma situação)}
\end{entry}

\begin{entry}{羡}{xian4}{12}{⽺}
  \definition{v.}{admirar; invejar}
\end{entry}

\begin{entry}{羡慕}{xian4mu4}{12,14}{⽺、⼼}
  \definition{v.}{invejar | admirar}
\end{entry}

\begin{entry}{献}{xian4}{13}{⽝}[HSK 5]
  \definition{v.}{oferecer; apresentar; dedicar; doar | mostrar; apresentar; exibir | exibir-se; mostrar-se para que os outros vejam}
\end{entry}

\begin{entry}{乡}{xiang1}{3}{⼄}[HSK 5]
  \definition[个]{s.}{país; campo; vilarejo; área rural | local de origem; vila ou cidade natal | município (uma unidade administrativa rural subordinada ao condado) | vila natal; cidade natal | terra ou local famoso por produzir algo}
\end{entry}

\begin{entry}{乡巴佬}{xiang1ba1lao3}{3,4,8}{⼄、⼰、⼈}
  \definition{s.}{aldeão | caipira}
\end{entry}

\begin{entry}{乡村}{xiang1 cun1}{3,7}{⼄、⽊}[HSK 5]
  \definition{adj.}{rural | rústico}
  \definition{s.}{vila; campo; área rural; principalmente envolvido na agricultura; áreas com distribuição populacional mais dispersa em relação às cidades}
\end{entry}

\begin{entry}{相}{xiang1}{9}{⽬}
  \definition*{s.}{sobrenome Xiang}
  \definition{adv.}{uns aos outros; mutuamente | (para uma ação realizada por uma pessoa em relação a outra) | indica a ação de uma parte em relação à outra parte}
  \definition{s.}{qualidade; substância}
  \definition{v.}{ver por si mesmo (se algo ou algo é do seu agrado)}
  \seeref{相}{xiang4}
\end{entry}

\begin{entry}{相比}{xiang1 bi3}{9,4}{⽬、⽐}[HSK 3]
  \definition{v.}{combinar; comparar com | comparar mutuamente, usar uma coisa como padrão, perceber as características de outra coisa ou obter uma opinião}
\end{entry}

\begin{entry}{相处}{xiang1chu3}{9,5}{⽬、⼡}[HSK 4]
  \definition{v.}{dar-se bem; viver juntos; dar-se bem (uns com os outros); viver uns com os outros; entrar em contato uns com os outros, tratar uns aos outros}
\end{entry}

\begin{entry}{相当}{xiang1dang1}{9,6}{⽬、⼹}[HSK 3]
  \definition{adj.}{adequado; apropriado}
  \definition{adv.}{bastante; razoavelmente; consideravelmente; indica um grau relativamente alto e profundo}
  \definition{v.}{combinar; equilibrar; corresponder a; ser aproximadamente igual a; ser proporcional a}
\end{entry}

\begin{entry}{相等}{xiang1deng3}{9,12}{⽬、⽵}[HSK 5]
  \definition{v.}{ser igual a; possuir a mesma quantidade, peso, tamanho e grau}
\end{entry}

\begin{entry}{相反}{xiang1fan3}{9,4}{⽬、⼜}[HSK 4]
  \definition{adj.}{oposto; contrário; dois aspectos das coisas são contraditórios e mutuamente exclusivos}
  \definition{conj.}{pelo contrário; usado no início ou no meio de uma frase para indicar uma contradição de significado com o que foi dito anteriormente.}
\end{entry}

\begin{entry}{相关}{xiang1guan1}{9,6}{⽬、⼋}[HSK 3]
  \definition{v.}{estar mutuamente relacionado; estar intimamente relacionado; estar inter-relacionado}
\end{entry}

\begin{entry}{相互}{xiang1 hu4}{9,4}{⽬、⼆}[HSK 3]
  \definition{adj.}{mútuo; recíproco; entre duas pessoas ou coisas}
  \definition{adv.}{mutuamente; um ao outro; tratamento recíproco}
\end{entry}

\begin{entry}{相聚}{xiang1ju4}{9,14}{⽬、⽿}
  \definition{v.}{reunir-se | montar}
\end{entry}

\begin{entry}{相亲}{xiang1qin1}{9,9}{⽬、⼇}
  \definition{s.}{encontro às cegas | entrevista arranjada para avaliar a proposta de um parceiro de casamento | apegar-se profundamente um ao outro}
\end{entry}

\begin{entry}{相思病}{xiang1si1bing4}{9,9,10}{⽬、⼼、⽧}
  \definition{s.}{saudade de amor}
\end{entry}

\begin{entry}{相似}{xiang1si4}{9,6}{⽬、⼈}[HSK 3]
  \definition{v.}{assemelhar-se; ser semelhante; ser parecido}
\end{entry}

\begin{entry}{相同}{xiang1tong2}{9,6}{⽬、⼝}[HSK 2]
  \definition{adj.}{semelhante; similar; igual; idêntico; o mesmo; consistentes entre si, sem diferença}
\end{entry}

\begin{entry}{相信}{xiang1xin4}{9,9}{⽬、⼈}[HSK 2]
  \definition{v.}{acreditar em; estar convencido de; ter fé em; acreditar que algo é certo ou verdadeiro sem dúvida}
\end{entry}

\begin{entry}{相宜}{xiang1yi2}{9,8}{⽬、⼧}
  \definition{adj.}{adequado | apropriado}
  \definition{v.}{ser adequado ou apropriado}
\end{entry}

\begin{entry}{相应}{xiang1ying4}{9,7}{⽬、⼴}[HSK 5]
  \definition{v.}{corresponder}
\end{entry}

\begin{entry}{相遇}{xiang1yu4}{9,12}{⽬、⾡}
  \definition{v.}{encontrar (reunião, encontro, etc.)}
\end{entry}

\begin{entry}{香}{xiang1}{9}{⾹}[HSK 3][Kangxi 186]
  \definition*{s.}{sobrenome Xiang}
  \definition{adj.}{aromático; perfumado; fragrante; cheiroso; oposto a 臭 | saboroso; saboroso; delicioso; apetitoso | com gosto; com bom apetite | (sono) profundo; dormir confortavelmente e tranquilamente | popular; valorizado; apreciado}
  \definition[根,炷]{s.}{especiaria; perfume; fragrância; aromatizante; substância com aroma intenso | incenso; bastão de incenso; tiras finas feitas de serragem e especiarias, queimadas em rituais para honrar os antepassados ou deuses e budas, e também usadas para afastar odores desagradáveis ou mosquitos| antigamente, referia-se a coisas relacionadas com mulheres ou mulheres}
  \seealsoref{臭}{chou4}
\end{entry}

\begin{entry}{香槟酒}{xiang1bin1jiu3}{9,14,10}{⾹、⽊、⾣}
  \definition[杯]{s.}{(empréstimo linguístico) \emph{champagne}}
\end{entry}

\begin{entry}{香波}{xiang1bo1}{9,8}{⾹、⽔}
  \definition{s.}{xampu}
\end{entry}

\begin{entry}{香肠}{xiang1chang2}{9,7}{⾹、⾁}[HSK 5]
  \definition[根]{s.}{salsicha; linguiça; alimento feito com intestino de porco, recheado com carne picada e temperos}
\end{entry}

\begin{entry}{香港}{xiang1gang3}{9,12}{⾹、⽔}
  \definition*{s.}{Hong Kong}
  \seealsoref{香港岛}{xiang1gang3 dao3}
\end{entry}

\begin{entry}{香港岛}{xiang1gang3 dao3}{9,12,7}{⾹、⽔、⼭}
  \definition*{s.}{Ilha de Hong Kong}
  \seealsoref{香港}{xiang1gang3}
\end{entry}

\begin{entry}{香蕉}{xiang1jiao1}{9,15}{⾹、⾋}[HSK 3]
  \definition[枝,根,个,把,串,束,弓]{s.}{banana}
\end{entry}

\begin{entry}{香炉}{xiang1lu2}{9,8}{⾹、⽕}
  \definition{s.}{incensário (para queimar incenso) | queimador de incenso | insensório, turíbulo}
\end{entry}

\begin{entry}{香气}{xiang1qi4}{9,4}{⾹、⽓}
  \definition{s.}{fragrância | aroma | incenso}
\end{entry}

\begin{entry}{香味}{xiang1wei4}{9,8}{⾹、⼝}
  \definition[股]{s.}{fragrância | cheiro doce}
\end{entry}

\begin{entry}{香蕈}{xiang1xun4}{9,15}{⾹、⾋}
  \definition{s.}{\emph{shiitake}, cogumelo comestível}
\end{entry}

\begin{entry}{香烟}{xiang1yan1}{9,10}{⾹、⽕}
  \definition[支,条]{s.}{cigarro | fumaça de incenso queimado}
\end{entry}

\begin{entry}{香艳}{xiang1yan4}{9,10}{⾹、⾊}
  \definition{adj.}{atraente | erótico | romântico}
\end{entry}

\begin{entry}{香皂}{xiang1zao4}{9,7}{⾹、⽩}
  \definition{s.}{sabonete | sabonete perfumado}
\end{entry}

\begin{entry}{箱}{xiang1}{15}{⾋}[HSK 4]
  \definition{s.}{caixa; estojo; baú | qualquer coisa no formato de caixa}
\end{entry}

\begin{entry}{箱子}{xiang1 zi5}{15,3}{⾋、⼦}[HSK 4]
  \definition[个,只]{s.}{baú; caixa; estojo; maleta; pasta executiva}
\end{entry}

\begin{entry}{详细}{xiang2xi4}{8,8}{⾔、⽷}[HSK 5]
  \definition{adj.}{explícito; detalhado; minucioso; circunstancial; meticuloso}
\end{entry}

\begin{entry}{享受}{xiang3shou4}{8,8}{⼇、⼜}[HSK 5]
  \definition[种]{s.}{prazer}
  \definition{v.}{aproveitar; desfrutar}
\end{entry}

\begin{entry}{响}{xiang3}{9}{⼝}[HSK 2]
  \definition{adj.}{barulhento; ressonante}
  \definition[声,阵]{s.}{som; ruído; barulho | eco}
  \definition{v.}{tocar; soar; ressoar; fazer um som | soar; fazer algo emitir um som}
\end{entry}

\begin{entry}{想}{xiang3}{13}{⼼}[HSK 1]
  \definition{v.}{pensar; ponderar; refletir | supor; contar; considerar; pensar; estimar | querer; gostaria de; sentir vontade (de fazer algo) | lembrar com saudade; sentir falta}
\end{entry}

\begin{entry}{想到}{xiang3 dao4}{13,8}{⼼、⼑}[HSK 2]
  \definition{v.}{pensar em; trazer à mente; ter no coração; ter uma ideia (na mente); ter uma ideia (no coração)}
\end{entry}

\begin{entry}{想法}{xiang3 fa3}{13,8}{⼼、⽔}[HSK 2]
  \definition[种]{s.}{ideia; opinião; pensamento; noção; o que alguém tem em mente; visões e opiniões sobre alguém ou algo obtidas através do pensamento}
  \definition{s.}{maneira de pensar | opinião | noção}
  \definition{v.}{tentar; pensar em uma maneira (de fazer algo); fazer o que puder; encontrar um jeito}
\end{entry}

\begin{entry}{想念}{xiang3nian4}{13,8}{⼼、⼼}[HSK 4]
  \definition{v.}{sentir falta; pensar em; lembrar com carinho; ficar doente por; desejar ver novamente; lembrar com saudade}
\end{entry}

\begin{entry}{想起}{xiang3 qi3}{13,10}{⼼、⾛}[HSK 2]
  \definition{v.}{recordar; lembrar; pensar em; trazer à mente; cruzar pelos pensamentos de alguém; passar pelo pensamento de alguém}
\end{entry}

\begin{entry}{想想看}{xiang3xiang3kan4}{13,13,9}{⼼、⼼、⽬}
  \definition{v.}{pensar sobre isso}
\end{entry}

\begin{entry}{想象}{xiang3xiang4}{13,11}{⼼、⾗}[HSK 4]
  \definition[个]{s.}{imaginação; refere-se ao processo mental de processamento e transformação de representações armazenadas na mente para formar novas imagens}
  \definition{v.}{imaginar; vislumbrar; visualizar; refere-se a ter uma imagem concreta de algo que não está na frente dos olhos}
\end{entry}

\begin{entry}{向}{xiang4}{6}{⼝}[HSK 2]
  \definition*{s.}{sobrenome Xiang}
  \definition{adv.}{sempre; o tempo todo}
  \definition{prep.}{em direção a; para}
  \definition{s.}{direção | a janela voltada para o norte}
  \definition{v.}{encarar; virar-se para | estar do lado de; ser parcial com; tomar o partido de alguém}
\end{entry}

\begin{entry}{向导}{xiang4dao3}{6,6}{⼝、⼨}[HSK 5]
  \definition{s.}{guia}
  \definition{v.}{agir como um guia; mostrar a alguém o caminho; levar alguém a algum lugar}
\end{entry}

\begin{entry}{向前}{xiang4 qian2}{6,9}{⼝、⼑}[HSK 5]
  \definition{adv.}{para frente; adiante;}
  \definition{v.}{avançar; ir em direção à frente; mover-se para frente; avançar um pouco mais}
\end{entry}

\begin{entry}{向上}{xiang4 shang4}{6,3}{⼝、⼀}[HSK 5]
  \definition{v.}{mover-se; subir; ir para um lugar mais alto; ir para um lugar mais alto em relação a um determinado ponto; ir para um desenvolvimento mais alto que o atual | avançar; continuar se aperfeiçoar; subir na vida; desenvolver-se em direção ao progresso}
\end{entry}

\begin{entry}{向汪}{xiang4wang1}{6,7}{⼝、⽔}
  \definition{v.}{esperar que}
\end{entry}

\begin{entry}{向往}{xiang4wang3}{6,8}{⼝、⼻}
  \definition{v.}{ansiar por | esperar ansiosamente por}
\end{entry}

\begin{entry}{相}{xiang4}{9}{⽬}
  \definition*{s.}{sobrenome Xiang}
  \definition{s.}{aparência | postura; porte; postura sentada, em pé, etc. | (física) fase; refere-se a uma parte homogênea de uma substância com a mesma composição e as mesmas propriedades físicas e químicas | fotografia | primeiro-ministro (na China antiga) | ministro; títulos oficiais de certos países | fácies marinha (carvão) | elefante, uma das peças do xadrez chinês | recepcionista (pessoa que ajuda o anfitrião a receber o hóspede); antigamente, referia-se a alguém que ajudava o anfitrião a receber convidados}
  \definition{v.}{olhar e avaliar; observe a aparência das coisas; julgar sua qualidade | assistir; ajudar; auxiliar}
  \seeref{相}{xiang1}
\end{entry}

\begin{entry}{相机}{xiang4 ji1}{9,6}{⽬、⽊}[HSK 2]
  \definition[台,部,架,个]{s.}{câmera; máquina fotográfica}
  \definition{v.}{ficar atento a uma oportunidade; procurar oportunidades}
\end{entry}

\begin{entry}{相片}{xiang4 pian4}{9,4}{⽬、⽚}[HSK 4]
  \definition[张]{s.}{foto; fotografia; uma imagem de uma pessoa ou objeto feita pela exposição de papel fotográfico a um negativo fotográfico e, em seguida, revelando e fixando a imagem.}
\end{entry}

\begin{entry}{相声}{xiang4sheng5}{9,7}{⽬、⼠}[HSK 5]
  \definition[个,段]{s.}{conversa cruzada; diálogo cômico; forma de performance humorística, em que os atores usam piadas, canções e imitações para satirizar e elogiar}
\end{entry}

\begin{entry}{项}{xiang4}{9}{⾴}[HSK 4]
  \definition*{s.}{sobrenome Xiang}
  \definition{clas.}{para itens discriminados; taxonomia}
  \definition{s.}{nuca (do pescoço); a parte de trás do pescoço | soma (de dinheiro); fundos para fins especiais | termo; em álgebra, significa uma única fórmula que não é unida por um sinal de mais ou de menos | item}
\end{entry}

\begin{entry}{项目}{xiang4mu4}{9,5}{⾴、⽬}[HSK 4]
  \definition{s.}{evento | item; projeto; trabalhos de engenharia, acadêmicos, etc., de conteúdo específico}
\end{entry}

\begin{entry}{象征}{xiang4zheng1}{11,8}{⾗、⼻}[HSK 5]
  \definition[种]{s.}{símbolo; emblema; insígnia; \emph{token}; objeto concreto que simboliza um significado especial}
  \definition{v.}{simbolizar; significar; representar; expressar um significado especial através de algo concreto}
\end{entry}

\begin{entry}{像}{xiang4}{13}{⼈}[HSK 2]
  \definition{adv.}{parecer; parecer como se}
  \definition{s.}{imagem; retrato; semelhança a alguém | imagem}
  \definition{v.}{assemelhar-se; ser como; parecer-se com | ser como; ser tal como}
\end{entry}

\begin{entry}{消除}{xiao1chu2}{10,9}{⽔、⾩}[HSK 5]
  \definition{v.}{dissipar; remover; eliminar; limpar; enxugar}
\end{entry}

\begin{entry}{消毒}{xiao1du2}{10,9}{⽔、⽏}[HSK 5]
  \definition{v.}{desinfetar; esterilizar; matar os microrganismos causadores de doenças por meios físicos ou químicos}
\end{entry}

\begin{entry}{消防}{xiao1fang2}{10,6}{⽔、⾩}[HSK 5]
  \definition{s.}{combate a incêncios; controle de incêndios}
\end{entry}

\begin{entry}{消防员}{xiao1fang2yuan2}{10,6,7}{⽔、⾩、⼝}
  \definition{s.}{bombeiro}
\end{entry}

\begin{entry}{消费}{xiao1fei4}{10,9}{⽔、⾙}[HSK 3]
  \definition{v.}{gastar; consumir; consumir materiais para satisfazer as necessidades de produção ou de vida (geralmente refere-se ao consumo doméstico) | consumir (recursos naturais)}
\end{entry}

\begin{entry}{消费者}{xiao1 fei4 zhe3}{10,9,8}{⽔、⾙、⽼}[HSK 5]
  \definition{s.}{consumidor; cliente; consumo; indivíduos membros da sociedade que compram e utilizam bens e serviços para consumo pessoal}
\end{entry}

\begin{entry}{消化}{xiao1hua4}{10,4}{⽔、⼔}[HSK 4]
  \definition{v.}{digerir (alimentos) | digerir (conhecimento); pensar e absorver; uma metáfora para a compreensão total de novos conhecimentos ou informações e a capacidade de transformá-los em algo que possa ser usado}
\end{entry}

\begin{entry}{消极}{xiao1ji2}{10,7}{⽔、⽊}[HSK 5]
  \definition{adj.}{negativo; oposto; adverso | passivo; inativo; sem ambição; sem iniciativa; desanimado; apático}
\end{entry}

\begin{entry}{消失}{xiao1shi1}{10,5}{⽔、⼤}[HSK 3]
  \definition{v.}{desaparecer; desvanecer; dissolver; dissipar; evaporar; sumir}
\end{entry}

\begin{entry}{消息}{xiao1xi5}{10,10}{⽔、⼼}[HSK 3]
  \definition[个,条,篇,些]{s.}{notícias; informação; reportagem sobre pessoas ou situações | notícias; novidades;}
\end{entry}

\begin{entry}{销售}{xiao1shou4}{12,11}{⾦、⼝}[HSK 4]
  \definition{v.}{vender; comercializar}
\end{entry}

\begin{entry}{嚣张}{xiao1zhang1}{18,7}{⼝、⼸}
  \definition{adj.}{desenfreado | arrogante | agressivo}
\end{entry}

\begin{entry}{小}{xiao3}{3}{⼩}[HSK 1,2][Kangxi 42]
  \definition*{s.}{sobrenome Xiao}
  \definition{adj.}{menor; pequeno; insignificante; pouco; volume, área, quantidade, intensidade, etc. não são grandes | jovem | expressões humildes, referindo-se a si mesmo ou a pessoas ou coisas relacionadas a si mesmo | por um tempo; por um curto período; por um curto período de tempo | o mais novo; o último na ordem de antiguidade; em último lugar na classificação}
  \definition{pref.}{usado antes do sobrenome, nome, posição na família, etc.}
  \definition{s.}{os jovens; pessoas mais jovens | concubina}
\end{entry}

\begin{entry}{小白菜}{xiao3bai2cai4}{3,5,11}{⼩、⽩、⾋}
  \definition[棵]{s.}{\emph{bok choy} | couve chinesa}
\end{entry}

\begin{entry}{小吃}{xiao3chi1}{3,6}{⼩、⼝}[HSK 4]
  \definition{s.}{lanche; petiscos; comida com especialidades locais, não muito para uma porção | prato frio; prato feito; cortes de frios na culinária ocidental | pratos pequenos e baratos; pratos simples em restaurantes com porções pequenas e preços baixos}
\end{entry}

\begin{entry}{小狗}{xiao3 gou3}{3,8}{⼩、⽝}
  \definition{s.}{filhote de cachorro}
\end{entry}

\begin{entry}{小孩儿}{xiao3hai2r5}{3,9,2}{⼩、⼦、⼉}[HSK 1]
  \definition[个]{s.}{criança; bebê}
\end{entry}

\begin{entry}{小伙子}{xiao3huo3zi5}{3,6,3}{⼩、⼈、⼦}[HSK 4]
  \definition[个]{s.}{rapaz jovem; jovem colega}
\end{entry}

\begin{entry}{小姐}{xiao3jie5}{3,8}{⼩、⼥}[HSK 1]
  \definition[个,位]{s.}{jovem senhora; anteriormente, era assim que se referiam às filhas de famílias ricas. | senhorita; título honorífico para mulheres jovens | (gíria) prostituta}
\end{entry}

\begin{entry}{小朋友}{xiao3 peng2 you3}{3,8,4}{⼩、⽉、⼜}[HSK 1]
  \definition[个]{s.}{criança; crianças; refere-se a crianças e adolescentes | (termo usado por um adulto para se dirigir a uma criança) amiguinho; menino (ou menina); termo carinhoso para se referir a crianças e adolescentes}
\end{entry}

\begin{entry}{小气鬼}{xiao3qi4gui3}{3,4,9}{⼩、⽓、⿁}
  \definition{adj.}{avarento | mão-de-vaca | miserável | pão-duro}
\end{entry}

\begin{entry}{小区}{xiao3qu1}{3,4}{⼩、⼖}
  \definition{s.}{conjunto habitacional, comunidade, bairro | célula (telecomunicações)}
\end{entry}

\begin{entry}{小声}{xiao3 sheng1}{3,7}{⼩、⼠}[HSK 2]
  \definition{v.}{falar em voz baixa; falar baixinho; sussurar}
\end{entry}

\begin{entry}{小时}{xiao3shi2}{3,7}{⼩、⽇}[HSK 1]
  \definition{clas.}{hora; unidade de medida legal do tempo, 1 hora equivale a 60 minutos, é 1/24 de um dia}
  \definition[个]{s.}{hora; refere-se a um período de uma hora}
\end{entry}

\begin{entry}{小时候}{xiao3 shi2 hou5}{3,7,10}{⼩、⽇、⼈}[HSK 2]
  \definition{s.}{na infância; quando alguém era jovem; refere-se à infância}
\end{entry}

\begin{entry}{小树}{xiao3shu4}{3,9}{⼩、⽊}
  \definition[棵]{s.}{muda | arbusto | árvore pequena}
\end{entry}

\begin{entry}{小说}{xiao3shuo1}{3,9}{⼩、⾔}[HSK 2]
  \definition[本,部,篇,章]{s.}{história; romance; ficção; uma forma literária que reflete a vida social por meio da descrição de personagens, ambiente e enredo}
\end{entry}

\begin{entry}{小偷儿}{xiao3 tou1er5}{3,11,2}{⼩、⼈、⼉}[HSK 5]
  \definition{s.}{ladrão insignificante (ou furtivo); ladrãozinho | ladrão}
\end{entry}

\begin{entry}{小腿}{xiao3tui3}{3,13}{⼩、⾁}
  \definition{s.}{perna (do joelho ao calcanhar) | haste}
\end{entry}

\begin{entry}{小屋}{xiao3wu1}{3,9}{⼩、⼫}
  \definition{s.}{cabana | chalé | cabine}
\end{entry}

\begin{entry}{小小}{xiao3xiao3}{3,3}{⼩、⼩}
  \definition{adj.}{muito pequeno}
\end{entry}

\begin{entry}{小心}{xiao3xin1}{3,4}{⼩、⼼}[HSK 2]
  \definition{adj.}{cuidadoso; atento; com cautela}
  \definition{v.}{ter cuidado; ser cauteloso; estar atento; tomar cuidado; prestar atenção}
\end{entry}

\begin{entry}{小型}{xiao3 xing2}{3,9}{⼩、⼟}[HSK 4]
  \definition{adj.}{de tamanho pequeno; em pequena escala; miniatura; tipo pequeno; tamanho de bolso; tipo compacto}
  \definition{s.}{(Mediterrâneo) escunas, pequenos veleiros de pesca ou turismo | pequeno \emph{rover} lunar (duas pessoas)}
\end{entry}

\begin{entry}{小学}{xiao3 xue2}{3,8}{⼩、⼦}[HSK 1]
  \definition[个]{s.}{escola primária (ou fundamental); escolas que oferecem ensino fundamental básico | estudos filológicos; antigamente, referia-se ao estudo da escrita, da fonética e da exegese}
\end{entry}

\begin{entry}{小学生}{xiao3 xue2 sheng1}{3,8,5}{⼩、⼦、⽣}[HSK 1]
  \definition{s.}{aluno; estudante; estudante do sexo masculino (男); estudante do sexo feminino (女) | um aluno mais novo (do que os outros da sua turma) | (dialeto) um menino pequeno}
  \seealsoref{男}{nan2}
  \seealsoref{女}{nv3}
\end{entry}

\begin{entry}{小洋白菜}{xiao3 yang2bai2cai4}{3,9,5,11}{⼩、⽔、⽩、⾋}
  \definition{s.}{couve de bruxelas}
\end{entry}

\begin{entry}{小众}{xiao3zhong4}{3,6}{⼩、⼈}
  \definition{s.}{minoria da população | nicho (mercado, etc.)}
\end{entry}

\begin{entry}{小组}{xiao3 zu3}{3,8}{⼩、⽷}[HSK 2]
  \definition[个,名,位]{s.}{grupo; um pequeno grupo de pessoas}
\end{entry}

\begin{entry}{哮喘}{xiao4chuan3}{10,12}{⼝、⼝}
  \definition{s.}{asma}
\end{entry}

\begin{entry}{效果}{xiao4guo3}{10,8}{⽁、⽊}[HSK 3]
  \definition[种,个]{s.}{efeito; resultado | efeitos sonoros; vários sons ou fenômenos naturais criados para combinar com o enredo em dramas e filmes, como vento e chuva, tiros, fogo, neve, etc.}
\end{entry}

\begin{entry}{效率}{xiao4lv4}{10,11}{⽁、⽞}[HSK 4]
  \definition[种]{s.}{eficiência; produtividade}
\end{entry}

\begin{entry}{校}{xiao4}{10}{⽊}
  \definition[所]{s.}{oficial militar | escola}
  \seeref{校}{jiao4}
\end{entry}

\begin{entry}{校服}{xiao4fu2}{10,8}{⽊、⽉}
  \definition{s.}{uniforme escolar}
\end{entry}

\begin{entry}{校规}{xiao4gui1}{10,8}{⽊、⾒}
  \definition{s.}{regras e regulamentos escolares}
\end{entry}

\begin{entry}{校监}{xiao4jian1}{10,10}{⽊、⽫}
  \definition{s.}{diretor | supervisor (de escola)}
\end{entry}

\begin{entry}{校园}{xiao4 yuan2}{10,7}{⽊、⼞}[HSK 2]
  \definition[个]{s.}{campus; pátio da escola; refere-se a todos os terrenos e edifícios dentro da área escolar}
\end{entry}

\begin{entry}{校长}{xiao4zhang3}{10,4}{⽊、⾧}[HSK 2]
  \definition[个,位,名]{s.}{diretor; presidente; reitor; o mais alto líder administrativo e empresarial de uma escola}
\end{entry}

\begin{entry}{笑}{xiao4}{10}{⽵}[HSK 1]
  \definition{adj.}{ridículo; engraçado; risível; hilário}
  \definition{v.}{sorrir; rir; mostrar expressão de alegria; emitir sons de alegria | ridicularizar; rir de; zombar}
\end{entry}

\begin{entry}{笑话儿}{xiao4 hua4r5}{10,8,2}{⽵、⾔、⼉}[HSK 2]
  \definition{s.}{piada; brincadeira; gracejo}
\end{entry}

\begin{entry}{笑话}{xiao4hua5}{10,8}{⽵、⾔}[HSK 2]
  \definition[个]{s.}{piada; brincadeira; uma conversa ou história que faz as pessoas rirem; algo que as pessoas usam como piada}
  \definition{v.}{ridicularizar; zombar; rir de;}
\end{entry}

\begin{entry}{笑容}{xiao4rong2}{10,10}{⽵、⼧}
  \definition[副]{s.}{sorriso | expressão sorridente}
\end{entry}

\begin{entry}{些}{xie1}{8}{⼆}[HSK 4]
  \definition{adv.}{um pouco; um pouco mais; usado após um adjetivo ou parte de um verbo para indicar uma pequena quantidade, equivalente a 一点儿}
  \definition{clas.}{alguns; um pouco; denota uma quantidade indefinida}
  \seealsoref{一点儿}{yi4dian3r5}
\end{entry}

\begin{entry}{些许}{xie1xu3}{8,6}{⼆、⾔}
  \definition{num.}{um pouco}
\end{entry}

\begin{entry}{歇}{xie1}{13}{⽋}[HSK 5]
  \definition*{s.}{sobrenome Xie}
  \definition{s.}{um pouco de tempo}
  \definition{v.}{descansar; fazer uma pausa | parar (o trabalho); encerrar o expediente | dormir; ir para a cama}
\end{entry}

\begin{entry}{协议}{xie2yi4}{6,5}{⼗、⾔}[HSK 5]
  \definition[项]{s.}{acordo; tratado; decisão conjunta alcançada através de negociação e consulta}
  \definition{v.}{concordar em}
\end{entry}

\begin{entry}{协议书}{xie2 yi4 shu1}{6,5,4}{⼗、⾔、⼄}[HSK 5]
  \definition{s.}{contrato | protocolo}
\end{entry}

\begin{entry}{斜}{xie2}{11}{⽃}[HSK 5]
  \definition{adj.}{oblíquo; inclinado | enviesado; chanfrado; diagonal; torto; nem paralelo nem perpendicular a um plano ou linha}
  \definition{v.}{virar de lado; inclinar}
\end{entry}

\begin{entry}{斜阳}{xie2yang2}{11,6}{⽃、⾩}
  \definition{s.}{sol poente}
\end{entry}

\begin{entry}{谐}{xie2}{11}{⾔}
  \definition{adj.}{harmonioso | humorístico}
\end{entry}

\begin{entry}{鞋}{xie2}{15}{⾰}[HSK 2]
  \definition[双,只]{s.}{sapatos; usado nos pés; algo que toca o chão ao caminhar; sem cano alto}
\end{entry}

\begin{entry}{写}{xie3}{5}{⼍}[HSK 1]
  \definition{v.}{escrever | compor; escrever (como autor, repórter, etc.) | descrever; retratar | pintar; desenhar | expressar a imagem das coisas através da linguagem e da escrita | desenhar (pintura)}
\end{entry}

\begin{entry}{写意}{xie3yi4}{5,13}{⼍、⼼}
  \definition{s.}{estilo de pintura chinesa à mão livre, caracterizado por traços ousados em vez de detalhes precisos}
  \definition{v.}{sugerir (em vez de descrever em detalhes)}
  \seeref{写意}{xie4yi4}
\end{entry}

\begin{entry}{写照}{xie3zhao4}{5,13}{⼍、⽕}
  \definition{s.}{retrato}
\end{entry}

\begin{entry}{写真}{xie3zhen1}{5,10}{⼍、⼗}
  \definition{s.}{retrato}
  \definition{v.}{descrever algo com precisão}
\end{entry}

\begin{entry}{写字}{xie3zi4}{5,6}{⼍、⼦}
  \definition{v.}{escrever (à mão) | praticar caligrafia}
\end{entry}

\begin{entry}{写字匠}{xie3zi4 jiang4}{5,6,6}{⼍、⼦、⼕}
  \definition{s.}{calígrafo}
\end{entry}

\begin{entry}{写作}{xie3zuo4}{5,7}{⼍、⼈}[HSK 3]
  \definition{v.}{escrever artigos; escrever livros, etc.; também se refere especificamente à criação de obras literárias}
\end{entry}

\begin{entry}{血}{xie3}{6}{⾎}[Kangxi 143]
  \seeref{血}{xue4}
\end{entry}

\begin{entry}{写意}{xie4yi4}{5,13}{⼍、⼼}
  \definition{adj.}{confortável | agradável | relaxado}
  \seeref{写意}{xie3yi4}
\end{entry}

\begin{entry}{泄气}{xie4qi4}{8,4}{⽔、⽓}
  \definition{adj.}{decepcionante | frustrante | patético}
  \definition{v.+compl.}{perder o coração | sentir-se desencorajado | ficar desanimado}
\end{entry}

\begin{entry}{谢病}{xie4bing4}{12,10}{⾔、⽧}
  \definition{v.}{desculpar-se por causa de doença}
\end{entry}

\begin{entry}{谢恩}{xie4'en1}{12,10}{⾔、⼼}
  \definition{v.}{agradecer a alguém pelo favor (especialmente imperador ou oficial superior)}
\end{entry}

\begin{entry}{谢媒}{xie4mei2}{12,12}{⾔、⼥}
  \definition{v.}{agradecer ao casamenteiro}
\end{entry}

\begin{entry}{谢世}{xie4shi4}{12,5}{⾔、⼀}
  \definition{v.}{morrer | falecer}
\end{entry}

\begin{entry}{谢天谢地}{xie4tian1xie4di4}{12,4,12,6}{⾔、⼤、⾔、⼟}
  \definition{expr.}{agradecer a Deus | agradecer aos céus}
\end{entry}

\begin{entry}{谢谢}{xie4xie5}{12,12}{⾔、⾔}[HSK 1]
  \definition{interj.}{Obrigado!}
  \definition{v.}{agradecer; agradecer a gentileza dos outros}
\end{entry}

\begin{entry}{谢意}{xie4yi4}{12,13}{⾔、⼼}
  \definition{s.}{gratidão}
\end{entry}

\begin{entry}{心}{xin1}{4}{⼼}[HSK 3][Kangxi 61]
  \definition*{s.}{Xin, uma das mansões lunares; uma das 28 constelações}
  \definition[颗,个]{s.}{o coraçã; órgão que impulsiona a circulação sanguínea no corpo humano e nos vertebrados| coração; mente; sentimento; intenção; refere-se aos órgãos do pensamento e ao pensamento, sentimentos, etc. | centro; núcleo; parte central}
\end{entry}

\begin{entry}{心机}{xin1ji1}{4,6}{⼼、⽊}
  \definition{s.}{pensamento | esquema}
\end{entry}

\begin{entry}{心里}{xin1 li3}{4,7}{⼼、⾥}[HSK 2]
  \definition[个]{s.}{no coração; no coração de alguém | no coração; na mente; na cabeça e no peito}
\end{entry}

\begin{entry}{心理}{xin1li3}{4,11}{⼼、⽟}[HSK 4]
  \definition{adj.}{psicológico}
  \definition{s.}{mentalidade; refere-se à reflexão da mente humana sobre coisas objetivas, incluindo sensação, percepção, memória, pensamento e emoções | psicologia}
\end{entry}

\begin{entry}{心情}{xin1qing2}{4,11}{⼼、⼼}[HSK 2]
  \definition{s.}{humor; tom de sentimento; estado de espírito; estado emocional interior}
\end{entry}

\begin{entry}{心声}{xin1sheng1}{4,7}{⼼、⼠}
  \definition{s.}{desejo sincero | voz interior | aspiração}
\end{entry}

\begin{entry}{心态}{xin1tai4}{4,8}{⼼、⼼}[HSK 5]
  \definition[种]{s.}{mentalidade; psicologia; estado mental}
\end{entry}

\begin{entry}{心疼}{xin1teng2}{4,10}{⼼、⽧}[HSK 5]
  \definition{adj.}{angustiado}
  \definition{v.}{amar profundamente; sentir pena porque coisas valiosas foram destruídas ou perdidas; não querer se separar delas | sentir pena; ficar angustiado; preocupar-se e sofrer pelo sofrimento dos outros; estar disposto a cuidar mais por causa do carinho}
\end{entry}

\begin{entry}{心中}{xin1zhong1}{4,4}{⼼、⼁}[HSK 2]
  \definition{s.}{no coração; na mente}
\end{entry}

\begin{entry}{芯片}{xin1pian4}{7,4}{⾋、⽚}
  \definition{s.}{chip de computador | microchip}
\end{entry}

\begin{entry}{辛苦}{xin1ku3}{7,8}{⾟、⾋}[HSK 5]
  \definition{adj.}{difícil; trabalhoso; árduo; descreve muito trabalho, alta intensidade e pouco descanso}
  \definition{s.}{dificuldades}
  \definition{v.}{trabalhar duro; passar por grandes dificuldades; passar por dificuldades}
\end{entry}

\begin{entry}{欣赏}{xin1shang3}{8,12}{⽋、⾙}[HSK 5]
  \definition{v.}{apreciar; admirar; valorizar; apreciar as coisas boas e descubrir o prazer que elas proporcionam | apreciar; gostar; considerar bom}
\end{entry}

\begin{entry}{新}{xin1}{13}{⽄}[HSK 1]
  \definition*{s.}{sobrenome Xin}
  \definition*{s.}{abreviação de Xinjiang (新疆)}
  \definition*{s.}{abreviação de Singapura (新加坡)}
  \definition{adj.}{novo; fresco; inovador; atualizado; aparecer ou ser experimentado pela primeira vez | nunca utilizado; novo; não foi usado ou foi usado por pouco tempo | recém-casado}
  \definition{adv.}{recém; recentemente; há pouco tempo}
  \definition{pref.}{(química) meso-}
  \definition{v.}{atualizar; renovar}
  \seealsoref{新加坡}{xin1jia1po1}
  \seealsoref{新疆}{xin1jiang1}
\end{entry}

\begin{entry}{新加坡}{xin1jia1po1}{13,5,8}{⽄、⼒、⼟}
  \definition*{s.}{Singapura}
\end{entry}

\begin{entry}{新疆}{xin1jiang1}{13,19}{⽄、⼸}
  \definition*{s.}{Xinjiang}
\end{entry}

\begin{entry}{新疆维吾尔自治区}{xin1jiang1 wei2wu2'er3 zi4zhi4qu1}{13,19,11,7,5,6,8,4}{⽄、⼸、⽷、⼝、⼩、⾃、⽔、⼖}
  \definition*{s.}{Região Autônoma Uigur de Xinjiang}
\end{entry}

\begin{entry}{新郎}{xin1lang2}{13,8}{⽄、⾢}[HSK 4]
  \definition[位,个]{s.}{noivo; homens no momento do casamento}
\end{entry}

\begin{entry}{新年}{xin1 nian2}{13,6}{⽄、⼲}[HSK 1]
  \definition*[个]{s.}{Ano Novo}
\end{entry}

\begin{entry}{新娘}{xin1niang2}{13,10}{⽄、⼥}[HSK 4]
  \definition[位,个]{s.}{noiva; a mulher no momento do casamento}
  \seealsoref{新娘子}{xin1niang2zi5}
\end{entry}

\begin{entry}{新娘服装}{xin1niang2 fu2zhuang1}{13,10,8,12}{⽄、⼥、⽉、⾐}
  \definition{s.}{roupas de noiva}
\end{entry}

\begin{entry}{新娘子}{xin1niang2zi5}{13,10,3}{⽄、⼥、⼦}
  \definition{s.}{noiva}
  \seealsoref{新娘}{xin1niang2}
\end{entry}

\begin{entry}{新闻}{xin1wen2}{13,9}{⽄、⾨}[HSK 2]
  \definition[个,条,则,版]{s.}{notícias; notícias nacionais e internacionais reportadas em jornais, estações de rádio, etc. | notícias; refere-se a coisas importantes ou novas que aconteceram recentemente na sociedade}
\end{entry}

\begin{entry}{新鲜}{xin1xian1}{13,14}{⽄、⿂}
  \definition{adj.}{fresco (experiência, alimento, etc.)}
  \definition{s.}{frescor}
\end{entry}

\begin{entry}{新型}{xin1 xing2}{13,9}{⽄、⼟}[HSK 4]
  \definition[种]{s.}{ultimo modelo; novo tipo; novo padrão; novo estilo}
\end{entry}

\begin{entry}{信}{xin4}{9}{⼈}[HSK 2,3]
  \definition*{s.}{sobrenome Xin}
  \definition{adj.}{verdade}
  \definition[封,个,张]{s.}{carta; correio | mensagem; notícia; informação | sinal; evidência | confiança; fé; crédito | detonador (de bombas, etc.) | arsênico}
  \definition{v.}{acreditar; fazer um balanço; dar crédito | deixar à vontade; deixar à mercê; deixar ao acaso | professar fé em; acreditar em}
\end{entry}

\begin{entry}{信访}{xin4fang3}{9,6}{⼈、⾔}
  \definition{s.}{carta de reclamação | carta de petição}
  \seealsoref{上访}{shang4fang3}
\end{entry}

\begin{entry}{信封}{xin4feng1}{9,9}{⼈、⼨}[HSK 3]
  \definition[个,封]{s.}{envelope para cartas}
\end{entry}

\begin{entry}{信号}{xin4hao4}{9,5}{⼈、⼝}[HSK 2]
  \definition[个,道]{s.}{sinal; luz, ondas de rádio, som, movimento, etc. usados para transmitir mensagens ou comandos | ponte de sinalização; marcação para chamar a atenção, ajudar na identificação e na memória}
\end{entry}

\begin{entry}{信经}{xin4jing1}{9,8}{⼈、⽷}
  \definition[个]{s.}{crença | credo (seção da missa católica)}
\end{entry}

\begin{entry}{信念}{xin4nian4}{9,8}{⼈、⼼}[HSK 5]
  \definition[个]{s.}{fé; crença; convicção; concepções consideradas corretas e acreditadas com convicção}
\end{entry}

\begin{entry}{信任}{xin4ren4}{9,6}{⼈、⼈}[HSK 3]
  \definition{s.}{confiança; um estado mental positivo e conexão emocional}
  \definition{v.}{confiar; ter confiança em; acreditar e ousar confiar}
\end{entry}

\begin{entry}{信息}{xin4xi1}{9,10}{⼈、⼼}[HSK 2]
  \definition[个,条,段,些]{s.}{notícias; informações; as últimas notícias sobre alguém ou alguma coisa | mensagem; informação; na teoria da informação, uma mensagem transmitida usando símbolos, cujo conteúdo é desconhecido pelo receptor}
\end{entry}

\begin{entry}{信箱}{xin4 xiang1}{9,15}{⼈、⾋}[HSK 5]
  \definition{s.}{caixa de correio; caixa postal instalada pelos correios para que as pessoas possam depositar cartas | caixa postal; caixas com números, localizadas nos correios, que podem ser alugadas para receber correspondência; chamadas de caixas postais exclusivas}
\end{entry}

\begin{entry}{信心}{xin4xin1}{9,4}{⼈、⼼}[HSK 2]
  \definition[个]{s.}{confiança; fé (em alguém ou algo) ; a crença de que os desejos se tornarão realidade}
\end{entry}

\begin{entry}{信用}{xin4yong4}{9,5}{⼈、⽤}
  \definition{s.}{crédito (comércio)}
\end{entry}

\begin{entry}{信用卡}{xin4yong4ka3}{9,5,5}{⼈、⽤、⼘}[HSK 2]
  \definition[张]{s.}{cartão de crédito; moeda eletrônica emitida por um banco ou outra instituição especializada para consumidores; os titulares do cartão podem usá-lo para sacar dinheiro ou fazer compras de acordo com os regulamentos}
\end{entry}

\begin{entry}{兴}{xing1}{6}{⼋}
  \definition*{s.}{sobrenome Xing}
  \definition{adv.}{talvez (dialeto)}
  \definition{v.}{subir | florescer | tornar-se popular | começar | encorajar | levantar-se | (frequentemente usado em negativas) permitir (dialeto)}
  \seeref{兴}{xing4}
\end{entry}

\begin{entry}{兴奋}{xing1fen4}{6,8}{⼋、⼤}[HSK 4]
  \definition{adj.}{animado; excitante; empolgante;}
  \definition{s.}{excitação; empolgação}
  \definition{v.}{excitar; intoxicar}
\end{entry}

\begin{entry}{星表}{xing1biao3}{9,8}{⽇、⾐}
  \definition{s.}{catálogo de estrelas}
\end{entry}

\begin{entry}{星辰}{xing1chen2}{9,7}{⽇、⾠}
  \definition{s.}{estrelas}
\end{entry}

\begin{entry}{星火}{xing1huo3}{9,4}{⽇、⽕}
  \definition{s.}{trilha de meteoro (usada principalmente em expressões como 急如星火) | faísca}
\end{entry}

\begin{entry}{星期}{xing1qi1}{9,12}{⽇、⽉}[HSK 1]
  \definition[个]{s.}{semana | dias da semana; usado em conjunto com 日, 一, 二, 三, 四, 五, 六, 天, indica um determinado dia da semana | abreviação de domingo}
  \seealsoref{星期二}{xing1 qi1 er4}
  \seealsoref{星期六}{xing1 qi1 liu4}
  \seealsoref{星期日}{xing1 qi1 ri4}
  \seealsoref{星期三}{xing1 qi1 san1}
  \seealsoref{星期四}{xing1 qi1 si4}
  \seealsoref{星期天}{xing1 qi1 tian1}
  \seealsoref{星期五}{xing1 qi1 wu3}
  \seealsoref{星期一}{xing1 qi1 yi1}
\end{entry}

\begin{entry}{星期二}{xing1 qi1 er4}{9,12,2}{⽇、⽉、⼆}[HSK 1]
  \definition{s.}{terça-feira}
\end{entry}

\begin{entry}{星期六}{xing1 qi1 liu4}{9,12,4}{⽇、⽉、⼋}[HSK 1]
  \definition{s.}{sábado}
\end{entry}

\begin{entry}{星期日}{xing1 qi1 ri4}{9,12,4}{⽇、⽉、⽇}[HSK 1]
  \definition{s.}{domingo}
  \seealsoref{星期天}{xing1 qi1 tian1}
\end{entry}

\begin{entry}{星期三}{xing1 qi1 san1}{9,12,3}{⽇、⽉、⼀}[HSK 1]
  \definition{s.}{quarta-feira}
\end{entry}

\begin{entry}{星期四}{xing1 qi1 si4}{9,12,5}{⽇、⽉、⼞}[HSK 1]
  \definition{s.}{quinta-feira}
\end{entry}

\begin{entry}{星期天}{xing1 qi1 tian1}{9,12,4}{⽇、⽉、⼤}[HSK 1]
  \definition{s.}{domingo}
  \seealsoref{星期日}{xing1 qi1 ri4}
\end{entry}

\begin{entry}{星期五}{xing1 qi1 wu3}{9,12,4}{⽇、⽉、⼆}[HSK 1]
  \definition{s.}{sexta-feira}
\end{entry}

\begin{entry}{星期一}{xing1 qi1 yi1}{9,12,1}{⽇、⽉、⼀}[HSK 1]
  \definition{s.}{segunda-feira}
\end{entry}

\begin{entry}{星星}{xing1 xing5}{9,9}{⽇、⽇}[HSK 2]
  \definition[颗,群,片]{s.}{estrela; em astronomia, refere-se aos corpos celestes luminosos no universo, como as estrelas que brilham no céu noturno | estrela; uma metáfora para alguém ou algo que se destaca em um determinado campo e atrai atenção | objetos em forma de estrela}
\end{entry}

\begin{entry}{星座}{xing1zuo4}{9,10}{⽇、⼴}
  \definition[张]{s.}{signo astrológico | constelação}
\end{entry}

\begin{entry}{猩猩}{xing1xing5}{12,12}{⽝、⽝}
  \definition{s.}{orangotango}
\end{entry}

\begin{entry}{行}{xing2}{6}{⾏}[HSK 1][Kangxi 144]
  \definition*{s.}{sobrenome Xing}
  \definition{adj.}{de viajar; relacionado a viagens | temporário; improvisado; provisório | capaz; competente}
  \definition{adv.}{em breve}
  \definition{s.}{comportamento; conduta | caligrafia cursiva (na caligrafia chinesa); escrita cursiva}
  \definition{v.}{ir | fazer uma viagem | estar em voga; prevalecer; circular | fazer; executar; realizar; envolver-se em | estar tudo bem; O.K. | indica a realização de uma determinada atividade (usado principalmente antes de verbos dissilábicos) | (em medicina) fazer efeito}
  \seeref{行}{hang2}
  \seeref{行}{heng2}
\end{entry}

\begin{entry}{行动}{xing2dong4}{6,6}{⾏、⼒}[HSK 2]
  \definition[次,场,项]{s.}{ação; operação; comportamento;}
  \definition{v.}{circular; mover-se; andar | agir; tomar medidas; atividades para atingir um determinado propósito}
\end{entry}

\begin{entry}{行进}{xing2jin4}{6,7}{⾏、⾡}
  \definition{s.}{avançar | movimentar-se para frente}
\end{entry}

\begin{entry}{行礼}{xing2li3}{6,5}{⾏、⽰}
  \definition{v.}{saudar | fazer saudação}
\end{entry}

\begin{entry}{行李}{xing2li5}{6,7}{⾏、⽊}[HSK 3]
  \definition[点,个]{s.}{bagagem, malas, cestas de vime, etc. que você leva quando sai de casa}
\end{entry}

\begin{entry}{行人}{xing2ren2}{6,2}{⾏、⼈}[HSK 2]
  \definition[个]{s.}{pedestre; transeunte; viajante à pé; pessoas caminhando na estrada}
\end{entry}

\begin{entry}{行驶}{xing2 shi3}{6,8}{⾏、⾺}[HSK 5]
  \definition{v.}{ir; navegar; viajar (utilizando um veículo, navio, etc.);}
\end{entry}

\begin{entry}{行为}{xing2wei2}{6,4}{⾏、⼂}[HSK 2]
  \definition[个,种,类]{s.}{ação; comportamento; conduta; atividades que são controladas por pensamentos e manifestadas externamente}
\end{entry}

\begin{entry}{行星}{xing2xing1}{6,9}{⾏、⽇}
  \definition[颗]{s.}{planeta}
  \seealsoref{惑星}{huo4xing1}
\end{entry}

\begin{entry}{行凶}{xing2xiong1}{6,4}{⾏、⼐}
  \definition{v.+compl.}{cometer agressão física ou assassinato | fazer algo violento}
\end{entry}

\begin{entry}{形成}{xing2cheng2}{7,6}{⼺、⼽}[HSK 3]
  \definition{v.}{moldar; formar; tomar forma; tornar-se algo ou surgir uma situação após mudanças e desenvolvimentos}
\end{entry}

\begin{entry}{形而上学}{xing2'er2shang4xue2}{7,6,3,8}{⼺、⽽、⼀、⼦}
  \definition{s.}{metafísica}
\end{entry}

\begin{entry}{形容}{xing2rong2}{7,10}{⼺、⼧}[HSK 4]
  \definition{s.}{aparência; semblante}
  \definition{v.}{descrever}
\end{entry}

\begin{entry}{形式}{xing2shi4}{7,6}{⼺、⼷}[HSK 3]
  \definition[种,个]{s.}{forma; formato; modalidade; a aparência, estrutura ou estado das coisas, etc.}
\end{entry}

\begin{entry}{形势}{xing2shi4}{7,8}{⼺、⼒}[HSK 4]
  \definition[个]{s.}{terreno; características topográficas; situação geográfica, principalmente de uma perspectiva militar | situação; circunstâncias; a situação geral, a tendência de como as coisas estão se desenvolvendo e mudando | geralmente não é usado em situações pessoais}
\end{entry}

\begin{entry}{形态}{xing2tai4}{7,8}{⼺、⼼}[HSK 5]
  \definition{s.}{forma; forma como as coisas se apresentam | forma; padrão; postura | morfologia; forma; (gramática) refere-se às formas internas de mudança das palavras, incluindo a formação de palavras e as mudanças morfológicas}
\end{entry}

\begin{entry}{形象}{xing2xiang4}{7,11}{⼺、⾗}[HSK 3]
  \definition{adj.}{vívido; expressão concreta e vívida}
  \definition[个,种]{s.}{imagem; forma; figura; formas ou posturas específicas que podem despertar pensamentos ou emoções nas pessoas | imagem literária; imagem artística; pessoas ou coisas com características diferentes criadas na literatura, no cinema e em outras artes}
\end{entry}

\begin{entry}{形状}{xing2zhuang4}{7,7}{⼺、⽝}[HSK 3]
  \definition[个,种]{s.}{forma; aparência ; aspecto; a aparência de um objeto ou figura, representada pela combinação de superfícies ou linhas externas}
\end{entry}

\begin{entry}{型}{xing2}{9}{⼟}[HSK 4]
  \definition{s.}{molde; modelo | modelo; tipo; padrão}
\end{entry}

\begin{entry}{型号}{xing2 hao4}{9,5}{⼟、⼝}[HSK 4]
  \definition[个,种]{s.}{modelo; tipo; refere-se ao desempenho, às especificações e ao tamanho de aeronaves, máquinas, implementos agrícolas, etc.}
\end{entry}

\begin{entry}{省}{xing3}{9}{⽬}
  \definition{v.}{examinar-se criticamente; verificar (os próprios pensamentos, palavras e ações) | visitar (especialmente os pais ou pessoas mais velhas) | estar ciente; tornar-se consciente; compreender; tomar consciência | examinar minuciosamente; inspecionar; escrutinar}
  \seeref{省}{sheng3}
\end{entry}

\begin{entry}{省悟}{xing3wu4}{9,10}{⽬、⼼}
  \definition{v.}{voltar a si | constatar | ver a verdade | acordar para a realidade}
\end{entry}

\begin{entry}{醒}{xing3}{16}{⾣}[HSK 4]
  \definition{adj.}{impressionante; notável; admirável; atraente; chamativo}
  \definition{v.}{ficar sóbrio; voltar a si; recuperar a consciência; retornar à normalidade após intoxicação, anestesia ou coma | despertar; estar acordado | ter a mente clara; mover a consciência da confusão para a compreensão | vir a entender; tornar-se ciente de; tomar consciência de}
\end{entry}

\begin{entry}{兴}{xing4}{6}{⼋}
  \definition{s.}{sentimento ou desejo de fazer algo | interesse em algo | excitação}
  \seeref{兴}{xing1}
\end{entry}

\begin{entry}{兴趣}{xing4 qu4}{6,15}{⼋、⾛}[HSK 4]
  \definition[个]{s.}{interesse (desejo de conhecer sobre alguma coisa ou coisa no qual está interessado) | \emph{hobby}}
\end{entry}

\begin{entry}{姓}{xing4}{8}{⼥}[HSK 2]
  \definition[个]{s.}{sobrenome; nome de família; um caractere que representa um sistema familiar, os chineses colocam o sobrenome em primeiro lugar e o nome em segundo}
  \definition{v.}{ter como sobrenome; tratar um ou mais caracteres como sobrenome}
\end{entry}

\begin{entry}{姓名}{xing4ming2}{8,6}{⼥、⼝}[HSK 2]
  \definition{s.}{nome; nome completo; sobrenome e nome próprio}
\end{entry}

\begin{entry}{姓氏}{xing4shi4}{8,4}{⼥、⽒}
  \definition{s.}{sobrenome}
\end{entry}

\begin{entry}{幸福}{xing4fu2}{8,13}{⼲、⽰}[HSK 3]
  \definition{adj.}{feliz; a vida, a família e outras circunstâncias deixam as pessoas satisfeitas e felizes}
  \definition{s.}{felicidade; bem estar; sensação ou experiência satisfatória e feliz, etc.}
\end{entry}

\begin{entry}{幸亏}{xing4kui1}{8,3}{⼲、⼆}
  \definition{adv.}{felizmente}
\end{entry}

\begin{entry}{幸运}{xing4yun4}{8,7}{⼲、⾡}[HSK 3]
  \definition{adj.}{sortudo; feliz; afortunado}
  \definition[个,点,丝]{s.}{boa sorte; boa fortuna}
\end{entry}

\begin{entry}{幸运抽奖}{xing4yun4chou1jiang3}{8,7,8,9}{⼲、⾡、⼿、⼤}
  \definition{s.}{loteria | sorteio}
\end{entry}

\begin{entry}{幸运儿}{xing4yun4'er2}{8,7,2}{⼲、⾡、⼉}
  \definition{s.}{pessoa de sorte}
\end{entry}

\begin{entry}{性}{xing4}{8}{⼼}[HSK 3]
  \definition[个]{s.}{natureza; caráter; personalidade | propriedade; qualidade; natureza e características das coisas | sexo; gênero | sexualidade; relacionado com a reprodução e a sexualidade | caráter; temperamento}
  \definition{suf.}{indica uma determinada propriedade ou característica de algo; segue um substantivo, verbo ou adjetivo, formando um substantivo abstrato ou um adjetivo que expressa uma propriedade}
\end{entry}

\begin{entry}{性别}{xing4bie2}{8,7}{⼼、⼑}[HSK 3]
  \definition[种]{s.}{sexo; gênero}
\end{entry}

\begin{entry}{性格}{xing4ge2}{8,10}{⼼、⽊}[HSK 3]
  \definition[种,个]{s.}{caráter; temperamento; as características psicológicas manifestadas na atitude e no comportamento em relação às pessoas e às coisas}
\end{entry}

\begin{entry}{性能}{xing4neng2}{8,10}{⼼、⾁}[HSK 5]
  \definition{s.}{natureza; propriedade; desempenho; função (de uma máquina, etc.); grau de conformidade dos produtos mecânicos ou outros produtos industriais com os requisitos de projeto}
\end{entry}

\begin{entry}{性侵}{xing4qin1}{8,9}{⼼、⼈}
  \definition{s.}{agressão sexual}
  \definition{v.}{agredir sexualmente}
\end{entry}

\begin{entry}{性生活}{xing4sheng1huo2}{8,5,9}{⼼、⽣、⽔}
  \definition{s.}{vida sexual}
\end{entry}

\begin{entry}{性质}{xing4zhi4}{8,8}{⼼、⾙}[HSK 4]
  \definition[个,种,类]{s.}{natureza; qualidade; caráter; propriedade; propriedade fundamental que distingue uma coisa de outra}
\end{entry}

\begin{entry}{凶}{xiong1}{4}{⼐}
  \definition{adj.}{sinistro; desfavorável; azarado (oposto de "吉") | ruim para as colheitas; improdutivo; ameaçado pela fome | feroz; vicioso; cruel | medroso; terrível}
  \definition[个]{s.}{mal; assassinato; ato de violência; atos de matar ou ferir pessoas | assassino; malfeitor; criminoso; pessoa má; pessoa violenta}
  \definition{v.}{ser feroz; tratar cruelmente}
  \seealsoref{吉}{ji2}
\end{entry}

\begin{entry}{兄弟}{xiong1di4}{5,7}{⼉、⼸}[HSK 4]
  \definition{adj.}{fraternal}
  \definition{pron.}{eu, me (termo de uso humilde por homens em discurso público)}
  \definition[个,对]{s.}{irmãos; irmão}
\end{entry}

\begin{entry}{匈奴}{xiong1nu2}{6,5}{⼓、⼥}
  \definition*{s.}{Xiongnu, um povo da estepe oriental que criou um império que floresceu na época das dinastias Qin e Han}
\end{entry}

\begin{entry}{汹涌}{xiong1yong3}{7,10}{⽔、⽔}
  \definition{adj.}{turbulento}
  \definition{v.}{aumentar ou emergir violentamente (oceano, rio, lago, etc.)}
\end{entry}

\begin{entry}{胸}{xiong1}{10}{⾁}
  \definition{s.}{peito | tórax}
\end{entry}

\begin{entry}{胸部}{xiong1 bu4}{10,10}{⾁、⾢}[HSK 4]
  \definition{s.}{peito; tórax; seios}
\end{entry}

\begin{entry}{雄伟}{xiong2wei3}{12,6}{⾫、⼈}[HSK 5]
  \definition{adj.}{magnífico; magnificente | imponente; magnífico}
\end{entry}

\begin{entry}{熊}{xiong2}{14}{⽕}[HSK 5]
  \definition*{s.}{sobrenome Xiong}
  \definition[把]{s.}{urso}
  \definition{v.}{repreender; censurar;}
\end{entry}

\begin{entry}{熊猫}{xiong2mao1}{14,11}{⽕、⽝}
  \definition[把,只]{s.}{panda gigante}
  \seealsoref{猫熊}{mao1xiong2}
\end{entry}

\begin{entry}{休兵}{xiu1bing1}{6,7}{⼈、⼋}
  \definition{s.}{armistício}
  \definition{v.}{cessar fogo}
\end{entry}

\begin{entry}{休假}{xiu1 jia4}{6,11}{⼈、⼈}[HSK 2]
  \definition{v.+compl.}{ter um feriado; tirar férias; sair de férias}
\end{entry}

\begin{entry}{休憩}{xiu1qi4}{6,16}{⼈、⼼}
  \definition{v.}{relaxar | descansar | dar um tempo}
\end{entry}

\begin{entry}{休息室}{xiu1xi1shi4}{6,10,9}{⼈、⼼、⼧}
  \definition{s.}{saguão | salão}
\end{entry}

\begin{entry}{休息}{xiu1xi5}{6,10}{⼈、⼼}[HSK 1]
  \definition{s.}{descanço}
  \definition{v.}{descansar; descansar um pouco; fazer uma pausa; interromper o trabalho, os estudos ou as atividades para recuperar as energias | dormir}
\end{entry}

\begin{entry}{休闲}{xiu1xian2}{6,7}{⼈、⾨}[HSK 5]
  \definition{s.}{ócio; lazer; tempo livre}
  \definition{v.}{desfrutar do lazer; sair de férias; aproveitar o tempo livre; parar de trabalhar ou estudar, estar em um estado de lazer e descontração | ficar ocioso}
\end{entry}

\begin{entry}{休整}{xiu1zheng3}{6,16}{⼈、⽁}
  \definition{v.}{(militar) descansar e reorganizar}
\end{entry}

\begin{entry}{修}{xiu1}{9}{⼈}[HSK 3]
  \definition*{s.}{sobrenome Xiu}
  \definition{adj.}{comprido; alto e esbelto}
  \definition{s.}{revisionismo}
  \definition{v.}{embelezar; decorar | consertar; reparar; reformar | escrever; redigir; compilar | estudar; cultivar; aprender e praticar para aperfeiçoar ou melhorar (o caráter e o conhecimento) | construir; edificar | cortar ou aparar, para deixar bonito e arrumado | dedicar-se à prática da religião}
\end{entry}

\begin{entry}{修复}{xiu1fu4}{9,9}{⼈、⼢}[HSK 5]
  \definition{v.}{reparar; restaurar; renovar | reparar; melhorar e restaurar (o relacionamento)}
\end{entry}

\begin{entry}{修改}{xiu1gai3}{9,7}{⼈、⽁}[HSK 3]
  \definition{v.}{revisar; retocar; corrigir erros e falhas em artigos, planos, etc.}
\end{entry}

\begin{entry}{修规}{xiu1gui1}{9,8}{⼈、⾒}
  \definition{s.}{plano de construção}
\end{entry}

\begin{entry}{修建}{xiu1jian4}{9,8}{⼈、⼵}[HSK 5]
  \definition{v.}{construir; erguer; animar; edificar; construir com tijolos, telhas, madeira, cimento, areia, etc.}
\end{entry}

\begin{entry}{修理}{xiu1li3}{9,11}{⼈、⽟}[HSK 4]
  \definition{v.}{consertar; reparar; restaurar algo danificado à sua forma ou função original | aparar; podar; cortar com tesouras e outras ferramentas para deixar árvores, flores, cabelos, etc. arrumados | culpar; punir; criticar ou punir uma pessoa para mostrar que ela está errada}
\end{entry}

\begin{entry}{修养}{xiu1yang3}{9,9}{⼈、⼋}[HSK 5]
  \definition[种]{s.}{treinamento; domínio; realização; refere-se a um determinado nível em termos de teoria, conhecimento, arte, pensamento, etc. | auto-cultivo; refere-se à atitude e ao comportamento cultivados ao longo do tempo, em conformidade com as exigências sociais}
\end{entry}

\begin{entry}{宿}{xiu3}{11}{⼧}
  \definition{s.}{usado para calcular a noite}[谈了半宿。___Conversamos por metade da noite.]
  \seeref{宿}{su4}
  \seeref{宿}{xiu3}
\end{entry}

\begin{entry}{绣}{xiu4}{10}{⽷}
  \definition{s.}{bordado}
  \definition{v.}{bordar}
\end{entry}

\begin{entry}{臭}{xiu4}{10}{⾃}
  \definition{s.}{odor; cheiro}
  \definition{v.}{cheirar; farejar; o mesmo que 嗅}
  \seeref{臭}{chou4}
  \seealsoref{嗅}{xiu4}
\end{entry}

\begin{entry}{袖}{xiu4}{10}{⾐}
  \definition{s.}{manga (de camisa, de camiseta, etc.)}
\end{entry}

\begin{entry}{宿}{xiu4}{11}{⼧}
  \definition{s.}{(astronomia) um termo antigo para constelação}
  \seeref{宿}{su4}
  \seeref{宿}{xiu4}
\end{entry}

\begin{entry}{嗅}{xiu4}{13}{⼝}
  \definition{v.}{cheirar; farejar; identificar odores pelo nariz}
\end{entry}

\begin{entry}{虚}{xu1}{11}{⾌}
  \definition*{s.}{Xu, a décima primeira das vinte e oito constelações (Vinte e Oito Constelações) em que a esfera celeste foi dividida, consistindo de duas estrelas em linha reta, uma em Aquário e a outra em Equuleus | Xu, uma das mansões lunares}
  \definition*{s.}{sobrenome Xu}
  \definition{adj.}{vazio; oco; desocupado | desconfiado; tímido | falso; nominal (oposto a 实) | humilde; modesto | fraco; com saúde debilitada | (física) virtual}
  \definition{adv.}{em vão}
  \definition{s.}{vazio; nulidade; anulação | resumo; teoria; princípios orientadores; ideologia política e outros aspectos}
  \definition{v.}{1. reservar espaço}
  \seealsoref{实}{shi2}
\end{entry}

\begin{entry}{虚伪}{xu1wei3}{11,6}{⾌、⼈}
  \definition{adj.}{falso | hipócrita | artificial}
\end{entry}

\begin{entry}{虚心}{xu1xin1}{11,4}{⾌、⼼}[HSK 5]
  \definition{adj.}{modesto; humilde; de mente aberta; não ser presunçoso, ser capaz de aceitar as opiniões dos outros}
\end{entry}

\begin{entry}{需求}{xu1qiu2}{14,7}{⾬、⽔}[HSK 3]
  \definition[种]{s.}{necessidades; demanda; exigência; solicitações decorrentes de necessidades}
\end{entry}

\begin{entry}{需要}{xu1yao4}{14,9}{⾬、⾑}[HSK 3]
  \definition[种]{s.}{necessidade; desejo ou exigência em relação a algo}
  \definition{v.}{precisar; querer; exigir; demandar; solicitar}
\end{entry}

\begin{entry}{许}{xu3}{6}{⾔}
  \definition*{s.}{sobrenome Xu}
  \definition{adv.}{um pouco | talvez}
  \definition{v.}{permitir | prometer | elogiar}
\end{entry}

\begin{entry}{许多}{xu3duo1}{6,6}{⾔、⼣}[HSK 2]
  \definition{num.}{muitos; muito; numerosos; uma grande quantidade de}
\end{entry}

\begin{entry}{许可}{xu3ke3}{6,5}{⾔、⼝}[HSK 5]
  \definition{v.}{permitir; autorizar}
\end{entry}

\begin{entry}{T-恤}{xu4}{9}{⼼}
  \definition{s.}{camiseta | pulôver | suéter}
\end{entry}

\begin{entry}{畜}{xu4}{10}{⽥}
  \definition{v.}{criar (animais domésticos)}
  \seeref{畜}{chu4}
\end{entry}

\begin{entry}{宣布}{xuan1bu4}{9,5}{⼧、⼱}[HSK 3]
  \definition{v.}{declarar; proclamar; pronunciar; anunciar; informar oficialmente a todos sobre as últimas decisões e situações}
\end{entry}

\begin{entry}{宣传}{xuan1chuan2}{9,6}{⼧、⼈}[HSK 3]
  \definition[个]{v.}{propagar; divulgar; fazer propaganda; explicar e esclarecer às pessoas, para que elas acreditem e sigam as ações}
\end{entry}

\begin{entry}{宣扬}{xuan1yang2}{9,6}{⼧、⼿}
  \definition{v.}{divulgar | anunciar | espalhar por toda parte}
\end{entry}

\begin{entry}{玄学}{xuan2xue2}{5,8}{⽞、⼦}
  \definition{s.}{Escola Philosófica Wei e Jin amalgamando os ideais daoísta e confucionistas | tradução da metafísica (形而上学)}
  \seealsoref{形而上学}{xing2'er2shang4xue2}
\end{entry}

\begin{entry}{悬挂}{xuan2gua4}{11,9}{⼼、⼿}
  \definition{s.}{(veículo) suspensão}
  \definition{v.}{suspender}
\end{entry}

\begin{entry}{悬崖}{xuan2ya2}{11,11}{⼼、⼭}
  \definition{s.}{precipício | penhasco}
\end{entry}

\begin{entry}{旋转}{xuan2zhuan3}{11,8}{⽅、⾞}
  \definition{v.}{girar}
\end{entry}

\begin{entry}{选}{xuan3}{9}{⾡}[HSK 2]
  \definition{s.}{pessoa ou coisa selecionada | seleções; antologia; trabalhos selecionados e compilados}
  \definition{v.}{selecionar; escolher | eleger}
\end{entry}

\begin{entry}{选手}{xuan3shou3}{9,4}{⾡、⼿}[HSK 3]
  \definition[位,名,个,些]{s.}{jogador; (selecionado) competidor; atleta selecionado para uma competição esportiva; participantes selecionados entre um grande número de candidatos}
\end{entry}

\begin{entry}{选修}{xuan3 xiu1}{9,9}{⾡、⼈}[HSK 5]
  \definition{v.}{fazer um curso eletivo; selecionar os cursos a serem estudados entre os cursos disponíveis}
\end{entry}

\begin{entry}{选择}{xuan3ze2}{9,8}{⾡、⼿}[HSK 4]
  \definition[个,种,次]{s.}{escolha; opção; resultado da escolha; possibilidade de escolha}
  \definition{v.}{selecionar; escolher}
\end{entry}

\begin{entry}{学}{xue2}{8}{⼦}[HSK 1]
  \definition[所]{s.}{aprendizagem; conhecimento; sabedoria; erudição | objeto de estudo; ramo do conhecimento | escola; faculdade | teoria; doutrina}
  \definition{v.}{estudar; aprender | imitar; copiar}
\end{entry}

\begin{entry}{学费}{xue2 fei4}{8,9}{⼦、⾙}[HSK 3]
  \definition[笔]{s.}{mensalidade (taxa); prêmio; taxas que os alunos devem pagar para estudar na escola, conforme estabelecido pela escola | preço pelo que se aprendeu ao custo do próprio bolso; a metáfora do preço a pagar para obter uma determinada experiência | custo; preço; todas as despesas necessárias durante o período de estudos do aluno}
\end{entry}

\begin{entry}{学分}{xue2fen1}{8,4}{⼦、⼑}[HSK 4]
  \definition{s.}{créditos de um curso; uma unidade de medida do peso e do tempo do curso no ensino superior; cada curso vale um crédito para uma aula por semana durante um semestre; alunos devem concluir o número necessário de créditos para se formar}
\end{entry}

\begin{entry}{学好}{xue2hao3}{8,6}{⼦、⼥}
  \definition{v.}{seguir bons exemplos | aprender bem}
\end{entry}

\begin{entry}{学会}{xue2hui4}{8,6}{⼦、⼈}
  \definition{s.}{instituto | associação (acadêmica) | sociedade científica, douta ou erudita}
  \definition{v.}{aprender | dominar (um assunto)}
\end{entry}

\begin{entry}{学科}{xue2 ke1}{8,9}{⼦、⽲}[HSK 5]
  \definition{s.}{ramo do aprendizado; disciplina | disciplina escolar; curso de estudo | cursos teóricos oferecidos em treinamento militar ou físico (oposto a 术科)  | disciplina acadêmica | curso | assunto; tema}
  \seealsoref{术科}{shu4ke1}
\end{entry}

\begin{entry}{学年}{xue2 nian2}{8,6}{⼦、⼲}[HSK 4]
  \definition{s.}{ano letivo; ano acadêmico}
\end{entry}

\begin{entry}{学期}{xue2qi1}{8,12}{⼦、⽉}[HSK 2]
  \definition[个,段]{s.}{semestre; período escolar; um ano acadêmico é dividido em dois semestres, um semestre do início do outono até as férias de inverno e um semestre do início da primavera até as férias de verão}
\end{entry}

\begin{entry}{学生}{xue2sheng5}{8,5}{⼦、⽣}[HSK 1]
  \definition{s.}{aluno; estudante; pupilo}
\end{entry}

\begin{entry}{学生证}{xue2sheng5zheng4}{8,5,7}{⼦、⽣、⾔}
  \definition{s.}{cartão de identidade de estudante}
\end{entry}

\begin{entry}{学时}{xue2 shi2}{8,7}{⼦、⽇}[HSK 4]
  \definition{s.}{hora-aula; hora de aula | período}
\end{entry}

\begin{entry}{学术}{xue2shu4}{8,5}{⼦、⽊}[HSK 4]
  \definition[个]{s.}{aprendizagem; aprendizado; ciências; aprendizado sistemático e especializado}
\end{entry}

\begin{entry}{学位}{xue2wei4}{8,7}{⼦、⼈}[HSK 5]
  \definition{s.}{grau; grau acadêmico; título concedido com base no nível acadêmico profissional, como doutorado, mestrado, etc.}
\end{entry}

\begin{entry}{学问}{xue2wen4}{8,6}{⼦、⾨}[HSK 4]
  \definition[个]{s.}{aprendizado, conhecimento, erudição; a compreensão correta do mundo objetivo que alguém tem | conhecimento; aprendizado sistemático; conhecimento sistemático sobre algo ou uma ciência que pode ser aprendido em um livro ou em uma experiência prática}
\end{entry}

\begin{entry}{学习}{xue2xi2}{8,3}{⼦、⼄}[HSK 1]
  \definition{s.}{estudo}
  \definition{v.}{estudar; aprender; adquirir conhecimentos ou habilidades através da leitura, da audição, da pesquisa e da prática}
\end{entry}

\begin{entry}{学校}{xue2xiao4}{8,10}{⼦、⽊}[HSK 1]
  \definition[所,个]{s.}{escola; instituição de ensino}
\end{entry}

\begin{entry}{学院}{xue2yuan4}{8,9}{⼦、⾩}[HSK 1]
  \definition[个,所]{s.}{academia; instituto; um tipo de instituição de ensino superior que se concentra em uma determinada área de especialização, como faculdades de engenharia, faculdades de música, faculdades de educação, etc.}
\end{entry}

\begin{entry}{学者}{xue2 zhe3}{8,8}{⼦、⽼}[HSK 5]
  \definition[位]{s.}{erudito; homem culto; pessoas que fazem pesquisas acadêmicas geralmente se referem àquelas que alcançaram certo sucesso acadêmico}
\end{entry}

\begin{entry}{雪}{xue3}{11}{⾬}[HSK 2]
  \definition*{s.}{sobrenome Xue}
  \definition[场,层]{s.}{neve | algo parecido com neve}
  \definition{v.}{limpar; enxugar; remover}
\end{entry}

\begin{entry}{雪板}{xue3ban3}{11,8}{⾬、⽊}
  \definition{s.}{prancha de \emph{snowboard}}
  \definition{v.}{praticar \textit{snowboard}}
\end{entry}

\begin{entry}{雪糕}{xue3gao1}{11,16}{⾬、⽶}
  \definition{s.}{picolé}
\end{entry}

\begin{entry}{雪花}{xue3hua1}{11,7}{⾬、⾋}
  \definition{s.}{floco de neve}
\end{entry}

\begin{entry}{雪葩}{xue3pa1}{11,12}{⾬、⾋}
  \definition{s.}{sorvete}
\end{entry}

\begin{entry}{雪人}{xue3ren2}{11,2}{⾬、⼈}
  \definition{s.}{boneco de neve | \emph{Yeti}}
\end{entry}

\begin{entry}{雪山}{xue3shan1}{11,3}{⾬、⼭}
  \definition{s.}{montanha coberta de neve}
\end{entry}

\begin{entry}{雪鞋}{xue3xie2}{11,15}{⾬、⾰}
  \definition[双]{s.}{sapatos de neve}
\end{entry}

\begin{entry}{血}{xue4}{6}{⾎}[HSK 3][Kangxi 143]
  \definition[滴,袋,口,毫升]{s.}{sangue | parente consanguíneo; com laços de parentesco | pessoa ativa e animada; metáfora para uma personalidade ou espírito forte e sincero | medicina tradicional chinesa refere-se à menstruação}
  \seeref{血}{xie3}
\end{entry}

\begin{entry}{血汗}{xue4han4}{6,6}{⾎、⽔}
  \definition{s.}{(fig.) suor e labuta, trabalho duro}
\end{entry}

\begin{entry}{熏香}{xun1xiang1}{14,9}{⽕、⾹}
  \definition{s.}{incenso}
\end{entry}

\begin{entry}{寻求}{xun2 qiu2}{6,7}{⼨、⽔}[HSK 5]
  \definition{v.}{procurar; perseguir; explorar; ir em busca de}
\end{entry}

\begin{entry}{寻找}{xun2zhao3}{6,7}{⼨、⼿}[HSK 4]
  \definition{v.}{buscar; procurar; pesquisar; encontrar, que pode ser usado tanto para coisas concretas quanto para coisas abstratas}
\end{entry}

\begin{entry}{巡逻}{xun2luo2}{6,11}{⾡、⾡}
  \definition{s.}{patrulha}
  \definition{v.}{patrulhar (polícia, exército ou marinha)}
\end{entry}

\begin{entry}{询问}{xun2wen4}{8,6}{⾔、⾨}[HSK 5]
  \definition{v.}{indagar; perguntar sobre; pedir conselho}
\end{entry}

\begin{entry}{训练}{xun4lian4}{5,8}{⾔、⽷}[HSK 3]
  \definition{v.}{treinar; exercitar; planejar e executar de forma sistemática o desenvolvimento de habilidades ou competências específicas}
\end{entry}

\begin{entry}{迅速}{xun4su4}{6,10}{⾡、⾡}[HSK 4]
  \definition{adv.}{rapidamente; velozmente; prontamente}
\end{entry}

%%%%% EOF %%%%%

