%%%
%%% K
%%%

\section*{K}\addcontentsline{toc}{section}{K}

\begin{entry}{咖啡}{ka1fei1}{8,11}{⼝、⼝}[HSK 3]
  \definition[杯]{s.}{(empréstimo linguístico) café}
\end{entry}

\begin{entry}{咖啡馆}{ka1fei1guan3}{8,11,11}{⼝、⼝、⾷}
  \definition[家]{s.}{cafeteria}
\end{entry}

\begin{entry}{咖啡色}{ka1fei1 se4}{8,11,6}{⼝、⼝、⾊}
  \definition{s.}{cor café}
\end{entry}

\begin{entry}{卡}{ka3}{5}{⼘}[HSK 2]
  \definition{clas.}{para calorias}
  \definition{s.}{cartão}
  \definition{v.}{bloquear | verificar | agarrar}
  \seeref{卡}{qia3}
\end{entry}

\begin{entry}{卡车司机}{ka3che1 si1ji1}{5,4,5,6}{⼘、⾞、⼝、⽊}
  \definition{s.}{motorista de caminhão}
\end{entry}

\begin{entry}{卡片}{ka3pian4}{5,4}{⼘、⽚}
  \definition{s.}{cartão}
\end{entry}

\begin{entry}{卡片游戏}{ka3pian4 you2xi4}{5,4,12,6}{⼘、⽚、⽔、⼽}
  \definition{s.}{carta de baralho}
\end{entry}

\begin{entry}{卡通}{ka3tong1}{5,10}{⼘、⾡}
  \definition{s.}{(empréstimo linguístico) \emph{cartoon}}
\end{entry}

\begin{entry}{开}{kai1}{4}{⼶}[HSK 1]
  \definition{clas.}{quilate (ouro)}
  \definition{v.}{abrir | ligar | dirigir | iniciar (alguma coisa) | começar | ferver | escrever  (uma receita, cheque, fatura, etc.) | operar (um veículo) | abreviação de Kelvin 开尔文}
  \seeref{开尔文}{kai1'er3wen2}
\end{entry}

\begin{entry}{开车}{kai1 che1}{4,4}{⼶、⾞}[HSK 1]
  \definition{v.+compl.}{conduzir | dirigir}
\end{entry}

\begin{entry}{开尔文}{kai1'er3wen2}{4,5,4}{⼶、⼩、⽂}
  \definition{s.}{Kelvin, temperatura absoluta | K, escala de temperatura}
\end{entry}

\begin{entry}{开发}{kai1fa1}{4,5}{⼶、⼜}[HSK 3]
  \definition{v.}{explorar | tornar acessível}
\end{entry}

\begin{entry}{开发区}{kai1fa1qu1}{4,5,4}{⼶、⼜、⼖}
  \definition{s.}{zona de desenvolvimento}
\end{entry}

\begin{entry}{开放}{kai1fang4}{4,8}{⼶、⽅}[HSK 3]
  \definition{adj.}{de mente aberta; sem restrições por convenções}
  \definition{v.}{florescer | abrir (para o público) | diminuir uma proibição, restrição, etc. (de política)}
\end{entry}

\begin{entry}{开花}{kai1hua1}{4,7}{⼶、⾋}[HSK 4]
  \definition{v.}{florescer; desabrochar; estar em flor; entrar em flor;  metáfora para um coração feliz ou um rosto sorridente | explodir}
\end{entry}

\begin{entry}{开会}{kai1 hui4}{4,6}{⼶、⼈}[HSK 1]
  \definition{v.+compl.}{realizar uma reunião | ter uma reunião | participar de uma reunião (conferência)}
\end{entry}

\begin{entry}{开机}{kai1 ji1}{4,6}{⼶、⽊}[HSK 2]
  \definition{v.}{começar a filmar um filme ou programa de TV | iniciar uma máquina}
\end{entry}

\begin{entry}{开口}{kai1kou3}{4,3}{⼶、⼝}
  \definition{v.}{abrir a boca de alguém | começar a falar}
\end{entry}

\begin{entry}{开启}{kai1qi3}{4,7}{⼶、⼝}
  \definition{v.}{abrir | iniciar | (computação) ativar}
\end{entry}

\begin{entry}{开始}{kai1shi3}{4,8}{⼶、⼥}[HSK 3]
  \definition{adv.}{inicial}
  \definition[个]{s.}{começo; início; estágio inicial}
  \definition{v.}{começar; iniciar}
\end{entry}

\begin{entry}{开水}{kai1shui3}{4,4}{⼶、⽔}[HSK 4]
  \definition[杯,瓶]{s.}{água fervida; água fervente}
\end{entry}

\begin{entry}{开锁}{kai1suo3}{4,12}{⼶、⾦}
  \definition{v.}{desbloquear | destravar}
\end{entry}

\begin{entry}{开头}{kai1tou2}{4,5}{⼶、⼤}
  \definition{s.}{início | começo}
  \definition{v.+compl.}{iniciar | começar | fazer um começo}
\end{entry}

\begin{entry}{开玩笑}{kai1 wan2xiao4}{4,8,10}{⼶、⽟、⽵}[HSK 1]
  \definition{v.}{contar uma piada | brincar | fazer piada de | pregar uma peça | provocar}
\end{entry}

\begin{entry}{开心}{kai1xin1}{4,4}{⼶、⼼}[HSK 2]
  \definition{v.}{sentir-se feliz | regozijar-se | divertir-se | tirar sarro de alguém}
\end{entry}

\begin{entry}{开学}{kai1 xue2}{4,8}{⼶、⼦}[HSK 2]
  \definition{v.}{iniciar as aulas | iniciar o semestre | começar as aulas}
\end{entry}

\begin{entry}{开业}{kai1 ye4}{4,5}{⼶、⼀}[HSK 3]
  \definition{v.}{iniciar um negócio; abrir para negócios | abrir um consultório particular}
\end{entry}

\begin{entry}{开夜车}{kai1ye4che1}{4,8,4}{⼶、⼣、⾞}
  \definition{expr.}{trabalho noturno | (literalmente) ``conduzir carro à noite''}
\end{entry}

\begin{entry}{开展}{kai1zhan3}{4,10}{⼶、⼫}[HSK 3]
  \definition{v.}{lançar; desenvolver | abrir; inaugurar}
\end{entry}

\begin{entry}{看}{kan1}{9}{⽬}
  \definition{v.}{cuidar | vigiar}
  \seeref{看}{kan4}
\end{entry}

\begin{entry}{砍}{kan3}{9}{⽯}
  \definition{v.}{cortar}
\end{entry}

\begin{entry}{砍刀}{kan3dao1}{9,2}{⽯、⼑}
  \definition{s.}{facão | machete}
\end{entry}

\begin{entry}{砍掉}{kan3diao4}{9,11}{⽯、⼿}
  \definition{v.}{amputar}
\end{entry}

\begin{entry}{砍断}{kan3duan4}{9,11}{⽯、⽄}
  \definition{v.}{cortar}
\end{entry}

\begin{entry}{砍价}{kan3jia4}{9,6}{⽯、⼈}
  \definition{v.}{barganhar | cortar ou derrubar um preço}
\end{entry}

\begin{entry}{砍杀}{kan3sha1}{9,6}{⽯、⽊}
  \definition{v.}{atacar com arma branca}
\end{entry}

\begin{entry}{砍伤}{kan3shang1}{9,6}{⽯、⼈}
  \definition{v.}{ferir com lâmina ou machado}
\end{entry}

\begin{entry}{砍树}{kan3shu4}{9,9}{⽯、⽊}
  \definition{v.}{derrubar árvores}
\end{entry}

\begin{entry}{砍死}{kan3si3}{9,6}{⽯、⽍}
  \definition{v.}{matar com um machado}
\end{entry}

\begin{entry}{砍头}{kan3tou2}{9,5}{⽯、⼤}
  \definition{v.}{decapitar}
\end{entry}

\begin{entry}{看}{kan4}{9}{⽬}[HSK 1]
  \definition{interj.}{Cuidado! (para um perigo)}
  \definition{part.}{(depois de um verbo) tentar}
  \definition{v.}{olhar | ver | assistir | ler | visitar (pessoas)}
  \seeref{看}{kan1}
\end{entry}

\begin{entry}{看病}{kan4 bing4}{9,10}{⽬、⽧}[HSK 1]
  \definition{v.+compl.}{(médico) ver um paciente | (paciente) consultar (ver) um médico}
\end{entry}

\begin{entry}{看不起}{kan4bu5qi3}{9,4,10}{⽬、⼀、⾛}[HSK 4]
  \definition{v.}{desprezar; desdenhar; menosprezar; ter desprezo; olhar de cima para baixo}
\end{entry}

\begin{entry}{看淡}{kan4dan4}{9,11}{⽬、⽔}
  \definition{v.}{considerar sem importância | ser indiferente a (fama, riqueza, etc.) | (de uma economia ou mercado) enfraquecer, ficar mais lento, diminuir a velocidade}
\end{entry}

\begin{entry}{看到}{kan4 dao4}{9,8}{⽬、⼑}[HSK 1]
  \definition{v.}{ver}
\end{entry}

\begin{entry}{看法}{kan4fa3}{9,8}{⽬、⽔}[HSK 2]
  \definition[个]{s.}{modo de olhar alguma coisa | ponto de vista | opinião}
\end{entry}

\begin{entry}{看见}{kan4 jian4}{9,4}{⽬、⾒}[HSK 1]
  \definition{v.}{encontrar | enxergar | ver | avistar}
\end{entry}

\begin{entry}{看来}{kan4 lai2}{9,7}{⽬、⽊}[HSK 4]
  \definition{adv.}{parecer; parecer como se (ou embora); refere-se a um julgamento aproximado; expressa um julgamento por observação}
  \definition{v.}{ser considerado; na visão de alguém; na opinião de alguém; expressar a ideia aproximada que o locutor tem da situação}
\end{entry}

\begin{entry}{看起来}{kan4 qi3 lai5}{9,10,7}{⽬、⾛、⽊}[HSK 3]
  \definition{v.}{parecer; parecer com}
\end{entry}

\begin{entry}{看上去}{kan4 shang4 qu4}{9,3,5}{⽬、⼀、⼛}[HSK 3]
  \definition{adv.}{parece que}
\end{entry}

\begin{entry}{看望}{kan4wang4}{9,11}{⽬、⽉}[HSK 4]
  \definition{v.}{ver; visitar; ligar; dar uma olhada; ir até os pais, idosos, professores ou amigos para cumprimentá-los}
\end{entry}

\begin{entry}{扛}{kang2}{6}{⼿}
  \definition{v.}{carregar no ombro de alguém |  (fig.) assumir (um fardo, dever, etc.)}
  \seeref{扛}{gang1}
\end{entry}

\begin{entry}{考}{kao3}{6}{⽼}[HSK 1]
  \definition*{s.}{sobrenome Kao}
  \definition{s.}{o pai falecido de alguém}
  \definition{v.}{examinar | dar (fazer) um exame, prova ou teste | verificar | inspecionar |estudar | investigar}
\end{entry}

\begin{entry}{考察}{kao3cha2}{6,14}{⽼、⼧}[HSK 4]
  \definition{v.}{inspecionar; investigar; observar e estudar}
\end{entry}

\begin{entry}{考虑}{kao3lv4}{6,10}{⽼、⾌}[HSK 4]
  \definition{v.}{considerar; refletir sobre; levar em conta}
\end{entry}

\begin{entry}{考生}{kao3 sheng1}{6,5}{⽼、⽣}[HSK 2]
  \definition{s.}{candidato a exame}
\end{entry}

\begin{entry}{考试}{kao3shi4}{6,8}{⽼、⾔}[HSK 1]
  \definition[次]{s.}{teste | prova | exame}
  \definition{v.+compl.}{submeter-se a uma prova | fazer um teste}
\end{entry}

\begin{entry}{考验}{kao3yan4}{6,10}{⽼、⾺}[HSK 3]
  \definition[场,个,种]{s.}{teste; julgamento}
  \definition{v.}{testar}
\end{entry}

\begin{entry}{烤}{kao3}{10}{⽕}
  \definition{v.}{assar | grelhar}
\end{entry}

\begin{entry}{烤肉}{kao3rou4}{10,6}{⽕、⾁}
  \definition{s.}{churrasco}
\end{entry}

\begin{entry}{靠}{kao4}{15}{⾮}[HSK 2]
  \definition{prep.}{para | (chegar) perto |por | por força de | em}
  \definition{v.}{encostar-se (em) | chegar perto de | estar perto de | depender de | confiar em |confiar}
\end{entry}

\begin{entry}{科}{ke1}{9}{⽲}[HSK 2]
  \definition*{s.}{sobrenome Ke}
  \definition{s.}{um ramo de estudo acadêmico ou profissional |uma divisão ou subdivisão de uma unidade administrativa | família | instruções de palco no drama chinês clássico}
\end{entry}

\begin{entry}{科技}{ke1 ji4}{9,7}{⽲、⼿}[HSK 3]
  \definition{s.}{ciência e tecnologia}
\end{entry}

\begin{entry}{科学}{ke1xue2}{9,8}{⽲、⼦}[HSK 2]
  \definition{adj.}{científico}
  \definition[门]{s.}{ciência}
\end{entry}

\begin{entry}{科学家}{ke1xue2jia1}{9,8,10}{⽲、⼦、⼧}
  \definition[个]{s.}{cientista}
\end{entry}

\begin{entry}{棵}{ke1}{12}{⽊}[HSK 4]
  \definition{clas.}{para plantas, árvores}
\end{entry}

\begin{entry}{颗}{ke1}{14}{⾴}
  \definition{clas.}{para grãos, pérolas, dentes, corações, satelites, pequenas esferas, etc.}
\end{entry}

\begin{entry}{蝌蚪}{ke1dou3}{15,10}{⾍、⾍}
  \definition{s.}{girino}
\end{entry}

\begin{entry}{壳}{ke2}{7}{⼠}
  \definition{s.}{casca (de ovo, noz, caranguejo, etc.) | caixa | invólucro | alojamento (de uma máquina ou dispositivo)}
\end{entry}

\begin{entry}{咳嗽}{ke2sou5}{9,14}{⼝、⼝}
  \definition{v.}{ter tosse | tossir}
\end{entry}

\begin{entry}{可}{ke3}{5}{⼝}
  \definition{adv.}{muito | realmente}
\end{entry}

\begin{entry}{可爱}{ke3'ai4}{5,10}{⼝、⽖}[HSK 2]
  \definition{adj.}{adorável | querido | fofo}
\end{entry}

\begin{entry}{可编程}{ke3bian1cheng2}{5,12,12}{⼝、⽷、⽲}
  \definition{adj.}{programável}
\end{entry}

\begin{entry*}{可擦写可编程只读存储器}{ke3ca1xie3ke3bian1cheng2zhi1du2cun2chu3qi4}{5,17,5,5,12,12,5,10,6,12,16}{⼝、⼿、⼍、⼝、⽷、⽲、⼝、⾔、⼦、⼈、⼝}
  \definition{s.}{EPROM (\emph{erasable programmable read-only memory})}
\end{entry*}

\begin{entry}{可见}{ke3jian4}{5,4}{⼝、⾒}[HSK 4]
  \definition{adj.}{visível; concebível; algo que é óbvio ou evidente}
  \definition{conj.}{isso mostra; isto prova; é, portanto, claro (ou evidente, óbvio) que}
  \definition{v.}{ser ou estar visível ; ser ou estar claro}
\end{entry}

\begin{entry}{可卡因}{ke3ka3yin1}{5,5,6}{⼝、⼘、⼞}
  \definition{s.}{(empréstimo linguístico) cocaína}
\end{entry}

\begin{entry}{可靠}{ke3kao4}{5,15}{⼝、⾮}[HSK 3]
  \definition{adj.}{confiável | verdadeiro; autêntico}
\end{entry}

\begin{entry}{可口可乐}{ke3kou3ke3le4}{5,3,5,5}{⼝、⼝、⼝、⼃}
  \definition*{s.}{(empréstimo linguístico) Coca-Cola}
\end{entry}

\begin{entry}{可乐}{ke3 le4}{5,5}{⼝、⼃}[HSK 3]
  \definition*{s.}{\emph{coke}; coca; coca-cola}
\end{entry}

\begin{entry}{可能}{ke3neng2}{5,10}{⼝、⾁}[HSK 2]
  \definition{adj.}{possível | provável}
  \definition{adv.}{possivelmente | provavelmente}
  \definition[个]{s.}{possibilidade | probabilidade}
\end{entry}

\begin{entry}{可怕}{ke3pa4}{5,8}{⼝、⼼}[HSK 2]
  \definition{adj.}{horrível | terrível | formidável | assustador | hediondo}
  \definition{adv.}{terrivelmente}
\end{entry}

\begin{entry}{可是}{ke3shi4}{5,9}{⼝、⽇}[HSK 2]
  \definition{adv.}{(usado para dar ênfase) de fato}
  \definition{conj.}{porém | contudo | mas}
\end{entry}

\begin{entry}{可惜}{ke3xi1}{5,11}{⼝、⼼}
  \definition{adj.}{é uma pena | que pena}
  \definition{adv.}{infelizmente | que pena | é uma pena}
\end{entry}

\begin{entry}{可以}{ke3yi3}{5,4}{⼝、⼈}[HSK 2]
  \definition{v.}{ser capaz de | poder}
\end{entry}

\begin{entry}{渴}{ke3}{12}{⽔}[HSK 1]
  \definition{adj.}{sedento}
\end{entry}

\begin{entry}{克}{ke4}{7}{⼗}[HSK 2]
  \definition{clas.}{grama (g)}
  \definition{v.}{pode | ser capaz de | restringir | controlar | superar | subjugar | capturar (uma cidade, etc.) | digerir | cortar | reduzir | definir um limite de tempo}
\end{entry}

\begin{entry}{克服}{ke4fu2}{7,8}{⼗、⽉}[HSK 3]
  \definition{v.}{sobrepujar; superar; conquistar | suportar (dificuldades, inconveniências, etc.)}
\end{entry}

\begin{entry}{刻}{ke4}{8}{⼑}[HSK 2]
  \definition{clas.}{para curtos intervalos de tempo}
  \definition{s.}{quarto (de hora)}
  \definition{v.}{esculpir | cortar | gravar}
\end{entry}

\begin{entry}{刻画}{ke4hua4}{8,8}{⼑、⽥}
  \definition{v.}{retratar | tirar um retrato}
\end{entry}

\begin{entry}{刻钟}{ke4 zhong1}{8,9}{⼑、⾦}
  \definition{s.}{um quarto de hora}
\end{entry}

\begin{entry}{客观}{ke4guan1}{9,6}{⼧、⾒}[HSK 3]
  \definition{adj.}{objetivo; justo e razoável; imparcial}
  \definition{s.}{objetivo}
\end{entry}

\begin{entry}{客气}{ke4qi5}{9,4}{⼧、⽓}
  \definition{adj.}{cortês | delicado | modesto | educado}
  \definition{v.}{fazer cerimônia}
\end{entry}

\begin{entry}{客人}{ke4ren2}{9,2}{⼧、⼈}[HSK 2]
  \definition{s.}{visitante | convidado | cliente | passageiro | viajante}
\end{entry}

\begin{entry}{客厅}{ke4ting1}{9,4}{⼧、⼚}
  \definition[间]{s.}{sala de estar | sala de visitas}
\end{entry}

\begin{entry}{课}{ke4}{10}{⾔}[HSK 1]
  \definition{s.}{aula | curso | lição | imposto | taxa |seção}
\end{entry}

\begin{entry}{课本}{ke4 ben3}{10,5}{⾔、⽊}[HSK 1]
  \definition[本]{s.}{livro do aluno | manual}
\end{entry}

\begin{entry}{课程}{ke4cheng2}{10,12}{⾔、⽲}[HSK 3]
  \definition[个,堂,节,门]{s.}{curso; currículo}
\end{entry}

\begin{entry}{课堂}{ke4 tang2}{10,11}{⾔、⼟}[HSK 2]
  \definition[间]{s.}{sala de aula}
\end{entry}

\begin{entry}{课文}{ke4 wen2}{10,4}{⾔、⽂}[HSK 1]
  \definition{s.}{texto (de uma lição)}
\end{entry}

\begin{entry}{肯定}{ken3ding4}{8,8}{⾁、⼧}
  \definition{adv.}{com certeza | certamente | definitivamente | afirmativo (resposta)}
  \definition{v.}{afirmar | ter a certeza | ser positivo | dar reconhecimento}
\end{entry}

\begin{entry}{坑}{keng1}{7}{⼟}
  \definition{s.}{poço | depressão | túnel | buraco no chão}
  \definition{v.}{enganar | trapacear}
\end{entry}

\begin{entry}{坑人}{keng1ren2}{7,2}{⼟、⼈}
  \definition{v.+compl.}{trapacear alguém}
\end{entry}

\begin{entry}{空}{kong1}{8}{⽳}[HSK 3]
  \definition*{s.}{sobrenome Kong}
  \definition{adj.}{vazio; oco; nulo}
  \definition{adv.}{por nada; em vão}
  \definition{s.}{céu; ar | vazio; vazio do mundo dos sentidos}
  \seeref{空}{kong4}
\end{entry}

\begin{entry}{空间}{kong1jian1}{8,7}{⽳、⾨}[HSK 4]
  \definition[个]{s.}{espaço; recinto; cômodo; espaço em branco; interespaço}
\end{entry}

\begin{entry}{空间站}{kong1jian1zhan4}{8,7,10}{⽳、⾨、⽴}
  \definition{s.}{estação espacial}
\end{entry}

\begin{entry}{空姐}{kong1jie3}{8,8}{⽳、⼥}
  \definition{s.}{aeromoça | comissária de bordo | abreviação de 空中小姐}
  \seeref{空中小姐}{kong1zhong1xiao3jie3}
\end{entry}

\begin{entry}{空气}{kong1qi4}{8,4}{⽳、⽓}[HSK 2]
  \definition{s.}{ar | atmosfera}
\end{entry}

\begin{entry}{空调}{kong1tiao2}{8,10}{⽳、⾔}[HSK 3]
  \definition[台]{s.}{ar-condicionado;  condicionador de ar}
\end{entry}

\begin{entry}{空心菜}{kong1xin1cai4}{8,4,11}{⽳、⼼、⾋}
  \definition{s.}{espinafre aquático | \emph{ong choy} | repolho do pântano | convolvulus aquático | glória-da-manhã aquática}
  \seealsoref{蕹菜}{weng4cai4}
\end{entry}

\begin{entry}{空中小姐}{kong1zhong1xiao3jie3}{8,4,3,8}{⽳、⼁、⼩、⼥}
  \definition{s.}{aeromoça}
\end{entry}

\begin{entry}{孔}{kong3}{4}{⼦}
  \definition*{s.}{sobrenome Kong}
  \definition{clas.}{para habitações em cavernas}
  \definition[个]{s.}{buraco}
\end{entry}

\begin{entry}{孔夫子}{kong3fu1zi3}{4,4,3}{⼦、⼤、⼦}
  \definition*{s.}{Confúcio (551-479 aC), pensador e filósofo social chinês}
  \seealsoref{孔子}{kong3zi3}
\end{entry}

\begin{entry}{孔雀}{kong3que4}{4,11}{⼦、⾫}
  \definition{s.}{pavão}
\end{entry}

\begin{entry}{孔子}{kong3zi3}{4,3}{⼦、⼦}
  \definition*{s.}{Confúcio (551-479 aC), pensador e filósofo social chinês}
  \seealsoref{孔夫子}{kong3fu1zi3}
\end{entry}

\begin{entry}{孔子学院}{kong3zi3 xue2yuan4}{4,3,8,9}{⼦、⼦、⼦、⾩}
  \definition*{s.}{Instituto Confúcio, organização estabelecida internacionalmente pela República Popular da China, que promove a língua e a cultura chinesas}
\end{entry}

\begin{entry}{恐怖主义}{kong3bu4zhu3yi4}{10,8,5,3}{⼼、⼼、⼂、⼂}
  \definition{adj.}{terrorista}
  \definition{s.}{terrorismo}
\end{entry}

\begin{entry}{恐龙}{kong3long2}{10,5}{⼼、⿓}
  \definition[头,只]{s.}{dinossauro}
\end{entry}

\begin{entry}{恐怕}{kong3pa4}{10,8}{⼼、⼼}[HSK 3]
  \definition{adv.}{talvez; provavelmente; pode ser | por medo de}
  \definition{v.}{ter medo de; temer; recear}
\end{entry}

\begin{entry}{空}{kong4}{8}{⽳}[HSK 4]
  \definition*{s.}{sobrenome Quan}
  \definition{adj.}{vazio; oco; nulo; que não contém nada; que não tem nada ou nenhum conteúdo; impraticável}
  \definition{adv.}{para nada; em vão; sem efeito}
  \definition{s.}{céu; ar | vazio; ausência do mundo dos sentidos}
  \seeref{空}{kong1}
\end{entry}

\begin{entry}{空儿}{kong4r5}{8,2}{⽳、⼉}[HSK 3]
  \definition{s.}{tempo livre | espaço (não utilizado)}
  \definition{v.}{ter tempo livre}
\end{entry}

\begin{entry}{控制}{kong4zhi4}{11,8}{⼿、⼑}
  \definition{v.}{controlar}
\end{entry}

\begin{entry}{口}{kou3}{3}{⼝}[HSK 1][Kangxi 30]
  \definition{clas.}{para coisas com bocas (pessoas, animais domésticos, canhões, etc.) | para mordidas ou bocados}
  \definition{s.}{boca}
\end{entry}

\begin{entry}{口袋}{kou3dai4}{3,11}{⼝、⾐}[HSK 4]
  \definition[个]{s.}{bolso | saco; sacola; artigos de tecido ou couro}
\end{entry}

\begin{entry}{口袋妖怪}{kou3dai4 yao1guai4}{3,11,7,8}{⼝、⾐、⼥、⼼}
  \definition*{s.}{\emph{Pokémon}}
\end{entry}

\begin{entry}{口香糖}{kou3xiang1tang2}{3,9,16}{⼝、⾹、⽶}
  \definition{s.}{goma de mascar | chiclete}
\end{entry}

\begin{entry}{口音}{kou3yin1}{3,9}{⼝、⾳}
  \definition{s.}{sons da fala oral (linguística)}
  \seeref{口音}{kou3yin5}
\end{entry}

\begin{entry}{口音}{kou3yin5}{3,9}{⼝、⾳}
  \definition{s.}{sotaque | voz}
  \seeref{口音}{kou3yin1}
\end{entry}

\begin{entry}{口语}{kou3 yu3}{3,9}{⼝、⾔}[HSK 4]
  \definition[门]{s.}{linguagem oral; linguagem falada; linguagem coloquial; linguagem usada em conversas}
\end{entry}

\begin{entry}{枯木}{ku1mu4}{9,4}{⽊、⽊}
  \definition{s.}{árvore morta | madeira morta}
\end{entry}

\begin{entry}{哭}{ku1}{10}{⼝}[HSK 2]
  \definition{v.}{chorar}
\end{entry}

\begin{entry}{哭墙}{ku1qiang2}{10,14}{⼝、⼟}
  \definition*{s.}{Muro das Lamentações (Jerusalém)}
\end{entry}

\begin{entry}{苦}{ku3}{8}{⾋}[HSK 4]
  \definition{adj.}{amargo | difícil; doloroso; sofrido | desgastado; gasto demais}
  \definition{adv.}{meticulosamente; fazendo o máximo possível; de forma árdua; pacientemente}
  \definition{v.}{causar sofrimento a alguém; causar dificuldades a alguém | sofrer com; ser incomodado por; sentir-se angustiado com uma situação}
\end{entry}

\begin{entry}{苦瓜}{ku3gua1}{8,5}{⾋、⽠}
  \definition{s.}{melão amargo (cabaça amarga, pêra bálsamo, maçã bálsamo, pepino amargo)}
\end{entry}

\begin{entry}{裤子}{ku4zi5}{12,3}{⾐、⼦}[HSK 3]
  \definition[条]{s.}{calças}
\end{entry}

\begin{entry}{酷}{ku4}{14}{⾣}
  \definition{adj.}{impiedoso | forte (por exemplo, vinho) | (empréstimo linguístico) legal, \emph{cool}}
\end{entry}

\begin{entry}{酷斯拉}{ku4si1la1}{14,12,8}{⾣、⽄、⼿}
  \definition*{s.}{Godzilla (Japonês ゴジラ Gojira)}
  \seealsoref{哥斯拉}{ge1si1la1}
\end{entry}

\begin{entry}{会}{kuai4}{6}{⼈}
  \definition{s.}{contabilidade}
  \definition{v.}{equilibrar uma conta}
  \seeref{会}{hui4}
\end{entry}

\begin{entry}{会计}{kuai4ji4}{6,4}{⼈、⾔}[HSK 4]
  \definition[个,位,名]{s.}{contabilidade | contador; contabilista; guarda-livros; pessoal que trabalha como contador}
\end{entry}

\begin{entry}{块}{kuai4}{7}{⼟}[HSK 1]
  \definition{clas.}{(coloquial) para dinheiro e unidades monetárias | para peças ou pedaços de roupa, bolos, sabão, etc.}
  \definition{s.}{pedaço | pedaço (de terra) | peça}
\end{entry}

\begin{entry}{快}{kuai4}{7}{⼼}[HSK 1]
  \definition{adj.}{quase | rápido | depressa}
  \definition{v.}{apressar-se}
\end{entry}

\begin{entry}{快餐}{kuai4 can1}{7,16}{⼼、⾷}[HSK 2]
  \definition[份,顿]{s.}{comida rápida | \emph{fast food}}
\end{entry}

\begin{entry}{快递}{kuai4 di4}{7,10}{⼼、⾡}[HSK 4]
  \definition[个]{s.}{correio rápido; entrega expressa; entrega rápida}
  \definition{v.}{entregar (serviço de entrega rápida por transportadoras especializadas)}
\end{entry}

\begin{entry}{快点儿}{kuai4 dian3r5}{7,9,2}{⼼、⽕、⼉}[HSK 2]
  \definition{v.}{apressar-se}
\end{entry}

\begin{entry}{快乐}{kuai4le4}{7,5}{⼼、⼃}[HSK 2]
  \definition{adj.}{feliz | alegre}
  \definition{s.}{felicidade | alegria}
\end{entry}

\begin{entry}{快速}{kuai4 su4}{7,10}{⼼、⾡}[HSK 3]
  \definition{adj.}{rápido; veloz; de alta velocidade}
\end{entry}

\begin{entry}{快要}{kuai4 yao4}{7,9}{⼼、⾑}[HSK 2]
  \definition{adv.}{estar prestes a | estar indo para | estar à beira de | em breve | em nenhum momento}
\end{entry}

\begin{entry}{筷子}{kuai4zi5}{13,3}{⽵、⼦}[HSK 2]
  \definition[对,根,把,双]{s.}{pauzinhos | \emph{chopsticks}}
\end{entry}

\begin{entry}{宽}{kuan1}{10}{⼧}[HSK 4]
  \definition*{s.}{sobrenome Kuan}
  \definition{adj.}{largo; amplo; grandes distâncias horizontais | leniente; generoso; indulgente | bem de vida; confortável | espaçoso}
  \definition{s.}{largura; amplitude}
  \definition{v.}{relaxar; aliviar}
\end{entry}

\begin{entry}{宽广}{kuan1 guang3}{10,3}{⼧、⼴}[HSK 4]
  \definition{adj.}{vasto; amplo; espaçoso; extenso}
\end{entry}

\begin{entry}{宽影片}{kuan1ying3pian4}{10,15,4}{⼧、⼺、⽚}
  \definition{s.}{filme \emph{widescreen}}
\end{entry}

\begin{entry}{款}{kuan3}{12}{⽋}
  \definition{clas.}{para versões ou modelos (de um produto)}
  \definition[笔,个]{s.}{montante de dinheiro | fundos | parágrafo | seção}
\end{entry}

\begin{entry}{窾}{kuan3}{17}{⽳}
  \definition{adj.}{oco}
  \definition{s.}{rachadura | cavidade | (onomatopéia) água atingindo a rocha}
  \definition{v.}{escavar um buraco}
  \seeref{窾}{cuan4}
\end{entry}

\begin{entry}{狂欢节}{kuang2huan1jie2}{7,6,5}{⽝、⽋、⾋}
  \definition*{s.}{Carnaval}
\end{entry}

\begin{entry}{况且}{kuang4qie3}{7,5}{⼎、⼀}
  \definition{conj.}{além disso | além do mais}
\end{entry}

\begin{entry}{旷野}{kuang4ye3}{7,11}{⽇、⾥}
  \definition{s.}{região selvagem}
\end{entry}

\begin{entry}{矿泉水}{kuang4quan2shui3}{8,9,4}{⽯、⽔、⽔}[HSK 4]
  \definition[瓶,杯]{s.}{água mineral de nascente}
\end{entry}

\begin{entry}{葵花}{kui2hua1}{12,7}{⾋、⾋}
  \definition{s.}{girassol (flor)}
\end{entry}

\begin{entry}{困}{kun4}{7}{⼞}[HSK 3]
  \definition{adj.}{cansado | sonolento}
  \definition{v.}{estar encalhado; estar em grande pressão | cercar; prender; sitiar; cercar; rodear}
\end{entry}

\begin{entry}{困难}{kun4nan5}{7,10}{⼞、⾫}[HSK 3]
  \definition{adj.}{dificuldades financeiras; circunstâncias difíceis | complicado; nodoso; difícil; duro;}
  \definition[种]{s.}{dificuldade; situação difícil}
\end{entry}

\begin{entry}{扩大}{kuo4da4}{6,3}{⼿、⼤}[HSK 4]
  \definition{v.}{ampliar; expandir; estender; alargar}
\end{entry}

\begin{entry}{扩展}{kuo4 zhan3}{6,10}{⼿、⼫}[HSK 4]
  \definition{v.}{esticar; expandir; estender; espalhar}
\end{entry}

\begin{entry}{括号}{kuo4 hao4}{9,5}{⼿、⼝}[HSK 4]
  \definition{s.}{chaves, colchetes e parênteses (em fórmulas aritméticas ou algébricas, os símbolos que indicam a combinação e a ordem de vários números ou termos) | colchetes e parênteses usados como um tipo de sinal de pontuação para mostrar a parte explicativa de uma passagem em um texto}
\end{entry}

%%%%% EOF %%%%%

