%%%
%%% K
%%%

\section*{K}\addcontentsline{toc}{section}{K}

\begin{entry}{咖}{ka1}{8}{⼝}
  \definition[杯]{s.}{classe | café | graduação}
\end{entry}

\begin{entry}{咖啡}{ka1fei1}{8,11}{⼝、⼝}[HSK 3]
  \definition[杯,瓶,罐,壶,包,袋,盒]{s.}{(empréstimo linguístico) café}
\end{entry}

\begin{entry}{咖啡馆}{ka1fei1guan3}{8,11,11}{⼝、⼝、⾷}
  \definition[家]{s.}{cafeteria}
\end{entry}

\begin{entry}{咖啡色}{ka1fei1 se4}{8,11,6}{⼝、⼝、⾊}
  \definition{s.}{cor café}
\end{entry}

\begin{entry}{卡}{ka3}{5}{⼘}[HSK 2]
  \definition{clas.}{calorias (cal)}
  \definition[张,片]{s.}{cartão; documento semelhante a um cartão | cassete; dispositivo tipo compartimento para colocar fitas cassete no gravador | caminhão}
  \seeref{卡}{qia3}
\end{entry}

\begin{entry}{卡车司机}{ka3che1 si1ji1}{5,4,5,6}{⼘、⾞、⼝、⽊}
  \definition{s.}{motorista de caminhão}
\end{entry}

\begin{entry}{卡片}{ka3pian4}{5,4}{⼘、⽚}
  \definition{s.}{cartão}
\end{entry}

\begin{entry}{卡片游戏}{ka3pian4 you2xi4}{5,4,12,6}{⼘、⽚、⽔、⼽}
  \definition{s.}{carta de baralho}
\end{entry}

\begin{entry}{卡通}{ka3tong1}{5,10}{⼘、⾡}
  \definition{s.}{(empréstimo linguístico) \emph{cartoon}}
\end{entry}

\begin{entry}{开}{kai1}{4}{⼶}[HSK 1]
  \definition*{s.}{Sobrenome Kai}
  \definition{clas.}{divisão do papel de impressão de tamanho padrão (uma parte da folha inteira) | quilate; unidade de cálculo da quantidade de ouro puro contida no ouro}
  \definition{s.}{porcentagem; percentual}
  \definition{v.}{abrir; estar ligado; ligar | recuperar; abrir; fazer uma abertura; escavar; abrir caminho; desbravar | abrir para fora; soltar-se | descongelar (rios); tornar-se navegável | levantar; libertar | iniciar; operar; manobrar | mover; estabelecer | executar; configurar | começar; iniciar | manter | escrever; fazer uma lista de | pagamento (salários, tarifas, etc.) | ferver}
  \definition{v.aux.}{usado após um verbo, indica ampliação ou expansão | usado após um verbo, indica o início e a continuidade}
  \seealsoref{开尔文}{kai1'er3wen2}
\end{entry}

\begin{entry}{开车}{kai1 che1}{4,4}{⼶、⾞}[HSK 1]
  \definition{v.+compl.}{dirigir um carro, trem, etc. | colocar uma máquina em funcionamento | (de um trem, etc.) partida | dirigir veículos motorizados}
\end{entry}

\begin{entry}{开创}{kai1 chuang4}{4,6}{⼶、⼑}[HSK 6]
  \definition{v.}{começar; iniciar; fundar; ser pioneiro; estabelecer; criar}
\end{entry}

\begin{entry}{开尔文}{kai1'er3wen2}{4,5,4}{⼶、⼩、⽂}
  \definition{s.}{Kelvin, temperatura absoluta | K, escala de temperatura}
\end{entry}

\begin{entry}{开发}{kai1fa1}{4,5}{⼶、⼜}[HSK 3]
  \definition{v.}{explorar; trabalhar com recursos naturais como terras baldias, minas, florestas e energia hidráulica para fins de aproveitamento | tornar acessível; descobrir ou explorar talentos, tecnologias, etc. para aproveitamento}
\end{entry}

\begin{entry}{开发区}{kai1fa1qu1}{4,5,4}{⼶、⼜、⼖}
  \definition{s.}{zona de desenvolvimento}
\end{entry}

\begin{entry}{开放}{kai1fang4}{4,8}{⼶、⽅}[HSK 3]
  \definition{adj.}{de mente aberta; sem restrições por convenções; pensamento e ambiente não conservadores, disposto a aceitar coisas novas e novas ideias; personalidade alegre}
  \definition{v.}{florescer | abrir (para o público); levantar bloqueios, proibições, restrições, etc. | diminuir uma proibição, restrição, etc. (de política); (economia) reduzir as restrições políticas, com justificativas específicas}
\end{entry}

\begin{entry}{开关}{kai1 guan1}{4,6}{⼶、⼋}[HSK 6]
  \definition[个,种,些]{s.}{interruptor; um dispositivo que conecta e desconecta o circuito de um dispositivo elétrico | registro; um dispositivo instalado em uma tubulação de fluido para controlar o fluxo}
\end{entry}

\begin{entry}{开花}{kai1hua1}{4,7}{⼶、⾋}[HSK 4]
  \definition{v.}{florescer; desabrochar; estar em flor; entrar em flor;  metáfora para um coração feliz ou um rosto sorridente | explodir}
\end{entry}

\begin{entry}{开会}{kai1 hui4}{4,6}{⼶、⼈}[HSK 1]
  \definition{v.+compl.}{realizar uma reunião; ter uma reunião; participar de uma reunião (conferência)}
\end{entry}

\begin{entry}{开机}{kai1 ji1}{4,6}{⼶、⽊}[HSK 2]
  \definition{v.}{começar a filmar um filme ou programa de TV; refere-se ao início das filmagens (de filmes, séries de TV, etc.) | ligar uma máquina}
\end{entry}

\begin{entry}{开口}{kai1kou3}{4,3}{⼶、⼝}
  \definition{v.}{abrir a boca de alguém | começar a falar}
\end{entry}

\begin{entry}{开幕}{kai1 mu4}{4,13}{⼶、⼱}[HSK 5]
  \definition{v.}{começar a apresentação; iniciar o espetáculo; levantar das cortinas | abrir; inaugurar; iniciar (uma conferência, exposição, etc.)}
\end{entry}

\begin{entry}{开幕式}{kai1mu4shi4}{4,13,6}{⼶、⼱、⼷}[HSK 5]
  \definition[场,次,届]{s.}{cerimônia de abertura; cerimônias e apresentações antes de eventos esportivos ou grandes eventos}
\end{entry}

\begin{entry}{开启}{kai1qi3}{4,7}{⼶、⼝}
  \definition{v.}{abrir | iniciar | (computação) ativar}
\end{entry}

\begin{entry}{开设}{kai1 she4}{4,6}{⼶、⾔}[HSK 6]
  \definition{v.}{montar; estabelecer; abrir (uma loja, fábrica, etc.); estabelecer novas instituições ou campos | oferecer (um curso na faculdade, etc.)}
\end{entry}

\begin{entry}{开始}{kai1shi3}{4,8}{⼶、⼥}[HSK 3]
  \definition[个]{s.}{começo; início; estágio inicial}
  \definition{v.}{começar; iniciar; começar a fazer algo}
\end{entry}

\begin{entry}{开水}{kai1shui3}{4,4}{⼶、⽔}[HSK 4]
  \definition[杯,瓶]{s.}{água fervida; água fervente}
\end{entry}

\begin{entry}{开锁}{kai1suo3}{4,12}{⼶、⾦}
  \definition{v.}{desbloquear | destravar}
\end{entry}

\begin{entry}{开通}{kai1 tong1}{4,10}{⼶、⾡}[HSK 6]
  \definition{v.}{limpar; dragar; remover obstáculos de; abrir o canal; desbloquear}
  \seeref{开通}{kai1 tong5}
\end{entry}

\begin{entry}{开通}{kai1 tong5}{4,10}{⼶、⾡}
  \definition{adj.}{liberal; mente aberta; mente moderna; mente liberal; sábio e sensato; não conservador ou teimoso}
  \seeref{开通}{kai1 tong1}
\end{entry}

\begin{entry}{开头}{kai1 tou2}{4,5}{⼶、⼤}[HSK 6]
  \definition{s.}{início; começo; o momento ou estágio do início; antecedente no tempo}
  \definition{v.+compl.}{começar, iniciar; a primeira ocorrência de um evento, ação, fenômeno, etc. | pôr-se a pé; começar}
\end{entry}

\begin{entry}{开玩笑}{kai1wan2xiao4}{4,8,10}{⼶、⽟、⽵}[HSK 1]
  \definition{v.}{fazer (ou brincar, fazer) uma piada; gracejar; zombar de; provocar; fazer uma brincadeira; zombar de alguém | tratar casualmente; dar pouca importância a; considerar como um assunto insignificante; insignificante | fazer uma brincadeira; pregar uma peça; brincar; em tom de brincadeira}
\end{entry}

\begin{entry}{开心}{kai1xin1}{4,4}{⼶、⼼}[HSK 2]
  \definition{adj.}{feliz; alegre; exultante; encantado}
  \definition{v.}{provocar; brincar; tirar sarro de alguém; zombar; divertir-se}
\end{entry}

\begin{entry}{开学}{kai1 xue2}{4,8}{⼶、⼦}[HSK 2]
  \definition{v.}{iniciar as aulas; iniciar o semestre; começar as aulas}
\end{entry}

\begin{entry}{开业}{kai1 ye4}{4,5}{⼶、⼀}[HSK 3]
  \definition[场]{v.}{iniciar um negócio; abrir para negócios | abrir um consultório particular}
\end{entry}

\begin{entry}{开夜车}{kai1 ye4 che1}{4,8,4}{⼶、⼣、⾞}[HSK 6]
  \definition{v.+compl.}{trabalhar até tarde da noite; ficar acordado até tarde da noite estudando ou trabalhando para cumprir prazos | (literalmente) ``conduzir carro à noite''}
\end{entry}

\begin{entry}{开展}{kai1zhan3}{4,10}{⼶、⼫}[HSK 3]
  \definition{v.}{lançar; desenvolver | abrir; inaugurar}
\end{entry}

\begin{entry}{看}{kan1}{9}{⽬}
  \definition{v.}{cuidar de; tomar conta de; cuidar de; proteger | manter sob vigilância}
  \seeref{看}{kan4}
\end{entry}

\begin{entry}{看管}{kan1 guan3}{9,14}{⽬、⽵}[HSK 6]
  \definition{v.}{cuidar; atender | guardar; vigiar; ficar de olho em | assumir o comando; estar no comando}
\end{entry}

\begin{entry}{砍}{kan3}{9}{⽯}
  \definition{v.}{cortar}
\end{entry}

\begin{entry}{砍刀}{kan3dao1}{9,2}{⽯、⼑}
  \definition{s.}{facão | machete}
\end{entry}

\begin{entry}{砍掉}{kan3diao4}{9,11}{⽯、⼿}
  \definition{v.}{amputar}
\end{entry}

\begin{entry}{砍断}{kan3duan4}{9,11}{⽯、⽄}
  \definition{v.}{cortar}
\end{entry}

\begin{entry}{砍价}{kan3jia4}{9,6}{⽯、⼈}
  \definition{v.}{barganhar | cortar ou derrubar um preço}
\end{entry}

\begin{entry}{砍杀}{kan3sha1}{9,6}{⽯、⽊}
  \definition{v.}{atacar com arma branca}
\end{entry}

\begin{entry}{砍伤}{kan3shang1}{9,6}{⽯、⼈}
  \definition{v.}{ferir com lâmina ou machado}
\end{entry}

\begin{entry}{砍树}{kan3shu4}{9,9}{⽯、⽊}
  \definition{v.}{derrubar árvores}
\end{entry}

\begin{entry}{砍死}{kan3si3}{9,6}{⽯、⽍}
  \definition{v.}{matar com um machado}
\end{entry}

\begin{entry}{砍头}{kan3tou2}{9,5}{⽯、⼤}
  \definition{v.}{decapitar}
\end{entry}

\begin{entry}{看}{kan4}{9}{⽬}[HSK 1,6]
  \definition{interj.}{Cuidado! (para um perigo)}
  \definition{part.}{tentar, usado depois de outros verbos}
  \definition{v.}{ver; olhar para; observar; fazer contato visual com pessoas ou objetos | pensar; considerar; observar; julgar; observar e analisar | visitar; ver; fazer uma visita | olhar para; considerar; tratar | tratar (um paciente ou uma doença) | cuidar | ficar atento; ficar de olho | depender de; ser dependente de | ler}
  \seeref{看}{kan1}
\end{entry}

\begin{entry}{看病}{kan4 bing4}{9,10}{⽬、⽧}[HSK 1]
  \definition{v.+compl.}{(de um médico) ver um paciente | (de um paciente) ver (consultar) um médico}
\end{entry}

\begin{entry}{看不起}{kan4bu5qi3}{9,4,10}{⽬、⼀、⾛}[HSK 4]
  \definition{v.}{desprezar; desdenhar; menosprezar; ter desprezo; olhar de cima para baixo}
\end{entry}

\begin{entry}{看成}{kan4 cheng2}{9,6}{⽬、⼽}[HSK 5]
  \definition{v.}{olhar como; considerar como; tratar como; pensar como; ter como}
\end{entry}

\begin{entry}{看出}{kan4 chu1}{9,5}{⽬、⼐}[HSK 5]
  \definition{v.}{perceber; descobrir; estar ciente de; ver}
\end{entry}

\begin{entry}{看待}{kan4dai4}{9,9}{⽬、⼻}[HSK 5]
  \definition{v.}{tratar; considerar; olhar com atenção; ter uma certa atitude ou visão em relação a alguém ou alguma coisa}
\end{entry}

\begin{entry}{看淡}{kan4dan4}{9,11}{⽬、⽔}
  \definition{v.}{considerar sem importância | ser indiferente a (fama, riqueza, etc.) | (de uma economia ou mercado) enfraquecer, ficar mais lento, diminuir a velocidade}
\end{entry}

\begin{entry}{看到}{kan4 dao4}{9,8}{⽬、⼑}[HSK 1]
  \definition{v.}{ver; avistar}
\end{entry}

\begin{entry}{看得见}{kan4 de5 jian4}{9,11,4}{⽬、⼻、⾒}[HSK 6]
  \definition{adj.}{perceptível; visível; tangível}
\end{entry}

\begin{entry}{看得起}{kan4 de5 qi3}{9,11,10}{⽬、⼻、⾛}[HSK 6]
  \definition{v.}{ter uma boa opinião sobre; pensar muito (ou muito) sobre}
\end{entry}

\begin{entry}{看法}{kan4fa3}{9,8}{⽬、⽔}[HSK 2]
  \definition[个,种,点]{s.}{opinião; perspectiva; (ponto de) vista; uma maneira de ver uma coisa | opinião desfavorável (ou crítica) sobre alguém}
\end{entry}

\begin{entry}{看好}{kan4 hao3}{9,6}{⽬、⼥}[HSK 6]
  \definition{v.}{elogiar; apreciar; encorajar; acreditar que pessoas ou coisas terão uma boa tendência | estar prestes a surgir uma boa tendência}
\end{entry}

\begin{entry}{看见}{kan4jian4}{9,4}{⽬、⾒}[HSK 1]
  \definition{v.}{ver; avistar; ao olhar, descobrir alguém ou algo}
\end{entry}

\begin{entry}{看来}{kan4 lai2}{9,7}{⽬、⽊}[HSK 4]
  \definition{adv.}{parecer; parecer como se (ou embora); refere-se a um julgamento aproximado; expressa um julgamento por observação}
  \definition{v.}{ser considerado; na visão de alguém; na opinião de alguém; expressar a ideia aproximada que o locutor tem da situação}
\end{entry}

\begin{entry}{看起来}{kan4 qi3 lai5}{9,10,7}{⽬、⾛、⽊}[HSK 3]
  \definition{v.}{parecer; aparentar; dar a impressão de (ou como se)}
\end{entry}

\begin{entry}{看上去}{kan4 shang4 qu4}{9,3,5}{⽬、⼀、⼛}[HSK 3]
  \definition{adv.}{parece que}
\end{entry}

\begin{entry}{看望}{kan4wang4}{9,11}{⽬、⽉}[HSK 4]
  \definition{v.}{ver; visitar; ligar; dar uma olhada; ir até os pais, idosos, professores ou amigos para cumprimentá-los}
\end{entry}

\begin{entry}{看作}{kan4 zuo4}{9,7}{⽬、⼈}[HSK 6]
  \definition{v.}{considerar como; olhar como}
\end{entry}

\begin{entry}{康}{kang1}{11}{⼴}
  \definition*{s.}{Sobrenome Kang}
  \definition{adj.}{saudável |  fácil; pacífico; abundante | amplo; largo | Dialeto: de baixa qualidade; inferior}
  \definition{s.}{bem-estar; saúde | palha; farelo; casca}
  \definition{v.}{(normalmente de um rabanete) tornar-se esponjoso}
\end{entry}

\begin{entry}{康复}{kang1 fu4}{11,9}{⼴、⼢}[HSK 6]
  \definition{v.}{Saúde: estaurar; recuperar; reabilitar}
\end{entry}

\begin{entry}{扛}{kang2}{6}{⼿}
  \definition{v.}{carregar objetos nos ombros |  suportar; aguentar | lidar; assumir}
  \seeref{扛}{gang1}
\end{entry}

\begin{entry}{抗}{kang4}{7}{⼿}[HSK 6]
  \definition*{s.}{Sobrenome Kang}
  \definition{pref.}{anti-}
  \definition{v.}{resistir; combater; lutar | recusar; desafiar}
\end{entry}

\begin{entry}{抗议}{kang4yi4}{7,5}{⼿、⾔}[HSK 6]
  \definition{v.}{protestar; reconsiderar; levantar objeções fortes}
\end{entry}

\begin{entry}{考}{kao3}{6}{⽼}[HSK 1]
  \definition*{s.}{Sobrenome Kao}
  \definition{adj.}{antigo; velho; com idade avançada}
  \definition{s.}{o pai falecido de alguém}
  \definition{v.}{examinar; dar (fazer) um exame, teste ou questionário | verificar; inspecionar | estudar; verificar; investigar | perguntar; testar; fazer perguntas para que o outro responda, a fim de testar suas habilidades em determinada área}
\end{entry}

\begin{entry}{考察}{kao3cha2}{6,14}{⽼、⼧}[HSK 4]
  \definition{v.}{inspecionar; investigar; observar e estudar}
\end{entry}

\begin{entry}{考场}{kao3 chang3}{6,6}{⽼、⼟}[HSK 6]
  \definition{s.}{sala de exames}
\end{entry}

\begin{entry}{考核}{kao3he2}{6,10}{⽼、⽊}[HSK 5]
  \definition{v.}{examinar; checar; avaliar; avaliar (a proficiência de alguém)}
\end{entry}

\begin{entry}{考虑}{kao3lv4}{6,10}{⽼、⾌}[HSK 4]
  \definition{v.}{considerar; refletir sobre; levar em conta}
\end{entry}

\begin{entry}{考生}{kao3 sheng1}{6,5}{⽼、⽣}[HSK 2]
  \definition{s.}{candidato a exame; alunos inscritos para o exame de admissão}
\end{entry}

\begin{entry}{考试}{kao3shi4}{6,8}{⽼、⾔}[HSK 1]
  \definition[次]{s.}{teste; exame; prova; atividades realizadas para verificar conhecimentos ou habilidades}
  \definition{v.+compl.}{testar; avaliar; avaliar conhecimentos e habilidades por meio de perguntas escritas ou orais.}
\end{entry}

\begin{entry}{考题}{kao3 ti2}{6,15}{⽼、⾴}[HSK 6]
  \definition{s.}{questões de exame; prova de exame; tópicos de exame}
\end{entry}

\begin{entry}{考验}{kao3yan4}{6,10}{⽼、⾺}[HSK 3]
  \definition[场,个,种]{s.}{teste; julgamento; atividade realizada para verificar se as habilidades, ideias, moral e qualidades de uma pessoa atendem aos requisitos}
  \definition{v.}{testar; testar as capacidades, ideias, moral e qualidades de uma pessoa através de situações, ações ou ambientes difíceis, para verificar se elas atendem aos requisitos}
\end{entry}

\begin{entry}{烤}{kao3}{10}{⽕}
  \definition{v.}{assar | grelhar}
\end{entry}

\begin{entry}{烤肉}{kao3 rou4}{10,6}{⽕、⾁}[HSK 5]
  \definition[块,串,片,盘]{s.}{churrasco (literalmente carne assada)}
\end{entry}

\begin{entry}{烤鸭}{kao3ya1}{10,10}{⽕、⿃}[HSK 5]
  \definition{s.}{pato assado; pato recheado e assado em um forno especial após ser abatido}
\end{entry}

\begin{entry}{靠}{kao4}{15}{⾮}[HSK 2]
  \definition{prep.}{manter (em); aproximar-se (de); ao longo de | por; graças a; com base em; de acordo com}
  \definition{s.}{armadura de palco (feita de seda bordada); armadura usada pelos generais militares antigos nas peças teatrais}
  \definition{v.}{inclinar-se; sentado ou em pé, deixar parte do peso do corpo ser suportado por outra pessoa ou objeto (pessoa) | encostar-se (em); apoiar-se ou levantar-se com a ajuda de alguma coisa | aproximar-se; estar perto de | confiar em; depender de | confiar}
\end{entry}

\begin{entry}{靠近}{kao4 jin4}{15,7}{⾮、⾡}[HSK 5]
  \definition{adv.}{próximo; perto de; ao lado de}
  \definition{v.}{aproximar-se; chegar perto; avançar em direção a um determinado objetivo de modo que a distância fique cada vez menor}
\end{entry}

\begin{entry}{科}{ke1}{9}{⽲}[HSK 2]
  \definition*{s.}{Sobrenome Ke}
  \definition{s.}{um ramo de estudo acadêmico ou profissional |uma divisão ou subdivisão de uma unidade administrativa | família | instruções de palco no drama chinês clássico; nos roteiros de peças clássicas, termos usados para indicar as ações dos personagens | nível; classificação; categoria | sessão de exames; refere-se às disciplinas, notas e anos das provas para a seleção de candidatos a cargos públicos militares e civis na antiguidade | tecnológico | assunto | lei; regulamento; decreto | penalidade; pena; punição | treinamento profissional ou formal; curso profissionalizante}
  \definition{v.}{proferir uma sentença (penal)}
\end{entry}

\begin{entry}{科技}{ke1 ji4}{9,7}{⽲、⼿}[HSK 3]
  \definition{s.}{ciência e tecnologia}
\end{entry}

\begin{entry}{科学}{ke1xue2}{9,8}{⽲、⼦}[HSK 2]
  \definition{adj.}{científico; em conformidade com as leis da ciência}
  \definition[门,个,种]{s.}{ciência; um conjunto de conhecimentos que reflete as leis objetivas da natureza, da sociedade, do pensamento, etc.}
\end{entry}

\begin{entry}{科学家}{ke1xue2jia1}{9,8,10}{⽲、⼦、⼧}
  \definition[位,名,个]{s.}{cientista; pessoas com realizações significativas no campo da pesquisa científica}
\end{entry}

\begin{entry}{科研}{ke1 yan2}{9,9}{⽲、⽯}[HSK 6]
  \definition{s.}{pesquisa científica}
  \definition{v.}{envolver-se em pesquisa científica}
\end{entry}

\begin{entry}{棵}{ke1}{12}{⽊}[HSK 4]
  \definition{clas.}{para plantas, árvores}
\end{entry}

\begin{entry}{颗}{ke1}{14}{⾴}[HSK 5]
  \definition{clas.}{para grãos, pérolas, dentes, corações, satelites, pequenas esferas, etc.}
  \definition{s.}{grão; partícula; pequenas coisas redondas}
\end{entry}

\begin{entry}{蝌}{ke1}{15}{⾍}
  \definition[只]{s.}{girino}
\end{entry}

\begin{entry}{蝌蚪}{ke1dou3}{15,10}{⾍、⾍}
  \definition{s.}{girino}
\end{entry}

\begin{entry}{壳}{ke2}{7}{⼠}
  \definition{s.}{casca (de ovo, noz, caranguejo, etc.) | caixa | invólucro | alojamento (de uma máquina ou dispositivo)}
\end{entry}

\begin{entry}{咳}{ke2}{9}{⼝}[HSK 5]
  \definition{v.}{tossir}
  \seeref{咳}{hai1}
\end{entry}

\begin{entry}{咳嗽}{ke2sou5}{9,14}{⼝、⼝}
  \definition{v.}{ter tosse | tossir}
\end{entry}

\begin{entry}{可}{ke3}{5}{⼝}[HSK 5]
  \definition*{s.}{Sobrenome Ke}
  \definition{adv.}{indica ênfase | indica o fortalecimento de perguntas retóricas | indica um tom de questionamento mais forte | sobre; a respeito de}
  \definition{conj.}{mas; ainda}
  \definition{v.}{aprovar; concordar com | poder; permitir; ser capaz de | precisar (fazer); valer a pena (fazer); merecer | ajustar; adequar | estar pronto para; estar disposto a; pretender}
  \seeref{可}{ke4}
\end{entry}

\begin{entry}{可爱}{ke3'ai4}{5,10}{⼝、⽖}[HSK 2]
  \definition{adj.}{adorável; simpático; encantador | bonitinho; adorável | amado; querido; encantador; cativante; relacionamento próximo, sentimentos profundos | fofo; bonito}
\end{entry}

\begin{entry}{可编程}{ke3bian1cheng2}{5,12,12}{⼝、⽷、⽲}
  \definition{adj.}{programável}
\end{entry}

\begin{entry*}{可擦写可编程只读存储器}{ke3ca1xie3ke3bian1cheng2zhi1du2cun2chu3qi4}{5,17,5,5,12,12,5,10,6,12,16}{⼝、⼿、⼍、⼝、⽷、⽲、⼝、⾔、⼦、⼈、⼝}
  \definition{s.}{EPROM (\emph{erasable programmable read-only memory})}
\end{entry*}

\begin{entry}{可见}{ke3jian4}{5,4}{⼝、⾒}[HSK 4]
  \definition{adj.}{visível; concebível; algo que é óbvio ou evidente}
  \definition{conj.}{isso mostra; isto prova; é, portanto, claro (ou evidente, óbvio) que}
  \definition{v.}{ser ou estar visível ; ser ou estar claro}
\end{entry}

\begin{entry}{可卡因}{ke3ka3yin1}{5,5,6}{⼝、⼘、⼞}
  \definition{s.}{(empréstimo linguístico) cocaína}
\end{entry}

\begin{entry}{可靠}{ke3kao4}{5,15}{⼝、⾮}[HSK 3]
  \definition{adj.}{confiável; digno de confiança | verdadeiro; autêntico; descrever notícias e outras informações como verdadeiras, de modo que as pessoas possam acreditar nelas}
\end{entry}

\begin{entry}{可口可乐}{ke3kou3ke3le4}{5,3,5,5}{⼝、⼝、⼝、⼃}
  \definition*{s.}{Empréstimo linguístico: Coca-Cola}
\end{entry}

\begin{entry}{可乐}{ke3 le4}{5,5}{⼝、⼃}[HSK 3]
  \definition*[罐,杯,瓶,听,口]{s.}{\emph{coke}; coca; coca-cola}
  \definition{adj.}{engraçado; divertido; risível}
\end{entry}

\begin{entry}{可怜}{ke3lian2}{5,8}{⼝、⼼}[HSK 5]
  \definition{adj.}{pobre; lamentável; lastimável | miserável (de quantidade ou qualidade); descreve um número pequeno ou um lugar tão pequeno que não vale a pena falar sobre ele}
  \definition{v.}{ter pena; ter piedade de; ter simpatia por pessoas que tiveram coisas muito ruins acontecendo com elas}
\end{entry}

\begin{entry}{可能}{ke3neng2}{5,10}{⼝、⾁}[HSK 2]
  \definition{adj.}{possível}
  \definition{adv.}{possivelmente}
  \definition[种]{s.}{possibilidade; tendências ou oportunidades que podem se tornar realidade}
\end{entry}

\begin{entry}{可怕}{ke3pa4}{5,8}{⼝、⼼}[HSK 2]
  \definition{adj.}{assustador; terrível; hediondo; medonho; horrível; aterrorizante}
  \definition{adv.}{terrivelmente}
\end{entry}

\begin{entry}{可是}{ke3shi4}{5,9}{⼝、⽇}[HSK 2]
  \definition{adv.}{de fato (usado para dar ênfase), equivalente a 的确}
  \definition{conj.}{mas; no entanto; contudo; conecta frases, expressa uma relação de transição, equivalente a 但是}
  \seealsoref{但是}{dan4 shi4}
  \seealsoref{的确}{di2que4}
\end{entry}

\begin{entry}{可惜}{ke3xi1}{5,11}{⼝、⼼}[HSK 5]
  \definition{adj.}{é uma pena; é muito ruim; é lamentável}
  \definition{adv.}{infelizmente}
\end{entry}

\begin{entry}{可以}{ke3yi3}{5,4}{⼝、⼈}[HSK 2]
  \definition{adj.}{aceitável; nada mal; muito bom | impressionante; espantoso; tremendo}
  \definition{v.}{poder; ter condições, capacidade e tempo para fazer algo ou ter alguma utilidade | permitir; poder | valer a pena fazer; considerar que vale a pena, recomendar fazer algo}
\end{entry}

\begin{entry}{渴}{ke3}{12}{⽔}[HSK 1]
  \definition{adj.}{sedento}
  \definition{adv.}{ansiosamente; metáfora de urgência}
  \definition{v.}{desejar; ansiar por}
\end{entry}

\begin{entry}{渴望}{ke3wang4}{12,11}{⽔、⽉}[HSK 5]
  \definition{v.}{aspirar; (ter sede, ansiar, desejar) por}
\end{entry}

\begin{entry}{可}{ke4}{5}{⼝}
  \definition{s.}{governante supremo de uma tribo nômade do norte; Khan (可汗), título do governante supremo dos antigos grupos étnicos xianbei, turco, uigur e mongol}
  \seeref{可}{ke3}
  \seealsoref{可汗}{ke4han2}
\end{entry}

\begin{entry}{可汗}{ke4han2}{5,6}{⼝、⽔}
  \definition{s.}{khan (empréstimo linguístico); cham}
\end{entry}

\begin{entry}{克}{ke4}{7}{⼗}[HSK 2]
  \definition*{s.}{Sobrenome Ke}
  \definition{clas.}{g, grama, unidade de peso | unidade tibetana de volume ou medida seca (com capacidade para cerca de 25 斤, de cevada) | unidade tibetana de área de terra equivalente a cerca de 1 亩}
  \definition{v.}{poder; ser capaz de | tolerar; conter; restringir; suprimir| subjugar; capturar; conquistar (uma cidade, etc.) | digerir (alimentos) | reduzir; diminuir | definir um limite de tempo}
  \seealsoref{斤}{jin1}
  \seealsoref{亩}{mu3}
\end{entry}

\begin{entry}{克服}{ke4fu2}{7,8}{⼗、⽉}[HSK 3]
  \definition{v.}{sobrepujar; superar; conquistar; vencer com força de vontade e determinação (deficiências, erros, fenômenos negativos, condições desfavoráveis, etc.) | aguentar; suportar (dificuldades, inconveniências, etc.)}
\end{entry}

\begin{entry}{刻}{ke4}{8}{⼑}[HSK 2,5]
  \definition{adj.}{cruel; severo; rude; indelicado | no mais alto grau}
  \definition{clas.}{um quarto (de uma hora, 15min)}
  \definition[件]{s.}{quarto (de hora); momento}
  \definition{v.}{esculpir; inscrever; gravar; talhar com uma faca (padrões, texto, etc.) | definir um limite de tempo | imprimir (CD)}
\end{entry}

\begin{entry}{刻画}{ke4hua4}{8,8}{⼑、⽥}
  \definition{v.}{retratar | tirar um retrato}
\end{entry}

\begin{entry}{刻钟}{ke4 zhong1}{8,9}{⼑、⾦}
  \definition{s.}{um quarto de hora}
\end{entry}

\begin{entry}{客}{ke4}{9}{⼧}
  \definition*{s.}{Sobrenome Ke}
  \definition{adj.}{objetivo; independente da consciência humana | estrangeiro; não desta região, unidade ou indústria}
  \definition{clas.}{porção (de comida, bebida, etc.); em algumas áreas, é usado para vender alimentos e bebidas em porções}
  \definition[个,位,名,些]{s.}{convidado; visitante; aquele que é convidado; aquele que vem visitar (em oposição a 主) | viajante; passageiro | comerciante viajante; refere-se especificamente a comerciantes que transportam mercadorias de um lugar para o outro | cliente; patrono; consumidor | uma pessoa envolvida em alguma atividade específica; pessoas que viajam fazendo algum tipo de atividade}
  \definition{v.}{ser um estranho; estabelecer-se (ou viver) em um lugar estranho; estar longe de casa ou morar no exterior}
  \seealsoref{主}{zhu3}
\end{entry}

\begin{entry}{客车}{ke4 che1}{9,4}{⼧、⾞}[HSK 6]
  \definition[辆,列,次,趟]{s.}{ônibus; veículo de passageiros; veículos que transportam passageiros em ferrovias e estradas}
\end{entry}

\begin{entry}{客观}{ke4guan1}{9,6}{⼧、⾒}[HSK 3]
  \definition{adj.}{objetivo; justo e razoável; imparcial; com base na situação real, sem preconceitos pessoais}
  \definition{s.}{objetivo; existe fora da consciência, sem depender da consciência subjetiva}
\end{entry}

\begin{entry}{客户}{ke4hu4}{9,4}{⼧、⼾}[HSK 5]
  \definition{s.}{cliente; consumidor}
\end{entry}

\begin{entry}{客气}{ke4qi5}{9,4}{⼧、⽓}[HSK 5]
  \definition{adj.}{educado; modesto; cortês}
  \definition{v.}{ser educado; ser cortês; fazer comentários educados ou agir educadamente}
\end{entry}

\begin{entry}{客人}{ke4ren2}{9,2}{⼧、⼈}[HSK 2]
  \definition[位,个,桌,拨,批]{s.}{visitante; convidado | cliente; passageiro; hóspede; viajante}
\end{entry}

\begin{entry}{客厅}{ke4ting1}{9,4}{⼧、⼚}[HSK 5]
  \definition[间,个]{s.}{sala de estar; sala de visitas; sala para receber convidados}
\end{entry}

\begin{entry}{课}{ke4}{10}{⾔}[HSK 1]
  \definition{clas.}{aula; lição; unidade de tempo de ensino; parágrafo do material didático}
  \definition[门,节]{s.}{classe; aula; ensino por etapas planejado | disciplina; curso | imposto; antiga referência a impostos | seção; departamentos de escritório criados no antigo governo}
  \definition{v.}{cobrar; impor; taxar}
\end{entry}

\begin{entry}{课本}{ke4 ben3}{10,5}{⾔、⽊}[HSK 1]
  \definition[本]{s.}{livro didático; livro-texto}
\end{entry}

\begin{entry}{课程}{ke4cheng2}{10,12}{⾔、⽲}[HSK 3]
  \definition[个,堂,节,门]{s.}{curso; currículo; as disciplinas e o programa letivo da escola}
\end{entry}

\begin{entry}{课堂}{ke4 tang2}{10,11}{⾔、⼟}[HSK 2]
  \definition[间,节,个]{s.}{sala de aula; local onde se realizam as aulas; local onde se realizam as atividades de ensino}
\end{entry}

\begin{entry}{课题}{ke4ti2}{10,15}{⾔、⾴}[HSK 5]
  \definition{s.}{uma questão para estudo ou discussão; principais questões a serem pesquisadas ou discutidas, ou assuntos importantes que precisam ser resolvidos com urgência | tarefa; problema; questões a serem resolvidas}
\end{entry}

\begin{entry}{课文}{ke4 wen2}{10,4}{⾔、⽂}[HSK 1]
  \definition[篇,段]{s.}{texto (de uma lição); texto principal do livro didático (diferente das notas de rodapé, exercícios, etc.)}
\end{entry}

\begin{entry}{肯}{ken3}{8}{⾁}[HSK 6]
  \definition{s.}{carne presa ao osso}
  \definition{v.}{concordar; consentir}
  \definition{v.aux.}{estar disposto a; estar pronto para; para expressar vontade subjetiva; vontade de aceitar}
\end{entry}

\begin{entry}{肯定}{ken3ding4}{8,8}{⾁、⼧}[HSK 5]
  \definition{adj.}{certo; definitivo; positivo; afirmativo | positivo; afirmativo; aceitável}
  \definition{adv.}{com certeza | certamente | definitivamente | afirmativo (resposta)}
  \definition{adv.}{certamente; definitivamente; sem dúvida; sem dúvida alguma}
  \definition{v.}{afirmar; aprovar; confirmar; considerar positivo; reconhecer a existência de algo ou sua autenticidade ou racionalidade (em oposição à 否定)}
  \seealsoref{否定}{fou3ding4}
\end{entry}

\begin{entry}{坑}{keng1}{7}{⼟}
  \definition{s.}{poço | depressão | túnel | buraco no chão}
  \definition{v.}{enganar | trapacear}
\end{entry}

\begin{entry}{坑人}{keng1ren2}{7,2}{⼟、⼈}
  \definition{v.+compl.}{trapacear alguém}
\end{entry}

\begin{entry}{空}{kong1}{8}{⽳}[HSK 3]
  \definition*{s.}{Sobrenome Kong}
  \definition{adj.}{vazio; oco; nulo; não inclui nada; não contém nada ou não tem conteúdo; irrealista}
  \definition{adv.}{por nada; em vão; sem efeito}
  \definition{s.}{céu; ar | vazio; vazio do mundo dos sentidos}
  \seeref{空}{kong4}
\end{entry}

\begin{entry}{空间}{kong1jian1}{8,7}{⽳、⾨}[HSK 4]
  \definition[个]{s.}{espaço; recinto; cômodo; espaço em branco; interespaço}
\end{entry}

\begin{entry}{空间站}{kong1jian1zhan4}{8,7,10}{⽳、⾨、⽴}
  \definition{s.}{estação espacial}
\end{entry}

\begin{entry}{空姐}{kong1jie3}{8,8}{⽳、⼥}
  \definition{s.}{aeromoça | comissária de bordo | abreviação de 空中小姐}
  \seealsoref{空中小姐}{kong1zhong1xiao3jie3}
\end{entry}

\begin{entry}{空军}{kong1 jun1}{8,6}{⽳、⼍}[HSK 6]
  \definition[名,位,个,支]{s.}{força aérea; um exército que luta no ar, geralmente composto por várias unidades de aviação e unidades terrestres da força aérea}
\end{entry}

\begin{entry}{空气}{kong1qi4}{8,4}{⽳、⽓}[HSK 2]
  \definition[缕,股,份,阵]{s.}{ar; gases que compõe a atmosfera terrestre | atmosfera}
\end{entry}

\begin{entry}{空调}{kong1tiao2}{8,10}{⽳、⾔}[HSK 3]
  \definition[台,个]{s.}{ar-condicionado;  condicionador de ar}
\end{entry}

\begin{entry}{空心菜}{kong1xin1cai4}{8,4,11}{⽳、⼼、⾋}
  \definition{s.}{espinafre aquático | \emph{ong choy} | repolho do pântano | convolvulus aquático | glória-da-manhã aquática}
  \seealsoref{蕹菜}{weng4cai4}
\end{entry}

\begin{entry}{空中}{kong1 zhong1}{8,4}{⽳、⼁}[HSK 5]
  \definition{adj.}{aéreo; aerotransportado; refere-se à transmissão de sinais de rádio}
  \definition{s.}{no céu; no ar}
\end{entry}

\begin{entry}{空中小姐}{kong1zhong1xiao3jie3}{8,4,3,8}{⽳、⼁、⼩、⼥}
  \definition{s.}{aeromoça}
\end{entry}

\begin{entry}{孔}{kong3}{4}{⼦}
  \definition*{s.}{Sobrenome Kong}
  \definition{clas.}{para habitações em cavernas}
  \definition[个]{s.}{buraco}
\end{entry}

\begin{entry}{孔夫子}{kong3fu1zi3}{4,4,3}{⼦、⼤、⼦}
  \definition*{s.}{Confúcio (551-479 aC), pensador e filósofo social chinês}
  \seealsoref{孔子}{kong3zi3}
\end{entry}

\begin{entry}{孔雀}{kong3que4}{4,11}{⼦、⾫}
  \definition{s.}{pavão}
\end{entry}

\begin{entry}{孔子}{kong3zi3}{4,3}{⼦、⼦}
  \definition*{s.}{Confúcio (551-479 aC), pensador e filósofo social chinês}
  \seealsoref{孔夫子}{kong3fu1zi3}
\end{entry}

\begin{entry}{孔子学院}{kong3zi3 xue2yuan4}{4,3,8,9}{⼦、⼦、⼦、⾩}
  \definition*{s.}{Instituto Confúcio, organização estabelecida internacionalmente pela República Popular da China, que promove a língua e a cultura chinesas}
\end{entry}

\begin{entry}{恐}{kong3}{10}{⼼}
  \definition{adv.}{talvez; provavelmente}
  \definition{v.}{temer; recear; ter medo de | ameaçar; aterrorizar; intimidar}
\end{entry}

\begin{entry}{恐怖主义}{kong3bu4zhu3yi4}{10,8,5,3}{⼼、⼼、⼂、⼂}
  \definition{adj.}{terrorista}
  \definition{s.}{terrorismo}
\end{entry}

\begin{entry}{恐龙}{kong3long2}{10,5}{⼼、⿓}
  \definition[头,只]{s.}{dinossauro}
\end{entry}

\begin{entry}{恐怕}{kong3pa4}{10,8}{⼼、⼼}[HSK 3]
  \definition{adv.}{talvez; provavelmente; pode ser; expressa suposição; estimativa. | por medo de; expressar estimativa e preocupação}
  \definition{v.}{ter medo de; temer; recear}
\end{entry}

\begin{entry}{空}{kong4}{8}{⽳}[HSK 4]
  \definition{adj.}{vazio; oco; nulo; que não contém nada; que não tem nada ou nenhum conteúdo; impraticável}
  \definition{adv.}{para nada; em vão; sem efeito}
  \definition{s.}{céu; ar | vazio; ausência do mundo dos sentidos}
  \seeref{空}{kong1}
\end{entry}

\begin{entry}{空儿}{kong4r5}{8,2}{⽳、⼉}[HSK 3]
  \definition[个]{s.}{tempo livre; sem horário específico | sala; espaço (não utilizado); área ainda não utilizada}
  \definition{v.}{ter tempo livre}
\end{entry}

\begin{entry}{控}{kong4}{11}{⼿}
  \definition{v.}{acusar; cobrar | controlar; dominar | manter (parte do corpo em uma determinada posição) sem apoio | virar (um recipiente) de cabeça para baixo para deixar o líquido escorrer}
\end{entry}

\begin{entry}{控制}{kong4zhi4}{11,8}{⼿、⼑}[HSK 5]
  \definition{v.}{controlar; restringir; dominar; fazer com que não ultrapasse um determinado limite | controlar; dominar; comandar; ocupar, fazer com que não se perca}
\end{entry}

\begin{entry}{口}{kou3}{3}{⼝}[HSK 1][Kangxi 30]
  \definition*{s.}{Sobrenome Kou}
  \definition{clas.}{usado para coisas com bocas (pessoas, animais domésticos, canhões, etc.) | usado para mordidas ou bocados | usado para idiomas}
  \definition{s.}{boca | borda; boca; o espaço externo ao recipiente | saída; entrada; local de entrada e saída | o gosto de alguém | corte; buraco; ferida |  a borda de uma faca; lâminas de facas, espadas, tesouras, etc. | a idade de um animal de tração | seção; departamento; sistema integrado de departamentos relacionados | conversa, discurso; pronunciamento; referência à fala | um portão da Grande Muralha (frequentemente usado em nomes de lugares)}
\end{entry}

\begin{entry}{口吃}{kou3chi1}{3,6}{⼝、⼝}
  \definition{s.}{gagueira; espasmofemia; balbucinato; mogilalia; battarismo; battarismo; iscnofonia; pselismo; o fenômeno de repetir palavras ou interromper frases ao falar é um defeito habitual de linguagem comumente conhecido como gagueira}
\end{entry}

\begin{entry}{口吃病}{kou3chi1 bing4}{3,6,10}{⼝、⼝、⽧}
  \definition{s.}{doença da gagueira}
\end{entry}

\begin{entry}{口袋}{kou3dai4}{3,11}{⼝、⾐}[HSK 4]
  \definition[个]{s.}{bolso | saco; sacola; artigos de tecido ou couro}
\end{entry}

\begin{entry}{口袋妖怪}{kou3dai4 yao1guai4}{3,11,7,8}{⼝、⾐、⼥、⼼}
  \definition*{s.}{Pokémon}
\end{entry}

\begin{entry}{口号}{kou3 hao4}{3,5}{⼝、⼝}[HSK 5]
  \definition[个]{s.}{\emph{slogan}; palavra de ordem; lema}
\end{entry}

\begin{entry}{口试}{kou3 shi4}{3,8}{⼝、⾔}[HSK 6]
  \definition{s.}{exame oral (ou teste); um tipo de exame que exige que os candidatos respondam a perguntas oralmente (em oposição a 笔试)}
  \definition{v.}{examinar oralmente}
  \seealsoref{笔试}{bi3 shi4}
\end{entry}

\begin{entry}{口香糖}{kou3xiang1tang2}{3,9,16}{⼝、⾹、⽶}
  \definition{s.}{goma de mascar | chiclete}
\end{entry}

\begin{entry}{口音}{kou3yin1}{3,9}{⼝、⾳}
  \definition{s.}{sons da fala oral (linguística)}
  \seeref{口音}{kou3yin5}
\end{entry}

\begin{entry}{口音}{kou3yin5}{3,9}{⼝、⾳}
  \definition{s.}{sotaque | voz}
  \seeref{口音}{kou3yin1}
\end{entry}

\begin{entry}{口语}{kou3 yu3}{3,9}{⼝、⾔}[HSK 4]
  \definition[门]{s.}{linguagem oral; linguagem falada; linguagem coloquial; linguagem usada em conversas}
\end{entry}

\begin{entry}{扣}{kou4}{6}{⼿}[HSK 6]
  \definition*{s.}{Sobrenome Kou}
  \definition{clas.}{giro; volta; uma volta de uma rosca}
  \definition[个,颗,粒]{s.}{nó | fivela; botão | círculo de rosca (em um parafuso)}
  \definition{v.}{fivela; abotoar; amarrar ou prender com um laço ou anel | colocar uma xícara, tigela etc. de cabeça para baixo; cobrir com uma xícara, tigela etc. invertida; colocar a boca do recipiente para baixo | deter; prender; levar sob custódia | cravar; esmagar (a bola); arremessar ou bater (em uma bola) com força de cima para baixo | atracar; deduzir; descontar; subtrair uma parte do valor original | puxar; pressionar | impor; marcar sem fundamento; acusar injustamente; impor ou atribuir (um crime ou má fama) a alguém}
\end{entry}

\begin{entry}{枯}{ku1}{9}{⽊}
  \definition{adj.}{murcho | (de um poço, rio, etc.) seco | chato; desinteressante | magro e abatido; emaciado}
  \definition[片]{s.}{borra; resíduo}
\end{entry}

\begin{entry}{枯木}{ku1mu4}{9,4}{⽊、⽊}
  \definition{s.}{árvore morta | madeira morta}
\end{entry}

\begin{entry}{哭}{ku1}{10}{⼝}[HSK 2]
  \definition{v.}{chorar; soluçar; lamentar-se; chorar de dor ou emoção}
\end{entry}

\begin{entry}{哭墙}{ku1qiang2}{10,14}{⼝、⼟}
  \definition*{s.}{Muro das Lamentações (Jerusalém)}
\end{entry}

\begin{entry}{苦}{ku3}{8}{⾋}[HSK 4]
  \definition{adj.}{amargo | difícil; doloroso; sofrido | desgastado; gasto demais}
  \definition{adv.}{meticulosamente; fazendo o máximo possível; de forma árdua; pacientemente}
  \definition{v.}{causar sofrimento a alguém; causar dificuldades a alguém | sofrer com; ser incomodado por; sentir-se angustiado com uma situação}
\end{entry}

\begin{entry}{苦瓜}{ku3gua1}{8,5}{⾋、⽠}
  \definition{s.}{melão amargo (cabaça amarga, pêra bálsamo, maçã bálsamo, pepino amargo)}
\end{entry}

\begin{entry}{库}{ku4}{7}{⼴}[HSK 5]
  \definition{s.}{depósito; tesouraria; armazém; almoxarifado; edifícios e equipamentos para armazenamento de mercadorias | banco de dados}
\end{entry}

\begin{entry}{裤}{ku4}{12}{⾐}
  \definition[条]{s.}{calças}
\end{entry}

\begin{entry}{裤子}{ku4zi5}{12,3}{⾐、⼦}[HSK 3]
  \definition[条]{s.}{calças; calções; roupas usadas abaixo da cintura, com cós, virilha e duas pernas}
\end{entry}

\begin{entry}{酷}{ku4}{14}{⾣}[HSK 6]
  \definition{adj.}{cruel; opressivo | feroz; escaldante | brutal | \emph{cool} (empréstimo linguístico); legal; excelente; moderno; ótimo | elegante e sóbrio; gracioso e severo}
  \definition{adv.}{muito; extremamente}
\end{entry}

\begin{entry}{酷斯拉}{ku4si1la1}{14,12,8}{⾣、⽄、⼿}
  \definition*{s.}{Godzilla. do Japonês Gojira, ゴジラ}
  \seealsoref{哥斯拉}{ge1si1la1}
\end{entry}

\begin{entry}{跨}{kua4}{13}{⾜}[HSK 6]
  \definition{adj.}{localizado ao lado de; anexo a}
  \definition{v.}{dar um passo; andar a passos largos | disputar; ficar de pernas abertas | atravessar; ir além (dos limites de uma certa quantidade, tempo, região, etc.)}
\end{entry}

\begin{entry}{会}{kuai4}{6}{⼈}
  \definition[个,场,次]{s.}{contabilidade}
  \definition{v.}{computar; calcular; equilibrar uma conta}
  \seeref{会}{hui4}
\end{entry}

\begin{entry}{会计}{kuai4ji4}{6,4}{⼈、⾔}[HSK 4]
  \definition[个,位,名]{s.}{contabilidade | contador; contabilista; guarda-livros; pessoal que trabalha como contador}
\end{entry}

\begin{entry}{块}{kuai4}{7}{⼟}[HSK 1]
  \definition{clas.}{usado para coisas em pedaços | usado para coisas em pedaços ou em algumas formas de folhas | usado para moedas de prata ou notas de papel equivalentes a 圆}
  \definition{s.}{pedaço; pedaço (de terra); peça; algo que forma um pedaço ou massa}
  \seealsoref{圆}{yuan2}
\end{entry}

\begin{entry}{快}{kuai4}{7}{⼼}[HSK 1]
  \definition*{s.}{Sobrenome Kuai}
  \definition{adj.}{rápido; veloz (oposto a 慢) | apressado | perspicaz; ágil; inteligente; de ​​mente rápida | (de uma faca, espada, etc.) afiado (oposto a 钝) | direto; franco; sem rodeios | satisfeito; feliz; gratificado | rápido; veloz; alta velocidade; tempo de execução curto | satisfeito; feliz; contente | engenhoso; ágil | afiado; facas, tesouras, machados e outros objetos afiados | sincero}
  \definition{adv.}{em breve; antes de muito tempo; estar prestes a | rapidamente}
  \definition{s.}{policial; polícia | (antigo) oficial encarregado de efetuar prisões}
  \seealsoref{钝}{dun4}
  \seealsoref{慢}{man4}
\end{entry}

\begin{entry}{快餐}{kuai4 can1}{7,16}{⼼、⾷}[HSK 2]
  \definition[份,顿]{s.}{pedido (comida) rápido; \emph{fast food}; refere-se a refeições simples preparadas com antecedência e que podem ser servidas rapidamente}
\end{entry}

\begin{entry}{快车}{kuai4 che1}{7,4}{⼼、⾞}[HSK 6]
  \definition{s.}{trem ou ônibus expresso (em oposição a 慢车); um trem ou ônibus com menos paradas e tempos de viagem mais curtos (usado principalmente para transporte de passageiros)}
  \seealsoref{慢车}{man4 che1}
\end{entry}

\begin{entry}{快递}{kuai4 di4}{7,10}{⼼、⾡}[HSK 4]
  \definition[个]{s.}{correio rápido; entrega expressa; entrega rápida}
  \definition{v.}{entregar (serviço de entrega rápida por transportadoras especializadas)}
\end{entry}

\begin{entry}{快点儿}{kuai4 dian3r5}{7,9,2}{⼼、⽕、⼉}[HSK 2]
  \definition{v.}{apressar-se}
\end{entry}

\begin{entry}{快活}{kuai4huo5}{7,9}{⼼、⽔}[HSK 5]
  \definition{adj.}{feliz; alegre; contente; animado}
\end{entry}

\begin{entry}{快乐}{kuai4le4}{7,5}{⼼、⼃}[HSK 2]
  \definition{adj.}{feliz; alegre; animado; prazeiroso}
  \definition{s.}{felicidade | alegria}
\end{entry}

\begin{entry}{快速}{kuai4 su4}{7,10}{⼼、⾡}[HSK 3]
  \definition{adj.}{rápido; veloz; de alta velocidade; descreve o tempo curto gasto para caminhar, fazer algo, etc.}
\end{entry}

\begin{entry}{快要}{kuai4 yao4}{7,9}{⼼、⾑}[HSK 2]
  \definition{adv.}{estar prestes a; estar indo para; estar à beira de; em breve; em pouco tempo; indica que a situação está prestes a ocorrer}
\end{entry}

\begin{entry}{筷}{kuai4}{13}{⽵}
  \definition[双,根,个]{s.}{pauzinhos para comer}
\end{entry}

\begin{entry}{筷子}{kuai4zi5}{13,3}{⽵、⼦}[HSK 2]
  \definition[根,双,副,把,对]{s.}{pauzinhos; \emph{chopsticks}; dois bastôes finos feitos de bambu, madeira, metal ou outro material, usados para segurar comida ou outros objetos}
\end{entry}

\begin{entry}{宽}{kuan1}{10}{⼧}[HSK 4]
  \definition*{s.}{Sobrenome Kuan}
  \definition{adj.}{largo; amplo; grandes distâncias horizontais | leniente; generoso; indulgente | bem de vida; confortável | espaçoso}
  \definition{s.}{largura; amplitude}
  \definition{v.}{relaxar; aliviar}
\end{entry}

\begin{entry}{宽度}{kuan1 du4}{10,9}{⼧、⼴}[HSK 5]
  \definition{s.}{largura; amplitude; duração; o grau de largura e estreiteza; a distância horizontal (no caso de um retângulo, a distância entre os dois lados mais longos)}
\end{entry}

\begin{entry}{宽广}{kuan1 guang3}{10,3}{⼧、⼴}[HSK 4]
  \definition{adj.}{vasto; amplo; espaçoso; extenso}
\end{entry}

\begin{entry}{宽阔}{kuan1 kuo4}{10,12}{⼧、⾨}[HSK 6]
  \definition{adj.}{amplo; largo; espaçoso | tolerante; mente aberta; descreve uma mente alegre e ampla}
\end{entry}

\begin{entry}{宽影片}{kuan1ying3pian4}{10,15,4}{⼧、⼺、⽚}
  \definition{s.}{filme \emph{widescreen}}
\end{entry}

\begin{entry}{款}{kuan3}{12}{⽋}
  \definition{clas.}{para versões ou modelos (de um produto)}
  \definition[笔,个]{s.}{montante de dinheiro | fundos | parágrafo | seção}
\end{entry}

\begin{entry}{窾}{kuan3}{17}{⽳}
  \definition{adj.}{oco}
  \definition{s.}{rachadura; cavidade | (onomatopéia) água batendo na rocha}
  \definition{v.}{escavar um buraco}
  \seeref{窾}{cuan4}
\end{entry}

\begin{entry}{狂}{kuang2}{7}{⽝}[HSK 5]
  \definition*{s.}{Sobrenome Kuang}
  \definition{adj.}{louco; maluco | violento; selvagem | selvagem; delirante; furioso; desenfreado; desinibido; sem restrições | arrogante; autoritário}
\end{entry}

\begin{entry}{狂欢节}{kuang2huan1jie2}{7,6,5}{⽝、⽋、⾋}
  \definition*{s.}{Carnaval}
\end{entry}

\begin{entry}{况}{kuang4}{7}{⼎}
  \definition*{s.}{Sobrenome Kuang}
  \definition{conj.}{além disso | mesmo; muito menos; sem mencionar}
  \definition{s.}{condição; situação}
  \definition{v.}{comparar}
\end{entry}

\begin{entry}{况且}{kuang4qie3}{7,5}{⼎、⼀}
  \definition{conj.}{além disso | além do mais}
\end{entry}

\begin{entry}{旷}{kuang4}{7}{⽇}
  \definition*{s.}{Sobrenome Kuang}
  \definition{adj.}{vasto; espaçoso | livre de preocupações e ideias mesquinhas | folgado}
  \definition{v.}{negligenciar ou desperdiçar | estar ausente de | desperdiçar; abandonar; negligenciar}
\end{entry}

\begin{entry}{旷野}{kuang4ye3}{7,11}{⽇、⾥}
  \definition{s.}{região selvagem}
\end{entry}

\begin{entry}{矿}{kuang4}{8}{⽯}[HSK 6]
  \definition[个,座]{s.}{depósito de minério | minério | mina}
\end{entry}

\begin{entry}{矿泉水}{kuang4quan2shui3}{8,9,4}{⽯、⽔、⽔}[HSK 4]
  \definition[瓶,杯]{s.}{água mineral de nascente}
\end{entry}

\begin{entry}{亏}{kui1}{3}{⼆}[HSK 5]
  \definition{adv.}{felizmente; por sorte; graças a | contrariamente, expressando sarcasmo}
  \definition{s.}{prejuízo}
  \definition{v.}{perder dinheiro, etc.; ter um déficit; ter prejuízo | ter falta de; ser deficiente; carecer de | tratar injustamente; causar prejuízo; trair a confiança}
\end{entry}

\begin{entry}{葵}{kui2}{12}{⾋}
  \definition*{s.}{Sobrenome Kui}
  \definition[朵]{s.}{certas ervas de flores grandes}
\end{entry}

\begin{entry}{葵花}{kui2hua1}{12,7}{⾋、⾋}
  \definition{s.}{girassol (flor)}
\end{entry}

\begin{entry}{困}{kun4}{7}{⼞}[HSK 3]
  \definition{adj.}{cansado; exausto; fatigado | difícil; complicado; difícil e penoso; pobre e miserável | sonolento; com sono; cansado, com vontade de dormir}
  \definition{v.}{ficar encalhado; estar em apuros; preso em dificuldades e sofrimentos ou limitado por circunstâncias e condições que não pode escapar | cercar; envolver; imobilizar; controlar dentro de um determinado limite | dormir}
\end{entry}

\begin{entry}{困难}{kun4nan5}{7,10}{⼞、⾫}[HSK 3]
  \definition{adj.}{dificuldades financeiras; circunstâncias difíceis | complicado; complexo; difícil; árduo; a situação é complexa e há muitos obstáculos}
  \definition[种]{s.}{dificuldade; situação difícil; problemas ou situações difíceis de resolver no trabalho e na vida}
\end{entry}

\begin{entry}{困扰}{kun4 rao3}{7,7}{⼞、⼿}[HSK 5]
  \definition{v.}{perturbar; deixar perplexo; perseguir}
\end{entry}

\begin{entry}{扩}{kuo4}{6}{⼿}
  \definition{v.}{expandir; ampliar; estender; alargar}
\end{entry}

\begin{entry}{扩大}{kuo4da4}{6,3}{⼿、⼤}[HSK 4]
  \definition{v.}{ampliar; expandir; estender; alargar}
\end{entry}

\begin{entry}{扩展}{kuo4 zhan3}{6,10}{⼿、⼫}[HSK 4]
  \definition{v.}{esticar; expandir; estender; espalhar}
\end{entry}

\begin{entry}{括}{kuo4}{9}{⼿}
  \definition{v.}{unir (músculos, etc.); contrair | incluir | adicionar colchetes a | amarrar; empacotar}
\end{entry}

\begin{entry}{括号}{kuo4 hao4}{9,5}{⼿、⼝}[HSK 4]
  \definition{s.}{chaves, colchetes e parênteses (em fórmulas aritméticas ou algébricas, os símbolos que indicam a combinação e a ordem de vários números ou termos) | colchetes e parênteses usados como um tipo de sinal de pontuação para mostrar a parte explicativa de uma passagem em um texto}
\end{entry}

\begin{entry}{阔}{kuo4}{12}{⾨}[HSK 6]
  \definition{adj.}{amplo; amplo; vasto | rico | longo, no sentido de ``há muito tempo'' | vazio; impraticável}
\end{entry}

%%%%% EOF %%%%%

