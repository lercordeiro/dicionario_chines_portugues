%%%
%%% Y
%%%

\section*{Y}\addcontentsline{toc}{section}{Y}

\begin{entry}{压}{ya1}{6}{⼚}[HSK 3]
  \definition{v.}{pressionar; empurrar para baixo; segurar; pesar | manter sob controle; controlar; manter sob; reprimir |exercer pressão sobre; suprimir; desencorajar; intimidar | aproximar-se; estar chegando perto}
  \seeref{压}{ya4}
\end{entry}

\begin{entry}{压力}{ya1li4}{6,2}{⼚、⼒}[HSK 3]
  \definition[份,个]{s.}{pressão; força atuando perpendicularmente à superfície de um objeto | pressão; força esmagadora; metáfora para a força que coage e intimida as pessoas (principalmente nos aspectos espirituais e psicológicos) | tensão; fardo; fardos econômicos, psicológicos e espirituais que o mundo exterior traz às pessoas}
\end{entry}

\begin{entry}{压岁钱}{ya1sui4qian2}{6,6,10}{⼚、⼭、⾦}
  \definition{s.}{dinheiro da sorte | dinheiro dado às crianças como presente no Ano Novo Chinês}
\end{entry}

\begin{entry}{压碎}{ya1sui4}{6,13}{⼚、⽯}
  \definition{v.}{esmagar em pedaços}
\end{entry}

\begin{entry}{压韵}{ya1yun4}{6,13}{⼚、⾳}
  \variantof{押韵}
\end{entry}

\begin{entry}{押}{ya1}{8}{⼿}
  \definition{v.}{deter sob custódia | escoltar e proteger | hipotecar | penhorar}
\end{entry}

\begin{entry}{押后}{ya1hou4}{8,6}{⼿、⼝}
  \definition{v.}{encerrar | adiar}
\end{entry}

\begin{entry}{押金}{ya1jin1}{8,8}{⼿、⾦}[HSK 5]
  \definition[笔,份,些]{s.}{caução; sinal; depósito; dinheiro como garantia}
\end{entry}

\begin{entry}{押送}{ya1song4}{8,9}{⼿、⾡}
  \definition{v.}{enviar sob escolta | transportar um detido}
\end{entry}

\begin{entry}{押运}{ya1yun4}{8,7}{⼿、⾡}
  \definition{v.}{escoltar sob guarda | escoltar (bens ou fundos)}
\end{entry}

\begin{entry}{押韵}{ya1yun4}{8,13}{⼿、⾳}
  \definition{v.}{rimar}
\end{entry}

\begin{entry}{押注}{ya1zhu4}{8,8}{⼿、⽔}
  \definition{v.}{apostar}
\end{entry}

\begin{entry}{押租}{ya1zu1}{8,10}{⼿、⽲}
  \definition{s.}{depósito de aluguel}
\end{entry}

\begin{entry}{鸭}{ya1}{10}{⿃}
  \definition[只]{s.}{pato | (gíria) prostituto}
\end{entry}

\begin{entry}{鸭子}{ya1 zi5}{10,3}{⿃、⼦}[HSK 5]
  \definition[只]{s.}{pato | (gíria) prostituto}
\end{entry}

\begin{entry}{牙}{ya2}{4}{⽛}[HSK 4][Kangxi 92]
  \definition*{s.}{sobrenome Ya}
  \definition[颗]{s.}{dente | marfim | algo semelhante a um dente}
\end{entry}

\begin{entry}{牙齿}{ya2chi3}{4,8}{⽛、⿒}
  \definition{adv.}{dental}
  \definition[颗]{s.}{dente}
\end{entry}

\begin{entry}{牙膏}{ya2gao1}{4,14}{⽛、⾁}
  \definition[管]{s.}{pasta de dente}
\end{entry}

\begin{entry}{牙行}{ya2hang2}{4,6}{⽛、⾏}
  \definition{s.}{corretor | \emph{broker}}
\end{entry}

\begin{entry}{牙刷}{ya2 shua1}{4,8}{⽛、⼑}[HSK 4]
  \definition[把]{s.}{escova de dentes}
\end{entry}

\begin{entry}{牙线}{ya2xian4}{4,8}{⽛、⽷}
  \definition[条]{s.}{fio dental}
\end{entry}

\begin{entry}{牙医}{ya2yi1}{4,7}{⽛、⼖}
  \definition{s.}{dentista}
\end{entry}

\begin{entry}{崖}{ya2}{11}{⼭}
  \definition{s.}{precipício | penhasco}
\end{entry}

\begin{entry}{亚军}{ya4jun1}{6,6}{⼆、⼍}[HSK 5]
  \definition{s.}{segundo lugar; vice-campeão; medalhista de prata}
\end{entry}

\begin{entry}{亚细亚洲}{ya4xi4ya4zhou1}{6,8,6,9}{⼆、⽷、⼆、⽔}
  \definition*{s.}{Ásia}
\end{entry}

\begin{entry}{亚运会}{ya4 yun4 hui4}{6,7,6}{⼆、⾡、⼈}[HSK 4]
  \definition*{s.}{Jogos Asiáticos}
\end{entry}

\begin{entry}{亚洲}{ya4zhou1}{6,9}{⼆、⽔}
  \definition*{s.}{Ásia, abreviação de~亚细亚洲}
  \seealsoref{亚细亚洲}{ya4xi4ya4zhou1}
\end{entry}

\begin{entry}{亚洲人}{ya4zhou1ren2}{6,9,2}{⼆、⽔、⼈}
  \definition{s.}{asiático | pessoa ou povo da Ásia}
\end{entry}

\begin{entry}{压}{ya4}{6}{⼚}
  \definition{adv.}{fundamentalmente; nunca (usado principalmente em frases negativas)}
  \seeref{压}{ya1}
  \seealsoref{压根儿}{ya4 gen1r5}
\end{entry}

\begin{entry}{压根儿}{ya4 gen1r5}{6,10,2}{⼚、⽊、⼉}
  \definition{adv.}{fundamentalmente; nunca (usado principalmente em frases negativas)}
\end{entry}

\begin{entry}{呀}{ya5}{7}{⼝}[HSK 4]
  \definition{part.}{usado no lugar de 啊 quando a palavra anterior termina com o som a, e, i, o ou ü}
  \seealsoref{啊}{a5}
\end{entry}

\begin{entry}{烟}{yan1}{10}{⽕}[HSK 3]
  \definition[根]{s.}{fumaça; gases produzidos quando substâncias queimam | névoa; vapor; algo como fumaça | tabaco; planta de tabaco | fumo; cigarro; termo geral para cigarros, charutos, etc. | ópio}
  \definition{v.}{ficar irritado com a fumaça (olhos)}
\end{entry}

\begin{entry}{烟草}{yan1cao3}{10,9}{⽕、⾋}
  \definition{s.}{tabaco}
\end{entry}

\begin{entry}{烟囱}{yan1cong1}{10,7}{⽕、⼞}
  \definition{s.}{chaminé}
\end{entry}

\begin{entry}{烟花}{yan1hua1}{10,7}{⽕、⾋}
  \definition{s.}{fogos de artifício}
\end{entry}

\begin{entry}{烟火}{yan1huo3}{10,4}{⽕、⽕}
  \definition{s.}{fogo de artifício}
\end{entry}

\begin{entry}{烟头}{yan1tou2}{10,5}{⽕、⼤}
  \definition[根]{s.}{bituca de cigarro}
\end{entry}

\begin{entry}{烟叶}{yan1ye4}{10,5}{⽕、⼝}
  \definition{s.}{folha de tabaco}
\end{entry}

\begin{entry}{烟雨}{yan1yu3}{10,8}{⽕、⾬}
  \definition{s.}{chuvisco | garoa}
\end{entry}

\begin{entry}{延长}{yan2chang2}{6,4}{⼵、⾧}[HSK 4]
  \definition{v.}{estender; prolongar; alongar; aumentar o tempo, a distância ou a duração de algo específico}
\end{entry}

\begin{entry}{延期}{yan2qi1}{6,12}{⼵、⽉}[HSK 4]
  \definition{v.+compl.}{atrasar; adiar; postergar}
\end{entry}

\begin{entry}{延伸}{yan2shen1}{6,7}{⼵、⼈}[HSK 5]
  \definition{v.}{estender; esticar; alongar; estender-se}
\end{entry}

\begin{entry}{延续}{yan2xu4}{6,11}{⼵、⽷}[HSK 4]
  \definition{v.}{durar; continuar; prosseguir; continuar como antes; prolongar}
\end{entry}

\begin{entry}{严}{yan2}{7}{⼀}[HSK 4]
  \definition*{s.}{sobrenome Yan}
  \definition{adj.}{rígido; rigoroso; estrito; severo}
  \definition{s.}{pai; refere-se ao pai}
\end{entry}

\begin{entry}{严格}{yan2ge2}{7,10}{⼀、⽊}[HSK 4]
  \definition{adj.}{rígido; estrito; rigoroso; muito consciente e meticuloso na implementação de sistemas e no domínio de padrões}
  \definition{v.}{tornar (sistemas, provisões, etc.) rigorosos;}
\end{entry}

\begin{entry}{严厉}{yan2li4}{7,5}{⼀、⼚}[HSK 5]
  \definition{adj.}{severo; rigoroso}
\end{entry}

\begin{entry}{严肃}{yan2su4}{7,8}{⼀、⾀}[HSK 5]
  \definition{adj.}{sério; solene; sincero; (expressão, atmosfera, etc.) faz as pessoas se sentirem admiradas e desconfortáveis | sóbrio; grave; sério; sincero}
  \definition{v.}{aplicar rigorosamente; fazer algo sério}
\end{entry}

\begin{entry}{严重}{yan2zhong4}{7,9}{⼀、⾥}[HSK 4]
  \definition{adj.}{sério; grave; crítico; severo}
\end{entry}

\begin{entry}{严重打伤}{yan2zhong4 da3 shang1}{7,9,5,6}{⼀、⾥、⼿、⼈}
  \definition{s.}{gravemente ferido}
\end{entry}

\begin{entry}{严重地}{yan2zhong4 di4}{7,9,6}{⼀、⾥、⼟}
  \definition{adv.}{seriamente | gravemente}
\end{entry}

\begin{entry}{严重关切}{yan2zhong4guan1qie4}{7,9,6,4}{⼀、⾥、⼋、⼑}
  \definition{s.}{preocupação séria}
\end{entry}

\begin{entry}{严重后果}{yan2zhong4hou4guo3}{7,9,6,8}{⼀、⾥、⼝、⽊}
  \definition{s.}{consequências sérias | repercursões graves}
\end{entry}

\begin{entry}{严重破坏}{yan2zhong4 po4huai4}{7,9,10,7}{⼀、⾥、⽯、⼟}
  \definition{s.}{destruição grave}
\end{entry}

\begin{entry}{严重伤害}{yan2zhong4 shang1hai4}{7,9,6,10}{⼀、⾥、⼈、⼧}
  \definition{s.}{ferimento grave}
\end{entry}

\begin{entry}{严重危害}{yan2zhong4wei1hai4}{7,9,6,10}{⼀、⾥、⼙、⼧}
  \definition{s.}{danos graves}
\end{entry}

\begin{entry}{严重问题}{yan2zhong4wen4ti2}{7,9,6,15}{⼀、⾥、⾨、⾴}
  \definition{s.}{problema sério}
\end{entry}

\begin{entry}{严重性}{yan2zhong4xing4}{7,9,8}{⼀、⾥、⼼}
  \definition{s.}{seriedade | gravidade}
\end{entry}

\begin{entry}{言论}{yan2lun4}{7,6}{⾔、⾔}
  \definition{s.}{expressão de opinião |  visualizações | comentários | argumentos}
\end{entry}

\begin{entry}{言语}{yan2 yu3}{7,9}{⾔、⾔}[HSK 5]
  \definition{s.}{verbal; fala; linguagem falada; conversa; palavras}
\end{entry}

\begin{entry}{炎热}{yan2re4}{8,10}{⽕、⽕}
  \definition{adj.}{extremamente quente | escaldante (clima)}
\end{entry}

\begin{entry}{研究}{yan2jiu1}{9,7}{⽯、⽳}[HSK 4]
  \definition{v.}{estudar; pesquisar | discutir; considerar}
\end{entry}

\begin{entry}{研究生}{yan2 jiu1 sheng1}{9,7,5}{⽯、⽳、⽣}[HSK 4]
  \definition[位,名]{s.}{pós-graduado; estudante de pós-graduação}
\end{entry}

\begin{entry}{研究所}{yan2 jiu1 suo3}{9,7,8}{⽯、⽳、⼾}[HSK 5]
  \definition[个]{s.}{instituto de pesquisa; instituição de pesquisa científica envolvida em pesquisas em um determinado campo}
\end{entry}

\begin{entry}{研制}{yan2 zhi4}{9,8}{⽯、⼑}[HSK 4]
  \definition{v.}{desenvolver; fabricar; produzir | triturar; (medicina chinesa) moer}
\end{entry}

\begin{entry}{盐}{yan2}{10}{⽫}[HSK 4]
  \definition*{s.}{sobrenome Yan}
  \definition{s.}{sal; sais}
\end{entry}

\begin{entry}{颜}{yan2}{15}{⾴}
  \definition*{s.}{sobrenome Yan}
  \definition{s.}{cor | face | semblante}
\end{entry}

\begin{entry}{颜色}{yan2 se4}{15,6}{⾴、⾊}[HSK 2]
  \definition{s.}{cor | pigmento | tintura}
\end{entry}

\begin{entry}{广}{yan3}{3}{⼴}
  \definition[家]{s.}{casa ou edifício construído contra ou ao longo da encosta de uma montanha ou penhasco}
  \seeref{广}{an1}
  \seeref{广}{guang3}
\end{entry}

\begin{entry}{眼}{yan3}{11}{⽬}[HSK 2]
  \definition{clas.}{para grandes coisas ocas: poços, fogões, panelas, etc.}
  \definition[只,双]{s.}{ponto crucial (de um assunto) | olho | pequeno buraco}
\end{entry}

\begin{entry}{眼柄}{yan3bing3}{11,9}{⽬、⽊}
  \definition{s.}{pedúnculo ocular (de crustáceo, etc.)}
\end{entry}

\begin{entry}{眼袋}{yan3dai4}{11,11}{⽬、⾐}
  \definition{s.}{inchaço sob os olhos}
\end{entry}

\begin{entry}{眼光}{yan3guang1}{11,6}{⽬、⼉}[HSK 5]
  \definition{s.}{olho; visão | visão; percepção; previsão; capacidade de observar e identificar coisas | vista; ponto de vista}
\end{entry}

\begin{entry}{眼花缭乱}{yan3hua1liao2luan4}{11,7,15,7}{⽬、⾋、⽷、⼄}
  \definition{v.}{ficar deslumbrado | deslumbrar}
\end{entry}

\begin{entry}{眼镜}{yan3jing4}{11,16}{⽬、⾦}[HSK 4]
  \definition[副]{s.}{óculos; óculos de grau}
\end{entry}

\begin{entry}{眼睛}{yan3jing5}{11,13}{⽬、⽬}[HSK 2]
  \definition[只,双]{s.}{olho(s)}
\end{entry}

\begin{entry}{眼泪}{yan3 lei4}{11,8}{⽬、⽔}[HSK 4]
  \definition[滴,行]{s.}{lágrimas; termo genérico para lágrimas; fluido incolor e transparente secretado pelas glândulas lacrimais no olho, que serve para proteger o olho}
\end{entry}

\begin{entry}{眼里}{yan3 li3}{11,7}{⽬、⾥}[HSK 4]
  \definition{s.}{nos olhos de uma pessoa; dentro de sua visão}
\end{entry}

\begin{entry}{眼前}{yan3 qian2}{11,9}{⽬、⼑}[HSK 3]
  \definition{adv.}{agora; (no) momento}
  \definition{s.}{diante dos olhos; diante de; momento}
\end{entry}

\begin{entry}{眼证}{yan3zheng4}{11,7}{⽬、⾔}
  \definition{s.}{testemunha ocular}
\end{entry}

\begin{entry}{演}{yan3}{14}{⽔}[HSK 3]
  \definition{v.}{desenvolver; evoluir | deduzir; elaborar | exercitar; praticar | representar; atuar}
\end{entry}

\begin{entry}{演唱}{yan3 chang4}{14,11}{⽔、⼝}[HSK 3]
  \definition{v.}{cantar em uma performance; apresentar canções, óperas, dramas, etc.}
\end{entry}

\begin{entry}{演唱会}{yan3 chang4 hui4}{14,11,6}{⽔、⼝、⼈}[HSK 3]
  \definition[个,场]{s.}{concerto; recital vocal; concerto vocal}
\end{entry}

\begin{entry}{演出}{yan3chu1}{14,5}{⽔、⼐}[HSK 3]
  \definition[场,次]{s.}{show; concerto; performance}
  \definition{v.}{apresentar; representar; fazer um show; apresentar peças de teatro, dança, arte popular, acrobacias, etc. para o público aproveitar}
\end{entry}

\begin{entry}{演讲}{yan3jiang3}{14,6}{⽔、⾔}[HSK 4]
  \definition[场,次]{s.}{palestra; discurso; ato ou a atividade de apresentar ou expressar ideias, opiniões ou informações oralmente em público ou diante de um público}
  \definition{v.}{dar uma palestra; fazer um discurso; informar o público sobre uma determinada área de conhecimento ou opinião sobre um determinado assunto}
\end{entry}

\begin{entry}{演员}{yan3yuan2}{14,7}{⽔、⼝}[HSK 3]
  \definition[个,位,名]{s.}{ator; performer; participantes de teatro, cinema, dança, arte popular, acrobacia e outras apresentações}
\end{entry}

\begin{entry}{扬雄}{yang2xiong2}{6,12}{⼿、⾫}
  \definition*{s.}{Yang Xiong (53 AC-18 DC), estudioso, poeta e lexicógrafo, autor do primeiro dicionário de dialeto chinês 方言}
  \seealsoref{方言}{fang1yan2}
\end{entry}

\begin{entry}{羊}{yang2}{6}{⽺}[HSK 3][Kangxi 123]
  \definition*{s.}{sobrenome Yang}
  \definition{s.}{carneiro; ovelha; bode; cabra}
\end{entry}

\begin{entry}{阳}{yang2}{6}{⾩}
  \definition*{s.}{Yang (o princípio positivo de Yin e Yang)}
  \definition{s.}{positivo (eletricidade) | sol}
  \seealsoref{阴}{yin1}
  \seealsoref{阴阳}{yin1yang2}
\end{entry}

\begin{entry}{阳光}{yang2guang1}{6,6}{⾩、⼉}[HSK 3]
  \definition{adj.}{alegre; otimista; personalidade positiva e alegre; cheia de vitalidade juvenil | aberto; transparente; público; conduzido sob supervisão pública}
  \definition[缕,束,道]{s.}{luz do sol; brilho do sol; raio de sol}
\end{entry}

\begin{entry}{阳台}{yang2tai2}{6,5}{⾩、⼝}[HSK 4]
  \definition{s.}{varanda; terraço; sacada; pequeno terraço do edifício com grades para se refrescar, tomar sol ou olhar o horizonte}
\end{entry}

\begin{entry}{洋葱}{yang2cong1}{9,12}{⽔、⾋}
  \definition{s.}{cebola}
\end{entry}

\begin{entry}{养}{yang3}{9}{⼋}[HSK 2]
  \definition{v.}{criar (animais ou filhos), plantar (flores), etc. | dar a luz}
\end{entry}

\begin{entry}{养成}{yang3cheng2}{9,6}{⼋、⼽}[HSK 4]
  \definition{v.}{cultivar; desenvolver; cultivar para formar; nutrir para crescer}
\end{entry}

\begin{entry}{养分}{yang3fen4}{9,4}{⼋、⼑}
  \definition{s.}{nutriente}
\end{entry}

\begin{entry}{养料}{yang3liao4}{9,10}{⼋、⽃}
  \definition{s.}{nutrição}
\end{entry}

\begin{entry}{氧}{yang3}{10}{⽓}
  \definition{s.}{oxigênio}
\end{entry}

\begin{entry}{样}{yang4}{10}{⽊}
  \definition{s.}{aparência | forma | modelo}
\end{entry}

\begin{entry}{样品}{yang4pin3}{10,9}{⽊、⼝}
  \definition{s.}{amostra | espécime}
\end{entry}

\begin{entry}{样儿}{yang4r5}{10,2}{⽊、⼉}
  \definition{s.}{aparência | forma | modelo}
  \seealsoref{样子}{yang4zi5}
\end{entry}

\begin{entry}{样样}{yang4yang4}{10,10}{⽊、⽊}
  \definition{adv.}{todos os tipos}
\end{entry}

\begin{entry}{样章}{yang4zhang1}{10,11}{⽊、⾳}
  \definition{s.}{capítulo de amostra}
\end{entry}

\begin{entry}{样子}{yang4zi5}{10,3}{⽊、⼦}[HSK 2]
  \definition{s.}{aparência | forma | modelo}
  \seealsoref{样儿}{yang4r5}
\end{entry}

\begin{entry}{约}{yao1}{6}{⽷}
  \definition{adj.}{econômico; frugal | simples; breve | indistinto}
  \definition{adv.}{cerca de; ao redor; aproximadamente}
  \definition{s.}{pacto; acordo; nomeação; coisa prometida}
  \definition{v.}{marcar uma consulta; organizar | perguntar ou convidar com antecedência | restringir; conter | reduzir (fração aproximada)}
  \seeref{约}{yue1}
\end{entry}

\begin{entry}{妖}{yao1}{7}{⼥}
  \definition{adj.}{enfeitiçante | encantador}
  \definition{s.}{\emph{goblin} | bruxa | diabo | monstro | fantasma | demônio}
\end{entry}

\begin{entry}{祅}{yao1}{8}{⽰}
  \definition{s.}{espírito maligno | \emph{goblin} | bruxaria}
  \variantof{妖}
\end{entry}

\begin{entry}{要}{yao1}{9}{⾑}[HSK 1]
  \definition*{s.}{sobrenome Yao}
  \definition{v.}{exigir; pedir; requerer; solicitar; buscar; insistir com base em algo em que se apoia | forçar; coagir; ameaçar}
  \seeref{要}{yao4}
\end{entry}

\begin{entry}{要求}{yao1qiu2}{9,7}{⾑、⽔}[HSK 2]
  \definition[点]{s.}{requerimento}
  \definition{v.}{pedir | exigir | solicitar | fazer uma reivindicação}
\end{entry}

\begin{entry}{要挟}{yao1xie2}{9,9}{⾑、⼿}
  \definition{v.}{chantagear | ameaçar}
\end{entry}

\begin{entry}{腰}{yao1}{13}{⾁}[HSK 4]
  \definition*{s.}{sobrenome Yao}
  \definition[个]{s.}{cintura; região lombar | cós | bolso | parte do meio das coisas | lombo}
\end{entry}

\begin{entry}{腰包}{yao1bao1}{13,5}{⾁、⼓}
  \definition{s.}{pochete | bolso}
\end{entry}

\begin{entry}{腰椎}{yao1zhui1}{13,12}{⾁、⽊}
  \definition{s.}{vértebra lombar (espinha dorsal inferior)}
\end{entry}

\begin{entry}{邀请}{yao1qing3}{16,10}{⾡、⾔}[HSK 5]
  \definition[份,个]{s.}{convite}
  \definition{v.}{convidar; solicitar; convidar pessoas para irem à sua casa ou a um local combinado}
\end{entry}

\begin{entry}{摇}{yao2}{13}{⼿}[HSK 4]
  \definition{v.}{chacoalhar; ondular; balançar; fazer com que um objeto se mova para frente e para trás | agitar algo | sacudir; chacoalhar; agitar algo para que se mova}
\end{entry}

\begin{entry}{摇晃}{yao2huang4}{13,10}{⼿、⽇}
  \definition{v.}{sacudir | agitar | balançar | chacoalhar}
\end{entry}

\begin{entry}{摇头}{yao2tou2}{13,5}{⼿、⼤}[HSK 5]
  \definition{v.+compl.}{sacudir; balançar a cabeça; balançar a cabeça para a esquerda e para a direita, indicando negação, desacordo ou impedimento}
\end{entry}

\begin{entry}{遥控}{yao2kong4}{13,11}{⾡、⼿}
  \definition{s.}{controle remoto}
  \definition{v.}{dirigir operações de um local remoto | controlar remotamente}
\end{entry}

\begin{entry}{咬}{yao3}{9}{⼝}[HSK 5]
  \definition{v.}{morder; estalar; pressionar os dentes superiores e inferiores com força | latir | agarrar; morder | incriminar outra pessoa (geralmente inocente) quando culpada ou interrogada | pronunciar; articular; pronunciar corretamente | corroer (metais); irritar (a pele) | ser minucioso (com relação ao uso de palavras) | aproximar-se de; pressionar em direção a; avançar sobre}
\end{entry}

\begin{entry}{药}{yao4}{9}{⾋}[HSK 2]
  \definition[种,服,味]{s.}{medicamento | remédio | droga}
\end{entry}

\begin{entry}{药补}{yao4bu3}{9,7}{⾋、⾐}
  \definition{s.}{suplemento dietético medicinal que ajuda a melhorar a saúde}
\end{entry}

\begin{entry}{药典}{yao4dian3}{9,8}{⾋、⼋}
  \definition{s.}{farmacopéia}
\end{entry}

\begin{entry}{药店}{yao4 dian4}{9,8}{⾋、⼴}[HSK 2]
  \definition{s.}{farmácia | drogaria | loja de produtos químicos}
\end{entry}

\begin{entry}{药房}{yao4fang2}{9,8}{⾋、⼾}
  \definition{s.}{farmácia | drogaria}
\end{entry}

\begin{entry}{药罐}{yao4guan4}{9,23}{⾋、⽸}
  \definition{s.}{frasco de remédio}
\end{entry}

\begin{entry}{药片}{yao4 pian4}{9,4}{⾋、⽚}[HSK 2]
  \definition[片]{s.}{uma pílula ou comprimido (remédio)}
\end{entry}

\begin{entry}{药品}{yao4pin3}{9,9}{⾋、⼝}
  \definition{s.}{medicamento | remédio | droga}
\end{entry}

\begin{entry}{药签}{yao4qian1}{9,13}{⾋、⽵}
  \definition{s.}{cotonete médico}
\end{entry}

\begin{entry}{药膳}{yao4shan4}{9,16}{⾋、⾁}
  \definition{s.}{dieta medicinal}
\end{entry}

\begin{entry}{药水}{yao4 shui3}{9,4}{⾋、⽔}[HSK 2]
  \definition{s.}{remédio engarrafado | loção | medicamento em forma líquida}
\end{entry}

\begin{entry}{药丸}{yao4wan2}{9,3}{⾋、⼂}
  \definition[粒]{s.}{pílula}
\end{entry}

\begin{entry}{药物}{yao4 wu4}{9,8}{⾋、⽜}[HSK 4]
  \definition{s.}{droga; medicamento; remédio; substâncias que controlam doenças, pragas, etc.}
\end{entry}

\begin{entry}{要}{yao4}{9}{⾑}[HSK 4]
  \definition{adj.}{importante; essencial}
  \definition{conj.}{suponha; no caso; se, indicando um relacionamento hipotético | ou; ou\dots ou\dots}
  \definition{s.}{ponto principal; manchete; conteúdo importante}
  \definition{v.}{querer; desejar; pensar | querer; pedir; deseja; querer obter; querer manter | recuperar algo; dizer a alguém para guardar algo para você ou devolver | pedir (ou querer) que alguém faça algo; pedir a alguém para fazer algo, quando usado para conseguir que alguém faça algo, tem um tom de comando e pode ser indelicado | precisar; tomar; pegar | deve; deveria; é necessário (imperativo, essencial) que\dots | estar indo para | querer; ter um desejo por; expressar determinação ou desejo de fazer algo | poder; dever;  indica uma estimativa, usada para comparação}
  \seeref{要}{yao1}
  \seealsoref{要是}{yao4shi5}
\end{entry}

\begin{entry}{要不}{yao4bu4}{9,4}{⾑、⼀}
  \definition{conj.}{de outra forma | se não | outro | ou}
\end{entry}

\begin{entry}{要不然}{yao4bu4ran2}{9,4,12}{⾑、⼀、⽕}
  \definition{conj.}{de outra forma | se não | outro | ou}
\end{entry}

\begin{entry}{要点}{yao4dian3}{9,9}{⾑、⽕}
  \definition{s.}{pontos principais | essencial}
\end{entry}

\begin{entry}{要好}{yao4hao3}{9,6}{⾑、⼥}
  \definition{v.}{ser amigos íntimos | estar em boas condições}
\end{entry}

\begin{entry}{要谎}{yao4huang3}{9,11}{⾑、⾔}
  \definition{v.}{pedir um preço enorme (como primeiro passo de negociação)}
\end{entry}

\begin{entry}{要么……要么……}{yao4me5 yao4me5}{9,3,9,3}{⾑、⼃、⾑、⼃}
  \definition{conj.}{ou\dots ou\dots}
\end{entry}

\begin{entry}{要强}{yao4qiang2}{9,12}{⾑、⼸}
  \definition{adj.}{ansioso para se destacar | ansioso para progredir na vida | obstinado}
\end{entry}

\begin{entry}{要是}{yao4shi5}{9,9}{⾑、⽇}[HSK 3]
  \definition{conj.}{se; suponha; no caso de; conecta frases, expressa uma relação hipotética, equivalente a 如果, e pode ser usado em conjunto com 的话}
  \seealsoref{的话}{de5 hua4}
  \seealsoref{如果}{ru2guo3}
\end{entry}

\begin{entry}{要是……的话}{yao4shi5 de5hua4}{9,9,8,8}{⾑、⽇、⽩、⾔}[HSK 2,3]
  \definition{conj.}{se\dots no caso de}
\end{entry}

\begin{entry}{要死}{yao4si3}{9,6}{⾑、⽍}
  \definition{adv.}{extremamente | muito}
\end{entry}

\begin{entry}{要义}{yao4yi4}{9,3}{⾑、⼂}
  \definition{s.}{resumo | o essencial}
\end{entry}

\begin{entry}{钥匙}{yao4shi5}{9,11}{⾦、⼔}
  \definition[把]{s.}{chave}
\end{entry}

\begin{entry}{钥匙洞孔}{yao4shi5dong4kong3}{9,11,9,4}{⾦、⼔、⽔、⼦}
  \definition{s.}{buraco da fechadura}
\end{entry}

\begin{entry}{钥匙卡}{yao4shi5ka3}{9,11,5}{⾦、⼔、⼘}
  \definition{s.}{cartão de acesso}
\end{entry}

\begin{entry}{钥匙孔}{yao4shi5kong3}{9,11,4}{⾦、⼔、⼦}
  \definition{s.}{buraco da fechadura}
\end{entry}

\begin{entry}{钥匙圈}{yao4shi5quan1}{9,11,11}{⾦、⼔、⼞}
  \definition{s.}{chaveiro}
\end{entry}

\begin{entry}{椰汁}{ye1zhi1}{12,5}{⽊、⽔}
  \definition{s.}{água de coco}
\end{entry}

\begin{entry}{噎}{ye1}{15}{⼝}
  \definition{v.}{engasgar | sufocar}
\end{entry}

\begin{entry}{爷爷}{ye2ye5}{6,6}{⽗、⽗}[HSK 1]
  \definition[个,位]{s.}{avô (paterno)}
\end{entry}

\begin{entry}{也}{ye3}{3}{⼄}[HSK 1]
  \definition*{s.}{sobrenome Ye}
  \definition{adv.}{também; igualmente; assim como; da mesma forma; usado em frases simples, implica que é igual a outra coisa | assim como (expressar ênfase) | (expressar que as consequências são as mesmas) | também (expressar ufemismo; expressar um tom diplomático) | usado em frases compostas paralelas, indica que duas ou mais coisas têm algo em comum (pode ser usado em todas as frases ou apenas na última frase)}
  \definition{part.}{usado no meio de uma frase, destacando um elemento da frase sobre o qual deve ser feita uma afirmação | usado no final de uma frase, indicando uma explicação ou um julgamento; usado no final da frase, indica tom afirmativo e também pode reforçar o tom interrogativo, exclamativo ou imperativo}
\end{entry}

\begin{entry}{也好}{ye3 hao3}{3,6}{⼄、⼥}[HSK 5]
  \definition{part.}{pode não ser uma má ideia; também pode ser | (reduplicado) se\dots ou\dots; não importa se | pode não ser uma má ideia | se\dots ou\dots; usado em conjunto, significa que não está condicionado a uma determinada situação}
\end{entry}

\begin{entry}{也就是}{ye3jiu4shi4}{3,12,9}{⼄、⼪、⽇}
  \definition{adv.}{i.e., isso é | ou seja}
\end{entry}

\begin{entry}{也就是说}{ye3jiu4shi4shuo1}{3,12,9,9}{⼄、⼪、⽇、⾔}
  \definition{adv.}{em outras palavras | então | isto é | por isso}
\end{entry}

\begin{entry}{也许}{ye3xu3}{3,6}{⼄、⾔}[HSK 2]
  \definition{adv.}{possivelmente | talvez}
\end{entry}

\begin{entry}{也有今天}{ye3you3jin1tian1}{3,6,4,4}{⼄、⽉、⼈、⼤}
  \definition{expr.}{obter apenas o que merece | todo cachorro tem seu dia | obter a sua parte (coisas boas ou ruins) | servir alguém bem}
\end{entry}

\begin{entry}{野}{ye3}{11}{⾥}
  \definition{adj.}{selvagem | rude}
  \definition{s.}{campo | espaço aberto | limite}
\end{entry}

\begin{entry}{野生}{ye3sheng1}{11,5}{⾥、⽣}
  \definition{adj.}{selvagem | não domesticado}
\end{entry}

\begin{entry}{业务}{ye4wu4}{5,5}{⼀、⼒}[HSK 5]
  \definition[项,笔,个,类,种]{s.}{negócios; trabalho vocacional; trabalho profissional}
\end{entry}

\begin{entry}{业余}{ye4yu2}{5,7}{⼀、⼈}[HSK 4]
  \definition{adj.}{tempo livre; depois do expediente; fora do horário de trabalho | amador; não profissional}
\end{entry}

\begin{entry}{叶子}{ye4zi5}{5,3}{⼝、⼦}[HSK 4]
  \definition[片]{s.}{folha; termo genérico para as folhas de uma planta}
\end{entry}

\begin{entry}{页}{ye4}{6}{⾴}[HSK 1][Kangxi 181]
  \definition{clas.}{página; folha de papel; lâmina; antigamente, referia-se a uma folha de um livro encadernado; atualmente, refere-se a uma das faces de um livro impresso em ambos os lados}
  \definition{s.}{página; folha de papel; folhas soltas de um livro}
\end{entry}

\begin{entry}{夜}{ye4}{8}{⼣}[HSK 2]
  \definition{s.}{noite}
\end{entry}

\begin{entry}{夜场}{ye4chang3}{8,6}{⼣、⼟}
  \definition{s.}{show noturno (em um teatro, etc.) | local de entretenimento noturno (bar, boate, discoteca, etc.)}
\end{entry}

\begin{entry}{夜店}{ye4dian4}{8,8}{⼣、⼴}
  \definition{s.}{boate | \emph{nightclub}}
\end{entry}

\begin{entry}{夜间}{ye4 jian1}{8,7}{⼣、⾨}[HSK 5]
  \definition{s.}{noite; à noite; noturno; durante a noite}
\end{entry}

\begin{entry}{夜里}{ye4li5}{8,7}{⼣、⾥}[HSK 2]
  \definition{adv.}{à noite | durante a noite | período noturno}
\end{entry}

\begin{entry}{夜幕}{ye4mu4}{8,13}{⼣、⼱}
  \definition{s.}{cortina da noite}
\end{entry}

\begin{entry}{夜鸟}{ye4niao3}{8,5}{⼣、⿃}
  \definition{s.}{ave noturna}
\end{entry}

\begin{entry}{夜深人静}{ye4shen1ren2jing4}{8,11,2,14}{⼣、⽔、⼈、⾭}
  \definition{expr.}{``Na calada da noite.''}
\end{entry}

\begin{entry}{夜生活}{ye4sheng1huo2}{8,5,9}{⼣、⽣、⽔}
  \definition{s.}{vida noturna}
\end{entry}

\begin{entry}{夜晚}{ye4wan3}{8,11}{⼣、⽇}
  \definition[个]{s.}{noite}
\end{entry}

\begin{entry}{夜夜}{ye4ye4}{8,8}{⼣、⼣}
  \definition{adv.}{toda noite}
\end{entry}

\begin{entry}{液体}{ye4ti3}{11,7}{⽔、⼈}
  \definition{adj./s.}{líquido}
\end{entry}

\begin{entry}{一}{yi1}[(quando usado sozinho)]{1}{⼀}[HSK 1][Kangxi 1]
  \definition{adv.}{uma vez; assim que; indica que duas ações ocorreram em um intervalo de tempo muito curto, uma terminando e a outra começando imediatamente em seguida | indica que primeiro se realiza uma ação e, em seguida, o resultado dessa ação  | indica uma ação única, indicando que a ação é muito curta ou apenas uma tentativa}
  \definition{num.}{um; 1 | pronunciado como \dpy{yao1} quando dito número a número | igual; refere-se ao mesmo ou igual | inteiro; todo; por toda parte | exclusivo ou único | refere-se a algo específico | também; caso contrário; referindo-se a outro ou mais um}
  \definition{part.}{antes de certas palavras para dar ênfase}
  \definition{prep.}{cada; por; toda vez}
  \definition{s.}{uma nota da escala em Gongchepu (工尺谱), correspondente ao 17 na notação musical numerada}
  \seeref{一}{yi2}
  \seeref{一}{yi4}
  \seealsoref{工尺谱}{gong1 che3 pu3}
\end{entry}

\begin{entry}{一……就……}{yi1 jiu4}{1,12}{⼀、⼪}
  \definition{expr.}{logo que |  uma vez que}
\end{entry}

\begin{entry}{一行}{yi1xing2}{1,6}{⼀、⾏}
  \definition{s.}{festa | delegação}
\end{entry}

\begin{entry}{伊朗}{yi1lang3}{6,10}{⼈、⽉}
  \definition*{s.}{Irã}
\end{entry}

\begin{entry}{伊马姆}{yi1ma3mu3}{6,3,8}{⼈、⾺、⼥}
  \definition*{s.}{Islã}
  \seealsoref{伊玛目}{yi1ma3mu4}
  \seealsoref{伊曼}{yi1man4}
  \seealsoref{伊斯兰}{yi1si1lan2}
\end{entry}

\begin{entry}{伊玛目}{yi1ma3mu4}{6,7,5}{⼈、⽟、⽬}
  \definition*{s.}{Islã}
  \seealsoref{伊马姆}{yi1ma3mu3}
  \seealsoref{伊曼}{yi1man4}
  \seealsoref{伊斯兰}{yi1si1lan2}
\end{entry}

\begin{entry}{伊曼}{yi1man4}{6,11}{⼈、⽈}
  \definition*{s.}{Islã}
  \seealsoref{伊马姆}{yi1ma3mu3}
  \seealsoref{伊玛目}{yi1ma3mu4}
  \seealsoref{伊斯兰}{yi1si1lan2}
\end{entry}

\begin{entry}{伊斯兰}{yi1si1lan2}{6,12,5}{⼈、⽄、⼋}
  \definition*{s.}{Islã}
  \seealsoref{伊马姆}{yi1ma3mu3}
  \seealsoref{伊玛目}{yi1ma3mu4}
  \seealsoref{伊曼}{yi1man4}
\end{entry}

\begin{entry}{衣}{yi1}{6}{⾐}[Kangxi 145]
  \definition[件]{s.}{roupa}
  \seeref{衣}{yi4}
\end{entry}

\begin{entry}{衣服}{yi1fu5}{6,8}{⾐、⽉}[HSK 1]
  \definition[套,件]{s.}{roupas; vestuário; algo que se veste para cobrir o corpo e se proteger do frio}
\end{entry}

\begin{entry}{衣柜}{yi1gui4}{6,8}{⾐、⽊}
  \definition[个]{s.}{armário | guarda-roupa}
\end{entry}

\begin{entry}{衣甲}{yi1jia3}{6,5}{⾐、⽥}
  \definition{s.}{armadura}
\end{entry}

\begin{entry}{衣架}{yi1 jia4}{6,9}{⾐、⽊}[HSK 3]
  \definition[个]{s.}{cabideiro; móvel para pendurar roupas | estatura; figura; refere-se ao formato do corpo de uma pessoa; estrutura corporal}
\end{entry}

\begin{entry}{医}{yi1}{7}{⼖}
  \definition{s.}{médico | medicina}
  \definition{v.}{curar | tratar}
\end{entry}

\begin{entry}{医疗}{yi1 liao2}{7,7}{⼖、⽧}[HSK 4]
  \definition{s.}{tratamento médico; tratamento de doenças}
\end{entry}

\begin{entry}{医生}{yi1sheng1}{7,5}{⼖、⽣}[HSK 1]
  \definition[位,个,名]{s.}{médico; clínico; pessoa que possui conhecimentos médicos e cuja profissão é tratar doenças}
\end{entry}

\begin{entry}{医学}{yi1 xue2}{7,8}{⼖、⼦}[HSK 4]
  \definition{s.}{medicina; iatrologia; ciência médica; ciência da prevenção e do tratamento de doenças e da proteção e promoção da saúde humana}
\end{entry}

\begin{entry}{医院}{yi1yuan4}{7,9}{⼖、⾩}[HSK 1]
  \definition[家,所,个]{s.}{hospital; instituições que tratam e cuidam de pacientes, e também realizam exames de saúde, prevenção de doenças, etc.}
\end{entry}

\begin{entry}{依法}{yi1 fa3}{8,8}{⼈、⽔}[HSK 5]
  \definition{adv.}{e acordo com regras (ou métodos) fixas | de acordo com a lei; por força da lei; em conformidade com as disposições legais}
\end{entry}

\begin{entry}{依旧}{yi1jiu4}{8,5}{⼈、⽇}[HSK 5]
  \definition{adv.}{ainda; como antes; como sempre}
\end{entry}

\begin{entry}{依据}{yi1ju4}{8,11}{⼈、⼿}[HSK 5]
  \definition{prep.}{julgando por; de acordo com; à luz de; com base em; de acordo com; introduzir algo que possa servir como premissa ou base}
  \definition{s.}{base; evidência; fundamento; base para tomar uma decisão ou realizar uma ação}
  \definition{v.}{basear-se em; confiar em; depdender de; usar algo como premissa ou base}
\end{entry}

\begin{entry}{依靠}{yi1kao4}{8,15}{⼈、⾮}[HSK 4]
  \definition{s.}{apoio; suporte; algo em que se apoiar; alguém ou algo em quem você pode confiar}
  \definition{v.}{depender de; confiar em (alguém ou alguma coisa para atingir um determinado objetivo)}
\end{entry}

\begin{entry}{依然}{yi1ran2}{8,12}{⼈、⽕}[HSK 4]
  \definition{adv.}{ainda; como antes;}
  \definition{v.}{estar quieto; estar como antes; estar como o original, sem alterações}
\end{entry}

\begin{entry}{依偎}{yi1wei1}{8,11}{⼈、⼈}
  \definition{v.}{aninhar-se | aconchegar-se}
\end{entry}

\begin{entry}{依照}{yi1 zhao4}{8,13}{⼈、⽕}[HSK 5]
  \definition{prep.}{de acordo com; à luz de; introduzir certos padrões para os eventos, o que equivale a 按照}
  \definition{v.}{seguir (com base em algo)}
  \seealsoref{按照}{an4zhao4}
\end{entry}

\begin{entry}{毉}{yi1}{18}{⼖}
  \variantof{医}
\end{entry}

\begin{entry}{一}{yi2}[(antes de quarto tom)]{1}{⼀}[HSK 1]
  \definition{num.}{um; 1 | um (artigo)}
  \seeref{一}{yi1}
  \seeref{一}{yi4}
\end{entry}

\begin{entry}{一半}{yi2ban4}{1,5}{⼀、⼗}[HSK 1]
  \definition{num.}{metade; em parte; uma metade}
\end{entry}

\begin{entry}{一辈子}{yi2bei4zi5}{1,12,3}{⼀、⾞、⼦}[HSK 5]
  \definition{s.}{uma vida inteira; vida inteira; toda a vida; durante toda a vida; enquanto se vive; todo o tempo entre o nascimento e a morte}
\end{entry}

\begin{entry}{一部分}{yi2 bu4 fen4}{1,10,4}{⼀、⾢、⼑}[HSK 2]
  \definition{adv.}{parcialmente}
  \definition{num.}{parte | porção | seção | fração}
  \definition[把]{s.}{parcial}
\end{entry}

\begin{entry}{一带}{yi2 dai4}{1,9}{⼀、⼱}[HSK 5]
  \definition{s.}{a área em torno de um determinado local; refere-se a um determinado local e suas proximidades.}
\end{entry}

\begin{entry}{一旦}{yi2dan4}{1,5}{⼀、⽇}[HSK 5]
  \definition{adv.}{uma vez; no caso; agora que | de repente; uma vez}
  \definition{s.}{em um único dia; em um tempo muito curto;}
\end{entry}

\begin{entry}{一道}{yi2dao4}{1,12}{⼀、⾡}
  \definition{adv.}{juntos | ao lado}
\end{entry}

\begin{entry}{一定}{yi2ding4}{1,8}{⼀、⼧}[HSK 2]
  \definition{adv.}{certamente | definitivamente}
\end{entry}

\begin{entry}{一个样}{yi2ge5yang4}{1,3,10}{⼀、⼈、⽊}
  \definition{adj.}{igual | mesmo}
  \seealsoref{一样}{yi2yang4}
\end{entry}

\begin{entry}{一共}{yi2gong4}{1,6}{⼀、⼋}[HSK 2]
  \definition{adv.}{completamente | no total | no todo | em suma}
\end{entry}

\begin{entry}{一会儿}{yi2 hui4r5}{1,6,2}{⼀、⼈、⼉}[HSK 1,2]
  \definition{adv.}{agora\dots agora\dots; um momento\dots o próximo\dots; usado antes de dois antônimos, indica a alternância de situações}
  \definition{s.}{um pouquinho de tempo; muito pouco tempo}
\end{entry}

\begin{entry}{一句话}{yi2 ju4 hua4}{1,5,8}{⼀、⼝、⾔}[HSK 5]
  \definition{s.}{em resumo; em uma palavra; expressar um conteúdo complexo de forma sucinta | trabalho fácil; fácil de fazer; descrever uma tarefa ou trabalho como muito simples e fácil de realizar}
\end{entry}

\begin{entry}{一块}{yi2kuai4}{1,7}{⼀、⼟}
  \definition{adv.}{(principalmente mandarim) juntos}
\end{entry}

\begin{entry}{一块儿}{yi2 kuai4r5}{1,7,2}{⼀、⼟、⼉}[HSK 1]
  \definition{adv.}{juntos; em conjunto}
  \definition{s.}{no mesmo lugar; no mesmo local}
\end{entry}

\begin{entry}{一路}{yi2 lu4}{1,13}{⼀、⾜}[HSK 5]
  \definition{adv.}{o tempo todo; persistentemente; continuamente | juntos; sem parar; continuamente}
  \definition{s.}{o mesmo caminho; a mesma rota; ao longo de toda a viagem, ao longo do caminho | do mesmo tipo; da mesma categoria}
\end{entry}

\begin{entry}{一路平安}{yi2 lu4 ping2 an1}{1,13,5,6}{⼀、⾜、⼲、⼧}[HSK 2]
  \definition{expr.}{Boa viagem!}
  \definition{v.}{ter uma viagem agradável}
\end{entry}

\begin{entry}{一路顺风}{yi2 lu4 shun4 feng1}{1,13,9,4}{⼀、⾜、⾴、⾵}[HSK 2]
  \definition{expr.}{ter uma viagem agradável}
\end{entry}

\begin{entry}{一律}{yi2lv4}{1,9}{⼀、⼻}[HSK 4]
  \definition{adj.}{igual; semelhante; uniforme; parecido; idêntico}
  \definition{adv.}{todos; tudo; sem exceção; enfatiza que todos devem ser assim, sem exceção, e é usado principalmente em regulamentos ou requisitos}
\end{entry}

\begin{entry}{一切}{yi2qie4}{1,4}{⼀、⼑}[HSK 3]
  \definition{pron.}{tudo; todo; todas as coisas}
\end{entry}

\begin{entry}{一下}{yi2xia4}{1,3}{⼀、⼀}
  \definition{adv.}{em um curto tempo | rapidamente}
\end{entry}

\begin{entry}{一下儿}{yi2 xia4r5}{1,3,2}{⼀、⼀、⼉}[HSK 1,5]
  \definition{s.}{um tempo; um momento}
\end{entry}

\begin{entry}{一下子}{yi2 xia4 zi5}{1,3,3}{⼀、⼀、⼦}[HSK 5]
  \definition{adv.}{tudo de uma vez; de repente; em pouco tempo; em um curto espaço de tempo}
\end{entry}

\begin{entry}{一向}{yi2xiang4}{1,6}{⼀、⼝}[HSK 5]
  \definition{adv.}{desde o início; indica do passado até o presente}
\end{entry}

\begin{entry}{一样}{yi2yang4}{1,10}{⼀、⽊}[HSK 1]
  \definition{adj.}{o mesmo; igualmente; semelhante; tão\dots quanto\dots}
  \definition{part.}{na mesma medida; anexado a verbos ou palavras nominais, indica uma comparação ou semelhança, equivalente a 似的}
  \seealsoref{似的}{shi4de5}
\end{entry}

\begin{entry}{一再}{yi2zai4}{1,6}{⼀、⼌}[HSK 4]
  \definition{adv.}{repetidamente; de novo e de novo; repetidas vezes}
\end{entry}

\begin{entry}{一战}{yi2zhan4}{1,9}{⼀、⼽}
  \definition*{s.}{Primeira Guerra Mundial}
\end{entry}

\begin{entry}{一致}{yi2zhi4}{1,10}{⼀、⾄}[HSK 4]
  \definition{adj.}{equado; idêntico; uniforme; unânime; nenhuma diferença (de opinião ou ação)}
  \definition{adv.}{juntos; em conjunto}
\end{entry}

\begin{entry}{仪式}{yi2shi4}{5,6}{⼈、⼷}
  \definition{s.}{cerimônia}
\end{entry}

\begin{entry}{移}{yi2}{11}{⽲}[HSK 4]
  \definition*{s.}{sobrenome Yi}
  \definition{v.}{mover; remover; deslocar; mudar | mudar; alterar}
\end{entry}

\begin{entry}{移动}{yi2dong4}{11,6}{⽲、⼒}[HSK 4]
  \definition{v.}{deslocar; mover; mudar}
\end{entry}

\begin{entry}{移民}{yi2min2}{11,5}{⽲、⽒}[HSK 4]
  \definition{s.}{emigrante; migrantes; aqueles que se mudam para um país ou estado estrangeiro para se estabelecer}
  \definition{v.}{migrar; imigrar}
\end{entry}

\begin{entry}{遗案}{yi2'an4}{12,10}{⾡、⽊}
  \definition{s.}{(lei) caso não resolvido}
\end{entry}

\begin{entry}{遗产}{yi2chan3}{12,6}{⾡、⼇}[HSK 4]
  \definition[笔,份]{s.}{legado; herança; patrimônio; propriedade deixada pelo falecido | patrimônio; riqueza cultural ou riqueza material transmitida pela história}
\end{entry}

\begin{entry}{遗传}{yi2chuan2}{12,6}{⾡、⼈}[HSK 4]
  \definition{v.}{herdar, descender, transmitir, passar adiante}
\end{entry}

\begin{entry}{遗骸}{yi2hai2}{12,15}{⾡、⾻}
  \definition{v.}{restos mortais}
\end{entry}

\begin{entry}{遗憾}{yi2han4}{12,16}{⾡、⼼}
  \definition{v.}{ter pena de | lamentar}
\end{entry}

\begin{entry}{遗迹}{yi2ji4}{12,9}{⾡、⾡}
  \definition{s.}{vestígios históricos | remanescente | vestígio}
\end{entry}

\begin{entry}{遗落}{yi2luo4}{12,12}{⾡、⾋}
  \definition{v.}{esquecer | deixar para trás (inadvertidamente) | deixar de fora | omitir}
\end{entry}

\begin{entry}{遗男}{yi2nan2}{12,7}{⾡、⽥}
  \definition{s.}{órfão | filho póstumo}
\end{entry}

\begin{entry}{遗嘱}{yi2zhu3}{12,15}{⾡、⼝}
  \definition{s.}{testamento}
\end{entry}

\begin{entry}{颐和园}{yi2he2yuan2}{13,8,7}{⾴、⼝、⼞}
  \definition*{s.}{Palácio de Verão}
\end{entry}

\begin{entry}{疑问}{yi2wen4}{14,6}{⽦、⾨}[HSK 4]
  \definition[个]{s.}{dúvida; consulta; pergunta; questionamento; coisas que não podem ser determinadas ou explicadas}
\end{entry}

\begin{entry}{乙}{yi3}{1}{⼄}[HSK 5]
  \definition*{s.}{sobrenome Yi}
  \definition*{s.}{o segundo lugar do Tian Gan}
  \definition{num.}{segundo}
  \definition{s.}{uma nota da escala em Gongchepu (工尺谱); nível superior na música tradicional chinesa}
  \seealsoref{工尺谱}{gong1 che3 pu3}
\end{entry}

\begin{entry}{已}{yi3}{3}{⼰}[HSK 3]
  \definition{adv.}{já | depois; mais tarde; depois de um tempo}
\end{entry}

\begin{entry}{已故}{yi3gu4}{3,9}{⼰、⽁}
  \definition{adj.}{morto | atrasado}
\end{entry}

\begin{entry}{已婚}{yi3hun1}{3,11}{⼰、⼥}
  \definition{adj.}{casado}
\end{entry}

\begin{entry}{已经}{yi3jing1}{3,8}{⼰、⽷}[HSK 2]
  \definition{adv.}{já}
\end{entry}

\begin{entry}{已久}{yi3jiu3}{3,3}{⼰、⼃}
  \definition{adv.}{já faz muito tempo}
\end{entry}

\begin{entry}{已灭}{yi3mie4}{3,5}{⼰、⽕}
  \definition{adj.}{extinto}
\end{entry}

\begin{entry}{已然}{yi3ran2}{3,12}{⼰、⽕}
  \definition{adv.}{já | já ser assim}
\end{entry}

\begin{entry}{已知}{yi3zhi1}{3,8}{⼰、⽮}
  \definition{v.}{conhecido (ter ciência)}
\end{entry}

\begin{entry}{以便}{yi3bian4}{4,9}{⼈、⼈}[HSK 5]
  \definition{conj.}{para que; de modo que; a fim de; com o objetivo de; para o propósito de; usado no início da segunda parte da frase, indica que o objetivo mencionado na segunda parte será facilmente alcançado}
\end{entry}

\begin{entry}{以此}{yi3ci3}{4,6}{⼈、⽌}
  \definition{adv.}{devido a esta | deste modo | por isso | com isso}
\end{entry}

\begin{entry}{以后}{yi3 hou4}{4,6}{⼈、⼝}[HSK 2]
  \definition{adv.}{depois de | depois | após}
\end{entry}

\begin{entry}{以及}{yi3ji2}{4,3}{⼈、⼃}[HSK 4]
  \definition{conj.}{assim como; juntamente como; bem como; também}
\end{entry}

\begin{entry}{以来}{yi3lai2}{4,7}{⼈、⽊}[HSK 3]
  \definition{prep.}{desde (um evento anterior); indica um período de um certo tempo no passado até o presente}
\end{entry}

\begin{entry}{以免}{yi3mian3}{4,7}{⼈、⼉}
  \definition{conj.}{para evitar isso}
\end{entry}

\begin{entry}{以内}{yi3 nei4}{4,4}{⼈、⼌}[HSK 4]
  \definition{adv.}{dentro de; menos que; não mais que; dentro de certos limites de tempo, premissas, quantidade e escopo}
\end{entry}

\begin{entry}{以期}{yi3qi1}{4,12}{⼈、⽉}
  \definition{v.}{tentando | esperando | esperando por}
\end{entry}

\begin{entry}{以前}{yi3qian2}{4,9}{⼈、⼑}[HSK 2]
  \definition{adv.}{antes de | antes}
\end{entry}

\begin{entry}{以求}{yi3qiu2}{4,7}{⼈、⽔}
  \definition{conj.}{a fim de}
\end{entry}

\begin{entry}{以色列}{yi3se4lie4}{4,6,6}{⼈、⾊、⼑}
  \definition*{s.}{Israel}
\end{entry}

\begin{entry}{以上}{yi3 shang4}{4,3}{⼈、⼀}[HSK 2]
  \definition{s.}{mais que | sobre | acima | o acima | o precedente | o acima mencionado}
\end{entry}

\begin{entry}{以外}{yi3 wai4}{4,5}{⼈、⼣}[HSK 2]
  \definition{s.}{além | exceto | fora | diferente de}
\end{entry}

\begin{entry}{以往}{yi3wang3}{4,8}{⼈、⼻}[HSK 5]
  \definition{s.}{antes; anterior; no passado}
\end{entry}

\begin{entry}{以为}{yi3wei2}{4,4}{⼈、⼂}[HSK 2]
  \definition{v.}{pensar, ou seja, considerar que\dots (geralmente há uma implicação de que a noção está errada --- exceto ao expressar a própria opnião atual)}
\end{entry}

\begin{entry}{以下}{yi3 xia4}{4,3}{⼈、⼀}[HSK 2]
  \definition[所]{s.}{abaixo | sob | seguinte}
\end{entry}

\begin{entry}{以至}{yi3zhi4}{4,6}{⼈、⾄}
  \definition{adv.}{até}
  \definition{conj.}{a tal ponto que\dots}
  \seealsoref{以至于}{yi3zhi4yu2}
\end{entry}

\begin{entry}{以至于}{yi3zhi4yu2}{4,6,3}{⼈、⾄、⼆}
  \definition{adv.}{até}
  \definition{conj.}{na medida em que\dots}
  \seealsoref{以至}{yi3zhi4}
\end{entry}

\begin{entry}{椅子}{yi3zi5}{12,3}{⽊、⼦}[HSK 2]
  \definition[把,套]{s.}{cadeira}
\end{entry}

\begin{entry}{一}{yi4}{1}{⼀}[HSK 1]
  \definition{adv.}{uma vez | assim que | ao}
  \definition{num.}{um; 1 | um (artigo)}
  \seeref{一}{yi1}
  \seeref{一}{yi2}
\end{entry}

\begin{entry}{一般}{yi4ban1}{1,10}{⼀、⾈}[HSK 2]
  \definition{adj.}{geral | comum | normal}
  \definition{adv.}{normalmente}
\end{entry}

\begin{entry}{一般来说}{yi4 ban1 lai2 shuo1}{1,10,7,9}{⼀、⾈、⽊、⾔}[HSK 4]
  \definition{expr.}{de modo geral; na média; no caso usual; a declaração usual}
\end{entry}

\begin{entry}{一边}{yi4bian1}{1,5}{⼀、⾡}[HSK 1]
  \definition{adj.}{igual; idêntico; da mesma forma}
  \definition{adv.}{enquanto; ao mesmo tempo; simultaneamente; indica que uma ação ocorre simultaneamente a outra ação}
  \definition{s.}{lado; um lado; um aspecto | ambos os lados; ao lado de}
\end{entry}

\begin{entry}{一点点}{yi4 dian3 dian3}{1,9,9}{⼀、⽕、⽕}[HSK 2]
  \definition{adj.}{um pouco}
\end{entry}

\begin{entry}{一点儿}{yi4dian3r5}{1,9,2}{⼀、⽕、⼉}[HSK 1]
  \definition{adv.}{um pouco; uma pitada; uma gota; uma amostra; uma pequena quantidade; ({adj.} + (一)点儿, 一点儿 + {s.} ou 有 + (一)点儿 + {s.})}
\end{entry}

\begin{entry}{一方面}{yi4 fang1 mian4}{1,4,9}{⼀、⽅、⾯}[HSK 3]
  \definition{s.}{um lado; um dos dois aspectos opostos ou um lado de algo que está relacionado a outro}
\end{entry}

\begin{entry}{一方面……,一方面……}{yi4 fang1 mian4 yi4 fang1 mian4}{1,4,9,1,4,9}{⼀、⽅、⾯、⼀、⽅、⾯}[HSK 3]
  \definition{conj.}{por um lado\dots, por outro lado\dots; conecta duas orações paralelas (devem ser usadas juntas)}
\end{entry}

\begin{entry}{一口气}{yi4 kou3 qi4}{1,3,4}{⼀、⼝、⽓}[HSK 5]
  \definition{adv.}{em um só fôlego; sem pausa}
\end{entry}

\begin{entry}{一流}{yi4liu2}{1,10}{⼀、⽔}[HSK 5]
  \definition{adj.}{clássico; de primeira linha; de primeira classe; o melhor}
  \definition[些]{s.}{tipo; mesmo tipo; da mesma classe; da mesma categoria; uma categoria}
\end{entry}

\begin{entry}{一齐}{yi4qi2}{1,6}{⼀、⿑}
  \definition{adv.}{tudo ao mesmo tempo | em uníssono | junto}
\end{entry}

\begin{entry}{一起}{yi4qi3}{1,10}{⼀、⾛}[HSK 1]
  \definition{adv.}{juntos; em companhia; indica o mesmo local, ao mesmo tempo que se faz algo | no total; em todos; no conjunto}
  \definition{s.}{no mesmo lugar}
\end{entry}

\begin{entry}{一身}{yi4 shen1}{1,7}{⼀、⾝}[HSK 5]
  \definition{s.}{o corpo inteiro; em todo o corpo | um terno; (um conjunto completo de) roupas | sozinho; uma única pessoa; relativo a uma única pessoa}
\end{entry}

\begin{entry}{一生}{yi4 sheng1}{1,5}{⼀、⽣}[HSK 2]
  \definition{s.}{toda a vida | ao longo da vida | a vida de alguém}
\end{entry}

\begin{entry}{一时}{yi4shi2}{1,7}{⼀、⽇}
  \definition{adv.}{por pouco tempo | por um tempo | temporariamente | momentaneamente | uma vez | de tempos em tempos | ocasionalmente}
\end{entry}

\begin{entry}{一同}{yi4tong2}{1,6}{⼀、⼝}
  \definition{adv.}{juntos, ao mesmo tempo}
\end{entry}

\begin{entry}{一些}{yi4 xie1}{1,8}{⼀、⼆}[HSK 1]
  \definition{clas.}{alguns; um número de; quantidade indeterminada | um pouco; uma pequena quantidade | mais de um; mais de uma vez; indica mais de um ou mais de uma vez, etc. | uma ligeira mudança no grau, intensidade; usado após certos verbos, adjetivos, etc., para indicar uma quantidade muito pequena}
  \definition{pron.}{uns; alguns}
\end{entry}

\begin{entry}{一直}{yi4zhi2}{1,8}{⼀、⽬}[HSK 2]
  \definition{adv.}{diretamente | sempre em frente | o tempo todo | sempre | constantemente}
\end{entry}

\begin{entry}{义务}{yi4wu4}{3,5}{⼂、⼒}[HSK 4]
  \definition{s.}{dever; obrigação; responsabilidades perante a lei, em oposição a 权利}
  \seealsoref{权利}{quan2li4}
\end{entry}

\begin{entry}{亿}{yi4}{3}{⼈}[HSK 2]
  \definition{num.}{cem milhões; 100.000.000; 1.0000.0000}
\end{entry}

\begin{entry}{艺人}{yi4ren2}{4,2}{⾋、⼈}
  \definition{s.}{artista | ator}
\end{entry}

\begin{entry}{艺术}{yi4shu4}{4,5}{⾋、⽊}[HSK 3]
  \definition{adj.}{artístico,único e lindo}
  \definition[个]{s.}{arte; literatura e arte | habilidade; arte; ofício; abordagem criativa}
\end{entry}

\begin{entry}{议论}{yi4lun4}{5,6}{⾔、⾔}[HSK 4]
  \definition{s.}{comentário; discussão; opiniões ou pontos de vista sobre o que é bom ou ruim, certo ou errado em relação a pessoas ou coisas}
  \definition{v.}{discutir; comentar; falar sobre; expressar opiniões e trocar pontos de vista sobre o bom, o ruim, o certo e o errado de pessoas ou coisas}
\end{entry}

\begin{entry}{亦}{yi4}{6}{⼇}
  \definition{adv.}{também | igualmente | apenas | embora | já}
\end{entry}

\begin{entry}{异常}{yi4chang2}{6,11}{⼶、⼱}
  \definition{adj.}{extraordinário | anormal}
  \definition{adv.}{extraordinariamente | excepcionalmente}
  \definition{s.}{anormalidade}
\end{entry}

\begin{entry}{衣}{yi4}{6}{⾐}[Kangxi 145]
  \definition{v.}{vestir | vestir-se}
  \seeref{衣}{yi1}
\end{entry}

\begin{entry}{易}{yi4}{8}{⽇}
  \definition*{s.}{sobrenome Yi}
  \definition{adj.}{fácil | amigável; pacífico}
  \definition{v.}{modificar; transformar | trocar | subestimar; desprezar}
\end{entry}

\begin{entry}{意见}{yi4jian4}{13,4}{⼼、⾒}[HSK 2]
  \definition[点,条]{s.}{reclamação | ideia | objeção | opinião | sugestão}
\end{entry}

\begin{entry}{意识}{yi4shi2}{13,7}{⼼、⾔}[HSK 5]
  \definition{s.}{consciência}
  \definition{s.}{consciência; percepção; grau de reconhecimento e importância atribuído a uma determinada questão}
  \definition{v.}{perceber; despertar para; estar ciente de; sentir, descobrir o que antes não se sentia ou não se descobria; geralmente é usado junto com 到}
  \seealsoref{到}{dao4}
\end{entry}

\begin{entry}{意思}{yi4si5}{13,9}{⼼、⼼}[HSK 2]
  \definition[个]{s.}{interesse}
\end{entry}

\begin{entry}{意外}{yi4wai4}{13,5}{⼼、⼣}[HSK 3]
  \definition{adj.}{inesperado; imprevisto}
  \definition{adv.}{acidentalmente}
  \definition[个]{s.}{acidente; infortúnio; infortúnio inesperado}
\end{entry}

\begin{entry}{意味着}{yi4wei4zhe5}{13,8,11}{⼼、⼝、⽬}[HSK 5]
  \definition{v.}{significar; subentender}
\end{entry}

\begin{entry}{意义}{yi4yi4}{13,3}{⼼、⼂}[HSK 3]
  \definition[个]{s.}{sentido; significado; significado expresso por palavras ou outros sinais; significado indicado por comportamento ou aquisição |valor; efeito; significância; impacto}
\end{entry}

\begin{entry}{意译}{yi4yi4}{13,7}{⼼、⾔}
  \definition{s.}{tradução livre | significado (de expressão estrangeira) | paráfrase | tradução do significado (em oposição à tradução literal)}
  \seealsoref{直译}{zhi2yi4}
\end{entry}

\begin{entry}{意指}{yi4zhi3}{13,9}{⼼、⼿}
  \definition{v.}{implicar | significar}
\end{entry}

\begin{entry}{意志}{yi4zhi4}{13,7}{⼼、⼼}[HSK 5]
  \definition[个,股]{s.}{vontade; determinação; desejo; força de vontade}
\end{entry}

\begin{entry}{因}{yin1}{6}{⼞}
  \definition*{s.}{sobrenome Yin}
  \definition{conj.}{porque; orações de conexão, indicando relações de causa e efeito}
  \definition{prep.}{com base em; à luz de; de acordo com; a introdução da ação comportamental equivale a 按照 ou 根据}
  \definition{s.}{causa; motivo; condições em que algo ocorre ou causa um determinado resultado (em oposição a 果)}
  \definition{v.}{seguir; continuar; fazer como sempre fez | estar em conformidade com; estar de acordo com; depender; contar com}
  \seealsoref{按照}{an4zhao4}
  \seealsoref{根据}{gen1ju4}
  \seealsoref{果}{guo3}
\end{entry}

\begin{entry}{因此}{yin1ci3}{6,6}{⼞、⽌}[HSK 3]
  \definition{conj.}{assim; por isso; portanto; consequentemente}
\end{entry}

\begin{entry}{因此就}{yin1ci3 jiu4}{6,6,12}{⼞、⽌、⼪}
  \definition{conj.}{portanto}
\end{entry}

\begin{entry}{因而}{yin1'er2}{6,6}{⼞、⽽}[HSK 5]
  \definition{conj.}{; como resultado; com o resultado que; conecta frases, indicando relação de causa e efeito}
\end{entry}

\begin{entry}{因为}{yin1wei4}{6,4}{⼞、⼂}[HSK 2]
  \definition{conj.}{porque | devido a | por conta de}
\end{entry}

\begin{entry}{因为……所以……}{yin1wei4 suo3yi3}{6,4,8,4}{⼞、⼂、⼾、⼈}[HSK 2]
  \definition{conj.}{porque\dots portanto\dots}
\end{entry}

\begin{entry}{阴}{yin1}{6}{⾩}[HSK 2]
  \definition*{s.}{sobrenome Yin}
  \definition*{s.}{Yin (o princípio negativo de Yin e Yang)}
  \definition{adj.}{nublado | sombrio | escondido | implícito}
  \definition{s.}{negativo (eletricidade) | lua}
  \seealsoref{阳}{yang2}
  \seealsoref{阴阳}{yin1yang2}
\end{entry}

\begin{entry}{阴天}{yin1 tian1}{6,4}{⾩、⼤}[HSK 2]
  \definition{adj.}{céu nublado | céu cinzento}
\end{entry}

\begin{entry}{阴阳}{yin1yang2}{6,6}{⾩、⾩}
  \definition*{s.}{Yin e Yang}
  \seealsoref{阳}{yang2}
  \seealsoref{阴}{yin1}
\end{entry}

\begin{entry}{音节}{yin1 jie2}{9,5}{⾳、⾋}[HSK 2]
  \definition{s.}{sílaba}
\end{entry}

\begin{entry}{音乐}{yin1yue4}{9,5}{⾳、⼃}[HSK 2]
  \definition[张,曲,段]{s.}{música}
\end{entry}

\begin{entry}{音乐光碟}{yin1yue4guang1die2}{9,5,6,14}{⾳、⼃、⼉、⽯}
  \definition{s.}{CD de música}
\end{entry}

\begin{entry}{音乐会}{yin1 yue4 hui4}{9,5,6}{⾳、⼃、⼈}[HSK 2]
  \definition[场]{s.}{concerto}
\end{entry}

\begin{entry}{音乐家}{yin1yue4jia1}{9,5,10}{⾳、⼃、⼧}
  \definition{s.}{músico}
\end{entry}

\begin{entry}{音乐节}{yin1yue4jie2}{9,5,5}{⾳、⼃、⾋}
  \definition{s.}{festival de música}
\end{entry}

\begin{entry}{音乐厅}{yin1yue4ting1}{9,5,4}{⾳、⼃、⼚}
  \definition{s.}{auditório | teatro | \emph{concert hall}}
\end{entry}

\begin{entry}{音乐学}{yin1yue4xue2}{9,5,8}{⾳、⼃、⼦}
  \definition{s.}{musicologia}
\end{entry}

\begin{entry}{音乐学院}{yin1yue4xue2yuan4}{9,5,8,9}{⾳、⼃、⼦、⾩}
  \definition{s.}{conservatório | academia de música}
\end{entry}

\begin{entry}{音乐院}{yin1yue4yuan4}{9,5,9}{⾳、⼃、⾩}
  \definition{s.}{conservatório | instituto de música}
\end{entry}

\begin{entry}{吟诗}{yin2shi1}{7,8}{⼝、⾔}
  \definition{v.}{recitar poesia}
\end{entry}

\begin{entry}{银}{yin2}{11}{⾦}[HSK 3]
  \definition*{s.}{sobrenome Yin}
  \definition{adj.}{relativo a moeda ou dinheiro
prateado; como a cor da prata}
  \definition[锭]{s.}{prata (Ag)}
\end{entry}

\begin{entry}{银行}{yin2hang2}{11,6}{⾦、⾏}[HSK 2]
  \definition[家,个]{s.}{banco | agência bancária}
\end{entry}

\begin{entry}{银行卡}{yin2 hang2 ka3}{11,6,5}{⾦、⾏、⼘}[HSK 2]
  \definition{s.}{cartão bancário}
\end{entry}

\begin{entry}{银河}{yin2he2}{11,8}{⾦、⽔}
  \definition*{s.}{Via Láctea}
  \seealsoref{银河系}{yin2he2xi4}
\end{entry}

\begin{entry}{银河系}{yin2he2xi4}{11,8,7}{⾦、⽔、⽷}
  \definition*{s.}{Galáxia Via Láctea}
  \seealsoref{银河}{yin2he2}
\end{entry}

\begin{entry}{银牌}{yin2 pai2}{11,12}{⾦、⽚}[HSK 3]
  \definition[枚]{s.}{medalha de prata; um tipo de medalha concedida ao segundo colocado}
\end{entry}

\begin{entry}{银色}{yin2 se4}{11,6}{⾦、⾊}
  \definition{s.}{cor prata; prateado}
\end{entry}

\begin{entry}{引}{yin3}{4}{⼸}[HSK 4]
  \definition*{s.}{sobrenome Yin}
  \definition{clas.}{uma unidade de comprimento (=33⅓ metros)}
  \definition{v.}{puxar; esticar | liderar; conduzir; guiar | sair; deixar | sobressair | atrair; provocar; trazer à existência | causar; provocar | citar; ser usado como evidência ou justificativa}
\end{entry}

\begin{entry}{引导}{yin3dao3}{4,6}{⼸、⼨}[HSK 4]
  \definition{v.}{conduzir; guiar; liderar; andar na frente e deixar que os outros sigam atrás para ver ou andar; usar imagens ou sinais para mostrar às pessoas para onde ir | esclarecer; fornecer orientação em termos de ideias, métodos, conceitos, etc.}
\end{entry}

\begin{entry}{引进}{yin3 jin4}{4,7}{⼸、⾡}[HSK 4]
  \definition{v.}{importar; trazer de fora | recomendar; dar uma indicação}
\end{entry}

\begin{entry}{引起}{yin3qi3}{4,10}{⼸、⾛}[HSK 4]
  \definition{v.}{causar; despertar; levar a; desencadear; dar origem a}
\end{entry}

\begin{entry}{引擎}{yin3qing2}{4,16}{⼸、⼿}
  \definition[台]{s.}{motor | (empréstimo linguístico) \emph{engine}}
\end{entry}

\begin{entry}{听}{yin3}{7}{⼝}
  \definition[个]{s.}{lata; embalagem metálica}
  \seeref{听}{ting1}
\end{entry}

\begin{entry}{饮料}{yin3liao4}{7,10}{⾷、⽃}[HSK 5]
  \definition[杯,瓶,种]{s.}{bebida; drinque; líquidos processados e fabricados para consumo, como vinho, chá, refrigerantes, suco de laranja, etc.}
\end{entry}

\begin{entry}{饮食}{yin3shi2}{7,9}{⾷、⾷}[HSK 5]
  \definition{s.}{dieta; alimentos e bebidas}
  \definition{v.}{comer; beber}
\end{entry}

\begin{entry}{印刷}{yin4shua1}{5,8}{⼙、⼑}[HSK 5]
  \definition{v.}{imprimir; imprimir textos, imagens, etc. em papel}
\end{entry}

\begin{entry}{印象}{yin4xiang4}{5,11}{⼙、⾗}[HSK 3]
  \definition[种]{s.}{impressão; marca; ideia; os vestígios deixados por coisas objetivas na mente das pessoas}
\end{entry}

\begin{entry}{应}{ying1}{7}{⼴}[HSK 4,5]
  \definition{v.}{ecoar; responder; responder a; responder às chamadas, saudações, perguntas, etc. de outras pessoas | conceder; cumprir | adequar; adaptar; responder a | lidar com; enfrentar; abordar | tornar-se realidade; ser cumprido}
\end{entry}

\begin{entry}{应当}{ying1 dang1}{7,6}{⼴、⼹}[HSK 3]
  \definition{v.}{dever}
\end{entry}

\begin{entry}{应该}{ying1gai1}{7,8}{⼴、⾔}[HSK 2]
  \definition{v.}{dever | ter de}
\end{entry}

\begin{entry}{英国}{ying1guo2}{8,8}{⾋、⼞}
  \definition*{s.}{Reino Unido}
\end{entry}

\begin{entry}{英国人}{ying1guo2ren2}{8,8,2}{⾋、⼞、⼈}
  \definition{s.}{inglês | pessoa ou povo do Reino Unido}
\end{entry}

\begin{entry}{英文}{ying1 wen2}{8,4}{⾋、⽂}[HSK 2]
  \definition{s.}{inglês, língua inglesa}
\end{entry}

\begin{entry}{英雄}{ying1xiong2}{8,12}{⾋、⾫}
  \definition[个]{s.}{herói}
\end{entry}

\begin{entry}{英勇}{ying1yong3}{8,9}{⾋、⼒}[HSK 4]
  \definition{adj.}{heroico; valente; bravo; corajoso; extraordinariamente corajoso}
\end{entry}

\begin{entry}{英语}{ying1 yu3}{8,9}{⾋、⾔}[HSK 2]
  \definition{s.}{inglês, língua inglesa}
\end{entry}

\begin{entry}{樱桃}{ying1tao2}{15,10}{⽊、⽊}
  \definition{s.}{cereja}
\end{entry}

\begin{entry}{鹦鹉}{ying1wu3}{16,13}{⿃、⿃}
  \definition{s.}{papagaio (ave)}
\end{entry}

\begin{entry}{迎接}{ying2jie1}{7,11}{⾡、⼿}[HSK 3]
  \definition{v.}{conhecer; cumprimentar; dar as boas-vindas}
\end{entry}

\begin{entry}{营养}{ying2yang3}{11,9}{⾋、⼋}[HSK 3]
  \definition[种]{s.}{nutrição; a função de um organismo de absorver as substâncias necessárias do mundo exterior para manter o crescimento, o desenvolvimento e outras atividades vitais; o ato ou processo de fornecer nutrição}
\end{entry}

\begin{entry}{营业}{ying2ye4}{11,5}{⾋、⼀}[HSK 4]
  \definition{v.}{fazer negócios; estar aberto para negócios}
\end{entry}

\begin{entry}{赢}{ying2}{17}{⾙}[HSK 3]
  \definition[个]{s.}{ganho (lucro)}
  \definition{v.}{ganhar; derrotar; superar; conquistar}
\end{entry}

\begin{entry}{赢得}{ying2 de2}{17,11}{⾙、⼻}[HSK 4]
  \definition{v.}{ganhar; obter; conquistar; assegurar; garantir}
\end{entry}

\begin{entry}{影片}{ying3 pian4}{15,4}{⼺、⽚}[HSK 2]
  \definition[部]{s.}{filme | imagem}
\end{entry}

\begin{entry}{影视}{ying3 shi4}{15,8}{⼺、⾒}[HSK 3]
  \definition{s.}{cinema e televisão combinados}
\end{entry}

\begin{entry}{影响}{ying3xiang3}{15,9}{⼺、⼝}[HSK 2]
  \definition{s.}{efeito | influência}
  \definition{v.}{afetar | influenciar}
\end{entry}

\begin{entry}{影像}{ying3xiang4}{15,13}{⼺、⼈}
  \definition{s.}{imagem}
\end{entry}

\begin{entry}{影子}{ying3zi5}{15,3}{⼺、⼦}[HSK 4]
  \definition[个,片]{s.}{sombra; imagem projetada por um objeto, etc., que bloqueia a luz | reflexão; reflexo; imagem de um objeto, etc., conforme aparece em um refletor, como um espelho, uma superfície de água, etc. | sinal; vestígio; vaga impressão}
\end{entry}

\begin{entry}{应对}{ying4dui4}{7,5}{⼴、⼨}
  \definition{v.}{responder | manusear | lidar}
\end{entry}

\begin{entry}{应用}{ying4yong4}{7,5}{⼴、⽤}[HSK 3]
  \definition{adj.}{aplicado (na vida ou na produção); usado diretamente na vida ou na produção}
  \definition{s.}{aplicativo}
  \definition{v.}{usar; aplicar}
\end{entry}

\begin{entry}{应用程序}{ying4yong4cheng2xu4}{7,5,12,7}{⼴、⽤、⽲、⼴}
  \definition{s.}{aplicativo | programa de computador}
\end{entry}

\begin{entry*}{应用程序编程接口}{ying4yong4cheng2xu4bian1cheng2jie1kou3}{7,5,12,7,12,12,11,3}{⼴、⽤、⽲、⼴、⽷、⽲、⼿、⼝}
  \definition{s.}{API (\emph{application programming interface})}
  \seealsoref{应用程序接口}{ying4yong4cheng2xu4jie1kou3}
\end{entry*}

\begin{entry}{应用程序接口}{ying4yong4cheng2xu4jie1kou3}{7,5,12,7,11,3}{⼴、⽤、⽲、⼴、⼿、⼝}
  \definition{s.}{API (\emph{application programming interface})}
  \seealsoref{应用程序编程接口}{ying4yong4cheng2xu4bian1cheng2jie1kou3}
\end{entry}

\begin{entry}{硬}{ying4}{12}{⽯}[HSK 4,5]
  \definition{adj.}{duro; rígido; resistente;  objeto resistente e não se deforma facilmente quando submetido a forças externas (em oposição a 软) | firme; forte; resistente; obstinado; (vontade, atitude, etc.) inabalável, forte e poderoso | capaz (pessoa); boa (qualidade) | rígido; severo; sem flexibilidade | duro; rígido; rigoroso; imutável}
  \definition{adv.}{conseguir fazer algo com dificuldade; indica fazer algo à força, independentemente das circunstâncias}
  \seealsoref{软}{ruan3}
\end{entry}

\begin{entry}{硬件}{ying4jian4}{12,6}{⽯、⼈}[HSK 5]
  \definition{s.}{\emph{hardware}; nome genérico dado aos vários elementos, componentes e dispositivos que constituem um computador | máquina, materiais; equipamento; referência a máquinas, equipamentos, materiais físicos, etc., utilizados nos processos de produção, pesquisa científica, gestão, etc.}
\end{entry}

\begin{entry}{拥抱}{yong1bao4}{8,8}{⼿、⼿}[HSK 5]
  \definition[个,次]{s.}{abraço;}
  \definition{v.}{abraçar; segurar em seus braços; abraçar para demonstrar afeto}
\end{entry}

\begin{entry}{拥有}{yong1you3}{8,6}{⼿、⽉}[HSK 5]
  \definition{v.}{possuir; deter; ter (grande quantidade de terras, população, bens, etc.)}
\end{entry}

\begin{entry}{永不}{yong3bu4}{5,4}{⽔、⼀}
  \definition{adv.}{nunca}
\end{entry}

\begin{entry}{永远}{yong3yuan3}{5,7}{⽔、⾡}[HSK 2]
  \definition{adv.}{para sempre, sempre | permanentemente}
\end{entry}

\begin{entry}{泳池}{yong3chi2}{8,6}{⽔、⽔}
  \definition{s.}{piscina}
  \seealsoref{游泳池}{you2 yong3 chi2}
  \seealsoref{游泳馆}{you2yong3guan3}
\end{entry}

\begin{entry}{泳衣}{yong3yi1}{8,6}{⽔、⾐}
  \definition{s.}{roupa de banho | maiô}
  \seealsoref{游泳衣}{you2yong3yi1}
\end{entry}

\begin{entry}{勇敢}{yong3gan3}{9,11}{⼒、⽁}[HSK 4]
  \definition{adj.}{bravo; valente; galante; corajoso}
\end{entry}

\begin{entry}{勇气}{yong3qi4}{9,4}{⼒、⽓}[HSK 4]
  \definition[种,股]{s.}{coragem; arrojo; nervos; coragem para agir sem medo}
\end{entry}

\begin{entry}{勇士}{yong3shi4}{9,3}{⼒、⼠}
  \definition{s.}{um guerreiro | uma pessoa corajosa}
\end{entry}

\begin{entry}{用}{yong4}{5}{⽤}[HSK 1][Kangxi 101]
  \definition*{s.}{sobrenome Yong}
  \definition{conj.}{portanto; por isso; assim sendo; razões para a introdução, equivalentes a 因}
  \definition{prep.}{com; ação de introduzir ferramentas, meios, etc. utilizados ou empregados}
  \definition{s.}{despesas; gastos; custos | uso; utilidade; eficácia}
  \definition{v.}{usar; aplicar; empregar | necessitar (normalmente na forma negativa) | respeitosamente: comer; beber}
  \seealsoref{因}{yin1}
\end{entry}

\begin{entry}{用不着}{yong4 bu4 zhao2}{5,4,11}{⽤、⼀、⽬}[HSK 5]
  \definition{v.}{não precisar; não ter utilidade para; não haver necessidade de}
\end{entry}

\begin{entry}{用处}{yong4chu5}{5,5}{⽤、⼡}
  \definition[个]{s.}{usabilidade | utilidade}
\end{entry}

\begin{entry}{用户}{yong4hu4}{5,4}{⽤、⼾}[HSK 5]
  \definition[个,位,名]{s.}{usuário; consumidor; entidades e indivíduos que utilizam determinados equipamentos públicos ou bens de consumo}
\end{entry}

\begin{entry}{用来}{yong4 lai2}{5,7}{⽤、⽊}[HSK 5]
  \definition{v.}{ser usado para; depender (dele) ou usar (ele) para atingir algum objetivo}
\end{entry}

\begin{entry}{用料}{yong4liao4}{5,10}{⽤、⽃}
  \definition{s.}{ingredientes | materiais}
\end{entry}

\begin{entry}{用途}{yong4tu2}{5,10}{⽤、⾡}[HSK 4]
  \definition[个,种]{s.}{uso; aplicação; aspectos ou escopo da aplicação}
\end{entry}

\begin{entry}{用心}{yong4xin1}{5,4}{⽤、⼼}
  \definition{s.}{motivo | intenção}
  \definition{v.+compl.}{ser diligente ou atencioso}
\end{entry}

\begin{entry}{用于}{yong4 yu2}{5,3}{⽤、⼆}[HSK 5]
  \definition{v.}{usar para; ser usado para; usar em}
\end{entry}

\begin{entry}{优}{you1}{6}{⼈}
  \definition{adj.}{excelente | superior}
\end{entry}

\begin{entry}{优等}{you1deng3}{6,12}{⼈、⽵}
  \definition{adj.}{excelente | de primeira linha | alta classe | da mais alta ordem, superior}
\end{entry}

\begin{entry}{优点}{you1dian3}{6,9}{⼈、⽕}[HSK 3]
  \definition[个,项,种,些]{s.}{mérito; virtude; vantagem; ponto forte}
\end{entry}

\begin{entry}{优格}{you1ge2}{6,10}{⼈、⽊}
  \definition{s.}{iogurte}
\end{entry}

\begin{entry}{优厚}{you1hou4}{6,9}{⼈、⼚}
  \definition{adj.}{generoso}
\end{entry}

\begin{entry}{优惠}{you1hui4}{6,12}{⼈、⼼}[HSK 5]
  \definition{adj.}{especial; pechincha; reduzido; com desconto | favorável; preferencial; melhores condições ou tratamento do que o normal, permitindo que as pessoas obtenham mais benefícios}
\end{entry}

\begin{entry}{优良}{you1 liang2}{6,7}{⼈、⾉}[HSK 4]
  \definition{adj.}{ótimo; bom; excelente; (variedade, qualidade, desempenho, estilo, etc.) muito bom}
\end{entry}

\begin{entry}{优伶}{you1ling2}{6,7}{⼈、⼈}
  \definition{s.}{ator}
\end{entry}

\begin{entry}{优美}{you1mei3}{6,9}{⼈、⽺}[HSK 4]
  \definition{adj.}{fino; elegante; gracioso; bonito}
\end{entry}

\begin{entry}{优盘}{you1pan2}{6,11}{⼈、⽫}
  \definition{s.}{unidade de memória USB}
  \seealsoref{闪存盘}{shan3cun2pan2}
\end{entry}

\begin{entry}{优势}{you1shi4}{6,8}{⼈、⼒}[HSK 3]
  \definition[种,个]{s.}{vantagem; superioridade; preponderância; posição dominante}
\end{entry}

\begin{entry}{优先}{you1xian1}{6,6}{⼈、⼉}[HSK 5]
  \definition{adj.}{anterior; sênior; subjacente}
  \definition{v.}{ter prioridade; ter precedência; colocar-se à frente de outras pessoas ou assuntos}
\end{entry}

\begin{entry}{优秀}{you1xiu4}{6,7}{⼈、⽲}[HSK 4]
  \definition{adj.}{esplêndido; excelente; extraordinário; excepcional; notável; descreve moral, qualidades, realizações, aprendizado, etc. muito bons.}
\end{entry}

\begin{entry}{优选}{you1xuan3}{6,9}{⼈、⾡}
  \definition{v.}{otimizar}
\end{entry}

\begin{entry}{优于}{you1yu2}{6,3}{⼈、⼆}
  \definition{v.}{superar}
\end{entry}

\begin{entry}{优裕}{you1yu4}{6,12}{⼈、⾐}
  \definition{adj.}{abundante | bastante}
  \definition{s.}{abundância}
\end{entry}

\begin{entry}{优质}{you1zhi4}{6,8}{⼈、⾙}
  \definition{adj.}{excelente qualidade}
\end{entry}

\begin{entry}{忧郁}{you1yu4}{7,8}{⼼、⾢}
  \definition{adj.}{deprimido | melancólico | desanimado}
  \definition{s.}{depressão | melancolia}
\end{entry}

\begin{entry}{幽默}{you1mo4}{9,16}{⼳、⿊}[HSK 5]
  \definition{adj.}{humorístico; interessante ou engraçado, mas com um significado profundo}
  \definition{s.}{humor; lado engraçado; graça; características, temperamento, palavras ou comportamentos interessantes, engraçados ou significativos}
\end{entry}

\begin{entry}{尤其}{you2qi2}{4,8}{⼪、⼋}[HSK 5]
  \definition{adv.}{especialmente; particularmente; indica um grau mais avançado, equivalente a 更加}
  \seealsoref{更加}{geng4 jia1}
\end{entry}

\begin{entry}{由}{you2}{5}{⽥}[HSK 3]
  \definition*{s.}{sobrenome You}
  \definition{prep.}{por causa de; devido a | por; através de | indica que alguém é responsável por fazer algo | indica dependência de | de; indica o ponto de partida}
  \definition[个]{s.}{causa; razão}
  \definition{v.}{atravessar | seguir; obedecer}
\end{entry}

\begin{entry}{由此}{you2 ci3}{5,6}{⽥、⽌}[HSK 5]
  \definition{adv.}{assim; por meio disto; disto; daí; por causa disto; portanto; daqui; de agora em diante}
\end{entry}

\begin{entry}{由于}{you2yu2}{5,3}{⽥、⼆}[HSK 3]
  \definition{conj.}{porque; uma vez que; visto que; porque, usado no início da frase anterior para expressar a causa, e na frase seguinte para expressar o resultado.}
  \definition{prep.}{devido a; graças a; por causa de; em virtude de; como resultado de; apresentar as razões pelas quais um evento ou ação ocorre}
\end{entry}

\begin{entry}{犹豫}{you2yu4}{7,15}{⽝、⾗}[HSK 5]
  \definition{adj.}{hesitante; indeciso, incapaz de decidir ou agir}
  \definition{v.}{hesitar; ser indeciso}
\end{entry}

\begin{entry}{邮包}{you2bao1}{7,5}{⾢、⼓}
  \definition{s.}{encomenda postal}
\end{entry}

\begin{entry}{邮递}{you2di4}{7,10}{⾢、⾡}
  \definition{v.}{enviar por correio}
\end{entry}

\begin{entry}{邮电}{you2dian4}{7,5}{⾢、⽥}
  \definition*{s.}{Correios e Telecomunicações}
\end{entry}

\begin{entry}{邮费}{you2fei4}{7,9}{⾢、⾙}
  \definition{s.}{postagem}
  \definition{v.}{postar}
\end{entry}

\begin{entry}{邮件}{you2 jian4}{7,6}{⾢、⼈}[HSK 3]
  \definition[封,个]{s.}{correspondência; correio; assunto postal; um termo geral para cartas, encomendas, etc. recebidas, transportadas e entregues pelos correios | \emph{e-mail}; mensagens enviadas e recebidas por meio eletrônico}
\end{entry}

\begin{entry}{邮局}{you2ju2}{7,7}{⾢、⼫}[HSK 4]
  \definition[家]{s.}{correio; agência dos correios; organizações que lidam com serviços postais}
\end{entry}

\begin{entry}{邮迷}{you2mi2}{7,9}{⾢、⾡}
  \definition{s.}{filatelista | colecionador de selos}
\end{entry}

\begin{entry}{邮票}{you2 piao4}{7,11}{⾢、⽰}[HSK 3]
  \definition[枚,张,套,版]{s.}{selo; selo postal; um \emph{voucher} vendido pelos correios e afixado na correspondência para indicar que a postagem foi paga}
\end{entry}

\begin{entry}{邮市}{you2shi4}{7,5}{⾢、⼱}
  \definition{s.}{mercado postal}
\end{entry}

\begin{entry}{邮箱}{you2 xiang1}{7,15}{⾢、⾋}[HSK 3]
  \definition{s.}{caixa de correio | \emph{mailbox}; refere-se ao endereço de \emph{e-mail}}
\end{entry}

\begin{entry}{邮资}{you2zi1}{7,10}{⾢、⾙}
  \definition{s.}{postagem}
\end{entry}

\begin{entry}{油}{you2}{8}{⽔}[HSK 2]
  \definition{adj.}{oleoso | gorduroso | superficial | astuto}
  \definition{s.}{óleo | gordura | graxa | petróleo}
  \definition{v.}{aplicar óleo de tungue, tinta ou verniz}
\end{entry}

\begin{entry}{游}{you2}{12}{⽔}[HSK 3]
  \definition*{s.}{sobrenome You}
  \definition{adj.}{itinerante; errante; não fixo; frequentemente em movimento}
  \definition{s.}{parte de um rio; uma seção do rio}
  \definition{v.}{nadar; pessoas ou animais se movendo na água | vagar por aí; vagar; viajar; passear | associar com (comunicação)}
\end{entry}

\begin{entry}{游客}{you2 ke4}{12,9}{⽔、⼧}[HSK 2]
  \definition{s.}{viajante | turista | (jogo online) jogador convidado}
\end{entry}

\begin{entry}{游艇}{you2ting3}{12,12}{⽔、⾈}
  \definition[只]{s.}{barcaça | iate}
\end{entry}

\begin{entry}{游戏}{you2xi4}{12,6}{⽔、⼽}[HSK 3]
  \definition[场]{s.}{jogo; recreação; atividades recreativas, como esconde-esconde, adivinhação de enigmas de lanternas e algumas atividades esportivas informais, como bola recreativa, também são chamadas de jogos}
  \definition{v.}{jogar; fazer atividades relaxantes e prazerosas sozinho ou com outras pessoas}
\end{entry}

\begin{entry}{游泳}{you2yong3}{12,8}{⽔、⽔}[HSK 3]
  \definition[次]{s.}{natação; refere-se ao esporte ou atividade de natação}
  \definition{v.+compl.}{nadar; pessoas ou animais nadando na água}
\end{entry}

\begin{entry}{游泳池}{you2 yong3 chi2}{12,8,6}{⽔、⽔、⽔}[HSK 5]
  \definition[场,个]{s.}{piscina; piscinas artificiais para natação, divididas em duas categorias: internas e externas}
  \seealsoref{泳池}{yong3chi2}
  \seealsoref{游泳馆}{you2yong3guan3}
\end{entry}

\begin{entry}{游泳馆}{you2yong3guan3}{12,8,11}{⽔、⽔、⾷}
  \definition[场]{s.}{piscina}
  \seealsoref{泳池}{yong3chi2}
  \seealsoref{游泳池}{you2 yong3 chi2}
\end{entry}

\begin{entry}{游泳镜}{you2yong3jing4}{12,8,16}{⽔、⽔、⾦}
  \definition{s.}{óculos de natação}
\end{entry}

\begin{entry}{游泳衣}{you2yong3yi1}{12,8,6}{⽔、⽔、⾐}
  \definition{s.}{roupa de banho}
  \seealsoref{泳衣}{yong3yi1}
\end{entry}

\begin{entry}{友好}{you3hao3}{4,6}{⼜、⼥}[HSK 2]
  \definition{adj.}{amigável}
  \definition{s.}{amigo próximo, íntimo}
\end{entry}

\begin{entry}{友谊}{you3yi4}{4,10}{⼜、⾔}[HSK 5]
  \definition[段,份]{s.}{amizade; amizade entre amigos}
\end{entry}

\begin{entry}{有}{you3}{6}{⽉}[HSK 1]
  \definition*{s.}{sobrenome You}
  \definition{pref.}{usado antes do nome de certas dinastias ou etnias}
  \definition{v.}{ter; possuir; indica posse ou propriedade | existe; há; indica que certas coisas existem em certos lugares | fazer uma estimativa ou uma comparação; expressar estimativa ou comparação | indicar ação; indica que algo aconteceu ou ocorreu | usado antes de substantivos abstratos, indica quantidade ou grandeza | em termos gerais, semelhante a 某; refere-se de maneira geral a algo semelhante | usado antes de pessoa, hora e lugar, indica a existência parcial | usado antes de certos verbos para formar uma expressão idiomática, indicando cortesia, polidez}
  \seealsoref{某}{mou3}
\end{entry}

\begin{entry}{有道理}{you3dao4li5}{6,12,11}{⽉、⾡、⽟}
  \definition{adj.}{razoável}
  \definition{v.}{fazer sentido}
\end{entry}

\begin{entry}{有的}{you3 de5}{6,8}{⽉、⽩}[HSK 1]
  \definition{pron.}{algum, alguns}
\end{entry}

\begin{entry}{有的时候}{you3 de5 shi2 hou4}{6,8,7,10}{⽉、⽩、⽇、⼈}
  \definition{adv.}{às vezes; ocasionalmente}
  \seealsoref{有时}{you3 shi2}
  \seealsoref{有时候}{you3 shi2 hou5}
\end{entry}

\begin{entry}{有的是}{you3 de5 shi4}{6,8,9}{⽉、⽩、⽇}[HSK 3]
  \definition{expr.}{tem bastante; não falta; enfatiza que existem muitos}
\end{entry}

\begin{entry}{有毒}{you3 du2}{6,9}{⽉、⽏}[HSK 5]
  \definition{adj.}{venenoso; tóxico; nocivo; geralmente é usada para descrever as propriedades nocivas à saúde de produtos químicos, plantas ou animais.}
\end{entry}

\begin{entry}{有害}{you3 hai4}{6,10}{⽉、⼧}[HSK 5]
  \definition{adj.}{prejudicial; nocivo; danoso; que pode causar danos ou prejuízos a algo}
\end{entry}

\begin{entry}{有劲儿}{you3 jin4er5}{6,7,2}{⽉、⼒、⼉}[HSK 4]
  \definition{adj.}{forte; enérgico; energético}
\end{entry}

\begin{entry}{有空儿}{you3 kong4r5}{6,8,2}{⽉、⽳、⼉}[HSK 2]
  \definition{v.}{ser livre}
\end{entry}

\begin{entry}{有力}{you3 li4}{6,2}{⽉、⼒}[HSK 5]
  \definition{adj.}{forte; vigoroso; poderoso; energético}
\end{entry}

\begin{entry}{有利}{you3li4}{6,7}{⽉、⼑}[HSK 3]
  \definition{adj.}{benéfico; favorável; vantajoso}
\end{entry}

\begin{entry}{有利于}{you3 li4 yu2}{6,7,3}{⽉、⼑、⼆}[HSK 5]
  \definition{prep.}{em favor de; fazer para; em benefício de; aproveitar}
\end{entry}

\begin{entry}{有名}{you3ming2}{6,6}{⽉、⼝}[HSK 1]
  \definition{adj.}{conhecido; famoso; célebre; nome conhecido por todos}
\end{entry}

\begin{entry}{有名无实}{you3ming2wu2shi2}{6,6,4,8}{⽉、⼝、⽆、⼧}
  \definition{v.}{(literal) tem um nome, mas não tem realidade | existe apenas no nome}
\end{entry}

\begin{entry}{有趣}{you3qu4}{6,15}{⽉、⾛}[HSK 4]
  \definition{adj.}{interessante; fascinante; divertido; excitante; estimulante}
\end{entry}

\begin{entry}{有人}{you3 ren2}{6,2}{⽉、⼈}[HSK 2]
  \definition{pron.}{qualquer um | alguém}
  \definition[所]{s.}{ocupado (como no banheiro) | pessoas}
\end{entry}

\begin{entry}{有时}{you3 shi2}{6,7}{⽉、⽇}[HSK 1]
  \definition{expr.}{às vezes; ocasionalmente; de vez em quando}
  \seealsoref{有的时候}{you3 de5 shi2 hou4}
  \seealsoref{有时候}{you3 shi2 hou5}
\end{entry}

\begin{entry}{有时候}{you3 shi2 hou5}{6,7,10}{⽉、⽇、⼈}[HSK 1]
  \definition{adv.}{às vezes; indica um momento incerto, mas não frequente}
  \seealsoref{有的时候}{you3 de5 shi2 hou4}
  \seealsoref{有时}{you3 shi2}
\end{entry}

\begin{entry}{有限}{you3 xian4}{6,8}{⽉、⾩}[HSK 4]
  \definition{adj.}{finito; limitado; restrito | indica baixo grau; indica pouco número; número baixo; nível baixo}
\end{entry}

\begin{entry}{有限公司}{you3xian4gong1si1}{6,8,4,5}{⽉、⾩、⼋、⼝}
  \definition{s.}{companhia limitada | corporação}
\end{entry}

\begin{entry}{有效}{you3 xiao4}{6,10}{⽉、⽁}[HSK 3]
  \definition{adj.}{válido; eficiente; eficaz}
\end{entry}

\begin{entry}{有些}{you3 xie1}{6,8}{⽉、⼆}[HSK 1]
  \definition{adv.}{um pouco; bastante; ligeiramente}
  \definition{pron.}{uma parte; alguns}
  \definition{v.}{usado para indicar que há alguns, mas não muitos;}
  \seealsoref{有(一)些}{you3 (yi4) xie1}
\end{entry}

\begin{entry}{有(一)点儿}{you3yi4dian3r5}{6,1,9,2}{⽉、⼀、⽕、⼉}[HSK 2]
  \definition{adv.}{um pouco (有点儿 + {s.} ou {v. mental})}
\end{entry}

\begin{entry}{有(一)些}{you3 (yi4) xie1}{6,1,8}{⽉、⼀、⼆}[HSK 1]
  \definition{adv.}{em vez disso; em vez de; de certa forma}
  \definition{pron.}{de certa forma}
  \seealsoref{有些}{you3 xie1}
\end{entry}

\begin{entry}{有意思}{you3 yi4 si5}{6,13,9}{⽉、⼼、⼼}[HSK 2]
  \definition{adj.}{interessante | agradável | significativo | divertido}
\end{entry}

\begin{entry}{有用}{you3yong4}{6,5}{⽉、⽤}[HSK 1]
  \definition{adj.}{útil; prático; funcional}
\end{entry}

\begin{entry}{有着}{you3 zhe5}{6,11}{⽉、⽬}[HSK 5]
  \definition{v.}{ter; possuir; haver; existir}
\end{entry}

\begin{entry}{又}{you4}{2}{⼜}[HSK 2][Kangxi 29]
  \definition{adv.}{mais uma vez | (usado para dar ênfase) de qualquer maneira | e ainda | e também}
\end{entry}

\begin{entry}{又称}{you4cheng1}{2,10}{⼜、⽲}
  \definition{s.}{também conhecido como}
\end{entry}

\begin{entry}{又及}{you4ji2}{2,3}{⼜、⼃}
  \definition{s.}{P.S., \emph{postscript}}
\end{entry}

\begin{entry}{又名}{you4ming2}{2,6}{⼜、⼝}
  \definition{s.}{também conhecido como | nome alternativo}
\end{entry}

\begin{entry}{又一次}{you4yi2ci4}{2,1,6}{⼜、⼀、⽋}
  \definition{adv.}{outra vez | mais uma vez | de novo}
\end{entry}

\begin{entry}{右}{you4}{5}{⼝}[HSK 1]
  \definition*{s.}{sobrenome You}
  \definition{adj.}{conservador; reacionário}
  \definition{s.}{a direita; o lado direito | oeste; na antiguidade, referia-se especificamente à direção oeste (com base na orientação para o sul) | o lado direito como o lado de precedência; posição ou nível mais elevado (os antigos costumavam considerar a direita como mais respeitável)}
  \definition{v.}{favorecer; apoiar; reverenciar}
\end{entry}

\begin{entry}{右边}{you4bian5}{5,5}{⼝、⾡}[HSK 1]
  \definition{s.}{a direita; o lado direito; do lado direito}
\end{entry}

\begin{entry}{右侧}{you4ce4}{5,8}{⼝、⼈}
  \definition{s.}{lateral direita | lado direito}
\end{entry}

\begin{entry}{右面}{you4mian4}{5,9}{⼝、⾯}
  \definition{s.}{lado direito}
\end{entry}

\begin{entry}{右倾}{you4qing1}{5,10}{⼝、⼈}
  \definition{adj.}{conservador | reacionário}
\end{entry}

\begin{entry}{右手}{you4shou3}{5,4}{⼝、⼿}
  \definition{s.}{mão direita | lado direito}
\end{entry}

\begin{entry}{右袒}{you4tan3}{5,10}{⼝、⾐}
  \definition{v.}{ser tendencioso | ser parcial | favorecer um lado | tomar partido}
\end{entry}

\begin{entry}{右转}{you4zhuan3}{5,8}{⼝、⾞}
  \definition{v.}{virar à direita}
\end{entry}

\begin{entry}{幼儿园}{you4'er2yuan2}{5,2,7}{⼳、⼉、⼞}[HSK 4]
  \definition{s.}{jardim de infância; escola maternal; escola infantil; instituição para a educação de crianças pequenas}
\end{entry}

\begin{entry}{诱人}{you4ren2}{9,2}{⾔、⼈}
  \definition{adj.}{atraente | cativante}
\end{entry}

\begin{entry}{淤泥}{yu1ni2}{11,8}{⽔、⽔}
  \definition{s.}{lodo}
\end{entry}

\begin{entry}{于}{yu2}{3}{⼆}
  \definition*{s.}{sobrenome Yu}
  \definition{prep.}{indica tempo, local, extensão, etc. | indica a direção da ação | usada após um verbo para indicar doação, entrega, etc. | relacionamento do objeto ou da entidade introduzida | indica o ponto inicial ou o ponto de partida | indica comparação}
\end{entry}

\begin{entry}{于是}{yu2shi4}{3,9}{⼆、⽇}[HSK 4]
  \definition{conj.}{então; portanto; consequentemente; como resultado; indica que o último segue o primeiro e que o último é frequentemente causado pelo primeiro}
\end{entry}

\begin{entry}{鱼}{yu2}{8}{⿂}[HSK 2][Kangxi 195]
  \definition*{s.}{sobrenome Yu}
  \definition[条,尾]{s.}{peixe}
\end{entry}

\begin{entry}{鱼船}{yu2chuan2}{8,11}{⿂、⾈}
  \definition{s.}{barco de pesca}
  \seealsoref{渔船}{yu2chuan2}
\end{entry}

\begin{entry}{鱼具}{yu2ju4}{8,8}{⿂、⼋}
  \variantof{渔具}
\end{entry}

\begin{entry}{鱼片}{yu2pian4}{8,4}{⿂、⽚}
  \definition{s.}{fatia de peixe | filé de peixe}
\end{entry}

\begin{entry}{鱼网}{yu2wang3}{8,6}{⿂、⽹}
  \variantof{渔网}
\end{entry}

\begin{entry}{鱼香}{yu2xiang1}{8,9}{⿂、⾹}
  \definition{s.}{um tempero da culinária chinesa que normalmente contém alho, cebolinha, gengibre, açúcar, sal, pimenta, etc. (Embora 鱼香 signifique literalmente ``fragrância de peixe'', não contém frutos do mar.)}
\end{entry}

\begin{entry}{鱼香肉丝}{yu2xiang1rou4si1}{8,9,6,5}{⿂、⾹、⾁、⼀}
  \definition{s.}{tiras de carne de porco salteadas com molho picante (prato)}
  \seealsoref{鱼香}{yu2xiang1}
\end{entry}

\begin{entry}{鱼汛}{yu2xun4}{8,6}{⿂、⽔}
  \variantof{渔汛}
\end{entry}

\begin{entry}{舁}{yu2}{9}{⾅}
  \definition{v.}{levantar; elevar | aumentar}
\end{entry}

\begin{entry}{渔}{yu2}{11}{⽔}
  \definition[条]{s.}{pescador}
  \definition{v.}{pescar}
\end{entry}

\begin{entry}{渔场}{yu2chang3}{11,6}{⽔、⼟}
  \definition{s.}{área de pesca}
\end{entry}

\begin{entry}{渔船}{yu2chuan2}{11,11}{⽔、⾈}
  \definition[条]{s.}{barco de pesca}
  \seealsoref{鱼船}{yu2chuan2}
\end{entry}

\begin{entry}{渔船队}{yu2chuan2dui4}{11,11,4}{⽔、⾈、⾩}
  \definition{s.}{frota pesqueira}
\end{entry}

\begin{entry}{渔夫}{yu2fu1}{11,4}{⽔、⼤}
  \definition{s.}{pescador}
\end{entry}

\begin{entry}{渔具}{yu2ju4}{11,8}{⽔、⼋}
  \definition{s.}{equipamento de pesca}
\end{entry}

\begin{entry}{渔捞}{yu2lao1}{11,10}{⽔、⼿}
  \definition{s.}{pesca (como atividade comercial)}
\end{entry}

\begin{entry}{渔猎}{yu2lie4}{11,11}{⽔、⽝}
  \definition{s.}{pesca e caça}
  \definition{v.}{saquear | pilhar}
\end{entry}

\begin{entry}{渔笼}{yu2long2}{11,11}{⽔、⽵}
  \definition{s.}{gaiola de pesca | armadilha de pesca}
\end{entry}

\begin{entry}{渔轮}{yu2lun2}{11,8}{⽔、⾞}
  \definition{s.}{navio de pesca}
\end{entry}

\begin{entry}{渔民}{yu2min2}{11,5}{⽔、⽒}
  \definition{s.}{pescadores | povo pescador}
\end{entry}

\begin{entry}{渔网}{yu2wang3}{11,6}{⽔、⽹}
  \definition{s.}{rede de pesca}
\end{entry}

\begin{entry}{渔汛}{yu2xun4}{11,6}{⽔、⽔}
  \definition{s.}{temporada de pesca}
\end{entry}

\begin{entry}{愉快}{yu2kuai4}{12,7}{⼼、⼼}
  \definition{adj.}{alegre | delicioso | prazeroso | agradável | feliz | encantado}
  \definition{adv.}{alegremente | agradavelmente}
\end{entry}

\begin{entry}{瑜伽}{yu2jia1}{13,7}{⽟、⼈}
  \definition*{s.}{Ioga}
\end{entry}

\begin{entry}{瑜珈}{yu2jia1}{13,9}{⽟、⽟}
  \variantof{瑜伽}
\end{entry}

\begin{entry}{与}{yu3}{3}{⼀}
  \definition{conj.}{e, com}
  \seeref{与}{yu4}
\end{entry}

\begin{entry}{与其}{yu3qi2}{3,8}{⼀、⼋}
  \definition{conj.}{mais do que}
\end{entry}

\begin{entry}{与其……不如……}{yu3qi2 bu4ru2}{3,8,4,6}{⼀、⼋、⼀、⼥}
  \definition{conj.}{ao invés de\dots melhor que\dots}
\end{entry}

\begin{entry}{与其……宁可……}{yu3qi2 ning4ke3}{3,8,5,5}{⼀、⼋、⼧、⼝}
  \definition{conj.}{ao invés de\dots melhor que\dots}
\end{entry}

\begin{entry}{宇航员}{yu3hang2yuan2}{6,10,7}{⼧、⾈、⼝}
  \definition{s.}{astronauta}
\end{entry}

\begin{entry}{宇宙}{yu3zhou4}{6,8}{⼧、⼧}
  \definition{s.}{universo | cosmos}
\end{entry}

\begin{entry}{羽冠}{yu3guan1}{6,9}{⽻、⼍}
  \definition{s.}{crista emplumada (de pássaro)}
\end{entry}

\begin{entry}{羽林}{yu3lin2}{6,8}{⽻、⽊}
  \definition{s.}{escolta armada}
\end{entry}

\begin{entry}{羽流}{yu3liu2}{6,10}{⽻、⽔}
  \definition{s.}{pluma}
\end{entry}

\begin{entry}{羽毛}{yu3mao2}{6,4}{⽻、⽑}
  \definition{s.}{pena | plumagem | pluma}
\end{entry}

\begin{entry}{羽毛笔}{yu3mao2bi3}{6,4,10}{⽻、⽑、⽵}
  \definition{s.}{caneta de pena}
\end{entry}

\begin{entry}{羽毛球}{yu3mao2qiu2}{6,4,11}{⽻、⽑、⽟}[HSK 5]
  \definition[只]{s.}{\emph{badminton}; esporte com bola, as regras e equipamentos são bastante semelhantes ao tênis | peteca}
\end{entry}

\begin{entry}{羽绒服}{yu3rong2fu2}{6,9,8}{⽻、⽷、⽉}[HSK 5]
  \definition[件]{s.}{jaqueta de plumas; peça de vestuário com enchimento de plumas; casaco cujo interior é preenchido com penas de pato ou ganso}
\end{entry}

\begin{entry}{雨}{yu3}{8}{⾬}[HSK 1][Kangxi 173]
  \definition*{s.}{sobrenome Yu}
  \definition[场,阵,滴]{s.}{chuva; água que cai das nuvens para o solo}
  \seeref{雨}{yu4}
\end{entry}

\begin{entry}{雨伞}{yu3san3}{8,6}{⾬、⼈}
  \definition[把]{s.}{guarda-chuva}
\end{entry}

\begin{entry}{雨蚀}{yu3shi2}{8,9}{⾬、⾷}
  \definition{s.}{erosão da chuva}
\end{entry}

\begin{entry}{雨水}{yu3 shui3}{8,4}{⾬、⽔}[HSK 5]
  \definition{s.}{água da chuva; precipitação; chuva; água proveniente da chuva}
\end{entry}

\begin{entry}{雨靴}{yu3xue1}{8,13}{⾬、⾰}
  \definition[双]{s.}{botas de chuva}
\end{entry}

\begin{entry}{雨衣}{yu3yi1}{8,6}{⾬、⾐}
  \definition[件]{s.}{impermeável}
\end{entry}

\begin{entry}{语}{yu3}{9}{⾔}
  \definition{s.}{língua; linguagem | dito; provérbio; refere-se especialmente a coloquialismos, provérbios, expressões idiomáticas ou palavras de livros antigos | sinal; meio não linguístico de comunicar ideias ; ações ou sinais que substituem palavras para expressar significado | palavras; expressão; refere-se a uma palavra, frase ou sentença}
  \definition{v.}{dizer; falar | (de pássaros, insetos, etc.) gorjear; pipilar}
  \seeref{语}{yu4}
\end{entry}

\begin{entry}{语调}{yu3diao4}{9,10}{⾔、⾔}
  \definition[个]{s.}{entonação}
\end{entry}

\begin{entry}{语法}{yu3fa3}{9,8}{⾔、⽔}[HSK 4]
  \definition[个]{s.}{gramática; maneira como o idioma é estruturado, incluindo a formação e as variações de palavras, a organização de frases e sentenças | estudo da gramática; estudo das regras de estrutura linguística}
\end{entry}

\begin{entry}{语法术语}{yu3fa3shu4yu3}{9,8,5,9}{⾔、⽔、⽊、⾔}
  \definition{s.}{termo gramatical}
\end{entry}

\begin{entry}{语气}{yu3qi4}{9,4}{⾔、⽓}
  \definition[个]{s.}{maneira de falar | tom}
\end{entry}

\begin{entry}{语言}{yu3yan2}{9,7}{⾔、⾔}[HSK 2]
  \definition[门,种]{s.}{linguagem | língua}
\end{entry}

\begin{entry}{语言实验室}{yu3yan2shi2yan4shi4}{9,7,8,10,9}{⾔、⾔、⼧、⾺、⼧}
  \definition{s.}{laboratório de línguas}
\end{entry}

\begin{entry}{语音}{yu3 yin1}{9,9}{⾔、⾳}[HSK 4]
  \definition{s.}{voz; pronúncia; sons da fala; som de alguém falando | pronúncia; som do idioma}
\end{entry}

\begin{entry}{与}{yu4}{3}{⼀}
  \definition{v.}{fazer parte de}
  \seeref{与}{yu3}
\end{entry}

\begin{entry}{玉}{yu4}{5}{⽟}[HSK 4][Kangxi 96]
  \definition*{s.}{sobrenome Yu}
  \definition{adj.}{(pessoa, especialmente uma mulher) pura; justa; bonita; bela | cristalino, branco e belo como o jade | (vida) rica; luxuosa}
  \definition{pron.}{seu; um termo de respeito, usado para honrar o corpo, as ações ou as coisas associadas à outra pessoa}
  \definition[块]{s.}{jade}
\end{entry}

\begin{entry}{玉米}{yu4mi3}{5,6}{⽟、⽶}[HSK 4]
  \definition[个,株,粒]{s.}{milho}
\end{entry}

\begin{entry}{玉米饼}{yu4mi3bing3}{5,6,9}{⽟、⽶、⾷}
  \definition{s.}{tortilha mexicana | bolo de milho}
\end{entry}

\begin{entry}{玉米粉}{yu4mi3fen3}{5,6,10}{⽟、⽶、⽶}
  \definition{s.}{amido de milho | farinha de milho}
\end{entry}

\begin{entry}{玉米糕}{yu4mi3gao1}{5,6,16}{⽟、⽶、⽶}
  \definition{s.}{bolo de milho | polenta}
\end{entry}

\begin{entry}{玉米花}{yu4mi3hua1}{5,6,7}{⽟、⽶、⾋}
  \definition{s.}{pipoca}
\end{entry}

\begin{entry}{玉米面}{yu4mi3mian4}{5,6,9}{⽟、⽶、⾯}
  \definition{s.}{fubá | farinha de milho}
\end{entry}

\begin{entry}{玉米片}{yu4mi3pian4}{5,6,4}{⽟、⽶、⽚}
  \definition{s.}{flocos de milho | chips de tortilha}
\end{entry}

\begin{entry}{玉米糁}{yu4mi3san3}{5,6,14}{⽟、⽶、⽶}
  \definition{s.}{grãos de milho}
\end{entry}

\begin{entry}{玉米笋}{yu4mi3sun3}{5,6,10}{⽟、⽶、⽵}
  \definition{s.}{broto de milho}
\end{entry}

\begin{entry}{芋头}{yu4tou5}{6,5}{⾋、⼤}
  \definition{s.}{taro, similar ao inhame e batata doce}
\end{entry}

\begin{entry}{芋头色}{yu4tou5se4}{6,5,6}{⾋、⼤、⾊}
  \definition{s.}{cor lilás}
\end{entry}

\begin{entry}{郁郁葱葱}{yu4yu4cong1cong1}{8,8,12,12}{⾢、⾢、⾋、⾋}
  \definition{expr.}{verdejante e exuberante}
\end{entry}

\begin{entry}{雨}{yu4}{8}{⾬}
  \definition{v.}{cair (chuva, neve, etc.) | precipitar | chover | molhar}
  \seeref{雨}{yu3}
\end{entry}

\begin{entry}{语}{yu4}{9}{⾔}
  \definition{v.}{contar; informar}
  \seeref{语}{yu3}
\end{entry}

\begin{entry}{预}{yu4}{10}{⾴}
  \definition{adv.}{antecipadamente}
  \definition{v.}{avançar | preparar}
\end{entry}

\begin{entry}{预报}{yu4bao4}{10,7}{⾴、⼿}[HSK 3]
  \definition[个]{s.}{boletim meteorológico; previsões meteorológicas antecipadas}
  \definition{v.}{prever (o tempo); relato de coisas antes que elas aconteçam, usado principalmente em clima, astronomia, desastres naturais, etc.}
\end{entry}

\begin{entry}{预备}{yu4 bei4}{10,8}{⾴、⼡}[HSK 5]
  \definition{v.}{preparar-se; ficar pronto}
\end{entry}

\begin{entry}{预测}{yu4 ce4}{10,9}{⾴、⽔}[HSK 4]
  \definition{v.}{prever; prognosticar; predizer}
\end{entry}

\begin{entry}{预订}{yu4ding4}{10,4}{⾴、⾔}[HSK 4]
  \definition{v.}{reservar; fazer uma reserva}
\end{entry}

\begin{entry}{预定}{yu4ding4}{10,8}{⾴、⼧}
  \definition{v.}{agendar com antecedência}
\end{entry}

\begin{entry}{预防}{yu4fang2}{10,6}{⾴、⾩}[HSK 3]
  \definition{v.}{prevenir; proteger-se contra; tomar precauções contra}
\end{entry}

\begin{entry}{预付}{yu4fu4}{10,5}{⾴、⼈}
  \definition{s.}{pré-pago}
  \definition{v.}{pagar antecipadamente}
\end{entry}

\begin{entry}{预感}{yu4gan3}{10,13}{⾴、⼼}
  \definition{s.}{premonição}
  \definition{v.}{ter uma premonição}
\end{entry}

\begin{entry}{预购}{yu4gou4}{10,8}{⾴、⾙}
  \definition{s.}{compra antecipada}
  \definition{v.}{comprar antecipadamente}
\end{entry}

\begin{entry}{预计}{yu4 ji4}{10,4}{⾴、⾔}[HSK 3]
  \definition{v.}{estimar; esperar; calcular com antecedência}
\end{entry}

\begin{entry}{预见}{yu4jian4}{10,4}{⾴、⾒}
  \definition{s.}{previsão; intuição; vislumbre}
  \definition{v.}{prever}
\end{entry}

\begin{entry}{预警}{yu4jing3}{10,19}{⾴、⾔}
  \definition{s.}{aviso | aviso antecipado}
\end{entry}

\begin{entry}{预览}{yu4lan3}{10,9}{⾴、⾒}
  \definition{s.}{visualização}
  \definition{v.}{visualizar}
\end{entry}

\begin{entry}{预留}{yu4liu2}{10,10}{⾴、⽥}
  \definition{v.}{separar | reservar}
\end{entry}

\begin{entry}{预谋}{yu4mou2}{10,11}{⾴、⾔}
  \definition{adj.}{premeditado}
  \definition{v.}{planejar algo com antecedência (especialmente um crime)}
\end{entry}

\begin{entry}{预判}{yu4pan4}{10,7}{⾴、⼑}
  \definition{v.}{prever | antecipar}
\end{entry}

\begin{entry}{预配}{yu4pei4}{10,10}{⾴、⾣}
  \definition{s.}{pré-alocado | pré-cabeado}
  \definition{v.}{pré-alocar | pré-cabear}
\end{entry}

\begin{entry}{预期}{yu4qi1}{10,12}{⾴、⽉}[HSK 5]
  \definition{v.}{esperar; antecipar; imaginar; antecipar com expectativa}
\end{entry}

\begin{entry}{预提}{yu4ti2}{10,12}{⾴、⼿}
  \definition{s.}{retenção}
  \definition{v.}{reter (imposto)}
\end{entry}

\begin{entry}{预习}{yu4xi2}{10,3}{⾴、⼄}[HSK 3]
  \definition{v.}{pré-visualizar; preparar uma lição; estudar com antecedência as aulas que irá assistir}
\end{entry}

\begin{entry}{预约}{yu4yue1}{10,6}{⾴、⽷}
  \definition{s.}{reserva}
  \definition{v.}{agendar | marcar compromisso}
\end{entry}

\begin{entry}{预祝}{yu4zhu4}{10,9}{⾴、⽰}
  \definition{v.}{parabenizar de antemão | oferecer os melhores votos para}
\end{entry}

\begin{entry}{欲}{yu4}{11}{⽋}
  \definition{adj.}{desejo | apetite | paixão | luxúria | ganância}
  \definition{v.}{desejar}
\end{entry}

\begin{entry}{喻}{yu4}{12}{⼝}
  \definition{s.}{analogia | símile | metáfora | alegoria}
  \definition{v.}{descrever algo como}
\end{entry}

\begin{entry}{寓意}{yu4yi4}{12,13}{⼧、⼼}
  \definition{s.}{moral (de uma história),  lição a ser aprendida, implicação, mensagem, significado metafórico}
\end{entry}

\begin{entry}{遇}{yu4}{12}{⾡}[HSK 4]
  \definition*{s.}{sobrenome Yu}
  \definition{s.}{chance; oportunidade}
  \definition{v.}{encontrar; deparar-se com; encontrar-se | tratar; receber}
\end{entry}

\begin{entry}{遇到}{yu4dao4}{12,8}{⾡、⼑}[HSK 4]
  \definition{v.}{esbarrar em; encontrar; deparar-se com; conhecer alguém ou algo (inesperado)}
\end{entry}

\begin{entry}{遇见}{yu4 jian4}{12,4}{⾡、⾒}[HSK 4]
  \definition{v.}{encontrar; deparar-se com}
\end{entry}

\begin{entry}{愈}{yu4}{13}{⼼}
  \definition{adv.}{mais e mais | ainda mais}
  \definition{v.}{recuperar | curar}
\end{entry}

\begin{entry}{豫}{yu4}{15}{⾗}
  \definition{adj.}{feliz despreocupado | à vontade}
  \seealsoref{预}{yu4}
\end{entry}

\begin{entry}{元}{yuan2}{4}{⼉}[HSK 1]
  \definition*{s.}{sobrenome Yuan}
  \definition*{s.}{Dinastia Yuan (1271-1368)}
  \definition{adj.}{primeiro; principal; primário | chefe; diretor; líder | básico; fundamental; principal | unidade; componente; formando um todo}
  \definition{clas.}{yuan, a unidade monetária da China}
  \definition{s.}{moeda com valor e peso fixos | origem; elemento}
\end{entry}

\begin{entry}{元旦}{yuan2dan4}{4,5}{⼉、⽇}[HSK 5]
  \definition*{s.}{Dia de Ano Novo (1 de janeiro)}
\end{entry}

\begin{entry}{元来}{yuan2lai2}{4,7}{⼉、⽊}
  \variantof{原来}
\end{entry}

\begin{entry}{元气}{yuan2qi4}{4,4}{⼉、⽓}
  \definition{s.}{força | vigor | vitalidade | energial vital}
\end{entry}

\begin{entry}{元宵}{yuan2xiao1}{4,10}{⼉、⼧}
  \definition*{s.}{Festival das Lanternas}
  \seealsoref{元宵节}{yuan2xiao1jie2}
  \seealsoref{元夜}{yuan2ye4}
\end{entry}

\begin{entry}{元宵节}{yuan2xiao1jie2}{4,10,5}{⼉、⼧、⾋}
  \definition*{s.}{Festival das Lanternas (15º~dia do primeiro mês lunar)}
  \seealsoref{元宵}{yuan2xiao1}
  \seealsoref{元夜}{yuan2ye4}
\end{entry}

\begin{entry}{元夜}{yuan2ye4}{4,8}{⼉、⼣}
  \definition*{s.}{Festival das Lanternas}
  \seealsoref{元宵}{yuan2xiao1}
  \seealsoref{元宵节}{yuan2xiao1jie2}
\end{entry}

\begin{entry}{员}{yuan2}{7}{⼝}[HSK 3]
  \definition{clas.}{para comandantes militares}
  \definition{s.}{uma pessoa envolvida em algum campo de atividade; refere-se a pessoas que trabalham ou estudam | membro; refere-se aos membros de um grupo ou organização}
\end{entry}

\begin{entry}{员工}{yuan2gong1}{7,3}{⼝、⼯}[HSK 3]
  \definition[位,名,个]{s.}{funcionário; atendente; balconista; empregado; trabalhador; pessoal}
\end{entry}

\begin{entry}{园林}{yuan2lin2}{7,8}{⼞、⽊}[HSK 5]
  \definition{s.}{parque; jardim; área paisagística com plantas e árvores para as pessoas apreciarem e descansarem.}
\end{entry}

\begin{entry}{原来}{yuan2lai2}{10,7}{⼚、⽊}[HSK 2]
  \definition{adv.}{originalmente | como se vê | na verdade}
  \definition{v.}{vir a ser}
\end{entry}

\begin{entry}{原理}{yuan2li3}{10,11}{⼚、⽟}[HSK 5]
  \definition{s.}{princípio; axioma; teoria; teoria básica ou princípio científico de significado universal}
\end{entry}

\begin{entry}{原料}{yuan2liao4}{10,10}{⼚、⽃}[HSK 4]
  \definition[种,个]{s.}{matéria-prima; refere-se a materiais que não foram processados e fabricados, como minérios para metalurgia e algodão para têxteis}
\end{entry}

\begin{entry}{原木}{yuan2mu4}{10,4}{⼚、⽊}
  \definition{s.}{registro | \emph{logs}}
\end{entry}

\begin{entry}{原色}{yuan2 se4}{10,6}{⼚、⾊}
  \definition{s.}{cor primária}
\end{entry}

\begin{entry}{原始}{yuan2shi3}{10,8}{⼚、⼥}[HSK 5]
  \definition{s.}{original; de primeira mão | primitivo; mais antigo; não desenvolvido; não civilizado}
\end{entry}

\begin{entry}{原先}{yuan2xian1}{10,6}{⼚、⼉}[HSK 5]
  \definition{adj.}{antigo; original}
  \definition{s.}{antigamente; no início}
\end{entry}

\begin{entry}{原因}{yuan2yin1}{10,6}{⼚、⼞}[HSK 2]
  \definition[个]{s.}{causa | razão | motivo}
\end{entry}

\begin{entry}{原有}{yuan2 you3}{10,6}{⼚、⽉}[HSK 5]
  \definition{v.}{já estar pronto, não é necessário fazer ou procurar nada; ser o original}
\end{entry}

\begin{entry}{原则}{yuan2ze2}{10,6}{⼚、⼑}[HSK 4]
  \definition{adv.}{em geral; em princípio; refere-se a um aspecto geral; geralmente}
  \definition[个]{s.}{princípios; leis ou padrões pelos quais alguém fala ou age}
\end{entry}

\begin{entry}{圆}{yuan2}{10}{⼞}[HSK 4]
  \definition*{s.}{sobrenome Yuan}
  \definition{adj.}{redondo; circular; esférico; arredondado | diplomático; satisfatório}
  \definition[个]{s.}{círculo; circunferência | uma moeda de valor e peso fixos}
  \definition{v.}{tornar plausível; justificar; tornar completo; completar}
\end{entry}

\begin{entry}{圆满}{yuan2man3}{10,13}{⼞、⽔}[HSK 4]
  \definition{adj.}{perfeito; satisfatório; sem defeitos}
\end{entry}

\begin{entry}{援助}{yuan2zhu4}{12,7}{⼿、⼒}
  \definition{s.}{assistência}
  \definition{v.}{ajudar | apoiar | assistir}
\end{entry}

\begin{entry}{缘}{yuan2}{12}{⽷}
  \definition{s.}{causa | razão | karma | destino | predestinação}
\end{entry}

\begin{entry}{缘分}{yuan2fen4}{12,4}{⽷、⼑}
  \definition{s.}{destino ou acaso que une as pessoas | afinidade ou relacionamento predestinado | destino (Budismo)}
\end{entry}

\begin{entry}{源}{yuan2}{13}{⽔}
  \definition*{s.}{sobrenome Yuan}
  \definition{s.}{nascente (de um rio); fonte | fonte; origem; causa}
  \definition{v.}{originar-se; provir de}
\end{entry}

\begin{entry}{远}{yuan3}{7}{⾡}[HSK 1]
  \definition*{s.}{sobrenome Yuan}
  \definition{adj.}{distante (no tempo ou no espaço); longe; remoto; Longa distância espacial ou temporal (em oposição a 近) | (relações de parentesco) distante | com grande diferença}
  \definition{v.}{manter-se afastado de; não se aproximar}
  \seealsoref{近}{jin4}
\end{entry}

\begin{entry}{远处}{yuan3 chu4}{7,5}{⾡、⼡}[HSK 5]
  \definition{s.}{distância; lugar distante}
\end{entry}

\begin{entry}{远方}{yuan3fang1}{7,4}{⾡、⽅}
  \definition{s.}{longe | um local distante}
\end{entry}

\begin{entry}{远天}{yuan3tian1}{7,4}{⾡、⼤}
  \definition{s.}{paraíso | o céu distante}
\end{entry}

\begin{entry}{远远}{yuan3yuan3}{7,7}{⾡、⾡}
  \definition{adv.}{de longe}
\end{entry}

\begin{entry}{远征}{yuan3zheng1}{7,8}{⾡、⼻}
  \definition{s.}{uma expedição militar | marcha para regiões remotas}
\end{entry}

\begin{entry}{怨}{yuan4}{9}{⼼}[HSK 5]
  \definition{s.}{ressentimento; inimizade; rancor}
  \definition{v.}{culpar; reclamar}
\end{entry}

\begin{entry}{院}{yuan4}{9}{⾩}[HSK 2]
  \definition[个]{s.}{pátio | instituição}
\end{entry}

\begin{entry}{院长}{yuan4zhang3}{9,4}{⾩、⾧}[HSK 2]
  \definition[个]{s.}{presidente de um conselho | reitor | chefe de departamento | primeiro-ministro da República da China | presidente de uma universidade}
\end{entry}

\begin{entry}{院子}{yuan4zi5}{9,3}{⾩、⼦}[HSK 2]
  \definition[个]{s.}{pátio | jardim | quintal}
\end{entry}

\begin{entry}{愿}{yuan4}{14}{⽕}[HSK 5]
  \definition{adj.}{honesto e prudente}
  \definition{s.}{esperança; desejo; vontade; a ideia de alcançar algum objetivo no futuro | voto (feito perante o Buda ou um deus); o desejo de retribuição feito ao rezar para os deuses e Buda}
  \definition{v.}{estar disposto; estar pronto; de bom grado, concordar porque está de acordo com seus desejos | ter esperança; desejar; qerer alcançar algum desejo}
\end{entry}

\begin{entry}{愿望}{yuan4wang4}{14,11}{⽕、⽉}[HSK 3]
  \definition[个]{s.}{desejo; aspiração; a ideia de esperar atingir um determinado objetivo no futuro}
\end{entry}

\begin{entry}{愿意}{yuan4yi4}{14,13}{⽕、⼼}[HSK 2]
  \definition{s.}{desejo | esperança}
  \definition{v.}{estar disposto | estar pronto}
\end{entry}

\begin{entry}{约}{yue1}{6}{⽷}[HSK 3]
  \definition*{s.}{sobrenome Yue}
  \definition{adj.}{econômico; frugal | simples; breve | indistinto}
  \definition{adv.}{cerca de; ao redor; aproximadamente}
  \definition{s.}{pacto; acordo; nomeação; coisa prometida}
  \definition{v.}{marcar uma consulta; organizar | perguntar ou convidar com antecedência | restringir; conter | reduzir (fração aproximada)}
  \seeref{约}{yao1}
\end{entry}

\begin{entry}{约会}{yue1hui4}{6,6}{⽷、⼈}[HSK 4]
  \definition[个,次]{s.}{data; compromisso; engajamento; reunião pré-agendada}
  \definition{v.}{marcar uma reunião; marcar um encontro;}
\end{entry}

\begin{entry}{约束}{yue1shu4}{6,7}{⽷、⽊}[HSK 5]
  \definition{adj.}{amarrado}
  \definition{s.}{restrição; constrangimento; engajamento}
  \definition{v.}{amarrar; prender; reprimir; restringir; manter dentro de si}
\end{entry}

\begin{entry}{月}{yue4}{4}{⽉}[HSK 1][Kangxi 74]
  \definition*{s.}{sobrenome Yue}
  \definition[个]{s.}{mês | por mês | lua | redondo; em forma de lua cheia}
\end{entry}

\begin{entry}{月饼}{yue4 bing3}{4,9}{⽉、⾷}[HSK 5]
  \definition[个,块,盒]{s.}{bolinho da lua; comida típica do Festival do Meio Outono; redonda e recheada; simboliza a reunião familiar}
\end{entry}

\begin{entry}{月底}{yue4 di3}{4,8}{⽉、⼴}[HSK 4]
  \definition[个]{s.}{final do mês; últimos dias do mês}
\end{entry}

\begin{entry}{月份}{yue4 fen4}{4,6}{⽉、⼈}[HSK 2]
  \definition{s.}{mês}
\end{entry}

\begin{entry}{月径}{yue4jing4}{4,8}{⽉、⼻}
  \definition{s.}{diâmetro da lua | diâmetro da órbita da lua | caminho iluminado pela lua}
\end{entry}

\begin{entry}{月亮}{yue4liang5}{4,9}{⽉、⼇}[HSK 2]
  \definition{s.}{lua}
\end{entry}

\begin{entry}{月球}{yue4 qiu2}{4,11}{⽉、⽟}[HSK 5]
  \definition*{s.}{Lua}
  \definition{pref.}{seleno-; seleni-}
\end{entry}

\begin{entry}{月壤}{yue4rang3}{4,20}{⽉、⼟}
  \definition{s.}{solo lunar}
\end{entry}

\begin{entry}{月相}{yue4xiang4}{4,9}{⽉、⽬}
  \definition{s.}{fases da lua, a saber: lua nova 朔, lua crescente 上弦, lua cheia 望 e lua minguante 下弦}
\end{entry}

\begin{entry}{月月}{yue4yue4}{4,4}{⽉、⽉}
  \definition{adv.}{todo mês}
\end{entry}

\begin{entry}{乐}{yue4}{5}{⼃}
  \definition*{s.}{sobrenome Yue}
  \definition{s.}{música}
  \seeref{乐}{le4}
\end{entry}

\begin{entry}{乐队}{yue4 dui4}{5,4}{⼃、⾩}[HSK 3]
  \definition[支]{s.}{orquestra; banda}
\end{entry}

\begin{entry}{阅兵式}{yue4bing1shi4}{10,7,6}{⾨、⼋、⼷}
  \definition{s.}{parada militar}
\end{entry}

\begin{entry}{阅读}{yue4du2}{10,10}{⾨、⾔}[HSK 4]
  \definition{s.}{leitura}
  \definition{v.}{ler; examinar; olhar (livros, jornais, etc.) e entender seu conteúdo}
\end{entry}

\begin{entry}{阅读广度}{yue4du2guang3du4}{10,10,3,9}{⾨、⾔、⼴、⼴}
  \definition{s.}{intervalo de leitura}
\end{entry}

\begin{entry}{阅读理解}{yue4du2li3jie3}{10,10,11,13}{⾨、⾔、⽟、⾓}
  \definition{s.}{compreensão de leitura}
\end{entry}

\begin{entry}{阅读器}{yue4du2qi4}{10,10,16}{⾨、⾔、⼝}
  \definition{s.}{leitor (\emph{software})}
\end{entry}

\begin{entry}{阅读时间}{yue4du2shi2jian1}{10,10,7,7}{⾨、⾔、⽇、⾨}
  \definition{s.}{tempo de leitura}
\end{entry}

\begin{entry}{阅读障碍}{yue4du2zhang4ai4}{10,10,13,13}{⾨、⾔、⾩、⽯}
  \definition{s.}{dislexia}
\end{entry}

\begin{entry}{阅读装置}{yue4du2zhuang1zhi4}{10,10,12,13}{⾨、⾔、⾐、⽹}
  \definition{s.}{dispositivo de leitura (por exemplo, para códigos de barras, etiquetas RFID, etc.)}
\end{entry}

\begin{entry}{阅览室}{yue4 lan3 shi4}{10,9,9}{⾨、⾒、⼧}[HSK 5]
  \definition[间]{s.}{sala de leitura; a biblioteca dispõe de salas para leitura e pesquisa, equipadas com mesas e cadeiras adequadas, livros, jornais, revistas, etc.}
\end{entry}

\begin{entry}{粤语}{yue4yu3}{12,9}{⾔、⾔}
  \definition{s.}{cantonês | língua cantonesa}
\end{entry}

\begin{entry}{越}{yue4}{12}{⾛}[HSK 2]
  \definition{adv.}{quanto mais\dots mais}
  \definition{v.}{subir | exceder | superar}
\end{entry}

\begin{entry}{越境}{yue4jing4}{12,14}{⾛、⼟}
  \definition{v.}{cruzar uma fronteira (geralmente ilegalmente) | entrar ou sair furtivamente de um país}
\end{entry}

\begin{entry}{越来越……}{yue4lai2yue4}{12,7,12}{⾛、⽊、⾛}[HSK 2]
  \definition{adv.}{cada vez mais\dots}
\end{entry}

\begin{entry}{越……越……}{yue4 yue4}{12,12}{⾛、⾛}[HSK 2]
  \definition{expr.}{quanto mais\dots tanto mais\dots}
\end{entry}

\begin{entry}{越障}{yue4zhang4}{12,13}{⾛、⾩}
  \definition{s.}{curso com obstáculos para treinamento de tropas}
  \definition{v.}{superar obstáculos}
\end{entry}

\begin{entry}{龠}{yue4}{17}{⿕}[Kangxi 214]
  \definition{clas.}{yue, uma unidade de medida seca para grãos (= 0,5 decilitro);}
  \definition{s.}{uma flauta curta antiga}
\end{entry}

\begin{entry}{云}{yun2}{4}{⼆}[HSK 2]
  \definition*{s.}{sobrenome Yun}
  \definition[朵]{s.}{nuvem}
\end{entry}

\begin{entry}{云端}{yun2duan1}{4,14}{⼆、⽴}
  \definition{s.}{alto nas nuvens | (computação) a nuvem}
\end{entry}

\begin{entry}{云南}{yun2nan2}{4,9}{⼆、⼗}
  \definition*{s.}{Yunnan}
\end{entry}

\begin{entry}{云云}{yun2yun2}{4,4}{⼆、⼆}
  \definition{adv.}{e assim por diante | assim e assim}
\end{entry}

\begin{entry}{运}{yun4}{7}{⾡}[HSK 5]
  \definition*{s.}{sobrenome Yun}
  \definition{s.}{sorte; destino; fortuna}
  \definition{v.}{mover; deslocar | transportar; levar | usar; empunhar; utilizar}
\end{entry}

\begin{entry}{运动}{yun4dong4}{7,6}{⾡、⼒}[HSK 2]
  \definition[场]{s.}{esporte | desporto}
  \definition{v.}{exercitar | mover-se}
\end{entry}

\begin{entry}{运动病}{yun4dong4bing4}{7,6,10}{⾡、⼒、⽧}
  \definition{s.}{enjôo (movimento, carro, etc.)}
\end{entry}

\begin{entry}{运动场}{yun4dong4chang3}{7,6,6}{⾡、⼒、⼟}
  \definition{s.}{campo desportivo | campo de jogos}
\end{entry}

\begin{entry}{运动服}{yun4dong4fu2}{7,6,8}{⾡、⼒、⽉}
  \definition{s.}{roupa para prática de esporte}
\end{entry}

\begin{entry}{运动会}{yun4 dong4 hui4}{7,6,6}{⾡、⼒、⼈}[HSK 4]
  \definition[个]{s.}{jogos; encontro esportivo; dia de esportes; reunião atlética}
\end{entry}

\begin{entry}{运动家}{yun4dong4jia1}{7,6,10}{⾡、⼒、⼧}
  \definition{s.}{ativista | atleta | esportista}
\end{entry}

\begin{entry}{运动衫}{yun4dong4shan1}{7,6,8}{⾡、⼒、⾐}
  \definition[件]{s.}{moletom | camisa esportiva}
\end{entry}

\begin{entry}{运动鞋}{yun4dong4xie2}{7,6,15}{⾡、⼒、⾰}
  \definition{s.}{tênis | sapatos esportivos}
\end{entry}

\begin{entry}{运动学}{yun4dong4xue2}{7,6,8}{⾡、⼒、⼦}
  \definition{s.}{cinemática}
\end{entry}

\begin{entry}{运动员}{yun4 dong4 yuan2}{7,6,7}{⾡、⼒、⼝}[HSK 4]
  \definition[名,个]{s.}{jogador; atleta; esportista; pessoas que participam de competições esportivas}
\end{entry}

\begin{entry}{运河}{yun4he2}{7,8}{⾡、⽔}
  \definition{s.}{canal (em um rio)}
\end{entry}

\begin{entry}{运气}{yun4qi5}{7,4}{⾡、⽓}[HSK 4]
  \definition{adj.}{sortudo; afortunado}
  \definition{s.}{sorte; fortuna}
\end{entry}

\begin{entry}{运输}{yun4shu1}{7,13}{⾡、⾞}[HSK 3]
  \definition{v.}{enviar; transportar; usar um carro, navio, avião, etc. para transportar pessoas ou coisas de um lugar para outro}
\end{entry}

\begin{entry}{运行}{yun4xing2}{7,6}{⾡、⾏}[HSK 5]
  \definition{v.}{(corpos celestes, etc.) mover-se ao longo do curso | (figurativo) funcionar, estar em operação | (serviço de trem, etc.) operar | (computador) executar um programa}
\end{entry}

\begin{entry}{运用}{yun4yong4}{7,5}{⾡、⽤}[HSK 4]
  \definition{v.}{usar; utilizar; manejar; aplicar; explorar as coisas de acordo com suas características}
\end{entry}

%%%%% EOF %%%%%

