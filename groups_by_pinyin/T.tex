%%%
%%% T
%%%

\section*{T}\addcontentsline{toc}{section}{T}

\begin{entry}{他}{ta1}{5}{⼈}[HSK 1]
  \definition{pron.}{ele | outro; referindo-se a outro; diferente | usado após o verbo, indica referência vaga | alguém; todos; usado em conjunto com 你, significa qualquer pessoa ou muitas pessoas | em outro lugar; outro lugar}
  \seealsoref{你}{ni3}
  \seealsoref{怹}{tan1}
\end{entry}

\begin{entry}{他的}{ta1 de5}{5,8}{⼈、⽩}
  \definition{pron.}{dele}
\end{entry}

\begin{entry}{他妈的}{ta1ma1de5}{5,6,8}{⼈、⼥、⽩}
  \definition{interj.}{Dane-se! | Foda-se!}
\end{entry}

\begin{entry}{他们}{ta1men5}{5,5}{⼈、⼈}[HSK 1]
  \definition{pron.}{eles}
\end{entry}

\begin{entry}{他们的}{ta1men5 de5}{5,5,8}{⼈、⼈、⽩}
  \definition{pron.}{deles}
\end{entry}

\begin{entry}{它}{ta1}{5}{⼧}[HSK 2]
  \definition*{s.}{sobrenome Ta}
  \definition{pron.}{ele; referência a algo além da pessoa (para objetos inanimados) | ele; usado após o verbo, indica referência vaga}
\end{entry}

\begin{entry}{它们}{ta1 men5}{5,5}{⼧、⼈}[HSK 2]
  \definition{pron.}{eles; usado para se referir a mais de uma coisa não humana; geralmente se refere a animais, objetos ou conceitos abstratos}
\end{entry}

\begin{entry}{她}{ta1}{6}{⼥}[HSK 1]
  \definition{pron.}{ela | ela; referir-se a coisas que se ama ou aprecia, como a pátria, a bandeira nacional, etc.}
\end{entry}

\begin{entry}{她的}{ta1 de5}{6,8}{⼥、⽩}
  \definition{pron.}{dela}
\end{entry}

\begin{entry}{她们}{ta1men5}{6,5}{⼥、⼈}[HSK 1]
  \definition{pron.}{elas; referindo-se a várias mulheres: em textos escritos, use 她们 quando todas as pessoas forem mulheres e 他们 quando houver homens e mulheres}
  \seealsoref{他们}{ta1men5}
\end{entry}

\begin{entry}{她们的}{ta1men5 de5}{6,5,8}{⼥、⼈、⽩}
  \definition{pron.}{delas}
\end{entry}

\begin{entry}{踏板}{ta4ban3}{15,8}{⾜、⽊}
  \definition{s.}{pedal (em um carro, em um piano, etc.) |  apoio para os pés | estribo}
\end{entry}

\begin{entry}{台}{tai2}{5}{⼝}[HSK 3]
  \definition*{s.}{sobrenome Tai}
  \definition{clas.}{usado para certas máquinas, aparelhos, instrumentos, etc | usado para uma performance completa, como drama, música e dança}
  \definition{s.}{torre | plataforma; palco | suporte; pedestal | qualquer coisa em forma de plataforma ou palco | mesa; escrivaninha | estação de transmissão; refere-se a estações de rádio | um serviço telefônico especial; refere-se à estação telefônica | ``seu'' (um termo respeitoso usado antigamente para se dirigir a alguém) | tufão}
\end{entry}

\begin{entry}{台风}{tai2feng1}{5,4}{⼝、⾵}[HSK 5]
  \definition[场,阵,级]{s.}{tufão; classificação de um ciclone tropical ocorrido no oeste do Pacífico Norte | postura; presença de palco; comportamento ou estilo que os atores demonstram no palco}
\end{entry}

\begin{entry}{台阶}{tai2jie1}{5,6}{⼝、⾩}[HSK 4]
  \definition[个,级]{s.}{escada; escadaria | passos; metáfora para uma maneira ou oportunidade de evitar constrangimentos causados ​​por um impasse | nova fase; novo nível; novo patamar; metáfora para novas conquistas ou novos patamares alcançados no estudo ou no trabalho}
\end{entry}

\begin{entry}{台上}{tai2 shang4}{5,3}{⼝、⼀}[HSK 4]
  \definition{s.}{no palco}
\end{entry}

\begin{entry}{台下}{tai2xia4}{5,3}{⼝、⼀}
  \definition{s.}{platéia | fora do palco}
\end{entry}

\begin{entry}{抬}{tai2}{8}{⼿}[HSK 5]
  \definition{clas.}{para objetos que precisam ser carregados por pessoas quando transportados (por exemplo, móveis)}
  \definition{v.}{levantar; elevar; puxar para cima | (por duas ou mais pessoas) carregar; transportar; duas ou mais pessoas carregando algo com as mãos ou nos ombros | discutir, debater (geralmente sem sentido ou sem importância)}
\end{entry}

\begin{entry}{抬杠}{tai2gang4}{8,7}{⼿、⽊}
  \definition{v.+compl.}{discutir pelo prazer em discutir | discutir obstinadamente | brigar}
\end{entry}

\begin{entry}{抬头}{tai2 tou2}{8,5}{⼿、⼤}[HSK 5]
  \definition{s.}{(em recibos, contas, etc.) nome do comprador ou beneficiário, ou espaço para preencher esse nome | nome do comprador ou beneficiário; refere-se ao cabeçalho do documento ou da fatura}
  \definition{v.}{levantar a cabeça | ganhar terreno; olhar para cima; subir | começar uma nova linha, como sinal de respeito, ao mencionar o destinatário em cartas, correspondência oficial, etc.}
\end{entry}

\begin{entry}{太}{tai4}{4}{⼤}[HSK 1]
  \definition*{s.}{sobrenome Tai}
  \definition{adj.}{mais alto; maior; mais distante | maior; extremo | bisavô; mais velho ou mais antigo; o de maior posição social ou hierarquia}
  \definition{adv.}{demais; expressa um grau excessivo (usado principalmente para coisas indesejáveis) | muito; extremamente; excessivamente; indica um grau extremamente elevado | muito; usado após o advérbio negativo 不, enfraquece o grau de negação e contém um tom diplomático}
\end{entry}

\begin{entry}{太极拳}{tai4ji2quan2}{4,7,10}{⼤、⽊、⼿}
  \definition*{s.}{Tai Chi Chuan, Taiji, T'aichi ou T'aichichuan; forma tradicional de exercício físico ou relaxamento}
\end{entry}

\begin{entry}{太空}{tai4kong1}{4,8}{⼤、⽳}[HSK 5]
  \definition[把]{s.}{firmamento; espaço sideral; espaço além da atmosfera terrestre; o céu vasto e infinito}
\end{entry}

\begin{entry}{太平洋}{tai4ping2 yang2}{4,5,9}{⼤、⼲、⽔}
  \definition*{s.}{Oceano Pacífico}
\end{entry}

\begin{entry}{太太}{tai4tai5}{4,4}{⼤、⼤}[HSK 2]
  \definition[位,名,个,些]{s.}{senhora; madame; títulos para mulheres casadas | esposa; senhora; madame; referir-se à própria esposa ou à esposa de outra pessoa}
\end{entry}

\begin{entry}{太阳窗}{tai4yang2chuang1}{4,6,12}{⼤、⾩、⽳}
  \definition{s.}{teto solar (de veículos)}
\end{entry}

\begin{entry}{太阳灯}{tai4yang2deng1}{4,6,6}{⼤、⾩、⽕}
  \definition{s.}{lâmpada solar (com células fotovoltaicas)}
\end{entry}

\begin{entry}{太阳风}{tai4yang2feng1}{4,6,4}{⼤、⾩、⾵}
  \definition{s.}{vento solar}
\end{entry}

\begin{entry}{太阳镜}{tai4yang2jing4}{4,6,16}{⼤、⾩、⾦}
  \definition{s.}{óculos de sol}
\end{entry}

\begin{entry}{太阳日}{tai4yang2ri4}{4,6,4}{⼤、⾩、⽇}
  \definition{s.}{dia solar}
\end{entry}

\begin{entry}{太阳穴}{tai4yang2xue2}{4,6,5}{⼤、⾩、⽳}
  \definition{s.}{têmpora (nas laterais da cabeça humana)}
\end{entry}

\begin{entry}{太阳翼}{tai4yang2yi4}{4,6,17}{⼤、⾩、⽻}
  \definition{s.}{painel solar}
\end{entry}

\begin{entry}{太阳雨}{tai4yang2yu3}{4,6,8}{⼤、⾩、⾬}
  \definition{s.}{banho de sol}
\end{entry}

\begin{entry}{太阳}{tai4yang5}{4,6}{⼤、⾩}[HSK 2]
  \definition[个,轮,枚,颗,盏]{s.}{o Sol | luz do sol; luz solar}
\end{entry}

\begin{entry}{态度}{tai4du5}{8,9}{⼼、⼴}[HSK 2]
  \definition[种,个]{s.}{maneira; comportamento; atitude; comportamento e expressão facial das pessoas | atitude; abordagem; opinião sobre o assunto e medidas tomadas}
\end{entry}

\begin{entry}{贪婪}{tan1lan2}{8,11}{⾙、⼥}
  \definition{adj.}{avaro | ambicioso | voraz | insaciável}
\end{entry}

\begin{entry}{怹}{tan1}{9}{⼼}
  \definition{pron.}{ele, ela (cortês, em oposição a 他)}
  \seealsoref{他}{ta1}
\end{entry}

\begin{entry}{谈}{tan2}{10}{⾔}[HSK 3]
  \definition*{s.}{sobrenome Tan}
  \definition{s.}{o que é dito ou falado; discurso}
  \definition{v.}{falar; bater papo; discutir}
\end{entry}

\begin{entry}{谈话}{tan2 hua4}{10,8}{⾔、⾔}[HSK 3]
  \definition[次]{s.}{declaração; opiniões (principalmente políticas) expressas na forma de conversas}
  \definition{v.+compl.}{conversar; discutir | falar; refere-se especificamente ao uso da conversa para entender a situação, fazer trabalho ideológico, etc. (usado principalmente por superiores para subordinados)}
\end{entry}

\begin{entry}{谈恋爱}{tan2lian4'ai4}{10,10,10}{⾔、⼼、⽖}
  \definition{v.}{namorar | apaixonar-se}
\end{entry}

\begin{entry}{谈判}{tan2pan4}{10,7}{⾔、⼑}[HSK 3]
  \definition{v.}{negociar; manter conversações; para resolver um grande problema, as partes relevantes trocaram opiniões entre si, na esperança de encontrar uma solução com a qual todos pudessem concordar}
\end{entry}

\begin{entry}{弹}{tan2}{11}{⼸}[HSK 5]
  \definition{s.}{bola; pelota; pequenas bolas disparadas com um estilingue | bomba; bala; explosivos que podem ser lançados ou arremessados, com poder destrutivo e letal}
\end{entry}

\begin{entry}{坦克}{tan3ke4}{8,7}{⼟、⼗}
  \definition{s.}{(empréstimo linguístico) tanque (veículo militar)}
\end{entry}

\begin{entry}{探亲}{tan4qin1}{11,9}{⼿、⼇}
  \definition{v.+compl.}{ir para casa para visitar a família}
\end{entry}

\begin{entry}{碳}{tan4}{14}{⽯}
  \definition{s.}{carbono (elemento químico)}
\end{entry}

\begin{entry}{汤}{tang1}{6}{⽔}[HSK 3]
  \definition*{s.}{sobrenome Tang}
  \definition[勺,碗,杯,锅]{s.}{água quente; água fervente | fontes termais | água utilizada para ferver algo| sopa; caldo | uma preparação líquida de ervas medicinais; decocção}
  \seeref{汤}{shang1}
\end{entry}

\begin{entry}{唐人街}{tang2ren2 jie1}{10,2,12}{⼝、⼈、⾏}
  \definition*{s.}{Bairro Chinês | \emph{Chinatown}}
  \seealsoref{中国城}{zhong1guo2cheng2}
\end{entry}

\begin{entry}{糖}{tang2}{16}{⽶}[HSK 3]
  \definition[包,斤,勺,袋,块]{s.}{açúcar; um tipo de açúcar; um tipo de composto orgânico, que pode ser dividido em três tipos: monossacarídeos, dissacarídeos e polissacarídeos; é a principal substância que produz energia térmica no corpo humano, como glicose, sacarose, lactose, amido, etc. | açúcar; açúcar comestível; termo geral para açúcar | doces; balas | carboidrato; algo doce e calórico}
\end{entry}

\begin{entry}{糖醋鱼}{tang2cu4yu2}{16,15,8}{⽶、⾣、⿂}
  \definition{s.}{peixe guisado em molho agridoce (prato)}
\end{entry}

\begin{entry}{倘或}{tang3huo4}{10,8}{⼈、⼽}
  \definition{conj.}{se | supondo que | no caso}
\end{entry}

\begin{entry}{倘若}{tang3ruo4}{10,8}{⼈、⾋}
  \definition{conj.}{se | supondo que | no caso}
\end{entry}

\begin{entry}{倘使}{tang3shi3}{10,8}{⼈、⼈}
  \definition{conj.}{se | supondo que | no caso}
\end{entry}

\begin{entry}{躺}{tang3}{15}{⾝}[HSK 4]
  \definition{v.}{deitar; reclinar}
\end{entry}

\begin{entry}{滔天}{tao1tian1}{13,4}{⽔、⼤}
  \definition{adj.}{(ondas, raiva, desastres, crimes, etc.) imponente, avassalador, imenso}
\end{entry}

\begin{entry}{逃}{tao2}{9}{⾡}[HSK 5]
  \definition{v.}{fugir; escapar; correr; dar no pé | evadir; esquivar-se; escapar}
\end{entry}

\begin{entry}{逃跑}{tao2 pao3}{9,12}{⾡、⾜}[HSK 5]
  \definition{v.}{fugir; escapar; correr; partir para fugir de um ambiente ou de coisas que não lhe são favoráveis}
\end{entry}

\begin{entry}{逃走}{tao2 zou3}{9,7}{⾡、⾛}[HSK 5]
  \definition{v.}{escapar; afastar-se de pessoas, coisas ou lugares que não são bons para você ou que você não gosta}
\end{entry}

\begin{entry}{桃}{tao2}{10}{⽊}[HSK 5]
  \definition*{s.}{sobrenome Tao}
  \definition{s.}{pêssego | em forma de pêssego | pessegueiro}
\end{entry}

\begin{entry}{桃花}{tao2 hua1}{10,7}{⽊、⾋}[HSK 5]
  \definition{s.}{(figurativo) caso amoroso | flor de pessegueiro}
\end{entry}

\begin{entry}{桃树}{tao2 shu4}{10,9}{⽊、⽊}[HSK 5]
  \definition[株]{s.}{pêssego (árvore) | pessegueiro; pêssegos}
\end{entry}

\begin{entry}{讨论}{tao3lun4}{5,6}{⾔、⾔}[HSK 2]
  \definition{v.}{discutir; conversar sobre; trocar opiniões ou debater as questões levantadas}
\end{entry}

\begin{entry}{讨生活}{tao3sheng1huo2}{5,5,9}{⾔、⽣、⽔}
  \definition{v.}{ganhar a vida}
\end{entry}

\begin{entry}{讨厌}{tao3yan4}{5,6}{⾔、⼚}[HSK 5]
  \definition{adj.}{desagradável; repugnante; repulsivo; irritante; incômodo |}
  \definition{v.}{odiar; não gostar; sentir repulsa por}
\end{entry}

\begin{entry}{套}{tao4}{10}{⼤}[HSK 2]
  \definition{clas.}{usado para coisas agrupadas: conjuntos, coleções, séries, etc.}
  \definition{s.}{estojo; capa; bainha | local onde o rio ou a cordilheira faz uma curva (usado principalmente em nomes de lugares) | enchimento de algodão em roupas e edredons | arreios; corda para amarrar animais | nó; laço; um objeto circular feito com corda ou algo semelhante | cortersia; convenção; conversa fiada; métodos repetitivos | armadilha; truque; conspiração}
  \definition{v.}{sobrepor; interligar | deslizar sobre; cobrir por fora | atrelar; engatar; usar um cinto de segurança | copiar; imitar; seguir o modelo de | extrair; induzir a falar; persuadir alguém a revelar um segredo; induzir; provocar | tentar vencer; aproximar-se de; aproximar-se intencionalmente de outras pessoas para algum propósito | fazer a rosca de um parafuso; usar um macho de rosca ou uma chave de rosca para fazer roscas}
\end{entry}

\begin{entry}{套餐}{tao4 can1}{10,16}{⼤、⾷}[HSK 4]
  \definition{s.}{combo; pacote de produtos; pacote de serviços; metaforicamente, bens ou projetos que são combinados e levados ao mercado | refeição preparada; pacotes de refeições completos}
\end{entry}

\begin{entry}{套问}{tao4wen4}{10,6}{⼤、⾨}
  \definition{s.}{retórica}
  \definition{v.}{descobrir por meio de questionamento indireto diplomático}
\end{entry}

\begin{entry}{特别}{te4bie2}{10,7}{⽜、⼑}[HSK 2]
  \definition{adj.}{especial; particular; fora do comum; diferente dos outros, com características próprias}
  \definition{adv.}{especialmente; particularmente | ainda mais; em particular; frequentemente usado com 是 | especialmente; deliberadamente; para um propósito específico}
  \seealsoref{是}{shi4}
\end{entry}

\begin{entry}{特地}{te4di4}{10,6}{⽜、⼟}
  \definition{adv.}{especialmente | propositalmente}
\end{entry}

\begin{entry}{特点}{te4dian3}{10,9}{⽜、⽕}[HSK 2]
  \definition[个,大]{s.}{característica; peculiaridade; traço distintivo; a singularidade de uma pessoa ou coisa}
\end{entry}

\begin{entry}{特定}{te4ding4}{10,8}{⽜、⼧}[HSK 5]
  \definition{adj.}{específico; especialmente designado | dado; especificado; específico (pessoa, hora, lugar, local, ambiente, etc.)}
\end{entry}

\begin{entry}{特技}{te4ji4}{10,7}{⽜、⼿}
  \definition{s.}{efeito especial | dublê}
\end{entry}

\begin{entry}{特价}{te4 jia4}{10,6}{⽜、⼈}[HSK 4]
  \definition{s.}{oferta especial; preço de barganha; preço especial reduzido}
\end{entry}

\begin{entry}{特色}{te4se4}{10,6}{⽜、⾊}[HSK 3]
  \definition{s.}{característica; característica distintiva; a cor única, estilo, etc. de um objeto}
\end{entry}

\begin{entry}{特殊}{te4shu1}{10,10}{⽜、⽍}[HSK 4]
  \definition{adj.}{especial; particular; peculiar; excepcional; incomum}
\end{entry}

\begin{entry}{特性}{te4 xing4}{10,8}{⽜、⼼}[HSK 5]
  \definition[个]{s.}{propriedade específica (ou característica) | característica; sabores | propriedade}
\end{entry}

\begin{entry}{特有}{te4 you3}{10,6}{⽜、⽉}[HSK 5]
  \definition{adj.}{específico; peculiar; característico; único; exclusivo; especial}
\end{entry}

\begin{entry}{特征}{te4zheng1}{10,8}{⽜、⼻}[HSK 4]
  \definition[个,种]{s.}{característica; aparência ou o fenômeno característico de uma pessoa ou coisa que pode ser visto de fora}
\end{entry}

\begin{entry}{疼}{teng2}{10}{⽧}[HSK 2]
  \definition{adj.}{dolorido; doído; sensação de extremo desconforto causada por ferimentos, doenças, etc.}
  \definition{v.}{ferir; machucar | adorar; amar profundamente; gostar muito; cuidar}
\end{entry}

\begin{entry}{梯恩梯}{ti1'en1ti1}{11,10,11}{⽊、⼼、⽊}
  \definition{s.}{(empréstimo linguístico) TNT, trinitrotolueno}
\end{entry}

\begin{entry}{踢}{ti1}{15}{⾜}
  \definition{v.}{chutar | jogar (por exemplo, futebol) | dar pontapés em}
\end{entry}

\begin{entry}{踢爆}{ti1bao4}{15,19}{⾜、⽕}
  \definition{v.}{expor | revelar}
\end{entry}

\begin{entry}{踢蹋舞}{ti1ta4wu3}{15,17,14}{⾜、⾜、⾇}
  \definition{s.}{sapateado | passo de dança}
\end{entry}

\begin{entry}{提}{ti2}{12}{⼿}[HSK 2]
  \definition*{s.}{sobrenome Ti}
  \definition{s.}{concha; utensílio para servir óleo ou vinho | traço ascendente (em caracteres chineses)}
  \definition{v.}{carregar (na mão, com o braço para baixo) ; segurar com as mãos para baixo | elevar; levantar; promover | avançar; antecipar uma data; mudar para uma data anterior; adiar o prazo previsto | levantar; apresentar; indicar ou citar | extrair; retirar (tirar) | (prisioneiros) trazer; entregar | mencionar; referir-se a; abordar}
\end{entry}

\begin{entry}{提倡}{ti2chang4}{12,10}{⼿、⼈}[HSK 5]
  \definition{v.}{promover; incentivar; recomendar; apresentar as vantagens de algo para incentivar as pessoas a usá-lo ou implementá-lo}
\end{entry}

\begin{entry}{提出}{ti2 chu1}{12,5}{⼿、⼐}[HSK 2]
  \definition{v.}{levantar; propor; apresentar; expressar seus desejos, ideias, sugestões, etc. por meio de palavras ou textos}
\end{entry}

\begin{entry}{提到}{ti2 dao4}{12,8}{⼿、⼑}[HSK 2]
  \definition{v.}{mencionar; referir-se a; levantar (assunto)}
\end{entry}

\begin{entry}{提高}{ti2gao1}{12,10}{⼿、⾼}[HSK 2]
  \definition{v.}{elevar; aprimorar; aumentar; melhorar a posição, o nível, a quantidade, a qualidade e outros aspectos em relação ao estado original}
\end{entry}

\begin{entry}{提供}{ti2gong1}{12,8}{⼿、⼈}[HSK 4]
  \definition{v.}{oferecer; fornecer; suprir; prover; proporcionar}
\end{entry}

\begin{entry}{提及}{ti2ji2}{12,3}{⼿、⼃}
  \definition{v.}{mencionar | levantar (um assunto) | chamar a atenção de alguém}
\end{entry}

\begin{entry}{提起}{ti2 qi3}{12,10}{⼿、⾛}[HSK 5]
  \definition{v.}{mencionar; falar sobre; abordar | levantar; despertar; estimular; revigorar; alegrar/animar | iniciar; instituir; propor | levantar; pegar}
\end{entry}

\begin{entry}{提前}{ti2qian2}{12,9}{⼿、⼑}[HSK 3]
  \definition{adv.}{antecipadamente; faça uma coisa antes de fazer outra}
  \definition{v.}{avançar; adiantar; mudar para uma data anterior; trazer para frente}
\end{entry}

\begin{entry}{提升}{ti2sheng1}{12,4}{⼿、⼗}
  \definition{v.}{promover (para uma posição de classificação mais alta) | levantar | içar | (figurativo) elevar, levantar, melhorar}
\end{entry}

\begin{entry}{提示}{ti2shi4}{12,5}{⼿、⽰}[HSK 5]
  \definition[个]{s.}{dica; lembrete; pistas ou informações fornecidas para chamar a atenção, fazer com que a outra pessoa pense ou compreenda}
  \definition{v.}{solicitar; lembrar; indicar; alertar; levantar questões que o outro não tenha pensado ou não tenha imaginado, para chamar a atenção dele}
\end{entry}

\begin{entry}{提问}{ti2wen4}{12,6}{⼿、⾨}[HSK 3]
  \definition{v.}{\emph{quiz}; fazer uma pergunta; colocar questões para}
\end{entry}

\begin{entry}{提醒}{ti2xing3}{12,16}{⼿、⾣}[HSK 4]
  \definition{v.+compl.}{alertar; avisar; advertir; lembrar; apontar para ou chamar a atenção para}
\end{entry}

\begin{entry}{题}{ti2}{15}{⾴}[HSK 2]
  \definition*{s.}{sobrenome Ti}
  \definition[个,道]{s.}{tópico; título; assunto; problema; frases que indicam o conteúdo de poemas ou discursos | questão; questões que devem ser respondidas durante os exercícios ou exames | antigamente, referia-se à testa}
  \definition{v.}{inscrever; escrever; assinar}
\end{entry}

\begin{entry}{题材}{ti2cai2}{15,7}{⾴、⽊}[HSK 5]
  \definition{s.}{tema; assunto; material que compõe as obras literárias e artísticas, ou seja, os eventos ou fenômenos da vida descritos concretamente nas obras}
\end{entry}

\begin{entry}{题目}{ti2mu4}{15,5}{⾴、⽬}[HSK 3]
  \definition[个,道]{s.}{título; assunto; tópico; o título de um poema ou discurso | quebra-cabeça; problema de exercício; questões a serem respondidas em exercícios ou provas}
\end{entry}

\begin{entry}{体操}{ti3 cao1}{7,16}{⼈、⼿}[HSK 4]
  \definition{s.}{ginástica; esportes, exercícios ou performances de vários movimentos, sem armas ou com o auxílio de determinados equipamentos}
\end{entry}

\begin{entry}{体会}{ti3hui4}{7,6}{⼈、⼈}[HSK 3]
  \definition[个,些,种]{s.}{conhecimento; compreensão; experiência pessoal}
  \definition{v.}{perceber; saber (ou aprender) com a experiência}
\end{entry}

\begin{entry}{体积}{ti3ji1}{7,10}{⼈、⽲}[HSK 5]
  \definition[个]{s.}{volume; quantidade; o tamanho do espaço ocupado pelo objeto}
\end{entry}

\begin{entry}{体检}{ti3 jian3}{7,11}{⼈、⽊}[HSK 4]
  \definition{s.}{exame clínico}
  \definition{v.}{fazer um exame médico}
\end{entry}

\begin{entry}{体力}{ti3 li4}{7,2}{⼈、⼒}[HSK 5]
  \definition{s.}{força física; vigor físico (ou corporal); a força do corpo humano para sustentar suas próprias atividades}
\end{entry}

\begin{entry}{体内}{ti3nei4}{7,4}{⼈、⼌}
  \definition{adj.}{dentro do corpo | \emph{in vivo} (versus \emph{in vitro} | interno a}
\end{entry}

\begin{entry}{体现}{ti3xian4}{7,8}{⼈、⾒}[HSK 3]
  \definition{v.}{refletir; incorporar; encarnar; uma certa qualidade ou fenômeno se manifesta especificamente em uma determinada coisa}
\end{entry}

\begin{entry}{体验}{ti3yan4}{7,10}{⼈、⾺}[HSK 3]
  \definition[种]{s.}{experiência; a sensação adquirida pela experiência pessoal}
  \definition{v.}{aprender através da prática; aprender através da experiência pessoal; entender as coisas através da experiência pessoal}
\end{entry}

\begin{entry}{体育}{ti3yu4}{7,8}{⼈、⾁}[HSK 2]
  \definition{s.}{cultura física; treinamento físico; educação cuja principal tarefa é desenvolver a capacidade física e fortalecer a constituição física, alcançada através da participação em várias atividades esportivas | esportes; atividades esportivas; refere-se a esportes}
\end{entry}

\begin{entry}{体育场}{ti3 yu4 chang3}{7,8,6}{⼈、⾁、⼟}[HSK 2]
  \definition[个,座]{s.}{estádio; campo esportivo; espaço ao ar livre para a prática de exercícios físicos ou competições esportivas}
\end{entry}

\begin{entry}{体育馆}{ti3 yu4 guan3}{7,8,11}{⼈、⾁、⾷}[HSK 2]
  \definition[个,座,家]{s.}{ginásio; locais esportivos ou competições em ambientes fechados geralmente têm arquibancadas fixas}
\end{entry}

\begin{entry}{体重}{ti3 zhong4}{7,9}{⼈、⾥}[HSK 4]
  \definition{s.}{peso corporal}
\end{entry}

\begin{entry}{替}{ti4}{12}{⽈}[HSK 4]
  \definition{prep.}{para; em nome de}
  \definition{s.}{decadência; declínio; enfraquecimento}
  \definition{v.}{substituir; substituir por; tomar o lugar de}
\end{entry}

\begin{entry}{替代}{ti4 dai4}{12,5}{⽈、⼈}[HSK 4]
  \definition{v.}{substituir; suplantar}
\end{entry}

\begin{entry}{天}{tian1}{4}{⼤}[HSK 1]
  \definition*{s.}{sobrenome Tian}
  \definition{adj.}{localizado no topo; suspenso no ar | inato; natural}
  \definition{clas.}{usado para contar dias}
  \definition{s.}{céu; paraíso; espaço onde se encontram o sol, a lua e as estrelas | dia; as 24 horas do dia, às vezes referindo-se especificamente ao período diurno | um período de tempo em um dia; em algum momento do dia | temporada; estação do ano | clima | natureza | Deus; céu; o criador | paraíso; refere-se ao local onde residem os deuses, budas e imortais}
\end{entry}

\begin{entry}{天才}{tian1cai2}{4,3}{⼤、⼿}[HSK 5]
  \definition{adj.}{talentoso | superdotado | genial}
  \definition[个]{s.}{dom; genialidade; talento natural; inteligência e sabedoria acima da média}
\end{entry}

\begin{entry}{天鹅}{tian1'e2}{4,12}{⼤、⿃}
  \definition{s.}{cisne}
\end{entry}

\begin{entry}{天公}{tian1gong1}{4,4}{⼤、⼋}
  \definition{s.}{céu, paraíso | senhor do céu}
\end{entry}

\begin{entry}{天花板}{tian1hua1ban3}{4,7,8}{⼤、⾋、⽊}
  \definition{s.}{teto}
\end{entry}

\begin{entry}{天空}{tian1kong1}{4,8}{⼤、⽳}[HSK 3]
  \definition{s.}{o céu; o firmamento}
\end{entry}

\begin{entry}{天气}{tian1qi4}{4,4}{⼤、⽓}[HSK 1]
  \definition{s.}{clima, tempo; mudanças meteorológicas que ocorrem na atmosfera em uma determinada área e durante um determinado período de tempo, tais como temperatura, umidade, pressão atmosférica, precipitação, vento, nuvens, etc.}
\end{entry}

\begin{entry}{天然}{tian1ran2}{4,12}{⼤、⽕}
  \definition{adj.}{natural}
\end{entry}

\begin{entry}{天然气}{tian1ran2qi4}{4,12,4}{⼤、⽕、⽓}[HSK 5]
  \definition{s.}{gás; gás natural; gás combustível produzido em campos petrolíferos, carboníferos e pântanos}
\end{entry}

\begin{entry}{天上}{tian1 shang4}{4,3}{⼤、⼀}[HSK 2]
  \definition[期]{s.}{o céu; o paraíso}
\end{entry}

\begin{entry}{天使}{tian1shi3}{4,8}{⼤、⼈}
  \definition{s.}{anjo}
\end{entry}

\begin{entry}{天堂}{tian1tang2}{4,11}{⼤、⼟}
  \definition{s.}{paraíso, céu}
\end{entry}

\begin{entry}{天天}{tian1tian1}{4,4}{⼤、⼤}
  \definition{adv.}{todo dia}
\end{entry}

\begin{entry}{天文}{tian1wen2}{4,4}{⼤、⽂}[HSK 5]
  \definition[对]{s.}{astronomia; a distribuição e o movimento dos corpos celestes, como o sol, a lua e as estrelas, no universo}
\end{entry}

\begin{entry}{天下}{tian1xia4}{4,3}{⼤、⼀}
  \definition{s.}{terra sob o céu | o mundo todo | toda a China | reino}
\end{entry}

\begin{entry}{天择}{tian1ze2}{4,8}{⼤、⼿}
  \definition{s.}{seleção natural}
\end{entry}

\begin{entry}{天真}{tian1zhen1}{4,10}{⼤、⼗}[HSK 4]
  \definition{adj.}{ingênuo; inocente; ignorante; simples de coração, direto por natureza, livre de fingimento e hipocrisia}
\end{entry}

\begin{entry}{天柱}{tian1zhu4}{4,9}{⼤、⽊}
  \definition{s.}{pilar celestial, que sustenta o céu}
\end{entry}

\begin{entry}{兲}{tian1}{6}{⼋}
  \variantof{天}
\end{entry}

\begin{entry}{田}{tian2}{5}{⽥}[Kangxi 102]
  \definition*{s.}{sobrenome Tian}
  \definition[片]{s.}{fazenda | campo}
\end{entry}

\begin{entry}{田园}{tian2yuan2}{5,7}{⽥、⼞}
  \definition{adj.}{bucólico}
  \definition{s.}{campo | interior | rural}
\end{entry}

\begin{entry}{钿}{tian2}{10}{⾦}
  \definition{s.}{(dialeto) moeda | dinheiro; moeda | uma quantia de dinheiro}
  \seeref{钿}{dian4}
\end{entry}

\begin{entry}{甜}{tian2}{11}{⽢}[HSK 3]
  \definition{adj.}{doce; melado | agradável; confortável; fazer as pessoas se sentirem confortáveis e felizes | (sono) profundo | feliz; descreve o sentimento de felicidade}
\end{entry}

\begin{entry}{甜酒}{tian2jiu3}{11,10}{⽢、⾣}
  \definition{s.}{licor doce}
\end{entry}

\begin{entry}{甜菊}{tian2ju2}{11,11}{⽢、⾋}
  \definition{s.}{estévia, arbusto cujas folhas produzem um substituto para o açúcar}
\end{entry}

\begin{entry}{甜品}{tian2pin3}{11,9}{⽢、⼝}
  \definition{s.}{sobremesa}
\end{entry}

\begin{entry}{甜食}{tian2shi2}{11,9}{⽢、⾷}
  \definition{s.}{doces | sobremesa}
\end{entry}

\begin{entry}{甜酸}{tian2suan1}{11,14}{⽢、⾣}
  \definition{adj.}{agridoce}
\end{entry}

\begin{entry}{甜甜圈}{tian2tian2quan1}{11,11,11}{⽢、⽢、⼞}
  \definition{s.}{rosquinha | \emph{doughnut}}
\end{entry}

\begin{entry}{甜筒}{tian2tong3}{11,12}{⽢、⽵}
  \definition{s.}{sorvete de casquinha}
\end{entry}

\begin{entry}{甜头}{tian2tou5}{11,5}{⽢、⼤}
  \definition{s.}{benefício | sabor doce (de poder, sucesso, etc.)}
\end{entry}

\begin{entry}{甜心}{tian2xin1}{11,4}{⽢、⼼}
  \definition{s.}{querido}
\end{entry}

\begin{entry}{甜言}{tian2yan2}{11,7}{⽢、⾔}
  \definition{s.}{boa conversa | palavras amáveis}
\end{entry}

\begin{entry}{甜玉米}{tian2 yu4mi3}{11,5,6}{⽢、⽟、⽶}
  \definition{s.}{milho doce}
\end{entry}

\begin{entry}{甜稚}{tian2zhi4}{11,13}{⽢、⽲}
  \definition{s.}{doce e inocente}
\end{entry}

\begin{entry}{填}{tian2}{13}{⼟}
  \definition{v.}{encher; rechear | reabastecer; suplementar; complementar | preencher; escrever dados em uma caixa (em um questionário ou formulário da \emph{Web})}
\end{entry}

\begin{entry}{填空}{tian2kong4}{13,8}{⼟、⽳}[HSK 4]
  \definition{v.}{preencher o espaço em branco (por exemplo, em um teste)}
\end{entry}

\begin{entry}{挑}{tiao1}{9}{⼿}[HSK 4]
  \definition{clas.}{para coisas que são escolhidas ou selecionadas | para coisas que podem ser usadas como palhetas}
  \definition{s.}{vara comprida com algo pendurado nas pontas; haste de transporte}
  \definition{v.}{escolher; selecionar | fazer picuinhas; ser hipercrítico; ser meticuloso; ser excessivamente rigoroso nos detalhes | carregar com uma haste de transporte; carregar no ombro; pendurar coisas nas pontas de varas longas e carregá-las em seus ombros}
  \seeref{挑}{tiao3}
\end{entry}

\begin{entry}{挑选}{tiao1 xuan3}{9,9}{⼿、⾡}[HSK 4]
  \definition{v.}{escolher; optar; selecionar; escolher a pessoa ou coisa certa para o trabalho}
\end{entry}

\begin{entry}{条}{tiao2}{7}{⽊}[HSK 2]
  \definition*{s.}{sobrenome Tiao}
  \definition{clas.}{usado para objetos longos e finos; usado para sintetizar certas coisas longas e retangulares em quantidades fixas | usado para itemização | aplicado ao corpo humano}
  \definition{s.}{galho; galhos finos e longos | tira; faixa | item; artigo | ordem; método | nota; anotação em papel}
\end{entry}

\begin{entry}{条幅}{tiao2fu2}{7,12}{⽊、⼱}
  \definition{s.}{faixa | banner | pergaminho de parede (para pintura ou caligrafia)}
\end{entry}

\begin{entry}{条贯}{tiao2guan4}{7,8}{⽊、⾙}
  \definition{s.}{ordem | procedimentos | sequência | sistema}
\end{entry}

\begin{entry}{条件}{tiao2jian4}{7,6}{⽊、⼈}[HSK 2]
  \definition[个,项,些]{s.}{condição; termo; fator; fatores que restringem a ocorrência, existência ou desenvolvimento das coisas | requisito; pré-requisito; qualificação; requisitos ou padrões estabelecidos para determinadas coisas | situação; estado; condição}
\end{entry}

\begin{entry}{条例}{tiao2li4}{7,8}{⽊、⼈}
  \definition{s.}{código de conduta | ordenanças | regulamentos | regras | estatutos}
\end{entry}

\begin{entry}{条目}{tiao2mu4}{7,5}{⽊、⽬}
  \definition{s.}{cláusulas e subcláusulas (em documento formal) | verbete (em um dicionário, enciclopédia, etc.)}
\end{entry}

\begin{entry}{调}{tiao2}{10}{⾔}[HSK 3]
  \definition{adj.}{harmonioso; boa coordenação}
  \definition{v.}{misturar; ajustar; fazer o ajuste uniforme e apropriado | provocar; importunar; fazer pouco de | incitar; instigar; provocar; semear discórdia | mediar; trazer harmonia}
  \seeref{调}{diao4}
\end{entry}

\begin{entry}{调节}{tiao2jie2}{10,5}{⾔、⾋}[HSK 5]
  \definition{v.}{regular; ajustar; ajustar e controlar de várias maneiras para atender aos requisitos}
\end{entry}

\begin{entry}{调解}{tiao2jie3}{10,13}{⾔、⾓}[HSK 5]
  \definition{v.}{mediar; fazer as pazes; resolver conflitos através da persuasão}
\end{entry}

\begin{entry}{调律}{tiao2lv4}{10,9}{⾔、⼻}
  \definition{v.}{afinar (por exemplo, um piano)}
\end{entry}

\begin{entry}{调皮}{tiao2pi2}{10,5}{⾔、⽪}[HSK 4]
  \definition{adj.}{travesso; malicioso; malandro | indisciplinado; desordeiro; indomável; astuto | inteligente e desonesto}
\end{entry}

\begin{entry}{调整}{tiao2zheng3}{10,16}{⾔、⽁}[HSK 3]
  \definition{v.}{ajustar; revisar; regularizar; fazer as alterações apropriadas no estado original para se adaptar à nova situação}
\end{entry}

\begin{entry}{挑}{tiao3}{9}{⼿}[HSK 4]
  \definition{s.}{um dos traços dos caracteres chineses; inclinado para cima da esquerda para a direita}
  \definition{v.}{levantar; elevar; erguer | levantar ou apoiar com uma extremidade de uma vara ou objeto semelhante; segurar ou apoiar com a ponta de uma vara etc. | causar conflitos deliberadamente; provocar deliberadamente um conflito | (método de bordado) usar uma agulha para levantar os fios de urdidura ou trama, com a agulha e a linha passando por baixo para formar padrões e desenhos}
  \seeref{挑}{tiao1}
\end{entry}

\begin{entry}{挑衅}{tiao3xin4}{9,11}{⼿、⾎}
  \definition{s.}{provocação}
  \definition{v.}{provocar}
\end{entry}

\begin{entry}{挑战}{tiao3zhan4}{9,9}{⼿、⼽}[HSK 4]
  \definition{v.}{desafiar; deixar um oponente deliberadamente irritado e sair para lutar ou lutar consigo mesmo; estimular um oponente a lutar consigo mesmo}
\end{entry}

\begin{entry}{跳}{tiao4}{13}{⾜}[HSK 3]
  \definition{v.}{pular; saltar | mover para cima e para baixo | pular (por cima); fazer omissões | quicar; a força elástica faz com que o objeto se mova repentinamente para cima | pulsar; palpitar; contrair-se | pular sobre;  saltar sobre; cruzar}
\end{entry}

\begin{entry}{跳挡}{tiao4dang3}{13,9}{⾜、⼿}
  \definition{v.}{pular marcha (de um carro) | perder a marcha}
\end{entry}

\begin{entry}{跳电}{tiao4dian4}{13,5}{⾜、⽥}
  \definition{v.}{desarmar (um disjuntor ou interruptor)}
\end{entry}

\begin{entry}{跳高}{tiao4 gao1}{13,10}{⾜、⾼}[HSK 3]
  \definition{s.}{salto em altura (atletismo)}
  \definition{v.}{saltar em altura}
\end{entry}

\begin{entry}{跳频}{tiao4pin2}{13,13}{⾜、⾴}
  \definition{s.}{FHSS, \emph{Frequency-Hopping Spread Spectrum}, método de transmissão de sinais de rádio}
\end{entry}

\begin{entry}{跳伞}{tiao4san3}{13,6}{⾜、⼈}
  \definition{s.}{paraquedas}
  \definition{v.}{saltar de paraquedas}
\end{entry}

\begin{entry}{跳绳}{tiao4sheng2}{13,11}{⾜、⽷}
  \definition{v.}{pular corda}
\end{entry}

\begin{entry}{跳水}{tiao4shui3}{13,4}{⾜、⽔}
  \definition{s.}{mergulho esportivo}
  \definition{v.}{mergulhar (na água) | cometer suicídio pulando na água | (figurativo, preços das ações, etc.) cair dramaticamente}
\end{entry}

\begin{entry}{跳跳糖}{tiao4tiao4tang2}{13,13,16}{⾜、⾜、⽶}
  \definition{s.}{\emph{Pop Rocks}, \emph{popping candy}}
\end{entry}

\begin{entry}{跳舞}{tiao4wu3}{13,14}{⾜、⾇}[HSK 3]
  \definition{v.+compl.}{dançar (como performance); executar dança, especialmente dança de salão}
\end{entry}

\begin{entry}{跳远}{tiao4 yuan3}{13,7}{⾜、⾡}[HSK 3]
  \definition{s.}{salto em distância (atletismo)}
\end{entry}

\begin{entry}{跳蚤}{tiao4zao5}{13,9}{⾜、⾍}
  \definition{s.}{pulga}
\end{entry}

\begin{entry}{贴}{tie1}{9}{⾙}[HSK 4]
  \definition{adj.}{submisso; obediente}
  \definition{clas.}{para uso em gessos, emplastros}
  \definition{s.}{subsídio; subvenção}
  \definition{v.}{grudar; colar | aninhar-se a; aconchegar-se a | subsidiar; ajudar financeiramente}
\end{entry}

\begin{entry}{铁}{tie3}{10}{⾦}[HSK 3]
  \definition*{s.}{sobrenome Tie}
  \definition{adj.}{duro; forte; sólido como ferro; metáfora para natureza dura; vontade forte | violento | inabalável; inalterável; determinado; metáfora para violência ou crueldade}
  \definition{s.}{ferro (Fe) | arma; armamento; refere-se a facas, armas de fogo, etc.}
  \definition{v.}{resolver; determinar}
\end{entry}

\begin{entry}{铁轨}{tie3gui3}{10,6}{⾦、⾞}
  \definition[根]{s.}{trilho | trilho ferroviário}
\end{entry}

\begin{entry}{铁路}{tie3 lu4}{10,13}{⾦、⾜}[HSK 3]
  \definition[条,公里]{s.}{ferrovia; estrada de ferro; uma estrada com trilhos de aço dispostos no leito da estrada para a circulação de trens}
\end{entry}

\begin{entry}{厅}{ting1}{4}{⼚}[HSK 5]
  \definition{s.}{salão; sala grande para reuniões ou receber convidados | escritório; nome de um departamento administrativo de uma grande organização | departamento governamental a nível provincial; nomes de alguns órgãos estaduais}
\end{entry}

\begin{entry}{听}{ting1}{7}{⼝}[HSK 1]
  \definition{clas.}{latas; usado para bebidas e alimentos para levar consigo}
  \definition{s.}{lata; embalagem metálica; recipiente cilíndrico utilizado para armazenar bebidas, alimentos, etc.}
  \definition{v.}{ouvir; escutar | obedecer; dar ouvidos; aceitar | supervisionar; administrar; tratar (assuntos políticos); julgar (casos) | permitir; deixar ser; deixar fazer}
  \seeref{听}{yin3}
\end{entry}

\begin{entry}{听到}{ting1 dao4}{7,8}{⼝、⼑}[HSK 1]
  \definition{v.}{ouvir, escutar; ouvir atentamente, escutar atentamente}
\end{entry}

\begin{entry}{听断}{ting1duan4}{7,11}{⼝、⽄}
  \definition{v.}{ouvir e decidir | julgar (ou seja, ouvir e julgar em um tribunal)}
\end{entry}

\begin{entry}{听骨}{ting1gu3}{7,9}{⼝、⾻}
  \definition{s.}{ossículos (do ouvido médio)}
  \seealsoref{听小骨}{ting1xiao3gu3}
\end{entry}

\begin{entry}{听会}{ting1hui4}{7,6}{⼝、⼈}
  \definition{v.}{participar de uma reunião (e ouvir o que é discutido)}
\end{entry}

\begin{entry}{听见}{ting1 jian4}{7,4}{⼝、⾒}[HSK 1]
  \definition{v.}{ouvir; conseguir ouvir}
\end{entry}

\begin{entry}{听讲}{ting1 jiang3}{7,6}{⼝、⾔}[HSK 2]
  \definition{v.+compl.}{assistir a uma palestra; ouvir palestras ou discursos}
\end{entry}

\begin{entry}{听来}{ting1lai2}{7,7}{⼝、⽊}
  \definition{v.}{ouvir de algum lugar | soar (antigo, estrangeiro, excitante, certo, etc.) | soar como se (ou seja, dar uma impressão ao ouvinte)}
\end{entry}

\begin{entry}{听力}{ting1li4}{7,2}{⼝、⼒}[HSK 3]
  \definition{s.}{audição; capacidade auditiva | compreensão auditiva (na aprendizagem de línguas)}
\end{entry}

\begin{entry}{听力理解}{ting1li4li3jie3}{7,2,11,13}{⼝、⼒、⽟、⾓}
  \definition{s.}{compreensão auditiva}
\end{entry}

\begin{entry}{听命}{ting1ming4}{7,8}{⼝、⼝}
  \definition{v.}{obedecer ordens | receber ordens}
\end{entry}

\begin{entry}{听凭}{ting1ping2}{7,8}{⼝、⼏}
  \definition{v.}{permitir (alguém a fazer o que desejar)}
\end{entry}

\begin{entry}{听说}{ting1 shuo1}{7,9}{⼝、⾔}[HSK 2]
  \definition{v.}{ser informado; ouvir falar de; ouvir dizer | ouvir e falar}
\end{entry}

\begin{entry}{听随}{ting1sui2}{7,11}{⼝、⾩}
  \definition{v.}{permitir | obedecer}
\end{entry}

\begin{entry}{听戏}{ting1xi4}{7,6}{⼝、⼽}
  \definition{v.}{assistir a uma ópera | ver uma ópera}
\end{entry}

\begin{entry}{听小骨}{ting1xiao3gu3}{7,3,9}{⼝、⼩、⾻}
  \definition{s.}{ossículos (do ouvido médio)}
  \seealsoref{听骨}{ting1gu3}
\end{entry}

\begin{entry}{听写}{ting1 xie3}{7,5}{⼝、⼍}[HSK 1]
  \definition{s.}{ditado}
  \definition{v.}{ouvir e escrever}
\end{entry}

\begin{entry}{听众}{ting1 zhong4}{7,6}{⼝、⼈}[HSK 3]
  \definition[位,名,个]{s.}{audiência; ouvintes; pessoas que ouvem palestras, música ou transmissões}
\end{entry}

\begin{entry}{聼}{ting1}{19}{⼼}
  \variantof{听}
\end{entry}

\begin{entry}{亭}{ting2}{9}{⼇}
  \definition{s.}{pavilhão | cabine | quiosque}
\end{entry}

\begin{entry}{停}{ting2}{11}{⼈}[HSK 2]
  \definition{adj.}{pronto; resolvido; bem organizado}
  \definition{clas.}{usado para partes (de um total); porções}
  \definition{v.}{parar; interromper; cessar; fazer uma pausa | permanecer; ficar; fazer uma parada (para descansar) | estacionar; ancorar; atracar}
\end{entry}

\begin{entry}{停办}{ting2ban4}{11,4}{⼈、⼒}
  \definition{v.}{cancelar | sair do negócio | desligar | terminar}
\end{entry}

\begin{entry}{停车}{ting2 che1}{11,4}{⼈、⾞}[HSK 2]
  \definition{v.}{(veículo) parar; frear | estacionar o veículo | parar; deixar de funcionar}
\end{entry}

\begin{entry}{停车场}{ting2 che1 chang3}{11,4,6}{⼈、⾞、⼟}[HSK 2]
  \definition[个]{s.}{estacionamento; área de estacionamento; local para estacionamento de veículos}
\end{entry}

\begin{entry}{停当}{ting2dang5}{11,6}{⼈、⼹}
  \definition{adj.}{realizado | preparado | assentado}
\end{entry}

\begin{entry}{停电}{ting2dian4}{11,5}{⼈、⽥}
  \definition{s.}{corte de energia}
  \definition{v.}{ter uma falha de energia}
\end{entry}

\begin{entry}{停工}{ting2gong1}{11,3}{⼈、⼯}
  \definition{v.}{parar de trabalhar | parar a produção}
\end{entry}

\begin{entry}{停火}{ting2huo3}{11,4}{⼈、⽕}
  \definition{s.}{cessar-fogo}
  \definition{v.+compl.}{cessar fogo}
\end{entry}

\begin{entry}{停课}{ting2ke4}{11,10}{⼈、⾔}
  \definition{v.}{fechar (escola) | parar as aulas}
\end{entry}

\begin{entry}{停留}{ting2 liu2}{11,10}{⼈、⽥}[HSK 5]
  \definition{v.}{permanecer; ficar por muito tempo; parar temporariamente em algum lugar, sem continuar avançando | permanecer; parar por um longo tempo; parar em um determinado estágio ou nível, sem evoluir}
\end{entry}

\begin{entry}{停息}{ting2xi1}{11,10}{⼈、⼼}
  \definition{v.}{cessar | parar}
\end{entry}

\begin{entry}{停下}{ting2 xia4}{11,3}{⼈、⼀}[HSK 4]
  \definition{v.}{encerrar; desligar; parar}
\end{entry}

\begin{entry}{停歇}{ting2xie1}{11,13}{⼈、⽋}
  \definition{v.}{parar para descansar}
\end{entry}

\begin{entry}{停业}{ting2ye4}{11,5}{⼈、⼀}
  \definition{v.}{cessar a negociação (temporária ou permanentemente) | fechar}
\end{entry}

\begin{entry}{停用}{ting2yong4}{11,5}{⼈、⽤}
  \definition{v.}{desabilitar | descontinuar | parar de usar | suspender}
\end{entry}

\begin{entry}{停止}{ting2 zhi3}{11,4}{⼈、⽌}[HSK 3]
  \definition{v.}{parar; suspender; cessar; cancelar}
\end{entry}

\begin{entry}{挺}{ting3}{9}{⼿}[HSK 2,4]
  \definition{adj.}{rígido; ereto; vertical; reto | notável; destacado; distinto}
  \definition{adv.}{muito; bastante}
  \definition{clas.}{usado para metralhadoras}
  \definition{v.}{sobressair; endireitar-se; protrudir (protuberância ou saliência) | suportar; aguentar; resistir; perseverar}
\end{entry}

\begin{entry}{挺拔}{ting3ba2}{9,8}{⼿、⼿}
  \definition{adj.}{alto e reto}
\end{entry}

\begin{entry}{挺杆}{ting3gan3}{9,7}{⼿、⽊}
  \definition{s.}{tucho (peça de máquina)}
\end{entry}

\begin{entry}{挺过}{ting3guo4}{9,6}{⼿、⾡}
  \definition{s.}{sobreviver}
\end{entry}

\begin{entry}{挺好}{ting3 hao3}{9,6}{⼿、⼥}[HSK 2]
  \definition{adj.}{nada mal; surpreendentemente bom}
\end{entry}

\begin{entry}{挺进}{ting3jin4}{9,7}{⼿、⾡}
  \definition{s.}{progresso | avanço}
  \definition{v.}{progredir | avançar}
\end{entry}

\begin{entry}{挺立}{ting3li4}{9,5}{⼿、⽴}
  \definition{v.}{ficar ereto | ficar de pé}
\end{entry}

\begin{entry}{挺身}{ting3shen1}{9,7}{⼿、⾝}
  \definition{v.}{endireitar as costas}
\end{entry}

\begin{entry}{挺尸}{ting3shi1}{9,3}{⼿、⼫}
  \definition{v.}{(coloquial) dormir | (literalmente) ficar deitado duro como um cadáver}
\end{entry}

\begin{entry}{挺腰}{ting3yao1}{9,13}{⼿、⾁}
  \definition{v.}{arquear as costas | endireitar as costas}
\end{entry}

\begin{entry}{挺住}{ting3zhu4}{9,7}{⼿、⼈}
  \definition{v.}{permanecer firme | manter-se firme (diante da adversidade ou da dor)}
\end{entry}

\begin{entry}{通}{tong1}{10}{⾡}[HSK 2]
  \definition*{s.}{sobrenome Tong}
  \definition{adj.}{lógico; coerente | geral; comum | tudo; inteiro | aberto; através de | total}
  \definition{clas.}{(antigo) usado para cartas, telegramas, documentos oficiais, etc.}
  \definition{s.}{autoridade; especialista}
  \definition{suf.}{especialista}
  \definition{v.}{abrir; atravessar | abrir ou limpar cutucando ou espetando | levar a; ir a | conectar; comunicar | notificar; informar | compreender; saber | cutucar; dar uma pancada | transmitir; conectar; interagir | dominar; compreender; entender}
  \seeref{通}{tong4}
\end{entry}

\begin{entry}{通常}{tong1chang2}{10,11}{⾡、⼱}[HSK 3]
  \definition{adj.}{usual; normal; geral}
  \definition{adv.}{habitualmente; usualmente; geralmente; ordinariamente}
\end{entry}

\begin{entry}{通牒}{tong1die2}{10,13}{⾡、⽚}
  \definition{s.}{nota diplomática}
\end{entry}

\begin{entry}{通观}{tong1guan1}{10,6}{⾡、⾒}
  \definition{v.}{ter uma visão geral de algo}
\end{entry}

\begin{entry}{通过}{tong1guo4}{10,6}{⾡、⾡}[HSK 2]
  \definition{prep.}{por; através de; por meio de; por meio de; meios, métodos, etc. para introduzir ações}
  \definition{v.}{atravessar; passar por; transitar | aprovar; adotar | solicitar o consentimento ou aprovação de}
\end{entry}

\begin{entry}{通识}{tong1shi2}{10,7}{⾡、⾔}
  \definition{s.}{conhecimento comum | erudição | conhecimento geral | amplamente conhecido}
\end{entry}

\begin{entry}{通信}{tong1 xin4}{10,9}{⾡、⼈}[HSK 3]
  \definition{v.+compl.}{corresponder; comunicar por carta; comunicar situações e informações escrevendo cartas | transmitir (ou transportar) mensagem; passar (ou transmitir) informação; usar ondas de rádio e outros sinais para transmitir texto, imagens, etc.}
\end{entry}

\begin{entry}{通用}{tong1yong4}{10,5}{⾡、⽤}[HSK 5]
  \definition{adj.}{de uso comum; universal; (em um determinado âmbito) de uso generalizado | intercambiável; alguns caracteres chineses com grafia diferente, mas pronúncia igual, podem ser usados indistintamente (alguns limitados a um determinado significado)}
\end{entry}

\begin{entry}{通知}{tong1zhi1}{10,8}{⾡、⽮}[HSK 2]
  \definition[份,个,张]{s.}{aviso; circular; notificação por escrito ou verbal}
  \definition{v.}{aconselhar; notificar; informar; dar aviso prévio}
\end{entry}

\begin{entry}{通知书}{tong1 zhi1 shu1}{10,8,4}{⾡、⽮、⼄}[HSK 4]
  \definition{s.}{aviso; observação; notificação}
\end{entry}

\begin{entry}{同}{tong2}{6}{⼝}
  \definition{adj.}{junto}
  \definition{adv.}{junto com}
\end{entry}

\begin{entry}{同伙}{tong2huo3}{6,6}{⼝、⼈}
  \definition[个]{s.}{cúmplice | colega}
\end{entry}

\begin{entry}{同流合污}{tong2liu2he2wu1}{6,10,6,6}{⼝、⽔、⼝、⽔}
  \definition{expr.}{chafurdar na lama com alguém | seguir o mau exemplo dos outros}
\end{entry}

\begin{entry}{同情}{tong2qing2}{6,11}{⼝、⼼}[HSK 4]
  \definition{s.}{simpatia}
  \definition{v.}{simpatizar com; solidarizar-se; compadecer-se; ter empatia emocional pelo que os outros estão passando}
\end{entry}

\begin{entry}{同时}{tong2shi2}{6,7}{⼝、⽇}[HSK 2]
  \definition{conj.}{além disso; além do mais; ainda mais; indica uma relação de equivalência, geralmente com um significado mais profundo}
  \definition{s.}{enquanto isso; ao mesmo tempo}
\end{entry}

\begin{entry}{同事}{tong2shi4}{6,8}{⼝、⼅}[HSK 2]
  \definition[个,位,名]{s.}{companheiro; colega; colega de trabalho; pessoas que trabalham juntas}
  \definition{v.}{trabalhar no mesmo lugar; trabalhar juntos; trabalhar na mesma unidade}
\end{entry}

\begin{entry}{同屋}{tong2wu1}{6,9}{⼝、⼫}
  \definition[个]{s.}{companheiro de quarto | colega de quarto}
\end{entry}

\begin{entry}{同性恋}{tong2xing4lian4}{6,8,10}{⼝、⼼、⼼}
  \definition{s.}{homossexualidade | pessoa gay | amor gay}
\end{entry}

\begin{entry}{同学}{tong2xue2}{6,8}{⼝、⼦}[HSK 1]
  \definition[位,个,些]{s.}{colega de escola; colega de turma; colega de estudos; pessoas que estudam na mesma escola}
\end{entry}

\begin{entry}{同砚}{tong2yan4}{6,9}{⼝、⽯}
  \definition[位,个]{s.}{colega de classe | colega estudante}
\end{entry}

\begin{entry}{同样}{tong2 yang4}{6,10}{⼝、⽊}[HSK 2]
  \definition{adj.}{igual; semelhante; similar; idêntico; sem diferença}
\end{entry}

\begin{entry}{同意}{tong2yi4}{6,13}{⼝、⼼}[HSK 3]
  \definition{v.}{concordar; consentir; aprovar; concordar com; dizer sim}
\end{entry}

\begin{entry}{童话}{tong2hua4}{12,8}{⽴、⾔}[HSK 4]
  \definition[个,部]{s.}{conto de fadas; gênero de literatura infantil no qual as histórias adequadas para a diversão das crianças são escritas com muita imaginação, fantasia e exagero}
\end{entry}

\begin{entry}{童年}{tong2 nian2}{12,6}{⽴、⼲}[HSK 4]
  \definition{s.}{infância; primeiros anos de vida}
\end{entry}

\begin{entry}{统计}{tong3ji4}{9,4}{⽷、⾔}[HSK 4]
  \definition{v.}{compilar estatísticas; refere-se à realização de trabalho estatístico, ou seja, coletar, reunir, analisar e extrapolar dados sobre um fenômeno | somar; adicionar; contar}
\end{entry}

\begin{entry}{统一}{tong3yi1}{9,1}{⽷、⼀}[HSK 4]
  \definition{adj.}{unificado; unitário; centralizado; consistente}
  \definition{v.}{unificar; unir; integrar; padronizar}
\end{entry}

\begin{entry}{通}{tong4}{10}{⾡}
  \definition{clas.}{usado para uma atividade, tomada em sua totalidade (discurso de abuso, período de reprodução de música, bebedeira, etc.)}
  \seeref{通}{tong1}
\end{entry}

\begin{entry}{痛}{tong4}{12}{⽧}[HSK 3]
  \definition{adv.}{extremamente; profundamente; amargamente}
  \definition{s.}{dor; sofrimento | tristeza; pesar}
\end{entry}

\begin{entry}{痛苦}{tong4ku3}{12,8}{⽧、⾋}[HSK 3]
  \definition{adj.}{doloroso; angustiado; sentindo-se muito desconfortável física ou mentalmente}
  \definition[降,种]{s.}{dor; agonia; sofrimento; refere-se a um estado ou sentimento de extremo desconforto físico ou mental}
\end{entry}

\begin{entry}{痛快}{tong4kuai4}{12,7}{⽧、⼼}[HSK 4]
  \definition{adj.}{encantado; alegre; muito feliz; confortável | franco; direto; simples e direto}
\end{entry}

\begin{entry}{痛骂}{tong4ma4}{12,9}{⽧、⾺}
  \definition{v.}{repreender severamente}
\end{entry}

\begin{entry}{偷}{tou1}{11}{⼈}[HSK 5]
  \definition{adv.}{furtivamente; secretamente; às escondidas}
  \definition{s.}{ladrão; furtador}
  \definition{v.}{roubar; furtar; levar sem pagar; roubar os bens alheios às escondidas | encontrar (tempo) | deixar-se levar; viver apenas para o presente, sem se preocupar com o futuro}
\end{entry}

\begin{entry}{偷安}{tou1'an1}{11,6}{⼈、⼧}
  \definition{v.}{buscar facilidade temporária}
\end{entry}

\begin{entry}{偷渡}{tou1du4}{11,12}{⼈、⽔}
  \definition{s.}{contrabando | imigração ilegal | clandestino (em um navio)}
  \definition{v.}{executar um bloqueio | roubar através da fronteira internacional}
\end{entry}

\begin{entry}{偷窃}{tou1qie4}{11,9}{⼈、⽳}
  \definition{v.}{furtar | roubar}
\end{entry}

\begin{entry}{偷情}{tou1qing2}{11,11}{⼈、⼼}
  \definition{v.}{manter um caso de amor clandestino}
\end{entry}

\begin{entry}{偷税}{tou1shui4}{11,12}{⼈、⽲}
  \definition{s.}{evasão fiscal}
\end{entry}

\begin{entry}{偷听}{tou1ting1}{11,7}{⼈、⼝}
  \definition{v.}{bisbilhotar; monitorar (secretamente)}
\end{entry}

\begin{entry}{偷偷}{tou1 tou1}{11,11}{⼈、⼈}[HSK 5]
  \definition{adv.}{secretamente; dissimuladamente; furtivamente; às escondidas}
\end{entry}

\begin{entry}{偷袭}{tou1xi2}{11,11}{⼈、⾐}
  \definition{s.}{ataque surpresa}
  \definition{v.}{montar um ataque furtivo | invadir}
\end{entry}

\begin{entry}{偸}{tou1}{11}{⼈}
  \variantof{偷}
\end{entry}

\begin{entry}{头}{tou2}{5}{⼤}[HSK 2,3]
  \definition{adj.}{(antes de um numeral) primeiro | (antes de 年 ou 天) último; anterior}
  \definition{clas.}{usado para suínos ou gado (animais de criação) | usado para cabeças de alho ou coisas com formato de cabeça}
  \definition{num.}{primeiro}
  \definition{prep.}{antes de; perto de; introduz o tempo de uma ação, equivalente a  在……之前 ou 临近 | (entre dois algarismos, indicando um número aproximado) cerca de}
  \definition[个,颗]{s.}{cabeça; a parte do corpo humano ou animal que possui órgãos como boca, nariz, olhos e ouvidos | cabelo ou penteado | topo; fim; a parte superior ou final de um objeto | começo ou fim; o ponto inicial ou final de algo | fim; remanescente; os restos de algo | cabeça; chefe; líder | lado; aspecto}
  \seeref{头}{tou5}
  \seealsoref{临近}{lin2jin4}
  \seealsoref{年}{nian2}
  \seealsoref{天}{tian1}
  \seealsoref{在}{zai4}
  \seealsoref{之前}{zhi1 qian2}
\end{entry}

\begin{entry}{头发}{tou2fa5}{5,5}{⼤、⼜}[HSK 2]
  \definition[根,缕,头]{s.}{cabelo}
\end{entry}

\begin{entry}{头号}{tou2hao4}{5,5}{⼤、⼝}
  \definition{adj.}{primeira classe | número um | \emph{top rank}}
\end{entry}

\begin{entry}{头脑}{tou2 nao3}{5,10}{⼤、⾁}[HSK 3]
  \definition{s.}{inteligência; mente | pista; tópicos principais | chefe; líder; capitão}
\end{entry}

\begin{entry}{头脑风暴}{tou2nao3feng1bao4}{5,10,4,15}{⼤、⾁、⾵、⽇}
  \definition{s.}{\emph{brainstorm}}
\end{entry}

\begin{entry}{头头}{tou2tou2}{5,5}{⼤、⼤}
  \definition{s.}{chefe | o cabeça}
\end{entry}

\begin{entry}{头像}{tou2xiang4}{5,13}{⼤、⼈}
  \definition{s.}{retrato | busto | avatar | imagem de perfil (computação)}
\end{entry}

\begin{entry}{投}{tou2}{7}{⼿}[HSK 4]
  \definition*{s.}{sobrenome Tou}
  \definition{pron.}{para; indica tempo, equivalente a 到, 临 | para; em direção a; indica orientação, direção, equivalente a 朝 ou 向}
  \definition{s.}{um jogo durante uma festa em que o vencedor era decidido pelo número de flechas lançadas em um pote distante | jogo de dados}
  \definition{v.}{lançar; arremessar; atirar | deixar cair; colocar em; lançar | mergulhar em; lançar-se em; pular dentro | lançar; projetar; sombrear | entregar; postar; enviar | ir até; ir para; buscar; juntar-se | sentir-se atraído por; adaptar-se a; concordar com; atender a}
  \seealsoref{朝}{chao2}
  \seealsoref{到}{dao4}
  \seealsoref{临}{lin2}
  \seealsoref{向}{xiang4}
\end{entry}

\begin{entry}{投递}{tou2di4}{7,10}{⼿、⾡}
  \definition{v.}{despachar | enviar}
\end{entry}

\begin{entry}{投票}{tou2piao4}{7,11}{⼿、⽰}
  \definition{v.+compl.}{votar | depositar um voto}
\end{entry}

\begin{entry}{投入}{tou2ru4}{7,2}{⼿、⼊}[HSK 4]
  \definition{adj.}{sisudo; dedicado; devotado; absorto}
  \definition{s.}{investimento; insumo; refere-se à aplicação de recursos}
  \definition{v.}{lançar em; colocar em; jogar em; por em | entrar em uma situação; participar de | aplicar; investir; colocar fundos em}
\end{entry}

\begin{entry}{投诉}{tou2su4}{7,7}{⼿、⾔}[HSK 4]
  \definition{v.}{reclamar; queixar-se; reclamar às autoridades ou pessoas envolvidas}
\end{entry}

\begin{entry}{投资}{tou2zi1}{7,10}{⼿、⾙}[HSK 4]
  \definition[次]{s.}{investimento}
  \definition{v.}{investir; aplicar dinheiro; investir dinheiro em negócios}
\end{entry}

\begin{entry}{投资风险}{tou2zi1feng1xian3}{7,10,4,9}{⼿、⾙、⾵、⾩}
  \definition{s.}{risco de investimento}
\end{entry}

\begin{entry}{投资回报率}{tou2zi1hui2bao4lv4}{7,10,6,7,11}{⼿、⾙、⼞、⼿、⽞}
  \definition{s.}{retorno sobre o investimento (ROI)}
\end{entry}

\begin{entry}{投资家}{tou2zi1jia1}{7,10,10}{⼿、⾙、⼧}
  \definition{s.}{investidor}
  \seealsoref{投资人}{tou2zi1ren2}
  \seealsoref{投资者}{tou2zi1zhe3}
\end{entry}

\begin{entry}{投资人}{tou2zi1ren2}{7,10,2}{⼿、⾙、⼈}
  \definition{s.}{investidor}
  \seealsoref{投资家}{tou2zi1jia1}
  \seealsoref{投资者}{tou2zi1zhe3}
\end{entry}

\begin{entry}{投资者}{tou2zi1zhe3}{7,10,8}{⼿、⾙、⽼}
  \definition{s.}{investidor}
  \seealsoref{投资家}{tou2zi1jia1}
  \seealsoref{投资人}{tou2zi1ren2}
\end{entry}

\begin{entry}{透}{tou4}{10}{⾡}[HSK 4]
  \definition{adv.}{totalmente; completamente; minuciosamente | profundamente; extremamente}
  \definition{v.}{penetrar; passar através de; infiltrar-se através de | revelar; deixar transparecer; contar secretamente |mostrar; aparecer}
\end{entry}

\begin{entry}{透彻}{tou4che4}{10,7}{⾡、⼻}
  \definition{adj.}{minucioso | incisivo | penetrante}
\end{entry}

\begin{entry}{透澈}{tou4che4}{10,15}{⾡、⽔}
  \variantof{透彻}
\end{entry}

\begin{entry}{透顶}{tou4ding3}{10,8}{⾡、⾴}
  \definition{adv.}{completamente}
\end{entry}

\begin{entry}{透过}{tou4guo4}{10,6}{⾡、⾡}
  \definition{v.}{passar através | penetrar}
\end{entry}

\begin{entry}{透亮}{tou4liang4}{10,9}{⾡、⼇}
  \definition{adj.}{brilhante | claro como cristal}
\end{entry}

\begin{entry}{透露}{tou4lu4}{10,21}{⾡、⾬}
  \definition{v.}{divulgar | vazar | revelar}
\end{entry}

\begin{entry}{透明}{tou4ming2}{10,8}{⾡、⽇}[HSK 4]
  \definition{adj.}{transparente; diáfano; capaz de transmitir luz | evidente; transparente; situação ou assunto que seja aberto e não oculto | transparente; diáfano; indica pureza, ausência de impurezas}
\end{entry}

\begin{entry}{透辟}{tou4pi4}{10,13}{⾡、⾟}
  \definition{adj.}{incisivo | penetrante}
\end{entry}

\begin{entry}{透气}{tou4qi4}{10,4}{⾡、⽓}
  \definition{v.}{respirar (sobre tecido, etc.) | fluir livremente (sobre ar) | respirar ar fresco | ventilar}
\end{entry}

\begin{entry}{透水}{tou4shui3}{10,4}{⾡、⽔}
  \definition{adj.}{permeável}
  \definition{s.}{vazamento de água}
\end{entry}

\begin{entry}{透支}{tou4zhi1}{10,4}{⾡、⽀}
  \definition{v.}{cheque especial (bancário) | saque a descoberto}
\end{entry}

\begin{entry}{头}{tou5}{5}{⼤}
  \definition{suf.}{adicionado após componentes nominais comuns | adicionado após o componente verbal, forma um substantivo abstrato, geralmente indicando que vale a pena realizar essa ação | adicionado após um componente adjetival, forma um substantivo, geralmente indicando algo abstrato | adicionado após o componente substantivo que indica a direção}
  \seeref{头}{tou2}
\end{entry}

\begin{entry}{突出}{tu1chu1}{9,5}{⽳、⼐}[HSK 3]
  \definition{adj.}{proeminente; excelente; mais que a média}
  \definition{v.}{romper | enfatizar; destacar; dar destaque a | sobressair; projetar-se; destacar-se}
\end{entry}

\begin{entry}{突破}{tu1po4}{9,10}{⽳、⽯}[HSK 5]
  \definition{v.}{romper; fazer uma descoberta revolucionária; concentrar esforços em um único ponto de ataque, reunir o sucesso | quebrar (limite); superar (dificuldade); superar dificuldades; ultrapassar os números ou limites anteriores, superar recordes anteriores, etc.; romper com as limitações e restrições anteriores}
\end{entry}

\begin{entry}{突然}{tu1ran2}{9,12}{⽳、⽕}[HSK 3]
  \definition{adj.}{repentino; abrupto; inesperado}
  \definition{adv.}{de repente; abruptamente; inesperadamente; subitamente}
\end{entry}

\begin{entry}{图}{tu2}{8}{⼞}[HSK 3]
  \definition*{s.}{sobrenome Tu}
  \definition[张]{s.}{mapa; gráfico; imagem; desenho | plano; esquema; tentativa}
  \definition{v.}{procurar; perseguir; esperar obter| desenhar; retratar; pintar | imaginar; planejar; pensar; maquinar}
\end{entry}

\begin{entry}{图案}{tu2'an4}{8,10}{⼞、⽊}[HSK 4]
  \definition{s.}{padrão; desenho; padrões e gráficos usados para decoração de edifícios, tecidos, artes e artesanato, etc.}
\end{entry}

\begin{entry}{图画}{tu2 hua4}{8,8}{⼞、⽥}[HSK 3]
  \definition[幅,张,套]{s.}{desenho; imagem; pintura}
\end{entry}

\begin{entry}{图片}{tu2 pian4}{8,4}{⼞、⽚}[HSK 2]
  \definition[张,幅]{s.}{imagem; fotografia; um termo geral para imagens, fotografias, decalques, etc. usados para ilustrar algo}
\end{entry}

\begin{entry}{图书馆}{tu2shu1guan3}{8,4,11}{⼞、⼄、⾷}[HSK 1]
  \definition[个,家]{s.}{biblioteca; instituição que coleta, organiza e armazena livros e materiais para leitura e consulta}
\end{entry}

\begin{entry}{徒手}{tu2shou3}{10,4}{⼻、⼿}
  \definition{adj.}{com as mãos vazias | desarmado | mão livre (desenho) | lutando mão-a-mão}
\end{entry}

\begin{entry}{途中}{tu2 zhong1}{10,4}{⾡、⼁}[HSK 4]
  \definition{adv.}{no caminho; ao longo do caminho}
\end{entry}

\begin{entry}{土}{tu3}{3}{⼟}[HSK 3][Kangxi 32]
  \definition*{s.}{sobrenome Tu}
  \definition{adj.}{local; nativo; com características regionais| caseiro; indígena; o que é tradicional no país; popular | não refinado; não esclarecido; não está na moda; não é popular}
  \definition[堆,捧,层]{s.}{solo; terra | terra; território | ópio bruto | cidade natal; terra natal; pátria}
\end{entry}

\begin{entry}{土地}{tu3di4}{3,6}{⼟、⼟}[HSK 4]
  \definition[片,块]{s.}{terra; solo; chão; superfície terrestre da Terra usada para cultivar, construir edifícios e viver | território}
  \seeref{土地}{tu3di5}
\end{entry}

\begin{entry}{土地}{tu3di5}{3,6}{⼟、⼟}
  \definition{s.}{deus da audeia; deus local; \emph{genius loci} deidade protetora de um local; (superstição) refere-se ao deus da terra que governa uma pequena área}
  \seeref{土地}{tu3di4}
\end{entry}

\begin{entry}{土豆}{tu3dou4}{3,7}{⼟、⾖}[HSK 5]
  \definition[个,片,块,斤]{s.}{batata; denominação comum da batata}
\end{entry}

\begin{entry}{土豆泥}{tu3dou4ni2}{3,7,8}{⼟、⾖、⽔}
  \definition{s.}{purê de batatas}
\end{entry}

\begin{entry}{土鸡}{tu3ji1}{3,7}{⼟、⿃}
  \definition{s.}{galinha caipira}
\end{entry}

\begin{entry}{吐}{tu3}{6}{⼝}[HSK 5]
  \definition{v.}{cuspir; sair pela boca | surgir ou aparecer pela boca ou por uma fenda | dizer; contar; falar abertamente}
  \seeref{吐}{tu4}
\end{entry}

\begin{entry}{吐}{tu4}{6}{⼝}[HSK 5]
  \definition{v.}{vomitar; sair pela boca | vomitar; expelir; metáfora para ser forçado a devolver bens usurpados}
  \seeref{吐}{tu3}
\end{entry}

\begin{entry}{兔}{tu4}{8}{⼉}[HSK 5]
  \definition[只]{s.}{lebre; coelho}
\end{entry}

\begin{entry}{兔子}{tu4zi5}{8,3}{⼉、⼦}
  \definition[只]{s.}{coelho | lebre}
\end{entry}

\begin{entry}{团}{tuan2}{6}{⼞}[HSK 3]
  \definition*{s.}{Liga da Juventude Comunista da China; Liga}
  \definition{adj.}{redondo; circular}
  \definition{clas.}{usado para algo em forma de bola}
  \definition[个]{s.}{bolinho de massa; comida em forma de bola feita de arroz ou farinha | algo em forma de bola | grupo; corpo; sociedade; organização; um grupo envolvido em um determinado trabalho ou atividade | regimento; unidade organizacional militar, geralmente abaixo do nível de divisão e acima do nível de batalhão}
  \definition{v.}{enrolar algo para formar uma bola; rolar | reunir; unir; conglomerar}
\end{entry}

\begin{entry}{团队}{tuan2dui4}{6,4}{⼞、⾩}
  \definition{s.}{equipe}
\end{entry}

\begin{entry}{团结}{tuan2jie2}{6,9}{⼞、⽷}[HSK 3]
  \definition{adj.}{unido; amigável; harmonioso; relação harmoniosa e coexistência harmoniosa}
  \definition{v.}{unir; reunir}
\end{entry}

\begin{entry}{团体}{tuan2ti3}{6,7}{⼞、⼈}[HSK 3]
  \definition[种,个]{s.}{equipe; grupo; organização; um grupo de pessoas com objetivos e interesses comuns}
\end{entry}

\begin{entry}{团长}{tuan2 zhang3}{6,4}{⼞、⾧}[HSK 5]
  \definition{s.}{comandante do regimento | chefe (ou presidente) de uma delegação, trupe, etc. | líder de uma delegação}
\end{entry}

\begin{entry}{推}{tui1}{11}{⼿}[HSK 2]
  \definition{v.}{empurrar; dar um encontrão | girar um moinho ou uma pedra de amolar; moer | cortar; aparar | impulsionar; promover; avançar | inferir; deduzir | afastar; fugir; deslocar | adiar | eleger; escolher | ter em alta estima; elogiar muito | declinar | selecionar | elogiar muito}
\end{entry}

\begin{entry}{推迟}{tui1chi2}{11,7}{⼿、⾡}[HSK 4]
  \definition{v.}{adiar; postergar; tardar; deixar para mais tarde}
\end{entry}

\begin{entry}{推动}{tui1 dong4}{11,6}{⼿、⼒}[HSK 3]
  \definition{v.}{promover; atuar; impulsionar; empurrar para a frente; dar ímpeto a; começar ou avançar algo (com alguma força); começar a trabalhar}
\end{entry}

\begin{entry}{推广}{tui1guang3}{11,3}{⼿、⼴}[HSK 3]
  \definition{v.}{espalhar; estender; promover; popularizar; expandir o escopo de uso ou função de algo}
\end{entry}

\begin{entry}{推介}{tui1jie4}{11,4}{⼿、⼈}
  \definition{s.}{promoção}
  \definition{v.}{promover | introduzir e recomendar}
\end{entry}

\begin{entry}{推进}{tui1 jin4}{11,7}{⼿、⾡}[HSK 3]
  \definition{v.}{avançar; empurrar; levar adiante; dar ímpeto a; promover o trabalho e fazê-lo avançar | empurrar; dirigir; avançar; seguir em frente; seguir em frente}
\end{entry}

\begin{entry}{推开}{tui1 kai1}{11,4}{⼿、⼶}[HSK 3]
  \definition{v.}{declinar; rejeitar | empurrar para longe; aplicar força em uma determinada direção para mover uma pessoa ou objeto para longe de seu lugar original | empurrar para abrir (um portão, etc.); empurrar para fora para abrir algo que está fechado | estender; popularizar; promover para um alcance mais amplo e realizar em uma escala mais ampla}
\end{entry}

\begin{entry}{推销}{tui1xiao1}{11,12}{⼿、⾦}[HSK 4]
  \definition{v.}{vender; comercializar; promover vendas; promover a comercialização de mercadorias}
\end{entry}

\begin{entry}{推行}{tui1 xing2}{11,6}{⼿、⾏}[HSK 5]
  \definition{v.}{realizar; prosseguir; praticar | implementar; praticar; implementação generalizada; divulgar (experiências, métodos, etc.)}
\end{entry}

\begin{entry}{腿}{tui3}{13}{⾁}[HSK 2]
  \definition[条,双]{s.}{perna; as partes dos humanos e dos animais que sustentam o corpo e permitem caminhar | um suporte em forma de perna; a parte inferior de um objeto que atua como uma perna e serve de suporte | presunto}
\end{entry}

\begin{entry}{腿号}{tui3hao4}{13,5}{⾁、⼝}
  \definition{s.}{anilha numerada (por exemplo, usada para identificar pássaros)}
  \seealsoref{腿号箍}{tui3hao4gu1}
\end{entry}

\begin{entry}{腿号箍}{tui3hao4gu1}{13,5,14}{⾁、⼝、⽵}
  \definition{s.}{anilha numerada (por exemplo, usada para identificar pássaros)}
  \seealsoref{腿号}{tui3hao4}
\end{entry}

\begin{entry}{退}{tui4}{9}{⾡}[HSK 3]
  \definition{v.}{recuar; mover-se para trás  (oposto de 進) | remover; retirar; fazer recuar; mover para trás | desistir; retirar-se de | refluir; declinar; retroceder | aposentar-se; deixar o emprego por atingir a idade estipulada ou por problemas de saúde | retornar; reembolsar; devolver | romper; cancelar o que foi decidido}
  \seealsoref{进}{jin4}
\end{entry}

\begin{entry}{退出}{tui4 chu1}{9,5}{⾡、⼐}[HSK 3]
  \definition{v.}{desistir; retirar-se; separar-se; retirar-se de; abandonar o local ou outro lugar e parar de participar; abandonaar o grupo ou organização}
\end{entry}

\begin{entry}{退休}{tui4xiu1}{9,6}{⾡、⼈}[HSK 3]
  \definition{v.+compl.}{aposentar-se; os trabalhadores que deixarem o emprego por velhice ou invalidez causada pelo trabalho receberão as despesas de subsistência conforme o cronograma}
\end{entry}

\begin{entry}{屯}{tun2}{4}{⼬}
  \definition*{s.}{sobrenome Tun}
  \definition{s.}{vila (geralmente usado em nomes de vilas); vilarejos; aldeias; povoados}
  \definition{v.}{coletar; estocar; armazenar; acumular | estacionar (tropas); aquartelar}
  \seeref{屯}{zhun1}
\end{entry}

\begin{entry}{拖拉机}{tuo1la1ji1}{8,8,6}{⼿、⼿、⽊}
  \definition[台]{s.}{trator}
\end{entry}

\begin{entry}{拖鞋}{tuo1xie2}{8,15}{⼿、⾰}
  \definition[双,只]{s.}{chinelos | sandálias}
\end{entry}

\begin{entry}{脱}{tuo1}{11}{⾁}[HSK 4]
  \definition{conj.}{se; no caso;}
  \definition{v.}{(cabelo, pele) soltar-se; desprender-se; cair | retirar peça de roupa do corpo | sair de; escapar de | perder (palavras) | livrar-se de algo}
\end{entry}

\begin{entry}{脱离}{tuo1li2}{11,10}{⾁、⼇}[HSK 5]
  \definition{v.}{separar-se; divorciar-se; afastar-se; sair (de um determinado ambiente ou situação); romper (uma determinada relação)}
\end{entry}

\begin{entry}{脱毛}{tuo1mao2}{11,4}{⾁、⽑}
  \definition{s.}{depilação}
  \definition{v.}{perder cabelo ou penas | depilar | fazer a barba}
\end{entry}

\begin{entry}{脱险}{tuo1xian3}{11,9}{⾁、⾩}
  \definition{v.}{sair do perigo}
\end{entry}

\begin{entry}{鸵鸟}{tuo2niao3}{10,5}{⿃、⿃}
  \definition{s.}{avestruz}
\end{entry}

\begin{entry}{唾骂}{tuo4ma4}{11,9}{⼝、⾺}
  \definition{v.}{insultar | amaldiçoar}
\end{entry}

%%%%% EOF %%%%%

