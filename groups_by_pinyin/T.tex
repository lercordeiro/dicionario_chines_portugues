%%%
%%% T
%%%

\section*{T}\addcontentsline{toc}{section}{T}

\begin{entry}{T-恤}{[t]-xu4}{0,9}
  \definition{s.}{camiseta | pulôver | suéter}
\end{entry}

\begin{entry}{㐌}{ta1}{5}[Radical 乙]
  \variantof{它}
\end{entry}

\begin{entry}{他}{ta1}{5}[Radical 人][HSK 1]
  \definition{pron.}{ele | se, o, lhe | si, consigo, ele}
  \seeref{怹}{tan1}
\end{entry}

\begin{entry}{他的}{ta1 de5}{5,8}
  \definition{pron.}{dele}
\end{entry}

\begin{entry}{他妈的}{ta1ma1de5}{5,6,8}
  \definition{interj.}{Dane-se! | Foda-se!}
\end{entry}

\begin{entry}{他们}{ta1men5}{5,5}[HSK 1]
  \definition{pron.}{eles | se, os, lhes | si, consigo, eles}
\end{entry}

\begin{entry}{他们的}{ta1men5 de5}{5,5,8}
  \definition{pron.}{deles}
\end{entry}

\begin{entry}{它}{ta1}{5}[Radical 宀][HSK 2]
  \definition{pron.}{ele (para objetos inanimados) | se, o, lhe | si, consigo, eles}
\end{entry}

\begin{entry}{它们}{ta1 men5}{5,5}[HSK 2]
  \definition{pron.}{eles (para objetos inanimados) | se, os, lhes | si, consigo, eles}
\end{entry}

\begin{entry}{她}{ta1}{6}[Radical 女][HSK 1]
  \definition{pron.}{ela | se, a, lhe | si, consigo, ela}
\end{entry}

\begin{entry}{她的}{ta1 de5}{6,8}
  \definition{pron.}{dela}
\end{entry}

\begin{entry}{她们}{ta1men5}{6,5}[HSK 1]
  \definition{pron.}{elas | se, as, lhes | si, consigo, elas}
\end{entry}

\begin{entry}{她们的}{ta1men5 de5}{6,5,8}
  \definition{pron.}{delas}
\end{entry}

\begin{entry}{踏板}{ta4ban3}{15,8}
  \definition{s.}{pedal (em um carro, em um piano, etc.) |  apoio para os pés | estribo}
\end{entry}

\begin{entry}{台}{tai2}{5}[Radical 口]
  \definition*{s.}{sobrenome Tai}
  \definition{clas.}{para aparelhos e máquinas}
  \definition{s.}{estação de transmissão | contador | \emph{help desk} | suporte técnico | plataforma | terraço | tufão}
\end{entry}

\begin{entry}{台风}{tai2feng1}{5,4}
  \definition{s.}{tufão}
\end{entry}

\begin{entry}{台下}{tai2xia4}{5,3}
  \definition{s.}{platéia | fora do palco}
\end{entry}

\begin{entry}{抬杠}{tai2gang4}{8,7}
  \definition{v.+compl.}{discutir pelo prazer em discutir | discutir obstinadamente | brigar}
\end{entry}

\begin{entry}{太}{tai4}{4}[Radical 大][HSK 1]
  \definition{adv.}{excessivamente | demais | muito}
\end{entry}

\begin{entry}{太极拳}{tai4ji2quan2}{4,7,10}
  \definition*{s.}{Tai Chi Chuan, Taiji, T'aichi ou T'aichichuan; forma tradicional de exercício físico ou relaxamento}
\end{entry}

\begin{entry}{太空}{tai4kong1}{4,8}
  \definition{s.}{espaço sideral | espaço exterior}
\end{entry}

\begin{entry}{太平洋}{tai4ping2 yang2}{4,5,9}
  \definition*{s.}{Oceano Pacífico}
\end{entry}

\begin{entry}{太太}{tai4tai5}{4,4}[HSK 2]
  \definition[个,位]{s.}{esposa | madame| mulher casada}
\end{entry}

\begin{entry}{太阳窗}{tai4yang2chuang1}{4,6,12}
  \definition{s.}{teto solar (de veículos)}
\end{entry}

\begin{entry}{太阳灯}{tai4yang2deng1}{4,6,6}
  \definition{s.}{lâmpada solar (com células fotovoltaicas)}
\end{entry}

\begin{entry}{太阳风}{tai4yang2feng1}{4,6,4}
  \definition{s.}{vento solar}
\end{entry}

\begin{entry}{太阳镜}{tai4yang2jing4}{4,6,16}
  \definition{s.}{óculos de sol}
\end{entry}

\begin{entry}{太阳日}{tai4yang2ri4}{4,6,4}
  \definition{s.}{dia solar}
\end{entry}

\begin{entry}{太阳穴}{tai4yang2xue2}{4,6,5}
  \definition{s.}{têmpora (nas laterais da cabeça humana)}
\end{entry}

\begin{entry}{太阳翼}{tai4yang2yi4}{4,6,17}
  \definition{s.}{painel solar}
\end{entry}

\begin{entry}{太阳雨}{tai4yang2yu3}{4,6,8}
  \definition{s.}{banho de sol}
\end{entry}

\begin{entry}{太阳}{tai4yang5}{4,6}[HSK 2]
  \definition[个]{s.}{sol | abreviação de 太阳穴}
  \seeref{太阳穴}{tai4yang2xue2}
\end{entry}

\begin{entry}{态度}{tai4du5}{8,9}[HSK 2]
  \definition[个]{s.}{maneira | comportamento | atitude | atitude | abordagem}
\end{entry}

\begin{entry}{贪婪}{tan1lan2}{8,11}
  \definition{adj.}{avaro | ambicioso | voraz | insaciável}
\end{entry}

\begin{entry}{怹}{tan1}{9}[Radical 心]
  \definition{pron.}{ele, ela (cortês, em oposição a 他)}
  \seeref{他}{ta1}
\end{entry}

\begin{entry}{谈话}{tan2hua4}{10,8}
  \definition[次]{s.}{conversa | fala | papo | declaração}
  \definition{v.+compl.}{conversar | falar | declarar}
\end{entry}

\begin{entry}{谈恋爱}{tan2lian4'ai4}{10,10,10}
  \definition{v.}{namorar | apaixonar-se}
\end{entry}

\begin{entry}{坦克}{tan3ke4}{8,7}
  \definition{s.}{(empréstimo linguístico) tanque (veículo militar)}
\end{entry}

\begin{entry}{探亲}{tan4qin1}{11,9}
  \definition{v.+compl.}{ir para casa para visitar a família}
\end{entry}

\begin{entry}{碳}{tan4}{14}[Radical 石]
  \definition{s.}{carbono (elemento químico)}
\end{entry}

\begin{entry}{汤}{tang1}{6}[Radical 水]
  \definition*{s.}{sobrenome Tang}
  \definition{s.}{sopa | caldo | decocção de ervas medicinais | água quente ou fervente | água em que algo foi fervido}
  \seeref{汤}{shang1}
\end{entry}

\begin{entry}{唐人街}{tang2ren2 jie1}{10,2,12}
  \definition*{s.}{Bairro Chinês | \emph{Chinatown}}
  \seealsoref{中国城}{zhong1guo2cheng2}
\end{entry}

\begin{entry}{糖}{tang2}{16}[Radical 米]
  \definition[颗,块]{s.}{açúcar | doces}
\end{entry}

\begin{entry}{糖醋鱼}{tang2cu4yu2}{16,15,8}
  \definition{s.}{peixe guisado em molho agridoce (prato)}
\end{entry}

\begin{entry}{倘或}{tang3huo4}{10,8}
  \definition{conj.}{se | supondo que | no caso}
\end{entry}

\begin{entry}{倘若}{tang3ruo4}{10,8}
  \definition{conj.}{se | supondo que | no caso}
\end{entry}

\begin{entry}{倘使}{tang3shi3}{10,8}
  \definition{conj.}{se | supondo que | no caso}
\end{entry}

\begin{entry}{滔天}{tao1tian1}{13,4}
  \definition{adj.}{(ondas, raiva, desastres, crimes, etc.) imponente, avassalador, imenso}
\end{entry}

\begin{entry}{逃}{tao2}{9}[Radical 辵]
  \definition{v.}{escapar | fugir}
\end{entry}

\begin{entry}{桃}{tao2}{10}[Radical 木]
  \definition{s.}{pêssego}
\end{entry}

\begin{entry}{讨论}{tao3lun4}{5,6}[HSK 2]
  \definition{v.}{discutir | falar sobre}
\end{entry}

\begin{entry}{讨生活}{tao3sheng1huo2}{5,5,9}
  \definition{v.}{ganhar a vida}
\end{entry}

\begin{entry}{套}{tao4}{10}[Radical 大][HSK 2]
  \definition{clas.}{para conjuntos, coleções}
  \definition{s.}{cobertura | fórmula | laço de corda}
  \definition{v.}{cobrir | envolver | intercalar | sobrepor}
\end{entry}

\begin{entry}{套问}{tao4wen4}{10,6}
  \definition{s.}{retórica}
  \definition{v.}{descobrir por meio de questionamento indireto diplomático}
\end{entry}

\begin{entry}{特别}{te4bie2}{10,7}[HSK 2]
  \definition{adj.}{especial | paricular | incomum}
  \definition{adv.}{especialmente | particularmente | propositalmente}
\end{entry}

\begin{entry}{特地}{te4di4}{10,6}
  \definition{adv.}{especialmente | propositalmente}
\end{entry}

\begin{entry}{特点}{te4dian3}{10,9}[HSK 2]
  \definition[个]{s.}{característica | peculiaridade | característica distintiva}
\end{entry}

\begin{entry}{特技}{te4ji4}{10,7}
  \definition{s.}{efeito especial | dublê}
\end{entry}

\begin{entry}{疼}{teng2}{10}[Radical 疒][HSK 2]
  \definition{adj.}{dolorido | doído}
  \definition{v.}{doer | amar ternamente}
\end{entry}

\begin{entry}{梯恩梯}{ti1'en1ti1}{11,10,11}
  \definition{s.}{(empréstimo linguístico) TNT, trinitrotolueno}
\end{entry}

\begin{entry}{踢}{ti1}{15}[Radical 足]
  \definition{v.}{chutar | jogar (por exemplo, futebol) | dar pontapés em}
\end{entry}

\begin{entry}{踢爆}{ti1bao4}{15,19}
  \definition{v.}{expor | revelar}
\end{entry}

\begin{entry}{踢蹋舞}{ti1ta4wu3}{15,17,14}
  \definition{s.}{sapateado | passo de dança}
\end{entry}

\begin{entry}{提}{ti2}{12}[Radical 手][HSK 2]
  \definition*{s.}{sobrenome Ti}
  \definition{s.}{concha | traço ascendente (em caracteres chineses)}
  \definition{v.}{carregar (na mão com o braço para baixo) | levantar | elevar | promover | avançar | mudar para um momento anterior | mover uma data para a frente | trazer à tona | apresentar | extrair | tirar | trazer | entregar | mencionar | referir-se a}
\end{entry}

\begin{entry}{提出}{ti2 chu1}{12,5}[HSK 2]
  \definition{v.}{levantar | propor | expor | apresentar}
\end{entry}

\begin{entry}{提到}{ti2 dao4}{12,8}[HSK 2]
  \definition{v.}{mencionar | referir-se a | levantar (assunto)}
\end{entry}

\begin{entry}{提高}{ti2gao1}{12,10}[HSK 2]
  \definition{v.}{melhorar | aumentar | elevar}
\end{entry}

\begin{entry}{提及}{ti2ji2}{12,3}
  \definition{v.}{mencionar | levantar (um assunto) | chamar a atenção de alguém}
\end{entry}

\begin{entry}{提升}{ti2sheng1}{12,4}
  \definition{v.}{promover (para uma posição de classificação mais alta) | levantar | içar | (figurativo) elevar, levantar, melhorar}
\end{entry}

\begin{entry}{题}{ti2}{15}[Radical 頁][HSK 2]
  \definition*{s.}{sobrenome Ti}
  \definition[道]{s.}{assunto | título | tópico | problema}
  \definition{v.}{inscrever | escrever}
\end{entry}

\begin{entry}{体内}{ti3nei4}{7,4}
  \definition{adj.}{dentro do corpo | \emph{in vivo} (versus \emph{in vitro} | interno a}
\end{entry}

\begin{entry}{体验}{ti3yan4}{7,10}
  \definition{v.}{vivenciar | experimentar por si mesmo}
\end{entry}

\begin{entry}{体育}{ti3yu4}{7,8}[HSK 2]
  \definition{s.}{treinamento físico | esportes | atividades esportivas}
\end{entry}

\begin{entry}{体育场}{ti3 yu4 chang3}{7,8,6}[HSK 2]
  \definition[个,座]{s.}{estádio | campo de esportes}
\end{entry}

\begin{entry}{体育馆}{ti3 yu4 guan3}{7,8,11}[HSK 2]
  \definition[个]{s.}{ginásio | estádio}
\end{entry}

\begin{entry}{天}{tian1}{4}[Radical 大][HSK 1]
  \definition{s.}{dia | céu | paraíso}
\end{entry}

\begin{entry}{天才}{tian1cai2}{4,3}
  \definition{adj.}{talentoso | superdotado | genial}
  \definition{s.}{talento | dom | gênio}
\end{entry}

\begin{entry}{天鹅}{tian1'e2}{4,12}
  \definition{s.}{cisne}
\end{entry}

\begin{entry}{天公}{tian1gong1}{4,4}
  \definition{s.}{céu, paraíso | senhor do céu}
\end{entry}

\begin{entry}{天花板}{tian1hua1ban3}{4,7,8}
  \definition{s.}{teto}
\end{entry}

\begin{entry}{天气}{tian1qi4}{4,4}[HSK 1]
  \definition{s.}{clima, tempo}
\end{entry}

\begin{entry}{天然}{tian1ran2}{4,12}
  \definition{adj.}{natural}
\end{entry}

\begin{entry}{天上}{tian1 shang4}{4,3}[HSK 2]
  \definition{s.}{o céu | paraíso}
\end{entry}

\begin{entry}{天使}{tian1shi3}{4,8}
  \definition{s.}{anjo}
\end{entry}

\begin{entry}{天堂}{tian1tang2}{4,11}
  \definition{s.}{paraíso, céu}
\end{entry}

\begin{entry}{天天}{tian1tian1}{4,4}
  \definition{adv.}{todo dia}
\end{entry}

\begin{entry}{天下}{tian1xia4}{4,3}
  \definition{s.}{terra sob o céu | o mundo todo | toda a China | reino}
\end{entry}

\begin{entry}{天择}{tian1ze2}{4,8}
  \definition{s.}{seleção natural}
\end{entry}

\begin{entry}{天柱}{tian1zhu4}{4,9}
  \definition{s.}{pilar celestial, que sustenta o céu}
\end{entry}

\begin{entry}{兲}{tian1}{6}[Radical 八]
  \variantof{天}
\end{entry}

\begin{entry}{田}{tian2}{5}[Radical 田][Kangxi 102]
  \definition*{s.}{sobrenome Tian}
  \definition[片]{s.}{fazenda | campo}
\end{entry}

\begin{entry}{田园}{tian2yuan2}{5,7}
  \definition{adj.}{bucólico}
  \definition{s.}{campo | interior | rural}
\end{entry}

\begin{entry}{钿}{tian2}{10}[Radical 金]
  \definition{s.}{(dialeto) moeda, dinheiro}
  \seeref{钿}{dian4}
\end{entry}

\begin{entry}{甜}{tian2}{11}[Radical 甘]
  \definition{adj.}{doce}
\end{entry}

\begin{entry}{甜酒}{tian2jiu3}{11,10}
  \definition{s.}{licor doce}
\end{entry}

\begin{entry}{甜菊}{tian2ju2}{11,11}
  \definition{s.}{estévia, arbusto cujas folhas produzem um substituto para o açúcar}
\end{entry}

\begin{entry}{甜品}{tian2pin3}{11,9}
  \definition{s.}{sobremesa}
\end{entry}

\begin{entry}{甜食}{tian2shi2}{11,9}
  \definition{s.}{doces | sobremesa}
\end{entry}

\begin{entry}{甜酸}{tian2suan1}{11,14}
  \definition{adj.}{agridoce}
\end{entry}

\begin{entry}{甜甜圈}{tian2tian2quan1}{11,11,11}
  \definition{s.}{rosquinha | \emph{doughnut}}
\end{entry}

\begin{entry}{甜筒}{tian2tong3}{11,12}
  \definition{s.}{sorvete de casquinha}
\end{entry}

\begin{entry}{甜头}{tian2tou5}{11,5}
  \definition{s.}{benefício | sabor doce (de poder, sucesso, etc.)}
\end{entry}

\begin{entry}{甜心}{tian2xin1}{11,4}
  \definition{s.}{querido}
\end{entry}

\begin{entry}{甜言}{tian2yan2}{11,7}
  \definition{s.}{boa conversa | palavras amáveis}
\end{entry}

\begin{entry}{甜玉米}{tian2 yu4mi3}{11,5,6}
  \definition{s.}{milho doce}
\end{entry}

\begin{entry}{甜稚}{tian2zhi4}{11,13}
  \definition{s.}{doce e inocente}
\end{entry}

\begin{entry}{条}{tiao2}{7}[Radical 木][HSK 2]
  \definition{clas.}{para coisas longas e finas (fita, rio, estrada, calças, etc.)}
  \definition{s.}{artigo | cláusula (de lei ou tratado) | item | faixa}
\end{entry}

\begin{entry}{条幅}{tiao2fu2}{7,12}
  \definition{s.}{faixa | banner | pergaminho de parede (para pintura ou caligrafia)}
\end{entry}

\begin{entry}{条贯}{tiao2guan4}{7,8}
  \definition{s.}{ordem | procedimentos | sequência | sistema}
\end{entry}

\begin{entry}{条件}{tiao2jian4}{7,6}[HSK 2]
  \definition[个]{s.}{circunstâncias | condição | fator | pré-requisito | qualificação | requisito}
\end{entry}

\begin{entry}{条例}{tiao2li4}{7,8}
  \definition{s.}{código de conduta | ordenanças | regulamentos | regras | estatutos}
\end{entry}

\begin{entry}{条目}{tiao2mu4}{7,5}
  \definition{s.}{cláusulas e subcláusulas (em documento formal) | verbete (em um dicionário, enciclopédia, etc.)}
\end{entry}

\begin{entry}{调律}{tiao2lv4}{10,9}
  \definition{v.}{afinar (por exemplo, um piano)}
\end{entry}

\begin{entry}{挑衅}{tiao3xin4}{9,11}
  \definition{s.}{provocação}
  \definition{v.}{provocar}
\end{entry}

\begin{entry}{跳}{tiao4}{13}[Radical 足]
  \definition{v.}{pular | saltar}
\end{entry}

\begin{entry}{跳挡}{tiao4dang3}{13,9}
  \definition{v.}{pular marcha (de um carro) | perder a marcha}
\end{entry}

\begin{entry}{跳电}{tiao4dian4}{13,5}
  \definition{v.}{desarmar (um disjuntor ou interruptor)}
\end{entry}

\begin{entry}{跳频}{tiao4pin2}{13,13}
  \definition{s.}{FHSS, \emph{Frequency-Hopping Spread Spectrum}, método de transmissão de sinais de rádio}
\end{entry}

\begin{entry}{跳伞}{tiao4san3}{13,6}
  \definition{s.}{paraquedas}
  \definition{v.}{saltar de paraquedas}
\end{entry}

\begin{entry}{跳绳}{tiao4sheng2}{13,11}
  \definition{v.}{pular corda}
\end{entry}

\begin{entry}{跳水}{tiao4shui3}{13,4}
  \definition{s.}{mergulho esportivo}
  \definition{v.}{mergulhar (na água) | cometer suicídio pulando na água | (figurativo, preços das ações, etc.) cair dramaticamente}
\end{entry}

\begin{entry}{跳跳糖}{tiao4tiao4tang2}{13,13,16}
  \definition{s.}{\emph{Pop Rocks}, \emph{popping candy}}
\end{entry}

\begin{entry}{跳舞}{tiao4wu3}{13,14}
  \definition{v.+compl.}{dançar}
\end{entry}

\begin{entry}{跳远}{tiao4yuan3}{13,7}
  \definition{v.+compl.}{salto em distância (atletismo)}
\end{entry}

\begin{entry}{跳蚤}{tiao4zao5}{13,9}
  \definition{s.}{pulga}
\end{entry}

\begin{entry}{铁}{tie3}{10}[Radical 金]
  \definition*{s.}{sobrenome Tie}
  \definition{adj.}{duro | forte | violento | inabalável | determinado | (gíria) apertado}
  \definition{s.}{ferro (metal) | arma}
\end{entry}

\begin{entry}{铁轨}{tie3gui3}{10,6}
  \definition[根]{s.}{trilho | trilho ferroviário}
\end{entry}

\begin{entry}{铁路}{tie3lu4}{10,13}
  \definition[条]{s.}{ferrovia}
\end{entry}

\begin{entry}{听}{ting1}{7}[Radical 口][HSK 1]
  \definition{clas.}{para bebidas enlatadas}
  \definition{s.}{lata de bebida (empréstimo linguístico, do inglês ``\emph{tin}'')}
  \definition{v.}{ouvir | escutar | obedecer}
\end{entry}

\begin{entry}{听到}{ting1dao4}{7,8}[HSK 1]
  \definition{v.}{ouvir | notar}
\end{entry}

\begin{entry}{听断}{ting1duan4}{7,11}
  \definition{v.}{ouvir e decidir | julgar (ou seja, ouvir e julgar em um tribunal)}
\end{entry}

\begin{entry}{听骨}{ting1gu3}{7,9}
  \definition{s.}{ossículos (do ouvido médio)}
  \seealsoref{听小骨}{ting1xiao3gu3}
\end{entry}

\begin{entry}{听会}{ting1hui4}{7,6}
  \definition{v.}{participar de uma reunião (e ouvir o que é discutido)}
\end{entry}

\begin{entry}{听见}{ting1 jian4}{7,4}[HSK 1]
  \definition{v.}{ouvir}
\end{entry}

\begin{entry}{听讲}{ting1 jiang3}{7,6}[HSK 2]
  \definition{v.+compl.}{assistir a uma palestra; ouvir uma conversa}
\end{entry}

\begin{entry}{听来}{ting1lai2}{7,7}
  \definition{v.}{ouvir de algum lugar | soar (antigo, estrangeiro, excitante, certo, etc.) | soar como se (ou seja, dar uma impressão ao ouvinte)}
\end{entry}

\begin{entry}{听力}{ting1li4}{7,2}
  \definition{s.}{audição | capacidade de compreensão oral}
\end{entry}

\begin{entry}{听力理解}{ting1li4li3jie3}{7,2,11,13}
  \definition{s.}{compreensão auditiva}
\end{entry}

\begin{entry}{听命}{ting1ming4}{7,8}
  \definition{v.}{obedecer ordens | receber ordens}
\end{entry}

\begin{entry}{听凭}{ting1ping2}{7,8}
  \definition{v.}{permitir (alguém a fazer o que desejar)}
\end{entry}

\begin{entry}{听说}{ting1 shuo1}{7,9}[HSK 2]
  \definition{v.}{ouvir dizer}
\end{entry}

\begin{entry}{听随}{ting1sui2}{7,11}
  \definition{v.}{permitir | obedecer}
\end{entry}

\begin{entry}{听戏}{ting1xi4}{7,6}
  \definition{v.}{assistir a uma ópera | ver uma ópera}
\end{entry}

\begin{entry}{听小骨}{ting1xiao3gu3}{7,3,9}
  \definition{s.}{ossículos (do ouvido médio)}
  \seealsoref{听骨}{ting1gu3}
\end{entry}

\begin{entry}{听写}{ting1xie3}{7,5}[HSK 1]
  \definition{s.}{ditado}
  \definition{v.}{transcrever música de ouvido | escrever (em um exercício de ditado)}
\end{entry}

\begin{entry}{聼}{ting1}{19}[Radical 耳]
  \variantof{听}
\end{entry}

\begin{entry}{亭}{ting2}{9}[Radical 亠]
  \definition{s.}{pavilhão | cabine | quiosque}
\end{entry}

\begin{entry}{停}{ting2}{11}[Radical 人][HSK 2]
  \definition{v.}{parar | estacionar (um carro)}
\end{entry}

\begin{entry}{停办}{ting2ban4}{11,4}
  \definition{v.}{cancelar | sair do negócio | desligar | terminar}
\end{entry}

\begin{entry}{停车}{ting2 che1}{11,4}[HSK 2]
  \definition{v.}{parar de trabalhar (uma máquina) | estacionar | parar (um veículo) | paralisar}
\end{entry}

\begin{entry}{停车场}{ting2 che1 chang3}{11,4,6}[HSK 2]
  \definition{s.}{parque de estacionamento}
\end{entry}

\begin{entry}{停当}{ting2dang5}{11,6}
  \definition{adj.}{realizado | preparado | assentado}
\end{entry}

\begin{entry}{停电}{ting2dian4}{11,5}
  \definition{s.}{corte de energia}
  \definition{v.}{ter uma falha de energia}
\end{entry}

\begin{entry}{停工}{ting2gong1}{11,3}
  \definition{v.}{parar de trabalhar | parar a produção}
\end{entry}

\begin{entry}{停火}{ting2huo3}{11,4}
  \definition{s.}{cessar-fogo}
  \definition{v.+compl.}{cessar fogo}
\end{entry}

\begin{entry}{停课}{ting2ke4}{11,10}
  \definition{v.}{fechar (escola) | parar as aulas}
\end{entry}

\begin{entry}{停留}{ting2liu2}{11,10}
  \definition{v.}{ficar em algum lugar temporariamente | demorar | permanecer}
\end{entry}

\begin{entry}{停息}{ting2xi1}{11,10}
  \definition{v.}{cessar | parar}
\end{entry}

\begin{entry}{停歇}{ting2xie1}{11,13}
  \definition{v.}{parar para descansar}
\end{entry}

\begin{entry}{停业}{ting2ye4}{11,5}
  \definition{v.}{cessar a negociação (temporária ou permanentemente) | fechar}
\end{entry}

\begin{entry}{停用}{ting2yong4}{11,5}
  \definition{v.}{desabilitar | descontinuar | parar de usar | suspender}
\end{entry}

\begin{entry}{停止}{ting2zhi3}{11,4}
  \definition{v.}{cessar | encerrar | parar}
\end{entry}

\begin{entry}{挺}{ting3}{9}[Radical 手][HSK 2]
  \definition{adj.}{ereto | fora do comum | direto}
  \definition{adv.}{bastante, ou melhor, bonito | muito (coloquial)}
  \definition{clas.}{para metralhadoras}
  \definition{v.}{endireitar (fisicamente) | sobressair (uma parte do corpo) | dar suporte | resistir}
\end{entry}

\begin{entry}{挺拔}{ting3ba2}{9,8}
  \definition{adj.}{alto e reto}
\end{entry}

\begin{entry}{挺杆}{ting3gan3}{9,7}
  \definition{s.}{tucho (peça de máquina)}
\end{entry}

\begin{entry}{挺过}{ting3guo4}{9,6}
  \definition{s.}{sobreviver}
\end{entry}

\begin{entry}{挺好}{ting3 hao3}{9,6}[HSK 2]
  \definition{adj.}{muito bom}
\end{entry}

\begin{entry}{挺进}{ting3jin4}{9,7}
  \definition{s.}{progresso | avanço}
  \definition{v.}{progredir | avançar}
\end{entry}

\begin{entry}{挺立}{ting3li4}{9,5}
  \definition{v.}{ficar ereto | ficar de pé}
\end{entry}

\begin{entry}{挺身}{ting3shen1}{9,7}
  \definition{v.}{endireitar as costas}
\end{entry}

\begin{entry}{挺尸}{ting3shi1}{9,3}
  \definition{v.}{(coloquial) dormir | (literalmente) ficar deitado duro como um cadáver}
\end{entry}

\begin{entry}{挺腰}{ting3yao1}{9,13}
  \definition{v.}{arquear as costas | endireitar as costas}
\end{entry}

\begin{entry}{挺住}{ting3zhu4}{9,7}
  \definition{v.}{permanecer firme | manter-se firme (diante da adversidade ou da dor)}
\end{entry}

\begin{entry}{通}{tong1}{10}[Radical 辵][HSK 2]
  \definition{clas.}{para cartas, telegramas, telefonemas, etc.}
  \definition{suf.}{especialista}
  \definition{v.}{ligar para | conseguir a ligação}
  \seeref{通}{tong4}
\end{entry}

\begin{entry}{通牒}{tong1die2}{10,13}
  \definition{s.}{nota diplomática}
\end{entry}

\begin{entry}{通观}{tong1guan1}{10,6}
  \definition{v.}{ter uma visão geral de algo}
\end{entry}

\begin{entry}{通过}{tong1guo4}{10,6}[HSK 2]
  \definition{adv.}{por meio de | através de | via}
  \definition{v.}{passar por | adotar (uma resolução), aprovar (legislação) | passar (em um teste)}
\end{entry}

\begin{entry}{通识}{tong1shi2}{10,7}
  \definition{s.}{conhecimento comum | erudição | conhecimento geral | amplamente conhecido}
\end{entry}

\begin{entry}{通知}{tong1zhi1}{10,8}[HSK 2]
  \definition[份,个,张]{s.}{aviso | circular}
  \definition{v.}{aconselhar | notificar | informar | dar aviso}
\end{entry}

\begin{entry}{同}{tong2}{6}[Radical 口]
  \definition{adj.}{junto}
  \definition{adv.}{junto com}
\end{entry}

\begin{entry}{同伙}{tong2huo3}{6,6}
  \definition[个]{s.}{cúmplice | colega}
\end{entry}

\begin{entry}{同流合污}{tong2liu2he2wu1}{6,10,6,6}
  \definition{expr.}{chafurdar na lama com alguém | seguir o mau exemplo dos outros}
\end{entry}

\begin{entry}{同情}{tong2qing2}{6,11}
  \definition{s.}{simpatia}
  \definition{v.}{simpatizar com}
\end{entry}

\begin{entry}{同时}{tong2shi2}{6,7}[HSK 2]
  \definition{conj.}{além disso}
  \definition{s.}{enquanto isso | ao mesmo tempo}
\end{entry}

\begin{entry}{同事}{tong2shi4}{6,8}[HSK 2]
  \definition{s.}{colega | colega de trabalho | companheiro}
\end{entry}

\begin{entry}{同屋}{tong2wu1}{6,9}
  \definition[个]{s.}{companheiro de quarto | colega de quarto}
\end{entry}

\begin{entry}{同性恋}{tong2xing4lian4}{6,8,10}
  \definition{s.}{homossexualidade | pessoa gay | amor gay}
\end{entry}

\begin{entry}{同学}{tong2xue2}{6,8}[HSK 1]
  \definition[位,个]{s.}{colega de classe | colega estudante}
\end{entry}

\begin{entry}{同砚}{tong2yan4}{6,9}
  \definition[位,个]{s.}{colega de classe | colega estudante}
\end{entry}

\begin{entry}{同样}{tong2 yang4}{6,10}[HSK 2]
  \definition{adj.}{igual | similar}
\end{entry}

\begin{entry}{同意}{tong2yi4}{6,13}
  \definition{v.}{concordar | aprovar | consentir}
\end{entry}

\begin{entry}{童年}{tong2nian2}{12,6}
  \definition{s.}{infância}
\end{entry}

\begin{entry}{通}{tong4}{10}[Radical 辵]
  \definition{clas.}{para uma atividade, tomada em sua totalidade (discurso de abuso, período de reprodução de música, bebedeira, etc.)}
  \seeref{通}{tong1}
\end{entry}

\begin{entry}{痛骂}{tong4ma4}{12,9}
  \definition{v.}{repreender severamente}
\end{entry}

\begin{entry}{偷}{tou1}{11}[Radical 人]
  \definition{adv.}{furtivamente}
  \definition{v.}{furtar | roubar}
\end{entry}

\begin{entry}{偷安}{tou1'an1}{11,6}
  \definition{v.}{buscar facilidade temporária}
\end{entry}

\begin{entry}{偷渡}{tou1du4}{11,12}
  \definition{s.}{contrabando | imigração ilegal | clandestino (em um navio)}
  \definition{v.}{executar um bloqueio | roubar através da fronteira internacional}
\end{entry}

\begin{entry}{偷窃}{tou1qie4}{11,9}
  \definition{v.}{furtar | roubar}
\end{entry}

\begin{entry}{偷情}{tou1qing2}{11,11}
  \definition{v.}{manter um caso de amor clandestino}
\end{entry}

\begin{entry}{偷税}{tou1shui4}{11,12}
  \definition{s.}{evasão fiscal}
\end{entry}

\begin{entry}{偷听}{tou1ting1}{11,7}
  \definition{v.}{bisbilhotar; monitorar (secretamente)}
\end{entry}

\begin{entry}{偷袭}{tou1xi2}{11,11}
  \definition{s.}{ataque surpresa}
  \definition{v.}{montar um ataque furtivo | invadir}
\end{entry}

\begin{entry}{偸}{tou1}{11}[Radical 亻]
  \variantof{偷}
\end{entry}

\begin{entry}{头}{tou2}{5}[Radical 大][HSK 2]
  \definition{clas.}{para suínos ou gado}
  \definition[个]{s.}{cabeça}
  \seeref{头}{tou5}
\end{entry}

\begin{entry}{头发}{tou2fa5}{5,5}[HSK 2]
  \definition{s.}{cabelo}
\end{entry}

\begin{entry}{头号}{tou2hao4}{5,5}
  \definition{adj.}{primeira classe | número um | \emph{top rank}}
\end{entry}

\begin{entry}{头脑风暴}{tou2nao3feng1bao4}{5,10,4,15}
  \definition{s.}{\emph{brainstorm}}
\end{entry}

\begin{entry}{头头}{tou2tou2}{5,5}
  \definition{s.}{chefe | o cabeça}
\end{entry}

\begin{entry}{头像}{tou2xiang4}{5,13}
  \definition{s.}{retrato | busto | avatar | imagem de perfil (computação)}
\end{entry}

\begin{entry}{投递}{tou2di4}{7,10}
  \definition{v.}{despachar | enviar}
\end{entry}

\begin{entry}{投票}{tou2piao4}{7,11}
  \definition{v.+compl.}{votar | depositar um voto}
\end{entry}

\begin{entry}{投资}{tou2zi1}{7,10}
  \definition{s.}{investimento}
  \definition{v.}{investir}
\end{entry}

\begin{entry}{投资风险}{tou2zi1feng1xian3}{7,10,4,9}
  \definition{s.}{risco de investimento}
\end{entry}

\begin{entry}{投资回报率}{tou2zi1hui2bao4lv4}{7,10,6,7,11}
  \definition{s.}{retorno sobre o investimento (ROI)}
\end{entry}

\begin{entry}{投资家}{tou2zi1jia1}{7,10,10}
  \definition{s.}{investidor}
  \seealsoref{投资人}{tou2zi1ren2}
  \seealsoref{投资者}{tou2zi1zhe3}
\end{entry}

\begin{entry}{投资人}{tou2zi1ren2}{7,10,2}
  \definition{s.}{investidor}
  \seealsoref{投资家}{tou2zi1jia1}
  \seealsoref{投资者}{tou2zi1zhe3}
\end{entry}

\begin{entry}{投资者}{tou2zi1zhe3}{7,10,8}
  \definition{s.}{investidor}
  \seealsoref{投资家}{tou2zi1jia1}
  \seealsoref{投资人}{tou2zi1ren2}
\end{entry}

\begin{entry}{透}{tou4}{10}[Radical 辵]
  \definition{adj.}{completo | total}
  \definition{adv.}{completamente | totalmente}
  \definition{v.}{aparecer | passar através | penetrar}
\end{entry}

\begin{entry}{透彻}{tou4che4}{10,7}
  \definition{adj.}{minucioso | incisivo | penetrante}
\end{entry}

\begin{entry}{透澈}{tou4che4}{10,15}
  \variantof{透彻}
\end{entry}

\begin{entry}{透顶}{tou4ding3}{10,8}
  \definition{adv.}{completamente}
\end{entry}

\begin{entry}{透过}{tou4guo4}{10,6}
  \definition{v.}{passar através | penetrar}
\end{entry}

\begin{entry}{透亮}{tou4liang4}{10,9}
  \definition{adj.}{brilhante | claro como cristal}
\end{entry}

\begin{entry}{透露}{tou4lu4}{10,21}
  \definition{v.}{divulgar | vazar | revelar}
\end{entry}

\begin{entry}{透明}{tou4ming2}{10,8}
  \definition{adj.}{transparente | (figurativo) transparente, aberto a escrutínio}
\end{entry}

\begin{entry}{透辟}{tou4pi4}{10,13}
  \definition{adj.}{incisivo | penetrante}
\end{entry}

\begin{entry}{透气}{tou4qi4}{10,4}
  \definition{v.}{respirar (sobre tecido, etc.) | fluir livremente (sobre ar) | respirar ar fresco | ventilar}
\end{entry}

\begin{entry}{透水}{tou4shui3}{10,4}
  \definition{adj.}{permeável}
  \definition{s.}{vazamento de água}
\end{entry}

\begin{entry}{透支}{tou4zhi1}{10,4}
  \definition{v.}{cheque especial (bancário) | saque a descoberto}
\end{entry}

\begin{entry}{头}{tou5}{5}[Radical 大]
  \definition{suf.}{sufixo para nomes}
  \seeref{头}{tou2}
\end{entry}

\begin{entry}{突然}{tu1ran2}{9,12}
  \definition{adv.}{de repente | abruptamente | inesperadamente}
\end{entry}

\begin{entry}{图}{tu2}{8}[Radical 囗]
  \definition[张]{s.}{diagrama | imagem | desenho | gráfico | mapa}
  \definition{v.}{planejar | esquematizar | tentar | perseguir | procurar}
\end{entry}

\begin{entry}{图片}{tu2 pian4}{8,4}[HSK 2]
  \definition[张,幅]{s.}{imagem | fotografia}
\end{entry}

\begin{entry}{图书馆}{tu2shu1guan3}{8,4,11}[HSK 1]
  \definition[家,个]{s.}{biblioteca}
\end{entry}

\begin{entry}{徒手}{tu2shou3}{10,4}
  \definition{adj.}{com as mãos vazias | desarmado | mão livre (desenho) | lutando mão-a-mão}
\end{entry}

\begin{entry}{土地}{tu3di4}{3,6}
  \definition[片,块]{s.}{terra | solo | território}
  \seeref{土地}{tu3di4}
\end{entry}

\begin{entry}{土地}{tu3di5}{3,6}
  \definition{s.}{deus local | \emph{genius loci} deidade protetora de um local}
  \seeref{土地}{tu3di4}
\end{entry}

\begin{entry}{土豆}{tu3dou4}{3,7}
  \definition[个,颗]{s.}{batata}
\end{entry}

\begin{entry}{土豆泥}{tu3dou4ni2}{3,7,8}
  \definition{s.}{purê de batatas}
\end{entry}

\begin{entry}{土鸡}{tu3ji1}{3,7}
  \definition{s.}{galinha caipira}
\end{entry}

\begin{entry}{吐}{tu3}{6}[Radical 口]
  \definition{v.}{cuspir | enviar (seda de um bicho-da-seda, cápsulas de flores de algodão etc.) | dizer | despejar (suas queixas)}
  \seeref{吐}{tu4}
\end{entry}

\begin{entry}{吐}{tu4}{6}[Radical 口]
  \definition{v.}{vomitar}
  \seeref{吐}{tu3}
\end{entry}

\begin{entry}{兔子}{tu4zi5}{8,3}
  \definition[只]{s.}{coelho | lebre}
\end{entry}

\begin{entry}{团队}{tuan2dui4}{6,4}
  \definition{s.}{equipe}
\end{entry}

\begin{entry}{团结}{tuan2jie2}{6,9}
  \definition{adj.}{unido}
  \definition{s.}{unidade | solidariedade}
  \definition{v.}{unir}
\end{entry}

\begin{entry}{推}{tui1}{11}[Radical 手][HSK 2]
  \definition{v.}{empurrar | girar um moinho ou uma pedra de amolar | moer | impulsionar | promover | avançar | estender | deduzir | inferir | declinar | empurrar para longe | deslocar | adiar | diferir | eleger | selecionar | escolher | ter em alta estima | elogiar muito}
\end{entry}

\begin{entry}{推迟}{tui1chi2}{11,7}
  \definition{v.}{adiar | deixar para mais tarde | tardar}
\end{entry}

\begin{entry}{推介}{tui1jie4}{11,4}
  \definition{s.}{promoção}
  \definition{v.}{promover | introduzir e recomendar}
\end{entry}

\begin{entry}{腿}{tui3}{13}[Radical 肉][HSK 2]
  \definition[条]{s.}{perna | osso do quadril}
\end{entry}

\begin{entry}{腿号}{tui3hao4}{13,5}
  \definition{s.}{anilha numerada (por exemplo, usada para identificar pássaros)}
  \seealsoref{腿号箍}{tui3hao4gu1}
\end{entry}

\begin{entry}{腿号箍}{tui3hao4gu1}{13,5,14}
  \definition{s.}{anilha numerada (por exemplo, usada para identificar pássaros)}
  \seealsoref{腿号}{tui3hao4}
\end{entry}

\begin{entry}{退休}{tui4xiu1}{9,6}
  \definition{v.+compl.}{aposentar-se}
\end{entry}

\begin{entry}{拖拉机}{tuo1la1ji1}{8,8,6}
  \definition[台]{s.}{trator}
\end{entry}

\begin{entry}{拖鞋}{tuo1xie2}{8,15}
  \definition[双,只]{s.}{chinelos | sandálias}
\end{entry}

\begin{entry}{脱毛}{tuo1mao2}{11,4}
  \definition{s.}{depilação}
  \definition{v.}{perder cabelo ou penas | depilar | fazer a barba}
\end{entry}

\begin{entry}{脱险}{tuo1xian3}{11,9}
  \definition{v.}{sair do perigo}
\end{entry}

\begin{entry}{鸵鸟}{tuo2niao3}{10,5}
  \definition{s.}{avestruz}
\end{entry}

\begin{entry}{唾骂}{tuo4ma4}{11,9}
  \definition{v.}{insultar | amaldiçoar}
\end{entry}

%%%%% EOF %%%%%

