%%%
%%% S
%%%

\section*{S}\addcontentsline{toc}{section}{S}

\begin{entry}{撒旦}{sa1dan4}{15,5}{⼿、⽇}
  \definition*{s.}{Satã}
\end{entry}

\begin{entry}{撒旦主义}{sa1dan4 zhu3yi4}{15,5,5,3}{⼿、⽇、⼂、⼂}
  \definition*{s.}{Satanismo}
\end{entry}

\begin{entry}{撒但}{sa1dan4}{15,7}{⼿、⼈}
  \variantof{撒旦}
\end{entry}

\begin{entry}{洒}{sa3}{9}{⽔}[HSK 5]
  \definition{adj.}{natural e sem restrições; confortável (sem restrições)}
  \definition{v.}{derramar; espalhar; borrifar; salpicar; fazer com que (água ou outra coisa) caia de forma dispersa | derramar; cair de forma dispersa}
\end{entry}

\begin{entry}{洒水}{sa3shui3}{9,4}{⽔、⽔}
  \definition{v.}{borrifar}
\end{entry}

\begin{entry}{飒飒}{sa4sa4}{9,9}{⾵、⾵}
  \definition{s.}{o farfalhar | sussurro | murmúrio (do vento nas árvores, o mar, etc.)}
\end{entry}

\begin{entry}{赛}{sai4}{14}{⾙}
  \definition{s.}{competição}
  \definition{v.}{competir | superar | destacar-se}
\end{entry}

\begin{entry}{赛车}{sai4che1}{14,4}{⾙、⾞}
  \definition{s.}{corrida de automóvel | corrida de bicicleta | carro de corrida}
\end{entry}

\begin{entry}{三}{san1}{3}{⼀}[HSK 1]
  \definition*{s.}{sobrenome San}
  \definition{num.}{três; 3 | muitos; vários; mais de dois; referindo-se a muitos ou à maioria | alguns; poucos; menos; não muitos}
\end{entry}

\begin{entry}{三角}{san1jiao3}{3,7}{⼀、⾓}
  \definition{s.}{triângulo}
\end{entry}

\begin{entry}{三角恋爱}{san1jiao3lian4'ai4}{3,7,10,10}{⼀、⾓、⼼、⽖}
  \definition{s.}{triângulo amoroso}
\end{entry}

\begin{entry}{三轮车}{san1lun2che1}{3,8,4}{⼀、⾞、⾞}
  \definition{s.}{triciclo}
\end{entry}

\begin{entry}{三明治}{san1ming2zhi4}{3,8,8}{⼀、⽇、⽔}
  \definition{s.}{(empréstimo linguístico) sanduíche}
\end{entry}

\begin{entry}{伞}{san3}{6}{⼈}[HSK 4]
  \definition*{s.}{sobrenome San}
  \definition[把]{s.}{guarda-chuva; proteção contra chuva ou sol | algo que tem o formato de um guarda-chuva}
\end{entry}

\begin{entry}{散}{san3}{12}{⽁}[HSK 5]
  \definition{adj.}{disperso; fragmentado; não integrado}
  \definition{s.}{medicamento em forma de pó}
  \definition{v.}{divergir; espalhar-se; separar-se; soltar-se; não se manter unido;  desintegrar}
  \seeref{散}{san4}
\end{entry}

\begin{entry}{散文}{san3wen2}{12,4}{⽁、⽂}[HSK 5]
  \definition[个]{s.}{ensaio; prosa; gênero literário, na antiguidade, referia-se a textos em prosa, em oposição à poesia e à prosa paralela; atualmente, refere-se a obras literárias que não sejam poesia, teatro ou romance, incluindo ensaios, contos, crônicas, relatos de viagem, etc.}
\end{entry}

\begin{entry}{散}{san4}{12}{⽁}
  \definition{v.}{quebrar; fragmentar; dispersar | dar; distribuir; disseminar; divulgar | dissipar; deixar sai  | terminar um acordo ou contrato; demitir}
  \seeref{散}{san3}
\end{entry}

\begin{entry}{散步}{san4bu4}{12,7}{⽁、⽌}[HSK 3]
  \definition{v.+compl.}{dar uma volta; dar um passeio; dar uma caminhada}
\end{entry}

\begin{entry}{散心}{san4xin1}{12,4}{⽁、⼼}
  \definition{v.+compl.}{aliviar o tédio | desfrutar de uma diversão | estar despreocupado}
\end{entry}

\begin{entry}{丧钟}{sang1zhong1}{8,9}{⼗、⾦}
  \definition{s.}{sentença de morte}
\end{entry}

\begin{entry}{桑}{sang1}{10}{⽊}
  \definition*{s.}{sobrenome Sang}
  \definition{s.}{amoreira}
\end{entry}

\begin{entry}{桑巴舞}{sang1ba1wu3}{10,4,14}{⽊、⼰、⾇}
  \definition{s.}{samba}
\end{entry}

\begin{entry}{桑树}{sang1shu4}{10,9}{⽊、⽊}
  \definition{s.}{amoreira, suas folhas são utilizadas para alimentar bichos-da-seda}
\end{entry}

\begin{entry}{骚乱}{sao1luan4}{12,7}{⾺、⼄}
  \definition{s.}{rebelião | perturbação | tumulto}
  \definition{v.}{criar um distúrbio}
\end{entry}

\begin{entry}{扫}{sao3}{6}{⼿}[HSK 4]
  \definition{v.}{varrer; limpar | passar rapidamente ao longo ou sobre; varrer | juntar tudo}
  \seeref{扫}{sao4}
\end{entry}

\begin{entry}{扫兴}{sao3xing4}{6,6}{⼿、⼋}
  \definition{v.+compl.}{sentir-se decepcionado | entristecer alguém}
\end{entry}

\begin{entry}{嫂子}{sao3zi5}{12,3}{⼥、⼦}
  \definition{s.}{esposa do irmão mais velho}
\end{entry}

\begin{entry}{扫}{sao4}{6}{⼿}
  \seeref{扫}{sao3}
  \seealsoref{扫帚}{sao4zhou5}
\end{entry}

\begin{entry}{扫帚}{sao4zhou5}{6,8}{⼿、⼱}
  \definition[把]{s.}{vassoura; ferramenta de varredura feita de varas de bambu, etc., maior que uma vassora}
\end{entry}

\begin{entry}{色}{se4}{6}{⾊}[HSK 4][Kangxi 139]
  \definition[种]{s.}{cor | aparência; semblante; expressão | tipo; gênero; descrição | cena; cenário;  paisagem | qualidade (de metais preciosos, mercadorias, etc.) | aparência feminina; beleza feminina}
  \seeref{色}{shai3}
\end{entry}

\begin{entry}{色彩}{se4cai3}{6,11}{⾊、⼺}[HSK 4]
  \definition[种,丝]{s.}{cor; matiz; tonalidade | cor; sabor; característica; metáfora para um determinado estado de espírito ou tendência de pensamento}
\end{entry}

\begin{entry}{色狼}{se4lang2}{6,10}{⾊、⽝}
  \definition*{s.}{Sátiro}
  \definition{adj.}{depravado | tarado}
\end{entry}

\begin{entry}{森林}{sen1lin2}{12,8}{⽊、⽊}[HSK 4]
  \definition[片,座,处]{s.}{floresta; bosque; normalmente, refere-se a uma grande área de árvores em crescimento; na silvicultura, refere-se a um grande número de árvores que crescem em uma área razoavelmente grande de terra, juntamente com os animais e outras plantas}
\end{entry}

\begin{entry}{僧}{seng1}{14}{⼈}
  \definition{s.}{monge Budista, abreviação de 僧伽}
  \seealsoref{僧伽}{seng1qie2}
\end{entry}

\begin{entry}{僧伽}{seng1qie2}{14,7}{⼈、⼈}
  \definition{s.}{sangha ou sanga (Budismo) | a comunidade monástica | monge}
\end{entry}

\begin{entry}{杀}{sha1}{6}{⽊}[HSK 5]
  \definition{adv.}{em extremo; excessivamente; usado após um verbo, indica grau intenso}
  \definition{v.}{matar; abater; esquartejar | lutar; entrar em batalha | enfraquecer; reduzir; diminuir | decolar; neutralizar}
\end{entry}

\begin{entry}{杀毒}{sha1 du2}{6,9}{⽊、⽏}[HSK 5]
  \definition{s.}{(computação) antivírus}
  \definition{v.}{esterilizar; desinfetar | (computação) eliminar um vírus}
\end{entry}

\begin{entry}{杀气}{sha1qi4}{6,4}{⽊、⽓}
  \definition{s.}{espírito assassino | aura de morte}
  \definition{v.}{desabafar a raiva de alguém}
\end{entry}

\begin{entry}{沙}{sha1}{7}{⽔}
  \definition*{s.}{sobrenome Sha}
  \definition[粒]{s.}{areia | cascalho | grânulo | pó}
\end{entry}

\begin{entry}{沙发}{sha1fa1}{7,5}{⽔、⼜}[HSK 3]
  \definition[套,组,个,张]{s.}{sofá; assentos com molas ou espuma plástica espessa, etc., com apoios de braços em ambos os lados}
\end{entry}

\begin{entry}{沙漠}{sha1mo4}{7,13}{⽔、⽔}[HSK 5]
  \definition[个]{s.}{deserto; superfície totalmente coberta por areia, sem água corrente, clima seco e vegetação escassa}
\end{entry}

\begin{entry}{沙特}{sha1te4}{7,10}{⽔、⽜}
  \definition*{s.}{Saudita | abreviação de 沙特阿拉伯}
  \seealsoref{沙特阿拉伯}{sha1te4 a1la1bo2}
\end{entry}

\begin{entry}{沙特阿拉伯}{sha1te4 a1la1bo2}{7,10,7,8,7}{⽔、⽜、⾩、⼿、⼈}
  \definition*{s.}{Arábia Saudita}
\end{entry}

\begin{entry}{沙鱼}{sha1yu2}{7,8}{⽔、⿂}
  \variantof{鲨鱼}
\end{entry}

\begin{entry}{沙子}{sha1 zi5}{7,3}{⽔、⼦}[HSK 3]
  \definition[粒,把,堆,袋,车]{s.}{areia; grão; pequenas pedras | \emph{pellets}; grãos pequenos; coisas parecidas com areia}
\end{entry}

\begin{entry}{刹}{sha1}{8}{⼑}
  \definition{v.}{acionar o(s) freio(s); frear; brecar}
  \seeref{刹}{cha4}
\end{entry}

\begin{entry}{刹多罗}{sha1duo1luo2}{8,6,8}{⼑、⼣、⽹}
  \definition*{s.}{Kshatara, sânscrito ``ksetra''}
\end{entry}

\begin{entry}{砂}{sha1}{9}{⽯}
  \variantof{沙}
\end{entry}

\begin{entry}{莎莎舞}{sha1sha1wu3}{10,10,14}{⾋、⾋、⾇}
  \definition{s.}{salsa (dança)}
\end{entry}

\begin{entry}{鲨鱼}{sha1yu2}{15,8}{⿂、⿂}
  \definition{s.}{tubarão}
\end{entry}

\begin{entry}{啥}{sha2}{11}{⼝}
  \definition{adv.}{Equivalente a 什么 (dialeto)}
\end{entry}

\begin{entry}{傻}{sha3}{13}{⼈}[HSK 5]
  \definition{adj.}{estúpido; confuso; burro; idiota; inflexível}
\end{entry}

\begin{entry}{傻瓜}{sha3gua1}{13,5}{⼈、⽠}
  \definition{adj.}{tolo | burro | simplório | idiota}
  \definition{v.}{enganar | iludir | lograr}
\end{entry}

\begin{entry}{傻眼}{sha3yan3}{13,11}{⼈、⽬}
  \definition{adj.}{estupefato | atordoado}
\end{entry}

\begin{entry}{嗄}{sha4}{13}{⼝}
  \definition{adj.}{rouco}
\end{entry}

\begin{entry}{色}{shai3}{6}{⾊}
  \definition[4]{s.}{cor}
  \seeref{色}{se4}
\end{entry}

\begin{entry}{晒}{shai4}{10}{⽇}[HSK 4]
  \definition{v.}{(sol) brilhar sobre | aquecer-se; secar ao sol; colocar algo sob a luz do sol para secar | ignorar (alguém) | mostrar; divulgar o conteúdo de sua vida privada na Internet}
\end{entry}

\begin{entry}{晒干}{shai4gan1}{10,3}{⽇、⼲}
  \definition{v.}{secar ao sol}
\end{entry}

\begin{entry}{山}{shan1}{3}{⼭}[HSK 1][Kangxi 46]
  \definition*{s.}{sobrenome Shan}
  \definition[座]{s.}{colina; maciço; montanha | qualquer coisa que se assemelhe a uma montanha | arbustos nos quais os bichos-da-seda tecem seus casulos; referindo-se a casulos de bicho-da-seda | eco; metáfora para um som muito alto}
\end{entry}

\begin{entry}{山顶}{shan1ding3}{3,8}{⼭、⾴}
  \definition{s.}{cume da montanha}
\end{entry}

\begin{entry}{山东}{shan1dong1}{3,5}{⼭、⼀}
  \definition*{s.}{Shandong}
\end{entry}

\begin{entry}{山谷}{shan1gu3}{3,7}{⼭、⾕}
  \definition{s.}{vale | ravina}
\end{entry}

\begin{entry}{山区}{shan1 qu1}{3,4}{⼭、⼖}[HSK 5]
  \definition[个]{s.}{área montanhosa; região montanhosa | colina; serra; montanha | distrito montanhoso}
\end{entry}

\begin{entry}{山体}{shan1ti3}{3,7}{⼭、⼈}
  \definition{s.}{forma de uma montanha}
\end{entry}

\begin{entry}{山羊}{shan1yang2}{3,6}{⼭、⽺}
  \definition{s.}{cabra | (ginástica) cavalo de salto de pequeno porte}
\end{entry}

\begin{entry}{山寨}{shan1zhai4}{3,14}{⼭、⼧}
  \definition{s.}{fortaleza fortificada da vila | fortaleza da montanha (especialmente de bandidos) | falsificação | imitação | (fig.) pechincha}
\end{entry}

\begin{entry}{扇}{shan1}{10}{⼾}[HSK 5]
  \definition{s.}{ventilar; agitar um leque para fazer o ar circular | dar um tapa; bater com a palma da mão | bater asas; esvoaçar | incitar; instigar; estimular; agitar}
  \seeref{扇}{shan4}
\end{entry}

\begin{entry}{闪}{shan3}{5}{⾨}[HSK 4]
  \definition*{s.}{sobrenome Shan}
  \definition{s.}{relâmpago}
  \definition{v.}{esquivar-se; desviar; sair do caminho | torcer; distender | surgir de repente | cintilar; brilhar | deixar para trás; abandonar | (corpo) oscilar dramaticamente}
\end{entry}

\begin{entry}{闪存盘}{shan3cun2pan2}{5,6,11}{⾨、⼦、⽫}
  \definition{s.}{unidade de memória \emph{USB} | cartão de memória}
  \seealsoref{优盘}{you1pan2}
\end{entry}

\begin{entry}{闪电}{shan3dian4}{5,5}{⾨、⽥}[HSK 4]
  \definition[道]{s.}{relâmpago; descargas elétricas entre nuvens ou entre nuvens e o solo}
  \seealsoref{雷电}{lei2dian4}
\end{entry}

\begin{entry}{单}{shan4}{8}{⼗}
  \definition*{s.}{sobrenome Shan}
  \definition{s.}{material de tecido de largura simples (dupla) | número singular (plural)}
  \seeref{单}{chan2}
  \seeref{单}{dan1}
\end{entry}

\begin{entry}{扇}{shan4}{10}{⼾}[HSK 5]
  \definition{clas.}{para portas, janelas, etc.}
  \definition[把]{s.}{leque | folha; algo em forma de placa ou folha}
\end{entry}

\begin{entry}{扇子}{shan4zi5}{10,3}{⼾、⼦}[HSK 5]
  \definition[把]{s.}{leque; abano; abanador; utensílios que produzem vento ao serem agitados}
\end{entry}

\begin{entry}{善}{shan4}{12}{⼝}
  \definition*{s.}{sobrenome Shan}
  \definition{adj.}{bom; bem | bom; satisfatório | gentil; amigável | familiar}
  \definition{adv.}{bom; bem}
  \definition{s.}{boa ação; ato benevolente; coisas boas (em oposição a 恶)}
  \definition{v.}{fazer sucesso; fazer bem; fazer acontecer | ser bom em; ser especialista (versado) em | ser apto a}
  \seealsoref{恶}{e4}
\end{entry}

\begin{entry}{善良}{shan4liang2}{12,7}{⼝、⾉}[HSK 4]
  \definition{adj.}{de bom coração; bom e honesto; de bom coração e cheio de boa vontade}
\end{entry}

\begin{entry}{善意}{shan4yi4}{12,13}{⼝、⼼}
  \definition{s.}{boa vontade | benevolência | bondade}
\end{entry}

\begin{entry}{善于}{shan4yu2}{12,3}{⼝、⼆}[HSK 4]
  \definition{adv./v.}{ser bom em; ser hábil em}
\end{entry}

\begin{entry}{禅}{shan4}{12}{⽰}
  \definition{v.}{abdicar e entregar a coroa a outra pessoa}
  \seeref{禅}{chan2}
\end{entry}

\begin{entry}{擅自}{shan4zi4}{16,6}{⼿、⾃}
  \definition{adv.}{sem permissão ou autorização | por iniciativa própria}
\end{entry}

\begin{entry}{伤}{shang1}{6}{⼈}[HSK 3]
  \definition*{s.}{sobrenome Shang}
  \definition[处]{s.}{ferida; ferimento}
  \definition{v.}{ferir; machucar | ter os sentimentos feridos | estar angustiado | enjoar de algo; desenvolver aversão a algo | ser prejudicial a; entravar}
\end{entry}

\begin{entry}{伤害}{shang1hai4}{6,10}{⼈、⼧}[HSK 4]
  \definition{v.}{ferir; prejudicar; machucar; magoar; causar danos físicos ou mentais}
\end{entry}

\begin{entry}{伤心}{shang1xin1}{6,4}{⼈、⼼}[HSK 3]
  \definition{v.+compl.}{estar triste; lamentar; estar com o coração partido; sentir-se triste por causa de infortúnio ou decepção}
\end{entry}

\begin{entry}{汤}{shang1}{6}{⽔}
  \definition{s.}{correnteza forte}
  \seeref{汤}{tang1}
\end{entry}

\begin{entry}{商标}{shang1biao1}{11,9}{⼝、⽊}[HSK 5]
  \definition[个]{s.}{marca; marca registrada; \emph{trademark}; marca ou símbolo (desenho, padrão, texto, etc.) gravado ou impresso na superfície ou embalagem de um produto, para diferenciá-lo de outros produtos semelhantes}
\end{entry}

\begin{entry}{商场}{shang1 chang3}{11,6}{⼝、⼟}[HSK 1]
  \definition[家]{s.}{mercado; shopping center; loja de departamentos; loja de grande área com uma variedade completa de produtos | o mundo dos negócios; referindo-se ao mundo dos negócios | mercado; mercado composto por várias lojas reunidas em um ou vários edifícios interligados}
\end{entry}

\begin{entry}{商店}{shang1dian4}{11,8}{⼝、⼴}[HSK 1]
  \definition[间,家,个]{s.}{loja; armazém; local de venda de mercadorias em recinto fechado}
\end{entry}

\begin{entry}{商量}{shang1liang5}{11,12}{⼝、⾥}[HSK 2]
  \definition{v.}{consultar; discutir; conversar sobre; discutir e trocar opiniões}
\end{entry}

\begin{entry}{商贸}{shang1mao4}{11,9}{⼝、⾙}
  \definition{s.}{comércio}
\end{entry}

\begin{entry}{商品}{shang1pin3}{11,9}{⼝、⼝}[HSK 3]
  \definition[种,个,件,批]{s.}{bens; mercadoria; \emph{merchande}; os produtos do trabalho produzidos para troca têm a dupla natureza de valor de uso e valor; as mercadorias incorporam diferentes relações de produção em diferentes sistemas sociais}
\end{entry}

\begin{entry}{商人}{shang1 ren2}{11,2}{⼝、⼈}[HSK 2]
  \definition[位,名]{s.}{comerciante; mercador; empresário; homem de negócios; pessoas que trabalham com a distribuição de mercadorias}
\end{entry}

\begin{entry}{商务}{shang1wu4}{11,5}{⼝、⼒}[HSK 4]
  \definition[种,类,项]{s.}{negócios; assuntos de negócios; assuntos comerciais}
\end{entry}

\begin{entry}{商业}{shang1ye4}{11,5}{⼝、⼀}[HSK 3]
  \definition[个,种]{s.}{barganha; negócio; comércio; atividade econômica que circula mercadorias por meio de compra e venda}
\end{entry}

\begin{entry}{上}{shang3}{3}{⼀}
  \definition{s.}{tom descendente-ascendente; significa o segundo tom dos quatro tons do mandarim, e também se refere ao terceiro tom do mandarim padrão}
  \seeref{上}{shang4}
\end{entry}

\begin{entry}{上声}{shang3sheng1}{3,7}{⼀、⼠}
  \definition{s.}{tom descendente e ascendente | terceiro tom no mandarim moderno}
\end{entry}

\begin{entry}{赏}{shang3}{12}{⾙}[HSK 4]
  \definition*{s.}{sobrenome Shang}
  \definition{s.}{recompensa; prêmio}
  \definition{v.}{conceder (outorgar) uma recompensa; recompensar; premiar | admirar; desfrutar; apreciar; valorizar}
\end{entry}

\begin{entry}{赏赐}{shang3ci4}{12,12}{⾙、⾙}
  \definition{s.}{recompensa | prêmio}
  \definition{v.}{recompensar | premiar}
\end{entry}

\begin{entry}{赏心悦目}{shang3xin1yue4mu4}{12,4,10,5}{⾙、⼼、⼼、⽬}
  \definition{expr.}{``Aquece o coração e encanta os olhos.''}
\end{entry}

\begin{entry}{上}{shang4}{3}{⼀}[HSK 1]
  \definition*{v.aux.}{usado após um verbo para indicar início e continuidade}
  \definition{adj.}{mais recente; último; anterior; tempo ou a sequência anterior | superior; mais alto; melhor; indica uma posição elevada em termos de qualidade, nível, etc. | lugar elevado; posição superior (em oposição a 下)}
  \definition{s.}{superior; acima; para cima; um lugar alto ou mais alto do que um determinado local | na superfície de um objeto; usado após um substantivo, indica a superfície de um objeto | indica estar dentro do escopo de algo; usado após um substantivo, indica que algo está dentro do âmbito de determinada coisa | indica um aspecto específico | antigamente, referia-se ao imperador | usado após palavras que indicam idade, equivale a ``\dots 的时候'' | o primeiro nível da escala da música folclórica chinesa, usado como um símbolo de nota na notação musical, equivalente ao '1' na notação simplificada.}
  \definition{v.}{subir; montar; embarcar; entrar | ir para; partir para | estar ocupado (com trabalho, estudos, etc.) em um horário fixo; começar a trabalhar ou estudar na hora marcada, etc. | seguir em frente; prosseguir | encher; abastecer; servir; melhorar; aumentar | aparecer no palco; entrar | colocar algo em posição; ajustar; fixar; montar as duas partes de algo | aplicar; pintar; espalhar | ser registrado; ser publicado (em uma publicação) | atingir; ser suficiente (uma determinada quantidade ou grau) | submeter; enviar; apresentar; submeter à aprovação superior | ventilar; apertar; torcer | trazer; servir; colocar comida, pratos, chá e outras coisas na mesa para os convidados | indicar que uma ação tem um resultado | pesquisar na \emph{Internet} | emaranhar-se; ficar emaranhado; enredar-se}
  \seeref{上}{shang3}
  \seealsoref{的时候}{de5 shi2hou4}
  \seealsoref{下}{xia4}
\end{entry}

\begin{entry}{上班}{shang4ban1}{3,10}{⼀、⽟}[HSK 1]
  \definition{v.+compl.}{ir trabalhar; começar a trabalhar; estar de plantão; ir trabalhar no local de trabalho regular no horário especificado}
\end{entry}

\begin{entry}{上边}{shang4 bian5}{3,5}{⼀、⾡}[HSK 1]
  \definition{s.}{topo; acima; sobre; superior}
\end{entry}

\begin{entry}{上车}{shang4 che1}{3,4}{⼀、⾞}[HSK 1]
  \definition{v.}{entrar; subir (em um ônibus, trem, carro etc.)}
\end{entry}

\begin{entry}{上次}{shang4 ci4}{3,6}{⼀、⽋}[HSK 1]
  \definition{adv.}{última vez}
\end{entry}

\begin{entry}{上当}{shang4dang4}{3,6}{⼀、⼹}
  \definition{v.+compl.}{ser enganado | morder uma isca | ser manipulado | ser joguete nas mãos de alguém}
\end{entry}

\begin{entry}{上访}{shang4fang3}{3,6}{⼀、⾔}
  \definition{v.}{buscar uma audiência com superiores (especialmente funcionários do governo) para fazer uma petição por algo}
\end{entry}

\begin{entry}{上个月}{shang4 ge4 yue4}{3,3,4}{⼀、⼈、⽉}[HSK 4]
  \definition{s.}{mês passado; refere-se à hora de um mês atrás, ou seja, o mês passado mais próximo da hora atual}
\end{entry}

\begin{entry}{上古}{shang4gu3}{3,5}{⼀、⼝}
  \definition{s.}{o passado distante | tempos antigos | antiguidade}
\end{entry}

\begin{entry}{上海}{shang4hai3}{3,10}{⼀、⽔}
  \definition*{s.}{Shangai (Xangai)}
\end{entry}

\begin{entry}{上级}{shang4ji2}{3,6}{⼀、⽷}[HSK 5]
  \definition[个]{s.}{nível superior; organização ou pessoa em nível superior; organizações ou pessoas de nível superior dentro do mesmo sistema organizacional}
\end{entry}

\begin{entry}{上课}{shang4 ke4}{3,10}{⼀、⾔}[HSK 1]
  \definition{v.+compl.}{frequentar aulas; ir às aulas; dar uma aula}
\end{entry}

\begin{entry}{上来}{shang4 lai2}{3,7}{⼀、⽊}[HSK 3]
  \definition{v.}{subir (para a minha localização) | estar no começo; começar; iniciar | surgir; de um lugar baixo para um lugar alto (o interlocutor está em um lugar alto) | usado após o verbo, indica que algo foi concluído com sucesso}
\end{entry}

\begin{entry}{上楼}{shang4 lou2}{3,13}{⼀、⽊}[HSK 4]
  \definition{v.}{subir as escadas; ir para o andar de cima}
\end{entry}

\begin{entry}{上门}{shang4 men2}{3,3}{⼀、⾨}[HSK 4]
  \definition{v.}{chamar; visitar; aparecer; ir ou vir para ver alguém; ir até a porta; ir até a casa de alguém | trancar a porta; fechar a porta durante a noite | casar-se e morar com a família da noiva}
\end{entry}

\begin{entry}{上面}{shang4 mian4}{3,9}{⼀、⾯}[HSK 3]
  \definition{s.}{uma posição mais alta que algo; uma posição acima/acima de algo | superfície do objeto | aspecto | a parte acima mencionada; a parte que vem primeiro na ordem; a parte de um artigo ou discurso que vem antes da presente | autoridades superiores | os mais velhos; a geração mais velha da família}
\end{entry}

\begin{entry}{上坡路}{shang4po1lu4}{3,8,13}{⼀、⼟、⾜}
  \definition{s.}{aclive | progresso | (fig.) tendência ascendente}
\end{entry}

\begin{entry}{上去}{shang4 qu4}{3,5}{⼀、⼛}[HSK 3]
  \definition{v.}{subir (a partir da minha localização) | ascender a um lugar (ou estado) considerado mais elevado (ou acima); usado depois de um verbo para indicar movimento, de baixo para cima ou de perto para longe}
\end{entry}

\begin{entry}{上升}{shang4 sheng1}{3,4}{⼀、⼗}[HSK 3]
  \definition{v.}{elevar; subir; mover-se para cima; mover de baixo para cima; aumentar em nível, grau, quantidade, etc.}
\end{entry}

\begin{entry}{上网}{shang4wang3}{3,6}{⼀、⽹}[HSK 1]
  \definition{v.}{conectar-se à \emph{Internet}; acessar a \emph{Internet}; entrar na \emph{Internet}; acessar a rede; refere-se especificamente ao computador do usuário conectado à Internet para pesquisar e consultar informações, etc.}
\end{entry}

\begin{entry}{上午}{shang4wu3}{3,4}{⼀、⼗}[HSK 1]
  \definition[个]{s.}{manhã; \emph{ante meridiem} (a.m.); geralmente refere-se ao período entre a manhã e o meio-dia}
\end{entry}

\begin{entry}{上下}{shang4 xia4}{3,3}{⼀、⼀}[HSK 5]
  \definition{adv.}{para cima e para baixo}
  \definition[顶]{s.}{alto e baixo | de cima para baixo; para cima e para baixo | superioridade ou inferioridade relativa | (após números redondos) aproximadamente; mais ou menos; por aí | velhos e jovens; hierarquia em termos de cargo e posição social}
  \definition{v.}{subir ou descer | subir e descer; da alta para a baixa ou da baixa para a alta}
\end{entry}

\begin{entry}{上学}{shang4 xue2}{3,8}{⼀、⼦}[HSK 1]
  \definition{v.}{ir à escola; frequentar a escola; estar na escola; ir à escola para estudar | começar a escola; começar a estudar no ensino fundamental}
\end{entry}

\begin{entry}{上询}{shang4 xun2}{3,8}{⼀、⾔}
  \definition{adv.}{primeira dezena do mês}
\end{entry}

\begin{entry}{上演}{shang4yan3}{3,14}{⼀、⽔}
  \definition{s.}{exibição | encenação}
  \definition{v.}{exibir (um filme) | encenar (uma peça)}
\end{entry}

\begin{entry}{上衣}{shang4 yi1}{3,6}{⼀、⾐}[HSK 3]
  \definition[件]{s.}{jaqueta; roupas para a parte superior do corpo}
\end{entry}

\begin{entry}{上涨}{shang4 zhang3}{3,10}{⼀、⽔}[HSK 5]
  \definition{v.}{subir; ir para cima; ascender}
\end{entry}

\begin{entry}{上周}{shang4 zhou1}{3,8}{⼀、⼝}[HSK 2]
  \definition{s.}{semana passada}
\end{entry}

\begin{entry}{尚且}{shang4qie3}{8,5}{⼩、⼀}
  \definition{conj.}{até | ainda}
\end{entry}

\begin{entry}{尚且……何况……}{shang4qie3 he2kuang4}{8,5,7,7}{⼩、⼀、⼈、⼎}
  \definition{conj.}{ainda que\dots, \dots}
\end{entry}

\begin{entry}{烧}{shao1}{10}{⽕}[HSK 4]
  \definition[次]{s.}{febre; temperatura corporal mais alta do que o normal}
  \definition{v.}{queimar; pegar fogo | cozinhar; aquecer; assar | guisar depois de fritar ou fritar depois de guisar | assar; grelhar os ingredientes dos alimentos diretamente sobre o fogo | ter febre; estar com febre | danificar (matar ou murchar) as plantas pelo uso excessivo (ou inadequado) de fertilizantes | tornar-se arrogante ou presunçoso; metáfora de estar em uma boa posição e se deixar levar}
\end{entry}

\begin{entry}{烧烤}{shao1kao3}{10,10}{⽕、⽕}
  \definition{s.}{churrasco}
  \definition{v.}{assar}
\end{entry}

\begin{entry}{稍}{shao1}{12}{⽲}[HSK 5]
  \definition{adv.}{ligeiramente; um pouco; um pouquinho}
\end{entry}

\begin{entry}{稍微}{shao1wei1}{12,13}{⽲、⼻}[HSK 5]
  \definition{adv.}{um pouco; um pouquinho; uma ninharia; indica que a quantidade é pequena ou o grau é superficial}
\end{entry}

\begin{entry}{少}{shao3}{4}{⼩}[HSK 1]
  \definition{adj.}{menos; pouco (oposto a 多); escasso; não atingir a quantidade original ou esperada}
  \definition{adv.}{um momento; um instante; provisoriamente; ligeiramente}
  \definition{v.}{faltar; ser insuficiente | dever | perder; desaparecer; extraviar | parar; desistir}
  \seeref{少}{shao4}
  \seealsoref{多}{duo1}
\end{entry}

\begin{entry}{少数}{shao3 shu4}{4,13}{⼩、⽁}[HSK 2]
  \definition{s.}{número pequeno; poucos; minoria}
\end{entry}

\begin{entry}{少}{shao4}{4}{⼩}
  \definition*{s.}{sobrenome Shao}
  \definition{s.}{jovem (em oposição a 老)}
  \definition{s.}{jovem mestre; filho de uma família rica}
  \seeref{少}{shao3}
  \seealsoref{老}{lao3}
\end{entry}

\begin{entry}{少年}{shao4 nian2}{4,6}{⼩、⼲}[HSK 2]
  \definition[个,名,位]{s.}{adolescente; juventude; atualmente, a faixa etária geralmente referida é de 10 anos ou mais a 18 anos ou mais | menor; jovem; juvenil; refere-se a menores na faixa etária anterior | jovem; adolescente; rapaz}
\end{entry}

\begin{entry}{舌头}{she2tou5}{6,5}{⾆、⼤}
  \definition[个]{s.}{língua | soldado inimigo capturado com o propósito de extrair informações}
\end{entry}

\begin{entry}{折}{she2}{7}{⼿}
  \definition{v.}{estalar; quebrar | perder dinheiro em um negócio}
  \seeref{折}{zhe1}
  \seeref{折}{zhe2}
\end{entry}

\begin{entry}{蛇}{she2}{11}{⾍}[HSK 5]
  \definition[条]{s.}{cobra; serpente}
\end{entry}

\begin{entry}{舍不得}{she3bu5de5}{8,4,11}{⾆、⼀、⼻}[HSK 5]
  \definition{v.}{não se pode abandonar ou deixar, não se quer usar ou descartar; detestar separar-me ou usar}
\end{entry}

\begin{entry}{舍得}{she3 de5}{8,11}{⾆、⼻}[HSK 5]
  \definition{v.}{não guardar rancor; estar disposto a abrir mão de algo; estar disposto a gastar dinheiro, tempo, etc.; estar disposto a abrir mão de pessoas, oportunidades, coisas, etc. que são importantes para você}
\end{entry}

\begin{entry}{设备}{she4bei4}{6,8}{⾔、⼡}[HSK 3]
  \definition[台,套]{s.}{instalação; equipamento; montagem; um conjunto de edifícios ou equipamentos necessários para executar uma determinada tarefa ou suprir uma determinada necessidade}
\end{entry}

\begin{entry}{设计}{she4ji4}{6,4}{⾔、⾔}[HSK 3]
  \definition[份]{s.}{plano; esquema; refere-se a um plano de design ou a um projeto para um plano, etc.}
  \definition{v.}{planejar; projetar; formular métodos, desenhos, etc. com antecedência, de acordo com determinados requisitos de finalidade, antes de iniciar oficialmente um trabalho | arquitetar; idear; tramar; fazer um plano}
\end{entry}

\begin{entry}{设立}{she4li4}{6,5}{⾔、⽴}[HSK 3]
  \definition{v.}{fundar; estabelecer; começar}
\end{entry}

\begin{entry}{设施}{she4shi1}{6,9}{⾔、⽅}[HSK 4]
  \definition{s.}{facilidade; instalação; instituições, sistemas, organizações, edifícios, etc., estabelecidos para realizar um trabalho ou atender a uma necessidade}
\end{entry}

\begin{entry}{设想}{she4xiang3}{6,13}{⾔、⼼}[HSK 5]
  \definition[个,种]{s.}{plano provisório (ou ideia); (item, tipo) refere-se a algo hipotético ou imaginário}
  \definition{v.}{imaginar; prever; conceber; supor | ter consideração por}
\end{entry}

\begin{entry}{设置}{she4zhi4}{6,13}{⾔、⽹}[HSK 4]
  \definition{v.}{estabelecer; colocar em prática; estabelecer ou criar instituições, empregos, profissões ou códigos, etc. | encaixar; ajustar; instalar; configurar; colocar}
\end{entry}

\begin{entry}{社}{she4}{7}{⽰}[HSK 5]
  \definition[个]{s.}{agência; sociedade; órgão organizado; organização; comunidade | comuna popular | o deus da terra, sacrifícios a ele ou altares para tais sacrifícios; na antiguidade, o deus da terra, o local onde ele era venerado, o dia da veneração e o ritual eram chamados de 社 | agência de notícias |  imprensa}
\end{entry}

\begin{entry}{社会}{she4hui4}{7,6}{⽰、⼈}[HSK 3]
  \definition[个,种]{s.}{sociedade; em um determinado estágio do desenvolvimento histórico, a relação geral entre as pessoas nas atividades de produção | comunidade; geralmente se refere a um grupo de pessoas que estão conectadas por atividades comuns}
\end{entry}

\begin{entry}{社区}{she4qu1}{7,4}{⽰、⼖}[HSK 5]
  \definition{s.}{bairro; comunidade residencial; bairros da cidade, divididos de acordo com a localização geográfica | distrito; comunidade (para pessoas da mesma classe social, etc.) ; lugar onde pessoas com características comuns, como classe social, vivem juntas}
\end{entry}

\begin{entry}{射}{she4}{10}{⼨}[HSK 5]
  \definition*{s.}{sobrenome She}
  \definition{v.}{atirar; disparar | descarregar em jato; jorrar | emitir (luz, calor, etc.) | irradiar | aludir a algo ou alguém; insinuar}
\end{entry}

\begin{entry}{射击}{she4ji1}{10,5}{⼨、⼐}[HSK 5]
  \definition{s.}{tiro; tiro ao alvo}
  \definition{v.}{disparar; atirar}
\end{entry}

\begin{entry}{摄氏}{she4shi4}{13,4}{⼿、⽒}
  \definition{s.}{graus Celsius (°C), centígrado}
\end{entry}

\begin{entry}{摄像}{she4 xiang4}{13,13}{⼿、⼈}[HSK 5]
  \definition{v.}{gravar; filmar; filmar com câmera; fazer uma gravação de vídeo (com uma câmera de vídeo ou TV)}
\end{entry}

\begin{entry}{摄像机}{she4 xiang4 ji1}{13,13,6}{⼿、⼈、⽊}[HSK 5]
  \definition[个,部]{s.}{câmera de vídeo; dispositivo que pode ser usado para converter imagens captadas em sinais de imagem de televisão}
\end{entry}

\begin{entry}{摄影}{she4ying3}{13,15}{⼿、⼺}[HSK 5]
  \definition{s.}{fotografia}
  \definition{v.}{fotografar; tirar uma foto; tirar fotos ou filmar}
\end{entry}

\begin{entry}{摄影师}{she4 ying3 shi1}{13,15,6}{⼿、⼺、⼱}[HSK 5]
  \definition[个]{s.}{fotógrafo; cinegrafista; operador de câmera; técnico de fotografia em estúdio fotográfico}
\end{entry}

\begin{entry}{谁}{shei2}{10}{⾔}[HSK 1]
  \definition{pron.}{quem? | (em pergunta retórica) quem?; usado em perguntas retóricas, para indicar que não há ninguém | refere-se a pessoas que não têm certeza, incluindo aquelas que não sabem | alguém; qualquer pessoa; indica qualquer pessoa ou qualquer um | repetido em uma frase para se referir a uma pessoa | (repetido em duas frases) quem quer que seja; fazer com que o sujeito e o objeto se refiram a duas pessoas diferentes}
  \seeref{谁}{shui2}
\end{entry}

\begin{entry}{申请}{shen1qing3}{5,10}{⽥、⾔}[HSK 4]
  \definition[份,批,项]{s.}{a solicitação para; o requerimento para; um pedido para ser visto pelos superiores ou departamentos relevantes}
  \definition{v.}{solicitar; apresentar uma solicitação; apresentar os motivos e fazer o pedido aos superiores ou aos departamentos competentes}
\end{entry}

\begin{entry}{伸}{shen1}{7}{⼈}[HSK 5]
  \definition{v.}{alongar; esticar; estender}
\end{entry}

\begin{entry}{身边}{shen1 bian1}{7,5}{⾝、⾡}[HSK 2]
  \definition{adv.}{ao redor; ao lado de alguém; perto do corpo | carregar consigo (transportar); à mão}
\end{entry}

\begin{entry}{身材}{shen1cai2}{7,7}{⾝、⽊}[HSK 4]
  \definition[副,种,个,具]{s.}{figura; estatura; altura e peso corporal}
\end{entry}

\begin{entry}{身份}{shen1fen4}{7,6}{⾝、⼈}[HSK 4]
  \definition[种]{s.}{status; capacidade; identidade; refere-se à origem, ao status e às qualificações de uma pessoa | dignidade; posição honrada; referência especial ao status respeitável}
\end{entry}

\begin{entry}{身份证}{shen1 fen4 zheng4}{7,6,7}{⾝、⼈、⾔}[HSK 3]
  \definition[张]{s.}{ID; bilhete de identidade; carteira de identidade}
\end{entry}

\begin{entry}{身高}{shen1 gao1}{7,10}{⾝、⾼}[HSK 4]
  \definition[个,种,段]{s.}{estatura; altura (de uma pessoa);}
\end{entry}

\begin{entry}{身上}{shen1 shang5}{7,3}{⾝、⼀}[HSK 1]
  \definition{s.}{no corpo de alguém | em um;  com um}
\end{entry}

\begin{entry}{身体}{shen1ti3}{7,7}{⾝、⼈}[HSK 1]
  \definition[具,个]{s.}{corpo | saúde; saúde das pessoas}
\end{entry}

\begin{entry}{身体能力}{shen1ti3 neng2li4}{7,7,10,2}{⾝、⼈、⾁、⼒}
  \definition{s.}{habilidade física}
\end{entry}

\begin{entry}{身体乳}{shen1ti3 ru3}{7,7,8}{⾝、⼈、⼄}
  \definition{s.}{loção corporal}
\end{entry}

\begin{entry}{身亡}{shen1wang2}{7,3}{⾝、⼇}
  \definition{v.}{morrer}
\end{entry}

\begin{entry}{深}{shen1}{11}{⽔}[HSK 3]
  \definition*{s.}{sobrenome Shen}
  \definition{adj.}{profundo | difícil; intenso; profundo | completo; penetrante; intenso; profundo | próximo; íntimo; afeição profunda; relacionamento próximo | escuro; profundo | tardio}
  \definition{adv.}{muito; grandemente; profundamente}
  \definition{s.}{profundidade}
  \seealsoref{浅}{qian3}
\end{entry}

\begin{entry}{深处}{shen1 chu4}{11,5}{⽔、⼡}[HSK 5]
  \definition{s.}{profundidades; recantos; recessos | profundezas}
\end{entry}

\begin{entry}{深度}{shen1 du4}{11,9}{⽔、⼴}[HSK 5]
  \definition{adj.}{(em grau ou extensão) profundo; sério; grave}
  \definition{s.}{profundidade; grau de profundidade; | profundidade; rigor; meticulosidade; grau de contato com a essência das coisas | estágio avançado (ou em deterioração) de desenvolvimento; grau de crescimento e desenvolvimento das coisas}
\end{entry}

\begin{entry}{深厚}{shen1hou4}{11,9}{⽔、⼚}[HSK 4]
  \definition{adj.}{profundo; sentimentos fortes | sólido; profundamente enraizado; fundação sólida}
\end{entry}

\begin{entry}{深刻}{shen1ke4}{11,8}{⽔、⼑}[HSK 3]
  \definition{adj.}{profundo; instenso; chegar à essência de um assunto ou problema}
\end{entry}

\begin{entry}{深入}{shen1 ru4}{11,2}{⽔、⼊}[HSK 3]
  \definition{adj.}{profundo; completo}
  \definition{v.}{ir fundo em; penetrar em; penetrar o exterior; alcançar o interior ou o centro de algo}
\end{entry}

\begin{entry}{深深}{shen1shen1}{11,11}{⽔、⽔}
  \definition{adj.}{profundo}
  \definition{adv.}{profundamente}
\end{entry}

\begin{entry}{深夜}{shen1ye4}{11,8}{⽔、⼣}
  \definition{adv.}{tarde da noite}
\end{entry}

\begin{entry}{什么}{shen2me5}{4,3}{⼈、⼃}[HSK 1]
  \definition{pron.}{o que?; expressar dúvida, perguntar sobre o mundo, locais, pessoas ou coisas | usado para se referir a algo indefinido; expressar incerteza | qualquer; todos; refere-se a todas as pessoas ou coisas | dois 什么 são usados juntos, indicando que o primeiro determina o segundo | usado para expressar surpresa ou insatisfação | usado para expressar discordância com o que foi dito; expressar negação | usado antes de elementos paralelos para indicar que a lista é infinita}
\end{entry}

\begin{entry}{什么时候}{shen2me5shi2hou5}{4,3,7,10}{⼈、⼃、⽇、⼈}
  \definition{adv.}{quando? | a que horas?}
\end{entry}

\begin{entry}{什么样}{shen2 me5 yang4}{4,3,10}{⼈、⼃、⽊}[HSK 2]
  \definition{pron.}{que tipo?; usado para perguntar sobre a natureza, características ou aparência de algo |  o quê?; de que tipo?; usado para perguntar sobre a situação ou o estado de alguém ou algo}
\end{entry}

\begin{entry}{神}{shen2}{9}{⽰}[HSK 5]
  \definition*{s.}{sobrenome Shen}
  \definition*{s.}{Deus}
  \definition{adj.}{inteligente; esperto | mágico; sobrenatural}
  \definition[个,位,尊]{s.}{divindade; deidade | espírito; mente; refere-se ao espírito, energia ou atenção de uma pessoa | olhar; expressão; expressões que refletem o estado interior}
\end{entry}

\begin{entry}{神话}{shen2hua4}{9,8}{⽰、⾔}[HSK 4]
  \definition[个]{s.}{mito; mitologia; conto de fadas; refere-se a deuses e deusas lendários e histórias de heróis antigos deificados | lorota; refere-se a alegações ridículas e infundadas}
\end{entry}

\begin{entry}{神经}{shen2jing1}{9,8}{⽰、⽷}[HSK 5]
  \definition{adj.}{excêntrico; estranho; peculiar; descreve anormalidade neurológica}
  \definition{s.}{nervo; um tipo de tecido presente no corpo humano ou animal que conecta o cérebro aos órgãos, transmitindo as sensações ao cérebro e as informações do cérebro aos órgãos}
\end{entry}

\begin{entry}{神经病的}{shen2jing1bing4de5}{9,8,10,8}{⽰、⽷、⽧、⽩}
  \definition{adj.}{neurótico}
\end{entry}

\begin{entry}{神经病学}{shen2jing1bing4xue2}{9,8,10,8}{⽰、⽷、⽧、⼦}
  \definition{s.}{neurologia}
\end{entry}

\begin{entry}{神秘}{shen2mi4}{9,10}{⽰、⽲}[HSK 4]
  \definition{adj.}{místico; misterioso}
\end{entry}

\begin{entry}{神明}{shen2ming2}{9,8}{⽰、⽇}
  \definition{s.}{divindades | deuses}
\end{entry}

\begin{entry}{神奇}{shen2qi2}{9,8}{⽰、⼤}[HSK 5]
  \definition{adj.}{mágico; peculiar; místico; milagroso; algo que parece muito novo; algo que ninguém imaginaria, mas que geralmente traz bons resultados}
  \definition{s.}{mágica; milagre}
\end{entry}

\begin{entry}{神器}{shen2qi4}{9,16}{⽰、⼝}
  \definition{s.}{objeto mágico | objeto simbólico do poder imperial | arma fina | ferramenta muito útil}
\end{entry}

\begin{entry}{神情}{shen2 qing2}{9,11}{⽰、⼼}[HSK 5]
  \definition{s.}{aparência; expressão; atividades internas reveladas no rosto das pessoas}
\end{entry}

\begin{entry}{神兽}{shen2shou4}{9,11}{⽰、⼋}
  \definition{s.}{animal mitológico | fera}
\end{entry}

\begin{entry}{甚而}{shen4'er2}{9,6}{⽢、⽽}
  \definition{conj.}{(ir) tão longe quanto | tanto que}
\end{entry}

\begin{entry}{甚或}{shen4huo4}{9,8}{⽢、⼽}
  \definition{conj.}{(ir) tão longe quanto | tanto que}
\end{entry}

\begin{entry}{甚至}{shen4zhi4}{9,6}{⽢、⾄}[HSK 4]
  \definition{conj.}{e até mesmo; nem mesmo; para apresentar uma situação típica e especial, para enfatizar a profundidade e a seriedade de uma situação}
\end{entry}

\begin{entry}{升}{sheng1}{4}{⼗}[HSK 3]
  \definition*{s.}{sobrenome Sheng}
  \definition{clas.}{litro (l)}
  \definition{s.}{sheng, uma unidade de medida seca para grãos (= 1 litro), um décimo de 斗}
  \definition{v.}{elevar; içar; subir; ascender; subir ou subir mais alto (oposto de 降) | promover; melhorar (nível)}
  \seealsoref{斗}{dou4}
  \seealsoref{降}{jiang4}
\end{entry}

\begin{entry}{升高}{sheng1 gao1}{4,10}{⼗、⾼}[HSK 5]
  \definition{v.}{subir; ascender | promover; elevar; intensificar; potencializar; melhorar}
\end{entry}

\begin{entry}{升起}{sheng1qi3}{4,10}{⼗、⾛}
  \definition{v.}{levantar | içar | subir}
\end{entry}

\begin{entry}{生}{sheng1}{5}{⽣}[HSK 2,3][Kangxi 100]
  \definition*{s.}{sobrenome Sheng}
  \definition{adj.}{vivo; vital | verde; não maduro | cru; não cozido; mal cozido | bruto; não refinado; não processado | estranho; desconhecido; não familiarizado | rígido; mecânico; forçado}
  \definition{adv.}{muito; usado antes de certas palavras que expressam emoções e sentimentos | verdadeiramente; realmente; forçosamente}
  \definition{s.}{vida | meio de subsistência | aluno; estudante | estudioso; antigamente chamados de eruditos | o tipo de personagem masculino na ópera de Pequim, etc.}
  \definition{suf.}{certos sufixos substantivos que se referem a pessoas (学生) | sufixos de certos advérbios (好生)}
  \definition{v.}{dar à luz; ter um filho | nascer | crescer; cultivar | viver; existir; sobreviver | favorecer; gerar; ocorrer | acender (uma fogueira); fazer o combustível queimar}
  \seealsoref{好生}{hao3sheng1}
  \seealsoref{学生}{xue2sheng5}
\end{entry}

\begin{entry}{生病}{sheng1bing4}{5,10}{⽣、⽧}[HSK 1]
  \definition{v.}{adoecer; ficar doente; ficar mal; contrair uma doença}
\end{entry}

\begin{entry}{生菜}{sheng1cai4}{5,11}{⽣、⾋}
  \definition{s.}{alface}
\end{entry}

\begin{entry}{生产}{sheng1chan3}{5,6}{⽣、⼇}[HSK 3]
  \definition{v.}{produzir; fabricar; utilizar ferramentas para mudar o objeto de trabalho e criar meios de produção e meios de subsistência | dar à luz uma criança; ter filhos}
\end{entry}

\begin{entry}{生成}{sheng1 cheng2}{5,6}{⽣、⼽}[HSK 5]
  \definition{v.}{formar; gerar; produzir | ter por natureza; nascer com}
\end{entry}

\begin{entry}{生词}{sheng1 ci2}{5,7}{⽣、⾔}[HSK 2]
  \definition[个,组,堆,条]{s.}{nova palavra; palavras que não aprendi, não conheço ou não entendo}
\end{entry}

\begin{entry}{生存}{sheng1cun2}{5,6}{⽣、⼦}[HSK 3]
  \definition{v.}{viver; sobreviver; subsistir; manter a vida; estar vivo}
\end{entry}

\begin{entry}{生的}{sheng1de5}{5,8}{⽣、⽩}
  \definition{conj.}{para evitar isso | para que\dots não\dots}
\end{entry}

\begin{entry}{生动}{sheng1dong4}{5,6}{⽣、⼒}[HSK 3]
  \definition{adj.}{vívido; animado; descreve a linguagem e as formas de expressão como sendo ativas e em movimento}
\end{entry}

\begin{entry}{生活}{sheng1huo2}{5,9}{⽣、⽔}[HSK 2]
  \definition[个,段,种]{s.}{vida; subsistência; as diversas atividades realizadas por pessoas ou seres vivos para sobreviver e se desenvolver | estilo de vida; condições de vida; situação em termos de vestuário, alimentação, habitação e transporte | trabalho (principalmente nas áreas industrial, agrícola e artesanal)}
  \definition{v.}{viver; realizar várias atividades | sobreviver}
\end{entry}

\begin{entry}{生活垃圾}{sheng1huo2la1ji1}{5,9,8,6}{⽣、⽔、⼟、⼟}
  \definition{s.}{lixo doméstico}
\end{entry}

\begin{entry}{生活型}{sheng1huo2 xing2}{5,9,9}{⽣、⽔、⼟}
  \definition{s.}{forma de vida}
\end{entry}

\begin{entry}{生理}{sheng1li3}{5,11}{⽣、⽟}
  \definition{adj.}{fisiológico}
  \definition{s.}{fisiologia}
\end{entry}

\begin{entry}{生命}{sheng1ming4}{5,8}{⽣、⼝}[HSK 3]
  \definition{s.}{vida; não envolve apenas a existência e as atividades dos organismos, mas também inclui experiências de vida humana, valores e elementos-chave da sobrevivência e do desenvolvimento de várias coisas}
\end{entry}

\begin{entry}{生气}{sheng1 qi4}{5,4}{⽣、⽓}[HSK 1]
  \definition{s.}{vitalidade; vigor; energia da vida}
  \definition{v.+compl.}{ficar com raiva; ficar ofendido; ficar zangado; encontrar algo que não é do seu agrado e sentir-se descontente}
\end{entry}

\begin{entry}{生日}{sheng1ri4}{5,4}{⽣、⽇}[HSK 1]
  \definition[个,次]{s.}{aniversário; dia de nascimento, também se refere ao dia em que se completa um ano de idade a cada ano}
\end{entry}

\begin{entry}{生态}{sheng1tai4}{5,8}{⽣、⼼}
  \definition{adj.}{ecológico}
  \definition{s.}{ecologia}
\end{entry}

\begin{entry}{生物}{sheng1wu4}{5,8}{⽣、⽜}
  \definition{adj.}{biológico}
  \definition{s.}{biologia (disciplina) | organismo | ser vivo}
\end{entry}

\begin{entry}{生意}{sheng1yi4}{5,13}{⽣、⼼}
  \definition[笔,种,次]{s.}{tendência a crescer; vitalidade; vigor; energia}
  \seeref{生意}{sheng1yi5}
\end{entry}

\begin{entry}{生意}{sheng1yi5}{5,13}{⽣、⼼}[HSK 3]
  \definition[笔,种,次]{s.}{comércio, compra e venda; negócios; indústria; colegas do mesmo setor}
  \seeref{生意}{sheng1yi4}
\end{entry}

\begin{entry}{生鱼片}{sheng1yu2pian4}{5,8,4}{⽣、⿂、⽚}
  \definition{s.}{fatias de peixe cru, \emph{sashimi}}
\end{entry}

\begin{entry}{生长}{sheng1zhang3}{5,4}{⽣、⾧}[HSK 3]
  \definition{v.}{cresçer; sob certas condições de vida, o volume e o peso dos organismos aumentam gradualmente | nascer e crescer}
\end{entry}

\begin{entry}{声}{sheng1}{7}{⼠}[HSK 5]
  \definition{clas.}{indica o número de vezes que um som é emitido}
  \definition{s.}{som; voz | reputação | consoante inicial (de uma sílaba chinesa) | tom; tom de voz | informação; notícia}
  \definition{v.}{declarar; anunciar; emitir um som}
\end{entry}

\begin{entry}{声明}{sheng1ming2}{7,8}{⼠、⽇}[HSK 3]
  \definition[项,份]{s.}{declaração}
  \definition{v.}{declarar; anunciar; expressar publicamente a sua atitude ou dizer a verdade}
\end{entry}

\begin{entry}{声音}{sheng1yin1}{7,9}{⼠、⾳}[HSK 2]
  \definition[个,种]{s.}{som; voz; a percepção auditiva das ondas sonoras}
\end{entry}

\begin{entry}{绳子}{sheng2zi5}{11,3}{⽷、⼦}
  \definition[条]{s.}{corda | cordão}
\end{entry}

\begin{entry}{省}{sheng3}{9}{⽬}[HSK 2]
  \definition*{s.}{sobrenome Sheng}
  \definition{s.}{província; unidade administrativa, subordinada diretamente ao governo central | capital provincial; refere-se à capital da província, localização da administração provincial | abreviação (de palavras)}
  \definition{v.}{economizar; poupar; reduzir o consumo (em oposição a 费) | omitir; deixar de fora}
  \seeref{省}{xing3}
  \seealsoref{费}{fei4}
\end{entry}

\begin{entry}{省城}{sheng3cheng2}{9,9}{⽬、⼟}
  \definition{s.}{capital da província}
\end{entry}

\begin{entry}{省会}{sheng3hui4}{9,6}{⽬、⼈}
  \definition{s.}{capital da província}
\end{entry}

\begin{entry}{省俭}{sheng3jian3}{9,9}{⽬、⼈}
  \definition{s.}{econômico | frugal}
  \definition{v.}{economizar}
\end{entry}

\begin{entry}{省力}{sheng3li4}{9,2}{⽬、⼒}
  \definition{v.}{economizar esforço ou trabalho}
\end{entry}

\begin{entry}{省钱}{sheng3qian2}{9,10}{⽬、⾦}
  \definition{v.}{economizar dinheiro}
\end{entry}

\begin{entry}{省却}{sheng3que4}{9,7}{⽬、⼙}
  \definition{v.}{livrar-se (para economizar espaço) | salvar}
\end{entry}

\begin{entry}{省心}{sheng3xin1}{9,4}{⽬、⼼}
  \definition{adj.}{despreocupado}
  \definition{v.}{ser poupado de preocupações | despreocupar-se}
\end{entry}

\begin{entry}{省长}{sheng3zhang3}{9,4}{⽬、⾧}
  \definition*{s.}{Governador | governador de uma província}
\end{entry}

\begin{entry}{圣诞节}{sheng4dan4jie2}{5,8,5}{⼟、⾔、⾋}
  \definition*{s.}{Natal}
\end{entry}

\begin{entry}{圣地}{sheng4di4}{5,6}{⼟、⼟}
  \definition{s.}{terra santa (de uma religião) | lugar sagrado | santuário | cidade santa (como Jerusalém, Meca, etc.) | centro de interesse histórico}
\end{entry}

\begin{entry}{胜}{sheng4}{9}{⾁}[HSK 3]
  \definition{adj.}{soberbo; maravilhoso; adorável}
  \definition[场]{s.}{vitória; sucesso | penteado de mulher; joias usadas pelas mulheres na antiguidade}
  \definition{v.}{vencer (oposto de 负, 败) | derrotar | (frequentemente seguido por 于, etc.) superar; ser superior a; levar a melhor sobre | vencer; ter sucesso; derrotar o adversário | ultrapassar; ser superior ao outro | suportar; ser capaz de suportar ou aguentar}
  \seealsoref{败}{bai4}
  \seealsoref{负}{fu4}
  \seealsoref{于}{yu2}
\end{entry}

\begin{entry}{胜负}{sheng4fu4}{9,6}{⾁、⾙}[HSK 5]
  \definition{s.}{vitória ou derrota; sucesso ou fracasso}
\end{entry}

\begin{entry}{胜利}{sheng4li4}{9,7}{⾁、⼑}[HSK 3]
  \definition{adv.}{com sucesso; triunfantemente; atingir o objetivo previsto}
  \definition{v.}{ganhar; vencer; triunfar; ter sucesso}
\end{entry}

\begin{entry}{胜算}{sheng4suan4}{9,14}{⾁、⽵}
  \definition{s.}{probabilidade de sucesso | estratégia que garante o sucesso}
  \definition{v.}{ter certeza do sucesso}
\end{entry}

\begin{entry}{乘}{sheng4}{10}{⽲}
  \definition{clas.}{usado para carruagens de guerra puxada por quatro cavalos}
  \definition{s.}{obras históricas; livros de história geral | antigamente, uma carruagem puxada por quatro cavalos}
  \seeref{乘}{cheng2}
\end{entry}

\begin{entry}{盛宴}{sheng4yan4}{11,10}{⽫、⼧}
  \definition{s.}{celebração}
\end{entry}

\begin{entry}{剩}{sheng4}{12}{⼑}[HSK 5]
  \definition*{s.}{sobrenome Sheng}
  \definition{v.}{permanecer; ser deixado (para trás);}
\end{entry}

\begin{entry}{剩下}{sheng4 xia4}{12,3}{⼑、⼀}[HSK 5]
  \definition{v.}{permanecer; ser deixado (para trás); consumir e utilizar, restando apenas os resíduos}
\end{entry}

\begin{entry}{失}{shi1}{5}{⼤}
  \definition{s.}{deslize; erro; defeito; acidente}
  \definition{v.}{perder (oposto de 得) | perder; deixar escapar | não agir de acordo com; negligenciar; violar | perder o controle de | errar; cometer um deslize; apresentar defeito em | não consiguir encontrar | não conseguir atingir o objetivo | desviar-se do normal | quebrar (uma promessa); voltar atrás (na palavra dada) | não conseguir obter | se perder}
  \seealsoref{得}{de2}
\end{entry}

\begin{entry}{失败}{shi1bai4}{5,8}{⼤、⾒}[HSK 4]
  \definition{adj.}{insatisfatório; a maneira como as coisas aconteceram deixou muito a desejar; o resultado final deixou muito a desejar}
  \definition{v.}{perder; ser derrotado; não vencer em uma guerra ou competição | falhar; fracassar; não dar em nada; falhar em atingir um objetivo ou meta desejada (trabalho, carreira, etc.)}
\end{entry}

\begin{entry}{失落}{shi1luo4}{5,12}{⼤、⾋}
  \definition{s.}{frustração | decepção | perda}
  \definition{v.}{perder (algo) | cair (algo) | sentir uma sensação de perda}
\end{entry}

\begin{entry}{失眠}{shi1mian2}{5,10}{⼤、⽬}
  \definition{s.}{insônia}
  \definition{v.}{ter insônia}
\end{entry}

\begin{entry}{失去}{shi1qu4}{5,5}{⼤、⼛}[HSK 3]
  \definition{v.}{perder}
\end{entry}

\begin{entry}{失望}{shi1wang4}{5,11}{⼤、⽉}[HSK 4]
  \definition{adj.}{desapontado; decepcionado}
  \definition{v.}{ficar desapontado; ficar decepcionado; estar desapontado;}
\end{entry}

\begin{entry}{失误}{shi1wu4}{5,9}{⼤、⾔}[HSK 5]
  \definition[个]{s.}{erro; engano; equívoco; erros causados por negligência ou medidas inadequadas}
  \definition{v.}{cometer um erro; cometer um equívoco}
\end{entry}

\begin{entry}{失业}{shi1ye4}{5,5}{⼤、⼀}[HSK 4]
  \definition{v.}{não ter emprego; estar desempregado; estar sem trabalho}
\end{entry}

\begin{entry}{失意}{shi1yi4}{5,13}{⼤、⼼}
  \definition{adj.}{desapontado | frustrado}
\end{entry}

\begin{entry}{师}{shi1}{6}{⼱}
  \definition*{s.}{sobrenome Shi}
  \definition{s.}{professor | mestre | especialista | modelo | divisão do exército}
  \definition{v.}{despachar tropas}
\end{entry}

\begin{entry}{师傅}{shi1fu5}{6,12}{⼱、⼈}[HSK 5]
  \definition[个,位,名]{s.}{mestre; um trabalhador qualificado; título honorífico para pessoas habilidosas | mestre; professor (em certos ofícios); pessoas que ensinam técnicas em áreas como engenharia, comércio e teatro}
\end{entry}

\begin{entry}{诗}{shi1}{8}{⾔}[HSK 4]
  \definition*{s.}{sobrenome Shi}
  \definition*{s.}{O Livro das Canções《诗经》}
  \definition{s.}{poesia; verso; poema}
  \seealsoref{诗经}{shi1jing1}
\end{entry}

\begin{entry}{诗词}{shi1ci2}{8,7}{⾔、⾔}
  \definition{s.}{verso}
\end{entry}

\begin{entry}{诗歌}{shi1 ge1}{8,14}{⾔、⽋}[HSK 5]
  \definition[本,首,段]{s.}{poesia; poemas e canções; refere-se a todos os tipos de poesia}
\end{entry}

\begin{entry}{诗经}{shi1jing1}{8,8}{⾔、⽷}
  \definition*{s.}{Shijing, o Livro das Canções, antiga coleção de poemas chineses e um dos Cinco Clássicos do Confucionismo}
\end{entry}

\begin{entry}{诗句}{shi1ju4}{8,5}{⾔、⼝}
  \definition[行]{s.}{verso | versículo}
\end{entry}

\begin{entry}{诗人}{shi1 ren2}{8,2}{⾔、⼈}[HSK 4]
  \definition{s.}{poeta; escritor de poesia}
\end{entry}

\begin{entry}{诗意}{shi1yi4}{8,13}{⾔、⼼}
  \definition{adj.}{poético}
  \definition{s.}{poesia}
\end{entry}

\begin{entry}{湿}{shi1}{12}{⽔}[HSK 4]
  \definition{adj.}{molhado; úmido; algo com água ou com muita água dentro}
\end{entry}

\begin{entry}{十}{shi2}{2}{⼗}[HSK 1][Kangxi 24]
  \definition*{s.}{sobrenome Shi}
  \definition{num.}{dez; 10 | dezena | completo; no topo; máximo; referindo-se a algo que atingiu o ápice da perfeição ou plenitude | um monte de; indica que há muitos}
\end{entry}

\begin{entry}{十分}{shi2fen1}{2,4}{⼗、⼑}[HSK 2]
  \definition{adv.}{muito; totalmente; completamente; extremamente; indica um nível muito alto}
\end{entry}

\begin{entry}{十足}{shi2zu2}{2,7}{⼗、⾜}[HSK 5]
  \definition{adj.}{puro e simples; apenas este componente ou esta característica é muito evidente | 100\%; completo; total; muito satisfatório; muito adequado}
\end{entry}

\begin{entry}{石头}{shi2tou5}{5,5}{⽯、⼤}[HSK 3]
  \definition[块,堆,些]{s.}{rocha; pedra; uma substância muito dura que é o principal material da superfície da Terra}
\end{entry}

\begin{entry}{石油}{shi2you2}{5,8}{⽯、⽔}[HSK 3]
  \definition[桶,吨,升]{s.}{óleo; óleo fóssil; petróleo; um líquido inflamável extraído do solo, geralmente marrom escuro, preto ou verde escuro, do qual gasolina e outras substâncias podem ser obtidas}
\end{entry}

\begin{entry}{时}{shi2}{7}{⽇}[HSK 3]
  \definition*{s.}{sobrenome Shi}
  \definition{adj.}{atual; presente | temporário; oportuno}
  \definition{adv.}{de vez em quando; ocasionalmente; de ​​tempos em tempos; equivalente a 常常 ou 经常 | às vezes\dots às vezes\dots; dois caracteres 时 usados juntos são equivalentes a ``有时……有时……'' e ``一会儿……一会儿……''}
  \definition{clas.}{hora, cada uma das 24 partes iguais de um dia e uma noite; também usada como unidade legal de tempo}
  \definition{s.}{dias; tempos; longo período de tempo; refere-se a um período de tempo | tempo; tempo fixo; refere-se ao tempo especificado | hora; hora do dia | temporada | chance; oportunidade; momento oportuno | atual; presente | tempo verbal; uma categoria gramatical que utiliza certas formas gramaticais para indicar o momento em que uma ação ocorre; geralmente é dividida em presente, pretérito e futuro}
  \seealsoref{常常}{chang2 chang2}
  \seealsoref{经常}{jing1chang2}
  \seealsoref{一会儿……一会儿……}{yi1hui4r5 yi1hui4r5}
  \seealsoref{有时……有时……}{you3shi2 you3shi2}
\end{entry}

\begin{entry}{时差}{shi2cha1}{7,9}{⽇、⼯}
  \definition{s.}{diferença de tempo | \emph{jet lag}}
\end{entry}

\begin{entry}{时常}{shi2chang2}{7,11}{⽇、⼱}[HSK 5]
  \definition{adv.}{frequentemente; com frequência}
\end{entry}

\begin{entry}{时代}{shi2dai4}{7,5}{⽇、⼈}[HSK 3]
  \definition[个]{s.}{idade; era; tempos; época; períodos e fases históricas divididas de acordo com condições econômicas, políticas, culturais e outras | um período na vida de alguém; uma fase na vida de uma pessoa}
\end{entry}

\begin{entry}{时光}{shi2guang1}{7,6}{⽇、⼉}[HSK 5]
  \definition[台]{s.}{tempo; passagem do tempo | dias; horas; anos; épocas; períodos}
\end{entry}

\begin{entry}{时候}{shi2hou5}{7,10}{⽇、⼈}[HSK 1]
  \definition[个]{s.}{(um ponto no) tempo; momento; um determinado momento no tempo | (a duração do) tempo; um período de tempo com início e fim}
\end{entry}

\begin{entry}{时机}{shi2ji1}{7,6}{⽇、⽊}[HSK 5]
  \definition{s.}{oportunidade; momento oportuno}
\end{entry}

\begin{entry}{时间}{shi2jian1}{7,7}{⽇、⾨}[HSK 1]
  \definition[段]{s.}{tempo; refere-se à forma de existência do movimento da matéria, um sistema contínuo composto pelo passado, presente e futuro | tempo; período (duração); um período de tempo com início e fim | tempo (um ponto); em algum momento do tempo}
\end{entry}

\begin{entry}{时刻}{shi2ke4}{7,8}{⽇、⼑}[HSK 3]
  \definition{adv.}{constantemente; sempre; a cada momento; frequentemente}
  \definition[个,段]{s.}{tempo; hora; momento; conjuntura; um ponto no tempo}
\end{entry}

\begin{entry}{时时}{shi2shi2}{7,7}{⽇、⽇}
  \definition{adv.}{muitas vezes | constantemente}
\end{entry}

\begin{entry}{时事}{shi2shi4}{7,8}{⽇、⼅}[HSK 5]
  \definition{s.}{acontecimentos atuais; assuntos atuais; eventos atuais | tendências atuais | como as coisas estão indo | a situação atual}
\end{entry}

\begin{entry}{实惠}{shi2hui4}{8,12}{⼧、⼼}[HSK 5]
  \definition{adj.}{sólido; substancial; benefícios práticos}
  \definition{s.}{benefício material; benefícios tangíveis; benefícios reais}
\end{entry}

\begin{entry}{实际}{shi2ji4}{8,7}{⼧、⾩}[HSK 2]
  \definition{adj.}{real; efetivo; concreto; prático | factual; prático; realista; de acordo com os fatos}
  \definition{s.}{realidade; prática; coisas e situações que existem objetivamente}
\end{entry}

\begin{entry}{实际上}{shi2 ji4 shang4}{8,7,3}{⼧、⾩、⼀}[HSK 3]
  \definition{adv.}{de fato; na verdade}
\end{entry}

\begin{entry}{实力}{shi2li4}{8,2}{⼧、⼒}[HSK 3]
  \definition{s.}{força real; geralmente se refere à força militar e econômica de um país, grupo ou indivíduo, e também se refere à capacidade de um indivíduo ou grupo em uma competição}
\end{entry}

\begin{entry}{实施}{shi2shi1}{8,9}{⼧、⽅}[HSK 4]
  \definition{v.}{colocar em vigor; implementar (leis, políticas, etc.); executar; trazer (colocar) algo em vigor; fazer cumprir; colocar algo em (prática)}
\end{entry}

\begin{entry}{实习}{shi2xi2}{8,3}{⼧、⼄}[HSK 2]
  \definition{s.}{estagiário; prática; estágio}
  \definition{v.}{aplicar e testar os conhecimentos teóricos aprendidos no trabalho prático, a fim de exercitar a capacidade profissional}
\end{entry}

\begin{entry}{实现}{shi2xian4}{8,8}{⼧、⾒}[HSK 2]
  \definition{v.}{alcançar; atingir; realizar; concretizar; tornar (ideais, planos, etc.) realidade}
\end{entry}

\begin{entry}{实行}{shi2xing2}{8,6}{⼧、⾏}[HSK 3]
  \definition{v.}{praticar; implementar; executar; colocar em prática; realizar (programa, política, plano, etc.) por meio de ação}
\end{entry}

\begin{entry}{实验}{shi2yan4}{8,10}{⼧、⾺}[HSK 3]
  \definition[个,次]{s.}{teste; experimento; trabalho de laboratório}
  \definition{v.}{testar; experimentar; realizar uma operação ou se envolver em uma atividade para testar uma teoria ou hipótese científica}
\end{entry}

\begin{entry}{实验室}{shi2 yan4 shi4}{8,10,9}{⼧、⾺、⼧}[HSK 3]
  \definition[个,间]{s.}{laboratório; salas especiais para experimentos científicos}
\end{entry}

\begin{entry}{实用}{shi2yong4}{8,5}{⼧、⽤}[HSK 4]
  \definition{adj.}{prático; pragmático; funcional; atende aos requisitos reais da aplicação}
  \definition{v.}{colocar em uso prático}
\end{entry}

\begin{entry}{实在}{shi2zai4}{8,6}{⼧、⼟}[HSK 2]
  \definition{adj.}{honesto; sincero | verdadeiro; honesto; realista; não é falso, não é enganador}
  \definition{adv.}{verdadeiramente; de fato; na verdade; usado para reforçar o tom afirmativo, enfatizando que a situação é realmente assim}
\end{entry}

\begin{entry}{拾}{shi2}{9}{⼿}[HSK 5]
  \definition{num.}{dez (usado no lugar do numeral 十 em cheques, notas bancárias, etc., para evitar erros ou alterações)}
  \definition{v.}{pegar (do chão); recolher}
\end{entry}

\begin{entry}{食品}{shi2 pin3}{9,9}{⾷、⼝}[HSK 3]
  \definition[种]{s.}{comida; gêneros alimentícios; provisões; alimentos vendidos em lojas que passaram por algum processamento}
\end{entry}

\begin{entry}{食堂}{shi2 tang2}{9,11}{⾷、⼟}[HSK 4]
  \definition[个,间]{s.}{cantina; refeitório}
\end{entry}

\begin{entry}{食物}{shi2wu4}{9,8}{⾷、⽜}[HSK 2]
  \definition[种]{s.}{comida; alimentos; comestíveis}
\end{entry}

\begin{entry}{使}{shi3}{8}{⼈}[HSK 3]
  \definition{conj.}{se; supondo; usado como a primeira cláusula de uma frase complexa; indica uma relação hipotética; equivalente a 假如}
  \definition{s.}{enviado; mensageiro; pessoas em uma missão}
  \definition{v.}{enviar; despachar; dizer a alguém para fazer algo | usar; empregar; aplicar | deixar; chamar; habilitar}
  \seealsoref{假如}{jia3ru2}
\end{entry}

\begin{entry}{使得}{shi3 de5}{8,11}{⼈、⼻}[HSK 5]
  \definition{v.}{ser utilizável; poder ser usado | ser viável; ser exequível; ser possível;  poder fazer | fazer; tornar; causar um determinado resultado (intenção, plano, coisa)}
\end{entry}

\begin{entry}{使劲}{shi3 jin4}{8,7}{⼈、⼒}[HSK 4]
  \definition{v.+compl.}{colocar energia; exercer toda a sua força | esforçar-se para ajudar; colocar energia para ajudar}
\end{entry}

\begin{entry}{使用}{shi3yong4}{8,5}{⼈、⽤}[HSK 2]
  \definition{v.}{usar; empregar; aplicar; fazer com que pessoas, equipamentos, fundos, etc. sirvam a um determinado propósito}
\end{entry}

\begin{entry}{始终}{shi3zhong1}{8,8}{⼥、⽷}[HSK 3]
  \definition{adv.}{sempre; o tempo todo; durante todo; do começo ao fim; indica continuidade do início ao fim}
  \definition{s.}{todo o processo do começo ao fim}
\end{entry}

\begin{entry}{屎}{shi3}{9}{⼫}
  \definition{s.}{fezes | excrementos | (forma ligada) secreção (do ouvido, olho, etc.)}
\end{entry}

\begin{entry}{士兵}{shi4bing1}{3,7}{⼠、⼋}[HSK 4]
  \definition[名,个]{s.}{soldado; militar; termo coletivo para oficiais não comissionados e soldados; os membros mais jovens do exército}
\end{entry}

\begin{entry}{世代}{shi4dai4}{5,5}{⼀、⼈}
  \definition{adv.}{por muitas gerações, eras}
  \definition{s.}{geração | era}
\end{entry}

\begin{entry}{世纪}{shi4ji4}{5,6}{⼀、⽷}[HSK 3]
  \definition[个,段]{s.}{século; uma unidade para calcular anos, cem anos é um século}
\end{entry}

\begin{entry}{世界}{shi4jie4}{5,9}{⼀、⽥}[HSK 3]
  \definition[个,片,种]{s.}{mundo; todos os lugares da Terra | a soma da natureza e da sociedade humana; refere-se à soma de toda a existência objetiva na natureza e na sociedade humana | campo; refere-se a uma determinada área ou campo | o universo sem limites; costumava ser um termo budista, mas agora também se refere ao mundo natural ilimitado e à sociedade humana | situação social; a situação ou atmosfera social de um determinado período}
\end{entry}

\begin{entry}{世界杯}{shi4jie4bei1}{5,9,8}{⼀、⽥、⽊}[HSK 3]
  \definition*{s.}{Copa do Mundo; troféu da copa do mundo}
\end{entry}

\begin{entry}{世锦赛}{shi4jin3sai4}{5,13,14}{⼀、⾦、⾙}
  \definition*{s.}{Campeonato Mundial}
\end{entry}

\begin{entry}{市}{shi4}{5}{⼱}[HSK 2]
  \definition{s.}{mercado; lugar onde se concentra o comércio | cidade; município; áreas densamente povoadas, com indústrias, comércio e cultura desenvolvidos | relativo ao sistema tradicional chinês de pesos e medidas; unidades administrativas, incluindo cidades sob jurisdição direta e cidades sob jurisdição provincial (ou autônoma) | unidade padrão de mercado; pertencente ao sistema municipal (unidades de medida) | preço de transação no mercado}
  \definition{v.}{comprar ou vender; fazer transações}
\end{entry}

\begin{entry}{市场}{shi4chang3}{5,6}{⼱、⼟}[HSK 3]
  \definition[家]{s.}{mercado (também no abstrato); um lugar fixo onde as pessoas compram e vendem coisas juntas | área de \emph{marketing}; região onde o produto é vendido | âmbito de influência (figurado); uma metáfora para o escopo e o grau em que uma determinada ideia ou comportamento é aceito por outros}
\end{entry}

\begin{entry}{市尺}{shi4 chi3}{5,4}{⼱、⼫}
  \definition{clas.}{chi, unidade chinesa de comprimento igual a um terço de um metro; a principal unidade de comprimento no sistema urbano é um chi}
\end{entry}

\begin{entry}{市区}{shi4 qu1}{5,4}{⼱、⼖}[HSK 4]
  \definition[个]{s.}{\emph{downtown}; centro da cidade; distrito urbano; áreas que ficam dentro dos limites da cidade e geralmente têm uma alta concentração de população e estoque de moradias.}
\end{entry}

\begin{entry}{市长}{shi4 zhang3}{5,4}{⼱、⾧}[HSK 2]
  \definition[个,位,名]{s.}{prefeito; chefe administrativo responsável pela administração de uma cidade}
\end{entry}

\begin{entry}{市中心}{shi4zhong1xin1}{5,4,4}{⼱、⼁、⼼}
  \definition{s.}{centro da cidade}
\end{entry}

\begin{entry}{示范}{shi4fan4}{5,9}{⽰、⾋}[HSK 5]
  \definition{v.}{demonstrar; dar o exemplo; criar um modelo que todos possam aprender}
\end{entry}

\begin{entry}{似的}{shi4de5}{6,8}{⼈、⽩}[HSK 4]
  \definition{part.}{como; como\dots como; como se (embora); usada após uma palavra ou frase para indicar uma semelhança com algo ou uma situação | usada para indicar alto grau}
\end{entry}

\begin{entry}{式}{shi4}{6}{⼷}[HSK 5]
  \definition*{s.}{sobrenome Shi}
  \definition{s.}{tipo; estilo | forma; padrão | ritual; cerimônia | fórmula; conjunto de símbolos que expressam uma lei natural na ciência natural | humor; modo; categoria gramatical que expressa a atitude subjetiva do falante em relação ao que está sendo dito, como narrativa, imperativa e condicional}
\end{entry}

\begin{entry}{事}{shi4}{8}{⼅}[HSK 1]
  \definition[件,桩,回]{s.}{assunto; questão; coisa; negócio | problema; acidente | emprego; trabalho | responsabilidade; envolvimento | caso, coisa; o que aconteceu}
  \definition{v.}{servir; atender | estar envolvido em; dedicar-se a}
\end{entry}

\begin{entry}{事故}{shi4gu4}{8,9}{⼅、⽁}[HSK 3]
  \definition[起,桩,次,场]{s.}{acidente; perdas ou desastres repentinos, muitas vezes relacionados ao transporte, produção, trabalho e segurança pessoal}
\end{entry}

\begin{entry}{事件}{shi4jian4}{8,6}{⼅、⼈}[HSK 3]
  \definition[个,件,次]{s.}{evento; incidente; grandes eventos na história ou na sociedade}
\end{entry}

\begin{entry}{事情}{shi4qing5}{8,11}{⼅、⼼}[HSK 2]
  \definition[件,个,些,种]{s.}{assunto; questão; coisa; negócio | erro; acidente; infortúnio | (coloquial) emprego; trabalho}
\end{entry}

\begin{entry}{事儿}{shi4r5}{8,2}{⼅、⼉}
  \definition[件,桩]{s.}{o emprego | negócio | afazeres | assunto que precisa ser resolvido | matéria}
\end{entry}

\begin{entry}{事实}{shi4shi2}{8,8}{⼅、⼧}[HSK 3]
  \definition[个,件]{s.}{mito; lenda; uma narrativa sobre alguém ou algo que foi transmitida oralmente}
  \definition{v.}{dizer; contar; ser dito; contar a história}
\end{entry}

\begin{entry}{事实上}{shi4 shi2 shang4}{8,8,3}{⼅、⼧、⼀}[HSK 3]
  \definition{adv.}{realmente; de fato; na realidade; na verdade; de fato}
\end{entry}

\begin{entry}{事物}{shi4wu4}{8,8}{⼅、⽜}[HSK 4]
  \definition{s.}{coisa; objeto; todos os objetos e fenômenos que existem objetivamente}
\end{entry}

\begin{entry}{事先}{shi4xian1}{8,6}{⼅、⼉}[HSK 4]
  \definition{adv.}{antes; de antemão; com antecedência; antecipadamente}
\end{entry}

\begin{entry}{事业}{shi4ye4}{8,5}{⼅、⼀}[HSK 3]
  \definition[个]{s.}{causa; carreira; empreendimento; atividades regulares realizadas por pessoas com um determinado objetivo, escala e sistema que têm impacto no desenvolvimento social | instituição; instalações; unidade de trabalho apoiada financeiramente pelo governo; refere-se especificamente a empresas que não têm rendimentos de produção, são financiadas pelo Estado e não realizam contabilidade económica}
\end{entry}

\begin{entry}{势力}{shi4li4}{8,2}{⼒、⼒}[HSK 5]
  \definition{s.}{força; poder; influência; forças políticas, econômicas, militares, etc.}
\end{entry}

\begin{entry}{视角}{shi4jiao3}{8,7}{⾒、⾓}
  \definition{s.}{ângulo do qual se observa um objeto | (figurativo) perspectiva, ponto de vista, quadro de referência | (cinematografia) ângulo da câmera | (percepção visual) ângulo visual (o ângulo que um objeto visto subtende no olho) | (fotografia) ângulo de visão}
\end{entry}

\begin{entry}{视频}{shi4pin2}{8,13}{⾒、⾴}[HSK 5]
  \definition[个,段]{s.}{vídeo; videoclipe}
\end{entry}

\begin{entry}{视为}{shi4 wei2}{8,4}{⾒、⼂}[HSK 5]
  \definition{v.}{considerar; ver como; considerar como; considerar ser; achar que é}
\end{entry}

\begin{entry}{试}{shi4}{8}{⾔}[HSK 1]
  \definition{s.}{teste; exame; avaliação de conhecimentos ou habilidades através de métodos específicos}
  \definition{v.}{tentar; investigar resultados ou verificar a natureza, não se envolver formalmente (em determinada atividade)}
\end{entry}

\begin{entry}{试卷}{shi4juan4}{8,8}{⾔、⼙}[HSK 4]
  \definition[分,张]{s.}{folha de teste; folha de exame; papel usado para escrever as respostas nos exames}
\end{entry}

\begin{entry}{试题}{shi4 ti2}{8,15}{⾔、⾴}[HSK 3]
  \definition[道]{s.}{questões de um exame}
\end{entry}

\begin{entry}{试图}{shi4tu2}{8,8}{⾔、⼞}[HSK 5]
  \definition{v.}{tentar; pretender, fazer o possível para realizar algo}
\end{entry}

\begin{entry}{试验}{shi4yan4}{8,10}{⾔、⾺}[HSK 3]
  \definition{v.}{testar; fazer um teste; fazer um experimento; para examinar o efeito ou desempenho de algo, primeiro experimente em um laboratório ou em uma escala menor}
\end{entry}

\begin{entry}{室}{shi4}{9}{⼧}[HSK 3]
  \definition*{s.}{sobrenome Shi}
  \definition*{s.}{Shi, a décima terceira das vinte e oito constelações da esfera celeste, composta por duas estrelas em linha reta na constelação de Pégaso}
  \definition{s.}{sala; quarto; casa | departamento; sala como unidade administrativa ou de trabalho; órgãos públicos, fábricas, escolas e outras unidades de trabalho internas | esposa; familiares ou esposa | família; clã | cavidade; órgão com forma semelhante a uma câmara}
\end{entry}

\begin{entry}{是}{shi4}{9}{⽇}[HSK 1]
  \definition*{s.}{sobrenome Shi}
  \definition{adj.}{correto; certo | verdadeiro}
  \definition{adv.}{(expressar afirmação firme) de fato; realmente}
  \definition{pron.}{isso; isto |  todos; qualquer um; usado antes de substantivos, tem o significado de 凡是}
  \definition{s.}{assuntos (importantes); grandes planos}
  \definition{v.}{usado como “ser” antes de substantivos ou pronomes para identificar, descrever ou ampliar o sujeito; indica que duas coisas são iguais, ou que a segunda explica a primeira | usado entre duas palavras idênticas; relacionar duas palavras semelhantes |  (usado antes de substantivos) ser exatamente; ser corretamente; usado antes de substantivos, tem o significado de 适合 | elogiar; justificar | expressar afirmação ou concordância (frequentemente usado sozinho) | usado para escolher perguntas, perguntas sim/não ou perguntas retóricas | (usado no início de uma frase) enfatizar uma determinada parte de uma frase | usado em perguntas sim-não}
  \seealsoref{凡是}{fan2shi4}
  \seealsoref{适合}{shi4he2}
\end{entry}

\begin{entry}{是不是}{shi4 bu2 shi4}{9,4,9}{⽇、⼀、⽇}[HSK 1]
  \definition{expr.}{sim ou não; é ou não é; se ou não; questões levantadas sobre a confirmação e a negação dos fatos}
\end{entry}

\begin{entry}{是的}{shi4de5}{9,8}{⽇、⽩}
  \definition{adv.}{sim | está certo}
\end{entry}

\begin{entry}{是否}{shi4fou3}{9,7}{⽇、⼝}[HSK 4]
  \definition{adv.}{se; se ou não}
\end{entry}

\begin{entry}{适合}{shi4he2}{9,6}{⾡、⼝}[HSK 3]
  \definition{v.}{servir; caber; se adequar; atender às necessidades de uma determinada situação ou pessoa}
\end{entry}

\begin{entry}{适应}{shi4ying4}{9,7}{⾡、⼴}[HSK 3]
  \definition{v.}{ajustar-se; adequar-se; adaptar-se; fazer as alterações correspondentes para se adequar à medida que as condições mudam}
\end{entry}

\begin{entry}{适用}{shi4 yong4}{9,5}{⾡、⽤}[HSK 3]
  \definition{adj.}{adequado; aplicável}
\end{entry}

\begin{entry}{收}{shou1}{6}{⽁}[HSK 2]
  \definition{expr.}{aos cuidados de (usado na linha de endereço após o nome)}
  \definition{v.}{recolocar; juntar; reunir e juntar coisas espalhadas ou dispersas | recolher; cobrar | ganhar; obter (benefícios econômicos) | colher; recolher; colher ou cortar frutas, legumes, cereais maduros, etc. | aceitar; receber; acolher | controlar; restringir; restringir, controlar os sentimentos ou ações, para voltar ao estado normal | finalizar; parar; concluir; encerrar | prender; deter; colocar sob custódia}
\end{entry}

\begin{entry}{收到}{shou1 dao4}{6,8}{⽁、⼑}[HSK 2]
  \definition{v.}{conseguir; obter; receber; alcançar}
\end{entry}

\begin{entry}{收费}{shou1 fei4}{6,9}{⽁、⾙}[HSK 3]
  \definition{v.}{cobrar; cobrar taxas}
\end{entry}

\begin{entry}{收购}{shou1 gou4}{6,8}{⽁、⾙}[HSK 5]
  \definition{v.}{comprar; adquirir; comprar muito em vários lugares | adquirir uma empresa; obter o controle efetivo de uma empresa por meio de dinheiro, transações de ações, etc.}
\end{entry}

\begin{entry}{收回}{shou1 hui2}{6,6}{⽁、⼞}[HSK 4]
  \definition{v.}{retomar; recuperar; relembrar; recordar; receber de volta o que foi enviado ou emprestado, ou o dinheiro que foi emprestado ou usado | sacar; retirar; recolher; rescindir; cancelar (uma opinião, ordem, etc.)}
\end{entry}

\begin{entry}{收获}{shou1huo4}{6,10}{⽁、⾋}[HSK 4]
  \definition[次,番,份]{s.}{resultados; ganhos; metaforicamente falando, conhecimento, experiência, etc. obtidos em estudo ou trabalho; os resultados obtidos por meio de trabalho árduo | colheita; colheita de safras}
  \definition{v.}{colher; juntar as colheitas;}
\end{entry}

\begin{entry}{收集}{shou1 ji2}{6,12}{⽁、⾫}[HSK 5]
  \definition{v.}{coletar; reunir; recolher}
\end{entry}

\begin{entry}{收据}{shou1ju4}{6,11}{⽁、⼿}
  \definition[张]{s.}{recibo | \emph{voucher}}
\end{entry}

\begin{entry}{收看}{shou1 kan4}{6,9}{⽁、⽬}[HSK 3]
  \definition{v.}{assistir (a um programa de TV)}
\end{entry}

\begin{entry}{收敛}{shou1lian3}{6,11}{⽁、⽁}
  \definition{v.}{diminuir | desaparecer | fazer desaparecer | exercer restrição | conter (alegria, arrogância, etc.) | constringir | (matemática) convergir}
\end{entry}

\begin{entry}{收买}{shou1mai3}{6,6}{⽁、⼄}
  \definition{v.}{subornar | comprar}
\end{entry}

\begin{entry}{收入}{shou1ru4}{6,2}{⽁、⼊}[HSK 2]
  \definition[笔,个]{s.}{renda; salário; dinheiro recebido}
  \definition{v.}{receber dinheiro | coletar; receber}
\end{entry}

\begin{entry}{收拾}{shou1shi5}{6,9}{⽁、⼿}[HSK 5]
  \definition{v.}{arrumar; empacotar; limpar; organizar, policiar, restaurar a normalidade em situações adversas | consertar; reparar; restaurar algo que está danificado ao seu estado ou função original |  punir; punir alguém, geralmente com medidas mais severas | matar}
\end{entry}

\begin{entry}{收听}{shou1 ting1}{6,7}{⽁、⼝}[HSK 3]
  \definition{v.}{ouvir (rádio)}
\end{entry}

\begin{entry}{收益}{shou1yi4}{6,10}{⽁、⽫}[HSK 4]
  \definition{s.}{lucro; renda; benefício; ganhos; vantagens ou benefícios obtidos}
\end{entry}

\begin{entry}{收音机}{shou1yin1ji1}{6,9,6}{⽁、⾳、⽊}[HSK 3]
  \definition[部,台]{s.}{rádio; sem fio; um termo geral para receptores de rádio}
\end{entry}

\begin{entry}{手}{shou3}{4}{⼿}[HSK 1][Kangxi 64]
  \definition{adj.}{prático; conveniente}
  \definition{adv.}{pessoalmente | para habilidade ou destreza}
  \definition{clas.}{usado para habilidades e competências | usado para indicar o número de vezes em que algo foi feito}
  \definition[双,只]{s.}{mão | pessoa proficiente em determinada atividade | habilidade; meios; referência a habilidades, técnicas ou meios | uma pessoa que faz ou é boa em determinado trabalho}
  \definition{v.}{ter na mão; segurar}
\end{entry}

\begin{entry}{手臂}{shou3bi4}{4,17}{⼿、⾁}
  \definition{s.}{braço}
\end{entry}

\begin{entry}{手边}{shou3bian1}{4,5}{⼿、⾡}
  \definition{adv.}{à mão | na mão}
\end{entry}

\begin{entry}{手表}{shou3biao3}{4,8}{⼿、⾐}[HSK 2]
  \definition[块,只,个]{s.}{relógio de pulso}
\end{entry}

\begin{entry}{手段}{shou3 duan4}{4,9}{⼿、⽎}[HSK 5]
  \definition[种]{s.}{meios; meio; medida; método; métodos e técnicas utilizados para atingir um determinado objetivo | truque; artifício; métodos inadequados de lidar com as pessoas | habilidade; capacidade; delicadeza; sutileza; técnica}
\end{entry}

\begin{entry}{手法}{shou3fa3}{4,8}{⼿、⽔}[HSK 5]
  \definition{s.}{habilidade; técnica; técnicas de criação (de obras literárias e artísticas) | truque; artifício; artimanha; refere-se a métodos inadequados usados para lidar com as pessoas}
\end{entry}

\begin{entry}{手工}{shou3gong1}{4,3}{⼿、⼯}[HSK 4]
  \definition{s.}{trabalho manual; trabalho feito à mão | método de operação manual; método manual, sem máquina | remuneração por trabalho manual, braçal; custo de mão de obra braçal}
\end{entry}

\begin{entry}{手工艺人}{shou3gong1 yi4ren2}{4,3,4,2}{⼿、⼯、⾋、⼈}
  \definition{s.}{artesão}
\end{entry}

\begin{entry}{手机}{shou3ji1}{4,6}{⼿、⽊}[HSK 1]
  \definition[部,台,个]{s.}{celular; telefone celular; telefone móvel}
\end{entry}

\begin{entry}{手里}{shou3 li3}{4,7}{⼿、⾥}[HSK 4]
  \definition[个]{s.}{(uma situação está) nas mãos de alguém | em mãos}
\end{entry}

\begin{entry}{手刹}{shou3sha1}{4,8}{⼿、⼑}
  \definition{s.}{freio de mão}
\end{entry}

\begin{entry}{手术}{shou3shu4}{4,5}{⼿、⽊}[HSK 4]
  \definition[个]{s.}{cirurgia; operação (cirúrgica); método de tratamento no qual o médico usa uma faca, tesoura etc. para fazer uma incisão em uma parte do corpo do paciente}
  \definition{v.}{realizar uma cirurgia}
\end{entry}

\begin{entry}{手套}{shou3tao4}{4,10}{⼿、⼤}[HSK 4]
  \definition[副,套,双,种]{s.}{luvas; itens usados ​​nas mãos, feitos de algodão, lã, couro, etc., para proteger as mãos ou manter o frio longe}
\end{entry}

\begin{entry}{手续}{shou3xu4}{4,11}{⼿、⽷}[HSK 3]
  \definition[项]{s.}{processo; formalidade; procedimento; procedimentos realizados de acordo com os regulamentos}
\end{entry}

\begin{entry}{手指}{shou3zhi3}{4,9}{⼿、⼿}[HSK 3]
  \definition[个,根,只]{s.}{dedo da mão}
\end{entry}

\begin{entry}{守}{shou3}{6}{⼧}[HSK 4]
  \definition*{s.}{sobrenome Shou}
  \definition{adv.}{próximo; perto de; perto de algum lugar em posição, perto de algum lugar}
  \definition{v.}{guardar; defender; estar presente para cuidar; não ir embora | manter vigilância; defender do ataque do oponente em uma luta ou confronto | observar; cumprir; respeitar; fazer as coisas como elas devem ser feitas | manter, observar a integridade; honrar a palavra de alguém; manter a palavra de alguém}
\end{entry}

\begin{entry}{守门员}{shou3men2yuan2}{6,3,7}{⼧、⾨、⼝}
  \definition{s.}{goleiro}
\end{entry}

\begin{entry}{首}{shou3}{9}{⾸}[HSK 4][Kangxi 185]
  \definition*{s.}{sobrenome Shou}
  \definition{adj.}{primeiro}
  \definition{adv.}{inicialmente; como o primeiro; em primeiro lugar}
  \definition{clas.}{para canções e poemas}
  \definition{s.}{cabeça | cabeça; chefe; líder | capital (cidade)}
  \definition{v.}{apresentar acusações contra alguém}
\end{entry}

\begin{entry}{首都}{shou3du1}{9,10}{⾸、⾢}[HSK 3]
  \definition[个,座]{s.}{capital (cidade); a sede do mais alto poder político do país e o centro político do país}
\end{entry}

\begin{entry}{首席执行官}{shou3xi2 zhi2xing2 guan1}{9,10,6,6,8}{⾸、⼱、⼿、⾏、⼧}
  \definition{s.}{\emph{chief executive officer}, CEO}
\end{entry}

\begin{entry}{首先}{shou3xian1}{9,6}{⾸、⼉}[HSK 3]
  \definition{adv.}{primeiramente; antes de todos os outros}
  \definition{conj.}{acima de tudo; primeiramente; em primeiro lugar}
\end{entry}

\begin{entry}{首相}{shou3xiang4}{9,9}{⾸、⽬}
  \definition*{s.}{Primeiro-Ministro (Japão, UK, etc.)}
\end{entry}

\begin{entry}{掱}{shou3}{12}{⼿}
  \variantof{手}
\end{entry}

\begin{entry}{寿司}{shou4 si1}{7,5}{⼨、⼝}[HSK 5]
  \definition[份]{s.}{\emph{sushi}; iguaria tradicional japonesa}
\end{entry}

\begin{entry}{受}{shou4}{8}{⼜}[HSK 3]
  \definition{v.}{receber; aceitar | sofrer; ser submetido a | aguentar; suportar; tolerar | ser agradável}
\end{entry}

\begin{entry}{受不了}{shou4bu5liao3}{8,4,2}{⼜、⼀、⼅}[HSK 4]
  \definition{adj.}{intolerável; insuportável}
  \definition{v.}{ser insuportável; não poder suportar algo; não suportar algo}
\end{entry}

\begin{entry}{受到}{shou4dao4}{8,8}{⼜、⼑}[HSK 2]
  \definition{v.}{receber; receber itens, mensagens, instruções, etc. fornecidos por outras pessoas}
\end{entry}

\begin{entry}{受得了}{shou4de5liao3}{8,11,2}{⼜、⼻、⼅}
  \definition{v.}{suportar | aguentar}
\end{entry}

\begin{entry}{受伤}{shou4shang1}{8,6}{⼜、⼈}[HSK 3]
  \definition{v.}{ser ferido; sofrer uma lesão}
\end{entry}

\begin{entry}{受限}{shou4xian4}{8,8}{⼜、⾩}
  \definition{v.}{ser limitado | ser restrito | ser constrangido}
\end{entry}

\begin{entry}{受灾}{shou4 zai1}{8,7}{⼜、⽕}[HSK 5]
  \definition{v.}{ser atingido por um desastre natural (ou calamidade) | ser atingido por uma adversidade natural}
\end{entry}

\begin{entry}{售货员}{shou4huo4yuan2}{11,8,7}{⼝、⾙、⼝}[HSK 4]
  \definition[个]{s.}{vendedor; balconista; assistente de loja; equipe que vende produtos em lojas}
\end{entry}

\begin{entry}{瘦}{shou4}{14}{⽧}[HSK 5]
  \definition{adj.}{magro; esquelético (oposto de 胖, 肥) | magro (oposto de 肥) | apertado (oposto de 肥) | infértil; pobre | esquelético; pouca gordura; pouca carne (em oposição a 或 ou 肥) | (roupas, sapatos, meias, etc.) apertado (em oposição a 肥) |magra; (carne comestível) com baixo teor de gordura (em oposição a 肥)}
  \definition{v.}{perder peso}
  \seealsoref{肥}{fei2}
  \seealsoref{或}{huo4}
  \seealsoref{胖}{pang4}
\end{entry}

\begin{entry}{书}{shu1}{4}{⼄}[HSK 1]
  \definition*{s.}{sobrenome Shu}
  \definition[本,册,部,套,卷]{s.}{livro; obras encadernadas | carta; carta especial | documento | estilo de caligrafia; escrita}
  \definition{v.}{escrever; registrar}
\end{entry}

\begin{entry}{书包}{shu1 bao1}{4,5}{⼄、⼓}[HSK 1]
  \definition[个,款]{s.}{mochila para guardar livros e materiais escolares}
\end{entry}

\begin{entry}{书店}{shu1 dian4}{4,8}{⼄、⼴}[HSK 1]
  \definition[个,家]{s.}{livraria; lojas que vendem livros}
\end{entry}

\begin{entry}{书法}{shu1fa3}{4,8}{⼄、⽔}[HSK 5]
  \definition[幅,卷,种,派]{s.}{caligrafia; arte de escrever caracteres, especialmente arte de escrever caracteres chineses com um pincel}
\end{entry}

\begin{entry}{书柜}{shu1 gui4}{4,8}{⼄、⽊}[HSK 5]
  \definition{s.}{estante; armário de livros}
\end{entry}

\begin{entry}{书记}{shu1ji5}{4,5}{⼄、⾔}
  \definition{s.}{secretário (chefe de um ramo de um partido socialista ou comunista) | atendente | balconista | escriturário}
\end{entry}

\begin{entry}{书架}{shu1jia4}{4,9}{⼄、⽊}[HSK 3]
  \definition[个,种,套]{s.}{estante de livros}
\end{entry}

\begin{entry}{书桌}{shu1 zhuo1}{4,10}{⼄、⽊}[HSK 5]
  \definition[个,张]{s.}{escrivaninha; mesa para ler e escrever}
\end{entry}

\begin{entry}{叔叔}{shu1shu5}{8,8}{⼜、⼜}
  \definition[个]{s.}{tio; irmão mais novo do pai | tio, dirigindo-se a um homem da mesma geração que o pai e mais jovem em idade}
\end{entry}

\begin{entry}{疏}{shu1}{12}{⽦}
  \definition*{s.}{sobrenome Shu}
  \definition{adj.}{fino; esparso; disperso (oposto a 密) | espalhado; disperso; difuso; a distância entre as coisas é grande; as lacunas entre as partes das coisas são grandes | distante; relacionamento distante; não próximo (de relações familiares ou sociais) | não familiarizado com; desconhecido | escasso; vazio}
  \definition{s.}{memorial; memorial ao trono; um texto em que um ministro na era feudal apresentava seus assuntos ao monarca em detalhes | comentário; anotações mais detalhadas de livros antigos do que 注}
  \definition{v.}{dragar (um rio, etc.) | negligenciar | dispersar; espalhar}
  \seealsoref{密}{mi4}
  \seealsoref{注}{zhu4}
\end{entry}

\begin{entry}{舒服}{shu1fu5}{12,8}{⾆、⽉}[HSK 2]
  \definition{adj.}{confortável; sentir-se relaxado e feliz, tanto física quanto mentalmente}
\end{entry}

\begin{entry}{舒适}{shu1shi4}{12,9}{⾆、⾡}[HSK 4]
  \definition{adj.}{aconchegante; confortável; acolhedor; cômodo}
\end{entry}

\begin{entry}{输}{shu1}{13}{⾞}[HSK 3]
  \definition{v.}{transportar; entregar | contribuir com dinheiro; doar | perder; falhar; ser batido; ser derrotado}
\end{entry}

\begin{entry}{输出}{shu1 chu1}{13,5}{⾞、⼐}[HSK 5]
  \definition{v.}{exportar (de dentro para fora); transportar (de dentro) para fora | exportar; vender ou distribuir no exterior ou fora do país | emitir informações, programas, dados, sinais, etc. a partir de uma máquina; enviar por uma determinada instituição ou dispositivo (energia, sinal, etc.)}
\end{entry}

\begin{entry}{输入}{shu1ru4}{13,2}{⾞、⼊}[HSK 3]
  \definition{v.}{introduzir; importar; comprar bens, introduzir tecnologia, contratar mão de obra, introduzir capital, etc. | inserir informações, programas, dados, sinais, etc. em uma máquina}
\end{entry}

\begin{entry}{蔬菜}{shu1cai4}{15,11}{⾋、⾋}[HSK 5]
  \definition[样,种]{s.}{verduras; legumes; vegetais; ervas que podem ser usadas na culinária}
\end{entry}

\begin{entry}{熟}{shu2}{15}{⽕}[HSK 2]
  \definition{adj.}{maduro (frutos) | pronto; cozido | processado, fabricado ou exercitado | familiar, bem conhecido; conhecido por ser comum ou frequentemente utilizado | habilidoso;  (trabalho, tecnologia) experiente; não é novato | profundo; sólido}
\end{entry}

\begin{entry}{熟练}{shu2lian4}{15,8}{⽕、⽷}[HSK 4]
  \definition{adj.}{especializado; proficiente; qualificado; habilidoso}
\end{entry}

\begin{entry}{熟人}{shu2 ren2}{15,2}{⽕、⼈}[HSK 3]
  \definition[位,名,个,些]{s.}{amigo; conhecido; pessoas que se conhecem há muito tempo; pessoas que são muito familiares}
\end{entry}

\begin{entry}{熟悉}{shu2xi1}{15,11}{⽕、⼼}[HSK 5]
  \definition{adj.}{familiarizado com; não ser estranho}
  \definition{v.}{estar familiarizado com; saber claramente que | conhecer bem algo ou alguém; compreender e dominar (a situação) através da observação ou da experiência}
\end{entry}

\begin{entry}{属}{shu3}{12}{⼫}[HSK 3]
  \definition{s.}{categoria | gênero | membros da família; dependentes; familiares; parentes}
  \definition{v.}{estar sob; subordinado a | pertencer a | nascer no ano de (um dos doze animais do zodíaco)}
  \seeref{属}{zhu3}
\end{entry}

\begin{entry}{属于}{shu3yu2}{12,3}{⼫、⼆}[HSK 3]
  \definition{v.}{pertencer a; fazer parte de; pertencer ou ser propriedade de uma determinada parte}
\end{entry}

\begin{entry}{暑假}{shu3 jia4}{12,11}{⽇、⼈}[HSK 4]
  \definition[个]{s.}{férias de verão; feriado de verão; férias escolares de verão, na China, durante o sétimo e o oitavo meses do calendário gregoriano}
\end{entry}

\begin{entry}{黍}{shu3}{12}{⿉}[Kangxi 202]
  \definition{s.}{painço}
\end{entry}

\begin{entry}{数}{shu3}{13}{⽁}[HSK 2]
  \definition{v.}{contar (número); contar (número) um a um | ser considerado excepcionalmente (bom, ruim, etc.) | enumerar; listar}
  \seeref{数}{shu4}
  \seeref{数}{shuo4}
\end{entry}

\begin{entry}{鼠}{shu3}{13}{⿏}[HSK 5][Kangxi 208]
  \definition[只]{s.}{rato; camundongo}
\end{entry}

\begin{entry}{鼠标}{shu3biao1}{13,9}{⿏、⽊}[HSK 5]
  \definition[个]{s.}{\emph{mouse} (de computador); dispositivo de entrada externo para computadores, usado para controlar o movimento do cursor na tela do computador, selecionar objetos de operação, executar vários comandos, etc.}
\end{entry}

\begin{entry}{薯}{shu3}{16}{⾋}
  \definition{s.}{batata | inhame}
\end{entry}

\begin{entry}{术科}{shu4ke1}{5,9}{⽊、⽲}
  \definition{s.}{cursos técnicos oferecidos em treinamento militar ou físico (oposto a 学科)}
  \seealsoref{学科}{xue2 ke1}
\end{entry}

\begin{entry}{束}{shu4}{7}{⽊}[HSK 3]
  \definition*{s.}{sobrenome Shu}
  \definition{clas.}{usado para cachos, molhos, feixes, feixes de luz, etc.}
  \definition{s.}{monte; pacote; maço; feixe; cacho; coisas agrupadas ou reunidas em tiras}
  \definition{v.}{atar; amarrar; vincular | controlar; restringir}
\end{entry}

\begin{entry}{束腰}{shu4yao1}{7,13}{⽊、⾁}
  \definition{s.}{cinto | cinta | cinturão}
\end{entry}

\begin{entry}{树}{shu4}{9}{⽊}[HSK 1]
  \definition*{s.}{sobrenome Shu}
  \definition[棵,株]{s.}{árvore; nome comum das plantas lenhosas}
  \definition{v.}{plantar; cultivar | configurar; manter; estabelecer}
\end{entry}

\begin{entry}{树林}{shu4 lin2}{9,8}{⽊、⽊}[HSK 4]
  \definition{s.}{bosque; muitas árvores que crescem em fragmentos, menores que as florestas}
\end{entry}

\begin{entry}{树莓}{shu4mei2}{9,10}{⽊、⾋}
  \definition{s.}{framboesa}
\end{entry}

\begin{entry}{树木}{shu4mu4}{9,4}{⽊、⽊}
  \definition{s.}{árvore}
\end{entry}

\begin{entry}{树叶}{shu4ye4}{9,5}{⽊、⼝}[HSK 4]
  \definition[片,枚,堆]{s.}{folha; folhagem;}
\end{entry}

\begin{entry}{竖}{shu4}{9}{⽴}
  \definition*{s.}{sobrenome Shu}
  \definition{adj.}{vertical; ereto; perpendicular ao solo}
  \definition{s.}{traço vertical (em caracteres chineses) | empregados domésticos; jovens criados}
  \definition{v.}{colocar em pé; erguer; ficar de pé; colocar o objeto perpendicular ao solo}
\end{entry}

\begin{entry}{数}{shu4}{13}{⽁}
  \definition{num.}{vários; alguns}
  \definition{s.}{número; cifra; figura | número (conceito matemático básico que representa a quantidade de coisas) | número; indica a quantidade de coisas a que se referem os substantivos ou pronomes | destino; sorte}
  \seeref{数}{shu3}
  \seeref{数}{shuo4}
\end{entry}

\begin{entry}{数据}{shu4ju4}{13,11}{⽁、⼿}[HSK 4]
  \definition[些,个]{s.}{dados; valores com base nos quais são realizadas estatísticas, cálculos, pesquisas científicas ou projetos técnicos}
\end{entry}

\begin{entry}{数量}{shu4liang4}{13,12}{⽁、⾥}[HSK 3]
  \definition[个,种]{s.}{quantidade; quantum; quantia; magnitude; número}
\end{entry}

\begin{entry}{数码}{shu4ma3}{13,8}{⽁、⽯}[HSK 4]
  \definition{s.}{dígito; numeral; algarismo | número; quantidade (usado principalmente na linguagem falada)}
  \definition{v.}{digitalizar}
\end{entry}

\begin{entry}{数目}{shu4 mu4}{13,5}{⽁、⽬}[HSK 5]
  \definition{s.}{número; quantidade; quantidade de algo expressa em uma determinada medida padrão (como unidades de medida, etc.)}
\end{entry}

\begin{entry}{数学}{shu4xue2}{13,8}{⽁、⼦}
  \definition{s.}{matemática (disciplina)}
\end{entry}

\begin{entry}{数字}{shu4zi4}{13,6}{⽁、⼦}[HSK 2]
  \definition{adj.}{digital; usando tecnologia digital}
  \definition[个,串]{s.}{dígito; número; um caractere que representa um número | numeral; símbolos que representam números, como algarismos arábicos, algarismos romanos, etc. | quantidade; montante}
\end{entry}

\begin{entry}{刷}{shua1}{8}{⼑}[HSK 4]
  \definition{s.}{escova; pincel | (onomatopéia) farfalhar; descreve o som de uma passagem rápida}
  \definition{v.}{escovar; esfregar; remover com uma escova | borrar; colar; aplicar com um pincel | eliminar; remover; limpar}
  \seeref{刷}{shua4}
\end{entry}

\begin{entry}{刷牙}{shua1ya2}{8,4}{⼑、⽛}[HSK 4]
  \definition{s.}{escovar os dentes}
\end{entry}

\begin{entry}{刷子}{shua1zi5}{8,3}{⼑、⼦}[HSK 4]
  \definition[把]{s.}{escova; escovão; utensílio feito de lã, fio de plástico, fio de metal, etc., para remover sujeira ou aplicar óleo de unção, etc., geralmente longo ou oval, alguns com alças}
\end{entry}

\begin{entry}{耍}{shua3}{9}{⽽}
  \definition{v.}{brincar com | empunhar | agir (legal, calmo, tranquilo, descolado, etc.) | exibir (uma habilidade, o temperamento de alguém, etc.)}
\end{entry}

\begin{entry}{耍赖}{shua3lai4}{9,13}{⽽、⾙}
  \definition{v.}{agir descaradamente | recusar -se a reconhecer que alguém perdeu o jogo ou fez uma promessa, etc. | agir como um idiota | agir como se algo nunca tivesse acontecido}
\end{entry}

\begin{entry}{刷}{shua4}{8}{⼑}
  \definition{v.}{selecionar}
  \seeref{刷}{shua1}
\end{entry}

\begin{entry}{摔}{shuai1}{14}{⼿}[HSK 5]
  \definition{v.}{cair; tropeçar; perder o equilíbrio | mergulhar; precipitar-se; cair de uma altura elevada | quebrar; fazer cair e quebrar | lançar; atirar; arremessar; joguar coisas com força e para baixo | bater; golpear; bater com força para que o que está grudado cair}
\end{entry}

\begin{entry}{摔倒}{shuai1dao3}{14,10}{⼿、⼈}[HSK 5]
  \definition{v.}{cair; tropeçar; perder o equilíbrio e cair}
\end{entry}

\begin{entry}{帅}{shuai4}{5}{⼱}[HSK 4]
  \definition*{s.}{sobrenome Shuai}
  \definition{adj.}{bonito; arrojado; elegante; inteligente}
  \definition{interj.}{Legal!}
  \definition[位,名]{s.}{comandante em chefe; o mais alto comandante do exército | comandante em chefe, a peça principal no xadrez chinês}
\end{entry}

\begin{entry}{帅哥}{shuai4 ge1}{5,10}{⼱、⼝}[HSK 4]
  \definition[个,位]{s.}{rapaz bonito; um garoto que é bonito e atraente na aparência}
\end{entry}

\begin{entry}{率领}{shuai4ling3}{11,11}{⽞、⾴}[HSK 5]
  \definition{v.}{liderar (equipe ou grupo); chefiar; comandar}
\end{entry}

\begin{entry}{率先}{shuai4 xian1}{11,6}{⽞、⼉}[HSK 4]
  \definition{v.}{tomar a iniciativa de fazer algo; ser o primeiro a fazer algo; assumir a liderança}
\end{entry}

\begin{entry}{双}{shuang1}{4}{⼜}[HSK 3]
  \definition*{s.}{sobrenome Shuang}
  \definition{adj.}{dois; gêmeo; par; dual; em oposição a 单 | números pares | duplo; dobro}
  \definition{clas.}{usado para certos membros, órgãos ou coisas pareadas que são bilateralmente simétricas, por exemplo, sapatos, meias, pauzinhos, etc.}
  \seealsoref{单}{dan1}
\end{entry}

\begin{entry}{双层床}{shuang1ceng2chuang2}{4,7,7}{⼜、⼫、⼴}
  \definition{s.}{beliche}
\end{entry}

\begin{entry}{双打}{shuang1da3}{4,5}{⼜、⼿}
  \definition[场]{s.}{duplas (em esportes)}
\end{entry}

\begin{entry}{双方}{shuang1fang1}{4,4}{⼜、⽅}[HSK 3]
  \definition{s.}{ambos os lados; as duas partes; duas pessoas ou dois grupos frente a frente em um determinado relacionamento ou situação}
\end{entry}

\begin{entry}{双方同意}{shuang1fang1tong2yi4}{4,4,6,13}{⼜、⽅、⼝、⼼}
  \definition{s.}{acordo bilateral}
\end{entry}

\begin{entry}{双手}{shuang1 shou3}{4,4}{⼜、⼿}[HSK 5]
  \definition{s.}{com as duas mãos; ambas as mãos; par de mãos}
\end{entry}

\begin{entry}{霜}{shuang1}{17}{⾬}
  \definition{s.}{geada | pó branco ou creme espalhado por uma superfície | glacê | creme de pele}
\end{entry}

\begin{entry}{爽}{shuang3}{11}{⽘}
  \definition{adj.}{claro; nítido; brilhante |franco; de coração aberto; direto | relaxado; confortável}
  \definition{v.}{desviar; afastar | tornar confortável; ficar confortável}
\end{entry}

\begin{entry}{谁}{shui2}{10}{⾔}[HSK 1]
  \seeref{谁}{shei2}
\end{entry}

\begin{entry}{水}{shui3}{4}{⽔}[HSK 1][Kangxi 85]
  \definition*{s.}{sobrenome Shui}
  \definition*{s.}{a nacionalidade Shui, que vive principalmente em Guizhou}
  \definition{adj.}{de má qualidade; mal feito; de baixa qualidade e conteúdo}
  \definition{clas.}{usado para número de lavagens}
  \definition[条,杯]{s.}{água | rio | termo geral para rios, lagos, mares, etc.; água | corrente; fluxo de água | um líquido; suco ralo | teor de prata nas moedas | encargos adicionais ou receitas | água, um dos cinco elementos}
\end{entry}

\begin{entry}{水边}{shui3bian1}{4,5}{⽔、⾡}
  \definition{s.}{beira d'água | beira-mar | costa (de mar, lago ou rio)}
\end{entry}

\begin{entry}{水波}{shui3bo1}{4,8}{⽔、⽔}
  \definition{s.}{ondulação (na água) | onda}
\end{entry}

\begin{entry}{水槽}{shui3cao2}{4,15}{⽔、⽊}
  \definition{s.}{pia (de cozinha)}
\end{entry}

\begin{entry}{水产品}{shui3 chan3 pin3}{4,6,9}{⽔、⼇、⼝}[HSK 5]
  \definition{s.}{produto aquático (peixes, camarões, etc.)}
\end{entry}

\begin{entry}{水分}{shui3 fen4}{4,4}{⽔、⼑}[HSK 5]
  \definition{s.}{teor de umidade; água contida em um objeto | exagero; metáfora de algo falso}
\end{entry}

\begin{entry}{水果}{shui3guo3}{4,8}{⽔、⽊}[HSK 1]
  \definition[个]{s.}{fruta; um nome genérico para frutas com alto teor de água que podem ser consumidas, como peras, pêssegos, maçãs, etc.}
\end{entry}

\begin{entry}{水饺}{shui3jiao3}{4,9}{⽔、⾷}
  \definition{s.}{\emph{dumplings} | pastéis chineses cozidos}
\end{entry}

\begin{entry}{水库}{shui3 ku4}{4,7}{⽔、⼴}[HSK 5]
  \definition[座]{s.}{reservatório; lago artificial construído pelo homem, que utiliza barragens e outras estruturas para represar a água e regular o fluxo, podendo ser utilizado para armazenamento de água, geração de energia e piscicultura, entre outros fins}
\end{entry}

\begin{entry}{水灵}{shui3ling2}{4,7}{⽔、⽕}
  \definition{adj.}{cheio de vida (sobre uma pessoa, etc.) | úmido e brilhante (sobre os olhos) | fresco (sobre frutas, etc.) | brilhante | aparência saudável}
\end{entry}

\begin{entry}{水路}{shui3lu4}{4,13}{⽔、⾜}
  \definition{s.}{hidrovia}
\end{entry}

\begin{entry}{水培}{shui3pei2}{4,11}{⽔、⼟}
  \definition{v.}{cultivar plantas hidroponicamente}
\end{entry}

\begin{entry}{水平}{shui3ping2}{4,5}{⽔、⼲}[HSK 2]
  \definition{adj.}{horizontal; nivelado; paralelo à superfície da água}
  \definition{s.}{padrão; nível; o nível alcançado em determinado aspecto}
\end{entry}

\begin{entry}{水平尺}{shui3ping2chi3}{4,5,4}{⽔、⼲、⼫}
  \definition{s.}{nível espiritual}
\end{entry}

\begin{entry}{水平度}{shui3ping2 du4}{4,5,9}{⽔、⼲、⼴}
  \definition{s.}{nivelamento}
\end{entry}

\begin{entry}{水平面}{shui3ping2mian4}{4,5,9}{⽔、⼲、⾯}
  \definition{s.}{plano horizontal | nível-da-água | superfície horizontal}
\end{entry}

\begin{entry}{水平视差}{shui3ping2 shi4cha1}{4,5,8,9}{⽔、⼲、⾒、⼯}
  \definition{s.}{paralaxe horizontal}
\end{entry}

\begin{entry}{水平仪}{shui3ping2yi2}{4,5,5}{⽔、⼲、⼈}
  \definition{s.}{nível (dispositivo para determinar horizontal) | nível espiritual | nível de topógrafo}
\end{entry}

\begin{entry}{水平以下}{shui3ping2 yi3xia4}{4,5,4,3}{⽔、⼲、⼈、⼀}
  \definition{s.}{sub-nível}
\end{entry}

\begin{entry}{水平轴}{shui3ping2zhou2}{4,5,9}{⽔、⼲、⾞}
  \definition{s.}{eixo horizontal}
\end{entry}

\begin{entry}{水瓶}{shui3 ping2}{4,10}{⽔、⽡}
  \definition{s.}{garrada de água}
\end{entry}

\begin{entry}{水豚}{shui3tun2}{4,11}{⽔、⾗}
  \definition{s.}{capivara}
\end{entry}

\begin{entry}{水污染}{shui3wu1ran3}{4,6,9}{⽔、⽔、⽊}
  \definition{s.}{poluição da água}
\end{entry}

\begin{entry}{水灾}{shui3 zai1}{4,7}{⽔、⽕}[HSK 5]
  \definition{s.}{inundação; desastres causados por excesso de chuvas, entre outros motivos}
\end{entry}

\begin{entry}{说}{shui4}{9}{⾔}
  \definition{v.}{persuadir}
  \seeref{说}{shuo1}
\end{entry}

\begin{entry}{税}{shui4}{12}{⽲}
  \definition{s.}{taxas | impostos}
\end{entry}

\begin{entry}{睡}{shui4}{13}{⽬}[HSK 1]
  \definition{v.}{dormir | deitar-se}
\end{entry}

\begin{entry}{睡觉}{shui4jiao4}{13,9}{⽬、⾒}[HSK 1]
  \definition{v.+compl.}{dormir; ir para a cama; entrar em estado de sono}
\end{entry}

\begin{entry}{睡懒觉}{shui4lan3jiao4}{13,16,9}{⽬、⼼、⾒}
  \definition{v.}{levantar-se tarde | passar o tempo a dormir}
\end{entry}

\begin{entry}{睡眠}{shui4 mian2}{13,10}{⽬、⽬}[HSK 5]
  \definition{s.}{sono; \emph{somnus}; sonolência}
\end{entry}

\begin{entry}{睡衣}{shui4yi1}{13,6}{⽬、⾐}
  \definition{s.}{pijamas | roupas de dormir}
\end{entry}

\begin{entry}{睡着}{shui4 zhao2}{13,11}{⽬、⽬}[HSK 4]
  \definition{v.}{dormir; adormecer; cair no sono}
\end{entry}

\begin{entry}{顺}{shun4}{9}{⾴}
  \definition{adj.}{correr bem | favorável}
\end{entry}

\begin{entry}{顺便}{shun4bian4}{9,9}{⾴、⼈}
  \definition{adv.}{convenientemente | de passagem | sem muito esforço extra}
\end{entry}

\begin{entry}{顺畅}{shun4chang4}{9,8}{⾴、⽥}
  \definition{adj.}{liso e sem obstáculos | fluente}
\end{entry}

\begin{entry}{顺从}{shun4cong2}{9,4}{⾴、⼈}
  \definition{v.}{obedecer | submeter-se}
\end{entry}

\begin{entry}{顺当}{shun4dang5}{9,6}{⾴、⼹}
  \definition{adv.}{suavemente}
\end{entry}

\begin{entry}{顺耳}{shun4'er3}{9,6}{⾴、⽿}
  \definition{adj.}{agradável ao ouvido}
\end{entry}

\begin{entry}{顺境}{shun4jing4}{9,14}{⾴、⼟}
  \definition{s.}{circunstâncias favoráveis}
\end{entry}

\begin{entry}{顺利}{shun4li4}{9,7}{⾴、⼑}[HSK 2]
  \definition{adj.}{sem problemas; com sucesso; sem dificuldades; sem contratempos; sem obstáculos; sem obstáculos ou dificuldades significativas no desempenho das tarefas}
\end{entry}

\begin{entry}{顺水}{shun4shui3}{9,4}{⾴、⽔}
  \definition{v.}{ir com o fluxo}
\end{entry}

\begin{entry}{顺心}{shun4xin1}{9,4}{⾴、⼼}
  \definition{adj.}{satisfatório | satisfeito}
\end{entry}

\begin{entry}{顺序}{shun4xu4}{9,7}{⾴、⼴}[HSK 4]
  \definition{adv.}{por sua vez; na ordem correta; na devida ordem; na ordem adequada; na ordem apropriada}
  \definition[个]{s.}{ordem; sequência; sucessão; subsequência; sequência simples; ordem de prioridade}
\end{entry}

\begin{entry}{顺叙}{shun4xu4}{9,9}{⾴、⼜}
  \definition{s.}{narrativa cronológica}
\end{entry}

\begin{entry}{顺延}{shun4yan2}{9,6}{⾴、⼵}
  \definition{v.}{adiar | procrastinar}
\end{entry}

\begin{entry}{顺眼}{shun4yan3}{9,11}{⾴、⽬}
  \definition{adj.}{agradável aos olhos}
\end{entry}

\begin{entry}{顺嘴}{shun4zui3}{9,16}{⾴、⼝}
  \definition{v.}{deixar escapar (sem pensar) | ler suavemente (texto) | adequar-se  ao gosto (comida)}
\end{entry}

\begin{entry}{说}{shuo1}{9}{⾔}[HSK 1]
  \definition{s.}{uma teoria (normalmente o último caractere, como em 日心说, teoria heliocêntrica); ensinamentos; doutrina}
  \definition{v.}{falar; conversar; dizer | explicar | repreender | atuar como casamenteiro | referir-se a; indicar | criticar; aconselhar | fazer uma combinação; conciliar; mediar | discutir; falar sobre; conversar sobre | uma forma de expressão linguística da arte cênica}
  \seeref{说}{shui4}
  \seealsoref{日心说}{ri4 xin1 shuo1}
\end{entry}

\begin{entry}{说不定}{shuo1bu5ding4}{9,4,8}{⾔、⼀、⼧}[HSK 4]
  \definition{adv.}{talvez; indica uma estimativa, possivelmente, provavelmente}
  \definition{v.}{não ter certeza; não estar certo; ser impreciso}
\end{entry}

\begin{entry}{说法}{shuo1 fa3}{9,8}{⾔、⽔}[HSK 5]
  \definition[种]{s.}{maneira de dizer uma coisa; palavras ou frases usadas para expressar significado | declaração; versão; argumento; opinião; ponto de vista | motivo; razão; motivos ou bases para a resolução do problema}
\end{entry}

\begin{entry}{说服}{shuo1fu2}{9,8}{⾔、⽉}[HSK 4]
  \definition{v.}{persuadir; convencer; convencer a outra parte com palavras bem fundamentadas}
\end{entry}

\begin{entry}{说好}{shuo1hao3}{9,6}{⾔、⼥}
  \definition{v.}{chegar a um acordo | concluir negociações}
\end{entry}

\begin{entry}{说话}{shuo1hua4}{9,8}{⾔、⾔}[HSK 1]
  \definition{adv.}{imediatamente; em um minuto; refere-se ao tempo que leva para falar, indicando um período muito curto}
  \definition{v.}{falar; conversar; dizer; expressar o significado através da linguagem | conversar (conversa fiada); bater papo | fofocar; conversar; criticar; censurar}
\end{entry}

\begin{entry}{说谎}{shuo1huang3}{9,11}{⾔、⾔}
  \definition{v.+compl.}{mentir | contar uma mentira}
\end{entry}

\begin{entry}{说理}{shuo1li3}{9,11}{⾔、⽟}
  \definition{v.}{racionalizar | discutir logicamente}
\end{entry}

\begin{entry}{说明}{shuo1ming2}{9,8}{⾔、⽇}[HSK 2]
  \definition[本,个]{s.}{legenda; instrução; explicação}
  \definition{v.}{mostrar; explicar; ilustrar | indicar; mostrar; provar; demonstrar; usar materiais confiáveis para demonstrar ou determinar a autenticidade de pessoas ou coisas}
\end{entry}

\begin{entry}{说完}{shuo1-wan2}{9,7}{⾔、⼧}
  \definition{expr.}{acabar/terminar palavras}
\end{entry}

\begin{entry}{硕士}{shuo4shi4}{11,3}{⽯、⼠}[HSK 5]
  \definition[个,位,名]{s.}{mestrado}
\end{entry}

\begin{entry}{数}{shuo4}{13}{⽁}
  \definition{adv.}{com frequência; repetidamente; indica uma ação frequente, equivalente a 屡次}
  \seeref{数}{shu3}
  \seeref{数}{shu4}
  \seealsoref{屡次}{lv3ci4}
\end{entry}

\begin{entry}{丝}{si1}{5}{⼀}
  \definition{adj.}{filiforme | delgado como um fio | que se assemelha a um fio}
  \definition{clas.}{um traço (de fumaça, etc.) | um pouquinho, etc.}
  \definition{s.}{seda | (cozinha) pedaços ou tiras de julienne, tiras cortadas finas}
\end{entry}

\begin{entry}{司机}{si1ji1}{5,6}{⼝、⽊}[HSK 2]
  \definition[个,名,位]{s.}{motorista; motorista particular; chofer; motoristas de veículos de transporte público, como trens, ônibus e bondes}
\end{entry}

\begin{entry}{私}{si1}{7}{⽲}
  \definition{adj.}{pessoal; privado (oposição a 公) | egoísta (oposto a 公) | secreto; privado | ilícito; ilegal}
  \definition{s.}{interesse privado (ou egoísta); motivo (ou ideia) egoísta (oposição a 公) | contrabando; mercadorias contrabandeadas | propriedade privada | interesses privados; ganho pessoal}
  \definition{s.}{sobrenome Si}
  \seealsoref{公}{gong1}
\end{entry}

\begin{entry}{私人}{si1ren2}{7,2}{⽲、⼈}[HSK 5]
  \definition{adj.}{privado; pertencente a um indivíduo ou exercido a título individual; não público | interpessoal}
  \definition[个]{s.}{algo privado; pessoas que se aproximam de você por motivos pessoais ou interesses próprios}
\end{entry}

\begin{entry}{私人信件}{si1ren2 xin4jian4}{7,2,9,6}{⽲、⼈、⼈、⼈}
  \definition{s.}{carta pessoal}
\end{entry}

\begin{entry}{私人钥匙}{si1ren2yao4shi5}{7,2,9,11}{⽲、⼈、⾦、⼔}
  \definition{s.}{(criptografia) chave privada}
\end{entry}

\begin{entry}{私人诊所}{si1ren2 zhen3suo3}{7,2,7,8}{⽲、⼈、⾔、⼾}
  \definition[些]{s.}{clínica privada}
\end{entry}

\begin{entry}{私生活}{si1sheng1huo2}{7,5,9}{⽲、⽣、⽔}
  \definition{s.}{vida privada}
\end{entry}

\begin{entry}{私自}{si1zi4}{7,6}{⽲、⾃}
  \definition{adj.}{privado | pessoal}
  \definition{adv.}{secretamente | sem aprovação explícita}
\end{entry}

\begin{entry}{思考}{si1kao3}{9,6}{⼼、⽼}[HSK 4]
  \definition{v.}{pensar; ponderar; considerar; deliberar; envolver-se em atividades de pensamento, como análise, síntese, julgamento, raciocínio e generalização}
\end{entry}

\begin{entry}{思维}{si1wei2}{9,11}{⼼、⽷}[HSK 5]
  \definition[种]{s.}{pensamento; reflexão; organizar e transformar os materiais obtidos através do conhecimento sensorial para formar conceitos, julgamentos e raciocínios}
  \definition{v.}{pensar;}
\end{entry}

\begin{entry}{思想}{si1xiang3}{9,13}{⼼、⼼}[HSK 3]
  \definition[个,种]{s.}{reflexão; pensamento; ideologia; a existência objetiva é refletida na consciência das pessoas por meio de atividades de pensamento, que pertencem à cognição racional | ideia; pensamento}
\end{entry}

\begin{entry}{斯巴达}{si1ba1da2}{12,4,6}{⽄、⼰、⾡}
  \definition*{s.}{Esparta}
\end{entry}

\begin{entry}{死}{si3}{6}{⽍}[HSK 3]
  \definition{adj.}{até a morte | implacável; mortal | fixo; rígido; inflexível | intransitável; fechado | (expressando raiva, reclamação, etc., às vezes jocosamente) maldito}
  \definition{adv.}{(frequentemente no negativo) teimosamente; inflexivelmente}
  \definition{v.}{morrer; estar morto (oposto a 生 e 活)}
  \seealsoref{活}{huo2}
  \seealsoref{生}{sheng1}
\end{entry}

\begin{entry}{死亡}{si3wang2}{6,3}{⽍、⼇}
  \definition{s.}{morte}
  \definition{v.}{morrer}
\end{entry}

\begin{entry}{四}{si4}{5}{⼞}[HSK 1]
  \definition*{s.}{sobrenome Si}
  \definition{num.}{quatro; 4}
  \definition{s.}{uma nota da escala em Gongchepu (工尺谱), correspondente ao 6 na notação musical numerada}
  \seealsoref{工尺谱}{gong1 che3 pu3}
\end{entry}

\begin{entry}{四川}{si4chuan1}{5,3}{⼞、⼮}
  \definition*{s.}{Sichuan}
\end{entry}

\begin{entry}{四季分明}{si4ji4-fen1ming2}{5,8,4,8}{⼞、⼦、⼑、⽇}
  \definition{expr.}{as quatro estações são muito distintas}
\end{entry}

\begin{entry}{四季如春}{si4ji4-ru2chun1}{5,8,6,9}{⼞、⼦、⼥、⽇}
  \definition{expr.}{é primavera todo o ano | clima favorável durante todo o ano | quatro estações como a primavera}
\end{entry}

\begin{entry}{四周}{si4 zhou1}{5,8}{⼞、⼝}[HSK 5]
  \definition{s.}{ao redor; por todos os lados; a parte que circunda o centro}
\end{entry}

\begin{entry}{似曾相识}{si4ceng2xiang1shi2}{6,12,9,7}{⼈、⽈、⽬、⾔}
  \definition{s.}{\emph{déjà vu} (a experiência de ver exatamente a mesma situação pela segunda vez) | situação aparentemente familiar}
\end{entry}

\begin{entry}{似乎}{si4hu1}{6,5}{⼈、⼃}[HSK 4]
  \definition{adv.}{como se; aparentemente; se parece como}
\end{entry}

\begin{entry}{寺}{si4}{6}{⼨}
  \definition{s.}{Templo Budista | Mesquita}
\end{entry}

\begin{entry}{寺庙}{si4miao4}{6,8}{⼨、⼴}
  \definition{s.}{templo | mosteiro | santuário}
\end{entry}

\begin{entry}{肆}{si4}{13}{⾀}
  \definition*{s.}{sobrenome Si}
  \definition{adj.}{arbitrário; desenfreado; sem limites; descuidado; imprudente}
  \definition{num.}{quatro (usado para o numeral 四 em cheques, etc., para evitar erros ou alterações)}
  \definition{s.}{loja}
  \seealsoref{四}{si4}
\end{entry}

\begin{entry}{厕}{si5}{8}{⼚}
  \definition{s.}{componente formador de palavras | latrina; fossa sanitária}
  \seealsoref{茅厕}{mao2ce4}
\end{entry}

\begin{entry}{松}{song1}{8}{⽊}[HSK 4]
  \definition*{s.}{sobrenome Song}
  \definition{adj.}{solto; frouxo; folgado | abastado; rico; próspero | leve e crocante; macio}
  \definition[棵]{s.}{pinheiro | fio de carne seca; carne moída seca; alimentos macios ou quebradiços |}
  \definition{v.}{afrouxar; relaxar; soltar}
\end{entry}

\begin{entry}{松木}{song1mu4}{8,4}{⽊、⽊}
  \definition{s.}{pinheiro}
\end{entry}

\begin{entry}{松树}{song1 shu4}{8,9}{⽊、⽊}[HSK 4]
  \definition[棵]{s.}{pinheiro; conífera comum, geralmente com folhas longas e pontiagudas e cones lenhosos}
\end{entry}

\begin{entry}{宋}{song4}{7}{⼧}
  \definition*{s.}{sobrenome Song}
  \definition{s.}{Dinastia Song (960-1279) | Song das dinastias do sul (420-479)}
\end{entry}

\begin{entry}{送}{song4}{9}{⾡}[HSK 1]
  \definition*{s.}{sobrenome Song}
  \definition{v.}{transportar; entregar | dar; dar como presente; presentear | acompanhar; despedir-se de alguém (ao sair); acompanhar a pessoa que está partindo até o destino ou caminhar um trecho com ela | escoltar}
\end{entry}

\begin{entry}{送到}{song4 dao4}{9,8}{⾡、⼑}[HSK 2]
  \definition{v.}{enviar para (lugar)}
\end{entry}

\begin{entry}{送给}{song4 gei3}{9,9}{⾡、⽷}[HSK 2]
  \definition{v.}{dar a (alguém ou organização); dar como algo gratuito; dar como presente}
\end{entry}

\begin{entry}{㮸}{song4}{14}{⽊}
  \variantof{送}
\end{entry}

\begin{entry}{搜}{sou1}{12}{⼿}[HSK 5]
  \definition{v.}{procurar | pesquisar | coletar; reunir | revistar}
\end{entry}

\begin{entry}{搜索}{sou1suo3}{12,10}{⼿、⽷}[HSK 5]
  \definition{v.}{procurar; caçar; explorar; pesquisar cuidadosamente; refere-se especificamente à busca militar para identificar situações suspeitas em determinada região, área marítima ou aérea}
\end{entry}

\begin{entry}{苏格兰}{su1ge2lan2}{7,10,5}{⾋、⽊、⼋}
  \definition*{s.}{Escócia}
\end{entry}

\begin{entry}{速度}{su4du4}{10,9}{⾡、⼴}[HSK 3]
  \definition[个,种]{s.}{velocidade; taxa; ritmo; andamento; uma quantidade física que indica a velocidade e a direção do movimento de um objeto, ou seja, a distância que um objeto percorre em uma direção por unidade de tempo | velocidade; rapidez; geralmente se refere ao grau de velocidade}
\end{entry}

\begin{entry}{宿舍}{su4she4}{11,8}{⼧、⾆}[HSK 5]
  \definition[间,幢]{s.}{alojamento; dormitório; república; albergue; casas onde escolas, empresas, etc. acomodam seus alunos ou funcionários}
\end{entry}

\begin{entry}{塑料}{su4 liao4}{13,10}{⼟、⽃}[HSK 4]
  \definition[块,种]{s.}{plástico; compostos de polímeros feitos de resinas naturais ou sintéticas como componente principal}
\end{entry}

\begin{entry}{塑料袋}{su4liao4dai4}{13,10,11}{⼟、⽃、⾐}[HSK 4]
  \definition{s.}{saco plástico; sacola de plástico}
\end{entry}

\begin{entry}{痠}{suan1}{12}{⽧}
  \definition{v.}{doer | estar dolorido}
  \variantof{酸}
\end{entry}

\begin{entry}{酸}{suan1}{14}{⾣}[HSK 4]
  \definition{adj.}{azedo; ácido | aflito; angustiado; doente do coração | pedante; descreve uma pessoa que finge ser culta e também descreve uma pessoa que é muito inflexível com suas próprias ideias e não está disposta a mudá-las para atender às exigências da época, é usado principalmente para satirizar intelectuais que fingem ser capazes de escrever poemas e artigos | ciumento; invejoso; sentimentos desconfortáveis porque outra pessoa é melhor do que você e, em geral, também apresenta comportamento hostil}
  \definition{s.}{ácido; produto químico que tem um sabor ácido quando misturado com água}
  \definition{v.}{estar dolorido (devido à fadiga ou doença); descreve a sensação de não ter força muscular e um pouco de dor por estar doente ou muito cansado}
\end{entry}

\begin{entry}{酸辣汤}{suan1la4tang1}{14,14,6}{⾣、⾟、⽔}
  \definition{s.}{sopa avinagrada e picante (prato)}
\end{entry}

\begin{entry}{酸奶}{suan1 nai3}{14,5}{⾣、⼥}[HSK 4]
  \definition[瓶,杯,盒,袋]{s.}{iogurte; produto lácteo fermentado por bactérias de ácido láctico}
\end{entry}

\begin{entry}{酸甜苦辣}{suan1 tian2 ku3 la4}{14,11,8,14}{⾣、⽢、⾋、⾟}[HSK 5]
  \definition{expr.}{os altos e baixos da vida; as experiências agridoces da vida; os aspectos doces, azedos, amargos e picantes da vida; refere-se a todos os tipos de sabores, como metáfora para experiências diversas, como felicidade, sofrimento, etc. | azedo, doce, amargo, picante — alegrias e tristezas da vida}
\end{entry}

\begin{entry}{算}{suan4}{14}{⽵}[HSK 2]
  \definition{adv.}{finalmente; por fim; no final; significa que, após um longo período de tempo ou muitas dificuldades, finalmente se alcançou o objetivo, equivalente a 总算}
  \definition{v.}{calcular; estimar; computar | contar; incluir | planejar; calcular; projetar | pensar; supor; especular | considerar; considerar como; contar como; reconhecer como | (aritmética) contar; ter peso | deixe estar; deixe passar; seguido por 了: desistir, não se importar mais}
  \seealsoref{了}{le5}
  \seealsoref{总算}{zong3suan4}
\end{entry}

\begin{entry}{算了}{suan4le5}{14,2}{⽵、⼅}
  \definition{v.}{deixar | deixe estar | deixe passar | esqueça isso}
\end{entry}

\begin{entry}{算命}{suan4ming4}{14,8}{⽵、⼝}
  \definition{s.}{cartomante}
  \definition{v.}{ler a sorte | fazer advinhações}
\end{entry}

\begin{entry}{尿}{sui1}{7}{⼫}
  \definition{s.}{(coloquial) urina}
  \seeref{尿}{niao4}
\end{entry}

\begin{entry}{虽}{sui1}{9}{⾍}
  \definition{conj.}{no entanto | embora | mesmo se/embora}
\end{entry}

\begin{entry}{虽然}{sui1 ran2}{9,12}{⾍、⽕}[HSK 2]
  \definition{conj.}{apesar de; embora (frequentemente usado correlativamente com 可是, 但是, etc); geralmente é usado no início de uma frase para indicar que o fato anterior foi reconhecido, mas não mudará o que acontecerá em seguida}
  \seealsoref{但是}{dan4 shi4}
  \seealsoref{可是}{ke3shi4}
\end{entry}

\begin{entry}{随}{sui2}{11}{⾩}[HSK 3]
  \definition*{s.}{sobrenome Sui}
  \definition{adv.}{fazer algo imediatamente assim que ocorre, sem demora ou hesitação; usado antes de dois verbos ou frases verbais para indicar que a última ação segue a anterior}
  \definition{prep.}{junto com (alguma outra ação) | apresentando as condições das quais a ação depende}
  \definition{v.}{seguir; vir (ou ir) junto com | concordar com; adaptar-se a | deixar (alguém fazer o que quiser) | (dialeto) parecer-se com; assemelhar-se a | seguir ou agir de acordo com a condição ou circunstância da qual a ação depende}
\end{entry}

\begin{entry}{随便}{sui2bian4}{11,9}{⾩、⼈}[HSK 2]
  \definition{adj.}{relaxado; descontraído; sem restrições; sem limitações | aleatório; casual; descuidado; indiferente; distraído, não pensa bem antes de falar ou agir | casual; informal; não dá importância aos detalhes}
  \definition{conj.}{qualquer; qualquer que seja; não importa}
  \definition{v.}{deixar alguém à vontade}
\end{entry}

\begin{entry}{随处}{sui2chu4}{11,5}{⾩、⼡}
  \definition{adv.}{em qualquer lugar}
\end{entry}

\begin{entry}{随地}{sui2di4}{11,6}{⾩、⼟}
  \definition{adv.}{qualquer lugar | todo lugar}
\end{entry}

\begin{entry}{随后}{sui2 hou4}{11,6}{⾩、⼝}[HSK 5]
  \definition{adv.}{logo em seguida; logo depois; indica que segue imediatamente após a ação ou situação anterior (geralmente usado em conjunto com 就)}
  \seealsoref{就}{jiu4}
\end{entry}

\begin{entry}{随机存取存储器}{sui2ji1cun2qu3cun2chu3qi4}{11,6,6,8,6,12,16}{⾩、⽊、⼦、⼜、⼦、⼈、⼝}
  \definition{s.}{RAM (\emph{random access memory})}
  \seealsoref{内存}{nei4cun2}
  \seealsoref{随机存取记忆体}{sui2ji1cun2qu3ji4yi4ti3}
\end{entry}

\begin{entry}{随机存取记忆体}{sui2ji1cun2qu3ji4yi4ti3}{11,6,6,8,5,4,7}{⾩、⽊、⼦、⼜、⾔、⼼、⼈}
  \definition{s.}{RAM (\emph{random access memory})}
  \seealsoref{内存}{nei4cun2}
  \seealsoref{随机存取存储器}{sui2ji1cun2qu3cun2chu3qi4}
\end{entry}

\begin{entry}{随时}{sui2shi2}{11,7}{⾩、⽇}[HSK 2]
  \definition{adv.}{a qualquer momento; em todos os momentos}
\end{entry}

\begin{entry}{随手}{sui2shou3}{11,4}{⾩、⼿}[HSK 4]
  \definition{adv.}{convenientemente; sem problemas adicionais; casualmente}
\end{entry}

\begin{entry}{随意}{sui2yi4}{11,13}{⾩、⼼}[HSK 5]
  \definition{adj.}{aleatório; casual; à vontade; como se deseja}
\end{entry}

\begin{entry}{随着}{sui2zhe5}{11,11}{⾩、⽬}[HSK 5]
  \definition{prep.}{junto com; na esteira de; em sintonia com; usado no início da frase ou antes do verbo, indica as condições necessárias para que uma ação, comportamento ou evento ocorra}
\end{entry}

\begin{entry}{岁}{sui4}{6}{⼭}[HSK 1]
  \definition{clas.}{usado para anos (de idade)}
  \definition{s.}{ano (literário) | colheita do ano (literário) | idade | tempo (literário) | ano (de idade) | ano (para as colheitas)}
\end{entry}

\begin{entry}{岁月}{sui4yue4}{6,4}{⼭、⽉}[HSK 5]
  \definition{s.}{anos; ano e mês; refere-se a tempo em geral}
\end{entry}

\begin{entry}{碎}{sui4}{13}{⽯}[HSK 5]
  \definition*{s.}{sobrenome Sui}
  \definition{adj.}{quebrado; fragmentado | tagarela; falante}
  \definition{v.}{(transitivo ou intransitivo) quebrar em pedaços; esmagar}
\end{entry}

\begin{entry}{隧道}{sui4dao4}{14,12}{⾩、⾡}
  \definition{s.}{túnel}
\end{entry}

\begin{entry}{孙女}{sun1nv3}{6,3}{⼦、⼥}[HSK 4]
  \definition{s.}{filha do filho; neta}
\end{entry}

\begin{entry}{孙武}{sun1wu3}{6,8}{⼦、⽌}
  \definition*{s.}{Sun Wu, também conhecido por Sun Tzu (孙子), general, estrategista e filósofo autor do ``Arte da Guerra'' (孙子兵法)}
  \seealsoref{孙子}{sun1zi3}
  \seealsoref{孙子兵法}{sun1zi3 bing1fa3}
\end{entry}

\begin{entry}{孙子}{sun1zi3}{6,3}{⼦、⼦}
  \definition*{s.}{Sun Tzu, também conhecido por Sun Wu (孙武), general, estrategista e filósofo autor do ``Arte da Guerra'' (孙子兵法)}
  \seeref{孙子}{sun1zi5}
  \seealsoref{孙武}{sun1wu3}
  \seealsoref{孙子兵法}{sun1zi3 bing1fa3}
\end{entry}

\begin{entry}{孙子兵法}{sun1zi3 bing1fa3}{6,3,7,8}{⼦、⼦、⼋、⽔}
  \definition*{s.}{``Arte da Guerra'', escrito por Sun Tzu (孫子)}
  \seealsoref{孙武}{sun1wu3}
  \seealsoref{孙子}{sun1zi3}
\end{entry}

\begin{entry}{孙子}{sun1zi5}{6,3}{⼦、⼦}[HSK 4]
  \definition{s.}{filho do filho; neto}
  \seeref{孙子}{sun1zi3}
\end{entry}

\begin{entry}{损害}{sun3 hai4}{10,10}{⼿、⼧}[HSK 5]
  \definition{v.}{prejudicar; danificar; causar danos}
\end{entry}

\begin{entry}{损失}{sun3shi1}{10,5}{⼿、⼤}[HSK 5]
  \definition{s.}{perda; desperdício; algo que se consome ou se perde sem custo algum}
  \definition{v.}{perder; consumir ou perder}
\end{entry}

\begin{entry}{笋}{sun3}{10}{⽵}
  \definition{s.}{broto de bambu}
\end{entry}

\begin{entry}{缩短}{suo1duan3}{14,12}{⽷、⽮}[HSK 4]
  \definition{v.}{encurtar; reduzir; diminuir}
\end{entry}

\begin{entry}{缩小}{suo1 xiao3}{14,3}{⽷、⼩}[HSK 4]
  \definition{v.}{reduzir, estreitar, encolher;  tornar menor (em oposição a 放大)}
  \seealsoref{放大}{fang4da4}
\end{entry}

\begin{entry}{缩影卡片}{suo1ying3 ka3pian4}{14,15,5,4}{⽷、⼺、⼘、⽚}
  \definition{s.}{cartão em miniatura}
\end{entry}

\begin{entry}{所}{suo3}{8}{⼾}[HSK 3]
  \definition*{s.}{sobrenome Suo}
  \definition{clas.}{usado para casas, etc.}
  \definition{part.}{usado com 为 ou 被 para indicar voz passiva | usado antes do verbo para formar um substantivo ou para qualificar um substantivo | usado antes do verbo na estrutura sujeito-predicado usada como complemento, indica que o termo central é o objeto}
  \definition{s.}{lugar | usado como nome de órgãos governamentais ou outros locais de trabalho}
  \seealsoref{被}{bei4}
  \seealsoref{为}{wei4}
\end{entry}

\begin{entry}{所长}{suo3 chang2}{8,4}{⼾、⾧}
  \definition{s.}{aquilo em que alguém é bom; o ponto forte de alguém; o forte de alguém}
  \seeref{所长}{suo3 zhang3}
\end{entry}

\begin{entry}{所以}{suo3 yi3}{8,4}{⼾、⼈}[HSK 2]
  \definition{conj.}{assim; portanto; como resultado; conecta frases, expressa resultados e costuma corresponder a expressões como 因为 e 由于}
  \definition[个]{s.}{motivo real; causa real; comportamento adequado}
  \seealsoref{因为}{yin1wei4}
  \seealsoref{由于}{you2yu2}
\end{entry}

\begin{entry}{所有}{suo3you3}{8,6}{⼾、⽉}[HSK 2]
  \definition{adj.}{todo | tudo}
  \definition{adj.}{tudo}
  \definition{s.}{bens; posses;}
  \definition{v.}{possuir; ter}
\end{entry}

\begin{entry}{所在}{suo3 zai4}{8,6}{⼾、⼟}[HSK 5]
  \definition[个]{s.}{lugar; local; localização | o lugar onde alguém ou algo está}
\end{entry}

\begin{entry}{所长}{suo3 zhang3}{8,4}{⼾、⾧}[HSK 3]
  \definition{s.}{chefe de um instituto, etc. | superintendente}
  \seeref{所长}{suo3 chang2}
\end{entry}

\begin{entry}{索性}{suo3xing4}{10,8}{⽷、⼼}
  \definition{adv.}{poderia muito bem | simplesmente | apenas}
\end{entry}

\begin{entry}{锁}{suo3}{12}{⾦}[HSK 5]
  \definition[把]{s.}{fechadura; dispositivo que impede que as pessoas abram facilmente a parte que se abre e fecha | correntes; cadeado e correntes | qualquer coisa com a forma de um cadeado antigo}
  \definition{v.}{trancar; trancar com chave | costurar com ponto fixo | tricotar}
\end{entry}

%%%%% EOF %%%%%

