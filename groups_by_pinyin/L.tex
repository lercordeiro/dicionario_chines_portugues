%%%
%%% L
%%%

\section*{L}\addcontentsline{toc}{section}{L}

\begin{entry}{垃}{la1}{8}{⼟}
  \definition[堆]{s.}{lixo}
\end{entry}

\begin{entry}{垃圾}{la1 ji1}{8,6}{⼟、⼟}[HSK 4]
  \definition{adj.}{lixo; inútil, ruim ou prejudicial}
  \definition[个]{s.}{entulho; lixo; refugo; rejeito; resíduo; coisa inútil que é jogada fora; metáfora para alguém ou algo que perdeu seu valor ou serve a um propósito ruim}
\end{entry}

\begin{entry}{垃圾车}{la1ji1che1}{8,6,4}{⼟、⼟、⾞}
  \definition{s.}{caminhão de lixo}
\end{entry}

\begin{entry}{垃圾电邮}{la1ji1dian4you2}{8,6,5,7}{⼟、⼟、⽥、⾢}
  \definition{s.}{\emph{e-mail} de \emph{spam}}
\end{entry}

\begin{entry}{垃圾堆}{la1ji1dui1}{8,6,11}{⼟、⼟、⼟}
  \definition{s.}{depósito de lixo}
\end{entry}

\begin{entry}{垃圾工}{la1ji1gong1}{8,6,3}{⼟、⼟、⼯}
  \definition{s.}{lixeiro | gari}
\end{entry}

\begin{entry}{垃圾食品}{la1ji1shi2pin3}{8,6,9,9}{⼟、⼟、⾷、⼝}
  \definition{s.}{\emph{junk food}}
\end{entry}

\begin{entry}{垃圾筒}{la1ji1tong3}{8,6,12}{⼟、⼟、⽵}
  \definition{s.}{cesto de lixo}
\end{entry}

\begin{entry}{垃圾箱}{la1ji1xiang1}{8,6,15}{⼟、⼟、⾋}
  \definition{s.}{cesto de lixo}
\end{entry}

\begin{entry}{垃圾邮件}{la1ji1you2jian4}{8,6,7,6}{⼟、⼟、⾢、⼈}
  \definition{s.}{\emph{spam}, \emph{e-mail} não solicitado}
\end{entry}

\begin{entry}{拉}{la1}{8}{⼿}[HSK 2]
  \definition{s.}{abreviação de América Latina, 拉丁美洲}
  \definition{v.}{puxar; arrastar; rebocar | transportar por veículo; rebocar | arrastar (ou puxar) para fora | mover (tropas para um lugar) | dar uma mãozinha; ajudar | arrastar para dentro; implicar; envolver | criar (criança) | atrair; conquistar; solicitar; angariar votos | bater-papo | organizar; preparar | ter dívidas; estar endividado | pressionar; recrutar à força | (no tênis, tênis de mesa, etc.) levantar (a bola) | tocar (certos instrumentos musicais); puxar uma parte do instrumento para que ele emita som | prolongar; espaçar | envolver-se em | (coloquial) esvaziar os intestinos | levantar, uma das técnicas do tênis de mesa | destruir; esmagar; quebrar}
  \seeref{拉}{la4}
  \seealsoref{拉丁美洲}{la1ding1 mei3zhou1}
\end{entry}

\begin{entry}{拉丁美洲}{la1ding1 mei3zhou1}{8,2,9,9}{⼿、⼀、⽺、⽔}
  \definition*{s.}{América Latina, nome coletivo dos países da América Central e do Sul, devido ao fato de a maioria de seus habitantes ser descendente de povos latinos e de a língua falada ser do grupo latino}
\end{entry}

\begin{entry}{拉开}{la1 kai1}{8,4}{⼿、⼶}[HSK 4]
  \definition{v.}{puxar para abrir; recuar| ampliar; espaçar; distanciar; afastar; separar}
\end{entry}

\begin{entry}{拉拉队}{la1la1dui4}{8,8,4}{⼿、⼿、⾩}
  \definition{s.}{claque | torcida}
\end{entry}

\begin{entry}{拉萨}{la1sa4}{8,11}{⼿、⾋}
  \definition*{s.}{Lhasa, capital da Região Autônoma do Tibete, 西藏自治区}
  \seealsoref{西藏自治区}{xi1zang4 zi4zhi4qu1}
\end{entry}

\begin{entry}{啦}{la1}{11}{⼝}
  \definition{s.}{(onomatoméia) som de canto, aplausos etc.; usado para palavras como 呼啦啦, 哗啦啦, 哩哩啦啦, etc.}
  \seeref{啦}{la5}
  \seealsoref{呼啦啦}{hu1 la1 la1}
  \seealsoref{哗啦啦}{hua1la1 la5}
  \seealsoref{哩哩啦啦}{li1 li1 la1 la1}
\end{entry}

\begin{entry}{拉}{la4}{8}{⼿}
  \definition{s.}{usado em 拉拉蛄 \dpy{la4la4gu3}}
  \seeref{拉}{la1}
  \seealsoref{拉拉蛄}{la4la4gu3}
\end{entry}

\begin{entry}{拉拉蛄}{la4la4gu3}{8,8,11}{⼿、⼿、⾍}
  \variantof{蝲蝲蛄}
\end{entry}

\begin{entry}{落}{la4}{12}{⾋}[HSK 5]
  \definition{v.}{deixar de fora; estar ausente | deixar para trás; esquecer de trazer; deixar algo em algum lugar e esquecer de levar| ficar para trás (ou cair); não conseguir acompanhar}
  \seeref{落}{lao4}
  \seeref{落}{luo4}
\end{entry}

\begin{entry}{蜡}{la4}{14}{⾍}
  \definition{s.}{cera; óleos produzidos por animais, minerais ou plantas | vela}
\end{entry}

\begin{entry}{蜡烛}{la4zhu2}{14,10}{⾍、⽕}
  \definition[根,支]{s.}{vela | círio | peça, geralmente de cera, que possui um pavio e se utiliza para iluminar}
\end{entry}

\begin{entry}{辣}{la4}{14}{⾟}[HSK 4]
  \definition{adj.}{apimentado; picante; pungente; quente | cruel; implacável; venenoso; vicioso}
  \definition{v.}{queimar; picar; formigar; ter uma irritação picante (boca, nariz ou olhos)}
\end{entry}

\begin{entry}{蝲}{la4}{15}{⾍}
  \definition{s.}{lagostim de água doce}
  \seealsoref{蝲蛄}{la4gu3}
\end{entry}

\begin{entry}{蝲蛄}{la4gu3}{15,11}{⾍、⾍}
  \definition{s.}{lagostim; lagostim de água doce}
\end{entry}

\begin{entry}{蝲蝲蛄}{la4la4gu3}{15,15,11}{⾍、⾍、⾍}
  \definition{s.}{grilo toupeira}
\end{entry}

\begin{entry}{啦}{la5}{11}{⼝}[HSK 6]
  \definition{part.}{uma palavra composta de 了 e 啊, que tem o significado de ambos}
  \seeref{啦}{la1}
  \seealsoref{啊}{a5}
  \seealsoref{了}{le5}
\end{entry}

\begin{entry}{来}{lai2}{7}{⽊}[HSK 1]
  \definition*{s.}{Sobrenome Lai}
  \definition{part.}{usado após uma palavra numérica ou de quantidade; indica uma quantidade aproximada | usado depois de numerais como 一, 二, 三; para listar razões ou fatos, etc.}
  \definition{s.}{usado após uma expressão de tempo para indicar uma duração que vai do passado ao presente}
  \definition{v.}{vir; chegar; de outro lugar para o lugar onde o interlocutor se encontra | aparecer; acontecer; vir; (problemas, coisas, etc.) ocorrerem; surgirem | substitui um verbo com significado específico, indicando a realização de uma ação específica | estar indo para; usado antes de outro verbo, indica que algo será feito | vir para fazer algo; usado após outro verbo, indica que se vai fazer algo | usado para indicar um propósito; expressar o objetivo, fazer algo usando o método, a atitude ou a direção anteriores | usado com 得 ou 不 para indicar possibilidade, capacidade ou hábito}
  \seealsoref{不}{bu4}
  \seealsoref{得}{de5}
\end{entry}

\begin{entry}{来不及}{lai2bu5ji2}{7,4,3}{⽊、⼀、⼃}[HSK 4]
  \definition{v.}{ser tarde demais; não ter tempo; não ter tempo suficiente (para fazer algo); não ser possível participar ou se atualizar devido a restrições de tempo}
\end{entry}

\begin{entry}{来到}{lai2 dao4}{7,8}{⽊、⼑}[HSK 1]
  \definition{v.}{chegar; vir}
\end{entry}

\begin{entry}{来得及}{lai2de5ji2}{7,11,3}{⽊、⼻、⼃}[HSK 4]
  \definition{v.}{ainda ter tempo; ser capaz de fazê-lo; ser capaz de fazer algo a tempo; ainda ter tempo de chegar lá ou de se atualizar}
\end{entry}

\begin{entry}{来往}{lai2 wang3}{7,8}{⽊、⼻}[HSK 6]
  \definition{s.}{negociação; contato com alguém; interações sociais}
  \definition{v.}{ir e vir | ter negócios com alguém}
\end{entry}

\begin{entry}{来信}{lai2 xin4}{7,9}{⽊、⼈}[HSK 5]
  \definition{s.}{sua carta; carta recebida; carta ao interlocutor}
  \definition{v.}{enviar uma carta para aqui; enviar uma carta para o remetente}
\end{entry}

\begin{entry}{来源}{lai2yuan2}{7,13}{⽊、⽔}[HSK 4]
  \definition{s.}{origem; causa; fonte; tabula rasa (ou seja, o lugar de onde as coisas vêm)}
  \definition{v.}{originar-se; surgir; ter origem; (algo) originar (seguido de 于)}
  \seealsoref{于}{yu2}
\end{entry}

\begin{entry}{来自}{lai2zi4}{7,6}{⽊、⾃}[HSK 2]
  \definition{v.}{vir de (um local) | \emph{From:} (cabeçalho de \emph{e -mail})}
\end{entry}

\begin{entry}{赖}{lai4}{13}{⾙}[HSK 6]
  \definition*{s.}{Sobrenome Lai}
  \definition{adj.}{ruim; pobre; não é bom}
  \definition{v.}{confiar em; depender de | permanecer em um lugar; prolongar a permanência de alguém em um lugar; ficar e recusar-se a sair | negar o próprio erro ou responsabilidade; voltar atrás na palavra; repudiar; negar; não admitir culpa; não assumir responsabilidade | colocar a culpa nos outros; incriminar falsamente (acusar); acusar alguém de algo errado; acusar alguém falsamente | culpar}
\end{entry}

\begin{entry}{兰}{lan2}{5}{⼋}
  \definition*{s.}{Sobrenome Lan}
  \definition{s.}{orquídea | lírio magnólia}
\end{entry}

\begin{entry}{兰花}{lan2hua1}{5,7}{⼋、⾋}
  \definition{s.}{orquídea}
\end{entry}

\begin{entry}{兰州}{lan2zhou1}{5,6}{⼋、⼮}
  \definition*{s.}{Lanzhou. capital da província de Gansu, 甘肃}
  \seealsoref{甘肃}{gan1su4}
\end{entry}

\begin{entry}{栏}{lan2}{9}{⽊}
  \definition{s.}{cerca; corrimão; balaustrada | curral; galpão; celeiro; chiqueiro | coluna (de uma página ou tabela, ou de um jornal) | quadro (de avisos); prancha; tabuleiro | Esporte: obstáculo}
\end{entry}

\begin{entry}{栏目}{lan2mu4}{9,5}{⽊、⽬}[HSK 6]
  \definition[个,档]{s.}{coluna; programa; seções nomeadas de jornais, revistas, etc. divididas de acordo com a natureza de seu conteúdo}
\end{entry}

\begin{entry}{蓝}{lan2}{13}{⾋}[HSK 2]
  \definition*{s.}{Sobrenome Lan}
  \definition{adj.}{azul}
  \definition{s.}{planta índigo; anil | plantas azuis; refere-se a certas plantas que podem ser usadas como corante azul ou certas plantas cujas folhas são azul-esverdeadas}
\end{entry}

\begin{entry}{蓝领}{lan2 ling3}{13,11}{⾋、⾴}[HSK 6]
  \definition[名,位,个]{s.}{trabalhador braçal}
\end{entry}

\begin{entry}{蓝色}{lan2 se4}{13,6}{⾋、⾊}[HSK 2]
  \definition[抹,片,缕,团,块]{s.}{cor azul}
\end{entry}

\begin{entry}{篮}{lan2}{16}{⽵}
  \definition[个]{s.}{cesto | o anel de ferro e a rede na cesta de basquete}
\end{entry}

\begin{entry}{篮球}{lan2qiu2}{16,11}{⽵、⽟}[HSK 2]
  \definition[个,只]{s.}{basquetebol | bola de basquete; refere-se à bola utilizada no basquetebol}
\end{entry}

\begin{entry}{懒}{lan3}{16}{⼼}[HSK 6]
  \definition{adj.}{indolente; preguiçoso (oposto de 勤) | lento; lânguido | ocioso; preguiçoso}
  \seealsoref{勤}{qin2}
\end{entry}

\begin{entry}{懒虫}{lan3chong2}{16,6}{⼼、⾍}
  \definition{s.}{desleixado ocioso | (insulto) sujeito preguiçoso}
\end{entry}

\begin{entry}{懒怠}{lan3dai4}{16,9}{⼼、⼼}
  \definition{s.}{preguiça}
\end{entry}

\begin{entry}{懒得}{lan3de5}{16,11}{⼼、⼻}
  \definition{adv.}{demasiado preguiçoso}
  \definition{v.}{não sentir vontade (de fazer algo)}
\end{entry}

\begin{entry}{懒惰}{lan3duo4}{16,12}{⼼、⼼}
  \definition{adj.}{preguiçoso}
\end{entry}

\begin{entry}{懒鬼}{lan3gui3}{16,9}{⼼、⿁}
  \definition{s.}{cara preguiçoso}
\end{entry}

\begin{entry}{懒汉}{lan3han4}{16,5}{⼼、⽔}
  \definition{s.}{sujeito ocioso | vagabundo | preguiçosos}
\end{entry}

\begin{entry}{懒人}{lan3ren2}{16,2}{⼼、⼈}
  \definition{s.}{pessoa preguiçosa}
\end{entry}

\begin{entry}{懒散}{lan3san3}{16,12}{⼼、⽁}
  \definition{adj.}{inativo | indolente | preguiçoso | negligente}
\end{entry}

\begin{entry}{懒腰}{lan3yao1}{16,13}{⼼、⾁}
  \definition[个]{s.}{alongamento (do corpo)}
\end{entry}

\begin{entry}{烂}{lan4}{9}{⽕}[HSK 5]
  \definition{adj.}{macio; pastoso; amassado | podre; deteriorado | quebrado; esfarrapado; gasto | desorganizado; indigno}
  \definition{adv.}{totalmente; extremamente; completamente; expressa um grau muito profundo}
  \definition{v.}{apodrecer; infeccionar; decompor-se}
\end{entry}

\begin{entry}{廊}{lang2}{11}{⼴}
  \definition[个]{s.}{varanda; corredor}
\end{entry}

\begin{entry}{廊坊}{lang2fang2}{11,7}{⼴、⼟}
  \definition*{s.}{Cidade de Langfang em Hebei}
\end{entry}

\begin{entry}{朗}{lang3}{10}{⽉}
  \definition*{s.}{Sobrenome Lang}
  \definition{adj.}{claro; brilhante | alto e claro (som)}
\end{entry}

\begin{entry}{朗读}{lang3du2}{10,10}{⽉、⾔}[HSK 5]
  \definition{v.}{ler em voz alta; recitar com voz clara e alta}
\end{entry}

\begin{entry}{浪}{lang4}{10}{⽔}
  \definition*{s.}{Sobrenome Lang}
  \definition{adj.}{desenfreado; perdulário}
  \definition{adv.}{livremente}
  \definition[朵,阵,波]{s.}{onda; vagalhão; rebentação | algo ondulatório | coisas ondulando como ondas}
  \definition{v.}{passear; divagar}
\end{entry}

\begin{entry}{浪费}{lang4fei4}{10,9}{⽔、⾙}[HSK 3]
  \definition{adj.}{desperdiçado; extravagante; não econômico}
  \definition{v.}{desperdiçar; dissipar; esbanjar; ser extravagante; uso excessivo ou inadequado de bens, recursos humanos, tempo, etc.}
\end{entry}

\begin{entry}{浪花}{lang4hua1}{10,7}{⽔、⾋}
  \definition[朵]{s.}{\emph{spray} | \emph{spray} do oceano | (figurativo) acontecimentos de sua vida}
\end{entry}

\begin{entry}{浪漫}{lang4man4}{10,14}{⽔、⽔}[HSK 5]
  \definition{adj.}{romântico; poético | não convencional; boêmio; abandonado; libertino; devasso; comportar-se de maneira descuidada e descuidada (geralmente se referindo a relacionamentos entre homens e mulheres) | irrealista; impraticável}
\end{entry}

\begin{entry}{捞}{lao1}{10}{⼿}
  \definition{v.}{pescar | dragar}
\end{entry}

\begin{entry}{劳}{lao2}{7}{⼒}
  \definition*{s.}{Sobrenome Lao}
  \definition{adj.}{difícil; cansativo; cansado}
  \definition{s.}{fadiga; trabalho árduo | ação meritória; serviço; conquistas | trabalhador | mérito | trabalhador braçal}
  \definition{v.}{trabalho; labor | esforço; exercício intenso | (pedir um favor a alguém, também 有劳) colocar alguém no trabalho de | expressar apreço (ao executor de uma tarefa); recompensar | colocar alguém no trabalho de; incomodar alguém com algo | trazer presentes para}
  \seealsoref{有劳}{you3lao2}
\end{entry}

\begin{entry}{劳动}{lao2dong4}{7,6}{⼒、⼒}[HSK 5]
  \definition[次]{s.}{trabalho; mão de obra; atividades intelectuais ou físicas que podem criar valor | trabalho físico; trabalho manual; referindo-se especificamente ao trabalho físico}
  \definition{v.}{realizar trabalho físico}
\end{entry}

\begin{entry}{劳工同事}{lao2gong1 tong2shi4}{7,3,6,8}{⼒、⼯、⼝、⼅}
  \definition{s.}{colaborador | colega de trabalho}
\end{entry}

\begin{entry}{牢}{lao2}{7}{⼧}[HSK 6]
  \definition*{s.}{Sobrenome Lao}
  \definition{adj.}{firme; durável}
  \definition{s.}{prisão; cadeia | (cercado para animais) curral; baia; galinheiro; estábulo; estrebaria; cocheira | (arcaico) animal de sacrifício}
\end{entry}

\begin{entry}{老}{lao3}{6}{⽼}[HSK 1,2][Kangxi 125]
  \definition*{s.}{Sobrenome Lao}
  \definition{adj.}{velho; envelhecido; idade avançada | antigo; de longa data; existe há muito tempo | antigo; desatualizado; obsoleto; ultrapassado  | antigo; tradicional; original | coberto de vegetação; os vegetais cresceram além do período ideal para serem consumidos | resistente; endurecido; alimentos muito cozidos | escuro; profundo; (sobre cores) | último nascido; o mais novo | veterano; experiente; sofisticado}
  \definition{adv.}{longo; por muito tempo | sempre (fazendo algo) | muito}
  \definition{pref.}{usado para designar pessoas, ordem de classificação, certos nomes de animais e plantas}
  \definition{s.}{idosos; pessoas mais velhas | ancião; sênior; um título respeitoso para pessoas mais velhas}
  \definition{v.}{envelhecer | morrer; referindo-se à morte de um idoso}
\end{entry}

\begin{entry}{老百姓}{lao3bai3xing4}{6,6,8}{⽼、⽩、⼥}[HSK 3]
  \definition[些]{s.}{povo; civis; pessoas comuns; residentes (em contraste com militares e funcionários públicos)}
\end{entry}

\begin{entry}{老板}{lao3ban3}{6,8}{⽼、⽊}[HSK 3]
  \definition[个,位]{s.}{chefe; dono; líder; refere-se ao gerente de uma empresa comercial ou industrial | antigo título honorífico dado a atores famosos de ópera ou atores que também eram diretores de companhias de ópera}
\end{entry}

\begin{entry}{老兵}{lao3bing1}{6,7}{⽼、⼋}
  \definition{s.}{velho soldado | veterano de guerra | veterano (alguém que tem muita experiência em algum domínio)}
\end{entry}

\begin{entry}{老公}{lao3 gong1}{6,4}{⽼、⼋}[HSK 4]
  \definition[个]{s.}{marido; esposo}
\end{entry}

\begin{entry}{老虎}{lao3hu3}{6,8}{⽼、⾌}
  \definition[只]{s.}{tigre}
  \seealsoref{虎}{hu3}
\end{entry}

\begin{entry}{老家}{lao3 jia1}{6,10}{⽼、⼧}[HSK 4]
  \definition{s.}{cidade natal; local de origem | lugar nativo; refere-se às gerações anteriores da família ou ao local onde a pessoa nasceu ou viveu}
\end{entry}

\begin{entry}{老年}{lao3 nian2}{6,6}{⽼、⼲}[HSK 2]
  \definition[个]{s.}{idoso; velhice; idade acima de 60 ou 70 anos}
\end{entry}

\begin{entry}{老朋友}{lao3 peng2 you3}{6,8,4}{⽼、⽉、⼜}[HSK 2]
  \definition[个,位,名]{s.}{velho amigo; refere-se a amigos que conhecemos há muito tempo e com quem temos uma relação íntima}
\end{entry}

\begin{entry}{老婆}{lao3po2}{6,11}{⽼、⼥}[HSK 4]
  \definition[个]{s.}{esposa}
\end{entry}

\begin{entry}{老人}{lao3 ren2}{6,2}{⽼、⼈}[HSK 1]
  \definition[位]{s.}{homem ou mulher de idade avançada; o idoso; o velho}
\end{entry}

\begin{entry}{老人家}{lao3 ren2 jia1}{6,2,10}{⽼、⼈、⼧}
  \definition{s.}{senhor ancião | madame | senhora | termo educado para mulher ou homem velho}
\end{entry}

\begin{entry}{老师}{lao3shi1}{6,6}{⽼、⼱}[HSK 1]
  \definition[个,位]{s.}{professor; título honorífico para professores; refere-se, de maneira geral, a pessoas que transmitem cultura e tecnologia ou que são dignas de admiração em termos de ideias, moralidade e conhecimentos profissionais}
\end{entry}

\begin{entry}{老是}{lao3 shi4}{6,9}{⽼、⽇}[HSK 2]
  \definition{adv.}{sempre; indica que a ação continua ou que o estado permanece inalterado, equivalente a 一直}
  \seealsoref{一直}{yi4zhi2}
\end{entry}

\begin{entry}{老实}{lao3shi5}{6,8}{⽼、⼧}[HSK 4]
  \definition{adj.}{franco; sincero; honesto | bom; bem-comportado | ingênuo; simplório; meio bobo; facilmente enganado; eufemismo para pouco inteligente}
\end{entry}

\begin{entry}{老太太}{lao3 tai4 tai5}{6,4,4}{⽼、⼤、⼤}[HSK 3]
  \definition[位,名,个]{s.}{velha senhora; (em tratamento direto)Venerável Senhora; uma maneira respeitosa de chamar uma senhora idosa; título honorífico para mulheres idosas | (forma de tratamento) sua velha mãe; minha velha mãe, avó ou sogra; referindo-se à própria mãe, à mãe do outro ou à mãe de outra pessoa, à sogra ou à sogra política}
\end{entry}

\begin{entry}{老头儿}{lao3 tou2r5}{6,5,2}{⽼、⼤、⼉}[HSK 3]
  \definition{s.}{(coloquial) (com um tom de intimidade) velho; velho amigo}
  \seealsoref{老头子}{lao3 tou2zi5}
\end{entry}

\begin{entry}{老头子}{lao3 tou2zi5}{6,5,3}{⽼、⼤、⼦}
  \definition{s.}{velho antiquado (ou velho rabugento) | (referindo-se ao marido idoso) meu velho | chefe de uma sociedade secreta | (coloquial) velho; velho rabugento}
  \seealsoref{老头儿}{lao3 tou2r5}
\end{entry}

\begin{entry}{老乡}{lao3 xiang1}{6,3}{⽼、⼄}[HSK 6]
  \definition[个,位]{s.}{conterrâneo; conterrâneo | uma maneira de chamar um fazendeiro cujo nome você não conhece}
\end{entry}

\begin{entry}{落}{lao4}{12}{⾋}
  \definition{v.}{cair; cair de uma altura elevada | se abaixar; descer; ir para baixo | permanecer; fazer uma parada; deixar para trás | obter; ter; receber}
  \seeref{落}{la4}
  \seeref{落}{lao4}
\end{entry}

\begin{entry}{乐}{le4}{5}{⼃}[HSK 3]
  \definition*{s.}{Sobrenome Le}
  \definition{adj.}{feliz; contente; rejubilante; animado; bem disposto}
  \definition{s.}{prazer; diversão}
  \definition{v.}{desfrutar; ficar feliz em; amar; encontrar prazer em | rir; divertir-se}
  \seeref{乐}{yue4}
\end{entry}

\begin{entry}{乐高}{le4gao1}{5,10}{⼃、⾼}
  \definition*{s.}{Lego (brinquedo)}
\end{entry}

\begin{entry}{乐观}{le4guan1}{5,6}{⼃、⾒}[HSK 3]
  \definition{adj.}{esperançoso; otimista; confiante; espírito alegre, confiante no futuro (oposto a 悲观)}
  \seealsoref{悲观}{bei1guan1}
\end{entry}

\begin{entry}{乐趣}{le4qu4}{5,15}{⼃、⾛}[HSK 4]
  \definition[个,种,些,点]{s.}{alegria; deleite; prazer; implicação de fazer alguém se sentir feliz; um humor de preferência}
\end{entry}

\begin{entry}{乐园}{le4yuan2}{5,7}{⼃、⼞}
  \definition{s.}{paraíso}
\end{entry}

\begin{entry}{了}{le5}{2}{⼅}[HSK 1,3]
  \definition{part.}{usada após verbos ou adjetivos para indicar a conclusão de uma ação, em um momento no passado ou antes do início de outra ação, ou uma ação esperada ou presumida | usada para indicar uma mudança de situação ou estado, seja real ou prevista | comandos ou solicitações em resposta a uma situação alterada; usada para xpressar urgência ou dissuadir | usada para indicar que algo chegou ao extremo; usada no final da frase ou em pausas no meio da frase, para expressar um tom de exclamação}
  \seeref{了}{liao3}
\end{entry}

\begin{entry}{累}{lei2}{11}{⽷}
  \definition*{s.}{Sobrenome Lei}
  \definition{adj.}{incômodo; complicado}
  \definition{s.}{corda; cordão | touro na época de acasalamento}
  \definition{v.}{amarrar; prender; atar | copular}
  \seeref{累}{lei3}
  \seeref{累}{lei4}
\end{entry}

\begin{entry}{雷}{lei2}{13}{⾬}
  \definition*{s.}{Sobrenome Lei}
  \definition[声,个,颗]{s.}{trovão | (militar) mina}
\end{entry}

\begin{entry}{雷电}{lei2dian4}{13,5}{⾬、⽥}
  \definition{s.}{trovão e relâmpago; raio}
\end{entry}

\begin{entry}{雷亚尔}{lei2ya4'er3}{13,6,5}{⾬、⼆、⼩}
  \definition*{s.}{Real Brasileiro}
\end{entry}

\begin{entry}{累}{lei3}{11}{⽷}
  \definition*{s.}{Sobrenome Lei}
  \definition{adj.}{em andamento; repetido; contínuo}
  \definition{v.}{acumular; empilhar; colocar em cima de outro | envolver; implicar | construir empilhando tijolos, pedras, terra, etc.}
  \seeref{累}{lei2}
  \seeref{累}{lei4}
\end{entry}

\begin{entry}{絫}{lei3}{12}{⽷}
  \variantof{累}
\end{entry}

\begin{entry}{泪}{lei4}{8}{⽔}[HSK 4]
  \definition[滴]{s.}{lágrima}
\end{entry}

\begin{entry}{泪水}{lei4 shui3}{8,4}{⽔、⽔}[HSK 4]
  \definition{s.}{lágrima}
\end{entry}

\begin{entry}{类}{lei4}{9}{⽶}[HSK 3]
  \definition*{s.}{Sobrenome Lei}
  \definition{clas.}{tipo; espécie; categoria usada para pessoas ou coisas}
  \definition{s.}{classe; categoria; tipo; variedade; a combinação de muitas coisas semelhantes ou iguais}
  \definition{v.}{assemelhar-se a; ser semelhante a}
\end{entry}

\begin{entry}{类似}{lei4si4}{9,6}{⽶、⼈}[HSK 3]
  \definition{adj.}{semelhante; análogo}
\end{entry}

\begin{entry}{类型}{lei4xing2}{9,9}{⽶、⼟}[HSK 4]
  \definition[种,个]{s.}{tipo; espécie; categoria; tipos formados por coisas com características comuns}
\end{entry}

\begin{entry}{累}{lei4}{11}{⽷}[HSK 1]
  \definition{adj.}{cansado; exausto; fatigado}
  \definition{v.}{cansar; desgastar; fatigar; esgotar | labutar; trabalhar duro}
  \seeref{累}{lei2}
  \seeref{累}{lei3}
\end{entry}

\begin{entry}{冷}{leng3}{7}{⼎}[HSK 1]
  \definition*{s.}{Sobrenome Leng}
  \definition{adj.}{frio; baixa temperatura; sensação de frio | gelado; frio por natureza; sem entusiasmo; sem gentileza | desolado; pouco frequentado; quieto; sem agitação | negligenciado; indesejável; ignorado por todos | raro; estranho; incomum | feito em segredo; filmado de forma escondida; lançado secretamente}
  \definition{v.}{esfriar; resfriar | esfriar; congelar; tornar-se indiferente, apático | ignorar}
\end{entry}

\begin{entry}{冷静}{leng3jing4}{7,14}{⼎、⾭}[HSK 4]
  \definition{adj.}{calmo; descreve uma pessoa que consegue ficar atenta em uma situação importante ou de emergência e não toma decisões aleatórias por causa de seus sentimentos no momento | (lugar) tranquilo; quieto; deserto}
\end{entry}

\begin{entry}{冷门}{leng3men2}{7,3}{⼎、⾨}
  \definition{s.}{uma profissão, ofício ou ramo de aprendizagem que recebe pouca atenção | um vencedor inesperado; azarão}
\end{entry}

\begin{entry}{冷气}{leng3 qi4}{7,4}{⼎、⽓}[HSK 6]
  \definition[股,阵]{s.}{ar frio (ou fresco); correntes de ar frio | ar condicionado; ar resfriado por equipamento de refrigeração | ar condicionado; equipamentos de ar condicionado}
\end{entry}

\begin{entry}{冷水}{leng3 shui3}{7,4}{⼎、⽔}[HSK 6]
  \definition[杯,瓶]{s.}{água fria | água não fervida}
\end{entry}

\begin{entry}{哩哩啦啦}{li1 li1 la1 la1}{10,10,11,11}{⼝、⼝、⼝、⼝}
  \definition{adj.}{espalhado; disperso; disseminado; difuso; esporádico; aqui e ali}
\end{entry}

\begin{entry}{厘}{li2}{9}{⼚}
  \definition*{s.}{Sobrenome Li}
  \definition{clas.}{li, uma unidade tradicional de comprimento, igual a 0,001 chi (市尺), e equivalente a 0,333 milímetro ou 0,013 polegada | li, uma unidade tradicional de peso, igual a 0,0001 jin (市斤), e equivalente a 5 centigramas ou 0,771 grãos | li, uma unidade tradicional de área, igual a 0,01 mu (市亩), e equivalente a 0,667 metro quadrado ou 0,797 jarda quadrada | li, unidade monetária chinesa, igual a 0,1 fen ou 0,001 yuan | li, unidade de taxa de juros, igual a 0,1\% de juros mensais ou 1\% de juros anuais | quantidade muito pequena; fração; o mínimo}
  \definition{v.}{regular; retificar | administrar}
  \seealsoref{市尺}{shi4 chi3}
  \seealsoref{市斤}{shi4jin1}
  \seealsoref{市亩}{shi4mu3}
\end{entry}

\begin{entry}{厘米}{li2mi3}{9,6}{⼚、⽶}[HSK 4]
  \definition{clas.}{centímetro; unidade de comprimento, símbolo cm, 1 metro é igual a 100 centímetros}
\end{entry}

\begin{entry}{离}{li2}{10}{⼇}[HSK 2]
  \definition*{s.}{Um dos Oito Diagramas | Sobrenome Li}
  \definition{prep.}{(ser longe) de\dots até\dots}
  \definition{v.}{partir; separar-se; afastar-se; estar longe de | prescindir; dispensar; ser independente de | mudar de; desviar-se de | mudar de; desviar-se de; trair; ser incompatível}
\end{entry}

\begin{entry}{离不开}{li2 bu4 kai1}{10,4,4}{⼇、⼀、⼶}[HSK 4]
  \definition{v.}{não pode prescindir; ser inseparável de; não ser capaz de se separar ou deixar uma pessoa, coisa ou circunstância}
\end{entry}

\begin{entry}{离婚}{li2hun1}{10,11}{⼇、⼥}[HSK 3]
  \definition{v.+compl.}{divórciar; romper um casamento; obter o divórcio}
\end{entry}

\begin{entry}{离开}{li2kai1}{10,4}{⼇、⼶}[HSK 2]
  \definition{v.}{deixar; partir; desviar-se; separar-se das pessoas, dos lugares e das coisas}
\end{entry}

\begin{entry}{梨}{li2}{11}{⽊}[HSK 5]
  \definition*{s.}{Sobrenome Li}
  \definition[个,颗]{s.}{perira; árvore de pera | pera}
\end{entry}

\begin{entry}{黎}{li2}{15}{⿉}
  \definition*{s.}{Etnia Li, uma das minorias nacionais da província de Hainan | Sobrenome Li}
  \definition{adj.}{numeroso}
  \definition{s.}{multidão}
\end{entry}

\begin{entry}{礼}{li3}{5}{⽰}[HSK 5]
  \definition*{s.}{Sobrenome Li}
  \definition[份]{s.}{observâncias cerimoniais em geral; cerimônia; rito | cortesia; etiqueta; boas maneiras | presente; oferta}
\end{entry}

\begin{entry}{礼拜}{li3 bai4}{5,9}{⽰、⼿}[HSK 5]
  \definition[个]{s.}{dia da semana; usado em conjunto com 一, 二, 三, 四, 五, 六, 日(或天, indica um dia específico da semana | semana; referência à semana | domingo}
  \definition{v.}{prestar homenagem aos deuses que veneram; rezar; orar}
\end{entry}

\begin{entry}{礼节}{li3jie2}{5,5}{⽰、⾋}
  \definition{s.}{protocolo | cerimônia | etiqueta}
\end{entry}

\begin{entry}{礼貌}{li3mao4}{5,14}{⽰、⾘}[HSK 5]
  \definition{adj.}{educado; descreve uma pessoa que fala e age respeitando os outros, sem arrogância, de acordo com as exigências das relações sociais}
  \definition{s.}{cortesia; educação; boas maneiras}
\end{entry}

\begin{entry}{礼让}{li3rang4}{5,5}{⽰、⾔}
  \definition{s.}{cortesia}
  \definition{v.}{mostrar consideração por (outros) | ceder a (outro veículo, etc.)}
\end{entry}

\begin{entry}{礼堂}{li3 tang2}{5,11}{⽰、⼟}[HSK 6]
  \definition[个,座,处]{s.}{auditórios; salão de assembleias; um salão para reuniões ou cerimônias}
\end{entry}

\begin{entry}{礼物}{li3wu4}{5,8}{⽰、⽜}[HSK 2]
  \definition[份,件,个,分,些]{s.}{presente; lembrança; itens oferecidos como forma de respeito ou celebração, referindo-se de maneira geral a itens oferecidos como presente}
\end{entry}

\begin{entry}{李}{li3}{7}{⽊}
  \definition*{s.}{Sobrenome Li}
  \definition{s.}{ameixa}
\end{entry}

\begin{entry}{李四}{li3si4}{7,5}{⽊、⼞}
  \definition*{s.}{Li Si | Zé Ninguém | Nome para uma pessoa não especificada, 2 de 3}
  \seealsoref{王五}{wang2wu3}
  \seealsoref{张三}{zhang1san1}
\end{entry}

\begin{entry}{李子}{li3zi5}{7,3}{⽊、⼦}
  \definition[个]{s.}{ameixa}
\end{entry}

\begin{entry}{里}{li3}{7}{⾥}[HSK 1][Kangxi 166]
  \definition*{s.}{Sobrenome Li}
  \definition{clas.}{li, uma unidade chinesa de comprimento (= 1/2 quilômetro)}
  \definition{s.}{forro; revestimento; interior; parte de trás do tecido | interno; dentro; no interior | vizinhança; vizinhos | cidade natal; local de origem}
\end{entry}

\begin{entry}{里边}{li3 bian5}{7,5}{⾥、⾡}[HSK 1]
  \definition{s.}{em; dentro; no interior}
\end{entry}

\begin{entry}{里面}{li3 mian4}{7,9}{⾥、⾯}[HSK 3]
  \definition{s.}{dentro; interior}
\end{entry}

\begin{entry}{里斯本}{li3si1ben3}{7,12,5}{⾥、⽄、⽊}
  \definition*{s.}{Lisboa}
\end{entry}

\begin{entry}{里斯本大学}{li3si1ben3 da4xue2}{7,12,5,3,8}{⾥、⽄、⽊、⼤、⼦}
  \definition*{s.}{Universidade de Lisboa}
\end{entry}

\begin{entry}{里头}{li3 tou5}{7,5}{⾥、⼤}[HSK 2]
  \definition{s.}{dentro}
\end{entry}

\begin{entry}{哩}{li3}{10}{⼝}
  \definition{clas.}{milha (unidade de comprimento igual a 1.609,344 m)}
  \seeref{哩}{li5}
\end{entry}

\begin{entry}{理}{li3}{11}{⽟}[HSK 6]
  \definition*{s.}{Sobrenome Li}
  \definition{s.}{textura; grão (em madeira, pele, etc.) | ordem; sequência | razão; lógica; verdade | ciências naturais (especialmente física)}
  \definition{v.}{gerenciar; executar | colocar em ordem; arrumar | (geralmente no negativo) prestar atenção a; fazer um gesto ou falar com | tratar | colocar em ordem; limpar | tomar conhecimento de; prestar atenção a; expressar uma atitude; expressar uma opinião}
\end{entry}

\begin{entry}{理财}{li3 cai2}{11,7}{⽟、⾙}[HSK 6]
  \definition{v.}{administrar questões financeiras; conduzir transações financeiras; administrar propriedade; ser responsável pelo trabalho financeiro}
\end{entry}

\begin{entry}{理发}{li3fa4}{11,5}{⽟、⼜}[HSK 3]
  \definition{v.+compl.}{cortar e aparar o cabelo; ter (dar) um corte de cabelo}
\end{entry}

\begin{entry}{理解}{li3jie3}{11,13}{⽟、⾓}[HSK 3]
  \definition{v.}{entender; compreender; compreender o significado por trás de algo através da reflexão e do aprendizado | entender com empatia; achar que os outros não conseguem fazer determinada coisa e demonstrar compaixão, perdão e não crítica}
\end{entry}

\begin{entry}{理论}{li3lun4}{11,6}{⽟、⾔}[HSK 3]
  \definition[套,个]{s.}{teoria; uma série de conclusões tiradas pelas pessoas sobre atividades naturais ou sociais}
  \definition{v.}{argumentar; raciocinar com alguém; discutir com outras pessoas sobre quem está certo ou errado}
\end{entry}

\begin{entry}{理想}{li3xiang3}{11,13}{⽟、⼼}[HSK 2]
  \definition{adj.}{ideal; perfeito | conforme o desejado; satisfatório}
  \definition{adv.}{idealmente}
  \definition[个,种]{s.}{ideal; sonho; aspiração}
\end{entry}

\begin{entry}{理由}{li3you2}{11,5}{⽟、⽥}[HSK 3]
  \definition[个,条,种,堆]{s.}{razão; justificativa; fundamento; a razão pela qual as coisas são feitas desta ou daquela maneira}
\end{entry}

\begin{entry}{理智}{li3zhi4}{11,12}{⽟、⽇}[HSK 6]
  \definition{adj.}{racional; sensato; cabeça fria; sóbrio; calmo}
  \definition{s.}{sentido; razão; intelecto; a capacidade de distinguir o certo do errado, analisar e julgar e controlar as emoções e o comportamento de acordo}
\end{entry}

\begin{entry}{力}{li4}{2}{⼒}[HSK 3][Kangxi 19]
  \definition*{s.}{Sobrenome Li}
  \definition{adj.}{forte; eficiente; capaz | forte; poderoso; referência geral à função das coisas}
  \definition{adv.}{energicamente; arduamente; vigorosamente; com todo o esforço; com toda a dedicação}
  \definition{s.}{força; energia; poder; (física) refere-se à ação de alterar o estado de movimento ou a forma de um objeto |poder; força; habilidade; capacidade; funções dos órgãos do corpo humano | força física; resistência física}
\end{entry}

\begin{entry}{力量}{li4liang5}{2,12}{⼒、⾥}[HSK 3]
  \definition[出]{s.}{força física; força espiritual | habilidade; capacidade | eficácia; efeito | força (pessoa ou grupo que tem muito poder ou influência); referência a uma pessoa ou grupo que pode desempenhar um papel importante}
\end{entry}

\begin{entry}{力气}{li4qi5}{2,4}{⼒、⽓}[HSK 4]
  \definition[点,把]{s.}{força física | esforço}
\end{entry}

\begin{entry}{历}{li4}{4}{⼚}
  \definition{adj.}{todas as anteriores (ocasiões, sessões, etc.)}
  \definition{adv.}{por toda parte; um por um}
  \definition{s.}{experiência; registro | almanaque; anuário; calendário}
  \definition{v.}{passar por; sofrer; experimentar | passar através; atravessar}
\end{entry}

\begin{entry}{历史}{li4shi3}{4,5}{⼚、⼝}[HSK 4]
  \definition[个,门,段]{s.}{histórico; registro do passado; processo de desenvolvimento da natureza e da sociedade humana; processo de desenvolvimento de uma coisa ou pessoa | história; eventos passados; experiência | história; refere-se ao tema da história}
\end{entry}

\begin{entry}{厉}{li4}{5}{⼚}
  \definition*{s.}{Sobrenome Li}
  \definition{adj.}{rigoroso; estrito | severo; sombrio; sério}
\end{entry}

\begin{entry}{厉害}{li4hai5}{5,10}{⼚、⼧}[HSK 5]
  \definition{adj.}{feroz; severo; descreve uma situação como sendo muito grave | severo; duro; descreve uma pessoa que é exigente com os outros, muito severa, muitas vezes deixando os outros um pouco assustados | incrível; talentoso; impressionante; usado para avaliar a capacidade de uma pessoa ou algo que ela fez que é notável | aterrorizante; assustador; descreve animais ferozes e assustadores.}
\end{entry}

\begin{entry}{立}{li4}{5}{⽴}[HSK 5][Kangxi 117]
  \definition{adj.}{ereto; vertical; na vertical}
  \definition{adv.}{imediatamente; instantaneamente}
  \definition{v.}{ficar em pé, com os pés no chão ou apoiados em algum objeto; o objeto deve estar na vertical | erguer; colocar (ou levantar) algo; colocar em pé | encontrar; criar; elaborar; formular; estabelecer | configurar; fundar; estabelecer | viver; existir | ascender ao trono; antigamente, referia-se à ascensão ao trono de um monarca | nomear; designar; antigamente, significava estabelecer uma determinada posição ou status}
\end{entry}

\begin{entry}{立场}{li4chang3}{5,6}{⽴、⼟}[HSK 5]
  \definition[个]{s.}{posição; postura; a posição e a atitude adotadas ao reconhecer e lidar com os problemas | ponto de vista; refere-se especificamente à atitude de reconhecer e lidar com questões a partir dos interesses de uma determinada classe, ou seja, a posição de classe}
\end{entry}

\begin{entry}{立法}{li4fa3}{5,8}{⽴、⽔}
  \definition{s.}{legislação}
  \definition{v.}{promulgar leis | legislar}
\end{entry}

\begin{entry}{立即}{li4ji2}{5,7}{⽴、⼙}[HSK 4]
  \definition{adv.}{prontamente; imediatamente; de imediato}
\end{entry}

\begin{entry}{立刻}{li4ke4}{5,8}{⽴、⼑}[HSK 3]
  \definition{adv.}{imediatamente; de ​​uma vez; indica que algo acontecerá imediatamente após um determinado momento}
\end{entry}

\begin{entry}{利}{li4}{7}{⼑}[HSK 6]
  \definition*{s.}{Sobrenome Li}
  \definition{adj.}{afiado; cortante | favorável; conveniente; sem dificuldades; sem ou com poucas dificuldades}
  \definition{s.}{benefício; vantagem | lucro; ganhos; juros}
  \definition{v.}{beneficiar; tornar vantajoso}
\end{entry}

\begin{entry}{利润}{li4run4}{7,10}{⼑、⽔}[HSK 5]
  \definition[笔]{s.}{lucro; o dinheiro ganho com atividades comerciais e industriais}
\end{entry}

\begin{entry}{利息}{li4xi1}{7,10}{⼑、⼼}[HSK 4]
  \definition{s.}{acréscimo; juros; dinheiro recebido além do valor principal como resultado de depósitos ou empréstimos (diferenciado de 本金)}
  \seealsoref{本金}{ben3 jin1}
\end{entry}

\begin{entry}{利益}{li4yi4}{7,10}{⼑、⽫}[HSK 4]
  \definition[个,种]{s.}{ganho; lucro; juros; benefício}
\end{entry}

\begin{entry}{利用}{li4yong4}{7,5}{⼑、⽤}[HSK 3]
  \definition{v.}{usar; utilizar; fazer uso de; fazer com que algo ou alguém funcione bem| explorar; tirar vantagem de; usar meios para fazer com que pessoas ou coisas sirvam aos seus interesses}
\end{entry}

\begin{entry}{例}{li4}{8}{⼈}
  \definition{adj.}{regular; rotineiro}
  \definition{s.}{exemplo; instância | precedente | caso; instância | regras; estatutos; regulamentos}
  \definition{v.}{analogizar}
\end{entry}

\begin{entry}{例如}{li4ru2}{8,6}{⼈、⼥}[HSK 2]
  \definition{conj.}{por exemplo; tal como; como por exemplo; colocado antes do exemplo, indica que o exemplo vem a seguir}
\end{entry}

\begin{entry}{例外}{li4wai4}{8,5}{⼈、⼣}[HSK 5]
  \definition[个]{s.}{exceção; situações que não se enquadram nas regras gerais ou nas leis comuns}
  \definition{v.}{ser excepcional; ser uma exceção}
\end{entry}

\begin{entry}{例子}{li4 zi5}{8,3}{⼈、⼦}[HSK 2]
  \definition[个]{s.}{exemplo; algo usado para ajudar a explicar ou provar uma determinada situação ou afirmação}
\end{entry}

\begin{entry}{隶}{li4}{8}{⾪}[Kangxi 171]
  \definition*{s.}{Sobrenome Li}
  \definition{s.}{escravo; uma pessoa em servidão; uma pessoa humilde | um dos estilos antigos da caligrafia chinesa}
  \definition{v.}{estar subordinado a}
\end{entry}

\begin{entry}{荔}{li4}{9}{⾋}
  \definition[颗]{s.}{lichia | (arcaico) uma espécie de grama semelhante à taboa}
\end{entry}

\begin{entry}{荔枝}{li4zhi1}{9,8}{⾋、⽊}
  \definition{s.}{lichia}
\end{entry}

\begin{entry}{鬲}{li4}{10}{⿀}[Kangxi 193]
  \definition{s.}{recipiente de cerâmica antigo com três pernas usado para cozinhar, com marcas de cordão na parte externa e pernas ocas}
  \seeref{鬲}{ge2}
\end{entry}

\begin{entry}{詈}{li4}{12}{⾔}
  \definition{v.}{xingar; usar linguagem severa}
\end{entry}

\begin{entry}{詈骂}{li4ma4}{12,9}{⾔、⾺}
  \definition{v.}{xingar | abusar}
\end{entry}

\begin{entry}{哩}{li5}{10}{⼝}
  \definition{part.}{(dialeto) final modal semelhante a 呢 ou 啦, usado em um tom definido, mas um tanto exagerado}
  \seealsoref{啦}{la5}
  \seealsoref{呢}{ne5}
\end{entry}

\begin{entry}{俩}{lia3}{9}{⼈}[HSK 4]
  \definition{num.}{ambos; dois; contração de 两个 | alguns; vários; refere-se a um pequeno número}
\end{entry}

\begin{entry}{俩钱}{lia3qian2}{9,10}{⼈、⾦}
  \definition{s.}{uma pequena quantia de dinheiro}
\end{entry}

\begin{entry}{连}{lian2}{7}{⾡}[HSK 3]
  \definition*{s.}{Sobrenome Lian}
  \definition{adv.}{em sucessão; um após o outro; repetidamente}
  \definition{prep.}{incluindo; incluido | até mesmo}
  \definition[个]{s.}{companhia; unidades organizacionais das forças armadas}
  \definition{v.}{ligar; juntar; conectar | envolver-se (em problemas); implicar; incriminar | costurar; coser}
\end{entry}

\begin{entry}{连接}{lian2 jie1}{7,11}{⾡、⼿}[HSK 5]
  \definition[条]{s.}{conexão}
  \definition{v.}{ligar; unir; relacionar, conectar; anexar}
\end{entry}

\begin{entry}{连忙}{lian2mang2}{7,6}{⾡、⼼}[HSK 3]
  \definition{adv.}{imediatamente; de imediato; com pressa; apressadamente}
\end{entry}

\begin{entry}{连锁反应}{lian2suo3fan3ying4}{7,12,4,7}{⾡、⾦、⼜、⼴}
  \definition{s.}{reação em cadeia}
\end{entry}

\begin{entry}{连续}{lian2xu4}{7,11}{⾡、⽷}[HSK 3]
  \definition{adv.}{continuamente; sucessivamente; em uma fileira; um após o outro}
\end{entry}

\begin{entry}{连续剧}{lian2 xu4 ju4}{7,11,10}{⾡、⽷、⼑}[HSK 3]
  \definition[部,集]{s.}{série; novela; drama dividido em vários episódios, transmitido continuamente pela rádio ou televisão, com enredo contínuo}
\end{entry}

\begin{entry}{帘}{lian2}{8}{⼱}
  \definition{s.}{cortina | tela (pendurada) | bandeira usada como placa de loja}
\end{entry}

\begin{entry}{莲}{lian2}{10}{⾋}
  \definition*{s.}{Sobrenome Lian}
  \definition[粒]{s.}{lótus}
\end{entry}

\begin{entry}{莲花}{lian2hua1}{10,7}{⾋、⾋}
  \definition{s.}{flor de lótus | lírio aquático}
\end{entry}

\begin{entry}{莲藕}{lian2'ou3}{10,18}{⾋、⾋}
  \definition{s.}{raiz de Lotus}
\end{entry}

\begin{entry}{联}{lian2}{12}{⽿}
  \definition{s.}{dísticos (antitéticos)}
  \definition{v.}{aliar-se a; unir-se; juntar-se a}
\end{entry}

\begin{entry}{联合}{lian2he2}{12,6}{⽿、⼝}[HSK 3]
  \definition{adj.}{conjunto; unido; federal; combinado}
  \definition{s.}{aliado; união; aliança; conectar-se ou unir-se para agir em conjunto}
\end{entry}

\begin{entry}{联合国}{lian2 he2 guo2}{12,6,8}{⽿、⼝、⼞}[HSK 3]
  \definition*{s.}{Nações Unidas; Organização internacional fundada em 1945, após o fim da Segunda Guerra Mundial, com sede em Nova Iorque, Estados Unidos ; as suas principais instituições são a Assembleia Geral, o Conselho de Segurança, o Conselho Econômico e Social e o Secretariado; de acordo com a Carta das Nações Unidas, os seus principais objetivos são manter a paz e a segurança internacionais, desenvolver relações amigáveis entre os países e promover a cooperação internacional nas áreas econômica e cultural}
\end{entry}

\begin{entry}{联合会}{lian2he2hui4}{12,6,6}{⽿、⼝、⼈}
  \definition{s.}{federação}
\end{entry}

\begin{entry}{联络}{lian2luo4}{12,9}{⽿、⽷}[HSK 5]
  \definition{v.}{entrar em contato; comunicar-se; entrar em contato com}
\end{entry}

\begin{entry}{联盟}{lian2meng2}{12,13}{⽿、⽫}[HSK 6]
  \definition{s.}{aliança; coalizão; liga; união}
\end{entry}

\begin{entry}{联赛}{lian2 sai4}{12,14}{⽿、⾙}[HSK 6]
  \definition{s.}{jogos da liga | liga (esportiva) | torneio da liga}
\end{entry}

\begin{entry}{联手}{lian2 shou3}{12,4}{⽿、⼿}[HSK 6]
  \definition{v.}{dar as mãos; cooperar | Literário: dar as mãos | agir em conjunto}
\end{entry}

\begin{entry}{联系}{lian2xi4}{12,7}{⽿、⽷}[HSK 3]
  \definition[个,种,层]{s.}{relacionamento; relacionamento entre duas coisas}
  \definition{v.}{entrar em contato; contatar; comunicar-se com alguém por telefone, e-mail ou carta | agendar; entrar em contato com; estabelecer algum tipo de relação com a outra parte | relacionar; combinar; integrar}
\end{entry}

\begin{entry}{联想}{lian2xiang3}{12,13}{⽿、⼼}[HSK 5]
  \definition*{s.}{Lenovo (empresa)}
  \definition{v.}{associar-se a; estabelecer uma conexão mental; lembrar-se de algo; lembrar-se de outras pessoas ou coisas relacionadas devido a alguém ou algo; evocar outros conceitos relacionados devido a um determinado conceito |}
\end{entry}

\begin{entry}{脸}{lian3}{11}{⾁}[HSK 2]
  \definition[张,个]{s.}{rosto (de pessoas ou animais); a parte frontal da cabeça, da testa ao queixo | parte frontal de algo | cara; autoestima; aparência | rosto; expressões faciais}
\end{entry}

\begin{entry}{脸盆}{lian3 pen2}{11,9}{⾁、⽫}[HSK 5]
  \definition[个]{s.}{lavatório; bacia para lavar as mãos e o rosto}
\end{entry}

\begin{entry}{脸色}{lian3 se4}{11,6}{⾁、⾊}[HSK 5]
  \definition{s.}{aparência; tez; cor da pele | aparência; expressão facial | (indicando a condição física de alguém) aparência; cor}
\end{entry}

\begin{entry}{练}{lian4}{8}{⽷}[HSK 2]
  \definition*{s.}{Sobrenome Lian}
  \definition{adj.}{habilidoso; experiente; bem treinado}
  \definition{s.}{seda branca}
  \definition{v.}{tratar, amaciar e branquear a seda por meio de fervura; cozinhar seda crua ou tecidos de seda crua | treinar; praticar; exercitar}
\end{entry}

\begin{entry}{练习}{lian4xi2}{8,3}{⽷、⼄}[HSK 2]
  \definition[项,次]{s.}{exercício (em livros); tarefas ou exercícios organizados para consolidar os resultados da aprendizagem}
  \definition{v.}{praticar; exercitar; repitir várias vezes até ficar bem treinado}
\end{entry}

\begin{entry}{炼}{lian4}{9}{⽕}
  \definition{v.}{fundir; refinar | temperar (um metal) com fogo | pesar a palavra; procurar a frase certa; polir | trabalhar; tornar uma substância pura ou resistente por aquecimento, etc. | polir; fazer as palavras bonitas e concisas}
\end{entry}

\begin{entry}{恋}{lian4}{10}{⼼}
  \definition*{s.}{Sobrenome Lian}
  \definition{v.}{amor (romântico) | ansiar por; sentir-se apegado a | amar; apaixonar-se por | não querendo se separar de; sentir sua falta para sempre; não suportar ficar separado}
\end{entry}

\begin{entry}{恋爱}{lian4'ai4}{10,10}{⼼、⽖}[HSK 5]
  \definition[个,场,段]{s.}{namoro; afeto; amor romântico; ações que demonstram o amor mútuo}
  \definition{v.}{amar; estar apaixonado}
\end{entry}

\begin{entry}{良}{liang2}{7}{⾉}
  \definition*{s.}{Sobrenome Liang}
  \definition{adj.}{bom; ótimo; agradável}
  \definition{adv.}{muito; muito mesmo; de fato}
  \definition{s.}{boas pessoas; pessoas gentis; talentos excepcionais}
\end{entry}

\begin{entry}{良好}{liang2hao3}{7,6}{⾉、⼥}[HSK 4]
  \definition{adj.}{bom; ótimo; bem}
\end{entry}

\begin{entry}{良田}{liang2tian2}{7,5}{⾉、⽥}
  \definition{s.}{terra agrícola boa | terra fértil}
\end{entry}

\begin{entry}{良心}{liang2xin1}{7,4}{⾉、⼼}
  \definition{s.}{consciência}
\end{entry}

\begin{entry}{凉}{liang2}{10}{⼎}[HSK 2]
  \definition{adj.}{frio; gelado; ligeiramente fria (menos do que 冷) | sombrio; desolado; sem animação | desanimado; desapontado | usado para prevenir o calor e manter a temperatura amena; para proteção contra o calor}
  \definition{s.}{frio; refere-se a um ambiente fresco ou a uma brisa fresca}
  \seeref{凉}{liang4}
  \seealsoref{冷}{leng3}
\end{entry}

\begin{entry}{凉快}{liang2kuai5}{10,7}{⼎、⼼}[HSK 2]
  \definition{adj.}{fresco; refrescante}
  \definition{v.}{refrescar; refrescar-se; deixar o corpo fresco e revigorado}
\end{entry}

\begin{entry}{凉水}{liang2 shui3}{10,4}{⼎、⽔}[HSK 3]
  \definition{s.}{água fria; água não aquecida | água não fervida}
\end{entry}

\begin{entry}{凉鞋}{liang2 xie2}{10,15}{⼎、⾰}[HSK 6]
  \definition[双,只]{s.}{sandália; alpargata; alpercata; alparca ; sapatos de verão com cabedal ventilado}
\end{entry}

\begin{entry}{量}{liang2}{12}{⾥}[HSK 4]
  \definition{v.}{medir | estimar; dimensionar}
  \seeref{量}{liang4}
\end{entry}

\begin{entry}{粮}{liang2}{13}{⽶}
  \definition[斤,粒]{s.}{grãos; alimentos; provisões | imposto sobre grãos | nutrição | imposto agrícola; grãos como imposto agrícola}
\end{entry}

\begin{entry}{粮食}{liang2shi5}{13,9}{⽶、⾷}[HSK 4]
  \definition[种,斤,吨,袋]{s.}{alimentos; grãos; termo geral para os vários tipos de arroz, feijão, etc. que podem ser consumidos}
\end{entry}

\begin{entry}{两}{liang3}{7}{⼀}[HSK 1,2]
  \definition*{s.}{Sobrenome Liang}
  \definition{clas.}{liang, uma unidade de peso (=50 gramas)}
  \definition{num.}{dois (sempre usado antes de classificadores) | poucos; alguns; indica um número indeterminado}
  \definition{s.}{ambos (lados); qualquer (lado)}
\end{entry}

\begin{entry}{两岸}{liang3 an4}{7,8}{⼀、⼭}[HSK 5]
  \definition{s.}{ambos os lados; ambas as margens; ambas as costas; entre os dois lados do estreito; bilateral}
\end{entry}

\begin{entry}{两边}{liang3 bian1}{7,5}{⼀、⾡}[HSK 4]
  \definition{s.}{ambos os lados; ambas as direções; ambos os lugares | ambas as partes; ambos os lados}
\end{entry}

\begin{entry}{两侧}{liang3 ce4}{7,8}{⼀、⼈}[HSK 6]
  \definition{s.}{dois flancos; dois (ambos) lados; ambos}
\end{entry}

\begin{entry}{两码事}{liang3ma3shi4}{7,8,8}{⼀、⽯、⼅}
  \definition{expr.}{duas coisas completamente diferentes}
\end{entry}

\begin{entry}{两手}{liang3 shou3}{7,4}{⼀、⼿}[HSK 6]
  \definition{s.}{ambas as mãos | ambos os aspectos; táticas duplas | Coloquial: habilidade; capacidade}
\end{entry}

\begin{entry}{亮}{liang4}{9}{⼇}[HSK 2]
  \definition*{s.}{Sobrenome Lian}
  \definition{adj.}{brilhante; claro | alto e claro; retumbante | esclarecido; aberto e claro}
  \definition{s.}{luz}
  \definition{v.}{iluminar; clarear; brilhar | elevar a voz; ressoar; tornar o som mais alto | revelar; mostrar; aparecer; exibir}
\end{entry}

\begin{entry}{凉}{liang4}{10}{⼎}
  \definition{v.}{deixar algo esfriar; deixar um objeto quente descansar por um tempo para que a temperatura diminua}
  \seeref{凉}{liang2}
\end{entry}

\begin{entry}{辆}{liang4}{11}{⾞}[HSK 2]
  \definition{clas.}{usado para automóveis, veículos, etc.}
\end{entry}

\begin{entry}{量}{liang4}{12}{⾥}
  \definition{s.}{instrumento de medida; antigamente, o termo se referia a objetos como baldes e litros, que medem o volume | capacidade (para tolerância ou ingestão de alimentos ou bebidas); refere-se ao limite do que pode ser acomodado | quantidade; valor; volume; número}
  \definition{v.}{estimar; medir; pesar}
  \seeref{量}{liang2}
\end{entry}

\begin{entry}{疗}{liao2}{7}{⽧}
  \definition{v.}{tratar; curar | recuperar}
\end{entry}

\begin{entry}{疗养}{liao2 yang3}{7,9}{⽧、⼋}[HSK 4]
  \definition{v.}{recuperar; convalescer; tratar pessoas com doenças crônicas ou debilitantes em instituições médicas especializadas com foco na recuperação}
\end{entry}

\begin{entry}{聊}{liao2}{11}{⽿}[HSK 6]
  \definition*{s.}{Sobrenome Liao}
  \definition{adv.}{apenas; meramente; provisoriamente; por enquanto | um pouco; ligeiramente}
  \definition{v.}{tagarelar; conversar; bater papo | confiar (ou depender, recorrer) a}
\end{entry}

\begin{entry}{聊天}{liao2tian1}{11,4}{⽿、⼤}
  \definition{v.+compl.}{papear | bater papo}
\end{entry}

\begin{entry}{聊天儿}{liao2 tian1r5}{11,4,2}{⽿、⼤、⼉}[HSK 6]
  \definition{v.+compl.}{conversar; fofocar; bater papo; duas ou mais pessoas conversando sem um tópico ou propósito específico}
\end{entry}

\begin{entry}{了}{liao3}{2}{⼅}
  \definition*{s.}{Sobrenome Liao}
  \definition{adv.}{inteiramente; um pouco; totalmente (mais usado em negativas)}
  \definition{v.}{terminar; concluir; encerrar; cumprir; eliminar; resolver | compreender; saber; perceber; saber claramente | expressar possibilidade ou impossibilidade; usado com 得 ou 不 após o verbo, indica possibilidade ou impossibilidade}
  \seeref{了}{le5}
  \seealsoref{不}{bu4}
  \seealsoref{得}{de5}
\end{entry}

\begin{entry}{了不起}{liao3bu5qi3}{2,4,10}{⼅、⼀、⾛}[HSK 4]
  \definition{adj.}{incrível; fantástico; extraordinário | sério; grave}
\end{entry}

\begin{entry}{了解}{liao3jie3}{2,13}{⼅、⾓}[HSK 4]
  \definition{v.}{entender; compreender | investigar; indagar sobre}
\end{entry}

\begin{entry}{料}{liao4}{10}{⽃}[HSK 6]
  \definition{clas.}{usado na medicina tradicional chinesa para preparar pílulas | unidade usada para calcular um pedaço de madeira, é a seção transversal em ambas as extremidades, que é de 1 pé (quadrado) com 7 pés de comprimento}
  \definition{s.}{material; coisa | (grão) alimento; forragem | artigos de vidro; vidros coloridos opacos | (para pílulas de medicina chinesa) prescrição}
  \definition{v.}{supor; esperar; antecipar | gerenciar; cuidar de | prever}
\end{entry}

\begin{entry}{列}{lie4}{6}{⼑}[HSK 4]
  \definition{v.}{organizar; formar uma linha; alinhar | listar; inserir em uma lista}
\end{entry}

\begin{entry}{列车}{lie4che1}{6,4}{⼑、⾞}[HSK 4]
  \definition{s.}{trem; trem em uma composição contínua, puxado por uma locomotiva e equipado com uma tripulação e marcações prescritas; geralmente um trem de passageiros}
\end{entry}

\begin{entry}{列入}{lie4 ru4}{6,2}{⼑、⼊}[HSK 4]
  \definition{v.}{incluir em uma lista}
\end{entry}

\begin{entry}{列为}{lie4 wei2}{6,4}{⼑、⼂}[HSK 4]
  \definition{v.}{ser classificado como; ser listado como}
\end{entry}

\begin{entry}{烈}{lie4}{10}{⽕}
  \definition*{s.}{Sobrenome Lie}
  \definition{adj.}{forte; violento; intenso; feroz | justo; severo | firme; convicto; rigoroso}
  \definition{s.}{pessoa que morreu por uma causa justa | conquistas; façanhas | mártir sacrificando-se por uma causa justa}
\end{entry}

\begin{entry}{烈士}{lie4shi4}{10,3}{⽕、⼠}
  \definition{s.}{mártir}
\end{entry}

\begin{entry}{猎}{lie4}{11}{⽝}
  \definition[个]{s.}{traje de caça}
  \definition{v.}{caçar | procurar; perseguir}
\end{entry}

\begin{entry}{猎物}{lie4wu4}{11,8}{⽝、⽜}
  \definition{s.}{presa (vítima de um predador)}
\end{entry}

\begin{entry}{裂}{lie4}{12}{⾐}[HSK 6]
  \definition{s.}{entalhe; incisão; entalhes grandes e profundos nas bordas das folhas ou corolas | brecha; lacuna; rachadura; refere-se à rachadura ou divisão que aparece na superfície ou no interior de um objeto}
  \definition{v.}{dividir; rachar; rasgar | (figurativo) quebrar; esmagar; arruinar}
\end{entry}

\begin{entry}{邻}{lin2}{7}{⾢}
  \definition{adj.}{vizinho; perto; adjacente; perto; próximo}
  \definition{s.}{vizinho | bairro; vizinhança}
\end{entry}

\begin{entry}{邻居}{lin2ju1}{7,8}{⾢、⼫}[HSK 5]
  \definition[个,位,家]{s.}{vizinho; pessoas ou famílias que moram muito perto}
\end{entry}

\begin{entry}{临}{lin2}{9}{⼁}
  \definition*{s.}{Sobrenome Lin}
  \definition{adv.}{pouco antes; prestes a; no ponto de}
  \definition{v.}{encarar; enfrentar; aproximar-se | chegar; estar presente | copiar (um modelo de caligrafia ou pintura); traçar sobre as palavras ou figuras | olhar de cima para baixo | ir de cima para baixo}
\end{entry}

\begin{entry}{临近}{lin2jin4}{9,7}{⼁、⾡}
  \definition{v.}{aproximar-se; estar perto de}
\end{entry}

\begin{entry}{临时}{lin2shi2}{9,7}{⼁、⽇}[HSK 4]
  \definition{adj.}{temporário; provisório; por um breve período}
  \definition{adv.}{no momento em que algo acontece (quando as coisas dão errado)}
\end{entry}

\begin{entry}{淋}{lin2}{11}{⽔}
  \definition{v.}{borrifar | pingar | derramar | encharcar}
  \seeref{淋}{lin4}
\end{entry}

\begin{entry}{淋}{lin4}{11}{⽔}
  \definition{s.}{gonorréia}
  \definition{v.}{filtrar | coar | drenar}
  \seeref{淋}{lin2}
\end{entry}

\begin{entry}{令}{ling2}{5}{⼈}
  \definition*{s.}{Antigo nome geográfico, na região atual de Linyi, província de Shanxi | Sobrenome Ling}
  \seeref{令}{ling3}
  \seeref{令}{ling4}
\end{entry}

\begin{entry}{灵}{ling2}{7}{⽕}
  \definition*{s.}{Sobrenome Ling}
  \definition{adj.}{rápido; inteligente; afiado | eficaz; efetivo | flexível; hábil}
  \definition{s.}{espírito; alma | inteligência; mente | fada; duende; elfo | restos mortais do falecido; esquife | carro funerário; caixão ou algo relacionado aos mortos}
\end{entry}

\begin{entry}{灵感}{ling2gan3}{7,13}{⽕、⼼}
  \definition{s.}{inspiração | explosão de criatividade em empreendimento científico ou artístico}
\end{entry}

\begin{entry}{灵魂}{ling2hun2}{7,13}{⽕、⿁}
  \definition{s.}{alma | espírito}
\end{entry}

\begin{entry}{灵活}{ling2huo2}{7,9}{⽕、⽔}[HSK 6]
  \definition[种,点,些]{adj.}{ágil; rápido; ligeiro; descreve a capacidade de fazer rapidamente mudanças apropriadas com base na situação ao lidar com as coisas | flexível; elástico; descreve reações rápidas, como movimentos e funções cerebrais}
\end{entry}

\begin{entry}{铃}{ling2}{10}{⾦}[HSK 5]
  \definition{s.}{sino; instrumento musical feito de metal | objetos em forma de sino | cápsula; botão; broto}
\end{entry}

\begin{entry}{铃声}{ling2 sheng1}{10,7}{⾦、⼠}[HSK 5]
  \definition{s.}{o tilintar de sinos; o som de um sino tocando}
\end{entry}

\begin{entry}{陵}{ling2}{10}{⾩}
  \definition*{s.}{Sobrenome Ling}
  \definition{s.}{colina; monte | túmulo imperial; mausoléu}
  \definition{v.}{(literário) intimidar; violar}
\end{entry}

\begin{entry}{陵园}{ling2yuan2}{10,7}{⾩、⼞}
  \definition{s.}{cemitério}
\end{entry}

\begin{entry}{菱}{ling2}{11}{⾋}
  \definition{s.}{maruca; caltrop aquático; castanha d'água}
\end{entry}

\begin{entry}{菱角}{ling2jiao5}{11,7}{⾋、⾓}
  \definition{s.}{castanha d'água}
\end{entry}

\begin{entry}{零}{ling2}{13}{⾬}[HSK 1]
  \definition*{s.}{Sobrenome Ling}
  \definition{adj.}{ímpar; dispersos; fragmentados (em oposição a 整)}
  \definition{num.}{zero; 0; também grafado como 〇; representa um número menor que qualquer número positivo e maior que qualquer número negativo; representa a ausência de quantidade | zero grau no termômetro | usado para indicar qualidade, comprimento, tempo, idade, etc. Entre dois dígitos, indica que a quantidade da unidade mais alta é acompanhada pela quantidade da unidade mais baixa | sinal de zero (0); nulo; espaço em branco para indicar números em caracteres chineses maiúsculos}
  \definition{s.}{fragmento; fração; lote ímpar; um número fracionário que não é suficiente para uma determinada unidade; um ponto decimal diferente de um inteiro}
  \definition{v.}{(de chuva, lágrimas, etc.) cair | murchar e cair}
  \seealsoref{整}{zheng3}
\end{entry}

\begin{entry}{零散}{ling2san3}{13,12}{⾬、⽁}
  \definition{adj.}{espalhado; disperso}
\end{entry}

\begin{entry}{零食}{ling2shi2}{13,9}{⾬、⾷}[HSK 4]
  \definition[包,袋,盒,箱,堆]{s.}{lanches; refrescos; petiscos entre as refeições; alimentação esporádica, além das refeições normais}
\end{entry}

\begin{entry}{零下}{ling2 xia4}{13,3}{⾬、⼀}[HSK 2]
  \definition{s.}{abaixo de zero; negativo}
\end{entry}

\begin{entry}{令}{ling3}{5}{⼈}
  \definition{clas.}{resma (de papel); unidade de medida de papel: 500 folhas inteiras de papel original produzidas mecanicamente equivalem a 1 resma}
\end{entry}

\begin{entry}{岭}{ling3}{8}{⼭}
  \definition{s.}{cordilheira}
\end{entry}

\begin{entry}{领}{ling3}{11}{⾴}[HSK 3]
  \definition{clas.}{usado para roupas, mantos, esteiras, tapetes, telas, etc.}
  \definition{s.}{pescoço; gargalo | gola; colarinho; faixa de pescoço | esboço; ponto principal; essência}
  \definition{v.}{conduzir; guiar; orientar | possuir; ser o possuidor de; ter jurisdição sobre | obter; conseguir; receber (o que foi distribuído) | aceitar; tomar |entender; compreender (o significado)}
\end{entry}

\begin{entry}{领带}{ling3 dai4}{11,9}{⾴、⼱}[HSK 5]
  \definition[条]{s.}{colar; gargantilha; gravata}
\end{entry}

\begin{entry}{领导}{ling3dao3}{11,6}{⾴、⼨}[HSK 3]
  \definition[个,位,名,些]{s.}{líder; liderança; pessoa que ocupa uma posição de liderança}
  \definition{v.}{liderar; exercer liderança; (elogio) liderar, gerenciar outras pessoas;  trabalhar com outras pessoas ou avançar em direção a um objetivo}
\end{entry}

\begin{entry}{领情}{ling3qing2}{11,11}{⾴、⼼}
  \definition{v.+compl.}{sentir-se grato a alguém}
\end{entry}

\begin{entry}{领取}{ling3 qu3}{11,8}{⾴、⼜}[HSK 6]
  \definition{v.}{sacar; receber; obter; receber o que lhe é enviado}
\end{entry}

\begin{entry}{领先}{ling3xian1}{11,6}{⾴、⼉}[HSK 3]
  \definition{v.}{liderar; assumir a liderança; estar na liderança; (velocidade, desempenho, etc.) superar pessoas ou coisas semelhantes, estar na vanguarda}
\end{entry}

\begin{entry}{领袖}{ling3xiu4}{11,10}{⾴、⾐}[HSK 6]
  \definition[个,位,名]{s.}{líder de estados, grupos políticos, organizações de massa, etc.}
\end{entry}

\begin{entry}{令}{ling4}{5}{⼈}[HSK 5]
  \definition{adj.}{bom; excelente | termos de cortesia usados para se referir aos familiares e parentes da outra pessoa}
  \definition{s.}{ordem; decreto; comando; ordem emitida pela autoridade superior | um título oficial; administradores de certos departamentos governamentais na antiguidade | temporada; estação; clima e fenologia de uma determinada estação | poema-canção; letra curta}
  \definition{v.}{ordenar; comandar | fazer com que alguém; fazer com que; permitir que}
  \seeref{令}{ling2}
  \seeref{令}{ling3}
\end{entry}

\begin{entry}{令人}{ling4ren2}{5,2}{⼈、⼈}
  \definition{v.}{causar alguém (a fazer alguma coisa) | fazer alguém ficar zangado, encantado, etc.}
\end{entry}

\begin{entry}{另}{ling4}{5}{⼝}[HSK 6]
  \definition*{s.}{Sobrenome Ling}
  \definition{adv.}{além disso; indica que está fora do escopo da declaração | no lugar de; em vez de}
  \definition{pron.}{(com substantivos) outro; diferente; refere-se a pessoas ou coisas fora do escopo do que é dito}
\end{entry}

\begin{entry}{另外}{ling4wai4}{5,5}{⼝、⼣}[HSK 3]
  \definition{adv.}{além disso; em adição; ademais; além do mais; além de que; além do que já foi dito}
  \definition{conj.}{além disso; usada entre duas ou mais frases, indica algo além do que foi mencionado anteriormente}
  \definition{pron.}{outro; além das pessoas ou coisas mencionadas anteriormente}
\end{entry}

\begin{entry}{另一方面}{ling4 yi4 fang1 mian4}{5,1,4,9}{⼝、⼀、⽅、⾯}[HSK 3]
  \definition{adv./conj.}{outro aspecto | por outro lado; por sua vez; em contrapartida}
\end{entry}

\begin{entry}{刘}{liu2}{6}{⼑}
  \definition*{s.}{Sobrenome Liu}
  \definition{s.}{Clássico: um tipo de machado de batalha}
  \definition{v.}{matar}
\end{entry}

\begin{entry}{流}{liu2}{10}{⽔}[HSK 2]
  \definition*{s.}{Sobrenome Liu}
  \definition{adj.}{fluente; tão suave quanto a água corrente}
  \definition{clas.}{lúmen; abreviação de lumens, 流明}
  \definition[名,个]{s.}{corrente de água | corrente; algo que se assemelha a um fluxo de água | razão; taxa; classe; grau; ramificação; facção; hierarquia}
  \definition{v.}{(de líquido) fluir | vaguear; vagar; mover-se de um lugar para outro; movimentar-se sem direção fixa | espalhar; circular; transmitir; divulgar | degenerar; mudar para pior; tender (aspecto negativo) | banir; enviar para o exílio | correr (ou fluir) como líquido; refere-se à parte do rio após deixar sua nascente (em contraste com a 源)}
  \seealsoref{流明}{liu2ming2}
  \seealsoref{源}{yuan2}
\end{entry}

\begin{entry}{流传}{liu2chuan2}{10,6}{⽔、⼈}[HSK 4]
  \definition{v.}{espalhar; circular; passar adiante}
\end{entry}

\begin{entry}{流动}{liu2 dong4}{10,6}{⽔、⼒}[HSK 5]
  \definition{v.}{fluir; correr; circular | ir de um lugar para outro; mover-se; mudar frequentemente de posição}
\end{entry}

\begin{entry}{流感}{liu2 gan3}{10,13}{⽔、⼼}[HSK 6]
  \definition{s.}{gripe; influenza; abreviação de 流行性感冒}
  \seealsoref{流行性感冒}{liu2xing2 xing4 gan3mao4}
\end{entry}

\begin{entry}{流利}{liu2li4}{10,7}{⽔、⼑}[HSK 2]
  \definition{adj.}{fluente; suave; lúcido; falar e escrever com fluência e clareza | com fluência; sem dificuldades}
\end{entry}

\begin{entry}{流明}{liu2ming2}{10,8}{⽔、⽇}
  \definition{s.}{(empréstimo linguístico) lúmen (unidade de fluxo luminoso)}
\end{entry}

\begin{entry}{流水}{liu2shui3}{10,4}{⽔、⽔}
  \definition{s.}{água corrente | (negócio) rotatividade}
\end{entry}

\begin{entry}{流通}{liu2tong1}{10,10}{⽔、⾡}[HSK 5]
  \definition{v.}{(de ar, dinheiro, mercadorias, etc.) fluir; circular}
\end{entry}

\begin{entry}{流星}{liu2xing1}{10,9}{⽔、⽇}
  \definition{s.}{meteoro | estrela cadente}
\end{entry}

\begin{entry}{流行}{liu2xing2}{10,6}{⽔、⾏}[HSK 2]
  \definition{adj.}{popular; na moda; muito popular}
  \definition{v.}{ser popular; prevalecer; espalhar-se amplamente; divulgar amplamente}
\end{entry}

\begin{entry}{流行性感冒}{liu2xing2 xing4 gan3mao4}{10,6,8,13,9}{⽔、⾏、⼼、⼼、⽇}
  \definition{s.}{gripe muito forte; influenza}
\end{entry}

\begin{entry}{留}{liu2}{10}{⽥}[HSK 2]
  \definition*{s.}{Sobrenome Liu}
  \definition{v.}{ficar; permanecer; parar em um determinado local ou posição; não se afastar | estudar no exterior (geralmente seguido pelo nome de um país com uma sílaba) | pedir a alguém para ficar; manter alguém onde está | concentrar-se em; concentrar a atenção em algo | manter; guardar; reservar; não joger fora | acumular; deixar crescer | aceitar; receber | transmitir (legado); deixar para trás}
\end{entry}

\begin{entry}{留神}{liu2shen2}{10,9}{⽥、⽰}
  \definition{v.+compl.}{tomar cuidado | prestar atenção | manter os olhos abertos}
\end{entry}

\begin{entry}{留下}{liu2 xia4}{10,3}{⽥、⼀}[HSK 2]
  \definition{v.}{deixar; parar em algum lugar}
\end{entry}

\begin{entry}{留学}{liu2xue2}{10,8}{⽥、⼦}[HSK 3]
  \definition{v.}{estudar no exterior; permanecer no estrangeiro para estudar ou pesquisar}
\end{entry}

\begin{entry}{留学生}{liu2 xue2 sheng1}{10,8,5}{⽥、⼦、⽣}[HSK 2]
  \definition[个,位,名,批,帮]{s.}{estudante estrangeiro; estudante que retornou; estudante que estuda no exterior}
\end{entry}

\begin{entry}{留言}{liu2 yan2}{10,7}{⽥、⾔}[HSK 6]
  \definition[条]{s.}{mensagem; recado}
  \definition{v.}{deixar uma mensagem; deixar seus comentários}
\end{entry}

\begin{entry}{柳}{liu3}{9}{⽊}
  \definition*{s.}{Sobrenome Liu}
  \definition{s.}{salgueiro}
\end{entry}

\begin{entry}{柳橙汁}{liu3cheng2zhi1}{9,16,5}{⽊、⽊、⽔}
  \definition[瓶,杯,罐,盒]{s.}{suco de laranja}
  \seealsoref{橙汁}{cheng2zhi1}
  \seealsoref{橘子汁}{ju2zi5zhi1}
\end{entry}

\begin{entry}{六}{liu4}{4}{⼋}[HSK 1]
  \definition*{s.}{Sobrenome Liu}
  \definition{num.}{seis; 6}
  \definition{s.}{símbolo musical utilizado na partitura da música tradicional chinesa, representando o primeiro grau da escala musical, equivalente ao ``5'' da notação musical simplificada}
\end{entry}

\begin{entry}{陆}{liu4}{7}{⾩}
  \definition{num.}{seis, usado para o numeral 六 em cheques, etc. para evitar erros ou alterações}
  \seeref{陆}{lu4}
  \seealsoref{六}{liu4}
\end{entry}

\begin{entry}{遛}{liu4}{13}{⾡}
  \definition{v.}{passear | andar (um animal) | caminhar conduzindo um animal doméstico}
\end{entry}

\begin{entry}{遛狗}{liu4gou3}{13,8}{⾡、⽝}
  \definition{v.+compl.}{passear com um cachorro}
\end{entry}

\begin{entry}{龙}{long2}{5}{⿓}[HSK 3][Kangxi 212]
  \definition*{s.}{Sobrenome Long}
  \definition[条]{s.}{dragão; animal mítico e sobrenatural, com chifres, escamas, garras e bigodes, capaz de voar e mergulhar na água, provocar nuvens e chuva | dinossauro; um enorme réptil extinto; referência a certos répteis gigantes da antiguidade | do imperador; dragão como símbolo do imperador; usado na era feudal como símbolo do imperador; também se refere a coisas pertencentes ao imperador | em forma de dragão; com um desenho de dragão; refere-se a certos objetos que formam uma sequência semelhante a um dragão ou decorados com motivos de dragões}
\end{entry}

\begin{entry}{龙山}{long2shan1}{5,3}{⿓、⼭}
  \definition*{s.}{Longshan}
\end{entry}

\begin{entry}{龙虾}{long2xia1}{5,9}{⿓、⾍}
  \definition{s.}{lagosta}
\end{entry}

\begin{entry}{笼}{long2}{11}{⽵}
  \definition{s.}{armação fechada de bambu, arame, etc. | jaula | gaiola}
  \seeref{笼}{long3}
\end{entry}

\begin{entry}{笼子}{long2zi5}{11,3}{⽵、⼦}
  \definition{s.}{jaula | cesta | gaiola | recipiente}
  \seeref{笼子}{long3zi5}
\end{entry}

\begin{entry}{笼}{long3}{11}{⽵}
  \definition{v.}{envolver | cobrir}
  \seeref{笼}{long2}
\end{entry}

\begin{entry}{笼子}{long3zi5}{11,3}{⽵、⼦}
  \definition{s.}{caixa grande | porta-malas}
  \seeref{笼子}{long2zi5}
\end{entry}

\begin{entry}{弄}{long4}{7}{⼶}
  \definition{s.}{rua estreita; beco; viela; travessa}
  \seeref{弄}{nong4}
\end{entry}

\begin{entry}{楼}{lou2}{13}{⽊}[HSK 1]
  \definition*{s.}{Sobrenome Lou}
  \definition{clas.}{andar, piso}
  \definition[层,座,栋]{s.}{um prédio com muitos andares | piso; andar | superestrutura; uma estrutura com um convés superior; um andar adicional construído sobre uma casa ou outro edifício | nome usado para certas lojas ou locais de entretenimento | arco ornamental; certas construções decorativas altas com passagens por baixo}
\end{entry}

\begin{entry}{楼道}{lou2 dao4}{13,12}{⽊、⾡}[HSK 6]
  \definition[个]{s.}{corredor; passagem | passagem (em edifício de vários andares)}
\end{entry}

\begin{entry}{楼房}{lou2 fang2}{13,8}{⽊、⼾}[HSK 6]
  \definition[栋,幢,座,套,层]{s.}{um edifício de dois ou mais andares}
\end{entry}

\begin{entry}{楼上}{lou2 shang4}{13,3}{⽊、⼀}[HSK 1]
  \definition{s.}{no andar de cima | autor anterior em um tópico do fórum; em plataformas como fóruns na internet, refere-se à pessoa que se manifesta antes de você.}
\end{entry}

\begin{entry}{楼梯}{lou2 ti1}{13,11}{⽊、⽊}[HSK 4]
  \definition[个]{s.}{escada; escadaria; degraus no meio de dois andares para permitir que as pessoas subam ou desçam as escadas}
\end{entry}

\begin{entry}{楼下}{lou2 xia4}{13,3}{⽊、⼀}[HSK 1]
  \definition{s.}{no andar de baixo}
\end{entry}

\begin{entry}{漏}{lou4}{14}{⽔}[HSK 5]
  \definition{s.}{relógio de água; ampulheta | falha; ponto fraco | gonorreia; a medicina tradicional chinesa refere-se a certas doenças que causam secreção de pus, sangue e muco | unidade de tempo medida por um relógio de água durante a noite}
  \definition{v.}{(líquido, gás, etc.) pingar; vazar; escorrer; cair (de um buraco ou fenda) | vazar; deixar escapar; divulgar | perder; deixar de fora por engano | vazar; o objeto tem poros e pode vazar coisas | há uma fuga de ar}
\end{entry}

\begin{entry}{漏电}{lou4dian4}{14,5}{⽔、⽥}
  \definition{v.}{vazar eletricidade}
\end{entry}

\begin{entry}{漏洞}{lou4 dong4}{14,9}{⽔、⽔}[HSK 5]
  \definition[个]{s.}{vazamento; rachadura; lacunas ou buracos desnecessários que permitem que coisas vazem | falha; defeito; lacuna; (fala, ação, método, etc.) imperfeições}
\end{entry}

\begin{entry}{露}{lou4}{21}{⾬}[HSK 6]
  \definition{v.}{mostrar; apresentar (uma certa emoção ou olhar no rosto) | mostrar; aparentar; fazer algo visível; as pessoas podem ver}
  \seeref{露}{lu4}
\end{entry}

\begin{entry}{卢}{lu2}{5}{⼘}
  \definition*{s.}{Luxemburgo, abreviação de 卢森堡 | Sobrenome Lu}
  \definition{s.}{Aarcaico: preta (cor)}
  \seealsoref{卢森堡}{lu2sen1bao3}
\end{entry}

\begin{entry}{卢森堡}{lu2sen1bao3}{5,12,12}{⼘、⽊、⼟}
  \definition*{s.}{Luxemburgo}
\end{entry}

\begin{entry}{卢旺达}{lu2wang4da2}{5,8,6}{⼘、⽇、⾡}
  \definition*{s.}{Ruanda}
\end{entry}

\begin{entry}{芦}{lu2}{7}{⾋}
  \definition*{s.}{Sobrenome Lu}
  \definition{s.}{junco}
\end{entry}

\begin{entry}{芦笋}{lu2sun3}{7,10}{⾋、⽵}
  \definition{s.}{aspargos}
\end{entry}

\begin{entry}{陆}{lu4}{7}{⾩}
  \definition*{s.}{Sobrenome Lu}
  \definition[个]{s.}{terra; terreno | rota terrestre; por terra}
  \seeref{陆}{liu4}
\end{entry}

\begin{entry}{陆地}{lu4di4}{7,6}{⾩、⼟}[HSK 4]
  \definition[块,片]{s.}{terra; terra seca (em oposição ao mar); superfície da Terra, excluindo os oceanos (e, às vezes, rios e lagos)}
\end{entry}

\begin{entry}{陆军}{lu4 jun1}{7,6}{⾩、⼍}[HSK 6]
  \definition{s.}{força terrestre; exército}
\end{entry}

\begin{entry}{陆路}{lu4lu4}{7,13}{⾩、⾜}
  \definition{s.}{rota terrestre}
\end{entry}

\begin{entry}{陆续}{lu4xu4}{7,11}{⾩、⽷}[HSK 4]
  \definition{adv.}{sucessivamente; um após o outro; intermitentemente}
\end{entry}

\begin{entry}{录}{lu4}{8}{⼹}[HSK 3]
  \definition{s.}{registro; cadastro; coleção; seleções}
  \definition{v.}{copiar; gravar; escrever; copiar; registrar | contratar; selecionar; empregar; adotar ou nomear | gravar em fita magnética}
\end{entry}

\begin{entry}{录取}{lu4qu3}{8,8}{⼹、⼜}[HSK 4]
  \definition{v.}{aceitar; admitir; recrutar; entrar; matricular (os aprovados no exame)}
\end{entry}

\begin{entry}{录像}{lu4 xiang4}{8,13}{⼹、⼈}[HSK 6]
  \definition[段,个,些,盘]{s.}{vídeo; gravação; fita de vídeo; imagens gravadas com celulares, câmeras, etc.}
  \definition{v.+compl.}{gravar bídeo; gravar em fita de vídeo | usar celulares, câmeras e outros dispositivos para salvar registros de vídeo}
\end{entry}

\begin{entry}{录像带}{lu4xiang4dai4}{8,13,9}{⼹、⼈、⼱}
  \definition[盘]{s.}{video-cassete}
\end{entry}

\begin{entry}{录像机}{lu4xiang4ji1}{8,13,6}{⼹、⼈、⽊}
  \definition[台]{s.}{gravador de vídeo | VCR}
\end{entry}

\begin{entry}{录音}{lu4yin1}{8,9}{⼹、⾳}[HSK 3]
  \definition[段,个]{s.}{gravação de som; som gravado com equipamento especializado}
  \definition{v.+compl.}{gravar; converter o som em sinal elétrico e, em seguida, gravá-lo por meios mecânicos, ópticos ou eletromagnéticos}
\end{entry}

\begin{entry}{录音机}{lu4 yin1 ji1}{8,9,6}{⼹、⾳、⽊}[HSK 6]
  \definition[台]{s.}{gravador de som; máquina de gravação (de fita)}
\end{entry}

\begin{entry}{鹿}{lu4}{11}{⿅}[Kangxi 198]
  \definition{s.}{cervo | veado}
\end{entry}

\begin{entry}{路}{lu4}{13}{⾜}[HSK 1]
  \definition*{s.}{Sobrenome Lu}
  \definition{clas.}{tipo; classe | linha; coluna; usado para um grupo de pessoas ou uma equipe; para organizar em ordem}
  \definition[条]{s.}{estrada; caminho; via | viagem; jornada; distância | maneira; meios | sequência; linha; lógica | região; distrito | rota | classe; classificação; grau | linha; fileira}
\end{entry}

\begin{entry}{路边}{lu4 bian1}{13,5}{⾜、⾡}[HSK 2]
  \definition{s.}{calçada; beira da estrada; margem da rua}
\end{entry}

\begin{entry}{路过}{lu4 guo4}{13,6}{⾜、⾡}[HSK 6]
  \definition{v.}{passar por (algum lugar); atravessar}
\end{entry}

\begin{entry}{路口}{lu4 kou3}{13,3}{⾜、⼝}[HSK 1]
  \definition[个]{s.}{cruzamento; intersecção; onde as estradas se encontram}
\end{entry}

\begin{entry}{路上}{lu4 shang5}{13,3}{⾜、⼀}[HSK 1]
  \definition{s.}{na estrada | a caminho; na rota; em processo de mudança de um lugar para outro}
\end{entry}

\begin{entry}{路线}{lu4 xian4}{13,8}{⾜、⽷}[HSK 3]
  \definition[条]{s.}{rota; caminho; linha; a estrada percorrida de um lugar a outro | linha; diretriz (de política, ideologia, campo de trabalho); a via fundamental a seguir em termos ideológicos, políticos ou profissionais}
\end{entry}

\begin{entry}{露}{lu4}{21}{⾬}[HSK 6]
  \definition{adj.}{fora de uma casa, tenda, etc., sem cobertura}
  \definition{s.}{orvalho; gotas de água condensadas | xarope; suco de fruta; bebida destilada de flores, folhas ou frutos}
  \definition{v.}{revelar; expor; mostrar; trair}
  \seeref{露}{lou4}
\end{entry}

\begin{entry}{露珠}{lu4zhu1}{21,10}{⾬、⽟}
  \definition{s.}{orvalho}
\end{entry}

\begin{entry}{乱}{luan4}{7}{⼄}[HSK 3]
  \definition{adj.}{em desordem; em confusão; em desarrumação; sem ordem nem organização | em um estado mental confuso | (de uma sociedade) turbulento; agitado | (de relações sexuais) impróprio; promíscuo}
  \definition{adv.}{aleatoriamente; arbitrariamente; indiscriminadamente; sem restrições; à vontade}
  \definition{s.}{motim; agitação; tumulto; revolta; guerra; calamidade}
  \definition{v.}{confundir; embaralhar; misturar; causar desordem}
\end{entry}

\begin{entry}{伦}{lun2}{6}{⼈}
  \definition*{s.}{Sobrenome Lun}
  \definition{s.}{relações humanas (especialmente como concebidas pela ética feudal) | lógica; ordem | par; correspondência; (mesma) classe | ética; relações humanas | sequência lógica; ordem | o mesmo tipo; semelhante; igual}
\end{entry}

\begin{entry}{伦敦}{lun2dun1}{6,12}{⼈、⽁}
  \definition*{s.}{Londres}
\end{entry}

\begin{entry}{论}{lun2}{6}{⾔}
  \definition*{s.}{Os Analectos de Confúcio, registro dos ditos e feitos de Confúcio e seus discípulos}
  \seeref{论}{lun4}
\end{entry}

\begin{entry}{轮}{lun2}{8}{⾞}[HSK 4]
  \definition{clas.}{para sol vermelho, lua brilhante, etc. | para rodadas | doze anos de idade (os doze ramos terrestres são usados para lembrar o gênero humano e cada doze anos de idade é um ciclo)}
  \definition{s.}{roda | anel; disco; objeto semelhante a uma roda | navio a vapor; barco a vapor}
  \definition{v.}{revezar; substituir um ao outro em sequência (para fazer algo)}
\end{entry}

\begin{entry}{轮船}{lun2chuan2}{8,11}{⾞、⾈}[HSK 4]
  \definition[艘]{s.}{navio}
\end{entry}

\begin{entry}{轮回}{lun2hui2}{8,6}{⾞、⼞}
  \definition[个]{s.}{reencarnação (Budismo) | ciclo}
  \definition{v.}{reencarnar}
\end{entry}

\begin{entry}{轮椅}{lun2 yi3}{8,12}{⾞、⽊}[HSK 4]
  \definition{s.}{cadeira de rodas; dispositivo de assento especialmente projetado com rodas para pessoas com dificuldade de locomoção, que pode ser acionado por um disco de roda ou manivela operados manualmente}
\end{entry}

\begin{entry}{轮子}{lun2 zi5}{8,3}{⾞、⼦}[HSK 4]
  \definition[个]{s.}{roda; peças circulares de veículos ou máquinas com capacidade de rotação}
\end{entry}

\begin{entry}{论}{lun4}{6}{⾔}
  \definition*{s.}{Sobrenome Lun}
  \definition{prep.}{por (uma certa unidade de medida) | de acordo com (um certo sistema ou princípio)}
  \definition{s.}{visão; opinião; declaração | (frequentemente em títulos) dissertação; ensaio; tratado | teoria; doutrina | ideia; palavras ou artigos que analisam e explicam coisas}
  \definition{v.}{discutir; falar sobre; discursar sobre; comentar | mencionar; considerar; falar de | decidir sobre; determinar | decidir sobre a natureza da culpa; punir | argumentar; analisar e explicar coisas | considerar; ponderar; medir; avaliar}
  \seeref{论}{lun2}
\end{entry}

\begin{entry}{论文}{lun4wen2}{6,4}{⾔、⽂}[HSK 4]
  \definition[篇]{s.}{tese; redação; artigo; artigo que discute ou examina uma questão}
\end{entry}

\begin{entry}{罗}{luo2}{8}{⽹}
  \definition*{s.}{Sobrenome Luo}
  \definition{v.}{coletar | juntar | pegar | peneirar}
\end{entry}

\begin{entry}{逻}{luo2}{11}{⾡}
  \definition{s.}{patrulha | (literário) a beira de um riacho de montanha}
  \definition{v.}{patrulhar; fazer rondas}
\end{entry}

\begin{entry}{逻辑}{luo2ji5}{11,13}{⾡、⾞}[HSK 5]
  \definition{s.}{lógica; lei objetiva; a objetividade das leis que regem o desenvolvimento das coisas | lógica; razão; regras para o pensamento | lógica como ciência do raciocínio, do pensamento; disciplina que estuda a lógica}
\end{entry}

\begin{entry}{螺}{luo2}{17}{⾍}
  \definition{s.}{concha em espiral | caracol | búzio}
\end{entry}

\begin{entry}{螺丝}{luo2si1}{17,5}{⾍、⼀}
  \definition{s.}{parafuso}
\end{entry}

\begin{entry}{骆}{luo4}{9}{⾺}
  \definition*{s.}{Sobrenome Luo}
  \definition[只]{s.}{Arcaico: um cavalo branco com crina preta, mencionado em antigos livros chineses}
\end{entry}

\begin{entry}{骆驼}{luo4tuo5}{9,8}{⾺、⾺}
  \definition[峰,匹,头]{s.}{camelo | coloquial: cabeça-dura, idiota}
\end{entry}

\begin{entry}{落}{luo4}{12}{⾋}[HSK 4]
  \definition*{s.}{Sobrenome Luo}
  \definition{s.}{paradeiro; lugar para ficar; local de descanso | assentamento; local de reunião | parte curta; área pequena; refere-se a um pequeno lugar ou área}
  \definition{v.}{cair; cair de uma altura elevada | se abaixar; descer; ir para baixo | abaixar; deixar cair (ou descer); fazer descer | afundar; declinar; descer | ficar para trás; ficar para trás ou ficar de fora | permanecer; fazer uma parada; deixar para trás | cair sobre; repousar com | obter; ter; receber | anotar; escrever no papel | cair em; entrar em; ficar preso}
  \seeref{落}{la4}
  \seeref{落}{lao4}
\end{entry}

\begin{entry}{落后}{luo4hou4}{12,6}{⾋、⼝}[HSK 3]
  \definition{adj.}{atrasado; trabalho em atraso, nível de desenvolvimento ou grau de reconhecimento (em oposição a 进步)}
  \definition{v.}{ficar para trás; ficar atrasado; ficar para trás em relação aos outros durante o avanço ou o progresso do trabalho}
  \seealsoref{进步}{jin4bu4}
\end{entry}

\begin{entry}{落花生}{luo4 hua1 sheng1}{12,7,5}{⾋、⾋、⽣}
  \definition{s.}{amendoim | noz de macaco}
\end{entry}

\begin{entry}{落日}{luo4ri4}{12,4}{⾋、⽇}
  \definition{s.}{pôr do sol}
\end{entry}

\begin{entry}{落实}{luo4shi2}{12,8}{⾋、⼧}[HSK 5]
  \definition{adj.}{sentimento de tranquilidade; (humor) estável; seguro}
  \definition{v.}{implementar; ser praticável; tornar os planos, políticas, medidas, etc. específicos e compreensíveis, de modo a que possam ser realizados | implementar; colocar em prática; pôr em prática significa que os planos, políticas e medidas são específicos e claros, e podem ser realizados}
\end{entry}

\begin{entry}{落汤鸡}{luo4tang1ji1}{12,6,7}{⾋、⽔、⿃}
  \definition{s.}{uma pessoa que parece encharcada e acamada| sofrimento profundo}
\end{entry}

\begin{entry}{驴}{lv2}{7}{⾺}
  \definition[头]{s.}{burro | asno | jumento | jegue}
\end{entry}

\begin{entry}{旅}{lv3}{10}{⽅}
  \definition{adv.}{juntos; conjuntamente}
  \definition[个]{s.}{brigada; unidade organizacional militar, abaixo do nível de divisão e acima do nível de regimento ou batalhão | força; tropas; geralmente se refere aos militares | viajante; passageiro; turista | viagem; jornada | pessoas}
  \definition{v.}{viajar; ficar longe de casa; ir para longe; morar longe de casa}
\end{entry}

\begin{entry}{旅程}{lv3cheng2}{10,12}{⽅、⽲}
  \definition{s.}{jornada | viagem}
\end{entry}

\begin{entry}{旅店}{lv3 dian5}{10,8}{⽅、⼴}[HSK 6]
  \definition[家,个]{s.}{pousada; albergue; hotel}
\end{entry}

\begin{entry}{旅馆}{lv3 guan3}{10,11}{⽅、⾷}[HSK 3]
  \definition[家,个,所]{s.}{pousada; hotel; local comercial destinado ao alojamento de viajantes}
\end{entry}

\begin{entry}{旅客}{lv3 ke4}{10,9}{⽅、⼧}[HSK 2]
  \definition[名,位,个,些]{s.}{viajante; passageiro; as agências de transporte e turismo referem-se às pessoas que viajam}
\end{entry}

\begin{entry}{旅行}{lv3xing2}{10,6}{⽅、⾏}[HSK 2]
  \definition{v.}{viajar; passear; para tratar de assuntos ou passear, ir de um lugar para outro (geralmente se refere a distâncias longas)}
\end{entry}

\begin{entry}{旅行社}{lv3 xing2 she4}{10,6,7}{⽅、⾏、⽰}[HSK 3]
  \definition[家]{s.}{agência de viagens; agência especializada em serviços relacionados a viagens, que providencia hospedagem, transporte e outros serviços para viajantes}
\end{entry}

\begin{entry}{旅游}{lv3you2}{10,12}{⽅、⽔}[HSK 2]
  \definition{v.}{viajar para outros lugares para passear e fazer turismo}
\end{entry}

\begin{entry}{屡}{lv3}{12}{⼫}
  \definition{adv.}{uma e outra vez; repetidamente | frequentemente}
\end{entry}

\begin{entry}{屡次}{lv3ci4}{12,6}{⼫、⽋}
  \definition{adv.}{repetidamente | uma e outra vez | muitas vezes}
\end{entry}

\begin{entry}{律}{lv4}{9}{⼻}
  \definition*{s.}{Sobrenome Lü}
  \definition{s.}{lei; regra; estatuto; regulamento}
  \definition{v.}{restringir; disciplinar; manter sob controle}
\end{entry}

\begin{entry}{律师}{lv4shi1}{9,6}{⼻、⼱}[HSK 4]
  \definition[名,个,位]{s.}{advogado; procurador; profissionais encarregados pelas partes ou nomeados pelo tribunal para auxiliar as partes no litígio, para comparecer ao tribunal para defesa e para tratar de assuntos jurídicos relacionados, de acordo com a lei}
\end{entry}

\begin{entry}{率}{lv4}{11}{⽞}
  \definition{s.}{taxa; razão; proporção; a relação proporcional entre duas grandezas relacionadas}
  \seeref{率}{shuai4}
\end{entry}

\begin{entry}{绿}{lv4}{11}{⽷}[HSK 2]
  \definition*{s.}{Sobrenome Lü}
  \definition{adj.}{verde}
  \definition{v.}{tornar-se verde; ficar verde}
\end{entry}

\begin{entry}{绿茶}{lv4 cha2}{11,9}{⽷、⾋}[HSK 3]
  \definition{s.}{chá verde; chá produzido apenas através dos processos de maturação, enrolamento (ou sem enrolamento) e secagem, sem passar por fermentação, com cor verde-claro}
\end{entry}

\begin{entry}{绿豆}{lv4dou4}{11,7}{⽷、⾖}
  \definition{s.}{vagens}
\end{entry}

\begin{entry}{绿豆芽}{lv4dou4 ya2}{11,7,7}{⽷、⾖、⾋}
  \definition{s.}{broto de feijão verde}
\end{entry}

\begin{entry}{绿化}{lv4 hua4}{11,4}{⽷、⼔}[HSK 6]
  \definition{v.}{tornar verde plantando árvores, flores, etc.; arborizar; reflorestar; plantar árvores, flores e plantas para embelezar o ambiente ou prevenir a erosão do solo}
\end{entry}

\begin{entry}{绿色}{lv4 se4}{11,6}{⽷、⾊}[HSK 2]
  \definition{adj.}{verde; ecológico; sem poluição; em conformidade com os requisitos ambientais}
  \definition{s.}{cor verde}
\end{entry}

\begin{entry}{略}{lve4}{11}{⽥}
  \definition{adv.}{ligeiramente | marginalmente | aproximadamente}
\end{entry}

\begin{entry}{略微}{lve4wei1}{11,13}{⽥、⼻}
  \definition{adv.}{ligeiramente | marginalmente | aproximadamente}
\end{entry}

%%%%% EOF %%%%%

