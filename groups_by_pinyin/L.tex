%%%
%%% L
%%%

\section*{L}\addcontentsline{toc}{section}{L}

\begin{entry}{垃圾}{la1 ji1}{8,6}{⼟、⼟}[HSK 4]
  \definition{adj.}{lixo; inútil, ruim ou prejudicial}
  \definition[个]{s.}{entulho; lixo; refugo; rejeito; resíduo; coisa inútil que é jogada fora; metáfora para alguém ou algo que perdeu seu valor ou serve a um propósito ruim}
\end{entry}

\begin{entry}{垃圾车}{la1ji1che1}{8,6,4}{⼟、⼟、⾞}
  \definition{s.}{caminhão de lixo}
\end{entry}

\begin{entry}{垃圾电邮}{la1ji1dian4you2}{8,6,5,7}{⼟、⼟、⽥、⾢}
  \definition{s.}{\emph{e-mail} de \emph{spam}}
\end{entry}

\begin{entry}{垃圾堆}{la1ji1dui1}{8,6,11}{⼟、⼟、⼟}
  \definition{s.}{depósito de lixo}
\end{entry}

\begin{entry}{垃圾工}{la1ji1gong1}{8,6,3}{⼟、⼟、⼯}
  \definition{s.}{lixeiro | gari}
\end{entry}

\begin{entry}{垃圾食品}{la1ji1shi2pin3}{8,6,9,9}{⼟、⼟、⾷、⼝}
  \definition{s.}{\emph{junk food}}
\end{entry}

\begin{entry}{垃圾筒}{la1ji1tong3}{8,6,12}{⼟、⼟、⽵}
  \definition{s.}{cesto de lixo}
\end{entry}

\begin{entry}{垃圾箱}{la1ji1xiang1}{8,6,15}{⼟、⼟、⾋}
  \definition{s.}{cesto de lixo}
\end{entry}

\begin{entry}{垃圾邮件}{la1ji1you2jian4}{8,6,7,6}{⼟、⼟、⾢、⼈}
  \definition{s.}{\emph{spam}, \emph{e-mail} não solicitado}
\end{entry}

\begin{entry}{拉}{la1}{8}{⼿}[HSK 2]
  \definition{v.}{puxar | arrastar | desenhar | conversar | (coloquial) esvaziar as entranhas}
  \seeref{拉}{la4}
\end{entry}

\begin{entry}{拉开}{la1 kai1}{8,4}{⼿、⼶}[HSK 4]
  \definition{v.}{puxar para abrir; recuar| ampliar; espaçar; distanciar; afastar; separar}
\end{entry}

\begin{entry}{拉拉队}{la1la1dui4}{8,8,4}{⼿、⼿、⾩}
  \definition{s.}{claque | torcida}
\end{entry}

\begin{entry}{拉}{la4}{8}{⼿}
  \definition{s.}{usado em 拉拉蛄 \dpy{la4la4gu3}}
  \seeref{拉}{la1}
  \seeref{拉拉蛄}{la4la4gu3}
\end{entry}

\begin{entry}{拉拉蛄}{la4la4gu3}{8,8,11}{⼿、⼿、⾍}
  \variantof{蝲蝲蛄}
\end{entry}

\begin{entry}{落}{la4}{12}{⾋}
  \definition{v.}{deixar de fora; estar ausente | deixar para trás; esquecer de trazer | ficar para trás (ou cair)}
  \seeref{落}{lao4}
  \seeref{落}{luo4}
\end{entry}

\begin{entry}{蜡烛}{la4zhu2}{14,10}{⾍、⽕}
  \definition[根,支]{s.}{vela | círio | peça, geralmente de cera, que possui um pavio e se utiliza para iluminar}
\end{entry}

\begin{entry}{辣}{la4}{14}{⾟}[HSK 4]
  \definition{adj.}{apimentado; picante; pungente; quente | cruel; implacável; venenoso; vicioso}
  \definition{v.}{queimar; picar; formigar; ter uma irritação picante (boca, nariz ou olhos)}
\end{entry}

\begin{entry}{蝲蝲蛄}{la4la4gu3}{15,15,11}{⾍、⾍、⾍}
  \definition{s.}{grilo toupeira}
\end{entry}

\begin{entry}{来}{lai2}{7}{⽊}[HSK 1]
  \definition{v.}{vir | chegar | trazer}
\end{entry}

\begin{entry}{来不及}{lai2bu5ji2}{7,4,3}{⽊、⼀、⼃}[HSK 4]
  \definition{v.}{ser tarde demais; não ter tempo; não ter tempo suficiente (para fazer algo); não ser possível participar ou se atualizar devido a restrições de tempo}
\end{entry}

\begin{entry}{来到}{lai2 dao4}{7,8}{⽊、⼑}[HSK 1]
  \definition{v.}{chegar | vir}
\end{entry}

\begin{entry}{来得及}{lai2de5ji2}{7,11,3}{⽊、⼻、⼃}[HSK 4]
  \definition{v.}{ainda ter tempo; ser capaz de fazê-lo; ser capaz de fazer algo a tempo; ainda ter tempo de chegar lá ou de se atualizar}
\end{entry}

\begin{entry}{来源}{lai2yuan2}{7,13}{⽊、⽔}[HSK 4]
  \definition{s.}{origem; causa; fonte; tabula rasa (ou seja, o lugar de onde as coisas vêm)}
  \definition{v.}{originar-se; surgir; ter origem; (algo) originar (seguido de ``于'')}
  \seealsoref{于}{yu2}
\end{entry}

\begin{entry}{来自}{lai2zi4}{7,6}{⽊、⾃}[HSK 2]
  \definition{v.}{vir de (um local) | \emph{From:} (cabeçalho de \emph{e -mail})}
\end{entry}

\begin{entry}{赖}{lai4}{13}{⾙}
  \definition*{s.}{sobrenome Lai}
  \definition{v.}{depender | aguentar em um lugar | renegar (promessa) | isolar-se | culpar | colocar a culpa em}
\end{entry}

\begin{entry}{兰花}{lan2hua1}{5,7}{⼋、⾋}
  \definition{s.}{orquídea}
\end{entry}

\begin{entry}{蓝}{lan2}{13}{⾋}[HSK 2]
  \definition*{s.}{sobrenome Lan}
  \definition{adj.}{azul}
\end{entry}

\begin{entry}{蓝色}{lan2 se4}{13,6}{⾋、⾊}[HSK 2]
  \definition{s.}{cor azul}
\end{entry}

\begin{entry}{篮球}{lan2qiu2}{16,11}{⽵、⽟}[HSK 2]
  \definition[个,只]{s.}{basquetebol}
\end{entry}

\begin{entry}{懒}{lan3}{16}{⼼}
  \definition{adj.}{preguiçoso | indolente | vadio}
\end{entry}

\begin{entry}{懒虫}{lan3chong2}{16,6}{⼼、⾍}
  \definition{s.}{desleixado ocioso | (insulto) sujeito preguiçoso}
\end{entry}

\begin{entry}{懒怠}{lan3dai4}{16,9}{⼼、⼼}
  \definition{s.}{preguiça}
\end{entry}

\begin{entry}{懒得}{lan3de5}{16,11}{⼼、⼻}
  \definition{adv.}{demasiado preguiçoso}
  \definition{v.}{não sentir vontade (de fazer algo)}
\end{entry}

\begin{entry}{懒惰}{lan3duo4}{16,12}{⼼、⼼}
  \definition{adj.}{preguiçoso}
\end{entry}

\begin{entry}{懒鬼}{lan3gui3}{16,9}{⼼、⿁}
  \definition{s.}{cara preguiçoso}
\end{entry}

\begin{entry}{懒汉}{lan3han4}{16,5}{⼼、⽔}
  \definition{s.}{sujeito ocioso | vagabundo | preguiçosos}
\end{entry}

\begin{entry}{懒人}{lan3ren2}{16,2}{⼼、⼈}
  \definition{s.}{pessoa preguiçosa}
\end{entry}

\begin{entry}{懒散}{lan3san3}{16,12}{⼼、⽁}
  \definition{adj.}{inativo | indolente | preguiçoso | negligente}
\end{entry}

\begin{entry}{懒腰}{lan3yao1}{16,13}{⼼、⾁}
  \definition[个]{s.}{alongamento (do corpo)}
\end{entry}

\begin{entry}{浪费}{lang4fei4}{10,9}{⽔、⾙}[HSK 3]
  \definition{adj.}{desperdiçado}
  \definition{adv.}{extravagantemente}
  \definition{v.}{desperdiçar; dissipar; esbanjar; ser extravagante}
\end{entry}

\begin{entry}{浪花}{lang4hua1}{10,7}{⽔、⾋}
  \definition[朵]{s.}{\emph{spray} | \emph{spray} do oceano | (figurativo) acontecimentos de sua vida}
\end{entry}

\begin{entry}{浪漫}{lang4man4}{10,14}{⽔、⽔}
  \definition{adj.}{romântico}
\end{entry}

\begin{entry}{捞}{lao1}{10}{⼿}
  \definition{v.}{pescar | dragar}
\end{entry}

\begin{entry}{劳工同事}{lao2gong1 tong2shi4}{7,3,6,8}{⼒、⼯、⼝、⼅}
  \definition{s.}{colaborador | colega de trabalho}
\end{entry}

\begin{entry}{老}{lao3}{6}{⽼}[HSK 1,2][Kangxi 125]
  \definition{adj.}{velho (pessoa) | venerável (pessoa) | experiente | ultrapassado | duro (carne, etc.)}
  \definition{adv.}{de longa data | sempre | o tempo todo | do passado}
  \definition{pref.}{prefixo de nome: Lao}
\end{entry}

\begin{entry}{老百姓}{lao3bai3xing4}{6,6,8}{⽼、⽩、⼥}[HSK 3]
  \definition[个]{s.}{povo(s); civis; pessoas comuns; pessoas ordinárias}
\end{entry}

\begin{entry}{老板}{lao3ban3}{6,8}{⽼、⽊}[HSK 3]
  \definition[个,位]{s.}{chefe; dono | tratamento respeitoso a uma estrela de ópera ou a um líder de trupe}
\end{entry}

\begin{entry}{老兵}{lao3bing1}{6,7}{⽼、⼋}
  \definition{s.}{velho soldado | veterano de guerra | veterano (alguém que tem muita experiência em algum domínio)}
\end{entry}

\begin{entry}{老公}{lao3 gong1}{6,4}{⽼、⼋}[HSK 4]
  \definition[个]{s.}{marido; esposo}
\end{entry}

\begin{entry}{老虎}{lao3hu3}{6,8}{⽼、⾌}
  \definition[只]{s.}{tigre}
  \seealsoref{虎}{hu3}
\end{entry}

\begin{entry}{老家}{lao3 jia1}{6,10}{⽼、⼧}[HSK 4]
  \definition{s.}{cidade natal; local de origem | lugar nativo; refere-se às gerações anteriores da família ou ao local onde a pessoa nasceu ou viveu}
\end{entry}

\begin{entry}{老年}{lao3 nian2}{6,6}{⽼、⼲}[HSK 2]
  \definition{s.}{idoso | velhice}
\end{entry}

\begin{entry}{老朋友}{lao3 peng2 you3}{6,8,4}{⽼、⽉、⼜}[HSK 2]
  \definition[个]{s.}{velho amigo}
\end{entry}

\begin{entry}{老婆}{lao3po2}{6,11}{⽼、⼥}[HSK 4]
  \definition[个]{s.}{esposa}
\end{entry}

\begin{entry}{老人}{lao3 ren2}{6,2}{⽼、⼈}[HSK 1]
  \definition[位]{s.}{pessoa idosa | o idoso | o velho}
\end{entry}

\begin{entry}{老人家}{lao3 ren2 jia1}{6,2,10}{⽼、⼈、⼧}
  \definition{s.}{senhor ancião | madame | senhora | termo educado para mulher ou homem velho}
\end{entry}

\begin{entry}{老师}{lao3shi1}{6,6}{⽼、⼱}[HSK 1]
  \definition[个,位]{s.}{professor}
\end{entry}

\begin{entry}{老是}{lao3 shi4}{6,9}{⽼、⽇}[HSK 2]
  \definition{adv.}{sempre | todas as vezes}
\end{entry}

\begin{entry}{老实}{lao3shi5}{6,8}{⽼、⼧}[HSK 4]
  \definition{adj.}{franco; sincero; honesto | bom; bem-comportado | ingênuo; simplório; meio bobo; facilmente enganado; eufemismo para pouco inteligente}
\end{entry}

\begin{entry}{老太太}{lao3 tai4 tai5}{6,4,4}{⽼、⼤、⼤}[HSK 3]
  \definition[位]{s.}{velha senhora; (em tratamento direto)Venerável Senhora; uma maneira respeitosa de chamar uma senhora idosa | (forma de tratamento) sua velha mãe; minha velha mãe, avó ou sogra}
\end{entry}

\begin{entry}{老头儿}{lao3 tou2r5}{6,5,2}{⽼、⼤、⼉}[HSK 3]
  \definition{s.}{(coloquial) velho; velho rabugento}
  \seealsoref{老头子}{lao3 tou2zi5}
\end{entry}

\begin{entry}{老头子}{lao3 tou2zi5}{6,5,3}{⽼、⼤、⼦}
  \definition{s.}{(coloquial) velho; velho rabugento}
  \seealsoref{老头儿}{lao3 tou2r5}
\end{entry}

\begin{entry}{落}{lao4}{12}{⾋}
  \definition{v.}{cair | descer | ficar; fazer escala; deixar para trás | obter; ter; receber}
  \seeref{落}{la4}
  \seeref{落}{lao4}
\end{entry}

\begin{entry}{乐}{le4}{5}{⼃}[HSK 3]
  \definition*{s.}{sobrenome Le}
  \definition{adj.}{feliz; contente; rejubilante; animado; bem disposto}
  \definition{adv.}{alegremente; felizmente; desejosamente}
  \definition{s.}{prazer; diversão}
  \definition{v.}{desfrutar; ficar feliz em; amar; encontrar prazer em
rir; divertir-se}
  \seeref{乐}{yue4}
\end{entry}

\begin{entry}{乐高}{le4gao1}{5,10}{⼃、⾼}
  \definition*{s.}{Lego (brinquedo)}
\end{entry}

\begin{entry}{乐观}{le4guan1}{5,6}{⼃、⾒}[HSK 3]
  \definition{adj.}{esperançoso; otimista; de um vermelho intenso}
\end{entry}

\begin{entry}{乐趣}{le4qu4}{5,15}{⼃、⾛}[HSK 4]
  \definition[个,种,些,点]{s.}{alegria; deleite; prazer; implicação de fazer alguém se sentir feliz; um humor de preferência}
\end{entry}

\begin{entry}{乐园}{le4yuan2}{5,7}{⼃、⼞}
  \definition{s.}{paraíso}
\end{entry}

\begin{entry}{了}{le5}{2}{⼅}[HSK 1]
  \definition{part.}{usada depois de verbos ou adjetivos para indicar que uma ação ou mudança foi concluída | usada no final de uma frase ou em uma pausa na frase, indica uma mudança, significa o surgimento de uma nova situação e expressa uma insistência ou um conselho contra algo}
  \seeref{了}{liao3}
  \seeref{了}{liao4}
\end{entry}

\begin{entry}{累}{lei2}{11}{⽷}
  \definition*{s.}{sobrenome Lei}
  \definition{s.}{corda}
  \definition{v.}{amarrar | torcer}
  \seeref{累}{lei3}
  \seeref{累}{lei4}
\end{entry}

\begin{entry}{雷电}{lei2dian4}{13,5}{⾬、⽥}
  \definition{s.}{trovão e relâmpago; raio}
\end{entry}

\begin{entry}{雷亚尔}{lei2ya4'er3}{13,6,5}{⾬、⼆、⼩}
  \definition*{s.}{Real Brasileiro}
\end{entry}

\begin{entry}{累}{lei3}{11}{⽷}
  \definition{adj.}{contínuo | repetido}
  \definition{v.}{acumular | envolver ou implicar}
  \seeref{累}{lei2}
  \seeref{累}{lei4}
\end{entry}

\begin{entry}{絫}{lei3}{12}{⽷}
  \variantof{累}
\end{entry}

\begin{entry}{泪}{lei4}{8}{⽔}[HSK 4]
  \definition[滴]{s.}{lágrima}
\end{entry}

\begin{entry}{泪水}{lei4 shui3}{8,4}{⽔、⽔}[HSK 4]
  \definition{s.}{lágrima}
\end{entry}

\begin{entry}{类}{lei4}{9}{⽶}[HSK 3]
  \definition*{s.}{sobrenome Lei}
  \definition{s.}{classe; categoria; tipo; espécie}
  \definition{v.}{assemelhar-se a; ser semelhante a}
\end{entry}

\begin{entry}{类似}{lei4si4}{9,6}{⽶、⼈}[HSK 3]
  \definition{adj.}{semelhante; análogo}
\end{entry}

\begin{entry}{类型}{lei4xing2}{9,9}{⽶、⼟}[HSK 4]
  \definition[种,个]{s.}{tipo; espécie; categoria; tipos formados por coisas com características comuns}
\end{entry}

\begin{entry}{累}{lei4}{11}{⽷}[HSK 1]
  \definition{adj.}{cansado | fatigado}
  \definition{v.}{forçar | desgastar | trabalhar duro}
  \seeref{累}{lei2}
  \seeref{累}{lei3}
\end{entry}

\begin{entry}{冷}{leng3}{7}{⼎}[HSK 1]
  \definition*{s.}{sobrenome Leng}
  \definition{adj.}{frio}
\end{entry}

\begin{entry}{冷静}{leng3jing4}{7,14}{⼎、⾭}[HSK 4]
  \definition{adj.}{calmo; descreve uma pessoa que consegue ficar atenta em uma situação importante ou de emergência e não toma decisões aleatórias por causa de seus sentimentos no momento | (lugar) tranquilo; quieto; deserto}
\end{entry}

\begin{entry}{厘米}{li2mi3}{9,6}{⼚、⽶}[HSK 4]
  \definition{clas.}{centímetro; unidade de comprimento, símbolo cm, 1 metro é igual a 100 centímetros}
\end{entry}

\begin{entry}{离}{li2}{10}{⼇}[HSK 2]
  \definition*{s.}{sobrenome Li}
  \definition{prep.}{(ser longe) de\dots até\dots}
  \definition{v.}{ficar longe de | deixar | separar-se de}
\end{entry}

\begin{entry}{离不开}{li2 bu4 kai1}{10,4,4}{⼇、⼀、⼶}[HSK 4]
  \definition{v.}{não pode prescindir; ser inseparável de; não ser capaz de se separar ou deixar uma pessoa, coisa ou circunstância}
\end{entry}

\begin{entry}{离婚}{li2hun1}{10,11}{⼇、⼥}[HSK 3]
  \definition{v.+compl.}{divórciar; romper um casamento; obter o divórcio}
\end{entry}

\begin{entry}{离开}{li2kai1}{10,4}{⼇、⼶}[HSK 2]
  \definition{v.}{partir| deixar}
\end{entry}

\begin{entry}{黎}{li2}{15}{⿉}
  \definition*{s.}{a nacionalidade Li, uma das minorias nacionais da província de Hainan | sobrenome Li}
  \definition{adj.}{numeroso}
  \definition{s.}{multidão}
\end{entry}

\begin{entry}{礼节}{li3jie2}{5,5}{⽰、⾋}
  \definition{s.}{protocolo | cerimônia | etiqueta}
\end{entry}

\begin{entry}{礼让}{li3rang4}{5,5}{⽰、⾔}
  \definition{s.}{cortesia}
  \definition{v.}{mostrar consideração por (outros) | ceder a (outro veículo, etc.)}
\end{entry}

\begin{entry}{礼物}{li3wu4}{5,8}{⽰、⽜}[HSK 2]
  \definition[件,个,份]{s.}{prenda | lembrança | presente}
\end{entry}

\begin{entry}{李}{li3}{7}{⽊}
  \definition*{s.}{sobrenome Li}
  \definition{s.}{ameixa}
\end{entry}

\begin{entry}{李四}{li3si4}{7,5}{⽊、⼞}
  \definition*{s.}{Li Si | Zé Ninguém | nome para uma pessoa não especificada, 2 de 3}
  \seealsoref{王五}{wang2wu3}
  \seealsoref{张三}{zhang1san1}
\end{entry}

\begin{entry}{李子}{li3zi5}{7,3}{⽊、⼦}
  \definition[个]{s.}{ameixa}
\end{entry}

\begin{entry}{里}{li3}{7}{⾥}[HSK 1][Kangxi 166]
  \definition*{s.}{sobrenome Li}
  \definition{adv.}{em | dentro | interior | interno}
  \definition{s.}{vizinhança | bairro | li, medida antiga de comprimento, aproximadamente 500m | unidade administrativa antiga de 25 famílias}
\end{entry}

\begin{entry}{里边}{li3 bian5}{7,5}{⾥、⾡}[HSK 1]
  \definition{prep.}{em | dentro}
\end{entry}

\begin{entry}{里面}{li3 mian4}{7,9}{⾥、⾯}[HSK 3]
  \definition{s.}{dentro; interior}
\end{entry}

\begin{entry}{里斯本}{li3si1ben3}{7,12,5}{⾥、⽄、⽊}
  \definition*{s.}{Lisboa}
\end{entry}

\begin{entry}{里斯本大学}{li3si1ben3 da4xue2}{7,12,5,3,8}{⾥、⽄、⽊、⼤、⼦}
  \definition*{s.}{Universidade de Lisboa}
\end{entry}

\begin{entry}{里头}{li3 tou5}{7,5}{⾥、⼤}[HSK 2]
  \definition{s.}{dentro}
\end{entry}

\begin{entry}{理发}{li3fa4}{11,5}{⽟、⼜}[HSK 3]
  \definition{v.+compl.}{cortar o cabelo; ter (dar) um corte de cabelo}
\end{entry}

\begin{entry}{理解}{li3jie3}{11,13}{⽟、⾓}[HSK 3]
  \definition{v.}{entender; compreender | entender com empatia}
\end{entry}

\begin{entry}{理论}{li3lun4}{11,6}{⽟、⾔}[HSK 3]
  \definition[套,个]{s.}{teoria}
  \definition{v.}{argumentar; raciocinar com alguém}
\end{entry}

\begin{entry}{理想}{li3xiang3}{11,13}{⽟、⼼}[HSK 2]
  \definition[个]{adj./s.}{ideal}
  \definition{adv.}{idealmente}
\end{entry}

\begin{entry}{理由}{li3you2}{11,5}{⽟、⽥}[HSK 3]
  \definition[个]{s.}{razão; justificativa; fundamento}
\end{entry}

\begin{entry}{力}{li4}{2}{⼒}[HSK 3][Kangxi 19]
  \definition*{s.}{sobrenome Li}
  \definition{adv.}{energicamente; arduamente; vigorosamente}
  \definition{s.}{poder; força; habilidade; capacidade | força; energia; poder | força física}
  \definition{v.}{fazer tudo o que puder; fazer todo o esforço}
\end{entry}

\begin{entry}{力量}{li4liang5}{2,12}{⼒、⾥}[HSK 3]
  \definition[出]{s.}{força física; força espiritual | habilidade; capacidade | eficácia; efeito | força (pessoa ou grupo que tem muito poder ou influência)}
\end{entry}

\begin{entry}{力气}{li4qi5}{2,4}{⼒、⽓}[HSK 4]
  \definition[点,把]{s.}{força física | esforço}
\end{entry}

\begin{entry}{历史}{li4shi3}{4,5}{⼚、⼝}[HSK 4]
  \definition[个,门,段]{s.}{histórico; registro do passado; processo de desenvolvimento da natureza e da sociedade humana; processo de desenvolvimento de uma coisa ou pessoa | história; eventos passados; experiência | história; refere-se ao tema da história}
\end{entry}

\begin{entry}{厉害}{li4hai5}{5,10}{⼚、⼧}
  \definition{adj.}{severo | rigoroso | exigente | radical | violento | feroz}
\end{entry}

\begin{entry}{立法}{li4fa3}{5,8}{⽴、⽔}
  \definition{s.}{legislação}
  \definition{v.}{promulgar leis | legislar}
\end{entry}

\begin{entry}{立即}{li4ji2}{5,7}{⽴、⼙}[HSK 4]
  \definition{adv.}{prontamente; imediatamente; de imediato}
\end{entry}

\begin{entry}{立刻}{li4ke4}{5,8}{⽴、⼑}[HSK 3]
  \definition{adv.}{imediatamente; de ​​uma vez}
\end{entry}

\begin{entry}{利息}{li4xi1}{7,10}{⼑、⼼}[HSK 4]
  \definition{s.}{acréscimo; juros; dinheiro recebido além do valor principal como resultado de depósitos ou empréstimos (diferenciado de ``本金'')}
  \seealsoref{本金}{ben3 jin1}
\end{entry}

\begin{entry}{利益}{li4yi4}{7,10}{⼑、⽫}[HSK 4]
  \definition[个,种]{s.}{ganho; lucro; juros; benefício}
\end{entry}

\begin{entry}{利用}{li4yong4}{7,5}{⼑、⽤}[HSK 3]
  \definition{v.}{usar; utilizar; fazer uso de | explorar; tirar vantagem de}
\end{entry}

\begin{entry}{例如}{li4ru2}{8,6}{⼈、⼥}[HSK 2]
  \definition{conj.}{por exemplo | como}
\end{entry}

\begin{entry}{例子}{li4 zi5}{8,3}{⼈、⼦}[HSK 2]
  \definition[个]{s.}{exemplo}
\end{entry}

\begin{entry}{隶}{li4}{8}{⾪}[Kangxi 171]
  \definition*{s.}{sobrenome Li}
  \definition{s.}{escravo; uma pessoa em servidão; uma pessoa humilde | um dos estilos antigos da caligrafia chinesa}
  \definition{v.}{estar subordinado a}
\end{entry}

\begin{entry}{荔枝}{li4zhi1}{9,8}{⾋、⽊}
  \definition{s.}{lichia}
\end{entry}

\begin{entry}{鬲}{li4}{10}{⿀}[Kangxi 193]
  \definition{s.}{um antigo tripé de cozinha com pernas ocas; uma grande panela de barro}
  \seeref{鬲}{ge2}
\end{entry}

\begin{entry}{詈骂}{li4ma4}{12,9}{⾔、⾺}
  \definition{v.}{xingar | abusar}
\end{entry}

\begin{entry}{俩}{lia3}{9}{⼈}[HSK 4]
  \definition{num.}{ambos; dois; contração de ``两个'' | alguns; vários; refere-se a um pequeno número}
\end{entry}

\begin{entry}{俩钱}{lia3qian2}{9,10}{⼈、⾦}
  \definition{s.}{uma pequena quantia de dinheiro}
\end{entry}

\begin{entry}{连}{lian2}{7}{⾡}[HSK 3]
  \definition*{s.}{sobrenome Lian}
  \definition{adv.}{em sucessão; um após o outro; repetidamente | até}
  \definition{prep.}{incluindo}
  \definition{s.}{companhia | conjunção}
  \definition{v.}{ligar; juntar; conectar | envolver (em problemas); implicar | costurar; coser}
\end{entry}

\begin{entry}{连忙}{lian2mang2}{7,6}{⾡、⼼}[HSK 3]
  \definition{adv.}{prontamente; imediatamente; apressadamente}
\end{entry}

\begin{entry}{连锁反应}{lian2suo3fan3ying4}{7,12,4,7}{⾡、⾦、⼜、⼴}
  \definition{s.}{reação em cadeia}
\end{entry}

\begin{entry}{连续}{lian2xu4}{7,11}{⾡、⽷}[HSK 3]
  \definition{adv.}{continuamente; sucessivamente; em uma fileira}
\end{entry}

\begin{entry}{连续剧}{lian2 xu4 ju4}{7,11,10}{⾡、⽷、⼑}[HSK 3]
  \definition{s.}{série; novela}
\end{entry}

\begin{entry}{帘}{lian2}{8}{⼱}
  \definition{s.}{cortina | tela (pendurada) | bandeira usada como placa de loja}
\end{entry}

\begin{entry}{莲花}{lian2hua1}{10,7}{⾋、⾋}
  \definition{s.}{flor de lótus | lírio aquático}
\end{entry}

\begin{entry}{莲藕}{lian2'ou3}{10,18}{⾋、⾋}
  \definition{s.}{raiz de Lotus}
\end{entry}

\begin{entry}{联合}{lian2he2}{12,6}{⽿、⼝}[HSK 3]
  \definition{adj.}{conjunto; unido; federal; combinado}
  \definition{s.}{aliado; união; aliança}
\end{entry}

\begin{entry}{联合国}{lian2 he2 guo2}{12,6,8}{⽿、⼝、⼞}[HSK 3]
  \definition*{s.}{Nações Unidas}
\end{entry}

\begin{entry}{联合会}{lian2he2hui4}{12,6,6}{⽿、⼝、⼈}
  \definition{s.}{federação}
\end{entry}

\begin{entry}{联系}{lian2xi4}{12,7}{⽿、⽷}[HSK 3]
  \definition{s.}{relacionamento; conexão}
  \definition[个,种,层]{v.}{entrar em contato; contatar | organizar; entrar em contato com | relacionar; combinar; integrar}
\end{entry}

\begin{entry}{脸}{lian3}{11}{⾁}[HSK 2]
  \definition[张,个]{s.}{cara | rosto | face}
\end{entry}

\begin{entry}{脸色}{lian3se4}{11,6}{⾁、⾊}
  \definition{s.}{compleição; tez; face}
\end{entry}

\begin{entry}{练}{lian4}{8}{⽷}[HSK 2]
  \definition{s.}{exercício | (literário) seda branca}
  \definition{v.}{praticar | treinar | aperfeiçoar (habilidade) | ferver e esfregar seda crua}
\end{entry}

\begin{entry}{练习}{lian4xi2}{8,3}{⽷、⼄}[HSK 2]
  \definition[个]{s.}{prática | exercício}
  \definition{v.}{praticar | exercitar}
\end{entry}

\begin{entry}{恋爱}{lian4'ai4}{10,10}{⼼、⽖}
  \definition[个,场]{s.}{amor (romântico)}
  \definition{v.}{sentir-se profundamente apegado a}
\end{entry}

\begin{entry}{良好}{liang2hao3}{7,6}{⾉、⼥}[HSK 4]
  \definition{adj.}{bom; ótimo; bem}
\end{entry}

\begin{entry}{良田}{liang2tian2}{7,5}{⾉、⽥}
  \definition{s.}{terra agrícola boa | terra fértil}
\end{entry}

\begin{entry}{良心}{liang2xin1}{7,4}{⾉、⼼}
  \definition{s.}{consciência}
\end{entry}

\begin{entry}{凉}{liang2}{10}{⼎}[HSK 2]
  \definition{adj.}{frio | legal}
  \seeref{凉}{liang4}
\end{entry}

\begin{entry}{凉快}{liang2kuai5}{10,7}{⼎、⼼}[HSK 2]
  \definition{adj.}{agradável e frio | agradavelmente fresco}
\end{entry}

\begin{entry}{凉水}{liang2 shui3}{10,4}{⼎、⽔}[HSK 3]
  \definition{s.}{água fria | água gelada | água não fervida}
\end{entry}

\begin{entry}{凉鞋}{liang2xie2}{10,15}{⼎、⾰}
  \definition{s.}{sandália | alpargata | alpercata | alparca}
\end{entry}

\begin{entry}{量}{liang2}{12}{⾥}[HSK 4]
  \definition{v.}{medir | estimar; dimensionar}
  \seeref{量}{liang4}
\end{entry}

\begin{entry}{粮食}{liang2shi5}{13,9}{⽶、⾷}[HSK 4]
  \definition[种,斤,吨,袋]{s.}{alimentos; grãos; termo geral para os vários tipos de arroz, feijão, etc. que podem ser consumidos}
\end{entry}

\begin{entry}{两}{liang3}{7}{⼀}[HSK 1,2]
  \definition{adv.}{ambos (lados) | cada (lado)}
  \definition{clas.}{liang, uma unidade de peso (=50 gramas)}
  \definition{num.}{dois (sempre usado antes de classificadores) | poucos; alguns}
\end{entry}

\begin{entry}{两边}{liang3 bian1}{7,5}{⼀、⾡}[HSK 4]
  \definition{s.}{ambos os lados; ambas as direções; ambos os lugares | ambas as partes; ambos os lados}
\end{entry}

\begin{entry}{两码事}{liang3ma3shi4}{7,8,8}{⼀、⽯、⼅}
  \definition{expr.}{duas coisas completamente diferentes}
\end{entry}

\begin{entry}{亮}{liang4}{9}{⼇}[HSK 2]
  \definition*{s.}{sobrenome Lian}
  \definition{adj.}{brilhante | alto e claro | retumbante | iluminado | aberto e claro}
  \definition{s.}{luz}
  \definition{v.}{iluminar | brilhar | elevar a voz | ressoar | revelar | mostrar | aparecer | exibir}
\end{entry}

\begin{entry}{凉}{liang4}{10}{⼎}
  \definition{v.}{esfriar | tornar ou tornar-se frio | deixar esfriar pelo ar}
  \seeref{凉}{liang2}
\end{entry}

\begin{entry}{辆}{liang4}{11}{⾞}[HSK 2]
  \definition{clas.}{para automóveis, veículos, etc.}
\end{entry}

\begin{entry}{量}{liang4}{12}{⾥}
  \definition{s.}{instrumento de medida; antigamente, o termo se referia a objetos como baldes e litros, que medem o volume | capacidade (para tolerância ou ingestão de alimentos ou bebidas); refere-se ao limite do que pode ser acomodado | quantidade; valor; volume; número}
  \definition{v.}{estimar; medir; pesar}
  \seeref{量}{liang2}
\end{entry}

\begin{entry}{疗养}{liao2 yang3}{7,9}{⽧、⼋}[HSK 4]
  \definition{v.}{recuperar; convalescer; tratar pessoas com doenças crônicas ou debilitantes em instituições médicas especializadas com foco na recuperação}
\end{entry}

\begin{entry}{聊天}{liao2tian1}{11,4}{⽿、⼤}
  \definition{v.+compl.}{papear | bater papo}
\end{entry}

\begin{entry}{了}{liao3}{2}{⼅}
  \definition{v.}{terminar | alcançar | entender claramente}
  \seeref{了}{le5}
  \seeref{了}{liao4}
\end{entry}

\begin{entry}{了不起}{liao3bu5qi3}{2,4,10}{⼅、⼀、⾛}[HSK 4]
  \definition{adj.}{incrível; fantástico; extraordinário | sério; grave}
\end{entry}

\begin{entry}{了解}{liao3jie3}{2,13}{⼅、⾓}[HSK 4]
  \definition{v.}{entender; compreender | investigar; indagar sobre}
\end{entry}

\begin{entry}{了}{liao4}{2}{⼅}
  \definition{adj.}{brilhantes (olhos)}
  \definition{v.}{observar | olhar para fora | olhar para baixo de um lugar mais alto | compreender claramente}
  \seeref{了}{le5}
  \seeref{了}{liao3}
\end{entry}

\begin{entry}{列}{lie4}{6}{⼑}[HSK 4]
  \definition{v.}{organizar; formar uma linha; alinhar | listar; inserir em uma lista}
\end{entry}

\begin{entry}{列车}{lie4che1}{6,4}{⼑、⾞}[HSK 4]
  \definition{s.}{trem; trem em uma composição contínua, puxado por uma locomotiva e equipado com uma tripulação e marcações prescritas; geralmente um trem de passageiros}
\end{entry}

\begin{entry}{列入}{lie4 ru4}{6,2}{⼑、⼊}[HSK 4]
  \definition{v.}{incluir em uma lista}
\end{entry}

\begin{entry}{列为}{lie4 wei2}{6,4}{⼑、⼂}[HSK 4]
  \definition{v.}{ser classificado como; ser listado como}
\end{entry}

\begin{entry}{烈士}{lie4shi4}{10,3}{⽕、⼠}
  \definition{s.}{mártir}
\end{entry}

\begin{entry}{猎物}{lie4wu4}{11,8}{⽝、⽜}
  \definition{s.}{presa (vítima de um predador)}
\end{entry}

\begin{entry}{邻居}{lin2ju1}{7,8}{⾢、⼫}
  \definition[个]{s.}{vizinho}
\end{entry}

\begin{entry}{临}{lin2}{9}{⼁}
  \definition*{s.}{sobrenome Lin}
  \definition{adv.}{pouco antes; prestes a; no ponto de}
  \definition{v.}{encarar; enfrentar; aproximar-se | chegar; estar presente | copiar (um modelo de caligrafia ou pintura); traçar sobre as palavras ou figuras | olhar de cima para baixo | ir de cima para baixo}
\end{entry}

\begin{entry}{临时}{lin2shi2}{9,7}{⼁、⽇}[HSK 4]
  \definition{adj.}{temporário; provisório; por um breve período}
  \definition{adv.}{no momento em que algo acontece (quando as coisas dão errado)}
\end{entry}

\begin{entry}{淋}{lin2}{11}{⽔}
  \definition{v.}{borrifar | pingar | derramar | encharcar}
  \seeref{淋}{lin4}
\end{entry}

\begin{entry}{淋}{lin4}{11}{⽔}
  \definition{s.}{gonorréia}
  \definition{v.}{filtrar | coar | drenar}
  \seeref{淋}{lin2}
\end{entry}

\begin{entry}{灵感}{ling2gan3}{7,13}{⽕、⼼}
  \definition{s.}{inspiração | explosão de criatividade em empreendimento científico ou artístico}
\end{entry}

\begin{entry}{灵魂}{ling2hun2}{7,13}{⽕、⿁}
  \definition{s.}{alma | espírito}
\end{entry}

\begin{entry}{陵园}{ling2yuan2}{10,7}{⾩、⼞}
  \definition{s.}{cemitério}
\end{entry}

\begin{entry}{菱角}{ling2jiao5}{11,7}{⾋、⾓}
  \definition{s.}{castanha d'água}
\end{entry}

\begin{entry}{零/〇}{ling2 ling2}{13,13}{⾬、⾬}[HSK 1]
  \definition{adj.}{extra}
  \definition{num.}{zero; 0}
  \definition{s.}{(matemática) resto (após a divisão) | fração | nada}
\end{entry}

\begin{entry}{零食}{ling2shi2}{13,9}{⾬、⾷}[HSK 4]
  \definition[包,袋,盒,箱,堆]{s.}{lanches; refrescos; petiscos entre as refeições; alimentação esporádica, além das refeições normais}
\end{entry}

\begin{entry}{零下}{ling2 xia4}{13,3}{⾬、⼀}[HSK 2]
  \definition{s.}{abaixo de zero}
\end{entry}

\begin{entry}{岭}{ling3}{8}{⼭}
  \definition{s.}{cordilheira}
\end{entry}

\begin{entry}{领}{ling3}{11}{⾴}[HSK 3]
  \definition{adj.}{territorial (sob jurisdição de; em posse de)}
  \definition{clas.}{para roupas, tapetes, telas, etc.}
  \definition{s.}{pescoço; gargalo | colarinho; faixa de pescoço | esboço; ponto principal; essência}
  \definition{v.}{encabeçar; liderar; conduzir | possuir; ser o possuidor de | receber; obter; conseguir | aceitar; tomar |entender; compreender | adotar}
\end{entry}

\begin{entry}{领导}{ling3dao3}{11,6}{⾴、⼨}[HSK 3]
  \definition[个,位]{s.}{líder; liderança}
  \definition{v.}{liderar; exercer liderança}
\end{entry}

\begin{entry}{领情}{ling3qing2}{11,11}{⾴、⼼}
  \definition{v.+compl.}{sentir-se grato a alguém}
\end{entry}

\begin{entry}{领先}{ling3xian1}{11,6}{⾴、⼉}[HSK 3]
  \definition{v.}{liderar; assumir a liderança; estar na liderança}
\end{entry}

\begin{entry}{令人}{ling4ren2}{5,2}{⼈、⼈}
  \definition{v.}{causar alguém (a fazer alguma coisa) | fazer alguém ficar zangado, encantado, etc.}
\end{entry}

\begin{entry}{另外}{ling4wai4}{5,5}{⼝、⼣}[HSK 3]
  \definition{adv.}{além disso; em adição; ademais; além do mais; além de que}
  \definition{pron.}{além disso}
\end{entry}

\begin{entry}{另一方面}{ling4 yi4 fang1 mian4}{5,1,4,9}{⼝、⼀、⽅、⾯}[HSK 3]
  \definition{adv./conj.}{outro aspecto | por outro lado; por sua vez; em contrapartida}
\end{entry}

\begin{entry}{刘}{liu2}{6}{⼑}
  \definition*{s.}{sobrenome Liu}
  \definition{s.}{(clássico) um tipo de machado de batalha}
  \definition{v.}{matar}
\end{entry}

\begin{entry}{流}{liu2}{10}{⽔}[HSK 2]
  \definition[名,个]{s.}{fluxo de água | correnteza | córrego | algo que se assemelha a um fluxo de água | corrente | fluxo | classe | grau | taxa (de variação)}
  \definition{v.}{fluir
deriva; mover; vagar
espalhar
degenerar; mudar para pior
enviar para o exílio; banir}
\end{entry}

\begin{entry}{流传}{liu2chuan2}{10,6}{⽔、⼈}[HSK 4]
  \definition{v.}{espalhar; circular; passar adiante}
\end{entry}

\begin{entry}{流利}{liu2li4}{10,7}{⽔、⼑}[HSK 2]
  \definition{adj.}{fluente (em uma língua)}
\end{entry}

\begin{entry}{流水}{liu2shui3}{10,4}{⽔、⽔}
  \definition{s.}{água corrente | (negócio) rotatividade}
\end{entry}

\begin{entry}{流星}{liu2xing1}{10,9}{⽔、⽇}
  \definition{s.}{meteoro | estrela cadente}
\end{entry}

\begin{entry}{流行}{liu2xing2}{10,6}{⽔、⾏}[HSK 2]
  \definition{adj.}{(estilo de roupa, música, etc.) popular, na moda}
  \definition{v.}{(doença contagiosa, etc.) espalhar | propagar}
\end{entry}

\begin{entry}{留}{liu2}{10}{⽥}[HSK 2]
  \definition{v.}{permanecer | ficar | pedir para alguém ficar | manter alguém onde ele está | concentrar-se em | reservar | manter | salvar | deixar crescer | crescer | vestir | aceitar | tomar | deixar para trás | estudar no exterior}
\end{entry}

\begin{entry}{留神}{liu2shen2}{10,9}{⽥、⽰}
  \definition{v.+compl.}{tomar cuidado | prestar atenção | manter os olhos abertos}
\end{entry}

\begin{entry}{留下}{liu2 xia4}{10,3}{⽥、⼀}[HSK 2]
  \definition{v.}{deixar}
\end{entry}

\begin{entry}{留学}{liu2xue2}{10,8}{⽥、⼦}[HSK 3]
  \definition{v.}{estudar no exterior}
\end{entry}

\begin{entry}{留学生}{liu2 xue2 sheng1}{10,8,5}{⽥、⼦、⽣}[HSK 2]
  \definition[个,位,名,批]{s.}{estudante estrangeiro | estudante estudando no exterior}
\end{entry}

\begin{entry}{柳}{liu3}{9}{⽊}
  \definition*{s.}{sobrenome Liu}
  \definition{s.}{salgueiro}
\end{entry}

\begin{entry}{柳橙汁}{liu3cheng2zhi1}{9,16,5}{⽊、⽊、⽔}
  \definition[瓶,杯,罐,盒]{s.}{suco de laranja}
  \seealsoref{橙汁}{cheng2zhi1}
  \seealsoref{橘子汁}{ju2zi5zhi1}
\end{entry}

\begin{entry}{六}{liu4}{4}{⼋}[HSK 1]
  \definition{num.}{seis; 6}
\end{entry}

\begin{entry}{遛狗}{liu4gou3}{13,8}{⾡、⽝}
  \definition{v.+compl.}{passear com um cachorro}
\end{entry}

\begin{entry}{龙}{long2}{5}{⿓}[HSK 3][Kangxi 212]
  \definition*{s.}{sobrenome Long}
  \definition{adj.}{imperial}
  \definition[条]{s.}{dragão
um enorme réptil extinto}
\end{entry}

\begin{entry}{龙山}{long2shan1}{5,3}{⿓、⼭}
  \definition*{s.}{Longshan}
\end{entry}

\begin{entry}{龙虾}{long2xia1}{5,9}{⿓、⾍}
  \definition{s.}{lagosta}
\end{entry}

\begin{entry}{笼}{long2}{11}{⽵}
  \definition{s.}{armação fechada de bambu, arame, etc. | jaula | gaiola}
  \seeref{笼}{long3}
\end{entry}

\begin{entry}{笼子}{long2zi5}{11,3}{⽵、⼦}
  \definition{s.}{jaula | cesta | gaiola | recipiente}
  \seeref{笼子}{long3zi5}
\end{entry}

\begin{entry}{笼}{long3}{11}{⽵}
  \definition{v.}{envolver | cobrir}
  \seeref{笼}{long2}
\end{entry}

\begin{entry}{笼子}{long3zi5}{11,3}{⽵、⼦}
  \definition{s.}{caixa grande | porta-malas}
  \seeref{笼子}{long2zi5}
\end{entry}

\begin{entry}{弄}{long4}{7}{⼶}
  \definition{s.}{beco | viela | travessa}
  \seeref{弄}{nong4}
\end{entry}

\begin{entry}{楼}{lou2}{13}{⽊}[HSK 1]
  \definition*{s.}{sobrenome Lou}
  \definition{clas.}{andar, piso}
  \definition[层,座,栋]{s.}{edifício | prédio | sobrado | casa com 2 ou mais andares}
\end{entry}

\begin{entry}{楼上}{lou2 shang4}{13,3}{⽊、⼀}[HSK 1]
  \definition{adv.}{no andar de cima | (gíria da Internet) post anterior em um fio de um fórum}
\end{entry}

\begin{entry}{楼梯}{lou2 ti1}{13,11}{⽊、⽊}[HSK 4]
  \definition[个]{s.}{escada; escadaria; degraus no meio de dois andares para permitir que as pessoas subam ou desçam as escadas}
\end{entry}

\begin{entry}{楼下}{lou2 xia4}{13,3}{⽊、⼀}[HSK 1]
  \definition{adv.}{no andar de baixo}
\end{entry}

\begin{entry}{漏}{lou4}{14}{⽔}
  \definition{s.}{relógio d'água ou ampulheta}
  \definition{v.}{vazar | divulgar | deixar de fora por engano}
\end{entry}

\begin{entry}{漏电}{lou4dian4}{14,5}{⽔、⽥}
  \definition{v.}{vazar eletricidade}
\end{entry}

\begin{entry}{卢旺达}{lu2wang4da2}{5,8,6}{⼘、⽇、⾡}
  \definition*{s.}{Ruanda}
\end{entry}

\begin{entry}{芦笋}{lu2sun3}{7,10}{⾋、⽵}
  \definition{s.}{aspargos}
\end{entry}

\begin{entry}{陆地}{lu4di4}{7,6}{⾩、⼟}[HSK 4]
  \definition[块,片]{s.}{terra; terra seca (em oposição ao mar); superfície da Terra, excluindo os oceanos (e, às vezes, rios e lagos)}
\end{entry}

\begin{entry}{陆路}{lu4lu4}{7,13}{⾩、⾜}
  \definition{s.}{rota terrestre}
\end{entry}

\begin{entry}{陆续}{lu4xu4}{7,11}{⾩、⽷}[HSK 4]
  \definition{adv.}{sucessivamente; um após o outro; intermitentemente}
\end{entry}

\begin{entry}{录}{lu4}{8}{⼹}[HSK 3]
  \definition*{s.}{sobrenome Lu}
  \definition{s.}{nota; ata; registro; coleção; seleções}
  \definition{v.}{gravar; escrever; copiar | selecionar; empregar; contratar | gravar em fita}
\end{entry}

\begin{entry}{录取}{lu4qu3}{8,8}{⼹、⼜}[HSK 4]
  \definition{v.}{aceitar; admitir; recrutar; entrar; matricular (os aprovados no exame)}
\end{entry}

\begin{entry}{录像带}{lu4xiang4dai4}{8,13,9}{⼹、⼈、⼱}
  \definition[盘]{s.}{video-cassete}
\end{entry}

\begin{entry}{录像机}{lu4xiang4ji1}{8,13,6}{⼹、⼈、⽊}
  \definition[台]{s.}{gravador de vídeo | VCR}
\end{entry}

\begin{entry}{录音}{lu4yin1}{8,9}{⼹、⾳}[HSK 3]
  \definition[段,个]{s.}{gravação de som; gravação de filme}
  \definition{v.+compl.}{gravar (som, filme)}
\end{entry}

\begin{entry}{录音机}{lu4yin1ji1}{8,9,6}{⼹、⾳、⽊}
  \definition[台]{s.}{gravador de áudio}
\end{entry}

\begin{entry}{鹿}{lu4}{11}{⿅}[Kangxi 198]
  \definition{s.}{cervo | veado}
\end{entry}

\begin{entry}{路}{lu4}{13}{⾜}[HSK 1]
  \definition*{s.}{sobrenome Lu}
  \definition[条]{s.}{caminho | estrada | via | jornada | linha (ônibus, etc.) | rota}
\end{entry}

\begin{entry}{路边}{lu4 bian1}{13,5}{⾜、⾡}[HSK 2]
  \definition{s.}{meio-fio | acostamento}
\end{entry}

\begin{entry}{路口}{lu4kou3}{13,3}{⾜、⼝}[HSK 1]
  \definition{s.}{cruzamento | interseção (de estradas)}
\end{entry}

\begin{entry}{路上}{lu4shang5}{13,3}{⾜、⼀}[HSK 1]
  \definition{adv.}{na estrada | no caminho | a caminho}
\end{entry}

\begin{entry}{路线}{lu4 xian4}{13,8}{⾜、⽷}[HSK 3]
  \definition[条]{s.}{rota; caminho; linha | linha; diretriz (de política, ideologia, campo de trabalho)}
\end{entry}

\begin{entry}{露珠}{lu4zhu1}{21,10}{⾬、⽟}
  \definition{s.}{orvalho}
\end{entry}

\begin{entry}{乱}{luan4}{7}{⼄}[HSK 3]
  \definition{adj.}{bagunçado; confuso; desordenado | turbulento; perturbado (estado de espírito) | arbitrário; aleatório}
  \definition{adv.}{em confusão ou desordem; em um estado de espírito confuso}
  \definition{s.}{caos; tumulto; agitação; turbilhão | comportamento sexual promíscuo; promiscuidade}
  \definition{v.}{confundir; embaralhar; misturar}
\end{entry}

\begin{entry}{伦敦}{lun2dun1}{6,12}{⼈、⽁}
  \definition*{s.}{Londres}
\end{entry}

\begin{entry}{轮}{lun2}{8}{⾞}[HSK 4]
  \definition{clas.}{para sol vermelho, lua brilhante, etc. | para rodadas | doze anos de idade (os doze ramos terrestres são usados para lembrar o gênero humano e cada doze anos de idade é um ciclo)}
  \definition{s.}{roda | anel; disco; objeto semelhante a uma roda | navio a vapor; barco a vapor}
  \definition{v.}{revezar; substituir um ao outro em sequência (para fazer algo)}
\end{entry}

\begin{entry}{轮船}{lun2chuan2}{8,11}{⾞、⾈}[HSK 4]
  \definition[艘]{s.}{navio}
\end{entry}

\begin{entry}{轮回}{lun2hui2}{8,6}{⾞、⼞}
  \definition[个]{s.}{reencarnação (Budismo) | ciclo}
  \definition{v.}{reencarnar}
\end{entry}

\begin{entry}{轮椅}{lun2 yi3}{8,12}{⾞、⽊}[HSK 4]
  \definition{s.}{cadeira de rodas; dispositivo de assento especialmente projetado com rodas para pessoas com dificuldade de locomoção, que pode ser acionado por um disco de roda ou manivela operados manualmente}
\end{entry}

\begin{entry}{轮子}{lun2 zi5}{8,3}{⾞、⼦}[HSK 4]
  \definition[个]{s.}{roda; peças circulares de veículos ou máquinas com capacidade de rotação}
\end{entry}

\begin{entry}{论文}{lun4wen2}{6,4}{⾔、⽂}[HSK 4]
  \definition[篇]{s.}{tese; redação; artigo; artigo que discute ou examina uma questão}
\end{entry}

\begin{entry}{罗}{luo2}{8}{⽹}
  \definition*{s.}{sobrenome Luo}
  \definition{v.}{coletar | juntar | pegar | peneirar}
\end{entry}

\begin{entry}{螺}{luo2}{17}{⾍}
  \definition{s.}{concha em espiral | caracol | búzio}
\end{entry}

\begin{entry}{螺丝}{luo2si1}{17,5}{⾍、⼀}
  \definition{s.}{parafuso}
\end{entry}

\begin{entry}{骆驼}{luo4tuo5}{9,8}{⾺、⾺}
  \definition[峰,匹,头]{s.}{camelo | (coloquial) cabeça-dura, idiota}
\end{entry}

\begin{entry}{落}{luo4}{12}{⾋}[HSK 4]
  \definition*{s.}{sobrenome Luo}
  \definition{s.}{paradeiro; lugar para ficar; local de descanso | assentamento; local de reunião | parte curta; área pequena; refere-se a um pequeno lugar ou área}
  \definition{v.}{cair | descer | baixar; deixar cair (ou descer) | afundar; declinar; cair; (figurativo) mudança da prosperidade para o declínio | ficar para trás; deixar para trás ou do lado de fora | ficar; parar; deixar para trás | cair em cima de; descansar com | obter; ter; receber | escrever; colocar a caneta no papel | cair em; entrar em}
  \seeref{落}{la4}
  \seeref{落}{lao4}
\end{entry}

\begin{entry}{落后}{luo4hou4}{12,6}{⾋、⼝}[HSK 3]
  \definition{adj.}{atrasado}
  \definition{s.}{atraso}
  \definition{v.}{ficar para trás | atrasar}
  \definition{v.}{atrasar-se; ficar para trás}
\end{entry}

\begin{entry}{落日}{luo4ri4}{12,4}{⾋、⽇}
  \definition{s.}{pôr do sol}
\end{entry}

\begin{entry}{落汤鸡}{luo4tang1ji1}{12,6,7}{⾋、⽔、⿃}
  \definition{s.}{uma pessoa que parece encharcada e acamada| sofrimento profundo}
\end{entry}

\begin{entry}{驴}{lv2}{7}{⾺}
  \definition[头]{s.}{burro | asno | jumento | jegue}
\end{entry}

\begin{entry}{旅程}{lv3cheng2}{10,12}{⽅、⽲}
  \definition{s.}{jornada | viagem}
\end{entry}

\begin{entry}{旅馆}{lv3 guan3}{10,11}{⽅、⾷}[HSK 3]
  \definition[家,个,所]{s.}{pousada; hotel}
\end{entry}

\begin{entry}{旅客}{lv3 ke4}{10,9}{⽅、⼧}[HSK 2]
  \definition{s.}{viajante | turista}
\end{entry}

\begin{entry}{旅行}{lv3xing2}{10,6}{⽅、⾏}[HSK 2]
  \definition{v.}{viajar}
\end{entry}

\begin{entry}{旅行社}{lv3 xing2 she4}{10,6,7}{⽅、⾏、⽰}[HSK 3]
  \definition[家]{s.}{agência de viagens}
\end{entry}

\begin{entry}{旅游}{lv3you2}{10,12}{⽅、⽔}[HSK 2]
  \definition[趟,次,个]{s.}{jornada | viagem}
  \definition{v.}{viajar}
\end{entry}

\begin{entry}{屡次}{lv3ci4}{12,6}{⼫、⽋}
  \definition{adv.}{repetidamente | uma e outra vez | muitas vezes}
\end{entry}

\begin{entry}{律师}{lv4shi1}{9,6}{⼻、⼱}[HSK 4]
  \definition[名,个,位]{s.}{advogado; procurador; profissionais encarregados pelas partes ou nomeados pelo tribunal para auxiliar as partes no litígio, para comparecer ao tribunal para defesa e para tratar de assuntos jurídicos relacionados, de acordo com a lei}
\end{entry}

\begin{entry}{绿}{lv4}{11}{⽷}[HSK 2]
  \definition{adj.}{verde}
\end{entry}

\begin{entry}{绿茶}{lv4 cha2}{11,9}{⽷、⾋}[HSK 3]
  \definition{s.}{chá verde}
\end{entry}

\begin{entry}{绿豆}{lv4dou4}{11,7}{⽷、⾖}
  \definition{s.}{vagens}
\end{entry}

\begin{entry}{绿豆芽}{lv4dou4 ya2}{11,7,7}{⽷、⾖、⾋}
  \definition{s.}{broto de feijão verde}
\end{entry}

\begin{entry}{绿色}{lv4 se4}{11,6}{⽷、⾊}[HSK 2]
  \definition{s.}{cor verde}
\end{entry}

\begin{entry}{略}{lve4}{11}{⽥}
  \definition{adv.}{ligeiramente | marginalmente | aproximadamente}
\end{entry}

\begin{entry}{略微}{lve4wei1}{11,13}{⽥、⼻}
  \definition{adv.}{ligeiramente | marginalmente | aproximadamente}
\end{entry}

%%%%% EOF %%%%%

