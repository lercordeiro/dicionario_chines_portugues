%%%
%%% W
%%%

\section*{W}\addcontentsline{toc}{section}{W}

\begin{entry}{哇塞}{wa1sai1}{9,13}{⼝、⼟}
  \definition{interj.}{(gíria) Uau!}
\end{entry}

\begin{entry}{哇噻}{wa1sai1}{9,16}{⼝、⼝}
  \variantof{哇塞}
\end{entry}

\begin{entry}{挖}{wa1}{9}{⼿}
  \definition{v.}{cavar | escavar}
\end{entry}

\begin{entry}{挖掘机}{wa1jue2ji1}{9,11,6}{⼿、⼿、⽊}
  \definition{s.}{escavadeira | escavador | escavadora | pá mecânica}
\end{entry}

\begin{entry}{瓦}{wa3}{4}{⽡}[Kangxi 98]
  \definition{s.}{telha | abreviação de 瓦特}
  \seealsoref{瓦特}{wa3te4}
\end{entry}

\begin{entry}{瓦努阿图}{wa3nu3'a1tu2}{4,7,7,8}{⽡、⼒、⾩、⼞}
  \definition*{s.}{Vanuatu, país do sudoeste do Oceano Pacífico}
\end{entry}

\begin{entry}{瓦特}{wa3te4}{4,10}{⽡、⽜}
  \definition{s.}{(empréstimo linguístico) watt | medida de potência}
\end{entry}

\begin{entry}{袜子}{wa4zi5}{10,3}{⾐、⼦}[HSK 4]
  \definition[双,只,对]{s.}{meias; peúgas; meias-calças}
\end{entry}

\begin{entry}{歪}{wai1}{9}{⽌}
  \definition{adj.}{torto | tortuoso | nocivo}
\end{entry}

\begin{entry}{歪果仁}{wai1 guo3 ren2}{9,8,4}{⽌、⽊、⼈}
  \definition{s.}{gíria na \emph{Internet} para estrangeiro (外国人)}
  \seealsoref{外国人}{wai4 guo2 ren2}
\end{entry}

\begin{entry}{外}{wai4}{5}{⼣}[HSK 1]
  \definition{adj.}{outro (que não o próprio) | não íntimo; não intimamente relacionado | não oficial | exterior; externo; do lado de fora | outros; referindo-se a um local fora da sua localização atual | do lado da mãe, da irmã ou da filha; referir-se a parentes do lado materno, irmãs ou filhas | informal; não oficial}
  \definition{adv.}{adicionalmente; além disso | para fora; para o exterior; fora | extra; além disso}
  \definition{s.}{fora; externo; exterior (oposto a 内, 里) | outro local; outro lugar | estrangeiro; país estrangeiro | lado externo | parentes de sua mãe, irmãs ou filhas}
  \seealsoref{里}{li3}
  \seealsoref{内}{nei4}
\end{entry}

\begin{entry}{外边}{wai4 bian5}{5,5}{⼣、⾡}[HSK 1]
  \definition{s.}{fora; exterior; externo; além de um determinado limite | local diferente de onde se vive ou trabalha; referindo-se a lugares distantes | exterior; externo; superfície}
\end{entry}

\begin{entry}{外插}{wai4cha1}{5,12}{⼣、⼿}
  \definition{v.}{extrapolar | (computação) conectar (um dispositivo periférico, etc.)}
\end{entry}

\begin{entry}{外地}{wai4 di4}{5,6}{⼣、⼟}[HSK 2]
  \definition{s.}{não local; outros lugares; locais fora da área local}
\end{entry}

\begin{entry}{外公}{wai4gong1}{5,4}{⼣、⼋}
  \definition{s.}{avô materno}
\end{entry}

\begin{entry}{外国}{wai4 guo2}{5,8}{⼣、⼞}[HSK 1]
  \definition[个]{s.}{país estrangeiro}
\end{entry}

\begin{entry}{外国人}{wai4 guo2 ren2}{5,8,2}{⼣、⼞、⼈}
  \definition[个]{s.}{estrangeiro | alienígena}
\end{entry}

\begin{entry}{外海}{wai4hai3}{5,10}{⼣、⽔}
  \definition{s.}{mar aberto}
\end{entry}

\begin{entry}{外号}{wai4hao4}{5,5}{⼣、⼝}
  \definition{s.}{apelido}
\end{entry}

\begin{entry}{外汇}{wai4 hui4}{5,5}{⼣、⽔}[HSK 4]
  \definition{s.}{câmbio estrangeiro; moeda estrangeira; moedas estrangeiras e títulos, como cheques, letras de câmbio, notas promissórias, etc., conversíveis em moedas estrangeiras, usados na compensação do comércio internacional}
\end{entry}

\begin{entry}{外积}{wai4ji1}{5,10}{⼣、⽲}
  \definition{s.}{produto exterior | (matemática) o produto vetorial de dois vetores}
\end{entry}

\begin{entry}{外交}{wai4jiao1}{5,6}{⼣、⼇}[HSK 3]
  \definition[个]{s.}{diplomacia; relações exteriores; atividades de um país nas relações internacionais, como participar de organizações e conferências internacionais, trocar enviados com outros países, conduzir negociações, assinar tratados e acordos, etc.}
\end{entry}

\begin{entry}{外交官}{wai4 jiao1 guan1}{5,6,8}{⼣、⼇、⼧}[HSK 4]
  \definition{s.}{diplomata}
\end{entry}

\begin{entry}{外界}{wai4jie4}{5,9}{⼣、⽥}[HSK 5]
  \definition{s.}{o exterior; o mundo externo; área fora de um determinado âmbito; sociedade externa}
\end{entry}

\begin{entry}{外卖}{wai4 mai4}{5,8}{⼣、⼗}[HSK 2]
  \definition[份,单,盒]{s.}{comida para viagem; levar para viagem}
  \definition{v.}{entregar; oferecer; refere-se à ação do comerciante entregar alimentos no local especificado pelo cliente}
\end{entry}

\begin{entry}{外贸}{wai4mao4}{5,9}{⼣、⾙}
  \definition{s.}{comércio exterior}
\end{entry}

\begin{entry}{外貌协会}{wai4mao4xie2hui4}{5,14,6,6}{⼣、⾘、⼗、⼈}
  \definition{s.}{``o clube da boa aparência'': pessoas que dão grande importância à aparência de uma pessoa}
  \seealsoref{外协}{wai4xie2}
\end{entry}

\begin{entry}{外面}{wai4 mian4}{5,9}{⼣、⾯}[HSK 3]
  \definition{s.}{o lado de fora; fora de um certo intervalo | exterior; aparência externa; a superfície de um objeto}
\end{entry}

\begin{entry}{外婆}{wai4po2}{5,11}{⼣、⼥}
  \definition{s.}{avó materna}
\end{entry}

\begin{entry}{外事}{wai4shi4}{5,8}{⼣、⼅}
  \definition{s.}{assuntos ou relações exteriores}
\end{entry}

\begin{entry}{外水}{wai4shui3}{5,4}{⼣、⽔}
  \definition{s.}{renda extra}
\end{entry}

\begin{entry}{外孙}{wai4sun1}{5,6}{⼣、⼦}
  \definition{s.}{filho da filha}
\end{entry}

\begin{entry}{外孙女}{wai4sun1nv3}{5,6,3}{⼣、⼦、⼥}
  \definition{s.}{filha da filha}
\end{entry}

\begin{entry}{外套}{wai4 tao4}{5,10}{⼣、⼤}[HSK 4]
  \definition[件,套]{s.}{casaco; jaqueta; paletó; sobretudo}
\end{entry}

\begin{entry}{外围}{wai4wei2}{5,7}{⼣、⼞}
  \definition{adv.}{arredores}
\end{entry}

\begin{entry}{外文}{wai4 wen2}{5,4}{⼣、⽂}[HSK 3]
  \definition[种,门]{s.}{língua ou escrita estrangeira}
\end{entry}

\begin{entry}{外协}{wai4xie2}{5,6}{⼣、⼗}
  \definition{s.}{terceirização | pessoas que julgam os outros pela aparência}
  \seealsoref{外貌协会}{wai4mao4xie2hui4}
\end{entry}

\begin{entry}{外衣}{wai4yi1}{5,6}{⼣、⾐}
  \definition{s.}{aparência | roupa de cima}
\end{entry}

\begin{entry}{外语}{wai4 yu3}{5,9}{⼣、⾔}[HSK 1]
  \definition[种,门]{s.}{língua estrangeira}
\end{entry}

\begin{entry}{弯}{wan1}{9}{⼸}[HSK 4]
  \definition{adj.}{curvo; dobrado; torto; flexível; tortuoso}
  \definition{s.}{curva; dobra}
  \definition{v.}{curvar; dobrar; flexionar}
\end{entry}

\begin{entry}{豌豆}{wan1dou4}{15,7}{⾖、⾖}
  \definition{s.}{ervilha}
\end{entry}

\begin{entry}{完}{wan2}{7}{⼧}[HSK 2]
  \definition*{s.}{sobrenome Wan}
  \definition{adj.}{inteiro; intacto; completo}
  \definition{v.}{acabar; terminar; completar | pagar | estar terminado; estar pronto para | esgotar; ser usado}
\end{entry}

\begin{entry}{完备}{wan2bei4}{7,8}{⼧、⼡}
  \definition{adj.}{completo | impecável | perfeito}
  \definition{v.}{não deixar nada a desejar}
\end{entry}

\begin{entry}{完毕}{wan2bi4}{7,6}{⼧、⽐}
  \definition{v.}{completar | terminar | acabar}
\end{entry}

\begin{entry}{完成}{wan2cheng2}{7,6}{⼧、⼽}[HSK 2]
  \definition{v.}{realizar; completar; terminar; cumprir; levar ao sucesso}
\end{entry}

\begin{entry}{完了}{wan2 le5}{7,2}{⼧、⼅}[HSK 5]
  \definition{v.}{acabar; terminar; concluir; chegar ao fim}
\end{entry}

\begin{entry}{完满}{wan2man3}{7,13}{⼧、⽔}
  \definition{adj.}{satisfatório | bem-sucedido}
\end{entry}

\begin{entry}{完美}{wan2mei3}{7,9}{⼧、⽺}[HSK 3]
  \definition{adj.}{perfeito; impecável; consumado}
\end{entry}

\begin{entry}{完全}{wan2quan2}{7,6}{⼧、⼊}[HSK 2]
  \definition{adj.}{inteiro; completo; não falta nada, está tudo completo}
  \definition{adv.}{completamente; representa tudo}
\end{entry}

\begin{entry}{完人}{wan2ren2}{7,2}{⼧、⼈}
  \definition{s.}{pessoa perfeita}
\end{entry}

\begin{entry}{完善}{wan2shan4}{7,12}{⼧、⼝}[HSK 3]
  \definition{adj.}{perfeito; consumado}
  \definition{v.}{refinar; melhorar; tornar perfeito}
\end{entry}

\begin{entry}{完税}{wan2shui4}{7,12}{⼧、⽲}
  \definition{v.}{pagar imposto}
\end{entry}

\begin{entry}{完完全全}{wan2wan2quan2quan2}{7,7,6,6}{⼧、⼧、⼊、⼊}
  \definition{adv.}{completamente}
\end{entry}

\begin{entry}{完整}{wan2zheng3}{7,16}{⼧、⽁}[HSK 3]
  \definition{adj.}{intacto; inteiro; completo; integrado; nenhum dano ou mutilação}
\end{entry}

\begin{entry}{玩}{wan2}{8}{⽟}
  \definition{s.}{brinquedo | algo usado para diversão}
  \definition{v.}{divertir-se | manter algo para entretenimento | brincar com}
\end{entry}

\begin{entry}{玩伴}{wan2ban4}{8,7}{⽟、⼈}
  \definition{s.}{parceiro de brincadeira}
\end{entry}

\begin{entry}{玩遍}{wan2bian4}{8,12}{⽟、⾡}
  \definition{v.}{passear (todo o país, toda a cidade, etc.) | visitar (um grande número de lugares)}
\end{entry}

\begin{entry}{玩家}{wan2jia1}{8,10}{⽟、⼧}
  \definition{s.}{entusiasta (áudio, modelos de aviões, etc.) | jogador (de um jogo)}
\end{entry}

\begin{entry}{玩具}{wan2ju4}{8,8}{⽟、⼋}[HSK 3]
  \definition[个,件,套]{s.}{brinquedo; coisas para brincar}
\end{entry}

\begin{entry}{玩具厂}{wan2ju4chang3}{8,8,2}{⽟、⼋、⼚}
  \definition{s.}{fábrica de brinquedos}
\end{entry}

\begin{entry}{玩具车}{wan2ju4 che1}{8,8,4}{⽟、⼋、⾞}
  \definition{s.}{carrinho de brinquedo}
\end{entry}

\begin{entry}{玩偶}{wan2'ou3}{8,11}{⽟、⼈}
  \definition{s.}{estatueta de brinquedo | boneco de ação | bicho de pelúcia | boneca}
\end{entry}

\begin{entry}{玩儿}{wan2r5}{8,2}{⽟、⼉}[HSK 1]
  \definition{v.}{divertir-se; (entretenimento) relaxar ou experimentar alguma atividade}
\end{entry}

\begin{entry}{玩耍}{wan2shua3}{8,9}{⽟、⽽}
  \definition{v.}{divertir-me | brincar (como as crianças fazem)}
\end{entry}

\begin{entry}{玩味}{wan2wei4}{8,8}{⽟、⼝}
  \definition{v.}{ponderar sutilezas | ruminar (pensamentos)}
\end{entry}

\begin{entry}{玩艺}{wan2yi4}{8,4}{⽟、⾋}
  \variantof{玩意}
\end{entry}

\begin{entry}{玩意}{wan2yi4}{8,13}{⽟、⼼}
  \definition{s.}{ato | brinquedo | coisa | truque (em uma performance, show de palco, acrobacias, etc.)}
\end{entry}

\begin{entry}{玩者}{wan2zhe3}{8,8}{⽟、⽼}
  \definition{s.}{jogador}
\end{entry}

\begin{entry}{顽强}{wan2qiang2}{10,12}{⾴、⼸}
  \definition{adj.}{persistente | tenaz | difícil de derrotar}
\end{entry}

\begin{entry}{埦}{wan3}{11}{⼟}
  \variantof{碗}
\end{entry}

\begin{entry}{晚}{wan3}{11}{⽇}[HSK 1]
  \definition*{s.}{sobrenome Wan}
  \definition{adj.}{tarde; tardio; passado o prazo acordado | júnior; mais jovem | mais tarde no tempo}
  \definition{s.}{noite; à noite; após o pôr do sol | últimos anos; última vida; um período posterior; refere-se especificamente à velhice de uma pessoa | pôr do sol; ao pôr do sol}
\end{entry}

\begin{entry}{晚安}{wan3'an1}{11,6}{⽇、⼧}[HSK 2]
  \definition{expr.}{Tenha uma boa noite; uma frase educada usada para se despedir ou cumprimentar as pessoas à noite}
\end{entry}

\begin{entry}{晚报}{wan3 bao4}{11,7}{⽇、⼿}[HSK 2]
  \definition[份,张]{s.}{jornal vespertino; um jornal publicado todas as tardes}
\end{entry}

\begin{entry}{晚餐}{wan3 can1}{11,16}{⽇、⾷}[HSK 2]
  \definition[份,顿,次]{s.}{ceia; jantar}
\end{entry}

\begin{entry}{晚点}{wan3 dian3}{11,9}{⽇、⽕}[HSK 4]
  \definition{adj.}{atrasado}
  \definition{s.}{jantar leve}
  \definition{v.}{atrasar; retardar; adiar; (carro, navio, avião) partir, correr ou chegar mais tarde do que o horário especificado}
\end{entry}

\begin{entry}{晚饭}{wan3 fan4}{11,7}{⽇、⾷}[HSK 1]
  \definition[顿]{s.}{jantar}
\end{entry}

\begin{entry}{晚会}{wan3hui4}{11,6}{⽇、⼈}[HSK 2]
  \definition[场,个,次]{s.}{festa noturna; entretenimento noturno}
\end{entry}

\begin{entry}{晚近}{wan3jin4}{11,7}{⽇、⾡}
  \definition{adj.}{recente | mais recente no passado}
  \definition{adv.}{ultimamente | recentemente}
\end{entry}

\begin{entry}{晚景}{wan3jing3}{11,12}{⽇、⽇}
  \definition{s.}{circunstâncias dos anos de declínio de alguém | cena noturna}
\end{entry}

\begin{entry}{晚上}{wan3shang5}{11,3}{⽇、⼀}[HSK 1]
  \definition[个]{s.}{noite; o período entre o pôr do sol e a madrugada}
\end{entry}

\begin{entry}{晚育}{wan3yu4}{11,8}{⽇、⾁}
  \definition{s.}{parto tardio}
  \definition{v.}{ter um filho mais tarde}
\end{entry}

\begin{entry}{碗}{wan3}{13}{⽯}[HSK 2]
  \definition*{s.}{sobrenome Wan}
  \definition{clas.}{usado para medição de alimentos e bebidas}
  \definition[只,个]{s.}{tigela | objeto em forma de tigela |}
\end{entry}

\begin{entry}{碗柜}{wan3gui4}{13,8}{⽯、⽊}
  \definition{s.}{armário}
\end{entry}

\begin{entry}{碗子}{wan3zi5}{13,3}{⽯、⼦}
  \definition{s.}{tigela}
\end{entry}

\begin{entry}{万}{wan4}{3}{⼀}[HSK 2]
  \definition*{s.}{sobrenome Wan}
  \definition{adv.}{absolutamente; indica um grau extremamente alto, equivalente a 完全, 绝对 e 极}
  \definition{num.}{dez mil; 10.000; 1.0000 | miríade; um número muito grande}
  \seealsoref{极}{ji2}
  \seealsoref{绝对}{jue2dui4}
  \seealsoref{完全}{wan2quan2}
\end{entry}

\begin{entry}{万福}{wan4fu2}{3,13}{⼀、⽰}
  \definition{s.}{(antigo) reverência feminina; reverência}
\end{entry}

\begin{entry}{万圣节}{wan4sheng4jie2}{3,5,5}{⼀、⼟、⾋}
  \definition*{s.}{Dia de Todos os Santos}
  \seealsoref{万圣节前夕}{wan4sheng4jie2qian2xi1}
\end{entry}

\begin{entry}{万圣节前夕}{wan4sheng4jie2qian2xi1}{3,5,5,9,3}{⼀、⼟、⾋、⼑、⼣}
  \definition*{s.}{Véspera do Dia de Todos os Santos | \emph{Halloween}}
  \seealsoref{万圣节}{wan4sheng4jie2}
\end{entry}

\begin{entry}{万万}{wan4wan4}{3,3}{⼀、⼀}
  \definition{adv.}{absolutamente | totalmente}
\end{entry}

\begin{entry}{万一}{wan4yi1}{3,1}{⼀、⼀}[HSK 4]
  \definition{conj.}{por via das dúvidas; se por acaso; só por precaução; expressa uma suposição muito improvável (usado para coisas desagradáveis)}
  \definition{num.}{um décimo milionésimo; uma porcentagem muito pequena}
  \definition{s.}{contingência; eventualidade; contingências muito improváveis}
\end{entry}

\begin{entry}{王}{wang2}{4}{⽟}[HSK 4]
  \definition*{s.}{sobrenome Wang}
  \definition{adj.}{grande; ótimo; honoríficos antigos para avós}
  \definition{s.}{rei; monarca; imperador; governante supremo de uma monarquia | cabeça; chefe; líder | o primeiro, maior ou mais forte de seu tipo | duque; príncipe; o título mais alto da sociedade feudal após a dinastia Han}
  \seeref{王}{wang4}
\end{entry}

\begin{entry}{王朝}{wang2chao2}{4,12}{⽟、⽉}
  \definition{s.}{dinastia}
\end{entry}

\begin{entry}{王五}{wang2wu3}{4,4}{⽟、⼆}
  \definition{s.}{Wang Wu | Zé Ninguém | nome para uma pessoa não especificada, 3 de 3}
  \seealsoref{李四}{li3si4}
  \seealsoref{张三}{zhang1san1}
\end{entry}

\begin{entry}{网}{wang3}{6}{⽹}[HSK 2][Kangxi 122]
  \definition[张]{s.}{rede; um dispositivo feito de corda ou barbante para capturar peixes e pássaros | algo que parece uma rede | rede; uma rede de organizações; um sistema}
  \definition{v.}{pegar com uma rede | cobrir como com uma rede}
\end{entry}

\begin{entry}{网罟}{wang3gu3}{6,10}{⽹、⽹}
  \definition{s.}{(fig.) a rede da justiça | rede usada para capturar peixes (ou outros animais, como pássaros)}
\end{entry}

\begin{entry}{网际网路}{wang3ji4wang3lu4}{6,7,6,13}{⽹、⾩、⽹、⾜}
  \definition*{s.}{\emph{Internet}}
  \seealsoref{互联网}{hu4lian2wang3}
  \seealsoref{网际网络}{wang3ji4wang3luo4}
  \seealsoref{网路}{wang3lu4}
\end{entry}

\begin{entry}{网际网络}{wang3ji4wang3luo4}{6,7,6,9}{⽹、⾩、⽹、⽷}
  \definition*{s.}{\emph{Internet}}
  \seealsoref{互联网}{hu4lian2wang3}
  \seealsoref{网际网路}{wang3ji4wang3lu4}
  \seealsoref{网路}{wang3lu4}
\end{entry}

\begin{entry}{网路}{wang3lu4}{6,13}{⽹、⾜}
  \definition{s.}{\emph{Internet}}
  \seealsoref{互联网}{hu4lian2wang3}
  \seealsoref{网际网路}{wang3ji4wang3lu4}
  \seealsoref{网际网络}{wang3ji4wang3luo4}
\end{entry}

\begin{entry}{网络}{wang3luo4}{6,9}{⽹、⽷}[HSK 4]
  \definition{s.}{rede; um sistema que consiste em ramificações interconectadas; em um sistema elétrico, um circuito ou parte de um circuito que consiste em vários elementos que permitem a transmissão de sinais elétricos de acordo com determinados requisitos | rede; rede de computadores}
\end{entry}

\begin{entry}{网球}{wang3qiu2}{6,11}{⽹、⽟}[HSK 2]
  \definition[个,颗,些]{s.}{tênis (esporte) | bola de tênis}
\end{entry}

\begin{entry}{网上}{wang3 shang4}{6,3}{⽹、⼀}[HSK 1]
  \definition{s.}{\emph{online}; refere-se a acessar a Internet através de um computador ou celular para pesquisar e consultar informações na rede}
\end{entry}

\begin{entry}{网上银行}{wang3shang4yin2hang2}{6,3,11,6}{⽹、⼀、⾦、⾏}
  \definition[个]{s.}{banco \emph{online} | acesso a operações bancárias via \emph{Internet}}
  \seealsoref{网银}{wang3yin2}
\end{entry}

\begin{entry}{网银}{wang3yin2}{6,11}{⽹、⾦}
  \definition{s.}{banco \emph{online} | acesso a operações bancárias via \emph{Internet}}
  \seealsoref{网上银行}{wang3shang4yin2hang2}
\end{entry}

\begin{entry}{网友}{wang3 you3}{6,4}{⽹、⼜}[HSK 1]
  \definition{s.}{internauta; usuário da \emph{Internet}; amigos que se conhecem pela Internet; também usado como forma de tratamento entre internautas}
\end{entry}

\begin{entry}{网站}{wang3zhan4}{6,10}{⽹、⽴}[HSK 2]
  \definition[个,家]{s.}{\emph{web}; \emph{website}; um site virtual na Internet para uma organização ou indivíduo, geralmente consistindo em uma página inicial e muitas páginas da web}
\end{entry}

\begin{entry}{网址}{wang3 zhi3}{6,7}{⽹、⼟}[HSK 4]
  \definition{s.}{\emph{website}; endereço da \emph{web}; endereço de um \emph{site} na \emph{Internet}, que os usuários podem acessar, consultar e obter recursos de informações nesse \emph{site} clicando nele}
\end{entry}

\begin{entry}{往}{wang3}{8}{⼻}[HSK 2]
  \definition{adj.}{passado; anterior}
  \definition{prep.}{para; em direção a; na direção de}
  \definition{v.}{ir}
\end{entry}

\begin{entry}{往程}{wang3cheng2}{8,12}{⼻、⽲}
  \definition{s.}{saída (de uma viagem de ônibus ou trem, etc.)}
\end{entry}

\begin{entry}{往返}{wang3fan3}{8,7}{⼻、⾡}
  \definition{s.}{ida e volta}
  \definition{v.}{ir e voltar | ir e vir}
\end{entry}

\begin{entry}{往复}{wang3fu4}{8,9}{⼻、⼢}
  \definition{s.}{para trás e para frente (por exemplo, da ação do pistão ou da bomba)}
  \definition{v.}{ir e voltar | fazer uma viagem de volta}
\end{entry}

\begin{entry}{往迹}{wang3ji4}{8,9}{⼻、⾡}
  \definition{s.}{eventos passados}
\end{entry}

\begin{entry}{往来}{wang3lai2}{8,7}{⼻、⽊}
  \definition{s.}{contatos | negociações}
\end{entry}

\begin{entry}{往例}{wang3li4}{8,8}{⼻、⼈}
  \definition{s.}{prática (habitual) do passado | precedente}
\end{entry}

\begin{entry}{往日}{wang3ri4}{8,4}{⼻、⽇}
  \definition{adv.}{dias passados}
  \definition{s.}{o passado}
\end{entry}

\begin{entry}{往生}{wang3sheng1}{8,5}{⼻、⽣}
  \definition{v.}{renascer | morrer | (Budismo) viver no paraíso}
\end{entry}

\begin{entry}{往事}{wang3shi4}{8,8}{⼻、⼅}
  \definition{s.}{acontecimentos anteriores | eventos passados}
\end{entry}

\begin{entry}{往往}{wang3wang3}{8,8}{⼻、⼻}[HSK 3]
  \definition{adv.}{frequentemente; muitas vezes; mais frequentemente do que não; indica que uma situação existe ou ocorre com frequência}
\end{entry}

\begin{entry}{往昔}{wang3xi1}{8,8}{⼻、⽇}
  \definition{s.}{o passado}
\end{entry}

\begin{entry}{罔}{wang3}{8}{⼌}
  \definition{v.}{enganar}
\end{entry}

\begin{entry}{王}{wang4}{4}{⽟}
  \definition{v.}{reger; governar; reinar; dominar}
  \seeref{王}{wang2}
\end{entry}

\begin{entry}{忘}{wang4}{7}{⼼}[HSK 1]
  \definition{v.}{esquecer | ignorar; negligenciar}
\end{entry}

\begin{entry}{忘本}{wang4ben3}{7,5}{⼼、⽊}
  \definition{v.}{esquecer as próprias raízes}
\end{entry}

\begin{entry}{忘餐}{wang4can1}{7,16}{⼼、⾷}
  \definition{v.}{esquecer as refeições}
\end{entry}

\begin{entry}{忘掉}{wang4diao4}{7,11}{⼼、⼿}
  \definition{v.}{esquecer}
\end{entry}

\begin{entry}{忘恩}{wang4'en1}{7,10}{⼼、⼼}
  \definition{v.}{ser ingrato}
\end{entry}

\begin{entry}{忘怀}{wang4huai2}{7,7}{⼼、⼼}
  \definition{v.}{esquecer}
\end{entry}

\begin{entry}{忘记}{wang4ji4}{7,5}{⼼、⾔}[HSK 1]
  \definition{v.}{esquecer | ignorar; negligenciar | sair da memória de alguém; não ser lembrado | descartar da mente; ignorar}
\end{entry}

\begin{entry}{忘却}{wang4que4}{7,7}{⼼、⼙}
  \definition{v.}{esquecer}
\end{entry}

\begin{entry}{危}{wei1}{6}{⼙}
  \definition*{s.}{sobrenome Wei}
  \definition*{s.}{Wei, a décima segunda das vinte e oito constelações em que a esfera celeste foi dividida, consistindo de três estrelas em forma de triângulo obtuso, uma em Aquário e duas em Pégaso | Wei, uma das mansões lunares}
  \definition{adj.}{arriscado; inseguro; perigoso (oposto a 安) | estar gravemente doente; estar morrendo | alto; íngreme}
  \definition{s.}{perigo | cumeeira (de um telhado)}
  \definition{v.}{pôr em perigo; colocar em perigo; comprometer}
  \seealsoref{安}{an1}
\end{entry}

\begin{entry}{危害}{wei1hai4}{6,10}{⼙、⼧}[HSK 3]
  \definition[个,种]{s.}{prejuízo; perigo; dano}
  \definition{v.}{destruir; prejudicar; pôr em perigo; pôr em risco}
\end{entry}

\begin{entry}{危急}{wei1ji2}{6,9}{⼙、⼼}
  \definition{adj.}{crítico | desesperadora (situação)}
\end{entry}

\begin{entry}{危难}{wei1nan4}{6,10}{⼙、⾫}
  \definition{s.}{calamidade}
\end{entry}

\begin{entry}{危险}{wei1xian3}{6,9}{⼙、⾩}[HSK 3]
  \definition{adj.}{arriscado; perigoso}
\end{entry}

\begin{entry}{微博}{wei1 bo2}{13,12}{⼻、⼗}[HSK 5]
  \definition*{s.}{Weibo (um aplicativo de mídia social chinês)}
  \definition[条]{s.}{\emph{microblog}}
\end{entry}

\begin{entry}{微风}{wei1feng1}{13,4}{⼻、⾵}
  \definition{s.}{brisa | vento leve}
\end{entry}

\begin{entry}{微软}{wei1ruan3}{13,8}{⼻、⾞}
  \definition*{s.}{\emph{Microsoft Corporation}}
\end{entry}

\begin{entry}{微笑}{wei1xiao4}{13,10}{⼻、⽵}[HSK 4]
  \definition[个,丝]{s.}{sorriso;}
  \definition{v.}{sorrir}
\end{entry}

\begin{entry}{微信}{wei1 xin4}{13,9}{⼻、⼈}[HSK 4]
  \definition*{s.}{WeChat; aplicativo gratuito lançado pela Tencent em 21 de janeiro de 2011 para fornecer serviços de mensagens instantâneas para terminais inteligentes}
\end{entry}

\begin{entry}{微型}{wei1xing2}{13,9}{⼻、⼟}
  \definition{pref.}{micro-}
  \definition{s.}{miniatura}
\end{entry}

\begin{entry}{为}{wei2}{4}{⼂}[HSK 3]
  \definition*{s.}{sobrenome Wei}
  \definition{part.}{frequentemente usado com 何 em uma pergunta retórica}
  \definition{prep.}{por; usado em frases passivas para introduzir o agente da ação, equivalente a 被 (frequentemente usado com 所)}
  \definition{suf.}{é anexado a alguns adjetivos ou advérbios monossilábicos para formar advérbios dissilábicos que expressam grau ou amplitude, geralmente modificando adjetivos ou verbos dissilábicos}
  \definition{v.}{fazer; agir | tornar-se; transformar-se em | ser; significar | servir como; agir como; desempenhar o papel de | fazer; trabalhar; indica certas ações e comportamentos, incluindo os significados de governança, engajamento, cenário e pesquisa}
  \seeref{为}{wei4}
  \seealsoref{被}{bei4}
  \seealsoref{何}{he2}
  \seealsoref{所}{suo3}
\end{entry}

\begin{entry}{为难}{wei2nan2}{4,10}{⼂、⾫}[HSK 5]
  \definition{adj.}{envergonhado; sentir-se constrangido; sentir-se sobrecarregado; sentir-se incapaz de lidar com algo}
  \definition{v.}{dificultar as coisas para; dificultar; contrariar}
\end{entry}

\begin{entry}{为期}{wei2qi1}{4,12}{⼂、⽉}[HSK 5]
  \definition{s.}{tempo restante}
  \definition{v.}{a ser concluído (até uma data definida, por um determinado período de tempo)}
\end{entry}

\begin{entry}{为止}{wei2 zhi3}{4,4}{⼂、⽌}[HSK 5]
  \definition{adv.}{até; até um determinado momento}
\end{entry}

\begin{entry}{为主}{wei2 zhu3}{4,5}{⼂、⼂}[HSK 5]
  \definition{v.}{dar prioridade a; dar preferência a; dar importância a}
\end{entry}

\begin{entry}{围}{wei2}{7}{⼞}[HSK 3]
  \definition*{s.}{sobrenome Wei}
  \definition{clas.}{o comprimento das duas mãos com os polegares e os dedos indicadores juntos ou dos dois braços juntos}
  \definition{s.}{em volta de tudo; ao redor}
  \definition{v.}{cercar; rodear; circundar; encurralar; cercar tudo, impedindo a passagem entre o interior e o exterior | envolver; contornar}
\end{entry}

\begin{entry}{围巾}{wei2jin1}{7,3}{⼞、⼱}[HSK 4]
  \definition[条]{s.}{lenço; cachecol; echarpe; gravata; tiras longas de malha ou tecido usadas ao redor do pescoço para aquecimento, proteção do colarinho ou decoração}
\end{entry}

\begin{entry}{围绕}{wei2rao4}{7,9}{⼞、⽷}[HSK 5]
  \definition{v.}{girar; circundar; dar voltas; girar em torno de algo; cercar | concentrar-se em; centrar-se em; centrar-se em uma questão ou evento (para realizar atividades)}
\end{entry}

\begin{entry}{违法}{wei2 fa3}{7,8}{⾡、⽔}[HSK 5]
  \definition{v.}{ser ilegal; infringir a lei; violar a lei ou os regulamentos}
\end{entry}

\begin{entry}{违反}{wei2fan3}{7,4}{⾡、⼜}[HSK 5]
  \definition{v.}{violar; transgredir; contrariar; não estar em conformidade (com as regras, regulamentos, etc.)}
\end{entry}

\begin{entry}{违规}{wei2 gui1}{7,8}{⾡、⾒}[HSK 5]
  \definition{v.}{violar (regras); infringir as regras e regulamentos}
\end{entry}

\begin{entry}{违宪}{wei2xian4}{7,9}{⾡、⼧}
  \definition{adj.}{inconstitucional}
\end{entry}

\begin{entry}{唯一}{wei2yi1}{11,1}{⼝、⼀}[HSK 5]
  \definition{adj.}{único; exclusivo; singular}
\end{entry}

\begin{entry}{维持}{wei2chi2}{11,9}{⽷、⼿}[HSK 4]
  \definition{v.}{manter; conservar; guardar; manter vivo}
\end{entry}

\begin{entry}{维护}{wei2hu4}{11,7}{⽷、⼿}[HSK 4]
  \definition{v.}{defender; proteger; manter; preservar}
\end{entry}

\begin{entry}{维吾尔}{wei2wu2'er3}{11,7,5}{⽷、⼝、⼩}
  \definition*{s.}{Grupo étnico Uigur de Xinjiang}
\end{entry}

\begin{entry}{维修}{wei2xiu1}{11,9}{⽷、⼈}[HSK 4]
  \definition{v.}{prestar serviços; manter; reparar; manter em (bom) estado de conservação}
\end{entry}

\begin{entry}{伟}{wei3}{6}{⼈}
  \definition{adj.}{grande | ótimo}
\end{entry}

\begin{entry}{伟大}{wei3da4}{6,3}{⼈、⼤}[HSK 3]
  \definition{adj.}{ótimo; excelente; extrovertido; descreve uma pessoa com moral e qualidades excelentes, habilidades e realizações excepcionais, que inspira grande respeito | ótimo; magnífico; descreve algo de grande importância, com impacto significativo, acima do normal, algo notável}
\end{entry}

\begin{entry}{伪}{wei3}{6}{⼈}
  \definition{adj.}{falso; falsificado | fantoche; colaboracionista; ilegal | forjado; falso}
  \definition{pref.}{pseudo-; quasi-; quase-}
\end{entry}

\begin{entry}{尾巴}{wei3ba5}{7,4}{⼫、⼰}[HSK 4]
  \definition{s.}{cauda; projeções na extremidade do corpo de certos animais | parte semelhante a uma cauda; refere-se, em geral, ao final de algo | apêndice; anexo; adepto servil; pessoa que segue ou concorda com outra pessoa | (figura de linguagem) alguém que faz sombra a outro | fim; remanescente; parte restante (ou inacabada)}
\end{entry}

\begin{entry}{委内瑞拉}{wei3nei4rui4la1}{8,4,13,8}{⼥、⼌、⽟、⼿}
  \definition*{s.}{Venezuela}
\end{entry}

\begin{entry}{委托}{wei3tuo1}{8,6}{⼥、⼿}[HSK 5]
  \definition{v.}{confiar; confiar uma tarefa a outra pessoa ou instituição (para que seja realizada)}
\end{entry}

\begin{entry}{卫生}{wei4 sheng1}{3,5}{⼙、⽣}[HSK 3]
  \definition{adj.}{bom para a saúde; higiênico; limpo; capaz de prevenir doenças e benéfico para a saúde}
  \definition{s.}{higiene; saneamento; situação limpa}
\end{entry}

\begin{entry}{卫生部}{wei4sheng1bu4}{3,5,10}{⼙、⽣、⾢}
  \definition*{s.}{Ministério da Saúde}
\end{entry}

\begin{entry}{卫生防疫}{wei4sheng1 fang2yi4}{3,5,6,9}{⼙、⽣、⾩、⽧}
  \definition{s.}{prevenção contra a epidemia}
\end{entry}

\begin{entry}{卫生间}{wei4sheng1jian1}{3,5,7}{⼙、⽣、⾨}[HSK 3]
  \definition[间,个]{s.}{banheiro; sanitário; \emph{toilette}; quartos com instalações sanitárias em hotéis ou residências}
\end{entry}

\begin{entry}{卫生巾}{wei4sheng1jin1}{3,5,3}{⼙、⽣、⼱}
  \definition{s.}{absorvente higiênico}
\end{entry}

\begin{entry}{卫生局}{wei4sheng1ju2}{3,5,7}{⼙、⽣、⼫}
  \definition*{s.}{Departamento de Saúde | Escritório de Saúde}
\end{entry}

\begin{entry}{卫生棉}{wei4sheng1mian2}{3,5,12}{⼙、⽣、⽊}
  \definition{s.}{absorvente | algodão absorvente esterilizado (usado para curativos ou limpeza de feridas) | absorvente tampão}
\end{entry}

\begin{entry}{卫生球}{wei4sheng1qiu2}{3,5,11}{⼙、⽣、⽟}
  \definition{s.}{naftalina}
\end{entry}

\begin{entry}{卫生署}{wei4sheng1shu3}{3,5,13}{⼙、⽣、⽹}
  \definition*{s.}{Agência de Saúde (ou Escritório, ou Departamento)}
\end{entry}

\begin{entry}{卫生套}{wei4sheng1tao4}{3,5,10}{⼙、⽣、⼤}
  \definition[只]{s.}{preservativo | camisinha}
\end{entry}

\begin{entry}{卫生厅}{wei4sheng1ting1}{3,5,4}{⼙、⽣、⼚}
  \definition*{s.}{Departamento de Saúde (da província)}
\end{entry}

\begin{entry}{卫生纸}{wei4sheng1zhi3}{3,5,7}{⼙、⽣、⽷}
  \definition{s.}{papel higiênico}
\end{entry}

\begin{entry}{卫星}{wei4xing1}{3,9}{⼙、⽇}[HSK 5]
  \definition[个,颗]{s.}{satélite; lua; corpos celestes orbitando planetas | satélite artificial | algo que gira em torno de um centro}
\end{entry}

\begin{entry}{为}{wei4}{4}{⼂}[HSK 2,3]
  \definition*{s.}{sobrenome Wei}
  \definition{part.}{com 何 em uma pergunta retórica para expressar dúvida}
  \definition{prep.}{por; usado em frases passivas para introduzir o agente da ação, equivalente a 被 (frequentemente usado com 所)}
  \definition{suf.}{é anexado a alguns adjetivos ou advérbios monossilábicos para formar advérbios dissilábicos que expressam grau ou amplitude, geralmente modificando adjetivos ou verbos dissilábicos}
  \definition{v.}{fazer; agir | tornar-se; transformar-se em | ser;  significar | servir como; agir como; desempenhar o papel de | fazer; trabalhar; indica certas ações e comportamentos, incluindo os significados de governança, engajamento, cenário e pesquisa}
  \seeref{为}{wei2}
  \seealsoref{被}{bei4}
  \seealsoref{何}{he2}
  \seealsoref{所}{suo3}
\end{entry}

\begin{entry}{为了}{wei4le5}{4,2}{⼂、⼅}[HSK 3]
  \definition{prep.}{para; por causa de; a fim de; o objetivo da introdução de ações comportamentais}
\end{entry}

\begin{entry}{为什么}{wei4shen2me5}{4,4,3}{⼂、⼈、⼃}[HSK 2]
  \definition{adv.}{por que?; por que é que?; como é que?;  nota: 为什么不 geralmente tem o significado de conselho, o mesmo que 何不}
  \seealsoref{何不}{he2bu4}
\end{entry}

\begin{entry}{未}{wei4}{5}{⽊}
  \definition{adv.}{não ter | ainda não}
\end{entry}

\begin{entry}{未必}{wei4bi4}{5,5}{⽊、⼼}[HSK 4]
  \definition{adv.}{não tenho certeza; talvez não; não necessariamente}
\end{entry}

\begin{entry}{未来}{wei4lai2}{5,7}{⽊、⽊}[HSK 4]
  \definition{adj.}{próximo (refere-se ao tempo)}
  \definition[个]{s.}{futuro; o amanhã}
\end{entry}

\begin{entry}{位}{wei4}{7}{⼈}[HSK 2]
  \definition*{s.}{sobrenome Wei}
  \definition{clas.}{usado para pessoas (com cortesia, respeito) | usado para bits binários}[十六位___16 bits]
  \definition{s.}{lugar; localização; o lugar onde ou onde alguém está localizado | posto; \emph{status}; posição; a posição de uma pessoa em uma determinada área da vida social | trono; refere-se especificamente ao status do imperador | lugar; dígito; a posição de cada dígito em um número}
\end{entry}

\begin{entry}{位居}{wei4ju1}{7,8}{⼈、⼫}
  \definition{v.}{estar localizado em}
\end{entry}

\begin{entry}{位于}{wei4yu2}{7,3}{⼈、⼆}[HSK 4]
  \definition{v.}{estar localizado; estar situado}
\end{entry}

\begin{entry}{位置}{wei4zhi4}{7,13}{⼈、⽹}[HSK 4]
  \definition[通,个]{s.}{assento; lugar; localização | lugar; posição; \emph{status} | posição (por exemplo: cargo no escritório)}
\end{entry}

\begin{entry}{位子}{wei4zi5}{7,3}{⼈、⼦}
  \definition{s.}{lugar | assento}
\end{entry}

\begin{entry}{味}{wei4}{8}{⼝}
  \definition{clas.}{para medicamentos}
  \definition{s.}{cheiro | gosto}
\end{entry}

\begin{entry}{味道}{wei4dao5}{8,12}{⼝、⾡}[HSK 2]
  \definition[个,股,种]{s.}{gosto; sabor | sensação; gosto; experiência | interesse; deleite | cheiro; odor}
\end{entry}

\begin{entry}{味儿}{wei4r5}{8,2}{⼝、⼉}[HSK 4]
  \definition{s.}{gosto; sabor; propriedade de uma substância que dá à língua uma determinada sensação de sabor | cheiro; odor; propriedade de uma substância que dá ao nariz um determinado sentido de cheiro | interesse; significado; deleite}
\end{entry}

\begin{entry}{胃}{wei4}{9}{⾁}[HSK 5]
  \definition*{s.}{Wei, uma das mansões lunares; uma das vinte e oito constelações}
  \definition{s.}{estômago; parte do aparelho digestivo}
\end{entry}

\begin{entry}{胃口}{wei4kou3}{9,3}{⾁、⼝}
  \definition{s.}{apetite}
\end{entry}

\begin{entry}{喂}{wei4}{12}{⼝}[HSK 2,4]
  \definition{interj.}{Ei!, Olá!, para chamar atenção | Alô? (quando respondendo uma chamada telefônica, pronuncia-se como \dpy{wei2})}
  \definition{v.}{criar; alimentar (animais); dar comida a um animal | alimentar (pessoas); colocar alimentos, medicamentos, etc. na boca de alguém}
\end{entry}

\begin{entry}{喂哺}{wei4bu3}{12,10}{⼝、⼝}
  \definition{v.}{alimentar (um bebê)}
\end{entry}

\begin{entry}{喂料}{wei4liao4}{12,10}{⼝、⽃}
  \definition{v.}{alimentar (também no sentido figurativo)}
\end{entry}

\begin{entry}{喂母乳}{wei4mu3ru3}{12,5,8}{⼝、⽏、⼄}
  \definition{s.}{amamentação}
\end{entry}

\begin{entry}{喂奶}{wei4nai3}{12,5}{⼝、⼥}
  \definition{v.}{amamentar}
\end{entry}

\begin{entry}{喂食}{wei4shi2}{12,9}{⼝、⾷}
  \definition{v.}{alimentar}
\end{entry}

\begin{entry}{喂养}{wei4yang3}{12,9}{⼝、⼋}
  \definition{v.}{alimentar (uma criança, animal doméstico, etc.) | manter | criar (um animal)}
\end{entry}

\begin{entry}{慰问}{wei4wen4}{15,6}{⼼、⾨}[HSK 5]
  \definition{v.}{visitar; consolar; expressar simpatia por; confortar e cumprimentar com palavras e presentes;  enfatizar o conforto e o cumprimento, frequentemente usado por superiores para subordinados}
\end{entry}

\begin{entry}{温度}{wen1du4}{12,9}{⽔、⼴}[HSK 2]
  \definition[度,级,档,个]{s.}{temperatura}
\end{entry}

\begin{entry}{温度表}{wen1du4biao3}{12,9,8}{⽔、⼴、⾐}
  \definition{s.}{termômetro}
\end{entry}

\begin{entry}{温度计}{wen1du4ji4}{12,9,4}{⽔、⼴、⾔}
  \definition{s.}{termógrafo | termômetro}
\end{entry}

\begin{entry}{温度梯度}{wen1du4ti1du4}{12,9,11,9}{⽔、⼴、⽊、⼴}
  \definition{s.}{gradiente de temperatura}
\end{entry}

\begin{entry}{温和}{wen1he2}{12,8}{⽔、⼝}[HSK 5]
  \definition{adj.}{gentil; suave; moderado}
\end{entry}

\begin{entry}{温暖}{wen1nuan3}{12,13}{⽔、⽇}[HSK 3]
  \definition{adj.}{caloroso; gentil; amigável | caloroso; quente}
  \definition{v.}{aquecer; fazer com que se sinta calor}
\end{entry}

\begin{entry}{温柔}{wen1rou2}{12,9}{⽔、⽊}
  \definition{adj.}{gentil e suave | terno | doce (comumente usado para descrever uma menina ou mulher)}
\end{entry}

\begin{entry}{文化}{wen2hua4}{4,4}{⽂、⼔}[HSK 3]
  \definition[个,种]{s.}{cultura; civilização; tudo o que os seres humanos criaram em termos materiais e espirituais ao longo da história social | cultura; alfabetização; escolaridade; educação; o nível de conhecimento das pessoas e a capacidade de usar a linguagem escrita}
\end{entry}

\begin{entry}{文化层}{wen2hua4ceng2}{4,4,7}{⽂、⼔、⼫}
  \definition{s.}{nível de cultura (em sítio arqueológico)}
\end{entry}

\begin{entry}{文化宫}{wen2hua4gong1}{4,4,9}{⽂、⼔、⼧}
  \definition{s.}{palácio cultural}
\end{entry}

\begin{entry}{文化圈}{wen2hua4quan1}{4,4,11}{⽂、⼔、⼞}
  \definition{s.}{esfera de influência cultural}
\end{entry}

\begin{entry}{文化热}{wen2hua4re4}{4,4,10}{⽂、⼔、⽕}
  \definition{s.}{mania cultural | febre cultural}
\end{entry}

\begin{entry}{文化史}{wen2hua4shi3}{4,4,5}{⽂、⼔、⼝}
  \definition*{s.}{História Cultural}
\end{entry}

\begin{entry}{文化水平}{wen2hua4 shui3ping2}{4,4,4,5}{⽂、⼔、⽔、⼲}
  \definition{s.}{nível educacional}
\end{entry}

\begin{entry}{文化障碍}{wen2hua4zhang4'ai4}{4,4,13,13}{⽂、⼔、⾩、⽯}
  \definition{s.}{barreira cultural}
\end{entry}

\begin{entry}{文件}{wen2jian4}{4,6}{⽂、⼈}[HSK 3]
  \definition[份,堆,叠]{s.}{documentos oficiais; papéis; instrumentos; termo genérico para documentos oficiais, cartas, etc. | os arquivos no computador; informações registradas no celular ou computador | artigos ou trabalhos sobre teorias políticas, atualidades, pesquisas acadêmicas, etc.; textos ou artigos sobre teoria política, políticas, etc.}
\end{entry}

\begin{entry}{文明}{wen2ming2}{4,8}{⽂、⽇}[HSK 3]
  \definition{adj.}{civilizado; sociedade desenvolvida e com alto nível cultural}
  \definition[个,种]{s.}{cultura; civilização}
\end{entry}

\begin{entry}{文学}{wen2xue2}{4,8}{⽂、⼦}[HSK 3]
  \definition[个,种]{s.}{literatura; a arte de moldar imagens e refletir a vida social através da linguagem e da escrita, incluindo romances, poesia, prosa, teatro, etc.}
\end{entry}

\begin{entry}{文学系}{wen2xue2 xi4}{4,8,7}{⽂、⼦、⽷}
  \definition*{s.}{Faculdade de Letras}
\end{entry}

\begin{entry}{文艺}{wen2yi4}{4,4}{⽂、⾋}[HSK 5]
  \definition{s.}{termo genérico para literatura e arte | performance (arte); refere-se especificamente às artes performativas, como música e dança}
\end{entry}

\begin{entry}{文章}{wen2zhang1}{4,11}{⽂、⾳}[HSK 3]
  \definition[篇,段,页,系列]{s.}{ensaio; artigo; texto independente; também se refere a obras literárias em geral | significado oculto; significado implícito | trabalho; coisas que podem ser feitas}
\end{entry}

\begin{entry}{文字}{wen2zi4}{4,6}{⽂、⼦}[HSK 3]
  \definition[种,类,段,行,篇]{s.}{caracteres; caligrafia; escrita; símbolos escritos para registrar a linguagem| linguagem escrita; a forma escrita da língua}
\end{entry}

\begin{entry}{纹路}{wen2lu4}{7,13}{⽷、⾜}
  \definition{s.}{padrão de linhas | rugas | veias | veias (em mármore ou impressão digital) | grãos (em madeira, etc.)}
\end{entry}

\begin{entry}{闻}{wen2}{9}{⾨}[HSK 2]
  \definition*{s.}{sobrenome Wen}
  \definition{adj.}{bem conhecido; famoso}
  \definition{s.}{notícia; história | reputação | boato; rumor}
  \definition{v.}{cheirar | ouvir}
\end{entry}

\begin{entry}{蚊香}{wen2xiang1}{10,9}{⾍、⾹}
  \definition{s.}{incenso ou espiral repelente de mosquitos}
\end{entry}

\begin{entry}{蚊子}{wen2zi5}{10,3}{⾍、⼦}
  \definition{s.}{pernilongo}
\end{entry}

\begin{entry}{稳}{wen3}{14}{⽲}[HSK 4]
  \definition{adj.}{constante; estável; firme | estável; estático; sedado | seguro; confiável; certo}
  \definition{adv.}{certamente; com certeza; seguramente; sem dúvida}
  \definition{v.}{estabilizar, manter estável}
\end{entry}

\begin{entry}{稳定}{wen3ding4}{14,8}{⽲、⼧}[HSK 4]
  \definition{adj.}{estável; firme; descreve uma natureza, um estado, etc. relativamente fixo; não muda significativamente}
  \definition{s.}{estabilidade}
  \definition{v.}{manter estável; estabilizar; liquidar; resolver a situação}
\end{entry}

\begin{entry}{问}{wen4}{6}{⾨}[HSK 1]
  \definition*{s.}{sobrenome Wen}
  \definition{prep.}{de; introduzir o objeto da ação, equivalente a 向 e 跟}
  \definition{v.}{perguntar; indagar; fazer com que as pessoas respondam ou esclareçam coisas que não sabem ou não têm certeza | perguntar (ou indagar) sobre | examinar; interrogar | intervir; responsabilizar; investigar | cuidar; preocupar-se; gerenciar; interferir}
  \seealsoref{跟}{gen1}
  \seealsoref{向}{xiang4}
\end{entry}

\begin{entry}{问安}{wen4'an1}{6,6}{⾨、⼧}
  \definition{s.}{saudações}
  \definition{v.}{dar cumprimentos a | prestar homenagem}
\end{entry}

\begin{entry}{问鼎}{wen4ding3}{6,12}{⾨、⿍}
  \definition{v.}{visar (o primeiro lugar, etc.) | aspirar ao trono}
\end{entry}

\begin{entry}{问候}{wen4hou4}{6,10}{⾨、⼈}[HSK 4]
  \definition{s.}{homenagem | saudação}
  \definition{v.}{prestar homenagem; enviar uma saudação;  dar os respeitos (cumprimentos) a alguém | (fig.) (coloquial) fazer referência ofensiva a (alguém querido pela pessoa com quem se está falando)}
\end{entry}

\begin{entry}{问卷}{wen4juan4}{6,8}{⾨、⼙}
  \definition[份]{s.}{questionário}
\end{entry}

\begin{entry}{问路}{wen4 lu4}{6,13}{⾨、⾜}[HSK 2]
  \definition{v.}{perguntar o caminho; pedir direções}
\end{entry}

\begin{entry}{问市}{wen4shi4}{6,5}{⾨、⼱}
  \definition{v.}{chegar ao mercado | bater o mercado | atingir o mercado}
\end{entry}

\begin{entry}{问题}{wen4ti2}{6,15}{⾨、⾴}[HSK 2]
  \definition{adj.}{desqualificado; indesejável; anormal, não atende aos requisitos}
  \definition[个,种,类,串]{s.}{pergunta; problema; perguntas a serem respondidas | problema; questão; contradições que precisam ser estudadas e resolvidas | problema; acidente; incidente | chave; ponto crucial; pontos importantes}
\end{entry}

\begin{entry}{嗡嗡}{weng1weng1}{13,13}{⼝、⼝}
  \definition{s.}{zumbido}
  \definition{v.}{zumbir}
\end{entry}

\begin{entry}{蕹菜}{weng4cai4}{16,11}{⾋、⾋}
  \definition{s.}{espinafre aquático | \emph{ong choy} | repolho do pântano | convolvulus aquático | glória-da-manhã aquática}
  \seealsoref{空心菜}{kong1xin1cai4}
\end{entry}

\begin{entry}{我}{wo3}{7}{⼽}[HSK 1]
  \definition{pron.}{eu; mim | um; qualquer um; usado para contrastar 他 e 我; refere-se a muitas pessoas em geral}
  \seealsoref{他}{ta1}
\end{entry}

\begin{entry}{我的}{wo3 de5}{7,8}{⼽、⽩}
  \definition{pron.}{meu, meus}
\end{entry}

\begin{entry}{我们}{wo3men5}{7,5}{⼽、⼈}[HSK 1]
  \definition{pron.}{nós; nos}
\end{entry}

\begin{entry}{我们的}{wo3men5 de5}{7,5,8}{⼽、⼈、⽩}
  \definition{pron.}{nosso, nossos}
\end{entry}

\begin{entry}{我去}{wo3qu4}{7,5}{⼽、⼛}
  \definition{interj.}{(gíria) O que\dots!! | Oh meu Deus! | Isso é insano!}
\end{entry}

\begin{entry}{卧}{wo4}{8}{⾂}
  \definition{v.}{agachar | deitar}
\end{entry}

\begin{entry}{卧病}{wo4bing4}{8,10}{⾂、⽧}
  \definition{s.}{acamado | doente na cama}
\end{entry}

\begin{entry}{卧舱}{wo4cang1}{8,10}{⾂、⾈}
  \definition{s.}{cabine de dormir em um barco ou trem}
\end{entry}

\begin{entry}{卧车}{wo4che1}{8,4}{⾂、⾞}
  \definition{s.}{um carro-leito | vagão-leito}
\end{entry}

\begin{entry}{卧床}{wo4chuang2}{8,7}{⾂、⼴}
  \definition{adj.}{acamado}
  \definition{s.}{cama}
  \definition{v.}{deitar na cama}
\end{entry}

\begin{entry}{卧倒}{wo4dao3}{8,10}{⾂、⼈}
  \definition{v.}{cair no chão | deitar-se}
\end{entry}

\begin{entry}{卧式}{wo4shi4}{8,6}{⾂、⼷}
  \definition{adj.}{horizontal}
\end{entry}

\begin{entry}{卧室}{wo4shi4}{8,9}{⾂、⼧}[HSK 5]
  \definition[间,个]{s.}{quarto de dormir; quarto de uma casa usado para dormir}
\end{entry}

\begin{entry}{卧榻}{wo4ta4}{8,14}{⾂、⽊}
  \definition{s.}{um sofá | uma cama estreita}
\end{entry}

\begin{entry}{卧推}{wo4tui1}{8,11}{⾂、⼿}
  \definition{s.}{supino}
\end{entry}

\begin{entry}{握}{wo4}{12}{⼿}[HSK 5]
  \definition{v.}{segurar; agarrar | agarrar; segurar; empunhar; controlar | pegar pela mão}
\end{entry}

\begin{entry}{握手}{wo4shou3}{12,4}{⼿、⼿}[HSK 3]
  \definition{v.+compl.}{apertar as mãos; dar um aperto de mão; estender a mão e apertar a mão do outro é uma forma de saudação ao se encontrar ou se despedir, e também é usado para expressar felicitações ou condolências}
\end{entry}

\begin{entry}{斡旋}{wo4xuan2}{14,11}{⽃、⽅}
  \definition{v.}{mediar (um conflito, etc.)}
\end{entry}

\begin{entry}{乌龟}{wu1gui1}{4,7}{⼃、⿔}
  \definition{s.}{tartaruga}
\end{entry}

\begin{entry}{乌克兰}{wu1ke4lan2}{4,7,5}{⼃、⼗、⼋}
  \definition*{s.}{Ucrânia}
\end{entry}

\begin{entry}{污染}{wu1ran3}{6,9}{⽔、⽊}[HSK 5]
  \definition{s.}{poluição}
  \definition{v.}{poluir; contaminar com substâncias nocivas e prejudiciais; refere-se especificamente à destruição do ambiente natural causada por substâncias nocivas, tais como gases, líquidos e resíduos emitidos por indústrias, minas, veículos, etc. | contaminar; metáfora de que pensamentos prejudiciais causam efeitos negativos nas pessoas}
\end{entry}

\begin{entry}{污染区}{wu1ran3qu1}{6,9,4}{⽔、⽊、⼖}
  \definition{s.}{área contaminada}
\end{entry}

\begin{entry}{污染物}{wu1ran3wu4}{6,9,8}{⽔、⽊、⽜}
  \definition{s.}{poluente}
  \seealsoref{污染物质}{wu1ran3 wu4zhi4}
\end{entry}

\begin{entry}{污染物质}{wu1ran3 wu4zhi4}{6,9,8,8}{⽔、⽊、⽜、⾙}
  \definition{s.}{poluente}
  \seealsoref{污染物}{wu1ran3wu4}
\end{entry}

\begin{entry}{污水}{wu1shui3}{6,4}{⽔、⽔}[HSK 5]
  \definition{s.}{água suja (ou poluída, residual); esgoto; lodo | efluente; drenagem; água suja; água poluída; água residual}
\end{entry}

\begin{entry}{屋}{wu1}{9}{⼫}[HSK 5]
  \definition[间,座]{s.}{casa | quarto}
\end{entry}

\begin{entry}{屋子}{wu1zi5}{9,3}{⼫、⼦}[HSK 3]
  \definition[间,座,栋]{s.}{quarto; sala}
\end{entry}

\begin{entry}{无}{wu2}{4}{⽆}[HSK 4][Kangxi 71]
  \definition{adv.}{não; não ter algo; não há\dots}
  \definition{conj.}{independentemente de; não importa se, o que, etc.}
  \definition{v.}{não ter; estar sem; não existir;}
\end{entry}

\begin{entry}{无敌}{wu2di2}{4,10}{⽆、⾆}
  \definition{adj.}{invencível | inigualável}
\end{entry}

\begin{entry}{无法}{wu2 fa3}{4,8}{⽆、⽔}[HSK 4]
  \definition{adj.}{incapaz; incapacitado}
  \definition{v.}{não há nada a ser feito}
\end{entry}

\begin{entry}{无骨}{wu2 gu3}{4,9}{⽆、⾻}
  \definition{adj.}{desossado}
\end{entry}

\begin{entry}{无故}{wu2gu4}{4,9}{⽆、⽁}
  \definition{adv.}{sem causa ou razão | sem motivo}
\end{entry}

\begin{entry}{无聊}{wu2liao2}{4,11}{⽆、⽿}[HSK 4]
  \definition{adj.}{entediado; aborrecido; sentir-se desinteressado porque não há nada para fazer | tolo; bobo; sem sentido; descreve palavras ou coisas ditas ou feitas como sem sentido e irritantes; descreve pessoas ou coisas como sem sentido e pouco atraentes}
\end{entry}

\begin{entry}{无论}{wu2lun4}{4,6}{⽆、⾔}[HSK 4]
  \definition{conj.}{não importa o quê; não importa como; independentemente de; indica que as condições são diferentes, mas resultado é o mesmo |}
  \seealsoref{无论……也……}{wu2lun4 ye3}
\end{entry}

\begin{entry}{无论……也……}{wu2lun4 ye3}{4,6,3}{⽆、⾔、⼄}
  \definition{conj.}{não apenas\dots, (o que, quem, como, etc.), \dots}
\end{entry}

\begin{entry}{无奈}{wu2nai4}{4,8}{⽆、⼤}[HSK 5]
  \definition{conj.}{mas (infelizmente); no entanto}
  \definition{v.}{não poder evitar; não ter alternativa; não ter escolha; não haver nada a fazer}
\end{entry}

\begin{entry}{无人}{wu2ren2}{4,2}{⽆、⼈}
  \definition{adj.}{não tripulado | desabitado}
\end{entry}

\begin{entry}{无人机}{wu2ren2ji1}{4,2,6}{⽆、⼈、⽊}
  \definition{s.}{\emph{drone} | veículo aéreo não tripulado}
\end{entry}

\begin{entry}{无视}{wu2shi4}{4,8}{⽆、⾒}
  \definition{v.}{ignorar | desconsiderar}
\end{entry}

\begin{entry}{无数}{wu2shu4}{4,13}{⽆、⽁}[HSK 4]
  \definition{adj.}{incontável; inumerável | inseguro; incerto; não conhecer a história ou os detalhes internos; não ter certeza}
\end{entry}

\begin{entry}{无所谓}{wu2suo3wei4}{4,8,11}{⽆、⼾、⾔}[HSK 4]
  \definition{v.}{não pode ser designado como; não merece o nome de; ser incapaz de dizer ou contar | não ter importância; ser indiferente;}
\end{entry}

\begin{entry}{无限}{wu2 xian4}{4,8}{⽆、⾩}[HSK 4]
  \definition{adj.}{infinito; ilimitado; sem limites; sem fim à vista}
\end{entry}

\begin{entry}{无氧}{wu2yang3}{4,10}{⽆、⽓}
  \definition{adj.}{anaeróbico}
\end{entry}

\begin{entry}{无疑}{wu2 yi2}{4,14}{⽆、⽦}[HSK 5]
  \definition{adv.}{indubitavelmente; sem dúvida; sem sombra de dúvida}
\end{entry}

\begin{entry}{吾}{wu2}{7}{⼝}
  \definition*{s.}{sobrenome Wu}
  \definition{pron.}{eu | (antigo) meu}
\end{entry}

\begin{entry}{五}{wu3}{4}{⼆}[HSK 1]
  \definition*{s.}{sobrenome Wu}
  \definition{num.}{cinco; 5}
  \definition{s.}{uma nota da escala em Gongchepu (工尺谱), correspondente a 6 na notação musical numerada}
  \seealsoref{工尺谱}{gong1 che3 pu3}
\end{entry}

\begin{entry}{五体投地}{wu3ti3tou2di4}{4,7,7,6}{⼆、⼈、⼿、⼟}
  \definition{expr.}{prostrar-se em admiração | adular alguém}
\end{entry}

\begin{entry}{五五}{wu3wu3}{4,4}{⼆、⼆}
  \definition{num.}{50-50}
  \definition{s.}{igual (partilha, parceria, etc.)}
\end{entry}

\begin{entry}{五颜六色}{wu3 yan2 liu4 se4}{4,15,4,6}{⼆、⾴、⼋、⾊}[HSK 4]
  \definition{adj.}{todas as cores sob o sol; multicolorido; colorido}
\end{entry}

\begin{entry}{午}{wu3}{4}{⼗}
  \definition{s.}{período entre 11h00 e 13h00, meio-dia}
\end{entry}

\begin{entry}{午餐}{wu3 can1}{4,16}{⼗、⾷}[HSK 2]
  \definition[份,顿,次]{s.}{almoço}
  \seealsoref{午饭}{wu3 fan4}
\end{entry}

\begin{entry}{午饭}{wu3 fan4}{4,7}{⼗、⾷}[HSK 1]
  \definition[顿]{s.}{almoço}
  \seealsoref{午餐}{wu3 can1}
\end{entry}

\begin{entry}{午后}{wu3hou4}{4,6}{⼗、⼝}
  \definition{s.}{tarde | período da tarde}
\end{entry}

\begin{entry}{午前}{wu3qian2}{4,9}{⼗、⼑}
  \definition{s.}{\emph{A.M.} | manhã | período da manhã}
\end{entry}

\begin{entry}{午睡}{wu3 shui4}{4,13}{⼗、⽬}[HSK 2]
  \definition{s.}{\emph{siesta}; cochilo da tarde; soneca do meio-dia}
  \definition{v.}{tirar uma soneca depois do almoço}
\end{entry}

\begin{entry}{午休}{wu3xiu1}{4,6}{⼗、⼈}
  \definition{s.}{pausa para almoço | cochilo na hora do almoço | intervalo do meio-dia}
\end{entry}

\begin{entry}{午宴}{wu3yan4}{4,10}{⼗、⼧}
  \definition{s.}{banquete de almoço}
\end{entry}

\begin{entry}{午夜}{wu3ye4}{4,8}{⼗、⼣}
  \definition{s.}{meia-noite}
\end{entry}

\begin{entry}{武}{wu3}{8}{⽌}
  \definition*{s.}{sobrenome Wu}
  \definition{s.}{arte marcial}
\end{entry}

\begin{entry}{武大戏}{wu3 da4xi4}{8,3,6}{⽌、⼤、⼽}
  \definition*{s.}{Drama de Luta Acrobática | Drama Wu}
\end{entry}

\begin{entry}{武断}{wu3duan4}{8,11}{⽌、⽄}
  \definition{adj.}{arbitrário | dogmático | subjetivo}
\end{entry}

\begin{entry}{武官}{wu3guan1}{8,8}{⽌、⼧}
  \definition{s.}{oficial militar}
\end{entry}

\begin{entry}{武力}{wu3li4}{8,2}{⽌、⼒}
  \definition{s.}{forças armadas | militares}
\end{entry}

\begin{entry}{武器}{wu3qi4}{8,16}{⽌、⼝}[HSK 3]
  \definition[批,个,件,种]{s.}{arma; equipamentos e dispositivos utilizados diretamente para matar inimigos ou destruir suas instalações defensivas e ofensivas | armas; armamento; metáfora usada como ferramenta de luta}
\end{entry}

\begin{entry}{武士}{wu3shi4}{8,3}{⽌、⼠}
  \definition{s.}{samurai | guerreiro}
\end{entry}

\begin{entry}{武术}{wu3shu4}{8,5}{⽌、⽊}[HSK 3]
  \definition[种,套,门]{s.}{arte marcial; autodefesa; \emph{wushu}; um esporte tradicional chinês que utiliza técnicas com os punhos, pernas, pés ou armas como facas e espadas}
\end{entry}

\begin{entry}{武艺}{wu3yi4}{8,4}{⽌、⾋}
  \definition{s.}{arte marcial | habilidade militar}
\end{entry}

\begin{entry}{武装}{wu3zhuang1}{8,12}{⽌、⾐}
  \definition{s.}{forças armadas | militar | arma}
  \definition{v.}{armar}
\end{entry}

\begin{entry}{舞}{wu3}{14}{⾇}[HSK 5]
  \definition{s.}{dança | palco; metáfora do domínio das atividades sociais}
  \definition{v.}{mover-se como numa dança | dançar com algo nas mãos; brincar com | florescer; empunhar; brandir | esvoaçar | fazer malabarismos; brincar com}
\end{entry}

\begin{entry}{舞抃}{wu3bian4}{14,7}{⾇、⼿}
  \definition{s.}{dançar por prazer}
\end{entry}

\begin{entry}{舞蹈}{wu3dao3}{14,17}{⾇、⾜}
  \definition{s.}{dança (ato performático)}
\end{entry}

\begin{entry}{舞会}{wu3hui4}{14,6}{⾇、⼈}
  \definition{s.}{baile}
\end{entry}

\begin{entry}{舞会舞}{wu3hui4wu3}{14,6,14}{⾇、⼈、⾇}
  \definition{s.}{baile}
\end{entry}

\begin{entry}{舞台}{wu3 tai2}{14,5}{⾇、⼝}[HSK 3]
  \definition[个]{s.}{palco; plataforma elevada usada exclusivamente para apresentações artísticas, geralmente localizada na parte frontal de teatros e auditórios | palco; metáfora do campo das atividades sociais}
\end{entry}

\begin{entry}{舞厅}{wu3ting1}{14,4}{⾇、⼚}
  \definition[间]{s.}{salão de dança | salão de baile}
\end{entry}

\begin{entry}{舞厅舞}{wu3ting1wu3}{14,4,14}{⾇、⼚、⾇}
  \definition{s.}{dança de salão}
\end{entry}

\begin{entry}{务必}{wu4bi4}{5,5}{⼒、⼼}
  \definition{adv.}{deve; ter certeza de; necessariamente; usado principalmente em frases afirmativas}
\end{entry}

\begin{entry}{务实}{wu4shi2}{5,8}{⼒、⼧}
  \definition{adj.}{pragmático}
  \definition{v.}{lidar com assuntos concretos}
\end{entry}

\begin{entry}{物价}{wu4 jia4}{8,6}{⽜、⼈}[HSK 5]
  \definition[个]{s.}{preços das commodities; preços das matérias-primas; preço das mercadorias}
\end{entry}

\begin{entry}{物理}{wu4li3}{8,11}{⽜、⽟}
  \definition{s.}{física (disciplina)}
\end{entry}

\begin{entry}{物业}{wu4ye4}{8,5}{⽜、⼀}[HSK 5]
  \definition[处]{s.}{propriedade; gestão de propriedades; gestão patrimonial; administração de imóveis | empresa de administração de imóveis; empresa de gestão imobiliária; empresa de administração de bens imóveis}
\end{entry}

\begin{entry}{物质}{wu4zhi4}{8,8}{⽜、⾙}[HSK 5]
  \definition[个]{s.}{matéria; substância; algo que existe além do espírito, que pode ser visto, tocado, cheirado ou detectado por instrumentos científicos | material; meios de subsistência; coisas que permitem às pessoas viver ou viver melhor, como comida, roupas, casas, dinheiro, etc.}
\end{entry}

\begin{entry}{误点}{wu4dian3}{9,9}{⾔、⽕}
  \definition{v.+compl.}{atrasar | chegar tarde}
\end{entry}

\begin{entry}{误会}{wu4hui4}{9,6}{⾔、⼈}
  \definition[场]{s.}{mal-entendido; desentendimentos ou conflitos decorrentes de mal-entendidos}
  \definition{v.}{entender mal; entender errado; interpretar mal; não entender; não entender corretamente o significado}
\end{entry}

\begin{entry}{误解}{wu4jie3}{9,13}{⾔、⾓}[HSK 5]
  \definition[种]{s.}{equívoco; mal-entendido; desentendimento}
  \definition{v.}{interpretar mal; interpretar erroneamente; não compreender corretamente}
\end{entry}

\begin{entry}{恶}{wu4}{10}{⼼}
  \definition{v.}{não gostar; odiar; detestar; repugnar}
\end{entry}

\begin{entry}{雾气}{wu4qi4}{13,4}{⾬、⽓}
  \definition{s.}{nevoeiro | névoa | vapor}
\end{entry}

%%%%% EOF %%%%%

