%%%
%%% F
%%%

\section*{F}\addcontentsline{toc}{section}{F}

\begin{entry}{发}{fa1}{5}[Radical ⼜][HSK 2]
  \definition{clas.}{para tiros (rodadas)}
  \definition{v.}{enviar | mandar}
  \seeref{发}{fa4}
\end{entry}

\begin{entry}{发表}{fa1biao3}{5,8}[HSK 3]
  \definition{v.}{publicar; entregar; emitir; expressar; anunciar | publicar}
\end{entry}

\begin{entry}{发财}{fa1cai2}{5,7}
  \definition{v.+compl.}{ficar rico | fazer fortuna}
\end{entry}

\begin{entry}{发愁}{fa1chou2}{5,13}
  \definition{v.+compl.}{preocupar-se | ficar ansioso | ficar triste}
\end{entry}

\begin{entry}{发出}{fa1 chu1}{5,5}[HSK 3]
  \definition{v.}{fazer; produzir; deixar sair | emitir; anunciar | enviar; partir | dar; emitir}
\end{entry}

\begin{entry}{发达}{fa1da2}{5,6}[HSK 3]
  \definition{adj.}{desenvolvido; florescente}
  \definition{v.}{desenvolver; promover; florescer}
\end{entry}

\begin{entry}{发动}{fa1dong4}{5,6}[HSK 3]
  \definition{v.}{iniciar; lançar; ligar motor; dar a partida (motor de combustão interna) | chamar à ação; mobilizar; estimular; despertar}
\end{entry}

\begin{entry}{发动机}{fa1dong4ji1}{5,6,6}
  \definition[台]{s.}{motor}
\end{entry}

\begin{entry}{发抖}{fa1dou3}{5,7}
  \definition{v.}{tremer | sacudir | estremecer}
\end{entry}

\begin{entry}{发明}{fa1ming2}{5,8}[HSK 3]
  \definition[个]{s.}{invenção}
  \definition{v.}{inventar | expor; explicar}
\end{entry}

\begin{entry}{发明者}{fa1ming2zhe3}{5,8,8}
  \definition{s.}{inventor}
\end{entry}

\begin{entry}{发票}{fa1piao4}{5,11}
  \definition{s.}{fatura | recibo | conta}
\end{entry}

\begin{entry}{发烧}{fa1shao1}{5,10}
  \definition{v.}{ter febre}
\end{entry}

\begin{entry}{发生}{fa1sheng1}{5,5}[HSK 3]
  \definition{v.}{ocorrer; acontecer; tomar lugar}
\end{entry}

\begin{entry}{发送}{fa1 song4}{5,9}[HSK 3]
  \definition{v.}{enviar; despachar | transmitir; enviar}
\end{entry}

\begin{entry}{发现}{fa1xian4}{5,8}[HSK 2]
  \definition{s.}{descoberta}
  \definition{v.}{perceber, tornar-se ciente de | descobrir, encontrar, detectar}
\end{entry}

\begin{entry}{发现者}{fa1xian4 zhe3}{5,8,8}
  \definition{s.}{descobridor}
\end{entry}

\begin{entry}{发言}{fa1yan2}{5,7}[HSK 3]
  \definition[个]{s.}{discurso; declaração; palestra}
  \definition{v.+compl.}{falar; fazer uma declaração (discurso)}
\end{entry}

\begin{entry}{发音}{fa1yin1}{5,9}
  \definition{s.}{pronúncia}
  \definition{v.}{pronunciar}
\end{entry}

\begin{entry}{发展}{fa1zhan3}{5,10}[HSK 3]
  \definition{s.}{desenvolvimento}
  \definition{v.}{crescer; expandir; avançar; desenvolver | recrutar; expandir; admitir}
\end{entry}

\begin{entry}{罚}{fa2}{9}[Radical 网]
  \definition{v.}{castigar | punir}
\end{entry}

\begin{entry}{罚款}{fa2kuan3}{9,12}
  \definition{s.}{multa (monetária) | pena}
  \definition{v.+compl.}{aplicar uma multa | multar}
\end{entry}

\begin{entry}{筏}{fa2}{12}[Radical 竹]
  \definition{s.}{jangada (de troncos, bambus, etc.)}
\end{entry}

\begin{entry}{法}{fa3}{8}[Radical 水]
  \definition*{s.}{França, abreviação de~法国}
  \seealsoref{法国}{fa3guo2}
\end{entry}

\begin{entry}{法国}{fa3guo2}{8,8}
  \definition*{s.}{França}
\end{entry}

\begin{entry}{法国人}{fa3guo2ren2}{8,8,2}
  \definition{s.}{francês | pessoa ou povo da França}
\end{entry}

\begin{entry}{法网}{fa3wang3}{8,6}
  \definition*{s.}{Torneio de Roland Garros (French Open), torneio de tênis}
\end{entry}

\begin{entry}{法文}{fa3wen2}{8,4}
  \definition*{s.}{françês, língua francesa}
\end{entry}

\begin{entry}{法语}{fa3yu3}{8,9}
  \definition{s.}{françês, língua francesa}
\end{entry}

\begin{entry}{发}{fa4}{5}[Radical ⼜]
  \definition{s.}{cabelo}
  \seeref{发}{fa1}
\end{entry}

\begin{entry}{发型}{fa4xing2}{5,9}
  \definition{s.}{penteado}
\end{entry}

\begin{entry}{发簪}{fa4zan1}{5,18}
  \definition{s.}{grampo de cabelo}
\end{entry}

\begin{entry}{番茄}{fan1qie2}{12,8}
  \definition{s.}{tomate}
\end{entry}

\begin{entry}{蕃茄}{fan1qie2}{15,8}
  \variantof{番茄}
\end{entry}

\begin{entry}{翻过}{fan1guo4}{18,6}
  \definition{v.}{virar |  transformar}
\end{entry}

\begin{entry}{翻脸}{fan1lian3}{18,11}
  \definition{v.+compl.}{brigar com alguém | tornar-se hostil}
\end{entry}

\begin{entry}{翻译}{fan1yi4}{18,7}
  \definition[个,位,名]{s.}{tradução | tradutor | interpretação | intérprete}
  \definition{v.}{traduzir; interpretar}
\end{entry}

\begin{entry}{反对}{fan3dui4}{4,5}[HSK 3]
  \definition{v.}{lutar; opor-se; objetar a; ser contra}
\end{entry}

\begin{entry}{反对党}{fan3dui4dang3}{4,5,10}
  \definition{s.}{partido de oposição}
\end{entry}

\begin{entry}{反对派}{fan3dui4pai4}{4,5,9}
  \definition{s.}{facção de oposição}
\end{entry}

\begin{entry}{反对票}{fan3dui4piao4}{4,5,11}
  \definition{s.}{voto dissidente}
\end{entry}

\begin{entry}{反复}{fan3fu4}{4,9}[HSK 3]
  \definition{adv.}{repetidamente; de ​​novo e de novo}
  \definition{s.}{reversão; recaída}
  \definition{v.}{recuar; cortar e mudar}
\end{entry}

\begin{entry}{反省}{fan3xing3}{4,9}
  \definition{v.}{examinar a consciência | questionar-se | refletir sobre si mesmo | sondar a alma}
\end{entry}

\begin{entry}{反应}{fan3ying4}{4,7}[HSK 3]
  \definition[个]{s.}{reação; resposta}
  \definition{v.}{reagir; responder}
\end{entry}

\begin{entry}{反正}{fan3zheng4}{4,5}[HSK 3]
  \definition{adv.}{de qualquer forma | tudo igual; em qualquer caso}
\end{entry}

\begin{entry}{犯法}{fan4fa3}{5,8}
  \definition{v.}{violar (quebrar) a lei}
\end{entry}

\begin{entry}{犯罪}{fan4zui4}{5,13}
  \definition{v.+compl.}{cometer  um crime (uma ofensa)}
\end{entry}

\begin{entry}{饭}{fan4}{7}[Radical 食][HSK 1]
  \definition[碗]{s.}{arroz cozido}
  \definition[顿]{s.}{refeição}
  \definition{s.}{(empréstimo linguístico) fã, devoto}
\end{entry}

\begin{entry}{饭店}{fan4dian4}{7,8}[HSK 1]
  \definition[家,个]{s.}{restaurante | hotel}
\end{entry}

\begin{entry}{饭馆}{fan4 guan3}{7,11}[HSK 2]
  \definition[家,个]{s.}{restaurante | lanchonete}
\end{entry}

\begin{entry}{范围}{fan4wei2}{9,7}[HSK 3]
  \definition[个]{s.}{escopo; limite; alcance}
  \definition{v.}{estabelecer limites para; limitar o escopo de}
\end{entry}

\begin{entry}{方案}{fang1'an4}{4,10}
  \definition[个,套]{s.}{plano | programa (para uma ação, etc.) | proposta | proposta de projeto de lei}
\end{entry}

\begin{entry}{方便}{fang1bian4}{4,9}[HSK 2]
  \definition{adj.}{conveniente | adequado}
  \definition{v.}{facilitar, facilitar as coisas | ter dinheiro de sobra | (eufemismo) aliviar-se}
\end{entry}

\begin{entry}{方便面}{fang1 bian4 mian4}{4,9,9}[HSK 2]
  \definition{s.}{macarrão instantâneo}
\end{entry}

\begin{entry}{方法}{fang1fa3}{4,8}[HSK 2]
  \definition[个]{s.}{método | meio}
\end{entry}

\begin{entry}{方面}{fang1mian4}{4,9}[HSK 2]
  \definition[个]{s.}{lado | campo | aspecto}
\end{entry}

\begin{entry}{方式}{fang1shi4}{4,6}[HSK 3]
  \definition[种,个]{s.}{maneira; método}
\end{entry}

\begin{entry}{方向}{fang1xiang4}{4,6}[HSK 2]
  \definition[个]{s.}{direção | orientação | alvo | meta | objetivo}
\end{entry}

\begin{entry}{方言}{fang1yan2}{4,7}
  \definition*{s.}{o primeiro dicionário de dialeto chinês, editado por Yang Xiong 扬雄 no século I, contendo mais de 9.000 caracteres}
  \definition{s.}{dialeto}
  \seealsoref{扬雄}{yang2xiong2}
\end{entry}

\begin{entry}{防}{fang2}{6}[HSK 3]
  \definition*{s.}{sobrenome Fang}
  \definition{s.}{defesa | barragem; dique; aterro}
  \definition{v.}{prover contra; defender contra; proteger contra}
\end{entry}

\begin{entry}{防护}{fang2hu4}{6,7}
  \definition{v.}{defender | proteger}
\end{entry}

\begin{entry}{防晒}{fang2shai4}{6,10}
  \definition{s.}{protetor solar}
\end{entry}

\begin{entry}{防止}{fang2zhi3}{6,4}[HSK 3]
  \definition{v.}{evitar; prevenir; prevenir; proteger contra}
\end{entry}

\begin{entry}{房东}{fang2dong1}{8,5}[HSK 3]
  \definition[个,位]{s.}{dono;  proprietário; senhorio}
\end{entry}

\begin{entry}{房间}{fang2jian1}{8,7}[HSK 1]
  \definition[间,个]{s.}{quarto}
\end{entry}

\begin{entry}{房屋}{fang2 wu1}{8,9}[HSK 3]
  \definition[间,所,套]{s.}{casas; habitação; edifícios}
\end{entry}

\begin{entry}{房主}{fang2zhu3}{8,5}
  \definition{s.}{proprietário | dono de um imóvel}
\end{entry}

\begin{entry}{房子}{fang2zi5}{8,3}[HSK 1]
  \definition[栋,幢,座,套,间,个]{s.}{apartamento | casa | quarto}
\end{entry}

\begin{entry}{房租}{fang2 zu1}{8,10}[HSK 3]
  \definition[笔]{s.}{aluguel}
\end{entry}

\begin{entry}{访问}{fang3wen4}{6,6}[HSK 3]
  \definition{v.}{visitar; ligar; entrevistar | visitar um \emph{site}}
\end{entry}

\begin{entry}{放}{fang4}{8}[Radical 攴][HSK 1]
  \definition{v.}{liberar | libertar | deixar ir | colocar | por | detonar (fogos de artifício)}
\end{entry}

\begin{entry}{放鞭炮}{fang4bian1pao4}{8,18,9}
  \definition{s.}{um conjunto de bombinhas ou traques}
\end{entry}

\begin{entry}{放出}{fang4chu1}{8,5}
  \definition{v.}{liberar | libertar}
\end{entry}

\begin{entry}{放大}{fang4da4}{8,3}
  \definition{v.}{ampliar}
\end{entry}

\begin{entry}{放到}{fang4 dao4}{8,8}[HSK 3]
  \definition{v.}{colocar em; meter}
\end{entry}

\begin{entry}{放电}{fang4dian4}{8,5}
  \definition{s.}{descarga elétrica}
\end{entry}

\begin{entry}{放飞}{fang4fei1}{8,3}
  \definition{s.}{deixar voar}
\end{entry}

\begin{entry}{放过}{fang4guo4}{8,6}
  \definition{v.}{deixar | deixar alguém escapar impune | passar despercebido}
\end{entry}

\begin{entry}{放假}{fang4 jia4}{8,11}[HSK 1]
  \definition{v.}{ter férias ou feriado}
\end{entry}

\begin{entry}{放弃}{fang4qi4}{8,7}
  \definition{v.}{abandonar | desistir de | renunciar}
\end{entry}

\begin{entry}{放弃权利}{fang4qi4 quan2li4}{8,7,6,7}
  \definition{s.}{renúncia}
\end{entry}

\begin{entry}{放弃者}{fang4qi4zhe3}{8,7,8}
  \definition{s.}{desistente}
\end{entry}

\begin{entry}{放任}{fang4ren4}{8,6}
  \definition{v.}{ignorar | saciar-se | deixar sozinho}
\end{entry}

\begin{entry}{放肆}{fang4si4}{8,13}
  \definition{adj.}{atrevido | pesunçoso | devasso}
\end{entry}

\begin{entry}{放松}{fang4song1}{8,8}
  \definition{adj.}{relaxado | afrouxado}
  \definition{v.}{relaxar | afrouxar}
\end{entry}

\begin{entry}{放下}{fang4 xia4}{8,3}[HSK 2]
  \definition{v.}{deitar | colocar para baixo | deixar ir | liberar | desistir | colocar em algum lugar}
\end{entry}

\begin{entry}{放心}{fang4xin1}{8,4}[HSK 2]
  \definition{adj.}{despreocupado}
  \definition{v.}{sentir-se aliviado | sentir-se tranquilo | ficar à vontade}
  \definition{v.+compl.}{confiar | ter confiança em alguém | estar à vontade | sentir-se aliviado}
\end{entry}

\begin{entry}{放学}{fang4 xue2}{8,8}[HSK 1]
  \definition{v.+compl.}{sair da escola | acabar as aulas | terminar a aula (por hoje)}
\end{entry}

\begin{entry}{放养}{fang4yang3}{8,9}
  \definition{v.}{criar (gado, peixes, culturas, etc.) | crescer | criar}
\end{entry}

\begin{entry}{放走}{fang4zou3}{8,7}
  \definition{v.}{permitir (uma pessoa ou um animal) ir | liberar | libertar}
\end{entry}

\begin{entry}{飞}{fei1}{3}[Radical 飛][HSK 1]
  \definition*{s.}{sobrenome Fei}
  \definition{adj.}{inesperado | acidental | infundado | sem fundamento}
  \definition{adv.}{rapidamente}
  \definition{s.}{roda livre de uma bicicleta}
  \definition{v.}{voar | esvoaçar | flutuar no ar | volatilizar}
\end{entry}

\begin{entry}{飞船}{fei1chuan2}{3,11}
  \definition{s.}{espaçonave | dirigível | aeronave}
\end{entry}

\begin{entry}{飞碟}{fei1die2}{3,14}
  \definition{s.}{disco-voador, OVNI, \emph{UFO} | \emph{frisbee}}
\end{entry}

\begin{entry}{飞机}{fei1ji1}{3,6}[HSK 1]
  \definition[架]{s.}{avião}
\end{entry}

\begin{entry}{飞机票}{fei1ji1piao4}{3,6,11}
  \definition[张]{s.}{bilhete de avião}
  \seealsoref{机票}{ji1piao4}
\end{entry}

\begin{entry}{飞行}{fei1 xing2}{3,6}[HSK 3]
  \definition{s.}{voo | aviação}
  \definition{v.}{voar; fazer um voo | (aviões, foguetes, etc.) voar no ar}
\end{entry}

\begin{entry}{非}{fei1}{8}[Radical ⾮][Kangxi 175]
  \definition*{s.}{África, abreviação de 非洲}
  \definition{adv.}{não ser | não é | não}
  \seealsoref{非洲}{fei1zhou1}
\end{entry}

\begin{entry}{非常}{fei1chang2}{8,11}[HSK 1]
  \definition{adv.}{extraordinário | altamente | muito}
\end{entry}

\begin{entry}{非洲}{fei1zhou1}{8,9}
  \definition*{s.}{África}
\end{entry}

\begin{entry}{非洲人}{fei1zhou1ren2}{8,9,2}
  \definition{s.}{africano | pessoa ou povo da África}
\end{entry}

\begin{entry}{狒狒}{fei4fei4}{8,8}
  \definition{s.}{babuíno}
\end{entry}

\begin{entry}{费}{fei4}{9}[Radical 貝][HSK 3]
  \definition*{s.}{Fei}
  \definition{s.}{taxa; despesa; encargo}
  \definition{v.}{custar; gastar; desperdiçar}
\end{entry}

\begin{entry}{费用}{fei4 yong4}{9,5}[HSK 3]
  \definition[笔,个]{s.}{custo; despesa; desembolso}
\end{entry}

\begin{entry}{分}{fen1}{4}[Radical ⼑][HSK 1]
  \definition{s.}{parte ou subdivisão | fração | um décimo (de certas unidades) | unidade de comprimento equivalente a 0,33cm | minuto (unidade de tempo) | minuto (unidade de medida angular) | um ponto (em esportes e jogos) | 0,01 yuan (unidade de dinheiro)}
  \definition{v.}{dividir | separar | distribuir | atribuir | distinguir (bom e mau)}
  \seeref{分}{fen4}
\end{entry}

\begin{entry}{分别}{fen1bie2}{4,7}[HSK 3]
  \definition{adv.}{diferentemente; de ​​maneiras diferentes}
  \definition{s.}{diferença}
  \definition{v.}{partir; deixar um ao outro | distinguir; diferenciar}
\end{entry}

\begin{entry}{分公司}{fen1gong1si1}{4,4,5}
  \definition{s.}{sucursal | filial de companhia}
\end{entry}

\begin{entry}{分开}{fen1 kai1}{4,4}[HSK 2]
  \definition{v.+compl.}{separar | dividir | desacoplar | desempacotar | quebrar | desmembrar | romper | desfazer | desvincular | distribuir | separar de (em) | dividir \dots de \dots | separar de}
\end{entry}

\begin{entry}{分量}{fen1liang4}{4,12}
  \definition{s.}{componente vetorial}
  \seeref{分量}{fen4liang4}
  \seeref{分量}{fen4liang5}
\end{entry}

\begin{entry}{分配}{fen1pei4}{4,10}[HSK 3]
  \definition{v.}{atribuir; dispor | atribuir; compartilhar; distribuir}
\end{entry}

\begin{entry}{分手}{fen1shou3}{4,4}
  \definition{v.+compl.}{separar | separar-se do companheiro | dizer adeus}
\end{entry}

\begin{entry}{分数}{fen1 shu4}{4,13}[HSK 2]
  \definition{s.}{fração | número fracionário | marca | nota | ponto}
\end{entry}

\begin{entry}{分钟}{fen1zhong1}{4,9}[HSK 2]
  \definition{s.}{minuto (usado em intervalos de tempo)}
\end{entry}

\begin{entry}{分子}{fen1zi3}{4,3}
  \definition{s.}{molécula | (matemática) numerador de uma fração}
  \seeref{分子}{fen4zi3}
\end{entry}

\begin{entry}{分组}{fen1 zu3}{4,8}[HSK 3]
  \definition{v.}{agrupar; dividir em grupos}
\end{entry}

\begin{entry}{焚香}{fen2xiang1}{12,9}
  \definition{v.}{queimar incenso}
\end{entry}

\begin{entry}{粉}{fen3}{10}[Radical ⽶]
  \definition{s.}{pó | pó cosmético facial | alimento preparado a partir de amido | macarrão feito de qualquer tipo de farinha}
  \definition{v.}{tornar algo em pó | ser um fã de}
\end{entry}

\begin{entry}{粉色}{fen3 se4}{10,6}
  \definition{s.}{cor-de-rosa}
\end{entry}

\begin{entry}{粉丝}{fen3si1}{10,5}
  \definition{s.}{(empréstimo linguístico) fã | entusiasta de alguém ou alguma coisa}
  \definition[把]{s.}{aletria de amido de feijão | aletria chinesa | macarrão de celofane ou macarrão de vidro (transparente)}
\end{entry}

\begin{entry}{分}{fen4}{4}[Radical 刀][HSK 2]
  \definition{s.}{parte | ingrediente | componente}
  \seeref{分}{fen1}
\end{entry}

\begin{entry}{分量}{fen4liang4}{4,12}
  \definition{s.}{tamanho da porção (comida)}
  \seeref{分量}{fen1liang4}
  \seeref{分量}{fen4liang5}
\end{entry}

\begin{entry}{分量}{fen4liang5}{4,12}
  \definition{s.}{quantidade | peso | medida}
  \seeref{分量}{fen1liang4}
  \seeref{分量}{fen4liang4}
\end{entry}

\begin{entry}{分子}{fen4zi3}{4,3}
  \definition{s.}{membros de uma classe ou grupo | elementos políticos (como intelectuais ou extremistas)}
  \seeref{分子}{fen1zi3}
\end{entry}

\begin{entry}{份}{fen4}{6}[Radical 人][HSK 2]
  \definition{clas.}{para presentes, jornais, revistas, papéis, relatórios, contratos, etc. ou pratos (refeição)}
\end{entry}

\begin{entry}{奋战}{fen4zhan4}{8,9}
  \definition{v.}{lutar bravamente | trabalhar duro}
\end{entry}

\begin{entry}{愤怒}{fen4nu4}{12,9}
  \definition{adj.}{zangado | indignado}
  \definition{s.}{ira}
\end{entry}

\begin{entry}{愤世嫉俗}{fen4shi4ji2su2}{12,5,13,9}
  \definition{v.}{ser cínico | ser amargurado}
\end{entry}

\begin{entry}{丰富}{feng1fu4}{4,12}[HSK 3]
  \definition{adj.}{rico; abundante; pleno}
  \definition{v.}{enriquecer}
\end{entry}

\begin{entry}{丰收}{feng1shou1}{4,6}
  \definition{s.}{colheita abundante}
\end{entry}

\begin{entry}{风}{feng1}{4}[Radical 風][Kangxi 182][HSK 1]
  \definition[阵,丝]{s.}{vento}
\end{entry}

\begin{entry}{风景}{feng1jing3}{4,12}
  \definition{s.}{cenário | paisagem}
\end{entry}

\begin{entry}{风扇}{feng1shan4}{4,10}
  \definition{s.}{ventilador elétrico}
\end{entry}

\begin{entry}{风险}{feng1xian3}{4,9}[HSK 3]
  \definition[个,种,项,类]{s.}{risco; perigo}
\end{entry}

\begin{entry}{风筝}{feng1zheng5}{4,12}
  \definition{s.}{pipa | papagaio | pandorga}
\end{entry}

\begin{entry}{枫叶}{feng1ye4}{8,5}
  \definition{s.}{folha de bordo (maple, tipo de árvore)}
\end{entry}

\begin{entry}{封}{feng1}{9}[Radical 寸][HSK 2]
  \definition*{s.}{sobrenome Feng}
  \definition{clas.}{para objetos selados, especialmente cartas}
  \definition{v.}{conceder um título | conferir | conceder | selar}
\end{entry}

\begin{entry}{封闭}{feng1bi4}{9,6}
  \definition{v.}{fechar | selar | confinado}
\end{entry}

\begin{entry}{封底}{feng1di3}{9,8}
  \definition{s.}{contracapa de um livro}
\end{entry}

\begin{entry}{封冻}{feng1dong4}{9,7}
  \definition{v.}{congelar (água ou terra)}
\end{entry}

\begin{entry}{封盖}{feng1gai4}{9,11}
  \definition{s.}{boné | capa | selo}
  \definition{v.}{cobrir}
\end{entry}

\begin{entry}{封建}{feng1jian4}{9,8}
  \definition{adj.}{feudal}
  \definition{s.}{feudalismo}
\end{entry}

\begin{entry}{封口}{feng1kou3}{9,3}
  \definition{v.}{selar | fechar | curar (uma ferida) | manter os lábios selados}
\end{entry}

\begin{entry}{封面}{feng1mian4}{9,9}
  \definition{s.}{capa (de uma publicação) | sobrecapa}
\end{entry}

\begin{entry}{封印}{feng1yin4}{9,5}
  \definition{s.}{selo (em envelopes)}
\end{entry}

\begin{entry}{封斋}{feng1zhai1}{9,10}
  \definition*{s.}{Ramadã (Islã)}
\end{entry}

\begin{entry}{疯狂}{feng1kuang2}{9,7}
  \definition{adj.}{louco | frenético | selvagem}
\end{entry}

\begin{entry}{缝纫}{feng2ren4}{13,6}
  \definition{v.}{costurar}
\end{entry}

\begin{entry}{缝纫机}{feng2ren4ji1}{13,6,6}
  \definition[架]{s.}{máquina de costura}
\end{entry}

\begin{entry}{凤凰}{feng4huang2}{4,11}
  \definition{s.}{fênix}
\end{entry}

\begin{entry}{佛}{fo2}{7}[Radical 人]
  \definition*{s.}{Buda, abreviação de 佛陀 | Budismo}
  \seeref{佛}{fu2}
  \seealsoref{佛陀}{fo2tuo2}
\end{entry}

\begin{entry}{佛陀}{fo2tuo2}{7,7}
  \definition{s.}{Buda (uma pessoa que atingiu a Budeidade, ou especificamente Siddhartha Gautama)}
\end{entry}

\begin{entry}{否定}{fou3ding4}{7,8}[HSK 3]
  \definition{adj.}{negativo}
  \definition{s.}{negativo (resposta); negação}
  \definition{v.}{rejeitar; negar}
\end{entry}

\begin{entry}{否认}{fou3ren4}{7,4}[HSK 3]
  \definition{v.}{negar; repudiar}
\end{entry}

\begin{entry}{否则}{fou3ze2}{7,6}
  \definition{conj.}{caso contrário | ou}
\end{entry}

\begin{entry}{夫妻}{fu1qi1}{4,8}
  \definition{s.}{casal | marido e eposa}
\end{entry}

\begin{entry}{佛}{fu2}{7}[Radical 人]
  \definition{adv.}{aparentemente}
  \definition{s.}{ornamento da cabeça (feminino)}
  \seeref{佛}{fo2}
\end{entry}

\begin{entry}{扶梯}{fu2ti1}{7,11}
  \definition{s.}{escada rolante}
\end{entry}

\begin{entry}{服}{fu2}{8}[Radical ⽉]
  \definition{s.}{roupas | vestido | vestuário | roupa de luto}
  \definition{v.}{servir (nas forças armadas, uma sentença de prisão, etc.) | obedecer | ser convencido (por um argumento) | convencer | admirar | aclimatar | tomar (medicamento) | usar roupas de luto}
  \seeref{服}{fu4}
\end{entry}

\begin{entry}{服务}{fu2 wu4}{8,5}[HSK 2]
  \definition{v.}{prestar serviço a | estar a serviço de | servir | trabalhar | servir}
\end{entry}

\begin{entry}{服务员}{fu2wu4yuan2}{8,5,7}
  \definition{s.}{atendente | garçom | garçonete | pessoal de atendimento ao cliente}
\end{entry}

\begin{entry}{服装}{fu2zhuang1}{8,12}[HSK 3]
  \definition[套,件,身]{s.}{roupas; trajes; fantasias}
\end{entry}

\begin{entry}{浮力}{fu2li4}{10,2}
  \definition{s.}{flutuabilidade}
\end{entry}

\begin{entry}{浮图}{fu2tu2}{10,8}
  \definition*{s.}{Termo alternativo para 佛陀}
  \variantof{浮屠}
  \seealsoref{佛陀}{fo2tuo2}
\end{entry}

\begin{entry}{浮屠}{fu2tu2}{10,11}
  \definition*{s.}{Buda | Templo (Stupa) Budista (transliteração de Pali Thuo)}
\end{entry}

\begin{entry}{符合}{fu2he2}{11,6}
  \definition{conj.}{de acordo com | concordando com | contando com | alinhado com}
  \definition{v.}{concordar com | estar em conformidade com | corresponder com | gerenciar | lidar}
\end{entry}

\begin{entry}{福}{fu2}{13}[Radical 示][HSK 3]
  \definition*{s.}{sobrenome Fu}
  \definition{s.}{benção; felicidade; boa sorte; boa fortuna}
  \definition{v.}{curvar-se; reverenciar}
\end{entry}

\begin{entry}{福克斯}{fu2ke4si1}{13,7,12}
  \definition*{s.}{Fox (empresa de mídia) | Focus (automóvel fabricado pela Ford)}
\end{entry}

\begin{entry}{福泽}{fu2ze2}{13,8}
  \definition{s.}{boa sorte}
\end{entry}

\begin{entry}{父母}{fu4 mu3}{4,5}[HSK 3]
  \definition{s.}{pai e mãe; pais}
\end{entry}

\begin{entry}{父母亲}{fu4mu3qin1}{4,5,9}
  \definition{s.}{pais}
\end{entry}

\begin{entry}{父亲}{fu4qin1}{4,9}[HSK 3]
  \definition[个,位]{s.}{pai}
\end{entry}

\begin{entry}{付}{fu4}{5}[Radical 人][HSK 3]
  \definition*{s.}{sobrenome Fu}
  \definition{clas.}{para pares ou conjuntos de coisas | para expressões faciais}
  \definition{v.}{comprometer-se a; entregar (entregar) a; entregar | pagar}
\end{entry}

\begin{entry}{付款}{fu4kuan3}{5,12}
  \definition{s.}{pagamento}
  \definition{v.+compl.}{pagar uma quantia em dinheiro}
\end{entry}

\begin{entry}{负责}{fu4ze2}{6,8}[HSK 3]
  \definition{adj.}{consciencioso}
  \definition{v.}{ser responsável por; estar encarregado de}
\end{entry}

\begin{entry}{附近}{fu4jin4}{7,7}
  \definition{adv.}{aqui perto | perto daqui}
\end{entry}

\begin{entry}{服}{fu4}{8}[Radical ⽉]
  \definition{clas.}{(para remédio) dose}
  \seeref{服}{fu2}
\end{entry}

\begin{entry}{复活节}{fu4huo2jie2}{9,9,5}
  \definition*{s.}{Páscoa}
\end{entry}

\begin{entry}{复刻}{fu4ke4}{9,8}
  \definition{v.}{reimprimir (um trabalho que esteve fora do catálogo) | reeditar (um disco de vinil, um CD, etc.) | replicar | recriar | (empréstimo linguístico) (computação) \emph{fork}}
\end{entry}

\begin{entry}{复习}{fu4xi2}{9,3}[HSK 2]
  \definition{s.}{revisão}
  \definition{v.}{rever | revisar}
\end{entry}

\begin{entry}{复印}{fu4yin4}{9,5}[HSK 3]
  \definition{v.}{fotografar; fotocopiar; duplicar}
\end{entry}

\begin{entry}{复杂}{fu4za1}{9,6}[HSK 3]
  \definition{adj.}{complexo; complicado}
\end{entry}

\begin{entry}{副}{fu4}{11}[Radical 刀]
  \definition{clas.}{para pares, conjuntos de coisas e expressões faciais | para óculos, luvas, etc.}
\end{entry}

\begin{entry}{富}{fu4}{12}[Radical 宀][HSK 3]
  \definition*{s.}{sobrenome Fu}
  \definition{adj.}{rico; póspero | rico; abundante}
  \definition{s.}{fortuna; riqueza}
\end{entry}

\begin{entry}{覆盆子}{fu4pen2zi5}{18,9,3}
  \definition{s.}{framboesa}
\end{entry}

%%%%% EOF %%%%%

