%%%
%%% M
%%%

\section*{M}\addcontentsline{toc}{section}{M}

\begin{entry}{妈}{ma1}{6}{⼥}[HSK 1]
  \definition[个,位]{s.}{mãe; mamãe | uma forma de tratamento para uma mulher casada uma geração mais velha | (antigo) uma forma de tratamento para uma empregada doméstica de meia-idade ou velha}
  \seealsoref{妈妈}{ma1 ma5}
\end{entry}

\begin{entry}{妈妈}{ma1 ma5}{6,6}{⼥、⼥}[HSK 1]
  \definition[个,位]{s.}{mamãe; mãe | uma forma de chamar uma mulher de meia-idade; títulos de respeito para mulheres mais velhas}
\end{entry}

\begin{entry}{吗}{ma2}{6}{⼝}
  \definition{adv.}{(coloquial) que?}
  \seeref{吗}{ma3}
  \seeref{吗}{ma5}
\end{entry}

\begin{entry}{麻烦}{ma2fan5}{11,10}{⿇、⽕}[HSK 3]
  \definition{adj.}{problemático; inconveniente}
  \definition[个]{s.}{problema; inconveniência}
  \definition{v.}{preocupar; incomodar; amolar; azucrinar; incomodar alguém; enfadar; aborrecer}
\end{entry}

\begin{entry}{麻将}{ma2jiang4}{11,9}{⿇、⼨}
  \definition[副]{s.}{\emph{mahjong}}
\end{entry}

\begin{entry}{麻辣豆腐}{ma2la4 dou4fu5}{11,14,7,14}{⿇、⾟、⾖、⾁}
  \definition{s.}{tofú guisado em molho picante (prato)}
\end{entry}

\begin{entry}{马}{ma3}{3}{⾺}[HSK 3][Kangxi 187]
  \definition*{s.}{sobrenome Ma}
  \definition{adj.}{grande}
  \definition[匹]{s.}{cavalo | a peça do cavalo no xadrez chinês}
\end{entry}

\begin{entry}{马耳他}{ma3'er3ta1}{3,6,5}{⾺、⽿、⼈}
  \definition*{s.}{Malta}
\end{entry}

\begin{entry}{马克思列宁主义}{ma3ke4si1 lie4ning2zhu3yi4}{3,7,9,6,5,5,3}{⾺、⼗、⼼、⼑、⼧、⼂、⼂}
  \definition*{s.}{Marxismo-Leninismo}
\end{entry}

\begin{entry}{马路}{ma3lu4}{3,13}{⾺、⾜}[HSK 1]
  \definition[条]{s.}{estrada; rua; avenida; estradas largas e planas para o tráfego de carros e cavalos nas cidades ou nos subúrbios}
\end{entry}

\begin{entry}{马马虎虎}{ma3ma3hu3hu3}{3,3,8,8}{⾺、⾺、⾌、⾌}
  \definition{adj.}{descuidado | casual | tolerável | vago | mais ou menos}
\end{entry}

\begin{entry}{马上}{ma3shang4}{3,3}{⾺、⼀}[HSK 1]
  \definition{adv.}{imediatamente; de uma só vez; em um piscar de olhos | em breve; em um futuro próximo; em um curto espaço de tempo}
\end{entry}

\begin{entry}{马尾}{ma3wei3}{3,7}{⾺、⼫}
  \definition{s.}{(penteado) rabo de cavalo | cauda de cavalo}
\end{entry}

\begin{entry}{吗}{ma3}{6}{⼝}
  \definition{s.}{usada em 吗啡, morfina}
  \seealsoref{吗啡}{ma3fei1}
\end{entry}

\begin{entry}{吗啡}{ma3fei1}{6,11}{⼝、⼝}
  \definition{s.}{morfina (empréstimo linguístico)}
\end{entry}

\begin{entry}{码头}{ma3tou2}{8,5}{⽯、⼤}[HSK 5]
  \definition[个]{s.}{doca; cais; píer; molhe; edifícios à beira-mar ou à beira do rio destinados exclusivamente à atracação de embarcações, embarque e desembarque de passageiros e carga e descarga de mercadorias | cidade portuária; centro comercial e de transportes; refere-se a uma cidade comercial com transporte terrestre e marítimo bem desenvolvido.}
\end{entry}

\begin{entry}{蚂蚁}{ma3yi3}{9,9}{⾍、⾍}
  \definition{s.}{formiga}
\end{entry}

\begin{entry}{骂}{ma4}{9}{⾺}[HSK 5]
  \definition{v.}{abusar; xingar; insultar; insultar alguém com palavras grosseiras ou maliciosas | repreender; censurar; condenar}
\end{entry}

\begin{entry}{骂街}{ma4jie1}{9,12}{⾺、⾏}
  \definition{v.}{gritar abusos na rua}
\end{entry}

\begin{entry}{骂名}{ma4ming2}{9,6}{⾺、⼝}
  \definition{s.}{infâmia}
\end{entry}

\begin{entry}{吗}{ma5}{6}{⼝}[HSK 1]
  \definition{part.}{usado no final de uma pergunta | como uma pausa em uma frase antes de introduzir o ponto principal | usado no final de uma pergunta retórica}
  \seeref{吗}{ma2}
  \seeref{吗}{ma3}
\end{entry}

\begin{entry}{埋伏}{mai2fu2}{10,6}{⼟、⼈}
  \definition{s.}{emboscada}
  \definition{v.}{emboscar}
\end{entry}

\begin{entry}{买}{mai3}{6}{⼄}[HSK 1]
  \definition*{s.}{sobrenome Mai}
  \definition{v.}{comprar; adquirir | comprar; subornar; usar dinheiro ou outros meios para angariar apoio| pedir; obter; trocar dinheiro por coisas}
\end{entry}

\begin{entry}{买东西}{mai3dong1xi5}{6,5,6}{⼄、⼀、⾑}
  \definition{v.}{fazer compras}
\end{entry}

\begin{entry}{买卖}{mai3 mai4}{6,8}{⼄、⼗}[HSK 5]
  \definition[种,笔]{s.}{negócio; compra e venda; transação | (privado) loja; armazém;}
\end{entry}

\begin{entry}{麦当劳}{mai4dang1lao2}{7,6,7}{⿆、⼹、⼒}
  \definition*{s.}{McDonald's (empresa de \emph{fast-food})}
  \seealsoref{麦当劳叔叔}{mai4dang1lao2 shu1shu5}
\end{entry}

\begin{entry}{麦当劳叔叔}{mai4dang1lao2 shu1shu5}{7,6,7,8,8}{⿆、⼹、⼒、⼜、⼜}
  \definition*{s.}{Ronald McDonald}
  \seealsoref{麦当劳}{mai4dang1lao2}
\end{entry}

\begin{entry}{麦淇淋}{mai4qi2lin2}{7,11,11}{⿆、⽔、⽔}
  \definition{s.}{(empréstimo linguístico) margarina}
\end{entry}

\begin{entry}{卖}{mai4}{8}{⼗}[HSK 2]
  \definition*{s.}{sobrenome Mai}
  \definition{clas.}{um prato (nos tempos antigos); antigamente, os restaurantes chamavam cada prato vendido de 一卖 (uma porção)}
  \definition{v.}{vender (oposto de 买) | trair (o próprio país ou amigos); alcançar objetivos pessoais à custa dos interesses do país, da nação e dos outros | não poupar esforços; esforçar-se ao máximo; tentar fazer o máximo possível | mostrar-se intencionalmente; exibir-se | vender o próprio trabalho; trabalhar em troca de dinheiro}
  \seealsoref{买}{mai3}
\end{entry}

\begin{entry}{馒头}{man2tou5}{14,5}{⾷、⼤}
  \definition{s.}{pão cozido no vapor}
\end{entry}

\begin{entry}{满}{man3}{13}{⽔}[HSK 2]
  \definition*{s.}{sobrenome Man}
  \definition*{s.}{etnia Manchu}
  \definition{adj.}{cheio; repleto; lotado; totalmente cheio; atingindo o limite da capacidade | tudo; inteiro; completo | presunçoso; complacente; orgulhoso}
  \definition{adv.}{muito; um tanto; bastante | completamente; inteiramente; perfeitamente}
  \definition{v.}{encher | sentir-se satisfeito; sentir que já é o suficiente | expirar; atingir o limite; atingir um determinado prazo ou limite}
\end{entry}

\begin{entry}{满分}{man3fen1}{13,4}{⽔、⼑}
  \definition{s.}{pontuação completa}
\end{entry}

\begin{entry}{满满}{man3man3}{13,13}{⽔、⽔}
  \definition{adj.}{completo | densamente empacotado}
\end{entry}

\begin{entry}{满意}{man3yi4}{13,13}{⽔、⼼}[HSK 2]
  \definition{adj.}{satisfeito; contente; gratificado}
  \definition{v.}{estar satisfeito; sentir-se contente; satisfazer os seus desejos; estar de acordo com os seus desejos}
\end{entry}

\begin{entry}{满足}{man3zu2}{13,7}{⽔、⾜}[HSK 3]
  \definition{v.}{estar satisfeito; contentar-se | satisfazer; causar satisfação; contentar}
\end{entry}

\begin{entry}{谩骂}{man4ma4}{13,9}{⾔、⾺}
  \definition{v.}{ridicularizar | abusar}
\end{entry}

\begin{entry}{慢}{man4}{14}{⼼}[HSK 1]
  \definition*{s.}{sobrenome Man}
  \definition{adj.}{lento; devagar; baixa velocidade; longa duração (em oposição a 快) | rude; arrogante; sem educação com as pessoas | frouxo; lento}
  \definition{adv.}{lentamente}
  \seealsoref{快}{kuai4}
\end{entry}

\begin{entry}{慢动作}{man4dong4zuo4}{14,6,7}{⼼、⼒、⼈}
  \definition{s.}{(cinema) câmera lenta}
\end{entry}

\begin{entry}{慢慢}{man4 man4}{14,14}{⼼、⼼}[HSK 3]
  \definition{adv.}{lentamente; vagarosamente; gradualmente}
\end{entry}

\begin{entry}{漫长}{man4chang2}{14,4}{⽔、⾧}[HSK 5]
  \definition{adj.}{muito longo; interminável; (tempo, espaço) dura muito tempo}
\end{entry}

\begin{entry}{漫画}{man4hua4}{14,8}{⽔、⽥}[HSK 5]
  \definition[幅,本,张,套]{s.}{desenho animado; caricatura}
\end{entry}

\begin{entry}{漫骂}{man4ma4}{14,9}{⽔、⾺}
  \variantof{谩骂}
\end{entry}

\begin{entry}{蔓草}{man4cao3}{14,9}{⾋、⾋}
  \definition{s.}{videira | trepadeira}
\end{entry}

\begin{entry}{忙}{mang2}{6}{⼼}[HSK 1]
  \definition*{s.}{sobrenome Mang}
  \definition{adj.}{ocupado; movimentado; totalmente ocupado; muitas coisas para fazer, sem tempo livre (oposto de 闲) | imperativo; ansioso; urgente}
  \definition{v.}{apressar-se; agitar-se; fazer algo com urgência e constantemente | trabalhar; fazer}
  \seealsoref{闲}{xian2}
\end{entry}

\begin{entry}{盲目}{mang2mu4}{8,5}{⽬、⽬}
  \definition{adj.}{ignorante | sem compreensão}
  \definition{adv.}{cegamente}
  \definition{s.}{cego}
\end{entry}

\begin{entry}{猫}{mao1}{11}{⽝}[HSK 2]
  \definition[只,种,群,窝,个]{s.}{gato |  (empréstimo linguístico) MODEM}
  \definition{v.}{esconder-se; entrar em esconderijo | inclinar-se para a frente; curvar-se}
  \seeref{猫}{mao2}
\end{entry}

\begin{entry}{猫熊}{mao1xiong2}{11,14}{⽝、⽕}
  \definition[把,只]{s.}{panda gigante}
  \seealsoref{熊猫}{xiong2mao1}
\end{entry}

\begin{entry}{毛}{mao2}{4}{⽑}[HSK 1,3][Kangxi 82]
  \definition*{s.}{sobrenome Mao}
  \definition{adj.}{bruto; semiacabado | grosseiro | pequeno | fino | descuidado; rude; precipitado | assustado; nervoso; em pânico | impetuoso | rústico; sem acabamento | impuro | (de moeda) que não vale mais seu valor nominal; depreciado}
  \definition{clas.}{mao, uma unidade fracionária de dinheiro na China; dez centavos; uma peça de dez centavos}
  \definition[根]{s.}{(de um animal, planta, etc.) cabelo; pena; penugem | (de humanos) cabelo; barba | planta; colheita | lã | mofo; bolor}
  \definition{v.}{depreciar; desvalorizar; refere-se à desvalorização da moeda | (de cavalos, gado, etc.) assustar-se}
\end{entry}

\begin{entry}{毛笔}{mao2 bi3}{4,10}{⽑、⽵}[HSK 5]
  \definition[支,枝,根]{s.}{pincel para escrever; pincel chinês; canetas feitas com pelos de coelho, carneiro, doninha, etc., são materiais tradicionais utilizados para escrever caracteres chineses e pintar pinturas tradicionais chinesas}
\end{entry}

\begin{entry}{毛病}{mao2bing4}{4,10}{⽑、⽧}[HSK 3]
  \definition[个]{s.}{doença ou deficiência | problema; fracasso | mau hábito; deficiência}
\end{entry}

\begin{entry}{毛巾}{mao2jin1}{4,3}{⽑、⼱}[HSK 4]
  \definition[条]{s.}{toalha; toalha de banho}
\end{entry}

\begin{entry}{毛衣}{mao2 yi1}{4,6}{⽑、⾐}[HSK 4]
  \definition[件]{s.}{suéter; blusa feita de lã}
\end{entry}

\begin{entry}{矛}{mao2}{5}{⽭}[Kangxi 110]
  \definition{s.}{lança; lanceta}
\end{entry}

\begin{entry}{矛盾}{mao2dun4}{5,9}{⽭、⽬}[HSK 5]
  \definition{adj.}{contraditório; descreve pessoas ou coisas que se opõem ou se repelem mutuamente}
  \definition{s.}{problema; contradição; discrepância; inconsistência | disputas e conflitos; relacionamento de oposição entre as duas partes devido a diferenças de opinião ou abordagem}
  \definition{v.}{opor-se; entrar em conflito; contradizer; nesta situação, apenas uma das opções está correta ou é verdadeira; não é possível que ambas estejam corretas ao mesmo tempo}
\end{entry}

\begin{entry}{牦牛}{mao2niu2}{8,4}{⽜、⽜}
  \definition{s.}{iaque}
\end{entry}

\begin{entry}{猫}{mao2}{11}{⽝}
  \definition{v.}{utilizado em 猫腰 \dpy{mao2yao1}}
  \seealsoref{猫腰}{mao2yao1}
\end{entry}

\begin{entry}{猫腰}{mao2yao1}{11,13}{⽝、⾁}
  \definition{v.}{curvar-se}
\end{entry}

\begin{entry}{冒}{mao4}{9}{⽇}[HSK 5]
  \definition*{s.}{sobrenome Mao}
  \definition{adv.}{com ousadia; precipitadamente | fingidamente; falsamente; fraudulentamente}
  \definition{v.}{emitir; liberar; enviar (para cima) | arriscar; ser corajoso}
\end{entry}

\begin{entry}{冒险}{mao4xian3}{9,9}{⽇、⾩}
  \definition{adj.}{corajoso}
  \definition{s.}{risco | aventura}
  \definition{v.+compl.}{correr risco | arriscar-se | aventurar-se em}
\end{entry}

\begin{entry}{贸易}{mao4yi4}{9,8}{⾙、⽇}[HSK 5]
  \definition[笔,宗,项]{s.}{comércio; troca; negócios; refere-se a atividades comerciais, como a troca de mercadorias}
  \definition{v.}{fazer uma transação comercial}
\end{entry}

\begin{entry}{帽子}{mao4zi5}{12,3}{⼱、⼦}[HSK 4]
  \definition[顶,个,种]{s.}{boné; chapéu; capacete | etiqueta; rótulo; marca}
\end{entry}

\begin{entry}{没}{mei2}{7}{⽔}[HSK 1]
  \definition{adv.}{não; nunca; negar que uma ação ou situação tenha ocorrido, com o significado de 不曾}
  \definition{pref.}{não (prefixo negativo para verbos, traduzido para outras línguas com verbos no pretérito)}
  \definition{v.}{não possuir; não ter | não existe; não há | ninguém; usado antes de 谁, 什么, 哪个, significa 全都不 | não ser tão bom quanto; ser inferior a; não chega a; não é tão bom quanto | menor que; insuficiente}
  \seeref{没}{mo4}
  \seealsoref{不曾}{bu4 ceng2}
  \seealsoref{哪个}{na3ge5}
  \seealsoref{全都不}{quan2dou1 bu4}
  \seealsoref{谁}{shei2}
  \seealsoref{什么}{shen2me5}
\end{entry}

\begin{entry}{没错}{mei2 cuo4}{7,13}{⽔、⾦}[HSK 4]
  \definition{adv.}{está certo; é isso mesmo; não há como errar}
\end{entry}

\begin{entry}{没法儿}{mei2 fa3r5}{7,8,2}{⽔、⽔、⼉}[HSK 4]
  \definition{adv.}{não pode; sem chance}
\end{entry}

\begin{entry}{没关系}{mei2guan1xi5}{7,6,7}{⽔、⼋、⽷}[HSK 1]
  \definition{v.}{está tudo bem; não é nada; não importa; não se preocupe}
  \seealsoref{没有关系}{mei2you3guan1xi5}
\end{entry}

\begin{entry}{没了}{mei2le5}{7,2}{⽔、⼅}
  \definition{v.}{estar morto | deixar de existir}
\end{entry}

\begin{entry}{没什么}{mei2 shen2 me5}{7,4,3}{⽔、⼈、⼃}[HSK 1]
  \definition{expr.}{não é nada; está tudo bem; não importa}
\end{entry}

\begin{entry}{没事儿}{mei2 shi4r5}{7,8,2}{⽔、⼅、⼉}[HSK 1]
  \definition{expr.}{fora de perigo; nada sério | não importa; não é nada; está tudo bem; não importa | está tudo bem; sem problemas; não se preocupe com isso; não é grande coisa; não há nada errado}
  \definition{v.}{não ter nada para fazer; ser livre; estar perdido | estar desempregado; estar sem trabalho | não ter responsabilidade}
\end{entry}

\begin{entry}{没想到}{mei2 xiang3 dao4}{7,13,8}{⽔、⼼、⼑}[HSK 4]
  \definition{expr.}{não esperava; inesperado}
\end{entry}

\begin{entry}{没用}{mei2 yong4}{7,5}{⽔、⽤}[HSK 3]
  \definition{adj.}{inútil; imprestável; sem valor; sem préstimo; vão; que não serve para nada}
\end{entry}

\begin{entry}{没有}{mei2 you3}{7,6}{⽔、⽉}[HSK 1]
  \definition{adv.}{ainda não; (usado com o pretérito) não; ação ou estado negativo ocorreu}
  \definition{v.}{não há; não tem; não existe}
\end{entry}

\begin{entry}{没有次序}{mei2you3 ci4xu4}{7,6,6,7}{⽔、⽉、⽋、⼴}
  \definition{adj.}{sem ordem; nenhuma ordem}
\end{entry}

\begin{entry}{没有关系}{mei2you3guan1xi5}{7,6,6,7}{⽔、⽉、⼋、⽷}
  \definition{v.}{não ter problema | não ter importância | não fazer mal}
  \seealsoref{没关系}{mei2guan1xi5}
\end{entry}

\begin{entry}{没有意思}{mei2you3yi4si5}{7,6,13,9}{⽔、⽉、⼼、⼼}
  \definition{adj.}{tedioso | chato | sem interesse}
\end{entry}

\begin{entry}{眉}{mei2}{9}{⽬}
  \definition{s.}{sobrancelha | margem superior}
\end{entry}

\begin{entry}{眉毛}{mei2mao5}{9,4}{⽬、⽑}
  \definition[根]{s.}{sobrancelha}
\end{entry}

\begin{entry}{眉头}{mei2tou2}{9,5}{⽬、⼤}
  \definition{s.}{testa}
\end{entry}

\begin{entry}{媒体}{mei2ti3}{12,7}{⼥、⼈}[HSK 3]
  \definition[家,个,种]{s.}{mídia; mídia de massa}
\end{entry}

\begin{entry}{煤}{mei2}{13}{⽕}[HSK 5]
  \definition[吨,堆,块]{s.}{carvão; carvão vegetal; minério sólido preto}
\end{entry}

\begin{entry}{煤气}{mei2 qi4}{13,4}{⽕、⽓}[HSK 5]
  \definition[把]{s.}{gás; gás de carvão; gás obtido a partir do processamento do carvão não tem cor nem odor, é tóxico e pode ser queimado ou utilizado como matéria-prima na indústria química | envenenamento por monóxido de carbono}
\end{entry}

\begin{entry}{每}{mei3}{7}{⽏}[HSK 3]
  \definition*{s.}{sobrenome Mei}
  \definition{adv.}{frequentemente; todo}
  \definition{pron.}{cada; cada um; cada qual;  todo}
\end{entry}

\begin{entry}{每次}{mei3ci4}{7,6}{⽏、⽋}
  \definition{adv.}{toda vez | cada vez}
\end{entry}

\begin{entry}{每个人}{mei3ge5ren2}{7,3,2}{⽏、⼈、⼈}
  \definition{pron.}{todo mundo | todos}
\end{entry}

\begin{entry}{每天}{mei3tian1}{7,4}{⽏、⼤}
  \definition{adv.}{todo dia | cada dia}
\end{entry}

\begin{entry}{美}{mei3}{9}{⽺}[HSK 3]
  \definition*{s.}{Abreviatura de América (美洲) | Abreviatura de Estados Unidos da América (美国)}
  \definition{adj.}{lindo; bonito; belo; atraente | satisfatório; bom; agradável}
  \definition{v.}{embelezar; enfeitar | orgulhar-se de; estar satisfeito consigo mesmo}
  \seealsoref{美国}{mei3guo2}
  \seealsoref{美洲}{mei3zhou1}
\end{entry}

\begin{entry}{美国}{mei3guo2}{9,8}{⽺、⼞}
  \definition*{s.}{Estados Unidos da América}
\end{entry}

\begin{entry}{美国人}{mei3guo2ren2}{9,8,2}{⽺、⼞、⼈}
  \definition{s.}{americano | pessoa ou povo dos Estados Unidos da América}
\end{entry}

\begin{entry}{美好}{mei3 hao3}{9,6}{⽺、⼥}[HSK 3]
  \definition{adj.}{bem; feliz; glorioso}
\end{entry}

\begin{entry}{美甲}{mei3jia3}{9,5}{⽺、⽥}
  \definition{s.}{manicure e/ou pedicure}
\end{entry}

\begin{entry}{美金}{mei3 jin1}{9,8}{⽺、⾦}[HSK 4]
  \definition{s.}{USD; dólar americano: a moeda local dos Estados Unidos}
\end{entry}

\begin{entry}{美丽}{mei3li4}{9,7}{⽺、⼀}[HSK 3]
  \definition{adj.}{bonito; lindo}
\end{entry}

\begin{entry}{美女}{mei3 nv3}{9,3}{⽺、⼥}[HSK 4]
  \definition[个,位]{s.}{beldade; mulher bonita; uma jovem linda}
\end{entry}

\begin{entry}{美食}{mei3 shi2}{9,9}{⽺、⾷}[HSK 3]
  \definition[种,道,桌]{s.}{iguaria; comida deliciosa}
\end{entry}

\begin{entry}{美术}{mei3shu4}{9,5}{⽺、⽊}[HSK 3]
  \definition[种]{s.}{arte; belas artes | pintura}
\end{entry}

\begin{entry}{美味}{mei3wei4}{9,8}{⽺、⼝}
  \definition{adj.}{delicioso}
  \definition{s.}{comida deliciosa | delicadeza (\emph{delicacy})}
\end{entry}

\begin{entry}{美学}{mei3xue2}{9,8}{⽺、⼦}
  \definition{s.}{estética}
\end{entry}

\begin{entry}{美元}{mei3yuan2}{9,4}{⽺、⼉}[HSK 3]
  \definition*[元,笔,沓]{s.}{Dólar Americano}
\end{entry}

\begin{entry}{美洲}{mei3zhou1}{9,9}{⽺、⽔}
  \definition*{s.}{América (incluindo Norte, Central e Sul)}
\end{entry}

\begin{entry}{美洲人}{mei3zhou1ren2}{9,9,2}{⽺、⽔、⼈}
  \definition{s.}{americano | pessoa ou povo do continente Americano}
\end{entry}

\begin{entry}{妹}{mei4}{8}{⼥}[HSK 1]
  \definition*{s.}{sobrenome Mei}
  \definition[个]{s.}{irmã mais nova | parente do sexo feminino da mesma geração | jovem garota; jovem mulher ou menina}
  \seealsoref{妹妹}{mei4 mei5}
\end{entry}

\begin{entry}{妹夫}{mei4fu5}{8,4}{⼥、⼤}
  \definition{s.}{marido da irmã mais nova}
\end{entry}

\begin{entry}{妹妹}{mei4 mei5}{8,8}{⼥、⼥}[HSK 1]
  \definition[个]{s.}{irmã mais nova}
\end{entry}

\begin{entry}{魅力}{mei4li4}{14,2}{⿁、⼒}
  \definition{s.}{charme | fascínio | glamour | carisma}
\end{entry}

\begin{entry}{闷热}{men1re4}{7,10}{⾨、⽕}
  \definition{adj.}{abafado | quente e abafado | sufocantemente quente | quente e sensual}
\end{entry}

\begin{entry}{门}{men2}{3}{⾨}[HSK 1][Kangxi 169]
  \definition*{s.}{sobrenome Men}
  \definition{clas.}{para equipamentos de artilharia (por exemplo: canhões) | para trabalhos escolares, ciência e tecnologia, etc. | para idiomas | para casamentos | para parentes}
  \definition[个,把,道,扇]{s.}{entradas e saídas de edifícios, veículos, navios, aviões, etc. | válvula; interruptor; algo que funciona como um interruptor ou como uma porta | habilidade; método; acesso; maneira de fazer algo | família; ramo de uma família ou clã | seita (religiosa); escola (de pensamento); faculdades acadêmicas, ideológicas ou religiosas | classe; categoria; ramo de estudo; refere-se à categoria geral de coisas | filo; segundo nível da classificação biológica | (computador) \emph{gate}; porta (lógica) | porta; portão; entrada; refere-se a uma porta que pode ser aberta e fechada, instalada na entrada e saída | qualquer abertura; partes de objetos que podem ser abertas e fechadas | orifício no corpo humano; refere-se especificamente aos orifícios do corpo humano | estudar com o mesmo professor; refere-se especificamente ao professor ou mestre | posição em um jogo de apostas (em relação ao local onde se senta ou onde se faz uma aposta)}
\end{entry}

\begin{entry}{门口}{men2 kou3}{3,3}{⾨、⼝}[HSK 1]
  \definition[个]{s.}{porta; portão; entrada; porta de entrada}
\end{entry}

\begin{entry}{门票}{men2 piao4}{3,11}{⾨、⽰}[HSK 1]
  \definition{s.}{bilhete de entrada; bilhete de admissão; ingressos para locais de turismo, entretenimento, etc.}
\end{entry}

\begin{entry}{门诊}{men2 zhen3}{3,7}{⾨、⾔}[HSK 5]
  \definition{s.}{(no hospital) clínica ambulatorial; seção para pacientes ambulatoriais; local onde os médicos atendem pacientes que não estão internados no hospital}
\end{entry}

\begin{entry}{们}{men5}{5}{⼈}[HSK 1]
  \definition{suf.}{usado após pronomes ou substantivos que se referem a pessoas para indicar pluralidade}
\end{entry}

\begin{entry}{蒙面}{meng2mian4}{13,9}{⾋、⾯}
  \definition{adj.}{descarado | desavergonhado | mascarado}
  \definition{v.}{cobrir o rosto | usar uma máscara}
\end{entry}

\begin{entry}{猛}{meng3}{11}{⽝}
  \definition{adj.}{feroz | violento | corajoso | abrupto | (gíria) incrível}
  \definition{adv.}{de repente}
\end{entry}

\begin{entry}{猛然}{meng3ran2}{11,12}{⽝、⽕}
  \definition{adv.}{de repente | abruptamente}
\end{entry}

\begin{entry}{懵懂}{meng3dong3}{18,15}{⼼、⼼}
  \definition{adj.}{confuso | ignorante}
\end{entry}

\begin{entry}{梦}{meng4}{11}{⼣}[HSK 4]
  \definition*{s.}{sobrenome Meng}
  \definition[场,个]{s.}{sonho; atividade de representação no cérebro durante o sono}
  \definition{v.}{sonhar; ter um sonho}
\end{entry}

\begin{entry}{梦见}{meng4 jian4}{11,4}{⼣、⾒}[HSK 4]
  \definition{v.}{sonhar; sonhar com; ver em um sonho}
\end{entry}

\begin{entry}{梦想}{meng4xiang3}{11,13}{⼣、⼼}[HSK 4]
  \definition[个]{s.}{sonhar; esperança vã; sonho inalcançável}
  \definition{v.}{sonhar; sonhar com carinho; desejar ardentemente}
\end{entry}

\begin{entry}{眯}{mi1}{11}{⽬}
  \definition{v.}{estreitar os olhos | esmagar | (dialeto) tirar uma soneca}
  \seeref{眯}{mi2}
\end{entry}

\begin{entry}{迷}{mi2}{9}{⾡}[HSK 3]
  \definition*{s.}{sobrenome Mi}
  \definition{adj.}{perdido; confuso}
  \definition{s.}{fã; entusiasta; fanático}
  \definition{v.}{estar confuso; perder o rumo; se perder-se | ficar fascinado por; entregar-se a; ficar encantado com (por); ser louco por | confundir; desorientar; fascinar; encantar}
\end{entry}

\begin{entry}{迷宫}{mi2gong1}{9,9}{⾡、⼧}
  \definition{s.}{labirinto}
\end{entry}

\begin{entry}{迷恋}{mi2lian4}{9,10}{⾡、⼼}
  \definition{adj.}{obcecado}
  \definition{v.}{estar/ser apaixonado por | ficar encantado por | estar/ser obcecado por}
\end{entry}

\begin{entry}{迷路}{mi2lu4}{9,13}{⾡、⾜}
  \definition{s.}{labirinto | ouvido interno}
  \definition{v.+compl.}{perder o caminho | perder-se | seguir pelo caminho errado | não conseguir encontrar o caminho}
\end{entry}

\begin{entry}{迷你}{mi2ni3}{9,7}{⾡、⼈}
  \definition{adj.}{(empréstimo linguístico) mini, como em minissaia ou \emph{Mini Cooper}}
\end{entry}

\begin{entry}{迷人}{mi2ren2}{9,2}{⾡、⼈}[HSK 5]
  \definition{adj.}{encantador; fascinante; sedutor; hipnotizante}
  \definition{v.}{confundir; intrigar; enganar}
\end{entry}

\begin{entry}{迷信}{mi2xin4}{9,9}{⾡、⼈}[HSK 5]
  \definition{s.}{superstição; crença supersticiosa; fé cega; adoração cega; crença em deuses, espíritos e fantasmas}
  \definition{v.}{ter fé cega em; fazer um fetiche de}
\end{entry}

\begin{entry}{眯}{mi2}{11}{⽬}
  \definition{v.}{cegar (como com poeira)}
  \seeref{眯}{mi1}
\end{entry}

\begin{entry}{米}{mi3}{6}{⽶}[HSK 2,3][Kangxi 119]
  \definition*{s.}{sobrenome Mi}
  \definition{clas.}{metro (m); unidade principal de comprimento do sistema métrico}
  \definition{s.}{arroz | sementes descascadas; refere-se a sementes comestíveis descascadas ou sem casca | qualquer coisa que se assemelhe a um grão de arroz}
\end{entry}

\begin{entry}{米饭}{mi3fan4}{6,7}{⽶、⾷}[HSK 1]
  \definition{s.}{arroz (cozido)}
\end{entry}

\begin{entry}{秘密}{mi4mi4}{10,11}{⽲、⼧}[HSK 4]
  \definition{adj.}{secreto}
  \definition[个]{s.}{segredo; algo secreto; coisas que você não quer que as pessoas saibam}
\end{entry}

\begin{entry}{秘书}{mi4shu1}{10,4}{⽲、⼄}[HSK 4]
  \definition[个,位,名]{s.}{o cargo de secretário; funções de secretariado | secretário; pessoas encarregadas da correspondência e que auxiliam o chefe do órgão ou departamento na condução diária de seu trabalho}
\end{entry}

\begin{entry}{密}{mi4}{11}{⼧}[HSK 4]
  \definition*{s.}{sobrenome Mi}
  \definition{adj.}{fechado; denso; espesso | íntimo; próximo; afetuoso | delicado; fino; cuidadoso; meticuloso}
  \definition{adv.}{secretamente}
  \definition{s.}{segredo | densidade}
\end{entry}

\begin{entry}{密码}{mi4ma3}{11,8}{⼧、⽯}[HSK 4]
  \definition[个]{s.}{código; senha;}
\end{entry}

\begin{entry}{密切}{mi4qie4}{11,4}{⼧、⼑}[HSK 4]
  \definition{adj.}{próximo; íntimo; relacionamento próximo}
  \definition{adv.}{cuidadosamente; atentamente; intimamente}
  \definition{v.}{tornar-se próximo; tornar-se íntimo; conectar-se}
\end{entry}

\begin{entry}{蜜桃}{mi4tao2}{14,10}{⾍、⽊}
  \definition{s.}{pêssego suculento}
\end{entry}

\begin{entry}{棉}{mian2}{12}{⽊}
  \definition{s.}{termo genérico para algodão ou paina | algodão | acolchoado ou estofado com algodão}
\end{entry}

\begin{entry}{免得}{mian3de5}{7,11}{⼉、⼻}
  \definition{conj.}{de modo a não | para evitar | para que não}
\end{entry}

\begin{entry}{免费}{mian3fei4}{7,9}{⼉、⾙}[HSK 4]
  \definition{s.}{gratuito; sem custo}
  \definition{v.+compl.}{isentar de taxas; tonar grátis}
\end{entry}

\begin{entry}{免税}{mian3shui4}{7,12}{⼉、⽲}
  \definition{adj.}{isento de impostos (tributação)}
  \definition{s.}{livre de impostos | isenção de impostos}
  \definition{v.+compl.}{isentar impostos}
\end{entry}

\begin{entry}{靣}{mian4}{8}{⼀}
  \variantof{面}
\end{entry}

\begin{entry}{面}{mian4}{9}{⾯}[HSK 2][Kangxi 176]
  \definition*{s.}{sobrenome Mian}
  \definition{adj.}{macio e farinhento; descreve algo que é muito macio ao comer | superficial}
  \definition{adv.}{diretamente; pessoalmente; na frente de alguém; cara a cara}
  \definition{clas.}{usado para objetos planos | usado para indicar o número de vezes que as pessoas se encontram}
  \definition[斤,两,碗]{s.}{face; parte frontal da cabeça; rosto | topo; superfície | capa; exterior; a parte externa de um objeto ou a face frontal de um tecido (em oposição à 里) |
(matemática) superfície | cara; sentimento; emoção | geral; área total; abrangente; toda a região | lado; aspecto | escopo; escala; extensão; alcance; âmbito |
farinha; farinha de trigo | pó; algo em pó | macarrão; \emph{noodle}}
  \definition{suf.}{sufixo para localização ou direção; anexado ao final de palavras que indicam localização, equivalente a 边}
  \definition{v.}{encarar algo | encontrar; revelar-se}
  \seealsoref{边}{bian1}
  \seealsoref{里}{li3}
\end{entry}

\begin{entry}{面包}{mian4bao1}{9,5}{⾯、⼓}[HSK 1]
  \definition[个,片,袋,块]{s.}{pão}[我买八个面包了。(Comprei oito pães.) | 他在吃两片面包。(Ele está comendo duas fatias de pão.) | 我在家里带了一袋面包。(Trouxe um saco de pão para casa.) | 我拿了一块面包。(Peguei um pedaço de pão.)]
\end{entry}

\begin{entry}{面对}{mian4dui4}{9,5}{⾯、⼨}[HSK 3]
  \definition{v.}{enfrentar; defrontar | confrontar (problema)}
\end{entry}

\begin{entry}{面对面}{mian4dui4mian4}{9,5,9}{⾯、⼨、⾯}
  \definition{expr.}{cara a cara}
\end{entry}

\begin{entry}{面对面吃面}{mian4dui4mian4 chi1 mian4}{9,5,9,6,9}{⾯、⼨、⾯、⼝、⾯}
  \definition{expr.}{Comer macarrão cara a cara; indica que o seu estado atual, ou algumas das posições em que você está, ou algumas das coisas que você fez são muito claras}
\end{entry}

\begin{entry}{面积}{mian4ji1}{9,10}{⾯、⽲}[HSK 3]
  \definition{s.}{área (de um andar, pedaço de terreno, etc.); área de uma superfície}
\end{entry}

\begin{entry}{面临}{mian4lin2}{9,9}{⾯、⼁}[HSK 4]
  \definition{v.}{ser confrontado com; encontrar (uma situação) na frente de}
\end{entry}

\begin{entry}{面貌}{mian4mao4}{9,14}{⾯、⾘}[HSK 5]
  \definition[种,个]{s.}{rosto; traços faciais; formato do rosto; aparência |
aparência; aspecto; aparência (das coisas) |}
\end{entry}

\begin{entry}{面前}{mian4 qian2}{9,9}{⾯、⼑}[HSK 2]
  \definition{s.}{antes; na frente de; diante de}
\end{entry}

\begin{entry}{面试}{mian4 shi4}{9,8}{⾯、⾔}[HSK 4]
  \definition[次]{s.}{entrevista; audição}
\end{entry}

\begin{entry}{面条}{mian4tiao2}{9,7}{⾯、⽊}
  \definition{s.}{macarrão | espaguete}
\end{entry}

\begin{entry}{面条儿}{mian4 tiao2r5}{9,7,2}{⾯、⽊、⼉}[HSK 1]
  \definition{s.}{macarrão; \emph{noodles}}
\end{entry}

\begin{entry}{面团}{mian4tuan2}{9,6}{⾯、⼞}
  \definition{s.}{massa | pasta}
\end{entry}

\begin{entry}{面子}{mian4zi5}{9,3}{⾯、⼦}[HSK 5]
  \definition{s.}{face; exterior; parte externa; superfície do objeto | imagem; reputação; prestígio; decência; vaidade superficial | sentimentos; sensibilidades | pó}
\end{entry}

\begin{entry}{糆}{mian4}{15}{⽶}
  \variantof{面}
\end{entry}

\begin{entry}{麫}{mian4}{15}{⿆}
  \variantof{面}
\end{entry}

\begin{entry}{描述}{miao2 shu4}{11,8}{⼿、⾡}[HSK 4]
  \definition[段,种]{s.}{descrição; trecho que descreve um evento ou uma cena}
  \definition{v.}{descrever; representar}
\end{entry}

\begin{entry}{描写}{miao2xie3}{11,5}{⼿、⼍}[HSK 4]
  \definition{v.}{representar; retratar; descrever; usar a linguagem e as palavras para transmitir uma imagem concreta de uma pessoa, evento ou situação}
\end{entry}

\begin{entry}{秒}{miao3}{9}{⽲}[HSK 5]
  \definition{adv.}{(coloquial) instantaneamente}
  \definition{s.}{segundo (unidade de tempo) | segundo (unidade de medida angular)}
\end{entry}

\begin{entry}{妙招}{miao4zhao1}{7,8}{⼥、⼿}
  \definition{adj.}{escorregadio}
  \definition{s.}{movimento inteligente | maneira inteligente de fazer algo}
\end{entry}

\begin{entry}{灭火}{mie4huo3}{5,4}{⽕、⽕}
  \definition{s.}{combate a incêndios}
  \definition{v.}{extinguir um incêndio}
\end{entry}

\begin{entry}{民间}{min2jian1}{5,7}{⽒、⾨}[HSK 3]
  \definition{s.}{povo (entre as pessoas) | não governamental; de pessoa para pessoa}
\end{entry}

\begin{entry}{民众}{min2zhong4}{5,6}{⽒、⼈}
  \definition{s.}{a população | as massas | as pessoas comuns}
\end{entry}

\begin{entry}{民主}{min2zhu3}{5,5}{⽒、⼂}
  \definition{adj.}{democrático}
  \definition{s.}{democracia}
\end{entry}

\begin{entry}{民族}{min2zu2}{5,11}{⽒、⽅}[HSK 3]
  \definition[个]{s.}{nação | grupo étnico}
\end{entry}

\begin{entry}{敏感}{min3gan3}{11,13}{⽁、⼼}[HSK 5]
  \definition{adj.}{sensível; descreve pessoas ou animais que rapidamente percebem mudanças ou estímulos externos | reativo; sensível; fácil de causar reações intensas}
\end{entry}

\begin{entry}{名}{ming2}{6}{⼝}[HSK 2]
  \definition*{s.}{sobrenome Ming}
  \definition*{v.}{nome próprio (é) | expressar; descrever | possuir; tomar; ter}
  \definition{adj.}{notável; famoso; conhecido; renomado}
  \definition{clas.}{usado para pessoas | usado para classificação por ordem}
  \definition{s.}{nome; denominação | desculpa; pretexto | fama; reputação}
\end{entry}

\begin{entry}{名称}{ming2 cheng1}{6,10}{⼝、⽲}[HSK 2]
  \definition[个,种]{s.}{nomes, apelidos e formas de se referir a pessoas ou coisas}
\end{entry}

\begin{entry}{名单}{ming2 dan1}{6,8}{⼝、⼗}[HSK 2]
  \definition[个,份]{s.}{lista com nomes de pessoas ou nomes de organizações}
\end{entry}

\begin{entry}{名牌儿}{ming2 pai2r5}{6,12,2}{⼝、⽚、⼉}[HSK 4]
  \definition*{s.}{Marca famosa}
\end{entry}

\begin{entry}{名片}{ming2pian4}{6,4}{⼝、⽚}[HSK 4]
  \definition[张,盒,叠]{s.}{cartão de visita; um pedaço de papel retangular com o nome, o cargo, o endereço etc. impressos}
\end{entry}

\begin{entry}{名人}{ming2 ren2}{6,2}{⼝、⼈}[HSK 4]
  \definition{s.}{celebridade; pessoa famosa}
\end{entry}

\begin{entry}{名字}{ming2zi5}{6,6}{⼝、⼦}[HSK 1]
  \definition[个]{s.}{nome; nome próprio | nome (de uma coisa)}
\end{entry}

\begin{entry}{明白}{ming2bai5}{8,5}{⽇、⽩}[HSK 1]
  \definition{adj.}{claro; óbvio; evidente; inequívoco | sensato; razoável | aberto; franco; inequívoco; explícito}
  \definition{v.}{entender; compreender; saber}
\end{entry}

\begin{entry}{明亮}{ming2 liang4}{8,9}{⽇、⼇}[HSK 5]
  \definition{adj.}{claro; bem iluminado | brilhante; resplandecente | claro; simples; compreensível}
\end{entry}

\begin{entry}{明明}{ming2ming2}{8,8}{⽇、⽇}[HSK 5]
  \definition{adv.}{obviamente; claramente; sem dúvida; indica que o fenômeno ou princípio é evidente}
\end{entry}

\begin{entry}{明年}{ming2 nian2}{8,6}{⽇、⼲}[HSK 1]
  \definition{s.}{próximo ano}
\end{entry}

\begin{entry}{明确}{ming2que4}{8,12}{⽇、⽯}[HSK 3]
  \definition{adj.}{claro; definido; específico}
  \definition{v.}{deixar claro; tornar definitivo}
\end{entry}

\begin{entry}{明天}{ming2tian1}{8,4}{⽇、⼤}[HSK 1]
  \definition{s.}{amanhã | futuro próximo}
\end{entry}

\begin{entry}{明显}{ming2xian3}{8,9}{⽇、⽇}[HSK 3]
  \definition{adj.}{claro; óbvio; distinto}
\end{entry}

\begin{entry}{明星}{ming2xing1}{8,9}{⽇、⽇}[HSK 2]
  \definition[个,位,颗,名]{s.}{estrela; ator, atleta, cantor famosos, etc. | talento de ponta; profissional de destaque; também é usado como metáfora para pessoas ou grupos que se destacam pelo seu bom desempenho ou excelência | estrela brilhante; estrela resplandecente; referindo-se a estrelas muito brilhantes}
\end{entry}

\begin{entry}{明珠}{ming2zhu1}{8,10}{⽇、⽟}
  \definition{s.}{pérola | jóia (de grande valor)}
\end{entry}

\begin{entry}{鸣}{ming2}{8}{⿃}
  \definition{v.}{chorar (pássaros, animais e insetos) | fazer um som | dar voz (gratidão, queixas, etc.)}
\end{entry}

\begin{entry}{命令}{ming4ling4}{8,5}{⼝、⼈}[HSK 5]
  \definition[道,个]{s.}{ordem; comando; instruções emitidas pelos superiores aos subordinados}
  \definition{v.}{ordenar; comandar}
\end{entry}

\begin{entry}{命运}{ming4yun4}{8,7}{⼝、⾡}[HSK 3]
  \definition[个]{s.}{tendência de desenvolvimento; tendência de futuro | destino; sina; sorte}
\end{entry}

\begin{entry}{摸}{mo1}{13}{⼿}[HSK 4]
  \definition{v.}{sentir; acariciar; tocar; tocar (um objeto) levemente com a mão e depois removê-lo ou mover a mão suavemente sobre a superfície do objeto | sentir para; tatear para; sentir algo com as mãos | descobrir; sentir; sondar; explorar; tentar fazer ou entender | sentir o caminho; tatear no escuro; andar por estradas que você não consegue reconhecer | furtar; roubar}
\end{entry}

\begin{entry}{模范}{mo2fan4}{14,9}{⽊、⾋}[HSK 5]
  \definition{adj.}{exemplar}
  \definition{s.}{modelo; exemplo excelente; pessoa exemplar; coisa exemplar; pessoas ou coisas exemplares que servem de modelo}
\end{entry}

\begin{entry}{模仿}{mo2fang3}{14,6}{⽊、⼈}[HSK 5]
  \definition{v.}{copiar; imitar; aprender a fazer algo seguindo um modelo pronto}
\end{entry}

\begin{entry}{模糊}{mo2hu5}{14,15}{⽊、⽶}[HSK 5]
  \definition{adj.}{vago; confuso; indistinto}
  \definition{v.}{confundir; desorientar}
\end{entry}

\begin{entry}{模式}{mo2shi4}{14,6}{⽊、⼷}[HSK 5]
  \definition{s.}{modelo; modo; padrão; a forma padrão de algo ou o modelo padrão que as pessoas podem seguir}
\end{entry}

\begin{entry}{模特儿}{mo2 te4r5}{14,10,2}{⽊、⽜、⼉}[HSK 4]
  \definition[个]{s.}{modelo (pessoa que posa para um fotógrafo ou pintor ou escultor); objeto de representação ou referência usado por artistas para esboços e esculturas, como o corpo humano, objetos, modelos etc.; também se refere aos arquétipos que os estudiosos da literatura usam para retratar seus personagens | modelo (uma pessoa que usa roupas para exibir modas); pessoa ou manequim usado para exibir estilos de roupas}
\end{entry}

\begin{entry}{模型}{mo2xing2}{14,9}{⽊、⼟}[HSK 4]
  \definition[个]{s.}{modelo; padrão; itens feitos em escala com base em objetos ou desenhos | molde; padrão; molde para fundir máquinas, objetos, etc.}
\end{entry}

\begin{entry}{膜拜}{mo2bai4}{14,9}{⾁、⼿}
  \definition{v.}{ajoelhar-se e se curvar com as mãos unidas no nível da testa | ter ou mostrar sentimentos fortes de respeito e admiração por um deus}
\end{entry}

\begin{entry}{摩擦}{mo2ca1}{15,17}{⼿、⼿}[HSK 5]
  \definition{s.}{atrito; desacordo; conflito (entre duas partes); a ação de impedir o movimento relativo entre dois objetos em contato, produzida na superfície de contato | atrito; metáfora para o conflito entre as duas partes}
  \definition{v.}{esfregar}
\end{entry}

\begin{entry}{摩托}{mo2 tuo1}{15,6}{⼿、⼿}[HSK 5]
  \definition[辆]{s.}{(empréstimo linguístico) motor; motor de combustão interna | (empréstimo linguístico) motocicleta, abreviação de 摩托车}
  \seealsoref{摩托车}{mo2tuo1che1}
\end{entry}

\begin{entry}{摩托车}{mo2tuo1che1}{15,6,4}{⼿、⼿、⾞}
  \definition[辆,部]{s.}{(empréstimo linguístico) motocicleta}
\end{entry}

\begin{entry}{磨}{mo2}{16}{⽯}
  \definition{v.}{moer | polir | afiar | desgastar | esfregar}
  \seeref{磨}{mo4}
\end{entry}

\begin{entry}{磨菇}{mo2gu5}{16,11}{⽯、⾋}
  \variantof{蘑菇}
\end{entry}

\begin{entry}{蘑菇}{mo2gu5}{19,11}{⾋、⾋}
  \definition{s.}{cogumelo}
  \definition{v.}{mandriar | embromar | amofinar | incomodar alguém com solicitações ou interrupções frequentes ou persistentes}
\end{entry}

\begin{entry}{魔术}{mo2shu4}{20,5}{⿁、⽊}
  \definition{s.}{magia}
\end{entry}

\begin{entry}{魔头}{mo2tou2}{20,5}{⿁、⼤}
  \definition{s.}{monstro | diabo}
\end{entry}

\begin{entry}{抹泪}{mo3lei4}{8,8}{⼿、⽔}
  \definition{v.}{limpar as lágrimas | (figurativo) derramar lágrimas}
\end{entry}

\begin{entry}{末}{mo4}{5}{⽊}[HSK 4]
  \definition{adj.}{último; final}
  \definition{s.}{ponta; terminal; extremidade; o final de algo | não essenciais; detalhes secundários | fim; final | pó; poeira | um papel na ópera tradicional}
\end{entry}

\begin{entry}{没}{mo4}{7}{⽔}
  \definition{adj.}{último; final}
  \definition{v.}{afundar na água; submergir | transbordar; subir além; exceder ou ultrapassar | esconder-se; desaparecer; sumir; ocultar-se | confiscar; expropriar | morrer}
  \variantof{没}
\end{entry}

\begin{entry}{莫名其妙}{mo4ming2qi2miao4}{10,6,8,7}{⾋、⼝、⼋、⼥}
  \definition{adj.}{desconcertante | bizzaro | inexplicável | perplexo}
\end{entry}

\begin{entry}{墨镜}{mo4jing4}{15,16}{⿊、⾦}
  \definition[只,双,副]{s.}{óculos escuros}
\end{entry}

\begin{entry}{磨}{mo4}{16}{⽯}
  \definition{s.}{mó (pedra pesada e redonda para moinho)}
  \definition{v.}{moer}
\end{entry}

\begin{entry}{默默}{mo4mo4}{16,16}{⿊、⿊}[HSK 4]
  \definition{adj.}{mudo; silencioso}
  \definition{adv.}{silenciosamente}
\end{entry}

\begin{entry}{默契}{mo4qi4}{16,9}{⿊、⼤}
  \definition{adj.}{(de membros da equipe) bem coordenados}
  \definition{s.}{entendimento tácito | entendimento mútuo | conectado em um nível mútuo profundo | (de membros da equipe) bem coordenados}
\end{entry}

\begin{entry}{某}{mou3}{9}{⽊}[HSK 3]
  \definition{pron.}{um certo alguém ou coisa; algum | usado para substituir seu próprio nome}
\end{entry}

\begin{entry}{模具}{mu2ju4}{14,8}{⽊、⼋}
  \definition{s.}{molde | matriz | padrão}
\end{entry}

\begin{entry}{模样}{mu2yang4}{14,10}{⽊、⽊}[HSK 5]
  \definition[副,种]{s.}{aparência; a aparência ou o estilo de vestir de uma pessoa |
indicando uma estimativa aproximada de tempo ou idade; expressão de estimativas relativas a tempo, idade, etc. | tendência; situação; inclinação}
\end{entry}

\begin{entry}{母亲}{mu3qin1}{5,9}{⽏、⼇}[HSK 3]
  \definition[位,个]{s.}{mãe}
\end{entry}

\begin{entry}{母语}{mu3yu3}{5,9}{⽏、⾔}
  \definition{s.}{língua materna | língua nativa}
\end{entry}

\begin{entry}{亩}{mu3}{7}{⼇}
  \definition{clas.}{usado para campos | unidade de área igual a um décimo quinto de um hectare}
\end{entry}

\begin{entry}{木偶}{mu4'ou3}{4,11}{⽊、⼈}
  \definition{s.}{fantoche, marionete}
\end{entry}

\begin{entry}{木头}{mu4tou5}{4,5}{⽊、⼤}[HSK 3]
  \definition{adj.}{estúpido; cabeça-dura}
  \definition[块,根]{s.}{tronco; madeira; viga; prancha}
\end{entry}

\begin{entry}{目标}{mu4biao1}{5,9}{⽬、⽊}[HSK 3]
  \definition[个]{s.}{alvo; objetivo | objetivo; destino}
\end{entry}

\begin{entry}{目的}{mu4di4}{5,8}{⽬、⽩}[HSK 2]
  \definition[个,些,种]{s.}{objetivo; meta; alvo; finalidade; propósito; o lugar ou situação que se deseja alcançar; o resultado que se deseja obter; o centro do alvo}
\end{entry}

\begin{entry}{目光}{mu4guang1}{5,6}{⽬、⼉}[HSK 5]
  \definition[道,束,种]{s.}{olhar fixo; a expressão e atitude reveladas pelos olhos | visão; vista; percepção visual; a linha imaginária formada entre os olhos e o objeto quando se olha para ele | perspicácia (capacidade de observar e reconhecer coisas); conhecimento adquirido através do contato com as coisas, capacidade de observar as coisas}
\end{entry}

\begin{entry}{目前}{mu4qian2}{5,9}{⽬、⼑}[HSK 3]
  \definition{adv.}{agora; recentemente; no momento; no presente}
\end{entry}

\begin{entry}{幕}{mu4}{13}{⼱}
  \definition{s.}{cortina ou tela | dossel ou tenda | quartel de um general | ato (de uma peça)}
\end{entry}

%%%%% EOF %%%%%

