%%%
%%% G
%%%

\section*{G}\addcontentsline{toc}{section}{G}

\begin{entry}{夹}{ga1}{6}{⼤}
  \definition{s.}{axila; sovaco; atualmente, costuma-se escrever ``胳肢窝'' (axila)}
  \seeref{夹}{jia1}
  \seeref{夹}{jia2}
  \seealsoref{胳肢窝}{ga1 zhi1 wo1}
\end{entry}

\begin{entry}{胳肢窝}{ga1 zhi1 wo1}{10,8,12}{⾁、⾁、⽳}
  \definition{s.}{axila; sovaco; também escrito ``夹肢窝''}
  \seealsoref{夹肢窝}{jia1 zhi1 wo1}
\end{entry}

\begin{entry}{该}{gai1}{8}{⾔}[HSK 2]
  \definition{v.}{deveria | é a vez de alguém fazer algo | merecer | dever}
\end{entry}

\begin{entry}{改}{gai3}{7}{⽁}[HSK 2]
  \definition*{s.}{sobrenome Gai}
  \definition{v.}{mudar | transformar | revisar | alterar | modificar | retificar | corrigir | mudar para (fazer outra coisa)}
\end{entry}

\begin{entry}{改变}{gai3bian4}{7,8}{⽁、⼜}[HSK 2]
  \definition{v.}{mudar | alterar | transformar | virar | converter | moldar | modificar}
\end{entry}

\begin{entry}{改革}{gai3ge2}{7,9}{⽁、⾰}[HSK 5]
  \definition[项,次,种]{s.}{reforma; reformação; iniciativas para aprimorar a inovação}
  \definition{v.}{reformar; transformar as antigas partes irracionais das coisas em novas que possam ser adaptadas à situação objetiva}
\end{entry}

\begin{entry}{改进}{gai3jin4}{7,7}{⽁、⾡}[HSK 3]
  \definition[个]{s.}{melhoria}
  \definition{v.}{aprimorar; aperfeiçoar; melhorar; tornar melhor
modificar}
\end{entry}

\begin{entry}{改良}{gai3liang2}{7,7}{⽁、⾉}
  \definition{v.}{melhorar (algo) | reformar (um sistema)}
\end{entry}

\begin{entry}{改善}{gai3shan4}{7,12}{⽁、⼝}[HSK 4]
  \definition{v.}{melhorar; amenizar; mudar a situação original para torná-la melhor}
\end{entry}

\begin{entry}{改善关系}{gai3shan4guan1xi5}{7,12,6,7}{⽁、⼝、⼋、⽷}
  \definition{v.}{melhorar a relação}
\end{entry}

\begin{entry}{改善通讯}{gai3shan4tong1xun4}{7,12,10,5}{⽁、⼝、⾡、⾔}
  \definition{v.}{melhorar a comunicação}
\end{entry}

\begin{entry}{改造}{gai3 zao4}{7,10}{⽁、⾡}[HSK 3]
  \definition{v.}{transformar; renovar | remodelar}
\end{entry}

\begin{entry}{改正}{gai3 zheng4}{7,5}{⽁、⽌}[HSK 4]
  \definition{v.}{corrigir; emendar; mudar o errado para o correto}
\end{entry}

\begin{entry}{芥}{gai4}{7}{⾋}
  \definition{s.}{usado em 芥蓝 \dpy{gai4lan2}}
  \seeref{芥蓝}{gai4lan2}
  \seeref{芥}{jie4}
\end{entry}

\begin{entry}{芥兰}{gai4lan2}{7,5}{⾋、⼋}
  \variantof{芥蓝}
\end{entry}

\begin{entry}{芥蓝}{gai4lan2}{7,13}{⾋、⾋}
  \definition{s.}{brócolis chinês | couve chinesa | mostarda}
  \seeref{格兰菜}{ge2lan2cai4}
\end{entry}

\begin{entry}{盖}{gai4}{11}{⽫}[HSK 4]
  \definition*{s.}{sobrenome Gai}
  \definition{adj.}{excelente; soberbo; fantástico}
  \definition{adv.}{cerca de; ao redor; aproximadamente; expressa um julgamento especulativo sobre algo, ou uma explicação da causa, o que é equivalente a ``大概'' ou ``原来''}
  \definition{conj.}{para; porque; dando continuidade à frase anterior, afirmando a razão ou causa, com tom incerto}
  \definition{s.}{tampa; capa; cobertura; algo que cobre ou sela a parte superior de um objeto | carapaça; concha (de tartaruga, caranguejo, etc.); ossos em formato de crânio em certas partes do corpo humano; as conchas nas costas de certos animais | dossel; capota; toldo | nivelador (uma ferramenta agrícola usada para nivelar terras)}
  \definition{v.}{cobrir; proteger; colocar uma capa em; colocar uma tampa em um objeto | selar; afixar um selo em | superar; sobressair; sobrepujar; ultrapassar | construir; colocar para cima | esconder; ocultar; encobrir | nivelar o terreno com um nivelador (ferramenta agrícola)}
  \seeref{盖}{ge3}
  \seealsoref{大概}{da4gai4}
  \seealsoref{原来}{yuan2lai2}
\end{entry}

\begin{entry}{概括}{gai4kuo4}{13,9}{⽊、⼿}[HSK 4]
  \definition{adj.}{genérico; simples e claro, captando o conteúdo principal}
  \definition{s.}{generalização}
  \definition{v.}{generalizar; resumir}
\end{entry}

\begin{entry}{概念}{gai4nian4}{13,8}{⽊、⼼}[HSK 3]
  \definition[个]{s.}{ideia; noção; conceito; concepção}
\end{entry}

\begin{entry}{干}{gan1}{3}{⼲}[HSK 1][Kangxi 51]
  \definition*{s.}{sobrenome Gan}
  \definition{v.}{preocupar | ignorar | interferir}
  \seeref{干}{gan4}
\end{entry}

\begin{entry}{干杯}{gan1bei1}{3,8}{⼲、⽊}[HSK 2]
  \definition{interj.}{Saúde!}
  \definition{v.+compl.}{fazer um brinde | brindar até a última gota}
\end{entry}

\begin{entry}{干脆}{gan1cui4}{3,10}{⼲、⾁}[HSK 5]
  \definition{adj.}{claro; direto; (falando, fazendo coisas) sem hesitação; atitude clara}
  \definition{adv.}{justamente; diretamente; sem maiores considerações}
\end{entry}

\begin{entry}{干净}{gan1jing4}{3,8}{⼲、⼎}[HSK 1]
  \definition{adj.}{limpo | arrumado}
\end{entry}

\begin{entry}{干你屁事}{gan1 ni3 pi4shi4}{3,7,7,8}{⼲、⼈、⼫、⼅}
  \definition{interj.}{Foda-se!}
\end{entry}

\begin{entry}{干扰}{gan1rao3}{3,7}{⼲、⼿}[HSK 5]
  \definition{v.}{perturbar; incomodar | interferir; interromper o funcionamento adequado de equipamentos eletrônicos com sinais eletrônicos dispersos}
\end{entry}

\begin{entry}{干与}{gan1yu4}{3,3}{⼲、⼀}
  \variantof{干预}
\end{entry}

\begin{entry}{干预}{gan1yu4}{3,10}{⼲、⾴}[HSK 5]
  \definition{s.}{intromissão; intervenção}
  \definition{v.}{intrometer-se; intervir; interpor-se;}
\end{entry}

\begin{entry}{甘薯}{gan1shu3}{5,16}{⽢、⾋}
  \definition{s.}{batata doce}
\end{entry}

\begin{entry}{甘心}{gan1xin1}{5,4}{⽢、⼼}
  \definition{v.}{estar disposto a | resignar-se a}
\end{entry}

\begin{entry}{赶}{gan3}{10}{⾛}[HSK 3]
  \definition*{s.}{sobrenome Gan}
  \definition{prep.}{por; até}
  \definition{v.}{ultrapassar; alcançar | perseguir; correr para; correr atrás; tentar pegar | dirigir | expulsar; afastar | encontrar; deparar-se com; esbarrar em; acontecer com; encontrar-se em (uma situação); aproveitar-se de (uma oportunidade) | ir para}
\end{entry}

\begin{entry}{赶到}{gan3 dao4}{10,8}{⾛、⼑}[HSK 3]
  \definition{v.}{apressar (para algum lugar); avançar de súbito}
\end{entry}

\begin{entry}{赶赴}{gan3fu4}{10,9}{⾛、⾛}
  \definition{v.}{apressar}
\end{entry}

\begin{entry}{赶集}{gan3ji2}{10,12}{⾛、⾫}
  \definition{v.}{ir a uma feira | ir ao mercado}
\end{entry}

\begin{entry}{赶脚}{gan3jiao3}{10,11}{⾛、⾁}
  \definition{v.}{transportar mercadorias para ganhar a vida (especialmente de burro) | trabalhar como carroceiro ou porteiro}
\end{entry}

\begin{entry}{赶紧}{gan3jin3}{10,10}{⾛、⽷}[HSK 3]
  \definition{adv.}{apressadamente; sem demora}
\end{entry}

\begin{entry}{赶快}{gan3kuai4}{10,7}{⾛、⼼}[HSK 3]
  \definition{adv.}{rapidamente; imediatamente}
\end{entry}

\begin{entry}{赶路}{gan3lu4}{10,13}{⾛、⾜}
  \definition{v.}{apressar a jornada | apressar-se}
\end{entry}

\begin{entry}{赶忙}{gan3mang2}{10,6}{⾛、⼼}
  \definition{v.}{acelerar | apressar | se apressar}
\end{entry}

\begin{entry}{赶跑}{gan3pao3}{10,12}{⾛、⾜}
  \definition{v.}{afastar | forçar a saída | repelir}
\end{entry}

\begin{entry}{赶上}{gan3shang4}{10,3}{⾛、⼀}
  \definition{adv.}{a tempo para}
  \definition{v.}{alcançar | ultrapassar}
\end{entry}

\begin{entry}{赶早}{gan3zao3}{10,6}{⾛、⽇}
  \definition{adv.}{o mais breve possível | na primeira oportunidade | antes que seja tarde | quanto antes melhor}
\end{entry}

\begin{entry}{赶走}{gan3zou3}{10,7}{⾛、⾛}
  \definition{v.}{expulsar | voltar atrás}
\end{entry}

\begin{entry}{敢}{gan3}{11}{⽁}[HSK 3]
  \definition{adj.}{ousado; corajoso; bravo}
  \definition{adv.}{talvez; provavelmente}
  \definition{v.}{ousar; aventurar-se | ter a confiança de; ter certeza | tornar ousado; aventurar-se}
\end{entry}

\begin{entry}{敢情}{gan3qing5}{11,11}{⽁、⼼}
  \definition{adv.}{claro | acontece que\dots}
\end{entry}

\begin{entry}{感到}{gan3 dao4}{13,8}{⼼、⼑}[HSK 2]
  \definition{v.}{sentir | perceber}
\end{entry}

\begin{entry}{感动}{gan3dong4}{13,6}{⼼、⼒}[HSK 2]
  \definition{v.}{mover (alguém) | tocar (alguém emocionalmente)}
\end{entry}

\begin{entry}{感觉}{gan3jue2}{13,9}{⼼、⾒}[HSK 2]
  \definition{s.}{sentimento | impressão | sensação}
  \definition{v.}{sentir | perceber}
\end{entry}

\begin{entry}{感冒}{gan3mao4}{13,9}{⼼、⽇}[HSK 3]
  \definition{adj.}{interessado}
  \definition[场,次]{s.}{resfriado; resfriado comum; gripe}
  \definition{v.}{pegar (ter) um resfriado}
\end{entry}

\begin{entry}{感情}{gan3qing2}{13,11}{⼼、⼼}[HSK 3]
  \definition[份,个,种]{s.}{emoção; sentimento | amor; afeição; apego}
\end{entry}

\begin{entry}{感染}{gan3ran3}{13,9}{⼼、⽊}
  \definition{s.}{infecção}
  \definition{v.}{infectar | (figurativo) influenciar}
\end{entry}

\begin{entry}{感受}{gan3shou4}{13,8}{⼼、⼜}[HSK 3]
  \definition{s.}{percepção ; sentimento; experiência}
  \definition{v.}{sentir; sentir (através dos sentidos); experimentar}
\end{entry}

\begin{entry}{感想}{gan3xiang3}{13,13}{⼼、⼼}[HSK 5]
  \definition[个,条]{s.}{pensamentos; impressões; reflexões; resposta do pensamento decorrente da exposição ao mundo exterior}
\end{entry}

\begin{entry}{感谢}{gan3xie4}{13,12}{⼼、⾔}[HSK 2]
  \definition{s.}{gratidão | agradecimento}
\end{entry}

\begin{entry}{感兴趣}{gan3xing4qu4}{13,6,15}{⼼、⼋、⾛}[HSK 4]
  \definition{v.}{estar interessado}
  \seeref{对……感兴趣}{dui4 gan3xing4qu4}
\end{entry}

\begin{entry}{橄榄球}{gan3lan3qiu2}{15,13,11}{⽊、⽊、⽟}
  \definition{s.}{futebol jogado com bola oval (rúgbi, futebol americano, regras australianas, etc.)}
\end{entry}

\begin{entry}{干}{gan4}{3}{⼲}[HSK 1]
  \definition{v.}{fazer | gerenciar | trabalhar | (gíria) matar | (vulgar) foder}
  \seeref{干}{gan1}
\end{entry}

\begin{entry}{干活}{gan4huo2}{3,9}{⼲、⽔}
  \definition{v.+compl.}{trabalhar | trabalhar em um emprego}
\end{entry}

\begin{entry}{干活儿}{gan4huo2r5}{3,9,2}{⼲、⽔、⼉}[HSK 2]
  \definition{v.}{trabalhar em um emprego}
\end{entry}

\begin{entry}{干吗}{gan4 ma2}{3,6}{⼲、⼝}[HSK 3]
  \definition{pron.}{por que?}
  \definition{v.}{o que fazer?}
\end{entry}

\begin{entry}{干什么}{gan4 shen2 me5}{3,4,3}{⼲、⼈、⼃}[HSK 1]
  \definition{v.}{o que fazer? | o que está fazendo?}
\end{entry}

\begin{entry}{刚}{gang1}{6}{⼑}[HSK 2]
  \definition{adj.}{duro (sentido de difícil) | forte}
  \definition{adv.}{apenas | exatamente | há pouco tempo | por muito pouco | assim que}
\end{entry}

\begin{entry}{刚才}{gang1cai2}{6,3}{⼑、⼿}[HSK 2]
  \definition{adv.}{ainda agora | há pouco tempo}
\end{entry}

\begin{entry}{刚刚}{gang1 gang1}{6,6}{⼑、⼑}[HSK 2]
  \definition{adv.}{apenas | apenas agora | um momento atrás | por muito pouco}
\end{entry}

\begin{entry}{扛}{gang1}{6}{⼿}
  \definition{v.}{levantar com as duas mãos | carregar alguma coisa juntos (duas ou mais pessoas)}
  \seeref{扛}{kang2}
\end{entry}

\begin{entry}{杠}{gang1}{7}{⽊}
  \definition{s.}{mastro de bandeira | poste | passarela}
  \seeref{杠}{gang4}
\end{entry}

\begin{entry}{钢}{gang1}{9}{⾦}
  \definition{s.}{aço}
\end{entry}

\begin{entry}{钢笔}{gang1 bi3}{9,10}{⾦、⽵}[HSK 5]
  \definition[支]{s.}{caneta-tinteiro; canetas com ponta metálica}
\end{entry}

\begin{entry}{钢琴}{gang1qin2}{9,12}{⾦、⽟}[HSK 5]
  \definition[架]{s.}{piano}
\end{entry}

\begin{entry}{钢丝}{gang1si1}{9,5}{⾦、⼀}
  \definition{s.}{cabo de aço | corda bamba}
\end{entry}

\begin{entry}{杠}{gang4}{7}{⽊}
  \definition{s.}{vara grossa | barra | linha grossa | padrão, critério | hífen, traço}
  \definition{v.}{marcar com uma linha grossa | afiar (faca, navalha, etc.)}
  \seeref{杠}{gang1}
\end{entry}

\begin{entry}{高}{gao1}{10}{⾼}[HSK 1][Kangxi 189]
  \definition*{s.}{sobrenome Gao}
  \definition{adj.}{alto | acima da média}
  \definition{pron.}{Seu (honorífico)}
\end{entry}

\begin{entry}{高潮}{gao1chao2}{10,15}{⾼、⽔}[HSK 4]
  \definition[个,场]{s.}{maré alta; o nível mais alto da maré em um ciclo de maré | pico; aumento; maré alta; uma metáfora para o estágio mais próspero de desenvolvimento das coisas (diferente de ``低潮'') | (ficção, drama e filmes) clímax}
  \seealsoref{低潮}{di1chao2}
\end{entry}

\begin{entry}{高大}{gao1 da4}{10,3}{⾼、⼤}[HSK 5]
  \definition{adj.}{alto e grande; alto | elevado; sublime; nobre}
\end{entry}

\begin{entry}{高度}{gao1 du4}{10,9}{⾼、⼴}[HSK 5]
  \definition{adj.}{alto; elevado; avançado; alto grau | alta concentração; intenso}
  \definition[个]{s.}{altura; altitude; elevação; distância de baixo para cima; o grau e o nível em que as coisas se desenvolveram}
\end{entry}

\begin{entry}{高尔夫}{gao1'er3fu1}{10,5,4}{⾼、⼩、⼤}
  \definition{s.}{(empréstimo linguístico) \emph{golf}}
\end{entry}

\begin{entry}{高跟鞋}{gao1 gen1 xie2}{10,13,15}{⾼、⾜、⾰}[HSK 5]
  \definition{s.}{salto alto; sapatos de salto alto; sapato feminino com salto mais alto e mais distante do chão}
\end{entry}

\begin{entry}{高级}{gao1ji2}{10,6}{⾼、⽷}[HSK 2]
  \definition{adj.}{sênior | alto escalão | alto nível | alto grau | grau superior | alta qualidade | avançado}
\end{entry}

\begin{entry}{高价}{gao1 jia4}{10,6}{⾼、⼈}[HSK 4]
  \definition{s.}{preço alto; bilhete caro; custo elevado; dispendioso}
\end{entry}

\begin{entry}{高楼}{gao1lou2}{10,13}{⾼、⽊}
  \definition[座]{s.}{edifício alto | edifício de muitos andares | arranha-céu}
\end{entry}

\begin{entry}{高尚}{gao1shang4}{10,8}{⾼、⼩}[HSK 4]
  \definition{adj.}{nobre; elevado; descreve um alto padrão moral e uma boa qualidade de pensamento | significativo e não de mau gosto}
\end{entry}

\begin{entry}{高手}{gao1shou3}{10,4}{⾼、⼿}
  \definition{s.}{\emph{expert} | mestre}
\end{entry}

\begin{entry}{高速}{gao1 su4}{10,10}{⾼、⾡}[HSK 3]
  \definition{adj.}{alta velocidade}
  \definition{s.}{auto-estrada; via expressa}
\end{entry}

\begin{entry}{高速公路}{gao1su4gong1lu4}{10,10,4,13}{⾼、⾡、⼋、⾜}[HSK 3]
  \definition[条]{s.}{via expressa; rodovia; auto-estrada}
\end{entry}

\begin{entry}{高铁}{gao1 tie3}{10,10}{⾼、⾦}[HSK 4]
  \definition{s.}{trem de alta velocidade; trem bala}
\end{entry}

\begin{entry}{高温}{gao1 wen1}{10,12}{⾼、⽔}[HSK 5]
  \definition{s.}{alta temperatura; temperatura elevada; hipertermia; megatemperatura; inferno;}
\end{entry}

\begin{entry}{高效}{gao1xiao4}{10,10}{⾼、⽁}
  \definition{adj.}{eficiente | altamente eficaz}
\end{entry}

\begin{entry}{高兴}{gao1xing4}{10,6}{⾼、⼋}[HSK 1]
  \definition{adj.}{feliz | contente | disposto (a fazer alguma coisa) | de bom humor}
\end{entry}

\begin{entry}{高于}{gao1 yu2}{10,3}{⾼、⼆}[HSK 5]
  \definition{v.}{ser mais alto do que; sobrepujar}
\end{entry}

\begin{entry}{高原}{gao1 yuan2}{10,10}{⾼、⼚}[HSK 5]
  \definition[片]{s.}{planalto continental; planalto | platô}
\end{entry}

\begin{entry}{高中}{gao1 zhong1}{10,4}{⾼、⼁}[HSK 2]
  \definition{s.}{escola secundária | escola de segundo grau}
\end{entry}

\begin{entry}{糕点}{gao1dian3}{16,9}{⽶、⽕}
  \definition{s.}{bolos | pastéis}
\end{entry}

\begin{entry}{糕点店}{gao1dian3 dian4}{16,9,8}{⽶、⽕、⼴}
  \definition{s.}{confeitaria}
\end{entry}

\begin{entry}{糕点师}{gao1dian3 shi1}{16,9,6}{⽶、⽕、⼱}
  \definition{s.}{confeiteiro}
\end{entry}

\begin{entry}{搞}{gao3}{13}{⼿}[HSK 5]
  \definition{v.}{fazer; realizar; estar envolvido em; engajar-se em um estudo, fazer algo em relação a, etc. | fazer; produzir; gerar; trabalhar | iniciar; estabelecer; organizar; configurar | consertar (mudar) alguém; fazer alguém sofrer | obter; assegurar; agarrar |  (seguido de um complemento) fazer com que se torne; produzir um determinado efeito ou resultado}
\end{entry}

\begin{entry}{搞错}{gao3cuo4}{13,13}{⼿、⾦}
  \definition{v.}{cometer um erro}
\end{entry}

\begin{entry}{搞定}{gao3ding4}{13,8}{⼿、⼧}
  \definition{v.}{consertar | resolver}
\end{entry}

\begin{entry}{搞鬼}{gao3gui3}{13,9}{⼿、⿁}
  \definition{v.}{fazer travessuras | fazer truques}
\end{entry}

\begin{entry}{搞好}{gao3 hao3}{13,6}{⼿、⼥}[HSK 5]
  \definition{v.}{fazer um bom trabalho; fazer bem; suar; tornar submisso, tornar útil, por meio de solicitações e presentes amigáveis; amolecer}
\end{entry}

\begin{entry}{搞混}{gao3hun4}{13,11}{⼿、⽔}
  \definition{v.}{confundir}
\end{entry}

\begin{entry}{搞乱}{gao3luan4}{13,7}{⼿、⼄}
  \definition{v.}{estragar | confundir | bagunçar}
\end{entry}

\begin{entry}{搞钱}{gao3qian2}{13,10}{⼿、⾦}
  \definition{v.}{fazer dinheiro | acumular dinheiro}
\end{entry}

\begin{entry}{搞通}{gao3tong1}{13,10}{⼿、⾡}
  \definition{v.}{entender algo}
\end{entry}

\begin{entry}{搞笑}{gao3xiao4}{13,10}{⼿、⽵}
  \definition{adj.}{engraçado | hilário}
  \definition{v.}{fazer as pessoas rirem}
\end{entry}

\begin{entry}{稿纸}{gao3zhi3}{15,7}{⽲、⽷}
  \definition{s.}{rascunho | manuscrito}
\end{entry}

\begin{entry}{告别}{gao4bie2}{7,7}{⼝、⼑}[HSK 3]
  \definition{v.+compl.}{dizer adeus a | deixar; partir de | prestar as últimas homenagens ao falecido}
\end{entry}

\begin{entry}{告急}{gao4ji2}{7,9}{⼝、⼼}
  \definition{v.}{estar em estado de emergência | relatar uma emergência | solicitar assistência de emergência}
\end{entry}

\begin{entry}{告诉}{gao4su4}{7,7}{⼝、⾔}
  \definition{v.}{apresentar queixa | registar uma reclamação}
  \seeref{告诉}{gao4su5}
\end{entry}

\begin{entry}{告诉}{gao4su5}{7,7}{⼝、⾔}[HSK 1]
  \definition{v.}{contar | dar a conhecer | informar}
  \seeref{告诉}{gao4su4}
\end{entry}

\begin{entry}{哥}{ge1}{10}{⼝}[HSK 1]
  \definition{s.}{irmão mais velho}
  \seeref{哥哥}{ge1 ge5}
\end{entry}

\begin{entry}{哥哥}{ge1 ge5}{10,10}{⼝、⼝}[HSK 1]
  \definition[个,位]{s.}{irmão mais velho}
\end{entry}

\begin{entry}{哥们}{ge1men5}{10,5}{⼝、⼈}
  \definition{expr.}{\emph{Brothers!}}
  \definition{s.}{(coloquial) cara | irmão (forma diminuta de tratamento entre homens)}
\end{entry}

\begin{entry}{哥斯拉}{ge1si1la1}{10,12,8}{⼝、⽄、⼿}
  \definition*{s.}{Godzilla}
  \seealsoref{酷斯拉}{ku4si1la1}
\end{entry}

\begin{entry}{鸽子}{ge1zi5}{11,3}{⿃、⼦}
  \definition{s.}{pombo}
\end{entry}

\begin{entry}{搁浅}{ge1qian3}{12,8}{⼿、⽔}
  \definition{v.}{ficar encalhado (navio) | encalhar | (figurativo) encontrar dificuldades e parar}
\end{entry}

\begin{entry}{歌}{ge1}{14}{⽋}[HSK 1]
  \definition[支,首]{s.}{canção | canto}
\end{entry}

\begin{entry}{歌迷}{ge1 mi2}{14,9}{⽋、⾡}
  \definition{s.}{fã de um cantor}
\end{entry}

\begin{entry}{歌曲}{ge1 qu3}{14,6}{⽋、⽈}[HSK 5]
  \definition{s.}{música; obra para as pessoas cantarem, uma combinação de poesia e música}
\end{entry}

\begin{entry}{歌声}{ge1 sheng1}{14,7}{⽋、⼠}[HSK 3]
  \definition{s.}{voz cantada; som de canto}
\end{entry}

\begin{entry}{歌手}{ge1 shou3}{14,4}{⽋、⼿}[HSK 3]
  \definition[个,位,名]{s.}{cantor; vocalista}
\end{entry}

\begin{entry}{阁下}{ge2xia4}{9,3}{⾨、⼀}
  \definition{pron.}{Sua Excelência | Sua Majestade | \emph{Sire}}
\end{entry}

\begin{entry}{格兰菜}{ge2lan2cai4}{10,5,11}{⽊、⼋、⾋}
  \definition{s.}{brócolis chinês | couve chinesa | mostarda}
  \seeref{芥蓝}{gai4lan2}
\end{entry}

\begin{entry}{格外}{ge2wai4}{10,5}{⽊、⼣}[HSK 4]
  \definition{adv.}{especialmente; particularmente; ainda mais; indica mais do que a média | adicionalmente; indica adicional ou extra}
\end{entry}

\begin{entry}{鬲}{ge2}{10}{⿀}
  \definition{s.}{um antigo tripé de cozinha com pernas ocas; uma grande panela de barro}
  \seeref{鬲}{li4}
\end{entry}

\begin{entry}{隔}{ge2}{12}{⾩}[HSK 4]
  \definition{adj.}{seguinte; vizinho}
  \definition{v.}{dividir; separar; bloquear; obstruir | estar a uma distância de, após ou em um intervalo de}
\end{entry}

\begin{entry}{隔壁}{ge2bi4}{12,16}{⾩、⼟}[HSK 5]
  \definition{s.}{vizinho; casas ou pessoas vizinhas | septo; distante (socialmente distante) | anteparo; partição}
\end{entry}

\begin{entry}{隔开}{ge2 kai1}{12,4}{⾩、⼶}[HSK 4]
  \definition{v.}{separar; manter separado; barricar; separar completamente duas pessoas (ou coisas) ou duas partes de uma coisa que estão intimamente unidas}
\end{entry}

\begin{entry}{个}{ge3}{3}{⼈}
  \definition{pron.}{usado em 自个儿}
  \seeref{个}{ge4}
  \seeref{自个儿}{zi4ge3r5}
\end{entry}

\begin{entry}{盖}{ge3}{11}{⽫}
  \definition*{s.}{sobrenome Ge}
  \seeref{盖}{gai4}
\end{entry}

\begin{entry}{个}{ge4}{3}{⼈}[HSK 1]
  \definition{clas.}{para objetos e pessoas em geral}
  \definition{pron.}{isto | aquilo}
  \definition{s.}{indivíduo | tamanho}
  \seeref{个}{ge3}
\end{entry}

\begin{entry}{个别}{ge4bie2}{3,7}{⼈、⼑}[HSK 4]
  \definition{adj.}{muito poucos; excepcionais}
  \definition{adv.}{separadamente; individualmente; isoladamente}
\end{entry}

\begin{entry}{个儿}{ge4r5}{3,2}{⼈、⼉}[HSK 5]
  \definition{s.}{tamanho; altura; estatura; tamanho do corpo ou do objeto |
pessoas ou coisas consideradas isoladamente; referir-se a uma pessoa ou coisa individualmente}
\end{entry}

\begin{entry}{个人}{ge4ren2}{3,2}{⼈、⼈}[HSK 3]
  \definition{pron.}{pessoal; si mesmo}
  \definition[个]{s.}{indivíduo}
\end{entry}

\begin{entry}{个体}{ge4ti3}{3,7}{⼈、⼈}[HSK 4]
  \definition{s.}{pessoa ou organismo individual}
\end{entry}

\begin{entry}{个性}{ge4xing4}{3,8}{⼈、⼼}[HSK 3]
  \definition{s.}{caráter individual; individualidade; personalidade}
\end{entry}

\begin{entry}{个子}{ge4zi5}{3,3}{⼈、⼦}[HSK 2]
  \definition{s.}{altura | estatura}
\end{entry}

\begin{entry}{各}{ge4}{6}{⼝}[HSK 3]
  \definition{adv.}{indica que mais de uma pessoa ou coisa está fazendo algo ou tem um determinado atributo}
  \definition{pron.}{todo; todos; cada | diferentes entre si; vários}
\end{entry}

\begin{entry}{各地}{ge4 di4}{6,6}{⼝、⼟}[HSK 3]
  \definition{s.}{todos os lugares; vários lugares}
\end{entry}

\begin{entry}{各个}{ge4 ge4}{6,3}{⼝、⼈}[HSK 4]
  \definition{adv./pron.}{cada | um a um; um após o outro}
\end{entry}

\begin{entry}{各位}{ge4 wei4}{6,7}{⼝、⼈}[HSK 3]
  \definition{pron.}{todos | cada}
\end{entry}

\begin{entry}{各种}{ge4 zhong3}{6,9}{⼝、⽲}[HSK 3]
  \definition{adv.}{todos os tipos; vários; cada tipo}
\end{entry}

\begin{entry}{各自}{ge4zi4}{6,6}{⼝、⾃}[HSK 3]
  \definition{pron.}{cada; respectivo; por si mesmo}
\end{entry}

\begin{entry}{给}{gei3}{9}{⽷}[HSK 1]
  \definition{prep.}{a | para}
  \definition{v.}{dar | permitir | fazer alguma coisa (para alguém)}
  \seeref{给}{ji3}
\end{entry}

\begin{entry}{给……打电话}{gei3 da3 dian4 hua4}{9,5,5,8}{⽷、⼿、⽥、⾔}
  \definition{expr.}{telefonar para alguém}
  \seeref{打电话}{da3 dian4 hua4}
\end{entry}

\begin{entry}{根}{gen1}{10}{⽊}[HSK 4]
  \definition*{s.}{sobrenome Gen}
  \definition{adv.}{completamente; minuciosamente; radicalmente}
  \definition{clas.}{para objetos finos, alongados}
  \definition{s.}{raiz (de uma planta) | descendentes; posteridade; analogia com as gerações futuras | raiz (abreviação de raiz quadrada) | radical (química, refere-se a radicais carregados) | base; pé; raiz; parte inferior, base ou parte de um objeto que está presa a outra coisa | a parte de baixo das coisas; fonte; a origem  das coisas | base; fundamento}
\end{entry}

\begin{entry}{根本}{gen1ben3}{10,5}{⽊、⽊}[HSK 3]
  \definition{adj.}{básico; essencial; fundamental}
  \definition{adv.}{sempre; simplesmente; absolutamente; de qualquer modo | radically; thoroughly}
  \definition[个]{s.}{base; raiz; fundação}
\end{entry}

\begin{entry}{根据}{gen1ju4}{10,11}{⽊、⼿}[HSK 4]
  \definition[个]{prep.}{com base em; de acordo com; à luz de}
  \definition{s.}{base; fundamentos; razão; fundo; alicerce}
  \definition{v.}{basear; usar algo como premissa para uma conclusão ou como base para uma ação verbal}
\end{entry}

\begin{entry}{跟}{gen1}{13}{⾜}[HSK 1]
  \definition{conj.}{e; com}
  \definition{prep.}{com}
  \definition{v.}{acompanhar junto | seguir de perto | ir com}
\end{entry}

\begin{entry}{跟前}{gen1qian2}{13,9}{⾜、⼑}[HSK 5]
  \definition{s.}{próximo; perto de; na frente de; (na ou para) a presença de alguém | o tempo imediatamente anterior a algum evento; tempo que se aproxima}
  \seeref{跟前}{gen1qian5}
\end{entry}

\begin{entry}{跟前}{gen1qian5}{13,9}{⾜、⼑}
  \definition{v.}{(dos filhos de alguém) viver com alguém (exclusivamente com relação à presença ou ausência de crianças)}
  \seeref{跟前}{gen1qian2}
\end{entry}

\begin{entry}{跟随}{gen1sui2}{13,11}{⾜、⾩}[HSK 5]
  \definition{s.}{seguidor; usado para se referir a alguém que seguiu}
  \definition{v.}{seguir; ir atrás; acompanhar}
\end{entry}

\begin{entry}{更}{geng1}{7}{⽈}
  \definition{s.}{vigia (por exemplo, de sentinela ou guarda)}
  \definition{v.}{alterar ou substituir | experimentar}
  \seeref{更}{geng4}
\end{entry}

\begin{entry}{更换}{geng1 huan4}{7,10}{⽈、⼿}[HSK 5]
  \definition{v.}{alterar; mudar; substituir; comutar}
\end{entry}

\begin{entry}{更新}{geng1xin1}{7,13}{⽈、⽄}[HSK 5]
  \definition{v.}{renovar; atualizar; substituir; remover o antigo e substituir pelo novo}
\end{entry}

\begin{entry}{耕}{geng1}{10}{⽾}
  \definition{v.}{lavrar; arar; cultivar | ganhar a vida; buscar o próprio sustento}
\end{entry}

\begin{entry}{更}{geng4}{7}{⽈}[HSK 2]
  \definition{adv.}{mais | ainda mais}
  \seeref{更}{geng1}
\end{entry}

\begin{entry}{更加}{geng4 jia1}{7,5}{⽈、⼒}[HSK 3]
  \definition{adv.}{mais; ainda mais; em maior grau}
\end{entry}

\begin{entry}{工}{gong1}{3}{⼯}[Kangxi 48]
  \definition{s.}{trabalho | trabalhador | habilidade | profissão | comércio | ofício}
\end{entry}

\begin{entry}{工厂}{gong1chang3}{3,2}{⼯、⼚}[HSK 3]
  \definition[家,座,个]{s.}{fábrica; moinho; planta; obras}
\end{entry}

\begin{entry}{工尺谱}{gong1 che3 pu3}{3,4,14}{⼯、⼫、⾔}
  \definition{s.}{notação musical tradicional chinesa que usa caracteres chineses para representar notas musicais}
\end{entry}

\begin{entry}{工程}{gong1 cheng2}{3,12}{⼯、⽲}[HSK 4]
  \definition[个,项]{s.}{projeto; programa; trabalhos que utilizam equipamentos grandes e complexos, como projetos de reconstrução urbana e projetos de cestas de alimentos, etc. | engenharia; departamentos de produção e manufatura usam equipamentos grandes e complexos para realizar seu trabalho}
\end{entry}

\begin{entry}{工程师}{gong1cheng2shi1}{3,12,6}{⼯、⽲、⼱}[HSK 3]
  \definition[个,名]{s.}{engenheiro}
\end{entry}

\begin{entry}{工夫}{gong1 fu1}{3,4}{⼯、⼤}
  \definition[个]{s.}{tempo | tempo livre; lazer}
  \seeref{工夫}{gong1 fu5}
\end{entry}

\begin{entry}{工夫}{gong1 fu5}{3,4}{⼯、⼤}[HSK 3]
  \definition{s.}{(um período de) tempo | tempo livre}
  \seeref{工夫}{gong1 fu1}
\end{entry}

\begin{entry}{工具}{gong1ju4}{3,8}{⼯、⼋}[HSK 3]
  \definition[个]{s.}{ferramenta; implemento | ferramenta; meio; instrumento}
\end{entry}

\begin{entry}{工龄}{gong1ling2}{3,13}{⼯、⿒}
  \definition{s.}{tempo de serviço | senioridade}
\end{entry}

\begin{entry}{工人}{gong1ren2}{3,2}{⼯、⼈}[HSK 1]
  \definition{s.}{trabalhador | operário | mão de obra de fábrica}
\end{entry}

\begin{entry}{工业}{gong1ye4}{3,5}{⼯、⼀}[HSK 3]
  \definition{s.}{indústria}
\end{entry}

\begin{entry}{工艺}{gong1 yi4}{3,4}{⼯、⾋}[HSK 5]
  \definition{s.}{técnica; tecnologia; arte industrial; técnicas ou métodos de fabricação e processamento de produtos | artesanato; arte artesanal}
\end{entry}

\begin{entry}{工艺品}{gong1 yi4 pin3}{3,4,9}{⼯、⾋、⼝}[HSK 5]
  \definition[个,件]{s.}{trabalho manual; artesanato; habilidade manual; artigo artesanal; itens delicados produzidos com técnicas artesanais. Por exemplo, esculturas em jade, esmaltes Jingtailan, bordados, etc.}
\end{entry}

\begin{entry}{工资}{gong1zi1}{3,10}{⼯、⾙}[HSK 3]
  \definition[份,个,年,月,天]{s.}{pagamento; salário}
\end{entry}

\begin{entry}{工作}{gong1zuo4}{3,7}{⼯、⼈}[HSK 1]
  \definition[个,份,项]{s.}{trabalho | tarefa}
  \definition{v.}{trabalhar | operar (uma máquina)}
\end{entry}

\begin{entry}{工作日}{gong1 zuo4 ri4}{3,7,4}{⼯、⼈、⽇}[HSK 5]
  \definition{s.}{dia de trabalho; dia útil; dias em que você deveria estar trabalhando de acordo com as regras | horas de trabalho por dia; horas do dia para fazer o trabalho necessário}
\end{entry}

\begin{entry}{公布}{gong1bu4}{4,5}{⼋、⼱}[HSK 3]
  \definition{v.}{promulgar; anunciar; publicar; tornar público}
\end{entry}

\begin{entry}{公车}{gong1che1}{4,4}{⼋、⾞}
  \definition{s.}{abreviação de~公共汽车, ônibus}
  \seeref{公共汽车}{gong1gong4qi4che1}
  \seealsoref{公共}{gong1 gong4}
\end{entry}

\begin{entry}{公告}{gong1gao4}{4,7}{⼋、⼝}[HSK 5]
  \definition{s.}{anúncio; notificação de assuntos importantes ao público em geral pelo governo ou por um órgão importante}
  \definition{v.}{anunciar; o governo ou órgão governamental informa publicamente às pessoas algo importante}
\end{entry}

\begin{entry}{公共}{gong1 gong4}{4,6}{⼋、⼋}[HSK 3]
  \definition{adj.}{público; comum; comunal}
  \definition{s.}{ônibus}
  \seealsoref{公车}{gong1che1}
  \seealsoref{公共汽车}{gong1gong4qi4che1}
\end{entry}

\begin{entry}{公共汽车}{gong1gong4qi4che1}{4,6,7,4}{⼋、⼋、⽔、⾞}[HSK 2]
  \definition[辆,班]{s.}{ônibus}
  \seeref{公车}{gong1che1}
  \seealsoref{公共}{gong1 gong4}
\end{entry}

\begin{entry}{公交车}{gong1 jiao1 che1}{4,6,4}{⼋、⼇、⾞}[HSK 2]
  \definition[辆]{s.}{ônibus urbano | veículo de transporte público}
\end{entry}

\begin{entry}{公斤}{gong1jin1}{4,4}{⼋、⽄}[HSK 2]
  \definition{clas.}{quilograma (kg)}
\end{entry}

\begin{entry}{公开}{gong1kai1}{4,4}{⼋、⼶}[HSK 3]
  \definition{adj.}{aberto; público}
  \definition{v.}{tornar público}
\end{entry}

\begin{entry}{公克}{gong1ke4}{4,7}{⼋、⼗}
  \definition{s.}{grama (medida de peso)}
\end{entry}

\begin{entry}{公里}{gong1li3}{4,7}{⼋、⾥}[HSK 2]
  \definition{s.}{quilômetro}
\end{entry}

\begin{entry}{公路}{gong1 lu4}{4,13}{⼋、⾜}[HSK 2]
  \definition[条]{s.}{rodovia | via de trânsito | estrada | auto-estrada}
\end{entry}

\begin{entry}{公民}{gong1min2}{4,5}{⼋、⽒}[HSK 3]
  \definition{s.}{cidadão; civil}
\end{entry}

\begin{entry}{公平}{gong1ping2}{4,5}{⼋、⼲}[HSK 2]
  \definition{adj.}{justo | imparcial | equitativo}
\end{entry}

\begin{entry}{公认}{gong1ren4}{4,4}{⼋、⾔}[HSK 5]
  \definition{v.}{(geralmente) reconhecer; (universalmente) aceitar}
\end{entry}

\begin{entry}{公式}{gong1shi4}{4,6}{⼋、⼷}[HSK 5]
  \definition[个]{s.}{fórmula; expressão}
\end{entry}

\begin{entry}{公司}{gong1si1}{4,5}{⼋、⼝}[HSK 2]
  \definition[家]{s.}{empresa | companhia | corporação | firma}
\end{entry}

\begin{entry}{公司治理}{gong1si1zhi4li3}{4,5,8,11}{⼋、⼝、⽔、⽟}
  \definition{s.}{governança corporativa}
\end{entry}

\begin{entry}{公务员}{gong1 wu4 yuan2}{4,5,7}{⼋、⼒、⼝}[HSK 3]
  \definition[个,名]{s.}{funcionário público}
\end{entry}

\begin{entry}{公用电话}{gong1yong4dian4hua4}{4,5,5,8}{⼋、⽤、⽥、⾔}
  \definition[部]{s.}{telefone público}
\end{entry}

\begin{entry}{公寓}{gong1yu4}{4,12}{⼋、⼧}
  \definition[套]{s.}{prédio de apartamentos | pensão}
\end{entry}

\begin{entry}{公元}{gong1yuan2}{4,4}{⼋、⼉}[HSK 4]
  \definition{s.}{D.C. (Depois de~Cristo); a era cristã; um método internacionalmente aceito de registro de datas, o ano lendário do nascimento de Jesus é 1 d.C., também conhecido como o primeiro ano da Era Comum, e é denotado por D.C.}
  \seealsoref{前}{qian2}
\end{entry}

\begin{entry}{公园}{gong1yuan2}{4,7}{⼋、⼞}[HSK 2]
  \definition[座]{s.}{parque (para recreação pública)}
\end{entry}

\begin{entry}{公正}{gong1zheng4}{4,5}{⼋、⽌}[HSK 5]
  \definition{adj.}{justo; equitativo; imparcial; de mente justa; equidade e integridade sem favoritismo}
\end{entry}

\begin{entry}{功臣}{gong1chen2}{5,6}{⼒、⾂}
  \definition{s.}{oficial meritório | pessoa que presta serviço excepcional, herói | (fig.) alguém que desempenha um papel vital}
\end{entry}

\begin{entry}{功夫}{gong1fu5}{5,4}{⼒、⼤}[HSK 3]
  \definition*{s.}{Gongfu (Kung Fu), arte marcial}
  \definition[番]{s.}{habilidade; feitura | luta acrobática; habilidade em artes marciais | esforço; tempo e energia}
\end{entry}

\begin{entry}{功课}{gong1 ke4}{5,10}{⼒、⾔}[HSK 3]
  \definition[份,门]{s.}{trabalho escolar; dever de casa | tarefa; lições; lição escolar}
\end{entry}

\begin{entry}{功能}{gong1neng2}{5,10}{⼒、⾁}[HSK 3]
  \definition[种,项]{s.}{função}
\end{entry}

\begin{entry}{供应}{gong1 ying4}{8,7}{⼈、⼴}[HSK 4]
  \definition{v.}{fornecer; acomodar}
\end{entry}

\begin{entry}{共}{gong4}{6}{⼋}[HSK 4]
  \definition*{s.}{Abreviação de Partido Comunista | sobrenome Gong}
  \definition{adj.}{conjunto; mútuo; geral; comum; o mesmo para todos}
  \definition{adv.}{juntos; juntamente; conjuntamente | em sua totalidade; em todos}
  \definition{v.}{compartilhar com; empreender ou realizar em conjunto}
\end{entry}

\begin{entry}{共产}{gong4chan3}{6,6}{⼋、⼇}
  \definition{adj.}{comunista}
  \definition{s.}{comunismo}
\end{entry}

\begin{entry}{共产党}{gong4chan3dang3}{6,6,10}{⼋、⼇、⼉}
  \definition*{s.}{Partido Comunista}
\end{entry}

\begin{entry}{共计}{gong4ji4}{6,4}{⼋、⾔}[HSK 5]
  \definition{s.}{total; total geral; agregado; montante}
  \definition{v.}{contar até; somar até; totalizar}
\end{entry}

\begin{entry}{共同}{gong4tong2}{6,6}{⼋、⼝}[HSK 3]
  \definition{adj.}{comum; compartilhado; colaborativo}
  \definition{adv.}{juntos; conjuntamente}
\end{entry}

\begin{entry}{共同体}{gong4tong2ti3}{6,6,7}{⼋、⼝、⼈}
  \definition{s.}{comunidade}
\end{entry}

\begin{entry}{共享}{gong4 xiang3}{6,8}{⼋、⼇}[HSK 5]
  \definition{v.}{compartilhar; desfrutar juntos; aproveitar as coisas boas juntos}
\end{entry}

\begin{entry}{共有}{gong4 you3}{6,6}{⼋、⽉}[HSK 3]
  \definition{v.}{ter completamente; compartilhar; possuir (por todos)}
\end{entry}

\begin{entry}{勾}{gou1}{4}{⼓}
  \definition*{s.}{sobrenome Gou}
  \definition{v.}{atrair | excitar | marcar | atacar | delinear | conspirar}
  \variantof{钩}
  \seeref{勾}{gou4}
\end{entry}

\begin{entry}{沟}{gou1}{7}{⽔}[HSK 5]
  \definition[条]{s.}{canal; vala; sarjeta; trincheira; cursos d'água ou fortificações escavados | ranhura; sulco raso; uma depressão que se assemelha a uma vala | ravina; barranco; cursos d'água}
\end{entry}

\begin{entry}{沟通}{gou1tong1}{7,10}{⽔、⾡}[HSK 5]
  \definition{v.}{comunicar; comunicar-se para entender as ideias, opiniões, etc. | conectar; ligar; estabelecer um paralelo entre os dois}
\end{entry}

\begin{entry}{钩}{gou1}{9}{⾦}
  \definition{s.}{gancho | \emph{check mark} | \emph{tick}}
  \definition{v.}{enganchar | costurar}
\end{entry}

\begin{entry}{狗}{gou3}{8}{⽝}[HSK 2]
  \definition[条,只]{s.}{cão | cachorro}
\end{entry}

\begin{entry}{勾}{gou4}{4}{⼓}
  \definition{s.}{usado em 勾当}
  \seeref{勾}{gou1}
  \seeref{勾当}{gou4dang4}
\end{entry}

\begin{entry}{勾当}{gou4dang4}{4,6}{⼓、⼹}
  \definition{s.}{negócio obscuro}
\end{entry}

\begin{entry}{句}{gou4}{5}{⼝}
  \variantof{勾}
  \seeref{句}{ju4}
\end{entry}

\begin{entry}{构}{gou4}{8}{⽊}
  \definition{s.}{composição literária}
  \definition{v.}{construir | formar | compor}
  \variantof{够}
\end{entry}

\begin{entry}{构成}{gou4cheng2}{8,6}{⽊、⼽}[HSK 4]
  \definition{s.}{parte; componente; composição; estrutura}
  \definition{v.}{formar; compor; constituir; compor; encaixar muitas partes para formar um todo | consistir; causar; formar (principalmente em termos jurídicos)}
\end{entry}

\begin{entry}{构造}{gou4 zao4}{8,10}{⽊、⾡}[HSK 4]
  \definition[种]{s.}{estrutura; construção; disposição, organização e inter-relação dos componentes}
  \definition{v.}{formar; construir}
\end{entry}

\begin{entry}{诟骂}{gou4ma4}{8,9}{⾔、⾺}
  \definition{v.}{abusar verbalmente | insultar | criticar}
\end{entry}

\begin{entry}{购买}{gou4 mai3}{8,6}{⾙、⼄}[HSK 4]
  \definition{v.}{comprar; adquirir; empobrecer}
\end{entry}

\begin{entry}{购物}{gou4wu4}{8,8}{⾙、⽜}[HSK 4]
  \definition{s.}{compras; itens comprados; \emph{shopping}}
\end{entry}

\begin{entry}{够}{gou4}{11}{⼣}[HSK 2]
  \definition{adj.}{suficiente}
  \definition{adv.}{(antes do adj.) realmente}
  \definition{v.}{bastar | chegar}
\end{entry}

\begin{entry}{够本}{gou4ben3}{11,5}{⼣、⽊}
  \definition{v.}{empatar | fazer valer o dinheiro}
\end{entry}

\begin{entry}{够不着}{gou4bu5zhao2}{11,4,11}{⼣、⼀、⽬}
  \definition{v.}{ser incapaz de alcançar}
\end{entry}

\begin{entry}{够得着}{gou4de5zhao2}{11,11,11}{⼣、⼻、⽬}
  \definition{v.}{estar à altura | alcançar}
\end{entry}

\begin{entry}{够格}{gou4ge2}{11,10}{⼣、⽊}
  \definition{adj.}{apto | qualificado | apresentável}
\end{entry}

\begin{entry}{够朋友}{gou4peng2you5}{11,8,4}{⼣、⽉、⼜}
  \definition{v.}{ser um amigo verdadeiro}
\end{entry}

\begin{entry}{够呛}{gou4qiang4}{11,7}{⼣、⼝}
  \definition{adj.}{suficiente | terrível | insuportável | improvável}
\end{entry}

\begin{entry}{够戗}{gou4qiang4}{11,8}{⼣、⼽}
  \variantof{够呛}
\end{entry}

\begin{entry}{够味}{gou4wei4}{11,8}{⼣、⼝}
  \definition{adj.}{excelente | na medida}
\end{entry}

\begin{entry}{彀}{gou4}{13}{⼸}
  \definition{s.}{calcance de um arco e flecha}
  \definition{v.}{puxar um arco ao máximo}
\end{entry}

\begin{entry}{估计}{gu1ji4}{7,4}{⼈、⾔}[HSK 5]
  \definition{v.}{fazer contas; estimar; calcular; julgar a natureza, quantidade, mudança, etc. de uma coisa em uma determinada situação | parecer; parecer como se; aparentar; fazer inferências aproximadas sobre a natureza, a quantidade e a mudança das coisas com base em determinadas circunstâncias}
\end{entry}

\begin{entry}{姑娘}{gu1niang5}{8,10}{⼥、⼥}[HSK 3]
  \definition[位,个]{s.}{menina; jovem senhora; mulher solteira | filha}
\end{entry}

\begin{entry}{姑且}{gu1qie3}{8,5}{⼥、⼀}
  \definition{adv.}{provisoriamente | por enquanto}
\end{entry}

\begin{entry}{孤独}{gu1du2}{8,9}{⼦、⽝}
  \definition{adj.}{solitário}
\end{entry}

\begin{entry}{古}{gu3}{5}{⼝}[HSK 3]
  \definition*{s.}{sobrenome Gu}
  \definition{adj.}{arcaico; antigo; antiquíssimo}
  \definition{pref.}{``paleo''; ``arqueo''}
  \definition{s.}{antiguidade; arcaísmo | livros ou ortodoxias de antigos sábios | uma forma de poesia pré-Tang}
\end{entry}

\begin{entry}{古城}{gu3cheng2}{5,9}{⼝、⼟}
  \definition{s.}{cidade antiga}
\end{entry}

\begin{entry}{古代}{gu3dai4}{5,5}{⼝、⼈}[HSK 3]
  \definition{s.}{tempos antigos | sociedade antiga; sociedade primitiva | antigamente}
\end{entry}

\begin{entry}{古老}{gu3 lao3}{5,6}{⼝、⽼}[HSK 5]
  \definition{adj.}{antigo; antiquado; histórico}
\end{entry}

\begin{entry}{古人}{gu3ren2}{5,2}{⼝、⼈}
  \definition{s.}{pessoas dos tempos antigos | os antigos | espécies humanas extintas, como \emph{Homo erectus} ou \emph{Homo neanderthalensis} | (literário) pessoa falecida}
\end{entry}

\begin{entry}{古铜色}{gu3tong2 se4}{5,11,6}{⼝、⾦、⾊}
  \definition{s.}{cor bronze}
\end{entry}

\begin{entry}{谷}{gu3}{7}{⾕}[Kangxi 150]
  \definition{adj.}{bom; gentil;}
  \definition{s.}{vale; ravina; desfiladeiro; garganta; faixa estreita de terra com uma saída no meio de duas colinas ou dois platôs | arroz não descascado | salário de funcionário (na época feudal) |calha; cocho; canal | fossa sob o cerebelo (anatomia); valécula | dificuldade; dilema}
  \definition{v.}{criar (filhos) | crescer}
\end{entry}

\begin{entry}{骨}{gu3}{9}{⾻}[Kangxi 188]
  \definition{s.}{osso}
\end{entry}

\begin{entry}{骨头}{gu3tou5}{9,5}{⾻、⼤}[HSK 4]
  \definition[根,块]{s.}{osso; tecidos mais duros no corpo de uma pessoa ou de alguns animais que sustentam o corpo ou protegem os órgãos do corpo | caráter de uma pessoa; refere-se à qualidade do caráter de uma pessoa}
\end{entry}

\begin{entry}{鼓}{gu3}{13}{⿎}[HSK 5]
  \definition*{s.}{sobrenome Gu}
  \definition{adj.}{abaulado; inchado; saliente; protuberante}
  \definition{clas.}{unidades antigas de cronometragem noturna; vigílias da noite}
  \definition{s.}{tambor; instrumento de percussão |
coisas semelhantes a tambores; formato, som e função semelhantes aos de um tambor |}
  \definition{v.}{soar; bater; golpear; fazer um objeto soar | ventilar; soprar com fole | agitar; despertar; ativar; incitar; revigorar | bater asas | aumentar; fazer beicinho}
\end{entry}

\begin{entry}{鼓励}{gu3li4}{13,7}{⿎、⼒}[HSK 5]
  \definition{v.}{incitar; encorajar; provocar e incentivar}
\end{entry}

\begin{entry}{鼓掌}{gu3zhang3}{13,12}{⿎、⼿}[HSK 5]
  \definition{v.+compl.}{aplaudir; bater palmas, principalmente para expressar felicidade, aprovação ou boas-vindas}
\end{entry}

\begin{entry}{固定}{gu4ding4}{8,8}{⼞、⼧}[HSK 4]
  \definition{adj.}{fixo; regular; inalterado ou imóvel}
  \definition{v.}{consertar; tornar fixo, não mover novamente; colocar as coisas em ordem, não mudá-las novamente}
\end{entry}

\begin{entry}{故}{gu4}{9}{⽁}
  \definition{conj.}{por isso | portanto | então}
\end{entry}

\begin{entry}{故宫}{gu4gong1}{9,9}{⽁、⼧}
  \definition*{s.}{Palácio Imperial | Cidade Proibida}
\end{entry}

\begin{entry}{故事}{gu4shi4}{9,8}{⽁、⼅}
  \definition{s.}{prática antiga}
  \seeref{故事}{gu4shi5}
\end{entry}

\begin{entry}{故事}{gu4shi5}{9,8}{⽁、⼅}[HSK 2]
  \definition{s.}{narrativa | história | conto}
  \seeref{故事}{gu4shi4}
\end{entry}

\begin{entry}{故乡}{gu4xiang1}{9,3}{⽁、⼄}[HSK 3]
  \definition[个]{s.}{cidade natal; terra natal}
\end{entry}

\begin{entry}{故意}{gu4yi4}{9,13}{⽁、⼼}[HSK 2]
  \definition{adv.}{intencionalmente | deliberadamente | propositalmente}
\end{entry}

\begin{entry}{顾客}{gu4ke4}{10,9}{⾴、⼧}[HSK 2]
  \definition[位]{s.}{cliente}
\end{entry}

\begin{entry}{顾问}{gu4wen4}{10,6}{⾴、⾨}[HSK 5]
  \definition{s.}{conselheiro; consultor; assessor; pessoas com conhecimento especializado ou experiência contratadas para prestar consultoria a organizações ou indivíduos}
\end{entry}

\begin{entry}{瓜}{gua1}{5}{⽠}[HSK 4][Kangxi 97]
  \definition*{s.}{sobrenome Gua}
  \definition[个]{s.}{qualquer tipo de melão ou cabaça | companheiro (termo depreciativo para uma pessoa)}
\end{entry}

\begin{entry}{刮}{gua1}{8}{⼑}
  \definition{v.}{ventar | soprar (vento)}
\end{entry}

\begin{entry}{刮风}{gua1feng1}{8,4}{⼑、⾵}
  \definition{v.+compl.}{ventar | fazer vento}
\end{entry}

\begin{entry}{挂}{gua4}{9}{⼿}[HSK 3]
  \definition{clas.}{para conjuntos ou sequência de itens}
  \definition{v.}{pendurar; colocar; suspender | interromper chamada (telefônica) | colocar alguém em contato com; ligar; telefonar
pegar carona; ser pego | ter em mente; estar preocupado com | ser revestido com; ser coberto com | colocar em registro; registrar}
\end{entry}

\begin{entry}{挂号}{gua4hao4}{9,5}{⼿、⼝}
  \definition{v.+compl.}{registrar-se (em um hospital, etc.) | enviar através de carta registrada}
\end{entry}

\begin{entry}{挂号信}{gua4hao4xin4}{9,5,9}{⼿、⼝、⼈}
  \definition{s.}{carta registrada}
\end{entry}

\begin{entry}{乖乖}{guai1guai1}{8,8}{⼃、⼃}
  \definition{adj.}{bem-comportado (criança) | obediente}
  \seeref{乖乖}{guai1guai5}
\end{entry}

\begin{entry}{乖乖}{guai1guai5}{8,8}{⼃、⼃}
  \definition{expr.}{Graças a Deus! | Oh meu Deus!}
  \seeref{乖乖}{guai1guai1}
\end{entry}

\begin{entry}{拐}{guai3}{8}{⼿}
  \definition{s.}{bengala | muleta}
  \definition{v.}{virar (uma esquina, etc.) | cortar | sequestrar | fraudar | apropriar-se indevidamente}
\end{entry}

\begin{entry}{怪}{guai4}{8}{⼼}[HSK 4,5]
  \definition*{s.}{sobrenome Guai}
  \definition{adj.}{estranho; esquisito; peculiar; excêntrico; pitoresco; monstruoso; desconcertante; anormal; incomum}
  \definition{adv.}{bastante; muito}
  \definition{s.}{monstro; demônio; diabo; ser maligno}
  \definition{v.}{achar algo estranho; admirar; ficar surpreso | culpar; repreender}
\end{entry}

\begin{entry}{怪癖}{guai4pi3}{8,18}{⼼、⽧}
  \definition{adj.}{peculiar}
  \definition{s.}{excentricidade | peculiaridade | hobby estranho}
\end{entry}

\begin{entry}{怪兽}{guai4shou4}{8,11}{⼼、⼋}
  \definition{s.}{animal raro | animal mítico | monstro}
\end{entry}

\begin{entry}{关}{guan1}{6}{⼋}[HSK 1,4]
  \definition*{s.}{sobrenome Guan}
  \definition{s.}{passagem; ponto de controle | alfândega; escritórios de cobrança de impostos para exportação e importação de mercadorias | ponto de inflexão ou barreira; ponto de virada ou dificuldade | momento crítico; mecanismo}
  \definition{v.}{fechar; encerrar; amarrar algo | fechar; trancar | encerrar; sair do mercado; falir | conceder ou sacar o pagamento de um salário | desligar | envolver; preocupar-se; conectar-se}
\end{entry}

\begin{entry}{关闭}{guan1bi4}{6,6}{⼋、⾨}[HSK 4]
  \definition{v.}{fechar | (empresa) falir}
\end{entry}

\begin{entry}{关怀}{guan1huai2}{6,7}{⼋、⼼}[HSK 5]
  \definition{v.}{mostrar cuidado amoroso por; mostrar solicitude por; cuidar, amar, apoiar ou ajudar os fracos ou grupos em dificuldade | geralmente usado para superiores para subordinados, anciãos para juniores ou organizações para indivíduos}
\end{entry}

\begin{entry}{关机}{guan1 ji1}{6,6}{⼋、⽊}[HSK 2]
  \definition{v.}{encerrar | finalizar | desligar}
\end{entry}

\begin{entry}{关键}{guan1jian4}{6,13}{⼋、⾦}[HSK 5]
  \definition{adj.}{crucial; decisivo; importante; que pode determinar o curso e o resultado dos eventos}
  \definition[个]{s.}{chave; ponto crucial; aspectos ou condições mais importantes que determinam o desenvolvimento e o resultado de algo}
\end{entry}

\begin{entry}{关上}{guan1 shang5}{6,3}{⼋、⼀}[HSK 1]
  \definition{v.}{fechar (uma porta) | fechar | desligar (luz, equipamento elétrico etc.)}
\end{entry}

\begin{entry}{关系}{guan1xi5}{6,7}{⼋、⽷}[HSK 3]
  \definition[个,种]{s.}{relações; conexões; relacionamento | consequência; impacto; significado | causa; razão (geralmente usado com 由于 ou 因为) | credenciais que mostram filiação a uma organização}
  \definition{v.}{preocupar; afetar; ter influência sobre; ter a ver com}
  \seealsoref{因为}{yin1wei4}
  \seealsoref{由于}{you2yu2}
\end{entry}

\begin{entry}{关心}{guan1xin1}{6,4}{⼋、⼼}[HSK 2]
  \definition{v.}{cuidar de | preocupar-se com | expressar interesse em | mostrar solicitude por}
\end{entry}

\begin{entry}{关于}{guan1yu2}{6,3}{⼋、⼆}[HSK 4]
  \definition{prep.}{sobre; relativo a; pertencente a; uma questão de; com relação a}
\end{entry}

\begin{entry}{关注}{guan1 zhu4}{6,8}{⼋、⽔}[HSK 3]
  \definition{s.}{preocupação; interesse; atenção}
  \definition{v.}{prestar atenção em; seguir algo de perto; seguir (nas redes sociais)}
\end{entry}

\begin{entry}{观察}{guan1cha2}{6,14}{⾒、⼧}[HSK 3]
  \definition{v.}{assistir; pesquisar; observar}
\end{entry}

\begin{entry}{观点}{guan1dian3}{6,9}{⾒、⽕}[HSK 2]
  \definition{s.}{ponto de vista | perspectiva}
\end{entry}

\begin{entry}{观看}{guan1 kan4}{6,9}{⾒、⽬}[HSK 3]
  \definition{v.}{assistir; ver}
\end{entry}

\begin{entry}{观念}{guan1nian4}{6,8}{⾒、⼼}[HSK 3]
  \definition[个]{s.}{ideia; conceito}
\end{entry}

\begin{entry}{观众}{guan1zhong4}{6,6}{⾒、⼈}[HSK 3]
  \definition[位,名,批,个]{s.}{espectador; audiência}
\end{entry}

\begin{entry}{官}{guan1}{8}{⼧}[HSK 4]
  \definition*{s.}{sobrenome Guan}
  \definition{adj.}{propriedade do governo; pertencente ao governo ou ao público | público}
  \definition[个,位]{s.}{funcionário do governo; oficial; servidor público; titular de cargo; funcionário público nomeado acima de um determinado nível | órgão (parte do tecido do corpo)}
\end{entry}

\begin{entry}{官方}{guan1fang1}{8,4}{⼧、⽅}[HSK 4]
  \definition{s.}{autoridade; (do ou pelo) governo | oficial (de uma organização ou instituição)}
\end{entry}

\begin{entry}{官桂}{guan1gui4}{8,10}{⼧、⽊}
  \definition{s.}{canela}
  \seealsoref{肉桂}{rou4gui4}
\end{entry}

\begin{entry}{冠}{guan1}{9}{⼍}
  \definition{s.}{chapéu | coroa | brasão | boné}
  \seeref{冠}{guan4}
\end{entry}

\begin{entry}{棺}{guan1}{12}{⽊}
  \definition{s.}{caixão | esquife | ataúde}
\end{entry}

\begin{entry}{管}{guan3}{14}{⽵}[HSK 3]
  \definition*{s.}{sobrenome Guan}
  \definition{adj.}{estreito; restrito; limitado; pequeno}
  \definition{clas.}{para objetos cilíndricos finos}
  \definition{conj.}{não importa (o que, como, etc.)}
  \definition{prep.}{a função é semelhante a ``把'', usada especificamente em conjunto com ``叫''.}
  \definition[根,条,排]{s.}{cano; tubo | instrumento musical de sopro | válvula; tubo | duto; canal; vasos}
  \definition{v.}{estar encarregado de; gerenciar; executar; supervisionar | administrar; governar | sujeitar alguém à disciplina | assumir; arcar | interferir; incomodar | garantir; assegurar; fornecer}
\end{entry}

\begin{entry}{管家}{guan3jia1}{14,10}{⽵、⼧}
  \definition{s.}{mordomo | governanta}
  \definition{v.}{administrar uma casa}
\end{entry}

\begin{entry}{管理}{guan3li3}{14,11}{⽵、⽟}[HSK 3]
  \definition{v.}{gerenciar; executar; administrar; governar; estar encarregado de | controlar; gerenciar | cuidar de}
\end{entry}

\begin{entry}{冠}{guan4}{9}{⼍}
  \definition*{s.}{sobrenome Guan}
  \definition{v.}{colocar um chapéu | ser o primeiro | dublar}
  \seeref{冠}{guan1}
\end{entry}

\begin{entry}{冠军}{guan4jun1}{9,6}{⼍、⼍}[HSK 5]
  \definition[个]{s.}{campeão; medalhista de ouro; primeiro lugar em esportes e outras competições}
\end{entry}

\begin{entry}{光}{guang1}{6}{⼉}[HSK 3]
  \definition*{s.}{sobrenome Guang}
  \definition{adj.}{suave; brilhante | nu; despido; descoberto | esgotado; sem nada sobrando | glorioso; gracioso | brilhante}
  \definition{adv.}{somente; sozinho; meramente}
  \definition{s.}{luz; raio | cenário | honra; glória; brilho | claridade | favor; graça | momento | corpo celeste}
  \definition{v.}{glorificar; recuperar; reconquistar | estar nu | brilhar}
\end{entry}

\begin{entry}{光临}{guang1lin2}{6,9}{⼉、⼁}[HSK 4]
  \definition{v.}{honrar com sua presença, uma palavra de honra, usada para dizer que um convidado chegou}
\end{entry}

\begin{entry}{光明}{guang1ming2}{6,8}{⼉、⽇}[HSK 3]
  \definition{adj.}{brilhante | ingênuo | justo; honesto}
  \definition{s.}{luz}
\end{entry}

\begin{entry}{光盘}{guang1pan2}{6,11}{⼉、⽫}[HSK 4]
  \definition[片,张]{s.}{CD; disco compacto; um disco circular feito de plástico rígido composto que usa um laser para registrar e ler informações}
\end{entry}

\begin{entry}{光槃}{guang1pan2}{6,14}{⼉、⽊}
  \variantof{光盘}
\end{entry}

\begin{entry}{光荣}{guang1rong2}{6,9}{⼉、⾋}[HSK 5]
  \definition{adj.}{honroso; honrado; glorioso; por fazer algo que é benéfico para o país ou para a coletividade e que é considerado por todos como digno de respeito ou elogio}
  \definition{s.}{honra; glória; crédito; sentimento de honra decorrente do fato de ser respeitado ou elogiado por fazer algo importante ou grandioso}
\end{entry}

\begin{entry}{光污染}{guang1 wu1ran3}{6,6,9}{⼉、⽔、⽊}
  \definition{s.}{poluição luminosa}
\end{entry}

\begin{entry}{光线}{guang1 xian4}{6,8}{⼉、⽷}[HSK 5]
  \definition[条,道]{s.}{luz; feixe luminoso; raio de luz}
\end{entry}

\begin{entry}{广}{guang3}{3}{⼴}[HSK 5]
  \definition{adj.}{largo; vasto; amplo; extenso; (oposto a ``狭'')}
  \seeref{广}{an1}
  \seeref{广}{yan3}
  \seealsoref{狭}{xia2}
\end{entry}

\begin{entry}{广播}{guang3bo1}{3,15}{⼴、⼿}[HSK 3]
  \definition[个]{s.}{programa de rádio; transmissão (de rádio)}
  \definition{v.}{transmitir; estar no ar | espalhar-se amplamente; ser conhecido em toda parte}
\end{entry}

\begin{entry}{广场}{guang3chang3}{3,6}{⼴、⼟}[HSK 2]
  \definition{s.}{praça | praça pública | esplanada}
\end{entry}

\begin{entry}{广场舞}{guang3chang3wu3}{3,6,14}{⼴、⼟、⾇}
  \definition{s.}{quadrilha, uma rotina de exercícios tocada com música em quadrados públicos, parques e praças, popular especialmente entre mulheres de meia-idade e aposentados na China}
\end{entry}

\begin{entry}{广大}{guang3da4}{3,3}{⼴、⼤}[HSK 3]
  \definition{adj.}{muito difundido | (uma área ou espaço) vasto; extenso; em grande escala | numeroso}
\end{entry}

\begin{entry}{广东}{guang3dong1}{3,5}{⼴、⼀}
  \definition*{s.}{Guangdong}
\end{entry}

\begin{entry}{广泛}{guang3fan4}{3,7}{⼴、⽔}[HSK 5]
  \definition{adj.}{amplo; extenso; de grande alcance; disseminado; escopo e cobertura amplos}
\end{entry}

\begin{entry}{广告}{guang3gao4}{3,7}{⼴、⼝}[HSK 2]
  \definition[项]{s.}{publicidade | anúncio publicitário}
\end{entry}

\begin{entry}{逛}{guang4}{10}{⾡}[HSK 4]
  \definition{v.}{perambular; passear; vaguear}
\end{entry}

\begin{entry}{归}{gui1}{5}{⼹}[HSK 4]
  \definition*{s.}{sobrenome Gui}
  \definition{s.}{divisão no ábaco com divisor de um dígito}
  \definition{v.}{retornar; voltar para; voltar (ou ir) | devolver algo a; dar de volta a | convergir; juntar-se | encarregar alguém de algo | atribuir a; pertencer a | usado entre os mesmos verbos, indicando que a ação não levou ao resultado correspondente}
\end{entry}

\begin{entry}{龟速}{gui1su4}{7,10}{⿔、⾡}
  \definition{adv.}{tão lento quanto uma tartaruga}
\end{entry}

\begin{entry}{规定}{gui1ding4}{8,8}{⾒、⼧}[HSK 3]
  \definition[个,条,项]{s.}{regra; regulamento; estipulação}
  \definition{v.}{estipular; prover; prescrever}
\end{entry}

\begin{entry}{规范}{gui1fan4}{8,9}{⾒、⾋}[HSK 3]
  \definition{adj.}{regular; normal; padrão; atendendo às especificações}
  \definition{s.}{norma; padrão}
  \definition{v.}{regular; padronizar}
\end{entry}

\begin{entry}{规划}{gui1hua4}{8,6}{⾒、⼑}[HSK 5]
  \definition{s.}{plano; projeto; planejamento; programa; programação; esquematização; plano de desenvolvimento de longo prazo mais abrangente |}
  \definition{v.}{planejar; programar;}
\end{entry}

\begin{entry}{规律}{gui1lv4}{8,9}{⾒、⼻}[HSK 4]
  \definition{adj.}{estável; regular; coisas, comportamentos, fenômenos, etc. que ocorrem em um determinado momento}
  \definition{s.}{lei; padrão regular; conexão essencial e recorrente entre as coisas}
\end{entry}

\begin{entry}{规模}{gui1mo2}{8,14}{⾒、⽊}[HSK 4]
  \definition[个]{s.}{escala; escopo; dimensões; padrão, forma ou escopo (de um empreendimento, instituição, projeto, movimento, etc.)}
\end{entry}

\begin{entry}{规则}{gui1ze2}{8,6}{⾒、⼑}[HSK 4]
  \definition{adj.}{ordenado; regular; descreve a forma, estrutura, arranjo, etc., que se conformam a uma determinada maneira organizada}
  \definition{s.}{regra; regulamento; sistema ou código de conduta prescrito para observância comum | lei; norma}
\end{entry}

\begin{entry}{鬼}{gui3}{9}{⿁}[HSK 5]
  \definition*{s.}{Gui, uma das mansões lunares | sobrenome Gui}
  \definition{adj.}{evasivo; furtivo; sub-reptício; ardiloso; enganoso, malicioso; obscuro | terrível; ruim; severo; vil | esperto; astuto; inteligente}
  \definition{s.}{espírito; fantasma; aparição; refere-se à alma de uma pessoa após a morte | usado para formar um termo de abuso para caráter ignóbil; refere-se a pessoas que têm maus hábitos ou cujo comportamento é repugnante | companheiro; pessoa que é considerada divertida}
\end{entry}

\begin{entry}{鬼怪}{gui3guai4}{9,8}{⿁、⼼}
  \definition{s.}{\emph{hobgoblin} | bicho-papão | fantasma}
\end{entry}

\begin{entry}{鬼火}{gui3huo3}{9,4}{⿁、⽕}
  \definition{s.}{fogo-fátuo | boitatá | fogo corredor | fogo de santelmo}
\end{entry}

\begin{entry}{柜子}{gui4 zi5}{8,3}{⽊、⼦}[HSK 5]
  \definition[个]{s.}{gabinete; armário; dispositivo para guardar roupas, documentos, livros, etc.}
\end{entry}

\begin{entry}{贵}[⻉]{gui4}[中一⻉]{9}{⾙}[HSK 1]
  \definition{adj.}{caro | nobre | precioso}
\end{entry}

\begin{entry}{贵姓}{gui4xing4}{9,8}{⾙、⼥}
  \definition{expr.}{qual seu sobrenome?}
\end{entry}

\begin{entry}{跪拜}{gui4bai4}{13,9}{⾜、⼿}
  \definition{v.}{prostrar-se | ajoelhar-se e adorar}
\end{entry}

\begin{entry}{滚}{gun3}{13}{⽔}[HSK 5]
  \definition*{s.}{sobrenome Gun}
  \definition{adj.}{rolante | fervente | precipitado; torrencial}
  \definition{adv.}{muito; em um grau elevado}
  \definition{v.}{rolar; girar; virar | escapar; fugir; ir embora | ferver | amarrar; aparar; fazer bainha}
\end{entry}

\begin{entry}{滚滚}{gun3gun3}{13,13}{⽔、⽔}
  \definition*{s.}{Apelido para um panda}
  \definition{v.}{mover-se | rolar | fluir continuamente}
\end{entry}

\begin{entry}{滚轮}{gun3lun2}{13,8}{⽔、⾞}
  \definition{s.}{pneu | dial rotativo | roda de rolagem (\emph{scroll})  (mouse de computador)}
\end{entry}

\begin{entry}{过}{guo1}{6}{⾡}
  \definition*{s.}{sobrenome Guo}
  \seeref{过}{guo4}
  \seeref{过}{guo5}
\end{entry}

\begin{entry}{锅}{guo1}{12}{⾦}[HSK 5]
  \definition[口,只]{s.}{panela; frigideira; utensílios de cozinha, redondos e côncavos, feitos principalmente de ferro, alumínio, etc. | parte que se parece com um pote em alguns objetos}
\end{entry}

\begin{entry}{囯}{guo2}{7}{⼞}
  \variantof{国}
\end{entry}

\begin{entry}{国}{guo2}{8}{⼞}[HSK 1]
  \definition*{s.}{sobrenome Guo}
  \definition[个]{s.}{país | nação}
\end{entry}

\begin{entry}{国宾馆}{guo2bin1guan3}{8,10,11}{⼞、⼧、⾷}
  \definition{s.}{pousada estadual}
\end{entry}

\begin{entry}{国歌}{guo2ge1}{8,14}{⼞、⽋}
  \definition{s.}{hino nacional}
\end{entry}

\begin{entry}{国籍}{guo2ji2}{8,20}{⼞、⽵}[HSK 5]
  \definition{s.}{nacionalidade; cidadania; refere-se à identidade de um indivíduo como pertencente a um Estado | identidade nacional (de um avião, navio, etc.)}
\end{entry}

\begin{entry}{国际}{guo2ji4}{8,7}{⼞、⾩}[HSK 2]
  \definition{adj.}{internacional}
\end{entry}

\begin{entry}{国际儿童节}{guo2ji4 er2tong2jie2}{8,7,2,12,5}{⼞、⾩、⼉、⽴、⾋}
  \definition*{s.}{Dia Internacional das Crianças (1~de~junho)}
\end{entry}

\begin{entry}{国际妇女节}{guo2ji4 fu4nv3jie2}{8,7,6,3,5}{⼞、⾩、⼥、⼥、⾋}
  \definition*{s.}{Dia Internacional das Mulheres (8~de~março)}
\end{entry}

\begin{entry}{国际劳动节}{guo2ji4 lao2dong4 jie2}{8,7,7,6,5}{⼞、⾩、⼒、⼒、⾋}
  \definition*{s.}{Dia Internacional dos Trabalhadores (1~de~maio)}
\end{entry}

\begin{entry}{国家}{guo2jia1}{8,10}{⼞、⼧}[HSK 1]
  \definition[个]{s.}{país | nação | estado}
\end{entry}

\begin{entry}{国民}{guo2 min2}{8,5}{⼞、⽒}[HSK 5]
  \definition[个]{s.}{membro de uma nação; povo de uma nação}
\end{entry}

\begin{entry}{国内}{guo2 nei4}{8,4}{⼞、⼌}[HSK 3]
  \definition{s.}{interno (a um país); doméstico; lar}
\end{entry}

\begin{entry}{国旗}{guo2qi2}{8,14}{⼞、⽅}
  \definition[面]{s.}{bandeira (de um país)}
\end{entry}

\begin{entry}{国庆}{guo2 qing4}{8,6}{⼞、⼴}[HSK 3]
  \definition*{s.}{Dia Nacional}
\end{entry}

\begin{entry}{国庆节}{guo2qing4jie2}{8,6,5}{⼞、⼴、⾋}
  \definition*{s.}{Dia Nacional (1~de~outubro)}
\end{entry}

\begin{entry}{国人}{guo2ren2}{8,2}{⼞、⼈}
  \definition{s.}{compatriota}
\end{entry}

\begin{entry}{国外}{guo2 wai4}{8,5}{⼞、⼣}[HSK 1]
  \definition{adj.}{no exterior | externo (assuntos) | estrangeiro}
\end{entry}

\begin{entry}{国语}{guo2yu3}{8,9}{⼞、⾔}
  \definition*{s.}{Língua Chinesa (Mandarim), enfatizando sua natureza nacional}
\end{entry}

\begin{entry}{果酱}{guo3jiang4}{8,13}{⽊、⾣}
  \definition{s.}{geléia | compota ou doce (de frutas)}
\end{entry}

\begin{entry}{果然}{guo3ran2}{8,12}{⽊、⽕}[HSK 3]
  \definition{adv.}{realmente; como esperado; com certeza}
  \definition{conj.}{se realmente; se de fato}
\end{entry}

\begin{entry}{果实}{guo3shi2}{8,8}{⽊、⼧}[HSK 4]
  \definition[种]{s.}{fruta; o órgão que se desenvolve a partir do ovário ou com outras partes da flor após a fertilização da flor | ganhos; frutos;  uma metáfora para conquista ou recompensa por trabalho árduo}
\end{entry}

\begin{entry}{果汁}{guo3zhi1}{8,5}{⽊、⽔}[HSK 3]
  \definition[杯,瓶,种]{s.}{suco; suco de fruta}
\end{entry}

\begin{entry}{果子}{guo3zi5}{8,3}{⽊、⼦}
  \definition{s.}{fruta}
\end{entry}

\begin{entry}{过}{guo4}{6}{⾡}[HSK 1,2]
  \definition{part.}{passado}
  \definition{v.}{atravessar | passar (tempo)}
  \seeref{过}{guo1}
  \seeref{过}{guo5}
\end{entry}

\begin{entry}{过不惯}{guo4 bu5 guan4}{6,4,11}{⾡、⼀、⼼}
  \definition{v.}{não se acostumar | não se habituar}
  \seeref{过惯}{guo4guan4}
\end{entry}

\begin{entry}{过程}{guo4cheng2}{6,12}{⾡、⽲}[HSK 3]
  \definition[个]{s.}{curso dos eventos; processo}
\end{entry}

\begin{entry}{过度}{guo4du4}{6,9}{⾡、⼴}[HSK 5]
  \definition{adj.}{excessivo; acima do limite; além do limite; além do que é apropriado}
\end{entry}

\begin{entry}{过分}{guo4fen4}{6,4}{⾡、⼑}[HSK 4]
  \definition{adj.}{excessivo; muito longe; demais; falar ou agir além dos limites ou graus adequados}
  \definition{adv.}{excessivamente; indevidamente; muito mesmo}
\end{entry}

\begin{entry}{过关}{guo4guan1}{6,6}{⾡、⼋}
  \definition{v.+compl.}{passar uma barreira | passar por uma provação | passar em um teste | atingir um padrão | passar pela alfândega}
\end{entry}

\begin{entry}{过惯}{guo4guan4}{6,11}{⾡、⼼}
  \definition{v.}{estar acostumado (a um certo estilo de vida, etc.)}
  \seeref{过不惯}{guo4 bu5 guan4}
\end{entry}

\begin{entry}{过节}{guo4jie2}{6,5}{⾡、⾋}
  \definition{v.+compl.}{celebrar festividades | comemorar um festival}
\end{entry}

\begin{entry}{过来}{guo4 lai2}{6,7}{⾡、⽊}[HSK 2]
  \definition{v.}{atravessar (para a minha localização) | vir até aqui}
\end{entry}

\begin{entry}{过敏}{guo4min3}{6,11}{⾡、⽁}[HSK 5]
  \definition{adj.}{sensível; excessivamente sensível; resposta acima do normal; ceticismo excessivo}
  \definition{v.}{ser alérgico a}
\end{entry}

\begin{entry}{过年}{guo4 nian2}{6,6}{⾡、⼲}[HSK 2]
  \definition{v.+compl.}{comemorar o Ano Novo | comemorar o Festival da Primavera | passar o Ano Novo | passar o Festival da Primavera}
\end{entry}

\begin{entry}{过期}{guo4qi1}{6,12}{⾡、⽉}
  \definition{v.+compl.}{exceder a data | passar a data | expirar (passar a data de expiração)}
\end{entry}

\begin{entry}{过去}{guo4 qu4}{6,5}{⾡、⼛}[HSK 2,3]
  \definition{v.}{atravessar, passar por (a partir da minha localização) | falecer}
\end{entry}

\begin{entry}{过瘾}{guo4yin3}{6,16}{⾡、⽧}
  \definition{adj.}{gratificante | imensamente agradável | satisfatório}
  \definition{v.+compl.}{satisfazer um desejo | se divertir com algo}
\end{entry}

\begin{entry}{过于}{guo4yu2}{6,3}{⾡、⼆}[HSK 5]
  \definition{adv.}{demais; indevidamente; excessivamente; advérbios de grau ou quantidade excessiva}
\end{entry}

\begin{entry}{过}{guo5}{6}{⾡}
  \definition{part.}{(marcador de ação experiente)}
  \seeref{过}{guo1}
  \seeref{过}{guo4}
\end{entry}

%%%%% EOF %%%%%

