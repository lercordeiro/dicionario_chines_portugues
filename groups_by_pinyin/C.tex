%%%
%%% C
%%%

\section*{C}\addcontentsline{toc}{section}{C}

\begin{entry}{擦}{ca1}{17}{⼿}[HSK 4]
  \definition{v.}{enxugar; esfregar; apagar; limpar; limpar esfregando com um pano, toalha de mão, etc. | espalhar sobre; colocar sobre | passar raspando | ralar (em pedaços); ralar frutas em um ralador para fazer fios finos}
\end{entry}

\begin{entry}{擦拭}{ca1shi4}{17,9}{⼿、⼿}
  \definition{v.}{limpar com um pano}
\end{entry}

\begin{entry}{猜}{cai1}{11}{⽝}[HSK 5]
  \definition{v.}{adivinhar; conjecturar; especular | suspeitar; ser cauteloso com os outros; desconfiar dos outros}
\end{entry}

\begin{entry}{猜测}{cai1 ce4}{11,9}{⽝、⽔}[HSK 5]
  \definition[个,种]{s.}{advinhação; conjectura; suposição; especulação}
  \definition{v.}{adivinhar; conjecturar; especular; estimar a partir da imaginação}
\end{entry}

\begin{entry}{才}{cai2}{3}{⼿}[HSK 2,4]
  \definition*{s.}{sobrenome Cai}
  \definition{adv.}{há pouco; agora mesmo | (precedido por uma expressão de tempo) não até | (precedido por uma expressão de razão ou condição) não a menos que; não até que; então e somente então; por nenhuma outra razão | (seguido por uma expressão numérica) apenas; indica um intervalo pequeno ou uma quantidade reduzida, equivalente a 仅仅 ou 只 | (em uma afirmação ou negação, enfatizando o que vem antes de 才, geralmente com 呢 no final da frase) na verdade; realmente | dica que algo acontece tarde ou termina tarde | (precedido por uma expressão de tempo) não até; indicando que não era assim, mas agora surgiu uma nova situação | (precedido por uma expressão de razão ou condição) a menos que; indica que só em determinadas condições e, em seguida, como | (expressa ênfase )}
  \definition{s.}{habilidade; talento; dom | pessoa competente | pessoas de um determinado tipo (frequentemente usado como sufixo) | dotação; talento; habilidade}
  \seealsoref{呢}{ne5}
\end{entry}

\begin{entry}{才华}{cai2hua2}{3,6}{⼿、⼗}
  \definition[份]{s.}{talento}
\end{entry}

\begin{entry}{才略}{cai2lve4}{3,11}{⼿、⽥}
  \definition{s.}{habilidade e sagacidade}
\end{entry}

\begin{entry}{才能}{cai2 neng2}{3,10}{⼿、⾁}[HSK 3]
  \definition[间]{s.}{talento; habilidade; dom; capacidade; inteligência e habilidade}
\end{entry}

\begin{entry}{材}{cai2}{7}{⽊}
  \definition[份]{s.}{madeira | material; geralmente se refere a coisas que podem ser transformadas diretamente em produtos acabados | material; materiais para escrita ou referência | pessoa capaz; pessoas talentosas | habilidade; talento; aptidão | caixão}
\end{entry}

\begin{entry}{材料}{cai2liao4}{7,10}{⽊、⽃}[HSK 4]
  \definition[份,个,种]{s.}{material; algo para fazer um produto acabado | material (figura de linguagem) | dados; material para estudo, pesquisa, etc.; conteúdo de uma obra}
\end{entry}

\begin{entry}{财}{cai2}{7}{⾙}
  \definition[笔]{s.}{riqueza; dinheiro; fortuna | propriedade; objetos de valor; um termo geral para dinheiro e materiais}
\end{entry}

\begin{entry}{财产}{cai2chan3}{7,6}{⾙、⼇}[HSK 4]
  \definition{s.}{ativos; propriedade; pertences; refere-se à posse de riqueza material, como dinheiro, bens, casas, terras, etc.}
\end{entry}

\begin{entry}{财富}{cai2fu4}{7,12}{⾙、⼧}[HSK 4]
  \definition{s.}{riqueza; fortuna}
\end{entry}

\begin{entry}{裁}{cai2}{12}{⾐}
  \definition{clas.}{divisão de papel de impressão de tamanho padrão}
  \definition{s.}{planejamento | tipo de escrita | planejamento mental; arranjo e seleção, usados principalmente na literatura e na arte | sanção; restrição | estilo; forma | (impressão) tamanho do corte de papel}
  \definition{v.}{cortar (papel, tecido, etc.) em partes | reduzir; cortar; dispensar | julgar; decidir | verificar; sancionar | cortar; eliminar; remover coisas desnecessárias ou redundantes | discernir; medir; julgar}
\end{entry}

\begin{entry}{裁判}{cai2pan4}{12,7}{⾐、⼑}[HSK 5]
  \definition[个,位,名]{s.}{árbitro; juiz; pessoa que desempenha funções de arbitragem em esportes e outras competições}
  \definition{v.}{arbitrar; atuar como árbitro; em esportes e outras atividades competitivas, julgar o desempenho dos atletas, vitórias e derrotas, classificações e problemas que ocorrem durante a competição de acordo com as regras da competição | julgar; refere-se a um terceiro que faz um julgamento quando surge uma disputa entre duas partes}
\end{entry}

\begin{entry}{采}{cai3}{8}{⾤}
  \definition*{s.}{sobrenome Cai}
  \definition{s.}{espírito; tez; cor e expressão facial | cores}
  \definition{v.}{escolher; arrancar; reunir; colher (flores, folhas, frutas) | minerar; extrair | reunir; coletar | adotar; pegar; selecionar}
  \seeref{采}{cai4}
\end{entry}

\begin{entry}{采访}{cai3fang3}{8,6}{⾤、⾔}[HSK 4]
  \definition{s.}{cobertura; entrevista; coleta de notícias; entrevistas, pesquisas, gravações de áudio e vídeo, etc., com o objetivo de coletar os materiais necessários}
  \definition{v.}{cobrir; entrevistar; reunir novas informações}
\end{entry}

\begin{entry}{采购}{cai3gou4}{8,8}{⾤、⾙}[HSK 5]
  \definition{s.}{comprador; responsável pelas compras}
  \definition{v.}{adquirir; comprar; fazer compras para uma organização; fazer compras para uma empresa}
\end{entry}

\begin{entry}{采取}{cai3qu3}{8,8}{⾤、⼜}[HSK 3]
  \definition{v.}{adotar; escolha da implementação (diretrizes, políticas, métodos, ações, etc.) | reunir; coletar; tomar; assumir}
\end{entry}

\begin{entry}{采用}{cai3 yong4}{8,5}{⾤、⽤}[HSK 3]
  \definition{v.}{selecionar e usar; adotar; considerar adequado e utilizar}
\end{entry}

\begin{entry}{彩}{cai3}{11}{⼺}
  \definition{s.}{cor | aplausos; vivas | variedade; brilho; esplendor | prêmio; loteria | sangue de uma ferida | habilidades especiais empregadas em mágica ou ópera para alcançar um efeito desejado | seda colorida | cores variadas | graça na arte; graciosidade | prêmio de loteria; ganhos | efeitos especiais no teatro chinês (simbolizando sangue, fogo, etc.)}
\end{entry}

\begin{entry}{彩虹}{cai3hong2}{11,9}{⼺、⾍}
  \definition[道]{s.}{arco-íris}
\end{entry}

\begin{entry}{彩票}{cai3piao4}{11,11}{⼺、⽰}[HSK 5]
  \definition[张]{s.}{bilhete de loteria}
\end{entry}

\begin{entry}{彩色}{cai3 se4}{11,6}{⼺、⾊}[HSK 3]
  \definition[个,种]{s.}{multicolorido; cor; várias cores}
\end{entry}

\begin{entry}{采}{cai4}{8}{⾤}
  \definition{s.}{atribuição a um nobre feudal; a terra (incluindo os escravos que cultivavam a terra) concedida pelos antigos príncipes aos nobres; também chamada de feudo}
  \seeref{采}{cai3}
\end{entry}

\begin{entry}{菜}{cai4}{11}{⾋}[HSK 1]
  \definition*{s.}{sobrenome Cai}
  \definition{adj.}{pouca habilidade; baixo nível; baixa capacidade}
  \definition[棵,个,道]{s.}{legumes; verduras; plantas que podem ser usadas como alimentos complementares | óleo de canola | prato; item ou prato do cardápio (seja de carne ou de vegetais)}
\end{entry}

\begin{entry}{菜单}{cai4dan1}{11,8}{⾋、⼗}[HSK 2]
  \definition[个,分,张]{s.}{menu; lista de pratos | menu (para computadores); lista utilizada para selecionar várias operações diferentes}
\end{entry}

\begin{entry}{菜刀}{cai4dao1}{11,2}{⾋、⼑}
  \definition[把]{s.}{faca de vegetais | faca de cozinha | cutelo}
\end{entry}

\begin{entry}{参}{can1}{8}{⼛}
  \definition{v.}{juntar-se; entrar; tomar parte em; participar | referir; consultar; comparar com outros materiais | ligar para prestar homenagem a; fazer uma visita |  (significado antigo) acusar um funcionário perante o imperador; relatar ou expor ao imperador | explorar e compreender (verdade, significado, etc.)}
\end{entry}

\begin{entry}{参观}{can1guan1}{8,6}{⼛、⾒}[HSK 2]
  \definition{v.}{visitar; dar uma olhada; observação no local (resultados do trabalho, carreira, instalações, locais históricos e pontos turísticos, etc.)}
\end{entry}

\begin{entry}{参加}{can1jia1}{8,5}{⼛、⼒}[HSK 2]
  \definition{v.}{aderir (a organizações); participar; participar (de atividades); participar de alguma organização ou atividade | dar (conselho, sugestão, etc.)}
\end{entry}

\begin{entry}{参考}{can1kao3}{8,6}{⼛、⽼}[HSK 4]
  \definition{v.}{consultar; referir-se a; acessar informações relevantes para estudo ou pesquisa | consultar; referir-se a; lidar com coisas, observar, ler, aprender e usar materiais relevantes}
\end{entry}

\begin{entry}{参与}{can1yu4}{8,3}{⼛、⼀}[HSK 4]
  \definition{v.}{participar de; tomar parte em; ter uma mão em; envolver-se em; participar (no planejamento, discussão e condução dos assuntos)}
\end{entry}

\begin{entry}{餐}{can1}{16}{⾷}
  \definition{clas.}{comer; fazer uma refeição}
  \definition{clas.}{usado para refeições}
  \definition{s.}{comida; refeição}
\end{entry}

\begin{entry}{餐馆}{can1 guan3}{16,11}{⾷、⾷}[HSK 5]
  \definition[家]{s.}{restaurante;}
\end{entry}

\begin{entry}{餐厅}{can1ting1}{16,4}{⾷、⼚}[HSK 5]
  \definition[家]{s.}{restaurante; refeitório em um hotel | cantina, refeitório; também é chamado de 食堂}
  \definition[间]{s.}{sala de jantar}
  \seealsoref{食堂}{shi2 tang2}
\end{entry}

\begin{entry}{餐饮}{can1 yin3}{16,7}{⾷、⾷}[HSK 5]
  \definition{s.}{comidas e bebidas; refere-se a atividades de bufê em restaurantes e hotéis}
\end{entry}

\begin{entry}{残}{can2}{9}{⽍}
  \definition{adj.}{incompleto; fragmentário; deficiente | remanescente; restante | cruel; feroz | opressivo; selvagem; bárbaro}
  \definition{v.}{ferir; danificar | estragar; prejudicar; destruir}
\end{entry}

\begin{entry}{残疾人}{can2ji2ren2}{9,10,2}{⽍、⽧、⼈}
  \definition{s.}{pessoa com deficiência}
\end{entry}

\begin{entry}{残酷}{can2ku4}{9,14}{⽍、⾣}
  \definition{adj.}{cruel}
  \definition{s.}{crueldade}
\end{entry}

\begin{entry}{蚕}{can2}{10}{⾍}
  \definition[只,条]{s.}{bicho-da-seda; um inseto que pode fiar seda e fazer casulos}
\end{entry}

\begin{entry}{蚕纸}{can2zhi3}{10,7}{⾍、⽷}
  \definition{s.}{papel onde o bicho-da-seda põe seus ovos}
\end{entry}

\begin{entry}{惨}{can3}{11}{⽕}
  \definition{adj.}{miserável; trágico | cruel; brutal; implacável | desastroso; terrível; esmagador | lamentável; desaventurado | em um grau sério; grau grave; dano grave | selvagem; desumano; vicioso; cruel}
\end{entry}

\begin{entry}{舱}{cang1}{10}{⾈}
  \definition{s.}{cabine (de um avião ou navio) | módulo (de uma nave espacial) | espaço em um navio ou aeronave para transportar pessoas, carga ou máquinas}
\end{entry}

\begin{entry}{操}{cao1}{16}{⼿}
  \definition*{s.}{sobrenome Cao}
  \definition[节,套]{s.}{exercício; ginástica | conduta; comportamento; moralidade, a moral e o código de conduta que as pessoas seguem}
  \definition{v.}{segurar; agarrar; segurar na mão | fazer algo; envolver-se em | falar (uma língua ou dialeto) | treinar (tropas); exercitar (corpo); praticar ou treinar de acordo com uma determinada forma ou postura | dirigir; manusear}
\end{entry}

\begin{entry}{操场}{cao1chang3}{16,6}{⼿、⼟}[HSK 4]
  \definition[个]{s.}{\emph{playground}; campo esportivo; locais para exercícios físicos ou exercícios militares}
\end{entry}

\begin{entry}{操心}{cao1xin1}{16,4}{⼿、⼼}
  \definition{v.+compl.}{preocupar-se com}
\end{entry}

\begin{entry}{操作}{cao1zuo4}{16,7}{⼿、⼈}[HSK 4]
  \definition{s.}{operação}
  \definition{v.}{operar; seguir os requisitos e procedimentos prescritos| implementar; realizar; executar; refere-se à implementação concreta (planos, medidas, etc.)}
\end{entry}

\begin{entry}{槽}{cao2}{15}{⽊}
  \definition{clas.}{usado para portas | usado para porcos}
  \definition[个,道]{s.}{cocho | sulco; entalhe | canal | manjedoura (para água, ração animal, vinho, cuba); um recipiente para alimentar o gado, geralmente é retangular, alto em todos os lados e côncavo no meio, como uma caixa sem tampa | tanque de fermentação; cuba de vinho; geralmente se refere a certos utensílios com lados altos e côncavos no meio | leito do rio; fossa; refere-se a certos cursos d'água ou valas com lados altos e um meio côncavo | ranhura; fenda; uma depressão semelhante a um sulco em um objeto}
\end{entry}

\begin{entry}{草}{cao3}{9}{⾋}[HSK 2]
  \definition*{s.}{sobrenome Cao}
  \definition{adj.}{descuidado; rude | rascunho; inicial | femea; na linguagem coloquial, refere-se a animais domésticos e aves fêmeas | precipitado; pouco cuidadoso | rascunho; não definitivo; preliminar; informal}
  \definition[种,棵,撮,株,根]{s.}{grama; gramado | palha | campo; zona rural; área selvagem | letra cursiva | letra cursiva (ou caligráfica) de um alfabeto fonético | rascunho | caligrafia cursiva; um tipo de escrita chinesa}
  \definition{v.}{esboçar; redigir}
\end{entry}

\begin{entry}{草地}{cao3 di4}{9,6}{⾋、⼟}[HSK 2]
  \definition[片,块]{s.}{prado; gramado; campo; pastagem ou grande área de terra plantada com pastagem | gramado; relvado; local com grama alta ou gramado}
\end{entry}

\begin{entry}{草莓}{cao3mei2}{9,10}{⾋、⾋}
  \definition[颗]{s.}{morango}
\end{entry}

\begin{entry}{草原}{cao3 yuan2}{9,10}{⾋、⼚}[HSK 5]
  \definition[片,个]{s.}{estepe; pradaria; grandes áreas de terra coberta de vegetação em áreas semiáridas, intercaladas com árvores tolerantes à seca}
\end{entry}

\begin{entry}{草纸}{cao3zhi3}{9,7}{⾋、⽷}
  \definition{s.}{papel pardo | pergaminho | papel de palha áspero | papel higiênico}
\end{entry}

\begin{entry}{肏}{cao4}{8}{⼊}
  \definition{v.}{(vulgar) foder; palavras sujas usadas para insultar pessoas; refere-se à relação sexual masculina}
\end{entry}

\begin{entry}{册}{ce4}{5}{⼌}[HSK 5]
  \definition{clas.}{usado para cópias de livros}
  \definition{s.}{volume; livro | cópia; volume | ordem imperial para conferir um título}
  \definition{v.}{conferir um título}
\end{entry}

\begin{entry}{厕}{ce4}{8}{⼚}
  \definition[个,间]{s.}{latrina; fossa sanitária; (componente formador de palavras)}
  \seeref{厕}{si5}
  \seealsoref{茅厕}{mao2ce4}
\end{entry}

\begin{entry}{厕所}{ce4suo3}{8,8}{⼚、⼾}
  \definition[间,处]{s.}{lavatório | \emph{toilette}}
\end{entry}

\begin{entry}{厕纸}{ce4zhi3}{8,7}{⼚、⽷}
  \definition{s.}{papel higiênico}
\end{entry}

\begin{entry}{测}{ce4}{9}{⽔}[HSK 4]
  \definition{v.}{pesquisar; sondar; medir | conjecturar; advinhar}
\end{entry}

\begin{entry}{测量}{ce4liang2}{9,12}{⽔、⾥}[HSK 4]
  \definition{v.}{aferir; pesquisar; medir; determinar valores relevantes para espaço, tempo, temperatura, velocidade, função, etc.}
\end{entry}

\begin{entry}{测试}{ce4 shi4}{9,8}{⽔、⾔}[HSK 4]
  \definition[个]{s.}{exame; teste; medição do conhecimento humano, das habilidades ou do funcionamento de máquinas, ferramentas ou instrumentos}
  \definition{v.}{examinar | testar, medição do desempenho e da precisão de máquinas, instrumentos, aparelhos, etc.}
\end{entry}

\begin{entry}{策}{ce4}{12}{⽵}
  \definition*{s.}{sobrenome Ce}
  \definition[个,项,根]{s.}{plano; esquema | tiras de bambu ou madeira usadas para escrever na China antiga | questões sobre atualidades definidas para os exames imperiais | chicote de montaria antigo | um tipo de ensaio na China antiga; um estilo de escrita para exames antigos | estratégia; método}
  \definition{v.}{chicotear (um cavalo) com um chicote de montaria | incitar com um chicote de cavalo, espora}
\end{entry}

\begin{entry}{策划}{ce4hua4}{12,6}{⽵、⼑}
  \definition{s.}{planejador | produtor | plano}
  \definition{v.}{esquematizar | engenhar | planejar}
\end{entry}

\begin{entry}{层}{ceng2}{7}{⼫}[HSK 2]
  \definition{clas.}{usado para coisas que se sobrepõem e se acumulam, como andares, camadas e estratos | usado para coisas que podem ser divididas em itens e etapas | usado para coisas que podem ser removidas ou apagadas da superfície de um objeto}
  \definition{s.}{camada; nível; coisas que se sobrepõem | nível; classificação; camada}
  \definition{v.}{sobrepor; empilhar camada sobre camada}
\end{entry}

\begin{entry}{层层}{ceng2ceng2}{7,7}{⼫、⼫}
  \definition{s.}{camada sobre camada}
\end{entry}

\begin{entry}{层次}{ceng2ci4}{7,6}{⼫、⽋}[HSK 5]
  \definition{s.}{disposição ordenada do conteúdo (de um discurso ou texto) | nível ou estrutura administrativa; distinções entre a mesma coisa devido a diferenças de tamanho, altura, etc. | nível; níveis de afiliação}
\end{entry}

\begin{entry}{曾}{ceng2}{12}{⽈}[HSK 4]
  \definition{adv.}{indica que uma ação já aconteceu ou um estado já existiu}
  \seeref{曾}{zeng1}
\end{entry}

\begin{entry}{曾经}{ceng2jing1}{12,8}{⽈、⽷}[HSK 3]
  \definition{adv.}{uma vez; indica que houve algum comportamento ou situação}
\end{entry}

\begin{entry}{叉}{cha1}{3}{⼜}[HSK 5]
  \definition{s.}{garfo; forquilha | símbolo de cruz, ``×''}
  \definition{v.}{trabalhar com um garfo; garfar; pegar coisas com um garfo}
  \seeref{叉}{cha2}
  \seeref{叉}{cha3}
\end{entry}

\begin{entry}{叉子}{cha1zi5}{3,3}{⼜、⼦}[HSK 5]
  \definition[把]{s.}{garfo; ferramenta com mais de duas pontas em uma extremidade | tridente; forquilha; ferramentas de agricultura antigas}
\end{entry}

\begin{entry}{差}{cha1}{9}{⼯}
  \definition{adj.}{diferente; diferente ou inconsistente com um determinado padrão}
  \definition{adv.}{ligeiramente; comparativamente; um pouco}
  \definition{s.}{diferença; resto após a subtração de dois números | erro; engano}
  \seeref{差}{cha4}
  \seeref{差}{chai1}
\end{entry}

\begin{entry}{差别}{cha1bie2}{9,7}{⼯、⼑}[HSK 5]
  \definition{s.}{diferença; disparidade; dissimilaridade; distinção; não semelhança; diferenças na forma ou no conteúdo}
\end{entry}

\begin{entry}{差距}{cha1ju4}{9,11}{⼯、⾜}[HSK 5]
  \definition[个,些,段]{s.}{lacuna; disparidade; discrepância; diferença; grau de diferença entre as coisas, especialmente em termos de distância de algum padrão.}
\end{entry}

\begin{entry}{差(一)点儿}{cha1yi4dian3r5}{9,1,9,2}{⼯、⼀、⽕、⼉}[HSK 5]
  \definition{adv.}{quase; à beira de; praticamente; aproximadamente; significa que algo está perto de ser alcançado, mas não foi alcançado, ou algo foi alcançado, mas mal foi alcançado}
\end{entry}

\begin{entry}{插}{cha1}{12}{⼿}[HSK 5]
  \definition{v.}{perfurar; inserir | interpor; inserir; colocar no meio}
\end{entry}

\begin{entry}{插话}{cha1hua4}{12,8}{⼿、⾔}
  \definition{s.}{interrupção | digressão}
  \definition{v.+compl.}{interromper (a fala de alguém)}
\end{entry}

\begin{entry}{插手}{cha1shou3}{12,4}{⼿、⼿}
  \definition{v.+compl.}{envolver-se em | dar uma mão | ter (tomar) uma mão | cutucar o nariz de alguém | intrometer-se}
\end{entry}

\begin{entry}{叉}{cha2}{3}{⼜}
  \definition{v.}{bloquear; emperrar; congestionar}
  \seeref{叉}{cha1}
  \seeref{叉}{cha3}
\end{entry}

\begin{entry}{查}{cha2}{9}{⽊}[HSK 2]
  \definition{v.}{examinar; verificar cuidadosamente | examinar; investigar; entender bem a situação | procurar; consultar; revisar (documentos bibliográficos)}
  \seeref{查}{zha1}
\end{entry}

\begin{entry}{查询}{cha2 xun2}{9,8}{⽊、⾔}[HSK 5]
  \definition{v.}{indagar; inquirir; perguntar sobre}
\end{entry}

\begin{entry}{茶}{cha2}{9}{⾋}[HSK 1]
  \definition{adj.}{moreno; fulvo; amarelo-acastanhado}
  \definition[杯,壶]{s.}{chá (a bebida); bebida feita com folhas de chá | chá (a planta) | certos tipos de bebidas ou alimentos líquidos | árvore de chá-de-óleo | camélia}
\end{entry}

\begin{entry}{茶叶}{cha2 ye4}{9,5}{⾋、⼝}[HSK 4]
  \definition[盒,罐,包,片]{s.}{chá; folhas de chá; as folhas jovens da planta do chá que são processadas para produzir bebidas}
\end{entry}

\begin{entry}{叉}{cha3}{3}{⼜}
  \definition{v.}{separar de modo a formar uma bifurcação; bifurcar}
  \seeref{叉}{cha1}
  \seeref{叉}{cha2}
\end{entry}

\begin{entry}{刹}{cha4}{8}{⼑}
  \definition*{s.}{abreviação de Kshatara, 刹多罗, sânscrito ``ksetra''}
  \definition{s.}{mosteiro, templo ou santuário budista}
  \seeref{刹}{sha1}
  \seealsoref{刹多罗}{sha1duo1luo2}
\end{entry}

\begin{entry}{差}{cha4}{9}{⼯}[HSK 1]
  \definition{adj.}{não está de acordo com o padrão; pobre; ruim; inferior | errado; incorreto | mesmo significado de 差 \dpy{cha1}}
  \definition{v.}{faltar}
  \seeref{差}{cha1}
  \seeref{差}{chai1}
\end{entry}

\begin{entry}{差不多}{cha4bu5duo1}{9,4,6}{⼯、⼀、⼣}[HSK 2]
  \definition{adj.}{semelhante; aproximadamente igual | não muito longe; quase certo (suficiente); basicamente, próximo dos padrões e requisitos; normal | prestes a (terminar; acabar); descreve que (algo) está quase acabando; (uma tarefa) está quase concluída}
  \definition{adv.}{quase; perto; indica proximidade}
\end{entry}

\begin{entry}{差点儿}{cha4dian3r5}{9,9,2}{⼯、⽕、⼉}
  \definition{adv.}{por pouco | por um triz | quase}
\end{entry}

\begin{entry}{拆}{chai1}{8}{⼿}[HSK 5]
  \definition{v.}{rasgar; desmontar; separar o que está unido | derrubar; desmantelar; demolir; refere-se especificamente à demolição de edifícios}
\end{entry}

\begin{entry}{拆除}{chai1 chu2}{8,9}{⼿、⾩}[HSK 5]
  \definition{v.}{desmantelar; demolir; derrubar; remover (um edifício, etc.)}
\end{entry}

\begin{entry}{差}{chai1}{9}{⼯}
  \definition{s.}{tarefa; trabalho; ser enviado para fazer algo; deveres oficiais; posição | corvéia; mensageiro ou oficial de justiça em um yamen feudal; (velho) refere-se a pessoas que são enviadas para fazer coisas}
  \definition{v.}{enviar uma mensagem; despachar; fnviar (para fazer algo)}
  \seeref{差}{cha1}
  \seeref{差}{cha4}
\end{entry}

\begin{entry}{单}{chan2}{8}{⼗}
  \definition{s.}{usado em 单于 \dpy{chan2yu2}}
  \seeref{单}{dan1}
  \seeref{单}{shan4}
  \seealsoref{单于}{chan2yu2}
\end{entry}

\begin{entry}{单于}{chan2yu2}{8,3}{⼗、⼆}
  \definition{s.}{rei de Xiongnu (匈奴)}
  \seealsoref{匈奴}{xiong1nu2}
\end{entry}

\begin{entry}{禅}{chan2}{12}{⽰}
  \definition{s.}{contemplação prolongada e intensa; meditação profunda | budista; refere-se geralmente a coisas relacionadas ao budismo}
  \seeref{禅}{shan4}
\end{entry}

\begin{entry}{蝉}{chan2}{14}{⾍}
  \definition[只,个]{s.}{cigarra}
  \seealsoref{知了}{zhi1liao3}
\end{entry}

\begin{entry}{产}{chan3}{6}{⼇}
  \definition*{s.}{sobrenome Chan}
  \definition{s.}{produto | propriedade; espólio | (abreviação) indústria}
  \definition{v.}{dar à luz; ser entregue a | produzir; render | separar um ser humano ou animal de sua mãe}
\end{entry}

\begin{entry}{产后}{chan3hou4}{6,6}{⼇、⼝}
  \definition{s.}{pós-parto}
\end{entry}

\begin{entry}{产品}{chan3pin3}{6,9}{⼇、⼝}[HSK 4]
  \definition[个,件,种,批,项,类]{s.}{produto; item produzido}
\end{entry}

\begin{entry}{产生}{chan3sheng1}{6,5}{⼇、⽣}[HSK 3]
  \definition{v.}{produzir; evoluir; emergir; provocar; vir a ser; dar origem a; criar coisas novas e novos fenômenos a partir do que já existe}
\end{entry}

\begin{entry}{产业}{chan3ye4}{6,5}{⼇、⼀}[HSK 5]
  \definition{s.}{patrimônio; propriedade; bens pessoais, como terrenos, casas, fábricas, etc. | indústria; refere-se especificamente à produção industrial moderna | setor; indústria; indústrias e setores da economia nacional}
\end{entry}

\begin{entry}{铲}{chan3}{11}{⾦}
  \definition[个,把]{s.}{pá}
  \definition{v.}{trabalhar com uma pá (ou enxada) | levantar (mover) com uma pá}
\end{entry}

\begin{entry}{铲车}{chan3che1}{11,4}{⾦、⾞}
  \definition[台]{s.}{empilhadeira}
\end{entry}

\begin{entry}{长}{chang2}{4}{⾧}[HSK 2][Kangxi 168]
  \definition*{s.}{sobrenome Chang}
  \definition{adj.}{longo; comprido; a distância entre uma extremidade e outra é grande (em oposição a 短 ) | para sempre; duradouro; a distância entre o ponto inicial e o ponto final de um determinado período é grande | excesso; o que sobra; o que é desnecessário}
  \definition{s.}{comprimento; na direção horizontal, a distância percorrida por um objeto de uma extremidade à outra | ponto forte; especialidade; especialização em determinada área; vantagem}
  \definition{v.}{ser bom em; ser forte em; ser muito bom em algo; ser especialista em algo}
  \seeref{长}{zhang3}
  \seealsoref{短}{duan3}
\end{entry}

\begin{entry}{长城}{chang2cheng2}{4,9}{⾧、⼟}[HSK 3]
  \definition*[段,座,条]{s.}{A Grande Muralha da China; também é usado como metáfora para uma força poderosa e inabalável, um obstáculo intransponível, etc.}
\end{entry}

\begin{entry}{长处}{chang2 chu4}{4,5}{⾧、⼡}[HSK 3]
  \definition[个]{s.}{forte; boas qualidades; pontos fortes; especialização em determinada área}
\end{entry}

\begin{entry}{长度}{chang2 du4}{4,9}{⾧、⼴}[HSK 5]
  \definition{s.}{comprimento; extensão; distância entre dois pontos}
\end{entry}

\begin{entry}{长颈鹿}{chang2jing3lu4}{4,11,11}{⾧、⾴、⿅}
  \definition[只]{s.}{girafa}
\end{entry}

\begin{entry}{长期}{chang2 qi1}{4,12}{⾧、⽉}[HSK 3]
  \definition{adj.}{secular; longo prazo; longo alcance; durante um longo período de tempo}
  \definition{s.}{longo prazo; por muito tempo}
\end{entry}

\begin{entry}{长寿}{chang2 shou4}{4,7}{⾧、⼨}[HSK 5]
  \definition{adj.}{vida longa; longevidade}
\end{entry}

\begin{entry}{长途}{chang2tu2}{4,10}{⾧、⾡}[HSK 4]
  \definition{adj.}{de longa distância; longe}
  \definition[段,次,程]{s.}{longa-distância; referindo-se especificamente a chamadas telefônicas de longa distância ou ônibus de longa distância}
\end{entry}

\begin{entry}{场}{chang2}{6}{⼟}
  \definition{clas.}{usado para descrever o desenrolar dos acontecimentos}
  \definition{s.}{eira; espaço aberto e plano; um terreno plano, geralmente usado para secar grãos e moer cereais | mercado; feira rural}
  \seeref{场}{chang3}
\end{entry}

\begin{entry}{肠}{chang2}{7}{⾁}[HSK 5]
  \definition[根,段,片]{s.}{intestinos; parte do sistema digestivo | salsicha; linguiça; alimentos de tripas recheadas com carne, peixe, etc. | sentimentos; emoções; humor}
\end{entry}

\begin{entry}{尝}{chang2}{9}{⼩}[HSK 5]
  \definition{adv.}{alguma vez; uma vez}
  \definition{v.}{provar; experimentar o sabor de | provar; experimentar; conhecer | tentar; testar}
\end{entry}

\begin{entry}{尝试}{chang2shi4}{9,8}{⼩、⾔}[HSK 5]
  \definition{v.}{tentar; provar; experimentar}
\end{entry}

\begin{entry}{倘}{chang2}{10}{⼈}
  \seeref{倘}{tang3}
\end{entry}

\begin{entry}{常}{chang2}{11}{⼱}[HSK 1]
  \definition*{s.}{sobrenome Chang}
  \definition{adj.}{normal; comum; ordinário; indica frequência, normalidade, universalidade | constante; invariável; imutável; permanente}
  \definition{adv.}{frequentemente; geralmente; com frequência;}
  \definition{s.}{normas; disciplina, ordem social e lei e ordem do Estado}
\end{entry}

\begin{entry}{常常}{chang2 chang2}{11,11}{⼱、⼱}[HSK 1]
  \definition{adv.}{frequentemente; muitas vezes; geralmente; indica que a ação ocorreu várias vezes}
\end{entry}

\begin{entry}{常见}{chang2 jian4}{11,4}{⼱、⾒}[HSK 2]
  \definition{adj.}{comum; frequentemente visto}
\end{entry}

\begin{entry}{常识}{chang2shi2}{11,7}{⼱、⾔}[HSK 4]
  \definition{s.}{senso comum; conhecimento geral; conhecimento que uma pessoa comum deve ter}
\end{entry}

\begin{entry}{常问问题}{chang2wen4wen4ti2}{11,6,6,15}{⼱、⾨、⾨、⾴}
  \definition{s.}{FAQ; perguntas frequentes}
\end{entry}

\begin{entry}{常用}{chang2 yong4}{11,5}{⼱、⽤}[HSK 2]
  \definition{adj.}{em uso comum; frequentemente utilizado}
\end{entry}

\begin{entry}{厂}{chang3}{2}{⼚}[HSK 3]
  \definition[家,间]{s.}{fábrica; moinho; planta; obra | pátio; depósito; refere-se a um estabelecimento comercial com um amplo espaço para armazenamento de mercadorias e processamento}
  \seeref{厂}{an1}
  \seeref{厂}{han3}
\end{entry}

\begin{entry}{厂长}{chang3 zhang3}{2,4}{⼚、⾧}[HSK 5]
  \definition[任,位,个]{s.}{diretor de fábrica; gerente de fábrica; líder responsável pela produção, pela vida e por todos os outros assuntos de toda a fábrica}
\end{entry}

\begin{entry}{场}{chang3}{6}{⼟}[HSK 2]
  \definition*{s.}{sobrenome Chang}
  \definition{clas.}{usado para atividades culturais, recreativas e esportivas | usado para pequenos trechos de uma peça}
  \definition{s.}{um local amplo utilizado para um fim específico | palco; campo | cena | (física) campo (por exemplo: campo manético) | (para atividades recreativas, esportivas ou outras) | um lugar onde as pessoas se reúnem | fazenda; quinta | abertura; encerramento; refere-se ao processo completo de uma apresentação ou competição | local; ponto; o local onde ocorreu o incidente}
  \seeref{场}{chang2}
\end{entry}

\begin{entry}{场合}{chang3he2}{6,6}{⼟、⼝}[HSK 3]
  \definition[个,些,种,类]{s.}{ocasião; situação; um certo tempo, lugar ou situação}
\end{entry}

\begin{entry}{场景}{chang3jing3}{6,12}{⼟、⽇}
  \definition{s.}{cena | cenário | situação | contexto}
\end{entry}

\begin{entry}{场面}{chang3mian4}{6,9}{⼟、⾯}[HSK 5]
  \definition[个,种,番]{s.}{espetáculo; cena (em teatro, ficção, etc.); uma cena em uma produção teatral, cinematográfica ou televisiva que consiste em um cenário, música e personagens | cena; ocasião; literatura narrativa que consiste em situações da vida em que os personagens se relacionam entre si em determinadas ocasiões | orquestra ou instrumentos de acompanhamento (em ópera); refere-se às pessoas e aos instrumentos musicais que acompanham a apresentação de uma ópera, divididos em dois tipos: música de sopro e cordas é uma cena cultural, e gongos e tambores são uma cena marcial | situação; referência geral a uma situação em um determinado contexto | frente; fachada; aparência; espetáculo superficial}
\end{entry}

\begin{entry}{场所}{chang3suo3}{6,8}{⼟、⼾}[HSK 3]
  \definition{s.}{lugar; sítio; arena; local da atividade}
\end{entry}

\begin{entry}{畅}{chang4}{8}{⽥}
  \definition*{s.}{sobrenome Chang}
  \definition{adj.}{suave; desimpedido; sem obstáculos; desobstruído | livre; desinibido}
\end{entry}

\begin{entry}{倡}{chang4}{10}{⼈}
  \definition{v.}{iniciar; propor; defender | promover; assumir a liderança}
\end{entry}

\begin{entry}{倡导}{chang4dao3}{10,6}{⼈、⼨}[HSK 5]
  \definition{v.}{iniciar; propor; promover; defender; advogar}
\end{entry}

\begin{entry}{鬯}{chang4}{10}{⾿}[Kangxi 192]
  \definition{adj.}{suave; desimpedido | livre; desinibido}
  \definition{s.}{um tipo de vinho usado em sacrifícios antigos | (antigo) estojo ou bolsa para arco | o mesmo que 畅}
  \seealsoref{畅}{chang4}
\end{entry}

\begin{entry}{唱}{chang4}{11}{⼝}[HSK 1]
  \definition*{s.}{sobrenome Chang}
  \definition{s.}{uma música ou uma parte cantada de uma ópera chinesa; canções; letras de óperas tradicionais}
  \definition{v.}{cantar; seguir o ritmo da música | chorar; chamar; gritar, falar ou recitar em voz alta}
\end{entry}

\begin{entry}{唱歌}{chang4 ge1}{11,14}{⼝、⽋}[HSK 1]
  \definition{v.+compl.}{cantar (uma música); emitir sons com entonação ritmada e melodiosa; emitir sons (musicais) com a boca; emitir sons de acordo com a melodia}
\end{entry}

\begin{entry}{唱片}{chang4 pian4}{11,4}{⼝、⽚}[HSK 4]
  \definition[枚,张]{s.}{disco; disco feito de goma-laca, plástico, etc. com ranhuras em espiral na superfície para registrar alterações no som que podem reproduzir o som gravado em um fonógrafo}
\end{entry}

\begin{entry}{抄}{chao1}{7}{⼿}[HSK 4]
  \definition*{s.}{sobrenome Chao}
  \definition{v.}{copiar; transcrever | plagiar | registrar as leituras de um medidor | revistar e confiscar; fazer uma incursão em  | pegar um atalho | dobrar (os braços) | agarrar; pegar | ir (andar) embora com}
\end{entry}

\begin{entry}{抄表}{chao1 biao3}{7,8}{⼿、⾐}
  \definition{s.}{leitura do medidor}
\end{entry}

\begin{entry}{抄写}{chao1 xie3}{7,5}{⼿、⼍}[HSK 4]
  \definition{v.}{copiar; transcrever}
\end{entry}

\begin{entry}{超}{chao1}{12}{⾛}
  \definition{adj.}{super; extremamente; maior (ou menor) que o padrão geral}
  \definition{v.}{exceder; ultrapassar; vir para a frente por trás; prevalecer | transcender; ir além; não ser sujeito a certas restrições; ir além de um certo intervalo | exceder; superar; exceder o limite prescrito}
\end{entry}

\begin{entry}{超过}{chao1guo4}{12,6}{⾛、⾡}[HSK 2]
  \definition{v.}{ultrapassar; superar (algo ou alguém); passar de trás para a frente de alguém ou algo | exceder; ser mais do que; ultrapassar (um padrão)}
\end{entry}

\begin{entry}{超级}{chao1ji2}{12,6}{⾛、⽷}[HSK 3]
  \definition{adj.}{super; além do nível geral}
  \definition{pref.}{super-; ultra-; hiper-}
\end{entry}

\begin{entry}{超级市场}{chao1 ji2 shi4 chang3}{12,6,5,6}{⾛、⽷、⼱、⼟}
  \definition[个,间,所,家]{s.}{supermercado; hipermercado}
\end{entry}

\begin{entry}{超声}{chao1sheng1}{12,7}{⾛、⼠}
  \definition{adj.}{ultrasônico}
  \definition{s.}{ultrasom}
\end{entry}

\begin{entry}{超市}{chao1shi4}{12,5}{⾛、⼱}[HSK 2]
  \definition[家]{s.}{supermercado; abreviação de 超级市场}
  \seealsoref{超级市场}{chao1 ji2 shi4 chang3}
\end{entry}

\begin{entry}{超越}{chao1yue4}{12,12}{⾛、⾛}[HSK 5]
  \definition{v.}{ultrapassar; superar; passar por cima; transcender}
\end{entry}

\begin{entry}{巢}{chao2}{11}{⼮}
  \definition*{s.}{sobrenome Chao}
  \definition[个]{s.}{ninho (de aves, insetos, etc.)}
\end{entry}

\begin{entry}{朝}{chao2}{12}{⽉}[HSK 3]
  \definition*{s.}{sobrenome Chao}
  \definition{prep.}{para; em direção a; a direção ou o objeto da ação introduzida, equivalente a 向 ou 对}
  \definition[个]{s.}{corte real; governo; assembleia realizada por um soberano; também se refere à posição no poder, em oposição ao 野 | dinastia, todo o período de governo transmitido de geração em geração por um determinado sobrenome imperial | reinado (de um soberano); o período de reinado de um determinado monarca}
  \definition{v.}{fazer uma peregrinação para; ter uma audiência com (um rei, um imperador, etc.) | estar voltado para; estar em frente a}
  \seeref{朝}{zhao1}
  \seealsoref{对}{dui4}
  \seealsoref{向}{xiang4}
  \seealsoref{野}{ye3}
\end{entry}

\begin{entry}{朝廷}{chao2ting2}{12,6}{⽉、⼵}
  \definition{s.}{corte imperial | dinastia}
\end{entry}

\begin{entry}{朝鲜}{chao2xian3}{12,14}{⽉、⿂}
  \definition*{s.}{Coréia do Norte}
\end{entry}

\begin{entry}{潮}{chao2}{15}{⽔}[HSK 4]
  \definition{adj.}{úmido; molhado | inferior; de qualidade ruim | inferior; não muito habilidoso}
  \definition{s.}{maré; água da maré | surto; corrente; maré; uma metáfora para mudanças sociais em grande escala ou para os altos e baixos de um movimento (social)}
  \definition{s.}{Chaozhou, uma cidade na província de Guangdong}
\end{entry}

\begin{entry}{潮流}{chao2liu2}{15,10}{⽔、⽔}[HSK 4]
  \definition{s.}{maré; corrente de maré; movimento da água devido às marés | tendência; analogia com mudanças sociais ou tendências de desenvolvimento}
\end{entry}

\begin{entry}{潮湿}{chao2shi1}{15,12}{⽔、⽔}[HSK 4]
  \definition{adj.}{molhado; úmido; umedecido; que contém mais água do que o normal}
\end{entry}

\begin{entry}{潮绣}{chao2xiu4}{15,10}{⽔、⽷}
  \definition*{s.}{Bordado Chaozhou}
\end{entry}

\begin{entry}{鼂}{chao2}{18}{⿌}
  \definition*{s.}{sobrenome Chao}
  \definition{s.}{tartaruga marinha}
\end{entry}

\begin{entry}{吵}{chao3}{7}{⼝}[HSK 3]
  \definition{adj.}{barulhento; ruidoso e perturbador}
  \definition{v.}{brigar; discutir; disputar}
\end{entry}

\begin{entry}{吵架}{chao3jia4}{7,9}{⼝、⽊}[HSK 3]
  \definition{v.+compl.}{brigar; discutir; ter uma discussão acalorada}
\end{entry}

\begin{entry}{炒}{chao3}{8}{⽕}
  \definition{v.}{saltear; refogar; aquecer os alimentos em uma panela e mexer repetidamente para cozinhá-los ou secá-los | especular (na bolsa de valores, etc.) | exagerar; dar publicidade exagerada; a fim de ampliar a influência, por meio de publicidade repetida e exagerada na mídia | demitir; despedir}
\end{entry}

\begin{entry}{车}{che1}{4}{⾞}[HSK 1][Kangxi 159]
  \definition*{s.}{sobrenome Che}
  \definition[辆]{s.}{veículo; meios de transporte terrestres sobre rodas | máquina ou instrumento com rodas; ferramentas com eixo giratório | máquina}
  \definition{v.}{tornear; usinar com torno mecânico | elevar água por meio de uma roda d'água; usar caminhão-pipa para coletar água | girar, geralmente se refere ao corpo}
  \seeref{车}{ju1}
\end{entry}

\begin{entry}{车次}{che1ci4}{4,6}{⾞、⽋}
  \definition{s.}{número do trem}
\end{entry}

\begin{entry}{车库}{che1ku4}{4,7}{⾞、⼴}
  \definition{s.}{garagem}
\end{entry}

\begin{entry}{车辆}{che1 liang4}{4,11}{⾞、⾞}[HSK 2]
  \definition{s.}{veículo; carro; termo genérico para todos os tipos de veículos}
\end{entry}

\begin{entry}{车牌}{che1pai2}{4,12}{⾞、⽚}
  \definition{s.}{matrícula | placa de carro}
\end{entry}

\begin{entry}{车票}{che1 piao4}{4,11}{⾞、⽰}[HSK 1]
  \definition{s.}{passagem de trem ou ônibus; bilhete; bilhete de transporte público}
\end{entry}

\begin{entry}{车上}{che1 shang4}{4,3}{⾞、⼀}[HSK 1]
  \definition{adv.}{no carro; no interior do veículo}
\end{entry}

\begin{entry}{车水马龙}{che1shui3-ma3long2}{4,4,3,5}{⾞、⽔、⾺、⿓}
  \definition{expr.}{tráfego engarrafado | engarrafamento | (literalmente) ``fluxo interminável de cavalos e carruagens''}
\end{entry}

\begin{entry}{车站}{che1 zhan4}{4,10}{⾞、⽴}[HSK 1]
  \definition[个,处]{s.}{estação; estação ferroviária; parada; pontos de parada estabelecidos nas linhas de transporte rodoviário são locais para embarque e desembarque de passageiros ou carga e descarga de mercadorias}
\end{entry}

\begin{entry}{车主}{che1 zhu3}{4,5}{⾞、⼂}[HSK 5]
  \definition{s.}{proprietário do carro; uma pessoa física ou família que possui um veículo motorizado}
\end{entry}

\begin{entry}{车子}{che1zi5}{4,3}{⾞、⼦}
  \definition{s.}{qualquer veículo (carro, bicicleta, caminhão, etc)}
\end{entry}

\begin{entry}{尺}{che3}{4}{⼫}
  \definition{s.}{(tom) uma nota da escala em Gongchepu (工尺谱), correspondente a 2 na notação musical numerada}
  \seealsoref{工尺谱}{gong1 che3 pu3}
\end{entry}

\begin{entry}{彻}{che4}{7}{⼻}
  \definition{adj.}{minucioso; completo; penetrante}
  \definition{adv.}{minuciosamente; profundamente}
\end{entry}

\begin{entry}{彻底}{che4di3}{7,8}{⼻、⼴}[HSK 4]
  \definition{adj.}{minucioso; completo; exaustivo; profundo e completo; nada é deixado de fora}
\end{entry}

\begin{entry}{撤}{che4}{15}{⼿}
  \definition{v.}{remover, tirar | demitir; liberar | retirar-se; evacuar}
\end{entry}

\begin{entry}{沉}{chen2}{7}{⽔}[HSK 4]
  \definition{adj.}{profundo | pesado | pesado (sentir-se pesado)}
  \definition{v.}{afundar; submergir; imergir | manter baixo; abaixar | descansar; parar}
  \seeref{沉}{chen2}
\end{entry}

\begin{entry}{沉默}{chen2mo4}{7,16}{⽔、⿊}[HSK 4]
  \definition{adj.}{silencioso; reticente; taciturno; não comunicativo}
  \definition{v.}{silenciar; não falar por causa de alguma coisa}
\end{entry}

\begin{entry}{沉重}{chen2zhong4}{7,9}{⽔、⾥}[HSK 4]
  \definition{adj.}{(pressão, fardo, etc.) muito pesado; profundo | sério; pesado; humor pouco animador; fardo pesado de pensamentos}
\end{entry}

\begin{entry}{衬}{chen4}{8}{⾐}
  \definition[件,个]{s.}{forro}
  \definition{v.}{forrar; colocar algo embaixo | fornecer um pano de fundo para; destacar; servir como contraste para}
\end{entry}

\begin{entry}{衬衫}{chen4shan1}{8,8}{⾐、⾐}[HSK 3]
  \definition[件,个]{s.}{camisa; blusa; camisa ocidental usada por baixo}
\end{entry}

\begin{entry}{衬衣}{chen4 yi1}{8,6}{⾐、⾐}[HSK 3]
  \definition[件,个]{s.}{camisa; também se refere a uma peça de roupa usada por baixo do casaco}
\end{entry}

\begin{entry}{称}{chen4}{10}{⽲}
  \definition{adj.}{ajustado; encaixado; adequado}
  \definition{v.}{ajustar; adequar; combinar; estar em conformidade com; ser adequado para | ter; possuir}
  \seeref{称}{cheng1}
\end{entry}

\begin{entry}{趁}{chen4}{12}{⾛}
  \definition{prep.}{aproveitar-se de; tirar vantagem de (tempo, oportunidade, etc.); indica o tempo e as condições de uso}
  \definition{v.}{ser rico em; possuir}
\end{entry}

\begin{entry}{称}{cheng1}{10}{⽲}[HSK 2,5]
  \definition*{s.}{sobrenome Cheng}
  \definition{s.}{nome}
  \definition{v.}{chamar; ser chamado | dizer; declarar | elogiar; louvar; expressar afirmação ou elogio a pessoas ou coisas por meio de palavras | pesar; medir o peso | elevar; levantar; erguer | aplaudir; concordar; expressar suas opiniões ou sentimentos por meio de palavras ou ações | declarar-se como; declarar que é; reivindicar ser alguém em virtude do próprio poder}
  \seeref{称}{chen4}
\end{entry}

\begin{entry}{称号}{cheng1hao4}{10,5}{⽲、⼝}[HSK 5]
  \definition{s.}{título; nome; designação; nome dado a alguém, a uma organização ou a alguma coisa (geralmente usado de forma honrosa)}
\end{entry}

\begin{entry}{称为}{cheng1 wei2}{10,4}{⽲、⼂}[HSK 3]
  \definition{v.}{ser chamado de; ser conhecido como; denominar}
\end{entry}

\begin{entry}{称赞}{cheng1zan4}{10,16}{⽲、⾙}[HSK 4]
  \definition[句,声,番,次]{s.}{elogio; aclamação; louvor; avaliação positiva de um desempenho ou conquista}
  \definition{v.}{elogiar; aclamar; louvar; usar palavras para expressar um carinho pelas virtudes de uma pessoa ou coisa}
\end{entry}

\begin{entry}{成}{cheng2}{6}{⼽}[HSK 2]
  \definition*{s.}{sobrenome Cheng}
  \definition{adj.}{capaz; competente | totalmente crescido; totalmente desenvolvido; maduro | estabelecido; Já definido; pronto para uso | em números ou quantidades consideráveis; inteiro; suficiente: enfatiza a quantidade ou a duração}
  \definition{clas.}{um décimo}
  \definition{interj.}{O.K.; tudo bem}
  \definition{s.}{resultado; conquista}
  \definition{v.}{ter sucesso; conseguir; ser bem-sucedido | tornar-se; transformar-se | ajudar a completar; realizar}
\end{entry}

\begin{entry}{成本}{cheng2ben3}{6,5}{⼽、⽊}[HSK 5]
  \definition{s.}{custo principal; custo; custo capitalizado; custo final; primeiro custo; custo próprio; custo de produção de um produto}
\end{entry}

\begin{entry}{成都}{cheng2du1}{6,10}{⼽、⾢}
  \definition*{s.}{Chengdu}
\end{entry}

\begin{entry}{成功}{cheng2gong1}{6,5}{⼽、⼒}[HSK 3]
  \definition{adj.}{bem-sucedido; frutífero}
  \definition[个,次]{s.}{sucesso}
  \definition{v.}{ter sucesso; obter os resultados esperados}
\end{entry}

\begin{entry}{成果}{cheng2guo3}{6,8}{⼽、⽊}[HSK 3]
  \definition[个]{s.}{realização; resultado; conquista; recompensas no trabalho ou na carreira}
\end{entry}

\begin{entry}{成婚}{cheng2hun1}{6,11}{⼽、⼥}
  \definition{v.}{casar-se}
\end{entry}

\begin{entry}{成活}{cheng2huo2}{6,9}{⼽、⽔}
  \definition{v.}{sobreviver}
\end{entry}

\begin{entry}{成吉思汗}{cheng2ji2si1han2}{6,6,9,6}{⼽、⼝、⼼、⽔}
  \definition*{s.}{Genghis Khan (1162-1227), fundador e governante do Império Mongol}
\end{entry}

\begin{entry}{成绩}{cheng2ji4}{6,11}{⼽、⽷}[HSK 2]
  \definition[项,个]{s.}{realização; sucesso; resultado (de trabalho ou estudo); refere-se à pontuação obtida em exames e competições; classificação, também se refere aos resultados alcançados no trabalho}
\end{entry}

\begin{entry}{成家}{cheng2jia1}{6,10}{⼽、⼧}
  \definition{v.}{tornar-se um especialista reconhecido | estabelecer-se e casar-se (de um homem)}
\end{entry}

\begin{entry}{成交}{cheng2jiao1}{6,6}{⼽、⼇}[HSK 5]
  \definition{v.+compl.}{fechar um acordo; fazer uma barganha; concluir uma transação}
\end{entry}

\begin{entry}{成就}{cheng2jiu4}{6,12}{⼽、⼪}[HSK 3]
  \definition[个,项]{s.}{realização; conquista; sucesso; realizações profissionais}
  \definition{v.}{realizar; alcançar; completar; concluir (carreira)}
\end{entry}

\begin{entry}{成立}{cheng2li4}{6,5}{⼽、⽴}[HSK 3]
  \definition{v.}{fundar; estabelecer; criar; (organizações, instituições, etc.) começar a existir e a funcionar | ser válido; ser sustentável; fazer sentido; (teorias, pontos de vista, razões, etc.) fundamentados e válidos}
\end{entry}

\begin{entry}{成批}{cheng2pi1}{6,7}{⼽、⼿}
  \definition{s.}{em lotes | a granel}
\end{entry}

\begin{entry}{成器}{cheng2qi4}{6,16}{⼽、⼝}
  \definition{v.}{tornar-se uma pessoa digna de respeito | fazer algo de si mesmo}
\end{entry}

\begin{entry}{成人}{cheng2ren2}{6,2}{⼽、⼈}[HSK 4]
  \definition[个]{s.}{adulto; crescido; pessoa adulta}
  \definition{v.}{crescer; tornar-se adulto}
\end{entry}

\begin{entry}{成色}{cheng2se4}{6,6}{⼽、⾊}
  \definition{v.}{sair-se bem | ser bem sucedido}
\end{entry}

\begin{entry}{成熟}{cheng2shu2}{6,15}{⼽、⽕}[HSK 3]
  \definition{adj.}{maduro; amadurecido; totalmente desenvolvido; descreve que as oportunidades, condições, etc. estão perfeitas e que não haverá nenhum problema}
  \definition{v.}{amadurecer; atingir a maturidade; estar totalmente desenvolvido; frutas e outros frutos totalmente maduros, referindo-se ao desenvolvimento completo de organismos vivos}
\end{entry}

\begin{entry}{成为}{cheng2wei2}{6,4}{⼽、⼂}[HSK 2]
  \definition{v.}{tornar-se; transformar-se; revelar-se; passar de uma situação, identidade ou estado para outro}
\end{entry}

\begin{entry}{成效}{cheng2xiao4}{6,10}{⼽、⽁}[HSK 5]
  \definition{s.}{efeito; resultado}
\end{entry}

\begin{entry}{成语}{cheng2yu3}{6,9}{⼽、⾔}[HSK 5]
  \definition{s.}{expressão idiomática; frase de conjunto (frases de quatro caracteres em chinês, geralmente com alusões literárias)}
\end{entry}

\begin{entry}{成员}{cheng2yuan2}{6,7}{⼽、⼝}[HSK 3]
  \definition[个,些,名,位]{s.}{membro; membros de um grupo ou família}
\end{entry}

\begin{entry}{成长}{cheng2zhang3}{6,4}{⼽、⾧}[HSK 3]
  \definition{v.}{crescer; amadurecer; tornar-se adulto; o desenvolvimento de seres humanos, animais ou plantas desde a infância até a maturidade}
\end{entry}

\begin{entry}{承}{cheng2}{8}{⼿}
  \definition*{s.}{sobrenome Cheng}
  \definition{v.}{suportar; segurar; carregar; sustentar | empreender; contratar (para fazer um trabalho) | estar em dívida (com alguém por uma gentileza); receber um favor | continuar; prosseguir | receber de cima (instruções, mandato)}
\end{entry}

\begin{entry}{承办}{cheng2ban4}{8,4}{⼿、⼒}[HSK 5]
  \definition{v.}{empreender}
\end{entry}

\begin{entry}{承担}{cheng2dan1}{8,8}{⼿、⼿}[HSK 4]
  \definition{v.}{suportar; empreender; assumir; tomar conta de algo}
\end{entry}

\begin{entry}{承认}{cheng2ren4}{8,4}{⼿、⾔}[HSK 4]
  \definition{s.}{reconhecimento (diplomático, artístico, etc.)}
  \definition{v.}{admitir; reconhecer | dar reconhecimento diplomático; reconhecer}
\end{entry}

\begin{entry}{承受}{cheng2shou4}{8,8}{⼿、⼜}[HSK 4]
  \definition{v.}{suportar; resistir; realizar (tarefas, dificuldades, pressões, etc.); submeter-se a (testes, etc.) | herdar}
\end{entry}

\begin{entry}{诚}{cheng2}{8}{⾔}
  \definition{adj.}{sincero; honesto; verdadeiro}
  \definition{adv.}{na verdade; realmente; de fato}
  \definition{s.}{sinceridade; genuinidade; seriedade}
\end{entry}

\begin{entry}{诚实}{cheng2shi2}{8,8}{⾔、⼧}[HSK 4]
  \definition{adj.}{honesto; sincero e honesto, não hipócrita}
\end{entry}

\begin{entry}{诚实地}{cheng2shi2 di4}{8,8,6}{⾔、⼧、⼟}
  \definition{adv.}{honestamente}
\end{entry}

\begin{entry}{诚信}{cheng2 xin4}{8,9}{⾔、⼈}[HSK 4]
  \definition{adj.}{honesto e confiável}
  \definition[种]{s.}{fé; honestidade; padrão e princípio de comportamento: não contar mentiras, prometer aos outros o que eles podem fazer e ter a confiança dos outros}
\end{entry}

\begin{entry}{城}{cheng2}{9}{⼟}[HSK 3]
  \definition*{s.}{sobrenome Cheng}
  \definition[座,道,个]{s.}{muralha da cidade; muralha | cidade | centro de um determinado tipo (por exemplo, negócios, entretenimento, etc.)}
\end{entry}

\begin{entry}{城堡}{cheng2bao3}{9,12}{⼟、⼟}
  \definition[座,个]{s.}{forte; castelo; cidadela; uma pequena cidade com muralhas que facilitam a defesa}
\end{entry}

\begin{entry}{城度}{cheng2du4}{9,9}{⼟、⼴}[HSK 3]
  \definition*{s.}{Cidade}
\end{entry}

\begin{entry}{城里}{cheng2 li3}{9,7}{⼟、⾥}[HSK 5]
  \definition{s.}{na cidade; dentro da cidade; originalmente referia-se à área dentro das muralhas da cidade, agora refere-se principalmente à área urbana}
\end{entry}

\begin{entry}{城市}{cheng2shi4}{9,5}{⼟、⼱}[HSK 3]
  \definition[个,座]{s.}{cidade; regiões com alta densidade populacional, comércio e indústria desenvolvidos e cuja população é predominantemente não agrícola são geralmente centros políticos, econômicos e culturais das regiões vizinhas}
\end{entry}

\begin{entry}{乘}{cheng2}{10}{⽲}[HSK 5]
  \definition*{s.}{sobrenome Cheng}
  \definition{s.}{uma divisão principal das escolas budistas; uma seita ou doutrina do budismo}
  \definition{v.}{cavalgar; andar a cavalo; utilizar um veículo ou animal em vez de caminhar | aproveitar-se de; valer-se de; tirar vantagem de; tirar proveito de | multiplicar; realizar multiplicação | perseguir; caçar}
  \seeref{乘}{sheng4}
\end{entry}

\begin{entry}{乘车}{cheng2 che1}{10,4}{⽲、⾞}[HSK 5]
  \definition{v.}{montar; dirigir; conduzir; andar a cavalo, de moto, de bicicleta, etc.}
\end{entry}

\begin{entry}{乘积}{cheng2ji1}{10,10}{⽲、⽲}
  \definition{s.}{(matemática) produto (resultado da multiplicação)}
\end{entry}

\begin{entry}{乘客}{cheng2 ke4}{10,9}{⽲、⼧}[HSK 5]
  \definition[个,位,名]{s.}{passageiro}
\end{entry}

\begin{entry}{乘客数}{cheng2ke4 shu4}{10,9,13}{⽲、⼧、⽁}
  \definition{s.}{número de passageiros}
\end{entry}

\begin{entry}{乘坐}{cheng2zuo4}{10,7}{⽲、⼟}[HSK 5]
  \definition{v.}{pegar (um trem, ônibus, etc.); andar de (bicicleta, moto, etc.)}
\end{entry}

\begin{entry}{盛}{cheng2}{11}{⽫}
  \definition{v.}{encher; encher com uma concha; colocar as coisas em recipientes; especialmente colocar alimentos em tigelas, pratos e outros recipientes | segurar; conter; acomodar}
  \seeref{盛}{sheng4}
\end{entry}

\begin{entry}{惩}{cheng2}{12}{⼼}
  \definition{v.}{receber ou dar aviso | punir; penalizar}
\end{entry}

\begin{entry}{惩处}{cheng2chu3}{12,5}{⼼、⼡}
  \definition{v.}{administrar justiça | punir}
\end{entry}

\begin{entry}{惩罚}{cheng2fa2}{12,9}{⼼、⽹}
  \definition{v.}{punir | penalizar}
\end{entry}

\begin{entry}{程}{cheng2}{12}{⽲}
  \definition{s.}{regra; regulamento; lei | ordem; procedimento | jornada; etapa de uma jornada; estrada; um trecho de estrada | distância percorrida ou movida por um objeto | programação | medição; termo geral para pesos e medidas}
\end{entry}

\begin{entry}{程度}{cheng2du4}{12,9}{⽲、⼴}[HSK 3]
  \definition[种]{s.}{nível; grau (de cultura, educação, aprendizagem, etc.) | extensão; grau; a situação, o nível ou o estágio em que as coisas mudam}
\end{entry}

\begin{entry}{程控}{cheng2kong4}{12,11}{⽲、⼿}
  \definition{s.}{programado | sob controle automático}
\end{entry}

\begin{entry}{程序}{cheng2xu4}{12,7}{⽲、⼴}[HSK 4]
  \definition[个,套,种]{s.}{ordem; curso; sequência; procedimento; ordem em que algo é feito; também, um determinado número de etapas em um trabalho | programa; conjunto de instruções de computador projetado em sequência para permitir que um computador execute uma ou mais operações}
\end{entry}

\begin{entry}{程序库}{cheng2xu4ku4}{12,7,7}{⽲、⼴、⼴}
  \definition{s.}{biblioteca de funções e procedimentos para programas de computador}
\end{entry}

\begin{entry}{程序设计}{cheng2xu4she4ji4}{12,7,6,4}{⽲、⼴、⾔、⾔}
  \definition{s.}{programação de computadores}
\end{entry}

\begin{entry}{橙}{cheng2}{16}{⽊}
  \definition{s.}{laranja; fruta da laranjeira | laranjeira; pé de laranja | cor laranja}
\end{entry}

\begin{entry}{橙色}{cheng2 se4}{16,6}{⽊、⾊}
  \definition{s.}{cor de laranja}
\end{entry}

\begin{entry}{橙汁}{cheng2zhi1}{16,5}{⽊、⽔}
  \definition[瓶,杯,罐,盒]{s.}{suco de laranja}
  \seealsoref{橘子汁}{ju2zi5zhi1}
  \seealsoref{柳橙汁}{liu3cheng2zhi1}
\end{entry}

\begin{entry}{吃}{chi1}{6}{⼝}[HSK 1]
  \definition{s.}{alimentos; necessidades básicas}
  \definition{v.}{comer; pegar; fazer; colocar alimentos na boca, mastigar e engolir (incluindo sugar e beber) | viver; depender de algo para viver | aniquilar; eliminar (usado principalmente em jogos de guerra e jogos de tabuleiro) | esgotar; exaurir; ser um fardo; ser um esforço | absorver | sofrer; incorrer | entender; compreender | entrar um objeto em outro | expressar aceitação psicológica | fazer suas refeições; comer}
\end{entry}

\begin{entry}{吃饭}{chi1 fan4}{6,7}{⼝、⾷}[HSK 1]
  \definition{v.+compl.}{comer; ter (comer) uma refeição | manter-se vivo;  ganhar a vida; refere-se à vida ou à sobrevivência em geral}
\end{entry}

\begin{entry}{吃惊}{chi1jing1}{6,11}{⼝、⼼}[HSK 4]
  \definition{v.+compl.}{ficar assustado; ficar chocado; ficar espantado; pegar de surpresa; ficar assustado inesperadamente}
\end{entry}

\begin{entry}{吃力}{chi1li4}{6,2}{⼝、⼒}[HSK 5]
  \definition{adj.}{suado; extenuante; trabalhoso; laborioso | cansado; fatigado}
\end{entry}

\begin{entry}{吃屎}{chi1 shi3}{6,9}{⼝、⼫}
  \definition{expr.}{Coma merda!}
\end{entry}

\begin{entry}{池}{chi2}{6}{⽔}
  \definition*{s.}{sobrenome Chi}
  \definition[个,片]{s.}{piscina; lagoa | qualquer espaço fechado com laterais elevadas | baias (em um teatro); a parte frontal do salão principal do teatro | fosso}
\end{entry}

\begin{entry}{池子}{chi2 zi5}{6,3}{⽔、⼦}[HSK 5]
  \definition{s.}{lago; lagoa; viveiro | piscina; piscina do balneário | (antigo) arquibancada (primeiras fileiras em um teatro) | pista de dança de um salão de baile}
\end{entry}

\begin{entry}{迟}{chi2}{7}{⾡}[HSK 5]
  \definition*{s.}{sobrenome Chi}
  \definition{adj.}{lento; tardio; demorado | atrasado | lento; obtuso}
\end{entry}

\begin{entry}{迟到}{chi2dao4}{7,8}{⾡、⼑}[HSK 4]
  \definition{v.}{chegar atrasado; atrasar-se}
\end{entry}

\begin{entry}{持}{chi2}{9}{⼿}
  \definition{v.}{segurar; agarrar | opor; confrontar | apoiar; manter | gerenciar; supervisionar | sequestrar; agarrar (controlar; forçar)}
\end{entry}

\begin{entry}{持续}{chi2xu4}{9,11}{⼿、⽷}[HSK 3]
  \definition{v.}{durar; continuar; sustentar; manter a situação ou as condições como estão, sem alterações}
\end{entry}

\begin{entry}{尺}{chi3}{4}{⼫}[HSK 4]
  \definition{clas.}{chi, uma unidade de comprimento (=13 metros)}
  \definition[支,把]{s.}{régua; instrumentos de medição | um instrumento no formato de uma régua}
  \seeref{尺}{che3}
\end{entry}

\begin{entry}{尺寸}{chi3 cun4}{4,3}{⼫、⼨}[HSK 4]
  \definition{s.}{tamanho; medida; dimensão}
\end{entry}

\begin{entry}{尺子}{chi3zi5}{4,3}{⼫、⼦}[HSK 4]
  \definition[把]{s.}{régua de madeira ou metal para orientar a caneta ou o lápis para desenhar linhas ou fazer medições}
\end{entry}

\begin{entry}{齿}{chi3}{8}{⿒}[Kangxi 211]
  \definition[颗]{s.}{dente | uma parte de qualquer coisa semelhante a um dente; parte dentada de um objeto | idade (de uma pessoa); faixa etária}
  \definition{v.}{mencionar; falar de}
\end{entry}

\begin{entry}{齿儿}{chi3r5}{8,2}{⿒、⼉}
  \definition{s.}{dentes}
\end{entry}

\begin{entry}{斥}{chi4}{5}{⽄}
  \definition*{s.}{sobrenome Chi}
  \definition{adj.}{(do solo) salino; alcalino}
  \definition{s.}{terra impregnada de sal, portanto estéril}
  \definition{v.}{repreender; censurar; denunciar; reprimir | repelir; excluir; expulsar | fornecer; prover | (literário) abrir; expandir | culpar; reprovar | estender; ampliar | (datado) reconhecer; detectar}
\end{entry}

\begin{entry}{斥骂}{chi4ma4}{5,9}{⽄、⾺}
  \definition{v.}{repreender}
\end{entry}

\begin{entry}{赤}{chi4}{7}{⾚}[Kangxi 155]
  \definition*{s.}{sobrenome Chi}
  \definition{adj.}{vermelho | um tipo de vermelho um pouco mais claro que o vermelhão | (história)  revolucionário; comunista | leal; sincero | nú; exposto}
  \definition{s.}{ouro puro}
\end{entry}

\begin{entry}{充}{chong1}{6}{⼉}
  \definition{adj.}{suficiente; completo; amplo; cheio}
  \definition{s.}{sobrenome Chong}
  \definition{v.}{encher; carregar; atulhar | servir como; agir como | fingir ser; posar como; passar algo como}
\end{entry}

\begin{entry}{充电}{chong1 dian4}{6,5}{⼉、⽥}[HSK 4]
  \definition{v.}{carregar (uma bateria); conectar uma fonte de alimentação CC aos terminais da bateria para recarregar a bateria | relaxar; passar o tempo livre; ``recarregar as baterias''; estudar para adquirir mais conhecimento; reabastecer (ou ampliar) o conhecimento; metaforicamente falando, para reabastecer a força física e a energia por meio do descanso e da recreação; também metaforicamente falando, para reabastecer novos conhecimentos e desenvolver novas habilidades por meio do reaprendizado}
\end{entry}

\begin{entry}{充电器}{chong1dian4qi4}{6,5,16}{⼉、⽥、⼝}[HSK 4]
  \definition{s.}{carregador de bateria; dispositivo para alimentar uma bateria com energia, forçando uma corrente através dela}
\end{entry}

\begin{entry}{充分}{chong1fen4}{6,4}{⼉、⼑}[HSK 4]
  \definition{adj.}{cheio; amplo; abundante; suficiente; adequado}
  \definition{adv.}{totalmente; até o fim}
\end{entry}

\begin{entry}{充满}{chong1man3}{6,13}{⼉、⽔}[HSK 3]
  \definition{v.}{preencher; encher; cobrir completamente| estar cheio de; estar repleto de; estar transbordando de; estar impregnado de}
\end{entry}

\begin{entry}{充足}{chong1zu2}{6,7}{⼉、⾜}[HSK 5]
  \definition{adj.}{bastante; adequado; suficiente}
\end{entry}

\begin{entry}{冲}{chong1}{6}{⼎}[HSK 4]
  \definition{s.}{via pública; local importante; via de passagem; via local importante | um trecho de planície em uma área montanhosa | (astronomia) oposição; os planetas externos orbitam até ficarem alinhados com a Terra e o Sol, e a Terra está no meio}
  \definition{v.}{atacar; apressar; correr; passar rapidamente; passar por um obstáculo | colidir; chocar; bater | despejar água fervente sobre | enxaguar; dar descarga; lavar | revelar (filme) | neutralizar a má sorte}
  \seeref{冲}{chong4}
\end{entry}

\begin{entry}{冲动}{chong1dong4}{6,6}{⼎、⼒}[HSK 5]
  \definition{adj.}{impulsivo; impetuoso}
  \definition{s.}{impulso; impetuosidade; impulso de movimento; fenômeno psicológico no qual as emoções são particularmente fortes e o controle racional é fraco}
  \definition{v.}{ficar animado; ser impetuoso; agir por impulso}
\end{entry}

\begin{entry}{冲锋}{chong1feng1}{6,12}{⼎、⾦}
  \definition{v.}{cobrar | tomar de assalto}
\end{entry}

\begin{entry}{冲浪}{chong1lang4}{6,10}{⼎、⽔}
  \definition{s.}{surfe}
  \definition{v.}{surfar}
\end{entry}

\begin{entry}{冲突}{chong1tu1}{6,9}{⼎、⽳}[HSK 5]
  \definition{v.}{chocar-se; entrar em conflito; conflitar | contradizer; duas coisas opostas que interferem uma na outra}
\end{entry}

\begin{entry}{憧}{chong1}{15}{⼼}
  \definition{adj.}{irresoluto; indeciso | estúpido; imbecil; confuso}
\end{entry}

\begin{entry}{憧憬}{chong1jing3}{15,15}{⼼、⼼}
  \definition{v.}{ansiar por | esperar por}
\end{entry}

\begin{entry}{虫}{chong2}{6}{⾍}[Kangxi 142]
  \definition[只,条]{s.}{inseto; verme | (pejorativo) pessoas que se comportam de forma desprezível | fã; viciado | forma inferior de vida animal, incluindo insetos, larvas de insetos, vermes e criaturas semelhantes | pessoa com uma característica indesejável específica}
\end{entry}

\begin{entry}{虫子}{chong2 zi5}{6,3}{⾍、⼦}[HSK 4]
  \definition[条,只,种]{s.}{percevejo; besouro; inseto; verme; criaturas semelhantes a insetos}
\end{entry}

\begin{entry}{重}{chong2}{9}{⾥}
  \definition*{s.}{sobrenome Chong}
  \definition{adv.}{novamente; mais uma vez}
  \definition{clas.}{usado para camadas}
  \definition{v.}{repetir; duplicar}
  \seeref{重}{zhong4}
\end{entry}

\begin{entry}{重重}{chong2chong2}{9,9}{⾥、⾥}
  \definition{adv.}{camada após camada | um após o outro}
  \seeref{重重}{zhong4zhong4}
\end{entry}

\begin{entry}{重点}{chong2dian3}{9,9}{⾥、⽕}
  \definition[个]{adj./adv./s.}{nota principal; ponto-chave; ponto focal; ênfase}
  \seeref{重点}{zhong4dian3}
\end{entry}

\begin{entry}{重逢}{chong2feng2}{9,10}{⾥、⾡}
  \definition{s.}{reunião}
  \definition{v.}{encontrar-se novamente | reunir-se}
\end{entry}

\begin{entry}{重复}{chong2fu4}{9,9}{⾥、⼢}[HSK 2]
  \definition{v.}{repetir; iterar; duplicar; reduplicar | fazer algo novamente; repetir as mesmas palavras, fazer as mesmas coisas}
\end{entry}

\begin{entry}{重新}{chong2xin1}{9,13}{⾥、⽄}[HSK 2]
  \definition{adv.}{novamente; de novo; significa repetir uma ação ou comportamento já realizado | indica que se deve começar do início (mudança de método ou conteúdo)}
\end{entry}

\begin{entry}{重阳节}{chong2yang2jie2}{9,6,5}{⾥、⾩、⾋}
  \definition*{s.}{Festa do Duplo Nove, Festival Yang, dia de subir aos lugares mais altos para evitar calamidades e dia do culto aos antepassados (9º dia do nono mês lunar)}
\end{entry}

\begin{entry}{崇}{chong2}{11}{⼭}
  \definition*{s.}{sobrenome Chong}
  \definition{adj.}{alto; elevado; sublime}
  \definition{v.}{adorar; reverenciar; venerar; estimar | respeitar}
\end{entry}

\begin{entry}{宠}{chong3}{8}{⼧}
  \definition*{s.}{sobrenome Chong}
  \definition{v.}{mimar; estragar; conceder favor a | regalar; encontrar favor com alguém; estar nas boas graças de alguém}
\end{entry}

\begin{entry}{宠物}{chong3wu4}{8,8}{⼧、⽜}
  \definition{s.}{animal de estimação}
\end{entry}

\begin{entry}{冲}{chong4}{6}{⼎}
  \definition{adj.}{poderoso; com vigor; com muita força; vigoroso | forte; odor forte e pungente (olfato)}
  \definition{prep.}{de frente; em direção a | na força de; com base em; em virtude de}
  \definition{v.}{estampar (máquina de estamparia)}
  \seeref{冲}{chong1}
\end{entry}

\begin{entry}{抽}{chou1}{8}{⼿}[HSK 4]
  \definition{v.}{retirar; tirar (do meio); retirar, puxar ou arrancar algo que está preso ou emaranhado em outra coisa | tirar, retirar (uma parte de um todo) | (certas plantas) começar a crescer, produzir | bombear | encolher; contrair | chicotear; açoitar; surrar | dirigir; conduzir | encontrar tempo; libertar-se; sair de alguma coisa}
\end{entry}

\begin{entry}{抽奖}{chou1 jiang3}{8,9}{⼿、⼤}[HSK 4]
  \definition{s.}{loteria; sorteio de loteria}
\end{entry}

\begin{entry}{抽烟}{chou1yan1}{8,10}{⼿、⽕}[HSK 4]
  \definition{v.+compl.}{fumar (um cigarro ou um cachimbo)}
\end{entry}

\begin{entry}{愁}{chou2}{13}{⼼}[HSK 5]
  \definition{adj.}{triste; pesaroso; angustiado; desconsolado; preocupado; deprimido}
  \definition{s.}{pesar; sofrimento; dor; tristeza}
  \definition{v.}{preocupar-se; estar preocupado; ficar ansioso; sentir ansiedade}
\end{entry}

\begin{entry}{酬}{chou2}{13}{⾣}
  \definition[份]{s.}{recompensa; pagamento}
  \definition{v.}{trocar amigavelmente | cumprir; perceber | (literário) propor um brinde; brindar | pagar; reembolsar | completar; concluir}
\end{entry}

\begin{entry}{酬劳}{chou2lao2}{13,7}{⾣、⼒}
  \definition{s.}{recompensa}
\end{entry}

\begin{entry}{丑}{chou3}{4}{⼀}[HSK 5]
  \definition*{s.}{sobrenome Chou}
  \definition{adj.}{feio, sem atrativos; em oposição a 美 | vergonhoso; desavergonhado; escandaloso; censurável; questionável}
  \definition{s.}{palhaço na ópera de Pequim, etc. | o segundo dos Doze Ramos Terrestres}
  \seealsoref{美}{mei3}
\end{entry}

\begin{entry}{臭}{chou4}{10}{⾃}[HSK 5]
  \definition{adj.}{sujo; malcheiroso; fedorento; contrário de 香 | repugnante; nojento; repulsivo | ruim; pobre; péssimo}
  \definition{adv.}{severamente; firmemente}
  \definition{v.}{falhar em detonar (bala)}
  \seeref{臭}{xiu4}
  \seealsoref{香}{xiang1}
\end{entry}

\begin{entry}{臭气}{chou4qi4}{10,4}{⾃、⽓}
  \definition{s.}{fedor}
\end{entry}

\begin{entry}{出}{chu1}{5}{⼐}[HSK 1]
  \definition{clas.}{usado para dramas, peças, óperas, etc.}
  \definition{v.}{deixar; sair (ir); de dentro para fora | vir; chegar | exceder; ir além | emitir; levar para fora | produzir; despejar | surgir; acontecer; ter lugar | publicar; divulgar | ventilar; emitir; descarregar | aparecer; revelar | gastar; pagar}
\end{entry}

\begin{entry}{出版}{chu1ban3}{5,8}{⼐、⽚}[HSK 5]
  \definition{v.}{aparecer; publicar; sair; sair da imprensa}
\end{entry}

\begin{entry}{出版社}{chu1ban3she4}{5,8,7}{⼐、⽚、⽰}
  \definition{s.}{editora}
\end{entry}

\begin{entry}{出差}{chu1chai1}{5,9}{⼐、⼯}[HSK 5]
  \definition{v.+compl.}{fazer uma viagem de negócios | assumir tarefas de curto prazo em transporte, construção, etc.}
\end{entry}

\begin{entry}{出发}{chu1fa1}{5,5}{⼐、⼜}[HSK 2]
  \definition{v.}{sair; partir; ir embora; deixar; sair do lugar onde se está e ir para outro lugar | começar a partir de; partir de; considerar ou tratar uma questão a partir de um determinado ponto de vista}
\end{entry}

\begin{entry}{出国}{chu1 guo2}{5,8}{⼐、⼞}[HSK 2]
  \definition{v.+compl.}{ir para o exterior; deixar a terra natal; viajar para o exterior}
\end{entry}

\begin{entry}{出汗}{chu1 han4}{5,6}{⼐、⽔}[HSK 5]
  \definition{v.}{suar; transpirar}
\end{entry}

\begin{entry}{出击}{chu1ji1}{5,5}{⼐、⼐}
  \definition{v.}{atacar}
\end{entry}

\begin{entry}{出家}{chu1 jia1}{5,10}{⼐、⼧}
  \definition{v.}{renunciar à família (para se tornar monge ou monja) (oposto a 在家) | tornar-se monge, monja ou sacerdote taoísta}
  \seealsoref{在家}{zai4 jia1}
\end{entry}

\begin{entry}{出口}{chu1kou3}{5,3}{⼐、⼝}[HSK 2,4]
  \definition[个]{s.}{saída; porta ou passagem que dá acesso ao exterior}
  \definition{v.+compl.}{falar; proferir; manifestar-se | exportar mercadorias do país ou da região para venda no exterior ou em outro lugar | deixar o porto (de um navio)}
\end{entry}

\begin{entry}{出来}{chu1 lai2}{5,7}{⼐、⽊}[HSK 1]
  \definition{v.}{emergir; sair; (para a minha direção) |  surgir; aparecer; emergir | concluir ou algo acontecer}
\end{entry}

\begin{entry}{出门}{chu1 men2}{5,3}{⼐、⾨}[HSK 2]
  \definition{v.+compl.}{sair | sair de casa; estar longe de casa; viajar para longe de casa | casar-se}
\end{entry}

\begin{entry}{出去}{chu1 qu4}{5,5}{⼐、⼛}[HSK 1]
  \definition{v.}{sair; ir para fora;  (a partir da minha localização)}
\end{entry}

\begin{entry}{出色}{chu1se4}{5,6}{⼐、⾊}[HSK 4]
  \definition{adj.}{esplêndido; extraordinário; notável; excepcionalmente bom; acima da média}
\end{entry}

\begin{entry}{出生}{chu1sheng1}{5,5}{⼐、⽣}[HSK 2]
  \definition{v.}{nascer}
\end{entry}

\begin{entry}{出售}{chu1 shou4}{5,11}{⼐、⼝}[HSK 4]
  \definition{v.}{vender; oferecer para venda}
\end{entry}

\begin{entry}{出席}{chu1xi2}{5,10}{⼐、⼱}[HSK 4]
  \definition{v.}{comparecer; estar presente; participar de reuniões com o direito de falar e votar; juntar-se a uma organização ou atividade}
\end{entry}

\begin{entry}{出现}{chu1xian4}{5,8}{⼐、⾒}[HSK 2]
  \definition{v.}{aparecer; surgir; emergir; crescer; revelar}
\end{entry}

\begin{entry}{出行}{chu1xing2}{5,6}{⼐、⾏}
  \definition{v.}{sair para algum lugar (viagem relativamente curta) | partir em uma viagem (viagem mais longa)}
\end{entry}

\begin{entry}{出于}{chu1 yu2}{5,3}{⼐、⼆}[HSK 5]
  \definition{prep.}{de; fora de; por causa de; em função de; de um certo ponto de vista}
  \definition{v.}{iniciar a partir de; originar-se de; prosseguir a partir de}
\end{entry}

\begin{entry}{出院}{chu1 yuan4}{5,9}{⼐、⾩}[HSK 2]
  \definition{v.}{sair do hospital; estar fora do hospital; receber alta do hospital}
\end{entry}

\begin{entry}{出站}{chu1 zhan4}{5,10}{⼐、⽴}
  \definition{s.}{saída da estação}
\end{entry}

\begin{entry}{出租}{chu1 zu1}{5,10}{⼐、⽲}[HSK 2]
  \definition{s.}{taxi; abreviação de 出租车}
  \definition{v.}{alugar; arrendar; receber dinheiro de outras pessoas para permitir que elas utilizem algo (como uma casa, um carro, livros, etc.) por um determinado período de tempo}
  \seealsoref{出租车}{chu1zu1che1}
\end{entry}

\begin{entry}{出租车}{chu1zu1che1}{5,10,4}{⼐、⽲、⾞}[HSK 2]
  \definition[辆]{s.}{táxi; carro de aluguel; veículos de transporte urbano disponíveis para aluguel, com cobrança por quilometragem ou tempo}
  \seealsoref{出租汽车}{chu1zu1qi4che1}
\end{entry}

\begin{entry}{出租汽车}{chu1zu1qi4che1}{5,10,7,4}{⼐、⽲、⽔、⾞}
  \definition[辆]{s.}{táxi}
  \seealsoref{出租车}{chu1zu1che1}
\end{entry}

\begin{entry}{出租司机}{chu1zu1si1ji1}{5,10,5,6}{⼐、⽲、⼝、⽊}
  \definition{s.}{motorista de táxi}
\end{entry}

\begin{entry}{初}{chu1}{7}{⾐}[HSK 3]
  \definition*{s.}{sobrenome Chu}
  \definition{adj.}{primeiro (em ordem) | elementar; rudimentar | original}
  \definition{adv.}{pela primeira vez; apenas começando; indica que a ação está ocorrendo pela primeira vez ou acabou de começar}
  \definition{pref.}{anexado a alguns substantivos ou números cardinais, indicando o primeiro}
  \definition{s.}{no início de; na primeira parte de | o estágio júnior (pleno; sênior)}
\end{entry}

\begin{entry}{初步}{chu1bu4}{7,7}{⾐、⽌}[HSK 3]
  \definition{adj.}{inicial; preliminar; imaturo, incompleto}
\end{entry}

\begin{entry}{初级}{chu1ji2}{7,6}{⾐、⽷}[HSK 3]
  \definition{adj.}{elementar; primário; júnior; inicial; o nível mais baixo; de baixa qualidade}
\end{entry}

\begin{entry}{初期}{chu1 qi1}{7,12}{⾐、⽉}[HSK 5]
  \definition{s.}{primórdio; estágio inicial; primeiros dias; estágio preliminar; período inicial}
\end{entry}

\begin{entry}{初心}{chu1xin1}{7,4}{⾐、⼼}
  \definition{s.}{intenção original de alguém, aspiração, etc. | (budismo) ``mente do iniciante'' (ter a mente aberta quando estudando um assunto como um iniciante no assunto teria)}
\end{entry}

\begin{entry}{初中}{chu1 zhong1}{7,4}{⾐、⼁}[HSK 3]
  \definition[所,个]{s.}{ensino médio; ensino fundamental}
\end{entry}

\begin{entry}{除}{chu2}{9}{⾩}
  \definition*{s.}{sobrenome Chu}
  \definition{prep.}{exceto; não incluído | além do mais}
  \definition{s.}{degraus de uma casa; degraus de uma porta; escadaria}
  \definition{v.}{remover; livrar-se de; eliminar; limpar | dividir; executar operação de divisão | nomear para o cargo}
\end{entry}

\begin{entry}{除非}{chu2fei1}{9,8}{⾩、⾮}[HSK 5]
  \definition{conj.}{a menos que; somente se; indica a única condição, equivalente a 只有, frequentemente combinada com 才, 否则, 不然, etc.}
  \seealsoref{不然}{bu4ran2}
  \seealsoref{才}{cai2}
  \seealsoref{否则}{fou3ze2}
  \seealsoref{只有}{zhi3 you3}
\end{entry}

\begin{entry}{除了}{chu2le5}{9,2}{⾩、⼅}[HSK 3]
  \definition{prep.}{exceto; à parte; indica que o que foi dito não é levado em consideração | além disso; além de; usado em conjunto com 还, 也 e 只, indica que, além de algo, há ainda outra coisa | ou \dots ou \dots; usado em conjunto com 就是, significa "ou assim ou assado"}
  \seealsoref{还}{hai2}
  \seealsoref{就是}{jiu4 shi4}
  \seealsoref{也}{ye3}
  \seealsoref{只}{zhi3}
\end{entry}

\begin{entry}{除夕}{chu2xi1}{9,3}{⾩、⼣}[HSK 5]
  \definition*{s.}{Véspera de Ano Novo Lunar; a noite do último dia do ano, também se refere ao último dia do ano}
\end{entry}

\begin{entry}{厨}{chu2}{12}{⼚}
  \definition[个]{s.}{cozinha}
\end{entry}

\begin{entry}{厨房}{chu2fang2}{12,8}{⼚、⼾}[HSK 5]
  \definition[间,个]{s.}{cozinha}
\end{entry}

\begin{entry}{处}{chu3}{5}{⼡}[HSK 4]
  \definition*{s.}{sobrenome Chu}
  \definition{v.}{morar; habitar; viver em um lugar | dar-se bem (com alguém); relacionar-se; interagir | estar situado em; estar em uma determinada condição; estar em (um lugar, período ou ocasião) | gerenciar; manejar; lidar com | punir; sentenciar; tomar medidas disciplinares contra (alguém)}
  \seeref{处}{chu4}
\end{entry}

\begin{entry}{处罚}{chu3 fa2}{5,9}{⼡、⽹}[HSK 5]
  \definition[个]{s.}{punição; castigo; penalidade}
  \definition{v.}{punir; disciplinar; castigar; advertir o transgressor ou infrator sobre perdas políticas ou financeiras}
\end{entry}

\begin{entry}{处分}{chu3fen4}{5,4}{⼡、⼑}[HSK 5]
  \definition{s.}{punição; castigo; refere-se a uma decisão de impor uma penalidade ou uma disposição}
  \definition{v.}{punir; tomar medidas disciplinares contra; fornecer algum tratamento ou disposição para aqueles que cometeram erros ou falhas}
\end{entry}

\begin{entry}{处理}{chu3li3}{5,11}{⼡、⽟}[HSK 3]
  \definition{s.}{manuseio; descarte}
  \definition{v.}{lidar com; dispor de; organizar; resolver | resolver; punir; lidar | vender a preços reduzidos; liquidar | lidar com; processar; processar algo de uma maneira ou método específico; processar uma peça de trabalho ou produto de uma maneira específica para que a peça de trabalho ou produto obtenha o desempenho necessário}
\end{entry}

\begin{entry}{处于}{chu3 yu2}{5,3}{⼡、⼆}[HSK 4]
  \definition{v.}{estar em (uma condição, estado)}
\end{entry}

\begin{entry}{处在}{chu3 zai4}{5,6}{⼡、⼟}[HSK 5]
  \definition{v.}{estar situado em; encontrar-se em; estar em (algum estado, posição ou condição)}
\end{entry}

\begin{entry}{处}{chu4}{5}{⼡}
  \definition{clas.}{usado para lugares ou para ocorrências ou atividades em lugares diferentes}
  \definition{s.}{lugar; local; instalação; dependência | parte; ponto; aspecto ou parte de um objeto | escritório; departamento; nomes de determinados órgãos, organizações ou unidades em órgãos por empresa}
  \seeref{处}{chu3}
\end{entry}

\begin{entry}{处处}{chu4chu4}{5,5}{⼡、⼡}
  \definition{adv.}{em todos os lugares | em todos os aspectos}
\end{entry}

\begin{entry}{畜}{chu4}{10}{⽥}
  \definition*{s.}{sobrenome Chu}
  \definition{s.}{animal doméstico; gado; bestas, principalmente referindo-se ao gado}
  \seeref{畜}{xu4}
\end{entry}

\begin{entry}{穿}{chuan1}{9}{⽳}[HSK 1]
  \definition{adj.}{direto; através; usado após certos verbos, indica um estado de revelação completa}
  \definition{s.}{vestuário; roupas; refere-se a roupas, sapatos, meias, etc.}
  \definition{v.}{usar; vestir; estar vestido; ter \dots vestido;  vestir roupas, sapatos, meias, etc. | perfurar através de; penetrar; formar orifícios por meio de cinzéis, brocas ou pontas afiadas | enfiar; amarrar; usar cordas e fios para ligar coisas | passar por; atravessar; passar por; através de (buracos, fendas, espaços vazios, etc.)}
\end{entry}

\begin{entry}{穿上}{chuan1 shang4}{9,3}{⽳、⼀}[HSK 4]
  \definition{v.}{vestir (roupas, etc.); colocar roupas}
\end{entry}

\begin{entry}{传}{chuan2}{6}{⼈}[HSK 3]
  \definition{v.}{passar; passar adiante | passar adiante; legar; passar de\dots para\dots; passar da geração anterior para a seguinte | transmitir (conhecimento, habilidade, etc.); comunicar; ensinar | espalhar; propagar | transmitir; conduzir; transferir | transmitir; expressar | convocar; dar ordem para chamar alguém | infectar; ser contagioso | enviar documentos por e-mail ou fax}
  \seeref{传}{zhuan4}
\end{entry}

\begin{entry}{传播}{chuan2bo1}{6,15}{⼈、⼿}[HSK 3]
  \definition{v.}{espalhar; difundir; propagar; disseminar}
\end{entry}

\begin{entry}{传承}{chuan2cheng2}{6,8}{⼈、⼿}
  \definition{s.}{herança | tradição continuada}
  \definition{v.}{transmitir (para as gerações futuras) | passar adiante (desde os tempos antigos)}
\end{entry}

\begin{entry}{传达}{chuan2da2}{6,6}{⼈、⾡}[HSK 5]
  \definition{s.}{recepção e registro de chamadas em um estabelecimento público | zelador; recepcionista}
  \definition{v.}{passar adiante (informações, etc.); transmitir; retransmitir; comunicar}
\end{entry}

\begin{entry}{传递}{chuan2 di4}{6,10}{⼈、⾡}[HSK 5]
  \definition{v.}{transmitir; entregar; transferir; passar adiante}
\end{entry}

\begin{entry}{传给}{chuan2gei3}{6,9}{⼈、⽷}
  \definition{v.}{passar para | transferir para | entregar a}
\end{entry}

\begin{entry}{传来}{chuan2 lai2}{6,7}{⼈、⽊}[HSK 3]
  \definition{v.}{(um som) passar; transmitir de algum lugar para o local onde o locutor se encontra | (notícias) chegar; transmitir mensagens ou informações}
\end{entry}

\begin{entry}{传说}{chuan2shuo1}{6,9}{⼈、⾔}[HSK 3]
  \definition[个,种,段]{s.}{lenda; conto popular; folclore; coisas lendárias; especificamente, lendas populares}
  \definition{v.}{dizer que; ser dito; passar de boca em boca; transmitir oralmente, segundo a tradição}
\end{entry}

\begin{entry}{传统}{chuan2tong3}{6,9}{⼈、⽷}[HSK 4]
  \definition{adj.}{tradicional; histórico; transmitido de geração em geração | antiquado, conservador e fora de sintonia com os tempos}
  \definition[个]{s.}{tradição; costume; fatores sociais, como costumes, moral, ideias, estilos, artes, instituições etc., que são transmitidos de uma geração para outra e que são característicos da sociedade}
\end{entry}

\begin{entry}{传真}{chuan2zhen1}{6,10}{⼈、⼗}[HSK 5]
  \definition[台,部,份]{s.}{\emph{FAX}, facsímile; texto, diagramas, fotografias, etc., transmitidos por aparelho de fax}
  \definition{v.}{enviar um fax}
\end{entry}

\begin{entry}{船}{chuan2}{11}{⾈}[HSK 2]
  \definition*{s.}{sobrenome Chuan}
  \definition[条,艘,叶,只]{s.}{barco; navio | embarcação; meio de transporte aquático, nome genérico para embarcações}
\end{entry}

\begin{entry}{创}{chuang1}{6}{⼑}
  \definition{s.}{ferimento; trauma}
  \seeref{创}{chuang4}
\end{entry}

\begin{entry}{窗}{chuang1}{12}{⽳}
  \definition[扇,个]{s.}{janela}
\end{entry}

\begin{entry}{窗户}{chuang1hu5}{12,4}{⽳、⼾}[HSK 4]
  \definition[个,扇,面,排]{s.}{janela; dispositivo de ventilação e transmissão de luz nas paredes}
\end{entry}

\begin{entry}{窗帘}{chuang1lian2}{12,8}{⽳、⼱}[HSK 5]
  \definition[副,幅,个,套,片,对]{s.}{cortinas para janelas}
\end{entry}

\begin{entry}{窗台}{chuang1 tai2}{12,5}{⽳、⼝}[HSK 4]
  \definition{s.}{parapeito da janela; peitoril; parte plana de uma janela que segura a moldura}
\end{entry}

\begin{entry}{窗子}{chuang1 zi5}{12,3}{⽳、⼦}[HSK 4]
  \definition{s.}{janela}
\end{entry}

\begin{entry}{床}{chuang2}{7}{⼴}[HSK 1]
  \definition{clas.}{usado para colchas, roupas de cama, etc.}
  \definition[张]{s.}{cama; sofá; móveis para dormir | algo com o formato de uma cama}
\end{entry}

\begin{entry}{闯}{chuang3}{6}{⾨}[HSK 5]
  \definition*{s.}{sobrenome Chuang}
  \definition{v.}{apressar-se; correr | moderar a si mesmo (lutando contra dificuldades e perigos); aventurar-se no mundo | incorrer; causar (um desastre, etc.)}
\end{entry}

\begin{entry}{创}{chuang4}{6}{⼑}
  \definition{v.}{começar (fazer algo); alcançar (algo pela primeira vez); estabelecer; fazer pela primeira vez | estabelecer; fundar; criar; perceber algo novo, como um começo | ferir; machucar}
  \seeref{创}{chuang1}
\end{entry}

\begin{entry}{创立}{chuang4li4}{6,5}{⼑、⽴}[HSK 5]
  \definition{v.}{fundar; originar; estabelecer}
\end{entry}

\begin{entry}{创新}{chuang4xin1}{6,13}{⼑、⽄}[HSK 3]
  \definition[个,种,次]{s.}{inovação; algo novo ou diferente, uma ideia}
  \definition{v.}{trazer novas ideias; inovar; abrir novos caminhos; criar ou fazer algo novo, diferente do que era antes}
\end{entry}

\begin{entry}{创业}{chuang4ye4}{6,5}{⼑、⼀}[HSK 3]
  \definition{s.}{empreendedorismo}
  \definition{v.}{começar um empreendimento; iniciar/fundar um negócio, uma empresa;}
\end{entry}

\begin{entry}{创意}{chuang4yi4}{6,13}{⼑、⼼}
  \definition{adj.}{criativo}
  \definition{s.}{criatividade}
\end{entry}

\begin{entry}{创造}{chuang4zao4}{6,10}{⼑、⾡}[HSK 3]
  \definition{s.}{criação; inovação; primeiro a concluir ou a alcançar resultados}
  \definition{v.}{criar; produzir; trazer à tona; fazer ou estabelecer pela primeira vez; referir-se de maneira geral a fazer ou estabelecer}
\end{entry}

\begin{entry}{创作}{chuang4zuo4}{6,7}{⼑、⼈}[HSK 3]
  \definition[个]{s.}{criação; trabalho criativo; obras literárias e artísticas}
  \definition{v.}{escrever; criar; produzir; compor; criar obras artísticas}
\end{entry}

\begin{entry}{吹}{chui1}{7}{⼝}[HSK 2]
  \definition{v.}{soprar; baforar | tocar (instrumentos de sopro) | (do vento) soprar | gabar-se; vangloriar-se | elogiar; louvar aos céus; adular | (relacionamento) romper; separar-se; (assunto) fracassar}
\end{entry}

\begin{entry}{吹牛}{chui1niu2}{7,4}{⼝、⽜}
  \definition{v.+compl.}{ogulhar-se | gabar-se | destacar-se}
\end{entry}

\begin{entry}{锤}{chui2}{13}{⾦}
  \definition[把,个]{s.}{uma bola de metal com uma alça ou corrente, usada como arma; maça | algo como um martelo | martelo}
  \definition{v.}{martelar para dar forma; bater com um martelo}
\end{entry}

\begin{entry}{春}{chun1}{9}{⽇}
  \definition*{s.}{sobrenome Chun}
  \definition{s.}{primavera | amor; luxúria | vida; vitalidade}
\end{entry}

\begin{entry}{春季}{chun1 ji4}{9,8}{⽇、⼦}[HSK 4]
  \definition{s.}{primavera; primeiro trimestre do ano, que na China se refere ao período de três meses entre o início da primavera e o início do verão, e também se refere aos três meses do calendário lunar, a saber, o primeiro, o segundo e o terceiro meses}
\end{entry}

\begin{entry}{春节}{chun1 jie2}{9,5}{⽇、⾋}[HSK 2]
  \definition*[个]{s.}{Festival da Primavera (Ano Novo Chinês); o primeiro dia do primeiro mês do calendário lunar, também se refere aos dias seguintes ao primeiro dia do primeiro mês}
\end{entry}

\begin{entry}{春天}{chun1 tian1}{9,4}{⽇、⼤}
  \definition[个,段,季,番]{s.}{primavera; época da primavera | primavera; renascimento; uma atmosfera cheia de energia e esperança}
\end{entry}

\begin{entry}{纯}{chun2}{7}{⽷}[HSK 4]
  \definition{adj.}{puro; não misturado; livre de impurezas | simples; puro e simples | habilidoso; proficiente; bem versado}
  \definition{adv.}{puramente; completamente; totalmente | genuinamente}
\end{entry}

\begin{entry}{纯净水}{chun2 jing4 shui3}{7,8,4}{⽷、⼎、⽔}[HSK 4]
  \definition{s.}{água purificada}
\end{entry}

\begin{entry}{纯真}{chun2zhen1}{7,10}{⽷、⼗}
  \definition{adj.}{inocente e não afetado | puro e não adulterado}
  \definition{s.}{inocência}
\end{entry}

\begin{entry}{唇}{chun2}{10}{⼝}
  \definition[片]{s.}{lábios}
\end{entry}

\begin{entry}{绰}{chuo4}{11}{⽷}
  \definition{adj.}{amplo; espaçoso | (do porte de uma menina) graciosa; flexível}
\end{entry}

\begin{entry}{绰号}{chuo4hao4}{11,5}{⽷、⼝}
  \definition{s.}{apelido}
\end{entry}

\begin{entry}{刺}{ci1}{8}{⼑}
  \definition{s.}{(onomatopéia) som de rasgo, fricção, etc.}
\end{entry}

\begin{entry}{词}{ci2}{7}{⾔}[HSK 2]
  \definition[个,组,句,段,首]{s.}{palavra; termo; antigamente, referia-se a palavras vazias; atualmente, refere-se a palavras com forma fonética fixa e significado específico na língua; a menor unidade que pode ser usada de forma independente | discurso; declaração; linguagem; texto | ci (um tipo de poesia clássica chinesa, originária da dinastia Tang e plenamente desenvolvida na dinastia Song); gênero poético escrito de acordo com uma estrutura fixa, com versos de comprimentos variados | palavras; redação; refere-se genericamente ao teatro; a parte da letra cantada em harmonia com a melodia em canções e certas artes vocais}
\end{entry}

\begin{entry}{词典}{ci2dian3}{7,8}{⾔、⼋}[HSK 2]
  \definition[本,部]{s.}{dicionário, livro de referência que reúne palavras e explicações para consulta}
  \seealsoref{字典}{zi4 dian3}
\end{entry}

\begin{entry}{词汇}{ci2hui4}{7,5}{⾔、⽔}[HSK 4]
  \definition[个,组,批,串,堆]{s.}{vocabulário; termo geral para palavras usadas em um idioma}
\end{entry}

\begin{entry}{词语}{ci2yu3}{7,9}{⾔、⾔}[HSK 2]
  \definition[个,租]{s.}{termo; palavra; expressão; conjunto de palavras e frases}
\end{entry}

\begin{entry}{瓷}{ci2}{10}{⽡}
  \definition{adj.}{(dialeto) (de relação) próxima; íntima}
  \definition{s.}{artigos de porcelana}
\end{entry}

\begin{entry}{辞}{ci2}{13}{⾟}
  \definition[首]{s.}{dicção; fraseologia | um tipo de literatura clássica chinesa; um gênero da literatura clássica | uma forma de poesia clássica}
  \definition{v.}{despedir-se | declinar | renunciar | dispensar; demitir | fugir; evitar}
\end{entry}

\begin{entry}{辞典}{ci2 dian3}{13,8}{⾟、⼋}[HSK 5]
  \definition[本,部]{s.}{dicionário; coleção de termos especializados ou enciclopédicos, organizados em uma determinada ordem e explicados, para fins de referência}
  \variantof{词典}
\end{entry}

\begin{entry}{辞职}{ci2zhi2}{13,11}{⾟、⽿}[HSK 5]
  \definition{v.+compl.}{renunciar; deixar o cargo; entregar a renúncia; pedir para ser dispensado de suas funções}
\end{entry}

\begin{entry}{磁}{ci2}{14}{⽯}
  \definition[块]{s.}{porcelana | (física) magnetismo; propriedade de atrair ferro, níquel, etc. | (dialeto)  (de relação) próximo; íntimo}
\end{entry}

\begin{entry}{磁带}{ci2dai4}{14,9}{⽯、⼱}
  \definition[盘,盒]{s.}{cassete | fita magnética}
\end{entry}

\begin{entry}{磁盘}{ci2pan2}{14,11}{⽯、⽫}
  \definition{s.}{disquete}
\end{entry}

\begin{entry}{磁铁}{ci2tie3}{14,10}{⽯、⾦}
  \definition{s.}{imã | magneto}
  \seealsoref{吸铁石}{xi1tie3shi2}
\end{entry}

\begin{entry}{此}{ci3}{6}{⽌}[HSK 4]
  \definition*{s.}{sobrenome Ci}
  \definition{pron.}{esse; essa; isso; este; esta; isto; indica ou se refere a uma pessoa ou coisa que está mais próxima, equivalente a 这 ou 这个 (em oposição a 彼) | aqui e agora; refere-se a um tempo ou lugar recente, equivalente a 这会儿 ou 这里}
  \seealsoref{彼}{bi3}
  \seealsoref{这}{zhe4}
  \seealsoref{这会儿}{zhe4 hui4r5}
  \seealsoref{这里}{zhe4 li3}
  \seealsoref{这个}{zhe4ge5}
\end{entry}

\begin{entry}{此后}{ci3 hou4}{6,6}{⽌、⼝}[HSK 5]
  \definition{s.}{daqui em diante; doravante; depois disso; após isso; de agora em diante}
\end{entry}

\begin{entry}{此刻}{ci3 ke4}{6,8}{⽌、⼑}[HSK 5]
  \definition{s.}{agora; no momento; exatamente agora; neste momento}
\end{entry}

\begin{entry}{此时}{ci3 shi2}{6,7}{⽌、⽇}[HSK 5]
  \definition{s.}{agora; no presente; agora mesmo; neste momento; por enquanto}
\end{entry}

\begin{entry}{此外}{ci3wai4}{6,5}{⽌、⼣}[HSK 4]
  \definition{conj.}{além disso; em adição; além das coisas ou situações mencionadas acima}
\end{entry}

\begin{entry}{次}{ci4}{6}{⽋}[HSK 1,4]
  \definition*{s.}{sobrenome Ci}
  \definition{adj.}{de segunda categoria; de qualidade inferior}
  \definition{clas.}{usado para coisas ou ações que podem ser repetidas}
  \definition{num.}{segundo; próximo}
  \definition{pref.}{(química) hipo-, radical ácido ou composto contendo dois átomos de oxigênio a menos}
  \definition{s.}{ordem; sequência; classificação | local de parada em uma viagem; escala}
\end{entry}

\begin{entry}{刺}{ci4}{8}{⼑}[HSK 4]
  \definition*{s.}{sobrenome Ci}
  \definition{s.}{espinho; farpa; algo afiado como uma agulha | cartão de visita | saliências; projeções pequenas e pontiagudas na superfície de um objeto ou na pele de uma pessoa}
  \definition{v.}{esfaquear; perfurar | irritar; estimular | assassinar | espionar; detectar | criticar}
  \seeref{刺}{ci1}
\end{entry}

\begin{entry}{刺激}{ci4ji1}{8,16}{⼑、⽔}[HSK 4]
  \definition{adj.}{animado; entusiasmado; sensação de empolgação e nervosismo}
  \definition[个]{s.}{estímulo; estimulação; fortes efeitos físicos ou psicológicos}
  \definition{v.}{irritar; provocar; estimular | incentivar; estimular; incitar; (por algum meio) para mudar as coisas para melhor, para fazer coisas positivas}
\end{entry}

\begin{entry}{刺猬}{ci4wei5}{8,12}{⼑、⽝}
  \definition{s.}{porco-espinho | ouriço}
\end{entry}

\begin{entry}{匆}{cong1}{5}{⼓}
  \definition{adj.}{apressado}
  \definition{adv.}{apressadamente}
\end{entry}

\begin{entry}{匆匆}{cong1cong1}{5,5}{⼓、⼓}
  \definition{adv.}{apressadamente}
\end{entry}

\begin{entry}{葱}{cong1}{12}{⾋}
  \definition{adj.}{verde; turquesa}
  \definition[根,把,捆]{s.}{cebola; cebolinha}
\end{entry}

\begin{entry}{聪}{cong1}{15}{⽿}
  \definition{adj.}{audição aguçada | brilhante; inteligente; esperto | perspicaz}
  \definition{s.}{(literário) faculdades auditivas}
\end{entry}

\begin{entry}{聪慧}{cong1hui4}{15,15}{⽿、⼼}
  \definition{adj.}{inteligente | brilhante}
\end{entry}

\begin{entry}{聪明}{cong1ming5}{15,8}{⽿、⽇}[HSK 5]
  \definition{adj.}{brilhante; esperto; inteligente; intelecto bem desenvolvido com boa memória e capacidade de compreensão}
\end{entry}

\begin{entry}{从}{cong2}{4}{⼈}[HSK 1]
  \definition*{s.}{sobrenome Cong}
  \definition{adj.}{secundário; acessório | relacionamento entre primos, etc., do mesmo avô paterno, bisavô ou de um ancestral comum ainda mais antigo; do mesmo clã}
  \definition{adv.}{(seguido de uma negativa) jamais | jamais; usado antes de palavras negativas, indica que algo nunca aconteceu desde o passado, equivalente a 从来}
  \definition{prep.}{de (um tempo, um lugar ou um ponto de vista) | via, através ou após (um local) | de; via; através de; passado (um lugar); introdução das ações, trajetórias e locais| (de comportamento) de; introdução de ações e comportamentos com base em referências e fundamentos}
  \definition{s.}{seguidor; acompanhante}
  \definition{v.}{seguir; cumprir; obedecer | participar; estar envolvido em | seguir o princípio de; empregar o método de | estar envolvido em | seguir; adotar (um determinado princípio ou atitude)}
  \seealsoref{从来}{cong2lai2}
\end{entry}

\begin{entry}{从不}{cong2bu4}{4,4}{⼈、⼀}
  \definition{adv.}{nunca}
\end{entry}

\begin{entry}{从此}{cong2ci3}{4,6}{⼈、⽌}[HSK 4]
  \definition{conj.}{doravante; portanto; a partir deste momento; de agora em diante; a partir de então}
\end{entry}

\begin{entry}{从而}{cong2'er2}{4,6}{⼈、⽽}[HSK 5]
  \definition{conj.}{assim; por isso; portanto; desse modo; por esse motivo; conjunção usada no início do texto seguinte para expressar o resultado, propósito ou ação posterior, o que é equivalente a 因此就}
  \seealsoref{因此就}{yin1ci3 jiu4}
\end{entry}

\begin{entry}{从来}{cong2lai2}{4,7}{⼈、⽊}[HSK 3]
  \definition{adv.}{sempre; o tempo todo; em todos os momentos; do passado até o presente}
\end{entry}

\begin{entry}{从前}{cong2qian2}{4,9}{⼈、⼑}[HSK 3]
  \definition{s.}{antes; antigamente; no passado | era uma vez; há muito tempo atrás}
\end{entry}

\begin{entry}{从事}{cong2shi4}{4,8}{⼈、⼅}[HSK 3]
  \definition{v.}{trabalhar; empreender; empenhar-se em; envolver-se em; dedicar-se ou comprometer-se (a uma causa); participar (de algo) | lidar com; deve ter um advérbio antes e não pode ter um objeto depois}
\end{entry}

\begin{entry}{从未}{cong2wei4}{4,5}{⼈、⽊}
  \definition{adv.}{nunca}
\end{entry}

\begin{entry}{从小}{cong2 xiao3}{4,3}{⼈、⼩}[HSK 2]
  \definition{adv.}{desde a infância; desde muito jovem; quando criança}
\end{entry}

\begin{entry}{从中}{cong2 zhong1}{4,4}{⼈、⼁}[HSK 5]
  \definition{adv.}{de; dentre; daí}
\end{entry}

\begin{entry}{粗}{cu1}{11}{⽶}[HSK 4]
  \definition{adj.}{largo (em diâmetro); grosso | grosseiro; rude; áspero | áspero; rouco | descuidado; negligente | rude; sem refinamento; vulgar}
  \definition{adv.}{grosseiramente; vagamente}
\end{entry}

\begin{entry}{粗糙}{cu1cao1}{11,16}{⽶、⽶}
  \definition{adj.}{áspero | grosseiro}
\end{entry}

\begin{entry}{粗心}{cu1xin1}{11,4}{⽶、⼼}[HSK 4]
  \definition{adj.}{descuidado; irrefletido; (fazer as coisas) de forma desleixada, sem cuidado}
\end{entry}

\begin{entry}{粗心地做}{cu1xin1 di4 zuo4}{11,4,6,11}{⽶、⼼、⼟、⼈}
  \definition{adj.}{feito descuidadamente}
\end{entry}

\begin{entry}{促}{cu4}{9}{⼈}
  \definition{adj.}{curto; apressado; urgente}
  \definition{v.}{urgir; promover | estar perto de; estar perto}
\end{entry}

\begin{entry}{促进}{cu4jin4}{9,7}{⼈、⾡}[HSK 4]
  \definition{v.}{impulsionar; promover; avançar; incentivar o desenvolvimento}
\end{entry}

\begin{entry}{促使}{cu4shi3}{9,8}{⼈、⼈}[HSK 4]
  \definition{v.}{incitar; estimular; impelir; causar; provocar uma mudança em alguém ou em algo}
\end{entry}

\begin{entry}{促销}{cu4 xiao1}{9,12}{⼈、⾦}[HSK 4]
  \definition{v.}{promover vendas}
\end{entry}

\begin{entry}{酢}{cu4}{12}{⾣}
  \definition{s.}{vinagre | (figurativo) ciúme (como em um caso de amor)}
  \variantof{醋}
\end{entry}

\begin{entry}{醋}{cu4}{15}{⾣}
  \definition[瓶,坛,碟,碗]{s.}{(condimento) vinagre | ciúme (como em caso de amor); uma metáfora para o ciúme, referindo-se principalmente aos relacionamentos entre pessoas}
\end{entry}

\begin{entry}{窾}{cuan4}{17}{⽳}
  \definition{adj.}{vazio | seco | destituído; pobre}
  \definition{s.}{buraco | lei}
  \definition{v.}{esconder}
  \seeref{窾}{kuan3}
\end{entry}

\begin{entry}{脆}{cui4}{10}{⾁}[HSK 5]
  \definition{adj.}{frágil; quebradiço (oposto a 韧) | crocante | (voz) clara; nítida | puro}
  \seealsoref{韧}{ren4}
\end{entry}

\begin{entry}{村}{cun1}{7}{⽊}[HSK 3]
  \definition{adj.}{rústico; grosseiro}
  \definition[个,座]{s.}{aldeia; vila | área povoada de certo tipo}
\end{entry}

\begin{entry}{村儿}{cun1r5}{7,2}{⽊、⼉}
  \definition{s.}{vila; aldeia}
\end{entry}

\begin{entry}{存}{cun2}{6}{⼦}[HSK 3]
  \definition{v.}{existir; viver; sobreviver | armazenar; manter | acumular; coletar | depositar | sair com; verificar | reservar; reter | permanecer em equilíbrio; estar em estoque | estimar; abrigar}
\end{entry}

\begin{entry}{存款}{cun2 kuan3}{6,12}{⼦、⽋}[HSK 5]
  \definition[笔]{s.}{depósito; poupança bancária}
  \definition{v.}{depositar dinheiro; colocar dinheiro no banco}
\end{entry}

\begin{entry}{存在}{cun2zai4}{6,6}{⼦、⼟}[HSK 3]
  \definition{s.}{existência; ser; ente; o mundo objetivo, que não depende da consciência humana para mudar, ou seja, a matéria}
  \definition{v.}{existir; ser; as coisas ocupam continuamente o tempo e o espaço; na verdade, ainda não desapareceram}
\end{entry}

\begin{entry}{寸}{cun4}{3}{⼨}[HSK 5][Kangxi 41]
  \definition*{s.}{sobrenome Cun}
  \definition{adj.}{muito pouco; muito curto; pequeno | (dialeto) coincidência}
  \definition{clas.}{cun, uma unidade tradicional de comprimento, igual a 0,1 市尺 e equivalente a 3,333 centímetros ou 1,312 polegadas | cun, uma unidade de comprimento (=13 decímetros)}
  \seealsoref{市尺}{shi4 chi3}
\end{entry}

\begin{entry}{搓}{cuo1}{12}{⼿}
  \definition{s.}{torção}
  \definition{v.}{esfregar ou rolar entre as mãos ou dedos |  (no tênis, tênis de mesa, críquete, etc.) cortar | (de roupa, etc.) torcer}
\end{entry}

\begin{entry}{鹾}{cuo2}{16}{⿄}
  \definition{adj.}{salgado}
  \definition{s.}{sal}
\end{entry}

\begin{entry}{挫}{cuo4}{10}{⼿}
  \definition{v.}{frustrar | diminuir; embotar; desinflar | pressionar para baixo; abaixar}
\end{entry}

\begin{entry}{挫折}{cuo4zhe2}{10,7}{⼿、⼿}
  \definition{s.}{revés | reverso | derrota | frustração | decepção}
  \definition{v.}{frustrar | desencorajar | subjugar}
\end{entry}

\begin{entry}{措}{cuo4}{11}{⼿}
  \definition{s.}{iniciativa; solução; medida}
  \definition{v.}{organizar; gerenciar; lidar | fazer planos; administrar; organizar}
\end{entry}

\begin{entry}{措施}{cuo4shi1}{11,9}{⼿、⽅}[HSK 4]
  \definition{s.}{medida; etapa; passo; abordagem adotada para lidar com as coisas}
\end{entry}

\begin{entry}{错}{cuo4}{13}{⾦}[HSK 1]
  \definition{adj.}{errado; equivocado; errôneo | (na negativa) nada ruim; muito bom | entrelaçado e recortado; intrincado; complexo | ruim; pobre; péssimo (usado apenas em negativas)}
  \definition{s.}{falha; demérito | erro; engano | (arcaico) pedra de amolar para polir jade}
  \definition{v.}{estar entrelaçado e serrilhado; ser intrincado | moer; esfregar | abrir caminho; sair do caminho | alternar; escalonar | estar fora de alinhamento | deslocar | evitar; fazer com que não se encontre ou não entre em conflito | polir; polir pedras preciosas | (literário) incrustar ou revestir com ouro, prata, etc. | interseccionar; cruzar; entrecruzar}
\end{entry}

\begin{entry}{错误}{cuo4wu4}{13,9}{⾦、⾔}[HSK 3]
  \definition{adj.}{equivocado; errado; errôneo; incorreto; não condizente com a realidade objetiva}
  \definition[个,次]{s.}{engano; erro; erro grosseiro; falha; coisas, comportamentos, etc. incorretos}
\end{entry}

%%%%% EOF %%%%%

