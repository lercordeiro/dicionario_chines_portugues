%%%
%%% C
%%%

\section*{C}\addcontentsline{toc}{section}{C}

\begin{entry}{擦拭}{ca1shi4}{17,9}
  \definition{v.}{limpar com um pano}
\end{entry}

\begin{entry}{猜}{cai1}{11}[Radical 犬]
  \definition{v.}{advinhar}
\end{entry}

\begin{entry}{才}{cai2}{3}[Radical 手][HSK 2]
  \definition{adv.}{apenas (seguido por uma cláusula numérica) | meramente | só (indicando que algo está acontecendo mais tarde do que o esperado) | só depois | só então | não\dots até (precedido por uma cláusula de condição ou razão) | há um momento atrás | há pouco tempo}
  \definition{conj.}{apenas quando}
  \definition{s.}{um indivíduo capaz | habilidade | talento}
\end{entry}

\begin{entry}{才华}{cai2hua2}{3,6}
  \definition[份]{s.}{talento}
\end{entry}

\begin{entry}{才略}{cai2lve4}{3,11}
  \definition{s.}{habilidade e sagacidade}
\end{entry}

\begin{entry}{才能}{cai2 neng2}{3,10}[HSK 3]
  \definition[间]{s.}{talento | habilidade | dom | capacidade}
\end{entry}

\begin{entry}{裁}{cai2}{12}[Radical 衣]
  \definition{s.}{decisão | julgamento}
  \definition{v.}{recortar (tecido de uma roupa) | cortar | aparar | reduzir | diminuir | cortar pessoal de uma equipe}
\end{entry}

\begin{entry}{采访}{cai3fang3}{8,6}
  \definition{s.}{entrevista}
  \definition{v.}{entrevistar | reunir notícias | cobrir (eventos)}
\end{entry}

\begin{entry}{采取}{cai3qu3}{8,8}[HSK 3]
  \definition{v.}{adotar | reunir | coletar | tomar | assumir}
\end{entry}

\begin{entry}{采用}{cai3 yong4}{8,5}[HSK 3]
  \definition{v.}{selecionar e usar | adotar}
\end{entry}

\begin{entry}{彩虹}{cai3hong2}{11,9}
  \definition[道]{s.}{arco-íris}
\end{entry}

\begin{entry}{彩色}{cai3 se4}{11,6}[HSK 3]
  \definition{s.}{multicolorido; cor}
\end{entry}

\begin{entry}{菜}{cai4}{11}[Radical 艸][HSK 1]
  \definition[棵]{s.}{hortaliça | verdura | legume}
  \definition[样,道,盘]{s.}{prato (tipo de alimento) | o tipo (de alguém) | (características de alguém, etc.) fraco, pobre}
\end{entry}

\begin{entry}{菜单}{cai4dan1}{11,8}[HSK 2]
  \definition[份,张]{s.}{menu | cardápio}
\end{entry}

\begin{entry}{菜刀}{cai4dao1}{11,2}
  \definition[把]{s.}{faca de vegetais | faca de cozinha | cutelo}
\end{entry}

\begin{entry}{参观}{can1guan1}{8,6}[HSK 2]
  \definition{v.}{visitar}
\end{entry}

\begin{entry}{参加}{can1jia1}{8,5}[HSK 2]
  \definition{v.}{participar de | tomar parte em | assistir}
\end{entry}

\begin{entry}{餐厅}{can1ting1}{16,4}
  \definition[家]{s.}{restaurante}
  \definition[间]{s.}{sala de jantar}
\end{entry}

\begin{entry}{残疾人}{can2ji2ren2}{9,10,2}
  \definition{s.}{pessoa com deficiência}
\end{entry}

\begin{entry}{残酷}{can2ku4}{9,14}
  \definition{adj.}{cruel}
  \definition{s.}{crueldade}
\end{entry}

\begin{entry}{蚕纸}{can2zhi3}{10,7}
  \definition{s.}{papel onde o bicho-da-seda põe seus ovos}
\end{entry}

\begin{entry}{惨}{can3}{11}[Radical 心]
  \definition{adj.}{miserável | cruel | desumano | desastroso | trágico | sombrio}
\end{entry}

\begin{entry}{舱}{cang1}{10}[Radical ⾈]
  \definition{s.}{cabine | porão (de carga) de um navio ou avião}
\end{entry}

\begin{entry}{操心}{cao1xin1}{16,4}
  \definition{v.+compl.}{preocupar-se com}
\end{entry}

\begin{entry}{操作}{cao1zuo4}{16,7}
  \definition{s.}{operação}
  \definition{v.}{trabalhar | operar | manipular}
\end{entry}

\begin{entry}{槽}{cao2}{15}[Radical ⽊]
  \definition{s.}{calha | canal | sulco | manjedoura}
\end{entry}

\begin{entry}{草}{cao3}{9}[Radical 艸][HSK 2]
  \definition[棵,撮,株,根]{s.}{erva | grama}
\end{entry}

\begin{entry}{草地}{cao3 di4}{9,6}[HSK 2]
  \definition[片]{s.}{relva | pastagem}
\end{entry}

\begin{entry}{草莓}{cao3mei2}{9,10}
  \definition[颗]{s.}{morango}
\end{entry}

\begin{entry}{草纸}{cao3zhi3}{9,7}
  \definition{s.}{papel pardo | pergaminho | papel de palha áspero | papel higiênico}
\end{entry}

\begin{entry}{肏}{cao4}{8}[Radical 肉]
  \definition{v.}{(vulgar) foder}
\end{entry}

\begin{entry}{厕所}{ce4suo3}{8,8}
  \definition[间,处]{s.}{lavatório | \emph{toilette}}
\end{entry}

\begin{entry}{厕纸}{ce4zhi3}{8,7}
  \definition{s.}{papel higiênico}
\end{entry}

\begin{entry}{策划}{ce4hua4}{12,6}
  \definition{s.}{planejador | produtor | plano}
  \definition{v.}{esquematizar | engenhar | planejar}
\end{entry}

\begin{entry}{层}{ceng2}{7}[Radical ⼫][HSK 2]
  \definition{clas.}{para andar, piso}
\end{entry}

\begin{entry}{层层}{ceng2ceng2}{7,7}
  \definition{s.}{camada sobre camada}
\end{entry}

\begin{entry}{层次}{ceng2ci4}{7,6}
  \definition{s.}{camada | nível | graduação | arranjo de ideias}
\end{entry}

\begin{entry}{曾经}{ceng2jing1}{12,8}[HSK 3]
  \definition{adv.}{uma vez | antes | costumava | no passado}
\end{entry}

\begin{entry}{插话}{cha1hua4}{12,8}
  \definition{s.}{interrupção | digressão}
  \definition{v.+compl.}{interromper (a fala de alguém)}
\end{entry}

\begin{entry}{插手}{cha1shou3}{12,4}
  \definition{v.+compl.}{envolver-se em | dar uma mão | ter (tomar) uma mão | cutucar o nariz de alguém | intrometer-se}
\end{entry}

\begin{entry}{查}{cha2}{9}[Radical 木][HSK 2]
  \definition{v.}{verificar | examinar | investigar |consultar}
  \seeref{查}{zha1}
\end{entry}

\begin{entry}{茶}{cha2}{9}[Radical 艸][HSK 1]
  \definition[杯,壶]{s.}{chá | pé (planta) de chá}
\end{entry}

\begin{entry}{刹}{cha4}{8}[Radical 刀]
  \definition{s.}{mosteiro, templo ou santuário budista | abreviação de 刹多罗 | sânscrito "ksetra"}
  \seeref{刹多罗}{cha4duo1luo2}
  \seeref{刹}{sha1}
\end{entry}

\begin{entry}{刹多罗}{cha4duo1luo2}{8,6,8}
  \definition*{s.}{Kshatara, sânscrito ``ksetra''}
\end{entry}

\begin{entry}{差}{cha4}{9}[Radical 工][HSK 1]
  \definition{adv.}{ligeiramente | comparativamente | um pouco}
  \definition{s.}{differença | dissimilaridade | engano | equívoco}
\end{entry}

\begin{entry}{差不多}{cha4bu5duo1}{9,4,6}[HSK 2]
  \definition{adj.}{mais ou menos}
  \definition{adv.}{quase perto}
\end{entry}

\begin{entry}{差点儿}{cha4dian3r5}{9,9,2}
  \definition{adv.}{por pouco | por um triz | quase}
\end{entry}

\begin{entry}{拆}{chai1}{8}[Radical 手]
  \definition{v.}{remover | tirar do seu lugar | desfazer | desmontar}
\end{entry}

\begin{entry}{单}{chan2}{8}[Radical 十]
  \definition{s.}{usado em 单于 \dpy{chan2yu2}}
  \seeref{单于}{chan2yu2}
  \seeref{单}{dan1}
  \seeref{单}{shan4}
\end{entry}

\begin{entry}{单于}{chan2yu2}{8,3}
  \definition{s.}{rei de Xiongnu (匈奴)}
  \seealsoref{匈奴}{xiong1nu2}
\end{entry}

\begin{entry}{禅}{chan2}{12}[Radical 示]
  \definition*{s.}{Zen}
  \definition{s.}{meditação (Budismo)}
  \seeref{禅}{shan4}
\end{entry}

\begin{entry}{蝉}{chan2}{14}[Radical 虫]
  \definition{s.}{cigarra}
\end{entry}

\begin{entry}{产后}{chan3hou4}{6,6}
  \definition{s.}{pós-parto}
\end{entry}

\begin{entry}{产生}{chan3sheng1}{6,5}[HSK 3]
  \definition{v.}{produzir; evoluir; emergir; provocar; vir a ser; dar origem a}
\end{entry}

\begin{entry}{铲车}{chan3che1}{11,4}
  \definition[台]{s.}{empilhadeira}
\end{entry}

\begin{entry}{长}{chang2}{4}[Radical 長][Kangxi 168][HSK 2]
  \definition{adj.}{comprido | longo}
  \seeref{长}{zhang3}
\end{entry}

\begin{entry}{长城}{chang2cheng2}{4,9}[HSK 3]
  \definition*{s.}{A Grande Muralha}
\end{entry}

\begin{entry}{长处}{chang2 chu4}{4,5}[HSK 3]
  \definition{s.}{força; boas qualidades; pontos fortes}
\end{entry}

\begin{entry}{长颈鹿}{chang2jing3lu4}{4,11,11}
  \definition[只]{s.}{girafa}
\end{entry}

\begin{entry}{长期}{chang2 qi1}{4,12}[HSK 3]
  \definition{adj.}{secular; longo prazo; longo alcance; durante um longo período de tempo}
  \definition{s.}{longo prazo}
\end{entry}

\begin{entry}{常}{chang2}{11}[Radical 巾][HSK 1]
  \definition*{s.}{sobrenome Chang}
  \definition{adv.}{muitas vezes | frequentemente}
\end{entry}

\begin{entry}{常常}{chang2chang2}{11,11}[HSK 1]
  \definition{adv.}{frequentemente | com frequência}
\end{entry}

\begin{entry}{常见}{chang2 jian4}{11,4}[HSK 2]
  \definition{adj.}{comum}
\end{entry}

\begin{entry}{常问问题}{chang2wen4wen4ti2}{11,6,6,15}
  \definition{s.}{FAQ; perguntas frequentes}
\end{entry}

\begin{entry}{常用}{chang2 yong4}{11,5}[HSK 2]
  \definition{adj.}{em uso comum}
\end{entry}

\begin{entry}{厂}{chang3}{2}[Radical 厂][HSK 3]
  \definition[家]{s.}{fábrica; moinho; planta; obra | pátio; depósito}
  \seeref{厂}{han3}
\end{entry}

\begin{entry}{场}{chang3}{6}[Radical ⼟][HSK 2]
  \definition{clas.}{para número de exames | para atividades esportivas ou recreativas}
  \definition{s.}{local grande usado para um propósito específico | cena (de uma peça) | palco}
\end{entry}

\begin{entry}{场合}{chang3he2}{6,6}[HSK 3]
  \definition[种]{s.}{ocasião; situação}
\end{entry}

\begin{entry}{场景}{chang3jing3}{6,12}
  \definition{s.}{cena | cenário | situação | contexto}
\end{entry}

\begin{entry}{场面}{chang3mian4}{6,9}
  \definition{s.}{cena | espetáculo | ocasião | situação}
\end{entry}

\begin{entry}{场所}{chang3suo3}{6,8}[HSK 3]
  \definition{s.}{lugar; sítio; arena}
\end{entry}

\begin{entry}{唱}{chang4}{11}[Radical ⼝][HSK 1]
  \definition{v.}{cantar}
\end{entry}

\begin{entry}{唱歌}{chang4ge1}{11,14}[HSK 1]
  \definition{v.+compl.}{cantar}
\end{entry}

\begin{entry}{超过}{chao1guo4}{12,6}[HSK 2]
  \definition{v.}{passar | ultrapassar (alguém ou algo) | exceder | ser mais do que | estar acima de (um padrão)}
\end{entry}

\begin{entry}{超级}{chao1ji2}{12,6}[HSK 3]
  \definition{adj.}{super}
  \definition{pref.}{``super'' | ``ultra'' | ``hiper''}
\end{entry}

\begin{entry}{超声}{chao1sheng1}{12,7}
  \definition{adj.}{ultrasônico}
  \definition{s.}{ultrasom}
\end{entry}

\begin{entry}{超市}{chao1shi4}{12,5}[HSK 2]
  \definition[家]{s.}{supermercado}
\end{entry}

\begin{entry}{巢}{chao2}{11}[Radical ⼮]
  \definition*{s.}{sobrenome Chao}
  \definition{s.}{ninho (de aves, etc.)}
\end{entry}

\begin{entry}{朝}{chao2}{12}[Radical ⽉][HSK 3]
  \definition*{s.}{sobrenome Chao}
  \definition{prep.}{para; em direção a}
  \definition{s.}{tribunal; governo | dinastia | o reino de um imperador}
  \definition{v.}{ter uma audiência com (um rei, um imperador, etc.); fazer uma peregrinação a | encarar; olhar}
  \seeref{朝}{zhao1}
\end{entry}

\begin{entry}{朝廷}{chao2ting2}{12,6}
  \definition{s.}{corte imperial | dinastia}
\end{entry}

\begin{entry}{朝鲜}{chao2xian3}{12,14}
  \definition*{s.}{Coréia do Norte}
\end{entry}

\begin{entry}{潮流}{chao2liu2}{15,10}
  \definition{s.}{tendência | onda | corrente}
\end{entry}

\begin{entry}{吵}{chao3}{7}[Radical ⼝][HSK 3]
  \definition{adj.}{barulhento; ruidoso}
  \definition{v.}{perturbar fazendo barulho; fazer barulho | discutir; brigar; disputar}
\end{entry}

\begin{entry}{吵架}{chao3jia4}{7,9}[HSK 3]
  \definition{v.+compl.}{brigar; discutir; ter uma briga}
\end{entry}

\begin{entry}{炒}{chao3}{8}[Radical ⽕]
  \definition{v.}{saltear | demitir (alguém)}
\end{entry}

\begin{entry}{车}{che1}{4}[Radical 車][Kangxi 159][HSK 1]
  \definition*{s.}{sobrenome Che}
  \definition[辆]{s.}{carro | veículo | viatura}
  \seeref{车}{ju1}
\end{entry}

\begin{entry}{车次}{che1ci4}{4,6}
  \definition{s.}{número do trem}
\end{entry}

\begin{entry}{车库}{che1ku4}{4,7}
  \definition{s.}{garagem}
\end{entry}

\begin{entry}{车辆}{che1 liang4}{4,11}[HSK 2]
  \definition{s.}{veículo | carro}
\end{entry}

\begin{entry}{车牌}{che1pai2}{4,12}
  \definition{s.}{matrícula | placa de carro}
\end{entry}

\begin{entry}{车票}{che1piao4}{4,11}[HSK 1]
  \definition{s.}{bilhete (de ônibus, trem, metrô)}
\end{entry}

\begin{entry}{车上}{che1 shang5}{4,3}[HSK 1]
  \definition{adv.}{no carro | dentro do veículo}
\end{entry}

\begin{entry}{车水马龙}{che1shui3-ma3long2}{4,4,3,5}
  \definition{expr.}{tráfego engarrafado | engarrafamento | (literalmente) ``fluxo interminável de cavalos e carruagens''}
\end{entry}

\begin{entry}{车站}{che1zhan4}{4,10}[HSK 1]
  \definition[处,个]{s.}{estação | ponto de ônibus}
\end{entry}

\begin{entry}{车主}{che1zhu3}{4,5}
  \definition{s.}{proprietário do carro}
\end{entry}

\begin{entry}{车子}{che1zi5}{4,3}
  \definition{s.}{qualquer veículo (carro, bicicleta, caminhão, etc)}
\end{entry}

\begin{entry}{撤}{che4}{15}[Radical 手]
  \definition{v.}{remover, tirar}
\end{entry}

\begin{entry}{沉}{chen2}{7}[Radical 水]
  \definition{adj.}{profundo}
  \definition{v.}{submergir | imergir | mergulhar | afundar}
\end{entry}

\begin{entry}{沉默}{chen2mo4}{7,16}
  \definition{adj.}{taciturno | não comunicativo | silencioso}
\end{entry}

\begin{entry}{衬衫}{chen4shan1}{8,8}[HSK 3]
  \definition[件]{s.}{camisa; blusa}
\end{entry}

\begin{entry}{衬衣}{chen4 yi1}{8,6}[HSK 3]
  \definition[件]{s.}{camisa}
\end{entry}

\begin{entry}{称}{chen4}{10}[Radical 禾]
  \definition{v.}{ajustar | combinar}
  \seeref{称}{cheng1}
\end{entry}

\begin{entry}{称}{cheng1}{10}[Radical 禾][HSK 2]
  \definition*{s.}{sobrenome Cheng}
  \definition{s.}{nome}
  \definition{v.}{chamar | dizer | elogiar | louvar | pesar | levantar | começar}
  \seeref{称}{chen4}
\end{entry}

\begin{entry}{称为}{cheng1 wei2}{10,4}[HSK 3]
  \definition{v.}{chamar; ser chamado; ser conhecido como}
\end{entry}

\begin{entry}{成}{cheng2}{6}[Radical ⼽][HSK 2]
  \definition*{s.}{sobrenome Cheng}
  \definition{v.}{sair-se bem | ser bem sucedido}
\end{entry}

\begin{entry}{成都}{cheng2du1}{6,10}
  \definition*{s.}{Chengdu}
\end{entry}

\begin{entry}{成功}{cheng2gong1}{6,5}[HSK 3]
  \definition{adj.}{bem-sucedido | frutífero}
  \definition[个,次]{s.}{sucesso}
  \definition{v.}{ter sucesso}
\end{entry}

\begin{entry}{成果}{cheng2guo3}{6,8}[HSK 3]
  \definition{s.}{realização; resultado}
\end{entry}

\begin{entry}{成婚}{cheng2hun1}{6,11}
  \definition{v.}{casar-se}
\end{entry}

\begin{entry}{成活}{cheng2huo2}{6,9}
  \definition{v.}{sobreviver}
\end{entry}

\begin{entry}{成吉思汗}{cheng2ji2si1han2}{6,6,9,6}
  \definition*{s.}{Genghis Khan (1162-1227), fundador e governante do Império Mongol}
\end{entry}

\begin{entry}{成绩}{cheng2ji4}{6,11}[HSK 2]
  \definition[项,个]{s.}{nota | classificação}
\end{entry}

\begin{entry}{成家}{cheng2jia1}{6,10}
  \definition{v.}{tornar-se um especialista reconhecido | estabelecer-se e casar-se (de um homem)}
\end{entry}

\begin{entry}{成就}{cheng2jiu4}{6,12}[HSK 3]
  \definition[个]{s.}{realização; sucesso}
  \definition{v.}{realizar; atingir; completar}
\end{entry}

\begin{entry}{成立}{cheng2li4}{6,5}[HSK 3]
  \definition{v.}{fundar; estabelecer; montar | ser válido; ser sustentável; reter água}
\end{entry}

\begin{entry}{成批}{cheng2pi1}{6,7}
  \definition{s.}{em lotes | a granel}
\end{entry}

\begin{entry}{成器}{cheng2qi4}{6,16}
  \definition{v.}{tornar-se uma pessoa digna de respeito | fazer algo de si mesmo}
\end{entry}

\begin{entry}{成色}{cheng2se4}{6,6}
  \definition{v.}{sair-se bem | ser bem sucedido}
\end{entry}

\begin{entry}{成熟}{cheng2shu2}{6,15}[HSK 3]
  \definition{adj./s.}{maduro; totalmente crescido}
  \definition{v.}{amadurecer; estar maduro; estar totalmente crescido}
\end{entry}

\begin{entry}{成为}{cheng2wei2}{6,4}[HSK 2]
  \definition{s.}{tornar-se | transformar-se em}
\end{entry}

\begin{entry}{成员}{cheng2yuan2}{6,7}[HSK 3]
  \definition[个]{s.}{membro}
\end{entry}

\begin{entry}{成长}{cheng2zhang3}{6,4}[HSK 3]
  \definition{v.}{crescer; amadurecer; amadurar}
\end{entry}

\begin{entry}{承认}{cheng2ren4}{8,4}
  \definition{s.}{reconhecimento (diplomático, artístico, etc.)}
  \definition{v.}{admitir | conceder | reconhecer}
\end{entry}

\begin{entry}{诚实}{cheng2shi2}{8,8}
  \definition{adj.}{honesto}
\end{entry}

\begin{entry}{诚实地}{cheng2shi2 di4}{8,8,6}
  \definition{adv.}{honestamente}
\end{entry}

\begin{entry}{城}{cheng2}{9}[Radical 土][HSK 3]
  \definition*{s.}{sobrenome Cheng}
  \definition[座,道,个]{s.}{muralha da cidade; muro | cidade}
\end{entry}

\begin{entry}{城堡}{cheng2bao3}{9,12}
  \definition*{s.}{castelo | torre (peça de xadrez)}
\end{entry}

\begin{entry}{城度}{cheng2du4}{9,9}[HSK 3]
  \definition*{s.}{Cidade}
\end{entry}

\begin{entry}{城市}{cheng2shi4}{9,5}[HSK 3]
  \definition[个,座]{s.}{cidade}
\end{entry}

\begin{entry}{乘客}{cheng2ke4}{10,9}
  \definition{s.}{passageiro}
\end{entry}

\begin{entry}{乘客数}{cheng2ke4 shu4}{10,9,13}
  \definition{s.}{número de passageiros}
\end{entry}

\begin{entry}{惩处}{cheng2chu3}{12,5}
  \definition{v.}{administrar justiça | punir}
\end{entry}

\begin{entry}{惩罚}{cheng2fa2}{12,9}
  \definition{v.}{punir | penalizar}
\end{entry}

\begin{entry}{程控}{cheng2kong4}{12,11}
  \definition{s.}{programado | sob controle automático}
\end{entry}

\begin{entry}{程序}{cheng2xu4}{12,7}
  \definition{s.}{procedimento | sequência | ordem | programa de computador}
\end{entry}

\begin{entry}{程序库}{cheng2xu4ku4}{12,7,7}
  \definition{s.}{biblioteca de funções e procedimentos para programas de computador}
\end{entry}

\begin{entry}{程序设计}{cheng2xu4she4ji4}{12,7,6,4}
  \definition{s.}{programação de computadores}
\end{entry}

\begin{entry}{橙色}{cheng2 se4}{16,6}
  \definition{s.}{cor de laranja}
\end{entry}

\begin{entry}{橙汁}{cheng2zhi1}{16,5}
  \definition[瓶,杯,罐,盒]{s.}{suco de laranja}
  \seealsoref{橘子汁}{ju2zi5zhi1}
  \seealsoref{柳橙汁}{liu3cheng2zhi1}
\end{entry}

\begin{entry}{吃}{chi1}{6}[Radical ⼝][HSK 1]
  \definition{v.}{comer | consumir | comer em (uma cafeteria, etc.) | erradicar | destruir | absorver}
\end{entry}

\begin{entry}{吃饭}{chi1fan4}{6,7}[HSK 1]
  \definition{v.+compl.}{comer | ter (comer) uma refeição | manter vivo | ganhar a vida}
\end{entry}

\begin{entry}{吃屎}{chi1 shi3}{6,9}
  \definition{expr.}{Coma merda!}
\end{entry}

\begin{entry}{池}{chi2}{6}[Radical 水]
  \definition*{s.}{sobrenome Chi}
  \definition{s.}{lagoa | reservatório | fosso}
\end{entry}

\begin{entry}{迟到}{chi2dao4}{7,8}
  \definition{v.}{chegar atrasado | tardar}
\end{entry}

\begin{entry}{持续}{chi2xu4}{9,11}[HSK 3]
  \definition{v.}{durar; continuar; sustentar}
\end{entry}

\begin{entry}{斥骂}{chi4ma4}{5,9}
  \definition{v.}{repreender}
\end{entry}

\begin{entry}{充满}{chong1man3}{6,13}[HSK 3]
  \definition{v.}{preencher | encher-se de; transbordar de; permear-se de}
\end{entry}

\begin{entry}{冲}{chong1}{6}[Radical ⼎]
  \definition{s.}{via pública}
  \definition{v.}{(água) correr contra | misturar com água | infundir | enxaguar | dar a descarga | revelar (um filme) | subir no ar | chocar-se | colidir com | ir em frente | apressar-se}
  \seeref{冲}{chong4}
\end{entry}

\begin{entry}{冲锋}{chong1feng1}{6,12}
  \definition{v.}{cobrar | tomar de assalto}
\end{entry}

\begin{entry}{冲浪}{chong1lang4}{6,10}
  \definition{s.}{surfe}
  \definition{v.}{surfar}
\end{entry}

\begin{entry}{冲突}{chong1tu1}{6,9}
  \definition{s.}{conflito | choque de forças opostas | colisão (de interesses)}
\end{entry}

\begin{entry}{憧憬}{chong1jing3}{15,15}
  \definition{v.}{ansiar por | esperar por}
\end{entry}

\begin{entry}{重}{chong2}{9}[Radical ⾥]
  \definition*{s.}{sobrenome Chong}
  \definition{adv.}{novamente; mais uma vez}
  \definition{clas.}{para camadas}
  \definition{v.}{repetir; duplicar}
  \seeref{重}{zhong4}
\end{entry}

\begin{entry}{重重}{chong2chong2}{9,9}
  \definition{adv.}{camada após camada | um após o outro}
  \seeref{重重}{zhong4zhong4}
\end{entry}

\begin{entry}{重点}{chong2dian3}{9,9}
  \definition{adj./adv./s.}{nota-chave | ponto-chave | ponto focal | ênfase}
  \seeref{重点}{zhong4dian3}
\end{entry}

\begin{entry}{重逢}{chong2feng2}{9,10}
  \definition{s.}{reunião}
  \definition{v.}{encontrar-se novamente | reunir-se}
\end{entry}

\begin{entry}{重复}{chong2fu4}{9,9}[HSK 2]
  \definition{v.}{repetir | iterar | duplicar | reduplicar | fazer algo de novo}
\end{entry}

\begin{entry}{重新}{chong2xin1}{9,13}[HSK 2]
  \definition{adv.}{de novo | novamente}
\end{entry}

\begin{entry}{重阳节}{chong2yang2jie2}{9,6,5}
  \definition*{s.}{Festa do Duplo Nove, Festival Yang, dia de subir aos lugares mais altos para evitar calamidades e dia do culto aos antepassados (9º dia do nono mês lunar)}
\end{entry}

\begin{entry}{崇}{chong2}{11}[Radical ⼭]
  \definition*{s.}{sobrenome Chong}
  \definition{adj.}{alto | sublime | elevado}
  \definition{v.}{estimar | adorar}
\end{entry}

\begin{entry}{宠物}{chong3wu4}{8,8}
  \definition{s.}{animal de estimação}
\end{entry}

\begin{entry}{冲}{chong4}{6}[Radical ⼎]
  \definition{adj.}{poderoso | vigoroso | pungente}
  \definition{adv.}{em direção | em vista de}
  \seeref{冲}{chong1}
\end{entry}

\begin{entry}{酬劳}{chou2lao2}{13,7}
  \definition{s.}{recompensa}
\end{entry}

\begin{entry}{臭}{chou4}{10}[Radical ⾃]
  \definition{adj.}{fétido | repulsivo | repugnante | malcheiroso}
  \definition{s.}{fedor}
  \definition{v.}{feder}
  \seeref{臭}{xiu4}
\end{entry}

\begin{entry}{臭气}{chou4qi4}{10,4}
  \definition{s.}{fedor}
\end{entry}

\begin{entry}{出}{chu1}{5}[Radical ⼐][HSK 1]
  \definition{clas.}{para dramas, peças, óperas, etc.}
  \definition{v.}{sair | ir para fora | vir para fora}
\end{entry}

\begin{entry}{出版}{chu1ban3}{5,8}
  \definition{v.}{publicar | editar}
\end{entry}

\begin{entry}{出版社}{chu1ban3she4}{5,8,7}
  \definition{s.}{editora}
\end{entry}

\begin{entry}{出差}{chu1chai1}{5,9}
  \definition{v.+compl.}{fazer uma viagem oficial ou de negócios}
\end{entry}

\begin{entry}{出发}{chu1fa1}{5,5}[HSK 2]
  \definition{v.}{partir | começar (uma jornada)}
\end{entry}

\begin{entry}{出国}{chu1 guo2}{5,8}[HSK 2]
  \definition{v.+compl.}{ir para o exterior | deixar a terra natal}
\end{entry}

\begin{entry}{出汗}{chu1han4}{5,6}
  \definition{v.}{transpirar | suar}
\end{entry}

\begin{entry}{出击}{chu1ji1}{5,5}
  \definition{v.}{atacar}
\end{entry}

\begin{entry}{出口}{chu1kou3}{5,3}[HSK 2]
  \definition[个]{s.}{exportação}
  \definition{v.+compl.}{exportar}
\end{entry}

\begin{entry}{出来}{chu1 lai2}{5,7}[HSK 1]
  \definition{v.}{sair | vir para fora (para a minha localização)}
\end{entry}

\begin{entry}{出门}{chu1 men2}{5,3}[HSK 2]
  \definition{v.+compl.}{sair | sair de casa | estar longe de casa | fazer uma viagem | casar}
\end{entry}

\begin{entry}{出去}{chu1 qu4}{5,5}[HSK 1]
  \definition{v.}{sair | ir para fora (a partir da minha localização)}
\end{entry}

\begin{entry}{出生}{chu1sheng1}{5,5}[HSK 2]
  \definition{v.}{nascer}
\end{entry}

\begin{entry}{出现}{chu1xian4}{5,8}[HSK 2]
  \definition{v.}{aparecer | surgir | emergir | crescer}
\end{entry}

\begin{entry}{出行}{chu1xing2}{5,6}
  \definition{v.}{sair para algum lugar (viagem relativamente curta) | partir em uma viagem (viagem mais longa)}
\end{entry}

\begin{entry}{出院}{chu1 yuan4}{5,9}[HSK 2]
  \definition{v.}{deixar o hospital | estar fora do hospital | ter alta do hospital}
\end{entry}

\begin{entry}{出站}{chu1 zhan4}{5,10}
  \definition{s.}{saída da estação}
\end{entry}

\begin{entry}{出租}{chu1 zu1}{5,10}[HSK 2]
  \definition{v.}{alugar | arrendar}
\end{entry}

\begin{entry}{出租车}{chu1zu1che1}{5,10,4}[HSK 2]
  \definition{s.}{táxi}
  \seealsoref{出租汽车}{chu1zu1qi4che1}
\end{entry}

\begin{entry}{出租汽车}{chu1zu1qi4che1}{5,10,7,4}
  \definition[辆]{s.}{táxi}
  \seealsoref{出租车}{chu1zu1che1}
\end{entry}

\begin{entry}{出租司机}{chu1zu1si1ji1}{5,10,5,6}
  \definition{s.}{motorista de táxi}
\end{entry}

\begin{entry}{初}{chu1}{7}[Radical 衣][HSK 3]
  \definition*{s.}{sobrenome Chu}
  \definition{adj.}{primeiro (em ordem) | elementar; rudimentar | original}
  \definition{adv.}{pela primeira vez}
  \definition{pref.}{anexado aos numerais de um a dez para indicar ordem (primeiro ao décimo)}
  \definition{s.}{no início de; na primeira parte de | o estágio júnior (pleno; sênior)}
\end{entry}

\begin{entry}{初步}{chu1bu4}{7,7}[HSK 3]
  \definition{adj.}{inicial; preliminar}
\end{entry}

\begin{entry}{初级}{chu1ji2}{7,6}[HSK 3]
  \definition{adj.}{elementar; primário; júnior; inicial}
\end{entry}

\begin{entry}{初心}{chu1xin1}{7,4}
  \definition{s.}{intenção original de alguém, aspiração, etc. | (budismo) ``mente do iniciante'' (ter a mente aberta quando estudando um assunto como um iniciante no assunto teria)}
\end{entry}

\begin{entry}{初中}{chu1 zhong1}{7,4}[HSK 3]
  \definition[所,个]{s.}{ensino médio; ensino fundamental}
\end{entry}

\begin{entry}{除非}{chu2fei1}{9,8}
  \definition{conj.}{a menos que | somente se}
\end{entry}

\begin{entry}{除了}{chu2le5}{9,2}[HSK 3]
  \definition{prep.}{exceto; à parte | além disso; além de | ou \dots ou \dots}
\end{entry}

\begin{entry}{厨房}{chu2fang2}{12,8}
  \definition[间]{s.}{cozinha}
\end{entry}

\begin{entry}{处}{chu3}{5}[Radical ⼡]
  \definition{v.}{residir | viver | habitar | estar dentro | estar situado em | ficar | se dar bem com | estar em uma posição de | lidar com | disciplinar | punir}
  \seeref{处}{chu4}
\end{entry}

\begin{entry}{处罚}{chu3fa2}{5,9}
  \definition{v.}{penalizar | punir}
\end{entry}

\begin{entry}{处理}{chu3li3}{5,11}[HSK 3]
  \definition{s.}{manuseio; descarte}
  \definition{v.}{lidar com; dispor de | resolver; punir; lidar | vender a preços reduzidos; liquidar | lidar com; processar}
\end{entry}

\begin{entry}{处}{chu4}{5}[Radical ⼡]
  \definition{clas.}{para locais ou itens de danos: lugar, local}
  \definition{s.}{local | localização | lugar | ponto | escritório | departamento}
  \seeref{处}{chu3}
\end{entry}

\begin{entry}{处处}{chu4chu4}{5,5}
  \definition{adv.}{em todos os lugares | em todos os aspectos}
\end{entry}

\begin{entry}{畜}{chu4}{10}[Radical ⽥]
  \definition{s.}{gado | animal domesticado | animal doméstico}
  \seeref{畜}{xu4}
\end{entry}

\begin{entry}{穿}{chuan1}{9}[Radical ⽳][HSK 1]
  \definition{v.}{vestir}
\end{entry}

\begin{entry}{传}{chuan2}{6}[Radical 人][HSK 3]
  \definition{v.}{passar; passar adiante | passar adiante; legar; passar de \dots para \dots | transmitir (conhecimento, habilidade, etc.); comunicar; ensinar | espalhar; propagar | transmitir; conduzir; transferir | transmitir; expressar |convocar | infectar; ser contagioso}
  \seeref{传}{zhuan4}
\end{entry}

\begin{entry}{传播}{chuan2bo1}{6,15}[HSK 3]
  \definition{v.}{espalhar; difundir; propagar; disseminar}
\end{entry}

\begin{entry}{传承}{chuan2cheng2}{6,8}
  \definition{s.}{herança | tradição continuada}
  \definition{v.}{transmitir (para as gerações futuras) | passar adiante (desde os tempos antigos)}
\end{entry}

\begin{entry}{传给}{chuan2gei3}{6,9}
  \definition{v.}{passar para | transferir para | entregar a}
\end{entry}

\begin{entry}{传来}{chuan2 lai2}{6,7}[HSK 3]
  \definition{v.}{(um som) passar | (notícias) chegar}
\end{entry}

\begin{entry}{传说}{chuan2shuo1}{6,9}[HSK 3]
  \definition{s.}{lenda | conto popular | folclore}
  \definition{v.}{dizer que; ser dito; passar de boca em boca}
\end{entry}

\begin{entry}{传统}{chuan2tong3}{6,9}
  \definition{adj.}{tradicional | convencional}
  \definition[个]{s.}{tradição | convenção}
\end{entry}

\begin{entry}{传真}{chuan2zhen1}{6,10}
  \definition{s.}{fax, facsímile}
\end{entry}

\begin{entry}{船}{chuan2}{11}[Radical ⾈][HSK 2]
  \definition[条,艘,只]{s.}{barco | navio}
\end{entry}

\begin{entry}{创作}{chuan4zuo4}{6,7}[HSK 3]
  \definition[个]{s.}{criação; trabalho criativo}
  \definition{v.}{escrever; criar; produzir; compor}
\end{entry}

\begin{entry}{窗帘}{chuang1lian2}{12,8}
  \definition{s.}{cortina}
\end{entry}

\begin{entry}{床}{chuang2}{7}[Radical ⼴][HSK 1]
  \definition{clas.}{para camas}
  \definition[张]{s.}{cama}
\end{entry}

\begin{entry}{创新}{chuang4xin1}{6,13}[HSK 3]
  \definition[个,种,次]{s.}{inovação}
  \definition{v.}{trazer novas ideias; inovar; abrir novos caminhos; criar algo novo}
\end{entry}

\begin{entry}{创业}{chuang4ye4}{6,5}[HSK 3]
  \definition{s.}{empreendedorismo}
  \definition{v.}{começar um empreendimento; iniciar um negócio, uma empresa | esculpir}
\end{entry}

\begin{entry}{创意}{chuang4yi4}{6,13}
  \definition{adj.}{criativo}
  \definition{s.}{criatividade}
\end{entry}

\begin{entry}{创造}{chuang4zao4}{6,10}[HSK 3]
  \definition{s.}{criação; inovação}
  \definition{v.}{criar; produzir; trazer à tona}
\end{entry}

\begin{entry}{吹}{chui1}{7}[Radical 口][HSK 2]
  \definition{v.}{soprar | tocar (instrumentos de sopro) | bajular |  louvar aos céus | separar (casal)  | fracassar}
\end{entry}

\begin{entry}{吹牛}{chui1niu2}{7,4}
  \definition{v.+compl.}{ogulhar-se | gabar-se | destacar-se}
\end{entry}

\begin{entry}{锤}{chui2}{13}[Radical 金]
  \definition{s.}{martelo | marreta}
  \definition{s.}{pesos (por exemplo, de uma balança)}
  \definition{v.}{marterlar para dar forma | atacar com um martelo}
\end{entry}

\begin{entry}{春}{chun1}{9}[Radical 日]
  \definition*{s.}{sobrenome Chun}
  \definition{s.}{primavera | amor | luxúria | vida | vitalidade}
\end{entry}

\begin{entry}{春节}{chun1 jie2}{9,5}[HSK 2]
  \definition*{s.}{Festival da Primavera (Ano Novo Chinês)}
\end{entry}

\begin{entry}{春天}{chun1 tian1}{9,4}
  \definition[个]{s.}{primavera}
\end{entry}

\begin{entry}{纯真}{chun2zhen1}{7,10}
  \definition{adj.}{inocente e não afetado | puro e não adulterado}
  \definition{s.}{inocência}
\end{entry}

\begin{entry}{唇}{chun2}{10}[Radical ⼝]
  \definition{s.}{lábios}
\end{entry}

\begin{entry}{绰号}{chuo4hao4}{11,5}
  \definition{s.}{apelido}
\end{entry}

\begin{entry}{词}{ci2}{7}[Radical 言][HSK 2]
  \definition[个,组]{s.}{discurso | declaração | linhas de jogo | um tipo de poesia clássica chinesa, originária da Dinastia Tang e totalmente desenvolvida na Dinastia Song | palavra  | termo}
\end{entry}

\begin{entry}{词典}{ci2dian3}{7,8}[HSK 2]
  \definition[部,本]{s.}{dicionário}
  \seealsoref{字典}{zi4dian3}
\end{entry}

\begin{entry}{词语}{ci2yu3}{7,9}[HSK 2]
  \definition{s.}{palavra (termo geral, incluindo desdemonossilábicas até frases curtas) | termo (por exemplo, termo técnico) | expressão}
\end{entry}

\begin{entry}{瓷}{ci2}{10}[Radical ⽡]
  \definition{s.}{artigos de porcelana}
\end{entry}

\begin{entry}{辞典}{ci2dian3}{13,8}
  \variantof{词典}
\end{entry}

\begin{entry}{磁带}{ci2dai4}{14,9}
  \definition[盘,盒]{s.}{cassete | fita magnética}
\end{entry}

\begin{entry}{磁盘}{ci2pan2}{14,11}
  \definition{s.}{disquete}
\end{entry}

\begin{entry}{磁铁}{ci2tie3}{14,10}
  \definition{s.}{imã | magneto}
  \seealsoref{吸铁石}{xi1tie3shi2}
\end{entry}

\begin{entry}{次}{ci4}{6}[Radical ⽋][HSK 1]
  \definition{clas.}{para frequência (número de vezes)}
\end{entry}

\begin{entry}{刺}{ci4}{8}[Radical 刀]
  \definition{s.}{espinho | picada}
  \definition{v.}{picar | perfurar | esfaquear | assassinar}
\end{entry}

\begin{entry}{刺猬}{ci4wei5}{8,12}
  \definition{s.}{porco-espinho | ouriço}
\end{entry}

\begin{entry}{匆匆}{cong1cong1}{5,5}
  \definition{adv.}{apressadamente}
\end{entry}

\begin{entry}{葱}{cong1}{12}[Radical 艸]
  \definition{s.}{cebolinha}
\end{entry}

\begin{entry}{聪慧}{cong1hui4}{15,15}
  \definition{adj.}{inteligente | brilhante}
\end{entry}

\begin{entry}{聪明}{cong1ming5}{15,8}
  \definition{adj.}{inteligente | brilhante | esperto}
\end{entry}

\begin{entry}{从}{cong2}{4}[Radical ⼈][HSK 1]
  \definition*{s.}{sobrenome Cong}
  \definition{prep.}{de | desde | a partir de}
\end{entry}

\begin{entry}{从不}{cong2bu4}{4,4}
  \definition{adv.}{nunca}
\end{entry}

\begin{entry}{从而}{cong2'er2}{4,6}
  \definition{conj.}{assim | desse modo}
\end{entry}

\begin{entry}{从来}{cong2lai2}{4,7}[HSK 3]
  \definition{adv.}{sempre; o tempo todo; em todos os momentos}
\end{entry}

\begin{entry}{从前}{cong2qian2}{4,9}[HSK 3]
  \definition{s.}{antes; antigamente; no passado | era uma vez; há muito tempo atrás}
\end{entry}

\begin{entry}{从事}{cong2shi4}{4,8}[HSK 3]
  \definition{v.}{trabalhar; empreender; empenhar-se em; envolver-se em | lidar com; manusear}
\end{entry}

\begin{entry}{从未}{cong2wei4}{4,5}
  \definition{adv.}{nunca}
\end{entry}

\begin{entry}{从小}{cong2 xiao3}{4,3}[HSK 2]
  \definition{adv.}{desde a infância | desde muito jovem | quando criança}
\end{entry}

\begin{entry}{粗糙}{cu1cao1}{11,16}
  \definition{adj.}{áspero | grosseiro}
\end{entry}

\begin{entry}{粗心}{cu1xin1}{11,4}
  \definition{adj.}{descuido}
\end{entry}

\begin{entry}{粗心地做}{cu1xin1 di4 zuo4}{11,4,6,11}
  \definition{adj.}{feito descuidadamente}
\end{entry}

\begin{entry}{酢}{cu4}{12}[Radical 酉]
  \variantof{醋}
\end{entry}

\begin{entry}{醋}{cu4}{15}[Radical ⾣]
  \definition{s.}{vinagre}
\end{entry}

\begin{entry}{窾}{cuan4}{17}[Radical 穴]
  \definition{v.}{esconder}
  \seeref{窾}{kuan3}
\end{entry}

\begin{entry}{村}{cun1}{7}[Radical ⽊][HSK 3]
  \definition{adj.}{rústico; grosseiro}
  \definition{s.}{aldeia; vila}
\end{entry}

\begin{entry}{存}{cun2}{6}[Radical 子][HSK 3]
  \definition{v.}{existir; viver; sobreviver | armazenar; manter | acumular; coletar | depositar | sair com; verificar |reservar; reter | permanecer em equilíbrio; estar em estoque | estimar; abrigar}
\end{entry}

\begin{entry}{存在}{cun2zai4}{6,6}[HSK 3]
  \definition{s.}{existência; ser; ente}
  \definition{v.}{existir; ser}
\end{entry}

\begin{entry}{搓}{cuo1}{12}[Radical 手]
  \definition{s.}{torção}
  \definition{v.}{esfregar ou rolar entre as mãos ou dedos | torcer}
\end{entry}

\begin{entry}{挫折}{cuo4zhe2}{10,7}
  \definition{s.}{revés | reverso | derrota | frustração | decepção}
  \definition{v.}{frustrar | desencorajar | subjugar}
\end{entry}

\begin{entry}{错}{cuo4}{13}[Radical 金][HSK 1]
  \definition*{s.}{sobrenome Cuo}
  \definition{adj.}{errado | enganado}
\end{entry}

\begin{entry}{错误}{cuo4wu4}{13,9}[HSK 3]
  \definition{adj.}{equivocado; errado; errôneo}
  \definition[个,次]{s.}{engano; erro; erro grosseiro; falha}
\end{entry}

%%%%% EOF %%%%%

