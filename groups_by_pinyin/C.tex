%%%
%%% C
%%%

\section*{C}\addcontentsline{toc}{section}{C}

\begin{entry}{擦}{ca1}{17}{⼿}[HSK 4]
  \definition{v.}{enxugar; esfregar; apagar; limpar; limpar esfregando com um pano, toalha de mão, etc. | espalhar sobre; colocar sobre | passar raspando | ralar (em pedaços); ralar frutas em um ralador para fazer fios finos}
\end{entry}

\begin{entry}{擦拭}{ca1shi4}{17,9}{⼿、⼿}
  \definition{v.}{limpar com um pano}
\end{entry}

\begin{entry}{猜}{cai1}{11}{⽝}
  \definition{v.}{advinhar}
\end{entry}

\begin{entry}{才}{cai2}{3}{⼿}[HSK 2,4]
  \definition*{s.}{sobrenome Cai}
  \definition{adv.}{indica que algo aconteceu há pouco tempo, agora mesmo | indica que algo acontece ou termina tarde | indica que algo só acontece sob certas condições, ou por um motivo ou propósito específico, seguido do que acontece depois, geralmente é precedida por palavras como “somente”, “deve”, “porque” ou “devido a” | em comparação, indica uma pequena quantidade, poucas ocorrências, pouca habilidade, etc.; meramente | indica ênfase no que está sendo dito, e o caractere “呢” é frequentemente usado no final da frase}
  \definition{conj.}{apenas quando}
  \definition{s.}{capacidade; talento; dom | pessoa capacitada}
  \seealsoref{呢}{ne5}
\end{entry}

\begin{entry}{才华}{cai2hua2}{3,6}{⼿、⼗}
  \definition[份]{s.}{talento}
\end{entry}

\begin{entry}{才略}{cai2lve4}{3,11}{⼿、⽥}
  \definition{s.}{habilidade e sagacidade}
\end{entry}

\begin{entry}{才能}{cai2 neng2}{3,10}{⼿、⾁}[HSK 3]
  \definition[间]{s.}{talento | habilidade | dom | capacidade}
\end{entry}

\begin{entry}{材料}{cai2liao4}{7,10}{⽊、⽃}[HSK 4]
  \definition[份,个,种]{s.}{material; algo para fazer um produto acabado | material (figura de linguagem) | dados; material para estudo, pesquisa, etc.; conteúdo de uma obra}
\end{entry}

\begin{entry}{财产}{cai2chan3}{7,6}{⾙、⼇}[HSK 4]
  \definition{s.}{ativos; propriedade; pertences; refere-se à posse de riqueza material, como dinheiro, bens, casas, terras, etc.}
\end{entry}

\begin{entry}{财富}{cai2fu4}{7,12}{⾙、⼧}[HSK 4]
  \definition{s.}{riqueza; fortuna}
\end{entry}

\begin{entry}{裁}{cai2}{12}{⾐}
  \definition{s.}{decisão | julgamento}
  \definition{v.}{recortar (tecido de uma roupa) | cortar | aparar | reduzir | diminuir | cortar pessoal de uma equipe}
\end{entry}

\begin{entry}{采访}{cai3fang3}{8,6}{⾤、⾔}[HSK 4]
  \definition{s.}{cobertura; entrevista; coleta de notícias; entrevistas, pesquisas, gravações de áudio e vídeo, etc., com o objetivo de coletar os materiais necessários}
  \definition{v.}{cobrir; entrevistar; reunir novas informações}
\end{entry}

\begin{entry}{采取}{cai3qu3}{8,8}{⾤、⼜}[HSK 3]
  \definition{v.}{adotar | reunir | coletar | tomar | assumir}
\end{entry}

\begin{entry}{采用}{cai3 yong4}{8,5}{⾤、⽤}[HSK 3]
  \definition{v.}{selecionar e usar | adotar}
\end{entry}

\begin{entry}{彩虹}{cai3hong2}{11,9}{⼺、⾍}
  \definition[道]{s.}{arco-íris}
\end{entry}

\begin{entry}{彩色}{cai3 se4}{11,6}{⼺、⾊}[HSK 3]
  \definition{s.}{multicolorido; cor}
\end{entry}

\begin{entry}{菜}{cai4}{11}{⾋}[HSK 1]
  \definition[棵]{s.}{hortaliça | verdura | legume}
  \definition[样,道,盘]{s.}{prato (tipo de alimento) | o tipo (de alguém) | (características de alguém, etc.) fraco, pobre}
\end{entry}

\begin{entry}{菜单}{cai4dan1}{11,8}{⾋、⼗}[HSK 2]
  \definition[份,张]{s.}{menu | cardápio}
\end{entry}

\begin{entry}{菜刀}{cai4dao1}{11,2}{⾋、⼑}
  \definition[把]{s.}{faca de vegetais | faca de cozinha | cutelo}
\end{entry}

\begin{entry}{参观}{can1guan1}{8,6}{⼛、⾒}[HSK 2]
  \definition{v.}{visitar}
\end{entry}

\begin{entry}{参加}{can1jia1}{8,5}{⼛、⼒}[HSK 2]
  \definition{v.}{participar de | tomar parte em | assistir}
\end{entry}

\begin{entry}{参考}{can1kao3}{8,6}{⼛、⽼}[HSK 4]
  \definition{v.}{consultar; referir-se a; acessar informações relevantes para estudo ou pesquisa | consultar; referir-se a; lidar com coisas, observar, ler, aprender e usar materiais relevantes}
\end{entry}

\begin{entry}{参与}{can1yu4}{8,3}{⼛、⼀}[HSK 4]
  \definition{v.}{participar de; tomar parte em; ter uma mão em; envolver-se em; participar (no planejamento, discussão e condução dos assuntos)}
\end{entry}

\begin{entry}{餐厅}{can1ting1}{16,4}{⾷、⼚}
  \definition[家]{s.}{restaurante}
  \definition[间]{s.}{sala de jantar}
\end{entry}

\begin{entry}{残疾人}{can2ji2ren2}{9,10,2}{⽍、⽧、⼈}
  \definition{s.}{pessoa com deficiência}
\end{entry}

\begin{entry}{残酷}{can2ku4}{9,14}{⽍、⾣}
  \definition{adj.}{cruel}
  \definition{s.}{crueldade}
\end{entry}

\begin{entry}{蚕纸}{can2zhi3}{10,7}{⾍、⽷}
  \definition{s.}{papel onde o bicho-da-seda põe seus ovos}
\end{entry}

\begin{entry}{惨}{can3}{11}{⽕}
  \definition{adj.}{miserável | cruel | desumano | desastroso | trágico | sombrio}
\end{entry}

\begin{entry}{舱}{cang1}{10}{⾈}
  \definition{s.}{cabine | porão (de carga) de um navio ou avião}
\end{entry}

\begin{entry}{操场}{cao1chang3}{16,6}{⼿、⼟}[HSK 4]
  \definition[个]{s.}{\emph{playground}; campo esportivo; locais para exercícios físicos ou exercícios militares}
\end{entry}

\begin{entry}{操心}{cao1xin1}{16,4}{⼿、⼼}
  \definition{v.+compl.}{preocupar-se com}
\end{entry}

\begin{entry}{操作}{cao1zuo4}{16,7}{⼿、⼈}[HSK 4]
  \definition{s.}{operação}
  \definition{v.}{operar; seguir os requisitos e procedimentos prescritos| implementar; realizar; executar; refere-se à implementação concreta (planos, medidas, etc.)}
\end{entry}

\begin{entry}{槽}{cao2}{15}{⽊}
  \definition{s.}{calha | canal | sulco | manjedoura}
\end{entry}

\begin{entry}{草}{cao3}{9}{⾋}[HSK 2]
  \definition[棵,撮,株,根]{s.}{erva | grama}
\end{entry}

\begin{entry}{草地}{cao3 di4}{9,6}{⾋、⼟}[HSK 2]
  \definition[片]{s.}{relva | pastagem}
\end{entry}

\begin{entry}{草莓}{cao3mei2}{9,10}{⾋、⾋}
  \definition[颗]{s.}{morango}
\end{entry}

\begin{entry}{草纸}{cao3zhi3}{9,7}{⾋、⽷}
  \definition{s.}{papel pardo | pergaminho | papel de palha áspero | papel higiênico}
\end{entry}

\begin{entry}{肏}{cao4}{8}{⼊}
  \definition{v.}{(vulgar) foder}
\end{entry}

\begin{entry}{厕所}{ce4suo3}{8,8}{⼚、⼾}
  \definition[间,处]{s.}{lavatório | \emph{toilette}}
\end{entry}

\begin{entry}{厕纸}{ce4zhi3}{8,7}{⼚、⽷}
  \definition{s.}{papel higiênico}
\end{entry}

\begin{entry}{测}{ce4}{9}{⽔}[HSK 4]
  \definition{v.}{pesquisar; sondar; medir | conjecturar; inferir}
\end{entry}

\begin{entry}{测量}{ce4liang2}{9,12}{⽔、⾥}[HSK 4]
  \definition{v.}{aferir; pesquisar; medir; determinar valores relevantes para espaço, tempo, temperatura, velocidade, função, etc.}
\end{entry}

\begin{entry}{测试}{ce4 shi4}{9,8}{⽔、⾔}[HSK 4]
  \definition[个]{s.}{exame; teste; medição do conhecimento humano, das habilidades ou do funcionamento de máquinas, ferramentas ou instrumentos}
  \definition{v.}{examinar | testar, medição do desempenho e da precisão de máquinas, instrumentos, aparelhos, etc.}
\end{entry}

\begin{entry}{策划}{ce4hua4}{12,6}{⽵、⼑}
  \definition{s.}{planejador | produtor | plano}
  \definition{v.}{esquematizar | engenhar | planejar}
\end{entry}

\begin{entry}{层}{ceng2}{7}{⼫}[HSK 2]
  \definition{clas.}{para andar, piso}
\end{entry}

\begin{entry}{层层}{ceng2ceng2}{7,7}{⼫、⼫}
  \definition{s.}{camada sobre camada}
\end{entry}

\begin{entry}{层次}{ceng2ci4}{7,6}{⼫、⽋}
  \definition{s.}{camada | nível | graduação | arranjo de ideias}
\end{entry}

\begin{entry}{曾}{ceng2}{12}{⽈}[HSK 4]
  \definition{adv.}{uma vez; antigamente; há algum tempo; usado para indicar ação ou estado passado}
  \seeref{曾}{zeng1}
\end{entry}

\begin{entry}{曾经}{ceng2jing1}{12,8}{⽈、⽷}[HSK 3]
  \definition{adv.}{uma vez | antes | costumava | no passado}
\end{entry}

\begin{entry}{插话}{cha1hua4}{12,8}{⼿、⾔}
  \definition{s.}{interrupção | digressão}
  \definition{v.+compl.}{interromper (a fala de alguém)}
\end{entry}

\begin{entry}{插手}{cha1shou3}{12,4}{⼿、⼿}
  \definition{v.+compl.}{envolver-se em | dar uma mão | ter (tomar) uma mão | cutucar o nariz de alguém | intrometer-se}
\end{entry}

\begin{entry}{查}{cha2}{9}{⽊}[HSK 2]
  \definition{v.}{verificar | examinar | investigar |consultar}
  \seeref{查}{zha1}
\end{entry}

\begin{entry}{茶}{cha2}{9}{⾋}[HSK 1]
  \definition[杯,壶]{s.}{chá | pé (planta) de chá}
\end{entry}

\begin{entry}{茶叶}{cha2 ye4}{9,5}{⾋、⼝}[HSK 4]
  \definition[盒,罐,包,片]{s.}{chá; folhas de chá; as folhas jovens da planta do chá que são processadas para produzir bebidas}
\end{entry}

\begin{entry}{刹}{cha4}{8}{⼑}
  \definition{s.}{mosteiro, templo ou santuário budista | abreviação de 刹多罗 | sânscrito ``ksetra''}
  \seeref{刹多罗}{cha4duo1luo2}
  \seeref{刹}{sha1}
\end{entry}

\begin{entry}{刹多罗}{cha4duo1luo2}{8,6,8}{⼑、⼣、⽹}
  \definition*{s.}{Kshatara, sânscrito ``ksetra''}
\end{entry}

\begin{entry}{差}{cha4}{9}{⼯}[HSK 1]
  \definition{adv.}{ligeiramente | comparativamente | um pouco}
  \definition{s.}{differença | dissimilaridade | engano | equívoco}
\end{entry}

\begin{entry}{差不多}{cha4bu5duo1}{9,4,6}{⼯、⼀、⼣}[HSK 2]
  \definition{adj.}{mais ou menos}
  \definition{adv.}{quase perto}
\end{entry}

\begin{entry}{差点儿}{cha4dian3r5}{9,9,2}{⼯、⽕、⼉}
  \definition{adv.}{por pouco | por um triz | quase}
\end{entry}

\begin{entry}{拆}{chai1}{8}{⼿}
  \definition{v.}{remover | tirar do seu lugar | desfazer | desmontar}
\end{entry}

\begin{entry}{单}{chan2}{8}{⼗}
  \definition{s.}{usado em 单于 \dpy{chan2yu2}}
  \seeref{单于}{chan2yu2}
  \seeref{单}{dan1}
  \seeref{单}{shan4}
\end{entry}

\begin{entry}{单于}{chan2yu2}{8,3}{⼗、⼆}
  \definition{s.}{rei de Xiongnu (匈奴)}
  \seealsoref{匈奴}{xiong1nu2}
\end{entry}

\begin{entry}{禅}{chan2}{12}{⽰}
  \definition*{s.}{Zen}
  \definition{s.}{meditação (Budismo)}
  \seeref{禅}{shan4}
\end{entry}

\begin{entry}{蝉}{chan2}{14}{⾍}
  \definition{s.}{cigarra}
\end{entry}

\begin{entry}{产后}{chan3hou4}{6,6}{⼇、⼝}
  \definition{s.}{pós-parto}
\end{entry}

\begin{entry}{产品}{chan3pin3}{6,9}{⼇、⼝}[HSK 4]
  \definition[个,件,种,批,项,类]{s.}{produto; item produzido}
\end{entry}

\begin{entry}{产生}{chan3sheng1}{6,5}{⼇、⽣}[HSK 3]
  \definition{v.}{produzir; evoluir; emergir; provocar; vir a ser; dar origem a}
\end{entry}

\begin{entry}{铲车}{chan3che1}{11,4}{⾦、⾞}
  \definition[台]{s.}{empilhadeira}
\end{entry}

\begin{entry}{长}{chang2}{4}{⾧}[HSK 2][Kangxi 168]
  \definition{adj.}{comprido | longo}
  \seeref{长}{zhang3}
\end{entry}

\begin{entry}{长城}{chang2cheng2}{4,9}{⾧、⼟}[HSK 3]
  \definition*{s.}{A Grande Muralha}
\end{entry}

\begin{entry}{长处}{chang2 chu4}{4,5}{⾧、⼡}[HSK 3]
  \definition{s.}{força; boas qualidades; pontos fortes}
\end{entry}

\begin{entry}{长颈鹿}{chang2jing3lu4}{4,11,11}{⾧、⾴、⿅}
  \definition[只]{s.}{girafa}
\end{entry}

\begin{entry}{长期}{chang2 qi1}{4,12}{⾧、⽉}[HSK 3]
  \definition{adj.}{secular; longo prazo; longo alcance; durante um longo período de tempo}
  \definition{s.}{longo prazo}
\end{entry}

\begin{entry}{长途}{chang2tu2}{4,10}{⾧、⾡}[HSK 4]
  \definition{adj.}{de longa distância; longe}
  \definition[段,次,程]{s.}{longa-distância; referindo-se especificamente a chamadas telefônicas de longa distância ou ônibus de longa distância}
\end{entry}

\begin{entry}{常}{chang2}{11}{⼱}[HSK 1]
  \definition*{s.}{sobrenome Chang}
  \definition{adv.}{muitas vezes | frequentemente}
\end{entry}

\begin{entry}{常常}{chang2 chang2}{11,11}{⼱、⼱}[HSK 1]
  \definition{adv.}{frequentemente | com frequência}
\end{entry}

\begin{entry}{常见}{chang2 jian4}{11,4}{⼱、⾒}[HSK 2]
  \definition{adj.}{comum}
\end{entry}

\begin{entry}{常识}{chang2shi2}{11,7}{⼱、⾔}[HSK 4]
  \definition{s.}{senso comum; conhecimento geral; conhecimento que uma pessoa comum deve ter}
\end{entry}

\begin{entry}{常问问题}{chang2wen4wen4ti2}{11,6,6,15}{⼱、⾨、⾨、⾴}
  \definition{s.}{FAQ; perguntas frequentes}
\end{entry}

\begin{entry}{常用}{chang2 yong4}{11,5}{⼱、⽤}[HSK 2]
  \definition{adj.}{em uso comum}
\end{entry}

\begin{entry}{厂}{chang3}{2}{⼚}[HSK 3]
  \definition[家]{s.}{fábrica; moinho; planta; obra | pátio; depósito}
  \seeref{厂}{han3}
\end{entry}

\begin{entry}{场}{chang3}{6}{⼟}[HSK 2]
  \definition{clas.}{para número de exames | para atividades esportivas ou recreativas}
  \definition{s.}{local grande usado para um propósito específico | cena (de uma peça) | palco}
\end{entry}

\begin{entry}{场合}{chang3he2}{6,6}{⼟、⼝}[HSK 3]
  \definition[种]{s.}{ocasião; situação}
\end{entry}

\begin{entry}{场景}{chang3jing3}{6,12}{⼟、⽇}
  \definition{s.}{cena | cenário | situação | contexto}
\end{entry}

\begin{entry}{场面}{chang3mian4}{6,9}{⼟、⾯}
  \definition{s.}{cena | espetáculo | ocasião | situação}
\end{entry}

\begin{entry}{场所}{chang3suo3}{6,8}{⼟、⼾}[HSK 3]
  \definition{s.}{lugar; sítio; arena}
\end{entry}

\begin{entry}{畅}{chang4}{8}{⽥}
  \definition*{s.}{sobrenome Chang}
  \definition{adj.}{suave; desimpedido; sem problemas; sem obstáculos; desobstruído | livre; desinibido}
\end{entry}

\begin{entry}{鬯}{chang4}{10}{⾿}[Kangxi 192]
  \definition{s.}{um antigo vinho para sacrifícios | estojo para arco | o mesmo que ``畅''}
  \seealsoref{畅}{chang4}
\end{entry}

\begin{entry}{唱}{chang4}{11}{⼝}[HSK 1]
  \definition{v.}{cantar}
\end{entry}

\begin{entry}{唱歌}{chang4 ge1}{11,14}{⼝、⽋}[HSK 1]
  \definition{v.+compl.}{cantar}
\end{entry}

\begin{entry}{唱片}{chang4 pian4}{11,4}{⼝、⽚}[HSK 4]
  \definition[枚,张]{s.}{disco; disco feito de goma-laca, plástico, etc. com ranhuras em espiral na superfície para registrar alterações no som que podem reproduzir o som gravado em um fonógrafo}
\end{entry}

\begin{entry}{抄}{chao1}{7}{⼿}[HSK 4]
  \definition*{s.}{sobrenome Chao}
  \definition{v.}{copiar; transcrever | plagiar | revistar e confiscar; fazer uma batida | pegar um atalho | dobrar (os braços) | agarrar; pegar}
\end{entry}

\begin{entry}{抄写}{chao1 xie3}{7,5}{⼿、⼍}[HSK 4]
  \definition{v.}{copiar; transcrever}
\end{entry}

\begin{entry}{超过}{chao1guo4}{12,6}{⾛、⾡}[HSK 2]
  \definition{v.}{passar | ultrapassar (alguém ou algo) | exceder | ser mais do que | estar acima de (um padrão)}
\end{entry}

\begin{entry}{超级}{chao1ji2}{12,6}{⾛、⽷}[HSK 3]
  \definition{adj.}{super}
  \definition{pref.}{super-; ultra-; hiper-}
\end{entry}

\begin{entry}{超声}{chao1sheng1}{12,7}{⾛、⼠}
  \definition{adj.}{ultrasônico}
  \definition{s.}{ultrasom}
\end{entry}

\begin{entry}{超市}{chao1shi4}{12,5}{⾛、⼱}[HSK 2]
  \definition[家]{s.}{supermercado}
\end{entry}

\begin{entry}{巢}{chao2}{11}{⼮}
  \definition*{s.}{sobrenome Chao}
  \definition{s.}{ninho (de aves, etc.)}
\end{entry}

\begin{entry}{朝}{chao2}{12}{⽉}[HSK 3]
  \definition*{s.}{sobrenome Chao}
  \definition{prep.}{para; em direção a}
  \definition{s.}{tribunal; governo | dinastia | o reino de um imperador}
  \definition{v.}{ter uma audiência com (um rei, um imperador, etc.); fazer uma peregrinação a | encarar; olhar}
  \seeref{朝}{zhao1}
\end{entry}

\begin{entry}{朝廷}{chao2ting2}{12,6}{⽉、⼵}
  \definition{s.}{corte imperial | dinastia}
\end{entry}

\begin{entry}{朝鲜}{chao2xian3}{12,14}{⽉、⿂}
  \definition*{s.}{Coréia do Norte}
\end{entry}

\begin{entry}{潮}{chao2}{15}{⽔}[HSK 4]
  \definition{adj.}{úmido; molhado | inferior; de qualidade ruim | inferior; não muito habilidoso}
  \definition{s.}{maré; água da maré | surto; corrente; maré; uma metáfora para mudanças sociais em grande escala ou para os altos e baixos de um movimento (social)}
\end{entry}

\begin{entry}{潮流}{chao2liu2}{15,10}{⽔、⽔}[HSK 4]
  \definition{s.}{maré; corrente de maré; movimento da água devido às marés | tendência; analogia com mudanças sociais ou tendências de desenvolvimento}
\end{entry}

\begin{entry}{潮湿}{chao2shi1}{15,12}{⽔、⽔}[HSK 4]
  \definition{adj.}{molhado; úmido; umedecido; que contém mais água do que o normal}
\end{entry}

\begin{entry}{鼂}{chao2}{18}{⿌}
  \definition*{s.}{sobrenome Chao}
  \definition{s.}{tartaruga marinha}
\end{entry}

\begin{entry}{吵}{chao3}{7}{⼝}[HSK 3]
  \definition{adj.}{barulhento; ruidoso}
  \definition{v.}{perturbar fazendo barulho; fazer barulho | discutir; brigar; disputar}
\end{entry}

\begin{entry}{吵架}{chao3jia4}{7,9}{⼝、⽊}[HSK 3]
  \definition{v.+compl.}{brigar; discutir; ter uma briga}
\end{entry}

\begin{entry}{炒}{chao3}{8}{⽕}
  \definition{v.}{saltear | demitir (alguém)}
\end{entry}

\begin{entry}{车}{che1}{4}{⾞}[HSK 1][Kangxi 159]
  \definition*{s.}{sobrenome Che}
  \definition[辆]{s.}{carro | veículo | viatura}
  \seeref{车}{ju1}
\end{entry}

\begin{entry}{车次}{che1ci4}{4,6}{⾞、⽋}
  \definition{s.}{número do trem}
\end{entry}

\begin{entry}{车库}{che1ku4}{4,7}{⾞、⼴}
  \definition{s.}{garagem}
\end{entry}

\begin{entry}{车辆}{che1 liang4}{4,11}{⾞、⾞}[HSK 2]
  \definition{s.}{veículo | carro}
\end{entry}

\begin{entry}{车牌}{che1pai2}{4,12}{⾞、⽚}
  \definition{s.}{matrícula | placa de carro}
\end{entry}

\begin{entry}{车票}{che1 piao4}{4,11}{⾞、⽰}[HSK 1]
  \definition{s.}{bilhete (de ônibus, trem, metrô)}
\end{entry}

\begin{entry}{车上}{che1 shang4}{4,3}{⾞、⼀}[HSK 1]
  \definition{adv.}{no carro | dentro do veículo}
\end{entry}

\begin{entry}{车水马龙}{che1shui3-ma3long2}{4,4,3,5}{⾞、⽔、⾺、⿓}
  \definition{expr.}{tráfego engarrafado | engarrafamento | (literalmente) ``fluxo interminável de cavalos e carruagens''}
\end{entry}

\begin{entry}{车站}{che1 zhan4}{4,10}{⾞、⽴}[HSK 1]
  \definition[处,个]{s.}{estação | ponto de ônibus}
\end{entry}

\begin{entry}{车主}{che1zhu3}{4,5}{⾞、⼂}
  \definition{s.}{proprietário do carro}
\end{entry}

\begin{entry}{车子}{che1zi5}{4,3}{⾞、⼦}
  \definition{s.}{qualquer veículo (carro, bicicleta, caminhão, etc)}
\end{entry}

\begin{entry}{尺}{che3}{4}{⼫}
  \definition{s.}{(tom) uma nota da escala em gongchepu (工尺谱), correspondente a 2 na notação musical numerada}
  \seealsoref{工尺谱}{gong1 che3 pu3}
\end{entry}

\begin{entry}{彻底}{che4di3}{7,8}{⼻、⼴}[HSK 4]
  \definition{adj.}{minucioso; completo; exaustivo; profundo e completo; nada é deixado de fora}
\end{entry}

\begin{entry}{撤}{che4}{15}{⼿}
  \definition{v.}{remover, tirar}
\end{entry}

\begin{entry}{沉}{chen2}{7}{⽔}[HSK 4]
  \definition{adj.}{profundo | pesado | pesado (sentir-se pesado)}
  \definition{v.}{afundar; submergir; imergir | manter baixo; abaixar | descansar; parar}
  \seeref{沉}{chen2}
\end{entry}

\begin{entry}{沉默}{chen2mo4}{7,16}{⽔、⿊}[HSK 4]
  \definition{adj.}{silencioso; reticente; taciturno; não comunicativo}
  \definition{v.}{silenciar; não falar por causa de alguma coisa}
\end{entry}

\begin{entry}{沉重}{chen2zhong4}{7,9}{⽔、⾥}[HSK 4]
  \definition{adj.}{(pressão, fardo, etc.) muito pesado; profundo | sério; pesado; humor pouco animador; fardo pesado de pensamentos}
\end{entry}

\begin{entry}{衬衫}{chen4shan1}{8,8}{⾐、⾐}[HSK 3]
  \definition[件]{s.}{camisa; blusa}
\end{entry}

\begin{entry}{衬衣}{chen4 yi1}{8,6}{⾐、⾐}[HSK 3]
  \definition[件]{s.}{camisa}
\end{entry}

\begin{entry}{称}{chen4}{10}{⽲}
  \definition{v.}{ajustar | combinar}
  \seeref{称}{cheng1}
\end{entry}

\begin{entry}{称}{cheng1}{10}{⽲}[HSK 2]
  \definition*{s.}{sobrenome Cheng}
  \definition{s.}{nome}
  \definition{v.}{chamar | dizer | elogiar | louvar | pesar | levantar | começar}
  \seeref{称}{chen4}
\end{entry}

\begin{entry}{称为}{cheng1 wei2}{10,4}{⽲、⼂}[HSK 3]
  \definition{v.}{chamar; ser chamado; ser conhecido como}
\end{entry}

\begin{entry}{称赞}{cheng1zan4}{10,16}{⽲、⾙}[HSK 4]
  \definition[句,声,番,次]{s.}{elogio; aclamação; louvor; avaliação positiva de um desempenho ou conquista}
  \definition{v.}{elogiar; aclamar; louvar; usar palavras para expressar um carinho pelas virtudes de uma pessoa ou coisa}
\end{entry}

\begin{entry}{成}{cheng2}{6}{⼽}[HSK 2]
  \definition*{s.}{sobrenome Cheng}
  \definition{v.}{sair-se bem | ser bem sucedido}
\end{entry}

\begin{entry}{成都}{cheng2du1}{6,10}{⼽、⾢}
  \definition*{s.}{Chengdu}
\end{entry}

\begin{entry}{成功}{cheng2gong1}{6,5}{⼽、⼒}[HSK 3]
  \definition{adj.}{bem-sucedido | frutífero}
  \definition[个,次]{s.}{sucesso}
  \definition{v.}{ter sucesso}
\end{entry}

\begin{entry}{成果}{cheng2guo3}{6,8}{⼽、⽊}[HSK 3]
  \definition{s.}{realização; resultado}
\end{entry}

\begin{entry}{成婚}{cheng2hun1}{6,11}{⼽、⼥}
  \definition{v.}{casar-se}
\end{entry}

\begin{entry}{成活}{cheng2huo2}{6,9}{⼽、⽔}
  \definition{v.}{sobreviver}
\end{entry}

\begin{entry}{成吉思汗}{cheng2ji2si1han2}{6,6,9,6}{⼽、⼝、⼼、⽔}
  \definition*{s.}{Genghis Khan (1162-1227), fundador e governante do Império Mongol}
\end{entry}

\begin{entry}{成绩}{cheng2ji4}{6,11}{⼽、⽷}[HSK 2]
  \definition[项,个]{s.}{nota | classificação}
\end{entry}

\begin{entry}{成家}{cheng2jia1}{6,10}{⼽、⼧}
  \definition{v.}{tornar-se um especialista reconhecido | estabelecer-se e casar-se (de um homem)}
\end{entry}

\begin{entry}{成就}{cheng2jiu4}{6,12}{⼽、⼪}[HSK 3]
  \definition[个]{s.}{realização; sucesso}
  \definition{v.}{realizar; atingir; completar}
\end{entry}

\begin{entry}{成立}{cheng2li4}{6,5}{⼽、⽴}[HSK 3]
  \definition{v.}{fundar; estabelecer; montar | ser válido; ser sustentável; reter água}
\end{entry}

\begin{entry}{成批}{cheng2pi1}{6,7}{⼽、⼿}
  \definition{s.}{em lotes | a granel}
\end{entry}

\begin{entry}{成器}{cheng2qi4}{6,16}{⼽、⼝}
  \definition{v.}{tornar-se uma pessoa digna de respeito | fazer algo de si mesmo}
\end{entry}

\begin{entry}{成人}{cheng2ren2}{6,2}{⼽、⼈}[HSK 4]
  \definition[个]{s.}{adulto; crescido; pessoa adulta}
  \definition{v.}{crescer; tornar-se adulto}
\end{entry}

\begin{entry}{成色}{cheng2se4}{6,6}{⼽、⾊}
  \definition{v.}{sair-se bem | ser bem sucedido}
\end{entry}

\begin{entry}{成熟}{cheng2shu2}{6,15}{⼽、⽕}[HSK 3]
  \definition{adj./s.}{maduro; totalmente crescido}
  \definition{v.}{amadurecer; estar maduro; estar totalmente crescido}
\end{entry}

\begin{entry}{成为}{cheng2wei2}{6,4}{⼽、⼂}[HSK 2]
  \definition{s.}{tornar-se | transformar-se em}
\end{entry}

\begin{entry}{成员}{cheng2yuan2}{6,7}{⼽、⼝}[HSK 3]
  \definition[个]{s.}{membro}
\end{entry}

\begin{entry}{成长}{cheng2zhang3}{6,4}{⼽、⾧}[HSK 3]
  \definition{v.}{crescer; amadurecer; amadurar}
\end{entry}

\begin{entry}{承担}{cheng2dan1}{8,8}{⼿、⼿}[HSK 4]
  \definition{v.}{suportar; empreender; assumir; tomar conta de algo}
\end{entry}

\begin{entry}{承认}{cheng2ren4}{8,4}{⼿、⾔}[HSK 4]
  \definition{s.}{reconhecimento (diplomático, artístico, etc.)}
  \definition{v.}{admitir; reconhecer | dar reconhecimento diplomático; reconhecer}
\end{entry}

\begin{entry}{承受}{cheng2shou4}{8,8}{⼿、⼜}[HSK 4]
  \definition{v.}{suportar; resistir; realizar (tarefas, dificuldades, pressões, etc.); submeter-se a (testes, etc.) | herdar}
\end{entry}

\begin{entry}{诚实}{cheng2shi2}{8,8}{⾔、⼧}[HSK 4]
  \definition{adj.}{honesto; sincero e honesto, não hipócrita}
\end{entry}

\begin{entry}{诚实地}{cheng2shi2 di4}{8,8,6}{⾔、⼧、⼟}
  \definition{adv.}{honestamente}
\end{entry}

\begin{entry}{诚信}{cheng2 xin4}{8,9}{⾔、⼈}[HSK 4]
  \definition{adj.}{honesto e confiável}
  \definition[种]{s.}{fé; honestidade; padrão e princípio de comportamento: não contar mentiras, prometer aos outros o que eles podem fazer e ter a confiança dos outros}
\end{entry}

\begin{entry}{城}{cheng2}{9}{⼟}[HSK 3]
  \definition*{s.}{sobrenome Cheng}
  \definition[座,道,个]{s.}{muralha da cidade; muro | cidade}
\end{entry}

\begin{entry}{城堡}{cheng2bao3}{9,12}{⼟、⼟}
  \definition*{s.}{castelo | torre (peça de xadrez)}
\end{entry}

\begin{entry}{城度}{cheng2du4}{9,9}{⼟、⼴}[HSK 3]
  \definition*{s.}{Cidade}
\end{entry}

\begin{entry}{城市}{cheng2shi4}{9,5}{⼟、⼱}[HSK 3]
  \definition[个,座]{s.}{cidade}
\end{entry}

\begin{entry}{乘客}{cheng2ke4}{10,9}{⽲、⼧}
  \definition{s.}{passageiro}
\end{entry}

\begin{entry}{乘客数}{cheng2ke4 shu4}{10,9,13}{⽲、⼧、⽁}
  \definition{s.}{número de passageiros}
\end{entry}

\begin{entry}{惩处}{cheng2chu3}{12,5}{⼼、⼡}
  \definition{v.}{administrar justiça | punir}
\end{entry}

\begin{entry}{惩罚}{cheng2fa2}{12,9}{⼼、⽹}
  \definition{v.}{punir | penalizar}
\end{entry}

\begin{entry}{程度}{cheng2du4}{12,9}{⽲、⼴}[HSK 3]
  \definition[种]{s.}{nível; grau (de cultura, educação, aprendizagem, etc.) | extensão; grau}
\end{entry}

\begin{entry}{程控}{cheng2kong4}{12,11}{⽲、⼿}
  \definition{s.}{programado | sob controle automático}
\end{entry}

\begin{entry}{程序}{cheng2xu4}{12,7}{⽲、⼴}[HSK 4]
  \definition[个,套,种]{s.}{ordem; curso; sequência; procedimento; ordem em que algo é feito; também, um determinado número de etapas em um trabalho | programa; conjunto de instruções de computador projetado em sequência para permitir que um computador execute uma ou mais operações}
\end{entry}

\begin{entry}{程序库}{cheng2xu4ku4}{12,7,7}{⽲、⼴、⼴}
  \definition{s.}{biblioteca de funções e procedimentos para programas de computador}
\end{entry}

\begin{entry}{程序设计}{cheng2xu4she4ji4}{12,7,6,4}{⽲、⼴、⾔、⾔}
  \definition{s.}{programação de computadores}
\end{entry}

\begin{entry}{橙色}{cheng2 se4}{16,6}{⽊、⾊}
  \definition{s.}{cor de laranja}
\end{entry}

\begin{entry}{橙汁}{cheng2zhi1}{16,5}{⽊、⽔}
  \definition[瓶,杯,罐,盒]{s.}{suco de laranja}
  \seealsoref{橘子汁}{ju2zi5zhi1}
  \seealsoref{柳橙汁}{liu3cheng2zhi1}
\end{entry}

\begin{entry}{吃}{chi1}{6}{⼝}[HSK 1]
  \definition{v.}{comer | consumir | comer em (uma cafeteria, etc.) | erradicar | destruir | absorver}
\end{entry}

\begin{entry}{吃饭}{chi1 fan4}{6,7}{⼝、⾷}[HSK 1]
  \definition{v.+compl.}{comer | ter (comer) uma refeição | manter vivo | ganhar a vida}
\end{entry}

\begin{entry}{吃惊}{chi1jing1}{6,11}{⼝、⼼}[HSK 4]
  \definition{v.+compl.}{ficar assustado; ficar chocado; ficar espantado; pegar de surpresa; ficar assustado inesperadamente}
\end{entry}

\begin{entry}{吃屎}{chi1 shi3}{6,9}{⼝、⼫}
  \definition{expr.}{Coma merda!}
\end{entry}

\begin{entry}{池}{chi2}{6}{⽔}
  \definition*{s.}{sobrenome Chi}
  \definition{s.}{lagoa | reservatório | fosso}
\end{entry}

\begin{entry}{迟到}{chi2dao4}{7,8}{⾡、⼑}[HSK 4]
  \definition{v.}{chegar atrasado; atrasar-se}
\end{entry}

\begin{entry}{持续}{chi2xu4}{9,11}{⼿、⽷}[HSK 3]
  \definition{v.}{durar; continuar; sustentar}
\end{entry}

\begin{entry}{尺}{chi3}{4}{⼫}[HSK 4]
  \definition{clas.}{chi, uma unidade de comprimento (=13 metros)}
  \definition[把]{s.}{régua; instrumentos de medição | um instrumento no formato de uma régua}
  \seeref{尺}{che3}
\end{entry}

\begin{entry}{尺寸}{chi3 cun4}{4,3}{⼫、⼨}[HSK 4]
  \definition{s.}{tamanho; medida; dimensão}
\end{entry}

\begin{entry}{尺子}{chi3zi5}{4,3}{⼫、⼦}[HSK 4]
  \definition[把]{s.}{régua de madeira ou metal para orientar a caneta ou o lápis para desenhar linhas ou fazer medições}
\end{entry}

\begin{entry}{齿}{chi3}{8}{⿒}[Kangxi 211]
  \definition[颗]{s.}{dente | uma parte de qualquer coisa semelhante a um dente; parte dentada de um objeto | idade (de uma pessoa); faixa etária}
  \definition{v.}{mencionar}
\end{entry}

\begin{entry}{斥骂}{chi4ma4}{5,9}{⽄、⾺}
  \definition{v.}{repreender}
\end{entry}

\begin{entry}{赤}{chi4}{7}{⾚}[Kangxi 155]
  \definition*{s.}{sobrenome Chi}
  \definition{adj.}{vermelho; de cor vermelha | leal; sincero; de coração único | nu; sem roupa}
\end{entry}

\begin{entry}{充电}{chong1 dian4}{6,5}{⼉、⽥}[HSK 4]
  \definition{v.}{carregar (uma bateria); conectar uma fonte de alimentação CC aos terminais da bateria para recarregar a bateria | relaxar; passar o tempo livre; ``recarregar as baterias''; estudar para adquirir mais conhecimento; reabastecer (ou ampliar) o conhecimento; metaforicamente falando, para reabastecer a força física e a energia por meio do descanso e da recreação; também metaforicamente falando, para reabastecer novos conhecimentos e desenvolver novas habilidades por meio do reaprendizado}
\end{entry}

\begin{entry}{充电器}{chong1dian4qi4}{6,5,16}{⼉、⽥、⼝}[HSK 4]
  \definition{s.}{carregador de bateria; dispositivo para alimentar uma bateria com energia, forçando uma corrente através dela}
\end{entry}

\begin{entry}{充分}{chong1fen4}{6,4}{⼉、⼑}[HSK 4]
  \definition{adj.}{cheio; amplo; abundante; suficiente; adequado}
  \definition{adv.}{totalmente; até o fim}
\end{entry}

\begin{entry}{充满}{chong1man3}{6,13}{⼉、⽔}[HSK 3]
  \definition{v.}{preencher | encher-se de; transbordar de; permear-se de}
\end{entry}

\begin{entry}{冲}{chong1}{6}{⼎}[HSK 4]
  \definition{s.}{via pública; local importante; via de passagem; via local importante | um trecho de planície em uma área montanhosa | oposição; os planetas externos orbitam até ficarem alinhados com a Terra e o Sol, e a Terra está no meio}
  \definition{v.}{atacar; apressar; correr; passar rapidamente; passar por um obstáculo | colidir; chocar; bater | despejar água fervente sobre | enxaguar; dar descarga; lavar | revelar (filme) | neutralizar a má sorte}
  \seeref{冲}{chong4}
\end{entry}

\begin{entry}{冲锋}{chong1feng1}{6,12}{⼎、⾦}
  \definition{v.}{cobrar | tomar de assalto}
\end{entry}

\begin{entry}{冲浪}{chong1lang4}{6,10}{⼎、⽔}
  \definition{s.}{surfe}
  \definition{v.}{surfar}
\end{entry}

\begin{entry}{冲突}{chong1tu1}{6,9}{⼎、⽳}
  \definition{s.}{conflito | choque de forças opostas | colisão (de interesses)}
\end{entry}

\begin{entry}{憧憬}{chong1jing3}{15,15}{⼼、⼼}
  \definition{v.}{ansiar por | esperar por}
\end{entry}

\begin{entry}{虫子}{chong2 zi5}{6,3}{⾍、⼦}[HSK 4]
  \definition[条,只,种]{s.}{percevejo; besouro; inseto; verme; criaturas semelhantes a insetos}
\end{entry}

\begin{entry}{重}{chong2}{9}{⾥}
  \definition*{s.}{sobrenome Chong}
  \definition{adv.}{novamente; mais uma vez}
  \definition{clas.}{para camadas}
  \definition{v.}{repetir; duplicar}
  \seeref{重}{zhong4}
\end{entry}

\begin{entry}{重重}{chong2chong2}{9,9}{⾥、⾥}
  \definition{adv.}{camada após camada | um após o outro}
  \seeref{重重}{zhong4zhong4}
\end{entry}

\begin{entry}{重点}{chong2dian3}{9,9}{⾥、⽕}
  \definition{adj./adv./s.}{nota-chave | ponto-chave | ponto focal | ênfase}
  \seeref{重点}{zhong4dian3}
\end{entry}

\begin{entry}{重逢}{chong2feng2}{9,10}{⾥、⾡}
  \definition{s.}{reunião}
  \definition{v.}{encontrar-se novamente | reunir-se}
\end{entry}

\begin{entry}{重复}{chong2fu4}{9,9}{⾥、⼢}[HSK 2]
  \definition{v.}{repetir | iterar | duplicar | reduplicar | fazer algo de novo}
\end{entry}

\begin{entry}{重新}{chong2xin1}{9,13}{⾥、⽄}[HSK 2]
  \definition{adv.}{de novo | novamente}
\end{entry}

\begin{entry}{重阳节}{chong2yang2jie2}{9,6,5}{⾥、⾩、⾋}
  \definition*{s.}{Festa do Duplo Nove, Festival Yang, dia de subir aos lugares mais altos para evitar calamidades e dia do culto aos antepassados (9º dia do nono mês lunar)}
\end{entry}

\begin{entry}{崇}{chong2}{11}{⼭}
  \definition*{s.}{sobrenome Chong}
  \definition{adj.}{alto | sublime | elevado}
  \definition{v.}{estimar | adorar}
\end{entry}

\begin{entry}{宠物}{chong3wu4}{8,8}{⼧、⽜}
  \definition{s.}{animal de estimação}
\end{entry}

\begin{entry}{冲}{chong4}{6}{⼎}
  \definition{adj.}{poderoso; com vigor; com muita força; vigoroso | forte; odor forte e pungente (olfato)}
  \definition{adv.}{de frente; em direção a | na força de; com base em; em virtude de}
  \definition{v.}{fazer face a (algo ou alguém) | estampar (máquina de estamparia)}
  \seeref{冲}{chong1}
\end{entry}

\begin{entry}{抽}{chou1}{8}{⼿}[HSK 4]
  \definition{v.}{retirar; tirar (do meio); retirar, puxar ou arrancar algo que está preso ou emaranhado em outra coisa | tirar, retirar (uma parte de um todo) | (certas plantas) começar a crescer, produzir | bombear | encolher; contrair |
chicotear; açoitar; surrar | dirigir; conduzir | encontrar tempo; libertar-se; sair de alguma coisa}
\end{entry}

\begin{entry}{抽奖}{chou1 jiang3}{8,9}{⼿、⼤}[HSK 4]
  \definition{s.}{loteria; sorteio de loteria}
\end{entry}

\begin{entry}{抽烟}{chou1yan1}{8,10}{⼿、⽕}[HSK 4]
  \definition{v.+compl.}{fumar (um cigarro ou um cachimbo)}
\end{entry}

\begin{entry}{酬劳}{chou2lao2}{13,7}{⾣、⼒}
  \definition{s.}{recompensa}
\end{entry}

\begin{entry}{臭}{chou4}{10}{⾃}
  \definition{adj.}{fétido | repulsivo | repugnante | malcheiroso}
  \definition{s.}{fedor}
  \definition{v.}{feder}
  \seeref{臭}{xiu4}
\end{entry}

\begin{entry}{臭气}{chou4qi4}{10,4}{⾃、⽓}
  \definition{s.}{fedor}
\end{entry}

\begin{entry}{出}{chu1}{5}{⼐}[HSK 1]
  \definition{clas.}{para dramas, peças, óperas, etc.}
  \definition{v.}{sair | ir para fora | vir para fora}
\end{entry}

\begin{entry}{出版}{chu1ban3}{5,8}{⼐、⽚}
  \definition{v.}{publicar | editar}
\end{entry}

\begin{entry}{出版社}{chu1ban3she4}{5,8,7}{⼐、⽚、⽰}
  \definition{s.}{editora}
\end{entry}

\begin{entry}{出差}{chu1chai1}{5,9}{⼐、⼯}
  \definition{v.+compl.}{fazer uma viagem oficial ou de negócios}
\end{entry}

\begin{entry}{出发}{chu1fa1}{5,5}{⼐、⼜}[HSK 2]
  \definition{v.}{partir | começar (uma jornada)}
\end{entry}

\begin{entry}{出国}{chu1 guo2}{5,8}{⼐、⼞}[HSK 2]
  \definition{v.+compl.}{ir para o exterior | deixar a terra natal}
\end{entry}

\begin{entry}{出汗}{chu1han4}{5,6}{⼐、⽔}
  \definition{v.}{transpirar | suar}
\end{entry}

\begin{entry}{出击}{chu1ji1}{5,5}{⼐、⼐}
  \definition{v.}{atacar}
\end{entry}

\begin{entry}{出口}{chu1kou3}{5,3}{⼐、⼝}[HSK 2,4]
  \definition[个]{s.}{saída; porta ou passagem para o exterior}
  \definition{v.+compl.}{falar; proferir; manifestar-se | exportar mercadorias do país ou da região para venda no exterior ou em outro lugar | deixar o porto (um navio)}
\end{entry}

\begin{entry}{出来}{chu1 lai2}{5,7}{⼐、⽊}[HSK 1]
  \definition{v.}{sair | vir para fora (para a minha localização)}
\end{entry}

\begin{entry}{出门}{chu1 men2}{5,3}{⼐、⾨}[HSK 2]
  \definition{v.+compl.}{sair | sair de casa | estar longe de casa | fazer uma viagem | casar}
\end{entry}

\begin{entry}{出去}{chu1 qu4}{5,5}{⼐、⼛}[HSK 1]
  \definition{v.}{sair | ir para fora (a partir da minha localização)}
\end{entry}

\begin{entry}{出色}{chu1se4}{5,6}{⼐、⾊}[HSK 4]
  \definition{adj.}{esplêndido; extraordinário; notável; excepcionalmente bom; acima da média}
\end{entry}

\begin{entry}{出生}{chu1sheng1}{5,5}{⼐、⽣}[HSK 2]
  \definition{v.}{nascer}
\end{entry}

\begin{entry}{出售}{chu1 shou4}{5,11}{⼐、⼝}[HSK 4]
  \definition{v.}{vender; oferecer para venda}
\end{entry}

\begin{entry}{出席}{chu1xi2}{5,10}{⼐、⼱}[HSK 4]
  \definition{v.}{comparecer; estar presente; participar de reuniões com o direito de falar e votar; juntar-se a uma organização ou atividade}
\end{entry}

\begin{entry}{出现}{chu1xian4}{5,8}{⼐、⾒}[HSK 2]
  \definition{v.}{aparecer | surgir | emergir | crescer}
\end{entry}

\begin{entry}{出行}{chu1xing2}{5,6}{⼐、⾏}
  \definition{v.}{sair para algum lugar (viagem relativamente curta) | partir em uma viagem (viagem mais longa)}
\end{entry}

\begin{entry}{出院}{chu1 yuan4}{5,9}{⼐、⾩}[HSK 2]
  \definition{v.}{deixar o hospital | estar fora do hospital | ter alta do hospital}
\end{entry}

\begin{entry}{出站}{chu1 zhan4}{5,10}{⼐、⽴}
  \definition{s.}{saída da estação}
\end{entry}

\begin{entry}{出租}{chu1 zu1}{5,10}{⼐、⽲}[HSK 2]
  \definition{v.}{alugar | arrendar}
\end{entry}

\begin{entry}{出租车}{chu1zu1che1}{5,10,4}{⼐、⽲、⾞}[HSK 2]
  \definition{s.}{táxi}
  \seealsoref{出租汽车}{chu1zu1qi4che1}
\end{entry}

\begin{entry}{出租汽车}{chu1zu1qi4che1}{5,10,7,4}{⼐、⽲、⽔、⾞}
  \definition[辆]{s.}{táxi}
  \seealsoref{出租车}{chu1zu1che1}
\end{entry}

\begin{entry}{出租司机}{chu1zu1si1ji1}{5,10,5,6}{⼐、⽲、⼝、⽊}
  \definition{s.}{motorista de táxi}
\end{entry}

\begin{entry}{初}{chu1}{7}{⾐}[HSK 3]
  \definition*{s.}{sobrenome Chu}
  \definition{adj.}{primeiro (em ordem) | elementar; rudimentar | original}
  \definition{adv.}{pela primeira vez}
  \definition{pref.}{anexado aos numerais de um a dez para indicar ordem (primeiro ao décimo)}
  \definition{s.}{no início de; na primeira parte de | o estágio júnior (pleno; sênior)}
\end{entry}

\begin{entry}{初步}{chu1bu4}{7,7}{⾐、⽌}[HSK 3]
  \definition{adj.}{inicial; preliminar}
\end{entry}

\begin{entry}{初级}{chu1ji2}{7,6}{⾐、⽷}[HSK 3]
  \definition{adj.}{elementar; primário; júnior; inicial}
\end{entry}

\begin{entry}{初心}{chu1xin1}{7,4}{⾐、⼼}
  \definition{s.}{intenção original de alguém, aspiração, etc. | (budismo) ``mente do iniciante'' (ter a mente aberta quando estudando um assunto como um iniciante no assunto teria)}
\end{entry}

\begin{entry}{初中}{chu1 zhong1}{7,4}{⾐、⼁}[HSK 3]
  \definition[所,个]{s.}{ensino médio; ensino fundamental}
\end{entry}

\begin{entry}{除非}{chu2fei1}{9,8}{⾩、⾮}
  \definition{conj.}{a menos que | somente se}
\end{entry}

\begin{entry}{除了}{chu2le5}{9,2}{⾩、⼅}[HSK 3]
  \definition{prep.}{exceto; à parte | além disso; além de | ou \dots ou \dots}
\end{entry}

\begin{entry}{厨房}{chu2fang2}{12,8}{⼚、⼾}
  \definition[间]{s.}{cozinha}
\end{entry}

\begin{entry}{处}{chu3}{5}{⼡}[HSK 4]
  \definition*{s.}{sobrenome Chu}
  \definition{v.}{morar; habitar; viver em um lugar | dar-se bem (com alguém); relacionar-se; interagir | estar situado em; estar em uma determinada condição; estar em (um lugar, período ou ocasião) | gerenciar; manejar; lidar com | punir; sentenciar; tomar medidas disciplinares contra (alguém)}
  \seeref{处}{chu4}
\end{entry}

\begin{entry}{处罚}{chu3fa2}{5,9}{⼡、⽹}
  \definition{v.}{penalizar | punir}
\end{entry}

\begin{entry}{处理}{chu3li3}{5,11}{⼡、⽟}[HSK 3]
  \definition{s.}{manuseio; descarte}
  \definition{v.}{lidar com; dispor de | resolver; punir; lidar | vender a preços reduzidos; liquidar | lidar com; processar}
\end{entry}

\begin{entry}{处于}{chu3 yu2}{5,3}{⼡、⼆}[HSK 4]
  \definition{v.}{estar em (uma condição, estado)}
\end{entry}

\begin{entry}{处}{chu4}{5}{⼡}
  \definition{clas.}{para locais ou itens de danos: lugar, local}
  \definition{s.}{lugar; local; instalação; dependência | parte; ponto; aspecto ou parte de um objeto | escritório; departamento; nomes de determinados órgãos, organizações ou unidades em órgãos por empresa}
  \seeref{处}{chu3}
\end{entry}

\begin{entry}{处处}{chu4chu4}{5,5}{⼡、⼡}
  \definition{adv.}{em todos os lugares | em todos os aspectos}
\end{entry}

\begin{entry}{畜}{chu4}{10}{⽥}
  \definition{s.}{gado | animal domesticado | animal doméstico}
  \seeref{畜}{xu4}
\end{entry}

\begin{entry}{穿}{chuan1}{9}{⽳}[HSK 1]
  \definition{v.}{vestir}
\end{entry}

\begin{entry}{穿上}{chuan1 shang4}{9,3}{⽳、⼀}[HSK 4]
  \definition{v.}{vestir (roupas, etc.); colocar roupas}
\end{entry}

\begin{entry}{传}{chuan2}{6}{⼈}[HSK 3]
  \definition{v.}{passar; passar adiante | passar adiante; legar; passar de \dots para \dots | transmitir (conhecimento, habilidade, etc.); comunicar; ensinar | espalhar; propagar | transmitir; conduzir; transferir | transmitir; expressar |convocar | infectar; ser contagioso}
  \seeref{传}{zhuan4}
\end{entry}

\begin{entry}{传播}{chuan2bo1}{6,15}{⼈、⼿}[HSK 3]
  \definition{v.}{espalhar; difundir; propagar; disseminar}
\end{entry}

\begin{entry}{传承}{chuan2cheng2}{6,8}{⼈、⼿}
  \definition{s.}{herança | tradição continuada}
  \definition{v.}{transmitir (para as gerações futuras) | passar adiante (desde os tempos antigos)}
\end{entry}

\begin{entry}{传给}{chuan2gei3}{6,9}{⼈、⽷}
  \definition{v.}{passar para | transferir para | entregar a}
\end{entry}

\begin{entry}{传来}{chuan2 lai2}{6,7}{⼈、⽊}[HSK 3]
  \definition{v.}{(um som) passar | (notícias) chegar}
\end{entry}

\begin{entry}{传说}{chuan2shuo1}{6,9}{⼈、⾔}[HSK 3]
  \definition{s.}{lenda | conto popular | folclore}
  \definition{v.}{dizer que; ser dito; passar de boca em boca}
\end{entry}

\begin{entry}{传统}{chuan2tong3}{6,9}{⼈、⽷}[HSK 4]
  \definition{adj.}{tradicional; histórico; transmitido de geração em geração | antiquado, conservador e fora de sintonia com os tempos}
  \definition[个]{s.}{tradição; costume; fatores sociais, como costumes, moral, ideias, estilos, artes, instituições etc., que são transmitidos de uma geração para outra e que são característicos da sociedade}
\end{entry}

\begin{entry}{传真}{chuan2zhen1}{6,10}{⼈、⼗}
  \definition{s.}{fax, facsímile}
\end{entry}

\begin{entry}{船}{chuan2}{11}{⾈}[HSK 2]
  \definition[条,艘,只]{s.}{barco | navio}
\end{entry}

\begin{entry}{窗户}{chuang1hu5}{12,4}{⽳、⼾}[HSK 4]
  \definition[个,扇,面,排]{s.}{janela; dispositivo de ventilação e transmissão de luz nas paredes}
\end{entry}

\begin{entry}{窗帘}{chuang1lian2}{12,8}{⽳、⼱}
  \definition{s.}{cortina}
\end{entry}

\begin{entry}{窗台}{chuang1 tai2}{12,5}{⽳、⼝}[HSK 4]
  \definition{s.}{parapeito da janela; peitoril; parte plana de uma janela que segura a moldura}
\end{entry}

\begin{entry}{窗子}{chuang1 zi5}{12,3}{⽳、⼦}[HSK 4]
  \definition{s.}{janela}
\end{entry}

\begin{entry}{床}{chuang2}{7}{⼴}[HSK 1]
  \definition{clas.}{para camas}
  \definition[张]{s.}{cama}
\end{entry}

\begin{entry}{创新}{chuang4xin1}{6,13}{⼑、⽄}[HSK 3]
  \definition[个,种,次]{s.}{inovação}
  \definition{v.}{trazer novas ideias; inovar; abrir novos caminhos; criar algo novo}
\end{entry}

\begin{entry}{创业}{chuang4ye4}{6,5}{⼑、⼀}[HSK 3]
  \definition{s.}{empreendedorismo}
  \definition{v.}{começar um empreendimento; iniciar um negócio, uma empresa | esculpir}
\end{entry}

\begin{entry}{创意}{chuang4yi4}{6,13}{⼑、⼼}
  \definition{adj.}{criativo}
  \definition{s.}{criatividade}
\end{entry}

\begin{entry}{创造}{chuang4zao4}{6,10}{⼑、⾡}[HSK 3]
  \definition{s.}{criação; inovação}
  \definition{v.}{criar; produzir; trazer à tona}
\end{entry}

\begin{entry}{创作}{chuang4zuo4}{6,7}{⼑、⼈}[HSK 3]
  \definition[个]{s.}{criação; trabalho criativo}
  \definition{v.}{escrever; criar; produzir; compor}
\end{entry}

\begin{entry}{吹}{chui1}{7}{⼝}[HSK 2]
  \definition{v.}{soprar | tocar (instrumentos de sopro) | bajular |  louvar aos céus | separar (casal)  | fracassar}
\end{entry}

\begin{entry}{吹牛}{chui1niu2}{7,4}{⼝、⽜}
  \definition{v.+compl.}{ogulhar-se | gabar-se | destacar-se}
\end{entry}

\begin{entry}{锤}{chui2}{13}{⾦}
  \definition{s.}{martelo | marreta}
  \definition{s.}{pesos (por exemplo, de uma balança)}
  \definition{v.}{marterlar para dar forma | atacar com um martelo}
\end{entry}

\begin{entry}{春}{chun1}{9}{⽇}
  \definition*{s.}{sobrenome Chun}
  \definition{s.}{primavera | amor | luxúria | vida | vitalidade}
\end{entry}

\begin{entry}{春季}{chun1 ji4}{9,8}{⽇、⼦}[HSK 4]
  \definition{s.}{primavera; primeiro trimestre do ano, que na China se refere ao período de três meses entre o início da primavera e o início do verão, e também se refere aos três meses do calendário lunar, a saber, o primeiro, o segundo e o terceiro meses}
\end{entry}

\begin{entry}{春节}{chun1 jie2}{9,5}{⽇、⾋}[HSK 2]
  \definition*{s.}{Festival da Primavera (Ano Novo Chinês)}
\end{entry}

\begin{entry}{春天}{chun1 tian1}{9,4}{⽇、⼤}
  \definition[个]{s.}{primavera}
\end{entry}

\begin{entry}{纯}{chun2}{7}{⽷}[HSK 4]
  \definition{adj.}{puro; não misturado; livre de impurezas | simples; puro e simples | habilidoso; proficiente; bem versado}
\end{entry}

\begin{entry}{纯净水}{chun2 jing4 shui3}{7,8,4}{⽷、⼎、⽔}[HSK 4]
  \definition{s.}{água purificada}
\end{entry}

\begin{entry}{纯真}{chun2zhen1}{7,10}{⽷、⼗}
  \definition{adj.}{inocente e não afetado | puro e não adulterado}
  \definition{s.}{inocência}
\end{entry}

\begin{entry}{唇}{chun2}{10}{⼝}
  \definition{s.}{lábios}
\end{entry}

\begin{entry}{绰号}{chuo4hao4}{11,5}{⽷、⼝}
  \definition{s.}{apelido}
\end{entry}

\begin{entry}{刺}{ci1}{8}{⼑}
  \definition{s.}{(onomatopéia) som de rasgo, fricção, etc.}
\end{entry}

\begin{entry}{词}{ci2}{7}{⾔}[HSK 2]
  \definition[个,组]{s.}{discurso | declaração | linhas de jogo | um tipo de poesia clássica chinesa, originária da Dinastia Tang e totalmente desenvolvida na Dinastia Song | palavra  | termo}
\end{entry}

\begin{entry}{词典}{ci2dian3}{7,8}{⾔、⼋}[HSK 2]
  \definition[部,本]{s.}{dicionário}
  \seealsoref{字典}{zi4 dian3}
\end{entry}

\begin{entry}{词汇}{ci2hui4}{7,5}{⾔、⽔}[HSK 4]
  \definition[个,组,批,串,堆]{s.}{vocabulário; termo geral para palavras usadas em um idioma}
\end{entry}

\begin{entry}{词语}{ci2yu3}{7,9}{⾔、⾔}[HSK 2]
  \definition{s.}{palavra (termo geral, incluindo desdemonossilábicas até frases curtas) | termo (por exemplo, termo técnico) | expressão}
\end{entry}

\begin{entry}{瓷}{ci2}{10}{⽡}
  \definition{s.}{artigos de porcelana}
\end{entry}

\begin{entry}{辞典}{ci2dian3}{13,8}{⾟、⼋}
  \variantof{词典}
\end{entry}

\begin{entry}{磁带}{ci2dai4}{14,9}{⽯、⼱}
  \definition[盘,盒]{s.}{cassete | fita magnética}
\end{entry}

\begin{entry}{磁盘}{ci2pan2}{14,11}{⽯、⽫}
  \definition{s.}{disquete}
\end{entry}

\begin{entry}{磁铁}{ci2tie3}{14,10}{⽯、⾦}
  \definition{s.}{imã | magneto}
  \seealsoref{吸铁石}{xi1tie3shi2}
\end{entry}

\begin{entry}{此}{ci3}{6}{⽌}[HSK 4]
  \definition{pron.}{esse; essa; isso; este; esta; isto | aqui e agora}
\end{entry}

\begin{entry}{此外}{ci3wai4}{6,5}{⽌、⼣}[HSK 4]
  \definition{conj.}{além disso; em adição; além das coisas ou situações mencionadas acima}
\end{entry}

\begin{entry}{次}{ci4}{6}{⽋}[HSK 1,4]
  \definition*{s.}{sobrenome Ci}
  \definition{adj.}{de segunda categoria; de qualidade inferior}
  \definition{clas.}{para coisas ou ações que podem se repetir;}
  \definition{num.}{segundo; próximo}
  \definition{pref.}{(química) hipo-, raízes ácidas ou compostos contendo dois átomos de oxigênio a menos}
  \definition{s.}{ordem; sequência; classificação | local de parada em uma viagem; escala}
\end{entry}

\begin{entry}{刺}{ci4}{8}{⼑}[HSK 4]
  \definition*{s.}{sobrenome Ci}
  \definition{s.}{espinho; farpa; algo afiado como uma agulha | cartão de visita | saliências; projeções pequenas e pontiagudas na superfície de um objeto ou na pele de uma pessoa}
  \definition{v.}{esfaquear; perfurar | irritar; estimular | assassinar | espionar; detectar | criticar}
  \seeref{刺}{ci1}
\end{entry}

\begin{entry}{刺激}{ci4ji1}{8,16}{⼑、⽔}[HSK 4]
  \definition{adj.}{animado; entusiasmado; sensação de empolgação e nervosismo}
  \definition[个]{s.}{estímulo; estimulação; fortes efeitos físicos ou psicológicos}
  \definition{v.}{irritar; provocar; estimular | incentivar; estimular; incitar; (por algum meio) para mudar as coisas para melhor, para fazer coisas positivas}
\end{entry}

\begin{entry}{刺猬}{ci4wei5}{8,12}{⼑、⽝}
  \definition{s.}{porco-espinho | ouriço}
\end{entry}

\begin{entry}{匆匆}{cong1cong1}{5,5}{⼓、⼓}
  \definition{adv.}{apressadamente}
\end{entry}

\begin{entry}{葱}{cong1}{12}{⾋}
  \definition{s.}{cebolinha}
\end{entry}

\begin{entry}{聪慧}{cong1hui4}{15,15}{⽿、⼼}
  \definition{adj.}{inteligente | brilhante}
\end{entry}

\begin{entry}{聪明}{cong1ming5}{15,8}{⽿、⽇}
  \definition{adj.}{inteligente | brilhante | esperto}
\end{entry}

\begin{entry}{从}{cong2}{4}{⼈}[HSK 1]
  \definition*{s.}{sobrenome Cong}
  \definition{prep.}{de | desde | a partir de}
\end{entry}

\begin{entry}{从不}{cong2bu4}{4,4}{⼈、⼀}
  \definition{adv.}{nunca}
\end{entry}

\begin{entry}{从此}{cong2ci3}{4,6}{⼈、⽌}[HSK 4]
  \definition{conj.}{doravante; portanto; a partir deste momento; de agora em diante; a partir de então}
\end{entry}

\begin{entry}{从而}{cong2'er2}{4,6}{⼈、⽽}
  \definition{conj.}{assim | desse modo}
\end{entry}

\begin{entry}{从来}{cong2lai2}{4,7}{⼈、⽊}[HSK 3]
  \definition{adv.}{sempre; o tempo todo; em todos os momentos}
\end{entry}

\begin{entry}{从前}{cong2qian2}{4,9}{⼈、⼑}[HSK 3]
  \definition{s.}{antes; antigamente; no passado | era uma vez; há muito tempo atrás}
\end{entry}

\begin{entry}{从事}{cong2shi4}{4,8}{⼈、⼅}[HSK 3]
  \definition{v.}{trabalhar; empreender; empenhar-se em; envolver-se em | lidar com; manusear}
\end{entry}

\begin{entry}{从未}{cong2wei4}{4,5}{⼈、⽊}
  \definition{adv.}{nunca}
\end{entry}

\begin{entry}{从小}{cong2 xiao3}{4,3}{⼈、⼩}[HSK 2]
  \definition{adv.}{desde a infância | desde muito jovem | quando criança}
\end{entry}

\begin{entry}{粗}{cu1}{11}{⽶}[HSK 4]
  \definition{adj.}{largo (em diâmetro); grosso | grosseiro; rude; áspero | áspero; rouco | descuidado; negligente | rude; sem refinamento; vulgar}
  \definition{adv.}{grosseiramente; vagamente}
\end{entry}

\begin{entry}{粗糙}{cu1cao1}{11,16}{⽶、⽶}
  \definition{adj.}{áspero | grosseiro}
\end{entry}

\begin{entry}{粗心}{cu1xin1}{11,4}{⽶、⼼}[HSK 4]
  \definition{adj.}{descuidado; irrefletido; (fazer as coisas) de forma desleixada, sem cuidado}
\end{entry}

\begin{entry}{粗心地做}{cu1xin1 di4 zuo4}{11,4,6,11}{⽶、⼼、⼟、⼈}
  \definition{adj.}{feito descuidadamente}
\end{entry}

\begin{entry}{促进}{cu4jin4}{9,7}{⼈、⾡}[HSK 4]
  \definition{v.}{impulsionar; promover; avançar; incentivar o desenvolvimento}
\end{entry}

\begin{entry}{促使}{cu4shi3}{9,8}{⼈、⼈}[HSK 4]
  \definition{v.}{incitar; estimular; impelir; causar; provocar uma mudança em alguém ou em algo}
\end{entry}

\begin{entry}{促销}{cu4 xiao1}{9,12}{⼈、⾦}[HSK 4]
  \definition{v.}{promover vendas}
\end{entry}

\begin{entry}{酢}{cu4}{12}{⾣}
  \variantof{醋}
\end{entry}

\begin{entry}{醋}{cu4}{15}{⾣}
  \definition{s.}{vinagre}
\end{entry}

\begin{entry}{窾}{cuan4}{17}{⽳}
  \definition{v.}{esconder}
  \seeref{窾}{kuan3}
\end{entry}

\begin{entry}{村}{cun1}{7}{⽊}[HSK 3]
  \definition{adj.}{rústico; grosseiro}
  \definition{s.}{aldeia; vila}
\end{entry}

\begin{entry}{存}{cun2}{6}{⼦}[HSK 3]
  \definition{v.}{existir; viver; sobreviver | armazenar; manter | acumular; coletar | depositar | sair com; verificar |reservar; reter | permanecer em equilíbrio; estar em estoque | estimar; abrigar}
\end{entry}

\begin{entry}{存在}{cun2zai4}{6,6}{⼦、⼟}[HSK 3]
  \definition{s.}{existência; ser; ente}
  \definition{v.}{existir; ser}
\end{entry}

\begin{entry}{搓}{cuo1}{12}{⼿}
  \definition{s.}{torção}
  \definition{v.}{esfregar ou rolar entre as mãos ou dedos | torcer}
\end{entry}

\begin{entry}{鹾}{cuo2}{16}{⿄}
  \definition{adj.}{salgado}
  \definition{s.}{sal}
\end{entry}

\begin{entry}{挫折}{cuo4zhe2}{10,7}{⼿、⼿}
  \definition{s.}{revés | reverso | derrota | frustração | decepção}
  \definition{v.}{frustrar | desencorajar | subjugar}
\end{entry}

\begin{entry}{措施}{cuo4shi1}{11,9}{⼿、⽅}[HSK 4]
  \definition{s.}{medida; etapa; passo; abordagem adotada para lidar com as coisas}
\end{entry}

\begin{entry}{错}{cuo4}{13}{⾦}[HSK 1]
  \definition*{s.}{sobrenome Cuo}
  \definition{adj.}{errado | enganado}
\end{entry}

\begin{entry}{错误}{cuo4wu4}{13,9}{⾦、⾔}[HSK 3]
  \definition{adj.}{equivocado; errado; errôneo}
  \definition[个,次]{s.}{engano; erro; erro grosseiro; falha}
\end{entry}

%%%%% EOF %%%%%

