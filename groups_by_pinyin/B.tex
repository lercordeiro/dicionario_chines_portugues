%%%
%%% B
%%%

\section*{B}\addcontentsline{toc}{section}{B}

\begin{entry}{八}{ba1}{2}{⼋}[HSK 1][Kangxi 12]
  \definition{num.}{oito; 8}
\end{entry}

\begin{entry}{八八六}{ba1 ba1 liu4}{2,2,4}{⼋、⼋、⼋}
  \definition{expr.}{\emph{Bye bye!} (em salas de bate-papo e mensagens de texto)}
\end{entry}

\begin{entry}{巴勒斯坦}{ba1le4si1tan3}{4,11,12,8}{⼰、⼒、⽄、⼟}
  \definition*{s.}{Palestina}
\end{entry}

\begin{entry}{巴士}{ba1 shi4}{4,3}{⼰、⼠}[HSK 4]
  \definition[辆]{s.}{ônibus; transliteração da palavra inglesa ``\emph{bus}''}
\end{entry}

\begin{entry}{巴西}{ba1xi1}{4,6}{⼰、⾑}
  \definition*{s.}{Brasil}
\end{entry}

\begin{entry}{巴西人}{ba1xi1ren2}{4,6,2}{⼰、⾑、⼈}
  \definition[个,位]{s.}{brasileiro | pessoa ou povo do Brasil}[他是巴西人。(Ele é brasileiro.)]
\end{entry}

\begin{entry}{巴西战舞}{ba1xi1zhan4wu3}{4,6,9,14}{⼰、⾑、⼽、⾇}
  \definition{s.}{capoeira}
\end{entry}

\begin{entry}{吧}{ba1}{7}{⼝}
  \definition{s.}{som de estalo, som crepitante}
  \definition{v.}{puxar o cachimbo; fumar | abreviação de ``bar''}
  \seeref{吧}{ba5}
\end{entry}

\begin{entry}{拔}{ba2}{8}{⼿}[HSK 5]
  \definition{v.aux.}{puxar para cima; puxar para fora; arrastar para fora |extrair; sugar | escolher; selecionar | superar; destacar-se entre | apreender; capturar | esfriar na água; mergulhar algo em água fria para que esfrie}
\end{entry}

\begin{entry}{拔尖}{ba2jian1}{8,6}{⼿、⼩}
  \definition{adj.}{topo de linha | fora do comum | o melhor}
  \definition{v.+compl.}{empurrar-se para a frente | sentir que é superior aos outros}
\end{entry}

\begin{entry}{把}{ba3}{7}{⼿}[HSK 3]
  \definition{clas.}{para objetos com alça | para objetos pequenos:~punhado}
  \definition{part.}{partícula tornando o substantivo seguinte um objeto direto}
  \definition{v.}{conter | alcançar | segurar}
  \seeref{把}{ba4}
\end{entry}

\begin{entry}{把柄}{ba3bing3}{7,9}{⼿、⽊}
  \definition{s.}{(figurativo) informações que podem ser usadas contra alguém}
\end{entry}

\begin{entry}{把持}{ba3chi2}{7,9}{⼿、⼿}
  \definition{v.}{controlar | dominar | monopolizar}
\end{entry}

\begin{entry}{把风}{ba3feng1}{7,4}{⼿、⾵}
  \definition{v.}{estar atento | vigiar (durante uma atividade clandestina)}
\end{entry}

\begin{entry}{把关}{ba3guan1}{7,6}{⼿、⼋}
  \definition{v.}{verificar estritamente | examinar cuidadosamente para ver se algo é feito de acordo com um padrão fixo | fazer a verificação final | guardar uma passagem, fronteira}
\end{entry}

\begin{entry}{把脉}{ba3mai4}{7,9}{⼿、⾁}
  \definition{v.}{sentir ou tomar o pulso de alguém}
\end{entry}

\begin{entry}{把式}{ba3shi4}{7,6}{⼿、⼷}
  \definition{s.}{pessoa qualificada em um comércio}
\end{entry}

\begin{entry}{把守}{ba3shou3}{7,6}{⼿、⼧}
  \definition{v.}{vigiar | guardar}
\end{entry}

\begin{entry}{把玩}{ba3wan2}{7,8}{⼿、⽟}
  \definition{v.}{brincar com | mexer com}
\end{entry}

\begin{entry}{把稳}{ba3wen3}{7,14}{⼿、⽲}
  \definition{adj.}{confiável}
\end{entry}

\begin{entry}{把握}{ba3wo4}{7,12}{⼿、⼿}[HSK 3]
  \definition{s.}{seguro | garantia | certeza}
  \definition{v.}{agarrar | segurar | aproveitar}
\end{entry}

\begin{entry}{把戏}{ba3xi4}{7,6}{⼿、⼽}
  \definition{s.}{acrobacia | malabarismo | truque barato}
\end{entry}

\begin{entry}{把}{ba4}{7}{⼿}
  \definition{v.}{lidar}
  \seeref{把}{ba3}
\end{entry}

\begin{entry}{爸}{ba4}{8}{⽗}[HSK 1]
  \definition[个,位]{s.}{(informal) pai}
  \seealsoref{爸爸}{ba4ba5}
\end{entry}

\begin{entry}{爸爸}{ba4ba5}{8,8}{⽗、⽗}[HSK 1]
  \definition[个,位,名,群]{s.}{(informal) pai; papai; papa}
  \seealsoref{爸}{ba4}
\end{entry}

\begin{entry}{爸妈}{ba4ma1}{8,6}{⽗、⼥}
  \definition{s.}{pai e mãe}
\end{entry}

\begin{entry}{罢}{ba4}{10}{⽹}
  \definition{v.}{parar | cessar | demitir | suspender | desistir | terminar}
  \seeref{罢}{ba5}
\end{entry}

\begin{entry}{霸权}{ba4quan2}{21,6}{⾬、⽊}
  \definition{s.}{hegemonia | supremacia}
\end{entry}

\begin{entry}{吧}{ba5}{7}{⼝}[HSK 1]
  \definition{part.}{indica discussão, sugestão, solicitação ou comando no final de uma frase | indica concordância ou aprovação no final de uma frase | indica uma pergunta ou especulação no final de uma frase | indica incerteza no final de uma frase | em uma frase, indica uma pausa, carrega um tom hipotético, frequentemente apresenta um contraste e implica um dilema}
  \seeref{吧}{ba1}
\end{entry}

\begin{entry}{罢}{ba5}{10}{⽹}
  \definition{part.}{partícula final, a mesma que 吧}
  \seeref{罢}{ba4}
  \seealsoref{吧}{ba5}
\end{entry}

\begin{entry}{白}{bai2}{5}{⽩}[HSK 1,3][Kangxi 106]
  \definition*{s.}{sobrenome Bai}
  \definition{adj.}{branco | claro | puro; claro; simples; sem mistura; em branco | branco (como símbolo de reação) | escrito incorretamente ou pronunciado incorretamente | grátis; sem custos}
  \definition{adv.}{em vão; sem propósito; sem resultados}
  \definition{s.}{parte falada em ópera, etc.; frases de peças de teatro, etc. | dialeto local | funeral}
  \definition{v.}{explicar; apresentar; esclarecer; declarar | branquear | olhar para as pessoas com o branco dos olhos (olhar vazio, de desaprovação)}
\end{entry}

\begin{entry}{白菜}{bai2 cai4}{5,11}{⽩、⾋}[HSK 3]
  \definition[棵,个]{s.}{acelga | repolho chinês}
\end{entry}

\begin{entry}{白痴}{bai2chi1}{5,13}{⽩、⽧}
  \definition{adj./s.}{estúpido | imbecil}
\end{entry}

\begin{entry}{白蛋白}{bai2dan4bai2}{5,11,5}{⽩、⾍、⽩}
  \definition{s.}{albumina}
\end{entry}

\begin{entry}{白鹄}{bai2hu2}{5,12}{⽩、⿃}
  \definition{s.}{cisne branco}
\end{entry}

\begin{entry}{白拣}{bai2jian3}{5,8}{⽩、⼿}
  \definition{s.}{uma escolha barata}
  \definition{v.}{escolher algo que não custa nada}
\end{entry}

\begin{entry}{白酒}{bai2 jiu3}{5,10}{⽩、⾣}[HSK 5]
  \definition{s.}{aguardente branca; aguardente (geralmente destilada de sorgo ou milho); bebidas destiladas tradicionais chinesas, feitas de sorgo, milho, etc., transparentes e incolores, com alto teor alcoólico}
\end{entry}

\begin{entry}{白萝卜}{bai2luo2bo5}{5,11,2}{⽩、⾋、⼘}
  \definition{s.}{rabanete branco}
\end{entry}

\begin{entry}{白色}{bai2 se4}{5,6}{⽩、⾊}[HSK 2]
  \definition{s.}{a cor branca}
\end{entry}

\begin{entry}{白天}{bai2 tian1}{5,4}{⽩、⼤}[HSK 1]
  \definition{adv.}{dia;  de dia}
  \definition[个]{s.}{dia; horário diurno; durante o dia}
\end{entry}

\begin{entry}{白苋}{bai2xian4}{5,7}{⽩、⾋}
  \definition{s.}{amaranto branco | brotos e folhas tenras de espinafre chinês usados como alimento}
\end{entry}

\begin{entry}{百}{bai3}{6}{⽩}[HSK 1]
  \definition*{s.}{sobrenome Bai}
  \definition{adj.}{multifacetado; numeroso}
  \definition{adv.}{muito; sempre}
  \definition{num.}{cem; 100 | centena; cento}
\end{entry}

\begin{entry}{百般}{bai3ban1}{6,10}{⽩、⾈}
  \definition{adv.}{de todas as maneiras possíveis | por todos os meios}
\end{entry}

\begin{entry}{百分}{bai3fen1}{6,4}{⽩、⼑}
  \definition{num.}{por cento}
  \definition{s.}{porcentagem}
\end{entry}

\begin{entry}{百货}{bai3 huo4}{6,8}{⽩、⾙}[HSK 4]
  \definition{s.}{mercadorias em geral; loja de departamentos; um termo geral para bens que incluem principalmente roupas, utensílios e necessidades diárias gerais}
\end{entry}

\begin{entry}{柏树}{bai3shu4}{9,9}{⽊、⽊}
  \definition{s.}{cipreste}
\end{entry}

\begin{entry}{摆}{bai3}{13}{⼿}[HSK 4]
  \definition*{s.}{sobrenome Bai}
  \definition*{s.}{Festival de Ganbai; uma reunião realizada nas áreas Dai durante festivais religiosos, para celebrar uma boa colheita ou para trocar materiais; geralmente se refere a uma reunião em massa}
  \definition{s.}{pêndulo; dispositivo mecânico que controla a frequência de oscilação em relógios e instrumentos |  a bainha inferior de um vestido, jaqueta ou saia}
  \definition{v.}{colocar; posicionar; organizar | assumir; mostrar intencionalmente | balançar; ondular; balançar para frente e para trás | revelar; listar; afirmar claramente | dizer; falar; declarar | libertar-se; livrar-se}
\end{entry}

\begin{entry}{摆动}{bai3 dong4}{13,6}{⼿、⼒}[HSK 4]
  \definition{v.}{balançar; balançar para frente e para trás; oscilar; vibrar}
\end{entry}

\begin{entry}{摆烂}{bai3lan4}{13,9}{⼿、⽕}
  \definition{v.}{(neologismo, gíria) parar de lutar (especialmente quando se sabe que não pode ter sucesso) | deixar tudo ir para o inferno}
\end{entry}

\begin{entry}{摆手}{bai3shou3}{13,4}{⼿、⼿}
  \definition{v.+compl.}{gesticular com a mão (acenando, acenando adeus, etc.) | balançar os braços | acenar com as mãos}
\end{entry}

\begin{entry}{摆脱}{bai3tuo1}{13,11}{⼿、⾁}[HSK 4]
  \definition{v.}{sacudir; rejeitar; romper com; libertar-se (ou desembaraçar-se) de; livrar-se de dificuldades, escravidão, controle, etc.}
\end{entry}

\begin{entry}{败}{bai4}{8}{⾒}[HSK 4]
  \definition{adj.}{dilapidado; decadente; murcho; em declínio}
  \definition{v.}{derrota; bater | falhar | quebrar; neutralizar; dissipar | arruinar; estragar; corromper | ser derrotado; perder}
\end{entry}

\begin{entry}{拜访}{bai4fang3}{9,6}{⼿、⾔}[HSK 5]
  \definition{v.}{visitar; fazer uma visita (respeitosamente)}
\end{entry}

\begin{entry}{班}{ban1}{10}{⽟}[HSK 1]
  \definition*{s.}{sobrenome Ban}
  \definition{adj.}{regular; programado; executado regularmente; com horários fixos (meios de transporte)}
  \definition{clas.}{um grupo de; uma classe de; usado para pessoas | meios de transporte com horários fixos}
  \definition[个]{s.}{equipe; turma; organização estruturada | dever; turno; período de trabalho dentro de um dia | equipe; esquadrão; unidade básica das forças armadas | nome usado antigamente para designar uma companhia teatral}
  \definition{v.}{mover; implantar; implementar}
\end{entry}

\begin{entry}{班级}{ban1 ji2}{10,6}{⽟、⽷}[HSK 3]
  \definition[个]{s.}{classe | série (na escola)}
\end{entry}

\begin{entry}{班长}{ban1 zhang3}{10,4}{⽟、⾧}[HSK 2]
  \definition[个,位,名]{s.}{monitor de turma; líder de equipe; alunos responsáveis nas turmas da escola | líder de esquadrão; responsável por uma turma de soldados, geralmente com patente de sargento}
\end{entry}

\begin{entry}{般}{ban1}{10}{⾈}
  \definition{s.}{espécie | tipo | classe | caminho | maneira}
  \seeref{般}{bo1}
  \seeref{般}{pan2}
\end{entry}

\begin{entry}{搬}{ban1}{13}{⼿}[HSK 3]
  \definition{v.}{copiar indiscriminadamente | mover-se (ou seja, mudar-se) | mover-se (algo relativamente pesado ou volumoso) | mudar | mudar-se}
\end{entry}

\begin{entry}{搬动}{ban1dong4}{13,6}{⼿、⼒}
  \definition{v.}{mover-se (alguma coisa) | se mudar}
\end{entry}

\begin{entry}{搬家}{ban1jia1}{13,10}{⼿、⼧}[HSK 3]
  \definition{s.}{mudança}
  \definition{v.+compl.}{mudar-se de casa}
\end{entry}

\begin{entry}{搬口}{ban1kou3}{13,3}{⼿、⼝}
  \definition{v.}{tagarelar | (idioma) transmitir histórias | semear dissensão | contar histórias}
\end{entry}

\begin{entry}{搬弄}{ban1nong4}{13,7}{⼿、⼶}
  \definition{v.}{causar problemas | mexer com alguém | mostrar (o que se pode fazer)}
\end{entry}

\begin{entry}{搬运}{ban1yun4}{13,7}{⼿、⾡}
  \definition{s.}{frete | transporte}
  \definition{v.}{carregar | transportar}
\end{entry}

\begin{entry}{搬走}{ban1zou3}{13,7}{⼿、⾛}
  \definition{v.}{carregar}
\end{entry}

\begin{entry}{板}{ban3}{8}{⽊}[HSK 3]
  \definition{adj.}{rígido; não natural | duro}
  \definition{clas.}{para cartões, papéis}
  \definition{s.}{tábua; placa; prato | veneziana; persiana; refere-se especificamente aos painéis de portas de lojas | badalos (instrumento musical que marca o ritmo) | uma batida acentuada (ritmo) na música e na ópera tradicional | chefe}
  \definition{v.}{parecer sério | livrar-se de maus hábitos ou falhas}
\end{entry}

\begin{entry}{版}{ban3}{8}{⽚}[HSK 5]
  \definition{clas.}{usado como uma palavra de medida para materiais impressos (por exemplo, livros, jornais, edições)}
  \definition{s.}{chapa, placa ou bloco de impressão | edição (livros impressos) | página (de um jornal) | moldes ou fromas de construção}
\end{entry}

\begin{entry}{办}{ban4}{4}{⼒}[HSK 2]
  \definition{v.}{fazer; lidar com; gerenciar; cuidar de | executar; configurar | preparar algo; comprar uma quantidade razoável de | punir; levar à justiça; punir com medidas}
\end{entry}

\begin{entry}{办法}{ban4fa3}{4,8}{⼒、⽔}[HSK 2]
  \definition[个,种]{s.}{método; meio; medida; caminho; maneira; método de lidar com situações ou resolver problemas}
\end{entry}

\begin{entry}{办公}{ban4gong1}{4,4}{⼒、⼋}
  \definition{v.+compl.}{lidar com negócios oficiais | trabalhar (especialmente em um escritório)}
\end{entry}

\begin{entry}{办公室}{ban4gong1shi4}{4,4,9}{⼒、⼋、⼧}[HSK 2]
  \definition[个,间]{s.}{órgãos, escolas, grupos, empresas e outras entidades que lidam com assuntos administrativos cotidianos | escritório; sala de escritório}
\end{entry}

\begin{entry}{办理}{ban4li3}{4,11}{⼒、⽟}[HSK 3]
  \definition{v.}{conduzir | manusear | transacionar}
\end{entry}

\begin{entry}{办事}{ban4 shi4}{4,8}{⼒、⼅}[HSK 4]
  \definition{v.}{trabalhar | lidar com assuntos; manipular transações}
\end{entry}

\begin{entry}{半}{ban4}{5}{⼗}[HSK 1]
  \definition{adj.}{incompleto}
  \definition{adv.}{parcialmente; indica incompletude}
  \definition{num.}{(depois de um número) ``e meio''}
  \definition{pref.}{semi-}
  \definition{s.}{metade | na metade; no meio | muito pouco; a menor parte}
\end{entry}

\begin{entry}{半年}{ban4 nian2}{5,6}{⼗、⼲}[HSK 1]
  \definition{s.}{meio ano}
\end{entry}

\begin{entry}{半球}{ban4qiu2}{5,11}{⼗、⽟}
  \definition{s.}{hemisfério}
\end{entry}

\begin{entry}{半天}{ban4 tian1}{5,4}{⼗、⼤}[HSK 1]
  \definition{s.}{metade do dia; metade do dia dividida pelo meio-dia | um longo tempo; bastante tempo; refere-se a um período de tempo relativamente longo (com um tom exagerado)}
\end{entry}

\begin{entry}{半夜}{ban4 ye4}{5,8}{⼗、⼣}[HSK 2]
  \definition{s.}{no meio da noite; metade da noite | por volta da meia-noite, também se refere à madrugada}
\end{entry}

\begin{entry}{半音}{ban4yin1}{5,9}{⼗、⾳}
  \definition{s.}{semitom}
\end{entry}

\begin{entry}{伴侣}{ban4lv3}{7,8}{⼈、⼈}
  \definition{s.}{companheiro | parceiro}
\end{entry}

\begin{entry}{扮演}{ban4yan3}{7,14}{⼿、⽔}[HSK 5]
  \definition{v.}{desempenhar o papel de; ter um papel (em uma peça, etc.); atuar}
\end{entry}

\begin{entry}{帮}{bang1}{9}{⼱}[HSK 1]
  \definition*{s.}{sobrenome Bang}
  \definition{clas.}{um grupo de; um bando de; uma gangue de; grupo de pessoas}
  \definition{s.}{lateral; superior; partes ao lado ou ao redor do objeto | folha externa; parte mais grossa das folhas externas dos vegetais | gangue; banda; grupo; conglomerado}
  \definition{v.}{ajudar; assistir; auxiliar | trabalho; refere-se ao trabalho assalariado}
\end{entry}

\begin{entry}{帮教}{bang1jiao4}{9,11}{⼱、⽁}
  \definition{v.}{orientar}
\end{entry}

\begin{entry}{帮忙}{bang1 mang2}{9,6}{⼱、⼼}[HSK 1]
  \definition{v.+compl.}{ajudar; dar uma mão; dar uma mãozinha; fazer um favor; fazer uma boa ação; ajudar os outros a fazer algo, referindo-se, de maneira geral, a oferecer ajuda quando alguém está com dificuldades}
\end{entry}

\begin{entry}{帮佣}{bang1yong1}{9,7}{⼱、⼈}
  \definition{s.}{ajudante doméstico | servo}
\end{entry}

\begin{entry}{帮助}{bang1zhu4}{9,7}{⼱、⼒}[HSK 2]
  \definition[个,次,回,份,种]{s.}{ajuda; auxílio; socorro; função de promoção ou auxílio}
  \definition{v.}{ajudar; assistir; apoiar; quando alguém está passando por dificuldades, oferecer apoio financeiro ou material, ou ainda apoio moral, dar conselhos, pensar em soluções, fazer coisas por essa pessoa, etc.}
\end{entry}

\begin{entry}{棒}{bang4}{12}{⽊}[HSK 5]
  \definition{adj.}{bom; forte; excelente}
  \definition[根]{s.}{porrete; vara; bastão; cacete; haste}
\end{entry}

\begin{entry}{棒棒糖}{bang4bang4tang2}{12,12,16}{⽊、⽊、⽶}
  \definition[根]{s.}{pirulito}
\end{entry}

\begin{entry}{棒冰}{bang4bing1}{12,6}{⽊、⼎}
  \definition{s.}{picolé}
\end{entry}

\begin{entry}{包}{bao1}{5}{⼓}[HSK 1]
  \definition*{s.}{sobrenome Bao}
  \definition{clas.}{pacote; embalagem; embrulho; para coisas empacotadas}
  \definition[个,只]{s.}{feixe; pacote; encomenda; algo embrulhado | saco; sacola; saco para guardar coisas | caroço; inchaço; protuberância; inchaço ou protuberância no corpo ou em objetos | tenda; tenda com cúpula feita de feltro}
  \definition{v.}{embrulhar; envolver com papel, tecido, etc. | cercar; rodear; envolver; envelopar | incluir; conter | realizar todo o processo; assumir toda a responsabilidade | assegurar; garantir | contratar; reservar; fretar; comprar ou alugar tudo; acordar uso exclusivo}
\end{entry}

\begin{entry}{包办}{bao1ban4}{5,4}{⼓、⼒}
  \definition{v.}{comandar todo o show | comprometer-se a fazer tudo sozinho}
\end{entry}

\begin{entry}{包干}{bao1gan1}{5,3}{⼓、⼲}
  \definition{s.}{tarefa alocada}
  \definition{v.}{ter a responsabilidade total sobre um trabalho}
\end{entry}

\begin{entry}{包裹}{bao1guo3}{5,14}{⼓、⾐}[HSK 4]
  \definition[个]{s.}{pacote; embrulho}
  \definition{v.}{embrulhar; amarrar; enrolar coisas em pano ou outra coisa}
\end{entry}

\begin{entry}{包含}{bao1han2}{5,7}{⼓、⼝}[HSK 4]
  \definition{v.}{conter; implicar; incluir; conter dentro, resumir, enfatizar o que está contido dentro, focar em relações internas, muitas vezes coisas abstratas}
\end{entry}

\begin{entry}{包括}{bao1kuo4}{5,9}{⼓、⼿}[HSK 4]
  \definition{v.}{incluir; compreender; consistir em; conter, conter dentro, resumir junto, enfatizar a listagem de todas as partes, ou a citação de uma parte delas, que podem ser coisas abstratas ou concretas}
\end{entry}

\begin{entry}{包容}{bao1rong2}{5,10}{⼓、⼧}
  \definition{adj.}{inclusivo}
  \definition{v.}{perdoar | mostrar tolerância | conter | segurar}
\end{entry}

\begin{entry}{包围}{bao1wei2}{5,7}{⼓、⼞}[HSK 5]
  \definition{v.}{circundar; cercar; rodear}
\end{entry}

\begin{entry}{包装}{bao1zhuang1}{5,12}{⼓、⾐}[HSK 5]
  \definition[个,款]{s.}{embalagem; materiais usados para embalar produtos, como papel, sacolas, garrafas ou caixas}
  \definition{v.}{embalar; embrulhar; empacotar |
aumentar a fama e o apelo de alguém ou algo por meio de publicidade | tornar alguém ou algo mais comercialmente viável ou atraente por meio de embelezamento ou publicidade}
\end{entry}

\begin{entry}{包子}{bao1 zi5}{5,3}{⼓、⼦}[HSK 1]
  \definition[个]{s.}{pão recheado cozido no vapor; alimentos, com recheio de vegetais, carne ou açúcar, etc., com massa levedada como invólucro, embrulhados e cozidos no vapor}
\end{entry}

\begin{entry}{包租}{bao1zu1}{5,10}{⼓、⽲}
  \definition{s.}{aluguel fixo para terras agrícolas}
  \definition{v.}{fretar | alugar | alugar um terreno ou uma casa para subarrendar}
\end{entry}

\begin{entry}{薄}{bao2}{16}{⾋}[HSK 4]
  \definition{adj.}{fino; frágil; pouca espessura |  frio; indiferente; carente de calor; emocionalmente frio; não profundo | leve; fraco | pobre; infértil}
  \seeref{薄}{bo2}
\end{entry}

\begin{entry}{宝}{bao3}{8}{⼧}[HSK 4]
  \definition*{s.}{sobrenome Bao}
  \definition{adj.}{antigo; precioso; estimado}
  \definition[个,件]{s.}{tesouro; objeto estimado; coisa preciosa | dispositivo de jogo; ferramenta de jogo | dinheiro; moeda; moeda antiga com furo quadrado no centro; moeda de prata}
\end{entry}

\begin{entry}{宝宝}{bao3 bao5}{8,8}{⼧、⼧}[HSK 4]
  \definition[个]{s.}{querida; \emph{darling}; \emph{baby}; apelido para crianças}
\end{entry}

\begin{entry}{宝贝}{bao3bei4}{8,4}{⼧、⾙}[HSK 4]
  \definition{adj.}{excêntrico; estranho; imprestável; um termo depreciativo para uma pessoa incompetente ou ridícula}
  \definition[个,件]{s.}{tesouro; objeto estimado; coisa preciosa | querida; \emph{darling}; \emph{baby}; apelido para crianças}
\end{entry}

\begin{entry}{宝贵}{bao3gui4}{8,9}{⼧、⾙}[HSK 4]
  \definition{adj.}{precioso; extremamente valioso, muito raro, pode ser usado para descrever coisas específicas, também pode ser usado para descrever coisas abstratas | valioso; como um tesouro}
\end{entry}

\begin{entry}{宝石}{bao3 shi2}{8,5}{⼧、⽯}[HSK 4]
  \definition[颗,枚,块]{s.}{gema; jóia; pedra preciosa; mineral precioso que tem um brilho lindo e uma dureza de mais de sete graus, não é afetado pela atmosfera ou por produtos químicos e pode ser usado como decoração, suporte de instrumentos ou abrasivos}
\end{entry}

\begin{entry}{饱}{bao3}{8}{⾷}[HSK 2]
  \definition{adj.}{cheio; comer até ficar satisfeito | cheio; rechonchudo}
  \definition{adv.}{totalmente; completamente; plenamente}
  \definition{v.}{satisfazer}
\end{entry}

\begin{entry}{保}{bao3}{9}{⼈}[HSK 3]
  \definition*{s.}{sobrenome Bao}
  \definition{s.}{fiador
oficial responsável
sistema administrativo}
  \definition{v.}{defender | proteger |manter | preservar | conservar em boas condições | garantir | assegurar | ficar como fiador de alguém.}
\end{entry}

\begin{entry}{保安}{bao3 an1}{9,6}{⼈、⼧}[HSK 3]
  \definition{s.}{guarda de segurança}
  \definition{v.}{manter seguro | garantir a segurança}
\end{entry}

\begin{entry}{保持}{bao3chi2}{9,9}{⼈、⼿}[HSK 3]
  \definition{v.}{manter | segurar | reter | preservar}
\end{entry}

\begin{entry}{保存}{bao3cun2}{9,6}{⼈、⼦}[HSK 3]
  \definition{v.}{conservar | preservar | (computação) salvar (um arquivo, etc.)}
\end{entry}

\begin{entry}{保护}{bao3hu4}{9,7}{⼈、⼿}[HSK 3]
  \definition{s.}{proteção | salvaguarda}
  \definition{v.}{proteger | defender | salvaguardar}
\end{entry}

\begin{entry}{保护国}{bao3hu4guo2}{9,7,8}{⼈、⼿、⼞}
  \definition{s.}{protetorado}
\end{entry}

\begin{entry}{保护剂}{bao3hu4ji4}{9,7,8}{⼈、⼿、⼑}
  \definition{s.}{agente protetor}
\end{entry}

\begin{entry}{保护区}{bao3hu4qu1}{9,7,4}{⼈、⼿、⼖}
  \definition{s.}{área protegida | área de conservação}
\end{entry}

\begin{entry}{保护色}{bao3hu4se4}{9,7,6}{⼈、⼿、⾊}
  \definition{s.}{camuflagem}
\end{entry}

\begin{entry}{保护神}{bao3hu4shen2}{9,7,9}{⼈、⼿、⽰}
  \definition{s.}{anjo da guarda | santo patrono}
\end{entry}

\begin{entry}{保护物}{bao3hu4 wu4}{9,7,8}{⼈、⼿、⽜}
  \definition{s.}{protetor}
\end{entry}

\begin{entry}{保护性}{bao3hu4xing4}{9,7,8}{⼈、⼿、⼼}
  \definition{s.}{proteção}
\end{entry}

\begin{entry}{保护者}{bao3hu4zhe3}{9,7,8}{⼈、⼿、⽼}
  \definition{s.}{protetor | segurador}
\end{entry}

\begin{entry}{保护主义}{bao3hu4zhu3yi4}{9,7,5,3}{⼈、⼿、⼂、⼂}
  \definition{s.}{protecionismo}
\end{entry}

\begin{entry}{保留}{bao3liu2}{9,10}{⼈、⽥}[HSK 3]
  \definition{v.}{reter | continuar a ter | segurar | reservar}
\end{entry}

\begin{entry}{保密}{bao3mi4}{9,11}{⼈、⼧}[HSK 4]
  \definition{v.}{manter segredo; manter algo em segredo; manter a confidencialidade}
\end{entry}

\begin{entry}{保守}{bao3shou3}{9,6}{⼈、⼧}[HSK 4]
  \definition{adj.}{retrógrado; conservador; pensamentos e conceitos que são retrógrados e não conseguem acompanhar o desenvolvimento da situação}
  \definition{v.}{manter; guardar; evitar perder}
\end{entry}

\begin{entry}{保卫}{bao3wei4}{9,3}{⼈、⼙}[HSK 5]
  \definition{v.}{defender; proteger; salvaguardar}
\end{entry}

\begin{entry}{保险}{bao3xian3}{9,9}{⼈、⾩}[HSK 3]
  \definition[个]{adj./s.}{seguro}
  \definition{v.}{ter certeza | estar vinculado a}
\end{entry}

\begin{entry}{保养}{bao3yang3}{9,9}{⼈、⼋}[HSK 5]
  \definition{v.}{preservar; cuidar bem (ou conservar) da saúde |  fazer manutenção; conservar; manter; manter em bom estado de conservação}
\end{entry}

\begin{entry}{保证}{bao3zheng4}{9,7}{⼈、⾔}[HSK 3]
  \definition[个]{s.}{garantia}
  \definition{v.}{garantir}
\end{entry}

\begin{entry}{报}{bao4}{7}{⼿}[HSK 3]
  \definition[份,张]{s.}{jornal | recompensa | relatório | vingança}
  \definition{v.}{anunciar | informar}
\end{entry}

\begin{entry}{报酬}{bao4chou5}{7,13}{⼿、⾣}
  \definition{s.}{recompensa | remuneração}
\end{entry}

\begin{entry}{报答}{bao4da2}{7,12}{⼿、⽵}[HSK 5]
  \definition{v.}{reembolsar; devolver; retribuir; pagar de volta; mostrar seu apreço de forma tangível}
\end{entry}

\begin{entry}{报到}{bao4dao4}{7,8}{⼿、⼑}[HSK 3]
  \definition{v.+compl.}{apresentar-se para o serviço | fazer check-in | registrar-se | assinar}
\end{entry}

\begin{entry}{报道}{bao4dao4}{7,12}{⼿、⾡}[HSK 3]
  \definition[个,篇,分]{s.}{história | reportagem}
  \definition{v.}{cobrir | relatar (notícias)}
\end{entry}

\begin{entry}{报告}{bao4gao4}{7,7}{⼿、⼝}[HSK 3]
  \definition[份,篇,分,个,通]{s.}{relatório | discurso | palestra | aconselhamento}
  \definition{v.}{relatar | dar a conhecer | informar}
\end{entry}

\begin{entry}{报警}{bao4jing3}{7,19}{⼿、⾔}[HSK 5]
  \definition{v.}{relatar (um incidente) à polícia; relatar uma situação crítica ou sinalizar uma emergência às autoridades competentes}
\end{entry}

\begin{entry}{报名}{bao4ming2}{7,6}{⼿、⼝}[HSK 2]
  \definition{v.+compl.}{inscrever-se; alistar-se; registrar seu nome; cadastrar-se; matricular-se; informar seu nome à pessoa responsável, órgão, grupo etc., indicando que você deseja participar de alguma atividade ou organização}
\end{entry}

\begin{entry}{报纸}{bao4zhi3}{7,7}{⼿、⽷}[HSK 2]
  \definition[分,期,张]{s.}{jornal; publicações periódicas cujo conteúdo principal é notícias, geralmente referem-se a jornais diários | papel jornal; um tipo de papel usado para imprimir jornais ou publicações em geral}
\end{entry}

\begin{entry}{抱}{bao4}{8}{⼿}[HSK 4]
  \definition*{s.}{sobrenome Bao}
  \definition{clas.}{braçada; medida dos dois braços}
  \definition{v.}{carregar no peito; segurar com ambos os braços; abraçar | ter o primeiro filho ou neto | adotar um bebê ou criança | ficar juntos, unidos | encaixar ou servir perfeitamente (roupas e sapatos do tamanho certo) | estimar; nutrir; abrigar; ter em mente | continuar; sobrecarregar com | chocar ovos}
\end{entry}

\begin{entry}{抱怨}{bao4yuan4}{8,9}{⼿、⼼}[HSK 5]
  \definition{v.}{reclamar ou expressar descontentamento ou insatisfação; falar com os outros sobre pessoas ou coisas com as quais você não está satisfeito}
\end{entry}

\begin{entry}{豹子}{bao4zi5}{10,3}{⾘、⼦}
  \definition[头]{s.}{leopardo}
\end{entry}

\begin{entry}{暴力}{bao4li4}{15,2}{⽇、⼒}
  \definition{adj.}{violento}
  \definition{s.}{violência}
\end{entry}

\begin{entry}{暴乱}{bao4luan4}{15,7}{⽇、⼄}
  \definition{s.}{rebelião | revolta | tumulto}
\end{entry}

\begin{entry}{暴行}{bao4xing2}{15,6}{⽇、⾏}
  \definition{s.}{ato selvagem | atrocidade | indignação}
\end{entry}

\begin{entry}{暴雨}{bao4yu3}{15,8}{⽇、⾬}
  \definition[场,阵]{s.}{tempestade | chuva torrencial}
\end{entry}

\begin{entry}{暴躁}{bao4zao4}{15,20}{⽇、⾜}
  \definition{adj.}{irascível | irritável}
\end{entry}

\begin{entry}{爆米花}{bao4mi3hua1}{19,6,7}{⽕、⽶、⾋}
  \definition{s.}{pipoca (de milho) | pipoca de arroz}
\end{entry}

\begin{entry}{爆炸}{bao4zha4}{19,9}{⽕、⽕}
  \definition{s.}{explosão}
  \definition{v.}{explodir | detonar}
\end{entry}

\begin{entry}{杯}{bei1}{8}{⽊}[HSK 1]
  \definition{clas.}{para certos recipientes de líquidos: copo, xícara, etc.}
  \definition[只,个]{s.}{copo; caneca; xícara | taça; troféu; prêmio em forma de taça}
\end{entry}

\begin{entry}{杯具}{bei1ju4}{8,8}{⽊、⼋}
  \definition{s.}{parachoque | fiasco | (gíria) tragédia}
\end{entry}

\begin{entry}{杯子}{bei1 zi5}{8,3}{⽊、⼦}[HSK 1]
  \definition[个,只,种]{s.}{xícara; copo; recipiente para bebidas ou outros líquidos, geralmente cilíndrico ou com a parte inferior ligeiramente mais estreita, com capacidade geralmente pequena}
\end{entry}

\begin{entry}{背}{bei1}{9}{⾁}[HSK 2]
  \definition{clas.}{carga; pacote; para transportar coisas nas costas}
  \definition{v.}{carregar nas costas | suportar; carregar}
  \seeref{背}{bei4}
\end{entry}

\begin{entry}{背包}{bei1 bao1}{9,5}{⾁、⼓}[HSK 5]
  \definition[个]{s.}{mochila; mochila de ataque; mochila de infantaria; pacotes de roupas carregados nas costas quando marcham}
\end{entry}

\begin{entry}{悲剧}{bei1 ju4}{12,10}{⽕、⼑}[HSK 5]
  \definition[部,出]{s.}{tragédia; drama trágico; uma das principais categorias de teatro, caracterizada basicamente pela representação do conflito irreconciliável entre o protagonista e a realidade e seu final trágico | tragédia; evento triste; metáfora para encontro infeliz}
\end{entry}

\begin{entry}{悲伤}{bei1 shang1}{12,6}{⽕、⼈}[HSK 5]
  \definition{adj.}{triste; pesaroso}
\end{entry}

\begin{entry}{北}{bei3}{5}{⼔}[HSK 1]
  \definition*{s.}{sobrenome Bei}
  \definition*{s.}{Norte (os países desenvolvidos)}
  \definition*{s.}{Especifica a parte norte da China}
  \definition{s.}{norte; uma das quatro direções básicas, a esquerda quando se está de frente para o sol pela manhã (oposta ao 南)}
  \definition{v.}{ser derrotado}
  \seealsoref{南}{nan2}
\end{entry}

\begin{entry}{北边}{bei3 bian1}{5,5}{⼔、⾡}[HSK 1]
  \definition{s.}{norte; o lado norte}
\end{entry}

\begin{entry}{北部}{bei3 bu4}{5,10}{⼔、⾢}[HSK 3]
  \definition{s.}{parte norte}
\end{entry}

\begin{entry}{北大西洋公约组织}{bei3 da4xi1 yang2 gong1 yue1 zu3zhi1}{5,3,6,9,4,6,8,8}{⼔、⼤、⾑、⽔、⼋、⽷、⽷、⽷}
  \definition*{s.}{Organização do Tratado do Atlântico Norte, OTAN}
\end{entry}

\begin{entry}{北方}{bei3fang1}{5,4}{⼔、⽅}[HSK 2]
  \definition{s.}{norte; indicando a direção norte | o Norte; a parte norte da China, especialmente a área ao norte do rio Huang He}
\end{entry}

\begin{entry}{北极}{bei3ji2}{5,7}{⼔、⽊}[HSK 5]
  \definition*{s.}{Pólo Norte; Pólo Ártico}
  \definition{s.}{pólo norte magnético; o ponto mais setentrional da Terra, também se refere à região mais setentrional da Terra}
\end{entry}

\begin{entry}{北京}{bei3 jing1}{5,8}{⼔、⼇}[HSK 1]
  \definition*{s.}{Beijing (Pequim), Capital da República Popular da China | Capital da China, localizada no nordeste do país, fundada em 700 a.C., a cidade é um importante centro comercial, industrial e cultural}
\end{entry}

\begin{entry}{北面}{bei3mian4}{5,9}{⼔、⾯}
  \definition{s.}{lado norte}
\end{entry}

\begin{entry}{北约}{bei3yue1}{5,6}{⼔、⽷}
  \definition*{s.}{OTAN (Organização do Tratado do Atlântico Norte), abreviação de 北大西洋公约组织}
  \seealsoref{北大西洋公约组织}{bei3 da4xi1 yang2 gong1 yue1 zu3zhi1}
\end{entry}

\begin{entry}{贝}{bei4}{4}{⾙}[Kangxi 154]
  \definition*{s.}{sobrenome Bei}
  \definition{s.}{búzios; conchas; mariscos; um termo geral para moluscos com concha, como amêijoas e caracóis |moeda antiga feita de conchas}
\end{entry}

\begin{entry}{备份}{bei4fen4}{8,6}{⼡、⼈}
  \definition{s.}{cópia de segurança | \emph{backup}}
\end{entry}

\begin{entry}{备胎}{bei4tai1}{8,9}{⼡、⾁}
  \definition{s.}{pneu sobressalente | (gíria) substituto}
\end{entry}

\begin{entry}{背}{bei4}{9}{⾁}[HSK 3]
  \definition{adj.}{azarado | fora do caminho; um lugar muito distante do centro movimentado, onde poucas pessoas aparecem | deficiente auditivo}
  \definition{s.}{parte posterior do corpo; costas; coluna vertebral; parte do tronco entre os ombros e a região lombar | parte de trás de um objeto}
  \definition{v.}{afastar-se; virar as costas | decorar; memorizar; recitar de memória | esconder algo de; fazer algo em segredo | sair, ir embora; partir; abandonar | quebrar; violar; agir de forma contrária a}
  \seeref{背}{bei1}
\end{entry}

\begin{entry}{背后}{bei4 hou4}{9,6}{⾁、⼝}[HSK 3]
  \definition{s.}{parte de trás | traseira | nas costas de alguém}
\end{entry}

\begin{entry}{背景}{bei4jing3}{9,12}{⾁、⽇}[HSK 4]
  \definition[种]{s.}{pano de fundo; fundo; cenário de teatro, filme ou drama de TV | fundo; cenário que permeia a imagem principal na tela | condições sociais; ambientes históricos (significativamente influentes para algo ou alguém) | poder que dá suporte a alguém}
\end{entry}

\begin{entry}{倍}{bei4}{10}{⼈}[HSK 4]
  \definition{adv.}{mais; especialmente}
  \definition{clas.}{vezes; para obter um número igual ao número original, você pode multiplicar o número por esse múltiplo}
  \definition{s.}{dobro; duas vezes mais}
\end{entry}

\begin{entry}{被}{bei4}{10}{⾐}[HSK 3]
  \definition*{s.}{sobrenome Bei}
  \definition{part.}{usada antes de verbos para formar frases verbais passivas}
  \definition{prep.}{usado em uma frase para indicar que o sujeito é o receptor da ação}
  \definition{s.}{colcha}
  \definition{v.}{cobrir; espalhar
sofrer}
\end{entry}

\begin{entry}{被单}{bei4dan1}{10,8}{⾐、⼗}
  \definition[床]{s.}{lençol}
\end{entry}

\begin{entry}{被动}{bei4dong4}{10,6}{⾐、⼒}[HSK 5]
  \definition{adj.}{passivo;  agir com base em um impulso externo (o oposto de 主动) | passivo; impossibilidade de prosseguir como pretendido devido a resistência ou interferência}
  \seealsoref{主动}{zhu3dong4}
\end{entry}

\begin{entry}{被告}{bei4gao4}{10,7}{⾐、⼝}
  \definition{s.}{réu}
\end{entry}

\begin{entry}{被迫}{bei4 po4}{10,8}{⾐、⾡}[HSK 4]
  \definition{v.}{ser forçado; ser coagido; ser compelido; ser constrangido; ser forçado a fazer algo por força externa}
\end{entry}

\begin{entry}{被套}{bei4tao4}{10,10}{⾐、⼤}
  \definition{s.}{capa de \emph{edredon}}
  \definition{v.}{ter dinheiro preso (em ações, imóveis, etc.)}
\end{entry}

\begin{entry}{被窝}{bei4wo1}{10,12}{⾐、⽳}
  \definition{s.}{colcha}
\end{entry}

\begin{entry}{被子}{bei4zi5}{10,3}{⾐、⼦}[HSK 3]
  \definition[床]{s.}{colcha}
\end{entry}

\begin{entry}{辈}{bei4}{12}{⾞}[HSK 5]
  \definition{s.}{geração da família | semelhante; círculo familiar; pessoas de um determinado tipo | vida útil; tempo de vida}
\end{entry}

\begin{entry}{本}{ben3}{5}{⽊}[HSK 1]
  \definition*{s.}{sobrenome Ben}
  \definition{adj.}{original; inerente | principal; central}
  \definition{adv.}{originalmente}
  \definition{clas.}{para livros, dicionários, periódicos, arquivos, etc. | para filmes com uma certa duração | para peças de teatro}
  \definition{prep.}{de acordo com; em consonância com; em conformidade com; equivalentes a 依照 e 按照}
  \definition{pron.}{nativo; próprio; refere-se ao próprio interlocutor ou ao grupo, instituição, empresa, local, etc. ao qual o interlocutor pertence | isto; atual; presente}
  \definition[个]{s.}{caule ou raiz de plantas | base; origem; fundamento; fundação;  alicerce | capital; capital social | livro; caderno; livreto | edição; versão | cópia; roteiro; manuscrito | memorial do trono; na era feudal, referia-se a um documento oficial}
  \definition{v.}{seguir; basear-se em; estar de acordo com}
  \seealsoref{按照}{an4zhao4}
  \seealsoref{依照}{yi1 zhao4}
\end{entry}

\begin{entry}{本金}{ben3 jin1}{5,8}{⽊、⾦}
  \definition{s.}{capital; capital para a operação do comércio e da indústria; capital para a operação de negócios |
valor principal; dinheiro retirado ao depositar ou tomar emprestado (diferente de 利息)}
  \seealsoref{利息}{li4xi1}
\end{entry}

\begin{entry}{本科}{ben3ke1}{5,9}{⽊、⽲}[HSK 4]
  \definition{s.}{graduação; bacharelado; o curso básico de uma universidade ou faculdade}
\end{entry}

\begin{entry}{本来}{ben3lai2}{5,7}{⽊、⽊}[HSK 3]
  \definition{adv.}{originalmente | apropriadamente | legalmente}
\end{entry}

\begin{entry}{本领}{ben3 ling3}{5,11}{⽊、⾴}[HSK 3]
  \definition[项,个]{s.}{capacidade | faculdade | poder | habilidade | talento}
\end{entry}

\begin{entry}{本人}{ben3ren2}{5,2}{⽊、⼈}[HSK 5]
  \definition{pron.}{eu (mim, mim mesmo); o orador refere-se a si mesmo | a si mesmo; em pessoa; refere-se à própria pessoa ou à pessoa mencionada anteriormente}
\end{entry}

\begin{entry}{本事}{ben3shi4}{5,8}{⽊、⼅}
  \definition{s.}{habilidade | capacidade | \emph{status} | poder | posição | autoridade}
  \seeref{本事}{ben3shi5}
\end{entry}

\begin{entry}{本事}{ben3shi5}{5,8}{⽊、⼅}[HSK 3]
  \definition{s.}{habilidade | capacidade |\emph{status} | poder | posição | autoridade}
  \seeref{本事}{ben3shi4}
\end{entry}

\begin{entry}{本子}{ben3 zi5}{5,3}{⽊、⼦}[HSK 1]
  \definition[个,本]{s.}{livro; caderno | edição | impressão | licença; certificado de competência emitido por uma instituição especializada, obtido após aprovação no exame | \emph{script}; roteiro}
\end{entry}

\begin{entry}{笨}{ben4}{11}{⽵}[HSK 4]
  \definition{adj.}{estúpido; sem graça; tolo; de pouca habilidade; sem inteligência | desajeitado; tosco; inflexível | incômodo; pesado; desajeitado; difícil de manejar; trabalhoso}
\end{entry}

\begin{entry}{笨蛋}{ben4dan4}{11,11}{⽵、⾍}
  \definition{s.}{bobalhão | cabeça-oca | cabeça-dura}
  \definition{v.}{iludir | enganar}
\end{entry}

\begin{entry}{崩}{beng1}{11}{⼭}
  \definition{s.}{morte de rei ou imperador | desaparecimento}
  \definition{v.}{entrar em colapso | cair em ruínas}
\end{entry}

\begin{entry}{绷带}{beng1dai4}{11,9}{⽷、⼱}
  \definition{s.}{curativo | (empréstimo linguístico) \emph{bandage}}
\end{entry}

\begin{entry}{甭}{beng2}{9}{⽤}
  \definition{v.}{contração de 不用 | não precisar}
  \seealsoref{不用}{bu2 yong4}
\end{entry}

\begin{entry}{蹦极}{beng4ji2}{18,7}{⾜、⽊}
  \definition{s.}{\emph{bungee jumping}}
\end{entry}

\begin{entry}{鼻子}{bi2zi5}{14,3}{⿐、⼦}[HSK 5]
  \definition[个,只]{s.}{nariz; órgão da face, responsável pela respiração e pelo olfato}
\end{entry}

\begin{entry}{比}{bi3}{4}{⽐}[HSK 1][Kangxi 81]
  \definition*{s.}{Bélgica, abreviação de 比利时}
  \definition{adj.}{específico}
  \definition{adv.}{recentemente}
  \definition{part.}{partícula usada para comparação (superioridade)}
  \definition{prep.}{que; do que | (seguido por um substantivo e adjetivo) mais \{adj.\} do que \{s.\}}
  \definition{s.}{razão; proporção | contraste; comparação | metáfora em poesia; técnica de composição poética}
  \definition{v.}{estar ao lado de; estar próximo a | igualar; comparar; competir; contrastar; emular; comparar superioridade, inferioridade, comprimento, distância, qualidade, etc. | assemelhar-se a; comparar com; fazer uma analogia | gesticular; fazer gestos | ser treinado em; ser direcionado a | copiar; imitar | poder ser comparado | apegar-se a; depender}
  \seealsoref{比利时}{bi3li4shi2}
\end{entry}

\begin{entry}{比方}{bi3fang1}{4,4}{⽐、⽅}[HSK 5]
  \definition{conj.}{se; suponha que; expressa uma hipótese, equivalente a 如果 (com eufemismos)}
  \definition{s.}{analogia; exemplo; instância; expressão que usa uma coisa para descrever outra (expressão idiomática); (figurativo) usar uma coisa para descrever outra}
  \definition{v.}{ilustrar; exemplificar; fazer uma analogia; usar uma coisa para descrever outra (expressão idiomática)}
  \seealsoref{如果}{ru2guo3}
\end{entry}

\begin{entry}{比分}{bi3 fen1}{4,4}{⽐、⼑}[HSK 4]
  \definition{s.}{pontuação; comparação de pontuações entre as duas equipes em uma partida}
\end{entry}

\begin{entry}{比较}{bi3jiao4}{4,10}{⽐、⾞}[HSK 3]
  \definition{adv.}{comparativamente | relativamente}
  \definition{s.}{comparação}
  \definition{v.}{comparar}
\end{entry}

\begin{entry}{比利时}{bi3li4shi2}{4,7,7}{⽐、⼑、⽇}
  \definition*{s.}{Bélgica}
\end{entry}

\begin{entry}{比例}{bi3li4}{4,8}{⽐、⼈}[HSK 3]
  \definition{s.}{escala | razão | proporção}
\end{entry}

\begin{entry}{比拼}{bi3pin1}{4,9}{⽐、⼿}
  \definition{s.}{concurso}
  \definition{v.}{competir ferozmente}
\end{entry}

\begin{entry}{比如}{bi3ru2}{4,6}{⽐、⼥}[HSK 2]
  \definition{conj.}{por exemplo; tal como; suponha; digamos; a seguir, apresentamos alguns exemplos; na linguagem coloquial, também se pode dizer 比如说}
  \seealsoref{比如说}{bi3 ru2 shuo1}
\end{entry}

\begin{entry}{比如说}{bi3 ru2 shuo1}{4,6,9}{⽐、⼥、⾔}[HSK 2]
  \definition{adv.}{por exemplo}
  \seealsoref{比如}{bi3ru2}
\end{entry}

\begin{entry}{比萨饼}{bi3sa4bing3}{4,11,9}{⽐、⾋、⾷}
  \definition[张]{s.}{pizza}
\end{entry}

\begin{entry}{比赛}{bi3sai4}{4,14}{⽐、⾙}[HSK 3]
  \definition[场,次]{s.}{competição | concurso}
  \definition{v.}{competir}
\end{entry}

\begin{entry}{比亚迪}{bi3ya4di2}{4,6,8}{⽐、⼆、⾡}
  \definition*{s.}{Montadora BYD}
\end{entry}

\begin{entry}{比重}{bi3zhong4}{4,9}{⽐、⾥}[HSK 5]
  \definition{s.}{proporção; o peso da parte em relação ao todo | densidade específica; a relação entre o peso de um objeto e seu volume}
\end{entry}

\begin{entry}{彼此}{bi3ci3}{8,6}{⼻、⽌}[HSK 5]
  \definition{pron.}{um ao outro; uns com os outros; este e aquele têm algum tipo de relacionamento; ambas as partes}
\end{entry}

\begin{entry}{笔}{bi3}{10}{⽵}[HSK 2]
  \definition{clas.}{usado para grandes quantias de dinheiro, compras, negócios, propriedades, etc. | usado em caligrafia e pintura, etc.}
  \definition[支,枝]{s.}{caneta; lápis; pincel para escrever; ferramentas para escrever ou desenhar |
técnica de escrita; caligrafia ou desenho | traço}
  \definition[支,枝]{v.}{escrever à mão}
\end{entry}

\begin{entry}{笔记}{bi3 ji4}{10,5}{⽵、⾔}[HSK 2]
  \definition[篇,本,个]{s.}{notas; anotações feitas durante aulas, palestras e leituras | ensaios; esboços}
  \definition{v.}{tomar nota (por escrito)}
\end{entry}

\begin{entry}{笔记本}{bi3ji4ben3}{10,5,5}{⽵、⾔、⽊}[HSK 2]
  \definition[个,本]{s.}{caderno para anotações | \emph{laptop}; refere-se a um computador portátil}
  \definition{s.}{\emph{laptop}}
\end{entry}

\begin{entry}{必}{bi4}{5}{⼼}[HSK 5]
  \definition{adv.}{certamente; necessariamente; indica que algo é certo ou que alguém acredita que esteja correto}
\end{entry}

\begin{entry}{必定}{bi4ding4}{5,8}{⼼、⼧}
  \definition{adv.}{sem falta | certamente | com certeza | definitivamente | inevitavelmente | com determinação}
  \definition{v.}{estar vinculado a | ter certeza de}
\end{entry}

\begin{entry}{必然}{bi4ran2}{5,12}{⼼、⽕}[HSK 3]
  \definition{adj.}{certo | inevitável | necessário}
  \definition{adv.}{inevitavelmente}
  \definition{s.}{necessidade}
\end{entry}

\begin{entry}{必须}{bi4xu1}{5,9}{⼼、⾴}[HSK 2]
  \definition{adv.}{necessariamente; obrigatoriamente; indica a necessidade lógica e emocional | deve; tem que; é obrigado a}
\end{entry}

\begin{entry}{必需}{bi4 xu1}{5,14}{⼼、⾬}[HSK 5]
  \definition{v.}{ser essencial; ser indispensável}
\end{entry}

\begin{entry}{必要}{bi4yao4}{5,9}{⼼、⾑}[HSK 3]
  \definition{adj.}{necessário | essencial | indispensável}
  \definition[个,些]{s.}{necessidade}
\end{entry}

\begin{entry}{毕竟}{bi4jing4}{6,11}{⽐、⾳}[HSK 5]
  \definition{adv.}{afinal de contas; quando tudo estiver dito e feito; em última análise; indica um resultado que não pode ser alterado, enfatizando que se trata de uma causa ou fato que precisa ser enfocado para referência | significa 到底, 究竟, 终究, indicando a conclusão final alcançada}
  \seealsoref{到底}{dao4di3}
  \seealsoref{究竟}{jiu1jing4}
  \seealsoref{终究}{zhong1jiu1}
\end{entry}

\begin{entry}{毕业}{bi4ye4}{6,5}{⽐、⼀}[HSK 4]
  \definition{s.}{formatura}
  \definition{v.+compl.}{formar-se}
\end{entry}

\begin{entry}{毕业生}{bi4 ye4 sheng1}{6,5,5}{⽐、⼀、⽣}[HSK 4]
  \definition[个]{s.}{diplomado; graduado; bacharel; pessoa que recebeu um diploma, grau ou certificado}
\end{entry}

\begin{entry}{闭幕}{bi4 mu4}{6,13}{⾨、⼱}[HSK 5]
  \definition{v.+compl.}{fechar; concluir; (conferência, exposição, etc.) terminar | cair a cortina; abaixar a cortina; terminar a apresentação e a cortina se fechar em frente ao palco}
\end{entry}

\begin{entry}{闭幕式}{bi4 mu4 shi4}{6,13,6}{⾨、⼱、⼷}[HSK 5]
  \definition{s.}{cerimônia de encerramento; cerimônia formal realizada no final de uma conferência ou exposição}
\end{entry}

\begin{entry}{闭嘴}{bi4zui3}{6,16}{⾨、⼝}
  \definition{expr.}{Cale-se!}
\end{entry}

\begin{entry}{壁虎}{bi4hu3}{16,8}{⼟、⾌}
  \definition{s.}{lagartixa}
\end{entry}

\begin{entry}{壁纸}{bi4zhi3}{16,7}{⼟、⽷}
  \definition{s.}{papel de parede}
\end{entry}

\begin{entry}{避}{bi4}{16}{⾌}[HSK 4]
  \definition{v.}{evitar; evadir; esquivar-se; buscar abrigo; fugir | impedir; manter afastado; repelir; previnir}
\end{entry}

\begin{entry}{避免}{bi4mian3}{16,7}{⾌、⼉}[HSK 4]
  \definition{v.}{evitar; desviar; abster-se de; tentar não fazer com que algo aconteça; prevenir; tentar impedir (que algo ruim aconteça) com antecedência}
\end{entry}

\begin{entry}{边}{bian1}{5}{⾡}[HSK 2]
  \definition*{s.}{sobrenome Bian}
  \definition{adv.}{dois ou mais 边 são usados separadamente antes de diferentes verbos, indicando que diferentes ações ocorrem simultaneamente}
  \definition[条,个]{s.}{lado (de uma figura geométrica) | borda; lado; margem; aba; rebordo | fronteira; limite | ao lado de; lugar próximo a; perto de um objeto; lateral | aro; aba; borda; decoração em forma de faixa incrustada ou pintada na borda de um objeto}
  \definition{suf.}{lado; anexado a palavras de localização monossilábicas, formando palavras de localização dissílabas}
  \seeref{边}{bian5}
\end{entry}

\begin{entry}{边防}{bian1fang2}{5,6}{⾡、⾩}
  \definition{s.}{defesa da fronteira}
\end{entry}

\begin{entry}{边关}{bian1guan1}{5,6}{⾡、⼋}
  \definition{s.}{posto de fronteira | posição defensiva estratégica na fronteira}
\end{entry}

\begin{entry}{边境}{bian1jing4}{5,14}{⾡、⼟}[HSK 5]
  \definition{s.}{fronteira}
\end{entry}

\begin{entry}{编}{bian1}{12}{⽷}[HSK 4]
  \definition*{s.}{sobrenome Bian}
  \definition{s.}{livro; volume; parte de um livro}
  \definition{v.}{tecer; trançar; entrançar | fazer uma lista; organizar em uma lista; organizar; agrupar | editar; compilar | compor; escrever | fabricar; inventar; fazer; preparar}
\end{entry}

\begin{entry}{编程}{bian1cheng2}{12,12}{⽷、⽲}
  \definition{s.}{programa de computador}
  \definition{v.}{programar computador}
\end{entry}

\begin{entry}{编辑}{bian1ji2}{12,13}{⽷、⾞}[HSK 5]
  \definition{v.}{editar; compilar; organizar e processar dados ou trabalhos existentes}
  \seeref{编辑}{bian1ji5}
\end{entry}

\begin{entry}{编辑}{bian1ji5}{12,13}{⽷、⾞}[HSK 5]
  \definition{s.}{editor; compilador; pessoa que organiza e processa dados ou trabalhos existentes}
  \seeref{编辑}{bian1ji2}
\end{entry}

\begin{entry}{邉}{bian1}{17}{⾡}
  \variantof{边}
\end{entry}

\begin{entry}{变}{bian4}{8}{⼜}[HSK 2]
  \definition{adj.}{alterado; mutável; que pode mudar; que está mudando ou já mudou}
  \definition{s.}{uma reviravolta inesperada nos acontecimentos; mudanças significativas repentinas}
  \definition{v.}{mudar; tornar-se diferente; fazer mudanças |tornar-se; transformar-se; natureza, estado ou situação diferentes dos originais | alterar; mudar; transformar}
\end{entry}

\begin{entry}{变成}{bian4 cheng2}{8,6}{⼜、⼽}[HSK 2]
  \definition{v.}{crescer; tornar-se; fazer; desenvolver-se; revelar-se; resultar; acontecer; passar a ser; passar para; acumular-se; converter-se; transformar-se; transformar-se em; mudar-se em; transformação da situação ou condição anterior para a situação ou condição atual}
\end{entry}

\begin{entry}{变动}{bian4 dong4}{8,6}{⼜、⼒}[HSK 5]
  \definition{v.}{mudar; alterar; oscilar; modificar; variar}
\end{entry}

\begin{entry}{变更}{bian4geng1}{8,7}{⼜、⽈}
  \definition{v.}{alterar | mudar | modificar}
\end{entry}

\begin{entry}{变化}{bian4hua4}{8,4}{⼜、⼔}[HSK 3]
  \definition[个]{s.}{mudança | variação}
  \definition{v.}{(intransitivo) mudar, variar}
\end{entry}

\begin{entry}{变节}{bian4jie2}{8,5}{⼜、⾋}
  \definition{s.}{traição | deserção | vira-casaca}
  \definition{v.}{mudar de lado politicamente}
\end{entry}

\begin{entry}{变迁}{bian4qian1}{8,6}{⼜、⾡}
  \definition{s.}{mudanças | vicissitudes}
\end{entry}

\begin{entry}{变数}{bian4shu4}{8,13}{⼜、⽁}
  \definition{s.}{(matemática) variável}
\end{entry}

\begin{entry}{变为}{bian4 wei2}{8,4}{⼜、⼂}[HSK 3]
  \definition{v.}{transformar-se em | tornar-se | mudar para}
\end{entry}

\begin{entry}{变心}{bian4xin1}{8,4}{⼜、⼼}
  \definition{v.+compl.}{deixar de ser fiel}
\end{entry}

\begin{entry}{变性}{bian4xing4}{8,8}{⼜、⼼}
  \definition{s.}{desnaturação | transexual}
  \definition{v.}{desnaturar | mudar de sexo}
\end{entry}

\begin{entry}{变异}{bian4yi4}{8,6}{⼜、⼶}
  \definition{s.}{variação | mutação}
\end{entry}

\begin{entry}{变装}{bian4zhuang1}{8,12}{⼜、⾐}
  \definition{v.}{trocar de roupa | vestir-se | vestir uma fantasia | disfarçar-se ou fantasiar-se de personagem real ou ficcional, \emph{cosplay} | travestir-se}
\end{entry}

\begin{entry}{便利}{bian4li4}{9,7}{⼈、⼑}[HSK 5]
  \definition{adj.}{fácil; conveniente;}
  \definition{s.}{facilidade; conveniência; coisas ou condições convenientes}
  \definition{v.}{facilitar; fornecer ajuda para que os outros se sintam confortáveis}
\end{entry}

\begin{entry}{便条}{bian4tiao2}{9,7}{⼈、⽊}[HSK 5]
  \definition[张,个]{s.}{nota ou mensagem informal; geralmente uma mensagem ou notificação}
\end{entry}

\begin{entry}{便宜}{bian4yi2}{9,8}{⼈、⼧}
  \definition{adj.}{prático; conveniente; adequado}
  \seeref{便宜}{pian2yi5}
\end{entry}

\begin{entry}{便于}{bian4yu2}{9,3}{⼈、⼆}[HSK 5]
  \definition{v.}{ser fácil para; ser conveniente para (algo ou fazer algo)}
\end{entry}

\begin{entry}{遍}{bian4}{12}{⾡}[HSK 2]
  \definition{adv.}{por toda parte; em toda parte; em todos os lugares}
  \definition{clas.}{usado para a repetição de ações de leitura, fala ou escrita}
\end{entry}

\begin{entry}{辩论}{bian4lun4}{16,6}{⾟、⾔}[HSK 4]
  \definition[场,次]{s.}{debate; argumento; a atividade comportamental em si de argumentar ou refutar diferentes pontos de vista ou afirmações, ou uma ocasião ou situação em que tal argumentação ou refutação é feita}
  \definition{v.}{debater; obter um entendimento unificado ou correto, ambos os lados usam linguagem, palavras etc. para explicar seus pontos de vista, apontar os erros ou as contradições do outro lado}
\end{entry}

\begin{entry}{辫子}{bian4zi5}{17,3}{⾟、⼦}
  \definition[根,条]{s.}{trança | um erro ou falha que pode ser explorado por um oponente | alça}
\end{entry}

\begin{entry}{边}{bian5}{5}{⾡}
  \definition[条,个]{suf.}{sufixo de uma palavra de localidade (lado); indica posição e direção, usado após palavras que indicam direção, como 上, 下, 前, 后, 左, 右}
  \seeref{边}{bian1}
\end{entry}

\begin{entry}{标题}{biao1ti2}{9,15}{⽊、⾴}[HSK 3]
  \definition[个,条,篇]{s.}{título | manchete | cabeçalho}
\end{entry}

\begin{entry}{标志}{biao1zhi4}{9,7}{⽊、⼼}[HSK 4]
  \definition[个,种]{s.}{sinal; marca; logotipo; símbolo; emblema; marcações que caracterizam um objeto para facilitar a identificação}
  \definition{v.}{marcar; indicar; simbolizar; identificar}
\end{entry}

\begin{entry}{标准}{biao1zhun3}{9,10}{⽊、⼎}[HSK 3]
  \definition{adj.}{criterioso | padronizado | normatizado}
  \definition[个]{s.}{critério | padrão (oficial) | norma}
\end{entry}

\begin{entry}{髟}{biao1}{10}{⾽}[Kangxi 190]
  \definition{adj.}{cabelo solto, caído}
\end{entry}

\begin{entry}{镖}{biao1}{16}{⾦}
  \definition{s.}{dardo | arma de arremesso | mercadorias enviadas sob a proteção de uma escolta armada}
\end{entry}

\begin{entry}{表}{biao3}{8}{⾐}[HSK 2]
  \definition*{s.}{sobrenome Biao}
  \definition{s.}{exterior; superfície; externo | a relação entre os filhos ou netos de um irmão e uma irmã ou de irmãs | modelo; exemplo; padrão | memorial a um imperador; um tipo de petição da antiguidade, frequentemente usado para expressar intenções; mais tarde, também usado para expressar opiniões sobre eventos importantes | formulário; lista; gráfico; tabela | medidor; instrumento para medir uma determinada quantidade | relógio; um dispositivo para medir o tempo, menor que um relógio, que geralmente pode ser carregado no bolso | medidor de luz solar; antiga vara de madeira para medir o tempo através da sombra do sol | coluna usada antigamente para marcação}
  \definition{v.}{mostrar; expressar; expressar ideias, pensamentos, sentimentos, etc. | administrar medicamentos para aliviar o resfriado; na medicina tradicional chinesa refere-se ao uso de medicamentos para dissipar o frio e o vento que afetam o corpo}
\end{entry}

\begin{entry}{表白}{biao3bai2}{8,5}{⾐、⽩}
  \definition{s.}{declaração | confissão}
  \definition{v.}{confessar a si mesmo | expressar | revelar pensamentos ou sentimentos de alguém}
\end{entry}

\begin{entry}{表达}{biao3da2}{8,6}{⾐、⾡}[HSK 3]
  \definition{v.}{entregar | expressar | mostrar | transmitir | comunicar}
\end{entry}

\begin{entry}{表格}{biao3ge2}{8,10}{⾐、⽊}[HSK 3]
  \definition[份,张]{s.}{tabela | formulário}
\end{entry}

\begin{entry}{表面}{biao3mian4}{8,9}{⾐、⾯}[HSK 3]
  \definition{s.}{superfície | lado de fora | aparência | superficialidade}
\end{entry}

\begin{entry}{表明}{biao3ming2}{8,8}{⾐、⽇}[HSK 3]
  \definition{v.}{deixar claro | tornar conhecido | declarar claramente}
\end{entry}

\begin{entry}{表情}{biao3qing2}{8,11}{⾐、⼼}[HSK 4]
  \definition[个,种,幅]{s.}{expressão; expressão facial; expressão de pensamentos e sentimentos internos por meio de mudanças faciais ou de gestos}
  \definition{v.}{expressar pensamentos e sentimentos internos por meio de mudanças faciais ou de gestos}
\end{entry}

\begin{entry}{表示}{biao3shi4}{8,5}{⾐、⽰}[HSK 2]
  \definition{s.}{expressão; indicação}
  \definition{v.}{mostrar; expressar; indicar | significar | expressar pensamentos e sentimentos através de palavras, ações ou expressões faciais}
\end{entry}

\begin{entry}{表现}{biao3xian4}{8,8}{⾐、⾒}[HSK 3]
  \definition[个,种,份]{s.}{desempenho | expressão  manifestação | comportamento}
  \definition{v.}{mostrar | expressar | exibir | manifestar | descrever}
\end{entry}

\begin{entry}{表演}{biao3yan3}{8,14}{⾐、⽔}[HSK 3]
  \definition[场]{s.}{representação | atuação | exposição}
  \definition{v.}{executar | atuar | jogar | demonstrar | agir | fingir}
\end{entry}

\begin{entry}{表演赛}{biao3yan3sai4}{8,14,14}{⾐、⽔、⾙}
  \definition{s.}{partida ou jogo de exibição}
\end{entry}

\begin{entry}{表演特技}{biao3yan3 te4ji4}{8,14,10,7}{⾐、⽔、⽜、⼿}
  \definition{s.}{acrobacia | pirueta | façanha}
\end{entry}

\begin{entry}{表演艺术}{biao3yan3 yi4shu4}{8,14,4,5}{⾐、⽔、⾋、⽊}
  \definition{s.}{arte performática}
\end{entry}

\begin{entry}{表演游戏}{biao3yan3 you2xi4}{8,14,12,6}{⾐、⽔、⽔、⼽}
  \definition{s.}{exibição dramática}
\end{entry}

\begin{entry}{表演者}{biao3yan3 zhe3}{8,14,8}{⾐、⽔、⽼}
  \definition{s.}{ator}
\end{entry}

\begin{entry}{表扬}{biao3yang2}{8,6}{⾐、⼿}[HSK 4]
  \definition[次,种,份]{s.}{elogios públicos por boas ações}
  \definition{v.}{elogiar; louvar}
\end{entry}

\begin{entry}{表扬信}{biao3yang2 xin4}{8,6,9}{⾐、⼿、⼈}
  \definition{s.}{carta de elogio | depoimento}
\end{entry}

\begin{entry}{别}{bie2}{7}{⼑}[HSK 1,4]
  \definition*{s.}{sobrenome Bie}
  \definition{adv.}{não; nada de (pedir a alguém para não fazer); é melhor não | talvez, usado em conjunto com a palavra 是 para indicar especulação.}
  \definition{pron.}{outro; algum outro}
  \definition{s.}{distinção; diferença | classificação}
  \definition{v.}{deixar; partir; separar | diferenciar; distinguir; encontrar aspectos diferentes | fixar objetos com pinos | girar; virar | aderir; colar; preder}
  \seeref{别}{bie4}
  \seealsoref{是}{shi4}
\end{entry}

\begin{entry}{别的}{bie2 de5}{7,8}{⼑、⽩}[HSK 1]
  \definition{pron.}{outro; o resto}
\end{entry}

\begin{entry}{别人}{bie2 ren2}{7,2}{⼑、⼈}[HSK 1]
  \definition{pron.}{outros; outras pessoas}
  \definition{s.}{outros; pessoas; outras pessoas; refere-se a alguém diferente de si mesmo}
\end{entry}

\begin{entry}{别说}{bie2shuo1}{7,9}{⼑、⾔}
  \definition{v.}{não falar de | não mencionar}
\end{entry}

\begin{entry}{别}{bie4}{7}{⼑}
  \definition{v.}{fazer com que alguém mude seus hábitos, opiniões, etc.}
  \seeref{别}{bie2}
\end{entry}

\begin{entry}{宾馆}{bin1guan3}{10,11}{⼧、⾷}[HSK 5]
  \definition[家,个,座]{s.}{hotel; acomodações públicas para hóspedes}
\end{entry}

\begin{entry}{冰}{bing1}{6}{⼎}[HSK 4]
  \definition{adj.}{frio (pessoa)| hostil}
  \definition[块]{s.}{gelo; água em estado sólido |  (gíria) metanfetamina}
  \definition{v.}{colocar gelo; colocar gelo ao redor; colocar no gelo; resfriar objetos com gelo ou água fria | sentir frio}
\end{entry}

\begin{entry}{冰糕}{bing1gao1}{6,16}{⼎、⽶}
  \definition{s.}{sorvete | picolé}
\end{entry}

\begin{entry}{冰棍}{bing1gun4}{6,12}{⼎、⽊}
  \definition[根]{s.}{picolé}
\end{entry}

\begin{entry}{冰激凌}{bing1ji1ling2}{6,16,10}{⼎、⽔、⼎}
  \definition{s.}{sorvete}
\end{entry}

\begin{entry}{冰球}{bing1qiu2}{6,11}{⼎、⽟}
  \definition{s.}{hóquei no gelo}
\end{entry}

\begin{entry}{冰天雪地}{bing1tian1-xue3di4}{6,4,11,6}{⼎、⼤、⾬、⼟}
  \definition{expr.}{um mundo de gelo e neve}
\end{entry}

\begin{entry}{冰箱}{bing1xiang1}{6,15}{⼎、⾋}[HSK 4]
  \definition[台,个]{s.}{geladeira; freezer; refrigerador; aparelhos para congelar alimentos ou medicamentos com gelo para mantê-los frios}
\end{entry}

\begin{entry}{冰雪}{bing1 xue3}{6,11}{⼎、⾬}[HSK 4]
  \definition{adj.}{puro como gelo e neve; descreve uma pessoa como pura}
  \definition{s.}{gelo e neve}
\end{entry}

\begin{entry}{兵}{bing1}{7}{⼋}[HSK 4]
  \definition[名]{s.}{armas; armamentos | soldado; pessoal militar | exército; tropas | soldado raso; membro mais jovem do exército | assuntos militares (estratégia) | peão, uma das peças do xadrez chinês}
\end{entry}

\begin{entry}{兵器}{bing1qi4}{7,16}{⼋、⼝}
  \definition{s.}{armas | armamento}
\end{entry}

\begin{entry}{饼}{bing3}{9}{⾷}[HSK 5]
  \definition[张]{s.}{um bolo redondo e plano; massa assada ou cozida no vapor | algo que tem o formato de um bolo; semelhante a uma torta}
\end{entry}

\begin{entry}{饼干}{bing3gan1}{9,3}{⾷、⼲}[HSK 5]
  \definition[块,片,包,盒,袋]{s.}{biscoito; bolacha; \emph{cookie}; alimentos, pedaços pequenos e finos cozidos em farinha com açúcar, ovos, leite, etc.}
\end{entry}

\begin{entry}{并}{bing4}{6}{⼲}[HSK 3,4]
  \definition{adv.}{igualmente; simultaneamente; lado a lado; coisas diferentes existem ao mesmo tempo; coisas diferentes estão acontecendo ao mesmo tempo | em absoluto (usado antes de uma negativa para dar ênfase);  usado antes de uma palavra negativa para reforçar o tom e refutá-la ligeiramente}
  \definition{conj.}{além de; e}
  \definition{v.}{combinar; fundir; incorporar; anexar; juntar}
\end{entry}

\begin{entry}{并排}{bing4pai2}{6,11}{⼲、⼿}
  \definition{adv.}{lado a lado}
\end{entry}

\begin{entry}{并且}{bing4qie3}{6,5}{⼲、⼀}[HSK 3]
  \definition{conj.}{além disso | o que é mais | e}
\end{entry}

\begin{entry}{幷}{bing4}{8}{⼲}
  \variantof{并}
\end{entry}

\begin{entry}{倂}{bing4}{10}{⼈}
  \variantof{并}
\end{entry}

\begin{entry}{病}{bing4}{10}{⽧}[HSK 1]
  \definition[场]{s.}{doença; enfermidade | doença; males | falha; defeito; desvantagem; erro}
  \definition{v.}{adoecer; ficar doente | ferir; causar danos a | angustiar; desaprovar}
\end{entry}

\begin{entry}{病毒}{bing4du2}{10,9}{⽧、⽏}[HSK 5]
  \definition[种,株,类]{s.}{vírus; patógenos que são menores que os germes e visíveis somente com um microscópio eletrônico | vírus de computador}
\end{entry}

\begin{entry}{病人}{bing4 ren2}{10,2}{⽧、⼈}[HSK 1]
  \definition[个,位]{s.}{doente; paciente; pessoas doentes; pessoas em tratamento}
\end{entry}

\begin{entry}{拨转}{bo1zhuan3}{8,8}{⼿、⾞}
  \definition{v.}{transferir (fundos, etc.) | virar | dar a volta}
\end{entry}

\begin{entry}{波}{bo1}{8}{⽔}
  \definition*{s.}{Polônia, abreviação de 波兰}
  \definition{s.}{onda | ondulação | tempestade | surto}
  \seealsoref{波兰}{bo1lan2}
\end{entry}

\begin{entry}{波兰}{bo1lan2}{8,5}{⽔、⼋}
  \definition*{s.}{Polônia}
\end{entry}

\begin{entry}{波音}{bo1yin1}{8,9}{⽔、⾳}
  \definition*{s.}{Boeing (empresa aeroespacial)}
  \definition{s.}{mordente (música)}
\end{entry}

\begin{entry}{玻璃}{bo1li5}{9,14}{⽟、⽟}[HSK 5]
  \definition[张,块]{s.}{vidro; corpo duro, quebradiço e transparente, geralmente feito de areia, calcário, carbonato de sódio, etc. | \emph{nylon}; plástico; refere-se a determinados plásticos que se assemelham ao vidro.}
\end{entry}

\begin{entry}{般}{bo1}{10}{⾈}
  \definition{s.}{utilizado em 般若}
  \seealsoref{般若}{bo1re3}
\end{entry}

\begin{entry}{般若}{bo1re3}{10,8}{⾈、⾋}
  \definition*{s.}{Prajna (sânscrito), \emph{insight} sobre a verdadeira natureza da realidade}
  \definition*{s.}{(Budismo) sabedoria}
\end{entry}

\begin{entry}{啵}{bo1}{11}{⼝}
  \definition{s.}{(onomatopéia) borbulhar}
  \seeref{啵}{bo5}
\end{entry}

\begin{entry}{菠菜}{bo1cai4}{11,11}{⾋、⾋}
  \definition[棵]{s.}{espinafre}
\end{entry}

\begin{entry}{播出}{bo1 chu1}{15,5}{⼿、⼐}[HSK 3]
  \definition{v.}{transmitir | estar no ar}
\end{entry}

\begin{entry}{播放}{bo1fang4}{15,8}{⼿、⽅}[HSK 3]
  \definition{v.}{ir ao ar | transmitir por rádio | mostrar | transmitir (um programa de TV)}
\end{entry}

\begin{entry}{播音}{bo1yin1}{15,9}{⼿、⾳}
  \definition{s.}{transmissão}
  \definition{v.+compl.}{transmitir}
\end{entry}

\begin{entry}{脖子}{bo2zi5}{11,3}{⾁、⼦}
  \definition[个]{s.}{pescoço}
\end{entry}

\begin{entry}{博客}{bo2 ke4}{12,9}{⼗、⼧}[HSK 5]
  \definition{s.}{\emph{blog}; página da Web ou site gerenciado por um indivíduo, geralmente composto por postagens organizadas da mais recente para a mais antiga | blogueiro; \emph{blogger}; pessoas que possuem ou escrevem \emph{blogs}}
\end{entry}

\begin{entry}{博览会}{bo2lan3hui4}{12,9,6}{⼗、⾒、⼈}[HSK 5]
  \definition[次]{s.}{exposição; feira internacional; exposições de produtos em grande escala}
\end{entry}

\begin{entry}{博士}{bo2shi4}{12,3}{⼗、⼠}[HSK 5]
  \definition{s.}{doutorado; grau de doutor; nível mais alto de um diploma; também, uma pessoa que obteve esse diploma | doutor; antigo título honorífico para uma pessoa que é habilidosa em um determinado ofício ou especializada em uma determinada ocupação | doutor; autoridades que ensinavam as escrituras na China nos tempos antigos}
\end{entry}

\begin{entry}{博文}{bo2wen2}{12,4}{⼗、⽂}
  \definition{s.}{artigo em um blog}
  \definition{v.}{escrever um artigo em um blog}
\end{entry}

\begin{entry}{博物馆}{bo2wu4guan3}{12,8,11}{⼗、⽜、⾷}[HSK 5]
  \definition[个]{s.}{museu; locais para coleta, armazenamento, pesquisa, exibição e exposição de relíquias culturais ou espécimes relacionados à história, cultura, arte, ciências naturais, ciência e tecnologia, etc.}
\end{entry}

\begin{entry}{博主}{bo2zhu3}{12,5}{⼗、⼂}
  \definition{s.}{blogueiro}
\end{entry}

\begin{entry}{薄}{bo2}{16}{⾋}
  \definition{adj.}{ligeiro; escasso; pequeno | mesquinho; pouco generoso; cruel | frívolo; fútil; leviano}
  \seeref{薄}{bao2}
\end{entry}

\begin{entry}{薄弱}{bo2ruo4}{16,10}{⾋、⼸}[HSK 5]
  \definition{adj.}{fraco; frágil}
\end{entry}

\begin{entry}{啵}{bo5}{11}{⼝}
  \definition{part.}{partícula gramaticalmente equivalente a 吧}
  \seeref{啵}{bo1}
  \seealsoref{吧}{ba5}
\end{entry}

\begin{entry}{不}{bu2}[(antes de quarto tom)]{4}{⼀}[HSK 1]
  \seeref{不}{bu4}
  \seeref{不}{bu5}
\end{entry}

\begin{entry}{不必}{bu2 bi4}{4,5}{⼀、⼼}[HSK 3]
  \definition{adv.}{não precisa | não tem que}
\end{entry}

\begin{entry}{不错}{bu2 cuo4}{4,13}{⼀、⾦}[HSK 2]
  \definition{adj.}{certo; correto | nada mal; muito bom}
\end{entry}

\begin{entry}{不大}{bu2 da4}{4,3}{⼀、⼤}[HSK 1]
  \definition{adv.}{não muito (indicando um grau baixo); não demasiado | não com frequência; raramente; dificilmente}
\end{entry}

\begin{entry}{不大离}{bu2da4li2}{4,3,10}{⼀、⼤、⼇}
  \definition{adj.}{bem perto | quase certo | nada mal}
\end{entry}

\begin{entry}{不但}{bu2 dan4}{4,7}{⼀、⼈}[HSK 2]
  \definition{conj.}{não só\dots mas também; usado na primeira parte de uma frase composta que expressa progressão, a segunda parte geralmente contém conjunções como 而且,  并且 ou advérbios como 也, 还 que correspondem à primeira parte}
  \seealsoref{并且}{bing4qie3}
  \seealsoref{而且}{er2 qie3}
  \seealsoref{还}{hai2}
  \seealsoref{也}{ye3}
\end{entry}

\begin{entry}{不但……而且……}{bu2 dan4 er2qie3}{4,7,6,5}{⼀、⼈、⽽、⼀}[HSK 2]
  \definition{conj.}{não só\dots mas também\dots}
\end{entry}

\begin{entry}{不到}{bu2dao4}{4,8}{⼀、⼑}
  \definition{adj.}{insuficiente}
  \definition{adv.}{menos que}
  \definition{v.}{não chegar}
\end{entry}

\begin{entry}{不断}{bu2duan4}{4,11}{⼀、⽄}[HSK 3]
  \definition{adv.}{continuamente | sem fim}
\end{entry}

\begin{entry}{不对}{bu2 dui4}{4,5}{⼀、⼨}[HSK 1]
  \definition{adj.}{incorreto; errado | anormal; anômalo; estranho | desarmonia; incompatibilidade; discórdia}
\end{entry}

\begin{entry}{不够}{bu2 gou4}{4,11}{⼀、⼣}[HSK 2]
  \definition{adv.}{insuficiente; indica que não atingiu o nível esperado}
  \definition{v.}{não ser suficiente; indica que é inferior ao exigido em quantidade ou grau}
\end{entry}

\begin{entry}{不顾}{bu2gu4}{4,10}{⼀、⾴}[HSK 5]
  \definition{v.}{não considerar; desconsiderar | desconsiderar; não levar em consideração; ignorar; não se preocupar com}
\end{entry}

\begin{entry}{不过}{bu2guo4}{4,6}{⼀、⾡}[HSK 2]
  \definition{adv.}{apenas; meramente; nada mais do que; indica que não excede um determinado limite, equivalente a 仅 ou 只 | como intensificador após certos adjetivos}
  \definition{conj.}{mas; no entanto; apenas; usado no início da segunda parte da frase, indica o contrário do sentido anterior e modifica ou complementa o significado anterior}
\end{entry}

\begin{entry}{不客气}{bu2 ke4 qi5}{4,9,4}{⼀、⼧、⽓}[HSK 1]
  \definition{adj.}{rude; indelicado; duro | franco; sincero; direto}
  \definition{expr.}{de modo algum; não mencione isso; de nada}
  \definition{v.}{dizer palavras ou fazer gestos indelicados}
\end{entry}

\begin{entry}{不利}{bu2 li4}{4,7}{⼀、⼑}[HSK 5]
  \definition{adj.}{desfavorável; desvantajoso; nocivo; prejudicial | malsucedido | desfavorável; adverso}
\end{entry}

\begin{entry}{不论}{bu2 lun4}{4,6}{⼀、⾔}[HSK 3]
  \definition{conj.}{não importa (o que, quem, como, etc.) | se \dots ou \dots}
\end{entry}

\begin{entry}{不论……都……}{bu2lun4 dou1}{4,6,10}{⼀、⾔、⾢}
  \definition{conj.}{não apenas\dots, (o que, quem, como, etc.), \dots}
\end{entry}

\begin{entry}{不论……也……}{bu2lun4 ye3}{4,6,3}{⼀、⾔、⼄}
  \definition{conj.}{não apenas\dots, (o que, quem, como, etc.), \dots}
\end{entry}

\begin{entry}{不耐烦}{bu2nai4fan2}{4,9,10}{⼀、⽽、⽕}[HSK 5]
  \definition{adj.}{impaciente; significa não ser capaz de suportar coisas tediosas ou que causam distração}
\end{entry}

\begin{entry}{不日}{bu2ri4}{4,4}{⼀、⽇}
  \definition{adv.}{em alguns dias}
\end{entry}

\begin{entry}{不是话}{bu2shi4hua4}{4,9,8}{⼀、⽇、⾔}
  \definition{expr.}{sem razão | demasiado irracionável}
  \seealsoref{不像话}{bu2xiang4hua4}
  \seealsoref{不成话}{bu4cheng2hua4}
\end{entry}

\begin{entry}{不太}{bu2 tai4}{4,4}{⼀、⼤}[HSK 2]
  \definition{adv.}{não exatamente | não muito bom}
\end{entry}

\begin{entry}{不像话}{bu2xiang4hua4}{4,13,8}{⼀、⼈、⾔}
  \definition{expr.}{sem razão | demasiado irracionável}
  \seealsoref{不是话}{bu2shi4hua4}
  \seealsoref{不成话}{bu4cheng2hua4}
\end{entry}

\begin{entry}{不幸}{bu2 xing4}{4,8}{⼀、⼲}[HSK 5]
  \definition{adj.}{triste; infeliz; lamentável; azarado | infeliz; indica o mais indesejável (que aconteceu)}
  \definition[个]{s.}{morte; desastre; infortúnio; adversidade; calamidade}
\end{entry}

\begin{entry}{不要}{bu2 yao4}{4,9}{⼀、⾑}[HSK 2]
  \definition{adv.}{nada de (pedir a alguém para não fazer) | não; expressa proibição e dissuasão}
\end{entry}

\begin{entry}{不要紧}{bu2yao4jin3}{4,9,10}{⼀、⾑、⽷}[HSK 4]
  \definition{adj.}{sem importância; sem seriedade; não problemático | não importa; não é um obstáculo | parece estar tudo bem, mas | à primeira vista, isso não parece atrapalhar}
\end{entry}

\begin{entry}{不易}{bu2 yi4}{4,8}{⼀、⽇}[HSK 5]
  \definition{v.}{não ser fácil; ser difícil}
\end{entry}

\begin{entry}{不用}{bu2 yong4}{4,5}{⼀、⽤}[HSK 1]
  \definition{v.}{não precisar; não ter necessidade; indicar que, na verdade, não é necessário}
\end{entry}

\begin{entry}{不在乎}{bu2 zai4 hu1}{4,6,5}{⼀、⼟、⼃}[HSK 4]
  \definition{v.}{não se importar; não dar a mínima; não dar atenção}
\end{entry}

\begin{entry}{不注意}{bu2zhu4yi4}{4,8,13}{⼀、⽔、⼼}
  \definition{adj.}{impensado | distraído}
  \definition{s.}{descuido | distração}
\end{entry}

\begin{entry}{补}{bu3}{7}{⾐}[HSK 3]
  \definition*{s.}{sobrenome Bu}
  \definition{s.}{benefício | ajuda | uso}
  \definition{v.}{consertar | remendar | preencher | adicionar suplemento | suprir | compensar |nutrir}
\end{entry}

\begin{entry}{补偿}{bu3chang2}{7,11}{⾐、⼈}[HSK 5]
  \definition{v.}{compensar (perda, consumo); compensar (deficiências, diferenças)}
\end{entry}

\begin{entry}{补充}{bu3chong1}{7,6}{⾐、⼉}[HSK 3]
  \definition{adj.}{adicional | suplementar}
  \definition[个]{s.}{aditivo | suplemento}
  \definition{v.}{reabastecer | suplementar | complementar}
\end{entry}

\begin{entry}{补贴}{bu3tie1}{7,9}{⾐、⾙}[HSK 5]
  \definition[笔,项,种,份]{s.}{subsídio; ajuda de custo; custos de indenização ou assistência concedida a empresas ou indivíduos pelo estado ou governo}
  \definition{v.}{subsidiar; compensar a falta de dinheiro ou coisas; refere-se principalmente à compensação financeira ou ajuda dada pelo estado ou governo a empresas ou indivíduos}
\end{entry}

\begin{entry}{不}{bu4}{4}{⼀}[HSK 1]
  \definition{adv.}{(antes de verbos, adjetivos e outros advérbios; nunca antes do verbo 有) não; não vai; não quer | em algumas expressões educadas, significa que não é necessário fazer isso, o que equivale a 不用 ou 不要 | | (entre um verbo e seu complemento) não pode | usado com 就 para indicar escolha}
  \definition{part.}{no final da frase para indicar uma pergunta; (usar sozinho ou com uma partícula nas respostas) não}
  \definition{pref.}{(antes de certos substantivos para formar um adjetivo) un-; in-}
  \seeref{不}{bu2}
  \seeref{不}{bu5}
  \seealsoref{不要}{bu2 yao4}
  \seealsoref{不用}{bu2 yong4}
  \seealsoref{就}{jiu4}
  \seealsoref{有}{you3}
\end{entry}

\begin{entry}{不安}{bu4'an1}{4,6}{⼀、⼧}[HSK 3]
  \definition{adj.}{inquieto | instável | intranquilo | pesaroso}
\end{entry}

\begin{entry}{不曾}{bu4 ceng2}{4,12}{⼀、⽈}[HSK 5]
  \definition{adv.}{nunca (ter feito algo); indica que não aconteceu (negação de 曾经)}
  \seealsoref{曾经}{ceng2jing1}
\end{entry}

\begin{entry}{不成话}{bu4cheng2hua4}{4,6,8}{⼀、⼽、⾔}
  \definition{expr.}{sem razão | demasiado irracionável}
  \seealsoref{不是话}{bu2shi4hua4}
  \seealsoref{不像话}{bu2xiang4hua4}
\end{entry}

\begin{entry}{不得不}{bu4de2bu4}{4,11,4}{⼀、⼻、⼀}[HSK 3]
  \definition{adv.}{tem que | não tem escolha a não ser}
\end{entry}

\begin{entry}{不得了}{bu4de2liao3}{4,11,2}{⼀、⼻、⼅}[HSK 5]
  \definition{adj.}{terrível; horrível; extremamente sério; indica uma situação grave}
  \definition{adv.}{muito; extremamente; excessivamente; indica um grau profundo}
\end{entry}

\begin{entry}{不敢当}{bu4gan3dang1}{4,11,6}{⼀、⽁、⼹}[HSK 5]
  \definition{expr.}{Eu realmente não mereço isso.; Eu não sou digno de tais elogios.; Não estou à altura da honra.; Você me lisonjeia.; palavra de humildade, para mostrar que você não pode pagar (hospitalidade, elogios, etc.)}
\end{entry}

\begin{entry}{不公}{bu4gong1}{4,4}{⼀、⼋}
  \definition{adj.}{injusto}
\end{entry}

\begin{entry}{不管}{bu4guan3}{4,14}{⼀、⽵}[HSK 4]
  \definition{conj.}{não importa (o que, como, etc.); independentemente de; indica que, embora as condições ou circunstâncias tenham mudado, o resultado permanece o mesmo}
  \seealsoref{不管……都……}{bu4guan3 dou1}
  \seealsoref{不管……也……}{bu4guan3 ye3}
\end{entry}

\begin{entry}{不管……都……}{bu4guan3 dou1}{4,14,10}{⼀、⽵、⾢}
  \definition{conj.}{não apenas\dots, (o que, quem, como, etc.), \dots}
\end{entry}

\begin{entry}{不管……也……}{bu4guan3 ye3}{4,14,3}{⼀、⽵、⼄}
  \definition{conj.}{não apenas\dots, (o que, quem, como, etc.), \dots}
\end{entry}

\begin{entry}{不光}{bu4 guang1}{4,6}{⼀、⼉}[HSK 3]
  \definition{adv.}{não é o único}
  \definition{conj.}{não somente}
\end{entry}

\begin{entry}{不好意思}{bu4 hao3 yi4 si5}{4,6,13,9}{⼀、⼥、⼼、⼼}[HSK 2]
  \definition{adj.}{envergonhado; desconfortável; constrangido; sem jeito |}
  \definition{interj.}{com licença; peço desculpas; desculpe-me}
\end{entry}

\begin{entry}{不仅}{bu4jin3}{4,4}{⼀、⼈}[HSK 3]
  \definition{adv.}{não apenas (em número, quantidade ou extensão)}
  \definition{conj.}{não somente}
\end{entry}

\begin{entry}{不久}{bu4 jiu3}{4,3}{⼀、⼃}[HSK 2]
  \definition{adv.}{em breve; dentro em breve; num futuro próximo | logo depois; pouco tempo depois | não muito tempo (antes ou depois de algo)}
\end{entry}

\begin{entry}{不可避免}{bu4ke3bi4mian3}{4,5,16,7}{⼀、⼝、⾌、⼉}
  \definition{adj./adv.}{inevitável}
\end{entry}

\begin{entry}{不良}{bu4 liang2}{4,7}{⼀、⾉}[HSK 5]
  \definition{adj.}{ruim; prejudicial; nocivo; insalubre}
\end{entry}

\begin{entry}{不满}{bu4 man3}{4,13}{⼀、⽔}[HSK 2]
  \definition{adj.}{ressentido; insatisfeito; descontente}
  \definition{v.}{estar descontente com; insatisfação ou descontentamento com alguém ou alguma coisa |ser menor que; quantidade ou tempo insuficientes ou inadequados}
\end{entry}

\begin{entry}{不免}{bu4mian3}{4,7}{⼀、⼉}[HSK 5]
  \definition{adv.}{inevitavelmente}
\end{entry}

\begin{entry}{不能不}{bu4 neng2 bu4}{4,10,4}{⼀、⾁、⼀}[HSK 5]
  \definition{adv.}{tem que; não pode, mas; necessariamente; definitivamente}
\end{entry}

\begin{entry}{不然}{bu4ran2}{4,12}{⼀、⽕}[HSK 4]
  \definition{adj.}{não é assim; não é o caso}
  \definition{conj.}{se não; caso contrário; indica outra consequência ou circunstância que teria ocorrido se não fosse}
\end{entry}

\begin{entry}{不如}{bu4ru2}{4,6}{⼀、⼥}[HSK 2]
  \definition{conj.}{em vez de; melhor do que; seria melhor; preferiria; seria melhor; usado no início da segunda parte da frase, indica uma escolha feita após comparação (geralmente em correspondência com o termo 与其 no texto anterior)}
  \definition{v.}{ser inferior a; não ser igual a; não ser tão bom quanto;  não poder fazer melhor que}
  \seealsoref{与其}{yu3qi2}
\end{entry}

\begin{entry}{不少}{bu4 shao3}{4,4}{⼀、⼩}[HSK 2]
  \definition{adj.}{muitos; bastante; não poucos; indica uma quantidade considerável, equivalente a muitos ou bastante}
\end{entry}

\begin{entry}{不时}{bu4shi2}{4,7}{⼀、⽇}[HSK 5]
  \definition{adv.}{frequentemente; de tempos em tempos | a qualquer momento}
\end{entry}

\begin{entry}{不是……而是}{bu4shi4 er2 shi4}{4,9,6,9}{⼀、⽇、⽽、⽇}
  \definition{conj.}{não somente\dots mas também\dots, expressam um relacionamento mais profundo e avançado em significado, mas as orações antes e depois são consistentes em expressar significados negativos e afirmativos, entretanto, a primeira metade da frase expressa negação, e a segunda metade expressa afirmação, e o significado das orações anteriores e seguintes não pode ser de um nível mais alto}
\end{entry}

\begin{entry}{不停}{bu4 ting2}{4,11}{⼀、⼈}[HSK 5]
  \definition{adv.}{sem parar; sem interrupção; continuamente}
\end{entry}

\begin{entry}{不同}{bu4 tong2}{4,6}{⼀、⼝}[HSK 2]
  \definition{adj.}{diferente; distinto; não semelhante;}
\end{entry}

\begin{entry}{不行}{bu4 xing2}{4,6}{⼀、⾏}[HSK 2]
  \definition{adj.}{não funciona; não é bom; falta de capacidade e habilidade; nível baixo}
  \definition{adv.}{profundamente; terrivelmente; extremamente; expressa um grau muito profundo; incrível (usado após o caractere 得 como complemento)}
  \definition{v.}{não servir; não ser permitido; estar fora de questão | estar à beira da morte}
  \seealsoref{得}{de5}
\end{entry}

\begin{entry}{不许}{bu4 xu3}{4,6}{⼀、⾔}[HSK 5]
  \definition{v.}{não permitir; ser proibido; proibir firmemente | não pode (usado em perguntas retóricas)}
\end{entry}

\begin{entry}{不一定}{bu4 yi2 ding4}{4,1,8}{⼀、⼀、⼧}[HSK 2]
  \definition{adv.}{talvez; incerto; não tenho certeza; não necessariamente assim; refere-se a algo que não pode ser determinado}
\end{entry}

\begin{entry}{不一会儿}{bu4 yi2 hui4r5}{4,1,6,2}{⼀、⼀、⼈、⼉}[HSK 2]
  \definition{expr.}{em um momento; em pouco tempo; em breve; depois de algum tempo}
\end{entry}

\begin{entry}{不止}{bu4zhi3}{4,4}{⼀、⽌}[HSK 5]
  \definition{adv.}{mais do que; não limitado a; indica mais do que esse valor ou intervalo}
  \definition{v.}{exceder; superar; não ser possível interromper a ação}
\end{entry}

\begin{entry}{不足}{bu4zu2}{4,7}{⼀、⾜}[HSK 5]
  \definition{adj.}{não o bastante; inadequado; insuficiente}
  \definition{s.}{deficiência; inadequação; desvantagens, não é bom o suficiente}
  \definition{v.}{não exceder um determinado número | não valer a pena; ser inferior; não merecer | não pode; não deveria}
\end{entry}

\begin{entry}{布}{bu4}{5}{⼱}[HSK 3]
  \definition*{s.}{sobrenome Bu}
  \definition[块,幅,匹]{s.}{pano | tecido | uma moeda de cobre nos tempos antigos}
  \definition{v.}{anunciar | declarar | tornar conhecido | proclamar | publicar | espalhar | disseminar |organizar | implantar | dispor}
\end{entry}

\begin{entry}{布谷鸟}{bu4gu3niao3}{5,7,5}{⼱、⾕、⿃}
  \definition{s.}{cuco (pássaro)}
  \seealsoref{杜鹃}{du4juan1}
  \seealsoref{杜鹃鸟}{du4juan1niao3}
  \seealsoref{杜宇}{du4yu3}
\end{entry}

\begin{entry}{布署}{bu4shu3}{5,13}{⼱、⽹}
  \variantof{部署}
\end{entry}

\begin{entry}{布置}{bu4zhi4}{5,13}{⼱、⽹}[HSK 4]
  \definition{v.}{arrumar; organizar; decorar; colocar adequadamente objetos ou paisagismo, conforme necessário | designar; tomar providências para; dar instruções sobre; organizar trabalho, atividades, etc.}
\end{entry}

\begin{entry}{步}{bu4}{7}{⽌}[HSK 3]
  \definition*{s.}{sobrenome Bu}
  \definition{clas.}{uma unidade antiga para medida de comprimento, equivalente a cinco chi}
  \definition{s.}{ritmo | passo | estágio | passo | condição | situação | estado}
  \definition{v.}{ir a pé | andar | pisar | contar passos}
\end{entry}

\begin{entry}{步行}{bu4 xing2}{7,6}{⽌、⾏}[HSK 4]
  \definition{v.}{caminhar; ir a pé; andar a pé (diferente de andar de carro, a cavalo, etc.)}
\end{entry}

\begin{entry}{部}{bu4}{10}{⾢}[HSK 3]
  \definition{clas.}{para obras de literatura, filmes, máquinas etc.}
  \definition[根]{s.}{departamento | divisão | ministério | seção | parte | tropas}
\end{entry}

\begin{entry}{部队}{bu4dui4}{10,4}{⾢、⾩}
  \definition[个]{s.}{exército | forças armadas | tropas | unidades}
\end{entry}

\begin{entry}{部分}{bu4fen5}{10,4}{⾢、⼑}[HSK 2]
  \definition[个,些,快,份]{s.}{parte; seção; porção; parte do todo; alguns indivíduos dentro do todo | ramo; parte separada de um sistema ou entidade}
\end{entry}

\begin{entry}{部门}{bu4men2}{10,3}{⾢、⾨}[HSK 3]
  \definition[个]{s.}{filial | departamento | divisão | seção}
\end{entry}

\begin{entry}{部属}{bu4shu3}{10,12}{⾢、⼫}
  \definition{s.}{afiliado a um ministério | subordinado | tropas sob comando de alguém}
\end{entry}

\begin{entry}{部署}{bu4shu3}{10,13}{⾢、⽹}
  \definition{s.}{implantação}
  \definition{v.}{implantar}
\end{entry}

\begin{entry}{部位}{bu4wei4}{10,7}{⾢、⼈}[HSK 5]
  \definition{s.}{lugar; posição (usado principalmente para o corpo humano)}
\end{entry}

\begin{entry}{部下}{bu4xia4}{10,3}{⾢、⼀}
  \definition{s.}{subordinado | tropas sob comando de alguém}
\end{entry}

\begin{entry}{部长}{bu4 zhang3}{10,4}{⾢、⾧}[HSK 3]
  \definition[个,位,名]{s.}{ministro | chefe de departamento | chefe de seção}
\end{entry}

\begin{entry}{部族}{bu4zu2}{10,11}{⾢、⽅}
  \definition{adj.}{tribal}
  \definition{s.}{tribo}
\end{entry}

\begin{entry}{不}{bu5}{4}{⼀}[HSK 1]
  \definition{adv.}{não (em expressões \{v.\} + 不 + \{v.\})}
  \seeref{不}{bu2}
  \seeref{不}{bu4}
\end{entry}

%%%%% EOF %%%%%

