%%%
%%% B
%%%

\section*{B}\addcontentsline{toc}{section}{B}

\begin{entry}{八}{ba1}{2}[HSK 1][Kangxi 12][Radical ⼋]
  \definition{num.}{oito; 8}
\end{entry}

\begin{entry}{八八六}{ba1 ba1 liu4}{2,2,4}[Radicais ⼋、⼋、⼋]
  \definition{expr.}{\emph{Bye bye!} (em salas de bate-papo e mensagens de texto)}
\end{entry}

\begin{entry}{巴勒斯坦}{ba1le4si1tan3}{4,11,12,8}[Radicais ⼰、⼒、⽄、⼟]
  \definition*{s.}{Palestina}
\end{entry}

\begin{entry}{巴士}{ba1 shi4}{4,3}[HSK 4][Radicais ⼰、⼠]
  \definition[辆]{s.}{ônibus; transliteração da palavra inglesa ``bus''}
\end{entry}

\begin{entry}{巴西}{ba1xi1}{4,6}[Radicais ⼰、⾑]
  \definition*{s.}{Brasil}
\end{entry}

\begin{entry}{巴西人}{ba1xi1ren2}{4,6,2}[Radicais ⼰、⾑、⼈]
  \definition[个,位]{s.}{brasileiro | pessoa ou povo do Brasil}
  \example{他是巴西人。}[Ele é brasileiro.]
\end{entry}

\begin{entry}{巴西战舞}{ba1xi1zhan4wu3}{4,6,9,14}[Radicais ⼰、⾑、⼽、⾇]
  \definition{s.}{capoeira}
\end{entry}

\begin{entry}{吧}{ba1}{7}[Radical ⼝]
  \definition{s.}{som de estalo, som crepitante}
  \definition{v.}{puxar o cachimbo; fumar | abreviação de ``bar''}
  \seeref{吧}{ba5}
\end{entry}

\begin{entry}{拔尖}{ba2jian1}{8,6}[Radicais ⼿、⼩]
  \definition{adj.}{topo de linha | fora do comum | o melhor}
  \definition{v.+compl.}{empurrar-se para a frente | sentir que é superior aos outros}
\end{entry}

\begin{entry}{把}{ba3}{7}[HSK 3][Radical ⼿]
  \definition{clas.}{para objetos com alça | para objetos pequenos:~punhado}
  \definition{part.}{partícula tornando o substantivo seguinte um objeto direto}
  \definition{v.}{conter | alcançar | segurar}
  \seeref{把}{ba4}
\end{entry}

\begin{entry}{把柄}{ba3bing3}{7,9}[Radicais ⼿、⽊]
  \definition{s.}{(figurativo) informações que podem ser usadas contra alguém}
\end{entry}

\begin{entry}{把持}{ba3chi2}{7,9}[Radicais ⼿、⼿]
  \definition{v.}{controlar | dominar | monopolizar}
\end{entry}

\begin{entry}{把风}{ba3feng1}{7,4}[Radicais ⼿、⾵]
  \definition{v.}{estar atento | vigiar (durante uma atividade clandestina)}
\end{entry}

\begin{entry}{把关}{ba3guan1}{7,6}[Radicais ⼿、⼋]
  \definition{v.}{verificar estritamente | examinar cuidadosamente para ver se algo é feito de acordo com um padrão fixo | fazer a verificação final | guardar uma passagem, fronteira}
\end{entry}

\begin{entry}{把脉}{ba3mai4}{7,9}[Radicais ⼿、⾁]
  \definition{v.}{sentir ou tomar o pulso de alguém}
\end{entry}

\begin{entry}{把式}{ba3shi4}{7,6}[Radicais ⼿、⼷]
  \definition{s.}{pessoa qualificada em um comércio}
\end{entry}

\begin{entry}{把守}{ba3shou3}{7,6}[Radicais ⼿、⼧]
  \definition{v.}{vigiar | guardar}
\end{entry}

\begin{entry}{把玩}{ba3wan2}{7,8}[Radicais ⼿、⽟]
  \definition{v.}{brincar com | mexer com}
\end{entry}

\begin{entry}{把稳}{ba3wen3}{7,14}[Radicais ⼿、⽲]
  \definition{adj.}{confiável}
\end{entry}

\begin{entry}{把握}{ba3wo4}{7,12}[HSK 3][Radicais ⼿、⼿]
  \definition{s.}{seguro | garantia | certeza}
  \definition{v.}{agarrar | segurar | aproveitar}
\end{entry}

\begin{entry}{把戏}{ba3xi4}{7,6}[Radicais ⼿、⼽]
  \definition{s.}{acrobacia | malabarismo | truque barato}
\end{entry}

\begin{entry}{把}{ba4}{7}[Radical ⼿]
  \definition{v.}{lidar}
  \seeref{把}{ba3}
\end{entry}

\begin{entry}{爸}{ba4}{8}[HSK 1][Radical ⽗]
  \definition[个,位]{s.}{(informal) pai}
  \seeref{爸爸}{ba4ba5}
  \seealsoref{爸爸}{ba4ba5}
\end{entry}

\begin{entry}{爸爸}{ba4ba5}{8,8}[HSK 1][Radicais ⽗、⽗]
  \definition[个,位,名,群]{s.}{(informal) pai; papai; papa}
  \seeref{爸}{ba4}
\end{entry}

\begin{entry}{爸妈}{ba4ma1}{8,6}[Radicais ⽗、⼥]
  \definition{s.}{pai e mãe}
\end{entry}

\begin{entry}{罢}{ba4}{10}[Radical ⽹]
  \definition{v.}{parar | cessar | demitir | suspender | desistir | terminar}
  \seeref{罢}{ba5}
\end{entry}

\begin{entry}{霸权}{ba4quan2}{21,6}[Radicais ⾬、⽊]
  \definition{s.}{hegemonia | supremacia}
\end{entry}

\begin{entry}{吧}{ba5}{7}[HSK 1][Radical ⼝]
  \definition{part.}{indica discussão, sugestão, solicitação ou comando no final de uma frase | indica concordância ou aprovação no final de uma frase | indica uma pergunta ou especulação no final de uma frase | indica incerteza no final de uma frase | em uma frase, indica uma pausa, carrega um tom hipotético, frequentemente apresenta um contraste e implica um dilema}
  \seeref{吧}{ba1}
\end{entry}

\begin{entry}{罢}{ba5}{10}[Radical ⽹]
  \definition{part.}{partícula final, a mesma que 吧}
  \seeref{罢}{ba4}
  \seealsoref{吧}{ba5}
\end{entry}

\begin{entry}{白}{bai2}{5}[HSK 1,3][Kangxi 106][Radical ⽩]
  \definition*{s.}{sobrenome Bai}
  \definition{adj.}{branco | claro | puro; claro; simples; sem mistura; em branco | branco (como símbolo de reação) | escrito incorretamente ou pronunciado incorretamente | grátis; sem custos}
  \definition{adv.}{em vão; sem propósito; sem resultados}
  \definition{s.}{parte falada em ópera, etc.; frases de peças de teatro, etc. | dialeto local | funeral}
  \definition{v.}{explicar; apresentar; esclarecer; declarar | branquear | olhar para as pessoas com o branco dos olhos (olhar vazio, de desaprovação)}
\end{entry}

\begin{entry}{白菜}{bai2 cai4}{5,11}[HSK 3][Radicais ⽩、⾋]
  \definition[棵,个]{s.}{acelga | repolho chinês}
\end{entry}

\begin{entry}{白痴}{bai2chi1}{5,13}[Radicais ⽩、⽧]
  \definition{adj./s.}{estúpido | imbecil}
\end{entry}

\begin{entry}{白蛋白}{bai2dan4bai2}{5,11,5}[Radicais ⽩、⾍、⽩]
  \definition{s.}{albumina}
\end{entry}

\begin{entry}{白鹄}{bai2hu2}{5,12}[Radicais ⽩、⿃]
  \definition{s.}{cisne branco}
\end{entry}

\begin{entry}{白拣}{bai2jian3}{5,8}[Radicais ⽩、⼿]
  \definition{s.}{uma escolha barata}
  \definition{v.}{escolher algo que não custa nada}
\end{entry}

\begin{entry}{白萝卜}{bai2luo2bo5}{5,11,2}[Radicais ⽩、⾋、⼘]
  \definition{s.}{rabanete branco}
\end{entry}

\begin{entry}{白色}{bai2 se4}{5,6}[HSK 2][Radicais ⽩、⾊]
  \definition{s.}{cor branca}
\end{entry}

\begin{entry}{白天}{bai2 tian1}{5,4}[HSK 1][Radicais ⽩、⼤]
  \definition{adv.}{dia | de dia}
  \definition[个]{s.}{dia}
\end{entry}

\begin{entry}{白苋}{bai2xian4}{5,7}[Radicais ⽩、⾋]
  \definition{s.}{amaranto branco | brotos e folhas tenras de espinafre chinês usados como alimento}
\end{entry}

\begin{entry}{百}{bai3}{6}[HSK 1][Radical ⽩]
  \definition*{s.}{sobrenome Bai}
  \definition{num.}{cem; 100 | centena | cento}
\end{entry}

\begin{entry}{百般}{bai3ban1}{6,10}[Radicais ⽩、⾈]
  \definition{adv.}{de todas as maneiras possíveis | por todos os meios}
\end{entry}

\begin{entry}{百分}{bai3fen1}{6,4}[Radicais ⽩、⼑]
  \definition{num.}{por cento}
  \definition{s.}{porcentagem}
\end{entry}

\begin{entry}{百货}{bai3 huo4}{6,8}[HSK 4][Radicais ⽩、⾙]
  \definition{s.}{mercadorias em geral; loja de departamentos; um termo geral para bens que incluem principalmente roupas, utensílios e necessidades diárias gerais}
\end{entry}

\begin{entry}{柏树}{bai3shu4}{9,9}[Radicais ⽊、⽊]
  \definition{s.}{cipreste}
\end{entry}

\begin{entry}{摆}{bai3}{13}[HSK 4][Radical ⼿]
  \definition*{s.}{sobrenome Bai | Festival de Ganbai; uma reunião realizada nas áreas Dai durante festivais religiosos, para celebrar uma boa colheita ou para trocar materiais; geralmente se refere a uma reunião em massa}
  \definition{s.}{pêndulo; um dispositivo mecânico que controla a frequência de vibração em relógios e instrumentos | a bainha inferior de um vestido, jaqueta ou saia}
  \definition{v.}{colocar; organizar | vestir; assumir | balançar; acenar; agitar para frente e para trás | expor; declarar claramente; listar | dizer; falar | libertar-se}
\end{entry}

\begin{entry}{摆动}{bai3 dong4}{13,6}[HSK 4][Radicais ⼿、⼒]
  \definition{v.}{balançar; balançar para frente e para trás; oscilar; vibrar}
\end{entry}

\begin{entry}{摆烂}{bai3lan4}{13,9}[Radicais ⼿、⽕]
  \definition{v.}{(neologismo, gíria) parar de lutar (especialmente quando se sabe que não pode ter sucesso) | deixar tudo ir para o inferno}
\end{entry}

\begin{entry}{摆手}{bai3shou3}{13,4}[Radicais ⼿、⼿]
  \definition{v.+compl.}{gesticular com a mão (acenando, acenando adeus, etc.) | balançar os braços | acenar com as mãos}
\end{entry}

\begin{entry}{摆脱}{bai3tuo1}{13,11}[HSK 4][Radicais ⼿、⾁]
  \definition{v.}{sacudir; rejeitar; romper com; libertar-se (ou desembaraçar-se) de; livrar-se de dificuldades, escravidão, controle, etc.}
\end{entry}

\begin{entry}{败}{bai4}{8}[HSK 4][Radical ⾒]
  \definition{adj.}{dilapidado; decadente; murcho; em declínio}
  \definition{v.}{derrota; bater | falhar | quebrar; neutralizar; dissipar | arruinar; estragar; corromper | ser derrotado; perder}
\end{entry}

\begin{entry}{班}{ban1}{10}[HSK 1][Radical ⽟]
  \definition*{s.}{sobrenome Ban}
  \definition{clas.}{para grupos}
  \definition[个]{s.}{equipe| time | esquadrão | turno de trabalho | classificação}
\end{entry}

\begin{entry}{班级}{ban1 ji2}{10,6}[HSK 3][Radicais ⽟、⽷]
  \definition[个]{s.}{classe | série (na escola)}
\end{entry}

\begin{entry}{班长}{ban1 zhang3}{10,4}[HSK 2][Radicais ⽟、⾧]
  \definition[个]{s.}{monitor de classe | líder de equipe | líder de esquadrão}
\end{entry}

\begin{entry}{般}{ban1}{10}[Radical ⾈]
  \definition{s.}{espécie | tipo | classe | caminho | maneira}
  \seeref{般}{bo1}
  \seeref{般}{pan2}
\end{entry}

\begin{entry}{搬}{ban1}{13}[HSK 3][Radical ⼿]
  \definition{v.}{copiar indiscriminadamente | mover-se (ou seja, mudar-se) | mover-se (algo relativamente pesado ou volumoso) | mudar | mudar-se}
\end{entry}

\begin{entry}{搬动}{ban1dong4}{13,6}[Radicais ⼿、⼒]
  \definition{v.}{mover-se (alguma coisa) | se mudar}
\end{entry}

\begin{entry}{搬家}{ban1jia1}{13,10}[HSK 3][Radicais ⼿、⼧]
  \definition{s.}{mudança}
  \definition{v.+compl.}{mudar-se de casa}
\end{entry}

\begin{entry}{搬口}{ban1kou3}{13,3}[Radicais ⼿、⼝]
  \definition{v.}{tagarelar | (idioma) transmitir histórias | semear dissensão | contar histórias}
\end{entry}

\begin{entry}{搬弄}{ban1nong4}{13,7}[Radicais ⼿、⼶]
  \definition{v.}{causar problemas | mexer com alguém | mostrar (o que se pode fazer)}
\end{entry}

\begin{entry}{搬运}{ban1yun4}{13,7}[Radicais ⼿、⾡]
  \definition{s.}{frete | transporte}
  \definition{v.}{carregar | transportar}
\end{entry}

\begin{entry}{搬走}{ban1zou3}{13,7}[Radicais ⼿、⾛]
  \definition{v.}{carregar}
\end{entry}

\begin{entry}{板}{ban3}{8}[HSK 3][Radical ⽊]
  \definition{adj.}{rígido; não natural | duro}
  \definition{clas.}{para cartões, papéis}
  \definition{s.}{tábua; placa; prato | veneziana; persiana; refere-se especificamente aos painéis de portas de lojas | badalos (instrumento musical que marca o ritmo) | uma batida acentuada (ritmo) na música e na ópera tradicional | chefe}
  \definition{v.}{parecer sério | livrar-se de maus hábitos ou falhas}
\end{entry}

\begin{entry}{办}{ban4}{4}[HSK 2][Radical ⼒]
  \definition{v.}{lidar com | lidar | gerenciar | configurar}
\end{entry}

\begin{entry}{办法}{ban4fa3}{4,8}[HSK 2][Radicais ⼒、⽔]
  \definition[条,个]{s.}{meio (de se fazer alguma coisa) | método | medida}
\end{entry}

\begin{entry}{办公}{ban4gong1}{4,4}[Radicais ⼒、⼋]
  \definition{v.+compl.}{lidar com negócios oficiais | trabalhar (especialmente em um escritório)}
\end{entry}

\begin{entry}{办公室}{ban4gong1shi4}{4,4,9}[HSK 2][Radicais ⼒、⼋、⼧]
  \definition[间]{s.}{gabinete | escritório}
\end{entry}

\begin{entry}{办理}{ban4li3}{4,11}[HSK 3][Radicais ⼒、⽟]
  \definition{v.}{conduzir | manusear | transacionar}
\end{entry}

\begin{entry}{办事}{ban4 shi4}{4,8}[HSK 4][Radicais ⼒、⼅]
  \definition{v.}{trabalhar | lidar com assuntos; manipular transações}
\end{entry}

\begin{entry}{半}{ban4}{5}[HSK 1][Radical ⼗]
  \definition{adj.}{incompleto}
  \definition{num.}{(depois de um número) ``e meio''}
  \definition{pref.}{semi}
  \definition{s.}{metade}
\end{entry}

\begin{entry}{半年}{ban4 nian2}{5,6}[HSK 1][Radicais ⼗、⼲]
  \definition{s.}{meio ano}
\end{entry}

\begin{entry}{半球}{ban4qiu2}{5,11}[Radicais ⼗、⽟]
  \definition{s.}{hemisfério}
\end{entry}

\begin{entry}{半天}{ban4 tian1}{5,4}[HSK 1][Radicais ⼗、⼤]
  \definition{s.}{metade do dia | muito tempo | bastante tempo}
\end{entry}

\begin{entry}{半夜}{ban4 ye4}{5,8}[HSK 2][Radicais ⼗、⼣]
  \definition{adv.}{no meio da noite | metade de uma noite}
  \definition{s.}{meia-noite}
\end{entry}

\begin{entry}{半音}{ban4yin1}{5,9}[Radicais ⼗、⾳]
  \definition{s.}{semitom}
\end{entry}

\begin{entry}{伴侣}{ban4lv3}{7,8}[Radicais ⼈、⼈]
  \definition{s.}{companheiro | parceiro}
\end{entry}

\begin{entry}{帮}{bang1}{9}[HSK 1][Radical ⼱]
  \definition{clas.}{para alguém (como uma ajuda)}
  \definition{s.}{gangue | grupo | contratado (como trabalhador) | camada externa | festa | sociedade secreta}
  \definition{v.}{ajudar | apoiar}
\end{entry}

\begin{entry}{帮教}{bang1jiao4}{9,11}[Radicais ⼱、⽁]
  \definition{v.}{orientar}
\end{entry}

\begin{entry}{帮忙}{bang1 mang2}{9,6}[HSK 1][Radicais ⼱、⼼]
  \definition{v.+compl.}{ajudar | dar uma mão | estender a mão | fazer um favor}
\end{entry}

\begin{entry}{帮佣}{bang1yong1}{9,7}[Radicais ⼱、⼈]
  \definition{s.}{ajudante doméstico | servo}
\end{entry}

\begin{entry}{帮助}{bang1zhu4}{9,7}[HSK 2][Radicais ⼱、⼒]
  \definition[种]{s.}{ajuda | assistência}
  \definition{v.}{ajudar | dar assistência}
\end{entry}

\begin{entry}{棒棒糖}{bang4bang4tang2}{12,12,16}[Radicais ⽊、⽊、⽶]
  \definition[根]{s.}{pirulito}
\end{entry}

\begin{entry}{棒冰}{bang4bing1}{12,6}[Radicais ⽊、⼎]
  \definition{s.}{picolé}
\end{entry}

\begin{entry}{包}{bao1}{5}[HSK 1][Radical ⼓]
  \definition*{s.}{sobrenome Bao}
  \definition{clas.}{pacotes, sacos, sacolas, embrulhos}
  \definition[个,只]{s.}{bolsa | pacote | recipiente | embrulho}
  \definition{v.}{contratar | cobrir | segurar ou abraçar | incluir | assumir o comando | embrulhar}
\end{entry}

\begin{entry}{包办}{bao1ban4}{5,4}[Radicais ⼓、⼒]
  \definition{v.}{comandar todo o show | comprometer-se a fazer tudo sozinho}
\end{entry}

\begin{entry}{包干}{bao1gan1}{5,3}[Radicais ⼓、⼲]
  \definition{s.}{tarefa alocada}
  \definition{v.}{ter a responsabilidade total sobre um trabalho}
\end{entry}

\begin{entry}{包裹}{bao1guo3}{5,14}[HSK 4][Radicais ⼓、⾐]
  \definition[个]{s.}{pacote; embrulho}
  \definition{v.}{embrulhar; amarrar; enrolar coisas em pano ou outra coisa}
\end{entry}

\begin{entry}{包含}{bao1han2}{5,7}[HSK 4][Radicais ⼓、⼝]
  \definition{v.}{conter; implicar; incluir; conter dentro, resumir, enfatizar o que está contido dentro, focar em relações internas, muitas vezes coisas abstratas}
\end{entry}

\begin{entry}{包括}{bao1kuo4}{5,9}[HSK 4][Radicais ⼓、⼿]
  \definition{v.}{incluir; compreender; consistir em; conter, conter dentro, resumir junto, enfatizar a listagem de todas as partes, ou a citação de uma parte delas, que podem ser coisas abstratas ou concretas}
\end{entry}

\begin{entry}{包容}{bao1rong2}{5,10}[Radicais ⼓、⼧]
  \definition{adj.}{inclusivo}
  \definition{v.}{perdoar | mostrar tolerância | conter | segurar}
\end{entry}

\begin{entry}{包子}{bao1 zi5}{5,3}[HSK 1][Radicais ⼓、⼦]
  \definition[个]{s.}{pão recheado cozido no vapor}
\end{entry}

\begin{entry}{包租}{bao1zu1}{5,10}[Radicais ⼓、⽲]
  \definition{s.}{aluguel fixo para terras agrícolas}
  \definition{v.}{fretar | alugar | alugar um terreno ou uma casa para subarrendar}
\end{entry}

\begin{entry}{薄}{bao2}{16}[HSK 4][Radical ⾋]
  \definition{adj.}{fino; frágil; pouca espessura |  frio; indiferente; carente de calor; emocionalmente frio; não profundo | leve; fraco | pobre; infértil}
  \seeref{薄}{bo2}
\end{entry}

\begin{entry}{宝}{bao3}{8}[HSK 4][Radical ⼧]
  \definition{adj.}{antigo; precioso; estimado}
  \definition[个,件]{s.}{tesouro; objeto estimado; coisa preciosa | dispositivo de jogo; ferramenta de jogo | dinheiro; moeda; moeda antiga com furo quadrado no centro; moeda de prata}
  \definition{s.}{sobrenome Bao}
\end{entry}

\begin{entry}{宝宝}{bao3 bao5}{8,8}[HSK 4][Radicais ⼧、⼧]
  \definition[个]{s.}{querida; \emph{darling}; \emph{baby}; apelido para crianças}
\end{entry}

\begin{entry}{宝贝}{bao3bei4}{8,4}[HSK 4][Radicais ⼧、⾙]
  \definition{adj.}{excêntrico; estranho; imprestável; um termo depreciativo para uma pessoa incompetente ou ridícula}
  \definition[个,件]{s.}{tesouro; objeto estimado; coisa preciosa | querida; \emph{darling}; \emph{baby}; apelido para crianças}
\end{entry}

\begin{entry}{宝贵}{bao3gui4}{8,9}[HSK 4][Radicais ⼧、⾙]
  \definition{adj.}{precioso; extremamente valioso, muito raro, pode ser usado para descrever coisas específicas, também pode ser usado para descrever coisas abstratas | valioso; como um tesouro}
\end{entry}

\begin{entry}{宝石}{bao3 shi2}{8,5}[HSK 4][Radicais ⼧、⽯]
  \definition[颗,枚,块]{s.}{gema; jóia; pedra preciosa; mineral precioso que tem um brilho lindo e uma dureza de mais de sete graus, não é afetado pela atmosfera ou por produtos químicos e pode ser usado como decoração, suporte de instrumentos ou abrasivos}
\end{entry}

\begin{entry}{饱}{bao3}{8}[HSK 2][Radical ⾷]
  \definition{adj.}{ter comido até ficar satisfeito | estar cheio | cheio}
  \definition{adv.}{completamente | até estar cheio}
  \definition{v.}{satisfazer}
\end{entry}

\begin{entry}{保}{bao3}{9}[HSK 3][Radical ⼈]
  \definition*{s.}{sobrenome Bao}
  \definition{s.}{fiador
oficial responsável
sistema administrativo}
  \definition{v.}{defender | proteger |manter | preservar | conservar em boas condições | garantir | assegurar | ficar como fiador de alguém.}
\end{entry}

\begin{entry}{保安}{bao3 an1}{9,6}[HSK 3][Radicais ⼈、⼧]
  \definition{s.}{guarda de segurança}
  \definition{v.}{manter seguro | garantir a segurança}
\end{entry}

\begin{entry}{保持}{bao3chi2}{9,9}[HSK 3][Radicais ⼈、⼿]
  \definition{v.}{manter | segurar | reter | preservar}
\end{entry}

\begin{entry}{保存}{bao3cun2}{9,6}[HSK 3][Radicais ⼈、⼦]
  \definition{v.}{conservar | preservar | (computação) salvar (um arquivo, etc.)}
\end{entry}

\begin{entry}{保护}{bao3hu4}{9,7}[HSK 3][Radicais ⼈、⼿]
  \definition{s.}{proteção | salvaguarda}
  \definition{v.}{proteger | defender | salvaguardar}
\end{entry}

\begin{entry}{保护国}{bao3hu4guo2}{9,7,8}[Radicais ⼈、⼿、⼞]
  \definition{s.}{protetorado}
\end{entry}

\begin{entry}{保护剂}{bao3hu4ji4}{9,7,8}[Radicais ⼈、⼿、⼑]
  \definition{s.}{agente protetor}
\end{entry}

\begin{entry}{保护区}{bao3hu4qu1}{9,7,4}[Radicais ⼈、⼿、⼖]
  \definition{s.}{área protegida | área de conservação}
\end{entry}

\begin{entry}{保护色}{bao3hu4se4}{9,7,6}[Radicais ⼈、⼿、⾊]
  \definition{s.}{camuflagem}
\end{entry}

\begin{entry}{保护神}{bao3hu4shen2}{9,7,9}[Radicais ⼈、⼿、⽰]
  \definition{s.}{anjo da guarda | santo patrono}
\end{entry}

\begin{entry}{保护物}{bao3hu4 wu4}{9,7,8}[Radicais ⼈、⼿、⽜]
  \definition{s.}{protetor}
\end{entry}

\begin{entry}{保护性}{bao3hu4xing4}{9,7,8}[Radicais ⼈、⼿、⼼]
  \definition{s.}{proteção}
\end{entry}

\begin{entry}{保护者}{bao3hu4zhe3}{9,7,8}[Radicais ⼈、⼿、⽼]
  \definition{s.}{protetor | segurador}
\end{entry}

\begin{entry}{保护主义}{bao3hu4zhu3yi4}{9,7,5,3}[Radicais ⼈、⼿、⼂、⼂]
  \definition{s.}{protecionismo}
\end{entry}

\begin{entry}{保留}{bao3liu2}{9,10}[HSK 3][Radicais ⼈、⽥]
  \definition{v.}{reter | continuar a ter | segurar | reservar}
\end{entry}

\begin{entry}{保密}{bao3mi4}{9,11}[HSK 4][Radicais ⼈、⼧]
  \definition{v.}{manter segredo; manter algo em segredo; manter a confidencialidade}
\end{entry}

\begin{entry}{保守}{bao3shou3}{9,6}[HSK 4][Radicais ⼈、⼧]
  \definition{adj.}{retrógrado; conservador; pensamentos e conceitos que são retrógrados e não conseguem acompanhar o desenvolvimento da situação}
  \definition{v.}{manter; guardar; evitar perder}
\end{entry}

\begin{entry}{保险}{bao3xian3}{9,9}[HSK 3][Radicais ⼈、⾩]
  \definition[个]{adj./s.}{seguro}
  \definition{v.}{ter certeza | estar vinculado a}
\end{entry}

\begin{entry}{保证}{bao3zheng4}{9,7}[HSK 3][Radicais ⼈、⾔]
  \definition[个]{s.}{garantia}
  \definition{v.}{garantir}
\end{entry}

\begin{entry}{报}{bao4}{7}[HSK 3][Radical ⼿]
  \definition[份,张]{s.}{jornal | recompensa | relatório | vingança}
  \definition{v.}{anunciar | informar}
\end{entry}

\begin{entry}{报酬}{bao4chou5}{7,13}[Radicais ⼿、⾣]
  \definition{s.}{recompensa | remuneração}
\end{entry}

\begin{entry}{报到}{bao4dao4}{7,8}[HSK 3][Radicais ⼿、⼑]
  \definition{v.+compl.}{apresentar-se para o serviço | fazer check-in | registrar-se | assinar}
\end{entry}

\begin{entry}{报道}{bao4dao4}{7,12}[HSK 3][Radicais ⼿、⾡]
  \definition[个,篇,分]{s.}{história | reportagem}
  \definition{v.}{cobrir | relatar (notícias)}
\end{entry}

\begin{entry}{报告}{bao4gao4}{7,7}[HSK 3][Radicais ⼿、⼝]
  \definition[份,篇,分,个,通]{s.}{relatório | discurso | palestra | aconselhamento}
  \definition{v.}{relatar | dar a conhecer | informar}
\end{entry}

\begin{entry}{报名}{bao4ming2}{7,6}[HSK 2][Radicais ⼿、⼝]
  \definition{v.+compl.}{matricular-se | alistar-se | inscrever-se | inserir o nome de alguém}
\end{entry}

\begin{entry}{报纸}{bao4zhi3}{7,7}[HSK 2][Radicais ⼿、⽷]
  \definition[张]{s.}{jornal | diário}
\end{entry}

\begin{entry}{抱}{bao4}{8}[HSK 4][Radical ⼿]
  \definition*{s.}{sobrenome Bao}
  \definition{clas.}{braçada; medida dos dois braços}
  \definition{v.}{carregar no peito; segurar com ambos os braços; abraçar | ter o primeiro filho ou neto | adotar um bebê ou criança | ficar juntos, unidos | encaixar ou servir perfeitamente (roupas e sapatos do tamanho certo) | estimar; nutrir; abrigar; ter em mente | continuar; sobrecarregar com | chocar ovos}
\end{entry}

\begin{entry}{抱怨}{bao4yuan4}{8,9}[Radicais ⼿、⼼]
  \definition{v.}{reclamar | resmungar | abrir uma reclamação | sentir-se insatisfeito}
\end{entry}

\begin{entry}{豹子}{bao4zi5}{10,3}[Radicais ⾘、⼦]
  \definition[头]{s.}{leopardo}
\end{entry}

\begin{entry}{暴力}{bao4li4}{15,2}[Radicais ⽇、⼒]
  \definition{adj.}{violento}
  \definition{s.}{violência}
\end{entry}

\begin{entry}{暴乱}{bao4luan4}{15,7}[Radicais ⽇、⼄]
  \definition{s.}{rebelião | revolta | tumulto}
\end{entry}

\begin{entry}{暴行}{bao4xing2}{15,6}[Radicais ⽇、⾏]
  \definition{s.}{ato selvagem | atrocidade | indignação}
\end{entry}

\begin{entry}{暴雨}{bao4yu3}{15,8}[Radicais ⽇、⾬]
  \definition[场,阵]{s.}{tempestade | chuva torrencial}
\end{entry}

\begin{entry}{暴躁}{bao4zao4}{15,20}[Radicais ⽇、⾜]
  \definition{adj.}{irascível | irritável}
\end{entry}

\begin{entry}{爆米花}{bao4mi3hua1}{19,6,7}[Radicais ⽕、⽶、⾋]
  \definition{s.}{pipoca (de milho) | pipoca de arroz}
\end{entry}

\begin{entry}{爆炸}{bao4zha4}{19,9}[Radicais ⽕、⽕]
  \definition{s.}{explosão}
  \definition{v.}{explodir | detonar}
\end{entry}

\begin{entry}{杯}{bei1}{8}[HSK 1][Radical ⽊]
  \definition{clas.}{para certos recipientes de líquidos: copo, xícara, etc.}
  \definition{s.}{copo | caneca | xícara | taça | troféu}
\end{entry}

\begin{entry}{杯具}{bei1ju4}{8,8}[Radicais ⽊、⼋]
  \definition{s.}{parachoque | fiasco | (gíria) tragédia}
\end{entry}

\begin{entry}{杯子}{bei1 zi5}{8,3}[HSK 1][Radicais ⽊、⼦]
  \definition[个,只]{s.}{copo | caneca | xícara | taça}
\end{entry}

\begin{entry}{背}{bei1}{9}[HSK 2][Radical ⾁]
  \definition{v.}{estar sobrecarregado | carregar nas costas ou no ombro}
  \seeref{背}{bei4}
\end{entry}

\begin{entry}{北}{bei3}{5}[HSK 1][Radical ⼔]
  \definition{s.}{norte}
  \definition{v.}{(clássico) ser derrotado}
\end{entry}

\begin{entry}{北边}{bei3 bian1}{5,5}[HSK 1][Radicais ⼔、⾡]
  \definition{adv.}{lado norte | ao norte de}
\end{entry}

\begin{entry}{北部}{bei3 bu4}{5,10}[HSK 3][Radicais ⼔、⾢]
  \definition{s.}{parte norte}
\end{entry}

\begin{entry}{北大西洋公约组织}{bei3 da4xi1 yang2 gong1 yue1 zu3zhi1}{5,3,6,9,4,6,8,8}[Radicais ⼔、⼤、⾑、⽔、⼋、⽷、⽷、⽷]
  \definition*{s.}{Organização do Tratado do Atlântico Norte, OTAN}
\end{entry}

\begin{entry}{北方}{bei3fang1}{5,4}[HSK 2][Radicais ⼔、⽅]
  \definition{s.}{norte | a parte norte de um país}
\end{entry}

\begin{entry}{北极}{bei3ji2}{5,7}[Radicais ⼔、⽊]
  \definition*{s.}{Ártico | Pólo Norte}
  \definition{s.}{pólo norte magnético}
\end{entry}

\begin{entry}{北京}{bei3 jing1}{5,8}[HSK 1][Radicais ⼔、⼇]
  \definition*{s.}{Beijing (Pequim), Capital da República Popular da China | Beijing (Pequim), governo da RPC}
\end{entry}

\begin{entry}{北面}{bei3mian4}{5,9}[Radicais ⼔、⾯]
  \definition{s.}{lado norte}
\end{entry}

\begin{entry}{北约}{bei3yue1}{5,6}[Radicais ⼔、⽷]
  \definition*{s.}{OTAN (Organização do Tratado do Atlântico Norte), abreviação de 北大西洋公约组织}
  \seeref{北大西洋公约组织}{bei3 da4xi1 yang2 gong1 yue1 zu3zhi1}
\end{entry}

\begin{entry}{备份}{bei4fen4}{8,6}[Radicais ⼡、⼈]
  \definition{s.}{cópia de segurança | \emph{backup}}
\end{entry}

\begin{entry}{备胎}{bei4tai1}{8,9}[Radicais ⼡、⾁]
  \definition{s.}{pneu sobressalente | (gíria) substituto}
\end{entry}

\begin{entry}{背}{bei4}{9}[HSK 3][Radical ⾁]
  \definition{adv.}{a parte de trás de um corpo ou objeto}
  \definition{s.}{costas | (gíria) azarado}
  \definition{v.}{esconder algo de | decorar | recitar de memória | virar as costas}
  \seeref{背}{bei1}
\end{entry}

\begin{entry}{背后}{bei4 hou4}{9,6}[HSK 3][Radicais ⾁、⼝]
  \definition{s.}{parte de trás | traseira | nas costas de alguém}
\end{entry}

\begin{entry}{背景}{bei4jing3}{9,12}[HSK 4][Radicais ⾁、⽇]
  \definition[种]{s.}{pano de fundo; fundo; cenário de teatro, filme ou drama de TV | fundo; cenário que permeia a imagem principal na tela | condições sociais; ambientes históricos (significativamente influentes para algo ou alguém) | poder que dá suporte a alguém}
\end{entry}

\begin{entry}{倍}{bei4}{10}[HSK 4][Radical ⼈]
  \definition{adv.}{mais; especialmente}
  \definition{clas.}{vezes; para obter um número igual ao número original, você pode multiplicar o número por esse múltiplo}
  \definition{s.}{dobro; duas vezes mais}
\end{entry}

\begin{entry}{被}{bei4}{10}[HSK 3][Radical ⾐]
  \definition*{s.}{sobrenome Bei}
  \definition{part.}{usada antes de verbos para formar frases verbais passivas}
  \definition{prep.}{usado em uma frase para indicar que o sujeito é o receptor da ação}
  \definition{s.}{colcha}
  \definition{v.}{cobrir; espalhar
sofrer}
\end{entry}

\begin{entry}{被单}{bei4dan1}{10,8}[Radicais ⾐、⼗]
  \definition[床]{s.}{lençol}
\end{entry}

\begin{entry}{被动}{bei4dong4}{10,6}[Radicais ⾐、⼒]
  \definition{adj.}{passivo}
\end{entry}

\begin{entry}{被告}{bei4gao4}{10,7}[Radicais ⾐、⼝]
  \definition{s.}{réu}
\end{entry}

\begin{entry}{被迫}{bei4 po4}{10,8}[HSK 4][Radicais ⾐、⾡]
  \definition{v.}{ser forçado; ser coagido; ser compelido; ser constrangido; ser forçado a fazer algo por força externa}
\end{entry}

\begin{entry}{被套}{bei4tao4}{10,10}[Radicais ⾐、⼤]
  \definition{s.}{capa de \emph{edredon}}
  \definition{v.}{ter dinheiro preso (em ações, imóveis, etc.)}
\end{entry}

\begin{entry}{被窝}{bei4wo1}{10,12}[Radicais ⾐、⽳]
  \definition{s.}{colcha}
\end{entry}

\begin{entry}{被子}{bei4zi5}{10,3}[HSK 3][Radicais ⾐、⼦]
  \definition[床]{s.}{colcha}
\end{entry}

\begin{entry}{本}{ben3}{5}[HSK 1][Radical ⽊]
  \definition{adj.}{o atual | original | inerente}
  \definition{adv.}{originalmente}
  \definition{clas.}{para livros, dicionários, periódicos, arquivos, etc.}
  \definition{s.}{raiz | caule | origem | fonte}
\end{entry}

\begin{entry}{本金}{ben3 jin1}{5,8}[Radicais ⽊、⾦]
  \definition{s.}{capital; capital para a operação do comércio e da indústria; capital para a operação de negócios |
valor principal; dinheiro retirado ao depositar ou tomar emprestado (diferente de ``利息'')}
  \seealsoref{利息}{li4xi1}
\end{entry}

\begin{entry}{本科}{ben3ke1}{5,9}[HSK 4][Radicais ⽊、⽲]
  \definition{s.}{graduação; bacharelado; o curso básico de uma universidade ou faculdade}
\end{entry}

\begin{entry}{本来}{ben3lai2}{5,7}[HSK 3][Radicais ⽊、⽊]
  \definition{adv.}{originalmente | apropriadamente | legalmente}
\end{entry}

\begin{entry}{本领}{ben3 ling3}{5,11}[HSK 3][Radicais ⽊、⾴]
  \definition[项,个]{s.}{capacidade | faculdade | poder | habilidade | talento}
\end{entry}

\begin{entry}{本事}{ben3shi4}{5,8}[Radicais ⽊、⼅]
  \definition{s.}{habilidade | capacidade | \emph{status} | poder | posição | autoridade}
  \seeref{本事}{ben3shi5}
\end{entry}

\begin{entry}{本事}{ben3shi5}{5,8}[HSK 3][Radicais ⽊、⼅]
  \definition{s.}{habilidade | capacidade |\emph{status} | poder | posição | autoridade}
  \seeref{本事}{ben3shi4}
\end{entry}

\begin{entry}{本子}{ben3 zi5}{5,3}[HSK 1][Radicais ⽊、⼦]
  \definition[本]{s.}{caderno}
\end{entry}

\begin{entry}{笨}{ben4}{11}[HSK 4][Radical ⽵]
  \definition{adj.}{estúpido; sem graça; tolo; de pouca habilidade; sem inteligência | desajeitado; tosco; inflexível | incômodo; pesado; desajeitado; difícil de manejar; trabalhoso}
\end{entry}

\begin{entry}{笨蛋}{ben4dan4}{11,11}[Radicais ⽵、⾍]
  \definition{s.}{bobalhão | cabeça-oca | cabeça-dura}
  \definition{v.}{iludir | enganar}
\end{entry}

\begin{entry}{崩}{beng1}{11}[Radical ⼭]
  \definition{s.}{morte de rei ou imperador | desaparecimento}
  \definition{v.}{entrar em colapso | cair em ruínas}
\end{entry}

\begin{entry}{绷带}{beng1dai4}{11,9}[Radicais ⽷、⼱]
  \definition{s.}{curativo | (empréstimo linguístico) \emph{bandage}}
\end{entry}

\begin{entry}{甭}{beng2}{9}[Radical ⽤]
  \definition{v.}{contração de 不用 | não precisar}
  \seeref{不用}{bu2 yong4}
\end{entry}

\begin{entry}{蹦极}{beng4ji2}{18,7}[Radicais ⾜、⽊]
  \definition{s.}{\emph{bungee jumping}}
\end{entry}

\begin{entry}{鼻子}{bi2zi5}{14,3}[Radicais ⿐、⼦]
  \definition[个,只]{s.}{nariz}
\end{entry}

\begin{entry}{比}{bi3}{4}[HSK 1][Kangxi 81][Radical ⽐]
  \definition*{s.}{Bélgica, abreviação de 比利时}
  \definition{part.}{partícula usada para comparação (superioridade)}
  \definition{prep.}{que | do que | (seguido por um substantivo e adjetivo) mais \{adj.\} do que \{s.\}}
  \definition{s.}{razão (taxa)}
  \definition{v.}{comparar | contrastar | gesticular (com as mãos)}
  \seeref{比利时}{bi3li4shi2}
\end{entry}

\begin{entry}{比分}{bi3 fen1}{4,4}[HSK 4][Radicais ⽐、⼑]
  \definition{s.}{pontuação; comparação de pontuações entre as duas equipes em uma partida}
\end{entry}

\begin{entry}{比较}{bi3jiao4}{4,10}[HSK 3][Radicais ⽐、⾞]
  \definition{adv.}{comparativamente | relativamente}
  \definition{s.}{comparação}
  \definition{v.}{comparar}
\end{entry}

\begin{entry}{比利时}{bi3li4shi2}{4,7,7}[Radicais ⽐、⼑、⽇]
  \definition*{s.}{Bélgica}
\end{entry}

\begin{entry}{比例}{bi3li4}{4,8}[HSK 3][Radicais ⽐、⼈]
  \definition{s.}{escala | razão | proporção}
\end{entry}

\begin{entry}{比拼}{bi3pin1}{4,9}[Radicais ⽐、⼿]
  \definition{s.}{concurso}
  \definition{v.}{competir ferozmente}
\end{entry}

\begin{entry}{比如}{bi3ru2}{4,6}[HSK 2][Radicais ⽐、⼥]
  \definition{conj.}{por exemplo | como}
\end{entry}

\begin{entry}{比如说}{bi3 ru2 shuo1}{4,6,9}[HSK 2][Radicais ⽐、⼥、⾔]
  \definition{adv.}{por exemplo}
\end{entry}

\begin{entry}{比萨饼}{bi3sa4bing3}{4,11,9}[Radicais ⽐、⾋、⾷]
  \definition[张]{s.}{pizza}
\end{entry}

\begin{entry}{比赛}{bi3sai4}{4,14}[HSK 3][Radicais ⽐、⾙]
  \definition[场,次]{s.}{competição | concurso}
  \definition{v.}{competir}
\end{entry}

\begin{entry}{比亚迪}{bi3ya4di2}{4,6,8}[Radicais ⽐、⼆、⾡]
  \definition*{s.}{Montadora BYD}
\end{entry}

\begin{entry}{笔}{bi3}{10}[HSK 2][Radical ⽵]
  \definition{clas.}{para somas de dinheiro, negócios}
  \definition[支,枝]{s.}{caneta | lápis}
\end{entry}

\begin{entry}{笔记}{bi3 ji4}{10,5}[HSK 2][Radicais ⽵、⾔]
  \definition[篇,本,个]{s.}{notas | ensaios | esboços}
  \definition{v.}{tomar nota (por escrito)}
\end{entry}

\begin{entry}{笔记本}{bi3ji4ben3}{10,5,5}[HSK 2][Radicais ⽵、⾔、⽊]
  \definition[本]{s.}{caderno}
  \definition{s.}{\emph{laptop}}
\end{entry}

\begin{entry}{必定}{bi4ding4}{5,8}[Radicais ⼼、⼧]
  \definition{adv.}{sem falta | certamente | com certeza | definitivamente | inevitavelmente | com determinação}
  \definition{v.}{estar vinculado a | ter certeza de}
\end{entry}

\begin{entry}{必然}{bi4ran2}{5,12}[HSK 3][Radicais ⼼、⽕]
  \definition{adj.}{certo | inevitável | necessário}
  \definition{adv.}{inevitavelmente}
  \definition{s.}{necessidade}
\end{entry}

\begin{entry}{必须}{bi4xu1}{5,9}[HSK 2][Radicais ⼼、⾴]
  \definition{adv.}{necessariamente | obrigatoriamente}
\end{entry}

\begin{entry}{必要}{bi4yao4}{5,9}[HSK 3][Radicais ⼼、⾑]
  \definition{adj.}{necessário | essencial | indispensável}
  \definition[个,些]{s.}{necessidade}
\end{entry}

\begin{entry}{毕业}{bi4ye4}{6,5}[HSK 4][Radicais ⽐、⼀]
  \definition{s.}{formatura}
  \definition{v.+compl.}{formar-se}
\end{entry}

\begin{entry}{毕业生}{bi4 ye4 sheng1}{6,5,5}[HSK 4][Radicais ⽐、⼀、⽣]
  \definition[个]{s.}{diplomado; graduado; bacharel; pessoa que recebeu um diploma, grau ou certificado}
\end{entry}

\begin{entry}{闭嘴}{bi4zui3}{6,16}[Radicais ⾨、⼝]
  \definition{expr.}{Cale-se!}
\end{entry}

\begin{entry}{壁虎}{bi4hu3}{16,8}[Radicais ⼟、⾌]
  \definition{s.}{lagartixa}
\end{entry}

\begin{entry}{壁纸}{bi4zhi3}{16,7}[Radicais ⼟、⽷]
  \definition{s.}{papel de parede}
\end{entry}

\begin{entry}{避}{bi4}{16}[HSK 4][Radical ⾌]
  \definition{v.}{evitar; evadir; esquivar-se; buscar abrigo; fugir | impedir; manter afastado; repelir; previnir}
\end{entry}

\begin{entry}{避免}{bi4mian3}{16,7}[HSK 4][Radicais ⾌、⼉]
  \definition{v.}{evitar; desviar; abster-se de; tentar não fazer com que algo aconteça; prevenir; tentar impedir (que algo ruim aconteça) com antecedência}
\end{entry}

\begin{entry}{边}{bian1}{5}[HSK 2][Radical ⾡]
  \definition{adv.}{simultaneamente}
  \definition[个]{s.}{fronteira | limite | borda | margem | lado}
  \seeref{边}{bian5}
\end{entry}

\begin{entry}{边防}{bian1fang2}{5,6}[Radicais ⾡、⾩]
  \definition{s.}{defesa da fronteira}
\end{entry}

\begin{entry}{边关}{bian1guan1}{5,6}[Radicais ⾡、⼋]
  \definition{s.}{posto de fronteira | posição defensiva estratégica na fronteira}
\end{entry}

\begin{entry}{编}{bian1}{12}[HSK 4][Radical ⽷]
  \definition*{s.}{sobrenome Bian}
  \definition{s.}{livro; volume; parte de um livro}
  \definition{v.}{tecer; trançar; entrançar | fazer uma lista; organizar em uma lista; organizar; agrupar | editar; compilar | compor; escrever | fabricar; inventar; fazer; preparar}
\end{entry}

\begin{entry}{编程}{bian1cheng2}{12,12}[Radicais ⽷、⽲]
  \definition{s.}{programa de computador}
  \definition{v.}{programar computador}
\end{entry}

\begin{entry}{邉}{bian1}{17}[Radical ⾡]
  \variantof{边}
\end{entry}

\begin{entry}{变}{bian4}{8}[HSK 2][Radical ⼜]
  \definition{v.}{mudar | transformar | variar}
\end{entry}

\begin{entry}{变成}{bian4 cheng2}{8,6}[HSK 2][Radicais ⼜、⼽]
  \definition{v.}{mudar | transformar-se em | tornar-se}
\end{entry}

\begin{entry}{变更}{bian4geng1}{8,7}[Radicais ⼜、⽈]
  \definition{v.}{alterar | mudar | modificar}
\end{entry}

\begin{entry}{变化}{bian4hua4}{8,4}[HSK 3][Radicais ⼜、⼔]
  \definition[个]{s.}{mudança | variação}
  \definition{v.}{(intransitivo) mudar, variar}
\end{entry}

\begin{entry}{变节}{bian4jie2}{8,5}[Radicais ⼜、⾋]
  \definition{s.}{traição | deserção | vira-casaca}
  \definition{v.}{mudar de lado politicamente}
\end{entry}

\begin{entry}{变迁}{bian4qian1}{8,6}[Radicais ⼜、⾡]
  \definition{s.}{mudanças | vicissitudes}
\end{entry}

\begin{entry}{变数}{bian4shu4}{8,13}[Radicais ⼜、⽁]
  \definition{s.}{(matemática) variável}
\end{entry}

\begin{entry}{变为}{bian4 wei2}{8,4}[HSK 3][Radicais ⼜、⼂]
  \definition{v.}{transformar-se em | tornar-se | mudar para}
\end{entry}

\begin{entry}{变心}{bian4xin1}{8,4}[Radicais ⼜、⼼]
  \definition{v.+compl.}{deixar de ser fiel}
\end{entry}

\begin{entry}{变性}{bian4xing4}{8,8}[Radicais ⼜、⼼]
  \definition{s.}{desnaturação | transexual}
  \definition{v.}{desnaturar | mudar de sexo}
\end{entry}

\begin{entry}{变异}{bian4yi4}{8,6}[Radicais ⼜、⼶]
  \definition{s.}{variação | mutação}
\end{entry}

\begin{entry}{变装}{bian4zhuang1}{8,12}[Radicais ⼜、⾐]
  \definition{v.}{trocar de roupa | vestir-se | vestir uma fantasia | disfarçar-se ou fantasiar-se de personagem real ou ficcional, \emph{cosplay} | travestir-se}
\end{entry}

\begin{entry}{遍}{bian4}{12}[HSK 2][Radical ⾡]
  \definition{adv.}{em todos os lugares | por toda parte}
  \definition{clas.}{para a repetição de ações de leitura, fala ou escrita}
\end{entry}

\begin{entry}{辩论}{bian4lun4}{16,6}[HSK 4][Radicais ⾟、⾔]
  \definition[场,次]{s.}{debate; argumento; a atividade comportamental em si de argumentar ou refutar diferentes pontos de vista ou afirmações, ou uma ocasião ou situação em que tal argumentação ou refutação é feita}
  \definition{v.}{debater; obter um entendimento unificado ou correto, ambos os lados usam linguagem, palavras etc. para explicar seus pontos de vista, apontar os erros ou as contradições do outro lado}
\end{entry}

\begin{entry}{辫子}{bian4zi5}{17,3}[Radicais ⾟、⼦]
  \definition[根,条]{s.}{trança | um erro ou falha que pode ser explorado por um oponente | alça}
\end{entry}

\begin{entry}{边}{bian5}{5}[Radical ⾡]
  \definition{suf.}{sufixo de uma palavra de localidade}
  \seeref{边}{bian1}
\end{entry}

\begin{entry}{标题}{biao1ti2}{9,15}[HSK 3][Radicais ⽊、⾴]
  \definition[个,条,篇]{s.}{título | manchete | cabeçalho}
\end{entry}

\begin{entry}{标志}{biao1zhi4}{9,7}[HSK 4][Radicais ⽊、⼼]
  \definition[个,种]{s.}{sinal; marca; logotipo; símbolo; emblema; marcações que caracterizam um objeto para facilitar a identificação}
  \definition{v.}{marcar; indicar; simbolizar; identificar}
\end{entry}

\begin{entry}{标准}{biao1zhun3}{9,10}[HSK 3][Radicais ⽊、⼎]
  \definition{adj.}{criterioso | padronizado | normatizado}
  \definition[个]{s.}{critério | padrão (oficial) | norma}
\end{entry}

\begin{entry}{镖}{biao1}{16}[Radical ⾦]
  \definition{s.}{dardo | arma de arremesso | mercadorias enviadas sob a proteção de uma escolta armada}
\end{entry}

\begin{entry}{表}{biao3}{8}[HSK 2][Radical ⾐]
  \definition*{s.}{sobrenome Biao}
  \definition{s.}{superfície externa | a relação entre os filhos ou netos de um irmão e uma irmã ou de irmãs | exemplo | modelo | memorial a um imperador dos tempos antigos | gráfico | formulário | lista | tabela | medidor | relógio de pulso}
\end{entry}

\begin{entry}{表白}{biao3bai2}{8,5}[Radicais ⾐、⽩]
  \definition{s.}{declaração | confissão}
  \definition{v.}{confessar a si mesmo | expressar | revelar pensamentos ou sentimentos de alguém}
\end{entry}

\begin{entry}{表达}{biao3da2}{8,6}[HSK 3][Radicais ⾐、⾡]
  \definition{v.}{entregar | expressar | mostrar | transmitir | comunicar}
\end{entry}

\begin{entry}{表格}{biao3ge2}{8,10}[HSK 3][Radicais ⾐、⽊]
  \definition[份,张]{s.}{tabela | formulário}
\end{entry}

\begin{entry}{表面}{biao3mian4}{8,9}[HSK 3][Radicais ⾐、⾯]
  \definition{s.}{superfície | lado de fora | aparência | superficialidade}
\end{entry}

\begin{entry}{表明}{biao3ming2}{8,8}[HSK 3][Radicais ⾐、⽇]
  \definition{v.}{deixar claro | tornar conhecido | declarar claramente}
\end{entry}

\begin{entry}{表情}{biao3qing2}{8,11}[HSK 4][Radicais ⾐、⼼]
  \definition[个,种,幅]{s.}{expressão; expressão facial; expressão de pensamentos e sentimentos internos por meio de mudanças faciais ou de gestos}
  \definition{v.}{expressar pensamentos e sentimentos internos por meio de mudanças faciais ou de gestos}
\end{entry}

\begin{entry}{表示}{biao3shi4}{8,5}[HSK 2][Radicais ⾐、⽰]
  \definition{s.}{expressão | indicação}
  \definition{v.}{expressar | mostrar | indicar | significar}
\end{entry}

\begin{entry}{表现}{biao3xian4}{8,8}[HSK 3][Radicais ⾐、⾒]
  \definition[个,种,份]{s.}{desempenho | expressão  manifestação | comportamento}
  \definition{v.}{mostrar | expressar | exibir | manifestar | descrever}
\end{entry}

\begin{entry}{表演}{biao3yan3}{8,14}[HSK 3][Radicais ⾐、⽔]
  \definition[场]{s.}{representação | atuação | exposição}
  \definition{v.}{executar | atuar | jogar | demonstrar | agir | fingir}
\end{entry}

\begin{entry}{表演赛}{biao3yan3sai4}{8,14,14}[Radicais ⾐、⽔、⾙]
  \definition{s.}{partida ou jogo de exibição}
\end{entry}

\begin{entry}{表演特技}{biao3yan3 te4ji4}{8,14,10,7}[Radicais ⾐、⽔、⽜、⼿]
  \definition{s.}{acrobacia | pirueta | façanha}
\end{entry}

\begin{entry}{表演艺术}{biao3yan3 yi4shu4}{8,14,4,5}[Radicais ⾐、⽔、⾋、⽊]
  \definition{s.}{arte performática}
\end{entry}

\begin{entry}{表演游戏}{biao3yan3 you2xi4}{8,14,12,6}[Radicais ⾐、⽔、⽔、⼽]
  \definition{s.}{exibição dramática}
\end{entry}

\begin{entry}{表演者}{biao3yan3 zhe3}{8,14,8}[Radicais ⾐、⽔、⽼]
  \definition{s.}{ator}
\end{entry}

\begin{entry}{表扬}{biao3yang2}{8,6}[HSK 4][Radicais ⾐、⼿]
  \definition[次,种,份]{s.}{elogios públicos por boas ações}
  \definition{v.}{elogiar; louvar}
\end{entry}

\begin{entry}{表扬信}{biao3yang2 xin4}{8,6,9}[Radicais ⾐、⼿、⼈]
  \definition{s.}{carta de elogio | depoimento}
\end{entry}

\begin{entry}{别}{bie2}{7}[HSK 1,4][Radical ⼑]
  \definition*{s.}{sobrenome Bie}
  \definition{adv.}{não; nada de (pedir a alguém para não fazer); é melhor não | talvez, usado em conjunto com a palavra ``是'' para indicar especulação.}
  \definition{pron.}{outro; algum outro}
  \definition{s.}{distinção; diferença | classificação}
  \definition{v.}{deixar; partir; separar | diferenciar; distinguir; encontrar aspectos diferentes | fixar objetos com pinos | girar; virar | aderir; colar; preder}
  \seeref{别}{bie4}
  \seealsoref{是}{shi4}
\end{entry}

\begin{entry}{别的}{bie2 de5}{7,8}[HSK 1][Radicais ⼑、⽩]
  \definition{pron.}{outro}
\end{entry}

\begin{entry}{别人}{bie2ren5}{7,2}[Radicais ⼑、⼈]
  \definition{pron.}{outra pessoa | outro povo | outros}
\end{entry}

\begin{entry}{别说}{bie2shuo1}{7,9}[Radicais ⼑、⾔]
  \definition{v.}{não falar de | não mencionar}
\end{entry}

\begin{entry}{别}{bie4}{7}[Radical ⼑]
  \definition{v.}{fazer com que alguém mude seus hábitos, opiniões, etc.}
  \seeref{别}{bie2}
\end{entry}

\begin{entry}{宾馆}{bin1guan3}{10,11}[Radicais ⼧、⾷]
  \definition[个,家]{s.}{casa de hóspedes | hotel}
\end{entry}

\begin{entry}{冰}{bing1}{6}[HSK 4][Radical ⼎]
  \definition{adj.}{frio (pessoa)| hostil}
  \definition[块]{s.}{gelo; água em estado sólido |  (gíria) metanfetamina}
  \definition{v.}{colocar gelo; colocar gelo ao redor; colocar no gelo; resfriar objetos com gelo ou água fria | sentir frio}
\end{entry}

\begin{entry}{冰糕}{bing1gao1}{6,16}[Radicais ⼎、⽶]
  \definition{s.}{sorvete | picolé}
\end{entry}

\begin{entry}{冰棍}{bing1gun4}{6,12}[Radicais ⼎、⽊]
  \definition[根]{s.}{picolé}
\end{entry}

\begin{entry}{冰激凌}{bing1ji1ling2}{6,16,10}[Radicais ⼎、⽔、⼎]
  \definition{s.}{sorvete}
\end{entry}

\begin{entry}{冰球}{bing1qiu2}{6,11}[Radicais ⼎、⽟]
  \definition{s.}{hóquei no gelo}
\end{entry}

\begin{entry}{冰天雪地}{bing1tian1-xue3di4}{6,4,11,6}[Radicais ⼎、⼤、⾬、⼟]
  \definition{expr.}{um mundo de gelo e neve}
\end{entry}

\begin{entry}{冰箱}{bing1xiang1}{6,15}[HSK 4][Radicais ⼎、⾋]
  \definition[台,个]{s.}{geladeira; freezer; refrigerador; aparelhos para congelar alimentos ou medicamentos com gelo para mantê-los frios}
\end{entry}

\begin{entry}{冰雪}{bing1 xue3}{6,11}[HSK 4][Radicais ⼎、⾬]
  \definition{adj.}{puro como gelo e neve; descreve uma pessoa como pura}
  \definition{s.}{gelo e neve}
\end{entry}

\begin{entry}{兵}{bing1}{7}[HSK 4][Radical ⼋]
  \definition[名]{s.}{armas; armamentos | soldado; pessoal militar | exército; tropas | soldado raso; membro mais jovem do exército | assuntos militares (estratégia) | peão, uma das peças do xadrez chinês}
\end{entry}

\begin{entry}{兵器}{bing1qi4}{7,16}[Radicais ⼋、⼝]
  \definition{s.}{armas | armamento}
\end{entry}

\begin{entry}{饼}{bing3}{9}[Radical ⾷]
  \definition[张]{s.}{panqueca | biscoito | torta}
\end{entry}

\begin{entry}{饼干}{bing3gan1}{9,3}[Radicais ⾷、⼲]
  \definition[片,块]{s.}{bolacha | biscoito}
\end{entry}

\begin{entry}{并}{bing4}{6}[HSK 3,4][Radical ⼲]
  \definition{adv.}{igualmente; simultaneamente; lado a lado; coisas diferentes existem ao mesmo tempo; coisas diferentes estão acontecendo ao mesmo tempo | em absoluto (usado antes de uma negativa para dar ênfase);  usado antes de uma palavra negativa para reforçar o tom e refutá-la ligeiramente}
  \definition{conj.}{além de; e}
  \definition{v.}{combinar; fundir; incorporar; anexar; juntar}
\end{entry}

\begin{entry}{并排}{bing4pai2}{6,11}[Radicais ⼲、⼿]
  \definition{adv.}{lado a lado}
\end{entry}

\begin{entry}{并且}{bing4qie3}{6,5}[HSK 3][Radicais ⼲、⼀]
  \definition{conj.}{além disso | o que é mais | e}
\end{entry}

\begin{entry}{幷}{bing4}{8}[Radical ⼲]
  \variantof{并}
\end{entry}

\begin{entry}{倂}{bing4}{10}[Radical ⼈]
  \variantof{并}
\end{entry}

\begin{entry}{病}{bing4}{10}[HSK 1][Radical ⽧]
  \definition[场]{s.}{doença}
  \definition{v.}{adoecer | estar doente}
\end{entry}

\begin{entry}{病人}{bing4 ren2}{10,2}[HSK 1][Radicais ⽧、⼈]
  \definition{s.}{doente | paciente}
\end{entry}

\begin{entry}{拨转}{bo1zhuan3}{8,8}[Radicais ⼿、⾞]
  \definition{v.}{transferir (fundos, etc.) | virar | dar a volta}
\end{entry}

\begin{entry}{波}{bo1}{8}[Radical ⽔]
  \definition*{s.}{Polônia, abreviação de 波兰}
  \definition{s.}{onda | ondulação | tempestade | surto}
  \seeref{波兰}{bo1lan2}
\end{entry}

\begin{entry}{波兰}{bo1lan2}{8,5}[Radicais ⽔、⼋]
  \definition*{s.}{Polônia}
\end{entry}

\begin{entry}{波音}{bo1yin1}{8,9}[Radicais ⽔、⾳]
  \definition*{s.}{Boeing (empresa aeroespacial)}
  \definition{s.}{mordente (música)}
\end{entry}

\begin{entry}{玻璃}{bo1li5}{9,14}[Radicais ⽟、⽟]
  \definition[张,塊]{s.}{vidro | (gíria) homossexual masculino}
\end{entry}

\begin{entry}{般}{bo1}{10}[Radical ⾈]
  \definition{s.}{utilizado em 般若 \dpy{bo1re3}}
  \seeref{般若}{bo1re3}
\end{entry}

\begin{entry}{般若}{bo1re3}{10,8}[Radicais ⾈、⾋]
  \definition*{s.}{Prajna (sânscrito), \emph{insight} sobre a verdadeira natureza da realidade | (Budismo) sabedoria}
\end{entry}

\begin{entry}{啵}{bo1}{11}[Radical ⼝]
  \definition{s.}{(onomatopéia) borbulhar}
  \seeref{啵}{bo5}
\end{entry}

\begin{entry}{菠菜}{bo1cai4}{11,11}[Radicais ⾋、⾋]
  \definition[棵]{s.}{espinafre}
\end{entry}

\begin{entry}{播出}{bo1 chu1}{15,5}[HSK 3][Radicais ⼿、⼐]
  \definition{v.}{transmitir | estar no ar}
\end{entry}

\begin{entry}{播放}{bo1fang4}{15,8}[HSK 3][Radicais ⼿、⽅]
  \definition{v.}{ir ao ar | transmitir por rádio | mostrar | transmitir (um programa de TV)}
\end{entry}

\begin{entry}{播音}{bo1yin1}{15,9}[Radicais ⼿、⾳]
  \definition{s.}{transmissão}
  \definition{v.+compl.}{transmitir}
\end{entry}

\begin{entry}{脖子}{bo2zi5}{11,3}[Radicais ⾁、⼦]
  \definition[个]{s.}{pescoço}
\end{entry}

\begin{entry}{博文}{bo2wen2}{12,4}[Radicais ⼗、⽂]
  \definition{s.}{artigo em um blog}
  \definition{v.}{escrever um artigo em um blog}
\end{entry}

\begin{entry}{博物馆}{bo2wu4guan3}{12,8,11}[Radicais ⼗、⽜、⾷]
  \definition{s.}{museu}
\end{entry}

\begin{entry}{博主}{bo2zhu3}{12,5}[Radicais ⼗、⼂]
  \definition{s.}{blogueiro}
\end{entry}

\begin{entry}{薄}{bo2}{16}[Radical ⾋]
  \definition{adj.}{ligeiro; escasso; pequeno | mesquinho; pouco generoso; cruel | frívolo; fútil; leviano}
  \seeref{薄}{bao2}
\end{entry}

\begin{entry}{啵}{bo5}{11}[Radical ⼝]
  \definition{part.}{partícula gramaticalmente equivalente a 吧}
  \seeref{啵}{bo1}
  \seealsoref{吧}{ba5}
\end{entry}

\begin{entry}{不}{bu2}[(antes de quarto tom)]{4}[HSK 1][Radical ⼀]
  \definition{adv.}{não}
  \definition{pref.}{prefixo negativo}
  \seeref{不}{bu4}
  \seeref{不}{bu5}
\end{entry}

\begin{entry}{不必}{bu2 bi4}{4,5}[HSK 3][Radicais ⼀、⼼]
  \definition{adv.}{não precisa | não tem que}
\end{entry}

\begin{entry}{不错}{bu2 cuo4}{4,13}[HSK 2][Radicais ⼀、⾦]
  \definition{adj.}{correto | não (é) mau | bastante bom | certo}
\end{entry}

\begin{entry}{不大}{bu2 da4}{4,3}[HSK 1][Radicais ⼀、⼤]
  \definition{adv.}{não muito | não frequentemente | raramente |dificilmente | escassamente}
\end{entry}

\begin{entry}{不大离}{bu2da4li2}{4,3,10}[Radicais ⼀、⼤、⼇]
  \definition{adj.}{bem perto | quase certo | nada mal}
\end{entry}

\begin{entry}{不但}{bu2 dan4}{4,7}[HSK 2][Radicais ⼀、⼈]
  \definition{conj.}{não somente}
\end{entry}

\begin{entry}{不但……而且……}{bu2 dan4 er2qie3}{4,7,6,5}[HSK 2][Radicais ⼀、⼈、⽽、⼀]
  \definition{conj.}{não só\dots mas também\dots}
\end{entry}

\begin{entry}{不到}{bu2dao4}{4,8}[Radicais ⼀、⼑]
  \definition{adj.}{insuficiente}
  \definition{adv.}{menos que}
  \definition{v.}{não chegar}
\end{entry}

\begin{entry}{不断}{bu2duan4}{4,11}[HSK 3][Radicais ⼀、⽄]
  \definition{adv.}{continuamente | sem fim}
\end{entry}

\begin{entry}{不对}{bu2 dui4}{4,5}[HSK 1][Radicais ⼀、⼨]
  \definition{adj.}{incorreto | errado | anormal | estranho | estar em desacordo com | ser difícil de conviver}
\end{entry}

\begin{entry}{不够}{bu2 gou4}{4,11}[HSK 2][Radicais ⼀、⼣]
  \definition{adv.}{insuficiente}
  \definition{v.}{não ser suficiente}
\end{entry}

\begin{entry}{不过}{bu2guo4}{4,6}[HSK 2][Radicais ⼀、⾡]
  \definition{conj.}{mas | contudo | no entanto}
\end{entry}

\begin{entry}{不客气}{bu2 ke4 qi5}{4,9,4}[HSK 1][Radicais ⼀、⼧、⽓]
  \definition{adj.}{indelicado | rude | brusco}
  \definition{expr.}{de nada | não há de que | não mencione isso}
\end{entry}

\begin{entry}{不论}{bu2 lun4}{4,6}[HSK 3][Radicais ⼀、⾔]
  \definition{conj.}{não importa (o que, quem, como, etc.) | se \dots ou \dots}
\end{entry}

\begin{entry}{不论……都……}{bu2lun4 dou1}{4,6,10}[Radicais ⼀、⾔、⾢]
  \definition{conj.}{não apenas\dots, (o que, quem, como, etc.), \dots}
\end{entry}

\begin{entry}{不论……也……}{bu2lun4 ye3}{4,6,3}[Radicais ⼀、⾔、⼄]
  \definition{conj.}{não apenas\dots, (o que, quem, como, etc.), \dots}
\end{entry}

\begin{entry}{不日}{bu2ri4}{4,4}[Radicais ⼀、⽇]
  \definition{adv.}{em alguns dias}
\end{entry}

\begin{entry}{不是话}{bu2shi4hua4}{4,9,8}[Radicais ⼀、⽇、⾔]
  \definition{expr.}{sem razão | demasiado irracionável}
  \seeref{不像话}{bu2xiang4hua4}
  \seeref{不成话}{bu4cheng2hua4}
\end{entry}

\begin{entry}{不太}{bu2 tai4}{4,4}[HSK 2][Radicais ⼀、⼤]
  \definition{adv.}{não bastante | não muito}
\end{entry}

\begin{entry}{不像话}{bu2xiang4hua4}{4,13,8}[Radicais ⼀、⼈、⾔]
  \definition{expr.}{sem razão | demasiado irracionável}
  \seeref{不是话}{bu2shi4hua4}
  \seeref{不成话}{bu4cheng2hua4}
\end{entry}

\begin{entry}{不要}{bu2 yao4}{4,9}[HSK 2][Radicais ⼀、⾑]
  \definition{adv.}{nada de (pedir a alguém para não fazer) | não}
\end{entry}

\begin{entry}{不要紧}{bu2yao4jin3}{4,9,10}[HSK 4][Radicais ⼀、⾑、⽷]
  \definition{adj.}{sem importância; sem seriedade; não problemático | não importa; não é um obstáculo | parece estar tudo bem, mas | à primeira vista, isso não parece atrapalhar}
\end{entry}

\begin{entry}{不用}{bu2 yong4}{4,5}[HSK 1][Radicais ⼀、⽤]
  \definition{v.}{não precisar}
  \seeref{甭}{beng2}
\end{entry}

\begin{entry}{不在乎}{bu2 zai4 hu1}{4,6,5}[HSK 4][Radicais ⼀、⼟、⼃]
  \definition{v.}{não se importar; não dar a mínima; não dar atenção}
\end{entry}

\begin{entry}{不注意}{bu2zhu4yi4}{4,8,13}[Radicais ⼀、⽔、⼼]
  \definition{adj.}{impensado | distraído}
  \definition{s.}{descuido | distração}
\end{entry}

\begin{entry}{补}{bu3}{7}[HSK 3][Radical ⾐]
  \definition*{s.}{sobrenome Bu}
  \definition{s.}{benefício | ajuda | uso}
  \definition{v.}{consertar | remendar | preencher | adicionar suplemento | suprir | compensar |nutrir}
\end{entry}

\begin{entry}{补充}{bu3chong1}{7,6}[HSK 3][Radicais ⾐、⼉]
  \definition{adj.}{adicional | suplementar}
  \definition[个]{s.}{aditivo | suplemento}
  \definition{v.}{reabastecer | suplementar | complementar}
\end{entry}

\begin{entry}{不}{bu4}{4}[HSK 1][Radical ⼀]
  \definition{adv.}{não}
  \definition{pref.}{prefixo negativo}
  \seeref{不}{bu2}
  \seeref{不}{bu5}
\end{entry}

\begin{entry}{不安}{bu4'an1}{4,6}[HSK 3][Radicais ⼀、⼧]
  \definition{adj.}{inquieto | instável | intranquilo | pesaroso}
\end{entry}

\begin{entry}{不成话}{bu4cheng2hua4}{4,6,8}[Radicais ⼀、⼽、⾔]
  \definition{expr.}{sem razão | demasiado irracionável}
  \seeref{不是话}{bu2shi4hua4}
  \seeref{不像话}{bu2xiang4hua4}
\end{entry}

\begin{entry}{不得不}{bu4de2bu4}{4,11,4}[HSK 3][Radicais ⼀、⼻、⼀]
  \definition{adv.}{tem que | não tem escolha a não ser}
\end{entry}

\begin{entry}{不公}{bu4gong1}{4,4}[Radicais ⼀、⼋]
  \definition{adj.}{injusto}
\end{entry}

\begin{entry}{不管}{bu4guan3}{4,14}[HSK 4][Radicais ⼀、⽵]
  \definition{conj.}{não importa (o que, como, etc.); independentemente de; indica que, embora as condições ou circunstâncias tenham mudado, o resultado permanece o mesmo}
  \seeref{不管……都……}{bu4guan3 dou1}
  \seeref{不管……也……}{bu4guan3 ye3}
\end{entry}

\begin{entry}{不管……都……}{bu4guan3 dou1}{4,14,10}[Radicais ⼀、⽵、⾢]
  \definition{conj.}{não apenas\dots, (o que, quem, como, etc.), \dots}
\end{entry}

\begin{entry}{不管……也……}{bu4guan3 ye3}{4,14,3}[Radicais ⼀、⽵、⼄]
  \definition{conj.}{não apenas\dots, (o que, quem, como, etc.), \dots}
\end{entry}

\begin{entry}{不光}{bu4 guang1}{4,6}[HSK 3][Radicais ⼀、⼉]
  \definition{adv.}{não é o único}
  \definition{conj.}{não somente}
\end{entry}

\begin{entry}{不好意思}{bu4 hao3 yi4 si5}{4,6,13,9}[HSK 2][Radicais ⼀、⼥、⼼、⼼]
  \definition{adj.}{pedir desculpas (por incomodar alguém) | sentir-se envergonhado | achar isso embaraçoso}
\end{entry}

\begin{entry}{不仅}{bu4jin3}{4,4}[HSK 3][Radicais ⼀、⼈]
  \definition{adv.}{não apenas (em número, quantidade ou extensão)}
  \definition{conj.}{não somente}
\end{entry}

\begin{entry}{不久}{bu4 jiu3}{4,3}[HSK 2][Radicais ⼀、⼃]
  \definition{adj.}{em breve | futuro próximo | logo depois | não muito depois | não muito tempo (antes ou depois de algo)}
\end{entry}

\begin{entry}{不可避免}{bu4ke3bi4mian3}{4,5,16,7}[Radicais ⼀、⼝、⾌、⼉]
  \definition{adj./adv.}{inevitável}
\end{entry}

\begin{entry}{不满}{bu4 man3}{4,13}[HSK 2][Radicais ⼀、⽔]
  \definition{adj.}{ressentido | insatisfeito | descontente}
  \definition{v.}{estar descontente com |ser menor que}
\end{entry}

\begin{entry}{不然}{bu4ran2}{4,12}[HSK 4][Radicais ⼀、⽕]
  \definition{adj.}{não é assim; não é o caso}
  \definition{conj.}{se não; caso contrário; indica outra consequência ou circunstância que teria ocorrido se não fosse}
\end{entry}

\begin{entry}{不如}{bu4ru2}{4,6}[HSK 2][Radicais ⼀、⼥]
  \definition{conj.}{em vez de | melhor que | seria melhor}
  \definition{v.}{ser inferior a | não ser igual a | não ser tão bom quanto | não poder fazer melhor que}
\end{entry}

\begin{entry}{不少}{bu4 shao3}{4,4}[HSK 2][Radicais ⼀、⼩]
  \definition{adj.}{muitos | bastante | não poucos}
\end{entry}

\begin{entry}{不是……而是}{bu4shi4 er2 shi4}{4,9,6,9}[Radicais ⼀、⽇、⽽、⽇]
  \definition{conj.}{não somente\dots mas também\dots, expressam um relacionamento mais profundo e avançado em significado, mas as orações antes e depois são consistentes em expressar significados negativos e afirmativos, entretanto, a primeira metade da frase expressa negação, e a segunda metade expressa afirmação, e o significado das orações anteriores e seguintes não pode ser de um nível mais alto}
\end{entry}

\begin{entry}{不同}{bu4 tong2}{4,6}[HSK 2][Radicais ⼀、⼝]
  \definition{adj.}{diferente | distinto}
\end{entry}

\begin{entry}{不行}{bu4 xing2}{4,6}[HSK 2][Radicais ⼀、⾏]
  \definition{adj.}{não funciona | não é bom}
  \definition{adv.}{profundamente | terrivelmente | extremamente}
  \definition{v.}{não fazer | não ser permitido | estar fora de questão | estar à beira da morte}
\end{entry}

\begin{entry}{不一定}{bu4 yi2 ding4}{4,1,8}[HSK 2][Radicais ⼀、⼀、⼧]
  \definition{adv.}{talvez | incerto | não tenho certeza | não necessariamente}
\end{entry}

\begin{entry}{不一会儿}{bu4 yi2 hui4r5}{4,1,6,2}[HSK 2][Radicais ⼀、⼀、⼈、⼉]
  \definition{expr.}{em um momento | em pouco tempo |em breve}
\end{entry}

\begin{entry}{不止}{bu4zhi3}{4,4}[Radicais ⼀、⽌]
  \definition{adv.}{incessantemente | sem fim | mais que | não limitado a}
\end{entry}

\begin{entry}{布}{bu4}{5}[HSK 3][Radical ⼱]
  \definition*{s.}{sobrenome Bu}
  \definition[块,幅,匹]{s.}{pano | tecido | uma moeda de cobre nos tempos antigos}
  \definition{v.}{anunciar | declarar | tornar conhecido | proclamar | publicar | espalhar | disseminar |organizar | implantar | dispor}
\end{entry}

\begin{entry}{布谷鸟}{bu4gu3niao3}{5,7,5}[Radicais ⼱、⾕、⿃]
  \definition{s.}{cuco (pássaro)}
  \seealsoref{杜鹃}{du4juan1}
  \seealsoref{杜鹃鸟}{du4juan1niao3}
  \seealsoref{杜宇}{du4yu3}
\end{entry}

\begin{entry}{布署}{bu4shu3}{5,13}[Radicais ⼱、⽹]
  \variantof{部署}
\end{entry}

\begin{entry}{布置}{bu4zhi4}{5,13}[HSK 4][Radicais ⼱、⽹]
  \definition{v.}{arrumar; organizar; decorar; colocar adequadamente objetos ou paisagismo, conforme necessário | designar; tomar providências para; dar instruções sobre; organizar trabalho, atividades, etc.}
\end{entry}

\begin{entry}{步}{bu4}{7}[HSK 3][Radical ⽌]
  \definition*{s.}{sobrenome Bu}
  \definition{clas.}{uma unidade antiga para medida de comprimento, equivalente a cinco chi}
  \definition{s.}{ritmo | passo | estágio | passo | condição | situação | estado}
  \definition{v.}{ir a pé | andar | pisar | contar passos}
\end{entry}

\begin{entry}{步行}{bu4 xing2}{7,6}[HSK 4][Radicais ⽌、⾏]
  \definition{v.}{caminhar; ir a pé; andar a pé (diferente de andar de carro, a cavalo, etc.)}
\end{entry}

\begin{entry}{部}{bu4}{10}[HSK 3][Radical ⾢]
  \definition{clas.}{para obras de literatura, filmes, máquinas etc.}
  \definition[根]{s.}{departamento | divisão | ministério | seção | parte | tropas}
\end{entry}

\begin{entry}{部队}{bu4dui4}{10,4}[Radicais ⾢、⾩]
  \definition[个]{s.}{exército | forças armadas | tropas | unidades}
\end{entry}

\begin{entry}{部分}{bu4fen5}{10,4}[HSK 2][Radicais ⾢、⼑]
  \definition[个]{s.}{parte | parte de | uma parte de | pedaço | secção}
\end{entry}

\begin{entry}{部门}{bu4men2}{10,3}[HSK 3][Radicais ⾢、⾨]
  \definition[个]{s.}{filial | departamento | divisão | seção}
\end{entry}

\begin{entry}{部属}{bu4shu3}{10,12}[Radicais ⾢、⼫]
  \definition{s.}{afiliado a um ministério | subordinado | tropas sob comando de alguém}
\end{entry}

\begin{entry}{部署}{bu4shu3}{10,13}[Radicais ⾢、⽹]
  \definition{s.}{implantação}
  \definition{v.}{implantar}
\end{entry}

\begin{entry}{部下}{bu4xia4}{10,3}[Radicais ⾢、⼀]
  \definition{s.}{subordinado | tropas sob comando de alguém}
\end{entry}

\begin{entry}{部长}{bu4 zhang3}{10,4}[HSK 3][Radicais ⾢、⾧]
  \definition[个,位,名]{s.}{ministro | chefe de departamento | chefe de seção}
\end{entry}

\begin{entry}{部族}{bu4zu2}{10,11}[Radicais ⾢、⽅]
  \definition{adj.}{tribal}
  \definition{s.}{tribo}
\end{entry}

\begin{entry}{不}{bu5}{4}[HSK 1][Radical ⼀]
  \definition{adv.}{não (em expressões ``v.+不+v.'')}
  \seeref{不}{bu2}
  \seeref{不}{bu4}
\end{entry}

%%%%% EOF %%%%%

