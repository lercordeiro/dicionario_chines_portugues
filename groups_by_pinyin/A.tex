%%%
%%% A
%%%

\section*{A}\addcontentsline{toc}{section}{A}

\begin{entry}{阿}{a1}{7}{⾩}
  \definition{pref.}{em dialetos do sul para formar termos carinhosos, antes de nomes de animais de estimação, sobrenomes monossilábicos ou números que denotam ordem de antiguidade em uma; anexado a 大, 二, 三,\dots\ para indicar classificação (e, às vezes, intimidade) | antes dos termos de parentesco; na frente de um sobrenome, de um nome próprio ou de um determinado título, com uma conotação de intimidade | em alguns contextos, pode soar infantil ou muito informal (por exemplo, chamar um colega de trabalho por ``阿 + Nome'' sem intimidade)}[阿妈___mamãe | 阿明 ___forma carinhosa de chamar alguém chamado Ming]
  \seeref{阿}{e1}
\end{entry}

\begin{entry}{阿哥}{a1ge1}{7,10}{⾩、⼝}
  \definition{s.}{irmão mais velho (afetivo)}[阿哥,帮我拿一下书包!___Irmão, ajude-me com minha mochila escolar!]
\end{entry}

\begin{entry}{阿姨}{a1yi2}{7,9}{⾩、⼥}[HSK 4]
  \definition[个,位]{s.}{tia; uma forma de tratamento para uma mulher da geração dos pais; dirigir-se a uma mulher que tem aproximadamente a mesma idade da sua mãe, geralmente não é parente | babá em uma família; professora em um jardim de infância | tia; irmã da mãe (mais comum no sul da China)}[阿姨,生日快乐!___Tia, feliz aniversário! | 阿姨,这个苹果多少钱一斤?___Tia/Senhora, quanto custa o quilo dessas maçãs? | 阿姨,我想喝水。___Tia/Babá, eu quero beber água.]
\end{entry}

\begin{entry}{呵}{a1}{8}{⼝}
  \variantof{啊}
  \seeref{呵}{he1}
\end{entry}

\begin{entry}{啊}{a1}{10}{⼝}[HSK 2]
  \definition{interj.}{Ah!; Oh!; expressar surpresa ou admiração}
  \seeref{啊}{a2}
  \seeref{啊}{a3}
  \seeref{啊}{a4}
  \seeref{啊}{a5}
\end{entry}

\begin{entry}{啊呀}{a1ya1}{10,7}{⼝、⼝}
  \definition{interj.}{Oh meu Deus! | interjeição de surpresa}
\end{entry}

\begin{entry}{啊哟}{a1yo5}{10,9}{⼝、⼝}
  \definition{interj.}{Meu Deus! | Oh! | Ai! | interjeição de surpresa ou dor}
\end{entry}

\begin{entry}{啊}{a2}{10}{⼝}[HSK 2]
  \definition{interj.}{Eh?; Ei?; Que?; Por que?; expressar questionamento, dúvida ou solicitar opinião}
  \seeref{啊}{a1}
  \seeref{啊}{a3}
  \seeref{啊}{a4}
  \seeref{啊}{a5}
\end{entry}

\begin{entry}{嗄}{a2}{13}{⼝}
  \variantof{啊}
  \seeref{嗄}{sha4}
\end{entry}

\begin{entry}{啊}{a3}{10}{⼝}[HSK 2]
  \definition{interj.}{Eh?; Meu!; E aí?; Que?; expressar surpresa e dúvida}
  \seeref{啊}{a1}
  \seeref{啊}{a2}
  \seeref{啊}{a4}
  \seeref{啊}{a5}
\end{entry}

\begin{entry}{啊}{a4}{10}{⼝}[HSK 2]
  \definition{interj.}{Bem!; Sim!; expressa concordância, pronúncia mais curta | Oh!; Ah!; indica que compreendeu, com pronúncia mais longa | Oh!; expressa surpresa ou admiração, com pronúncia mais longa | Desgraça!; expressar tristeza, pesar ou pesar}
  \seeref{啊}{a1}
  \seeref{啊}{a2}
  \seeref{啊}{a3}
  \seeref{啊}{a5}
\end{entry}

\begin{entry}{啊}{a5}{10}{⼝}[HSK 2,4]
  \definition{part.}{usado no final da frase para expressar admiração | usado no final da frase para expressar afirmação, justificativa, insistência, recomendação, etc. | usado no final da frase para indicar dúvida | usado para fazer uma pequena pausa na frase, chamando a atenção para o que vem a seguir | usado após os itens enumerados | usado após verbos repetitivos, indica um processo longo}
  \seeref{啊}{a1}
  \seeref{啊}{a2}
  \seeref{啊}{a3}
  \seeref{啊}{a4}
\end{entry}

\begin{entry}{矮}{ai3}{13}{⽮}[HSK 4]
  \definition{adj.}{baixo; baixa estatura; pequeno em altura | baixo (refere-se a grau, classificação, nível, etc.)}[他比我矮。___Ele é mais baixo que eu. | 这栋楼很矮,只有三层。___Esse prédio é baixo, tem só três andares. | 她虽然矮,但是跑得很快!___Ela pode ser baixinha, mas corre muito rápido!]
\end{entry}

\begin{entry}{矮凳}{ai3deng4}{13,14}{⽮、⼏}
  \definition{s.}{banquinho baixo | banqueta}[这个矮凳是木制的,很结实。___Este banquinho é de madeira e bem resistente.]
\end{entry}

\begin{entry}{矮林}{ai3lin2}{13,8}{⽮、⽊}
  \definition{s.}{mata rasteira | bosque baixo}[这片矮林里有很多野兔和鸟类。___Neste bosque baixo há muitos coelhos selvagens e pássaros. | 山坡上长满了矮林,远看像绿色的地毯。___A encosta está coberta de mata rasteira, que de longe parece um tapete verde.]
\end{entry}

\begin{entry}{矮胖}{ai3pang4}{13,9}{⽮、⾁}
  \definition{adj.}{atarracado; gorducho; rechonchudo; roliço; baixo e robusto | chamar alguém diretamente de 矮胖 pode ser ofensivo}[我家猫矮胖矮胖的,像个毛球。___Meu gato é baixinho e gordinho, parece uma bolinha de pelo.]
\end{entry}

\begin{entry}{矮人}{ai3ren2}{13,2}{⽮、⼈}
  \definition{s.}{anão; pessoa de baixa estatura (indivíduo) | homúnculo; figuras criadas artificialmente pelos alquimistas em frascos de destilação | nanismo}[他虽然是矮人,但很有力气。___Embora ele seja baixo, é muito forte. | 北欧神话中的矮人是技艺高超的工匠。___Na mitologia nórdica, os anões são artesãos habilidosos. | 他因为身高被嘲笑为‘矮人’,这让他很伤心。___Ele foi zombado por ser chamado de ‘anão’ devido à sua altura, o que o magoou.]
\end{entry}

\begin{entry}{矮树}{ai3shu4}{13,9}{⽮、⽊}
  \definition{s.}{arbusto | árvore pequena, baixa}[矮树比高树更容易修剪。___Árvores baixas são mais fáceis de podar do que árvores altas. | 我们种了些矮树作为花园的边界。___Plantamos alguns arbustos como cerca natural do jardim.]
\end{entry}

\begin{entry}{矮小}{ai3 xiao3}{13,3}{⽮、⼩}[HSK 4]
  \definition{adj.}{subdimensionado; curto e pequeno; baixo e pequeno | quando usado para pessoas, pode soar depreciativo se não for em contexto neutro ou afetuoso}[这位矮小的老人是村里的智者。___Este idoso baixinho é o sábio da vila. | 这种矮小的灌木适合盆栽。___Este tipo de arbusto pequeno é ideal para vasos. | 山脚下有一片矮小的房屋,显得格外宁静。___Ao pé da montanha, havia casas baixas que transmitiam uma tranquilidade única.]
\end{entry}

\begin{entry}{矮星}{ai3xing1}{13,9}{⽮、⽇}
  \definition{s.}{estrela anã}[白矮星是恒星演化的最终阶段之一。___Anãs brancas são um dos estágios finais da evolução estelar.]
\end{entry}

\begin{entry}{矮子}{ai3zi5}{13,3}{⽮、⼦}
  \definition{s.}{pessoa baixa; anão; baixinho}[白雪公主和七个小矮子住在森林里。___Branca de Neve e os sete anões vivem na floresta. | 用`矮子'称呼他人是不礼貌的。___Chamar alguém de `baixinho' é falta de educação.]
\end{entry}

\begin{entry}{爱}{ai4}{10}{⽖}[HSK 1]
  \definition*{s.}{sobrenome Ai}
  \definition[个]{s.}{amor; afeição profunda; preocupação profunda; especialmente amor entre pessoas}[爱是理解和包容。___O amor é compreensão e tolerância.]
  \definition{v.}{amar; ter sentimentos profundos por pessoas ou coisas | gostar; gostar de; estar interessado em |  cuidar; valorizar; ter em alta estima; cuidar bem de | estar apto a; ter o hábito de}[他们深深爱着对方。___Eles se amam profundamente. | 我爱我的家人。___Eu amo minha família. | 我爱旅行。___Eu adoro viajar.]
\end{entry}

\begin{entry}{爱爱}{ai4'ai5}{10,10}{⽖、⽖}
  \definition{v.}{(coloquial) fazer amor ou relações íntimas | pode ser usado como um apelido entre casais, transmitindo ternura | pode soar vulgar se usado em contextos inadequados}[他们俩刚结婚,天天都想爱爱。___Eles acabaram de se casar e querem fazer amor todo dia. | 爱爱,你今天好漂亮!___Amor, você está linda hoje!]
\end{entry}

\begin{entry}{爱抚}{ai4fu3}{10,7}{⽖、⼿}
  \definition{v.}{acariciar; afagar; cuidar (com ternura)}[他轻轻爱抚她的头发。___Ele afagou suavemente o cabelo dela. | 母亲爱抚婴儿的脸颊。___A mãe acaricia a bochecha do bebê. | 她爱抚着小猫的耳朵。___Ela acariciou as orelhas do gatinho.]
\end{entry}

\begin{entry}{爱国}{ai4 guo2}{10,8}{⽖、⼞}[HSK 4]
  \definition{adj.}{patriótico; patriotismo}[爱国是每个公民的责任。___O patriotismo é o dever de todo cidadão. | 这部电影讲述了英雄的爱国故事。___Este filme conta a história patriótica de um herói.]
  \definition{v.}{ser patriota; amar o seu país}
\end{entry}

\begin{entry}{爱好}{ai4 hao4}{10,6}{⽖、⼥}[HSK 1]
  \definition[个,种]{s.}{passatempo; interesse; \emph{hobby}; sentimentos de interesse especial ou afeição por algo | 爱好 é mais usado para atividades regulares (esportes, música), enquanto 喜欢 é para preferências gerais}[他的爱好是收集邮票。___Seu hobby era colecionar selos.  | 我的爱好是读书和旅行。___Meus hobbies são ler e viajar.]
  \definition{v.}{estar interessado em; ter prazer em; ter um forte interesse em algo; ter sentimentos profundos por alguém ou algo}
  \seealsoref{喜欢}{xi3huan5}
\end{entry}

\begin{entry}{爱好者}{ai4 hao4 zhe3}{10,6,8}{⽖、⼥、⽼}
  \definition{s.}{hobbista; amador; entusiasta; fã; amante (de arte, esportes, etc.)}[他是一位摄影爱好者。___Ele é um entusiasta de fotografia. | 她是位潜水爱好者,经常去东南亚潜水。___Ela é uma mergulhadora amadora e frequentemente mergulha no Sudeste Asiático.  | 我们为书法爱好者创建了一个微信群。___Criamos um grupo no WeChat para amantes de caligrafia.]
\end{entry}

\begin{entry}{爱护}{ai4hu4}{10,7}{⽖、⼿}[HSK 4]
  \definition{v.}{acalentar; valorizar; salvaguardar; cuidar bem de}[全社会都应爱护老年人。___Toda a sociedade deve tratar os idosos com cuidado e respeito. | 请爱护公园里的小动物。___Por favor, tratem os animais do parque com cuidado.]
\end{entry}

\begin{entry}{爱情}{ai4qing2}{10,11}{⽖、⼼}[HSK 2]
  \definition{s.}{amor (entre pessoas); afeição}[爱情是盲目的。___O amor é cego. | 爱情如同玫瑰,美丽却带刺。___O amor é como uma rosa, bela mas com espinhos.  | 这首歌讲述了破碎的爱情故事。___Esta música conta uma história de amor fracassado.]
\end{entry}

\begin{entry}{爱人}{ai4 ren5}{10,2}{⽖、⼈}[HSK 2]
  \definition[个]{s.}{amante; \emph{dollbaby}; namorado(a) | marido ou esposa; mais usado em ocasiões formais}[这是我的爱人。___Este é o meu/minha esposo/companheiro. | 她是我一生的爱人。___Ela é o amor da minha vida. | 请携带爱人出席晚宴。___Por favor, traga seu cônjuge para o jantar.]
\end{entry}

\begin{entry}{爱上}{ai4shang4}{10,3}{⽖、⼀}
  \definition{v.}{perder o coração por; apaixonar-se por}[他在旅行时爱上了一位法国女孩。___Ele se apaixonou por uma garota francesa durante a viagem.  | 来到杭州后,我爱上了龙井茶。___Depois de chegar em Hangzhou, me apaixonei pelo chá Longjing. | 我从来没想过自己会爱上健身。___Eu nunca imaginei que iria me apaixonar por academia.]
\end{entry}

\begin{entry}{爱心}{ai4xin1}{10,4}{⽖、⼼}[HSK 3]
  \definition[片]{s.}{amor; carinho; compaixão; um sentimento de preocupação e carinho por outras pessoas ou animais}
\end{entry}

\begin{entry}{碍}{ai4}{13}{⽯}
  \definition{v.}{atrapalhar; dificultar; obstruir; estar no caminho de | levar em consideração}
\end{entry}

\begin{entry}{碍事}{ai4shi4}{13,8}{⽯、⼅}
  \definition{s.}{(usualmente em frases negativas) sem consequência, não importa}
  \definition{v.+compl.}{estar no caminho | ser um obstáculo}
\end{entry}

\begin{entry}{厂}{an1}{2}{⼚}
  \definition{s.}{usado principalmente em nomes pessoais}[他名中有个厂字。___O nome dele contém a palavra `An'.]
  \seeref{厂}{chang3}
  \seeref{厂}{han3}
\end{entry}

\begin{entry}{广}{an1}{3}{⼴}
  \definition{s.}{mais comum em nomes de pessoas; o mesmo que 庵}[广安是我的朋友。___An'an é meu amigo.]
  \seeref{广}{guang3}
  \seeref{广}{yan3}
  \seealsoref{庵}{an1}
\end{entry}

\begin{entry}{安}{an1}{6}{⼧}[HSK 4]
  \definition*{s.}{sobrenome An}
  \definition{adj.}{pacífico; quieto; tranquilo; calmo | seguro; protegido (oposto a 危) | com boa saúde | em paz; bem}
  \definition{adv.}{pacificamente; silenciosamente | com segurança; em segurança | (em perguntas retóricas) como?}
  \definition{pron.}{usado como pronome interrogativo, como em 哪里,怎么; 谁,何,如何}
  \definition{s.}{segurança; proteção; paz | ampère; (eletricidade) abreviação de ampère, 安培}
  \definition{v.}{tranquilizar (a mente de alguém); acalmar | contentar-se; ficar satisfeito | colocar em uma posição adequada; encontrar um lugar para | instalar; consertar; encaixar; configurar | trazer (uma acusação contra alguém); dar (a alguém um apelido); reivindicar (crédito por algo) | abrigar (uma intenção) | acalmar; estabilizar | sentir-se satisfeito e à vontade}
  \seealsoref{安培}{an1pei2}
  \seealsoref{何}{he2}
  \seealsoref{哪里}{na3 li3}
  \seealsoref{如何}{ru2he2}
  \seealsoref{谁}{shei2}
  \seealsoref{危}{wei1}
  \seealsoref{怎么}{zen3me5}
\end{entry}

\begin{entry}{安家}{an1jia1}{6,10}{⼧、⼧}
  \definition{v.+compl.}{montar uma casa | estabelecer-se}
\end{entry}

\begin{entry}{安静}{an1jing4}{6,14}{⼧、⾭}[HSK 2]
  \definition{adj.}{silencioso; tranquilo; sem som; sem barulho e sem algazarra}
\end{entry}

\begin{entry}{安排}{an1pai2}{6,11}{⼧、⼿}[HSK 3]
  \definition{s.}{plano; programação; organização; tabela do plano de atividades ou horários}
  \definition{v.}{organizar (assuntos) de acordo com a sequência ou regras; tratar as coisas de acordo com uma determinada ordem ou regras | atribuir tarefas a alguém; colocar as pessoas nos cargos de trabalho determinados, conforme planejado}
\end{entry}

\begin{entry}{安培}{an1pei2}{6,11}{⼧、⼟}
  \definition{clas.}{A; (empréstimo linguístico) ampere; (física) unidade de corrente elétrica}
\end{entry}

\begin{entry}{安全}{an1quan2}{6,6}{⼧、⼊}[HSK 2]
  \definition{adj.}{seguro; protegido; sem perigo; sem ameaças; sem acidentes}
  \definition{s.}{segurança; proteção; refere-se a um estado ou conceito, geralmente indicando ausência de ameaças ou perigo}
\end{entry}

\begin{entry}{安神}{an1shen2}{6,9}{⼧、⽰}
  \definition{v.+compl.}{acalmar os nervos | aliviar a inquietação pela tranquilização da mente e do corpo}
\end{entry}

\begin{entry}{安慰}{an1wei4}{6,15}{⼧、⼼}[HSK 5]
  \definition{adj.}{confortar; tranquilizar; consolar; apaziguar;}
  \definition[个]{s.}{conforto; consolo; comportamento que alivia a dor de alguém e o acalma com palavras ou gestos}
  \definition{v.}{confortar; consolar; acalmar e confortar; deixar a mente tranquila}
\end{entry}

\begin{entry}{安置}{an1zhi4}{6,13}{⼧、⽹}[HSK 4]
  \definition{v.}{providenciar; encontrar um lugar para; ajudar a estabelecer-se; colocar pessoas ou coisas em uma determinada posição ou organizá-las adequadamente}
\end{entry}

\begin{entry}{安装}{an1zhuang1}{6,12}{⼧、⾐}[HSK 3]
  \definition{v.}{instalar; consertar; configurar; fixar máquinas ou equipamentos (geralmente conjuntos) em um determinado local, de acordo com métodos e especificações específicos}
\end{entry}

\begin{entry}{庵}{an1}{11}{⼴}
  \definition*{s.}{sobrenome An}
  \definition[个,座]{s.}{cabana | convento de freiras; templos budistas, principalmente onde vivem as freiras}
\end{entry}

\begin{entry}{岸}{an4}{8}{⼭}[HSK 5]
  \definition{adj.}{arrogante; orgulhoso; grandioso (de maneira sombria ou condescendente)}
  \definition[条,道,段,面]{s.}{margem; costa; litoral; terreno à beira da água}
\end{entry}

\begin{entry}{岸上}{an4 shang4}{8,3}{⼭、⼀}[HSK 5]
  \definition{s.}{em terra; costa; margem | na margem do rio; na beira do rio}
\end{entry}

\begin{entry}{按}{an4}{9}{⼿}[HSK 3]
  \definition{prep.}{de acordo com; à luz de; com base em; em conformidade com}
  \definition{v.}{pressionar; empurrar para baixo; pressionar ou apertar com a mão ou os dedos | pôr de parte; deixar de lado; deixar para mais tarde | restringir; controlar; inibir; parar | verificar; consultar | comentar ou anotar (por um editor ou autor)}
\end{entry}

\begin{entry}{按摩}{an4mo2}{9,15}{⼿、⼿}[HSK 5]
  \definition{s.}{massagem; empurrar, pressionar, beliscar e amassar o corpo de uma pessoa com as mãos para promover a circulação sanguínea, aumentar a resistência da pele e regular a função dos nervos}
\end{entry}

\begin{entry}{按时}{an4shi2}{9,7}{⼿、⽇}[HSK 4]
  \definition{adv.}{na hora; no horário; pontualmente; de acordo com o tempo estipulado}
\end{entry}

\begin{entry}{按照}{an4zhao4}{9,13}{⼿、⽕}[HSK 3]
  \definition{prep.}{de acordo com; em conformidade com; à luz de; com base em; apresentar os fundamentos ou critérios de julgamento para fazer algo}
\end{entry}

\begin{entry}{暗}{an4}{13}{⽇}[HSK 4]
  \definition{adj.}{escuro; opaco; sem graça; pouca luz | escondido; secreto; não revelado | pouco claro; nebuloso; vago; confuso | subterrâneo}
  \definition{adv.}{secretamente | no escuro}
\end{entry}

\begin{entry}{暗恋}{an4lian4}{13,10}{⽇、⼼}
  \definition{s.}{amor secreto}
  \definition{v.}{estar secretamente apaixonado por}
\end{entry}

\begin{entry}{暗示}{an4shi4}{13,5}{⽇、⽰}[HSK 4]
  \definition[个]{s.}{sugestão; insinuação; intimação; (psicologia) refere-se ao uso de palavras, gestos, expressões, etc. para fazer as pessoas aceitarem involuntariamente uma determinada opinião ou fazerem algo}
  \definition{v.}{dar uma dica; sugerir secretamente; indicar algo a alguém usando outras palavras, expressões faciais ou gestos sem dizer em voz alta}
\end{entry}

\begin{entry}{暗香}{an4xiang1}{13,9}{⽇、⾹}
  \definition{s.}{fragrância sutil}
\end{entry}

\begin{entry}{奥}{ao4}{12}{⼤}
  \definition*{s.}{sobrenome Ao}
  \definition*{s.}{Oersted (a unidade eletromagnética de intensidade do campo magnético), abreviação de 奥斯特}
  \definition{adj.}{profundo e difícil de entender; abstruso | significado profundo, não é fácil de entender}
  \definition{s.}{canto secreto da casa; antigamente, referia-se ao canto sudoeste de uma casa e também, de modo geral, à profundidade de uma casa}
  \seealsoref{奥斯特}{ao4 si1 te4}
\end{entry}

\begin{entry}{奥林匹克运动会}{ao4lin2pi3ke4 yun4dong4hui4}{12,8,4,7,7,6,6}{⼤、⽊、⼖、⼗、⾡、⼒、⼈}
  \definition*{s.}{Jogos Olímpicos, Olimpíadas}
\end{entry}

\begin{entry}{奥斯特}{ao4 si1 te4}{12,12,10}{⼤、⽄、⽜}
  \definition{s.}{Oersted}
\end{entry}

\begin{entry}{奥特曼}{ao4te4man4}{12,10,11}{⼤、⽜、⽈}
  \definition*{s.}{\emph{Ultraman},  super-herói de ficção científica japonesa}
\end{entry}

\begin{entry}{奥运}{ao4yun4}{12,7}{⼤、⾡}
  \definition*{s.}{Jogos Olímpicos, Olimpíadas, abreviação de 奥林匹克运动会}
  \seealsoref{奥林匹克运动会}{ao4lin2pi3ke4 yun4dong4hui4}
\end{entry}

\begin{entry}{奥运会}{ao4yun4hui4}{12,7,6}{⼤、⾡、⼈}
  \definition*{s.}{Jogos Olímpicos, Olimpíadas, abreviação de 奥林匹克运动会}
  \seealsoref{奥林匹克运动会}{ao4lin2pi3ke4 yun4dong4hui4}
\end{entry}

\begin{entry}{澳}{ao4}{15}{⽔}
  \definition*{s.}{Austrália, abreviação de 澳大利亚}
  \definition{s.}{sobrenome Ao}
  \definition{s.}{baía; uma entrada do mar; um lugar curvo na costa onde os barcos podem ser atracados, frequentemente usado em nomes de lugares}
  \seealsoref{澳大利亚}{ao4da4li4ya4}
\end{entry}

\begin{entry}{澳大利亚}{ao4da4li4ya4}{15,3,7,6}{⽔、⼤、⼑、⼆}
  \definition*{s.}{Austrália}
\end{entry}

%%%%% EOF %%%%%

