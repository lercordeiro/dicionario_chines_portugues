%%%
%%% P
%%%

\section*{P}\addcontentsline{toc}{section}{P}

\begin{entry}{扒}{pa2}{5}{⼿}
  \definition{v.}{reunir; juntar; reunir ou espalhar coisas com as mãos ou com um ancinho | roubar; furtar | arranhar; coçar com as mãos | cozinhar; refogar; cozinhar os alimentos em fogo baixo}
  \seeref{扒}{ba1}
\end{entry}

\begin{entry}{扒犁}{pa2li2}{5,11}{⼿、⽜}
  \definition{s.}{trenó}
  \seealsoref{爬犁}{pa2li2}
\end{entry}

\begin{entry}{爬}{pa2}{8}{⽖}[HSK 2]
  \definition{v.}{rastejar; arrastar-se; engatinhar | escalar; trepar; subir com dificuldade | sentar-se; levantar-se; levantar-se da posição deitada ou sentada}
\end{entry}

\begin{entry}{爬杆}{pa2gan1}{8,7}{⽖、⽊}
  \definition{s.}{escalada em poste}
  \definition{v.}{escalar um poste}
\end{entry}

\begin{entry}{爬竿}{pa2gan1}{8,9}{⽖、⽵}
  \definition{s.}{poste de escalada | escalada em poste (como ginástica ou ato de circo)}
\end{entry}

\begin{entry}{爬犁}{pa2li2}{8,11}{⽖、⽜}
  \definition{s.}{trenó}
  \seealsoref{扒犁}{pa2li2}
\end{entry}

\begin{entry}{爬墙}{pa2qiang2}{8,14}{⽖、⼟}
  \definition{v.}{escalar uma parede}
\end{entry}

\begin{entry}{爬山}{pa2shan1}{8,3}{⽖、⼭}[HSK 2]
  \definition{v.+compl.}{escalar uma montanha;}
\end{entry}

\begin{entry}{爬上}{pa2shang4}{8,3}{⽖、⼀}
  \definition{v.}{escalar}
\end{entry}

\begin{entry}{爬升}{pa2sheng1}{8,4}{⽖、⼗}
  \definition{v.}{ascender | ganhar promoção | subir (números de vendas, etc.) | aumentar}
\end{entry}

\begin{entry}{爬梳}{pa2shu1}{8,11}{⽖、⽊}
  \definition{v.}{vasculhar (documentos históricos, etc.) | desvendar}
\end{entry}

\begin{entry}{爬行}{pa2xing2}{8,6}{⽖、⾏}
  \definition{v.}{rastejar | arrastar | engatinhar}
\end{entry}

\begin{entry}{怕}{pa4}{8}{⼼}[HSK 2]
  \definition{adv.}{(expressando suposição, julgamento, estimativa, etc.) talvez; suponho; receio (que)}
  \definition{adv.}{por medo; talvez; suponho}
  \definition{v.}{temer; ter medo; recear; sentir medo, ficar nervoso | estar preocupado com; estar preocupado por (ou sobre); ter medo de que algo possa acontecer | ser afetado por; não conseguir suportar; não aguentar mais}
\end{entry}

\begin{entry}{拍}{pai1}{8}{⼿}[HSK 3]
  \definition[个,副,对]{s.}{bastão; raquete | batida; tempo; (música) uma unidade para medir a duração de uma nota musical}
  \definition{v.}{tirar (uma foto); usar uma câmera para capturar imagens de pessoas e objetos em filme | dar um tapinha; bater suavemente com as mãos ou ferramentas | bater asas | bater (ondas do mar) | enviar (um telegrama, etc.) | bajular}
\end{entry}

\begin{entry}{拍马}{pai1ma3}{8,3}{⼿、⾺}
  \definition{v.}{instigar um cavalo dando tapinhas em seu traseiro | lisonjear | bajular}
  \seealsoref{拍马屁}{pai1ma3pi4}
\end{entry}

\begin{entry}{拍马屁}{pai1ma3pi4}{8,3,7}{⼿、⾺、⼫}
  \definition{s.}{puxa-saco | bajulador}
  \definition{v.}{puxar o saco | bajular}
  \seealsoref{拍马}{pai1ma3}
\end{entry}

\begin{entry}{拍摄}{pai1 she4}{8,13}{⼿、⼿}[HSK 5]
  \definition{s.}{fotografar; tirar (uma foto); usar uma câmera fotográfica para capturar imagens de pessoas e objetos}
\end{entry}

\begin{entry}{拍照}{pai1 zhao4}{8,13}{⼿、⽕}[HSK 4]
  \definition{v.+compl.}{fotografar; tirar uma foto}
\end{entry}

\begin{entry}{排}{pai2}{11}{⼿}[HSK 2,3]
  \definition{clas.}{usado para linhas, filas; coisas usadas para formar filas}
  \definition{s.}{linha; fileira; fileiras horizontais | pelotão; unidade militar, abaixo do nível de companhia, acima do nível de pelotão | jangada; balsa; um meio de transporte aquático feito de bambu e madeira unidos lado a lado; também se refere a bambu e madeira amarrados em fileiras para facilitar o transporte aquático | torta; bolo de carne; bolinho assado; comida cozida no vapor}
  \definition{v.}{organizar; alinhar; colocar em ordem; posicionar ou organizar em uma determinada ordem; ordenar | ensaiar | ejetar; excluir; dispensar; remover; eliminar | empurrar o obstáculo para fora do caminho}
\end{entry}

\begin{entry}{排除}{pai2chu2}{11,9}{⼿、⾩}[HSK 5]
  \definition{v.}{remover; superar; excluir; eliminar; livrar-se de}
\end{entry}

\begin{entry}{排队}{pai2dui4}{11,4}{⼿、⾩}[HSK 2]
  \definition{v.+compl.}{formar uma fila; alinhar-se; enfileirar-se; organizar em sequência | listar; classificar}
\end{entry}

\begin{entry}{排列}{pai2lie4}{11,6}{⼿、⼑}[HSK 4]
  \definition{v.}{classificar; colocar; variar; organizar; pôr em ordem}
\end{entry}

\begin{entry}{排名}{pai2 ming2}{11,6}{⼿、⼝}[HSK 3]
  \definition{s.}{classificação; resultado; organizado de acordo com determinados critérios}
\end{entry}

\begin{entry}{排球}{pai2 qiu2}{11,11}{⼿、⽟}[HSK 2]
  \definition[场,只,个]{s.}{voleibol; bola de voleibol}
\end{entry}

\begin{entry}{排水}{pai2shui3}{11,4}{⼿、⽔}
  \definition{v.}{drenar}
\end{entry}

\begin{entry}{牌}{pai2}{12}{⽚}[HSK 4]
  \definition[块]{s.}{placa; tabuleta; quadro; placar | marca; marca registrada; marca comercial | cartas, dominó, etc. | a tonalidade de uma música}
\end{entry}

\begin{entry}{牌子}{pai2 zi5}{12,3}{⽚、⼦}[HSK 3]
  \definition[个,种,块]{s.}{sinal; placa; placas feitas de madeira ou outros materiais, geralmente com texto nelas | marca; marca registrada; um nome especial dado por uma empresa ao seu próprio produto}
\end{entry}

\begin{entry}{派}{pai4}{9}{⽔}[HSK 3]
  \definition{adj.}{elegante; bonito; imponente}
  \definition{clas.}{usado para grupos, escolas de pensamento ou arte, etc. | usado para um discursos, situações, cenas, etc.}
  \definition[个,块,种]{s.}{panelinha; facção; pessoas com ideias, visões e estilos semelhantes | torta; um alimento recheado comumente consumido pelos ocidentais, geralmente doce | maneira e ar; estilo ou comportamento | afluente; braço de rio}
  \definition{v.}{enviar; despachar; arranjar ou ordenar que uma pessoa faça algo; providenciar transporte | alocar; repartir; distribuir}
\end{entry}

\begin{entry}{攀}{pan1}{19}{⼿}
  \definition{v.}{escalar; escalar | buscar conexões em altos cargos | envolver; implicar | agarrar; agarrar-se; segurar-se a}
\end{entry}

\begin{entry}{攀爬}{pan1pa2}{19,8}{⼿、⽖}
  \definition{v.}{escalar}
\end{entry}

\begin{entry}{攀岩}{pan1yan2}{19,8}{⼿、⼭}
  \definition{s.}{alpinista}
  \definition{v.}{escalar uma montanha}
\end{entry}

\begin{entry}{爿}{pan2}{4}{⽙}[Kangxi 90]
  \definition{clas.}{para faixas de terra ou bambu, lojas, fábricas etc.}
\end{entry}

\begin{entry}{胖}{pan2}{9}{⾁}
  \definition{adj.}{saudável}
  \seeref{胖}{pang4}
\end{entry}

\begin{entry}{般}{pan2}{10}{⾈}
  \definition{adj.}{feliz; bem-aventurado}
\end{entry}

\begin{entry}{般乐}{pan2le4}{10,5}{⾈、⼃}
  \definition{v.}{jogar | divertir-se}
\end{entry}

\begin{entry}{盘}{pan2}{11}{⽫}[HSK 4]
  \definition*{s.}{Sobrenome Pan}
  \definition{clas.}{para pratos, pedras de moer, etc. | para jogos de xadrez e de bola | para as coisas que estão entrelaçadas, emaranhadas}
  \definition{s.}{bandeja; tabuleiro | recipiente plano e raso, como uma bandeja, prato, travessa etc.  | preço atual; cotação de mercado; refere-se ao preço básico pelo qual as commodities são negociadas}
  \definition{v.}{enrolar; torcer; enrolar (para cima); entrelaçar; cercar | construir (assentando tijolos, pedras, etc.) | checar; examinar; interrogar; verificar um por um ou repetidamente (quantidade, situação, etc.) | transferir a propriedade de; passar para outra pessoa | carregar; transportar}
\end{entry}

\begin{entry}{盘子}{pan2zi5}{11,3}{⽫、⼦}[HSK 4]
  \definition[个,叠,套,只]{s.}{prato; utensílio de fundo raso para guardar objetos, maior do que um pires, geralmente de formato redondo | situação de mercado; taxa de mercado; transação comercial}
\end{entry}

\begin{entry}{槃}{pan2}{14}{⽊}
  \variantof{盘}
\end{entry}

\begin{entry}{判}{pan4}{7}{⼑}[HSK 6]
  \definition{adv.}{obviamente há uma diferença}
  \definition{v.}{distinguir; discriminar; separar | julgar; decidir; avaliar | sentenciar; condenar}
\end{entry}

\begin{entry}{判断}{pan4duan4}{7,11}{⼑、⽄}[HSK 3]
  \definition[个,项]{s.}{julgamento; conclusões tiradas após reflexão e análise}
  \definition{v.}{julgar; decidir}
\end{entry}

\begin{entry}{乓}{pang1}{6}{⼃}
  \definition{interj.}{(onomatopéia) barulho repentino feito por tiros, uma porta batendo, coisas quebrando, etc.; estrondo; estouro; batida; colisão}
\end{entry}

\begin{entry}{旁}{pang2}{10}{⽅}[HSK 5]
  \definition{adj.}{outro | abundante; abrangente}
  \definition{s.}{lado | radical lateral de um caractere chinês}
\end{entry}

\begin{entry}{旁边}{pang2bian1}{10,5}{⽅、⾡}[HSK 1]
  \definition{s.}{junto a; próximo de; ao lado}
\end{entry}

\begin{entry}{胖}{pang4}{9}{⾁}[HSK 3]
  \definition{adj.}{gordo; robusto; rechonchudo; (corpo humano) com muita gordura ou carne (em oposição a 瘦)}
  \seeref{胖}{pan2}
  \seealsoref{瘦}{shou4}
\end{entry}

\begin{entry}{胖子}{pang4 zi5}{9,3}{⾁、⼦}[HSK 4]
  \definition{s.}{obeso; gordo; pessoa gorda}
\end{entry}

\begin{entry}{泡}{pao1}{8}{⽔}
  \definition{adj.}{esponjoso; oco e macio; não duro}
  \definition{clas.}{usado para fezes e urina}
  \definition[串,个]{s.}{algo fofo e macio | pequeno lago}
  \seeref{泡}{pao4}
\end{entry}

\begin{entry}{炮}{pao2}{9}{⽕}
  \definition{v.}{(medicina tradicional chinesa) preparar a medicina chinesa assando-a em uma panela de ferro quente até dourar e estalar}
  \seeref{炮}{bao1}
  \seeref{炮}{pao4}
\end{entry}

\begin{entry}{跑}{pao2}{12}{⾜}
  \definition{v.}{(de animais) bater com a pata (no chão); (de animais) escavar o solo com suas garras ou cascos}
  \seeref{跑}{pao3}
\end{entry}

\begin{entry}{跑}{pao3}{12}{⾜}[HSK 1]
  \definition{v.}{correr; pessoas ou animais que se movem rapidamente para a frente com as pernas e os pés | caminhar; passear | fugir; escapar | correr de um lado para outro; fazer rondas; correr atrás de algo | de um líquido ou gás) vazar; evaporar | (como complemento de um verbo) fora; longe | participar de uma corrida}
  \seeref{跑}{pao2}
\end{entry}

\begin{entry}{跑步}{pao3bu4}{12,7}{⾜、⽌}[HSK 3]
  \definition{v.+compl.}{correr; trotar}
\end{entry}

\begin{entry}{跑调}{pao3diao4}{12,10}{⾜、⾔}
  \definition{v.}{(coloquial) estar fora do tom ou desafinado (enquanto canta)}
\end{entry}

\begin{entry}{跑掉}{pao3diao4}{12,11}{⾜、⼿}
  \definition{v.}{fugir}
\end{entry}

\begin{entry}{跑肚}{pao3du4}{12,7}{⾜、⾁}
  \definition{v.}{(coloquial) ter diarréia}
\end{entry}

\begin{entry}{跑酷}{pao3ku4}{12,14}{⾜、⾣}
  \definition*{s.}{Eempréstimo linguístico: Parkour}
\end{entry}

\begin{entry}{跑马}{pao3ma3}{12,3}{⾜、⾺}
  \definition{s.}{corrida de cavalos}
  \definition{v.}{andar a cavalo em ritmo acelerado}
\end{entry}

\begin{entry}{跑题}{pao3ti2}{12,15}{⾜、⾴}
  \definition{v.}{divagar | fugir do assunto | tergiversar}
\end{entry}

\begin{entry}{跑腿}{pao3tui3}{12,13}{⾜、⾁}
  \definition{v.}{realizar tarefas}
\end{entry}

\begin{entry}{泡}{pao4}{8}{⽔}[HSK 6]
  \definition[串,个]{s.}{bolha | algo em forma de bolha}
  \definition{v.}{mergulhar; encharcar | despejar água fervente em (chá, sopa instantânea, etc.) | enrolar; demorar-se; ficar por aí | (coloquial) (de um homem) brincar no campo; brincar com uma mulher | perder tempo; matar o tempo deliberadamente}
  \seeref{泡}{pao1}
\end{entry}

\begin{entry}{炮}{pao4}{9}{⽕}[HSK 6]
  \definition{s.}{arma grande; canhão; peça de artilharia | fogo de artifício | buraco de explosão cheio de dinamite | canhão, uma das peças do xadrez chinês}
  \seeref{炮}{bao1}
  \seeref{炮}{pao2}
\end{entry}

\begin{entry}{胚}{pei1}{9}{⾁}
  \definition{s.}{embrião}
\end{entry}

\begin{entry}{陪}{pei2}{10}{⾩}[HSK 5]
  \definition{v.}{servir; acompanhar; cuidar; fazer companhia a alguém | auxiliar; ajudar}
\end{entry}

\begin{entry}{培}{pei2}{11}{⼟}
  \definition{v.}{aterrar com terra; aterrar | fomentar; treinar | cultivar; crescer e desenvolver-se propositalmente}
\end{entry}

\begin{entry}{培训}{pei2xun4}{11,5}{⼟、⾔}[HSK 4]
  \definition{v.}{treinar (trabalhadores técnicos, quadros profissionais, etc.)}
\end{entry}

\begin{entry}{培训班}{pei2 xun4 ban1}{11,5,10}{⼟、⾔、⽟}[HSK 4]
  \definition{s.}{aula de treinamento; curso de treinamento}
\end{entry}

\begin{entry}{培养}{pei2yang3}{11,9}{⼟、⼋}[HSK 4]
  \definition{v.}{cultivar (plantas, microorganismos) | promover; treinar ou desenvolver; educar e treinar para um determinado propósito durante um longo período de tempo; fazer crescer | progredir gradualmente; desenvolver ou cultivar gradualmente (hábito, qualidade, caráter, emoção, estilo, interesse, habilidade, etc.)}
\end{entry}

\begin{entry}{培育}{pei2yu4}{11,8}{⼟、⾁}[HSK 4]
  \definition{v.}{criar; fomentar; educar; procriar; nutrir; cultivar}
\end{entry}

\begin{entry}{赔}{pei2}{12}{⾙}[HSK 5]
  \definition{v.}{compensar; pagar por; indenizar | sofrer uma perda; fazer negócios e perder dinheiro}
\end{entry}

\begin{entry}{赔偿}{pei2chang2}{12,11}{⾙、⼈}[HSK 5]
  \definition{v.}{indenizar; compensar; pagar por; indenizar outras pessoas ou grupos por perdas causadas por suas próprias ações}
\end{entry}

\begin{entry}{赔钱}{pei2qian2}{12,10}{⾙、⾦}
  \definition{v.+compl.}{perder dinheiro | compensar; compensar com dinheiro os prejuízos causados a terceiros}
\end{entry}

\begin{entry}{佩}{pei4}{8}{⼈}
  \definition{s.}{um ornamento usado como pingente amarrados em cintos nos tempos antigos}
  \definition{v.}{vestir (na cintura, etc.) | (arcaico) admirar | (arcaico) usar, especialmente uma pistola ou espada, na cintura}
\end{entry}

\begin{entry}{佩服}{pei4fu2}{8,8}{⼈、⽉}
  \definition{v.}{admirar}
\end{entry}

\begin{entry}{配}{pei4}{10}{⾣}[HSK 3]
  \definition{adj.}{adequado; bem combinado}
  \definition{s.}{cônjuge (geralmente referindo-se a uma esposa)}
  \definition{v.}{unir-se em matrimônio | (animais) acasalar; copular | compor; combinar; mesclar; amalgamar; misturar | distribuir de forma planejada; repartir | encontrar algo para encaixar ou substituir outra coisa; compensar as partes faltantes de acordo com certos padrões | combinar; harmonizar com; estar em harmonia com | exilar; banir; nos tempos antigos, referia-se ao exílio de criminosos}
  \definition{v.aux.}{adequar-se a; merecer; ser qualificado; ser digno de}
\end{entry}

\begin{entry}{配备}{pei4bei4}{10,8}{⾣、⼡}[HSK 5]
  \definition{s.}{equipamento; material; conjunto completo de utensílios, etc.}
  \definition{v.}{fornecer; alocar; equipar; distribuir conforme necessário | posicionar; dispor (tropas, etc.)}
\end{entry}

\begin{entry}{配合}{pei4he2}{10,6}{⾣、⼝}[HSK 3]
  \definition{v.}{cooperar; coordenar; todas as partes trabalham juntas para concluir tarefas comuns}
\end{entry}

\begin{entry}{配套}{pei4tao4}{10,10}{⾣、⼤}[HSK 5]
  \definition{v.+compl.}{formar um conjunto ou sistema completo; combinar vários elementos relacionados em um conjunto completo}
\end{entry}

\begin{entry}{喷}{pen1}{12}{⼝}[HSK 5]
  \definition{v.}{jorrar; esguichar; expelir sob pressão | borrifar; espalhar; pulverizar}
  \seeref{喷}{pen4}
\end{entry}

\begin{entry}{盆}{pen2}{9}{⽫}[HSK 5]
  \definition*{s.}{Sobrenome Pen}
  \definition[个]{s.}{bacia; banheira; panela; utensílios para guardar ou lavar coisas}
\end{entry}

\begin{entry}{盆友}{pen2you3}{9,4}{⽫、⼜}
  \definition{s.}{(gíria na \emph{Internet}) amigo (trocadilho com 朋友)}
  \seealsoref{朋友}{peng2you5}
\end{entry}

\begin{entry}{喷}{pen4}{12}{⼝}
  \definition{s.}{na época; tempo no mercado; época em que frutas, peixes e camarões são comercializados em grande quantidade | colheita; número de vezes que floresceu e frutificou; número de vezes que foi colhido na maturação}
  \seeref{喷}{pen1}
\end{entry}

\begin{entry}{朋}{peng2}{8}{⽉}
  \definition*{s.}{Sobrenome Peng}
  \definition{s.}{amigo}
  \definition{v.}{(literário) rivalizar; igualar; comparar | (literário) reunir-se em grupo; juntar-se em grupo}
\end{entry}

\begin{entry}{朋友}{peng2you5}{8,4}{⽉、⼜}[HSK 1]
  \definition[个,位,帮,群]{s.}{amigo; pessoas que têm um bom relacionamento, uma boa relação, se entendem e se ajudam mutuamente | namorado; namorada}
\end{entry}

\begin{entry}{膨}{peng2}{16}{⾁}
  \definition{v.}{inchar; inflar | expandir; aumentar o comprimento ou o volume de um objeto}
\end{entry}

\begin{entry}{膨胀}{peng2zhang4}{16,8}{⾁、⾁}
  \definition{v.}{expandir | inflar | inchar}
\end{entry}

\begin{entry}{碰}{peng4}{13}{⽯}[HSK 2]
  \definition{v.}{tocar; bater; esbarrar | encontrar; esbarrar | arriscar; tentar | tentar a sorte | reunir-se para discutir; ter uma reunião curta}
\end{entry}

\begin{entry}{碰到}{peng4 dao4}{13,8}{⽯、⼑}[HSK 2]
  \definition{v.}{encontrar (com); esbarrar; cruzar}
\end{entry}

\begin{entry}{碰见}{peng4 jian4}{13,4}{⽯、⾒}[HSK 2]
  \definition{v.}{encontrar; encontrar-se; sem combinar, encontrar-se por acaso}
\end{entry}

\begin{entry}{碰头}{peng4tou2}{13,5}{⽯、⼤}
  \definition{s.}{colisão | conflito}
  \definition{v.}{colidir}
  \definition{v.+compl.}{conhecer e discutir | juntar ideias | ver-se}
\end{entry}

\begin{entry}{碰运气}{peng4yun4qi5}{13,7,4}{⽯、⾡、⽓}
  \definition{v.}{deixar algo ao acaso | tentar a sorte}
\end{entry}

\begin{entry}{批}{pi1}{7}{⼿}[HSK 4]
  \definition{adj.}{(compra ou venda) atacado; a granel; em grandes quantidades}
  \definition{clas.}{para mercadorias a granel, grande número de pessoas}
  \definition{s.}{fibras de algodão, linho, etc., prontas para serem estiradas e torcidas | anotação; comentário}
  \definition{v.}{escrever comentários ou críticas sobre documentos subordinados, textos de outras pessoas, tarefas etc. | refutar; criticar | dar um tapa}
\end{entry}

\begin{entry}{批评}{pi1ping2}{7,7}{⼿、⾔}[HSK 3]
  \definition{v.}{criticar; comentar sobre deficiências e erros | criticar; apontar vantagens e desvantagens; comentar sobre o que é bom e o que é ruim}
\end{entry}

\begin{entry}{批准}{pi1zhun3}{7,10}{⼿、⼎}[HSK 3]
  \definition{v.}{aprovar}
\end{entry}

\begin{entry}{披}{pi1}{8}{⼿}[HSK 5]
  \definition{v.}{colocar sobre os ombros; enrolar em volta; cobrir ou colocar sobre os ombros | abrir; desenrolar; espalhar | abrir-se; rachar}
\end{entry}

\begin{entry}{皮}{pi2}{5}{⽪}[HSK 3][Kangxi 107]
  \definition*{s.}{Sobrenome Pi}
  \definition{adj.}{macios e encharcados; não mais crocantes | malandro; travesso | apático; endurecido; indiferente devido a repetidas repreensões | pegajoso; tenaz; resiliente}
  \definition{pref.}{pico- (um trilhonésimo)}
  \definition[层,块,张,个]{s.}{pele; casca; uma camada de tecido na superfície dos organismos animais e vegetais | pele; couro; couro processado | capa; embalagem; a camada externa que envolve algo | superfície do objeto | folha; peça larga e plana (de algum material fino) | borracha}
\end{entry}

\begin{entry}{皮包}{pi2 bao1}{5,5}{⽪、⼓}[HSK 3]
  \definition[个,只,款]{s.}{bolsa; pasta; portfólio; bolsas de couro}
\end{entry}

\begin{entry}{皮肤}{pi2fu1}{5,8}{⽪、⾁}[HSK 5]
  \definition{adj.}{superficial}
  \definition[层,块]{s.}{pele; couro; derme}
\end{entry}

\begin{entry}{皮卡}{pi2ka3}{5,5}{⽪、⼘}
  \definition{s.}{(empréstimo linguístico) \emph{pick-up} | caminhonete}
\end{entry}

\begin{entry}{皮卡丘}{pi2ka3qiu1}{5,5,5}{⽪、⼘、⼀}
  \definition*{s.}{Pikachu (Pokémon, 口袋妖怪)}
  \seealsoref{口袋妖怪}{kou3dai4 yao1guai4}
\end{entry}

\begin{entry}{皮下}{pi2xia4}{5,3}{⽪、⼀}
  \definition{adj.}{(injeção) subcutâneo | sob a pele}
\end{entry}

\begin{entry}{皮鞋}{pi2xie2}{5,15}{⽪、⾰}[HSK 5]
  \definition[双,只,款]{s.}{sapatos feitos de couro}
\end{entry}

\begin{entry}{啤}{pi2}{11}{⼝}
  \definition{s.}{cerveja}
\end{entry}

\begin{entry}{啤酒}{pi2jiu3}{11,10}{⼝、⾣}[HSK 3]
  \definition[杯,瓶,罐,桶,缸]{s.}{(empréstimo linguístico) cerveja; uma bebida de baixo teor alcoólico feita de malte de cevada e lúpulo, com espuma e aroma especial}
\end{entry}

\begin{entry}{啤酒馆}{pi2jiu3guan3}{11,10,11}{⼝、⾣、⾷}
  \definition{s.}{cervejaria}
\end{entry}

\begin{entry}{脾}{pi2}{12}{⾁}
  \definition{s.}{baço}
\end{entry}

\begin{entry}{脾气}{pi2qi5}{12,4}{⾁、⽓}[HSK 5]
  \definition[发,个]{s.}{temperamento; disposição; referindo-se ao caráter de uma pessoa | mau humor; temperamento irascível}
\end{entry}

\begin{entry}{匹}{pi3}{4}{⼖}[HSK 5]
  \definition{adj.}{solitário}
  \definition{clas.}{para cavalos, mulas, etc. | para rolos inteiros de seda ou tecido}
  \definition{v.}{ser igual a; ser compatível com}
\end{entry}

\begin{entry}{否}{pi3}{7}{⼝}
  \definition{adj.}{ruim; maligno; perverso}
  \definition{v.}{censurar}
  \seeref{否}{fou3}
\end{entry}

\begin{entry}{屁}{pi4}{7}{⼫}
  \definition{s.}{vento (ou gás) (dos intestinos); peido | (vulgar) bobagem; merda; lixo | quadril; bunda}
\end{entry}

\begin{entry}{屁股}{pi4gu5}{7,8}{⼫、⾁}
  \definition{s.}{nádega | quadris}
\end{entry}

\begin{entry}{屁话}{pi4hua4}{7,8}{⼫、⾔}
  \definition{s.}{absurdo | tolice | besteira}
\end{entry}

\begin{entry}{譬}{pi4}{20}{⾔}
  \definition{s.}{exemplo; analogia; metáfora}
  \definition{v.}{dar um exemplo; fazer uma analogia}
\end{entry}

\begin{entry}{譬如}{pi4ru2}{20,6}{⾔、⼥}
  \definition{conj.}{por exemplo | como}
\end{entry}

\begin{entry}{片}{pian1}{4}{⽚}
  \definition{s.}{película; filme; refere-se a filmes com imagens, paisagens ou imagens gravadas com som}
  \seeref{片}{pian4}
\end{entry}

\begin{entry}{片儿}{pian1r5}{4,2}{⽚、⼉}
  \definition{s.}{folha | película; filme}
\end{entry}

\begin{entry}{扁}{pian1}{9}{⼾}
  \definition{adj.}{pequeno | fora do caminho; remoto}
  \seeref{扁}{bian3}
\end{entry}

\begin{entry}{扁舟}{pian1 zhou1}{9,6}{⼾、⾈}
  \definition[叶,艘]{s.}{pequeno barco; esquife}
\end{entry}

\begin{entry}{偏}{pian1}{11}{⼈}[HSK 6]
  \definition{adj.}{parcial; preconceituoso; injusto; focando apenas em um lado | torto; inclinado (oposto de 正) | não dominante; auxiliar | remoto; periférico; longe do centro; incomum}
  \definition{adv.}{intencionalmente; insistentemente; persistentemente; indica ir intencionalmente contra o senso comum ou a solicitação de outra pessoa}
  \definition{expr.}{uma expressão educada para indicar que alguém já tomou chá ou comeu}
  \definition{v.}{divergir; não ser igual a; ser diferente de; exceder ou ficar aquém dos padrões normais | desviar-se; afastar-se; sair na direção certa}
  \seealsoref{正}{zheng4}
\end{entry}

\begin{entry}{偏偏}{pian1pian1}{11,11}{⼈、⼈}
  \definition{adv.}{voluntariamente | insistentemente | persistentemente | ao contrário da expectativa | infelizmente (indicando que alguma coisa aconteceu ao contrário do que se esperava) | teimosamente (indicando que algo é o oposto ao que seria normal ou razoável) | precisamente (indicando que alguém ou um grupo é escolhido)}
\end{entry}

\begin{entry}{篇}{pian1}{15}{⽵}[HSK 2]
  \definition*{s.}{Sobrenome Pian}
  \definition{clas.}{usado para folhas de papel, páginas de livros, artigos, etc.}
  \definition{s.}{um pedaço de escrita | folha (de papel, etc.) | (para papel, folhas de livros, artigos, etc.) folha; página; pedaço}
\end{entry}

\begin{entry}{便}{pian2}{9}{⼈}
  \definition*{s.}{Sobrenome Pian}
  \definition{adj.}{silencioso e confortável}
  \seeref{便}{bian4}
\end{entry}

\begin{entry}{便宜}{pian2yi5}{9,8}{⼈、⼧}[HSK 2]
  \definition{adj.}{barato; acessível}
  \definition[个,份,件]{s.}{vantagem em algum aspecto | ganho; lucro; vantagem; benefício indevido}
  \definition{v.}{deixar alguém escapar impune; obter algum benefício}
  \seeref{便宜}{bian4yi2}
\end{entry}

\begin{entry}{片}{pian4}{4}{⽚}[HSK 2][Kangxi 91]
  \definition*{s.}{Sobrenome Pian}
  \definition{adj.}{breve; parcial; incompleto; fragmentário; esporádico; breve | unilateral}
  \definition{clas.}{usado para coisas em forma de lâminas | usado para terrenos ou superfícies aquáticas com a mesma paisagem e que estão conectados entre si | usado para paisagens, clima, sons, linguagem, intenções, etc. (usado em conjunto com o numeral 一)}
  \definition{s.}{plano, fatia; floco; pedaço fino; algo plano e fino | seção; parte de uma grande área; uma pequena parte do todo ou uma área menor dividida dentro de uma área maior | filme; peça de TV; referência ao filme}
  \definition{v.}{fatiar; cortar em fatias; cortar em fatias finas com uma faca | abrir; cortar; separar}
  \seeref{片}{pian1}
  \seealsoref{一}{yi1}
\end{entry}

\begin{entry}{片面}{pian4mian4}{4,9}{⽚、⾯}[HSK 4]
  \definition{adj.}{unilateral (em oposição a 全面)}
  \seealsoref{全面}{quan2mian4}
\end{entry}

\begin{entry}{骗}{pian4}{12}{⾺}[HSK 5]
  \definition{v.}{enganar; trapacear; iludir; ludibriar; usar mentiras ou meios fraudulentos para fazer alguém acreditar ou ser enganado | enganar; fraudar | montar (um cavalo); balançar (ou saltar) para a sela}
\end{entry}

\begin{entry}{骗子}{pian4 zi5}{12,3}{⾺、⼦}[HSK 5]
  \definition[个]{s.}{trapaceiro; vigarista; fraudador; impostor; golpista; pessoa que obtém bens de forma fraudulenta}
\end{entry}

\begin{entry}{漂}{piao1}{14}{⽔}
  \definition{v.}{flutuar | estar a deriva}
  \seeref{漂}{piao3}
  \seeref{漂}{piao4}
\end{entry}

\begin{entry}{漂流}{piao1liu2}{14,10}{⽔、⽔}
  \definition{s.}{\emph{rafting}}
  \definition{v.}{ser levado pela correnteza | flutuar ao longo ou sobre}
\end{entry}

\begin{entry}{飘}{piao1}{15}{⾵}
  \definition{adj.}{complacente | frívolo | fraco | instável | bambo | cambaleante}
  \definition{v.}{flutuar (no ar) | esvoaçar | tremular}
\end{entry}

\begin{entry}{漂}{piao3}{14}{⽔}
  \definition{v.}{alvejar | branquear}
  \seeref{漂}{piao1}
  \seeref{漂}{piao4}
\end{entry}

\begin{entry}{票}{piao4}{11}{⽰}[HSK 1]
  \definition{clas.}{para grupos, lotes, transações comerciais}
  \definition[张]{s.}{bilhete; passagem; ingresso | cédula | nota bancária; conta | pessoa mantida em cativeiro por sequestradores para obter resgate; refém | apresentação amadora (de ópera de Pequim, etc.); peças teatrais amadoras}
  \definition{v.}{atuar como amador (na ópera de Pequim)}
\end{entry}

\begin{entry}{票价}{piao4 jia4}{11,6}{⽰、⼈}[HSK 3]
  \definition[个]{s.}{o preço de um ingresso; taxa de admissão; taxa de entrada}
\end{entry}

\begin{entry}{漂}{piao4}{14}{⽔}
  \definition{adj.}{usado em 漂亮}
  \seeref{漂}{piao1}
  \seeref{漂}{piao3}
  \seealsoref{漂亮}{piao4liang5}
\end{entry}

\begin{entry}{漂亮}{piao4liang5}{14,9}{⽔、⼇}[HSK 2]
  \definition{adj.}{bonito; lindo; atraente; de boa aparência; esteticamente agradável | excelente; notável | não pode ser utilizado para descrever homens}
\end{entry}

\begin{entry}{拼}{pin1}{9}{⼿}[HSK 5]
  \definition{v.}{montar; juntar as peças | dar tudo de si no trabalho; estar disposto a arriscar a vida (em lutas, no trabalho, etc.); fazer tudo o que for preciso; arriscar tudo}
\end{entry}

\begin{entry}{拼命}{pin1ming4}{9,8}{⼿、⼝}
  \definition{adv.}{com toda a força | desesperadamente}
  \definition{v.+compl.}{arriscar a vida de alguém | desafiar a morte | colocar-se em uma luta desesperada | fazer algo desesperadamente | exercer a maior força}
\end{entry}

\begin{entry}{拼音}{pin1yin1}{9,9}{⼿、⾳}
  \definition{s.}{escrita fonética | pinyin (romanização chinesa)}
\end{entry}

\begin{entry}{贫}{pin2}{8}{⾙}
  \definition{adj.}{pobre; empobrecido | inadequado; deficiente; insuficiente | tagarela; loquaz; falante; chato e irritante}
\end{entry}

\begin{entry}{贫民窟}{pin2min2ku1}{8,5,13}{⾙、⽒、⽳}
  \definition{s.}{favela}
\end{entry}

\begin{entry}{频}{pin2}{13}{⾴}
  \definition*{s.}{Sobrenome Pin}
  \definition{adj.}{frequente}
  \definition{adv.}{frequentemente; repetidamente}
  \definition{s.}{Física: frequência; o número de vezes que um objeto vibra por segundo}
\end{entry}

\begin{entry}{频道}{pin2dao4}{13,12}{⾴、⾡}[HSK 5]
  \definition[个]{s.}{canal; canal de frequência; televisão e rádio, os sinais de som e imagem ocupam um determinado canal de frequência}
\end{entry}

\begin{entry}{频繁}{pin2fan2}{13,17}{⾴、⽷}[HSK 5]
  \definition{adj.}{frequentemente}
  \definition{adj.}{frequente}
\end{entry}

\begin{entry}{品}{pin3}{9}{⼝}[HSK 5]
  \definition*{s.}{Sobrenome Pin}
  \definition{s.}{artigo; produto | grau; classe; classificação; nível | caráter; qualidade | classificação; os graus dos funcionários públicos antigos, num total de nove graus}
  \definition{v.}{provar; saborear; degustar algo com discernimento | soprar; tocar (instrumentos de sopro) | avaliar; distinguir o bom do ruim}
\end{entry}

\begin{entry}{品德}{pin3de2}{9,15}{⼝、⼻}
  \definition{s.}{caráter moral | moralidade}
\end{entry}

\begin{entry}{品质}{pin3zhi4}{9,8}{⼝、⾙}[HSK 4]
  \definition[个,种]{s.}{qualidade; caráter; natureza do pensamento, da compreensão, do caráter, etc., conforme expresso no comportamento, no estilo, etc. | qualidade (de produtos, mercadorias, etc.)}
\end{entry}

\begin{entry}{品种}{pin3zhong3}{9,9}{⼝、⽲}[HSK 5]
  \definition[个]{s.}{raça; linhagem; variedade; refere-se a um grupo de organismos com características genéticas comuns, formados por meio da seleção e cultivo artificiais de culturas, gado, aves, etc. | variedade; sortimento; referência geral ao tipo de item}
\end{entry}

\begin{entry}{乒}{ping1}{6}{⼃}
  \definition{interj.}{(onomatopéia) estalo; estouro; estrondo | (onomatopéia)  ``ping''}
  \definition{s.}{(abreviação) tênis de mesa; pingue-pongue | (abreviação) bola de tênis de mesa; bola de pingue-pongue}
\end{entry}

\begin{entry}{乒乓球}{ping1pang1qiu2}{6,6,11}{⼃、⼃、⽟}
  \definition[个]{s.}{tênis de mesa |ping-pong}
\end{entry}

\begin{entry}{平}{ping2}{5}{⼲}[HSK 2]
  \definition*{s.}{Sobrenome Ping}
  \definition{adj.}{plano; nivelado; uniforme; liso | igual; justo | mesma pontuação; empatado | médio; comum | silencioso; tranquilo | no mesmo nível; altura igual; sem diferença | imparcial; médio; equitativo | calmo; estável; tranquilo | comum;  frequente}
  \definition{s.}{no mesmo nível; em pé de igualdade com; igual | tom nivelado, um dos quatro tons do chinês clássico}
  \definition{v.}{tornar nivelado ou uniforme; nivelar | reprimir; suprimir | acalmar; tornar pacífico; silenciar (acalmar); conter a raiva | estar no mesmo nível | acalmar; amenizar; controlar a raiva}
\end{entry}

\begin{entry}{平安}{ping2'an1}{5,6}{⼲、⼧}[HSK 2]
  \definition{s.}{seguro; bem; sem contratempos; sem acidentes; são e salvo}
\end{entry}

\begin{entry}{平常}{ping2chang2}{5,11}{⼲、⼱}[HSK 2]
  \definition{adj.}{comum; normal; ordinário; nada de especial}
  \definition{adv.}{normalmente; geralmente; como regra geral}
\end{entry}

\begin{entry}{平等}{ping2deng3}{5,12}{⼲、⽵}[HSK 2]
  \definition{adj.}{igual; igualdade; refere-se ao fato de as pessoas gozarem de tratamento igualitário nos aspectos sociais, políticos, econômicos e jurídicos}
\end{entry}

\begin{entry}{平地}{ping2di4}{5,6}{⼲、⼟}
  \definition{v.}{nivelar a terra | aplanar}
\end{entry}

\begin{entry}{平方}{ping2fang1}{5,4}{⼲、⽅}[HSK 4]
  \definition{s.}{quadrado}
\end{entry}

\begin{entry}{平方米}{ping2fang1 mi3}{5,4,6}{⼲、⽅、⽶}
  \definition{clas.}{unidade de medida de área, 1 metro quadrado equivale a 10.000 centímetros quadrados}
\end{entry}

\begin{entry}{平方市丈}{ping2fang1 shi4 zhang4}{5,4,5,3}{⼲、⽅、⼱、⼀}
  \definition{clas.}{pés quadrados}
\end{entry}

\begin{entry}{平静}{ping2jing4}{5,14}{⼲、⾭}[HSK 4]
  \definition{adj.}{(humor, ambiente, etc.) calmo; quieto; pacífico; tranquilo}
\end{entry}

\begin{entry}{平均}{ping2jun1}{5,7}{⼲、⼟}[HSK 4]
  \definition{adj.}{igual; médio}
  \definition{s.}{média}
  \definition{v.}{calcular a média de um conjunto de números}
\end{entry}

\begin{entry}{平时}{ping2shi2}{5,7}{⼲、⽇}[HSK 2]
  \definition{s.}{em tempos normais; em tempos comuns | em tempo de paz; refere-se a períodos normais}
\end{entry}

\begin{entry}{平台}{ping2tai2}{5,5}{⼲、⼝}
  \definition{s.}{plataforma | terraço | edifício de telhado plano}
\end{entry}

\begin{entry}{平坦}{ping2tan3}{5,8}{⼲、⼟}[HSK 5]
  \definition{adj.}{plano; uniforme; nivelado; liso; sem elevações ou depressões (referindo-se principalmente ao relevo)}
\end{entry}

\begin{entry}{平稳}{ping2 wen3}{5,14}{⼲、⽲}[HSK 4]
  \definition{adj.}{firme; estável; suave e constante; sem oscilações ou flutuações}
\end{entry}

\begin{entry}{平原}{ping2yuan2}{5,10}{⼲、⼚}[HSK 5]
  \definition[片]{s.}{campo; planície; terreno plano e extenso}
\end{entry}

\begin{entry}{评}{ping2}{7}{⾔}[HSK 6]
  \definition*{s.}{Sobrenome Ping}
  \definition{v.}{comentar; criticar; revisar | julgar; avaliar}
\end{entry}

\begin{entry}{评估}{ping2gu1}{7,7}{⾔、⼈}[HSK 5]
  \definition{v.}{estimar; avaliar; apreciar; avaliar e estimar (coisas abstratas)}
\end{entry}

\begin{entry}{评价}{ping2jia4}{7,6}{⾔、⼈}[HSK 3]
  \definition[个,项,条,份]{s.}{avaliação; apreciação; comentários ou opiniões de pessoas sobre alguém ou algo}
  \definition{v.}{estimar valor; avaliar valor}
\end{entry}

\begin{entry}{评论}{ping2lun4}{7,6}{⾔、⾔}[HSK 5]
  \definition[篇]{s.}{revisão; comentário; artigos ou comentários críticos}
  \definition{v.}{discutir; comentar sobre algo ou alguém}
\end{entry}

\begin{entry}{凭}{ping2}{8}{⼏}[HSK 5]
  \definition{prep.}{introduzir a ação ou o comportamento com base em algo; quando a frase nominal após 凭 é longa, pode-se adicionar 着 após 凭}
  \definition[张]{s.}{prova; evidência}
  \definition{v.}{apoiar-se; encostar-se | confiar em; depender de | basear-se em; tomar como base}
  \seealsoref{着}{zhe5}
\end{entry}

\begin{entry}{苹}{ping2}{8}{⾋}
  \definition[个]{s.}{uma espécie de artemísia | maçã | lentilha-d'água}
\end{entry}

\begin{entry}{苹果}{ping2guo3}{8,8}{⾋、⽊}[HSK 3]
  \definition[个,斤,筐,箱,棵,种]{s.}{maçã}
\end{entry}

\begin{entry}{瓶}{ping2}{10}{⽡}[HSK 2]
  \definition*{s.}{Sobrenome Ping}
  \definition{clas.}{usado para coisas que são engarrafadas; quantidade contida em um frasco, vaso, garrafa}
  \definition[个]{s.}{jarra; vaso; frasco; garrafa;}
\end{entry}

\begin{entry}{瓶盖}{ping2gai4}{10,11}{⽡、⽫}
  \definition{s.}{tampa de garrafa}
\end{entry}

\begin{entry}{瓶装}{ping2zhuang1}{10,12}{⽡、⾐}
  \definition{adj.}{engarrafado}
\end{entry}

\begin{entry}{瓶子}{ping2zi5}{10,3}{⽡、⼦}[HSK 2]
  \definition[个,只,种]{s.}{garrafa; recipientes com gargalo feitos de cerâmica, vidro, plástico, etc., geralmente em forma cilíndrica}
\end{entry}

\begin{entry}{甁}{ping2}{12}{⽡}
  \variantof{瓶}
\end{entry}

\begin{entry}{坡}{po1}{8}{⼟}[HSK 6]
  \definition{adj.}{inclinado}
  \definition{s.}{declive | encosta}
\end{entry}

\begin{entry}{泼}{po1}{8}{⽔}[HSK 5]
  \definition{adj.}{rude e irracional; mal-humorado}
  \definition{v.}{espalhar; salpicar; derramar; derramar ou espalhar o líquido com força para fora |}
\end{entry}

\begin{entry}{颇}{po1}{11}{⽪}
  \definition*{s.}{Sobrenome Po}
  \definition{adv.}{muito, bastante (linguagem escrita)}
\end{entry}

\begin{entry}{迫}{po4}{8}{⾡}
  \definition{adj.}{urgente; premente}
  \definition{s.}{morteiro; artilharia}
  \definition{v.}{compelir; forçar; pressionar | aproximar-se; ir em direção a (ou perto de)}
\end{entry}

\begin{entry}{迫切}{po4qie4}{8,4}{⾡、⼑}[HSK 4]
  \definition{adj.}{urgente; premente; muito ansiosamente, a ponto de ser difícil esperar}
\end{entry}

\begin{entry}{破}{po4}{10}{⽯}[HSK 3]
  \definition{adj.}{quebrado; danificado; rasgado; desgastado | insignificante; péssimo; medíocre}
  \definition{v.}{quebrar; danificar | dividir; cortar; separar | trocar (dinheiro) | livrar-se de; destruir; romper com | derrotar; capturar (uma cidade, etc.) | gastar dinheiro | revelar a verdade sobre; expor | mudar; romper; quebrar (regras, hábitos, ideias, etc.)}
\end{entry}

\begin{entry}{破产}{po4chan3}{10,6}{⽯、⼇}[HSK 4]
  \definition{v.+compl.}{falir; ir à falência; tornar-se insolvente; entrar em liquidação; perder todo o patrimônio | falhar; fracassar; não dar em nada; figura de linguagem (geralmente com uma conotação depreciativa)}
\end{entry}

\begin{entry}{破坏}{po4huai4}{10,7}{⽯、⼟}[HSK 3]
  \definition{v.}{demolir; naufragar; soçobrar; destruir; obliterar | quebrar; violar (um acordo, regulamento, etc.); não cumprir (disposições legais, regras, acordos, princípios, etc.) | prejudicar; perturbar; sabotar; causar grande dano; causar danos às coisas | reverter; mudar (um sistema social, costume, etc.) completamente ou violentamente | destruir; decompor; danificar o tecido ou a estrutura de um objeto}
\end{entry}

\begin{entry}{破坏性}{po4huai4xing4}{10,7,8}{⽯、⼟、⼼}
  \definition{adj.}{destrutivo}
  \definition{s.}{poder destrutivo}
\end{entry}

\begin{entry}{扑}{pu1}{5}{⼿}[HSK 6]
  \definition{s.}{sopro; refere-se a gases, fragrâncias, cinzas, areia, etc. que se apresentam | espanador}
  \definition{v.}{atacar; lançar-se sobre; correr para frente com toda a sua força e, de repente, jogar todo o seu corpo em um objeto | dedicar; dedicar todas as energias a uma causa; colocar toda a sua energia em (trabalho, carreira, etc.) | bater asas; esvoaçar | inclinar-se}
\end{entry}

\begin{entry}{扑克}{pu1ke4}{5,7}{⼿、⼗}
  \definition{s.}{(empréstimo linguístico) (jogo) \emph{poker}  | baralho}
\end{entry}

\begin{entry}{铺}{pu1}{12}{⾦}[HSK 6]
  \definition{clas.}{usado para kang, etc.; kang, uma plataforma de alvenaria ou de barro em uma extremidade de um cômodo, aquecida no inverno por fogueiras embaixo e coberta com esteiras para dormir}
  \definition{v.}{espalhar; estender; desdobrar | colocar; pavimentar}
  \seeref{铺}{pu4}
\end{entry}

\begin{entry}{铺垫}{pu1dian4}{12,9}{⾦、⼟}
  \definition{s.}{cobre leito | colcha | roupa de cama}
  \definition{v.}{pavimentar}
\end{entry}

\begin{entry}{葡}{pu2}{12}{⾋}
  \definition*{s.}{Portugal, abreviação de 葡萄牙}
  \seealsoref{葡萄牙}{pu2tao2ya2}
\end{entry}

\begin{entry}{葡汉词典}{pu2-han4 ci2dian3}{12,5,7,8}{⾋、⽔、⾔、⼋}
  \definition{s.}{dicionário português-chinês}
  \seealsoref{汉葡词典}{han4-pu2 ci2dian3}
\end{entry}

\begin{entry}{葡萄酒}{pu2 tao2 jiu3}{12,11,10}{⾋、⾋、⾣}[HSK 5]
  \definition[瓶]{s.}{vinho (de uva)}
\end{entry}

\begin{entry}{葡萄牙}{pu2tao2ya2}{12,11,4}{⾋、⾋、⽛}
  \definition{s.}{Portugal}
\end{entry}

\begin{entry}{葡萄牙文}{pu2tao2ya2wen2}{12,11,4,4}{⾋、⾋、⽛、⽂}
  \definition{s.}{português, língua portuguesa}
  \seealsoref{葡文}{pu2wen2}
\end{entry}

\begin{entry}{葡萄牙语}{pu2tao2ya2yu3}{12,11,4,9}{⾋、⾋、⽛、⾔}
  \definition{s.}{português, língua portuguesa}
  \seealsoref{葡语}{pu2yu3}
\end{entry}

\begin{entry}{葡萄}{pu2tao5}{12,11}{⾋、⾋}[HSK 5]
  \definition[棵,串]{s.}{parreira | uva}
\end{entry}

\begin{entry}{葡文}{pu2wen2}{12,4}{⾋、⽂}
  \definition{s.}{português, língua portuguesa}
  \seealsoref{葡萄牙文}{pu2tao2ya2wen2}
\end{entry}

\begin{entry}{葡语}{pu2yu3}{12,9}{⾋、⾔}
  \definition{s.}{português, língua portuguesa}
  \seealsoref{葡萄牙语}{pu2tao2ya2yu3}
\end{entry}

\begin{entry}{普}{pu3}{12}{⽇}
  \definition*{s.}{Sobrenome Pu}
  \definition{adj.}{geral; universal}
\end{entry}

\begin{entry}{普遍}{pu3bian4}{12,12}{⽇、⾡}[HSK 3]
  \definition{adj.}{geral; comum; universal; difundido; a existência é muito ampla; tem semelhança}
\end{entry}

\begin{entry}{普及}{pu3ji2}{12,3}{⽇、⼃}[HSK 3]
  \definition{adj.}{popular; universal; onipresente; amplamente compreendido, aceito ou utilizado}
  \definition[种]{v.}{popularizar; disseminar; espalhar entre as pessoas; promover amplamente o conhecimento, a educação, a tecnologia, etc. para popularizá-los}
\end{entry}

\begin{entry}{普通}{pu3 tong1}{12,10}{⽇、⾡}[HSK 2]
  \definition{adj.}{comum; normal; geral; médio; em geral, nada de especial, como a maioria das pessoas ou coisas}
\end{entry}

\begin{entry}{普通话}{pu3tong1hua4}{12,10,8}{⽇、⾡、⾔}[HSK 2]
  \definition*{s.}{Mandarim (literalmente "linguagem comum") | Putonghua (fala comum da língua chinesa) | Língua oficial da China}
\end{entry}

\begin{entry}{铺}{pu4}{12}{⾦}
  \definition{s.}{pequena loja; depósito | uma cama feita de tábuas de madeira; geralmente se refere a uma cama | estação de correios; antiga estação de correios (usada principalmente em nomes de lugares)}
  \seeref{铺}{pu1}
\end{entry}

\begin{entry}{瀑}{pu4}{18}{⽔}
  \definition{s.}{cachoeira; catarata}
  \seeref{瀑}{bao4}
\end{entry}

\begin{entry}{瀑布}{pu4bu4}{18,5}{⽔、⼱}
  \definition{s.}{queda de água | cachoeira | cascata | catarata}
\end{entry}

%%%%% EOF %%%%%

