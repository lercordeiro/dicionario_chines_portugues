%%%
%%% R
%%%

\section*{R}\addcontentsline{toc}{section}{R}

\begin{entry}{儿}{r5}{2}{⼉}
  \definition{suf.}{sufixo diminutivo não silábico | final retroflexo}
  \seeref{儿}{er2}
  \seeref{儿}{ren2}
\end{entry}

\begin{entry}{然}{ran2}{12}{⽕}
  \definition{conj.}{mas | no entanto}
\end{entry}

\begin{entry}{然而}{ran2'er2}{12,6}{⽕、⽽}[HSK 4]
  \definition{conj.}{ainda; mas; contudo; todavia; usado no início de uma frase para indicar uma transição; para indicar uma transição, geralmente é precedido por uma conjunção como 虽然 para indicar concessão}
  \seealsoref{虽然}{sui1 ran2}
\end{entry}

\begin{entry}{然后}{ran2hou4}{12,6}{⽕、⼝}[HSK 2]
  \definition{conj.}{depois | logo | portanto}
\end{entry}

\begin{entry}{燃料}{ran2 liao4}{16,10}{⽕、⽃}[HSK 4]
  \definition{s.}{combustível; carburante; substâncias que podem gerar calor e energia luminosa quando queimadas podem ser divididas em três tipos de acordo com sua forma: combustível sólido (como carvão, carvão vegetal, madeira), combustível líquido (como gasolina, querosene) e combustível gasoso (como gás de carvão, biogás); também se refere a substâncias que podem gerar energia nuclear, como urânio, plutônio, etc.}
\end{entry}

\begin{entry}{燃烧}{ran2shao1}{16,10}{⽕、⽕}[HSK 4]
  \definition{s.}{combustão | flama}
  \definition{v.}{queimar; acender | arder; inflamar; ferver; metáfora para as emoções de uma pessoa serem muito fortes, como um fogo ardente}
\end{entry}

\begin{entry}{染}{ran3}{9}{⽊}[HSK 5]
  \definition*{s.}{sobrenome Ran}
  \definition{s.}{soja fermentada e temperada em forma de pasta}
  \definition{v.}{tingir; pintar | pegar (uma doença); cair em (um mau hábito, etc.) | sujar; contaminar | pegar (contrair) (uma doença) | adquirir (um mau hábito, etc.); contaminar}
\end{entry}

\begin{entry}{壤}{rang3}{20}{⼟}
  \definition{s.}{solo | terra | (literário) a terra (em contraste com o céu 天)}
\end{entry}

\begin{entry}{让}{rang4}{5}{⾔}[HSK 2]
  \definition{v.}{deixar alguém fazer alguma coisa |fazer alguém (sentir-se triste, etc.) | permitir | conceder}
\end{entry}

\begin{entry}{让步}{rang4bu4}{5,7}{⾔、⽌}
  \definition{v.+compl.}{fazer uma concessão | entregar | desistir | comprometer}
\end{entry}

\begin{entry}{绕}{rao4}{9}{⽷}[HSK 5]
  \definition*{s.}{sobrenome Rao}
  \definition{v.}{enrolar; bobinar; rebobinar | mover-se em círculo; girar; revolver | fazer um desvio; contornar; dar a volta | confundir; desorientar}
\end{entry}

\begin{entry}{热}{re4}{10}{⽕}[HSK 1]
  \definition{adj.}{quente; temperatura elevada | ardente; caloroso; profundamente afetuoso | ansioso; invejoso; descreve inveja e desejo de possuir algo | térmico; altamente radioativo | popular; muito procurado; muito apreciado por muitas pessoas}
  \definition{s.}{calor; energia liberada pelo movimento irregular das moléculas dentro de um objeto | febre; febre alta causada por doença | moda passageira; mania; febre}
  \definition{v.}{aquecer (geralmente se refere a alimentos)}
\end{entry}

\begin{entry}{热爱}{re4'ai4}{10,10}{⽕、⽖}[HSK 3]
  \definition{v.}{amar ardentemente; amar de coração; ter amor profundo por}
\end{entry}

\begin{entry}{热泪盈眶}{re4lei4ying2kuang4}{10,8,9,11}{⽕、⽔、⽫、⽬}
  \definition{expr.}{olhos cheios de lágrimas de emoção | extremamente emocionado}
\end{entry}

\begin{entry}{热量}{re4 liang4}{10,12}{⽕、⾥}[HSK 5]
  \definition{s.}{calor; quantidade de calor; calorias; em física, refere-se à energia transferida entre objetos com temperaturas diferentes, do objeto com temperatura mais alta para o objeto com temperatura mais baixa}
\end{entry}

\begin{entry}{热烈}{re4lie4}{10,10}{⽕、⽕}[HSK 3]
  \definition{adj.}{caloroso; calorosamente; fervoroso; ardente; entusiasmado}
\end{entry}

\begin{entry}{热门}{re4men2}{10,3}{⽕、⾨}[HSK 5]
  \definition{adj.}{popular}
  \definition{s.}{algo que desperta o interesse popular; metáfora para algo que está na moda e recebe a atenção de todos (em contraste com 冷门)}
  \seealsoref{冷门}{leng3men2}
\end{entry}

\begin{entry}{热闹}{re4nao5}{10,8}{⽕、⾾}[HSK 4]
  \definition{adj.}{animado; agitado; movimentado com barulho e excitação; descreve uma cena animada com uma atmosfera calorosa}
  \definition{s.}{uma vista emocionante; uma cena de agitação e excitação; atmosfera acolhedora}
  \definition{v.}{animar; divertir-se}
\end{entry}

\begin{entry}{热情}{re4qing2}{10,11}{⽕、⼼}[HSK 2]
  \definition{adj.}{caloroso | fervoroso | entusiasmado}
  \definition{s.}{entusiasmo | ardor | devoção | calor | zelo}
\end{entry}

\begin{entry}{热心}{re4xin1}{10,4}{⽕、⼼}[HSK 4]
  \definition{adj.}{ardente; sincero; entusiasmado; afetuoso; apaixonado; interessado}
  \definition{v.}{ser entusiasmado com alguma coisa}
\end{entry}

\begin{entry}{热血沸腾}{re4xue4fei4teng2}{10,6,8,13}{⽕、⾎、⽔、⾁}
  \definition{expr.}{ferver o sangue | apaixonar-se}
\end{entry}

\begin{entry}{人}{ren2}{2}{⼈}[HSK 1][Kangxi 9]
  \definition*{s.}{sobrenome Ren}
  \definition[个,名,位]{s.}{homem; pessoa; pessoas; ser humano | todos; cada um; todo mundo | adulto; crescido | uma pessoa envolvida em uma atividade específica | pessoas; outras pessoas | caráter; personalidade; qualidade, caráter ou reputação de uma pessoa | como alguém se sente; estado de saúde de alguém | mão de obra; força de trabalho}
\end{entry}

\begin{entry}{人才}{ren2cai2}{2,3}{⼈、⼿}[HSK 3]
  \definition{adj.}{aparência bonita, elegante}
  \definition[个]{s.}{talento; pessoal qualificado; pessoa com capacidade}
\end{entry}

\begin{entry}{人材}{ren2cai2}{2,7}{⼈、⽊}
  \variantof{人才}
\end{entry}

\begin{entry}{人道}{ren2dao4}{2,12}{⼈、⾡}
  \definition{s.}{solidariedade humana | humanitarismo | humano | a ``maneira humana'', um dos estágios do ciclo de reencarnação (budismo) | relação sexual}
\end{entry}

\begin{entry}{人工}{ren2gong1}{2,3}{⼈、⼯}[HSK 3]
  \definition{adj.}{feito pelo homem; artificial}
  \definition[个]{s.}{trabalho manual; trabalho feito à mão | mão de obra; homem-dia; uma unidade de cálculo da quantidade de trabalho realizado}
\end{entry}

\begin{entry}{人海}{ren2hai3}{2,10}{⼈、⽔}
  \definition{s.}{uma multidão | um mar de pessoas}
\end{entry}

\begin{entry}{人家}{ren2jia1}{2,10}{⼈、⼧}[HSK 4]
  \definition[对]{s.}{lar; família; família do noivo; casa do futuro marido}
  \seeref{人家}{ren2jia5}
\end{entry}

\begin{entry}{人家}{ren2jia5}{2,10}{⼈、⼧}
  \definition{pron.}{outros; uma pessoa ou pessoas diferentes do falante ou ouvinte; refere-se a alguém diferente de si mesmo ou de outra pessoa | certa pessoa ou pessoas (a pessoa ou pessoas mencionadas em um contexto próximo, aproximadamente equivalente ao pronome de terceira pessoa);  refere-se a uma pessoa ou algumas pessoas, com significado semelhante a 他 | eu; mim (usado retoricamente no lugar do primeiro pronome pessoal, muitas vezes expressando descontentamento de forma jocosa; geralmente usado quando se fala com pessoas próximas, para significar 自己, usado principamente por meninas)}
  \seeref{人家}{ren2jia1}
  \seealsoref{他}{ta1}
  \seealsoref{自己}{zi4ji3}
\end{entry}

\begin{entry}{人间}{ren2jian1}{2,7}{⼈、⾨}[HSK 5]
  \definition{s.}{o mundo humano | o Mundo; a Terra}
\end{entry}

\begin{entry}{人口}{ren2kou3}{2,3}{⼈、⼝}[HSK 2]
  \definition{s.}{pessoas | população}
\end{entry}

\begin{entry}{人类}{ren2lei4}{2,9}{⼈、⽶}[HSK 3]
  \definition[种]{s.}{humano; humanidade; raça humana}
\end{entry}

\begin{entry}{人力}{ren2 li4}{2,2}{⼈、⼒}[HSK 5]
  \definition{s.}{mão de obra; trabalho manual; força de trabalho}
\end{entry}

\begin{entry}{人们}{ren2 men5}{2,5}{⼈、⼈}[HSK 2]
  \definition{s.}{homens |  pessoas | o público}
\end{entry}

\begin{entry}{人民}{ren2 min2}{2,5}{⼈、⽒}[HSK 3]
  \definition[群,批,个]{s.}{o povo}
\end{entry}

\begin{entry}{人民币}{ren2min2bi4}{2,5,4}{⼈、⽒、⼱}[HSK 3]
  \definition*[块,张,元]{s.}{Renminbi (RMB); Yuan Chinês (CYN); nome da moeda chinesa}
\end{entry}

\begin{entry}{人权}{ren2quan2}{2,6}{⼈、⽊}
  \definition*{s.}{Direitos Humanos}
  \seealsoref{人权法}{ren2quan2fa3}
\end{entry}

\begin{entry}{人权法}{ren2quan2fa3}{2,6,8}{⼈、⽊、⽔}
  \definition*{s.}{Direitos Humanos}
  \seealsoref{人权}{ren2quan2}
\end{entry}

\begin{entry}{人群}{ren2 qun2}{2,13}{⼈、⽺}[HSK 3]
  \definition{s.}{multidão; ajuntamento; torpel; aglomeração; um grupo de pessoas}
\end{entry}

\begin{entry}{人生}{ren2sheng1}{2,5}{⼈、⽣}[HSK 3]
  \definition{s.}{vida (tempo de alguém na Terra)}
\end{entry}

\begin{entry}{人士}{ren2shi4}{2,3}{⼈、⼠}[HSK 5]
  \definition{s.}{pessoa; figura; personalidade; figura pública; pessoas com certa influência social}
\end{entry}

\begin{entry}{人数}{ren2 shu4}{2,13}{⼈、⽁}[HSK 2]
  \definition{s.}{número de pessoas}
\end{entry}

\begin{entry}{人物}{ren2wu4}{2,8}{⼈、⽜}[HSK 5]
  \definition[个,位,名]{s.}{personagem; personagens criados em obras literárias e artísticas | figura; personalidade; homem influente; refere-se a pessoas com grande talento e status; também se refere a pessoas com certas características ou que são representativas em algum aspecto | pintura figurativa; um tipo de pintura tradicional chinesa com personagens como tema}
\end{entry}

\begin{entry}{人像}{ren2xiang4}{2,13}{⼈、⼈}
  \definition{s.}{``retrato'' de uma pessoa (esboço, foto, escultura, etc.)}
\end{entry}

\begin{entry}{人行道}{ren2xing2dao4}{2,6,12}{⼈、⾏、⾡}
  \definition{s.}{calçada}
\end{entry}

\begin{entry}{人鱼}{ren2yu2}{2,8}{⼈、⿂}
  \definition{s.}{sereia | peixe-boi | salamandra gigante}
\end{entry}

\begin{entry}{人员}{ren2yuan2}{2,7}{⼈、⼝}[HSK 3]
  \definition[个,位,名]{s.}{funcionários | pessoal}
\end{entry}

\begin{entry}{儿}{ren2}{2}{⼉}
  \definition{s.}{pessoa, radical em caracteres chineses}
  \variantof{人}
  \seeref{儿}{er2}
  \seeref{儿}{r5}
\end{entry}

\begin{entry}{忍}{ren3}{7}{⼼}[HSK 5]
  \definition{v.}{suportar; aguentar; tolerar; aturar | ter coragem para; ser insensível o suficiente para; ser capaz de endurecer o coração e fazer coisas que não se devem fazer por uma questão de razão}
\end{entry}

\begin{entry}{忍不住}{ren3bu5zhu4}{7,4,7}{⼼、⼀、⼈}[HSK 5]
  \definition{v.}{incapaz de suportar; não conseguir evitar fazer algo; não conseguir se controlar}
\end{entry}

\begin{entry}{忍耐}{ren3nai4}{7,9}{⼼、⽽}
  \definition{s.}{paciência | resistência}
  \definition{v.}{suportar | resistir | exercer paciência}
\end{entry}

\begin{entry}{忍受}{ren3shou4}{7,8}{⼼、⼜}[HSK 5]
  \definition{v.}{suportar; sofrer; aguentar; tolerar; suportar com dificuldade o sofrimento, as dificuldades e as adversidades da vida}
\end{entry}

\begin{entry}{认}{ren4}{4}{⾔}[HSK 5]
  \definition{v.}{reconhecer; saber; distinguir; identificar | estabelecer uma determinada relação com; adotar | admitir; reconhecer; assumir | comprometer-se a fazer algo | (frequentemente seguido por 了) aceitar como inevitável; resignar-se}
  \seealsoref{了}{le5}
\end{entry}

\begin{entry}{认出}{ren4 chu1}{4,5}{⾔、⼐}[HSK 3]
  \definition{v.}{reconhecer; decifrar; identificar}
\end{entry}

\begin{entry}{认得}{ren4 de5}{4,11}{⾔、⼻}[HSK 3]
  \definition{v.}{saber; reconhecer}
\end{entry}

\begin{entry}{认定}{ren4ding4}{4,8}{⾔、⼧}[HSK 5]
  \definition{v.}{afirmar; manter; acreditar firmemente; considerar com certeza | decidir-se por algo; confirmar; chegar a uma conclusão afirmativa}
\end{entry}

\begin{entry}{认可}{ren4ke3}{4,5}{⾔、⼝}[HSK 3]
  \definition{v.}{aceitar; aprovar; confirmar; dar força legal a | permitir; concordar}
\end{entry}

\begin{entry}{认识}{ren4shi5}{4,7}{⾔、⾔}[HSK 1]
  \definition[份]{s.}{cognição; conhecimento; compreensão; refere-se à reflexão da mente humana sobre o mundo objetivo}
  \definition{v.}{saber; compreender; reconhecer}
\end{entry}

\begin{entry}{认为}{ren4wei2}{4,4}{⾔、⼂}[HSK 2]
  \definition{v.}{pensar | considerar | segurar | julgar}
\end{entry}

\begin{entry}{认真}{ren4zhen1}{4,10}{⾔、⼗}[HSK 1]
  \definition{adj.}{sério; sério e meticuloso}
  \definition{adv.}{seriamente}
  \definition{v.}{levar algo a sério; considerar como verdadeiro; levar a sério}
\end{entry}

\begin{entry}{任}{ren4}{6}{⼈}[HSK 3]
  \definition{clas.}{número de vezes que serviu em uma posição}
  \definition{conj.}{não importa (como, o que, etc.)}
  \definition{s.}{correio oficial; escritório}
  \definition{v.}{nomear; designar | assumir um posto; assumir um emprego | deixar; permitir; dar rédea solta a}
\end{entry}

\begin{entry}{任何}{ren4he2}{6,7}{⼈、⼈}[HSK 3]
  \definition{pron.}{qualquer; qualquer que seja; o que for}
\end{entry}

\begin{entry}{任务}{ren4wu5}{6,5}{⼈、⼒}[HSK 3]
  \definition[项,个]{s.}{tarefa; dever; missão; designação}
\end{entry}

\begin{entry}{扔}{reng1}{5}{⼿}[HSK 5]
  \definition{v.}{arremessar; lançar; atirar; jogar | esquecer; jogar fora; descartar | colocar casualmente; deixar as pessoas ou as coisas de lado, não se importar}
\end{entry}

\begin{entry}{扔掉}{reng1diao4}{5,11}{⼿、⼿}
  \definition{v.}{jogar fora}
\end{entry}

\begin{entry}{扔弃}{reng1qi4}{5,7}{⼿、⼶}
  \definition{v.}{abandonar | descartar | jogar fora}
\end{entry}

\begin{entry}{扔下}{reng1xia4}{5,3}{⼿、⼀}
  \definition{v.}{lançar (uma bomba) | derrubar}
\end{entry}

\begin{entry}{仍}{reng2}{4}{⼈}[HSK 3]
  \definition*{s.}{sobrenome Reng}
  \definition{adv.}{ainda}
  \definition{conj.}{por isso}
  \definition{v.}{permanecer}
\end{entry}

\begin{entry}{仍旧}{reng2jiu4}{4,5}{⼈、⽇}[HSK 5]
  \definition{adv.}{ainda; ainda assim; contudo}
  \definition{v.}{permanecer igual; continuar sendo}
\end{entry}

\begin{entry}{仍然}{reng2ran2}{4,12}{⼈、⽕}[HSK 3]
  \definition{adv.}{ainda; como antes}
\end{entry}

\begin{entry}{日}{ri4}{4}{⽇}[HSK 1][Kangxi 72]
  \definition*{s.}{Japão, abreviação de 日本}
  \definition{clas.}{usado para contar o número de dias}
  \definition{s.}{sol | dia (em oposição a 夜); período diurno | diariamente; todos os dias; a cada dia que passa | um dia específico; dia especial | tempo; refere-se a um período de tempo | dia; uma rotação da Terra}
  \seealsoref{日本}{ri4ben3}
  \seealsoref{夜}{ye4}
\end{entry}

\begin{entry}{日报}{ri4 bao4}{4,7}{⽇、⼿}[HSK 2]
  \definition[张]{s.}{diário | jornal diários}
\end{entry}

\begin{entry}{日本}{ri4ben3}{4,5}{⽇、⽊}
  \definition*{s.}{Japão}
\end{entry}

\begin{entry}{日本人}{ri4ben3ren2}{4,5,2}{⽇、⽊、⼈}
  \definition{s.}{japonês | pessoa ou povo do Japão}
\end{entry}

\begin{entry}{日常}{ri4chang2}{4,11}{⽇、⼱}[HSK 3]
  \definition{adv.}{usual; diário; cotidiano; dia a dia}
\end{entry}

\begin{entry}{日出}{ri4chu1}{4,5}{⽇、⼐}
  \definition{s.}{nascer do sol}
  \seealsoref{夕阳}{xi1yang2}
\end{entry}

\begin{entry}{日光灯}{ri4guang1deng1}{4,6,6}{⽇、⼉、⽕}
  \definition{s.}{lâmpada fluorescente}
\end{entry}

\begin{entry}{日记}{ri4ji4}{4,5}{⽇、⾔}[HSK 4]
  \definition[本,篇,册]{s.}{diário; artigo que registra eventos e pensamentos diários}
\end{entry}

\begin{entry}{日历}{ri4li4}{4,4}{⽇、⼚}[HSK 4]
  \definition[张,本]{s.}{caledário; livro com o ano, mês, dia, semana, termo solar, aniversário, etc. registrados, um livro por ano, uma página por dia, aberto diariamente}
\end{entry}

\begin{entry}{日期}{ri4qi1}{4,12}{⽇、⽉}[HSK 1]
  \definition[个,段]{s.}{data; a data ou período específico em que algo aconteceu}
\end{entry}

\begin{entry}{日心说}{ri4 xin1 shuo1}{4,4,9}{⽇、⼼、⾔}
  \definition{s.}{teoria heliocêntrica | a teoria de que o sol está no centro do universo}
\end{entry}

\begin{entry}{日子}{ri4zi5}{4,3}{⽇、⼦}[HSK 2]
  \definition{s.}{dia | uma data (calendário) | dias de vida de alguém}
\end{entry}

\begin{entry}{容貌}{rong2mao4}{10,14}{⼧、⾘}
  \definition{s.}{aparência | aspecto | características}
\end{entry}

\begin{entry}{容易}{rong2yi4}{10,8}{⼧、⽇}[HSK 3]
  \definition{adj.}{fácil; simples | provável; responsável; apto}
\end{entry}

\begin{entry}{柔软}{rou2ruan3}{9,8}{⽊、⾞}
  \definition{adj.}{macio | suave}
\end{entry}

\begin{entry}{揉}{rou2}{12}{⼿}
  \definition{v.}{amassar | massagear | esfregar}
\end{entry}

\begin{entry}{揉碎}{rou2sui4}{12,13}{⼿、⽯}
  \definition{v.}{esmagar | desintegrar-se em pedaços}
\end{entry}

\begin{entry}{肉}{rou4}{6}{⾁}[HSK 1][Kangxi 130]
  \definition{adj.}{não crocante; mole | lento (em movimento); preguiçoso | carnal; erótico}
  \definition[块]{s.}{carne (especialmente carne de porco) | carne | polpa (da fruta)}
\end{entry}

\begin{entry}{肉桂}{rou4gui4}{6,10}{⾁、⽊}
  \definition{s.}{canela}
  \seealsoref{官桂}{guan1gui4}
\end{entry}

\begin{entry}{如}{ru2}{6}{⼥}
  \definition{conj.}{por exemplo}
\end{entry}

\begin{entry}{如此}{ru2 ci3}{6,6}{⼥、⽌}[HSK 5]
  \definition{adv.}{assim; tal; dessa forma; dessa maneira; refere-se a uma situação mencionada anteriormente, equivalente a 这样}
  \seealsoref{这样}{zhe4 yang4}
\end{entry}

\begin{entry}{如果}{ru2guo3}{6,8}{⼥、⽊}[HSK 2]
  \definition{conj.}{se | caso | no caso de | no evento de | supondo que}
\end{entry}

\begin{entry}{如何}{ru2he2}{6,7}{⼥、⼈}[HSK 3]
  \definition{adv.}{como?; o que?}
\end{entry}

\begin{entry}{如画}{ru2hua4}{6,8}{⼥、⽥}
  \definition{adj.}{pitoresco}
\end{entry}

\begin{entry}{如今}{ru2jin1}{6,4}{⼥、⼈}[HSK 4]
  \definition{s.}{agora; hoje em dia; atualmente; no presente}
\end{entry}

\begin{entry}{如同}{ru2 tong2}{6,6}{⼥、⼝}[HSK 5]
  \definition{v.}{parecer que. usado principalmente em metáforas}
\end{entry}

\begin{entry}{如下}{ru2 xia4}{6,3}{⼥、⼀}[HSK 5]
  \definition{adv.}{como descrito ou listado abaixo; conforme segue; conforme abaixo}
\end{entry}

\begin{entry}{儒教}{ru2jiao4}{16,11}{⼈、⽁}
  \definition*{s.}{Confucionismo}
\end{entry}

\begin{entry}{乳房}{ru3fang2}{8,8}{⼄、⼾}
  \definition{s.}{seio | mama | úbere}
\end{entry}

\begin{entry}{辱骂}{ru3ma4}{10,9}{⾠、⾺}
  \definition{v.}{insultar | abusar}
\end{entry}

\begin{entry}{入党}{ru4dang3}{2,10}{⼊、⼉}
  \definition{v.}{ingressar em um partido político (especialmente o partido comunista)}
\end{entry}

\begin{entry}{入境}{ru4jing4}{2,14}{⼊、⼟}
  \definition{s.}{imigração}
  \definition{v.+compl.}{entrar em um país | imigrar}
\end{entry}

\begin{entry}{入口}{ru4kou3}{2,3}{⼊、⼝}[HSK 2]
  \definition[个]{s.}{entrada | enseada}
  \definition{v.}{entra na boca | importar}
\end{entry}

\begin{entry}{入门}{ru4 men2}{2,3}{⼊、⾨}[HSK 5]
  \definition{s.}{(geralmente em títulos de livros) curso básico; manual introdutório | ABC; guia; refere-se a leituras básicas; conhecimentos básicos}
  \definition{v.+compl.}{ultrapassar o limiar; aprender os rudimentos de um assunto | aprender o ABC de; ser introduzido a um assunto; aprender o básico}
\end{entry}

\begin{entry}{入乡随俗}{ru4xiang1-sui2su2}{2,3,11,9}{⼊、⼄、⾩、⼈}
  \definition{expr.}{Em roma, faça como os romanos!}
\end{entry}

\begin{entry}{软}{ruan3}{8}{⾞}[HSK 5]
  \definition*{s.}{sobrenome Ruan}
  \definition{adj.}{macio; flexível; maleável; maleável (oposto de 硬) | suave; brando; delicado | fraco; débil | de baixa qualidade, capacidade, etc. | facilmente movido (ou influenciado) | de maneira suave (ou gentil) | indulgente; tolerante | maleável; flexível | fácil de se emocionar ou abalar}
  \seealsoref{硬}{ying4}
\end{entry}

\begin{entry}{软件}{ruan3jian4}{8,6}{⾞、⼈}[HSK 5]
  \definition[款,个]{v.}{(computador) \emph{software}; programas de computador, procedimentos, regras e quaisquer arquivos, documentos e dados relacionados à operação do sistema de computador}
\end{entry}

\begin{entry}{锐}{rui4}{12}{⾦}
  \definition*{s.}{sobrenome Rui}
  \definition{adj.}{afiado; aguçado (oposto a 钝) | agudo; perspicaz | rápido; ágil; veloz}
  \definition{adv.}{rapidamente; de ​​repente}
  \definition{s.}{vigor; espírito de luta | armas afiadas}
  \seealsoref{钝}{dun4}
\end{entry}

\begin{entry}{若}{ruo4}{8}{⾋}
  \definition*{s.}{sobrenome Ruo}
  \definition{adv.}{como se; como se fosse; usado antes do verbo para indicar que o que foi dito é mais ou menos assim, equivalente a 好像}
  \definition{conj.}{se; usado na primeira parte de uma frase composta, expressa uma relação hipotética, equivalente a 如果}
  \definition{pron.}{você; referir-se ao interlocutor como 你 ou 你的}
  \definition{v.}{parecer}
  \seealsoref{好像}{hao3xiang4}
  \seealsoref{你}{ni3}
  \seealsoref{你的}{ni3 de5}
  \seealsoref{如果}{ru2guo3}
\end{entry}

\begin{entry}{弱}{ruo4}{10}{⼸}[HSK 4]
  \definition{adj.}{fraco; debilitado | jovem | inferior; pior | colocado depois de uma fração ou decimal para indicar que é um pouco menor que esse número}
  \definition{v.}{perder (através da morte)}
\end{entry}

%%%%% EOF %%%%%

