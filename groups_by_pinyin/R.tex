%%%
%%% R
%%%

\section*{R}\addcontentsline{toc}{section}{R}

\begin{entry}{儿}{r5}{2}[Radical 儿]
  \definition{suf.}{sufixo diminutivo não silábico | final retroflexo}
  \seeref{儿}{er2}
  \seeref{儿}{ren2}
\end{entry}

\begin{entry}{然}{ran2}{12}[Radical 火]
  \definition{conj.}{mas | no entanto}
\end{entry}

\begin{entry}{然而}{ran2'er2}{12,6}
  \definition{conj.}{mas | no entanto}
\end{entry}

\begin{entry}{然后}{ran2hou4}{12,6}[HSK 2]
  \definition{conj.}{depois | logo | portanto}
\end{entry}

\begin{entry}{燃烧}{ran2shao1}{16,10}
  \definition{s.}{combustão | flama}
  \definition{v.}{queimar | acender}
\end{entry}

\begin{entry}{壤}{rang3}{20}[Radical 土]
  \definition{s.}{solo | terra | (literário) a terra (em contraste com o céu 天)}
\end{entry}

\begin{entry}{让}{rang4}{5}[Radical 言][HSK 2]
  \definition{v.}{deixar alguém fazer alguma coisa |fazer alguém (sentir-se triste, etc.) | permitir | conceder}
\end{entry}

\begin{entry}{让步}{rang4bu4}{5,7}
  \definition{v.+compl.}{fazer uma concessão | entregar | desistir | comprometer}
\end{entry}

\begin{entry}{热}{re4}{10}[Radical 火][HSK 1]
  \definition{adj.}{quente (clima) | fervente | ardente | fervoroso}
  \definition{v.}{aquecer | ferver}
\end{entry}

\begin{entry}{热爱}{re4'ai4}{10,10}
  \definition{v.}{amar ardentemente | adorar}
\end{entry}

\begin{entry}{热泪盈眶}{re4lei4ying2kuang4}{10,8,9,11}
  \definition{expr.}{olhos cheios de lágrimas de emoção | extremamente emocionado}
\end{entry}

\begin{entry}{热闹}{re4nao5}{10,8}
  \definition{adj.}{animado | movimentado com barulho e excitação}
\end{entry}

\begin{entry}{热情}{re4qing2}{10,11}[HSK 2]
  \definition{adj.}{caloroso | fervoroso | entusiasmado}
  \definition{s.}{entusiasmo | ardor | devoção | calor | zelo}
\end{entry}

\begin{entry}{热心}{re4xin1}{10,4}
  \definition{adj.}{entusiasmado | ardente | zeloso}
\end{entry}

\begin{entry}{热血沸腾}{re4xue4fei4teng2}{10,6,8,13}
  \definition{expr.}{ferver o sangue | apaixonar-se}
\end{entry}

\begin{entry}{人}{ren2}{2}[Radical 人][Kangxi 9][HSK 1]
  \definition[个,位]{s.}{pessoa | gente}
\end{entry}

\begin{entry}{人才}{ren2cai2}{2,3}
  \definition{s.}{talento | pessoa talentosa}
\end{entry}

\begin{entry}{人材}{ren2cai2}{2,7}
  \variantof{人才}
\end{entry}

\begin{entry}{人道}{ren2dao4}{2,12}
  \definition{s.}{solidariedade humana | humanitarismo | humano | a ``maneira humana'', um dos estágios do ciclo de reencarnação (budismo) | relação sexual}
\end{entry}

\begin{entry}{人海}{ren2hai3}{2,10}
  \definition{s.}{uma multidão | um mar de pessoas}
\end{entry}

\begin{entry}{人间}{ren2jian1}{2,7}
  \definition{s.}{o mundo humano | a Terra}
\end{entry}

\begin{entry}{人口}{ren2kou3}{2,3}[HSK 2]
  \definition{s.}{pessoas | população}
\end{entry}

\begin{entry}{人类}{ren2lei4}{2,9}
  \definition{s.}{humanidade | raça humana}
\end{entry}

\begin{entry}{人们}{ren2 men5}{2,5}[HSK 2]
  \definition{s.}{homens |  pessoas | o público}
\end{entry}

\begin{entry}{人民}{ren2min2}{2,5}
  \definition[个]{s.}{povo | população}
\end{entry}

\begin{entry}{人民币}{ren2min2bi4}{2,5,4}
  \definition*{s.}{Renminbi (RMB) | Yuan Chinês (CYN) | nome da moeda chinesa}
\end{entry}

\begin{entry}{人权}{ren2quan2}{2,6}
  \definition*{s.}{Direitos Humanos}
  \seealsoref{人权法}{ren2quan2fa3}
\end{entry}

\begin{entry}{人权法}{ren2quan2fa3}{2,6,8}
  \definition*{s.}{Direitos Humanos}
  \seealsoref{人权}{ren2quan2}
\end{entry}

\begin{entry}{人生}{ren2sheng1}{2,5}
  \definition{s.}{vida (tempo de alguém na Terra)}
\end{entry}

\begin{entry}{人数}{ren2 shu4}{2,13}[HSK 2]
  \definition{s.}{número de pessoas}
\end{entry}

\begin{entry}{人像}{ren2xiang4}{2,13}
  \definition{s.}{``retrato'' de uma pessoa (esboço, foto, escultura, etc.)}
\end{entry}

\begin{entry}{人行道}{ren2xing2dao4}{2,6,12}
  \definition{s.}{calçada}
\end{entry}

\begin{entry}{人鱼}{ren2yu2}{2,8}
  \definition{s.}{sereia | peixe-boi | salamandra gigante}
\end{entry}

\begin{entry}{儿}{ren2}{2}[Radical 儿]
  \definition{s.}{pessoa, radical em caracteres chineses}
  \variantof{人}
  \seeref{儿}{er2}
  \seeref{儿}{r5}
\end{entry}

\begin{entry}{忍耐}{ren3nai4}{7,9}
  \definition{s.}{paciência | resistência}
  \definition{v.}{suportar | resistir | exercer paciência}
\end{entry}

\begin{entry}{认识}{ren4shi5}{4,7}[HSK 1]
  \definition{s.}{conhecimento | saber | entendimento}
  \definition{v.}{estar familiarizado com | conhecer alguém | saber | reconhecer}
\end{entry}

\begin{entry}{认为}{ren4wei2}{4,4}[HSK 2]
  \definition{v.}{pensar | considerar | segurar | julgar}
\end{entry}

\begin{entry}{认真}{ren4zhen1}{4,10}[HSK 1]
  \definition{adj.}{sério | consciencioso}
  \definition{adv.}{seriamente}
  \definition{v.}{levar a sério}
\end{entry}

\begin{entry}{任务}{ren4wu5}{6,5}
  \definition[项,个]{s.}{missão | atribuição | tarefa | obrigação | papel}
\end{entry}

\begin{entry}{扔}{reng1}{5}[Radical 手]
  \definition{v.}{lançar | atirar}
\end{entry}

\begin{entry}{扔掉}{reng1diao4}{5,11}
  \definition{v.}{jogar fora}
\end{entry}

\begin{entry}{扔弃}{reng1qi4}{5,7}
  \definition{v.}{abandonar | descartar | jogar fora}
\end{entry}

\begin{entry}{扔下}{reng1xia4}{5,3}
  \definition{v.}{lançar (uma bomba) | derrubar}
\end{entry}

\begin{entry}{仍然}{reng2ran2}{4,12}
  \definition{adv.}{ainda}
\end{entry}

\begin{entry}{日}{ri4}{4}[Radical 日][Kangxi 72][HSK 1]
  \definition*{s.}{Japão, abreviação de~日本}
  \definition{clas.}{dia (mais usado em escrita) | data, dia do mês}
  \seeref{日本}{ri4ben3}
\end{entry}

\begin{entry}{日报}{ri4 bao4}{4,7}[HSK 2]
  \definition[张]{s.}{diário | jornal diários}
\end{entry}

\begin{entry}{日本}{ri4ben3}{4,5}
  \definition*{s.}{Japão}
\end{entry}

\begin{entry}{日本人}{ri4ben3ren2}{4,5,2}
  \definition{s.}{japonês | pessoa ou povo do Japão}
\end{entry}

\begin{entry}{日常}{ri4chang2}{4,11}
  \definition{adv.}{diariamente | dia-a-dia | todo dia}
\end{entry}

\begin{entry}{日出}{ri4chu1}{4,5}
  \definition{s.}{nascer do sol}
  \seealsoref{夕阳}{xi1yang2}
\end{entry}

\begin{entry}{日光灯}{ri4guang1deng1}{4,6,6}
  \definition{s.}{lâmpada fluorescente}
\end{entry}

\begin{entry}{日期}{ri4qi1}{4,12}[HSK 1]
  \definition{s.}{data}
\end{entry}

\begin{entry}{日子}{ri4zi5}{4,3}[HSK 2]
  \definition{s.}{dia | uma data (calendário) | dias de vida de alguém}
\end{entry}

\begin{entry}{容貌}{rong2mao4}{10,14}
  \definition{s.}{aparência | aspecto | características}
\end{entry}

\begin{entry}{容易}{rong2yi4}{10,8}
  \definition{adj.}{fácil | responsável (por) | provável}
\end{entry}

\begin{entry}{柔软}{rou2ruan3}{9,8}
  \definition{adj.}{macio | suave}
\end{entry}

\begin{entry}{揉}{rou2}{12}[Radical 手]
  \definition{v.}{amassar | massagear | esfregar}
\end{entry}

\begin{entry}{揉碎}{rou2sui4}{12,13}
  \definition{v.}{esmagar | desintegrar-se em pedaços}
\end{entry}

\begin{entry}{肉}{rou4}{6}[Radical 肉][Kangxi 130][HSK 1]
  \definition{s.}{carne | polpa de uma fruta}
\end{entry}

\begin{entry}{肉桂}{rou4gui4}{6,10}
  \definition{s.}{canela}
  \seealsoref{官桂}{guan1gui4}
\end{entry}

\begin{entry}{如}{ru2}{6}[Radical 女]
  \definition{conj.}{por exemplo}
\end{entry}

\begin{entry}{如此}{ru2ci3}{6,6}
  \definition{adv.}{assim | então | tal}
\end{entry}

\begin{entry}{如果}{ru2guo3}{6,8}[HSK 2]
  \definition{conj.}{se | caso | no caso de | no evento de | supondo que}
\end{entry}

\begin{entry}{如画}{ru2hua4}{6,8}
  \definition{adj.}{pitoresco}
\end{entry}

\begin{entry}{儒教}{ru2jiao4}{16,11}
  \definition*{s.}{Confucionismo}
\end{entry}

\begin{entry}{乳房}{ru3fang2}{8,8}
  \definition{s.}{seio | mama | úbere}
\end{entry}

\begin{entry}{辱骂}{ru3ma4}{10,9}
  \definition{v.}{insultar | abusar}
\end{entry}

\begin{entry}{入党}{ru4dang3}{2,10}
  \definition{v.}{ingressar em um partido político (especialmente o partido comunista)}
\end{entry}

\begin{entry}{入境}{ru4jing4}{2,14}
  \definition{s.}{imigração}
  \definition{v.+compl.}{entrar em um país | imigrar}
\end{entry}

\begin{entry}{入口}{ru4kou3}{2,3}[HSK 2]
  \definition[个]{s.}{entrada | enseada}
  \definition{v.}{entra na boca | importar}
\end{entry}

\begin{entry}{入门}{ru4men2}{2,3}
  \definition{s.}{curso elementar | ABC | guia}
  \definition{v.+compl.}{atravessar o limiar | aprender o ABC de | introduzir um assunto | aprender os rudimentos de um assunto}
\end{entry}

\begin{entry}{入乡随俗}{ru4xiang1-sui2su2}{2,3,11,9}
  \definition{expr.}{Em roma, faça como os romanos!}
\end{entry}

\begin{entry}{软件}{ruan3jian4}{8,6}
  \definition{v.}{\emph{software}}
\end{entry}

%%%%% EOF %%%%%

