%%%
%%% E
%%%

\section*{E}\addcontentsline{toc}{section}{E}

\begin{entry}{阿}{e1}{7}{⾩}
  \definition*{s.}{Dong'e, um condado na província de Shandong | Sobrenome E}
  \definition{s.}{grande monte (ou colina) | um lugar sinuoso (montanha, água, etc.)}
  \definition{v.}{bajular; satisfazer}
  \seeref{阿}{a1}
\end{entry}

\begin{entry}{俄}{e2}{9}{⼈}
  \definition*{s.}{Rússia, abreviação de 俄罗斯}
  \definition{adv.}{muito em breve; em breve; de repente}
  \seealsoref{俄罗斯}{e2luo2si1}
\end{entry}

\begin{entry}{俄罗斯}{e2luo2si1}{9,8,12}{⼈、⽹、⽄}
  \definition*{s.}{Rússia}
\end{entry}

\begin{entry}{俄罗斯人}{e2luo2si1ren2}{9,8,12,2}{⼈、⽹、⽄、⼈}
  \definition{s.}{russo | pessoa ou povo da Rússia}
\end{entry}

\begin{entry}{哦}{e2}{10}{⼝}
  \definition{v.}{cantar suavemente (um poema)}
  \seeref{哦}{o2}
  \seeref{哦}{o4}
  \seeref{哦}{o5}
\end{entry}

\begin{entry}{恶}{e3}{10}{⼼}
  \definition{part.}{elementos formadores de palavras}
  \seeref{恶}{e4}
  \seeref{恶}{wu4}
\end{entry}

\begin{entry}{恶心}{e3xin1}{10,4}{⼼、⼼}[HSK 4]
  \definition{adj.}{nauseante; repugnante}
  \definition{s.}{enjoo; náusea; repugnância; sensação de enjoo; vontade de vomitar}
  \definition{v.}{repugnar; ser nauseante; vomitar}
  \seeref{恶心}{e4xin1}
\end{entry}

\begin{entry}{恶}{e4}{10}{⼼}
  \definition{adj.}{feroz | ruim; maligno; perverso | vicioso | feio | grosseiro}
  \definition{s.}{mal; vício; crime (oposto a 善) | maldade; comportamento muito ruim; coisas criminosas}
  \seeref{恶}{e3}
  \seeref{恶}{e4}
  \seealsoref{善}{shan4}
\end{entry}

\begin{entry}{恶心}{e4xin1}{10,4}{⼼、⼼}
  \definition{s.}{mau hábito | hábito vicioso | vício}
  \seeref{恶心}{e3xin1}
\end{entry}

\begin{entry}{饿}{e4}{10}{⾷}[HSK 1]
  \definition{adj.}{faminto}
  \definition{v.}{passar fome; causar fome}
\end{entry}

\begin{entry}{鳄}{e4}{17}{⿂}
  \definition{s.}{crocodilo;  jacaré}
\end{entry}

\begin{entry}{鳄鱼}{e4yu2}{17,8}{⿂、⿂}
  \definition[条]{s.}{jacaré | crocodilo}
\end{entry}

\begin{entry}{恩}{en1}{10}{⼼}
  \definition*{s.}{Sobrenome En}
  \definition{s.}{bondade; favor; graça; gentileza}
\end{entry}

\begin{entry}{恩赐}{en1ci4}{10,12}{⼼、⾙}
  \definition{s.}{favor | caridade}
  \definition{v.}{conceder (favor, caridade)}
\end{entry}

\begin{entry}{恩人}{en1 ren2}{10,2}{⼼、⼈}[HSK 6]
  \definition{s.}{benfeitor; uma pessoa que ajudou significativamente alguém}
\end{entry}

\begin{entry}{儿}{er2}{2}{⼉}[Kangxi 10]
  \definition{adj.}{macho}
  \definition{s.}{criança | jovem; juventude | filho}
  \definition{suf.}{adicionado a substantivos para expressar pequenez  | adicionado a verbos, adjetivos e classificadores para formar substantivos | adicionado a substantivos para formar substantivos com significados diferentes | sufixos de alguns verbos | anexado após adjetivos duplicados}
  \seeref{儿}{r5}
\end{entry}

\begin{entry}{儿科}{er2 ke1}{2,9}{⼉、⽲}[HSK 6]
  \definition{s.}{(departamento de) pediatria | pediatria; o ramo da medicina que trata do desenvolvimento, cuidado e doença das crianças}
\end{entry}

\begin{entry}{儿女}{er2 nv3}{2,3}{⼉、⼥}[HSK 5]
  \definition{s.}{crianças; filhos e filhas | homem e mulher jovens (apaixonados)}
\end{entry}

\begin{entry}{儿童}{er2tong2}{2,12}{⼉、⽴}[HSK 4]
  \definition[个,群]{s.}{criança; menor de idade (mais jovem do que 少年)}
  \seealsoref{少年}{shao4 nian2}
\end{entry}

\begin{entry}{儿媳}{er2xi2}{2,13}{⼉、⼥}
  \definition{s.}{esposa do filho}
\end{entry}

\begin{entry}{儿子}{er2zi5}{2,3}{⼉、⼦}[HSK 1]
  \definition[个]{s.}{filho}
  \seealsoref{女儿}{nv3'er2}
\end{entry}

\begin{entry}{而}{er2}{6}{⽽}[HSK 4][Kangxi 126]
  \definition{conj.}{e (coordenação) | e ainda (restrição) | conexão de componentes com continuidade semântica | conecxão de componentes afirmativos e negativos que se complementam | conexão de componentes com significados opostos para indicar um contraste |  conexão de componentes de causa e efeito no raciocínio | significa “chegar” ou “alcançar” | conexão de componentes que indicam tempo ou modo ao verbo | inserido entre o sujeito e o predicado, significa 如果}
  \seealsoref{如果}{ru2guo3}
\end{entry}

\begin{entry}{而况}{er2kuang4}{6,7}{⽽、⼎}
  \definition{conj.}{além disso | além do mais}
\end{entry}

\begin{entry}{而且}{er2 qie3}{6,5}{⽽、⼀}[HSK 2]
  \definition{conj.}{e também; indica igualdade | e isso; não só\dots mas (também); indica um passo adiante}
\end{entry}

\begin{entry}{而是}{er2 shi4}{6,9}{⽽、⽇}[HSK 4]
  \definition{conj.}{mas; em vez disso; geralmente usada em conjunto com 不是 para formar o correlativo 不是……而是, indicando uma relação paralela}
  \seealsoref{不是……而是}{bu4shi4 er2 shi4}
\end{entry}

\begin{entry}{耳}{er3}{6}{⽿}[Kangxi 128]
  \definition*{s.}{Sobrenome Er}
  \definition{part.}{(clássico) somente; apenas}
  \definition{s.}{orelha | coisa parecida com uma orelha | em ambos os lados; lado | orelha de um utensílio}
\end{entry}

\begin{entry}{耳朵}{er3duo5}{6,6}{⽿、⽊}[HSK 5]
  \definition[双,只,个,对]{s.}{orelha; ouvido; órgão da audição e do equilíbrio}
\end{entry}

\begin{entry}{耳机}{er3 ji1}{6,6}{⽿、⽊}[HSK 4]
  \definition[副,个,对]{s.}{fone de ouvido; receptor (de telefone); dispositivos que permitem que uma pessoa ouça sons sozinha, como ouvir música, histórias, chamadas telefônicas etc., usados na cabeça ou inseridos nos ouvidos}
\end{entry}

\begin{entry}{二}{er4}{2}{⼆}[HSK 1][Kangxi 7]
  \definition{adj.}{diferente; refere-se a duas coisas ou coisas diferentes | bobo; pateta; tolo; sem inteligência | desleal; infiel; indiferente; sem determinação}
  \definition{num.}{dois; 2}
\end{entry}

\begin{entry}{二胡}{er4hu2}{2,9}{⼆、⾁}
  \definition{s.}{erhu; um instrumento de arco de duas cordas com um registro mais baixo que o 京胡; um tipo de 胡琴, a caixa de som é feita de bambu, madeira, etc., coberta com pele de cobra, etc., tem duas cordas e o tom é baixo e suave}
  \seealsoref{胡琴}{hu2qin2}
  \seealsoref{京胡}{jing1hu2}
\end{entry}

\begin{entry}{二手}{er4 shou3}{2,4}{⼆、⼿}[HSK 4]
  \definition{adj.}{usado; de segunda mão; refere-se especificamente a usados e revendidos}
\end{entry}

\begin{entry}{二维码}{er4 wei2 ma3}{2,11,8}{⼆、⽷、⽯}[HSK 5]
  \definition{s.}{\emph{QR code}}
\end{entry}

\begin{entry}{二战}{er4zhan4}{2,9}{⼆、⼽}
  \definition*{s.}{Segunda Guerra Mundial}
\end{entry}

%%%%% EOF %%%%%

