%%%
%%% E
%%%

\section*{E}\addcontentsline{toc}{section}{E}

\begin{entry}{阿}{e1}{7}{⾩}
  \definition*{s.}{Dong'e (um condado na província de Shandong) | sobrenome E}
  \definition{s.}{grande monte (ou colina) | um lugar curvo (montanha, água, etc.)}
  \definition{v.}{fazer-se passar por; agradar;}
  \seeref{阿}{a1}
\end{entry}

\begin{entry}{俄}{e2}{9}{⼈}
  \definition*{s.}{Rússia, abreviação de 俄罗斯}
  \seealsoref{俄罗斯}{e2luo2si1}
\end{entry}

\begin{entry}{俄罗斯}{e2luo2si1}{9,8,12}{⼈、⽹、⽄}
  \definition*{s.}{Rússia}
\end{entry}

\begin{entry}{俄罗斯人}{e2luo2si1ren2}{9,8,12,2}{⼈、⽹、⽄、⼈}
  \definition{s.}{russo | pessoa ou povo da Rússia}
\end{entry}

\begin{entry}{哦}{e2}{10}{⼝}
  \definition{v.}{entoar cântico}
  \seeref{哦}{o2}
  \seeref{哦}{o4}
  \seeref{哦}{o5}
\end{entry}

\begin{entry}{恶心}{e3xin1}{10,4}{⼼、⼼}[HSK 4]
  \definition{adj.}{nauseante; repugnante}
  \definition{s.}{enjoo; náusea; repugnância; sensação de enjoo; vontade de vomitar}
  \definition{v.}{repugnar; ser nauseante; vomitar}
  \seeref{恶心}{e4xin1}
\end{entry}

\begin{entry}{恶心}{e4xin1}{10,4}{⼼、⼼}
  \definition{s.}{mau hábito | hábito vicioso | vício}
  \seeref{恶心}{e3xin1}
\end{entry}

\begin{entry}{饿}{e4}{10}{⾷}[HSK 1]
  \definition{adj.}{faminto}
  \definition{s.}{fome}
  \definition{v.}{morrer de fome}
\end{entry}

\begin{entry}{鳄鱼}{e4yu2}{17,8}{⿂、⿂}
  \definition[条]{s.}{jacaré | crocodilo}
\end{entry}

\begin{entry}{恩赐}{en1ci4}{10,12}{⼼、⾙}
  \definition{s.}{favor | caridade}
  \definition{v.}{conceder (favor, caridade)}
\end{entry}

\begin{entry}{儿}{er2}{2}{⼉}[Kangxi 10]
  \definition{s.}{criança | filho}
  \seeref{儿}{r5}
  \seeref{儿}{ren2}
\end{entry}

\begin{entry}{儿童}{er2tong2}{2,12}{⼉、⽴}[HSK 4]
  \definition[个,群]{s.}{criança; menor de idade (mais jovem do que ``少年'')}
  \seealsoref{少年}{shao4 nian2}
\end{entry}

\begin{entry}{儿媳}{er2xi2}{2,13}{⼉、⼥}
  \definition{s.}{esposa do filho}
\end{entry}

\begin{entry}{儿子}{er2zi5}{2,3}{⼉、⼦}
  \definition{s.}{filho}
  \seealsoref{女儿}{nv3'er2}
\end{entry}

\begin{entry}{而}{er2}{6}{⽽}[HSK 4][Kangxi 126]
  \definition{conj.}{e (coordenação) | e ainda (restrição) | conexão de componentes com continuidade semântica | conecxão de componentes afirmativos e negativos que se complementam | conexão de componentes com significados opostos para indicar um contraste |  conexão de componentes de causa e efeito no raciocínio | significa “chegar” ou “alcançar” | conexão de componentes que indicam tempo ou modo ao verbo | inserido entre o sujeito e o predicado, significa ``如果'' (se)}
  \seealsoref{如果}{ru2guo3}
\end{entry}

\begin{entry}{而况}{er2kuang4}{6,7}{⽽、⼎}
  \definition{conj.}{além disso | além do mais}
\end{entry}

\begin{entry}{而且}{er2 qie3}{6,5}{⽽、⼀}[HSK 2]
  \definition{conj.}{muito menos | além disso | além do mais}
\end{entry}

\begin{entry}{而是}{er2 shi4}{6,9}{⽽、⽇}[HSK 4]
  \definition{conj.}{mas; em vez disso; geralmente usada em conjunto com ``不是'' para formar o correlativo ``不是……而是'', indicando uma relação paralela}
  \seealsoref{不是……而是}{bu4shi4 er2 shi4}
\end{entry}

\begin{entry}{耳朵}{er3duo5}{6,6}{⽿、⽊}
  \definition[只,个,对]{s.}{orelha}
\end{entry}

\begin{entry}{耳机}{er3 ji1}{6,6}{⽿、⽊}[HSK 4]
  \definition[副,个,对]{s.}{fone de ouvido; receptor (de telefone); dispositivos que permitem que uma pessoa ouça sons sozinha, como ouvir música, histórias, chamadas telefônicas etc., usados na cabeça ou inseridos nos ouvidos}
\end{entry}

\begin{entry}{二}{er4}{2}{⼆}[HSK 1][Kangxi 7]
  \definition{num.}{dois; 2 | (dialeto de Pequim) estúpido}
\end{entry}

\begin{entry}{二手}{er4 shou3}{2,4}{⼆、⼿}[HSK 4]
  \definition{adj.}{usado; de segunda mão; refere-se especificamente a usados e revendidos}
\end{entry}

\begin{entry}{二战}{er4zhan4}{2,9}{⼆、⼽}
  \definition*{s.}{Segunda Guerra Mundial}
\end{entry}

%%%%% EOF %%%%%

