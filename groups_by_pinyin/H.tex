%%%
%%% H
%%%

\section*{H}\addcontentsline{toc}{section}{H}

\begin{entry}{哈哈}{ha1 ha1}{9,9}{⼝、⼝}[HSK 3]
  \definition{expr.}{(onomatopéia)  ha ha; o som de uma risada alta}
\end{entry}

\begin{entry}{哈马斯}{ha1ma3si1}{9,3,12}{⼝、⾺、⽄}
  \definition*{s.}{Hamas (Grupo Palestino)}
\end{entry}

\begin{entry}{还}{hai2}{7}{⾡}[HSK 1]
  \definition{adv.}{ainda | também | ainda mais | razoavelmente | bastante}
  \seeref{还}{huan2}
\end{entry}

\begin{entry}{还是}{hai2shi5}{7,9}{⾡、⽇}[HSK 1]
  \definition{adv.}{ainda (como antes) | inesperadamente | teve melhor}
  \definition{conj.}{ou (somente para frases interrogativas)}
\end{entry}

\begin{entry}{还有}{hai2 you3}{7,6}{⾡、⽉}[HSK 1]
  \definition{adv.}{além do mais | além disso | ainda permanece | ainda há}
\end{entry}

\begin{entry}{孩子}{hai2zi5}{9,3}{⼦、⼦}[HSK 1]
  \definition{s.}{criança | filho}
\end{entry}

\begin{entry}{海}{hai3}{10}{⽔}[HSK 2]
  \definition*{s.}{sobrenome Hai}
  \definition[个,片]{s.}{mar | oceano}
\end{entry}

\begin{entry}{海边}{hai3 bian1}{10,5}{⽔、⾡}[HSK 2]
  \definition{s.}{costa marítima | litoral | beira-mar | praia}
\end{entry}

\begin{entry}{海底}{hai3di3}{10,8}{⽔、⼴}
  \definition{adj.}{submarino}
  \definition{s.}{fundo do mar | solo oceânico | fundo do oceano}
\end{entry}

\begin{entry}{海风}{hai3feng1}{10,4}{⽔、⾵}
  \definition{s.}{brisa do mar | vento que vem do mar}
\end{entry}

\begin{entry}{海关}{hai3guan1}{10,6}{⽔、⼋}[HSK 3]
  \definition{s.}{alfândega}
\end{entry}

\begin{entry}{海浪}{hai3lang4}{10,10}{⽔、⽔}
  \definition{s.}{ondas do mar}
\end{entry}

\begin{entry}{海里}{hai3li3}{10,7}{⽔、⾥}
  \definition{s.}{milha náutica}
\end{entry}

\begin{entry}{海绵}{hai3mian2}{10,11}{⽔、⽷}
  \definition{s.}{(zoologia) esponja do mar | esponja (feita de poliéster ou celulose, etc.) | espuma de borracha}
\end{entry}

\begin{entry}{海鸥}{hai3'ou1}{10,9}{⽔、⿃}
  \definition{s.}{gaivota}
\end{entry}

\begin{entry}{海水}{hai3 shui3}{10,4}{⽔、⽔}[HSK 4]
  \definition[把]{s.}{água do mar; salmoura}
\end{entry}

\begin{entry}{海棠}{hai3tang2}{10,12}{⽔、⽊}
  \definition{s.}{begônia}
\end{entry}

\begin{entry}{海鲜}{hai3xian1}{10,14}{⽔、⿂}[HSK 4]
  \definition[种,份,桌,批,些]{s.}{frutos do mar; mariscos; peixes marinhos frescos, camarões, etc., para consumo |}
\end{entry}

\begin{entry}{害}{hai4}{10}{⼧}
  \definition{s.}{dano | mal | calamidade}
  \definition{v.}{causar danos a | causar problemas para}
\end{entry}

\begin{entry}{害怕}{hai4pa4}{10,8}{⼧、⼼}[HSK 3]
  \definition{v.}{estar assustado; ter medo}
\end{entry}

\begin{entry}{害羞}{hai4xiu1}{10,10}{⼧、⽺}
  \definition{adj.}{tímido | envergonhado}
\end{entry}

\begin{entry}{含}{han2}{7}{⼝}[HSK 4]
  \definition{v.}{manter na boca (sem engolir ou cuspir) | conter; incluir | cuidar; acalentar; abrigar}
\end{entry}

\begin{entry}{含金量}{han2jin1liang4}{7,8,12}{⼝、⾦、⾥}
  \definition{adj.}{conteúdo de ouro | (fig.) valioso}
\end{entry}

\begin{entry}{含量}{han2 liang4}{7,12}{⼝、⾥}[HSK 4]
  \definition{s.}{conteúdo; a quantidade de um componente contido em uma substância}
\end{entry}

\begin{entry}{含义}{han2yi4}{7,3}{⼝、⼂}[HSK 4]
  \definition[个,种,层]{s.}{sentido; mensagem; significado; implicação}
\end{entry}

\begin{entry}{含有}{han2 you3}{7,6}{⼝、⽉}[HSK 4]
  \definition{v.}{conter; ter; incluir}
\end{entry}

\begin{entry}{函数}{han2shu4}{8,13}{⼐、⽁}
  \definition{s.}{função (matemática)}
\end{entry}

\begin{entry}{涵}{han2}{11}{⽔}
  \definition{s.}{bueiro | galeria}
  \definition{v.}{conter | incluir | entupir}
\end{entry}

\begin{entry}{寒假}{han2jia4}{12,11}{⼧、⼈}[HSK 4]
  \definition[个]{s.}{férias de inverno (feriados); férias escolares no meio do inverno, em janeiro e fevereiro (na China)}
\end{entry}

\begin{entry}{寒冷}{han2 leng3}{12,7}{⼧、⼎}[HSK 4]
  \definition{adj.}{frio; frígido; gélido; gelado}
\end{entry}

\begin{entry}{韩国}{han2guo2}{12,8}{⾱、⼞}
  \definition*{s.}{Coréia do Sul}
\end{entry}

\begin{entry}{韩国人}{han2guo2ren2}{12,8,2}{⾱、⼞、⼈}
  \definition{s.}{coreano | pessoa ou povo da Coréia}
\end{entry}

\begin{entry}{厂}{han3}{2}{⼚}[Kangxi 27]
  \definition{s.}{radical ``penhasco'' em caracteres chineses (radical Kangxi 27)}
  \seeref{厂}{chang3}
\end{entry}

\begin{entry}{喊}{han3}{12}{⼝}[HSK 2]
  \definition{clas.}{gritar | berrar | chamar (uma pessoa)}
\end{entry}

\begin{entry}{汉}{han4}{5}{⽔}
  \definition{s.}{grupo étnico Han | chinês (língua) | dinastia Han (206 a.C.-220d.C.) | homem}
\end{entry}

\begin{entry}{汉堡包}{han4bao3bao1}{5,12,5}{⽔、⼟、⼓}
  \definition[个]{s.}{hambúrguer}
\end{entry}

\begin{entry}{汉堡王}{han4bao3wang2}{5,12,4}{⽔、⼟、⽟}
  \definition*{s.}{Burguer King (restaurante \emph{fast-food})}
\end{entry}

\begin{entry}{汉服}{han4fu2}{5,8}{⽔、⽉}
  \definition{s.}{vestido chinês tradicional Han}
\end{entry}

\begin{entry}{汉葡词典}{han4-pu2 ci2dian3}{5,12,7,8}{⽔、⾋、⾔、⼋}
  \definition[部,本]{s.}{dicionário chinês-português}
  \seealsoref{葡汉词典}{pu2-han4 ci2dian3}
\end{entry}

\begin{entry}{汉语}{han4yu3}{5,9}{⽔、⾔}[HSK 1]
  \definition[门]{s.}{língua chinesa, mandarim}
\end{entry}

\begin{entry}{汉字}{han4 zi4}{5,6}{⽔、⼦}[HSK 1]
  \definition[个]{s.}{caracter chinês}
\end{entry}

\begin{entry}{汗水}{han4shui3}{6,4}{⽔、⽔}
  \definition*{s.}{Rio Han (Hanshui)}
  \definition{s.}{suor | transpiração}
\end{entry}

\begin{entry}{汗腺}{han4xian4}{6,13}{⽔、⾁}
  \definition{s.}{glândula sudorípara}
\end{entry}

\begin{entry}{汗液}{han4ye4}{6,11}{⽔、⽔}
  \definition{s.}{suor}
\end{entry}

\begin{entry}{焊}{han4}{11}{⽕}
  \definition{v.}{soldar}
\end{entry}

\begin{entry}{撼}{han4}{16}{⼿}
  \definition{v.}{sacudir | vibrar}
\end{entry}

\begin{entry}{行}{hang2}{6}{⾏}[HSK 3]
  \definition{clas.}{para coisas usadas para viajar}
  \definition{s.}{linha; fileira | comércio; profissão; ramo de atividade | empresa de negócios}
  \seeref{行}{xing2}
\end{entry}

\begin{entry}{行业}{hang2ye4}{6,5}{⾏、⼀}[HSK 4]
  \definition[种,个]{s.}{comércio; indústria; setor; profissão; categorias em negócios e indústria referem-se a ocupações em geral}
\end{entry}

\begin{entry}{航班}{hang2ban1}{10,10}{⾈、⽟}[HSK 4]
  \definition[个,次]{s.}{número do voo; voo programado}
\end{entry}

\begin{entry}{航空}{hang2kong1}{10,8}{⾈、⽳}[HSK 4]
  \definition{s.}{viagem; aviação; refere-se ao voo de uma aeronave no ar}
\end{entry}

\begin{entry}{航天员}{hang2tian1yuan2}{10,4,7}{⾈、⼤、⼝}
  \definition{s.}{astronauta}
\end{entry}

\begin{entry}{号}{hao2}{5}{⼝}
  \definition[个]{s.}{rugido | choro}
  \seeref{号}{hao4}
\end{entry}

\begin{entry}{蚝}{hao2}{10}{⾍}
  \definition{s.}{ostra}
\end{entry}

\begin{entry}{毫不费力}{hao2bu2fei4li4}{11,4,9,2}{⽊、⼀、⾙、⼒}
  \definition{adj.}{sem esforço | não gastando o menor esforço}
\end{entry}

\begin{entry}{毫米}{hao2mi3}{11,6}{⽊、⽶}[HSK 4]
  \definition{clas.}{milímetro; unidade legal de medida de comprimento, 1 mm equivale a 0,1 cm}
\end{entry}

\begin{entry}{毫升}{hao2 sheng1}{11,4}{⽊、⼗}[HSK 4]
  \definition{clas.}{mililitro; unidade de volume, milésimo de um litro (ml)}
\end{entry}

\begin{entry}{豪华}{hao2hua2}{14,6}{⾗、⼗}
  \definition{adj.}{luxuoso}
\end{entry}

\begin{entry}{好}{hao3}{6}{⼥}[HSK 1,4]
  \definition{adj.}{bom; ótimo; agradável; vantajoso; satisfatório | amigável; gentil; amistoso; amável | saudável; bem | pronto; concluído; usado após um verbo para indicar conclusão ou perfeição | fácil (de fazer); conveniente; responsável (por)}
  \definition{adv.}{muito; bastante; tão; usado na frente de uma palavra de quantidade ou uma palavra de tempo para indicar muito ou por muito tempo | em que medida; como; usado antes de adjetivos e verbos para indicar profundidade e com exclamação}
  \definition{interj.}{O.K.; tudo bem; aprovação, acordo ou encerramento | (no início de uma frase ou oração) expressa concordância (ou desaprovação, surpresa, etc.)}
  \definition{prep.}{de modo a; para que | apaixonar-se}
  \definition{s.}{referindo-se a palavras de elogio ou aplauso | saudações; cumprimentos}
  \definition{suf.}{sufixo que indica conclusão ou prontidão | depois de um pronome significa ``olá''}
  \definition{v.}{dever; precisar; ter que}
  \seeref{好}{hao4}
\end{entry}

\begin{entry}{好吃}{hao3chi1}{6,6}{⼥、⼝}[HSK 1]
  \definition{adj.}{delicioso | saboroso}
  \seeref{好吃}{hao4chi1}
\end{entry}

\begin{entry}{好处}{hao3chu4}{6,5}{⼥、⼡}[HSK 2]
  \definition[个]{s.}{benefício | vantagem | ganho | lucro}
\end{entry}

\begin{entry}{好多}{hao3 duo1}{6,6}{⼥、⼣}[HSK 2]
  \definition{adj.}{muito | uma boa quantidade | um bom negócio | uma grande quantidade}
  \definition{pron.}{quanto}
\end{entry}

\begin{entry}{好汉}{hao3han4}{6,5}{⼥、⽔}
  \definition[条]{s.}{herói | pessoa forte e corajosa}
\end{entry}

\begin{entry}{好好}{hao3 hao3}{6,6}{⼥、⼥}[HSK 3]
  \definition{adj.}{realmente bom/bem; em perfeitas condições; quando tudo está bem}
  \definition{adv.}{diretamente; seriamente; cuidadosamente}
\end{entry}

\begin{entry}{好久}{hao3jiu3}{6,3}{⼥、⼃}[HSK 2]
  \definition{adv.}{por muito tempo | por eras (no passado)}
\end{entry}

\begin{entry}{好看}{hao3 kan4}{6,9}{⼥、⽬}[HSK 1]
  \definition{adj.}{boa aparência | bom (um filme, livro, programa de TV, etc.)}
\end{entry}

\begin{entry}{好人}{hao3 ren2}{6,2}{⼥、⼈}[HSK 2]
  \definition[个]{s.}{boa (legal) pessoa | uma pessoa saudável | uma pessoa que tenta se dar bem com todos}
\end{entry}

\begin{entry}{好生}{hao3sheng1}{6,5}{⼥、⽣}
  \definition{adv.}{bastante; extremamente | cuidadosamente; apropriadamente}
\end{entry}

\begin{entry}{好事}{hao3 shi4}{6,8}{⼥、⼅}[HSK 2]
  \definition[个]{s.}{boa ação | um ato de caridade | boas obras | evento feliz}
  \seeref{好事}{hao4 shi4}
\end{entry}

\begin{entry}{好听}{hao3 ting1}{6,7}{⼥、⼝}[HSK 1]
  \definition{adj.}{agradável de ouvir}
\end{entry}

\begin{entry}{好玩儿}{hao3 wan2r5}{6,8,2}{⼥、⽟、⼉}[HSK 1]
  \definition{adj.}{divertido | prazeroso | interessante}
\end{entry}

\begin{entry}{好象}{hao3xiang4}{6,11}{⼥、⾗}
  \variantof{好像}
\end{entry}

\begin{entry}{好像}{hao3xiang4}{6,13}{⼥、⼈}[HSK 2]
  \definition{adv.}{talvez fosse | parecer | ser como}
\end{entry}

\begin{entry}{好心}{hao3xin1}{6,4}{⼥、⼼}
  \definition{s.}{bondade | boas intenções}
\end{entry}

\begin{entry}{好学}{hao3xue2}{6,8}{⼥、⼦}
  \definition{adj.}{fácil de aprender}
  \seeref{好学}{hao4xue2}
\end{entry}

\begin{entry}{好用}{hao3yong4}{6,5}{⼥、⽤}
  \definition{adj.}{fácil de usar | adequado ao uso}
\end{entry}

\begin{entry}{好友}{hao3you3}{6,4}{⼥、⼜}[HSK 4]
  \definition[位,个]{s.}{bom amigo; amigo próximo}
\end{entry}

\begin{entry}{号}{hao4}{5}{⼝}[HSK 1]
  \definition{clas.}{para indicar o número de pessoas}
  \definition{num.}{dia do mês | usado para indicar o número de pessoas}
  \definition[个]{s.}{número ordinal | dia de um mês | marca | sinal | estabelecimento comercial | tamanho | buzina (instrumento de sopro) | toque de corneta | nome suposto}
  \definition{suf.}{sufixo de navio}
  \definition{v.}{tomar um pulso}
  \seeref{号}{hao2}
\end{entry}

\begin{entry}{号角}{hao4jiao3}{5,7}{⼝、⾓}
  \definition{s.}{corneta | trombeta}
\end{entry}

\begin{entry}{号码}{hao4ma3}{5,8}{⼝、⽯}[HSK 4]
  \definition[个,组,串]{s.}{número}
\end{entry}

\begin{entry}{好}{hao4}{6}{⼥}
  \definition*{s.}{sobrenome Hao}
  \definition{adv.}{algo que acontece com frequência, que é fácil de acontecer}
  \definition{v.}{gostar; amar; ter afeição por}
  \seeref{好}{hao3}
\end{entry}

\begin{entry}{好吃}{hao4chi1}{6,6}{⼥、⼝}
  \definition{v.}{gostar de comer | ser guloso}
  \seeref{好吃}{hao3chi1}
\end{entry}

\begin{entry}{好奇}{hao4qi2}{6,8}{⼥、⼤}[HSK 3]
  \definition{adj.}{curioso}
  \definition{s.}{curiosidade}
  \definition{v.}{ser ou estar curioso}
\end{entry}

\begin{entry}{好事}{hao4 shi4}{6,8}{⼥、⼅}
  \definition[行]{s.}{bençãos}
  \seeref{好事}{hao3 shi4}
\end{entry}

\begin{entry}{好学}{hao4xue2}{6,8}{⼥、⼦}
  \definition{s.}{estudioso | erudito}
  \seeref{好学}{hao3xue2}
\end{entry}

\begin{entry}{呵}{he1}{8}{⼝}
  \definition{expr.}{Meu Deus! | expelir a respiração}
  \seeref{呵}{a1}
\end{entry}

\begin{entry}{欱}{he1}{10}{⽋}
  \variantof{喝}
\end{entry}

\begin{entry}{喝}{he1}{12}{⼝}[HSK 1]
  \definition{interj.}{Meu Deus!}
  \definition{v.}{beber}
  \seeref{喝}{he4}
\end{entry}

\begin{entry}{喝醉}{he1zui4}{12,15}{⼝、⾣}
  \definition{v.}{ficar bêbado}
\end{entry}

\begin{entry}{合}{he2}{6}{⼝}[HSK 3]
  \definition*{s.}{sobrenome He}
  \definition{adj.}{todo; completo; inteiro}
  \definition{clas.}{para rodadas}
  \definition{s.}{conjunção}
  \definition{v.}{fechar | juntar; combinar | adequar-se; concordar; conformar-se a | ser igual a; somar}
\end{entry}

\begin{entry}{合法}{he2fa3}{6,8}{⼝、⽔}[HSK 3]
  \definition{adj.}{legal; legítimo; correto}
\end{entry}

\begin{entry}{合格}{he2ge2}{6,10}{⼝、⽊}[HSK 3]
  \definition{adj.}{qualificado; de acordo com o padrão}
\end{entry}

\begin{entry}{合理}{he2li3}{6,11}{⼝、⽟}[HSK 3]
  \definition{adj.}{racional; razoável; equitativo}
\end{entry}

\begin{entry}{合适}{he2shi4}{6,9}{⼝、⾡}[HSK 2]
  \definition{adj.}{certo | adequado | apropriado}
\end{entry}

\begin{entry}{合同}{he2tong5}{6,6}{⼝、⼝}[HSK 4]
  \definition[个,份]{s.}{contrato; acordo; uma disposição para observância mútua por duas ou mais partes na condução de um assunto com o objetivo de determinar seus respectivos direitos e obrigações.}
\end{entry}

\begin{entry}{合宪性}{he2xian4xing4}{6,9,8}{⼝、⼧、⼼}
  \definition{s.}{constitucionalismo}
\end{entry}

\begin{entry}{合资}{he2zi1}{6,10}{⼝、⾙}
  \definition{s.}{\emph{joint-venture} com capitais mistos}
\end{entry}

\begin{entry}{合作}{he2zuo4}{6,7}{⼝、⼈}[HSK 3]
  \definition[个]{s.}{cooperação; colaboração}
  \definition{v.}{cooperar; colaborar; trabalhar em conjunto}
\end{entry}

\begin{entry}{何}{he2}{7}{⼈}
  \definition*{s.}{sobrenome He}
  \definition{adv.}{expressa exclamação, equivalente a ``多么''}
  \definition{pron.}{O que?; Onde?; Por que? | expressa uma pergunta retórica, equivalente a ``岂'', ``怎''}
  \seealsoref{多么}{duo1me5}
  \seealsoref{岂}{qi3}
  \seealsoref{怎}{zen3}
\end{entry}

\begin{entry}{何况}{he2kuang4}{7,7}{⼈、⼎}
  \definition{conj.}{além disso | muito menos}
\end{entry}

\begin{entry}{和}{he2}{8}{⼝}[HSK 1]
  \definition*{s.}{sobrenome He}
  \definition{conj.}{e (somente para palavras) | junto com}
  \definition{s.}{união | paz | japonês (comida, roupa, etc.) | harmonia}
  \seeref{和}{he4}
  \seeref{和}{hu2}
  \seeref{和}{huo2}
  \seeref{和}{huo4}
\end{entry}

\begin{entry}{和平}{he2ping2}{8,5}{⼝、⼲}[HSK 3]
  \definition{adj.}{pacífico; não violento}
  \definition{s.}{paz}
\end{entry}

\begin{entry}{和平共处}{he2ping2gong4chu3}{8,5,6,5}{⼝、⼲、⼋、⼡}
  \definition{s.}{coexistência pacífica de nações, sociedades, etc.}
\end{entry}

\begin{entry}{和谐}{he2xie2}{8,11}{⼝、⾔}
  \definition{adj.}{harmonioso}
  \definition{s.}{harmonia}
  \definition{v.}{(eufemismo) censurar}
\end{entry}

\begin{entry}{河}{he2}{8}{⽔}[HSK 2]
  \definition[条,道]{s.}{rio}
\end{entry}

\begin{entry}{河蚌}{he2bang4}{8,10}{⽔、⾍}
  \definition{s.}{mexilhões | bivalves cultivados em rios e lagos}
\end{entry}

\begin{entry}{核}{he2}{10}{⽊}
  \definition{adj.}{nuclear}
  \definition{s.}{poço | pedra | núcleo}
  \definition{v.}{examinar | checar | verificar}
\end{entry}

\begin{entry}{荷}{he2}{10}{⾋}
  \definition{s.}{lótus}
  \seeref{荷}{he4}
\end{entry}

\begin{entry}{荷花}{he2hua1}{10,7}{⾋、⾋}
  \definition{s.}{lótus}
\end{entry}

\begin{entry}{盒}{he2}{11}{⽫}
  \definition{clas.}{caixa pequena}
  \definition{s.}{caixa pequena | estojo}
\end{entry}

\begin{entry}{和}{he4}{8}{⼝}
  \definition{v.}{compor um poema em resposta (ao poema de alguém) usando a mesma sequência de rimas | juntar-se à cantoria | cantar junto com outros}
  \seeref{和}{he2}
  \seeref{和}{hu2}
  \seeref{和}{huo2}
  \seeref{和}{huo4}
\end{entry}

\begin{entry}{贺}{he4}{9}{⾙}
  \definition*{s.}{sobrenome He}
  \definition{v.}{parabenizar | congratular}
\end{entry}

\begin{entry}{荷}{he4}{10}{⾋}
  \definition{s.}{carga | responsabilidade}
  \definition{v.}{carregar no ombro ou nas costas}
  \seeref{荷}{he2}
\end{entry}

\begin{entry}{喝}{he4}{12}{⼝}
  \definition{v.}{gritar bem alto}
  \seeref{喝}{he1}
\end{entry}

\begin{entry}{喝彩}{he4cai3}{12,11}{⼝、⼺}
  \definition{s.}{aclamar | torcer}
\end{entry}

\begin{entry}{褐色}{he4 se4}{14,6}{⾐、⾊}
  \definition{s.}{cor marrom}
\end{entry}

\begin{entry}{鹤}{he4}{15}{⿃}
  \definition{s.}{grou (ave)}
\end{entry}

\begin{entry}{黑}{hei1}{12}{⿊}[HSK 2][Kangxi 203]
  \definition{adj.}{preto | escuro | ilegal | secreto | sombrio | sinistro}
  \definition{v.}{esconder (algo) | difamar | (empréstimo linguístico) (computador) hackear}
\end{entry}

\begin{entry}{黑暗}{hei1 an4}{12,13}{⿊、⽇}[HSK 4]
  \definition{adj.}{escuro; sombrio; sem luz | maligno; corrupto; reacionário}
\end{entry}

\begin{entry}{黑板}{hei1ban3}{12,8}{⿊、⽊}[HSK 2]
  \definition[块,个]{s.}{quadro negro}
\end{entry}

\begin{entry}{黑客}{hei1ke4}{12,9}{⿊、⼧}
  \definition{s.}{(empréstimo linguístico) (computação) \emph{hacker}}
\end{entry}

\begin{entry}{黑色}{hei1 se4}{12,6}{⿊、⾊}[HSK 2]
  \definition{s.}{cor preta}
\end{entry}

\begin{entry}{很}{hen3}{9}{⼻}[HSK 1]
  \definition{adv.}{bastante | muito | terrivelmente | advérbio de grau}
\end{entry}

\begin{entry}{恨}{hen4}{9}{⼼}
  \definition{s.}{ódio}
  \definition{v.}{odiar}
\end{entry}

\begin{entry}{恒星系}{heng2xing1xi4}{9,9,7}{⼼、⽇、⽷}
  \definition{s.}{sistema estelar | galáxia}
\end{entry}

\begin{entry}{横竖}{heng2shu5}{15,9}{⽊、⽴}
  \definition{adv.}{de qualquer maneira | independentemente (linguagem falada)}
\end{entry}

\begin{entry}{轰鸣}{hong1ming2}{8,8}{⾞、⿃}
  \definition{s.}{bum (som de explosão) | estrondo}
\end{entry}

\begin{entry}{轰炸机}{hong1zha4ji1}{8,9,6}{⾞、⽕、⽊}
  \definition{s.}{bombardeiro (aeronave)}
\end{entry}

\begin{entry}{哄}{hong1}{9}{⼝}
  \definition{s.}{gargalhadas | risadas ruidosas | algazarra | rugido | clamor}
  \seeref{哄}{hong3}
  \seeref{哄}{hong4}
\end{entry}

\begin{entry}{红}{hong2}{6}{⽷}[HSK 2]
  \definition*{s.}{sobrenome Hong}
  \definition{adj.}{vermelho | popular | revolucionário}
  \definition{s.}{bônus}
\end{entry}

\begin{entry}{红包}{hong2 bao1}{6,5}{⽷、⼓}[HSK 4]
  \definition[个]{s.}{saco de papel vermelho ou envelope contendo dinheiro como presente, gorjeta ou bônus | suborno; propina}
\end{entry}

\begin{entry}{红宝石}{hong2bao3shi2}{6,8,5}{⽷、⼧、⽯}
  \definition{s.}{rubi}
\end{entry}

\begin{entry}{红茶}{hong2 cha2}{6,9}{⽷、⾋}[HSK 3]
  \definition[杯,壶,斤,种]{s.}{chá preto}
\end{entry}

\begin{entry}{红酒}{hong2 jiu3}{6,10}{⽷、⾣}[HSK 3]
  \definition{s.}{vinho tinto}
\end{entry}

\begin{entry}{红绿灯}{hong2lv4deng1}{6,11,6}{⽷、⽷、⽕}
  \definition[个]{s.}{semáforo | sinal de trânsito}
\end{entry}

\begin{entry}{红色}{hong2 se4}{6,6}{⽷、⾊}[HSK 2]
  \definition{s.}{cor vermelha}
\end{entry}

\begin{entry}{红烧}{hong2shao1}{6,10}{⽷、⽕}
  \definition{s.}{guisado em molho de soja (prato)}
\end{entry}

\begin{entry}{红薯}{hong2shu3}{6,16}{⽷、⾋}
  \definition{s.}{batata doce}
\end{entry}

\begin{entry}{红线}{hong2xian4}{6,8}{⽷、⽷}
  \definition{s.}{linha vermelha}
\end{entry}

\begin{entry}{洪水}{hong2shui3}{9,4}{⽔、⽔}
  \definition{s.}{enchente | inundação | dilúvio}
\end{entry}

\begin{entry}{哄}{hong3}{9}{⼝}
  \definition{v.}{enganar | persuadir | divertir (uma criança)}
  \seeref{哄}{hong1}
  \seeref{哄}{hong4}
\end{entry}

\begin{entry}{哄}{hong4}{9}{⼝}
  \definition{s.}{tumulto | agitação | perturbação}
  \seeref{哄}{hong1}
  \seeref{哄}{hong3}
\end{entry}

\begin{entry}{猴子}{hou2zi5}{12,3}{⽝、⼦}
  \definition[只]{s.}{macaco}
\end{entry}

\begin{entry}{后}{hou4}{6}{⼝}[HSK 1]
  \definition*{s.}{sobrenome Hou}
  \definition{adv.}{atrás | depois | mais tarde}
  \definition{s.}{traseiro | descendência | posteridade |imperatriz | rainha | soberano | governante}
\end{entry}

\begin{entry}{后边}{hou4 bian5}{6,5}{⼝、⾡}[HSK 1]
  \definition{adv.}{atrás | detrás}
\end{entry}

\begin{entry}{后果}{hou4guo3}{6,8}{⼝、⽊}[HSK 3]
  \definition{s.}{consequência; resultado}
\end{entry}

\begin{entry}{后来}{hou4lai2}{6,7}{⼝、⽊}[HSK 2]
  \definition{adv.}{mais tarde}
\end{entry}

\begin{entry}{后面}{hou4mian4}{6,9}{⼝、⾯}[HSK 3]
  \definition{adv.}{parte de trás; retaguarda; atrás | atrás; perto do fim; na parte de trás | mais tarde; depois}
  \seeref{后面}{hou4mian5}
\end{entry}

\begin{entry}{后面}{hou4mian5}{6,9}{⼝、⾯}[HSK 3]
  \definition{adv.}{parte de trás; retaguarda; atrás | atrás; perto do fim; na parte de trás | mais tarde; depois}
  \seeref{后面}{hou4mian4}
\end{entry}

\begin{entry}{后年}{hou4nian2}{6,6}{⼝、⼲}[HSK 3]
  \definition{s.}{o ano que vem; daqui a dois anos}
\end{entry}

\begin{entry}{后天}{hou4 tian1}{6,4}{⼝、⼤}[HSK 1]
  \definition{adv.}{depois de amanhã}
\end{entry}

\begin{entry}{后头}{hou4 tou5}{6,5}{⼝、⼤}[HSK 4]
  \definition{adv.}{posteriormente | atrás | mais tarde}
  \definition{s.}{a parte de trás | a parte traseira}
\end{entry}

\begin{entry}{厚}{hou4}{9}{⼚}[HSK 4]
  \definition*{s.}{sobrenome Hou}
  \definition{adj.}{grosso; espesso | profundo | bondoso; gentil; magnânimo | grande; generoso | rico ou forte em sabor}
  \definition{s.}{espessura; profundidade}
  \definition{v.}{favorecer; enfatizar}
\end{entry}

\begin{entry}{呼吸}{hu1xi1}{8,6}{⼝、⼝}[HSK 4]
  \definition{s.}{um suspiro; metáfora para um período de tempo muito curto}
  \definition{v.}{respirar}
\end{entry}

\begin{entry}{呼啸}{hu1xiao4}{8,11}{⼝、⼝}
  \definition{v.}{assobiar}
\end{entry}

\begin{entry}{忽然}{hu1ran2}{8,12}{⼼、⽕}[HSK 2]
  \definition{adv.}{de repente}
\end{entry}

\begin{entry}{忽视}{hu1shi4}{8,8}{⼼、⾒}[HSK 4]
  \definition{v.}{ignorar; negligenciar; menosprezar; desprezar; dar de ombros}
\end{entry}

\begin{entry}{和}{hu2}{8}{⼝}
  \definition{v.}{completar um conjunto de Mahjong ou cartas de baralho}
  \seeref{和}{he2}
  \seeref{和}{he4}
  \seeref{和}{huo2}
  \seeref{和}{huo4}
\end{entry}

\begin{entry}{胡萝卜}{hu2luo2bo5}{9,11,2}{⾁、⾋、⼘}
  \definition{s.}{cenoura}
\end{entry}

\begin{entry}{湖}{hu2}{12}{⽔}[HSK 2]
  \definition[个,片]{s.}{lago}
\end{entry}

\begin{entry}{湖南}{hu2nan2}{12,9}{⽔、⼗}
  \definition*{s.}{Hunan}
\end{entry}

\begin{entry}{葫芦}{hu2lu5}{12,7}{⾋、⾋}
  \definition{adj.}{confuso}
  \definition{s.}{cabaça | termo genérico para bloco e equipamento (ou partes dele)}
\end{entry}

\begin{entry}{糊里糊涂}{hu2li5hu2tu5}{15,7,15,10}{⽶、⾥、⽶、⽔}
  \definition{adj.}{desnorteado | perturbado}
\end{entry}

\begin{entry}{蝴蝶}{hu2die2}{15,15}{⾍、⾍}
  \definition[只]{s.}{borboleta}
\end{entry}

\begin{entry}{虎}{hu3}{8}{⾌}
  \definition{s.}{tigre}
  \seealsoref{老虎}{lao3hu3}
\end{entry}

\begin{entry}{虎虎}{hu3hu3}{8,8}{⾌、⾌}
  \definition{adj.}{formidável | forte | vigoroso}
\end{entry}

\begin{entry}{虎口}{hu3kou3}{8,3}{⾌、⼝}
  \definition{s.}{lugar perigoso | cova do tigre}
\end{entry}

\begin{entry}{虎鼬}{hu3you4}{8,18}{⾌、⿏}
  \definition{s.}{doninha}
\end{entry}

\begin{entry}{互}{hu4}{4}{⼆}
  \definition{adj.}{mútuo | recíproco}
\end{entry}

\begin{entry}{互动}{hu4dong4}{4,6}{⼆、⼒}
  \definition{s.}{interativo}
  \definition{v.}{interagir}
\end{entry}

\begin{entry}{互利}{hu4li4}{4,7}{⼆、⼑}
  \definition{s.}{benefício mútuo}
\end{entry}

\begin{entry}{互联网}{hu4lian2wang3}{4,12,6}{⼆、⽿、⽹}[HSK 3]
  \definition{s.}{\emph{Internet}}
  \seealsoref{网际网路}{wang3ji4wang3lu4}
  \seealsoref{网际网络}{wang3ji4wang3luo4}
  \seealsoref{网路}{wang3lu4}
\end{entry}

\begin{entry}{互相}{hu4xiang1}{4,9}{⼆、⽬}[HSK 3]
  \definition{adv.}{mutuamente; um ao outro}
\end{entry}

\begin{entry}{户}{hu4}{4}{⼾}[HSK 4][Kangxi 63]
  \definition*{s.}{sobrenome Hu}
  \definition[个]{s.}{porta com um painel; porta | domicílio; residência; família | status familiar | conta (banco)}
\end{entry}

\begin{entry}{护士}{hu4shi5}{7,3}{⼿、⼠}[HSK 4]
  \definition[名,位]{s.}{enfermeiro; pessoas especializadas em enfermagem em hospitais ou instituições epidemiológicas}
\end{entry}

\begin{entry}{护照}{hu4zhao4}{7,13}{⼿、⽕}[HSK 2]
  \definition[本,个]{s.}{passaporte}
\end{entry}

\begin{entry}{化}{hua1}{4}{⼔}[HSK 3]
  \definition*{s.}{sobrenome Hua}
  \definition{s.}{químico}
  \definition{suf.}{modernizar; modernização}
  \definition{v.}{mudar; converter; transformar | converter; influenciar | derreter; dissolver | digerir | queimar | morrer | pedir esmola}
  \variantof{花}
\end{entry}

\begin{entry}{花}{hua1}{7}{⾋}[HSK 1,2,4]
  \definition*{s.}{sobrenome Hua}
  \definition{adj.}{multicolorido; colorido | embaçado; obscuro; deslumbrado e confuso | extravagante; florido; vistoso}
  \definition[朵,支,束,把,盆,簇]{s.}{flor; órgãos de reprodução sexual de plantas com sementes | flor; planta ornamental |  qualquer coisa que se assemelhe a uma flor | fogos de artifício | padrão; design; design decorativo | flor; metáfora para a essência de uma causa | prostituta; cortesã; referindo-se a prostitutas ou a assuntos relacionados a prostitutas | algodão | varíola | ferimento; ferida; lesões traumáticas sofridas em combate}
  \definition{v.}{gastar; despender; consumir}
\end{entry}

\begin{entry}{花茶}{hua1cha2}{7,9}{⾋、⾋}
  \definition[杯,壶]{s.}{chá perfumado}
\end{entry}

\begin{entry}{花店}{hua1dian4}{7,8}{⾋、⼴}
  \definition{s.}{floricultura}
\end{entry}

\begin{entry}{花儿}{hua1r5}{7,2}{⾋、⼉}
  \definition[朵,支,束,把,盆,簇]{s.}{flor}
\end{entry}

\begin{entry}{花生}{hua1sheng1}{7,5}{⾋、⽣}
  \definition[粒]{s.}{amendoim}
\end{entry}

\begin{entry}{花样游泳}{hua1yang4you2yong3}{7,10,12,8}{⾋、⽊、⽔、⽔}
  \definition{s.}{nado sincronizado}
\end{entry}

\begin{entry}{花椰菜}{hua1ye1cai4}{7,12,11}{⾋、⽊、⾋}
  \definition{s.}{couve-flor}
\end{entry}

\begin{entry}{花园}{hua1 yuan2}{7,7}{⾋、⼞}[HSK 2]
  \definition[个,座]{s.}{jardim}
\end{entry}

\begin{entry}{划}{hua2}{6}{⼑}[HSK 4]
  \definition{adj.}{rentável; vale (o esforço); compensa (fazer alguma coisa)}
  \definition{v.}{remar | ser vantajoso para alguém; ser uma pechincha | arranhar; cortar a superfície de; cortar em outra coisa com um objeto pontiagudo | arranhar; golpear;  esfregar uma coisa ou varrer sobre outra}
  \seeref{划}{hua4}
\end{entry}

\begin{entry}{划船}{hua2 chuan2}{6,11}{⼑、⾈}[HSK 3]
  \definition[次,回]{s.}{remo (ato de remar); passeios de barco}
  \definition{v.}{remar um barco}
\end{entry}

\begin{entry}{划艇}{hua2ting3}{6,12}{⼑、⾈}
  \definition{s.}{barco a remo}
\end{entry}

\begin{entry}{华人}{hua2 ren2}{6,2}{⼗、⼈}[HSK 3]
  \definition{s.}{Chinês; chinês étnico | cidadãos estrangeiros de ascendência chinesa que adquiriram nacionalidade no seu país de residência}
\end{entry}

\begin{entry}{华盛顿}{hua2sheng4dun4}{6,11,10}{⼗、⽫、⾴}
  \definition*{s.}{Washington}
\end{entry}

\begin{entry}{华氏}{hua2shi4}{6,4}{⼗、⽒}
  \definition{s.}{graus Fahrenheit (°F)}
\end{entry}

\begin{entry}{华夏}{hua2xia4}{6,10}{⼗、⼢}
  \definition*{s.}{Huaxia, nome antigo da China | Catai}
\end{entry}

\begin{entry}{华裔}{hua2yi4}{6,13}{⼗、⾐}
  \definition{s.}{descendente de chinês}
\end{entry}

\begin{entry}{滑}{hua2}{12}{⽔}
  \definition*{s.}{sobrenome Hua}
  \definition{adj.}{deslizado}
  \definition{v.}{deslizar}
\end{entry}

\begin{entry}{滑雪}{hua2xue3}{12,11}{⽔、⾬}
  \definition{v.+compl.}{esquiar | praticar esqui}
\end{entry}

\begin{entry}{化学}{hua4xue2}{4,8}{⼔、⼦}
  \definition{s.}{química (disciplina)}
\end{entry}

\begin{entry}{划}{hua4}{6}{⼑}[HSK 4]
  \definition{s.}{traço de um caracter chinês}
  \definition{v.}{delimitar; diferenciar; delinear | transferir; ceder | planejar; programar | desenhar; marcar; delinear; fazer linhas ou escrever como marcadores com uma caneta ou objeto semelhante a uma caneta}
  \seeref{划}{hua2}
\end{entry}

\begin{entry}{画}{hua4}{8}{⽥}[HSK 2]
  \definition[幅,张]{s.}{quadro | pintura | traço de um caractere chinês (variante de 划) | (caligrafia) traço horizontal (variante de traço 划)}
  \definition{v.}{desenhar | pintar | traçar uma linha (variante de 划)}
  \seeref{划}{hua4}
\end{entry}

\begin{entry}{画地为牢}{hua4di4wei2lao2}{8,6,4,7}{⽥、⼟、⼂、⼧}
  \definition{expr.}{(literalmente) ser confinado dentro de um círculo desenhado no chão | (figurativo) limitar-se a uma gama restrita de atividades}
\end{entry}

\begin{entry}{画家}{hua4 jia1}{8,10}{⽥、⼧}[HSK 2]
  \definition[个,名,位]{s.}{pintor}
\end{entry}

\begin{entry}{画儿}{hua4r5}{8,2}{⽥、⼉}[HSK 2]
  \definition{s.}{imagem | desenho | pintura}
\end{entry}

\begin{entry}{话}{hua4}{8}{⾔}[HSK 1]
  \definition[种,席,句,口,番]{s.}{fala | linguagem | dialeto}
\end{entry}

\begin{entry}{话剧}{hua4 ju4}{8,10}{⾔、⼑}[HSK 3]
  \definition[台,部]{s.}{drama moderno; peça de teatro}
\end{entry}

\begin{entry}{话题}{hua4ti2}{8,15}{⾔、⾴}[HSK 3]
  \definition[个,种,项]{s.}{assunto de uma palestra; tópico de uma conversa}
\end{entry}

\begin{entry}{怀旧}{huai2jiu4}{7,5}{⼼、⽇}
  \definition{s.}{nostalgia}
  \definition{v.}{sentir-se nostálgico}
\end{entry}

\begin{entry}{怀念}{huai2nian4}{7,8}{⼼、⼼}[HSK 4]
  \definition{v.}{pensar em; valorizar a memória de}
\end{entry}

\begin{entry}{怀疑}{huai2yi2}{7,14}{⼼、⽦}[HSK 4]
  \definition{v.}{duvidar; suspeitar | supor}
\end{entry}

\begin{entry}{坏}{huai4}{7}{⼟}[HSK 1]
  \definition{adj.}{avariado | mau}
  \definition{v.}{perder o controle}
\end{entry}

\begin{entry}{坏处}{huai4 chu4}{7,5}{⼟、⼡}[HSK 2]
  \definition[个]{s.}{dano | problema}
\end{entry}

\begin{entry}{坏蛋}{huai4dan4}{7,11}{⼟、⾍}
  \definition{s.}{bastardo | canalha | pessoa má}
\end{entry}

\begin{entry}{坏人}{huai4 ren2}{7,2}{⼟、⼈}[HSK 2]
  \definition[个]{s.}{malfeitor | canalha | pessoa má}
\end{entry}

\begin{entry}{欢快}{huan1kuai4}{6,7}{⽋、⼼}
  \definition{adj.}{feliz e sem ansiedade | vívido}
\end{entry}

\begin{entry}{欢乐}{huan1le4}{6,5}{⽋、⼃}[HSK 3]
  \definition{adj.}{feliz; alegre}
\end{entry}

\begin{entry}{欢迎}{huan1ying2}{6,7}{⽋、⾡}[HSK 2]
  \definition{adj.}{bem-vindo}
  \definition{v.}{dar as boas-vindas | ser bem-vindo}
\end{entry}

\begin{entry}{还}{huan2}{7}{⾡}[HSK 1]
  \definition*{s.}{sobrenome Huan}
  \definition{v.}{devolver | restituir | pagar de volta}
  \seeref{还}{hai2}
\end{entry}

\begin{entry}{环}{huan2}{8}{⽟}[HSK 3]
  \definition*{s.}{sobrenome Huan}
  \definition{clas.}{para anéis}
  \definition{s.}{anel; arco | \emph{link}; ligação}
  \definition{v.}{cercar; rodear; circular; circundar}
\end{entry}

\begin{entry}{环保}{huan2 bao3}{8,9}{⽟、⼈}[HSK 3]
  \definition{adj.}{bom para o meio ambiente; não danifica o meio ambiente}
  \definition{s.}{proteção ambiental}
\end{entry}

\begin{entry}{环境}{huan2jing4}{8,14}{⽟、⼟}[HSK 3]
  \definition[个]{s.}{ambiente | arredores; circunstâncias}
\end{entry}

\begin{entry}{环境卫生}{huan2jing4wei4sheng1}{8,14,3,5}{⽟、⼟、⼙、⽣}
  \definition{s.}{saneamento ambiental}
  \seealsoref{环卫}{huan2wei4}
\end{entry}

\begin{entry}{环卫}{huan2wei4}{8,3}{⽟、⼙}
  \definition{s.}{limpeza pública | saneamento urbano | saneamento ambiental | abreviação de 环境卫生}
  \seeref{环境卫生}{huan2jing4wei4sheng1}
\end{entry}

\begin{entry}{缓解}{huan3jie3}{12,13}{⽶、⾓}[HSK 4]
  \definition{v.}{facilitar; aliviar; atenuar; amenizar; reduzir}
\end{entry}

\begin{entry}{幻觉}{huan4jue2}{4,9}{⼳、⾒}
  \definition{s.}{ilusão | alucinação}
\end{entry}

\begin{entry}{换}{huan4}{10}{⼿}[HSK 2]
  \definition{v.}{mudar | trocar | substituir | converter (moedas)}
\end{entry}

\begin{entry}{换钱}{huan4qian2}{10,10}{⼿、⾦}
  \definition{v.+compl.}{trocar dinheiro (em pequenas valores ou em outra moeda) | trocar (mercadorias) por dinheiro | vender}
\end{entry}

\begin{entry}{荒芜}{huang1wu2}{9,7}{⾋、⾋}
  \definition{adj.}{estéril}
\end{entry}

\begin{entry}{皇帝}{huang2di4}{9,9}{⽩、⼱}
  \definition[个]{s.}{imperador}
\end{entry}

\begin{entry}{黄}{huang2}{11}{⿈}[HSK 2][Kangxi 201]
  \definition*{s.}{sobrenome Huang ou Hwang}
  \definition{adj.}{amarelo | pornográfico}
\end{entry}

\begin{entry}{黄瓜}{huang2 gua1}{11,5}{⿈、⽠}[HSK 4]
  \definition[根,棵,株]{s.}{pepino}
\end{entry}

\begin{entry}{黄河}{huang2he2}{11,8}{⿈、⽔}
  \definition*{s.}{Rio Amarelo | Rio Huang He}
\end{entry}

\begin{entry}{黄昏}{huang2hun1}{11,8}{⿈、⽇}
  \definition{s.}{anoitecer}
\end{entry}

\begin{entry}{黄金}{huang2jin1}{11,8}{⿈、⾦}[HSK 4]
  \definition{adj.}{de primeira qualidade; dourado;}
  \definition[块,克,两]{s.}{ouro; \emph{aurum}; um tipo de metal, de cor amarela, mais precioso, abreviação de ``金'', frequentemente falado como ``金子''.}
  \seealsoref{金}{jin1}
  \seealsoref{金子}{jin1zi5}
\end{entry}

\begin{entry}{黄色}{huang2 se4}{11,6}{⿈、⾊}[HSK 2]
  \definition{s.}{cor amarela}
\end{entry}

\begin{entry}{黄油}{huang2you2}{11,8}{⿈、⽔}
  \definition[盒]{s.}{manteiga}
\end{entry}

\begin{entry}{谎话}{huang3hua4}{11,8}{⾔、⾔}
  \definition{s.}{mentira}
\end{entry}

\begin{entry}{灰色}{hui1 se4}{6,6}{⽕、⾊}
  \definition{s.}{cor cinza}
\end{entry}

\begin{entry}{挥汗如雨}{hui1han4ru2yu3}{9,6,6,8}{⼿、⽔、⼥、⾬}
  \definition{s.}{suor derramado}
  \definition{v.}{pingar com suor}
\end{entry}

\begin{entry}{囘}{hui2}{5}{⼞}
  \variantof{回}
\end{entry}

\begin{entry}{回}{hui2}{6}{⼞}[HSK 1,2]
  \definition{clas.}{de atos de uma peça de teatro}
  \definition{s.}{seção ou capítulo (de um livro clássico) | grupo étnico Hui (mulçumanos chineses)}
  \definition{v.}{regressar | voltar | dar a volta | responder | resolver | circular | curvar}
\end{entry}

\begin{entry}{回答}{hui2da2}{6,12}{⼞、⽵}[HSK 1]
  \definition{v.}{responder}
\end{entry}

\begin{entry}{回到}{hui2 dao4}{6,8}{⼞、⼑}[HSK 1]
  \definition{v.}{retornar a}
\end{entry}

\begin{entry}{回复}{hui2 fu4}{6,9}{⼞、⼢}[HSK 4]
  \definition{v.}{responder (a uma carta) | retornar ao estado normal; restaurar algo ao seu estado original}
\end{entry}

\begin{entry}{回国}{hui2 guo2}{6,8}{⼞、⼞}[HSK 2]
  \definition{v.}{retornar ao seu país (terra natal)}
\end{entry}

\begin{entry}{回家}{hui2 jia1}{6,10}{⼞、⼧}[HSK 1]
  \definition{v.}{ir (voltar) para casa | estar em casa | retornar para casa}
\end{entry}

\begin{entry}{回来}{hui2 lai5}{6,7}{⼞、⽊}[HSK 1]
  \definition{v.}{regressar | voltar | estar de volta | (para a minha localização)}
\end{entry}

\begin{entry}{回去}{hui2 qu4}{6,5}{⼞、⼛}[HSK 1]
  \definition{v.}{regressar | voltar | estar de volta | (a partir da minha localização)}
\end{entry}

\begin{entry}{回信}{hui2xin4}{6,9}{⼞、⼈}
  \definition{s.}{uma carta em resposta | uma mensagem verbal em resposta}
  \definition{v.+compl.}{escrever em resposta | escrever de volta | responder uma carta | responder verbalmente uma mensagem}
\end{entry}

\begin{entry}{回旋}{hui2xuan2}{6,11}{⼞、⽅}
  \definition{v.}{circular | rodar | dar a volta}
\end{entry}

\begin{entry}{廻}{hui2}{8}{⼵}
  \variantof{回}
\end{entry}

\begin{entry}{汇}{hui4}{5}{⽔}[HSK 4]
  \definition{s.}{coisas coletadas; conjunto; coleção}
  \definition{v.}{convergir | reunir-se | remeter; transferir por meio de agências postais e telegráficas, bancos}
\end{entry}

\begin{entry}{汇报}{hui4bao4}{5,7}{⽔、⼿}[HSK 4]
  \definition[份,次]{s.}{relatório; referindo-se ao conteúdo de declarações escritas ou orais feitas a um superior ou pessoa relevante para apresentar uma situação ou refletir um problema}
  \definition{v.}{relatar; fazer um relato de}
\end{entry}

\begin{entry}{汇率}{hui4lv4}{5,11}{⽔、⽞}[HSK 4]
  \definition[个]{s.}{taxa de câmbio; relação entre a moeda de um país e a de outro}
\end{entry}

\begin{entry}{会}{hui4}{6}{⼈}[HSK 1,2]
  \definition{adv.}{um momento}
  \definition{s.}{encontro | reunião}
  \definition{suf.}{união | grupo | associação}
  \definition{v.}{poder (ter a habilidade, saber como fazer) | saber | ter habilidade | saber como fazer | ser provável | ter certeza de | encontrar-se | reunir-se}
  \seeref{会}{kuai4}
\end{entry}

\begin{entry}{会首}{hui4shou3}{6,9}{⼈、⾸}
  \definition{s.}{chefe de uma sociedade | patrocinador de uma organização}
\end{entry}

\begin{entry}{会议}{hui4yi4}{6,5}{⼈、⾔}[HSK 3]
  \definition[场,届,个]{s.}{reunião; conferência | conselho; congresso}
\end{entry}

\begin{entry}{会员}{hui4 yuan2}{6,7}{⼈、⼝}[HSK 3]
  \definition[位]{s.}{membro; associado | filiação}
\end{entry}

\begin{entry}{婚礼}{hun1li3}{11,5}{⼥、⽰}[HSK 4]
  \definition[场]{s.}{casamento; núpcias; cerimônia de casamento}
\end{entry}

\begin{entry}{魂}{hun2}{13}{⿁}
  \definition{s.}{alma | espírito | alma imortal (que pode ser separada do corpo)}
\end{entry}

\begin{entry}{混饭}{hun4fan4}{11,7}{⽔、⾷}
  \definition{v.+compl.}{trabalhar para viver}
\end{entry}

\begin{entry}{混乱}{hun4luan4}{11,7}{⽔、⼄}
  \definition{adj.}{confuso | caótico | desordenado}
  \definition{s.}{caos}
\end{entry}

\begin{entry}{和}{huo2}{8}{⼝}
  \definition{v.}{combinar uma substância em pó (farinha, gesso, etc.) com água}
  \seeref{和}{he2}
  \seeref{和}{he4}
  \seeref{和}{hu2}
  \seeref{和}{huo4}
\end{entry}

\begin{entry}{活}{huo2}{9}{⽔}[HSK 3]
  \definition{adj.}{vivo; vivendo | vívido; animado; ativo | móvel; em movimento}
  \definition{adv.}{exatamente; simplesmente}
  \definition{s.}{trabalho | produto}
  \definition{v.}{viver | salvar (a vida de uma pessoa)}
\end{entry}

\begin{entry}{活动}{huo2dong4}{9,6}{⽔、⼒}[HSK 2]
  \definition[项,个]{s.}{atividade | evento | campanha}
  \definition{v.}{exercer | operar}
\end{entry}

\begin{entry}{活力}{huo2li4}{9,2}{⽔、⼒}
  \definition{s.}{energia | vitalidade | vigor | força vital}
\end{entry}

\begin{entry}{活路}{huo2lu4}{9,13}{⽔、⾜}
  \definition{s.}{maneira de sobreviver | meio de subsistência}
  \seeref{活路}{huo2lu5}
\end{entry}

\begin{entry}{活路}{huo2lu5}{9,13}{⽔、⾜}
  \definition{s.}{labor | trabalho físico}
  \seeref{活路}{huo2lu4}
\end{entry}

\begin{entry}{活着}{huo2zhe5}{9,11}{⽔、⽬}
  \definition{adj.}{vivo}
\end{entry}

\begin{entry}{火}{huo3}{4}{⽕}[HSK 3,4][Kangxi 86]
  \definition*{s.}{sobrenome Huo}
  \definition{adj.}{ardente; flamejante; vermelho como fogo | efervescente; próspero}
  \definition{adv.}{urgentemente}
  \definition{clas.}{para unidades militares (antigo)}
  \definition{s.}{fogo | armas de fogo; munições | calor interno (uma das seis causas de doenças) | a ação de lutar}
  \definition{v.}{ficar com raiva; perder a paciência}
\end{entry}

\begin{entry}{火柴}{huo3chai2}{4,10}{⽕、⽊}
  \definition[根,盒]{s.}{fósforo (palito de fósforo)}
\end{entry}

\begin{entry}{火车}{huo3 che1}{4,4}{⽕、⾞}[HSK 1]
  \definition[列,节,班,趟]{s.}{trem | comboio}
\end{entry}

\begin{entry}{火车司机}{huo3che1 si1ji1}{4,4,5,6}{⽕、⾞、⼝、⽊}
  \definition{s.}{maquinista de trem}
\end{entry}

\begin{entry}{火海}{huo3hai3}{4,10}{⽕、⽔}
  \definition{s.}{um mar de chamas}
\end{entry}

\begin{entry}{伙}{huo3}{6}{⼈}[HSK 4]
  \definition{clas.}{grupo; multidão; banda}
  \definition{s.}{iguaria; alimentação; refeições | parceiro; companheiro | coletivo de colegas}
  \definition{v.}{combinar; unir}
\end{entry}

\begin{entry}{伙伴}{huo3ban4}{6,7}{⼈、⼈}[HSK 4]
  \definition[个,位,群]{s.}{parceiro; companheiro; antigo sistema militar de dez pessoas para uma fogueira, o chefe da fogueira, uma pessoa encarregada de cozinhar, com a fogueira é chamado de parceiro da fogueira, agora se refere à participação comum em uma determinada organização ou engajada em certas atividades}
\end{entry}

\begin{entry}{和}{huo4}{8}{⼝}
  \definition{clas.}{para enxágues de roupas | para fervuras de ervas medicinais}
  \definition{v.}{misturar (ingredientes) | misturar}
  \seeref{和}{he2}
  \seeref{和}{he4}
  \seeref{和}{hu2}
  \seeref{和}{huo2}
\end{entry}

\begin{entry}{或}{huo4}{8}{⼽}[HSK 2]
  \definition{conj.}{ou | ou\dots ou\dots}
\end{entry}

\begin{entry}{或许}{huo4xu3}{8,6}{⼽、⾔}[HSK 4]
  \definition{adv.}{talvez; possivelmente; receio; não tenho certeza}
\end{entry}

\begin{entry}{或者}{huo4zhe3}{8,8}{⼽、⽼}[HSK 2]
  \definition{conj.}{ou (usado em expressões afirmativas)}
\end{entry}

\begin{entry}{货}{huo4}{8}{⾙}[HSK 4]
  \definition{s.}{dinheiro; moeda | bens; mercadorias; \emph{commodity} | palavras insultuosas dirigidas a alguém; maldição; xingamento}
\end{entry}

\begin{entry}{货车}{huo4che1}{8,4}{⾙、⾞}
  \definition{s.}{caminhão | van | vagão de carga}
\end{entry}

\begin{entry}{获}{huo4}{10}{⾋}[HSK 4]
  \definition*{s.}{sobrenome Huo}
  \definition{v.}{capturar; pegar | obter; ganhar; colher | colher; ceifar}
\end{entry}

\begin{entry}{获得}{huo4de2}{10,11}{⾋、⼻}[HSK 4]
  \definition{v.}{adquirir; ganhar; obter; alcançar}
\end{entry}

\begin{entry}{获奖}{huo4 jiang3}{10,9}{⾋、⼤}[HSK 4]
  \definition{v.}{ganhar prêmio; ser recompensado; ganhar um prêmio; receber um prêmio}
\end{entry}

\begin{entry}{获取}{huo4 qu3}{10,8}{⾋、⼜}[HSK 4]
  \definition{v.}{adquirir; obter; ganhar; colher}
\end{entry}

\begin{entry}{惑星}{huo4xing1}{12,9}{⼼、⽇}
  \definition{s.}{planeta}
  \seealsoref{行星}{xing2xing1}
\end{entry}

%%%%% EOF %%%%%

