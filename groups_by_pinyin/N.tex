%%%
%%% N
%%%

\section*{N}\addcontentsline{toc}{section}{N}

\begin{entry}{那}{na1}{6}[Radical 邑]
  \definition*{s.}{sobrenome Na}
\end{entry}

\begin{entry}{拿}{na2}{10}[Radical 手]
  \definition{part.}{usado da mesma forma que 把: para marcar o seguinte substantivo seguinte como objeto direto}
  \definition{v.}{segurar | tomar | pegar em}
\end{entry}

\begin{entry}{那}{na3}{6}[Radical 邑]
  \variantof{哪}
\end{entry}

\begin{entry}{哪}{na3}{9}[Radical 口]
  \definition{prep.}{que? | qual?}
\end{entry}

\begin{entry}{哪国人}{na3 guo2ren2}{9,8,2}
  \definition{expr.}{de qual país?}
\end{entry}

\begin{entry}{哪里}{na3li3}{9,7}
  \definition{adv.}{onde?}
\end{entry}

\begin{entry}{哪怕}{na3pa4}{9,8}
  \definition{conj.}{mesmo se/embora | até | não importa como}
\end{entry}

\begin{entry}{哪儿}{na3r5}{9,2}
  \definition{adv.}{onde?}
\end{entry}

\begin{entry}{哪些}{na3xie1}{9,8}
  \definition{pron.}{quais?}
\end{entry}

\begin{entry}{那}{na4}{6}[Radical 邑]
  \definition{conj.}{nessa situação | nesse caso}
  \definition{pron.}{aquele | aquilo}
\end{entry}

\begin{entry}{那里}{na4li5}{6,7}
  \definition{pron.}{lá | ali}
\end{entry}

\begin{entry}{那么}{na4me5}{6,3}
  \definition{adv.}{então | como aquele | dessa maneira}
\end{entry}

\begin{entry}{那末}{na4me5}{6,5}
  \variantof{那么}
\end{entry}

\begin{entry}{那麽}{na4me5}{6,14}
  \variantof{那么}
\end{entry}

\begin{entry}{那儿}{na4r5}{6,2}
  \definition{pron.}{lá | ali}
\end{entry}

\begin{entry}{那些}{na4xie1}{6,8}
  \definition{pron.}{aqueles}
\end{entry}

\begin{entry}{奶奶}{nai3nai5}{5,5}
  \definition[位]{s.}{avó (paterna) | (respeitoso) dona da casa}
\end{entry}

\begin{entry}{耐心}{nai4xin1}{9,4}
  \definition{s.}{paciência}
  \definition{v.}{ser paciente}
\end{entry}

\begin{entry}{男}{nan2}{7}[Radical 田]
  \definition{adj.}{masculino}
  \definition{s.}{Barão, o mais baixo de cinco ordens de nobreza}
\end{entry}

\begin{entry}{男孩儿}{nan2hai2r5}{7,9,2}
  \definition{s.}{menino | rapaz}
\end{entry}

\begin{entry}{男朋友}{nan2peng2you5}{7,8,4}
  \definition{s.}{namorado}
\end{entry}

\begin{entry}{南边}{nan2bian5}{9,5}
  \definition{adv.}{sul | lado sul | parte sul | ao sul de}
\end{entry}

\begin{entry}{南方}{nan2fang1}{9,4}
  \definition{s.}{sul | o Sul | a parte sul do país}
\end{entry}

\begin{entry}{南极}{nan2ji2}{9,7}
  \definition*{s.}{Antártico | Pólo Sul}
  \definition{s.}{pólo sul magnético}
\end{entry}

\begin{entry}{南面}{nan2mian4}{9,9}
  \definition{s.}{sul | lado sul}
\end{entry}

\begin{entry}{难}{nan2}{10}[Radical 隹]
  \definition{adj.}{difícil}
  \definition{s.}{dificuldade}
  \seeref{难}{nan4}
\end{entry}

\begin{entry}{难道}{nan2dao4}{10,12}
  \definition{adv.}{indica uma pergunta retórica | certamente não significa que\dots | é possível que\dots}
\end{entry}

\begin{entry}{难度}{nan2du4}{10,9}
  \definition{s.}{grau de dificuldade}
\end{entry}

\begin{entry}{难}{nan4}{10}[Radical 隹]
  \definition{s.}{desastre}
  \definition{v.}{repreender}
  \seeref{难}{nan2}
\end{entry}

\begin{entry}{孬}{nao1}{10}[Radical 子]
  \definition{adj.}{(dialeto) não (é) bom (contração de 不+好)}
\end{entry}

\begin{entry}{脑袋}{nao3dai5}{10,11}
  \definition[颗,个]{s.}{cabeça | crânio | cérebro | capacidade mental}
\end{entry}

\begin{entry}{脑瓜}{nao3gua1}{10,5}
  \definition{s.}{crânio | cérebro | cabeça | mente | mentalidade | ideia}
  \seealsoref{脑瓜子}{nao3gua1zi5}
\end{entry}

\begin{entry}{脑瓜子}{nao3gua1zi5}{10,5,3}
  \definition{s.}{crânio | cérebro | cabeça | mente | mentalidade | ideia}
  \seealsoref{脑瓜}{nao3gua1}
\end{entry}

\begin{entry}{呢}{ne5}{8}[Radical 口]
  \definition{part.}{(no final de uma frase declarativa) partícula que indica a continuação de um estado ou ação |  partícula para perguntar sobre a localização (``Onde está\dots?'') | partícula indicando  afirmação forte | partícula indicando que uma pergunta feita anteriormente deve ser aplicada à palavra anterior (``E quanto a\dots?'', ``E\dots?'') | partícula sinalizando uma pausa, para enfatizar as palavras anteriores e permitir que o ouvinte tenha tempo para compreendê-las (``ok?'', ``você está comigo ?'')}
  \seeref{呢}{ni2}
\end{entry}

\begin{entry}{内存}{nei4cun2}{4,6}
  \definition{s.}{armazenamento interno | memória do computador | RAM (\emph{random access memory})}
  \seealsoref{随机存取存储器}{sui2ji1cun2qu3cun2chu3qi4}
  \seealsoref{随机存取记忆体}{sui2ji1cun2qu3ji4yi4ti3}
\end{entry}

\begin{entry}{内燃机}{nei4ran2ji1}{4,16,6}
  \definition{s.}{motor de combustão interna}
\end{entry}

\begin{entry}{内省}{nei4xing3}{4,9}
  \definition{s.}{introspecção}
  \definition{v.}{refletir sobre si mesmo}
\end{entry}

\begin{entry}{能}{neng2}{10}[Radical 肉]
  \definition*{s.}{sobrenome Neng}
  \definition{adv.}{talvez}
  \definition{s.}{(física)nenergia | habilidade}
  \definition{v.}{poder | ser capaz de}
\end{entry}

\begin{entry}{能干}{neng2gan4}{10,3}
  \definition{adj.}{capaz | competente}
\end{entry}

\begin{entry}{能够}{neng2gou4}{10,11}
  \definition{v.}{ser capaz de}
\end{entry}

\begin{entry}{能上能下}{neng2shang4neng2xia4}{10,3,10,3}
  \definition{s.}{pronto para aceitar qualquer trabalho, alto ou baixo}
\end{entry}

\begin{entry}{呢}{ni2}{8}[Radical 口]
  \definition{s.}{material de lã}
  \seeref{呢}{ne5}
\end{entry}

\begin{entry}{泥}{ni2}{8}[Radical 水]
  \definition{s.}{lama | argila | pasta | polpa}
  \seeref{泥}{ni4}
\end{entry}

\begin{entry}{泥潭}{ni2tan2}{8,15}
  \definition{s.}{atoleiro | lamaçal | charco | pântano}
\end{entry}

\begin{entry}{伲}{ni3}{7}[Radical 人]
  \variantof{你}
\end{entry}

\begin{entry}{你}{ni3}{7}[Radical 人]
  \definition{pron.}{você (informal) | tu | te | ti | contigo}
  \seeref{您}{nin2}
\end{entry}

\begin{entry}{你的}{ni3 de5}{7,8}
  \definition{pron.}{seu}
\end{entry}

\begin{entry}{你好}{ni3hao3}{7,6}
  \definition{interj.}{Olá! | Oi!}
\end{entry}

\begin{entry}{你们}{ni3men5}{7,5}
  \definition{pron.}{vocês (informal) | vós | vos | convosco}
\end{entry}

\begin{entry}{你们的}{ni3men5 de5}{7,5,8}
  \definition{pron.}{vossos}
\end{entry}

\begin{entry}{袮}{ni3}{10}[Radical 衣]
  \definition{pron.}{Você, Tu (divindade)}
  \variantof{你}
\end{entry}

\begin{entry}{泥}{ni4}{8}[Radical 水]
  \definition{adj.}{contido}
  \seeref{泥}{ni2}
\end{entry}

\begin{entry}{逆境}{ni4jing4}{9,14}
  \definition{s.}{adversidade | tribulação}
\end{entry}

\begin{entry}{年}{nian2}{6}[Radical 干]
  \definition*{s.}{sobrenome Nian}
\end{entry}

\begin{entry}{年货}{nian2huo4}{6,8}
  \definition{s.}{mercadorias vendidas no Ano Novo Chinês}
\end{entry}

\begin{entry}{年级}{nian2ji2}{6,6}
  \definition[个]{s.}{classe | ano (escola)}
\end{entry}

\begin{entry}{年纪}{nian2ji4}{6,6}
  \definition[个]{s.}{grau | nota | classe | categoria | graduação | ano (na escola, faculdade, etc.)}
\end{entry}

\begin{entry}{年轻}{nian2qing1}{6,9}
  \definition{adj.}{jovem}
\end{entry}

\begin{entry}{碾碎}{nian3sui4}{15,13}
  \definition{v.}{pulverizar | esmagar}
\end{entry}

\begin{entry}{鸟儿}{niao3r5}{5,2}
  \definition[只]{s.}{pássaro | ave}
\end{entry}

\begin{entry}{尿}{niao4}{7}[Radical 尸]
  \definition[泡]{s.}{urina}
  \definition{v.}{urinar}
  \seeref{尿}{sui1}
\end{entry}

\begin{entry}{您}{nin2}{11}[Radical 心]
  \definition{pron.}{você (formal) | tu | te | ti | contigo}
  \seeref{你}{ni3}
\end{entry}

\begin{entry}{宁}{ning2}{5}[Radical 宀]
  \definition*{s.}{sobrenome Ning}
  \definition{adj.}{calmo, pacífico, sereno | saudável}
  \seeref{宁}{ning4}
\end{entry}

\begin{entry}{柠檬}{ning2meng2}{9,17}
  \definition{s.}{limão}
\end{entry}

\begin{entry}{拧开}{ning3kai1}{8,4}
  \definition{v.}{desaparafusar | desatarrachar | torcer (uma tampa) | abrir (uma torneira) | ligar (girando um botão) | girar (maçaneta da porta)}
\end{entry}

\begin{entry}{宁}{ning4}{5}[Radical 宀]
  \definition{conj.}{mais\dots do que\dots, melhor\dots do que\dots}
  \seeref{宁}{ning2}
\end{entry}

\begin{entry}{宁可}{ning4ke3}{5,5}
  \definition{conj.}{mais\dots do que\dots | melhor\dots do que\dots}
\end{entry}

\begin{entry}{宁可……也不……}{ning4ke3 ye3bu4}{5,5,3,4}
  \definition{conj.}{em vez de\dots}
\end{entry}

\begin{entry}{宁可……也要……}{ning4ke3 ye3yao4}{5,5,3,9}
  \definition{conj.}{mesmo que tenhamos que\dots nós iremos\dots}
\end{entry}

\begin{entry}{宁肯}{ning4ken3}{5,8}
  \definition{conj.}{mais\dots do que\dots, melhor\dots do que\dots}
\end{entry}

\begin{entry}{宁愿}{ning4yuan4}{5,14}
  \definition{conj.}{mais\dots do que\dots, melhor\dots do que\dots}
\end{entry}

\begin{entry}{牛}{niu2}{4}[Radical 牛][Kangxi 93]
  \definition*{s.}{sobrenome Niu}
  \definition[条,头]{s.}{boi | touro | vaca | (gíria) incrível}
\end{entry}

\begin{entry}{牛顿}{niu2dun4}{4,10}
  \definition*{s.}{Newton (nome) | newton (N, unidade de força do SI)}
\end{entry}

\begin{entry}{牛郎织女}{niu2lang2zhi1nv3}{4,8,8,3}
  \definition*{s.}{Vaqueiro e Tecelã (personagens de contos folclóricos) | amantes separados | Altair e Vega (estrelas)}
\end{entry}

\begin{entry}{牛奶}{niu2nai3}{4,5}
  \definition[瓶,杯]{s.}{leite de vaca}
\end{entry}

\begin{entry}{牛人}{niu2ren2}{4,2}
  \definition{s.}{(coloquial) o cara | verdadeiro especialista | \emph{badass}}
\end{entry}

\begin{entry}{牛肉}{niu2rou4}{4,6}
  \definition{s.}{carne de vaca | bife}
\end{entry}

\begin{entry}{牛仔裤}{niu2zai3ku4}{4,5,12}
  \definition[条]{s.}{calça de ganga, jeans}
\end{entry}

\begin{entry}{农村}{nong2cun1}{6,7}
  \definition[个]{s.}{campo rural | aldeia | povoação rústica}
\end{entry}

\begin{entry}{浓}{nong2}{9}[Radical 水]
  \definition{adj.}{concentrado | denso | forte (cheiro, etc.)}
\end{entry}

\begin{entry}{努力}{nu3li4}{7,2}
  \definition{adj.}{diligente | aplicado}
  \definition{s.}{esforçar-se | se esforçar}
\end{entry}

\begin{entry}{怒骂}{nu4ma4}{9,9}
  \definition{v.}{praguejar de raiva}
\end{entry}

\begin{entry}{暖}{nuan3}{13}[Radical 日]
  \definition{adj.}{quente}
  \definition{v.}{esquentar}
\end{entry}

\begin{entry}{暖和}{nuan3huo5}{13,8}
  \definition{adj.}{morno; agradável e quente}
\end{entry}

\begin{entry}{暖气}{nuan3qi4}{13,4}
  \definition{s.}{aquecimento central | aquecedor | ar quente}
\end{entry}

\begin{entry}{那}{nuo2}{6}[Radical 邑]
  \definition*{s.}{sobrenome Nuo}
\end{entry}

\begin{entry}{诺贝尔奖}{nuo4bei4'er3 jiang3}{10,4,5,9}
  \definition*{s.}{Prêmio Nobel}
\end{entry}

\begin{entry}{诺奖}{nuo4jiang3}{10,9}
  \definition*{s.}{Prêmio Nobel, abreviação de 诺贝尔奖}
  \seeref{诺贝尔奖}{nuo4bei4'er3 jiang3}
\end{entry}

\begin{entry}{女}{nv3}{3}[Radical 女][Kangxi 38]
  \definition{adj.}{feminino}
\end{entry}

\begin{entry}{女儿}{nv3'er2}{3,2}
  \definition{s.}{filha}
\end{entry}

\begin{entry}{女孩}{nv3hai2}{3,9}
  \definition{s.}{menina | garota}
\end{entry}

\begin{entry}{女朋友}{nv3peng2you5}{3,8,4}
  \definition{s.}{namorada}
\end{entry}

\begin{entry}{女王}{nv3wang2}{3,4}
  \definition{s.}{rainha}
\end{entry}

\begin{entry}{女婿}{nv3xu5}{3,12}
  \definition{s.}{marido da filha}
\end{entry}

%%%%% EOF %%%%%

