%%%
%%% N
%%%

\section*{N}\addcontentsline{toc}{section}{N}

\begin{entry}{那}{na1}{6}{⾢}
  \definition*{s.}{sobrenome Na}
\end{entry}

\begin{entry}{拿}{na2}{10}{⼿}[HSK 1]
  \definition{part.}{usado da mesma forma que 把: para marcar o seguinte substantivo seguinte como objeto direto}
  \definition{v.}{segurar | tomar | pegar em}
\end{entry}

\begin{entry}{拿出}{na2 chu1}{10,5}{⼿、⼐}[HSK 2]
  \definition{v.}{apresentar (evidências) | prover | apresentar (uma proposta) | colocar para fora | retirar}
\end{entry}

\begin{entry}{拿到}{na2 dao4}{10,8}{⼿、⼑}[HSK 2]
  \definition{v.}{pegar | obter}
\end{entry}

\begin{entry}{那}{na3}{6}{⾢}
  \variantof{哪}
\end{entry}

\begin{entry}{哪}{na3}{9}{⼝}[HSK 1,4]
  \definition{adv.}{para expressar uma pergunta retórica}
  \definition{pron.}{qual?; o que? | qualquer; ser usado em um sentido geral}
  \seeref{哪}{na5}
  \seeref{哪}{nei3}
\end{entry}

\begin{entry}{哪个}{na3ge5}{9,3}{⼝、⼈}
  \definition{pron.}{qual deles (pergunta sobre o objeto) | quem (perguntar a alguém ou indicar qualquer pessoa)}
\end{entry}

\begin{entry}{哪国人}{na3 guo2ren2}{9,8,2}{⼝、⼞、⼈}
  \definition{expr.}{de qual país?}
\end{entry}

\begin{entry}{哪里}{na3 li3}{9,7}{⼝、⾥}[HSK 1]
  \definition{adv.}{onde?}
\end{entry}

\begin{entry}{哪怕}{na3pa4}{9,8}{⼝、⼼}[HSK 4]
  \definition{conj.}{mesmo; mesmo se; mesmo que; não importa o quão}
\end{entry}

\begin{entry}{哪儿}{na3r5}{9,2}{⼝、⼉}[HSK 1]
  \definition{adv.}{onde?}
\end{entry}

\begin{entry}{哪些}{na3xie1}{9,8}{⼝、⼆}[HSK 1]
  \definition{pron.}{quais?}
\end{entry}

\begin{entry}{那}{na4}{6}{⾢}[HSK 1,2]
  \definition{conj.}{nessa situação | nesse caso}
  \definition{pron.}{aquele | aquilo}
\end{entry}

\begin{entry}{那边}{na4bian5}{6,5}{⾢、⾡}[HSK 1]
  \definition{pron.}{ali | acolá}
\end{entry}

\begin{entry}{那会儿}{na4 hui4r5}{6,6,2}{⾢、⼈、⼉}[HSK 2]
  \definition{pron.}{então | naquela época}
\end{entry}

\begin{entry}{那里}{na4 li3}{6,7}{⾢、⾥}[HSK 1]
  \definition{pron.}{lá | ali}
\end{entry}

\begin{entry}{那么}{na4 me5}{6,3}{⾢、⼃}[HSK 2]
  \definition{adv.}{então | como aquele | dessa maneira}
\end{entry}

\begin{entry}{那麽}{na4 me5}{6,14}{⾢、⿇}
  \variantof{那么}
\end{entry}

\begin{entry}{那儿}{na4r5}{6,2}{⾢、⼉}[HSK 1]
  \definition{pron.}{lá | ali}
\end{entry}

\begin{entry}{那时}{na4 shi2}{6,7}{⾢、⽇}[HSK 2]
  \definition{pron.}{então | naquela época | naqueles dias}
\end{entry}

\begin{entry}{那时候}{na4 shi2 hou5}{6,7,10}{⾢、⽇、⼈}[HSK 2]
  \definition{adv.}{naquela hora}
\end{entry}

\begin{entry}{那些}{na4xie1}{6,8}{⾢、⼆}[HSK 1]
  \definition{pron.}{aqueles}
\end{entry}

\begin{entry}{那样}{na4 yang4}{6,10}{⾢、⽊}[HSK 2]
  \definition{pron.}{assim | tal | como esse | desse tipo}
\end{entry}

\begin{entry}{哪}{na5}{9}{⼝}
  \definition{part.}{usado depois de uma palavra com a terminação -n, é equivalente a ``啊''}
  \seeref{哪}{na3}
  \seeref{哪}{nei3}
  \seealsoref{啊}{a5}
\end{entry}

\begin{entry}{奶}{nai3}{5}{⼥}[HSK 1]
  \definition[杯,滴,瓶,只,桶]{s.}{seios | leite}
  \definition{v.}{amamentar}
\end{entry}

\begin{entry}{奶茶}{nai3 cha2}{5,9}{⼥、⾋}[HSK 3]
  \definition[杯]{s.}{chá com leite}
\end{entry}

\begin{entry}{奶奶}{nai3nai5}{5,5}{⼥、⼥}[HSK 1]
  \definition[位]{s.}{avó (paterna) | (respeitoso) dona da casa}
\end{entry}

\begin{entry}{耐心}{nai4xin1}{9,4}{⽽、⼼}[HSK 5]
  \definition{adj.}{paciente}
  \definition{s.}{paciência; não se incomoda com as dificuldades e tem um caráter tolerante}
  \definition{v.}{ser paciente}
\end{entry}

\begin{entry}{男}{nan2}{7}{⽥}[HSK 1]
  \definition{adj.}{masculino}
  \definition{s.}{Barão, o mais baixo de cinco ordens de nobreza}
\end{entry}

\begin{entry}{男孩儿}{nan2hai2r5}{7,9,2}{⽥、⼦、⼉}[HSK 1]
  \definition{s.}{menino | rapaz}
\end{entry}

\begin{entry}{男女}{nan2 nv3}{7,3}{⽥、⼥}[HSK 4]
  \definition{s.}{homens e mulheres; masculino e feminino}
\end{entry}

\begin{entry}{男朋友}{nan2peng2you5}{7,8,4}{⽥、⽉、⼜}[HSK 1]
  \definition{s.}{namorado}
\end{entry}

\begin{entry}{男人}{nan2ren2}{7,2}{⽥、⼈}[HSK 1]
  \definition[个]{s.}{um homem | um macho | cavalheiro | marido}
\end{entry}

\begin{entry}{男生}{nan2sheng1}{7,5}{⽥、⽣}[HSK 1]
  \definition[个]{s.}{aluno | estudante do sexo masculino}
\end{entry}

\begin{entry}{男士}{nan2 shi4}{7,3}{⽥、⼠}[HSK 4]
  \definition{s.}{cavalheiro; \emph{gentleman}}
\end{entry}

\begin{entry}{男性}{nan2 xing4}{7,8}{⽥、⼼}[HSK 5]
  \definition{s.}{masculino; homem; masculinidade}
\end{entry}

\begin{entry}{男子}{nan2zi3}{7,3}{⽥、⼦}[HSK 3]
  \definition[名]{s.}{homem; macho}
\end{entry}

\begin{entry}{南}{nan2}{9}{⼗}[HSK 1]
  \definition*{s.}{sobrenome Nan}
  \definition{s.}{sul}
\end{entry}

\begin{entry}{南北}{nan2 bei3}{9,5}{⼗、⼔}[HSK 5]
  \definition{s.}{norte e sul | de norte a sul}
\end{entry}

\begin{entry}{南边}{nan2bian5}{9,5}{⼗、⾡}[HSK 1]
  \definition{adv.}{sul | lado sul | parte sul | ao sul de}
\end{entry}

\begin{entry}{南部}{nan2 bu4}{9,10}{⼗、⾢}[HSK 3]
  \definition{s.}{parte sul; sul | a parte sul}
\end{entry}

\begin{entry}{南方}{nan2 fang1}{9,4}{⼗、⽅}[HSK 2]
  \definition{s.}{sul | o Sul | a parte sul do país}
\end{entry}

\begin{entry}{南极}{nan2ji2}{9,7}{⼗、⽊}[HSK 5]
  \definition*{s.}{Polo Sul; Polo Antártico | Polo sul magnético}
  \definition{s.}{pólo sul magnético}
\end{entry}

\begin{entry}{南面}{nan2mian4}{9,9}{⼗、⾯}
  \definition{s.}{sul | lado sul}
\end{entry}

\begin{entry}{难}{nan2}{10}{⾫}[HSK 1]
  \definition{adj.}{difícil}
  \definition{s.}{dificuldade}
  \seeref{难}{nan4}
\end{entry}

\begin{entry}{难道}{nan2dao4}{10,12}{⾫、⾡}[HSK 3]
  \definition{adv.}{indica uma pergunta retórica | certamente não significa que\dots?; é possível que\dots?; não me diga\dots; poderia ser que\dots?}
\end{entry}

\begin{entry}{难得}{nan2de2}{10,11}{⾫、⼻}[HSK 5]
  \definition{adj.}{raro; difícil de encontrar; difícil de obter ou realizar, indicando que é valioso}
  \definition{adv.}{raramente; com pouca frequência}
\end{entry}

\begin{entry}{难度}{nan2 du4}{10,9}{⾫、⼴}[HSK 3]
  \definition{s.}{dificuldade; grau de dificuldade}
\end{entry}

\begin{entry}{难过}{nan2guo4}{10,6}{⾫、⾡}[HSK 2]
  \definition{adj.}{triste | ruim | pesaroso | arrependido | difícil}
\end{entry}

\begin{entry}{难看}{nan2 kan4}{10,9}{⾫、⽬}[HSK 2]
  \definition{adj.}{feio | antiestético | vergonhoso | embaraçoso | vergonhoso}
\end{entry}

\begin{entry}{难受}{nan2shou4}{10,8}{⾫、⼜}[HSK 2]
  \definition{adj.}{sofrer dor | sentir-se mal | desconfortável | sentir-se infeliz}
\end{entry}

\begin{entry}{难题}{nan2 ti2}{10,15}{⾫、⾴}[HSK 2]
  \definition[出]{s.}{desafio | problema difícil | pergunta difícil}
\end{entry}

\begin{entry}{难听}{nan2 ting1}{10,7}{⾫、⼝}[HSK 2]
  \definition{adj.}{desagradável de ouvir | ofensivo | grosseiro | escandaloso}
\end{entry}

\begin{entry}{难以}{nan2 yi3}{10,4}{⾫、⼈}[HSK 5]
  \definition{adj.}{difícil; complicado}
\end{entry}

\begin{entry}{难}{nan4}{10}{⾫}
  \definition{s.}{desastre}
  \definition{v.}{repreender}
  \seeref{难}{nan2}
\end{entry}

\begin{entry}{难免}{nan4mian3}{10,7}{⾫、⼉}[HSK 4]
  \definition{adj.}{inevitável; difícil de evitar}
\end{entry}

\begin{entry}{孬}{nao1}{10}{⼥}
  \definition{adj.}{(dialeto) não (é) bom (contração de 不+好)}
\end{entry}

\begin{entry}{脑袋}{nao3dai5}{10,11}{⾁、⾐}[HSK 4]
  \definition[颗,个]{s.}{cabeça; a parte mais alta do corpo humano ou a parte mais alta de um animal que contém órgãos como a boca, o nariz, os olhos etc. | mente; cérebro; capacidade de pensar, lembrar, etc.}
\end{entry}

\begin{entry}{脑瓜}{nao3gua1}{10,5}{⾁、⽠}
  \definition{s.}{crânio | cérebro | cabeça | mente | mentalidade | ideia}
  \seealsoref{脑瓜子}{nao3gua1zi5}
\end{entry}

\begin{entry}{脑瓜子}{nao3gua1zi5}{10,5,3}{⾁、⽠、⼦}
  \definition{s.}{crânio | cérebro | cabeça | mente | mentalidade | ideia}
  \seealsoref{脑瓜}{nao3gua1}
\end{entry}

\begin{entry}{脑子}{nao3 zi5}{10,3}{⾁、⼦}[HSK 5]
  \definition[个]{s.}{cérebro | mente; cabeça; cérebro; inteligência; poder mental; refere-se à capacidade de pensar, memorizar, raciocinar, etc.; inteligência}
\end{entry}

\begin{entry}{闹}{nao4}{8}{⾾}[HSK 4]
  \definition{adj.}{barulhento}
  \definition{v.}{fazer barulho; provocar problemas | dar vazão (à sua raiva, ressentimento, etc.) | sofrer de; ser incomodado por; ocorrer (um desastre ou coisa ruim) | fazer;  entrar em ação | agitar; perturbar | brincar; fazer bagunça}
\end{entry}

\begin{entry}{闹钟}{nao4 zhong1}{8,9}{⾾、⾦}[HSK 4]
  \definition[个,台,只]{s.}{despertador; relógios capazes de tocar alarmes em horários predeterminados}
\end{entry}

\begin{entry}{呢}{ne5}{8}{⼝}[HSK 1]
  \definition{part.}{(no final de uma frase declarativa) partícula que indica a continuação de um estado ou ação |  partícula para perguntar sobre a localização (``Onde está\dots?'') | partícula indicando  afirmação forte | partícula indicando que uma pergunta feita anteriormente deve ser aplicada à palavra anterior (``E quanto a\dots?'', ``E\dots?'') | partícula sinalizando uma pausa, para enfatizar as palavras anteriores e permitir que o ouvinte tenha tempo para compreendê-las (``ok?'', ``você está comigo ?'')}
  \seeref{呢}{ni2}
\end{entry}

\begin{entry}{哪}{nei3}{9}{⼝}
  \definition{part.}{qual? (interrogativo, seguido de classificador ou numeral-classificador)}
  \seeref{哪}{na3}
  \seeref{哪}{na5}
\end{entry}

\begin{entry}{内}{nei4}{4}{⼌}[HSK 3]
  \definition*{s.}{sobrenome Nei}
  \definition{adj.}{interno; interior}
  \definition{prep.}{dentro}
  \definition{s.}{interior; lado de dentro; parte de dentro | a esposa ou parentes dela}
\end{entry}

\begin{entry}{内部}{nei4bu4}{4,10}{⼌、⾢}[HSK 4]
  \definition{s.}{interior; dentro; interno; dentro de um determinado intervalo}
\end{entry}

\begin{entry}{内存}{nei4cun2}{4,6}{⼌、⼦}
  \definition{s.}{armazenamento interno | memória do computador | RAM (\emph{random access memory})}
  \seealsoref{随机存取存储器}{sui2ji1cun2qu3cun2chu3qi4}
  \seealsoref{随机存取记忆体}{sui2ji1cun2qu3ji4yi4ti3}
\end{entry}

\begin{entry}{内科}{nei4ke1}{4,9}{⼌、⽲}[HSK 4]
  \definition{s.}{medicina geral; clínica geral; clínica médica}
\end{entry}

\begin{entry}{内燃机}{nei4ran2ji1}{4,16,6}{⼌、⽕、⽊}
  \definition{s.}{motor de combustão interna}
\end{entry}

\begin{entry}{内容}{nei4rong2}{4,10}{⼌、⼧}[HSK 3]
  \definition[个]{s.}{conteúdo; substância}
\end{entry}

\begin{entry}{内心}{nei4 xin1}{4,4}{⼌、⼼}[HSK 3]
  \definition{s.}{coração; interior; íntimo do ser}
\end{entry}

\begin{entry}{内省}{nei4xing3}{4,9}{⼌、⽬}
  \definition{s.}{introspecção}
  \definition{v.}{refletir sobre si mesmo}
\end{entry}

\begin{entry}{内在}{nei4zai4}{4,6}{⼌、⼟}[HSK 5]
  \definition{adj.}{intrínseco; algo que existe em si mesmo, mas que não pode ser descoberto através da observação direta | interno; imanente; difícil de perceber}
\end{entry}

\begin{entry}{能}{neng2}{10}{⾁}[HSK 1]
  \definition*{s.}{sobrenome Neng}
  \definition{adv.}{talvez}
  \definition{s.}{(física)nenergia | habilidade}
  \definition{v.}{poder | ser capaz de}
\end{entry}

\begin{entry}{能不能}{neng2 bu4 neng2}{10,4,10}{⾁、⼀、⾁}[HSK 3]
  \definition{adv.}{pode ou não pode\dots?}
\end{entry}

\begin{entry}{能干}{neng2gan4}{10,3}{⾁、⼲}[HSK 4]
  \definition{adj.}{apto; capaz; competente}
\end{entry}

\begin{entry}{能够}{neng2 gou4}{10,11}{⾁、⼣}[HSK 2]
  \definition{v.}{ser capaz de}
\end{entry}

\begin{entry}{能力}{neng2li4}{10,2}{⾁、⼒}[HSK 3]
  \definition{s.}{habilidade; capacidade; aptidão}
\end{entry}

\begin{entry}{能量}{neng2liang4}{10,12}{⾁、⾥}[HSK 5]
  \definition[种]{s.}{energia; quantidade de energia; Uma grandeza física que mede a capacidade da matéria de realizar trabalho | capacidade; competências; capacidade e papel que uma pessoa pode desempenhar}
\end{entry}

\begin{entry}{能上能下}{neng2shang4neng2xia4}{10,3,10,3}{⾁、⼀、⾁、⼀}
  \definition{s.}{pronto para aceitar qualquer trabalho, alto ou baixo}
\end{entry}

\begin{entry}{呢}{ni2}{8}{⼝}
  \definition{s.}{material de lã}
  \seeref{呢}{ne5}
\end{entry}

\begin{entry}{泥}{ni2}{8}{⽔}
  \definition{s.}{lama | argila | pasta | polpa}
  \seeref{泥}{ni4}
\end{entry}

\begin{entry}{泥潭}{ni2tan2}{8,15}{⽔、⽔}
  \definition{s.}{atoleiro | lamaçal | charco | pântano}
\end{entry}

\begin{entry}{你}{ni3}{7}{⼈}[HSK 1]
  \definition{pron.}{você (informal) | tu | te | ti | contigo}
  \seeref{您}{nin2}
\end{entry}

\begin{entry}{你的}{ni3 de5}{7,8}{⼈、⽩}
  \definition{pron.}{seu}
\end{entry}

\begin{entry}{你好}{ni3hao3}{7,6}{⼈、⼥}
  \definition{interj.}{Olá! | Oi!}
\end{entry}

\begin{entry}{你们}{ni3men5}{7,5}{⼈、⼈}[HSK 1]
  \definition{pron.}{vocês (informal) | vós | vos | convosco}
\end{entry}

\begin{entry}{你们的}{ni3men5 de5}{7,5,8}{⼈、⼈、⽩}
  \definition{pron.}{vossos}
\end{entry}

\begin{entry}{伲}{ni4}{7}{⼈}
  \definition{pron.}{(dialeto) eu | meu | nosso | nós}
  \seeref{你}{ni3}
\end{entry}

\begin{entry}{泥}{ni4}{8}{⽔}
  \definition{adj.}{contido}
  \seeref{泥}{ni2}
\end{entry}

\begin{entry}{逆境}{ni4jing4}{9,14}{⾡、⼟}
  \definition{s.}{adversidade | tribulação}
\end{entry}

\begin{entry}{年}{nian2}{6}{⼲}[HSK 1]
  \definition*{s.}{sobrenome Nian}
  \definition[个]{clas./s.}{ano}
\end{entry}

\begin{entry}{年初}{nian2 chu1}{6,7}{⼲、⾐}[HSK 3]
  \definition{s.}{o começo do ano}
\end{entry}

\begin{entry}{年代}{nian2dai4}{6,5}{⼲、⼈}[HSK 3]
  \definition[个]{s.}{idade; anos; tempo | uma década de um século}
\end{entry}

\begin{entry}{年底}{nian2 di3}{6,8}{⼲、⼴}[HSK 3]
  \definition[个]{s.}{fim de ano; o fim do ano}
\end{entry}

\begin{entry}{年度}{nian2du4}{6,9}{⼲、⼴}[HSK 5]
  \definition{s.}{ano; de acordo com a natureza e as necessidades de um negócio, há um prazo de doze meses com data de início e término definidas}
\end{entry}

\begin{entry}{年货}{nian2huo4}{6,8}{⼲、⾙}
  \definition{s.}{mercadorias vendidas no Ano Novo Chinês}
\end{entry}

\begin{entry}{年级}{nian2ji2}{6,6}{⼲、⽷}[HSK 2]
  \definition[个]{s.}{classe | ano (escola)}
\end{entry}

\begin{entry}{年纪}{nian2ji4}{6,6}{⼲、⽷}[HSK 3]
  \definition{s.}{era; época; idade}
\end{entry}

\begin{entry}{年龄}{nian2ling2}{6,13}{⼲、⿒}[HSK 5]
  \definition[个]{s.}{idade; animais, plantas e outros seres vivos vivem e crescem no mundo durante um determinado número de anos}
\end{entry}

\begin{entry}{年前}{nian2 qian2}{6,9}{⼲、⼑}[HSK 5]
  \definition{s.}{antes do final do ano; antes do ano novo}
\end{entry}

\begin{entry}{年轻}{nian2qing1}{6,9}{⼲、⾞}[HSK 2]
  \definition{adj.}{jovem}
\end{entry}

\begin{entry}{碾碎}{nian3sui4}{15,13}{⽯、⽯}
  \definition{v.}{pulverizar | esmagar}
\end{entry}

\begin{entry}{念}{nian4}{8}{⼼}[HSK 3]
  \definition*{s.}{sobrenome Nian}
  \definition{num.}{vinte; 20}
  \definition{s.}{ideia; pensamento}
  \definition{v.}{ler em voz alta | estudar; frequentar a escola | considerar; levar em conta | sentir falta; pensar em}
\end{entry}

\begin{entry}{鸟}{niao3}{5}{⿃}[HSK 2][Kangxi 196]
  \definition[只,群]{s.}{pássaro}
  \seeref{鸟}{diao3}
\end{entry}

\begin{entry}{鸟儿}{niao3r5}{5,2}{⿃、⼉}
  \definition[只]{s.}{pássaro | ave}
\end{entry}

\begin{entry}{尿}{niao4}{7}{⼫}
  \definition[泡]{s.}{urina}
  \definition{v.}{urinar}
  \seeref{尿}{sui1}
\end{entry}

\begin{entry}{您}{nin2}{11}{⼼}[HSK 1]
  \definition{pron.}{você (formal) | tu | te | ti | contigo}
  \seeref{你}{ni3}
\end{entry}

\begin{entry}{宁}{ning2}{5}{⼧}
  \definition*{s.}{sobrenome Ning}
  \definition{adj.}{calmo, pacífico, sereno | saudável}
  \seeref{宁}{ning4}
\end{entry}

\begin{entry}{宁静}{ning2 jing4}{5,14}{⼧、⾭}[HSK 4]
  \definition{adj.}{calmo; tranquilo; pacífico}
\end{entry}

\begin{entry}{柠檬}{ning2meng2}{9,17}{⽊、⽊}
  \definition{s.}{limão}
\end{entry}

\begin{entry}{拧开}{ning3kai1}{8,4}{⼿、⼶}
  \definition{v.}{desaparafusar | desatarrachar | torcer (uma tampa) | abrir (uma torneira) | ligar (girando um botão) | girar (maçaneta da porta)}
\end{entry}

\begin{entry}{宁}{ning4}{5}{⼧}
  \definition{conj.}{mais\dots do que\dots, melhor\dots do que\dots}
  \seeref{宁}{ning2}
\end{entry}

\begin{entry}{宁可}{ning4ke3}{5,5}{⼧、⼝}
  \definition{conj.}{mais\dots do que\dots | melhor\dots do que\dots}
\end{entry}

\begin{entry}{宁可……也不……}{ning4ke3 ye3bu4}{5,5,3,4}{⼧、⼝、⼄、⼀}
  \definition{conj.}{em vez de\dots}
\end{entry}

\begin{entry}{宁可……也要……}{ning4ke3 ye3yao4}{5,5,3,9}{⼧、⼝、⼄、⾑}
  \definition{conj.}{mesmo que tenhamos que\dots nós iremos\dots}
\end{entry}

\begin{entry}{宁肯}{ning4ken3}{5,8}{⼧、⾁}
  \definition{conj.}{mais\dots do que\dots, melhor\dots do que\dots}
\end{entry}

\begin{entry}{宁愿}{ning4yuan4}{5,14}{⼧、⽕}
  \definition{conj.}{mais\dots do que\dots, melhor\dots do que\dots}
\end{entry}

\begin{entry}{牛}{niu2}{4}{⽜}[HSK 3,5][Kangxi 93]
  \definition*{s.}{sobrenome Niu}
  \definition{adj.}{muito capaz ou bom | teimoso; arrogante}
  \definition{clas.}{Newton (medida física de força)}
  \definition[头]{s.}{gado; boi | niu (nona das vinte e oito constelações em que a esfera celeste foi dividida, consistindo de seis estrelas, três em Áries e três em Sagitário)}
\end{entry}

\begin{entry}{牛顿}{niu2dun4}{4,10}{⽜、⾴}
  \definition*{s.}{Newton (nome) | newton (N, unidade de força do SI)}
\end{entry}

\begin{entry}{牛郎织女}{niu2lang2zhi1nv3}{4,8,8,3}{⽜、⾢、⽷、⼥}
  \definition*{s.}{Vaqueiro e Tecelã (personagens de contos folclóricos) | amantes separados | Altair e Vega (estrelas)}
\end{entry}

\begin{entry}{牛奶}{niu2nai3}{4,5}{⽜、⼥}[HSK 1]
  \definition[瓶,杯]{s.}{leite de vaca}
\end{entry}

\begin{entry}{牛人}{niu2ren2}{4,2}{⽜、⼈}
  \definition{s.}{(coloquial) o cara | verdadeiro especialista | \emph{badass}}
\end{entry}

\begin{entry}{牛肉}{niu2rou4}{4,6}{⽜、⾁}
  \definition{s.}{carne de vaca | bife}
\end{entry}

\begin{entry}{牛仔裤}{niu2zai3ku4}{4,5,12}{⽜、⼈、⾐}[HSK 5]
  \definition[条]{s.}{calças jeans}
\end{entry}

\begin{entry}{农产品}{nong2 chan3 pin3}{6,6,9}{⼍、⼇、⼝}[HSK 5]
  \definition{s.}{produtos agrícolas}
\end{entry}

\begin{entry}{农村}{nong2cun1}{6,7}{⼍、⽊}[HSK 3]
  \definition[个]{s.}{aldeia; campo; área rural}
\end{entry}

\begin{entry}{农民}{nong2min2}{6,5}{⼍、⽒}[HSK 3]
  \definition[个,位]{s.}{fazendeiro; camponês; campesinato}
\end{entry}

\begin{entry}{农业}{nong2ye4}{6,5}{⼍、⼀}[HSK 3]
  \definition{s.}{agricultura; lavoura}
\end{entry}

\begin{entry}{浓}{nong2}{9}{⽔}[HSK 4]
  \definition{adj.}{denso; espesso; concentrado; um líquido ou gás que contém mais de um determinado ingrediente | grande; forte; profundo (de grau ou extensão) | profundo; (algumas cores) escuro}
\end{entry}

\begin{entry}{弄}{nong4}{7}{⼶}[HSK 2]
  \definition{s.}{beco | viela | travessa}
  \seeref{弄}{long4}
\end{entry}

\begin{entry}{努力}{nu3li4}{7,2}{⼒、⼒}[HSK 2]
  \definition{adj.}{diligente | aplicado}
  \definition{s.}{esforçar-se | se esforçar}
\end{entry}

\begin{entry}{怒骂}{nu4ma4}{9,9}{⼼、⾺}
  \definition{v.}{praguejar de raiva}
\end{entry}

\begin{entry}{暖}{nuan3}{13}{⽇}[HSK 5]
  \definition{adj.}{caloroso; cordial}
  \definition{v.}{aquecer; esquentar; aquecer algo ou aquecer o corpo}
\end{entry}

\begin{entry}{暖和}{nuan3huo5}{13,8}{⽇、⼝}[HSK 3]
  \definition{adj.}{morno; agradável e quente}
  \definition{v.}{aquecer}
\end{entry}

\begin{entry}{暖气}{nuan3qi4}{13,4}{⽇、⽓}[HSK 4]
  \definition[个]{s.}{aquecedor; aquecimento; aquecimento central}
\end{entry}

\begin{entry}{那}{nuo2}{6}{⾢}
  \definition*{s.}{sobrenome Nuo}
\end{entry}

\begin{entry}{诺贝尔奖}{nuo4bei4'er3 jiang3}{10,4,5,9}{⾔、⾙、⼩、⼤}
  \definition*{s.}{Prêmio Nobel}
\end{entry}

\begin{entry}{诺奖}{nuo4jiang3}{10,9}{⾔、⼤}
  \definition*{s.}{Prêmio Nobel, abreviação de 诺贝尔奖}
  \seeref{诺贝尔奖}{nuo4bei4'er3 jiang3}
\end{entry}

\begin{entry}{女}{nv3}{3}{⼥}[HSK 1][Kangxi 38]
  \definition{adj.}{feminino}
\end{entry}

\begin{entry}{女儿}{nv3'er2}{3,2}{⼥、⼉}[HSK 1]
  \definition{s.}{filha}
  \seealsoref{儿子}{er2zi5}
\end{entry}

\begin{entry}{女孩}{nv3hai2}{3,9}{⼥、⼦}
  \definition{s.}{menina | garota}
\end{entry}

\begin{entry}{女孩儿}{nv3hai2r5}{3,9,2}{⼥、⼦、⼉}[HSK 1]
\end{entry}

\begin{entry}{女朋友}{nv3peng2you5}{3,8,4}{⼥、⽉、⼜}[HSK 1]
  \definition{s.}{namorada}
\end{entry}

\begin{entry}{女人}{nv3ren2}{3,2}{⼥、⼈}[HSK 1]
  \definition[个,位]{s.}{mulher}
\end{entry}

\begin{entry}{女生}{nv3sheng1}{3,5}{⼥、⽣}[HSK 1]
  \definition[个]{s.}{aluna | estudante so sexo feminino}
\end{entry}

\begin{entry}{女士}{nv3shi4}{3,3}{⼥、⼠}[HSK 4]
  \definition{pron.}{Sra.; Senhorita; Senhora; título honorífico para mulheres (agora usado em contextos diplomáticos)}
  \definition[位,个]{s.}{senhora; madame}
\end{entry}

\begin{entry}{女王}{nv3wang2}{3,4}{⼥、⽟}
  \definition{s.}{rainha}
\end{entry}

\begin{entry}{女性}{nv3 xing4}{3,8}{⼥、⼼}[HSK 5]
  \definition[位,名]{s.}{mulher; feminino; feminilidade}
\end{entry}

\begin{entry}{女婿}{nv3xu5}{3,12}{⼥、⼥}
  \definition{s.}{marido da filha}
\end{entry}

\begin{entry}{女子}{nv3 zi3}{3,3}{⼥、⼦}[HSK 3]
  \definition[位]{s.}{mulher; feminino}
\end{entry}

%%%%% EOF %%%%%

