%%%
%%% J
%%%

\section*{J}\addcontentsline{toc}{section}{J}

\begin{entry}{几}{ji1}{2}{⼏}[Kangxi 16]
  \definition{adv.}{quase; praticamente}
  \definition{s.}{uma mesa pequena}
  \seeref{几}{ji3}
\end{entry}

\begin{entry}{几乎}{ji1hu1}{2,5}{⼏、⼃}[HSK 4]
  \definition{adv.}{quase; praticamente; próximo a | perto de; quase; à beira de}
\end{entry}

\begin{entry}{几率}{ji1lv4}{2,11}{⼏、⽞}
  \definition{s.}{probabilidade; um evento pode ou não ocorrer sob as mesmas condições, a grandeza que indica a possibilidade de ocorrência é chamada de probabilidade}
\end{entry}

\begin{entry}{机}{ji1}{6}{⽊}
  \definition*{s.}{Sobrenome Ji}
  \definition{adj.}{flexível; perspicaz; destreza; agilidade}
  \definition[台]{s.}{máquina; motor | avião; aeronave; aeroplano; refere-se especificamente a aeronaves | ponto crucial; os fatores-chave para a ocorrência e mudança das coisas | chance; ocasião; oportunidade; um momento crítico ou oportuno para o desenvolvimento e mudança das coisas | organismo; funções vitais dos organismos | besta; mecanismo de disparo de flechas de madeira em uma besta antiga | assuntos importantes; assuntos extremamente importantes e confidenciais | ideia; intenção}
\end{entry}

\begin{entry}{机场}{ji1chang3}{6,6}{⽊、⼟}[HSK 1]
  \definition[个,家,处,座]{s.}{aeródromo; campo de aviação; aeroporto; campo de voo}
\end{entry}

\begin{entry}{机动车}{ji1 dong4 che1}{6,6,4}{⽊、⼒、⾞}[HSK 6]
  \definition{s.}{veículo motorizado (oposto a 人力车) | veículo automotor; automóvel de passageiros: veículo comercial concebido e tecnicamente adequado para o transporte de passageiros e respetiva bagagem, incluindo o banco do condutor}
  \seealsoref{人力车}{ren2 li4 che1}
\end{entry}

\begin{entry}{机构}{ji1gou4}{6,8}{⽊、⽊}[HSK 4]
  \definition[所]{s.}{órgão; organização; instituição; instalações; aparelhamento; configuração | mecanismo; funcionamento interno de uma máquina ou unidade | estrutura interna de uma organização}
\end{entry}

\begin{entry}{机关}{ji1 guan1}{6,6}{⽊、⼋}[HSK 6]
  \definition{adj.}{operado por máquina | controlado mecanicamente}
  \definition[个]{s.}{engrenagem; mecanismo; Antigo: refere-se a certos dispositivos controlados mecanicamente; também se refere às peças de frenagem de dispositivos mecânicos | escritório; órgão; corpo; instituição | esquema; maquinação; estratagema; um plano cuidadoso e inteligente}
\end{entry}

\begin{entry}{机会}{ji1hui4}{6,6}{⽊、⼈}[HSK 2]
  \definition[个,次,种,些]{s.}{chance; oportunidade; momento favorável raro}
\end{entry}

\begin{entry}{机甲}{ji1jia3}{6,5}{⽊、⽥}
  \definition{s.}{\emph{mecha} (robôs operados pelo homem em mangá japonês)}
\end{entry}

\begin{entry}{机票}{ji1 piao4}{6,11}{⽊、⽰}[HSK 1]
  \definition[张]{s.}{passagem aérea; passagem de avião}
  \seealsoref{飞机票}{fei1ji1 piao4}
\end{entry}

\begin{entry}{机器}{ji1qi4}{6,16}{⽊、⼝}[HSK 3]
  \definition[台,部,个]{s.}{máquina; maquinário; motor; dispositivos e máquinas que são montados a partir de peças, podem funcionar, transformar energia ou produzir trabalho útil podem ser usados como ferramentas de produção, reduzindo a intensidade do trabalho humano e aumentando a produtividade | aparato; sistema político e econômico}
\end{entry}

\begin{entry}{机器人}{ji1 qi4 ren2}{6,16,2}{⽊、⼝、⼈}[HSK 5]
  \definition[个]{s.}{androide; golem | pessoa mecânica | robô}
\end{entry}

\begin{entry}{机械}{ji1xie4}{6,11}{⽊、⽊}[HSK 6]
  \definition{adj.}{rígido; mecânico; inflexível; uma metáfora para uma abordagem rígida e imutável}
  \definition[台,部,个]{s.}{máquina; maquinário; mecanismo; vários dispositivos compostos por princípios mecânicos}
\end{entry}

\begin{entry}{机遇}{ji1yu4}{6,12}{⽊、⾡}[HSK 4]
  \definition[个]{s.}{chance; oportunidade; circunstâncias favoráveis}
\end{entry}

\begin{entry}{机制}{ji1 zhi4}{6,8}{⽊、⼑}[HSK 5]
  \definition{s.}{mecanismo; processado por máquina; feito por máquina}
\end{entry}

\begin{entry}{肌}{ji1}{6}{⾁}
  \definition[块,片]{s.}{músculo; carne | pele;}
\end{entry}

\begin{entry}{肌肉}{ji1rou4}{6,6}{⾁、⾁}[HSK 5]
  \definition[身,块]{s.}{músculo; um dos tecidos básicos dos músculos humanos e de alguns animais, composto principalmente de células musculares fibrosas, pode se contrair, é o movimento do corpo e o corpo de digestão, respiração, circulação, excreção e outros processos fisiológicos da fonte de energia; pode ser dividido em três tipos: músculo liso, músculo esquelético e músculo cardíaco}
\end{entry}

\begin{entry}{鸡}{ji1}{7}{⿃}[HSK 2]
  \definition*{s.}{Sobrenome Ji}
  \definition[只]{s.}{galo, galinha, frango | palavra ofensiva para uma mulher que ganha dinheiro fazendo sexo com homens}
\end{entry}

\begin{entry}{鸡蛋}{ji1dan4}{7,11}{⿃、⾍}[HSK 1]
  \definition[个,枚,筐,箱,打]{s.}{ovo de galinha}
\end{entry}

\begin{entry}{积}{ji1}{10}{⽲}
  \definition{adj.}{de longa data; pendente há muito tempo | antiquíssimo; acumulado ao longo de um longo período de tempo}
  \definition{s.}{(medicina chinesa) indigestão (em bebês e crianças) | (matemática)  abreviação de produto, 乘积}
  \definition{v.}{acumular; juntar; amontoar; reunir; coletar}
  \seealsoref{乘积}{cheng2ji1}
\end{entry}

\begin{entry}{积极}{ji1ji2}{10,7}{⽲、⽊}[HSK 3]
  \definition{adj.}{ativo; descreve uma atitude proativa e esforçada | positivo; que tem um efeito positivo e ajuda no desenvolvimento das coisas}
\end{entry}

\begin{entry}{积累}{ji1lei3}{10,11}{⽲、⽷}[HSK 4]
  \definition{s.}{acúmulo; acumulação}
  \definition{v.}{acumular}
\end{entry}

\begin{entry}{积木}{ji1mu4}{10,4}{⽲、⽊}
  \definition{s.}{blocos de montar (brinquedo)}
\end{entry}

\begin{entry}{基}{ji1}{11}{⼟}
  \definition{adj.}{chave; básico; primário; cardinal; fundamental}
  \definition{s.}{base; fundação | base; grupo; radical; (química) uma parte dos átomos contidos na molécula de um composto, quando considerada como uma unidade, é chamada de base}
\end{entry}

\begin{entry}{基本}{ji1ben3}{11,5}{⼟、⽊}[HSK 3]
  \definition{adj.}{básico; fundamental; elementar | principal}
  \definition{adv.}{basicamente; em geral; no geral; em termos gerais}
  \definition{s.}{fundação}
\end{entry}

\begin{entry}{基本法}{ji1ben3fa3}{11,5,8}{⼟、⽊、⽔}
  \definition{s.}{lei básica (constituição)}
\end{entry}

\begin{entry}{基本功}{ji1ben3gong1}{11,5,5}{⼟、⽊、⼒}
  \definition{s.}{habilidades | fundamentos básicos}
\end{entry}

\begin{entry}{基本上}{ji1 ben3 shang4}{11,5,3}{⼟、⽊、⼀}[HSK 3]
  \definition{adv.}{basicamente; principalmente | em geral; de modo geral}
\end{entry}

\begin{entry}{基础}{ji1chu3}{11,10}{⼟、⽯}[HSK 3]
  \definition[个,种,点,层]{s.}{base; fundamento; fundação; a essência ou o ponto de partida do desenvolvimento das coisas | básico; fundamental; refere-se às condições mínimas | fundação do edifício; base do edifício}
\end{entry}

\begin{entry}{基地}{ji1di4}{11,6}{⼟、⼟}[HSK 5]
  \definition{s.}{base; como base para alguns negócios | base; um local dedicado à realização de um negócio}
\end{entry}

\begin{entry}{基督教}{ji1 du1 jiao4}{11,13,11}{⼟、⽬、⽁}[HSK 6]
  \definition*{s.}{Cristianismo; A Religião Cristã | Cristão}
\end{entry}

\begin{entry}{基金}{ji1jin1}{11,8}{⼟、⾦}[HSK 5]
  \definition[项,支,种,个]{s.}{fundo; fundos reservados ou destinados ao estabelecimento ou desenvolvimento de uma empresa}
\end{entry}

\begin{entry}{基因}{ji1yin1}{11,6}{⼟、⼞}
  \definition{s.}{gene}
\end{entry}

\begin{entry}{激}{ji1}{16}{⽔}
  \definition*{s.}{Sobrenome Ji}
  \definition{adj.}{afiado; feroz; violento | vívido}
  \definition{adv.}{bruscamente; ferozmente; violentamente}
  \definition{s.}{o impacto de ondas fortes contra a costa}
  \definition{v.}{bater; avançar; correr | despertar; estimular; incitar; excitar | ficar doente por se molhar | esfriar (colocando água gelada, etc.)}
\end{entry}

\begin{entry}{激动}{ji1dong4}{16,6}{⽔、⼒}[HSK 4]
  \definition{adj.}{animado; entusiasmado; empolgado}
  \definition{v.}{agitar; excitar; tornar fortes os sentimentos de alguém}
\end{entry}

\begin{entry}{激烈}{ji1lie4}{16,10}{⽔、⽕}[HSK 4]
  \definition{adj.}{agudo; afiado; feroz; violento; intenso}
\end{entry}

\begin{entry}{激情}{ji1qing2}{16,11}{⽔、⼼}[HSK 6]
  \definition{s.}{paixão; emoções fortes e explosivas, como êxtase, raiva, etc.}
\end{entry}

\begin{entry}{鷄}{ji1}{21}{⿃}
  \variantof{鸡}
\end{entry}

\begin{entry}{及}{ji2}{3}{⼃}
  \definition{conj.}{e | bem como}
\end{entry}

\begin{entry}{及格}{ji2ge2}{3,10}{⼃、⽊}[HSK 4]
  \definition{v.+compl.}{passar; passar em um teste, exame, etc.}
\end{entry}

\begin{entry}{及时}{ji2shi2}{3,7}{⼃、⽇}[HSK 3]
  \definition{adj.}{oportuno; na hora certa; adequado; na ocasião certa}
  \definition{adv.}{prontamente; sem demora; imediatamente}
\end{entry}

\begin{entry}{吉}{ji2}{6}{⼝}
  \definition*{s.}{Província de Jilin, abreviação de 吉林 | Sobrenome Ji}
  \definition{adj.}{sortudo; propício; auspicioso (oposto de 凶)}
  \seealsoref{吉林}{ji2lin2}
  \seealsoref{凶}{xiong1}
\end{entry}

\begin{entry}{吉利}{ji2 li4}{6,7}{⼝、⼑}[HSK 6]
  \definition{adj.}{sortudo; auspicioso; propício}
\end{entry}

\begin{entry}{吉林}{ji2lin2}{6,8}{⼝、⽊}
  \definition*{s.}{Província de Jilin}
\end{entry}

\begin{entry}{吉他}{ji2ta1}{6,5}{⼝、⼈}
  \definition[把]{s.}{Empréstimo linguístico: guitarra}
\end{entry}

\begin{entry}{吉祥}{ji2xiang2}{6,10}{⼝、⽰}[HSK 6]
  \definition{adj.}{sortudo; auspicioso; propício}
  \definition[个,种]{s.}{sorte; auspiciosidade; propiciação; um sinal ou símbolo de boa sorte ou fortuna}
\end{entry}

\begin{entry}{级}{ji2}{6}{⽷}[HSK 2]
  \definition{clas.}{usado para degraus, escadas, pisos de torres, etc.}
  \definition[个,种]{s.}{nível; classificação; grau; classe | série; turma; qualquer uma das divisões anuais de um curso escolar | degrau}
\end{entry}

\begin{entry}{即}{ji2}{7}{⼙}
  \definition{conj.}{e | até | mesmo se/embora}
\end{entry}

\begin{entry}{即便}{ji2bian4}{7,9}{⼙、⼈}
  \definition{conj.}{mesmo se/embora}
\end{entry}

\begin{entry}{即便是}{ji2bian4 shi4}{7,9,9}{⼙、⼈、⽇}
  \definition{conj.}{mesmo que seja}
\end{entry}

\begin{entry}{即或}{ji2huo4}{7,8}{⼙、⼽}
  \definition{conj.}{mesmo se/embora}
\end{entry}

\begin{entry}{即将}{ji2jiang1}{7,9}{⼙、⼨}[HSK 4]
  \definition{adv.}{em breve; estar prestes a; estar a ponto de}
\end{entry}

\begin{entry}{即若}{ji2ruo4}{7,8}{⼙、⾋}
  \definition{conj.}{mesmo se/embora}
\end{entry}

\begin{entry}{即使}{ji2shi3}{7,8}{⼙、⼈}[HSK 5]
  \definition{conj.}{mesmo; mesmo que; mesmo se; apesar de; expressando uma concessão hipotética}
\end{entry}

\begin{entry}{即是}{ji2shi4}{7,9}{⼙、⽇}
  \definition{conj.}{aquilo é}
\end{entry}

\begin{entry}{极}{ji2}{7}{⽊}[HSK 4]
  \definition*{s.}{Sobrenome Ji}
  \definition{adj.}{máximo; extremo; final; supremo}
  \definition{adv.}{extremamente; excessivamente}
  \definition{s.}{o ponto máximo, mais alto; extremo; ápice; ponto culminante | pólo; as extremidades norte e sul da Terra; as extremidades de um ímã; a extremidade de uma fonte de alimentação ou de um aparelho elétrico onde a corrente entra ou sai do aparelho}
  \definition{v.}{chegar ao fim de; levar a extremos}
\end{entry}

\begin{entry}{极端}{ji2duan1}{7,14}{⽊、⽴}[HSK 6]
  \definition{adj.}{extremo; absoluto; sem quaisquer restrições}
  \definition{adv.}{excessivamente; extremamente; alto grau de expressão}
  \definition{s.}{extremo; extremidade; o auge do desenvolvimento}
\end{entry}

\begin{entry}{……极了}{ji2le5}{7,2}{⽊、⼅}[HSK 3]
  \definition{expr.}{extremamente; alto grau de expressão}
\end{entry}

\begin{entry}{极其}{ji2qi2}{7,8}{⽊、⼋}[HSK 4]
  \definition{adv.}{mais; extremamente; excessivamente}
\end{entry}

\begin{entry}{急}{ji2}{9}{⼼}[HSK 2]
  \definition{adj.}{impaciente; ansioso | irritado; aborrecido; incomodado | rápido e intenso (em oposição a 缓); veloz | urgente; premente}
  \definition{s.}{urgência; emergência; assunto urgente e grave}
  \definition{v.}{preocupar; deixar ansioso | estar ansioso para ajudar; tratar os problemas dos outros como se fossem urgentes e ajudar a resolvê-los imediatamente}
  \seealsoref{缓}{huan3}
\end{entry}

\begin{entry}{急救}{ji2 jiu4}{9,11}{⼼、⽁}[HSK 6]
  \definition{s.}{primeiros socorros; tratamento médico de emergência (para pessoas gravemente doentes ou gravemente feridas)}
  \definition{v.}{prestar primeiros socorros; dar tratamento de emergência}
\end{entry}

\begin{entry}{急忙}{ji2mang2}{9,6}{⼼、⼼}[HSK 4]
  \definition{adv.}{apressadamente; com pressa}
\end{entry}

\begin{entry}{疾}{ji2}{10}{⽧}
  \definition*{s.}{Sobrenome Ji}
  \definition{s.}{doença; enfermidade; moléstia; padecimento | sofrimento; dor; dificuldade; mazela}
\end{entry}

\begin{entry}{疾病}{ji2bing4}{10,10}{⽧、⽧}[HSK 6]
  \definition[种]{s.}{doença; enfermidade; termo geral para doença}
\end{entry}

\begin{entry}{集}{ji2}{12}{⾫}[HSK 6]
  \definition*{s.}{Sobrenome Ji}
  \definition{clas.}{parte; volume}
  \definition[个,本]{s.}{mercado; feira rural | coleção; conjunto; antologia | (matemática) conjunto}
  \definition{v.}{reunir; coletar; montar}
\end{entry}

\begin{entry}{集合}{ji2he2}{12,6}{⾫、⼝}[HSK 4]
  \definition{v.}{reunir-se; juntar-se | reunir, juntar, convocar}
\end{entry}

\begin{entry}{集体}{ji2ti3}{12,7}{⾫、⼈}[HSK 3]
  \definition{s.}{coletivo; comunidade; grupo; equipe; organizações ou grupos em que muitas pessoas trabalham, estudam e vivem juntas}
\end{entry}

\begin{entry}{集团}{ji2tuan2}{12,6}{⾫、⼞}[HSK 5]
  \definition[个]{s.}{anel; bloco; grupo; panelinha; círculo; grupo organizado para agir em conjunto com um determinado objetivo | grupo; entidade econômica com uma direção de negócios especializada, liderada por uma grande empresa com forte poder econômico e alta visibilidade, e formada pela combinação ou fusão de empresas relacionadas}
\end{entry}

\begin{entry}{集中}{ji2zhong1}{12,4}{⾫、⼁}[HSK 3]
  \definition{adj.}{centralizado; concentrado}
  \definition{v.}{concentrar; centralizar; focar; acumular; reunir (oposto de 分散) | reunir pessoas, coisas, forças, etc. dispersas; resumir opiniões, experiências, etc.}
  \seealsoref{分散}{fen1san4}
\end{entry}

\begin{entry}{嫉}{ji2}{13}{⼥}
  \definition{v.}{invejar | odiar | ter ciúmes; ter inveja}
\end{entry}

\begin{entry}{嫉妒}{ji2du4}{13,7}{⼥、⼥}
  \definition{v.}{estar com ciúmes de | invejar}
\end{entry}

\begin{entry}{几}{ji3}{2}{⼏}[HSK 1][Kangxi 16]
  \definition{adv.}{quanto?, usado para perguntar sobre quantidade e tempo}
  \definition{num.}{alguns; vários; poucos; indica um número indeterminado maior que um e menor que dez}
  \seeref{几}{ji1}
\end{entry}

\begin{entry}{几何}{ji3he2}{2,7}{⼏、⼈}
  \definition{s.}{geometria}
\end{entry}

\begin{entry}{纪}{ji3}{6}{⽷}
  \definition*{s.}{Sobrenome Ji}
  \definition{s.}{disciplina | um período de doze anos (na China antiga); um período de anos | (geologia) subdivisão de uma era geológica; período}
  \definition{v.}{colocar por escrito; registrar; mesmo significado de 记, usado principalmente em 记录, 纪年, 纪元, 纪传, etc. | classificar (fios de seda)}
  \seeref{纪}{ji4}
  \seealsoref{记}{ji4}
  \seealsoref{纪传}{ji4 zhuan4}
  \seealsoref{记录}{ji4lu4}
  \seealsoref{纪年}{ji4nian2}
  \seealsoref{纪元}{ji4yuan2}
\end{entry}

\begin{entry}{挤}{ji3}{9}{⼿}[HSK 5]
  \definition{adj.}{lotado; congestionado; descreve um grande número de pessoas ou coisas e muito pouco espaço}
  \definition{v.}{empacotar; amontoar; aglomerar | sacudir; empurrar contra; empurrar alguém ou algo para longe com seu corpo com toda a força que puder| pressionar; apertar; expulsar por pressão}
\end{entry}

\begin{entry}{给}{ji3}{9}{⽷}
  \definition{adj.}{abundante; próspero; bem provido para}
  \definition{v.}{fornecer; prover}
  \seeref{给}{gei3}
\end{entry}

\begin{entry}{给予}{ji3yu3}{9,4}{⽷、⼅}[HSK 6]
  \definition{v.}{dar; conceder; dar em troca}
\end{entry}

\begin{entry}{计}{ji4}{4}{⾔}
  \definition*{s.}{Sobrenome Ji}
  \definition{s.}{medidor; aferidor; indicador; um instrumento para medir ou calcular graus, tempo, etc. | ideia; ardil; estratagema; plano}
  \definition{v.}{contar; calcular; numerar | planejar; traçar; imaginar}
\end{entry}

\begin{entry}{计划}{ji4hua4}{4,6}{⾔、⼑}[HSK 2]
  \definition[个,项]{s.}{plano; projeto; programa; trabalho, ações, conteúdo e etapas previamente definidos}
  \definition{v.}{planejar; traçar um plano}
\end{entry}

\begin{entry}{计算}{ji4suan4}{4,14}{⾔、⽵}[HSK 3]
  \definition{v.}{contar; calcular; computar; enumerar; encontrar a variável desconhecida | planejar; considerar | conspirar secretamente contra os outros; planejar secretamente prejudicar os outros}
\end{entry}

\begin{entry}{计算机}{ji4 suan4 ji1}{4,14,6}{⾔、⽵、⽊}[HSK 2]
  \definition[部,台]{s.}{computador; calculadora; máquinas capazes de realizar cálculos matemáticos são feitas com dispositivos mecânicos, como calculadoras manuais, outras são feitas com componentes eletrônicos, como computadores eletrônicos}
\end{entry}

\begin{entry}{计算机程序}{ji4suan4ji1 cheng2xu4}{4,14,6,12,7}{⾔、⽵、⽊、⽲、⼴}
  \definition{s.}{programa de computador}
\end{entry}

\begin{entry}{记}{ji4}{5}{⾔}[HSK 1]
  \definition*{s.}{Sobrenome Ji}
  \definition{clas.}{tapas, palmadas, bofetadas, etc.; usado para indicar o número de vezes que uma determinada ação é realizada}
  \definition{s.}{assinatura; bloco de notas; livro ou artigo que registra fatos | insígnia; indicação; \& comercial; símbolo | marca de nascença; manchas escuras presentes na pele desde o nascimento}
  \definition{v.}{lembrar; ter em mente; guardar na memória; manter a imagem na mente | escrever (anotar); registrar; inscrever}
\end{entry}

\begin{entry}{记得}{ji4de5}{5,11}{⾔、⼻}[HSK 1]
  \definition{v.}{lembrar; recordar; lembrar-se; não esquecer | manter algo em mente; (informal) não se esquecer de fazer algo, usado para lembrar}
\end{entry}

\begin{entry}{记录}{ji4lu4}{5,8}{⾔、⼹}[HSK 3]
  \definition[份,名,位,个]{s.}{notas; registro | anotador; registrador; a pessoa que faz registros}
  \definition{v.}{tomar notas; registrar; escrever o que ouviu ou o que aconteceu; gravar o som ou a imagem com um gravador ou uma câmera de vídeo e transformar em algum tipo de obra}
\end{entry}

\begin{entry}{记性}{ji4xing5}{5,8}{⾔、⼼}
  \definition{s.}{memória (habilidade em reter informações)}
\end{entry}

\begin{entry}{记忆}{ji4yi4}{5,4}{⾔、⼼}[HSK 5]
  \definition[段]{s.}{memória; manter em sua mente uma imagem do passado}
  \definition{v.}{recordar; lembrar; lembrar-se ou recordar alguém ou algo do passado}
\end{entry}

\begin{entry}{记载}{ji4zai3}{5,10}{⾔、⾞}[HSK 4]
  \definition[段,份]{s.}{registro; conta; artigos e materiais que registram eventos}
  \definition{v.}{registrar; colocar por escrito}
\end{entry}

\begin{entry}{记者}{ji4zhe3}{5,8}{⾔、⽼}[HSK 3]
  \definition[群,名,位]{s.}{repórter; correspondente; jornalista; profissionais dedicados a entrevistar e reportar notícias para a mídia}
\end{entry}

\begin{entry}{记住}{ji4 zhu5}{5,7}{⾔、⼈}[HSK 1]
  \definition{v.}{lembrar; aprender de cor; ter em mente; guardar na memória}
\end{entry}

\begin{entry}{纪}{ji4}{6}{⽷}
  \definition*{s.}{Sobrenome Ji}
  \definition{s.}{disciplina | idade; época | (geologia) período | um período de doze anos (na China antiga); um período de anos | (geologia) subdivisão de uma era geológica}
  \definition{v.}{colocar por escrito; registrar | registrar, mesmo significado de 记, usado principalmente em 记录, 纪年, 纪元, 纪传, etc. | classificar (fios de seda)}
  \seeref{纪}{ji3}
  \seealsoref{记}{ji4}
  \seealsoref{纪传}{ji4 zhuan4}
  \seealsoref{记录}{ji4lu4}
  \seealsoref{纪年}{ji4nian2}
  \seealsoref{纪元}{ji4yuan2}
\end{entry}

\begin{entry}{纪录}{ji4lu4}{6,8}{⽷、⼹}[HSK 3]
  \definition[项,个]{s.}{recorde (esportes); o número mais alto ou mais baixo registrado em um determinado período de tempo}
\end{entry}

\begin{entry}{纪律}{ji4lv4}{6,9}{⽷、⼻}[HSK 4]
  \definition{s.}{disciplina; código de conduta que cada membro da vida coletiva deve observar}
\end{entry}

\begin{entry}{纪年}{ji4nian2}{6,6}{⽷、⼲}
  \definition{s.}{cronologia; uma maneira de numerar os anos | registro cronológico de eventos; anais; um dos gêneros de livros históricos é organizar fatos históricos em ordem cronológica}
\end{entry}

\begin{entry}{纪念}{ji4nian4}{6,8}{⽷、⼼}[HSK 3]
  \definition[个,次]{s.}{lembrança; recordação; usado para representar uma lembrança (objeto)}
  \definition{v.}{comemorar; expressar saudade por pessoas ou coisas através de objetos ou ações}
\end{entry}

\begin{entry}{纪元}{ji4yuan2}{6,4}{⽷、⼉}
  \definition{s.}{o início de uma era (por exemplo, o reinado de um imperador) | época; era}
\end{entry}

\begin{entry}{纪传}{ji4 zhuan4}{6,6}{⽷、⼈}
  \definition{s.}{crônica; biografia}
\end{entry}

\begin{entry}{纪传体}{ji4 zhuan4 ti3}{6,6,7}{⽷、⼈、⼈}
  \definition{s.}{história apresentada em uma série de biografias | gênero histórico baseado em biografia}
\end{entry}

\begin{entry}{技}{ji4}{7}{⼿}
  \definition[门,项]{s.}{destreza; habilidade; estratagema | técnica; tecnologia}
\end{entry}

\begin{entry}{技俩}{ji4liang3}{7,9}{⼿、⼈}
  \definition{s.}{truque | estratagema | ardil | esquema | estratégia | tática}
\end{entry}

\begin{entry}{技能}{ji4 neng2}{7,10}{⼿、⾁}[HSK 5]
  \definition[种,项]{s.}{habilidade técnica; domínio de uma habilidade ou técnica; capacidade de adquirir e aplicar conhecimento}
\end{entry}

\begin{entry}{技巧}{ji4qiao3}{7,5}{⼿、⼯}[HSK 4]
  \definition{s.}{habilidade; técnica; habilidades engenhosas expressas em artes, artesanato, esportes, etc.}
\end{entry}

\begin{entry}{技术}{ji4shu4}{7,5}{⼿、⽊}[HSK 3]
  \definition[种,门,项]{s.}{habilidade; técnica; tecnologia; a experiência e o conhecimento acumulados pelo ser humano no processo de utilização e transformação da natureza, e refletidos no trabalho produtivo, também se referem, de maneira geral, a outras habilidades operacionais}
\end{entry}

\begin{entry}{系}{ji4}{7}{⽷}
  \definition{v.}{amarrar; prender; abotoar; dar um nó}
  \seeref{系}{xi4}
\end{entry}

\begin{entry}{季}{ji4}{8}{⼦}[HSK 4]
  \definition*{s.}{Sobrenome Ji}
  \definition{s.}{estação; o ano é dividido em quatro estações, primavera, verão, outono e inverno, e uma estação dura três meses | temporada | o fim de uma era | o último mês de uma temporada | o quarto ou mais novo entre irmãos; último na ordem de precedência}
\end{entry}

\begin{entry}{季度}{ji4du4}{8,9}{⼦、⼴}[HSK 4]
  \definition[个]{s.}{trimestre; período de tempo trimestral}
\end{entry}

\begin{entry}{季节}{ji4jie2}{8,5}{⼦、⾋}[HSK 4]
  \definition[个]{s.}{estação (clima); um período característico do ano}
\end{entry}

\begin{entry}{既}{ji4}{9}{⽆}[HSK 4]
  \definition*{s.}{Sobrenome Ji}
  \definition{adv.}{já}
  \definition{conj.}{desde; como; agora que | assim como; e também; ambos\dots e\dots; usado em conjunto com advérbios como 且, 又, 也 para indicar uma combinação de ambas as situações}
  \seealsoref{且}{qie3}
  \seealsoref{也}{ye3}
  \seealsoref{又}{you4}
\end{entry}

\begin{entry}{既不……又不……}{ji4bu4 you4bu4}{9,4,2,4}{⽆、⼀、⼜、⼀}
  \definition{conj.}{nem mesmo\dots}
\end{entry}

\begin{entry}{既然}{ji4ran2}{9,12}{⽆、⽕}[HSK 4]
  \definition{conj.}{como; desde; agora que; usado na primeira metade de uma frase, muitas vezes repetido na segunda metade pelos advérbios 就, 也, 还 para indicar que a premissa é primeiro declarada e depois inferida}
  \seealsoref{还}{hai2}
  \seealsoref{就}{jiu4}
  \seealsoref{也}{ye3}
\end{entry}

\begin{entry}{既又}{ji4you4}{9,2}{⽆、⼜}
  \definition{conj.}{desde | como | agora isso | os dois e | assim como}
\end{entry}

\begin{entry}{继}{ji4}{10}{⽷}
  \definition{adv.}{então; depois}
  \definition{s.}{filhos; prole}
  \definition{v.}{continuar; ter sucesso; seguir}
\end{entry}

\begin{entry}{继承}{ji4cheng2}{10,8}{⽷、⼿}[HSK 5]
  \definition{v.}{herdar (o patrimônio de uma pessoa falecida, etc.) de acordo com a lei | continuar; geralmente se refere à aceitação do estilo, da cultura, do conhecimento, etc., daqueles que nos precederam | continuar; os descendentes continuam o trabalho deixado por seus antecessores.}
\end{entry}

\begin{entry}{继续}{ji4xu4}{10,11}{⽷、⽷}[HSK 3]
  \definition{s.}{continuação}
  \definition{v.}{continuar; prosseguir | prosseguir; continuar; seguir em frente (com); (atividades, eventos, etc.) continuar após uma pausa ou um determinado período de tempo}
\end{entry}

\begin{entry}{寂}{ji4}{11}{⼧}
  \definition{adj.}{quieto; parado; silencioso | solitário}
\end{entry}

\begin{entry}{寂寥}{ji4liao2}{11,14}{⼧、⼧}
  \definition{s.}{solidão | vasto e vazio | quieto e desolado (literário)}
\end{entry}

\begin{entry}{寂寞}{ji4mo4}{11,13}{⼧、⼧}
  \definition{adj.}{sozinho | solitário | (de um lugar) silencioso}
\end{entry}

\begin{entry}{寄}{ji4}{11}{⼧}[HSK 4]
  \definition{adj.}{adotado; fomentado; promovido}
  \definition{v.}{enviar; postar; remeter | confiar; depositar; colocar | depender de; apegar-se a}
\end{entry}

\begin{entry}{寄存}{ji4cun2}{11,6}{⼧、⼦}
  \definition{v.}{depositar | deixar algo com alguém | armazenar}
\end{entry}

\begin{entry}{寄递}{ji4di4}{11,10}{⼧、⾡}
  \definition{s.}{entrega de correspondência}
\end{entry}

\begin{entry}{寄放}{ji4fang4}{11,8}{⼧、⽅}
  \definition{v.}{deixar algo com alguém}
\end{entry}

\begin{entry}{寄居}{ji4ju1}{11,8}{⼧、⼫}
  \definition{s.}{morar longe de casa}
\end{entry}

\begin{entry}{寄卖}{ji4mai4}{11,8}{⼧、⼗}
  \definition{v.}{consignar para venda}
\end{entry}

\begin{entry}{寄生}{ji4sheng1}{11,5}{⼧、⽣}
  \definition{s.}{parasita | parasitismo}
  \definition{v.}{viver tirando vantagem dos outros | viver dentro ou sobre outro organismo como um parasita}
\end{entry}

\begin{entry}{寄生生活}{ji4sheng1sheng1huo2}{11,5,5,9}{⼧、⽣、⽣、⽔}
  \definition{s.}{parasitismo | vida parasitária}
\end{entry}

\begin{entry}{寄售}{ji4shou4}{11,11}{⼧、⼝}
  \definition{v.}{venda em consignação}
\end{entry}

\begin{entry}{寄送}{ji4song4}{11,9}{⼧、⾡}
  \definition{v.}{enviar | transmitir}
\end{entry}

\begin{entry}{寄宿}{ji4su4}{11,11}{⼧、⼧}
  \definition{s.}{embarque}
  \definition{v.}{embarcar}
\end{entry}

\begin{entry}{寄托}{ji4tuo1}{11,6}{⼧、⼿}
  \definition{v.}{investir (sua esperança, energia, etc.) em algo | confiar (a alguém) | colocar (a esperança, a energia, etc.) em}
\end{entry}

\begin{entry}{寄望}{ji4wang4}{11,11}{⼧、⽉}
  \definition{v.}{depositar esperanças em}
\end{entry}

\begin{entry}{寄养}{ji4yang3}{11,9}{⼧、⼋}
  \definition{v.}{embarcar | promover | colocar sob os cuidados de alguém (uma criança, animal de estimação, etc.)}
\end{entry}

\begin{entry}{寄予}{ji4yu3}{11,4}{⼧、⼅}
  \definition{v.}{expressar | colocar (esperança, importância, etc.) em | mostrar}
\end{entry}

\begin{entry}{旣}{ji4}{11}{⽆}
  \variantof{既}
\end{entry}

\begin{entry}{加}{jia1}{5}{⼒}[HSK 2]
  \definition*{s.}{Canadá, abreviação de 加拿大 | Sobrenome Jia}
  \definition{v.}{adicionar; somar | aumentar; incrementar; aumentar a quantidade ou o grau em relação ao original | inserir; adicionar; anexar; adicionar o que não existe; colocar no lugar | acrescentar; indica a realização de uma determinada ação | colocar uma coisa em cima da outra | impor ou aplicar algo a outra pessoa; atribuir um determinado comportamento a outra pessoa}
  \seealsoref{加拿大}{jia1na2da4}
\end{entry}

\begin{entry}{加班}{jia1ban1}{5,10}{⼒、⽟}[HSK 4]
  \definition{v.+compl.}{fazer horas extras; trabalhar horas extras}
\end{entry}

\begin{entry}{加工}{jia1gong1}{5,3}{⼒、⼯}[HSK 3]
  \definition{s.}{processo | trabalho (de uma máquina)}
  \definition{v.}{processar; realizar diversos trabalhos em matérias-primas e produtos semiacabados (como alterar dimensões, formas, propriedades, aumentar a precisão, pureza, etc.) para que atendam aos requisitos especificados | melhorar; polir; refere-se a todos os tipos de trabalho que tornam o produto final mais perfeito e refinado}
\end{entry}

\begin{entry}{加快}{jia1 kuai4}{5,7}{⼒、⼼}[HSK 3]
  \definition{v.}{acelerar; aumentar a velocidade; agilizar}
\end{entry}

\begin{entry}{加盟}{jia1 meng2}{5,13}{⼒、⽫}[HSK 6]
  \definition{v.}{aliar-se a; filiar-se a um sindicato; juntar-se a um grupo ou organização}
\end{entry}

\begin{entry}{加拿大}{jia1na2da4}{5,10,3}{⼒、⼿、⼤}
  \definition{s.}{Canadá}
\end{entry}

\begin{entry}{加拿大人}{jia1na2da4ren2}{5,10,3,2}{⼒、⼿、⼤、⼈}
  \definition{s.}{canadense | pessoa ou povo do Canadá}
\end{entry}

\begin{entry}{加强}{jia1 qiang2}{5,12}{⼒、⼸}[HSK 3]
  \definition{v.}{fortalecer; engrandecer; reforçar; tornar mais forte ou mais eficaz}
\end{entry}

\begin{entry}{加热}{jia1 re4}{5,10}{⼒、⽕}[HSK 5]
  \definition{v.}{aquecer; esquentar; aumentar a temperatura de um objeto}
\end{entry}

\begin{entry}{加入}{jia1ru4}{5,2}{⼒、⼊}[HSK 4]
  \definition{v.}{juntar-se; unir-se; aderir a; tornar-se um membro de uma organização, grupo | adicionar; colocar em}
\end{entry}

\begin{entry}{加上}{jia1 shang4}{5,3}{⼒、⼀}[HSK 5]
  \definition{conj.}{além disso; em adição}
  \definition{v.}{adicionar; acrescentar; dar; aumentar}
\end{entry}

\begin{entry}{加速}{jia1 su4}{5,10}{⼒、⾡}[HSK 5]
  \definition{v.}{acelerar; agilizar}
\end{entry}

\begin{entry}{加速度}{jia1su4du4}{5,10,9}{⼒、⾡、⼴}
  \definition{s.}{aceleração}
\end{entry}

\begin{entry}{加以}{jia1 yi3}{5,4}{⼒、⼈}[HSK 5]
  \definition{conj.}{além disso; em adição; indica outras razões ou condições}
  \definition{v.aux.}{usado na frente de palavras dissilábicas para indicar como um objeto mencionado deve ser tratado ou descartado | usado antes de um verbo polifônico ou de um substantivo formado a partir de um verbo para indicar como tratar ou lidar com o que foi mencionado anteriormente}
\end{entry}

\begin{entry}{加油}{jia1 you2}{5,8}{⼒、⽔}[HSK 2]
  \definition{v.+compl.}{abastecer com óleo; reabastecer; adicionar combustível ou óleo lubrificante | fazer um esforço extra; dar o máximo; (Vamos lá!) metáfora para se esforçar ainda mais}
\end{entry}

\begin{entry}{加油工}{jia1 you2 gong1}{5,8,3}{⼒、⽔、⼯}[HSK 6]
  \definition{s.}{frentista}
\end{entry}

\begin{entry}{加油站}{jia1you2zhan4}{5,8,10}{⼒、⽔、⽴}[HSK 4]
  \definition[个,家]{s.}{posto de gasolina; posto de combustível; postos de abastecimento para venda a varejo de gasolina e óleo para carros e outros veículos motorizados}
\end{entry}

\begin{entry}{夹}{jia1}{6}{⼤}[HSK 5]
  \definition{s.}{clipe, grampo, pasta, etc.}
  \definition{v.}{colocar no meio; pressionar de ambos os lados; aplicar força ou ação ao mesmo objeto de ambos os lados ao mesmo tempo | misturar; mesclar; intercalar}
  \seeref{夹}{ga1}
  \seeref{夹}{jia2}
\end{entry}

\begin{entry}{夹杂}{jia1 za2}{6,6}{⼤、⽊}
  \definition{v.}{ser misturado com; estar carregado de; adicionar (algo mais)}
\end{entry}

\begin{entry}{夹肢窝}{jia1 zhi1 wo1}{6,8,12}{⼤、⾁、⽳}
  \definition{s.}{axila; sovaco; também escrito como 胳肢窝}
  \seealsoref{胳肢窝}{ga1 zhi1 wo1}
\end{entry}

\begin{entry}{夹子}{jia1 zi5}{6,3}{⼤、⼦}
  \definition[个,堆,盒]{s.}{pasta; carteira; algo para guardar dinheiro, papel, etc. | clipe; grampo; pasta; pinça; ferramentas para prender coisas}
\end{entry}

\begin{entry}{茄}{jia1}{8}{⾋}
  \definition{s.}{caracter fonético usado em empréstimos linguísticos para o som "jia", embora 夹 seja mais comum}
  \seealsoref{夹}{jia1}
\end{entry}

\begin{entry}{家}{jia1}{10}{⼧}[HSK 1,2]
  \definition*{s.}{Sobrenome Jia}
  \definition{adj.}{domado; domesticado; criado; alimentado | interno}
  \definition{clas.}{usado para famílias ou estabelecimentos comerciais; para uso doméstico; lojas; fábricas, etc.}
  \definition{pron.}{Educado: meu (irmã, tio, etc.)}
  \definition[个]{s.}{família; domicílio; clã | lar; casa; residência da família | pessoa ou família envolvida em um determinado comércio; pessoas que trabalham em determinada profissão ou que possuem determinada identidade | especialista em um determinado campo; pessoa que possui conhecimentos especializados ou se dedica a atividades específicas | escola de pensamento; rscola acadêmica | (em cartas de baralho, mah-jong etc.) festa; lado; refere-se a jogar xadrez ou cartas, em que uma das partes joga contra a outra | nacionalidade; referindo-se à etnia | membros da família; parentes; pessoas ou famílias com quem você tem algum tipo de relação | membro do mesmo clã; pessoas com o mesmo sobrenome}
  \definition{suf.}{sufixo substantivo para designar um especialista em alguma atividade, como um músico ou revolucionário, para designar uma profissão como em -eiro, -ista, por exemplo 科学家}
  \seealsoref{科学家}{ke1xue2jia1}
\end{entry}

\begin{entry}{家电}{jia1 dian4}{10,5}{⼧、⽥}[HSK 6]
  \definition[件,台]{s.}{eletrodomésticos, abreviação de 家用电器}
  \seealsoref{家用电器}{jia1yong4 dian4qi4}
\end{entry}

\begin{entry}{家伙}{jia1huo5}{10,6}{⼧、⼈}
  \definition[些,个,群,帮]{s.}{ferramenta; utensílio; arma; refere-se a ferramentas ou armas | cara; companheiro; refere-se a pessoas (com desprezo ou humor)  | gado; animal doméstico}
\end{entry}

\begin{entry}{家具}{jia1ju4}{10,8}{⼧、⼋}[HSK 3]
  \definition[件,套,些,个]{s.}{móveis; mobiliário de casa; utensílios domésticos, incluem principalmente camas, mesas, cadeiras, armários, etc.}
\end{entry}

\begin{entry}{家俱}{jia1ju4}{10,10}{⼧、⼈}
  \definition{s.}{mobília}
\end{entry}

\begin{entry}{家里}{jia1 li3}{10,7}{⼧、⾥}[HSK 1]
  \definition{s.}{(em) casa; (em sua) família | esposa}
\end{entry}

\begin{entry}{家人}{jia1 ren2}{10,2}{⼧、⼈}[HSK 1]
  \definition{s.}{família (de alguém); membro da família; os membros de uma família}
\end{entry}

\begin{entry}{家属}{jia1shu3}{10,12}{⼧、⼫}[HSK 3]
  \definition{s.}{membros da família; dependentes (familiares); os membros da família que não sejam o próprio chefe da família, ou seja, os membros da família que não sejam o próprio trabalhador}
\end{entry}

\begin{entry}{家庭}{jia1ting2}{10,9}{⼧、⼴}[HSK 2]
  \definition[个,户]{s.}{família}
\end{entry}

\begin{entry}{家务}{jia1wu4}{10,5}{⼧、⼒}[HSK 4]
  \definition[堆,次,件]{s.}{trabalho doméstico; tarefas domésticas}
\end{entry}

\begin{entry}{家乡}{jia1xiang1}{10,3}{⼧、⼄}[HSK 3]
  \definition[片,座]{s.}{cidade natal; o lugar onde sua família vive há gerações}
\end{entry}

\begin{entry}{家用电器}{jia1yong4 dian4qi4}{10,5,5,16}{⼧、⽤、⽥、⼝}
  \definition{s.}{eletrodoméstico; refere-se a diversos aparelhos elétricos utilizados na vida doméstica e coletiva}
\end{entry}

\begin{entry}{家园}{jia1 yuan2}{10,7}{⼧、⼞}[HSK 6]
  \definition{s.}{casa; terra natal; um jardim em casa, geralmente referindo-se à cidade natal ou à família}
\end{entry}

\begin{entry}{家长}{jia1 zhang3}{10,4}{⼧、⾧}[HSK 2]
  \definition[位,名,个]{s.}{pais; patriarca; tutor; guardião; refere-se aos pais ou outros responsáveis legais}
\end{entry}

\begin{entry}{傢}{jia1}{12}{⼈}
  \definition{s.}{usado em 家伙  e 家俱}
  \variantof{家}
  \seealsoref{傢伙}{jia1huo5}
  \seealsoref{家俱}{jia1ju4}
\end{entry}

\begin{entry}{傢伙}{jia1huo5}{12,6}{⼈、⼈}
  \variantof{家伙}
\end{entry}

\begin{entry}{傢俱}{jia1ju4}{12,10}{⼈、⼈}
  \variantof{家俱}
\end{entry}

\begin{entry}{嘉}{jia1}{14}{⼝}
  \definition*{s.}{Sobrenome Jia}
  \definition{adj.}{bom; ótimo | auspicioso | excelente}
  \definition{v.}{elogiar; recomendar}
  \definition{v.}{elogiar}
\end{entry}

\begin{entry}{嘉宾}{jia1bin1}{14,10}{⼝、⼧}[HSK 6]
  \definition[个,位,名,些]{s.}{convidado}
\end{entry}

\begin{entry}{嘉年华}{jia1nian2hua2}{14,6,6}{⼝、⼲、⼗}
  \definition{s.}{(empréstimo linguístico) carnaval}
\end{entry}

\begin{entry}{夹}{jia2}{6}{⼤}
  \definition{adj.}{forrado; com camada dupla; duas camadas (roupas, colchas, etc.) | pinçado; voz deliberadamente engraçada}
  \seeref{夹}{ga1}
  \seeref{夹}{jia1}
\end{entry}

\begin{entry}{甲}{jia3}{5}{⽥}[HSK 5]
  \definition*{s.}{Sobrenome Jia}
  \definition{s.}{alfa; primeiro lugar; o primeiro dos caules celestiais, geralmente usado para indicar o primeiro em ordem ou classificação | concha; carapaça; crustáceos | unha; crostas queratinosas nos dedos das mãos e dos pés | armadura; equipamento de proteção feito de metal | unidade de administração civil composta por 10 residências | uma palavra substituta para uma pessoa ou coisa indefinida; usado como pronome |}
  \definition{v.}{ocupar o primeiro lugar; ser melhor do que}
\end{entry}

\begin{entry}{甲骨文}{jia3gu3wen2}{5,9,4}{⽥、⾻、⽂}
  \definition{s.}{escrituras de oráculos | inscrições em ossos de oráculos (forma original de escritura chinesa)}
\end{entry}

\begin{entry}{假}{jia3}{11}{⼈}[HSK 2]
  \definition{adj.}{falso; artificial}
  \definition{conj.}{se; caso; no caso de; conecta frases, expressa relação hipotética, geralmente usada com 如, 若 e 使, equivalente a 如果}
  \definition[个,天]{s.}{falsificação; coisas falsas, irreais ou forjadas}
  \definition{v.}{emprestar | valer-se de; aproveitar; utilizar | supor; presumir; pressupor}
  \seeref{假}{jia4}
  \seealsoref{如}{ru2}
  \seealsoref{如果}{ru2guo3}
  \seealsoref{若}{ruo4}
  \seealsoref{使}{shi3}
\end{entry}

\begin{entry}{假的}{jia3de5}{11,8}{⼈、⽩}
  \definition{adj.}{falso | substituto | simulado}
\end{entry}

\begin{entry}{假如}{jia3ru2}{11,6}{⼈、⼥}[HSK 4]
  \definition{conj.}{se; supondo; no caso}
\end{entry}

\begin{entry}{假声}{jia3sheng1}{11,7}{⼈、⼠}
  \definition{s.}{falsete}
  \seealsoref{真声}{zhen1sheng1}
\end{entry}

\begin{entry}{假使}{jia3shi3}{11,8}{⼈、⼈}
  \definition{conj.}{se | supondo | em caso}
\end{entry}

\begin{entry}{假证件}{jia3zheng4jian4}{11,7,6}{⼈、⾔、⼈}
  \definition{s.}{documentos falsos}
\end{entry}

\begin{entry}{价}{jia4}{6}{⼈}[HSK 5]
  \definition{s.}{preço | valor; (figurativo) valores (éticos, culturais etc.) | (química) valência}
\end{entry}

\begin{entry}{价格}{jia4ge2}{6,10}{⼈、⽊}[HSK 3]
  \definition[个,种]{s.}{preço; tarifa; o valor monetário da mercadoria}
\end{entry}

\begin{entry}{价钱}{jia4 qian2}{6,10}{⼈、⾦}[HSK 3]
  \definition[个,种,笔]{s.}{preço}
\end{entry}

\begin{entry}{价值}{jia4zhi2}{6,10}{⼈、⼈}[HSK 3]
  \definition{s.}{valor; o trabalho social necessário condensado nos produtos | valor; importância; efeitos positivos}
\end{entry}

\begin{entry}{驾}{jia4}{8}{⾺}
  \definition*{s.}{Sobrenome Jia}
  \definition{s.}{carruagem do imperador; refere-se especificamente ao carro do imperador, referindo-se ao imperador | referindo-se a um veículo, usado como um termo respeitoso para uma pessoa}
  \definition{v.}{atrelar; puxar (uma carroça, etc.) | dirigir (um veículo); pilotar (um avião); velejar (um barco) | montar; cavalgar}
\end{entry}

\begin{entry}{驾驶}{jia4shi3}{8,8}{⾺、⾺}[HSK 5]
  \definition{v.}{dirigir; pilotar; conduzir; guiar; operar (um carro, navio, avião, trator, etc.) para fazê-lo mover}
\end{entry}

\begin{entry}{驾照}{jia4 zhao4}{8,13}{⾺、⽕}[HSK 5]
  \definition[本,张]{s.}{carteira de motorista}
\end{entry}

\begin{entry}{架}{jia4}{9}{⽊}[HSK 3]
  \definition{clas.}{usado para coisas com pilares ou componentes mecânicos | quadrado (usado para montanhas)}
  \definition{s.}{estrutura; organização do corpo humano ou das coisas | prateleira; estante; suporte; componentes que sustentam objetos ou utensílios para colocar objetos, etc.}
  \definition{v.}{colocar para cima; erigir | brigar; discutir | resistir; repelir; afastar | sequestrar; levar alguém à força}
\end{entry}

\begin{entry}{架式}{jia4shi5}{9,6}{⽊、⼷}
  \variantof{架势}
\end{entry}

\begin{entry}{架势}{jia4shi5}{9,8}{⽊、⼒}
  \definition{s.}{postura | atitude | posição (sobre um assunto, etc.)}
\end{entry}

\begin{entry}{假}{jia4}{11}{⼈}
  \definition[个,天]{s.}{feriado; férias; período de suspensão temporária do trabalho ou dos estudos, legal ou aprovado | licença; afastamento temporário; período de licença temporária do trabalho ou dos estudos, após aprovação}
  \seeref{假}{jia3}
\end{entry}

\begin{entry}{假期}{jia4 qi1}{11,12}{⼈、⽉}[HSK 2]
  \definition[个,段,次,种]{s.}{férias; feriados; período de licença}
\end{entry}

\begin{entry}{假日}{jia4 ri4}{11,4}{⼈、⽇}[HSK 6]
  \definition[节]{s.}{feriado; dia de folga}
\end{entry}

\begin{entry}{奸}{jian1}{6}{⼥}
  \definition{adj.}{perverso; maligno; traiçoeiro; malicioso}
  \definition{s.}{traidor; espião | pessoa perversa; pessoa traiçoeira | relações sexuais ilícitas; comportamento sexual impróprio}
  \definition{v.}{ter relações sexuais ilícitas}
\end{entry}

\begin{entry}{奸夫}{jian1fu1}{6,4}{⼥、⼤}
  \definition{s.}{homem adúltero}
\end{entry}

\begin{entry}{尖}{jian1}{6}{⼩}[HSK 6]
  \definition{adj.}{pontiagudo; afilado; agudo | agudo; estridente; penetrante | mesquinho; pão-duro | mordaz; cáustico}
  \definition{s.}{ponto; ponta; topo | o melhor do seu tipo; a melhor escolha; a nata da safra; uma pessoa ou coisa notável}
  \definition{v.}{tornar (a voz, etc.) aguda; estridente}
\end{entry}

\begin{entry}{坚}{jian1}{7}{⼟}
  \definition*{s.}{Sobrenome Jian}
  \definition{adj.}{duro; firme; sólido; forte | firme; resoluto; constante}
  \definition{s.}{fortaleza; fortificação; um ponto fortemente fortificado; coisas sólidas, principalmente referindo-se a posições | armadura}
\end{entry}

\begin{entry}{坚持}{jian1chi2}{7,9}{⼟、⼿}[HSK 3]
  \definition{v.}{persistir em; perseverar em; defender; insistir em; manter-se fiel a; aderir a; persistir com determinação e não desistir quando se depara com dificuldades | aderir a; insistir em; não alterar (os princípios, opiniões, pontos de vista originais, etc.)}
\end{entry}

\begin{entry}{坚定}{jian1ding4}{7,8}{⼟、⼧}[HSK 5]
  \definition{adj.}{firme; inabalável; inamovível; (posição, opinião, vontade, etc.) firme e estável, inabalável}
  \definition{v.}{fortalecer}
\end{entry}

\begin{entry}{坚固}{jian1gu4}{7,8}{⼟、⼞}[HSK 4]
  \definition{adj.}{firme; sólido; robusto; forte; durável; firmemente unidos e inquebráveis}
\end{entry}

\begin{entry}{坚决}{jian1jue2}{7,6}{⼟、⼎}[HSK 3]
  \definition{adj.}{firme; resoluto; (atitude, opinião, ação, etc.) determinado e inabalável}
\end{entry}

\begin{entry}{坚强}{jian1qiang2}{7,12}{⼟、⼸}[HSK 3]
  \definition{adj.}{forte; firme; convicto; (qualidades humanas, personalidade, determinação, etc.) firme e forte, não vacila diante das dificuldades}
  \definition{v.}{fortalecer; tornar forte; é a qualidade, a determinação, etc., que não vacilam}
\end{entry}

\begin{entry}{坚守}{jian1shou3}{7,6}{⼟、⼧}
  \definition{v.}{agarrar-se}
\end{entry}

\begin{entry}{间}{jian1}{7}{⾨}[HSK 1]
  \definition{clas.}{a menor unidade de uma casa; a menor unidade habitacional; cômodo}
  \definition{s.}{espaço entre duas partes  | (em um) tempo ou espaço definido | sala; quarto | uma seção de uma sala ou o espaço lateral entre dois pares de pilares | com um tempo ou espaço definido}
  \seeref{间}{jian4}
\end{entry}

\begin{entry}{浅}{jian1}{8}{⽔}
  \definition{adj.}{murmurando, fluindo suavemente, gorgolejando suavemente}
  \definition{s.}{(onomatopéia) som de água em movimento |}
\end{entry}

\begin{entry}{肩}{jian1}{8}{⾁}[HSK 5]
  \definition*{s.}{Sobrenome Jian}
  \definition{s.}{ombro; torso}
  \definition{v.}{assumir; empreender; carregar; suportar; suportar um fardo}
\end{entry}

\begin{entry}{肩膀}{jian1bang3}{8,14}{⾁、⾁}
  \definition{s.}{ombro}
\end{entry}

\begin{entry}{艰}{jian1}{8}{⾉}
  \definition{adj.}{difícil; duro}
\end{entry}

\begin{entry}{艰苦}{jian1ku3}{8,8}{⾉、⾋}[HSK 5]
  \definition{adj.}{duro; resistente; árduo; difícil; condições de trabalho ou de vida ruins que tornam as pessoas miseráveis}
\end{entry}

\begin{entry}{艰难}{jian1nan2}{8,10}{⾉、⾫}[HSK 5]
  \definition{adj.}{duro; árduo; difícil}
\end{entry}

\begin{entry}{兼}{jian1}{10}{⼋}
  \definition{conj.}{e (ocupando dois ou mais cargos (oficiais) ao mesmo tempo)}
\end{entry}

\begin{entry}{监}{jian1}{10}{⽫}
  \definition{s.}{prisão; cadeia}
  \definition{v.}{supervisionar; inspecionar; observar}
\end{entry}

\begin{entry}{监测}{jian1 ce4}{10,9}{⽫、⽔}[HSK 6]
  \definition{v.}{monitorar; supervisionar e testar}
\end{entry}

\begin{entry}{监督}{jian1du1}{10,13}{⽫、⽬}[HSK 6]
  \definition[个,位,名]{s.}{monitoramento; supervisão; pessoas que supervisionam}
  \definition{v.}{controlar; supervisionar; superintender; monitorar e supervisionar de perto}
\end{entry}

\begin{entry}{监狱}{jian1yu4}{10,9}{⽫、⽝}
  \definition{s.}{prisão}
\end{entry}

\begin{entry}{渐}{jian1}{11}{⽔}
  \definition{v.}{encharcar; ficar saturado com | fluir para}
  \seeref{渐}{jian4}
\end{entry}

\begin{entry}{煎}{jian1}{13}{⽕}
  \definition{v.}{fritar | refogar}
\end{entry}

\begin{entry}{煎饼}{jian1bing3}{13,9}{⽕、⾷}
  \definition[张]{s.}{jianbing, crepe chinês | panqueca}
\end{entry}

\begin{entry}{煎蛋}{jian1dan4}{13,11}{⽕、⾍}
  \definition{s.}{ovos fritos}
\end{entry}

\begin{entry}{俭}{jian3}{9}{⼈}
  \definition*{s.}{Sobrenome Jian}
  \definition{adj.}{econômico; frugal | querendo; faltando; curto}
\end{entry}

\begin{entry}{俭省}{jian3sheng3}{9,9}{⼈、⽬}
  \definition{adj.}{econômico}
\end{entry}

\begin{entry}{柬}{jian3}{9}{⽊}
  \definition*{s.}{Sobrenome Jian}
  \definition[张,封]{s.}{cartão; nota; carta; um termo geral para cartas, cartões de visita, postagens, etc.}
\end{entry}

\begin{entry}{柬埔寨}{jian3pu3zhai4}{9,10,14}{⽊、⼟、⼧}
  \definition*{s.}{Camboja}
\end{entry}

\begin{entry}{捡}{jian3}{10}{⼿}[HSK 6]
  \definition{v.}{coletar; reunir; apanhar; pegar coisas do chão}
\end{entry}

\begin{entry}{减}{jian3}{11}{⼎}[HSK 4]
  \definition*{s.}{Sobrenome Jian}
  \definition{v.}{subtrair; remover uma parte da quantidade original | reduzir; diminuir; cortar}
\end{entry}

\begin{entry}{减肥}{jian3fei2}{11,8}{⼎、⾁}[HSK 4]
  \definition{v.+compl.}{perder peso; dieta, exercícios, medicamentos, massagem, cirurgia, etc., para reduzir o excesso de gordura corporal, de modo que o grau de obesidade seja reduzido}
\end{entry}

\begin{entry}{减轻}{jian3 qing1}{11,9}{⼎、⾞}[HSK 5]
  \definition{v.}{aliviar; remeter; clarear; facilitar; mitigar}
\end{entry}

\begin{entry}{减少}{jian3shao3}{11,4}{⼎、⼩}[HSK 4]
  \definition{v.}{cair; reduzir; diminuir; subtrair uma parte}
\end{entry}

\begin{entry}{剪}{jian3}{11}{⼑}[HSK 5]
  \definition[把]{s.}{tesouras; tesouras de poda; cortadores | pinças; tenazes}
  \definition{v.}{cortar; aparar; tosquiar; cortar (com uma tesoura) | exterminar; eliminar; acabar com}
\end{entry}

\begin{entry}{剪刀}{jian3dao1}{11,2}{⼑、⼑}[HSK 5]
  \definition[把]{s.}{tesoura; instrumento de ferro para cortar tecido, papel, barbante, etc., com duas lâminas interligadas que podem ser abertas e fechadas}
\end{entry}

\begin{entry}{剪子}{jian3 zi5}{11,3}{⼑、⼦}[HSK 5]
  \definition[把]{s.}{cortador; tosquiadeira | tesoura}
\end{entry}

\begin{entry}{检}{jian3}{11}{⽊}
  \definition*{s.}{Sobrenome Jian}
  \definition{v.}{verificar; inspecionar; examinar | conter-se; ter cuidado na conduta}
\end{entry}

\begin{entry}{检测}{jian3 ce4}{11,9}{⽊、⽔}[HSK 4]
  \definition{v.}{testar; detectar; verificar}
\end{entry}

\begin{entry}{检查}{jian3cha2}{11,9}{⽊、⽊}[HSK 2]
  \definition[份,个,次]{s.}{autocrítica; reconhecer e criticar os próprios erros verbais ou escritos}
  \definition{v.}{verificar; inspecionar; examinar; verificar cuidadosamente para descobrir o problema | criticar a si mesmo; identificar seus pontos fracos e erros, e criticar seu próprio comportamento}
\end{entry}

\begin{entry}{检验}{jian3yan4}{11,10}{⽊、⾺}[HSK 5]
  \definition{v.}{testar; examinar; inspecionar}
\end{entry}

\begin{entry}{简}{jian3}{13}{⽵}
  \definition*{s.}{Sobrenome Jian}
  \definition{adj.}{simples; simplificado; breve (oposto a 繁) | breve; em resumo; em poucas palavras}
  \definition{s.}{Arcaico: tiras de bambu (para escrever) | carta; correspondência}
  \definition{v.}{simplificar | (literário) selecionar; escolher}
  \seealsoref{繁}{fan2}
\end{entry}

\begin{entry}{简单}{jian3dan1}{13,8}{⽵、⼗}[HSK 3]
  \definition{adj.}{simples; descomplicado; estrutura simples; poucas complicações; fácil de entender, usar ou lidar | comum; lugar-comum; (experiência, capacidade, etc.) comum (usado principalmente em frases negativas) | casual; simplificado; precipitado; pouco cuidadoso}
\end{entry}

\begin{entry}{简介}{jian3 jie4}{13,4}{⽵、⼈}[HSK 6]
  \definition{s.}{breve introdução; sinopse; relato resumido}
  \definition{v.}{fazer um breve relato de (algo)}
\end{entry}

\begin{entry}{简历}{jian3li4}{13,4}{⽵、⼚}[HSK 4]
  \definition[个,份]{s.}{currículo; \emph{curriculum vitae} (CV); notas biográficas}
\end{entry}

\begin{entry}{简直}{jian3zhi2}{13,8}{⽵、⽬}[HSK 3]
  \definition{adv.}{simplesmente; de forma alguma; praticamente; significa “exatamente assim” (tom exagerado)}
\end{entry}

\begin{entry}{见}{jian4}{4}{⾒}[HSK 1][Kangxi 147]
  \definition*{s.}{Sobrenome Jian}
  \definition{part.}{usado antes de um verbo para indicar voz passiva ou para expressar como isso me afeta}
  \definition{s.}{visão; ideia; opinião sobre algo; ponto de vista}
  \definition{v.}{ver; avistar | encontrar-se com; ser exposto a | parecer ser; mostrar evidência de | ver; referir-se a; indicar a fonte ou o local onde deve ser consultado | ver; encontrar; convocar}
  \seeref{见}{xian4}
\end{entry}

\begin{entry}{见到}{jian4 dao4}{4,8}{⾒、⼑}[HSK 2]
  \definition{v.}{ver | encontrar; esbarrar; deparar-se com}
\end{entry}

\begin{entry}{见过}{jian4 guo4}{4,6}{⾒、⾡}[HSK 2]
  \definition{s.}{visto (ver); já viu alguém ou algo; indica um momento no passado; alguém já viu ou encontrou um determinado objeto}
\end{entry}

\begin{entry}{见面}{jian4mian4}{4,9}{⾒、⾯}[HSK 1]
  \definition{v.+compl.}{encontrar-se com alguém;  ver um ao outro; ver alguém face-a-face}
\end{entry}

\begin{entry}{件}{jian4}{6}{⼈}[HSK 2]
  \definition*{s.}{Sobrenome Jian}
  \definition{clas.}{item; peça; artigo; usado para coisas individuais}
  \definition{s.}{refere-se a coisas que podem ser contadas uma a uma | papel; carta; documento; correspondência}
\end{entry}

\begin{entry}{间}{jian4}{7}{⾨}
  \definition{s.}{espaço entre as duas partes; abertura; lacuna}
  \definition{v.}{separar | semear a discórdia | desbastar (mudas); podar; remover ou arrancar as mudas em excesso}
  \seeref{间}{jian1}
\end{entry}

\begin{entry}{间或}{jian4huo4}{7,8}{⾨、⼽}
  \definition{adv.}{às vezes | ocasionalmente | de vez em quando}
\end{entry}

\begin{entry}{间接}{jian4jie1}{7,11}{⾨、⼿}[HSK 5]
  \definition{adj.}{indireto; de segunda mão; em oposição a 直接}
  \seealsoref{直接}{zhi2jie1}
\end{entry}

\begin{entry}{建}{jian4}{8}{⼵}[HSK 3]
  \definition*{s.}{Província de Fujian | Rio Jian Jiang (na província de Fujian) | Sobrenome Jian}
  \definition{v.}{construir; construir; erigir | estabelecer; configurar; fundar | propor; defender; apresentar (suas próprias opiniões)}
\end{entry}

\begin{entry}{建成}{jian4 cheng2}{8,6}{⼵、⼽}[HSK 3]
  \definition{v.}{terminar a construção}
\end{entry}

\begin{entry}{建立}{jian4li4}{8,5}{⼵、⽴}[HSK 3]
  \definition{v.}{estabelecer; construir; começar a construir | vir a ser; começar a surgir; começar a se formar}
\end{entry}

\begin{entry}{建立者}{jian4li4zhe3}{8,5,8}{⼵、⽴、⽼}
  \definition{s.}{fundador}
\end{entry}

\begin{entry}{建设}{jian4she4}{8,6}{⼵、⾔}[HSK 3]
  \definition{s.}{reconstrução; desenvolvimento; trabalhos relacionados com a construção}
  \definition{v.}{construir; edificar; (Estado ou coletividade) criar novos empreendimentos ou aumento de novas instalações}
\end{entry}

\begin{entry}{建设性}{jian4she4xing4}{8,6,8}{⼵、⾔、⼼}
  \definition{adj.}{construtivo}
  \definition{s.}{construtividade}
\end{entry}

\begin{entry}{建设者}{jian4she4zhe3}{8,6,8}{⼵、⾔、⽼}
  \definition{s.}{construtor}
\end{entry}

\begin{entry}{建议}{jian4yi4}{8,5}{⼵、⾔}[HSK 3]
  \definition[个,点,条]{s.}{proposta; sugestão; recomendação; para que alguém ou alguma coisa evolua para melhor, para o coletivo; pontos de vista e opiniões apresentados pelos líderes, etc.}
  \definition{v.}{propor; sugerir; recomendar; em relação a determinada pessoa ou situação, apresentar seus pontos de vista e opiniões ao coletivo, aos líderes ou a indivíduos, para que as coisas evoluam para melhor}
\end{entry}

\begin{entry}{建造}{jian4 zao4}{8,10}{⼵、⾡}[HSK 5]
  \definition{adj.}{indireto; de segunda mão; ter um relacionamento por meio de um terceiro (em oposição a 直接)}
  \seealsoref{直接}{zhi2jie1}
\end{entry}

\begin{entry}{建筑}{jian4zhu4}{8,12}{⼵、⽵}[HSK 5]
  \definition[座,幢,排]{s.}{construção; estrutura; edifício; prédio}
  \definition{v.}{construir; erguer; edificar; construir casas, estradas, pontes, etc.}
\end{entry}

\begin{entry}{剑}{jian4}{9}{⼑}[HSK 6]
  \definition[把,口]{s.}{espada; sabre; florete}
\end{entry}

\begin{entry}{剑客}{jian4ke4}{9,9}{⼑、⼧}
  \definition{s.}{espada | esgrimista, espadachim}
\end{entry}

\begin{entry}{贱}{jian4}{9}{⾙}
  \definition*{s.}{Sobrenome Jian}
  \definition{adj.}{baixo preço; barato (oposto a 贵) | humilde (oposto a 贵) | baixo; básico; desprezível | humilde; baixa posição social}
  \definition{pron.}{meu (autodepreciativo)}
  \seealsoref{贵}{gui4}
\end{entry}

\begin{entry}{健}{jian4}{10}{⼈}
  \definition{adj.}{forte; saudável; bem definido | ser forte em; ser bom em; apresentar um grau superior à média em determinado aspecto}
  \definition{v.}{fortalecer; endurecer; revigorar}
\end{entry}

\begin{entry}{健康}{jian4kang1}{10,11}{⼈、⼴}[HSK 2]
  \definition{adj.}{em forma; saudável; descreve que a pessoa está em ótimo estado físico ou mental, sem nenhum problema | sudável; tudo está normal, sem problemas | saudável; livre de doenças; bom para a saúde}
  \definition{s.}{saúde; físico; estado de saúde}
\end{entry}

\begin{entry}{健全}{jian4quan2}{10,6}{⼈、⼊}[HSK 5]
  \definition{adj.}{saudável; íntegro; capaz; apto; robusto e sem mácula | sólido; completo; perfeito}
  \definition{v.}{aperfeiçoar; melhorar; fortalecer; reforçar}
\end{entry}

\begin{entry}{健身}{jian4shen1}{10,7}{⼈、⾝}[HSK 4]
  \definition{s.}{exercício físico | \emph{fitness}}
  \definition{v.+compl.}{exercitar-se; manter a forma; praticar um esporte, especialmente a ginástica, inclusive em aparelhos, para desenvolver força, flexibilidade, aumentar a resistência, melhorar a coordenação e o controle de todas as partes do corpo}
\end{entry}

\begin{entry}{渐}{jian4}{11}{⽔}
  \definition{adv.}{gradualmente; por graus}
  \seeref{渐}{jian1}
\end{entry}

\begin{entry}{渐渐}{jian4 jian4}{11,11}{⽔、⽔}[HSK 4]
  \definition{adv.}{gradualmente; pouco a pouco; passo a passo; indica um aumento ou diminuição gradual em grau ou quantidade}
\end{entry}

\begin{entry}{鉴}{jian4}{13}{⾦}[HSK 6]
  \definition*{s.}{Sobrenome Jian}
  \definition{expr.}{uma expressão idiomática antiga usada para escrever cartas, depois da saudação inicial para pedir que alguém leia a carta}
  \definition{s.}{espelho (feito de bronze ou latão); espelho de bronze antigo | advertência; lição objetiva}
  \definition{v.}{Literário: refletir; espelhar | inspecionar; examinar; escrutinar; olhar cuidadosamente}
\end{entry}

\begin{entry}{鉴定}{jian4ding4}{13,8}{⾦、⼧}[HSK 6]
  \definition{s.}{avaliação dos pontos fortes e fracos de uma pessoa; avaliação de pessoas ou coisas}
  \definition{v.}{avaliar; identificar; autenticar; determinar; identificar e determinar (a autenticidade e a qualidade das coisas) | conduzir uma avaliação; avaliar o desempenho de uma pessoa ao longo de um determinado período de tempo}
\end{entry}

\begin{entry}{键}{jian4}{13}{⾦}[HSK 5]
  \definition[个]{s.}{pino (para máquinas) | tecla (de uma máquina de escrever, piano, etc.) | chave | etapa crucial}
\end{entry}

\begin{entry}{键盘}{jian4pan2}{13,11}{⾦、⽫}[HSK 5]
  \definition[个]{s.}{braço; teclado; cravo; painel de teclas; porta-chaves}
\end{entry}

\begin{entry}{箭}{jian4}{15}{⽵}[HSK 6]
  \definition[支]{s.}{seta | distância percorrida por uma flecha}
\end{entry}

\begin{entry}{江}{jiang1}{6}{⽔}[HSK 4]
  \definition*{s.}{Rio Changjiang | Sobrenome Jiang}
  \definition[条,道]{s.}{rio grande}
\end{entry}

\begin{entry}{江南水乡}{jiang1nan2shui3xiang1}{6,9,4,3}{⽔、⼗、⽔、⼄}
  \definition*{s.}{Vila Aquática de Jiangnan | Cidades Aquáticas}
\end{entry}

\begin{entry}{江水}{jiang1shui3}{6,4}{⽔、⽔}
  \definition{s.}{água do rio}
\end{entry}

\begin{entry}{江苏}{jiang1su1}{6,7}{⽔、⾋}
  \definition*{s.}{Província de Jiangsu}
\end{entry}

\begin{entry}{江西}{jiang1xi1}{6,6}{⽔、⾑}
  \definition*{s.}{Jiangxi}
\end{entry}

\begin{entry}{姜}{jiang1}{9}{⼥}
  \definition*{s.}{Sobrenome Jiang}
  \definition{s.}{gengibre}
\end{entry}

\begin{entry}{将}{jiang1}{9}{⼨}[HSK 5]
  \definition*{s.}{Sobrenome Jiang}
  \definition{adv.}{estar indo para; parcialmente\dots parcialmente\dots}
  \definition{part.}{expressar uma direção, como 进来, 出去; usado no meio de verbos e complementos que indicam tendência, como 进来, 出去, etc.}
  \definition{prep.}{com; por meio de; por | usado da mesma forma que 把}
  \definition{v.}{fazer algo; lidar com (um assunto) | dar um cheque-mate | cuidar (da saúde) | incitar alguém a agir; desafiar; estimular | segurar; pegar | colocar; tirar | levar; trazer | dar suporte; dar apoio}
  \seeref{将}{jiang4}
  \seeref{将}{qiang1}
  \seealsoref{把}{ba3}
  \seealsoref{出去}{chu1 qu4}
  \seealsoref{进来}{jin4 lai2}
\end{entry}

\begin{entry}{将近}{jiang1jin4}{9,7}{⼨、⾡}[HSK 3]
  \definition{adv.}{quase}
\end{entry}

\begin{entry}{将军}{jiang1jun1}{9,6}{⼨、⼍}[HSK 6]
  \definition[位,名]{s.}{general; geralmente se refere a generais seniores}
  \definition{v.+compl.}{dar xeque-mate; atacar o general ou rei do oponente no xadrez; colocar alguém em grandes apuros; metáfora para dar a alguém um problema difícil ou dificultar a tarefa para essa pessoa}
\end{entry}

\begin{entry}{将来}{jiang1lai2}{9,7}{⼨、⽊}[HSK 3]
  \definition[个]{s.}{no futuro (geralmente se refere a um período mais longo)}
\end{entry}

\begin{entry}{将要}{jiang1 yao4}{9,9}{⼨、⾑}[HSK 5]
  \definition{adv.}{irá; deverá; estará prestes a; irá a; indica que um ato ou situação ocorre logo em seguida}
\end{entry}

\begin{entry}{讲}{jiang3}{6}{⾔}[HSK 2]
  \definition[种]{s.}{palestra; discurso}
  \definition{v.}{contar; falar | explicar; transmitir oralmente; esclarecer | negociar; barganhar | ser exigente com; valorizar; dar importância}
\end{entry}

\begin{entry}{讲话}{jiang3 hua4}{6,8}{⾔、⾔}[HSK 2]
  \definition[个]{s.}{discurso; palestra | guia; introdução}
  \definition{v.}{falar; conversar; dirigir-se a alguém | criticar}
\end{entry}

\begin{entry}{讲究}{jiang3jiu5}{6,7}{⾔、⽳}[HSK 4]
  \definition{adj.}{requintado; elegante; de bom gosto; exigente com a vida e com outros aspectos, buscando alto nível, qualidade e detalhes}
  \definition{s.}{estudo cuidadoso; algo que merece atenção; elementos e aspectos que merecem atenção especial}
  \definition{v.}{dar ênfase a; ser específico sobre; prestar atenção a}
\end{entry}

\begin{entry}{讲课}{jiang3 ke4}{6,10}{⾔、⾔}[HSK 6]
  \definition{v.}{ensinar; dar palestras; proferir uma palestra | dar uma lição (palestra)}
\end{entry}

\begin{entry}{讲述}{jiang3shu4}{6,8}{⾔、⾡}
  \definition{v.}{falar sobre | narrar | descrever}
\end{entry}

\begin{entry}{讲座}{jiang3zuo4}{6,10}{⾔、⼴}[HSK 4]
  \definition[个]{s.}{palestra; um curso de palestras; a forma de instrução usada para ensinar um determinado assunto ou tópico, geralmente por meio de palestras ao vivo, seriados de rádio ou televisão ou seriados de jornal.}
\end{entry}

\begin{entry}{奖}{jiang3}{9}{⼤}[HSK 4]
  \definition[个,次]{s.}{prêmio; recompensa | elogio; loa}
  \definition{v.}{elogiar; recompensar; recomendar; incentivar}
\end{entry}

\begin{entry}{奖金}{jiang3jin1}{9,8}{⼤、⾦}[HSK 4]
  \definition[个,笔]{s.}{bônus; recompensa; prêmio; prêmio em dinheiro; dinheiro de recompensa, dinheiro dado às pessoas para incentivá-las ou elogiá-las por terem se saído bem em alguma coisa}
\end{entry}

\begin{entry}{奖励}{jiang3li4}{9,7}{⼤、⼒}[HSK 5]
  \definition{s.}{prêmio; recompensa; dinheiro ou honras dadas em troca de elogios ou incentivos}
  \definition{v.}{recompensar; incentivar; encorajar}
\end{entry}

\begin{entry}{奖学金}{jiang3 xue2 jin1}{9,8,8}{⼤、⼦、⾦}[HSK 4]
  \definition[笔]{s.}{bolsa de estudos; exposição; prêmios concedidos por escolas, organizações ou indivíduos a alunos com bom desempenho acadêmico}
\end{entry}

\begin{entry}{匠}{jiang4}{6}{⼕}
  \definition{s.}{artesão}
\end{entry}

\begin{entry}{降}{jiang4}{8}{⾩}[HSK 4]
  \definition*{s.}{Sobrenome Jiang}
  \definition{v.}{cair; descer | diminuir; reduzir | nascer}
\end{entry}

\begin{entry}{降低}{jiang4di1}{8,7}{⾩、⼈}[HSK 4]
  \definition{v.}{reduzir; cortar; diminuir; rebaixar; cair; abaixar}
\end{entry}

\begin{entry}{降价}{jiang4 jia4}{8,6}{⾩、⼈}[HSK 4]
  \definition{v.}{ficar mais barato; cortar o preço; reduzir o preço}
\end{entry}

\begin{entry}{降落}{jiang4luo4}{8,12}{⾩、⾋}[HSK 4]
  \definition{v.}{aterrissar; descer; descer do céu}
\end{entry}

\begin{entry}{降温}{jiang4 wen1}{8,12}{⾩、⽔}[HSK 4]
  \definition{v.}{baixar a temperatura (como em uma oficina);  recusar | cair a temperatura | esfriar; resfriar; metáfora para um declínio no entusiasmo ou uma diminuição no ímpeto de algo}
\end{entry}

\begin{entry}{将}{jiang4}{9}{⼨}
  \definition{s.}{general; nome do posto; abaixo de marechal de campo; acima de coronel}
  \definition{v.}{comandar; liderar}
  \seeref{将}{jiang1}
  \seeref{将}{qiang1}
\end{entry}

\begin{entry}{强}{jiang4}{12}{⼸}
  \definition{adj.}{teimoso; inflexível}
  \seeref{强}{qiang2}
  \seeref{强}{qiang3}
\end{entry}

\begin{entry}{酱}{jiang4}{13}{⾣}[HSK 6]
  \definition{adj.}{marinado em molho de soja; cozido em molho de soja}
  \definition{s.}{molho espesso feito de soja, farinha, etc. | molho; pasta; geleia | um condimento pastoso feito de feijão, trigo fermentados e sal}
  \definition{v.}{cozinhar ou conservar em molho de soja}
\end{entry}

\begin{entry}{酱油}{jiang4you2}{13,8}{⾣、⽔}[HSK 6]
  \definition[袋,瓶,壶,桶]{s.}{molho de soja}
\end{entry}

\begin{entry}{犟}{jiang4}{16}{⽜}
  \variantof{强}
\end{entry}

\begin{entry}{交}{jiao1}{6}{⼇}[HSK 2]
  \definition*{s.}{Sobrenome Jiao}
  \definition{adv.}{mutuamente; recíprocamente; um ao outro | juntos; simultaneamente}
  \definition{s.}{amigo; conhecido; amizade; relacionamento | transação comercial; negócio; barganha | queda}
  \definition{v.}{entregar | (de lugares ou períodos de tempo) cruzar; encontrar; unir | chegar (a uma determinada hora ou estação); estabelecer-se; vir | cruzar; intersectar | associar-se a | ter relações sexuais | acasalar; reproduzir-se | transferir as coisas para as partes interessadas | unir (lugares ou períodos de tempo)}
\end{entry}

\begin{entry}{交班}{jiao1ban1}{6,10}{⼇、⽟}
  \definition{v.}{passar para o próximo turno de trabalho}
\end{entry}

\begin{entry}{交杯酒}{jiao1bei1jiu3}{6,8,10}{⼇、⽊、⾣}
  \definition{s.}{copo de vinho nupcial}
\end{entry}

\begin{entry}{交叉}{jiao1cha1}{6,3}{⼇、⼜}
  \definition{v.}{cruzar | sobrepor}
\end{entry}

\begin{entry}{交叉点}{jiao1cha1dian3}{6,3,9}{⼇、⼜、⽕}
  \definition{s.}{encruzilhada | cruzamento | junção}
\end{entry}

\begin{entry}{交叉口}{jiao1cha1kou3}{6,3,3}{⼇、⼜、⼝}
  \definition{s.}{intersecção (rodovia)}
\end{entry}

\begin{entry}{交代}{jiao1dai4}{6,5}{⼇、⼈}[HSK 5]
  \definition{v.}{contar; entregar | ordenar; insistir; contar aos outros sobre suas intenções, instruções | contar; admitir}
\end{entry}

\begin{entry}{交叠}{jiao1die2}{6,13}{⼇、⼜}
  \definition{s.}{sobreposição}
\end{entry}

\begin{entry}{交费}{jiao1 fei4}{6,9}{⼇、⾙}[HSK 3]
  \definition{v.}{pagar taxas ou impostos; pagar uma taxa ou imposto}
\end{entry}

\begin{entry}{交给}{jiao1 gei3}{6,9}{⼇、⽷}[HSK 2]
  \definition{v.}{entregar para | dar para}
\end{entry}

\begin{entry}{交媾}{jiao1gou4}{6,13}{⼇、⼥}
  \definition{v.}{copular | ter relações sexuais}
\end{entry}

\begin{entry}{交换}{jiao1huan4}{6,10}{⼇、⼿}[HSK 4]
  \definition{v.}{trocar; permutar; comutar; intercambiar}
\end{entry}

\begin{entry}{交际}{jiao1ji4}{6,7}{⼇、⾩}[HSK 4]
  \definition{s.}{contato; comunicação; relações sociais; contato interpessoal, socialização}
\end{entry}

\begin{entry}{交界}{jiao1jie4}{6,9}{⼇、⽥}
  \definition{s.}{fronteira comum | limite comum | interface}
\end{entry}

\begin{entry}{交警}{jiao1 jing3}{6,19}{⼇、⾔}[HSK 3]
  \definition{s.}{policial de trânsito, abreviação de 交通警察}
  \seealsoref{交通警察}{jiao1tong1jing3cha2}
\end{entry}

\begin{entry}{交流}{jiao1liu2}{6,10}{⼇、⽔}[HSK 3]
  \definition{v.}{trocar; interagir; comunicar-se; compartilhar o que cada um tem com o outro}
\end{entry}

\begin{entry}{交朋友}{jiao1 peng2 you3}{6,8,4}{⼇、⽉、⼜}[HSK 2]
  \definition{v.}{fazer amizade com alguém; fazer amigos}
\end{entry}

\begin{entry}{交通}{jiao1tong1}{6,10}{⼇、⾡}[HSK 2]
  \definition{s.}{tráfego | ligação; conexão | transporte; termo genérico para todos os tipos de transporte, como ferroviário e rodoviário}
  \definition{v.}{conspirar; fazer amizades; conchavar | estar conectado; estar ligado; estar vinculado | associar-se a; conspirar com}
\end{entry}

\begin{entry}{交通警察}{jiao1tong1jing3cha2}{6,10,19,14}{⼇、⾡、⾔、⼧}
  \definition{s.}{policial de trânsito}
  \seealsoref{交警}{jiao1 jing3}
\end{entry}

\begin{entry}{交往}{jiao1wang3}{6,8}{⼇、⼻}[HSK 3]
  \definition{v.}{estar em contato com; associar-se a; interagir}
\end{entry}

\begin{entry}{交响}{jiao1xiang3}{6,9}{⼇、⼝}
  \definition{s.}{sinfonia}
\end{entry}

\begin{entry}{交易}{jiao1yi4}{6,8}{⼇、⽇}[HSK 3]
  \definition[笔,桩,个,场]{s.}{negócio; comércio; transação comercial; transação; atividades de compra e venda de mercadorias}
  \definition{v.}{negociar; comprar e vender mercadorias}
\end{entry}

\begin{entry}{交运}{jiao1yun4}{6,7}{⼇、⾡}
  \definition{v.}{despachar (bagagem em um aeroporto, etc.) | entregar para transporte}
\end{entry}

\begin{entry}{郊}{jiao1}{8}{⾢}
  \definition*{s.}{Sobrenome Jiao}
  \definition{s.}{subúrbios; periferias; áreas ao redor da cidade}
\end{entry}

\begin{entry}{郊区}{jiao1 qu1}{8,4}{⾢、⼖}[HSK 5]
  \definition[个,片,块]{s.}{subúrbios; arredores; periferia; área ao redor da cidade que está administrativamente sob a jurisdição da cidade}
\end{entry}

\begin{entry}{骄}{jiao1}{9}{⾺}
  \definition{adj.}{orgulhoso; arrogante; vaidoso | Literário: feroz; intenso; forte; violento}
\end{entry}

\begin{entry}{骄傲}{jiao1'ao4}{9,12}{⾺、⼈}[HSK 6]
  \definition{adj.}{arrogante; vaidoso; orgulhoso}
  \definition{s.}{orgulho; pessoas ou coisas das quais se orgulhar}
\end{entry}

\begin{entry}{胶}{jiao1}{10}{⾁}
  \definition*{s.}{Sobrenome Jiao}
  \definition{adj.}{pegajoso; viscoso; grudento}
  \definition{s.}{cola; goma; adesivo | borracha | gel; colóide}
  \definition{v.}{colar com cola | colar; grudar}
\end{entry}

\begin{entry}{胶带}{jiao1 dai4}{10,9}{⾁、⼱}[HSK 5]
  \definition[卷]{s.}{fita adesiva | fita de gravação | correia de borracha}
\end{entry}

\begin{entry}{胶卷}{jiao1juan3}{10,8}{⾁、⼙}
  \definition{s.}{filme | rolo de filme}
\end{entry}

\begin{entry}{胶水}{jiao1shui3}{10,4}{⾁、⽔}[HSK 5]
  \definition[瓶]{s.}{cola; mucilagem; cola líquida}
\end{entry}

\begin{entry}{教}{jiao1}{11}{⽁}[HSK 1]
  \definition*{s.}{Sobrenome Jiao}
  \definition{prep.}{em uma frase passiva para introduzir o executor da ação}
  \definition{s.}{religião | professor; referência à educação ou aos professores}
  \definition{v.}{ensinar; instruir |  pedir; ordenar; dizer | permitir; possibilitar}
  \seeref{教}{jiao4}
\end{entry}

\begin{entry}{教会}{jiao1hui4}{11,6}{⽁、⼈}
  \definition{v.}{mostrar | ensinar}
  \seeref{教会}{jiao4hui4}
\end{entry}

\begin{entry}{焦}{jiao1}{12}{⽕}
  \definition*{s.}{Sobrenome Jiao}
  \definition{adj.}{queimado; chamuscado; carbonizado | preocupado; ansioso}
  \definition{clas.}{J; Joule, abreviação}
  \definition{pref.}{(química) piro-}
  \definition{s.}{Metalurgia: coque}
\end{entry}

\begin{entry}{焦点}{jiao1dian3}{12,9}{⽕、⽕}[HSK 6]
  \definition{s.}{foco; ponto focal; Matemática: refere-se a um ponto que tem uma relação especial com uma elipse, hipérbole, parábola, etc. | foco; ponto focal; Óptica: refere-se à intersecção de feixes de luz paralelos após serem refratados por uma lente ou refletidos por um espelho curvo | foco; questão central; metaforicamente, uma coisa ou princípio que chama a atenção para o foco}
\end{entry}

\begin{entry}{焦虑}{jiao1lv4}{12,10}{⽕、⾌}
  \definition{adj.}{ansioso | preocupado | apreensivo}
\end{entry}

\begin{entry}{角}{jiao3}{7}{⾓}[HSK 2][Kangxi 148]
  \definition*{s.}{Jiao, uma das mansões lunares}
  \definition{clas.}{uma unidade monetária fracionária na China (=1/10 de um yuan ou 10 fen)}
  \definition[个,只,对]{s.}{chifre; o objeto duro que cresce na cabeça de bovinos, ovinos, veados, etc. | buzina; corneta; instrumentos musicais tocados no exército antigo | algo com a forma de um chifre | cabo; promontório; península | esquina; canto; a junção entre duas arestas de um objeto | ângulo}
  \seeref{角}{jue2}
\end{entry}

\begin{entry}{角度}{jiao3du4}{7,9}{⾓、⼴}[HSK 2]
  \definition[个,种]{s.}{perspectiva; ponto de vista; o ponto de partida para ver as coisas | ângulo; o tamanho do ângulo; normalmente expresso em graus ou radianos}
\end{entry}

\begin{entry}{饺}{jiao3}{9}{⾷}
  \definition[盘,碗,顿,个]{s.}{bolinho de massa; \emph{dumpling}}
\end{entry}

\begin{entry}{饺子}{jiao3zi5}{9,3}{⾷、⼦}[HSK 2]
  \definition[个,盘,碗,锅]{s.}{jiaozi; bolinho chinês; bolinho de massa}
\end{entry}

\begin{entry}{脚}{jiao3}{11}{⾁}[HSK 2]
  \definition{clas.}{usado para chutes}
  \definition[只,双]{s.}{pé; a parte inferior das pernas de pessoas ou animais, que entra em contato com o solo | base; pé; a parte inferior do objeto | antigamente, referia-se ao trabalho físico de transporte de cargas | resíduos; sobras}
\end{entry}

\begin{entry}{脚步}{jiao3 bu4}{11,7}{⾁、⽌}[HSK 5]
  \definition{s.}{pé; passo; pisada; refere-se ao movimento das pernas ao caminhar | ritmo; passo; distância entre os pés dianteiros e traseiros ao caminhar}
\end{entry}

\begin{entry}{脚印}{jiao3 yin4}{11,5}{⾁、⼙}[HSK 6]
  \definition{s.}{trilha; pegada; marca de pé; os rastros deixados pelos passos}
\end{entry}

\begin{entry}{叫}{jiao4}{5}{⼝}[HSK 1,3]
  \definition{adj.}{macho (animal)}
  \definition{prep.}{usado em frases passivas; introduz o agente da ação; equivalente a 被 | combinado com 看, 说; usado para expressar suas ideias e pontos de vista}
  \definition{v.}{chorar; gritar; berrar | nomear; chamar | chamar; chamar a atenção | cumprimentar; saudar; dizer olá | pedir; ordenar; licitar | permitir; concordar com algo; concordar em fazer algo | contratar; encomendar; comprar o que você precisa}
  \seealsoref{被}{bei4}
  \seealsoref{看}{kan4}
  \seealsoref{说}{shuo1}
\end{entry}

\begin{entry}{叫作}{jiao4 zuo4}{5,7}{⼝、⼈}[HSK 2]
  \definition{v.}{ser chamado de; ser conhecido como}
\end{entry}

\begin{entry}{觉}{jiao4}{9}{⾒}[HSK 6]
  \definition[个]{s.}{sono; o processo desde adormecer até acordar}
  \seeref{觉}{jue2}
\end{entry}

\begin{entry}{校}{jiao4}{10}{⽊}
  \definition{v.}{verificar | comparar | revisar}
  \seeref{校}{xiao4}
\end{entry}

\begin{entry}{较}{jiao4}{10}{⾞}[HSK 3]
  \definition{adj.}{claro; óbvio; evidente}
  \definition{adv.}{comparativamente; relativamente; razoavelmente; bastante; bastante}
  \definition{prep.}{usado para comparar características e graus; introduzir o objeto de comparação; equivalente a 比}
  \definition{v.}{comparar | disputar}
  \seealsoref{比}{bi3}
\end{entry}

\begin{entry}{敎}{jiao4}{11}{⽁}
  \variantof{教}
\end{entry}

\begin{entry}{教}{jiao4}{11}{⽁}
  \definition*{s.}{Sobrenome Jiao}
  \definition{s.}{religião | ensinamento}
  \definition{v.}{causar | como fazer algo | contar (explicar como fazer algo)}
  \seeref{教}{jiao1}
\end{entry}

\begin{entry}{教材}{jiao4cai2}{11,7}{⽁、⽊}[HSK 3]
  \definition[本,套]{s.}{livro didático; materiais didáticos, incluindo livros didáticos, apostilas, materiais de referência, vídeos, imagens, etc.}
\end{entry}

\begin{entry}{教导}{jiao4dao3}{11,6}{⽁、⼨}
  \definition{s.}{instrução | orientação | ensino}
  \definition{v.}{instruir | orientar | ensinar}
\end{entry}

\begin{entry}{教官}{jiao4guan1}{11,8}{⽁、⼧}
  \definition{s.}{instrutor militar}
\end{entry}

\begin{entry}{教会}{jiao4hui4}{11,6}{⽁、⼈}
  \definition{s.}{igreja cristã}
  \seeref{教会}{jiao1hui4}
\end{entry}

\begin{entry}{教练}{jiao4lian4}{11,8}{⽁、⽷}[HSK 3]
  \definition[个,位,名]{s.}{instrutor; treinador (esportes); pessoas que trabalham como treinadores}
  \definition{v.}{treinar; treinar outras pessoas para dominarem uma determinada técnica (como esportes, dirigir carros, pilotar aviões, etc.)}
\end{entry}

\begin{entry}{教师}{jiao4 shi1}{11,6}{⽁、⼱}[HSK 2]
  \definition[个,位,名]{s.}{professor; professor de escola}
\end{entry}

\begin{entry}{教室}{jiao4shi4}{11,9}{⽁、⼧}[HSK 2]
  \definition[间]{s.}{sala de aula}
\end{entry}

\begin{entry}{教授}{jiao4shou4}{11,11}{⽁、⼿}[HSK 4]
  \definition[个,位]{s.}{professor (universitário)}
  \definition{v.}{ensinar; instruir; dar aulas; dar palestras}
\end{entry}

\begin{entry}{教堂}{jiao4tang2}{11,11}{⽁、⼟}[HSK 6]
  \definition[座,所,间]{s.}{igreja; capela; catedral; casa de deus; um lugar onde os cristãos realizam cerimônias religiosas}
\end{entry}

\begin{entry}{教学}{jiao4 xue2}{11,8}{⽁、⼦}[HSK 2]
  \definition[个,门]{s.}{ensino; educação; o processo de transmissão de conhecimentos e habilidades}
\end{entry}

\begin{entry}{教学楼}{jiao4 xue2 lou2}{11,8,13}{⽁、⼦、⽊}[HSK 1]
  \definition{s.}{prédio da escola; bloco de ensino; edifícios utilizados para atividades educacionais, geralmente incluindo salas de aula, laboratórios, auditórios, etc.}
\end{entry}

\begin{entry}{教训}{jiao4xun4}{11,5}{⽁、⾔}[HSK 4]
  \definition{s.}{moral; lição}
  \definition{v.}{repreender; educar; ensinar uma lição a alguém; dar uma bronca em alguém; dar um sermão em alguém (por ter cometido um erro, etc.)}
\end{entry}

\begin{entry}{教育}{jiao4yu4}{11,8}{⽁、⾁}[HSK 2]
  \definition{s.}{educação; refere-se a atividades sociais cujo objetivo direto é influenciar o desenvolvimento físico e mental das pessoas; refere-se principalmente ao processo de formação dos alunos nas escolas}
  \definition{v.}{ensinar; educar; inspirar, fazer compreender a razão}
\end{entry}

\begin{entry}{教育部}{jiao4 yu4 bu4}{11,8,10}{⽁、⾁、⾢}[HSK 6]
  \definition*{s.}{Ministério da Educação}
\end{entry}

\begin{entry}{教长}{jiao4zhang3}{11,4}{⽁、⾧}
  \definition{s.}{imã (Islã) | mulá}
\end{entry}

\begin{entry}{节}{jie1}{5}{⾋}
  \definition{adj.}{momento crucial; momento crítico; momento decisivo; metáfora para algo importante, decisivo ou oportuno}
  \seeref{节}{jie2}
\end{entry}

\begin{entry}{阶}{jie1}{6}{⾩}
  \definition{s.}{degrau; escada; escadaria | classificação | escala | ordem | estágio}
\end{entry}

\begin{entry}{阶段}{jie1duan4}{6,9}{⾩、⽎}[HSK 4]
  \definition{s.}{estágio; fase; período; bancada; gradação}
\end{entry}

\begin{entry}{皆}{jie1}{9}{⽩}
  \definition{adv.}{todos | em todos os casos}
\end{entry}

\begin{entry}{结}{jie1}{9}{⽷}
  \definition{v.}{dar (frutos); formar (sementes); produzir frutos ou sementes (uma planta)}
  \seeref{结}{jie2}
\end{entry}

\begin{entry}{结果}{jie1guo3}{9,8}{⽷、⽊}
  \definition{v.}{dar frutos}
  \seeref{结果}{jie2guo3}
\end{entry}

\begin{entry}{结实}{jie1shi5}{9,8}{⽷、⼧}[HSK 3]
  \definition{adj.}{sólido; resistente; durável | forte; resistente; robusto}
\end{entry}

\begin{entry}{接}{jie1}{11}{⼿}[HSK 2]
  \definition*{s.}{Sobrenome Jie}
  \definition{v.}{entrar em contato com; aproximar-se de | conectar; unir; juntar | continuar; prosseguir | assumir o controle; assumir o trabalho de outra pessoa e continuar a fazê-lo | pegar; agarrar; segurar ou sustentar com as mãos | receber; aceitar | encontrar; dar as boas-vindas}
\end{entry}

\begin{entry}{接班人}{jie1ban1ren2}{11,10,2}{⼿、⽟、⼈}
  \definition{s.}{sucessor}
\end{entry}

\begin{entry}{接触}{jie1chu4}{11,13}{⼿、⾓}[HSK 5]
  \definition{v.}{entrar em contato com | entrar em contato; tocar; interagir | engajar; o termo militar refere-se a fogo cruzado}
\end{entry}

\begin{entry}{接待}{jie1dai4}{11,9}{⼿、⼻}[HSK 3]
  \definition{v.}{receber (alguém); acolher; recepcionar; receber com cordialidade e generosidade}
\end{entry}

\begin{entry}{接到}{jie1 dao4}{11,8}{⼿、⼑}[HSK 2]
  \definition{v.}{receber (carta, etc.)}
\end{entry}

\begin{entry}{接(电话)}{jie1(dian4hua4)}{11,5,8}{⼿、⽥、⾔}
  \definition{v.}{atender (o telefone) | receber (uma ligação telefônica)}
\end{entry}

\begin{entry}{接近}{jie1jin4}{11,7}{⼿、⾡}[HSK 3]
  \definition{adj.}{perto; próximo; a diferença entre os dois é mínima}
  \definition{v.}{estar perto de; aproximar; aproximar-se}
\end{entry}

\begin{entry}{接连}{jie1lian2}{11,7}{⼿、⾡}[HSK 5]
  \definition{adv.}{no final; em sucessão; em uma fileira; um após o outro; seguindo o anterior; continuando}
\end{entry}

\begin{entry}{接收}{jie1 shou1}{11,6}{⼿、⽁}[HSK 6]
  \definition{v.}{aceitar; receber | assumir; expropriar; tomar posse (de uma instituição, propriedade, etc.) de acordo com a lei | admitir; aceitar; absorver}
\end{entry}

\begin{entry}{接受}{jie1shou4}{11,8}{⼿、⼜}[HSK 2]
  \definition{v.}{aceitar; não recusar (o que os outros oferecem) | concordar; não recusar (opiniões/sugestões/críticas/convites de outras pessoas, etc.)}
\end{entry}

\begin{entry}{接下来}{jie1 xia4 lai2}{11,3,7}{⼿、⼀、⽊}[HSK 2]
  \definition{expr.}{próximo; seguinte; indica uma sequência temporal subsequente}
\end{entry}

\begin{entry}{接着}{jie1zhe5}{11,11}{⼿、⽬}[HSK 2]
  \definition{adv.}{por sua vez; um após o outro; sucessivamente; conectado (à frase anterior); imediatamente após (a ação anterior)}
  \definition{v.}{seguir; prosseguir; continuar; seguir em frente; ficar ao lado | pegar com as mãos; apanhar}
\end{entry}

\begin{entry}{揭}{jie1}{12}{⼿}[HSK 6]
  \definition*{s.}{Sobrenome Jie}
  \definition{v.}{rasgar; arrancar; tirar | descobrir; levantar (a tampa, etc.) | expor; mostrar; trazer à luz | (literário) levantar; içar}
\end{entry}

\begin{entry}{街}{jie1}{12}{⾏}[HSK 2]
  \definition[条]{s.}{rua; avenida com prédios dos dois lados | mercado; feira rural}
\end{entry}

\begin{entry}{街道}{jie1dao4}{12,12}{⾏、⾡}[HSK 4]
  \definition[条]{s.}{caminho; rua; estrada; via pública com casas em ambos os lados, relativamente larga | escritório do subdistrito; tipo de organização responsável por gerenciar determinados aspectos da rua}
\end{entry}

\begin{entry}{街头}{jie1 tou2}{12,5}{⾏、⼤}[HSK 6]
  \definition{s.}{rua; esquina da rua}
\end{entry}

\begin{entry}{街舞}{jie1wu3}{12,14}{⾏、⾇}
  \definition{s.}{dança de rua, \emph{street dance} (por exemplo, \emph{breakdance})}
\end{entry}

\begin{entry}{节}{jie2}{5}{⾋}[HSK 2]
  \definition*{s.}{Sobrenome Jie}
  \definition{clas.}{nó (kn), velocidade de um barco | para seções, comprimentos}
  \definition[个]{s.}{junta; botão; nó; geralmente se refere à parte da grama ou caule da grama onde as folhas crescem ou à parte onde os galhos e troncos das plantas são conectados | parte; divisão; um trecho de algo interligado; uma parte do todo | festival; feriado; dia memorável; um período de tempo ou um dia com características específicas | item; assunto | castidade; integridade ética e moral | articulação; o local onde os ossos humanos ou animais se conectam | etiqueta; cerimonial | batida; ritmo | registro; documento utilizado na antiguidade para comprovar a identidade | estação do ano | sílaba}
  \definition{v.}{economizar; conservar; poupar | resumir; extrair; retirar uma parte do todo | controlar; restringir; moderar}
  \seeref{节}{jie1}
\end{entry}

\begin{entry}{节假日}{jie2 jia4 ri4}{5,11,4}{⾋、⼈、⽇}[HSK 6]
  \definition[个]{s.}{feriados; festivais e feriados}
\end{entry}

\begin{entry}{节目}{jie2mu4}{5,5}{⾋、⽬}[HSK 2]
  \definition[个,场,项,台]{s.}{programa; item (em um programa); programas artísticos ou projetos transmitidos por rádios e televisões}
\end{entry}

\begin{entry}{节能}{jie2 neng2}{5,10}{⾋、⾁}[HSK 6]
  \definition{v.}{economizar no consumo de energia; conservar energia}
\end{entry}

\begin{entry}{节日}{jie2ri4}{5,4}{⾋、⽇}[HSK 2]
  \definition[个,种,类]{s.}{festival; feriado; dia de comemoração tradicional; dia comemorativo estabelecido por lei}
\end{entry}

\begin{entry}{节省}{jie2sheng3}{5,9}{⾋、⽬}[HSK 4]
  \definition{adj.}{econômico; parcimonioso}
  \definition{v.}{economizar; conservar; usar com moderação; reduzir; eliminar ou minimizar o esgotamento de itens potencialmente esgotáveis}
\end{entry}

\begin{entry}{节约}{jie2yue1}{5,6}{⾋、⽷}[HSK 3]
  \definition{adj.}{econômico; sem luxo}
  \definition{v.}{guardar; economizar; usar com moderação; economizar gastos desnecessários}
\end{entry}

\begin{entry}{节奏}{jie2zou4}{5,9}{⾋、⼤}[HSK 6]
  \definition[个,种]{s.}{ritmo; o fenômeno da alternância regular de comprimento, força e fraqueza das notas na música | padrão regular; uma metáfora para um processo de ajuste adequado com tensão e relaxamento}
\end{entry}

\begin{entry}{杰}{jie2}{8}{⽊}
  \definition{adj.}{notável; proeminente; fora do comum}
  \definition[位,名,个,些]{s.}{pessoa excepcional; herói; uma pessoa com talentos excepcionais}
\end{entry}

\begin{entry}{杰出}{jie2chu1}{8,5}{⽊、⼐}[HSK 6]
  \definition{adj.}{notável; proeminente; (talento, realização) excepcional}
\end{entry}

\begin{entry}{拮}{jie2}{9}{⼿}
  \definition{adj.}{trabalhoso | sem dinheiro | antagônico | trabalhando duro | pressionado}
\end{entry}

\begin{entry}{拮据}{jie2ju1}{9,11}{⼿、⼿}
  \definition{adj.}{em circunstâncias difíceis; sem dinheiro; em dificuldades}
\end{entry}

\begin{entry}{结}{jie2}{9}{⽷}[HSK 4]
  \definition*{s.}{Sobrenome Jie}
  \definition{s.}{nó | declaração juramentada; garantia por escrito; documento que, antigamente, significava um reconhecimento de encerramento ou uma garantia de responsabilidade}
  \definition{v.}{amarrar; tricotar; dar nó; tecer | formar; forjar; cimentar; solidificar | resolver; concluir | combinar; formar um relacionamento}
  \seeref{结}{jie1}
\end{entry}

\begin{entry}{结构}{jie2gou4}{9,8}{⽷、⽊}[HSK 4]
  \definition[个,座]{s.}{estrutura; composição; construção; formação; constituição; tecido; forma; sistematização; mecânica; organização | arquitetura; estrutura; construção; construção de partes de edifícios com suporte de carga ou com carga externa | textura (geológico)}
\end{entry}

\begin{entry}{结果}{jie2guo3}{9,8}{⽷、⽊}[HSK 2]
  \definition{conj.}{como resultado | no final}
  \definition{s.}{resultado | conclusão | consequência}
  \definition{v.}{despachar | matar}
  \seeref{结果}{jie1guo3}
\end{entry}

\begin{entry}{结合}{jie2he2}{9,6}{⽷、⼝}[HSK 3]
  \definition{v.}{ligar; unir; combinar; integrar; formar uma relação estreita entre pessoas ou coisas | casar-se; unir-se em matrimônio; referir-se especificamente a casais}
\end{entry}

\begin{entry}{结婚}{jie2hun1}{9,11}{⽷、⼥}[HSK 3]
  \definition{v.+compl.}{casar; casar-se; casar-se bem;}
\end{entry}

\begin{entry}{结婚礼服}{jie2hun1 li3 fu2}{9,11,5,8}{⽷、⼥、⽰、⽉}
  \definition{s.}{vestido de casamento}
\end{entry}

\begin{entry}{结局}{jie2ju2}{9,7}{⽷、⼫}
  \definition{s.}{conclusão | fim | final}
\end{entry}

\begin{entry}{结论}{jie2lun4}{9,6}{⽷、⾔}[HSK 4]
  \definition[个]{s.}{conclusão; palavra final sobre uma pessoa ou coisa após investigação e pesquisa | veredito; julgamento deduzido de premissas também é chamado de conclusão}
\end{entry}

\begin{entry}{结社自由}{jie2she4zi4you2}{9,7,6,5}{⽷、⽰、⾃、⽥}
  \definition{s.}{(constitucional) liberdade de associação}
\end{entry}

\begin{entry}{结束}{jie2shu4}{9,7}{⽷、⽊}[HSK 3]
  \definition{v.}{finalizar; fechar; terminar; concluir; encerrar; desenvolver ou avançar até a fase final, sem continuidade}
\end{entry}

\begin{entry}{结束辩论}{jie2shu4 bian4 lun4}{9,7,16,6}{⽷、⽊、⾟、⾔}
  \definition{s.}{debate de encerramento}
\end{entry}

\begin{entry}{结束工作}{jie2shu4gong1zuo4}{9,7,3,7}{⽷、⽊、⼯、⼈}
  \definition{s.}{trabalho final}
  \definition{v.}{terminar o trabalho}
\end{entry}

\begin{entry}{结束剂}{jie2shu4 ji4}{9,7,8}{⽷、⽊、⼑}
  \definition{s.}{finalizador}
\end{entry}

\begin{entry}{结束区}{jie2shu4 qu1}{9,7,4}{⽷、⽊、⼖}
  \definition{s.}{zona final}
\end{entry}

\begin{entry}{结束文本}{jie2shu4 wen2ben3}{9,7,4,5}{⽷、⽊、⽂、⽊}
  \definition{s.}{texto final}
\end{entry}

\begin{entry}{结束语}{jie2shu4yu3}{9,7,9}{⽷、⽊、⾔}
  \definition{s.}{conclusões finais | considerações finais}
\end{entry}

\begin{entry}{捷}{jie2}{11}{⼿}
  \definition*{s.}{Sobrenome Jie}
  \definition{adj.}{rápido; ágil}
  \definition{s.}{vitória; triunfo; sucesso}
\end{entry}

\begin{entry}{捷径}{jie2jing4}{11,8}{⼿、⼻}
  \definition{s.}{atalho}
\end{entry}

\begin{entry}{截}{jie2}{14}{⼽}
  \definition{clas.}{seção; pedaço; comprimento}
  \definition{prep.}{por (um tempo especificado); até}
  \definition{v.}{cortar; romper | parar; verificar; interromper; interceptar}
\end{entry}

\begin{entry}{截止}{jie2zhi3}{14,4}{⼽、⽌}[HSK 6]
  \definition{adv.}{até (um certo limite de tempo); por (um tempo especificado)}
\end{entry}

\begin{entry}{截至}{jie2zhi4}{14,6}{⼽、⾄}[HSK 6]
  \definition{adv.}{a partir de; até (um certo limite de tempo); por (um tempo especificado)}
\end{entry}

\begin{entry}{姐}{jie3}{8}{⼥}[HSK 1]
  \definition[个,位]{s.}{irmã mais velha; irmã | termo genérico para mulheres jovens | mulheres da mesma geração que são mais velhas do que você (geralmente não inclui aquelas que podem ser chamadas de cunhadas) | um título respeitoso para mulheres jovens profissionais em determinados cargos}
  \seealsoref{姐姐}{jie3 jie5}
\end{entry}

\begin{entry}{姐夫}{jie3fu5}{8,4}{⼥、⼤}
  \definition{s.}{marido da irmã mais velha}
\end{entry}

\begin{entry}{姐姐}{jie3 jie5}{8,8}{⼥、⼥}[HSK 1]
  \definition[个]{s.}{irmã mais velha}
\end{entry}

\begin{entry}{姐妹}{jie3 mei4}{8,8}{⼥、⼥}[HSK 4]
  \definition[个]{s.}{irmãs}
\end{entry}

\begin{entry}{解}{jie3}{13}{⾓}[HSK 6]
  \definition{s.}{solução; o valor de uma variável desconhecida em uma equação algébrica}
  \definition{v.}{dividir; separar | desfazer; desatar; abrir algo que esteja amarrado ou encadernado | acalmar; dissipar; dispensar; eliminar | resolver; explicar; interpretar | entender; compreender | aliviar-se (excreção de urina e fezes) | dissolver; desintegrar | (cálculo analítico) resolver; solucionar}
\end{entry}

\begin{entry}{解除}{jie3chu2}{13,9}{⾓、⾩}[HSK 5]
  \definition{v.}{remover; aliviar; livrar-se de; eliminar}
\end{entry}

\begin{entry}{解放}{jie3fang4}{13,8}{⾓、⽅}[HSK 5]
  \definition{v.}{libertar; emancipar; eliminar as restrições para permitir o desenvolvimento da liberdade}
\end{entry}

\begin{entry}{解雇}{jie3gu4}{13,12}{⾓、⾫}
  \definition{v.}{demitir}
\end{entry}

\begin{entry}{解救}{jie3jiu4}{13,11}{⾓、⽁}
  \definition{v.}{resgatar | ajudar a sair de dificuldades | salvar a situação}
\end{entry}

\begin{entry}{解决}{jie3jue2}{13,6}{⾓、⼎}[HSK 3]
  \definition{v.}{solucionar; resolver; liquidar; resolver problemas com resultados | acabar com; descartar; eliminar (o inimigo)}
\end{entry}

\begin{entry}{解开}{jie3 kai1}{13,4}{⾓、⼶}[HSK 3]
  \definition{v.}{desatar; desamarrar; desabotoar; desamarrar ou desfazer nós}
\end{entry}

\begin{entry}{解释}{jie3shi4}{13,12}{⾓、⾤}[HSK 4]
  \definition{v.}{explicar; expor; interpretar | analisar; explicaro significado, razões, justificativas, etc.}
\end{entry}

\begin{entry}{解说}{jie3 shuo1}{13,9}{⾓、⾔}[HSK 6]
  \definition{v.}{narrar; comentar; fazer um comentário; explicar oralmente}
\end{entry}

\begin{entry}{解压}{jie3ya1}{13,6}{⾓、⼚}
  \definition{v.}{aliviar o estresse | (computação) descomprimir}
\end{entry}

\begin{entry}{介}{jie4}{4}{⼈}
  \definition*{s.}{Sobrenome Jie}
  \definition{adj.}{direto; honesto e franco; correto}
  \definition{s.}{armadura | concha (crustáceos e criaturas aquáticas) | preposição}
  \definition{v.}{estar situado entre; interpor | levar a sério; levar em conta; ter em mente}
\end{entry}

\begin{entry}{介绍}{jie4shao4}{4,8}{⼈、⽷}[HSK 1]
  \definition{s.}{introdução; apresentação}
  \definition{v.}{introduzir; apresentar | recomendar; sugerir | dar a conhecer; informar}
\end{entry}

\begin{entry}{戒}{jie4}{7}{⼽}[HSK 5]
  \definition[个]{s.}{advertência; exortação | disciplina monástica budista; preceitos budistas | anel (dedo)}
  \definition{v.}{proteger-se contra; estar preparado; estar atento | advertir; exortar; admoestar | abandonar; parar; desistir; desistir (de um hábito ruim)}
\end{entry}

\begin{entry}{芥}{jie4}{7}{⾋}
  \definition{s.}{mostarda}
  \seeref{芥}{gai4}
\end{entry}

\begin{entry}{芥兰}{jie4lan2}{7,5}{⾋、⼋}
  \definition{s.}{couve}
\end{entry}

\begin{entry}{届}{jie4}{8}{⼫}[HSK 5]
  \definition{clas.}{sessões (de uma conferência); anos (de graduação); quantificador, ligeiramente equivalente a 次, usado para reuniões regulares ou turmas de formandos, etc.}
  \definition{v.}{vencer o prazo}
  \seealsoref{次}{ci4}
\end{entry}

\begin{entry}{界}{jie4}{9}{⽥}[HSK 6]
  \definition{s.}{fronteira; limite | escopo; extensão | círculos | divisão primária; reino | era geológica | (matemática) limite | mundo; faixa dividida por ocupação, emprego ou gênero, etc. | grupo}
\end{entry}

\begin{entry}{界碑}{jie4bei1}{9,13}{⽥、⽯}
  \definition{s.}{marco de fronteira}
\end{entry}

\begin{entry}{借}{jie4}{10}{⼈}[HSK 2]
  \definition{adv.}{por meio de}
  \definition{v.}{emprestar | pedir emprestado | usar como pretexto | aproveitar; tirar proveito (de uma oportunidade, etc.)}
\end{entry}

\begin{entry}{借鉴}{jie4jian4}{10,13}{⼈、⾦}[HSK 6]
  \definition{s.}{tirar lições de; aproveitar a experiência de; ganhar experiência e lições com o passado ou com as experiências de outras pessoas}
\end{entry}

\begin{entry}{借书证}{jie4shu1zheng4}{10,4,7}{⼈、⼄、⾔}
  \definition{s.}{cartão de biblioteca | (literalmente) cartão para pedir emprestado livros}
\end{entry}

\begin{entry}{今}{jin1}{4}{⼈}
  \definition*{s.}{Sobrenome Jin}
  \definition{s.}{agora; o presente | moderno (em oposição a 古) | de hoje; deste ano | isso; isto}
  \seealsoref{古}{gu3}
\end{entry}

\begin{entry}{今后}{jin1 hou4}{4,6}{⼈、⼝}[HSK 2]
  \definition{s.}{a partir de agora; doravante; no futuro; desde o momento em que falamos}
\end{entry}

\begin{entry}{今年}{jin1 nian2}{4,6}{⼈、⼲}[HSK 1]
  \definition{adv.}{este ano}
\end{entry}

\begin{entry}{今日}{jin1 ri4}{4,4}{⼈、⽇}[HSK 5]
  \definition{s.}{hoje}
\end{entry}

\begin{entry}{今天}{jin1tian1}{4,4}{⼈、⼤}[HSK 1]
  \definition{adv.}{hoje; neste dia | agora; o momento ou a época atual}
\end{entry}

\begin{entry}{斤}{jin1}{4}{⽄}[HSK 2][Kangxi 69]
  \definition{clas.}{uma unidade de peso (=500 gramas)}
  \definition{s.}{machado; cutelo; ferramentas antigas para cortar árvores}
\end{entry}

\begin{entry}{金}{jin1}{8}{⾦}[HSK 3][Kangxi 167]
  \definition*{s.}{Dinastia Jin (1115-1234) | Sobrenome Jin}
  \definition{adj.}{dourado | altamente respeitado; precioso. metáfora de nobreza}
  \definition[锭,块]{s.}{ouro | metal | dinheiro | instrumento antigo de percussão de metal}
\end{entry}

\begin{entry}{金额}{jin1 e2}{8,15}{⾦、⾴}[HSK 6]
  \definition[份,笔]{s.}{quantidade de dinheiro; soma de dinheiro}
\end{entry}

\begin{entry}{金刚石}{jin1gang1shi2}{8,6,5}{⾦、⼑、⽯}
  \definition{s.}{diamante, também chamado de 钻石}
  \seealsoref{钻石}{zuan4shi2}
\end{entry}

\begin{entry}{金牌}{jin1 pai2}{8,12}{⾦、⽚}[HSK 3]
  \definition[枚]{s.}{medalha de ouro; refere-se à medalha conquistada pelo campeão em uma competição esportiva | ficha de ouro; placa de ouro usada como símbolo}
\end{entry}

\begin{entry}{金钱}{jin1 qian2}{8,10}{⾦、⾦}[HSK 6]
  \definition[沓,笔,堆]{s.}{dinheiro; moeda}
\end{entry}

\begin{entry}{金融}{jin1rong2}{8,16}{⾦、⿀}[HSK 6]
  \definition{s.}{finanças; serviços bancários; refere-se a atividades econômicas como a emissão, circulação e retirada de moeda, a concessão e retirada de empréstimos, o depósito e retirada de depósitos e transações de câmbio}
\end{entry}

\begin{entry}{金色}{jin1 se4}{8,6}{⾦、⾊}
  \definition{s.}{cor ouro; dourado}
\end{entry}

\begin{entry}{金子}{jin1zi5}{8,3}{⾦、⼦}
  \definition{s.}{ouro; elemento metálico, símbolo Au (aurum) amarelo-avermelhado, macio, dúctil, quimicamente estável é um metal precioso, usado para fabricar dinheiro, ornamentos etc.}
\end{entry}

\begin{entry}{矜}{jin1}{9}{⽭}
  \definition{adj.}{presunçoso; vaidoso | contido; reservado; determinado}
  \definition{v.}{ter pena; simpatizar com; compadecer-se}
\end{entry}

\begin{entry}{仅}{jin3}{4}{⼈}[HSK 3]
  \definition{adv.}{somente; meramente; por muito pouco}
\end{entry}

\begin{entry}{仅此而已}{jin3ci3'er2yi3}{4,6,6,3}{⼈、⽌、⽽、⼰}
  \definition{adv.}{apenas isso e nada mais | isso é tudo}
\end{entry}

\begin{entry}{仅仅}{jin3 jin3}{4,4}{⼈、⼈}[HSK 3]
  \definition{adv.}{somente; meramente; por muito pouco; indica que está limitado a um determinado âmbito}
\end{entry}

\begin{entry}{尽}{jin3}{6}{⼫}
  \definition{adv.}{na maior extensão possível | na extremidade mais distante de | usado antes de palavras que indicam direção, o mesmo que 最 | de agora em diante}
  \definition{prep.}{dentro dos limites de}
  \definition{v.}{dar prioridade a | deixar que certas pessoas ou coisas tenham precedência}
  \seeref{尽}{jin4}
  \seealsoref{最}{zui4}
\end{entry}

\begin{entry}{尽管}{jin3guan3}{6,14}{⼫、⽵}[HSK 5]
  \definition{adv.}{justo; livremente; faça o que quiser, não se preocupe, não há restrições de movimento ou comportamento}
  \definition{conj.}{no entanto; embora; apesar de ; normalmente usado no início de uma frase anterior para introduzir um fato, seguido de 但是, etc. para introduzir um resultado que o fato não deveria ter; às vezes, também pode ser usado no início de uma frase posterior.}
  \seealsoref{但是}{dan4 shi4}
\end{entry}

\begin{entry}{尽可能}{jin3 ke3 neng2}{6,5,10}{⼫、⼝、⾁}[HSK 5]
  \definition{adv.}{na medida do possível; com o melhor de sua capacidade; tentar fazer algo, atingir um determinado nível ou extensão}
\end{entry}

\begin{entry}{尽快}{jin3kuai4}{6,7}{⼫、⼼}[HSK 4]
  \definition{adv.}{com toda a velocidade; o mais rápido possível; o mais breve possível}
\end{entry}

\begin{entry}{尽量}{jin3liang4}{6,12}{⼫、⾥}[HSK 3]
  \definition{adv.}{tanto quanto possível; da melhor maneira possível}
\end{entry}

\begin{entry}{紧}{jin3}{10}{⽷}[HSK 3]
  \definition{adj.}{tenso; apertado; o estado em que um objeto se encontra após ser submetido a uma grande força de tração ou pressão.| seguro; firme | cerrado; apertado | urgente; premente; tenso | rigoroso; rígido; severo | difícil; sem dinheiro}
  \definition{v.}{apertar; tornar mais apertado}
\end{entry}

\begin{entry}{紧急}{jin3ji2}{10,9}{⽷、⼼}[HSK 3]
  \definition{adj./adj.}{urgente; premente; crítico}
\end{entry}

\begin{entry}{紧紧}{jin3 jin3}{10,10}{⽷、⽷}[HSK 5]
  \definition{adv.}{firmemente; estreitamente; apertadamente; prestar muita atenção (em algo)}
\end{entry}

\begin{entry}{紧密}{jin3 mi4}{10,11}{⽷、⼧}[HSK 4]
  \definition{adj.}{próximos; inseparáveis | incessante; rápido e intenso}
\end{entry}

\begin{entry}{紧张}{jin3zhang1}{10,7}{⽷、⼸}[HSK 3]
  \definition{adj.}{nervoso; tenso; mentalmente em estado de alerta, excitado e inquieto | apertado; em falta; o que está disponível não satisfaz os requisitos| tenso; intenso; intenso ou urgente, causando tensão mental}
\end{entry}

\begin{entry}{锦}{jin3}{13}{⾦}
  \definition*{s.}{Sobrenome Jin}
  \definition{adj.}{brilhante e bonito (cores brilhantes e lindas)}
  \definition[块]{s.}{brocado; tecidos de seda com padrões coloridos}
\end{entry}

\begin{entry}{锦上添花}{jin3 shang4 tian1 hua1}{13,3,11,7}{⾦、⼀、⽔、⾋}
  \definition{expr.}{adicionar flores ao brocado --- tornar o que é bom ainda melhor; melhorar | dourando o lírio}
\end{entry}

\begin{entry}{尽}{jin4}{6}{⼫}[HSK 6]
  \definition*{s.}{Sobrenome Jin}
  \definition{adj.}{exausto; acabado | ao máximo; ao limite | tudo; exaustivo}
  \definition{v.}{esgotar | tentar o seu melhor; fazer o melhor uso possível | morrer; falecer | terminar | chegar ao fim ao máximo; alcançar extremos}
  \seeref{尽}{jin3}
\end{entry}

\begin{entry}{尽力}{jin4li4}{6,2}{⼫、⼒}[HSK 4]
  \definition{v.+compl.}{esforçar-se ao máximo; esforçar-se ao máximo; usar toda a sua força; fazer algo com seu melhor esforço}
\end{entry}

\begin{entry}{近}{jin4}{7}{⾡}[HSK 2]
  \definition{adj.}{próximo; perto; distância espacial ou temporal curta (oposto de 远) | íntimo; intimamente relacionado; relação estreita | fácil de entender}
  \seealsoref{远}{yuan3}
\end{entry}

\begin{entry}{近代}{jin4dai4}{7,5}{⾡、⼈}[HSK 4]
  \definition{s.}{tempos modernos; era passada relativamente próxima à era moderna, geralmente referida na história chinesa como 1840 a 1919 | na história mundial, geralmente se refere à era capitalista}
\end{entry}

\begin{entry}{近来}{jin4lai2}{7,7}{⾡、⽊}[HSK 5]
  \definition{adv.}{ultimamente; recentemente; de ​​tarde; refere-se a um período de tempo entre o passado imediato e o presente}
\end{entry}

\begin{entry}{近期}{jin4 qi1}{7,12}{⾡、⽉}[HSK 3]
  \definition{adv.}{num futuro próximo; brevemente}
\end{entry}

\begin{entry}{近日}{jin4 ri4}{7,4}{⾡、⽇}[HSK 6]
  \definition{s.}{recentemente; nos últimos dias; apontando para o passado | nos próximos dias; refere-se ao futuro}
\end{entry}

\begin{entry}{近视}{jin4 shi4}{7,8}{⾡、⾒}[HSK 6]
  \definition{adj.}{miopia; uma deficiência visual em que a visão próxima é clara, mas a visão distante é turva | míope (figurativo); metáfora para miopia}
\end{entry}

\begin{entry}{进}{jin4}{7}{⾡}[HSK 1]
  \definition*{s.}{Sobrenome Jin}
  \definition{clas.}{para seções em um edifício ou complexo residencial; qualquer uma das várias fileiras de casas em um complexo residencial de estilo antigo}
  \definition{s.}{(matemática) base de um sistema numérico}
  \definition{v.}{avançar; ir adiante; seguir em frente; (oposto a 退) | entrar; entrar em; entrar ou sair; (oposto a 出) | receber | comer; tomar; beber | submeter; apresentar | marcar um gol}
  \definition{v.aux.}{usado após um verbo, significa ``para dentro''}
  \seealsoref{出}{chu1}
  \seealsoref{退}{tui4}
\end{entry}

\begin{entry}{进步}{jin4bu4}{7,7}{⾡、⽌}[HSK 3]
  \definition{adj.}{progressivo; adequado às tendências da época; que impulsiona o desenvolvimento social (em oposição a 落后)}
  \definition{v.}{avançar; progredir; melhorar}
  \seealsoref{落后}{luo4hou4}
\end{entry}

\begin{entry}{进出口}{jin4chu1kou3}{7,5,3}{⾡、⼐、⼝}
  \definition{s.}{importação e exportação}
  \definition{v.}{importar e exportar}
\end{entry}

\begin{entry}{进攻}{jin4gong1}{7,7}{⾡、⽁}[HSK 6]
  \definition{s.}{ofensiva}
  \definition{v.}{atacar; assaltar; tomar a ofensiva (oposto à 防守)}
  \seealsoref{防守}{fang2shou3}
\end{entry}

\begin{entry}{进化}{jin4hua4}{7,4}{⾡、⼔}[HSK 5]
  \definition[个]{s.}{evolução; os organismos se desenvolvem e evoluem do simples para o complexo e de níveis baixos para altos}
  \definition{v.}{evoluir; um termo geral usado para descrever uma mudança gradual para melhor}
\end{entry}

\begin{entry}{进口}{jin4kou3}{7,3}{⾡、⼝}[HSK 4]
  \definition{adj.}{importado}
  \definition{s.}{importação; entrada de um edifício ou local, também chamada de 入口}
  \definition{v.+compl.}{importar; comprar ou transportar mercadorias de outro país ou região | entrar no porto; navegar em direção a um porto}
  \seealsoref{入口}{ru4kou3}
\end{entry}

\begin{entry}{进来}{jin4 lai2}{7,7}{⾡、⽊}[HSK 1]
  \definition{v.}{entrar (para a minha localização)}
\end{entry}

\begin{entry}{进去}{jin4 qu4}{7,5}{⾡、⼛}[HSK 1]
  \definition{v.}{entrar (a partir da minha localização)}
  \definition{v.aux.}{usado depois de um verbo, significa ``ir para dentro''; para um determinado intervalo ou período de tempo}
\end{entry}

\begin{entry}{进入}{jin4 ru4}{7,2}{⾡、⼊}[HSK 2]
  \definition{v.}{entrar; entrar em}
\end{entry}

\begin{entry}{进行}{jin4xing2}{7,6}{⾡、⾏}[HSK 2]
  \definition{v.}{continuar; estar em andamento; estar em progresso | fazer; conduzir; realizar; executar | marchar; avançar; prosseguir; estar em marcha}
\end{entry}

\begin{entry}{进行编程}{jin4xing2bian1cheng2}{7,6,12,12}{⾡、⾏、⽷、⽲}
  \definition{s.}{programa de computador executável}
\end{entry}

\begin{entry}{进一步}{jin4 yi2 bu4}{7,1,7}{⾡、⼀、⽌}[HSK 3]
  \definition{adv.}{mais; dar um passo adiante; avançar um passo; indica que as coisas estão progredindo em um nível mais alto do que antes}
\end{entry}

\begin{entry}{进展}{jin4zhan3}{7,10}{⾡、⼫}[HSK 3]
  \definition{v.}{fazer progresso; progredir; avançar no desenvolvimento}
\end{entry}

\begin{entry}{禁}{jin4}{13}{⽰}
  \definition*{s.}{Sobrenome Jin}
  \definition{s.}{um tabu; assuntos não permitidos por lei ou costume | área proibida | residência real; o lugar onde o imperador viveu nos tempos antigos}
  \definition{v.}{proibir; banir | aprisionar; deter}
\end{entry}

\begin{entry}{禁止}{jin4zhi3}{13,4}{⽰、⽌}[HSK 4]
  \definition{v.}{banir; proibir; interditar}
\end{entry}

\begin{entry}{京}{jing1}{8}{⼇}
  \definition*{s.}{Pequim (Beijing), abreviação de 北京 | Sobrenome Jing}
  \definition{num.}{dez milhões (um numeral antigo); 10.000.000; 1000.0000}
  \definition{s.}{capital de um país}
  \seealsoref{北京}{bei3 jing1}
\end{entry}

\begin{entry}{京二胡}{jing1'er4hu2}{8,2,9}{⼇、⼆、⾁}
  \definition{s.}{um tipo de violino chinês semelhante ao 二胡 de duas cordas, usado principalmente para acompanhamento do canto da ópera de Pequim | também chamado de 京胡 | jing'erhu, um violino de duas cordas, intermediário em tamanho e tom entre o 京胡 e o 二胡, usado para acompanhar a ópera chinesa}
  \seealsoref{二胡}{er4hu2}
  \seealsoref{京胡}{jing1hu2}
\end{entry}

\begin{entry}{京胡}{jing1hu2}{8,9}{⼇、⾁}
  \definition{s.}{jinghu, um instrumento de arco de duas cordas com registro agudo; violino da ópera de Pequim | também chamado de 京二胡 | jinghu, um 二胡 (violino de duas cordas) menor e mais agudo, usado para acompanhar a ópera chinesa}
  \seealsoref{二胡}{er4hu2}
  \seealsoref{胡琴}{hu2qin2}
  \seealsoref{京二胡}{jing1'er4hu2}
\end{entry}

\begin{entry}{京剧}{jing1ju4}{8,10}{⼇、⼑}[HSK 3]
  \definition*[场,段]{s.}{Ópera de Pequim}
\end{entry}

\begin{entry}{经}{jing1}{8}{⽷}
  \definition*{s.}{Sobrenome Jing}
  \definition{s.}{livro sagrado | escritura | clássicos | longitude | menstruação | canal}
  \definition{v.}{passar | sofrer | suportar | deformar (têxtil)}
\end{entry}

\begin{entry}{经常}{jing1chang2}{8,11}{⽷、⼱}[HSK 2]
  \definition{adj.}{habitual; cotidiano; diário; do dia a dia}
  \definition{adv.}{frequentemente; regularmente; constantemente; com frequência; indica que a ação ocorre repetidamente}
\end{entry}

\begin{entry}{经典}{jing1dian3}{8,8}{⽷、⼋}[HSK 4]
  \definition{adj.}{clássico; (escritos ou obras, etc.) que são típicos, autorizados}
  \definition{s.}{clássicos; escritos tradicionais e valiosos; os livros mais importantes e fundamentais da religião | escrituras; escritos de doutrinas religiosas}
\end{entry}

\begin{entry}{经费}{jing1fei4}{8,9}{⽷、⾙}[HSK 5]
  \definition[笔]{s.}{fundos; desembolso; gastos | despesas; gastos}
\end{entry}

\begin{entry}{经过}{jing1guo4}{8,6}{⽷、⾡}[HSK 2]
  \definition{prep.}{depois; através; como resultado de; passar por uma atividade ou evento que traz novas mudanças para pessoas ou coisas}
  \definition[个,段,番]{s.}{processo; curso; experiência}
  \definition{v.}{passar; atravessar; passar por; através de (local, tempo, ação, etc.)}
\end{entry}

\begin{entry}{经济}{jing1ji4}{8,9}{⽷、⽔}[HSK 3]
  \definition{adj.}{econômico;  parcimonioso; descreve algo que custa pouco e rende muito; preço acessível}
  \definition{s.}{economia; a soma das relações de produção social em um determinado período histórico|econômico; de valor industrial ou econômico; refere-se à economia nacional; também se refere a um determinado setor da economia nacional | economia; refere-se às atividades econômicas, incluindo produção, circulação, distribuição e consumo, bem como atividades ou processos financeiros, de seguros, etc. | renda; situação financeira; refere-se à situação financeira de uma pessoa}
  \definition{v.}{governar o país e beneficiar o povo}
\end{entry}

\begin{entry}{经理}{jing1li3}{8,11}{⽷、⽟}[HSK 2]
  \definition[个,位,名]{s.}{gerente; diretor; pessoas responsáveis pela gestão e administração de empresas ou corporações}
\end{entry}

\begin{entry}{经历}{jing1li4}{8,4}{⽷、⼚}[HSK 3]
  \definition[个,次,段,种]{s.}{experiência; coisas que você viu, fez ou sofreu pessoalmente}
  \definition{v.}{passar por; atravessar; ter visto, feito ou sofrido pessoalmente}
\end{entry}

\begin{entry}{经验}{jing1yan4}{8,10}{⽷、⾺}[HSK 3]
  \definition[个,次,种]{s.}{experiência; conhecimento ou habilidades adquiridos através da prática}
  \definition{v.}{experimentar; passar por; ter visto, feito ou sofrido pessoalmente}
\end{entry}

\begin{entry}{经营}{jing1ying2}{8,11}{⽷、⾋}[HSK 3]
  \definition{v.}{executar; gerenciar; operar; envolver-se em; planejar e gerenciar (empresas, etc.) | gerenciar; refere-se a planos e organizações em geral}
\end{entry}

\begin{entry}{惊}{jing1}{11}{⼼}
  \definition{v.}{assustar; ficar assustado; ficar nervoso devido a estímulo repentino; ficar com medo | surpreender; chocar; alarmar}
\end{entry}

\begin{entry}{惊呆}{jing1dai1}{11,7}{⼼、⼝}
  \definition{adj.}{estupefato | chocado}
\end{entry}

\begin{entry}{惊人}{jing1 ren2}{11,2}{⼼、⼈}[HSK 6]
  \definition{adj.}{surpreso; espantado; atônito; surpreendente}
\end{entry}

\begin{entry}{惊喜}{jing1 xi3}{11,12}{⼼、⼝}[HSK 6]
  \definition{s.}{boa surpresa; agradavelmente surpreso}
\end{entry}

\begin{entry}{精}{jing1}{14}{⽶}[HSK 6]
  \definition{adv.}{muito; extremamente; antes de certos adjetivos, significa 十分 ou 非常}
  \definition{s.}{refinado; escolhido; escolha; purificado ou selecionado | perfeito; excelente; melhor | fino (em oposição a 粗); preciso; meticuloso | inteligente; astuto; esperto | habilidoso; versado; proficiente | extrato; essência; essência refinada ou selecionada; extraída | energia; espírito | semente; esperma; sêmen | \emph{goblin}; espírito; elfo; demônio}
  \seealsoref{粗}{cu1}
  \seealsoref{非常}{fei1chang2}
  \seealsoref{十分}{shi2fen1}
\end{entry}

\begin{entry}{精彩}{jing1cai3}{14,11}{⽶、⼺}[HSK 3]
  \definition{adj.}{brilhante; esplêndido; maravilhoso}
\end{entry}

\begin{entry}{精力}{jing1li4}{14,2}{⽶、⼒}[HSK 4]
  \definition[些]{s.}{energia; vigor; força mental e física}
\end{entry}

\begin{entry}{精灵}{jing1ling2}{14,7}{⽶、⽕}
  \definition{s.}{espírito | fada | elfo | duende | gênio}
\end{entry}

\begin{entry}{精美}{jing1 mei3}{14,9}{⽶、⽺}[HSK 6]
  \definition{adj.}{elegante; requintado}
\end{entry}

\begin{entry}{精品}{jing1pin3}{14,9}{⽶、⼝}[HSK 6]
  \definition[个]{s.}{belas obras (de arte); objetos de arte | produtos de qualidade; artigos de excelente qualidade; produto \emph{premium}}
\end{entry}

\begin{entry}{精神}{jing1shen2}{14,9}{⽶、⽰}[HSK 3]
  \definition[种,个,类,股]{s.}{espírito; mente; estado mental; refere-se à consciência, às atividades mentais e ao estado psicológico geral de uma pessoa | substância; espírito; essência; propósito; significado principal}
  \seeref{精神}{jing1shen5}
\end{entry}

\begin{entry}{精神}{jing1shen5}{14,9}{⽶、⽰}[HSK 3]
  \definition{adj.}{animado; espirituoso; vigoroso; descreve uma pessoa como cheia de energia | muito bonito; boa aparência, bom físico}
  \definition[种,个,类,股]{s.}{impulso; vigor; vitalidade}
  \seeref{精神}{jing1shen2}
\end{entry}

\begin{entry}{精致}{jing1zhi4}{14,10}{⽶、⾄}
  \definition{adj.}{delicado | exótico | refinado}
\end{entry}

\begin{entry}{鲸}{jing1}{16}{⿂}
  \definition[头,只,条]{s.}{baleia; cetáceo}
\end{entry}

\begin{entry}{鲸鲨}{jing1sha1}{16,15}{⿂、⿂}
  \definition{s.}{tubarão baleia}
\end{entry}

\begin{entry}{鲸鱼}{jing1yu2}{16,8}{⿂、⿂}
  \definition{s.}{baleia}
\end{entry}

\begin{entry}{井}{jing3}{4}{⼆}[HSK 6]
  \definition*{s.}{Jing, uma das mansões lunares | Sobrenome Jing}
  \definition{adj.}{limpo; organizado}
  \definition[口]{s.}{poço; um buraco profundo cavado no chão para tirar água | algo em forma de poço | vila natal ou cidade natal}
\end{entry}

\begin{entry}{景}{jing3}{12}{⽇}[HSK 6]
  \definition*{s.}{Sobrenome Jing}
  \definition{adj.}{grandioso; elevado; grande}
  \definition{s.}{vista; cenário; cena | situação; condição | cenário (de uma peça ou filme) | cena (de uma peça)}
  \definition{v.}{admirar; reverenciar; respeitar}
\end{entry}

\begin{entry}{景点}{jing3 dian3}{12,9}{⽇、⽕}[HSK 6]
  \definition[个,处]{s.}{local cênico; atração turística; um lugar onde se concentram as atrações turísticas, incluindo atrações naturais e culturais}
\end{entry}

\begin{entry}{景色}{jing3se4}{12,6}{⽇、⾊}[HSK 3]
  \definition[片,幅,道,处]{s.}{vista; cena; cenário; paisagem}
\end{entry}

\begin{entry}{景象}{jing3 xiang4}{12,11}{⽇、⾗}[HSK 5]
  \definition[个]{s.}{cena; visão; vista; quadro}
\end{entry}

\begin{entry}{警}{jing3}{19}{⾔}
  \definition{s.}{policial}
  \definition{v.}{alertar | avisar}
\end{entry}

\begin{entry}{警察}{jing3cha2}{19,14}{⾔、⼧}[HSK 3]
  \definition[个,位,名,群,队]{s.}{polícia; policial; oficial de polícia; as forças armadas que mantêm a segurança social do país são uma parte importante do aparato estatal; também se refere aos membros dessas forças armadas}
\end{entry}

\begin{entry}{警告}{jing3gao4}{19,7}{⾔、⼝}[HSK 5]
  \definition[个]{s.}{advertência (como medida disciplinar); uma forma de punição}
  \definition{v.}{avisar; advertir; admoestar}
\end{entry}

\begin{entry}{警官}{jing3guan1}{19,8}{⾔、⼧}
  \definition{s.}{polícia | policial}
\end{entry}

\begin{entry}{净}{jing4}{8}{⼎}[HSK 6]
  \definition{adj.}{limpo | (depois de um verbo) terminado; sem nada sobrando | líquido | vazio; oco; nu}
  \definition{adv.}{todo; o tempo todo | somente; meramente; nada além de | inteiramente; indica puro e nada mais}
  \definition{s.}{o “rosto pintado”, comumente conhecido como Hualian, 花脸, um tipo de personagem da ópera de Pequim, etc.}
  \definition{v.}{tornar limpo | limpar; lavar; esfregar para limpar}
  \seealsoref{花脸}{hua1lian3}
\end{entry}

\begin{entry}{竞}{jing4}{10}{⽴}
  \definition{adj.}{forte; poderoso}
  \definition{v.}{competir; contender; disputar | contestar}
\end{entry}

\begin{entry}{竞赛}{jing4sai4}{10,14}{⽴、⾙}[HSK 5]
  \definition[个]{s.}{concurso; competição; partida; corrida}
  \definition{v.}{correr; competir; competir uns com os outros por superioridade; em esportes, produção e outras atividades, para comparar competência, habilidade etc., usado principalmente na linguagem falada}
\end{entry}

\begin{entry}{竞争}{jing4zheng1}{10,6}{⽴、⼑}[HSK 5]
  \definition{v.}{competir; disputar; lutar; entre duas ou mais partes; em prol de seus próprios interesses; lutar pela vitória por meio de uma disputa de sua própria força contra outra}
\end{entry}

\begin{entry}{竟}{jing4}{11}{⾳}
  \definition{adj.}{todo; por toda parte; do começo ao fim}
  \definition{adv.}{no final; eventualmente | na verdade; inesperadamente; significa algo inesperado, equivalente a 居然}
  \definition{v.}{terminar; completar | investigar}
  \seealsoref{居然}{ju1ran2}
\end{entry}

\begin{entry}{竟然}{jing4ran2}{11,12}{⾳、⽕}[HSK 4]
  \definition{adv.}{de fato; inesperadamente; para surpresa de alguém; chegar ao ponto de; indica que algo é um pouco inesperado}
\end{entry}

\begin{entry}{敬}{jing4}{12}{⽁}
  \definition*{s.}{Sobrenome Jing}
  \definition{adj.}{respeitoso; reverente}
  \definition{adv.}{respeitosamente}
  \definition{v.}{respeitar; honrar; estimar | oferecer educadamente | envolver-se em; dedicar-se a}
\end{entry}

\begin{entry}{敬礼}{jing4li3}{12,5}{⽁、⽰}
  \definition{s.}{saudação}
  \definition{v.}{saudar}
\end{entry}

\begin{entry}{静}{jing4}{14}{⾭}[HSK 3]
  \definition*{s.}{Sobrenome Jing}
  \definition{adj.}{tranquilo;  sossegado; calmo; imóvel | silencioso; quieto; sem emitir nenhum som | calmo, sereno; serenidade; (interior) paz}
  \definition{v.}{acalmar-se; aquietar-se; tranquilizar (o coração)}
\end{entry}

\begin{entry}{镜}{jing4}{16}{⾦}
  \definition*{s.}{Sobrenome Jing}
  \definition[面,副]{s.}{espelho | lente; vidro; dispositivos para auxiliar a visão ou conduzir experimentos ópticos}
  \definition{v.}{espelhar | perceber | usar como referência}
\end{entry}

\begin{entry}{镜头}{jing4tou2}{16,5}{⾦、⼤}[HSK 4]
  \definition[个]{s.}{lente de câmera; objetiva; combinação de várias lentes, usada para formar uma imagem | foto; cena}
\end{entry}

\begin{entry}{镜子}{jing4zi5}{16,3}{⾦、⼦}[HSK 4]
  \definition[面,个]{s.}{espelho; instrumento de reflexão de imagem liso e plano, antigamente esmerilhado a partir de um disco grosso de cobre fundido, atualmente feito de vidro plano revestido de prata ou alumínio | óculos; óculos de grau}
\end{entry}

\begin{entry}{纠}{jiu1}{5}{⽷}
  \definition*{s.}{Sobrenome Jiu}
  \definition{v.}{emaranhar | reunir-se | corrigir; retificar | supervisionar; superintender}
\end{entry}

\begin{entry}{纠纷}{jiu1fen1}{5,7}{⽷、⽷}[HSK 6]
  \definition[个,次]{s.}{questão; disputa; existem contradições ou conflitos de interesse entre as duas partes que precisam ser resolvidos}
\end{entry}

\begin{entry}{纠葛}{jiu1ge2}{5,12}{⽷、⾋}
  \definition{s.}{emaranhado | disputa}
\end{entry}

\begin{entry}{纠正}{jiu1zheng4}{5,5}{⽷、⽌}[HSK 6]
  \definition{v.}{fazer certo; corrigir (deficiências ou erros em pensamentos, ações, métodos, etc.)}
\end{entry}

\begin{entry}{究}{jiu1}{7}{⽳}
  \definition{adv.}{na verdade; realmente; afinal}
  \definition{v.}{estudar cuidadosamente; aprofundar; investigar; rastrear}
\end{entry}

\begin{entry}{究竟}{jiu1jing4}{7,11}{⽳、⾳}[HSK 4]
  \definition{adv.}{de fato; exatamente; usado em frases interrogativas para buscar | afinal de contas, no final; ênfase em fatos ou motivos}
  \definition{s.}{resultado; desfecho; a causa, o efeito ou a história completa do que aconteceu}
\end{entry}

\begin{entry}{九}{jiu3}{2}{⼄}[HSK 1]
  \definition*{s.}{Sobrenome Jiu}
  \definition{adj.}{muitos; numerosos; indica várias vezes ou a maioria das vezes}
  \definition{num.}{nove; 9}
  \definition{s.}{cada um dos nove períodos de nove dias começando no dia seguinte ao solstício de inverno}
\end{entry}

\begin{entry}{久}{jiu3}{3}{⼃}[HSK 3]
  \definition{adj.}{por muito tempo; longo período de tempo | duração de tempo especificada}
\end{entry}

\begin{entry}{韭}{jiu3}{9}{⾲}[Kangxi 179]
  \definition{s.}{alho de flor perfumada; cebolinha chinesa}
\end{entry}

\begin{entry}{韭菜}{jiu3cai4}{9,11}{⾲、⾋}
  \definition{s.}{cebolinha chinesa | (figurativo) investidores de varejo que perdem seu dinheiro para operadores mais experientes (ou seja, são ``colhidos'' como cebolinhas)}
\end{entry}

\begin{entry}{酒}{jiu3}{10}{⾣}[HSK 2]
  \definition*{s.}{Sobrenome Jiu}
  \definition[口,杯,瓶,罐,桶,缸]{s.}{bebida alcoólica; vinho; licor; bebidas destiladas}
\end{entry}

\begin{entry}{酒吧}{jiu3ba1}{10,7}{⾣、⼝}[HSK 4]
  \definition[家,个]{s.}{bar; \emph{pub}; um local onde são vendidas bebidas alcoólicas e onde as pessoas podem beber e conversar, referindo-se principalmente a um restaurante ou hotel de estilo ocidental especializado na venda de bebidas alcoólicas.}
\end{entry}

\begin{entry}{酒店}{jiu3 dian4}{10,8}{⾣、⼴}[HSK 2]
  \definition[家,个]{s.}{hotel; Estabelecimento comercial que oferece hospedagem e alimentação aos hóspedes | restaurante}
\end{entry}

\begin{entry}{酒馆}{jiu3guan3}{10,11}{⾣、⾷}
  \definition{s.}{bar | taverna | adega}
\end{entry}

\begin{entry}{酒鬼}{jiu3gui3}{10,9}{⾣、⿁}[HSK 5]
  \definition{s.}{bebedor de vinho; beberrão; ébrio | alcoólatra}
\end{entry}

\begin{entry}{酒水}{jiu3 shui3}{10,4}{⾣、⽔}[HSK 6]
  \definition{s.}{bebidas; bebidas e álcool | Dialeto: festa; banquete}
\end{entry}

\begin{entry}{旧}{jiu4}{5}{⽇}[HSK 3]
  \definition{adj.}{passado; antigo; velho; ultrapassado (em oposição a 新)| usado; desgastado; velho; descolorido ou deformado devido ao uso prolongado ou ao tempo | antigo; único; que já existiu; anterior}
  \definition{s.}{velha amizade; velho amigo}
  \seealsoref{新}{xin1}
\end{entry}

\begin{entry}{救}{jiu4}{11}{⽁}[HSK 3]
  \definition*{s.}{Sobrenome Jiu}
  \definition{v.}{resgatar; salvar | salvar de; aliviar (angústia, etc.) | resgatar; livrar alguém de um desastre ou perigo | ajudar; aliviar; socorrer; livrar pessoas e coisas de desastres e perigos}
\end{entry}

\begin{entry}{救出}{jiu4chu1}{11,5}{⽁、⼐}
  \definition{v.}{resgatar | tirar do perigo}
\end{entry}

\begin{entry}{救护车}{jiu4hu4che1}{11,7,4}{⽁、⼿、⾞}
  \definition[辆]{s.}{ambulância}
\end{entry}

\begin{entry}{救命}{jiu4 ming4}{11,8}{⽁、⼝}[HSK 6]
  \definition{interj.}{Socorro!; Salve-me!}
  \definition{v.+compl.}{ajudar; salvar a vida de alguém}
\end{entry}

\begin{entry}{救援}{jiu4 yuan2}{11,12}{⽁、⼿}[HSK 6]
  \definition{v.}{resgatar; socorrer; vir em auxílio de alguém (resgate)}
\end{entry}

\begin{entry}{救灾}{jiu4 zai1}{11,7}{⽁、⽕}[HSK 5]
  \definition{v.}{ajudar as vítimas de desastres, aliviar o desastre; resgatar pessoas afetadas por desastres; recuperar danos causados por desastres}
\end{entry}

\begin{entry}{就}{jiu4}{12}{⼪}[HSK 1]
  \definition{adv.}{de imediato; imediatamente; indica que algo ocorrerá em breve | tão cedo quanto; já; há muito tempo; indica que a ação ocorreu há muito tempo | assim que; logo depois; indica que os eventos se sucedem imediatamente | nesse caso; então; indica que, sob determinadas condições, ocorre naturalmente um determinado resultado | exatamente; precisamente; indica que é exatamente assim | apenas; meramente; somente | tantos quanto; enfatiza a quantidade | apenas; simplesmente; reforço da afirmação | colocado entre dois componentes idênticos, significa tolerância ou indiferença}
  \definition{prep.}{tirar proveito de alguém (algo); expressa condições, oportunidades, etc., equivalente a 趁 | quando se trata de alguém (algo); relativo a; com relação a; sobre; objeto ou escopo da introdução da ação |no local; introduz o local próximo ao qual a ação ocorreu}
  \definition{v.}{ser comido com; ir com; pratos, frutas, etc., acompanhados de alimentos básicos ou bebidas alcoólicas | aproximar-se; mover-se em direção a | ir para; assumir; empreender; envolver-se em; entrar em | realizar; fazer | tirar proveito de; acomodar-se a; adequar-se; encaixar-se | assumir; começar a entrar ou a exercer | seguir; acompanhar}
  \seealsoref{趁}{chen4}
\end{entry}

\begin{entry}{就是}{jiu4 shi4}{12,9}{⼪、⽇}[HSK 3]
  \definition{adv.}{exatamente; precisamente; expressar concordância com a afirmação da outra pessoa ou confirmar que a afirmação da outra pessoa está correta | apenas; simplesmente; expressa afirmação, determinação ou ênfase, o significado específico deve ser determinado com base no contexto anterior ou posterior | usado para indicar escolha}
  \definition{conj.}{ainda que; mesmo que se reconheça que essa situação é verdadeira, a situação posterior não mudará}
  \definition{part.}{usado no final de uma frase para expressar afirmação}
\end{entry}

\begin{entry}{就是说}{jiu4 shi4 shuo1}{12,9,9}{⼪、⽇、⾔}[HSK 6]
  \definition{interj.}{ou seja; isto é; em outras palavras; é frequentemente usado como uma interjeição em uma frase para indicar que as palavras seguintes são uma explicação ou esclarecimento das anteriores}
\end{entry}

\begin{entry}{就算}{jiu4 suan4}{12,14}{⼪、⽵}[HSK 6]
  \definition{conj.}{mesmo que; concedido que; expressam uma relação hipotética e concessiva, frequentemente usadas com 也, equivalente a 即使}
  \seealsoref{即使}{ji2shi3}
  \seealsoref{也}{ye3}
\end{entry}

\begin{entry}{就要}{jiu4 yao4}{12,9}{⼪、⾑}[HSK 2]
  \definition{adv.}{estar prestes a; estar indo para; estar no ponto de}
\end{entry}

\begin{entry}{就业}{jiu4ye4}{12,5}{⼪、⼀}[HSK 3]
  \definition{v.+compl.}{conseguir um emprego; obter emprego; assumir uma ocupação; começar a trabalhar}
\end{entry}

\begin{entry}{就职}{jiu4zhi2}{12,11}{⼪、⽿}
  \definition{v.}{assumir o cargo | assumir um posto}
\end{entry}

\begin{entry}{车}{ju1}{4}{⾞}[Kangxi 159]
  \definition{s.}{torre; castelo; carruagem, uma das peças do xadrez chinês}
  \seeref{车}{che1}
\end{entry}

\begin{entry}{居}{ju1}{8}{⼫}
  \definition*{s.}{Sobrenome Ju}
  \definition{s.}{residência; casa | restaurante (em nomes de restaurantes)}
  \definition{v.}{residir; morar; viver | ocupar uma determinada posição; ocupar (um lugar); estar (em uma determinada posição) | reivindicar; afirmar | armazenar; guardar | ficar parado; estar parado}
\end{entry}

\begin{entry}{居民}{ju1min2}{8,5}{⼫、⽒}[HSK 4]
  \definition[个,户,位]{s.}{residente; habitante; pessoas que estão fixas em um único lugar}
\end{entry}

\begin{entry}{居然}{ju1ran2}{8,12}{⼫、⽕}[HSK 5]
  \definition{adv.}{inesperadamente; para surpresa de alguém; além da expectativa (expressão idiomática) |}
  \definition{v.}{ir tão longe a ponto de; ter a impudência de; ter o descaramento de;}
\end{entry}

\begin{entry}{居住}{ju1zhu4}{8,7}{⼫、⼈}[HSK 4]
  \definition{v.}{viver; residir; morar; habitar}
\end{entry}

\begin{entry}{据}{ju1}{11}{⼿}
  \definition{part.}{elemento formador de palavras}
  \seealsoref{拮据}{jie2ju1}
\end{entry}

\begin{entry}{局}{ju2}{7}{⼫}[HSK 4,6]
  \definition{adj.}{limitado; confinado}
  \definition{clas.}{\emph{set}; jogo; turno}
  \definition{s.}{tabuleiro de xadrez | situação; estado de coisas | generosidade de espírito; extensão da tolerância de alguém | festa; reunião; refere-se a certas reuniões | ardil; armadilha | parte; porção; papel | escritório; agência; agências governamentais divididas por negócios | significa ``loja'' em nomes de lojas | departamento; agência; nomes de certas entidades empresariais | escritório; usado como nome de uma instituição ou outro local de negócios}
\end{entry}

\begin{entry}{局面}{ju2mian4}{7,9}{⼫、⾯}[HSK 5]
  \definition[种]{s.}{aspecto; fase; situação; o estado das coisas em um período de tempo, em sua maior parte abstraído | escopo; escala}
\end{entry}

\begin{entry}{局长}{ju2 zhang3}{7,4}{⼫、⾧}[HSK 5]
  \definition[位,个]{s.}{comissário; diretor; principais chefes de gabinete do governo}
\end{entry}

\begin{entry}{橘}{ju2}{16}{⽊}
  \definition[只,棵]{s.}{tangerina}
\end{entry}

\begin{entry}{橘子汁}{ju2zi5zhi1}{16,3,5}{⽊、⼦、⽔}
  \definition[瓶,杯,罐,盒]{s.}{suco de laranja}
  \seealsoref{橙汁}{cheng2zhi1}
  \seealsoref{柳橙汁}{liu3cheng2zhi1}
\end{entry}

\begin{entry}{柜}{ju3}{8}{⽊}
  \definition{s.}{faia; salgueiro}
  \seeref{柜}{gui4}
\end{entry}

\begin{entry}{举}{ju3}{9}{⼂}[HSK 2]
  \definition*{s.}{Sobrenome Ju}
  \definition{adj.}{inteiro; completo}
  \definition{s.}{ato; ação; movimento; comportamento | (nas dinastias Ming e Qing) candidato aprovado nos exames imperiais a nível provincial}
  \definition{v.}{levantar; erguer; sustentar | começar; iniciar; surgir | eleger; escolher; recomendar; selecionar | citar; enumerar; propor; revelar}
\end{entry}

\begin{entry}{举办}{ju3ban4}{9,4}{⼂、⼒}[HSK 3]
  \definition{v.}{conduzir; organizar; realizar}
\end{entry}

\begin{entry}{举动}{ju3dong4}{9,6}{⼂、⼒}[HSK 5]
  \definition{s.}{ato; atividade; movimento; ação}
\end{entry}

\begin{entry}{举手}{ju3 shou3}{9,4}{⼂、⼿}[HSK 2]
  \definition{v.}{levantar a mão ou as mãos; levantar a mão para sinalizar ou responder a uma pergunta}
\end{entry}

\begin{entry}{举行}{ju3xing2}{9,6}{⼂、⾏}[HSK 2]
  \definition{v.}{realizar (uma reunião, cerimônia, etc.); realizar (atividades formais ou solenes)}
\end{entry}

\begin{entry}{巨}{ju4}{4}{⼯}
  \definition*{s.}{Sobrenome Ju}
  \definition{adj.}{enorme; tremendo; gigantesco}
\end{entry}

\begin{entry}{巨大}{ju4da4}{4,3}{⼯、⼤}[HSK 4]
  \definition{adj.}{enorme; tremendo; enorme; gigantesco; imenso}
\end{entry}

\begin{entry}{句}{ju4}{5}{⼝}[HSK 2]
  \definition{clas.}{para sentenças, frases ou linhas de versos}
  \definition{s.}{frase; sentença}
  \seeref{句}{gou4}
\end{entry}

\begin{entry}{句子}{ju4zi5}{5,3}{⼝、⼦}[HSK 2]
  \definition[个,句]{s.}{sentença; uma unidade linguística composta por palavras ou frases que expressa um significado completo}
\end{entry}

\begin{entry}{拒}{ju4}{7}{⼿}
  \definition{v.}{resistir; repelir | recusar; rejeitar}
\end{entry}

\begin{entry}{拒绝}{ju4jue2}{7,9}{⼿、⽷}[HSK 5]
  \definition{v.}{recusar; rejeitar; declinar; não aceitar (pedidos, sugestões ou presentes)}
\end{entry}

\begin{entry}{足}{ju4}{7}{⾜}[Kangxi 157]
  \definition{adj.}{excessivo}
  \seeref{足}{zu2}
\end{entry}

\begin{entry}{具}{ju4}{8}{⼋}
  \definition*{s.}{Sobrenome Ju}
  \definition{clas.}{(literário) usado para caixões, cadáveres e certos objetos}
  \definition{s.}{utensílio; ferramenta; implemento | capacidade; habilidade}
  \definition{v.}{possuir; ter | fornecer; prover | declarar; enumerar}
\end{entry}

\begin{entry}{具备}{ju4bei4}{8,8}{⼋、⼡}[HSK 4]
  \definition{v.}{ter; possuir; ser provido de}
\end{entry}

\begin{entry}{具体}{ju4ti3}{8,7}{⼋、⼈}[HSK 3]
  \definition{adj.}{específico; particular | concreto; específico; mais detalhado; muito detalhado; muito claro | concreto; real; não é abstrato, tem uma forma definida; pode ser visto ou sentido}
  \definition{v.}{incorporar; objetivar; combinar teorias, princípios, padrões, etc. com pessoas ou coisas específicas}
\end{entry}

\begin{entry}{具有}{ju4 you3}{8,6}{⼋、⽉}[HSK 3]
  \definition{v.}{ter; possuir; ser provido de}
\end{entry}

\begin{entry}{俱}{ju4}{10}{⼈}
  \definition{adv.}{(literário) tudo; completamente; inteiramente}
\end{entry}

\begin{entry}{俱乐部}{ju4le4bu4}{10,5,10}{⼈、⼃、⾢}[HSK 5]
  \definition[个]{s.}{clube; grupos e locais para atividades sociais, políticas, literárias, recreativas e outras}
\end{entry}

\begin{entry}{剧}{ju4}{10}{⼑}[HSK 6]
  \definition*{s.}{Sobrenome Ju}
  \definition{adj.}{agudo; grave; intenso; violento}
  \definition[部,个,种]{s.}{obra teatral; drama; peça; ópera}
\end{entry}

\begin{entry}{剧本}{ju4ben3}{10,5}{⼑、⽊}[HSK 5]
  \definition{s.}{cenário; roteiro (para drama, filme, etc.); gênero de obra literária que consiste em diálogos entre personagens (às vezes cantados) e indicações de palco}
\end{entry}

\begin{entry}{剧场}{ju4 chang3}{10,6}{⼑、⼟}[HSK 3]
  \definition[个,坐]{s.}{teatro; local para apresentações teatrais, musicais, etc.}
\end{entry}

\begin{entry}{据}{ju4}{11}{⼿}[HSK 6]
  \definition*{s.}{Sobrenome Ju}
  \definition{prep.}{de acordo com; com base em}
  \definition{s.}{evidência; certificado; prova}
  \definition{v.}{ocupar; apreender | confiar em; depender de}
  \seeref{据}{ju1}
\end{entry}

\begin{entry}{据说}{ju4shuo1}{11,9}{⼿、⾔}[HSK 3]
  \definition{v.}{é o que dizem; é o que se diz}
\end{entry}

\begin{entry}{距}{ju4}{11}{⾜}
  \definition{s.}{distância | espora (de um galo, etc.)}
  \definition{v.}{estar separado (longe) de; estar distante de}
\end{entry}

\begin{entry}{距离}{ju4li2}{11,10}{⾜、⼇}[HSK 4]
  \definition[个]{s.}{distância}
  \definition{v.}{estar distante de}
\end{entry}

\begin{entry}{聚}{ju4}{14}{⽿}[HSK 4]
  \definition{v.}{reunir-se; juntar-se}
\end{entry}

\begin{entry}{聚会}{ju4hui4}{14,6}{⽿、⼈}[HSK 4]
  \definition[个,次]{s.}{reunião; encontro; confraternização; festa}
  \definition{v.}{encontrar-se; reunir-se}
\end{entry}

\begin{entry}{聚散}{ju4san4}{14,12}{⽿、⽁}
  \definition{s.}{juntos e separados | agregação e dissipação}
\end{entry}

\begin{entry}{捐}{juan1}{10}{⼿}[HSK 6]
  \definition{s.}{imposto}
  \definition{v.}{renunciar; abandonar | 2. contribuir; doar; assinar}
\end{entry}

\begin{entry}{捐款}{juan1 kuan3}{10,12}{⼿、⽋}[HSK 6]
  \definition[笔]{s.}{doação; contribuição (de dinheiro); valor doado}
  \definition{v.+compl.}{doar; contribuir com dinheiro}
\end{entry}

\begin{entry}{捐赠}{juan1 zeng4}{10,16}{⼿、⾙}[HSK 6]
  \definition{v.}{apresentar; contribuir (como um presente); doar (itens para um país ou grupo)}
\end{entry}

\begin{entry}{捐助}{juan1 zhu4}{10,7}{⼿、⼒}[HSK 6]
  \definition{v.}{oferecer (assistência financeira ou material); contribuir; doar}
\end{entry}

\begin{entry}{圈}{juan1}{11}{⼞}
  \definition{v.}{prender aves e animais de criação | prender; colocar na cadeia, prisão | confinar}
  \seeref{圈}{juan4}
  \seeref{圈}{quan1}
\end{entry}

\begin{entry}{卷}{juan3}{8}{⼙}[HSK 4]
  \definition{clas.}{para pequenas coisas enroladas (maço de papel dinheiro, carretel de filme, etc.) | para rolos, carretéis, bobinas, etc.}
  \definition[张]{s.}{rolo; carretel; bobina}
  \definition{v.}{enrolar; dobrar algo em um cilindro ou semicírculo | varrer; carregar; levar junto | envolver-se; participar}
  \seeref{卷}{juan4}
\end{entry}

\begin{entry}{卷}{juan4}{8}{⼙}[HSK 4]
  \definition{clas.}{para capítulos, seções ou volumes; fascículos}
  \definition{s.}{livro; livros e pinturas que são enrolados para coleção; geralmente se refere a pinturas e caligrafia | papel de exame | arquivo; dossiê}
  \seeref{卷}{juan3}
\end{entry}

\begin{entry}{圈}{juan4}{11}{⼞}
  \definition*{s.}{Sobrenome Juan}
  \definition{s.}{curral; local onde o gado ou as aves são mantidos, geralmente cercado ou murado, alguns com galpões}
  \seeref{圈}{juan1}
  \seeref{圈}{quan1}
\end{entry}

\begin{entry}{决}{jue2}{6}{⼎}
  \definition{v.}{decidir; determinar | executar uma pessoa | (de um dique, etc.) romper; desabar}
\end{entry}

\begin{entry}{决不}{jue2 bu4}{6,4}{⼎、⼀}[HSK 5]
  \definition{adv.}{definitivamente não; certamente não; sob nenhuma circunstância; de forma alguma}
\end{entry}

\begin{entry}{决策}{jue2ce4}{6,12}{⼎、⽵}[HSK 6]
  \definition{s.}{decisão política; decisão de importância estratégica; estratégia ou método de decisão}
  \definition{v.}{formular políticas; tomar uma decisão estratégica; decidir sobre uma estratégia ou abordagem}
\end{entry}

\begin{entry}{决定}{jue2ding4}{6,8}{⼎、⼧}[HSK 3]
  \definition{adj.}{decisivo; as leis objetivas levam as coisas a se desenvolverem e mudarem em determinada direção}
  \definition[项,个]{s.}{decisão; resolução; assuntos decididos}
  \definition{v.}{decidir; determinar; algo se torna a base ou o pré-requisito para outra coisa; desempenha um papel dominante | decidir; resolver; tomar uma decisão; propor uma forma de agir}
\end{entry}

\begin{entry}{决赛}{jue2sai4}{6,14}{⼎、⾙}[HSK 3]
  \definition[场]{s.}{finais (de uma competição); em competições esportivas, a última partida ou rodada disputada para determinar a classificação}
\end{entry}

\begin{entry}{决心}{jue2xin1}{6,4}{⼎、⼼}[HSK 3]
  \definition{s.}{resolução; determinação; determinação inabalável}
  \definition{v.}{secidir-se; decidir fazer algo e não vacilar nem mudar de ideia}
\end{entry}

\begin{entry}{角}{jue2}{7}{⾓}[Kangxi 148]
  \definition*{s.}{Sobrenome Jue}
  \definition{s.}{papel (teatro)}
  \definition{v.}{competir}
  \seeref{角}{jiao3}
\end{entry}

\begin{entry}{角色}{jue2se4}{7,6}{⾓、⾊}[HSK 4]
  \definition{s.}{papel; personagem em uma peça; personagem representado por um ator | papel; função; parte}
\end{entry}

\begin{entry}{绝}{jue2}{9}{⽷}[HSK 6]
  \definition{adj.}{exausto; esgotado; acabado | desesperado; sem esperança | único; soberbo; incomparável | não deixar margem de manobra; não fazer concessões; intransigente}
  \definition{adv.}{extremamente; mais | (antes de uma negativa) absolutamente; no mínimo; por qualquer meio; em qualquer conta}
  \definition{s.}{(literário) jueju, um poema de quatro linhas}
  \definition{v.}{cortar; romper | parar de respirar; morrer}
\end{entry}

\begin{entry}{绝版}{jue2ban3}{9,8}{⽷、⽚}
  \definition{adj.}{esgotado | fora de catálogo}
\end{entry}

\begin{entry}{绝不}{jue2bu4}{9,4}{⽷、⼀}
  \definition{adv.}{definitivamente não | de forma alguma | sob nenhuma circunstância}
\end{entry}

\begin{entry}{绝大多数}{jue2 da4 duo1 shu4}{9,3,6,13}{⽷、⼤、⼣、⽁}[HSK 6]
  \definition{expr.}{maioria absoluta | uma maioria esmagadora}
\end{entry}

\begin{entry}{绝对}{jue2dui4}{9,5}{⽷、⼨}[HSK 3]
  \definition{adj.}{absoluto; sem condições; sem restrições | absoluto; extremo; incompleto; sem margem para negociação ou alteração}
  \definition{adv.}{absolutamente; completamente; com certeza}
\end{entry}

\begin{entry}{绝望}{jue2 wang4}{9,11}{⽷、⽉}[HSK 5]
  \definition{v.+compl.}{desesperar; desistir de toda esperança; perder toda esperança de}
\end{entry}

\begin{entry}{绝招}{jue2zhao1}{9,8}{⽷、⼿}
  \definition{s.}{habilidade única | movimento delicado inesperado (como último recurso) | golpe de mestre | golpe final}
\end{entry}

\begin{entry}{觉}{jue2}{9}{⾒}
  \definition{s.}{sentimento; senso; percepção e discriminação de estímulos externos}
  \definition{v.}{sentir; perceber | acordar | tornar-se consciente; tornar-se desperto; despertar; entender}
  \seeref{觉}{jiao4}
\end{entry}

\begin{entry}{觉得}{jue2de5}{9,11}{⾒、⼻}[HSK 1]
  \definition{v.}{sentir; estar ciente; pressentir; causar uma sensação | pensar; sentir; encontrar; considerar (tom menos assertivo)}
\end{entry}

\begin{entry}{觉悟}{jue2wu4}{9,10}{⾒、⼼}[HSK 6]
  \definition{s.}{consciência; percepção; compreensão; nível de consciência}
  \definition{v.}{vir a compreender; tornar-se consciente de; tornar-se politicamente desperto; despertar}
\end{entry}

\begin{entry}{脚}{jue2}{11}{⾁}
  \variantof{角}
\end{entry}

\begin{entry}{军}{jun1}{6}{⼍}
  \definition*{s.}{Sobrenome Jun}
  \definition{s.}{forças armadas; exército; tropas | exército; contingente; muitas pessoas participando de uma atividade | exército; unidades militares}
\end{entry}

\begin{entry}{军队}{jun1dui4}{6,4}{⼍、⾩}[HSK 6]
  \definition[支,个]{s.}{forças armadas; exército; tropas}
\end{entry}

\begin{entry}{军舰}{jun1 jian4}{6,10}{⼍、⾈}[HSK 6]
  \definition[艘,只]{s.}{navio de guerra; embarcação naval | \emph{warcraft}; um termo geral para embarcações militares equipadas com armas e equipamentos que podem executar missões de combate, incluindo principalmente navios de guerra, cruzadores, contratorpedeiros, porta-aviões, submarinos, torpedeiros, etc.}
\end{entry}

\begin{entry}{军人}{jun1 ren2}{6,2}{⼍、⼈}[HSK 5]
  \definition{s.}{soldado; militar; pessoal militar; pessoas com status militar; pessoas servindo nas forças armadas}
\end{entry}

\begin{entry}{军事}{jun1shi4}{6,8}{⼍、⼅}[HSK 6]
  \definition{s.}{militar; assuntos militares; assuntos relativos aos militares e à guerra}
\end{entry}

\begin{entry}{军装}{jun1zhuang1}{6,12}{⼍、⾐}
  \definition{s.}{uniforme militar}
\end{entry}

\begin{entry}{君}{jun1}{7}{⼝}
  \definition*{s.}{Sobrenome Jun}
  \definition[个,位,名,些]{s.}{monarca; soberano; governante supremo | (como título) Senhor; Sr. | (literário) (em trato direto) você; senhor | cavalheiro | governante}
\end{entry}

\begin{entry}{君主立宪制}{jun1zhu3li4xian4zhi4}{7,5,5,9,8}{⼝、⼂、⽴、⼧、⼑}
  \definition{s.}{monarquia constitucional}
\end{entry}

%%%%% EOF %%%%%

