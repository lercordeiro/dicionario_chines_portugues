%%%
%%% 16画
%%%

\section*{16画}\addcontentsline{toc}{section}{16画}

\begin{Entry}{儒}{16}{⼈}
  \begin{Phonetics}{儒}{ru2}
    \definition*{s.}{Confucionismo; Confucionista | Sobrenome Ru}
    \definition{s.}{(antigo) erudito; homem culto}
  \end{Phonetics}
\end{Entry}

\begin{Entry}{儒教}{16,11}{⼈、⽁}
  \begin{Phonetics}{儒教}{ru2jiao4}
    \definition*{s.}{Confucionismo}
  \end{Phonetics}
\end{Entry}

\begin{Entry}{嘴}{16}{⼝}
  \begin{Phonetics}{嘴}{zui3}[][HSK 2]
    \definition[张]{s.}{boca; boca humana ou animal | qualquer coisa com formato ou função semelhante a uma boca | fala | comida}
  \end{Phonetics}
\end{Entry}

\begin{Entry}{嘴巴}{16,4}{⼝、⼰}
  \begin{Phonetics}{嘴巴}{zui3 ba5}[][HSK 4]
    \definition[张]{s.}{boca}
  \end{Phonetics}
\end{Entry}

\begin{Entry}{嘴巴子}{16,4,3}{⼝、⼰、⼦}
  \begin{Phonetics}{嘴巴子}{zui3ba5zi5}
    \definition{s.}{tapa | bofetada}
  \end{Phonetics}
\end{Entry}

\begin{Entry}{器}{16}{⼝}
  \begin{Phonetics}{器}{qi4}
    \definition[台]{s.}{dispositivo | ferramenta | utensílio}
  \end{Phonetics}
\end{Entry}

\begin{Entry}{器官}{16,8}{⼝、⼧}
  \begin{Phonetics}{器官}{qi4guan1}[][HSK 4]
    \definition[个,种]{s.}{órgão; aparelho; parte de um organismo que consiste em vários tipos de tecidos celulares que podem desempenhar uma função fisiológica separada}
  \end{Phonetics}
\end{Entry}

\begin{Entry}{壁}{16}{⼟}
  \begin{Phonetics}{壁}{bi4}
    \definition*{s.}{Bi, a décima quarta das vinte e oito constelações em que a esfera celeste foi dividida, consistindo em duas estrelas em linha reta, uma em Pégaso e a outra em Andrômeda | A Estrela Bìxìu, uma das Vinte e Oito Mansões da astronomia tradicional chinesa}
    \definition[道]{s.}{parede | superfície plana como uma parede | penhasco | muralha; parapeito | barreira}
  \end{Phonetics}
\end{Entry}

\begin{Entry}{壁纸}{16,7}{⼟、⽷}
  \begin{Phonetics}{壁纸}{bi4zhi3}
    \definition{s.}{papel de parede; papel colado em paredes internas para decoração ou proteção, com diversos tipos e cores}
  \end{Phonetics}
\end{Entry}

\begin{Entry}{壁画}{16,8}{⼟、⽥}
  \begin{Phonetics}{壁画}{bi4hua4}[][HSK 7-9]
    \definition[幅]{s.}{mural; afresco; desenhos nas paredes ou tetos de edifícios}
  \end{Phonetics}
\end{Entry}

\begin{Entry}{壁虎}{16,8}{⼟、⾌}
  \begin{Phonetics}{壁虎}{bi4hu3}
    \definition{s.}{lagartixa}
  \end{Phonetics}
\end{Entry}

\begin{Entry}{懒}{16}{⼼}
  \begin{Phonetics}{懒}{lan3}[][HSK 6]
    \definition{adj.}{indolente; preguiçoso (oposto de 勤) | lento; lânguido | ocioso; preguiçoso}
  \seealsoref{勤}{qin2}
  \end{Phonetics}
\end{Entry}

\begin{Entry}{懒人}{16,2}{⼼、⼈}
  \begin{Phonetics}{懒人}{lan3ren2}
    \definition{s.}{pessoa preguiçosa}
  \end{Phonetics}
\end{Entry}

\begin{Entry}{懒汉}{16,5}{⼼、⽔}
  \begin{Phonetics}{懒汉}{lan3han4}
    \definition{s.}{sujeito ocioso | vagabundo | preguiçosos}
  \end{Phonetics}
\end{Entry}

\begin{Entry}{懒虫}{16,6}{⼼、⾍}
  \begin{Phonetics}{懒虫}{lan3chong2}
    \definition{s.}{desleixado ocioso | (insulto) sujeito preguiçoso}
  \end{Phonetics}
\end{Entry}

\begin{Entry}{懒怠}{16,9}{⼼、⼼}
  \begin{Phonetics}{懒怠}{lan3dai4}
    \definition{s.}{preguiça}
  \end{Phonetics}
\end{Entry}

\begin{Entry}{懒鬼}{16,9}{⼼、⿁}
  \begin{Phonetics}{懒鬼}{lan3gui3}
    \definition{s.}{cara preguiçoso}
  \end{Phonetics}
\end{Entry}

\begin{Entry}{懒得}{16,11}{⼼、⼻}
  \begin{Phonetics}{懒得}{lan3de5}
    \definition{adv.}{demasiado preguiçoso}
    \definition{v.}{não sentir vontade (de fazer algo)}
  \end{Phonetics}
\end{Entry}

\begin{Entry}{懒惰}{16,12}{⼼、⼼}
  \begin{Phonetics}{懒惰}{lan3duo4}
    \definition{adj.}{preguiçoso}
  \end{Phonetics}
\end{Entry}

\begin{Entry}{懒散}{16,12}{⼼、⽁}
  \begin{Phonetics}{懒散}{lan3san3}
    \definition{adj.}{inativo | indolente | preguiçoso | negligente}
  \end{Phonetics}
\end{Entry}

\begin{Entry}{懒腰}{16,13}{⼼、⾁}
  \begin{Phonetics}{懒腰}{lan3yao1}
    \definition[个]{s.}{alongamento (do corpo)}
  \end{Phonetics}
\end{Entry}

\begin{Entry}{撼}{16}{⼿}
  \begin{Phonetics}{撼}{han4}
    \definition{v.}{agitar; sacudir}
  \end{Phonetics}
\end{Entry}

\begin{Entry}{擅}{16}{⼿}
  \begin{Phonetics}{擅}{shan4}
    \definition{adv.}{sem autorização; arbitrariamente | fazer algo por conta própria}
    \definition{v.}{ser bom em; ser especialista em | arrogar-se a si mesmo; fazer algo por conta própria | reivindicar arbitrariamente; ir além do escopo e ajir arbitrariamente}
  \end{Phonetics}
\end{Entry}

\begin{Entry}{擅自}{16,6}{⼿、⾃}
  \begin{Phonetics}{擅自}{shan4zi4}
    \definition{adv.}{sem permissão ou autorização | por iniciativa própria}
  \end{Phonetics}
\end{Entry}

\begin{Entry}{操}{16}{⼿}
  \begin{Phonetics}{操}{cao1}
    \definition*{s.}{Sobrenome Cao}
    \definition[节,套]{s.}{exercício; ginástica | conduta; comportamento; moralidade, a moral e o código de conduta que as pessoas seguem}
    \definition{v.}{segurar; agarrar; segurar na mão | fazer algo; envolver-se em | falar (uma língua ou dialeto) | treinar (tropas); exercitar (corpo); praticar ou treinar de acordo com uma determinada forma ou postura | dirigir; manusear}
  \end{Phonetics}
\end{Entry}

\begin{Entry}{操心}{16,4}{⼿、⼼}
  \begin{Phonetics}{操心}{cao1/xin1}[][HSK 7-9]
    \definition{v.+compl.}{se esforçar; preocupar-se com; incomodar-se com}
  \end{Phonetics}
\end{Entry}

\begin{Entry}{操场}{16,6}{⼿、⼟}
  \begin{Phonetics}{操场}{cao1chang3}[][HSK 4]
    \definition[个,片,座,处]{s.}{\emph{playground}; campo esportivo; locais para exercícios físicos ou exercícios militares}
  \end{Phonetics}
\end{Entry}

\begin{Entry}{操作}{16,7}{⼿、⼈}
  \begin{Phonetics}{操作}{cao1zuo4}[][HSK 4]
    \definition{v.}{operar; seguir os requisitos e procedimentos prescritos| implementar; realizar; executar; refere-se à implementação concreta (planos, medidas, etc.)}
  \end{Phonetics}
\end{Entry}

\begin{Entry}{操劳}{16,7}{⼿、⼒}
  \begin{Phonetics}{操劳}{cao1lao2}[][HSK 7-9]
    \definition{v.}{trabalhar duro | cuidar; cuidar de}
  \end{Phonetics}
\end{Entry}

\begin{Entry}{操纵}{16,7}{⼿、⽷}
  \begin{Phonetics}{操纵}{cao1zong4}[][HSK 6]
    \definition{v.}{operar; controlar (uma máquina, instrumento, etc.) | manipular; controlar secretamente; assumir o controle de (uma pessoa, organização, situação, etc.)}
  \end{Phonetics}
\end{Entry}

\begin{Entry}{操控}{16,11}{⼿、⼿}
  \begin{Phonetics}{操控}{cao1kong4}[][HSK 7-9]
    \definition{v.}{controlar; manipular}
  \end{Phonetics}
\end{Entry}

\begin{Entry}{整}{16}{⽁}
  \begin{Phonetics}{整}{zheng3}[][HSK 3]
    \definition*{s.}{Sobrenome Zheng}
    \definition{adj.}{cheio; integral; inteiro; completo; sem defeitos | limpo; arrumado; organizado; em boa ordem | redondo (não é uma fração)}
    \definition{s.}{número inteiro (não fracionário)}
    \definition{v.}{retificar; corrigir; pôr em ordem | consertar; renovar; reparar | corrigir; punir; causar sofrimento;  fazer alguém sofrer | fazer; realizar; trabalhar; em algumas regiões, significa 做, 搞}
  \seealsoref{搞}{gao3}
  \seealsoref{做}{zuo4}
  \end{Phonetics}
\end{Entry}

\begin{Entry}{整个}{16,3}{⽁、⼈}
  \begin{Phonetics}{整个}{zheng3ge4}[][HSK 3]
    \definition{adj.}{total; inteiro; completo}
  \end{Phonetics}
\end{Entry}

\begin{Entry}{整天}{16,4}{⽁、⼤}
  \begin{Phonetics}{整天}{zheng3 tian1}[][HSK 3]
    \definition{s.}{o dia inteiro; o dia todo; durante todo o dia; de manhã à noite}
  \end{Phonetics}
\end{Entry}

\begin{Entry}{整齐}{16,6}{⽁、⿑}
  \begin{Phonetics}{整齐}{zheng3qi2}[][HSK 3]
    \definition{adj.}{arrumado; organizado; em boa ordem | uniforme; regular; tamanho, comprimento, grau, etc. são relativamente consistentes | usado para descrever que todas as coisas necessárias estão prontas}
    \definition{v.}{estar em boas condições; manter a ordem e a organização}
  \end{Phonetics}
\end{Entry}

\begin{Entry}{整体}{16,7}{⽁、⼈}
  \begin{Phonetics}{整体}{zheng3ti3}[][HSK 3]
    \definition[个]{s.}{um todo; totalidade}
  \end{Phonetics}
\end{Entry}

\begin{Entry}{整治}{16,8}{⽁、⽔}
  \begin{Phonetics}{整治}{zheng3 zhi4}[][HSK 6]
    \definition{v.}{renovar; consertar; arrumar; dragar (um rio, etc.) | punir; fazer alguém sofrer}
  \end{Phonetics}
\end{Entry}

\begin{Entry}{整顿}{16,10}{⽁、⾴}
  \begin{Phonetics}{整顿}{zheng3dun4}[][HSK 6]
    \definition{v.}{retificar; consolidar; reorganizar; tornar fenômenos, disciplinas e estilos desordenados e irracionais, ordenados e razoáveis}
  \end{Phonetics}
\end{Entry}

\begin{Entry}{整理}{16,11}{⽁、⽟}
  \begin{Phonetics}{整理}{zheng3li3}[][HSK 3]
    \definition{v.}{organizar; reorganizar; classificar; ordenar; colocar em ordem}
  \end{Phonetics}
\end{Entry}

\begin{Entry}{整整}{16,16}{⽁、⽁}
  \begin{Phonetics}{整整}{zheng3 zheng3}[][HSK 3]
    \definition{adv.}{inteiramente; completamente; solidamente; continuamente}
  \end{Phonetics}
\end{Entry}

\begin{Entry}{橘}{16}{⽊}
  \begin{Phonetics}{橘}{ju2}
    \definition[只,棵]{s.}{tangerina}
  \end{Phonetics}
\end{Entry}

\begin{Entry}{橘子汁}{16,3,5}{⽊、⼦、⽔}
  \begin{Phonetics}{橘子汁}{ju2zi5zhi1}
    \definition[瓶,杯,罐,盒]{s.}{suco de laranja}
  \seealsoref{橙汁}{cheng2zhi1}
  \seealsoref{柳橙汁}{liu3cheng2zhi1}
  \end{Phonetics}
\end{Entry}

\begin{Entry}{橙}{16}{⽊}
  \begin{Phonetics}{橙}{cheng2}
    \definition{s.}{laranja; fruta da laranjeira | laranjeira; pé de laranja | cor laranja}
  \end{Phonetics}
\end{Entry}

\begin{Entry}{橙汁}{16,5}{⽊、⽔}
  \begin{Phonetics}{橙汁}{cheng2zhi1}[][HSK 7-9]
    \definition[瓶,杯,罐,盒]{s.}{laranjada; suco de laranja}
  \seealsoref{橘子汁}{ju2zi5zhi1}
  \seealsoref{柳橙汁}{liu3cheng2zhi1}
  \end{Phonetics}
\end{Entry}

\begin{Entry}{橙色}{16,6}{⽊、⾊}
  \begin{Phonetics}{橙色}{cheng2 se4}
    \definition{s.}{cor de laranja}
  \end{Phonetics}
\end{Entry}

\begin{Entry}{激}{16}{⽔}
  \begin{Phonetics}{激}{ji1}
    \definition*{s.}{Sobrenome Ji}
    \definition{adj.}{afiado; feroz; violento | vívido}
    \definition{adv.}{bruscamente; ferozmente; violentamente}
    \definition{s.}{o impacto de ondas fortes contra a costa}
    \definition{v.}{bater; avançar; correr | despertar; estimular; incitar; excitar | ficar doente por se molhar | esfriar (colocando água gelada, etc.)}
  \end{Phonetics}
\end{Entry}

\begin{Entry}{激化}{16,4}{⽔、⼔}
  \begin{Phonetics}{激化}{ji1hua4}[][HSK 7-9]
    \definition{v.}{aguçar; intensificar; tornar agudo}
  \end{Phonetics}
\end{Entry}

\begin{Entry}{激发}{16,5}{⽔、⼜}
  \begin{Phonetics}{激发}{ji1fa1}[][HSK 7-9]
    \definition{v.}{despertar; desencadear; estimular; motivar; inspirar | excitar; mudar moléculas e átomos de um estado de energia mais baixo para um estado de energia mais alto}
  \end{Phonetics}
\end{Entry}

\begin{Entry}{激光}{16,6}{⽔、⼉}
  \begin{Phonetics}{激光}{ji1guang1}[][HSK 7-9]
    \definition*{s.}{LASER; \emph{Light Amplification by Stimulated Emission of Radiation}; amplificação de luz por emissão estimulada de radiação}
  \end{Phonetics}
\end{Entry}

\begin{Entry}{激动}{16,6}{⽔、⼒}
  \begin{Phonetics}{激动}{ji1dong4}[][HSK 4]
    \definition{adj.}{animado; entusiasmado; empolgado}
    \definition{v.}{agitar; excitar; tornar fortes os sentimentos de alguém}
  \end{Phonetics}
\end{Entry}

\begin{Entry}{激励}{16,7}{⽔、⼒}
  \begin{Phonetics}{激励}{ji1li4}[][HSK 7-9]
    \definition{v.}{instar; impelir; inspirar; encorajar; usar palavras ou ações de outras pessoas para encorajar as pessoas a trabalhar mais e fazer melhor | dirigir; excitar; estimular ou excitar uma reação ou atividade}
  \end{Phonetics}
\end{Entry}

\begin{Entry}{激活}{16,9}{⽔、⽔}
  \begin{Phonetics}{激活}{ji1huo2}[][HSK 7-9]
    \definition{v.}{ativar; estimular certas substâncias no corpo para torná-las ativas | colocar em jogo; revigorar; estimular metaforicamente, influenciar algo, torná-lo ativo}
  \end{Phonetics}
\end{Entry}

\begin{Entry}{激烈}{16,10}{⽔、⽕}
  \begin{Phonetics}{激烈}{ji1lie4}[][HSK 4]
    \definition{adj.}{agudo; afiado; feroz; violento; intenso}
  \end{Phonetics}
\end{Entry}

\begin{Entry}{激素}{16,10}{⽔、⽷}
  \begin{Phonetics}{激素}{ji1su4}[][HSK 7-9]
    \definition{s.}{Fisiológico: hormônio; uma substância química que regula a atividade celular}
  \end{Phonetics}
\end{Entry}

\begin{Entry}{激起}{16,10}{⽔、⾛}
  \begin{Phonetics}{激起}{ji1qi3}[][HSK 7-9]
    \definition{v.}{excitar; agitar; despertar; evocar; desencadear}
  \end{Phonetics}
\end{Entry}

\begin{Entry}{激情}{16,11}{⽔、⼼}
  \begin{Phonetics}{激情}{ji1qing2}[][HSK 6]
    \definition{s.}{paixão; emoções fortes e explosivas, como êxtase, raiva, etc.}
  \end{Phonetics}
\end{Entry}

\begin{Entry}{燃}{16}{⽕}
  \begin{Phonetics}{燃}{ran2}
    \definition{v.}{queimar | acender; inflamar}
  \end{Phonetics}
\end{Entry}

\begin{Entry}{燃料}{16,10}{⽕、⽃}
  \begin{Phonetics}{燃料}{ran2 liao4}[][HSK 4]
    \definition[种]{s.}{combustível; carburante; substâncias que podem gerar calor e energia luminosa quando queimadas podem ser divididas em três tipos de acordo com sua forma: combustível sólido (como carvão, carvão vegetal, madeira), combustível líquido (como gasolina, querosene) e combustível gasoso (como gás de carvão, biogás); também se refere a substâncias que podem gerar energia nuclear, como urânio, plutônio, etc.}
  \end{Phonetics}
\end{Entry}

\begin{Entry}{燃烧}{16,10}{⽕、⽕}
  \begin{Phonetics}{燃烧}{ran2shao1}[][HSK 4]
    \definition{v.}{queimar; acender | arder; inflamar; ferver; metáfora para as emoções de uma pessoa serem muito fortes, como um fogo ardente}
  \end{Phonetics}
\end{Entry}

\begin{Entry}{犟}{16}{⽜}
  \begin{Phonetics}{犟}{jiang4}
    \variantof{强}
  \end{Phonetics}
\end{Entry}

\begin{Entry}{磨}{16}{⽯}
  \begin{Phonetics}{磨}{mo2}[][HSK 6]
    \definition{v.}{esfregar; desgastar | moer; refletir; polir | desgastar; esgotar; cansar; exaurir | incomodar; causar problemas | destruir; obliterar; extinguir-se | ficar ocioso; perder tempo; perder tempo; procrastinar}
  \end{Phonetics}
  \begin{Phonetics}{磨}{mo4}
    \definition[盘]{s.}{mó (pedra pesada e redonda para moinho)}
    \definition{v.}{moer; esfarelar; triturar | virar; inverter a marcha}
  \end{Phonetics}
\end{Entry}

\begin{Entry}{磨菇}{16,11}{⽯、⾋}
  \begin{Phonetics}{磨菇}{mo2gu5}
    \variantof{蘑菇}
  \end{Phonetics}
\end{Entry}

\begin{Entry}{篮}{16}{⽵}
  \begin{Phonetics}{篮}{lan2}
    \definition[个]{s.}{cesto | o anel de ferro e a rede na cesta de basquete}
  \end{Phonetics}
\end{Entry}

\begin{Entry}{篮球}{16,11}{⽵、⽟}
  \begin{Phonetics}{篮球}{lan2qiu2}[][HSK 2]
    \definition[个,只]{s.}{basquetebol | bola de basquete; refere-se à bola utilizada no basquetebol}
  \end{Phonetics}
\end{Entry}

\begin{Entry}{糕}{16}{⽶}
  \begin{Phonetics}{糕}{gao1}
    \definition{s.}{bolo; alimentos feitos de farinha de arroz, farinha de trigo, etc.}
  \end{Phonetics}
\end{Entry}

\begin{Entry}{糕点}{16,9}{⽶、⽕}
  \begin{Phonetics}{糕点}{gao1dian3}
    \definition{s.}{bolos | pastéis}
  \end{Phonetics}
\end{Entry}

\begin{Entry}{糕点师}{16,9,6}{⽶、⽕、⼱}
  \begin{Phonetics}{糕点师}{gao1dian3 shi1}
    \definition{s.}{confeiteiro}
  \end{Phonetics}
\end{Entry}

\begin{Entry}{糕点店}{16,9,8}{⽶、⽕、⼴}
  \begin{Phonetics}{糕点店}{gao1dian3 dian4}
    \definition{s.}{confeitaria}
  \end{Phonetics}
\end{Entry}

\begin{Entry}{糖}{16}{⽶}
  \begin{Phonetics}{糖}{tang2}[][HSK 3]
    \definition[包,斤,勺,袋,块]{s.}{açúcar; um tipo de açúcar; um tipo de composto orgânico, que pode ser dividido em três tipos: monossacarídeos, dissacarídeos e polissacarídeos; é a principal substância que produz energia térmica no corpo humano, como glicose, sacarose, lactose, amido, etc. | açúcar; açúcar comestível; termo geral para açúcar | doces; balas | carboidrato; algo doce e calórico}
  \end{Phonetics}
\end{Entry}

\begin{Entry}{糖醋鱼}{16,15,8}{⽶、⾣、⿂}
  \begin{Phonetics}{糖醋鱼}{tang2cu4yu2}
    \definition{s.}{peixe guisado em molho agridoce (prato)}
  \end{Phonetics}
\end{Entry}

\begin{Entry}{膨}{16}{⾁}
  \begin{Phonetics}{膨}{peng2}
    \definition{v.}{inchar; inflar | expandir; aumentar o comprimento ou o volume de um objeto}
  \end{Phonetics}
\end{Entry}

\begin{Entry}{膨胀}{16,8}{⾁、⾁}
  \begin{Phonetics}{膨胀}{peng2zhang4}
    \definition{v.}{expandir | inflar | inchar}
  \end{Phonetics}
\end{Entry}

\begin{Entry}{蕹}{16}{⾋}
  \begin{Phonetics}{蕹}{weng4}
    \definition{s.}{espinafre-d’água ou \emph{ong choy}, usado como vegetal no sul da China e no sudeste da Ásia}
  \end{Phonetics}
\end{Entry}

\begin{Entry}{蕹菜}{16,11}{⾋、⾋}
  \begin{Phonetics}{蕹菜}{weng4cai4}
    \definition{s.}{espinafre aquático | \emph{ong choy} | repolho do pântano | convolvulus aquático | glória-da-manhã aquática}
  \seealsoref{空心菜}{kong1xin1cai4}
  \end{Phonetics}
\end{Entry}

\begin{Entry}{薄}{16}{⾋}
  \begin{Phonetics}{薄}{bao2}[][HSK 4]
    \definition{adj.}{fino; frágil | frio; indiferente; carente de calor | leve; fraco | pobre; infértil}
  \end{Phonetics}
  \begin{Phonetics}{薄}{bo2}
    \definition*{s.}{Sobrenome Bo}
    \definition{adj.}{pequeno; leve; magro | mau; cruel; mesquinho | frívolo; fútil; não solene | fraco; frágil}
    \definition{v.}{desprezar; tratar com desprezo; menosprezar | aproximar-se}
  \end{Phonetics}
  \begin{Phonetics}{薄}{bo4}
    \definition{s.}{menta; uma erva perene com aroma refrescante nos caules e folhas}
  \end{Phonetics}
\end{Entry}

\begin{Entry}{薄弱}{16,10}{⾋、⼸}
  \begin{Phonetics}{薄弱}{bo2ruo4}[][HSK 5]
    \definition{adj.}{fraco; frágil; não é firme; não é sólido}
  \end{Phonetics}
\end{Entry}

\begin{Entry}{薛}{16}{⾋}
  \begin{Phonetics}{薛}{xue1}
    \definition*{s.}{Estado vassalo durante a Dinastia Zhou (1046-256 a.C.) | Sobrenome Xue}
    \definition{s.}{erva semelhante ao absinto (clássico)}
  \end{Phonetics}
\end{Entry}

\begin{Entry}{薛稷}{16,15}{⾋、⽲}
  \begin{Phonetics}{薛稷}{xue1 ji4}
    \definition*{s.}{Xue Ji (649-713), um dos quatro grandes calígrafos do início da dinastia Tang, 唐初四大家}
  \seealsoref{唐初四大家}{tang2 chu1 si4 da4jia1}
  \end{Phonetics}
\end{Entry}

\begin{Entry}{薪}{16}{⾋}
  \begin{Phonetics}{薪}{xin1}
    \definition{s.}{lenha; combustível | salário; ordenado; pagamento}
  \end{Phonetics}
\end{Entry}

\begin{Entry}{薪水}{16,4}{⾋、⽔}
  \begin{Phonetics}{薪水}{xin1shui3}[][HSK 6]
    \definition[份,笔]{s.}{pagamento; salário; ordenados; dinheiro ou bens pagos regularmente aos trabalhadores como compensação pelo seu trabalho}
  \end{Phonetics}
\end{Entry}

\begin{Entry}{薯}{16}{⾋}
  \begin{Phonetics}{薯}{shu3}
    \definition{s.}{batata | inhame}
  \end{Phonetics}
\end{Entry}

\begin{Entry}{薯片}{16,4}{⾋、⽚}
  \begin{Phonetics}{薯片}{shu3 pian4}[][HSK 6]
    \definition{s.}{batatas fritas (\emph{chips}); batatas fritas crocantes ; flocos finos feitos de batatas}
  \end{Phonetics}
\end{Entry}

\begin{Entry}{薯条}{16,7}{⾋、⽊}
  \begin{Phonetics}{薯条}{shu3 tiao2}[][HSK 6]
    \definition{s.}{batatas fritas (palito)}
  \end{Phonetics}
\end{Entry}

\begin{Entry}{融}{16}{⿀}
  \begin{Phonetics}{融}{rong2}
    \definition*{s.}{Sobrenome Rong}
    \definition{adj.}{permanente; longo prazo; duradouro | muito brilhante | circulante; corrente}
    \definition{s.}{fogo | plena luz do dia}
    \definition{v.}{derreter; descongelar | misturar; fundir; estar em harmonia | circular (dinheiro, etc.)}
  \end{Phonetics}
\end{Entry}

\begin{Entry}{融入}{16,2}{⿀、⼊}
  \begin{Phonetics}{融入}{rong2 ru4}[][HSK 6]
    \definition{v.}{integrar em; juntar-se, integrar-se ao grupo | misturar-se; enfatizar a mistura e a combinação com o ambiente circundante para se tornar harmonioso e consistente | encher com (um certo sentimento); imbuir com (uma certa qualidade); preparar (chá, ervas, etc.); imergir; infundir (drogas, etc.)}
  \end{Phonetics}
\end{Entry}

\begin{Entry}{融合}{16,6}{⿀、⼝}
  \begin{Phonetics}{融合}{rong2 he2}[][HSK 6]
    \definition{v.}{fundir; mesclar; misturar; combinar várias coisas diferentes em uma}
  \end{Phonetics}
\end{Entry}

\begin{Entry}{衡}{16}{⾏}
  \begin{Phonetics}{衡}{heng2}
    \definition*{s.}{Sobrenome Heng}
    \definition[个]{s.}{braço graduado de uma balança | balança; aparelho de pesagem}
    \definition{v.}{pesar; medir; julgar}
  \end{Phonetics}
\end{Entry}

\begin{Entry}{衡量}{16,12}{⾏、⾥}
  \begin{Phonetics}{衡量}{heng2 liang2}[][HSK 6]
    \definition{v.}{pesar; medir; comparar; avaliar | considerar; pensar sobre; deliberar}
  \end{Phonetics}
\end{Entry}

\begin{Entry}{赞}{16}{⾙}
  \begin{Phonetics}{赞}{zan4}
    \definition{v.}{patrocinar | apoiar | elogiar | (gíria na \emph{Internet}) para curtir (uma postagem \emph{on-line})}
  \end{Phonetics}
\end{Entry}

\begin{Entry}{赞成}{16,6}{⾙、⼽}
  \begin{Phonetics}{赞成}{zan4cheng2}[][HSK 4]
    \definition{v.}{endossar; favorecer; aprovar; concordar com; concordar ou apoiar as ideias, os planos, as propostas ou o comportamento de outra pessoa}
  \end{Phonetics}
\end{Entry}

\begin{Entry}{赞扬}{16,6}{⾙、⼿}
  \begin{Phonetics}{赞扬}{zan4yang2}
    \definition{v.}{elogiar | aprovar | demonstrar aprovação}
  \end{Phonetics}
\end{Entry}

\begin{Entry}{赞助}{16,7}{⾙、⼒}
  \begin{Phonetics}{赞助}{zan4zhu4}[][HSK 4]
    \definition{v.}{apoiar; patrocinar; concordar e ajudar (refere-se principalmente a oferecer dinheiro para ajudar)}
  \end{Phonetics}
\end{Entry}

\begin{Entry}{赞赏}{16,12}{⾙、⾙}
  \begin{Phonetics}{赞赏}{zan4 shang3}[][HSK 4]
    \definition{v.}{admirar; apreciar; valorizar}
  \end{Phonetics}
\end{Entry}

\begin{Entry}{赠}{16}{⾙}
  \begin{Phonetics}{赠}{zeng4}[][HSK 5]
    \definition{s.}{um presente (de despedida), uma lembrança; honrarias póstumas; uma patente de título}
    \definition{v.}{dar um presente; presentear com um brinde}
  \end{Phonetics}
\end{Entry}

\begin{Entry}{赠送}{16,9}{⾙、⾡}
  \begin{Phonetics}{赠送}{zeng4song4}[][HSK 5]
    \definition{v.}{dar; dar de presente; dar algo de graça a alguém}
  \end{Phonetics}
\end{Entry}

\begin{Entry}{踹}{16}{⾜}
  \begin{Phonetics}{踹}{chuai4}[][HSK 7-9]
    \definition{v.}{chutar (com a sola do pé) | pisar; pisotear; pisar em}
  \end{Phonetics}
\end{Entry}

\begin{Entry}{辨}{16}{⾟}
  \begin{Phonetics}{辨}{bian4}[][HSK 7-9]
    \definition{v.}{diferenciar; distinguir; discriminar | reconhecer; distinguir; identificar; discernir}
  \end{Phonetics}
\end{Entry}

\begin{Entry}{辨认}{16,4}{⾟、⾔}
  \begin{Phonetics}{辨认}{bian4ren4}[][HSK 7-9]
    \definition{v.}{identificar; reconhecer; identificar e julgar com base em características para encontrar ou identificar um objeto}
  \end{Phonetics}
\end{Entry}

\begin{Entry}{辨别}{16,7}{⾟、⼑}
  \begin{Phonetics}{辨别}{bian4bie2}[][HSK 7-9]
    \definition{v.}{diferenciar; distinguir; discriminar; encontrar características de diferentes coisas e diferenciá-las}
  \end{Phonetics}
\end{Entry}

\begin{Entry}{辩}{16}{⾟}
  \begin{Phonetics}{辩}{bian4}
    \definition{v.}{argumentar; disputar; debater}
  \end{Phonetics}
\end{Entry}

\begin{Entry}{辩论}{16,6}{⾟、⾔}
  \begin{Phonetics}{辩论}{bian4lun4}[][HSK 4]
    \definition{v.}{debater; obter um entendimento unificado ou correto, ambos os lados usam linguagem, palavras etc. para explicar seus pontos de vista, apontar os erros ou as contradições do outro lado}
  \end{Phonetics}
\end{Entry}

\begin{Entry}{辩护}{16,7}{⾟、⼿}
  \begin{Phonetics}{辩护}{bian4hu4}[][HSK 7-9]
    \definition{v.}{pleitear; defender | defender; argumentar em favor de}
  \end{Phonetics}
\end{Entry}

\begin{Entry}{辩解}{16,13}{⾟、⾓}
  \begin{Phonetics}{辩解}{bian4jie3}[][HSK 7-9]
    \definition{v.}{fornecer uma explicação; tentar se defender; explicar uma visão ou comportamento criticado; eliminar a crítica ou reduzir sua gravidade}
  \end{Phonetics}
\end{Entry}

\begin{Entry}{避}{16}{⾌}
  \begin{Phonetics}{避}{bi4}[][HSK 4]
    \definition{v.}{evitar; evadir; esquivar-se; buscar abrigo; fugir | impedir; manter afastado; repelir; previnir}
  \end{Phonetics}
\end{Entry}

\begin{Entry}{避免}{16,7}{⾌、⼉}
  \begin{Phonetics}{避免}{bi4mian3}[][HSK 4]
    \definition{v.}{evitar; desviar; abster-se de; tentar não fazer com que algo aconteça; prevenir; tentar impedir (que algo ruim aconteça) com antecedência}
  \end{Phonetics}
\end{Entry}

\begin{Entry}{避难}{16,10}{⾌、⾫}
  \begin{Phonetics}{避难}{bi4/nan4}[][HSK 7-9]
    \definition{s.}{refúgio}
    \definition{v.+compl.}{refugiar-se; buscar asilo (político etc.)}
  \end{Phonetics}
\end{Entry}

\begin{Entry}{避暑}{16,12}{⾌、⽇}
  \begin{Phonetics}{避暑}{bi4/shu3}[][HSK 7-9]
    \definition{v.+compl.}{ir de férias em um resort de verão; ir para um lugar fresco para evitar o calor do verão | prevenir insolação}
  \end{Phonetics}
\end{Entry}

\begin{Entry}{邀}{16}{⾡}
  \begin{Phonetics}{邀}{yao1}
    \definition{v.}{convidar; requerer | (literário)  buscar aprovação; pedir permissão | interceptar}
  \end{Phonetics}
\end{Entry}

\begin{Entry}{邀请}{16,10}{⾡、⾔}
  \begin{Phonetics}{邀请}{yao1qing3}[][HSK 5]
    \definition[份,个]{s.}{convite}
    \definition{v.}{convidar; solicitar; convidar pessoas para irem à sua casa ou a um local combinado}
  \end{Phonetics}
\end{Entry}

\begin{Entry}{醒}{16}{⾣}
  \begin{Phonetics}{醒}{xing3}[][HSK 4]
    \definition{adj.}{impressionante; notável; admirável; atraente; chamativo}
    \definition{v.}{ficar sóbrio; voltar a si; recuperar a consciência; retornar à normalidade após intoxicação, anestesia ou coma | despertar; estar acordado | ter a mente clara; mover a consciência da confusão para a compreensão | vir a entender; tornar-se ciente de; tomar consciência de}
  \end{Phonetics}
\end{Entry}

\begin{Entry}{镖}{16}{⾦}
  \begin{Phonetics}{镖}{biao1}
    \definition{s.}{dardo | arma de arremesso | mercadorias enviadas sob a proteção de uma escolta armada}
  \end{Phonetics}
\end{Entry}

\begin{Entry}{镜}{16}{⾦}
  \begin{Phonetics}{镜}{jing4}
    \definition*{s.}{Sobrenome Jing}
    \definition[面,副]{s.}{espelho | lente; vidro; dispositivos para auxiliar a visão ou conduzir experimentos ópticos}
    \definition{v.}{espelhar | perceber | usar como referência}
  \end{Phonetics}
\end{Entry}

\begin{Entry}{镜子}{16,3}{⾦、⼦}
  \begin{Phonetics}{镜子}{jing4zi5}[][HSK 4]
    \definition[面,个]{s.}{espelho; instrumento de reflexão de imagem liso e plano, antigamente esmerilhado a partir de um disco grosso de cobre fundido, atualmente feito de vidro plano revestido de prata ou alumínio | óculos; óculos de grau}
  \end{Phonetics}
\end{Entry}

\begin{Entry}{镜头}{16,5}{⾦、⼤}
  \begin{Phonetics}{镜头}{jing4tou2}[][HSK 4]
    \definition[个,组]{s.}{lente de câmera; objetiva; combinação de várias lentes, usada para formar uma imagem | foto; cena}
  \end{Phonetics}
\end{Entry}

\begin{Entry}{雕}{16}{⾫}
  \begin{Phonetics}{雕}{diao1}[][HSK 7-9]
    \definition*{s.}{Sobrenome Diao}
    \definition{s.}{abutre; águia | escultura ou obras esculpidas}
    \definition{v.}{esculpir; gravar}
  \end{Phonetics}
\end{Entry}

\begin{Entry}{雕刻}{16,8}{⾫、⼑}
  \begin{Phonetics}{雕刻}{diao1ke4}[][HSK 7-9]
    \definition[件,尊,个]{s.}{escultura; entalhe; obras de arte esculpidas}
    \definition{v.}{esculpir; gravar; esculpir uma imagem em metal, marfim, osso ou outros materiais}
  \end{Phonetics}
\end{Entry}

\begin{Entry}{雕塑}{16,13}{⾫、⼟}
  \begin{Phonetics}{雕塑}{diao1su4}[][HSK 7-9]
    \definition[座,件,尊,个]{s.}{escultura; obras de arte tridimensionais feitas de madeira, pedra ou metal por meio de entalhe, empilhamento, batida, etc.}
    \definition{v.}{esculpir; entalhar e moldar; usar madeira, pedra ou metal para criar formas artísticas tridimensionais esculpindo, empilhando, batendo, etc.}
  \end{Phonetics}
\end{Entry}

\begin{Entry}{霍}{16}{⾬}
  \begin{Phonetics}{霍}{huo4}
    \definition*{s.}{Sobrenome Huo}
    \definition{adv.}{Literário: de repente; rapidamente}
  \end{Phonetics}
\end{Entry}

\begin{Entry}{霍乱}{16,7}{⾬、⼄}
  \begin{Phonetics}{霍乱}{huo4luan4}[][HSK 7-9]
    \definition{s.}{cólera; uma doença altamente contagiosa causada pelo Vibrio Cholerae | gastroenterite aguda (geralmente se refere a sintomas como vômitos intensos, diarreia, dor abdominal e cólicas)}
  \end{Phonetics}
\end{Entry}

\begin{Entry}{颠}{16}{⾴}
  \begin{Phonetics}{颠}{dian1}
    \definition*{s.}{Sobrenome Dian}
    \definition{adj.}{mentalmente perturbado; insano; o mesmo que 癫}
    \definition{s.}{coroa (da cabeça) | topo; cume}
    \definition{v.}{sacudir; bater | cair; virar; tombar | Dialeto: correr; ir embora}
  \seealsoref{癫}{dian1}
  \end{Phonetics}
\end{Entry}

\begin{Entry}{颠倒}{16,10}{⾴、⼈}
  \begin{Phonetics}{颠倒}{dian1dao3}[][HSK 7-9]
    \definition{adj.}{confuso; desordenado}
    \definition{v.}{inverter; reverter; virar de cabeça para baixo}
  \end{Phonetics}
\end{Entry}

\begin{Entry}{颠覆}{16,18}{⾴、⾑}
  \begin{Phonetics}{颠覆}{dian1fu4}[][HSK 7-9]
    \definition{v.}{derrubar; subverter; virar; tombar | tombar; derrubar (um regime legítimo) por conspiração}
  \end{Phonetics}
\end{Entry}

\begin{Entry}{飙}{16}{⾵}
  \begin{Phonetics}{飙}{biao1}
    \definition{s.}{tempestade; furacão; redemoinho | Literário: vento violento; redemoinho}
  \end{Phonetics}
\end{Entry}

\begin{Entry}{飙升}{16,4}{⾵、⼗}
  \begin{Phonetics}{飙升}{biao1sheng1}[][HSK 7-9]
    \definition{v.}{disparar; subir rapidamente; (preço, quantidade, etc.) aumentam rapidamente}
  \end{Phonetics}
\end{Entry}

\begin{Entry}{餐}{16}{⾷}
  \begin{Phonetics}{餐}{can1}[][HSK 6]
    \definition{clas.}{comer; fazer uma refeição}
    \definition{clas.}{usado para refeições}
    \definition{s.}{comida; refeição}
  \end{Phonetics}
\end{Entry}

\begin{Entry}{餐厅}{16,4}{⾷、⼚}
  \begin{Phonetics}{餐厅}{can1ting1}[][HSK 5]
    \definition[间]{s.}{sala de jantar}
    \definition[间,家,个]{s.}{restaurante; refeitório em um hotel | cantina, refeitório; também é chamado de 食堂}
  \seealsoref{食堂}{shi2 tang2}
  \end{Phonetics}
\end{Entry}

\begin{Entry}{餐饮}{16,7}{⾷、⾷}
  \begin{Phonetics}{餐饮}{can1 yin3}[][HSK 5]
    \definition[个]{s.}{comidas e bebidas; refere-se a atividades de bufê em restaurantes e hotéis}
  \end{Phonetics}
\end{Entry}

\begin{Entry}{餐桌}{16,10}{⾷、⽊}
  \begin{Phonetics}{餐桌}{can1zhuo1}[][HSK 7-9]
    \definition[张]{s.}{mesa de jantar}
  \end{Phonetics}
\end{Entry}

\begin{Entry}{餐馆}{16,11}{⾷、⾷}
  \begin{Phonetics}{餐馆}{can1 guan3}[][HSK 5]
    \definition[家,个]{s.}{restaurante}
  \end{Phonetics}
\end{Entry}

\begin{Entry}{鲸}{16}{⿂}
  \begin{Phonetics}{鲸}{jing1}
    \definition[头,只,条]{s.}{baleia; cetáceo}
  \end{Phonetics}
\end{Entry}

\begin{Entry}{鲸鱼}{16,8}{⿂、⿂}
  \begin{Phonetics}{鲸鱼}{jing1yu2}
    \definition{s.}{baleia}
  \end{Phonetics}
\end{Entry}

\begin{Entry}{鲸鲨}{16,15}{⿂、⿂}
  \begin{Phonetics}{鲸鲨}{jing1sha1}
    \definition{s.}{tubarão baleia}
  \end{Phonetics}
\end{Entry}

\begin{Entry}{鹦}{16}{⿃}
  \begin{Phonetics}{鹦}{ying1}
    \definition[只]{s.}{papagaio}
  \end{Phonetics}
\end{Entry}

\begin{Entry}{鹦鹉}{16,13}{⿃、⿃}
  \begin{Phonetics}{鹦鹉}{ying1wu3}
    \definition{s.}{papagaio (ave)}
  \end{Phonetics}
\end{Entry}

\begin{Entry}{鹾}{16}{⿄}
  \begin{Phonetics}{鹾}{cuo2}
    \definition{adj.}{salgado}
    \definition{s.}{sal}
  \end{Phonetics}
\end{Entry}

\begin{Entry}{默}{16}{⿊}
  \begin{Phonetics}{默}{mo4}
    \definition*{s.}{Sobrenome Mo}
    \definition{adj.}{taciturno; reservado | silencioso}
    \definition{v.}{escrever de memória}
  \end{Phonetics}
\end{Entry}

\begin{Entry}{默契}{16,9}{⿊、⼤}
  \begin{Phonetics}{默契}{mo4qi4}
    \definition{adj.}{(de membros da equipe) bem coordenados}
    \definition{s.}{entendimento tácito | entendimento mútuo | conectado em um nível mútuo profundo | (de membros da equipe) bem coordenados}
  \end{Phonetics}
\end{Entry}

\begin{Entry}{默默}{16,16}{⿊、⿊}
  \begin{Phonetics}{默默}{mo4mo4}[][HSK 4]
    \definition{adj.}{mudo; quieto; silencioso}
    \definition{adv.}{silenciosamente}
  \end{Phonetics}
\end{Entry}

%%%%% EOF %%%%%

