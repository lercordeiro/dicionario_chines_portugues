%%%
%%% 16画
%%%

\section*{16画}\addcontentsline{toc}{section}{16画}

\begin{entry}{儒}{16}{⼈}
  \begin{phonetics}{儒}{ru2}
    \definition*{s.}{Confucionismo; Confucionista | Sobrenome Ru}
    \definition{s.}{(antigo) erudito; homem culto}
  \end{phonetics}
\end{entry}

\begin{entry}{儒教}{16,11}{⼈、⽁}
  \begin{phonetics}{儒教}{ru2jiao4}
    \definition*{s.}{Confucionismo}
  \end{phonetics}
\end{entry}

\begin{entry}{嘴}{16}{⼝}
  \begin{phonetics}{嘴}{zui3}[][HSK 2]
    \definition[张]{s.}{boca; boca humana ou animal | qualquer coisa com formato ou função semelhante a uma boca | fala | comida}
  \end{phonetics}
\end{entry}

\begin{entry}{嘴巴}{16,4}{⼝、⼰}
  \begin{phonetics}{嘴巴}{zui3 ba5}[][HSK 4]
    \definition[张]{s.}{boca}
  \end{phonetics}
\end{entry}

\begin{entry}{嘴巴子}{16,4,3}{⼝、⼰、⼦}
  \begin{phonetics}{嘴巴子}{zui3ba5zi5}
    \definition{s.}{tapa | bofetada}
  \end{phonetics}
\end{entry}

\begin{entry}{器}{16}{⼝}
  \begin{phonetics}{器}{qi4}
    \definition[台]{s.}{dispositivo | ferramenta | utensílio}
  \end{phonetics}
\end{entry}

\begin{entry}{器官}{16,8}{⼝、⼧}
  \begin{phonetics}{器官}{qi4guan1}[][HSK 4]
    \definition[个]{s.}{órgão; aparelho; parte de um organismo que consiste em vários tipos de tecidos celulares que podem desempenhar uma função fisiológica separada}
  \end{phonetics}
\end{entry}

\begin{entry}{壁}{16}{⼟}
  \begin{phonetics}{壁}{bi4}
    \definition*{s.}{Bi, a décima quarta das vinte e oito constelações em que a esfera celeste foi dividida, consistindo em duas estrelas em linha reta, uma em Pégaso e a outra em Andrômeda | A Estrela Bìxìu, uma das Vinte e Oito Mansões da astronomia tradicional chinesa}
    \definition[道]{s.}{parede | superfície plana como uma parede | penhasco | muralha; parapeito | barreira}
  \end{phonetics}
\end{entry}

\begin{entry}{壁纸}{16,7}{⼟、⽷}
  \begin{phonetics}{壁纸}{bi4zhi3}
    \definition{s.}{papel de parede; papel colado em paredes internas para decoração ou proteção, com diversos tipos e cores}
  \end{phonetics}
\end{entry}

\begin{entry}{壁虎}{16,8}{⼟、⾌}
  \begin{phonetics}{壁虎}{bi4hu3}
    \definition{s.}{lagartixa}
  \end{phonetics}
\end{entry}

\begin{entry}{懒}{16}{⼼}
  \begin{phonetics}{懒}{lan3}[][HSK 6]
    \definition{adj.}{indolente; preguiçoso (oposto de 勤) | lento; lânguido | ocioso; preguiçoso}
  \seealsoref{勤}{qin2}
  \end{phonetics}
\end{entry}

\begin{entry}{懒人}{16,2}{⼼、⼈}
  \begin{phonetics}{懒人}{lan3ren2}
    \definition{s.}{pessoa preguiçosa}
  \end{phonetics}
\end{entry}

\begin{entry}{懒汉}{16,5}{⼼、⽔}
  \begin{phonetics}{懒汉}{lan3han4}
    \definition{s.}{sujeito ocioso | vagabundo | preguiçosos}
  \end{phonetics}
\end{entry}

\begin{entry}{懒虫}{16,6}{⼼、⾍}
  \begin{phonetics}{懒虫}{lan3chong2}
    \definition{s.}{desleixado ocioso | (insulto) sujeito preguiçoso}
  \end{phonetics}
\end{entry}

\begin{entry}{懒怠}{16,9}{⼼、⼼}
  \begin{phonetics}{懒怠}{lan3dai4}
    \definition{s.}{preguiça}
  \end{phonetics}
\end{entry}

\begin{entry}{懒鬼}{16,9}{⼼、⿁}
  \begin{phonetics}{懒鬼}{lan3gui3}
    \definition{s.}{cara preguiçoso}
  \end{phonetics}
\end{entry}

\begin{entry}{懒得}{16,11}{⼼、⼻}
  \begin{phonetics}{懒得}{lan3de5}
    \definition{adv.}{demasiado preguiçoso}
    \definition{v.}{não sentir vontade (de fazer algo)}
  \end{phonetics}
\end{entry}

\begin{entry}{懒惰}{16,12}{⼼、⼼}
  \begin{phonetics}{懒惰}{lan3duo4}
    \definition{adj.}{preguiçoso}
  \end{phonetics}
\end{entry}

\begin{entry}{懒散}{16,12}{⼼、⽁}
  \begin{phonetics}{懒散}{lan3san3}
    \definition{adj.}{inativo | indolente | preguiçoso | negligente}
  \end{phonetics}
\end{entry}

\begin{entry}{懒腰}{16,13}{⼼、⾁}
  \begin{phonetics}{懒腰}{lan3yao1}
    \definition[个]{s.}{alongamento (do corpo)}
  \end{phonetics}
\end{entry}

\begin{entry}{撼}{16}{⼿}
  \begin{phonetics}{撼}{han4}
    \definition{v.}{agitar; sacudir}
  \end{phonetics}
\end{entry}

\begin{entry}{擅}{16}{⼿}
  \begin{phonetics}{擅}{shan4}
    \definition{adv.}{sem autorização; arbitrariamente | fazer algo por conta própria}
    \definition{v.}{ser bom em; ser especialista em | arrogar-se a si mesmo; fazer algo por conta própria | reivindicar arbitrariamente; ir além do escopo e ajir arbitrariamente}
  \end{phonetics}
\end{entry}

\begin{entry}{擅自}{16,6}{⼿、⾃}
  \begin{phonetics}{擅自}{shan4zi4}
    \definition{adv.}{sem permissão ou autorização | por iniciativa própria}
  \end{phonetics}
\end{entry}

\begin{entry}{操}{16}{⼿}
  \begin{phonetics}{操}{cao1}
    \definition*{s.}{Sobrenome Cao}
    \definition[节,套]{s.}{exercício; ginástica | conduta; comportamento; moralidade, a moral e o código de conduta que as pessoas seguem}
    \definition{v.}{segurar; agarrar; segurar na mão | fazer algo; envolver-se em | falar (uma língua ou dialeto) | treinar (tropas); exercitar (corpo); praticar ou treinar de acordo com uma determinada forma ou postura | dirigir; manusear}
  \end{phonetics}
\end{entry}

\begin{entry}{操心}{16,4}{⼿、⼼}
  \begin{phonetics}{操心}{cao1xin1}
    \definition{v.+compl.}{preocupar-se com}
  \end{phonetics}
\end{entry}

\begin{entry}{操场}{16,6}{⼿、⼟}
  \begin{phonetics}{操场}{cao1chang3}[][HSK 4]
    \definition[个]{s.}{\emph{playground}; campo esportivo; locais para exercícios físicos ou exercícios militares}
  \end{phonetics}
\end{entry}

\begin{entry}{操作}{16,7}{⼿、⼈}
  \begin{phonetics}{操作}{cao1zuo4}[][HSK 4]
    \definition{s.}{operação}
    \definition{v.}{operar; seguir os requisitos e procedimentos prescritos| implementar; realizar; executar; refere-se à implementação concreta (planos, medidas, etc.)}
  \end{phonetics}
\end{entry}

\begin{entry}{操纵}{16,7}{⼿、⽷}
  \begin{phonetics}{操纵}{cao1zong4}[][HSK 6]
    \definition{v.}{operar; controlar (uma máquina, instrumento, etc.) | manipular; controlar secretamente; assumir o controle de (uma pessoa, organização, situação, etc.)}
  \end{phonetics}
\end{entry}

\begin{entry}{整}{16}{⽁}
  \begin{phonetics}{整}{zheng3}[][HSK 3]
    \definition*{s.}{Sobrenome Zheng}
    \definition{adj.}{cheio; integral; inteiro; completo; sem defeitos | limpo; arrumado; organizado; em boa ordem | redondo (não é uma fração)}
    \definition{s.}{número inteiro (não fracionário)}
    \definition{v.}{retificar; corrigir; pôr em ordem | consertar; renovar; reparar | corrigir; punir; causar sofrimento;  fazer alguém sofrer | fazer; realizar; trabalhar; em algumas regiões, significa 做, 搞}
  \seealsoref{搞}{gao3}
  \seealsoref{做}{zuo4}
  \end{phonetics}
\end{entry}

\begin{entry}{整个}{16,3}{⽁、⼈}
  \begin{phonetics}{整个}{zheng3ge4}[][HSK 3]
    \definition{adj.}{total; inteiro; completo}
  \end{phonetics}
\end{entry}

\begin{entry}{整天}{16,4}{⽁、⼤}
  \begin{phonetics}{整天}{zheng3 tian1}[][HSK 3]
    \definition{s.}{o dia inteiro; o dia todo; durante todo o dia; de manhã à noite}
  \end{phonetics}
\end{entry}

\begin{entry}{整齐}{16,6}{⽁、⿑}
  \begin{phonetics}{整齐}{zheng3qi2}[][HSK 3]
    \definition{adj.}{arrumado; organizado; em boa ordem | uniforme; regular; tamanho, comprimento, grau, etc. são relativamente consistentes | usado para descrever que todas as coisas necessárias estão prontas}
    \definition{v.}{estar em boas condições; manter a ordem e a organização}
  \end{phonetics}
\end{entry}

\begin{entry}{整体}{16,7}{⽁、⼈}
  \begin{phonetics}{整体}{zheng3ti3}[][HSK 3]
    \definition[个]{s.}{um todo; totalidade}
  \end{phonetics}
\end{entry}

\begin{entry}{整理}{16,11}{⽁、⽟}
  \begin{phonetics}{整理}{zheng3li3}[][HSK 3]
    \definition{v.}{organizar; reorganizar; classificar; ordenar; colocar em ordem}
  \end{phonetics}
\end{entry}

\begin{entry}{整整}{16,16}{⽁、⽁}
  \begin{phonetics}{整整}{zheng3 zheng3}[][HSK 3]
    \definition{adv.}{inteiramente; completamente; solidamente; continuamente}
  \end{phonetics}
\end{entry}

\begin{entry}{橘}{16}{⽊}
  \begin{phonetics}{橘}{ju2}
    \definition[只,棵]{s.}{tangerina}
  \end{phonetics}
\end{entry}

\begin{entry}{橘子汁}{16,3,5}{⽊、⼦、⽔}
  \begin{phonetics}{橘子汁}{ju2zi5zhi1}
    \definition[瓶,杯,罐,盒]{s.}{suco de laranja}
  \seealsoref{橙汁}{cheng2zhi1}
  \seealsoref{柳橙汁}{liu3cheng2zhi1}
  \end{phonetics}
\end{entry}

\begin{entry}{橙}{16}{⽊}
  \begin{phonetics}{橙}{cheng2}
    \definition{s.}{laranja; fruta da laranjeira | laranjeira; pé de laranja | cor laranja}
  \end{phonetics}
\end{entry}

\begin{entry}{橙汁}{16,5}{⽊、⽔}
  \begin{phonetics}{橙汁}{cheng2zhi1}
    \definition[瓶,杯,罐,盒]{s.}{suco de laranja}
  \seealsoref{橘子汁}{ju2zi5zhi1}
  \seealsoref{柳橙汁}{liu3cheng2zhi1}
  \end{phonetics}
\end{entry}

\begin{entry}{橙色}{16,6}{⽊、⾊}
  \begin{phonetics}{橙色}{cheng2 se4}
    \definition{s.}{cor de laranja}
  \end{phonetics}
\end{entry}

\begin{entry}{激}{16}{⽔}
  \begin{phonetics}{激}{ji1}
    \definition*{s.}{Sobrenome Ji}
    \definition{adj.}{afiado; feroz; violento | vívido}
    \definition{adv.}{bruscamente; ferozmente; violentamente}
    \definition{s.}{o impacto de ondas fortes contra a costa}
    \definition{v.}{bater; avançar; correr | despertar; estimular; incitar; excitar | ficar doente por se molhar | esfriar (colocando água gelada, etc.)}
  \end{phonetics}
\end{entry}

\begin{entry}{激动}{16,6}{⽔、⼒}
  \begin{phonetics}{激动}{ji1dong4}[][HSK 4]
    \definition{adj.}{animado; entusiasmado; empolgado}
    \definition{v.}{agitar; excitar; tornar fortes os sentimentos de alguém}
  \end{phonetics}
\end{entry}

\begin{entry}{激烈}{16,10}{⽔、⽕}
  \begin{phonetics}{激烈}{ji1lie4}[][HSK 4]
    \definition{adj.}{agudo; afiado; feroz; violento; intenso}
  \end{phonetics}
\end{entry}

\begin{entry}{燃}{16}{⽕}
  \begin{phonetics}{燃}{ran2}
    \definition{v.}{queimar | acender; inflamar}
  \end{phonetics}
\end{entry}

\begin{entry}{燃料}{16,10}{⽕、⽃}
  \begin{phonetics}{燃料}{ran2 liao4}[][HSK 4]
    \definition{s.}{combustível; carburante; substâncias que podem gerar calor e energia luminosa quando queimadas podem ser divididas em três tipos de acordo com sua forma: combustível sólido (como carvão, carvão vegetal, madeira), combustível líquido (como gasolina, querosene) e combustível gasoso (como gás de carvão, biogás); também se refere a substâncias que podem gerar energia nuclear, como urânio, plutônio, etc.}
  \end{phonetics}
\end{entry}

\begin{entry}{燃烧}{16,10}{⽕、⽕}
  \begin{phonetics}{燃烧}{ran2shao1}[][HSK 4]
    \definition{s.}{combustão | flama}
    \definition{v.}{queimar; acender | arder; inflamar; ferver; metáfora para as emoções de uma pessoa serem muito fortes, como um fogo ardente}
  \end{phonetics}
\end{entry}

\begin{entry}{犟}{16}{⽜}
  \begin{phonetics}{犟}{jiang4}
    \variantof{强}
  \end{phonetics}
\end{entry}

\begin{entry}{磨}{16}{⽯}
  \begin{phonetics}{磨}{mo2}[][HSK 6]
    \definition{v.}{esfregar; desgastar | moer; refletir; polir | desgastar; esgotar; cansar; exaurir | incomodar; causar problemas | destruir; obliterar; extinguir-se | ficar ocioso; perder tempo; perder tempo; procrastinar}
  \end{phonetics}
  \begin{phonetics}{磨}{mo4}
    \definition[盘]{s.}{mó (pedra pesada e redonda para moinho)}
    \definition{v.}{moer; esfarelar; triturar | virar; inverter a marcha}
  \end{phonetics}
\end{entry}

\begin{entry}{磨菇}{16,11}{⽯、⾋}
  \begin{phonetics}{磨菇}{mo2gu5}
    \variantof{蘑菇}
  \end{phonetics}
\end{entry}

\begin{entry}{篮}{16}{⽵}
  \begin{phonetics}{篮}{lan2}
    \definition[个]{s.}{cesto | o anel de ferro e a rede na cesta de basquete}
  \end{phonetics}
\end{entry}

\begin{entry}{篮球}{16,11}{⽵、⽟}
  \begin{phonetics}{篮球}{lan2qiu2}[][HSK 2]
    \definition[个,只]{s.}{basquetebol | bola de basquete; refere-se à bola utilizada no basquetebol}
  \end{phonetics}
\end{entry}

\begin{entry}{糕}{16}{⽶}
  \begin{phonetics}{糕}{gao1}
    \definition{s.}{bolo; alimentos feitos de farinha de arroz, farinha de trigo, etc.}
  \end{phonetics}
\end{entry}

\begin{entry}{糕点}{16,9}{⽶、⽕}
  \begin{phonetics}{糕点}{gao1dian3}
    \definition{s.}{bolos | pastéis}
  \end{phonetics}
\end{entry}

\begin{entry}{糕点师}{16,9,6}{⽶、⽕、⼱}
  \begin{phonetics}{糕点师}{gao1dian3 shi1}
    \definition{s.}{confeiteiro}
  \end{phonetics}
\end{entry}

\begin{entry}{糕点店}{16,9,8}{⽶、⽕、⼴}
  \begin{phonetics}{糕点店}{gao1dian3 dian4}
    \definition{s.}{confeitaria}
  \end{phonetics}
\end{entry}

\begin{entry}{糖}{16}{⽶}
  \begin{phonetics}{糖}{tang2}[][HSK 3]
    \definition[包,斤,勺,袋,块]{s.}{açúcar; um tipo de açúcar; um tipo de composto orgânico, que pode ser dividido em três tipos: monossacarídeos, dissacarídeos e polissacarídeos; é a principal substância que produz energia térmica no corpo humano, como glicose, sacarose, lactose, amido, etc. | açúcar; açúcar comestível; termo geral para açúcar | doces; balas | carboidrato; algo doce e calórico}
  \end{phonetics}
\end{entry}

\begin{entry}{糖醋鱼}{16,15,8}{⽶、⾣、⿂}
  \begin{phonetics}{糖醋鱼}{tang2cu4yu2}
    \definition{s.}{peixe guisado em molho agridoce (prato)}
  \end{phonetics}
\end{entry}

\begin{entry}{膨}{16}{⾁}
  \begin{phonetics}{膨}{peng2}
    \definition{v.}{inchar; inflar | expandir; aumentar o comprimento ou o volume de um objeto}
  \end{phonetics}
\end{entry}

\begin{entry}{膨胀}{16,8}{⾁、⾁}
  \begin{phonetics}{膨胀}{peng2zhang4}
    \definition{v.}{expandir | inflar | inchar}
  \end{phonetics}
\end{entry}

\begin{entry}{蕹}{16}{⾋}
  \begin{phonetics}{蕹}{weng4}
    \definition{s.}{espinafre-d’água ou \emph{ong choy}, usado como vegetal no sul da China e no sudeste da Ásia}
  \end{phonetics}
\end{entry}

\begin{entry}{蕹菜}{16,11}{⾋、⾋}
  \begin{phonetics}{蕹菜}{weng4cai4}
    \definition{s.}{espinafre aquático | \emph{ong choy} | repolho do pântano | convolvulus aquático | glória-da-manhã aquática}
  \seealsoref{空心菜}{kong1xin1cai4}
  \end{phonetics}
\end{entry}

\begin{entry}{薄}{16}{⾋}
  \begin{phonetics}{薄}{bao2}[][HSK 4]
    \definition{adj.}{fino; frágil | frio; indiferente; carente de calor | leve; fraco | pobre; infértil}
  \end{phonetics}
  \begin{phonetics}{薄}{bo2}
    \definition*{s.}{Sobrenome Bo}
    \definition{adj.}{pequeno; leve; magro | mau; cruel; mesquinho | frívolo; fútil; não solene | fraco; frágil}
    \definition{v.}{desprezar; tratar com desprezo; menosprezar | aproximar-se}
  \end{phonetics}
  \begin{phonetics}{薄}{bo4}
    \definition{s.}{menta; uma erva perene com aroma refrescante nos caules e folhas}
  \end{phonetics}
\end{entry}

\begin{entry}{薄弱}{16,10}{⾋、⼸}
  \begin{phonetics}{薄弱}{bo2ruo4}[][HSK 5]
    \definition{adj.}{fraco; frágil}
  \end{phonetics}
\end{entry}

\begin{entry}{薯}{16}{⾋}
  \begin{phonetics}{薯}{shu3}
    \definition{s.}{batata | inhame}
  \end{phonetics}
\end{entry}

\begin{entry}{赞}{16}{⾙}
  \begin{phonetics}{赞}{zan4}
    \definition{v.}{patrocinar | apoiar | elogiar | (gíria na \emph{Internet}) para curtir (uma postagem \emph{on-line})}
  \end{phonetics}
\end{entry}

\begin{entry}{赞成}{16,6}{⾙、⼽}
  \begin{phonetics}{赞成}{zan4cheng2}[][HSK 4]
    \definition{v.}{endossar; favorecer; aprovar; concordar com; concordar ou apoiar as ideias, os planos, as propostas ou o comportamento de outra pessoa}
  \end{phonetics}
\end{entry}

\begin{entry}{赞扬}{16,6}{⾙、⼿}
  \begin{phonetics}{赞扬}{zan4yang2}
    \definition{v.}{elogiar | aprovar | demonstrar aprovação}
  \end{phonetics}
\end{entry}

\begin{entry}{赞助}{16,7}{⾙、⼒}
  \begin{phonetics}{赞助}{zan4zhu4}[][HSK 4]
    \definition{s.}{patrocinador}
    \definition{v.}{apoiar; patrocinar; concordar e ajudar (refere-se principalmente a oferecer dinheiro para ajudar)}
  \end{phonetics}
\end{entry}

\begin{entry}{赞赏}{16,12}{⾙、⾙}
  \begin{phonetics}{赞赏}{zan4 shang3}[][HSK 4]
    \definition{v.}{admirar; apreciar; valorizar}
  \end{phonetics}
\end{entry}

\begin{entry}{赠}{16}{⾙}
  \begin{phonetics}{赠}{zeng4}[][HSK 5]
    \definition{v.}{dar um presente; presentear com um brinde}
  \end{phonetics}
\end{entry}

\begin{entry}{赠送}{16,9}{⾙、⾡}
  \begin{phonetics}{赠送}{zeng4song4}[][HSK 5]
    \definition{v.}{dar; dar de presente; dar algo de graça a alguém}
  \end{phonetics}
\end{entry}

\begin{entry}{辩}{16}{⾟}
  \begin{phonetics}{辩}{bian4}
    \definition{v.}{argumentar; disputar; debater}
  \end{phonetics}
\end{entry}

\begin{entry}{辩论}{16,6}{⾟、⾔}
  \begin{phonetics}{辩论}{bian4lun4}[][HSK 4]
    \definition[场,次]{s.}{debate; argumento; a atividade comportamental em si de argumentar ou refutar diferentes pontos de vista ou afirmações, ou uma ocasião ou situação em que tal argumentação ou refutação é feita}
    \definition{v.}{debater; obter um entendimento unificado ou correto, ambos os lados usam linguagem, palavras etc. para explicar seus pontos de vista, apontar os erros ou as contradições do outro lado}
  \end{phonetics}
\end{entry}

\begin{entry}{避}{16}{⾌}
  \begin{phonetics}{避}{bi4}[][HSK 4]
    \definition{v.}{evitar; evadir; esquivar-se; buscar abrigo; fugir | impedir; manter afastado; repelir; previnir}
  \end{phonetics}
\end{entry}

\begin{entry}{避免}{16,7}{⾌、⼉}
  \begin{phonetics}{避免}{bi4mian3}[][HSK 4]
    \definition{v.}{evitar; desviar; abster-se de; tentar não fazer com que algo aconteça; prevenir; tentar impedir (que algo ruim aconteça) com antecedência}
  \end{phonetics}
\end{entry}

\begin{entry}{邀}{16}{⾡}
  \begin{phonetics}{邀}{yao1}
    \definition{v.}{convidar; requerer | (literário)  buscar aprovação; pedir permissão | interceptar}
  \end{phonetics}
\end{entry}

\begin{entry}{邀请}{16,10}{⾡、⾔}
  \begin{phonetics}{邀请}{yao1qing3}[][HSK 5]
    \definition[份,个]{s.}{convite}
    \definition{v.}{convidar; solicitar; convidar pessoas para irem à sua casa ou a um local combinado}
  \end{phonetics}
\end{entry}

\begin{entry}{醒}{16}{⾣}
  \begin{phonetics}{醒}{xing3}[][HSK 4]
    \definition{adj.}{impressionante; notável; admirável; atraente; chamativo}
    \definition{v.}{ficar sóbrio; voltar a si; recuperar a consciência; retornar à normalidade após intoxicação, anestesia ou coma | despertar; estar acordado | ter a mente clara; mover a consciência da confusão para a compreensão | vir a entender; tornar-se ciente de; tomar consciência de}
  \end{phonetics}
\end{entry}

\begin{entry}{镖}{16}{⾦}
  \begin{phonetics}{镖}{biao1}
    \definition{s.}{dardo | arma de arremesso | mercadorias enviadas sob a proteção de uma escolta armada}
  \end{phonetics}
\end{entry}

\begin{entry}{镜}{16}{⾦}
  \begin{phonetics}{镜}{jing4}
    \definition*{s.}{Sobrenome Jing}
    \definition[面,副]{s.}{espelho | lente; vidro; dispositivos para auxiliar a visão ou conduzir experimentos ópticos}
    \definition{v.}{espelhar | perceber | usar como referência}
  \end{phonetics}
\end{entry}

\begin{entry}{镜子}{16,3}{⾦、⼦}
  \begin{phonetics}{镜子}{jing4zi5}[][HSK 4]
    \definition[面,个]{s.}{espelho; instrumento de reflexão de imagem liso e plano, antigamente esmerilhado a partir de um disco grosso de cobre fundido, atualmente feito de vidro plano revestido de prata ou alumínio | óculos; óculos de grau}
  \end{phonetics}
\end{entry}

\begin{entry}{镜头}{16,5}{⾦、⼤}
  \begin{phonetics}{镜头}{jing4tou2}[][HSK 4]
    \definition[个]{s.}{lente de câmera; objetiva; combinação de várias lentes, usada para formar uma imagem | foto; cena}
  \end{phonetics}
\end{entry}

\begin{entry}{雕}{16}{⾫}
  \begin{phonetics}{雕}{diao1}
    \definition*{s.}{Sobrenome Diao}
    \definition{s.}{abutre; águia | escultura ou obras esculpidas}
    \definition{v.}{esculpir; gravar}
  \end{phonetics}
\end{entry}

\begin{entry}{雕刻}{16,8}{⾫、⼑}
  \begin{phonetics}{雕刻}{diao1ke4}
    \definition{s.}{escultura}
    \definition{v.}{esculpir | gravar}
  \end{phonetics}
\end{entry}

\begin{entry}{餐}{16}{⾷}
  \begin{phonetics}{餐}{can1}[][HSK 6]
    \definition{clas.}{comer; fazer uma refeição}
    \definition{clas.}{usado para refeições}
    \definition{s.}{comida; refeição}
  \end{phonetics}
\end{entry}

\begin{entry}{餐厅}{16,4}{⾷、⼚}
  \begin{phonetics}{餐厅}{can1ting1}[][HSK 5]
    \definition[家]{s.}{restaurante; refeitório em um hotel | cantina, refeitório; também é chamado de 食堂}
    \definition[间]{s.}{sala de jantar}
  \seealsoref{食堂}{shi2 tang2}
  \end{phonetics}
\end{entry}

\begin{entry}{餐饮}{16,7}{⾷、⾷}
  \begin{phonetics}{餐饮}{can1 yin3}[][HSK 5]
    \definition{s.}{comidas e bebidas; refere-se a atividades de bufê em restaurantes e hotéis}
  \end{phonetics}
\end{entry}

\begin{entry}{餐馆}{16,11}{⾷、⾷}
  \begin{phonetics}{餐馆}{can1 guan3}[][HSK 5]
    \definition[家]{s.}{restaurante;}
  \end{phonetics}
\end{entry}

\begin{entry}{鲸}{16}{⿂}
  \begin{phonetics}{鲸}{jing1}
    \definition[头,只,条]{s.}{baleia; cetáceo}
  \end{phonetics}
\end{entry}

\begin{entry}{鲸鱼}{16,8}{⿂、⿂}
  \begin{phonetics}{鲸鱼}{jing1yu2}
    \definition{s.}{baleia}
  \end{phonetics}
\end{entry}

\begin{entry}{鲸鲨}{16,15}{⿂、⿂}
  \begin{phonetics}{鲸鲨}{jing1sha1}
    \definition{s.}{tubarão baleia}
  \end{phonetics}
\end{entry}

\begin{entry}{鹦}{16}{⿃}
  \begin{phonetics}{鹦}{ying1}
    \definition[只]{s.}{papagaio}
  \end{phonetics}
\end{entry}

\begin{entry}{鹦鹉}{16,13}{⿃、⿃}
  \begin{phonetics}{鹦鹉}{ying1wu3}
    \definition{s.}{papagaio (ave)}
  \end{phonetics}
\end{entry}

\begin{entry}{鹾}{16}{⿄}
  \begin{phonetics}{鹾}{cuo2}
    \definition{adj.}{salgado}
    \definition{s.}{sal}
  \end{phonetics}
\end{entry}

\begin{entry}{默}{16}{⿊}
  \begin{phonetics}{默}{mo4}
    \definition*{s.}{Sobrenome Mo}
    \definition{adj.}{taciturno; reservado | silencioso}
    \definition{v.}{escrever de memória}
  \end{phonetics}
\end{entry}

\begin{entry}{默契}{16,9}{⿊、⼤}
  \begin{phonetics}{默契}{mo4qi4}
    \definition{adj.}{(de membros da equipe) bem coordenados}
    \definition{s.}{entendimento tácito | entendimento mútuo | conectado em um nível mútuo profundo | (de membros da equipe) bem coordenados}
  \end{phonetics}
\end{entry}

\begin{entry}{默默}{16,16}{⿊、⿊}
  \begin{phonetics}{默默}{mo4mo4}[][HSK 4]
    \definition{adj.}{mudo; silencioso}
    \definition{adv.}{silenciosamente}
  \end{phonetics}
\end{entry}

%%%%% EOF %%%%%

