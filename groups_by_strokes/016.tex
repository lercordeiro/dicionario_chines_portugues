%%%
%%% 16画
%%%

\section*{16画}\addcontentsline{toc}{section}{16画}

\begin{entry}{儒教}{16,11}
  \begin{phonetics}{儒教}{ru2jiao4}
    \definition*{s.}{Confucionismo}
  \end{phonetics}
\end{entry}

\begin{entry}{嘴}{16}[Radical 口]
  \begin{phonetics}{嘴}{zui3}[][HSK 2]
    \definition[张]{s.}{boca | qualquer coisa com formato ou função semelhante a uma boca}
    \definition{v.}{falar}
  \end{phonetics}
\end{entry}

\begin{entry}{嘴巴}{16,4}
  \begin{phonetics}{嘴巴}{zui3ba5}
    \definition[张]{s.}{boca}
    \definition[个]{s.}{bofetada na cara}
  \end{phonetics}
\end{entry}

\begin{entry}{嘴巴子}{16,4,3}
  \begin{phonetics}{嘴巴子}{zui3ba5zi5}
    \definition{s.}{tapa | bofetada}
  \end{phonetics}
\end{entry}

\begin{entry}{器}{16}[Radical 口]
  \begin{phonetics}{器}{qi4}
    \definition[台]{s.}{dispositivo | ferramenta | utensílio}
  \end{phonetics}
\end{entry}

\begin{entry}{壁纸}{16,7}
  \begin{phonetics}{壁纸}{bi4zhi3}
    \definition{s.}{papel de parede}
  \end{phonetics}
\end{entry}

\begin{entry}{壁虎}{16,8}
  \begin{phonetics}{壁虎}{bi4hu3}
    \definition{s.}{lagartixa}
  \end{phonetics}
\end{entry}

\begin{entry}{懒}{16}[Radical 心]
  \begin{phonetics}{懒}{lan3}
    \definition{adj.}{preguiçoso | indolente | vadio}
  \end{phonetics}
\end{entry}

\begin{entry}{懒人}{16,2}
  \begin{phonetics}{懒人}{lan3ren2}
    \definition{s.}{pessoa preguiçosa}
  \end{phonetics}
\end{entry}

\begin{entry}{懒汉}{16,5}
  \begin{phonetics}{懒汉}{lan3han4}
    \definition{s.}{sujeito ocioso | vagabundo | preguiçosos}
  \end{phonetics}
\end{entry}

\begin{entry}{懒虫}{16,6}
  \begin{phonetics}{懒虫}{lan3chong2}
    \definition{s.}{desleixado ocioso | (insulto) sujeito preguiçoso}
  \end{phonetics}
\end{entry}

\begin{entry}{懒怠}{16,9}
  \begin{phonetics}{懒怠}{lan3dai4}
    \definition{s.}{preguiça}
  \end{phonetics}
\end{entry}

\begin{entry}{懒鬼}{16,9}
  \begin{phonetics}{懒鬼}{lan3gui3}
    \definition{s.}{cara preguiçoso}
  \end{phonetics}
\end{entry}

\begin{entry}{懒得}{16,11}
  \begin{phonetics}{懒得}{lan3de5}
    \definition{adv.}{demasiado preguiçoso}
    \definition{v.}{não sentir vontade (de fazer algo)}
  \end{phonetics}
\end{entry}

\begin{entry}{懒惰}{16,12}
  \begin{phonetics}{懒惰}{lan3duo4}
    \definition{adj.}{preguiçoso}
  \end{phonetics}
\end{entry}

\begin{entry}{懒散}{16,12}
  \begin{phonetics}{懒散}{lan3san3}
    \definition{adj.}{inativo | indolente | preguiçoso | negligente}
  \end{phonetics}
\end{entry}

\begin{entry}{懒腰}{16,13}
  \begin{phonetics}{懒腰}{lan3yao1}
    \definition[个]{s.}{alongamento (do corpo)}
  \end{phonetics}
\end{entry}

\begin{entry}{撼}{16}[Radical 手]
  \begin{phonetics}{撼}{han4}
    \definition{v.}{sacudir | vibrar}
  \end{phonetics}
\end{entry}

\begin{entry}{擅自}{16,6}
  \begin{phonetics}{擅自}{shan4zi4}
    \definition{adv.}{sem permissão ou autorização | por iniciativa própria}
  \end{phonetics}
\end{entry}

\begin{entry}{操心}{16,4}
  \begin{phonetics}{操心}{cao1xin1}
    \definition{v.+compl.}{preocupar-se com}
  \end{phonetics}
\end{entry}

\begin{entry}{操作}{16,7}
  \begin{phonetics}{操作}{cao1zuo4}
    \definition{s.}{operação}
    \definition{v.}{trabalhar | operar | manipular}
  \end{phonetics}
\end{entry}

\begin{entry}{整天}{16,4}
  \begin{phonetics}{整天}{zheng3tian1}
    \definition{adv.}{dia todo | o dia inteiro}
  \end{phonetics}
\end{entry}

\begin{entry}{橘子汁}{16,3,5}
  \begin{phonetics}{橘子汁}{ju2zi5zhi1}
    \definition[瓶,杯,罐,盒]{s.}{suco de laranja}
  \seealsoref{橙汁}{cheng2zhi1}
  \seealsoref{柳橙汁}{liu3cheng2zhi1}
  \end{phonetics}
\end{entry}

\begin{entry}{橙汁}{16,5}
  \begin{phonetics}{橙汁}{cheng2zhi1}
    \definition[瓶,杯,罐,盒]{s.}{suco de laranja}
  \seealsoref{橘子汁}{ju2zi5zhi1}
  \seealsoref{柳橙汁}{liu3cheng2zhi1}
  \end{phonetics}
\end{entry}

\begin{entry}{橙色}{16,6}
  \begin{phonetics}{橙色}{cheng2 se4}
    \definition{s.}{cor de laranja}
  \end{phonetics}
\end{entry}

\begin{entry}{激动}{16,6}
  \begin{phonetics}{激动}{ji1dong4}
    \definition{v.}{excitar | mover-se emocionalmente | agitar (emoções)}
  \end{phonetics}
\end{entry}

\begin{entry}{燃烧}{16,10}
  \begin{phonetics}{燃烧}{ran2shao1}
    \definition{s.}{combustão | flama}
    \definition{v.}{queimar | acender}
  \end{phonetics}
\end{entry}

\begin{entry}{犟}{16}[Radical 牛]
  \begin{phonetics}{犟}{jiang4}
    \variantof{强}
  \end{phonetics}
\end{entry}

\begin{entry}{磨}{16}[Radical 石]
  \begin{phonetics}{磨}{mo2}
    \definition{v.}{moer | polir | afiar | desgastar | esfregar}
  \end{phonetics}
  \begin{phonetics}{磨}{mo4}
    \definition{s.}{mó (pedra pesada e redonda para moinho)}
    \definition{v.}{moer}
  \end{phonetics}
\end{entry}

\begin{entry}{磨菇}{16,11}
  \begin{phonetics}{磨菇}{mo2gu5}
    \variantof{蘑菇}
  \end{phonetics}
\end{entry}

\begin{entry}{篮球}{16,11}
  \begin{phonetics}{篮球}{lan2qiu2}[][HSK 2]
    \definition[个,只]{s.}{basquetebol}
  \end{phonetics}
\end{entry}

\begin{entry}{糕点}{16,9}
  \begin{phonetics}{糕点}{gao1dian3}
    \definition{s.}{bolos | pastéis}
  \end{phonetics}
\end{entry}

\begin{entry}{糕点师}{16,9,6}
  \begin{phonetics}{糕点师}{gao1dian3 shi1}
    \definition{s.}{confeiteiro}
  \end{phonetics}
\end{entry}

\begin{entry}{糕点店}{16,9,8}
  \begin{phonetics}{糕点店}{gao1dian3 dian4}
    \definition{s.}{confeitaria}
  \end{phonetics}
\end{entry}

\begin{entry}{糖}{16}[Radical 米]
  \begin{phonetics}{糖}{tang2}
    \definition[颗,块]{s.}{açúcar | doces}
  \end{phonetics}
\end{entry}

\begin{entry}{糖醋鱼}{16,15,8}
  \begin{phonetics}{糖醋鱼}{tang2cu4yu2}
    \definition{s.}{peixe guisado em molho agridoce (prato)}
  \end{phonetics}
\end{entry}

\begin{entry}{膨胀}{16,8}
  \begin{phonetics}{膨胀}{peng2zhang4}
    \definition{v.}{expandir | inflar | inchar}
  \end{phonetics}
\end{entry}

\begin{entry}{蕹菜}{16,11}
  \begin{phonetics}{蕹菜}{weng4cai4}
    \definition{s.}{espinafre aquático | \emph{ong choy} | repolho do pântano | convolvulus aquático | glória-da-manhã aquática}
  \seealsoref{空心菜}{kong1xin1cai4}
  \end{phonetics}
\end{entry}

\begin{entry}{薯}{16}[Radical 艸]
  \begin{phonetics}{薯}{shu3}
    \definition{s.}{batata | inhame}
  \end{phonetics}
\end{entry}

\begin{entry}{赞}{16}[Radical 貝]
  \begin{phonetics}{赞}{zan4}
    \definition{v.}{patrocinar | apoiar | elogiar | (gíria na \emph{Internet}) para curtir (uma postagem \emph{on-line})}
  \end{phonetics}
\end{entry}

\begin{entry}{赞扬}{16,6}
  \begin{phonetics}{赞扬}{zan4yang2}
    \definition{v.}{elogiar | aprovar | demonstrar aprovação}
  \end{phonetics}
\end{entry}

\begin{entry}{赞助}{16,7}
  \begin{phonetics}{赞助}{zan4zhu4}
    \definition{s.}{patrocinador}
    \definition{v.}{apoiar | auxiliar | patrocinar}
  \end{phonetics}
\end{entry}

\begin{entry}{避免}{16,7}
  \begin{phonetics}{避免}{bi4mian3}
    \definition{v.}{evitar | prevenir | abster-se de}
  \end{phonetics}
\end{entry}

\begin{entry}{镖}{16}[Radical 金]
  \begin{phonetics}{镖}{biao1}
    \definition{s.}{dardo | arma de arremesso | mercadorias enviadas sob a proteção de uma escolta armada}
  \end{phonetics}
\end{entry}

\begin{entry}{雕刻}{16,8}
  \begin{phonetics}{雕刻}{diao1ke4}
    \definition{s.}{escultura}
    \definition{v.}{esculpir | gravar}
  \end{phonetics}
\end{entry}

\begin{entry}{餐厅}{16,4}
  \begin{phonetics}{餐厅}{can1ting1}
    \definition[家]{s.}{restaurante}
    \definition[间]{s.}{sala de jantar}
  \end{phonetics}
\end{entry}

\begin{entry}{鲸鱼}{16,8}
  \begin{phonetics}{鲸鱼}{jing1yu2}
    \definition{s.}{baleia}
  \end{phonetics}
\end{entry}

\begin{entry}{鲸鲨}{16,15}
  \begin{phonetics}{鲸鲨}{jing1sha1}
    \definition{s.}{tubarão baleia}
  \end{phonetics}
\end{entry}

\begin{entry}{鹦鹉}{16,13}
  \begin{phonetics}{鹦鹉}{ying1wu3}
    \definition{s.}{papagaio (ave)}
  \end{phonetics}
\end{entry}

\begin{entry}{默契}{16,9}
  \begin{phonetics}{默契}{mo4qi4}
    \definition{adj.}{(de membros da equipe) bem coordenados}
    \definition{s.}{entendimento tácito | entendimento mútuo | conectado em um nível mútuo profundo | (de membros da equipe) bem coordenados}
  \end{phonetics}
\end{entry}

%%%%% EOF %%%%%

