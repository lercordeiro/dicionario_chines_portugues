%%%
%%% 18画
%%%

\section*{18画}\addcontentsline{toc}{section}{18画}

\begin{Entry}{嚣}{18}{⼝}
  \begin{Phonetics}{嚣}{xiao1}
    \definition*{s.}{Sobrenome Xiao}
    \definition{adj.}{lazer}
    \definition{v.}{clamar; fazer barulho}
  \end{Phonetics}
\end{Entry}

\begin{Entry}{嚣张}{18,7}{⼝、⼸}
  \begin{Phonetics}{嚣张}{xiao1zhang1}
    \definition{adj.}{desenfreado | arrogante | agressivo}
  \end{Phonetics}
\end{Entry}

\begin{Entry}{懵}{18}{⼼}
  \begin{Phonetics}{懵}{meng3}
    \definition{adj.}{confuso; ignorante; irracional | inconsciente; entorpecido}
  \end{Phonetics}
\end{Entry}

\begin{Entry}{懵懂}{18,15}{⼼、⼼}
  \begin{Phonetics}{懵懂}{meng3dong3}
    \definition{adj.}{confuso | ignorante}
  \end{Phonetics}
\end{Entry}

\begin{Entry}{戳}{18}{⼽}
  \begin{Phonetics}{戳}{chuo1}[][HSK 7-9]
    \definition{s.}{selo; carimbo, abreviação de 戳记}
    \definition{v.}{cutucar; esfaquear | Dialeto: torcer; embotar | Dialeto: ficar em pé}
  \seealsoref{戳记}{chuo1ji4}
  \end{Phonetics}
\end{Entry}

\begin{Entry}{戳记}{18,5}{⼽、⾔}
  \begin{Phonetics}{戳记}{chuo1ji4}
    \definition{s.}{carimbo; selo}
  \end{Phonetics}
\end{Entry}

\begin{Entry}{毉}{18}{⼖}
  \begin{Phonetics}{毉}{yi1}
    \variantof{医}
  \end{Phonetics}
\end{Entry}

\begin{Entry}{瀑}{18}{⽔}
  \begin{Phonetics}{瀑}{bao4}
    \definition{s.}{chuva torrencial; tempestade}
  \end{Phonetics}
  \begin{Phonetics}{瀑}{pu4}
    \definition{s.}{cachoeira; catarata}
  \end{Phonetics}
\end{Entry}

\begin{Entry}{瀑布}{18,5}{⽔、⼱}
  \begin{Phonetics}{瀑布}{pu4bu4}
    \definition{s.}{queda de água | cachoeira | cascata | catarata}
  \end{Phonetics}
\end{Entry}

\begin{Entry}{翻}{18}{⽻}
  \begin{Phonetics}{翻}{fan1}[][HSK 4]
    \definition{v.}{virar; dar a volta; inverter; mudar de posição; torcer; reverter | vasculhar; procurar; pesquisar; mover objetos para localizar algo | reverter; retrair; retirar | passar por cima; ultrapassar; cruzar | multiplicar | traduzir; decodificar | romper-se; cair; desentender-se com alguém}
  \end{Phonetics}
\end{Entry}

\begin{Entry}{翻天覆地}{18,4,18,6}{⽻、⼤、⾑、⼟}
  \begin{Phonetics}{翻天覆地}{fan1tian1-fu4di4}[][HSK 7-9]
    \definition{expr.}{virar o mundo de cabeça para baixo; uma mudança tremenda; abalar a terra; marcar época; virar o céu e a terra; sacudir o próprio chão (mundo); virar o mundo de cabeça para baixo; mudanças titânicas; ``Céu e terra virados de cabeça para baixo.''}
  \end{Phonetics}
\end{Entry}

\begin{Entry}{翻过}{18,6}{⽻、⾡}
  \begin{Phonetics}{翻过}{fan1guo4}
    \definition{v.}{virar |  transformar}
  \end{Phonetics}
\end{Entry}

\begin{Entry}{翻来覆去}{18,7,18,5}{⽻、⽊、⾑、⼛}
  \begin{Phonetics}{翻来覆去}{fan1lai2-fu4qu4}[][HSK 7-9]
    \definition{expr.}{jogar de um lado para o outro; tocar a mesma corda; repetir várias vezes; virar e se virar; dizer repetidamente; jogar inquieto de um lado para o outro; virar de um lado para o outro}
  \end{Phonetics}
\end{Entry}

\begin{Entry}{翻译}{18,7}{⽻、⾔}
  \begin{Phonetics}{翻译}{fan1yi4}[][HSK 4]
    \definition[个,位,名]{s.}{tradutor; intérprete; pessoas que fazem trabalhos de tradução}
    \definition{v.}{traduzir; interpretar; colocar o significado de palavras de um idioma em palavras de outro idioma (expressão idiomática); expressar um significado em outro idioma}
  \end{Phonetics}
\end{Entry}

\begin{Entry}{翻脸}{18,11}{⽻、⾁}
  \begin{Phonetics}{翻脸}{fan1/lian3}
    \definition{v.+compl.}{brigar com alguém | tornar-se hostil}
  \end{Phonetics}
\end{Entry}

\begin{Entry}{翻番}{18,12}{⽻、⽥}
  \begin{Phonetics}{翻番}{fan1/fan1}[][HSK 7-9]
    \definition{v.+compl.}{aumentar em um número especificado de vezes; dobrar}
  \end{Phonetics}
\end{Entry}

\begin{Entry}{覆}{18}{⾑}
  \begin{Phonetics}{覆}{fu4}
    \definition{v.}{cobrir; encapar | derrubar; perturbar; virar de cabeça para baixo}
  \end{Phonetics}
\end{Entry}

\begin{Entry}{覆盆子}{18,9,3}{⾑、⽫、⼦}
  \begin{Phonetics}{覆盆子}{fu4pen2zi5}
    \definition{s.}{framboesa}
  \end{Phonetics}
\end{Entry}

\begin{Entry}{覆盖}{18,11}{⾑、⽫}
  \begin{Phonetics}{覆盖}{fu4gai4}[][HSK 7-9]
    \definition{s.}{vegetação; cobertura vegetal; refere-se às plantas que cobrem o solo}
    \definition{v.}{cobrir}
  \end{Phonetics}
\end{Entry}

\begin{Entry}{蹦}{18}{⾜}
  \begin{Phonetics}{蹦}{beng4}[][HSK 7-9]
    \definition{v.}{pular; saltar; quicar}
  \end{Phonetics}
\end{Entry}

\begin{Entry}{蹦极}{18,7}{⾜、⽊}
  \begin{Phonetics}{蹦极}{beng4ji2}
    \definition{s.}{\emph{bungee jumping}}
  \end{Phonetics}
\end{Entry}

\begin{Entry}{鞭}{18}{⾰}
  \begin{Phonetics}{鞭}{bian1}
    \definition[条]{s.}{chicote; oçoite; chibata | um bastão de ferro usado como arma na China antiga | algo parecido com um chicote | uma série de pequenos fogos de artifício | pênis de animal; refere-se ao pênis de certos mamíferos usado para fins medicinais ou comestíveis}
    \definition{v.}{açoitar; chicotear; flagelar}
  \end{Phonetics}
\end{Entry}

\begin{Entry}{鞭炮}{18,9}{⾰、⽕}
  \begin{Phonetics}{鞭炮}{bian1pao4}[][HSK 7-9]
    \definition[串,挂,盒,捆,箱,个]{s.}{\emph{maroon}, um tipo de foguete usado como alarme ou aviso; fogos de artifício; um termo geral para fogos de artifício grandes e pequenos}
  \end{Phonetics}
\end{Entry}

\begin{Entry}{鞭策}{18,12}{⾰、⽵}
  \begin{Phonetics}{鞭策}{bian1ce4}[][HSK 7-9]
    \definition{v.}{estimular; incitar; incentivar}
  \end{Phonetics}
\end{Entry}

\begin{Entry}{鼂}{18}{⿌}
  \begin{Phonetics}{鼂}{chao2}
    \definition*{s.}{Sobrenome Chao}
    \definition{s.}{tartaruga marinha}
  \end{Phonetics}
\end{Entry}

%%%%% EOF %%%%%

