%%%
%%% 20画
%%%
\section*{20画}\addcontentsline{toc}{section}{20画}\addcontentsline{loh}{figure}{\#\#\#\# 20画}

%%%%%%%%%% 嚼 %%%%%%%%%%
\subsection*{嚼}\addcontentsline{loh}{figure}{嚼}

\begin{Entry}{嚼}{20}{⼝}
  \begin{Phonetics}{嚼}{jiao2}[][HSK 7-9]
    \definition{v.}{mastigar; mascar; limitado para uso em 过屠门而大嚼}
  \seealsoref{过屠门而大嚼}{guo4 tu2men2 er2 da4 jiao2}
  \end{Phonetics}
  \begin{Phonetics}{嚼}{jiao4}
    \definition{v.}{mascar; ruminar}
  \end{Phonetics}
  \begin{Phonetics}{嚼}{jue2}
    \definition{v.}{mastigar; morder; mastigar completamente; é usado em algumas palavras compostas e expressões idiomáticas; usado em 咀嚼}
  \seealsoref{咀嚼}{ju3jue2}
  \end{Phonetics}
\end{Entry}

%%%%%%%%%% 壤 %%%%%%%%%%
\subsection*{壤}\addcontentsline{loh}{figure}{壤}

\begin{Entry}{壤}{20}{⼟}
  \begin{Phonetics}{壤}{rang3}
    \definition{s.}{solo | terra | (literário) a terra (em contraste com o céu 天)}
  \end{Phonetics}
\end{Entry}

%%%%%%%%%% 灌 %%%%%%%%%%
\subsection*{灌}\addcontentsline{loh}{figure}{灌}

\begin{Entry}{灌}{20}{⽔}
  \begin{Phonetics}{灌}{guan4}[][HSK 7-9]
    \definition*{s.}{Sobrenome: Guan}
    \definition{s.}{arbusto; aglomerados de árvores baixas | irrigação}
    \definition{v.}{irrigar (rega e irrigação do solo) | encher; despejar; injetar | gravar; refere-se à gravação (música)}
  \end{Phonetics}
\end{Entry}

\begin{Entry}{灌溉}{20,12}{⽔、⽔}
  \begin{Phonetics}{灌溉}{guan4gai4}[][HSK 7-9]
    \definition{v.}{regar; irrigar}
  \end{Phonetics}
\end{Entry}

\begin{Entry}{灌输}{20,13}{⽔、⾞}
  \begin{Phonetics}{灌输}{guan4shu1}[][HSK 7-9]
    \definition{v.}{implantar; incutir em; inculcar; imbuir com (ideias, conhecimento); transmitir (ideias, conhecimento, etc.) | canalizar água; despejar água em; direcionar a água para onde ela é necessária}
  \end{Phonetics}
\end{Entry}

%%%%%%%%%% 譬 %%%%%%%%%%
\subsection*{譬}\addcontentsline{loh}{figure}{譬}

\begin{Entry}{譬}{20}{⾔}
  \begin{Phonetics}{譬}{pi4}
    \definition{s.}{exemplo; analogia; metáfora}
    \definition{v.}{dar um exemplo; fazer uma analogia}
  \end{Phonetics}
\end{Entry}

\begin{Entry}{譬如}{20,6}{⾔、⼥}
  \begin{Phonetics}{譬如}{pi4ru2}
    \definition{conj.}{por exemplo | como}
  \end{Phonetics}
\end{Entry}

%%%%%%%%%% 魔 %%%%%%%%%%
\subsection*{魔}\addcontentsline{loh}{figure}{魔}

\begin{Entry}{魔}{20}{⿁}
  \begin{Phonetics}{魔}{mo2}
    \definition{adj.}{místico; misterioso; mágico}
    \definition{s.}{espírito maligno; demônio; diabo; monstro | mágico; místico}
  \end{Phonetics}
\end{Entry}

\begin{Entry}{魔头}{20,5}{⿁、⼤}
  \begin{Phonetics}{魔头}{mo2tou2}
    \definition[个,些]{s.}{diabo; demônio; monstro; espírito maligno}
  \end{Phonetics}
\end{Entry}

\begin{Entry}{魔术}{20,5}{⿁、⽊}
  \begin{Phonetics}{魔术}{mo2shu4}[][HSK 7-9]
    \definition[个,场]{s.}{magia; ilusionismo; prestidigitação; truques; utilizando princípios físicos e químicos ou dispositivos especiais, os objetos podem aparecer, desaparecer ou sofrer mudanças maravilhosas de maneira sutil e imperceptível}
  \end{Phonetics}
\end{Entry}

\begin{Entry}{魔鬼}{20,9}{⿁、⿁}
  \begin{Phonetics}{魔鬼}{mo2gui3}[][HSK 7-9]
    \definition[个,些,群]{s.}{diabo; demônio; monstro; na religião ou mitologia, refere-se a fantasmas ou monstros malignos; metaforicamente, também pode se referir a pessoas perversas que cometem atos malignos}
  \end{Phonetics}
\end{Entry}

%%%%% EOF %%%%%

