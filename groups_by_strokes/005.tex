%%%
%%% 5画
%%%
\section*{5画}\addcontentsline{toc}{section}{5画}

%%%%%%%%%% 且 %%%%%%%%%%
\subsection*{且}

\begin{Entry}{且}{5}{⼀}
  \begin{Phonetics}{且}{qie3}
    \definition*{s.}{Sobrenome: Qie}
    \definition{adv.}{apenas; por enquanto | por um longo tempo}
    \definition{conj.}{mesmo; até; até mesmo; usado na primeira cláusula de uma frase complexa para expressar concessão, equivalente a 尚且 | ambos\dots e\dots; conecta adjetivos ou verbos para expressar relacionamento paralelo, equivalente a 而且 e 又……又……}
  \seealsoref{而且}{er2 qie3}
  \seealsoref{尚且}{shang4 qie3}
  \seealsoref{又……又……}{you4 you4}
  \end{Phonetics}
\end{Entry}

%%%%%%%%%% 世 %%%%%%%%%%
\subsection*{世}

\begin{Entry}{世}{5}{⼀}
  \begin{Phonetics}{世}{shi4}
    \definition*{s.}{Sobrenome: Shi}
    \definition{s.}{vida; tempo de vida; vida humana | geração; geração após geração | idade; era | o mundo; sociedade | (geologia) época, abaixo de ``período''}
  \end{Phonetics}
\end{Entry}

\begin{Entry}{世代}{5,5}{⼀、⼈}
  \begin{Phonetics}{世代}{shi4dai4}
    \definition{adv.}{por muitas gerações, eras}
    \definition{s.}{geração | era}
  \end{Phonetics}
\end{Entry}

\begin{Entry}{世纪}{5,6}{⼀、⽷}
  \begin{Phonetics}{世纪}{shi4ji4}[][HSK 3]
    \definition[个,段]{s.}{século; uma unidade para calcular anos, cem anos é um século}
  \end{Phonetics}
\end{Entry}

\begin{Entry}{世界}{5,9}{⼀、⽥}
  \begin{Phonetics}{世界}{shi4jie4}[][HSK 3]
    \definition[个,片,种]{s.}{mundo; todos os lugares da Terra | a soma da natureza e da sociedade humana; refere-se à soma de toda a existência objetiva na natureza e na sociedade humana | campo; refere-se a uma determinada área ou campo | o universo sem limites; costumava ser um termo budista, mas agora também se refere ao mundo natural ilimitado e à sociedade humana | situação social; a situação ou atmosfera social de um determinado período}
  \end{Phonetics}
\end{Entry}

\begin{Entry}{世界杯}{5,9,8}{⼀、⽥、⽊}
  \begin{Phonetics}{世界杯}{shi4jie4bei1}[][HSK 3]
    \definition*{s.}{Copa do Mundo; Troféu da Copa do Mundo}
  \end{Phonetics}
\end{Entry}

\begin{Entry}{世锦赛}{5,13,14}{⼀、⾦、⾙}
  \begin{Phonetics}{世锦赛}{shi4jin3sai4}
    \definition*{s.}{Campeonato Mundial}
  \end{Phonetics}
\end{Entry}

%%%%%%%%%% 丘 %%%%%%%%%%
\subsection*{丘}

\begin{Entry}{丘}{5}{⼀}
  \begin{Phonetics}{丘}{qiu1}
    \definition*{s.}{Sobrenome: Qiu}
    \definition[个]{s.}{monte; outeiro | (literário) sepultura}
  \end{Phonetics}
\end{Entry}

\begin{Entry}{丘陵}{5,10}{⼀、⾩}
  \begin{Phonetics}{丘陵}{qiu1ling2}
    \definition{s.}{colinas}
  \end{Phonetics}
\end{Entry}

%%%%%%%%%% 丙 %%%%%%%%%%
\subsection*{丙}

\begin{Entry}{丙}{5}{⼀}
  \begin{Phonetics}{丙}{bing3}[][HSK 7-9]
    \definition*{s.}{o terceiro dos Dez Troncos Celestiais}
    \definition{s.}{terceiro | Literário: fogo}
  \end{Phonetics}
\end{Entry}

%%%%%%%%%% 业 %%%%%%%%%%
\subsection*{业}

\begin{Entry}{业}{5}{⼀}
  \begin{Phonetics}{业}{ye4}
    \definition*{s.}{Sobrenome Ye}
    \definition{adv.}{já; indica que a ação foi concluída, equivalente a 已经}
    \definition{s.}{comércio; indústria; ramo de negócios | emprego; ocupação; profissão | curso de estudo | causa; empreendimento | propriedade | carma; o budismo se refere a todas as ações, palavras e pensamentos humanos como carma, que são chamados de carma corporal, carma da fala e carma mental; o carma inclui aspectos bons e ruins, geralmente referindo-se ao destino ou ao pecado}
    \definition{v.}{envolver-se em; exercer uma determinada profissão}
  \seealsoref{已经}{yi3jing1}
  \end{Phonetics}
\end{Entry}

\begin{Entry}{业务}{5,5}{⼀、⼒}
  \begin{Phonetics}{业务}{ye4wu4}[][HSK 5]
    \definition[项,笔,个,类,种]{s.}{negócios; trabalho vocacional; trabalho profissional}
  \end{Phonetics}
\end{Entry}

\begin{Entry}{业余}{5,7}{⼀、⼈}
  \begin{Phonetics}{业余}{ye4yu2}[][HSK 4]
    \definition{adj.}{tempo livre; depois do expediente; fora do horário de trabalho | amador; não profissional}
  \end{Phonetics}
\end{Entry}

%%%%%%%%%% 丛 %%%%%%%%%%
\subsection*{丛}

\begin{Entry}{丛}{5}{⼀}
  \begin{Phonetics}{丛}{cong2}
    \definition*{s.}{Sobrenome: Cong}
    \definition{s.}{aglomerado; matagal; bosque | coleção; multidão}
  \end{Phonetics}
\end{Entry}

\begin{Entry}{丛林}{5,8}{⼀、⽊}
  \begin{Phonetics}{丛林}{cong2lin2}[][HSK 7-9]
    \definition[片]{s.}{selva; floresta | mosteiro budista}
  \end{Phonetics}
\end{Entry}

%%%%%%%%%% 东 %%%%%%%%%%
\subsection*{东}

\begin{Entry}{东}{5}{⼀}
  \begin{Phonetics}{东}{dong1}[][HSK 1]
    \definition*{s.}{Sobrenome: Dong}
    \definition{s.}{leste; uma das quatro direções básicas, o lado onde o sol nasce | proprietário; dono | anfitrião (antigamente, o anfitrião ficava a leste e os convidados a oeste)}
  \end{Phonetics}
\end{Entry}

\begin{Entry}{东方}{5,4}{⼀、⽅}
  \begin{Phonetics}{东方}{dong1 fang1}[][HSK 2]
    \definition*{s.}{Sobrenome: Dongfang}
    \definition{s.}{leste | oriente; o leste; o Oriente}
  \end{Phonetics}
\end{Entry}

\begin{Entry}{东方学院}{5,4,8,9}{⼀、⽅、⼦、⾩}
  \begin{Phonetics}{东方学院}{dong1fang1 xue2yuan4}
    \definition*{s.}{Instituto Oriental}
  \end{Phonetics}
\end{Entry}

\begin{Entry}{东北}{5,5}{⼀、⼔}
  \begin{Phonetics}{东北}{dong1 bei3}[][HSK 2]
    \definition*{s.}{Nordeste da China; O Nordeste | Manchúria}
    \definition{s.}{nordeste}
  \end{Phonetics}
\end{Entry}

\begin{Entry}{东半球}{5,5,11}{⼀、⼗、⽟}
  \begin{Phonetics}{东半球}{dong1ban4qiu2}
    \definition*{s.}{Hemisfério Oriental}
  \end{Phonetics}
\end{Entry}

\begin{Entry}{东边}{5,5}{⼀、⾡}
  \begin{Phonetics}{东边}{dong1 bian5}[][HSK 1]
    \definition{s.}{leste; o lado leste; refere-se à fronteira oriental}
  \end{Phonetics}
\end{Entry}

\begin{Entry}{东西}{5,6}{⼀、⾑}
  \begin{Phonetics}{东西}{dong1xi1}
    \definition{s.}{leste e oeste | de leste a oeste; a distância de um lugar de leste a oeste}
  \end{Phonetics}
  \begin{Phonetics}{东西}{dong1xi5}[][HSK 1]
    \definition[个,件]{s.}{coisa; refere-se a todos os tipos de coisas concretas ou abstratas | coisa; criatura; refere-se especificamente a pessoas ou coisas que causam repulsa ou simpatia}
  \end{Phonetics}
\end{Entry}

\begin{Entry}{东张西望}{5,7,6,11}{⼀、⼸、⾑、⽉}
  \begin{Phonetics}{东张西望}{dong1zhang1-xi1wang4}[][HSK 7-9]
    \definition{expr.}{olhar (ou espreitar) ao redor; olhar para um lado e para o outro; olhar em todas as direções | olhar (olhar) para um lado e para o outro; olhar (espreitar) ao redor; olhar para a direita e para a esquerda; olhar para todos os lados; olhar para leste e oeste; olhar em todas as direções | olhar (espiar) ao redor}
  \end{Phonetics}
\end{Entry}

\begin{Entry}{东奔西走}{5,8,6,7}{⼀、⼤、⾑、⾛}
  \begin{Phonetics}{东奔西走}{dong1ben1-xi1zou3}[][HSK 7-9]
    \definition{expr.}{correr de um lado para o outro; correr atarefadamente; ir em todas as direções em busca de algo; correr ou se apressar aqui e ali (procurando emprego, ganhando a vida, tentando a sorte em algo ou outro, etc.)}
  \end{Phonetics}
\end{Entry}

\begin{Entry}{东南}{5,9}{⼀、⼗}
  \begin{Phonetics}{东南}{dong1 nan2}[][HSK 2]
    \definition*{s.}{Sudeste da China; O Sudeste; refere-se à região costeira sudeste da China, incluindo as províncias e cidades de Xangai, Jiangsu, Zhejiang, Fujian, Taiwan, etc.}
    \definition{s.}{sudeste}
  \end{Phonetics}
\end{Entry}

\begin{Entry}{东面}{5,9}{⼀、⾯}
  \begin{Phonetics}{东面}{dong1mian4}
    \definition{s.}{lado leste (de algo)}
  \end{Phonetics}
\end{Entry}

\begin{Entry}{东部}{5,10}{⼀、⾢}
  \begin{Phonetics}{东部}{dong1 bu4}[][HSK 3]
    \definition{s.}{o leste; parte oriental; a parte oriental de uma determinada região}
  \end{Phonetics}
\end{Entry}

\begin{Entry}{东道主}{5,12,5}{⼀、⾡、⼂}
  \begin{Phonetics}{东道主}{dong1dao4zhu3}[][HSK 7-9]
    \definition[个,些,位]{s.}{anfitrião; aquele que paga por uma refeição; o anfitrião do banquete}
  \end{Phonetics}
\end{Entry}

%%%%%%%%%% 丝 %%%%%%%%%%
\subsection*{丝}

\begin{Entry}{丝}{5}{⼀}
  \begin{Phonetics}{丝}{si1}
    \definition{clas.}{si, uma unidade de peso (=0,0005 gramas) | usado para expressar a aparência ou expressão de uma pessoa | um décimo de milésimo de certas unidades de medida (medida de comprimento) | usado para representar coisas abstratas}
    \definition[些,种,类,跟,缕]{s.}{seda | uma coisa semelhante a um fio; itens semelhantes à seda | cordas; instrumentos de corda}
  \end{Phonetics}
\end{Entry}

%%%%%%%%%% 主 %%%%%%%%%%
\subsection*{主}

\begin{Entry}{主}{5}{⼂}
  \begin{Phonetics}{主}{zhu3}
    \definition*{s.}{Deus; Senhor; o nome do Deus em que se acredita o cristianismo, o judaísmo, etc.}
    \definition{adj.}{principal; primário; o mais básico; o mais importante | de si mesmo; por vontade própria; próprio; do próprio}
    \definition[位,名,个]{s.}{anfitrião; alguém que convida e recebe convidados (oposto de 宾 e 客) | mestre; dono; uma pessoa que possui poder ou propriedade; uma pessoa em posição dominante | pessoa ou parte interessada | decisão; opinião; visão definitiva | placa espiritual (ou memorial)}
    \definition{v.}{dirigir; administrar; assumir o comando de; presidir; assumir a responsabilidade primária | decidir; reivindicar | significar; indicar; prever um certo resultado}
  \seealsoref{宾}{bin1}
  \seealsoref{客}{ke4}
  \end{Phonetics}
\end{Entry}

\begin{Entry}{主人}{5,2}{⼂、⼈}
  \begin{Phonetics}{主人}{zhu3ren2}[][HSK 2]
    \definition[个,位]{s.}{mestre; uma pessoa que empregava tutores, contadores, etc. antigamente; uma pessoa que empregava empregados domésticos | anfitrião; Aaguém que entretém convidados (em oposição a 客人) | proprietário; uma pessoa que possui um certo tipo de bens ou poder}
  \seealsoref{客人}{ke4ren2}
  \end{Phonetics}
\end{Entry}

\begin{Entry}{主义}{5,3}{⼂、⼂}
  \begin{Phonetics}{主义}{zhu3yi4}
    \definition[种]{s.}{doutrina; um determinado sistema social ou sistema político e econômico | estilo de pensamento; um certo ponto de vista ou estilo | ideologia; teorias e doutrinas sistemáticas sobre a natureza, a sociedade humana, etc.}
    \definition{suf.}{-ismo}
  \end{Phonetics}
\end{Entry}

\begin{Entry}{主办}{5,4}{⼂、⼒}
  \begin{Phonetics}{主办}{zhu3ban4}[][HSK 5]
    \definition{v.}{manter; hospedar; dirigir; patrocinar}
  \end{Phonetics}
\end{Entry}

\begin{Entry}{主任}{5,6}{⼂、⼈}
  \begin{Phonetics}{主任}{zhu3ren4}[][HSK 3]
    \definition[个,位,名]{s.}{chefe; diretor; presidente; o principal responsável por um departamento ou instituição}
  \end{Phonetics}
\end{Entry}

\begin{Entry}{主动}{5,6}{⼂、⼒}
  \begin{Phonetics}{主动}{zhu3dong4}[][HSK 3]
    \definition{adj.}{ativo; positivo; agir sem esperar por um impulso externo (em oposição a 被动) | iniciativo; capaz de impulsionar as coisas por vontade própria; capaz de criar uma situação favorável e fazer as coisas acontecerem de acordo com suas próprias intenções (em oposição a 被动)}
  \seealsoref{被动}{bei4dong4}
  \end{Phonetics}
\end{Entry}

\begin{Entry}{主导}{5,6}{⼂、⼨}
  \begin{Phonetics}{主导}{zhu3dao3}[][HSK 5]
    \definition{adj.}{líder; dominante; guiado; principais e guias para que as coisas se desenvolvam em uma determinada direção}
    \definition{s.}{fator principal (ou orientador)}
  \end{Phonetics}
\end{Entry}

\begin{Entry}{主观}{5,6}{⼂、⾒}
  \begin{Phonetics}{主观}{zhu3guan1}[][HSK 5]
    \definition{adj.}{subjetivo; não com base nas condições reais, mas com base nos próprios desejos | subjetivo; filosoficamente, refere-se à consciência e aos aspectos espirituais dos seres humanos}
  \end{Phonetics}
\end{Entry}

\begin{Entry}{主体}{5,7}{⼂、⼈}
  \begin{Phonetics}{主体}{zhu3 ti3}[][HSK 5]
    \definition[个,些,种,群]{s.}{corpo principal; parte principal; parte principal; esteio; a parte principal das coisas | Filosofia: sujeito}
  \end{Phonetics}
\end{Entry}

\begin{Entry}{主张}{5,7}{⼂、⼸}
  \begin{Phonetics}{主张}{zhu3zhang1}[][HSK 3]
    \definition[个,项,些,种]{s.}{vista; posição; proposição}
    \definition{v.}{defender; apoiar; manter; representar; ter uma opinião sobre como agir, fazer uma sugestão}
  \end{Phonetics}
\end{Entry}

\begin{Entry}{主角}{5,7}{⼂、⾓}
  \begin{Phonetics}{主角}{zhu3 jue2}[][HSK 6]
    \definition[个,位,名]{s.}{liderança; papel principal; protagonista; um papel importante em uma peça, filme, etc.; um ator que desempenha um papel importante | (figurado) algo que tem grande influência em uma determinada área; refere-se ao personagem principal}
  \end{Phonetics}
\end{Entry}

\begin{Entry}{主持}{5,9}{⼂、⼿}
  \begin{Phonetics}{主持}{zhu3chi2}[][HSK 3]
    \definition[位,名]{s.}{anfitrião; a pessoa responsável por administrar e lidar com uma determinada atividade}
    \definition{v.}{dirigir; administrar; assumir o comando; encarregar-se de; ser responsável por gerenciar, organizar uma determinada atividade ou lidar com um determinado assunto | defender; apoiar; preservar; manter}
  \end{Phonetics}
\end{Entry}

\begin{Entry}{主持人}{5,9,2}{⼂、⼿、⼈}
  \begin{Phonetics}{主持人}{zhu3 chi2 ren2}[][HSK 6]
    \definition[个,位]{s.}{anfitrião; âncora; apresentador}
  \end{Phonetics}
\end{Entry}

\begin{Entry}{主要}{5,9}{⼂、⾑}
  \begin{Phonetics}{主要}{zhu3yao4}[][HSK 2]
    \definition{adj.}{principal; chefe; o mais importante na questão; o decisivo | principal; núcleo; a raiz ou parte mais importante de algo}
  \end{Phonetics}
\end{Entry}

\begin{Entry}{主席}{5,10}{⼂、⼱}
  \begin{Phonetics}{主席}{zhu3xi2}[][HSK 4]
    \definition*{s.}{Presidente (da China)}
    \definition[个,位,名]{s.}{presidente, \emph{chairman} (de uma reunião) | chefe; presidente (de uma organização ou estado)}
  \end{Phonetics}
\end{Entry}

\begin{Entry}{主席台}{5,10,5}{⼂、⼱、⼝}
  \begin{Phonetics}{主席台}{zhu3xi2tai2}
    \definition[个]{s.}{plataforma | tribuna}
  \end{Phonetics}
\end{Entry}

\begin{Entry}{主席团}{5,10,6}{⼂、⼱、⼞}
  \begin{Phonetics}{主席团}{zhu3xi2tuan2}
    \definition{s.}{presídio}
  \end{Phonetics}
\end{Entry}

\begin{Entry}{主流}{5,10}{⼂、⽔}
  \begin{Phonetics}{主流}{zhu3liu2}[][HSK 6]
    \definition{s.}{corrente principal; corrente mãe; convencional | tendência principal; aspecto essencial ou principal; falando metaforicamente, os principais aspectos do desenvolvimento das coisas}
  \end{Phonetics}
\end{Entry}

\begin{Entry}{主意}{5,13}{⼂、⼼}
  \begin{Phonetics}{主意}{zhu3yi5}[][HSK 3]
    \definition[个,种]{s.}{ideia; plano; decisão; método}
  \end{Phonetics}
\end{Entry}

\begin{Entry}{主管}{5,14}{⼂、⽵}
  \begin{Phonetics}{主管}{zhu3guan3}[][HSK 5]
    \definition[位,名,个,些]{s.}{pessoa responsável, como supervisor, gerente, diretor, etc.}
    \definition{v.}{estar encarregado de; ser responsável por; ser o principal responsável pela gestão de um trabalho; assumir a responsabilidade primária pela gestão (um certo aspecto)}
  \end{Phonetics}
\end{Entry}

\begin{Entry}{主题}{5,15}{⼂、⾴}
  \begin{Phonetics}{主题}{zhu3ti2}[][HSK 4]
    \definition[个]{s.}{tema; assunto; motivo; lema; ideias básicas expressas em toda a obra de literatura e arte por meio de imagens artísticas concretas | pontos/conteúdos principais; referência geral ao conteúdo principal de artigos, discursos, conferências, etc.}
  \end{Phonetics}
\end{Entry}

%%%%%%%%%% 乐 %%%%%%%%%%
\subsection*{乐}

\begin{Entry}{乐}{5}{⼃}
  \begin{Phonetics}{乐}{le4}[][HSK 3]
    \definition*{s.}{Sobrenome: Le}
    \definition{adj.}{feliz; contente; rejubilante; animado; bem disposto}
    \definition{s.}{prazer; diversão; felicidade}
    \definition{v.}{desfrutar; ficar feliz em; amar; encontrar prazer em | rir; divertir-se}
  \end{Phonetics}
  \begin{Phonetics}{乐}{yue4}
    \definition*{s.}{Sobrenome Yue}
    \definition{s.}{música}
  \end{Phonetics}
\end{Entry}

\begin{Entry}{乐队}{5,4}{⼃、⾩}
  \begin{Phonetics}{乐队}{yue4 dui4}[][HSK 3]
    \definition[支,个]{s.}{orquestra; banda; um grupo composto por muitas pessoas que tocam diferentes instrumentos musicais}
  \end{Phonetics}
\end{Entry}

\begin{Entry}{乐曲}{5,6}{⼃、⽈}
  \begin{Phonetics}{乐曲}{yue4 qu3}[][HSK 6]
    \definition[支,首,段]{s.}{música; composição musical}
  \end{Phonetics}
\end{Entry}

\begin{Entry}{乐观}{5,6}{⼃、⾒}
  \begin{Phonetics}{乐观}{le4guan1}[][HSK 3]
    \definition{adj.}{esperançoso; otimista; confiante; espírito alegre, confiante no futuro (oposto a 悲观)}
  \seealsoref{悲观}{bei1guan1}
  \end{Phonetics}
\end{Entry}

\begin{Entry}{乐园}{5,7}{⼃、⼞}
  \begin{Phonetics}{乐园}{le4yuan2}[][HSK 7-9]
    \definition[座,个,家]{s.}{parque infantil; parque de diversões; um parque ou jardim para as pessoas desfrutarem; geralmente se refere a um lugar que traz felicidade | paraíso; no cristianismo, refere-se ao Céu ou ao Jardim do Éden.}
  \end{Phonetics}
\end{Entry}

\begin{Entry}{乐高}{5,10}{⼃、⾼}
  \begin{Phonetics}{乐高}{le4gao1}
    \definition*{s.}{Lego (brinquedo)}
  \end{Phonetics}
\end{Entry}

\begin{Entry}{乐意}{5,13}{⼃、⼼}
  \begin{Phonetics}{乐意}{le4yi4}[][HSK 7-9]
    \definition{adj.}{contente; satisfeito; feliz}
    \definition{v.}{estar disposto; estar pronto para}
  \end{Phonetics}
\end{Entry}

\begin{Entry}{乐趣}{5,15}{⼃、⾛}
  \begin{Phonetics}{乐趣}{le4qu4}[][HSK 4]
    \definition[个,种,些,点]{s.}{alegria; deleite; prazer; implicação de fazer alguém se sentir feliz; um humor de preferência}
  \end{Phonetics}
\end{Entry}

%%%%%%%%%% 仔 %%%%%%%%%%
\subsection*{仔}

\begin{Entry}{仔}{5}{⼈}
  \begin{Phonetics}{仔}{zi1}
    \definition{s./v.}{usado em 仔肩}
  \seealsoref{仔肩}{zi1jian1}
  \end{Phonetics}
  \begin{Phonetics}{仔}{zi3}
    \definition{adj.}{jovem | cuidadoso; pequeno; fino}
  \end{Phonetics}
\end{Entry}

\begin{Entry}{仔细}{5,8}{⼈、⽷}
  \begin{Phonetics}{仔细}{zi3xi4}[][HSK 5]
    \definition{adj.}{cuidadoso; atencioso; descreve alguém que é cuidadoso e meticuloso ao fazer as coisas; não é descuidado | frugal; econômico; descreve o uso moderado de dinheiro ou bens, sem desperdício}
    \definition{v.}{ter cuidado; prestar atenção; ter muito cuidado e evitar que aconteçam coisas ruins}
  \end{Phonetics}
\end{Entry}

\begin{Entry}{仔肩}{5,8}{⼈、⾁}
  \begin{Phonetics}{仔肩}{zi1jian1}
    \definition{s.}{encargos oficiais (ou responsabilidades)}
    \definition{v.}{assumir a responsabilidade por algo}
  \end{Phonetics}
\end{Entry}

%%%%%%%%%% 他 %%%%%%%%%%
\subsection*{他}

\begin{Entry}{他}{5}{⼈}
  \begin{Phonetics}{他}{ta1}[][HSK 1]
    \definition{pron.}{ele | outro; referindo-se a outro; diferente | usado após o verbo, indica referência vaga | alguém; todos; usado em conjunto com 你, significa qualquer pessoa ou muitas pessoas | em outro lugar; outro lugar}
  \seealsoref{你}{ni3}
  \seealsoref{怹}{tan1}
  \end{Phonetics}
\end{Entry}

\begin{Entry}{他们}{5,5}{⼈、⼈}
  \begin{Phonetics}{他们}{ta1men5}[][HSK 1]
    \definition{pron.}{eles}
  \end{Phonetics}
\end{Entry}

\begin{Entry}{他们的}{5,5,8}{⼈、⼈、⽩}
  \begin{Phonetics}{他们的}{ta1men5 de5}
    \definition{pron.}{deles}
  \end{Phonetics}
\end{Entry}

\begin{Entry}{他妈的}{5,6,8}{⼈、⼥、⽩}
  \begin{Phonetics}{他妈的}{ta1ma1de5}
    \definition{interj.}{Dane-se! | Foda-se!}
  \end{Phonetics}
\end{Entry}

\begin{Entry}{他的}{5,8}{⼈、⽩}
  \begin{Phonetics}{他的}{ta1 de5}
    \definition{pron.}{dele}
  \end{Phonetics}
\end{Entry}

%%%%%%%%%% 付 %%%%%%%%%%
\subsection*{付}

\begin{Entry}{付}{5}{⼈}
  \begin{Phonetics}{付}{fu4}[][HSK 3]
    \definition*{s.}{Sobrenome: Fu}
    \definition{clas.}{usado para pares ou conjuntos de coisas | usado para expressões faciais}
    \definition{v.}{comprometer-se com; entregar (ou transferir) para | pagar; refere-se especificamente a dar dinheiro}
  \end{Phonetics}
\end{Entry}

\begin{Entry}{付出}{5,5}{⼈、⼐}
  \begin{Phonetics}{付出}{fu4 chu1}[][HSK 4]
    \definition{v.}{pagar; gastar; entregar (dinheiro, consideração, etc.)}
  \end{Phonetics}
\end{Entry}

\begin{Entry}{付费}{5,9}{⼈、⾙}
  \begin{Phonetics}{付费}{fu4fei4}[][HSK 7-9]
    \definition{v.}{pagar}
  \end{Phonetics}
\end{Entry}

\begin{Entry}{付款}{5,12}{⼈、⽋}
  \begin{Phonetics}{付款}{fu4kuan3}[][HSK 7-9]
    \definition[次]{s.}{pagamento}
    \definition{v.}{pagar uma quantia em dinheiro}
  \end{Phonetics}
\end{Entry}

%%%%%%%%%% 仙 %%%%%%%%%%
\subsection*{仙}

\begin{Entry}{仙}{5}{⼈}
  \begin{Phonetics}{仙}{xian1}
    \definition{s.}{imortal}
  \end{Phonetics}
\end{Entry}

%%%%%%%%%% 代 %%%%%%%%%%
\subsection*{代}

\begin{Entry}{代}{5}{⼈}
  \begin{Phonetics}{代}{dai4}[][HSK 3]
    \definition*{s.}{Sobrenome: Dai}
    \definition{s.}{dinastia | geração; hierarquia familiar | era; o segundo nível da divisão geológica é o período, acima do qual está a era e abaixo do qual está o período, por exemplo, o Paleozóico, o Mesozóico e o Cenozóico pertencem à era Phanerozoico | período histórico; época}
    \definition{v.}{tomar o lugar de; estar no lugar de | agir em nome de; exercer}
  \end{Phonetics}
\end{Entry}

\begin{Entry}{代号}{5,5}{⼈、⼝}
  \begin{Phonetics}{代号}{dai4hao4}[][HSK 7-9]
    \definition[个]{s.}{codinome | designação; marca; código}
  \end{Phonetics}
\end{Entry}

\begin{Entry}{代价}{5,6}{⼈、⼈}
  \begin{Phonetics}{代价}{dai4jia4}[][HSK 5]
    \definition[种,个]{s.}{preço; material, energia gasta ou sacrifícios feitos para atingir um objetivo | custo; preço; dinheiro pago para obter algo}
  \end{Phonetics}
\end{Entry}

\begin{Entry}{代言}{5,7}{⼈、⾔}
  \begin{Phonetics}{代言}{dai4yan2}
    \definition{v.}{ser um porta-voz | ser um embaixador (para uma marca) | endossar}
  \end{Phonetics}
\end{Entry}

\begin{Entry}{代言人}{5,7,2}{⼈、⾔、⼈}
  \begin{Phonetics}{代言人}{dai4yan2ren2}[][HSK 7-9]
    \definition{s.}{porta-voz; uma pessoa que fala em nome de um determinado partido (classe, grupo, etc.)}
  \end{Phonetics}
\end{Entry}

\begin{Entry}{代表}{5,8}{⼈、⾐}
  \begin{Phonetics}{代表}{dai4biao3}[][HSK 3]
    \definition[位,名,个,些]{s.}{deputado; delegado; representante; pessoas eleitas para representar eleitores ou expressar opiniões, ou pessoas encarregadas ou designadas para representar indivíduos, grupos ou governos ou expressar opiniões | representante oficial; pessoas ou coisas que refletem as características comuns de um grupo específico}
    \definition{v.}{representar; defender | usar pessoas ou coisas para expressar um significado ou conceito específico}
  \end{Phonetics}
\end{Entry}

\begin{Entry}{代表团}{5,8,6}{⼈、⾐、⼞}
  \begin{Phonetics}{代表团}{dai4 biao3 tuan2}[][HSK 3]
    \definition[个]{s.}{delegação; contingente; um grupo temporário de grande dimensão formado para participar de uma determinada atividade em nome de um país, governo ou outra organização social}
  \end{Phonetics}
\end{Entry}

\begin{Entry}{代称}{5,10}{⼈、⽲}
  \begin{Phonetics}{代称}{dai4cheng1}
    \definition{s.}{nome alternativo | antonomásia}
    \definition{v.}{referir-se a algo ou alguém por outro nome}
  \end{Phonetics}
\end{Entry}

\begin{Entry}{代理}{5,11}{⼈、⽟}
  \begin{Phonetics}{代理}{dai4li3}[][HSK 5]
    \definition{v.}{agir em nome de alguém em uma posição de responsabilidade; substituir alguém | agir como procurador; agir como agente; ser encarregado pelas partes de realizar atividades e conduzir assuntos em seu nome dentro do escopo de sua autorização}
  \end{Phonetics}
\end{Entry}

\begin{Entry}{代理人}{5,11,2}{⼈、⽟、⼈}
  \begin{Phonetics}{代理人}{dai4li3ren2}[][HSK 7-9]
    \definition{s.}{agente; procurador; representante; uma pessoa encarregada de agir em nome de uma parte | advogado; procurador; refere-se a profissionais que possuem qualificações em prática jurídica e são encarregados de fornecer serviços jurídicos aos clientes}
  \end{Phonetics}
\end{Entry}

\begin{Entry}{代替}{5,12}{⼈、⽈}
  \begin{Phonetics}{代替}{dai4ti4}[][HSK 4]
    \definition{v.}{substituir; substituir por; tomar o lugar de}
  \end{Phonetics}
\end{Entry}

%%%%%%%%%% 令 %%%%%%%%%%
\subsection*{令}

\begin{Entry}{令}{5}{⼈}
  \begin{Phonetics}{令}{ling2}
    \definition*{s.}{Antigo nome geográfico, na região atual de Linyi, província de Shanxi | Sobrenome: Ling}
  \end{Phonetics}
  \begin{Phonetics}{令}{ling3}
    \definition{clas.}{resma (de papel); unidade de medida de papel: 500 folhas inteiras de papel original produzidas mecanicamente equivalem a 1 resma}
  \end{Phonetics}
  \begin{Phonetics}{令}{ling4}[][HSK 5]
    \definition{adj.}{bom; excelente | termos de cortesia usados para se referir aos familiares e parentes da outra pessoa}
    \definition{s.}{ordem; decreto; comando; ordem emitida pela autoridade superior | um título oficial; administradores de certos departamentos governamentais na antiguidade | temporada; estação; clima e fenologia de uma determinada estação | poema-canção; letra curta}
    \definition{v.}{ordenar; comandar | fazer com que alguém; fazer com que; permitir que}
  \end{Phonetics}
\end{Entry}

\begin{Entry}{令人}{5,2}{⼈、⼈}
  \begin{Phonetics}{令人}{ling4ren2}
    \definition{v.}{causar alguém (a fazer alguma coisa) | fazer alguém ficar zangado, encantado, etc.}
  \end{Phonetics}
\end{Entry}

%%%%%%%%%% 仪 %%%%%%%%%%
\subsection*{仪}

\begin{Entry}{仪}{5}{⼈}
  \begin{Phonetics}{仪}{yi2}
    \definition*{s.}{Sobrenome Yi}
    \definition{s.}{aparência; porte | cerimônia; rito | presente; dádiva | aparelho; instrumento}
    \definition{v.}{(literário) admirar; ansiar por}
  \end{Phonetics}
\end{Entry}

\begin{Entry}{仪式}{5,6}{⼈、⼷}
  \begin{Phonetics}{仪式}{yi2shi4}[][HSK 6]
    \definition{s.}{rito; cerimônia; procedimento e formato da cerimônia}
  \end{Phonetics}
\end{Entry}

\begin{Entry}{仪器}{5,16}{⼈、⼝}
  \begin{Phonetics}{仪器}{yi2qi4}[][HSK 6]
    \definition[台]{s.}{aparelho; instrumento; ferramentas ou equipamentos utilizados para observação, medição, inspeção, etc. em pesquisas ou experimentos científicos; geralmente, são relativamente precisos e padronizados}
  \end{Phonetics}
\end{Entry}

%%%%%%%%%% 们 %%%%%%%%%%
\subsection*{们}

\begin{Entry}{们}{5}{⼈}
  \begin{Phonetics}{们}{men5}[][HSK 1]
    \definition{suf.}{usado após pronomes ou substantivos que se referem a pessoas para indicar pluralidade}
  \end{Phonetics}
\end{Entry}

%%%%%%%%%% 兄 %%%%%%%%%%
\subsection*{兄}

\begin{Entry}{兄}{5}{⼉}
  \begin{Phonetics}{兄}{xiong1}
    \definition{s.}{irmão mais velho | parente mais velho do sexo masculino da mesma geração | uma forma cortês de tratamento entre amigos homens; um título respeitoso para amigos homens}
  \end{Phonetics}
\end{Entry}

\begin{Entry}{兄弟}{5,7}{⼉、⼸}
  \begin{Phonetics}{兄弟}{xiong1di4}[][HSK 4]
    \definition{adj.}{fraternal}
    \definition{pron.}{eu, me (termo de uso humilde por homens em discurso público)}
    \definition[个,位]{s.}{irmãos; irmão}
  \end{Phonetics}
\end{Entry}

%%%%%%%%%% 兰 %%%%%%%%%%
\subsection*{兰}

\begin{Entry}{兰}{5}{⼋}
  \begin{Phonetics}{兰}{lan2}
    \definition*{s.}{Sobrenome: Lan}
    \definition{s.}{orquídea | lírio magnólia}
  \end{Phonetics}
\end{Entry}

\begin{Entry}{兰州}{5,6}{⼋、⼮}
  \begin{Phonetics}{兰州}{lan2zhou1}
    \definition*{s.}{Lanzhou. capital da província de Gansu, 甘肃}
  \seealsoref{甘肃}{gan1su4}
  \end{Phonetics}
\end{Entry}

\begin{Entry}{兰花}{5,7}{⼋、⾋}
  \begin{Phonetics}{兰花}{lan2hua1}
    \definition{s.}{orquídea}
  \end{Phonetics}
\end{Entry}

%%%%%%%%%% 册 %%%%%%%%%%
\subsection*{册}

\begin{Entry}{册}{5}{⼌}
  \begin{Phonetics}{册}{ce4}[][HSK 5]
    \definition{clas.}{usado para cópias de livros}
    \definition{s.}{volume; livro | cópia; volume | ordem imperial para conferir um título}
    \definition{v.}{conferir um título}
  \end{Phonetics}
\end{Entry}

%%%%%%%%%% 写 %%%%%%%%%%
\subsection*{写}

\begin{Entry}{写}{5}{⼍}
  \begin{Phonetics}{写}{xie3}[][HSK 1]
    \definition{v.}{escrever | compor; escrever (como autor, repórter, etc.) | descrever; retratar | pintar; desenhar | expressar a imagem das coisas através da linguagem e da escrita | desenhar (pintura)}
  \end{Phonetics}
\end{Entry}

\begin{Entry}{写字}{5,6}{⼍、⼦}
  \begin{Phonetics}{写字}{xie3zi4}
    \definition{v.}{escrever (à mão) | praticar caligrafia}
  \end{Phonetics}
\end{Entry}

\begin{Entry}{写字台}{5,6,5}{⼍、⼦、⼝}
  \begin{Phonetics}{写字台}{xie3 zi4 tai2}[][HSK 6]
    \definition[个,张]{s.}{escrivaninha; secretária; escrivaninha de escrever; uma mesa retangular usada para escrever e trabalhar, com gavetas e algumas com pequenos armários}
  \end{Phonetics}
\end{Entry}

\begin{Entry}{写字匠}{5,6,6}{⼍、⼦、⼕}
  \begin{Phonetics}{写字匠}{xie3zi4 jiang4}
    \definition{s.}{calígrafo}
  \end{Phonetics}
\end{Entry}

\begin{Entry}{写字楼}{5,6,13}{⼍、⼦、⽊}
  \begin{Phonetics}{写字楼}{xie3 zi4 lou2}[][HSK 6]
    \definition{s.}{prédio de escritórios}
  \end{Phonetics}
\end{Entry}

\begin{Entry}{写作}{5,7}{⼍、⼈}
  \begin{Phonetics}{写作}{xie3zuo4}[][HSK 3]
    \definition{v.}{escrever artigos; escrever livros, etc.; também se refere especificamente à criação de obras literárias}
  \end{Phonetics}
\end{Entry}

\begin{Entry}{写真}{5,10}{⼍、⼗}
  \begin{Phonetics}{写真}{xie3zhen1}
    \definition{s.}{retrato}
    \definition{v.}{descrever algo com precisão}
  \end{Phonetics}
\end{Entry}

\begin{Entry}{写意}{5,13}{⼍、⼼}
  \begin{Phonetics}{写意}{xie3yi4}
    \definition{s.}{estilo de pintura chinesa à mão livre, caracterizado por traços ousados em vez de detalhes precisos}
    \definition{v.}{sugerir (em vez de descrever em detalhes)}
  \end{Phonetics}
  \begin{Phonetics}{写意}{xie4yi4}
    \definition{adj.}{confortável | agradável | relaxado}
  \end{Phonetics}
\end{Entry}

\begin{Entry}{写照}{5,13}{⼍、⽕}
  \begin{Phonetics}{写照}{xie3zhao4}
    \definition{s.}{retrato}
  \end{Phonetics}
\end{Entry}

%%%%%%%%%% 冬 %%%%%%%%%%
\subsection*{冬}

\begin{Entry}{冬}{5}{⼎}
  \begin{Phonetics}{冬}{dong1}
    \definition*{s.}{Sobrenome: Dong}
    \definition{s.}{inverno}
    \definition{s.}{onomatopéia: som de um tambor batendo, batendo na porta, etc.}
  \end{Phonetics}
\end{Entry}

\begin{Entry}{冬天}{5,4}{⼎、⼤}
  \begin{Phonetics}{冬天}{dong1 tian1}[][HSK 2]
    \definition[个]{s.}{inverno; a quarta estação do ano, na China, geralmente se refere aos três meses entre outubro e dezembro do calendário lunar}
  \end{Phonetics}
\end{Entry}

\begin{Entry}{冬瓜}{5,5}{⼎、⽠}
  \begin{Phonetics}{冬瓜}{dong1gua1}
    \definition{s.}{melão de inverno}
  \end{Phonetics}
\end{Entry}

\begin{Entry}{冬季}{5,8}{⼎、⼦}
  \begin{Phonetics}{冬季}{dong1 ji4}[][HSK 4]
    \definition[个,次,种]{s.}{inverno; o quarto trimestre do ano, habitualmente referido na China como o período de três meses entre o início do inverno e o início da primavera, e também referido como ``décimo, décimo primeiro e décimo segundo'' meses do calendário lunar}
  \end{Phonetics}
\end{Entry}

%%%%%%%%%% 凸 %%%%%%%%%%
\subsection*{凸}

\begin{Entry}{凸}{5}{⼐}
  \begin{Phonetics}{凸}{tu1}
    \definition{adj.}{saliente; elevado (oposto a 凹) | levantado; mais alto que o entorno}
    \definition{s.}{protuberância}
  \seealsoref{凹}{ao1}
  \end{Phonetics}
\end{Entry}

%%%%%%%%%% 凹 %%%%%%%%%%
\subsection*{凹}

\begin{Entry}{凹}{5}{⼐}
  \begin{Phonetics}{凹}{ao1}[][HSK 7-9]
    \definition{adj.}{afundado; amassado (oposto a 凸) | côncavo; oco; amassado; mais baixo que a área circundante}
    \definition{v.}{cavar; chanfrar | desabar; afundar}
  \seealsoref{凸}{tu1}
  \end{Phonetics}
  \begin{Phonetics}{凹}{wa1}
    \variantof{洼}
  \end{Phonetics}
\end{Entry}

%%%%%%%%%% 出 %%%%%%%%%%
\subsection*{出}

\begin{Entry}{出}{5}{⼐}
  \begin{Phonetics}{出}{chu1}[][HSK 1]
    \definition{clas.}{usado para dramas, peças, óperas, etc.}
    \definition{v.}{deixar; sair (ir); de dentro para fora | vir; chegar | exceder; ir além | emitir; levar para fora | produzir; despejar | surgir; acontecer; ter lugar | publicar; divulgar | ventilar; emitir; descarregar | aparecer; revelar | gastar; pagar}
  \end{Phonetics}
\end{Entry}

\begin{Entry}{出人意料}{5,2,13,10}{⼐、⼈、⼼、⽃}
  \begin{Phonetics}{出人意料}{chu1ren2yi4liao4}[][HSK 7-9]
    \definition{expr.}{excedendo todas as expectativas; além de todas as expectativas | surpreendente | inesperado}
  \end{Phonetics}
\end{Entry}

\begin{Entry}{出入}{5,2}{⼐、⼊}
  \begin{Phonetics}{出入}{chu1 ru4}[][HSK 6]
    \definition{s.}{discrepância; divergência; inconsistência; diferença}
    \definition{v.}{entrar e sair; sair e entrar}
  \end{Phonetics}
\end{Entry}

\begin{Entry}{出厂}{5,2}{⼐、⼚}
  \begin{Phonetics}{出厂}{chu1/chang3}[][HSK 7-9]
    \definition{v.+compl.}{(produtos) ser despachado da fábrica | sair da fábrica (produtos acabados)}
  \end{Phonetics}
\end{Entry}

\begin{Entry}{出于}{5,3}{⼐、⼆}
  \begin{Phonetics}{出于}{chu1 yu2}[][HSK 5]
    \definition{prep.}{de; fora de; por causa de; em função de; de um certo ponto de vista}
    \definition{v.}{iniciar a partir de; originar-se de; prosseguir a partir de}
  \end{Phonetics}
\end{Entry}

\begin{Entry}{出口}{5,3}{⼐、⼝}
  \begin{Phonetics}{出口}{chu1/kou3}[][HSK 2,4]
    \definition[个]{s.}{saída; porta ou passagem que dá acesso ao exterior}
    \definition{v.+compl.}{falar; proferir; manifestar-se | exportar mercadorias do país ou da região para venda no exterior ou em outro lugar | deixar o porto (de um navio)}
  \end{Phonetics}
\end{Entry}

\begin{Entry}{出口成章}{5,3,6,11}{⼐、⼝、⼽、⾳}
  \begin{Phonetics}{出口成章}{chu1kou3-cheng2zhang1}[][HSK 7-9]
    \definition{expr.}{``As palavras fluem da boca como da pena de um mestre.''; ``Seja um excelente orador.''  | (discurso de alguém) eloquente | articular}
  \end{Phonetics}
\end{Entry}

\begin{Entry}{出土}{5,3}{⼐、⼟}
  \begin{Phonetics}{出土}{chu1/tu3}[][HSK 7-9]
    \definition{v.+compl.}{ser desenterrado; ser escavado | sair do chão | subir; emergir}
  \end{Phonetics}
\end{Entry}

\begin{Entry}{出山}{5,3}{⼐、⼭}
  \begin{Phonetics}{出山}{chu1/shan1}[][HSK 7-9]
    \definition{v.+compl.}{deixar a área da montanha | sair da aposentadoria e assumir um cargo | deixar a aposentadoria e assumir um cargo governamental; tornar-se um funcionário público}
  \end{Phonetics}
\end{Entry}

\begin{Entry}{出门}{5,3}{⼐、⾨}
  \begin{Phonetics}{出门}{chu1/men2}[][HSK 2]
    \definition{v.+compl.}{sair | sair de casa; estar longe de casa; viajar para longe de casa | casar-se}
  \end{Phonetics}
\end{Entry}

\begin{Entry}{出丑}{5,4}{⼐、⼀}
  \begin{Phonetics}{出丑}{chu1/chou3}[][HSK 7-9]
    \definition{v.+compl.}{fazer alguém de bobo | fazer papel de bobo; tornar-se motivo de chacota; expor-se ao ridículo por falta de jeito (estranheza, incompetência, etc.) | envergonhar-se; expor os próprios pontos fracos}
  \end{Phonetics}
\end{Entry}

\begin{Entry}{出手}{5,4}{⼐、⼿}
  \begin{Phonetics}{出手}{chu1/shou3}[][HSK 7-9]
    \definition{s.}{a habilidade que você demonstra quando começa a fazer algo | comprimento da manga}
    \definition{v.+compl.}{vender; dispor de; tirar (bens acumulados, etc.) das mãos de alguém; vender mercadorias (usado principalmente para negociação, liquidação, etc.) | atacar; começar a agir; refere-se à luta física | oferecer; produzir; tirar (dinheiro ou algo); gastar dinheiro}
  \end{Phonetics}
\end{Entry}

\begin{Entry}{出毛病}{5,4,10}{⼐、⽑、⽧}
  \begin{Phonetics}{出毛病}{chu1 mao2 bing4}[][HSK 7-9]
    \definition{v.}{estar (sair) fora de ordem; dar errado; quebrar; ter problemas com; falhar; sofrer um acidente | Coloquial:
estar (ou ficar) fora de serviço}
  \end{Phonetics}
\end{Entry}

\begin{Entry}{出风头}{5,4,5}{⼐、⾵、⼤}
  \begin{Phonetics}{出风头}{chu1 feng1tou5}[][HSK 7-9]
    \definition{v.}{procurar (estar em; gostar de) holofotes; fazer uma figura elegante; impulsionar-se; exibir-se | Literário: buscar os holofotes}
  \end{Phonetics}
\end{Entry}

\begin{Entry}{出主意}{5,5,13}{⼐、⼂、⼼}
  \begin{Phonetics}{出主意}{chu1 zhu3yi5}[][HSK 7-9]
    \definition{v.}{oferecer conselhos; fornecer ideias; fazer sugestões; apresentar ideias}
  \end{Phonetics}
\end{Entry}

\begin{Entry}{出击}{5,5}{⼐、⼐}
  \begin{Phonetics}{出击}{chu1ji1}
    \definition{v.}{atacar}
  \end{Phonetics}
\end{Entry}

\begin{Entry}{出去}{5,5}{⼐、⼛}
  \begin{Phonetics}{出去}{chu1 qu4}[][HSK 1]
    \definition{v.}{sair; ir para fora;  (a partir da minha localização)}
  \end{Phonetics}
\end{Entry}

\begin{Entry}{出发}{5,5}{⼐、⼜}
  \begin{Phonetics}{出发}{chu1fa1}[][HSK 2]
    \definition{v.}{sair; partir; ir embora; deixar; sair do lugar onde se está e ir para outro lugar | começar a partir de; partir de; considerar ou tratar uma questão a partir de um determinado ponto de vista}
  \end{Phonetics}
\end{Entry}

\begin{Entry}{出发点}{5,5,9}{⼐、⼜、⽕}
  \begin{Phonetics}{出发点}{chu1fa1dian3}[][HSK 7-9]
    \definition{s.}{ponto de partida de uma jornada | ponto de vista (em uma discussão, argumento, etc.); ponto de partida | origem; ponto de partida}
  \end{Phonetics}
\end{Entry}

\begin{Entry}{出台}{5,5}{⼐、⼝}
  \begin{Phonetics}{出台}{chu1 tai2}[][HSK 6]
    \definition{v.}{aparecer no palco; entrar atores no palco | aparecer publicamente; metáfora (política, plano, programa, etc.) lançar oficialmente}
  \end{Phonetics}
\end{Entry}

\begin{Entry}{出头}{5,5}{⼐、⼤}
  \begin{Phonetics}{出头}{chu1/tou2}[][HSK 7-9]
    \definition{s.}{fração ímpar restante após uma divisão | (após um número redondo) um pouco superior; um pouco mais que}
    \definition{v.+compl.}{erguer a cabeça; libertar-se (da miséria, da perseguição, etc.); livrar-se de circunstâncias miseráveis; sair dos problemas | apresentar-se; aparecer em público; agir em nome de; assumir a liderança; avançar | (algo) mostrar sua ponta; apenas/se destacar (objeto) mostrando o topo}
  \end{Phonetics}
\end{Entry}

\begin{Entry}{出生}{5,5}{⼐、⽣}
  \begin{Phonetics}{出生}{chu1sheng1}[][HSK 2]
    \definition{v.}{nascer}
  \end{Phonetics}
\end{Entry}

\begin{Entry}{出示}{5,5}{⼐、⽰}
  \begin{Phonetics}{出示}{chu1shi4}[][HSK 7-9]
    \definition{v.}{mostrar; produzir}
  \end{Phonetics}
\end{Entry}

\begin{Entry}{出任}{5,6}{⼐、⼈}
  \begin{Phonetics}{出任}{chu1ren4}[][HSK 7-9]
    \definition{v.}{assumir o cargo de; assumir (uma determinada posição oficial)}
  \end{Phonetics}
\end{Entry}

\begin{Entry}{出众}{5,6}{⼐、⼈}
  \begin{Phonetics}{出众}{chu1zhong4}[][HSK 7-9]
    \definition{adj.}{fora do comum; excepcional}
  \end{Phonetics}
\end{Entry}

\begin{Entry}{出动}{5,6}{⼐、⼒}
  \begin{Phonetics}{出动}{chu1 dong4}[][HSK 6]
    \definition{v.}{partir; começar; passear em equipe | chamar; enviar; despachar; enviar forças militares | entrar em ação; manifestar-se; tomar uma atitude}
  \end{Phonetics}
\end{Entry}

\begin{Entry}{出名}{5,6}{⼐、⼝}
  \begin{Phonetics}{出名}{chu1 ming2}[][HSK 6]
    \definition{adj.}{famoso; bem conhecido; renomado}
    \definition{v.}{tornar-se famoso (ou conhecido)  | emprestar o próprio nome (para uma ocasião ou empresa); usar o nome de}
  \end{Phonetics}
\end{Entry}

\begin{Entry}{出场}{5,6}{⼐、⼟}
  \begin{Phonetics}{出场}{chu1 chang3}[][HSK 6]
    \definition{v.}{entrar no palco; aparecer em cena; entrar atores no palco (performance) | entrar na arena ou campo de esportes; entrar atletas no estádio (para participar de uma apresentação ou competição)}
  \end{Phonetics}
\end{Entry}

\begin{Entry}{出汗}{5,6}{⼐、⽔}
  \begin{Phonetics}{出汗}{chu1 han4}[][HSK 5]
    \definition{v.}{suar; transpirar}
  \end{Phonetics}
\end{Entry}

\begin{Entry}{出自}{5,6}{⼐、⾃}
  \begin{Phonetics}{出自}{chu1zi4}[][HSK 7-9]
    \definition{v.}{vir de; provir de; originar-se de (um certo tempo ou lugar)}
  \end{Phonetics}
\end{Entry}

\begin{Entry}{出色}{5,6}{⼐、⾊}
  \begin{Phonetics}{出色}{chu1se4}[][HSK 4]
    \definition{adj.}{esplêndido; extraordinário; notável; excepcionalmente bom; acima da média}
  \end{Phonetics}
\end{Entry}

\begin{Entry}{出血}{5,6}{⼐、⾎}
  \begin{Phonetics}{出血}{chu1/xie3}[][HSK 7-9]
    \definition{s.}{Medicina: hemorragia; sangramento}
    \definition{v.+compl.}{perder sangue; sangrar | desembolsar (dinheiro, etc.)}
  \end{Phonetics}
\end{Entry}

\begin{Entry}{出行}{5,6}{⼐、⾏}
  \begin{Phonetics}{出行}{chu1 xing2}[][HSK 6]
    \definition{v.}{viajar; sair}
  \end{Phonetics}
\end{Entry}

\begin{Entry}{出访}{5,6}{⼐、⾔}
  \begin{Phonetics}{出访}{chu1 fang3}[][HSK 6]
    \definition{v.}{ir ao exterior em visita oficial | ir visitar em caráter oficial ou para investigação}
  \end{Phonetics}
\end{Entry}

\begin{Entry}{出局}{5,7}{⼐、⼫}
  \begin{Phonetics}{出局}{chu1/ju2}[][HSK 7-9]
    \definition{v.+compl.}{estar fora (no beisebol, softball, etc.) | estar fora (de uma competição, etc.); abandonar; ser eliminado}
  \end{Phonetics}
\end{Entry}

\begin{Entry}{出来}{5,7}{⼐、⽊}
  \begin{Phonetics}{出来}{chu1 lai2}[][HSK 1]
    \definition{v.}{emergir; sair; (para a minha direção) |  surgir; aparecer; emergir | concluir ou algo acontecer}
  \end{Phonetics}
\end{Entry}

\begin{Entry}{出走}{5,7}{⼐、⾛}
  \begin{Phonetics}{出走}{chu1zou3}[][HSK 7-9]
    \definition{v.}{deixar a própria casa (ou país) sob compulsão; fugir; ir embora; sair de casa}
  \end{Phonetics}
\end{Entry}

\begin{Entry}{出身}{5,7}{⼐、⾝}
  \begin{Phonetics}{出身}{chu1shen1}[][HSK 7-9]
    \definition{s.}{origem familiar; origem (de classe) | experiência anterior (ou ocupação)}
    \definition{v.}{ser descendente de; vir de}
  \end{Phonetics}
\end{Entry}

\begin{Entry}{出事}{5,8}{⼐、⼅}
  \begin{Phonetics}{出事}{chu1 shi4}[][HSK 6]
    \definition{v.}{sofrer um acidente; ocorrer um acidente ou incidente}
  \end{Phonetics}
\end{Entry}

\begin{Entry}{出具}{5,8}{⼐、⼋}
  \begin{Phonetics}{出具}{chu1ju4}[][HSK 7-9]
    \definition{v.}{emitir; escrever; produzir ou escrever (algum tipo de prova)}
  \end{Phonetics}
\end{Entry}

\begin{Entry}{出卖}{5,8}{⼐、⼗}
  \begin{Phonetics}{出卖}{chu1mai4}[][HSK 7-9]
    \definition{v.}{vender; oferecer para venda; trocar trabalho por uma certa quantia de compensação | vender; trair; negociar; fazer coisas que beneficiam o inimigo para ganho pessoal, causando danos ao país, à nação, a parentes e amigos, etc.}
  \end{Phonetics}
\end{Entry}

\begin{Entry}{出国}{5,8}{⼐、⼞}
  \begin{Phonetics}{出国}{chu1/guo2}[][HSK 2]
    \definition{v.+compl.}{ir para o exterior; deixar a terra natal; viajar para o exterior}
  \end{Phonetics}
\end{Entry}

\begin{Entry}{出版}{5,8}{⼐、⽚}
  \begin{Phonetics}{出版}{chu1ban3}[][HSK 5]
    \definition{v.}{aparecer; publicar; sair; sair da imprensa}
  \end{Phonetics}
\end{Entry}

\begin{Entry}{出版社}{5,8,7}{⼐、⽚、⽰}
  \begin{Phonetics}{出版社}{chu1ban3she4}[][HSK 7-9]
    \definition{s.}{editora; instituições que se dedicam à edição e publicação de livros, periódicos, pinturas, produtos audiovisuais, etc.}
  \end{Phonetics}
\end{Entry}

\begin{Entry}{出现}{5,8}{⼐、⾒}
  \begin{Phonetics}{出现}{chu1xian4}[][HSK 2]
    \definition{v.}{aparecer; surgir; emergir; crescer; revelar}
  \end{Phonetics}
\end{Entry}

\begin{Entry}{出差}{5,9}{⼐、⼯}
  \begin{Phonetics}{出差}{chu1/chai1}[][HSK 5]
    \definition{v.+compl.}{fazer uma viagem de negócios | assumir tarefas de curto prazo em transporte, construção, etc.}
  \end{Phonetics}
\end{Entry}

\begin{Entry}{出洋相}{5,9,9}{⼐、⽔、⽬}
  \begin{Phonetics}{出洋相}{chu1 yang2xiang4}[][HSK 7-9]
    \definition{v.}{fazer papel de bobo}
  \end{Phonetics}
\end{Entry}

\begin{Entry}{出院}{5,9}{⼐、⾩}
  \begin{Phonetics}{出院}{chu1 yuan4}[][HSK 2]
    \definition{v.}{sair do hospital; estar fora do hospital; receber alta do hospital}
  \end{Phonetics}
\end{Entry}

\begin{Entry}{出面}{5,9}{⼐、⾯}
  \begin{Phonetics}{出面}{chu1 mian4}[][HSK 6]
    \definition{v.}{comparecer pessoalmente; agir em sua própria capacidade (ou em nome de alguém) | agir em sua própria capacidade (em nome de uma organização); apresentar-se; fazer algo individualmente ou coletivamente}
  \end{Phonetics}
\end{Entry}

\begin{Entry}{出家}{5,10}{⼐、⼧}
  \begin{Phonetics}{出家}{chu1 jia1}
    \definition{v.}{renunciar à família (para se tornar monge ou monja) (oposto a 在家) | tornar-se monge, monja ou sacerdote taoísta}
  \seealsoref{在家}{zai4 jia1}
  \end{Phonetics}
\end{Entry}

\begin{Entry}{出席}{5,10}{⼐、⼱}
  \begin{Phonetics}{出席}{chu1xi2}[][HSK 4]
    \definition{v.}{comparecer; estar presente; participar de reuniões com o direito de falar e votar; juntar-se a uma organização ou atividade}
  \end{Phonetics}
\end{Entry}

\begin{Entry}{出息}{5,10}{⼐、⼼}
  \begin{Phonetics}{出息}{chu1xi5}[][HSK 7-9]
    \definition{s.}{perspectivas; futuro brilhante; refere-se ao desenvolvimento ou ambição futura}
    \definition{v.+compl.}{progredir; progredir bem; há progresso e a capacidade se torna mais forte}
  \end{Phonetics}
\end{Entry}

\begin{Entry}{出租}{5,10}{⼐、⽲}
  \begin{Phonetics}{出租}{chu1 zu1}[][HSK 2]
    \definition{s.}{taxi; abreviação de 出租车}
    \definition{v.}{alugar; arrendar; receber dinheiro de outras pessoas para permitir que elas utilizem algo (como uma casa, um carro, livros, etc.) por um determinado período de tempo}
  \seealsoref{出租车}{chu1zu1che1}
  \end{Phonetics}
\end{Entry}

\begin{Entry}{出租车}{5,10,4}{⼐、⽲、⾞}
  \begin{Phonetics}{出租车}{chu1zu1che1}[][HSK 2]
    \definition[辆]{s.}{táxi; carro de aluguel; veículos de transporte urbano disponíveis para aluguel, com cobrança por quilometragem ou tempo}
  \seealsoref{出租汽车}{chu1zu1qi4che1}
  \end{Phonetics}
\end{Entry}

\begin{Entry}{出租司机}{5,10,5,6}{⼐、⽲、⼝、⽊}
  \begin{Phonetics}{出租司机}{chu1zu1si1ji1}
    \definition{s.}{motorista de táxi}
  \end{Phonetics}
\end{Entry}

\begin{Entry}{出租汽车}{5,10,7,4}{⼐、⽲、⽔、⾞}
  \begin{Phonetics}{出租汽车}{chu1zu1qi4che1}
    \definition[辆]{s.}{táxi}
  \seealsoref{出租车}{chu1zu1che1}
  \end{Phonetics}
\end{Entry}

\begin{Entry}{出站}{5,10}{⼐、⽴}
  \begin{Phonetics}{出站}{chu1 zhan4}
    \definition{s.}{saída da estação}
  \end{Phonetics}
\end{Entry}

\begin{Entry}{出资}{5,10}{⼐、⾙}
  \begin{Phonetics}{出资}{chu1zi1}[][HSK 7-9]
    \definition{v.}{financiar; investir capital; fornecer fundos (ou capital); fornecer financiamento}
  \end{Phonetics}
\end{Entry}

\begin{Entry}{出难题}{5,10,15}{⼐、⾫、⾴}
  \begin{Phonetics}{出难题}{chu1 nan2ti2}[][HSK 7-9]
    \definition{v.}{levantar uma questão difícil; dificultar as coisas para alguém | fazer perguntas difíceis; propor um problema difícil; definir uma tarefa muito difícil}
  \end{Phonetics}
\end{Entry}

\begin{Entry}{出售}{5,11}{⼐、⼝}
  \begin{Phonetics}{出售}{chu1 shou4}[][HSK 4]
    \definition{v.}{vender; oferecer para venda}
  \end{Phonetics}
\end{Entry}

\begin{Entry}{出游}{5,12}{⼐、⽔}
  \begin{Phonetics}{出游}{chu1you2}[][HSK 7-9]
    \definition{v.}{fazer um passeio turístico (ou viagem)}
  \end{Phonetics}
\end{Entry}

\begin{Entry}{出道}{5,12}{⼐、⾡}
  \begin{Phonetics}{出道}{chu1/dao4}[][HSK 7-9]
    \definition{v.+compl.}{(anteriormente um aprendiz) começar a trabalhar depois de cumprir seu aprendizado; refere-se à aprendizagem de uma habilidade e ao início do trabalho independente; agora geralmente se refere aos jovens que entram na sociedade pela primeira vez | tornar-se conhecido, estrear-se na sociedade; começar uma carreira (no show business, etc.); refere-se a um artista que entra formalmente na indústria do entretenimento e ganha reconhecimento na carreira}
  \end{Phonetics}
\end{Entry}

\begin{Entry}{出路}{5,13}{⼐、⾜}
  \begin{Phonetics}{出路}{chu1lu4}[][HSK 6]
    \definition[条,个]{s.}{saída; futuro; uma maneira de manter a sobrevivência ou progredir intermitentemente; também pode se referir ao futuro | saída; formas de vender produtos | saída; um caminho que leva para fora ou para frente}
  \end{Phonetics}
\end{Entry}

\begin{Entry}{出境}{5,14}{⼐、⼟}
  \begin{Phonetics}{出境}{chu1/jing4}[][HSK 7-9]
    \definition{v.+compl.}{sair do país | sair de uma determinada região | partir; sair; emigrar}
  \end{Phonetics}
\end{Entry}

\begin{Entry}{出演}{5,14}{⼐、⽔}
  \begin{Phonetics}{出演}{chu1yan3}[][HSK 7-9]
    \definition{s.}{uma aparição (no palco, etc.)}
    \definition{v.}{desempenhar o papel de; atuar | aparecer (em um show, etc.)}
  \end{Phonetics}
\end{Entry}

%%%%%%%%%% 刊 %%%%%%%%%%
\subsection*{刊}

\begin{Entry}{刊}{5}{⼑}
  \begin{Phonetics}{刊}{kan1}
    \definition{s.}{periódico; publicação (jornais, revistas, etc., excluindo livros) | (geralmente em um jornal) coluna especial}
    \definition{v.}{Literário: cortar; picar | Literário: esculpir; gravar | apagar ou corrigir | imprimir; publicar; publicar em um jornal ou revista}
  \end{Phonetics}
\end{Entry}

\begin{Entry}{刊物}{5,8}{⼑、⽜}
  \begin{Phonetics}{刊物}{kan1wu4}[][HSK 7-9]
    \definition[份,本,家,期]{s.}{jornal; periódico; publicação; revistas publicadas regularmente ou irregularmente geralmente contêm artigos, imagens, etc.}
  \end{Phonetics}
\end{Entry}

\begin{Entry}{刊登}{5,12}{⼑、⽨}
  \begin{Phonetics}{刊登}{kan1deng1}[][HSK 7-9]
    \definition{v.}{lançar; publicar}
  \end{Phonetics}
\end{Entry}

%%%%%%%%%% 功 %%%%%%%%%%
\subsection*{功}

\begin{Entry}{功}{5}{⼒}
  \begin{Phonetics}{功}{gong1}[][HSK 7-9]
    \definition*{s.}{Sobrenome: Gong}
    \definition[次,大]{s.}{mérito; façanha; serviço meritório (ação) | resultado; eficácia; realização | habilidade; habilidade técnica; tecnologia e qualificação técnica | trabalho; uma força faz com que um objeto se desloque uma certa distância na direção da força}
  \end{Phonetics}
\end{Entry}

\begin{Entry}{功力}{5,2}{⼒、⼒}
  \begin{Phonetics}{功力}{gong1li4}[][HSK 7-9]
    \definition{s.}{efeito; eficácia | habilidade}
  \end{Phonetics}
\end{Entry}

\begin{Entry}{功夫}{5,4}{⼒、⼤}
  \begin{Phonetics}{功夫}{gong1fu5}[][HSK 3]
    \definition*{s.}{Gongfu (Kung Fu), arte marcial}
    \definition[番]{s.}{habilidade; destreza; conhecimento | luta acrobática; habilidade em artes marciais | esforço; tempo e energia}
  \end{Phonetics}
\end{Entry}

\begin{Entry}{功臣}{5,6}{⼒、⾂}
  \begin{Phonetics}{功臣}{gong1chen2}[][HSK 7-9]
    \definition[个]{s.}{pessoa que prestou serviço excepcional (oposição a 罪人) | ministro que prestou serviço excepcional; um funcionário meritório se refere a alguém que fez contribuições significativas para uma determinada causa.}
  \seealsoref{罪人}{zui4ren2}
  \end{Phonetics}
\end{Entry}

\begin{Entry}{功劳}{5,7}{⼒、⼒}
  \begin{Phonetics}{功劳}{gong1lao2}[][HSK 7-9]
    \definition{s.}{contribuição; crédito; contribuição para a causa}
  \end{Phonetics}
\end{Entry}

\begin{Entry}{功底}{5,8}{⼒、⼴}
  \begin{Phonetics}{功底}{gong1di3}[][HSK 7-9]
    \definition{s.}{fundação; base sólida; formação profunda; conhecimento dos fundamentos; boas habilidades básicas}
  \end{Phonetics}
\end{Entry}

\begin{Entry}{功效}{5,10}{⼒、⽁}
  \begin{Phonetics}{功效}{gong1xiao4}[][HSK 7-9]
    \definition{s.}{efeito; eficiência; comportamento; eficácia; o efeito, função ou eficiência de um medicamento, método ou outra coisa}
  \end{Phonetics}
\end{Entry}

\begin{Entry}{功能}{5,10}{⼒、⾁}
  \begin{Phonetics}{功能}{gong1neng2}[][HSK 3]
    \definition[种,项]{s.}{função; os efeitos positivos produzidos por coisas ou métodos}
  \end{Phonetics}
\end{Entry}

\begin{Entry}{功课}{5,10}{⼒、⾔}
  \begin{Phonetics}{功课}{gong1 ke4}[][HSK 3]
    \definition[份,门]{s.}{trabalho escolar; dever de casa; refere-se aos trabalhos de casa atribuídos pelos professores aos alunos| tarefa; lições; lição escolar | preparações; preparação necessária antes de fazer algo}
  \end{Phonetics}
\end{Entry}

\begin{Entry}{功率}{5,11}{⼒、⽞}
  \begin{Phonetics}{功率}{gong1lv4}[][HSK 7-9]
    \definition[瓦,千瓦,兆瓦]{s.}{potência (W); uma grandeza física que indica a velocidade com que o trabalho é realizado; ou seja, o trabalho realizado ou consumido por unidade de tempo; a unidade é Watt}
  \end{Phonetics}
\end{Entry}

\begin{Entry}{功绩}{5,11}{⼒、⽷}
  \begin{Phonetics}{功绩}{gong1ji4}
    \definition{s.}{mérito e realização; feito; contribuição (oposto a 过失 )}
  \seealsoref{过失}{guo4shi1}
  \end{Phonetics}
\end{Entry}

%%%%%%%%%% 加 %%%%%%%%%%
\subsection*{加}

\begin{Entry}{加}{5}{⼒}
  \begin{Phonetics}{加}{jia1}[][HSK 2]
    \definition*{s.}{Canadá, abreviação de 加拿大 | Sobrenome: Jia}
    \definition{v.}{adicionar; somar | aumentar; incrementar; aumentar a quantidade ou o grau em relação ao original | inserir; adicionar; anexar; adicionar o que não existe; colocar no lugar | acrescentar; indica a realização de uma determinada ação | colocar uma coisa em cima da outra | impor ou aplicar algo a outra pessoa; atribuir um determinado comportamento a outra pessoa}
  \seealsoref{加拿大}{jia1na2da4}
  \end{Phonetics}
\end{Entry}

\begin{Entry}{加入}{5,2}{⼒、⼊}
  \begin{Phonetics}{加入}{jia1ru4}[][HSK 4]
    \definition{v.}{juntar-se; unir-se; aderir a; tornar-se um membro de uma organização, grupo | adicionar; colocar em}
  \end{Phonetics}
\end{Entry}

\begin{Entry}{加上}{5,3}{⼒、⼀}
  \begin{Phonetics}{加上}{jia1 shang4}[][HSK 5]
    \definition{conj.}{além disso; em adição}
    \definition{v.}{adicionar; acrescentar; dar; aumentar}
  \end{Phonetics}
\end{Entry}

\begin{Entry}{加工}{5,3}{⼒、⼯}
  \begin{Phonetics}{加工}{jia1gong1}[][HSK 3]
    \definition{s.}{processo | trabalho (de uma máquina)}
    \definition{v.}{processar; realizar diversos trabalhos em matérias-primas e produtos semiacabados (como alterar dimensões, formas, propriedades, aumentar a precisão, pureza, etc.) para que atendam aos requisitos especificados | melhorar; polir; refere-se a todos os tipos de trabalho que tornam o produto final mais perfeito e refinado}
  \end{Phonetics}
\end{Entry}

\begin{Entry}{加以}{5,4}{⼒、⼈}
  \begin{Phonetics}{加以}{jia1 yi3}[][HSK 5]
    \definition{conj.}{além disso; em adição; indica outras razões ou condições}
    \definition{v.aux.}{usado na frente de palavras dissilábicas para indicar como um objeto mencionado deve ser tratado ou descartado | usado antes de um verbo polifônico ou de um substantivo formado a partir de um verbo para indicar como tratar ou lidar com o que foi mencionado anteriormente}
  \end{Phonetics}
\end{Entry}

\begin{Entry}{加快}{5,7}{⼒、⼼}
  \begin{Phonetics}{加快}{jia1 kuai4}[][HSK 3]
    \definition{v.}{acelerar; aumentar a velocidade; agilizar}
  \end{Phonetics}
\end{Entry}

\begin{Entry}{加油}{5,8}{⼒、⽔}
  \begin{Phonetics}{加油}{jia1/you2}[][HSK 2]
    \definition{v.+compl.}{abastecer com óleo; reabastecer; adicionar combustível ou óleo lubrificante | fazer um esforço extra; dar o máximo; (Vamos lá!) metáfora para se esforçar ainda mais}
  \end{Phonetics}
\end{Entry}

\begin{Entry}{加油工}{5,8,3}{⼒、⽔、⼯}
  \begin{Phonetics}{加油工}{jia1 you2 gong1}[][HSK 6]
    \definition{s.}{frentista}
  \end{Phonetics}
\end{Entry}

\begin{Entry}{加油站}{5,8,10}{⼒、⽔、⽴}
  \begin{Phonetics}{加油站}{jia1you2zhan4}[][HSK 4]
    \definition[个,座,家]{s.}{posto de gasolina; posto de combustível; postos de abastecimento para venda a varejo de gasolina e óleo para carros e outros veículos motorizados}
  \end{Phonetics}
\end{Entry}

\begin{Entry}{加重}{5,9}{⼒、⾥}
  \begin{Phonetics}{加重}{jia1zhong4}[][HSK 7-9]
    \definition{v.}{tornar mais pesado; aumentar o peso de; aumentar a quantidade de | tornar mais sério; agravar}
  \end{Phonetics}
\end{Entry}

\begin{Entry}{加剧}{5,10}{⼒、⼑}
  \begin{Phonetics}{加剧}{jia1ju4}[][HSK 7-9]
    \definition{v.}{agravar; intensificar; exacerbar; tornar-se mais sério do que antes}
  \end{Phonetics}
\end{Entry}

\begin{Entry}{加拿大}{5,10,3}{⼒、⼿、⼤}
  \begin{Phonetics}{加拿大}{jia1na2da4}
    \definition{s.}{Canadá}
  \end{Phonetics}
\end{Entry}

\begin{Entry}{加拿大人}{5,10,3,2}{⼒、⼿、⼤、⼈}
  \begin{Phonetics}{加拿大人}{jia1na2da4ren2}
    \definition{s.}{canadense | pessoa ou povo do Canadá}
  \end{Phonetics}
\end{Entry}

\begin{Entry}{加热}{5,10}{⼒、⽕}
  \begin{Phonetics}{加热}{jia1 re4}[][HSK 5]
    \definition{v.}{aquecer; esquentar; aumentar a temperatura de um objeto}
  \end{Phonetics}
\end{Entry}

\begin{Entry}{加班}{5,10}{⼒、⽟}
  \begin{Phonetics}{加班}{jia1/ban1}[][HSK 4]
    \definition{v.+compl.}{fazer horas extras; trabalhar horas extras; aumentar o horário de trabalho ou os turnos além do limite de tempo prescrito}
  \end{Phonetics}
\end{Entry}

\begin{Entry}{加紧}{5,10}{⼒、⽷}
  \begin{Phonetics}{加紧}{jia1jin3}[][HSK 7-9]
    \definition{v.}{intensificar; acelerar; aumentar a velocidade ou intensidade}
  \end{Phonetics}
\end{Entry}

\begin{Entry}{加速}{5,10}{⼒、⾡}
  \begin{Phonetics}{加速}{jia1 su4}[][HSK 5]
    \definition{v.}{acelerar; agilizar}
  \end{Phonetics}
\end{Entry}

\begin{Entry}{加速度}{5,10,9}{⼒、⾡、⼴}
  \begin{Phonetics}{加速度}{jia1su4du4}
    \definition{s.}{velocidade acelerada; aceleração | Física: aceleração}
  \end{Phonetics}
\end{Entry}

\begin{Entry}{加深}{5,11}{⼒、⽔}
  \begin{Phonetics}{加深}{jia1shen1}[][HSK 7-9]
    \definition{v.}{aprofundar; aumentar a profundidade; torne-se mais profundo}
  \end{Phonetics}
\end{Entry}

\begin{Entry}{加强}{5,12}{⼒、⼸}
  \begin{Phonetics}{加强}{jia1 qiang2}[][HSK 3]
    \definition{v.}{fortalecer; engrandecer; reforçar; tornar mais forte ou mais eficaz}
  \end{Phonetics}
\end{Entry}

\begin{Entry}{加盟}{5,13}{⼒、⽫}
  \begin{Phonetics}{加盟}{jia1 meng2}[][HSK 6]
    \definition{v.}{aliar-se a; filiar-se a um sindicato; juntar-se a um grupo ou organização}
  \end{Phonetics}
\end{Entry}

%%%%%%%%%% 务 %%%%%%%%%%
\subsection*{务}

\begin{Entry}{务}{5}{⼒}
  \begin{Phonetics}{务}{wu4}
    \definition*{s.}{Sobrenome: Wu}
    \definition{s.}{caso; negócio | usado em nomes de lugares}
    \definition{v.}{engajar-se em; dedicar seus esforços a | procurar; perseguir; ir atrás | estar envolvido em; dedicar-se a; envolver-se em; comprometer-se com | deve; deveria; ter certeza de}
  \end{Phonetics}
\end{Entry}

\begin{Entry}{务必}{5,5}{⼒、⼼}
  \begin{Phonetics}{务必}{wu4bi4}
    \definition{adv.}{deve; ter certeza de; necessariamente; usado principalmente em frases afirmativas}
  \end{Phonetics}
\end{Entry}

\begin{Entry}{务实}{5,8}{⼒、⼧}
  \begin{Phonetics}{务实}{wu4shi2}
    \definition{adj.}{pragmático; prático; pé-no-chão}
    \definition{v.}{lidar com assuntos concretos; discutir e estudar questões específicas; envolver-se em trabalho específico}
  \end{Phonetics}
\end{Entry}

%%%%%%%%%% 包 %%%%%%%%%%
\subsection*{包}

\begin{Entry}{包}{5}{⼓}
  \begin{Phonetics}{包}{bao1}[][HSK 1]
    \definition*{s.}{Sobrenome: Bao}
    \definition{clas.}{pacote; embalagem; embrulho; usado para coisas empacotadas}
    \definition[个,只]{s.}{feixe; pacote; encomenda; algo embrulhado | saco; sacola; saco para guardar coisas | caroço; inchaço; protuberância; inchaço ou protuberância no corpo ou em objetos | tenda; tenda com cúpula feita de feltro}
    \definition{v.}{embrulhar; envolver com papel, tecido, etc. | cercar; rodear; envolver; envelopar | incluir; conter | realizar todo o processo; assumir toda a responsabilidade | assegurar; garantir | contratar; reservar; fretar; comprar ou alugar tudo; acordar uso exclusivo}
  \end{Phonetics}
\end{Entry}

\begin{Entry}{包子}{5,3}{⼓、⼦}
  \begin{Phonetics}{包子}{bao1 zi5}[][HSK 1]
    \definition[个]{s.}{pão recheado cozido no vapor; alimentos, com recheio de vegetais, carne ou açúcar, etc., com massa levedada como invólucro, embrulhados e cozidos no vapor}
  \end{Phonetics}
\end{Entry}

\begin{Entry}{包干}{5,3}{⼓、⼲}
  \begin{Phonetics}{包干}{bao1gan1}
    \definition{s.}{tarefa alocada}
    \definition{v.}{ter a responsabilidade total sobre um trabalho}
  \end{Phonetics}
\end{Entry}

\begin{Entry}{包办}{5,4}{⼓、⼒}
  \begin{Phonetics}{包办}{bao1ban4}
    \definition{v.}{cuidar de tudo que diz respeito a um trabalho | comandar todo o espetáculo; monopolizar tudo | assumir tudo; manter tudo em suas próprias mãos}
  \end{Phonetics}
\end{Entry}

\begin{Entry}{包扎}{5,4}{⼓、⼿}
  \begin{Phonetics}{包扎}{bao1za1}[][HSK 7-9]
    \definition{v.}{empacotar; amarrar; embrulhar}
  \end{Phonetics}
\end{Entry}

\begin{Entry}{包含}{5,7}{⼓、⼝}
  \begin{Phonetics}{包含}{bao1han2}[][HSK 4]
    \definition{v.}{conter; implicar; incluir; conter dentro, resumir, enfatizar o que está contido dentro, focar em relações internas, muitas vezes coisas abstratas}
  \end{Phonetics}
\end{Entry}

\begin{Entry}{包围}{5,7}{⼓、⼞}
  \begin{Phonetics}{包围}{bao1wei2}[][HSK 5]
    \definition{v.}{circundar; cercar; rodear}
  \end{Phonetics}
\end{Entry}

\begin{Entry}{包括}{5,9}{⼓、⼿}
  \begin{Phonetics}{包括}{bao1kuo4}[][HSK 4]
    \definition{v.}{incluir; compreender; consistir em; conter, conter dentro, resumir junto, enfatizar a listagem de todas as partes, ou a citação de uma parte delas, que podem ser coisas abstratas ou concretas}
  \end{Phonetics}
\end{Entry}

\begin{Entry}{包容}{5,10}{⼓、⼧}
  \begin{Phonetics}{包容}{bao1rong2}[][HSK 7-9]
    \definition{adj.}{inclusivo}
    \definition{v.}{perdoar; mostrar tolerância; fácil de aceitar ideias diferentes ou ser capaz de perdoar o comportamento ou linguagem hostil dos outros | segurar; conter; quantas pessoas ou coisas podem ser colocadas em um determinado espaço}
  \end{Phonetics}
\end{Entry}

\begin{Entry}{包租}{5,10}{⼓、⽲}
  \begin{Phonetics}{包租}{bao1zu1}
    \definition{s.}{aluguel fixo para terras agrícolas}
    \definition{v.}{fretar | alugar | alugar um terreno ou uma casa para subarrendar}
  \end{Phonetics}
\end{Entry}

\begin{Entry}{包袱}{5,11}{⼓、⾐}
  \begin{Phonetics}{包袱}{bao1fu5}[][HSK 7-9]
    \definition[个,堆,身]{s.}{um pacote embrulhado em pano; uma bolsa embrulhada em pano contendo roupas e outras necessidades diárias | carga; peso; fardo; uma metáfora para um fardo que afeta o pensamento ou a ação}
  \end{Phonetics}
\end{Entry}

\begin{Entry}{包装}{5,12}{⼓、⾐}
  \begin{Phonetics}{包装}{bao1zhuang1}[][HSK 5]
    \definition[个,款]{s.}{embalagem; materiais usados para embalar produtos, como papel, sacolas, garrafas ou caixas}
    \definition{v.}{embalar; embrulhar; empacotar | aumentar a fama e o apelo de alguém ou algo por meio de publicidade | tornar alguém ou algo mais comercialmente viável ou atraente por meio de embelezamento ou publicidade}
  \end{Phonetics}
\end{Entry}

\begin{Entry}{包裹}{5,14}{⼓、⾐}
  \begin{Phonetics}{包裹}{bao1guo3}[][HSK 4]
    \definition[个,件]{s.}{pacote; embrulho}
    \definition{v.}{embrulhar; amarrar; enrolar coisas em pano ou outra coisa}
  \end{Phonetics}
\end{Entry}

%%%%%%%%%% 匆 %%%%%%%%%%
\subsection*{匆}

\begin{Entry}{匆}{5}{⼓}
  \begin{Phonetics}{匆}{cong1}
    \definition{adj.}{apressado}
    \definition{adv.}{apressadamente}
  \end{Phonetics}
\end{Entry}

\begin{Entry}{匆匆}{5,5}{⼓、⼓}
  \begin{Phonetics}{匆匆}{cong1cong1}[][HSK 7-9]
    \definition{adj.}{apressado; com pressa}
  \end{Phonetics}
\end{Entry}

\begin{Entry}{匆忙}{5,6}{⼓、⼼}
  \begin{Phonetics}{匆忙}{cong1mang2}[][HSK 7-9]
    \definition{adj.}{apressado; com pressa; ocupado}
  \end{Phonetics}
\end{Entry}

%%%%%%%%%% 北 %%%%%%%%%%
\subsection*{北}

\begin{Entry}{北}{5}{⼔}
  \begin{Phonetics}{北}{bei3}[][HSK 1]
    \definition*{s.}{Norte (os países desenvolvidos) | Sobrenome: Bei}
    \definition{s.}{norte; uma das quatro direções básicas, a esquerda quando se está de frente para o sol pela manhã (oposta ao 南)}
    \definition{v.}{ser derrotado}
  \seealsoref{南}{nan2}
  \end{Phonetics}
\end{Entry}

\begin{Entry}{北大西洋公约组织}{5,3,6,9,4,6,8,8}{⼔、⼤、⾑、⽔、⼋、⽷、⽷、⽷}
  \begin{Phonetics}{北大西洋公约组织}{bei3 da4xi1 yang2 gong1 yue1 zu3zhi1}
    \definition*{s.}{Organização do Tratado do Atlântico Norte, OTAN}
  \end{Phonetics}
\end{Entry}

\begin{Entry}{北方}{5,4}{⼔、⽅}
  \begin{Phonetics}{北方}{bei3fang1}[][HSK 2]
    \definition{s.}{norte; indicando a direção norte | o Norte; a parte norte da China, especialmente a área ao norte do rio Huang He}
  \end{Phonetics}
\end{Entry}

\begin{Entry}{北边}{5,5}{⼔、⾡}
  \begin{Phonetics}{北边}{bei3 bian1}[][HSK 1]
    \definition{s.}{norte; o lado norte}
  \end{Phonetics}
\end{Entry}

\begin{Entry}{北约}{5,6}{⼔、⽷}
  \begin{Phonetics}{北约}{bei3yue1}
    \definition*{s.}{OTAN, Organização do Tratado do Atlântico Norte; Abreviação de 北大西洋公约组织}
  \seealsoref{北大西洋公约组织}{bei3 da4xi1 yang2 gong1 yue1 zu3zhi1}
  \end{Phonetics}
\end{Entry}

\begin{Entry}{北极}{5,7}{⼔、⽊}
  \begin{Phonetics}{北极}{bei3ji2}[][HSK 5]
    \definition*{s.}{Polo Norte; Polo Ártico}
    \definition{s.}{polo norte magnético; o ponto mais setentrional da Terra, também se refere à região mais setentrional da Terra}
  \end{Phonetics}
\end{Entry}

\begin{Entry}{北京}{5,8}{⼔、⼇}
  \begin{Phonetics}{北京}{bei3 jing1}[][HSK 1]
    \definition*{s.}{Pequim (Beijing), Capital da República Popular da China | Capital da China, localizada no nordeste do país, fundada em 700 a.C., a cidade é um importante centro comercial, industrial e cultural}
  \end{Phonetics}
\end{Entry}

\begin{Entry}{北面}{5,9}{⼔、⾯}
  \begin{Phonetics}{北面}{bei3mian4}
    \definition{s.}{norte; o lado norte}
  \end{Phonetics}
\end{Entry}

\begin{Entry}{北部}{5,10}{⼔、⾢}
  \begin{Phonetics}{北部}{bei3 bu4}[][HSK 3]
    \definition{s.}{parte norte de uma região ou país}
  \end{Phonetics}
\end{Entry}

%%%%%%%%%% 半 %%%%%%%%%%
\subsection*{半}

\begin{Entry}{半}{5}{⼗}
  \begin{Phonetics}{半}{ban4}[][HSK 1]
    \definition{adv.}{parcialmente; usado antes de verbos ou adjetivos para indicar incompletude}
    \definition{num.}{(depois de um número) ``e meio'' | meio; metade | na metade; no meio | muito pouco; o mínimo}
  \end{Phonetics}
\end{Entry}

\begin{Entry}{半天}{5,4}{⼗、⼤}
  \begin{Phonetics}{半天}{ban4 tian1}[][HSK 1]
    \definition{s.}{metade do dia; metade do dia dividida pelo meio-dia | um longo tempo; bastante tempo; refere-se a um período de tempo relativamente longo (com um tom exagerado)}
  \end{Phonetics}
\end{Entry}

\begin{Entry}{半边天}{5,5,4}{⼗、⾡、⼤}
  \begin{Phonetics}{半边天}{ban4bian1tian1}[][HSK 7-9]
    \definition{s.}{metade do céu; parte do céu | mulheres modernas; mulheres (do ditado de Mao Zedong ``As mulheres podem sustentar metade do céu.'')}
  \end{Phonetics}
\end{Entry}

\begin{Entry}{半决赛}{5,6,14}{⼗、⼎、⾙}
  \begin{Phonetics}{半决赛}{ban4 jue2 sai4}[][HSK 6]
    \definition{s.}{semifinais}
  \end{Phonetics}
\end{Entry}

\begin{Entry}{半场}{5,6}{⼗、⼟}
  \begin{Phonetics}{半场}{ban4chang3}[][HSK 7-9]
    \definition{s.}{metade de um jogo ou competição (tempo) | meia quadra (no basquete)}
  \end{Phonetics}
\end{Entry}

\begin{Entry}{半年}{5,6}{⼗、⼲}
  \begin{Phonetics}{半年}{ban4 nian2}[][HSK 1]
    \definition{s.}{meio ano}
  \end{Phonetics}
\end{Entry}

\begin{Entry}{半岛}{5,7}{⼗、⼭}
  \begin{Phonetics}{半岛}{ban4dao3}[][HSK 7-9]
    \definition[个]{s.}{península; terra que se estende até o mar ou lago, cercada por água em três lados e conectada à terra em um lado}
  \end{Phonetics}
\end{Entry}

\begin{Entry}{半夜}{5,8}{⼗、⼣}
  \begin{Phonetics}{半夜}{ban4 ye4}[][HSK 2]
    \definition{s.}{no meio da noite; metade da noite | por volta da meia-noite, também se refere à madrugada}
  \end{Phonetics}
\end{Entry}

\begin{Entry}{半信半疑}{5,9,5,14}{⼗、⼈、⼗、⽦}
  \begin{Phonetics}{半信半疑}{ban4xin4-ban4yi2}[][HSK 7-9]
    \definition{expr.}{meio acreditar e meio duvidar; ser bastante duvidoso sobre (acerca de) uma coisa; estar incerto quanto ao que acreditar; meio seriamente e meio cético; não totalmente convencido; bastante desconfiado | meio acreditando, meio duvidando}
  \end{Phonetics}
\end{Entry}

\begin{Entry}{半音}{5,9}{⼗、⾳}
  \begin{Phonetics}{半音}{ban4yin1}
    \definition{s.}{semitom; na música, uma oitava é dividida em doze notas e o intervalo entre duas notas adjacentes é chamado de semitom}
  \end{Phonetics}
\end{Entry}

\begin{Entry}{半真半假}{5,10,5,11}{⼗、⼗、⼗、⼈}
  \begin{Phonetics}{半真半假}{ban4zhen1-ban4jia3}[][HSK 7-9]
    \definition{expr.}{meio verdadeiro e meio falso | meio genuíno, meio falso; parcialmente verdadeiro, parcialmente falso | meio de brincadeira, meio a sério; meio brincando}
  \end{Phonetics}
\end{Entry}

\begin{Entry}{半途而废}{5,10,6,8}{⼗、⾡、⽽、⼴}
  \begin{Phonetics}{半途而废}{ban4tu2'er2fei4}[][HSK 7-9]
    \definition{expr.}{desistir no meio do caminho; deixar inacabado; fazer algo pela metade; parar no meio do caminho; metaforicamente, parar antes de concluir uma tarefa; não terminar o que foi iniciado}
  \end{Phonetics}
\end{Entry}

\begin{Entry}{半球}{5,11}{⼗、⽟}
  \begin{Phonetics}{半球}{ban4qiu2}
    \definition{s.}{hemisfério}
  \end{Phonetics}
\end{Entry}

\begin{Entry}{半数}{5,13}{⼗、⽁}
  \begin{Phonetics}{半数}{ban4shu4}[][HSK 7-9]
    \definition{s.}{metade do total; metade}
  \end{Phonetics}
\end{Entry}

\begin{Entry}{半路}{5,13}{⼗、⾜}
  \begin{Phonetics}{半路}{ban4lu4}[][HSK 7-9]
    \definition{adv.}{a caminho | em andamento}
    \definition{s.}{na metade do caminho; no meio do caminho}
  \end{Phonetics}
\end{Entry}

%%%%%%%%%% 占 %%%%%%%%%%
\subsection*{占}

\begin{Entry}{占}{5}{⼘}
  \begin{Phonetics}{占}{zhan1}
    \definition*{s.}{Sobrenome Zhan}
    \definition{v.}{praticar adivinhação; antigamente, as pessoas usavam cascos de tartaruga e mil-folhas para prever boa ou má sorte; mais tarde, a palavra passou a se referir à previsão de boa ou má sorte por vários meios}
  \end{Phonetics}
  \begin{Phonetics}{占}{zhan4}[][HSK 2]
    \definition{v.}{tomar; apreender; ocupar; obter e possuir (terra, lugar, etc.) pela força ou outros meios impróprios | manter; inventar; constituir; explicar; estar em (uma certa posição); pertencer a (uma certa situação) | usar; ocupar; tomar; possuir}
  \end{Phonetics}
\end{Entry}

\begin{Entry}{占有}{5,6}{⼘、⽉}
  \begin{Phonetics}{占有}{zhan4 you3}[][HSK 5]
    \definition{v.}{possuir; ter; ocupar e possuir | manter; ocupar; estar em (uma determinada posição) | possuir; deter; ter; dominar}
  \end{Phonetics}
\end{Entry}

\begin{Entry}{占据}{5,11}{⼘、⼿}
  \begin{Phonetics}{占据}{zhan4 ju4}[][HSK 6]
    \definition{v.}{segurar; ocupar; assumir; tomar ou ocupar à força (uma região, lugar, etc.)}
  \end{Phonetics}
\end{Entry}

\begin{Entry}{占领}{5,11}{⼘、⾴}
  \begin{Phonetics}{占领}{zhan4ling3}[][HSK 5]
    \definition{v.}{manter; tomar; ocupar; capturar; conquistar (posições ou territórios) com forças armadas | ocupar; capturar; possuir}
  \end{Phonetics}
\end{Entry}

%%%%%%%%%% 卡 %%%%%%%%%%
\subsection*{卡}

\begin{Entry}{卡}{5}{⼘}
  \begin{Phonetics}{卡}{ka3}[][HSK 2]
    \definition{clas.}{calorias (cal)}
    \definition[张,片]{s.}{cartão; documento semelhante a um cartão | cassete; dispositivo tipo compartimento para colocar fitas cassete no gravador | caminhão}
  \end{Phonetics}
  \begin{Phonetics}{卡}{qia3}
    \definition*{s.}{Sobrenome: Qia}
    \definition[张,片]{s.}{clipe; prendedor; pinça; utensílio para prender objetos | posto de controle; posto de guarda ou posto de controle localizado em vias de comunicação importantes ou em locais com terreno acidentado}
    \definition{v.}{encravar; ficar preso; impedir de se mover | parar; controlar; impedir | pressionar firmemente com a palma da mão}
  \end{Phonetics}
\end{Entry}

\begin{Entry}{卡片}{5,4}{⼘、⽚}
  \begin{Phonetics}{卡片}{ka3pian4}[][HSK 7-9]
    \definition[张,盒,套]{s.}{cartão; pedaços de papel usados ​​para registrar diversas informações para comparação, verificação e referência}
  \end{Phonetics}
\end{Entry}

\begin{Entry}{卡片游戏}{5,4,12,6}{⼘、⽚、⽔、⼽}
  \begin{Phonetics}{卡片游戏}{ka3pian4 you2xi4}
    \definition{s.}{carta de baralho; jogos de cartas}
  \end{Phonetics}
\end{Entry}

\begin{Entry}{卡车}{5,4}{⼘、⾞}
  \begin{Phonetics}{卡车}{ka3che1}[][HSK 7-9]
    \definition[辆]{s.}{caminhão; caminhões pesados ​​para transporte de mercadorias, equipamentos, etc.}
  \end{Phonetics}
\end{Entry}

\begin{Entry}{卡车司机}{5,4,5,6}{⼘、⾞、⼝、⽊}
  \begin{Phonetics}{卡车司机}{ka3che1 si1ji1}
    \definition{s.}{motorista de caminhão}
  \end{Phonetics}
\end{Entry}

\begin{Entry}{卡通}{5,10}{⼘、⾡}
  \begin{Phonetics}{卡通}{ka3tong1}[][HSK 7-9]
    \definition[本]{s.}{Empréstimo linguístico: \emph{cartoon}; desenho animado}
  \end{Phonetics}
\end{Entry}

%%%%%%%%%% 卢 %%%%%%%%%%
\subsection*{卢}

\begin{Entry}{卢}{5}{⼘}
  \begin{Phonetics}{卢}{lu2}
    \definition*{s.}{Luxemburgo, abreviação de 卢森堡 | Sobrenome: Lu}
    \definition{s.}{Aarcaico: preta (cor)}
  \seealsoref{卢森堡}{lu2sen1bao3}
  \end{Phonetics}
\end{Entry}

\begin{Entry}{卢旺达}{5,8,6}{⼘、⽇、⾡}
  \begin{Phonetics}{卢旺达}{lu2wang4da2}
    \definition*{s.}{Ruanda}
  \end{Phonetics}
\end{Entry}

\begin{Entry}{卢森堡}{5,12,12}{⼘、⽊、⼟}
  \begin{Phonetics}{卢森堡}{lu2sen1bao3}
    \definition*{s.}{Luxemburgo}
  \end{Phonetics}
\end{Entry}

%%%%%%%%%% 印 %%%%%%%%%%
\subsection*{印}

\begin{Entry}{印}{5}{⼙}
  \begin{Phonetics}{印}{yin4}[][HSK 6]
    \definition*{s.}{Sobrenome Yin}
    \definition[个,枚,道,条]{s.}{selo; lacre; carimbo; estampilha | marca; estampa; impressão}
    \definition{v.}{imprimir; gravar | corresponder; conformar; estar em conformidade com}
  \end{Phonetics}
\end{Entry}

\begin{Entry}{印刷}{5,8}{⼙、⼑}
  \begin{Phonetics}{印刷}{yin4shua1}[][HSK 5]
    \definition{v.}{imprimir; imprimir textos, imagens, etc. em papel}
  \end{Phonetics}
\end{Entry}

\begin{Entry}{印象}{5,11}{⼙、⾗}
  \begin{Phonetics}{印象}{yin4xiang4}[][HSK 3]
    \definition[种]{s.}{impressão; marca; ideia; os vestígios deixados por coisas objetivas na mente das pessoas}
  \end{Phonetics}
\end{Entry}

%%%%%%%%%% 厉 %%%%%%%%%%
\subsection*{厉}

\begin{Entry}{厉}{5}{⼚}
  \begin{Phonetics}{厉}{li4}
    \definition*{s.}{Sobrenome: Li}
    \definition{adj.}{rigoroso; estrito | severo; sombrio; sério}
  \end{Phonetics}
\end{Entry}

\begin{Entry}{厉害}{5,10}{⼚、⼧}
  \begin{Phonetics}{厉害}{li4hai5}[][HSK 5]
    \definition{adj.}{feroz; severo; descreve uma situação como sendo muito grave | severo; duro; descreve uma pessoa que é exigente com os outros, muito severa, muitas vezes deixando os outros um pouco assustados | incrível; talentoso; impressionante; usado para avaliar a capacidade de uma pessoa ou algo que ela fez que é notável | aterrorizante; assustador; descreve animais ferozes e assustadores}
  \end{Phonetics}
\end{Entry}

%%%%%%%%%% 厺 %%%%%%%%%%
\subsection*{厺}

\begin{Entry}{厺}{5}{⼤}
  \begin{Phonetics}{厺}{qu4}
    \variantof{去}
  \end{Phonetics}
\end{Entry}

%%%%%%%%%% 去 %%%%%%%%%%
\subsection*{去}

\begin{Entry}{去}{5}{⼛}
  \begin{Phonetics}{去}{qu4}[][HSK 1]
    \definition{adj.}{passado; último; refere-se ao tempo passado (um ano)}
    \definition{adv.}{muito; extremamente; usado depois de adjetivos como 大, 多 e 远, significa 极 ou 非常}
    \definition{s.}{tom descendente, um dos quatro tons do chinês clássico e o quarto tom na pronúncia padrão do chinês moderno}
    \definition{v.}{ir; partir; sair | estar separado de | perder | remover; livrar-se de | ir (a algum lugar) para fazer algo; sair do local onde o interlocutor se encontra para outro lugar (oposto a 来) | ir para; estar indo para (fazer algo lá); usado antes de outro verbo para indicar fazer algo | desempenhar o papel de; representar o papel de; interpretar papéis em óperas | enviar; fazer ir; despachar}
    \definition{v.aux.}{usado entre uma frase verbal (ou frase preposicional) e um verbo para indicar que o primeiro é um método ou atitude e o último é um propósito | usado depois de um verbo para indicar que a ação está longe da localização do falante}
  \seealsoref{大}{da4}
  \seealsoref{多}{duo1}
  \seealsoref{非常}{fei1chang2}
  \seealsoref{极}{ji2}
  \seealsoref{来}{lai2}
  \seealsoref{远}{yuan3}
  \end{Phonetics}
\end{Entry}

\begin{Entry}{去世}{5,5}{⼛、⼀}
  \begin{Phonetics}{去世}{qu4shi4}[][HSK 3]
    \definition{v.}{(usado apenas para adultos, com conotações solenes) morrer; falecer; deixar este mundo}
  \end{Phonetics}
\end{Entry}

\begin{Entry}{去年}{5,6}{⼛、⼲}
  \begin{Phonetics}{去年}{qu4nian2}[][HSK 1]
    \definition{s.}{ano passado}
  \end{Phonetics}
\end{Entry}

\begin{Entry}{去死}{5,6}{⼛、⽍}
  \begin{Phonetics}{去死}{qu4si3}
    \definition{interj.}{Caia morto! | Vá para o Inferno!}
  \end{Phonetics}
\end{Entry}

\begin{Entry}{去掉}{5,11}{⼛、⼿}
  \begin{Phonetics}{去掉}{qu4 diao4}[][HSK 6]
    \definition{v.}{livrar-se de; tirar; acabar com; abandonar; erradicar}
  \end{Phonetics}
\end{Entry}

%%%%%%%%%% 发 %%%%%%%%%%
\subsection*{发}

\begin{Entry}{发}{5}{⼜}
  \begin{Phonetics}{发}{fa1}[][HSK 2]
    \definition*{s.}{Sobrenome: Fa}
    \definition{clas.}{bala, usada para munições e cartuchos}
    \definition{v.}{distribuir; enviar; entregar | emitir; disparar; lançar; descarregar | produzir; gerar; criar (dar origem a) | proferir; emitir; expressar | expandir; desenvolver | prosperar; prosperidade graças à aquisição de bens materiais | crescer ou expandir quando fermentado ou embebido | difundir; dispersar; espalhar | expor; descobrir; revelar | transformar-se; tornar-se; entrar em um determinado estado | demonstrar seus sentimentos; expressar (sentimentos) | sentir; ter um sentimento | começar; estabelecer | fazer com que se faça; iniciar um empreendimento; começar a agir; provocar uma ação}
  \end{Phonetics}
  \begin{Phonetics}{发}{fa4}
    \definition*{s.}{Sobrenome: Fa}
    \definition[件]{s.}{cabelo}
  \end{Phonetics}
\end{Entry}

\begin{Entry}{发火}{5,4}{⼜、⽕}
  \begin{Phonetics}{发火}{fa1/huo3}[][HSK 7-9]
    \definition{v.+compl.}{ficar com raiva; explodir; ficar furioso; perder a paciência | detonar; explodir | inflamar; pegar fogo; acender; começar a queimar}
  \end{Phonetics}
\end{Entry}

\begin{Entry}{发出}{5,5}{⼜、⼐}
  \begin{Phonetics}{发出}{fa1 chu1}[][HSK 3]
    \definition{v.}{fazer; produzir; deixar sair; ocorrer (som, dúvida, etc.) | emitir; anunciar; publicar; divulgar (ordens, instruções) | enviar (mercadorias, cartas, etc.); partir (veículos, etc.) | emitir; exalar (cheiro, calor, etc.)}
  \end{Phonetics}
\end{Entry}

\begin{Entry}{发布}{5,5}{⼜、⼱}
  \begin{Phonetics}{发布}{fa1bu4}[][HSK 5]
    \definition{v.}{emitir; publicar; liberar; anunciar; fazer ordens públicas, anúncios, notícias, etc.}
  \end{Phonetics}
\end{Entry}

\begin{Entry}{发布会}{5,5,6}{⼜、⼱、⼈}
  \begin{Phonetics}{发布会}{fa1bu4hui4}[][HSK 7-9]
    \definition[次]{s.}{coletiva de imprensa; um formato de conferência usado para divulgar notícias ou responder perguntas da mídia e do público | \emph{briefing}; atividades de exposição para promover novos produtos, etc.}
  \end{Phonetics}
\end{Entry}

\begin{Entry}{发生}{5,5}{⼜、⽣}
  \begin{Phonetics}{发生}{fa1sheng1}[][HSK 3]
    \definition{v.}{ocorrer; acontecer; tomar lugar; surgir algo que não existia antes}
  \end{Phonetics}
\end{Entry}

\begin{Entry}{发电}{5,5}{⼜、⽥}
  \begin{Phonetics}{发电}{fa1 dian4}[][HSK 6]
    \definition{s.}{geração de energia elétrica; produção de eletricidade; fornecimento de energia}
    \definition{v.}{gerar eletricidade (ou energia elétrica) | enviar um telegrama}
  \end{Phonetics}
\end{Entry}

\begin{Entry}{发电机}{5,5,6}{⼜、⽥、⽊}
  \begin{Phonetics}{发电机}{fa1dian4ji1}[][HSK 7-9]
    \definition{s.}{gerador; dínamo | alternador; gerador elétrico}
  \end{Phonetics}
\end{Entry}

\begin{Entry}{发光}{5,6}{⼜、⼉}
  \begin{Phonetics}{发光}{fa1/guang1}[][HSK 7-9]
    \definition{s.}{luminescência}
    \definition{v.+compl.}{emitir luz; brilhar; ser luminoso; cintilar}
  \end{Phonetics}
\end{Entry}

\begin{Entry}{发动}{5,6}{⼜、⼒}
  \begin{Phonetics}{发动}{fa1dong4}[][HSK 3]
    \definition{v.}{iniciar; começar; lançar | chamar à ação; mobilizar; despertar | ligar o motor; dar a partida; dar o pontapé inicial (motor de combustão interna) | estimular; colocar em ação}
  \end{Phonetics}
\end{Entry}

\begin{Entry}{发动机}{5,6,6}{⼜、⼒、⽊}
  \begin{Phonetics}{发动机}{fa1dong4ji1}
    \definition[台]{s.}{motor}
  \end{Phonetics}
\end{Entry}

\begin{Entry}{发扬}{5,6}{⼜、⼿}
  \begin{Phonetics}{发扬}{fa1yang2}[][HSK 7-9]
    \definition{v.}{desenvolver; continuar; levar adiante; desenvolver e promover (boas práticas, tradições, etc.) | aproveitar ao máximo; fazer uso total de; exercer ou mostrar (algum poder, habilidade, etc.) tanto quanto possível}
  \end{Phonetics}
\end{Entry}

\begin{Entry}{发扬光大}{5,6,6,3}{⼜、⼿、⼉、⼤}
  \begin{Phonetics}{发扬光大}{fa1yang2-guang1da4}[][HSK 7-9]
    \definition{expr.}{levar adiante; desenvolver; aprimorar; fomentar e aprimorar; dar pleno uso a; dar maior escopo a; levar a um maior nível de desenvolvimento; desenvolver para um estágio mais alto; espalhar e florescer; ``O desenvolvimento e a promoção tornam-no cada vez mais grandioso.''}
  \end{Phonetics}
\end{Entry}

\begin{Entry}{发行}{5,6}{⼜、⾏}
  \begin{Phonetics}{发行}{fa1xing2}[][HSK 5]
    \definition{v.}{emitir; liberar; publicar; emitir ou vender de publicações recém-impressas, moeda, selos, etc.}
  \end{Phonetics}
\end{Entry}

\begin{Entry}{发达}{5,6}{⼜、⾡}
  \begin{Phonetics}{发达}{fa1da2}[][HSK 3]
    \definition{adj.}{desenvolvido; florescente; (coisas) Já estão bem desenvolvidas; (negócios) prosperam}
    \definition{v.}{desenvolver; promover; florescer; a pessoa tem um bom desempenho profissional e é muito bem-sucedida}
  \end{Phonetics}
\end{Entry}

\begin{Entry}{发作}{5,7}{⼜、⼈}
  \begin{Phonetics}{发作}{fa1zuo4}[][HSK 7-9]
    \definition{v.}{sair; mostrar efeito; a doença no corpo se manifesta repentinamente ou o álcool ou as drogas fazem efeito | explodir; ter um ataque de raiva; perder a paciência porque está muito zangado ou insatisfeito}
  \end{Phonetics}
\end{Entry}

\begin{Entry}{发抖}{5,7}{⼜、⼿}
  \begin{Phonetics}{发抖}{fa1dou3}[][HSK 7-9]
    \definition{v.}{tremer; sacudir; estremecer; tremer devido ao medo, raiva ou frio}
  \end{Phonetics}
\end{Entry}

\begin{Entry}{发言}{5,7}{⼜、⾔}
  \begin{Phonetics}{发言}{fa1/yan2}[][HSK 3]
    \definition[个]{s.}{discurso; declaração; palestra; opiniões publicadas}
    \definition{v.+compl.}{falar; fazer uma declaração (discurso); expressar opinião (geralmente em reuniões)}
  \end{Phonetics}
\end{Entry}

\begin{Entry}{发言人}{5,7,2}{⼜、⾔、⼈}
  \begin{Phonetics}{发言人}{fa1 yan2 ren2}[][HSK 6]
    \definition{s.}{porta-voz}
  \end{Phonetics}
\end{Entry}

\begin{Entry}{发财}{5,7}{⼜、⾙}
  \begin{Phonetics}{发财}{fa1/cai2}[][HSK 7-9]
    \definition{v.+compl.}{ficar rico; fazer fortuna; acumular fortuna; ganhar muito dinheiro ou propriedades | trabalhar; conseguir um emprego; uma maneira educada de perguntar a alguém onde ele trabalha é dizer onde ele fez fortuna}
  \end{Phonetics}
\end{Entry}

\begin{Entry}{发奋图强}{5,8,8,12}{⼜、⼤、⼞、⼸}
  \begin{Phonetics}{发奋图强}{fa1fen4-tu2qiang2}[][HSK 7-9]
    \definition{expr.}{fazer um esforço para se tornar forte (expressão idiomática); determinado a fazer melhor | arregaçar as mangas}
  \end{Phonetics}
\end{Entry}

\begin{Entry}{发放}{5,8}{⼜、⽅}
  \begin{Phonetics}{发放}{fa1 fang4}[][HSK 6]
    \definition{v.}{conceder; estender; fornecer; (governo, organização) distribuir dinheiro ou suprimentos para os necessitados | emitir; enviar}
  \end{Phonetics}
\end{Entry}

\begin{Entry}{发明}{5,8}{⼜、⽇}
  \begin{Phonetics}{发明}{fa1ming2}[][HSK 3]
    \definition[个,项,种]{s.}{invenção; novos produtos ou métodos inventados}
    \definition{v.}{inventar; pesquisa que cria (novos produtos ou novos métodos) | expor; explicar; explicação criativa}
  \end{Phonetics}
\end{Entry}

\begin{Entry}{发明者}{5,8,8}{⼜、⽇、⽼}
  \begin{Phonetics}{发明者}{fa1ming2zhe3}
    \definition{s.}{inventor}
  \end{Phonetics}
\end{Entry}

\begin{Entry}{发泄}{5,8}{⼜、⽔}
  \begin{Phonetics}{发泄}{fa1xie4}[][HSK 7-9]
    \definition{v.}{soltar; abreviar; dar vazão a; desabafar emoções ou desejos}
  \end{Phonetics}
\end{Entry}

\begin{Entry}{发炎}{5,8}{⼜、⽕}
  \begin{Phonetics}{发炎}{fa1yan2}[][HSK 6]
    \definition{s.}{inflamação}
    \definition{v.}{irritar; inflamar; reação complexa de organismos a fatores patogênicos, como microrganismos, substâncias químicas e estímulos físicos; os sintomas sistêmicos incluem aumento da temperatura corporal, alterações na composição do sangue, vermelhidão local, inchaço, febre, dor, etc.}
  \end{Phonetics}
\end{Entry}

\begin{Entry}{发现}{5,8}{⼜、⾒}
  \begin{Phonetics}{发现}{fa1xian4}[][HSK 2]
    \definition[个,项]{s.}{descoberta; achado}
    \definition{v.}{encontrar; descobrir; detectar; identificar; através de pesquisa, exploração, etc., ver ou encontrar coisas ou leis que os antepassados não viram | descobrir; perceber; perceber; notar; estar ciente de}
  \end{Phonetics}
\end{Entry}

\begin{Entry}{发现者}{5,8,8}{⼜、⾒、⽼}
  \begin{Phonetics}{发现者}{fa1xian4 zhe3}
    \definition{s.}{descobridor}
  \end{Phonetics}
\end{Entry}

\begin{Entry}{发育}{5,8}{⼜、⾁}
  \begin{Phonetics}{发育}{fa1yu4}[][HSK 7-9]
    \definition{s.}{crescimento}
    \definition{v.}{crescer; desenvolver; a estrutura e a função dos organismos evoluem do simples para o complexo ou do imaturo para o maduro}
  \end{Phonetics}
\end{Entry}

\begin{Entry}{发表}{5,8}{⼜、⾐}
  \begin{Phonetics}{发表}{fa1biao3}[][HSK 3]
    \definition{v.}{publicar; entregar; emitir; expressar; anunciar; expressar (opiniões) ou divulgar (assuntos) ao público, verbalmente ou por escrito | publicar em jornais (artigos, etc.)}
  \end{Phonetics}
\end{Entry}

\begin{Entry}{发型}{5,9}{⼜、⼟}
  \begin{Phonetics}{发型}{fa4xing2}[][HSK 7-9]
    \definition[个,种]{s.}{penteado}
  \end{Phonetics}
\end{Entry}

\begin{Entry}{发怒}{5,9}{⼜、⼼}
  \begin{Phonetics}{发怒}{fa1 nu4}[][HSK 6]
    \definition{v.}{ficar com raiva; explodir; perder a paciência | entrar em fúria | entrar em fúria (paixão)}
  \end{Phonetics}
\end{Entry}

\begin{Entry}{发挥}{5,9}{⼜、⼿}
  \begin{Phonetics}{发挥}{fa1hui1}[][HSK 4]
    \definition{v.}{colocar em jogo; dar jogo a; dar espaço a; dar rédea solta a; revelar a natureza ou a capacidade interior | expressar; desenvolver (uma ideia, um tema, etc.); elaborar; fazer valer o ponto ou o motivo}
  \end{Phonetics}
\end{Entry}

\begin{Entry}{发觉}{5,9}{⼜、⾒}
  \begin{Phonetics}{发觉}{fa1jue2}[][HSK 5]
    \definition{v.}{vir a saber; estar ciente (de); perceber; tornar-se consciente | encontrar; detectar; perceber; descobrir}
  \end{Phonetics}
\end{Entry}

\begin{Entry}{发送}{5,9}{⼜、⾡}
  \begin{Phonetics}{发送}{fa1 song4}[][HSK 3]
    \definition{v.}{enviar; despachar | transmitir (rádio)}
  \end{Phonetics}
\end{Entry}

\begin{Entry}{发音}{5,9}{⼜、⾳}
  \begin{Phonetics}{发音}{fa1yin1}
    \definition{s.}{pronúncia}
    \definition{v.}{pronunciar}
  \end{Phonetics}
\end{Entry}

\begin{Entry}{发射}{5,10}{⼜、⼨}
  \begin{Phonetics}{发射}{fa1she4}[][HSK 5]
    \definition{v.}{subir; disparar; lançar; irradiar; projetar; descarregar; enviar algo (como uma bala, um projétil, um satélite, etc.) de um dispositivo em uma velocidade muito alta}
  \end{Phonetics}
\end{Entry}

\begin{Entry}{发展}{5,10}{⼜、⼫}
  \begin{Phonetics}{发展}{fa1zhan3}[][HSK 3]
    \definition{v.}{crescer; expandir; avançar; desenvolver; a mudança das coisas de pequeno para grande, de simples para complexo, de inferior para superior | recrutar; admitir expandir (organização, escala, etc.)}
  \end{Phonetics}
\end{Entry}

\begin{Entry}{发烧}{5,10}{⼜、⽕}
  \begin{Phonetics}{发烧}{fa1shao1}[][HSK 4]
    \definition{v.}{ter febre; a temperatura corporal normal de uma pessoa é de cerca de 37ºC; se exceder 37,5ºC, é febre}
  \end{Phonetics}
\end{Entry}

\begin{Entry}{发热}{5,10}{⼜、⽕}
  \begin{Phonetics}{发热}{fa1/re4}[][HSK 7-9]
    \definition{pref.}{piro-}
    \definition{s.}{ebulição; febre; calor; pirexia}
    \definition{v.+compl.}{emitir calor; gerar calor; aquecer; esquentar | ter febre | ser cabeça quente}
  \end{Phonetics}
\end{Entry}

\begin{Entry}{发病}{5,10}{⼜、⽧}
  \begin{Phonetics}{发病}{fa1 bing4}[][HSK 6]
    \definition{v.}{(de uma doença) avanço | patogênese; morbidade | surto (de uma doença)}
  \end{Phonetics}
\end{Entry}

\begin{Entry}{发起}{5,10}{⼜、⾛}
  \begin{Phonetics}{发起}{fa1 qi3}[][HSK 6]
    \definition{s.}{iniciador; patrocinador}
    \definition{v.}{iniciar; patrocinar; começar; lançar}
  \end{Phonetics}
\end{Entry}

\begin{Entry}{发起人}{5,10,2}{⼜、⾛、⼈}
  \begin{Phonetics}{发起人}{fa1qi3ren2}[][HSK 7-9]
    \definition{s.}{iniciador; patrocinador | membro fundador; originadores; autores | propositor}
  \end{Phonetics}
\end{Entry}

\begin{Entry}{发掘}{5,11}{⼜、⼿}
  \begin{Phonetics}{发掘}{fa1jue2}[][HSK 7-9]
    \definition{v.}{explorar; escavar; desenterrar}
  \end{Phonetics}
\end{Entry}

\begin{Entry}{发票}{5,11}{⼜、⽰}
  \begin{Phonetics}{发票}{fa1piao4}[][HSK 4]
    \definition[张,种]{s.}{conta; recibo; fatura; recibos emitidos por lojas ou outros escritórios de cobrança}
  \end{Phonetics}
\end{Entry}

\begin{Entry}{发愣}{5,12}{⼜、⼼}
  \begin{Phonetics}{发愣}{fa1/leng4}[][HSK 7-9]
    \definition{v.+compl.}{olhar fixamente; estar em transe (ou atordoado)}
  \end{Phonetics}
\end{Entry}

\begin{Entry}{发脾气}{5,12,4}{⼜、⾁、⽓}
  \begin{Phonetics}{发脾气}{fa1 pi2qi5}[][HSK 7-9]
    \definition{v.}{ficar com raiva; perder a paciência; ficar furioso; fazer barulho ou xingar porque as coisas não saem do seu jeito}
  \end{Phonetics}
\end{Entry}

\begin{Entry}{发愁}{5,13}{⼜、⼼}
  \begin{Phonetics}{发愁}{fa1/chou2}[][HSK 7-9]
    \definition{v.+compl.}{preocupar-se; ficar ansioso; ficar triste; sentir-se deprimido por não ter ideias ou soluções}
  \end{Phonetics}
\end{Entry}

\begin{Entry}{发源地}{5,13,6}{⼜、⽔、⼟}
  \begin{Phonetics}{发源地}{fa1yuan2di4}[][HSK 7-9]
    \definition{s.}{fonte; berço; lar; terra natal; lugar de origem; terra de origem | local de nascimento}[青藏高原是藏族文化的发源地。===O Planalto Qinghai-Tibete é o berço da cultura tibetana.]
  \end{Phonetics}
\end{Entry}

\begin{Entry}{发誓}{5,14}{⼜、⾔}
  \begin{Phonetics}{发誓}{fa1/shi4}[][HSK 7-9]
    \definition{v.+compl.}{jurar; prometer; fazer um juramento; expressar solenemente a resolução e a promessa de fazer o que foi acordado ou dito}
  \end{Phonetics}
\end{Entry}

\begin{Entry}{发酵}{5,14}{⼜、⾣}
  \begin{Phonetics}{发酵}{fa1/jiao4}[][HSK 7-9]
    \definition{s.}{fermentação; zimólise; compostos orgânicos complexos são decompostos em substâncias mais simples sob a ação de microrganismos}
    \definition{v.+compl.}{fermentar}
  \end{Phonetics}
\end{Entry}

\begin{Entry}{发簪}{5,18}{⼜、⽵}
  \begin{Phonetics}{发簪}{fa4zan1}
    \definition{s.}{grampo de cabelo}
  \end{Phonetics}
\end{Entry}

%%%%%%%%%% 古 %%%%%%%%%%
\subsection*{古}

\begin{Entry}{古}{5}{⼝}
  \begin{Phonetics}{古}{gu3}[][HSK 3]
    \definition*{s.}{Cuba, abreviação de 古巴 | Sobrenome: Gu}
    \definition{adj.}{antigo; milenar; ancestral; secular | simples e sincero | velho | arcaico}
    \definition{pref.}{(distante no tempo; antigo; primitivo) paleo-; arqueo-}
    \definition{s.}{tempos antigos (oposto a 今) | antiguidade; ancestralidade | livros ou ortodoxias dos sábios antigos, a tradição do Tao | uma forma de poesia pré-Tang}
  \seealsoref{古巴}{gu3ba1}
  \seealsoref{今}{jin1}
  \end{Phonetics}
\end{Entry}

\begin{Entry}{古人}{5,2}{⼝、⼈}
  \begin{Phonetics}{古人}{gu3ren2}[][HSK 7-9]
    \definition{s.}{os antigos; antepassados ​​(em oposição a 今人) | pessoas dos tempos antigos | espécies humanas extintas, como \emph{Homo erectus} ou \emph{Homo neanderthalensis} | Lliterário: pessoa falecida}
  \seealsoref{今人}{jin1ren2}
  \end{Phonetics}
\end{Entry}

\begin{Entry}{古今中外}{5,4,4,5}{⼝、⼈、⼁、⼣}
  \begin{Phonetics}{古今中外}{gu3jin1-zhong1wai4}[][HSK 7-9]
    \definition{expr.}{``Antigo e moderno, chinês e estrangeiro.''; em todos os tempos e em todas as terras}
  \end{Phonetics}
\end{Entry}

\begin{Entry}{古巴}{5,4}{⼝、⼰}
  \begin{Phonetics}{古巴}{gu3ba1}
    \definition*{s.}{Cuba}
  \end{Phonetics}
\end{Entry}

\begin{Entry}{古代}{5,5}{⼝、⼈}
  \begin{Phonetics}{古代}{gu3dai4}[][HSK 3]
    \definition{s.}{tempos antigos; o passado é um período muito distante do presente (diferentemente de 近代 e 现代); na periodização histórica chinesa, geralmente se refere ao período anterior a meados do século XIX | sociedade antiga; sociedade primitiva; refere-se especificamente à era da sociedade escravista (às vezes também inclui a era comunal primitiva) |antigamente; tempos antigos; no passado}
  \seealsoref{近代}{jin4dai4}
  \seealsoref{现代}{xian4dai4}
  \end{Phonetics}
\end{Entry}

\begin{Entry}{古朴}{5,6}{⼝、⽊}
  \begin{Phonetics}{古朴}{gu3pu3}[][HSK 7-9]
    \definition{adj.}{simples e pouco sofisticado (arte, arquitetura, etc.); descreve a aparência sem muita decoração ou modificação, dando às pessoas uma sensação antiga, e também descreve o comportamento das pessoas como simples e sincero}
  \end{Phonetics}
\end{Entry}

\begin{Entry}{古老}{5,6}{⼝、⽼}
  \begin{Phonetics}{古老}{gu3 lao3}[][HSK 5]
    \definition{adj.}{antigo; antiquado; histórico}
  \end{Phonetics}
\end{Entry}

\begin{Entry}{古典}{5,8}{⼝、⼋}
  \begin{Phonetics}{古典}{gu3dian3}[][HSK 6]
    \definition{adj.}{clássico; descreve uma obra ou coisa como tendo características tradicionais ou exemplares}
    \definition{s.}{os clássicos}
  \end{Phonetics}
\end{Entry}

\begin{Entry}{古怪}{5,8}{⼝、⼼}
  \begin{Phonetics}{古怪}{gu3guai4}[][HSK 7-9]
    \definition{adj.}{pitoresco; excêntrico; esquisito; estranho; muito diferente do habitual, surpreendente; desconhecido e raro; raro e inovador}
  \end{Phonetics}
\end{Entry}

\begin{Entry}{古城}{5,9}{⼝、⼟}
  \begin{Phonetics}{古城}{gu3cheng2}
    \definition{s.}{cidade antiga}
  \end{Phonetics}
\end{Entry}

\begin{Entry}{古迹}{5,9}{⼝、⾡}
  \begin{Phonetics}{古迹}{gu3ji4}[][HSK 7-9]
    \definition[处]{s.}{sítio histórico; local de interesse histórico; construções antigas ou outras relíquias de grande importância}
  \end{Phonetics}
\end{Entry}

\begin{Entry}{古铜色}{5,11,6}{⼝、⾦、⾊}
  \begin{Phonetics}{古铜色}{gu3tong2 se4}
    \definition{s.}{cor bronze}
  \end{Phonetics}
\end{Entry}

\begin{Entry}{古董}{5,12}{⼝、⾋}
  \begin{Phonetics}{古董}{gu3dong3}[][HSK 7-9]
    \definition{adj.}{antiquado; uma metáfora para coisas ultrapassadas ou pessoas teimosas}
    \definition[件,个]{s.}{objeto de arte; raridade; antiguidade; artefatos transmitidos desde os tempos antigos podem ser usados ​​como referência para a compreensão da cultura antiga}
  \end{Phonetics}
\end{Entry}

\begin{Entry}{古装}{5,12}{⼝、⾐}
  \begin{Phonetics}{古装}{gu3 zhuang1}
    \definition[套]{s.}{traje antigo; roupas tradicionais; roupas de estilo antigo}
  \end{Phonetics}
\end{Entry}

%%%%%%%%%% 句 %%%%%%%%%%
\subsection*{句}

\begin{Entry}{句}{5}{⼝}
  \begin{Phonetics}{句}{gou4}
    \variantof{勾}
  \end{Phonetics}
  \begin{Phonetics}{句}{ju4}[][HSK 2]
    \definition{clas.}{para sentenças, frases ou linhas de versos}
    \definition{s.}{frase; sentença}
  \end{Phonetics}
\end{Entry}

\begin{Entry}{句子}{5,3}{⼝、⼦}
  \begin{Phonetics}{句子}{ju4zi5}[][HSK 2]
    \definition[个,句]{s.}{sentença; uma unidade linguística composta por palavras ou frases que expressa um significado completo}
  \end{Phonetics}
\end{Entry}

%%%%%%%%%% 另 %%%%%%%%%%
\subsection*{另}

\begin{Entry}{另}{5}{⼝}
  \begin{Phonetics}{另}{ling4}[][HSK 6]
    \definition*{s.}{Sobrenome: Ling}
    \definition{adv.}{além disso; indica que está fora do escopo da declaração | no lugar de; em vez de}
    \definition{pron.}{(com substantivos) outro; diferente; refere-se a pessoas ou coisas fora do escopo do que é dito}
  \end{Phonetics}
\end{Entry}

\begin{Entry}{另一方面}{5,1,4,9}{⼝、⼀、⽅、⾯}
  \begin{Phonetics}{另一方面}{ling4 yi4 fang1 mian4}[][HSK 3]
    \definition{adv./conj.}{outro aspecto | por outro lado; por sua vez; em contrapartida}
  \end{Phonetics}
\end{Entry}

\begin{Entry}{另外}{5,5}{⼝、⼣}
  \begin{Phonetics}{另外}{ling4wai4}[][HSK 3]
    \definition{adv.}{além disso; em adição; ademais; além do mais; além de que; além do que já foi dito}
    \definition{conj.}{além disso; usada entre duas ou mais frases, indica algo além do que foi mencionado anteriormente}
    \definition{pron.}{outro; além das pessoas ou coisas mencionadas anteriormente}
  \end{Phonetics}
\end{Entry}

%%%%%%%%%% 只 %%%%%%%%%%
\subsection*{只}

\begin{Entry}{只}{5}{⼝}
  \begin{Phonetics}{只}{zhi1}[][HSK 3]
    \definition{adj.}{solteiro; solitário; único; muito raro}
    \definition{clas.}{usado para um de um par | usado para animais pequenos (pássaros, gatos, cães, etc.) | usado para certos utensílios, aparelhos | usado para navios}
  \end{Phonetics}
  \begin{Phonetics}{只}{zhi3}[][HSK 2]
    \definition{adv.}{somente; apenas; meramente | simplesmente; usado para limitar o escopo, indicando que não há nada além disso, equivalente a 仅仅}
  \seealsoref{仅仅}{jin3 jin3}
  \end{Phonetics}
\end{Entry}

\begin{Entry}{只不过}{5,4,6}{⼝、⼀、⾡}
  \begin{Phonetics}{只不过}{zhi3 bu2 guo4}[][HSK 5]
    \definition{adv.}{somente; apenas; meramente; não mais do que}
  \end{Phonetics}
\end{Entry}

\begin{Entry}{只见}{5,4}{⼝、⾒}
  \begin{Phonetics}{只见}{zhi3 jian4}[][HSK 5]
    \definition{v.}{somente ver; ver; só vi, e de repente percebi uma certa situação}
  \end{Phonetics}
\end{Entry}

\begin{Entry}{只好}{5,6}{⼝、⼥}
  \begin{Phonetics}{只好}{zhi3hao3}[][HSK 3]
    \definition{v.}{ter que; ser forçado a; não ter escolha a não ser; significa que só pode ser assim, não há outra opção}
  \end{Phonetics}
\end{Entry}

\begin{Entry}{只有}{5,6}{⼝、⽉}
  \begin{Phonetics}{只有}{zhi3 you3}[][HSK 3]
    \definition{adv.}{somente; tem que; forçado a}
    \definition{conj.}{somente se; conecta frases, expressa condições necessárias, geralmente corresponde a 才 e 方}
  \seealsoref{才}{cai2}
  \seealsoref{方}{fang1}
  \end{Phonetics}
\end{Entry}

\begin{Entry}{只有……才……}{5,6,3}{⼝、⽉、⼿}
  \begin{Phonetics}{只有……才……}{zhi3you3 cai2}
    \definition{conj.}{só se\dots então\dots}
  \end{Phonetics}
\end{Entry}

\begin{Entry}{只身}{5,7}{⼝、⾝}
  \begin{Phonetics}{只身}{zhi1shen1}
    \definition{adv.}{sozinho | por si só}
  \end{Phonetics}
\end{Entry}

\begin{Entry}{只怕}{5,8}{⼝、⼼}
  \begin{Phonetics}{只怕}{zhi3pa4}
    \definition{adv.}{receio que\dots | talvez | muito provavelmente}
  \end{Phonetics}
\end{Entry}

\begin{Entry}{只是}{5,9}{⼝、⽇}
  \begin{Phonetics}{只是}{zhi3 shi4}[][HSK 3]
    \definition{adv.}{somente; meramente; apenas; expressa ênfase limitada a uma determinada situação ou âmbito}
    \definition{conj.}{somente; mas; exceto que; conecta frases, indicando uma ligeira transição, equivalente a 不过}
  \seealsoref{不过}{bu2guo4}
  \end{Phonetics}
\end{Entry}

\begin{Entry}{只要}{5,9}{⼝、⾑}
  \begin{Phonetics}{只要}{zhi3yao4}[][HSK 2]
    \definition{conj.}{desde que; se apenas; contanto que; indica condições necessárias (就 ou 可 são frequentemente usados depois)}
  \seealsoref{便}{bian4}
  \seealsoref{就}{jiu4}
  \end{Phonetics}
\end{Entry}

\begin{Entry}{只要……就……}{5,9,12}{⼝、⾑、⼪}
  \begin{Phonetics}{只要……就……}{zhi3yao4 jiu4}
    \definition{conj.}{contanto que/desde que/se somente\dots, então\dots}
  \end{Phonetics}
\end{Entry}

\begin{Entry}{只消}{5,10}{⼝、⽔}
  \begin{Phonetics}{只消}{zhi3xiao1}
    \definition{conj.}{desde que}
  \end{Phonetics}
\end{Entry}

\begin{Entry}{只能}{5,10}{⼝、⾁}
  \begin{Phonetics}{只能}{zhi3 neng2}[][HSK 2]
    \definition{adv.}{só pode; obrigado a fazer algo; isso significa que devido à limitação da capacidade pessoal ou às condições objetivas, não há outra escolha senão esta}
  \end{Phonetics}
\end{Entry}

\begin{Entry}{只读}{5,10}{⼝、⾔}
  \begin{Phonetics}{只读}{zhi3du2}
    \definition{s.}{somente leitura (computação) | \emph{read-only}}
  \end{Phonetics}
\end{Entry}

\begin{Entry}{只顾}{5,10}{⼝、⾴}
  \begin{Phonetics}{只顾}{zhi3 gu4}[][HSK 6]
    \definition{adv.}{meramente; simplesmente; apenas se importa com; indica que a atenção está focada em apenas um aspecto}
    \definition{v.}{considerar apenas uma coisa}
  \end{Phonetics}
\end{Entry}

\begin{Entry}{只得}{5,11}{⼝、⼻}
  \begin{Phonetics}{只得}{zhi3 de5}[][HSK 6]
    \definition{v.}{não ter alternativa senão; obrigado a; ter que; ser obrigado a}
  \end{Phonetics}
\end{Entry}

\begin{Entry}{只管}{5,14}{⼝、⽵}
  \begin{Phonetics}{只管}{zhi3 guan3}[][HSK 6]
    \definition{adv.}{por todos os meios; expressa incentivo para que os outros façam algo com confiança, sem se preocuparem com outras coisas | apenas; simplesmente; significa fazer uma coisa com seriedade, sem se preocupar com outras coisas}
  \end{Phonetics}
\end{Entry}

%%%%%%%%%% 叫 %%%%%%%%%%
\subsection*{叫}

\begin{Entry}{叫}{5}{⼝}
  \begin{Phonetics}{叫}{jiao4}[][HSK 1,3]
    \definition{adj.}{macho (animal)}
    \definition{prep.}{usado em frases passivas; introduz o agente da ação; equivalente a 被 | combinado com 看, 说; usado para expressar suas ideias e pontos de vista}
    \definition{v.}{chorar; gritar; berrar | nomear; chamar | chamar; chamar a atenção | cumprimentar; saudar; dizer olá | pedir; ordenar; licitar | permitir; concordar com algo; concordar em fazer algo | contratar; encomendar; comprar o que você precisa}
  \seealsoref{被}{bei4}
  \seealsoref{看}{kan4}
  \seealsoref{说}{shuo1}
  \end{Phonetics}
\end{Entry}

\begin{Entry}{叫好}{5,6}{⼝、⼥}
  \begin{Phonetics}{叫好}{jiao4/hao3}[][HSK 7-9]
    \definition{v.+compl.}{aplaudir; gritar ``Bravo!''; gritar ``Muito bem!'' | aplaudir | torcer}
  \end{Phonetics}
\end{Entry}

\begin{Entry}{叫作}{5,7}{⼝、⼈}
  \begin{Phonetics}{叫作}{jiao4 zuo4}[][HSK 2]
    \definition{v.}{ser chamado de; ser conhecido como}
  \end{Phonetics}
\end{Entry}

\begin{Entry}{叫板}{5,8}{⼝、⽊}
  \begin{Phonetics}{叫板}{jiao4/ban3}[][HSK 7-9]
    \definition{v.+compl.}{Coloquial: desafiar | sinalizar aos músicos (na ópera chinesa, prolongando uma palavra falada antes de iniciar uma canção)}
  \end{Phonetics}
\end{Entry}

%%%%%%%%%% 召 %%%%%%%%%%
\subsection*{召}

\begin{Entry}{召}{5}{⼝}
  \begin{Phonetics}{召}{shao4}
    \definition*{s.}{Sobrenome: Shao}
    \definition{s.}{(frequentemente em nomes de lugares mongóis) templo; mosteiro}
    \definition{v.}{convocar; intimar; invocar}
  \end{Phonetics}
  \begin{Phonetics}{召}{zhao4}
    \definition*{s.}{Sobrenome Zhao}
    \definition{s.}{templo}
    \definition{v.}{chamar; intimar; convocar; invocar}
  \end{Phonetics}
\end{Entry}

\begin{Entry}{召开}{5,4}{⼝、⼶}
  \begin{Phonetics}{召开}{zhao4kai1}[][HSK 4]
    \definition{v.}{convocar; chamar pessoas para uma reunião; realizar (uma reunião)}
  \end{Phonetics}
\end{Entry}

%%%%%%%%%% 叮 %%%%%%%%%%
\subsection*{叮}

\begin{Entry}{叮}{5}{⼝}
  \begin{Phonetics}{叮}{ding1}
    \definition{v.}{picar; ferroar | dizer ou perguntar novamente para ter certeza; verificar; insistir; certificar-se | sondar; perseguir}
  \end{Phonetics}
\end{Entry}

\begin{Entry}{叮嘱}{5,15}{⼝、⼝}
  \begin{Phonetics}{叮嘱}{ding1zhu3}[][HSK 7-9]
    \definition{v.}{advertir; exortar; insistir repetidamente; instruir repetidamente; dizer à outra pessoa para lembrar o que deve e o que não deve ser feito}
  \end{Phonetics}
\end{Entry}

%%%%%%%%%% 可 %%%%%%%%%%
\subsection*{可}

\begin{Entry}{可}{5}{⼝}
  \begin{Phonetics}{可}{ke3}[][HSK 5]
    \definition*{s.}{Sobrenome: Ke}
    \definition{adv.}{indica ênfase | indica o fortalecimento de perguntas retóricas | indica um tom de questionamento mais forte | sobre; a respeito de}
    \definition{conj.}{mas; ainda}
    \definition{v.}{aprovar; concordar com | poder; permitir; ser capaz de | precisar (fazer); valer a pena (fazer); merecer | ajustar; adequar | estar pronto para; estar disposto a; pretender}
  \end{Phonetics}
  \begin{Phonetics}{可}{ke4}
    \definition{s.}{governante supremo de uma tribo nômade do norte; Khan (可汗), título do governante supremo dos antigos grupos étnicos xianbei, turco, uigur e mongol}
  \seealsoref{可汗}{ke4han2}
  \end{Phonetics}
\end{Entry}

\begin{Entry}{可口}{5,3}{⼝、⼝}
  \begin{Phonetics}{可口}{ke3kou3}[][HSK 7-9]
    \definition{adj.}{bom; saboroso; apetitoso; gostoso de comer; (alimento ou bebida) tem uma textura agradável e um sabor bom}
  \end{Phonetics}
\end{Entry}

\begin{Entry}{可口可乐}{5,3,5,5}{⼝、⼝、⼝、⼃}
  \begin{Phonetics}{可口可乐}{ke3kou3ke3le4}
    \definition*{s.}{Empréstimo linguístico: Coca-Cola}
  \end{Phonetics}
\end{Entry}

\begin{Entry}{可不}{5,4}{⼝、⼀}
  \begin{Phonetics}{可不}{ke3bu4}
    \definition{adv.}{certamente; exatamente; em uma conversa, significa concordar plenamente com o que a outra pessoa diz}
  \seealsoref{可不是}{ke3bu2shi4}
  \end{Phonetics}
\end{Entry}

\begin{Entry}{可不是}{5,4,9}{⼝、⼀、⽇}
  \begin{Phonetics}{可不是}{ke3bu2shi4}[][HSK 7-9]
    \definition{adv.}{certamente; exatamente; é assim mesmo; para expressar concordância, é frequentemente usado como uma frase independente, mas também pode ser expresso como 可不}
  \seealsoref{可不}{ke3bu4}
  \end{Phonetics}
\end{Entry}

\begin{Entry}{可以}{5,4}{⼝、⼈}
  \begin{Phonetics}{可以}{ke3yi3}[][HSK 2]
    \definition{adj.}{aceitável; nada mal; muito bom | impressionante; espantoso; tremendo}
    \definition{v.}{poder; ter condições, capacidade e tempo para fazer algo ou ter alguma utilidade | permitir; poder | valer a pena fazer; considerar que vale a pena, recomendar fazer algo}
  \end{Phonetics}
\end{Entry}

\begin{Entry}{可见}{5,4}{⼝、⾒}
  \begin{Phonetics}{可见}{ke3jian4}[][HSK 4]
    \definition{adj.}{visível; concebível; algo que é óbvio ou evidente}
    \definition{conj.}{isso mostra; isto prova; é, portanto, claro (ou evidente, óbvio) que}
    \definition{v.}{ser ou estar visível ; ser ou estar claro}
  \end{Phonetics}
\end{Entry}

\begin{Entry}{可乐}{5,5}{⼝、⼃}
  \begin{Phonetics}{可乐}{ke3 le4}[][HSK 3]
    \definition*[罐,杯,瓶,听,口]{s.}{\emph{coke}; coca; coca-cola}
    \definition{adj.}{engraçado; divertido; risível}
  \end{Phonetics}
\end{Entry}

\begin{Entry}{可卡因}{5,5,6}{⼝、⼘、⼞}
  \begin{Phonetics}{可卡因}{ke3ka3yin1}
    \definition{s.}{(empréstimo linguístico) cocaína}
  \end{Phonetics}
\end{Entry}

\begin{Entry}{可汗}{5,6}{⼝、⽔}
  \begin{Phonetics}{可汗}{ke4han2}
    \definition{s.}{khan (empréstimo linguístico); cham}
  \end{Phonetics}
\end{Entry}

\begin{Entry}{可行}{5,6}{⼝、⾏}
  \begin{Phonetics}{可行}{ke3xing2}[][HSK 7-9]
    \definition{adj.}{viável; praticável; funcional}
  \end{Phonetics}
\end{Entry}

\begin{Entry}{可观}{5,6}{⼝、⾒}
  \begin{Phonetics}{可观}{ke3guan1}[][HSK 7-9]
    \definition{adj.}{que vale a pena ver; que vale a pena assistir | considerável; impressionante; substancial; de nível alcançado; de grau relativamente alto}
  \end{Phonetics}
\end{Entry}

\begin{Entry}{可怕}{5,8}{⼝、⼼}
  \begin{Phonetics}{可怕}{ke3pa4}[][HSK 2]
    \definition{adj.}{assustador; terrível; hediondo; medonho; horrível; aterrorizante}
    \definition{adv.}{terrivelmente}
  \end{Phonetics}
\end{Entry}

\begin{Entry}{可怜}{5,8}{⼝、⼼}
  \begin{Phonetics}{可怜}{ke3lian2}[][HSK 5]
    \definition{adj.}{pobre; lamentável; lastimável | miserável (de quantidade ou qualidade); descreve um número pequeno ou um lugar tão pequeno que não vale a pena falar sobre ele}
    \definition{v.}{ter pena; ter piedade de; ter simpatia por pessoas que tiveram coisas muito ruins acontecendo com elas}
  \end{Phonetics}
\end{Entry}

\begin{Entry}{可信}{5,9}{⼝、⼈}
  \begin{Phonetics}{可信}{ke3xin4}[][HSK 7-9]
    \definition{adj.}{confiável; em que (quem) se pode acreditar}
  \end{Phonetics}
\end{Entry}

\begin{Entry}{可是}{5,9}{⼝、⽇}
  \begin{Phonetics}{可是}{ke3shi4}[][HSK 2]
    \definition{adv.}{de fato (usado para dar ênfase), equivalente a 的确}
    \definition{conj.}{mas; no entanto; contudo; conecta frases, expressa uma relação de transição, equivalente a 但是}
  \seealsoref{但是}{dan4 shi4}
  \seealsoref{的确}{di2que4}
  \end{Phonetics}
\end{Entry}

\begin{Entry}{可贵}{5,9}{⼝、⾙}
  \begin{Phonetics}{可贵}{ke3gui4}[][HSK 7-9]
    \definition{adj.}{estimado; valioso; valorizado; louvável; recomendável}
  \end{Phonetics}
\end{Entry}

\begin{Entry}{可乘之机}{5,10,3,6}{⼝、⽲、⼂、⽊}
  \begin{Phonetics}{可乘之机}{ke3cheng2zhi1ji1}[][HSK 7-9]
    \definition{expr.}{uma oportunidade para aproveitar; oportunidade a explorar; abertura}
  \end{Phonetics}
\end{Entry}

\begin{Entry}{可恶}{5,10}{⼝、⼼}
  \begin{Phonetics}{可恶}{ke3wu4}[][HSK 7-9]
    \definition{adj.}{odioso; abominável; detestável; repugnante; extremamente irritante}
  \end{Phonetics}
\end{Entry}

\begin{Entry}{可爱}{5,10}{⼝、⽖}
  \begin{Phonetics}{可爱}{ke3'ai4}[][HSK 2]
    \definition{adj.}{adorável; simpático; encantador | bonitinho; adorável | amado; querido; encantador; cativante; relacionamento próximo, sentimentos profundos | fofo; bonito}
  \end{Phonetics}
\end{Entry}

\begin{Entry}{可笑}{5,10}{⼝、⽵}
  \begin{Phonetics}{可笑}{ke3xiao4}[][HSK 7-9]
    \definition{adj.}{absurdo; risível; ridículo; hilário; engraçado}
  \end{Phonetics}
\end{Entry}

\begin{Entry}{可耻}{5,10}{⼝、⽿}
  \begin{Phonetics}{可耻}{ke3chi3}[][HSK 7-9]
    \definition{adj.}{vergonhoso; ignominioso; desonroso}
  \end{Phonetics}
\end{Entry}

\begin{Entry}{可能}{5,10}{⼝、⾁}
  \begin{Phonetics}{可能}{ke3neng2}[][HSK 2]
    \definition{adj.}{possível}
    \definition{adv.}{possivelmente}
    \definition[种]{s.}{possibilidade; tendências ou oportunidades que podem se tornar realidade}
  \end{Phonetics}
\end{Entry}

\begin{Entry}{可惜}{5,11}{⼝、⼼}
  \begin{Phonetics}{可惜}{ke3xi1}[][HSK 5]
    \definition{adj.}{é uma pena; é muito ruim; é lamentável}
    \definition{adv.}{infelizmente}
  \end{Phonetics}
\end{Entry}

\begin{Entry}{可谓}{5,11}{⼝、⾔}
  \begin{Phonetics}{可谓}{ke3wei4}[][HSK 7-9]
    \definition{s.}{poder ser dito; poder ser considerado; poder ser chamado de}[这机会可谓是千载难逢。===Pode-se dizer que esta é a oportunidade da sua vida.]
  \end{Phonetics}
\end{Entry}

\begin{Entry}{可悲}{5,12}{⼝、⽕}
  \begin{Phonetics}{可悲}{ke3bei1}[][HSK 7-9]
    \definition{adj.}{triste; lamentável; deplorável; de partir o coração}
  \end{Phonetics}
\end{Entry}

\begin{Entry}{可编程}{5,12,12}{⼝、⽷、⽲}
  \begin{Phonetics}{可编程}{ke3bian1cheng2}
    \definition{adj.}{programável}
  \end{Phonetics}
\end{Entry}

\begin{Entry}{可想而知}{5,13,6,8}{⼝、⼼、⽽、⽮}
  \begin{Phonetics}{可想而知}{ke3xiang3'er2zhi1}[][HSK 7-9]
    \definition{expr.}{é fácil de imaginar; é óbvio; claramente; como você pode imaginar; você pode imaginar isso sem precisar de explicações; pode-se muito bem imaginar}
  \end{Phonetics}
\end{Entry}

\begin{Entry}{可歌可泣}{5,14,5,8}{⼝、⽋、⼝、⽔}
  \begin{Phonetics}{可歌可泣}{ke3ge1-ke3qi4}[][HSK 7-9]
    \definition{expr.}{``Uma história digna de elogios e lágrimas.''; capaz de evocar elogios e lágrimas; tanto alegre quanto trágico; comovente e digno de uma canção; tocante e digno de uma canção; digno de elogios, comovente até às lágrimas refere-se a feitos trágicos e heroicos que tocam profundamente as pessoas; digno de ser louvado e de comover até às lágrimas}
  \end{Phonetics}
\end{Entry}

\begin{Entry}{可疑}{5,14}{⼝、⽦}
  \begin{Phonetics}{可疑}{ke3yi2}[][HSK 7-9]
    \definition{adj.}{duvidoso; suspeito; questionável}
  \end{Phonetics}
\end{Entry}

\begin{Entry}{可靠}{5,15}{⼝、⾮}
  \begin{Phonetics}{可靠}{ke3kao4}[][HSK 3]
    \definition{adj.}{confiável; digno de confiança | verdadeiro; autêntico; descrever notícias e outras informações como verdadeiras, de modo que as pessoas possam acreditar nelas}
  \end{Phonetics}
\end{Entry}

\begin{Entry*}{可擦写可编程只读存储器}{5,17,5,5,12,12,5,10,6,12,16}{⼝、⼿、⼍、⼝、⽷、⽲、⼝、⾔、⼦、⼈、⼝}
  \begin{Phonetics}{可擦写可编程只读存储器}{ke3 ca1 xie3 ke3 bian1cheng2 zhi1 du2 cun2chu3qi4}
    \definition{s.}{EPROM (\emph{erasable programmable read-only memory})}
  \end{Phonetics}
\end{Entry*}

%%%%%%%%%% 台 %%%%%%%%%%
\subsection*{台}

\begin{Entry}{台}{5}{⼝}
  \begin{Phonetics}{台}{tai2}[][HSK 3]
    \definition*{s.}{Sobrenome: Tai}
    \definition{clas.}{usado para certas máquinas, aparelhos, instrumentos, etc | usado para uma performance completa, como drama, música e dança}
    \definition{s.}{torre | plataforma; palco | suporte; pedestal | qualquer coisa em forma de plataforma ou palco | mesa; escrivaninha | estação de transmissão; refere-se a estações de rádio | um serviço telefônico especial; refere-se à estação telefônica | ``seu'' (um termo respeitoso usado antigamente para se dirigir a alguém) | tufão}
  \end{Phonetics}
\end{Entry}

\begin{Entry}{台上}{5,3}{⼝、⼀}
  \begin{Phonetics}{台上}{tai2 shang4}[][HSK 4]
    \definition{s.}{no palco}
  \end{Phonetics}
\end{Entry}

\begin{Entry}{台下}{5,3}{⼝、⼀}
  \begin{Phonetics}{台下}{tai2xia4}
    \definition{s.}{platéia | fora do palco}
  \end{Phonetics}
\end{Entry}

\begin{Entry}{台风}{5,4}{⼝、⾵}
  \begin{Phonetics}{台风}{tai2feng1}[][HSK 5]
    \definition[场,阵,级]{s.}{tufão; classificação de um ciclone tropical ocorrido no oeste do Pacífico Norte | postura; presença de palco; comportamento ou estilo que os atores demonstram no palco}
  \end{Phonetics}
\end{Entry}

\begin{Entry}{台灯}{5,6}{⼝、⽕}
  \begin{Phonetics}{台灯}{tai2 deng1}[][HSK 6]
    \definition[个,盏]{s.}{luminária de mesa; luminária de leitura; uma luminária com base para uso sobre uma mesa}
  \end{Phonetics}
\end{Entry}

\begin{Entry}{台阶}{5,6}{⼝、⾩}
  \begin{Phonetics}{台阶}{tai2jie1}[][HSK 4]
    \definition[个,级]{s.}{escada; escadaria | passos; metáfora para uma maneira ou oportunidade de evitar constrangimentos causados ​​por um impasse | nova fase; novo nível; novo patamar; metáfora para novas conquistas ou novos patamares alcançados no estudo ou no trabalho}
  \end{Phonetics}
\end{Entry}

%%%%%%%%%% 右 %%%%%%%%%%
\subsection*{右}

\begin{Entry}{右}{5}{⼝}
  \begin{Phonetics}{右}{you4}[][HSK 1]
    \definition*{s.}{Sobrenome You}
    \definition{adj.}{conservador; reacionário}
    \definition{s.}{a direita; o lado direito | oeste; na antiguidade, referia-se especificamente à direção oeste (com base na orientação para o sul) | o lado direito como o lado de precedência; posição ou nível mais elevado (os antigos costumavam considerar a direita como mais respeitável)}
    \definition{v.}{favorecer; apoiar; reverenciar}
  \end{Phonetics}
\end{Entry}

\begin{Entry}{右手}{5,4}{⼝、⼿}
  \begin{Phonetics}{右手}{you4shou3}
    \definition{s.}{mão direita | lado direito}
  \end{Phonetics}
\end{Entry}

\begin{Entry}{右边}{5,5}{⼝、⾡}
  \begin{Phonetics}{右边}{you4bian5}[][HSK 1]
    \definition{s.}{a direita; o lado direito; do lado direito}
  \end{Phonetics}
\end{Entry}

\begin{Entry}{右侧}{5,8}{⼝、⼈}
  \begin{Phonetics}{右侧}{you4ce4}
    \definition{s.}{lateral direita | lado direito}
  \end{Phonetics}
\end{Entry}

\begin{Entry}{右转}{5,8}{⼝、⾞}
  \begin{Phonetics}{右转}{you4zhuan3}
    \definition{v.}{virar à direita}
  \end{Phonetics}
\end{Entry}

\begin{Entry}{右面}{5,9}{⼝、⾯}
  \begin{Phonetics}{右面}{you4mian4}
    \definition{s.}{lado direito}
  \end{Phonetics}
\end{Entry}

\begin{Entry}{右倾}{5,10}{⼝、⼈}
  \begin{Phonetics}{右倾}{you4qing1}
    \definition{adj.}{conservador | reacionário}
  \end{Phonetics}
\end{Entry}

\begin{Entry}{右袒}{5,10}{⼝、⾐}
  \begin{Phonetics}{右袒}{you4tan3}
    \definition{v.}{ser tendencioso | ser parcial | favorecer um lado | tomar partido}
  \end{Phonetics}
\end{Entry}

%%%%%%%%%% 叶 %%%%%%%%%%
\subsection*{叶}

\begin{Entry}{叶}{5}{⼝}
  \begin{Phonetics}{叶}{ye4}
    \definition*{s.}{Sobrenome Ye}
    \definition[枝]{s.}{folha; folhagem | coisa parecida com uma folha | página; folha | parte de um período histórico; segmentos de período mais longos | lóbulo; lóbulos do cérebro, pulmões e fígado}
  \end{Phonetics}
\end{Entry}

\begin{Entry}{叶子}{5,3}{⼝、⼦}
  \begin{Phonetics}{叶子}{ye4zi5}[][HSK 4]
    \definition[片]{s.}{folha; termo genérico para as folhas de uma planta}
  \end{Phonetics}
\end{Entry}

%%%%%%%%%% 号 %%%%%%%%%%
\subsection*{号}

\begin{Entry}{号}{5}{⼝}
  \begin{Phonetics}{号}{hao2}
    \definition{v.}{uivar; gritar; gritar em voz alta e prolongada | lamentar; chorar alto | uivar; (vento) assobiar, assoviar}
  \end{Phonetics}
  \begin{Phonetics}{号}{hao4}[][HSK 1]
    \definition{clas.}{usado para o número de pessoas |  tipo; espécie; classificação | usado para pessoas ou negócios; número de vezes utilizado para transações}
    \definition[把]{s.}{nome | nome presumido; nome alternativo; pseudônimo; apelido | casa de negócios; loja | marca; sinal; sinalização | número | data | ordem; no exército, as ordens são transmitidas verbalmente ou por meio de clarins | qualquer instrumento de sopro e latão; trombeta usada no exército ou em bandas | qualquer coisa usada como buzina | chamada de corneta; qualquer chamada feita em uma corneta; usar um apito para emitir um som com um significado específico | pessoa em uma condição especial; pessoas que se encontram em uma situação especial}
    \definition{suf.}{sufixo de navio}
    \definition{v.}{marcar; fazer uma marca | sentir; colocar a mão no pulso do paciente e avaliar a situação através do fluxo sanguíneo}
  \end{Phonetics}
\end{Entry}

\begin{Entry}{号召}{5,5}{⼝、⼝}
  \begin{Phonetics}{号召}{hao4zhao4}[][HSK 5]
    \definition{s.}{chamado; apelo; desejo ou pedido solene (de um governo, partido político, organização etc.) para que as massas façam algo}
    \definition{v.}{chamar;  (governo, partido político, organização, etc.) fazer um pedido solene às massas para que façam algo, na esperança de que todos se esforcem para alcançá-lo}
  \end{Phonetics}
\end{Entry}

\begin{Entry}{号角}{5,7}{⼝、⾓}
  \begin{Phonetics}{号角}{hao4jiao3}
    \definition{s.}{corneta | trombeta}
  \end{Phonetics}
\end{Entry}

\begin{Entry}{号码}{5,8}{⼝、⽯}
  \begin{Phonetics}{号码}{hao4ma3}[][HSK 4]
    \definition[个,组,串]{s.}{número}
  \end{Phonetics}
\end{Entry}

\begin{Entry}{号称}{5,10}{⼝、⽲}
  \begin{Phonetics}{号称}{hao4cheng1}[][HSK 7-9]
    \definition{v.}{ser conhecido como; ser conhecido por um certo nome | afirmar ser; alegar}
  \end{Phonetics}
\end{Entry}

%%%%%%%%%% 司 %%%%%%%%%%
\subsection*{司}

\begin{Entry}{司}{5}{⼝}
  \begin{Phonetics}{司}{si1}
    \definition*{s.}{Sobrenome: Si}
    \definition{s.}{departamento (sob um ministério); um departamento dentro de uma agência de nível ministerial}
    \definition{v.}{assumir o comando de; atender; administrar; operar; gerenciar}
  \end{Phonetics}
\end{Entry}

\begin{Entry}{司长}{5,4}{⼝、⾧}
  \begin{Phonetics}{司长}{si1 zhang3}[][HSK 6]
    \definition[位,名]{s.}{diretor-geral | chefe de gabinete}
  \end{Phonetics}
\end{Entry}

\begin{Entry}{司机}{5,6}{⼝、⽊}
  \begin{Phonetics}{司机}{si1ji1}[][HSK 2]
    \definition[个,名,位]{s.}{motorista; motorista particular; chofer; motoristas de veículos de transporte público, como trens, ônibus e bondes}
  \end{Phonetics}
\end{Entry}

%%%%%%%%%% 叹 %%%%%%%%%%
\subsection*{叹}

\begin{Entry}{叹}{5}{⼝}
  \begin{Phonetics}{叹}{tan4}
    \definition{v.}{suspirar | exclamar com admiração; aclamar; louvar |recitar com cadência; entoar cântico; entoar}
  \end{Phonetics}
\end{Entry}

\begin{Entry}{叹气}{5,4}{⼝、⽓}
  \begin{Phonetics}{叹气}{tan4qi4}[][HSK 6]
    \definition{v.}{suspirar; soltar um suspiro; soltar um longo suspiro e fazer um som devido à insatisfação ou desamparo}
  \end{Phonetics}
\end{Entry}

%%%%%%%%%% 叼 %%%%%%%%%%
\subsection*{叼}

\begin{Entry}{叼}{5}{⼝}
  \begin{Phonetics}{叼}{diao1}[][HSK 7-9]
    \definition{v.}{segurar na boca; segurar com a boca}
  \end{Phonetics}
\end{Entry}

%%%%%%%%%% 囘 %%%%%%%%%%
\subsection*{囘}

\begin{Entry}{囘}{5}{⼞}
  \begin{Phonetics}{囘}{hui2}
    \variantof{回}
  \end{Phonetics}
\end{Entry}

%%%%%%%%%% 囚 %%%%%%%%%%
\subsection*{囚}

\begin{Entry}{囚}{5}{⼞}
  \begin{Phonetics}{囚}{qiu2}
    \definition[个,群,位,名,些,批]{s.}{prisioneiro; condenado}
    \definition{v.}{aprisionar}
  \end{Phonetics}
\end{Entry}

\begin{Entry}{囚犯}{5,5}{⼞、⽝}
  \begin{Phonetics}{囚犯}{qiu2fan4}
    \definition[名]{s.}{prisioneiro; condenado}
  \end{Phonetics}
\end{Entry}

%%%%%%%%%% 四 %%%%%%%%%%
\subsection*{四}

\begin{Entry}{四}{5}{⼞}
  \begin{Phonetics}{四}{si4}[][HSK 1]
    \definition*{s.}{Sobrenome: Si}
    \definition{num.}{quatro; 4}
    \definition{s.}{uma nota da escala em Gongchepu (工尺谱), correspondente ao 6 na notação musical numerada}
  \seealsoref{工尺谱}{gong1 che3 pu3}
  \end{Phonetics}
\end{Entry}

\begin{Entry}{四川}{5,3}{⼞、⼮}
  \begin{Phonetics}{四川}{si4chuan1}
    \definition*{s.}{Província de Sichuan}
  \end{Phonetics}
\end{Entry}

\begin{Entry}{四处}{5,5}{⼞、⼡}
  \begin{Phonetics}{四处}{si4 chu4}[][HSK 6]
    \definition{adv.}{em volta; ao redor; em todos os lugares; em todas as direções}
  \end{Phonetics}
\end{Entry}

\begin{Entry}{四周}{5,8}{⼞、⼝}
  \begin{Phonetics}{四周}{si4 zhou1}[][HSK 5]
    \definition{s.}{ao redor; por todos os lados; a parte que circunda o centro}
  \end{Phonetics}
\end{Entry}

\begin{Entry}{四季分明}{5,8,4,8}{⼞、⼦、⼑、⽇}
  \begin{Phonetics}{四季分明}{si4ji4-fen1ming2}
    \definition{expr.}{as quatro estações são muito distintas}
  \end{Phonetics}
\end{Entry}

\begin{Entry}{四季如春}{5,8,6,9}{⼞、⼦、⼥、⽇}
  \begin{Phonetics}{四季如春}{si4ji4-ru2chun1}
    \definition{expr.}{é primavera todo o ano | clima favorável durante todo o ano | quatro estações como a primavera}
  \end{Phonetics}
\end{Entry}

%%%%%%%%%% 圣 %%%%%%%%%%
\subsection*{圣}

\begin{Entry}{圣}{5}{⼟}
  \begin{Phonetics}{圣}{sheng4}
    \definition*{s.}{Sobrenome: Sheng}
    \definition{adj.}{santo; sagrado | imperial}
    \definition{s.}{santo; sábio | imperador | o maior mestre de uma determinada arte ou habilidade}
  \end{Phonetics}
\end{Entry}

\begin{Entry}{圣地}{5,6}{⼟、⼟}
  \begin{Phonetics}{圣地}{sheng4di4}
    \definition{s.}{terra santa (de uma religião) | lugar sagrado | santuário | cidade santa (como Jerusalém, Meca, etc.) | centro de interesse histórico}
  \end{Phonetics}
\end{Entry}

\begin{Entry}{圣诞节}{5,8,5}{⼟、⾔、⾋}
  \begin{Phonetics}{圣诞节}{sheng4 dan4 jie2}[][HSK 6]
    \definition*{s.}{Natal; Nascimento de Jesus Cristo em 25 de dezembro}
  \end{Phonetics}
\end{Entry}

%%%%%%%%%% 处 %%%%%%%%%%
\subsection*{处}

\begin{Entry}{处}{5}{⼡}
  \begin{Phonetics}{处}{chu3}[][HSK 4]
    \definition*{s.}{Sobrenome: Chu}
    \definition{v.}{morar; habitar; viver em um lugar | dar-se bem (com alguém); relacionar-se; interagir | estar situado em; estar em uma determinada condição; estar em (um lugar, período ou ocasião) | gerenciar; manejar; lidar com | punir; sentenciar; tomar medidas disciplinares contra (alguém)}
  \end{Phonetics}
  \begin{Phonetics}{处}{chu4}
    \definition{clas.}{usado para lugares ou para ocorrências ou atividades em lugares diferentes}
    \definition{s.}{lugar; local; instalação; dependência | parte; ponto; aspecto ou parte de um objeto | escritório; departamento; nomes de determinados órgãos, organizações ou unidades em órgãos por empresa}
  \end{Phonetics}
\end{Entry}

\begin{Entry}{处于}{5,3}{⼡、⼆}
  \begin{Phonetics}{处于}{chu3 yu2}[][HSK 4]
    \definition{v.}{estar em (uma condição, estado)}
  \end{Phonetics}
\end{Entry}

\begin{Entry}{处分}{5,4}{⼡、⼑}
  \begin{Phonetics}{处分}{chu3fen4}[][HSK 5]
    \definition{s.}{punição; castigo; refere-se a uma decisão de impor uma penalidade ou uma disposição}
    \definition{v.}{punir; tomar medidas disciplinares contra; fornecer algum tratamento ou disposição para aqueles que cometeram erros ou falhas}
  \end{Phonetics}
\end{Entry}

\begin{Entry}{处方}{5,4}{⼡、⽅}
  \begin{Phonetics}{处方}{chu3fang1}[][HSK 7-9]
    \definition[张,份]{s.}{prescrição; receita; uma receita médica ou uma receita pronta para preparar medicamentos}
    \definition{v.}{escrever uma receita; prescrever}
  \end{Phonetics}
\end{Entry}

\begin{Entry}{处长}{5,4}{⼡、⾧}
  \begin{Phonetics}{处长}{chu4 zhang3}[][HSK 6]
    \definition{s.}{chefe de um departamento (ou escritório); chefe de seção}
  \end{Phonetics}
\end{Entry}

\begin{Entry}{处处}{5,5}{⼡、⼡}
  \begin{Phonetics}{处处}{chu4 chu4}[][HSK 6]
    \definition{adv.}{em todos os lugares; em todos os aspectos}
  \end{Phonetics}
\end{Entry}

\begin{Entry}{处在}{5,6}{⼡、⼟}
  \begin{Phonetics}{处在}{chu3 zai4}[][HSK 5]
    \definition{v.}{estar situado em; encontrar-se em; estar em (algum estado, posição ou condição)}
  \end{Phonetics}
\end{Entry}

\begin{Entry}{处罚}{5,9}{⼡、⽹}
  \begin{Phonetics}{处罚}{chu3 fa2}[][HSK 5]
    \definition[个,次,种]{s.}{punição; castigo; penalidade}
    \definition{v.}{punir; disciplinar; castigar; advertir o transgressor ou infrator sobre perdas políticas ou financeiras}
  \end{Phonetics}
\end{Entry}

\begin{Entry}{处理}{5,11}{⼡、⽟}
  \begin{Phonetics}{处理}{chu3li3}[][HSK 3]
    \definition{s.}{manuseio; descarte}
    \definition{v.}{lidar com; dispor de; organizar; resolver | resolver; punir; lidar | vender a preços reduzidos; liquidar | lidar com; processar; processar algo de uma maneira ou método específico; processar uma peça de trabalho ou produto de uma maneira específica para que a peça de trabalho ou produto obtenha o desempenho necessário}
  \end{Phonetics}
\end{Entry}

\begin{Entry}{处置}{5,13}{⼡、⽹}
  \begin{Phonetics}{处置}{chu3zhi4}[][HSK 7-9]
    \definition{v.}{lidar com; gerenciar; descartar | punir}
  \end{Phonetics}
\end{Entry}

\begin{Entry}{处境}{5,14}{⼡、⼟}
  \begin{Phonetics}{处境}{chu3jing4}[][HSK 7-9]
    \definition[种,个]{s.}{situação difícil; situação desfavorável}
  \end{Phonetics}
\end{Entry}

%%%%%%%%%% 外 %%%%%%%%%%
\subsection*{外}

\begin{Entry}{外}{5}{⼣}
  \begin{Phonetics}{外}{wai4}[][HSK 1]
    \definition{adj.}{outro (que não o próprio) | não íntimo; não intimamente relacionado | não oficial | exterior; externo; do lado de fora | outros; referindo-se a um local fora da sua localização atual | do lado da mãe, da irmã ou da filha; referir-se a parentes do lado materno, irmãs ou filhas | informal; não oficial}
    \definition{adv.}{adicionalmente; além disso | para fora; para o exterior; fora | extra; além disso}
    \definition{s.}{fora; externo; exterior (oposto a 内, 里) | outro local; outro lugar | estrangeiro; país estrangeiro | lado externo | parentes de sua mãe, irmãs ou filhas}
  \seealsoref{里}{li3}
  \seealsoref{内}{nei4}
  \end{Phonetics}
\end{Entry}

\begin{Entry}{外公}{5,4}{⼣、⼋}
  \begin{Phonetics}{外公}{wai4gong1}
    \definition{s.}{avô materno}
  \end{Phonetics}
\end{Entry}

\begin{Entry}{外币}{5,4}{⼣、⼱}
  \begin{Phonetics}{外币}{wai4 bi4}[][HSK 6]
    \definition[种]{s.}{moeda estrangeira}
  \end{Phonetics}
\end{Entry}

\begin{Entry}{外文}{5,4}{⼣、⽂}
  \begin{Phonetics}{外文}{wai4 wen2}[][HSK 3]
    \definition[种,门]{s.}{língua ou escrita estrangeira}
  \end{Phonetics}
\end{Entry}

\begin{Entry}{外水}{5,4}{⼣、⽔}
  \begin{Phonetics}{外水}{wai4shui3}
    \definition{s.}{renda extra}
  \end{Phonetics}
\end{Entry}

\begin{Entry}{外出}{5,5}{⼣、⼐}
  \begin{Phonetics}{外出}{wai4 chu1}[][HSK 6]
    \definition{v.}{sair, especialmente para ir a outro lugar a negócios}
  \end{Phonetics}
\end{Entry}

\begin{Entry}{外号}{5,5}{⼣、⼝}
  \begin{Phonetics}{外号}{wai4hao4}
    \definition[个]{s.}{apelido; nomes dados por outras pessoas, diferentes do nome real, muitas vezes têm conotações de carinho, brincadeira, elogio ou ódio}
  \end{Phonetics}
\end{Entry}

\begin{Entry}{外头}{5,5}{⼣、⼤}
  \begin{Phonetics}{外头}{wai4 tou5}[][HSK 6]
    \definition{s.}{Coloquial: fora; ao ar livre (oposto a 里头)}
  \seealsoref{里头}{li3 tou5}
  \end{Phonetics}
\end{Entry}

\begin{Entry}{外汇}{5,5}{⼣、⽔}
  \begin{Phonetics}{外汇}{wai4 hui4}[][HSK 4]
    \definition{s.}{câmbio estrangeiro; moeda estrangeira; moedas estrangeiras e títulos, como cheques, letras de câmbio, notas promissórias, etc., conversíveis em moedas estrangeiras, usados na compensação do comércio internacional}
  \end{Phonetics}
\end{Entry}

\begin{Entry}{外边}{5,5}{⼣、⾡}
  \begin{Phonetics}{外边}{wai4 bian5}[][HSK 1]
    \definition{s.}{fora; exterior; externo; além de um determinado limite | local diferente de onde se vive ou trabalha; referindo-se a lugares distantes | exterior; externo; superfície}
  \end{Phonetics}
\end{Entry}

\begin{Entry}{外交}{5,6}{⼣、⼇}
  \begin{Phonetics}{外交}{wai4jiao1}[][HSK 3]
    \definition[个]{s.}{diplomacia; relações exteriores; atividades de um país nas relações internacionais, como participar de organizações e conferências internacionais, trocar enviados com outros países, conduzir negociações, assinar tratados e acordos, etc.}
  \end{Phonetics}
\end{Entry}

\begin{Entry}{外交官}{5,6,8}{⼣、⼇、⼧}
  \begin{Phonetics}{外交官}{wai4 jiao1 guan1}[][HSK 4]
    \definition[位,名]{s.}{diplomata}
  \end{Phonetics}
\end{Entry}

\begin{Entry}{外协}{5,6}{⼣、⼗}
  \begin{Phonetics}{外协}{wai4xie2}
    \definition{s.}{terceirização | pessoas que julgam os outros pela aparência}
  \seealsoref{外貌协会}{wai4mao4xie2hui4}
  \end{Phonetics}
\end{Entry}

\begin{Entry}{外地}{5,6}{⼣、⼟}
  \begin{Phonetics}{外地}{wai4 di4}[][HSK 2]
    \definition{s.}{não local; outros lugares; locais fora da área local}
  \end{Phonetics}
\end{Entry}

\begin{Entry}{外孙}{5,6}{⼣、⼦}
  \begin{Phonetics}{外孙}{wai4sun1}
    \definition{s.}{filho da filha}
  \end{Phonetics}
\end{Entry}

\begin{Entry}{外孙女}{5,6,3}{⼣、⼦、⼥}
  \begin{Phonetics}{外孙女}{wai4sun1nv3}
    \definition{s.}{filha da filha}
  \end{Phonetics}
\end{Entry}

\begin{Entry}{外衣}{5,6}{⼣、⾐}
  \begin{Phonetics}{外衣}{wai4 yi1}[][HSK 6]
    \definition[件]{s.}{casaco; jaqueta; colete; sobreveste; envoltório; roupa externa (ou vestimenta); capa externa; vestido externo | semblante; aparência; feição}
  \end{Phonetics}
\end{Entry}

\begin{Entry}{外观}{5,6}{⼣、⾒}
  \begin{Phonetics}{外观}{wai4 guan1}[][HSK 6]
    \definition{s.}{aspecto; semblante; aparência; aparência exterior; a aparência de um objeto}
  \end{Phonetics}
\end{Entry}

\begin{Entry}{外围}{5,7}{⼣、⼞}
  \begin{Phonetics}{外围}{wai4wei2}
    \definition{adv.}{arredores}
  \end{Phonetics}
\end{Entry}

\begin{Entry}{外来}{5,7}{⼣、⽊}
  \begin{Phonetics}{外来}{wai4 lai2}[][HSK 6]
    \definition{adj.}{de fora; externo; estrangeiro}
  \end{Phonetics}
\end{Entry}

\begin{Entry}{外事}{5,8}{⼣、⼅}
  \begin{Phonetics}{外事}{wai4shi4}
    \definition{s.}{assuntos ou relações exteriores}
  \end{Phonetics}
\end{Entry}

\begin{Entry}{外卖}{5,8}{⼣、⼗}
  \begin{Phonetics}{外卖}{wai4 mai4}[][HSK 2]
    \definition[份,单,盒]{s.}{comida para viagem; levar para viagem}
    \definition{v.}{entregar; oferecer; refere-se à ação do comerciante entregar alimentos no local especificado pelo cliente}
  \end{Phonetics}
\end{Entry}

\begin{Entry}{外国}{5,8}{⼣、⼞}
  \begin{Phonetics}{外国}{wai4 guo2}[][HSK 1]
    \definition[个]{s.}{país estrangeiro}
  \end{Phonetics}
\end{Entry}

\begin{Entry}{外国人}{5,8,2}{⼣、⼞、⼈}
  \begin{Phonetics}{外国人}{wai4 guo2 ren2}
    \definition[个]{s.}{estrangeiro | alienígena}
  \end{Phonetics}
\end{Entry}

\begin{Entry}{外界}{5,9}{⼣、⽥}
  \begin{Phonetics}{外界}{wai4jie4}[][HSK 5]
    \definition{s.}{o exterior; o mundo externo; área fora de um determinado âmbito; sociedade externa}
  \end{Phonetics}
\end{Entry}

\begin{Entry}{外科}{5,9}{⼣、⽲}
  \begin{Phonetics}{外科}{wai4 ke1}[][HSK 6]
    \definition[名]{s.}{cirurgia; departamento cirúrgico; um departamento em uma instituição médica que usa principalmente cirurgia para tratar doenças internas e externas}
  \end{Phonetics}
\end{Entry}

\begin{Entry}{外语}{5,9}{⼣、⾔}
  \begin{Phonetics}{外语}{wai4 yu3}[][HSK 1]
    \definition[种,门]{s.}{língua estrangeira}
  \end{Phonetics}
\end{Entry}

\begin{Entry}{外贸}{5,9}{⼣、⾙}
  \begin{Phonetics}{外贸}{wai4mao4}
    \definition{s.}{comércio exterior}
  \end{Phonetics}
\end{Entry}

\begin{Entry}{外面}{5,9}{⼣、⾯}
  \begin{Phonetics}{外面}{wai4 mian4}[][HSK 3]
    \definition{s.}{o lado de fora; fora de um certo intervalo | exterior; aparência externa; a superfície de um objeto}
  \end{Phonetics}
\end{Entry}

\begin{Entry}{外套}{5,10}{⼣、⼤}
  \begin{Phonetics}{外套}{wai4 tao4}[][HSK 4]
    \definition[件,套,个]{s.}{casaco; jaqueta; paletó; sobretudo}
  \end{Phonetics}
\end{Entry}

\begin{Entry}{外海}{5,10}{⼣、⽔}
  \begin{Phonetics}{外海}{wai4hai3}
    \definition{s.}{mar aberto}
  \end{Phonetics}
\end{Entry}

\begin{Entry}{外积}{5,10}{⼣、⽲}
  \begin{Phonetics}{外积}{wai4ji1}
    \definition{s.}{produto exterior | (matemática) o produto vetorial de dois vetores}
  \end{Phonetics}
\end{Entry}

\begin{Entry}{外资}{5,10}{⼣、⾙}
  \begin{Phonetics}{外资}{wai4 zi1}[][HSK 6]
    \definition{s.}{capital estrangeiro (oposto a 内资); investimento estrangeiro; fundos estrangeiros; capital investido por países estrangeiros}
  \seealsoref{内资}{nei4 zi1}
  \end{Phonetics}
\end{Entry}

\begin{Entry}{外部}{5,10}{⼣、⾢}
  \begin{Phonetics}{外部}{wai4 bu4}[][HSK 6]
    \definition{s.}{fora; externo; fora de um certo intervalo | exterior; superfície}
  \end{Phonetics}
\end{Entry}

\begin{Entry}{外婆}{5,11}{⼣、⼥}
  \begin{Phonetics}{外婆}{wai4po2}
    \definition{s.}{avó materna}
  \end{Phonetics}
\end{Entry}

\begin{Entry}{外插}{5,12}{⼣、⼿}
  \begin{Phonetics}{外插}{wai4cha1}
    \definition{v.}{extrapolar | (computação) conectar (um dispositivo periférico, etc.)}
  \end{Phonetics}
\end{Entry}

\begin{Entry}{外貌协会}{5,14,6,6}{⼣、⾘、⼗、⼈}
  \begin{Phonetics}{外貌协会}{wai4mao4xie2hui4}
    \definition{s.}{``o clube da boa aparência'': pessoas que dão grande importância à aparência de uma pessoa}
  \seealsoref{外协}{wai4xie2}
  \end{Phonetics}
\end{Entry}

%%%%%%%%%% 失 %%%%%%%%%%
\subsection*{失}

\begin{Entry}{失}{5}{⼤}
  \begin{Phonetics}{失}{shi1}
    \definition{s.}{deslize; erro; defeito; acidente}
    \definition{v.}{perder (oposto de 得) | perder; deixar escapar | não agir de acordo com; negligenciar; violar | perder o controle de | errar; cometer um deslize; apresentar defeito em | não consiguir encontrar | não conseguir atingir o objetivo | desviar-se do normal | quebrar (uma promessa); voltar atrás (na palavra dada) | não conseguir obter | se perder}
  \seealsoref{得}{de2}
  \end{Phonetics}
\end{Entry}

\begin{Entry}{失业}{5,5}{⼤、⼀}
  \begin{Phonetics}{失业}{shi1ye4}[][HSK 4]
    \definition{v.}{não ter emprego; estar desempregado; estar sem trabalho; refere-se àqueles que estão dentro da idade legal para trabalhar, têm capacidade para trabalhar, estão desempregados e querem encontrar um emprego, mas não conseguem; embora se envolvam em certos trabalhos sociais, sua remuneração é menor do que o padrão mínimo de vida urbano local e são considerados desempregados}
  \end{Phonetics}
\end{Entry}

\begin{Entry}{失去}{5,5}{⼤、⼛}
  \begin{Phonetics}{失去}{shi1qu4}[][HSK 3]
    \definition{v.}{perder}
  \end{Phonetics}
\end{Entry}

\begin{Entry}{失败}{5,8}{⼤、⾒}
  \begin{Phonetics}{失败}{shi1bai4}[][HSK 4]
    \definition{adj.}{insatisfatório; a maneira como as coisas aconteceram deixou muito a desejar; o resultado final deixou muito a desejar}
    \definition{v.}{perder; ser derrotado; não vencer em uma guerra ou competição | falhar; fracassar; não dar em nada; falhar em atingir um objetivo ou meta desejada (trabalho, carreira, etc.)}
  \end{Phonetics}
\end{Entry}

\begin{Entry}{失误}{5,9}{⼤、⾔}
  \begin{Phonetics}{失误}{shi1wu4}[][HSK 5]
    \definition[个]{s.}{erro; engano; equívoco; erros causados por negligência ou medidas inadequadas}
    \definition{v.}{cometer um erro; cometer um equívoco}
  \end{Phonetics}
\end{Entry}

\begin{Entry}{失眠}{5,10}{⼤、⽬}
  \begin{Phonetics}{失眠}{shi1mian2}
    \definition{s.}{insônia}
    \definition{v.}{ter insônia}
  \end{Phonetics}
\end{Entry}

\begin{Entry}{失望}{5,11}{⼤、⽉}
  \begin{Phonetics}{失望}{shi1wang4}[][HSK 4]
    \definition{adj.}{desapontado; decepcionado}
    \definition{v.}{ficar desapontado; ficar decepcionado; estar desapontado; sentir-se sem esperança; perder a confiança}
  \end{Phonetics}
\end{Entry}

\begin{Entry}{失落}{5,12}{⼤、⾋}
  \begin{Phonetics}{失落}{shi1luo4}
    \definition{s.}{frustração | decepção | perda}
    \definition{v.}{perder (algo) | cair (algo) | sentir uma sensação de perda}
  \end{Phonetics}
\end{Entry}

\begin{Entry}{失意}{5,13}{⼤、⼼}
  \begin{Phonetics}{失意}{shi1yi4}
    \definition{adj.}{desapontado | frustrado}
  \end{Phonetics}
\end{Entry}

%%%%%%%%%% 头 %%%%%%%%%%
\subsection*{头}

\begin{Entry}{头}{5}{⼤}
  \begin{Phonetics}{头}{tou2}[][HSK 2,3]
    \definition{adj.}{(antes de um numeral) primeiro | (antes de 年 ou 天) último; anterior}
    \definition{clas.}{usado para suínos ou gado (animais de criação) | usado para cabeças de alho ou coisas com formato de cabeça}
    \definition{num.}{primeiro}
    \definition{prep.}{antes de; perto de; introduz o tempo de uma ação, equivalente a  在……之前 ou 临近 | (entre dois algarismos, indicando um número aproximado) cerca de}
    \definition[个,颗]{s.}{cabeça; a parte do corpo humano ou animal que possui órgãos como boca, nariz, olhos e ouvidos | cabelo ou penteado | topo; fim; a parte superior ou final de um objeto | começo ou fim; o ponto inicial ou final de algo | fim; remanescente; os restos de algo | cabeça; chefe; líder | lado; aspecto}
  \seealsoref{临近}{lin2jin4}
  \seealsoref{年}{nian2}
  \seealsoref{天}{tian1}
  \seealsoref{在}{zai4}
  \seealsoref{之前}{zhi1 qian2}
  \end{Phonetics}
  \begin{Phonetics}{头}{tou5}
    \definition{suf.}{adicionado após componentes nominais comuns | adicionado após o componente verbal, forma um substantivo abstrato, geralmente indicando que vale a pena realizar essa ação | adicionado após um componente adjetival, forma um substantivo, geralmente indicando algo abstrato | adicionado após o componente substantivo que indica a direção}
  \end{Phonetics}
\end{Entry}

\begin{Entry}{头发}{5,5}{⼤、⼜}
  \begin{Phonetics}{头发}{tou2fa5}[][HSK 2]
    \definition[根,缕,头]{s.}{cabelo}
  \end{Phonetics}
\end{Entry}

\begin{Entry}{头号}{5,5}{⼤、⼝}
  \begin{Phonetics}{头号}{tou2hao4}
    \definition{adj.}{primeira classe | número um | \emph{top rank}}
  \end{Phonetics}
\end{Entry}

\begin{Entry}{头头}{5,5}{⼤、⼤}
  \begin{Phonetics}{头头}{tou2tou2}
    \definition{s.}{chefe | o cabeça}
  \end{Phonetics}
\end{Entry}

\begin{Entry}{头疼}{5,10}{⼤、⽧}
  \begin{Phonetics}{头疼}{tou2 teng2}[][HSK 6]
    \definition{s.}{dor de cabeça}
    \definition{v.}{estar preocupado ou incomodado por alguém ou algo}
  \end{Phonetics}
\end{Entry}

\begin{Entry}{头脑}{5,10}{⼤、⾁}
  \begin{Phonetics}{头脑}{tou2 nao3}[][HSK 3]
    \definition{s.}{inteligência; mente | pista; tópicos principais | chefe; líder; capitão}
  \end{Phonetics}
\end{Entry}

\begin{Entry}{头脑风暴}{5,10,4,15}{⼤、⾁、⾵、⽇}
  \begin{Phonetics}{头脑风暴}{tou2nao3feng1bao4}
    \definition{s.}{\emph{brainstorm}}
  \end{Phonetics}
\end{Entry}

\begin{Entry}{头像}{5,13}{⼤、⼈}
  \begin{Phonetics}{头像}{tou2xiang4}
    \definition{s.}{retrato | busto | avatar | imagem de perfil (computação)}
  \end{Phonetics}
\end{Entry}

%%%%%%%%%% 奶 %%%%%%%%%%
\subsection*{奶}

\begin{Entry}{奶}{5}{⼥}
  \begin{Phonetics}{奶}{nai3}[][HSK 1]
    \definition{adj.}{bebê; infância; infantil}
    \definition[杯,滴,瓶,只,桶]{s.}{seios; mama | leite; produtos lácteos}
    \definition{v.}{amamentar; mamar}
  \end{Phonetics}
\end{Entry}

\begin{Entry}{奶牛}{5,4}{⼥、⽜}
  \begin{Phonetics}{奶牛}{nai3 niu2}[][HSK 6]
    \definition{s.}{vaca leiteira (ou leiteira); vaca}
  \end{Phonetics}
\end{Entry}

\begin{Entry}{奶奶}{5,5}{⼥、⼥}
  \begin{Phonetics}{奶奶}{nai3nai5}[][HSK 1]
    \definition[位]{s.}{avó (paterna) | vovó; avó; mulheres mais velhas | jovem senhora da casa}
  \end{Phonetics}
\end{Entry}

\begin{Entry}{奶茶}{5,9}{⼥、⾋}
  \begin{Phonetics}{奶茶}{nai3 cha2}[][HSK 3]
    \definition[杯]{s.}{chá com leite; chá com leite de vaca ou de ovelha}
  \end{Phonetics}
\end{Entry}

\begin{Entry}{奶粉}{5,10}{⼥、⽶}
  \begin{Phonetics}{奶粉}{nai3 fen3}[][HSK 6]
    \definition[袋,桶,罐,勺]{s.}{leite em pó}
  \end{Phonetics}
\end{Entry}

%%%%%%%%%% 宁 %%%%%%%%%%
\subsection*{宁}

\begin{Entry}{宁}{5}{⼧}
  \begin{Phonetics}{宁}{ning2}
    \definition*{s.}{Região Autônoma de Ningxia Hui, abreviação de 宁夏回族自治区 | outro nome para Nanquim, 南京 | Sobrenome: Ning}
    \definition{adj.}{calmo, pacífico, sereno | saudável}
    \definition{v.}{Literário: fazer uma visita (aos pais ou aos mais velhos); | Literário: pacificar; apaziguar}
  \seealsoref{南京}{nan2jing1}
  \seealsoref{宁夏回族自治区}{ning2xia4 hui2zu2 zi4zhi4qu1}
  \end{Phonetics}
  \begin{Phonetics}{宁}{ning4}
    \definition{conj.}{mais\dots do que\dots, melhor\dots do que\dots}
  \end{Phonetics}
\end{Entry}

\begin{Entry}{宁可}{5,5}{⼧、⼝}
  \begin{Phonetics}{宁可}{ning4ke3}
    \definition{conj.}{mais\dots do que\dots | melhor\dots do que\dots}
  \end{Phonetics}
\end{Entry}

\begin{Entry}{宁可……也不……}{5,5,3,4}{⼧、⼝、⼄、⼀}
  \begin{Phonetics}{宁可……也不……}{ning4ke3 ye3bu4}
    \definition{conj.}{preferiria\dots do que\dots}
  \end{Phonetics}
\end{Entry}

\begin{Entry}{宁可……也要……}{5,5,3,9}{⼧、⼝、⼄、⾑}
  \begin{Phonetics}{宁可……也要……}{ning4ke3 ye3yao4}
    \definition{conj.}{mesmo que tenhamos que\dots nós iremos\dots}
  \end{Phonetics}
\end{Entry}

\begin{Entry}{宁肯}{5,8}{⼧、⾁}
  \begin{Phonetics}{宁肯}{ning4ken3}
    \definition{conj.}{mais\dots do que\dots, melhor\dots do que\dots}
  \end{Phonetics}
\end{Entry}

\begin{Entry}{宁夏回族自治区}{5,10,6,11,6,8,4}{⼧、⼢、⼞、⽅、⾃、⽔、⼖}
  \begin{Phonetics}{宁夏回族自治区}{ning2xia4 hui2zu2 zi4zhi4qu1}
    \definition*{s.}{Região Autônoma de Ningxia Hui}
  \end{Phonetics}
\end{Entry}

\begin{Entry}{宁愿}{5,14}{⼧、⽕}
  \begin{Phonetics}{宁愿}{ning4yuan4}
    \definition{conj.}{mais\dots do que\dots, melhor\dots do que\dots}
  \end{Phonetics}
\end{Entry}

\begin{Entry}{宁静}{5,14}{⼧、⾭}
  \begin{Phonetics}{宁静}{ning2 jing4}[][HSK 4]
    \definition{adj.}{calmo; tranquilo; pacífico}
  \end{Phonetics}
\end{Entry}

%%%%%%%%%% 它 %%%%%%%%%%
\subsection*{它}

\begin{Entry}{它}{5}{⼧}
  \begin{Phonetics}{它}{ta1}[][HSK 2]
    \definition*{s.}{Sobrenome: Ta}
    \definition{pron.}{ele; referência a algo além da pessoa (para objetos inanimados) | ele; usado após o verbo, indica referência vaga}
  \end{Phonetics}
\end{Entry}

\begin{Entry}{它们}{5,5}{⼧、⼈}
  \begin{Phonetics}{它们}{ta1 men5}[][HSK 2]
    \definition{pron.}{eles; usado para se referir a mais de uma coisa não humana; geralmente se refere a animais, objetos ou conceitos abstratos}
  \end{Phonetics}
\end{Entry}

%%%%%%%%%% 对 %%%%%%%%%%
\subsection*{对}

\begin{Entry}{对}{5}{⼨}
  \begin{Phonetics}{对}{dui4}[][HSK 1,2]
    \definition{adj.}{certo; correto; em conformidade com determinados padrões | oposto; contrário}
    \definition{adv.}{mutuamente; cara a cara}
    \definition{clas.}{usado para pessoas ou coisas que formam pares; casais}
    \definition{prep.}{o que diz respeito a; relativo a; com relação a; introduz o objeto da ação}
    \definition[幅]{s.}{dístico; refere-se a um par de versos | par; parceiro; pessoas ou coisas que se complementam}
    \definition{v.}{responder; dar uma resposta | tratar; lidar com; combater | ser treinado para; ser direcionado para; enfrentar | colocar (duas coisas) em contato; encaixar uma na outra; combinar ou cooperar entre si | comparar; verificar; identificar; comparar e verificar se estão de acordo | definir; ajustar; ajustar para atender a determinados requisitos | misturar (refere-se principalmente a líquidos); adicionar | dividir ao meio; dividir em duas partes iguais | combinar; concordar; dar-se bem; harmonizar-se}
  \end{Phonetics}
\end{Entry}

\begin{Entry}{对于}{5,3}{⼨、⼆}
  \begin{Phonetics}{对于}{dui4yu2}[][HSK 4]
    \definition{prep.}{para; relativo a; no que diz respeito a; a respeito de}
  \end{Phonetics}
\end{Entry}

\begin{Entry}{对不起}{5,4,10}{⼨、⼀、⾛}
  \begin{Phonetics}{对不起}{dui4bu5qi3}[][HSK 1]
    \definition{interj.}{Desculpe! | Desculpe-me! | Perdoe-me! | Com licença?}
    \definition{v.}{desculpar; pedir desculpas; perdoar}
  \end{Phonetics}
\end{Entry}

\begin{Entry}{对手}{5,4}{⼨、⼿}
  \begin{Phonetics}{对手}{dui4shou3}[][HSK 3]
    \definition[个,名,位,对]{s.}{oponente; adversário na competição | igual; correspondente; refere-se especificamente ao adversário em uma competição em que as habilidades e o nível são praticamente iguais}
  \end{Phonetics}
\end{Entry}

\begin{Entry}{对方}{5,4}{⼨、⽅}
  \begin{Phonetics}{对方}{dui4fang1}[][HSK 3]
    \definition{s.}{outro lado; lado oposto; outra parte; a parte contrária ao sujeito da ação ou outras pessoas envolvidas em um determinado evento ou situação}
  \end{Phonetics}
\end{Entry}

\begin{Entry}{对比}{5,4}{⼨、⽐}
  \begin{Phonetics}{对比}{dui4bi3}[][HSK 4]
    \definition{s.}{razão; proporção | contraste; comparação; diferenças ou lacunas encontradas após comparação}
    \definition{v.}{contrastar; comparar}
  \end{Phonetics}
\end{Entry}

\begin{Entry}{对付}{5,5}{⼨、⼈}
  \begin{Phonetics}{对付}{dui4fu5}[][HSK 4]
    \definition{adj.}{em bons termos; estar em termos agradáveis ​​(frequentemente usado em negativas); dialeto usado para descrever duas pessoas que têm um bom relacionamento e se dão bem, frequentemente usado para negar}
    \definition{v.}{enfrentar; tratar; lidar com | fazer acontecer; (informal) fazer algo que você não quer fazer; aceitar algo que você não gosta}
  \end{Phonetics}
\end{Entry}

\begin{Entry}{对外}{5,5}{⼨、⼣}
  \begin{Phonetics}{对外}{dui4 wai4}[][HSK 6]
    \definition{adj.}{externo; para fora | estrangeiro; no exterior}
  \end{Phonetics}
\end{Entry}

\begin{Entry}{对白}{5,5}{⼨、⽩}
  \begin{Phonetics}{对白}{dui4bai2}[][HSK 7-9]
    \definition{s.}{diálogo | diálogo entre personagens em peças e filmes}
  \end{Phonetics}
\end{Entry}

\begin{Entry}{对立}{5,5}{⼨、⽴}
  \begin{Phonetics}{对立}{dui4li4}[][HSK 5]
    \definition{v.}{opor-se; contrastar; filosoficamente, refere-se a duas coisas ou dois aspectos da mesma coisa que se contradizem, se excluem ou entram em conflito entre si | opor-se; ser antagônico a}
  \end{Phonetics}
\end{Entry}

\begin{Entry}{对……有兴趣}{5,6,6,15}{⼨、⽉、⼋、⾛}
  \begin{Phonetics}{对……有兴趣}{dui4 you3xing4qu4}
    \definition{expr.}{estar interessado em\dots; ter interesse em\dots; interessar-se por\dots}
  \seealsoref{对……感兴趣}{dui4 gan3xing4qu4}
  \end{Phonetics}
\end{Entry}

\begin{Entry}{对应}{5,7}{⼨、⼴}
  \begin{Phonetics}{对应}{dui4ying4}[][HSK 5]
    \definition{adj.}{homólogo; correspondente}
    \definition{v.}{corresponder; ser equivalente a}
  \end{Phonetics}
\end{Entry}

\begin{Entry}{对抗}{5,7}{⼨、⼿}
  \begin{Phonetics}{对抗}{dui4kang4}[][HSK 6]
    \definition{v.}{antagonizar; confrontar | resistir; opor-se; contra-atacar}
  \end{Phonetics}
\end{Entry}

\begin{Entry}{对话}{5,8}{⼨、⾔}
  \begin{Phonetics}{对话}{dui4hua4}[][HSK 2]
    \definition[段,番,个]{s.}{diálogo; conversa; refere-se especificamente a diálogos entre personagens em obras literárias, como peças de teatro e romances}
    \definition{v.}{conversar com; comunicar-se com | manter um diálogo; conversar uns com os outros}
  \end{Phonetics}
\end{Entry}

\begin{Entry}{对峙}{5,9}{⼨、⼭}
  \begin{Phonetics}{对峙}{dui4zhi4}[][HSK 7-9]
    \definition{v.}{ficar de frente um para o outro; confrontar um ao outro}
  \end{Phonetics}
\end{Entry}

\begin{Entry}{对弈}{5,9}{⼨、⼶}
  \begin{Phonetics}{对弈}{dui4yi4}[][HSK 7-9]
    \definition{v.}{Literário: jogar go, xadrez etc.}
  \end{Phonetics}
\end{Entry}

\begin{Entry}{对待}{5,9}{⼨、⼻}
  \begin{Phonetics}{对待}{dui4dai4}[][HSK 3]
    \definition{v.}{tratar; abordar; manusear; estar em uma posição relacionada ou comparada a outra; expressar uma certa atitude ou agir de determinada maneira em relação a pessoas ou coisas}
  \end{Phonetics}
\end{Entry}

\begin{Entry}{对……说}{5,9}{⼨、⾔}
  \begin{Phonetics}{对……说}{dui4 shuo5}
    \definition{v.}{dizer a alguém}
  \end{Phonetics}
\end{Entry}

\begin{Entry}{对面}{5,9}{⼨、⾯}
  \begin{Phonetics}{对面}{dui4mian4}[][HSK 2]
    \definition{adv.}{cara a cara}
    \definition[面]{s.}{lado oposto; o outro lado; os nomes dados às duas margens opostas de ruas, rios, etc. | bem na frente; diretamente à frente}
  \end{Phonetics}
\end{Entry}

\begin{Entry}{对准}{5,10}{⼨、⼎}
  \begin{Phonetics}{对准}{dui4zhun3}[][HSK 7-9]
    \definition{s.}{alinhamento; mira; registro}
    \definition{v.}{treinar em; ser direcionado para; mirar (apontar; direcionar) para}
  \end{Phonetics}
\end{Entry}

\begin{Entry}{对称}{5,10}{⼨、⽲}
  \begin{Phonetics}{对称}{dui4chen4}[][HSK 7-9]
    \definition{adj.}{simétrico; refere-se a uma figura ou objeto que tem uma correspondência um-para-um em tamanho, forma e disposição em relação a um ponto, linha ou plano, como os lados esquerdo e direito de um corpo humano, um navio ou um avião}
  \end{Phonetics}
\end{Entry}

\begin{Entry}{对得起}{5,11,10}{⼨、⼻、⾛}
  \begin{Phonetics}{对得起}{dui4de5qi3}[][HSK 7-9]
    \definition{v.}{tratar alguém de forma justa; ser digno de; não decepcionar alguém}
  \end{Phonetics}
\end{Entry}

\begin{Entry}{对象}{5,11}{⼨、⾗}
  \begin{Phonetics}{对象}{dui4xiang4}[][HSK 3]
    \definition[个,位]{s.}{alvo; objeto; a pessoa ou coisa que serve como objetivo ao agir ou pensar | parceiro; namorado; namorada; refere-se especificamente à pessoa amada}
  \end{Phonetics}
\end{Entry}

\begin{Entry}{对策}{5,12}{⼨、⽵}
  \begin{Phonetics}{对策}{dui4ce4}[][HSK 7-9]
    \definition{s.}{contra-ataque; contramedida; maneira de lidar com uma situação; estratégias ou soluções para problemas a serem resolvidos}
    \definition{v.}{abordar; elaborar estratégias; traçar estratégias}
  \end{Phonetics}
\end{Entry}

\begin{Entry}{对联}{5,12}{⼨、⽿}
  \begin{Phonetics}{对联}{dui4lian2}[][HSK 7-9]
    \definition[副,幅]{s.}{dístico (escrito em pergaminhos, etc.); dísticos: frases paralelas escritas em papel, tecido ou esculpidas em bambu, madeira ou pilares}
  \end{Phonetics}
\end{Entry}

\begin{Entry}{对……感兴趣}{5,13,6,15}{⼨、⼼、⼋、⾛}
  \begin{Phonetics}{对……感兴趣}{dui4 gan3xing4qu4}
    \definition{expr.}{estar interessado em\dots; ter interesse em\dots; interessar-se por\dots}
  \seealsoref{对……有兴趣}{dui4 you3xing4qu4}
  \end{Phonetics}
\end{Entry}

\begin{Entry}{对照}{5,13}{⼨、⽕}
  \begin{Phonetics}{对照}{dui4zhao4}[][HSK 7-9]
    \definition{s.}{contraste; comparação}
    \definition{v.aux.}{contrastar; comparar ; fazer referência cruzada; fazer comparação}
  \end{Phonetics}
\end{Entry}

\begin{Entry}{对……熟悉}{5,15,11}{⼨、⽕、⼼}
  \begin{Phonetics}{对……熟悉}{dui4 shu2xi1}
    \definition{expr.}{estar familiarizado com\dots}
  \end{Phonetics}
\end{Entry}

%%%%%%%%%% 左 %%%%%%%%%%
\subsection*{左}

\begin{Entry}{左}{5}{⼯}
  \begin{Phonetics}{左}{zuo3}[][HSK 1]
    \definition*{s.}{Sobrenome Zuo}
    \definition{adj.}{estranho; herético; não ortodoxo | errado; incorreto | diferente; contrário; oposto | progressista; revolucionário; politicamente e ideologicamente progressista; radical}
    \definition{s.}{a esquerda; o lado esquerdo | leste; na antiguidade, referia-se especificamente à direção leste (com base na orientação para o sul) | a esquerda; ala esquerda; refere-se a uma posição inferior (na antiguidade, a direita era considerada superior e a esquerda, inferior)}
    \definition{v.}{assistir; auxiliar}
  \end{Phonetics}
\end{Entry}

\begin{Entry}{左右}{5,5}{⼯、⼝}
  \begin{Phonetics}{左右}{zuo3you4}[][HSK 3]
    \definition{s.}{os lados esquerdo e direito; esquerda e direita, também indicam os arredores | atendentes; acompanhantes; as pessoas que o acompanham | aproximadamente; mais ou menos; por aí; usado após números para indicar uma estimativa, com o mesmo significado de 上下}
    \definition{v.}{controlar; manipular; influenciar; dominar}
  \seealsoref{上下}{shang4 xia4}
  \end{Phonetics}
\end{Entry}

\begin{Entry}{左边}{5,5}{⼯、⾡}
  \begin{Phonetics}{左边}{zuo3bian5}[][HSK 1]
    \definition{s.}{esquerda; o lado esquerdo}
  \end{Phonetics}
\end{Entry}

\begin{Entry}{左派}{5,9}{⼯、⽔}
  \begin{Phonetics}{左派}{zuo3pai4}
    \definition{s.}{(política) esquerda | esquerdista}
  \end{Phonetics}
\end{Entry}

\begin{Entry}{左面}{5,9}{⼯、⾯}
  \begin{Phonetics}{左面}{zuo3mian4}
    \definition{s.}{esquerda | lado esquerdo}
  \end{Phonetics}
\end{Entry}

\begin{Entry}{左倾}{5,10}{⼯、⼈}
  \begin{Phonetics}{左倾}{zuo3qing1}
    \definition{s.}{esquerdista | progressivo}
  \end{Phonetics}
\end{Entry}

\begin{Entry}{左袒}{5,10}{⼯、⾐}
  \begin{Phonetics}{左袒}{zuo3tan3}
    \definition{v.}{ser tendencioso | ser parcial para | favorecer um lado | tomar partido com}
  \end{Phonetics}
\end{Entry}

\begin{Entry}{左舷}{5,11}{⼯、⾈}
  \begin{Phonetics}{左舷}{zuo3xian2}
    \definition{s.}{porto (lado de um navio)}
  \end{Phonetics}
\end{Entry}

\begin{Entry}{左翼}{5,17}{⼯、⽻}
  \begin{Phonetics}{左翼}{zuo3yi4}
    \definition{s.}{esquerda (política)}
  \end{Phonetics}
\end{Entry}

%%%%%%%%%% 巧 %%%%%%%%%%
\subsection*{巧}

\begin{Entry}{巧}{5}{⼯}
  \begin{Phonetics}{巧}{qiao3}[][HSK 3]
    \definition{adj.}{habilidoso; engenhoso; esperto | oportuno; coincidente; fortuito | astuto; enganoso; enganador; traiçoeiro; ardiloso | (de mão, língua) hábil; loquaz}
    \definition{s.}{(tecnologia, artesanato) habilidade; destreza}
  \end{Phonetics}
\end{Entry}

\begin{Entry}{巧合}{5,6}{⼯、⼝}
  \begin{Phonetics}{巧合}{qiao3he2}
    \definition{s.}{coincidência; (coisas) coincidentes ou idênticas}
  \end{Phonetics}
\end{Entry}

\begin{Entry}{巧克力}{5,7,2}{⼯、⼗、⼒}
  \begin{Phonetics}{巧克力}{qiao3ke4li4}[][HSK 4]
    \definition[块,颗,盒,包]{s.}{Empréstimo linguístico: chocolate; alimentos feitos com cacau em pó como principal matéria-prima, açúcar e especiarias}
  \end{Phonetics}
\end{Entry}

\begin{Entry}{巧妙}{5,7}{⼯、⼥}
  \begin{Phonetics}{巧妙}{qiao3miao4}[][HSK 6]
    \definition{adj.}{inteligente; engenhoso; (método ou técnica, etc.) inteligente, além do comum}
  \end{Phonetics}
\end{Entry}

%%%%%%%%%% 市 %%%%%%%%%%
\subsection*{市}

\begin{Entry}{市}{5}{⼱}
  \begin{Phonetics}{市}{shi4}[][HSK 2]
    \definition{s.}{mercado; lugar onde se concentra o comércio | cidade; município; áreas densamente povoadas, com indústrias, comércio e cultura desenvolvidos | relativo ao sistema tradicional chinês de pesos e medidas; unidades administrativas, incluindo cidades sob jurisdição direta e cidades sob jurisdição provincial (ou autônoma) | unidade padrão de mercado; pertencente ao sistema municipal (unidades de medida) | preço de transação no mercado}
    \definition{v.}{comprar ou vender; fazer transações}
  \end{Phonetics}
\end{Entry}

\begin{Entry}{市中心}{5,4,4}{⼱、⼁、⼼}
  \begin{Phonetics}{市中心}{shi4zhong1xin1}
    \definition{s.}{centro da cidade}
  \end{Phonetics}
\end{Entry}

\begin{Entry}{市区}{5,4}{⼱、⼖}
  \begin{Phonetics}{市区}{shi4 qu1}[][HSK 4]
    \definition[个]{s.}{\emph{downtown}; centro da cidade; distrito urbano; áreas que ficam dentro dos limites da cidade e geralmente têm uma alta concentração de população e estoque de moradias}
  \end{Phonetics}
\end{Entry}

\begin{Entry}{市升}{5,4}{⼱、⼗}
  \begin{Phonetics}{市升}{shi4sheng1}
    \definition{clas.}{sheng; uma unidade tradicional de volume, equivalente a 1 litro ou 1,76 \emph{pints} ou 0,22 galão}
  \end{Phonetics}
\end{Entry}

\begin{Entry}{市尺}{5,4}{⼱、⼫}
  \begin{Phonetics}{市尺}{shi4 chi3}
    \definition{clas.}{chi, uma unidade tradicional de comprimento, equivalente a 0,333 metros ou 1,094 pés}
  \end{Phonetics}
\end{Entry}

\begin{Entry}{市斤}{5,4}{⼱、⽄}
  \begin{Phonetics}{市斤}{shi4jin1}
    \definition{clas.}{jin, uma unidade tradicional de peso, cada uma contendo 10 liang (市两)  e equivalente a 0,5 quilogramas ou 1,102 libras}
  \seealsoref{市两}{shi4liang3}
  \end{Phonetics}
\end{Entry}

\begin{Entry}{市长}{5,4}{⼱、⾧}
  \begin{Phonetics}{市长}{shi4 zhang3}[][HSK 2]
    \definition[个,位,名]{s.}{prefeito; chefe administrativo responsável pela administração de uma cidade}
  \end{Phonetics}
\end{Entry}

\begin{Entry}{市民}{5,5}{⼱、⽒}
  \begin{Phonetics}{市民}{shi4 min2}[][HSK 6]
    \definition[位,名]{s.}{habitantes da cidade; residente da cidade; moradores da cidade | cidadão; refere-se especificamente aos artesãos e comerciantes de pequeno e médio porte nas cidades da sociedade feudal tardia}
  \end{Phonetics}
\end{Entry}

\begin{Entry}{市场}{5,6}{⼱、⼟}
  \begin{Phonetics}{市场}{shi4chang3}[][HSK 3]
    \definition[家]{s.}{mercado (também no abstrato); um lugar fixo onde as pessoas compram e vendem coisas juntas | área de \emph{marketing}; região onde o produto é vendido | âmbito de influência (figurado); uma metáfora para o escopo e o grau em que uma determinada ideia ou comportamento é aceito por outros}
  \end{Phonetics}
\end{Entry}

\begin{Entry}{市两}{5,7}{⼱、⼀}
  \begin{Phonetics}{市两}{shi4liang3}
    \definition{clas.}{liang, uma unidade tradicional de peso, igual a 0,1 jin (市斤), e equivalente a 50 gramas ou 1,763 onças}
  \seealsoref{市斤}{shi4jin1}
  \end{Phonetics}
\end{Entry}

\begin{Entry}{市亩}{5,7}{⼱、⼇}
  \begin{Phonetics}{市亩}{shi4mu3}
    \definition{clas.}{mu, uma unidade tradicional de área, igual a 60 zhang quadrados (平方市丈) e equivalente a 6,667 ares ou 0,165 acre}
  \seealsoref{平方市丈}{ping2fang1 shi4 zhang4}
  \end{Phonetics}
\end{Entry}

%%%%%%%%%% 布 %%%%%%%%%%
\subsection*{布}

\begin{Entry}{布}{5}{⼱}
  \begin{Phonetics}{布}{bu4}[][HSK 3]
    \definition*{s.}{Sobrenome: Bu}
    \definition[块,幅,匹]{s.}{tecido; tecido de algodão; algodão, linho ou fibras sintéticas tecidas, que podem ser utilizadas como material para confecção de roupas ou outros objetos | uma moeda antiga | algo parecido com um pano}
    \definition{v.}{declarar; anunciar; publicar; proclamar | divulgar; espalhar por toda parte; difundir amplamente | implantar; dispor; organizar}
  \end{Phonetics}
\end{Entry}

\begin{Entry}{布局}{5,7}{⼱、⼫}
  \begin{Phonetics}{布局}{bu4ju2}[][HSK 7-9]
    \definition{s.}{\emph{layout}; distribuição; arranjo geral; arranjo abrangente: planejamento e arranjo da estrutura geral das coisas; especialmente o arranjo de materiais e tramas na criação artística}
    \definition{v.}{planejar; compor uma imagem, ensaio, etc. (geralmente se refere a escrever, pintar, jogar xadrez, etc.) | posicionar as peças em um tabuleiro de xadrez}
  \end{Phonetics}
\end{Entry}

\begin{Entry}{布谷鸟}{5,7,5}{⼱、⾕、⿃}
  \begin{Phonetics}{布谷鸟}{bu4gu3niao3}
    \definition{s.}{cuco (pássaro)}
  \seealsoref{杜鹃}{du4juan1}
  \seealsoref{杜鹃鸟}{du4juan1niao3}
  \seealsoref{杜宇}{du4yu3}
  \end{Phonetics}
\end{Entry}

\begin{Entry}{布满}{5,13}{⼱、⽔}
  \begin{Phonetics}{布满}{bu4 man3}[][HSK 6]
    \definition{v.}{abundar em; estar cheio de; espalhar-se e preencher um certo espaço}
  \end{Phonetics}
\end{Entry}

\begin{Entry}{布置}{5,13}{⼱、⽹}
  \begin{Phonetics}{布置}{bu4zhi4}[][HSK 4]
    \definition{v.}{arrumar; organizar; decorar; colocar adequadamente objetos ou paisagismo, conforme necessário | designar; tomar providências para; dar instruções sobre; organizar trabalho, atividades, etc.}
  \end{Phonetics}
\end{Entry}

\begin{Entry}{布署}{5,13}{⼱、⽹}
  \begin{Phonetics}{布署}{bu4shu3}
    \variantof{部署}
  \end{Phonetics}
\end{Entry}

%%%%%%%%%% 帅 %%%%%%%%%%
\subsection*{帅}

\begin{Entry}{帅}{5}{⼱}
  \begin{Phonetics}{帅}{shuai4}[][HSK 4]
    \definition*{s.}{Sobrenome: Shuai}
    \definition{adj.}{bonito; arrojado; elegante; inteligente}
    \definition{interj.}{Legal!}
    \definition[位,名,个,些]{s.}{comandante em chefe; o mais alto comandante do exército | comandante em chefe, a peça principal no xadrez chinês}
  \end{Phonetics}
\end{Entry}

\begin{Entry}{帅哥}{5,10}{⼱、⼝}
  \begin{Phonetics}{帅哥}{shuai4 ge1}[][HSK 4]
    \definition[个,位,名,些]{s.}{rapaz bonito; um garoto que é bonito e atraente na aparência}
  \end{Phonetics}
\end{Entry}

%%%%%%%%%% 平 %%%%%%%%%%
\subsection*{平}

\begin{Entry}{平}{5}{⼲}
  \begin{Phonetics}{平}{ping2}[][HSK 2]
    \definition*{s.}{Sobrenome: Ping}
    \definition{adj.}{plano; nivelado; uniforme; liso | igual; justo | mesma pontuação; empatado | médio; comum | silencioso; tranquilo | no mesmo nível; altura igual; sem diferença | imparcial; médio; equitativo | calmo; estável; tranquilo | comum;  frequente}
    \definition{s.}{no mesmo nível; em pé de igualdade com; igual | tom nivelado, um dos quatro tons do chinês clássico}
    \definition{v.}{tornar nivelado ou uniforme; nivelar | reprimir; suprimir | acalmar; tornar pacífico; silenciar (acalmar); conter a raiva | estar no mesmo nível | acalmar; amenizar; controlar a raiva}
  \end{Phonetics}
\end{Entry}

\begin{Entry}{平凡}{5,3}{⼲、⼏}
  \begin{Phonetics}{平凡}{ping2fan2}[][HSK 6]
    \definition{adj.}{comum; ordinário; normal; não surpreendente}
  \end{Phonetics}
\end{Entry}

\begin{Entry}{平方}{5,4}{⼲、⽅}
  \begin{Phonetics}{平方}{ping2fang1}[][HSK 4]
    \definition{s.}{Matemática: segunda potência (de uma quantidade); quadrado | metro quadrado (m²)}[那间房有十二平方。===O quarto tem doze metros quadrados.]
  \end{Phonetics}
\end{Entry}

\begin{Entry}{平方市丈}{5,4,5,3}{⼲、⽅、⼱、⼀}
  \begin{Phonetics}{平方市丈}{ping2fang1 shi4 zhang4}
    \definition{clas.}{pés quadrados}
  \end{Phonetics}
\end{Entry}

\begin{Entry}{平方米}{5,4,6}{⼲、⽅、⽶}
  \begin{Phonetics}{平方米}{ping2 fang1 mi3}[][HSK 6]
    \definition{s.}{metro quadrado; a unidade legal de medida de área, 1 metro quadrado é igual a 10.000 centímetros quadrados}
  \end{Phonetics}
\end{Entry}

\begin{Entry}{平台}{5,5}{⼲、⼝}
  \begin{Phonetics}{平台}{ping2 tai2}[][HSK 6]
    \definition[个]{s.}{casa com telhado plano rebocado | terraço | plataforma móvel; metaforicamente, refere-se às áreas, oportunidades, ambientes, espaços, etc. que fornecem suporte e garantia para algo | plataforma; um sistema em um computador eletrônico que consiste em software e hardware básicos; tal sistema pode suportar a execução de programas aplicativos e softwares aplicativos podem ser desenvolvidos nesse sistema | plataforma; lugar; falando metaforicamente, o mesmo nível ou grau}
  \end{Phonetics}
\end{Entry}

\begin{Entry}{平地}{5,6}{⼲、⼟}
  \begin{Phonetics}{平地}{ping2di4}
    \definition{v.}{nivelar a terra | aplanar}
  \end{Phonetics}
\end{Entry}

\begin{Entry}{平安}{5,6}{⼲、⼧}
  \begin{Phonetics}{平安}{ping2'an1}[][HSK 2]
    \definition{s.}{seguro; bem; sem contratempos; sem acidentes; são e salvo}
  \end{Phonetics}
\end{Entry}

\begin{Entry}{平均}{5,7}{⼲、⼟}
  \begin{Phonetics}{平均}{ping2jun1}[][HSK 4]
    \definition{adj.}{igual; médio}
    \definition{s.}{média}
    \definition{v.}{calcular a média de um conjunto de números}
  \end{Phonetics}
\end{Entry}

\begin{Entry}{平时}{5,7}{⼲、⽇}
  \begin{Phonetics}{平时}{ping2shi2}[][HSK 2]
    \definition{s.}{em tempos normais; em tempos comuns | em tempo de paz; refere-se a períodos normais}
  \end{Phonetics}
\end{Entry}

\begin{Entry}{平坦}{5,8}{⼲、⼟}
  \begin{Phonetics}{平坦}{ping2tan3}[][HSK 5]
    \definition{adj.}{plano; uniforme; nivelado; liso; sem elevações ou depressões (referindo-se principalmente ao relevo)}
  \end{Phonetics}
\end{Entry}

\begin{Entry}{平原}{5,10}{⼲、⼚}
  \begin{Phonetics}{平原}{ping2yuan2}[][HSK 5]
    \definition[片,个]{s.}{campo; planície; terreno plano e extenso}
  \end{Phonetics}
\end{Entry}

\begin{Entry}{平常}{5,11}{⼲、⼱}
  \begin{Phonetics}{平常}{ping2chang2}[][HSK 2]
    \definition{adj.}{comum; normal; ordinário; nada de especial}
    \definition{adv.}{normalmente; geralmente; como regra geral}
  \end{Phonetics}
\end{Entry}

\begin{Entry}{平等}{5,12}{⼲、⽵}
  \begin{Phonetics}{平等}{ping2deng3}[][HSK 2]
    \definition{adj.}{igual; igualdade; refere-se ao fato de as pessoas gozarem de tratamento igualitário nos aspectos sociais, políticos, econômicos e jurídicos}
  \end{Phonetics}
\end{Entry}

\begin{Entry}{平稳}{5,14}{⼲、⽲}
  \begin{Phonetics}{平稳}{ping2 wen3}[][HSK 4]
    \definition{adj.}{firme; estável; suave e constante; sem oscilações ou flutuações}
  \end{Phonetics}
\end{Entry}

\begin{Entry}{平静}{5,14}{⼲、⾭}
  \begin{Phonetics}{平静}{ping2jing4}[][HSK 4]
    \definition{adj.}{(humor, ambiente, etc.) calmo; quieto; pacífico; tranquilo}
  \end{Phonetics}
\end{Entry}

\begin{Entry}{平衡}{5,16}{⼲、⾏}
  \begin{Phonetics}{平衡}{ping2 heng2}[][HSK 6]
    \definition{adj.}{balanceado; equilibrado; os aspectos opostos são iguais ou compensados ​​em quantidade ou qualidade | equilibrado; várias forças atuam sobre um objeto com magnitude igual e direções opostas para manter o objeto estável}
    \definition{v.}{equilibrar; trazer ou manter em equilíbrio; tornar as coisas ou alimentos iguais em quantidade, qualidade ou força}
  \end{Phonetics}
\end{Entry}

%%%%%%%%%% 幼 %%%%%%%%%%
\subsection*{幼}

\begin{Entry}{幼}{5}{⼳}
  \begin{Phonetics}{幼}{you4}
    \definition{adj.}{jovem; menor de idade (oposto a 老)}
    \definition{s.}{crianças; os jovens}
  \seealsoref{老}{lao3}
  \end{Phonetics}
\end{Entry}

\begin{Entry}{幼儿园}{5,2,7}{⼳、⼉、⼞}
  \begin{Phonetics}{幼儿园}{you4'er2yuan2}[][HSK 4]
    \definition[家,所]{s.}{jardim de infância; escola maternal; escola infantil; instituição para a educação de crianças pequenas}
  \end{Phonetics}
\end{Entry}

%%%%%%%%%% 弘 %%%%%%%%%%
\subsection*{弘}

\begin{Entry}{弘}{5}{⼸}
  \begin{Phonetics}{弘}{hong2}
    \definition*{s.}{Sobrenome: Hong}
    \definition{adj.}{grande; grandioso; magnífico}
    \definition{v.}{ampliar; expandir | promover}
  \end{Phonetics}
\end{Entry}

\begin{Entry}{弘扬}{5,6}{⼸、⼿}
  \begin{Phonetics}{弘扬}{hong2yang2}[][HSK 7-9]
    \definition{v.}{melhorar; levar adiante; desenvolver e expandir; promover; promover vigorosamente}
  \end{Phonetics}
\end{Entry}

%%%%%%%%%% 归 %%%%%%%%%%
\subsection*{归}

\begin{Entry}{归}{5}{⼹}
  \begin{Phonetics}{归}{gui1}[][HSK 4]
    \definition*{s.}{Sobrenome: Gui}
    \definition{s.}{divisão no ábaco com divisor de um dígito}
    \definition{v.}{retornar; voltar para; voltar (ou ir) | devolver algo a; dar de volta a | convergir; juntar-se | encarregar alguém de algo | atribuir a; pertencer a}
    \definition{v.aux.}{usado entre dois verbos idênticos, indicando que a ação não levou ao resultado correspondente}
  \end{Phonetics}
\end{Entry}

\begin{Entry}{归来}{5,7}{⼹、⽊}
  \begin{Phonetics}{归来}{gui1lai2}[][HSK 7-9]
    \definition{v.}{retornar; voltar ou estar de volta; retornar ao local de onde você começou ou partiu de outro lugar}
  \end{Phonetics}
\end{Entry}

\begin{Entry}{归纳}{5,7}{⼹、⽷}
  \begin{Phonetics}{归纳}{gui1na4}[][HSK 7-9]
    \definition{s.}{indução; método indutivo}
    \definition{v.}{induzir; concluir; mesclar e classificar; resumir (usado principalmente para coisas abstratas)}
  \end{Phonetics}
\end{Entry}

\begin{Entry}{归还}{5,7}{⼹、⾡}
  \begin{Phonetics}{归还}{gui1huan2}[][HSK 7-9]
    \definition{v.}{retornar; reverter (oposto a 借用); devolver dinheiro ou itens emprestados ao proprietário original}
  \seealsoref{借用}{jie4yong4}
  \end{Phonetics}
\end{Entry}

\begin{Entry}{归结}{5,9}{⼹、⽷}
  \begin{Phonetics}{归结}{gui1jie2}[][HSK 7-9]
    \definition{s.}{fim; final (de uma história, etc.) | resolução}
    \definition{v.}{chegar a uma conclusão; resumir; colocar em poucas palavras}
  \end{Phonetics}
\end{Entry}

\begin{Entry}{归根到底}{5,10,8,8}{⼹、⽊、⼑、⼴}
  \begin{Phonetics}{归根到底}{gui1gen1-dao4di3}[][HSK 7-9]
    \definition{expr.}{``Em última análise.'', significa que, no final, as coisas acabarão de uma certa maneira; ela vem de 《何典》, de 张南庄, da Dinastia Qing (清); na análise final (última); no longo prazo; afinal; na análise final; em essência; fundamentalmente}
  \seealsoref{何典}{he2 dian3}
  \seealsoref{清}{qing1}
  \seealsoref{张南庄}{zhang1 nan2zhuang1}
  \end{Phonetics}
\end{Entry}

\begin{Entry}{归宿}{5,11}{⼹、⼧}
  \begin{Phonetics}{归宿}{gui1su4}[][HSK 7-9]
    \definition{s.}{um lar para retornar; destino final; fim}
  \end{Phonetics}
\end{Entry}

\begin{Entry}{归属}{5,12}{⼹、⼫}
  \begin{Phonetics}{归属}{gui1shu3}[][HSK 7-9]
    \definition{v.}{pertencer a; estar sob a jurisdição de; definir afiliação}
  \end{Phonetics}
\end{Entry}

%%%%%%%%%% 必 %%%%%%%%%%
\subsection*{必}

\begin{Entry}{必}{5}{⼼}
  \begin{Phonetics}{必}{bi4}[][HSK 5]
    \definition{adv.}{certamente; necessariamente; indica que algo é certo ou que alguém acredita que esteja correto | deve; tem que}
  \end{Phonetics}
\end{Entry}

\begin{Entry}{必不可少}{5,4,5,4}{⼼、⼀、⼝、⼩}
  \begin{Phonetics}{必不可少}{bi4bu4ke3shao3}[][HSK 7-9]
    \definition{suf.}{indispensável; essencial; absolutamente necessário; insubstituível; inevitável}
  \end{Phonetics}
\end{Entry}

\begin{Entry}{必定}{5,8}{⼼、⼧}
  \begin{Phonetics}{必定}{bi4ding4}[][HSK 7-9]
    \definition{adv.}{certamente; estar vinculado a; ter certeza de; para expressar a certeza de um julgamento ou inferência | definitivamente; para expressar determinação de vontade; ter certeza de fazê-lo}
  \end{Phonetics}
\end{Entry}

\begin{Entry}{必修}{5,9}{⼼、⼈}
  \begin{Phonetics}{必修}{bi4 xiu1}[][HSK 6]
    \definition{adj.}{(de um curso acadêmico) obrigatório; compulsório; mandatório; obrigatório estudar de acordo com os regulamentos (em oposição a 选修)}
  \seealsoref{选修}{xuan3 xiu1}
  \end{Phonetics}
\end{Entry}

\begin{Entry}{必将}{5,9}{⼼、⼨}
  \begin{Phonetics}{必将}{bi4 jiang1}[][HSK 6]
    \definition{adv.}{certamente; certamente irá; usado para expressar inevitabilidade (ou necessidade)}
  \end{Phonetics}
\end{Entry}

\begin{Entry}{必要}{5,9}{⼼、⾑}
  \begin{Phonetics}{必要}{bi4yao4}[][HSK 3]
    \definition{adj.}{necessário; essencial; indispensável}
    \definition[个,些]{s.}{necessidade; características indispensáveis}
  \end{Phonetics}
\end{Entry}

\begin{Entry}{必须}{5,9}{⼼、⾴}
  \begin{Phonetics}{必须}{bi4xu1}[][HSK 2]
    \definition{adv.}{necessariamente; obrigatoriamente; indica a necessidade lógica e emocional | deve; tem que; é obrigado a}
  \end{Phonetics}
\end{Entry}

\begin{Entry}{必然}{5,12}{⼼、⽕}
  \begin{Phonetics}{必然}{bi4ran2}[][HSK 3]
    \definition{adj.}{certo; inevitável; necessário; definido e inalterável; imutável}
    \definition{adv.}{inevitavelmente}
    \definition{s.}{necessidade; em filosofia, refere-se às leis objetivas do desenvolvimento que não são influenciadas pela vontade humana}
  \end{Phonetics}
\end{Entry}

\begin{Entry}{必需}{5,14}{⼼、⾬}
  \begin{Phonetics}{必需}{bi4 xu1}[][HSK 5]
    \definition{adj.}{essencial; indispensável}
    \definition{v.}{ser essencial; ser indispensável}
  \end{Phonetics}
\end{Entry}

%%%%%%%%%% 扑 %%%%%%%%%%
\subsection*{扑}

\begin{Entry}{扑}{5}{⼿}
  \begin{Phonetics}{扑}{pu1}[][HSK 6]
    \definition{s.}{sopro; refere-se a gases, fragrâncias, cinzas, areia, etc. que se apresentam | espanador}
    \definition{v.}{atacar; lançar-se sobre; correr para frente com toda a sua força e, de repente, jogar todo o seu corpo em um objeto | dedicar; dedicar todas as energias a uma causa; colocar toda a sua energia em (trabalho, carreira, etc.) | bater asas; esvoaçar | inclinar-se}
  \end{Phonetics}
\end{Entry}

\begin{Entry}{扑克}{5,7}{⼿、⼗}
  \begin{Phonetics}{扑克}{pu1ke4}
    \definition{s.}{(empréstimo linguístico) (jogo) \emph{poker}  | baralho}
  \end{Phonetics}
\end{Entry}

%%%%%%%%%% 扒 %%%%%%%%%%
\subsection*{扒}

\begin{Entry}{扒}{5}{⼿}
  \begin{Phonetics}{扒}{ba1}[][HSK 7-9]
    \definition{v.}{segurar; agarrar-se a | cavar; varrer; puxar para baixo | empurrar para o lado | despir-se; tirar}
  \end{Phonetics}
  \begin{Phonetics}{扒}{pa2}
    \definition{v.}{reunir; juntar; reunir ou espalhar coisas com as mãos ou com um ancinho | roubar; furtar | arranhar; coçar com as mãos | cozinhar; refogar; cozinhar os alimentos em fogo baixo}
  \end{Phonetics}
\end{Entry}

\begin{Entry}{扒犁}{5,11}{⼿、⽜}
  \begin{Phonetics}{扒犁}{pa2li2}
    \definition{s.}{Dialeto: trenó; arado}
  \seealsoref{爬犁}{pa2li2}
  \end{Phonetics}
\end{Entry}

%%%%%%%%%% 打 %%%%%%%%%%
\subsection*{打}

\begin{Entry}{打}{5}{⼿}
  \begin{Phonetics}{打}{da2}
    \definition{clas./s.}{(empréstimo linguístico) dúzia}
  \end{Phonetics}
  \begin{Phonetics}{打}{da3}[][HSK 1,4,5]
    \definition{prep.}{de; desde; ponto de partida que indica lugar, tempo ou extensão; indica rotas e locais percorridos | devido a; origem da introdução de coisas novas}
    \definition{v.}{golpear; acertar; bater | quebrar; esmagar | lutar; atacar; espancar | entrar com uma ação judicial; negociar; fazer representações | construir; edificar | fabricar (em uma ferraria); forjar | misturar; mexer; bater | amarrar; embalar | tricotar; tecer | desenhar; pintar; deixar uma marca; imprimir | abrir; perfurar; cavar | içar; levantar | enviar; despachar; projetar | emitir ou receber (um certificado, etc.) | remover; livrar-se de | colher; tirar; retirar | comprar | capturar; caçar | reunir; coletar; colher; recolher através de ações como cortar e podar | estimar; calcular; contar; determinar | fazer; envolver-se em | jogar algum tipo de jogo | expressar certos movimentos corporais | adotar; usar; adotar uma determinada abordagem | pegar (um táxi) | indicar a melhora de seu estado mental; melhorar o estado mental}
  \end{Phonetics}
\end{Entry}

\begin{Entry}{打工}{5,3}{⼿、⼯}
  \begin{Phonetics}{打工}{da3gong1}[][HSK 2]
    \definition{v.}{contratar para trabalhar; trabalhar em tempo parcial; realizar trabalho manual (para alguém, geralmente temporariamente)}
  \end{Phonetics}
\end{Entry}

\begin{Entry}{打工人}{5,3,2}{⼿、⼯、⼈}
  \begin{Phonetics}{打工人}{da3gong1ren2}
    \definition{s.}{trabalhador}
  \end{Phonetics}
\end{Entry}

\begin{Entry}{打开}{5,4}{⼿、⼶}
  \begin{Phonetics}{打开}{da3 kai1}[][HSK 1]
    \definition{v.}{abrir; desdobrar; desenrolar | descobrir; revelar; desvendar | ativar; ligar; ligar o circuito | romper | abrir-se; espalhar-se; expandir; ampliar | abrir; iniciar o funcionamento do software, etc.}
  \end{Phonetics}
\end{Entry}

\begin{Entry}{打车}{5,4}{⼿、⾞}
  \begin{Phonetics}{打车}{da3 che1}[][HSK 1]
    \definition{v.}{pegar um táxi; chamar um táxi; dar sinal para um táxi}
  \end{Phonetics}
\end{Entry}

\begin{Entry}{打仗}{5,5}{⼿、⼈}
  \begin{Phonetics}{打仗}{da3/zhang4}[][HSK 7-9]
    \definition{v.+compl.}{lutar; ir à guerra; fazer guerra}
  \end{Phonetics}
\end{Entry}

\begin{Entry}{打击}{5,5}{⼿、⼐}
  \begin{Phonetics}{打击}{da3ji1}[][HSK 5]
    \definition{v.}{golpear; atacar; reprimir; atacar para frustrar; machucar | bater; bater (em um tambor, etc.); golpear ou bater em algo}
  \end{Phonetics}
\end{Entry}

\begin{Entry}{打包}{5,5}{⼿、⼓}
  \begin{Phonetics}{打包}{da3bao1}[][HSK 5]
    \definition{v.}{levar a comida embora; levar para viagem; refere-se especificamente a comer em um restaurante e levar as sobras em uma caixa, sacola ou outro recipiente | embalar; empacotar | desembalar; desempacotar}
  \end{Phonetics}
\end{Entry}

\begin{Entry}{打印}{5,5}{⼿、⼙}
  \begin{Phonetics}{打印}{da3yin4}[][HSK 2]
    \definition{v.}{imprimir; imprimir em papel ou outro suporte de gravação, como uma impressora}
  \end{Phonetics}
\end{Entry}

\begin{Entry}{打印机}{5,5,6}{⼿、⼙、⽊}
  \begin{Phonetics}{打印机}{da3 yin4 ji1}[][HSK 6]
    \definition[个,部,台]{s.}{impressora; uma máquina de escrever controlada por um microcomputador, sem teclado, que converte códigos de caracteres em caracteres e os imprime}
  \end{Phonetics}
\end{Entry}

\begin{Entry}{打发}{5,5}{⼿、⼜}
  \begin{Phonetics}{打发}{da3 fa5}[][HSK 6]
    \definition{v.}{enviar; despachar | dispensar; mandar embora | passar o tempo; matar o tempo}
  \end{Phonetics}
\end{Entry}

\begin{Entry}{打电话}{5,5,8}{⼿、⽥、⾔}
  \begin{Phonetics}{打电话}{da3 dian4 hua4}[][HSK 1]
    \definition{v.}{telefonar; fazer uma chamada telefônica; dar um telefonema}
  \seealsoref{给……打电话}{gei3 da3 dian4 hua4}
  \end{Phonetics}
\end{Entry}

\begin{Entry}{打交道}{5,6,12}{⼿、⼇、⾡}
  \begin{Phonetics}{打交道}{da3 jiao1dao5}[][HSK 7-9]
    \definition{v.}{mediar; formar equipe; entrar em contato com; fazer contato com; ter relações com}
  \end{Phonetics}
\end{Entry}

\begin{Entry}{打动}{5,6}{⼿、⼒}
  \begin{Phonetics}{打动}{da3 dong4}[][HSK 6]
    \definition{v.}{mover; tocar}[这番话打动了她的心。===Essas palavras tocaram seu coração.]
  \end{Phonetics}
\end{Entry}

\begin{Entry}{打压}{5,6}{⼿、⼚}
  \begin{Phonetics}{打压}{da3ya1}
    \definition{v.}{reprimir | derrotar}
  \end{Phonetics}
\end{Entry}

\begin{Entry}{打扫}{5,6}{⼿、⼿}
  \begin{Phonetics}{打扫}{da3sao3}[][HSK 4]
    \definition{v.}{varrer; limpar; varrer para limpar}
  \end{Phonetics}
\end{Entry}

\begin{Entry}{打听}{5,7}{⼿、⼝}
  \begin{Phonetics}{打听}{da3ting5}[][HSK 3]
    \definition{v.}{perguntar sobre; indagar sobre; obter uma linha sobre}
  \end{Phonetics}
\end{Entry}

\begin{Entry}{打屁股}{5,7,8}{⼿、⼫、⾁}
  \begin{Phonetics}{打屁股}{da3pi4gu5}
    \definition{v.}{dar um tapa no bumbum de alguém}
  \end{Phonetics}
\end{Entry}

\begin{Entry}{打岔}{5,7}{⼿、⼭}
  \begin{Phonetics}{打岔}{da3/cha4}[][HSK 7-9]
    \definition{s.}{interrupção}
    \definition{v.+compl.}{interromper; cortar | mudar de assunto | interromper (especialmente a fala)}
  \end{Phonetics}
\end{Entry}

\begin{Entry}{打扮}{5,7}{⼿、⼿}
  \begin{Phonetics}{打扮}{da3ban5}[][HSK 5]
    \definition{s.}{estilo de se vestir; o modo de se vestir; as roupas que se usa}
    \definition{v.}{vestir-se bem; maquiar-se; dar uma boa aparência e vestir-se bem; adornar}
  \end{Phonetics}
\end{Entry}

\begin{Entry}{打扰}{5,7}{⼿、⼿}
  \begin{Phonetics}{打扰}{da3rao3}[][HSK 5]
    \definition{v.}{perturbar; incomodar; interferir no trabalho normal, na vida ou no que as outras pessoas estão fazendo, etc. | usado para expressar um pedido de desculpas por ajuda; gratidão por ajuda; hospitalidade recebida}
  \end{Phonetics}
\end{Entry}

\begin{Entry}{打折}{5,7}{⼿、⼿}
  \begin{Phonetics}{打折}{da3/zhe2}[][HSK 4]
    \definition{v.+compl.}{dar desconto; dar um desconto; vender produtos a um preço reduzido em uma determinada porcentagem do preço original; metáfora para não cumprir 100\% do que foi originalmente padronizado ou prometido}
  \end{Phonetics}
\end{Entry}

\begin{Entry}{打针}{5,7}{⼿、⾦}
  \begin{Phonetics}{打针}{da3/zhen1}[][HSK 4]
    \definition{v.+compl.}{dar ou receber uma injeção; injetar um medicamento líquido em um organismo com uma seringa}
  \end{Phonetics}
\end{Entry}

\begin{Entry}{打官司}{5,8,5}{⼿、⼧、⼝}
  \begin{Phonetics}{打官司}{da3/guan1si5}[][HSK 6]
    \definition{v.+compl.}{ir ao tribunal (ou à lei); envolver-se em um processo judicial}
  \end{Phonetics}
\end{Entry}

\begin{Entry}{打招呼}{5,8,8}{⼿、⼿、⼝}
  \begin{Phonetics}{打招呼}{da3 zhao1hu5}[][HSK 7-9]
    \definition{v.}{cumprimentar alguém; dizer olá; tirar o chapéu; saudações por meio de palavras ou gestos | avisar; lembrar; informar; notificar com antecedência; dar aviso prévio}
  \end{Phonetics}
\end{Entry}

\begin{Entry}{打的}{5,8}{⼿、⽩}
  \begin{Phonetics}{打的}{da3/di1}
    \definition{v.+compl.}{(coloquial) pegar um táxi | ir de táxi}
  \end{Phonetics}
\end{Entry}

\begin{Entry}{打败}{5,8}{⼿、⾒}
  \begin{Phonetics}{打败}{da3 bai4}[][HSK 4]
    \definition{v.}{derrotar; vencer; piorar | sofrer uma derrota; ser derrotado}
  \end{Phonetics}
\end{Entry}

\begin{Entry}{打架}{5,9}{⼿、⽊}
  \begin{Phonetics}{打架}{da3/jia4}[][HSK 5]
    \definition{v.+compl.}{brigar; discutir; entrar em conflito | contradizer; conflitar; ser inconsistente}
  \end{Phonetics}
\end{Entry}

\begin{Entry}{打盹儿}{5,9,2}{⼿、⽬、⼉}
  \begin{Phonetics}{打盹儿}{da3/dun3r5}[][HSK 7-9]
    \definition{v.+compl.}{cochilar; tirar uma soneca}
  \end{Phonetics}
\end{Entry}

\begin{Entry}{打结}{5,9}{⼿、⽷}
  \begin{Phonetics}{打结}{da3jie2}
    \definition{v.}{dar um nó | amarrar}
  \end{Phonetics}
\end{Entry}

\begin{Entry}{打骂}{5,9}{⼿、⾺}
  \begin{Phonetics}{打骂}{da3ma4}
    \definition{v.}{bater e repreender}
  \end{Phonetics}
\end{Entry}

\begin{Entry}{打倒}{5,10}{⼿、⼈}
  \begin{Phonetics}{打倒}{da3/dao3}[][HSK 7-9]
    \definition{v.+compl.}{atacar e derrubar no chão; cair | derrubar; tombar}[打倒法西斯政权。===Derrubar o regime fascista.]
  \end{Phonetics}
\end{Entry}

\begin{Entry}{打捞}{5,10}{⼿、⼿}
  \begin{Phonetics}{打捞}{da3lao1}[][HSK 7-9]
    \definition{s.}{salvamento; resgate}
    \definition{v.}{sair da água; resgatar | encontrar e recuperar objetos que afundaram na água}
  \end{Phonetics}
\end{Entry}

\begin{Entry}{打破}{5,10}{⼿、⽯}
  \begin{Phonetics}{打破}{da3 po4}[][HSK 3]
    \definition{v.}{quebrar; esmagar; quebrar recordes, regras ou restrições existentes, etc.}
  \end{Phonetics}
\end{Entry}

\begin{Entry}{打通}{5,10}{⼿、⾡}
  \begin{Phonetics}{打通}{da3/tong1}[][HSK 7-9]
    \definition{v.+compl.}{passar; abrir-se | estabelecer contato | abrir o acesso | passar (uma conexão telefônica) | remover um bloco}
  \end{Phonetics}
\end{Entry}

\begin{Entry}{打造}{5,10}{⼿、⾡}
  \begin{Phonetics}{打造}{da3 zao4}[][HSK 6]
    \definition{v.}{forjar (trabalhar em metal); fabricar (principalmente objetos de metal) | fazer; criar; construir; desenvolver}
  \end{Phonetics}
\end{Entry}

\begin{Entry}{打断}{5,11}{⼿、⽄}
  \begin{Phonetics}{打断}{da3 duan4}[][HSK 6]
    \definition{v.}{interromper uma atividade (fala; pensamento ou ação) | fraturar (osso do corpo)  com força | arrombar; bater com força para quebrar}
  \end{Phonetics}
\end{Entry}

\begin{Entry}{打猎}{5,11}{⼿、⽝}
  \begin{Phonetics}{打猎}{da3/lie4}[][HSK 7-9]
    \definition{v.+compl.}{ir caçar}
  \end{Phonetics}
\end{Entry}

\begin{Entry}{打球}{5,11}{⼿、⽟}
  \begin{Phonetics}{打球}{da3 qiu2}[][HSK 1]
    \definition{v.}{jogar bola (com as mãos) | jogar (basquetebol, handbol, etc.) | jogar um jogo de bola}
  \end{Phonetics}
\end{Entry}

\begin{Entry}{打搅}{5,12}{⼿、⼿}
  \begin{Phonetics}{打搅}{da3jiao3}[][HSK 7-9]
    \definition{v.}{perturbar; incomodar | interromper}
  \end{Phonetics}
\end{Entry}

\begin{Entry}{打牌}{5,12}{⼿、⽚}
  \begin{Phonetics}{打牌}{da3 pai2}[][HSK 6]
    \definition{v.}{jogar cartas, usar cartas para entretenimento ou jogos de azar}
  \end{Phonetics}
\end{Entry}

\begin{Entry}{打量}{5,12}{⼿、⾥}
  \begin{Phonetics}{打量}{da3liang5}[][HSK 7-9]
    \definition{v.}{aumentar o tamanho; medir com o olho; olhar de cima a baixo; examinar; observar | pensar; calcular; supor; estimar}
  \end{Phonetics}
\end{Entry}

\begin{Entry}{打雷}{5,13}{⼿、⾬}
  \begin{Phonetics}{打雷}{da3 lei2}[][HSK 4]
    \definition{v.}{trovejar; produzir ruídos altos quando as nuvens descarregam eletricidade}
  \end{Phonetics}
\end{Entry}

\begin{Entry}{打算}{5,14}{⼿、⽵}
  \begin{Phonetics}{打算}{da3suan4}[][HSK 2]
    \definition[个,项]{s.}{plano; intenção; consideração; cálculo; ideias sobre a direção e os métodos da ação; pensamentos}
    \definition{v.}{pretender; planejar; calcular; considerar com antecedência}
  \end{Phonetics}
\end{Entry}

\begin{Entry}{打瞌睡}{5,15,13}{⼿、⽬、⽬}
  \begin{Phonetics}{打瞌睡}{da3ke1shui4}
    \definition{v.}{cochilar}
  \end{Phonetics}
\end{Entry}

\begin{Entry}{打磨}{5,16}{⼿、⽯}
  \begin{Phonetics}{打磨}{da3mo2}[][HSK 7-9]
    \definition{v.}{polir; dar brilho; fazer brilhar; esfregar a superfície de um objeto para torná-lo liso e delicado}
  \end{Phonetics}
\end{Entry}

%%%%%%%%%% 扔 %%%%%%%%%%
\subsection*{扔}

\begin{Entry}{扔}{5}{⼿}
  \begin{Phonetics}{扔}{reng1}[][HSK 5]
    \definition{v.}{arremessar; lançar; atirar; jogar | esquecer; jogar fora; descartar | colocar casualmente; deixar as pessoas ou as coisas de lado, não se importar}
  \end{Phonetics}
\end{Entry}

\begin{Entry}{扔下}{5,3}{⼿、⼀}
  \begin{Phonetics}{扔下}{reng1xia4}
    \definition{v.}{lançar (uma bomba) | derrubar}
  \end{Phonetics}
\end{Entry}

\begin{Entry}{扔弃}{5,7}{⼿、⼶}
  \begin{Phonetics}{扔弃}{reng1qi4}
    \definition{v.}{abandonar | descartar | jogar fora}
  \end{Phonetics}
\end{Entry}

\begin{Entry}{扔掉}{5,11}{⼿、⼿}
  \begin{Phonetics}{扔掉}{reng1diao4}
    \definition{v.}{jogar fora}
  \end{Phonetics}
\end{Entry}

%%%%%%%%%% 斥 %%%%%%%%%%
\subsection*{斥}

\begin{Entry}{斥}{5}{⽄}
  \begin{Phonetics}{斥}{chi4}
    \definition*{s.}{Sobrenome: Chi}
    \definition{adj.}{(solo) salino; alcalino}
    \definition{s.}{terra impregnada de sal, portanto estéril}
    \definition{v.}{repreender; censurar; denunciar; reprimir | repelir; excluir; expulsar | fornecer; prover | Literário: abrir; expandir | culpar; reprovar | estender; ampliar | Obsoleto: reconhecer; detectar}
  \end{Phonetics}
\end{Entry}

\begin{Entry}{斥骂}{5,9}{⽄、⾺}
  \begin{Phonetics}{斥骂}{chi4ma4}
    \definition{v.}{repreender}
  \end{Phonetics}
\end{Entry}

%%%%%%%%%% 旧 %%%%%%%%%%
\subsection*{旧}

\begin{Entry}{旧}{5}{⽇}
  \begin{Phonetics}{旧}{jiu4}[][HSK 3]
    \definition{adj.}{passado; antigo; velho; ultrapassado (em oposição a 新)| usado; desgastado; velho; descolorido ou deformado devido ao uso prolongado ou ao tempo | antigo; único; que já existiu; anterior}
    \definition{s.}{velha amizade; velho amigo}
  \seealsoref{新}{xin1}
  \end{Phonetics}
\end{Entry}

%%%%%%%%%% 未 %%%%%%%%%%
\subsection*{未}

\begin{Entry}{未}{5}{⽊}
  \begin{Phonetics}{未}{wei4}
    \definition*{s.}{Sobrenome: Wei}
    \definition{adv.}{Literário: não tem; não fez; (oposto a 已) | Literário: não}
    \definition{part.}{ou não; no final das perguntas, indicando dúvida}[今可以言未?===Posso falar agora?]
    \definition{s.}{wei (oitavo dos doze Ramos Terrestres)}
  \seealsoref{已}{yi3}
  \end{Phonetics}
\end{Entry}

\begin{Entry}{未必}{5,5}{⽊、⼼}
  \begin{Phonetics}{未必}{wei4bi4}[][HSK 4]
    \definition{adv.}{não tenho certeza; talvez não; não necessariamente}
  \end{Phonetics}
\end{Entry}

\begin{Entry}{未来}{5,7}{⽊、⽊}
  \begin{Phonetics}{未来}{wei4lai2}[][HSK 4]
    \definition{adj.}{próximo (refere-se ao tempo)}
    \definition[个,段,种]{s.}{futuro; o amanhã}
  \end{Phonetics}
\end{Entry}

%%%%%%%%%% 末 %%%%%%%%%%
\subsection*{末}

\begin{Entry}{末}{5}{⽊}
  \begin{Phonetics}{末}{mo4}[][HSK 4]
    \definition{adj.}{último; final}
    \definition{s.}{ponta; terminal; extremidade; o final de algo | não essenciais; detalhes secundários | fim; final | pó; poeira | um papel na ópera tradicional}
  \end{Phonetics}
\end{Entry}

%%%%%%%%%% 本 %%%%%%%%%%
\subsection*{本}

\begin{Entry}{本}{5}{⽊}
  \begin{Phonetics}{本}{ben3}[][HSK 1,6]
    \definition*{s.}{Sobrenome: Ben}
    \definition{adj.}{original; inerente | principal; central}
    \definition{adv.}{originalmente}
    \definition{clas.}{para livros, dicionários, periódicos, arquivos, etc. | para vídeos de uma determinada duração | para peças de teatro, ópera}
    \definition{prep.}{de acordo com; em consonância com; em conformidade com; equivalentes a 依照 e 按照}
    \definition{pron.}{nativo; próprio; refere-se ao próprio interlocutor ou ao grupo, instituição, empresa, local, etc. ao qual o interlocutor pertence | isto; atual; presente}
    \definition[个]{s.}{caule ou raiz de plantas | base; origem; fundamento; fundação;  alicerce | capital; capital social | livro; caderno; livreto | edição; versão | cópia; roteiro; manuscrito | memorial do trono; na era feudal, referia-se a um documento oficial}
    \definition{v.}{seguir; basear-se em; estar de acordo com}
  \seealsoref{按照}{an4zhao4}
  \seealsoref{依照}{yi1 zhao4}
  \end{Phonetics}
\end{Entry}

\begin{Entry}{本人}{5,2}{⽊、⼈}
  \begin{Phonetics}{本人}{ben3ren2}[][HSK 5]
    \definition{pron.}{eu (mim, mim mesmo); o orador refere-se a si mesmo | a si mesmo; em pessoa; refere-se à própria pessoa ou à pessoa mencionada anteriormente}
  \end{Phonetics}
\end{Entry}

\begin{Entry}{本土}{5,3}{⽊、⼟}
  \begin{Phonetics}{本土}{ben3 tu3}[][HSK 6]
    \definition{s.}{território metropolitano; pátria-mãe; refere-se ao território do país | país (ou terra) natal de alguém; nativo; cidade natal; local original de crescimento}
  \end{Phonetics}
\end{Entry}

\begin{Entry}{本子}{5,3}{⽊、⼦}
  \begin{Phonetics}{本子}{ben3 zi5}[][HSK 1]
    \definition[个,本]{s.}{livro; caderno | edição | impressão | licença; certificado de competência emitido por uma instituição especializada, obtido após aprovação no exame | \emph{script}; roteiro}
  \end{Phonetics}
\end{Entry}

\begin{Entry}{本分}{5,4}{⽊、⼑}
  \begin{Phonetics}{本分}{ben3fen4}[][HSK 7-9]
    \definition{adj.}{honesto; decente}
    \definition{s.}{o trabalho de alguém; o dever de alguém; as próprias responsabilidades e obrigações | estar satisfeito com sua posição e ambiente atuais}
  \end{Phonetics}
\end{Entry}

\begin{Entry}{本地}{5,6}{⽊、⼟}
  \begin{Phonetics}{本地}{ben3 di4}[][HSK 6]
    \definition{s.}{local; nativo; localidade; a área onde as pessoas e as coisas estão localizadas; uma área específica referida em uma narrativa}
  \end{Phonetics}
\end{Entry}

\begin{Entry}{本色}{5,6}{⽊、⾊}
  \begin{Phonetics}{本色}{ben3se4}[][HSK 7-9]
    \definition{s.}{(algo, geralmente tecidos não tingidos) cor natural; a cor original de algo (geralmente se refere a tecidos não tingidos)}
  \end{Phonetics}
\end{Entry}

\begin{Entry}{本来}{5,7}{⽊、⽊}
  \begin{Phonetics}{本来}{ben3lai2}[][HSK 3]
    \definition{adj.}{original}
    \definition{adv.}{anteriormente; originalmente; indica que antes disso | claro; em primeiro lugar; como deveria ser; indica que algo é natural ou óbvio}
  \end{Phonetics}
\end{Entry}

\begin{Entry}{本身}{5,7}{⽊、⾝}
  \begin{Phonetics}{本身}{ben3shen1}[][HSK 6]
    \definition{pron.}{próprio; em si mesmo; refere-se à pessoa, unidade ou coisa em si}
  \end{Phonetics}
\end{Entry}

\begin{Entry}{本事}{5,8}{⽊、⼅}
  \begin{Phonetics}{本事}{ben3shi4}
    \definition{s.}{habilidade; aptidão; capacidade; competência; refere-se às habilidades, capacidades ou talentos que uma pessoa possui em determinada área | habilidade; aptidão; capacidade; competência; a capacidade e os meios necessários para atingir um determinado objetivo ou concluir uma determinada tarefa | status; poder; posição; autoridade; refere-se à identidade, posição ou poder de uma pessoa.}
  \end{Phonetics}
  \begin{Phonetics}{本事}{ben3shi5}[][HSK 3]
    \definition{s.}{habilidade; capacidade; talento; aptidão}
  \end{Phonetics}
\end{Entry}

\begin{Entry}{本性}{5,8}{⽊、⼼}
  \begin{Phonetics}{本性}{ben3xing4}[][HSK 7-9]
    \definition{s.}{qualidade inerente | instintos naturais | natureza}
  \end{Phonetics}
\end{Entry}

\begin{Entry}{本质}{5,8}{⽊、⾙}
  \begin{Phonetics}{本质}{ben3zhi4}[][HSK 6]
    \definition{s.}{essência; natureza; caráter inato; qualidade intrínseca; refere-se aos atributos fundamentais inerentes às próprias coisas, que desempenham um papel decisivo na natureza, condição e desenvolvimento das coisas (distinguido de 现象)}
  \seealsoref{现象}{xian4xiang4}
  \end{Phonetics}
\end{Entry}

\begin{Entry}{本金}{5,8}{⽊、⾦}
  \begin{Phonetics}{本金}{ben3 jin1}
    \definition{s.}{capital; capital para a operação do comércio e da indústria; capital para a operação de negócios | valor principal; dinheiro retirado ao depositar ou tomar emprestado (diferente de 利息)}
  \seealsoref{利息}{li4xi1}
  \end{Phonetics}
\end{Entry}

\begin{Entry}{本科}{5,9}{⽊、⽲}
  \begin{Phonetics}{本科}{ben3ke1}[][HSK 4]
    \definition{s.}{graduação; bacharelado; o curso básico de uma universidade ou faculdade}
  \end{Phonetics}
\end{Entry}

\begin{Entry}{本能}{5,10}{⽊、⾁}
  \begin{Phonetics}{本能}{ben3neng2}[][HSK 7-9]
    \definition{adv.}{instintivamente; pela luz da natureza; inconscientemente, subconscientemente}
    \definition{s.}{instinto; as habilidades que humanos e animais desenvolvem durante o processo de evolução são fixadas pela hereditariedade e não precisam ser ensinadas}
  \end{Phonetics}
\end{Entry}

\begin{Entry}{本钱}{5,10}{⽊、⾦}
  \begin{Phonetics}{本钱}{ben3qian2}[][HSK 7-9]
    \definition{s.}{capital; dinheiro usado para gerar lucros, juros ou para participar de atividades como jogos de azar | habilidades, qualificações ou condições que podem ser usadas para fazer algo; metaforicamente falando, qualificações, habilidades e outras condições em que se pode confiar}
  \end{Phonetics}
\end{Entry}

\begin{Entry}{本着}{5,11}{⽊、⽬}
  \begin{Phonetics}{本着}{ben3zhe5}[][HSK 7-9]
    \definition{prep.}{com base em; de acordo com; em conformidade com; à luz de}
  \end{Phonetics}
\end{Entry}

\begin{Entry}{本领}{5,11}{⽊、⾴}
  \begin{Phonetics}{本领}{ben3 ling3}[][HSK 3]
    \definition[项,个,种]{s.}{habilidade; capacidade; faculdade; poder; destreza; talento}
  \end{Phonetics}
\end{Entry}

\begin{Entry}{本期}{5,12}{⽊、⽉}
  \begin{Phonetics}{本期}{ben3 qi1}[][HSK 6]
    \definition{adv.}{o período atual | este prazo (geralmente em finanças)}
  \end{Phonetics}
\end{Entry}

\begin{Entry}{本意}{5,13}{⽊、⼼}
  \begin{Phonetics}{本意}{ben3yi4}[][HSK 7-9]
    \definition{s.}{ideia original; intenção real (original)}
  \end{Phonetics}
\end{Entry}

%%%%%%%%%% 术 %%%%%%%%%%
\subsection*{术}

\begin{Entry}{术}{5}{⽊}
  \begin{Phonetics}{术}{shu4}
    \definition*{s.}{Sobrenome: Shu}
    \definition{s.}{arte; habilidade; técnica; tecnologia; acadêmico | método; tática; estratégia}
  \end{Phonetics}
  \begin{Phonetics}{术}{zhu2}
    \definition{s.}{vários gêneros de flores da família Asteraceae (margaridas e crisântemos)}
  \end{Phonetics}
\end{Entry}

\begin{Entry}{术科}{5,9}{⽊、⽲}
  \begin{Phonetics}{术科}{shu4ke1}
    \definition{s.}{cursos técnicos oferecidos em treinamento militar ou físico (oposto a 学科)}
  \seealsoref{学科}{xue2 ke1}
  \end{Phonetics}
\end{Entry}

%%%%%%%%%% 正 %%%%%%%%%%
\subsection*{正}

\begin{Entry}{正}{5}{⽌}
  \begin{Phonetics}{正}{zheng1}
    \definition{s.}{o primeiro mês do ano lunar; a primeira lua}
  \end{Phonetics}
  \begin{Phonetics}{正}{zheng4}[][HSK 1,3]
    \definition*{s.}{Sobrenome Zheng}
    \definition{adj.}{reto; ereto; vertical | principal; posicionado no meio | direito; anverso | honesto; íntegro; justo | puro; sem mistura (de cor ou sabor) | regular; padronizado; de acordo com a lei; correto | chefe; comandante; diretor | regular; as laterais e os ângulos do gráfico têm comprimentos e tamanhos iguais | positivo; (matemática), significa maior que zero; (física) significa perda de elétrons (oposto de 负) | exato; preciso; usado para indicar tempo, refere-se ao momento exato ou ao ponto médio de um período}
    \definition{adv.}{apenas; certo; exatamente; precisamente | agora mesmo; neste momento; indica a continuidade de uma ação ou a permanência de um estado}
    \definition{v.}{definir (colocar) corretamente; alinhar; endireitar | ajustar; corrigir; retificar}
  \seealsoref{负}{fu4}
  \end{Phonetics}
\end{Entry}

\begin{Entry}{正义}{5,3}{⽌、⼂}
  \begin{Phonetics}{正义}{zheng4yi4}[][HSK 5]
    \definition{adj.}{justo; íntegro}
    \definition{s.}{justiça; o que é certo; o que é benéfico para o povo | (frequentemente em títulos de livros) interpretação ortodoxa ou retificada (de textos antigos)}
  \end{Phonetics}
\end{Entry}

\begin{Entry}{正正}{5,5}{⽌、⽌}
  \begin{Phonetics}{正正}{zheng4zheng4}
    \definition{adv.}{na hora certa | ordenadamente}
  \end{Phonetics}
\end{Entry}

\begin{Entry}{正在}{5,6}{⽌、⼟}
  \begin{Phonetics}{正在}{zheng4zai4}[][HSK 1]
    \definition{adv.}{em processo de; em andamento; indica que uma ação está em andamento ou que uma situação está em curso.}
    \definition{v.}{estar a + {v.inf.} | estar + {v.ger.}}
  \end{Phonetics}
\end{Entry}

\begin{Entry}{正好}{5,6}{⽌、⼥}
  \begin{Phonetics}{正好}{zheng4hao3}[][HSK 2]
    \definition{adj.}{na hora certa; na hora certa; o suficiente}
    \definition{adv.}{acontecer com; chance de; como acontece}
  \end{Phonetics}
\end{Entry}

\begin{Entry}{正如}{5,6}{⽌、⼥}
  \begin{Phonetics}{正如}{zheng4 ru2}[][HSK 5]
    \definition{adv.}{exatamente como; assim como}
  \end{Phonetics}
\end{Entry}

\begin{Entry}{正式}{5,6}{⽌、⼷}
  \begin{Phonetics}{正式}{zheng4shi4}[][HSK 3]
    \definition{adj.}{formal; oficial; descreve uma atmosfera séria, atitudes ou comportamentos que não são fáceis ou descontraídos | formal; oficial; descreve o cumprimento de determinados trâmites e procedimentos}
  \end{Phonetics}
\end{Entry}

\begin{Entry}{正当}{5,6}{⽌、⼹}
  \begin{Phonetics}{正当}{zheng4dang1}[][HSK 6]
    \definition{adj./adv.}{exatamente quando; exatamente o momento para}
  \end{Phonetics}
  \begin{Phonetics}{正当}{zheng4dang4}[][HSK 6]
    \definition{adv.}{exatamente quando; exatamente o momento para; está em (um certo período ou estágio)}
  \end{Phonetics}
\end{Entry}

\begin{Entry}{正宗}{5,8}{⽌、⼧}
  \begin{Phonetics}{正宗}{zheng4zong1}
    \definition{adj.}{autêntico | genuíno | \emph{old school} | (fig.) tradicional}
  \end{Phonetics}
\end{Entry}

\begin{Entry}{正版}{5,8}{⽌、⽚}
  \begin{Phonetics}{正版}{zheng4 ban3}[][HSK 5]
    \definition{s.}{versão genuína; versão autorizada; versão publicada e distribuída oficialmente por uma editora legal (em contraste com a 盗版)}
  \seealsoref{盗版}{dao4 ban3}
  \end{Phonetics}
\end{Entry}

\begin{Entry}{正规}{5,8}{⽌、⾒}
  \begin{Phonetics}{正规}{zheng4gui1}[][HSK 5]
    \definition{adj.}{normal; regular; padrão; está em conformidade com padrões formalmente definidos ou geralmente reconhecidos}
  \end{Phonetics}
\end{Entry}

\begin{Entry}{正是}{5,9}{⽌、⽇}
  \begin{Phonetics}{正是}{zheng4 shi4}[][HSK 2]
    \definition{v.}{ser precisamente; ser exatamente}
  \end{Phonetics}
\end{Entry}

\begin{Entry}{正面}{5,9}{⽌、⾯}
  \begin{Phonetics}{正面}{zheng4mian4}
    \definition{adj.}{bom; positivo (em oposição a 反面 ou 负面 | direto; cara a cara}
    \definition{s.}{frente; fachada; a frente do corpo; o lado de um edifício voltado para uma praça ou rua, que é decorado de forma mais elegante; a direção da viagem (em oposição a 背面, 反面 ou 侧面 | lado anverso (ou direito); o lado de uma folha que é usado principalmente ou está em contato com o mundo externo (em oposição a 背面 ou 反面 | superfície; o lado diretamente exibido das coisas, problemas, etc.}
  \seealsoref{背面}{bei4mian4}
  \seealsoref{侧面}{ce4mian4}
  \seealsoref{反面}{fan3mian4}
  \seealsoref{负面}{fu4mian4}
  \end{Phonetics}
\end{Entry}

\begin{Entry}{正常}{5,11}{⽌、⼱}
  \begin{Phonetics}{正常}{zheng4chang2}[][HSK 2]
    \definition{adj.}{normal; regular; conforma-se com regras ou circunstâncias gerais}
  \end{Phonetics}
\end{Entry}

\begin{Entry}{正确}{5,12}{⽌、⽯}
  \begin{Phonetics}{正确}{zheng4que4}[][HSK 2]
    \definition{adj.}{correto; certo; próprio; conforma-se com fatos, razão ou algum padrão geralmente aceito}
  \end{Phonetics}
\end{Entry}

%%%%%%%%%% 母 %%%%%%%%%%
\subsection*{母}

\begin{Entry}{母}{5}{⽏}[Kangxi 80]
  \begin{Phonetics}{母}{mu3}[][HSK 6]
    \definition*{s.}{Sobrenome: Mu}
    \definition{adj.}{fêmea}
    \definition[位,名,个,些]{s.}{mãe | fêmea (animal) (oposto a 公) | origem; pais | parentes idosas; geralmente se refere a mulheres idosas | côncavo | fonte; algo que tem a capacidade ou função de produzir outras coisas}
  \seealsoref{公}{gong1}
  \end{Phonetics}
\end{Entry}

\begin{Entry}{母女}{5,3}{⽏、⼥}
  \begin{Phonetics}{母女}{mu3 nv3}[][HSK 6]
    \definition{s.}{mãe e filha}
  \end{Phonetics}
\end{Entry}

\begin{Entry}{母子}{5,3}{⽏、⼦}
  \begin{Phonetics}{母子}{mu3 zi3}[][HSK 6]
    \definition{s.}{mãe e filho}
  \end{Phonetics}
\end{Entry}

\begin{Entry}{母鸡}{5,7}{⽏、⿃}
  \begin{Phonetics}{母鸡}{mu3ji1}[][HSK 6]
    \definition{s.}{galinha}
  \end{Phonetics}
\end{Entry}

\begin{Entry}{母亲}{5,9}{⽏、⼇}
  \begin{Phonetics}{母亲}{mu3qin1}[][HSK 3]
    \definition[位,名,个,些]{s.}{mãe}
  \end{Phonetics}
\end{Entry}

\begin{Entry}{母语}{5,9}{⽏、⾔}
  \begin{Phonetics}{母语}{mu3yu3}
    \definition{s.}{língua materna | língua nativa}
  \end{Phonetics}
\end{Entry}

%%%%%%%%%% 民 %%%%%%%%%%
\subsection*{民}

\begin{Entry}{民}{5}{⽒}
  \begin{Phonetics}{民}{min2}
    \definition*{s.}{Sobrenome: Min}
    \definition{adj.}{folclórico ; civil (não militar)}
    \definition{s.}{pessoa | membro de um grupo étnico | uma pessoa de uma determinada ocupação | do povo; folclore | civil; cidadão | o povo | um membro de uma nacionalidade}
  \end{Phonetics}
\end{Entry}

\begin{Entry}{民工}{5,3}{⽒、⼯}
  \begin{Phonetics}{民工}{min2 gong1}[][HSK 6]
    \definition{s.}{trabalhador trabalhando em um projeto público | trabalhador temporário alistado em um projeto público | agricultor que trabalha em empregos temporários na cidade | trabalhador migrante}
  \end{Phonetics}
\end{Entry}

\begin{Entry}{民主}{5,5}{⽒、⼂}
  \begin{Phonetics}{民主}{min2zhu3}[][HSK 6]
    \definition{adj.}{democrático; em consonância com os princípios democráticos}
    \definition[个]{s.}{democracia; direitos democráticos; refere-se ao direito do povo de participar da vida política e dos assuntos do Estado e de expressar livremente suas opiniões}
  \end{Phonetics}
\end{Entry}

\begin{Entry}{民众}{5,6}{⽒、⼈}
  \begin{Phonetics}{民众}{min2zhong4}
    \definition{s.}{a população | as massas | as pessoas comuns}
  \end{Phonetics}
\end{Entry}

\begin{Entry}{民间}{5,7}{⽒、⾨}
  \begin{Phonetics}{民间}{min2jian1}[][HSK 3]
    \definition{s.}{entre o povo | não governamental; de pessoa para pessoa}
  \end{Phonetics}
\end{Entry}

\begin{Entry}{民族}{5,11}{⽒、⽅}
  \begin{Phonetics}{民族}{min2zu2}[][HSK 3]
    \definition[个]{s.}{nação; uma comunidade estável formada ao longo da história pela humanidade, com uma língua comum, uma região comum, uma vida econômica comum e uma mentalidade comum expressa em uma cultura comum | grupo étnico; refere-se, de maneira geral, às comunidades formadas ao longo da história por pessoas em diferentes estágios de desenvolvimento social}
  \end{Phonetics}
\end{Entry}

\begin{Entry}{民意}{5,13}{⽒、⼼}
  \begin{Phonetics}{民意}{min2 yi4}[][HSK 6]
    \definition{s.}{vontade do povo; vontade popular | opinião pública}
  \end{Phonetics}
\end{Entry}

\begin{Entry}{民歌}{5,14}{⽒、⽋}
  \begin{Phonetics}{民歌}{min2 ge1}[][HSK 6]
    \definition[支,首]{s.}{canção folclórica; os nomes dos autores das canções transmitidas oralmente são muitas vezes desconhecidos}
  \end{Phonetics}
\end{Entry}

\begin{Entry}{民警}{5,19}{⽒、⾔}
  \begin{Phonetics}{民警}{min2 jing3}[][HSK 6]
    \definition{s.}{polícia; policial}
  \end{Phonetics}
\end{Entry}

%%%%%%%%%% 永 %%%%%%%%%%
\subsection*{永}

\begin{Entry}{永}{5}{⽔}
  \begin{Phonetics}{永}{yong3}
    \definition*{s.}{Sobrenome Yong}
    \definition{adj.}{sempre; para sempre; perpetuamente}
    \definition{adv.}{para sempre; significa um tempo muito longo sem fim, o que equivale a 永远}
  \seealsoref{永远}{yong3yuan3}
  \end{Phonetics}
\end{Entry}

\begin{Entry}{永不}{5,4}{⽔、⼀}
  \begin{Phonetics}{永不}{yong3bu4}
    \definition{adv.}{nunca}
  \end{Phonetics}
\end{Entry}

\begin{Entry}{永远}{5,7}{⽔、⾡}
  \begin{Phonetics}{永远}{yong3yuan3}[][HSK 2]
    \definition{adv.}{sempre; para sempre; Indica um longo período de tempo sem fim}
    \definition{s.}{eternidade; um futuro que nunca acaba}
  \end{Phonetics}
\end{Entry}

%%%%%%%%%% 汇 %%%%%%%%%%
\subsection*{汇}

\begin{Entry}{汇}{5}{⽔}
  \begin{Phonetics}{汇}{hui4}[][HSK 4]
    \definition{s.}{montagem; coleção; coisas coletadas}
    \definition{v.}{convergir | reunir; coletar | remeter | trocar (câmbio de moedas)}
  \end{Phonetics}
\end{Entry}

\begin{Entry}{汇合}{5,6}{⽔、⼝}
  \begin{Phonetics}{汇合}{hui4he2}[][HSK 7-9]
    \definition{s.}{confluência; fusão}
    \definition{v.}{convergir; juntar; reunir; encontrar}
  \end{Phonetics}
\end{Entry}

\begin{Entry}{汇报}{5,7}{⽔、⼿}
  \begin{Phonetics}{汇报}{hui4bao4}[][HSK 4]
    \definition[份,次]{s.}{relatório; referindo-se ao conteúdo de declarações escritas ou orais feitas a um superior ou pessoa relevante para apresentar uma situação ou refletir um problema}
    \definition{v.}{relatar; fazer um relato de}
  \end{Phonetics}
\end{Entry}

\begin{Entry}{汇率}{5,11}{⽔、⽞}
  \begin{Phonetics}{汇率}{hui4lv4}[][HSK 4]
    \definition[个,种]{s.}{taxa de câmbio; relação entre a moeda de um país e a de outro}
  \end{Phonetics}
\end{Entry}

\begin{Entry}{汇款}{5,12}{⽔、⽋}
  \begin{Phonetics}{汇款}{hui4/kuan3}[][HSK 5]
    \definition[笔,个]{s.}{remessa; dinheiro enviado ou recebido}
    \definition{v.+compl.}{remeter dinheiro; fazer uma remessa; enviar dinheiro}
  \end{Phonetics}
\end{Entry}

\begin{Entry}{汇集}{5,12}{⽔、⾫}
  \begin{Phonetics}{汇集}{hui4ji2}[][HSK 7-9]
    \definition{v.}{aduzir; coletar; compilar | reunir-se; congestionar; convergir; reunir; juntar}
  \end{Phonetics}
\end{Entry}

\begin{Entry}{汇聚}{5,14}{⽔、⽿}
  \begin{Phonetics}{汇聚}{hui4ju4}[][HSK 7-9]
    \definition{v.}{juntar; montar; reunir; ajuntar; reunir-se; convergência e acumulação (usado principalmente para objetos)}
  \end{Phonetics}
\end{Entry}

%%%%%%%%%% 汉 %%%%%%%%%%
\subsection*{汉}

\begin{Entry}{汉}{5}{⽔}
  \begin{Phonetics}{汉}{han4}
    \definition*{s.}{Dinastia Han (206 a.C.-220 d.C.)  | Astronomia: A Via Láctea | Sobrenome: Han}
    \definition{s.}{grupo étnico Han | chinês (língua) | homem}
  \end{Phonetics}
\end{Entry}

\begin{Entry}{汉字}{5,6}{⽔、⼦}
  \begin{Phonetics}{汉字}{han4 zi4}[][HSK 1]
    \definition[个]{s.}{caractere chinês; ideograma chinês; sinograma; com pouquíssimas exceções, os caracteres chineses representam uma sílaba cada um}
  \end{Phonetics}
\end{Entry}

\begin{Entry}{汉服}{5,8}{⽔、⽉}
  \begin{Phonetics}{汉服}{han4fu2}
    \definition{s.}{vestido chinês tradicional Han}
  \end{Phonetics}
\end{Entry}

\begin{Entry}{汉语}{5,9}{⽔、⾔}
  \begin{Phonetics}{汉语}{han4yu3}[][HSK 1]
    \definition[门]{s.}{língua chinesa, mandarim}
  \end{Phonetics}
\end{Entry}

\begin{Entry}{汉堡王}{5,12,4}{⽔、⼟、⽟}
  \begin{Phonetics}{汉堡王}{han4bao3wang2}
    \definition*{s.}{Burguer King, restaurante de \emph{fast-food}}
  \end{Phonetics}
\end{Entry}

\begin{Entry}{汉堡包}{5,12,5}{⽔、⼟、⼓}
  \begin{Phonetics}{汉堡包}{han4bao3bao1}
    \definition[个]{s.}{hambúrguer}
  \end{Phonetics}
\end{Entry}

\begin{Entry}{汉葡词典}{5,12,7,8}{⽔、⾋、⾔、⼋}
  \begin{Phonetics}{汉葡词典}{han4-pu2 ci2dian3}
    \definition[部,本]{s.}{dicionário chinês-português}
  \seealsoref{葡汉词典}{pu2-han4 ci2dian3}
  \end{Phonetics}
\end{Entry}

%%%%%%%%%% 灭 %%%%%%%%%%
\subsection*{灭}

\begin{Entry}{灭}{5}{⽕}
  \begin{Phonetics}{灭}{mie4}[][HSK 6]
    \definition{v.}{extinguir-se | extinguir; apagar; desligar | afogar; inundar; submergir | perecer; destruir | exterminar; apagar; acabar com; tornar inexistente}
  \end{Phonetics}
\end{Entry}

\begin{Entry}{灭火}{5,4}{⽕、⽕}
  \begin{Phonetics}{灭火}{mie4huo3}
    \definition{s.}{combate a incêndios}
    \definition{v.}{extinguir um incêndio}
  \end{Phonetics}
\end{Entry}

%%%%%%%%%% 犯 %%%%%%%%%%
\subsection*{犯}

\begin{Entry}{犯}{5}{⽝}
  \begin{Phonetics}{犯}{fan4}[][HSK 6]
    \definition{s.}{criminoso}
    \definition{v.}{ofender; violar; ir contra | atacar; violar; trabalhar contra | fazer; ocorrer | voltar a; ter uma recorrência de; recair; retornar a (velhos hábitos)}
  \end{Phonetics}
\end{Entry}

\begin{Entry}{犯法}{5,8}{⽝、⽔}
  \begin{Phonetics}{犯法}{fan4fa3}
    \definition{v.}{violar (quebrar) a lei}
  \end{Phonetics}
\end{Entry}

\begin{Entry}{犯规}{5,8}{⽝、⾒}
  \begin{Phonetics}{犯规}{fan4 gui1}[][HSK 6]
    \definition{v.}{quebrar as regras; violar regras | Esporte: cometer uma falta contra}
  \end{Phonetics}
\end{Entry}

\begin{Entry}{犯愁}{5,13}{⽝、⼼}
  \begin{Phonetics}{犯愁}{fan4/chou2}[][HSK 7-9]
    \definition{v.+compl.}{preocupar-se; estar ansioso}
  \end{Phonetics}
\end{Entry}

\begin{Entry}{犯罪}{5,13}{⽝、⽹}
  \begin{Phonetics}{犯罪}{fan4/zui4}[][HSK 6]
    \definition{v.+compl.}{cometer  um crime}
  \end{Phonetics}
\end{Entry}

%%%%%%%%%% 玄 %%%%%%%%%%
\subsection*{玄}

\begin{Entry}{玄}{5}{⽞}[Kangxi 95]
  \begin{Phonetics}{玄}{xuan2}
    \definition*{s.}{Sobrenome Xuan}
    \definition{adj.}{preto; escuro | profundo; abstruso; escondido | não confiável; irrealista; não confiável}
  \end{Phonetics}
\end{Entry}

\begin{Entry}{玄学}{5,8}{⽞、⼦}
  \begin{Phonetics}{玄学}{xuan2xue2}
    \definition{s.}{Escola Philosófica Wei e Jin amalgamando os ideais daoísta e confucionistas | tradução da metafísica (形而上学) | Obsoleto: metafísica}
  \seealsoref{形而上学}{xing2'er2shang4xue2}
  \end{Phonetics}
\end{Entry}

%%%%%%%%%% 玉 %%%%%%%%%%
\subsection*{玉}

\begin{Entry}{玉}{5}{⽟}[Kangxi 96]
  \begin{Phonetics}{玉}{yu4}[][HSK 4]
    \definition*{s.}{Sobrenome Yu}
    \definition{adj.}{(pessoa, especialmente uma mulher) pura; justa; bonita; bela | cristalino, branco e belo como o jade | (vida) rica; luxuosa}
    \definition{pron.}{seu; um termo de respeito, usado para honrar o corpo, as ações ou as coisas associadas à outra pessoa}
    \definition[块,种]{s.}{jade}
  \end{Phonetics}
\end{Entry}

\begin{Entry}{玉米}{5,6}{⽟、⽶}
  \begin{Phonetics}{玉米}{yu4mi3}[][HSK 4]
    \definition[根,粒,棵,片]{s.}{milho}
  \end{Phonetics}
\end{Entry}

\begin{Entry}{玉米片}{5,6,4}{⽟、⽶、⽚}
  \begin{Phonetics}{玉米片}{yu4mi3pian4}
    \definition{s.}{flocos de milho | chips de tortilha}
  \end{Phonetics}
\end{Entry}

\begin{Entry}{玉米花}{5,6,7}{⽟、⽶、⾋}
  \begin{Phonetics}{玉米花}{yu4mi3hua1}
    \definition{s.}{pipoca}
  \end{Phonetics}
\end{Entry}

\begin{Entry}{玉米面}{5,6,9}{⽟、⽶、⾯}
  \begin{Phonetics}{玉米面}{yu4mi3mian4}
    \definition{s.}{fubá | farinha de milho}
  \end{Phonetics}
\end{Entry}

\begin{Entry}{玉米饼}{5,6,9}{⽟、⽶、⾷}
  \begin{Phonetics}{玉米饼}{yu4mi3bing3}
    \definition{s.}{tortilha mexicana | bolo de milho}
  \end{Phonetics}
\end{Entry}

\begin{Entry}{玉米笋}{5,6,10}{⽟、⽶、⽵}
  \begin{Phonetics}{玉米笋}{yu4mi3 sun3}
    \definition{s.}{broto de milho}
  \end{Phonetics}
\end{Entry}

\begin{Entry}{玉米粉}{5,6,10}{⽟、⽶、⽶}
  \begin{Phonetics}{玉米粉}{yu4mi3fen3}
    \definition{s.}{amido de milho | farinha de milho}
  \end{Phonetics}
\end{Entry}

\begin{Entry}{玉米糁}{5,6,14}{⽟、⽶、⽶}
  \begin{Phonetics}{玉米糁}{yu4mi3 san3}
    \definition{s.}{grãos de milho}
  \end{Phonetics}
\end{Entry}

\begin{Entry}{玉米糕}{5,6,16}{⽟、⽶、⽶}
  \begin{Phonetics}{玉米糕}{yu4mi3gao1}
    \definition{s.}{bolo de milho | polenta}
  \end{Phonetics}
\end{Entry}

\begin{Entry}{玉帛}{5,8}{⽟、⼱}
  \begin{Phonetics}{玉帛}{yu4bo2}
    \definition{s.}{objetos de jade e tecidos de seda, usados ​​como presentes de estado (oposto a 干戈); paz e harmonia}[化干戈为玉帛。===Transforme guerra em paz.]
  \seealsoref{干戈}{gan1ge1}
  \end{Phonetics}
\end{Entry}

%%%%%%%%%% 瓜 %%%%%%%%%%
\subsection*{瓜}

\begin{Entry}{瓜}{5}{⽠}[Kangxi 97]
  \begin{Phonetics}{瓜}{gua1}[][HSK 4]
    \definition*{s.}{Sobrenome: Gua}
    \definition[个]{s.}{qualquer tipo de melão ou cabaça | companheiro (termo depreciativo para uma pessoa)}
    \definition{v.}{fofocar}
  \end{Phonetics}
\end{Entry}

\begin{Entry}{瓜子}{5,3}{⽠、⼦}
  \begin{Phonetics}{瓜子}{gua1zi3}[][HSK 7-9]
    \definition[个,把,颗,粒,些]{s.}{sementes de melão; sementes de girassol; sementes de abóbora}
  \end{Phonetics}
\end{Entry}

\begin{Entry}{瓜分}{5,4}{⽠、⼑}
  \begin{Phonetics}{瓜分}{gua1fen1}[][HSK 7-9]
    \definition{v.}{cortar um melão -- cortar; desmembrar; dividir; particionar}
  \end{Phonetics}
\end{Entry}

%%%%%%%%%% 甘 %%%%%%%%%%
\subsection*{甘}

\begin{Entry}{甘}{5}{⽢}[Kangxi 99]
  \begin{Phonetics}{甘}{gan1}
    \definition*{s.}{Província de Gansu, abreviação de 甘肃 | Sobrenome: Gan}
    \definition{adj.}{doce; agradável; satisfatório}
    \definition{v.}{estar disposto a; estar contente ou satisfeito com}
  \seealsoref{甘肃}{gan1su4}
  \end{Phonetics}
\end{Entry}

\begin{Entry}{甘心}{5,4}{⽢、⼼}
  \begin{Phonetics}{甘心}{gan1xin1}[][HSK 7-9]
    \definition{v.}{estar contente com; estar disposto a | reconciliar-se com; resignar-se com; contentar-se com}
  \end{Phonetics}
\end{Entry}

\begin{Entry}{甘肃}{5,8}{⽢、⾀}
  \begin{Phonetics}{甘肃}{gan1su4}
    \definition*{s.}{Província de Gansu}
  \end{Phonetics}
\end{Entry}

\begin{Entry}{甘薯}{5,16}{⽢、⾋}
  \begin{Phonetics}{甘薯}{gan1shu3}
    \definition{s.}{batata doce}
  \end{Phonetics}
\end{Entry}

%%%%%%%%%% 生 %%%%%%%%%%
\subsection*{生}

\begin{Entry}{生}{5}{⽣}[Kangxi 100]
  \begin{Phonetics}{生}{sheng1}[][HSK 2,3]
    \definition*{s.}{Sobrenome: Sheng}
    \definition{adj.}{vivo; vital | verde; não maduro | cru; não cozido; mal cozido | bruto; não refinado; não processado | estranho; desconhecido; não familiarizado | rígido; mecânico; forçado}
    \definition{adv.}{muito; usado antes de certas palavras que expressam emoções e sentimentos | verdadeiramente; realmente; forçosamente}
    \definition{s.}{vida | meio de subsistência | aluno; estudante | estudioso; antigamente chamados de eruditos | o tipo de personagem masculino na ópera de Pequim, etc.}
    \definition{suf.}{certos sufixos substantivos que se referem a pessoas (学生) | sufixos de certos advérbios (好生)}
    \definition{v.}{dar à luz; ter um filho | nascer | crescer; cultivar | viver; existir; sobreviver | favorecer; gerar; ocorrer | acender (uma fogueira); fazer o combustível queimar}
  \seealsoref{好生}{hao3sheng1}
  \seealsoref{学生}{xue2sheng5}
  \end{Phonetics}
\end{Entry}

\begin{Entry}{生日}{5,4}{⽣、⽇}
  \begin{Phonetics}{生日}{sheng1ri4}[][HSK 1]
    \definition[个,次]{s.}{aniversário; dia de nascimento, também se refere ao dia em que se completa um ano de idade a cada ano}
  \end{Phonetics}
\end{Entry}

\begin{Entry}{生气}{5,4}{⽣、⽓}
  \begin{Phonetics}{生气}{sheng1/qi4}[][HSK 1]
    \definition{s.}{vitalidade; vigor; energia da vida}
    \definition{v.+compl.}{ficar com raiva; ficar ofendido; ficar zangado; encontrar algo que não é do seu agrado e sentir-se descontente}
  \end{Phonetics}
\end{Entry}

\begin{Entry}{生长}{5,4}{⽣、⾧}
  \begin{Phonetics}{生长}{sheng1zhang3}[][HSK 3]
    \definition{v.}{cresçer; sob certas condições de vida, o volume e o peso dos organismos aumentam gradualmente | nascer e crescer}
  \end{Phonetics}
\end{Entry}

\begin{Entry}{生产}{5,6}{⽣、⼇}
  \begin{Phonetics}{生产}{sheng1chan3}[][HSK 3]
    \definition{v.}{produzir; fabricar; utilizar ferramentas para mudar o objeto de trabalho e criar meios de produção e meios de subsistência | dar à luz uma criança; ter filhos}
  \end{Phonetics}
\end{Entry}

\begin{Entry}{生动}{5,6}{⽣、⼒}
  \begin{Phonetics}{生动}{sheng1dong4}[][HSK 3]
    \definition{adj.}{vívido; animado; descreve a linguagem e as formas de expressão como sendo ativas e em movimento}
  \end{Phonetics}
\end{Entry}

\begin{Entry}{生存}{5,6}{⽣、⼦}
  \begin{Phonetics}{生存}{sheng1cun2}[][HSK 3]
    \definition{v.}{viver; sobreviver; subsistir; manter a vida; estar vivo}
  \end{Phonetics}
\end{Entry}

\begin{Entry}{生成}{5,6}{⽣、⼽}
  \begin{Phonetics}{生成}{sheng1 cheng2}[][HSK 5]
    \definition{v.}{formar; gerar; produzir | ter por natureza; nascer com}
  \end{Phonetics}
\end{Entry}

\begin{Entry}{生词}{5,7}{⽣、⾔}
  \begin{Phonetics}{生词}{sheng1 ci2}[][HSK 2]
    \definition[个,组,堆,条]{s.}{nova palavra; palavras que não aprendi, não conheço ou não entendo}
  \end{Phonetics}
\end{Entry}

\begin{Entry}{生命}{5,8}{⽣、⼝}
  \begin{Phonetics}{生命}{sheng1ming4}[][HSK 3]
    \definition{s.}{vida; não envolve apenas a existência e as atividades dos organismos, mas também inclui experiências de vida humana, valores e elementos-chave da sobrevivência e do desenvolvimento de várias coisas}
  \end{Phonetics}
\end{Entry}

\begin{Entry}{生态}{5,8}{⽣、⼼}
  \begin{Phonetics}{生态}{sheng1tai4}
    \definition{adj.}{ecológico}
    \definition{s.}{ecologia}
  \end{Phonetics}
\end{Entry}

\begin{Entry}{生物}{5,8}{⽣、⽜}
  \begin{Phonetics}{生物}{sheng1wu4}
    \definition{adj.}{biológico}
    \definition{s.}{biologia (disciplina) | organismo | ser vivo}
  \end{Phonetics}
\end{Entry}

\begin{Entry}{生的}{5,8}{⽣、⽩}
  \begin{Phonetics}{生的}{sheng1de5}
    \definition{conj.}{para evitar isso | para que\dots não\dots}
  \end{Phonetics}
\end{Entry}

\begin{Entry}{生鱼片}{5,8,4}{⽣、⿂、⽚}
  \begin{Phonetics}{生鱼片}{sheng1yu2pian4}
    \definition{s.}{fatias de peixe cru, \emph{sashimi}}
  \end{Phonetics}
\end{Entry}

\begin{Entry}{生活}{5,9}{⽣、⽔}
  \begin{Phonetics}{生活}{sheng1huo2}[][HSK 2]
    \definition[个,段,种]{s.}{vida; subsistência; as diversas atividades realizadas por pessoas ou seres vivos para sobreviver e se desenvolver | estilo de vida; condições de vida; situação em termos de vestuário, alimentação, habitação e transporte | trabalho (principalmente nas áreas industrial, agrícola e artesanal)}
    \definition{v.}{viver; realizar várias atividades | sobreviver}
  \end{Phonetics}
\end{Entry}

\begin{Entry}{生活垃圾}{5,9,8,6}{⽣、⽔、⼟、⼟}
  \begin{Phonetics}{生活垃圾}{sheng1huo2la1ji1}
    \definition{s.}{lixo doméstico}
  \end{Phonetics}
\end{Entry}

\begin{Entry}{生活型}{5,9,9}{⽣、⽔、⼟}
  \begin{Phonetics}{生活型}{sheng1huo2 xing2}
    \definition{s.}{forma de vida}
  \end{Phonetics}
\end{Entry}

\begin{Entry}{生活费}{5,9,9}{⽣、⽔、⾙}
  \begin{Phonetics}{生活费}{sheng1 huo2 fei4}[][HSK 6]
    \definition{s.}{subsídio; despesas de subsistência; despesas necessárias para manter a vida diária}
  \end{Phonetics}
\end{Entry}

\begin{Entry}{生病}{5,10}{⽣、⽧}
  \begin{Phonetics}{生病}{sheng1bing4}[][HSK 1]
    \definition{v.}{adoecer; ficar doente; ficar mal; contrair uma doença}
  \end{Phonetics}
\end{Entry}

\begin{Entry}{生理}{5,11}{⽣、⽟}
  \begin{Phonetics}{生理}{sheng1li3}
    \definition{adj.}{fisiológico}
    \definition{s.}{fisiologia}
  \end{Phonetics}
\end{Entry}

\begin{Entry}{生菜}{5,11}{⽣、⾋}
  \begin{Phonetics}{生菜}{sheng1cai4}
    \definition{s.}{alface}
  \end{Phonetics}
\end{Entry}

\begin{Entry}{生意}{5,13}{⽣、⼼}
  \begin{Phonetics}{生意}{sheng1yi4}
    \definition[笔,种,次]{s.}{tendência a crescer; vitalidade; vigor; energia}
  \end{Phonetics}
  \begin{Phonetics}{生意}{sheng1yi5}[][HSK 3]
    \definition[笔,种,次]{s.}{comércio, compra e venda; negócios; indústria; colegas do mesmo setor}
  \end{Phonetics}
\end{Entry}

%%%%%%%%%% 用 %%%%%%%%%%
\subsection*{用}

\begin{Entry}{用}{5}{⽤}[Kangxi 101]
  \begin{Phonetics}{用}{yong4}[][HSK 1]
    \definition*{s.}{Sobrenome Yong}
    \definition{conj.}{portanto; por isso; assim sendo; razões para a introdução, equivalentes a 因}
    \definition{prep.}{com; ação de introduzir ferramentas, meios, etc. utilizados ou empregados}
    \definition{s.}{despesas; gastos; custos | uso; utilidade; eficácia}
    \definition{v.}{usar; aplicar; empregar | necessitar (normalmente na forma negativa) | respeitosamente: comer; beber}
  \seealsoref{因}{yin1}
  \end{Phonetics}
\end{Entry}

\begin{Entry}{用于}{5,3}{⽤、⼆}
  \begin{Phonetics}{用于}{yong4 yu2}[][HSK 5]
    \definition{v.}{usar para; ser usado para; usar em}
  \end{Phonetics}
\end{Entry}

\begin{Entry}{用不着}{5,4,11}{⽤、⼀、⽬}
  \begin{Phonetics}{用不着}{yong4 bu4 zhao2}[][HSK 5]
    \definition{v.}{não precisar; não ter utilidade para; não haver necessidade de}
  \end{Phonetics}
\end{Entry}

\begin{Entry}{用心}{5,4}{⽤、⼼}
  \begin{Phonetics}{用心}{yong4 xin1}[][HSK 6]
    \definition{adj.}{diligente; atento; com atenção concentrada}
    \definition{s.}{motivo; intenção; o verdadeiro propósito ou razão para fazer algo}
  \end{Phonetics}
\end{Entry}

\begin{Entry}{用户}{5,4}{⽤、⼾}
  \begin{Phonetics}{用户}{yong4hu4}[][HSK 5]
    \definition[个,位,名]{s.}{usuário; consumidor; entidades e indivíduos que utilizam determinados equipamentos públicos ou bens de consumo}
  \end{Phonetics}
\end{Entry}

\begin{Entry}{用处}{5,5}{⽤、⼡}
  \begin{Phonetics}{用处}{yong4 chu3}[][HSK 6]
    \definition[个]{s.}{uso; usabilidade; utilidade}
  \end{Phonetics}
\end{Entry}

\begin{Entry}{用来}{5,7}{⽤、⽊}
  \begin{Phonetics}{用来}{yong4 lai2}[][HSK 5]
    \definition{v.}{ser usado para; depender (dele) ou usar (ele) para atingir algum objetivo}
  \end{Phonetics}
\end{Entry}

\begin{Entry}{用法}{5,8}{⽤、⽔}
  \begin{Phonetics}{用法}{yong4 fa3}[][HSK 6]
    \definition[种,个]{s.}{uso; emprego; a maneira de usar}
  \end{Phonetics}
\end{Entry}

\begin{Entry}{用品}{5,9}{⽤、⼝}
  \begin{Phonetics}{用品}{yong4 pin3}[][HSK 6]
    \definition[批,件,种]{s.}{suprimentos; artigos para uso; itens para usar}
  \end{Phonetics}
\end{Entry}

\begin{Entry}{用料}{5,10}{⽤、⽃}
  \begin{Phonetics}{用料}{yong4liao4}
    \definition{s.}{ingredientes | materiais}
  \end{Phonetics}
\end{Entry}

\begin{Entry}{用途}{5,10}{⽤、⾡}
  \begin{Phonetics}{用途}{yong4tu2}[][HSK 4]
    \definition[个,种]{s.}{uso; aplicação; aspectos ou escopo da aplicação}
  \end{Phonetics}
\end{Entry}

\begin{Entry}{用得着}{5,11,11}{⽤、⼻、⽬}
  \begin{Phonetics}{用得着}{yong4 de5 zhao2}[][HSK 6]
    \definition{adj.}{útil; necessário}
    \definition{v.}{precisar; achar algo útil | ter necessidade de;  ser necessário; valer a pena}
  \end{Phonetics}
\end{Entry}

%%%%%%%%%% 田 %%%%%%%%%%
\subsection*{田}

\begin{Entry}{田}{5}{⽥}[Kangxi 102]
  \begin{Phonetics}{田}{tian2}[][HSK 6]
    \definition*{s.}{Sobrenome: Tian}
    \definition[亩,块,片]{s.}{campo; terra; terra de cultivo | área aberta rica em algum produto natural; campo}
    \definition{v.}{(arcaico) caçar}
  \end{Phonetics}
\end{Entry}

\begin{Entry}{田园}{5,7}{⽥、⼞}
  \begin{Phonetics}{田园}{tian2yuan2}
    \definition{adj.}{bucólico}
    \definition{s.}{campo | interior | rural}
  \end{Phonetics}
\end{Entry}

\begin{Entry}{田径}{5,8}{⽥、⼻}
  \begin{Phonetics}{田径}{tian2jing4}[][HSK 6]
    \definition{s.}{Esporte: atletismo}[他参加了这次的田径赛。===Ele participou da competição de atletismo.]
  \end{Phonetics}
\end{Entry}

%%%%%%%%%% 由 %%%%%%%%%%
\subsection*{由}

\begin{Entry}{由}{5}{⽥}
  \begin{Phonetics}{由}{you2}[][HSK 3]
    \definition*{s.}{Sobrenome You}
    \definition{prep.}{por causa de; devido a | por; indica que algo deve ser feito por alguém | indica confiança em; indica dependência em | de; indica o ponto de partida | por; através de}
    \definition[个]{s.}{causa; razão; motivo}
    \definition{v.}{atravessar; passar por; seguir o caminho de | obedecer; seguir}
  \end{Phonetics}
\end{Entry}

\begin{Entry}{由于}{5,3}{⽥、⼆}
  \begin{Phonetics}{由于}{you2yu2}[][HSK 3]
    \definition{conj.}{porque; uma vez que; visto que;  usado no início da frase anterior, indica a razão, e a frase seguinte indica o resultado}
    \definition{prep.}{devido a; graças a; por causa de; em virtude de; como resultado de; introduzir a causa da ocorrência de eventos, ações, etc.}
  \end{Phonetics}
\end{Entry}

\begin{Entry}{由此}{5,6}{⽥、⽌}
  \begin{Phonetics}{由此}{you2 ci3}[][HSK 5]
    \definition{adv.}{assim; por meio disto; disto; daí; por causa disto; portanto; daqui; de agora em diante}
  \end{Phonetics}
\end{Entry}

%%%%%%%%%% 甲 %%%%%%%%%%
\subsection*{甲}

\begin{Entry}{甲}{5}{⽥}
  \begin{Phonetics}{甲}{jia3}[][HSK 5]
    \definition*{s.}{Sobrenome: Jia}
    \definition{s.}{alfa; primeiro lugar; o primeiro dos caules celestiais, geralmente usado para indicar o primeiro em ordem ou classificação | concha; carapaça; crustáceos | unha; crostas queratinosas nos dedos das mãos e dos pés | armadura; equipamento de proteção feito de metal | Obsoleto: unidade de administração civil composta por 10 residências | uma palavra substituta para uma pessoa ou coisa indefinida; usado como pronome}
    \definition{v.}{ocupar o primeiro lugar; ser melhor do que}
  \end{Phonetics}
\end{Entry}

\begin{Entry}{甲骨文}{5,9,4}{⽥、⾻、⽂}
  \begin{Phonetics}{甲骨文}{jia3gu3wen2}
    \definition{s.}{escrituras de oráculos | inscrições em ossos de oráculos (forma original de escritura chinesa)}
  \end{Phonetics}
\end{Entry}

%%%%%%%%%% 申 %%%%%%%%%%
\subsection*{申}

\begin{Entry}{申}{5}{⽥}
  \begin{Phonetics}{申}{shen1}
    \definition*{s.}{O nono dos doze Ramos Terrestres | Outro nome para Xangai, 上海 | Sobrenome: Shen}
    \definition{v.}{declarar; explicar; expressar}
  \seealsoref{上海}{shang4hai3}
  \end{Phonetics}
\end{Entry}

\begin{Entry}{申请}{5,10}{⽥、⾔}
  \begin{Phonetics}{申请}{shen1qing3}[][HSK 4]
    \definition[份,批,项]{s.}{a solicitação para; o requerimento para; um pedido para ser visto pelos superiores ou departamentos relevantes}
    \definition{v.}{solicitar; apresentar uma solicitação; apresentar os motivos e fazer o pedido aos superiores ou aos departamentos competentes}
  \end{Phonetics}
\end{Entry}

%%%%%%%%%% 电 %%%%%%%%%%
\subsection*{电}

\begin{Entry}{电}{5}{⽥}
  \begin{Phonetics}{电}{dian4}[][HSK 1]
    \definition*{s.}{Sobrenome: Dian}
    \definition{s.}{eletricidade; energia elétrica | telegrama | relâmpago}
    \definition{v.}{dar ou receber um choque elétrico | enviar telegrama, telefonar ou enviar fax}
  \end{Phonetics}
\end{Entry}

\begin{Entry}{电力}{5,2}{⽥、⼒}
  \begin{Phonetics}{电力}{dian4 li4}[][HSK 6]
    \definition{s.}{energia elétrica; fornecimento de energia elétrica | energia elétrica | eletricidade}
  \end{Phonetics}
\end{Entry}

\begin{Entry}{电子}{5,3}{⽥、⼦}
  \begin{Phonetics}{电子}{dian4zi3}
    \definition{s.}{eletrônico | elétron}
  \end{Phonetics}
\end{Entry}

\begin{Entry}{电子名片}{5,3,6,4}{⽥、⼦、⼝、⽚}
  \begin{Phonetics}{电子名片}{dian4zi3 ming2pian4}
    \definition{s.}{cartão de visita eletrônico}
  \end{Phonetics}
\end{Entry}

\begin{Entry}{电子邮件}{5,3,7,6}{⽥、⼦、⾢、⼈}
  \begin{Phonetics}{电子邮件}{dian4zi3you2jian4}[][HSK 3]
    \definition[封,份,个,条]{s.}{correio eletrônico; \emph{e-mail}}
  \seealsoref{电邮}{dian4you2}
  \end{Phonetics}
\end{Entry}

\begin{Entry}{电子版}{5,3,8}{⽥、⼦、⽚}
  \begin{Phonetics}{电子版}{dian4 zi3 ban3}[][HSK 5]
    \definition[个]{s.}{edição eletrônica}
  \end{Phonetics}
\end{Entry}

\begin{Entry}{电车}{5,4}{⽥、⾞}
  \begin{Phonetics}{电车}{dian4 che1}[][HSK 6]
    \definition[辆,班,趟,路]{s.}{bonde; veículos de transporte público urbano movidos por linhas aéreas e acionados por motores de tração}
  \end{Phonetics}
\end{Entry}

\begin{Entry}{电车司机}{5,4,5,6}{⽥、⾞、⼝、⽊}
  \begin{Phonetics}{电车司机}{dian4che1 si1ji1}
    \definition{s.}{motorista de bonde}
  \end{Phonetics}
\end{Entry}

\begin{Entry}{电台}{5,5}{⽥、⼝}
  \begin{Phonetics}{电台}{dian4 tai2}[][HSK 3]
    \definition[个,家]{s.}{transceptor; transmissor-receptor | aparelho de rádio; estação de rádio; estação de transmissão}
  \end{Phonetics}
\end{Entry}

\begin{Entry}{电讯}{5,5}{⽥、⾔}
  \begin{Phonetics}{电讯}{dian4xun4}[][HSK 7-9]
    \definition{s.}{despacho (telegráfico) | telecomunicações; uma mensagem enviada por telefone, telégrafo ou rádio (telegráfico) | sinais de comunicação de rádio; sinais de rádio}
  \end{Phonetics}
\end{Entry}

\begin{Entry}{电冰箱}{5,6,15}{⽥、⼎、⾋}
  \begin{Phonetics}{电冰箱}{dian4bing1xiang1}
    \definition[台]{s.}{frigorífico | refrigerador}
  \end{Phonetics}
\end{Entry}

\begin{Entry}{电动}{5,6}{⽥、⼒}
  \begin{Phonetics}{电动}{dian4 dong4}[][HSK 6]
    \definition{adj.}{motorizado; acionado por energia elétrica; operado por energia elétrica elétrico}
  \end{Phonetics}
\end{Entry}

\begin{Entry}{电动车}{5,6,4}{⽥、⼒、⾞}
  \begin{Phonetics}{电动车}{dian4 dong4 che1}[][HSK 4]
    \definition{s.}{veículo elétrico (\emph{scooter}, bicicleta, carro, etc.)}
  \end{Phonetics}
\end{Entry}

\begin{Entry}{电池}{5,6}{⽥、⽔}
  \begin{Phonetics}{电池}{dian4chi2}[][HSK 5]
    \definition[节,块,组,个]{s.}{célula; bateria}
  \end{Phonetics}
\end{Entry}

\begin{Entry}{电灯}{5,6}{⽥、⽕}
  \begin{Phonetics}{电灯}{dian4 deng1}[][HSK 4]
    \definition[盏,个]{s.}{luz elétrica; lâmpada elétrica; lâmpadas que usam eletricidade como fonte de energia}
  \end{Phonetics}
\end{Entry}

\begin{Entry}{电灯泡}{5,6,8}{⽥、⽕、⽔}
  \begin{Phonetics}{电灯泡}{dian4deng1pao4}
    \definition{s.}{lâmpada elétrica | (gíria) terceiro convidado indesejado}
  \end{Phonetics}
\end{Entry}

\begin{Entry}{电网}{5,6}{⽥、⽹}
  \begin{Phonetics}{电网}{dian4wang3}[][HSK 7-9]
    \definition[张,个]{s.}{rede de arame eletrificada; emaranhamento de fios energizados | rede elétrica (ou grade)}
  \end{Phonetics}
\end{Entry}

\begin{Entry}{电报}{5,7}{⽥、⼿}
  \begin{Phonetics}{电报}{dian4bao4}[][HSK 7-9]
    \definition[封,份,个]{s.}{telegrama; cabo; telégrafo; mensagem}
  \end{Phonetics}
\end{Entry}

\begin{Entry}{电邮}{5,7}{⽥、⾢}
  \begin{Phonetics}{电邮}{dian4you2}
    \definition{s.}{correio eletrônico, \emph{e-mail} | abreviação de~电子邮件}
  \seealsoref{电子邮件}{dian4zi3you2jian4}
  \end{Phonetics}
\end{Entry}

\begin{Entry}{电饭锅}{5,7,12}{⽥、⾷、⾦}
  \begin{Phonetics}{电饭锅}{dian4 fan4 guo1}[][HSK 5]
    \definition[台,个]{s.}{panela elétrica de arroz}
  \end{Phonetics}
\end{Entry}

\begin{Entry}{电线}{5,8}{⽥、⽷}
  \begin{Phonetics}{电线}{dian4xian4}[][HSK 7-9]
    \definition[根,个]{s.}{fio; fio elétrico; condutor de corrente; condutor que transmite energia elétrica}
  \end{Phonetics}
\end{Entry}

\begin{Entry}{电视}{5,8}{⽥、⾒}
  \begin{Phonetics}{电视}{dian4shi4}[][HSK 1]
    \definition[部,台,个]{s.}{televisão; TV; televisor}
  \end{Phonetics}
\end{Entry}

\begin{Entry}{电视台}{5,8,5}{⽥、⾒、⼝}
  \begin{Phonetics}{电视台}{dian4 shi4 tai2}[][HSK 3]
    \definition[家,座,个]{s.}{canal de TV; estação de televisão; locais e instituições que transmitem programas de televisão}
  \end{Phonetics}
\end{Entry}

\begin{Entry}{电视机}{5,8,6}{⽥、⾒、⽊}
  \begin{Phonetics}{电视机}{dian4 shi4 ji1}[][HSK 1]
    \definition[个,台]{s.}{aparelho de TV; receptor de televisão; receptor de imagem; televisor; aparelho de televisão}
  \end{Phonetics}
\end{Entry}

\begin{Entry}{电视剧}{5,8,10}{⽥、⾒、⼑}
  \begin{Phonetics}{电视剧}{dian4 shi4 ju4}[][HSK 3]
    \definition[部,集,个]{s.}{série de TV; drama de TV; novela; drama escrito e gravado para transmissão pela televisão}
  \end{Phonetics}
\end{Entry}

\begin{Entry}{电话}{5,8}{⽥、⾔}
  \begin{Phonetics}{电话}{dian4 hua4}[][HSK 1]
    \definition[部]{s.}{telefone; aparelho telefônico; telefonia}
    \definition[通]{s.}{chamada telefônica; telefonema}
  \end{Phonetics}
\end{Entry}

\begin{Entry}{电信}{5,9}{⽥、⼈}
  \begin{Phonetics}{电信}{dian4xin4}[][HSK 7-9]
    \definition{s.}{telecomunicações; métodos de transmissão de informações usando tecnologias de comunicação com fio, rádio e óptica, incluindo telégrafo, telefone, etc.}
  \end{Phonetics}
\end{Entry}

\begin{Entry}{电脑}{5,10}{⽥、⾁}
  \begin{Phonetics}{电脑}{dian4nao3}[][HSK 1]
    \definition[个,台]{s.}{computador eletrônico}
  \end{Phonetics}
\end{Entry}

\begin{Entry}{电脑语言}{5,10,9,7}{⽥、⾁、⾔、⾔}
  \begin{Phonetics}{电脑语言}{dian4nao3yu3yan2}
    \definition{s.}{linguagem de programação | linguagem de computador}
  \end{Phonetics}
\end{Entry}

\begin{Entry}{电铃}{5,10}{⽥、⾦}
  \begin{Phonetics}{电铃}{dian4ling2}[][HSK 7-9]
    \definition[个]{s.}{campainha elétrica}
  \end{Phonetics}
\end{Entry}

\begin{Entry}{电梯}{5,11}{⽥、⽊}
  \begin{Phonetics}{电梯}{dian4ti1}[][HSK 4]
    \definition[部,台,架]{s.}{elevador}
  \end{Phonetics}
\end{Entry}

\begin{Entry}{电梯司机}{5,11,5,6}{⽥、⽊、⼝、⽊}
  \begin{Phonetics}{电梯司机}{dian4ti1 si1ji1}
    \definition{s.}{ascensorista}
  \end{Phonetics}
\end{Entry}

\begin{Entry}{电源}{5,13}{⽥、⽔}
  \begin{Phonetics}{电源}{dian4yuan2}[][HSK 4]
    \definition[台,个,套]{s.}{fonte de alimentação; fonte de energia; fonte de energia elétrica; dispositivo que fornece energia elétrica a um aparelho, como uma bateria, um gerador, etc.}
  \end{Phonetics}
\end{Entry}

\begin{Entry}{电影}{5,15}{⽥、⼺}
  \begin{Phonetics}{电影}{dian4ying3}[][HSK 1]
    \definition[部,片,幕,场]{s.}{filme; longa-metragem; cinema}
  \end{Phonetics}
\end{Entry}

\begin{Entry}{电影艺术}{5,15,4,5}{⽥、⼺、⾋、⽊}
  \begin{Phonetics}{电影艺术}{dian4ying3 yi4shu4}
    \definition{s.}{arte cinematográfica}
  \end{Phonetics}
\end{Entry}

\begin{Entry}{电影术}{5,15,5}{⽥、⼺、⽊}
  \begin{Phonetics}{电影术}{dian4ying3 shu4}
    \definition{s.}{cinematografia}
  \end{Phonetics}
\end{Entry}

\begin{Entry}{电影节}{5,15,5}{⽥、⼺、⾋}
  \begin{Phonetics}{电影节}{dian4ying3jie2}
    \definition{s.}{festival de cinema}
  \end{Phonetics}
\end{Entry}

\begin{Entry}{电影奖}{5,15,9}{⽥、⼺、⼤}
  \begin{Phonetics}{电影奖}{dian4ying3jiang3}
    \definition{s.}{premiações de cinema}
  \end{Phonetics}
\end{Entry}

\begin{Entry}{电影界}{5,15,9}{⽥、⼺、⽥}
  \begin{Phonetics}{电影界}{dian4ying3jie4}
    \definition{s.}{indústria cinematográfica}
  \end{Phonetics}
\end{Entry}

\begin{Entry}{电影院}{5,15,9}{⽥、⼺、⾩}
  \begin{Phonetics}{电影院}{dian4 ying3 yuan4}[][HSK 1]
    \definition[家,座,个]{s.}{cinema; sala de cinema; teatro; salão de cinema; local comercial dedicado à exibição de filmes}
  \end{Phonetics}
\end{Entry}

\begin{Entry}{电影音乐}{5,15,9,5}{⽥、⼺、⾳、⼃}
  \begin{Phonetics}{电影音乐}{dian4ying3 yin1yue4}
    \definition{s.}{música cinematográfica}
  \end{Phonetics}
\end{Entry}

\begin{Entry}{电影票}{5,15,11}{⽥、⼺、⽰}
  \begin{Phonetics}{电影票}{dian4ying3piao4}
    \definition{s.}{ingresso de filme}
  \end{Phonetics}
\end{Entry}

\begin{Entry}{电器}{5,16}{⽥、⼝}
  \begin{Phonetics}{电器}{dian4 qi4}[][HSK 6]
    \definition[件,种]{s.}{dispositivo elétrico; cargas em circuitos e dispositivos usados ​​para controlar, regular ou proteger circuitos, motores, etc.; como alto-falantes; interruptores; resistores; fusíveis, etc. | eletrodomésticos ou aparelhos elétricos domésticos; refere-se a eletrodomésticos, como televisores, gravadores, geladeiras, máquinas de lavar, etc.}
  \end{Phonetics}
\end{Entry}

%%%%%%%%%% 白 %%%%%%%%%%
\subsection*{白}

\begin{Entry}{白}{5}{⽩}[Kangxi 106]
  \begin{Phonetics}{白}{bai2}[][HSK 1,3]
    \definition*{s.}{Sobrenome: Bai}
    \definition{adj.}{branco | claro; entendível; compreendível | puro; claro; simples; sem mistura; em branco | branco (como símbolo de reação) | escrito incorretamente ou pronunciado incorretamente}
    \definition{adv.}{em vão; sem propósito; sem resultados | gratuito; sem custos}
    \definition{s.}{parte falada em ópera, etc.; frases de peças de teatro, etc. | dialeto local | funeral}
    \definition{v.}{explicar; apresentar; esclarecer; declarar | branquear | olhar para as pessoas com o branco dos olhos (olhar vazio, de desaprovação); olhar para alguém com desdém}
  \end{Phonetics}
\end{Entry}

\begin{Entry}{白天}{5,4}{⽩、⼤}
  \begin{Phonetics}{白天}{bai2 tian1}[][HSK 1]
    \definition{adv.}{dia;  de dia}
    \definition[个]{s.}{dia; horário diurno; durante o dia}
  \end{Phonetics}
\end{Entry}

\begin{Entry}{白白}{5,5}{⽩、⽩}
  \begin{Phonetics}{白白}{bai2bai2}[][HSK 7-9]
    \definition{adv.}{em vão; sem propósito; sem nenhum efeito | sem custo; gratuito; nenhum preço a pagar; pago mas não recebido}
  \end{Phonetics}
\end{Entry}

\begin{Entry}{白色}{5,6}{⽩、⾊}
  \begin{Phonetics}{白色}{bai2 se4}[][HSK 2]
    \definition{s.}{a cor branca}
  \end{Phonetics}
\end{Entry}

\begin{Entry}{白苋}{5,7}{⽩、⾋}
  \begin{Phonetics}{白苋}{bai2xian4}
    \definition{s.}{amaranto branco | brotos e folhas tenras de espinafre chinês usados como alimento}
  \end{Phonetics}
\end{Entry}

\begin{Entry}{白拣}{5,8}{⽩、⼿}
  \begin{Phonetics}{白拣}{bai2jian3}
    \definition{s.}{uma escolha barata}
    \definition{v.}{escolher algo que não custa nada}
  \end{Phonetics}
\end{Entry}

\begin{Entry}{白酒}{5,10}{⽩、⾣}
  \begin{Phonetics}{白酒}{bai2 jiu3}[][HSK 5]
    \definition[瓶,杯,壶]{s.}{aguardente branca; aguardente (geralmente destilada de sorgo ou milho); bebidas destiladas tradicionais chinesas, feitas de sorgo, milho, etc., transparentes e incolores, com alto teor alcoólico}
  \end{Phonetics}
\end{Entry}

\begin{Entry}{白菜}{5,11}{⽩、⾋}
  \begin{Phonetics}{白菜}{bai2 cai4}[][HSK 3]
    \definition[棵,种]{s.}{couve chinesa | \emph{pak choi}, um tipo de couve}
  \end{Phonetics}
\end{Entry}

\begin{Entry}{白萝卜}{5,11,2}{⽩、⾋、⼘}
  \begin{Phonetics}{白萝卜}{bai2luo2bo5}
    \definition{s.}{rabanete branco | \emph{daikon}}
  \end{Phonetics}
\end{Entry}

\begin{Entry}{白蛋白}{5,11,5}{⽩、⾍、⽩}
  \begin{Phonetics}{白蛋白}{bai2dan4bai2}
    \definition{s.}{albumina}
  \end{Phonetics}
\end{Entry}

\begin{Entry}{白领}{5,11}{⽩、⾴}
  \begin{Phonetics}{白领}{bai2 ling3}[][HSK 6]
    \definition[个,名,位,些]{s.}{colarinho branco; trabalhador de colarinho branco; refere-se a funcionários cujo trabalho principal envolve trabalho intelectual, são conhecidos por suas roupas elegantes, colarinhos e camisas brancas; atualmente é frequentemente usado para se referir àqueles que trabalham em cargos de gestão ou técnicos em empresas e ganham salários relativamente altos}
  \end{Phonetics}
\end{Entry}

\begin{Entry}{白鹄}{5,12}{⽩、⿃}
  \begin{Phonetics}{白鹄}{bai2hu2}
    \definition{s.}{cisne branco}
  \end{Phonetics}
\end{Entry}

\begin{Entry}{白痴}{5,13}{⽩、⽧}
  \begin{Phonetics}{白痴}{bai2chi1}
    \definition{adj.}{idiota; uma pessoa que sofre de idiotice; frequentemente usado para menosprezar alguém que é incompetente ou incapaz de fazer as coisas}
    \definition{s.}{idiotice; uma doença caracterizada por retardo mental, demência, fala arrastada, movimentos lentos e até mesmo incapacidade de cuidar de si mesmo}
  \end{Phonetics}
\end{Entry}

%%%%%%%%%% 皮 %%%%%%%%%%
\subsection*{皮}

\begin{Entry}{皮}{5}{⽪}[Kangxi 107]
  \begin{Phonetics}{皮}{pi2}[][HSK 3]
    \definition*{s.}{Sobrenome: Pi}
    \definition{adj.}{macios e encharcados; não mais crocantes | malandro; travesso | apático; endurecido; indiferente devido a repetidas repreensões | pegajoso; tenaz; resiliente}
    \definition{pref.}{pico- (um trilhonésimo)}
    \definition[层,块,张,个]{s.}{pele; casca; uma camada de tecido na superfície dos organismos animais e vegetais | pele; couro; couro processado | capa; embalagem; a camada externa que envolve algo | superfície do objeto | folha; peça larga e plana (de algum material fino) | borracha}
  \end{Phonetics}
\end{Entry}

\begin{Entry}{皮下}{5,3}{⽪、⼀}
  \begin{Phonetics}{皮下}{pi2xia4}
    \definition{adj.}{(injeção) subcutâneo | sob a pele}
  \end{Phonetics}
\end{Entry}

\begin{Entry}{皮包}{5,5}{⽪、⼓}
  \begin{Phonetics}{皮包}{pi2 bao1}[][HSK 3]
    \definition[个,只,款]{s.}{bolsa; pasta; portfólio; bolsas de couro}
  \end{Phonetics}
\end{Entry}

\begin{Entry}{皮卡}{5,5}{⽪、⼘}
  \begin{Phonetics}{皮卡}{pi2ka3}
    \definition{s.}{(empréstimo linguístico) \emph{pick-up} | caminhonete}
  \end{Phonetics}
\end{Entry}

\begin{Entry}{皮卡丘}{5,5,5}{⽪、⼘、⼀}
  \begin{Phonetics}{皮卡丘}{pi2ka3qiu1}
    \definition*{s.}{Pikachu (Pokémon, 口袋妖怪)}
  \seealsoref{口袋妖怪}{kou3dai4 yao1guai4}
  \end{Phonetics}
\end{Entry}

\begin{Entry}{皮肤}{5,8}{⽪、⾁}
  \begin{Phonetics}{皮肤}{pi2fu1}[][HSK 5]
    \definition{adj.}{superficial}
    \definition[种,块,片,层]{s.}{pele; couro; derme}
  \end{Phonetics}
\end{Entry}

\begin{Entry}{皮球}{5,11}{⽪、⽟}
  \begin{Phonetics}{皮球}{pi2 qiu2}[][HSK 6]
    \definition{s.}{bola (feita de borracha, couro etc.)}
  \end{Phonetics}
\end{Entry}

\begin{Entry}{皮鞋}{5,15}{⽪、⾰}
  \begin{Phonetics}{皮鞋}{pi2xie2}[][HSK 5]
    \definition[双,只,款]{s.}{sapatos feitos de couro}
  \end{Phonetics}
\end{Entry}

%%%%%%%%%% 目 %%%%%%%%%%
\subsection*{目}

\begin{Entry}{目}{5}{⽬}[Kangxi 109]
  \begin{Phonetics}{目}{mu4}
    \definition*{s.}{Sobrenome: Mu}
    \definition{s.}{olho | item | (biologia) ordem | lista de coisas; catálogo; sumário | buraco em uma rede; malha (abertura)  | (de documentos, teses, etc.) nome; título | ponto; ponto de território, um termo do Go; refere-se à intersecção das linhas verticais e horizontais no tabuleiro, uma intersecção é chamada de 一目, \dpy{yi2 mu4}}
    \definition{v.}{(literário) olhar; considerar}
  \end{Phonetics}
\end{Entry}

\begin{Entry}{目光}{5,6}{⽬、⼉}
  \begin{Phonetics}{目光}{mu4guang1}[][HSK 5]
    \definition[道,束,种]{s.}{olhar fixo; a expressão e atitude reveladas pelos olhos | visão; vista; percepção visual; a linha imaginária formada entre os olhos e o objeto quando se olha para ele | perspicácia (capacidade de observar e reconhecer coisas); conhecimento adquirido através do contato com as coisas, capacidade de observar as coisas}
  \end{Phonetics}
\end{Entry}

\begin{Entry}{目的}{5,8}{⽬、⽩}
  \begin{Phonetics}{目的}{mu4di4}[][HSK 2]
    \definition[个,些,种]{s.}{objetivo; meta; alvo; finalidade; propósito; o lugar ou situação que se deseja alcançar; o resultado que se deseja obter; o centro do alvo}
  \end{Phonetics}
\end{Entry}

\begin{Entry}{目前}{5,9}{⽬、⼑}
  \begin{Phonetics}{目前}{mu4qian2}[][HSK 3]
    \definition{adv.}{agora; recentemente; no momento; no presente}
  \end{Phonetics}
\end{Entry}

\begin{Entry}{目标}{5,9}{⽬、⽊}
  \begin{Phonetics}{目标}{mu4biao1}[][HSK 3]
    \definition[个,项]{s.}{alvo; objetivo; objeto de tiro, ataque ou busca| objetivo; meta; destino; a situação ou padrão que se deseja alcançar}
  \end{Phonetics}
\end{Entry}

%%%%%%%%%% 矛 %%%%%%%%%%
\subsection*{矛}

\begin{Entry}{矛}{5}{⽭}[Kangxi 110]
  \begin{Phonetics}{矛}{mao2}
    \definition{s.}{Arcaico: lança; lanceta}
  \end{Phonetics}
\end{Entry}

\begin{Entry}{矛盾}{5,9}{⽭、⽬}
  \begin{Phonetics}{矛盾}{mao2dun4}[][HSK 5]
    \definition{adj.}{contraditório; descreve pessoas ou coisas que se opõem ou se repelem mutuamente}
    \definition[对,个,种]{s.}{problema; contradição; discrepância; inconsistência | disputas e conflitos; relacionamento de oposição entre as duas partes devido a diferenças de opinião ou abordagem}
    \definition{v.}{opor-se; entrar em conflito; contradizer; nesta situação, apenas uma das opções está correta ou é verdadeira; não é possível que ambas estejam corretas ao mesmo tempo}
  \end{Phonetics}
\end{Entry}

%%%%%%%%%% 石 %%%%%%%%%%
\subsection*{石}

\begin{Entry}{石}{5}{⽯}[Kangxi 112]
  \begin{Phonetics}{石}{dan4}
    \definition{clas.}{dan, uma unidade de medida seca para grãos; unidade de capacidade, 10 斗 é igual a 1 石}
  \seealsoref{斗}{dou4}
  \end{Phonetics}
  \begin{Phonetics}{石}{shi2}
    \definition*{s.}{Sobrenome: Shi}
    \definition{s.}{pedra; rocha; o material duro que constitui a crosta terrestre é composto por uma coleção de minerais | inscrição em pedra; esculturas em pedra}
  \end{Phonetics}
\end{Entry}

\begin{Entry}{石头}{5,5}{⽯、⼤}
  \begin{Phonetics}{石头}{shi2tou5}[][HSK 3]
    \definition[块,堆,些]{s.}{rocha; pedra; uma substância muito dura que é o principal material da superfície da Terra}
  \end{Phonetics}
\end{Entry}

\begin{Entry}{石油}{5,8}{⽯、⽔}
  \begin{Phonetics}{石油}{shi2you2}[][HSK 3]
    \definition[桶,吨,升]{s.}{óleo; óleo fóssil; petróleo; um líquido inflamável extraído do solo, geralmente marrom escuro, preto ou verde escuro, do qual gasolina e outras substâncias podem ser obtidas}
  \end{Phonetics}
\end{Entry}

%%%%%%%%%% 示 %%%%%%%%%%
\subsection*{示}

\begin{Entry}{示}{5}{⽰}[Kangxi 113]
  \begin{Phonetics}{示}{shi4}
    \definition*{s.}{Sobrenome: Shi}
    \definition{s.}{(sua) carta  | missiva; instruções; palavras ou escritos para subordinados ou gerações mais jovens}
    \definition{v.}{mostrar; notificar; instruir | indicar; significar; mostrar ou apontar, fazer conhecido}
  \end{Phonetics}
\end{Entry}

\begin{Entry}{示范}{5,9}{⽰、⾋}
  \begin{Phonetics}{示范}{shi4fan4}[][HSK 5]
    \definition{v.}{demonstrar; dar o exemplo; criar um modelo que todos possam aprender}
  \end{Phonetics}
\end{Entry}

%%%%%%%%%% 礼 %%%%%%%%%%
\subsection*{礼}

\begin{Entry}{礼}{5}{⽰}
  \begin{Phonetics}{礼}{li3}[][HSK 5]
    \definition*{s.}{Sobrenome: Li}
    \definition[份]{s.}{observâncias cerimoniais em geral; cerimônia; rito | cortesia; etiqueta; boas maneiras | presente; oferta}
  \end{Phonetics}
\end{Entry}

\begin{Entry}{礼仪}{5,5}{⽰、⼈}
  \begin{Phonetics}{礼仪}{li3yi2}[][HSK 7-9]
    \definition[个,种]{s.}{rito; protocolo; etiqueta; cerimônia e decoro; etiqueta e cerimônia}
  \end{Phonetics}
\end{Entry}

\begin{Entry}{礼节}{5,5}{⽰、⾋}
  \begin{Phonetics}{礼节}{li3jie2}
    \definition{s.}{protocolo | cerimônia | etiqueta}
  \end{Phonetics}
\end{Entry}

\begin{Entry}{礼让}{5,5}{⽰、⾔}
  \begin{Phonetics}{礼让}{li3rang4}
    \definition{s.}{cortesia}
    \definition{v.}{mostrar consideração por (outros) | ceder a (outro veículo, etc.)}
  \end{Phonetics}
\end{Entry}

\begin{Entry}{礼服}{5,8}{⽰、⽉}
  \begin{Phonetics}{礼服}{li3fu2}[][HSK 7-9]
    \definition[件,个]{s.}{um vestido completo; um terno; traje formal; uma vestimenta cerimonial; vestuário usado em ocasiões formais ou durante cerimônias}
  \end{Phonetics}
\end{Entry}

\begin{Entry}{礼物}{5,8}{⽰、⽜}
  \begin{Phonetics}{礼物}{li3wu4}[][HSK 2]
    \definition[份,件,个,分,些]{s.}{presente; lembrança; itens oferecidos como forma de respeito ou celebração, referindo-se de maneira geral a itens oferecidos como presente}
  \end{Phonetics}
\end{Entry}

\begin{Entry}{礼品}{5,9}{⽰、⼝}
  \begin{Phonetics}{礼品}{li3pin3}[][HSK 7-9]
    \definition[个,份,件]{s.}{presente; dádiva; algo dado a alguém como sinal de gratidão ou bênção; algo oferecido como presente}
  \end{Phonetics}
\end{Entry}

\begin{Entry}{礼拜}{5,9}{⽰、⼿}
  \begin{Phonetics}{礼拜}{li3 bai4}[][HSK 5]
    \definition[个]{s.}{dia da semana; usado em conjunto com 一, 二, 三, 四, 五, 六, 日(或天, indica um dia específico da semana | semana; referência à semana | domingo}
    \definition{v.}{prestar homenagem aos deuses que veneram; rezar; orar}
  \end{Phonetics}
\end{Entry}

\begin{Entry}{礼堂}{5,11}{⽰、⼟}
  \begin{Phonetics}{礼堂}{li3 tang2}[][HSK 6]
    \definition[个,座,处]{s.}{auditórios; salão de assembleias; um salão para reuniões ou cerimônias}
  \end{Phonetics}
\end{Entry}

\begin{Entry}{礼貌}{5,14}{⽰、⾘}
  \begin{Phonetics}{礼貌}{li3mao4}[][HSK 5]
    \definition{adj.}{educado; descreve uma pessoa que fala e age respeitando os outros, sem arrogância, de acordo com as exigências das relações sociais}
    \definition{s.}{cortesia; educação; boas maneiras}
  \end{Phonetics}
\end{Entry}

%%%%%%%%%% 禾 %%%%%%%%%%
\subsection*{禾}

\begin{Entry}{禾}{5}{⽲}[Kangxi 115]
  \begin{Phonetics}{禾}{he2}
    \definition[棵]{s.}{mudas (especialmente de arroz) | painço}
  \end{Phonetics}
\end{Entry}

\begin{Entry}{禾苗}{5,8}{⽲、⾋}
  \begin{Phonetics}{禾苗}{he2miao2}[][HSK 7-9]
    \definition[棵,片]{s.}{mudas de cereais | muda (de arroz ou outro grão)}
  \end{Phonetics}
\end{Entry}

%%%%%%%%%% 立 %%%%%%%%%%
\subsection*{立}

\begin{Entry}{立}{5}{⽴}[Kangxi 117]
  \begin{Phonetics}{立}{li4}[][HSK 5]
    \definition{adj.}{ereto; vertical; na vertical}
    \definition{adv.}{imediatamente; instantaneamente}
    \definition{v.}{ficar em pé, com os pés no chão ou apoiados em algum objeto; o objeto deve estar na vertical | erguer; colocar (ou levantar) algo; colocar em pé | encontrar; criar; elaborar; formular; estabelecer | configurar; fundar; estabelecer | viver; existir | ascender ao trono; antigamente, referia-se à ascensão ao trono de um monarca | nomear; designar; antigamente, significava estabelecer uma determinada posição ou status}
  \end{Phonetics}
\end{Entry}

\begin{Entry}{立场}{5,6}{⽴、⼟}
  \begin{Phonetics}{立场}{li4chang3}[][HSK 5]
    \definition[个]{s.}{posição; postura; a posição e a atitude adotadas ao reconhecer e lidar com os problemas | ponto de vista; refere-se especificamente à atitude de reconhecer e lidar com questões a partir dos interesses de uma determinada classe, ou seja, a posição de classe}
  \end{Phonetics}
\end{Entry}

\begin{Entry}{立即}{5,7}{⽴、⼙}
  \begin{Phonetics}{立即}{li4ji2}[][HSK 4]
    \definition{adv.}{prontamente; imediatamente; de imediato}
  \end{Phonetics}
\end{Entry}

\begin{Entry}{立刻}{5,8}{⽴、⼑}
  \begin{Phonetics}{立刻}{li4ke4}[][HSK 3]
    \definition{adv.}{imediatamente; de ​​uma vez; indica que algo acontecerá imediatamente após um determinado momento}
  \end{Phonetics}
\end{Entry}

\begin{Entry}{立法}{5,8}{⽴、⽔}
  \begin{Phonetics}{立法}{li4fa3}
    \definition{s.}{legislação}
    \definition{v.}{promulgar leis | legislar}
  \end{Phonetics}
\end{Entry}

%%%%%%%%%% 纠 %%%%%%%%%%
\subsection*{纠}

\begin{Entry}{纠}{5}{⽷}
  \begin{Phonetics}{纠}{jiu1}
    \definition*{s.}{Sobrenome: Jiu}
    \definition{v.}{emaranhar | reunir-se | corrigir; retificar | supervisionar; superintender}
  \end{Phonetics}
\end{Entry}

\begin{Entry}{纠正}{5,5}{⽷、⽌}
  \begin{Phonetics}{纠正}{jiu1zheng4}[][HSK 6]
    \definition{v.}{fazer certo; corrigir (deficiências ou erros em pensamentos, ações, métodos, etc.)}
  \end{Phonetics}
\end{Entry}

\begin{Entry}{纠纷}{5,7}{⽷、⽷}
  \begin{Phonetics}{纠纷}{jiu1fen1}[][HSK 6]
    \definition[个,次]{s.}{questão; disputa; existem contradições ou conflitos de interesse entre as duas partes que precisam ser resolvidos}
  \end{Phonetics}
\end{Entry}

\begin{Entry}{纠葛}{5,12}{⽷、⾋}
  \begin{Phonetics}{纠葛}{jiu1ge2}
    \definition{s.}{emaranhado | disputa}
  \end{Phonetics}
\end{Entry}

\begin{Entry}{纠缠}{5,13}{⽷、⽷}
  \begin{Phonetics}{纠缠}{jiu1chan2}[][HSK 7-9]
    \definition{v.}{enredar-se; estar em apuros; entrelaçar | importunar; preocupar; perturbar; causar problemas}
  \end{Phonetics}
\end{Entry}

%%%%%%%%%% 艾 %%%%%%%%%%
\subsection*{艾}

\begin{Entry}{艾}{5}{⾋}
  \begin{Phonetics}{艾}{ai4}
    \definition*{s.}{Botânica: Artemísia chinesa (Artemisia argyi)}
    \definition*{s.}{Sobrenome: Ai}
    \definition{adj.}{Literário: gracioso; bonito; lindo}
    \definition{s.}{artemísia; absinto; artemísia chinesa}
    \definition{v.}{Literário: parar; terminar}
  \end{Phonetics}
  \begin{Phonetics}{艾}{yi4}
    \definition{adj.}{estável}
    \definition{v.}{ser corrigido; estar corrigido}
  \end{Phonetics}
\end{Entry}

\begin{Entry}{艾滋病}{5,12,10}{⾋、⽔、⽧}
  \begin{Phonetics}{艾滋病}{ai4zi1bing4}[][HSK 7-9]
    \definition*{s.}{Síndrome da Imunodeficiência Adquirida (AIDS)}
  \end{Phonetics}
\end{Entry}

%%%%%%%%%% 节 %%%%%%%%%%
\subsection*{节}

\begin{Entry}{节}{5}{⾋}
  \begin{Phonetics}{节}{jie1}
    \definition{adj.}{momento crucial; momento crítico; momento decisivo; metáfora para algo importante, decisivo ou oportuno}
  \end{Phonetics}
  \begin{Phonetics}{节}{jie2}[][HSK 2]
    \definition*{s.}{Sobrenome: Jie}
    \definition{clas.}{nó (kn), velocidade de um barco | para seções, comprimentos}
    \definition[个]{s.}{junta; botão; nó; geralmente se refere à parte da grama ou caule da grama onde as folhas crescem ou à parte onde os galhos e troncos das plantas são conectados | parte; divisão; um trecho de algo interligado; uma parte do todo | festival; feriado; dia memorável; um período de tempo ou um dia com características específicas | item; assunto | castidade; integridade ética e moral | articulação; o local onde os ossos humanos ou animais se conectam | etiqueta; cerimonial | batida; ritmo | registro; documento utilizado na antiguidade para comprovar a identidade | estação do ano | sílaba}
    \definition{v.}{economizar; conservar; poupar | resumir; extrair; retirar uma parte do todo | controlar; restringir; moderar}
  \end{Phonetics}
\end{Entry}

\begin{Entry}{节日}{5,4}{⾋、⽇}
  \begin{Phonetics}{节日}{jie2ri4}[][HSK 2]
    \definition[个,种,类]{s.}{festival; feriado; dia de comemoração tradicional; dia comemorativo estabelecido por lei}
  \end{Phonetics}
\end{Entry}

\begin{Entry}{节气}{5,4}{⾋、⽓}
  \begin{Phonetics}{节气}{jie2qi4}[][HSK 7-9]
    \definition{s.}{termo solar (é qualquer um dos 24 períodos nos calendários lunisolares tradicionais chineses); com base na duração do dia e da noite, na altura da sombra do meio-dia e em outros fatores, vários pontos são designados ao longo do ano, cada ponto é chamado de termo solar; os termos solares indicam a posição da Terra em sua órbita, ou seja, a posição do Sol na eclíptica; geralmente, também se referem ao dia da semana em que cada ponto ocorre}
  \end{Phonetics}
\end{Entry}

\begin{Entry}{节水}{5,4}{⾋、⽔}
  \begin{Phonetics}{节水}{jie2shui3}[][HSK 7-9]
    \definition{v.}{economizar água}
  \end{Phonetics}
\end{Entry}

\begin{Entry}{节目}{5,5}{⾋、⽬}
  \begin{Phonetics}{节目}{jie2mu4}[][HSK 2]
    \definition[个,场,项,台]{s.}{programa; item (em um programa); programas artísticos ou projetos transmitidos por rádios e televisões}
  \end{Phonetics}
\end{Entry}

\begin{Entry}{节约}{5,6}{⾋、⽷}
  \begin{Phonetics}{节约}{jie2yue1}[][HSK 3]
    \definition{adj.}{econômico; sem luxo}
    \definition{v.}{guardar; economizar; usar com moderação; economizar gastos desnecessários}
  \end{Phonetics}
\end{Entry}

\begin{Entry}{节衣缩食}{5,6,14,9}{⾋、⾐、⽷、⾷}
  \begin{Phonetics}{节衣缩食}{jie2yi1-suo1shi2}[][HSK 7-9]
    \definition{expr.}{``Reduza os gastos com comida e roupas.''; economizar em comida e roupas; ser mais econômico; viver frugalmente; praticar austeridade; praticar uma economia rigorosa}
  \end{Phonetics}
\end{Entry}

\begin{Entry}{节俭}{5,9}{⾋、⼈}
  \begin{Phonetics}{节俭}{jie2jian3}[][HSK 7-9]
    \definition{adj.}{econômico; frugal}
  \end{Phonetics}
\end{Entry}

\begin{Entry}{节奏}{5,9}{⾋、⼤}
  \begin{Phonetics}{节奏}{jie2zou4}[][HSK 6]
    \definition[个,种]{s.}{ritmo; o fenômeno da alternância regular de comprimento, força e fraqueza das notas na música | padrão regular; uma metáfora para um processo de ajuste adequado com tensão e relaxamento}
  \end{Phonetics}
\end{Entry}

\begin{Entry}{节省}{5,9}{⾋、⽬}
  \begin{Phonetics}{节省}{jie2sheng3}[][HSK 4]
    \definition{adj.}{econômico; parcimonioso}
    \definition{v.}{economizar; conservar; usar com moderação; reduzir; eliminar ou minimizar o esgotamento de itens potencialmente esgotáveis}
  \end{Phonetics}
\end{Entry}

\begin{Entry}{节能}{5,10}{⾋、⾁}
  \begin{Phonetics}{节能}{jie2 neng2}[][HSK 6]
    \definition{v.}{economizar no consumo de energia; conservar energia}
  \end{Phonetics}
\end{Entry}

\begin{Entry}{节假日}{5,11,4}{⾋、⼈、⽇}
  \begin{Phonetics}{节假日}{jie2 jia4 ri4}[][HSK 6]
    \definition[个]{s.}{feriados; festivais e feriados}
  \end{Phonetics}
\end{Entry}

%%%%%%%%%% 讨 %%%%%%%%%%
\subsection*{讨}

\begin{Entry}{讨}{5}{⾔}
  \begin{Phonetics}{讨}{tao3}
    \definition{v.}{enviar forças armadas para suprimir; enviar uma expedição punitiva contra; enviar exército ou despachar tropas para suprimir ou atacar | denunciar; condenar; censurar | exigir; pedir; implorar por | casar (com uma mulher) | incorrer; convidar | discutir; estudar | provocar; cortejar}
  \end{Phonetics}
\end{Entry}

\begin{Entry}{讨生活}{5,5,9}{⾔、⽣、⽔}
  \begin{Phonetics}{讨生活}{tao3sheng1huo2}
    \definition{v.}{ganhar a vida}
  \end{Phonetics}
\end{Entry}

\begin{Entry}{讨厌}{5,6}{⾔、⼚}
  \begin{Phonetics}{讨厌}{tao3yan4}[][HSK 5]
    \definition{adj.}{desagradável; repugnante; repulsivo; irritante; incômodo}
    \definition{v.}{odiar; não gostar; sentir repulsa por}
  \end{Phonetics}
\end{Entry}

\begin{Entry}{讨论}{5,6}{⾔、⾔}
  \begin{Phonetics}{讨论}{tao3lun4}[][HSK 2]
    \definition{v.}{discutir; conversar sobre; trocar opiniões ou debater as questões levantadas}
  \end{Phonetics}
\end{Entry}

%%%%%%%%%% 让 %%%%%%%%%%
\subsection*{让}

\begin{Entry}{让}{5}{⾔}
  \begin{Phonetics}{让}{rang4}[][HSK 2]
    \definition*{s.}{Sobrenome: Rang}
    \definition{prep.}{em uma frase passiva para introduzir o executor da ação | de acordo com; em conformidade com; à luz de; com base em; usado para expressar a opinião subjetiva de alguém}
    \definition{v.}{ceder; recuar; render-se; desistir; admitir | convidar; oferecer | deixar; permitir; fazer | deixar alguém ter algo por um preço justo | ser inferior a; não ser tão bom quanto | ceder; afastar-se | expressar desejos | esquivar-se; evitar; fugir | Usado antes de 我们, indica uma ordem ou sugestão para que todos façam algo juntos}
  \seealsoref{我们}{wo3men5}
  \end{Phonetics}
\end{Entry}

\begin{Entry}{让步}{5,7}{⾔、⽌}
  \begin{Phonetics}{让步}{rang4/bu4}
    \definition{v.+compl.}{ceder; fazer uma concessão; comprometer; em uma disputa, significa renunciar parcial ou totalmente às próprias opiniões ou interesses}
  \end{Phonetics}
\end{Entry}

\begin{Entry}{让座}{5,10}{⾔、⼴}
  \begin{Phonetics}{让座}{rang4 zuo4}[][HSK 6]
    \definition{v.}{oferecer seu lugar a alguém; ceder seu lugar a alguém | convidar os convidados para se sentarem}
  \end{Phonetics}
\end{Entry}

%%%%%%%%%% 训 %%%%%%%%%%
\subsection*{训}

\begin{Entry}{训}{5}{⾔}
  \begin{Phonetics}{训}{xun4}
    \definition{s.}{instrução; ensinamento; ensino | padrão; modelo; exemplo; regra; diretriz | explicação ou interpretação crítica de um texto | treinamento; exercício}
    \definition{v.}{instruir; admoestar; dar uma palestra a alguém; ensinar | explicar; instruir; explicação do significado da palavra | treinar}
  \end{Phonetics}
\end{Entry}

\begin{Entry}{训诂}{5,7}{⾔、⾔}
  \begin{Phonetics}{训诂}{xun4gu3}
    \definition{s.}{estudos exegéticos (de textos antigos); exegese}
    \definition{v.}{explicação de palavras e frases em livros antigos | interpretar e elaborar glossários e comentários sobre textos clássicos}
  \end{Phonetics}
\end{Entry}

\begin{Entry}{训练}{5,8}{⾔、⽷}
  \begin{Phonetics}{训练}{xun4lian4}[][HSK 3]
    \definition{v.}{treinar; exercitar; planejar e executar de forma sistemática o desenvolvimento de habilidades ou competências específicas}
  \end{Phonetics}
\end{Entry}

%%%%%%%%%% 议 %%%%%%%%%%
\subsection*{议}

\begin{Entry}{议}{5}{⾔}
  \begin{Phonetics}{议}{yi4}
    \definition[个,则,条]{s.}{opinião; visão}
    \definition{v.}{discutir; trocar pontos de vista sobre; conversar sobre | comentar; observar | fofocar; comentar}
  \end{Phonetics}
\end{Entry}

\begin{Entry}{议论}{5,6}{⾔、⾔}
  \begin{Phonetics}{议论}{yi4lun4}[][HSK 4]
    \definition[个]{s.}{comentário; discussão; opiniões ou pontos de vista sobre o que é bom ou ruim, certo ou errado em relação a pessoas ou coisas}
    \definition{v.}{discutir; comentar; falar sobre; expressar opiniões e trocar pontos de vista sobre o bom, o ruim, o certo e o errado de pessoas ou coisas}
  \end{Phonetics}
\end{Entry}

\begin{Entry}{议题}{5,15}{⾔、⾴}
  \begin{Phonetics}{议题}{yi4 ti2}[][HSK 6]
    \definition[项,个]{s.}{assunto; assunto em discussão; tópico para discussão}
  \end{Phonetics}
\end{Entry}

%%%%%%%%%% 记 %%%%%%%%%%
\subsection*{记}

\begin{Entry}{记}{5}{⾔}
  \begin{Phonetics}{记}{ji4}[][HSK 1]
    \definition*{s.}{Sobrenome: Ji}
    \definition{clas.}{tapas, palmadas, bofetadas, etc.; usado para indicar o número de vezes que uma determinada ação é realizada}
    \definition{s.}{assinatura; bloco de notas; livro ou artigo que registra fatos | insígnia; indicação; \& comercial; símbolo | marca de nascença; manchas escuras presentes na pele desde o nascimento}
    \definition{v.}{lembrar; ter em mente; guardar na memória; manter a imagem na mente | escrever (anotar); registrar; inscrever}
  \end{Phonetics}
\end{Entry}

\begin{Entry}{记忆}{5,4}{⾔、⼼}
  \begin{Phonetics}{记忆}{ji4yi4}[][HSK 5]
    \definition[段]{s.}{memória; manter em sua mente uma imagem do passado}
    \definition{v.}{recordar; lembrar; lembrar-se ou recordar alguém ou algo do passado}
  \end{Phonetics}
\end{Entry}

\begin{Entry}{记忆犹新}{5,4,7,13}{⾔、⼼、⽝、⽄}
  \begin{Phonetics}{记忆犹新}{ji4yi4-you2xin1}[][HSK 7-9]
    \definition{expr.}{estar fresco na memória; lembrar vividamente; estar ainda muito vivo na memória; ainda poder lembrar vividamente\dots; permanecer fresco na memória; ainda ter lembranças frescas de\dots; ainda reter memórias de\dots; a memória ainda está fresca.; a coisa está fresca na memória}
  \end{Phonetics}
\end{Entry}

\begin{Entry}{记号}{5,5}{⾔、⼝}
  \begin{Phonetics}{记号}{ji4hao5}[][HSK 7-9]
    \definition[个]{s.}{marca; sinal; marcas feitas para atrair atenção, auxiliar na identificação e na memória}
  \end{Phonetics}
\end{Entry}

\begin{Entry}{记住}{5,7}{⾔、⼈}
  \begin{Phonetics}{记住}{ji4 zhu5}[][HSK 1]
    \definition{v.}{lembrar; aprender de cor; ter em mente; guardar na memória}
  \end{Phonetics}
\end{Entry}

\begin{Entry}{记录}{5,8}{⾔、⼹}
  \begin{Phonetics}{记录}{ji4lu4}[][HSK 3]
    \definition[份,名,位,个]{s.}{notas; registro | anotador; registrador; a pessoa que faz registros}
    \definition{v.}{tomar notas; registrar; escrever o que ouviu ou o que aconteceu; gravar o som ou a imagem com um gravador ou uma câmera de vídeo e transformar em algum tipo de obra}
  \end{Phonetics}
\end{Entry}

\begin{Entry}{记性}{5,8}{⾔、⼼}
  \begin{Phonetics}{记性}{ji4xing5}
    \definition{s.}{memória (habilidade em reter informações)}
  \end{Phonetics}
\end{Entry}

\begin{Entry}{记者}{5,8}{⾔、⽼}
  \begin{Phonetics}{记者}{ji4zhe3}[][HSK 3]
    \definition[群,名,位]{s.}{repórter; correspondente; jornalista; profissionais dedicados a entrevistar e reportar notícias para a mídia}
  \end{Phonetics}
\end{Entry}

\begin{Entry}{记载}{5,10}{⾔、⾞}
  \begin{Phonetics}{记载}{ji4zai3}[][HSK 4]
    \definition[段,种,条]{s.}{registro; conta; artigos e materiais que registram eventos}
    \definition{v.}{registrar; colocar por escrito}
  \end{Phonetics}
\end{Entry}

\begin{Entry}{记得}{5,11}{⾔、⼻}
  \begin{Phonetics}{记得}{ji4de5}[][HSK 1]
    \definition{v.}{lembrar; recordar; lembrar-se; não esquecer | manter algo em mente; (informal) não se esquecer de fazer algo, usado para lembrar}
  \end{Phonetics}
\end{Entry}

%%%%%%%%%% 边 %%%%%%%%%%
\subsection*{边}

\begin{Entry}{边}{5}{⾡}
  \begin{Phonetics}{边}{bian1}[][HSK 2]
    \definition*{s.}{Sobrenome: Bian}
    \definition{adv.}{dois ou mais 边 são usados separadamente antes de diferentes verbos, indicando que diferentes ações ocorrem simultaneamente}
    \definition[条,个]{s.}{lado (de uma figura geométrica) | borda; lado; margem; aba; rebordo | fronteira; limite | ao lado de; lugar próximo a; perto de um objeto; lateral | aro; aba; borda; decoração em forma de faixa incrustada ou pintada na borda de um objeto}
    \definition{suf.}{lado; anexado a palavras de localização monossilábicas, formando palavras de localização dissílabas}
  \end{Phonetics}
  \begin{Phonetics}{边}{bian5}
    \definition{suf.}{sufixo de uma palavra de localidade (lado); indica posição e direção, usado após palavras que indicam direção, como 上, 下, 前, 后, 左, 右}
  \end{Phonetics}
\end{Entry}

\begin{Entry}{边关}{5,6}{⾡、⼋}
  \begin{Phonetics}{边关}{bian1guan1}
    \definition{s.}{posto de fronteira | posição defensiva estratégica na fronteira}
  \end{Phonetics}
\end{Entry}

\begin{Entry}{边防}{5,6}{⾡、⾩}
  \begin{Phonetics}{边防}{bian1fang2}
    \definition{s.}{defesa da fronteira}
  \end{Phonetics}
\end{Entry}

\begin{Entry}{边远}{5,7}{⾡、⾡}
  \begin{Phonetics}{边远}{bian1yuan3}[][HSK 7-9]
    \definition{adj.}{longe do centro; remoto; periférico; perto da fronteira; longe da área central}
  \end{Phonetics}
\end{Entry}

\begin{Entry}{边界}{5,9}{⾡、⽥}
  \begin{Phonetics}{边界}{bian1jie4}[][HSK 7-9]
    \definition{s.}{limite; linha de fronteira; fronteiras entre países ou regiões}
  \end{Phonetics}
\end{Entry}

\begin{Entry}{边缘}{5,12}{⾡、⽷}
  \begin{Phonetics}{边缘}{bian1yuan2}[][HSK 6]
    \definition{s.}{borda; beira; franja; uma área ou objeto próximo ao extremo |  borda; beira; algo está muito próximo de uma situação perigosa | interdisciplinar; relacionado a muitas coisas}
  \end{Phonetics}
\end{Entry}

\begin{Entry}{边境}{5,14}{⾡、⼟}
  \begin{Phonetics}{边境}{bian1jing4}[][HSK 5]
    \definition{s.}{fronteira; borda; perto da fronteira}
  \end{Phonetics}
\end{Entry}

\begin{Entry}{边疆}{5,19}{⾡、⼸}
  \begin{Phonetics}{边疆}{bian1jiang1}[][HSK 7-9]
    \definition{s.}{fronteira; área de fronteira; região de fronteira; território próximo à fronteira}
  \end{Phonetics}
\end{Entry}

%%%%%%%%%% 闪 %%%%%%%%%%
\subsection*{闪}

\begin{Entry}{闪}{5}{⾨}
  \begin{Phonetics}{闪}{shan3}[][HSK 4]
    \definition*{s.}{Sobrenome: Shan}
    \definition{s.}{relâmpago}
    \definition{v.}{esquivar-se; desviar; sair do caminho | torcer; distender | surgir de repente | cintilar; brilhar | deixar para trás; abandonar | (corpo) oscilar dramaticamente}
  \end{Phonetics}
\end{Entry}

\begin{Entry}{闪电}{5,5}{⾨、⽥}
  \begin{Phonetics}{闪电}{shan3dian4}[][HSK 4]
    \definition[道]{s.}{relâmpago; descargas elétricas entre nuvens ou entre nuvens e o solo}
  \seealsoref{雷电}{lei2dian4}
  \end{Phonetics}
\end{Entry}

\begin{Entry}{闪存盘}{5,6,11}{⾨、⼦、⽫}
  \begin{Phonetics}{闪存盘}{shan3cun2pan2}
    \definition{s.}{unidade de memória \emph{USB} | cartão de memória}
  \seealsoref{优盘}{you1pan2}
  \end{Phonetics}
\end{Entry}

%%%%%%%%%% 饥 %%%%%%%%%%
\subsection*{饥}

\begin{Entry}{饥}{5}{⾷}
  \begin{Phonetics}{饥}{ji1}
    \definition{adj.}{faminto; com fome}
    \definition[只]{s.}{fome; quebra de safra; colheita pobre ou inexistente}
  \end{Phonetics}
\end{Entry}

\begin{Entry}{饥饿}{5,10}{⾷、⾷}
  \begin{Phonetics}{饥饿}{ji1'e4}[][HSK 7-9]
    \definition{adj.}{faminto; esfomeado; estômago vazio; necessidade urgente de comer}[孩子们因饥饿哭泣。===As crianças choravam de fome.]
  \end{Phonetics}
\end{Entry}

%%%%%%%%%% 鸟 %%%%%%%%%%
\subsection*{鸟}

\begin{Entry}{鸟}{5}{⿃}[Kangxi 196]
  \begin{Phonetics}{鸟}{diao3}
    \definition{s.}{(em romances tradicionais, como termo pejorativo) maldito; condenado; fudido; o mesmo que 屌}
  \seealsoref{屌}{diao3}
  \end{Phonetics}
  \begin{Phonetics}{鸟}{niao3}[][HSK 2]
    \definition*{s.}{Sobrenome: Niao}
    \definition[只,群]{s.}{pássaro; ave}
  \end{Phonetics}
\end{Entry}

\begin{Entry}{鸟儿}{5,2}{⿃、⼉}
  \begin{Phonetics}{鸟儿}{niao3r5}
    \definition[只]{s.}{pássaro | ave}
  \end{Phonetics}
\end{Entry}

%%%%%%%%%% 龙 %%%%%%%%%%
\subsection*{龙}

\begin{Entry}{龙}{5}{⿓}[Kangxi 212]
  \begin{Phonetics}{龙}{long2}[][HSK 3]
    \definition*{s.}{Sobrenome: Long}
    \definition[条]{s.}{dragão; animal mítico e sobrenatural, com chifres, escamas, garras e bigodes, capaz de voar e mergulhar na água, provocar nuvens e chuva | dinossauro; um enorme réptil extinto; referência a certos répteis gigantes da antiguidade | do imperador; dragão como símbolo do imperador; usado na era feudal como símbolo do imperador; também se refere a coisas pertencentes ao imperador | em forma de dragão; com um desenho de dragão; refere-se a certos objetos que formam uma sequência semelhante a um dragão ou decorados com motivos de dragões}
  \end{Phonetics}
\end{Entry}

\begin{Entry}{龙山}{5,3}{⿓、⼭}
  \begin{Phonetics}{龙山}{long2shan1}
    \definition*{s.}{Longshan}
  \end{Phonetics}
\end{Entry}

\begin{Entry}{龙虾}{5,9}{⿓、⾍}
  \begin{Phonetics}{龙虾}{long2xia1}
    \definition{s.}{lagosta}
  \end{Phonetics}
\end{Entry}

%%%%% EOF %%%%%

