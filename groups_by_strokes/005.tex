%%%
%%% 5画
%%%

\section*{5画}\addcontentsline{toc}{section}{5画}

\begin{entry}{㐌}{5}[Radical 乙]
  \begin{phonetics}{㐌}{ta1}
    \variantof{它}
  \end{phonetics}
\end{entry}

\begin{entry}{世代}{5,5}
  \begin{phonetics}{世代}{shi4dai4}
    \definition{adv.}{por muitas gerações, eras}
    \definition{s.}{geração | era}
  \end{phonetics}
\end{entry}

\begin{entry}{世界}{5,9}
  \begin{phonetics}{世界}{shi4jie4}
    \definition[个]{s.}{mundo}
  \end{phonetics}
\end{entry}

\begin{entry}{世界杯}{5,9,8}
  \begin{phonetics}{世界杯}{shi4jie4bei1}
    \definition*{s.}{Copa do Mundo}
  \end{phonetics}
\end{entry}

\begin{entry}{世锦赛}{5,13,14}
  \begin{phonetics}{世锦赛}{shi4jin3sai4}
    \definition*{s.}{Campeonato Mundial}
  \end{phonetics}
\end{entry}

\begin{entry}{丘陵}{5,10}
  \begin{phonetics}{丘陵}{qiu1ling2}
    \definition{s.}{colinas}
  \end{phonetics}
\end{entry}

\begin{entry}{东}{5}[Radical ⼀]
  \begin{phonetics}{东}{dong1}[][HSK 1]
    \definition*{s.}{sobrenome Dong}
    \definition{s.}{leste}
  \end{phonetics}
\end{entry}

\begin{entry}{东方}{5,4}
  \begin{phonetics}{东方}{dong1 fang1}[][HSK 2]
    \definition*{s.}{sobrenome Dongfang}
    \definition{s.}{leste | oriente}
  \end{phonetics}
\end{entry}

\begin{entry}{东方学院}{5,4,8,9}
  \begin{phonetics}{东方学院}{dong1fang1 xue2yuan4}
    \definition*{s.}{Instituto Oriental}
  \end{phonetics}
\end{entry}

\begin{entry}{东北}{5,5}
  \begin{phonetics}{东北}{dong1 bei3}[][HSK 2]
    \definition*{s.}{Nordeste da China | Manchúria}
    \definition{s.}{nordeste}
  \end{phonetics}
\end{entry}

\begin{entry}{东半球}{5,5,11}
  \begin{phonetics}{东半球}{dong1ban4qiu2}
    \definition*{s.}{Hemisfério Oriental}
  \end{phonetics}
\end{entry}

\begin{entry}{东边}{5,5}
  \begin{phonetics}{东边}{dong1bian5}[][HSK 1]
    \definition{s.}{este | leste | lado leste | oriente}
  \end{phonetics}
\end{entry}

\begin{entry}{东西}{5,6}
  \begin{phonetics}{东西}{dong1xi1}[][HSK 0]
    \definition{s.}{leste e oeste}
  \end{phonetics}
  \begin{phonetics}{东西}{dong1xi5}[][HSK 1]
    \definition[个,件]{s.}{coisa | material | pessoa}
  \end{phonetics}
\end{entry}

\begin{entry}{东南}{5,9}
  \begin{phonetics}{东南}{dong1 nan2}[][HSK 2]
    \definition{s.}{sudeste | sudeste da China | o Sudeste}
  \end{phonetics}
\end{entry}

\begin{entry}{东面}{5,9}
  \begin{phonetics}{东面}{dong1mian4}
    \definition{s.}{lado leste (de algo)}
  \end{phonetics}
\end{entry}

\begin{entry}{东部}{5,10}
  \begin{phonetics}{东部}{dong1 bu4}[][HSK 3]
    \definition{s.}{o leste; parte oriental}
  \end{phonetics}
\end{entry}

\begin{entry}{丝}{5}[Radical 一]
  \begin{phonetics}{丝}{si1}
    \definition{adj.}{filiforme | delgado como um fio | que se assemelha a um fio}
    \definition{clas.}{um traço (de fumaça, etc.) | um pouquinho, etc.}
    \definition{s.}{seda | (cozinha) pedaços ou tiras de julienne, tiras cortadas finas}
  \end{phonetics}
\end{entry}

\begin{entry}{主人}{5,2}
  \begin{phonetics}{主人}{zhu3ren2}[][HSK 2]
    \definition[个,位]{s.}{mestre | anfitrião | proprietário | uma pessoa que tem um certo tipo de bens ou poder}
  \end{phonetics}
\end{entry}

\begin{entry}{主义}{5,3}
  \begin{phonetics}{主义}{zhu3yi4}
    \definition{s.}{ideologia}
    \definition{suf.}{"ismo"}
  \end{phonetics}
\end{entry}

\begin{entry}{主要}{5,9}
  \begin{phonetics}{主要}{zhu3yao4}[][HSK 2]
    \definition{adj.}{principal}
  \end{phonetics}
\end{entry}

\begin{entry}{主席}{5,10}
  \begin{phonetics}{主席}{zhu3xi2}
    \definition*[个,位]{s.}{Presidente (da China) | Primeiro-Ministro}
  \end{phonetics}
\end{entry}

\begin{entry}{主席台}{5,10,5}
  \begin{phonetics}{主席台}{zhu3xi2tai2}
    \definition[个]{s.}{plataforma | tribuna}
  \end{phonetics}
\end{entry}

\begin{entry}{主席团}{5,10,6}
  \begin{phonetics}{主席团}{zhu3xi2tuan2}
    \definition{s.}{presídio}
  \end{phonetics}
\end{entry}

\begin{entry}{乐观}{5,6}
  \begin{phonetics}{乐观}{le4guan1}
    \definition{adj.}{otimista | esperançoso}
  \end{phonetics}
\end{entry}

\begin{entry}{乐园}{5,7}
  \begin{phonetics}{乐园}{le4yuan2}
    \definition{s.}{paraíso}
  \end{phonetics}
\end{entry}

\begin{entry}{乐高}{5,10}
  \begin{phonetics}{乐高}{le4gao1}
    \definition*{s.}{Lego (brinquedo)}
  \end{phonetics}
\end{entry}

\begin{entry}{他}{5}[Radical 人]
  \begin{phonetics}{他}{ta1}[][HSK 1]
    \definition{pron.}{ele | se, o, lhe | si, consigo, ele}
    \seeref{怹}{tan1}
  \end{phonetics}
\end{entry}

\begin{entry}{他们}{5,5}
  \begin{phonetics}{他们}{ta1men5}[][HSK 1]
    \definition{pron.}{eles | se, os, lhes | si, consigo, eles}
  \end{phonetics}
\end{entry}

\begin{entry}{他们的}{5,5,8}
  \begin{phonetics}{他们的}{ta1men5 de5}
    \definition{pron.}{deles}
  \end{phonetics}
\end{entry}

\begin{entry}{他妈的}{5,6,8}
  \begin{phonetics}{他妈的}{ta1ma1de5}
    \definition{interj.}{Dane-se! | Foda-se!}
  \end{phonetics}
\end{entry}

\begin{entry}{他的}{5,8}
  \begin{phonetics}{他的}{ta1 de5}
    \definition{pron.}{dele}
  \end{phonetics}
\end{entry}

\begin{entry}{付}{5}[Radical 人]
  \begin{phonetics}{付}{fu4}[][HSK 3]
    \definition*{s.}{sobrenome Fu}
    \definition{clas.}{para pares ou conjuntos de coisas | para expressões faciais}
    \definition{v.}{comprometer-se a; entregar (entregar) a; entregar | pagar}
  \end{phonetics}
\end{entry}

\begin{entry}{付款}{5,12}
  \begin{phonetics}{付款}{fu4kuan3}
    \definition{s.}{pagamento}
    \definition{v.+compl.}{pagar uma quantia em dinheiro}
  \end{phonetics}
\end{entry}

\begin{entry}{仙}{5}[Radical 人]
  \begin{phonetics}{仙}{xian1}
    \definition{s.}{imortal}
  \end{phonetics}
\end{entry}

\begin{entry}{代}{5}[Radical 人]
  \begin{phonetics}{代}{dai4}[][HSK 3]
    \definition*{s.}{sobrenome Dai}
    \definition{s.}{período histórico | dinastia | geração | era}
    \definition{v.}{tomar o lugar de; estar no lugar de
agir em nome de; exercer}
  \end{phonetics}
\end{entry}

\begin{entry}{代价}{5,6}
  \begin{phonetics}{代价}{dai4jia4}
    \definition{s.}{preço | custo}
  \end{phonetics}
\end{entry}

\begin{entry}{代言}{5,7}
  \begin{phonetics}{代言}{dai4yan2}
    \definition{v.}{ser um porta-voz | ser um embaixador (para uma marca) | endossar}
  \end{phonetics}
\end{entry}

\begin{entry}{代表}{5,8}
  \begin{phonetics}{代表}{dai4biao3}[][HSK 3]
    \definition[位,个,名]{s.}{deputado; delegado; representante | representante oficial}
    \definition{v.}{representar; defender}
  \end{phonetics}
\end{entry}

\begin{entry}{代表团}{5,8,6}
  \begin{phonetics}{代表团}{dai4 biao3 tuan2}[][HSK 3]
    \definition[个]{s.}{delegação; contingente}
  \end{phonetics}
\end{entry}

\begin{entry}{代称}{5,10}
  \begin{phonetics}{代称}{dai4cheng1}
    \definition{s.}{nome alternativo | antonomásia}
    \definition{v.}{referir-se a algo ou alguém por outro nome}
  \end{phonetics}
\end{entry}

\begin{entry}{令人}{5,2}
  \begin{phonetics}{令人}{ling4ren2}
    \definition{v.}{causar alguém (a fazer alguma coisa) | fazer alguém ficar zangado, encantado, etc.}
  \end{phonetics}
\end{entry}

\begin{entry}{仪式}{5,6}
  \begin{phonetics}{仪式}{yi2shi4}
    \definition{s.}{cerimônia}
  \end{phonetics}
\end{entry}

\begin{entry}{们}{5}[Radical 人]
  \begin{phonetics}{们}{men5}[][HSK 1]
    \definition{part.}{sufixo para plural de pronomes e substantivos referentes a indivíduos}
  \end{phonetics}
\end{entry}

\begin{entry}{兄弟}{5,7}
  \begin{phonetics}{兄弟}{xiong1di4}
    \definition{adj.}{fraternal | \emph{brotherly}}
    \definition{pron.}{eu, me (termo de uso humilde por homens em discurso público)}
    \definition[个]{s.}{irmãos | irmão mais novo | \emph{brothers}}
  \end{phonetics}
\end{entry}

\begin{entry}{兰花}{5,7}
  \begin{phonetics}{兰花}{lan2hua1}
    \definition{s.}{orquídea}
  \end{phonetics}
\end{entry}

\begin{entry}{写}{5}[Radical 冖]
  \begin{phonetics}{写}{xie3}[][HSK 1]
    \definition{v.}{escrever}
  \end{phonetics}
\end{entry}

\begin{entry}{写字}{5,6}
  \begin{phonetics}{写字}{xie3zi4}
    \definition{v.}{escrever (à mão) | praticar caligrafia}
  \end{phonetics}
\end{entry}

\begin{entry}{写字匠}{5,6,6}
  \begin{phonetics}{写字匠}{xie3zi4 jiang4}
    \definition{s.}{calígrafo}
  \end{phonetics}
\end{entry}

\begin{entry}{写作}{5,7}
  \begin{phonetics}{写作}{xie3zuo4}
    \definition{s.}{escrita | redação | composição}
    \definition{v.}{escrever}
  \end{phonetics}
\end{entry}

\begin{entry}{写真}{5,10}
  \begin{phonetics}{写真}{xie3zhen1}
    \definition{s.}{retrato}
    \definition{v.}{descrever algo com precisão}
  \end{phonetics}
\end{entry}

\begin{entry}{写意}{5,13}
  \begin{phonetics}{写意}{xie3yi4}
    \definition{s.}{estilo de pintura chinesa à mão livre, caracterizado por traços ousados em vez de detalhes precisos}
    \definition{v.}{sugerir (em vez de descrever em detalhes)}
  \end{phonetics}
  \begin{phonetics}{写意}{xie4yi4}
    \definition{adj.}{confortável | agradável | relaxado}
  \end{phonetics}
\end{entry}

\begin{entry}{写照}{5,13}
  \begin{phonetics}{写照}{xie3zhao4}
    \definition{s.}{retrato}
  \end{phonetics}
\end{entry}

\begin{entry}{冬}{5}[Radical 冫]
  \begin{phonetics}{冬}{dong1}
    \definition*{s.}{sobrenome Dong}
    \definition{s.}{inverno}
  \end{phonetics}
\end{entry}

\begin{entry}{冬天}{5,4}
  \begin{phonetics}{冬天}{dong1 tian1}[][HSK 2]
    \definition{s.}{inverno}
  \end{phonetics}
\end{entry}

\begin{entry}{冬瓜}{5,5}
  \begin{phonetics}{冬瓜}{dong1gua1}
    \definition{s.}{melão de inverno}
  \end{phonetics}
\end{entry}

\begin{entry}{出}{5}[Radical ⼐]
  \begin{phonetics}{出}{chu1}[][HSK 1]
    \definition{clas.}{para dramas, peças, óperas, etc.}
    \definition{v.}{sair | ir para fora | vir para fora}
  \end{phonetics}
\end{entry}

\begin{entry}{出口}{5,3}
  \begin{phonetics}{出口}{chu1kou3}[][HSK 2]
    \definition[个]{s.}{exportação}
    \definition{v.+compl.}{exportar}
  \end{phonetics}
\end{entry}

\begin{entry}{出门}{5,3}
  \begin{phonetics}{出门}{chu1 men2}[][HSK 2]
    \definition{v.+compl.}{sair | sair de casa | estar longe de casa | fazer uma viagem | casar}
  \end{phonetics}
\end{entry}

\begin{entry}{出击}{5,5}
  \begin{phonetics}{出击}{chu1ji1}
    \definition{v.}{atacar}
  \end{phonetics}
\end{entry}

\begin{entry}{出去}{5,5}
  \begin{phonetics}{出去}{chu1 qu4}[][HSK 1]
    \definition{v.}{sair | ir para fora (a partir da minha localização)}
  \end{phonetics}
\end{entry}

\begin{entry}{出发}{5,5}
  \begin{phonetics}{出发}{chu1fa1}[][HSK 2]
    \definition{v.}{partir | começar (uma jornada)}
  \end{phonetics}
\end{entry}

\begin{entry}{出生}{5,5}
  \begin{phonetics}{出生}{chu1sheng1}[][HSK 2]
    \definition{v.}{nascer}
  \end{phonetics}
\end{entry}

\begin{entry}{出汗}{5,6}
  \begin{phonetics}{出汗}{chu1han4}
    \definition{v.}{transpirar | suar}
  \end{phonetics}
\end{entry}

\begin{entry}{出行}{5,6}
  \begin{phonetics}{出行}{chu1xing2}
    \definition{v.}{sair para algum lugar (viagem relativamente curta) | partir em uma viagem (viagem mais longa)}
  \end{phonetics}
\end{entry}

\begin{entry}{出来}{5,7}
  \begin{phonetics}{出来}{chu1 lai2}[][HSK 1]
    \definition{v.}{sair | vir para fora (para a minha localização)}
  \end{phonetics}
\end{entry}

\begin{entry}{出国}{5,8}
  \begin{phonetics}{出国}{chu1 guo2}[][HSK 2]
    \definition{v.+compl.}{ir para o exterior | deixar a terra natal}
  \end{phonetics}
\end{entry}

\begin{entry}{出版}{5,8}
  \begin{phonetics}{出版}{chu1ban3}
    \definition{v.}{publicar | editar}
  \end{phonetics}
\end{entry}

\begin{entry}{出版社}{5,8,7}
  \begin{phonetics}{出版社}{chu1ban3she4}
    \definition{s.}{editora}
  \end{phonetics}
\end{entry}

\begin{entry}{出现}{5,8}
  \begin{phonetics}{出现}{chu1xian4}[][HSK 2]
    \definition{v.}{aparecer | surgir | emergir | crescer}
  \end{phonetics}
\end{entry}

\begin{entry}{出差}{5,9}
  \begin{phonetics}{出差}{chu1chai1}
    \definition{v.+compl.}{fazer uma viagem oficial ou de negócios}
  \end{phonetics}
\end{entry}

\begin{entry}{出院}{5,9}
  \begin{phonetics}{出院}{chu1 yuan4}[][HSK 2]
    \definition{v.}{deixar o hospital | estar fora do hospital | ter alta do hospital}
  \end{phonetics}
\end{entry}

\begin{entry}{出租}{5,10}
  \begin{phonetics}{出租}{chu1 zu1}[][HSK 2]
    \definition{v.}{alugar | arrendar}
  \end{phonetics}
\end{entry}

\begin{entry}{出租车}{5,10,4}
  \begin{phonetics}{出租车}{chu1zu1che1}[][HSK 2]
    \definition{s.}{táxi}
  \seealsoref{出租汽车}{chu1zu1qi4che1}
  \end{phonetics}
\end{entry}

\begin{entry}{出租司机}{5,10,5,6}
  \begin{phonetics}{出租司机}{chu1zu1si1ji1}
    \definition{s.}{motorista de táxi}
  \end{phonetics}
\end{entry}

\begin{entry}{出租汽车}{5,10,7,4}
  \begin{phonetics}{出租汽车}{chu1zu1qi4che1}
    \definition[辆]{s.}{táxi}
  \seealsoref{出租车}{chu1zu1che1}
  \end{phonetics}
\end{entry}

\begin{entry}{出站}{5,10}
  \begin{phonetics}{出站}{chu1 zhan4}
    \definition{s.}{saída da estação}
  \end{phonetics}
\end{entry}

\begin{entry}{功夫}{5,4}
  \begin{phonetics}{功夫}{gong1fu5}[][HSK 3]
    \definition*{s.}{Gongfu (Kung Fu), arte marcial}
    \definition[番]{s.}{habilidade; feitura | luta acrobática; habilidade em artes marciais | esforço; tempo e energia}
  \end{phonetics}
\end{entry}

\begin{entry}{功臣}{5,6}
  \begin{phonetics}{功臣}{gong1chen2}
    \definition{s.}{oficial meritório | pessoa que presta serviço excepcional, herói | (fig.) alguém que desempenha um papel vital}
  \end{phonetics}
\end{entry}

\begin{entry}{功能}{5,10}
  \begin{phonetics}{功能}{gong1neng2}[][HSK 3]
    \definition[种,项]{s.}{função}
  \end{phonetics}
\end{entry}

\begin{entry}{功课}{5,10}
  \begin{phonetics}{功课}{gong1 ke4}[][HSK 3]
    \definition[份,门]{s.}{trabalho escolar; dever de casa | tarefa; lições; lição escolar}
  \end{phonetics}
\end{entry}

\begin{entry}{加}{5}[Radical 力]
  \begin{phonetics}{加}{jia1}[][HSK 2]
    \definition*{s.}{Canadá, abreviação de~加拿大 | sobrenome Jia}
    \seeref{加拿大}{jia1na2da4}
  \end{phonetics}
\end{entry}

\begin{entry}{加入}{5,2}
  \begin{phonetics}{加入}{jia1ru4}
    \definition{v.}{tornar-se um membro | juntar-se | participar de | adicionar em}
  \end{phonetics}
\end{entry}

\begin{entry}{加工}{5,3}
  \begin{phonetics}{加工}{jia1gong1}[][HSK 3]
    \definition{s.}{processo | trabalho (de uma máquina)}
    \definition{v.}{processar | melhorar; polir}
  \end{phonetics}
\end{entry}

\begin{entry}{加快}{5,7}
  \begin{phonetics}{加快}{jia1 kuai4}[][HSK 3]
    \definition{v.}{acelerar; aumentar a velocidade}
  \end{phonetics}
\end{entry}

\begin{entry}{加油}{5,8}
  \begin{phonetics}{加油}{jia1you2}[][HSK 2]
    \definition{v.+compl.}{lubrificar | encher o tanque de combustível | fazer um esforço maior | fazer um esforço extra}
  \end{phonetics}
\end{entry}

\begin{entry}{加拿大}{5,10,3}
  \begin{phonetics}{加拿大}{jia1na2da4}
    \definition{s.}{Canadá}
  \end{phonetics}
\end{entry}

\begin{entry}{加拿大人}{5,10,3,2}
  \begin{phonetics}{加拿大人}{jia1na2da4ren2}
    \definition{s.}{canadense | pessoa ou povo do Canadá}
  \end{phonetics}
\end{entry}

\begin{entry}{加速}{5,10}
  \begin{phonetics}{加速}{jia1su4}
    \definition{v.}{acelerar | agilizar}
  \end{phonetics}
\end{entry}

\begin{entry}{加速度}{5,10,9}
  \begin{phonetics}{加速度}{jia1su4du4}
    \definition{s.}{aceleração}
  \end{phonetics}
\end{entry}

\begin{entry}{加强}{5,12}
  \begin{phonetics}{加强}{jia1 qiang2}[][HSK 3]
    \definition{v.}{fortalecer; engrandecer; reforçar}
  \end{phonetics}
\end{entry}

\begin{entry}{务实}{5,8}
  \begin{phonetics}{务实}{wu4shi2}
    \definition{adj.}{pragmático}
    \definition{v.}{lidar com assuntos concretos}
  \end{phonetics}
\end{entry}

\begin{entry}{包}{5}[Radical 勹]
  \begin{phonetics}{包}{bao1}[][HSK 1]
    \definition*{s.}{sobrenome Bao}
    \definition{clas.}{pacotes, sacos, sacolas, embrulhos}
    \definition[个,只]{s.}{bolsa | pacote | recipiente | embrulho}
    \definition{v.}{contratar | cobrir | segurar ou abraçar | incluir | assumir o comando | embrulhar}
  \end{phonetics}
\end{entry}

\begin{entry}{包子}{5,3}
  \begin{phonetics}{包子}{bao1zi5}[][HSK 1]
    \definition[个]{s.}{pão recheado cozido no vapor}
  \end{phonetics}
\end{entry}

\begin{entry}{包干}{5,3}
  \begin{phonetics}{包干}{bao1gan1}
    \definition{s.}{tarefa alocada}
    \definition{v.}{ter a responsabilidade total sobre um trabalho}
  \end{phonetics}
\end{entry}

\begin{entry}{包办}{5,4}
  \begin{phonetics}{包办}{bao1ban4}
    \definition{v.}{comandar todo o show | comprometer-se a fazer tudo sozinho}
  \end{phonetics}
\end{entry}

\begin{entry}{包括}{5,9}
  \begin{phonetics}{包括}{bao1kuo4}
    \definition{v.}{compreender | consistir em | incluir | incorporar | envolver}
  \end{phonetics}
\end{entry}

\begin{entry}{包容}{5,10}
  \begin{phonetics}{包容}{bao1rong2}
    \definition{adj.}{inclusivo}
    \definition{v.}{perdoar | mostrar tolerância | conter | segurar}
  \end{phonetics}
\end{entry}

\begin{entry}{包租}{5,10}
  \begin{phonetics}{包租}{bao1zu1}
    \definition{s.}{aluguel fixo para terras agrícolas}
    \definition{v.}{fretar | alugar | alugar um terreno ou uma casa para subarrendar}
  \end{phonetics}
\end{entry}

\begin{entry}{匆匆}{5,5}
  \begin{phonetics}{匆匆}{cong1cong1}
    \definition{adv.}{apressadamente}
  \end{phonetics}
\end{entry}

\begin{entry}{北}{5}[Radical 匕]
  \begin{phonetics}{北}{bei3}[][HSK 1]
    \definition{s.}{norte}
    \definition{v.}{(clássico) ser derrotado}
  \end{phonetics}
\end{entry}

\begin{entry}{北大西洋公约组织}{5,3,6,9,4,6,8,8}
  \begin{phonetics}{北大西洋公约组织}{bei3 da4xi1 yang2 gong1 yue1 zu3zhi1}
    \definition*{s.}{Organização do Tratado do Atlântico Norte, OTAN}
  \end{phonetics}
\end{entry}

\begin{entry}{北方}{5,4}
  \begin{phonetics}{北方}{bei3fang1}[][HSK 2]
    \definition{s.}{norte | a parte norte de um país}
  \end{phonetics}
\end{entry}

\begin{entry}{北边}{5,5}
  \begin{phonetics}{北边}{bei3bian1}[][HSK 1]
    \definition{adv.}{lado norte | ao norte de}
  \end{phonetics}
\end{entry}

\begin{entry}{北约}{5,6}
  \begin{phonetics}{北约}{bei3yue1}
    \definition*{s.}{OTAN (Organização do Tratado do Atlântico Norte), abreviação de 北大西洋公约组织}
    \seeref{北大西洋公约组织}{bei3 da4xi1 yang2 gong1 yue1 zu3zhi1}
  \end{phonetics}
\end{entry}

\begin{entry}{北极}{5,7}
  \begin{phonetics}{北极}{bei3ji2}
    \definition*{s.}{Ártico | Pólo Norte}
    \definition{s.}{pólo norte magnético}
  \end{phonetics}
\end{entry}

\begin{entry}{北京}{5,8}
  \begin{phonetics}{北京}{bei3jing1}[][HSK 1]
    \definition*{s.}{Beijing (Pequim), Capital da República Popular da China | Beijing (Pequim), governo da RPC}
  \end{phonetics}
\end{entry}

\begin{entry}{北面}{5,9}
  \begin{phonetics}{北面}{bei3mian4}
    \definition{s.}{lado norte}
  \end{phonetics}
\end{entry}

\begin{entry}{北部}{5,10}
  \begin{phonetics}{北部}{bei3 bu4}[][HSK 3]
    \definition{s.}{parte norte}
  \end{phonetics}
\end{entry}

\begin{entry}{半}{5}[Radical 十]
  \begin{phonetics}{半}{ban4}[][HSK 1]
    \definition{adj.}{incompleto}
    \definition{num.}{(depois de um número) ``e meio''}
    \definition{pref.}{semi}
    \definition{s.}{metade}
  \end{phonetics}
\end{entry}

\begin{entry}{半天}{5,4}
  \begin{phonetics}{半天}{ban4tian1}[][HSK 1]
    \definition{s.}{metade do dia | muito tempo | bastante tempo}
  \end{phonetics}
\end{entry}

\begin{entry}{半年}{5,6}
  \begin{phonetics}{半年}{ban4 nian2}[][HSK 1]
    \definition{s.}{meio ano}
  \end{phonetics}
\end{entry}

\begin{entry}{半夜}{5,8}
  \begin{phonetics}{半夜}{ban4 ye4}[][HSK 2]
    \definition{adv.}{no meio da noite | metade de uma noite}
    \definition{s.}{meia-noite}
  \end{phonetics}
\end{entry}

\begin{entry}{半音}{5,9}
  \begin{phonetics}{半音}{ban4yin1}
    \definition{s.}{semitom}
  \end{phonetics}
\end{entry}

\begin{entry}{半球}{5,11}
  \begin{phonetics}{半球}{ban4qiu2}
    \definition{s.}{hemisfério}
  \end{phonetics}
\end{entry}

\begin{entry}{占}{5}[Radical 卜]
  \begin{phonetics}{占}{zhan1}
    \definition*{s.}{sobrenome Zhan}
    \definition{v.}{praticar adivinhação | advinhar}
  \end{phonetics}
  \begin{phonetics}{占}{zhan4}
    \definition{v.}{ocupar | apreender | tomar | constituir | manter | compor | dar conta de}
  \end{phonetics}
\end{entry}

\begin{entry}{卡}{5}[Radical 卜]
  \begin{phonetics}{卡}{ka3}
    \definition{clas.}{para calorias}
    \definition{s.}{cartão}
    \definition{v.}{bloquear | verificar | agarrar}
  \end{phonetics}
  \begin{phonetics}{卡}{qia3}
    \definition[张]{s.}{grampo | prendedor}
    \definition{s.}{posto de controle}
    \definition{v.}{cunhar | ficar preso | encravar}
  \end{phonetics}
\end{entry}

\begin{entry}{卡片}{5,4}
  \begin{phonetics}{卡片}{ka3pian4}
    \definition{s.}{cartão}
  \end{phonetics}
\end{entry}

\begin{entry}{卡片游戏}{5,4,12,6}
  \begin{phonetics}{卡片游戏}{ka3pian4 you2xi4}
    \definition{s.}{carta de baralho}
  \end{phonetics}
\end{entry}

\begin{entry}{卡车司机}{5,4,5,6}
  \begin{phonetics}{卡车司机}{ka3che1 si1ji1}
    \definition{s.}{motorista de caminhão}
  \end{phonetics}
\end{entry}

\begin{entry}{卡通}{5,10}
  \begin{phonetics}{卡通}{ka3tong1}
    \definition{s.}{(empréstimo linguístico) \emph{cartoon}}
  \end{phonetics}
\end{entry}

\begin{entry}{卢旺达}{5,8,6}
  \begin{phonetics}{卢旺达}{lu2wang4da2}
    \definition*{s.}{Ruanda}
  \end{phonetics}
\end{entry}

\begin{entry}{厉害}{5,10}
  \begin{phonetics}{厉害}{li4hai5}
    \definition{adj.}{severo | rigoroso | exigente | radical | violento | feroz}
  \end{phonetics}
\end{entry}

\begin{entry}{厺}{5}[Radical 厶]
  \begin{phonetics}{厺}{qu4}
    \variantof{去}
  \end{phonetics}
\end{entry}

\begin{entry}{去}{5}[Radical 厶]
  \begin{phonetics}{去}{qu4}[][HSK 1]
    \definition{v.}{ir | (eufenismo) morrer}
  \end{phonetics}
\end{entry}

\begin{entry}{去年}{5,6}
  \begin{phonetics}{去年}{qu4nian2}[][HSK 1]
    \definition{s.}{ano passado}
  \end{phonetics}
\end{entry}

\begin{entry}{去死}{5,6}
  \begin{phonetics}{去死}{qu4si3}
    \definition{interj.}{Caia morto! | Vá para o Inferno!}
  \end{phonetics}
\end{entry}

\begin{entry}{发}{5}[Radical ⼜]
  \begin{phonetics}{发}{fa1}
    \definition{clas.}{para tiros (rodadas)}
    \definition{v.}{enviar | mandar}
  \end{phonetics}
  \begin{phonetics}{发}{fa4}
    \definition{s.}{cabelo}
  \end{phonetics}
\end{entry}

\begin{entry}{发出}{5,5}
  \begin{phonetics}{发出}{fa1 chu1}[][HSK 3]
    \definition{v.}{fazer; produzir; deixar sair | emitir; anunciar | enviar; partir | dar; emitir}
  \end{phonetics}
\end{entry}

\begin{entry}{发生}{5,5}
  \begin{phonetics}{发生}{fa1sheng1}[][HSK 3]
    \definition{v.}{ocorrer; acontecer; tomar lugar}
  \end{phonetics}
\end{entry}

\begin{entry}{发动}{5,6}
  \begin{phonetics}{发动}{fa1dong4}[][HSK 3]
    \definition{v.}{iniciar; lançar; ligar motor; dar a partida (motor de combustão interna) | chamar à ação; mobilizar; estimular; despertar}
  \end{phonetics}
\end{entry}

\begin{entry}{发动机}{5,6,6}
  \begin{phonetics}{发动机}{fa1dong4ji1}
    \definition[台]{s.}{motor}
  \end{phonetics}
\end{entry}

\begin{entry}{发达}{5,6}
  \begin{phonetics}{发达}{fa1da2}[][HSK 3]
    \definition{adj.}{desenvolvido; florescente}
    \definition{v.}{desenvolver; promover; florescer}
  \end{phonetics}
\end{entry}

\begin{entry}{发抖}{5,7}
  \begin{phonetics}{发抖}{fa1dou3}
    \definition{v.}{tremer | sacudir | estremecer}
  \end{phonetics}
\end{entry}

\begin{entry}{发言}{5,7}
  \begin{phonetics}{发言}{fa1yan2}[][HSK 3]
    \definition[个]{s.}{discurso; declaração; palestra}
    \definition{v.+compl.}{falar; fazer uma declaração (discurso)}
  \end{phonetics}
\end{entry}

\begin{entry}{发财}{5,7}
  \begin{phonetics}{发财}{fa1cai2}
    \definition{v.+compl.}{ficar rico | fazer fortuna}
  \end{phonetics}
\end{entry}

\begin{entry}{发明}{5,8}
  \begin{phonetics}{发明}{fa1ming2}[][HSK 3]
    \definition[个]{s.}{invenção}
    \definition{v.}{inventar | expor; explicar}
  \end{phonetics}
\end{entry}

\begin{entry}{发明者}{5,8,8}
  \begin{phonetics}{发明者}{fa1ming2zhe3}
    \definition{s.}{inventor}
  \end{phonetics}
\end{entry}

\begin{entry}{发现}{5,8}
  \begin{phonetics}{发现}{fa1xian4}[][HSK 2]
    \definition{s.}{descoberta}
    \definition{v.}{perceber, tornar-se ciente de | descobrir, encontrar, detectar}
  \end{phonetics}
\end{entry}

\begin{entry}{发现者}{5,8,8}
  \begin{phonetics}{发现者}{fa1xian4 zhe3}
    \definition{s.}{descobridor}
  \end{phonetics}
\end{entry}

\begin{entry}{发表}{5,8}
  \begin{phonetics}{发表}{fa1biao3}[][HSK 3]
    \definition{v.}{publicar; entregar; emitir; expressar; anunciar | publicar}
  \end{phonetics}
\end{entry}

\begin{entry}{发型}{5,9}
  \begin{phonetics}{发型}{fa4xing2}
    \definition{s.}{penteado}
  \end{phonetics}
\end{entry}

\begin{entry}{发送}{5,9}
  \begin{phonetics}{发送}{fa1 song4}[][HSK 3]
    \definition{v.}{enviar; despachar | transmitir; enviar}
  \end{phonetics}
\end{entry}

\begin{entry}{发音}{5,9}
  \begin{phonetics}{发音}{fa1yin1}
    \definition{s.}{pronúncia}
    \definition{v.}{pronunciar}
  \end{phonetics}
\end{entry}

\begin{entry}{发展}{5,10}
  \begin{phonetics}{发展}{fa1zhan3}[][HSK 3]
    \definition{s.}{desenvolvimento}
    \definition{v.}{crescer; expandir; avançar; desenvolver | recrutar; expandir; admitir}
  \end{phonetics}
\end{entry}

\begin{entry}{发烧}{5,10}
  \begin{phonetics}{发烧}{fa1shao1}
    \definition{v.}{ter febre}
  \end{phonetics}
\end{entry}

\begin{entry}{发票}{5,11}
  \begin{phonetics}{发票}{fa1piao4}
    \definition{s.}{fatura | recibo | conta}
  \end{phonetics}
\end{entry}

\begin{entry}{发愁}{5,13}
  \begin{phonetics}{发愁}{fa1chou2}
    \definition{v.+compl.}{preocupar-se | ficar ansioso | ficar triste}
  \end{phonetics}
\end{entry}

\begin{entry}{发簪}{5,18}
  \begin{phonetics}{发簪}{fa4zan1}
    \definition{s.}{grampo de cabelo}
  \end{phonetics}
\end{entry}

\begin{entry}{古}{5}[Radical ⼝]
  \begin{phonetics}{古}{gu3}[][HSK 3]
    \definition*{s.}{sobrenome Gu}
    \definition{adj.}{arcaico; antigo; antiquíssimo}
    \definition{pref.}{``paleo''; ``arqueo''}
    \definition{s.}{antiguidade; arcaísmo | livros ou ortodoxias de antigos sábios | uma forma de poesia pré-Tang}
  \end{phonetics}
\end{entry}

\begin{entry}{古人}{5,2}
  \begin{phonetics}{古人}{gu3ren2}
    \definition{s.}{pessoas dos tempos antigos | os antigos | espécies humanas extintas, como \emph{Homo erectus} ou \emph{Homo neanderthalensis} | (literário) pessoa falecida}
  \end{phonetics}
\end{entry}

\begin{entry}{古代}{5,5}
  \begin{phonetics}{古代}{gu3dai4}[][HSK 3]
    \definition{s.}{tempos antigos | sociedade antiga; sociedade primitiva | antigamente}
  \end{phonetics}
\end{entry}

\begin{entry}{古老}{5,6}
  \begin{phonetics}{古老}{gu3lao3}
    \definition{adj.}{ancestral | antigo | velho}
  \end{phonetics}
\end{entry}

\begin{entry}{古城}{5,9}
  \begin{phonetics}{古城}{gu3cheng2}
    \definition{s.}{cidade antiga}
  \end{phonetics}
\end{entry}

\begin{entry}{古铜色}{5,11,6}
  \begin{phonetics}{古铜色}{gu3tong2 se4}
    \definition{s.}{cor bronze}
  \end{phonetics}
\end{entry}

\begin{entry}{句}{5}[Radical 口]
  \begin{phonetics}{句}{gou4}
    \variantof{勾}
  \end{phonetics}
  \begin{phonetics}{句}{ju4}
    \definition{clas.}{para orações, frases ou linhas de versos}
    \definition{s.}{sentença | cláusula | frase}
  \end{phonetics}
\end{entry}

\begin{entry}{句子}{5,3}
  \begin{phonetics}{句子}{ju4zi5}[][HSK 2]
    \definition[个]{s.}{sentença | frase | oração}
  \end{phonetics}
\end{entry}

\begin{entry}{另外}{5,5}
  \begin{phonetics}{另外}{ling4wai4}
    \definition{adv./pron.}{além disso}
  \end{phonetics}
\end{entry}

\begin{entry}{只}{5}[Radical 口]
  \begin{phonetics}{只}{zhi1}
    \definition{clas.}{para pássaros, gatos, cãezinhos, etc.}
  \end{phonetics}
  \begin{phonetics}{只}{zhi3}
    \definition{adv.}{apenas | só}
  \end{phonetics}
\end{entry}

\begin{entry}{只好}{5,6}
  \begin{phonetics}{只好}{zhi3hao3}
    \definition{adv.}{ser forçado a | ter que | sem nenhuma opção melhor | não ter outro remédio senão}
  \end{phonetics}
\end{entry}

\begin{entry}{只有……才……}{5,6,3}
  \begin{phonetics}{只有……才……}{zhi3you3 cai2}
    \definition{conj.}{só se\dots então\dots}
  \end{phonetics}
\end{entry}

\begin{entry}{只身}{5,7}
  \begin{phonetics}{只身}{zhi1shen1}
    \definition{adv.}{sozinho | por si só}
  \end{phonetics}
\end{entry}

\begin{entry}{只怕}{5,8}
  \begin{phonetics}{只怕}{zhi3pa4}
    \definition{adv.}{receio que\dots | talvez | muito provavelmente}
  \end{phonetics}
\end{entry}

\begin{entry}{只要}{5,9}
  \begin{phonetics}{只要}{zhi3yao4}[][HSK 2]
    \definition{conj.}{se apenas | contanto que}
  \end{phonetics}
\end{entry}

\begin{entry}{只要……就……}{5,9,12}
  \begin{phonetics}{只要……就……}{zhi3yao4 jiu4}
    \definition{conj.}{contanto que/desde que/se somente\dots, então\dots}
  \end{phonetics}
\end{entry}

\begin{entry}{只消}{5,10}
  \begin{phonetics}{只消}{zhi3xiao1}
    \definition{conj.}{desde que}
  \end{phonetics}
\end{entry}

\begin{entry}{只能}{5,10}
  \begin{phonetics}{只能}{zhi3 neng2}[][HSK 2]
    \definition{adv.}{só pode | obrigado a fazer algo}
  \end{phonetics}
\end{entry}

\begin{entry}{只读}{5,10}
  \begin{phonetics}{只读}{zhi3du2}
    \definition{s.}{somente leitura (computação) | \emph{read-only}}
  \end{phonetics}
\end{entry}

\begin{entry}{只顾}{5,10}
  \begin{phonetics}{只顾}{zhi3gu4}
    \definition{adv.}{exclusivamente preocupado (com uma coisa)}
    \definition{v.}{cuidar de apenas um aspecto}
  \end{phonetics}
\end{entry}

\begin{entry}{只得}{5,11}
  \begin{phonetics}{只得}{zhi3de5}
    \definition{v.}{ser obrigado a | não ter outra alternativa senão}
  \end{phonetics}
\end{entry}

\begin{entry}{叫}{5}[Radical 口]
  \begin{phonetics}{叫}{jiao4}[][HSK 1]
    \definition{v.}{ser chamado | chamar-se | pedir | por (indica agente na voz passiva) | chamar | gritar | ordenar}
  \end{phonetics}
\end{entry}

\begin{entry}{叫作}{5,7}
  \begin{phonetics}{叫作}{jiao4 zuo4}[][HSK 2]
    \definition{v.}{ser chamado de | ser conhecido como}
  \end{phonetics}
\end{entry}

\begin{entry}{叮嘱}{5,15}
  \begin{phonetics}{叮嘱}{ding1zhu3}
    \definition{v.}{exortar | avisar | insistir de novo e de novo}
  \end{phonetics}
\end{entry}

\begin{entry}{可}{5}[Radical 口]
  \begin{phonetics}{可}{ke3}
    \definition{adv.}{muito | realmente}
  \end{phonetics}
\end{entry}

\begin{entry}{可口可乐}{5,3,5,5}
  \begin{phonetics}{可口可乐}{ke3kou3ke3le4}
    \definition*{s.}{(empréstimo linguístico) Coca-Cola}
  \end{phonetics}
\end{entry}

\begin{entry}{可以}{5,4}
  \begin{phonetics}{可以}{ke3yi3}[][HSK 2]
    \definition{v.}{ser capaz de | poder}
  \end{phonetics}
\end{entry}

\begin{entry}{可卡因}{5,5,6}
  \begin{phonetics}{可卡因}{ke3ka3yin1}
    \definition{s.}{(empréstimo linguístico) cocaína}
  \end{phonetics}
\end{entry}

\begin{entry}{可怕}{5,8}
  \begin{phonetics}{可怕}{ke3pa4}[][HSK 2]
    \definition{adj.}{horrível | terrível | formidável | assustador | hediondo}
    \definition{adv.}{terrivelmente}
  \end{phonetics}
\end{entry}

\begin{entry}{可是}{5,9}
  \begin{phonetics}{可是}{ke3shi4}[][HSK 2]
    \definition{adv.}{(usado para dar ênfase) de fato}
    \definition{conj.}{porém | contudo | mas}
  \end{phonetics}
\end{entry}

\begin{entry}{可爱}{5,10}
  \begin{phonetics}{可爱}{ke3'ai4}[][HSK 2]
    \definition{adj.}{adorável | querido | fofo}
  \end{phonetics}
\end{entry}

\begin{entry}{可能}{5,10}
  \begin{phonetics}{可能}{ke3neng2}[][HSK 2]
    \definition{adj.}{possível | provável}
    \definition{adv.}{possivelmente | provavelmente}
    \definition[个]{s.}{possibilidade | probabilidade}
  \end{phonetics}
\end{entry}

\begin{entry}{可惜}{5,11}
  \begin{phonetics}{可惜}{ke3xi1}
    \definition{adj.}{é uma pena | que pena}
    \definition{adv.}{infelizmente | que pena | é uma pena}
  \end{phonetics}
\end{entry}

\begin{entry}{可编程}{5,12,12}
  \begin{phonetics}{可编程}{ke3bian1cheng2}
    \definition{adj.}{programável}
  \end{phonetics}
\end{entry}

\begin{entry*}{可擦写可编程只读存储器}{5,17,5,5,12,12,5,10,6,12,16}
  \begin{phonetics}{可擦写可编程只读存储器}{ke3ca1xie3ke3bian1cheng2zhi1du2cun2chu3qi4}
    \definition{s.}{EPROM (\emph{erasable programmable read-only memory})}
  \end{phonetics}
\end{entry*}

\begin{entry}{台}{5}[Radical 口]
  \begin{phonetics}{台}{tai2}
    \definition*{s.}{sobrenome Tai}
    \definition{clas.}{para aparelhos e máquinas}
    \definition{s.}{estação de transmissão | contador | \emph{help desk} | suporte técnico | plataforma | terraço | tufão}
  \end{phonetics}
\end{entry}

\begin{entry}{台下}{5,3}
  \begin{phonetics}{台下}{tai2xia4}
    \definition{s.}{platéia | fora do palco}
  \end{phonetics}
\end{entry}

\begin{entry}{台风}{5,4}
  \begin{phonetics}{台风}{tai2feng1}
    \definition{s.}{tufão}
  \end{phonetics}
\end{entry}

\begin{entry}{右}{5}[Radical 口]
  \begin{phonetics}{右}{you4}[][HSK 1]
    \definition{s.}{(política) a Direita}
    \definition{s.}{direita}
  \end{phonetics}
\end{entry}

\begin{entry}{右手}{5,4}
  \begin{phonetics}{右手}{you4shou3}
    \definition{s.}{mão direita | lado direito}
  \end{phonetics}
\end{entry}

\begin{entry}{右边}{5,5}
  \begin{phonetics}{右边}{you4bian5}[][HSK 1]
    \definition{adv.}{à direita | ao lado direito}
  \end{phonetics}
\end{entry}

\begin{entry}{右侧}{5,8}
  \begin{phonetics}{右侧}{you4ce4}
    \definition{s.}{lateral direita | lado direito}
  \end{phonetics}
\end{entry}

\begin{entry}{右转}{5,8}
  \begin{phonetics}{右转}{you4zhuan3}
    \definition{v.}{virar à direita}
  \end{phonetics}
\end{entry}

\begin{entry}{右面}{5,9}
  \begin{phonetics}{右面}{you4mian4}
    \definition{s.}{lado direito}
  \end{phonetics}
\end{entry}

\begin{entry}{右倾}{5,10}
  \begin{phonetics}{右倾}{you4qing1}
    \definition{adj.}{conservador | reacionário}
  \end{phonetics}
\end{entry}

\begin{entry}{右袒}{5,10}
  \begin{phonetics}{右袒}{you4tan3}
    \definition{v.}{ser tendencioso | ser parcial | favorecer um lado | tomar partido}
  \end{phonetics}
\end{entry}

\begin{entry}{号}{5}[Radical 口]
  \begin{phonetics}{号}{hao2}
    \definition[个]{s.}{rugido | choro}
  \end{phonetics}
  \begin{phonetics}{号}{hao4}
    \definition{clas.}{para indicar o número de pessoas}
    \definition{num.}{dia do mês | usado para indicar o número de pessoas}
    \definition[个]{s.}{número ordinal | dia de um mês | marca | sinal | estabelecimento comercial | tamanho | buzina (instrumento de sopro) | toque de corneta | nome suposto}
    \definition{suf.}{sufixo de navio}
    \definition{v.}{tomar um pulso}
  \end{phonetics}
\end{entry}

\begin{entry}{号角}{5,7}
  \begin{phonetics}{号角}{hao4jiao3}
    \definition{s.}{corneta | trombeta}
  \end{phonetics}
\end{entry}

\begin{entry}{号码}{5,8}
  \begin{phonetics}{号码}{hao4ma3}
    \definition[堆,个]{s.}{número}
  \end{phonetics}
\end{entry}

\begin{entry}{司机}{5,6}
  \begin{phonetics}{司机}{si1ji1}[][HSK 2]
    \definition{s.}{condutor | motorista | chofer}
  \end{phonetics}
\end{entry}

\begin{entry}{囘}{5}[Radical 囗]
  \begin{phonetics}{囘}{hui2}
    \variantof{回}
  \end{phonetics}
\end{entry}

\begin{entry}{四}{5}[Radical 囗]
  \begin{phonetics}{四}{si4}[][HSK 1]
    \definition{num.}{quatro; 4}
  \end{phonetics}
\end{entry}

\begin{entry}{四川}{5,3}
  \begin{phonetics}{四川}{si4chuan1}
    \definition*{s.}{Sichuan}
  \end{phonetics}
\end{entry}

\begin{entry}{四季分明}{5,8,4,8}
  \begin{phonetics}{四季分明}{si4ji4-fen1ming2}
    \definition{expr.}{as quatro estações são muito distintas}
  \end{phonetics}
\end{entry}

\begin{entry}{四季如春}{5,8,6,9}
  \begin{phonetics}{四季如春}{si4ji4-ru2chun1}
    \definition{expr.}{é primavera todo o ano | clima favorável durante todo o ano | quatro estações como a primavera}
  \end{phonetics}
\end{entry}

\begin{entry}{圣地}{5,6}
  \begin{phonetics}{圣地}{sheng4di4}
    \definition{s.}{terra santa (de uma religião) | lugar sagrado | santuário | cidade santa (como Jerusalém, Meca, etc.) | centro de interesse histórico}
  \end{phonetics}
\end{entry}

\begin{entry}{圣诞节}{5,8,5}
  \begin{phonetics}{圣诞节}{sheng4dan4jie2}
    \definition*{s.}{Natal}
  \end{phonetics}
\end{entry}

\begin{entry}{处}{5}[Radical ⼡]
  \begin{phonetics}{处}{chu3}
    \definition{v.}{residir | viver | habitar | estar dentro | estar situado em | ficar | se dar bem com | estar em uma posição de | lidar com | disciplinar | punir}
  \end{phonetics}
  \begin{phonetics}{处}{chu4}
    \definition{clas.}{para locais ou itens de danos: lugar, local}
    \definition{s.}{local | localização | lugar | ponto | escritório | departamento}
  \end{phonetics}
\end{entry}

\begin{entry}{处处}{5,5}
  \begin{phonetics}{处处}{chu4chu4}
    \definition{adv.}{em todos os lugares | em todos os aspectos}
  \end{phonetics}
\end{entry}

\begin{entry}{处罚}{5,9}
  \begin{phonetics}{处罚}{chu3fa2}
    \definition{v.}{penalizar | punir}
  \end{phonetics}
\end{entry}

\begin{entry}{处理}{5,11}
  \begin{phonetics}{处理}{chu3li3}[][HSK 3]
    \definition{s.}{manuseio; descarte}
    \definition{v.}{lidar com; dispor de | resolver; punir; lidar | vender a preços reduzidos; liquidar | lidar com; processar}
  \end{phonetics}
\end{entry}

\begin{entry}{外}{5}[Radical 夕]
  \begin{phonetics}{外}{wai4}[][HSK 1]
    \definition{s.}{fora | por fora | exterior | estrangeiro}
  \end{phonetics}
\end{entry}

\begin{entry}{外公}{5,4}
  \begin{phonetics}{外公}{wai4gong1}
    \definition{s.}{avô materno}
  \end{phonetics}
\end{entry}

\begin{entry}{外水}{5,4}
  \begin{phonetics}{外水}{wai4shui3}
    \definition{s.}{renda extra}
  \end{phonetics}
\end{entry}

\begin{entry}{外号}{5,5}
  \begin{phonetics}{外号}{wai4hao4}
    \definition{s.}{apelido}
  \end{phonetics}
\end{entry}

\begin{entry}{外边}{5,5}
  \begin{phonetics}{外边}{wai4bian5}[][HSK 1]
    \definition{adv.}{fora do país | superfície externa | fora | lugar diferente de sua casa}
  \end{phonetics}
\end{entry}

\begin{entry}{外交}{5,6}
  \begin{phonetics}{外交}{wai4jiao1}
    \definition{adj.}{diplomático}
    \definition[个]{s.}{diplomacia | relações exteriores}
  \end{phonetics}
\end{entry}

\begin{entry}{外协}{5,6}
  \begin{phonetics}{外协}{wai4xie2}
    \definition{s.}{terceirização | pessoas que julgam os outros pela aparência}
  \seealsoref{外貌协会}{wai4mao4xie2hui4}
  \end{phonetics}
\end{entry}

\begin{entry}{外地}{5,6}
  \begin{phonetics}{外地}{wai4 di4}[][HSK 2]
    \definition{s.}{não local | outros lugares}
  \end{phonetics}
\end{entry}

\begin{entry}{外孙}{5,6}
  \begin{phonetics}{外孙}{wai4sun1}
    \definition{s.}{filho da filha}
  \end{phonetics}
\end{entry}

\begin{entry}{外孙女}{5,6,3}
  \begin{phonetics}{外孙女}{wai4sun1nv3}
    \definition{s.}{filha da filha}
  \end{phonetics}
\end{entry}

\begin{entry}{外衣}{5,6}
  \begin{phonetics}{外衣}{wai4yi1}
    \definition{s.}{aparência | roupa de cima}
  \end{phonetics}
\end{entry}

\begin{entry}{外围}{5,7}
  \begin{phonetics}{外围}{wai4wei2}
    \definition{adv.}{arredores}
  \end{phonetics}
\end{entry}

\begin{entry}{外事}{5,8}
  \begin{phonetics}{外事}{wai4shi4}
    \definition{s.}{assuntos ou relações exteriores}
  \end{phonetics}
\end{entry}

\begin{entry}{外卖}{5,8}
  \begin{phonetics}{外卖}{wai4 mai4}[][HSK 2]
    \definition{s.}{para viagem | para fora}
    \definition{v.}{entregar | oferecer}
  \end{phonetics}
\end{entry}

\begin{entry}{外国}{5,8}
  \begin{phonetics}{外国}{wai4guo2}[][HSK 1]
    \definition[个]{s.}{país estrangeiro}
  \end{phonetics}
\end{entry}

\begin{entry}{外国人}{5,8,2}
  \begin{phonetics}{外国人}{wai4guo2ren2}
    \definition{s.}{estrangeiro | pessoa de fora do país}
  \end{phonetics}
\end{entry}

\begin{entry}{外语}{5,9}
  \begin{phonetics}{外语}{wai4yu3}[][HSK 1]
    \definition[门]{s.}{língua estrangeira}
  \end{phonetics}
\end{entry}

\begin{entry}{外贸}{5,9}
  \begin{phonetics}{外贸}{wai4mao4}
    \definition{s.}{comércio exterior}
  \end{phonetics}
\end{entry}

\begin{entry}{外面}{5,9}
  \begin{phonetics}{外面}{wai4mian4}
    \definition{adv.}{fora | por fora | exterior | superfície}
  \end{phonetics}
\end{entry}

\begin{entry}{外海}{5,10}
  \begin{phonetics}{外海}{wai4hai3}
    \definition{s.}{mar aberto}
  \end{phonetics}
\end{entry}

\begin{entry}{外积}{5,10}
  \begin{phonetics}{外积}{wai4ji1}
    \definition{s.}{produto exterior | (matemática) o produto vetorial de dois vetores}
  \end{phonetics}
\end{entry}

\begin{entry}{外婆}{5,11}
  \begin{phonetics}{外婆}{wai4po2}
    \definition{s.}{avó materna}
  \end{phonetics}
\end{entry}

\begin{entry}{外插}{5,12}
  \begin{phonetics}{外插}{wai4cha1}
    \definition{v.}{extrapolar | (computação) conectar (um dispositivo periférico, etc.)}
  \end{phonetics}
\end{entry}

\begin{entry}{外貌协会}{5,14,6,6}
  \begin{phonetics}{外貌协会}{wai4mao4xie2hui4}
    \definition{s.}{``o clube da boa aparência'': pessoas que dão grande importância à aparência de uma pessoa}
  \seealsoref{外协}{wai4xie2}
  \end{phonetics}
\end{entry}

\begin{entry}{失去}{5,5}
  \begin{phonetics}{失去}{shi1qu4}
    \definition{v.}{perder}
  \end{phonetics}
\end{entry}

\begin{entry}{失眠}{5,10}
  \begin{phonetics}{失眠}{shi1mian2}
    \definition{s.}{insônia}
    \definition{v.}{ter insônia}
  \end{phonetics}
\end{entry}

\begin{entry}{失望}{5,11}
  \begin{phonetics}{失望}{shi1wang4}
    \definition{adj.}{desapontado}
    \definition{v.}{perder a esperança | desesperar}
  \end{phonetics}
\end{entry}

\begin{entry}{失落}{5,12}
  \begin{phonetics}{失落}{shi1luo4}
    \definition{s.}{frustração | decepção | perda}
    \definition{v.}{perder (algo) | cair (algo) | sentir uma sensação de perda}
  \end{phonetics}
\end{entry}

\begin{entry}{失意}{5,13}
  \begin{phonetics}{失意}{shi1yi4}
    \definition{adj.}{desapontado | frustrado}
  \end{phonetics}
\end{entry}

\begin{entry}{头}{5}[Radical 大]
  \begin{phonetics}{头}{tou2}[][HSK 2]
    \definition{clas.}{para suínos ou gado}
    \definition[个]{s.}{cabeça}
  \end{phonetics}
  \begin{phonetics}{头}{tou5}[][HSK 0]
    \definition{suf.}{sufixo para nomes}
  \end{phonetics}
\end{entry}

\begin{entry}{头发}{5,5}
  \begin{phonetics}{头发}{tou2fa5}[][HSK 2]
    \definition{s.}{cabelo}
  \end{phonetics}
\end{entry}

\begin{entry}{头号}{5,5}
  \begin{phonetics}{头号}{tou2hao4}
    \definition{adj.}{primeira classe | número um | \emph{top rank}}
  \end{phonetics}
\end{entry}

\begin{entry}{头头}{5,5}
  \begin{phonetics}{头头}{tou2tou2}
    \definition{s.}{chefe | o cabeça}
  \end{phonetics}
\end{entry}

\begin{entry}{头脑风暴}{5,10,4,15}
  \begin{phonetics}{头脑风暴}{tou2nao3feng1bao4}
    \definition{s.}{\emph{brainstorm}}
  \end{phonetics}
\end{entry}

\begin{entry}{头像}{5,13}
  \begin{phonetics}{头像}{tou2xiang4}
    \definition{s.}{retrato | busto | avatar | imagem de perfil (computação)}
  \end{phonetics}
\end{entry}

\begin{entry}{奶}{5}[Radical 女]
  \begin{phonetics}{奶}{nai3}[][HSK 1]
    \definition[杯,滴,瓶,只,桶]{s.}{seios | leite}
    \definition{v.}{amamentar}
  \end{phonetics}
\end{entry}

\begin{entry}{奶奶}{5,5}
  \begin{phonetics}{奶奶}{nai3nai5}[][HSK 1]
    \definition[位]{s.}{avó (paterna) | (respeitoso) dona da casa}
  \end{phonetics}
\end{entry}

\begin{entry}{宁}{5}[Radical 宀]
  \begin{phonetics}{宁}{ning2}
    \definition*{s.}{sobrenome Ning}
    \definition{adj.}{calmo, pacífico, sereno | saudável}
  \end{phonetics}
  \begin{phonetics}{宁}{ning4}
    \definition{conj.}{mais\dots do que\dots, melhor\dots do que\dots}
  \end{phonetics}
\end{entry}

\begin{entry}{宁可}{5,5}
  \begin{phonetics}{宁可}{ning4ke3}
    \definition{conj.}{mais\dots do que\dots | melhor\dots do que\dots}
  \end{phonetics}
\end{entry}

\begin{entry}{宁可……也不……}{5,5,3,4}
  \begin{phonetics}{宁可……也不……}{ning4ke3 ye3bu4}
    \definition{conj.}{em vez de\dots}
  \end{phonetics}
\end{entry}

\begin{entry}{宁可……也要……}{5,5,3,9}
  \begin{phonetics}{宁可……也要……}{ning4ke3 ye3yao4}
    \definition{conj.}{mesmo que tenhamos que\dots nós iremos\dots}
  \end{phonetics}
\end{entry}

\begin{entry}{宁肯}{5,8}
  \begin{phonetics}{宁肯}{ning4ken3}
    \definition{conj.}{mais\dots do que\dots, melhor\dots do que\dots}
  \end{phonetics}
\end{entry}

\begin{entry}{宁愿}{5,14}
  \begin{phonetics}{宁愿}{ning4yuan4}
    \definition{conj.}{mais\dots do que\dots, melhor\dots do que\dots}
  \end{phonetics}
\end{entry}

\begin{entry}{它}{5}[Radical 宀]
  \begin{phonetics}{它}{ta1}[][HSK 2]
    \definition{pron.}{ele (para objetos inanimados) | se, o, lhe | si, consigo, eles}
  \end{phonetics}
\end{entry}

\begin{entry}{它们}{5,5}
  \begin{phonetics}{它们}{ta1 men5}[][HSK 2]
    \definition{pron.}{eles (para objetos inanimados) | se, os, lhes | si, consigo, eles}
  \end{phonetics}
\end{entry}

\begin{entry}{对}{5}[Radical ⼨]
  \begin{phonetics}{对}{dui4}[][HSK 1]
    \definition{adj.}{correto | sim}
    \definition{clas.}{para casais}
    \definition{prep.}{com | para | para com}
  \end{phonetics}
\end{entry}

\begin{entry}{对不起}{5,4,10}
  \begin{phonetics}{对不起}{dui4bu5qi3}[][HSK 1]
    \definition{interj.}{Desculpe! | Desculpe-me! | Perdoe-me! | Desculpe? (por favor, repita)}
    \definition{v.}{desculpar | pedir desculpas | perdoar}
  \end{phonetics}
\end{entry}

\begin{entry}{对手}{5,4}
  \begin{phonetics}{对手}{dui4shou3}[][HSK 3]
    \definition[个]{s.}{oponente; adversário}
  \end{phonetics}
\end{entry}

\begin{entry}{对方}{5,4}
  \begin{phonetics}{对方}{dui4fang1}[][HSK 3]
    \definition{s.}{outro lado; lado oposto; outra parte}
  \end{phonetics}
\end{entry}

\begin{entry}{对……有兴趣}{5,6,6,15}
  \begin{phonetics}{对……有兴趣}{dui4 you3xing4qu4}
    \definition{expr.}{estar interessado em\dots | ter interesse em\dots | interessar-se por\dots}
    \seeref{对……感兴趣}{dui4 gan3xing4qu4}
  \end{phonetics}
\end{entry}

\begin{entry}{对话}{5,8}
  \begin{phonetics}{对话}{dui4hua4}[][HSK 2]
    \definition[个]{s.}{diálogo | conversa}
    \definition{v.}{dialogar | conversar}
  \end{phonetics}
\end{entry}

\begin{entry}{对待}{5,9}
  \begin{phonetics}{对待}{dui4dai4}[][HSK 3]
    \definition{v.}{tratar; abordar; manusear; estar em uma posição relacionada ou comparada a outra}
  \end{phonetics}
\end{entry}

\begin{entry}{对……说}{5,9}
  \begin{phonetics}{对……说}{dui4 shuo5}
    \definition{v.}{dizer a alguém}
  \end{phonetics}
\end{entry}

\begin{entry}{对面}{5,9}
  \begin{phonetics}{对面}{dui4mian4}[][HSK 2]
    \definition{s.}{lado oposto}
  \end{phonetics}
\end{entry}

\begin{entry}{对得起}{5,11,10}
  \begin{phonetics}{对得起}{dui4de5qi3}
    \definition{v.}{não decepcionar alguém | tratar alguém de maneira justa | ser digno de}
  \end{phonetics}
\end{entry}

\begin{entry}{对象}{5,11}
  \begin{phonetics}{对象}{dui4xiang4}[][HSK 3]
    \definition[个]{s.}{alvo; objeto | parceiro; namorado; namorada}
  \end{phonetics}
\end{entry}

\begin{entry}{对……感兴趣}{5,13,6,15}
  \begin{phonetics}{对……感兴趣}{dui4 gan3xing4qu4}
    \definition{expr.}{estar interessado em\dots | ter interesse em\dots | interessar-se por\dots}
    \seeref{对……有兴趣}{dui4 you3xing4qu4}
  \end{phonetics}
\end{entry}

\begin{entry}{对……熟悉}{5,15,11}
  \begin{phonetics}{对……熟悉}{dui4 shu2xi1}
    \definition{expr.}{estar familiarizado com\dots}
  \end{phonetics}
\end{entry}

\begin{entry}{左}{5}[Radical 工]
  \begin{phonetics}{左}{zuo3}[][HSK 1]
    \definition*{s.}{sobrenome Zuo}
    \definition{s.}{esquerda}
  \end{phonetics}
\end{entry}

\begin{entry}{左右}{5,5}
  \begin{phonetics}{左右}{zuo3you4}
    \definition{adv.}{cerca de | aproximadamente}
  \end{phonetics}
\end{entry}

\begin{entry}{左边}{5,5}
  \begin{phonetics}{左边}{zuo3bian5}[][HSK 1]
    \definition{s.}{esquerda | lado esquerdo}
  \end{phonetics}
\end{entry}

\begin{entry}{左派}{5,9}
  \begin{phonetics}{左派}{zuo3pai4}
    \definition{s.}{(política) esquerda | esquerdista}
  \end{phonetics}
\end{entry}

\begin{entry}{左面}{5,9}
  \begin{phonetics}{左面}{zuo3mian4}
    \definition{s.}{esquerda | lado esquerdo}
  \end{phonetics}
\end{entry}

\begin{entry}{左倾}{5,10}
  \begin{phonetics}{左倾}{zuo3qing1}
    \definition{s.}{esquerdista | progressivo}
  \end{phonetics}
\end{entry}

\begin{entry}{左袒}{5,10}
  \begin{phonetics}{左袒}{zuo3tan3}
    \definition{v.}{ser tendencioso | ser parcial para | favorecer um lado | tomar partido com}
  \end{phonetics}
\end{entry}

\begin{entry}{左舷}{5,11}
  \begin{phonetics}{左舷}{zuo3xian2}
    \definition{s.}{porto (lado de um navio)}
  \end{phonetics}
\end{entry}

\begin{entry}{左翼}{5,17}
  \begin{phonetics}{左翼}{zuo3yi4}
    \definition{s.}{esquerda (política)}
  \end{phonetics}
\end{entry}

\begin{entry}{巧合}{5,6}
  \begin{phonetics}{巧合}{qiao3he2}
    \definition{s.}{coincidência}
    \definition{v.}{coincidir}
  \end{phonetics}
\end{entry}

\begin{entry}{巧克力}{5,7,2}
  \begin{phonetics}{巧克力}{qiao3ke4li4}
    \definition[块]{s.}{(empréstimo linguístico) chocolate}
  \end{phonetics}
\end{entry}

\begin{entry}{市}{5}[Radical 巾]
  \begin{phonetics}{市}{shi4}[][HSK 2]
    \definition*{s.}{sobrenome Shi}
    \definition{s.}{mercado | cidade | município | referente ao sistema chinês de pesos e medidas}
    \definition{v.}{comprar | vender | negociar}
  \end{phonetics}
\end{entry}

\begin{entry}{市中心}{5,4,4}
  \begin{phonetics}{市中心}{shi4zhong1xin1}
    \definition{s.}{centro da cidade}
  \end{phonetics}
\end{entry}

\begin{entry}{市区}{5,4}
  \begin{phonetics}{市区}{shi4qu1}
    \definition{s.}{centro da cidade | distrito urbano}
  \end{phonetics}
\end{entry}

\begin{entry}{市长}{5,4}
  \begin{phonetics}{市长}{shi4 zhang3}[][HSK 2]
    \definition[个]{s.}{prefeito}
  \end{phonetics}
\end{entry}

\begin{entry}{市场}{5,6}
  \begin{phonetics}{市场}{shi4chang3}
    \definition{s.}{mercado (também no abstrato)}
  \end{phonetics}
\end{entry}

\begin{entry}{布}{5}[Radical 巾]
  \begin{phonetics}{布}{bu4}[][HSK 3]
    \definition*{s.}{sobrenome Bu}
    \definition[块,幅,匹]{s.}{pano | tecido | uma moeda de cobre nos tempos antigos}
    \definition{v.}{anunciar | declarar | tornar conhecido | proclamar | publicar | espalhar | disseminar |organizar | implantar | dispor}
  \end{phonetics}
\end{entry}

\begin{entry}{布谷鸟}{5,7,5}
  \begin{phonetics}{布谷鸟}{bu4gu3niao3}
    \definition{s.}{cuco (pássaro)}
  \seealsoref{杜鹃}{du4juan1}
  \seealsoref{杜鹃鸟}{du4juan1niao3}
  \seealsoref{杜宇}{du4yu3}
  \end{phonetics}
\end{entry}

\begin{entry}{布署}{5,13}
  \begin{phonetics}{布署}{bu4shu3}
    \variantof{部署}
  \end{phonetics}
\end{entry}

\begin{entry}{帅}{5}[Radical 巾]
  \begin{phonetics}{帅}{shuai4}
    \definition*{s.}{sobrenome Shuai}
    \definition{adj.}{elegante | agradável à vista | gracioso | inteligente}
    \definition{interj.}{Legal!}
    \definition{s.}{comandante em chefe}
  \end{phonetics}
\end{entry}

\begin{entry}{平}{5}[Radical 干]
  \begin{phonetics}{平}{ping2}[][HSK 2]
    \definition*{s.}{sobrenome Ping}
    \definition{adj.}{calmo | pacífico}
    \definition{s.}{plano | nível}
    \definition{v.}{fazer a mesma pontuação | marcar uma pontuação}
  \end{phonetics}
\end{entry}

\begin{entry}{平台}{5,5}
  \begin{phonetics}{平台}{ping2tai2}
    \definition{s.}{plataforma | terraço | edifício de telhado plano}
  \end{phonetics}
\end{entry}

\begin{entry}{平地}{5,6}
  \begin{phonetics}{平地}{ping2di4}
    \definition{v.}{nivelar a terra | aplanar}
  \end{phonetics}
\end{entry}

\begin{entry}{平安}{5,6}
  \begin{phonetics}{平安}{ping2'an1}[][HSK 2]
    \definition{s.}{seguro | bem | sem contratempos | são e salvo}
  \end{phonetics}
\end{entry}

\begin{entry}{平时}{5,7}
  \begin{phonetics}{平时}{ping2shi2}[][HSK 2]
    \definition{adv.}{normalmente | em tempos normais | em tempos de paz}
  \end{phonetics}
\end{entry}

\begin{entry}{平常}{5,11}
  \begin{phonetics}{平常}{ping2chang2}[][HSK 2]
    \definition{adj.}{comum | ordinário | usual}
    \definition{adv.}{usualmente | geralmente | ordinariamente | como regra}
  \end{phonetics}
\end{entry}

\begin{entry}{平等}{5,12}
  \begin{phonetics}{平等}{ping2deng3}[][HSK 2]
    \definition{adj.}{igual | igualdade}
  \end{phonetics}
\end{entry}

\begin{entry}{幼儿园}{5,2,7}
  \begin{phonetics}{幼儿园}{you4'er2yuan2}
    \definition{s.}{jardim de infância | berçário}
  \end{phonetics}
\end{entry}

\begin{entry}{归}{5}[Radical ⼹]
  \begin{phonetics}{归}{gui1}
    \definition*{s.}{sobrenome Gui}
    \definition{s.}{divisão no ábaco com divisor de um dígito}
    \definition{v.}{retornar | voltar a | retribuir a | (uma responsabilidade) a ser resolvido por | pertencer | reunir-se | (usado entre dois verbos idênticos) apesar}
  \end{phonetics}
\end{entry}

\begin{entry}{必定}{5,8}
  \begin{phonetics}{必定}{bi4ding4}
    \definition{adv.}{sem falta | certamente | com certeza | definitivamente | inevitavelmente | com determinação}
    \definition{v.}{estar vinculado a | ter certeza de}
  \end{phonetics}
\end{entry}

\begin{entry}{必要}{5,9}
  \begin{phonetics}{必要}{bi4yao4}[][HSK 3]
    \definition{adj.}{necessário | essencial | indispensável}
    \definition[个,些]{s.}{necessidade}
  \end{phonetics}
\end{entry}

\begin{entry}{必须}{5,9}
  \begin{phonetics}{必须}{bi4xu1}[][HSK 2]
    \definition{adv.}{necessariamente | obrigatoriamente}
  \end{phonetics}
\end{entry}

\begin{entry}{必然}{5,12}
  \begin{phonetics}{必然}{bi4ran2}[][HSK 3]
    \definition{adj.}{certo | inevitável | necessário}
    \definition{adv.}{inevitavelmente}
    \definition{s.}{necessidade}
  \end{phonetics}
\end{entry}

\begin{entry}{扑克}{5,7}
  \begin{phonetics}{扑克}{pu1ke4}
    \definition{s.}{(empréstimo linguístico) (jogo) \emph{poker}  | baralho}
  \end{phonetics}
\end{entry}

\begin{entry}{扒犁}{5,11}
  \begin{phonetics}{扒犁}{pa2li2}
    \definition{s.}{trenó}
    \seeref{爬犁}{pa2li2}
  \end{phonetics}
\end{entry}

\begin{entry}{打}{5}[Radical 手]
  \begin{phonetics}{打}{da2}
    \definition{s.}{(empréstimo linguístico) dúzia}
  \end{phonetics}
  \begin{phonetics}{打}{da3}
    \definition{adv.}{desde}
    \definition{v.}{jogar (um jogo) | bater | atacar | acertar | quebrar | digitar | misturar | construir | lutar | pegar | fazer | amarrar | atirar | calcular}
  \end{phonetics}
\end{entry}

\begin{entry}{打工}{5,3}
  \begin{phonetics}{打工}{da3gong1}[][HSK 2]
    \definition{v.}{(para alunos) ter um emprego fora do horário de aula ou durante as férias | trabalhar em um emprego temporá rio ou casual}
  \end{phonetics}
\end{entry}

\begin{entry}{打工人}{5,3,2}
  \begin{phonetics}{打工人}{da3gong1ren2}
    \definition{s.}{trabalhador}
  \end{phonetics}
\end{entry}

\begin{entry}{打开}{5,4}
  \begin{phonetics}{打开}{da3 kai1}[][HSK 1]
    \definition{v.}{abrir | desdobrar | ligar | avançar | espalhar}
  \end{phonetics}
\end{entry}

\begin{entry}{打车}{5,4}
  \begin{phonetics}{打车}{da3 che1}[][HSK 1]
    \definition{v.}{pegar um táxi | chamar um táxi}
  \end{phonetics}
\end{entry}

\begin{entry}{打印}{5,5}
  \begin{phonetics}{打印}{da3yin4}[][HSK 2]
    \definition{v.}{imprimir}
  \end{phonetics}
\end{entry}

\begin{entry}{打电话}{5,5,8}
  \begin{phonetics}{打电话}{da3 dian4hua4}[][HSK 1]
    \definition{v.}{telefonar | fazer uma chamada telefônica | dar um telefonema}
  \seealsoref{给……打电话}{gei3 da3dian4hua4}
  \end{phonetics}
\end{entry}

\begin{entry}{打压}{5,6}
  \begin{phonetics}{打压}{da3ya1}
    \definition{v.}{reprimir | derrotar}
  \end{phonetics}
\end{entry}

\begin{entry}{打扫}{5,6}
  \begin{phonetics}{打扫}{da3sao3}
    \definition{v.}{varrer | limpar}
  \end{phonetics}
\end{entry}

\begin{entry}{打听}{5,7}
  \begin{phonetics}{打听}{da3ting5}[][HSK 3]
    \definition{v.}{perguntar sobre; indagar sobre; obter uma linha sobre}
  \end{phonetics}
\end{entry}

\begin{entry}{打屁股}{5,7,8}
  \begin{phonetics}{打屁股}{da3pi4gu5}
    \definition{v.}{dar um tapa no bumbum de alguém}
  \end{phonetics}
\end{entry}

\begin{entry}{打扮}{5,7}
  \begin{phonetics}{打扮}{da3ban5}
    \definition{v.}{arranjar-se | enfeitar-se}
  \end{phonetics}
\end{entry}

\begin{entry}{打扰}{5,7}
  \begin{phonetics}{打扰}{da3rao3}
    \definition{v.}{perturbar | incomodar}
  \end{phonetics}
\end{entry}

\begin{entry}{打针}{5,7}
  \begin{phonetics}{打针}{da3zhen1}
    \definition{v.+compl.}{dar injeção | levar injeção}
  \end{phonetics}
\end{entry}

\begin{entry}{打的}{5,8}
  \begin{phonetics}{打的}{da3di1}
    \definition{v.+compl.}{(coloquial) pegar um táxi | ir de táxi}
  \end{phonetics}
\end{entry}

\begin{entry}{打架}{5,9}
  \begin{phonetics}{打架}{da3jia4}
    \definition{v.+compl.}{lutar | brigar | participar de lutas, brigas}
  \end{phonetics}
\end{entry}

\begin{entry}{打结}{5,9}
  \begin{phonetics}{打结}{da3jie2}
    \definition{v.}{dar um nó | amarrar}
  \end{phonetics}
\end{entry}

\begin{entry}{打骂}{5,9}
  \begin{phonetics}{打骂}{da3ma4}
    \definition{v.}{bater e repreender}
  \end{phonetics}
\end{entry}

\begin{entry}{打破}{5,10}
  \begin{phonetics}{打破}{da3 po4}[][HSK 3]
    \definition{v.}{quebrar; esmagar}
  \end{phonetics}
\end{entry}

\begin{entry}{打猎}{5,11}
  \begin{phonetics}{打猎}{da3lie4}
    \definition{v.}{ir caçar}
  \end{phonetics}
\end{entry}

\begin{entry}{打球}{5,11}
  \begin{phonetics}{打球}{da3 qiu2}[][HSK 1]
    \definition{v.}{jogar bola (com as mãos) | jogar (basquetebol, handbol, etc.)}
  \end{phonetics}
\end{entry}

\begin{entry}{打搅}{5,12}
  \begin{phonetics}{打搅}{da3jiao3}
    \definition{v.}{perturbar | incomodar}
  \end{phonetics}
\end{entry}

\begin{entry}{打算}{5,14}
  \begin{phonetics}{打算}{da3suan4}[][HSK 2]
    \definition[个]{s.}{plano | intenção}
    \definition{v.}{pensar | planejar | pretender}
  \end{phonetics}
\end{entry}

\begin{entry}{打瞌睡}{5,15,13}
  \begin{phonetics}{打瞌睡}{da3ke1shui4}
    \definition{v.}{cochilar}
  \end{phonetics}
\end{entry}

\begin{entry}{打磨}{5,16}
  \begin{phonetics}{打磨}{da3mo2}
    \definition{v.}{polir | fazer brilhar}
  \end{phonetics}
\end{entry}

\begin{entry}{扔}{5}[Radical 手]
  \begin{phonetics}{扔}{reng1}
    \definition{v.}{lançar | atirar}
  \end{phonetics}
\end{entry}

\begin{entry}{扔下}{5,3}
  \begin{phonetics}{扔下}{reng1xia4}
    \definition{v.}{lançar (uma bomba) | derrubar}
  \end{phonetics}
\end{entry}

\begin{entry}{扔弃}{5,7}
  \begin{phonetics}{扔弃}{reng1qi4}
    \definition{v.}{abandonar | descartar | jogar fora}
  \end{phonetics}
\end{entry}

\begin{entry}{扔掉}{5,11}
  \begin{phonetics}{扔掉}{reng1diao4}
    \definition{v.}{jogar fora}
  \end{phonetics}
\end{entry}

\begin{entry}{斥骂}{5,9}
  \begin{phonetics}{斥骂}{chi4ma4}
    \definition{v.}{repreender}
  \end{phonetics}
\end{entry}

\begin{entry}{旧}{5}[Radical 日]
  \begin{phonetics}{旧}{jiu4}
    \definition{adj.}{velho | antigo | desgastado (com a idade)}
  \end{phonetics}
\end{entry}

\begin{entry}{未}{5}[Radical 木]
  \begin{phonetics}{未}{wei4}
    \definition{adv.}{não ter | ainda não}
  \end{phonetics}
\end{entry}

\begin{entry}{未必}{5,5}
  \begin{phonetics}{未必}{wei4bi4}
    \definition{adv.}{não pode | não necessariamente}
  \end{phonetics}
\end{entry}

\begin{entry}{本}{5}[Radical 木]
  \begin{phonetics}{本}{ben3}[][HSK 1]
    \definition{adj.}{o atual | original | inerente}
    \definition{adv.}{originalmente}
    \definition{clas.}{para livros, dicionários, periódicos, arquivos, etc.}
    \definition{s.}{raiz | caule | origem | fonte}
  \end{phonetics}
\end{entry}

\begin{entry}{本子}{5,3}
  \begin{phonetics}{本子}{ben3zi5}[][HSK 1]
    \definition[本]{s.}{caderno}
  \end{phonetics}
\end{entry}

\begin{entry}{本来}{5,7}
  \begin{phonetics}{本来}{ben3lai2}[][HSK 3]
    \definition{adv.}{originalmente | apropriadamente | legalmente}
  \end{phonetics}
\end{entry}

\begin{entry}{本事}{5,8}
  \begin{phonetics}{本事}{ben3 shi4}[][HSK 3]
    \definition{s.}{habilidade | capacidade | \emph{status} | poder | posição | autoridade}
  \end{phonetics}
  \begin{phonetics}{本事}{ben3 shi5}[][HSK 3]
    \definition{s.}{habilidade | capacidade |\emph{status} | poder | posição | autoridade}
  \end{phonetics}
\end{entry}

\begin{entry}{本领}{5,11}
  \begin{phonetics}{本领}{ben3 ling3}[][HSK 3]
    \definition[项,个]{s.}{capacidade | faculdade | poder | habilidade | talento}
  \end{phonetics}
\end{entry}

\begin{entry}{正}{5}[Radical 止]
  \begin{phonetics}{正}{zheng1}[][HSK 0]
    \definition{s.}{primeiro mês do ano lunar}
  \end{phonetics}
  \begin{phonetics}{正}{zheng4}[][HSK 1]
    \definition{adj.}{reto | vertical | adequado | principal | (matemática) positivo}
    \definition{adv.}{agora mesmo | no processo de}
    \definition{v.}{corrigir | retificar}
  \end{phonetics}
\end{entry}

\begin{entry}{正正}{5,5}
  \begin{phonetics}{正正}{zheng4zheng4}
    \definition{adv.}{na hora certa | ordenadamente}
  \end{phonetics}
\end{entry}

\begin{entry}{正在}{5,6}
  \begin{phonetics}{正在}{zheng4zai4}[][HSK 1]
    \definition{adv.}{no processo de | atualmente | em andamento}
    \definition{v.}{estar a~+~v.inf. | estar~+~v.ger.}
  \end{phonetics}
\end{entry}

\begin{entry}{正好}{5,6}
  \begin{phonetics}{正好}{zheng4hao3}[][HSK 2]
    \definition{adj.}{na medida certa | na hora certa | o suficiente}
    \definition{adv.}{acontecer com | chance de | como acontece}
  \end{phonetics}
\end{entry}

\begin{entry}{正宗}{5,8}
  \begin{phonetics}{正宗}{zheng4zong1}
    \definition{adj.}{autêntico | genuíno | \emph{old school} | (fig.) tradicional}
  \end{phonetics}
\end{entry}

\begin{entry}{正是}{5,9}
  \begin{phonetics}{正是}{zheng4 shi4}[][HSK 2]
    \definition{adv.}{precisamente | exatamente}
  \end{phonetics}
\end{entry}

\begin{entry}{正常}{5,11}
  \begin{phonetics}{正常}{zheng4chang2}[][HSK 2]
    \definition{adj.}{regular | normal | ordinário}
  \end{phonetics}
\end{entry}

\begin{entry}{正确}{5,12}
  \begin{phonetics}{正确}{zheng4que4}[][HSK 2]
    \definition{adj.}{correto | certo | próprio}
  \end{phonetics}
\end{entry}

\begin{entry}{母亲}{5,9}
  \begin{phonetics}{母亲}{mu3qin1}
    \definition[个]{s.}{mãe}
  \end{phonetics}
\end{entry}

\begin{entry}{母语}{5,9}
  \begin{phonetics}{母语}{mu3yu3}
    \definition{s.}{língua materna | língua nativa}
  \end{phonetics}
\end{entry}

\begin{entry}{民主}{5,5}
  \begin{phonetics}{民主}{min2zhu3}
    \definition{adj.}{democrático}
    \definition{s.}{democracia}
  \end{phonetics}
\end{entry}

\begin{entry}{民众}{5,6}
  \begin{phonetics}{民众}{min2zhong4}
    \definition{s.}{a população | as massas | as pessoas comuns}
  \end{phonetics}
\end{entry}

\begin{entry}{永不}{5,4}
  \begin{phonetics}{永不}{yong3bu4}
    \definition{adv.}{nunca}
  \end{phonetics}
\end{entry}

\begin{entry}{永远}{5,7}
  \begin{phonetics}{永远}{yong3yuan3}[][HSK 2]
    \definition{adv.}{para sempre, sempre | permanentemente}
  \end{phonetics}
\end{entry}

\begin{entry}{汉}{5}[Radical 水]
  \begin{phonetics}{汉}{han4}
    \definition{s.}{grupo étnico Han | chinês (língua) | dinastia Han (206 a.C.-220d.C.) | homem}
  \end{phonetics}
\end{entry}

\begin{entry}{汉字}{5,6}
  \begin{phonetics}{汉字}{han4zi4}[][HSK 1]
    \definition[个]{s.}{caracter chinês}
  \end{phonetics}
\end{entry}

\begin{entry}{汉服}{5,8}
  \begin{phonetics}{汉服}{han4fu2}
    \definition{s.}{vestido chinês tradicional Han}
  \end{phonetics}
\end{entry}

\begin{entry}{汉语}{5,9}
  \begin{phonetics}{汉语}{han4yu3}[][HSK 1]
    \definition[门]{s.}{língua chinesa, mandarim}
  \end{phonetics}
\end{entry}

\begin{entry}{汉堡王}{5,12,4}
  \begin{phonetics}{汉堡王}{han4bao3wang2}
    \definition*{s.}{Burguer King (restaurante \emph{fast-food})}
  \end{phonetics}
\end{entry}

\begin{entry}{汉堡包}{5,12,5}
  \begin{phonetics}{汉堡包}{han4bao3bao1}
    \definition[个]{s.}{hambúrguer}
  \end{phonetics}
\end{entry}

\begin{entry}{汉葡词典}{5,12,7,8}
  \begin{phonetics}{汉葡词典}{han4-pu2 ci2dian3}
    \definition[部,本]{s.}{dicionário chinês-português}
  \seealsoref{葡汉词典}{pu2-han4 ci2dian3}
  \end{phonetics}
\end{entry}

\begin{entry}{灭火}{5,4}
  \begin{phonetics}{灭火}{mie4huo3}
    \definition{s.}{combate a incêndios}
    \definition{v.}{extinguir um incêndio}
  \end{phonetics}
\end{entry}

\begin{entry}{犯法}{5,8}
  \begin{phonetics}{犯法}{fan4fa3}
    \definition{v.}{violar (quebrar) a lei}
  \end{phonetics}
\end{entry}

\begin{entry}{犯罪}{5,13}
  \begin{phonetics}{犯罪}{fan4zui4}
    \definition{v.+compl.}{cometer  um crime (uma ofensa)}
  \end{phonetics}
\end{entry}

\begin{entry}{玄学}{5,8}
  \begin{phonetics}{玄学}{xuan2xue2}
    \definition{s.}{Escola Philosófica Wei e Jin amalgamando os ideais daoísta e confucionistas | tradução da metafísica (形而上学)}
    \seeref{形而上学}{xing2'er2shang4xue2}
  \end{phonetics}
\end{entry}

\begin{entry}{玉}{5}[Radical 玉][Kangxi 96]
  \begin{phonetics}{玉}{yu4}
    \definition[块]{s.}{jade}
  \end{phonetics}
\end{entry}

\begin{entry}{玉米}{5,6}
  \begin{phonetics}{玉米}{yu4mi3}
    \definition[粒]{s.}{milho}
  \end{phonetics}
\end{entry}

\begin{entry}{玉米片}{5,6,4}
  \begin{phonetics}{玉米片}{yu4mi3pian4}
    \definition{s.}{flocos de milho | chips de tortilha}
  \end{phonetics}
\end{entry}

\begin{entry}{玉米花}{5,6,7}
  \begin{phonetics}{玉米花}{yu4mi3hua1}
    \definition{s.}{pipoca}
  \end{phonetics}
\end{entry}

\begin{entry}{玉米面}{5,6,9}
  \begin{phonetics}{玉米面}{yu4mi3mian4}
    \definition{s.}{fubá | farinha de milho}
  \end{phonetics}
\end{entry}

\begin{entry}{玉米饼}{5,6,9}
  \begin{phonetics}{玉米饼}{yu4mi3bing3}
    \definition{s.}{tortilha mexicana | bolo de milho}
  \end{phonetics}
\end{entry}

\begin{entry}{玉米笋}{5,6,10}
  \begin{phonetics}{玉米笋}{yu4mi3sun3}
    \definition{s.}{broto de milho}
  \end{phonetics}
\end{entry}

\begin{entry}{玉米粉}{5,6,10}
  \begin{phonetics}{玉米粉}{yu4mi3fen3}
    \definition{s.}{amido de milho | farinha de milho}
  \end{phonetics}
\end{entry}

\begin{entry}{玉米糁}{5,6,14}
  \begin{phonetics}{玉米糁}{yu4mi3san3}
    \definition{s.}{grãos de milho}
  \end{phonetics}
\end{entry}

\begin{entry}{玉米糕}{5,6,16}
  \begin{phonetics}{玉米糕}{yu4mi3gao1}
    \definition{s.}{bolo de milho | polenta}
  \end{phonetics}
\end{entry}

\begin{entry}{甘心}{5,4}
  \begin{phonetics}{甘心}{gan1xin1}
    \definition{v.}{estar disposto a | resignar-se a}
  \end{phonetics}
\end{entry}

\begin{entry}{甘薯}{5,16}
  \begin{phonetics}{甘薯}{gan1shu3}
    \definition{s.}{batata doce}
  \end{phonetics}
\end{entry}

\begin{entry}{生}{5}[Radical 生][Kangxi 100]
  \begin{phonetics}{生}{sheng1}[][HSK 2]
    \definition{adj.}{vida | estudante | cru | não cozido}
    \definition{v.}{nascer | dar a luz | crescer}
  \end{phonetics}
\end{entry}

\begin{entry}{生日}{5,4}
  \begin{phonetics}{生日}{sheng1ri4}[][HSK 1]
    \definition[个]{s.}{aniversário}
  \end{phonetics}
\end{entry}

\begin{entry}{生气}{5,4}
  \begin{phonetics}{生气}{sheng1 qi4}[][HSK 1]
    \definition{s.}{vitalidade | vigor}
    \definition{v.+compl.}{irritar-se | zangar-se | ofender-se | ficar com raiva}
  \end{phonetics}
\end{entry}

\begin{entry}{生长}{5,4}
  \begin{phonetics}{生长}{sheng1zhang3}
    \definition{v.}{crescer | amadurecer | ser criado}
  \end{phonetics}
\end{entry}

\begin{entry}{生词}{5,7}
  \begin{phonetics}{生词}{sheng1 ci2}[][HSK 2]
    \definition[个]{s.}{nova palavra}
  \end{phonetics}
\end{entry}

\begin{entry}{生态}{5,8}
  \begin{phonetics}{生态}{sheng1tai4}
    \definition{adj.}{ecológico}
    \definition{s.}{ecologia}
  \end{phonetics}
\end{entry}

\begin{entry}{生物}{5,8}
  \begin{phonetics}{生物}{sheng1wu4}
    \definition{adj.}{biológico}
    \definition{s.}{biologia (disciplina) | organismo | ser vivo}
  \end{phonetics}
\end{entry}

\begin{entry}{生的}{5,8}
  \begin{phonetics}{生的}{sheng1de5}
    \definition{conj.}{para evitar isso | para que\dots não\dots}
  \end{phonetics}
\end{entry}

\begin{entry}{生鱼片}{5,8,4}
  \begin{phonetics}{生鱼片}{sheng1yu2pian4}
    \definition{s.}{fatias de peixe cru, \emph{sashimi}}
  \end{phonetics}
\end{entry}

\begin{entry}{生活}{5,9}
  \begin{phonetics}{生活}{sheng1huo2}[][HSK 2]
    \definition[道]{s.}{vida | atividade | meios de subsistência}
    \definition{v.}{viver}
  \end{phonetics}
\end{entry}

\begin{entry}{生活垃圾}{5,9,8,6}
  \begin{phonetics}{生活垃圾}{sheng1huo2la1ji1}
    \definition{s.}{lixo doméstico}
  \end{phonetics}
\end{entry}

\begin{entry}{生活型}{5,9,9}
  \begin{phonetics}{生活型}{sheng1huo2 xing2}
    \definition{s.}{forma de vida}
  \end{phonetics}
\end{entry}

\begin{entry}{生病}{5,10}
  \begin{phonetics}{生病}{sheng1 bing4}[][HSK 1]
    \definition{v.}{ficar doente | estar adoecido}
  \end{phonetics}
\end{entry}

\begin{entry}{生理}{5,11}
  \begin{phonetics}{生理}{sheng1li3}
    \definition{adj.}{fisiológico}
    \definition{s.}{fisiologia}
  \end{phonetics}
\end{entry}

\begin{entry}{生菜}{5,11}
  \begin{phonetics}{生菜}{sheng1cai4}
    \definition{s.}{alface}
  \end{phonetics}
\end{entry}

\begin{entry}{生意}{5,13}
  \begin{phonetics}{生意}{sheng1yi4}
    \definition{s.}{força vital | vitalidade}
  \end{phonetics}
  \begin{phonetics}{生意}{sheng1yi5}
    \definition{s.}{negócio}
  \end{phonetics}
\end{entry}

\begin{entry}{用}{5}[Radical 用][Kangxi 101]
  \begin{phonetics}{用}{yong4}[][HSK 1]
    \definition{v.}{usar}
  \end{phonetics}
\end{entry}

\begin{entry}{用心}{5,4}
  \begin{phonetics}{用心}{yong4xin1}
    \definition{s.}{motivo | intenção}
    \definition{v.+compl.}{ser diligente ou atencioso}
  \end{phonetics}
\end{entry}

\begin{entry}{用处}{5,5}
  \begin{phonetics}{用处}{yong4chu5}
    \definition[个]{s.}{usabilidade | utilidade}
  \end{phonetics}
\end{entry}

\begin{entry}{用料}{5,10}
  \begin{phonetics}{用料}{yong4liao4}
    \definition{s.}{ingredientes | materiais}
  \end{phonetics}
\end{entry}

\begin{entry}{田}{5}[Radical 田][Kangxi 102]
  \begin{phonetics}{田}{tian2}
    \definition*{s.}{sobrenome Tian}
    \definition[片]{s.}{fazenda | campo}
  \end{phonetics}
\end{entry}

\begin{entry}{田园}{5,7}
  \begin{phonetics}{田园}{tian2yuan2}
    \definition{adj.}{bucólico}
    \definition{s.}{campo | interior | rural}
  \end{phonetics}
\end{entry}

\begin{entry}{由于}{5,3}
  \begin{phonetics}{由于}{you2yu2}
    \definition{conj.}{porque; desde}
    \definition{prep.}{devido a; graças a; devido a; em virtude de; como resultado de}
  \end{phonetics}
\end{entry}

\begin{entry}{甲骨文}{5,9,4}
  \begin{phonetics}{甲骨文}{jia3gu3wen2}
    \definition{s.}{escrituras de oráculos | inscrições em ossos de oráculos (forma original de escritura chinesa)}
  \end{phonetics}
\end{entry}

\begin{entry}{电}{5}[Radical 田]
  \begin{phonetics}{电}{dian4}[][HSK 1]
    \definition{s.}{eletricidade | telegrama | cabo}
    \definition{v.}{dar ou receber um choque elétrico | enviar um telegrama | telegrafar}
  \end{phonetics}
\end{entry}

\begin{entry}{电子}{5,3}
  \begin{phonetics}{电子}{dian4zi3}
    \definition{s.}{eletrônico | elétron}
  \end{phonetics}
\end{entry}

\begin{entry}{电子名片}{5,3,6,4}
  \begin{phonetics}{电子名片}{dian4zi3 ming2pian4}
    \definition{s.}{cartão de visita eletrônico}
  \end{phonetics}
\end{entry}

\begin{entry}{电子邮件}{5,3,7,6}
  \begin{phonetics}{电子邮件}{dian4zi3you2jian4}[][HSK 3]
    \definition[封,份]{s.}{correio eletrônico; \emph{e-mail}}
  \seealsoref{电邮}{dian4you2}
  \end{phonetics}
\end{entry}

\begin{entry}{电车司机}{5,4,5,6}
  \begin{phonetics}{电车司机}{dian4che1 si1ji1}
    \definition{s.}{motorista de bonde}
  \end{phonetics}
\end{entry}

\begin{entry}{电台}{5,5}
  \begin{phonetics}{电台}{dian4 tai2}[][HSK 3]
    \definition[个,家]{s.}{transceptor; transmissor-receptor | aparelho de rádio; estação de rádio; estação de transmissão}
  \end{phonetics}
\end{entry}

\begin{entry}{电冰箱}{5,6,15}
  \begin{phonetics}{电冰箱}{dian4bing1xiang1}
    \definition[台]{s.}{frigorífico | refrigerador}
  \end{phonetics}
\end{entry}

\begin{entry}{电动}{5,6}
  \begin{phonetics}{电动}{dian4dong4}
    \definition{adj.}{movido a eletricidade | elétrico}
  \end{phonetics}
\end{entry}

\begin{entry}{电动车}{5,6,4}
  \begin{phonetics}{电动车}{dian4dong4che1}
    \definition{s.}{veículo elétrico (\emph{scooter}, bicicleta, carro, etc.)}
  \end{phonetics}
\end{entry}

\begin{entry}{电灯泡}{5,6,8}
  \begin{phonetics}{电灯泡}{dian4deng1pao4}
    \definition{s.}{lâmpada elétrica | (gíria) terceiro convidado indesejado}
  \end{phonetics}
\end{entry}

\begin{entry}{电邮}{5,7}
  \begin{phonetics}{电邮}{dian4you2}
    \definition{s.}{correio eletrônico, \emph{e-mail} | abreviação de~电子邮件}
  \seealsoref{电子邮件}{dian4zi3you2jian4}
  \end{phonetics}
\end{entry}

\begin{entry}{电视}{5,8}
  \begin{phonetics}{电视}{dian4shi4}[][HSK 1]
    \definition[台,个]{s.}{televisão | TV | televisor}
  \end{phonetics}
\end{entry}

\begin{entry}{电视台}{5,8,5}
  \begin{phonetics}{电视台}{dian4 shi4 tai2}[][HSK 3]
    \definition[个]{s.}{canal de TV | estação de televisão}
  \end{phonetics}
\end{entry}

\begin{entry}{电视机}{5,8,6}
  \begin{phonetics}{电视机}{dian4shi4ji1}[][HSK 1]
    \definition[台]{s.}{aparelho de televisão | televisor}
  \end{phonetics}
\end{entry}

\begin{entry}{电视剧}{5,8,10}
  \begin{phonetics}{电视剧}{dian4 shi4 ju4}[][HSK 3]
    \definition[部]{s.}{série de TV; drama de TV; novela}
  \end{phonetics}
\end{entry}

\begin{entry}{电话}{5,8}
  \begin{phonetics}{电话}{dian4hua4}[][HSK 1]
    \definition[部]{s.}{telefone}
    \definition[通]{s.}{chamada telefônica}
  \end{phonetics}
\end{entry}

\begin{entry}{电脑}{5,10}
  \begin{phonetics}{电脑}{dian4nao3}[][HSK 1]
    \definition[台]{s.}{computador}
  \end{phonetics}
\end{entry}

\begin{entry}{电脑语言}{5,10,9,7}
  \begin{phonetics}{电脑语言}{dian4nao3yu3yan2}
    \definition{s.}{linguagem de programação | linguagem de computador}
  \end{phonetics}
\end{entry}

\begin{entry}{电梯}{5,11}
  \begin{phonetics}{电梯}{dian4ti1}
    \definition[台,部]{s.}{elevador | ascensor}
  \end{phonetics}
\end{entry}

\begin{entry}{电梯司机}{5,11,5,6}
  \begin{phonetics}{电梯司机}{dian4ti1 si1ji1}
    \definition{s.}{ascensorista}
  \end{phonetics}
\end{entry}

\begin{entry}{电影}{5,15}
  \begin{phonetics}{电影}{dian4ying3}[][HSK 1]
    \definition[部,片,幕,场]{s.}{filme}
  \end{phonetics}
\end{entry}

\begin{entry}{电影艺术}{5,15,4,5}
  \begin{phonetics}{电影艺术}{dian4ying3 yi4shu4}
    \definition{s.}{arte cinematográfica}
  \end{phonetics}
\end{entry}

\begin{entry}{电影术}{5,15,5}
  \begin{phonetics}{电影术}{dian4ying3 shu4}
    \definition{s.}{cinematografia}
  \end{phonetics}
\end{entry}

\begin{entry}{电影节}{5,15,5}
  \begin{phonetics}{电影节}{dian4ying3jie2}
    \definition{s.}{festival de cinema}
  \end{phonetics}
\end{entry}

\begin{entry}{电影奖}{5,15,9}
  \begin{phonetics}{电影奖}{dian4ying3jiang3}
    \definition{s.}{premiações de cinema}
  \end{phonetics}
\end{entry}

\begin{entry}{电影界}{5,15,9}
  \begin{phonetics}{电影界}{dian4ying3jie4}
    \definition{s.}{indústria cinematográfica}
  \end{phonetics}
\end{entry}

\begin{entry}{电影院}{5,15,9}
  \begin{phonetics}{电影院}{dian4ying3yuan4}[][HSK 1]
    \definition[次,家,座]{s.}{sala de cinema}
  \end{phonetics}
\end{entry}

\begin{entry}{电影音乐}{5,15,9,5}
  \begin{phonetics}{电影音乐}{dian4ying3 yin1yue4}
    \definition{s.}{música cinematográfica}
  \end{phonetics}
\end{entry}

\begin{entry}{电影票}{5,15,11}
  \begin{phonetics}{电影票}{dian4ying3piao4}
    \definition{s.}{ingresso de filme}
  \end{phonetics}
\end{entry}

\begin{entry}{电器}{5,16}
  \begin{phonetics}{电器}{dian4qi4}
    \definition{s.}{aparelho elétrico}
  \end{phonetics}
\end{entry}

\begin{entry}{白}{5}[Radical 白][Kangxi 106]
  \begin{phonetics}{白}{bai2}[][HSK 1]
    \definition*{s.}{sobrenome Bai}
    \definition{adj.}{branco | claro | puro | límpido | simples | em branco | grátis}
    \definition{adv.}{em vão | sem propósito | por nada}
    \definition{s.}{parte falada na ópera | diálogo | dialeto}
  \end{phonetics}
\end{entry}

\begin{entry}{白天}{5,4}
  \begin{phonetics}{白天}{bai2tian1}[][HSK 1]
    \definition{adv.}{dia | de dia}
    \definition[个]{s.}{dia}
  \end{phonetics}
\end{entry}

\begin{entry}{白色}{5,6}
  \begin{phonetics}{白色}{bai2 se4}[][HSK 2]
    \definition{s.}{cor branca}
  \end{phonetics}
\end{entry}

\begin{entry}{白苋}{5,7}
  \begin{phonetics}{白苋}{bai2xian4}
    \definition{s.}{amaranto branco | brotos e folhas tenras de espinafre chinês usados como alimento}
  \end{phonetics}
\end{entry}

\begin{entry}{白拣}{5,8}
  \begin{phonetics}{白拣}{bai2jian3}
    \definition{s.}{uma escolha barata}
    \definition{v.}{escolher algo que não custa nada}
  \end{phonetics}
\end{entry}

\begin{entry}{白菜}{5,11}
  \begin{phonetics}{白菜}{bai2 cai4}[][HSK 3]
    \definition[棵,个]{s.}{acelga | repolho chinês}
  \end{phonetics}
\end{entry}

\begin{entry}{白萝卜}{5,11,2}
  \begin{phonetics}{白萝卜}{bai2luo2bo5}
    \definition{s.}{rabanete branco}
  \end{phonetics}
\end{entry}

\begin{entry}{白蛋白}{5,11,5}
  \begin{phonetics}{白蛋白}{bai2dan4bai2}
    \definition{s.}{albumina}
  \end{phonetics}
\end{entry}

\begin{entry}{白鹄}{5,12}
  \begin{phonetics}{白鹄}{bai2hu2}
    \definition{s.}{cisne branco}
  \end{phonetics}
\end{entry}

\begin{entry}{白痴}{5,13}
  \begin{phonetics}{白痴}{bai2chi1}
    \definition{adj./s.}{estúpido | imbecil}
  \end{phonetics}
\end{entry}

\begin{entry}{皮}{5}[Radical 皮][Kangxi 107]
  \begin{phonetics}{皮}{pi2}
    \definition*{s.}{sobrenome Pi}
    \definition{adj.}{safado}
    \definition{pref.}{``pico'' (um trilhonésimo)}
    \definition[张]{s.}{couro | pele | pelagem}
  \end{phonetics}
\end{entry}

\begin{entry}{皮下}{5,3}
  \begin{phonetics}{皮下}{pi2xia4}
    \definition{adj.}{(injeção) subcutâneo | sob a pele}
  \end{phonetics}
\end{entry}

\begin{entry}{皮卡}{5,5}
  \begin{phonetics}{皮卡}{pi2ka3}
    \definition{s.}{(empréstimo linguístico) \emph{pick-up} | caminhonete}
  \end{phonetics}
\end{entry}

\begin{entry}{皮卡丘}{5,5,5}
  \begin{phonetics}{皮卡丘}{pi2ka3qiu1}
    \definition*{s.}{\emph{Pikachu} (Pokémon, 口袋妖怪)}
  \seealsoref{口袋妖怪}{kou3dai4 yao1guai4}
  \end{phonetics}
\end{entry}

\begin{entry}{皮肤}{5,8}
  \begin{phonetics}{皮肤}{pi2fu1}
    \definition[层,块]{s.}{pele}
  \end{phonetics}
\end{entry}

\begin{entry}{皮鞋}{5,15}
  \begin{phonetics}{皮鞋}{pi2xie2}
    \definition[双,只,款]{s.}{sapatos de couro}
  \end{phonetics}
\end{entry}

\begin{entry}{目的}{5,8}
  \begin{phonetics}{目的}{mu4di4}[][HSK 2]
    \definition[个]{s.}{objetivo | meta | alvo | propósito}
  \end{phonetics}
\end{entry}

\begin{entry}{礼节}{5,5}
  \begin{phonetics}{礼节}{li3jie2}
    \definition{s.}{protocolo | cerimônia | etiqueta}
  \end{phonetics}
\end{entry}

\begin{entry}{礼让}{5,5}
  \begin{phonetics}{礼让}{li3rang4}
    \definition{s.}{cortesia}
    \definition{v.}{mostrar consideração por (outros) | ceder a (outro veículo, etc.)}
  \end{phonetics}
\end{entry}

\begin{entry}{礼物}{5,8}
  \begin{phonetics}{礼物}{li3wu4}[][HSK 2]
    \definition[件,个,份]{s.}{prenda | lembrança | presente}
  \end{phonetics}
\end{entry}

\begin{entry}{立刻}{5,8}
  \begin{phonetics}{立刻}{li4ke4}
    \definition{adv.}{imediatamente}
  \end{phonetics}
\end{entry}

\begin{entry}{立法}{5,8}
  \begin{phonetics}{立法}{li4fa3}
    \definition{s.}{legislação}
    \definition{v.}{promulgar leis | legislar}
  \end{phonetics}
\end{entry}

\begin{entry}{纠葛}{5,12}
  \begin{phonetics}{纠葛}{jiu1ge2}
    \definition{s.}{emaranhado | disputa}
  \end{phonetics}
\end{entry}

\begin{entry}{节}{5}[Radical 艸]
  \begin{phonetics}{节}{jie2}[][HSK 2]
    \definition*{s.}{sobrenome Jie}
    \definition{clas.}{para nós, seções, comprimentos}
    \definition{s.}{junta | botão | nó | divisão | parte | festival | feriado | item | integridade moral | castidade}
  \end{phonetics}
\end{entry}

\begin{entry}{节日}{5,4}
  \begin{phonetics}{节日}{jie2ri4}[][HSK 2]
    \definition[个]{s.}{festival | feriado}
  \end{phonetics}
\end{entry}

\begin{entry}{节目}{5,5}
  \begin{phonetics}{节目}{jie2mu4}[][HSK 2]
    \definition{s.}{programa | item (em um programa)}
  \end{phonetics}
\end{entry}

\begin{entry}{节奏}{5,9}
  \begin{phonetics}{节奏}{jie2zou4}
    \definition{s.}{ritmo | cadência | batida}
  \end{phonetics}
\end{entry}

\begin{entry}{讨生活}{5,5,9}
  \begin{phonetics}{讨生活}{tao3sheng1huo2}
    \definition{v.}{ganhar a vida}
  \end{phonetics}
\end{entry}

\begin{entry}{讨论}{5,6}
  \begin{phonetics}{讨论}{tao3lun4}[][HSK 2]
    \definition{v.}{discutir | falar sobre}
  \end{phonetics}
\end{entry}

\begin{entry}{让}{5}[Radical 言]
  \begin{phonetics}{让}{rang4}[][HSK 2]
    \definition{v.}{deixar alguém fazer alguma coisa |fazer alguém (sentir-se triste, etc.) | permitir | conceder}
  \end{phonetics}
\end{entry}

\begin{entry}{让步}{5,7}
  \begin{phonetics}{让步}{rang4bu4}
    \definition{v.+compl.}{fazer uma concessão | entregar | desistir | comprometer}
  \end{phonetics}
\end{entry}

\begin{entry}{记}{5}[Radical 言]
  \begin{phonetics}{记}{ji4}[][HSK 1]
    \definition{clas.}{para tapas, palmadas, bofetadas, etc.}
    \definition{s.}{nota | registro | marca | sinal |marca de nascença}
    \definition{v.}{lembrar | ter em mente | memorizar | escrever (anotar) | registrar}
  \end{phonetics}
\end{entry}

\begin{entry}{记住}{5,7}
  \begin{phonetics}{记住}{ji4-zhu4}[][HSK 1]
    \definition{v.}{decorar | memorizar | ter em mente}
  \end{phonetics}
\end{entry}

\begin{entry}{记录}{5,8}
  \begin{phonetics}{记录}{ji4lu4}[][HSK 3]
    \definition[个,位]{s.}{notas; registro | anotador; registrador}
    \definition{v.}{tomar notas; registrar}
  \end{phonetics}
\end{entry}

\begin{entry}{记性}{5,8}
  \begin{phonetics}{记性}{ji4xing5}
    \definition{s.}{memória (habilidade em reter informações)}
  \end{phonetics}
\end{entry}

\begin{entry}{记者}{5,8}
  \begin{phonetics}{记者}{ji4zhe3}[][HSK 3]
    \definition[群,名,位]{s.}{repórter; correspondente; jornalista}
  \end{phonetics}
\end{entry}

\begin{entry}{记得}{5,11}
  \begin{phonetics}{记得}{ji4de5}[][HSK 1]
    \definition{v.}{lembrar | lembrar-se}
  \end{phonetics}
\end{entry}

\begin{entry}{边}{5}[Radical 辵]
  \begin{phonetics}{边}{bian1}[][HSK 2]
    \definition{adv.}{simultaneamente}
    \definition[个]{s.}{fronteira | limite | borda | margem | lado}
  \end{phonetics}
  \begin{phonetics}{边}{bian5}[][HSK 0]
    \definition{suf.}{sufixo de uma palavra de localidade}
  \end{phonetics}
\end{entry}

\begin{entry}{边关}{5,6}
  \begin{phonetics}{边关}{bian1guan1}
    \definition{s.}{posto de fronteira | posição defensiva estratégica na fronteira}
  \end{phonetics}
\end{entry}

\begin{entry}{边防}{5,6}
  \begin{phonetics}{边防}{bian1fang2}
    \definition{s.}{defesa da fronteira}
  \end{phonetics}
\end{entry}

\begin{entry}{闪存盘}{5,6,11}
  \begin{phonetics}{闪存盘}{shan3cun2pan2}
    \definition{s.}{unidade de memória \emph{USB} | cartão de memória}
  \seealsoref{优盘}{you1pan2}
  \end{phonetics}
\end{entry}

\begin{entry}{鸟}{5}[Radical 鳥]
  \begin{phonetics}{鸟}{diao3}[][HSK 0]
    \definition{s.}{pênis | órgão genital masculino | aves | aviário}
  \end{phonetics}
  \begin{phonetics}{鸟}{niao3}[][HSK 2]
    \definition[只,群]{s.}{pássaro}
  \end{phonetics}
\end{entry}

\begin{entry}{鸟儿}{5,2}
  \begin{phonetics}{鸟儿}{niao3r5}
    \definition[只]{s.}{pássaro | ave}
  \end{phonetics}
\end{entry}

\begin{entry}{龙}{5}[Radical 龍]
  \begin{phonetics}{龙}{long2}
    \definition*{s.}{sobrenome Long}
    \definition{adj.}{imperial}
    \definition[条]{s.}{dragão chinês | (fig.) imperador | dragão | (forma ligada) dinossauro}
  \end{phonetics}
\end{entry}

\begin{entry}{龙山}{5,3}
  \begin{phonetics}{龙山}{long2shan1}
    \definition*{s.}{Longshan}
  \end{phonetics}
\end{entry}

\begin{entry}{龙虾}{5,9}
  \begin{phonetics}{龙虾}{long2xia1}
    \definition{s.}{lagosta}
  \end{phonetics}
\end{entry}

%%%%% EOF %%%%%

